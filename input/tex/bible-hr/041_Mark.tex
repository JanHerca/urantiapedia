\begin{document}

\title{Marko}


\chapter{1}

\par 1 Početak Evanđelja Isusa Krista Sina Božjega. 
\par 2 Pisano je u  Izaiji proroku: Evo šaljem glasnika svoga pred licem tvojim da ti pripravi put. 
\par 3 Glas viče u pustinji: Pripravite put Gospodinu, poravnite mu staze! 
\par 4 Tako se pojavi Ivan: krstio je u pustinji i propovijedao  krst obraćenja na otpuštenje grijeha. 
\par 5 Grnula k njemu sva judejska  zemlja i svi Jeruzalemci: primali su od njega krštenje u rijeci  Jordanu ispovijedajući svoje grijehe. 
\par 6 Ivan bijaše odjeven u devinu dlaku, s kožnatim pojasom  oko bokova; hranio se skakavcima i divljim medom. 
\par 7 I propovijedao je: "Nakon mene dolazi jači od mene. Ja nisam dostojan sagnuti  se i odriješiti mu remenje na obući. 
\par 8 Ja vas krstim vodom,  a on će vas krstiti Duhom Svetim." 
\par 9 Onih dana dođe Isus iz Nazareta galilejskoga i primi u  Jordanu krštenje od Ivana. 
\par 10 I odmah, čim izađe iz vode, ugleda  otvorena nebesa i Duha poput goluba gdje silazi na nj, 
\par 11 a  glas se zaori s nebesa: Ti si Sin moj, Ljubljeni! U tebi mi  sva milina! 
\par 12 I odmah ga Duh nagna u pustinju. 
\par 13 I bijaše u pustinji  četrdeset dana, gdje ga je iskušavao Sotona; bijaše sa zvijerima, a anđeli mu služahu. 
\par 14 A pošto Ivan bijaše predan, otiđe Isus u Galileju. Propovijedao  je evanđelje Božje: 
\par 15 "Ispunilo se vrijeme, približilo se kraljevstvo  Božje! Obratite se i vjerujte evanđelju!" 
\par 16 I prolazeći uz Galilejsko more, ugleda Šimuna i Andriju, brata Šimunova, gdje ribare na moru; bijahu ribari. 
\par 17 I reče  im Isus: "Hajdete za mnom i učinit ću vas ribarima ljudi!" 
\par 18 Oni  odmah ostaviše mreže i pođoše za njim. 
\par 19 Pošavši malo naprijed, ugleda Jakova Zebedejeva i njegova  brata Ivana: u lađi su krpali mreže. 
\par 20 Odmah pozva i njih.  Oni ostave oca Zebedeja u lađi s nadničarima i otiđu za njim. 
\par 21 I stignu u Kafarnaum. Odmah u subotu uđe on u sinagogu  i poče naučavati. 
\par 22 Bijahu zaneseni njegovim naukom. Ta učio  ih je kao onaj koji ima vlast, a ne kao pismoznanci. 
\par 23 A u njihovoj se sinagogi upravo zatekao čovjek opsjednut  nečistim duhom. On povika: 
\par 24 "Što ti imaš s nama, Isuse Nazarećanine?  Došao si da nas uništiš? Znam tko si: Svetac Božji!" 
\par 25 Isus  mu zaprijeti: "Umukni i iziđi iz njega!" 
\par 26 Nato nečisti duh  potrese njime pa povika iz svega glasa i iziđe iz njega. 
\par 27 Svi se zaprepastiše te se zapitkivahu: "Što li je ovo?  Nova li i snažna nauka! Pa i samim nečistim dusima zapovijeda, i pokoravaju mu se." 
\par 28 I pročulo se odmah o njemu posvuda, po svoj okolici galilejskoj. 
\par 29 I odmah pošto iziđoše iz sinagoge, uđe s Jakovom i Ivanom  u kuću Šimunovu i Andrijinu. 
\par 30 A punica Šimunova ležala u ognjici.  I odmah mu kažu za nju. 
\par 31 On pristupi, prihvati je za ruku  i podiže. I pusti je ognjica. I posluživaše im. 
\par 32 Uvečer, kad sunce zađe, donošahu preda nj sve bolesne  i opsjednute. 
\par 33 I sav je grad nagrnuo k vratima. 
\par 34 I on ozdravi  bolesnike - a bijahu mnogi i razne im bolesti - i zloduhe mnoge  izagna. I ne dopusti zlodusima govoriti jer su ga znali. 
\par 35 Rano ujutro, još za mraka, ustane, iziđe i povuče se  na samotno mjesto i ondje se moljaše. 
\par 36 Potražiše ga Šimun  i njegovi drugovi. 
\par 37 Kad ga nađoše, rekoše mu: "Svi te traže." 
\par 38 Kaže im: "Hajdemo drugamo, u obližnja mjesta, da i ondje  propovijedam! Ta zato sam došao." 
\par 39 I prođe svom Galilejom:  propovijedao je u njihovim sinagogama i zloduhe izgonio. 
\par 40 I dođe k njemu neki gubavac, klekne i zamoli: "Ako hoćeš, možeš me očistiti!" 
\par 41 Isus ganut pruži ruku, dotače ga se  pa će mu: "Hoću, budi čist!" 
\par 42 I odmah nesta s njega gube i  očisti se. 
\par 43 Isus se otrese na nj i odmah ga otpravi 
\par 44 riječima:  "Pazi, nikomu ništa ne kazuj, nego idi, pokaži se svećeniku  i prinesi za svoje očišćenje što propisa Mojsije, njima za svjedočanstvo." 
\par 45 Ali čim iziđe, stane on uvelike pripovijedati i razglašavati  događaj tako da Isus više nije mogao javno ući u grad, nego se  zadržavao vani na samotnim mjestima. I dolažahu k njemu odasvud. 


\chapter{2}

\par 1 I pošto nakon nekoliko dana opet uđe u Kafarnaum, pročulo se  da je u kući. 
\par 2 I skupiše se mnogi te više nije bilo mjesta  ni pred vratima. On im navješćivaše Riječ. 
\par 3 I dođu noseći k  njemu uzetoga. Nosila ga četvorica. 
\par 4 Budući da ga zbog mnoštva  nisu mogli unijeti k njemu, otkriju krov nad mjestom gdje bijaše  Isus. Načinivši otvor, spuste postelju na kojoj je uzeti ležao. 
\par 5 Vidjevši njihovu vjeru, kaže Isus uzetome: "Sinko! Otpuštaju  ti se grijesi." 
\par 6 Sjedjeli su ondje neki pismoznanci koji počeše  mudrovati u sebi: 
\par 7 "Što to ovaj govori? Huli! Ta tko može grijehe  otpuštati doli Bog jedini?" 
\par 8 Isus duhom odmah proniknu da tako  mudruju u sebi, pa će im: "Što to mudrujete u sebi? 
\par 9 Ta što  je lakše? Reći uzetomu: 'Otpuštaju ti se grijesi' ili reći: 'Ustani, uzmi svoju postelju i hodi?' 
\par 10 Ali da znate: vlastan je Sin  Čovječji na zemlji otpuštati grijehe!" I reče uzetomu: 
\par 11 "Tebi  zapovijedam, ustani, uzmi postelju i pođi kući!" 
\par 12 I on usta, uze odmah postelju i iziđe na očigled svima. Svi su zaneseni  slavili Boga govoreći: "Takvo što nikad još ne vidjesmo!" 
\par 13 Isus ponovno iziđe k moru. Sve je ono mnoštvo grnulo  k njemu i on ih poučavaše. 
\par 14 Prolazeći ugleda Levija Alfejeva  gdje sjedi u carinarnici. I kaže mu: "Pođi za mnom!" On usta  i pođe za njim. 
\par 15 Kada zatim Isus bijaše za stolom u njegovoj kući, nađoše  se za stolom s njime i njegovim učenicima i mnogi carinici i  grešnici. Bilo ih je uistinu mnogo. A slijedili su ga 
\par 16 i pismoznanci  farizejske sljedbe pa vidjevši da jede s grešnicima i carinicima  rekoše njegovim učenicima: "Zašto jede s carinicima i grešnicima?" 
\par 17 Čuvši to, Isus im reče: "Ne treba zdravima liječnika, nego  bolesnima! Ne dođoh zvati pravednike, nego grešnike." 
\par 18 Ivanovi su učenici i farizeji postili. I dođu neki i  kažu mu: "Zašto učenici Ivanovi i učenici farizejski poste, a  tvoji učenici ne poste?" 
\par 19 Nato im Isus reče: "Mogu li svatovi  postiti dok je zaručnik s njima? Dokle god imaju zaručnika sa  sobom, ne mogu postiti. 
\par 20 Doći će već dani kad će im se ugrabiti  zaručnik i tada će postiti u onaj dan!" 
\par 21 "Nitko ne prišiva krpe od sirova sukna na staro odijelo.  Inače nova zakrpa vuče sa starog odijela pa nastane još veća  rupa." 
\par 22 "I nitko ne ulijeva novo vino u stare mješine. Inače  će vino poderati mješine pa propade i vino i mješine. Nego -  novo vino u nove mješine!" 
\par 23 Jedne je subote prolazio kroz usjeve. Njegovi učenici  počeše putem trgati klasje. A farizeji mu rekoše: 
\par 24 "Gle! Zašto  čine što subotom nije dopušteno?" 
\par 25 Isus im odgovori: "Zar nikad niste čitali što učini David  kad ogladnje te se nađe u potrebi on i njegovi pratioci? 
\par 26 Kako  za velikog svećenika Ebjatara uđe u Dom Božji i pojede prinesene  kruhove kojih ne smije jesti nitko osim svećenika; a on dade  i svojim pratiocima?" 
\par 27 I govoraše im: "Subota je stvorena radi čovjeka, a ne  čovjek radi subote. 
\par 28 Tako, Sin Čovječji gospodar je subote!" 


\chapter{3}

\par 1 Uđe ponovno u sinagogu. Bio je ondje čovjek usahle ruke. 
\par 2 A  oni vrebahu hoće li ga Isus u subotu izliječiti, da ga optuže. 
\par 3 On kaže čovjeku usahle ruke: "Stani na sredinu!" 
\par 4 A njima  će: "Je li subotom dopušteno činiti dobro ili činiti zlo, život  spasiti ili pogubiti?" No oni su šutjeli. 
\par 5 A on, ražalošćen okorjelošću srca njihova, srdito ih ošinu  pogledom pa reče tom čovjeku: "Ispruži ruku!" On ispruži - i  ruka mu zdrava! 
\par 6 Farizeji iziđu i dadnu se odmah s herodovcima na vijećanje  protiv njega kako da ga pogube. 
\par 7 Isus se s učenicima povuče k moru. Za njim je išao silan  svijet iz Galileje. I iz Judeje, 
\par 8 iz Jeruzalema, iz Idumeje, iz Transjordanije i iz okolice Tira i Sidona - silno je mnoštvo  čulo što čini i nagrnulo k njemu. 
\par 9 Stoga reče učenicima neka mu se zbog mnoštva pripravi  lađica da ga ne bi zgnjeli. 
\par 10 Jer mnoge je ozdravio pa su se  svi koji bijahu pogođeni kakvim zlom bacali na nj da bi ga se  dotakli. 
\par 11 A nečisti duhovi, čim bi ga spazili, padali bi preda  nj i vikali: "Ti si Sin Božji!" 
\par 12 A on im se oštro prijetio  da ga ne prokazuju. 
\par 13 Uziđe na goru i pozove koje sam htjede. I dođoše k njemu. 
\par 14 I ustanovi dvanaestoricu da budu s njime i da ih šalje propovijedati 
\par 15 s vlašću da izgone đavle. 
\par 16 Ustanovi dakle dvanaestoricu:  Šimuna, kojemu nadjenu ime Petar, 
\par 17 i Jakova Zebedejeva i Ivana, brata Jakovljeva, kojima nadjenu ime Boanerges, to jest Sinovi  groma, 
\par 18 i Andriju i Filipa i Bartolomeja i Mateja i Tomu i  Jakova Alfejeva i Tadeja i Šimuna Kananajca 
\par 19 i Judu Iškariotskoga, koji ga izda. 
\par 20 I dođe Isus u kuću. Opet se skupi toliko mnoštvo da nisu  mogli ni jesti. 
\par 21 Čuvši to, dođoše njegovi da ga obuzdaju jer  se govorilo: "Izvan sebe je!" 
\par 22 I pismoznanci što siđoše iz Jeruzalema govorahu: "Beelzebula  ima, po poglavici đavolskom izgoni đavle." 
\par 23 A on ih dozva pa im u prispodobama govoraše: "Kako može  Sotona Sotonu izgoniti? 
\par 24 Ako se kraljevstvo u sebi razdijeli, ono ne može opstati. 
\par 25 Ili: ako se kuća u sebi razdijeli,  ona ne može opstati. 
\par 26 Ako je dakle Sotona sam na sebe ustao  i razdijelio se, ne može opstati, nego mu je kraj. 
\par 27 Nitko, dakako, ne može u kuću jakoga ući i oplijeniti mu pokućstvo  ako prije jakoga ne sveže. Tada će mu kuću oplijeniti!" 
\par 28 Doista, kažem vam, sve će se oprostiti sinovima ljudskima, koliki god bili grijesi i hule kojima pohule. 
\par 29 No pohuli  li tko na Duha Svetoga, nema oproštenja dovijeka; krivac je grijeha  vječnoga." 
\par 30 Jer govorahu: "Duha nečistoga ima." 
\par 31 I dođu majka njegova i braća njegova. Ostanu vani, a  k njemu pošalju neka ga pozovu. 
\par 32 Oko njega je sjedjelo mnoštvo.  I reknu mu: "Eno vani majke tvoje i braće tvoje, traže te!" 
\par 33 On  im odgovori: "Tko je majka moja i braća moja?" 
\par 34 I okruži pogledom po onima što su sjedjeli oko njega  u krugu i kaže: "Evo majke moje, evo braće moje! 
\par 35 Tko god  vrši volju Božju, on mi je brat i sestra i majka." 


\chapter{4}

\par 1 I poče opet poučavati uz more. I zgrnu se k njemu silan svijet  te on uđe u lađu i sjede na moru, a sve ono mnoštvo bijaše uz  more, na kopnu. 
\par 2 Poučavao ih je u prispodobama mnogočemu. Govorio im u  pouci: 
\par 3 "Poslušajte! Gle, iziđe sijač sijati. 
\par 4 I dok je sijao, poneko zrno pade uz put, dođoše ptice i pozobaše ga. 
\par 5 Neko  opet pade na kamenito tlo gdje nemaše dosta zemlje. Odmah izniknu  jer nemaše duboke zemlje. 
\par 6 Ali kad ogranu sunce, izgorje; i  jer nemaše korijenja, osuši se. 
\par 7 Neko opet pade u trnje i trnje  uzraste i uguši ga te ploda ne donese. 
\par 8 Neko napokon pade u  dobru zemlju i dade plod, razraste se i razmnoži, te donese:  jedno tridesetostruko, jedno šezdesetostruko, jedno stostruko." 
\par 9 I doda: "Tko ima uši da čuje, neka čuje!" 
\par 10 Kad bijaše nasamo, oni oko njega zajedno s dvanaestoricom  pitahu ga o prispodobama. 
\par 11 I govoraše im: "Vama je dano otajstvo  kraljevstva Božjega, a onima vani sve biva u prispodobama: 
\par 12 da gledaju, gledaju - i ne vide, slušaju, slušaju - i ne razumiju, da se ne obrate pa da im se otpusti." 
\par 13 I kaže im: "Zar ne znate tu prispodobu? Kako ćete onda  razumjeti prispodobe uopće? 
\par 14 Sijač sije Riječ. 
\par 15 Oni uz  put, gdje je Riječ posijana, jesu oni kojima, netom čuju, odmah  dolazi Sotona i odnosi Riječ u njih posijanu. 
\par 16 Zasijani na  tlo kamenito jesu oni koji kad čuju Riječ, odmah je s radošću  prime, 
\par 17 ali nemaju u sebi korijena, nego su nestalni: kad  nastane nevolja ili progonstvo zbog Riječi, odmah se sablazne. 
\par 18 A drugi su oni u trnje zasijani. To su oni koji poslušaju  Riječ, 
\par 19 ali nadošle brige vremenite, zavodljivost bogatstva  i ostale požude uguše Riječ te ona ostane bez ploda. 
\par 20 A zasijani  na dobru zemlju jesu oni koji čuju i prime Riječ te urode: tridesetostruko, šezdesetostruko, stostruko. 
\par 21 I govoraše im: "Unosi li se svjetiljka da se pod posudu  stavi ili pod postelju? Zar ne da se stavi na svijećnjak? 
\par 22 Ta  ništa nije zastrto, osim zato da se očituje; i ništa skriveno, osim zato da dođe na vidjelo! 
\par 23 Ima li tko uši da čuje, neka  čuje." 
\par 24 I govoraše im: "Pazite što slušate. Mjerom kojom mjerite  mjerit će vam se. I nadodat će vam se. 
\par 25 Doista, onomu tko  ima dat će se, a onomu tko nema oduzet će se i ono što ima." 
\par 26 I govoraše im: "Kraljevstvo je Božje kao kad čovjek baci  sjeme u zemlju. 
\par 27 Spavao on ili bdio, noću i danju sjeme klija  i raste - sam ne zna kako; 
\par 28 zemlja sama od sebe donosi plod:  najprije stabljiku, onda klas i napokon puno zrnja na klasu. 
\par 29 A čim plod dopusti, brže se on laća srpa jer eto žetve." 
\par 30 I govoraše: "Kako da prispodobimo kraljevstvo nebesko  ili u kojoj da ga prispodobi iznesemo? 
\par 31 Kao kad se gorušičino  zrno posije u zemlju. Manje od svega sjemenja na zemlji, 
\par 32 jednoć  posijano, naraste i postane veće od svega povrća pa potjera velike  grane te se pod sjenom njegovom gnijezde ptice nebeske." 
\par 33 Mnogim takvim prispodobama navješćivaše im Riječ, kako  već mogahu slušati. 
\par 34 Bez prispodobe im ne govoraše, a nasamo  bi svojim učenicima sve razjašnjavao. 
\par 35 Uvečer istoga dana kaže im: "Prijeđimo prijeko!" 
\par 36 Oni  otpuste mnoštvo i povezu Isusa kako već bijaše u lađi. A pratile  su ga i druge lađe. 
\par 37 Najednom nasta žestoka oluja, na lađu  navale valovi te su je već gotovo napunili. 
\par 38 A on na krmi  spavaše na uzglavku. Probude ga i kažu mu: "Učitelju! Zar ne  mariš što ginemo?" 
\par 39 On se probudi, zaprijeti vjetru i reče  moru: "Utihni! Umukni!" I smiri se vjetar i nasta velika utiha. 
\par 40 Tada im reče: "Što ste bojažljivi? Kako nemate vjere?" 
\par 41 Oni  se silno prestrašiše pa se zapitkivahu: "Tko li je ovaj da mu  se i vjetar i more pokoravaju?" 


\chapter{5}

\par 1 Stigoše na onu stranu mora, u kraj gerazenski. 
\par 2 Čim iziđe  iz lađe, odmah mu iz grobnica pohiti u susret neki čovjek s nečistim  duhom. 
\par 3 Obitavalište je imao u grobnicama. I nitko ga više  nije mogao svezati ni lancima 
\par 4 jer je već često bio i okovima  i lancima svezan, ali je raskinuo okove i iskidao lance i nitko  ga nije mogao ukrotiti. 
\par 5 Po cijele bi noći i dane u grobnicama  i po brdima vikao i bio se kamenjem. 
\par 6 Kad izdaleka opazi Isusa, dotrči i pokloni mu se, 
\par 7 a  onda u sav glas povika: "Što ti imaš sa mnom, Isuse, Sine  Boga Svevišnjega? Zaklinjem te Bogom, ne muči me!" 
\par 8 Jer Isus  mu bijaše rekao: "Iziđi, duše nečisti, iz ovoga čovjeka!" 
\par 9 Isus  ga nato upita: "Kako ti je ime?" Kaže mu: "Legija mi je ime!  Ima nas mnogo!" 
\par 10 I uporno zaklinjaše Isusa da ih ne istjera  iz onoga kraja. 
\par 11 A ondje je pod brdom paslo veliko krdo svinja. 
\par 12 Zaklinjahu  ga dakle: "Pošalji nas u ove svinje da u njih uđemo!" 
\par 13 I on  im dopusti. Tada iziđoše nečisti duhovi i uđoše u svinje. I krdo  od oko dvije tisuće jurnu niz obronak u more i podavi se u moru. 
\par 14 Svinjari pobjegoše i razglasiše gradom i selima. A ljudi  pođoše vidjeti što se dogodilo. 
\par 15 Dođu Isusu. Ugledaju opsjednutoga:  sjedio je obučen i zdrave pameti - on koji ih je imao legiju.  I prestraše se. 
\par 16 A očevici im razlagahu kako je to bilo s  opsjednutim i ono o svinjama. 
\par 17 Tada ga stanu moliti da ode  iz njihova kraja. 
\par 18 Kad je ulazio u lađu, onaj što bijaše opsjednut molio  ga da bude uza nj. 
\par 19 No on mu ne dopusti, nego mu reče: "Pođi  kući k svojima pa im javi što ti je učinio Gospodin, kako ti  se smilovao." 
\par 20 On ode i poče razglašavati po Dekapolu što  mu učini Isus. I svi su se divili. 
\par 21 Kad se Isus lađom ponovno prebacio prijeko, zgrnu se  k njemu silan svijet. 
\par 22 Stajao je uz more. I dođe, gle, jedan  od nadstojnika sinagoge, imenom Jair. Ugledavši ga, padne mu  pred noge 
\par 23 pa ga usrdno moljaše: "Kćerkica mi je na umoru!  Dođi, stavi ruke na nju da ozdravi i ostane u životu!" 
\par 24 I  pođe s njima. A za njim je išao silan svijet i pritiskao ga. 
\par 25 A neka je žena dvanaest godina bolovala od krvarenja, 
\par 26 mnogo pretrpjela od pustih liječnika, razdala sve svoje  i ništa nije koristilo; štoviše, bivalo joj je sve gore. 
\par 27 Čuvši  za Isusa, priđe mu među mnoštvom odostraga i dotaknu se njegove  haljine. 
\par 28 Mislila je: "Dotaknem li se samo njegovih haljina, bit ću spašena." 
\par 29 I odmah prestane njezino krvarenje te osjeti  u tijelu da je ozdravila od zla. 
\par 30 Isus odmah u sebi osjeti da je iz njega izišla sila pa  se okrenu usred mnoštva i reče: "Tko se to dotaknu mojih haljina?" 
\par 31 A učenici mu rekoše: "Ta vidiš kako te mnoštvo odasvud pritišće  i još pitaš: 'Tko me se to dotaknu?'" 
\par 32 A on zaokruži pogledom  da vidi onu koja to učini. 
\par 33 Žena, sva u strahu i trepetu, svjesna onoga što joj se  dogodilo, pristupi i baci se preda nj pa mu kaza sve po istini. 
\par 34 On joj reče: "Kćeri, vjera te tvoja spasila! Pođi u miru  i budi zdrava od svojega zla!" 
\par 35 Dok je Isus još govorio, eto nadstojnikovih s porukom.  "Kći ti je umrla. Čemu dalje mučiti učitelja?" 
\par 36 Isus je čuo  taj razgovor, pa će nadstojniku: "Ne boj se! Samo vjeruj!" 
\par 37 I  ne dopusti da ga itko drugi prati osim Petra i Jakova i Ivana, brata Jakovljeva. 
\par 38 I dođu u kuću nadstojnikovu. Ugleda buku  i one koji plakahu i naricahu u sav glas. 
\par 39 Uđe i kaže im:  "Što bučite i plačete? Dijete nije umrlo, nego spava." 
\par 40 A  oni mu se podsmjehivahu. No on ih sve izbaci, uzme sa sobom djetetova oca i majku  i svoje pratioce pa uđe onamo gdje bijaše dijete. 
\par 41 Primi dijete  za ruku govoreći: "Talita, kum!" što znači: "Djevojko! Zapovijedam  ti, ustani!" 
\par 42 I djevojka odmah usta i poče hodati. Bijaše  joj dvanaest godina. I u tren ostadoše zapanjeni, u čudu veliku. 
\par 43 On im dobro poprijeti neka toga nitko ne dozna; i reče da  djevojci dadnu jesti. 


\chapter{6}

\par 1 I otišavši odande, dođe u svoj zavičaj. A doprate ga učenici. 
\par 2 I kada dođe subota, poče učiti u sinagogi. I mnogi što su  ga slušali preneraženi govorahu: "Odakle to ovome? Kakva li mu  je mudrost dana? I kakva se to silna djela događaju po njegovim  rukama? 
\par 3 Nije li ovo drvodjelja, sin Marijin, i brat Jakovljev, i Josipov, i Judin, i Šimunov? I nisu li mu sestre ovdje među  nama?" I sablažnjavahu se o njega. 
\par 4 A Isus im govoraše: "Nije prorok bez časti doli u svom  zavičaju i među rodbinom i u svom domu." 
\par 5 I ne mogaše ondje  učiniti ni jedno čudo, osim što ozdravi nekoliko nemoćnika stavivši  ruke na njih. 
\par 6 I čudio se njihovoj nevjeri. Obilazio je selima uokolo i naučavao. 
\par 7 Dozva dvanaestoricu te ih poče slati dva po dva dajući  im vlast nad nečistim dusima. 
\par 8 I zapovjedi im da na put ne  nose ništa osim štapa: ni kruha, ni torbe, ni novaca o pojasu, 
\par 9 nego da nose samo sandale i da ne oblače dviju haljina. 
\par 10 I govoraše im: "Kad uđete gdje u kuću, u njoj ostanite  dok ne odete odande. 
\par 11 Ako vas gdje ne prime te vas ne poslušaju, iziđite odande i otresite prah ispod svojih nogu njima za svjedočanstvo." 
\par 12 Otišavši, propovijedali su obraćenje, 
\par 13 izgonili mnoge  zloduhe i mnoge su nemoćnike mazali uljem i oni su ozdravljali. 
\par 14 Dočuo to i kralj Herod jer se razglasilo Isusovo ime  te se govorilo: "Ivan Krstitelj uskrsnuo od mrtvih i zato čudesne  sile djeluju u njemu." 
\par 15 A drugi govorahu: "Ilija je!" Treći  opet: "Prorok, kao jedan od proroka." 
\par 16 Herod pak na to govoraše:  "Uskrsnu Ivan kojemu ja odrubih glavu." 
\par 17 Herod doista bijaše dao uhititi Ivana i svezati ga u  tamnici zbog Herodijade, žene brata svoga Filipa, kojom se bio  oženio. 
\par 18 Budući da je Ivan govorio Herodu: "Ne smiješ imati  žene brata svojega!", 
\par 19 Herodijada ga mrzila i htjela ga ubiti, ali nije mogla 
\par 20 jer se Herod bojao Ivana; znao je da je on  čovjek pravedan i svet pa ga je štitio. I kad god bi ga slušao, uvelike bi se zbunio, a rado ga je slušao. 
\par 21 I dođe zgodan dan kad Herod o svom rođendanu priredi  gozbu svojim velikašima, časnicima i prvacima galilejskim. 
\par 22 Uđe  kći Herodijadina i zaplesa. Svidje se Herodu i sustolnicima.  Kralj reče djevojci: "Zaišti od mene što god hoćeš i dat ću ti!" 
\par 23 I zakle joj se: "Što god zaišteš od mene, dat ću ti, pa  bilo to i pol mojega kraljevstva." 
\par 24 Ona iziđe pa će svojoj  materi: "Što da zaištem?" A ona će: "Glavu Ivana Krstitelja!" 
\par 25 I odmah žurno uđe kralju te zaište: "Hoću da mi odmah dadeš  na pladnju glavu Ivana Krstitelja!" 
\par 26 Ožalosti se kralj, ali zbog zakletve i sustolnika na  htjede je odbiti. 
\par 27 Kralj odmah posla krvnika i naredi da donese  glavu Ivanovu. On ode, odrubi mu glavu u tamnici, 
\par 28 donese  je na pladnju i dade je djevojci, a djevojka materi. 
\par 29 Kad  za to dočuše Ivanovi učenici, dođu, uzmu njegovo tijelo i polože  ga u grob. 
\par 30 Uto se apostoli skupe oko Isusa i izvijeste ga o svemu  što su činili i naučavali. 
\par 31 I reče im: "Hajdete i vi u osamu  na samotno mjesto, i otpočinite malo." Jer mnogo je svijeta dolazilo  i odlazilo pa nisu imali kada ni jesti. 
\par 32 Otploviše dakle lađom  na samotno mjesto, u osamu. 
\par 33 No kad su odlazili, mnogi ih  vidješe i prepoznaše te se pješice iz svih gradova strčaše onamo  i pretekoše ih. 
\par 34 Kad iziđe, vidje silan svijet i sažali mu se jer bijahu  kao ovce bez pastira pa ih stane poučavati u mnogočemu. 
\par 35 A u kasni već sat pristupe mu učenici pa mu reknu: "Pust  je ovo kraj i već je kasno. 
\par 36 Otpusti ih da odu po okolnim  zaseocima i selima i kupe sebi što za jelo." 
\par 37 No on im odgovori:  "Podajte im vi jesti." Kažu mu: "Da pođemo i kupimo za dvjesta  denara kruha pa da im damo jesti?" 
\par 38 A on će im: "Koliko kruhova imate? Idite i vidite!" Pošto  izvidješe, kažu: "Pet, i dvije ribe." 
\par 39 I zapovjedi im da sve, u skupinama, posjedaju po zelenoj travi. 
\par 40 I pružiše se po  sto i po pedeset na svaku lijehu. 
\par 41 On uze pet kruhova i dvije ribe, pogleda na nebo, izreče  blagoslov pa razlomi kruhove i davaše učenicima da posluže ljude.  Tako i dvije ribe razdijeli svima. 
\par 42 I jeli su svi i nasitili se. 
\par 43 I od ulomaka nakupiše dvanaest  punih košara, a i od riba. 
\par 44 A jelo je pet tisuća muškaraca. 
\par 45 On odmah prisili učenike da uđu u lađu i da se prebace  prijeko, prema Betsaidi, dok on otpusti mnoštvo. 
\par 46 I pošto  se rasta s ljudima, otiđe u goru da se pomoli. 
\par 47 Uvečer pak lađa bijaše posred mora, a on sam na kraju. 
\par 48 Vidjevši kako se muče veslajući, jer im bijaše protivan vjetar, oko četvrte noćne straže dođe k njima hodeći po moru. I htjede  ih mimoići. 
\par 49 A oni, vidjevši kako hodi po moru, pomisliše  da je utvara pa kriknuše. 
\par 50 Jer svi su ga vidjeli i prestrašili  se. A on im odmah progovori: "Hrabro samo! Ja sam! Ne bojte se!" 
\par 51 I uziđe k njima u lađu, a vjetar utihnu. I veoma se, prekomjerno, snebivahu; 
\par 52 još ne shvatiše ono o kruhovima, nego im srce  bijaše stvrdnuto. 
\par 53 Pošto doploviše na kraj, dođu u Genezaret i pristanu. 
\par 54 Kad iziđu iz lađe, ljudi ga odmah prepoznaju 
\par 55 pa oblete  sav onaj kraj. I počnu donositi na nosilima bolesnike onamo gdje  bi čuli da se on nalazi. 
\par 56 I kamo bi god ulazio - u sela, u  gradove, u zaseoke - po trgovima bi stavljali bolesnike i molili  ga da se dotaknu makar skuta njegove haljine. I koji bi ga se  god dotakli, ozdravljali bi. 


\chapter{7}

\par 1 Skupe se oko njega farizeji i neki od pismoznanaca koji dođoše  iz Jeruzalema. 
\par 2 I opaze da neki njegovi učenici jedu kruh nečistih, to jest neopranih ruku. 
\par 3 A farizeji i svi Židovi ne jedu ako prije temeljito ne  operu ruke; drže se predaje starih. 
\par 4 Niti s trga što jedu ako  prije ne operu. Mnogo toga još ima što zbog predaje drže: pranje  čaša, vrčeva i lonaca. 
\par 5 Zato farizeji i pismoznanci upitaju Isusa: "Zašto tvoji  učenici ne postupaju po predaji starih, nego nečistih ruku blaguju?" 
\par 6 A on im reče: "Dobro prorokova Izaija o vama, licemjeri, kad napisa: Ovaj me narod usnama časti, a srce mu je daleko od mene. 
\par 7 Uzalud me štuju naučavajući nauke - uredbe ljudske. 
\par 8 Napustili ste zapovijed Božju, a držite se predaje ljudske." 
\par 9 Još im govoraše: "Lijepo! Dokidate Božju zapovijed da  biste sačuvali svoju predaju. 
\par 10 Mojsije doista reče: Poštuj  oca svoga i majku svoju. I: Tko prokune oca ili majku, smrću neka se kazni. 
\par 11 A vi velite: 'Rekne li tko ocu  ili majci: Pomoć koja te od mene ide neka bude 'korban', to jest  sveti dar', 
\par 12 takvome više ne dopuštate ništa učiniti za oca  ili majku. 
\par 13 Tako dokidate riječ Božju svojom predajom, koju  sami sebi predadoste. I još štošta tomu slično činite." 
\par 14 Tada  ponovno dozove mnoštvo i stane govoriti: "Poslušajte me svi i  razumijte! 
\par 15 Ništa što izvana ulazi u čovjeka ne može ga onečistiti, nego što iz čovjeka izlazi - to ga onečišćuje. 
\par 16 Tko ima uši  da čuje, neka čuje!" 
\par 17 I kad od mnoštva uđe u kuću, upitaše ga učenici za prispodobu. 
\par 18 I reče im: "Tako? Ni vi ne razumijete? Ne shvaćate li da  čovjeka ne može onečistiti što u nj ulazi 
\par 19 jer mu ne ulazi  u srce, nego u utrobu te izlazi u zahod?" Tako on očisti sva  jela. 
\par 20 Još dometnu: "Što iz čovjeka izlazi, te onečišćuje čovjeka. 
\par 21 Ta iznutra, iz srca čovječjega, izlaze zle namisli, bludništva, krađe, ubojstva, 
\par 22 preljubi, lakomstva, opakosti, prijevara, razuzdanost, zlo oko, psovka, uznositost, bezumlje. 
\par 23 Sva  ta zla iznutra izlaze i onečišćuju čovjeka." 
\par 24 Odande otiđe u kraj tirski. I uđe u neku kuću. Htio je  da nitko ne sazna, ali se nije mogao sakriti, 
\par 25 nego odmah  doču žena koje kćerkica imaše duha nečistoga. Ona dođe i pade  mu pred noge. 
\par 26 A žena bijaše Grkinja, Sirofeničanka rodom.  I moljaše ga da joj iz kćeri istjera zloduha. 
\par 27 A on joj govoraše:  "Pusti da se prije nasite djeca! Ne priliči uzeti kruh djeci  i baciti ga psićima." 
\par 28 A ona će mu: "Da, Gospodine! Ali i  psići ispod stola jedu od mrvica dječjih." 
\par 29 Reče joj: "Zbog  te riječi idi, izišao je iz tvoje kćeri zloduh." 
\par 30 I ode kući  te nađe dijete gdje leži na postelji, a zloduh je bio izišao. 
\par 31 Zatim se ponovno vrati iz krajeva tirskih pa preko Sidona  dođe Galilejskom moru, u krajeve dekapolske. 
\par 32 Donesu mu nekoga gluhog mucavca pa ga zamole da stavi  na nj ruku. 
\par 33 On ga uzme nasamo od mnoštva, utisne svoje prste  u njegove uši, zatim pljune i dotakne se njegova jezika. 
\par 34 Upravi  pogled u nebo, uzdahne i kaže mu: "Effata!" - to će reći: "Otvori  se!" 
\par 35 I odmah mu se otvoriše uši i razdriješi spona jezika  te stade govoriti razgovijetno. 
\par 36 A Isus im zabrani da nikome ne kazuju. No što im je on  više branio, oni su to više razglašavali 
\par 37 i preko svake mjere  zadivljeni govorili: "Dobro je sve učinio! Gluhima daje čuti, nijemima govoriti!" 


\chapter{8}

\par 1 Onih se dana opet skupio silan svijet. Budući da nisu imali  što jesti, dozva Isus učenike pa im reče: 
\par 2 "Žao mi je naroda  jer su već tri dana uza me i nemaju što jesti. 
\par 3 Ako ih otpravim  gladne njihovim kućama, klonut će putom. A neki su od njih došli  iz daleka." 
\par 4 Učenici mu odgovore: "Otkuda bi ih tko ovdje u  pustinji mogao nahraniti kruhom?" 
\par 5 On ih zapita: "Koliko kruhova  imate?" Oni odgovore: "Sedam." 
\par 6 Nato zapovjedi mnoštvu da posjeda  po zemlji. I uze sedam kruhova, zahvali, razlomi i davaše svojim  učenicima da posluže. I poslužiše mnoštvu. 
\par 7 A imali su i malo  ribica. Blagoslovi i njih te reče da i to posluže. 
\par 8 I jeli  su i nasitili se. A od preteklih ulomaka odniješe sedam košara. 
\par 9 Bilo ih je oko četiri tisuće. Tada ih otpusti, 
\par 10 a sam sa  svojim učenicima odmah uđe u lađu i ode u kraj dalmanutski. 
\par 11 Tada istupiše farizeji i počeše raspravljati s njime.  Iskušavajući ga, zatraže od njega znak s neba. 
\par 12 On uzdahnu  iz sve duše i reče: "Zašto ovaj naraštaj traži znak? Zaista,  kažem vam, ovome se naraštaju neće dati znak." 
\par 13 Tada ih ostavi, ponovno uđe u lađu pa otiđe prijeko. 
\par 14 A zaboraviše ponijeti kruha; imali su samo jedan kruh  sa sobom na lađi. 
\par 15 Nato ih Isus opomenu: "Pazite, čuvajte  se kvasca farizejskog i kvasca Herodova!" 
\par 16 Oni, zamišljeni, među sobom govorahu: "Kruha nemamo." 
\par 17 Zamijetio to Isus pa  im reče: "Zašto ste zamišljeni što kruha nemate? Zar još ne shvaćate  i ne razumijete? Zar vam je srce stvrdnuto? 
\par 18 Oči imate, a ne vidite; uši imate, a ne čujete? Zar se ne sjećate? 
\par 19 Kad sam ono razlomio pet kruhova na  pet tisuća, koliko punih košara ulomaka odnijeste?" Kažu mu:  "Dvanaest." 
\par 20 "A kada razlomih sedam na četiri tisuće, koliko  punih košara ulomaka odnijeste?" Odgovore: "Sedam." 
\par 21 A on  će njima: "I još ne razumijete?" 
\par 22 Dođu u Betsaidu, dovedu mu slijepca pa ga zamole da ga  se dotakne. 
\par 23 On uhvati slijepca za ruku, izvede ga iz sela, pljunu mu u oči, stavi na nj ruke i zapita ga: "Vidiš li što?" 
\par 24 Slijepac upilji pogled i reče: "Opažam ljude; vidim nešto  kao drveće ... hodaju." 
\par 25 Tada mu Isus opet stavi ruke na oči  i slijepac progleda i ozdravi te je mogao sve jasno na daleko  vidjeti. 
\par 26 Tada ga posla kući i reče mu: "Ne ulazi u selo." 
\par 27 I krenu Isus i njegovi učenici u sela Cezareje Filipove.  Putem on upita učenike: "Što govore ljudi, tko sam ja?" 
\par 28 Oni  mu rekoše: "Da si Ivan Krstitelj, drugi da si Ilija, treći opet  da si neki od proroka." 
\par 29 On njih upita: "A vi, što vi kažete, tko sam ja?" Petar prihvati i reče: "Ti si Pomazanik - Krist!" 
\par 30 I zaprijeti im da nikomu ne kazuju o njemu. 
\par 31 I poče ih poučavati kako Sin Čovječji treba da mnogo  pretrpi, da ga starješine, glavari svećenički i pismoznanci odbace, da bude ubijen i nakon tri dana da ustane. 
\par 32 Otvoreno im to  govoraše. Petar ga uze u stranu i poče odvraćati. 
\par 33 A on se  okrenu, pogleda svoje učenike pa zaprijeti Petru: "Nosi se od  mene, sotono, jer ti nije na pameti što je Božje, nego što je  ljudsko!" 
\par 34 Tada dozva narod i učenike pa im reče: "Hoće li tko za  mnom, neka se odrekne samoga sebe, neka uzme svoj križ i neka  ide za mnom. 
\par 35 Tko hoće život svoj spasiti, izgubit će ga;  a tko izgubi život svoj poradi mene i evanđelja, spasit će ga. 
\par 36 Ta što koristi čovjeku steći sav svijet, a životu svojemu  nauditi? 
\par 37 Ta što da čovjek dadne u zamjenu za život svoj? 
\par 38 Doista, tko se zastidi mene i mojih riječi u ovom preljubničkom  i grešničkom naraštaju - njega će se stidjeti i Sin Čovječji  kada dođe u slavi Oca svoga zajedno sa svetim anđelima." 


\chapter{9}

\par 1 Još im govoraše: "Zaista, kažem vam, neki od ovdje nazočnih  neće okusiti smrti dok ne vide da je kraljevstvo Božje došlo  u sili." 
\par 2 Nakon šest dana uze Isus sa sobom Petra, Jakova i Ivana  i povede ih na goru visoku, u osamu, same, i preobrazi se pred  njima. 
\par 3 I haljine mu postadoše sjajne, bijele veoma - nijedan  ih bjelilac na zemlji ne bi mogao tako izbijeliti. 
\par 4 I ukaza  im se Ilija s Mojsijem te razgovarahu s Isusom. 
\par 5 A Petar prihvati i reče Isusu: "Učitelju, dobro nam je  ovdje biti! Načinimo tri sjenice: tebi jednu, Mojsiju jednu i  Iliji jednu." 
\par 6 Doista nije znao što da kaže jer bijahu prestrašeni. 
\par 7 I pojavi se oblak i zasjeni ih, a iz oblaka se začu glas:  "Ovo je Sin moj, Ljubljeni! Slušajte ga!" 
\par 8 I odjednom, obazrevši se uokolo, nikoga uza se ne vidješe doli Isusa sama. 
\par 9 Dok su silazili s gore, naloži im da nikomu ne pripovijedaju  što su vidjeli dok Sin Čovječji od mrtvih ne ustane. 
\par 10 Oni  održaše tu riječ, ali se među sobom pitahu što znači to njegovo  "od mrtvih ustati" 
\par 11 pa ga upitaju: "Zašto pismoznanci govore  da prije treba da dođe Ilija?" 
\par 12 A on im reče: "Ilija će,  doduše, prije doći i sve obnoviti. Pa kako ipak piše  o Sinu Čovječjem da će mnogo pretrpjeti i biti prezren? 
\par 13 Ali, velim vam: Ilija je već došao i oni učiniše s njim što im se  prohtjelo, kao što piše o njemu." 
\par 14 Kada dođoše k učenicima, ugledaše oko njih silan svijet  i pismoznance kako raspravljaju s njima. 
\par 15 Čim ga sve ono mnoštvo  ugleda, iznenađeno brže pohrli pozdraviti ga. 
\par 16 A on ih upita:  "Što to raspravljaste s njima?" 
\par 17 Odvrati netko iz mnoštva:  "Učitelju, dovedoh k tebi svoga sina koji ima nijemoga duha. 
\par 18 Gdje ga god zgrabi, obara ga, a on pjeni, škripi zubima i  koči se. Rekoh tvojim učenicima da ga izagnaju, ali ne mogoše." 
\par 19 On im odvrati: "O rode nevjerni! Dokle mi je biti s vama?  Dokle li vas podnositi? Dovedite ga k meni!" 
\par 20 I dovedoše ga  k njemu. Čim zloduh ugleda Isusa, potrese dječakom i on se, oboren  na zemlju, stane valjati i pjeniti. 
\par 21 Isus upita njegova oca:  "Koliko je vremena kako mu se to događa?" On reče: "Od djetinjstva! 
\par 22 A često ga znade baciti i u vatru i u vodu da ga upropasti.  Nego, ako što možeš, pomozi nam, imaj samilosti s nama!" 
\par 23 Nato  mu Isus reče: "Što? Ako možeš? Sve je moguće onomu koji vjeruje!" 
\par 24 Dječakov otac brže povika: "Vjerujem! Pomozi mojoj nevjeri!" 
\par 25 Vidjevši da svijet odasvud grne, Isus zaprijeti nečistomu  duhu: "Nijemi i gluhi duše, ja ti zapovijedam, iziđi iz njega  i da nisi više u nj ušao!" 
\par 26 Zloduh nato zaviče, žestoko strese  dječaka te iziđe, a on osta kao mrtav te su mnogi govorili da  je umro. 
\par 27 No Isus ga dohvati za ruku, podiže ga i on ustade. 
\par 28 Kad Isus uđe u kuću, upitaše ga učenici nasamo: "Kako  to da ga mi ne mogosmo izagnati?" 
\par 29 Odgovori im: "Ovaj se rod  ničim drugim ne može izagnati osim molitvom i postom." 
\par 30 Otišavši odande, prolažahu kroz Galileju. On ne htjede  da to itko sazna. 
\par 31 Jer poučavaše svoje učenike. Govoraše im:  "Sin Čovječji predaje se u ruke ljudima. Ubit će ga, ali će on, ubijen, nakon tri dana ustati." 
\par 32 No oni ne razumješe te besjede, a bojahu ga se pitati. 
\par 33 I dođoše u Kafarnaum. I već u kući upita ih: "Što ste  putem raspravljali?" 
\par 34 A oni umukoše jer putem među sobom razgovarahu  o tome tko je najveći. 
\par 35 On sjede i dozove dvanaestoricu te  im reče: "Ako tko želi biti prvi, neka bude od svih posljednji  i svima poslužitelj!" 
\par 36 I uzme dijete, postavi ga posred njih, zagrli ga i reče im: 
\par 37 "Tko god jedno ovakvo dijete primi  u moje ime, mene prima. A tko mene prima, ne prima mene, nego  onoga koji mene posla." 
\par 38 Reče mu Ivan: "Učitelju, vidjesmo jednoga kako u tvoje  ime izgoni zloduhe. Mi smo mu branili jer ne ide s nama." 
\par 39 A  Isus reče: "Ne branite mu! Jer nitko ne može učiniti nešto silno  u moje ime pa me ubrzo zatim pogrditi. 
\par 40 Tko nije protiv nas, za nas je." 
\par 41 "Uistinu, tko vas napoji čašom vode u ime toga što ste  Kristovi, zaista, kažem vam, neće mu propasti plaća." 
\par 42 "Onomu naprotiv tko bi sablaznio jednoga od ovih najmanjih  što vjeruju, daleko bi bolje bilo da s mlinskim kamenom o vratu  bude bačen u more." 
\par 43 "Pa ako te ruka sablažnjava, odsijeci je. Bolje ti je  sakatu ući u život, nego s obje ruke otići u pakao, u oganj neugasivi. 
\par 44 # 
\par 45 I ako te noga sablažnjava, odsijeci je. Bolje ti je  hromu ući u život, nego s obje noge bit bačen u pakao. 
\par 46 # 
\par 47 I ako te oko sablažnjava, iskopaj ga. Bolje ti je jednooku  ući u kraljevstvo Božje, nego s oba oka biti bačen u pakao, 
\par 48 gdje  crv njihov ne gine niti se oganj gasi. 
\par 49 Uistinu, ognjem će svaki od njih biti posoljen. 
\par 50 Dobra je sol. Ali ako sol postane neslana, čime ćete  nju začiniti? Imajte sol u sebi, a mir među sobom!" 


\chapter{10}

\par 1 Krenuvši odande, dođe u judejski kraj i na onu stranu Jordana.  I opet mnoštvo nagrnu k njemu, a on ih po svojem običaju ponovno  poučavaše. 
\par 2 A pristupe farizeji pa, da ga iskušaju, upitaše: "Je li  mužu dopušteno otpustiti ženu?" 
\par 3 On im odgovori: "Što vam zapovjedi  Mojsije?" 
\par 4 Oni rekoše: "Mojsije je dopustio napisati otpusno  pismo i - otpustiti." 
\par 5 A Isus će im: "Zbog okorjelosti  srca vašega napisa vam on tu zapovijed. 
\par 6 Od početka stvorenja  muško i žensko stvori ih. 
\par 7 Stoga će čovjek ostaviti oca  i majku da prione uza svoju ženu; 
\par 8 i dvoje njih bit će jedno  tijelo. Tako više nisu dvoje, nego jedno tijelo. 
\par 9 Što dakle  Bog združi, čovjek neka ne rastavlja!" 
\par 10 U kući su ga učenici ponovno o tome ispitivali. 
\par 11 I  reče im: "Tko otpusti svoju ženu pa se oženi drugom, čini prema  prvoj preljub. 
\par 12 I ako žena napusti svoga muža pa se uda za  drugoga, čini preljub." 
\par 13 Donosili mu dječicu da ih se dotakne, a učenici im branili. 
\par 14 Opazivši to, Isus se ozlovolji i reče im: "Pustite dječicu  neka dolaze k meni; ne priječite im jer takvih je kraljevstvo  Božje! 
\par 15 Zaista, kažem vam, tko ne primi kraljevstva Božjega  kao dijete, ne, u nj neće ući." 
\par 16 Nato ih zagrli pa ih blagoslivljaše  polažući na njih ruke. 
\par 17 I dok je izlazio na put, dotrči netko, klekne preda nj  pa ga upita: "Učitelju dobri, što mi je činiti da baštinim život  vječni?" 
\par 18 Isus mu reče: "Što me zoveš dobrim? Nitko nije dobar  doli Bog jedini! 
\par 19 Zapovijedi znadeš: Ne ubij! Ne čini preljuba! Ne ukradi! Ne svjedoči lažno! Ne otmi! Poštuj oca svoga i majku!" 
\par 20 On mu odgovori: "Učitelju, sve sam to čuvao od svoje  mladosti." 
\par 21 Isus ga nato pogleda, zavoli ga i rekne mu: "Jedno  ti nedostaje! Idi i što imaš, prodaj i podaj siromasima pa ćeš  imati blago na nebu. A onda dođi i idi za mnom." 
\par 22 On se na  tu riječ smrkne i ode žalostan jer imaše velik imetak. 
\par 23 Isus zaokruži pogledom pa će svojim učenicima: "Kako  li će teško imućnici u kraljevstvo Božje!" 
\par 24 Učenici ostadoše  zapanjeni tim njegovim riječima. Zato im Isus ponovi: "Djeco, kako je teško u kraljevstvo Božje! 
\par 25 Lakše je devi kroz ušice  iglene nego bogatašu u kraljevstvo Božje." 
\par 26 Oni se još većma snebivahu te će jedan drugome: "Pa tko  se onda može spasiti?" 
\par 27 Isus upre u njih pogled i reče: "Ljudima  je nemoguće, ali ne Bogu! Ta Bogu je sve moguće!" 
\par 28 Petar mu poče govoriti: "Evo, mi sve ostavismo i pođosmo  za tobom." 
\par 29 Reče Isus: "Zaista, kažem vam, nema ga tko ostavi  kuću, ili braću, ili sestre, ili majku, ili oca, ili djecu, ili  polja poradi mene i poradi evanđelja, 
\par 30 a da ne bi sada, u  ovom vremenu, s progonstvima primio stostruko kuća, i braće,  i sestara, i majki, i djece, i polja - i u budućem vijeku život  vječni. 
\par 31 A mnogi prvi bit će posljednji i posljednji prvi." 
\par 32 Putovali su tako uzlazeći u Jeruzalem. Isus je išao pred  njima te bijahu zaprepašteni, a oni koji su išli za njima, prestrašeni. Tada Isus opet uze dvanaestoricu i poče im kazivati što će  ga zadesiti: 
\par 33 "Evo, uzlazimo u Jeruzalem i Sin Čovječji bit  će predan glavarima svećeničkim i pismoznancima. Osudit će ga  na smrt, predati poganima, 
\par 34 izrugati i popljuvati. Izbičevat  će ga, ubit će ga, ali on će nakon tri dana ustati." 
\par 35 I pristupe mu Jakov i Ivan, sinovi Zebedejevi, govoreći  mu: "Učitelju, htjeli bismo da nam učiniš što te zaištemo." 
\par 36 A  on će im: "Što hoćete da vam učinim?" 
\par 37 Oni mu rekoše: "Daj  nam da ti u slavi tvojoj sjednemo jedan zdesna, a drugi slijeva." 
\par 38 A Isus im reče: "Ne znate što ištete. Možete li piti čašu  koju ja pijem, ili krstiti se krstom kojim se ja krstim?" 
\par 39 Oni  mu rekoše: "Možemo." A Isus će im: "Čašu koju ja pijem pit ćete  i krstom kojim se ja krstim bit ćete kršteni, 
\par 40 ali sjesti  meni zdesna ili slijeva nisam ja vlastan dati - to je onih kojima  je pripravljeno." 
\par 41 Kad su to čula ostala desetorica, počeše se gnjeviti  na Jakova i Ivana. 
\par 42 Zato ih Isus dozva i reče im: "Znate da  oni koji se smatraju vladarima gospoduju svojim narodima i velikaši  njihovi drže ih pod vlašću. 
\par 43 Nije tako među vama! Naprotiv, tko hoće da među vama bude najveći, neka vam bude poslužitelj! 
\par 44 I tko hoće da među vama bude prvi, neka bude svima sluga. 
\par 45 Jer ni Sin Čovječji nije došao da bude služen, nego da služi  i život svoj dade kao otkupninu za mnoge." 
\par 46 Dođu tako u Jerihon. Kad je Isus s učenicima i sa silnim  mnoštvom izlazio iz Jerihona, kraj puta je sjedio slijepi prosjak  Bartimej, sin Timejev. 
\par 47 Kad je čuo da je to Isus Nazarećanin, stane vikati: "Sine Davidov, Isuse, smiluj mi se!" 
\par 48 Mnogi  ga ušutkivahu, ali on još jače vikaše: "Sine Davidov, smiluj  mi se!" 
\par 49 Isus se zaustavi i reče: "Pozovite ga!" I pozovu  slijepca sokoleći ga: "Ustani! Zove te!" 
\par 50 On baci sa sebe  ogrtač, skoči i dođe Isusu. 
\par 51 Isus ga upita: "Što hoćeš da  ti učinim?" Slijepac mu reče: "Učitelju moj, da progledam." 
\par 52 Isus  će mu: "Idi, vjera te tvoja spasila!" I on odmah progleda i uputi  se za njim. 


\chapter{11}

\par 1 Kad se približe Jeruzalemu, Betfagi i Betaniji, do Maslinske  gore, pošalje dva učenika 
\par 2 i kaže im: "Hajdete u selo pred  vama. Čim u nj uđete, naći ćete privezano magare koje još nitko  nije zajahao. Odriješite ga i vodite. 
\par 3 Ako vam tko reče: 'Što  to radite?' recite: 'Gospodinu treba', i odmah će ga ipak ovamo  pustiti." 
\par 4 Otiđoše i nađoše magare privezano uz vrata vani na cesti  i odriješe ga. 
\par 5 A neki od nazočnih upitaše: "Što radite? Što  driješite magare?" 
\par 6 Oni im odvrate kako im reče Isus. I pustiše  ih. 
\par 7 I dovedu magare Isusu, prebace preko njega svoje haljine  i on zajaha na nj. 
\par 8 Mnogi prostriješe svoje haljine po putu, a drugi narezaše zelenih grana po poljima. 
\par 9 I oni pred njim  i oni za njim klicahu: "Hosana! Blagoslovljen Onaj koji dolazi  u ime Gospodnje! 
\par 10 Blagoslovljeno kraljevstvo oca našega  Davida koji dolazi! Hosana u visinama!" 
\par 11 I uđe u Jeruzalem, u Hram. I sve uokolo razgleda, pa  kako već bijaše kasno, pođe s dvanaestoricom u Betaniju. 
\par 12 Sutradan su izlazili iz Betanije, a on ogladnje. 
\par 13 Ugleda  izdaleka lisnatu smokvu i priđe ne bi li na njoj što našao. Ali  došavši bliže, ne nađe ništa osim lišća jer ne bijaše vrijeme  smokvama. 
\par 14 Tada reče smokvi: "Nitko nikada više ne jeo s tebe!"  Čuli su to njegovi učenici. 
\par 15 Stignu tako u Jeruzalem. On uđe u Hram i stane izgoniti  one koji su prodavali i kupovali u Hramu. Mjenjačima isprevrta  stolove i prodavačima golubova klupe. 
\par 16 I ne dopusti da itko  išta pronese kroz Hram. 
\par 17 Učio ih je i govorio: "Nije li pisano:  Dom će se moj zvati Dom molitve za sve narode? A vi od  njega načinili pećinu razbojničku!" 
\par 18 Kada su za to dočuli glavari svećenički i pismoznanci, tražili su kako da ga pogube. Uistinu, bojahu ga se jer je sav  narod bio očaran njegovim naukom. 
\par 19 A kad se uvečerilo, izlazili  su iz grada. 
\par 20 Kad su ujutro prolazili mimo one smokve, opaze da je  usahla do korijena. 
\par 21 Petar se prisjeti pa će Isusu: "Učitelju, pogledaj! Smokva koju si prokleo usahnu." 
\par 22 Isus im odvrati: "Imajte vjeru Božju. 
\par 23 Zaista, kažem  vam, rekne li tko ovoj gori: 'Digni se i baci u more!' i u srcu  svome ne posumnja, nego vjeruje da će se dogoditi to što kaže  - doista, bit će mu! 
\par 24 Stoga vam kažem: Sve što god zamolite  i zaištete, vjerujte da ste postigli i bit će vam! 
\par 25 No kad  ustanete na molitvu, otpustite ako što imate protiv koga da i  vama Otac vaš, koji je na nebesima, otpusti vaše prijestupke." 
\par 26 # 
\par 27 I dođu opet u Jeruzalem. Dok je obilazio Hramom, dođu  k njemu glavari svećenički, pismoznanci i starješine. 
\par 28 I govorahu  mu: "Kojom vlašću to činiš? Ili tko ti dade tu vlast da to činiš?" 
\par 29 A Isus im reče: "Jedno ću vas upitati. Odgovorite mi, pa  ću vam kazati kojom vlašću ovo činim. 
\par 30 Krst Ivanov bijaše  li od Neba ili od ljudi? Odgovorite mi!" 
\par 31 A oni umovahu među  sobom: "Reknemo li 'od Neba', odvratit će 'Zašto mu dakle ne  povjerovaste?' 
\par 32 Nego, da reknemo 'od ljudi!'" - Bojahu se  mnoštva. Ta svi Ivana smatrahu doista prorokom. 
\par 33 I odgovore  Isusu: "Ne znamo." A Isus će im: "Ni ja vama neću kazati kojom  vlašću ovo činim." 


\chapter{12}

\par 1 I uze im zboriti u prispodobama: Čovjek vinograd posadi, ogradom ogradi, iskopa tijesak i kulu podiže pa ga iznajmi  vinogradarima i otputova. 
\par 2 I u svoje vrijeme posla vinogradarima  slugu da od njih uzme dio uroda vinogradarskoga. 
\par 3 A oni ga  pograbiše, istukoše i otposlaše praznih ruku. 
\par 4 I opet posla  k njima drugog slugu: i njemu razbiše glavu i izružiše ga. 
\par 5 Trećega  također posla: njega ubiše. Tako i mnoge druge: jedne istukoše, druge pobiše." 
\par 6 "Još jednoga imaše, sina ljubljenoga. Njega naposljetku  posla k njima misleći: 'Poštovat će sina moga.' 
\par 7 Ali ti vinogradari  među sobom rekoše: 'Ovo je baštinik! Hajde da ga ubijemo i baština  će biti naša.' 
\par 8 I pograbe ga, ubiju i izbace iz vinograda." 
\par 9 "Što li će učiniti gospodar vinograda? Doći će i pobiti  te vinogradare i dati vinograd drugima. 
\par 10 Niste li čitali ovo Pismo: Kamen što ga odbaciše graditelji, postade kamen zaglavni. 
\par 11 Gospodnje je to djelo - kakvo čudo u očima našim!" 
\par 12 I tražili su da ga uhvate, ali se pobojaše mnoštva. Razumješe  da je protiv njih izrekao prispodobu pa ga ostave i odu. 
\par 13 I pošalju k njemu neke od farizeja i herodovaca da ga  uhvate u riječi. 
\par 14 Oni dođu i kažu mu: "Učitelju, znamo da  si istinit i ne mariš tko je tko jer nisi pristran, nego po istini  učiš putu Božjemu. Je li dopušteno dati porez caru ili nije?  Da damo ili da ne damo?" 
\par 15 A on im reče prozirući njihovo licemjerje:  "Što me iskušavate? Donesite mi denar da vidim!" 
\par 16 Oni doniješe.  I reče im: "Čija je ovo slika i natpis?" A oni će mu: "Carev." 
\par 17 A Isus im reče: "Caru podajte carevo, a Bogu Božje!" I divili  su mu se. 
\par 18 Dođu k njemu saduceji, koji vele da nema uskrsnuća, i  upitaju ga: 
\par 19 "Učitelju, Mojsije nam napisa: Umre li čiji  brat i ostavi ženu, a ne ostavi djeteta, neka njegov brat  uzme tu ženu te podigne porod bratu svomu. 
\par 20 Sedmero braće  bijaše. Prvi uze ženu i umrije ne ostavivši poroda. 
\par 21 I drugi  je uze te umrije ne ostavivši poroda. I treći jednako tako. 
\par 22 I  sedmorica ne ostaviše poroda. Najposlije i žena umrije. 
\par 23 Komu  će biti žena o uskrsnuću, kad uskrsnu? Jer sedmorica su je imala  za ženu." 
\par 24 Reče im Isus: "Niste li u zabludi zbog toga što ne razumijete  Pisama ni sile Božje? 
\par 25 Ta kad od mrtvih ustaju, niti se žene  niti udavaju, nego su kao anđeli na nebesima. 
\par 26 A što se tiče  mrtvih, da ustaju, niste li čitali u knjizi Mojsijevoj ono o  grmu, kako Mojsiju reče Bog: Ja sam Bog Abrahamov i Bog Izakov  i Bog Jakovljev? 
\par 27 Nije on Bog mrtvih, nego živih. Uvelike  se varate." 
\par 28 Tada pristupi jedan od pismoznanaca koji je slušao njihovu  raspravu. Vidjevši da im je dobro odgovorio, upita ga: "Koja  je zapovijed prva od sviju?" 
\par 29 Isus odgovori: "Prva je: Slušaj, Izraele! Gospodin Bog naš Gospodin je jedini. 
\par 30 Zato ljubi  Gospodina Boga svojega iz svega srca svojega, i iz sve duše svoje,  i iz svega uma svoga, i iz sve snage svoje!" 
\par 31 "Druga je: Ljubi svoga bližnjega kao sebe samoga.  Nema druge zapovijedi veće od tih." 
\par 32 Nato će mu pismoznanac: "Dobro, učitelju! Po istini si  kazao: On je jedini, nema drugoga osim njega. 
\par 33 Njega ljubiti  iz svega srca, iz svega razuma i iz sve snage i ljubiti bližnjega  kao sebe samoga - više je nego sve paljenice i žrtve." 
\par 34 Kad Isus vidje kako je pametno odgovorio, reče mu: "Nisi  daleko od kraljevstva Božjega!" I nitko se više nije usuđivao  pitati ga. 
\par 35 A naučavajući u Hramu, uze Isus govoriti: "Kako pismoznanci  kažu da je Krist sin Davidov? 
\par 36 A sam David reče u Duhu Svetome: Reče Gospod Gospodinu mojemu: 'Sjedni mi zdesna dok ne položim neprijatelje tvoje za podnožje nogama tvojim!' 
\par 37 Sam ga David zove Gospodinom. Kako mu je onda sin?" Silan ga je svijet s užitkom slušao. 
\par 38 A on im u pouci  svojoj govoraše: "Čuvajte se pismoznanaca, koji rado idu u dugim  haljinama, vole pozdrave na trgovima, 
\par 39 prva sjedala u sinagogama  i pročelja na gozbama; 
\par 40 proždiru kuće udovičke, još pod izlikom  dugih molitava. Stići će ih to oštrija osuda!" 
\par 41 Potom sjede nasuprot riznici te promatraše kako narod  baca sitniš u riznicu. Mnogi bogataši bacahu mnogo. 
\par 42 Dođe  i neka siromašna udovica i baci dva novčića, to jest jedan kvadrant. 
\par 43 Tada dozove svoje učenike i reče im: "Doista, kažem vam,  ova je sirota udovica ubacila više od svih koji ubacuju u riznicu. 
\par 44 Svi su oni zapravo ubacili od svoga suviška, a ona je od  svoje sirotinje ubacila sve što je imala, sav svoj žitak." 


\chapter{13}

\par 1 Kad je izlazio iz Hrama, rekne mu jedan od njegovih učenika:  "Učitelju, gledaj! Kakva li kamenja, kakvih li zdanja!" 
\par 2 Isus  mu odvrati: "Vidiš li ta veličanstvena zdanja? Ne, neće se ostaviti  ni kamen na kamenu nerazvaljen." 
\par 3 Dok je zatim na Maslinskoj gori sjedio sučelice Hramu, upitaju ga nasamo Petar, Jakov, Ivan i Andrija: 
\par 4 "Reci nam  kada će to biti i na koji se znak sve to ima svršiti?" 
\par 5 Tada im Isus poče govoriti: "Pazite da vas tko ne zavede. 
\par 6 Mnogi će doći u moje ime i govoriti: Ja sam! I mnoge  će zavesti. 
\par 7 Kada pak čujete za ratove i za glasove o ratovima, ne uznemirujte se. Treba da se to dogodi, ali to još nije svršetak." 
\par 8 "Narod će ustati protiv naroda, kraljevstvo protiv  kraljevstva. Bit će potresa po raznim mjestima, bit će gladi.  To je početak trudova." 
\par 9 "Vi pak pazite sami na sebe. Predavat će vas vijećima  i tući vas u sinagogama, pred upraviteljima i kraljevima stajat  ćete zbog mene, njima za svjedočanstvo. 
\par 10 A treba da se najprije  svim narodima propovijeda evanđelje." 
\par 11 "Kad vas budu vodili na izručenje, ne brinite se unaprijed  što ćete govoriti, nego govorite što vam bude dano u onaj čas.  Ta niste vi koji govorite, nego Duh Sveti. 
\par 12 Predavat će na  smrt brat brata i otac sina. Djeca će ustajati na roditelje  i ubijati ih. 
\par 13 Svi će vas zamrziti zbog imena moga. Ali tko  ustraje do svršetka, bit će spašen." 
\par 14 "I kad vidite da grozota pustoši stoluje gdje  joj nije mjesto - tko čita, neka razumije - koji se tada zateknu  u Judeji, neka bježe u gore! 
\par 15 Tko bude na krovu, neka ne silazi  i ne ulazi u kuću da iz nje što uzme. 
\par 16 I tko bude u polju, neka se ne okreće natrag da uzme ogrtač!" 
\par 17 "Jao trudnicama i dojiljama u one dane! 
\par 18 A molite  da to ne bude zimi 
\par 19 jer će onih dana biti tjeskoba kakve  ne bi od početka stvorenja, koje stvori Bog, sve do sada,  a neće je ni biti. 
\par 20 I kad Gospodin ne bi skratio dane  one, nitko se ne bi spasio. No poradi izabranih, koje on sebi  izabra, skratio je on te dane." 
\par 21 Ako vam tada tko rekne: 'Evo Krista ovdje! Eno ondje!'  - ne vjerujte. 
\par 22 Ustat će doista lažni kristi i lažni proroci  i tvorit će znamenja i čudesa da, bude li moguće, zavedu  izabrane. 
\par 23 Vi dakle budite na oprezu! Evo, prorekao sam vam  sve!" 
\par 24 Nego, u one dane, nakon one nevolje, sunce će pomrčati i mjesec neće više svijetljeti 
\par 25 a zvijezde će s neba padati i sile će se nebeske poljuljati. 
\par 26 Tada će ugledati Sina Čovječjega gdje dolazi na oblacima  s velikom moći i slavom. 
\par 27 I razaslat će anđele i sabrati  svoje izabranike s četiri vjetra, s kraja zemlje do  na kraj neba." 
\par 28 A od smokve se naučite prispodobi! Kad joj grana već  omekša i lišće potjera, znate: ljeto je blizu. 
\par 29 Tako i vi  kad vidite da se to zbiva, znajte: blizu je, na vratima! 
\par 30 Zaista, kažem vam, ne, neće uminuti naraštaj ovaj dok se sve to ne zbude. 
\par 31 Nebo će i zemlja uminuti, ali riječi moje ne, neće uminuti." 
\par 32 "A o onom danu i času nitko ne zna, pa ni anđeli na nebu, ni Sin, nego samo Otac." 
\par 33 "Pazite! Bdijte jer ne znate kada je čas. 
\par 34 Kao kad ono čovjek neki polazeći na put ostavi svoju  kuću, upravu povjeri slugama, svakomu svoj posao, a vrataru zapovjedi  da bdije. 
\par 35 Bdijte, dakle, jer ne znate kad će se domaćin vratiti  - da li uvečer ili o ponoći, da li za prvih pijetlova ili ujutro  - 
\par 36 da vas ne bi našao pozaspale ako iznenada dođe." 
\par 37 "Što vama kažem, svima kažem: Bdijte!" 


\chapter{14}

\par 1 Za dva dana bijaše Pasha i Beskvasni kruhovi. Glavari svećenički  i pismoznanci tražili su kako da ga na prijevaru uhvate i ubiju. 
\par 2 Jer se govorilo: "Nikako ne na Blagdan da ne nastane pobuna  naroda." 
\par 3 I kad je u Betaniji, u kući Šimuna Gubavca, bio za stolom, dođe neka žena s alabastrenom posudicom prave skupocjene nardove  pomasti. Razbi posudicu i poli ga po glavi. 
\par 4 A neki negodovahu  te će jedan drugomu: "Čemu to rasipanje pomasti? 
\par 5 Mogla se  pomast prodati za više od tristo denara i dati siromasima." I  otresahu se na nju. 
\par 6 A Isus reče: "Pustite je, što joj dodijavate?  Dobro djelo učini na meni. 
\par 7 Ta siromaha svagda imate uza se  i kad god hoćete možete im dobro činiti, a mene nemate svagda. 
\par 8 Učinila je što je mogla: unaprijed mi pomaza tijelo za ukop. 
\par 9 Zaista, kažem vam, gdje se god bude propovijedalo evanđelje, po svem svijetu, navješćivat će se i ovo što ona učini - njoj  na spomen." 
\par 10 A Juda Iškariotski, jedan od dvanaestorice, ode glavarima  svećeničkim da im ga preda. 
\par 11 Kad su oni to čuli, obradovali  su se i obećali mu dati novca. I tražio je zgodu da ga preda. 
\par 12 Prvoga dana Beskvasnih kruhova, kad se žrtvovala pasha, upitaju učenici Isusa: "Gdje hoćeš blagovati pashu, da odemo  i pripravimo?" 
\par 13 On pošalje dvojicu učenika i rekne im: "Idite  u grad i namjerit ćete se na čovjeka koji nosi krčag vode. Pođite  za njim 
\par 14 pa gdje on uđe, recite domaćinu: 'Učitelj pita: Gdje  mi je svratište u kojem bih blagovao pashu sa svojim učenicima?' 
\par 15 I on će vam pokazati na katu veliko blagovalište, prostrto  i spremljeno. Ondje nam pripravite." 
\par 16 Učenici odu, dođu u grad i nađu kako im on reče te priprave  pashu. 
\par 17 A uvečer dođe on s dvanaestoricom. 
\par 18 I dok bijahu za stolom te blagovahu, reče Isus: "Zaista, kažem vam, jedan će me od vas izdati - koji sa mnom blaguje." 
\par 19 Ožalošćeni, stanu mu govoriti jedan za drugim: "Da nisam  ja?" 
\par 20 A on im reče: "Jedan od dvanaestorice koji umače sa  mnom u zdjelicu. 
\par 21 Sin Čovječji, istina, odlazi kako je o njemu  pisano, ali jao čovjeku onomu koji ga predaje. Tomu bi čovjeku  bolje bilo da se ni rodio nije!" 
\par 22 I dok su blagovali, on uze kruh, izreče blagoslov pa  razlomi, dade im i reče: "Uzmite, ovo je tijelo moje." 
\par 23 I  uze čašu, zahvali i dade im. I svi su iz nje pili. 
\par 24 A on im  reče: "Ovo je krv moja, krv Saveza, koja se za mnoge prolijeva. 
\par 25 Zaista, kažem vam, ne, neću više piti od ovoga roda trsova  do onoga dana kad ću ga - novoga - piti u kraljevstvu Božjem." 
\par 26 Otpjevavši hvalospjeve, zaputiše se prema Maslinskoj  gori. 
\par 27 I reče im Isus: "Svi ćete se sablazniti. Ta pisano  je:  Udarit ću pastira i ovce će se razbjeći. 
\par 28 Ali kad uskrsnem, ići ću pred vama u Galileju." 
\par 29 Nato  će mu Petar: "Ako se i svi sablazne, ja neću!" 
\par 30 A Isus mu  reče: "Zaista, kažem ti, baš ti, danas, ove noći, prije nego  se pijetao dvaput oglasi, triput ćeš me zatajiti." 
\par 31 Ali on  je upornije uvjeravao: "Bude li trebalo i umrijeti s tobom -  ne, neću te zatajiti." A tako su svi govorili. 
\par 32 I dođu u predio imenom Getsemani. I kaže Isus svojim  učenicima: "Sjednite ovdje dok se ne pomolim." 
\par 33 I povede sa  sobom Petra, Jakova i Ivana. Spopade ga užas i tjeskoba 
\par 34 pa  im reče: "Duša mi je nasmrt žalosna! Ostanite ovdje  i bdijte!" 
\par 35 Ode malo dalje i rušeći se na zemlju molio je  da ga, ako je moguće, mimoiđe ovaj čas. 
\par 36 Govoraše: "Abba!  Oče! Tebi je sve moguće! Otkloni čašu ovu od mene! Ali ne što  ja hoću, nego što hoćeš ti!" 
\par 37 I dođe, nađe ih pozaspale pa reče Petru: "Šimune, spavaš?  Jedan sat nisi mogao probdjeti? 
\par 38 Bdijte i molite da ne padnete  u napast. Duh je, istina, voljan, no tijelo je slabo." 
\par 39 Opet ode i pomoli se istim riječima. 
\par 40 Ponovno dođe  i nađe ih pozaspale. Oči im se sklapale i nisu znali što da mu  odgovore. 
\par 41 Dođe i treći put i reče im: "Samo spavajte i počivajte!  Gotovo je! Dođe čas! Evo, predaje se Sin Čovječji u ruke grešničke! 
\par 42 Ustanite, hajdemo! Evo, izdajica se moj približio!" 
\par 43 Uto, dok je on još govorio, stiže Juda, jedan od dvanaestorice, i s njime svjetina s mačevima i toljagama, poslana od glavara  svećeničkih, pismoznanaca i starješina. 
\par 44 A izdajica im njegov  dade znak: "Koga poljubim, taj je! Uhvatite ga i oprezno odvedite!" 
\par 45 I kako dođe, odmah pristupi k njemu i reče: "Učitelju!" I  poljubi ga. 
\par 46 Oni podignu na nj ruke i uhvate ga. 
\par 47 A jedan  od nazočnih trgnu mač, udari slugu velikoga svećenika i odsiječe  mu uho. 
\par 48 Isus im prozbori: "Kao na razbojnika iziđoste s mačevima  i toljagama da me uhvatite. 
\par 49 Danomice bijah vam u Hramu, naučavah  i ne uhvatiste me. No neka se ispune Pisma!" 
\par 50 I svi ga ostave i pobjegnu. 
\par 51 A jedan je mladić išao za njim, ogrnut samo plahtom.  I njega htjedoše uhvatiti, 
\par 52 no on ostavi plahtu i gol pobježe. 
\par 53 Zatim odvedoše Isusa velikom svećeniku. I skupe se svi  glavari svećenički, starješine i pismoznanci. 
\par 54 Petar je izdaleka  išao za njim do u dvor velikog svećenika. Tu je sjedio sa stražarima  i grijao se uz vatru. 
\par 55 A glavari svećenički i cijelo Vijeće, da bi mogli pogubiti  Isusa, tražili su protiv njega kakvo svjedočanstvo, ali nikako  da ga nađu. 
\par 56 Mnogi su doduše lažno svjedočili protiv njega, ali im se svjedočanstva ne slagahu. 
\par 57 Ustali su neki i lažno  svjedočili protiv njega: 
\par 58 "Mi smo ga čuli govoriti: 'Ja ću  razvaliti ovaj rukotvoreni Hram i za tri dana sagraditi drugi, nerukotvoreni!'" 
\par 59 Ali ni u tom im svjedočanstvo ne bijaše  složno. 
\par 60 Usta nato veliki svećenik na sredinu i upita Isusa: "Zar  ništa ne odgovaraš? Što to ovi svjedoče protiv tebe? 
\par 61 A on  je šutio i ništa mu nije odgovarao. Veliki ga svećenik ponovo  upita: "Ti li si Krist, Sin Blagoslovljenoga?" 
\par 62 A Isus mu  reče: "Ja jesam! I gledat ćete Sina Čovječjega gdje  sjedi zdesna Sile i dolazi s oblacima nebeskim." 
\par 63 Nato  veliki svećenik razdrije haljine i reče: "Što nam još trebaju  svjedoci? 
\par 64 Čuli ste hulu. Što vam se čini?" Oni svi presudiše  da zaslužuje smrt. 
\par 65 I neki stanu pljuvati po njemu, zastirati mu lice i udarati  ga govoreći: "Proreci!" I sluge ga stadoše pljuskati. 
\par 66 I dok je Petar bio dolje u dvoru, dođe jedna sluškinja  velikoga svećenika; 
\par 67 ugledavši Petra gdje se grije, upre u  nj pogled i reče: "I ti bijaše s Nazarećaninom, Isusom." 
\par 68 On  zanijeka: "Niti znam niti razumijem što govoriš." I iziđe van  u predvorje, a pijetao se oglasi. 
\par 69 Sluškinja ga ugleda i poče  opet govoriti nazočnima: "Ovaj je od njih!" 
\par 70 On opet nijekaše.  Domalo nazočni opet stanu govoriti Petru: "Doista, i ti si od  njih! Ta Galilejac si!" 
\par 71 On se tada stane kleti i preklinjati:  "Ne znam čovjeka o kom govorite!" 
\par 72 I odmah se po drugi put  oglasi pijetao. I spomenu se Petar one besjede, kako mu ono Isus  reče: "Prije nego se pijetao dvaput oglasi, triput ćeš me zatajiti."  I briznu u plač. 


\chapter{15}

\par 1 Odmah izjutra glavari svećenički zajedno sa starješinama i  pismoznancima - cijelo Vijeće - upriličili su vijećanje pa Isusa  svezali, odveli i predali Pilatu. 
\par 2 I upita ga Pilat: "Ti li  si kralj židovski?" On mu odgovori: "Ti kažeš." 
\par 3 I glavari  ga svećenički teško optuživahu. 
\par 4 Pilat ga opet upita: "Ništa  ne odgovaraš? Gle, koliko te optužuju." 
\par 5 A Isus ništa više  ne odgovori te se Pilat čudio. 
\par 6 O Blagdanu bi im pustio uznika koga bi zaiskali. 
\par 7 A  zajedno s pobunjenicima koji u pobuni počiniše umorstvo bijaše  u okove bačen čovjek zvani Baraba. 
\par 8 I uziđe svjetina te poče  od Pilata iskati ono što im običavaše činiti. 
\par 9 A on im odgovori:  "Hoćete li da vam pustim kralja židovskoga?" 
\par 10 Znao je doista  da ga glavari svećenički bijahu predali iz zavisti. 
\par 11 Ali glavari  svećenički podjare svjetinu da traži neka im radije pusti Barabu. 
\par 12 Pilat ih opet upita: "Što dakle da učinim s ovim kojega zovete  kraljem židovskim?" 
\par 13 A oni opet povikaše: "Raspni ga!" 
\par 14 Reče  im Pilat: "Ta što je zla učinio?" Povikaše još jače: "Raspni  ga!" 
\par 15 Hoteći ugoditi svjetini, Pilat im pusti Barabu, a Isusa  izbičeva i preda da se razapne. 
\par 16 Vojnici ga odvedu u unutarnjost dvora, to jest u pretorij, pa sazovu cijelu četu 
\par 17 i zaogrnu ga grimizom; spletu trnov  vijenac i stave mu na glavu 
\par 18 te ga stanu pozdravljati: "Zdravo, kralju židovski!" 
\par 19 I udarahu ga trskom po glavi, pljuvahu  po njemu i klanjahu mu se prigibajući koljena. 
\par 20 A pošto ga  izrugaše, svukoše mu grimiz i obukoše mu njegove haljine. I izvedu ga da ga razapnu. 
\par 21 I prisile nekog prolaznika  koji je dolazio s polja, Šimuna Cirenca, oca Aleksandrova i Rufova, da mu ponese križ. 
\par 22 I dovuku ga na mjesto Golgotu, što znači Lubanjsko mjesto. 
\par 23 I nuđahu mu piti namirisana vina, ali on ne uze. 
\par 24 Kad  ga razapeše, razdijele među se haljine njegove bacivši za  njih kocku - što će tko uzeti. 
\par 25 A bijaše treća ura kad  ga razapeše. 
\par 26 Bijaše napisan i natpis o njegovoj krivici:  "Kralj židovski." 
\par 27 A zajedno s njime razapnu i dva razbojnika, jednoga njemu zdesna, drugoga slijeva. 
\par 28 # 
\par 29 Prolaznici su ga pogrđivali mašući glavama: "Ej, ti,  koji razvaljuješ Hram i sagradiš ga za tri dana, 
\par 30 spasi sam  sebe, siđi s križa!" 
\par 31 Slično i glavari svećenički s pismoznancima  rugajući se govorahu jedni drugima: "Druge je spasio, sebe ne  može spasiti! 
\par 32 Krist, kralj Izraelov! Neka sad siđe s križa  da vidimo i povjerujemo!" Vrijeđahu ga i oni koji bijahu s njim  raspeti. 
\par 33 A o šestoj uri tama nasta po svoj zemlji - sve do ure  devete. 
\par 34 O devetoj uri povika Isus iza glasa: "Eloi, Eloi  lama sabahtani?" To znači: "Bože moj, Bože moj, zašto  si me ostavio?" 
\par 35 Neki od nazočnih čuvši to govorahu: "Gle, Iliju zove." 
\par 36 A jedan otrča, natopi spužvu octom,  natakne na trsku i pruži mu piti govoreći: "Pustite da  vidimo hoće li doći Ilija da ga skine." 
\par 37 A Isus zavapi jakim glasom i izdahnu. 
\par 38 I zavjesa se hramska razdrije nadvoje, odozgor dodolje. 
\par 39 A kad satnik koji stajaše njemu nasuprot vidje da tako izdahnu, reče: "Zaista, ovaj čovjek bijaše Sin Božji!" 
\par 40 Izdaleka promatrahu i neke žene: među njima Marija Magdalena  i Marija, majka Jakova Mlađega i Josipa, i Saloma - 
\par 41 te su  ga pratile kad bijaše u Galileji i posluživale mu - i mnoge druge  koje uziđoše s njim u Jeruzalem. 
\par 42 A uvečer, budući da je bila Priprava, to jest predvečerje  subote, 
\par 43 dođe Josip iz Arimateje, ugledan vijećnik, koji također  isčekivaše kraljevstvo Božje: odvaži se, uđe k Pilatu i zaiska  tijelo Isusovo. 
\par 44 Pilat se začudi da je već umro pa dozva satnika  i upita ga je li odavna umro. 
\par 45 Kad sazna od satnika, darova  Josipu tijelo. 
\par 46 Josip kupi platno, skine tijelo i zavije ga  u platno te položi u grob, koji bijaše izduben iz stijene. I  dokotrlja kamen na grobna vrata. 
\par 47 A Marija Magdalena i Marija Josipova promatrahu kamo  ga polažu. 



\chapter{16}

\par 1 Kad prođe subota, Marija Magdalena i Marija Jakovljeva i Saloma  kupiše miomirisa da odu pomazati Isusa. 
\par 2 I prvoga dana u tjednu, veoma rano, o izlasku sunčevu, dođu na grob. 
\par 3 I razgovarahu među sobom: "Tko će nam otkotrljati kamen  s vrata grobnih?" 
\par 4 Pogledaju, a ono kamen otkotrljan. Bijaše  doista veoma velik. 
\par 5 I ušavši u grob, ugledaju mladića zaogrnuta  bijelom haljinom gdje sjedi zdesna. I preplaše se. 
\par 6 A on će  im: "Ne plašite se! Isusa tražite, Nazarećanina, Raspetoga? Uskrsnu!  Nije ovdje! Evo mjesta kamo ga položiše. 
\par 7 Nego idite, recite  njegovim učenicima i Petru: Ide pred vama u Galileju! Ondje ćete  ga vidjeti, kamo vam reče!" 
\par 8 One iziđu i stanu bježati od groba: spopade ih strah i  trepet. I nikomu ništa ne rekoše jer se bojahu. 
\par 9 Uskrsnuvši dakle rano prvog dana u tjednu, ukaza se najprije  Mariji Magdaleni iz koje bijaše istjerao sedam zloduha. 
\par 10 Ona  ode i dojavi njegovima, tužnima i zaplakanima. 
\par 11 Kad su oni  čuli da je živ i da ga je ona vidjela, ne povjerovaše. 
\par 12 Nakon toga ukazao se u drugome obličju dvojici od njih  na putu dok su išli u selo. 
\par 13 I oni odu i dojave drugima. Ni  njima ne povjerovaše. 
\par 14 Napokon se ukaza jedanaestorici dok bijahu za stolom.  Prekori njihovu nevjeru i okorjelost srca što ne povjerovaše  onima koji ga vidješe uskrsla od mrtvih. 
\par 15 I reče im: "Pođite po svem svijetu, propovijedajte evanđelje  svemu stvorenju. 
\par 16 Tko uzvjeruje i pokrsti se, spasit će se, a tko ne uzvjeruje, osudit će se. 
\par 17 A ovi će znakovi pratiti  one koji uzvjeruju: u ime će moje izganjati zloduhe, novim će  jezicima zboriti, 
\par 18 zmije uzimati; i popiju li što smrtonosno, ne, neće im nauditi; na nemoćnike će ruke polagati, i bit će  im dobro." 
\par 19 I Gospodin Isus, pošto im to reče, bude uzet na nebo  i sjede zdesna Bogu. 
\par 20 Oni pak odoše i propovijedahu posvuda, a Gospodin surađivaše  i utvrđivaše Riječ popratnim znakovima. 




\end{document}