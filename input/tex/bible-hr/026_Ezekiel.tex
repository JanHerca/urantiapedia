\begin{document}

\title{Ezekiel}


\chapter{1}

\par 1 Godine tridesete, četvrtoga mjeseca, petoga dana, kad bijah  među izgnanicima na rijeci Kebaru, otvoriše se nebesa i ja ugledah  božanska viđenja. 
\par 2 Petoga dana istoga mjeseca - godine pete  otkako odvedoše u izgnanstvo kralja Jojakima - 
\par 3 riječ Jahvina  dođe Ezekielu, sinu Buzijevu, svećeniku u zemlji kaldejskoj,  na rijeci Kebaru. Spusti se na me ruka Jahvina. 
\par 4 Pogledah, kad ono sa sjevera udario silan vihor, velik  oblak, bukteći oganj obavijen sjajem; usred njega, usred ognja, nešto nalik na sjajnu kovinu. 
\par 5 Usred toga nešto kao četiri  bića, obličjem slična čovjeku; 
\par 6 svako od njih sa četiri obraza, u svakoga četiri krila. 
\par 7 Noge im ravne, a stopala kao u teleta;  sijevahu poput glatke mjedi. 
\par 8 Ispod krila imahu na sve četiri  strane ruke čovječje. I svako od njih četvero imaše svoj obraz  i svoja krila. 
\par 9 Krila im se spajahu jedno s drugim. Idući,  ne okretahu se: svako se naprijed kretaše. 
\par 10 I u sva četiri bijaše lice čovječje; u sva četiri zdesna  lice lavlje; u sva četiri slijeva lice volujsko; i lice orlovsko  u sva četiri. 
\par 11 Krila im bijahu gore raskriljena. Svako imaše  dva krila što se spajahu i dva krila kojim tijelo pokrivahu. 
\par 12 I svako iđaše samo naprijed. A iđahu onamo kamo ih je duh  gonio. I ne okretahu se idući. 
\par 13 A posred tih bića vidjelo se kao neko užareno ugljevlje, kao goruće zublje koje se među njima kretahu; iz ognja sijevaše  i munje bljeskahu. 
\par 14 Bića trčahu i opet se vraćahu poput munje. 
\par 15 Dok ja promatrah, gle: na zemlji uza svako od četiri  bića po jedan točak. 
\par 16 Točkovi bijahu slični krizolitu, sva  četiri istoga oblika; oblikom i napravom bijahu kao da je jedan  točak u drugome. 
\par 17 U kretanju mogli su ići u sva četiri smjera  a nisu se morali okretati. 
\par 18 Naplatnice im bijahu visoke, a kad bolje promotrih, gle, na sve strane pune očiju. 
\par 19 Kad bi bića krenula, krenuli bi  s njima i točkovi; kad bi se bića sa tla podigla, i točkovi se  podizahu. 
\par 20 Kuda ih je duh gonio, onuda se kretahu, a zajedno  se s njima i točkovi podizali, jer duh bića bijaše u točkovima. 
\par 21 Pa kad su bića krenula, i točkovi bi krenuli, a kad bi se  ona zaustavila, ustavljali se i točkovi; kad se ona sa tla dizahu, i točkovi se s njima podizahu, jer duh bića bijaše u točkovima. 
\par 22 Nad glavama bića bijaše nešto kao svod nebeski, nalik  na sjajan prozirac, uzdignut nad njihovim glavama. 
\par 23 A pod  svodom raskriljena krila, jedno prema drugome: svakome po dva  krila pokrivahu tijelo. 
\par 24 Čuh lepet njihovih krila kao huk  velikih voda, kao glas Svesilnog, kao silan vihor, kao graju  u taboru. Kad bi se bića zaustavila, spustila bi krila. 
\par 25 Sa  svoda nad njihovim glavama čula se grmljavina. 
\par 26 Ispod svoda nad njihovim glavama bijaše nešto kao kamen  safir, poput prijestolja: na tom kao prijestolju, gore na njemu, kao neki čovjek. 
\par 27 I vidjeh kao sjajnu kovinu, iznutra i uokolo  kao oganj; od njegovih bokova naviše i od njegovih bokova naniže  nešto poput ognja i blijeska na sve strane. 
\par 28 Taj blijesak  na sve strane bijaše poput duge što se za kišnih dana javlja  u oblaku. To bijaše nešto kao slava Jahvina. Vidjeh, padoh ničice  i čuh glas koji mi govoraše. 


\chapter{2}

\par 1 I reče mi: "Sine čovječji, na noge se, da s tobom govorim!" 
\par 2 OI uđe u me duh, kako mi progovori, te me podiže na noge i  ja čuh glas onoga koji mi govoraše. 
\par 3 I reče mi: "Sine čovječji, šaljem te k sinovima Izraelovim, k narodu odmetničkom što se  odvrže od mene. Oni i oci njihovi griješili su protiv mene sve  do dana današnjega. 
\par 4 Šaljem te k sinovima tvrdokorna pogleda  i okorjela srca. Reci im: Ovako govori Jahve Gospod! 
\par 5 I poslušali  oni ili ne poslušali - rod su odmetnički - neka znaju da je prorok  među vama. 
\par 6 A ti, sine čovječji, ne boj ih se i ne plaši se  riječi njihovih: 'Trnje te okružuje i sjediš među samim skorpijama.'  Ne plaši se riječi njihovih i ne boj se nimalo njihova pogleda  jer oni su rod odmetnički. 
\par 7 Govori im moje riječi, poslušali  oni ili ne poslušali, jer rod su odmetnički. 
\par 8 A ti, sine čovječji, poslušaj što ću ti sada reći: Ne budi odmetnik kao što su oni  rod odmetnički! Otvori usta i progutaj što ću ti sada dati!" 
\par 9 I pogledah, a to ruka k meni ispružena i u njoj, gle,  svitak knjige. 
\par 10 I razvi se knjiga preda mnom: bijaše ispisana  izvana i iznutra, a u njoj napisano: "Naricanje! Jecanje! Jauk!" 


\chapter{3}

\par 1 I reče mi: "Sine čovječji, progutaj što je pred tobom! Pojedi  taj svitak, te idi i propovijedaj domu Izraelovu!" 
\par 2 Otvorih  usta, a on mi dade da progutam svitak 
\par 3 i reče: "Sine čovječji, nahrani trbuh i nasiti utrobu svitkom što ti ga dajem!" I pojedoh  ga, i bijaše mi u ustima sladak kao med. 
\par 4 Reče mi: "Sine čovječji, idi domu Izraelovu i prenesi  mu moju poruku. 
\par 5 Ne šaljem te k narodu nepoznata jezika i nerazumljiva  govora, već te šaljem domu Izraelovu. 
\par 6 Ne šaljem te k mnogim  narodima nepoznata jezika i nerazumljiva govora koje ti ne bi  mogao razumjeti. A kad bih te k njima i poslao, oni bi te poslušali. 
\par 7 A dom te Izraelov neće poslušati, jer ni mene ne sluša, jer  dom je Izraelov tvrde glave i okorjela srca. 
\par 8 Evo, zato ću  sada otvrdnuti tvoje lice kao što je i njihovo i glavu ću tvoju  učiniti tvrdoglavom kao što je njihova. 
\par 9 I ne boj ih se i ne  plaši, jer oni su rod odmetnički!" 
\par 10 Reče mi: "Sine čovječji, sve riječi što ću ti reći uzmi  k srcu i poslušaj ih svojim ušima. 
\par 11 I hajde izgnanicima, sinovima  svojega naroda, i reci im: Ovako govori Jahve Gospod! - poslušali  ili ne poslušali!" 
\par 12 Uto me duh podiže i ja za sobom čuh silnu tutnjavu. Slava  se Jahvina podigla sa svojega mjesta. 
\par 13 Čuh lepet krila onih  bića - udarahu jedno o drugo - i snažnu škripu točkova što se  s njima kretahu i zaglušnu jeku jakoga glasa. 
\par 14 Tada me duh prihvati i ponese. I ja iđah ogorčen i gnjevna  srca, a ruka me Jahvina čvrsto pritisla. 
\par 15 Tako stigoh u Tel  Abib, k izgnanicima koji življahu na rijeci Kebaru - onamo gdje  se bijahu nastanili - te ostadoh među njima sedam dana kao omamljen. 
\par 16 Poslije sedam dana dođe mi opet riječ Jahvina: 
\par 17 "Sine  čovječji, postavljam te za čuvara doma Izraelova. I ti ćeš riječi  iz mojih usta slušati i opominjat ćeš ih u moje ime. 
\par 18 Kad  bezbožniku reknem: 'Umrijet ćeš', a ti ga ne opomeneš i ne odvratiš  od zla puta njegova kako bi mu život spasio, on će umrijeti sa  svojega bezakonja, ali ću ja od tebe tražiti račun za krv njegovu. 
\par 19 A kad opomeneš bezbožnika, a on se ne odvrati od bezakonja  i od zla puta svojega, on će umrijeti zbog svoje krivice, a ti  ćeš spasiti svoj život. 
\par 20 Isto tako, odvrati li se pravednik  od svoje pravednosti i stane činiti nepravdu, postavit ću mu  zamku i umrijet će jer ga ti ne opomenu zbog njegova grijeha;  umrijet će, i njegova se pravedna djela više neće spominjati, ali ću od tebe tražiti račun za krv njegovu. 
\par 21 Ako li ti pravednika  opomeneš da ne griješi, i on zaista prestane griješiti, živjet  će jer je prihvatio opomenu, a i ti ćeš spasiti život svoj." 
\par 22 Ondje me opet zahvati ruka Jahvina i on mi reče: "Ustani  i siđi u dolinu da ondje s tobom govorim!" 
\par 23 Ustadoh tada i  siđoh u dolinu, i gle: Slava Jahvina stajaše ondje, slična Slavi  koju vidjeh na rijeci Kebaru te padoh ničice. 
\par 24 Jahvin duh  uđe u me, osovi me na noge i reče: "Idi i zatvori se u domu svojemu! 
\par 25 Na te ću, evo, sine čovječji, staviti užad i svezati te i  više nećeš izlaziti. 
\par 26 I jezik ću ti zalijepiti za nepce te  ćeš onijemjeti i nećeš ih više karati, jer su rod odmetnički. 
\par 27 A kad ti ja progovorim, otvorit ću ti usta i ti ćeš im reći:  Ovako govori Jahve Gospod! I tko hoće slušati, neka sluša, a  tko neće, neka ne sluša, jer su rod odmetnički. 


\chapter{4}

\par 1 A ti, sine čovječji, uzmi opeku, postavi je preda se i nacrtaj  na njoj grad Jeruzalem. 
\par 2 Oko njega postavi opsadu, sagradi  prema njemu utvrdu, podigni nasip, iskopaj oko njega opkop, razvrstaj  vojsku i porazmjesti zidodere uokolo. 
\par 3 Zatim uzmi gvozdenu  ploču i postavi je kao gvozden bedem između sebe i grada te k  njemu okreni lice, i bit će opsjednut. Pritisni ga! To je znak  domu Izraelovu! 
\par 4 Zatim lezi na svoju lijevu stranu i stavi  na se grijeh doma Izraelova: koliko dana budeš tako ležao, toliko  ćeš dana nositi njihov grijeh. 
\par 5 Dajem ti po dan za godine grijeha  njihovih: sto i devedeset dana nosit ćeš grijeh doma Izraelova. 
\par 6 A kad to završiš, četrdeset ćeš dana ležati na desnoj strani  da nosiš grijeh doma Judina; dajem ti po dan za svaku godinu. 
\par 7 Tad okreni lice prema opsjedanom Jeruzalemu, pruži golu desnicu  i prorokuj protiv njega. 
\par 8 A ja ću te užetima vezati da se ne  možeš okretati s jedne strane na drugu dok ne navršiš dane svoje  opsade. 
\par 9 Uzmi pšenice, ječma, boba, leće, prosa i raži, stavi to  u jednu posudu i pripravi od toga sebi kruh. Jest ćeš ga onoliko  dana koliko budeš ležao na svojoj strani: sto i devedeset dana. 
\par 10 Jelo što ćeš ga jesti bit će izmjereno; dvadeset šekela na  dan; a jest ćeš ga u određeno vrijeme. 
\par 11 I vodu ćeš piti na  mjeru: šestinu hina. Pit ćeš je u određeno vrijeme. 
\par 12 A jest  ćeš pogaču od ječma što ćeš je pred njima ispeći na ljudskim  izmetinama." 
\par 13 I reče: "Tako će sinovi Izraelovi jesti svoj nečisti  kruh među narodima među koje ću ih izagnati." 
\par 14 Ja mu odgovorih:  "Jao, Jahve Gospode, gle, moja duša nije okaljana, jer se od  djetinjstva još ne okusih ničega uginulog ni rastrganog niti  u moja usta ikad uđe meso nečisto." 
\par 15 A on će: "Gle, dajem  ti kravlju balegu umjesto ljudskih izmetina da na njoj ispečeš  kruh!" 
\par 16 Još mi reče: "Sine čovječji, uništit ću u Jeruzalemu  posljednju pričuvu kruha, i jest će kruh na mjeru i s tjeskobom, i pit će vodu na mjeru i sa zebnjom. 
\par 17 Neka im nestane kruha  i vode, neka usahnu zbog bezakonja svojega i jedan za drugim  neka poginu! 


\chapter{5}

\par 1 A ti, sine čovječji, uzmi mač naoštren, uzmi ga kao britvu  brijačku i obrij glavu i bradu. Zatim uzmi mjerice i porazdijeli. 
\par 2 Trećinu spali posred grada ognjem kad se navrše dani tvoje  opsade; trećinu uzmi i sasijeci mačem oko grada; trećinu baci  u vjetar - i svoj ću mač trgnuti na njih. 
\par 3 Uzmi malo i zaveži  u skute haljine. 
\par 4 Od toga opet nešto uzmi, baci u vatru i spali:  odatle će se razgorjeti vatra svemu domu Izraelovu!" 
\par 5 Ovako govori Jahve Gospod: "Ovo je Jeruzalem! Postavih  ga u središte naroda, okružih ga zemljama! 
\par 6 Ali se on odupro  mojim naredbama većma nego pogani, zakonima mojim većma nego  zemlje koje ga okružuju." 
\par 7 Stoga ovako govori Jahve Gospod:  "Buntovniji ste od naroda koji vas okružuju, ne hodite po mojim  zakonima i ne vršite ni mojih naredaba ni naredaba okolnih naroda." 
\par 8 Zato ovako govori Jahve Gospod: "Evo i mene protiv tebe! Izvršit ću sud svoj nad tobom na  oči svih naroda. 
\par 9 Zbog tvojih gadosti učinit ću s tobom što  još ne učinih nikada nit ću ikad učiniti: 
\par 10 posred tebe očevi  će jesti sinove, a sinovi očeve; izvršit ću sud svoj nad tobom  i sav ostatak tvoj predati svim vjetrovima! 
\par 11 Života mi moga!  - riječ je Jahve Gospoda - svakojakim grozotama i gadostima ti  uistinu oskvrnu moje Svetište. I ja ću sada brijati i oko se  moje neće sažaliti i neću se smilovati: 
\par 12 trećina će tvojih  žitelja posred tebe od kuge skončati i od gladi umrijeti; trećina  će oko tebe od mača pasti; trećinu ću predati vjetrovima - i  mač ću svoj trgnuti na njih! 
\par 13 Tako ću iskaliti gnjev svoj  i smirit će se jarost moja kad im se osvetim. I kad iskalim jarost  svoju nad njima, spoznat će da sam to ja, Jahve, u ljubomori  svojoj bio rekao. 
\par 14 Opustošit ću te, izvrgnut ću te ruglu naroda  koji te okružuju, na oči svim prolaznicima. 
\par 15 Da, bit ćeš na  ruglo i sramotu, opomena i užas okolnim narodima kad izvršim  protiv tebe sve svoje sudove kažnjavajući gnjevno, jarosno. Ja, Jahve, rekoh! 
\par 16 I kad na vas pustim ljute strijele gladi što  zatiru, koje ću pustiti na vas da vas uništim i glađu zatrem  - uništit ću vam i posljednju pričuvu kruha. 
\par 17 A povrh gladi  pustit ću na vas i divlje zvijeri koje će ti djecu rastrgati;  kuga će te i krv preplaviti: pod mač ću te svoj okrenuti. Ja, Jahve, rekoh!" 


\chapter{6}

\par 1 Tada mi dođe riječ Jahvina i reče: 
\par 2 "Sine čovječji, okreni  lice prema gorama Izraelovim i prorokuj protiv njih. 
\par 3 Reci: 'Gore Izraelove, čujte riječ Jahve Gospoda! Ovako govori  Jahve Gospod: Gore i brežuljci, jaruge i doline, evo, spustit  ću mač na vas i oborit ću uzvišice vaše! 
\par 4 Opustjet će žrtvenici  vaši i porušit će se stupovi vaši, a vaše poginule pred kumire  ću vam baciti. 
\par 5 Pobacat ću trupla sinova Izraelovih pred kumire  njihove i rasijat ću kosti vaše oko žrtvenika vaših! 
\par 6 Gdje  god boravili, gradovi će vaši biti opustošeni, uzvišice poharane, žrtvenici će vam opustjeti i biti uništeni, kumiri će vaši biti  oboreni i nestat će ih, stupovi će vaši biti smrvljeni, sva će  djela vaša propasti. 
\par 7 Među vas će padati poginuli, i znat ćete  da sam ja Jahve! 
\par 8 Ali ću ipak poštedjeti neke od vas: ti će među narodima  uteći maču kad se raspršite po zemljama. 
\par 9 Tada će se preživjeli  među vama spomenuti mene među narodima kamo budu odvedeni u izgnanstvo, kad im slomim srce preljubničko što se odmetnulo od mene i kad  im iskopam preljubničke oči što pođoše za kumirima njihovim.  I tada će sami sebi omrznuti zbog nedjela što ih počiniše gadostima  svojim. 
\par 10 I spoznat će da sam ja Jahve: nisam im zaludu govorio  da ću ih udariti svim tim zlom.'" 
\par 11 Ovako govori Jahve Gospod: "Pljesni rukama i lupni nogama, te reci: Jao! zbog svih gadnih nedjela dom će Izraelov pasti  od mača, gladi i kuge! 
\par 12 Tko bude daleko, od kuge će umrijeti;  tko bude blizu, od mača će pasti, i tko bude opkoljen, od gladi  će izdahnuti! Tako ću gnjev iskaliti na njima 
\par 13 i spoznat će  da sam ja Jahve kad im poginuli budu ležali među kumirima oko  žrtvenika na svakome povišem brežuljku, nad svim vrhovima planinskim, pod svakim stablom zelenim, pod svakim hrastom granatim, gdje  se god prinosio ugodan miris kumirima njihovim. 
\par 14 Ruku ću podići  na njih i svu ću im zemlju pretvoriti u pustoš, od pustinje do  Rible, posvuda gdje borave! I spoznat će da sam ja Jahve!" 


\chapter{7}

\par 1 Opet mi dođe riječ Jahvina i reče: 
\par 2 "Ti, sine čovječji, reci:  Ovako govori Jahve Gospod zemlji Izraelovoj: 'Primiče se kraj: bliži se konac zemlji na sve četiri strane  svijeta! 
\par 3 Sada je i tebi kraj: gnjev ću svoj na te izliti,  sudit ću ti prema putovima tvojim i na te ću oboriti sve gadosti  tvoje! 
\par 4 I moje te oči neće požaliti, neću ti se smilovati,  nego ću te nagraditi prema putovima tvojim, tvoje će gadosti  u tebi ostati. I znat ćete da sam ja Jahve.'" 
\par 5 Ovako govori Jahve: "Jedna nesreća, evo, dolazi! 
\par 6 Kraj  dolazi, dolazi ti kraj, evo, dolazi! 
\par 7 Kolo ti udesa dolazi, stanovniče zemlje! Dolazi tvoj čas, bliži se dan: strava je, a ne radost u gorama. 
\par 8 Eto, uskoro ću na te izliti gnjev i  iskalit ću na tebi srdžbu svoju! Sudit ću ti prema putovima tvojim  i oborit ću na te sve gadosti tvoje. 
\par 9 I moje te oči neće požaliti, neću ti se smilovati, nego ću ti platiti prema putovima tvojim  i tvoje će gadosti u tebi ostati! I spoznat ćete da sam ja Jahve  koji bije. 
\par 10 Evo, evo dolazi, kolo ti udesa dolazi, prut već  cvjeta i oholost pupa, 
\par 11 a nasilje se podiže kao žezlo bezbožnosti!  I nitko neće ostati od njih, nitko od njihova mnoštva. Ništa  od njihove buke, nema u njima vrijednosti. 
\par 12 Ide vrijeme, bliži  se dan! Tko kupuje, neka se ne raduje, a tko prodaje, neka ne  tuguje, jer se gnjev izlijeva na sve bogatstvo njegovo. 
\par 13 Jer  tko proda, neće više dobiti što je prodao, i nitko neće bezakonjem  ojačati život! 
\par 14 Trube trublje i sve je spremno, ali nitko  ne kreće u boj, jer gnjev se moj izlijeva na sve bučno mnoštvo. 
\par 15 Vani - mač, a unutra - kuga i glad! I tko je u polju, od mača će poginuti, a tko u gradu, glad će ga i kuga uništiti. 
\par 16 Koji uteku, sklonit će se u gore kao dolinski golubovi, a  ja ću ih sve istrijebiti, svakoga zbog bezakonja njegova, 
\par 17 i  sve će ruke klonuti, a koljena će svima malaksati. 
\par 18 U kostrijet  će se odjenuti, trepet će ih obuzeti, sva će lica sramota pokriti, sve će im glave oćelavjeti! 
\par 19 Srebro svoje pobacat će na ulice, a zlato će smatrati izmetom: u dan srdžbe Jahvine ni srebro  ni zlato neće ih izbaviti, duše im neće moći nasititi ni trbuha  napuniti, jer se o to spotakoše na grijeh. 
\par 20 Uzoholiše se zbog  divnoga nakita svojega; od njega napraviše kumire - grozote i  gadosti svoje: zato im ga pretvorih u izmet. 
\par 21 Dat ću ga u  ruke tuđincima da oplijene, razgrabe i oskvrnu. 
\par 22 Odvratit  ću od njih lice svoje: i neka se samo oskvrnjuje moja dragocjenost, neka u nju uđu provalnici i neka je oskvrnu! 
\par 23 Spremaj lance, jer je zemlja puna krvi i zločina koji zaslužuju smrt i grad  prepun nasilja! 
\par 24 Zato ću dovesti najgore narode da baštine  njihove domove. Slomit ću oholost nasilnika, i svetišta njihova  bit će oskvrnjena. 
\par 25 Dolazi tjeskoba! Tražit će mir, a mira  biti neće! 
\par 26 Dolazit će nevolja za nevoljom, jedna zla vijest  za drugom! I tražit će se viđenje u proroka; u svećenika neće  više biti Zakona ni u starješina savjeta! 
\par 27 Kralj će protužiti, a kneza će spopasti užas i ruke će puku zadrhtati, jer ću ih  nagraditi prema putovima njihovim i sudit ću im prema sudovima  njihovim. I znat će da sam ja Jahve." 


\chapter{8}

\par 1 Godine šeste, šestoga mjeseca, petoga dana, dok sjeđah u svojoj  kući, a preda mnom starješine judejske, spusti se na me ruka  Jahvina. 
\par 2 Pogledah, i gle: tu kao neki čovjek; od njegovih kao bokova  naniže oganj, a od njegovih kao bokova naviše bljeskanje, nešto  poput usijane kovine. 
\par 3 Ispruži nešto nalik na ruku i uhvati  me za kosu na glavi. Uto me duh podiže između zemlje i neba i  ponese me u božanskome viđenju u Jeruzalem, na ulaz unutrašnjih  vrata, što su okrenuta prema sjeveru gdje stoji kumir, ljubomora  koja izaziva ljubomoru. 
\par 4 I gle, ondje bijaše Slava Boga Izraelova, kao što je vidjeh u dolini. 
\par 5 I reče mi: "Sine čovječji, podigni  oči prema sjeveru!" I podigoh oči prema sjeveru. I gle, kumir, ljubomora, bijaše i na sjeveru, kraj vrata žrtvenika, na strani  prema ulazu. 
\par 6 I reče mi: "Sine čovječji, vidiš li što oni ovdje čine?  Velike su to gnusobe što ih dom Izraelov ovdje čini, samo da  me udalji iz mojega Svetišta. A vidjet ćeš i gorih gnusoba!" 
\par 7 I povede me do vrata predvorja. Pogledah, i gle: u zidu pukotina. 
\par 8 I reče mi: "Sine čovječji, probij taj zid!" Probih zid, a  ono - ulaz! 
\par 9 I reče mi: "Uđi i pogledaj strahovite gadosti  što se ovdje čine!" 
\par 10 Uđoh i pogledah. I gle, svakojake slike  gmazova i gnusnih životinja - sve kumiri doma Izraelova, našarani  na zidu, svuda uokolo. 
\par 11 A pred tim kumirima sedamdesetorica  ljudi od starješina doma Izraelova, i među njima i Šafanov sin  Jaazanija. I svakome od njih u ruci kadionica iz koje se podiže  oblak kada miomirisnoga. 
\par 12 I reče mi: "Sine čovječji, vidiš li što u toj tami rade  starješine doma Izraelova, svatko u svojoj oslikanoj komori?  I još govore: Jahve nas ne vidi jer je Jahve napustio zemlju!" 
\par 13 I reče mi još: "A vidjet ćeš i gorih gnusoba što se ovdje  čine!" 
\par 14 I povede me do vrata Doma Jahvina što su okrenuta  prema sjeveru. I gle, ondje sjeđahu žene i oplakivahu Tamuza. 
\par 15 I reče mi: "Vidiš li, sine čovječji? A vidjet ćeš i gorih  gnusoba od ovih!" 
\par 16 I povede me u unutrašnje predvorje Doma  Jahvina. Ondje, na ulazu u Hekal Jahvin, između trijema i žrtvenika, bijaše oko dvadeset i pet ljudi, okrenutih leđima Hekalu Jahvinu, a licem prema istoku, i ondje se prema istoku klanjahu suncu. 
\par 17 I reče mi: "Vidiš li to, sine čovječji? Malo li je domu  Judinu svih ovih gnusoba što ih ovdje čine, nego mi još zemlju  pune i nasiljem, i ponovo me izazivaju i granama pred nosom mašu? 
\par 18 Zato ću i ja sada postupiti s njima jarosno i oči se moje  više neće sažaliti i neću im se smilovati. I kad stanu iza glasa  vikati na moje uši, neću ih uslišiti." 


\chapter{9}

\par 1 Tada zagrmje na moje uši i reče: "Kazne grada! Priđite svaka  sa svojim zatornim oružjem u ruci!" 
\par 2 I gle, dođoše šestorica  ljudi s gornjih vrata, što su okrenuta k sjeveru, svaki sa svojim  zatornim oružjem u ruci. Među njima bijaše i jedan odjeven u  lan, s pisarskim priborom za pojasom. Uđoše oni i stadoše uz  tučani žrtvenik. 
\par 3 A Slava Boga Izraelova vinu se s kerubina, nad kojima lebdijaše, prema pragu Doma. I pozva čovjeka odjevena u lan, koji imaše za pojasom pisarski  pribor, 
\par 4 te mu reče: "Prođi gradom Jeruzalemom i znakom 'tau'  obilježi čela sviju koji tuguju i plaču zbog gnusoba što se u  njemu čine!" 
\par 5 A drugima reče na moje uši: "Pođite za njim gradom  i ubijajte bez milosrđa. Oči vaše neka se ne sažale i nemajte  smilovanja. 
\par 6 Pobijte starce, mladiće, djevojke, djecu i žene;  istrijebite ih sve do posljednjega. Ali na kome bude znak 'tau', njega ne dirajte. Počnite od mojega Svetišta!" I oni počeše  od starješina koji stajahu pred Domom. 
\par 7 I reče im: "Oskvrnite  Dom moj i napunite mu predvorje truplima! Krenite!" I oni iziđoše  te zaredaše ubijati gradom. 
\par 8 Dok su oni klali, ja ostadoh, bacih se ničice i zavapih:  "Jao, Jahve Gospode! Zar ćeš zaista uništiti sve što preostade  od Izraela da iskališ svoj gnjev nad Jeruzalemom?" 
\par 9 Reče mi:  "Veoma je veliko bezakonje doma Izraelova i doma Judina; zemlja  je puna krvi, a grad krcat zločina. Govore: 'Jahve je ostavio  zemlju! Ne vidi Jahve!' 
\par 10 I zato se moje oči neće sažaliti  i neću im se smilovati: djela ću im njihova oboriti na glavu!" 
\par 11 I gle, čovjek odjeven u lan, koji imaše za pojasom pisarski  pribor, javi vijesti: "Učinih kako si mi zapovjedio!" 


\chapter{10}

\par 1 Pogledah, i gle: na svodu nad glavama kreubinÄa pojavi se  nešto kao kamen safir, kao nekakvo prijestolje. 
\par 2 I prozbori  čovjeku odjevenom u lan: "Uđi među točkove pod kerubinima, uzmi  pune pregršti žeravice između kerubina i prospi je nad gradom!"  - I on na moje oči uđe među kerubine. 
\par 3 A kerubini stajahu s  desne strane Doma kad čovjek uđe među njih. I oblak ispuni sve  unutrašnje predvorje, 
\par 4 a Slava Jahvina vinu se s kerubinÄa  prema pragu Doma. Dom se ispuni oblakom, a predvorje napuni svjetlost  Slave Jahvine. 
\par 5 Huka kerubinskih krila razliježe se do vanjskoga  predvorja, kao kad zagrmi glas Svevišnjega. 
\par 6 A kad on zapovjedi čovjeku odjevenom u lan: "Uzmi ognja  između točkova što su pod kerubinima", čovjek uđe i stade kraj  točkova. 
\par 7 Jedan kerubin pruži ruku prema ognju što bijaše među  kerubinima, uze ga i stavi u ruke čovjeku odjevenom u lan. On  ga primi i iziđe. 
\par 8 A ispod kerubinskih krila ukaza se nešto  kao ruka čovječja. 
\par 9 Pogledah, i gle: uz kerubine četiri točka, po jedan uza svakoga. A točkovi bijahu nalik na kamen krizolit; 
\par 10 sva četiri istog oblika i kao da je jedan točak u drugome. 
\par 11 U kretanju mogahu ići u sva četiri smjera, sve bez zakretanja.  Kamo bi se glava usmjerila, onamo bi krenuli, a da se, krećući  se, nisu morali okretati. 
\par 12 Cijelo tijelo u kerubinÄa - leđa, ruke, krila i sva četiri točka njihova - sve im bijaše posvud  naokolo puno očiju. 
\par 13 A točkovi, koliko sam čuo, zvahu se "Kovitlac". 
\par 14 Svaki  imaše po četiri lica: lice prvoga kerubinsko, lice drugoga čovječje, a u trećega lice lavlje, u četvrtoga orlovsko. 
\par 15 Tada se kerubini  podigoše u visine. Bijahu to ista bića što ih vidjeh na rijeci  Kebaru. 
\par 16 Kad bi kerubini krenuli, krenuli bi i točkovi uz  njih, kad bi kerubini krilima mahnuli da se od zemlje podignu, točkovi se ne bi od njih odmicali. 
\par 17 Kad bi se zaustavili, i točkovi bi stali; a kad bi se sa zemlje podigli, i točkovi  se podizahu, jer duh bića bijaše u njima. 
\par 18 Uto se Slava Jahvina vinu s praga Doma i stade nad kerubinima. 
\par 19 Tada kerubini raširiše krila i podigoše se sa zemlje pred  mojim očima. A kad oni krenuše, i točkovi za njima krenuše. I  zaustaviše se nad istočnim vratima Doma Jahvina, a Slava Boga  Izraelova bijaše nad njima. 
\par 20 Bijahu to ista bića što ih vidjeh  pred Bogom Izraelovim na rijeci Kebaru. I tako spoznah da ono  bijahu kerubini. 
\par 21 U svakoga po četiri lica i po četiri krila, a pod krilima nešto kao ruka čovječja. 
\par 22 Lica im ista kao  ona što ih vidjeh na rijeci Kebaru. I svako se naprijed kretaše. 


\chapter{11}

\par 1 Tada se duh podiže i ponese me do istočnih vrata Doma Jahvina, što su okrenuta k istoku. I gle: na ulazu vrata dvadeset i pet  ljudi, među kojima vidjeh i Jaazaniju, sina Azurova, i Pelatju, sina Benajina, knezove narodne. 
\par 2 I reče mi: "Sine čovječji, evo ljudi koji smišljaju opačine i koji u ovom gradu daju zle  savjete: 
\par 3 'Nije li čas da gradimo domove? Ovaj je grad kotao, a mi smo meso.' 
\par 4 Zato prorokuj protiv njih, prorokuj, sine  čovječji!" 
\par 5 I duh Jahvin siđe nada me i kaza mi: "Reci: Ovako veli  Jahve Gospod: 'Ne govoriš li tako, dome Izraelov? Ali ja poznajem  misli vašega srca! 
\par 6 Množite ubojstva u ovome gradu i njegove  ulice punite truplima.' 
\par 7 Zato ovako govori Jahve Gospod: 'Oni  koje vi probodoste i razbacaste po gradu - oni su meso, a grad  je kotao. Zato ću vas ja izvesti sada iz njega. 
\par 8 Od mača strahujete, i mač ću na vas dovesti - riječ je Jahve Gospoda! 
\par 9 Izvest  ću vas iz grada i predati vas u ruke tuđincima, i sud ću svoj  izvršiti nad vama: 
\par 10 od mača ćete pasti! Na međi Izraelovoj  sudit ću vam, i tada ćete znati da sam ja Jahve! 
\par 11 A ovaj grad  više vam neće biti kotao i vi nećete biti meso njegovo. Na međi  Izraelovoj sudit ću vam, 
\par 12 i tada ćete znati da sam ja Jahve  po čijim uredbama ne živjeste i čijih zakona ne izvršavaste,  nego živjeste po zakonima okolnih naroda!'" 
\par 13 Dok ja tako prorokovah, umrije Pelatja, sin Benajin.  I ja padoh ničice te zavapih iz svega glasa: "Jao, Jahve Gospode, zar ćeš doista uništiti sav Ostatak doma Izraelova?" 
\par 14 I dođe  mi riječ Jahvina: 
\par 15 "Sine čovječji, tvojoj braći, i tvojim rođacima, i svem  domu Izraelovu Jeruzalemci govore: 'Daleko ste od Jahve! Nama  je ova zemlja dana u posjed!' 
\par 16 Zato im reci: Ovako govori  Jahve Gospod: 'Ako ih i odagnah među daleke narode, ako ih i  rasprših po zemljama, ja ću im sam uskoro biti Svetište u zemljama  u kojima se nalaze.' 
\par 17 Stoga im reci: Ovako govori Jahve Gospod:  'Sabrat ću vas iz narodÄa, vratit ću vas iz zemalja u kojima  ste bili raspršeni i dat ću vam opet zemlju Izraelovu! 
\par 18 I  kad se u nju vrate, istrijebit će iz nje sve grozote i gadosti. 
\par 19 I ja ću im dati novo srce i nov ću duh udahnuti u njih: iščupat  ću iz njih njihovo kameno srce i stavit ću u njih srce od mesa, 
\par 20 da hode po mojim naredbama i da čuvaju i vrše sve moje zakone.  I bit će oni moj narod, a ja Bog njihov! 
\par 21 A onima kojima srce  hodi za grozotama i gadostima oborit ću na glavu njihov put'  - riječ je Jahve Gospoda." 
\par 22 Kerubini podigoše krila i točkovi se digoše za njima, a Slava Boga Izraelova lebdijaše nad njima. 
\par 23 Slava se Jahvina  vinu iz grada i zaustavi se na gori, istočno od grada. 
\par 24 A  mene duh podiže i ponese duhom Božjim k izgnanicima u zemlju  kaldejsku. I iščeznu viđenje koje gledah. 
\par 25 I pripovjedih izgnanicima  sve što mi Jahve bijaše objavio. 


\chapter{12}

\par 1 Opet mi dođe riječ Jahvina: 
\par 2 "Sine čovječji! Ti boraviš  u rodu odmetničkom koji ima oči, a ne vidi, uši ima, a ne čuje, jer su rod odmetnički. 
\par 3 Zato, sine čovječji, spremi izgnanički  zavežljaj i njima na oči obdan se seli: seli se iz svojega mjesta  u drugo, ne bi li uvidjeli da su rod odmetnički. 
\par 4 Obdan, njima  na oči, iznesi zavežljaj, zavežljaj izgnanički, a iziđi obnoć  na njihove oči kao što se odlazi u izgnanstvo. 
\par 5 Njima na oči  prokopaj zid i kroza nj izađi. 
\par 6 I njima na oči vrgni zavežljaj  na ramena i po mrkloj noći iziđi. Pokrij lice da ne vidiš zemlju, jer te postavih kao znamenje domu Izraelovu!" 
\par 7 Učinih kako mi bijaše zapovjeđeno: obdan iznesoh zavežljaj, zavežljaj izgnanički, a obnoć prokopah zid rukama i njima na  oči po mrkloj noći vrgoh zavežljaj na ramena. 
\par 8 Ujutro mi dođe  riječ Jahvina: 
\par 9 "Sine čovječji, zapita li te dom Izraelov,  dom odmetnički: 'Što to radiš?' 
\par 10 ti mu reci: 'Ovako govori  Jahve Gospod! Ovo je proroštvo knezu jeruzalemskom i svemu domu  Izraelovu koji je u Jeruzalemu.' 
\par 11 Reci: 'Ja sam vam znamenje!  Kako ja uradih, tako će biti njima: svi ćete se morati seliti  u izgnanstvo! 
\par 12 Knez njihov morat će vrći zavežljaj na ramena  i po mrkloj noći izaći. Prokopat će zid da izađe kroza nj i lice  će pokriti rukama da očima ne vidi zemlje. 
\par 13 Ja ću mu razapeti  mrežu, i uhvatit će se u moju zamku, i odvest ću ga u Babilon, u zemlju kaldejsku. Ali je on neće ugledati i ondje će život  ostaviti. 
\par 14 A sve one oko njega, pomagače i čete, raspršit  ću u sve vjetrove i svoj mač ću trgnuti na njih. 
\par 15 A kad ih  raspršim među narode i rasijem po zemljama, znat će da sam ja  Jahve. 
\par 16 Ipak, ostavit ću nekolicinu koji će umaći maču, gladi  i kugi, da među narodima kamo prispiju pripovijedaju svoje gadosti;  neka se zna da sam ja Jahve.'" 
\par 17 I dođe mi riječ Jahvina: 
\par 18 "Sine čovječji, jedi kruha  zabrinuto i pij vode sa zebnjom i sa strepnjom! 
\par 19 I reci puku  zemlje: 'Ovako govori Jahve Gospod Jeruzalemcima u zemlji Izraelovoj:  Zabrinuto će jesti kruha i sa strepnjom piti vode, jer će im  zemlja opustjeti i ostat će bez igdje ičega s bezakonja žitelja  svojih. 
\par 20 I svi gradovi, sada napučeni, bit će poharani, a  sva zemlja opustošena. I znat će da sam ja Jahve!'" 
\par 21 I dođe mi riječ Jahvina: 
\par 22 "Sine čovječji, kakve su  vam to priče o zemlji Izraelovoj? Govori se: 'Gle, prolaze dani, a od proroštva ništa!' 
\par 23 Zato im reci: Ovako govori Jahve  Gospod: 'Dokončat ću te priče i neće se više ponavljati u Izraelu.'  Reci im: 'Bliže se već dani i sva će se proroštva moja ispuniti! 
\par 24 Jer neće više biti u domu Izraelovu varavih viđenja, ni lažnih  proroštava kojima ljude bijahu zavodili. 
\par 25 Jer što ja, Jahve  Gospod, govorim, to će i biti, i riječ se neće odgoditi! Da!  Još za vaših dana, rode odmetnički, riječ ću izgovoriti i izvršiti.'  Tako govori Jahve Gospod!" 
\par 26 I dođe mi riječ Jahvina: 
\par 27 "Sine čovječji! Evo što  se govori u domu Izraelovu: 'Viđenje što ga ovaj ugleda za dane  je daleke! Prorokuje za daleka vremena!' 
\par 28 Zato im reci: Ovako  govori Jahve Gospod: 'Nijedna riječ moja neće se više odgoditi!  Što rekoh, rečeno je, i sve će se ispuniti!' - riječ je Jahve  Gospoda." 


\chapter{13}

\par 1 I opet mi dođe riječ Jahvina: 
\par 2 "Sine čovječji! Prorokuj  protiv onih koji se grade prorocima u Izraelu! 
\par 3 Reci tim prorocima  koji prorokuju po svojoj glavi: 'Čujte riječ Jahvinu! Ovako govori  Jahve Gospod: Jao prorocima bezumnim koji duh svoj slijede a  ništa ne vide! 
\par 4 Ti su tvoji proroci, Izraele, kao lisice usred  ruševina. 
\par 5 Vi se ne popeste na proboje i ne zidaste zida oko  doma Izraelova da se održi u boju u dan Jahvin. 
\par 6 Viđenja su  njihova isprazna, i lažna su njihova proricanja. Govore 'Riječ  Jahvina!' - a Jahve ih nije poslao. I još očekuju da će im se  riječi ispuniti. 
\par 7 Zar ne vidite da su vam viđenja isprazna  i da su vam lažna proricanja kad govorite 'Riječ Jahvina!' -  a ja nisam govorio.' 
\par 8 Stoga ovako govori Jahve Gospod: 'Zato što govorite isprazno  i laž vidite, evo me protiv vas' - riječ je Jahve Gospoda! 
\par 9 Evo, ruka moja bit će protiv proroka koji vide isprazno i laž proriču:  neće više biti u zboru mojega naroda, neće biti upisani u knjigu  doma Izraelova, nikad više neće stupiti na tlo Izraelovo! I znat  će da sam ja Jahve Gospod! 
\par 10 Jer narod moj obmanjuju govoreći  'Mir' kad mira nema. I dok jedni hoće da se zid utvrdi, oni hoće  da se samo ožbuka. 
\par 11 Reci onima koji hoće da se samo ožbuka:  'Past će!' Udarit će silan pljusak, oborit ću na nj grÓad kao  kamenje, bjesnjet će olujni vihori. 
\par 12 Evo, zid će pasti! Neće  li vas tada pitati: 'Gdje vam je sada žbuka kojom ste ga ožbukali?' 
\par 13 Zato ovako govori Jahve Gospod: 'U svojoj jarosti razbjesnit  ću olujne vihore, u srdžbi ću svojoj udariti silnim pljuskom  da ga zatrem, u gnjevu ću na nj oboriti grÓad kao kamenje. 
\par 14 Obalit  ću zid što ga vi žbukom ožbukaste, na zemlju ću ga oboriti da  mu se razgole temelji. Past će zid, i vi ćete pod njim izginuti!  Tada ćete znati da sam ja Jahve! 
\par 15 Tako ću iskaliti gnjev nad  zidom i nad onima koji ga žbukom ožbukaše. A vama ću reći: Nema  više zida! Nema onih koji ga žbukom ožbukaše. 
\par 16 Nema izraelskih  proroka koji Jeruzalemu proricahu i koji mu mir vidješe kad mira  ne bijaše.' Tako govori Jahve Gospod." 
\par 17 "Sine čovječji, okreni lice protiv kćeri svojega naroda  koje prorokuju po svojoj glavi! Prorokuj protiv njih: 
\par 18 Ovako  govori Jahve Gospod: 'Jao onima koje vezu poveze za svačije ruke  i koje prave prijevjese za glave svake veličine da ulove duše!  Mislite li uloviti sve duše mojega naroda a svoje duše sačuvati  žive? 
\par 19 Obeščašćujete me pred mojim narodom za šaku ječma,  za zalogaj kruha, ubijajući duše koje ne bi trebale da umru,  a spasavajući one koje ne bi trebale da žive; i obmanjujete tako  narod moj koji rado sluša vaše laži.' 
\par 20 Zato ovako govori Jahve Gospod: 'Evo me protiv vaših  poveza kojima lovite duše kao ptice! Rastrgat ću sve to na vašim  rukama i oslobodit ću duše koje time hvatate kao ptice! 
\par 21 Poderat  ću vaše prijevjese i oslobodit ću svoj narod da ne bude više  plijen vaših ruku. I znat ćete da sam ja Jahve! 
\par 22 Jer vi lažju  ražalostiste srce pravednika, koje ja ražalostiti ne htjedoh, a okrijepiste ruke bezbožnika da se ne obrati od zla puta bezbožničkog  pa da život spasi. 
\par 23 Zato nećete više vidjeti isprazno niti  ćete laž proricati: ja ću osloboditi narod svoj iz vaših ruku.  I znat ćete da sam ja Jahve!'" 


\chapter{14}

\par 1 Uto k meni dođoše neki od starješina Izraelovih i sjedoše  preda me. 
\par 2 I dođe mi riječ Jahvina: 
\par 3 "Sine čovječji! Ti ljudi  nose kumire u srcu i upiru oči u ono što ih na grijeh potiče.  Pa zar da trpim da u mene traže savjeta? 
\par 4 Zato im reci: Ovako  govori Jahve Gospod: 'Tko god iz doma Izraelova nosi u srcu kumire  i upire oči u ono što ga na grijeh potiče, a dolazi k proroku, ja, Jahve, odgovorit ću mu prema mnoštvu njegovih kumira, 
\par 5 da  uhvatim za srce dom Izraelov koji se zbog idola svojih odmetnu  od mene.' 
\par 6 Zato reci domu Izraelovu: Ovako govori Jahve Gospod: 'Obratite  se, odvratite se od kumira svojih! Odvratite lice od gnusoba  svojih! 
\par 7 Jer tko se god iz doma Izraelova i od došljaka koji  se nastaniše u Izraelu odmetne od mene i u srcu nosi kumire i  upire oči u ono što ga potiče na grijeh, pa unatoč tome dođe  k proroku da preko njega u mene traži savjeta, njemu ću ja, Jahve, sam odgovoriti; 
\par 8 okrenut ću se protiv njega i učinit ću od  njega poslovičan primjer: iskorijenit ću ga iz svojega naroda!  I znat ćete da sam ja Jahve. 
\par 9 Ako li se prorok dadne zavesti i progovori, bilo bi to  kao da sam ja, Jahve, zaveo toga proroka: ruku ću podići na njega  i iskorijenit ću ga iz svojega naroda izraelskoga. 
\par 10 Obojica  će podjednako snositi grijeh svoj: grijeh prorokov jednak je  grijehu onoga koji je u njega tražio savjeta. 
\par 11 I tako se dom  Izraelov više neće odmetati od mene i neće se više kaljati svojim  opačinama: on će biti narod moj, a ja ću biti njegov Bog' - riječ  je Jahve Gospoda." 
\par 12 I dođe mi riječ Jahvina: 
\par 13 "Sine čovječji, zgriješi  li koja zemlja protiv mene nevjerom i ja podignem ruku na nju  te joj uništim i posljednju pričuvu kruha i pustim na nju glad  da zatrem u njoj sve ljude i stoku; 
\par 14 preostanu li u njoj samo  tri čovjeka - Noa, Daniel i Job - ti će se svojom pravednošću  spasiti - riječ je Jahve Gospoda. 
\par 15 Također, ako na tu zemlju  pustim divlje zvijeri da joj djecu unište a nju pretvore u pustinju, kojom se zbog zvijeri više nitko neće usuditi proći; 
\par 16 preostanu  li u njoj samo ta tri čovjeka, života mi moga - riječ je Jahve  Gospoda - oni neće spasiti ni sinova ni kćeri nego samo sebe, a zemlja će njihova postati prava pustinja. 
\par 17 Ili, ako ja  trgnem mač na tu zemlju govoreći: 'Maču, prođi ovom zemljom!'  da istrijebim u njoj sve ljude i stoku, 
\par 18 a u njoj se nađu  samo ona tri čovjeka, života mi moga - riječ je Jahve Gospoda  - oni neće spasiti ni sinova ni kćeri nego samo sebe. 
\par 19 Ili, ako ja pošaljem na tu zemlju kugu te izlijem na nju gnjev i  pokolj da zatrem u njoj sve ljude i stoku, 
\par 20 a u njoj preostanu  samo ona tri čovjeka, Noa, Daniel i Job, života mi moga - riječ  je Jahve Gospoda - oni neće spasiti ni sinova ni kćeri nego samo  sebe svojom pravednošću." 
\par 21 Ovako govori Jahve Gospod: "Ipak, ako na Jeruzalem pustim  sva svoja četiri ljuta biča - mač, glad, divlju zvjerad i kugu  - da zatrem u njemu sve ljude i stoku, 
\par 22 u njemu će ipak preživjeti  Ostatak koji će spasiti sinove i kćeri. I evo, oni će doći k  vama da vidite njihovo vladanje i njihova djela i da se utješite, jer ćete upoznati: što god poduzeh protiv Jeruzalema, ne učinih  bez razloga. 
\par 23 Da, kad vidite njihovo vladanje i njihova djela, utješit ćete se, jer ćete upoznati da ne učinih bez razloga  što god poduzeh protiv Jeruzalema - riječ je Jahve Gospoda." 


\chapter{15}

\par 1 I dođe mi riječ Jahvina: 
\par 2 "Sine čovječji! U čemu je trs loze bolji od drugih šumskih drveta? 
\par 3 Služi li da se od njega štogod načini? Djelja li se od njega klin da se o njega što objesi? 
\par 4 Gle, baca se u oganj da izgori: kad mu oganj sažeže oba kraja i sredinu spali, može li još čemu poslužiti? 
\par 5 Eto, ni onda kad bijaše čitav ništa se od njega ne mogaše načiniti. Pa kako će, dakle, čemu poslužiti kad ga plamen sažga?" 
\par 6 Zato ovako govori Jahve Gospod: "Kao što sam trs loze, među drugim drvetima, bacio u oganj da izgori, tako ću postupati i s Jeruzalemcima! 
\par 7 Upravit ću lice svoje na njih, i kada se iz jednog ognja izbave, drugi će ih proždrijeti. I spoznat ćete da sam ja Jahve kad lice svoje upravim na njih 
\par 8 i svu im zemlju opustošim jer mi bijahu nevjerni! - riječ je Jahve Gospoda." 


\chapter{16}

\par 1 I dođe mi riječ Jahvina: 
\par 2 Sine čovječji! Pokaži Jeruzalemu  sve gadosti njegove! 
\par 3 Reci: Ovako Jahve Gospod govori Jeruzalemu, nevjernici: 'Podrijetlom i rodom iz zemlje si kanaanske, otac  ti Amorejac, mati Hetitkinja. 
\par 4 Kad si svijet ugledala, na dan  rođenja tvojega pupka ti ne odrezaše niti te vodom opraše da  te očiste; solju te ne osoliše niti te povojima poviše. 
\par 5 Nijedno  se oko na te ne sažali niti se tko smilova da ti to učini, nego  te na dan rođenja tvojega gadnu baciše napolje. 
\par 6 A ja prođoh kraj tebe i vidjeh gdje se koprcaš u krvi.  I rekoh ti dok si još u krvi bila: 'Živi!' U krvi ti tvojoj rekoh:  'Živi! 
\par 7 Razrasti se kao izdanak u polju!' I umnožih te, i ti  se razraste i velika postade, i dođe vrijeme da sazreš. Dojke  ti se raspupale, kosa ti narasla, ali si još gola i naga bila. 
\par 8 Prođoh kraj tebe i u te se zagledah: i gle, dob tvoja - dob  je ljubavi! Raširih na te skute svoje i pokrih ti golotinju. Prisegoh  ti i sklopih Savez s tobom - riječ je Jahve Gospoda - i ti moja  postade. 
\par 9 Okupah te u vodi, krv saprah s tebe i uljem te pomazah. 
\par 10 Obukoh te u šarene haljine, na noge ti obuh sandale od fine  kože; opasah te bezom i pokrih te prijevjesom svilenim. 
\par 11 Uresih  te nakitima: na ruke ti stavih narukvice, oko vrata ogrlice; 
\par 12 prstenom ti nos uresih, uši naušnicama, a glavu ti ovjenčah  vijencem najljepšim. 
\par 13 I tako se sva u srebru i zlatu pojavi, u haljini od beza, svilom izvezenoj. Za hranu ti dadoh najfinije  brašno, med i ulje. Bila si tako lijepa, prelijepa, za kraljicu  podobna! 
\par 14 Glas o ljepoti tvojoj puče među narodima, jer ti bijaše  tako lijepa u nakitu mojem što ga djenuh na tebe - riječ je Jahve  Gospoda. 
\par 15 Ali te ljepota tvoja zanijela, zbog glasa se svojega  bludu podade: blud si svoj nudila obilno svakom prolazniku, njegova  si bila. 
\par 16 Od haljina si svojih šarene uzvišice pravila i na  njima se bludu odavala ... 
\par 17 I nakite uze zlatne i srebrne, kojima te ja bijah uresio, i od njih načini sebi muške likove  da s njima bludničiš. 
\par 18 Uze šarene, vezene haljine da njima  odjeneš kumire svoje i njima si prinosila moje ulje i moj kad. 
\par 19 A hranu što ti je dadoh - najfinije brašno, med i ulje kojima  te hranjah - pred njih si stavljala na ugodan miris. Da, tako  to bijaše - riječ je Jahve Gospoda! 
\par 20 Sinove si svoje i kćeri  uzimala koje meni porodi i njima ih za hranu klala. Malo ti bijaše  tvoga bludničenja, 
\par 21 pa si čak i djecu moju davala da se njima  na čast kroz oganj provedu! 
\par 22 U svim tim gnusobama i bludu  svojemu ne spomenu se dana mladosti svoje, kad si se gola i naga  u krvi svojoj koprcala. 
\par 23 I povrh svega zla - Jao! Jao! riječ je Jahve Gospoda  - 
\par 24 sagradi sebi humke, posvud diže uzvišice. 
\par 25 Na svim raskršćima  podiže uzvišice i na njima blatiš svoju ljepotu, nudiš se svakom  prolazniku množeć' svoje bludničenje. 
\par 26 Bludu se podade sa  sinovima Egipta, snažna tijela, bludničenje si množila da me  razjariš. 
\par 27 Zato, evo, ruku digoh na te, smanjivši ti obrok  hrane i predavši te bijesu tvojih mrziteljica, kćeri filistejskih, koje se stide sramotnoga tvojeg vladanja. 
\par 28 Tjerala si blud  i sa sinovima Asira i nisi se zasitila; i s njima si blud tjerala, ali se nisi zasitila. 
\par 29 Umnožila si bludničenje svoje i sa  zemljom kanaanskom, sa zemljom kaldejskom, ali se ni onda nisi  zasitila. 
\par 30 O, kako li slabo bijaše tvoje srce - riječ je Jahve Gospoda  - kad činjaše ono što rade bludnice najrazvratnije. 
\par 31 Na svim  raskrsnicama humak sebi podiže, posvuda sagradi sebi uzvišice.  Ali ne kao druge bludnice, jer si prezirala plaću bludničku, 
\par 32 nego kao preljubnica: mjesto muža, strance si primala. 
\par 33 Svima  se bludnicama plaća, a ti si sama ljubavnike svoje plaćala i  još si ih u bludnosti svojoj darovima mamila da ti dođu odasvuda. 
\par 34 Ti bijaše bludnica kakvih nema: nitko za tobom nije trčao  da s tobom blud provodi, nego si sama davala plaću bludničku, a nisu je tebi plaćali. Toliko si bila opaka! 
\par 35 Stoga, razvratnice, čuj riječ Jahvinu: 
\par 36 Ovako govori  Jahve Gospod: Jer si svlačila svoju sramotu i u bludu golotinju  otkrivala pred svima svojim ljubavnicima i gnusnim kumirima,  i zbog krvi svojih sinova što si ih njima prinosila, 
\par 37 evo, skupit ću sve tvoje ljubavnike s kojima si se naslađivala, sve  koje si voljela i koje si mrzila, skupit ću ih odasvud protiv  tebe i razotkriti im tvoju golotinju, neka vide sramotu tvoju. 
\par 38 Sudit ću ti kao što se sudi preljubnicama i krvnicama i predati  te bijesu njihovu. 
\par 39 Predat ću te u ruke njihove da poruše  tvoje humke, da razore uzvišice tvoje. I zderat će sa tebe haljine, oteti nakit i ostaviti te golu, sasvim nagu. 
\par 40 A zatim će  na te dovesti svjetinu da te kamenuje i da te sasiječe mačevima. 
\par 41 Kuće će ti ognjem spaliti i naočigled svim ženama izvršiti  pravdu nad tobom. Tako ću dokrajčiti tvoje bludničenje, nećeš  više davati plaću bludničku. 
\par 42 Iskalit ću gnjev svoj nad tobom  i povući ću svoju ljubomoru od tebe. Smirit ću se i neću se više  gnjeviti. 
\par 43 I jer se ne spomenu svoje mladosti, već me svim  tim izazivaše, oborit ću ti na glavu sve postupke tvoje - riječ  je Jahve Gospoda: nećeš više dodavati bestidnosti na sve svoje  gadosti! 
\par 44 I sastavljač poslovica narugat će ti se poslovicom: 'Kakva  mati, takva kći.' 
\par 45 Prava si kći svoje matere, koja ostavi  muža i djecu; sestra si sestara svojih, koje ostaviše muževe  svoje i djecu: Hetitkinja vam mati bijaše, otac Amorejac: 
\par 46 Samarija, sestra tvoja starija, sa svojim kćerima tebi slijeva stoji;  Sodoma, tvoja mlađa sestra, sa kćerima svojim zdesna ti stoji. 
\par 47 A ti ne samo da si njihovim putem hodila i činila njihove  gadosti - to bi tebi bilo premalo - već bijaše od njih pokvarenija  na svojim putovima. 
\par 48 Života mi mojega - riječ je Jahve Gospoda  - tvoja sestra Sodoma sa svojim kćerima ne učini što si ti počinila  zajedno sa kćerima svojim. 
\par 49 Evo opačina sestre tvoje Sodome: gizdavo, u izobilju  kruha i bezbrižno življaše ona i kćeri njezine, a sirotinju i  bijednike ne pomagahu. 
\par 50 Uzoholiše se i gadosti pred očima  mojim činjahu, i zato ih zatrijeh, kao što vidje! 
\par 51 A sestra  ti Samarija ne počini ni polovicu grijeha tvojih, i tako ti počini  više gadosti nego one obje zajedno, opravdavši sestre svoje svojim  gadostima. 
\par 52 Zato snosi sad sramotu grijeha kojima si sestre  svoje opravdala; zbog grijeha kojima se više od njih nagrdi,  one izađoše pravednije. Postidi se, dakle, i snosi sramotu svoju  kojom sestre opravda. 
\par 53 A ja ću okrenuti udes njihov, udes Sodome i kćeri njenih, udes Samarije i kćeri njenih; i tvoj ću udes okrenuti među njima, 
\par 54 da snosiš sramotu svoju i da se postidiš za sve što si počinila, njima na utjehu. 
\par 55 Sestra tvoja Sodoma i kćeri njene vratit  će se u stanje prijašnje; sestra tvoja Samarija i kćeri njene  vratit će se u stanje prijašnje; ali i ti i kćeri tvoje vratit  ćete se u stanje prijašnje. 
\par 56 Zar se nije spominjala sestra  tvoja Sodoma dok ti bijaše ponosita, 
\par 57 prije negoli se golotinja  tvoja otkrila? Budi sada za ruglo kćerima edomskim, susjedama  njenim i kćerima filistejskim koje ti se sa svih strana rugaju. 
\par 58 Snosi, dakle, svoju sramotu i svoje gadosti - riječ je Jahve  Gospoda!' 
\par 59 Jer ovako govori Jahve Gospod: 'Postupit ću s tobom onako  kako ti učini kad pogazi zakletvu i raskinu Savez. 
\par 60 Ali ću  se ja ipak spomenuti svojega Saveza s tobom što ga sklopih u  dane mladosti tvoje i uspostavit ću s tobom Savez vječan. 
\par 61 I  ti ćeš se opomenuti svojih putova i postidjet ćeš se kad primiš  svoje sestre, stariju i mlađu, koje ću ti dati za kćeri, ali  ne snagom tvog Saveza. 
\par 62 Sklopit ću s tobom savez svoj i znat  ćeš da sam ja Jahve, 
\par 63 da se opomeneš i da se postidiš i da  od sramote više ne otvoriš usta kad ti oprostim sve što učini!  To je riječ Jahve Gospoda.'" 


\chapter{17}

\par 1 Dođe mi riječ Jahvina: 
\par 2 "Sine čovječji, smisli zagonetku  i iznesi prispodobu domu Izraelovu! Reci: 
\par 3 'Ovako govori Jahve  Gospod:  Velik orao, velikih krila, duga perja, gusta, šarena paperja, doletje na Libanon i zgrabi cedrov vrh; 
\par 4 odlomi mu najvišu grančicu, odnese je u zemlju trgovaca i spusti je u grad prodavača. 
\par 5 Onda uze izdanak iz zemlje, u plodnu ga njivu posadi, kraj obilnih voda stavi, kao vrbu usadi. 
\par 6 Izdanak proklija, bujan izbi čokot, onizak izraste, mladice mu k orlu segnuše, a pod njim mu žilje bješe; u bujan se razvi čokot, potjera izdanke, mladice razgrana. 
\par 7 Bijaše i drugi orao, velik i velikih krila, gusta perja. I gle, čokot k njemu žilje pruži, k njemu upravi grančice svoje da ga natapa bolje od tla u koje bi zasađen. 
\par 8 Na plodnoj njivi, kraj obilnih voda, bješe zasađen: mogao je tjerat' mladice, uroditi rodom, k'o veličanstveni trs izrasti.' 
\par 9 Reci: 'Ovako govori Jahve Gospod: Hoće l' uspijevati? Neće l' mu orao sve žilje izguliti? Neće l' mu sve plodove potrgati? Neće l' mu sve mladice, čim izbiju, sasušiti? Da, i bez snažne mišice, i bez mnoštva naroda, iščupat će ga iz korijena! 
\par 10 Gle, zasađen je! Hoće l' uspijevati? Čim ga takne istočnjak, neće l' sav usahnuti? Da, na lijehama iz kojih niče uvenut će.'" 
\par 11 Dođe mi riječ Jahvina: 
\par 12 "Reci domu odmetničkome: 'Zar  ne znate što ovo znači?' Reci im: 'Eto, dođe kralj babilonski  u Jeruzalem, zarobi mu kralja i sve knezove, odvede ih k sebi  u Babilon. 
\par 13 Odvede i izdanak iz kraljevskoga koljena, sklopi  s njima savez i zakletvom se obveza, pošto odvede sve mogućnike  iz zemlje, 
\par 14 da će kraljevstvo ostati neznatno i da se neće  dizati, nego će čuvati i držati savez s njime. 
\par 15 Ali se on od njega odmetnu; poslanike uputi u Egipat, tražeći od njega konje i jaku vojsku. Hoće li uspjeti? Može  li umaći onaj tko tako radi? Raskinu savez, pa da umakne? 
\par 16 Života  mi moga, riječ je Jahve Gospoda: jer prezre zakletvu kralja koji  ga na prijestolje posadi i razvrže savez s njime, u njegovoj  će zemlji umrijeti, usred Babilona! 
\par 17 Svojom silnom vojskom  i mnoštvom naroda faraon mu neće pomoći u boju kad onaj digne  nasipe i sagradi kule opsadne da mu zatre mnogo ljudstvo. 
\par 18 Prezreo  je zakletvu, razvrgao savez. Da, iako ruku bijaše dao, sve to  učini! Ne, neće umaći!' 
\par 19 Stoga ovako govori Jahve Gospod: 'Života mi moga, zakletvu  što je prezre i savez što ga razvrže oborit ću na glavu njegovu! 
\par 20 Mrežu ću nad njim razapeti i uhvatit će se u moju zamku,  pa ću ga odvesti u Babilon i ondje mu suditi zbog nevjere kojom  mi se iznevjeri. 
\par 21 Cvijet vojske njegove od mača će pasti,  a ostatak se raspršiti u sve vjetrove. I spoznat ćete da ja,  Jahve, tako rekoh.' 
\par 22 Ovako govori Jahve Gospod: 'S vrha cedra velikoga, s vrška mladih grana njegovih, odlomit ću grančicu i posadit' je na gori visokoj, najvišoj. 
\par 23 Na najvišoj gori izraelskoj nju ću zasaditi: razgranat će se ona, plodom uroditi. 
\par 24 I sve će poljsko drveće znati da ja sam Jahve koji visoko drvo ponizujem, a nisko uzvisujem; zeleno drvo sušim, a drvu suhu dajem da rodi. Ja, Jahve, rekoh i učinit ću!'" 


\chapter{18}

\par 1 Dođe mi riječ Jahvina: 
\par 2 "Što vam je te o Izraelu ponavljate  poslovicu: 'Oci jedoše kiselo grožđe, sinovima trnu zubi!' 
\par 3 Života  mi moga, riječ je Jahve Gospoda: nitko od vas neće više u Izraelu  ponavljati tu poslovicu; 
\par 4 jer, svi su životi moji, kako život  očev tako i život sinovlji. I evo, onaj koji zgriješi, taj će  umrijeti. 
\par 5 Tko je pravedan i poštuje zakon i pravdu 
\par 6 i ne blaguje  po gorama i očiju ne podiže kumirima doma Izraelova, ne oskvrnjuje  žene bližnjega svoga i ne prilazi ženi dok je nečista; 
\par 7 nikomu  ne nanosi nasilja, vraća što je u zalog primio i ništa ne otima;  kruh svoj dijeli s gladnim, gologa odijeva, 
\par 8 ne posuđuje uz  dobit i ne uzima pridavka, ruku usteže od nedjela, po istini  presuđuje, 
\par 9 po mojim naredbama hodi i čuva moje zakone, postupajući  po istini - taj je zaista pravedan i taj će živjeti, riječ je  Jahve Gospoda. 
\par 10 Ali, porodi li on sina nasilnika, koji krv prolijeva  ili bratu takvo što učini, 
\par 11 a ne radi kao njegov roditelj, nego blaguje po gorama, oskvrnjuje ženu bližnjega; 
\par 12 ubogu  i bijednu nanosi nasilje, otima, ne vraća što je u zalog primio, oči podiže kumirima čineći gadosti; 
\par 13 posuđuje uz dobit i  uzima pridavak - ne, takav sin neće živjeti! Učinio je te gadosti  i umrijet će, a krv će njegova na njega pasti. 
\par 14 A porodi li on sina koji uvidi sve grijehe što ih njegov  otac počini, uvidi ih i tako više ne učini; 
\par 15 ne blaguje po  gorama, očiju ne podiže kumirima doma Izraelova, ne oskvrnjuje  žene bližnjega; 
\par 16 nikomu ne nanosi nasilja, ne prisvaja zaloga, ništa ne otima, kruh svoj dijeli s gladnim, gologa odijeva; 
\par 17 ruku usteže od nedjela, ne uzima dobiti ni pridavka, vrši  moje zakone i hodi po mojim naredbama - ne, taj neće umrijeti  zbog grijeha očeva, on će živjeti. 
\par 18 A njegov otac, koji je  nemilice tlačio i pljačkao bližnjega, čineći u narodu što ne  valja, zbog svojega će grijeha umrijeti. 
\par 19 Ali vi kažete: 'Zašto da sin ne snosi očev grijeh?' Zato  što sin vrši zakon i pravdu, čuva i vrši sve moje naredbe, živjet  će. 
\par 20 Onaj koji zgriješi, taj će i umrijeti. Sin neće snositi  grijeha očeva, ni otac grijeha sinovljega. Na pravedniku će biti  pravda njegova, a na bezbožniku bezbožnost njegova. 
\par 21 Ako se bezbožnik odvrati od svih grijeha što ih počini, i bude čuvao sve moje naredbe i vršio zakon i pravdu, živjet  će i neće umrijeti. 
\par 22 Sva njegova nedjela što ih počini bit  će zaboravljena: zbog pravednosti što je čini, živjet će. 
\par 23 Jer, zar je meni do toga da umre bezbožnik - riječ je Jahve Gospoda  - a ne da se odvrati od svojih zlih putova i da živi? 
\par 24 Ako li se pravednik odvrati od svoje pravednosti i stane  činiti nepravdu i sve gadosti koje radi bezbožnik - hoće li živjeti?  Sva pravedna djela koja bijaše činio zaboravit će se, a zbog  svoje nevjere kojom se iznevjerio i zbog grijeha što ih počini, umrijet će. 
\par 25 A vi velite: 'Put Jahvin nije pravedan!' Čuj, dome Izraelov: Moj put da nije pravedan? Nisu li vaši putovi  nepravedni? 
\par 26 Ako li se pravednik odvrati od svoje pravednosti  i stane činiti nepravdu, pa zbog toga umre, umrijet će zbog nepravde  što je počini. 
\par 27 A ako se bezbožnik odvrati od svoje bezbožnosti  što je bijaše činio, pa stane vršiti moj zakon i pravdu, živjet  će i neće umrijeti. 
\par 28 Jer je uvidio i odvratio se od svojih  nedjela što ih bijaše počinio, živjet će i neće umrijeti. 
\par 29 Ali dom Izraelov kaže: 'Put Gospodnji nije pravedan!'  Putovi moji da nisu pravedni, dome Izraelov? Nisu li vaši putovi  nepravedni? 
\par 30 Dome Izraelov, ja ću suditi svakome po njegovim  putovima - riječ je Jahve Gospoda. Obratite se, dakle, i povratite  od svih svojih nedjela, i grijeh vam vaš neće biti na propast! 
\par 31 Odbacite od sebe sva nedjela koja ste činili i načinite sebi  novo srce i nov duh! Zašto da umirete, dome Izraelov? 
\par 32 Ja  ne želim smrti nikoga koji umre - riječ je Jahve Gospoda. Obratite  se, dakle, i živite! 


\chapter{19}

\par 1 A ti, sine čovječji, protuži tužaljkom za knezovima izraelskim. 
\par 2 Reci: Što bijaše tvoja mati? Lavica među lavovima, ležala je među lavićima, hraneći mladunčad svoju. 
\par 3 I othrani jedno mlado, koje lavom posta. Naučiv se plijen derati, stade ljude proždirati! 
\par 4 Narodi se protiv njega udružiše, lav upade u jamu njihovu, na lancu ga odvedoše u zemlju egipatsku. 
\par 5 A kad mati vidje da uzalud čeka i da joj nada propade, uze drugo mlado i od njega lava učini. 
\par 6 Živeć' tako među lavovima, i on lavom posta. Naučiv se plijen derati, stade ljude proždirati, 
\par 7 utvrde im rušiti, pustošiti gradove. Uzdrhta zemlja i sve na njoj od silne rike njegove. 
\par 8 Ali se ljudi iz okolnih mjesta protiv njega podigoše i zamke mu postaviše; i lav se uhvati u jamu njihovu. 
\par 9 Okovana u kavez ga zatvoriše, odvedoše kralju babilonskom, ondje ga u kulu zatočiše, da mu se više ne čuje rika po gorama izraelskim. 
\par 10 Mati tvoja bješe kao loza pokraj vode zasađena, rodna i granata od obilja vode! 
\par 11 Imala je jaku granu za palicu vladalačku: uzdiže se nad krošnju, naočita visinom, mnoštvom grančica. 
\par 12 Al' u gnjevu bješe iščupana i na zemlju bačena. Istočnjak joj rod sasuši: polomi se i uvenu jaka grana njezina i vatra je svu proguta. 
\par 13 U pustinju bje presađena, u zemlju suhu, bezvodnu. 
\par 14 Al' liznu oganj iz pruta njezina i spali joj grane i plodove! I nema više na njoj grane jake za palicu vladalačku." To je, evo tužaljka, i ostat će tužaljka. 


\chapter{20}

\par 1 Godine sedme, petoga mjeseca, desetoga dana, dođoše k meni  neke od starješina izraelskih da se s Jahvom svjetuju. Posjedaše  preda me. 
\par 2 I dođe mi riječ Jahvina: "Sine čovječji! Govori starješinama Izraelovim! 
\par 3 Reci im:  Ovako govori Jahve Gospod: 'Došli ste me pitati za savjet? Života  mi moga, nećete me pitati!' - riječ je Jahve Gospoda. 
\par 4 Hoćeš  li im suditi, hoćeš li suditi, sine čovječji? Pokaži im gadosti  otaca njihovih. 
\par 5 Reci im: Ovako govori Jahve Gospod: 'Onoga  dana kad izabrah Izraela i ruku stavih na potomstvo doma Jakovljeva  te im se objavih u zemlji egipatskoj, zakleh im se: Ja sam Jahve, Bog vaš! 
\par 6 Toga im se dana rukom podignutom zakleh da ću ih  izvesti iz zemlje egipatske u zemlju koju za njih izabrah, u  zemlju kojom teče med i mlijeko, od svih zemalja najljepšu. 
\par 7 I  rekoh im: Odbacite od sebe sve gadosti što vam oči privlače i  ne kaljajte se kumirima egipatskim jer - ja sam Jahve, Bog vaš!' 
\par 8 Ali se oni odvrgoše od mene i ne htjedoše me poslušati:  nijedan ne odbaci gadosti koje mu oči zaniješe i ne okani se  kumira egipatskih. Tad odlučih izliti gnjev svoj na njih i iskaliti  srdžbu na njima u zemlji egipatskoj. 
\par 9 Ali radi imena svojega  - da se ne kalja na oči naroda među kojima obitavahu i pred kojima  im bijah objavio da ću ih izvesti - 
\par 10 izvedoh ih iz zemlje  egipatske i odvedoh ih u pustinju; 
\par 11 i dadoh im svoje uredbe  i objavih svoje zakone, koje svatko mora vršiti da bi živio; 
\par 12 dadoh im i svoje subote, kao znak između sebe i njih, neka  znaju da sam ja Jahve koji ih posvećujem. 
\par 13 Ali se i u pustinji dom Izraelov odmetnu od mene: nisu  hodili po mojim uredbama; odbaciše moje zakone, koje svatko mora  vršiti da bi živio; subote moje oskvrnjivahu. I zato odlučih  u pustinji gnjev svoj na njih izliti da ih zatrem. 
\par 14 Ali ni  toga ne učinih radi svojeg imena, da se ono ne kalja pred narodima  kojima ih naočigled izvedoh. 
\par 15 Ali im se zakleh u pustinji  da ih neću uvesti u zemlju koju sam im bio dao, u zemlju kojom  teče med i mlijeko, od svih zemalja najljepšu, 
\par 16 jer odbaciše  moje zakone, i ne hodiše po mojim uredbama, i subote moje oskvrnjivahu, a srce im iđaše za njihovim kumirima. 
\par 17 Oči se moje ipak sažališe  da ih ne zatrem. I tako ih u pustinji ne uništih, 
\par 18 nego rekoh  sinovima njihovim u pustinji: 'Ne hodite po uredbama svojih otaca, ne čuvajte zakona njihovih i ne kaljajte se kumirima njihovim! 
\par 19 Ja sam Jahve, Bog vaš! Po uredbama mojim hodite, čuvajte  i vršite moje zakone 
\par 20 i svetkujte moje subote, neka one budu  znak između mene i vas, kako bi se znalo da sam ja Jahve, Bog  vaš!' 
\par 21 Ali se i sinovi odmetnuše od mene: po mojim uredbama  nisu hodili i nisu čuvali ni vršili mojih zakona, koje svatko  mora vršiti da bi živio, a subote su moje oskvrnjivali. I zato  odlučih gnjev svoj izliti i iskaliti srdžbu svoju na njima u  pustinji. 
\par 22 Ali opet ruku svoju sustegoh radi svojeg imena, da se ono ne kalja pred narodima kojima ih naočigled izvedoh. 
\par 23 No zakleh se u pustinji da ću ih raspršiti među narode i  rasijati po zemljama, 
\par 24 jer nisu vršili mojih zakona i jer  prezreše moje uredbe i jer subote moje oskvrnjivahu i oči upirahu  u kumire svojih otaca. 
\par 25 I zato im dadoh uredbe koje ne bijahu  dobre, zakone koji usmrćuju: 
\par 26 da se oskvrnjuju svojim prinosima, provodeći kroz oganj svoju prvorođenčad. Htjedoh tako da ih  zastrašim, neka znaju da sam ja Jahve. 
\par 27 Sine čovječji, reci domu Izraelovu: 'Ovako govori Jahve  Gospod! I ovim me oci vaši još uvrijediše: nevjerom mi se iznevjeriše! 
\par 28 Kad ih uvedoh u zemlju koju im se zakleh dati, gdje god bi  ugledali povišen brežuljak ili stablo krošnjato, prinosili bi  žrtve, donosili izazovne prinose, metali mirise ugodne, nalijevali  ljevanice. 
\par 29 Upitah ih: što li znači ta uzvišica na koju se  penjete?' I tako osta ime 'bama', uzvišica, do dana današnjega. 
\par 30 Zato reci domu Izraelovu: 'Ovako govori Jahve Gospod:  Ne kaljate li se i vi kao oci vaši, ne provodite li i vi blud  s gadostima njihovim? 
\par 31 Kaljate se prinoseći im darove, provodeći  kroz oganj svoje sinove u čast svim kumirima svojim sve do dana  današnjega. I da me onda za savjet pitaš, dome Izraelov! Života  mi moga - riječ je Jahve Gospoda - nećete me za savjet pitati! 
\par 32 I neće se zbiti o čemu sanjate kad govorite: 'Bit ćemo kao  drugi narodi, kao narodi ostalih zemalja što služe drveću i kamenju.' 
\par 33 Života mi moga - riječ je Jahve Gospoda - vladat ću vama  rukom krepkom i mišicom uzdignutom, u svoj žestini svoje jarosti. 
\par 34 Izvest ću vas iz naroda, skupiti vas iz svih zemalja u koje  bijaste raspršeni rukom krepkom i mišicom uzdignutom, u svem  plamu jarosti moje! 
\par 35 Odvest ću vas u pustinju naroda i ondje  vam licem u lice suditi! 
\par 36 Kao što sudih ocima vašim u pustinji  zemlje egipatske, i vama ću suditi - riječ je Jahve Gospoda! 
\par 37 Provest ću vas ispod štapa svojega, podvrći vas brojenju: 
\par 38 razlučit ću između vas sve koji se pobuniše i odvrgoše od  mene: izvest ću ih iz zemlje u kojoj kao došljaci borave, ali  - u zemlju Izraelovu nikad ući neće! I znat ćete da sam ja Jahve!' 
\par 39 'A vi, dome Izraelov' - ovako govori Jahve Gospod - 'samo  idite i dalje služite svaki svom kumiru! Jednom ćete, kunem vas  se, poslušati i nećete više kaljati moje sveto ime svojim prinosima  i kumirima: 
\par 40 na Svetoj gori mojoj, na visokoj gori Izraelovoj  - riječ je Jahve Gospoda - služit će mi sav dom Izraelov, u svojoj  zemlji. Ondje će mi oni omiljeti i ondje ću iskati vaše podizanice  i prinose vaših prvina sa svim svetinjama. 
\par 41 Omiljet ćete mi  kao miris ugodan kad vas izvedem iz narodÄa i skupim iz zemalja  u kojima bjeste rasijani. I na vama ću očitovati svetost svoju  naočigled svih naroda. 
\par 42 Tada ćete znati da sam ja Jahve, kada  vas dovedem u zemlju Izraelovu, u zemlju koju se zakleh dati  očevima vašim. 
\par 43 Ondje ćete se spomenuti svih svojih putova  i nedjela kojima se okaljaste: sami ćete sebi omrznuti zbog nedjela  što ih počiniste. 
\par 44 I tada ćete spoznati da sam ja Jahve kad, radi imena svojega, ne postupim s vama po zloći vaših putova  ni po vašim pokvarenim djelima, dome Izraelov! Tako govori Jahve  Gospod!'" 
\par 45 (21:1) I dođe mi riječ Jahvina: 
\par 46 (21:2) "Sine čovječji, okreni lice k  jugu i prospi besjedu prema jugu te prorokuj protiv šume u kraju  negepskom. 
\par 47 (21:3) Reci šumi negepskoj: 'Poslušaj riječ Jahvinu! Ovako  govori Jahve Gospod: Evo, zapalit ću usred tebe oganj i on će  proždrijeti u tebi svako drvo, zeleno i suho! Razgorjeli se oganj  neće utrnuti dok sve ne izgori od sjevera do juga. 
\par 48 (21:4) I svi će  vidjeti da sam ja, Jahve, zapalio taj oganj i neće se ugasiti.'" 
\par 49 (21:5) Rekoh na to: "Jao, Jahve Gospode, tÓa oni će za mene reći:  'Evo opet pričalice s pričama!'" 


\chapter{21}

\par 1 (21:6) I dođe mi riječ Jahvina: 
\par 2 (21:7) "Sine čovječji, okreni lice  prema Jeruzalemu i prospi besjedu protiv njegova Svetišta i prorokuj  protiv zemlje Izraelove. 
\par 3 (21:8) Reci zemlji Izraelovoj: 'Ovako govori  Jahve Gospod: Evo me na te! Trgnut ću mač iz korica, istrijebit  ću iz tebe sve - i pravedna i bezbožna! 
\par 4 (21:9) Da iz tebe istrijebim  pravedna i bezbožna, trgnut ću evo mač iz korica na svako tijelo, od sjevera do juga. 
\par 5 (21:10) I svako će tijelo spoznati da sam ja, Jahve, isukao mač svoj iz korica i da ga više neću u njih vratiti! 
\par 6 (21:11) A ti, sine čovječji, kukaj kao da su ti sva rebra polomljena, kukaj gorko, njima na oči! 
\par 7 (21:12) Ako li te zapitaju: 'Što toliko  kukaš?' reci im: 'Zbog vijesti koja stiže, od koje će sva srca  zamrijeti i sve ruke klonuti, svaki duh biti utučen i svako koljeno  klecati. Evo, dolazi, već je tu!' Tako govori Jahve Gospod." 
\par 8 (21:13) I dođe mi riječ Jahvina: 
\par 9 (21:14) "Sine čovječji, prorokuj!  Ovako govori Jahve Gospod. Reci: 'Mač! Mač! Naoštren i osvjetlan! 
\par 10 (21:15) Za klanje naoštren, osvjetlan da sijeva. 
\par 11 (21:16) Osvjetlan da ga ruka prihvati, mač naoštren, osvjetlan da se stavi u ruke ubojici. 
\par 12 (21:17) A ti, sine čovječji, plači, nariči! Jer, evo, mač je već na narod moj isukan, mač na izraelske knezove: svi su oni s mojim narodom maču izručeni! Udri se stoga u slabine!' 
\par 13 (21:18) Dođe kušnja, i odbačenoga žezla više biti neće - riječ je Jahve Gospoda. 
\par 14 (21:19) A ti, sine čovječji, prorokuj i rukama plješći. Neka se udvostruči, neka se utrostruči taj mač pokolja, mač pokolja golema što ih odasvud okružuje. 
\par 15 (21:20) Da zadršću srca, da bude žrtava nebrojenih, na svaka sam vrata postavio mač, pripravljen da k'o munja sijeva, za pokolje naoštren. 
\par 16 (21:21) Natrag! Desno! Naprijed! Lijevo! 
\par 17 (21:22) I ja ću pljeskati rukama, iskaliti gnjev svoj na njima! Ja, Jahve, rekoh!" 
\par 18 (21:23) I dođe mi riječ Jahvina: 
\par 19 (21:24) "Sine čovječji, zacrtaj  dva puta kuda da pođe mač kralja babilonskoga. Neka oba puta  izlaze iz iste zemlje! Na raskršću puta ka gradu stavi putokaz. 
\par 20 (21:25) Zacrtaj maču put da dođe u Rabat Bene Amon i u Judeju, u  utvrđeni Jeruzalem. 
\par 21 (21:26) Jer kralj babilonski stoji na početku  puta, na raspuću dvaju putova, i pita znamenja - miješa strijele, ispituje terafime i motri jetru. 
\par 22 (21:27) Znamenja mu u desnici kažu:  na Jeruzalem; da ondje namjesti zidodere, da naredi pokolj, da  podigne zidodere protiv vrata, da naspe nasip i sagradi opsadne  kule. 
\par 23 (21:28) Ali će se njima učiniti da je znamenje lažno, jer mu  se zakleše na vjernost. Ali će ih on tada podsjetiti na njihovo  vjerolomstvo u koje se uloviše. 
\par 24 (21:29) Zato, ovako govori Jahve  Gospod: 'Jer bez prestanka podsjećate na svoja bezakonja otkrivajući  opačine i pokazujući grijehe u svim svojim djelima - da, jer  bez prestanka na njih podsjećate, u njih ćete se uloviti. 
\par 25 (21:30) A  tebi, nečasni i bezbožnički kneže izraelski, tebi dođe dan i  čas posljednjega zločina.' 
\par 26 (21:31) Ovako govori Jahve Gospod: 'Skini  mitru s glave i odloži kraljevski vijenac! Jer sve se mijenja:  tko bi dolje, bit će uzvišen, a tko bi gore, bit će ponižen. 
\par 27 (21:32) Ruševine, ruševine, ruševine ću postaviti kakvih nije bilo, dok ne dođe onaj koji ima suditi, jer ja ću mu predati sud.' 
\par 28 (21:33) Sine čovječji, prorokuj: Ovako govori Jahve Gospod sinovima  Amonovim o njihovoj sramoti. Reci: 'Mač! Mač za pokolj isukan  i naoštren da siječe, da kao munja sijeva, 
\par 29 (21:34) a tebi dotle isprazno  viđaju, laž proriču - da se stavi pod vrat zlikovcima zloglasnim, kojima, eto, dođe dan i čas posljednjega zločina! 
\par 30 (21:35) Ali vrati  mač u korice! U mjestu gdje si nastao i u zemlji gdje si se rodio  ja ću ti suditi. 
\par 31 (21:36) Ondje ću na te gnjev svoj izliti i raspiriti  protiv tebe plamen srdžbe svoje i predati te u ruke okrutnim  ljudima, vještim zatornicima. 
\par 32 (21:37) I bit ćeš hrana ognju, a krv  će tvoja zemljom protjecati. I nitko te živ više neće spominjati!  Jer ja, Jahve, tako rekoh.'" 


\chapter{22}

\par 1 I dođe mi riječ Jahvina: 
\par 2 "Sine čovječji, hoćeš li suditi, hoćeš li suditi gradu krvničkom? Pokaži mu sve gnusobe njegove! 
\par 3 Reci: 'Ovako govori Jahve Gospod: Grade što u sebi krv toliku  prolijevaš i što svuda sebi kumire praviš da se okaljaš, kucnu  čas tvoj: 
\par 4 krvlju što je proli ti sagriješi i kumirima koje  napravi ti se okalja, skrativši tako dane svoje i ubrzavši svoje  godine. I zato ću te sada učiniti sramotom među narodima, ruglom  po svim zemljama. 
\par 5 I koji su ti blizu i koji su ti daleko,  podrugivat će se tebi: 'O sramotno ime, grade pokvareni!' 
\par 6 Eto, knezovi izraelski - svaki na svoju ruku - u tebi krv prolijevaju. 
\par 7 I u tebi se više ne poštuje ni otac ni majka, došljake tlače, siročad i udovice u tebi zlostavljaju! 
\par 8 Svetinje moje prezireš, subote oskvrnjuješ. 
\par 9 U tebi su klevetnici zbog kojih se krv  prolijeva; u tebi se po gorama blaguje, posred tebe čine sramote. 
\par 10 U tebi se raskriva sramota očeva, u tebi siluju žene dok  su nečiste. 
\par 11 Jedan čini gadost sa ženom susjeda svoga, drugi  djelom sramotnim oskvrnjuje snahu svoju, a treći u tebi siluje  sestru, kćerku oca svoga. 
\par 12 Ima ih koji i mito primaju da krv proliju. Uzimaš ujam  i pridatak, od bližnjega silom otimaš, a mene zaboravljaš - riječ  je Jahve Gospoda. 
\par 13 Zato, evo, ja rukama plješćem nad plijenom  što ga ti napljačka i nad krvlju što se lije u tebi. 
\par 14 Jer, hoće li srce tvoje izdržati i hoće li ruke tvoje odoljeti u  dane kad ja na te ustanem? Ja, Jahve, rekoh i učinit ću! 
\par 15 Zato  ću te raspršiti među narode, rasijat' te po zemljama, da uklonim  iz tebe svu nečistoću! 
\par 16 I bit ćeš opet moja baština naočigled  naroda. I znat ćeš da sam ja Jahve!'" 
\par 17 Dođe mi riječ Jahvina: 
\par 18 "Sine čovječji, dom Izraelov  troska mi postade: bakar, srebro, kositar, željezo i olovo u  peći - svi su oni troska! 
\par 19 Stoga ovako govori Jahve Gospod:  'Jer mi troska postadoste, skupit ću vas, evo, u Jeruzalemu. 
\par 20 Kao što se skuplja srebro, bakar, željezo, olovo i kositar  u peći te se okolo oganj potpiri da se sve rastali, tako ću i  ja vas skupiti u svojem gnjevu i u svojoj jarosti, složiti vas  i rastaliti. 
\par 21 Jest, skupit ću vas i potpiriti oko vas oganj  svoje jarosti da se usred grada rastalite. 
\par 22 Kao što se srebro  u peći topi, tako ćete se i vi u njemu rastopiti. I znat ćete  da ja, Jahve, gnjev svoj na vas izlijevam!'" 
\par 23 Dođe mi riječ Jahvina: 
\par 24 "Sine čovječji, reci još:  'Ti si zemlja još neočišćena, koju još ne opra kiša dana jarosnoga! 
\par 25 Knezovi njezini, poput lavova što riču i plijen razdiru,  ljude proždiru, otimlju im blago i dragocjenosti, množeći udovice  usred nje. 
\par 26 Svećenici njezini ne poštuju mog Zakona i oskvrnjuju  moje svetinje, ne razlikujući sveto od nesvetoga, ne učeći se  lučiti nečisto od čistoga. Zanemarili su subote moje, bez časti  sam u njihovoj sredini. 
\par 27 Starješine njezine, poput vukova  što plijen razdiru i krv prolijevaju, upropašćuju ljude, lakomi  na dobitak. 
\par 28 A proroci njezini sve to premazuju bjelilom i  prekrivaju ispraznim viđenjima i lažnim proricanjima zboreći:  'Ovako govori Jahve Gospod!' - a Jahve to ne reče. 
\par 29 Imućnici  pak čine svakojaka nasilja i otimačine, siromaha i bijednika  ugnjetavaju, a došljaka bespravno tlače. 
\par 30 Tražio sam među  njima nekoga da podigne zidine i stane na proboje preda me u  obranu zemlje, da je ne zatrem, i ne nađoh nikoga. 
\par 31 I zato  izlih na njih gnjev svoj pa ih zatrijeh ognjem svoje jarosti;  putove im njihove na glavu oborih' - riječ je Jahve Gospoda." 


\chapter{23}

\par 1 Dođe mi riječ Jahvina: 
\par 2 "Sine čovječji, bile dvije žene, kćeri jedne matere. 
\par 3 I odaše se bludu u Egiptu, blud činiše  u mladosti: ondje su im grudi stiskali, djevojačke dojke gnječili. 
\par 4 Starijoj bijaše ime Ohola, a sestri joj Oholiba. Obje moje  postadoše i rodiše mi sinove i kćeri. Evo im imenÄa: Samarija  je Ohola, Jeruzalem Oholiba. 
\par 5 Ohola, iako meni pripadaše, bludu se odala; uspalila se  za ljubavnicima, za Asircima, susjedima svojim, 
\par 6 u modri baršun  odjevenima, sve samim vojvodama i namjesnicima, pristalim momcima, vještim konjanicima. 
\par 7 I oda se bludu s njima, sve poizbor  sinovima asirskim; i usplamtjev za njima, okalja se svim njihovim  kumirima. 
\par 8 A ne okani se ni bluda s Egipćanima, koji s njome  ležahu od njezine mladosti, koji su joj djevojačke dojke gnječili  i na nju blud svoj izlijevali. 
\par 9 I zato je predah u ruke njenim  ljubavnicima, u ruke Asircima za kojima se uspalila. 
\par 10 I oni  je razgoliše, zarobiše joj sinove i kćeri, a nju samu mačem pogubiše.  I postade tako primjer svim ženama kako na njoj bi sud izvršen. 
\par 11 Vidje to sestra joj Oholiba, ali se još gore uspali i  gori blud činjaše. 
\par 12 Za sinovima se asirskim uspaljivala, sve  samim vojvodama i namjesnicima, svojim susjedima, raskošno odjevenim, vještim konjanicima, poizbor momcima. 
\par 13 I vidjeh kako se okaljala:  obje su istim putem pošle. 
\par 14 Ali se ova još gore bludu odala:  kad bi ugledala muškarca na zidu naslikana, likove Kaldejaca  crvenilom nacrtane, 
\par 15 bedara pasom opasanih, sa spuštenim povezima  na glavama - sve junake, prave Babilonce, rodom iz zemlje kaldejske  - 
\par 16 tek što bi ugledala priliku njihovu, sva bi se za njima  uspalila te im slala poslanike u zemlju kaldejsku. 
\par 17 Sinovi  babilonski k njoj bi dohrlili na ljubavnu postelju da je bludom  kaljaju. A kad bi se s njima okaljala, zgadili bi joj se. 
\par 18 Ali  se razglasilo njezino bludništvo, otkrila se njena golotinja, i duša se moja od nje odvratila, kao što se bješe odvratila  od sestre njene. 
\par 19 Jer ona se još gorem bludu predala, opominjući  se dana svoje mladosti kad se u Egiptu bludu odavala, 
\par 20 uspaljujući  se za razvratnicima kojima muška snaga bijaše kao u magaraca, a izljev kao u pastuha. 
\par 21 Tako se opet vrati sramoti svoje  mladosti, kad su joj u Egiptu grudi pritiskivali, djevičanske  dojke gnječili. 
\par 22 Zato, Oholibo, ovako govori Jahve Gospod: 'Gle, dignut  ću na te tvoje ljubavnike, koji ti se duši ogadiše, i dovest  ću ih odasvud na tebe: 
\par 23 Babilonce, sve Kaldejce, Pekođane, Šoance i Koance, a s njima sve sinove asirske - sve poizbor  momke, vojvode i namjesnike, na glasu junake, vješte konjanike. 
\par 24 I doći će na te sa sjevera sila bojnih kola i točkova s mnoštvom  naroda i svrstat' se odasvud protiv tebe sa štitovima, štitićima  i oklopima. Njima ću te na sud predati, i svojim će ti sudom  suditi. 
\par 25 Oborit ću na te svu svoju ljubomoru, neka s tobom  jarosno postupe: nos i uši neka ti odsijeku, a ostatak tvoj da  od mača padne; sinove i kćeri da ti odvedu, a ostatak tvoj da  oganj proguta. 
\par 26 I zderat će s tebe tvoje haljine i oteti sve  tvoje nakite. 
\par 27 Tako ću okončati svu tvoju sramotu i bludničenje, sve tamo od Egipta: nećeš više k njima oči dizati i nećeš se  više spominjati Egipta!' 
\par 28 Jer, ovako govori Jahve Gospod:  'Evo me! Predat ću te u ruke onima koji ti omrznuše, koji ti  se duši ogadiše. 
\par 29 Neka iskale na tebi svoju mržnju, neka ti  svu muku preotmu, a tebe nek' ostave golu i nagu! Neka se obnaži  sva golotinja tvoje bludnosti i besramnosti, tvojeg bludničenja. 
\par 30 Sve će te to stići zbog tvojeg bludničenja s narodima i jer  si se okaljala njihovim kumirima. 
\par 31 Putem si sestre svoje hodila:  dat ću ti u ruku čašu njezinu: 
\par 32 Ovako govori Jahve Gospod: Čašu sestre svoje ispit ćeš, čašu široku, duboku, i bit ćeš na podsmijeh i ruglo - mnogo u nju stane! - 
\par 33 napunit ćeš se pijanstva i žalosti! Čaša je to pustošenja, užasa - čaša sestre tvoje Samarije. 
\par 34 Pit ćeš je, do dna iskapiti, zatim u komade razbiti, grudi svoje izraniti. Jer, ja tako rekoh' - riječ je Jahve Gospoda! 
\par 35 Stoga ovako govori Jahve Gospod: 'Jer ti mene zaboravi  i leđa mi okrenu, snosi sada svu svoju sramotu i bestidnost!'" 
\par 36 I još mi reče Jahve: "Sine čovječji, hoćeš li suditi  Oholi i Oholibi, pokazati im njihove gadosti? 
\par 37 Preljub počiniše, ruke su im okrvavljene, s kumirima svojim preljub učiniše, djecu  koju mi porodiše provedoše kroz oganj da ih proguta. 
\par 38 Još  mi i ovo učiniše: onoga dana obeščastiše moje Svetište i subote  moje oskvrnuše. 
\par 39 Jer istoga dana kad djecu svoju kumirima  klaše, u Svetište moje dođoše da ga obeščaste. Eto, tako uradiše  usred Doma mojega. 
\par 40 Slale su čak po muškarce izdaleka, i oni  bi im pohrlili čim bi glasnici k njima stigli. A ti se za njih  kupala, oči svoje mazala i nakitom se kitila. 
\par 41 A potom bi  sjedala na raskošnu postelju pred kojom stol prostrt bijaše na  koji si stavljala moj tamjan i moje ulje. 
\par 42 Tu se čulo pocikivanje  bezbrižnog društva zbog velikog mnoštva dovedena sa svih strana  pustinje; stavljali su ženama na ruke narukvice i na glavu vijence  prekrasne. 
\par 43 I rekoh: 'Sa ženom ogrezlom u preljubu još blud  tjeraju, i sama se ona još bludu odaje!' 
\par 44 Prilaze joj kao  kakvoj bludnici! Da, prilazili su k Oholi i Oholibi, pokvarenicama. 
\par 45 Zato će im pravednici suditi kao što se sudi preljubnicama  i onima koji krv prolijevaju, jer - one su preljubnice, ruke  su im okrvavljene. 
\par 46 Jer ovako govori Jahve Gospod: 'Neka se protiv njih zbor  sazove da ih izvrgnem zlostavljanju i pljački. 
\par 47 Zbor neka  ih kamenuje i mačevima raskomada; sinove i kćeri neka im pokolje, a domove ognjem spali. 
\par 48 Tako ću iz zemlje istrijebiti sramotu, da se druge žene opomenu i ne čine djela vaših sramotnih. 
\par 49 A  na vas ću oboriti svu vašu bestidnost, ispaštat ćete grijehe  idolopoklonstva. I znat ćete da sam ja Jahve Gospod.'" 


\chapter{24}

\par 1 Godine devete, devetoga mjeseca, desetoga dana, dođe mi riječ  Jahvina: 
\par 2 "Sine čovječji, zapiši ovaj dan: upravo danas kralj  babilonski zaposjede Jeruzalem. 
\par 3 Pripovijedaj domu odmetničkom  prispodobu. Reci im: Ovako govori Jahve Gospod: 'Pristavi lonac, pristavi i nalij vode u nj! 
\par 4 Baci u nj komade, sve najbolje komade mesa, but i pleće! Napuni ga ponajboljim kostima! 
\par 5 Uzmi najbolje od stada. Pod loncem vatru naloži. Neka dobro uzavri, neka se u njemu skuhaju i kosti.' 
\par 6 Jer ovako govori Jahve Gospod: 'Jao gradu krvničkom, zahrđalu loncu s kojega se hrđa ne skida! A zatim komad po komad iz njega izvadi, ali za nj ne bacaj kÓockÄe! 
\par 7 Jer krv je njegova u njemu - na golu je kamenu ostavi, po zemlji je ne razlij gdje bi je prašina mogla prekriti! 
\par 8 Da se gnjev moj raspali, da mu odmazdim, ostavih krv njegovu na kamenu golom, da se ne pokrije.' 
\par 9 Stoga ovako govori Jahve Gospod: 'Jao gradu krvničkome, jer ću veliku lomaču naložiti! 
\par 10 Skupi drva, vatru potpali, skuhaj meso, primiješaj začina, nek' izgore i kosti! 
\par 11 A zatim ga prazna na žeravicu pristavi da mjed mu se usija i nečistoća njegova sva se rastopi, da se uništi hrđa na njemu! 
\par 12 Grdne li muke! Ali se velika hrđa ne dade s njega skinuti:  i vatri odolje. 
\par 13 Sramotan je grijeh tvoj: htjedoh te očistiti, ali se ti ne htjede od grijeha očistiti; i zbog toga se više  nećeš očistiti dok nad tobom ne iskalim gnjev svoj! 
\par 14 Ja, Jahve, rekoh! I riječ ću ispuniti; neću popustiti: I neću se smilovati  niti ću se pokajati! Sudit ću te po putovima tvojim i po djelima  tvojim! - riječ je Jahve Gospoda." 
\par 15 I dođe mi riječ Jahvina: 
\par 16 "Sine čovječji, evo, nenadanom  smrću oduzet ću ti radost očinju! Ne tuguj, ne plači i ne roni  suza! 
\par 17 Jecaj tiho, ali ne žali kao što se za mrtvima žali!  I povij oko glave povez, a na noge obuj sandale. Ne prekrivaj  brade i ne jedi žalobničke pogače." 
\par 18 Ujutro tako prorokovah narodu, a uveče mi žena umrije  te sutradan uradih kao što mi bijaše zapovjeđeno. 
\par 19 Narod me  na to zapita: "Nećeš li nam reći što znači za nas to što ti radiš?" 
\par 20 Ja im odgovorih: "Dođe mi riječ Jahvina: 
\par 21 "Reci domu Izraelovu:  Ovako govori Jahve Gospod: Evo, oskvrnut ću svoje Svetište, vaš  ponos snažni, radost vam očinju i čežnju duše vaše! I sinovi  i kćeri koje ostaviste, od mača će pasti! 
\par 22 Tada ćete uraditi  kao što i ja uradih: nećete prekrivati brade i nećete jesti žalobničke  pogače! 
\par 23 Povit ćete povez oko glave i obuti na noge sandale!  I nećete više tugovati ni plakati, nego ćete čiljeti zbog svojih  nedjela i jecati jedan za drugim! 
\par 24 A Ezekiel će vam biti primjer:  učinit ćete sve što je i on činio. Kad se to zbude, spoznat ćete  da sam ja Jahve!' 
\par 25 A ti, sine čovječji, doista u dan kad im oduzmem snagu, dičnu radost njihovu, radost im očinju, slast duše njihove,  sinove i kćeri njihove - 
\par 26 u taj će dan k tebi stići bjegunac  da ti to dojavi! 
\par 27 U taj će se dan tvoja usta otvoriti, i ti  ćeš tom bjeguncu progovoriti; nećeš više biti nijem. I tako ćeš  im biti znak. I oni će spoznati da sam ja Jahve!" 


\chapter{25}

\par 1 I dođe mi riječ Jahvina: 
\par 2 "Sine čovječji, okreni lice k  sinovima Amonovim te prorokuj protiv njih! 
\par 3 Reci sinovima Amonovim:  'Poslušajte riječ Jahve Gospoda! Ovako govori Jahve Gospod: Zato  što vi klicaste 'ha, ha!' nad mojim Svetištem kad ono bijaše  oskvrnuto, i nad zemljom Izraelovom kad ona bijaše opustošena, i nad domom Judinim kad odlažaše u izgnanstvo, 
\par 4 predat ću  vas, evo, u posjed sinovima Istoka da usred vas razapnu svoje  šatore, udare svoja prebivališta. Oni neka jedu tvoje plodove  i piju mlijeko tvoje! 
\par 5 Od Rabe ću pašnjake za deve načiniti, a u zemlji Amonovih sinova torove ću za ovce podići. I znat  ćete da sam ja Jahve!' 
\par 6 Jer ovako govori Jahve Gospod: 'Zato što si pljeskao rukama  i udarao nogama i svom se dušom radovao nad zemljom Izraelovom, 
\par 7 ja ću, evo, ruku na te podići i kao plijen te predati narodima!  Istrijebit ću te iz narodÄa, iskorijeniti iz zemalja! Zatrijet  ću te! I znat ćeš da sam ja Jahve! 
\par 8 Ovako govori Jahve Gospod: 'Zato što Moab i Seir govorahu:  'Gle, dom je Judin poput svih naroda', 
\par 9 otkrit ću, evo, obronke  moapske, da s kraja na kraj ostane bez gradova što bijahu ukras  zemlje: Bet Haješimot, Baal Meon i Kirjatajim. 
\par 10 Dat ću ih  u posjed sinovima Istoka, neprijateljima Amonaca, da se sinovi  Amonovi među narodima više ne spominju! 
\par 11 Tako ću izvršiti  sud nad Moabom. I znat će da sam ja Jahve!' 
\par 12 Ovako govori Jahve Gospod: 'Zato što se Edom osveti domu  Judinu i tom se osvetom teško ogriješi, 
\par 13 ovako govori Jahve  Gospod: Podići ću ruku na Edom, istrijebit ću iz njega ljude  i životinje! Pretvorit ću ga u pustinju: od Temana do Dedana  svi će od mača izginuti. 
\par 14 Tako ću se osvetiti Edomu rukom  svojega naroda izraelskog. Oni će postupiti s Edomom prema mojem  gnjevu i mojoj srdžbi. I spoznat će moju osvetu!' - riječ je  Jahve Gospoda. 
\par 15 Ovako govori Jahve Gospod: 'Zato što Filistejci izvršiše  odmazdu, krvavo se osvećujući s mržnjom u srcu, razarajući sve  zbog svojeg neprijateljstva, 
\par 16 ovako govori Jahve Gospod: Evo, podižem ruku na Filistejce, istrijebit ću Kerećane, uništit  ću sve što preostane na morskoj obali! 
\par 17 Tako ću im se strašno  osvetiti kaznama jarosnim. I kad im se osvetim, znat će da sam  ja Jahve.'" 


\chapter{26}

\par 1 Godine jedanaeste, prvoga dana u mjesecu, dođe mi riječ Jahvina: 
\par 2 "Sine čovječji, jer Tir nad Jeruzalemom klicaše: 'Ha, ha! Razbiše se ta vrata narÄodÄa, i k meni se okrenuše; obogatit ću se: on je opustošen' - 
\par 3 stoga ovako govori Jahve Gospod: 'Evo me protiv tebe, Tire, dići ću na te silne narode, kao što more valove diže! 
\par 4 Porušit će zidine tirske i razoriti sve kule njegove. A ja ću mu i prašinu pomesti, načinit' od njega pećinu golu! 
\par 5 Bit će sušilište mreža. Jer ja rekoh! - riječ je Jahve Gospoda. I narodima plijen će postati. 
\par 6 A sve kćeri njegove od mača će pasti u polju! Znat će da sam ja Jahve!' 
\par 7 Jer ovako govori Jahve Gospod: 'Gle, dovest ću na Tir sa sjevera Nabukodonozora, kralja babilonskoga, kralja nad kraljevima, s konjima i bojnim kolima, s konjanicima, četama i mnoštvom naroda! 
\par 8 Kćeri će tvoje u polju mačem posjeći! Protiv tebe dići će kule opsadne, nasuti protiv tebe nasipe i podić' protiv tebe štitove. 
\par 9 Na zidove će tvoje upraviti zidodere i tvoje će kule kukama oborit'! 
\par 10 Od nebrojenih konja njegovih svega će te prašina prekriti, a od štropota konjanika i točkova i bojnih kola njihovih zadrhtat će zidine tvoje, kad bude prolazio kroz vrata tvoja, k'o što se prolazi kroz grad osvojen. 
\par 11 Kopitima svojih konja zgazit će ti sve ulice; narod tvoj mačem će pobiti i srušiti stupovlje tvoje. 
\par 12 Poplijenit će bogatstvo tvoje, tvoje će razgrabiti blago! Razorit će tvoje zidine i kuće tvoje divne srušiti! Kamenje, drvo, prašinu tvoju u more će pobacati! 
\par 13 A ja ću prekinuti jeku tvojih pjesama, i zvuk se tvojih harfa više neće čuti! 
\par 14 Pretvorit ću te u pećinu golu, postat ćeš sušilište mrežÄa. Više se nikad nećeš podići, jer ja, Jahve, rekoh!' - to riječ je Jahve Gospoda." 
\par 15 Ovako Jahve Gospod govori Tiru: "A neće li od trijeska  pada tvojega i jecanja tvojih ranjenika, kad nastane u tebi pokolj  nemili, zadrhtati svi otoci? 
\par 16 I neće li tada svi morski knezovi  sići s prijestolja svojih, odbaciti svoje plašteve, i skinuti  vezene haljine, u strah se zaodjeti, na zemlju posjedati, dršćući  bez prestanka, užasnuti tvojim udesom? 
\par 17 A zatim će nad tobom  zakukati i reći ti: 'Kamo li propade? Kamo li s mora nestade, grade proslavljeni, što bijaše tako moćan na moru, ti i žitelji tvoji, koji strah zadavahu zemlji svoj? 
\par 18 Sada na dan pada tvojega otoci će zadrhtati, otoci u moru prestravit će se zbog propasti tvoje!' 
\par 19 Jer ovako govori Jahve Gospod: 'Kad te pretvorim u pusti  grad, kakvi su gradovi u kojima više nitko ne boravi, i kada  na tebe dovedem bezdane da te velike vode prekriju, 
\par 20 spustit  ću te s onima koji su sišli u grob, k narodu pradavnom, i smjestit  ću te u najdublje zemljine predjele, u vječnu samoću, s onima  što u grob siđoše, da se više ne vratiš u zemlju živih. 
\par 21 Pretvorit  ću te u užas i više te neće biti. Tražit će te, ali te više nikad  neće naći!' - riječ je Jahve Gospoda." 


\chapter{27}

\par 1 I dođe mi riječ Jahvina: 
\par 2 "A ti sine čovječji, udari u tužaljku  nad Tirom 
\par 3 i reci Tiru što leži na ulazu u more i trguje s  narodima bezbrojnih otoka: 'Ovako govori Jahve Gospod: Tire što govoraše: Ja sam lađa prekrasna, izvanredne ljepote. 
\par 4 Tvoje međe sežu u more duboko, graditelji tvoji besprimjerno te lijepa načiniše. 
\par 5 Od senirskih čempresa oplate ti sagradiše, cedar libanonski uzeše, jarbole ti podigoše; 
\par 6 od bašanskih hrastova istesaše ti vesla, od bjelokosti i šimšira s kitijimskog otočja palubu ti načiniše! 
\par 7 Od vezena lana egipatskog bijahu ti jedra da ti budu zastava! A grimiz i skrlet s eliških otoka staviše ti za krovišta. 
\par 8 Žitelji Sidona i Arvada bjehu ti veslači, a mudraci tvoji, Tire, bijahu ti kormilari! 
\par 9 Starješine gebalske i vještaci popravljahu kvarove tvoje. Sve morske lađe i mornari bijahu tvoji i s tobom trgovahu! 
\par 10 Perzijanci, Ludijci i Putijci u tvojoj vojsci bijahu ratnici, u tebi vješahu štitove i kacige; oni ti sjaj davahu. 
\par 11 Sinovi  arvadski s vojnicima na bedemima tvojim uokrug čuvahu ti kule.  O zidove ti uokolo štitove vješahu da uzveličaju jedinstvenu  ljepotu tvoju! 
\par 12 Zbog bogatstva tvoga golemog čak i Taršiš  s tobom trgovaše, plaćajući srebrom i gvožđem, olovom i kositrom  trg tvoj. 
\par 13 Javan i Tubal i Mešek s tobom trgovahu: davahu  ljude i suđe mjedeno za trg tvoj. 
\par 14 Oni iz Bet Togarme davahu  konje, trkaće konjiće i mazge. 
\par 15 I sinovi Dedanovi s tobom  trgovahu. Mnogi ti otoci bijahu podložni: plaćahu ti daću u bjelokosti  i ebanovini. 
\par 16 Zbog obilja robe tvoje Edom s tobom trgovaše.  Davahu ti za trg dragulje, purpur i vezivo, koralje, rubine i  bez; 
\par 17 i Judeja i zemlja Izraelova trgovahu s tobom: minitskim  žitom, voskom, medom, uljem i balzamom trg tvoj plaćahu! 
\par 18 Zbog  obilja trga tvojeg, silnoga ti blaga, i Damask s tobom trgovaše  za helbonsko vino i saharsku vunu. 
\par 19 I Dan i Javan iz Uzala  za trg tvoj prekaljeno gvožđe mijenjahu, cimet i slatku trsku. 
\par 20 Dedan s tobom trgovaše prostirkama jahačkim. 
\par 21 Arapi i  kedarski knezovi mijenjahu se s tobom, trg ti plaćajući jaganjcima, jarcima i ovnovima. 
\par 22 Trgovci iz Šebe i Rame trgovahu s tobom, za trg ti davahu najbolje dragulje i zlato. 
\par 23 Haran, Kane  i Eden, trgovci Šebe, Asirije i Kišmada trgovahu s tobom. Mijenjahu  za trg tvoj 
\par 24 skupocjene halje, purpurne i vezene plašteve, sagove šarene i užad čvrsto pletenu. 
\par 25 Taršiške su lađe nakrcane  prevozile robu tvoju! Bješe tako puna i teška veoma. 
\par 26 Na pučinu morsku, na mnoga te mora izvedoše veslači. Istočni te vjetar razbi na pučini morskoj! 
\par 27 Tvoje blago i trg ti, rukodjela tvoja, lađari tvoji i krmilari, popravljači pukotina, mjenjači trga tvojeg, svi ratnici na tebi i sve mnoštvo posred tebe potonut će na dno morsko na dan tvoga brodoloma! 
\par 28 Na vapaj ti krmilara obale će zadrhtati. 
\par 29 I sići će s lađa svojih svi veslači, svi lađari i svi krmilari i ostat će na kopnu. 
\par 30 Za tobom će glasno naricati i kukati gorko. Pepelom će posut glave, i valjat se u prašini; 
\par 31 zbog tebe će glave obrijati, kostrijet će pripasati, ojađene duše za tobom naricati, i kukati gorko. 
\par 32 U žalosti će ti tužbalicu zapjevati, nad tobom će protužiti: 'Koji grad k'o Tir ponosan bješe posred mora?' 
\par 33 Jer kad bi on blago iskrcao, mnoge bi narode njima nasitio! Obiljem bogatstva i trga mnoge bi kraljeve zemaljske usrećio. 
\par 34 A sada te, evo, valovi smrskaše, potonu u dubine morske! Blago tvoje i sva posada potonuše s tobom. 
\par 35 Svi žitelji otočki zbog tebe se prestraviše. Kraljevi se njini naježiše, glave pokunjiše. 
\par 36 Trgovci narodÄa zviždahu za tobom, jer ti strašilo posta i nestade zauvijek!'" 


\chapter{28}

\par 1 I dođe mi riječ Jahvina: 
\par 2 "Sine čovječji, kaži knezu tirskome: 'Ovako govori Jahve Gospod: Tvoje se srce uzoholi, ti reče: 'Ja sam bog! Na božjem prijestolju sjedim u srcu morskom.' Iako čovjek, a ne Bog, ti srce svoje izjednači s Božjim. 
\par 3 Bješe, eto, od Daniela mudriji, nijedna ti tajna ne bje skrivena! 
\par 4 Mudrošću svojom i razborom nateče bogatstva, riznicu napuni srebrom i zlatom! 
\par 5 Mudar li bijaše trgovac, bogatstvo svoje namnoži! Al' ti se s bogatstva srce uzoholi.' 
\par 6 Stog ovako govori Jahve Gospod: 'Jer svoje srce s Božjim izjednači, 
\par 7 dovest ću, evo, na te tuđince najnasilnije među narodima. Isukat će mačeve na mudrost ti divnu, i ljepotu će ti okaljati, 
\par 8 bacit će te u jamu da umreš nasilnijom smrću od onih što umiru na pučini morskoj! 
\par 9 Hoćeš li tada pred krvnikom reći: 'Ja sam bog'? Čovjek si, a ne bog, u ruci svojih ubojica. 
\par 10 Umrijet ćeš smrću neobrezanih od ruke tuđinske! Jer ja, Jahve, rekoh to' - riječ je Jahve Gospoda." 
\par 11 I dođe mi riječ Jahvina: 
\par 12 "Sine čovječji, zakukaj  tužaljku nad tirskim kraljem. Reci mu: 'Ovako govori Jahve Gospod: Gle, ti bješe uzor savršenstva, pun mudrosti i čudesno lijep! 
\par 13 U Edenu, vrtu Božjem, ti življaše, resio te dragulj svaki, sard, topaz i dijamant, krizolit, oniks i jaspis, safir, smaragd i zlato. Načinjeni bjehu bubnjevi i frule, na dan ti rođenja bjehu pripravljeni. 
\par 14 Postavih te kao raskriljena keruba zaštitnika: bio si na svetoj gori Božjoj, hodio si posred ognjena kamenja. 
\par 15 Savršen bješe na putima svojim od dana svojega rođenja dok ti se u srcu ne zače opačina. 
\par 16 Obilno trgujući, napuni se nasiljem i sagriješi. Zato te zbacih s gore Božje, istrgoh te, kerube zaštitniče, isred ognjenoga kamenja. 
\par 17 Srce ti se uzoholi zbog ljepote tvoje, mudrost svoju odnemari zbog svojega blaga! Na zemlju te bacih i predah te zemaljskim kraljevima da te prezirno gledaju. 
\par 18 Mnoštvom svog bezakonja, nepoštenim trgovanjem oskvrnu svoja svetišta! Pustih oganj posred tebe da te proždre. Pretvorih te na zemlji u pepeo na oči onih što te motre. 
\par 19 Svi koji te poznaju među narodima zgroziše se nad tobom! Jer ti strašilo posta, nestade zauvijek.'" 
\par 20 I dođe mi riječ Jahvina: 
\par 21 "Sine čovječji, okreni lice  k Sidonu, prorokuj protiv njega. 
\par 22 Reci: 'Ovako govori Jahve Gospod: Evo me protiv tebe, Sidone, proslavit ću se usred tebe! I znat će se da sam ja Jahve kada nad njim sud izvršim i svetost svoju pokažem u njemu. 
\par 23 I poslat ću na nj kugu i krv po ulicama njegovim; i mrtvi će posred njega padati od mača, koji ti odasvud prijeti, i znat će se tada da sam ja Jahve. 
\par 24 I više neće biti domu Izraelovu trna što ranjava nit' žaoke što razdire među svima uokolo koji ga preziru! I znat će se da sam ja Jahve!'" 
\par 25 Ovako govori Jahve Gospod: "A kad skupim sav dom Izraelov  između naroda po kojima su razasuti, očitovat ću u njima svoju  svetost pred očima narodÄa. I nastanit će se u svojoj zemlji  što je dadoh sluzi svome Jakovu. 
\par 26 I u njoj će živjeti u miru, gradit će domove i saditi vinograde. Živjet će u pouzdanju dok  budem izvršivao svoj sud nad svima koji ih naokolo prezirahu.  I znat će da sam ja Jahve, Bog njihov." 


\chapter{29}

\par 1 Godine desete, desetoga mjeseca, dvanaestoga dana, dođe mi  riječ Jahvina: 
\par 2 "Sine čovječji, okreni lice faraonu, kralju  egipatskom, i prorokuj protiv njega i protiv sveg Egipta. 
\par 3 Govori  i reci: 'Ovako govori Jahve Gospod: Evo me protiv tebe, faraone, kralju egipatski, golemi krokodile što ležiš usred rijeka svojih. Ti reče: 'Rijeke su moje, sebi sam ih načinio.' 
\par 4 I zato ću ti kuke zarit' u gubicu i sve ribe rijeka tvojih zalijepiti na krljušti tvoje. Izvući ću te isred rijeka tvojih sa svim ribama rijeka tvojih zalijepljenim na tvoje krljušti. 
\par 5 Bacit ću u pustinju tebe i sve ribe iz rijeka tvojih. Na tlo ćeš poljsko pasti, nitko te neće podić' ni sahraniti, zvijerima zemaljskim i nebeskim pticama dat ću te za hranu! 
\par 6 I znat će svi stanovnici Egipta da sam ja Jahve. Jer ti bješe trska za oslonac domu Izraelovu! 
\par 7 Kad te u ruku uhvatiše, ti se slomi i rane im otvori; a kad se na te osloniše, ti prepuče i bedra im sva izrani.' 
\par 8 Stog ovako govori Jahve Gospod: 'Gle, dovest ću mač svoj na te, istrijebit ću iz tebe i ljude i stoku! 
\par 9 Sva će zemlja egipatska pustoš biti i razvalina, i oni će znati da sam ja Jahve!' Jer ti reče: 'Rijeka je moja, sebi je načinih.' 
\par 10 'Zato  evo me na te i na rijeke tvoje da pretvorim zemlju egipatsku  u pustinju i pustoš od Migdola do Sevana i do granice etiopske! 
\par 11 Neće njome više prolaziti noga ljudska ni noga životinjska, ostat će nenaseljena četrdeset godina. 
\par 12 Od zemlje ću egipatske  načiniti pustoš sred zemalja opustošenih, a gradovi njezini bit  će četrdeset godina ruševine među razvaljenim gradovima. I raspršit  ću Egipćane među narode i rasijat ću ih po zemljama.' 
\par 13 Jer, ovako govori Jahve Gospod: 'Kad mine četrdeset godina, sakupit ću opet sve Egipćane između naroda kamo bijahu raspršeni. 
\par 14 Vratit ću izgnanike egipatske, vratit ću ih opet u zemlju  Patros, domovinu njihovu, da osnuju ondje slabo kraljevstvo. 
\par 15 Ono će biti najmanje od svih kraljevstava, da se više nikad  ne digne nad druge narode. Smanjit ću ga da više nikad ne podjarmi  drugih naroda 
\par 16 i da više ne bude uzdanje domu Izraelovu. Nek'  mu u pamet doziva grijehe koje bijaše počinio okrećući se za  njima. I oni će spoznati da sam ja Jahve.'" 
\par 17 Godine dvadeset i sedme, prvoga dana prvoga mjeseca,  dođe mi riječ Jahvina: 
\par 18 "Sine čovječji, kralj babilonski Nabukodonozor  krenu s vojskom na velik pohod protiv grada Tira. I svaka glava  ogolje i svako se rame odadrije. Ali ni on ni vojska mu ne imahu  nikakve dobiti od toga što krenuše na Tir. 
\par 19 Stoga ovako govori  Jahve Gospod: 'Gle, predat ću Nabukodonozoru, kralju babilonskome, zemlju egipatsku. Odnijet će joj blago, nagrabiti plijena i  opljačkati je. To će biti plaća vojsci njegovoj. 
\par 20 Za trud  što na Tir krenu dat ću mu svu zemlju egipatsku, jer za me bijaše  radio' - riječ je Jahve Gospoda. 
\par 21 'U onaj ću dan učiniti da  izraste rog domu Izraelovu, a tebi ću usta otvoriti među njima.  I znat će da sam ja Jahve.'" 


\chapter{30}

\par 1 I dođe mi riječ Jahvina: 
\par 2 "Sine čovječji, prorokuj i reci:  'Ovako govori Jahve Gospod: Kukajte: 'Jao dana!' 
\par 3 Jer se bliži dan, bliži se dan Jahvin!  Dan oblačan, vrijeme narodima određeno. 
\par 4 I mač će ući u Egipat, a strah će ophrvati Etiopiju kad  mrtvi stanu padati po Egiptu i kad se razgrabi njegovo blago  te kad mu temelje sve sruše. 
\par 5 Kuš, Put i Lud, sva Arabija i  Libija, i sinovi zemlje Krete s njima od mača će izginuti'! 
\par 6 Ovako govori Jahve Gospod: 'Past će koji podupiru Egipat  i srozat će se ponos njegove moći. Od Migdola do Sevana sve će  u njemu od mača pasti - riječ je Jahve Gospoda. 
\par 7 On će biti pustoš među opustošenim zemljama, a njegovi  gradovi ruševine među razrušenim gradovima. 
\par 8 I znat će da sam  ja Jahve kad zapalim svoj oganj u Egiptu i zatrem sve pomagače  njegove. 
\par 9 U onaj će dan glasnici od mene na lađama isploviti da  zastraše bezbrižnu Etiopiju. I strah će je ophrvati u dan egipatski.  Jer, evo, bliži se!' 
\par 10 Ovako govori Jahve Gospod: 'Uništit  ću mnoštvo egipatsko rukom Nabukodonozora, kralja babilonskoga! 
\par 11 On i njegov narod s njime - najokrutniji među narodima -  bit će dovedeni da zemlju zatru. I oni će isukati mač na Egipat  i svu će mu zemlju truplima ispuniti. 
\par 12 A ja ću isušiti rijeke  i zemlju predati u ruke silnicima, opustošit ću zemlju i što  je u njoj - rukom tuđinaca. Ja, Jahve, rekoh!' 
\par 13 Ovako govori Jahve Gospod: 'Razorit ću kumire i ništavila  istrijebiti iz Memfisa, i neće više biti knezova u egipatskoj  zemlji, a strah ću posijati u zemlji egipatskoj. 
\par 14 Opustošit  ću Patros, zapaliti Soan, izvršiti sud na Tebi. 
\par 15 Iskalit ću  gnjev nad Sinom, tvrđom egipatskom, istrijebit ću mnoštvo u Tebi. 
\par 16 Zapalit ću oganj pod Egiptom: Sin će uzdrhtati od strave, Teba će biti osvojena, a Memfis u tjeskobi dan za danom. 
\par 17 Mladići  Heliopola i Pi-Beseta od mača će pasti. A oni će biti odvedeni  u ropstvo! 
\par 18 Nad Tafnisom pomrčat će dan kad ondje slomim jaram  egipatski i kad se dokonča ponos moći u njemu! Nad njim će se  nadviti oblak, i njegove će kćeri biti odvedene u ropstvo! 
\par 19 Tako  ću izvršiti sud nad Egiptom, i znat će da sam ja Jahve.'" 
\par 20 Godine jedanaeste, prvoga mjeseca, sedmoga dana dođe  mi riječ Jahvina: 
\par 21 "Sine čovječji, gle, slomih mišicu faraonu, kralju egipatskom! I evo, nisu je ni povili: nisu metnuli lijekove  niti su je povojima obavili da je okrijepe kako bi se opet mogla  prihvatiti mača. 
\par 22 Stoga ovako govori Jahve Gospod: 'Evo me  protiv faraona, kralja egipatskoga, da mu slomim obje ruke, i  zdravu i slomljenu, i da mu mač izbijem iz ruke! 
\par 23 Razagnat  ću Egipćane među narode i rasijati ih po zemljama! 
\par 24 Ojačat  ću ruke kralju babilonskom i mač ću svoj staviti u njegovu ruku;  a faraonu ću slomiti ruke te će stenjati pred neprijateljem kao  ranjenik. 
\par 25 Da, ojačat ću ruke kralju babilonskom, a ruke će  faraonove klonuti. I znat će se da sam ja Jahve kad metnem mač  svoj u ruke kralju babilonskom i on ga zavitla nad zemljom egipatskom. 
\par 26 Raspršit ću Egipćane među narode i rasijati ih po zemljama.  I znat će da sam ja Jahve.'" 


\chapter{31}

\par 1 Jedanaeste godine, trećega mjeseca, prvoga dana, dođe mi riječ  Jahvina: 
\par 2 "Sine čovječji, kaži faraonu, kralju egipatskom,  i mnoštvu njegovu: 'Na koga naličiš veličinom svojom? 
\par 3 Usporedit ću te, evo, s cedrom libanonskim, lijepih grana, gusta lišća i debla visoka: vrh mu do oblaka seže. 
\par 4 Voda ga othrani i uzvisi bezdan; rijekama mu svojim nasad oblijevaše, rukave svoje slaše k svem drveću poljskom. 
\par 5 I zato rastom on nadvisi sve poljsko drveće. Grane mu se namnožiše, hvoje mu se razgranaše od obilne vode što mu dotjecaše; 
\par 6 ptice mu nebeske na granama gnijezda savijahu. Ispod hvoja njegovih legoše se divlje zvijeri. A u hladu njegovu svi veliki narodi sjeđahu. 
\par 7 Lijep on bijaše veličinom i širinom svojih grana; do dubokih voda žilje mu sezaše! 
\par 8 Ne bijahu mu ravni ni cedrovi u vrtu Božjem, ni čempresi se ne mogahu usporediti s granama njegovim, a platane ni kao hvoje njegove ne bijahu! Nijedno stablo u vrtu Božjem ne bješe mu po ljepoti ravno. 
\par 9 Ukrasih ga mnoštvom grana, i zaviđaše mu sve edensko drveće u vrtu Božjem.' 
\par 10 Stoga ovako govori Jahve Gospod: 'Jer se s rasta uzoholio  što mu vrh do oblaka sezaše i srce mu visina zanese, 
\par 11 predadoh  ga u ruke najmoćnijemu od svih naroda da učini s njime po zloći  njegovoj, i odbacih ga. 
\par 12 Tuđinci, najokrutniji od naroda,  posjekoše ga i oboriše, grane mu padahu po gorama i svim dolinama, hvoje mu se po svim uvalama polomiše, svi se narodi zemlje od  njegova hlada udaljiše, ostaviše ga! 
\par 13 Na njegovo oboreno stablo  sve ptice nebeske sletješe! Među njegovim se granama sve divlje zvijeri nastaniše! 
\par 14 Da  se rastom svojim nijedno stablo pokraj vode više ne uzvisi i  da vrh svoj među oblake ne uzdigne! I da se nijedno stablo koje  pije vode u visinu svoju ne uzdaje! Jer su svi predani smrti, bačeni u podzemne krajeve, posred sinova ljudskih, s onima što  slaze u jamu!' 
\par 15 Ovako govori Jahve Gospod: 'U dan kad on siđe u Podzemlje, u znak žalosti, zatvorih nad njim ponor i zaustavih rijeke njegove.  I velike vode presahnuše te sav Libanon zbog njega u tugu zaogrnuh  i sve se poljsko drveće zbog njega osuši! 
\par 16 Gromotom pada njegova  potresoh narode kad ga strmoglavih u Podzemlje s onima što u  jamu siđoše! I u podzemnom se kraju utješi sve drveće edensko, najizabranije i najljepše u Libanonu, sve što je vodu ispijalo. 
\par 17 I ono, mišica njegova, i oni među narodima koji u hladu njegovu  sjeđahu, siđoše s njim u Podzemlje, k onima što mačem bijahu  probodeni. 
\par 18 Na koga, dakle, među drvećem edenskim, naličiš svojom  moći, slavom i veličinom? A sad si s njima oboren u podzemni  kraj i s neobrezanima ležiš među onima što mačem bijahu probodeni.  To je faraon i sve njegovo mnoštvo' - riječ je Jahve Gospoda." 


\chapter{32}

\par 1 Godine dvanaeste, dvanaestoga mjeseca, prvoga dana, dođe mi  riječ Jahvina: 
\par 2 "Sine čovječji, zaplači nad faraonom, kraljem  egipatskim, i kaži mu: 'Laviću naroda, propao si! Ti bješe kao krokodil u vodi, bučio si u rijekama svojim, nogama si vodu mutio, valove joj podizao!' 
\par 3 Ovako govori Jahve Gospod: 'Gle, razapet ću mrežu nad tobom sa skupom mnogih naroda: oni će te u mojoj mreži izvući. 
\par 4 Ostavit će te na zemlji, tresnuti tobom o tlo. Sve ptice nebeske na te ću pustiti i zvijeri zemaljske tobom ću nahraniti! 
\par 5 Meso ću ti razbacat' po gorama, doline ću prekriti strvinom tvojom. 
\par 6 Istekom iz tebe zemlju ću napojiti, krvlju tvojom po gorama, i korita riječna njome napuniti. 
\par 7 A kada te utrnem, nebesa ću potamniti i zvijezde na njima ugasiti! Oblakom ću sunce zastrijeti, i mjesec svjetlošću neće svijetliti. 
\par 8 Sva ću svjetlila na nebu zbog tebe utrnuti i mrak ću nad zemljom razastrijeti!' - riječ je Jahve Gospoda. 
\par 9 'Ucvilit ću srca mnogih naroda kad izgnanike tvoje odvedem  u zemlje tebi nepoznate. 
\par 10 Narode će mnoge strava uhvatiti, a njihovi će se kraljevi nad tobom užasnuti njima naočigled.  I na dan pada tvojega svatko će za svoj život neprestano strepiti.' 
\par 11 Jer ovako govori Jahve Gospod: 'Mač kralja babilonskoga na  te će se spustiti. 
\par 12 Tvoje ću mnoštvo pobiti mačevima junaka, najljući od svih naroda opustošit će ponos Egipta i sve mnoštvo  njegovo zatrijeti. 
\par 13 Svu ću stoku njegovu uništiti pokraj voda  obilnih. Ljudska ih noga više neće gaziti niti će ih životinjski  papak mutiti. 
\par 14 Onda ću im vode opet stišati i učinit ću da  im rijeke kao ulje teku!' - riječ je Jahve Gospoda. 
\par 15 'Kad zemlju egipatsku opustošim, kad bude opljačkano  što je na njoj, kad udarim sve žitelje njezine, znat će da sam  ja Jahve. 
\par 16 Tužaljka je to koja će se naricati. Naricat će  je kćeri naroda. Naricat će je nad Egiptom i nad svim njegovim  mnoštvom' - riječ je Jahve Gospoda." 
\par 17 Godine dvanaeste, prvoga mjeseca, petnaestoga dana, dođe  mi riječ Jahvina: 
\par 18 "Sine čovječji, nariči za egipatskim mnoštvom, gurni ga, njega i kćeri naroda slavnih, u podzemni kraj, k onima  što siđoše u jamu. 
\par 19 Od koga si bolji? Siđi i počini s neobrezanima. 
\par 20 Oni će pasti među one što ih mač pokosi. Ležaj će dobiti  sa svim mnoštvom. 
\par 21 Najhrabriji junaci govorit će mu iz srca  Podzemlja: 'Ti i pomoćnici tvoji siđite i počinite s neobrezanima, mačem pokošenima!' 
\par 22 Ondje je i Asirac i sva njegova gomila oko groba njegova  - svi pobijeni, mačem pokošeni. 
\par 23 Grobovi im leže na dnu jame  i sva mu je gomila oko grobova njegova - svi, nekoć užas u zemlji  živih, sada pobijeni, mačem pokošeni. 
\par 24 Ondje Elam i sve mnoštvo njegovo oko groba njegova -  svi pobijeni, mačem pokošeni, neobrezani u podzemni kraj mrtvih  siđoše: nekoć užas u zemlji živih, snose sad sramotu svoju s  onima što u jamu siđoše. 
\par 25 Usred pobijenih ležaj smjestiše  njemu i mnoštvu oko groba njegova - sve neobrezani, mačem probodeni:  nekoć užas u zemlji živih, snose sad sramotu svoju s onima što  u jamu siđoše, među pobijene položeni. 
\par 26 Ondje je Mešek i Tubal i sve mnoštvo njegovo, s grobovima  oko groba njegova - svi neobrezani, mačem probodeni, nekoć užas  u zemlji živih. 
\par 27 Ne leže s junacima davno palima, u Podzemlje  siđoše s oružjem, s mačem pod glavom i sa štitom na kostima,  jer bijahu užas junacima u zemlji živih. 
\par 28 I ti ćeš ležati  usred neobrezanih, mačem pokošenih. 
\par 29 Ondje je Edom i svi njegovi kraljevi i knezovi: unatoč  svojemu junaštvu, i oni leže zajedno s onima što su mačem pokošeni, s neobrezanima, s onima koji u jamu siđoše. 
\par 30 Ondje su knezovi sjevera i svi Sidonci, i oni siđoše  među probodene. Unatoč užasu svojega junaštva, leže neobrezani, s mačem probodenima, snoseći svoju sramotu s onima što u jamu  siđoše. 
\par 31 Vidjet će ih faraon i utješit će se zbog svog mnoštva  - faraon i sva vojska njegova mačem pokošena, riječ je Jahve  Gospoda! 
\par 32 Jer je zadavao strah u zemlji živih, faraon i sve  mnoštvo njegovo leže s neobrezanima, s mačem pokošenima - riječ  je Jahve Gospoda." 


\chapter{33}

\par 1 Dođe mi riječ Jahvina: 
\par 2 "Sine čovječji, govori sinovima  naroda svojega! Reci: 'Ako ja na neku zemlju dovedem mač, a narod  te zemlje uzme jednoga između sebe i postavi ga za stražara, 
\par 3 a on - videći da mač dolazi na zemlju - zatrubi u rog i opomene  sav narod: 
\par 4 ako se tada onaj koji čuje glas roga ne da opomenuti  te mač dođe i pogubi ga - krv njegova past će na glavu njegovu: 
\par 5 jer, čuo je glas roga, ali se ne dade opomenuti - krv njegova  past će na njega. Da se dao opomenuti, spasio bi život. 
\par 6 A  opet, ako stražar - videći da mač dolazi na zemlju - ne zatrubi  u rog i ne opomene narod te mač dođe i pogubi koga od njih: taj  je, doduše, poginuo zbog svoga grijeha, ali ću ja krv njegovu  tražiti iz stražarove ruke.' 
\par 7 I tebe sam, sine čovječji, postavio za stražara domu Izraelovu:  kad čuješ riječ iz mojih usta, opomeni ih u moje ime. 
\par 8 Reknem  li bezbožniku: 'Bezbožniče, umrijet ćeš!' - a ti ne progovoriš  i ne opomeneš bezbožnika da se vrati od svojega zloga puta, bezbožnik  će umrijeti zbog svojega grijeha, ali krv njegovu tražit ću iz  tvoje ruke. 
\par 9 Ali ako bezbožnika opomeneš da se vrati od svojega  zloga puta, a on se ne vrati sa svojega puta: on će umrijeti  zbog svojega grijeha, a ti si spasio život svoj. 
\par 10 Sine čovječji, reci domu Izraelovu: Vi govorite: 'Prijestupi  i grijesi naši pritišću nas i zbog njih propadamo! I da još živimo?' 
\par 11 Odgovori im: 'Života mi moga - riječ je Jahve Gospoda - nije  meni do smrti bezbožnikove, nego da se odvrati od zloga puta  svojega i da živi! Obratite se, dakle, obratite od zloga puta  svojega! Zašto da umrete, dome Izraelov!' 
\par 12 Sine čovječji, reci sinovima naroda svoga: 'Pravednika  neće izbaviti pravednost njegova u dan kad sagriješi niti će  bezbožnik stradati zbog svoje bezbožnosti u dan kad se od nje  odvrati, kao što ni pravednik neće moći ostati na životu u dan  kad sagriješi. 
\par 13 Reknem li ja prevedniku: 'Živjet ćeš!' a on  se pouzda u svoju pravednost i stane činiti nepravdu, zaboravit  ću svu njegovu pravednost, i on će umrijeti zbog nepravde što  je počini! 
\par 14 A reknem li bezbožniku: 'Umrijet ćeš!' a on se  odvrati od grijeha svojega i stane raditi po zakonu i pravdi, 
\par 15 vrati zalog, plati oteto i stane živjeti po zakonima života, ne čineći bezakonja - živjet će, neće umrijeti! 
\par 16 I svi grijesi  njegovi što ih bijaše počinio bit će mu zaboravljeni. Radi po  zakonu i pravdi, živjet će!' 
\par 17 Ali sinovi naroda tvoga govore: 'Jahvin put nije pravedan!'  Njihov put nije pravedan! 
\par 18 Ako se pravednik odvrati od svoje  pravednosti i stane činiti nepravdu, on će stoga umrijeti. 
\par 19 A  ako se bezbožnik odvrati od svoje bezbožnosti i stane raditi  po zakonu i pravdi, on će zbog toga živjeti. 
\par 20 A vi velite:  'Jahvin put nije pravedan!' Svakome ću od vas suditi prema putovima  njegovim, dome Izraelov!" 
\par 21 Godine dvanaeste, desetoga mjeseca, petoga dana našeg  izgnanstva, dođe k meni bjegunac iz Jeruzalema i reče: "Pade  grad!" 
\par 22 Ruka se Jahvina spustila na me uveče, prije dolaska  toga bjegunca, i otvorila mi usta prije negoli on dođe k meni  ujutro! Otvoriše mi se, dakle, usta i ja više ne bijah nijem. 
\par 23 I dođe mi riječ Jahvina: 
\par 24 "Sine čovječji, oni koji  žive u ovim ruševinama zemlje Izraelove govore: 'Jedan bijaše  Abraham i baštini ovu zemlju, a nas je mnogo - nama je zemlja  dana u posjed!' 
\par 25 Stoga im reci: 'Ovako govori Jahve Gospod:  Vi blagujete po gorama, oči podižete kumirima svojim, krv prolijevate  - i još da posjedujete ovu zemlju? 
\par 26 Na svoj se mač oslanjate, činite gadosti, oskvrnjujete ženu bližnjega - i još da posjedujete  ovu zemlju?' 
\par 27 Ovo im reci: 'Ovako govori Jahve Gospod: Života  mi moga, oni koji su u ruševinama od mača će pasti; one koji  su u polju dat ću zvijerima da ih proždru; a koji su u utvrdama  i po pećinama od kuge će poginuti! 
\par 28 Tako ću zemlju ovu razoriti  i opustošiti i nestat će zauvijek drskoga njezina ponosa. Opustjet  će gore Izraelove i nitko više neće njima prolaziti. 
\par 29 I znat  će da sam ja Jahve kad zemlju njihovu razorim i opustošim zbog  svih gadosti što ih počiniše.' 
\par 30 A o tebi, sine čovječji, sinovi naroda tvoga kazuju uza  zidove i na kućnim vratima i govore jedan drugom: 'Hajde da čujemo  kakva je to riječ došla od Jahve!' 
\par 31 I hrle k tebi kao na zbor  narodni; i narod moj sjeda preda te i sluša tvoje riječi, ali  ih ne izvršuje: naslađuju se njima u ustima, a srce im ide za  nepravednim dobitkom. 
\par 32 I gle, ti si za njih kao slatka pjesma  uz glazbu otpjevana glasom umilnim: riječi ti slušaju, ali ih  ne izvršuju. 
\par 33 Ali kad sve ovo dođe - gle, već dolazi - znat  će da prorok bijaše među njima!" 


\chapter{34}

\par 1 I dođe mi riječ Jahvina: 
\par 2 "Sine čovječji, prorokuj protiv  Izraelovih pastira, prorokuj im i reci: 'Ovako govori Jahve Gospod:  Jao pastirima Izraelovim koji napasaju sami sebe! Ne moraju li  pastiri napasati stado? 
\par 3 Mlijekom se hranite, vunom odijevate, ovnove tovne koljete, a stada ne pasete. 
\par 4 Nemoćnih ne krijepite, bolesnih ne liječite, ranjenih ne povijate, zalutalih natrag  ne dovodite, izgubljenih ne tražite, nego nasilno i okrutno njima  gospodarite. 
\par 5 I tako se ovce raspršiše nemajuć' pastira, i  raspršene postadoše plijen zvijerima. 
\par 6 Ovce lutaju po svim  gorama i visokim bregovima; po svoj su zemlji raspršene ovce  moje i nitko za njih ne pita, nikoga nema da ih traži.' 
\par 7 Zato, pastiri, čujte riječ Jahvinu: 
\par 8 'Tako mi života, riječ je Jahve Gospoda: zato što ovce moje postadoše plijen  i hrana zvijerima, nemajuć' pastira, dok pastiri moji ovaca mojih  ne traže nego sami sebe pasu, a ne pasu stada mojega - 
\par 9 zato, pastiri, čujte riječ Jahvinu: 
\par 10 Ovako govori Jahve Gospod:  'Evo me na pastire! Ovce svoje tražit ću iz ruku njihovih i neću  im dati da mi više stado pasu ni da sami sebe pasu: istrgnut  ću ovce iz usta njihovih, neće im više biti hrana.' 
\par 11 Jer ovako govori Jahve Gospod: 'Evo me, sam ću potražiti  ovce svoje i sam ću ih pasti! 
\par 12 Kao što se pastir brine za  ovce svoje kad se nađe uza stado raspršeno, i ja ću se pobrinuti  za svoje ovce i skupit' ih iz svih mjesta u koje se raspršiše  u dan oblaka i mraka. 
\par 13 Izvest ću ih iz naroda, skupit ću ih  iz zemalja i dovesti ih u zemlju njihovu da ih pasem na gorama  izraelskim, po svim dolinama i travnjacima. 
\par 14 Past ću ih na  izvrsnim pašama, ovčinjaci će im biti na visokim gorama izraelskim;  ondje će počivati u dobrim ovčinjacima i past će na sočnim pašama, po gorama izraelskim. 
\par 15 Sam ću pasti ovce svoje i sam ću im  dati počinka - riječ je Jahve Gospoda. 
\par 16 Potražit ću izgubljenu, dovesti natrag zalutalu, povit ću ranjenu i okrijepiti nemoćnu, bdjeti nad pretilom i jakom - past ću ih pravedno.' 
\par 17 A vama, ovce moje, ovako govori Jahve Gospod: 'Evo me  da sudim između ovce i ovce, između ovnova i jaraca! 
\par 18 Zar  vam je malo pasti na dobroj paši te ostatak paše nogama gazite?  Malo vam je piti bistru vodu te ostatak nogama mutite? 
\par 19 A  moje ovce moraju pasti što vi nogama izgaziste, piti što vi nogama  zamutiste.' 
\par 20 Stoga ovako govori Jahve Gospod: 'Evo me da sudim između  ovce pretile i mršave! 
\par 21 Jer bokovima i plećima, bodući rogovima, slabe ovce guraste dok ih ne izguraste. 
\par 22 Ja ću izbaviti ovce  svoje da više ne budu plijenom i sudit ću između ovce i ovce. 
\par 23 Postavit ću im jednoga pastira koji će ih pasti, slugu svoga  Davida: on će ih sam pasti i bit će im pastir, 
\par 24 a ja, Jahve, bit ću njihov Bog, i moj sluga David bit će im knez. Ja, Jahve, rekoh! 
\par 25 I sklopit ću s njima Savez mira i uklonit ću iz zemlje  sve divlje zvijeri, i živjet će mirno u pustinji i spavati po  šumama. 
\par 26 Njih i sve oko brda svojega učinit ću blagoslovom  i dat ću im na vrijeme kišu, i bit će to kiša blagoslova. 
\par 27 I  drveće će poljsko donositi plodove, a zemlja će dati rod svoj.  I oni će mirno živjeti u svojoj zemlji i znat će da sam ja Jahve  kad slomim palice jarma njihova i kad ih izbavim iz ruku onih  što ih podjarmiše. 
\par 28 I neće više biti plijenom narodima, i  zvijeri ih više neće žderati, nego će mirno živjeti i nitko ih  neće plašiti. 
\par 29 I učinit ću da im probuja slavni nasad, i glad  ih više neće zatirati, u zemlji više neće podnositi rug narodÄa. 
\par 30 I znat će da sam ja, Jahve, Bog njihov, s njima i da su oni, dom Izraelov, narod moj - riječ je Jahve Gospoda. 
\par 31 Vi, ovce  moje, vi ste stado paše moje, a ja sam Bog vaš' - riječ je Jahve  Gospoda." 


\chapter{35}

\par 1 I dođe mi riječ Jahvina: 
\par 2 "Sine čovječji, okreni lice k  Seirskoj gori i prorokuj protiv nje! 
\par 3 Reci joj: 'Ovako govori Jahve Gospod: Evo me na te, Goro seirska! Ruku  ću podići na te i pretvoriti te u pustoš i pustinju. 
\par 4 Gradove  ću tvoje razvaliti i postat ćeš pustinjom. I znat ćeš da sam  ja Jahve! 
\par 5 Vječnu si mržnju gojila i maču predavala sinove  Izraelove kad bi ih nesreća pogodila i kad bi im kucnuo čas posljednjega  grijeha. 
\par 6 Zato, života mi moga - riječ je Jahve Gospoda - krvi  ću te predati i krv će te progoniti: od krvi nisi prezala, krv  će te progoniti! 
\par 7 Od Gore seirske učinit ću pustoš i pustinju, istrijebit ću iz nje polaznika i povratnika. 
\par 8 Gore njezine  napunit ću truplima: po tvojim brežuljcima, po tvojim dolinama  i po tvojim uvalama padat će mačem pokošeni. 
\par 9 Učinit ću od  tebe vječnu pustinju, gradovi se tvoji neće napučiti. I znat  ćete da sam ja Jahve!' 
\par 10 Ti reče: 'Ova dva naroda i ove dvije zemlje bit će moji;  mi ćemo ih zaposjesti, ako i jest Jahve bio ondje!' 
\par 11 'Zato, života mi moga - riječ je Jahve Gospoda - postupit  ću s tobom prema gnjevu i ljubomori s kojom ti postupi u svojoj  mržnji s njima! Upoznat ćeš me po tome kako ću ti suditi! 
\par 12 I  znat ćeš da sam ja, Jahve, čuo sve tvoje hule što ih izreče na  gore Izraelove govoreći: 'Opustješe, nama su dane za hranu!' 
\par 13 Razmetali ste se, protiv mene govorili i gomilali protiv  mene riječi; čuo sam ja!' 
\par 14 Ovako govori Jahve Gospod: 'Na  radost sve zemlje, od tebe ću učiniti pustoš. 
\par 15 Kako si se  ti radovala što opustje baština doma Izraelova, tako ću učiniti  s tobom: opustjet ćeš, Goro seirska, a s tobom i sav Edom! I  znat će se da sam ja Jahve!" 


\chapter{36}

\par 1 Sine čovječji, prorokuj gorama Izraelovim i reci: "O gore Izraelove, čujte riječ Jahvinu: 
\par 2 Ovako govori Jahve  Gospod: Neprijatelji vaši govore o vama: 'Ha! Ha! Visine vječne  postat će naš posjed!' 
\par 3 I zato prorokuj i reci: 'Ovako govori  Jahve Gospod: Sa svih vas strana pustoše i plijene da budete  posjed ostalim narodima i na jezike dođoste svjetini klevetničkoj. 
\par 4 Zato, gore Izraelove, čujte riječ Jahvinu! Ovako govori Jahve  Gospod gorama i brežuljcima, uvalama i dolinama, opustošenim  razvalinama i napuštenim gradovima koji postadoše plijen i ruglo  ostalim narodima uokolo - 
\par 5 ovako, dakle, govori Jahve Gospod:  Zaista sam govorio o ognju ljubomore svoje protiv ostalih naroda, protiv sveg Edoma, koji s radošću u srcu i s mržnjom u duši  sebi prisvoji u posjed zemlju moju da je oplijeni i opljačka.' 
\par 6 Zato prorokuj o zemlji Izraelovoj! Reci gorama i brežuljcima, uvalama i dolinama: 'Ovako govori Jahve Gospod! Evo, govorim  u ljubomori i jarosti jer moradoste podnositi rug naroda.' 
\par 7 Zato  ovako govori Jahve Gospod: 'Evo, dižem ruku i kunem se: narodi  koji su oko vas snosit će sami svoju sramotu! 
\par 8 A vi, gore Izraelove, razgranajte se i donesite rod narodu koji će skoro doći. 
\par 9 Jer, evo me k vama! K vama se okrenuh, i gajit ću vas i zasijati! 
\par 10 Razmnožit ću ljude po vama - sav dom Izraelov - gradove vam  napučiti, razvaline vaše opet podići! 
\par 11 Razmnožit ću po vama  ljude i stoku, oni će se namnožiti i naploditi - te ću vas napučiti  kao nekoć i obasuti vas dobrima više nego prije! I znat ćete  da sam ja Jahve! 
\par 12 Dovest ću k vama ljude, narod svoj, Izraela, i zaposjest će te i bit ćeš im baština i nećeš im više djecu  otimati.'" 
\par 13 Ovako govori Jahve Gospod: "A što se o tebi govori: 'Ti  si zemlja koja ljude proždire i svojem narodu djecu otima' - 
\par 14 ti više nećeš ljude proždirati ni narodu svome djece otimati  - riječ je Jahve Gospoda. 
\par 15 Ne dam da više slušaš rug pogana, ne dam da više budeš na sramotu narodima: nećeš više narodu  svojem djece otimati" - riječ je Jahve Gospoda. 
\par 16 Dođe mi riječ Jahvina: 
\par 17 "Sine čovječji, kad dom Izraelov  još življaše u svojoj zemlji, oskvrnu je svojim nedjelima i svojim  putovima. Putovi njihovi bijahu preda mnom kao nečistoća žene  nečiste. 
\par 18 I zato na njih izlih gnjev svoj zbog krvi što je  proliše i zbog kumira kojima je oskvrnuše. 
\par 19 Rasprših ih među  narode i rasijah po zemljama. Sudio sam im prema putovima i nedjelima  njihovim. 
\par 20 Ali u narodima među koje dođoše, među svim narodima  u koje dospješe, oskvrnjivahu moje sveto ime, jer o njima se  govorilo: 'To je Jahvin narod, a morade otići iz zemlje Jahvine!' 
\par 21 I meni se sažali moje sveto ime što ga dom Izraelov obeščasti  u narodima među koje dođe. 
\par 22 Reci zato domu Izraelovu: 'Ovako govori Jahve Gospod:  Što činim, ne činim radi vas, dome Izraelov, nego radi svetoga  imena svojega, koje vi oskvrnuste među narodima u koje dođoste. 
\par 23 Ja ću posvetiti ime svoje veliko koje vi oskvrnuste posred  narodÄa u koje dođoste! I znat će narodi da sam ja Jahve - riječ  je Jahve Gospoda - kad na vama, njima naočigled, pokažem svetost  svoju. 
\par 24 Tada ću vas sabrati iz svih naroda i skupiti iz svih  zemalja, natrag vas dovesti u vašu zemlju. 
\par 25 Poškropit ću vas  vodom čistom da se očistite. Očistit ću vas od svih vaših nečistoća  i od svih kumira vaših. 
\par 26 Dat ću vam novo srce, nov duh udahnut  ću u vas! Izvadit ću iz tijela vašega srce kameno i dat ću vam  srce od mesa. 
\par 27 Duh svoj udahnut ću u vas da hodite po mojim  zakonima i da čuvate i vršite moje naredbe. 
\par 28 I nastanit ćete  se u zemlji koju dadoh vašim ocima, i bit ćete moj narod, a ja  ću biti vaš Bog. 
\par 29 Izbavit ću vas od svih vaših nečistoća i  dozvat ću žito i umnožiti ga, i nikad vas više neću izvrći gladi. 
\par 30 Umnožit ću plod drveća i rod njiva da ne podnosite više zbog  gladi sramotu među narodima. 
\par 31 I tada ćete se spomenuti zlih  putova i nedjela svojih, i sami ćete sebe omrznuti zbog bezakonja  i gadosti svojih. 
\par 32 A što činim, znajte dobro, ne činim radi  vas - riječ je Jahve Gospoda! Postidite se i posramite zbog putova  svojih, dome Izraelov!' 
\par 33 Ovako govori Jahve Gospod: 'A kad vas očistim od svih  bezakonja vaših, napučit ću opet vaše gradove i sagraditi razvaline; 
\par 34 opustjela zemlja, nekoć pustinja naočigled svakom prolazniku, bit će opet obrađena. 
\par 35 Tada će se reći: 'Evo zemlje što bijaše  pusta, a postade kao vrt edenski! Gle gradova što bijahu pusti, same razvaline i ruševine, a sada su utvrđeni i napučeni!' 
\par 36 I  narodi oko vas koji preostanu znat će da ja, Jahve, razvaljeno  opet gradim, i što bi opustošeno, opet sadim. Ja, Jahve, rekoh  i učinit ću!' 
\par 37 Ovako govori Jahve Gospod: Još će ovo moliti dom Izraelov:  da im ljudstvo namnožim kao stada. 
\par 38 Kao svetim stadima, kao  stadima blagdanskih dana u Jeruzalemu, gradovi, nekoć razvaline, napučit će se ljudstvom. I znat će da sam ja Jahve." 


\chapter{37}

\par 1 I spusti se na me ruka Jahvina i Jahve me u svojem duhu izvede  i postavi usred doline pune kostiju. 
\par 2 Provede me kroz njih, svuda oko njih, i gle, bijaše ih u dolini veoma mnogo i bijahu  sasvim suhe! 
\par 3 Reče mi: "Sine čovječji, mogu li ove kosti oživjeti?"  Ja odgovorih: "Jahve Gospode, to samo ti znaš!" 
\par 4 Tad mi reče:  "Prorokuj ovim kostima i reci im: 'O suhe kosti, čujte riječ  Jahvinu!' 
\par 5 Ovako govori Jahve Gospod ovim kostima: 'Evo, duh  ću svoj udahnuti u vas i oživjet ćete! 
\par 6 Žilama ću vas ispreplesti, mesom obložiti, kožom vas obaviti i duh svoj udahnuti u vas  i oživjet ćete - i znat ćete da sam ja Jahve!'" 
\par 7 I ja stadoh prorokovati kao što mi bješe zapovjeđeno.  I dok sam prorokovao, nastade šuškanje i pomicanje i kosti se  stadoše pribirati. 
\par 8 Pogledah, i gle, po njima narasle žile  i meso; kožom se presvukoše, ali duha još ne bijaše u njima. 
\par 9 I reče mi: "Prorokuj duhu, sine čovječji, prorokuj i reci:  'Ovako govori Jahve Gospod: Od sva četiri vjetra dođi, duše,  i dahni u ova trupla da ožive!'" 
\par 10 I stadoh prorokovati kao  što mi zapovjedi, i duh uđe u njih i oživješe i stadoše na noge  - vojska vrlo, vrlo velika. 
\par 11 Reče mi: "Sine čovječji, te kosti - to je sav dom Izraelov.  Evo, oni vele: 'Usahnuše nam kosti i propade nam nada, pogibosmo!' 
\par 12 Zato prorokuj i reci im. 'Ovako govori Jahve Gospod: Ja ću  otvoriti vaše grobove, izvesti vas iz vaših grobova, narode moj, i odvesti vas u zemlju Izraelovu! 
\par 13 I znat ćete da sam ja  Jahve kad otvorim grobove vaše i kad vas izvedem iz vaših grobova, moj narode! 
\par 14 I duh svoj udahnut ću u vas da oživite, i dovest  ću vas u vašu zemlju, i znat ćete da ja, Jahve govorim i činim'  - riječ je Jahve Gospoda." 
\par 15 I dođe mi riječ Jahvina: 
\par 16 "Sine čovječji, uzmi drvo  i napiši na njemu: 'Juda i sinovi Izraelovi, njegovi saveznici!'  Onda uzmi drugo drvo i napiši na njemu: 'Josip - drvo Efrajimovo  - i sav dom Izraelov, njegov saveznik.' 
\par 17 I sastavi ih u jedno  drvo da budu kao jedno u tvojoj ruci! 
\par 18 A kad te sinovi tvojega  naroda zapitaju: 'Hoćeš li nam objasniti što to znači?' - 
\par 19 reci  im: 'Ovako govori Jahve Gospod: Evo, uzet ću drvo Josipovo, što  je u ruci Efrajimovoj, drvo Josipovo i Izraelovih plemena, njegovih  saveznika, i sastavit ću ga s drvetom Judinim te ću od njih načiniti  jedno; oba će biti jedno u mojoj ruci.' 
\par 20 Oba drveta na koja to napišeš neka ti budu u ruci, njima  naočigled. 
\par 21 I reci im: 'Ovako govori Jahve Gospod: Evo, skupit  ću sinove Izraelove iz naroda u koje dođoše, skupit ću ih odasvud  i odvesti ih u zemlju njihovu. 
\par 22 I načinit ću od njih jedan  narod u zemlji, u gorama Izraelovim, i bit će im svima jedan  kralj, i oni više neće biti dva naroda i neće više biti razdijeljeni  na dva kraljevstva. 
\par 23 I neće se više kaljati svojim kumirima, ni svojim grozotama, ni opačinama. Izbavit ću ih od svih njihovih  nevjera kojima zgriješiše i očistit ću ih, i oni će biti moj  narod, a ja njihov Bog. 
\par 24 I sluga moj David bit će im kralj  i svima će im biti jedan pastir. Živjet će po mojim zakonima, čuvajući i vršeći moje naredbe. 
\par 25 Boravit će u zemlji koju  dadoh sluzi svome Jakovu, u kojoj življahu oci vaši: u njoj će  stanovati oni i njihovi sinovi, i sinovi sinova njihovih dovijeka.  I moj sluga David bit će im knez dovijeka. 
\par 26 Sklopit ću s njima  savez mira; bit će to Savez vječan s njima. Utvrdit ću ih i razmnožiti  i postavit ću Svetište svoje zauvijek među njih. 
\par 27 Moj će Šator  biti među njima i ja ću biti Bog njihov, a oni narod moj! 
\par 28 I  kad Svetište moje bude zauvijek među njima, znat će svi narodi  da sam ja Jahve, koji posvećujem Izraela.'" 


\chapter{38}

\par 1 I dođe mi riječ Jahvina: 
\par 2 "Sine čovječji, okreni lice ka  Gogu, u zemlji Magogu, velikom knezu Mešeka i Tubala, prorokuj  protiv njega. 
\par 3 Reci: 'Ovako govori Jahve Gospod: Evo me na  te, Gože, veliki kneže Mešeka i Tubala! 
\par 4 Namamit ću te i metnut  ću ti žvale u čeljusti, izvest ću tebe i svu tvoju vojsku - konje  i konjanike, silno mnoštvo u potpunoj opremi - sve u oklopima  i sa štitovima, sve vične maču. 
\par 5 S njima je i Perzija, Etiopija i Put - svi sa štitovima  i pod kacigama; 
\par 6 zatim Gomer i sve čete njegove, Bet Togarma  s krajnjega sjevera i sve čete njezine - silan narod s tobom! 
\par 7 Dobro se spremi ti i sve mnoštvo što se oko tebe skupilo i  stani mu na čelo! 
\par 8 Poslije mnogo dana dobit ćeš zapovijed;  poslije mnogo godina navalit ćeš na zemlju, izbavljenu od mača  i skupljenu iz mojih naroda, na gore Izraelove, nekoć zadugo  puste: otkako ih izvedoh iz naroda, svi spokojno žive. 
\par 9 Dići  ćeš se, doći kao nevrijeme, kao oblak što prekrije zemlju, ti  i tvoje čete, a s vama sila naroda!' 
\par 10 Ovako govori Jahve Gospod: 'U onaj će ti se dan misli  rojiti u srcu i skovat ćeš zao naum. 
\par 11 Reći ćeš: 'Hajde da  se dignem na zemlju nebranjenu, da navalim na miran narod koji  spokojno živi bez zidina i bez prijevornica i bez vrata: 
\par 12 pa  da se plijena naplijenim i pljačke napljačkam - da ruku stavim  na razvaline opet napučene i na narod iz narÄodÄa sakupljen,  koji se bavi stadima i imanjem i živi u središtu zemlje.' 
\par 13 Šeba, Dedan i trgovci taršiški i svi njihovi lavići pitat će te: 'Zar  zato dolaziš plijeniti? I zar si radi pljačke toliku gomilu skupio  da odneseš srebro i zlato, da otmeš stoku i imanje i da se plijena  velikoga naplijeniš?' 
\par 14 Zato prorokuj, sine čovječji, i reci Gogu: 'Ovako govori  Jahve Gospod: U onaj dan kad narod moj izraelski bude spokojno  živio, ti ćeš se podići! 
\par 15 Doći ćeš iz svoga sjedišta, s krajnjega  sjevera, ti i s tobom mnogo naroda, sve samih konjanika, silno  mnoštvo, golema vojska. 
\par 16 Navalit ćeš na Izraela, narod moj, kao oblak kad pokrije zemlju. U posljednje dane dovest ću te  na svoju zemlju da me narodi upoznaju, kad na tebi, Gože, njima  naočigled, pokažem svetost svoju.' 
\par 17 Ovako govori Jahve Gospod: 'Nisi li ti onaj o kome sam  govorio, u davne dane, preko slugu svojih, proroka Izraelovih, koji u ono vrijeme prorokovaše da ću te na njih dovesti? 
\par 18 U  onaj dan kad Gog navali na zemlju Izraelovu - riječ je Jahve  Gospoda - gnjev će mi iz nosa planuti. 
\par 19 U ljubomori svojoj  i u ognju jarosti svoje odlučih: U onaj dan bit će silan potres  u zemlji Izraelovoj. 
\par 20 I trest će se poda mnom ribe morske  i ptice nebeske, poljske zvijeri i gmazovi što gmižu po zemlji  i svi ljudi što žive na njoj. Planine će se razvaliti, vrleti  popadati i sve se zidine porušiti! 
\par 21 I po svim svojim gorama  pozvat ću na njega mač - riječ je Jahve Gospoda - s mačem će  se brat na brata dići! 
\par 22 Sudit ću mu kugom i krvlju. I spustit  ću silan pljusak, i kamenje tÓuče, oganj i sumpor na nj, na njegove  čete i na mnogi narod koji bude s njime. 
\par 23 I uzveličat ću se, posvetiti i objaviti pred svim narodima, i znat će da sam ja  Jahve.' 


\chapter{39}

\par 1 Sine čovječji, prorokuj protiv Goga i reci: 'Ovako govori  Jahve Gospod: Evo me na te, Gože, veliki kneže Mešeka i Tubala! 
\par 2 Namamit ću te i povesti, podići te s krajnjega sjevera i dovesti  na gore Izraelove. 
\par 3 Izbit ću ti luk iz lijeve ruke i prosuti  strijele iz tvoje desnice. 
\par 4 Na gorama ćeš Izraelovim pasti, ti i sve tvoje čete i narodi koji budu s tobom: pticama grabljivicama, svemu krilatom, i zvijerima dadoh te za hranu. 
\par 5 Na otvorenom  ćeš polju pasti, jer ja tako rekoh - riječ je Jahve Gospoda. 
\par 6 Poslat ću oganj na Magog i na sve koji spokojno žive na otocima  - i znat će da sam ja Jahve. 
\par 7 A svoje sveto ime objavit ću  posred naroda svoga izraelskoga i neću dati da se više oskvrnjuje  moje sveto ime! I znat će svi narodi da sam ja, Jahve, Svetac  Izraelov. 
\par 8 Evo dolazi i biva - riječ je Jahve Gospoda! To je dan  o kojem sam govorio! 
\par 9 Izići će stanovnici izraelskih gradova, naložiti vatru  i spaliti oružje, štitove, štitiće, lukove i strelice, koplja  i sulice - ložit će njima vatru sedam godina. 
\par 10 Neće nositi  drva iz polja ni sjeći u šumama, nego će vatru oružjem ložiti.  I oplijenit će one koji su njih plijenili, opljačkati one koji  su njih pljačkali - riječ je Jahve Gospoda. 
\par 11 U onaj ću dan dati Gogu za grob glasovito mjesto u Izraelu:  dolinu Abarim, istočno od Mora, koja zatvara put prolaznicima;  ondje će pokopati Goga i sve njegovo mnoštvo. I dolina će se  prozvati Hamon-Gog. 
\par 12 I ukopavat će ih dom Izraelov, sedam  mjeseci, da očisti svu zemlju; 
\par 13 pokapat će ih sav narod zemlje.  I bit će im slavan dan u koji se proslavim, riječ je Jahve Gospoda. 
\par 14 Izabrat će ljude da neprestano prolaze zemljom pa da s prolaznicima  pokapaju one koji preostaše po zemlji, da je tako očiste. 
\par 15 I  kad koji, prolazeći zemljom, vidi ljudske kosti, podignut će  kraj njih nadgrobnik dok ih grobari ne ukopaju u dolini Hamon-Gog. 
\par 16 Hamona je ime i gradu. I tako će očistiti zemlju. 
\par 17 Sine  čovječji, ovako govori Jahve Gospod: Reci pticama, svemu krilatom  i svemu zvijerju: skupite se i dođite! Saberite se odasvud na  žrtvu moju koju koljem za vas, na veliku gozbu po izraelskim  gorama, da se najedete mesa i napijete krvi. 
\par 18 Najedite se  mesa od junaka i napijte se krvi zemaljskih knezova, ovnova,  janjaca, jaraca, junaca, ugojene stoke bašanske! 
\par 19 Najedite  se do sita pretiline i napijte se krvi mojih klanica koje sam  vam naklao. 
\par 20 Nasitite se za mojim stolom konja i konjanika, junaka i ratnika!' - riječ je Jahve Gospoda. 
\par 21 'Tako ću se proslaviti među narodima, i svi će narodi  vidjeti sud koji ću izvršiti i ruku što ću je na njih podići. 
\par 22 Znat će dom Izraelov da sam ja, Jahve, Bog njihov - od toga  dana zauvijek. 
\par 23 I znat će narodi da dom Izraelov bijaše odveden  u ropstvo zbog svojih nedjela: iznevjerio mi se, pa sakrih lice  svoje od njih i predadoh ih njihovim neprijateljima u ruke da  od mača poginu. 
\par 24 Postupih s njima po nečistoći njihovoj i  nedjelima te sakrih lice svoje od njih.' 
\par 25 Stoga ovako govori Jahve Gospod: 'Sad ću vratiti roblje  Jakovljevo i pomilovati sav dom Izraelov - ljubomoran na ime  svoje sveto, 
\par 26 oprostit ću im svu sramotu i nevjeru kojom mi  se iznevjeriše dok još spokojno življahu u zemlji i nikoga ne  bijaše da ih straši. 
\par 27 A kad ih dovedem iz naroda i skupim  iz zemalja dušmanskih i na njima, naočigled mnogih naroda, svetost  svoju pokažem, 
\par 28 znat će da sam ja Jahve, Bog njihov, koji  ih u izgnanstvo među narode odvedoh i koji ih opet skupljam u  njihovu zemlju, ne ostavivši ondje nijednoga od njih. 
\par 29 I nikada  više neću kriti lica od njih, jer ću duh svoj izliti na dom Izraelov'  - riječ je Jahve Gospoda." 


\chapter{40}

\par 1 Godine dvadeset i pete za našega izgnanstva, početkom godine, prvoga mjeseca, desetoga dana, a četrnaest godina otkako pade  Grad, upravo onoga dana spusti se na me ruka Jahvina. 
\par 2 I odvede  me u božanskom viđenju u zemlju Izraelovu te me postavi na veoma  visoku goru: Na njoj, s južne strane, bijaše nešto kao sazidan  grad. 
\par 3 Povede me onamo, i gle: čovjek, sjajan kao mjed, stajaše  na vratima, s lanenim užetom i mjeračkom trskom u ruci. 
\par 4 I  taj mi čovjek reče: "Sine čovječji, gledaj svojim očima i slušaj  svojim ušima, popamti sve što ću ti pokazati, jer si doveden  ovamo da ti pokažem. Objavi domu Izraelovu sve što ovdje vidiš." 
\par 5 I gle, zdanje sve uokolo opasano zidom. Čovjek držaše  u ruci mjeračku trsku od šest lakata, a svaki lakat bijaše za  jedan dlan duži od običnoga lakta. On izmjeri zdanje. Širina:  jedna trska, visina: jedna trska. 
\par 6 Zatim pođe k vratima što bijahu okrenuta k istoku. Uspe  se uza stepenice i izmjeri prag vrata. Širina: jedna trska. 
\par 7 A  svaka klijet jednu trsku dugačka i jednu trsku široka. Između  klijeti: pet lakata. Prag vrata sa strane njihova trijema, iznutra, jedna trska. 
\par 8 Izmjeri trijem vrata iznutra: bijaše osam lakata  širok, 
\par 9 a njegovi polustupovi dva lakta. Trijem vrata bijaše  s nutarnje strane. 
\par 10 Na svakoj strani istočnih vrata bijahu  po tri klijeti. I sve tri bijahu iste mjere. Tako i polustupovi:  s obje strane bijahu iste mjere. 
\par 11 Zatim izmjeri vrata: bijahu  deset lakata široka i trinaest lakata visoka. 
\par 12 Pred klijetima  bijaše s jedne i s druge strane ograda od jednog lakta. Svaka  klijet: šest lakata s jedne i šest lakata s druge strane. 
\par 13 A  zatim izmjeri vrata od stražnje strane jedne klijeti do stražnje  strane nasuprotne klijeti, u širinu: bijaše dvadeset i pet lakata;  otvor pred otvorom. 
\par 14 Izmjeri i trijem: dvadeset lakata. Predvorje  bijaše sve uokolo trijema vrata. 
\par 15 Od ulaznog pročelja vrata  do nasuprotne strane njihova trijema bijaše pedeset lakata. 
\par 16 Na  klijetima i njihovim dovracima, s unutrašnje strane sve uokolo, a tako i u trijemu, bijahu prozori s rešetkama. Takvi su prozori  bili iznutra, sve naokolo, a na polustupovima palme. 
\par 17 Zatim me povede u vanjsko predvorje Doma. I gle, sve  oko predvorja prostorije i pločnik. Trideset prostorija na pločniku. 
\par 18 Pločnik bijaše sa svake strane vrata i odgovaraše razini  vrata. To je donji pločnik. 
\par 19 On izmjeri širinu predvorja od  donjih vrata do unutrašnjega predvorja, s vanjske strane: sto  lakata na istok i na sjever. 
\par 20 Sjevernim vratima vanjskoga predvorja izmjeri širinu  i dužinu. 
\par 21 I ta su imala po tri klijeti sa svake strane, a  stupovi im i trijemovi istih mjera kao u prvih vrata: pedeset  lakata u dužinu i dvadeset i pet lakata u širinu. 
\par 22 Prozori  im, trijemovi i palme bijahu iste mjere kao na istočnim vratima, a na prilazu im sedam stepenica; trijem im bijaše s unutrašnje  strane. 
\par 23 Unutrašnje predvorje imaše vrata što bijahu nasuprot  sjevernim vratima; kao i istočna. On izmjeri: između jednih i  drugih vrata bijaše sto lakata. 
\par 24 Zatim me povede na jug, i gle: i ondje vrata. Izmjeri  ondje klijeti, polustupove i trijemove: bijahu iste mjere. 
\par 25 Ona, kao i njihovi trijemovi, imahu sve uokolo prozore što bijahu  kao i oni prvi. Dužina je i tu iznosila pedeset lakata, a širina  dvadeset i pet. 
\par 26 K vratima je vodilo sedam stuba; trijem im  je bio s unutrašnje strane, a na stupovima imahu po jednu palmu  sa svake strane. 
\par 27 Unutrašnje predvorje imaše jedna vrata i  s južne strane. On izmjeri: od tih vrata do južnih vrata - sto  lakata. 
\par 28 Zatim me na južna vrata uvede u unutrašnje predvorje.  I izmjeri južna vrata: bijahu istih mjera. 
\par 29 Klijeti, stupovi  i trijemovi bijahu istih mjera. Vrata i njihov trijem imahu svud  unaokolo prozore. Pedeset lakata bijaše tu u dužinu, dvadeset  i pet lakata u širinu. 
\par 30 A sve uokolo trijemovi: dvadeset i  pet lakata dugi, a pet lakata široki. 
\par 31 Trijemovi su se pružili  prema vanjskom predvorju. Na polustupovima njihovim palme, a  stubište im je imalo osam stuba. 
\par 32 Zatim me povede k istočnim vratima unutrašnjega predvorja.  I izmjeri vrata: bijahu istih mjera. 
\par 33 Klijeti im, polustupovi  i trijemovi bijahu također istih mjera. Vrata i njihov trijem  imahu svud naokolo prozore. U dužinu bješe pedeset lakata, u  širinu dvadeset i pet. 
\par 34 Trijem im se pružao prema vanjskom  predvorju. Na njihovim polustupovima s ove i s one strane bijahu  palme. Stubište im imaše osam stuba. 
\par 35 Zatim me povede k sjevernim vratima. I izmjeri ih: bijahu  istih mjera. 
\par 36 Klijeti im, polustupovi i trijemovi bijahu također  istih mjera. Vrata i njihov trijem imahu svud uokolo prozore.  Pedeset je lakata tu bilo u dužinu, a dvadeset i pet u širinu. 
\par 37 Trijem je sezao do vanjskoga predvorja. Na polustupovima  s ove i one strane bijahu palme. Stubište imaše osam stuba. 
\par 38 Uz trijemove vrata bijaše prostor s posebnim ulazom.  Ondje su se ispirale žrtve paljenice. 
\par 39 U trijemu vrata s jedne  i s druge strane bijahu po dva stola za klanje paljenicÄa, okajnicÄa  i naknadnicÄa. 
\par 40 I s vanjske strane onomu tko ulazi na ulaz  sjevernih vrata bijahu dva stola; i s druge strane, prema trijemu  vrata, dva stola. 
\par 41 Četiri stola, dakle, s jedne, a četiri  stola s druge strane vrata: u svemu osam stolova, na kojima se  klahu žrtve. 
\par 42 Osim toga, četiri stola za paljenice, od klesanoga  kamena. Bili su po lakat i pol široki i lakat visoki. Na njima  je stajao pribor za klanje žrtava paljenica i klanica. 
\par 43 Stolovi  bijahu sve uokolo obrubljeni žljebićima od jednoga dlana, zavrnutima  unutra. Na stolove se stavljalo žrtveno meso. 
\par 44 Zatim me povede u unutrašnje predvorje. U unutrašnjem  predvorju bijahu dvije prostorije: jedna bijaše sa strane sjevernih  vrata, okrenuta prema jugu, a druga sa strane južnih vrata, okrenuta  prema sjeveru. 
\par 45 I on mi reče: "Ta prostorija što je okrenuta  na jug određena je za svećenike koji obavljaju službu u Domu. 
\par 46 A prostorija što je okrenuta na sjever jest za svećenike  koji obavljaju službu na žrtveniku. To su sinovi Sadokovi, oni  između sinova Levijevih koji smiju prići k Jahvi da mu služe!" 
\par 47 On izmjeri predvorje. Dužina: sto lakata, širina: sto  lakata; bijaše četverouglasto. Pred Domom stajaše žrtvenik. 
\par 48 A zatim me povede k trijemu. Izmjeri polustupove trijema:  bijaše pet lakata na jednoj i pet lakata na drugoj strani. Vrata  bijahu široka tri lakta s jedne i tri lakta s druge strane. 
\par 49 Trijem  bijaše dugačak dvadeset lakata, a širok dvanaest lakata. Deset  je stepenica vodilo onamo. Na dovratnicima s jedne i s druge  strane stajaše po jedan stup. 


\chapter{41}

\par 1 Zatim me povede u Hekal. Izmjeri mu polustupove: bijahu široki  šest lakata s jedne i šest lakata s druge strane. 
\par 2 Vrata bijahu  široka deset lakata: sa svake strane po jedno krilo od pet lakata.  A zatim izmjeri Hekal: bijaše dugačak četrdeset, a širok dvadeset  lakata. 
\par 3 Onda uđe i izmjeri polustupove vrata: dva lakta; zatim  vrata: šest lakata; pa širinu vrata: sedam lakata. 
\par 4 Izmjeri  zatim unutrašnji prostor: dužina dvadeset lakata, širina ispred  Hekala dvadeset lakata. I reče mi: "To je Svetinja nad svetinjama." 
\par 5 Potom izmjeri zid Doma: šest lakata. Pobočne prostorije  bijahu široke četiri lakta, sve oko Doma. 
\par 6 Pobočne prostorije  bijahu jedna nad drugom, bijaše ih trideset na tri bÓoja. U hramskom  zidu bijahu, sve uokolo, zasjeci da prihvate pobočne prostorije.  Tako one ne bijahu ugrađene u hramski zid. 
\par 7 Širina se prostorija  povećavala od boja do boja, jer su one sve uokolo, na bojeve, okruživale Dom, a Dom je, kako se uzlazilo, ostavljao sve širi  prostor. S najdonjeg se boja uzlazilo na najgornji kroza srednji. 
\par 8 Onda vidjeh sve oko Doma neku uzvisinu. Osnove pobočnih  prostorija: cijela trska, šest lakata. 
\par 9 Debljina vanjskoga  zida pobočnih klijeti: pet lakata. Prolaz između pobočnih prostorija  Doma 
\par 10 i klijeti bijaše, sve uokolo Doma, dvadeset lakata širok. 
\par 11 Iz pobočne prostorije izlažahu na prolaz jedna vrata prema  sjeveru i jedna prema jugu. Prolaz bijaše širok pet lakata svud  uokolo. 
\par 12 Zdanje što zatvaraše ograđeni prostor sa zapada bijaše  široko sedamdeset lakata, a zid te građevine posvud uokolo bijaše  debeo pet lakata i dugačak devedeset lakata. 
\par 13 On izmjeri Dom:  bijaše dugačak stotinu lakata. Ograđeni prostor, zdanje mu i  zidovi, stotinu lakata dužine. 
\par 14 Širina pročelja Doma s ograđenim  prostorom prema istoku: sto lakata. 
\par 15 On izmjeri dužinu zdanja  duž ograđenog prostora što bijaše straga i hodnike s jedne i  s druge strane: stotinu lakata. Unutrašnjost Hekala, trijemovi predvorja, 
\par 16 pragovi, prozori  s rešetkama i hodnici na sve tri strane uokolo, nasuprot pragovima, bijahu sve uokolo drvetom obloženi od zemlje do prozora. Prozori  su bili zastrti. 
\par 17 Od ulaza sve do unutrašnjosti Doma, a tako  i izvana te po svem zidu uokolo, iznutra i izvana, 
\par 18 bijahu  likovi kerubina i palma. Po jedna palma između dva kerubina,  a svaki kerubin imaše dva lica: 
\par 19 prema palmi s jedne strane  lice čovječje, a prema palmi s druge strane lice lavlje. Tako  bijaše po svemu Domu sve uokolo: 
\par 20 od zemlje do ponad vrata  bijahu izdjeljani kerubini i palme, a tako i po zidu Hekala. 
\par 21 Dovratnici Hekala bijahu četverouglasti. 
\par 22 Pred Svetištem nešto kao žrtvenik od drveta: tri lakta  visok, dva lakta dugačak i dva lakta širok. Uglovi mu, podnožje  i stranice bijahu od drveta. I čovjek mi reče: "Evo stola koji  je pred licem Jahvinim!" 
\par 23 I Hekal i Svetište imahu po dvoja vrata, 
\par 24 a svaka  vrata po dva krila što se obrtahu: dva krila u jednih i dva krila  u drugih vrata. 
\par 25 A na vratima Hekala bijahu izdjeljani kerubini  i palme, kao što bijahu izdjeljani i po zidovima. Izvana pred  trijemom bijaše drvena nadstrešnica. 
\par 26 Prozori s rešetkama  i palme bijahu s jedne i s druge strane na trijemu, u pobočnim  prostorijama Doma i na nadstrešnici. 


\chapter{42}

\par 1 A zatim me povede na sjever, u vanjsko predvorje, i dovede  me do prostorija nasuprot ograđenom prostoru, nasuprot zdanju  prema sjeveru. 
\par 2 Pročelje im sa sjeverne strane bijaše dugo  sto lakata, a široko pedeset lakata. 
\par 3 Nasuprot vratima unutrašnjeg  predvorja i nasuprot pločniku vanjskoga predvorja bijahu hodnici  jedan prema drugome na tri boja. 
\par 4 Pred prostorijama bijaše  prolaz prema unutrašnjosti - deset lakata širok i sto lakata  dugačak. Vrata im bijahu okrenuta na sjever. 
\par 5 Gornje prostorije, jer im prostor oduzimahu hodnici, bijahu manje od donjih i srednjih. 
\par 6 Jer bijahu na tri boja, ali ne imahu stupova kao u predvorju.  Zato gornje prostorije bijahu uže od donjih i srednjih. 
\par 7 Vanjski  zid, duž klijeti, prema vanjskom predvorju, ispred klijeti, bijaše  dugačak pedeset lakata. 
\par 8 Jer dužina klijetima vanjskoga predvorja  bijaše pedeset lakata, a onima pred Hekalom sto lakata. 
\par 9 U  tih prostorija bijaše ulaz s istoka onomu tko im prilazi iz vanjskog  predvorja. 
\par 10 Po širini zida predvorja prema istoku, pred ograđenim  prostorom i pred samim zdanjem, bijaše još prostorijÄa. 
\par 11 Pred  njima bijaše prolaz kao ispred klijeti smještenih prema sjeveru:  jednake dužine i jednake širine; i svi im izlazi, raspored i  vrata bijahu jednaki. 
\par 12 Bili su kao ulazi u klijeti što bijahu  prema jugu: ulaz na početku svakog prolaza, nasuprot zidu zdanja, prema istoku onomu tko bi u njih ulazio. 
\par 13 I reče mi: "Sjeverne i južne prostorije ispred ograđenog  prostora jesu prostorije Svetišta: ondje svećenici koji prilaze  Jahvi blaguju najveće svetinje. Oni će ovdje odlagati najveće  svetinje, prinose, okajnice i naknadnice, jer je to mjesto sveto. 
\par 14 A kad svećenici budu ulazili, neće izlaziti iz Svetišta u  vanjsko predvorje, nego će tu ostavljati odjeću u kojoj bijahu  službu služili, jer je sveta, i oblačiti drugu odjeću da bi se  mogli približiti mjestu određenu za narod." 
\par 15 Izmjerivši unutrašnjost Doma, izvede me na istočna vrata  i izmjeri sve uokolo. 
\par 16 Mjeračkom trskom izmjeri istočnu stranu:  bijaše pet stotina trska, mjeračkih trska, 
\par 17 a zatim se okrenu  i izmjeri sjevernu stranu: bijaše pet stotina trska, mjeračkih  trska. 
\par 18 Tada se okrenu na južnu stranu i izmjeri: pet stotina  trska, mjeračkih trska. 
\par 19 Potom se okrenu na zapadnu stranu  i izmjeri: pet stotina trska, mjeračkih trska. 
\par 20 On izmjeri  zid na sve četiri strane uokolo: bijaše pet stotina trska dugačak  i pet stotina širok. Odvajao je sveto mjesto od nesvetoga. 


\chapter{43}

\par 1 Zatim me povede k vratima što gledaju na istok. 
\par 2 I gle,  Slava Boga Izraelova dolazi od istoka; šum joj kao šum velikih  voda: i zemlja se sjala od slave njegove. 
\par 3 To viđenje koje  gledah bijaše kao viđenje što ga vidjeh kad dođoh da uništim  grad i kao viđenje koje vidjeh na rijeci Kebaru. Padoh ničice. 
\par 4 A Slava Jahvina uđe u Dom na vrata koja gledaju na istok. 
\par 5 Tada me duh podiže i odvede u unutrašnje predvorje. I gle:  Dom bijaše pun Slave Jahvine. 
\par 6 I čuh glas koji mi iz Doma govori, a kraj mene netko stajaše. 
\par 7 I reče mi: "Sine čovječji, ovo je mjesto mojega prijestolja, ovo je  mjesto stopa mojih nogu: ovdje ću, posred sinova Izraelovih,  prebivati zauvijek. Izraelov dom neće više oskvrnjivati moje  sveto ime - ni oni ni njihovi kraljevi - svojim bludništvom i  truplima svojih kraljeva: 
\par 8 stavili su svoj prag do moga, svoje  dovratnike do mojih, tako da je bio samo zid između mene i njih, i oskvrnjivali su moje sveto ime gnusobama koje počiniše. I  zato ih zatrijeh u svojem gnjevu. 
\par 9 Sada će oni ukloniti daleko  od mene svoje bludništvo i trupla svojih kraljeva, a ja ću zauvijek  prebivati posred njih. 
\par 10 Sine čovječji, pokaži domu Izraelovu ovaj Dom da se posrame  sa svojih bezakonja. Neka mu izmjere razmjere. 
\par 11 Ako se posrame  zbog svega što učiniše, opiši im Dom i njegove razmjere, njegove  izlaze i ulaze, sve njegovo obličje, sve propise i sve zakone;  upoznaj ih i nacrtaj im da vide i da čuvaju i provedu sve njegovo  obličje i sve propise o njemu. 
\par 12 A ovo je zakon za Dom: navrh  gore, sav prostor uokolo, bit će najsvetija svetinja. 
\par 13 Ovo su mjere žrtvenika, na laktove - a lakat je ovdje  jedan lakat i pedalj: podnožje žrtvenika lakat dugo, lakat široko;  obrub kojim je obrubljen uokolo - jedan pedalj. Visina žrtvenika: 
\par 14 od podnožja na zemlji do donjega pojasa žrtvenika - dva lakta, a u širinu jedan lakat; od manjeg pojasa do većega četiri lakta, a u širinu jedan lakat. 
\par 15 A samo žrtvište: četiri lakta visoko.  A sa žrtvišta dižu se uvis četiri roga. 
\par 16 Žrtvište: dvanaest  lakata dugo, dvanaest lakata široko, četvorina, na sve četiri  strane. 
\par 17 A pojas: četrnaest lakata dug i četrnaest lakata  širok, na četiri strane; njegov rub uokolo pol lakta, a podnožje  oko njega uokolo jedan lakat; stepenice mu gledaju na istok." 
\par 18 I reče mi: "Sine čovječji, ovako govori Jahve Gospod:  'Ovo su propisi žrtveni po kojima se u svoje vrijeme mora podići  žrtvenik da se na njemu prinose paljenice i da se po njemu škropi  krvlju. 
\par 19 Svećenicima levitima, potomcima Sadokovim, koji pristupaju  k meni da mi služe - riječ je Jahve Gospoda - dat ćeš june za  žrtvu okajnicu. 
\par 20 Uzet ćeš njegove krvi i njome pomazati četiri  roga žrtvišta i četiri ugla pojasa i obrub sve uokolo da okajnicom  pomiriš žrtvenik. 
\par 21 Zatim uzmi june i spali ga na odijeljenom  mjestu Doma, izvan Svetišta. 
\par 22 Sutradan prinesi jarca bez mane  kao okajnicu, neka se njime okaje žrtvenik kao što je okajan  junetom. 
\par 23 A kad ga okaješ, prinesi junca bez mane i ovna bez  mane iz stada: 
\par 24 prikaži ih pred Jahvom, a svećenici neka ih  pospu solju i neka ih prinesu kao paljenicu Jahvi. 
\par 25 Sedam  dana svaki dan prinesi jednog jarca za grijeh; i neka se prinese  june i ovan iz stada, oba bez mane. 
\par 26 Sedam dana neka se pomiruje  žrtvenik i neka se čisti i posvećuje. 
\par 27 Pošto se navrše ti  dani, od osmoga dana unapredak neka svećenici žrtvuju na žrtveniku  vaše paljenice i pričesnice; i omiljet ćete mi' - riječ je Jahve  Gospoda." 


\chapter{44}

\par 1 Potom me odvede natrag k izvanjskim vratima Svetišta, koja  gledaju na istok: bijahu zatvorena. 
\par 2 I reče mi Jahve: "Ova  će vrata biti zatvorena; neka se ne otvaraju i nitko neka ne  ulazi na njih, jer ja, Jahve, Bog Izraelov, kroz njih prođoh  - zato neka budu zatvorena. 
\par 3 Samo knez, jer je knez, smije  sjesti tu i blagovati pred Jahvom; tada neka uđe kroz trijem  vrata i istim putem neka izađe." 
\par 4 Zatim me odvede k sjevernim vratima pred Dom. Pogledah, i gle: Slava Jahvina bijaše napunila Dom Jahvin. Padoh ničice. 
\par 5 Jahve mi reče: "Sine čovječji, pomno pripazi, dobro gledaj i pažljivo poslušaj  što ću ti reći o svim uredbama Doma Jahvina i o svim njegovim  zakonima. Dobro pazi tko smije ući u Dom i tko je iz Svetišta  odijeljen. 
\par 6 Reci rodu odmetničkom, domu Izraelovu: 'Ovako govori  Jahve Gospod: Previše je već vaših gnusoba, dome Izraelov! 
\par 7 Uvodili  ste tuđince, neobrezana srca i neobrezana tijela, te su ušli  u moje Svetište i oskvrnuli moj Dom dok ste vi prinosili moj  kruh i pretilinu i krv; i tako ste raskinuli moj Savez svim tim  gnusobama. 
\par 8 Niste sami čuvali moje svetinje, nego ste njih  namjestili namjesto sebe kao čuvare u mojem Svetištu.' 
\par 9 Ovako  govori Jahve Gospod: 'Nijedan tuđinac, neobrezana srca i neobrezana  tijela, da više ne ulazi u moje Svetište - nijedan tuđinac koji  živi među sinovima Izraelovim. 
\par 10 A leviti koji su se udaljili od mene, kad je ono Izrael  odlutao od mene za svojim kumirima, snosit će svoje bezakonje. 
\par 11 Služit će u Svetištu samo kao stražari na vratima Doma i  kao posluga Domu: klat će narodu paljenice i druge žrtve i bit  će mu na službu. 
\par 12 Služili su im pred njihovim kumirima i tako  naveli Dom Izraelov na bezakonje. Zato podigoh ruku na njih -  riječ je Jahve Gospoda - da snose svoje bezakonje. 
\par 13 Više neće pristupati k meni da mi služe kao svećenici  i neće više prilaziti mojim najsvetijim svetinjama, nego će snositi  svoju sramotu i gnusobe koje počiniše. 
\par 14 Postavit ću ih da  u Domu rade svaki posao i sve što treba u njemu svršiti. 
\par 15 Svećenici leviti, potomci Sadokovi, koji su mi vjerno  služili u mojem Svetištu kad su ono sinovi Izraelovi odlutali  od mene - oni smiju pristupati k meni da mi služe: služit će  preda mnom prinoseći mi pretilinu i krv - riječ je Jahve Gospoda. 
\par 16 Oni smiju ulaziti u moje Svetište i pristupati k mojem stolu  da mi služe i da vrše službu. 
\par 17 Kad budu ulazili na vrata unutrašnjega predvorja, neka  obuku lanene haljine: neka ne bude na njima ništa vuneno kad  služe na vratima unutrašnjega predvorja i Doma. 
\par 18 Na glavama  neka nose lanene kape, oko bokova gaće lanene: neka se ne pašu  ničim od čega bi se znojili. 
\par 19 Kad izlaze u vanjsko predvorje  k narodu, neka svuku haljine u kojima su služili i neka ih ostave  u prostorijama Svetišta, a neka obuku druge haljine, da ne posvete  puk svojim haljinama. 
\par 20 I neka ne briju glave, a ni bujne kose  neka ne puštaju, nego neka strigu kosu. 
\par 21 I nijedan svećenik, kad mu je poći u unutrašnje predvorje, neka ne pije vina. 
\par 22 Neka se ne žene udovicom ili puštenicom  nego samo djevojkom iz roda Izraelova ili udovicom svećenikovom. 
\par 23 Neka mi narod uče razlikovati sveto od nesvetoga, lučiti  nečisto od čistoga. 
\par 24 U parnicama oni neka budu suci: neka  sude po mojim zakonima; i neka čuvaju zakone i uredbe o svim  mojim blagdanima i neka svetkuju moje subote. 
\par 25 K mrtvacu neka ne prilaze da se ne okaljaju; samo za  ocem i za majkom, za sinom i kćerju, za bratom i sestrom još  neudatom smiju se okaljati. 
\par 26 Pošto se nakon toga koji očisti, neka mu se broji sedam dana: 
\par 27 a onda kad uđe u Svetište,  u unutrašnje predvorje da služi u Svetištu, neka prinese žrtvu  okajnicu - riječ je Jahve Gospoda. 
\par 28 Njima ne pripada nikakva baština - ja sam njihova baština;  i zato im ne dajte nikakva posjeda u Izraelu - ja sam posjed  njihov. 
\par 29 Hranit će se od žrtava prinosnica, okajnica i naknadnica, i sve zavjetovano u Izraelu njima pripada. 
\par 30 Najbolje od svih  vaših prvina i od svih vaših prinosa koje ćete prinositi pripada  svećenicima; njima ćete davati i najbolje brašno, da blagoslov  počiva na vašim domovima. 
\par 31 Svećenici ne smiju jesti mesa od  uginulih i razderanih životinja - bilo od ptica ili stoke. 


\chapter{45}

\par 1 Kad budete zemlju ždrijebom dijelili u baštinu, prinesite  kao prinos pridržan Jahvi jedan sveti dio zemlje, dugačak dvadeset  i pet tisuća lakata, širok deset tisuća; to neka bude sveto područje  uzduž i poprijeko. 
\par 2 Od toga neka bude za Svetište četvorina  od pet stotina lakata i čistina od deset lakata uokolo. 
\par 3 Od  toga područja izmjeri u dužinu dvadeset i pet tisuća lakata,  a u širinu deset tisuća: tu neka bude Svetište - Svetinja nad  svetinjama. 
\par 4 Taj sveti dio zemlje pripada svećenicima koji  služe u Svetištu i koji pristupaju k Jahvi da mu služe: tu neka  im bude mjesto za kuće; i to neka je sveto mjesto koje pripada  Svetištu. 
\par 5 Dvadeset i pet tisuća u dužinu i deset tisuća u  širinu neka bude levitima koji služe Domu: neka ondje sagrade  gradove u kojima će stanovati. 
\par 6 Za posjed gradu dodijelite pet tisuća lakata u širinu  i dvadeset i pet tisuća lakata u dužinu usporedo sa svetim područjem:  to će pripadati svemu domu Izraelovu. 
\par 7 Knezu pripada dio s obje strane svetoga područja i gradskoga  posjeda - duž svetoga područja i duž gradskoga posjeda - od zapadne  strane prema zapadu i od istočne strane prema istoku, a dužina  neka bude jednaka svakom tom dijelu, od zapadne do istočne granice. 
\par 8 To neka bude njegova zemlja, posjed u Izraelu, da knezovi  više ne tlače narod moj i da zemlju dadu domu Izraelovu po plemenima.' 
\par 9 Ovako govori Jahve Gospod: 'Dosta je, knezovi Izraelovi!  Okanite se nasilja i pljačke i vršite zakon i pravdu; izbavite  narod moj od svojih tražbina - riječ je Jahve Gospoda. 
\par 10 Mjerite  pravom mjerom: pravom efom i pravim batom. 
\par 11 Efa i bat neka  jednako hvataju: bat neka iznosi desetinu homera i efa desetinu  homera - neka im mjera bude prema homeru. 
\par 12 Šekel neka bude  dvadeset gera; mina neka vam bude dvadeset šekela, dvadeset i  pet šekela i petnaest šekela. 
\par 13 Ovo je prinos koji ćete prinositi: šestinu efe od svakoga  homera pšenice i šestinu efe od svakoga homera ječma. 
\par 14 A za  ulje ova je uredba: desetina bata od svakoga kora - deset bata  jedan je kor. 
\par 15 Od svakoga stada od dvije stotine ovaca sa sočnih izraelskih  pašnjaka po jednu ovcu za žrtvu prinosnicu, paljenicu i pričesnicu  - vama za pomirenje - riječ je Jahve Gospoda. 
\par 16 Sav narod zemlje  duguje ovaj prinos knezu Izraelovu. 
\par 17 A knez je dužan davati  žrtve paljenice, prinosnice i ljevanice za svetkovine i za mlađake, za subote i blagdane doma Izraelova: on neka se postara za okajnicu, za pomirnicu, prinosnicu, paljenicu i pričesnicu za pomirenje  doma Izraelova.' 
\par 18 Ovako govori Jahve Gospod: 'Prvoga mjeseca, prvoga dana  u mjesecu, uzmi june bez mane i okaj njime Svetište. 
\par 19 Svećenik  neka uzme krvi te žrtve okajnice i neka njome pomaže dovratnike  Doma i sva četiri ugla pojasa žrtveničkoga i dovratnike vrata  unutrašnjega predvorja. 
\par 20 Tako neka učini i sedmoga dana istoga  mjeseca za svakoga koji je sagriješio iz slabosti i neznanja.  Tako ćete dovršiti pomirenje Doma. 
\par 21 Prvoga mjeseca, četrnaestoga  dana u mjesecu, svetkujte Pashu, sedmodnevni blagdan, kad se  blaguju beskvasni hljebovi. 
\par 22 Toga dana neka knez za se i za  sav puk zemlje prinese june za okajnicu. 
\par 23 Sedam dana blagdana  neka prinosi za paljenicu Jahvi sedam junčića i sedam ovnova  bez mane - svaki dan tih sedam dana - i svaki dan jarca kao okajnicu. 
\par 24 A kao prinosnicu neka prinese efu po svakom juncu i efu po  ovnu i hin ulja na svaku efu. 
\par 25 Sedmoga mjeseca, petnaestoga dana u mjesecu, neka o blagdanu  isto toliko prinosi sedam dana: isto toliko okajnica, paljenica, prinosnica i ulja.' 


\chapter{46}

\par 1 Ovako govori Jahve Gospod: 'Vrata unutrašnjega predvorja,  koja gledaju na istok, neka budu zatvorena šest radnih dana,  a neka se otvaraju u dan subotnji; i u dan mlađaka neka se otvaraju. 
\par 2 Knez neka ulazi kroz njihov trijem i neka stane kod dovratnika;  svećenici će tada prinijeti njegovu paljenicu i pričesnicu, a  on će se pokloniti na pragu vrata i izaći. Neka se vrata ne zatvaraju  do večeri. 
\par 3 Na ulazu istih vrata subotom i na mlađake klanjat  će se Jahvi puk zemlje. 
\par 4 Paljenica koju će knez subotom prinositi Jahvi neka bude:  šest jaganjaca bez mane, ovan bez mane. 
\par 5 A prinosnica neka  bude efa po ovnu, a po jaganjcu koliko i kako tko može i hin  ulja po efi. 
\par 6 Na dan mlađaka neka se prinese june bez mane, šest jaganjaca i ovan bez mane. 
\par 7 A za prinosnicu neka se prinese  efa po junetu, efa po ovnu, a po jaganjcu koliko tko može i hin  ulja po efi. 
\par 8 Kad knez bude ulazio, neka ulazi kroz trijem vrata i istim  putem neka izlazi. 
\par 9 A kad puk zemlje o blagdanima dolazi pred  Jahvu, onaj koji na sjeverna vrata uđe da se pokloni neka izađe  na južna, a tko uđe na južna neka izađe na sjeverna: neka se  ne vraća na vrata na koja je ušao, nego neka izađe na suprotna. 
\par 10 I knez neka bude s njima: kad oni ulaze, neka i on uđe i  neka izađe kad oni izlaze. 
\par 11 O blagdanima i svetkovinama neka se kao prinosnica prinese  efa po juncu i efa po ovnu, a po jaganjcu koliko tko može i hin  ulja po efi. 
\par 12 Kad knez želi prinijeti dobrovoljnu paljenicu ili dobrovoljnu  pričesnicu Jahvi, neka mu se otvore vrata koja gledaju na istok, pa neka prinese paljenicu i svoju pričesnicu kao na dan subotnji;  potom neka iziđe, a kad iziđe, neka se zatvore vrata. 
\par 13 Svaki dan prinijet ćeš Jahvi za paljenicu janje od godine, bez mane; prinijet ćeš ga svako jutro. 
\par 14 A kao prinosnicu  na nj prinesi svako jutro šestinu efe i trećinu hina ulja da  se poškropi najfinije brašno; to neka bude svagdašnji prinos  Jahvi po vječnoj uredbi. 
\par 15 Treba dakle prinijeti jagnje, prinosnicu  i ulje svako jutro kao svagdašnju žrtvu paljenicu.' 
\par 16 Ovako govori Jahve Gospod: 'Dadne li knez dar komu svom  sinu od svoje baštine, taj će dar pripasti njima u posjed kao  baština. 
\par 17 Ako li od svoje baštine dade dar komu svom sluzi, tome će to pripadati do otpusne godine, a potom neka se vrati  knezu; baština pripada samo kneževim sinovima. 
\par 18 Knez ne smije  prisvojiti ništa od baštine naroda, pljačkajući narodni posjed.  Sinovima svojim neka dade baštinu od svojega posjeda, da se ne  raspe narod potjeran sa svojega posjeda.'" 
\par 19 Zatim me odvede kroz ulaz kraj vrata, u svećeničke prostorije  Svetišta što gledaju na sjever. I gle: ondje, u dnu, prema zapadu, jedan prostor! 
\par 20 I reče mi: "Ovo je mjesto gdje će svećenici  kuhati žrtve naknadnice i okajnice i žrtvu pomirnicu, gdje će  peći prinosnice da ih ne iznose u vanjsko predvorje te ne posvete  naroda." 
\par 21 Potom me izvede u vanjsko predvorje i provede kraj četiri  ugla predvorja, i gle, u svakom uglu predvorja bijaše malo predvorje. 
\par 22 Ta mala predvorja u četiri ugla predvorja bijahu četrdeset  lakata dugačka, trideset široka - sva četiri istih razmjera; 
\par 23 sva četiri zidom opasana, a pod zidom sve uokolo bijahu sagrađena  ognjišta. 
\par 24 I reče mi: "To su kuhinje gdje će sluge Doma kuhati  puku žrtve." 


\chapter{47}

\par 1 Zatim me odvede natrag k vratima Doma. I gle: voda izvirala  ispod praga Doma, prema istoku - jer pročelje Doma bijaše prema  istoku - i voda otjecaše ispod desne strane Doma, južno od žrtvenika. 
\par 2 Zatim me izvede na sjeverna vrata i provede me uokolo vanjskim  putem k vanjskim vratima koja gledaju na istok. I gle, voda izvirala  s desne strane. 
\par 3 Čovjek pođe prema istoku s užetom u ruci,  izmjeri tisuću lakata i prevede me preko vode, a voda mi sezaše  do gležanja. 
\par 4 Ondje opet izmjeri tisuću lakata i provede me  preko vode, a voda bijaše do koljena. I opet izmjeri tisuću lakata  i prevede me preko vode što bijaše do bokova. 
\par 5 Opet izmjeri  tisuću lakata, ali ondje bijaše potok koji ne mogoh prijeći jer  je voda nabujala te je trebalo plivati: bijaše to potok koji  se ne može prijeći. 
\par 6 I upita me: "Vidiš li, sine čovječji?" I odvede me natrag, na obalu potoka. 
\par 7 I kad se vratih, gle, na obali s obje strane  mnoga stabla. 
\par 8 I reče mi: "Ova voda teče u istočni kraj, spušta se u  Arabu i teče u more; a kad se u more izlije, vode mu ozdrave. 
\par 9 I kuda god potok protječe, sve živo što se miče oživi; i bit  će vrlo mnogo riba, jer kamo god dođe ova voda, sve ozdravi i  oživi - kuda god protječe ovaj potok. 
\par 10 I ribari će ribariti  duž mora: od En Gedija do En Eglajima sušit će se mreže; i bit  će vrlo mnogo svakovrsnih riba kao u Velikom moru. 
\par 11 A močvare  onoga mora i njegove bare neće ozdraviti: bit će za sol. 
\par 12 Duž  potoka na obje strane rast će svakovrsne voćke: lišće im neće  otpadati i s njih neće nestajati ploda; svakog će mjeseca roditi  novim plodom jer im voda dotječe iz Svetišta. Plod će njihov  biti za jelo, a lišće za lijek'. 
\par 13 Ovako govori Jahve Gospod: 'Ovo su granice u kojima ćete  podijeliti zemlju u baštinu među dvanaest plemena Izraelovih  - Josipu dva dijela. 
\par 14 Svakom će od vas pripasti podjednako  od zemlje koju se zakleh dati vašim ocima, a vama će pripasti  u baštinu. 
\par 15 Ovo su, dakle, granice zemlje: na sjeveru, od Velikoga  mora put Hetlona do Ulaza u Hamat: Sedad, 
\par 16 Berota, Sibrajim, između kraja damaščanskog i hamatskoga, i Haser Enon, prema  granici hauranskoj. 
\par 17 Granica će se, dakle, protezati od mora  do Haser Enona, kojemu je na sjeveru kraj damaščanski i hamatski  - sjeverna strana. 
\par 18 Istočna strana: između Haurana i Damaska, između Gileada i zemlje izraelske, pa Jordanom kao granicom  prema istočnomu moru do Tamara - istočna strana. 
\par 19 Južna strana:  prema jugu od Tamara do Meripskih voda i Kadeša pa potokom prema  Velikomu moru - južna strana, prema jugu. 
\par 20 A zapadna strana:  granica je Veliko more pa do nadomak Ulaza u Hamat - zapadna  strana. 
\par 21 Tu zemlju razdijelite među sobom po plemenima Izraelovim. 
\par 22 Razdijelit ćete je ždrijebom u baštinu između sebe i između  došljaka koji žive među vama i koji među vama djecu narodiše:  i njih ćete smatrati domorocima među Izraelovim sinovima, da  i oni dobiju ždrijebom baštinu među Izraelovim sinovima. 
\par 23 Svakome  tom došljaku dodijelite baštinu u plemenu u kojem živi - riječ  je Jahve Gospoda. 



\chapter{48}

\par 1 A ovo su imena plemenÄa: od krajnjega sjevera put Hetlona  prema Ulazu u Hamat i Haser Enon, od damaščanskoga kraja na sjeveru  duž Hamata, od istoka do zapada - dio Danov. 
\par 2 Uz područje Danovo, od istoka do zapada - dio Ašerov. 
\par 3 Uz područje Ašerovo, od  istoka do zapada - dio Naftalijev. 
\par 4 Uz područje Naftalijevo, od istoka do zapada - dio Manašeov. 
\par 5 Uz područje Manašeovo, od istoka do zapada - dio Efrajimov. 
\par 6 Uz područje Efrajimovo, od istoka do zapada - dio Rubenov. 
\par 7 Uz područje Rubenovo,  od istoka do zapada - dio Judin. 
\par 8 Uz područje Judino, od istoka  do zapada neka bude pridržano područje koje ćete Jahvi prinijeti:  dvadeset i pet tisuća lakata u širinu, a u dužinu kao svaki drugi  dio, od istoka do zapada. U sredini neka bude Svetište. 
\par 9 To pridržano područje koje ćete Jahvi prinijeti neka bude  dugačko dvadeset i pet tisuća lakata, široko deset tisuća. 
\par 10 To  sveto, prineseno područje za svećenike neka bude na sjeveru dvadeset  i pet tisuća lakata; prema zapadu široko deset tisuća, prema  istoku široko deset tisuća; prema jugu dugačko dvadeset i pet  tisuća. U sredini neka bude Jahvino Svetište. 
\par 11 A posvećenim  svećenicima, potomcima Sadokovim, koji su mi vjerno služili i  nisu, kao leviti, zastranili kad su ono zastranili sinovi Izraelovi: 
\par 12 njima će pripasti dio od toga najsvetijeg područja zemlje, uz područje levitsko. 
\par 13 A levitima, baš kao i području svećeničkom:  dvadeset i pet tisuća lakata u dužinu i deset tisuća lakata u  širinu - ukupno dvadeset i pet tisuća lakata u dužinu, deset  tisuća u širinu. 
\par 14 Od toga se ništa ne smije prodati ni zamijeniti;  ne smije se ni na koga prenijeti ta prvina zemlje, jer je Jahvi  posvećena. 
\par 15 Pet tisuća lakata u širinu, što ostane od onih dvadeset  i pet tisuća, neka bude opće područje: za grad, za naselje i  za čistinu. Grad neka bude u sredini. 
\par 16 Evo mjerÄa: sa sjevera  četiri tisuće i pet stotina lakata; s juga četiri tisuće i pet  stotina; s istoka četiri tisuće i pet stotina; sa zapada četiri  tisuće i pet stotina. 
\par 17 A čistina oko grada: dvije stotine  i pedeset lakata prema sjeveru, dvije stotine i pedeset prema  jugu, dvije stotine i pedeset prema istoku, dvije stotine i pedeset  prema zapadu. 
\par 18 Što ostane u dužinu, duž svetoga područja -  deset tisuća lakata prema istoku i deset tisuća prema zapadu, duž svetoga područja - to neka bude za uzdržavanje onih koji  služe gradu. 
\par 19 Ti koji služe gradu uzimat će se iz svih plemena  Izraelovih. 
\par 20 Sve, dakle, pridržano područje - dvadeset i pet  tisuća lakata sa dvadeset i pet tisuća, u četverokut - prinijet  ćete Jahvi: i sveto područje i posjed gradski. 
\par 21 Knezu pripada  što preostane: s obje strane svetoga područja i posjeda gradskoga  - prema istoku dvadeset i pet tisuća lakata, prema istočnoj strani, i prema zapadu dvadeset i pet tisuća lakata, prema zapadnoj  strani, usporedo s drugim područjima - sve je to kneževo. A u  sredini je sveto područje i Svetište Doma. 
\par 22 Od levitskoga  posjeda i od posjeda gradskoga - koje je usred kneževa - i između  Judina i Benjaminova područja: kneževo je. 
\par 23 Ostala plemena: od istoka do zapada - dio Benjaminov. 
\par 24 Uz područje Benjaminovo, od istoka do zapada - dio Šimunov. 
\par 25 uz područje Šimunovo, od istoka do zapada - dio Jisakarov. 
\par 26 Uz područje Jisakarovo, od istoka do zapada - dio Zebulunov. 
\par 27 Uz područje Zebulunovo, od istoka do zapada - dio Gadov. 
\par 28 Uz područje Gadovo, na južnoj strani, prema jugu, ide granica  od Tamara do Meripskih voda i Kadeša, pa potokom prema Velikome  moru. 
\par 29 To je zemlja koju ćete ždrijebom razdijeliti u baštinu  plemenima Izraelovim, to su njihovi dijelovi - riječ je Jahve  Gospoda. 
\par 30 [30a] A ovo su gradska vrata 
\par 31 [31a] koja će se zvati po Izraelovim  plemenima. [30b] Na sjevernoj strani - četiri tisuće i pet stotina  lakata u dužinu - [31b] troja vrata: Vrata Rubenova, Vrata Judina, Vrata Levijeva. 
\par 32 Na istočnoj strani - četiri tisuće i pet  stotina lakata u dužinu - troja vrata: Vrata Josipova, Vrata  Benjaminova, Vrata Danova. 
\par 33 Na južnoj strani - četiri tisuće  i pet stotina lakata u dužinu - troja vrata: Vrata Šimunova,  Vrata Jisakarova, Vrata Zebulunova. 
\par 34 Sa zapadne strane - četiri  tisuće i pet stotina lakata u dužinu - troja vrata: Vrata Gadova, Vrata Ašerova, Vrata Naftalijeva. 
\par 35 Sve uokolo: osamnaest  tisuća lakata. A ime će gradu unapredak biti: 'Jahve je ovdje.'" 




\end{document}