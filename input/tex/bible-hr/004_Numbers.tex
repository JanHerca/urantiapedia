\begin{document}

\title{Brojevi}


\chapter{1}

\par 1 U Sinajskoj pustinji, u Šatoru sastanka, reče Jahve Mojsiju  prvoga dana drugoga mjeseca, druge godine po izlasku iz zemlje  egipatske: 
\par 2 "Obavite popis sve zajednice izraelske po rodovima  i porodicama, navodeći imena svih muškaraca, glavu po glavu. 
\par 3 Od dvadeset godina naviše, za borbu sposobne u Izraelu, ti  i Aron pobilježite prema njihovim jedinicama. 
\par 4 Neka s vama  bude po jedan čovjek od svakoga plemena, glavari svoga pradjedovskog  doma. 
\par 5 Ovo su imena ljudi koji će vam pomagati: Elisur, sin Šedeurov, za pleme Rubenovo; 
\par 6 Šelumiel, sin Surišadajev, za pleme Šimunovo; 
\par 7 Nahšon, sin Aminadabov, za pleme Judino; 
\par 8 Netanel, sin Suarov, za pleme Jisakarovo; 
\par 9 Eliab, sin Helonov, za pleme Zebulunovo. 
\par 10 Za Josipove sinove: Elišama, sin Amihudov, za pleme Efrajimovo;  Gamliel, sin Pedahsurov, za pleme Manašeovo; 
\par 11 Abidan, sin  Gidonijev, za pleme Benjaminovo; 
\par 12 Ahiezer, sin Amišadajev, za pleme Danovo; 
\par 13 Pagiel, sin Okranov, za pleme Ašerovo; 
\par 14 Elijasaf, sin Deuelov, za pleme Gadovo; 
\par 15 Ahira, sin Enanov, za pleme Naftalijevo." 
\par 16 To bijahu sazivači zajednice, knezovi pradjedovskih plemena  i glavari rodova izraelskih. 
\par 17 Mojsije i Aron onda uzmu one ljude što su po imenu bili  određeni 
\par 18 te prvoga dana drugoga mjeseca sazovu svu zajednicu.  Tada se u popis prema rodovima i porodicama po redu unosio broj  osoba starijih od dvadeset godina. 
\par 19 Kako je Jahve naredio  Mojsiju, tako ih je on pobilježio u Sinajskoj pustinji. 
\par 20 Kad se utvrdi potomstvo Rubena, Izraelova prvorođenca, prema njegovim rodovima i porodicama, pribilježiše, glavu po  glavu, imena svih muškaraca od dvadeset godina naviše, svih za  borbu sposobnih. 
\par 21 Popisanih od Rubenova plemena bilo je četrdeset  i šest tisuća i pet stotina. 
\par 22 Bili su popisani Šimunovi potomci prema njihovim rodovima  i porodicama: pribilježiše se, glava po glava, imena svih muškaraca  od dvadeset godina naviše, svih za borbu sposobnih. 
\par 23 Popisanih  od Šimunova plemena bilo je pedeset i devet tisuća i tri stotine. 
\par 24 Kad se utvrdi potomstvo sinova Gadovih prema njihovim  rodovima i porodicama, pribilježiše se, glava po glava, imena  onih od dvadeset godina i više, svih za borbu sposobnih. 
\par 25 Popisanih  od Gadova plemena bilo je četrdeset i pet tisuća i šest stotina  i pedeset. 
\par 26 Kad se utvrdi potomstvo sinova Judinih prema njihovim  rodovima i porodicama, pribilježiše se, glava po glava, imena  onih od dvadeset godina i više, svih za borbu sposobnih. 
\par 27 Popisanih  od Judina plemena bilo je sedamdeset i četiri tisuće i šest stotina. 
\par 28 Kad se utvrdi potomstvo sinova Jisakarovih prema njihovim  rodovima i porodicama, pribilježiše se, glava po glava, imena  onih od dvadeset godina i više, svih za borbu sposobnih. 
\par 29 Popisanih  od Jisakarova plemena bilo je pedeset i četiri tisuće i četiri  stotine. 
\par 30 Kad se utvrdi potomstvo sinova Zebulunovih prema njihovim  rodovima i porodicama, pribilježiše se, glava po glava, imena  onih od dvadeset godina i više, svih za borbu sposobnih. 
\par 31 Popisanih  od Zebulunova plemena bilo je pedeset i sedam tisuća i četiri  stotine. 
\par 32 Josipovi sinovi: Kad se utvrdi potomstvo sinova Efrajimovih, prema njihovim rodovima i porodicama, pribilježiše se, glava  po glava, imena onih od dvadeset godina i više, svih za borbu  sposobnih. 
\par 33 Popisanih od Efrajimova plemena bilo je četrdeset  tisuća i pet stotina. 
\par 34 Kad se utvrdi potomstvo sinova Manašeovih, prema njihovim  rodovima i porodicama, pribilježiše se, glava po glava, imena  onih od dvadeset godina i više, svih za borbu sposobnih. 
\par 35 Popisanih  od Manašeova plemena bilo je trideset i dvije tisuće i dvjesta. 
\par 36 Kad se utvrdi potomstvo sinova Benjaminovih, prema njihovim  rodovima i porodicama, pribilježiše se, glava po glava, imena  onih od dvadeset godina i više, svih za borbu sposobnih. 
\par 37 Popisanih  od Benjaminova plemena bilo je trideset i pet tisuća i četiri  stotine. 
\par 38 Kad se utvrdi potomstvo sinova Danovih, prema njihovim  rodovima i porodicama, pribilježiše se, glava po glava, imena  onih od dvadeset godina i više, svih za borbu sposobnih. 
\par 39 Popisanih  od Danova plemena bilo je šezdeset i dvije tisuće i sedam stotina. 
\par 40 Kad se utvrdi potomstvo sinova Ašerovih, prema njihovim  rodovima i porodicama, pribilježiše se, glava po glava, imena  onih od dvadeset godina i više, svih za borbu sposobnih. 
\par 41 Popisanih  od Ašerova plemena bila je četrdeset i jedna tisuća i pet stotina. 
\par 42 Kad se utvrdi potomstvo sinova Naftalijevih, prema njihovim  rodovima i porodicama, pribilježiše se, glava po glava, imena  onih od dvadeset godina i više, svih za borbu sposobnih. 
\par 43 Popisanih  od Naftalijeva plemena bilo je pedeset i tri tisuće i četiri  stotine. 
\par 44 To su oni koje popisaše Mojsije i Aron sa dvanaest knezova  izraelskih, po jedan na svaki pradjedovski dom. 
\par 45 Bili su popisani  svi Izraelci, prema pradjedovskim domovima, od dvadeset godina  i više, svi za borbu sposobni u Izraelu. 
\par 46 Bilo je, dakle,  svih popisanih šest stotina i tri tisuće i pet stotina i pedeset. 
\par 47 Među te nisu se ubrojili Levijevci prema svojem pradjedovskom  plemenu. 
\par 48 Jahve je, naime, rekao Mojsiju: 
\par 49 "Nipošto nemoj  popisivati Levijeva plemena niti ga unosi u popis Izraelaca. 
\par 50 Nego ti sam odredi Levijevce za službu u Prebivalištu svjedočanstva;  za sav njegov namještaj i sve što na nj spada; neka oni nose  Prebivalište i sav njegov namještaj; neka oni u njemu poslužuju  i oko njega tabore. 
\par 51 Kad se Prebivalište mora premještati, neka ga Levijevci rastave; a kad se s Prebivalištem treba utaboriti, neka ga Levijevci opet podignu. Svjetovnjak koji bi mu se primakao  neka se pogubi. 
\par 52 Neka Izraelci logoruju svatko u svome taboru;  svatko kod svoje zastave, po četama. 
\par 53 Levijevci neka borave  oko Prebivališta svjedočanstva, da se gnjev ne obori na izraelsku  zajednicu. Neka tako Levijevci stražu straže oko Prebivališta  svjedočanstva." 
\par 54 Izraelci učine kako je Jahve Mojsiju naredio. U svemu  tako urade. 


\chapter{2}

\par 1 Jahve reče Mojsiju i Aronu: 
\par 2 "Neka Izraelci logoruju svatko  kod svoje zastave, pod znakovima svojih pradjedovskih domova;  neka se utabore oko Šatora sastanka, ali malo podalje. 
\par 3 Sprijeda, s istočne strane, zastava Judina tabora, prema  njihovim četama. Glavar je Judinih potomaka Nahšon, sin Aminadabov. 
\par 4 Njegova vojska broji sedamdeset i četiri tisuće i šest stotina  popisanih. 
\par 5 Do njega neka taboruje Jisakarovo pleme. Glavar je Jisakarovih  potomaka Netanel, sin Suarov. 
\par 6 Njegova vojska broji pedeset  četiri tisuće i četiri stotine popisanih. 
\par 7 Onda pleme Zebulunovo. Glavar je Zebulunovih potomaka  Eliab, sin Helonov. 
\par 8 Njegova vojska broji pedeset i sedam tisuća  i četiri stotine popisanih. 
\par 9 Prema njihovim četama, svih je upisanih u Judinu taboru  sto osamdeset i šest tisuća i četiri stotine. Neka oni prvi stupaju! 
\par 10 S juga je zastava tabora Rubenova, prema njihovim četama.  Glavar je Rubenovih potomaka Elisur, sin Šedeurov. 
\par 11 Njegova  vojska broji četrdeset i šest tisuća i pet stotina popisanih. 
\par 12 Do njega neka taboruje pleme Šimunovo. Glavar je Šimunovih  potomaka Šelumiel, sin Surišadajev. 
\par 13 Njegova vojska broji  pedeset i devet tisuća i tri stotine popisanih. 
\par 14 Onda pleme Gadovo. Glavar je Gadovih potomaka Elijasaf, sin Deuelov. 
\par 15 Njegova vojska broji četrdeset i pet tisuća  šest stotina i pedeset popisanih. 
\par 16 Prema njihovim četama, svih je upisanih u taboru Rubenovu  sto pedeset i jedna tisuća četiri stotine i pedeset. Neka oni  stupaju drugi! 
\par 17 Potom neka ide Šator sastanka, tako da tabor levitski  bude usred drugih tabora. Kako taboruju, onako neka i stupaju:  svatko pod svojom zastavom. 
\par 18 Sa zapada, zastava tabora Efrajimova, prema njihovim  četama. Glavar je Efrajimovih potomaka Elišama, sin Amihudov. 
\par 19 Njegova vojska broji četrdeset tisuća i pet stotina popisanih. 
\par 20 Do njega je pleme Manašeovo. Glavar je Manašeovih potomaka  Gamliel, sin Pedahsurov. 
\par 21 Njegova vojska broji trideset i  dvije tisuće i dvjesta popisanih. 
\par 22 Onda je pleme Benjaminovo. Glavar je potomaka Benjaminovih  Abidan, sin Gidonijev. 
\par 23 Njegova vojska broji trideset i pet  tisuća i četiri stotine popisanih. 
\par 24 Prema njihovim četama, svih je upisanih u Efrajimovu  taboru sto i osam tisuća i sto. Oni neka stupaju treći! 
\par 25 Sa sjevera, zastava tabora Danova, prema njihovim četama.  Glavar je Danovih potomaka Ahiezer, sin Amišadajev. 
\par 26 Njegova  vojska broji šezdeset i dvije tisuće i sedam stotina popisanih. 
\par 27 Do njega neka se utabori pleme Ašerovo. Glavar je Ašerovih  potomaka Pagiel, sin Okranov. 
\par 28 Njegova vojska broji četrdeset  i jednu tisuću i pet stotina popisanih. 
\par 29 Onda pleme Naftalijevo. Glavar je Naftalijevih potomaka  Ahira, sin Enanov. 
\par 30 Njegova vojska broji pedeset i tri tisuće  i četiri stotine popisanih. 
\par 31 Svih je popisanih u taboru Danovu sto pedeset i sedam  tisuća i šest stotina. Neka oni stupaju posljednji pod svojim  zastavama." 
\par 32 To su popisani Izraelci prema pradjedovskim domovima.  Svih je upisanih u taborima, po njihovim četama, šest stotina  i tri tisuće i pet stotina i pedeset. 
\par 33 Levijevci nisu bili  upisivani s Izraelcima, kako je Jahve naredio Mojsiju. 
\par 34 U svemu su Izraelci učinili kako je Jahve naredio Mojsiju.  Tako su taborovali pod svojim zastavama i tako išli, svatko prema  svom rodu i porodici. 


\chapter{3}

\par 1 Ovo je potomstvo Aronovo i Mojsijevo iz vremena kad je Jahve  Mojsiju govorio na Sinajskom brdu. 
\par 2 Ovo su bila imena Aronovih sinova: prvorođenac Nadab,  zatim Abihu, Eleazar i Itamar. 
\par 3 To su imena Aronovih sinova, svećenika pomazanih, za svećeništvo posvećenih. 
\par 4 Ali Nadab  i Abihu umriješe pred Jahvom kad su u Sinajskoj pustinji pred  njim prinosili neposvećenu vatru. Kako nisu imali sinova, to  su Eleazar i Itamar služili kao svećenici u nazočnosti svoga  oca Arona. 
\par 5 Jahve reče Mojsiju: 
\par 6 "Dozovi pleme Levijevo neka stane  pred svećenika Arona. Neka mu poslužuju; 
\par 7 neka vrše njegovu  dužnost i dužnost sve zajednice pred Šatorom sastanka, služeći  Prebivalištu. 
\par 8 Neka se brinu za sav namještaj u Šatoru sastanka, za dužnost sinova izraelovih, i obavljaju službu u Prebivalištu. 
\par 9 Podaj levite Aronu i njegovim sinovima. Neka mu ih Izraelci  potpuno daruju. 
\par 10 Arona i njegove sinove postavi da vrše svoju  svećeničku službu. A svjetovnjak koji bi se tome približio neka  se pogubi." 
\par 11 Jahve reče Mojsiju: 
\par 12 "Ja, evo, uzimam Levijevce između  Izraelaca namjesto svih prvorođenaca - onih koji otvaraju materinju  utrobu kod Izraelaca. Moji su, dakle, Levijevci! 
\par 13 Meni, naime, pripada svaki prvorođenac. Onoga dana kad sam pobio sve prvence  u zemlji egipatskoj, sebi sam posvetio sve prvorođence u Izraelu  - i od ljudi i od stoke. Oni su moji. Ja sam Jahve." 
\par 14 Jahve reče Mojsiju u Sinajskoj pustinji: 
\par 15 "Popiši  Levijevce po njihovim porodicama i rodovima; popiši sve muškarce  od jednoga mjeseca i više." 
\par 16 Na zapovijed Jahvinu Mojsije ih popisa, kako mu je bilo  naređeno. 
\par 17 Ovo su poimenice bili sinovi Levijevi: Geršon,  Kehat i Merari. 
\par 18 A ovo su imena Geršonovih sinova po njihovim rodovima:  Libni i Šimi. 
\par 19 A sinovi su Kehatovi po svojim rodovima: Amram, Jishar, Hebron i Uziel. 
\par 20 Sinovi su Merarijevi po svojim rodovima:  Mahli i Muši. To su Levijevi rodovi po svojim porodicama. 
\par 21 Od Geršona lozu vuče rod Libnijev i rod Šimijev. To su  rodovi Geršonovaca. 
\par 22 Njih je u popisu svih muškaraca od jednoga  mjeseca naviše ubilježeno sedam tisuća i pet stotina. 
\par 23 Rodovi  Geršonovaca taborovali su za Prebivalištem prema zapadu. 
\par 24 Glava  porodice Geršonovaca bijaše Elijasaf, sin Laelov. 
\par 25 Geršonovci  su se u Šatoru sastanka brinuli za Prebivalište, za Šator i njegov  krov, za zavjese na ulazu u Šator sastanka; 
\par 26 onda za dvorišne  zavjese, za zavjesu na ulazu u dvorište što je oko Prebivališta  i žrtvenika, za njihova užeta i za sve što spada na tu službu. 
\par 27 Od Kehata potječe rod Amramov, rod Jisharov, rod Hebronov  i rod Uzielov. To su rodovi Kehatovaca. 
\par 28 Kad se popisaše svi  muškarci od jednoga mjeseca naviše, bilo ih je osam tisuća i  šest stotina. Oni su se brinuli za Svetište. 
\par 29 Rodovi Kehatovaca taborovali su s južne strane Prebivališta. 
\par 30 Glava rodova u domu Kehatovu bijaše Elisafan, sin Uzielov. 
\par 31 Oni su se brinuli za Kovčeg, stol, svijećnjak, žrtvenik i  sveti pribor kojim su se služili i, konačno, za zavjesu i za  sve što joj pripada. 
\par 32 Vrhovni poglavar levita bio je Eleazar, sin svećenika  Arona. On je vršio nadzor nad onima koji su se brinuli za Svetište. 
\par 33 Od Merarija potječe rod Mahlijev i rod Mušijev. To su  Merarijevi rodovi. 
\par 34 Njih je u popisu svih muškaraca od jednoga  mjeseca i više ubilježeno šest tisuća i dvije stotine. 
\par 35 Glava  rodova u domu Merarijevu bijaše Suriel, sin Abihajilov. Oni su  taborovali sa sjeverne strane Prebivališta. 
\par 36 Merarijevci su  se brinuli za trenice Prebivališta, za njegove priječnice, za  stupce i njihova podnožja, za sav njegov pribor i za sve što  spada na njegovu službu. 
\par 37 Povrh toga, za stupove uokolo predvorja, njihova podnožja, kočiće i užeta. 
\par 38 Pred Prebivalištem prema istoku, pred Šatorom sastanka  s istočne strane, utaborivali se Mojsije, Aron i njihovi sinovi, kojima je u ime Izraelaca bila povjerena služba u Svetištu.  Svjetovnjak koji bi se približio imao se pogubiti. 
\par 39 Svih popisanih Levijevaca od jednoga mjeseca naviše,  koje je na Jahvinu zapovijed po njihovim rodovima popisao Mojsije  i Aron, bijaše dvadeset i dvije tisuće. 
\par 40 Jahve rekne Mojsiju: "Popiši sve muške prvorođence izraelske  od jednoga mjeseca naviše te načini popis njihovih imena. 
\par 41 I  levite dodijeli meni - ja sam Jahve - namjesto svih prvorođenaca  izraelskih, a stoku levitsku namjesto sve prvenčadi stoke izraelske." 
\par 42 Tako Mojsije popiše sve prvorođence izraelske, kako mu  je Jahve naredio. 
\par 43 Svih muških prvorođenaca od jednoga mjeseca  naviše bijaše u popisu imena dvadeset i dvije tisuće i dvije  stotine sedamdeset i tri. 
\par 44 Tada Jahve reče Mojsiju: 
\par 45 "Uzmi levite namjesto svih  prvorođenaca izraelskih, a stoku levitsku namjesto stoke njihove;  leviti neka budu moji. Ja sam Jahve. 
\par 46 A za otkupninu dvjesta  sedamdeset i triju izraelskih prvorođenaca što ih je više nego  levita, 
\par 47 uzmi pet šekela po glavi, uzmi ih prema hramskom  šekelu: dvadeset gera - jedan šekel. 
\par 48 Onda podaj taj novac  Aronu i njegovim sinovima za otkupninu onih kojih je odviše." 
\par 49 Tako Mojsije primi taj novac kao otkupninu za prvorođence  koji su nadilazili broj onih koje su leviti otkupili. 
\par 50 Od  izraelskih je prvorođenaca primio u srebru tisuću trista šezdeset  i pet šekela hramske mjere. 
\par 51 Po nalogu Jahvinu Mojsije predade  novac te otkupnine Aronu i njegovim sinovima, kako je Jahve Mojsiju  naredio. 


\chapter{4}

\par 1 Jahve reče Mojsiju i Aronu: 
\par 2 "Izdvojite između sinova Levijevih  glavare Kehatovih sinova po rodovima i porodicama njihovim: 
\par 3 od  trideset godina naviše, sve do pedeset godina - sve koji mogu  ući u red da vrše službe u Šatoru sastanka. 
\par 4 A služba je Kehatovih  sinova u Šatoru sastanka: briga za svetinje nad svetinjama. 
\par 5 Kad se tabor diže na put, neka uđu Aron i njegovi sinovi  te skinu zaštitnu zavjesu i njom pokriju Kovčeg svjedočanstva. 
\par 6 Neka onda na nj stave pokrivalo od fine kože, a po njemu neka  razastru platno, potpuno ljubičasto. Potom neka Kovčegu namjeste  motke. 
\par 7 Po stolu prinošenja neka prostru ljubičasto platno. Onda  neka na nj stave zdjele, žlice, krčage i vrčeve za ljevanice.  Kruh neprekidnog prinošenja neka također bude na njemu. 
\par 8 To  neka prekriju tamnocrvenim platnom, a preko njega neka prebace  pokrivalo od fine kože. Potom neka stolu namjeste motke. 
\par 9 Neka zatim uzmu ljubičasto platno i pokriju svijećnjak  za svjetlo i njegove svjetiljke, njegove usekače, njegove lugare  i sve posude za ulje kojima se ono poslužuje. 
\par 10 Neka ga stave  sa svim njegovim priborom na pokrivalo od fine kože pa polože  na nosiljku. 
\par 11 Po zlatnom žrtveniku neka razastru ljubičasto platno  i prekriju ga pokrivalom od fine kože. Potom neka mu namjeste  motke. 
\par 12 Neka sad uzmu sav pribor što se upotrebljava za službu  u Svetištu pa ga stave na ljubičasto platno i onda prekriju pokrivačem  od fine kože. Zatim neka sve to polože na nosiljku. 
\par 13 Neka  pometu pepeo sa žrtvenika i po njemu razastru crveno platno. 
\par 14 Na nj neka postave sav pribor što se upotrebljava za službu:  kadionike, viljuške, lopatice i zdjele - sve posuđe za žrtvenik.  Po njemu onda neka razastru pokrivalo od fine kože. Zatim neka  namjeste motke. 
\par 15 Pošto Aron i njegovi sinovi završe pokrivanje Svetišta  i svega svetog posuđa, u času kad imadne tabor krenuti na put, neka dođu potomci Kehatovi da to ponesu. No svetih se predmeta  ne smiju doticati da ne poginu. To je dužnost Kehatovih potomaka u Šatoru sastanka. 
\par 16 A Eleazar, sin svećenika Arona, neka se brine za ulje  svijećnjaka, za mirisni kad, za trajnu prinosnicu i za ulje pomazanja;  neka se brine za sve Prebivalište, za sve što je u njemu - za  Svetište i njegovo posuđe." 
\par 17 Jahve reče Mojsiju i Aronu: 
\par 18 "Ne dopustite da nestane  pleme rodova Kehatovih između levita. 
\par 19 Ovako postupajte s  njima, da žive i ne izginu primičući se najvećim svetinjama:  neka dođu Aron i njegovi sinovi da postave svakoga od njih na  njegovu službu i uz njegovu dužnost. 
\par 20 Oni neka ne ulaze ni  da začas pogledaju Svetište da ne bi poginuli." 
\par 21 Jahve reče Mojsiju: 
\par 22 "Popiši i Geršonove sinove po  njihovim porodicama i njihovim rodovima, od trideset godina naviše, sve do pedesete godine; 
\par 23 popiši ih sve koji mogu ići u red  da vrše službu u Šatoru sastanka. 
\par 24 A ovo je služba rodova Geršonovaca pri radu i prenošenju: 
\par 25 neka nose zavjese Prebivališta, Šator sastanka s njegovim  krovom, pokrivalo od fine kože što je povrh njega, i zavjesu  na ulazu u Šator sastanka; 
\par 26 onda, dvorišne zavjese, zavjesu  s vrata na ulazu u predvorje što opkoljuje Prebivalište i žrtvenik, konopce i sav pribor za njihovu službu; što god treba oko tih  stvari raditi, neka učine. 
\par 27 Neka Geršonovci obavljaju sve  svoje dužnosti - sve što imaju nositi i sve što imaju raditi  - po nalogu Arona i njegovih sinova. Njihovoj brizi povjerite  sve što treba da nose. 
\par 28 To je služba rodova Geršonovaca u  Šatoru sastanka. Njihova služba neka bude pod vodstvom Itamara, sina svećenika Arona." 
\par 29 "Sinove Merarijeve popiši po rodovima i porodicama njihovim. 
\par 30 Popiši ih od trideset godina naviše, sve do pedeset godina, koji mogu ući u red da vrše službu u Šatoru sastanka. 
\par 31 Za  sve njihove službe u Šatoru sastanka dužnost im je da nose trenice  za Prebivalište, njegove priječnice, njegove stupce i njegova  podnožja; 
\par 32 stupce što okružuju predvorje, njihova podnožja, njihove kočiće, njihove konopce, sa svim priborom za njihovu  službu. Poimenično popišite predmete što su im povjereni da ih  nose. 
\par 33 To je služba rodova Merarijevaca u svemu što imaju činiti  u Šatoru sastanka pod vodstvom Itamara, sina svećenika Arona." 
\par 34 Mojsije, Aron i glavari zajednice popisali su Kehatove  sinove po njihovim rodovima i porodicama - 
\par 35 sve koji mogu  ući u red da vrše službu u Šatoru sastanka, od trideset godina  naviše, sve do pedeset godina. 
\par 36 I popisanih po njihovim rodovima  bijaše dvije tisuće sedam stotina i pedeset. 
\par 37 To je popis  rodova Kehatovaca, svih koji su služili u Šatoru sastanka, a  koje popisa Mojsije i Aron na zapovijed što je Jahve dade Mojsiju. 
\par 38 Popisanih sinova Geršonovih po njihovim rodovima i porodicama, 
\par 39 od trideset godina naviše, sve do pedeset godina, svih koji  mogu ući u red da vrše službu u Šatoru sastanka - 
\par 40 popisanih, dakle, po njihovim rodovima i porodicama bijaše dvije tisuće  šest stotina i pedeset. 
\par 41 To je popis rodova Geršonovaca, svih  koji su služili u Šatoru sastanka, a koje popisa Mojsije i Aron  na Jahvinu zapovijed. 
\par 42 Popis rodova Merarijevih sinova po njihovim rodovima  i porodicama, 
\par 43 od trideset godina naviše, sve do pedeset godina, svih koji mogu ući u red da vrše službu u Šatoru sastanka - 
\par 44 popisanih, dakle, po njihovim rodovima bijaše tri tisuće  dvjesta. 
\par 45 To je popis Merarijevaca što su ga sastavili Mojsije  i Aron na zapovijed koju je Jahve dao Mojsiju. 
\par 46 Svih, dakle, popisanih levita koje su popisali Mojsije, Aron i glavari izraelski po njihovim rodovima i porodicama,  od trideset godina naviše do pedeset godina - 
\par 47 svih koji su  ušli u službu posluživanja i službu prenošenja u Šatoru sastanka  - 
\par 48 bilo je osam tisuća pet stotina i osamdeset. 
\par 49 Na zapovijed  koju je Jahve dao Mojsiju svakoga su unijeli u popis prema onom  u čemu je služio i što je prenosio. Popisali su ih kako je Jahve  zapovjedio Mojsiju. 


\chapter{5}

\par 1 Jahve reče Mojsiju: 
\par 2 "Naredi Izraelcima da iz tabora odstrane  svakoga gubavca, svakoga koji imadne izljev i svakoga koji se  onečisti mrtvim tijelom. 
\par 3 Odstranite i muške i ženske! Izvan  tabora ih istjerajte da ne onečiste svoje tabore u kojima ja  boravim među njima." 
\par 4 Izraelci tako učine: istjeraju ih iz tabora. Kako je Jahve  rekao Mojsiju, tako Izraelci učine. 
\par 5 Jahve reče Mojsiju: 
\par 6 "Kaži Izraelcima: Kad koji čovjek  ili žena počini bilo kakav grijeh na štetu čovjeka ogriješivši  se protiv Jahve, i osjeti se krivim, 
\par 7 neka prizna počinjeni  grijeh, nadoknadi štetu što bolje može te još doda tome petinu  i dadne onome kome je nanio nepravdu. 
\par 8 Ako čovjek ne bi imao bližeg rođaka kome bi se nadoknada  mogla uručiti, dužna nadoknada pripada Jahvi za svećenika, ne  računajući u to pomirbenoga ovna kojim će svećenik izvršiti nad  krivcem obred pomirenja. 
\par 9 I svaka podizanica od svih posvećenih  stvari što ih Izraelci svećeniku donose njemu pripada. 
\par 10 Svakome  idu stvari koje je posvetio; i neka svećeniku bude ono što njemu  tko dadne." 
\par 11 Jahve reče Mojsiju: 
\par 12 "Govori Izraelcima i reci im:  Ako nekome žena pođe stranputicom te mu se iznevjeri 
\par 13 i netko  s njom legne, ali to ostane sakriveno očima njezina muža i žena  ostane neotkrivena iako se oskvrnula te protiv nje ne bude svjedoka  budući da u činu nije bila uhvaćena - 
\par 14 i sad muža obuzme duh  ljubomore i on postane ljubomoran na svoju ženu koja se oskvrnula;  ili ako ga spopadne duh ljubomore te postane ljubomoran na svoju  ženu a da se ona nije oskvrnula - 
\par 15 neka taj muž dovede svoju  ženu svećeniku. Neka za nju donese prinos: desetinu efe ječmenog  brašna. Neka po njemu ne polijeva ulja niti na nj stavlja tamjana, jer to je prinosnica za ljubomoru, spomen-prinosnica da podsjeti  na grijeh. 
\par 16 Neka svećenik povede tu ženu i postavi je pred Jahvu. 
\par 17 Sad neka svećenik uzme posvećene vode u kakvu zemljanu posudu  i, uzevši prašine što je na podu Prebivališta, neka je svećenik  ubaci u vodu. 
\par 18 Pošto je svećenik postavio ženu pred Jahvu, neka joj otkrije glavu a na njezine ruke stavi spomen-prinosnicu, to jest žitnu prinosnicu za ljubomoru, s svećenik neka drži  u ruci vodu gorčine i prokletstva. 
\par 19 Zatim neka svećenik ženu  zakune. Neka joj reče: 'Ako nikad čovjek s tobom nije ležao te  ako nisi išla stranputicom i oskvrnula se dok si bila pod vlašću  svoga muža, budi pošteđena od ove vode gorčine i prokletstva! 
\par 20 Ali ako si išla stranputicom dok si bila pod vlašću svoga  muža te se oskvrnula; ako je koji čovjek osim tvoga muža legao  s tobom ...' 
\par 21 Ovdje neka svećenik zakune ženu ovom kletvom:  neka joj rekne: Jahve te postavio za prokletstvo i kletvu među  tvojim narodom, učinio da ti uvene rodnica i da ti se utroba  nadme! 
\par 22 Neka ova voda prokletstva zađe u tvoju utrobu! Trbuh  ti se od nje naduo, a rodnica uvenula! - A žena neka poprati:  Amen! Amen! 
\par 23 Potom neka ta prokletstva svećenik napiše na list pa  ih ispere u vodu gorčine. 
\par 24 Onda neka ženu napoji vodom gorčine  i prokletstva, da bi se voda gorčine po njoj razišla i napunila  je gorkošću. 
\par 25 Neka svećenik onda uzme iz ženine ruke prinosnicu  za ljubomoru, prinese je pred Jahvom kao žrtvu prikaznicu te  je donese na žrtvenik. 
\par 26 Zagrabivši od prinosnice punu pregršt  kao spomen-žrtvu, neka to sažeže u kad na žrtveniku. Napokon, neka ženu napoji vodom. 
\par 27 Pošto je napoji vodom, bude li oskvrnuta  iznevjerivši se svome mužu, voda prokletstva ući će u nju i napunit  će je gorčinom; njezina će se utroba naduti a rodnica uvenuti  - ta će žena postati prokletstvom u svome narodu. 
\par 28 A ako žena  ne bude oskvrnuta nego nevina, neće joj biti ništa i imat će  djece. 
\par 29 To je obred u slučaju ljubomore, kad žena pođe stranputicom  i oskvrne se dok je pod vlašću svoga muža; 
\par 30 ili kad kojega  čovjeka obuzme duh ljubomore te postane ljubomoran na svoju ženu.  Neka, dakle, postavi svoju ženu pred Jahvu, a svećenik neka nad  njom izvrši sav ovaj obred. 
\par 31 Neka je muž slobodan od krivnje, a žena neka snosi svoju krivnju." 


\chapter{6}

\par 1 Jahve reče Mojsiju: 
\par 2 "Govori Izraelcima i reci im: 'Ako tko, bilo čovjek ili žena, položi nazirejski zavjet te se posveti  Jahvi, 
\par 3 neka se suzdržava od vina i svakoga opojnog pića. Neka  ne pije ni ukiseljena vina niti ukiseljena opojnog pića; a niti  kakva soka od grožđa neka ne pije; neka ne jede grožđa, ni svježa  ni suha. 
\par 4 Sve vrijeme svoga nazireata ne smije jesti ništa  što rađa lozov trs - od zelena grožđa do komine.' 
\par 5 Sve dok traje njegov nazirejski zavjet, neka britva ne  prelazi preko njegove glave; dok se ne navrši vrijeme što ga  je Jahvi zavjetovao, neka bude posvećen i pusti kose da mu slobodno  rastu na glavi. 
\par 6 Za sve vrijeme svoga zavjeta Jahvi neka se  ne primiče nikakvu mrtvacu. 
\par 7 Neka se ne onečišćuje ni zbog  svoga oca, ni zbog svoje majke, svoga brata ili svoje sestre  ako bi umrli, jer na svojoj glavi nosi posvećenje svoga Boga. 
\par 8 Sve vrijeme svoga nazireata on je posvećen Jahvi. 
\par 9 Umre li tko nenadanom smrću pokraj njega, onečistivši  tako njegovu posvećenu glavu, neka na dan svoga očišćenja obrije  svoju glavu - neka je obrije sedmoga dana. 
\par 10 A osmoga dana  neka donese svećeniku, na ulazu u Šator sastanka, dvije grlice  ili dva golubića. 
\par 11 Neka svećenik prinese jedno kao žrtvu okajnicu, a drugo kao žrtvu paljenicu, zatim neka nad njim izvrši obred  pomirenja zbog ljage kojom se okaljao uz mrtvaca. Toga dana neka  posveti svoju glavu; 
\par 12 neka zavjetuje Jahvi dane svoga nazireata;  neka donese jednogodišnjeg janjca kao žrtvu naknadnicu. Prijašnje  vrijeme neka se ne računa, jer je njegov nazireat bio oskvrnjen. 
\par 13 Ovo je obred za nazirejca: na dan kad se navrši vrijeme  njegova nazireata, neka ga dovedu na ulaz Šatora sastanka. 
\par 14 Kao  svoj prinos neka Jahvi donese: jednogodišnjeg janjca bez mane  za žrtvu paljenicu; jednogodišnje žensko janje, bez mane, za  žrtvu okajnicu; jednoga ovna, bez mane, za žrtvu pričesnicu; 
\par 15 nadalje, košaru neukvasanih pogača od najboljeg brašna, u  ulju zamiješenih i neukvasanih kolača, namazanih uljem, s njihovim  prinosnicama i ljevanicama. 
\par 16 Svećenik, pošto to donese pred Jahvu, neka prinese njegovu  okajnicu i paljenicu. 
\par 17 Zatim neka prinese ovna Jahvi kao žrtvu  pričesnicu zajedno s košarom neukvasanih pogača. I njegovu prinosnicu  i njegovu ljevanicu neka prinese svećenik. 
\par 18 Na ulazu u Šator  sastanka neka nazirejac obrije svoju posvećenu glavu i, uzevši  uvojke sa svoje posvećene glave, neka ih stavi na vatru što gori  pred žrtvom pričesnicom. 
\par 19 Zatim neka svećenik uzme kuhano pleće ovna, jednu neukvasanu  pogaču iz košare i jedan neukvasani kolač i stavi to na ruke  nazirejcu pošto ovaj obrije svoje posvećene kose. 
\par 20 Neka to  svećenik prinese kao žrtvu prikaznicu pred Jahvom. To je svetinja  što pripada svećeniku, osim grudi prikaznice i stegna podizanice.  Poslije toga nazirejac može piti vina." 
\par 21 Ovo je obred nazirejca, ne računajući ono što bi još  mogla prinijeti njegova ruka. Ako je povrh svoga nazireata obećao  kakav dar, neka povrh obreda svoga nazireata učini kako je zavjetovao. 
\par 22 Jahve reče Mojsiju: 
\par 23 "Reci Aronu i njegovim sinovima:  'Ovako blagoslivljajte Izraelce govoreći im: 
\par 24 Neka te blagoslovi Jahve i neka te čuva! 
\par 25 Neka te Jahve licem svojim obasja, milostiv ti bude! 
\par 26 Neka pogled svoj Jahve svrati na te i mir ti donese!' Tako neka stavljaju moje ime nad sinove Izraelove, i ja ću  ih blagoslivljati." 
\par 27 


\chapter{7}

\par 1 U onaj dan kad Mojsije završi podizanje Prebivališta i kad  ga pomaza i posveti sa svim njegovim posuđem, a tako i žrtvenik  sa svim njegovim priborom, 
\par 2 pristupe glavari izraelski, starješine  njihovih pradjedovskih domova, to jest knezovi plemenski koji  su vodili popisivanje, 
\par 3 i dovedu svoje prinose pred Jahvu:  šestora teretna kola i dvanaest volova - jedna kola za dvojicu  glavara i vola za svakoga pojedinoga. Dovedu ih pred Prebivalište. 
\par 4 Tada Jahve progovori Mojsiju: 
\par 5 "Primi to od njih za upotrebu  pri službi u Šatoru sastanka; onda to podaj svakome levitu prema  njegovoj službi." 
\par 6 Mojsije uze kola i volove pa ih dade levitima. 
\par 7 Dvoja  kola i četiri vola dade Geršonovcima prema njihovoj službi, 
\par 8 a  četvera kola i osam volova dade Merarijevcima prema njihovoj  službi pod vodstvom Itamara, sina svećenika Arona. 
\par 9 Kehatovcima  nije dao ništa, jer je njihova zadaća bila nositi posvećene predmete  na ramenima. 
\par 10 Tada glavari prinesu prinos za posvetu žrtvenika na dan  njegova pomazanja. Dok su glavari prinosili svoje prinose pred  žrtvenik, 
\par 11 Jahve progovori Mojsiju: "Svakoga dana neka po  jedan glavar donese svoj prinos za posvetu žrtvenika!" 
\par 12 Prvoga dana donese svoj prinos Nahšon, sin Aminadabov, od plemena Judina. 
\par 13 Njegov je prinos bio: jedna srebrna zdjela  teška sto trideset šekela i jedan srebrni kotlić od sedamdeset  šekela, prema hramskom šekelu; jedno i drugo bijaše napunjeno  najboljim brašnom, zamiješenim u ulju, za prinosnicu. 
\par 14 Onda  jedna zlatna posudica od deset šekela puna tamjana; 
\par 15 jedan  junac, jedan ovan, jedno janje od godinu dana za paljenicu; 
\par 16 jedan  jarac za žrtvu okajnicu, 
\par 17 a za žrtvu pričesnicu: dva vola, pet ovnova, pet kozlića i pet jednogodišnjih janjaca. To je  bio prinos Nahšona, Aminadabova sina. 
\par 18 Drugoga dana donese svoj prinos Netanel, sin Suarov,  glavar Jisakarovaca. 
\par 19 Za svoj prinos donio je: jednu srebrnu  zdjelu tešku sto trideset šekela, jedan srebrni kotlić od sedamdeset  šekela, prema hramskom šekelu; oboje puno najboljeg brašna, zamiješena  u ulju, za prinosnicu; 
\par 20 onda jednu zlatnu posudicu od deset  šekela punu tamjana; 
\par 21 jednog junca, jednoga ovna, jedno janje  od godinu dana za paljenicu; 
\par 22 jednog jarca za okajnicu, 
\par 23 a  za pričesnicu: dva vola, pet ovnova, pet kozlića i pet jednogodišnjih  janjaca. To je bio prinos Netanela, Suarova sina. 
\par 24 Trećega dana donese svoj prinos glavar Zebulunovaca,  Eliab, sin Helonov. 
\par 25 Njegov je prinos bio: jedna srebrna zdjela  teška sto trideset šekela i jedan srebrni kotlić od sedamdeset  šekela, prema hramskom šekelu; oboje puno najboljeg brašna, zamiješena  u ulju, za prinosnicu; 
\par 26 jedna zlatna posudica puna tamjana; 
\par 27 jedan junac, jedan ovan, jedno janje od godinu dana za paljenicu; 
\par 28 jedan jarac za okajnicu, 
\par 29 a za pričesnicu: dva vola, pet  ovnova, pet kozlića i pet jednogodišnjih janjaca. To je bio prinos  Eliaba, Helonova sina. 
\par 30 Četvrtog dana donese svoj prinos glavar Rubenovaca, Elisur, sin Šedeurov. 
\par 31 Njegov je prinos bio: jedna srebrna zdjela  teška sto trideset šekela, jedan srebrni kotlić od sedamdeset  šekela, prema hramskom šekelu; oboje puno najboljeg brašna, zamiješena  u ulju, za prinosnicu; 
\par 32 onda jedna zlatna posudica od deset  šekela puna tamjana; 
\par 33 jedan junac, jedan ovan, jedno janje  od godinu dana za paljenicu; 
\par 34 jedan jarac za okajnicu, 
\par 35 a  za pričesnicu: dva vola, pet ovnova, pet kozlića i pet jednogodišnjih  janjaca. To je bio prinos Elisura, Šedeurova sina. 
\par 36 Petoga dana donese svoj prinos glavar Šimunovaca, Šelumiel, sim Surišadajev. 
\par 37 Njegov je prinos bio: jedna srebrna zdjela  teška sto trideset šekela, jedan srebrni kotlić od sedamdeset  šekela, prema hramskom šekelu; oboje napunjeno najboljim brašnom, zamiješenim u ulju, za prinosnicu; 
\par 38 onda jedna zlatna posudica  od deset šekela puna tamjana; 
\par 39 jedan junac, jedan ovan, jedno  janje od godinu dana za paljenicu; 
\par 40 jedan jarac za okajnicu, 
\par 41 a za pričesnicu: dva vola, pet ovnova, pet kozlića i pet  jednogodišnjih janjaca. To je bio prinos Šelumiela, Surišadajeva  sina. 
\par 42 Šestoga dana donese svoj prinos glavar Gadovaca, Elijasaf, sin Deuelov. 
\par 43 Njegov je prinos bio: jedna srebrna zdjela  teška sto trideset šekela, jedan srebrni kotlić od sedamdeset  šekela, prema hramskom šekelu; oboje napunjeno najboljim brašnom, zamiješenim u ulju, za prinosnicu; 
\par 44 onda jedna zlatna posudica  od deset šekela puna tamjana; 
\par 45 jedan junac, jedan ovan, jedno  janje od godinu dana za paljenicu; 
\par 46 jedan jarac za okajnicu, 
\par 47 a za pričesnicu: dva vola, pet ovnova, pet kozlića i pet  jednogodišnjih janjaca. To je bio prinos Elijasafa, Deuelova  sina. 
\par 48 Sedmoga dana donese svoj prinos glavar Efrajimovaca,  Elišama, sin Amihudov. 
\par 49 Njegov je prinos bio: jedna srebrna  zdjela teška sto trideset šekela i jedan srebrni kotlić od sedamdeset  šekela, prema hramskom šekelu; oboje puno najboljeg brašna, zamiješena  u ulju, za prinosnicu; 
\par 50 onda jedna zlatna posudica od deset  šekela puna tamjana; 
\par 51 jedan junac, jedan ovan, jedno janje  od godinu dana za paljenicu, 
\par 52 jedan jarac za okajnicu, 
\par 53 a  za pričesnicu: dva vola, pet ovnova, pet kozlića i pet jednogodišnjih  janjaca. To je bio prinos Elišama, Amihudova sina. 
\par 54 Osmoga dana donese svoj prinos glavar Manašeovaca, Gamliel, sin Pedahsurov. 
\par 55 Njegov je prinos bio: jedna srebrna zdjela  teška sto trideset šekela i jedan srebrni kotlić od sedamdeset  šekela, prema hramskom šekelu; oboje napunjeno najboljim brašnom, zamiješenim u ulju, za prinosnicu; 
\par 56 onda jedna zlatna posudica  od deset šekela puna tamjana; 
\par 57 jedan junac, jedan ovan, jedno  janje od godinu dana za paljenicu; 
\par 58 jedan jarac za okajnicu, 
\par 59 a za pričesnicu: dva vola, pet ovnova, pet kozlića i pet  jednogodišnjih janjaca. To je bio prinos Gamliela, Pedahsurova  sina. 
\par 60 Devetoga dana donese svoj prinos glavar Benjaminovaca, Abidan, sin Gidonijev. 
\par 61 Njegov je prinos bio: jedna srebrna  zdjela teška sto trideset šekela i jedan srebrni kotlić od sedamdeset  šekela, prema hramskom šekelu; oboje napunjeno najboljim brašnom, zamiješenim u ulju, za prinosnicu; 
\par 62 onda jedna zlatna posudica  od deset šekela puna tamjana, 
\par 63 jedan junac, jedan ovan, jedno  janje od godinu dana za paljenicu; 
\par 64 jedan jarac za okajnicu, 
\par 65 a za pričesnicu: dva vola, pet ovnova, pet kozlića i pet  jednogodišnjih janjaca. To je bio prinos Abidana, Gidonijeva  sina. 
\par 66 Desetoga dana donese svoj prinos glavar Danovaca, Ahiezer, sin Amišadajev. 
\par 67 Njegov je prinos bio: jedna srebrna zdjela  teška sto trideset šekela i jedan srebrni kotlić od sedamdeset  šekela, prema hramskom šekelu; oboje napunjeno najboljim brašnom, zamiješenim u ulju, za prikaznicu; 
\par 68 onda jedna zlatna posudica  od deset šekela puna tamjana; 
\par 69 jedan junac, jedan ovan, jedno  janje od godinu dana za paljenicu; 
\par 70 jedan jarac za okajnicu, 
\par 71 a za pričesnicu: dva vola, pet ovnova, pet kozlića i pet  jednogodišnjih janjaca. To je bio prinos Ahiezera, Amišadajeva  sina. 
\par 72 Jedanaestoga dana donese svoj prinos glavar Ašerovaca, Pagiel, sin Okranov. 
\par 73 Njegov je prinos bio: jedna srebrna  zdjela teška sto trideset šekela i jedan srebrni kotlić od sedamdeset  šekela, prema hramskom šekelu; oboje napunjeno najboljim brašnom, zamiješenim u ulju, za prinosnicu; 
\par 74 onda jedna zlatna posudica  od deset šekela puna tamjana; 
\par 75 jedan junac, jedan ovan, jedno  janje od godinu dana za paljenicu; 
\par 76 jedan jarac za okajnicu, 
\par 77 a za pričesnicu: dva vola, pet ovnova, pet kozlića i pet  jednogodišnjih janjaca. To je bio prinos Pagiela, Okranova sina. 
\par 78 Dvanaestoga dana donese svoj prinos glavar Naftalijevaca, Ahira, sin Enanov. 
\par 79 Njegov je prinos bio: jedna srebrna zdjela  teška sto trideset šekela i jedan srebrni kotlić od sedamdeset  šekela, prema hramskom šekelu; oboje napunjeno najboljim brašnom, zamiješenim u ulju, za prinosnicu; 
\par 80 onda jedna zlatna posudica  od deset šekela puna tamjana; 
\par 81 jedan junac, jedan ovan, jedno  janje od godinu dana za paljenicu; 
\par 82 jedan jarac za okajnicu, 
\par 83 a za pričesnicu: dva vola, pet ovnova, pet kozlića i pet  jednogodišnjih janjaca. To je bio prinos Ahire, Enanova sina. 
\par 84 To su bili prinosi glavara izraelskih za posvetu žrtvenika  na dan kad bijaše pomazan: dvanaest srebrnih zdjela, dvanaest  srebrnih kotlića i dvanaest zlatnih posudica. 
\par 85 Svaka srebrna  zdjela težila je sto trideset šekela; svaki kotlić sedamdeset  šekela. Svega srebra u posuđu bilo je dvije tisuće i četiri stotine  hramskih šekela. 
\par 86 Zlatnih posudica punih tamjana bilo je dvanaest, svaka posudica težila je deset hramskih šekela. Sve zlato u  posudicama težilo je sto dvadeset šekela. 
\par 87 Sve stoke za paljenicu: dvanaest junaca, dvanaest ovnova, dvanaest jednogodišnjih janjaca s njihovim prinosima. Za okajnicu  dvanaest jaraca. 
\par 88 Sve stoke za pričesnicu: dvadeset i četiri  vola, šezdeset ovnova, šezdeset kozlića i šezdeset janjaca od  godine dana. To je bio prinos za posvetu žrtvenika pošto bijaše pomazan. 
\par 89 Kad bi Mojsije ulazio u Šator sastanka da razgovara s  Njim, slušao bi glas kako mu govori ozgo s Pomirilišta što je  bilo na Kovčegu svjedočanstva, među dva kerubina. Tada bi mu  govorio. 


\chapter{8}

\par 1 Jahve reče Mojsiju: 
\par 2 "Govori Aronu i reci mu: 'Kad budeš  palio svjetionice, neka sedam svjetionica svijetli na prednjoj  strani svijećnjaka.'" 
\par 3 Aron i učini tako: smjesti svjetionice na prednju stranu  svijećnjaka, kako je Jahve Mojsiju naredio. 
\par 4 Svijećnjak bijaše  skovan od zlata; skovan od svoga podnožja do svoje čaške. Svijećnjak  je bio napravljen prema uzorku što ga je Jahve pokazao Mojsiju. 
\par 5 Jahve reče Mojsiju: 
\par 6 "Uzmi levite između Izraelaca i  očisti ih! 
\par 7 Ovako s njima postupi da ih očistiš: poškropi ih  vodom za okajavanje; a oni neka se obriju po svemu svome tijelu, neka operu svoju odjeću i bit će čisti. 
\par 8 Neka zatim uzmu jednog  junca i prinosnicu od najboljeg brašna, zamiješena u ulju. A  ti uzmi drugog junca za okajnicu. 
\par 9 Dovedi onda levite pred  Šator sastanka i skupi svu izraelsku zajednicu. 
\par 10 Kad dovedeš  levite pred Jahvu, neka Izraelci stave na njih svoje ruke. 
\par 11 Neka  zatim Aron prinese levite, kao prikaznicu pred Jahvom, u ime  Izraelaca. Tako će njihov posao biti da služe Jahvi. 
\par 12 Neka  potom leviti stave svoje ruke juncima na glave; onda jednoga  prinesi kao okajnicu, a drugoga kao paljenicu Jahvi, da se izvrši  obred pomirenja nad levitima. 
\par 13 Stavivši levite pred Arona  i njegove sinove, prikaži ih Jahvi žrtvom prikaznicom. 
\par 14 Odvoji  tako levite između Izraelaca da budu moji. 
\par 15 Poslije toga,  pošto ih očistiš i prineseš žrtvom prikaznicom, neka leviti uđu  u službu Šatora sastanka. 
\par 16 Jer oni su između Izraelaca meni  potpuno darovani; njih sam sebi uzeo namjesto svih koji otvaraju  majčinu utrobu, svih izraelskih prvorođenaca. 
\par 17 Svako, naime, prvorođenče među Izraelcima, kako čedo tako i živinče, moje  je; sebi sam ih posvetio onoga dana kad sam pobio svu prvorođenčad  u zemlji egipatskoj. 
\par 18 Tako sam uzeo levite namjesto svih izraelskih  prvorođenaca. 
\par 19 I predao sam levite između Izraelaca kao dar  Aronu i njegovim sinovima da mjesto Izraelaca obavljaju službu  u Šatoru sastanka; da nad njima obavljaju obred pomirenja, tako  da kakva nedaća ne bi pogodila Izraelce što bi se približili  Svetištu." 
\par 20 Mojsije, Aron i sva izraelska zajednica učine tako s  levitima; kako je Jahve naredio Mojsiju za levite, tako im Izraelci  i učine. 
\par 21 Leviti se očiste i operu svoju odjeću; onda ih Aron  prinese pred Jahvu žrtvom prikaznicom. Aron nad njima obavi obred  pomirenja da ih očisti. 
\par 22 Poslije toga uđu leviti u službu  u Šator sastanka, u nazočnosti Arona i njegovih sinova. Kako  je Jahve naredio Mojsiju za levite, tako su s njima i uradili. 
\par 23 Jahve reče Mojsiju: 
\par 24 "I ovo se tiče levita: od dvadeset  i pet godina naviše neka leviti po redu preuzimaju službu u Šatoru  sastanka. 
\par 25 A kad kome bude pedeset godina, neka istupi iz  službe i neka više ne služi. 
\par 26 Ali može pomagati svojoj braći  u vršenju njihovih dužnosti u Šatoru sastanka, no sam ne mora  vršiti službe. Tako postupi prema levitima za njihove dužnosti!" 


\chapter{9}

\par 1 Prvoga mjeseca druge godine nakon izlaska iz zemlje egipatske  Jahve reče Mojsiju u Sinajskoj pustinji: 
\par 2 "Neka Izraelci slave  Pashu u njezino vrijeme. 
\par 3 Slavite je u njezino vrijeme, u suton, četrnaestoga dana ovoga mjeseca; slavite je prema svim njezinim  propisima i običajima." 
\par 4 Tako Mojsije reče Izraelcima da slave Pashu. 
\par 5 I oni  su je slavili u Sinajskoj pustinji, u suton, prvoga mjeseca,  četrnaestoga dana u mjesecu. Kako je god Jahve Mojsiju naredio, tako su Izraelci i učinili. 
\par 6 A bijaše ljudi onečišćenih mrtvacem; ti nisu mogli slaviti  Pashu onoga dana. Dođu tako pred Mojsija i Arona istoga dana 
\par 7 pa reknu: "Mrtvacem smo se onečistili; ipak, zašto bi nam  bilo uskraćeno prinositi Jahvi žrtvu u njezino vrijeme usred  Izraelovih sinova?" 
\par 8 Mojsije im reče: "Strpite se da čujem  što će Jahve za vas odrediti." 
\par 9 I Jahve reče Mojsiju: 
\par 10 "Ovako kaži Izraelcima: 'Kad  se tko između vas ili vaših potomaka onečisti mrtvacem ili je  na daleku putu, neka ipak slavi Pashu Jahvi. 
\par 11 Neka je slave  u suton četrnaestog dana drugoga mjeseca. Neka je blaguju s neukvasanim  kruhom i gorkim zeljem; 
\par 12 neka ništa od nje ne ostavljaju za  ujutro; neka ni jedne kosti na njoj ne lome. Neka je slave prema  propisima Pashe. 
\par 13 Onaj koji je čist a ne bude na putovanju  pa ipak propusti proslaviti Pashu, neka se iskorijeni iz svoga  naroda. Budući da nije prinio Jahvi žrtve u njezino vrijeme,  takav neka snosi svoju krivnju. 
\par 14 Ako s vama boravi stranac i Pashu prinosi Jahvi, neka  je prinosi prema propisima i običajima njezinim. Neka bude jedan  zakon za vas, bio to stranac ili domorodac.'" 
\par 15 Na dan kad je podignuto Prebivalište oblak prekri Prebivalište, Šator svjedočanstva. Od večeri do jutra stajao je u obliku ognja  nad Prebivalištem. 
\par 16 Tako ga je oblak neprestano zaklanjao, a noću bijaše poput ognja. 
\par 17 Kad bi se god oblak digao sa  Šatora, Izraelci bi poslije toga krenuli. A gdje bi oblak stao, tu bi se i Izraelci utaborili. 
\par 18 Na zapovijed Jahvinu Izraelci  su kretali na put i na Jahvinu se zapovijed utaborivali. Sve  vrijeme što bi oblak stajao nad Prebivalištem oni su taborovali. 
\par 19 Ako bi oblak dugo stajao nad Prebivalištem, Izraelci su slušali  Jahvin nalog i ne bi polazili na put. 
\par 20 Ali ako bi se dogodilo  da oblak ostane nad Prebivalištem malo vremena, oni bi se na  Jahvinu zapovijed utaborili i na Jahvinu zapovijed opet krenuli  na put. 
\par 21 Ako bi se oblak digao pošto se zadržao od večeri  do jutra, oni bi tada ujutro krenuli na put. Danju ili noću,  kad bi se oblak digao, oni bi krenuli na put. 
\par 22 Dva dana ili  mjesec ili godinu - dok bi oblak ostajao nad Prebivalištem -  Izraelci su taborovali, ne krećući na put, a čim bi se digao, oni bi krenuli. 
\par 23 Po zapovijedi Jahvinoj stajahu u taboru  i po zapovijedi Jahvinoj kretahu na put. Držali su se Jahvina  naloga, kako Jahve bijaše zapovjedio Mojsiju. 


\chapter{10}

\par 1 Jahve reče Mojsiju: 
\par 2 "Napravi sebi dvije trube; napravi  ih od kovana srebra. Neka ti služe za sazivanje zajednice i za  pokretanje tabora. 
\par 3 Kad se u njih zatrubi, neka se sva zajednica  skupi k tebi na ulazu u Šator sastanka. 
\par 4 Ako li se zatrubi  u jednu, neka se k tebi skupe glavari izraelski, tisućnici. 
\par 5 Kad  popratite trubljenje bojnim poklikom, neka krenu logori utaboreni  na istočnoj strani. 
\par 6 Kad popratite trubljenje bojnim poklikom  po drugi put, neka krenu logori utaboreni s južne strane: neka  se trubljenje poprati bojnim poklikom da oni krenu. 
\par 7 Trubite  i da skupite zajednicu, ali bez bojnog poklika. 
\par 8 Neka u trube  trube svećenici, sinovi Aronovi. Neka vam to bude trajnom uredbom  za vaše naraštaje. 
\par 9 Kad u svojoj zemlji pođete u rat na neprijatelja koji  vas pritisne, zaorite na trube s bojnim poklikom, i Jahve, Bog  vaš, sjetit će se vas i bit ćete izbavljeni od svojih neprijatelja. 
\par 10 Na dan svoje svečanosti, svojih blagdana ili svojih mjesečevih  mlađaka, dok prinosite svoje paljenice i pričesnice, trubite  u trube. Neka to za vas bude spomen pred Bogom vašim. Ja sam  Jahve, Bog vaš." 
\par 11 Druge godine drugoga mjeseca dvadesetog dana u mjesecu  diže se oblak iznad Prebivališta svjedočanstva. 
\par 12 Tada se Izraelci  zapute iz Sinajske pustinje na svoja putovanja. Oblak se zaustavi  u pustinji Paranu. 
\par 13 Tako na Jahvinu zapovijed danu Mojsiju krenuše prvi put. 
\par 14 Prva je krenula zastava tabora Judinih sinova u svojim četama.  Nad njihovom vojskom bijaše Nahšon, sin Aminadabov; 
\par 15 nad vojskom  plemena Jisakarovaca stajaše Netanel, sin Suarov, 
\par 16 a nad vojskom  plemena Zebulunovaca bijaše Eliab, sin Helonov. 
\par 17 Zatim, pošto je rastavljeno Prebivalište, krenuše Geršonovci  i Merarijevci noseći Prebivalište. 
\par 18 Potom krenu zastava tabora Rubenova u svojim četama.  Nad njihovom vojskom bijaše Elisur, sin Šedeurov; 
\par 19 nad vojskom  plemena Šimunovaca stajao je Šelumiel, sin Surišadajev; 
\par 20 nad  vojskom plemena Gadovaca bio je Elijasaf, sin Deuelov. 
\par 21 Potom krenuše Kehatovci noseći posvećene predmete. Tako  je Prebivalište bilo podignuto prije njihova dolaska. 
\par 22 Onda krenu zastava tabora Efrajimovaca u svojim četama.  Nad njihovom vojskom bijaše Elišama, sin Amihudov, 
\par 23 nad vojskom  plemena Manašeovaca stajaše Gamliel, sin Pedahsurov; 
\par 24 nad  vojskom plemena Benjaminovaca bijaše Abidan, sin Gidonijev. 
\par 25 A kao zalazna straža za sve tabore krenu, u svojim četama, zastava tabora Danovaca. Nad njihovom je vojskom stajao Ahiezer, sin Amišadajev. 
\par 26 Nad vojskom plemena Ašerovaca bio je Pagiel, sin Okranov; 
\par 27 a nad vojskom plemena Naftalijevaca bio je  Ahira, sin Enanov. 
\par 28 Takav je bio red putovanja Izraelaca svrstanih u svoje  čete. Tako su putovali. 
\par 29 Mojsije reče Hobabu, sinu Midjanca Reuela, Mojsijeva  tasta: "Zaputili smo se u kraj o kojemu je Jahve rekao: 'Dat  ću vam ga!' Pođi s nama i dobro ćemo ti činiti, jer je Jahve  obećao sreću Izraelu." 
\par 30 "Ne idem", odgovori mu, "nego se vraćam u svoju zemlju;  k svojima se vraćam." 
\par 31 "Molim te, ne ostavljaj nas!" - reče. "Budući da znaš  gdje nam se treba u pustinji utaboriti, valjat ćeš nam kao oči. 
\par 32 Ako s nama pođeš, dobročinstva koja nam Jahve bude udijelio  s tobom ćemo dijeliti." 
\par 33 Od Jahvina brda putovali su tri dana hoda. Kovčeg Jahvina  saveza išao je pred njima ta tri dana hoda da im potraži mjesto  odmora. 
\par 34 Danju je opet Jahvin oblak bio nad njima, kako bi  se iz tabora zaputili. 
\par 35 Kad bi Kovčeg polazio, Mojsije bi rekao: "Ustani, Jahve! Neprijatelji tvoji neka se rasprše! Koji tebe mrze nek' bježe pred tobom!" 
\par 36 A kad bi se zaustavljao, popratio bi: "Vrati se, o Jahve! Izraelu ti si tisuće bezbrojne!" 


\chapter{11}

\par 1 I stade narod zlobno mrmljati u Jahvine uši. Kad to ču Jahve, planu gnjevom. Jahvin oganj izbi među njima i spali jedan kraj  tabora. 
\par 2 Narod zavapi Mojsiju, a Mojsije se pomoli Jahvi i  oganj se utiša. 
\par 3 Ono se mjesto prozva Tabera, jer je Jahvin  oganj ondje zaplamtio na njih. 
\par 4 Svjetinu koja se oko njih skupila obuzme pohlepa za jelom.  Izraelci se opet upuste u jadikovanje govoreći: "Tko će nas nasititi  mesom? 
\par 5 Sjećamo se kako smo u Egiptu jeli badava ribe, krastavaca, dinje, prÓase, luka i češnjaka. 
\par 6 Sad nam život vene; nema  ničega, osim mÓane, pred našim očima." 
\par 7 MÓana je bila kao zrno korijandera i nalik na bdelij. 
\par 8 Narod išao naokolo, skupljao je, a onda tro kamenom na kamenoj  ploči ili stÓupao u stÓupi. Kuhao ju je u loncu i od nje pravio  kolače. Okus joj bijaše kao okus kolača zgotovljena u ulju. 
\par 9 Kad  bi se noću spuštala rosa po taborištu, s njome bi se spustila  i mÓana. 
\par 10 Mojsije je slušao kako jadikuje narod u svojim obiteljima, svatko na ulazu u svoj šator. Gnjev Jahvin žestoko planu i Mojsije  se ražalosti. 
\par 11 "Zašto zlostavljaš slugu svoga?" - upravi Mojsije riječ  Jahvi. "Zašto nisam stekao milost u tvojim očima kad si na me  uprtio teret svega ovog naroda? 
\par 12 Zar je od mene potekao sav  ovaj narod? Zar sam ga ja rodio, kad veliš: 'Nosi ga u svome  krilu, kao što dojilja nosi dojenče, u zemlju što sam je pod  zakletvom obećao njihovim očevima!' 
\par 13 Odakle meni meso da ga  dam svemu ovom puku koji plače oko mene govoreći: 'Daj nam mesa  da jedemo!' 
\par 14 Ja sam ne mogu nositi sav ovaj narod. Preteško  je to za me. 
\par 15 Ako ćeš ovako sa mnom postupati, radije me ubij, ako sam stekao milost u tvojim očima, da više ne gledam svoga  jada." 
\par 16 Onda Jahve reče Mojsiju: "Skupi mi sedamdeset muževa  između starješina izraelskih za koje znaš da su starješine narodu  i njegovi nadglednici. Dovedi ih u Šator sastanka pa neka ondje  zauzmu svoja mjesta s tobom. 
\par 17 Ja ću sići i ondje s tobom govoriti;  uzet ću nešto duha koji je na tebi i stavit ću ga na njih. Tako  će s tobom nositi teret naroda da ga ne nosiš sam. 
\par 18 Nadalje, kaži narodu: Za sutra se posvetite i jest ćete  mesa, jer ste mrmljali u uši Jahvi govoreći: 'Tko će nas nasititi  mesa? U Egiptu nam je bilo dobro.' Jahve će vam, dakle, dati  mesa da jedete. 
\par 19 Nećete ga jesti samo jedan dan, ni dva dana, ni pet dana, ni deset dana, ni dvadeset dana, 
\par 20 nego cio mjesec, sve dok vam ne izbije na nosnice i ne ogadi vam se, jer ste  odbacili Jahvu koji je među vama mrmljajući pred njim riječima:  'Zašto smo uopće odlazili iz Egipta!'" 
\par 21 "Naroda u kojemu se nalazim", odgovori Mojsije, "ima  šest stotina tisuća pješaka, a ti kažeš: 'Mesa ću im dati da  jedu mjesec dana.' 
\par 22 Može li im se naklati sitne i krupne stoke  da im dostane? Mogu li im se sve ribe iz mora zgrnuti da im bude  dosta?" 
\par 23 Jahve reče Mojsiju: "Zar je ruka Jahvina tako kratka?  Sad ćeš vidjeti hoće li se obistiniti moja riječ ili neće." 
\par 24 Mojsije izađe i kaza narodu Jahvine riječi. Onda skupi  sedamdeset muževa između narodnih starješina i smjesti ih oko  Šatora. 
\par 25 Jahve siđe u oblaku i poče s njim govoriti. Zatim  uze od duha koji bijaše na njemu i stavi na onu sedamdesetoricu  starješina. Kad duh počinu na njima, počeše prorokovati, ali  to više nikad ne učiniše. 
\par 26 Dvojica ostadoše u taboru. Jednome je bilo ime Eldad, a drugome Medad. Duh je i na njima počinuo - bili su i oni među  upisanima, premda nisu došli u tabor - te počeše u taboru prorokovati. 
\par 27 Neki mladić otrča te javi Mojsiju: "Eldad i Medad", reče, "prorokuju u taboru!" 
\par 28 Jošua, sin Nunov, koji je posluživao  Mojsija od svoje mladosti, prozbori i reče: "Mojsije, gospodaru  moj, ušutkaj ih!" 
\par 29 Mojsije mu odgovori: "Zar si zavidan zbog  mene! Oh, kad bi sav narod Jahvin postao prorok! Kad bi Jahve  na njih izlio svoga duha!" 
\par 30 Potom se Mojsije i starješine  izraelske vrate u tabor. 
\par 31 Tada Jahve zapovjedi te zapuhnu vjetar i nanese prepelice  od mora i sasu ih na tabor, na dan hoda i s ove i s one strane  tabora, na dva lakta iznad zemlje. 
\par 32 Narod je ustao te je toga  cijeloga dana, svu noć i cio sutrašnji dan skupljao prepelice.  Onaj tko ih je skupio najmanje imao je deset homera. 
\par 33 Zatim  ih razastriješe oko tabora. Meso još bijaše među njihovim zubima  - još ga nisu prožvakali - kadli planu Jahvin gnjev protiv naroda:  Jahve udari narod strašnim pomorom. 
\par 34 Ono se mjesto prozva Kibrot Hataava, jer su ondje pokopali  one koji se bijahu polakomili. 
\par 35 Iz Kibrot Hataave narod se zaputi u Haserot. I utabori  se u Haserotu. 


\chapter{12}

\par 1 A Mirjam i Aron uzeše rogoboriti protiv Mojsija zbog žene  Kušanke kojom se oženio; jer bijaše uzeo za ženu jednu Kušanku. 
\par 2 "Zar je samo Mojsiju govorio Jahve?" - rekoše mu. "Zar  i nama nije govorio?" Jahve to ču. 
\par 3 Mojsije je bio veoma skroman  čovjek, najskromniji čovjek na zemlji. 
\par 4 I odmah reče Jahve Mojsiju, Aronu i Mirjami: "Vas se troje  pojavite u Šatoru sastanka." Njih se troje pojavi. 
\par 5 U stupu  oblaka siđe Jahve te stade na ulazu u Šator. Zovnu Arona i Mirjamu.  Kad njih dvoje istupi naprijed, 
\par 6 reče Jahve: "Saslušajte riječi moje: Nađe li se među vama prorok, u viđenju njemu ja se javljam, u snu njemu progovaram. 
\par 7 Ali nije tako sa slugom mojim Mojsijem. Od svih u kući mojoj najvjerniji je on. 
\par 8 Iz usta u usta njemu ja govorim, očevidnošću, a ne zagonetkama, i lik Jahvin on smije gledati. Kako se onda niste bojali govoriti protiv sluge moga Mojsija?" 
\par 9 Uskipjevši gnjevom na njih, Jahve ode. 
\par 10 Čim se od šatora  oblak udaljio, gle! Mirjam ogubavi, kao snijegom posuta. Aron  se okrenu prema Mirjami, a to guba na njoj. 
\par 11 Tada rekne Aron Mojsiju: "Gospodaru moj, ne svaljuj na  nas kazne za grijeh koji smo u ludosti počinili i kojega smo  krivci. 
\par 12 Ne daj da ona ostane kao mrtvo dijete kojemu je već  na izlasku iz majčine utrobe meso napol uništeno!" 
\par 13 Tada zavapi Mojsije Jahvi: "Bože, molim te, ozdravi je!" 
\par 14 "Da joj je otac njezin pljunuo u lice", reče Jahve Mojsiju, "zar se ne bi morala stidjeti sedam dana? Neka i ona bude odvojena  izvan tabora sedam dana, pa neka se poslije opet pripusti." 
\par 15 Tako je Mirjam bila odvojena izvan tabora sedam dana.  Narod nije na put polazio dok Mirjam nije opet bila pripuštena. 
\par 16 Poslije toga narod krenu iz Haserota i utabori se u pustinji  Paranu. 


\chapter{13}

\par 1 Jahve reče Mojsiju: 
\par 2 "Pošalji ljude, po jednoga čovjeka  iz pojedinog pradjedovskog plemena, da izvide kanaansku zemlju, koju dajem Izraelcima. Pošaljite sve njihove glavare!" 
\par 3 Na Jahvinu zapovijed Mojsije ih posla iz pustinje Parana.  Svi ti ljudi bijahu glavari Izraelaca. 
\par 4 A ovo su njihova imena: Šamua, sin Zakurov, od plemena Rubenova; 
\par 5 Šafat, sin Horijev, od plemena Šimunova; 
\par 6 Kaleb, sin Jefuneov, od plemena Judina; 
\par 7 Jigal, sin Josipov, od plemena Jisakarova; 
\par 8 Hošea, sin Nunov, od plemena Efrajimova; 
\par 9 Palti, sin Rafuov, od plemena Benjaminova; 
\par 10 Gadiel, sin Sodijev, od plemena Zebulunova; 
\par 11 Gadi, sin  Susijev, od plemena Josipova, od plemena Manašeova; 
\par 12 Amiel, sin Gemalijev, od plemena Danova; 
\par 13 Setur, sin Mikaelov, od  plemena Ašerova; 
\par 14 Nahbi, sin Vofsijev, od plemena Naftalijeva; 
\par 15 Geuel, sin Makijev, od plemena Gadova. 
\par 16 To su imena ljudi koje je Mojsije poslao da izvide zemlju.  A Hošeu, sina Nunova, Mojsije prozva Jošuom. 
\par 17 Posla ih Mojsije da izvide kanaansku zemlju pa im reče:  "Idite gore u Negeb, onda se popnite na brdo. 
\par 18 Razgledajte  zemlju kakva je. Je li narod koji u njoj živi jak ili slab, malobrojan  ili mnogobrojan? 
\par 19 Kakva je zemlja u kojoj živi: dobra ili  rđava? Kakvi su gradovi u kojima borave: otvoreni ili utvrđeni? 
\par 20 Kakvo je tlo: plodno ili mršavo? Ima li po njemu drveća ili  nema? Odvažni budite i ponesite plodova te zemlje." Bilo je upravo vrijeme ranog grožđa. 
\par 21 Odu oni gore da  izvide zemlju od pustinje Sina do Rehoba, koji je na ulazu u  Hamat. 
\par 22 Popnu se u Negeb i dođu do Hebrona, gdje su se nalazili  Ahiman, Šešaj i Talmaj, Anakovi potomci. - Hebron je osnovan  sedam godina prije nego Soan u Egiptu. - 
\par 23 Kada stigoše u Dolinu  Eškol, odrezaše ondje lozu s grozdom i ponesoše ga, udvoje, na  motki; ponesoše i mogranja i smokava. 
\par 24 Ono se mjesto prozva  Dolina Eškol zbog grozda koji su ondje Izraelci odrezali. 
\par 25 Nakon četrdeset dana vrate se iz zemlje koju su izviđali. 
\par 26 Odu k Mojsiju i Aronu i svoj izraelskoj zajednici u Kadeš, u Paranskoj pustinji. Podnesu njima i svoj zajednici izvještaj, a onda im pokažu plodove zemlje. 
\par 27 Izvijeste ga oni: "Išli smo u zemlju u koju si nas poslao.  Zaista njome teče med i mlijeko. Evo njezinih plodova. 
\par 28 Ali  je jak narod koji u onoj zemlji živi, gradovi su utvrđeni i vrlo  veliki. A vidjesmo ondje i potomke Anakove. 
\par 29 Amalečani borave  u negepskom kraju: Hetiti, Jebusejci i Amorejci žive u brdu;  a Kanaanci se nalaze uz more i duž Jordana." 
\par 30 Kaleb ušutka narod oko Mojsija i progovori: "Krenimo  ne oklijevajući i zauzmimo je, jer je možemo nadvladati!" 
\par 31 Ali ljudi što su s njim išli odvratiše: "Ne možemo ići  na onaj narod jer je jači od nas." 
\par 32 I počnu ozloglašivati Izraelcima zemlju koju su izviđali:  "Zemlja kroz koju smo prošli da je izvidimo zemlja je što proždire  svoje stanovništvo. Sav narod što ga u njoj vidjesmo ljudi su  krupna stasa. 
\par 33 Vidjesmo ondje i divove - Anakovo potomstvo  od divova. Činilo nam se da smo prema njima kao skakavci. Takvi  bijasmo i njima." 


\chapter{14}

\par 1 Tada zagraja sva zajednica i poče vikati. I te noći narod  plakaše. 
\par 2 Svi su Izraelci mrmljali protiv Mojsija i Arona.  Sva im je zajednica govorila: "Kamo sreće da smo pomrli u zemlji  egipatskoj! Ili da smo pomrli u ovoj pustinji! 
\par 3 Zašto nas Jahve  vodi u tu zemlju da padnemo od mača a žene naše i djeca da postanu  roblje! Zar nam ne bi bilo bolje da se vratimo u Egipat!" 
\par 4 Jedan  je drugome govorio: "Postavimo sebi vođu i vratimo se u Egipat!" 
\par 5 Mojsije i Aron padoše ničice pred svom okupljenom izraelskom  zajednicom. 
\par 6 A Jošua, sin Nunov, i Kaleb, sin Jefuneov, koji  bijahu među onima što su izviđali zemlju, razderaše svoju odjeću. 
\par 7 Zatim rekoše svoj zajednici izraelskoj: "Zemlja kroz koju  smo prošli da je istražimo izvanredno je dobra. 
\par 8 Ako nam Jahve  bude dobrostiv, u tu će nas zemlju dovesti i dat će nam je. To  je zemlja u kojoj teče med i mlijeko. 
\par 9 Samo, nemojte se buniti  protiv Jahve! Ne bojte se naroda one zemlje: tÓa on je zalogaj  za nas. Oni su bez zaštite, a s nama je Jahve! Ne bojte ih se!" 
\par 10 I dok je sva zajednica već mislila da ih kamenuje, pokaza  se Slava Jahvina u Šatoru sastanka svima sinovima Izraelovim. 
\par 11 Tada reče Mojsiju: "Dokle će me taj narod prezirati? Dokle  mi neće vjerovati unatoč svim znamenjima što sam ih među njima  izvodio? 
\par 12 Udarit ću ih pomorom i istrijebiti, a od tebe ću  učiniti narod veći i moćniji od njega." 
\par 13 Onda Mojsije reče Jahvi: "Egipćani su shvatili da si  ti, svojom moći, izveo ovaj narod između njih. 
\par 14 Oni su to  kazali žiteljima one zemlje. Već su saznali da si ti, Jahve,  usred ovog naroda, kojemu se očituješ licem u lice, i da ti,  Jahve, u oblaku stojiš nad njima; da obdan u stupu od oblaka, a obnoć u stupu od ognja ideš pred njima. 
\par 15 Zato, ako pobiješ  ovaj narod kao jednoga čovjeka, narodi koji su čuli glas o tebi  reći će: 
\par 16 'Jahve je bio nemoćan da dovede ovaj narod u zemlju  koju mu je pod zakletvom obećao, i zato ih je poubijao u pustinji.' 
\par 17 Zato neka se snaga moga Gospodina uzvisi, kako si najavio  rekavši: 
\par 18 'Jahve je spor na srdžbu, a bogat milosrđem; podnosi  opačinu i prijestup, ali krivca ne ostavlja nekažnjena, nego  opačinu otaca kažnjava na djeci do trećega i četvrtog koljena.' 
\par 19 Oprosti krivnju ovome narodu po veličini svoga milosrđa,  kao što si vodio ovaj narod od Egipta dovde." 
\par 20 "Opraštam po riječi tvojoj", reče Jahve. 
\par 21 "Ali ipak, tako ja živ bio i slave se Jahvine napunila sva zemlja, 
\par 22 ni  jedan od ljudi koji su vidjeli slavu moju i znamenja što sam  ih izveo u Egiptu i u pustinji, pa me ipak iskušavali već deset  puta ne hoteći poslušati moj glas, 
\par 23 neće vidjeti zemlje što  sam je pod zakletvom obećao njihovim ocima; nitko od onih koji  me preziru neće je vidjeti. 
\par 24 A slugu svoga Kaleba, jer je  u njemu drukčiji duh i jer mi bijaše poslušan, njega ću ja dovesti  u zemlju u koju je išao i njegovi će je potomci zaposjesti! Neka  Amalečani i Kanaanci samo ostanu u dolini. 
\par 25 Sutra se vratite  i krenite u pustinju put Crvenog mora." 
\par 26 Još reče Jahve Mojsiju i Aronu: 
\par 27 "Dokle će ta opaka  zajednica mrmljati protiv mene? Čuo sam tužbe što ih Izraelci  na me dižu. 
\par 28 Kaži im: Tako ja živ bio, objavljuje Jahve, kako  ste na moje uši govorili, tako ću vam i učiniti. 
\par 29 U ovoj pustinji  popadat će vaša mrtva tijela: svih vas koji ste ubilježeni u  bilo koji vaš popis od dvadeset godina pa naprijed, koji ste  rogoborili protiv mene. 
\par 30 Nećete ući u zemlju na koju sam svoju  ruku digao da vas u njoj nastanim, osim Kaleba, sina Jefuneova, i Jošue, sina Nunova. 
\par 31 A vašu djecu, o kojoj kažete da bi  postala roblje, njih ću uvesti da nastane zemlju što ste je vi  prezreli. 
\par 32 A vi? Neka vam tjelesa popadaju u ovoj pustinji! 
\par 33 Vaši sinovi neka lutaju pustinjom četrdeset godina, neka  trpe zbog vaše nevjere dok vam ne ispropadaju tjelesa u ovoj  pustinji. 
\par 34 Prema broju dana u koje ste istraživali zemlju  - dana četrdeset, za svaki dan jednu godinu - ispaštajte svoje  opačine četrdeset godina. Iskusite što znači mene napustiti. 
\par 35 Ja, Jahve, to kažem: tako ću postupiti s ovom opakom zajednicom  što se sjatila protiv mene. U ovoj istoj pustinji neka završi!  Tu neka izgine." 
\par 36 A oni ljudi koje Mojsije bijaše poslao da istraže zemlju  i koji su nakon povratka potakli svu zajednicu da rogobori protiv  njega ozloglašujući zemlju; 
\par 37 oni, dakle, ljudi koji su zlobno  ozloglasili zemlju bijahu pomoreni pred Jahvom. 
\par 38 Od onih ljudi  koji su išli da istraže zemlju ostadoše na životu jedino Jošua, sin Nunov, i Kaleb, sin Jefuneov. 
\par 39 Kad je Mojsije prenio te riječi svim Izraelcima, narod  se uvelike ražalosti. 
\par 40 I uranivši ujutro počnu se uspinjati  na vrh brda govoreći. "Evo uzlazimo na mjesto o kojem je govorio  Jahve jer smo zgriješili." 
\par 41 A Mojsije rekne: "Zašto kršite  zapovijed Jahvinu? Nećete uspjeti. 
\par 42 Ne penjite se, da vas  ne potuku vaši neprijatelji, jer Jahve nije među vama. 
\par 43 Ta  ondje se pred vama nalaze Amalečani i Kanaanci te ćete od mača  pasti jer ste se odvratili od Jahve i jer Jahve neće biti s vama." 
\par 44 Ali se oni prkosno penjahu prema vrhu brda, iako se ni Kovčeg  saveza Jahvina ni Mojsije nisu micali iz tabora. 
\par 45 Amalečani  i Kanaanci koji su živjeli na onome brdu spuste se, udare po  njima i rasprše ih sve do Horme. 


\chapter{15}

\par 1 Jahve reče Mojsiju: 
\par 2 "Govori Izraelcima i reci im: 'Kad  uđete u zemlju gdje ćete boraviti i koju vam ja dajem, 
\par 3 pa  budete prinosili Jahvi paljenu žrtvu, paljenicu ili klanicu,  zavjetnicu ili dragovoljnu žrtvu, ili žrtvu prigodom svojih svetkovina  - praveći tako od krupne ili sitne stoke ugodan miris Jahvi - 
\par 4 neka prinositelj prinese svoj dar Jahvi: prinosnicu od desetine  efe najboljeg brašna, zamiješena u četvrtini hina ulja. 
\par 5 Uz  paljenicu ili uz klanicu prinesi čevrtinu hina vina za ljevanicu  na svako janje. 
\par 6 Povrh ovna prinesi kao prinosnicu dvije desetine  efe najboljeg brašna, zamiješena u jednoj trećini hina ulja; 
\par 7 i vina za ljevanicu prinesi trećinu hina na ugodan miris Jahvi. 
\par 8 Ako Jahvi prinosiš junca kao paljenicu ili kao klanicu  da izvršiš zavjet ili kao pričesnicu, 
\par 9 neka se onda uz junca  prinesu tri desetine efe najboljeg brašna, zamiješena u pola  hina ulja, 
\par 10 a za ljevanicu prinesi pola hina vina kao paljenu  žrtvu na ugodan miris Jahvi. 
\par 11 Neka se tako postupi uza svakoga vola i uza svakoga ovna, uza svaku glavu sitne stoke, ovce ili koze: 
\par 12 koliko ih god  prinesete, za svako pojedino tako učinite, već prema njihovu  broju. 
\par 13 Svaki domorodac neka postupa ovako kad prinosi žrtvu  paljenu na ugodan miris Jahvi. 
\par 14 I ako koji stranac koji živi među vama, ili će biti među  vašim potomcima, htjedne prinijeti žrtvu paljenu na ugodan miris  Jahvi, neka radi kako i vi radite. 
\par 15 Neka je jedan zakon i  za vas i za stranca koji s vama boravi. To je trajan zakon za  vaše naraštaje: pred Jahvom, kako je s vama, tako neka bude i  sa strancem. 
\par 16 Jedan zakon i jedno pravo neka vrijedi za vas  i za stranca koji s vama boravi.'" 
\par 17 Jahve reče Mojsiju: 
\par 18 "Govori Izraelcima i reci im:  'Kad dođete u zemlju u koju vas vodim 
\par 19 i budete jeli kruh  te zemlje, prinesite podizanicu Jahvi. 
\par 20 Kao prvinu iz svojih  naćava prinesite jedan kolač kao podizanicu; prinesite ga kao  i podizanicu s gumna. 
\par 21 Od prvine svojih naćava davajte Jahvi  podizanicu od naraštaja do naraštaja.'" 
\par 22 "Ako nehotice pogriješite te ne budete obdržavali koju  od zapovijedi što ih je Jahve objavio po Mojsiju - 
\par 23 sve što  vam je Jahve zapovjedio po Mojsiju, odonda kad vam je izdao zapovijedi  pa dalje od koljena do koljena - 
\par 24 onda: ako je to počinjeno  nepažnjom zajednice, neka sva zajednica prinese jednoga junca  kao paljenicu na ugodan miris Jahvi s propisanom prikaznicom  i ljevanicom i jednoga jarca kao okajnicu. 
\par 25 Neka svećenik  obavi obred pomirenja nad svom izraelskom zajednicom, pa će im  biti oprošteno. Bila je samo nepažnja, a oni su prinijeli svoj  dar - paljenu žrtvu Jahvi - i okajnicu pred Jahvom za svoju nepažnju. 
\par 26 Bit će oprošteno svoj izraelskoj zajednici, a tako i strancu  koji među njima boravi, jer se sav narod iz nepažnje ogriješio. 
\par 27 Pogriješi li iz nepažnje pojedinac, neka prinese jedno  žensko kozle od godine dana kao okajnicu. 
\par 28 Neka svećenik obavi  obred pomirenja pred Jahvom nad osobom koja je nehotice pogriješila  od nepažnje. Kad nad njom obavi obred pomirenja, bit će joj oprošteno. 
\par 29 Kada tko pogriješi nepažnjom, neka vam jedan zakon vrijedi  i za domoroca i za stranca koji boravi među vama. 
\par 30 Ali onaj koji nešto učini naumice, bio on domorodac ili  stranac, taj na Jahvu huli. Takav neka se istrijebi između svoga  naroda 
\par 31 jer je prezreo Jahvinu riječ i prekršio njegovu zapovijed.  Neka se takav iskorijeni. Neka njegova krivnja padne na nj!" 
\par 32 Kad su Izraelci bili u pustinji, nađu čovjeka kako kupi  drva u subotnji dan. 
\par 33 I oni koji su ga našli da kupi drva  dovedu ga Mojsiju i Aronu i svoj zajednici. 
\par 34 Stave ga pod  stražu, jer još nije bilo određeno što treba s njim učiniti. 
\par 35 "Toga čovjeka treba pogubiti!" - reče Jahve Mojsiju.  "Neka ga kamenjem zaspe izvan tabora sva zajednica." 
\par 36 Sva ga zajednica izvede izvan tabora i zasu ga kamenjem  te on poginu, kako je Jahve zapovjedio Mojsiju. 
\par 37 Reče Jahve Mojsiju: 
\par 38 "Govori Izraelcima i reci im:  neka od naraštaja do naraštaja prave rese na skutovima svojih  haljina, a za resu svakoga skuta neka privezuju ljubičastu vrpcu. 
\par 39 Imat ćete rese zato da vas pogled na njih sjeća svih Jahvinih  zapovijedi. Vršite ih, a ne zanosite se svojim srcem i svojim  očima, što vas tako lako zavode na bludnost. 
\par 40 Tako ćete se  sjećati svih mojih zapovijedi, vršit ćete ih i bit ćete posvećeni  svome Bogu. 
\par 41 Ja sam Jahve, Bog vaš, koji sam vas izveo iz  zemlje egipatske da vam budem Bogom. Ja, Jahve, Bog vaš." 


\chapter{16}

\par 1 Korah, sin Jisharov, sin Kehatov, sin Levijev, pa Datan i  Abiram, sinovi Eliabovi, i On, sin Peletov - potomci Rubenovi  - 
\par 2 ustanu protiv Mojsija zajedno sa dvjesta pedeset Izraelaca, glavara zajednice, uglednih na skupštini i ljudi na glasu. 
\par 3 Oni  se sjate oko Mojsija i Arona govoreći im: "Vi prelazite mjeru!  Sva je zajednica, svi njezini članovi, posvećena i među njima  je Jahve. Zašto se onda uzvisujete iznad zajednice Jahvine!" 
\par 4 Kad to ču Mojsije, pade ničice. 
\par 5 Zatim reče Korahu i  svoj njegovoj družini: "Sutra će Jahve pokazati tko je njegov  i tko je posvećen, kome dopušta da mu se približi. Koga sebi  izabere, k sebi će ga i pustiti. 
\par 6 Učinite ovo: uzmite kadionike, Korah i sva njegova družina; 
\par 7 sutra stavite u njih vatre i  metnite odozgo tamjana pred Jahvom. Koga Jahve odabere, taj neka  bude posvećen. Vi prelazite mjeru, Levijevci!" 
\par 8 Potom Mojsije reče Korahu: "Poslušajte, Levijevci! 
\par 9 Zar  vam je malo što vas je Bog Izraelov izdvojio iz Izraelove zajednice  da vas približi k sebi te da vršite službu u Jahvinu prebivalištu  i da stojite pred zajednicom služeći joj? 
\par 10 Promaknuo je tebe  i s tobom svu tvoju braću Levijevce, a vi još tražite i svećeništvo! 
\par 11 Ti i sva tvoja družina, dakle, sjatili ste se protiv Jahve;  jer što je Aron da protiv njega rogoborite?" 
\par 12 Zatim posla Mojsije po Datana i Abirama, sinove Eliabove, ali oni odgovore: "Nećemo doći! 
\par 13 Zar je malo što si nas odveo  iz zemlje kojom teče med i mlijeko da nas pobiješ u ovoj pustinji, pa hoćeš da nasilno zagospodariš nad nama? 
\par 14 Nisi nas uveo  u zemlju kojom teče med i mlijeko i nisi nam dao u posjed njive  i vinograde! Misliš li iskopati oči ovim ljudima? Nećemo doći!" 
\par 15 Mojsije se vrlo razljuti i reče Jahvi: "Ne obaziri se na  njihovu prinosnicu! Ni jednoga njihova magarca nisam prisvojio  niti sam ijednoga od njih oštetio." 
\par 16 Zatim Mojsije reče Korahu: "Ti i sva tvoja družina stupite  sutra pred Jahvu; ti, i oni, i Aron. 
\par 17 Neka svaki uzme svoj  kadionik, stavi u nj tamjana i neka svaki donese svoj kadionik  pred Jahvu - dvjesta i pedeset kadionika. A i ti i Aron donesite  svaki svoj kadionik." 
\par 18 Svaki uzme svoj kadionik, stavi u nj vatre, onda odozgo  metne tamjana i stane s Mojsijem i Aronom kod ulaza u Šator sastanka. 
\par 19 Kad, naprama njima, sabra Korah svu zajednicu na ulazu u  Šator sastanka, onda se svoj zajednici pokaza slava Jahvina. 
\par 20 I reče Jahve Mojsiju i Aronu: 
\par 21 "Odvojite se od te zajednice  da je odmah satrem!" 
\par 22 Oni popadoše ničice i povikaše: "Bože! Bože životnog  duha u svakome tijelu! Zar ćeš se razgnjeviti na svu zajednicu  kad je samo jedan sagriješio!" 
\par 23 Onda Jahve reče Mojsiju: 
\par 24 "Reci toj zajednici: 'Uklonite  se iz okolice prebivališta Koraha, Datana i Abirama!'" 
\par 25 Mojsije ustade i pođe k Datanu i Abiramu. Za njim krenuše  izraelske starješine. 
\par 26 Zatim ovako progovori zajednici: "Odstupite  od šatora tih opakih ljudi! Ne dotičite se ničega što je njihovo, da ne budete uništeni zbog svih njihovih grijeha." 
\par 27 Tako se oni udalje iz okolice prebivališta Korahova,  Datanova i Abiramova. Uto izađu Datan i Abiram te stanu na ulazu  svojih šatora sa svojim ženama, svojim sinovima i svojom nejačadi. 
\par 28 "Po ovom ćete vidjeti", reče Mojsije, "da me Jahve poslao  da vršim sva ova djela, a da ih ne činim sam od sebe: 
\par 29 ako  ovi ljudi umru kao što umru i svi ljudi; ako ih pohodi sudbina  kakva pohodi sve ljude, onda me Jahve nije poslao. 
\par 30 Ali ako  Jahve učini nečuveno: ako zemlja rastvori svoje ralje i proguta  ih sa svim što je njihovo te živi siđu u Šeol, onda znajte da  su ovi ljudi prezreli Jahvu." 
\par 31 A kad on završi sve te riječi, tlo se pod njima raspukne; 
\par 32 zemlja rastvori svoje ralje i proguta ih s njihovim domovima, sa svim Korahovim ljudima i svim njihovim imanjem. 
\par 33 Živi  siđu u Šeol, oni i sve njihovo. Onda se nad njima zemlja zatvori  i oni iščeznu iz zbora. 
\par 34 Na njihov vrisak svi Izraelci što  su stajali oko njih pobjegoše govoreći: "Da i nas zemlja ne proguta!" 
\par 35 Ali sukne oganj od Jahve te proždre dvjesta i pedeset  ljudi koji su prinosili tamjan. 
\par 36 (17:1) Jahve reče Mojsiju: 
\par 37 (17:2) "Kaži Eleazaru, sinu svećenika Arona, da ukloni kadionike - jer su posvećeni - iz toga zgarišta, a  neposvećenu vatru iz njih neka razaspe podalje. 
\par 38 (17:3) Kadionici  onih koji su sagriješili i grijehom život pokopali neka se prekuju  u pločice za oblaganje žrtvenika. Doneseni su, naime, pred Jahvu, pa su posvećeni. Neka budu opomenom Izraelcima!" 
\par 39 (17:4) Tako svećenik Eleazar uze kadionike od tuča što su ih  prinosili oni koji izgorješe; prekovaše ih u pločice za oblaganje  žrtvenika. 
\par 40 (17:5) One su opomena Izraelcima da se nitko nepozvan  - nitko tko nije od Aronova potomstva - ne smije približiti da  pali tamjan pred Jahvom, kako mu se ne bi dogodilo kao Korahu  i njegovoj družini, prema onom što je kazao Jahve po Mojsiju. 
\par 41 (17:6) Sutradan je sva zajednica rogoborila protiv Mojsija i  Arona. "Pobili ste Jahvin narod!" - govorili su. 
\par 42 (17:7) Dok se zajednica  skupljala protiv Mojsija i Arona, oni se okrenuše prema Šatoru  sastanka, i gle! oblak ga prekri i slava se Jahvina pokaza. 
\par 43 (17:8) Tada Mojsije i Aron odoše pred Šator sastanka. 
\par 44 (17:9) I Jahve  reče Mojsiju: 
\par 45 (17:10) "Udaljite se od te zajednice; u tili ću je  čas uništiti!" Oni padoše ničice. 
\par 46 (17:11) Zatim Mojsije reče Aronu: "Uzmi kadionik, stavi u nj vatre sa žrtvenika, metni tamjana, a onda se žuri  do zajednice da obaviš nad njom obred pomirenja. Gnjev je Jahvin  već izbio i zlo je počelo!" 
\par 47 (17:12) Aron uze što mu je Mojsije rekao te otrča usred zbora, a kad tamo: pomor među narodom već počeo. Stavi tamjana te obavi  obred pomirenja nad narodom. 
\par 48 (17:13) Zatim stade između mrtvih i  živih i zlo se ustavi. 
\par 49 (17:14) Bilo ih je mrtvih od toga zla četrnaest  tisuća i sedam stotina, osim onih koji su poginuli zbog Koraha. 
\par 50 (17:15) Aron se vrati k Mojsiju na ulaz u Šator sastanka: pomor se  ustavi. 


\chapter{17}

\par 1 (17:16) Jahve reče Mojsiju: 
\par 2 (17:17) "Razloži Izraelcima te od njih  uzmi po jedan štap za svaki predjedovski dom; uzmi od svih njihovih  starješina za njihove pradjedovske domove dvanaest štapova. Ime  svakoga napiši na njegovu štapu. 
\par 3 (17:18) A kako ima po jedan štap  za svakoga starješinu pradjedovskih domova, Aronovo ime napiši  na Levijevu štapu. 
\par 4 (17:19) Onda ih pohrani u Šator sastanka pred  Svjedočanstvo; ondje gdje se s tobom sastajem. 
\par 5 (17:20) Štap onoga  čovjeka koga izaberem propupat će. Tako ću maknuti od sebe rogoborenje  Izraelaca kojim prigovaraju vama." 
\par 6 (17:21) Mojsije tako kaza Izraelcima. Sve njihove starješine  dadoše mu štap, po jedan štap za svakoga starješinu - dakle,  dvanaest štapova za njihove pradjedovske domove. Među njihovim  štapovima bio je i štap Aronov. 
\par 7 (17:22) Mojsije pohrani štapove pred  Jahvu u Šatoru svjedočanstva. 
\par 8 (17:23) Kad sutradan Mojsije uđe u  Šator svjedočanstva, gle: štap Arona iz doma Levijeva propupao!  Potjerala mladica, procvjetao cvijet i sazreli bademi. 
\par 9 (17:24) Tada iznese Mojsije ispred Jahve sve štapove sinovima  Izraelovim. Oni ih razgledaše, a onda svatko uze svoj štap. 
\par 10 (17:25) Jahve reče Mojsiju: "Opet stavi Aronov štap pred Svjedočanstvo, neka se čuva za znak buntovnim sinovima. Dokončaj tako njihovo  rogoborenje protiv mene da ne izginu." 
\par 11 (17:26) I učini Mojsije: kako mu je Jahve zapovjedio, tako učini. 
\par 12 (17:27) "Izgibosmo!" - rekoše Izraelci Mojsiju. "Propadosmo!  Svi odreda propadosmo! 
\par 13 (17:28) Tko god priđe Jahvinu prebivalištu, umire ... Zar ćemo svi izginuti?" 


\chapter{18}

\par 1 Tada Jahve reče Aronu: "Ti, tvoji sinovi i tvoj pradjedovski  dom s tobom bit ćete odgovorni za grijehe u Svetištu; ti i tvoji  sinovi s tobom bit ćete odgovorni za grijehe svoga svećeništva. 
\par 2 Pridruži k sebi i svoju braću od Levijeva plemena - tvoga  pradjedovskog doma - neka ti se priključe da ti poslužuju, tebi  i tvojim sinovima s tobom, pred Šatorom svjedočanstva. 
\par 3 Neka  stoje na službu tebi i svemu Šatoru, ali neka se ne približuju  pokućstvu u Svetištu niti žrtveniku, da ne poginu i oni i vi. 
\par 4 Neka su, dakle, tebi pridruženi i neka preuzmu brigu za Šator  sastanka, svaku službu oko Šatora. I neka se ni jedan svjetovnjak  ne približuje vama, 
\par 5 a vi vršite službu u Svetištu i službu  oko žrtvenika da se više ne izlijeva gnjev na Izraelce. 
\par 6 Uzeh, evo, vašu braću levite između Izraelaca vama za dar; kao darovani  pripadaju Jahvi da obavljaju službu oko Šatora sastanka. 
\par 7 Ti  i tvoji sinovi s tobom preuzmite svećeničke poslove oko svega  što spada na žrtvenik i iza zavjese. Službu koju dajem na dar  vašem svećeništvu vi obavljajte. A svjetovnjak koji se primakne  neka se pogubi." 
\par 8 Još reče Jahve Aronu: "Tebi, evo, povjeravam brigu o onom  što se meni prinosi. Sve što Izraelci posvećuju dodjeljujem tebi  i tvojim sinovima kao baštinu trajnim zakonom. 
\par 9 Ovo neka pripadne  tebi od svetinja nad svetinjama: od paljenih žrtava svi njihovi  darovi, za sve njihove prinosnice, za sve njihove okajnice i  za sve njihove naknadnice što ih budu meni uzvraćali; ta vrlo  sveta stvar neka pripadne tebi i tvojim sinovima! 
\par 10 Blagujte  ih kao najveće svetinje! Svaki muškarac može ih jesti. Neka ti  budu svete! 
\par 11 I ovo neka bude za te: ono što se uzima od izraelskih  prinosa da se prinese kao prikaznica - trajnim zakonom predajem  tebi, tvojim sinovima i tvojim kćerima s tobom. Svatko tko u  tvome domu bude čist može od toga jesti. 
\par 12 Najbolje od novoga ulja i najbolje od novoga vina i žita  - prvine koje se prinose Jahvi - predajem tebi. 
\par 13 Prvi rodovi  svega u njihovoj zemlji što ih budu donosili Jahvi neka budu  tvoji. Tko je god čist u tvome domu može ih jesti. 
\par 14 Sve što  u Izraelu bude određeno za 'herem' neka je tvoje. 
\par 15 Svako prvorođenče  svih bića - kako ljudi tako i životinja - što se prinose Jahvi  neka bude tvoje. Samo pusti da se otkupi prvenac od ljudi i prvenče  od nečiste stoke. 
\par 16 Kad budu stari mjesec dana, pusti da ih  otkupljuju. A njihovu otkupnu cijenu odredi: pet srebrnih šekela, prema hramskom šekelu, a to je dvadeset gera. 
\par 17 Ali prvenče  kravlje, prvenče ovčje i prvenče kozje neka se ne otkupljuje.  Oni su svetinja. Krv njihovu izlij na žrtvenik, a pretilinu njihovu  sažeži u kad kao žrtvu spaljenu na ugodan miris Jahvi. 
\par 18 Njihovo  meso neka pripadne tebi; kao i grudi žrtve prikaznice i desno  pleće. 
\par 19 Sve posvećene prinose što ih Izraelci podižu Jahvi predajem  trajnim zakonom tebi, tvojim sinovima i tvojim kćerima s tobom.  To je savez osoljen, trajan pred Jahvom, tebi i tvome potomstvu  s tobom." 
\par 20 "Nemoj imati baštine u zemlji njihovoj", reče Jahve Aronu, "niti sebi stječi posjeda među njima! Ja sam tvoj dio i tvoja  baština među Izraelcima." 
\par 21 "Levijevim sinovima, evo, predajem u baštinu sve desetine  u Izraelu za njihovu službu - za službu što je obavljaju u Šatoru  sastanka. 
\par 22 A Izraelci neka se više ne primiču Šatoru sastanka, da ne navuku na se grijeh i ne poginu. 
\par 23 Neka samo leviti  obavljaju službu u Šatoru sastanka; i neka oni budu odgovorni  za svoj grijeh. Trajna je to odredba za vaše naraštaje; među  Izraelcima neka nemaju posjeda, 
\par 24 jer ja im predajem u posjed  desetine što ih Izraelci prinose na dar Jahvi. Stoga sam za njih  rekao: neka oni nemaju posjeda među Izraelcima." 
\par 25 Jahve reče Mojsiju: 
\par 26 "Levitima govori i reci im: 'Kad  od Izraelaca primate desetinu, koju ja od njih dajem vama u baštinu, od toga onda vi prinesite podizanicu Jahvi: desetinu od desetine. 
\par 27 Prinos će vam biti zaračunan kao da je prinos s gumna i Óotoka  iz badnja. 
\par 28 Tako isto prinosite podizanicu Jahvi i od svih  svojih desetina što ih primate od Izraelaca. Od toga davajte  podizanicu Jahvinu svećeniku Aronu. 
\par 29 Od svih darova koje budete  primali podižite podizanicu Jahvi; od svega ono najbolje - onaj  dio koji treba posvećivati.' 
\par 30 Još im reci: 'Pošto od toga prinesete najbolji dio, neka  se to levitima uračuna kao prihod s gumna i prihod iz badnja. 
\par 31 Na svakome ga mjestu možete jesti, i vi i vaši ukućani, jer  to vam je nagrada za vašu službu u Šatoru sastanka. 
\par 32 Pošto  prinesete njegov najbolji dio, nećete navući na se grijeha; svetinja  Izraelaca nećete oskvrnjivati te nećete ginuti.'" 


\chapter{19}

\par 1 Jahve reče Mojsiju i Aronu: 
\par 2 "Ovo je zakonska odredba što  ju je Jahve naredio: Reci Izraelcima neka ti dovedu crvenu junicu, zdravu, na kojoj nema mane i na koju još nije stavljan jaram. 
\par 3 A vi je predajte svećeniku Eleazaru. Neka se zatim izvede  izvan tabora i zakolje pred njim. 
\par 4 Svećenik Eleazar neka uzme  njezine krvi na svoj prst pa njome poškropi sedam puta prema  pročelju Šatora sastanka. 
\par 5 Neka se onda junica spali na njegove  oči; neka joj se spale: koža, meso, krv i nečist. 
\par 6 Potom neka  svećenik uzme cedrovine, izopa i crvenoga prediva pa ih baci  usred vatre gdje se krava spaljuje. 
\par 7 Neka svećenik opere svoju  odjeću, a svoje tijelo u vodi okupa. Poslije toga neka se svećenik  vrati u tabor, ali neka je nečist do večeri. 
\par 8 I onaj koji ju  je spaljivao neka svoju odjeću opere i okupa svoje tijelo u vodi  te bude nečist do večeri. 
\par 9 A jedan čist čovjek neka pokupi  pepeo od junice pa ga pohrani izvan tabora na čisto mjesto da  se čuva izraelskoj zajednici za vodu očišćenja. To je žrtva okajnica. 
\par 10 I onaj koji skupi pepeo od junice neka opere svoju odjeću  i bude nečist do večeri. Neka to bude trajan zakon i za Izraelce i za stranca koji  među njima boravi." 
\par 11 "Tko se dotakne mrtva ljudskog tijela neka je nečist  sedam dana. 
\par 12 Takav neka se opere tom vodom trećega dana i  sedmoga dana pa će biti čist. Ako se ne opere trećega dana i  sedmoga dana, neće biti čist. 
\par 13 Tko se dotakne mrtvaca, tijela  preminula čovjeka, a ne opere se, oskvrnjuje Jahvino prebivalište.  Takav neka se iskorijeni iz Izraela. Budući da vodom za očišćenje  nije bio poliven, nečist je; njegova je nečistoća još na njemu." 
\par 14 "Ovo je zakon kad koji čovjek umre u šatoru; tko god  uđe u šator i tko god bude u šatoru neka je nečist sedam dana. 
\par 15 Svaka otvorena posuda koja ne bude zatvorena poklopcem neka  je nečista. 
\par 16 A na otvorenu polju tko se god dotakne poginuloga  od mača, ili mrtvaca, ili ljudskih kostiju, ili groba neka je  nečist sedam dana. 
\par 17 Neka se za onoga koji se onečistio uzme  pepela od životinje spaljene za okajnicu i na nj, u kakvu sudu, nalije žive vode. 
\par 18 Onda neka čist čovjek uzme izopa, zamoči  ga u vodu te poškropi po šatoru, po svemu posuđu, po ljudima  koji su tu bili, po onome koji se dotakao kostiju, ili ubijenoga, ili preminuloga, ili groba. 
\par 19 Neka čisti čovjek škropi nečistoga  trećega i sedmoga dana. Tako će ga na sedmi dan očistiti. Taj  onda neka opere svoju odjeću, okupa se u vodi i neka je navečer  čist. 
\par 20 A bude li tko nečist pa se ne očisti, neka se iskorijeni  iz zajednice, jer je oskvrnuo Jahvino svetište; vodom za očišćenje  nije bio poliven; nečist je! 
\par 21 Neka im i ovo bude trajnim zakonom: i onaj koji je škropio  vodom za očišćenje neka opere svoju odjeću; i onaj koji je dirnuo  vodu za očišćenje neka je nečist do večeri. 
\par 22 Čega se god nečisti  dotakne neka je nečisto; a osoba koja se njega dotakne neka je  nečista do večeri." 


\chapter{20}

\par 1 Potom stigoše Izraelci, sva zajednica, u pustinju Sin u prvome  mjesecu. Narod se nastani u Kadešu. Ondje umrije Mirjam i ondje  je sahraniše. 
\par 2 Nije bilo vode za zajednicu. Stoga se udruže protiv Mojsija  i protiv Arona. 
\par 3 Narod se poče svađati s Mojsijem i govoriti:  "Da smo bar izginuli kad su nam i braća poginula pred Jahvom! 
\par 4 Zašto ste doveli Jahvinu zajednicu u ovu pustinju da ovdje  pomremo i mi i naša stoka? 
\par 5 Zašto ste nas izveli iz Egipta  da nas dovedete u ovo nesretno mjesto; mjesto u kojem nema ni  žita, ni smokava, ni loze, ni mogranja? Nema ni vode da pijemo." 
\par 6 Mojsije i Aron odu ispred zajednice do ulaza u Šator sastanka  i padnu ničice. Tada im se pokaza slava Jahvina. 
\par 7 I Jahve reče  Mojsiju: 
\par 8 "Uzmi štap pa ti i tvoj brat Aron skupite zajednicu.  Onda, na njihove oči, progovorite pećini da ustupi svoje vode.  Iz pećine im izvedi vodu te napoj zajednicu i njezino blago." 
\par 9 Mojsije uzme štap ispred Jahve kako mu je naredio. 
\par 10 Zatim  Mojsije i Aron skupe zbor pred pećinu pa im Mojsije rekne: "Čujte, buntovnici! Hoćemo li vam iz ove pećine izvesti vodu?" 
\par 11 Zatim Mojsije podigne ruku i dvaput udari štapom o pećinu:  voda provali u obilju, pa su mogli piti i zajednica i njezino  blago. 
\par 12 Potom će Jahve Mojsiju i Aronu: "Budući da se niste pouzdavali  u me i niste me svetim očitovali u očima sinova Izraelovih, nećete  uvesti ovaj zbor u zemlju koju im dajem." 
\par 13 To su Meripske vode, kraj njih su se Izraelci prepirali  s Jahvom, a on se pokazao svetim. 
\par 14 Iz Kadeša pošalje Mojsije glasnike: "Kralju Edoma. Ovako  veli tvoj brat Izrael: 'Ti znaš sve jade koji su nas snašli. 
\par 15 Naši se preci spustiše u Egipat. U Egiptu smo proboravili  mnogo vremena. Egipćani su s nama i s našim precima loše postupali. 
\par 16 Stoga smo vapili Jahvi, i on ču naš glas i posla anđela koji  nas izbavi iz Egipta. Evo nas sad u Kadešu, gradu uz rub tvoga  područja. 
\par 17 Pusti nas da prođemo kroz tvoju zemlju. Nećemo  ići preko polja ni vinograda niti ćemo piti vodu iz bunara; ići  ćemo Kraljevskim putem, ne skrećući ni desno ni lijevo, dok ne  prođemo tvoje područje." 
\par 18 Edom mu odgovori: "Ne prolazi preko moje zemlje, jer  eto me s mačem preda te!" 
\par 19 "Ići ćemo utrenikom", rekoše Izraelci, "a budemo li pili  tvoje vode, mi i naša stada, za to ćemo ti platiti. Ništa više, samo da prođemo pješice." 
\par 20 "Ne prolazi!" - odgovori. I Edom mu izađe u susret s mnogo ljudi i s velikom silom. 
\par 21 Tako Edom nije dopustio Izraelu da prođe kroz njegovo  područje i Izrael se okrenu od njega. 
\par 22 Zaputivši se od Kadeša, stigoše Izraelci, sva zajednica, k brdu Horu. 
\par 23 Kod brda Hora, uz među edomsku, reče Jahve  Mojsiju i Aronu: 
\par 24 "Neka se Aron pridruži svojim precima! Neće  ući u zemlju koju dajem Izraelcima, jer ste se oprli mojoj zapovijedi  kod Meripskih voda. 
\par 25 Uzmi Arona i njegova sina Eleazara, pa  ih izvedi na brdo Hor. 
\par 26 I svuci Aronu njegove haljine pa ih  obuci njegovu sinu Eleazaru. Aron će se pridružiti precima, umrijet  će ondje." 
\par 27 Mojsije učini kako naredi Jahve. Pred svom zajednicom  popeše se na brdo Hor. 
\par 28 Mojsije svuče s Arona njegove haljine  te ih obuče njegovu sinu Eleazaru. Ondje navrh brda umrije Aron.  Zatim se Mojsije i Eleazar spustiše s brda. 
\par 29 Sva zajednica  vidje da je Aron preminuo i sav dom Izraelov oplakivaše Arona  trideset dana. 


\chapter{21}

\par 1 Kralj Arada, Kanaanac koji je živio u Negebu, ču da Izrael  dolazi Atarimskim putem, pa navali na Izraela i neke njegove  zarobi. 
\par 2 Tada se Izrael ovako zavjetova Jahvi: "Ako u moje  ruke izručiš ovaj narod, potpuno ću uništiti njegove gradove." 
\par 3 Jahve usliša glas Izraela i predade mu Kanaance. A Izrael  njih i njihove gradove 'heremom' uništi. Stoga se ono mjesto  prozva Horma. 
\par 4 Od brda Hora zapute se prema Crvenom moru da zaobiđu zemlju  edomsku. Narod putem postane nestrpljiv. 
\par 5 I počne govoriti  i protiv Boga i protiv Mojsija: "Zašto nas izvedoste iz Egipta  da pomremo u ovoj pustinji? Nema kruha, nema vode, a to bijedno  jelo već se ogadilo dušama našim." 
\par 6 Onda Jahve pošalje na narod  ljute zmije; ujedale ih one, tako te pomrije mnogo naroda u Izraelu. 
\par 7 Dođe narod k Mojsiju pa reče: "Sagriješili samo kad smo govorili  protiv Jahve i protiv tebe. Pomoli se Jahvi da ukloni zmije od  nas!" Mojsije se pomoli za narod, 
\par 8 i Jahve reče Mojsiju: "Napravi  otrovnicu i stavi je na stup: tko god bude ujeden, ostat će na  životu ako je pogleda." 
\par 9 Mojsije napravi zmiju od mjedi i postavi je na stup. Kad  bi koga ujela ljutica, pogledao bi u mjedenu zmiju i ozdravio. 
\par 10 Pođu Izraelci i utabore se u Obotu. 
\par 11 Potom se zapute  iz Obota i utabore se kraj Ije-Abarima, u pustinji što je nasuprot  Moabu, sa strane sunčeva izlaska. 
\par 12 Odande otputuju te se utabore  u dolini Zaredu. 
\par 13 Odande krenu i utabore se s onu stranu Arnona, koji je u pustinji a izvire u području Amorejaca. Jer je Arnon  granica moapska između Moabaca i Amorejaca. 
\par 14 Zato se veli  u "Knjizi Jahvinih vojni": "Vaheb kod Sufe i doline arnonske 
\par 15 i padine doline što se naginje prema mjestu Aru i naslanja se na granicu moapsku ..." 
\par 16 Odande odoše u Beer. To je bunar o kojem je Jahve rekao  Mojsiju: "Skupi narod da im dam vode!" 
\par 17 Tada Izrael zapjeva ovu pjesmu: "Proključaj, studenče! A vi ga uznosite: 
\par 18 knezovi ga iskopali, prvaci narodni izdubli žezlom, štapom svojim." Iz pustinje odu u Matanu, 
\par 19 iz Matane u Nahaliel, a iz  Nahaliela u Bamot; 
\par 20 iz Bamota u dolinu što se stere u moapskom  polju, prema vrhuncu Pisge, s koje se pruža vidik na pustaru. 
\par 21 Sad Izrael posla glasnike Sihonu, amorejskome kralju, s porukom: 
\par 22 "Pusti da prođem preko tvoje zemlje. Nećemo zalaziti  u polja i u vinograde, niti ćemo piti vode iz bunara. Ići ćemo  Kraljevskim putem dok ne prođemo tvoje područje." 
\par 23 Ali Sihon  ne dopusti Izraelu da prođe njegovim područjem, nego skupi sav  svoj narod te izađe u pustinju da presretne Izraelce. Stigavši  do Jahze, zavojuje na Izraela. 
\par 24 Ali ga Izrael potuče oštrim  mačem i osvoji njegovu zemlju od Arnona do Jaboka, do Amonaca, jer je Az ležao na granici Amonaca. 
\par 25 Izrael zauzme sve one gradove i Izrael se nastani u svim  onim gradovima Amorejaca; u Hešbonu i svim njegovim naseljima. 
\par 26 Kako je Hešbon bio glavni grad Sihona, amorejskog kralja, koji je ratovao protiv prijašnjega moapskoga kralja te osvojio  svu njegovu zemlju do Arnona, 
\par 27 kažu zato pjesnici: "Hrabro, o Hešbone, dobro sazdani, čvrsto posađeni grade Sihonov! 
\par 28 Iz Hešbona oganj suknu, plamen iz grada Sihonova, sažga Ar moapski, proždrije visove arnonske. 
\par 29 Teško tebi, Moabe! Propao si, narode Kemošev! Od sinova bjegunce učini, a od kćeri svojih ropkinje Sihonu, kralju amorejskom. 
\par 30 Pobili smo ih; propao je Hešbon do Dibona: sve smo razorili do Nofaha, što je blizu Medebe ..." 
\par 31 Tako se Izrael nastani u zemlji Amorejaca. 
\par 32 Mojsije  se uputi da izvidi Jazer. Potom zauzmu njegova naselja a rastjeraju  Amorejce koji bijahu ondje. 
\par 33 Okrenu se onda i pođu prema Bašanu. A bašanski kralj  Og presrete ih sa svim svojim narodom da zapodjene boj kod Edreja. 
\par 34 Ali Jahve reče Mojsiju: "Ne boj ga se! Predao sam u tvoje  ruke njega, sav njegov narod i njegovu zemlju. Postupi s njim  kako si postupio s amorejskim kraljem Sihonom koji je boravio  u Hešbonu." 
\par 35 I potukoše ga, i sinove njegove, i sav njegov narod,  tako da nitko ne uteče. Potom zaposjedoše njegovu zemlju. 


\chapter{22}

\par 1 Poslije toga Izraelci otputuju i utabore se na Moapskim poljanama, s onu stranu Jordana, nasuprot Jerihonu. 
\par 2 Balak, sin Siporov, vidje sve što Izrael učini Amorejcima. 
\par 3 Moab se uvelike poboja toga naroda jer je bio brojan. Moaba obuze strah od Izraelaca. 
\par 4 Zato reče Moab midjanskim  starješinama: "Sad će ova rulja oko nas sve popasti kao što vol  popase travu po polju." Balak, sin Siporov, bijaše moapski kralj u ono vrijeme. 
\par 5 On  pošalje glasnike Bileamu, sinu Beorovu, u Petoru, koji se nalazi  na Rijeci, u zemlji Amonaca. Pozove ga rekavši: "Evo je došao  neki narod iz Egipta; evo je prekrio lice zemlje i naselio se  uza me. 
\par 6 Zato dođi i prokuni mi ovaj narod jer je jači od mene.  Tako ću ga moći svladati i istjerati iz zemlje. A znam da je  blagoslovljen onaj koga blagosloviš, a proklet onaj koga prokuneš." 
\par 7 Starješine moapske i starješine midjanske krenu s nagradom  za vračanje u svojim rukama. Stignu Bileamu i prenesu mu Balakovu  poruku. 
\par 8 On im rekne: "Prenoćite ovdje te ću vam odgovoriti  prema onome što mi Jahve kaže." Tako moapski knezovi ostanu kod  Bileama. 
\par 9 Bog dođe Bileamu i upita: "Tko su ti ljudi s tobom?" Bileam  odgovori Bogu: 
\par 10 "Poslao ih k meni Balak, sin Siporov, moapski  kralj, s porukom: 
\par 11 'Evo je neki narod došao iz Egipta i prekrio  lice zemlje. Dođi da ga prokuneš. Tako ću ga moći svladati i  protjerati.'" 
\par 12 Ali Bog reče Bileamu: "Nemoj ići s njima! Nemoj  proklinjati onaj narod jer je blagoslovljen." 
\par 13 Ujutro Bileam ustane te će Balakovim knezovima: "Odlazite  u svoju zemlju jer mi ne da Jahve da pođem s vama." 
\par 14 Moapski  se knezovi dignu, odu Balaku pa mu reknu: "Bileam nije htio poći  s nama." 
\par 15 Balak opet pošalje knezove, brojnije i uglednije od prvih. 
\par 16 Oni dođu Bileamu i reknu mu: "Ovako je poručio Balak, sin  Siporov: 'Ne skanjuj se nego dođi k meni. 
\par 17 Bogato ću te nagraditi  i učinit ću sve što mi kažeš. Dođi, molim te, i prokuni mi ovaj  narod!'" 
\par 18 Ali Bileam odgovori Balakovim slugama: "Da mi Balak  dadne svoju kuću punu srebra i zlata, ne bih mogao prestupiti  zapovijedi Jahve, Boga svoga, da učinim išta, bilo veliko bilo  malo. 
\par 19 Ali provedite ovdje i vi noć da doznam što će mi Jahve  još kazati." 
\par 20 Noću Bog dođe Bileamu pa mu rekne: "Ako su ti  ljudi došli da te pozovu, ustani, pođi s njima! Ali da činiš  samo što ti ja reknem!" 
\par 21 Ustane Bileam ujutro, osamari svoju magaricu i ode s  moapskim knezovima. 
\par 22 No Božja srdžba usplamtje što je on pošao.  Zato anđeo Jahvin stade na put da ga spriječi. On je jahao na  svojoj magarici, a pratila ga njegova dva momka. 
\par 23 Kad magarica  opazi anđela Jahvina kako stoji na putu s isukanim mačem u ruci, skrene sa staze i pođe preko polja. Bileam poče tući magaricu  da je vrati na put. 
\par 24 Anđeo Jahvin tada stade na uskom prolazu, među vinogradima, a bijaše ograda i s ove i s one strane. 
\par 25 Magarica, spazivši Jahvina anđela, stisne se uza zid i o zid pritisne  Bileamovu nogu. On je opet poče tući. 
\par 26 Anđeo Jahvin pođe naprijed  te stade na usko mjesto gdje nije bilo prostora da se provuče  ni desno ni lijevo. 
\par 27 Kad je magarica ugledala Jahvina anđela, legne pod Bileamom. Bileam pobjesni i poče tući magaricu štapom. 
\par 28 Tada Jahve otvori usta magarici te ona progovori Bileamu:  "Što sam ti učinila da si me tukao tri puta?" 
\par 29 Bileam odgovori  magarici: "Što sa mnom zbijaš šalu! Da mi je mač u ruci, sad  bih te ubio!" 
\par 30 A magarica uzvrati Bileamu: "Zar ja nisam tvoja  magarica na kojoj si jahao svega svoga vijeka do danas? Jesam  li ti običavala ovako?" - "Nisi!" - odgovori on. 
\par 31 Tada Jahve otvori oči Bileamu i on opazi anđela Jahvina  kako stoji na putu s golim mačem u ruci. Pognu on glavu i pade  ničice. 
\par 32 Onda će mu anđeo Jahvin: "Zašto si tukao svoju magaricu  već tri puta? TÓa ja sam istupio da te spriječim, jer te put  meni naočigled vodi u propast. 
\par 33 Magarica me opazila i preda  mnom se uklonila sva tri puta. Da mi se nije uklanjala, već bih  te ubio, a nju ostavio na životu." 
\par 34 Onda će Bileam anđelu  Jahvinu: "Sagriješio sam! Nisam znao da ti preda mnom stojiš  na putu. Ali sad, ako je zlo u tvojim očima, ja ću se vratiti." 
\par 35 Ali anđeo Jahvin odvrati Bileamu: "Idi s tim ljudima, ali  samo ono govori što ti ja kažem." Tako Bileam ode s Balakovim  knezovima. 
\par 36 Kad je Balak čuo da Bileam dolazi, iziđe mu u susret  do grada Moaba što se nalazi na granici Arnona, na kraju područja. 
\par 37 "Zar nisam uporno po te slao i pozivao te? Zašto mi nisi  došao?" reče Balak Bileamu. "Zar te zaista ne mogu bogato nagraditi?" 
\par 38 "Evo sam ti došao", reče Bileam Balaku. "Ali hoću li ti moći  sada što kazati? Samo što mi Bog stavi na jezik, to ću govoriti." 
\par 39 Pođe zatim Bileam s Balakom i dođoše u Kirjat Husot. 
\par 40 Žrtvova  Balak i krupne i sitne stoke te od toga pruži Bileamu i knezovima  koji su ga pratili. 
\par 41 Sutradan uze Balak Bileama i odvede ga gore na Bamot-Baal, odakle mogaše vidjeti krajnji dio naroda. 


\chapter{23}

\par 1 I Bileam reče Balaku: "Ovdje mi načini sedam žrtvenika; ovdje  mi pripravi sedam junaca i sedam ovnova." 
\par 2 Balak učini kako  je Bileam rekao. A onda Balak i Bileam prinesu po jednoga junca  i ovna na svakome žrtveniku. 
\par 3 Potom će Bileam Balaku: "Ti stoj  kod svoje paljenice, a ja idem ne bih li se sreo s Jahvom, pa  što mi očituje, kazat ću ti." I ode na osamljeno mjesto. 
\par 4 I Bog srete Bileama, koji mu reče: "Sedam sam žrtvenika  podigao i prinio na svakome po jednoga junca i ovna." 
\par 5 A Jahve  stavi riječi u usta Bileamu te mu zapovjedi: "Vrati se Balaku  i ovako govori." 
\par 6 Bileam se vrati k njemu, a on stajaše uza svoju paljenicu  i s njim svi knezovi moapski. 
\par 7 Tada on poče svoju pjesmu i  reče:  "Iz Arama dovede me Balak, kralj Moaba, iz strana istočnih: 'Dođi, prokuni mi Jakova, dođi, gromom udri Izraela!' 
\par 8 Kako mogu proklinjati koga Bog ne proklinje? Kako gromom udarati koga Jahve ne udara? 
\par 9 Jer s vrha hridi ja ga gledam, s visoka ga motrim brijega. Gle naroda koji odvojeno živi, među narode on se ne broji. 
\par 10 Prah Jakovljev tko će prebrojiti; pijesak Izraela tko će izmjeriti! O, da mi je umrijeti smrću pravednika! O, da svršetak moj bude kao njegov!" 
\par 11 "Što mi to uradi!" - reče Balak Bileamu. "Dovedoh te  da prokuneš moje neprijatelje, a kad tamo, ti ih blagoslovom  obasu!" 
\par 12 On odgovori: "Zar mi nije dužnost kazati što mi Jahve  stavlja u usta?" 
\par 13 "Hajde sa mnom na drugo mjesto, odakle ga možeš svega  vidjeti" zamoli ga Balak. "Odavde mu vidiš samo jedan kraj, a  ne vidiš ga svega. Odande mi ga prokuni!" 
\par 14 Povede ga zatim na Sede Sofim, na vrh Pisge. Tu sagradi  sedam žrtvenika i na svakom žrtveniku prinese po jednoga junca  i ovna. 
\par 15 Bileam tada rekne Balaku: "Stoj ovdje kraj svoje  paljenice, a ja odoh onamo na susret Bogu." 
\par 16 Jahve sretne Bileama; stavi riječi u njegova usta te  mu zapovjedi: "Vrati se k Balaku i tako govori!" 
\par 17 I vrati  se on Balaku, koji stajaše uza svoju paljenicu i s njim moapski  knezovi. Balak ga zapita: "Što je Jahve rekao?" 
\par 18 Tada Bileam započe svoju pjesmu i reče: "Ustani, Balače, i poslušaj! Uhom me posluhni, sine Siporov! 
\par 19 Bog nije čovjek da bi slagao, nije sin Adama da bi se kajao. Zar on kada rekne, a ne učini, zar obeća, pa ne ispuni? 
\par 20 Gle, primih od Boga da blagoslovim, blagoslovit ću i povuć' neću blagoslova. 
\par 21 U Jakovu nesreće ne nazreh, nit' nevolje vidjeh u Izraelu. Jahve, Bog njegov, s njime je, poklik kralju u njemu odzvanja. 
\par 22 Iz Egipta Bog ga je izveo, on je njemu k'o rozi bivola. 
\par 23 Gatanja nema protiv Jakova nit' protiv Izraela vračanja. I kada budu rekli Jakovu i Izraelu: 'Što radi Bog?' 
\par 24 gle, ustat će narod k'o lavica, dići će se poput lava: leći neće dok plijen ne proguta, dok ne popije krv pobijenih." 
\par 25 Zatim Balak reče Bileamu: "Nemoj ga ni kletvom kleti, ali ni blagoslovom blagoslivljati." 
\par 26 Bileam odvrati Balaku.  "Zar ti nisam rekao: sve što Jahve kaže, to ću činiti." 
\par 27 Potom Balak reče Bileamu: "Hajde! Odvest ću te na drugo  mjesto. Možda će Bogu biti pravo da mi ga odande prokuneš." 
\par 28 I  odvede Balak Bileama na vrh Peora, odakle se pruža vidik na pustaru. 
\par 29 "Sagradi mi ovdje sedam žrtvenika", reče Bileam Balaku.  "Nadalje, pripremi mi ovdje sedam junaca i sedam ovnova." 
\par 30 Balak učini kako je Bileam rekao i prinese po jednoga  junca i ovna na svakome žrtveniku. 


\chapter{24}

\par 1 Kad opazi Bileam da je Jahvi drago što on blagoslivlja Izraela, ne htjede više ni ići kao prije u potragu za znamenjima, nego  se licem okrenu prema pustari. 
\par 2 Bileam podiže oči i vidje Izraela  utaborena po njegovim plemenima. Na nj siđe Duh Božji 
\par 3 i on  poče svoju pjesmu te reče: "Proročanstvo Bileama, sina Beorova, proročanstvo čovjeka pronicava pogleda, 
\par 4 proročanstvo onoga koji riječi Božje sluša, koji vidi viđenja Svesilnoga, koji pada i oči mu se otvaraju. 
\par 5 Kako su lijepi ti šatori, Jakove, i stanovi tvoji, Izraele! 
\par 6 Kao dolovi što se steru, kao vrtovi uz obalu rijeke, kao aloje što ih Jahve posadi, kao cedri pokraj voda! 
\par 7 Iz potomstva junak mu izlazi, nad mnogim on vlada narodima. Kralj će njegov nadvisit' Agaga, uzdiže se kraljevstvo njegovo. 
\par 8 Iz Egipta Bog ga izveo, on je njemu k'o rozi bivola. On proždire narode dušmanske, on njihove kosti drobi. 
\par 9 Skupio se, polegao poput lava, poput lavice: tko ga podići smije? Blagoslovljen bio tko te blagoslivlje, proklet da je tko tebe proklinje!" 
\par 10 I usplamtje srdžbom Balak na Bileama i udari rukom o  ruku. "Pozvao sam te da prokuneš moje neprijatelje", reče Balak  Bileamu, "a kad tamo, ti ih blagoslovi evo triput! 
\par 11 Nosi se  odmah u svoj kraj. Bio sam rekao: dostojno ću te počastiti! A  eto, Jahve te liši časti." 
\par 12 Nato Bileam odgovori Balaku: "Zar  nisam rekao i tvojim glasnicima koje si k meni poslao: 
\par 13 'Da  mi Balak dadne svoju kuću punu srebra i zlata, ne bih mogao prestupiti  zapovijed Jahvinu i po svojoj volji činiti bilo dobro, bilo zlo;  ono što kaže Jahve, to ću i ja reći.' 
\par 14 A sada, kad, evo, odlazim  k svome narodu, hajde da ti objavim što će ovaj narod učiniti  tvome narodu u budućnosti!" 
\par 15 I poče svoju pjesmu i reče: "Proročanstvo Bileama, sina Beorova, proročanstvo čovjeka pronicava pogleda, 
\par 16 proročanstvo onoga koji riječi Božje sluša, koji poznaje mudrost Svevišnjega, koji vidi viđenja Svesilnoga, koji pada i oči mu se otvaraju. 
\par 17 Vidim ga, ali ne sada: motrim ga, al' ne iz blizine: od Jakova zvijezda izlazi, od Izraela žezlo se diže. On Moabu razbija bokove i svu djecu Šetovu zatire! 
\par 18 Edom će njegovim postati posjedom, a Seir zemljom osvojenom. Razvija snagu svoju Izrael, 
\par 19 Jakov vlada nad neprijateljima i uništava preživjele iz Ira." 
\par 20 Bileam se zagleda u Amaleka te poče svoju pjesmu i reče: "Amalek je prvenac među narodima, ali vječna propast njegov je svršetak." 
\par 21 Onda se zagleda u Kenijce te poče svoju pjesmu i reče: "Tvrd je stan tvoj, Kajine, na timoru ti gnijezdo savijeno! 
\par 22 Al' gnijezdo pripada Beoru; dokle ćeš Ašuru robovati?" 
\par 23 Opet poče svoju pjesmu i reče: "Narodi pomorski sabiru se sa sjevera, 
\par 24 a brodovlje od strane Kitima. Podjarmljuju Ašur, podjarmljuju Heber, pa i njega će propast stići vječita." 
\par 25 Potom ustade Bileam te se uputi natrag u svoj kraj. A  i Balak ode svojim putem. 


\chapter{25}

\par 1 Dok je Izrael boravio u Šitimu, narod se upusti u blud s Moapkama. 
\par 2 One pozivahu narod na žrtvovanje svojim bogovima, a narod  sudjelovaše u njihovim gozbama i klanjaše se njihovim bogovima. 
\par 3 Tako se Izrael osramoti s Baalom peorskim. I Jahve planu gnjevom  na Izraela. 
\par 4 "Pokupi sve narodne glavare", reče Jahve Mojsiju. "Objesi  ih Jahvi usred bijela dana da se Jahvin gnjev odvrati od Izraela." 
\par 5 Onda Mojsije rekne izraelskim sucima: "Neka svatko pobije  one svoje ljude koji su se osramotili s Baalom peorskim." 
\par 6 Baš tada neki Izraelac dođe i dovede k svojoj braći jednu  Midjanku naočigled Mojsija i naočigled sve izraelske zajednice  koja zaplaka na ulazu u Šator sastanka. 
\par 7 Kad to opazi Pinhas, sin Eleazara, sina svećenika Arona, ustade ispred zajednice:  uze koplje u ruku 
\par 8 i pođe za Izraelcem u odaje i probode ih  oboje, Izraelca i ženu; nju kroza slabine. Tako pomor Izraelaca  prestade. 
\par 9 A onih koji su od pomora pomrli bilo je dvadeset  i četiri tisuće. 
\par 10 Jahve reče Mojsiju: 
\par 11 "Pinhas, sin Eleazara, sina svećenika  Arona, odvratio je moj gnjev od Izraelaca, obuzet među njima  mojim revnovanjem. Zato u svome revnovanju nisam istrijebio izraelskoga  naroda. 
\par 12 Kaži mu dakle: 'S njime, evo, sklapam savez mira. 
\par 13 Neka to bude za nj i njegove potomke poslije njega savez  vječnoga svećeništva, jer je revnovao za svoga Boga i izvršio  pomirenje za izraelski narod.'" 
\par 14 Ime Izraelcu koji je bio ubijen - onome što je ubijen  s Midjankom - bijaše Zimri. Bio je sin Salua, glavara jedne od  Šimunovih porodica. 
\par 15 A ime ubijene žene Midjanke bijaše Kozbi.  Bila je kći Surova. Sur je bio glavar jednog plemena, jedne porodice  u Midjanu. 
\par 16 Jahve reče Mojsiju: 
\par 17 "Navali na Midjance i potuci  ih, 
\par 18 jer su i oni navaljivali na vas svojim lukavštinama kad  su lukavo radili protiv vas u slučaju Peora i svoje sestre Kozbi, kćeri glavara midjanskoga, koja je zaglavila u vrijeme pomora  nastalog zbog Peora." 


\chapter{26}

\par 1 Poslije toga zla Jahve reče Mojsiju i Eleazaru, sinu svećenika  Arona: 
\par 2 "Obavite popis sve zajednice sinova Izraelovih, po  njihovim porodicama, popišite sve, od dvadeset godina pa naviše, koji su u Izraelu sposobni za borbu." 
\par 3 Mojsije, dakle, i svećenik Eleazar popišu ih na Moapskim  poljanama, uz Jordan blizu Jerihona, 
\par 4 sve od dvadeset godina  pa naviše, kako je Jahve naredio Mojsiju i Izraelcima. Sinovi  Izraelovi koji su izašli iz zemlje egipatske bili su: 
\par 5 Izraelov prvorođenac Ruben. Sinovi Rubenovi: od Henoka  rod Henokovaca; od Palua rod Paluovaca; 
\par 6 od Hesrona rod Hesronovaca  i od Karmija rod Karmijevaca. 
\par 7 To su rodovi Rubenovaca. Njih  je bilo četrdeset i tri tisuće sedam stotina i trideset. 
\par 8 Paluov sin bijaše Eliab, 
\par 9 a sinovi Eliabovi: Nemuel, Datan i Abiram. Taj Datan i Abiram bijahu ugledni članovi zajednice  koji se podigoše protiv Mojsija i Arona u buni Korahovoj, kad  se pobuniše protiv Jahve. 
\par 10 Nato je zemlja rastvorila svoje  ralje i progutala ih zajedno s Korahom, kad je smrt pograbila  tu skupinu i oganj proždro dvjesta i pedeset ljudi. Tako su postali  opomenom. 
\par 11 No sinovi Korahovi ne izginuše. 
\par 12 Sinovi Šimunovi po svojim rodovima: od Nemuela rod Nemuelovaca;  od Jamina rod Jaminovaca; od Jakina rod Jakinovaca; 
\par 13 od Zeraha  rod Zerahovaca i od Šaula rod Šaulovaca. 
\par 14 To su rodovi Šimunovaca, njih dvadeset i dvije tisuće i dvjesta. 
\par 15 Sinovi Gadovi po svojim rodovima: od Sefona rod Sefonovaca;  od Hagija rod Hagijevaca; od Šunija rod Šunijevaca; 
\par 16 od Oznija  rod Oznijevaca; od Erija rod Erijevaca; 
\par 17 od Aroda roda Arodovaca  i od Arelija rod Arelijevaca. 
\par 18 To su rodovi Gadovih potomaka.  Njih je upisano četrdeset tisuća i pet stotina. 
\par 19 Judini sinovi: Er i Onan. I Er i Onan umriješe u zemlji  kanaanskoj. 
\par 20 Sinovi Judini po svojim rodovima bijahu: od Šele rod  Šelinaca; od Peresa rod Peresovaca i od Zeraha rod Zerahovaca. 
\par 21 Peresovi su sinovi opet bili: od Hesrona rod Hesronovaca  i od Hamula rod Hamulovaca. 
\par 22 To su Judini rodovi. Njih je  upisano sedamdeset i šest tisuća i petsto. 
\par 23 Sinovi Jisakarovi, prema svojim rodovima: od Tole rod  Tolinaca; od Puve rod Puvinaca; 
\par 24 od Jašuba rod Jašubovaca  i od Šimrona rod Šimronovaca. 
\par 25 To su Jisakarovi rodovi. Njih  je upisano šezdeset i četiri tisuće i trista. 
\par 26 Sinovi Zebulunovi, po svojim rodovima: od Sereda rod  Seredovaca; od Elona rod Elonovaca i od Jahleela rod Jahleelovaca. 
\par 27 To su rodovi Zebulunovaca. Njih je upisano šezdeset tisuća  i pet stotina. 
\par 28 Sinovi Josipovi, po svojim rodovima: Manaše i Efrajim. 
\par 29 Sinovi Manašeovi: od Makira rod Makirovaca. Makiru se rodio  Gilead. Od Gileada rod Gileadovaca. 
\par 30 Ovo su bili sinovi Gileadovi:  od Jezera rod Jezerovaca; od Heleka rod Helekovaca; 
\par 31 od Asriela  rod Asrielovaca; od Šekema rod Šekemovaca; 
\par 32 od Šemide rod  Šemidinaca i od Hefera rod Heferovaca. 
\par 33 Heferov sin Selofhad  nije imao sinova, nego kćeri. Imena Selofhadovih kćeri bila su:  Mahla, Noa, Hogla, Milka i Tirsa. 
\par 34 To su Manašeovi rodovi.  Njih je upisano pedeset i dvije tisuće i sedam stotina. 
\par 35 Ovo su opet sinovi Efrajimovi, po svojim rodovima: od  Šutelaha rod Šutelahovaca; od Bekera rod Bekerovaca i od Tahana  rod Tahanovaca. 
\par 36 Ovo su sinovi Šutelahovi: od Erana rod Eranovaca. 
\par 37 To su rodovi Efrajimovih sinova. Njih je upisano trideset  i dvije tisuće i pet stotina. To su sinovi Josipovi, po svojim rodovima. 
\par 38 Sinovi Benjaminovi, po svojim rodovima: od Bele rod Belinaca;  od Ašbela rod Ašbelovaca; od Ahirama rod Ahiramovaca; 
\par 39 od  Šefufama rod Šefufamovaca i od Hufama rod Hufamovaca. 
\par 40 Belini  sinovi bili su: Ard i Naaman. I tako, od Arda rod Ardovaca, a  od Naamana rod Naamanovaca. 
\par 41 To su sinovi Benjaminovi, po  svojim rodovima. Njih je upisano četrdeset i pet tisuća i šest  stotina. 
\par 42 Ovo su sinovi Danovi, po svojim rodovima: od Šuhama rod  Šuhamovaca. To su sinovi Danovi, prema svojim rodovima. 
\par 43 Od  svih rodova Šuhamovaca bilo je upisano šezdeset i četiri tisuće  i četiri stotine. 
\par 44 Sinovi Ašerovi, po svojim rodovima: od Jimne rod Jimninaca;  od Jišvija rod Jišvijevaca i od Berije rod Berijevaca. 
\par 45 Od  sinova Berijinih: od Hebera rod Heberovaca i od Malkiela rod  Malkielovaca. 
\par 46 Ašerovoj kćeri bilo je ime Serah. 
\par 47 To su  rodovi Ašerovih sinova. Njih je upisano pedeset i tri tisuće  i četiri stotine. 
\par 48 Sinovi Naftalijevi, po svojim rodovima: od Jahseela rod  Jahseelovaca; od Gunija rod Gunijevaca; 
\par 49 od Jesera rod Jeserovaca  i od Šilema rod Šilemovaca. 
\par 50 To su rodovi Naftalijevaca. Po  njihovim rodovima upisano ih je četrdeset i pet tisuća i četiri  stotine. 
\par 51 Bilo je, dakle, upisanih Izraelaca šest stotina i jedna  tisuća i sedam stotina i trideset. 
\par 52 Jahve reče Mojsiju: 
\par 53 "Tima neka se razdijeli zemlja  u baštinu prema broju osoba. 
\par 54 Većem broju povećaj njegovu  baštinu, a manjem smanji njegovu baštinu; neka se svakomu dadne  njegova baština prema broju upisanih. 
\par 55 Ali zemlja neka se  podijeli kockom: neka se primi u baštinu prema djedovskim plemenskim  imenima. 
\par 56 Baština se ima podijeliti kockom svakom plemenu  prema njegovoj veličini." 
\par 57 Ovo je popis Levijevaca, po njihovim rodovima: od Geršona  rod Geršonovaca; od Kehata rod Kehatovaca i od Merarija rod Merarijevaca. 
\par 58 Ovo su rodovi Levijevaca: rod Libnijevaca, rod Hebronovaca, rod Mahlijevaca, rod Mušijevaca i rod Korahovaca. Kehatu se  rodio Amram. 
\par 59 Amramovoj ženi bijaše ime Jokebeda. Bila je  kći Levijeva, koja se Leviju rodila u Egiptu. Ona je Amramu rodila:  Arona, Mojsija i njihovu sestru Mirjam. 
\par 60 Aronu se rodili:  Nadab, Abihu, Eleazar i Itamar. 
\par 61 Nadab i Abihu poginuli su  kad su prinosili neposvećenu vatru pred Jahvom. 
\par 62 Svih je popisanih  muškaraca od jednog mjeseca pa naviše bilo dvadeset i tri tisuće.  Oni nisu bili popisani s Izraelcima i nije im bila dodijeljena  baština među Izraelcima. 
\par 63 To su, dakle, oni koje je popisao Mojsije i svećenik  Eleazar; oni su obavili ovaj popis Izraelaca uz Jordan, na Moapskim  poljanama nasuprot Jerihonu. 
\par 64 Među njima nije bilo ni jednoga  od onih koje su popisali Mojsije i svećenik Aron kad su popisivali  Izraelce u Sinajskoj pustinji. 
\par 65 Jer Jahve bijaše za njih rekao:  "Neka pomru u pustinji i neka nitko od njih ne ostane, osim Kaleba, sina Jefuneova, i Jošue, sina Nunova!" 


\chapter{27}

\par 1 Tada pristupiše kćeri Selofhada, sina Heferova, sina Gileadova, sina Makirova, sina Manašeova iz roda Josipova sina Manašea.  A imena kćeri bila su: Mahla, Noa, Hogla, Milka i Tirsa. 
\par 2 One  stanu pred Mojsija, pred svećenika Eleazara, pred glavare i svu  zajednicu na ulazu u Šator sastanka pa reknu: 
\par 3 "Naš je otac  umro u pustinji. Nije pripadao družini što se pobunila protiv  Jahve - Korahovoj družini - nego je umro od svoga vlastitoga  grijeha. Sinova nije imao. 
\par 4 Zašto bi se odstranilo ime našega  oca iz njegova roda? Budući da nije imao sina, daj nama posjed  među braćom našega oca!" 
\par 5 Mojsije iznese njihov slučaj pred Jahvu. 
\par 6 A Jahve reče  Mojsiju: 
\par 7 "Selofhadove kćeri pravo kažu. Treba svakako da im  dadneš posjed koji će biti njihova baština među braćom njihova  oca. Prenesi na njih baštinu njihova oca. 
\par 8 Nadalje, reci Izraelcima:  'Kad koji čovjek umre a ne imadne sina, prenesite njegovu baštinu  na njegovu kćer. 
\par 9 Ne imadne li ni kćeri, predajte baštinu njegovoj  braći. 
\par 10 Ako ne imadne ni braće, njegovu baštinu podajte braći  njegova oca. 
\par 11 Ako mu otac ne imadne braće, baštinu njegovu  podajte najbližem rođaku njegova roda: neka je on uzme u posjed.' Neka to bude zakonska odredba Izraelcima, kako je Jahve naredio  Mojsiju." 
\par 12 Jahve reče Mojsiju: "Popni se na ovo brdo Abarim i razgledaj  zemlju koju dajem Izraelcima. 
\par 13 A kad budeš razgledao, pridružit  ćeš se svojim precima i ti, kako se pridružio i tvoj brat Aron. 
\par 14 Jer ste se u pobuni zajednice u pustinji Sin usprotivili  mojim ustima umjesto da vodom očitujete moju svetost pred njihovim  očima."  (To su Meripske vode kod Kadeša u Sinskoj pustinji.) 
\par 15 A Jahvi Mojsije progovori ovako: 
\par 16 "Neka Jahve, Bog  duhova u svakom tijelu, postavi čovjeka nad ovom zajednicom 
\par 17 koji  će pred njom izlaziti; koji će pred njom stupati; koji će je  izvoditi i uvoditi tako da Jahvina zajednica ne bude kao stado  što nema pastira." 
\par 18 "Uzmi Jošuu, sina Nunova!" - reče Jahve Mojsiju. "To  je čovjek u kome ima duha. Na nj položi ruku svoju! 
\par 19 Onda  ga odvedi pred svećenika Eleazara i pred svu zajednicu te mu  na njihove oči daj naredbe! 
\par 20 Predaj mu dio svoje vlasti da  ga sluša sva zajednica sinova Izraelovih. 
\par 21 Neka pristupa k  svećeniku Eleazaru, koji će za nj tražiti odluke Urima pred Jahvom.  Na njegovu zapovijed neka izlaze i na njegovu zapovijed neka  ulaze, oni i svi Izraelci s njim - sva zajednica." 
\par 22 Mojsije učini kako mu je Jahve naredio: uzme Jošuu te  ga postavi pred svećenika Eleazara i pred svu zajednicu. 
\par 23 Položi zatim na nj svoje ruke i dade mu svoje naredbe, kako je Jahve zapovjedio preko Mojsija. 


\chapter{28}

\par 1 Jahve reče Mojsiju: 
\par 2 "Naredi Izraelcima i reci im: 'Točno  u određeno vrijeme prinosite mi moje prinose, moju hranu - žrtve  paljene meni na ugodan miris.' 
\par 3 Reci im: Ovo su žrtve paljene  koje ćete prinositi Jahvi: Svaki dan dva jednogodišnja janjca bez mane kao trajnu paljenicu. 
\par 4 Jedno janje prinosite jutrom, a drugo janje prinosite u suton. 
\par 5 A za prinosnicu desetinu efe najboljeg brašna, zamiješena  u četvrtini hina čistoga ulja. 
\par 6 Trajna je to paljenica koja  je već bila prinesena na Sinajskom brdu - žrtva spaljena na ugodan  miris Jahvi. 
\par 7 Njezina ljevanica neka se sastoji od četvrtine  hina za svako janje. Ljevanica vina neka se Jahvi izlijeva u  Svetištu. 
\par 8 Drugo janje prinosite u suton. Prinosi ga kao i  jutarnju prinosnicu i njezinu ljevanicu: kao žrtvu spaljenu Jahvi  na ugodan miris." 
\par 9 "Na subotnji dan: dva jednogodišnja janjeta bez mane i  dvije desetine efe najboljeg brašna, zamiješena s uljem, za prinosnicu, s njezinom ljevanicom. 
\par 10 Neka se subotnja paljenica svake  subote dodaje trajnoj paljenici i njezinoj ljevanici." 
\par 11 "Na početku vaših mjeseci prinosite Jahvi za paljenicu:  dva junca, jednoga ovna i sedam jednogodišnjih janjaca bez mane. 
\par 12 Za pojedinog junca kao prinosnicu: tri desetine najboljeg  brašna zamiješena s uljem; za svakog ovna kao prinosnicu: dvije  desetine efe najboljeg brašna zamiješena s uljem. 
\par 13 Za svako  janje jednu desetinu efe najboljeg brašna zamiješena s uljem  kao prinosnicu. To je paljenica spaljena na ugodan miris Jahvi. 
\par 14 Njihove ljevanice neka budu: na junca polovica hina vina;  na ovna trećina hina; na janje četvrtina hina. To neka bude mjesečna  paljenica na mlađak svakog mjeseca u godini. 
\par 15 Povrh trajne  paljenice neka se Jahvi prinosi jedan jarac kao okajnica s njezinom  ljevanicom." 
\par 16 "Prvoga mjeseca, četrnaestoga dana u mjesecu, Jahvina  je Pasha, 
\par 17 a petnaestoga dana toga mjeseca jest blagdan. Neka  se sedam dana jedu beskvasni hljebovi. 
\par 18 Prvog dana neka bude  sveti saziv. Nikakva težačkog posla nemojte raditi. 
\par 19 Prinesite  Jahvi žrtvu paljenu, žrtvu paljenicu: dva junca, jednoga ovna  i sedam jednogodišnjih janjaca. Neka vam budu bez mane. 
\par 20 Njihova  prinosnica, od najboljeg brašna zamiješena s uljem, neka bude:  tri desetine efe na junca, dvije desetine efe na ovna, 
\par 21 a  na svakoga od onih sedam janjaca neka bude jedna desetina efe. 
\par 22 Neka bude jedan jarac kao okajnica, da se nad vama izvrši  obred pomirenja. 
\par 23 Ovo prinosite povrh jutarnje paljenice,  koje se prinosi kao trajna paljenica. 
\par 24 Tako činite svaki dan  za sedam dana; to je hrana, žrtva paljena na ugodan miris Jahvi.  To neka se prinosi povrh trajne paljenice i njezine ljevanice. 
\par 25 Sedmoga dana neka vam bude sveti savez. Nikakva težačkog  posla nemojte raditi!" 
\par 26 "I na Dan prvina - na svoj Blagdan sedmica - kad budete  Jahvi prinosili prinosnicu, imajte sveti saziv: nikakva težačkog  posla nemojte raditi. 
\par 27 Za paljenicu na ugodan miris Jahvi  prinesite dva junca, jednoga ovna i sedam jednogodišnjih janjaca. 
\par 28 Njihova prinosnica, od najboljeg brašna zamiješena s uljem, neka bude: na pojedinog junca tri desetine efe, na pojedinoga  ovna dvije desetine efe, 
\par 29 a jedna desetina efe na svakoga  od onih sedam janjaca. 
\par 30 Neka bude i jedan jarac kao okajnica, da se nad vama izvrši obred pomirenja. 
\par 31 Prinosite ih povrh  trajne paljenice i njezine prinosnice, a neka vam budu bez mane  one i njihove ljevanice." 


\chapter{29}

\par 1 "U sedmome mjesecu, na prvi dan mjeseca, imajte sveti saziv.  Nikakva težačkog posla nemojte raditi. Neka vam to bude Dan sazivanja. 
\par 2 Za paljenicu na ugodan miris Jahvi prinesite: jednoga junca, jednoga ovna i sedam jednogodišnjih janjaca bez mane. 
\par 3 Njihova  prinosnica, od najboljeg brašna zamiješena s uljem, neka bude:  tri desetine efe na junca, dvije desetine efe na ovna 
\par 4 i jedna  desetina efe na svakoga od onih sedam janjaca. 
\par 5 Neka bude jedan  jarac kao okajnica, da se nad vama izvrši obred pomirenja. 
\par 6 Neka  to bude povrh paljenice o mlađaku mjesecu i njezine prinosnice, povrh trajne paljenice i njezine prinosnice i povrh njihovih  propisanih ljevanica, žrtva spaljena na ugodan miris Jahvi." 
\par 7 "A desetoga dana toga sedmog mjeseca imajte sveti saziv.  Postite i nemojte raditi nikakva posla. 
\par 8 Prinesite paljenicu  Jahvi na ugodan miris: jednoga junca, jednoga ovna i sedam jednogodišnjih  janjaca. Neka su vam bez mane. 
\par 9 Njihova prinosnica, od najboljeg  brašna zamiješena s uljem, neka bude: tri desetine efe na junca, dvije desetine na jednoga ovna 
\par 10 i jedna desetina efe na svakoga  od onih sedam janjaca. 
\par 11 Jedan jarac neka se prinese kao okajnica.  To je povrh okajnice na Dan pomirenja, povrh trajne paljenice  i njezine prinosnice i njihovih ljevanica." 
\par 12 "Na petnaesti dan sedmoga mjeseca imajte sveti saziv.  Nikakva težačkog posla nemojte raditi. Sedam dana svetkujte svečanost  Jahvi. 
\par 13 A za paljenicu, spaljenu na ugodan miris Jahvi, prinesite:  trinaest junaca, dva ovna i četrnaest jednogodišnjih janjaca.  Neka su bez mane. 
\par 14 Njihova prinosnica, od najboljeg brašna  zamiješena s uljem, neka bude: tri desetine efe na svakoga od  trinaest junaca, dvije desetine efe na svakoga od dvaju ovnova 
\par 15 i jedna desetina efe na svako pojedino od četrnaestero janjadi.  Neka se nadoda jedan jarac kao okajnica. 
\par 16 To neka bude povrh  trajne paljenice, njezine prinosnice i njezine ljevanice. 
\par 17 Drugog dana: dvanaest junaca, dva ovna, četrnaest jednogodišnjih  janjaca bez mane. 
\par 18 Njihovu prinosnicu i njihove ljevanice  prinesite propisno prema broju junaca, ovnova i janjaca. 
\par 19 Prinesite  jednoga jarca kao okajnicu povrh trajne paljenice, njezine prinosnice  i njezinih ljevanica. 
\par 20 Trećeg dana: jedanaest junaca, dva ovna, četrnaest jednogodišnjih  janjaca bez mane. 
\par 21 Njihovu prinosnicu i njihove ljevanice  prinesite propisno prema broju junaca, ovnova i janjaca. 
\par 22 Prinesite  jednoga jarca kao okajnicu povrh trajne paljenice, njezine prinosnice  i njezine ljevanice. 
\par 23 Četvrtog dana: deset junaca, dva ovna, četrnaest jednogodišnjih  janjaca bez mane. 
\par 24 Njihovu prinosnicu i njihove ljevanice  prinesite propisno prema broju junaca, ovnova i janjaca. 
\par 25 Jednog  jarca prinesite kao okajnicu povrh trajne paljenice, njezine  prinosnice i njezine ljevanice. 
\par 26 Petog dana: devet junaca, dva ovna, četrnaest jednogodišnjih  janjaca bez mane. 
\par 27 Njihovu prinosnicu i njihove ljevanice  prinesite propisno prema broju junaca, ovnova i janjaca. 
\par 28 Prinesite  jednog jarca kao okajnicu povrh trajne paljenice, njezine prinosnice  i njezine ljevanice. 
\par 29 Šestog dana: osam junaca, dva ovna, četrnaest jednogodišnjih  janjaca bez mane. 
\par 30 Njihovu prinosnicu i njihove ljevanice  prinesite propisno prema broju junaca, ovnova i janjaca. 
\par 31 Jednoga  jarca prinesite kao okajnicu povrh trajne paljenice, njezine  prinosnice i njezinih ljevanica. 
\par 32 Sedmog dana: sedam junaca, dva ovna, četrnaest jednogodišnjih  janjaca bez mane. 
\par 33 Njihovu prinosnicu i njihove ljevanice  prinesite propisno prema broju junaca, ovnova i janjaca. 
\par 34 Jednog  jarca prinesite kao okajnicu povrh trajne paljenice, njezine  prinosnice i njezine ljevanice. 
\par 35 Osmog dana imajte svečani zbor. Nikakva težačkog posla  nemojte raditi. 
\par 36 A za paljenicu, spaljenu na ugodan miris  Jahvi, prinesite: jednog junca, jednoga ovna i sedam jednogodišnjih  janjaca bez mane. 
\par 37 Njihovu prinosnicu i njihove ljevanice  prinesite propisno prema broju junaca, ovnova i janjaca. 
\par 38 Jednog  jarca prinesite kao okajnicu povrh trajne paljenice, njezine  prinosnice i njezine ljevanice. 
\par 39 Na svoje određene blagdane prinesite to Jahvi osim svojih  zavjetnica i svojih dragovoljnih žrtava, svojih paljenica, prinosnica, ljevanica i svojih pričesnica." 
\par 40 (30:1) Sve kako mu je Jahve naredio Mojsije kaza Izraelcima. 


\chapter{30}

\par 1 (30:2) Zatim reče Mojsije glavarima plemena Izraelovih: "Ovo  je Jahve naredio. 
\par 2 (30:3) Ako koji čovjek učini zavjet ili se uz zakletvu  obveže da će se nečega odreći, neka ne krši svoje riječi; neka  izvrši sve što iz njegovih usta izađe! 
\par 3 (30:4) Ako koja žena učini Jahvi zavjet ili se obveže da će se  nečega odreći dok je još mlada, u očevu domu, 
\par 4 (30:5) a otac joj sazna  za zavjet i obećanje kojim se obvezala pa joj ništa ne rekne, tada su valjani svi njezini zavjeti i valjano je svako obećanje  kojim se obvezala. 
\par 5 (30:6) Ali ako joj se otac usprotivi kad sazna, nikakav njezin zavjet ni njezino obećanje kojim se vezala ne  vrijedi. Jahve će joj oprostiti jer joj se otac usprotivio. 
\par 6 (30:7) Ako se uda dok je pod svojim zavjetima ili pod obećanjem  koje je nepromišljeno izišlo iz njezinih usta, 
\par 7 (30:8) pa njezin muž  sazna i pošto je saznao ništa joj ne rekne, tada vrijede njezini  zavjeti i vrijede obećanja kojima se obvezala. 
\par 8 (30:9) No ako se njezin  muž usprotivi kad o tom sazna, ukida se time njezin zavjet i  obećanje što je nepromišljeno izišlo iz njezinih usta. I Jahve  će joj oprostiti. 
\par 9 (30:10) A zavjet udovice ili žene otpuštene i sve obveze koje  je na se preuzela vrijede za nju. 
\par 10 (30:11) Ako se zavjetuje ili se obveže zakletvom na obećanje  dok je u kući svoga muža, 
\par 11 (30:12) pa njezin muž sazna i ništa joj  ne rekne, ne usprotivi joj se, svaki je njezin zavjet valjan  i valjano je svako obećanje kojim se obvezala. 
\par 12 (30:13) Ali ako ih  njezin muž proglasi ništetnim kad o njima sazna, tada ništa što  je izišlo iz njezinih usta, njezini zavjeti ili preuzete obveze  neće vrijediti. Muž ih je njezin poništio, i Jahve će joj oprostiti. 
\par 13 (30:14) Svaki zavjet i svaku zakletvu koja obvezuje ženu na neko  mrtvenje njezin muž može uzdržati na snazi ili poništiti. 
\par 14 (30:15) Ako  joj muž od dana do dana ništa ne rekne, time potvrđuje sve njezine  zavjete i sva njezina obećanja kojima se obvezala; on ih je učinio  valjanima ako ništa nije rekao kad je o njima čuo. 
\par 15 (30:16) Ali ako ih poništi kasnije, pošto je o njima već čuo, neka snosi njezinu krivnju." 
\par 16 (30:17) To su uredbe koje je Jahve Mojsiju izdao za muža i njegovu  ženu i za oca i njegovu kćer, koja, još mlada, živi u kući očevoj. 


\chapter{31}

\par 1 Jahve reče Mojsiju: 
\par 2 "Iskali osvetu Izraelaca na Midjancima, a poslije toga pridružit ćeš se svojim precima." 
\par 3 A Mojsije reče narodu: "Opremite ljude između sebe za  pohod na Midjance, 
\par 4 da na Midjancima izvrše Jahvinu osvetu.  Na vojnu opremite po jednu tisuću od svakoga izraelskog plemena!" 
\par 5 I tako su iz izraelskih porodica - tisuću po plemenu -  za vojnu skupili dvanaest tisuća. 
\par 6 Posla ih Mojsije - tisuću  po plemenu - na vojnu zajedno s Pinhasom, sinom svećenika Eleazara.  On je nosio posvećene stvari i trube. 
\par 7 Oni zavojuju na Midjance, kako je Jahve naredio Mojsiju, i pobiju sve muškarce. 
\par 8 Među ostalima pobili su i midjanske  kraljeve: Evija, Rekema, Sura, Hura i Rebu - pet midjanskih kraljeva.  Mačem pogube i Bileama, Beorova sina. 
\par 9 Odvedu tada Izraelci u ropstvo midjanske žene s njihovom  djecom i svu njihovu stoku, krupnu i sitnu, i zaplijene sve njihovo  blago. 
\par 10 Ognjem spale sve gradove njihove u kojima se živjeli  i sva njihova naselja, 
\par 11 a sve njihovo uzmu za plijen i pljačku, i ljude i životinje. 
\par 12 Onda u tabor na Moapskim poljanama  uz Jordan, nasuprot Jerihonu, dovedu Mojsiju, svećeniku Eleazaru  i svoj izraelskoj zajednici zarobljenike, plijen i pljačku. 
\par 13 Mojsije, svećenik Eleazar i svi glavari zajednice izađu  im u susret izvan tabora. 
\par 14 Mojsije se razljuti na zapovjednike  vojske, tisućnike i satnike, koji se bijahu vratili s toga bojnog  pohoda. 
\par 15 Reče im: "Što! Na životu ste ostavili sve ženskinje! 
\par 16 A baš su žene, po nagovoru Bileamovu, zavele Izraelce da  u Peorovu slučaju istupe protiv Jahve. Tako dođe pomor na Jahvinu  zajednicu. 
\par 17 Stoga svu mušku djecu pobijte! A ubijte i svaku  ženu koja je poznala muškarca! 
\par 18 A sve mlade djevojke koje  nisu poznale muškarca ostavite na životu za se. 
\par 19 Vi pak proboravite  izvan tabora sedam dana; svi vi koji ste koga ubili i koji ste  se ubijenoga dotakli. Čistite se i vi i vaši zarobljenici trećega  i sedmoga dana; 
\par 20 očistite svu odjeću, sve mješine, sve od  kostrijeti napravljeno i sve drvene predmete." 
\par 21 Zatim svećenik Eleazar progovori borcima koji su se vratili  iz boja: "Ovo je odredba koju je izdao Jahve Mojsiju: 
\par 22 'Zlato, srebro, bakar, gvožđe, mjed i olovo - 
\par 23 sve što podnosi vatru  - provucite kroz vatru i bit će očišćeno.' Ipak, neka se očisti  i vodom očišćenja. A sve što ne podnosi vatru provucite kroz  vodu. 
\par 24 Sedmoga dana operite svoju odjeću i bit ćete čisti.  Poslije toga možete se vratiti u tabor." 
\par 25 Jahve reče Mojsiju: 
\par 26 "Ti, svećenik Eleazar i obiteljske  starješine zajednice napravite popis ratnoga plijena, ljudstva  i stoke, 
\par 27 a onda ratni plijen podijeli napola: na borce koji  su išli u borbu i na svu ostalu zajednicu. 
\par 28 Od boraca koji  su išli u borbu ustavi ujam za Jahvu: jednu glavu od svakih pet  stotina, bilo ljudi, bilo krupnog blaga, magaradi ili sitne stoke. 
\par 29 Uzmi to od njihove polovice i podaj svećeniku Eleazaru kao  podizanicu za Jahvu. 
\par 30 A od polovice što zapadne druge Izraelce  uzmi po glavu od pedeset, bilo ljudi, bilo krupnog blaga, magaradi  ili sitne stoke - od svih životinja - pa ih podaj levitima koji  vode brigu o Jahvinu prebivalištu." 
\par 31 Mojsije i svećenik Eleazar učine kako je Jahve naredio  Mojsiju. 
\par 32 Ratnoga je plijena bilo, osim pljačke što su vojnici  napljačkali: šest stotina sedamdeset i pet tisuća grla sitne  stoke, 
\par 33 sedamdeset i dvije tisuće grla krupne stoke, 
\par 34 šezdeset  i jedna tisuća magaradi, 
\par 35 a ljudskih duša - žena koje nisu  poznale muškarca - bijaše u svemu trideset i dvije tisuće. 
\par 36 Prema tome, polovica što je dodijeljena onima koji su  išli u borbu bila je: tri stotine trideset i sedam tisuća i pet  stotina grla sitne stoke; 
\par 37 ujam za Jahvu od sitne stoke šest  stotina sedamdeset i pet grla; 
\par 38 krupne je stoke bilo trideset  i šest tisuća grla, a njihov ujam za Jahvu sedamdeset i dva grla; 
\par 39 magaradi je bilo trideset tisuća i pet stotina, a njihov  ujam za Jahvu šezdeset i jedno. 
\par 40 Ljudskih je duša bilo šesnaest  tisuća, a njihov ujam za Jahvu trideset i dvije osobe. 
\par 41 Ujam  predade Mojsije svećeniku Eleazaru za podizanicu Jahvi, kako  je Jahve naredio Mojsiju. 
\par 42 A od polovice koja je zapala druge Izraelce i koju Mojsije  odijeli od one što je pripala ljudima koji su se borili - 
\par 43 dakle, polovica što je pripala zajednici iznosila je: trista trideset  i sedam tisuća i pet stotina grla sitne stoke, 
\par 44 a krupne stoke  trideset i šest tisuća grla; 
\par 45 magaradi trideset tisuća i pet  stotina, 
\par 46 a ljudskih duša šesnaest tisuća. 
\par 47 Tako, od polovice  što je pripala Izraelcima Mojsije ostavi po jedno od pedeset, i od ljudstva i od stoke, te ih predade levitima koji su se  brinuli o Jahvinu prebivalištu, kako je Jahve naredio Mojsiju. 
\par 48 Onda pristupiše k Mojsiju vojnički zapovjednici, tisućnici  i satnici, 
\par 49 i rekoše mu: "Tvoje sluge prebrojile su borce  što bijahu pod našim zapovjedništvom i od nas nitko nije izgubljen. 
\par 50 Uz to smo donijeli svoje darove Jahvi: narukvica, orukvica, prstenja, naušnica i ogrlica - na kakvu je tko zlatninu već  naišao - da se nad nama obavi obred pomirenja pred Jahvom." 
\par 51 Mojsije i svećenik Eleazar prime od njih to zlato, to  jest sve te izrađene predmete. 
\par 52 Bilo je svega zlata što su  kao svoju podizanicu Jahvi donijeli tisućnici i satnici: šesnaest  tisuća sedam stotina i pedeset šekela. 
\par 53 Svaki je vojnik za  se zadržao svoj plijen. 
\par 54 Tako Mojsije i svećenik Eleazar uzmu  zlato od tisućnika i satnika te ga donesu u Šator sastanka na  spomen Izraelcima pred Jahvom. 


\chapter{32}

\par 1 Rubenovci i Gadovci imađahu mnogo, vrlo mnogo blaga. Opaze, međutim, da je zemlja jazerska i zemlja gileadska pogodna za  stočarstvo. 
\par 2 Zato Gadovci i Rubenovci dođu k Mojsiju, svećeniku  Eleazaru i glavarima zajednice pa reknu: 
\par 3 "Atarot, Dibon, Jazer, Nimra, Hešbon, Eleale, Sebam, Nebo i Beon - 
\par 4 kraj što ga Jahve  osvoji pred izraelskom zajednicom - kraj je pogodan za stočarstvo;  a sluge tvoje bave se stočarstvom. 
\par 5 Ako smo stekli blagonaklonost  u tvojim očima", nastave, "neka se ovaj kraj dade u posjed tvojim  slugama. Ne šalji nas preko Jordana!" 
\par 6 Mojsije odgovori Gadovcima i Rubenovcima: "Zar da vaša  braća idu u rat, a vi da ostanete ovdje? 
\par 7 Zašto odvraćate srca  Izraelaca da ne prijeđu u zemlju koju im je Jahve predao? 
\par 8 Tako  su učinili i vaši očevi kad sam ih poslao iz Kadeš Barnee da  izvide zemlju. 
\par 9 Popeli su se do Eškola i razgledali zemlju, ali su onda ubili srčanost u Izraelcima da ne odu u zemlju koju  im je Jahve dao. 
\par 10 Onog dana Jahve planu gnjevom. Zakle se  i reče: 
\par 11 'Ljudi što su izišli iz Egipta, kojima je dvadeset  ili više godina, jer me nisu vjerno slijedili, nikad neće vidjeti  zemlju što sam je pod zakletvom obećao Abrahamu, Izaku i Jakovu!' 
\par 12 Jahvu su jedino vjerno slijedili Kenižanin Kaleb, sin Jefuneov, i sin Nunov Jošua. 
\par 13 Jahve je gnjevom planuo na Izraelce pa  ih je pustinjom povlačio četrdeset godina, sve dok ne pomrije  sav naraštaj što je u očima Jahvinim zlo postupio. 
\par 14 A sad  vi - grešni naraštaj - ustajete namjesto svojih očeva da još  povećate srdžbu Jahvinu na Izraela. 
\par 15 Ako se od njega odvratite, on će još produžiti vaš boravak u pustinji; tako ćete upropastiti  sav taj narod." 
\par 16 Onda se oni primaknu k njemu i reknu: "Mi bismo ovdje  podigli torove za svoje blago i gradove za svoju nejačad, 
\par 17 a  sami ćemo pograbiti oružje i poći na čelu Izraelaca dok ih ne  dovedemo na njihovo mjesto. Naša nejačad neka ostane - zbog stanovništva  ove zemlje - u utvrđenim gradovima. 
\par 18 Mi se svojim kućama nećemo  vraćati sve dok svaki Izraelac ne zaposjedne svoju baštinu. 
\par 19 S  njima nećemo dijeliti svoje posjede s onu stranu Jordana niti  dalje, jer će nas zapasti naša baština s ovu stranu, na istok  od Jordana." 
\par 20 Mojsije im reče: "Ako tako uradite, ako pođete pred Jahvom  u boj; 
\par 21 ako vi svi naoružani prijeđete Jordan pred Jahvom  dok on ne rastjera ispred sebe svoje neprijatelje: 
\par 22 tada,  kad zemlja bude pokorena Jahvi, vi ćete se moći vratiti. Tako  ćete biti oslobođeni odgovornosti prema Jahvi i prema Izraelu, a ova će zemlja postati pred Jahvom vaše vlasništvo. 
\par 23 Ali  ako tako ne uradite, sagriješit ćete protiv Jahve i znajte da  će vas stići kazna za vaš grijeh. 
\par 24 Sazidajte, dakle, gradove  za svoju nejačad i torove za svoju stoku, ali izvršite što ste  obećali." 
\par 25 Gadovci i Rubenovci odgovore Mojsiju: "Tvoje će sluge  učiniti kako gospodar naš nalaže. 
\par 26 Naša nejačad, naše žene, naša stoka i sve naše blago neka ostanu ondje u gileadskim gradovima, 
\par 27 a tvoje sluge, svi koji su za boj sposobni, poći će pred  Jahvom u boj, kako naš gospodar nalaže." 
\par 28 Tada za njih Mojsije izda nalog svećeniku Eleazaru, Nunovu  sinu Jošui i glavarima obitelji izraelskih plemena. 
\par 29 I reče  im Mojsije: "Ako Gadovci i Rubenovci, svi oni koji nose oružje, s vama prijeđu Jordan da se bore pred Jahvom i zemlja bude pokorena  vama, onda im dajte gileadsku zemlju u vlasništvo. 
\par 30 Ali ako  ne prijeđu naoružani s vama, neka dobiju baštinu među vama u  zemlji kanaanskoj." 
\par 31 Nato odgovore Gadovci i Rubenovci: "Što je god Jahve  rekao tvojim slugama, to ćemo učiniti. 
\par 32 Mi ćemo naoružani  prijeći pred Jahvom u zemlju kanaansku, ali neka nam bude posjed  naše baštine s ove strane Jordana." 
\par 33 I tako njima - Gadovcima, Rubenovcima i polovici plemena  Manašea, sina Josipova - dadne kraljevstvo amorejskoga kralja  Sihona i kraljevstvo bašanskoga kralja Oga, zemlju s gradovima  u njihovim granicama, gradove okolne zemlje. 
\par 34 Gadovci sagrade: Dibon, Atarot i Aroer, 
\par 35 Atrot Šofan, Jazer, Jogbohu, 
\par 36 Bet Nimru i Bet Haran, utvrđene gradove  i torove za stada. 
\par 37 Rubenovci sagrade: Hešbon, Eleale, Kirjatajim, 
\par 38 Nebo, Baal Meon - nazivi su izmijenjeni - i Šibmu. Oni prozovu svojim  imenima gradove koje su oni podigli. 
\par 39 Sinovi Makira, sina Manašeova, odu u Gilead, osvoje ga  i protjeraju Amorejce koji bijahu ondje. 
\par 40 Mojsije preda Gilead  Manašeovu sinu Makiru, i on se u njemu nastani. 
\par 41 A Manašeov  sin Jair ode te zauzme njihova sela pa ih prozva "Jairova sela". 
\par 42 Potom ode Nobah i zauzme Kenat i njegova područja te ga nazove  svojim imenom "Nobah". 


\chapter{33}

\par 1 Ovo su postaje Izraelaca što ih prijeđoše kad iziđoše iz zemlje  egipatske u svojim četama pod vodstvom Mojsijevim i Aronovim. 
\par 2 Na zapovijed Jahvinu Mojsije je bilježio polazne točke njihova  putovanja. Ovo su njihove postaje prema njihovim polaznim točkama. 
\par 3 Iz Ramsesa krenuše u prvome mjesecu. Bio je petnaesti  dan prvoga mjeseca - sutradan poslije Pashe - kad se Izraelci  zaputiše uzdignutih pesnica i naočigled sviju Egipćana, 
\par 4 dok  su Egipćani pokopavali one koje je Jahve između njih pobio, to  jest sve prvorođence, i tako nad njihovim božanstvima izvršio  pravdu. 
\par 5 Krenu dakle Izraelci iz Ramsesa i utabore se u Sukotu. 
\par 6 Zatim odu iz Sukota i utabore se u Etamu, baš na rubu pustinje. 
\par 7 Pođu iz Etama, a onda okrenu prema Pi Hahirotu, koji se nalazi  nasuprot Baal Sefona. Tabore postave pred Migdolom. 
\par 8 Krenu  od Pi Hahirota i prijeđu posred mora u pustinju. Išli su tri  dana pustinjom Etanom, a onda se utabore u Mari. 
\par 9 Zatim odu  iz Mare i stignu u Elim. U Elimu je bilo dvanaest izvor-voda  i sedamdeset palma. Tu su se utaborili. 
\par 10 Potom krenu iz Elima  te se utabore uz Crveno more. 
\par 11 A otišavši od Crvenog mora, utabore se u pustinji Sinu. 
\par 12 Potom odu iz pustinje Sina i  postave tabore u Dofki. 
\par 13 Otišavši iz Dofke, utabore se u Alušu. 
\par 14 Krenu iz Aluša i utabore se u Refidimu. Tu narod nije imao  vode da pije. 
\par 15 Odu iz Refidima te se utabore u Sinajskoj pustinji. 
\par 16 Krenu iz Sinajske pustinje te se utabore u Kibrot Hataavi. 
\par 17 Odu iz Kibrot Hataave te se utabore u Haserotu. 
\par 18 Onda  odu iz Haserota i utabore se u Ritmi. 
\par 19 Krenu iz Ritme i utabore  se u Rimon Peresu. 
\par 20 Odu iz Rimon Peresa i utabore se u Libni. 
\par 21 Iz Libne odu i utabore se u Risi. 
\par 22 Odu iz Rise te se utabore  u Kehelati. 
\par 23 Odu iz Kehelate i utabore se na brdu Šeferu. 
\par 24 Odu s brda Šefera i utabore se u Haradi. 
\par 25 Odu iz Harade  i utabore se u Makhelotu. 
\par 26 Odu iz Makhelota te se utabore  u Tahatu. 
\par 27 Odu iz Tahata i utabore se u Tarahu. 
\par 28 Iz Taraha  odu i utabore se u Mitki. 
\par 29 Odu iz Mitke i utabore se u Hašmoni. 
\par 30 Iz Hašmone odu i utabore se u Moserotu. 
\par 31 Odu iz Moserota  i utabore se u Bene Jaakanu. 
\par 32 Odu iz Bene Jaakana i utabore  se u Hor Gidgadu. 
\par 33 Odu iz Hor Gidgada i utabore se u Jotbati. 
\par 34 Odu iz Jotbate i utabore se u Abroni. 
\par 35 Iz Abrone odu i  utabore se u Esion Geberu. 
\par 36 Iz Esion Gebera odu i utabore  se u pustinji Sinu, to jest u Kadešu. 
\par 37 Iz Kadeša krenu te se utabore na brdu Horu, na granici  zemlje edomske. 
\par 38 Na zapovijed Jahvinu svećenik se Aron pope na brdo Hor  i tu umre na prvi dan petoga mjeseca, u četrdesetoj godini nakon  izlaska Izraelaca iz egipatske zemlje. 
\par 39 Aronu je bilo stotinu  dvadeset i tri godine kad je preminuo na brdu Horu. 
\par 40 Aradski kralj, Kanaanac, koji je živio u kanaanskom kraju  Negebu, čuo je o dolasku Izraelaca. 
\par 41 S brda Hora odu te se utabore u Salmoni. 
\par 42 Odu iz Salmone  i utabore se u Punonu. 
\par 43 Odu iz Punona i utabore se u Obotu. 
\par 44 Odu iz Obota i utabore se na moapskom području u Ije-Abarimu. 
\par 45 Odu iz Ije-Abarima i utabore se u Dibon Gadu. 
\par 46 Iz Dibon  Gada odu i utabore se u Almon Diblatajimu. 
\par 47 Iz Almon Diblatajima  odu i utabore se na Abarimskim bregovima, pred Nebom. 
\par 48 Odu  s Abarimskih bregova i utabore se na Moapskim poljanama, uz Jordan, nasuprot Jerihonu; 
\par 49 taborovali su uz Jordan od Bet Haješimota  sve do Abel Hašitima na Moapskim poljanama. 
\par 50 Na Moapskim poljanama uz Jordan, nasuprot Jerihonu, Jahve  reče Mojsiju: 
\par 51 "Ovako reci Izraelcima: 'Kad prijeđete preko Jordana  u zemlju kanaansku, 
\par 52 potjerajte ispred sebe sve stanovnike  te zemlje, uništite sve njihove slike; uništite sve njihove salivene  kumire i sve njihove uzvišice porušite. 
\par 53 Onda zaposjednite  zemlju i u njoj se nastanite, jer sam vam je predao da je zaposjednete. 
\par 54 Zemlju razdijelite kockom među svoje rodove. Brojnijem povećajte  dio, a manjem smanjite. Gdje god kocka padne, bilo za koga, neka  je to njegovo, a prema otačkim plemenima dijelite im baštinu. 
\par 55 Ako stanovnike zemlje ispred sebe ne potjerate, onda će oni  koje od njih na životu ostavite biti trnje u vašim očima i bodljike  u vašim bokovima; dosađivat će vam u zemlji u kojoj budete živjeli 
\par 56 i postupit ću s vama kako sam mislio postupiti s njima.'" 


\chapter{34}

\par 1 Jahve reče Mojsiju: 
\par 2 "Izdaj Izraelcima naredbu i reci im:  'Kad uđete u kanaansku zemlju, ta će vam zemlja pripasti u baštinu, zemlja kanaanska sa svojim granicama. 
\par 3 Južna strana protezat će vam se od pustinje Sina uz Edom.  Južna će vam granica početi s kraja Slanog mora na istočnoj strani. 
\par 4 Onda će vam granica skrenuti na jug, prema Akrabimskoj strmini, i nastaviti se preko Sina. Doprijet će na jugu do Kadeš Barnee;  zatim će izaći prema Hasar Adaru i nastaviti se do Asmone. 
\par 5 Od  Asmone granica će skrenuti prema Egipatskom potoku i izaći će  na more. 
\par 6 Zapadna granica bit će vam Veliko more; neka vam je to  granica prema zapadu. 
\par 7 A ovo će vam biti sjeverna granica: od Velikog mora povucite  crtu na brdo Hor; 
\par 8 s brda Hora onda potegnite crtu do ulaza  u Hamat; završetak granice bit će Sedada. 
\par 9 Onda će se granica  protegnuti do Zifrona i završiti u Hasar Enanu. To će vam biti  sjeverna granica. 
\par 10 Za svoju istočnu granicu povucite crtu od Hasar Enana  do Šefama. 
\par 11 Granica će se spuštati od Šefama do Rible, istočno  od Ajina. Odande će se granica spustiti i doprijeti do istočne  obale Kineretskog jezera. 
\par 12 Iza toga spustit će se granica  niz Jordan da završi u Slanome moru. To će biti vaša zemlja sa svojim granicama naokolo.'" 
\par 13 Tada Mojsije naredi Izraelcima: "To je zemlja koju ćete  kockom dobiti u baštinu, a za koju je zapovjedio Jahve da je  dobije devet plemena i polovica jednog plemena. 
\par 14 Jer pleme  Rubenovaca prema svojim porodicama, zatim pleme Gadovaca prema  svojim porodicama već primiše svoju baštinu, kao što je svoju  baštinu primila i polovica plemena Manašeova. 
\par 15 Ta dva plemena  i pol primila su svoje baštine s one strane Jordana, nasuprot  Jerihonu, s istočne strane." 
\par 16 Jahve reče Mojsiju: 
\par 17 "Ovo su imena ljudi koji će vam  zemlju podijeliti: svećenik Eleazar i Nunov sin Jošua; 
\par 18 i  od svakoga plemena uzmi po jednoga glavara za razdiobu zemlje. 
\par 19 Ovo su imena tih ljudi: Kaleb, sin Jefuneov; od plemena Judina; 
\par 20 Šemuel, sin Amihudov, od plemena Šimunova; 
\par 21 Elidad, sin  Kislonov, od plemena Benjaminova; 
\par 22 knez Buki, sin Joglijev, od plemena Danovaca. 
\par 23 Od sinova Josipovih: knez Haniel, sin Efodov, od plemena  Manašeovaca; 
\par 24 knez Kemuel, sin Šiftanov, od plemena Efrajimovaca; 
\par 25 knez Elisafan, sin Parnakov, od plemena Zebulunovaca; 
\par 26 knez  Paltiel, sin Azanov, od plemena Jisakarovaca; 
\par 27 knez Ahihud, sin Šelomijev, od plemena Ašerovaca; 
\par 28 knez Pedahel, sin Amihudov, od plemena Naftalijevaca." 
\par 29 To su oni kojima je Jahve naložio da Izraelcima izdijele  baštinu u zemlji kanaanskoj. 


\chapter{35}

\par 1 Reče Jahve Mojsiju na Moapskim poljanama kod Jordana, nasuprot  Jerihonu: 
\par 2 "Naredi Izraelcima da ustupe levitima od baštine  koju posjeduju gradove gdje će stanovati i pašnjake oko gradova.  To dajte levitima. 
\par 3 Neka gradovi budu njima za stanovanje,  a okolni pašnjaci neka budu za njihova goveda, njihovo blago  i sve njihove životinje. 
\par 4 Pašnjaci uz gradove koje ustupite  levitima neka zahvate od gradskih zidina van do tisuću lakata  naokolo. 
\par 5 Izmjerite od grada van dvije tisuće lakata s istočne  strane, dvije tisuće lakata s južne strane, dvije tisuće lakata  sa zapadne strane i sa sjeverne strane dvije tisuće lakata, tako  da grad bude u sredini. To neka im budu gradski pašnjaci. 
\par 6 Od gradova koje budete dali levitima šest će ih biti gradovi-utočišta, koje ćete ustupiti da ubojica može tamo pobjeći. Ovima dodajte  još četrdeset i dva grada. 
\par 7 Tako će svih gradova koje ustupite  levitima biti četrdeset i osam gradova s njihovim pašnjacima. 
\par 8 A gradove koje budete izdvajali od vlasništva Izraelaca, od  onih koji ih imaju mnogo uzmite više, a manje od onih koji imaju  malo. Neka svatko ustupi gradove levitima prema omjeru baštine  koju bude primio." 
\par 9 Nadalje reče Jahve Mojsiju: 
\par 10 "Govori Izraelcima i reci  im: 'Kad prijeđete preko Jordana u zemlju kanaansku, 
\par 11 označite  sebi gradove koji će vam služiti kao gradovi-utočišta, kamo može  pobjeći ubojica koji nehotice koga ubije. 
\par 12 Ti gradovi neka  vam budu utočište od osvetnika, tako da ubojica ne moradne poginuti  dok ne stane na sud pred zajednicu. 
\par 13 Od gradova koje ustupite  bit će vam šest gradova za utočište. 
\par 14 Dodijelite tri grada  s onu stranu Jordana, a tri grada u zemlji kanaanskoj. Neka to  budu gradovi-utočišta. 
\par 15 Tih šest gradova neka budu za utočište  kako Izraelcima tako i strancu i došljaku koji među njima borave, kamo može pobjeći tko god ubije koga nehotice. 
\par 16 Ali ako tko udari koga gvozdenim predmetom te ga usmrti, to je onda ubojica. Ubojica mora glavom platiti. 
\par 17 Udari li  ga iz ruke kamenom od kojega čovjek može poginuti i zbilja pogine, to je opet ubojica. Ubojica mora glavom platiti. 
\par 18 Ili ako  ga udari iz ruke kakvim drvenim predmetom od kojega može umrijeti  i zbilja umre, i to je ubojica. Ubojica mora glavom platiti. 
\par 19 Krvni osvetnik mora sam ubojicu usmrtiti. Kad ga sretne,  neka ga ubije. 
\par 20 Nadalje, ako tko koga gurne iz mržnje ili na nj nešto  baci namjerno te ga usmrti, 
\par 21 ili ga udari rukom iz zlobe te  udareni umre, napadač mora zaglaviti - on je ubojica. Krvni osvetnik  neka ubojicu ubije čim ga sretne. 
\par 22 No gurne li ga slučajno, ne iz neprijateljstva, ili nešto na nj baci, ali ne iz zasjede, 
\par 23 ili iz nepažnje na njega obori kakav kamen od kojega čovjek  može poginuti te ga usmrti, a nije mu bio neprijatelj niti mu  je zlo želio - 
\par 24 tada neka zajednica prosudi između ubojice  i krvnog osvetnika prema ovim pravilima: 
\par 25 Zajednica mora izbaviti  ubojicu iz ruku krvnog osvetnika; onda neka ga zajednica vrati  u grad-utočište kamo je pobjegao; tu neka on ostane do smrti  velikoga svećenika koji je bio pomazan svetim uljem. 
\par 26 Ali  ako ubojica ikad izađe izvan granice utočišta kamo je pobjegao, 
\par 27 pa na nj nabasa krvni osvetnik izvan granica njegova grada-utočišta  te krvni osvetnik ubije ubojicu, to mu se ne računa u krvoproliće, 
\par 28 jer ubojica mora ostati u gradu-utočištu do smrti velikoga  svećenika. A poslije smrti velikoga svećenika može se vratiti  na svoj posjed. 
\par 29 Neka vam takvi budu sudbeni postupci od naraštaja do  naraštaja svuda gdje budete boravili. 
\par 30 Za svako ubojstvo čovjeka kazna smrti nad ubojicom može  se izvršiti na dokaz svjedoka. Nitko se ne može smrću kazniti  na dokaz samo jednog svjedoka. 
\par 31 Ne smijete primati otkupnine  za život ubojice koji je zaslužio smrt: on mora umrijeti. 
\par 32 Niti  smijete primati otkupnine od bilo koga koji, pošto je pobjegao  u svoj grad-utočište, hoće da se vrati i da živi na svome tlu  prije smrti velikoga svećenika. 
\par 33 Nemojte oskvrnjivati zemlje  u kojoj živite. A krvoprolićem zemlja se oskvrnjuje. Za zemlju  na kojoj je krv prolivena pomirenje se ne može pribaviti, osim  krvlju onoga koji ju je prolio. 
\par 34 Ne smije se obeščašćivati  zemlja u kojoj živite i usred koje ja boravim, jer ja, Jahve, prebivam među sinovima Izraelovim.'" 



\chapter{36}

\par 1 Tada pristupe obiteljski glavari od roda sinova Gileada, sina  Makirova, sina Manašeova, jednoga roda Josipovih sinova, te pred  Mojsijem i starješinama, glavarima obitelji, 
\par 2 reknu: "Jahve je naredio našemu gospodaru da kockom dade ovu zemlju  u baštinu Izraelcima; nadalje, našem je gospodaru naredio Jahve  da baštinu našega brata Selofhada dade njegovim kćerima. 
\par 3 Ali  ako se one udaju za koga iz drugog izraelskoga plemena, onda  će njihova baština biti otrgnuta od naše djedovske baštine i  biti priključena baštini plemena kojemu one pripadnu. Tako će  se okrnjiti baština koja kockom pripadne nama. 
\par 4 A kada nastupi  jubilej Izraelcima, baština će se tih žena dodati baštini plemena  kojemu pripadnu. Tako će njihova baština biti oduzeta od baštine  našega pradjedovskog plemena." 
\par 5 I po zapovijedi Jahvinoj Mojsije naredi Izraelcima: "Pleme  Josipovih sinova pravo govori. 
\par 6 Ovo naređuje Jahve za Selofhadove  kćeri: Neka se one udaju za onoga koji im se učini dobar, samo  neka se udaju u rod svoga očinskoga plemena. 
\par 7 Baština Izraelaca  ne smije se prenositi iz jednoga plemena u drugo; i svaki Izraelac  mora ostati privezan uz pradjedovsku baštinu svoga plemena. 
\par 8 Zato  se svaka djevojka koja steče baštinu u izraelskim plemenima mora  udati za nekoga u plemenu kojemu pripada rod joj očev, tako da  bi svaki Izraelac sačuvao baštinu svoga oca. 
\par 9 Tako se baština  neće prenositi iz jednoga plemena u drugo, nego će svako izraelsko  pleme prianjati uza svoju baštinu." 
\par 10 Kako je Jahve Mojsiju naredio, tako su i učinile kćeri  Selofhadove: 
\par 11 Mahla, Tirsa, Hogla, Milka i Noa, kćeri Selofhadove, udaše se za sinove svojih stričeva. 
\par 12 Kako su se udale u rod  potomstva Manašea, Josipova sina, njihova je baština ostala u  plemenu kojemu pripadaše rod im očev. 
\par 13 To su zapovijedi i zakoni koje je Jahve preko Mojsija  izdao Izraelcima na Moapskim poljanama uz Jordan, nasuprot Jerihonu. 





\end{document}