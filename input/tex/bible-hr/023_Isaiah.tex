\begin{document}

\title{Izaija}


\chapter{1}

\par 1 Viđenje Izaije, sina Amosova, koje je imao o Judeji i Jeruzalemu  u dane Uzije, Jotama, Ahaza i Ezekije, kraljeva judejskih. 
\par 2 Čujte, nebesa, poslušaj, zemljo, jer Jahve govori: "Sinove sam ti odgojio, podigao, al' se oni od mene odvrgoše. 
\par 3 Vo poznaje svog vlasnika, a magarac jasle gospodareve - Izrael ne poznaje, narod moj ne razumije." 
\par 4 Jao, grešna li naroda, puka u zlu ogrezla, roda zlikovačkog, pokvarenih sinova! Jahvu ostaviše, prezreše Sveca Izraelova, njemu su okrenuli leđa. 
\par 5 TÓa gdje da vas još udarim, odmetnici tvrdokorni? Sva je glava bolna, srce iznemoglo; 
\par 6 od pete do glave nidje zdrava mjesta, već ozljede, modrice, otvorene rane, ni očišćene, ni povijene, ni uljem ublažene. 
\par 7 Zemlja vam opustje, gradove oganj popali, njive vam na oči haraju tuđinci - pustoš k'o kad propade Sodoma. 
\par 8 Kći sionska ostade kao koliba u vinogradu, kao pojata u polju krastavaca, kao grad opsjednut. 
\par 9 Da nam Jahve nad Vojskama ne ostavi Ostatak, bili bismo k'o Sodoma, Gomori slični. 
\par 10 Čujte riječ Jahvinu, glavari sodomski, poslušaj zakon Boga našega, narode gomorski! 
\par 11 "Što će mi mnoštvo žrtava vaših?" - govori Jahve. - "Sit sam ovnujskih paljenica i pretiline gojne teladi. I krv mi se ogadi bikova, janjaca i jaradi. 
\par 12 Kad mi lice vidjet' dolazite, tko od vas ište da gazite mojim predvorjima? 
\par 13 Prestanite mi nositi ništavne prinose, kad mi omrznu. Mlađaka, subote i sazive - ne podnosim zborovanja i opačine. 
\par 14 Mlađake i svetkovine vaše iz sve duše mrzim - teški su mi, podnijet' ih ne mogu! 
\par 15 Kad na molitvu ruke širite, je od vas oči odvraćam. Molitve samo množite, ja vas ne slušam. Ruke su vam u krvi ogrezle, 
\par 16 operite se, očistite. Uklonite mi s očiju djela opaka, prestanite zlo činiti! 
\par 17 Učite se dobrim djelima: pravdi težite, ugnjetenom pritecite u pomoć, siroti pomozite do pravde, za udovu se zauzmite." 
\par 18 "Hajde, dakle, da se pravdamo," govori Jahve. "Budu l' vam grijesi kao grimiz, pobijeljet će poput snijega; kao purpur budu li crveni, postat će kao vuna. 
\par 19 Htjednete l' me poslušati, uživat ćete plodove zemaljske. 
\par 20 U buntovništvu ako ustrajete, proždrijet će vas mač." Tako usta Jahvina govorahu. 
\par 21 Kako li posta bludnicom tvrđa vjerna? Bješe puna pravičnosti, pravda u njoj stolovala, a sad - ubojice. 
\par 22 Srebro ti se u trosku obratilo, vino ti se razvodnjelo. 
\par 23 Knezovi se tvoji odmetnuli, s tatima se pobratili. Svi za mitom hlepe, za darovima lete. Siroti pravdu uskraćuju, udovička parnica ne stiže k njima. 
\par 24 Stog ovako govori Jahve, Gospod nad Vojskama, Junak Izraelov: "Ah, kad se iskalim na protivnicima i osvetim dušmanima! 
\par 25 Kada na te ruku pružim, da lužinom tvoju trosku očistim, da iz tebe uklonim olovo! 
\par 26 Da ti opet postavim suce kao negda, savjetnike kao u početku, pa da te zovu Gradom pravednim, Tvrđom vjernosti." 
\par 27 Sud pravedni otkupit će Sion, a pravda obraćenike njegove. 
\par 28 Otpadnici i grešnici skršit će se zajedno, a oni što Jahvu napuštaju poginut će. 
\par 29 Da, stidjet ćete se zbog hrastova što ih sad obožavate i crvenjet ćete zbog gajeva u kojima sad uživate. 
\par 30 Jer, bit ćete poput hrasta osušena lišća i poput gaja u kojem vode nema. 
\par 31 Junak će biti kučina, a iskra djelo njegovo, zajedno će izgorjeti, a nikoga da ugasi. 


\chapter{2}

\par 1 Viđenje Izaije, sina Amosova, o Judeji i Jeruzalemu: 
\par 2 Dogodit će se na kraju dana: Gora Doma Jahvina bit će postavljena vrh svih gora, uzvišena iznad svih bregova. K njoj će se stjecati svi narodi, 
\par 3 nagrnut će mnoga plemena i reći: "Hajde, uziđimo na Goru Jahvinu, pođimo u Dom Boga Jakovljeva! On će nas naučiti svojim putovima, hodit ćemo stazama njegovim. Jer će iz Siona Zakon doći, iz Jeruzalema riječ Jahvina." 
\par 4 On će biti sudac narodima, mnogim će sudit' plemenima, koji će mačeve prekovati u plugove, a koplja u srpove. Neće više narod dizat' mača protiv naroda nit' se više učit' ratovanju. 
\par 5 Hajde, dome Jakovljev, u Jahvinoj hodimo svjetlosti! 
\par 6 Da, ti si svoj odbacio narod, dom Jakovljev, jer je pun vračeva s istoka i gatara kao Filistejci, bratime se s tuđincima. 
\par 7 Zemlja mu je puna srebra i zlata i blagu mu kraja nema; zemlja mu je puna konja, kolima mu broja nema. 
\par 8 Zemlja mu je prepuna kumira i oni se klanjaju pred djelom ruku svojih, pred onim što njihovi načiniše prsti. 
\par 9 Smrtnik će se poviti, čovjek sniziti; ne praštaj im. 
\par 10 Uđi među pećine, skrij se u prašinu, pred užasom Jahvinim, pred sjajem veličanstva njegova, kad ustane da potrese zemlju. 
\par 11 Ohol pogled bit će skršen i bahatost ljudska ponižena. Jahve će se uzvisiti, on jedini - u dan onaj. 
\par 12 Da, bit će to dan Jahve nad Vojskama, protiv svih oholih i bahatih, protiv sviju što se uzvisiše, da ih obori; 
\par 13 protiv svih cedrova libanonskih i svih hrastova bašanskih; 
\par 14 protiv svih gora uznositih i svih bregova uzdignutih; 
\par 15 protiv svake visoke tvrđe i svih tvrdih zidina; 
\par 16 protiv sveg brodovlja taršiškog i svih brodova raskošnih. 
\par 17 Oholost ljudska skršit će se i bahatost ljudska poniziti. Jahve će se uzvisiti, on jedini - u dan onaj, 
\par 18 i kumiri će netragom nestati. 
\par 19 Uđite u rupe među pećinama i u spilje zemaljske pred užasom Jahvinim, pred sjajem veličanstva njegova, kad ustane da potrese zemlju. 
\par 20 U dan onaj: bacit će svaki svoje srebro i zlatne kumire koje sebi načini da im se klanja, 
\par 21 kad uteče u šupljine pećina i u raspukline stijena pred užasom Jahvinim, pred sjajem veličanstva njegova, kad ustane da potrese zemlju. 
\par 22 Čuvajte se, dakle, čovjeka koji ima samo jedan dah u nosnicama: jer što vrijedi? 


\chapter{3}

\par 1 Gle, Gospod, Jahve nad Vojskama, oduzima Jeruzalemu i Judeji svaku potporu, pomoć u kruhu i pomoć u vodi, 
\par 2 junaka i ratnika, suca i proroka, vrača i starješinu, pedesetnika i odličnika, 
\par 3 savjetnika i mudra gatara i onoga što se bavi čaranjem. 
\par 4 "A za glavare postavljam im djecu, dajem deranima da njima vladaju." 
\par 5 Ljudi se glože jedan s drugim i svaki s bližnjim svojim; dijete nasrće na starca, prostak na odličnika 
\par 6 te svatko brata hvata u očinskoj kući: "Ti imaš plašt, budi nam glavarom, uzmi u ruke ovo rasulo!" 
\par 7 A on će se, u dan onaj, braniti: "Neću da budem vidar, nema u mene ni kruha ni plašta: ne stavljajte me narodu za glavara." 
\par 8 Jeruzalem se ruši i pada Judeja, jer im se jezik i djela Jahvi protive te prkose pogledu Slave njegove. 
\par 9 Lice njihovo protiv njih svjedoči, razmeću se grijehom poput Sodome i ne kriju ga, jao njima, sami sebi propast spremaju. 
\par 10 Kažite: "Blago pravedniku, hranit će se plodom djela svojih! 
\par 11 Jao opakome, zlo će mu biti, na nj će pasti djela ruku njegovih." 
\par 12 Deran tlači narod moj i žene njime vladaju. O narode moj, vladaoci te tvoji zavode i raskapaju put kojim hodiš. 
\par 13 Ustade Jahve da se popravda s narodom svojim, 
\par 14 Jahve dolazi na sud sa starješinama i glavarima svog naroda: "Vinograd ste moj opustošili, u vašim je kućama što oteste siromahu. 
\par 15 S kojim pravom narod moj tlačite i gazite lice siromaha?" - riječ je Jahve, Gospoda nad Vojskama. 
\par 16 I reče Jahve: "Što se to ohole kćeri sionske te ispružena vrata hode, okolo okom namiguju, koracima sitnim koracaju, grivnama na nozi zveckaju? 
\par 17 Oćelavit će Gospod tjeme kćeri sionskih, obnažit će Jahve golotinju njihovu." 
\par 18 U onaj će dan Gospod strgnuti sve čime se ona ponosi:  ukosnice i mjesečiće, 
\par 19 naušnice, narukvice i koprene, 
\par 20 poveze, lančiće, pojaseve, bočice s miomirisima i privjese, 
\par 21 prstenje  i nosne prstenove, 
\par 22 skupocjene haljine i plašteve, prijevjese  i torbice, 
\par 23 zrcala i košuljice, povezače i rupce. 
\par 24 Mjesto miomirisa, smrad; mjesto pojasa, konopac; mjesto kovrča, tjeme obrijano; mjesto gizdave halje, kostrijet; mjesto ljepote, žig. 
\par 25 Muževi tvoji od mača će pasti, junaci tvoji u kreševu. 
\par 26 Vrata će tvoja kukat' i tugovati, na zemlji ćeš sjedit' napuštena. 


\chapter{4}

\par 1 I sedam će se žena jagmiti za jednoga čovjeka - u dan onaj: "Svoj ćemo kruh jesti," reći će, "i u halje se svoje oblačiti, daj nam samo da tvoje nosimo ime, skini sa nas svu sramotu našu." 
\par 2 U onaj će dan izdanak Jahvin biti na diku i na slavu, a plod zemlje na ponos i ures spašenima u Izraelu. 
\par 3 Koji ostanu na Sionu i prežive u Jeruzalemu, zvat će se "sveti" i bit će upisani da u Jeruzalemu žive. 
\par 4 Kad Gospod spere ljagu kćeri sionskih i obriše s Jeruzalema krv prolivenu dahom suda i dahom što spaljuje, 
\par 5 sazdat će Jahve nad svom Gorom sionskom i nad svima što ondje budu zborovali oblak s dimom danju, a noću sjaj ognja žarkoga. Jer, vrh svega Slava će biti zaklon 
\par 6 i sjenica da zasjenjuje danju od pripeke, štit i utočište od pljuska i oluje. 


\chapter{5}

\par 1 Zapjevat ću svojemu dragome, pjesmu svog ljubljenog njegovu vinogradu. Moj je dragi imao vinograd na brežuljku rodnome. 
\par 2 Okopa ga, iskrči kamenje, posadi ga lozom plemenitom. Posred njega kulu on podiže i u nj tijesak metnu. Nadaše se da će uroditi grožđem, a on izrodi vinjagu. 
\par 3 Sad, žitelji jeruzalemski i ljudi Judejci, presudite izmeđ' mene i vinograda mojega. 
\par 4 Što još mogoh učiniti za svoj vinograd a da nisam učinio? Nadah se da će uroditi grožđem, zašto vinjagu izrodi? 
\par 5 No sad ću vam reći što ću učiniti od svog vinograda: plot ću mu soriti da ga opustoše, zidinu razvaliti da ga izgaze. 
\par 6 U pustoš ću ga obratiti, ni obrezana ni okopana, nek' u drač i trnje sav zaraste; zabranit ću oblacima da dažde nad njime. 
\par 7 Vinograd Jahve nad Vojskama dom je Izraelov; izabrani nasad njegov ljudi Judejci. Nadao se pravdi, a eto nepravde, nadao se pravičnosti, a eto vapaja. 
\par 8 Jao vama koji kuću kući primičete i polje s poljem sastavljate, dok sve mjesto ne zauzmete te postanete jedini u zemlji. 
\par 9 Na uši moje reče Jahve nad Vojskama: "Doista, mnoge će kuće opustjeti, velike i lijepe, bit će bez žitelja. 
\par 10 Deset rali vinograda dat će samo bačvicu, mjera sjemena dat će samo mjericu." 
\par 11 Jao onima što već jutrom na uranku žestokim se pićem zalijevaju i kasno noću sjede vinom raspaljeni. 
\par 12 Na gozbama im harfe i citare, bubnjevi i frule uz vino, a za djelo Jahvino ne mare, ne gledaju djelo ruku njegovih. 
\par 13 Stoga će u ropstvo narod moj odvesti, jer nema razumnosti, odličnici njegovi od gladi će umirati, puk njegov od žeđi će gorjeti. 
\par 14 Da, Podzemlje će razvaliti ždrijelo, razjapit će ralje neizmjerne da se u njih strmoglave odličnici mu i mnoštvo sa svom grajom i veseljem! 
\par 15 Smrtnik će nikom poniknuti, ponizit' se čovjek, oborit će se pogled silnih. 
\par 16 Jahve nad Vojskama uzvisit će se sudom, i Bog će sveti otkrit' svetost svoju. 
\par 17 Jaganjci će pasti kao na pašnjacima, a jarci će brstiti po ruševinama bogataškim. 
\par 18 Jao onima koji na se krivnju vuku volovskom užadi i grijeh kolskim konopcem - 
\par 19 onima što govore: "Neka pohiti, neka poteče s djelom svojim da bismo ga vidjeli, neka se približi i završi naum Sveca Izraelova da bismo znali!" 
\par 20 Jao onima koji zlo dobrom nazivaju, a dobro zlom, koji od tame svjetlost prave, a od svjetlosti tamu, koji gorko slatkim čine, a slatko gorkim! 
\par 21 Jao onima koji su mudri u svojim očima i pametni sami pred sobom! 
\par 22 Jao onima koji su jaki u vinu i junaci u miješanju jakih pića; 
\par 23 onima koji za mito brane krivca, a pravedniku uskraćuju pravdu! 
\par 24 Zato, kao što plameni jezici proždiru slamu i kao što nestaje suha trava u plamenu, tako će korijen njihov istrunuti, poput praha razletjet' se pupoljak njihov, jer odbaciše Zakon Jahve nad Vojskama i prezreše riječ Sveca Izraelova. 
\par 25 Zato se raspali gnjev Jahvin protiv njegova naroda, i on diže ruku na nj i udari ga te se potresoše gore: trupla njihova leže k'o smeće po ulicama, ali gnjev se njegov još ne smiri, ruka mu je sveđer podignuta. 
\par 26 On podiže stijeg ratni narodu dalekom, zazviždi mu na kraj zemlje, i gle: brzo, spremno hita. 
\par 27 U njemu nema trudna ni sustala, ni dremljiva niti snena, oko boka pojas ne otpasuje, na obući ne driješi remena. 
\par 28 Strijele su mu dobro zašiljene, lukovi mu svi zapeti, kremen su kopita konja njegovih, vihor su točkovi bojnih mu kola. 
\par 29 Rika mu je k'o u lava i riče k'o lavovi mladi, reži, grabi plijen i odnosi, a nikoga da mu ga istrgne. 
\par 30 U dan onaj režat će na njega k'o što more buči. Pogledaš li zemlju - sve tmina, tjeskoba, svjetlost proguta tmina oblačna. 


\chapter{6}

\par 1 One godine kad umrije kralj Uzija, vidjeh Gospoda gdje sjedi  na prijestolju visoku i uzvišenu. Skuti njegova plašta ispunjahu  Svetište. 
\par 2 Iznad njega stajahu serafi; svaki je imao po šest  krila: dva krila da zakloni lice, dva da zakrije noge, a dvama  je krilima letio. 
\par 3 I klicahu jedan drugome: "Svet! Svet! Svet Jahve nad Vojskama! Puna je sva zemlja Slave njegove!" 
\par 4 Od gromka glasa onih koji klicahu stresoše se dovraci  na pragovima, a Dom se napuni dimom. 
\par 5 Rekoh: "Jao meni, propadoh, jer čovjek sam nečistih usana, u narodu nečistih usana prebivam, a oči mi vidješe Kralja, Jahvu nad Vojskama!" 
\par 6 Jedan od serafa doletje k meni: u ruci mu žerava koju  uze kliještima sa žrtvenika; 
\par 7 dotače se njome mojih usta i  reče:  "Evo, usne je tvoje dotaklo, krivica ti je skinuta i grijeh oprošten." 
\par 8 Tad čuh glas Gospodnji: "Koga da pošaljem? I tko će nam poći?" Ja rekoh: "Evo me, mene pošalji!" 
\par 9 On odgovori: "Idi i reci tom narodu: 'Slušajte dobro, al' nećete razumjeti, gledajte dobro, al' nećete spoznati.' 
\par 10 Otežaj salom srce tom narodu, ogluši mu uši, zaslijepi oči, da očima ne vidi, da ušima ne čuje i srcem da ne razumije kako bi se obratio i ozdravio." 
\par 11 Ja rekoh: "Dokle, o Gospode?" On mi odgovori: "Dok gradovi ne opuste i ne ostanu bez žitelja, dok kuće ne budu bez ikoga živa, i zemlja ne postane pustoš, 
\par 12 dok Jahve daleko ne protjera ljude. Haranje veliko pogodit će zemlju, 
\par 13 i ostane li u njoj još desetina, i ona će biti zatrta poput duba kad ga do panja posijeku. Panj će njihov biti sveto sjeme." 


\chapter{7}

\par 1 U dane judejskoga kralja Ahaza, sina Jotamova, sina Uzijina, aramski kralj Rason i izraelski kralj Pekah, sin Remalijin,  zavojštiše na Jeruzalem, ali ga ne mogoše zauzeti. 
\par 2 Tada dojaviše domu Davidovu: "Aramci se utaborili u Efrajimu."  Na tu vijest uzdrhta srce kraljevo i srce svega naroda, kao što  u šumi drveće ustrepti od vjetra. 
\par 3 I Jahve reče Izaiji: "Iziđi  pred Ahaza, ti i sin tvoj Šear Jašub, do nakraj vodovoda gornjeg  ribnjaka na putu u Valjarevo polje. 
\par 4 Reci mu: 'Pazi, smiri se, ne boj se, i nek' ti ne premire srce od ovih dvaju ugaraka zadimljenih, od raspaljenog bijesa Rasona aramskog i sina Remalijina, 
\par 5 jer Aram, Efrajim i sin Remalijin smisliše tvoju propast.' 
\par 6 Pođimo, rekoše, na Judeju, uplašimo je i osvojimo da u njoj zakraljimo sina Tabelova. 
\par 7 Ovako govori Jahve Gospod: 'To se neće zbiti: toga biti neće! 
\par 8 [8a] Damask je glava Aramcima, a Damasku je glava Rason; 
\par 9 Samarija je glava Efrajimcima, a Samariji glava je Remalija. [8b]Još šezdeset i pet godina, i Efrajim, razoren, neće više biti narod. [9b] Ako se na me ne oslonite, održat' se nećete!" 
\par 10 Jahve opet progovori Ahazu i reče mu: 
\par 11 "Zaišti od  Jahve, Boga svoga, jedan znak za sebe iz dubine Podzemlja ili  gore iz visina." 
\par 12 Ali Ahaz odgovori: "Ne, neću iskati i neću  iskušavati Jahvu." 
\par 13 Tada reče Izaija: "Čujte, dome Davidov. Zar vam je malo dodijavati ljudima, pa i Bogu mom dodijavate! 
\par 14 Zato, sam će vam Gospodin dati znak: Evo, začet će djevica i roditi sina i nadjenut će mu ime Emanuel! 
\par 15 Vrhnjem i medom on će se hraniti dok ne nauči odbacivat' zlo i birati dobro. 
\par 16 Jer prije nego dječak nauči odbacivat' zlo i birati dobro, opustjet će zemlja, zbog koje strepiš, od dvaju kraljeva. 
\par 17 Protiv tebe i protiv tvog naroda i protiv kuće oca tvojega dovest će Jahve dane kakvih ne bijaše otkad se Efrajim odvoji od Jude - kralja asirskoga. 
\par 18 U dan onaj zazviždat će Jahve muhama na ušću egipatskih rijeka i pčelama u zemlji asirskoj 
\par 19 da dođu i popadaju po strmim dolovima, po rasjelinama stijena, po svim trnjacima i svim pojilištima. 
\par 20 U dan onaj Gospod će obrijati britvom najmljenom s onu stranu Eufrata - kraljem asirskim - kosu s glave, dlake s nogu i bradu s obraza. 
\par 21 U dan onaj svatko će hraniti po kravu i dvije ovce 
\par 22 i od obilja mlijeka koje će mu dati hranit će se vrhnjem; vrhnjem i medom hranit će se koji god u zemlji preostanu. 
\par 23 U dan onaj gdje god bijaše tisuću čokota, vrijednih tisuću srebrnika, izrast će drač i trnje. 
\par 24 Onamo će polaziti sa strijelom i lukom, jer sva će zemlja u drač i trnje zarasti. 
\par 25 A po svim gorama gdje se motikom kopalo nitko više neće ići, strašeći se trnja i drača: onuda će goveda pasti i gaziti ovce." 


\chapter{8}

\par 1 Reče Jahve: "Uzmi veliku ploču i napiši na njoj ljudskim pismom:  Maher Šalal Haš Baz - Brz grabež - hitar plijen." 
\par 2 Potom uzeh  vjerne svjedoke, svećenika Uriju i Zahariju, sina Berekjina. 
\par 3 Približih se proročici te ona zače i rodi sina. Jahve  mi reče: "Nazovi ga Maher Šalal Haš Baz, 
\par 4 jer prije no što  dijete počne tepati 'tata' i 'mama', nosit će se pred kralja  asirskog sve bogatstvo Damaska i plijen Samarije." 
\par 5 I opet mi reče Jahve: 
\par 6 "Jer narod ovaj odbacuje mirne tekućice Šiloaha, a dršće pred Rasonom i pred sinom Remalijinim, 
\par 7 navest će Gospod na vas vodu Eufrata, silnu i veliku - kralja asirskog i svu slavu njegovu - i ona će izići iz rukava svoga, preliti se preko svih obala; 
\par 8 provalit će u Judeju, razlit' se i poplaviti je, popeti se do grla njezina; i krila će svoja raširiti preko cijele tvoje zemlje, o Emanuele." 
\par 9 Udružite se samo, narodi, al' bit ćete smrvljeni! Poslušajte, vi kraljevi daleki, pašite se, bit ćete smrvljeni, pašite se, bit ćete smrvljeni! 
\par 10 Kujte naum - bit će uništen, dogovarajte se samo, bit će uzalud, jer s nama je Bog! 
\par 11 Jer, ovako mi reče Jahve, kad me rukom uhvatio i opomenuo da ne idem putem kojim narod ovaj ide: 
\par 12 "Ne zovite urotom sve što narod ovaj urotom zove; ne bojte se čega se on boji i nemajte straha. 
\par 13 Jahve nad Vojskama - on jedini nek' vam svet bude; jedino se njega bojte, strah od njega nek' vas prožme. 
\par 14 On će vam biti zamka i kamen spoticanja i stijena posrtanja za obje kuće Izraelove, zamka i mreža svim Jeruzalemcima. 
\par 15 Mnogi će od njih posrnuti, pasti, razbiti se, zaplesti se, uhvatiti." 
\par 16 Pohrani ovo svjedočanstvo, zapečati ovu objavu među učenicima svojim: 
\par 17 Čekat ću Jahvu koji je lice svoje sakrio od doma Jakovljeva - u njega ja se uzdam. 
\par 18 Evo, ja i djeca koju mi Jahve dade Izraelu smo znak i znamenje od Jahve nad Vojskama što prebiva na Gori sionskoj. 
\par 19 Reknu li vam: "Duhove pitajte i vrače koji šapću i mrmljaju" - dakako, narod mora pitati svoje "bogove" i za žive u mrtvih se raspitivati! - 
\par 20 Uza Zakon! Uza svjedočanstvo! Tko ne rekne tako, zoru neće dočekati. 
\par 21 Lutat će zemljom potlačen i gladan, izgladnjela bijes će ga spopasti, proklinjat će svoga kralja i svog Boga. Okrene l' se k nebu, 
\par 22 pogleda l' po zemlji, gle, svuda samo mrak i strava, svuda tmina tjeskobna. Ali će se tama raspršiti, 


\chapter{9}

\par 1 (8:23) Jer više neće biti mraka gdje je sad tjeskoba. (8:24) U prvo vrijeme on obescijeni zemlju Zebulunovu i zemlju Naftalijevu, al' u vrijeme posljednje on će proslaviti Put uz more, s one strane Jordana - Galileju pogansku. 
\par 2 (9:1) Narod koji je u tmini hodio svjetlost vidje veliku; one što mrklu zemlju obitavahu svjetlost jarka obasja. 
\par 3 (9:2) Ti si radost umnožio, uvećao veselje, i oni se pred tobom raduju kao što se ljudi raduju žetvi, k'o što kliču kad se dijeli plijen. 
\par 4 (9:3) Teški jaram njegov, prečku što mu pleća pritiskaše, šibu njegova goniča slomi kao u dan midjanski. 
\par 5 (9:4) Da, sva bojna obuća, svaki plašt krvlju natopljen izgorjet će i bit će ognju hrana. 
\par 6 (9:5) Jer, dijete nam se rodilo, sina dobismo; na plećima mu je vlast. Ime mu je: Savjetnik divni, Bog silni, Otac vječni, Knez mironosni. 
\par 7 (9:6) Nadaleko vlast će mu se sterat' i miru neće biti kraja nad prijestoljem Davidovim, nad kraljevstvom njegovim: učvrstit će ga i utvrdit u pravu i pravednosti. Od sada i dovijeka učinit će to privržena ljubav Jahve nad Vojskama. 
\par 8 (9:7) Gospod posla riječ protiv Jakova i ona pade na Izraela. 
\par 9 (9:8) Sazna je sav narod njegov, Efrajim i stanovnici Samarije koji govorahu naduta i ohola srca: 
\par 10 (9:9) "Opeke nam popadaše, gradit ćemo od tesanika; sasjekoše nam divlje smokve, cedre ćemo posaditi." 
\par 11 (9:10) Al' Jahve podiže na brdo Sion njegove protivnike i podbada neprijatelje njegove: 
\par 12 (9:11) Aram s istoka, Filistejce sa zapada, da svim ustima proždiru Izraela. Na sve to gnjev se njegov neće smiriti, ruka će mu ostat' ispružena. 
\par 13 (9:12) Ali se narod nije obratio onom koji ga je b§io, ne tražiše Jahvu nad Vojskama. 
\par 14 (9:13) Zato Jahve odsiječe Izraelu glavu i rep, palmu i rogoz u jednom danu. 
\par 15 (9:14) Starješina i odličnik - to je glava; prorok, učitelj laži - to je rep. 
\par 16 (9:15) Oni što vode narod taj - zavode ga, a koji se vodit' daju - propali su. 
\par 17 (9:16) Stog' mu Gospod neće poštedjet' mladića, sirotama njegovim i udovicama smilovat' se neće. Sav je taj narod bezbožan i zao, na sva usta bezumno govori. Na sve to gnjev se njegov neće smiriti, ruka će mu ostat' ispružena. 
\par 18 (9:17) Da, bezbožnost se k'o oganj razmahala, drač i trnje proždire, pa upali šumsku guštaru, stupovi se dima podižu. 
\par 19 (9:18) Plamti zemlja od gnjeva Jahvina, narod ognju hrana postaje. Nitko ni brata svog ne štedi, [19b] svatko jede meso svog susjeda. 
\par 20 (9:19) [19a] Proždire zdesna, i opet je gladan; guta slijeva, i opet sit nije: 
\par 21 (9:20) Manaše Efrajima, Efrajim Manašea, obojica zajedno Judu. Na sve to gnjev se njegov neće smiriti, ruka će mu ostat' ispružena. 


\chapter{10}

\par 1 Jao onima koji izdaju odredbe nepravedne,  koji ispisuju propise tlačiteljske; 
\par 2 koji uskraćuju pravdu ubogima i otimlju pravo sirotinji mog naroda, da oplijene udovice, da opljačkaju sirote! 
\par 3 Što ćete činiti u dan kazne kad izdaleka propast dođe? Kom ćete se za pomoć uteći, gdje ostaviti blago svoje 
\par 4 da se ne morate među roblje pognuti, pasti među poklanima? Na sve to gnjev se njegov neće smiriti, ruka će mu ostat' ispružena. 
\par 5 Jao Asiru, šibi gnjeva mojega, prutu kojim srdžba moja zamahuje! 
\par 6 Na puk ga poslah nevjeran, na narod što me razjari, da ga oplijeni i opljačka, da ga izgazi k'o blato na ulici. 
\par 7 Ali on nije tako mislio i u srcu nije tako sudio, već u srcu zasnova zator, istrebljenje mnogih naroda. 
\par 8 Govoraše: "Nisu li svi knezovi moji kraljevi? 
\par 9 Nije li Kalno kao Karkemiš? Nije li Hamat kao Arpad, Samarija kao Damask? 
\par 10 Kao što mi ruka dosegnu kraljevstva kumira, bogatija kipovima od Jeruzalema i Samarije, 
\par 11 kao što učinih sa Samarijom i kumirima njenim, neću li učiniti s Jeruzalemom i s likovima njegovim?" 
\par 12 I kad dovrši Gospod sve djelo svoje na gori Sionu i u  Jeruzalemu, kaznit će plod ohola srca kralja asirskog i drskost  njegovih ponositih očiju. 
\par 13 Jer reče: "Učinih snagom svoje ruke i mudrošću svojom, jer sam uman; uklonih međe narodima i blaga njihova opljačkah, kao junak oborih one što sjede na prijestoljima. 
\par 14 Ruka moja kao gnijezda zgrabi bogatstva naroda. Kao što se kÓupe ostavljena jaja, zemlju svu sam pokupio i nikog ne bi krilima da zalepeće, kljun otvori, zapijuče." 
\par 15 Zar se hvali sjekira povrh onog koji njome siječe? Hoće li se oholit' pila povrh onog koji njome pili? K'o da šiba maše onim koji je podiže, il' štap diže onog koji drvo nije; 
\par 16 Jahve nad Vojskama poslat će stoga gojaznima njegovim skončanje, slavu će mu ognjem potpaliti, kao što se vatra potpiruje. 
\par 17 Svjetlost Izraelova bit će poput ognja, Svetac njegov kao plamen koji će zapalit' i proždrijeti drač njegov i trnje njegovo u jednome danu. 
\par 18 I krasotu njegovih šuma i voćnjaka uništit će od srčike do kore, ona će biti k'o bolesnik što se trne; 
\par 19 ostatak stabala šumskih bit će lako izbrojiti - dijete će ih lako popisati. 
\par 20 U onaj dan: Ostatak Izraelov i preživjeli iz kuće Jakovljeve neće se više oslanjati na onoga koji ih bije, već će se iskreno oslanjati na Jahvu, Sveca Izraelova. 
\par 21 Ostatak će se vratiti, ostatak Jakovljev Bogu jakome. 
\par 22 Zaista, o Izraele, sve da naroda tvojega ima kao pijeska  u moru, samo će se Ostatak njegov vratiti. Određeno je uništenje, pravda se prelila, 
\par 23 Jahve, Gospod nad Vojskama, poharat će, kako odredi, svu zemlju. 
\par 24 Zato ovako govori Jahve nad Vojskama: "O narode moj što prebivaš na Sionu, ne boj se Asira kad te šibom tuče, kad štap diže na tebe. 
\par 25 Jer, još malo, vrlo malo, i gnjev moj će prestati, moja će ih jarost uništiti." 
\par 26 Na nj će Jahve nad Vojskama bičem zamahnuti, kao kad udari Midjan na stijeni Orebu, i štap će dići nad more k'o na putu egipatskom. 
\par 27 U onaj dan: s leđa će ti breme pasti i jaram njegov s vrata će ti nestat'. 
\par 28 Ide on od Rimona, dolazi na Ajat, prelazi Migron, u Mikmasu breme odlaže. 
\par 29 Prelaze klance, u Gebi im je noćište; Rama dršće, Gibea Šaulova bježi. 
\par 30 Viči iza glasa, Bat Galime! Slušaj ga, Lajšo! Odgovori mu, Anatote! 
\par 31 Madmena pobježe, utekoše stanovnici gebimski. 
\par 32 Još danas zaustavit će se u Nobu, rukom prijeti gori Kćeri sionske, Brijegu jeruzalemskom. 
\par 33 Gle, Jahve, Gospod nad Vojskama, kreše grane silom strahovitom: najviši su vršci posječeni, ponajviši sniženi. 
\par 34 Pod sjekirom pada šumska guštara, sa slavom svojom pada Libanon. 


\chapter{11}

\par 1 Isklijat će mladica iz panja Jišajeva,  izdanak će izbit' iz njegova korijena. 
\par 2 Na njemu će duh Jahvin počivat', duh mudrosti i umnosti, duh savjeta i jakosti, duh znanja i straha Gospodnjeg. 
\par 3 Prodahnut će ga strah Gospodnji: neće suditi po viđenju, presuđivati po čuvenju, 
\par 4 već po pravdi će sudit' ubogima i sud prav izricat' bijednima na zemlji. Šibom riječi svoje ošinut će silnika, a dahom iz usta ubit' bezbožnika. 
\par 5 On će pravdom opasati bedra, a vjernošću bokove. 
\par 6 Vuk će prebivati s jagnjetom, ris ležati s kozlićem, tele i lavić zajedno će pasti, a djetešce njih će vodit'. 
\par 7 Krava i medvjedica zajedno će pasti, a mladunčad njihova skupa će ležati, lav će jesti slamu k'o govedo. 
\par 8 Nad rupom gujinom igrat će se dojenče, sisanče će ruku zavlačiti u leglo zmijinje. 
\par 9 Zlo se više neće činiti, neće se pustošiti na svoj svetoj gori mojoj: zemlja će se ispuniti spoznajom Jahvinom kao što se vodom pune mora. 
\par 10 U dan onaj: Jišajev izdanak, dignut kao stijeg narodima, puci će željno tražiti. I prebivalište njegovo bit će slavno. 
\par 11 U dan onaj: Jahve će drugi put ruku pružiti da otkupi Ostatak svoga naroda, one što ostanu iz Asira i iz Egipta, iz Patrosa, Kuša i Elama, iz Šineara, Hamata i s morskih otoka. 
\par 12 Podignut će stijeg narodima, sabrat će Izraelu prognanike i skupiti Judi raspršene sa sva četiri kraja zemlje. 
\par 13 Ljubomor će nestat' Efrajimov, bit će istrijebljeni dušmani Judini; Efrajim neće više zavidjeti Judi, a Juda neće biti neprijatelj Efrajimu. 
\par 14 Filistejcima na zapadu za vrat će sjesti, zajedno će plijeniti sinove Istoka; ruku će svoju pružit' na Edom i Moab, bit će im pokorni sinovi Amonovi. 
\par 15 Jahve će isušit' zaljev mora egipatskog, zamahnut će rukom protiv Eufrata; snagom daha razbit će ga na sedam potoka da se u obući može prelaziti: 
\par 16 i bit će cesta Ostatku njegova naroda, koji preživio bude iz Asira, kao što bijaše Izraelcima kad iziđoše iz zemlje egipatske. 


\chapter{12}

\par 1 Reći ćeš u dan onaj: Hvalim te, Jahve, razgnjevi se ti na mene, ali se odvratio gnjev tvoj i ti me utješi! 
\par 2 Evo, Bog je spasenje moje, uzdam se, ne bojim se više, jer je Jahve snaga moja i pjesma, on je moje spasenje. 
\par 3 I s radošću ćete crpsti vodu iz izvora spasenja. 
\par 4 Reći ćete u dan onaj: Hvalite Jahvu prizivajte ime njegovo! Objavite narodima djela njegova, razglašujte uzvišenost imena njegova! 
\par 5 Pjevajte Jahvi, jer stvori divote, neka je to znano po svoj zemlji! 
\par 6 Kličite i radujte se, stanovnici Siona, jer je velik među vama Svetac Izraelov! 


\chapter{13}

\par 1 Proroštvo o Babilonu koje vidje Izaija, sin Amosov. 
\par 2 Na goletnu brdu dignite zastavu, vičite im iz sveg grla. Mašite rukom neka dođu na vrata kneževska. 
\par 3 Zapovijed dadoh svetim svojim ratnicima, gnjevu svom pozvah svoje junake koji slave veličanstvo moje. 
\par 4 Čuj! Žagor na gorama kao od silna naroda. Čuj! Buka kraljevstava, sakupljenih naroda. To Jahve nad Vojskama za boj vojsku pregleda. 
\par 5 Iz daleka kraja, s granica neba dolaze oni - Jahve i oruđa gnjeva njegova - da svu zemlju poharaju. 
\par 6 Kukajte, jer je blizu Jahvin dan, k'o pohara dolazi od Svemoćnog. 
\par 7 I sve ruke stog' malakšu ... Svako ljudsko srce klone, 
\par 8 strava ih je obrvala, trudovi boli već ih spopadaju i grče se k'o rodilja. U prepasti jedan drugog motre, lica su im poput plamena. 
\par 9 Dolazi nesmiljeni Jahvin dan - gnjev i jarost - da u pustoš zemlju prometne, da istrijebi iz nje grešnike. 
\par 10 Jer nebeske zvijezde a ni Štapci neće više sjati svjetlošću, pomrčat će sunce ishodeći i mjesec neće više svijetliti. 
\par 11 Kaznit ću svijet za zloću, bezbožnike za bezakonje, dokrajčit ću ponos oholih, poniziti nadutost silnika. 
\par 12 Rjeđi će biti čovjek neg' žeženo zlato, rjeđi samrtnik od zlata ofirskog. 
\par 13 Nebesa ću potresti, maknut će se zemlja s mjesta od srdžbe Jahve nad Vojskama u dan kad se izlije gnjev njegov. 
\par 14 I tada, kao gazela preplašena, kao ovce koje nitko ne prikuplja, svatko će se vratit' svom narodu, svatko će u zemlju svoju pobjeći. 
\par 15 Koga stignu, probost će ga: koga uhvate, mačem će sasjeći; 
\par 16 pred očima smrskat će im dojenčad, opljačkati kuće, silovati žene. 
\par 17 Gle, podižem na njih Medijce što ne cijene srebra i preziru zlato. 
\par 18 Svi će mladići biti pokošeni, sve djevojke zatrte. Na plod utrobe neće se smilovati, nad djecom im se oko neće sažaliti. 
\par 19 Babilon, ures kraljevstava, ures i ponos kaldejski, bit će k'o Sodoma i Gomora kad ih Bog zatrije. 
\par 20 Nikad se više neće naseliti, od koljena do koljena ostat će nenapučen. Arapin ondje neće dizati šatora, nit' će pastiri ondje počivati. 
\par 21 Počivat će ondje zvijeri pustinjske, sove će im napuniti kuće, nojevi će ondje stanovati, jarci plesati. 
\par 22 Hijene će zavijati iz njegovih palača, a čaglji iz raskošnih dvorova... Vrijeme se njegovo bliži, dani mu se neće produžiti. 


\chapter{14}

\par 1 Da, smilovat će se Jahve Jakovu i opet izabrati Izraela, dati  mu da počine u svojoj zemlji. Pridružit će im se došljak i priključiti  se domu Jakovljevu. 
\par 2 Sami će narodi po njih doći i odvesti  ih u njihov kraj, a njih će Dom Izraelov baštiniti u Jahvinoj  zemlji kao sluge i sluškinje. I zarobit će one koji njih bijahu  zarobili i gospodovat će nad svojim tlačiteljima. 
\par 3 U dan kad ti Jahve dade da počineš od svojih stradanja, nemira i teškog robovanja kojima te pritisnuše, 
\par 4 zapjevat  ćeš ovu rugalicu kralju babilonskom: Kako nestade silnika? Kako nestade tlačenja? 
\par 5 Jahve slomi štap zlikovački i žezlo vladarsko 
\par 6 što je bijesno b§ilo narode udarcima bez kraja i konca, što je gnjevno vladalo narodima progoneć' ih nemilice. 
\par 7 Počiva, miruje sva zemlja kličući od radosti. 
\par 8 Nad tobom se raduju čempresi i cedri libanonski: "Otkako si pao, ne dolaze nas više sjeći!" 
\par 9 Zbog tebe se uzbudi Podzemlje da te dočeka kada dođeš. U tvoju čast ono budi sjene, sve zemaljske mogućnike, ono diže s prijestolja sve kraljeve naroda. 
\par 10 Svi ti oni progovaraju: "I ti si skršen k'o i mi, nama si sličan postao. 
\par 11 Oholost se tvoja sruši u Podzemlje i buka tvojih harfa; pod tobom je ležaj od truleži, a crvi tvoj su pokrivač. 
\par 12 Kako pade sa nebesa, Svjetlonošo, sine Zorin? Kako li si oboren na zemlju, ti, vladaru naroda? 
\par 13 U svom si srcu govorio: 'Uspet ću se na nebesa, povrh zvijezda Božjih prijesto ću sebi dići. Na zbornoj ću stolovati gori na krajnom sjeveru. 
\par 14 Uzaći ću u visine oblačne, bit ću jednak Višnjemu.' 
\par 15 A sruši se u Podzemlje, u dubine provalije!" 
\par 16 Koji te vide, motre te i o tebi razmišljaju: "Je li to čovjek koji je zemljom tresao i drmao kraljevstvima, 
\par 17 koji je u pustinju svijet obraćao i sa zemljom sravnjivao gradove, koji sužnjeva svojih nikad nije kući otpuštao?" 
\par 18 Kraljevi naroda svi u časti počivaju, svaki u svojoj grobnici. 
\par 19 A ti si iz groba izbačen kao smeće odvratno, pokriven poklanima, mačem probodenima, koji su na ploče grobne pobacani k'o strvina zgažena! 
\par 20 Nećeš se združiti s njima u grobu, jer si zemlju svoju uništio i narod svoj poklao. I nikad se više neće spominjati pleme zlikovačko. 
\par 21 Spremite pokolj djeci za bezakonje otaca njihovih, da se više ne dignu da osvajaju zemlju i da ne pokriju lice svijeta! 
\par 22 Ustat ću na njih - riječ je Jahve nad Vojskama - i zatrt  ću Babilonu ime i ostatak, rod i porod - riječ je Jahvina. 
\par 23 Dat  ću ga u posjed ježevima i močvarama. Pomest ću ga metlom zatornom  - riječ je Jahve nad Vojskama. 
\par 24 Zakleo se Jahve nad Vojskama: "Što zamislih, zbit će se, što naumih, izvršit će se! 
\par 25 Skršit ću Asur u svojoj zemlji, zgazit ću ga na svojim gorama. Jaram njegov s njih će spasti, spast će im s pleća breme njegovo." 
\par 26 Takva je odluka stvorena protiv čitave zemlje; takva je ruka pružena protiv svih naroda! 
\par 27 A kad Jahve nad Vojskama odluči, tko da se usprotivi? Kada ruku ispruži, tko da je odvrati? 
\par 28 One godine kad je umro kralj Ahaz, bi objavljeno ovo  proroštvo: 
\par 29 Ne raduj se, sva Filistejo, što se slomi štap koji te udarao, jer će iz korijena zmijinjeg izaći ljutica što će izleći zmaja krilatoga. 
\par 30 Ubogi će pasti na mojim pašnjacima i u pouzdanju počivati siromasi; glađu ću pomoriti rod tvoj i pobiti što od tebe ostane. 
\par 31 Kukajte, vrata! Zapomaži, grade! Strepi, sva Filistejo! Jer sa sjevera dim dolazi i čete njegove ne napušta nitko! 
\par 32 Što će se tad odgovoriti glasnicima naroda? "Jahve zasnova Sion, i u njem su zaštićeni siromasi njegova naroda." 


\chapter{15}

\par 1 Proroštvo o Moabu. Obnoć opustošen, šaptom pade Ar Moab! Obnoć opustošen, šaptom pade Kir Moab! 
\par 2 Uspinju se u hram dibonski, na uzvišice da plaču; nad Nebom i nad Medebom Moab nariče! Sve su glave ostrižene, a sve brade obrijane; 
\par 3 na ulicama oblače vreće, na njegovim krovovima leleču! Na njegovim trgovima svi nariču i suze prolijevaju. 
\par 4 Jauču Hešbon i Eleala, do Jahasa jauk se čuje. Zato dršću bokovi Moabu, strepi duša njegova; 
\par 5 srce Moabovo jeca, bjegunci mu idu do Soara. Da, uz brdo Luhit uspinju se plačući; putem horonajimskim razliježe se jauk nad propašću. 
\par 6 Da! Vode nimrimske postadoše pustinja: trava usahla, bilja nestalo, zelenila više nema. 
\par 7 Zato stečevinu koju stekoše i ono što prištedješe nose preko Potoka vrba. 
\par 8 Da! Jauk se razliježe po kraju moapskom: kukanje mu do Eglajima, kukanje mu do Beer Elima. 
\par 9 Da! Vode dimonske krvi su pune, i još jednu nesreću dodajem Dimonu: jednog lava na moapske bjegunce i na preživjele u zemlji. 


\chapter{16}

\par 1 Šaljite jaganjce vladaru zemlje, od Stijene prema pustinji do gore Kćeri sionske. 
\par 2 Kao razbjegle ptice, kao raspršeno gnijezdo bit će kćeri moapske na arnonskim gazovima. 
\par 3 Daj nam savjet, stvori odluku! Sred podneva sjenu svoju kao noć razastri. Sakrij izagnane, ne izdaj bjegunca. 
\par 4 Daj da kod tebe borave prognanici moapski, budi im okriljem pred pustošnikom. Kad se skonča tlačitelj, kad nestane pustošnika, kad ugnjetač iščezne iz zemlje, 
\par 5 učvrstit će se prijesto u blagosti i na njemu će vjerno stolovati, u šatoru Davidovu, sudac koji pravo ište i pravdu čini. 
\par 6 Čuli smo za nadutost Moaba, nadutost preveliku, za ponos, oholost i uznositost; isprazno je njegovo hvastanje. 
\par 7 Zato kuka Moab, nad Moabom svi jauču, za kolačima grožđanim iz Kir Heresa jauču posve slomljeni. 
\par 8 Jer uvenuše nasadi hešbonski, trsje sibmansko, i lišće su mu pomlatili gospodari naroda. Sezaše do Jazera, zamicaše u pustinju; izdanci mu dosezahu sve do mora. 
\par 9 Zato plačem za trsjem sibmanskim kao što plače Jazer, suzama te ja zalijevam, Hešbone i Elealo! Nad plodovima tvojim, nad jematvom, krik se začu; 
\par 10 nestade iz voćnjaka veselja i radosti. U vinogradima ne pocikuje se, ne kliče se od radosti; ne mastÄi se vino u kaci, zamuknu podvikivanje. 
\par 11 Zato utroba moja za Moabom poput harfe dršće, a grudi mi za Kir Herešom. 
\par 12 Zaludu se pokazuje Moab, umara se na uzvišicama dolazeći u svetište da se moli: ništa postići neće. 
\par 13 Ovo je riječ koju nekoć reče Jahve protiv Moaba. 
\par 14 A  sada govori Jahve ovako: "Za tri godine, godine najamničke, slava  će Moabova, sa svim velikim mnoštvom njegovim, potamnjeti, a  što od nje ostane, bit će maleno, slabo i nemoćno." 


\chapter{17}

\par 1 Proroštvo o Damasku. Gle, prestat će Damask biti gradom i postat će hrpom ruševina; 
\par 2 njegovi gradići, dovijek napušteni, bit će pašnjak stadima; ležat će u njima i nitko ih neće tjerati. 
\par 3 Izgubit će Efrajim utvrde, a Damask kraljevstvo; ostatku Arama zbit će se što i slavi sinova Izraelovih - riječ je Jahve nad Vojskama. 
\par 4 U onaj dan smanjit će se slava Jakovljeva, spast će salo tijela njegova. 
\par 5 Bit će k'o kad žetelac žito hvata, a ruka mu žanje klasje, kao kad se pabirče klasovi u refajimskoj dolini - 
\par 6 ostat će samo pabirci; ili kao kad se otresa maslina: dvije-tri uljike sasvim na vrhu, četiri ili pet na granama drveta - riječ je Jahve, Boga Izraelova. 
\par 7 U onaj dan čovjek će pogledati na svog Stvoritelja i upraviti  oči k Svecu Izraelovu. 
\par 8 Neće više gledati žrtvenika, djela  svojih ruku, neće više gledati onoga što njegovi prsti stvoriše:  ašere i sunčane stupove. 
\par 9 U onaj će dan gradovi tvoji biti napušteni, kao što bjehu napušteni hivijski i amorejski kad ih ostaviše pred Izraelcima, i opustjet će, 
\par 10 jer si zaboravio Boga svog spasenja i nisi se spomenuo Stijene svoje snage. Stog' i sadiš ljupke biljke i strane presađuješ mladice; 
\par 11 u dan kad ih posadiš, one izrastu, a ujutro procvatu tvoje sadnice, al' propada žetva u dan nevolje, u dan boli neizlječive. 
\par 12 Jao, buka naroda mnogobrojnih; buče kao što buči more; šum naroda koji šume k'o što šumore silne vode. 
\par 13 Šume narodi kao što silne vode šumore, al' kad On im zaprijeti, bježe daleko, po gorama razvijani kao pljeva na vjetru, k'o prašina u vihoru. 
\par 14 Navečer eto straha; prije svanuća više ga nema: to je sudba onih koji nas plijene i kob onih što nas pljačkaju. 


\chapter{18}

\par 1 Jao zemlji krilatih kukaca s one strane rijeka kuških, 
\par 2 tebi koja šalješ morem glasnike i vodom u čamcima rogoznim. Idite, brze skoroteče, k narodu stasitu, tamnoputu, k narodu kog se boje odvajkada, narodu žilavu, zavojevaču, čija je zemlja rijekama izbrazdana. 
\par 3 Svi stanovnici kruga zemaljskoga, i vi, žitelji zemlje, kad se zastava na brdu digne, gledajte; kad rog zatrubi, slušajte. 
\par 4 Jer ovako mi govori Jahve: "Mirno ću gledati s mjesta svojega, k'o prozirna žega podnevna, k'o rosan oblak za vrućine žetvene. 
\par 5 Prije berbe, kad loza ocvate i cvijet u grozdove dozri, nožima će posjeći mladice, povezati, ukloniti grane. 
\par 6 Svi će biti ostavljeni grabljivicama gorskim i zvijerima zemaljskim. Grabljivice će na njima ljetovati, zvijeri zemaljske zimovati." 
\par 7 U ono će vrijeme narod stasit i tamnoputan, narod kojega  se boje odvajkada, narod žilav i zavojevački, čija je zemlja  izbrazdana rijekama, donositi darove Jahvi nad Vojskama, k mjestu  imena Jahve nad Vojskama - na goru Sion. 


\chapter{19}

\par 1 Proroštvo o Egiptu. Gle, Jahve sjedi na brzu oblaku, u Egipat dolazi. Dršću pred njim idoli egipatski, u njedrima premire srce Egipćana. 
\par 2 Podbost ću Egipćane protiv Egipćana, i brat će se s bratom svojim boriti, drug s drugom, grad s gradom, a kraljevstvo s kraljevstvom. 
\par 3 Egiptu se pamet muti, ja sprečavam njegove naume; oni traže idole i vrače, opsjenare i gatare. 
\par 4 Egipćane ja predajem u ruke gospodaru okrutnu, kralj silovit njima će vladati - riječ je Jahve nad Vojskama. 
\par 5 Nestat će vode iz mora, presahnut će i presušiti Rijeka, 
\par 6 zaudarat će prokopi, spast će rukavci Rijeke egipatske i presušiti. Uvenut će trska i sita, 
\par 7 sva zelen pokraj Nila; usahnut će na Nilu svi usjevi, propast će, raspršit' se, iščeznuti. 
\par 8 Tužit će ribari, kukat će svi što u Nil udicu bacaju; jadikovat će oni što u vodi mrežu razapinju. 
\par 9 Postidjet će se lanari, grebenari i tkači bijela tkanja. 
\par 10 Snuždit će se tkalci, rastužiti radnici. 
\par 11 Pravi su luđaci knezovi soanski, mudri savjetnici faraonovi glupo svjetuju; kako se usuđujete reći faraonu: "Učenik sam mudraca, učenik drevnih kraljeva?" 
\par 12 TÓa gdje su tvoji mudraci? Nek' ti dojave i obznane što je Jahve nad Vojskama nakanio s Egiptom. 
\par 13 Ludi su knezovi taniški, prevareni knezovi memfiški, oni zavode Egipat, glavare njegovih plemena. 
\par 14 U njih je ulio Jahve duh vrtoglavi te zavode Egipat u svakom mu činu da tetura k'o pijanac kada bljuje. 
\par 15 U Egiptu više ne može uspjeti ništa od onog što čine glava i rep, palma i sita. 
\par 16 U onaj će dan Egipćani postati kao žene, drhtat će i  strepiti od zamaha ruke Jahve nad Vojskama kojom će zamahnuti  na njih. 
\par 17 Zemlja će Judina biti na užas Egiptu; kad god je  se sjeti, strah će ga obuzeti zbog onoga što je Jahve nad Vojskama  protiv njega naumio. 
\par 18 U onaj će se dan u zemlji egipatskoj  pet gradova što govore kanaanskim jezikom zakleti Jahvi nad Vojskama;  jedan će se od njih zvati Ir Hahres. 
\par 19 U onaj će dan biti žrtvenik  Jahvin usred zemlje egipatske i stup posvećen Jahvi blizu granice  njegove. 
\par 20 To će Jahvi nad Vojskama biti znak i svjedočanstvo  u egipatskoj zemlji. Kad zazovu Jahvu protiv tlačitelja, on će  im poslati spasitelja i vođu da ih izbavi. 
\par 21 I objavit će se  Jahve Egipćanima, i u onaj će dan Egipćani spoznati Jahvu; služit  će mu žrtvama i prinosima, zavjetovat će se i izvršavati zavjete. 
\par 22 Jahve će teško udariti Egipćane, ali će ih iscijeliti; obratit  će se oni Jahvi i on će ih uslišiti i iscijeliti. 
\par 23 U onaj  će dan ići cesta od Egipta do Asirije. Asirci će dolaziti u Egipat, a Egipćani u Asiriju. Egipat i Asirija služit će Jahvi. 
\par 24 U onaj će dan Izrael, treći s Egiptom i Asirijom, biti  blagoslovljen usred zemlje. 
\par 25 Jahve nad Vojskama blagoslovit  će ga: "Nek' je blagoslovljen", reći će, "moj narod egipatski, djelo mojih ruku Asirija i baština moja Izrael." 


\chapter{20}

\par 1 U godini kad je vrhovni vojskovođa, koga bijaše poslao Sargon, kralj asirski, došao u Ašdod, napao ga i zauzeo, 
\par 2 u to vrijeme  reče Jahve po Izaiji, sinu Amosovu: "Hajde, skini kostrijet s  bokova i obuću izuj s nogu." On to učini te iđaše gol i bos. 
\par 3 Tada reče Jahve: "Kao što je sluga moj Izaija tri godine išao  gol i bos kao znak i znamenje Egiptu i Kušu, 
\par 4 tako će kralj  asirski odvesti sužnje iz Egipta i izgnanike iz Kuša, mlade i  stare, gole i bose, otkrivenih zadnjica, sramote Egipta. 
\par 5 Zbunit  će se tad i posramiti zbog Kuša, svoje uzdanice, i Egipta, svojega  ponosa. 
\par 6 I reći će u onaj dan stanovnici ovog primorja: 'Eto, to je uzdanica naša kojoj se utjecasmo da nam pomogne i spasi  nas od kralja asirskoga. A kako da se mi spasimo?'" 


\chapter{21}

\par 1 Proroštvo primorskoj pustinji. Kao što vihori, hujeći nad Negebom, dolaze iz pustinje, kraja strahotna 
\par 2 - otkri mi se u strašnom viđenju - tako pljačkaš pljačka, pustošnik pustoši. "Navali, Elame, opsjedni, Medijo! Dokrajčit ću sve jauke." 
\par 3 Zato bedra moja probadaju grčevi; bolovi me spopadaju k'o trudovi porodilju; od smućenosti ogluših, od straha obnevidjeh. 
\par 4 Srce mi dršće, groza me obuze, sumrak za kojim čeznuh postade mi užas. 
\par 5 Postavljaju stol, prostiru stolnjak, jede se i pije ... Ustajte, knezovi, mažite štit! 
\par 6 Jer Gospod mi ovako reče: "Idi, postavi stražara! Što vidi, nek' javi. 
\par 7 Vidi li konjanike, jahače udvojene, jahače na magarcima, jahače na devama, neka dobro pazi s pažnjom napetom!" 
\par 8 A stražar viknu: "Povazdan, Gospodaru, stojim na stražarnici, čitavu noć na straži prostojim." 
\par 9 I gle, dolaze konjanici, jahači udvojeni. Oni mi viknuše, oni rekoše: "Pade, pade Babilon! Svi kipovi njegovih bogova o zemlju se razbiše." 
\par 10 Omlaćeno žito, čedo gumna moga, što čuh od Jahve nad Vojskama, Boga Izraelova, objavih vam! 
\par 11 Proroštvo o Edomu. Viču mi iz Seira: "Stražaru, koje je doba noći? Stražaru, koje je doba noći?" 
\par 12 Stražar odgovori: "Dolazi jutro, a zatim opet noć. Hoćete li pitati, pitajte, vratite se, dođite!" 
\par 13 Proroštvo o Arapima. U šikarama pustinjskim počivate, dedanske karavane. 
\par 14 Vodu iznesite pred žedne, stanovnici zemlje temske, iziđite s kruhom pred bjegunca. 
\par 15 Pred mačevima bježe oni, pred mačem trgnutim, pred lukom zapetim, pred bojem žestokim. 
\par 16 Da, ovako mi reče Gospod: "Još jedna godina, godina nadničarska, i nestat će sve slave Kedrove. 
\par 17 A od mnogobrojnih strijelaca  među hrabrim sinovima Kedrovim malo će ih ostati." Tako je govorio  Jahve, Bog Izraelov. 


\chapter{22}

\par 1 Proroštvo  o Dolini viđenja: Što ti je da uzlaziš sav na krovove, 
\par 2 grade pun vreve, grade bučni, grade razigrani? Pobijeni tvoji ne padoše od mača, nit' u boju poginuše mrtvi tvoji; 
\par 3 glavari svi ti zajedno pobjegoše, u ropstvo padoše luka ne napevši. Zasužnjeni su svi tvoji knezovi, zalud umakoše daleko. 
\par 4 Zato velim: "Ostavite me, zaplakat ću gorko; nemojte me tješiti zbog uništenja naroda moga." 
\par 5 Jer ovo je dan pometnje i rasula od Jahve, Gospoda nad Vojskama. U Dolini viđenja zid se ruši, vapaj za pomoć diže se u goru. 
\par 6 Elam tobolac uzima, Aram konja jaše, a Kir štit otkriva. 
\par 7 Krasne doline tvoje pune su bojnih kola, konjanici stižu pred vrata. 
\par 8 Tako pada Judi zaštita. U onaj dan svrnuo si pogled na oružje u Šumskoj kući. 
\par 9 Vidjeste da u gradu Davidovu ima mnogo pukotina. Sabraste vodu iz Donjega ribnjaka. 
\par 10 Prebrojiste kuće jeruzalemske, porušiste kuće da zid učvrstite. 
\par 11 Između dva zida načiniste spremište za vodu iz starog ribnjaka. Ali se ne obazreste na Tvorca, nit' vidjeste onog što davno sve smisli. 
\par 12 Gospod, Jahve nad Vojskama, pozivaše vas u dan onaj da plačete i tugujete, da obrijete glave i pripašete kostrijet. 
\par 13 A gle: radost i veselje, ubijaju goveda i kolju ovce; jedu meso i piju vino: "Jedimo i pijmo, jer sutra nam je mrijeti." 
\par 14 Ali Jahve nad Vojskama objavi mi: "Dok ne umrete, grijeh taj neće vam se oprostiti", reče Jahve, Gospod nad Vojskama. 
\par 15 Ovako govori Jahve, Gospod nad Vojskama: "Hajde, idi k onom dvorjaninu, k Šibni, nadstojniku dvora, 
\par 16 koji kleše sebi grob na uzvisini i u stijeni sebi prebivalište usijeca: 'Što tu posjeduješ i koga tu imaš da sebi klešeš grobnicu?' 
\par 17 Gle, Jahve će te baciti daleko udarcem jednim jedinim, snažno će te uhvatiti, 
\par 18 smotat će te u klupko, kao loptu te baciti po zemlji širokoj! Ondje ti ćeš umrijeti, s kolima što ti bijahu na slavu, ti, sramoto dvoru svoga gospodara!" 
\par 19 Lišit ću te tvoje službe, otjerat te sa tvog mjesta; 
\par 20 i pozvat ću svoga slugu Elijakima, sina Hilkijina. 
\par 21 Obući ću mu tvoju haljinu, tvojim ću ga pojasom opasat', tvoju ću mu vlast predati u ruke te će biti otac žiteljima jeruzalemskim i kući Judinoj. 
\par 22 Metnut ću mu na pleća ključ od kuće Davidove: kad otvori, nitko neće zatvoriti, kad zatvori, nitko neće otvoriti. 
\par 23 Kao klin zabit ću ga na tvrdu mjestu; i postat će prijesto slave domu oca svojega. 
\par 24 O njega će se vješati sva slava njegova doma očinskog, izdanci i potomstvo, sve malo posuđe, od zdjelica do ćupova. 
\par 25 U onaj dan - riječ je Jahve nad Vojskama - izvući će se klin  zabijen na tvrdu mjestu, slomit će se i pasti. I sav teret što  je visio na njemu razbit će se, jer je Jahve govorio. 


\chapter{23}

\par 1 Proroštvo o Tiru. Kukajte, lađe taršiške, jer vaša je tvrđa razorena! Javljeno im je dok se iz kitimske vraćahu zemlje. 
\par 2 Umuknite, stanovnici primorja, trgovci sidonski, kojih su glasnici brodili morem po vodi velikoj. 
\par 3 Sjetva Nila, žetva Rijeke, bijaše njegovo bogatstvo. On bijaše sajmište narodima. 
\par 4 Stidi se, Sidone, jer more govori: "Ne hvataju me trudovi niti rađam, ne odgajam momaka nit' podižem djevojaka." 
\par 5 Uzdrhtat će Egipćani kad o Tiru vijest čuju. 
\par 6 Otplovite u Taršiš, kukajte, stanovnici primorja. 
\par 7 Je li to vaš grad veseli što postoji od davnih davnina i noge ga daleko nosile da se ondje naseli? 
\par 8 Tko li je to odlučio protiv Tira okrunjenog, kojeg trgovci bijahu knezovi a prekupci odličnici zemlje? 
\par 9 Jahve nad Vojskama odluči tako da osramoti ponosnu slavu, da ponizi sve odličnike zemlje. 
\par 10 Obrađuj zemlju, kćeri taršiška, tvoje luke više nema! 
\par 11 Ruku svoju Gospod diže na more i kraljevstvima zadrma. Zapovjedi Jahve da se razore tvrđave kanaanske. 
\par 12 Rekao je: "Nećeš više klikovati, okaljana djevice, kćeri sidonska!" Ustani i idi u Kitim; ni ondje nećeš imati mira. 
\par 13 Evo zemlje kitimske ... podižu se kule opsadne, razaraju utvrde, sve je ruševina. 
\par 14 Kukajte, brodovi taršiški, razorena je vaša tvrđava! 
\par 15 Dogodit će se, u onaj dan, da će Tir biti zaboravljen  sedamdeset godina, kao dani jednoga kralja. A poslije sedamdeset  godina Tiru će biti kao bludnici iz pjesme: 
\par 16 "Uzmi citaru i skići se gradom, bludnice zaboravljena! Sviraj lijepo, pjevaj mnogo, da te se spomenu!" 
\par 17 Poslije sedamdeset godina pohodit će Jahve Tir. I grad  će opet dobivati svoju plaću bludničku. Podavat će se bludu sa  svim kraljevstvima svijeta na licu zemlje. 
\par 18 Ali će njegova  dobit i plaća biti posvećena Jahvi; neće se zgrtati ni čuvati, nego će njegova dobit biti za one koji prebivaju pred Jahvom  da imaju hrane do sita i doličnu odjeću. 


\chapter{24}

\par 1 Gle, Jahve razvaljuje zemlju, razara je, nakazi joj lice, raspršuje stanovnike njene. 
\par 2 Svećenik će biti k'o i narod, gospodar k'o i sluga, gospodarica k'o i sluškinja, prodavač k'o i kupac, zajmodavac k'o i zajmoprimac, vjerovnik k'o i dužnik. 
\par 3 Opustošena će biti zemlja, opljačkana sasvim, jer je Jahve odlučio. 
\par 4 Zemlja tuži, vene, svijet gine, gasne, nebo sa zemljom propada. 
\par 5 Oskvrnjena je zemlja pod žiteljima svojim, jer prestupiše Zakon, pogaziše odredbu, Savez vječni razvrgoše. 
\par 6 Zato prokletstvo proždire zemlju, okajavaju stanovnici njeni. Zato su sažgani žitelji zemljini i malo je ljudi preostalo. 
\par 7 Vino tuguje, loza vene, uzdišu svi što bijahu srca vesela. 
\par 8 Prestalo je veselje uz bubnje, zamrla je graja razigrana; umukla je glazba citara. 
\par 9 Ne pije se više vino uz pjesmu, ogrknu piće silovito. 
\par 10 Razoren je grad ništavila, zatvoren ulaz svim kućama. 
\par 11 Jauk po ulicama zbog vina, nesta svakog veselja, radost je iz zemlje prognana. 
\par 12 Tek pustoš ostade u gradu, u trijeske smrskana su vrata. 
\par 13 Jer tako se zbiva na zemlji, među narodima, kao kad se oberu masline il' paljetkuje grožđe nakon berbe. 
\par 14 Oni glas podižu, kliču od radosti; uznose s mora veličajnost Jahvinu. 
\par 15 I na istoku ime Jahve slave oni, na otocima mora ime Jahve, Boga Izraelova. 
\par 16 S kraja zemlje čujemo pjesme: "Slava Pravedniku!" Ali ja kažem: "Propadoh! Propadoh! Jao meni! Vjerolomci se iznevjeriše, nevjerom se, vjerolomci, iznevjeriše." 
\par 17 Strava, jama, zamka tebi, žitelju zemlje: 
\par 18 tko pred glasom strave uteče u jamu će upasti; tko se iz jame izvuče zamka će ga uhvatit'. Da, otvorit će se ustave u visini i zatresti zemlji temelji. 
\par 19 Zemlja će se grozno razbiti, zemlja će se strašno raspući, zemlja će se silno uzdrmati, 
\par 20 zemlja će zateturati poput pijanca, zanjihat se poput kolibe; toliko će joj otežati bezakonje njeno da će pasti i neće više ustati. 
\par 21 I dogodit će se u onaj dan: Jahve će kazniti u visini Vojsku nebesku, a na zemlji sve kraljeve zemaljske; 
\par 22 bit će sakupljeni i zasužnjeni u jami, zatvoreni u tamnicu i nakon mnogih dana kažnjeni. 
\par 23 Pocrvenjet će mjesec, postidjet se sunce, jer će kraljevati Jahve nad Vojskama na gori Sionu i u Jeruzalemu i Slava će mu sjati pred starješinama. 


\chapter{25}

\par 1 Jahve, ti si Bog moj, uznosim te, tvoje ime slavim, jer si proveo čudesan naum, smišljen od davnine, istinit i vjeran, 
\par 2 jer grad si pretvorio u hrpu kamenja, a mjesto utvrđeno u ruševine. Tvrđa neprijateljska više nije grad, dovijeka se više obnoviti neće. 
\par 3 Zato te slavi narod snažan, grad moćnih plemena tebe se boji; 
\par 4 jer ti si utočište nevoljnom, utočište ubogom u nevolji; ti si skrovište od pljuska i od žege zaklon, jer ćud je silnička kao pljusak zimski; 
\par 5 kao žega nad zemljom sušnom ti gušiš graju neprijatelja; kao žega sjenom oblaka prekinu se silniku pjesma pobjednička. 
\par 6 I Jahve nad Vojskama spremit će svim narodima na ovoj gori gozbu od pretiline, gozbu od izvrsna vina, od pretiline sočne, od vina staložena. 
\par 7 Na ovoj gori on će raskinuti zastor što zastiraše sve narode, pokrivač koji sva plemena pokrivaše 
\par 8 i uništit će smrt zasvagda. I suzu će sa svakog lica Jahve, Gospod, otrti - sramotu će svog naroda na svoj zemlji skinuti: tako Jahve reče. 
\par 9 I reći će se u onaj dan: "Gle, ovo je Bog naš, u njega se uzdasmo, on nas je spasio; ovo je Jahve u koga se uzdasmo! Kličimo i veselimo se spasenju njegovu, 
\par 10 jer ruka Jahvina na ovoj gori počiva!" Moab je izgažen na svome mjestu kao što se gazi slama na buništu; 
\par 11 ondje on razmahuje rukama kao što ih razmahuje plivač kada pliva. Ali Jahve obara njegovu ponositost i propinjanja ruku njegovih. 
\par 12 Visoku tvrđu tvojih zidina on razvaljuje, na zemlju baca, u prah ruši. 


\chapter{26}

\par 1 U onaj dan pjevat ću ovu pjesmu u zemlji Judinoj: "Tvrd grad mi imamo: za obranu nam On podiže zidove i predziđa. 
\par 2 Otvorite vrata! Nek' uđe narod pravedni koji čuva vjernost, 
\par 3 čiji je značaj čvrst, koji čuva mir jer se u te uzda. 
\par 4 Uzdajte se u Jahvu dovijeka, jer Jahve je Stijena vječna; 
\par 5 on obara one koji obitavahu visoko, razvaljuje tvrđu visoku, ruši u prah, sravnjuje sa zemljom, 
\par 6 te je gaze noge, noge ubogih i koraci nevoljnih." 
\par 7 Put je pravednikov prav, ti ravniš stazu pravednom. 
\par 8 Da, na stazi tvojih sudova željno te, Jahve, čekamo; ime tvoje i spomen tvoj duša nam žudi. 
\par 9 Dušom svojom žudim tebe noću i duhom svojim u sebi te tražim. Jer kad se na zemlji pojave tvoji sudovi, uče se pravdi stanovnici kruga zemaljskoga. 
\par 10 Ako se pomiluje opaki, on se ne uči pravednosti. U zemlji pravednosti on čini bezakonje i ne obazire se na veličanstvo Jahvino. 
\par 11 Jahve, ruka je tvoja podignuta, a oni je ne vide. Nek' vide i postide se, nek' ih proguta revnost za narod, nek' ih proždre oganj pripravljen dušmanima tvojim. 
\par 12 Jahve, ti mir nama daješ, jer ti si tvorac svih djela naših. 
\par 13 Jahve, Bože naš, gospodarili su nama osim tebe drugi gospodari, ali tebe jedinog, ime tvoje, častimo. 
\par 14 Mrtvi neće oživjeti, sjene neće uskrsnuti, jer ti si ih kaznio i uništio i zatro svaki spomen na njih! 
\par 15 Umnožio si narod, Jahve, umnožio si narod, proslavio se, proširio sve granice zemlje! 
\par 16 Jahve, tražili su te u nevolji; izlijevali tihu molitvu, kad ih je stigla tvoja kazna. 
\par 17 Kao što se trudna žena pred porođajem grči i viče u bolovima, takvi smo, Jahve, pred tobom. 
\par 18 Zatrudnjeli smo, u mukama smo kao da rađamo, nismo donijeli duha spasenja zemlji nit' se rodiše stanovnici svijeta. 
\par 19 Tvoji će mrtvi oživjeti, uskrsnut će tijela. Probudite se i kličite, stanovnici praha! Jer rosa je tvoja - rosa svjetlosti, i zemlja će sjene na svijet dati. 
\par 20 Hajde, narode moj, uđi u sobe i vrata za sobom zatvori. Sakrij se časkom dok jarost ne prođe. 
\par 21 Jer, gle, izići će Jahve iz svog prebivališta da stanovnike zemljine kazni što se o njeg' ogriješiše. Izbacit će zemlja svu krv što je na njoj prolivena i neće više kriti onih koji su na njoj poklani. 


\chapter{27}

\par 1 U onaj dan kaznit će Jahve mačem ljutim, velikim i jakim Levijatana, zmiju hitru, Levijatana, zmiju vijugavu, i ubit će zmaja morskoga. 
\par 2 U onaj dan pjevajte mu, vinogradu vinorodnom: 
\par 3 Ja, Jahve, njega čuvam, svaki čas ga zalijevam, i da ga tko ne ošteti, danju i noću stražim. 
\par 4 Nema gnjeva u meni! Nek' se trnje i drač samo pojavi, protiv njega ustat ću u boj, svega ću ga sažgati! 
\par 5 Ili u moje nek' dođe okrilje, neka sklopi mir sa mnom, mir neka sklopi sa mnom! 
\par 6 Dolaze dani kad će se ukorijeniti Jakov, razgranit' se i procvasti Izrael, i sav svijet plodovima napuniti. 
\par 7 Je li ga udario kako udari one koji njega udarahu? Je li ga ubio kako ubi one koji njega ubiše? 
\par 8 Za kaznu ga potjera, izagna, odnese ga silnim dahom svojim u dan istočnjaka. 
\par 9 Tako će se okajati bezakonje Jakovljevo; a ovo je sve plod oproštenja grijeha njegova. Neka se smrve svi kamenovi žrtvenika kao što se u prah drobi krečno kamenje! Nek' se više ne dižu ašere i sunčani stupovi! 
\par 10 Jer opustje tvrdi grad, naselje je poharano, napušteno kao pustinja. Telad ondje pase - leži ondje, grmlje brsti. 
\par 11 Kad mu se osuše grane, lome ih, dolaze žene i oganj pale. Jer to je narod nerazuman, zato ga neće žaliti Stvoritelj, Tvorac mu se neće smilovati. 
\par 12 Jahve će u dan onaj klasje vrijeći od Eufrata do Potoka egipatskog, i bit ćete pobrani jedan po jedan, djeco Izraelova. 
\par 13 U onaj dan zatrubit će velika trublja, i doći će izgubljeni u zemlji asirskoj, i koji bijahu izgnani u zemlju egipatsku, i poklonit će se Jahvi na Svetoj gori, u Jeruzalemu. 


\chapter{28}

\par 1 Teško gizdavu vijencu pijanica Efrajimovih,  uvelu cvijetu blistava mu nakita - onima što uvrh plodnog dola leže vinom opijeni! 
\par 2 Evo, od Gospoda jaki i moćni, kao pljusak s tučom, kao vihor razorni, prolom oblaka i povodanj, i svom ih snagom na zemlju baca. 
\par 3 Bit će izgažen nogama gizdav vijenac pijanica Efrajimovih 
\par 4 i uveo cvijet blistava mu nakita uvrh plodnih dolina; bit će kao rana smokva prije ljeta: čim je tko spazi, odmah je ubere. 
\par 5 U onaj dan Jahve nad Vojskama postat će kruna slave i sjajan vijenac Ostatku svoga naroda - 
\par 6 duh pravde onome koji sjedi na stolici sudačkoj i srčanost onome koji odbija napad na vrata. 
\par 7 Oni posrću od vina, teturaju od žestoka pića: svećenici i proroci od žestoka pića posrću; omami ih vino; teturaju od žestoka pića, posrću u viđenjima, ljuljaju se sudeći. 
\par 8 Svi su stolovi puni gnusnih bljuvotina, nigdje čista mjesta nema! 
\par 9 "Koga on to uči mudrosti, koga on upućuje u objavu? Zar djecu odviknutu od mlijeka odbijenu od prsiju? 
\par 10 Sav la-sav, sav la-sav, kav la-kav, kav la-kav, zeer šam, zeer šam." 
\par 11 Da, mucavim usnama i na stranom jeziku govorit će se ovom narodu. 
\par 12 On im reče: "Evo počinka, dajte umornom da otpočine! Evo odmora!" Ali ne htjedoše poslušati. 
\par 13 Zato će im Jahve ovako govoriti: "Sav la-sav, sav la-sav, kav la-kav, kav la-kav, zeer šam, zeer šam", da hodeći padnu nauznak, da se razbiju, zapletu i uhvate. 
\par 14 Stoga čujte riječ Jahvinu, vi podsmjevači, vi što vladate narodom ovim koji je u Jeruzalemu. 
\par 15 Vi velite: "Sklopismo savez sa smrću i s Podzemljem učinismo sporazum. Kad prođe bič razorni, ne, neće nas dohvatiti, jer od laži načinismo sebi sklonište i od obmane skrovište." 
\par 16 Stog ovako govori Jahve Gospod: "Evo, postavljam na Sion kamen odabrani, dragocjen kamen ugaoni, temeljac. Onaj koji u nj vjeruje neće propasti. 
\par 17 I uzet ću pravo za mjeru, a pravdu za tezulju." I tuča će vam zastrti sklonište od laži, a voda otplaviti skrovište; 
\par 18 propast će savez vaš sa smrću, vaš sporazum s Podzemljem održat' se neće. Kada bič razorni prođe, satrt će vas; 
\par 19 kad god prođe, dohvatit će vas; prolazit će svako jutro, danju i noću. Samo će vas strah uputit u objavu. 
\par 20 Prekratka će bit' postelja da se čovjek pruži, preuzak pokrivač da se umota. 
\par 21 Da, kao na gori Perasimu, Jahve će ustati, kao u Dolini gibeonskoj, on će se razjariti, da izvrši djelo svoje, djelo čudnovato, da ispuni naum svoj, naum tajnoviti. 
\par 22 Podsmijevanja se okanite, da vas jače okovi ne stegnu; jer čuh da je od Gospoda, Jahve nad Vojskama, zemlji ovoj dosuđeno uništenje. 
\par 23 Poslušajte i čujte glas moj, prisluhnite pomno moju besjedu. 
\par 24 Ore li orač svakog dana, brazdi, brana njivu svoju? 
\par 25 A kad joj poravna površinu, ne sije li grahor, ne sipa li kumin? Pšenicu gdje treba, proso i ječam, i napokon raž po rubovima? 
\par 26 Bog ga njegov upućuje, on ga uči djelu pravom. 
\par 27 Ne mlati se grahor cijepom, nećeš točkom po kuminu, već se grahor štapom mlati, a kumin se prutom lupa. 
\par 28 A da li se žito tare? Ne, nećeš ga dovijeka mlatiti: po njem ćeš pognat kolski točak i konje, ali ga zdrobiti nećeš. 
\par 29 I to dolazi od Jahve nad Vojskama, savjetom divnog, mudrošću velikog. 


\chapter{29}

\par 1 Teško Arielu! Arielu, gradu što ga opkoli David! Nek' se niže godina na godinu, nek' se izredaju blagdani, 
\par 2 pa ću pritisnuti Ariel i nastat će jauk i lelek. Za mene ćeš biti Ariel, 
\par 3 opkolit ću te kao David, rovovima okružiti, suprot tebi nasipe ću dići. 
\par 4 Oboren govorit ćeš sa zemlje, iz praha mucat' riječju prigušenom, glas će ti se iz zemlje dizat' kao pokojnikov, iz praha ćeš šaptati besjedu. 
\par 5 Kao sitna prašina bit će mnoštvo tvojih dušmana, kao pljeva razvijana - rulja silnika: i odjednom, u tren oka: 
\par 6 pohodit će te Jahve nad Vojskama grmljavinom, tutnjem, bukom velikom, vihorom, olujom i plamenim ognjem što proždire. 
\par 7 Bit će k'o san, utvara noćna: mnoštvo svih naroda što vojuje s Arielom i svih onih koji zavojštiše na nj i na utvrde njegove i koji ga odasvud pritijesniše. 
\par 8 Bit će kao kad gladan sanja da jede, a probudi se prazna želuca; i kao kad žedan sanja da pije, pa se, iznemogao, suha grla probudi. Tako će se dogoditi mnoštvu naroda koji vojuju protiv Gore sionske. 
\par 9 Stanite, skamenite se od čuda, oslijepite i obnevidite! Pijani su, ali ne od vina, posrću, ali ne od silna pića. 
\par 10 Jahve je izlio na vas duh obamrlosti, zatvorio je oči vaše - proroke, zastro glave vaše - vidioce. 
\par 11 Zato će vam svako viđenje biti kao riječi u zapečaćenoj  knjizi: dade li se kome tko zna čitati govoreći: "De, čitaj to!"  - on će odgovoriti: "Ne mogu jer je zapečaćena." 
\par 12 A dade li  se kome tko ne zna čitati govoreći: "Čitaj to!" - on će odvratiti:  "Ne znam čitati." 
\par 13 Jahve reče: "Jer mi narod ovaj samo ustima pristupa i samo me usnama časti, a srce mu je daleko od mene i njegovo štovanje naučena ljudska uredba, 
\par 14 zato ću, evo, i dalje čudno postupati s ovim narodom - čudno i prečudno: i propast će mudrost njegovih mudraca, pomračit se umnost njegovih umnika." 
\par 15 Teško onima koji se od Jahve kriju da bi svoje sakrili namjere i koji u mraku djeluju i zbore: "Tko nas vidi i tko nas pozna?" 
\par 16 Kolike li naopakosti vaše! Cijeni li se glina kao lončar, pa da djelo rekne svome tvorcu: "NIje me on načinio"? Ili lonac da rekne lončaru: "On ne razumije ništa"? 
\par 17 Neće li se naskoro Libanon u voćnjak pretvoriti, a voćnjak izroditi u šumu? 
\par 18 I čut će u onaj dan gluhi riječi knjige; oslobođene mraka i tmine, oči će slijepih vidjeti. 
\par 19 A siromasi će se opet radovati u Jahvi, najbjedniji će klicat' Svecu Izraelovu, 
\par 20 jer neće više biti silnika, nestat će podsmjevača, istrijebit će se svi koji zlo snuju: 
\par 21 oni koji riječju druge okrivljuju, oni koji na vratima sucu postavljaju zamku i nizašto obaraju pravednika. 
\par 22 Zato ovako govori Jahve, Bog kuće Jakovljeve, koji otkupi Abrahama: "Neće se odsad više stidjeti Jakov i više mu neće lice blijedjeti, 
\par 23 jer kad vidi usred sebe djelo mojih ruku, svetit će ime moje." Svetit će Sveca Jakovljeva, bojat će se Boga Izraelova. 
\par 24 Zabludjeli duhom urazumit će se, a oni što mrmljaju primit će pouku. 


\chapter{30}

\par 1 Teško sinovima odmetničkim! - riječ je Jahvina. Oni provode osnove koje nisu moje, sklapaju saveze koji nisu po mom duhu i grijeh na grijeh gomilaju. 
\par 2 Zaputiše se u Egipat, ne pitajući usta moja, da se uteku faraonovu zaklonu i da se zaštite u sjeni Egipta. 
\par 3 Zaklon faraonov bit će na sramotu, i na ruglo zaštita u sjeni Egipta. 
\par 4 Eno mu knezova već u Soanu, podanici stigoše u Hanes: 
\par 5 svi će se oni razočarati u narodu beskorisnom, neće im biti na pomoć ni na korist, već na sramotu i porugu. 
\par 6 Proroštvo o negepskim životinjama. Kroza zemlju nevolje i bijede, lavice i lava koji riču, ljutice i zmaja krilatog, nose oni blago na leđima magaraca i bogatstvo na grbi deva, nose ga narodu beskorisnom. 
\par 7 Jer prazna je i ništavna pomoć Egipta, zato ga i zovemo: Rahab - danguba. 
\par 8 Ded napiši na ploču i zapiši u knjigu da vremenima budućim svjedočanstvo ostane. 
\par 9 Ovo je narod odmetnički, sinovi lažljivi, sinovi koji neće da slušaju Zakon Jahvin. 
\par 10 Vidovitima oni govore: "Okanite se viđenja!" a vidiocima: "Ne prorokujte istinu! Govorite nam što je ugodno, opsjene nam prorokujte! 
\par 11 Skrenite s puta, zastranite sa staze, uklonite nam s očiju Sveca Izraelova!" 
\par 12 Stog' ovako zbori Svetac Izraelov: "Jer riječ ovu odbacujete, a uzdate se u opačinu i prijevaru i na njih se oslanjate, 
\par 13 grijeh će vam taj biti poput pukotine, visoko na zidu izbočene, koja prijeti rušenjem. 
\par 14 Da se sruši k'o što se glinen sud razbije, slupan nemilice, te mu se među krhotinama ne nađe ni rbine, žerave da uzmeš s ognjišta il' zagrabiš vode iz studenca." 
\par 15 Jer ovako govori Jahve Gospod, Svetac Izraelov: "Mir i obraćenje - spas vam je, u smirenu uzdanju snaga je vaša. Ali vi ne htjedoste. 
\par 16 Rekoste: 'Ne! Pobjeći ćemo na konjima!' - i zato, bježat ćete! 'Na brzim ćemo konjima jahati!' - i zato, bit će brži vaši neprijatelji!" 
\par 17 Pobjeći će vas tisuća kad jedan zaprijeti, zaprijete li petorica, u bijeg ćete nagnut' dok vas ne preostane k'o kopljača na vrhu gore il' na brijegu zastava. 
\par 18 Al' Jahve čeka čas da vam se smiluje, i stog izglÄedÄa da vam milost iskaže, jer Jahve je Bog pravedan - blago svima koji njega čekaju. 
\par 19 Da, puče sionski koji živiš u Jeruzalemu, više ne plači!  Čim začuje vapaj tvoj, odmah će ti se smilovati; čim te čuje, uslišit će te. 
\par 20 Hranit će vas Gospod kruhom tjeskobe, pojiti  vodom nevolje, al' se više neće kriti tvoj Učitelj - oči će ti  gledati Učitelja tvoga. 
\par 21 I uši će tvoje čuti riječ gdje iza  tebe govori: "To je put, njime idite", bilo da vam je krenuti  nadesno ili nalijevo. 
\par 22 Smatrat ćeš nečistima svoje srebrne  kumire i pozlatu svojih kipova; odbacit ćeš ih kao nečist i reći  im: "Napolje!" 
\par 23 A on će dati kišu tvojem sjemenu što ga posiješ  u zemlju, i kruh kojim zemlja urodi bit će obilat i hranjiv.  Stoka će tvoja pasti u onaj dan po prostranim pašnjacima. 
\par 24 Volovi  i magarci što obrađuju zemlju jest će osoljenu krmu, ovijanu  lopatom i vijačom. 
\par 25 I na svakoj gori i na svakome povišenom  brijegu bit će potoka i rječica - u dan silnoga pokolja kad se  kule budu rušile. 
\par 26 Tada će svjetlost mjesečeva biti kao svjetlost  sunčana, a svjetlost će sunčana postati sedam puta jača, kao  svjetlost sedam dana - u dan kad Jahve iscijeli prijelom svojemu  narodu, izliječi rane svojih udaraca. 
\par 27 Gle, ime Jahve izdaleka dolazi, gnjev njegov gori, dim je neizdrživ. Usne su mu pune jarosti, jezik mu oganj što proždire. 
\par 28 Dah mu je kao potok nabujali što do grla seže. On dolazi da prosije narode rešetom zatornim, da stavi uzde zavodljive u čeljusti naroda. 
\par 29 Tad će vam pjesma biti kao u noćima blagdanskim, kad su srca vesela kao u onoga koji uza zvuke frule hodočasti na Goru Jahvinu, k Stijeni Izraelovoj. 
\par 30 Jahve će zagrmjet glasom veličajnim i pokazat ruku svoju što udara u jarosnu gnjevu, sred ognja zatornog, iz olujna pljuska i krupÄe kamene. 
\par 31 Od glasa Jahvina prepast će se Asur, šibom ošinut. 
\par 32 I kad god ga udari šiba kaznena, kojom će ga Jahve išibati, nek' se oglase bubnjevi i citare - u sav jek boja on s njima ratuje! 
\par 33 Odavna je pripravljen Tofet za Moleka - lomača visoka, široka, mnogo ognja, mnogo drvlja. Dah gnjeva Jahvina, kao potok sumporni, njega će spaliti. 


\chapter{31}

\par 1 Teško onima što slaze u Egipat po pomoć i nadu u konje polažu te se uzdaju u mnoga kola i u mnoštvo konjanika, ne gledajuć' s uzdanjem u Sveca Izraelova i od Jahve savjeta ne tražeć'. 
\par 2 Al' i on je mudar i navalit će zlo, i neće poreć' svojih prijetnja; on će ustat' na dom zlikovački i na pomoć zločinačku. 
\par 3 Egipćanin je čovjek, a ne Bog; konji su mu meso, a ne duh; kada Jahve rukom mahne, posrnut će pomagač i past će onaj komu pomaže - svi će zajedno propasti. 
\par 4 Da, ovako mi reče Jahve: Kao što lav ili lavić nad plijenom reži, pa i kad se strči na njega mnoštvo pastira, on se ne prepada vike njihove, nit' za njihovu graju mari - tako će Jahve nad Vojskama sići da vojuje za goru Sion, za visinu njezinu. 
\par 5 Kao ptice što lepršaju krilima, Jahve nad Vojskama zaklanjat će Jeruzalem, zaklanjat' ga, izbaviti, poštedjet' ga i spasiti. 
\par 6 Vratite se onome od kog otpadoste tako duboko, sinovi Izraelovi. 
\par 7 Da, u onaj će dan svatko prezreti svoje kumire srebrne i zlatne što ih rukama sebi za grijeh načiniste. 
\par 8 Asur neće pasti od mača ljudskoga: proždrijet će ga mač, ali ne čovječji. Od mača će bježat', al' će mu satnici pod tlaku pasti. 
\par 9 Užasnut, ostavit će svoju hridinu, prestravljeni, knezovi od svoje će bježat' zastave - riječ je Jahve, čiji je oganj na Sionu i čija je peć u Jeruzalemu. 


\chapter{32}

\par 1 Evo po pravdi kralj kraljuje, po pravici vladaju knezovi: 
\par 2 svaki je kao zavjetrina, utočište od nevremena, kao u sušnoj zemlji potoci, kao sjena u žednoj pustari. 
\par 3 Oči vidovitih neće više biti slijepe, uši onih što čuju slušat će pozorno; 
\par 4 srce nerazumnih shvaćat će mudrost, mucavci će govorit' okretno i razgovijetno; 
\par 5 pokvarenjaka neće više zvati plemenitim, varalicu neće više držat' odličnikom. 
\par 6 Jer, pokvarenjak govori ludosti i srce mu bezakonje snuje, da počini zlodjela, da o Jahvi oholo govori; da gladnoga ostavi prazna želuca, da žednome napitak uskrati. 
\par 7 U varalice pakosno je oružje; on spletke samo kuje, da lažima upropasti uboge, pa i kad nevoljnik pravo dokazuje. 
\par 8 U plemenita nakane su plemenite i plemenito on djeluje. 
\par 9 Ustajte, žene nehajne, slušajte moj glas; kćeri lakoumne, čujte mi besjedu. 
\par 10 Za godinu i nekoliko dana drhtat ćete, lakoumnice, jer jematve neće biti, plodovi se neće brati. 
\par 11 Dršćite, nehajnice, strepite, lakoumnice, svucite se, obnažite, oko bedara kostrijet opašite! 
\par 12 Bijte se u prsa zbog ljupkih polja, plodnih vinograda; 
\par 13 zbog njiva naroda mojega što rađaju trnjem i dračem; zbog svih kuća veselih, grada razigranog. 
\par 14 Jer, napuštena bit će palača, opustjet će bučni grad; Ofel i kula postat će brlog dovijeka - bit će radost divljim magarcima, paša stadima, 
\par 15 dok se na nas ne izlije duh iz visina. Tad će pustinja postat' voćnjak, a voćnjak se u šumu pretvorit'. 
\par 16 U pustinji će se nastaniti pravo, i pravda će prebivati u voćnjaku. 
\par 17 Mir će biti djelo pravde, a plod pravednosti - trajan pokoj i uzdanje. 
\par 18 Narod će moj prebivati u nastambama pouzdanim, u bezbrižnim počivalištima. 
\par 19 A šuma će biti oborena, grad će biti snižen. 
\par 20 Blago vama: sijat ćete kraj svih voda, puštajući vola i magarca da slobodno idu! 


\chapter{33}

\par 1 Teško tebi koji nepustošen pustošiš,  koji pljačkaš nepljačkan, kad prestaneš, tebe će opustošiti, opljačkat' te kad prestaneš pljačkati. 
\par 2 Jahve, smiluj nam se, u te se uzdamo! Budi mišica naša svako jutro, naš spas u doba nevolje. 
\par 3 Od silna tutnja pobjegoše narodi, ti ustade, raspršiše se puci 
\par 4 i plijen se skuplja kao što se kupe šaške, na nj će navaliti kao jato skakavaca. 
\par 5 Uzvišen je Jahve jer u visini stoluje, on puni Sion pravom i pravednošću. 
\par 6 Pouzdan je tvoj vijek: mudrost i znanje spasonosno su blago - a strah Gospodnji njegovo bogatstvo. 
\par 7 Gle, stanovništvo Arielovo kuka po ulicama, glasnici mironosni plaču gorko. 
\par 8 Opustješe ceste, s putova nesta putnika; raskidaju se savezi, preziru se svjedoci, ni prema kome nema obzira. 
\par 9 Gine zemlja u žalosti, u stidu vene Libanon. Šaron je kao stepa, Bašan i Karmel ogolješe. 
\par 10 "Sada ću ustati", veli Jahve, "sada ću se dići, sada uzvisiti. 
\par 11 Začeste sijeno, rodit ćete slamu; dah moj proždrijet će vas kao oganj. 
\par 12 Narodi će biti sažgani u vapno, kao posječeno trnje što gori u vatri. 
\par 13 Čujte, vi koji ste daleko, što sam učinio, a vi koji ste blizu poznajte mi snagu!" 
\par 14 Na Sionu strepe grešnici, trepet spopada bezbožnika: "Tko će od nas opstati pred ognjem zatornim, tko će od nas opstati pred žarom vječnim?" 
\par 15 Onaj koji hodi u pravdi i pravo govori, koji prezire dobit od prinude, koji otresa ruku da ne primi mito, koji zatiskuje uši da ne čuje o krvoproliću, koji zatvara oči da ne vidi zlo: 
\par 16 on će prebivati u visinama, tvrđe na stijenama bit će mu utočište, imat će dosta kruha i vode će mu svagda dotjecati. 
\par 17 Oči će ti gledati kralja u njegovoj ljepoti, promatrat će zemlju nepreglednu. 
\par 18 Srce će ti u strahu misliti: Gdje li je onaj što je brojio, gdje li onaj što je mjerio, gdje li onaj što je prebrajao mladiće? 
\par 19 Nećeš više vidjeti divljega naroda, naroda nerazumljiva i neshvatljiva govora, jezika strana što ga nitko ne razumije. 
\par 20 Pogledaj na Sion, grad blagdana naših: oči će ti Jeruzalem vidjeti, prebivalište zaštićeno, šator što se ne prenosi, kojem se kolčići nikad ne vade, nit' mu se ijedno uže otkida. 
\par 21 Ondje nam je Jahve silni, umjesto rijeka i širokih rukavaca: neće onud proći nijedna lađa s veslima niti će koji bojni brod projedriti. 
\par 22 Jer Jahve je sudac naš, Jahve naš vojvoda, Jahve je kralj naš - on će nas spasiti. 
\par 23 Užad ti je popustila, ne može držati jarbola ni razviti stijega, pa se dijeli golemo blago oteto - kljasti će se naplijeniti plijena! 
\par 24 I nijedan građanin neće reći: "Bolestan sam!" Narodu što živi ondje krivnja će se oprostiti. 


\chapter{34}

\par 1 Pristupite, puci, da čujete, pomno slušajte, narodi; čuj, zemljo, i sve što te ispunja, kruže zemaljski i sve što raste po tebi! 
\par 2 Jer razgnjevi se Jahve na sve narode, razjari se na svu vojsku njihovu. Izruči ih uništenju, pokolju ih predade. 
\par 3 Leže njihovi pobijeni, smrad se diže od trupla mnogih, krv gorama proteče, 
\par 4 raspade se sva vojska nebeska. Nebesa se sviše kao knjiga i pada sva njihova vojska k'o što lozov list otpada, k'o što se trusi lišće smokovo. 
\par 5 Jer na nebu je opijeni mač moj: gle, na Edom on se obara da kazni narod što ga prokleh. 
\par 6 Mač Jahvin krvlju je opijen, omašćen pretilinom, krvlju janjećom i jarećom, pretilinom bubrega ovnujskih. Jer Jahvi se u Bosri žrtva prinosi, veliko klanje u zemlji edomskoj. 
\par 7 S njima će biti poklani bivoli i junad s bikovima. Zemlja će se njihovom napojiti krvlju, i prašina njihova omastit' pretilinom, 
\par 8 jer Jahvi je ovo dan odmazde, godina naplate da Sion osveti. 
\par 9 Potoci se njegovi obrću u smolu, prašina njegova u sumpor, i zemlja će mu postat smola goreća. 
\par 10 Ni noću ni danju ugasit' se neće, dim će joj se dizati dovijeka, iz koljena u koljeno pusta će ostati, za vjekove vjekova nitko neće prolaziti njome. 
\par 11 Zaposjest će je jež i čaplja, sova i gavran prebivat će u njoj. Rastegnut će nad njom uže pustoši i visak praznine. 
\par 12 Ondje će se nastaniti jarci, neće biti više plemića njezinih, ondje se više neće proglašavat' kraljevi, svi će joj knezovi biti uništeni. 
\par 13 Nići će trnje u njenim dvorcima, u tvrđavama kopriva i stričak, ona će biti jazbina čagljima, ležaj nojevima. 
\par 14 Ondje će se sretat divlje mačke s hijenama, jarci će dozivati jedan drugoga; ondje će se odmarati Lilit našav počivalište. 
\par 15 Ondje će se gnijezditi guja, odlagati jaja, ležat' na njima, u sjeni ih tvojoj izleći; onamo će slijetati jastrebovi jedan za drugim. 
\par 16 Istražujte u knjizi Jahvinoj i čitajte, nijedno od tog ne izosta, jer usta njegova tako narediše i duh njegov njih sakupi. 
\par 17 Jer on im je ždrijeb bacio i ruka im njegova užetom zemlju odmjeri: dovijeka će je oni posjedovati, od koljena do koljena nju će obitavati. 


\chapter{35}

\par 1 Nek' se uzraduje pustinja, zemlja sasušena, neka kliče stepa, nek' ljiljan procvjeta. 
\par 2 Nek' bujno cvatom cvate, da, neka od veselja kliče i nek' se raduje. Dana joj je slava Libanona, divota Karmela i Šarona; oni će vidjeti slavu Jahvinu, divotu Boga našega. 
\par 3 Ukrijepite ruke klonule, učvrstite koljena klecava! 
\par 4 Recite preplašenim srcima: "Budite jaki, ne bojte se! Evo Boga vašega, odmazda dolazi, Božja naplata, on sam hita da nas spasi!" 
\par 5 Sljepačke će oči progledati, uši će se gluhih otvoriti, 
\par 6 tad će hromi skakati k'o jelen, njemakov će jezik klicati. Jer će u pustinji provreti voda, i u stepi potoci, 
\par 7 sažgana će zemlja postat' jezero, a tlo žedno - izvori. U brlozima gdje ležahu čaglji izrast će rogoz i trska. 
\par 8 Bit će ondje čista cesta, a zvat će se Sveti put: nitko nečist njime neće proći, bezumnici njime neće lutati. 
\par 9 Ondje neće više biti lÓava, nit' će onud zvijer prolaziti, već će hodit' samo otkupljeni, 
\par 10 vraćati se otkupljenici Jahvini. Doći će u Sion kličuć' od radosti, s veseljem vječnim na čelima; pratit će ih radost i veselje, pobjeći će bol i jauci. 


\chapter{36}

\par 1 Četrnaeste godine Ezekijina kraljevanja asirski kralj Sanherib  napade sve utvrđene judejske gradove i osvoji ih. 
\par 2 Tada pošalje  asirski kralj iz Lakiša u Jeruzalem kralju Ezekiji velikoga peharnika  s jakom vojskom. On stade kod vodovoda Gornjeg ribnjaka, na putu  u Valjarevo polje. 
\par 3 K njemu iziđe upravitelj dvora Elijakim, sin Hilkijin, pisar Šebna i savjetnik Joah, sin Asafov. 
\par 4 Veliki im peharnik reče: "Kažite Ezekiji: Ovako govori  veliki kralj, kralj asirski: 'Kakvo je to uzdanje u koje se uzdaš? 
\par 5 Misliš li da su prazne riječi već i savjet i snaga za rat?  U koga se uzdaš da si se pobunio protiv mene? 
\par 6 Eto, oslanjaš  se na Egipat, na slomljenu trsku koja prodire i probada dlan  onomu tko se na nju nasloni. Takav je faraon, kralj egipatski, svima koji se uzdaju u njega.' 
\par 7 Možda ćete mi odgovoriti:  'Uzdamo se u Jahvu, Boga svojega.' Ali nije li njemu Ezekija  uklonio uzvišice i žrtvenike i zapovjedio Judejcima i Jeruzalemu:  'Samo se pred ovim žrtvenikom klanjajte!' 
\par 8 Hajde, okladi se  s mojim gospodarom, kraljem asirskim: dat ću ti dvije tisuće  konja ako mogneš naći jahače za njih. 
\par 9 Kako ćeš onda odoljeti  jednome jedinom od najmanjih slugu moga gospodara? Ali ti se  uzdaš u Egipat da će ti dati kola i konjanike! 
\par 10 Naposljetku, zar sam ja mimo volju Jahvinu krenuo protiv ove zemlje da je  razorim? Sam mi je Jahve rekao: 'Idi na tu zemlju i razori je!'" 
\par 11 Elijakim, Šebna i Joah rekoše velikom peharniku: "Molimo  te, govori svojim slugama aramejski, jer mi razumijemo; ne govori  s nama judejski da čuje narod koji je na zidinama." 
\par 12 Ali im  veliki peharnik odgovori: "Zar me moj gospodar poslao da ovo  kažem tvojem gospodaru i tebi, a ne upravo onim ljudima koji  sjede na zidinama, osuđeni da s vama jedu svoju nečist i piju  svoju mokraću?" 
\par 13 Tada se veliki peharnik uspravi i u sav glas povika na  judejskom ove riječi: "Čujte riječ velikoga kralja, kralja asirskoga! 
\par 14 Ovako veli kralj: 'Neka vas Ezekija ne zavarava, jer vas  ne može izbaviti iz moje ruke. 
\par 15 Neka vas Ezekija ne hrabri  pouzdanjem u Jahvu govoreći: Jahve će nas sigurno izbaviti: ovaj  grad neće pasti u ruke kralju asirskom. 
\par 16 Ne slušajte Ezekije, jer ovako veli asirski kralj: Sklopite mir sa mnom, predajte  mi se, pa neka svaki od vas jede plodove iz svoga vinograda i  sa svoje smokve i neka pije vode iz svojega studenca 
\par 17 dok  ne dođem i ne odvedem vas u zemlju kao što je vaša, u zemlju  pšenice i mošta, u zemlju kruha i vinograda. 
\par 18 Ne dajte da  vas Ezekija zaludi govoreći vam: Jahve će vas izbaviti. Jesu  li bogovi drugih naroda izbavili svoje zemlje iz ruku asirskoga  kralja? 
\par 19 Gdje su bogovi hamatski i arpadski, gdje su bogovi  sefarvajimski, gdje su bogovi samarijski da izbave Samariju iz  moje ruke? 
\par 20 Koji su među svim bogovima tih zemalja izbavili  svoju zemlju iz moje ruke, da bi Jahve izbavio Jeruzalem iz ruke  moje?'" 
\par 21 Šutjeli su i ni riječi mu nisu odgovorili, jer kralj  bijaše zapovjedio: "Ne odgovarajte mu!" 
\par 22 Upravitelj dvora  Elijakim, sin Hilkijin, pisar Šebna i savjetnik Joah, sin Asafov, dođoše k Ezekiji, razdrijevši haljine, i saopćiše mu riječi  velikoga peharnika. 


\chapter{37}

\par 1 Čuvši to, kralj Ezekija razdrije svoje haljine, obuče kostrijet  i ode u Dom Jahvin. 
\par 2 Zatim posla Elijakima, upravitelja dvora, kraljevskog pisara Šebnu i svećeničke starješine, odjevene u  kostrijet, k proroku Izaiji, sinu Amosovu. 
\par 3 Oni mu rekoše:  "Ovako veli Ezekija: 'Ovo je dan nevolje, kazne i rugla: prispješe  djeca do porođaja, a nema snage da se rode. 
\par 4 Možda je Jahve, Bog tvoj, čuo riječi velikog peharnika koga je kralj asirski, gospodar njegov, poslao da se izruguje Bogu živome, i možda  će Jahve, Bog tvoj, kazniti riječi koje je čuo! Pomoli se pobožno  za Ostatak koji je još preostao!'" 
\par 5 Kad su sluge kralja Ezekije stigle k Izaiji, 
\par 6 on im  reče: "Kažite svome gospodaru: 'Ovako veli Jahve: Ne boj se riječi  koje si čuo kada su na me hulile sluge kralja asirskoga. 
\par 7 Udahnut  ću u njega duh, i kad čuje jednu vijest, vratit će se u svoju  zemlju. I učinit ću da u svojoj zemlji pogine od mača.'" 
\par 8 Veliki peharnik vrati se i nađe asirskoga kralja gdje  opsjeda Libnu, jer bijaše čuo da je kralj otišao iz Lakiša. 
\par 9 Dočuo  je, naime, da je Tirhaka, etiopski kralj, zavojštio na njega. Tada Sanherib ponovo uputi poslanike da kažu Ezekiji: 
\par 10 "Ovako  recite judejskom kralju Ezekiji: 'Neka te ne vara tvoj Bog, u  koga se uzdaš, govoreći ti: Jeruzalem neće pasti u ruke asirskog  kralja. 
\par 11 Ti znaš što su kraljevi asirski učinili svim zemljama  izručivši ih prokletstvu! A ti, ti li ćeš se spasiti? 
\par 12 Jesu  li bogovi spasili narode koje su uništili moji oci: Gozance,  Harance, Resefce i Edence, u Tel Basaru? 
\par 13 Gdje je kralj hamatski, kralj arpadski, kralj Sefarvajima, Hene i Ive?'" 
\par 14 Ezekija primi pismo iz ruke poslanikove i pročita ga.  Zatim uđe u Dom Jahvin i razvi ga ondje pred Jahvom. 
\par 15 I pomoli  se Ezekija Jahvi ovako: 
\par 16 "Jahve nad Vojskama, Bože Izraelov, koji stoluješ nad kerubima, ti si Bog jedini nad svim kraljevstvima  na zemlji, ti si stvorio nebo i zemlju. 
\par 17 Prikloni uho, Jahve, i počuj! Otvori oči, Jahve, i vidi! Sanheribove čujder riječi koje poruči da izruga Boga živoga. 
\par 18 Istina je, o Jahve, asirski su kraljevi zatrli sve narode  i zemlje njihove; 
\par 19 pobacali im u oganj bogove, jer ne bijahu  bogovi to, već djela ruku ljudskih, od drva i kamena; zato ih  i uništiše. 
\par 20 Ali sada, Jahve, Bože naš, izbavi nas iz ruke  njegove, da spoznaju sva kraljevstva zemlje da si ti, Jahve,  Bog jedini!" 
\par 21 Tad Izaija, sin Amosov, poruči Ezekiji: "Ovako veli Jahve, Bog Izraelov: 'Uslišah molitvu koju mi uputi zbog Sanheriba, kralja asirskoga.' 
\par 22 Evo riječi što je Jahve objavi protiv  njega:  Prezire te, ruga ti se, djevica, Kći sionska; za tobom maše glavom kći jeruzalemska. 
\par 23 Koga si grdio, hulio? Na koga si glasno vikao, ohol pogled dizao? Na Sveca Izraelova! 
\par 24 Po slugama si svojim vrijeđao Gospoda. Govorio si: s mnoštvom kola ja popeh se na vrh gorÄa, na najviše vrhunce Libanona. Posjekoh mu ja cedre najviše i čemprese ponajljepše. Dosegoh mu vrh najviši, i vrt njegov šumoviti. 
\par 25 Kopao sam i pio sam vode tuđe; stopalima tad isuših sve rijeke egipatske. 
\par 26 Čuješ li dobro? Odavna to sam snovao, odiskona smišljao, sada to ostvarujem: na tebi je da prometneš gradove tvrde u razvaline; 
\par 27 stanovnici njini, nemoćni, prepadnuti i smeteni, bjehu kao trava u polju kao mlado zelenilo, kao trava vrh krovova opaljena vjetrom istočnim. 
\par 28 Znam kad se dižeš i kad sjedaš, kad izlaziš i kad se vraćaš. 
\par 29 Jer bjesnio si na me i jer obijest tvoja do ušiju mi dođe, prsten ću ti provući kroz nozdrve, uzde stavit' u žvale, vratit ću te putem kojim si došao! 
\par 30 A znak nek' ti bude ovo: ove će se godine jesti što se samo okrÄunÄi, dogodine što samo uzraste, a treće godine sijte i žanjite, sadite vinograde, jedite im rod. 
\par 31 Preživjeli iz kuće Judine, žilje će pustit' u dubinu, plodom rodit' u visinu. 
\par 32 Jer će iz Jeruzalema izaći Ostatak. Sačuvani s gore Siona. Sve će to učinit' ljubomora Jahve nad Vojskama. 
\par 33 Zato ovo govori Jahve o kralju asirskom: U ovaj grad on ući neće, ovamo strijele svoje neće izmetati, k njemu neće ni štit okrenuti, niti oko njega nasipe kopati. 
\par 34 Vratit će se putem kojim je i došao, u grad ovaj neće ući - riječ je Jahvina. 
\par 35 Grad ću ovaj štitit, zakriliti ga, sebe radi i rad sluge svoga Davida." 
\par 36 Tad iziđe Anđeo Jahvin i pobi u asirskom taboru sto osamdeset  i pet tisuća ljudi. Ujutru, kad je valjalo ustati, gle, bijahu  ondje sve sami mrtvaci. 
\par 37 Sanherib podiže tabor i ode. Vratio se u Ninivu. 
\par 38 Jednoga  dana, dok se klanjao u hramu svoga boga Nimroka, njegovi ga sinovi  Adramelek i Sareser ubiše mačem i pobjegoše u zemlju araratsku.  Na njegovo se mjesto zakralji sin mu Asar-Hadon. 


\chapter{38}

\par 1 U ono se vrijeme Ezekija razbolje nasmrt. Prorok Izaija, sin  Amosov, dođe mu i reče: "Ovako veli Jahve: 'Uredi kuću svoju, jer ćeš umrijeti, nećeš ozdraviti.'" 
\par 2 Ezekija se okrenu zidu  i ovako se pomoli Jahvi: 
\par 3 "Ah, Jahve, sjeti se da sam pred  tobom hodio vjerno i poštena srca i učinio što je dobro u tvojim  očima." I Ezekija briznu u gorak plač. 
\par 4 Tada dođe riječ Jahvina Izaiji: 
\par 5 "Idi i reci Ezekiji:  Ovako kaže Jahve, Bog oca tvoga Davida: 'Uslišao sam tvoju molitvu, vidio tvoje suze. Izliječit ću te; za tri dana uzići ćeš u Dom  Jahvin. Dodat ću tvome vijeku petnaest godina. 
\par 6 Izbavit ću  tebe i ovaj grad iz ruku asirskoga kralja. Jest, zakrilit ću  ovaj grad!'" 
\par 7 Izaija odgovori: "Evo ti znaka od Jahve da će učiniti što  je rekao: 
\par 8 sjenu koja je sišla po stupnjevima Ahazova sunčanika  vratit ću za deset stupnjeva natrag." I vrati se sunce deset  stupnjeva natrag po stupnjevima po kojima bijaše već sišlo. 
\par 9 Pjesan Ezekije, kralja judejskoga, kada se razbolio pa  ozdravio od svoje bolesti: 
\par 10 "Govorio sam: U podne dana svojih ja moram otići. Na vratima Podzemlja mjesto mi je dano za ostatak mojih ljeta. 
\par 11 Govorio sam: Vidjet više neću Jahve na zemlji živih, vidjet više neću nikoga od stanovnika ovog svijeta. 
\par 12 Stan je moj razvrgnut, bačen daleko, kao šator pastirski; poput tkalca moj si život namotao da bi me otkinuo od osnove. Od jutra do noći skončat ćeš me, 
\par 13 vičem sve do jutra; kao što lav mrska kosti moje, od jutra do noći skončat ćeš me. 
\par 14 Poput laste ja pijučem, zapomažem kao golubica, uzgor mi se okreću oči, zauzmi se, jamči za me. 
\par 15 Kako ću mu govoriti i što ću mu reći? TÓa on je koji djeluje. Slavit ću te sva ljeta svoja, premda s gorčinom u duši. 
\par 16 Gospodine, za tebe živjet će srce moje i živjet će moj duh. Ti ćeš me izliječiti i vratiti mi život, 
\par 17 bolest će mi se pretvorit' u zdravlje. Ti si spasio dušu moju od jame uništenja, za leđa si bacio sve moje grijehe. 
\par 18 Jer Podzemlje ne slavi te, ne hvali te Smrt; oni koji padnu u rupu u tvoju se vjernost više ne uzdaju. 
\par 19 Živi, živi, jedino on te slavi kao ja danas. Otac naučava sinovima tvoju vjernost. 
\par 20 U pomoć mi, Jahve priteci, i mi ćemo pjevati uz harfe sve dane svojega života pred Hramom Jahvinim." 
\par 21 Izaija naloži: "Donesite oblog od smokava, privijte mu  ga na čir i on će ozdraviti." 
\par 22 Ezekija upita: "Po kojem ću  znaku prepoznati da ću uzići u Dom Jahvin?" 


\chapter{39}

\par 1 U to vrijeme posla babilonski kralj Merodak Baladan, sin Baladanov, pisma s darom Ezekiji, jer bijaše čuo da se razbolio i ozdravio. 
\par 2 Ezekija se obradova tome i pokaza poslanicima svoju riznicu  - srebro, zlato, miomirise i mirisavo ulje - svoju oružanu i  sve što je bilo u skladištima. Nije bilo ničega u njegovu dvoru  i u svemu njegovu gospodarstvu što im Ezekija nije pokazao. 
\par 3 Tada prorok Izaija dođe kralju Ezekiji i upita ga: "Što  su rekli ti ljudi i odakle su došli k tebi?" Ezekija odgovori:  "Došli su iz daleke zemlje, iz Babilona." 
\par 4 Izaija upita dalje:  "Što su vidjeli u tvojem dvoru?" Ezekija odgovori: "Vidjeli su  sve što je u mojem dvoru; nema u mojim skladištima ničega što  im nisam pokazao." 
\par 5 Tad Izaija reče Ezekiji: "Čuj riječ Jahve nad Vojskama: 
\par 6 'Evo dolaze dani kada će sve što je u tvojem dvoru, sve što  su tvoji oci nakupili do danas, biti odneseno u Babilon. Ništa  neće ostati,' kaže Jahve. 
\par 7 'A od sinova koji poteku od tebe, koji ti se rode, neke će uzeti da budu uškopljeni dvorani babilonskoga  kralja.'" 
\par 8 Ezekija odgovori Izaiji: "Povoljna je riječ koju  ti je Jahve objavio." A mislio je: "Bit će barem mira i sigurnosti  za moga života." 


\chapter{40}

\par 1 "Tješite, tješite moj narod, govori Bog vaš. 
\par 2 Govorite srcu Jeruzalema, vičite mu da mu se ropstvo okonča, da mu je krivnja okajana, jer iz Jahvine ruke primi dvostruko za sve grijehe svoje." 
\par 3 Glas viče: "Pripravite Jahvi put kroz pustinju. Poravnajte u stepi stazu Bogu našemu. 
\par 4 Nek' se povisi svaka dolina, nek' se spusti svaka gora i brežuljak. Što je neravno, nek' se poravna, strmine nek' postanu ravni. 
\par 5 Otkrit će se tada Slava Jahvina i svako će je tijelo vidjeti, jer Jahvina su usta govorila." 
\par 6 Glas nalaže: "Viči!" Odgovorih: "Što da vičem?" - "Svako je tijelo k'o trava, k'o cvijet poljski sva mu dražest. 
\par 7 Sahne trava, vene cvijet, kad dah Jahvin preko njih prođe. Doista, narod je trava. 
\par 8 Sahne trava, vene cvijet, ali riječ Boga našeg ostaje dovijeka." 
\par 9 Na visoku se uspni goru, glasniče radosne vijesti, Sione! Podigni snažno svoj glas, glasniče radosne vijesti, Jeruzaleme! Podigni ga, ne boj se, reci judejskim gradovima: "Evo Boga vašega!" 
\par 10 Gle, Gospod Jahve dolazi u moći, mišicom svojom vlada! Evo s njim naplata njegova, a ispred njega njegova nagrada. 
\par 11 Kao pastir pase stado svoje, u ruke uzima jaganjce, nosi ih u svome naručju i brižljivo njeguje dojilice. 
\par 12 Tko je šakom izmjerio more i nebesa premjerio pedljem? Tko je mjericom izmjerio zemlju i planine na mjerila, a tezuljom bregove? 
\par 13 Tko je pokrenuo duh Jahvin, koji ga je uputio savjetnik? 
\par 14 S kim se on posvjetova, tko je njemu mudrost ulio, naučio ga putovima pravde? Tko li ga je naučio znanju, pokazao mu put k umnosti? 
\par 15 Gle, narodi su kao kap iz vjedra, vrijede kao prah na tezulji. Otoci, gle, lebde poput truna! 
\par 16 Libanon je malen za lomaču, a zvijeri njegovih nema dosta za paljenicu. 
\par 17 Svi narodi k'o ništa su pred njim, ništavilo su njemu i praznina. 
\par 18 S kime ćete prispodobit' Boga? I s kakvim ga likom usporedit'? 
\par 19 Ljevač lijeva idol, zlatar ga pozlaćuje i lijeva od srebra lančiće. 
\par 20 Siromah za prinos bira drvo koje ne trune; i vješta traži umjetnika. da mu načini kip nepomičan. 
\par 21 Zar ne znate? Zar niste čuli? Nije li vam odiskona otkriveno? Zar niste shvatili tko zasnova zemlju? 
\par 22 On stoluje vrh kruga zemaljskoga kom su stanovnici poput skakavaca. Kao zastor nebesa je razastro, kao šator za stan razapeo. 
\par 23 On obraća u ništa knezove, uništava suce zemaljske. 
\par 24 Tek što su zasađeni, tek što su posijani, tek što im stabljika u zemlju korijen pruži, on puhne na njih i oni posahnu, vihor ih k'o pljevu raznese. 
\par 25 "S kime ćete mene prispodobit', tko mi je ravan?" - kaže Svetac. 
\par 26 Podignite oči i gledajte: tko je to stvorio? Onaj koji na broj izvodi vojsku njihovu i koji ih sve zove po imenu. 
\par 27 Zašto kažeš, Jakove, i ti, Izraele, govoriš: "Moj put sakriven je Jahvi, Bogu mom izmiče moja pravica?" 
\par 28 Zar ne znaš? Zar nisi čuo? Jahve je Bog vječni, krajeva zemaljskih stvoritelj. On se ne umara, ne sustaje, i um je njegov neizmjerljiv. 
\par 29 Umornome snagu vraća, jača nemoćnoga. 
\par 30 Mladići se more i malakšu, iznemogli, momci posrću. 
\par 31 Al' onima što se u Jahvu uzdaju snaga se obnavlja, krila im rastu kao orlovima, trče i ne sustaju, hode i ne more se. 


\chapter{41}

\par 1 Umuknite preda mnom, otoci, nek' novu snagu narodi priberu. Nek' se primaknu i progovore; zajedno pristupimo k sudu. 
\par 2 "Tko je podigao s Istoka onog kog ukorak prati Pobjeda? Tko mu izručuje narode i kraljeve podlaže? Prah su pod mačem njegovim, k'o pljevu ih njegov luk raspršuje. 
\par 3 Goni ih, napreduje pouzdano, noge mu se ceste ne dotiču. 
\par 4 Tko je to učinio i izvršio? Onaj koji odiskona zove naraštaje, ja, Jahve, koji sam prvi i bit ću ovaj isti s posljednjima!" 
\par 5 Otoci gledaju i strah ih obuzima, dršću krajevi zemaljski, oni se bliže i već su tu. 
\par 6 Svatko pomaže svome drugu i bratu svom zbori: "Hrabro!" 
\par 7 Ljevač bodri zlatara, onaj koji gladi čekićem bodri onog koji kuje na nakovnju. On govori o spajanju: "Dobro je", i čavlima kip učvršćuje da se ne pomiče. 
\par 8 Ti, Izraele, slugo moja, Jakove, kog sam izabrao, potomče Abrahama, mojega ljubimca! 
\par 9 Ti koga uzeh s krajeva zemlje i pozvah s rubova njenih, ti kome rekoh: "Ti si sluga moj, izabrao sam te i nisam te odbacio." 
\par 10 Ne boj se jer ja sam s tobom; ne obaziri se plaho jer ja sam Bog tvoj. Ja te krijepim i pomažem ti, podupirem te pobjedničkom desnicom. 
\par 11 Gle, postidjet će se i smesti svi koji su na tebe bjesnjeli, bit će uništeni i propast će oni što se s tobom parbiše! 
\par 12 Tražit ćeš svoje protivnike, ali ih nećeš naći. Bit će uništeni, svedeni na ništa oni koji protiv tebe vojuju. 
\par 13 Jer ja, Jahve, Bog tvoj, krijepim desnicu tvoju i kažem ti: "Ne boj se, ja ti pomažem." 
\par 14 Ne boj se, Jakove, crviću, Izraele, ličinko, ja sam pomoć tvoja - riječ je Jahvina - Svetac Izraelov tvoj je otkupitelj. 
\par 15 Gle, činim te mlatilom oštrim, novim, dvostrukih zubaca; mlatit ćeš i satirati brda, u prah ćeš pretvoriti bregove. 
\par 16 Vijat ćeš ih, vjetar će ih odnijeti, vihor će ih raspršiti. A ti ćeš kliktati u Jahvi, dičit ćeš se Svecem Izraelovim. 
\par 17 Ubogi i bijedni vodu traže, a nje nema! Jezik im se osuši od žeđi. Ja, Jahve, njih ću uslišiti, ja, Bog Izraelov, ostavit' ih neću. 
\par 18 U goleti bregova otvorit ću rijeke i posred dolova izvore. Pustinju ću pretvoriti u močvaru, a u vrela sušnu zemlju. 
\par 19 Posadit ću u pustinji cedar, bagrem, mirtu i maslinu. Stepu ću pošumiti čempresom, brijestom i šimširom zajedno. 
\par 20 Nek' svi vide i nek' znaju, nek' promisle i nek' shvate: ruka Jahvina to učini, Svetac Izraelov stvori sve. 
\par 21 "Iznesite svoj spor, kaže Jahve, predočite dokaze, kaže kralj Jakovljev. 
\par 22 Nek' pristupe i nek' nam objave ono što će se zbiti. TÓa što su nam otkrili o onom što bijaše, što se ispunilo, da o tom mislimo? Il' objavite što će biti, da doznamo ono što dolazi. 
\par 23 Otkrijte nam što će se poslije zbiti, i poznat ćemo da ste bogovi. Učinite nešto, dobro ili zlo, da se začudimo i prepadnemo zajedno. 
\par 24 Ali vi niste ništa i djela su vam ništavna, gnusan je koji vas izabere." 
\par 25 Podigoh ga sa sjevera da dođe, zazvah ga po imenu s istoka. Kao blato gazio je namjesnike, kao što po glini lončar gazi. 
\par 26 Tko je to odiskona objavio da bismo znali, unaprijed prorekao da bismo rekli: istina je? Ali nikog nema tko bi objavio, niti koga da bi navijestio, niti koga da čuje riječi vaše. 
\par 27 Ja prvi rekoh Sionu: "Gle, evo ih"; prvi Jeruzalemu poslah glasnika vijesti radosne. 
\par 28 Gledao sam, ali ne bješe nikoga, ni jednoga od njih da savjet dade, da ih pitam i da odgovore. 
\par 29 Svi zajedno ništa su, ništavna su djela njihova, vjetar i ispraznost njihovi kipovi. 


\chapter{42}

\par 1 Evo Sluge mojega koga podupirem,  mog izabranika, miljenika duše moje. Na njega sam svoga duha izlio da donosi pravo narodima. 
\par 2 On ne viče, on ne diže glasa, niti se čuti može po ulicama. 
\par 3 On ne lomi napuknutu trsku niti gasi stijenj što tinja. Vjerno on donosi pravdu, 
\par 4 ne sustaje i ne malakše dok na zemlji ne uspostavi pravo. Otoci žude za njegovim naukom. 
\par 5 Ovako govori Jahve, Bog, koji stvori i razastrije nebesa, koji rasprostrije zemlju i njeno raslinje, koji dade dah narodima na njoj i dah bićima što njome hode. 
\par 6 Ja, Jahve, u pravdi te pozvah, čvrsto te za ruku uzeh; oblikovah te i postavih te za Savez narodu i svjetlost pucima, 
\par 7 da otvoriš oči slijepima, da izvedeš sužnje iz zatvora, iz tamnice one što žive u tami. 
\par 8 Ja, Jahve mi je ime, svoje slave drugom ne dam, niti časti svoje kipovima. 
\par 9 Što prije prorekoh, evo, zbi se, i nove događaje ja naviještam, i prije negoli se pokažu, vama ih objavljujem. 
\par 10 Pjevajte Jahvi pjesmu novu, i s kraja zemlje hvalu njegovu, neka ga slavi more sa svim što je u njem, otoci i njihovi žitelji! 
\par 11 Nek' digne glas pustinja i njeni gradovi, nek' odjeknu naselja gdje žive Kedarci! Nek' podvikuju stanovnici Stijene, neka kliču s gorskih vrhova! 
\par 12 Nek' daju čast Jahvi i hvalu mu naviještaju po otocima! 
\par 13 Kao junak izlazi Jahve, kao ratnik žar svoj podjaruje. Uz bojni poklik i viku ratnu ide junački na svog neprijatelja. 
\par 14 "Šutjeh dugo, gluh se činjah, svladavah se; sad vičem kao žena kada rađa, dašćem i uzdišem. 
\par 15 Isušit ću brda i bregove, sparušiti svu zelen po njima, rijeke ću u stepe pretvoriti i močvare isušiti. 
\par 16 Vodit ću slijepce po cestama, uputit' ih putovima. Pred njima ću tamu u svjetlost obratit', a neravno tlo u ravno. To ću učiniti i neću propustiti. 
\par 17 Uzmaknut će u golemu stidu koji se uzdaju u kipove, koji ljevenim likovima govore: 'Vi ste naši bogovi.'" 
\par 18 Čujte, gluhi! Progledajte, slijepi, da vidite! 
\par 19 Tko je slijep ako ne moj sluga, tko je gluh kao glasnik koga šaljem? Tko je slijep kao prijatelj, tko je gluh kao sluga Jahvin? 
\par 20 Mnogo si vidio, ali nisi mario, uši ti bjehu otvorene, ali nisi čuo! 
\par 21 Jahvi se svidjelo zbog njegove pravednosti da uzveliča i proslavi Zakon svoj. 
\par 22 A narod je ovaj opljačkan i oplijenjen, mladići mu stavljeni u klade, vrgnuti u zatvore. Plijene ih, a nikoga da ih izbavi; robe ih, a nitko da kaže: "Vrati!" 
\par 23 Tko od vas mari za to? Tko pazi i sluša unapredak? 
\par 24 Tko je pljačkašu izručio Jakova i otimačima Izraela? Nije li Jahve, protiv koga smo griješili, čijim putima ne htjedosmo hoditi, čiji Zakon nismo slušali? 
\par 25 Zato izli na Izraela žarki gnjev svoj i strahote ratne: plamen ga okruži odasvud, al' on ni to nije shvatio; sažeže ga, al' on ni to k srcu ne uze. 


\chapter{43}

\par 1 Sada ovako govori Jahve, koji te stvorio, Jakove, koji te sazdao, Izraele: "Ne boj se, jer ja sam te otkupio; imenom sam te zazvao: ti si moj! 
\par 2 Kad preko vode prelaziš, s tobom sam; ili preko rijeke, neće te preplaviti. Pođeš li kroz vatru, nećeš izgorjeti, plamen te opaliti neće. 
\par 3 Jer ja sam Jahve, Bog tvoj, Svetac Izraelov, tvoj spasitelj. Za otkupninu tvoju dajem Egipat, mjesto tebe dajem Kuš i Šebu. 
\par 4 Jer dragocjen si u mojim očima, vrijedan si i ja te ljubim. Stog i dajem ljude za tebe i narode za život tvoj. 
\par 5 Ne boj se jer ja sam s tobom. S istoka ću ti dovest' potomstvo i sabrat ću te sa zapada. 
\par 6 Reći ću sjeveru: 'Daj mi ga!' a jugu 'Ne zadržavaj ga!' Sinove mi dovedi izdaleka i kćeri moje s kraja zemlje, 
\par 7 sve koji se mojim zovu imenom i koje sam na svoju slavu stvorio, koje sam sazdao i načinio." 
\par 8 Izvedi narod slijep, premda oči ima, i gluh, premda uši ima. 
\par 9 Neka se saberu sva plemena i neka se skupe narodi. Tko je od njih to prorekao i davno navijestio? Nek' dovedu svjedoke da se opravdaju, neka se čuje da se može reći: "Istina je!" 
\par 10 Jer vi ste mi svjedoci, riječ je Jahvina, i moje sluge koje sam izabrao, da biste znali i vjerovali i uvidjeli da sam to ja. Prije mene nijedan bog nije bio načinjen i neće poslije mene biti. 
\par 11 Ja, ja sam Jahve, osim mene nema spasitelja. 
\par 12 Ja sam prorekao, spasio i navijestio, i nema među vama tuđinca! Vi ste mi svjedoci, riječ je Jahvina, a ja sam Bog 
\par 13 od vječnosti - ja jesam! I nitko iz ruke moje ne izbavlja; što učinim, tko izmijeniti može? 
\par 14 Ovako govori Jahve, otkupitelj vaš, Svetac Izraelov: "Radi vas poslah protiv Babilona, oborit ću prijevornice zatvorima i Kaldejci će udarit u kukanje. 
\par 15 Ja sam Jahve, Svetac vaš, stvoritelj Izraelov, kralj vaš!" 
\par 16 Ovako govori Jahve, koji put po moru načini i stazu po vodama silnim; 
\par 17 koji izvede bojna kola i konje, vojsku i junake, i oni padoše da više ne ustanu, zgasnuše, kao stijenj se utrnuše. 
\par 18 Ne spominjite se onog što se zbilo, nit' mislite na ono što je prošlo. 
\par 19 Evo, činim nešto novo; već nastaje. Zar ne opažate? Da, put ću napraviti u pustinji, a staze u pustoši. 
\par 20 Slavit će me divlje zvijeri, čaglji i nojevi, jer vodu ću stvorit' u pustinji, rijeke u stepi, da napojim svoj narod, izabranika svoga. 
\par 21 I narod koji sam sebi sazdao moju će kazivati hvalu! 
\par 22 Ali me ti, Jakove, nisi zazvao, niti si se zamorio oko mene, Izraele! 
\par 23 Nisi mi prinosio ovce za paljenicu, nisi me častio žrtvama. A ja te silio nisam na prinose, nisam ti dodijavao ištući kada. 
\par 24 Nisi mi kupovao za novac trsku, nisi me sitio salom svojih žrtava; nego si me grijesima svojim mučio, bezakonjem svojim dosađivao mi. 
\par 25 A ja, ja radi sebe opačine tvoje brišem i grijeha se tvojih ne spominjem. 
\par 26 Podsjeti me, zajedno se sporimo, govori ti da se opravdaš. 
\par 27 Prvi je otac tvoj sagriješio, posrednici tvoji od mene se odmetnuli, 
\par 28 knezovi su tvoji oskvrnuli Svetište. Tad izručih Jakova prokletstvu, i poruzi Izraela. 


\chapter{44}

\par 1 Sad čuj, Jakove, slugo moj, Izraele, kog sam izabrao. 
\par 2 Ovako kaže Jahve, koji te stvorio, koji te od utrobe sazdao i pomaže ti: "Ne boj se, Jakove, slugo moja, Ješurune, kog sam izabrao. 
\par 3 Jer na žednu ću zemlju vodu izliti i po tlu sušnome potoke. Izlit ću duh svoj na tvoje potomstvo i blagoslov na tvoja pokoljenja. 
\par 4 Rast će kao trava pokraj izvora, kao vrbe uz vode tekućice. 
\par 5 Jedan će reći: 'Ja sam Jahvin', drugi će se zvati imenom Jakovljevim. Treći će sebi na ruci napisati: 'Jahvin' i nazvat će se imenom Izraelovim." 
\par 6 Ovako govori kralj Izraelov i otkupitelj njegov, Jahve nad Vojskama: "Ja sam prvi i ja sam posljednji: osim mene Boga nema. 
\par 7 Tko je kao ja? Nek' ustane i govori, nek' navijesti i nek' mi razloži! Tko je od vječnosti otkrio što se zbilo? Nek' nam navijesti što će još doći! 
\par 8 Ne plašite se, ne bojte se: nisam li vam to odavna navijestio i otkrio? Vi ste mi svjedoci: ima li Boga osim mene? Ima li Stijene? Ja ne znam!" 
\par 9 Tko god pravi kipove, ništavan je, i dragocjenosti njegove  ne koriste ničemu. Svjedoci njihovi ništa ne vide i ništa ne  znaju, da im budu na sramotu. 
\par 10 Tko pravi boga i lijeva kip  da od toga korist ne očekuje? 
\par 11 Gle, svi će štovatelji likova  biti osramoćeni, izrađivači njihovi više od bilo koga. Nek' se  saberu svi i pojave: prepast će se i postidjeti odjednom. 
\par 12 Kovač ga izrađuje na živu ugljevlju, čekićem ga oblikuje, snažnom ga rukom obrađuje. Gladan je i iznemogao; ne pije vode, iscrpljuje se. 
\par 13 Drvodjelja uzima mjeru, pisaljkom lik ocrta, ostruže  ga dlijetom, šestarom ga zaokruži i izdjelja ga po uzoru na lik  ljudski, kao lijepo ljudsko obličje, da stoji u hramu. 
\par 14 Bijaše  sebi nasjekao cedre, uzeo čempres ili hrast koje je za se njegovao  među šumskim drvećem; ili je posadio bor koji raste od kiše. 
\par 15 Čovjeku su dobra za vatru; uzima ih da se ogrije; pali ih  da ispeče kruh. Ali od njih djelja i boga pred kojim pada ničice, pravi kip i klanja mu se. 
\par 16 Polovinom od toga naloži, dakle, oganj, peče meso na žeravi, jede pečenku i siti se: grije se  i govori: "Ah, grijem se i uživam uz vatru." 
\par 17 Ali od onoga  što preostane pravi sebi boga, svog kumira, pada pred njim ničice  i klanja mu se i moli: "Spasi me, jer si ti moj bog." 
\par 18 Ne znaju oni i ne razumiju: zaslijepljene su im oči pa  ne vide, i srce pa ne shvaćaju. 
\par 19 Takav ne razmišlja, nema  u njega znanja ni razbora da sebi kaže: "Polovinom od ovoga naložio  sam oganj, na žeravici ispekao kruh, ispržio meso koje sam pojeo, pa zar ću od ostatka načiniti gnusobu? Zar ću se komadu drveta  klanjati?" 
\par 20 On voli pepeo, zavodi ga prevareno srce. Neće spasti  svog života i nikad neće reći: "Nije li varka ovo u mojoj desnici?" 
\par 21 Sjeti se toga, Jakove, i ti, Izraele, jer si sluga moj! Ja sam te stvorio i sluga si mi, Izraele, neću te zaboraviti! 
\par 22 Kao maglu rastjerao sam tvoje opačine i grijehe tvoje poput oblaka. Meni se obrati jer ja sam te otkupio. 
\par 23 Kličite, nebesa, jer je Jahve učinio! Orite se, dubine zemljine! Odjekujte radošću, planine, i vi, šume, sa svim svojim drvećem! Jer Jahve je otkupio Jakova, proslavio se u Izraelu! 
\par 24 Ovako govori Jahve, otkupitelj tvoj i tvorac tvoj od utrobe: "Ja sam Jahve koji sam sve stvorio, koji sam nebesa sam razapeo i učvrstio zemlju bez pomoći ičije. 
\par 25 Ja osujećujem znamenja vrača, i čarobnjake u luđake promećem; silim mudrace da ustuknu i mudrost im obraćam u bezumlje, 
\par 26 ali potvrđujem riječ sluge svojega, ispunjam naum svojih glasnika. Ja govorim Jeruzalemu: 'Naseli se!' I gradovima judejskim: 'Sagradite se!' Iz razvalina ja ih podižem. 
\par 27 Ja govorim moru: 'Presahni! Presušujem ti rijeke.' 
\par 28 Ja govorim Kiru: 'Pastiru moj!' I on će sve želje moje ispuniti govoreći Jeruzalemu: 'Sagradi se!' i Hramu: 'Utemelji se!'" 


\chapter{45}

\par 1 Ovako govori Jahve o Kiru, pomazaniku svome: "Primih ga za desnicu da pred njim oborim narode i raspašem bokove kraljevima, da rastvorim pred njim vratnice, da mu nijedna vrata ne budu zatvorena. 
\par 2 Ja ću hoditi pred tobom da poravnam uzvisine, da razbijem mjedene vratnice, da slomim željezne prijevornice. 
\par 3 Dajem ti tajna blaga i skrivena bogatstva, da bi spoznao da sam ja Jahve koji te zovem po imenu, Bog Izraelov. 
\par 4 Radi sluge svog Jakova i Izraela, svog izabranika, po imenu ja te pozvah, imenovah te premda me znao nisi. 
\par 5 Ja sam Jahve i nema drugoga; osim mene Boga nema. Iako me ne poznaš, naoružah te: 
\par 6 nek' se znade od istoka do zapada da izvan mene sve je ništavilo." Ja sam Jahve i nema drugoga; 
\par 7 ja tvorim svjetlost i stvaram tamu. Ja stvaram sreću i dovodim nesreću, ja, Jahve, činim sve to. 
\par 8 Rosite, nebesa, odozgo, i oblaci, daždite pravednošću. Neka se rastvori zemlja da procvjeta spasenje, da proklija izbavljenje! Ja, Jahve, stvaram sve. 
\par 9 Jao onome tko raspravlja s tvorcem svojim, a sud je među glinenim sudovima! Kaže li glina lončaru: "Što radiš?" ili djelo njegovo: "Kljast si!" 
\par 10 Jao onom koji kaže ocu: "Što si rodio?" Ili ženi: "Što si na svijet dala?" 
\par 11 Ovako govori Jahve, Svetac Izraelov, njegov tvorac: "Zar je vaše da me o mojoj djeci pitate i da mi nad djelom ruku mojih zapovijedate? 
\par 12 Ja sam načinio zemlju i čovjeka na njoj stvorio; svojim sam rukama razapeo nebesa i zapovijedam svim vojskama njihovim. 
\par 13 Ja sam ga podigao da pobijedi i poravnao sam mu sve putove. On će obnoviti moj Grad i sužnje moje vratiti bez otkupnine i naknade." Tako kaže Jahve nad Vojskama. 
\par 14 Ovako govori Jahve: "Ratari Egipta i trgovci Kuša, i Sebejci, ljudi rasta visoka, prijeći će tebi i tvoji će biti; za tobom će ići okovani, tebi će se klanjati i molit će ti se: 'Jedino je kod tebe Bog, nema drugoga; ništavni su bogovi.'" 
\par 15 Doista ti si Bog skriveni, Bog Izraelov, Spasitelj. 
\par 16 Postidjet će se i poniknut će svi zajedno, otići će u ruglu oni koji prave kipove. 
\par 17 A Jahve će vječnim spasenjem spasit' Izraela. Nećete se postidjeti i nećete poniknuti dovijeka. 
\par 18 Da, ovako govori Jahve, nebesa Stvoritelj - on je Bog - koji je oblikovao i sazdao zemlju, koji ju je učvrstio i nije je stvorio pustu, već ju je uobličio za obitavanje: "Ja sam Jahve i nema drugoga. 
\par 19 Nisam govorio u tajnosti, u zakutku mračne zemlje. Nisam rekao potomstvu Jakovljevu: 'Tražite me u pustoši.' Ja, Jahve, govorim pravo i naviještam čestito." 
\par 20 "Saberite se i dođite, pristupite zajedno, svi preživjeli od naroda! Neznalice puke oni su što nose kip izrađen od drveta i mole boga koji ih spasit' ne može. 
\par 21 Objavite, iznesite svoje dokaze, svjetujte se zajedno: tko je to od davnine navijestio i od tada prorekao? Nisam li ja, Jahve? Nema drugoga boga do mene; Boga pravednog i Spasitelja osim mene nema. 
\par 22 Obratite se k meni da se spasite, svi krajevi zemlje, jer ja sam Bog i nema drugoga! 
\par 23 Sobom se samim kunem, iz mojih usta izlazi istina, riječ neopoziva, da će se preda mnom prignuti svako koljeno, mnome će se svaki jezik zaklinjati 
\par 24 govoreći: 'Jedino je u Jahvi pobjeda i snaga!'" K njemu će doći, postiđeni, svi što na nj su bjesnjeli. 
\par 25 U Jahvi će pobijediti i proslavit se sve potomstvo Izraelovo! 


\chapter{46}

\par 1 Pade Bel! Sruši se Nebo! Prte svoje kipove na životinje i stoku, nose ih kao breme, teret što zamara. 
\par 2 Padaju, ruše se svi zajedno, ne mogu spasiti one što ih nose, nego i sami u ropstvo odlaze. 
\par 3 "Slušajte me, kućo Jakovljeva, i svi koji ostadoste od kuće Izraelove! Ja sam vas ponio tek što se rodiste, i nosio vas od krila materina. 
\par 4 Do starosti vaše ja ću ostat' isti, do vaših sjedina podupirat ću vas. To sam činio; nosit ću vas i dalje, pomagati vas, izbavljati. 
\par 5 S kime biste me usporedili i izjednačili, s kime prispodobili: komu da sam sličan? 
\par 6 Vade zlato iz kese i tezuljom mjere srebro, pa naimlju zlatara da od njeg boga načini te pred njim padaju ničice i klanjaju se. 
\par 7 Dižu ga na rame i nose ga, onda ga stavljaju na njegovo mjesto; on stoji i ne miče se s njega. Prizivaju li ga, on ne odgovara i nikog ne spasava od nevolje njegove. 
\par 8 Sjetite se toga i budite ljudi, uzmite to k srcu, otpadnici, 
\par 9 sjetite se prošlosti pradavne: ja sam Bog i nema drugoga; Bog, nitko mi sličan nije! 
\par 10 Onaj sam koji od početka svršetak otkriva i unaprijed javlja što još se nije zbilo! Ja kažem: Odluka će se moja ispuniti, izvršit ću sve što mi je po volji. 
\par 11 S istoka zovem grabljivicu, iz daleke zemlje zovem čovjeka svog nauma. Rekao sam - ispunit ću, naumio sam - izvršit ću. 
\par 12 Slušajte me, vi koji gubite srčanost i koji ste daleko od pobjede. 
\par 13 Primičem svoju pobjedu, nije više daleko, spasenje moje zakasniti neće. Na Sion ću spas staviti, u Izraela svoju slavu." 


\chapter{47}

\par 1 Spusti se, sjedni u prašinu, djevice, kćeri babilonska! Sjedni na zemlju, bez prijestolja, kćeri kaldejska! Jer, neće te više zvati nježnom i tankoćutnom. 
\par 2 Uzmi mlin i melji brašno! Skini prijevjes, podigni skut, razgali bedra, prijeđi preko rijeke! 
\par 3 Nek' se golotinja tvoja otkrije, nek' se sramota tvoja pokaže! Ja ću se osvetiti, odvraćat' me nitko neće. 
\par 4 Otkupitelj naš, ime mu je Jahve nad Vojskama, Svetac Izraelov, kaže: 
\par 5 "Sjedi šutke, u mrak se povuci, kćeri kaldejska. Jer, neće te više zvati vladaricom kraljevstava. 
\par 6 Razgnjevih se na svoj narod, oskvrnuh svoju baštinu. Tebi ih u ruke izručih, a ti im ne iskaza milosti. Na starce si stavljala jaram svoj preteški. 
\par 7 Govorila si: 'Dovijeka gospodaricom ću ostati.' Nikad nisi to k srcu uzela ni pomislila kako će se završiti. 
\par 8 A sad poslušaj, razvratnice, koja sjediš bezbrižno i u srcu svom govoriš: 'Ja, i nitko drugi! Nikad neću obudovjeti, neću djece izgubiti!' 
\par 9 Stići će te oboje, za tren, u isti dan! Izgubit ćeš djecu, obudovjet ćeš! Punom će te mjerom snaći oboje, pokraj svega tvojeg vračanja i množine tvojih zaklinjanja! 
\par 10 U zloću si se svoju uzdala, govorila si: 'Nitko me ne vidi!' Mudrost tvoja i znanje zavedoše te. U svom si srcu govorila: 'Ja i nitko drugi!' 
\par 11 Zlo će te snaći - nećeš ga presresti; oborit će se na te nesreća - nećeš je odvratiti; doći će na te propast iznenada - nećeš je predvidjeti. 
\par 12 Ustraj, dakle, u svojim zaklinjanjima i u tolikim svojim čaranjima, oko kojih si se trudila od mladosti. Možda će ti biti od koristi? Možda ćeš s njima strah utjerati? 
\par 13 Izmoriše te mnogi tvoji savjetnici! Nek' ustanu samo da te spase oni koji premjeravaju nebesa, koji promatraju zvijezde i koji svakog mjeseca proriču ono što će te snaći. 
\par 14 Gle, oni će biti poput pljeve, oganj će ih sažeći. Ni sami sebe neće izbaviti iz zagrljaja plamenoga. Neće ostat' ni žerave da se tko ogrije, ni ognjišta da uza nj posjedne! 
\par 15 Takvi će ti biti vrači tvoji, oko kojih si se trudila od mladosti! Poći će svaki svojim putem, i nikog neće biti da te spasi." 


\chapter{48}

\par 1 Čujte ovo, kućo Jakovljeva, vi koji se zovete imenom Izraelovim i koji ste izišli iz voda Judinih! Vi koji se Jahvinim imenom kunete i slavite Boga Izraelova, ali ne u istini i pravdi. 
\par 2 Jer vi se nazivate po Svetome gradu i oslanjate se na Boga Izraelova, Jahve nad Vojskama njemu je ime. 
\par 3 Dogođaje prošle odavna sam navijestio, iz mojih su izišli usta i ja sam ih objavio, učinih brzo, i zbi se. 
\par 4 Jer znao sam da si tvrdokoran, da ti je šija žila gvozdena i čelo da ti je mjedeno. 
\par 5 Zato sam ti već onda navijestio, javio ti prije nego što se zbilo, da ne bi rekao: "Moj kip učini to, rezani moj lik i ljeveni kip zapovjediše tako!" 
\par 6 Čuo si i vidio sve to; zar ne priznaješ? A sada navijestit ću ti nešto novo, otajno, što još ne znaš; 
\par 7 ovog je trena stvoreno, a ne odavna, o tome dosad nisi ništa čuo, da ne bi rekao: "Znao sam." 
\par 8 Nisi o tome čuo ni znao, niti se uho tvoje prije otvorilo, jer znadoh da ćeš se iznevjeriti i da te od utrobe majčine zovu otpadnikom. 
\par 9 Radi imena svoga odgađah svoj gnjev, radi časti svoje susprezah se da te ne uništim. 
\par 10 Gle, pročistio sam te poput srebra, iskušao te u talioniku nevolje. 
\par 11 Sebe radi činih tako, sebe radi! TÓa zar da se ime moje obeščasti? Slave svoje drugome ne dam! 
\par 12 Čuj me, Jakove, Izraele, koga sam pozvao: Ja jesam, ja sam prvi, ja sam i posljednji. 
\par 13 Ruka moja zemlju utemelji, desnica mi razape nebesa. Pozovem ih samo, i odmah dolaze. 
\par 14 Saberite se svi i čujte: tko je od njih to prorekao? "Onaj koga Jahve ljubi ispunit će volju moju nad Babilonom i nad potomstvom kaldejskim." 
\par 15 Ja rekoh i pozvah ga, vodih ga i pomogoh u naumu. 
\par 16 Pristupite k meni i počujte ovo: "Od početka nisam vam govorio tajno, i kad se zbivalo, bijah ondje." - "A sada me Gospod Jahve šalje s duhom svojim." 
\par 17 Ovako govori Jahve, otkupitelj tvoj, Svetac Izraelov: "Ja, Jahve, Bog tvoj, tvojem dobru te učim, vodim te putem kojim ti je ići. 
\par 18 O, da si pazio na zapovijedi moje, kao rijeka sreća bi tvoja bila, a pravda tvoja kao morski valovi! 
\par 19 Potomstva bi tvojeg bilo kao pijeska, a poroda utrobe tvoje kao njegovih zrnaca! Nikad ti se ime ne bi zatrlo niti izbrisalo preda mnom!" 
\par 20 Izađite iz Babilona, bježite iz Kaldeje! Glasno kličući, kazujte, objavljujte, do nakraj zemlje razglasite! Govorite: "Jahve je otkupio slugu svoga Jakova! 
\par 21 Nisu žeđali dok ih je kroz pustinju vodio; iz stijene je za njih vodu izbio, rascijepio je pećinu i potekla je voda." 
\par 22 "Nema mira opakima," kaže Jahve. 


\chapter{49}

\par 1 Čujte me, otoci, slušajte pomno, narodi daleki! Jahve me pozvao od krila materina, od utrobe majke moje spomenuo se moga imena. 
\par 2 Od usta mojih britak mač je načinio, sakrio me u sjeni ruke svoje, od mene je oštru načinio strijelu, sakrio me u svome tobolcu. 
\par 3 Rekao mi: "Ti si Sluga moj, Izraele, u kom ću se proslaviti!" 
\par 4 A ja rekoh: "Zaludu sam se mučio, nizašto naprezao snagu." Ipak, kod Jahve je moje pravo, kod mog Boga nagrada je moja. A sad govori Jahve, koji me od utrobe Slugom svojim načini, da mu vratim natrag Jakova, da se sabere Izrael. 
\par 5 Proslavih se u očima Jahvinim, Bog moj bijaše mi snaga. 
\par 6 I reče mi: "Premalo je da mi budeš Sluga, da podigneš plemena Jakovljeva i vratiš Ostatak Izraelov, nego ću te postaviti za svjetlost narodima, da spas moj do nakraj zemlje doneseš." 
\par 7 Ovako govori Jahve, otkupitelj Izraelov, Svetac njegov, onome kog preziru i odbacuju narodi, sluzi silničkome: "Kad vide, dići će se kraljevi, bacit će se ničice knezovi, zbog Jahve, koji je vjernost svoju pokazao, Sveca Izraelova, koji te izabrao." 
\par 8 Ovako govori Jahve: "U vrijeme milosti ja ću te uslišiti, u dan spasa ja ću ti pomoći. Sazdao sam te i postavio za Savez narodu, da zemlju podignem, da nanovo razdijelim baštinu opustošenu, 
\par 9 da kažeš zasužnjenima: 'Iziđite!' a onima koji su u tami: 'Dođite na svjetlo!' Oni će pasti uzduž svih putova, i paša će im biti po svim goletima. 
\par 10 Neće više gladovat' i žeđati, neće ih mučiti žega ni sunce, jer vodit će ih onaj koji im se smiluje, dovest će ih k izvorima vode. 
\par 11 Sve gore svoje obratit će u putove, i ceste će se moje povisiti." 
\par 12 Gle, jedni dolaze izdaleka, drugi sa sjevera i sa zapada, a neki iz zemlje sinimske. 
\par 13 Kličite, nebesa, veseli se, zemljo, podvikujte, planine, od veselja; jer Jahve tješi narod svoj, on je milosrdan nevoljnima. 
\par 14 Sion reče: "Jahve me ostavi, Gospod me zaboravi." 
\par 15 "Može li žena zaboravit' svoje dojenče, ne imat' sućuti za čedo utrobe svoje? Pa kad bi koja i zaboravila, tebe ja zaboraviti neću. 
\par 16 Gle, u dlanove sam te svoje urezao, zidovi tvoji svagda su mi pred očima. 
\par 17 Obnovitelji tvoji hitaju, rušioci tvoji i pustošitelji odlaze od tebe. 
\par 18 Obazri se oko sebe i pogledaj: sabiru se svi i dolaze k tebi. Života mi moga" - riječ je Jahvina - "svima ćeš se njima zaodjenuti k'o nakitom, i njima ćeš se k'o nevjesta ukrasiti! 
\par 19 Jest, tvoje ruševine, tvoje razvaline i tvoja zemlja poharana pretijesna će biti žiteljima tvojim kad se udalje oni što te zatirahu. 
\par 20 Opet će na tvoje uši reći sinovi kojih si bila lišena: 'Pretijesno mi je mjesto ovo, makni se da se mogu smjestiti.' 
\par 21 I ti ćeš se u srcu svom zapitati: 'Tko mi rodi sve ove? Bijah bez djece, neplodna, prognana i odbačena, pa tko ih podiže? Bijah, eto, sama ostala, a oni gdje su bili?'" 
\par 22 Ovako govori Gospod Jahve: "Evo, dajem rukom znak narodima i zastavu svoju dižem plemenima. Vratit će ti u naručju sinove, nosit će ti kćeri na plećima. 
\par 23 Kraljevi će biti tvoji skrbnici, a kneginje im tvoje dojkinje. Klanjat će ti se licem do zemlje i prah će lizat s tvojih nogu. I znat ćeš da sam ja Jahve: koji se u me uzdaju, neće se posramiti." 
\par 24 Može li se otet plijen junaku? Može li sužanj pobjeć pobjedniku? 
\par 25 Da, ovako govori Jahve: "Bit će oduzet sužanj junaku, pobjeći će plijen pobjedniku! S onima koji se s tobom spore ja ću se sporiti, tvoju djecu ja ću izbaviti; 
\par 26 tlačiteljima ću tvojim dati njihovo meso za jelo i svojom krvlju opit će se kao moštom. I znat će svako tijelo da sam ja Jahve, Spasitelj tvoj, i da je tvoj okupitelj Silni Jakovljev." 


\chapter{50}

\par 1 Ovako govori Jahve: "Gdje je otpusno pismo vaše matere kojim sam je otpustio? Ili tko je od mojih vjerovnika taj komu sam vas prodao? Zbog bezakonja ste svojih prodani, zbog nevjere je mati vaša otpuštena. 
\par 2 Zašto ne nađoh nikoga kad sam došao? Zašto se ne odazva nitko kad sam zazvao? Zar mi je ruka prekratka da izbavi, ili u meni snage nema da oslobodim? Gle, prijetnjom svojom isušujem more, u pustinje rijeke pretvaram; njihove se ribe raspadaju jer vode nema i od žeđi ugibaju. 
\par 3 Nebesa oblačim u tminu i kostrijet im dajem za pokrivač!" 
\par 4 Gospod Jahve dade mi jezik vješt da znam riječju krijepiti umorne. Svako jutro on mi uho budi da ga slušam kao učenici. 
\par 5 Gospod Jahve uši mi otvori: ne protivih se niti uzmicah. 
\par 6 Leđa podmetnuh onima što me udarahu, a obraze onima što mi bradu čupahu, i lica svojeg ne zaklonih od uvreda ni od pljuvanja. 
\par 7 Gospod mi Jahve pomaže, zato se neću smesti. Zato učinih svoj obraz k'o kremen i znam da se neću postidjeti. 
\par 8 Blizu je onaj koji mi pravo daje. Tko će se pravdati sa mnom? Iziđimo zajedno! Tko je protivnik moj u parnici? Nek' mi se približi! 
\par 9 Gle, Gospod mi Jahve pomaže, tko će me osuditi? Svi će se oni k'o haljina izlizati, moljac će ih razjesti. 
\par 10 "Tko god se od vas Jahve boji, nek' posluša glas Sluge njegova! Tko u tmini hodi, bez tračka svjetlosti, nek' se uzda u ime Jahvino, nek' se na Boga svog osloni. 
\par 11 Ali svi vi oganj palite, raspirujete žeravicu. Idite u plamenove ognja svojega i u žeravu koju raspiriste. Tako će vam moja učiniti ruka: ležat ćete u mukama. 


\chapter{51}

\par 1 Čujte me, vi koji za pravdom težite,  koji Jahvu tražite; pogledajte stijenu iz koje ste isječeni i jamu duboku iz koje ste izvađeni. 
\par 2 Pogledajte oca svoga Abrahama i Saru koja vas porodi! Jest, sam bijaše kad sam ga pozvao, al' sam ga blagoslovio i umnožio." 
\par 3 Jest, Jahvi se sažalio Sion, sažalile mu se njegove razvaline. Pustaru će njegovu učiniti poput Edena, a stepu poput Vrta Jahvina. Klicanje i radost njim će odjekivat', i zahvalnice i glas hvalospjeva. 
\par 4 "Pomno me slušaj, puče moj, počujte me, narodi! Jer od mene Zakon dolazi i moje pravo za svjetlo narodima. 
\par 5 Brzo će stići pravda moja, moje će spasenje doći kao svjetlost. Moja će mišica suditi narodima. Mene očekuju otoci i u moju se mišicu uzdaju. 
\par 6 K nebu oči podignite, na zemlju dolje pogledajte. K'o dim će se rasplinut' nebesa, zemlja će se k'o haljina istrošit', kao komarci nestat će joj žitelji. Ali će spasenje moje trajati dovijeka, i pravdi mojoj neće biti kraja. 
\par 7 Poslušajte me, vi koji poznajete pravo, narode kojima je moj Zakon u srcu. Ne bojte se poruge ljudske, ne plašite se uvreda! 
\par 8 Jer moljac će ih razjesti kao haljinu, crv će ih rastočiti kao vunu. Ali će pravda moja trajati dovijeka i spas moj od koljena do koljena." 
\par 9 Probudi se! Probudi se! Opaši se snagom, mišice Jahvina! Probudi se, kao u dane davne, za drevnih naraštaja. Nisi li ti rasjekla Rahaba i probola Zmaja? 
\par 10 Nisi li ti isušila more, vodu velikog bezdana, i od dubine morske put načinila da njime prolaze otkupljeni? 
\par 11 Vratit će se oni što ih je oslobodio Jahve i s radosnim kricima doći će na Sion. Vječna će sreća biti nad glavama, pratit će ih klicanje i radost, nestat će tuge i jecaja. 
\par 12 Ja, ja sam tješitelj vaš. Tko si ti da se bojiš smrtna čovjeka i sina čovječjeg, koji je kao trava? 
\par 13 Zaboravio si Jahvu, svoga Stvoritelja, koji razastrije nebesa i koji zemlju utemelji; sveudilj strepiš, svaki dan, od tlačiteljeve jarosti. Kao da je pregnuo da te uništi. Ali gdje je sad jarost tlačiteljeva? 
\par 14 Doskora će biti slobodan sužanj, neće umrijeti u jami, niti će mu kruha nedostajati. 
\par 15 Jer, ja sam Jahve, Bog tvoj, koji burkam more da mu valovi buče, ime mi je Jahve nad Vojskama. 
\par 16 Svoje sam ti riječi stavio u usta, u sjenu svoje ruke sakrio  sam te kad sam razastro nebesa, utemeljio zemlju i rekao Sionu:  "Ti si narod moj!" 
\par 17 Probudi se, probudi se, ustani, Jeruzaleme! Ti koji si pio iz ruke Jahvine čašu gnjeva njegova. Ispio si pehar opojni, do dna ga iskapio. 
\par 18 Od svih sinova koje je rodio ne bješe nikog da ga vodi; od svih sinova koje je podigao ne bješe nikog da ga pridrži. 
\par 19 Ovo te dvoje pogodilo - tko da te požali? - pohara i rasap, glad i mač - tko da te utješi? 
\par 20 Sinovi ti leže obamrli po uglovima svih ulica, kao antilopa u mreži, puni gnjeva Jahvina, prijetnje Boga tvojega. 
\par 21 Zato čuj ovo, bijedniče, pijan, ali ne od vina. 
\par 22 Ovako govori Jahve, Gospod tvoj, tvoj Bog, branitelj tvoga naroda: "Iz ruke ti, evo, uzimam čašu opojnu, pehar gnjeva svojega: nećeš ga više piti. 
\par 23 Stavit ću je u ruke tvojim tlačiteljima, onima koji su ti govorili: 'Prigni se da prijeđemo!' I ti si im leđa kao tlo podmetao, kao put za prolaznike. 


\chapter{52}

\par 1 Probudi se! Probudi se! Odjeni se snagom, Sione! Odjeni se najsjajnijim haljinama, Jeruzaleme, grade sveti, jer više neće k tebi ulaziti neobrezani i nečisti. 
\par 2 Otresi prah sa sebe, ustani, izgnani Jeruzaleme! Skini okov sa svog vrata, izgnana kćeri sionska." 
\par 3 Jest, ovako govori Jahve: "Bili ste prodani nizašto i  bit ćete otkupljeni bez novaca." 
\par 4 Jest, ovako govori Gospod  Jahve: "Moj je narod sišao nekoć u Egipat da se ondje nastani  kao stranac, potom ga Asirci nizašto potlačiše. 
\par 5 Ali sada,  čemu sam ja ovdje - riječ je Jahvina - kad je moj narod bio bez  razloga porobljen, a gospodari njegovi likuju - riječ je Jahvina  - i bez prestanka se danomice ime moje huli. 
\par 6 Zato će narod  moj poznati moje ime i shvatit će u onaj dan da sam ja koji govorim:  'Evo me!'" 
\par 7 Kako su ljupke po gorama noge glasonoše radosti koji oglašava mir, nosi sreću, i spasenje naviješta govoreć Sionu: "Bog tvoj kraljuje!" 
\par 8 Čuj, stražari ti glas podižu, zajedno svi kliču od radosti, jer na svoje oči vide gdje se na Sion vraća Jahve. 
\par 9 Radujte se, kličite, razvaline jeruzalemske, jer je Jahve utješio narod svoj i otkupio Jeruzalem. 
\par 10 Ogolio je Jahve svetu svoju mišicu pred očima svih naroda, da svi krajevi zemaljski vide spasenje Boga našega. 
\par 11 Odlazite, odlazite, iziđite odatle, ne dotičite ništa nečisto! Iziđite iz njegove sredine! Očistite se, vi koji nosite posude Jahvine! 
\par 12 Jer nećete izići u hitnji, niti ćete ići bježeći, jer će vam prethodnica biti Jahve, a zalaznica Bog Izraelov! 
\par 13 Gle, uspjet će Sluga moj, podignut će se, uzvisit' i proslaviti! 
\par 14 Kao što se mnogi užasnuše vidjevši ga - tako mu je lice bilo neljudski iznakaženo te obličjem više nije naličio na čovjeka - 
\par 15 tako će on mnoge zadiviti narode i kraljevi će pred njim usta stisnuti videć' ono o čemu im nitko nije govorio, shvaćajuć' ono o čemu nikad čuli nisu: 


\chapter{53}

\par 1 "Tko da povjeruje u ono što nam je objavljeno, kome se otkri ruka Jahvina?" 
\par 2 Izrastao je pred njim poput izdanka, poput korijena iz zemlje sasušene. Ne bijaše na njem ljepote ni sjaja da bismo se u nj zagledali, ni ljupkosti da bi nam se svidio. 
\par 3 Prezren bješe, odbačen od ljudi, čovjek boli, vičan patnjama, od kog svatko lice otklanja, prezren bješe, odvrgnut. 
\par 4 A on je naše bolesti ponio, naše je boli na se uzeo, dok smo mi držali da ga Bog bije i ponižava. 
\par 5 Za naše grijehe probodoše njega, za opačine naše njega satriješe. Na njega pade kazna - radi našeg mira, njegove nas rane iscijeliše. 
\par 6 Poput ovaca svi smo lutali i svaki svojim putem je hodio. A Jahve je svalio na nj bezakonje nas sviju. 
\par 7 Zlostavljahu ga, a on puštaše, i nije otvorio usta svojih. K'o jagnje na klanje odvedoše ga; k'o ovca, nijema pred onima što je strižu, nije otvorio usta svojih. 
\par 8 Silom ga se i sudom riješiše; tko se brine za njegovu sudbinu? Da, iz zemlje živih ukloniše njega, za grijehe naroda njegova nasmrt ga izbiše. 
\par 9 Ukop mu odrediše među zločincima, a grob njegov bi s bogatima, premda nije počinio nepravde nit' su mu usta laži izustila. 
\par 10 Al' se Jahvi svidje da ga pritisne bolima. Žrtvuje li život svoj za naknadnicu, vidjet će potomstvo, produžit' sebi dane i Jahvina će se volja po njemu ispuniti. 
\par 11 Zbog patnje duše svoje vidjet će svjetlost i nasititi se spoznajom njezinom. Sluga moj pravedni opravdat će mnoge i krivicu njihovu na sebe uzeti. 
\par 12 Zato ću mu mnoštvo dati u baštinu i s mogućnicima plijen će dijeliti, jer sam se ponudio na smrt i među zlikovce bio ubrojen, da grijehe mnogih ponese na sebi i da se zauzme za zločince. 


\chapter{54}

\par 1 Kliči, nerotkinjo, koja nisi rađala;  podvikuj od radosti, ti što ne znaš za trudove! Jer osamljena više djece ima negoli udata, kaže Jahve. 
\par 2 Raširi prostor svog šatora, razastri, ne štedi platna svog prebivališta, produži mu užeta, kolčiće učvrsti! 
\par 3 Jer proširit ćeš se desno i lijevo. Tvoje će potomstvo zavladat' narodima i napučit će opustjele gradove. 
\par 4 Ne boj se, nećeš se postidjeti; na srami se, nećeš se crvenjeti. Zaboravit ćeš sramotu svoje mladosti i više se nećeš spominjati rugla udovištva svoga. 
\par 5 Jer suprug ti je tvoj Stvoritelj, ime mu je Jahve nad Vojskama; tvoj je Otkupitelj Svetac Izraelov, Bog zemlje svekolike on se zove. 
\par 6 Jest, k'o ženu ostavljenu, u duši ucviljenu, Jahve te pozvao. Zar se smije otpustiti žena svoje mladosti, pita Bog tvoj. 
\par 7 "Za kratak trenutak ostavih tebe, al' u sućuti velikoj opet ću te prigrliti. 
\par 8 U provali srdžbe sakrih načas od tebe lice svoje, al' u ljubavi vječnoj smilovah se tebi," govori Jahve, tvoj Otkupitelj. 
\par 9 "Bit će mi k'o za Noinih dana, kad se zakleh da vode Noine neće više preplaviti zemlju; tako se zaklinjem da se više neću na tebe srditi nit' ću ti prijetiti. 
\par 10 Nek' se pokrenu planine i potresu brijezi, al' se ljubav moja neće odmać' od tebe, nit' će se pokolebati moj Savez mira," kaže Jahve koji ti se smilovao. 
\par 11 "O nevoljnice, vihorom vitlana, neutješna, gle, postavit ću na smaragd tvoje kamenje i na safir tvoje temelje. 
\par 12 Od rubina dići ću ti kruništa, vrata tvoja od prozirca, ograde ti od dragulja. 
\par 13 Svi će ti sinovi Jahvini biti učenici, i velika će biti sreća djece tvoje. 
\par 14 Na pravdi ćeš biti zasnovana. Odbaci tjeskobu, nemaš se čega bojati, odbaci strah jer ti se neće primaći. 
\par 15 Ako li te napadnu, neće doći od mene; tko se na te digne, zbog tebe će pasti. 
\par 16 Gle, ja sam stvorio kovača koji raspaljuje žeravu i vadi iz nje oružje da ga kuje. Ali stvorih i zatornika da uništava. 
\par 17 Neće uspjeti oružje protiv tebe skovano. Dokazat ćeš da je zao svaki jezik što na te udari na sudu. To je baština slugu Jahvinih, to im je pobjeda od mene" - riječ je Jahvina. 


\chapter{55}

\par 1 "O svi vi koji ste žedni, dođite na vodu; ako novaca i nemate, dođite. Bez novaca i bez naplate kupite vina i mlijeka! 
\par 2 Zašto da trošite novac na ono što kruh nije i nadnicu svoju na ono što ne siti? Mene poslušajte, i dobro ćete jesti i sočna ćete uživati jela. 
\par 3 Priklonite uho i k meni dođite, poslušajte, i duša će vam živjeti. Sklopit ću s vama Savez vječan, Savez milosti Davidu obećanih." 
\par 4 Evo, učinih te svjedokom pucima, knezom i zapovjednikom narodima. 
\par 5 Evo, pozvat ćeš narod koji ne poznaješ, i narod koji te ne zna dohrlit će k tebi radi Jahve, Boga tvojega, i Sveca Izraelova, jer on te proslavio. 
\par 6 Tražite Jahvu dok se može naći, zovite ga dok je blizu! 
\par 7 Nek' bezbožnik put svoj ostavi, a zlikovac naume svoje. Nek' se vrati Gospodu, koji će mu se smilovati, k Bogu našem jer je velikodušan u praštanju. 
\par 8 "Jer misli vaše nisu moje misli i puti moji nisu vaši puti," riječ je Jahvina. 
\par 9 "Visoko je iznad zemlje nebo, tako su puti moji iznad vaših putova, i misli moje iznad vaših misli." 
\par 10 "Kao što daždi i sniježi s neba bez prestanka dok se  zemlja ne natopi, oplodi i ozeleni da bi dala sjeme sijaču i  kruha za jelo, 
\par 11 tako se riječ koja iz mojih usta izlazi ne  vraća k meni bez ploda, nego čini ono što sam htio i obistinjuje  ono zbog čega je poslah." 
\par 12 Da, s radošću ćete otići i u miru ćete biti vođeni. Gore će i brda klicati od radosti pred vama i sva će stabla u polju pljeskati. 
\par 13 Umjesto trnja rast će čempresi, umjesto koprive mirta će nicati. I bit će to Jahvi na slavu, kao znak vječan, neprolazan. 


\chapter{56}

\par 1 Ovako govori Jahve: "Držite se prava i činite pravdu, jer  će uskoro doći moj spas i objaviti se moja pravednost." 
\par 2 Blago čovjeku koji čini tako i sinu čovječjem što se toga  pridržava: koji poštuje subotu da je ne oskvrni i koji ruke svoje  čuva od svakoga zla djela. 
\par 3 Neka sin tuđinčev koji je prionuo uz Jahvu ne govori:  "Jamačno će me Jahve odvojiti od svojega naroda." Neka uškopljenik  ne govori: "Ja sam, evo, tek suho drvo." 
\par 4 Jer ovako govori Jahve: "S uškopljenicima koji obdržavaju  subotu, koji izabiru što je meni drago i ostanu postojani u Savezu  mome - 
\par 5 podići ću u kući svojoj i među svojim zidovima spomenik  i ime, bolje nego sinovima i kćerima, dat ću im vječno ime koje  neće biti iskorijenjeno. 
\par 6 A sinove tuđinske koji pristadoše uz Jahvu da mu služe  i da ljube ime Jahvino i da mu budu službenici, koji poštuju  subotu i ne oskvrnjuju je i postojani su u Savezu mome, 
\par 7 njih  ću dovesti na svoju svetu goru i razveseliti u svojem Domu molitve.  Njihove žrtve paljenice i klanice bit će ugodne na mojem žrtveniku, jer će se Dom moj zvati Dom molitve za sve narode." 
\par 8 Riječ je Gospoda Jahve koji sabire raspršene Izraelce:  "Sabrat ću ih još povrh onih koji su već sabrani." 
\par 9 Sve zvijeri poljske, dođite jesti, i sve vi, zvijeri šumske! 
\par 10 Svi su mu stražari slijepi, i ništa ne shvaćaju. Svi su oni psi nijemi, ne mogu lajati. Sanjaju i drijemlju, najmilije im spavati. 
\par 11 Psi su to proždrljivi, nezasitni; pastiri su to bez razbora: svaki svojim putem okreće, svaki za dobitkom svojim. 
\par 12 "Dođite, donijet ću vina; napit ćemo se pića žestoka, i sutra će biti kao danas, izobilje veliko, preveliko!" 


\chapter{57}

\par 1 Pravednik gine, i nitko ne mari. Uklanjaju ljude pobožne, i nitko ne shvaća. Da, zbog zla uklonjen je pravednik 
\par 2 da bi ušao u mir. Tko god je pravim putem hodio počiva na svom ležaju. 
\par 3 Pristupite sad, sinovi vračarini, leglo preljubničko i bludničko! 
\par 4 S kim se podrugujete, na koga razvaljujete usta i komu jezik plazite? Niste li vi porod grešan i leglo lažljivo? 
\par 5 Vi koji se raspaljujete među hrašćem, pod svakim zelenim drvetom, žrtvujući djecu u dolinama i u rasjelinama stijena! 
\par 6 Dio je tvoj među oblucima potočnim, oni, oni su baština tvoja. Njima izlijevaš ljevanicu, njima prinosiš darove! Zar da se time ja utješim? 
\par 7 Na gori visokoj, uzdignutoj, svoj si ležaj postavila i popela se onamo da prinosiš žrtvu klanicu. 
\par 8 Za vrata i dovratke metnula si spomen svoj; daleko od mene svoj ležaj raskrivaš, penješ se na nj i širiš ga. Pogađala si se s onima s kojima si voljela lijegati, sve si više bludničila s njima gledajuć' im mušku snagu. 
\par 9 S uljem za Molekom trčiš, s pomastima mnogim, nadaleko posla glasnike svoje, strovali ih u Podzemlje. 
\par 10 Iscrpljena si od tolikih lutanja, al' nisi rekla: "Beznadno je!" Snagu si svoju nanovo našla te nisi sustala. 
\par 11 Koga si se uplašila i pobojala da si se iznevjerila, da se više nisi mene spominjala, niti si me k srcu uzimala? Šutio sam, zatvarao oči, zato me se nisi bojala. 
\par 12 Ali ću objavit' o tvojoj pravdi i djela ti tvoja neće koristiti. 
\par 13 Kad uzmeš vikati, nek' te izbave kipovi koje si skupila, sve će ih vjetar raznijeti, vihor će ih otpuhnuti. A tko se u me uzda, baštinit će zemlju i zaposjest će svetu goru moju. 
\par 14 Govorit će se: Naspite, naspite, poravnajte put! Uklonite zapreke s puta mog naroda! 
\par 15 Jer ovako govori Višnji i Uzvišeni, koji vječno stoluje i ime mu je Sveti: "U prebivalištu visokom i svetom stolujem, ali ja sam i s potlačenim i poniženim, da oživim duh smjernih, da oživim srca skrušenih. 
\par 16 Jer neću se prepirati dovijeka ni vječno se ljutiti: preda mnom bi podlegao duh i duše što sam ih stvorio. 
\par 17 Zbog grijeha lakomosti njegove razgnjevih se, udarih ga i sakrih se rasrđen. Ali on okrenu za srcem svojim 
\par 18 i vidjeh putove njegove. Izliječit ću ga, voditi i utješit' one što s njime tuguju - 
\par 19 stavit ću hvalu na usne njihove. Mir, mir onom tko je daleko i tko je blizu," govori Jahve, "ja ću te izliječiti." 
\par 20 Al' opaki su poput mora uzburkanog koje se ne može smiriti, valovi mu mulj i blato izmeću. 
\par 21 "Nema mira grešnicima!" govori Bog moj. 


\chapter{58}

\par 1 Viči iz sveg grla, ne suspeži se! Glas svoj poput roga podigni. Objavi mom narodu njegove zločine, domu Jakovljevu grijehe njegove. 
\par 2 Dan za danom oni mene traže i žele znati moje putove, kao narod koji vrši pravdu i ne zaboravlja pravo Boga svoga. Od mene ištu pravedne sudove i žude da im se Bog približi: 
\par 3 "Zašto postimo ako ti ne vidiš, zašto se trapimo ako ti ne znaš?" Gle, u dan kad postite poslove nalazite i na posao gonite radnike svoje. 
\par 4 Gle, vi postite da se prepirete i svađate i da pesnicom bijete siromahe. Ne postite više kao danas, i čut će vam se glas u visini! 
\par 5 Zar je meni takav post po volji u dan kad se čovjek trapi? Spuštati kao rogoz glavu k zemlji, sterati poda se kostrijet i pepeo, hoćeš li to zvati postom i danom ugodnim Jahvi? 
\par 6 Ovo je post koji mi je po volji, riječ je Jahve Gospoda: Kidati okove nepravedne, razvezivat' spone jarmene, puštati na slobodu potlačene, slomiti sve jarmove; 
\par 7 podijeliti kruh svoj s gladnima, uvesti pod krov svoj beskućnike, odjenuti onog koga vidiš gola i ne kriti se od onog tko je tvoje krvi. 
\par 8 Tad će sinut' poput zore tvoja svjetlost, i zdravlje će tvoje brzo procvasti. Pred tobom će ići tvoja pravda, a Slava Jahvina bit će ti zalaznicom. 
\par 9 Vikneš li, Jahve će ti odgovorit, kad zavapiš, reći će: "Evo me!" Ukloniš li iz svoje sredine jaram, ispružen prst i besjedu bezbožnu, 
\par 10 dadeš li kruha gladnome, nasitiš li potlačenog, tvoja će svjetlost zasjati u tmini i tama će tvoja kao podne postati, 
\par 11 Jahve će te vodit' bez prestanka, sitit će te u sušnim krajevima. On će krijepit' kosti tvoje i bit ćeš kao vrt zaljeven, kao studenac kojem voda nikad ne presuši. 
\par 12 I ti ćeš gradit' na starim razvalinama, dići ćeš temelje budućih koljena. Zvat će te popravljačem pukotina i obnoviteljem cesta do naselja. 
\par 13 Zadržiš li nogu da ne pogaziš subotu i u sveti dan ne obavljaš poslove; nazoveš li subotu milinom a časnim dan Jahvi posvećen; častiš li ga odustajuć' od puta, bavljenja poslom i pregovaranja - 
\par 14 tad ćeš u Jahvi svoju milinu naći, i ja ću te provesti po zemaljskim visovima, dat ću ti da uživaš u baštini oca tvog Jakova, jer Jahvina su usta govorila. 


\chapter{59}

\par 1 Ne, nije ruka Jahvina prekratka da spasi, niti mu je uho otvrdlo da ne bi čuo, 
\par 2 nego su opačine vaše jaz otvorile između vas i Boga vašega. Vaši su grijesi lice njegovo zastrli, i on vas više ne sluša. 
\par 3 Jer ruke su vaše u krvi ogrezle, a vaši prsti u zločinima. Usne vam izgovaraju laž, a jezik podlost mrmlja. 
\par 4 Nitko s pravom tužbu ne podiže, niti koga sude po istini. U ništavilo se uzdaju, laž kazuju, začinju zloću, a rađaju bezakonje. 
\par 5 Legu jaja gujina, tkaju mrežu paukovu; pojede li tko njihovo jaje, umire, razbije li ga, iz njega ljutica izlazi. 
\par 6 Njihovim tkanjem nemoguće se odjenuti, ne možeš se pokriti njihovom rukotvorinom. Rukotvorine su njihove djela zločinačka, rukama svojim čine nasilje. 
\par 7 Noge njihove u zlo hitaju i brze su da krv nevinu proliju. Misli su im misli zločinačke, pustoš i propast na njinim su putima. 
\par 8 Put mira oni ne poznaju, na stazama njihovim nema pravice. Iskrivili su svoje putove, tko njima kreće mira ne poznaje. 
\par 9 Stog' se pravo od nas udaljilo, zato pravda ne dopire do nas. Nadasmo se svjetlosti, a ono tama; i vidjelu, a ono u tmini hodimo. 
\par 10 Pipamo kao slijepci duž zida, tapkamo kao bez očiju. Spotičemo se u podne k'o u sumraku, sasvim zdravi, kao da smo mrtvi. 
\par 11 Svi mumljamo kao medvjedi i gučemo tužno kao golubovi. Očekivasmo Sud, a njega nema, i spasenje - od nas je daleko. 
\par 12 Jer mnogo je naših opačina pred tobom i grijesi naši protiv nas svjedoče. Doista, prijestupi su naši pred nama, mi znademo svoju krivicu; 
\par 13 pobunili smo se i zanijekali Jahvu, odmetnuli se od Boga svojega, govorili podlo, odmetnički, mrmljali u srcu riječi lažljive. 
\par 14 Tako je potisnuto pravo, i pravda mora stajati daleko. Jer na trgu posrnu istina i poštenju nema više pristupa. 
\par 15 Vjernosti je ponestalo, a tko izbjegava zlo, bude opljačkan. Jahve vidje, i ne bi mu milo što nema pravice. 
\par 16 Vidje da nema čovjeka, začudi se što nema posrednika. Tad mu pomože njegova mišica i njegova ga pravda poduprije. 
\par 17 Pravednost je obukao k'o oklop, stavio na glavu kacigu spasenja. Osvetom se odjenuo k'o haljom, ogrnu se revnošću kao plaštem. 
\par 18 Vratit će svakome po njegovim djelima: gnjev svojim protivnicima, odmazdu dušmanima. 
\par 19 Sa zapada vidjet će ime Jahvino i Slavu njegovu s istoka sunčanog. Jer doći će kao uska rijeka koju goni dah Jahvin. 
\par 20 Ali doći će Otkupitelj Sionu, i onima od sinova Jakovljevih koji se obrate od svog otpadništva, riječ je Jahvina. 
\par 21 "A ovo je moj Savez s njima," govori Jahve. "Duh moj  koji je na tebi i riječi moje koje stavih u tvoja usta neće izići  iz usta tvojih ni usta tvojega potomstva, ni iz usta potomstva  tvojih potomaka, od sada pa dovijeka," veli Jahve. 


\chapter{60}

\par 1 Ustani, zasini, jer svjetlost tvoja dolazi, nad tobom blista Slava Jahvina. 
\par 2 A zemlju, evo, tmina pokriva, i mrklina narode! A tebe obasjava Jahve, i Slava se njegova javlja nad tobom. 
\par 3 K tvojoj svjetlosti koračaju narodi, i kraljevi k istoku tvoga sjaja. 
\par 4 Podigni oči, obazri se: svi se sabiru, k tebi dolaze. Sinovi tvoji dolaze izdaleka, kćeri ti nose u naručju. 
\par 5 Gledat ćeš tad i sjati radošću, igrat će srce i širit' se, jer k tebi će poteći bogatstvo mora, blago naroda k tebi će pritjecati. 
\par 6 Mnoštvo deva prekrit će te, jednogrbe deve iz Midjana i Efe. Svi će iz Šebe doći donoseći zlato i tamjan i hvale Jahvi pjevajući. 
\par 7 Sva stada kedarska u tebi će se sabrati, ovnovi nebajotski bit će ti na službu. Penjat će se k'o ugodna žrtva na moj žrtvenik, proslavit ću Dom Slave svoje! 
\par 8 Tko su oni što lebde poput oblaka, k'o golubovi prema golubinjacima svojim? 
\par 9 Da, to se zbog mene sabiru brodovi, lađe su taršiške pred njima da izdaleka dovezu tvoje sinove, a s njima srebro njihovo i zlato, zbog imena Jahve, Boga tvojega, zbog Sveca Izraelova koji te proslavi. 
\par 10 Zidine će tvoje obnoviti stranci i kraljevi njihovi služit će ti. U svojoj srdžbi ja sam te udario, al' u svojoj naklonosti opet ti se smilovah. 
\par 11 Vrata će tvoja biti otvorena svagda, ni danju ni noću neće se zatvarati, da propuste k tebi bogatstva naroda s kraljevima koji ih vode. 
\par 12 Jer propast će narod i kraljevstvo koje ti ne bude htjelo služiti, i ti će se narodi sasvim zatrti. 
\par 13 K tebi će doći slava Libanona, čempresi, jele i brijestovi skupa, da ukrase prostor mojega Svetišta, podnožje će moje proslaviti! 
\par 14 K tebi će, sagnuti, dolaziti sinovi tvojih tlačitelja, pred noge ti padat' koji te prezirahu. Nazivat će te Gradom Jahvinim, Sionom Sveca Izraelova. 
\par 15 Zato što si bio ostavljen, omražen, izbjegavan, učinit ću te vječnim ponosom, radošću od koljena do koljena. 
\par 16 Ti ćeš sisati mlijeko naroda, sisat ćeš grudi kraljeva. I znat ćeš da sam ja, Jahve, Spasitelj tvoj, Silni Jakovljev, tvoj Otkupitelj. 
\par 17 Mjesto mjedi, donijet ću zlato; mjesto željeza, donijet ću srebro; mjesto drva, mjed; mjesto kamena, željezo. Za glavara tvoga postavit ću Mir, Pravdu za vladara. 
\par 18 Više se neće slušat' o nasilju u tvojoj zemlji ni o pustošenju i razaranju na tvojem području. Zidine ćeš svoje nazivati Spasom, Slavom svoja vrata. 
\par 19 Neće ti više sunce biti svjetlost danju nit' će ti svijetlit' mjesečina, nego će Jahve biti tvoje vječno svjetlo i tvoj će Bog biti tvoj sjaj. 
\par 20 Sunce tvoje neće više zalaziti nit' će ti mjesec pomrčati, jer će Jahve biti tvoje vječno svjetlo, i okončat će se dani tvoje žalosti. 
\par 21 Svi u tvom narodu bit će pravednici i posjedovat će zemlju dovijeka, mog nasada izdanci, mojih ruku djelo, da bih se u njima proslavio. 
\par 22 Od najmanjega postat će tisuća, od neznatnoga moćan narod. Ja, Jahve, govorio sam; u pravo ću vrijeme izvršiti. 


\chapter{61}

\par 1 Duh Jahve Gospoda na meni je, jer me Jahve pomaza, posla me da radosnu vijest donesem ubogima, da iscijelim srca slomljena; da zarobljenima navijestim slobodu i oslobođenje sužnjevima; 
\par 2 da navijestim godinu milosti Jahvine i dan odmazde Boga našega; da razveselim ožalošćene na Sionu 
\par 3 i da im dadem vijenac mjesto pepela, ulje radosti mjesto ruha žalosti, pjesmu zahvalnicu mjesto duha očajna. I zvat će ih Hrastovima pravde, Nasadom Jahvinim - na slavu njegovu. 
\par 4 Oni će nanovo dići drevne razvaline, sazdati opet mjesta poharana, ruševine prošlih pokoljenja. 
\par 5 Tuđinci će doći da vam stada pasu, stranci će vam biti ratari i vinogradari. 
\par 6 A vas će zvati "Svećenici Jahvini", nazivat će vas "Službenici Boga našega". Uživat ćete bogatstva naroda, blagom se njihovim dičiti. 
\par 7 Dvostruka bijaše njihova sramota - rug i prezir bijahu im baština - zato će u zemlji svojoj baštinit' dvostruko, njihova će biti radost vječita. 
\par 8 Jer ja, Jahve, ljubim pravdu, a mrzim grabež nepravedni. Vjerno ću ih nagraditi i sklopiti s njima Savez vječni. 
\par 9 Slavno će im biti sjeme među pucima i potomstvo među narodima. Tko god ih vidi, prepoznat će da su sjeme što ga Jahve blagoslovi. 
\par 10 Radošću silnom u Jahvi se radujem, duša moja kliče u Bogu mojemu, jer me odjenu haljinom spasenja, zaogrnu plaštem pravednosti, kao ženik kad sebi vijenac stavi il' nevjesta kad se uresi nakitom. 
\par 11 Kao što zemlja tjera svoje klice, kao što u vrtu niče sjemenje, učinit će Gospod da iznikne pravda i hvala pred svim narodima. 


\chapter{62}

\par 1 Sionu za ljubav neću šutjeti, Jeruzalema radi neću mirovati dok pravda njegova ne zasine k'o svjetlost, dok njegovo spasenje ne plane k'o zublja. 
\par 2 I puci će vidjet' tvoju pravednost, i tvoju slavu svi kraljevi; prozvat će te novim imenom što će ga odrediti usta Jahvina. 
\par 3 U Jahvinoj ćeš ruci biti kruna divna i kraljevski vijenac na dlanu Boga svog. 
\par 4 Neće te više zvati Ostavljenom ni zemlju tvoju Opustošenom, nego će te zvati Moja milina, a zemlju tvoju Udata, jer ti si milje Jahvino i zemlja će tvoja imat' supruga. 
\par 5 Kao što se mladić ženi djevicom, tvoj će se graditelj tobom oženiti; i kao što se ženik raduje nevjesti, tvoj će se Bog tebi radovati. 
\par 6 Na zidine tvoje, Jeruzaleme, stražare sam postavio: ni danju ni noću ne smiju zašutjeti. O, vi koji podsjećate Jahvu, vama nema počinka! 
\par 7 I ne dajte mu mira dok ne obnovi Jeruzalem, dok ga opet slavom na zemlji ne učini. 
\par 8 Zakle se Gospod desnicom i mišicom svojom snažnom: "Neću više dati žita tvoga za hranu neprijateljima. Neće više tuđinci piti tvoga vina o kojem si teško radio. 
\par 9 Neka ga jedu oni koji su ga želi i neka hvale Jahvu, neka ga piju oni što su ga trgali u predvorju mojega Svetišta!" 
\par 10 Prođite, prođite kroz vrata, poravnajte put narodu! Nasipajte, nasipajte cestu, uklonite s nje kamenje. Podignite stijeg narodima! 
\par 11 Evo, Jahve oglasuje do nakraj zemlje: "Recite kćeri sionskoj: Evo, dolazi tvoj spasitelj. Evo, s njim naplata njegova i njegova nagrada ispred njega! 
\par 12 Oni će se zvati 'Sveti narod', 'Otkupljenici Jahvini'. A tebe će zvati 'Traženi' - 'Grad neostavljeni'." 


\chapter{63}

\par 1 Tko je taj što dolazi iz Edoma, iz Bosre, u haljinama crvenim? Tko je taj što veličanstveno odjenut pun snage korača? - Ja sam to koji naučavam pravdu, velik kad spasavam! 
\par 2 - Zašto je crvena tvoja haljina i odijelo kao u onog koji gazi u kaci? 
\par 3 - U kaci sam sam gazio, od narodÄa nikog ne bijaše. U gnjevu ih svom izgazih i zgnječih u svojoj jarosti. Krv mi njihova poprska haljine, iskaljah svu odjeću svoju. 
\par 4 Jer dan osvete bijaše mi u srcu, došla je godina mojeg otkupljenja. 
\par 5 Ogledah se, al' ne bješe pomoćnika! Začudih se, al' ne bješe potpore. Tada mi je moja mišica pomogla i moja me srdžba poduprla. 
\par 6 U gnjevu svom satrijeh narode, u bijesu sve ih izgazih i zemlju polih krvlju njihovom! 
\par 7 Slavit ću ljubav Jahvinu, slavna djela njegova - za sve što nam Jahve učini, za veliku dobrotu domu Izraelovu što nam je iskaza u svojoj samilosti, u obilju svoje ljubavi. 
\par 8 Reče: "Dosta, oni su narod moj, sinovi koji se neće iznevjeriti!" I on im posta Spasiteljem u svim njihovim tjeskobama. 
\par 9 Nije slao poslanika ni anđela nego ih je sam spasio. U svojoj ljubavi i samilosti sam ih je otkupio, podigao ih i nosio u sve dane od davnine. 
\par 10 Ali se oni odmetnuše, ožalostiše sveti Duh njegov. Zato im je postao neprijatelj i sam je na njih zavojštio. 
\par 11 Spomenuše se tad davnih dana i sluge njegova Mojsija: "Gdje li je onaj koji izvuče iz vode pastira stada svojega? Gdje je onaj koji udahnu u njega Duh svoj sveti? 
\par 12 Koji je Mojsijevu desnicu vodio veličanstveno svojom mišicom, koji vodu pred njima razdvoji i steče sebi ime vječno; 
\par 13 koji ih provede dnom bezdana kao konja po pustinji i nisu se spoticali? 
\par 14 Poput stoke što silazi u dolinu, Duh Jahvin vodio ih počivalištu. Tako si ti vodio narod svoj i slavno ime sebi stekao. 
\par 15 Pogledaj s nebesa i vidi iz prebivališta svoga svetog i slavnog. Gdje li je ljubomora tvoja i snaga? Zar se susteglo ganuće tvog srca i samilost tvoja prema meni? Ah, sućuti nam svoje ne ustegni, 
\par 16 jer Otac si naš! Abraham nas ne poznaje i ne spominje nas se Izrael; Jahve, ti si naš Otac, Otkupitelj naš - ime ti je oduvijek. 
\par 17 Zašto, o Jahve, zašto nas puštaš da lutamo daleko od tvojih putova, zašto si dao da nam srce otvrdne da se tebe više ne bojimo? Vrati se, radi slugu svojih i radi plemenÄa što su tvoja baština! 
\par 18 Zašto bezbožnici gaze tvoje Svetište, a neprijatelji naši blate tvoju svetinju? 
\par 19 Odavna postadosmo kao oni kojima više ne vladaš i koji tvoje ime više ne nose. O, da razdreš nebesa i siđeš, da ime svoje objaviš neprijateljima: pred licem tvojim tresla bi se brda, pred tobom bi drhtali narodi, 


\chapter{64}

\par 1 (63:19) Odavna postadosmo kao oni kojima više ne vladaš i koji tvoje ime više ne nose. O, da razdreš nebesa i siđeš, da ime svoje objaviš neprijateljima: pred licem tvojim tresla bi se brda, pred tobom bi drhtali narodi, 
\par 2 (64:1) kao kad oganj suho granje zapali  i vatra vodu zakuha! 
\par 3 (64:2) Čineć' djela strahotna, neočekivana, silazio si i brda su se tresla pred tobom! 
\par 4 (64:3) Odvijeka se čulo nije, uho nijedno nije slušalo, oko nijedno nije vidjelo, da bi koji bog, osim tebe, takvo što činio onima koji se uzdaju u njega. 
\par 5 (64:4) Pomažeš onima što pravdu čine radosno i tebe se spominju na putima tvojim; razgnjevismo te, griješismo, od tebe se odmetnusmo. 
\par 6 (64:5) Tako svi postasmo nečisti, a sva pravda naša k'o haljine okaljane. Svi mi k'o lišće otpadosmo i opačine naše k'o vjetar nas odnose. 
\par 7 (64:6) Nikog nema da tvoje ime prizove, da se probudi i osloni o tebe. Jer lice si svoje od nas sakrio i predao nas u ruke zločinima našim. 
\par 8 (64:7) Pa ipak, naš si otac, o Jahve: mi smo glina, a ti si naš lončar - svi smo mi djelo ruku tvojih. 
\par 9 (64:8) Ne srdi se, Jahve, odveć žestoko, ne spominji se bez prestanka naše krivice. De pogledaj - tÓa svi smo mi narod tvoj! 
\par 10 (64:9) Opustješe sveti gradovi tvoji, Sion pustinja posta, i pustoš Jeruzalem. 
\par 11 (64:10) Dom, svetinja naša i ponos naš, u kom te oci naši slavljahu, ognjem izgori i sve su nam dragocjenosti opljačkane. 
\par 12 (64:11) Zar ćeš se na sve to, Jahve, sustezati, zar ćeš šutjet' i ponižavati nas odveć žestoko? 


\chapter{65}

\par 1 Potražiše me koji ne pitahu za me, nađoše me koji me ne tražahu; rekoh: "Evo me! Evo me!" narodu koji ne prizivaše ime moje. 
\par 2 Svagda sam pružao ruku narodu odmetničkom, koji hodi putem zlim, za mislima svojim, 
\par 3 narodu koji me bez prestanka u lice srdi: žrtvuju po vrtovima, kad prinose na opekama, 
\par 4 na grobovima stanuju i noće na skrovitim mjestima, jedu svinjetinu, meću u zdjele jela nečista. 
\par 5 I još govore: "Ukloni se! Ne prilazi mi da te ne posvetim." Oni su mi dim u nosu, oganj što gori povazdan. 
\par 6 Evo, sve je napisano preda mnom: neću ušutjet' dok im ne platim, dok im ne platim u njedra, 
\par 7 za bezakonja vaša i vaših otaca, sva zajedno - govori Jahve. Koji su prinosili kad na gorama i pogrđivali me na brežuljcima - izmjerit ću im u krilo plaću za djela prijašnja. 
\par 8 Ovako govori Jahve: "Kao što o soku u grozdu vele: 'Ne uništavajte ga, u njemu je blagoslov!' tako ću učiniti i ja radi slugu svojih, neću sve uništiti. 
\par 9 Izvest ću iz Jakova potomstvo, a iz Jude baštinika gora svojih; baštinit će ih odabranici moji, i moje će se sluge ondje naseliti. 
\par 10 Šaron će postati pašnjak ovcama, a nizina akorska počivalište govedima - narodu mojem koji mene traži. 
\par 11 A vi koji ste Jahvu ostavili, koji ste zaboravili Svetu goru moju, koji pripremate stol Gadu, koji Meniju naljev lijevate, 
\par 12 za mač sam vas odredio - past ćete ničice da vas kolju. Jer zvao sam vas, a vi se niste odazvali, govorio sam, a vi niste slušali, nego ste činili što je zlo u očima mojim, izabirali ste što mi nije po volji." 
\par 13 Stog ovako Jahve Gospod govori: "Evo, sluge će moje jesti, a vi ćete gladovati. Evo, sluge će moje piti, a vi ćete žeđati. Evo, sluge će se moje veseliti, a vi ćete se stidjeti. 
\par 14 Evo, sluge će se moje radovati od sreće u srcu, a vi ćete vikati od boli u srcu i kukati duše slomljene! 
\par 15 Ime ćete svoje ostaviti za kletvu mojim izabranicima: 'Tako te ubio Jahve!' A sluge svoje on će zvati drugim imenom. 
\par 16 Tko se u zemlji bude blagoslivljao, nek' se blagoslivlje Bogom vjernim. I tko se u zemlji bude kleo, nek' se kune Bogom vjernim. Jer prijašnje će nevolje biti zaboravljene, od očiju mojih bit će sakrivene. 
\par 17 Jer, evo, ja stvaram nova nebesa i novu zemlju. Prijašnje se više neće spominjati niti će vam na um dolaziti. 
\par 18 Veselite se i dovijeka kličite zbog onoga što ja stvaram; jer, evo, od Jeruzalema stvaram klicanje, od naroda njegova radost. 
\par 19 I klicat ću nad Jeruzalemom, radovat' se nad svojim narodom. U njemu više neće čuti ni plača ni vapaja. 
\par 20 U njemu više neće biti novorođenčeta koje živi malo dana ni starca koji ne bi godina svojih navršio: najmlađi će umrijet' kao stogodišnjak, a tko ne doživi stotinu godina prokletim će se smatrati. 
\par 21 Gradit će kuće i stanovat' u njima, saditi vinograde i uživati rod njihov. 
\par 22 Neće se više graditi da drugi stanuju ni saditi da drugi uživa: vijek naroda moga bit će k'o vijek drveta, izabranici moji dugo će uživati plodove ruku svojih. 
\par 23 Neće se zalud mučiti i neće rađati za smrt preranu, jer će oni s potomcima svojim biti rod blagoslovljenika Jahvinih. 
\par 24 Prije nego me zazovu, ja ću im se odazvat'; još će govoriti, a ja ću ih već uslišiti. 
\par 25 Vuk i jagnje zajedno će pasti, lav će jesti slamu k'o govedo; al' će se zmija prahom hraniti. Nitko neće činiti zla ni štete na svoj Svetoj gori mojoj" - govori Jahve. 



\chapter{66}

\par 1 Ovako govori Jahve: "Nebesa su moje prijestolje, a zemlja podnožje nogama! Kakvu kuću da mi sagradite i gdje da bude mjesto mog prebivališta? 
\par 2 TÓa sve je moja ruka načinila i sve je moje" - riječ je Jahvina. "Ali na koga svoj pogled svraćam? Na siromaha i čovjeka duha ponizna koji od moje riječi dršće. 
\par 3 Ima ih koji kolju bika, ali i ljude ubijaju; žrtvuju ovcu, ali i psu vrat lome. Netko prinosi žrtvu, ali i krv svinjsku; prinose kad, ali časte i kipove. Kao što oni izabraše svoje putove i duši im se mile gnusobe njihove, 
\par 4 tako ću i ja izabrati za njih nevolje nesmiljene, pustit ću na njih ono čega se plaše. Jer zvao sam, a nitko se ne odazva, govorio sam, a nitko ne posluša, nego su činili što je zlo u očima mojim, izabrali ono što mi nije po volji." 
\par 5 Poslušajte riječ Jahvinu, vi koji od njegove riječi dršćete. "Govore braća vaša koja na vas mrze i odbacuju vas radi moga imena: 'Neka se proslavi Jahve, pa da radost vašu vidimo.' Ali oni će biti postiđeni." 
\par 6 Čuj! Buka iz grada, glas iz Hrama! Glas je to Jahve koji uzvraća svojim neprijateljima. 
\par 7 Prije neg' bolove oćutje, eto je rodila. Prije neg' trudove osjeti, porodi dječaka. 
\par 8 Tko je takvo što čuo, tko je takvo što vidio? Može li se zemlja u jednom danu napučiti? Može li se narod odjednom roditi? A tek što je osjetila trudove, Sionka rodi sinove! 
\par 9 "Zar bih ja otvorio krilo materino a da ono ne rodi?" - govori Jahve. "Zar bih ja, koji dajem rađanje, zatvorio maternicu?" - kaže Bog tvoj. 
\par 10 Veselite se s Jeruzalemom, kličite zbog njega svi koji ga ljubite! Radujte se, radujte s njime svi koji ste nad njim tugovali! 
\par 11 Nadojite se i nasitite na dojkama utjehe njegove da se nasišete i nasladite na grudima krepčine njegove. 
\par 12 Jer ovako govori Jahve: "Evo, mir ću na njih kao rijeku svratiti i kao potok nabujali bogatstvo naroda. Dojenčad ću njegovu na rukama nositi i milovati na koljenima. 
\par 13 Kao što mati tješi sina, tako ću i ja vas utješiti - utješit ćete se u Jeruzalemu." 
\par 14 Kad to vidite, srce će vam se radovati i procvast će vam kosti k'o mlada trava. Očitovat će se ruka Jahvina na njegovim slugama i gnjev nad neprijateljima njegovim. 
\par 15 Jer, evo, dolazi Jahve s ognjem - bojna su mu kola poput vihora - da u jarosti gnjev svoj iskali i prijetnje svoje u ognju žarkome. 
\par 16 Da, sudit će Jahve ognjem i mačem svakom smrtniku: pobijenih Jahvinih mnoštvo će biti. 
\par 17 Oni koji se posvećuju i čiste u vrtovima iza onog jednog u sredini, koji jedu svinjetinu, nečisto i miševe - svi će zajedno izginuti, riječ je Jahvina. 
\par 18 Ja dobro poznajem njihova djela i namjere njihove. "Dolazim da saberem sve puke i jezike, i oni će doći i vidjeti  moju Slavu! 
\par 19 Postavit ću im znak i poslat ću preživjele od  njih k narodima u Taršiš, Put, Lud, Mošek, Roš, Tubal i Javan  - k dalekim otocima koji nisu čuli glasa o meni ni vidjeli moje  Slave - i oni će naviještati Slavu moju narodima. 
\par 20 I dovest  će svu vašu braću između svih naroda kao prinos Jahvi - na konjima, na bojnim kolima, nosilima, na mazgama i jednogrbim devama -  na Svetu goru svoju u Jeruzalemu" - govori Jahve - "kao što sinovi  Izraelovi prinose prinos u čistim posudama u Domu Jahvinu. 
\par 21 I  uzet ću sebi između njih svećenike, levite" - govori Jahve. 
\par 22 "Jer, kao što će nova nebesa i zemlja nova, koju ću stvoriti, trajati preda mnom" - riječ je Jahvina - "tako će vam ime i potomstvo trajati. 
\par 23 Od mlađaka do mlađaka, od subote do subote, dolazit će svi ljudi da se poklone pred licem mojim" - govori Jahve. 
\par 24 Izlazeći, gledat ću trupla ljudi koji se od mene odmetnuše: crv njihov neće umrijeti i njihov se oganj neće ugasiti - bit će na gadost svim ljudima. 




\end{document}