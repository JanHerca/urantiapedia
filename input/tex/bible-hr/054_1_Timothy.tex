\begin{document}

\title{1 Timoteju}


\chapter{1}

\par 1 Pavao, apostol Krista Isusa po nalogu Boga, Spasitelja našega, i Krista Isusa, nade naše, 
\par 2 Timoteju, pravomu sinu u vjeri:  milost, milosrđe i mir od Boga Oca i Krista Isusa, Gospodina  našega! 
\par 3 Kao što sam te zamolio kad sam odlazio u Makedoniju, ostani  u Efezu da zapovijediš nekima neka ne naučavaju drugih nauka 
\par 4 i neka se ne zanose beskrajnim bajkama i rodoslovljima, koja  više pogoduju rasprama negoli rasporedbi Božjoj po vjeri. 
\par 5 Svrha  je te zapovijedi ljubav iz čista srca, dobre savjesti i vjere  neprijetvorne. 
\par 6 To su neki promašili i zastranili u praznorječje; 
\par 7 htjeli bi biti učitelji Zakona, a ne razumiju ni što govore  ni što tvrde. 
\par 8 A mi znamo da je Zakon dobar ako se tko njime služi zakonito, 
\par 9 svjestan toga da je Zakon tu ne za pravednika nego za bezakonike  i nepokornike, nepobožnike i grešnike, bezbožnike i svetogrdnike, ocoubojice i materoubojice, koljače, 
\par 10 bludnike, muškoložnike, trgovce ljudima, varalice, krivokletnike, i ima li još što protivno  zdravom nauku - 
\par 11 po evanđelju Slave blaženoga Boga koje je  meni povjereno. 
\par 12 Zahvalan sam Onome koji mi dade snagu - Kristu Isusu, Gospodinu našemu - jer me smatrao vrijednim povjerenja, kad  u službu postavi mene 
\par 13 koji prije bijah hulitelj, progonitelj  i nasilnik. Ali pomilovan sam jer sam to u neznanju učinio, još  u nevjeri. 
\par 14 I milost Gospodina našega preobilovala je zajedno  s vjerom i ljubavlju, u Kristu Isusu. 
\par 15 Vjerodostojna je riječ i vrijedna da se posve prihvati:  Isus Krist dođe na svijet spasiti grešnike, od kojih sam prvi  ja. 
\par 16 A pomilovan sam zato da na meni prvome Isus Krist pokaže  svu strpljivost i pruži primjer svima koji će povjerovati u njega  za život vječni. 
\par 17 A Kralju vjekova, besmrtnome, nevidljivome, jedinome Bogu čast i slava u vijeke vjekova. Amen. 
\par 18 Taj ti zadatak predajem, sine Timoteju, u skladu s proroštvima  nekoć nad tobom izrečenima: na njih oslonjen, bij boj plemeniti 
\par 19 imajući vjeru i dobru savjest, koju su neki odbacili i doživjeli  brodolom vjere. 
\par 20 Među njima je Himenej i Aleksandar, koje  sam predao Sotoni da nauče ne huliti. 


\chapter{2}

\par 1 Dakle, preporučujem prije svega da se obavljaju prošnje, molitve, molbenice i zahvalnice za sve ljude, 
\par 2 za kraljeve i sve koji  su na vlasti, da provodimo miran i spokojan život u svoj bogoljubnosti  i ozbiljnosti. 
\par 3 To je dobro i ugodno pred Spasiteljem našim  Bogom, 
\par 4 koji hoće da se svi ljudi spase i dođu do spoznanja  istine. 
\par 5 Jer jedan je Bog, jedan je i posrednik između Boga  i ljudi, čovjek - Krist Isus, 
\par 6 koji sebe samoga dade kao otkup  za sve. To je u svoje vrijeme dano svjedočanstvo, 
\par 7 za koje  sam ja postavljen propovjednikom i apostolom - istinu govorim, ne lažem - učiteljem naroda u vjeri i istini. 
\par 8 Hoću dakle da muškarci mole na svakome mjestu, podižući  čiste ruke bez srdžbe i raspre; 
\par 9 isto tako žene - u doličnu  držanju: neka se rese stidljivošću i razborom, ne pletenicama  i zlatom ili biserjem ili skupocjenim odijelom, 
\par 10 nego - dobrim  djelima, kako dolikuje ženama koje ispovijedaju bogoljubnost. 
\par 11 Žena neka u miru prima pouku sa svom podložnošću. 
\par 12 Poučavati  pak ženi ne dopuštam, ni vladati nad mužem, nego - neka bude  na miru. 
\par 13 Jer prvi je oblikovan Adam, onda Eva; 
\par 14 i Adam  nije zaveden, a žena je, zavedena, učinila prekršaj. 
\par 15 A spasit  će se rađanjem djece ako ustraje u vjeri, ljubavi i posvećivanju, s razborom. 


\chapter{3}

\par 1 Vjerodostojna je riječ: teži li tko za nadgledništvom, časnu  službu želi. 
\par 2 Treba stoga da nadglednik bude besprijekoran, jedne žene muž, trijezan, razuman, sređen, gostoljubiv, sposoban  poučavati, 
\par 3 ne vinu sklon, ne nasilan nego popustljiv, ne ratoboran, ne srebroljubac; 
\par 4 da svojom kućom dobro upravlja i sinove  drži u pokornosti sa svom ozbiljnošću - 
\par 5 a ne zna li netko  svojom kućom upravljati, kako će se brinuti za Crkvu Božju? - 
\par 6 ne novoobraćenik da se ne bi uzoholio i pao pod osudu đavlovu. 
\par 7 A treba da ima i lijepo svjedočanstvo od onih vani, da ne  bi u rug upao i zamku đavlovu. 
\par 8 Ðakoni isto tako treba da budu ozbiljni, ne dvolični,  ne odani mnogom vinu ni prljavu dobitku - 
\par 9 imajući otajstvo  vjere u čistoj savjesti. 
\par 10 I neka se najprije iskušavaju, pa  onda, budu li besprigovorni, neka obavljaju službu. 
\par 11 Žene isto tako neka budu ozbiljne, ne klevetnice nego  trijezne, vjerne u svemu. 
\par 12 đakoni neka budu jedne žene muževi, neka dobro upravljaju  djecom i svojim kućama. 
\par 13 Jer oni koji dobro obavljaju službu, stječu častan položaj i veliku smjelost u vjeri, vjeri u Isusu  Kristu. 
\par 14 Ovo ti pišem u nadi da ću ubrzo doći k tebi, 
\par 15 a okasnim  li, da znaš kako se treba vladati u kući Božjoj, koja je Crkva  Boga živoga, stup i uporište istine. 
\par 16 Da, po sveopćem uvjerenju, veliko je Otajstvo pobožnosti: On, očitovan u tijelu, opravdan u Duhu, viđen od anđela, propovijedan među narodima, vjerovan u svijetu, uznesen u slavu. 


\chapter{4}

\par 1 Duh izričito govori da će u posljednja vremena neki otpasti  od vjere i prikloniti se prijevarnim duhovima i zloduhovskim  naucima. 
\par 2 A sve to pod utjecajem himbe lažljivaca otupjele  savjesti 
\par 3 koji zabranjuju ženiti se i nameću uzdržavati se  od jela što ih je Bog stvorio da ih sa zahvalnošću uzimaju oni  koji vjeruju i znaju istinu. 
\par 4 Doista, svako je Božje stvorenje  dobro i ne valja odbaciti ništa što se uzima sa zahvalnošću 
\par 5 jer  se posvećuje riječju Božjom i molitvom. 
\par 6 To izlaži braći i bit ćeš dobar poslužitelj Krista Isusa, hranjen riječima vjere i dobroga nauka za kojim postojano ideš. 
\par 7 Svjetovne pak i bablje priče odbijaj! Vježbaj se u pobožnosti! 
\par 8 Uistinu, tjelesno vježbanje malo čemu koristi, a pobožnost  je svemu korisna jer joj je obećan život - sadašnji i budući. 
\par 9 Vjerodostojna je to riječ i vrijedna da se posve prihvati. 
\par 10 Ta za to se trudimo i borimo jer se pouzdajemo u Boga živoga  koji je Spasitelj svih ljudi, ponajpače vjernika. 
\par 11 Zapovijedaj  to i naučavaj! 
\par 12 Nitko neka ne prezire tvoje mladosti, nego budi uzor  vjernicima u riječi, u vladanju, u ljubavi, u vjeri, u čistoći. 
\par 13 Dok ne dođem, posveti se čitanju, poticanju, poučavanju. 
\par 14 Ne zanemari milosnog dara koji je u tebi, koji ti je dan  po proroštvu zajedno s polaganjem ruku starješinstva. 
\par 15 Oko  toga nastoj, sav u tom budi da tvoj napredak bude svima očit. 
\par 16 Pripazi na samog sebe i na poučavanje! Ustraj u tome! Jer  to vršeći, spasit ćeš i sebe i one koji te slušaju. 


\chapter{5}

\par 1 Na starca se ne otresaj, nego ga nagovaraj kao oca, mladiće  kao braću, 
\par 2 starice kao majke, djevojke kao sestre - u svoj  čistoći. 
\par 3 Udovice poštuj - one koje su zaista udovice. 
\par 4 Ako li  ipak koja udovica ima djecu ili unuke, neka najprije oni znaju  očitovati svoju pobožnost prema vlastitom domu i uzdarjem uzvraćati  roditeljima jer to je ugodno Bogu. 
\par 5 A ona koja je zaista udovica, posve sama, pouzdaje se u Boga, odana prošnjama i molitvama  noć i dan; 
\par 6 ona, naprotiv, koja provodi lagodan život, živa  je već umrla. 
\par 7 I to naglašuj da budu besprijekorne. 
\par 8 Ako  li se tkogod za svoje, navlastito za ukućane, ne stara, zanijekao  je vjeru i gori je od nevjernika. 
\par 9 U popis neka se unosi udovica ne mlađa od šezdeset godina, jednog muža žena, 
\par 10 koja ima svjedočanstvo dobrih djela: da  je djecu odgojila, da je bila gostoljubiva, da je svetima noge  prala, da je nevoljnima pomagala, da se svakom dobru djelu posvećivala. 
\par 11 Mlađe pak udovice odbijaj jer kad ih požuda odvrati od Krista, hoće se udati, 
\par 12 pa zasluže osudu što su pogazile prvotnu  vjernost. 
\par 13 A uz to se, obilazeći po kućama, nauče biti besposlene, i ne samo besposlene, nego i brbljave i nametljive, govoreći  što ne bi smjele. 
\par 14 Hoću dakle da se mlađe udaju, djecu rađaju, da budu kućevne te ne daju protivniku nikakva povoda za pogrđivanje. 
\par 15 Jer već su neke zastranile za Sotonom. 
\par 16 Ako koja vjernica  ima udovica, neka im pomaže, a neka se ne opterećuje Crkva, da  uzmogne pomoći onima koje su zaista udovice. 
\par 17 Starješine koji su dobri predstojnici dostojni su dvostruke  časti, ponajpače oni koji se trude oko Riječi i poučavanja. 
\par 18 Pismo  doista veli: Volu koji vrši ne zavezuj usta! I: Vrijedan  je radnik plaće svoje. 
\par 19 Protiv starješine ne primaj tužbe, osim na osnovi dvaju ili triju svjedoka. 
\par 20 One koji  griješe, pred svima ukori da i drugi imaju straha! 
\par 21 Zaklinjem  te pred Bogom i Kristom Isusom i izabranim anđelima da se toga  držiš bez predrasude i ništa ne činiš po naklonosti. 
\par 22 Ruku  prebrzo ni na koga ne polaži i ne budi dionikom tuđih grijeha!  Sebe čistim čuvaj! 
\par 23 Ne pij više samo vode, nego uzimaj malo  vina poradi želuca i čestih svojih slabosti. 
\par 24 Grijesi nekih ljudi očiti su i prije suda, nekih pak  samo nakon njega. 
\par 25 Tako su i dobra djela očita, a bila i drukčija, ne mogu se sakriti. 



\chapter{6}

\par 1 Koji su pod jarmom, robovi, neka svoje gospodare smatraju svake  časti dostojnima da se ne bi pogrđivalo ime Božje i nauk. 
\par 2 A  oni kojima su gospodari vjernici, neka ih ne cijene manje zato  što su braća, nego neka im još više služe jer ti koji primaju  njihovo dobročinstvo vjernici su i ljubljena braća. To naučavaj i preporučuj! 
\par 3 A tko drukčije naučava i ne  prianja uza zdrave riječi, riječi Gospodina našega Isusa Krista, i nauk u skladu s pobožnošću, 
\par 4 nadut je, puka neznalica, samo  boluje od raspra i rječoborstava, od kojih nastaje zavist, svađa, pogrde, zla sumnjičenja, 
\par 5 razračunavanja ljudi pokvarene pameti  i lišenih istine, što pobožnost smatraju dobitkom. 
\par 6 Pa i jest  dobitak velik pobožnost, zadovoljna onim što ima! 
\par 7 Ta ništa  nismo donijeli na svijet te iz njega ništa ni iznijeti ne možemo! 
\par 8 Imamo li dakle hranu i odjeću, zadovoljimo se time. 
\par 9 Jer  oni koji se hoće bogatiti, upadaju u napast, zamku i mnoge nerazumne  i štetne požude što ljude strovaljuju u zator i propast. 
\par 10 Zaista, korijen svih zala jest srebroljublje; njemu odani, mnogi odlutaše  od vjere i sami sebe isprobadaše mukama mnogima. 
\par 11 A ti se, Božji čovječe, toga kloni! Teži za pravednošću, pobožnošću, vjerom, ljubavlju, postojanošću, krotkošću! 
\par 12 Bij  dobar boj vjere, osvoji vječni život na koji si pozvan i radi  kojega si dao ono lijepo svjedočanstvo pred mnogim svjedocima! 
\par 13 Zapovijedam pred Bogom koji svemu život daje i pred Kristom  Isusom koji pred Poncijem Pilatom posvjedoči lijepo svjedočanstvo: 
\par 14 čuvaj Zapovijed, neokaljano i besprijekorno, do Pojavka Gospodina  našega Isusa Krista. 
\par 15 Njega će u svoje vrijeme pokazati On, Blaženi i jedini Vladar, Kralj kraljeva i Gospodar gospodara, 
\par 16 koji jedini ima besmrtnost, prebiva u svjetlu nedostupnu, koga nitko od ljudi ne vidje niti ga vidjeti može. Njemu čast i vlast vjekovječna! Amen. 
\par 17 Onima koji su u sadašnjem svijetu bogati zapovijedaj  neka ne budu bahati i neka se ne uzdaju u nesigurno bogatstvo, nego u Boga koji nam sve bogato daje na uživanje; 
\par 18 neka dobro  čine, neka se bogate dobrim djelima, neka budu darežljivi, zajedničari  - 
\par 19 prikupljajući sebi lijepu glavnicu za budućnost da osvoje  onaj pravi život. 
\par 20 Timoteju, poklad čuvaj kloneći se svjetovnoga praznoglasja  i proturječja nekog nazovispoznanja, 
\par 21 koje su neki ispovijedali  pa od vjere zastranili. Milost s vama 




\end{document}