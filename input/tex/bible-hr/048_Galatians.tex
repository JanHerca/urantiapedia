\begin{document}

\title{Galačanima}


\chapter{1}

\par 1 Pavao, apostol - ne od ljudi ni po kojem čovjeku, nego po Isusu  Kristu i Bogu Ocu koji ga uskrisi od mrtvih - 
\par 2 i sva braća  koja su sa mnom: Crkvama u Galaciji. 
\par 3 Milost vam i mir od Boga, Oca našega, i Gospodina Isusa Krista, 
\par 4 koji sam sebe dade  za grijehe naše da nas istrgne iz sadašnjega svijeta opakoga  kao što je volja Boga i Oca našega, 
\par 5 komu slava u vijeke vjekova!  Amen. 
\par 6 Čudim se da od Onoga koji vas pozva na milost Kristovu  tako brzo prelazite na neko drugo evanđelje, 
\par 7 koje uostalom  i ne postoji. Postoje samo neki koji vas zbunjuju i hoće prevratiti  evanđelje Kristovo. 
\par 8 Ali kad bismo vam mi, ili kad bi vam anđeo  s neba navješćivao neko evanđelje mimo onoga koje vam mi navijestismo, neka je proklet! 
\par 9 Što smo već rekli, to sad i ponavljam: navješćuje  li vam tko neko evanđelje mimo onoga koje primiste, neka je proklet. 
\par 10 Doista, nastojim li ovo pridobiti ljude ili Boga? Ili  idem li za tim da ljudima ugodim? Kad bih sveudilj nastojao ljudima  ugađati, ne bih bio Kristov sluga. 
\par 11 Obznanjujem vam, braćo: evanđelje koje sam navješćivao  nije od ljudi, 
\par 12 niti ga ja od kojeg čovjeka primih ili naučih, nego objavom Isusa Krista. 
\par 13 Ta čuli ste za moje negdašnje ponašanje u židovstvu:  preko svake sam mjere progonio i pustošio Crkvu Božju 
\par 14 te  sam u židovstvu, prerevno odan otačkim predajama, nadmašio mnoge  vršnjake u svojem narodu. 
\par 15 Ali kad se Onomu koji me odvoji već od majčine utrobe  i pozva milošću svojom, svidjelo 
\par 16 otkriti mi Sina svoga  da ga navješćujem među poganima, odmah, ne posavjetovah se s  tijelom i krvlju 
\par 17 i ne uziđoh u Jeruzalem k onima koji bijahu  apostoli prije mene, nego odoh u Arabiju pa se opet vratih u  Damask. 
\par 18 Onda nakon tri godine uziđoh u Jeruzalem potražiti Kefu  i ostadoh kod njega petnaest dana. 
\par 19 Od apostola ne vidjeh  nikoga drugog osim Jakova, brata Gospodinova. 
\par 20 Što vam pišem, Bog mi je svjedok, ne lažem. 
\par 21 Zatim dođoh u krajeve sirijske  i cilicijske. 
\par 22 Osobno pak bijah nepoznat Kristovim crkvama  u Judeji. 
\par 23 One su samo čule: "Negdašnji naš progonitelj sada  navješćuje vjeru koju je nekoć pustošio" 
\par 24 i slavile su Boga  zbog mene. 


\chapter{2}

\par 1 Zatim nakon četrnaest godina opet uziđoh u Jeruzalem s Barnabom, a povedoh sa sobom i Tita. 
\par 2 Uziđoh po objavi i izložih im  - napose uglednijima - evanđelje koje propovijedam među poganima  da ne bih možda, ili da nisam, trčao uzalud. 
\par 3 Čak ni Tit, pratilac moj, premda Grk, nije bio prisiljen  obrezati se, 
\par 4 i to radi uljeza, lažne braće, koja se ušuljaše  da vrebaju slobodu koju imamo u Kristu Isusu, ne bi li nas učinili  robovima. 
\par 5 Ne, ni načas im nismo popustili, nismo se podložili:  da istina evanđelja ostane kod vas! 
\par 6 A oni koji štogod znače - bili oni što bili, nije mi do  toga, Bog ne gleda tko je tko - ti uglednici, uistinu, ništa  nisu pridometnuli. 
\par 7 Nego naprotiv, vidjevši da mi je povjereno  evanđelje za neobrezane, kao Petru za obrezane - 
\par 8 jer Onaj  koji je bio na djelu po Petrovu apostolstvu među obrezanima,  bio je na djelu i po meni među poganima - 
\par 9 i spoznavši milost  koja mi je dana, Jakov, Kefa i Ivan, smatrani stupovima, pružiše  meni i Barnabi desnice zajedništva: mi ćemo među pogane, a oni  među obrezane! 
\par 10 Samo neka se sjećamo siromaha, što sam revno  i činio. 
\par 11 A kad Kefa stiže u Antiohiju, u lice mu se usprotivih  jer je zavrijedio osudu: 
\par 12 doista, prije nego stigoše neki  od Jakova, blagovao je zajedno s poganima; a kad oni dođoše,  počeo se povlačiti i odvajati bojeći se onih iz obrezanja. 
\par 13 Za  njim se povedoše i ostali Židovi te je i Barnaba zaveden tom  prijetvornošću. 
\par 14 Ali kad vidjeh da ne hode ravno, po istini evanđelja, rekoh Kefi pred svima: "Ako ti, Židov, poganski živiš, a ne  židovski, kako možeš siliti pogane da se požidove?" 
\par 15 Mi smo podrijetlom Židovi, a ne "grešnici iz poganstva". 
\par 16 Ali znamo: čovjek se ne opravdava po djelima Zakona, nego  vjerom u Isusa Krista. Zato i mi u Krista Isusa povjerovasmo  da se opravdamo po vjeri u Krista, a ne po djelima Zakona jer  se po djelima Zakona nitko neće opravdati. 
\par 17 Ako se pak po tome što zaiskasmo opravdati se u Kristu  očitovalo da smo i mi grešnici, nije li onda Krist u službi grijeha?  Nipošto! 
\par 18 Doista, ako ponovno gradim što sam bio srušio, pokazujem  da sam prijestupnik. 
\par 19 Ta po Zakonu ja Zakonu umrijeh da Bogu  živim. S Kristom sam razapet. 
\par 20 Živim, ali ne više ja, nego  živi u meni Krist. A što sada živim u tijelu, u vjeri živim u  Sina Božjega koji me ljubio i predao samoga sebe za mene. 
\par 21 Ne  dokidam milosti Božje! Doista, ako je opravdanje po Zakonu, onda  je Krist uzalud umro. 


\chapter{3}

\par 1 O bezumni Galaćani, tko li vas opčara? A pred očima vam je  Isus Krist bio ocrtan kao Raspeti. 
\par 2 Ovo bih samo htio doznati  od vas: jeste li primili Duha po djelima Zakona ili po vjeri  u Poruku? 
\par 3 Tako li ste bezumni? Započeli ste u Duhu pa da sada  u tijelu dovršite? 
\par 4 Zar ste toliko toga uzalud doživjeli? A  kad bi doista bilo uzalud! 
\par 5 Onaj dakle koji vam daje Duha i  čini među vama silna djela, čini li to zbog djela Zakona ili  zbog vjere u Poruku? 
\par 6 Tako Abraham povjerova Bogu i uračuna  mu se u pravednost. 
\par 7 Shvatite dakle: oni od vjere, to su sinovi Abrahamovi. 
\par 8 A Pismo, predvidjevši da Bog po vjeri opravdava pogane, unaprijed  navijesti Abrahamu: U tebi će blagoslovljeni biti svi narodi. 
\par 9 Tako: oni od vjere blagoslivlju se s vjernikom Abrahamom. 
\par 10 Doista, koji su god od djela Zakona, pod prokletstvom  su. Ta pisano je: Proklet tko se god ne drži i tko ne vrši  svega što je napisano u Knjizi Zakona. 
\par 11 A da se pred Bogom  nitko ne opravdava Zakonom, očito je jer: Pravednik će od  vjere živjeti. 
\par 12 Zakon pak nije od vjere, nego veli:  Tko ga vrši, u njemu će naći život. 
\par 13 Krist nas otkupi od prokletstva Zakona, postavši za nas  prokletstvom - jer pisano je: Proklet je tko god visi na drvetu  - 
\par 14 da u Kristu Isusu na pogane dođe blagoslov Abrahamov: da  Obećanje, Duha, primimo po vjeri. 
\par 15 Braćo, po ljudsku govorim: već i ljudski valjan savez  nitko ne poništava niti mu što dodaje. 
\par 16 A ova su obećanja  dana Abrahamu i potomstvu njegovu. Ne veli se: "i potomcima"  kao o mnogima, nego kao o jednomu: I potomstvu tvojem,  to jest Kristu. 
\par 17 Ovo hoću kazati: Saveza koji je Bog valjano  sklopio ne obeskrepljuje Zakon, koji je nastao četiri stotine  i trideset godina poslije, i ne dokida obećanja. 
\par 18 Doista,  ako se baština zadobiva po Zakonu, ne zadobiva se po obećanju.  A Abrahama je Bog po obećanju obdario. 
\par 19 Čemu onda Zakon? Dometnut je poradi prekršaja dok ne  dođe potomstvo komu je namijenjeno obećanje; sastavljen je po  anđelima preko posrednika. 
\par 20 Posrednika pak nema gdje je samo  jedan. A Bog je jedan. 
\par 21 Zar je dakle Zakon protiv obećanja Božjih? Nipošto! Jer  da je dan Zakon koji bi mogao oživljavati, pravednost bi doista  proizlazila iz Zakona. 
\par 22 Ali je Pismo sve zatvorilo pod grijeh  da se, po vjeri u Isusa Krista, obećano dade onima koji vjeruju. 
\par 23 Prije dolaska vjere, pod Zakonom zatvoreni, bili smo  čuvani za vjeru koja se imala objaviti. 
\par 24 Tako nam je Zakon  bio nadzirateljem sve do Krista da se po vjeri opravdamo. 
\par 25 A  otkako je nadošla vjera, nismo više pod nadzirateljem. 
\par 26 Uistinu, svi ste sinovi Božji, po vjeri u Kristu Isusu. 
\par 27 Doista, koji ste god u Krista kršteni, Kristom se zaodjenuste. 
\par 28 Nema više: Židov - Grk! Nema više: rob - slobodnjak! Nema  više: muško - žensko! Svi ste vi Jedan u Kristu Isusu! 
\par 29 Ako  li ste Kristovi, onda ste Abrahamovo potomstvo, baštinici po  obećanju. 


\chapter{4}

\par 1 Hoću reći: sve dok je baštinik maloljetan, ništa se ne razlikuje  od roba premda je gospodar svega: 
\par 2 pod skrbnicima je i upraviteljima  sve do dana koji je odredio otac. 
\par 3 Tako i mi: dok bijasmo maloljetni, robovasmo počelima svijeta. 
\par 4 A kada dođe punina vremena, odasla Bog Sina svoga: od žene bi rođen, Zakonu podložan 
\par 5 da podložnike Zakona otkupi te primimo posinstvo. 
\par 6 A budući da ste sinovi, odasla Bog u srca vaša Duha Sina svoga koji kliče: "Abba! Oče!" 
\par 7 Tako više nisi rob nego sin; ako pak sin, onda i baštinik po Bogu. 
\par 8 Onda dok još niste poznavali Boga, služili ste bogovima  koji po naravi to nisu. 
\par 9 Ali sada kad ste spoznali Boga - zapravo, kad je Bog spoznao vas - kako se sad opet vraćate k nemoćnim  i bijednim počelima i opet im, ponovno, hoćete robovati? 
\par 10 Dane  pomno opslužujete, i mjesece, i vremena, i godine! 
\par 11 Sve se  bojim za vas! Da se možda nisam uzalud trudio oko vas! 
\par 12 Postanite, braćo, molim vas, kao ja jer i ja postadoh  kao vi. Ničim me niste povrijedili. 
\par 13 Znate: prvi sam vam put  za bolesti navješćivao evanđelje. 
\par 14 Svoju kušnju, moje tijelo, niste ni prezreli ni odbacili, nego ste me primili kao anđela  Božjega, kao Krista Isusa. 
\par 15 Gdje je sada ono vaše blaženstvo?  Svjedočim vam doista: kad bi bilo moguće, oči biste svoje bili  iskopali i dali mi ih. 
\par 16 Tako? Postadoh li vam neprijateljem  propovijedajući vam istinu? 
\par 17 Oni revnuju za vas, ne časno, nego - odvojiti vas hoće  da onda vi za njih revnujete. 
\par 18 Dobro je da se za vas revnuje  u dobru uvijek, a ne samo kad sam nazočan kod vas, 
\par 19 dječice  moja, koju ponovno u trudovima rađam dok se Krist ne oblikuje  u vama. 
\par 20 Htio bih sada biti kod vas, pa i jezik promijeniti, jer ne znam što bih s vama. 
\par 21 Recite mi vi, koji želite biti pod Zakonom, zar ne čujete  Zakona? 
\par 22 Ta pisano je da je Abraham imao dva sina, jednoga  od ropkinje i jednoga od slobodne. 
\par 23 Ali onaj od ropkinje rođen  je po tijelu, a onaj od slobodne snagom obećanja. 
\par 24 To je slika.  Doista, te žene dva su Saveza: jedan s brda Sinaja, koji rađa  za ropstvo - to je Hagara. 
\par 25 Jer Hagara znači brdo Sinaj u  Arabiji i odgovara sadašnjem Jeruzalemu jer robuje zajedno sa  svojom djecom. 
\par 26 Onaj pak Jeruzalem gore slobodan je; on je  majka naša. 
\par 27 Pisano je doista: Kliči, nerotkinjo, koja ne rađaš, podvikuj od radosti, ti što ne znaš za trudove! Jer osamljena više djece ima negoli udana. 
\par 28 Vi ste, braćo, kao Izak, djeca obećanja. 
\par 29 I kao što  je onda onaj po tijelu rođeni progonio onoga po duhu rođenoga, tako je i sada. 
\par 30 Nego, što veli Pismo? Otjeraj sluškinju  i sina njezina jer sin sluškinje ne smije biti baštinik sa sinom  slobodne. 
\par 31 Zato, braćo, nismo djeca ropkinje nego slobodne. 


\chapter{5}

\par 1 Za slobodu nas Krist oslobodi! Držite se dakle i ne dajte se  ponovno u jaram ropstva! 
\par 2 Evo ja, Pavao, velim vam: ako se obrežete, Krist vam ništa  neće koristiti. 
\par 3 I ponovno jamčim svakom čovjeku koji se obreže:  dužan je opsluživati sav Zakon. 
\par 4 Prekinuli ste s Kristom vi  koji se u Zakonu mislite opravdati; iz milosti ste ispali. 
\par 5 Jer  mi po Duhu iz vjere očekujemo pravednost, nadu svoju. 
\par 6 Uistinu, u Kristu Isusu ništa ne vrijedi ni obrezanje ni neobrezanje, nego - vjera ljubavlju djelotvorna. 
\par 7 Dobro ste trčali; tko li vas je samo spriječio da se više  ne pokoravate istini? 
\par 8 Ta pobuda nije od Onoga koji vas zove! 
\par 9 Malo kvasca cijelo tijesto ukvasa. 
\par 10 Ja se uzdam u vas u  Gospodinu: vi nećete drukčije misliti. A tko vas zbunjuje, snosit  će osudu, tko god bio. 
\par 11 A ja, braćo, ako sveudilj propovijedam obrezanje, zašto  me sveudilj progone? Onda je obeskrijepljena sablazan križa! 
\par 12 Uškopili se oni koji vas podbunjuju! 
\par 13 Doista vi ste, braćo, na slobodu pozvani! Samo neka ta  sloboda ne bude izlikom tijelu, nego - ljubavlju služite jedni  drugima. 
\par 14 Ta sav je Zakon ispunjen u jednoj jedinoj riječi, u ovoj: Ljubi bližnjega svoga kao sebe samoga! 
\par 15 Ako  li pak jedni druge grizete i glođete, pazite da jedni druge ne  proždrete. 
\par 16 Hoću reći: po Duhu živite pa nećete ugađati požudi tijela! 
\par 17 Jer tijelo žudi protiv Duha, a Duh protiv tijela. Doista, to se jedno drugomu protivi da ne činite što hoćete. 
\par 18 Ali  ako vas Duh vodi, niste pod Zakonom. 
\par 19 A očita su djela tijela.  To su: bludnost, nečistoća, razvratnost, 
\par 20 idolopoklonstvo, vračanje, neprijateljstva, svađa, ljubomor, srdžbe, spletkarenja, razdori, strančarenja, 
\par 21 zavisti, pijančevanja, pijanke i  tome slično. Unaprijed vam kažem, kao što vam već rekoh: koji  takvo što čine, kraljevstva Božjega neće baštiniti. 
\par 22 Plod  je pak Duha: ljubav, radost, mir, velikodušnost, uslužnost, dobrota, vjernost, 
\par 23 blagost, uzdržljivost. Protiv tih nema zakona. 
\par 24 Koji su Kristovi, razapeše tijelo sa strastima i požudama. 
\par 25 Ako živimo po Duhu, po Duhu se i ravnajmo! 
\par 26 Ne hlepimo  za taštom slavom! Ne izazivajmo jedni druge, ne zaviđajmo jedni  drugima! 



\chapter{6}

\par 1 Braćo, ako se tko i zatekne u kakvu prijestupu, vi, duhovni, takva ispravljajte u duhu blagosti. A pazi na samoga sebe da  i ti ne podlegneš napasti. 
\par 2 Nosite jedni bremena drugih i tako  ćete ispuniti zakon Kristov! 
\par 3 Jer misli li tko da jest štogod, a nije ništa, sam sebe vara. 
\par 4 Svatko neka ispita sam svoje  djelo pa će onda u samom sebi imati čime se dičiti, a ne u usporedbi  s drugim. 
\par 5 Ta svatko će nositi svoj teret. 
\par 6 Koji se uči Riječi, neka sva dobra dijeli sa svojim učiteljem. 
\par 7 Ne varajte se: Bog se ne da izrugivati! Što tko sije,  to će i žeti! 
\par 8 Doista, tko sije u tijelo svoje, iz tijela će  žeti raspadljivost, a tko sije u duh, iz duha će žeti život vječni. 
\par 9 Neka nam ne dozlogrdi činiti dobro: ako ne sustanemo,  u svoje ćemo vrijeme žeti! 
\par 10 Dakle, dok imamo vremena, činimo  dobro svima, ponajpače domaćima u vjeri. 
\par 11 Gledajte kolikim vam slovima pišem svojom rukom. 
\par 12 Svi  koji se hoće praviti važni tijelom, sile vas na obrezanje, samo  da zbog križa Kristova ne bi trpjeli progonstvo. 
\par 13 Ta ni sami  obrezani ne opslužuju Zakona, ali hoće da se vi obrežete da bi  se mogli ponositi vašim tijelom. 
\par 14 A ja, Bože sačuvaj da bih  se ičim ponosio osim križem Gospodina našega Isusa Krista po  kojem je meni svijet raspet i ja svijetu. 
\par 15 Uistinu, niti je  što obrezanje niti neobrezanje, nego - novo stvorenje. 
\par 16 A  na sve koji se ovoga pravila budu držali, i na sveg Izraela Božjega  - mir i milosrđe! 
\par 17 Ubuduće neka mi nitko ne dodijava jer ja na svom tijelu  nosim biljege Isusove! 
\par 18 Milost Gospodina našega Isusa Krista s duhom vašim, braćo!  Amen 




\end{document}