\begin{document}

\title{2 Kraljevima}


\chapter{1}

\par 1 Poslije smrti Ahabove pobuni se Moab protiv Izraela. 
\par 2 Kako  Ahazja bijaše pao preko prozorske rešetke svoje gornje odaje  u Samariji i ozlijedio se, posla glasnike kojima reče: "Idite, pitajte Baal Zebuba, boga ekronskog, hoću li ozdraviti od ove  bolesti." 
\par 3 Ali je Anđeo Jahvin rekao Iliji Tišbijcu: "Ustani!  Idi u susret glasnicima samarijanskoga kralja i reci im: 'Zar  nema Boga u Izraelu te se idete savjetovati s Baal Zebubom, bogom  ekronskim?' 
\par 4 I zato veli Jahve ovako: 'Nećeš sići s postelje  u koju si se popeo; sigurno ćeš umrijeti.'" I ode Ilija. 
\par 5 Glasnici se vratiše k Ahazji, a on im reče: "Kako to da  ste se već vratili?" 
\par 6 Oni mu odgovoriše: "Sreo nas neki čovjek  i rekao nam: 'Idite, vratite se pred kralja koji vas je poslao  i recite mu: Ovako veli Jahve: Zar nema Boga u Izraelu te si  poslao po savjet k Baal Zebubu, bogu ekronskom? Zato nećeš sići  s postelje na koju si se popeo, nego ćeš umrijeti.'" 
\par 7 On ih  upita: "Kakav bijaše na oči taj čovjek koji vas je sreo i rekao  vam te riječi?" 
\par 8 A oni mu odgovoriše: "Bio je to čovjek u kožuhu  i s kožnim pojasom oko bedara." On reče: "To je Ilija Tišbijac!" 
\par 9 Tada mu posla pedesetnika s njegovom pedesetoricom i ode  taj k njemu i, našavši ga gdje sjedi na vrhu brijega, reče mu:  "Čovječe Božji! Kralj je naredio: Siđi!" 
\par 10 Ilija odgovori i  reče pedesetniku: "Ako sam čovjek Božji, neka oganj siđe s neba  i neka te proguta, tebe i tvoju pedesetoricu." I oganj se spusti  s neba i proguta ga, njega i njegovu pedesetoricu. 
\par 11 Kralj mu posla drugoga pedesetnika i njegovu pedesetoricu;  a taj, kad dođe, reče mu: "Čovječe Božji! Kralj je ovo zapovjedio:  Brže siđi!" 
\par 12 Ilija odgovori i reče mu: "Ako sam čovjek Božji, neka siđe oganj s neba i proguta tebe i tvoju pedesetoricu."  I spusti se oganj s neba i proguta ga, njega i njegovu pedesetoricu. 
\par 13 Kralj posla opet trećega pedesetnika i njegovu pedesetoricu.  Treći pedesetnik dođe, prignu koljena pred Ilijom i zamoli ga  ovako: "Čovječe Božji! Neka bude dragocjen u tvojim očima moj  život i život ovih pedeset tvojih slugu! 
\par 14 Oganj se spustio  s neba i progutao je oba pedesetnika s njihovom pedesetoricom;  ali sada neka barem moj život bude dragocjen u tvojim očima!" 
\par 15 Anđeo Jahvin reče Iliji: "Siđi s njim, ne boj se!" On  ustade i siđe s njim pred kralja 
\par 16 i reče mu: "Ovako veli Jahve:  zato što si slao glasnike Baal Zebubu, bogu ekronskom, po savjet, nećeš sići s postelje na koju si se popeo, nego ćeš umrijeti." 
\par 17 I umrije po riječi Jahvinoj koju je objavio Ilija. A  Joram, njegov brat, zakralji se mjesto njega druge godine Jorama, sina Jošafata, judejskoga kralja, jer ovaj nije imao sinova. 
\par 18 Ostala povijest Ahazje, sve što je učinio, zar to nije zapisano  u knjizi Ljetopisa kraljeva izraelskih? 


\chapter{2}

\par 1 Evo što se dogodilo kad je Jahve uznio Iliju na nebo u vihoru:  Ilija i Elizej pošli iz Gilgala. 
\par 2 I reče Ilija Elizeju: "Ostani  ovdje jer me Jahve šalje do Betela." Elizej odgovori: "Života  mi Jahvina i tvoga: ja te neću ostaviti!" I siđoše do Betela. 
\par 3 A proročki sinovi koji su boravili u Betelu iziđoše Elizeju  u susret i rekoše mu: "Znaš li da će danas Jahve uzeti tvoga  gospodara iznad tvoje glave?" On reče: "I ja to znam; tiho!" 
\par 4 Ilija mu reče: "Elizeju! Ostani ipak ovdje jer me Jahve šalje  do Jerihona." Ali on odgovori: "Života mi Jahvina i tvoga: ja  te neću ostaviti!" I uđoše u Jerihon. 
\par 5 Proročki sinovi koji  su živjeli u Jerihonu priđoše Elizeju i rekoše mu: "Znaš li da  će danas Jahve uzeti tvoga gospodara iznad tvoje glave?" On reče:  "I ja to znam; tiho!" 
\par 6 Ilija mu reče: "Ostani ipak ovdje jer  me Jahve šalje do Jordana." Ali on odgovori: "Života mi Jahvina  i tvoga: ja te neću ostaviti!" I tako pođoše obojica. 
\par 7 I pedeset proročkih sinova pođe i zaustavi se podalje, dok su se njih dvojica zadržala na obali Jordana. 
\par 8 Tada Ilija  uze svoj ogrtač, smota ga i udari njime po vodi, a voda se razdijeli  na dvije strane. I obojica prijeđoše po suhu. 
\par 9 A kad prijeđoše, Ilija će Elizeju: "Traži što da ti još učinim prije nego što  budem uznesen ispred tebe!" A Elizej odgovori: "Neka mi u dio  padne obilje tvoga duha!" 
\par 10 Ilija odgovori: "Mnogo tražiš:  ako me budeš vidio kad budem uznesen ispred tebe, bit će ti tako;  ako pak ne budeš vidio, neće ti biti." 
\par 11 I dok su tako išli  i razgovarali, gle: ognjena kola i ognjeni konji stadoše među  njih i Ilija u vihoru uziđe na nebo. 
\par 12 Elizej je gledao i vikao:  "Oče moj, oče moj! Kola Izraelova i konjanici njegovi!" I više  ga nije vidio. Uze tada svoje haljine i razdera ih nadvoje. 
\par 13 I  podiže Ilijin plašt, koji bijaše pao s njega, te se vrati i zaustavi  se na obali Jordana. 
\par 14 Uze onda Ilijin plašt i udari po vodi govoreći: "Gdje  je Jahve, Bog Ilijin?" I kad udari po vodi, ona se razdijeli  na dvije strane i Elizej prijeđe. 
\par 15 Proročki su sinovi to sa strane vidjeli pa rekoše: "Duh  je Ilijin počinuo na Elizeju!" I krenuše mu u susret, baciše  se pred njim na zemlju 
\par 16 i rekoše mu: "Evo ovdje s tvojim slugama  pedeset junaka. Dopusti im da idu tražiti tvoga gospodara; možda  ga je Duh Jahvin uzdigao i bacio na koju goru ili u kakvu dolinu."  On im odgovori: "Ne šaljite nikoga." 
\par 17 Ali kako su oni svejednako  navaljivali, reče im: "Pošaljite!" I poslaše pedesetoricu; tražili  su ga tri dana, ali ga nisu našli. 
\par 18 Vratiše se Elizeju, koji  je ostao u Jerihonu, i on im reče: "Nisam li vam rekao: 'Nemojte  ići!'" 
\par 19 Ljudi iz grada rekoše Elizeju: "Lijepo je u gradu, kako  to može vidjeti i naš gospodar, ali je voda loša i zemlja neplodna." 
\par 20 On reče: "Donesite mi novu zdjelu i metnite soli u nju!"  I oni mu je donesoše. 
\par 21 On tada ode na izvor, baci u nj soli  i reče: "Ovako govori Jahve: 'Ozdravljam ovu vodu. Neće od nje  više biti ni smrti ni neplodnosti.'" 
\par 22 I voda postade zdrava  i takva je do današnjeg dana, po riječi koju je izrekao Elizej. 
\par 23 Odatle je uzašao u Betel. Dok je išao putem, dječaci  bijahu izišli iz grada i rugahu mu se govoreći: "Hodi, ćelo!  Hodi, ćelo!" 
\par 24 On se obazre, pogleda ih i prokle ih u ime Jahvino.  I odmah iziđoše dva medvjeda iz šume i rastrgaše četrdeset i  dvoje djece. 
\par 25 Odatle ode on na goru Karmel, a odande se vrati  u Samariju. 


\chapter{3}

\par 1 Joram, sin Ahabov, zakralji se nad Izraelom u Samariji osamnaeste  godine Jošafatova kraljevanja u Judeji. I vladao je dvanaest  godina. 
\par 2 Činio je što je zlo u očima Jahvinim, ali ne kao njegov  otac i mati, jer je uklonio Baalov stup što ga bijaše podigao  njegov otac. 
\par 3 Ali je prianjao uz grijeh kojim je Jeroboam,  sin Nebatov, zavodio Izraela; i nije odstupao od njega. 
\par 4 Meša, kralj moapski, bio je stočar i slao je izraelskom  kralju u danak stotinu tisuća janjaca i vunu od stotine tisuća  ovnova. 
\par 5 Ali kad je umro Ahab, pobuni se kralj moapski protiv  izraelskog kralja. 
\par 6 U to je baš vrijeme kralj Joram izišao iz Samarije i izvršio  smotru svih Izraelaca. 
\par 7 Zatim je poručio judejskom kralju Jošafatu:  "Moapski se kralj pobunio protiv mene. Hoćeš li sa mnom u rat  protiv Moabaca?" Judejski kralj odgovori: "Hoću! Ja kao ti, moj  narod kao tvoj narod, moji konji kao i tvoji konji." 
\par 8 I doda:  "Kojim ćemo putem?" A drugi mu odgovori: "Kroz Edomsku pustinju." 
\par 9 I tako krenu izraelski kralj s judejskim kraljem i s kraljem  edomskim. Sedam su dana lutali, a nije bilo vode četama ni stoci  koja je išla za njima. 
\par 10 Tada povika kralj izraelski: "Jao, Jahve je pozvao ova tri kralja da ih preda u ruke Moapcima!" 
\par 11 Ali Jošafat reče: "Nema li tu proroka Jahvina da se preko  njega posavjetujemo s Jahvom?" Tada odgovori jedan između slugu  izraelskoga kralja: "Ovdje je Elizej, sin Šafatov, koji je lijevao  vodu na Ilijine ruke." 
\par 12 Jošafat reče: "U njega je riječ Božja."  I kralj izraelski, kralj judejski i kralj edomski odoše Elizeju. 
\par 13 A Elizej reče kralju izraelskom: "Što ja imam s tobom?  Potraži proroke svoga oca i proroke svoje majke!" Izraelski kralj  odgovori mu: "Ne! Jer Jahve je pozvao ova tri kralja da ih preda  u ruke Moapcima." 
\par 14 Elizej uzvrati: "Tako mi živoga Jahve Sebaota, komu služim, kad ne bih gledao na judejskog kralja Jošafata, ne bih ti obraćao pažnje niti bih te pogledao. 
\par 15 Sada mi dovedite  svirača."  I dok je glazbenik svirao, siđe ruka Jahvina nada nj. 
\par 16 I  on reče: "Ovako veli Jahve: 'Iskopajte u ovoj dolini mnogo jama. 
\par 17 Jer ovako veli Jahve: nećete osjetiti vjetra niti ćete vidjeti  dažda, a ova će se dolina napuniti vodom. I pit ćete vi, vaš  marva i vaša stoka.' 
\par 18 Ali to još nije ništa u očima Jahve:  on će predati Moab u vaše ruke. 
\par 19 Vi ćete zauzeti sve utvrđene  gradove, posjeći sve plodno drveće, zatrpati sve izvore i opustošiti  najbolja polja: kamenjem ćete ih zasijati." 
\par 20 I doista, ujutro, u vrijeme kad se prinosi žrtva, dođe  voda od Edoma i preplavi svu okolinu. 
\par 21 Kad su Moapci čuli da su kraljevi došli s njima ratovati, pozvaše sve koji bijahu sposobni za oružje i postaviše ih na  granicu. 
\par 22 Kad su ujutro ustali i kad je sunce granulo nad onom  vodom, Moapcima se sa strane voda učini crvenom kao krv. 
\par 23 I  rekoše: "To je krv! Zacijelo su se kraljevi međusobno pobili  i jedan drugoga pogubili. A sada: na plijen, Moapci!" 
\par 24 Ali kad su stigli do izraelskog tabora, digoše se Izraelci  i potukoše Moapce, tako te ovi pobjegoše pred njima. A Izraelci  pojuriše da dotuku Moapce. 
\par 25 Razorili su im gradove, bacali  svaki po kamen na najbolje njive da ih zaspu, zatrpali izvore  i posjekli sve plodno drveće. Konačno, ostao je samo grad Kir  Harešet; praćari su ga opkolili i tukli ga. 
\par 26 Kada je moapski  kralj vidio da neće izdržati bitku, uze sa sobom sedam stotina  ljudi naoružanih mačevima, pokuša se probiti i doći do kralja  edomskog, ali ne uspje. 
\par 27 Tada uze svoga sina prvenca, koji  ga imaše naslijediti, i prinese ga kao paljenicu na zidu. To se tako silno zgadilo Izraelcima te odoše od njih i vratiše  se u svoju zemlju. 


\chapter{4}

\par 1 Žena jednoga od proročkih sinova zamoli Elizeja ovako: "Tvoj  sluga, moj muž, umro je; a znaš da se tvoj sluga bojao Jahve.  Sada je došao vjerovnik da mi uzme oba sina i učini ih svojim  robovima." 
\par 2 Elizej joj reče: "Što ti mogu učiniti! Reci mi  što imaš u kući?" Ona odgovori: "Tvoja sluškinja nema ništa u  kući, osim vrča ulja." 
\par 3 Tada joj reče: "Idi i posudi od svih  svojih susjeda praznih sudova, ali neka ih ne bude premalo! 
\par 4 Zatim  se vrati kući, zatvori vrata za sobom i za svojim sinovima i  nalijevaj ulje u sve te sudove i pune stavljaj na stranu." 
\par 5 I  ode ona od njega, zatvori vrata za sobom i za svojim sinovima.  Oni su joj dodavali sudove, a ona ih punila. 
\par 6 I kad se sudovi  napuniše, reče ona svome sinu: "Dodaj mi još jedan sud!" Ali  joj on odgovori: "Nema više sudova." I ulje stade. 
\par 7 Ona ode  i kaza čovjeku Božjem, a on joj reče: "Idi, prodaj ulje i podmiri  svoj dug, a od ostatka živjet ćeš ti i tvoji sinovi!" 
\par 8 Jednoga je dana Elizej prolazio kroza Šunam. A živjela  ondje ugledna žena i ona ga pozva k stolu. Odonda, kad god prolazaše  onuda, uvratio bi se k njoj na jelo. 
\par 9 Ona reče svome mužu:  "Evo, znam i vidim da je svet onaj čovjek Božji što prolazi ovuda. 
\par 10 Načinimo mu sobicu na krovu, stavimo mu ondje postelju, stol, stolicu i svjetiljku: kad dođe k nama, povući će se onamo." 
\par 11 Jednoga dana dođe on onamo, povuče se u gornju sobu i počinu  ondje. 
\par 12 Reče zatim svome momku Gehaziju: "Pozovi tu Šunamku!"  On je pozva te ona stade preda nj. 
\par 13 I još mu reče: "Kaži joj:  'Lijepo se brineš za nas. Što možemo učiniti za te? Treba li  reći riječ za te kralju ili vojskovođi?'" Ali ona odgovori: "Ja  živim usred svoga naroda." 
\par 14 On nastavi: "Dakle, što da učinimo  za nju?" Gehazi odgovori: "Eto, nema sina, a muž joj je vremešan." 
\par 15 A on reče: "Pozovi je!" Pozva je, a ona stade kod ulaza. 
\par 16 "Dogodine u ovo doba", reče joj, "zagrlit ćeš sina u naručju."  A ona reče: "Ne, gospodaru moj, ne varaj službenice svoje!" 
\par 17 Ali  je žena doista zatrudnjela i rodila je sina druge godine u ono  doba, kako joj je rekao Elizej. 
\par 18 Dječak je rastao. Jednoga dana ode ocu kod žetelaca. 
\par 19 I potuži se ocu: "Jao, glava, glava moja!" A otac zapovjedi  jednom momku da ga odnese majci. 
\par 20 On ga uze i odvede ga njegovoj  majci. Na njenim je koljenima ostao do podne i onda umrije. 
\par 21 Ona  tada ode gore i položi ga u postelju Božjega čovjeka. Izišla  je zatim i zaključala vrata. 
\par 22 Potom je pozvala svoga muža  i rekla: "Pošalji mi jednoga od momaka i jednu magaricu; otrčat  ću do čovjeka Božjeg i vratit ću se." 
\par 23 On je upita: "Zašto  da danas pođeš k njemu? Nije ni mlađak niti je subota." Ali ona  odgovori: "Ostaj u miru!" 
\par 24 Pošto joj je momak osamario magaricu, ona će mu: "Povedi  i pođi! Ne zadržavaj me na putu, osim ako ti naredim." 
\par 25 Ode ona i dođe k čovjeku Božjem, na goru Karmel. Kada  je čovjek Božji ugleda izdaleka, reče svome momku Gehaziju: "Evo  one Šunamke. 
\par 26 Otrči pred nju i pitaj je: 'Kako si? Je li ti  muž dobro? Je li ti dijete zdravo?'" Ona odgovori: "Zdravi smo." 
\par 27 Kada je stigla do čovjeka Božjega na gori, obujmi mu noge.  Gehazi pristupi da je odmakne, ali mu čovjek Božji reče: "Pusti  je jer joj je duša ojađena. Jahve mi krije, nije mi ništa objavio." 
\par 28 A ona reče: "Zar sam ja tražila sina od svoga gospodara?  Nisam li ti govorila da me ne zavaravaš?" 
\par 29 On tada reče Gehaziju: "Opaši se, uzmi u ruku moj štap  pa idi! Ako koga susretneš, ne pozdravljaj ga; ako te tko pozdravi, ne odzdravljaj mu. Moj štap položi na dječaka." 
\par 30 Ali dječakova  majka reče: "Života mi Jahvina i tvoga, neću te ostaviti!" On  tada ustade i pođe za njom. 
\par 31 Gehazi je otišao prije njih i položio štap na dječaka, ali ne bješe ni glasa ni odziva. Vrati se on pred Elizeja i  javi mu: "Dječak se nije probudio." 
\par 32 Elizej uđe u kuću i nađe dječaka gdje mrtav leži na njegovoj  postelji. 
\par 33 Ušavši, zatvori vrata za sobom i pomoli se Jahvi. 
\par 34 Zatim se pope na postelju, leže na dječaka, položi svoja  usta na njegova usta, svoje oči na njegove oči, svoje ruke na  njegove ruke; disao je nad njim te se ugrijalo tijelo dječakovo. 
\par 35 Potom ustade i prošeta se po kući tamo-amo, zatim se opet  pope i disaše nad njim. A dječak tada kihnu sedam puta i otvori  oči. 
\par 36 I zovnu Elizej Gehazija i reče: "Pozovi tu Šunamku."  On je pozva. Kad je stigla preda nj, reče joj: "Uzmi svoga sina." 
\par 37 Ona, ušavši, pade mu pred noge i pokloni se do zemlje. Zatim  uze svoga sina te iziđe. 
\par 38 Elizej se vrati u Gilgal, a bijaše glad u zemlji. I kad  su proročki sinovi sjedili pred njim, reče svome momku: "Stavi  veliki lonac na vatru i skuhaj jelo sinovima proročkim." 
\par 39 Jedan  od njih ode u polje da nabere zelja, ali nađe divlju povijušu  i nabra s nje punu haljinu gorkih plodova. Vrati se i nareza  ih u lonac, jer nije znao kakvi su. 
\par 40 Usuše ljudima da jedu.  Ali kad su počeli jesti, povikaše: "Čovječe Božji! Smrt je u  loncu!" I nisu mogli jesti. 
\par 41 Tada će Elizej: "Donesite brašna!"  I baci ga u lonac i reče: "Uspite ljudima neka jedu!" I ništa  više nije bilo štetno u loncu. 
\par 42 Neki čovjek došao iz Baal Šališe i donio čovjeku Božjem  kruh od prvina, dvadeset ječmenih hljebova i kaše u torbi. A  on zapovjedi: "Daj ljudima neka jedu!" 
\par 43 Ali njegov momak odgovori:  "Kako to mogu postaviti pred stotinu ljudi?" On odgovori: "Podaj  ljudima i neka jedu, jer ovako veli Jahve: 'Jest će i preostat  će.'" 
\par 44 I postavi on pred njih. I jedoše i još preosta, prema  riječi Jahvinoj. 


\chapter{5}

\par 1 Naaman, vojskovođa aramskoga kralja, bijaše ugledan čovjek  i poštovan pred svojim gospodarom, jer je po njemu Jahve dao  pobjedu Aramejcima. Ali taj vrsni ratnik bješe gubav. 
\par 2 Jednom  su Aramejci otišli u pljačku i na području izraelskom zarobili  mladu djevojku, koja je zatim služila ženi Naamanovoj. 
\par 3 Ona  reče svojoj gospodarici: "Ah, kad bi se samo moj gospodar obratio  proroku koji je u Samariji! On bi ga zacijelo oslobodio gube!" 
\par 4 Naaman ode i obavijesti svoga gospodara: "Tako je i tako  rekla djevojka koja je došla iz zemlje izraelske." 
\par 5 Aramejski  kralj odgovori: "Idi onamo! Ja ću poslati pismo kralju izraelskom."  Naaman ode; ponio je deset talenata srebra; šest tisuća zlatnih  šekela i deset svečanih haljina. 
\par 6 I predade kralju izraelskom  pismo što kazivaše: "Uz pismo koje ti stiže, šaljem ti, evo,  svoga slugu Naamana da ga izliječiš od gube." 
\par 7 Kad je izraelski kralj pročitao pismo, razdera haljine  na sebi i reče: "Zar sam ja Bog da mogu usmrćivati i oživljavati  te ga ovaj šalje k meni da ga izliječim od njegove gube? Gledajte  samo kako traži povoda da me napadne!" 
\par 8 A kad je Elizej saznao da je kralj izraelski razderao  na sebi odjeću, poruči kralju: "Zašto si razderao haljine svoje?  Neka onaj samo dođe k meni i neka se uvjeri da ima prorok u Izraelu." 
\par 9 I tako Naaman stiže sa svojim konjima i kolima i stade  pred vratima Elizejeve kuće. 
\par 10 A Elizej poruči dolazniku: "Idi  i okupaj se sedam puta u Jordanu i tijelo će ti opet biti čisto." 
\par 11 Naaman se naljuti i pođe govoreći: "Gle, ja mišljah, izići  će preda me, zazvat će ime Jahve, Boga svoga, stavit će ruku  na bolesno mjesto i odnijeti mi gubu. 
\par 12 Nisu li rijeke u Damasku, Abana i Parpar, bolje od svih voda izraelskih? Ne bih li se  mogao u njima okupati da postanem čist?" Okrenu se i ode odande ljutit. 
\par 13 Ali mu pristupiše sluge  njegove i rekoše: "Oče moj, da ti je prorok odredio i teže, zar  ne bi učinio? A nekmoli kad ti je rekao: 'Okupaj se, i bit ćeš  čist.'" 
\par 14 I tako siđe, opra se sedam puta u Jordanu, prema  riječi čovjeka Božjega; i tijelo mu posta opet kao u malog djeteta  - očistio se! 
\par 15 Vrati se on Elizeju sa svom svojom pratnjom, uđe, stade  preda nj i reče mu: "Evo, sad znam da nema Boga na svoj zemlji, osim u Izraelu. Zato te molim, primi dar od svoga sluge." 
\par 16 Ali  on odgovori: "Tako mi živog Jahve, komu služim, ne primam." Naaman  navaljivaše da primi, ali on ne htjede. 
\par 17 Tada Naaman reče:  "Dobro, kad nećeš. Ali barem dopusti da meni, tvome sluzi, dadu  ove zemlje koliko mogu ponijeti dvije mazge. Jer sluga tvoj neće  više prinositi pomirnica ni klanica drugim bogovima nego samo  Jahvi. 
\par 18 A Jahve neka oprosti ovo sluzi tvome: kad moj gospodar  pođe u hram Rimonov da se ondje pokloni, pa se nasloni na moju  ruku, onda bih se i ja poklonio u hramu Rimonovu. Neka Jahve  oprosti taj čin sluzi tvome." 
\par 19 A on mu reče: "Idi s mirom."  I udalji se Naaman i prijeđe dio puta. 
\par 20 Gehazi, momak Elizeja, Božjega čovjeka, pomisli: "Moj  je gospodar poštedio Naamana, toga Aramejca, i nije primio ništa  od onoga što mu je ponudio. Tako mi živog Jahve, potrčat ću ja  za njim i uzet ću štogod od njega." 
\par 21 I Gehazi pohitje za Naamanom. Kada ga je Naaman vidio  da za njim trči, skoči mu sa svojih kola u susret i upita ga:  "Je li sve dobro?" 
\par 22 On odgovori: "Dobro je. Moj gospodar šalje  me da ti kažem: upravo su stigla dva mladića iz Efrajimove gore, dvojica od proročkih sinova. Daj za njih, molim te, talenat  srebra i dvoje haljine." 
\par 23 Naaman reče: "Uzmi, molim te, dva  telenta!" I navaljivaše da uzme. I zaveza dva talenta srebra  u dvije kese, i dvoje haljine, i predade ih dvojici svojih momaka  da ih nose pred njim. 
\par 24 Kad je Gehazi stigao do Ofela, uze  ih iz njihovih ruku i pohrani ih u kući. Zatim otpusti ljude  i oni odoše. 
\par 25 Kad je došao, stao je pred svoga gospodara. Elizej ga  upita "Odakle, Gehazi?" On odgovori: "Tvoj sluga nije nikamo  odlazio." 
\par 26 Ali Elizej reče: "Nije li Duh moj bio s tobom kad  je netko sišao sa svojih kola te izišao preda te? Sad si primio  srebro, pa možeš kupiti maslinike, vinograde, sitno i krupno  blago, sluge i sluškinje. 
\par 27 Ali će se guba Naamanova prilijepiti  za te i za tvoje potomstvo zauvijek." I Gehazi se udalji od njega, bijel od gube kao od snijega. 


\chapter{6}

\par 1 Proročki sinovi rekoše Elizeju: "Gle, tijesan nam je prostor  u tebe. 
\par 2 Nego da odemo do Jordana, pa da svaki ondje uzmemo  po brvno i načinimo sebi ondje prebivalište." On odgovori: "Idite." 
\par 3 Jedan od njih reče mu: "Udostoj se poći sa svojim slugama."  On odgovori: "Hoću." 
\par 4 I pođe s njima. Kad su stigli do Jordana, uzeše sjeći drva. 
\par 5 A dok je jedan od njih tesao gredu, pade  mu sjekira u vodu i on povika: "Jao, gospodaru! I još je bila  posuđena!" 
\par 6 A čovjek Božji upita ga: "Gdje je pala?" Onaj mu  pokaza mjesto. Tada on odsiječe komad drveta, baci ga na ono  mjesto i učini da sjekira ispliva. 
\par 7 I reče: "Izvadi je!" I  čovjek pruži ruku te je uze. 
\par 8 Aramejski kralj bio u ratu s Izraelom. Posavjetovao se  sa svojim časnicima i rekao: "Podignite šatore na tom mjestu." 
\par 9 Ali Elizej poruči izraelskom kralju: "Čuvaj se onoga mjesta  jer su se Aramejci ondje utaborili." 
\par 10 I kralj izraelski upozori  ljude na mjesto za koje mu je rekao čovjek Božji. On je upozoravao  i kralj se čuvao; a bilo je to više puta. 
\par 11 Srce aramejskog kralja uznemiri se zbog toga, pa on pozva  svoje časnike te ih upita: "Nećete li mi reći tko od naših drži  s kraljem Izraelovim?" 
\par 12 Jedan od časnika odgovori: "Ne, gospodaru  kralju; Elizej, prorok Izraelov, otkriva izraelskom kralju riječi  koje kazuješ u svojoj spavaonici." 
\par 13 On reče: "Idite i pogledajte  gdje je, pa ću već poslati da ga uhvate." I javiše mu: "Eno ga  u Dotanu." 
\par 14 Tada kralj posla onamo konje, kola i jake čete.  Oni stigoše noću i opkoliše grad. 
\par 15 Ujutro, ustavši, čovjek Božji iziđe, a to oko grada stoji  vojska s konjima i kolima! Njegov mu momak reče: "Ah, gospodaru  moj, što nam je činiti?" 
\par 16 A on odgovori: "Ne boj se jer ih  ima više s nama nego s njima." 
\par 17 I Elizej se pomoli ovako:  "Jahve, otvori mu oči da vidi!" I Jahve otvori oči momku i on  vidje: gora oko Elizeja sva prekrivena ognjenim konjima i kolima! 
\par 18 Kad su Aramejci sišli prema njemu, Elizej se ovako pomoli  Jahvi: "Udari sljepoćom ove ljude!" I na riječ Elizejevu udari  ih sljepoćom. 
\par 19 Elizej im reče: "Nije ovo put i nije ovo grad.  Pođite za mnom, ja ću vas odvesti čovjeku koga tražite." Ali  ih odvede u Samariju. 
\par 20 Kad su ulazili u Samariju, Elizej reče:  "Jahve, otvori ovima oči da progledaju." Jahve im otvori oči  i oni vidješe da su usred Samarije! 
\par 21 Kad ih vidje kralj Izraela, reče Elizeju: "Treba li ih  poubijati, oče moj?" 
\par 22 A on odgovori: "Nemoj ih ubiti. Zar  ćeš ubiti one koje nisi zarobio svojim lukom i mačem? Ponudi  im kruha i vode; neka jedu i piju i neka se vrate svome gospodaru." 
\par 23 Kralj im priredi veliku gozbu. Pošto su jeli i pili, otpusti  ih. I vratiše se svome gospodaru. I tako aramejski pljačkaši  nisu više zalazili na izraelsko tlo. 
\par 24 Dogodi se poslije toga te aramejski kralj Ben-Hadad skupi  svu svoju vojsku i uzađe i opkoli Samariju. 
\par 25 I nasta velika  glad u Samariji, a opsada potraja toliko da je magareća glava  stajala osamdeset šekela srebra, a četvrt kaba golubinje nečisti  pet šekela srebra. 
\par 26 Kada je kralj prolazio po zidinama, neka mu žena vikne:  "Pomozi, gospodaru kralju!" 
\par 27 On odgovori: "Neka ti pomogne  Jahve! Kako ću ti ja pomoći? Nečim s gumna ili iz tijeska?" 
\par 28 Još  joj kralj reče: "Što ti je?" Ona odgovori: "Ova mi je žena rekla:  'Daj svoga sina da ga pojedemo danas, a sutra ćemo pojesti moga!' 
\par 29 Skuhale smo moga sina i pojele ga. A sutradan rekoh joj:  'Daj svoga sina da ga pojedemo.' Ali je ona sakrila svoga sina." 
\par 30 Kada je kralj čuo riječi te žene, razdrije na sebi haljine.  I kad je išao po zidinama, narod vidje da mu je na tijelu kostrijet. 
\par 31 I reče tada kralj: "Neka mi Bog učini ovo zlo i doda drugo  ako glava Elizeja, sina Šafatova, ostane danas na njegovim ramenima!" 
\par 32 Elizej sjedio u svojoj kući i starješine sjedile s njim.  Kralj je ispred sebe poslao glasnika, ali Elizej reče starješinama, prije nego što je glasnik stigao do njega: "Vidite li da je  onaj krvnički sin naredio da mi skinu glavu? Pazite: kada glasnik  stigne, zatvorite vrata i odbijte ga od vrata. Ne čuje li se  topot koraka njegova gospodara za njim?" 
\par 33 Dok im je još govorio, kralj stupi preda nj i reče mu: "Ova je nevolja, gle, od Jahve!  Što da još očekujem od Jahve?" 


\chapter{7}

\par 1 Elizej reče tada: "Čuj riječ Jahvinu! Ovako veli Jahve: 'Sutra  će u ovo doba na vratima Samarije biti mjera finoga brašna za  šekel, a dvije mjere ječmenog brašna za šekel.'" 
\par 2 Dvorjanik, o čiju se ruku kralj oslanjao, odgovori čovjeku Božjemu: "I  kad bi Jahve načinio okna na nebu, bi li to moglo biti?" A Elizej  odgovori: "Vidjet ćeš svojim očima, ali nećeš jesti." 
\par 3 A pred gradskim vratima bijahu četiri gubavca; rekoše  oni jedan drugome: "Zašto stojimo ovdje i očekujemo smrt? 
\par 4 Ako  odlučimo ući u grad, glad je u gradu te ćemo ondje umrijeti;  ako ostanemo ovdje, opet ćemo umrijeti. Hajde! Pobjegnimo i prijeđimo  u aramejski tabor: ako nas ostave na životu, živjet ćemo; ako  nas ubiju, pa dobro: umrijet ćemo!" 
\par 5 U sumračje, ustavši, krenuše  odande u aramejski tabor. Stigoše do ruba tabora, i gle - ondje  nikoga! 
\par 6 Jer je Jahve učinio te se u taboru aramejskom čula  buka kola i konja, buka goleme vojske. I govorili su među sobom:  "Eto, kralj Izraela najmio je protiv nas kraljeve hetitske i  kraljeve egipatske da krenu protiv nas." 
\par 7 Digli su se i pobjegli  u sumraku: ostavili su svoje šatore, konje i magarce, sav tabor  kakav bijaše. Pobjegli su da iznesu živu glavu. 
\par 8 Kad su gubavci, dakle, došli do ruba tabora, uvukoše se  u jedan šator. Pošto su se najeli i napili, uzeše odande srebro, zlato i haljine pa odoše da ih sakriju. Vratiše se onda pa uđoše  u drugi šator: uzeše plijen iz njega te odoše i sakriše ga. 
\par 9 Rekoše tada jedan drugome: "Ne smijemo tako raditi. Današnji  je dan pun dobrih vijesti, a mi šutimo. Ako dočekamo jutro, bit  ćemo krivi. Zato pođimo! Javimo dvoru novost." 
\par 10 I vratiše  se, pozvaše gradsku stražu i javiše: "Otišli smo u tabor aramejski, a ondje nigdje čovjeka ni ljudskoga glasa; samo konji privezani  i magarci, a šatori ostavljeni kakvi jesu." 
\par 11 Stražari viknuše  i dojaviše u unutrašnjost dvora. 
\par 12 Kralj ustade noću i reče svojim časnicima: "Ja ću vam  objasniti što su nam učinili Aramejci. Kako znaju da smo gladni, izišli su iz tabora i sakrili se u polju, misleći: već će oni  izići iz grada, a mi ćemo ih žive pohvatati i ući u grad." 
\par 13 A  jedan između njegovih časnika odgovori: "Neka se uzme ipak pet  od preostalih konja. S njima će biti kao sa svim mnoštvom Izraelovim  koje je ovdje preostalo. Pošaljimo ih pa ćemo vidjeti." 
\par 14 I  uzeše dva konja kolska i kralj posla ljude za aramejskim taborom  govoreći: "Idite, izvidite!" 
\par 15 Išli su za njima do Jordana;  put bijaše sav prekriven haljinama i stvarima koje su Aramejci  pobacali u bijegu. Glasnici se vratiše i obavijestiše kralja. 
\par 16 I narod iziđe i uze pljačkati aramejski tabor: i bijaše  mjera finoga brašna za šekel, a dvije mjere ječmenoga za jedan  šekel, prema riječi Jahvinoj. 
\par 17 Kralj je postavio na gradska vrata onoga dvorjanika o  čiju se ruku oslanjao; a narod ga izgazi na vratima i on umrije, prema riječi što ju je rekao Božji čovjek kad mu kralj bijaše  došao. 
\par 18 Dogodilo se kako je čovjek Božji rekao kralju: "Sutra  u ovo doba na vratima Samarije bit će dvije mjere ječmenoga brašna  za šekel i mjera finoga brašna za šekel." 
\par 19 Dvorjanik je odgovorio  Elizeju: "Pa da Jahve načini i okna na nebu, bi li moglo biti  što kažeš?" Elizej mu je odgovorio: "Vidjet ćeš svojim očima, ali nećeš jesti." 
\par 20 I doista, tako mu se dogodilo: izgazio  ga narod na vratima te on umrije. 


\chapter{8}

\par 1 Elizej bijaše savjetovao ženi kojoj je oživio sina: "Ustani, pođi sa svojom obitelji i skloni se kao tuđinka bilo kamo, jer  je Jahve pustio glad; već je došla u zemlju za sedam godina." 
\par 2 Žena usta i učini kako joj je rekao čovjek Božji: otišla je, ona i njena obitelj, i ostala sedam godina u zemlji filistejskoj. 
\par 3 Na kraju sedme godine žena se vrati iz zemlje filistejske  i ode kralju da zatraži svoju kuću i njivu. 
\par 4 Upravo je kralj razgovarao s Gehazijem, momkom Božjega  čovjeka. Govorio mu je: "Pripovijedaj mi o svim velikim djelima  koja je Elizej učinio." 
\par 5 I kad je pripovijedao kralju o uskrisenju  djeteta, eto žene kojoj je Elizej oživio sina; ona se obrati  kralju radi svoje kuće i njive. A Gezahi reče: "Gospodaru kralju, evo one žene i evo njena sina koga je Elizej oživio." 
\par 6 Kralj  upita ženu i ona mu sve pripovjedi. Tada joj kralj dade jednoga  slugu, komu naredi: "Neka joj se vrati sve što je njeno i svi  prihodi od njive od dana kada je ostavila zemlju do danas!" 
\par 7 Elizej dođe u Damask. Ben-Hadad, kralj aramejski, bijaše  obolio. Odmah mu javiše: "Božji čovjek došao ovamo." 
\par 8 Tada  reče kralj Hazaelu: "Uzmi sa sobom dar pa idi pred Božjeg čovjeka.  I preko njega se posavjetuj s Jahvom da bi saznao hoću li se  izliječiti od ove bolesti." 
\par 9 Hazael ode pred Elizeja i donese mu u dar što bijaše od  ponajboljeg u Damasku, sve to natovareno na četrdeset deva. Dođe  on i stade preda nj i reče: "Tvoj sin Ben-Hadad, kralj aramejski, šalje me k tebi i pita hoće li ozdraviti od one bolesti." 
\par 10 Elizej  mu odgovori: "Idi i reci mu: 'Ozdravit ćeš, dakako!' Ali mi je  Jahve pokazao da će umrijeti." 
\par 11 I čovjek Božji uprije pogled  preda se, smeten, i zaplaka. 
\par 12 Hazael reče: "Zašto plačeš,  moj gospodaru?" Elizej odgovori: "Zato što znam sva zla koja  ćeš ti učiniti Izraelcima: spalit ćeš im utvrde, mačem ćeš poubijati  njihove ratnike, njihovu ćeš djecu satirati, a trudne žene parati." 
\par 13 Hazael reče: "Ali što je tvoj sluga? Zar je pas da učini  tako strašne stvari?" Elizej odgovori: "U jednoj Jahvinoj objavi  vidio sam tebe kao kralja aramejskog." 
\par 14 Hazael ode od Elizeja i vrati se svome gospodaru, koji  ga upita: "Što ti je rekao Elizej?" On odgovori: "Rekao mi je  da ćeš ozdraviti." 
\par 15 Ali sutradan uze pokrivač, namoči ga u  vodi i pokri kralja preko lica te on umrije. A na njegovo mjesto  zakralji se Hazael. 
\par 16 Pete godine kraljevanja Jorama, sina Ahabova, u Izraelu, postade judejskim kraljem Joram, sin Jošafatov. 
\par 17 Bile su  mu trideset i dvije godine kad se zakraljio, a kraljevao je osam  godina u Jeruzalemu. 
\par 18 Živio je poput izraelskih kraljeva,  kao i dom Ahabov, jer mu je kći Ahabova bila žena; radio je što  je zlo u Jahvinim očima. 
\par 19 Ipak Jahve ne htjede razoriti Judeje  zbog sluge svoga Davida, zato što mu obeća da će dati svjetiljku  njemu i njegovim sinovima zauvijek. 
\par 20 U njegovo se vrijeme Edomci odmetnuše ispod judejske  vlasti i postaviše sebi kralja. 
\par 21 Joram ode u Seir i s njim  sva bojna kola. Diže se noću i pobi Edomce koji su bili opkolili  njega i zapovjednike bojnih kola. Narod pobježe u svoje šatore. 
\par 22 Ipak su se Edomci oslobodili ispod judejske vlasti sve do  danas. U isto se doba odmetnu i Libna. 
\par 23 Ostala povijest Jorama, sve što je učinio, zar to nije  zapisano u knjizi Ljetopisa kraljeva judejskih? 
\par 24 Joram počinu  kraj svojih otaca i bi pokopan k svojim ocima u Davidovu gradu.  Njegov sin Ahazja zakralji se mjesto njega. 
\par 25 Dvanaeste godine Jorama, sina Ahabova, kralja Izraela, postade judejskim kraljem Ahazja, sin Joramov. 
\par 26 Ahazji bijahu  dvadeset i dvije godine kad se zakraljio, a kraljevao je godinu  dana u Jeruzalemu. Mati mu se zvala Atalija, a bila je kći izraelskog  kralja Omrija. 
\par 27 I on je hodio putem obitelji Ahabove i činio  je zlo u očima Jahvinim, kao i obitelj Ahabova, jer je s njom  bio u rodu. 
\par 28 On je pošao s Joramom, sinom Ahabovim, u Ramot Gilead  u boj protiv Hazaela, aramskog kralja. 
\par 29 Kralj Joram vratio  se u Jizreel da se liječi od rana što mu ih zadadoše u Rami kad  se borio s aramejskim kraljem Hazaelom. Joramov sin Ahazja, judejski  kralj, sišao je u Jizreel da posjeti Ahabova sina Jorama jer  se Joram razbolio. 


\chapter{9}

\par 1 Prorok Elizej pozva jednoga od proročkih sinova i reče mu:  "Opaši se, uzmi sa sobom ovu posudu s uljem pa idi u Ramot Gilead. 
\par 2 Kad onamo stigneš, potraži Jehua, sina Jošafatova, sina Nimšijeva.  Kad ga nađeš, izvedi ga između njegovih drugova i uvedi ga u  pokrajnju sobu. 
\par 3 I uzmi posudu s uljem, izlij mu je na glavu  i reci: 'Ovako veli Jahve: Pomazao sam te za kralja izraelskoga.'  Zatim otvori vrata i bježi, ne oklijevaj." 
\par 4 Tada mladi prorok ode u Ramot Gilead. 
\par 5 Kad je stigao, zapovjednici vojske upravo su sjedili na okupu. On reče: "Imam  ti riječ reći, zapovjedniče!" Jehu upita: "Komu od nas?" On odgovori:  "Tebi, zapovjedniče!" 
\par 6 Jehu tada ustade i uđe u kuću. Mladi  mu čovjek izli ulje na glavu i reče mu: "Ovako veli Jahve, Bog  Izraelov: 'Pomazao sam te za kralja nad Jahvinim narodom, nad  Izraelom. 
\par 7 Ti ćeš pobiti obitelj Ahaba, gospodara tvoga, a  ja ću osvetiti krv svojih slugu proroka i krv sviju službenika  Jahvinih na Izebeli 
\par 8 i na svoj obitelji Ahabovoj. Iskorijenit  ću Ahabu sve što mokri uza zid, robove i slobodnjake u Izraelu. 
\par 9 Učinit ću s domom Ahabovim kao s domom Jeroboama, sina Nebatova, i kao s domom Baše, sina Ahijina. 
\par 10 A Izebelu će proždrijeti  psi na polju jizreelskom i nitko je neće pokopati.'" - Zatim  otvori vrata i pobježe. 
\par 11 Jehu iziđe k časnicima svoga gospodara. Oni ga upitaše:  "Je li sve u miru? Zašto je ta budala dolazila k tebi?" On im  odgovori: "Znate čovjeka i besjedu njegovu." 
\par 12 Oni rekoše:  "Ne znamo! Kazuj nam!" On im reče: "Govorio mi je tako i tako  i rekao mi: 'Ovako veli Jahve: Pomazao sam te za kralja nad Izraelom.'" 
\par 13 Odmah oni uzeše svoje ogrtače i prostriješe ih pred njim  po stepenicama, zatrubiše u rogove i povikaše: "Jehu je kralj!" 
\par 14 Tako Jehu, sin Jošafata, sina Nimšijeva, skova urotu  protiv Jorama - Joram je tada branio Ramot Gilead sa svim Izraelcima  protiv Hazaela, aramejskog kralja. 
\par 15 Ali se kralj Joram vratio  u Jizreel da liječi rane koje su mu zadali Aramejci u boju s  Hazaelom, aramejskim kraljem. - I reče Jehu: "Ako vam je po volji, neka nitko ne utekne iz grada da odnese vijest u Jizreel." 
\par 16 Jehu se tada pope na kola i ode prema Jizreelu, jer je  Joram ondje bolovao, i Ahazja, kralj judejski, došao ga posjetiti. 
\par 17 Stražar koji je stajao na kuli u Jizreelu, videći da  dolazi Jehuova četa, javi: "Vidim nekakvu četu." Joram naredi:  "Uzmi konjanika i pošalji ga pred njih da upita: je li sve s  mirom." 
\par 18 Ode konjanik preda nj i reče: "Ovako veli kralj:  je li sve s mirom?" - Jehu odgovori: "Što te briga je li s mirom!  Hajde za mnom." Stražar javi: "Glasnik je stigao do njih, ali  se ne vraća." 
\par 19 Kralj posla drugoga konjanika. Taj dođe k njima  i upita: "Ovako veli kralj: je li sve s mirom?" - Jehu mu odgovori:  "Što te briga je li s mirom! Hajde za mnom." 
\par 20 Stražar opet  javi: "Došao je do njih, ali se ne vraća. A vožnja je kao vožnja  Jehua, sina Nimšijeva: vozi kao mahnit!" 
\par 21 Joram reče: "Preži!"  I upregoše u njegova kola. Joram, kralj Izraela, i judejski kralj  Ahazja iziđoše, svaki u svojim kolima, u susret Jehuu. Susretoše  ga u polju Nabota Jizreelca. 
\par 22 Kad Joram ugleda Jehua, upita ga: "Je li sve u miru,  Jehu?" Ovaj odgovori: "Kakvu miru dok traju bludništva tvoje  majke Izebele i njena mnoga čaranja!" 
\par 23 Joram okrenu i udari  u bijeg govoreći Ahazji: "Izdaja, Ahazja!" 
\par 24 Jehu se lati luka, ustrijeli Jorama među pleća: strijela mu prođe posred srca te  se on sruši u kola. 
\par 25 Jehu reče svome dvorjaniku Bidkaru: "Digni  ga i baci na njivu Nabota Jizreelca. Sjeti se: kad smo ja i ti  jahali za njegovim ocem Ahabom, kako Jahve izreče protiv njega: 
\par 26 'Kunem se: kako sinoć vidjeh krv Nabotovu i krv njegovih  sinova,' riječ je Jahvina, 'tako ću ti vratiti isto na ovome  polju,' riječ je Jahvina. Digni ga, dakle, i baci ga na to polje, prema riječi Jahvinoj." 
\par 27 Kada je to vidio judejski kralj Ahazja, pobježe prema  Bet Haganu, ali ga je Jehu gonio i naredio: "Ubijte i njega!"  Ranili su ga u kolima na brdu Guru, koje se nalazi kod Jibleama.  Ali je umakao u Megido i ondje umrije. 
\par 28 Njegove su ga sluge  u kolima prenijele u Jeruzalem i sahranile ga u grobnici kraj  njegovih otaca, u Davidovu gradu. 
\par 29 Jedanaeste godine kraljevanja  Jorama, sina Ahabova, Ahazja postade kralj nad Judejom. 
\par 30 A Jehu bijaše ušao u Jizreel. Kad je to čula Izebela, namaza oči, uresi glavu i pogleda s prozora. 
\par 31 I kad je Jehu  ulazio na vrata, ona reče: "Kako je, Zimri, ubojico svoga gospodara?" 
\par 32 Jehu okrenu lice prema prozoru i reče: "Tko je sa mnom, tko?"  I dva-tri dvoranina pogledaše prema njemu. 
\par 33 On reče: "Bacite  je dolje." I oni je baciše. Njena je krv poprskala zidove i konje, koji je pogaziše. 
\par 34 Ušao je on, jeo i pio, a zatim naredio:  "Pogledajte onu prokletnicu i sahranite je, jer je bila kraljevska  kći." 
\par 35 I odoše da je sahrane, ali ne nađoše ništa od nje,  osim lubanje, nogu i ruku. 
\par 36 Vratiše se i javiše, a Jehu reče:  "To je riječ koju je Jahve objavio preko svoga sluge Ilije Tišbijca:  'U polju jizreelskom psi će proždrijeti Izebelino tijelo. 
\par 37 Izebelino  truplo bit će kao gnoj u polju, da se neće moći kazati: Ovo je  Izebela.'" 


\chapter{10}

\par 1 U Samariji bijaše sedamdeset Ahabovih sinova. Jehu napisa  pismo i posla ga u Samariju zapovjednicima grada, starješinama  i skrbnicima Ahabove djece. Kazivaše u njemu: 
\par 2 "Sada, kad vam  stigne ovo pismo - vi, u kojih su sinovi vašeg gospodara, koji  imate kola i konje, tvrde gradove i oružje - 
\par 3 pogledajte koji  je između sinova vašeg gospodara najbolji i najdostojniji, pa  ga postavite na prijestolje njegova oca i borite se za dom svoga  gospodara." 
\par 4 Ali se oni veoma uplašiše i rekoše: "Eto, dva mu kralja  nisu mogla odoljeti, kako ćemo mu mi odoljeti?" 
\par 5 Upravitelj  dvora, zapovjednik grada, starješine i skrbnici poručiše ovo  Jehuu: "Mi smo tvoje sluge, činit ćemo sve što nam budeš naredio;  kraljem proglašavati nećemo nikoga. Čini što misliš da je dobro." 
\par 6 Jehu im napisa drugo pismo i u njemu reče: "Ako ste za  mene i želite me slušati, uzmite glave ljudi, sinova svoga gospodara, i potražite me sutra u ovo doba u Jizreelu." Sedamdeset je naime  kraljevih sinova bilo kod uglednih građana koji su ih odgajali. 
\par 7 I kad im je stiglo ovo pismo, uzeli su kraljeve sinove i pobili  ih svih sedamdeset. Njihove su glave metnuli u košare i poslali  su ih njemu u Jizreel. 
\par 8 Glasnik dođe i javi mu: "Donijeli su glave kraljevih sinova."  On reče: "Stavite ih do sutra kod ulaznih vrata, u dvije hrpe." 
\par 9 Ujutro iziđe, stade i reče svomu narodu: "Vi ste pravedni!  Ja sam se urotio protiv svoga gospodara i ja sam ga ubio, ali  tko pobi sve ove? 
\par 10 Znajte, dakle, da nije izostala nijedna  riječ koju reče Jahve o obitelji Ahabovoj; nego je Jahve izvršio  sve što je rekao preko sluge svoga Ilije." 
\par 11 I Jehu pobi sve  koji su u Jizreelu ostali iz kuće Ahabove, sve velikaše njegove, pouzdanike i svećenike njegove. Nije poštedio nikoga. 
\par 12 Potom usta Jehu i pođe u Samariju. Kad je bio na cesti  kod Bet Ekeda pastirskoga, 
\par 13 nađe braću judejskog kralja Ahazje  te ih upita: "Tko ste?" Oni mu odgovoriše: "Mi smo braća Ahazjina, a silazimo da pozdravimo sinove kraljeve i sinove kraljičine." 
\par 14 Tada zapovjedi: "Pohvatajte ih žive!" I žive ih pohvataše  i pobiše ih na studencu kod Bet Ekeda, njih četrdeset i dvojicu.  Nije ostavio ni jednoga od njih. 
\par 15 Otišavši odatle, nađe Jonadaba, sina Rekabova, koji mu  je dolazio u susret. On ga pozdravi i reče mu: "Je li tvoje srce  iskreno prema mome, kao što je moje prema tvome srcu?" Jonadab  odgovori: "Jest." - "Ako je tako, daj mi ruku." Jonadab mu pruži  ruku i Jehu ga posadi kraj sebe na kola. 
\par 16 I reče mu: "Hodi  sa mnom, divit ćeš se mojoj revnosti za Jahvu." I odvede ga na  svojim kolima. 
\par 17 Ušao je u Samariju i poubijao sve preživjele  iz obitelji Ahabove u Samariji. Sve ih je iskorijenio po riječi  koju Jahve bijaše rekao Iliji. 
\par 18 Jehu je sakupio sav narod i rekao mu: "Ahab je malo poštivao  Baala; Jehu će ga više poštivati. 
\par 19 Sada mi pozovite sve proroke  Baalove, sve njegove sluge i sve njegove svećenike, neka ni jedan  ne izostane, jer ću žrtvovati veliku žrtvu Baalu. Tko izostane, izgubit će život." Jehu je radio lukavo, da bi uništio Baalove  vjernike. 
\par 20 Jehu reče: "Sazovite svečani zbor Baalu." I sazvaše  ga. 
\par 21 Jehu je nato poslao glasnike po svem Izraelu i došli  su svi Baalovi vjernici: nije bilo ni jednoga da bi izostao.  Skupili su se u Baalov hram, koji se ispunio od jednoga zida  do drugoga. 
\par 22 Jehu reče čuvaru haljina: "Iznesi haljine svim  Baalovim vjernicima." I iznese im haljine. 
\par 23 Jehu uđe u hram  Baalov s Jonadabom, sinom Rekabovim, i reče Baalovim vjernicima:  "Provjerite dobro da nema ovdje među vama Jahvina sluge nego  samih Baalovih vjernika." 
\par 24 I pođe žrtvovati klanice i paljenice. Ali je Jehu postavio vani osamdeset svojih ljudi i rekao  im: "Ako koji od vas pusti da utekne i jedan od ovih ljudi što  ih predajem u vaše ruke, svojim će životom platiti njegov život." 
\par 25 Kad je Jehu završio prinos paljenice, naredi tjelesnoj straži  i dvoranima: "Uđite, pobijte ih! Nitko neka ne iziđe!" Tjelesna  straža i dvorani uđoše, pobiše ih oštricom mača i prodriješe  sve do svetišta Baalova hrama. 
\par 26 Iznesoše Baalov lik iz hrama  i spališe ga. 
\par 27 Raskopaše žrtvenik Baalov, srušiše i hram Baalov  i pretvoriše ga u jame za nečist, koje su ostale do danas. 
\par 28 Tako je Jehu istrijebio Baala iz Izraela. 
\par 29 Ali se  Jehu nije okrenuo od grijeha Jeroboama, sina Nebatova, kojima  je zavodio Izraela, od zlatnih telaca u Betelu i Danu. 
\par 30 Jahve  je rekao Jehuu: "Zato što si dobro izvršio ono što mi je po volji  i što si učinio sve što sam nosio u srcu protiv kuće Ahabove, tvoji će sinovi sve do četvrtoga koljena sjediti na prijestolju  Izraelovu." 
\par 31 Ali Jehu nije vjerno i svim srcem svojim slijedio  zakon Jahve, Boga Izraelova. Nije se odvratio od grijeha kojima  je Jeroboam zavodio Izraela. 
\par 32 U ono je vrijeme Jahve počeo krnjiti zemlju izraelsku, i Hazael se tukao s Izraelcima na svom području, 
\par 33 od Jordana  prema sunčevu izlasku, u svoj zemlji Gileadu, u zemlji Gadovoj, Rubenovoj i Manašeovoj, sve od Aroera na obali Arnona, do Gileada  i Bašana. 
\par 34 Ostala povijest Jehuova, sve što je učinio, sva  njegova djela, zar to nije zapisano u knjizi Ljetopisa kraljeva  izraelskih? 
\par 35 Počinuo je kraj svojih otaca i pokopaše ga u  Samariji. Joahaz, sin njegov, zakralji se mjesto njega. 
\par 36 Jehu  je vladao u Samariji nad Izraelom dvadeset i osam godina. 


\chapter{11}

\par 1 Zato Ahazjina mati Atalija, vidjevši gdje joj sin poginu,  ustade i posmica sav kraljevski rod. 
\par 2 Ali Jošeba, kći kralja  Jorama i sestra Ahazjina, uze Ahazjina sina Joaša; ukravši ga  između kraljevih sinova koje su ubijali, metnu ga s dojiljom  u ložnicu. Tako ga je sakrila od Atalije te nije pogubljen. 
\par 3 Bio  je sakriven u Domu Jahvinu šest godina, sve dok je zemljom vladala  Atalija. 
\par 4 Sedme godine Jojada posla po satnike Karijaca i tjelesnu  stražu i pozva ih k sebi u Dom Jahvin. Sklopi s njima savez,  zakle ih i pokaza im kraljeva sina. 
\par 5 I reče im: "Evo što valja  da učinite: trećina vas koji subotom ulazite u službu neka čuva  stražu kod kraljevskoga dvora. 
\par 6 Druga trećina, ona kod Surskih  vrata, i treća trećina, ona kod stražnjih stražarskih vrata,  neka čuvaju stražu kod ulaza u dvor; 
\par 7 a ostala dva vaša odreda, svi koji subotom izlaze iz službe, neka čuvaju stražu u Domu  Jahvinu kod kralja. 
\par 8 Tako ćete okružiti kralja, svaki s oružjem  u ruci. I tko god pokuša proći kroz vaše redove, neka bude pogubljen.  Budite uz kralja kamo god pođe ili izađe." 
\par 9 Satnici su učinili sve kako im je naredio svećenik Jojada.  Svaki je od njih uzeo svoje ljude koji subotom ulaze u službu  s onima koji subotom izlaze. I svi su došli svećeniku Jojadi. 
\par 10 Svećenik dade satnicima koplja i štitove kralja Davida što  su bili u Domu Jahvinu. 
\par 11 Stražari se svrstaše, s oružjem u  ruci, od južne do sjeverne strane Doma i prema žrtveniku i Domu  oko kralja unaokolo. 
\par 12 Tada Jojada izvede sina kraljeva, stavi  mu krunu i dade mu Svjedočanstvo te ga pomaza za kralja. Pljeskali  su i vikali: "Živio kralj!" 
\par 13 Kad Atalija ču viku naroda, dođe k narodu u Dom Jahvin. 
\par 14 Pogleda bolje, kad gle, kralj, po običaju, stoji na svojem  mjestu, a pred kraljem zapovjednici i svirači; sav puk klikće  od radosti i trubi u trube. Tad Atalija razdrije haljine i povika:  "Izdaja! Izdaja!" 
\par 15 Svećenik Jojada naredi satnicima i vojnim  zapovjednicima: "Izvedite je kroz redove i tko krene za njom  pogubite ga mačem." Još je svećenik dodao: "Nemojte je smaknuti  u Domu Jahvinu." 
\par 16 Staviše ruke na nju; a kad je kroz Konjska  vrata stigla do kraljevskog dvora, ondje je pogubiše. 
\par 17 Tada Jojada sklopi savez između Jahve, kralja i naroda  da narod bude narod Jahvin. 
\par 18 Potom sav narod ode u Baalov  hram i razoriše ga, porušiše žrtvenike i polomiše likove; a Baalova  svećenika Matana ubiše pred žrtvenicima. A svećenik opet postavi  straže kod Doma Jahvina. 
\par 19 Zatim uze satnike Karijaca, stražu  i sav narod. Oni izvedoše kralja iz Doma Jahvina i uvedoše ga  u dvor kroz Vrata stražarska. I Joaš sjede na kraljevsko prijestolje. 
\par 20 Sav se puk veselio i grad se smirio kad su Ataliju ubili  mačem u kraljevskom dvoru. 
\par 21 (12:1) Joašu je bilo sedam godina kad se zakraljio. 


\chapter{12}

\par 1 (12:2) Sedme godine  Jehuova kraljevanja Joaš je postao kraljem i kraljevao je četrdeset  godina u Jeruzalemu. Majka mu se zvala Sibja i bila je iz Beer  Šebe. 
\par 2 (12:3) Joaš je činio što je pravo u očima Jahve svega svog  vijeka jer ga je poučavao svećenik Jojada. 
\par 3 (12:4) Ali uzvišica nisu  srušili i narod je svejednako prinosio žrtve i kad na uzvišicama. 
\par 4 (12:5) Joaš reče svećenicima: "Sav novac od posvećenih darova  što se donosi u Dom Jahvin, novac koji je nekomu nametnut procjenom  i novac što ga tko od svoje volje donose u Dom Jahvin 
\par 5 (12:6) neka  svećenici uzimaju svaki od svoga znanca i oni neka tim poprave  Dom gdje god se nađe koje oštećenje." 
\par 6 (12:7) Ali u dvadeset i trećoj  godini kraljevanja Joaševa svećenici nisu još popravili Doma. 
\par 7 (12:8) Tada kralj Joaš pozva svećenika Jojadu i druge svećenike i  reče im: "Zašto ne popravljate Dom? Odsad ne smijete više sebi  uzimati novac od svojih znanaca nego ga morate dati za popravak  Doma." 
\par 8 (12:9) Svećenici pristadoše da ne uzimaju novac od naroda, ali ni Doma da ne popravljaju. 
\par 9 (12:10) Tada svećenik Jojada uze kovčeg, proreza rupu na zaklopcu  i stavi ga uza žrtvenik, zdesna od ulaza u Dom Jahvin. Svećenici, čuvari praga, stavljali su u nj sav novac sabran u Domu Jahvinu. 
\par 10 (12:11) Kad bi se vidjelo da u kovčegu ima mnogo novaca, došao bi  kraljev tajnik s velikim svećenikom te bi prebrojili i zavezali  novac koji se nalazio u Domu Jahvinu. 
\par 11 (12:12) Prebrojeni novac uručivao  se upraviteljima poslova oko popravka Doma Jahvina, a oni su  isplaćivali drvodjeljama i graditeljima koji su radili u Domu  Jahvinu 
\par 12 (12:13) i zidarima i klesačima kamena, i za nabavu drveta  i tesanog kamena određena za popravak Doma Jahvina, ukratko:  za troškove oko popravka Doma. 
\par 13 (12:14) Ali u Domu Jahvinu nisu se  pravile srebrne čaše, ni noževi, ni plitice, ni trube, niti bilo  kakav predmet od zlata ili srebra za novac koji je darovan, 
\par 14 (12:15) nego  su ga davali radnicima koje su najmili za popravak Jahvina Doma. 
\par 15 (12:16) Nije se tražio obračun od ljudi kojima su predavali novac  da ga daju radnicima, jer su oni radili savjesno. 
\par 16 (12:17) Novac naknadnice  i okajnice nije se unosio u Dom Jahvin, nego je pripao svećenicima. 
\par 17 (12:18) Tada Hazael, aramejski kralj, pođe u rat protiv Gata  i osvoji ga. Zatim odluči poći protiv Jeruzalema. 
\par 18 (12:19) Joaš, judejski  kralj, uze sve posvećene darove koje su posvetili judejski kraljevi, njegovi oci: Jošafat, Joram i Ahazja, sve što je sam prikazao  i sve zlato koje se našlo u riznicima Doma Jahvina i kraljevskog  dvora. Sve to posla Hazaelu, aramejskom kralju, i tako se ovaj  udalji od Jeruzalema. 
\par 19 (12:20) Ostala povijest Joaševa i sve što je učinio, zar sve  to nije zapisano u knjizi Ljetopisa kraljeva judejskih? 
\par 20 (12:21) Njegovi  časnici ustadoše i skovaše zavjeru; ubiše Joaša u Bet Milu kad  je u nj silazio. 
\par 21 (12:22) Njegovi časnici Jozakar, sin Šimatov, i  Jozabad, sin Šomerov, zadaše mu smrtni udarac. Pokopali su ga  kraj njegovih otaca u Davidovu gradu, a njegov sin Amasja zakralji  se mjesto njega. 


\chapter{13}

\par 1 Dvadeset i treće godine kraljevanja judejskog kralja Joaša, sina Ahazjina, postade Joahaz, sin Jehuov, izraelskim kraljem  u Samariji. Kraljevao je sedamnaest godina. 
\par 2 On je činio što  je zlo u očima Jahvinim i poveo se za grijesima Jeroboama, sina  Nebatova, koji je zavodio Izraela. Od njih nije odstupao. 
\par 3 Tada Jahve uskipje gnjevom na Izraela i predade ga u ruke  aramejskog kralja Hazaela i u ruke Ben-Hadada, sina Hazaelova, za sve ono vrijeme. 
\par 4 Ali je Joahaz ublažio lice Jahvino i  Jahve ga je uslišio, jer je vidio nevolju koju je aramejski kralj  nanosio Izraelu. 
\par 5 Jahve je dao Izraelu izbavitelja koji ga  je izbavio od ruke aramejske te su Izraelci živjeli u svojim  šatorima kao i prije. 
\par 6 Ali nisu odstupali od grijeha kojim  Jeroboam bijaše zaveo Izraela: ustrajali su u njemu, pa i ašere  ostadoše u Samariji. 
\par 7 Jahve je ostavio Joahazu samo pedeset  konjanika kao vojsku, deset bojnih kola i deset tisuća pješaka;  kralj aramejski bijaše ih uništio i zgazio ih kao prah u vršidbi. 
\par 8 Ostala povijest Joahazova, sve što je učinio i poduzimao, zar sve to nija zapisano u knjizi Ljetopisa kraljeva izraelskih? 
\par 9 Joahaz je počinuo sa svojim ocima i bi pokopan u Samariji, a njegov sin Joaš zakralji se mjesto njega. 
\par 10 Trideset i sedme godine kraljevanja judejskoga kralja  Joaša postade Joaš, sin Joahazov, izraelskim kraljem u Samariji;  kraljevao je šesnaest godina. 
\par 11 Činio je što je zlo u očima  Jahvinim. Nije odstupao od grijeha Jeroboama, sina Nebatova,  koji je zaveo Izraela. Za njim se poveo. 
\par 12 Ostala povijest Joaševa, sve što je učinio, junaštva  njegova, kako je ratovao s Amasjom, judejskim kraljem, zar sve  to nije zapisano u knjizi Ljetopisa kraljeva izraelskih? 
\par 13 Joaš  je počinuo sa svojim ocima, a Jeroboam se popeo na njegovo prijestolje.  Joaša pokopaše u Samariji uz izraelske kraljeve. 
\par 14 Kad se Elizej razbolio od bolesti od koje mu valjade  umrijeti, dođe mu izraelski kralj Joaš, rasplaka se nad njim  i reče mu: "Oče moj, oče moj! Kola Izraelova i konjanici njegovi!" 
\par 15 Elizej mu reče: "Uzmi luk i strijele." I on dohvati luk i  strijele. 
\par 16 Elizej će tada kralju: "Nategni luk!" I on ga nateže.  Elizej stavi ruke na ruke kraljeve, 
\par 17 zatim reče: "Otvori prozor  prema istoku." I on ga otvori, a nato će Elizej: "Odapni!" I  on odape, a Elizej reče: "Pobjedonosna strijela Jahvina! Pobjednička  strijela nad Aramejcima! Do nogu ćeš potući Aramejce kod Afeka." 
\par 18 I nastavi: "Uzmi strijele!" On ih uze. Elizej tada reče kralju:  "Udri o zemlju!" On udari tri puta i stade. 
\par 19 Tada se rasrdi  na njega Božji čovjek i reče: "Pet ili šest puta trebalo je da  udariš! Tada bi potpuno potukao Aramejce; ovako ćeš ih pobijediti  samo tri puta." 
\par 20 Elizej zatim umrije i pokopaše ga. A pljačkaške čete  Moabaca napadale zemlju svake godine. 
\par 21 Dogodilo se te su neki, sahranjujući čovjeka, opazili razbojnike: baciše mtrvaca u grob  Elizejev i odoše. Mrtvac, dotakavši se Elizejevih kostiju, oživje  i stade na noge. 
\par 22 Aramejski kralj Hazael ugnjetavaše Izraelce svega vijeka  Joahazova. 
\par 23 Ali im se Jahve smilova i ražali se nad njima.  Pogleda na njih zbog svoga Saveza koji je sklopio s Abrahamom, Izakom i Jakovom. Nije ih htio uništiti i nije ih odbacio daleko  od svoga lica do danas. 
\par 24 Hazael, aramejski kralj, umrije,  a njegov sin Ben-Hadad zavlada namjesto njega. 
\par 25 Tada Joaš, sin Joahazov, opet uze iz ruke Ben-Hadada, sina Hazaelova, gradove  koje Hazael u ratu bijaše oteo njegovu ocu Joahazu. Joaš ga je  tri puta potukao i vratio gradove Izraelove. 


\chapter{14}

\par 1 Druge godine kraljevanja Joaša, sina Joahazova, nad Izraelom, postade judejskim kraljem Amasja, sin Joašev. 
\par 2 Bilo mu je  dvadeset i pet godina kad se zakraljio, a kraljevao je dvadeset  i devet godina u Jeruzalemu. Mati mu se zvala Joadana i bila  je iz Jeruzalema. 
\par 3 Činio je što je pravo u Jahvinim očima,  ali ne sasvim kao praotac njegov David. U svemu je slijedio Joaša, svoga oca. 
\par 4 Ali uzvišica nije razrušio i narod je svejednako  prinosio žrtve i kad na uzvišicama. 
\par 5 Kad je učvrstio kraljevstvo, smakao je one časnike koji  su mu ubili oca. 
\par 6 Ali nije pogubio sinova onih ubojica, prema  onome što je napisano u knjizi Zakona Mojsijeva, gdje Jahve zapovijeda:  "Neka se očevi ne pogubljuju za sinove ni sinovi za očeve, nego  svatko neka gine za svoj grijeh." 
\par 7 On je potukao Edomce u Slanoj  dolini, deset tisuća njih, i u bitki je zauzeo Selu; dao joj  je ime Jokteel, koje nosi do današnjega dana. 
\par 8 Tada Amasja posla glasnike izraelskom kralju Joašu, sinu  Jehuova sina Joahaza, i poruči mu: "Dođi da se ogledamo!" 
\par 9 A  izraelski kralj Joaš odvrati judejskom kralju Amasji: "Libanonski  je trn jedanput poslao glasnike k libanonskom cedru: 'Daj kćer  mome sinu za ženu', ali su divlje zvijeri libanonske prošle i  trn izgazile. 
\par 10 Potukao si Edomce, pa ti se srce uzobijestilo  i tražiš slavu! Radije ostani kod kuće. Zašto izazivaš zlo i  hoćeš da propadneš ti i svi Judejci s tobom?" 
\par 11 Ali Amasja ne posluša. Izađe izraelski kralj Joaš te  se ogledaše u boju on i judejski kralj Amasja u Bet Šemešu u  Judeji. 
\par 12 Izraelci poraziše Judejce i oni pobjegoše svaki pod  svoj šator. 
\par 13 Izraelski kralj Joaš uhvati u Bet Šemešu judejskoga kralja  Amasju, sina Joaševa, sina Ahazjina, i odvede ga u Jeruzalem.  Tada sruši jeruzalemski zid od Efrajimovih vrata do Ugaonih vrata, u dužini od četiri stotine lakata. 
\par 14 Uzevši sve zlato, srebro  i posuđe što se nalazilo u Domu Jahvinu i u riznici kraljevskog  dvora, povrh toga i taoce, vrati se u Samariju. 
\par 15 Ostala povijest Joaševa, sve što je činio i poduzimao  i kako je ratovao s Amasjom, judejskim kraljem, zar sve to nije  zapisano u knjizi Ljetopisa izraelskih kraljeva? 
\par 16 Joaš je  počinuo sa svojim ocima i pokopan je u Samariji uz kraljeve izraelske.  Sin njegov Jeroboam zakralji se mjesto njega. 
\par 17 Amasja, sin Joašev, judejski kralj, živio je još petnaest  godina poslije smrti izraelskog kralja Joaša, sina Joahazova. 
\par 18 A ostala povijest Amasjina zar nije zapisana u knjizi Ljetopisa  kraljeva judejskih? 
\par 19 Protiv njega je skovana urota u Jeruzalemu.  Iako je on pobjegao u Lakiš, poslaše za njim u Lakiš ljude koji  ga ondje ubiše. 
\par 20 Odande su ga prenijeli na konjima i sahranili  u Jeruzalemu kraj njegovih otaca, u Davidovu gradu. 
\par 21 Tada  sav judejski narod uze Azahju, komu bijaše šesnaest godina, i  zakralji ga namjesto njegova oca Amasje. 
\par 22 On opet sagradi Elat povrativši ga Judeji, pošto je kralj  počinuo kod svojih otaca. 
\par 23 Petnaeste godine kraljevanja judejskog kralja Amasje, sina Joaševa, postade izraelskim kraljem u Samariji Jeroboam, sin Joašev. On je kraljevao četrdeset i jednu godinu. 
\par 24 Činio  je što je zlo u očima Jahvinim, nije se ostavio nijednoga grijeha  Jeroboama, sina Nebatova, koji je zaveo Izraela. 
\par 25 On je dobio natrag izraelsko područje od Ulaza u Hamat  do Mrtvoga mora, prema riječi koju je Jahve, Bog Izraelov, rekao  preko sluge svoga Jone, sina Amitajeva, proroka iz Gat Hahefera. 
\par 26 Jer je Jahve vidio ljutu nevolju Izraelovu da više nema ni  slobodnih ni robova i nikoga da pomogne Izraelu. 
\par 27 Ali Jahve  nije odlučio izbrisati ispod neba ime Izraelovo: spasio ga je  rukom Jeroboama, sina Joaševa. 
\par 28 Ostala povijest Jeroboama, sve što je učinio i sve što  je poduzimao, kako je ratovao i kako je vratio Damask Judi i  Izraelu, zar sve to nije zapisano u knjizi Ljetopisa kraljeva  izraelskih? 
\par 29 Jeroboam je počinuo sa svojim ocima. Pokopali  su ga u Samariji uz kraljeve izraelske, a njegov sin Zaharija  zakralji se mjesto njega. 


\chapter{15}

\par 1 Dvadeset i sedme godine kraljevanja Jeroboama, kralja izraelskog, postade judejskim kraljem Azarja, sin Amasjin. 
\par 2 Bilo mu je  šesnaest godina kad se zakraljio, a kraljevao je pedeset i dvije  godine u Jeruzalemu. Mati mu se zvala Jekolija, a bila je iz  Jeruzalema. 
\par 3 Činio je što je pravo u Jahvinim očima, sasvim  kao i njegov otac Amasja. 
\par 4 Samo uzvišica nije srušio i narod  je svejednako prinosio žrtve i kad na uzvišicama. 
\par 5 Ali Jahve udari kralja i ostade on gubav do smrti. Stanovao  je u odvojenoj kući. Kraljev sin Jotam bio upravitelj dvora i  sudio je puku zemlje. 
\par 6 Ostala povijest Azarjina i sve što je učinio, zar sve  to nije zapisano u knjizi Ljetopisa kraljeva judejskih? 
\par 7 Azarja  je počinuo i sahraniše ga kraj njegovih otaca u Davidovu gradu.  A na njegovo se mjesto zakralji sin mu Jotam. 
\par 8 Trideset i osme godine Azarjina kraljevanja u Judeji postade  izraelskim kraljem u Samariji za šest mjeseci Zaharija, sin Jeroboamov. 
\par 9 On je činio što je zlo u očima Jahvinim, kao što su činili  njegovi oci; nije odstupao od grijeha Jeroboama, sina Nebatova, koji je na grijeh naveo Izraela. 
\par 10 Šalum, sin Jabešov, uroti se protiv njega; udario ga  je i usmrtio u Jibleamu te se zakraljio mjesto njega. 
\par 11 Ostala povijest Zaharijina zapisana je u knjizi Ljetopisa  izraelskih kraljeva. 
\par 12 Ispunila se riječ koju je Jahve rekao  Jehuu: "Tvoji će sinovi sjediti na prijestolju Izraela sve do  četvrtog koljena." I tako je bilo. 
\par 13 Šalum, sin Jabešov, postade kraljem trideset i devete  godine kraljevanja Uzije, judejskog kralja, i kraljevao je mjesec  dana u Samariji. 
\par 14 Menahem, sin Gadijev, ode iz Tirse, uđe  u Samariju te udari Šaluma, sina Jabešova, usmrti ga i zakralji  se mjesto njega. 
\par 15 Ostala povijest Šalumova i urota koju je skovao, sve  je zapisano u knjizi Ljetopisa kraljeva izraelskih. 
\par 16 Tada  je Menahem razorio Tifnah i sve što je u njem bilo i njegovo  područje od Tirse jer mu nisu otvorili vrata. Razorio ga je i  rasporio sve trudnice u njemu. 
\par 17 Trideset i devete godine kraljevanja Azarje u Judeji  postade Menahem, sin Gadijev, kraljem Izraela. Kraljevao je deset  godina u Samariji. 
\par 18 Činio je što je zlo u očima Jahvinim;  nije odstupao od grijeha Jeroboama, sina Nebatova, koji je zaveo  Izraela.  U njegovo vrijeme 
\par 19 Pul, kralj Asirije, osvoji zemlju.  Menahem dade Pulu tisuću talenata srebra da mu pomogne učvrstiti  kraljevsku vlast u njegovim rukama. 
\par 20 Menahem ubra taj novac  od Izraela, od svih imućnih ljudi, da bi ga mogao dati asirskom  kralju. Po osobi je bilo pedeset šekela srebra. Tako se asirski  kralj vratio i nije ondje ostao u zemlji. 
\par 21 Ostala povijest Menahema i sve što je učinio, zar sve  to nije zapisano u knjizi Ljetopisa kraljeva izraelskih? 
\par 22 Menahem  je počinuo sa svojim ocima, a sin njegov Pekahja zakralji se  na njegovo mjesto. 
\par 23 Pedesete godine kraljevanja judejskog kralja Azarje postade  kraljem izraelskim u Samariji Pekahja, sin Menahemov. Kraljevao  je dvije godine. 
\par 24 On je činio što je zlo u očima Jahvinim;  nije odstupao od grijeha Jeroboama, sina Nebatova, koji je zaveo  Izraela. 
\par 25 Njegov dvoranin Pekah, sin Remalijin, uroti se protiv  njega i ubi ga u Samariji, u kuli kraljevskog dvora, s Argobom  i Arjeom. Imao je sa sobom pedeset ljudi iz Gileada. Ubio je  kralja i zakraljio se mjesto njega. 
\par 26 Ostala povijest Pekahje i sve što je učinio, sve je to  zapisano u knjizi Ljetopisa izraelskih kraljeva. 
\par 27 Pedeset i druge godine kraljevanja Azarje, judejskoga  kralja, postade kraljem u Samariji Pekah, sin Remalijin. Kraljevao  je dvadeset godina. 
\par 28 On je činio što je zlo u očima Jahvinim;  nije odstupao od grijeha Jeroboama, sina Nebatova, koji je zaveo  Izraela. 
\par 29 U vrijeme izraelskog kralja Pekaha došao je asirski kralj  Tiglat Pileser i zauzeo Ijon, Abel Bet Maaku, Janoah, Kedeš,  Hasor, Gilead, Galileju i svu zemlju Naftalijevu. I odveo je  stanovništvo u Asiriju. 
\par 30 Hošea, sin Elin, uroti se protiv  Pekaha, sina Remalijina, ubi ga i zakralji mjesto njega dvadesete  godine Jotama, sina Uzijina. 
\par 31 Ostala povijest Pekahova, sve što je učinio, sve je to  zapisano u knjizi Ljetopisa izraelskih kraljeva. 
\par 32 Druge godine kraljevanja Pekaha, sina Remalijina, nad  Izraelom, postade judejskim kraljem Jotam, sin Uzijin. 
\par 33 Bilo  mu je dvadeset i pet godina kad se zakraljio, a kraljevao je  šesnaest godina u Jeruzalemu. Materi mu bješe ime Jeruša, Sadokova  kći. 
\par 34 Činio je što je pravo u Jahvinim očima, sasvim kao i  otac mu Uzija. 
\par 35 Ali ni on nije srušio uzvišica; narod je svejednako  prinosio žrtve i kad na uzvišicama. On je sagradio Gornja vrata  na Domu Jahvinu. 
\par 36 Ostala povijest Jotama i sve što je učinio, zar to nije  sve zapisano u knjizi Ljetopisa judejskih kraljeva? 
\par 37 U njegove  je dane Jahve počeo slati protiv Judeje aramejskog kralja Resina  i Pekaha, sina Remalijina. 
\par 38 Tada Jotam počinu kod otaca i  sahraniše ga u gradu njegova praoca Davida. A na njegovo se mjesto  zakralji sin mu Ahaz. 


\chapter{16}

\par 1 Sedamnaeste godine vladanja Pekaha, sina Remalijina, postade  judejskim kraljem Ahaz, sin Jotamov. 
\par 2 Ahazu je bilo dvadeset  godina kad se zakraljio, a kraljevao je šesnaest godina u Jeruzalemu, ali nije činio što je pravo u očima Jahve, Boga njegova, kao  što je činio predak mu David. 
\par 3 Živio je poput izraelskih kraljeva  i sam je proveo svoga sina kroz oganj po gnusnom običaju naroda  što ih je Jahve protjerao pred Izraelovim sinovima. 
\par 4 Prinosio  je žrtve i kad po uzvišicama i brežuljcima i pod svakim zelenim  drvetom. 
\par 5 Tada aramejski kralj Resin i Pekah, sin Remalijin, kralj  Izraela, pođoše u rat protiv Jeruzalema. Opsjedoše ga, ali ga  ne mogoše osvojiti. 
\par 6 U to vrijeme aramejski kralj Resin vrati  Elat Edomcima; protjerao je Judejce iz Elata; ušli su Edomci  u njega i ondje su ostali do danas. - 
\par 7 Tada Ahaz uputi poslanike  asirskom kralju Tiglat-Pileseru da mu kažu: "Ja sam tvoj sluga  i sin tvoj! Dođi i izbavi me iz ruku aramejskog kralja i kralja  Izraela, koji su se digli protiv mene." 
\par 8 Ahaz je uzeo srebro  i zlato što se nalazilo u Domu Jahvinu i u riznicama kraljevskog  dvora i sve je poslao na dar asirskom kralju. 
\par 9 I posluša ga  asirski kralj: otišao je na Damask i osvojio ga. Stanovništvo  je odveo u sužanjstvo u Kir, a Resina je pogubio. 
\par 10 Kralj Ahaz otišao je u Damask u susret asirskom kralju  Tiglat-Pileseru. I vidio je žrtvenik koji bijaše u Damasku. Tada  kralj Ahaz posla svećeniku Uriji mjere žrtvenika, njegov nacrt  i sve pojedinosti njegove građe. 
\par 11 Svećenik Urija sagradi žrtvenik;  sve je upute što ih je kralj Ahaz uputio iz Damaska izvršio svećenik  Urija prije nego što se kralj Ahaz vratio iz Damaska. 
\par 12 Pošto  je kralj Ahaz stigao iz Damaska, vidio je žrtvenik, prišao mu  i popeo se na nj. 
\par 13 Spalio je na žrtveniku svoju paljenicu  i svoju prinosnicu, izlio svoju ljevanicu i krvlju pričesnica  poškropio žrtvenik. 
\par 14 A mjedeni žrtvenik, koji bijaše pred  Jahvom, maknuo je ispred Hrama, gdje je bio između novoga žrtvenika  i Doma Jahvina. I postavio ga je pokraj novoga žrtvenika sa sjevera. 
\par 15 Kralj Ahaz zapovjedio je svećeniku Uriji: "Na velikom ćeš  žrtveniku spaljivati jutarnju paljenicu i večernju prinosnicu, kraljevu paljenicu i njegovu prinosnicu, i paljenice, prinosnice  i ljevanice svega naroda. Po njemu ćeš izlijevati svu krv paljenica  i klanica. A o žrtveniku od mjedi još ću razmisliti." 
\par 16 Svećenik  Urija učini sve što mu je naredio kralj Ahaz. 
\par 17 Kralj Ahaz skinuo je okvire s podnožja; s njih je skinuo  i umivaonike. A mjedeno more skinuo je s volova koji su stajali  pod njim i stavio ga na kameni pod. 
\par 18 Pred asirskim je kraljem  uklonio iz Jahvina Doma Subotnji hodnik koji bijahu sagradili  i vanjski kraljevski prilaz. 
\par 19 Ostala povijest Ahazova i sve što je učinio, zar to nije  sve zapisano u knjizi Ljetopisa judejskih kraljeva? 
\par 20 Ahaz  je počinuo sa svojim ocima i sahranjen je u Davidovu gradu. Na  njegovo se mjesto zakraljio sin mu Ezekija. 


\chapter{17}

\par 1 Dvanaeste godine kraljevanja Ahaza u Judeji, postao je Hošea, sin Elin, izraelskim kraljem u Samariji. Kraljevao je devet  godina. 
\par 2 On je činio što je zlo u očima Jahvinim, ali ne kao  izraelski kraljevi, njegovi prethodnici. 
\par 3 Asirski kralj Salmanasar  pošao je protiv Hošee, koji mu se pokorio i plaćao mu danak. 
\par 4 Ali je asirski kralj otkrio da mu Hošea sprema zavjeru: još  je Hošea poslao poslanike egipatskom kralju Sou i nije platio  danaka asirskom kralju kao svake godine. Tada ga asirski kralj  baci u tamnicu. 
\par 5 Asirski kralj osvoji svu zemlju i krenu opsjedati Samariju.  Opsjedao ju je tri godine. 
\par 6 Devete godine Hošeine vladavine  zauze asirski kralj Samariju i odvede Izraelce u sužanjstvo u  Asiriju. Naselio ih je u Helahu, i na Haboru, rijeci u Gozanu, i u gradovima medijskim. 
\par 7 I tako se dogodilo zato što su Izraelci sagriješili protiv  Jahve, Boga svoga, koji ih je izveo iz zemlje egipatske, ispod  vlasti faraona, kralja egipatskog. Štovali su druge bogove, 
\par 8 slijedili  običaje naroda što ih je Jahve protjerao pred sinovima Izraelovim, živjeli po običajima što su ih uveli kraljevi Izraelovi. 
\par 9 Izraelci  i njihovi kraljevi potajno su činili neprikladna djela protiv  Jahve, Boga svoga. Podigli su uzvišice u svim svojim naseljima:  od stražarskih kula pa do utvrđenih gradova. 
\par 10 Podizali su  stupove i ašere na svakom humku i pod svakim zelenim drvetom. 
\par 11 Ondje su, na svim uzvišicama, palili kad po običaju naroda  što ih je Jahve protjerao ispred njih i činili su zla djela te  izazivali gnjev Jahvin. 
\par 12 Služili su idolima, premda im Jahve  bijaše rekao: "Ne činite toga!" 
\par 13 A Jahve opominjaše Izraelce i Judejce preko svih svojih  proroka i sviju vidjelaca: "Obratite se od zlog puta svoga",  govorio je, "i pokoravajte se naredbama i zapovijedima mojim  prema Zakonu koji sam naložio ocima vašim i prema svemu što sam  vam objavio preko slugu svojih - proroka." 
\par 14 Ali oni nisu poslušali  nego su ostali tvrdovrati kao i njihovi oci, koji nisu vjerovali  u Jahvu, Boga svoga. 
\par 15 Prezreli su njegove zakone i Savez koji  je sklopio s njihovim ocima i opomene njegove koje im je upućivao.  Težili su za ispraznošću, pa su i sami postali isprazni slijedeći  narode oko sebe, premda im je Jahve zapovjedio da ne čine kao  oni. 
\par 16 Odbacili su sve zapovijedi Jahve, Boga svoga, i načinili  su sebi salivene idole, dva teleta. Podigli su ašere, klanjali  se svoj vojsci nebeskoj i služili Baalu. 
\par 17 Provodili su svoje  sinove i kćeri kroz oganj, odavali se vračanju i gatanju, čineći  tako zlo u očima Jahvinim i razjarujući ga. 
\par 18 Tada se Jahve  razgnjevi na Izraela i odbaci ga ispred svoga lica. Ostalo je  samo pleme Judino. 
\par 19 Ali ni pleme Judino nije držalo zapovijedi Jahve, Boga  svoga, i slijedilo je običaje kojih su se držali Izraelci. 
\par 20 I  Jahve odbaci sav rod Izraela, ponizi ga i predade ga pljačkašima, dok ih konačno ne odbaci daleko od svoga lica. 
\par 21 On je, konačno, otrgnuo Izraelce od kuće Davidove, a Izrael je proglasio kraljem  Jeroboama, sina Nebatova. Jeroboam je odvratio Izraela od Jahve  i naveo ih na veliku grehotu. 
\par 22 Izraelci su slijedili svaki  grijeh koji je Jeroboam počinio i od njega se nisu odvraćali, 
\par 23 dok konačno Jahve nije odbacio Izraela ispred svoga lica, kako to bijaše objavio po svojim slugama, prorocima. Odveo je  Izraelce iz njihove zemlje u sužanjstvo u Asiriju, gdje su do  današnjega dana. 
\par 24 Asirski je kralj doveo ljude iz Babilona, iz Kute, iz  Ave, Hamata i iz Sefarvajima, i naselio ih u gradovima Samarije  mjesto Izraelaca. Oni su zaposjeli Samariju i nastanili se u  gradovima njezinim. 
\par 25 U vrijeme naseljavanja u zemlju nisu štovali Jahve i  on je poslao protiv njih lavove da ih rastrgaju. 
\par 26 Zato su  rekli asirskom kralju: "Narodi koje si preselio da ih nastaniš  u gradovima Samarije ne znaju kako valja štovati Boga ove zemlje  i on je na njih poslao lavove, koji ih usmrćuju, jer ti narodi  ne poznaju bogoštovlja ove zemlje." 
\par 27 Tada je asirski kralj  izdao ovu zapovijed: "Neka ide onamo jedan od svećenika koje  sam odande doveo u sužanjstvo; neka ide, neka se ondje nastani  i pouči ih u štovanju Boga one zemlje." 
\par 28 Tako ode jedan od  svećenika koji su bili odvedeni iz Samarije i nastani se u Betelu.  On ih je poučio kako treba štovati Jahvu. 
\par 29 Svaki je narod imao likove svojih bogova i postavili  su ih u hramove na uzvišicama koje su podigli Samarijanci, svaki  narod u svojim gradovima u kojima življaše. 
\par 30 Babilonci načiniše  Sukot Benota, Kušani Nergala, Hamaćani Ašimu; 
\par 31 Avijci načiniše  Nibhaza i Tartaka, a Sefarvajimci spaljivahu svoju djecu na ognju  u čast Adrameleka i Anameleka, sefarvajimskih bogova. 
\par 32 Oni  su štovali i Jahvu i postavili su neke između sebe za svećenike  uzvišica koji su im prinosili žrtve u hramovima uzvišica. 
\par 33 Štovali  su Jahvu i služili su svojim bogovima po običaju onih naroda  između kojih su ih preselili. 
\par 34 Oni se još i danas drže starih običaja. Ne štuju Jahve  i ne usklađuju svojih pravila i običaja sa Zakonom i zapovijedima  što ih je Jahve naredio djeci Jakova komu je nadjenuo ime Izrael. 
\par 35 Jahve bijaše s njima sklopio Savez i zapovjedio im: "Ne štujte  tuđih bogova niti im se klanjajte. Nemojte ih štovati niti im  žrtava prinositi. 
\par 36 Samo je Jahve onaj koji vas je velikom  snagom svoje ispružene ruke izveo iz zemlje egipatske; njega  štujte, njemu se klanjajte i njemu žrtve prinosite. 
\par 37 Držite  se pravila i običaja, zakona i naredaba koje vam je propisao  da ih vjerno ispunjavate uvijek i ne štujte tuđih bogova. 
\par 38 Nemojte  zaboraviti Saveza koji sam sklopio s vama i nemojte štovati drugih  bogova, 
\par 39 samo Jahvu, Boga svoga, poštujte i on će vas izbaviti  iz ruke svih vaših neprijatelja." 
\par 40 Ali oni nisu poslušali, nego su se i dalje držali svoga starog običaja. 
\par 41 Tako su ti narodi štovali Jahvu, a služili su i svojim  idolima. Njihovi sinovi i sinovi njihovih sinova čine do dana  današnjega onako kako su činili njihovi oci. 


\chapter{18}

\par 1 Treće godine kraljevanja Hošee, sina Elina, u Izraelu, postao  je judejskim kraljem Ezekija, sin Ahazov. 
\par 2 Bilo mu je dvadeset  i pet godina kad se zakraljio. Kraljevao je dvadeset i devet  godina u Jeruzalemu. Materi mu je bilo ime Abija, Zaharijina  kći. 
\par 3 Činio je što je pravo u očima Jahvinim, sasvim kao njegov  otac David. 
\par 4 On je uklonio uzvišice, srušio je stupove, sasjekao  je ašere i razbio zmiju od mjedi koju bijaše načinio Mojsije.  Izraelci su joj sve do tada prinosili žrtve. Zvali su je Nehuštan. 
\par 5 Pouzdavao se u Jahvu, Boga Izraelova. Ni prije njega ni  poslije njega ne bijaše mu ravna među kraljevima judejskim. 
\par 6 Prionuo  je uz Jahvu i nikada se nije okrenuo od njega. Držao je sve zapovijedi  što ih je Jahve dao preko Mojsija. 
\par 7 I Jahve bijaše s njim,  pomagaše ga u svim njegovim pothvatima. Pobunio se protiv asirskog  kralja i nije mu više bio podložan. 
\par 8 On je potukao Filistejce  do Gaze, opustošio njihovo područje od stražarskih kula sve do  utvrđenih gradova. 
\par 9 Četvrte godine vladavine Ezekijine, a to je bila sedma  godina kraljevanja izraelskog kralja Hošee, sina Elina, napade  asirski kralj Salmanasar Samariju i opsjede je. 
\par 10 Osvojio ju  je nakon tri godine. Šeste godine Ezekijine vladavine, a devete  godine izraelskog kralja Hošee, pala je Samarija. 
\par 11 Asirski  je kralj odveo Izraelce u sužanjstvo u Asiriju i naselio ih u  Halahu, na Haboru, rijeci gozanskoj, i u medijskim gradovima. 
\par 12 Bijaše to stoga što nisu poslušali glas Jahve, Boga svoga, i što su prekršili njegov Savez i sve što im je naredio Mojsije, sluga Jahvin. Nisu ništa slušali niti vršili. 
\par 13 Četrnaeste godine Ezekijina kraljevanja asirski kralj  Sanherib napade utvrđene judejske gradove i osvoji ih. 
\par 14 Tada  judejski kralj Ezekija poruči asirskom kralju u Lakiš: "Pogriješio  sam! Obustavi svoje napade na me. Snosit ću sve što mi nametneš."  Asirski kralj zatraži od Ezekije, judejskog kralja, tri stotine  talenata srebra i trideset talenata zlata. 
\par 15 I dade Ezekija  sve srebro što se našlo u Domu Jahvinu i u riznicama kraljevskog  dvora. 
\par 16 U to je vrijeme Ezekija obio vrata i vratnice na Svetištu  Jahvinu što ih bijaše pozlatio on sam, judejski kralj Ezekija, i posla to asirskom kralju. 
\par 17 Asirski je kralj poslao iz Lakiša u Jeruzalem kralju  Ezekiji vrhovnog zapovjednika vojske, velikog dvoranina i peharnika  s jakom vojskom. Krenuše oni, a kad su stigli u Jeruzalem, stadoše  kod vodovoda Gornjeg ribnjaka, na putu u Valjarevo polje. 
\par 18 Oni  pozvaše kralja. Pred njim je izašao upravitelj dvora Elijakim, sin Hilkijin, pisar Šebna i savjetnik Joah, sin Asafov. 
\par 19 Veliki peharnik reče im: "Kažite Ezekiji: 'Ovako veli  veliki kralj, kralj asirski: kakvo je to pouzdanje u koje se  uzdaš? 
\par 20 Misliš li da su prazne riječi već savjet i snaga za  rat? U koga se uzdaš da si se pobunio protiv mene? 
\par 21 Eto, oslanjaš  se na Egipat, na slomljenu trsku koja probada i prodire dlan  onomu tko se na nju nasloni. Takav je faraon, kralj egipatski, svima koji se uzdaju u njega.' 
\par 22 Možda ćete mi odgovoriti:  'Uzdamo se u Jahvu, Boga svojega.' Ali nije li njemu Ezekija  uklonio uzvišice i žrtvenike i zapovjedio Judejcima i Jeruzalemu:  'Samo se pred ovim žrtvenikom u Jeruzalemu klanjajte.' 
\par 23 Hajde, okladi se s mojim gospodarom, asirskim kraljem: dat ću ti dvije  tisuće konja ako mogneš naći jahače za njih! 
\par 24 Kako ćeš onda  odoljeti jednome jedinom od najmanjih slugu moga gospodara? Ali  se ti uzdaš u Egipat da će ti dati kola i konjanika. 
\par 25 Naposljetku, zar sam ja mimo volju Jahvinu krenuo protiv ovoga mjesta da  ga razorim? Sam mi je Jahve rekao: 'Idi na tu zemlju i razori  je!'" 
\par 26 Elijakim, Šebna i Joah rekoše velikom peharniku: "Molimo  te, govori svojim slugama aramejski, jer mi razumijemo; ne govori  s nama judejski da čuje narod koji je na zidinama!" 
\par 27 Ali im  veliki peharnik odgovori: "Zar me moj gospodar poslao da ovo  kažem tvome gospodaru i tebi, a ne upravo onim ljudima koji sjede  na zidinama, osuđeni da s vama jedu svoju nečist i piju svoju  mokraću?" 
\par 28 Tada se veliki peharnik uspravi i u sav glas povika na  judejskom ove riječi: "Čujte riječ velikoga kralja, kralja asirskog! 
\par 29 Ovako veli kralj: 'Neka vas Ezekija ne zavarava, jer vas  ne može izbaviti iz moje ruke. 
\par 30 Neka vas Ezekija ne hrabri  pouzdanjem u Jahvu govoreći: Jahve će nas sigurno izbaviti, ovaj  grad neće pasti u ruke kralju asirskom.' 
\par 31 Ne slušajte Ezekije, jer ovako veli asirski kralj: 'Sklopite mir sa mnom, predajte  mi se, pa neka svaki od vas jede plodove iz svoga vinograda i  sa svoje smokve i neka pije vode iz svojega studenca 
\par 32 dok  ne dođem i ne odvedem vas u zemlju kao što je vaša, u zemlju  pšenice i mošta, u zemlju kruha i vinograda, u zemlju ulja i  meda da biste živjeli i da ne pomrete. Ne dajte da vas Ezekija  zaludi govoreći vam: Jahve će vas izbaviti.' 
\par 33 Jesu li bogovi  drugih naroda izbavili svoje zemlje iz ruku asirskoga kralja? 
\par 34 Gdje su bogovi hamatski i arpadski, gdje su bogovi sefarvajimski, henski i ivski, gdje su bogovi samarijski da izbave Samariju  iz ruke moje? 
\par 35 Koji su među svim bogovima tih zemalja izbavili  svoju zemlju iz moje ruke da bi Jahve izbavio Jeruzalem iz ruke  moje?" 
\par 36 Šutjeli su i ni riječi mu nisu odgovorili, jer kralj  bijaše zapovjedio: "Ne odgovarajte mu!" 
\par 37 Upravitelj dvora  Elijakim, sin Hilkijin, pisar Šebna i savjetnik Joah, sin Asafov, dođoše k Ezekiji, razdrijevši haljine, i saopćiše mu riječi  velikoga peharnika 


\chapter{19}

\par 1 Čuvši to, kralj Ezekija razdrije svoje haljine, obuče kostrijet  i ode u Dom Jahvin. 
\par 2 Zatim posla Elijakima, upravitelja dvora, pisara Šebnu i svećeničke starješine, odjevene u kostrijet,  k proroku Izaiji, sinu Amosovu. 
\par 3 Oni mu rekoše: "Ovako veli  Ezekija: 'Ovo je dan nevolje, kazne i rugla. Prispješe djeca  do rođenja, a nema snage da se rode. 
\par 4 Možda je Jahve, Bog tvoj, čuo što je rekao veliki peharnik koga je asirski kralj, gospodar  njegov, poslao da se izruguje Bogu živome i možda će Jahve, Bog  tvoj, kazniti riječi koje je čuo! Pomoli se pobožno za Ostatak  koji je još preostao.'" 
\par 5 Kad su sluge kralja Ezekije stigle k Izaiji, 
\par 6 on im  reče: "Kažite svome gospodaru: 'Ovako veli Jahve: Ne boj se riječi  koje si čuo kada su na me hulile sluge kralja asirskoga. 
\par 7 Udahnut  ću u njega duh i kad čuje jednu vijest, vratit će se u svoju  zemlju. I učinit ću da u svojoj zemlji pogine od mača.'" 
\par 8 Veliki peharnik vrati se i nađe asirskoga kralja gdje  opsjeda Libnu, jer bijaše čuo da je kralj otišao iz Lakiša. 
\par 9 Dočuo  je, naime, vijest o Tirhaku, kralju etiopskome: "Evo, izašao  je da se bori protiv tebe." Tada Sanherib ponovo uputi poslanike da kažu Ezekiji: 
\par 10 "Ovako  recite judejskom kralju Ezekiji: 'Neka te ne vara tvoj Bog, u  koga se uzdaš, govoreći ti: Jeruzalem neće pasti u ruke asirskog  kralja! 
\par 11 Ti znaš što su asirski kraljevi učinili svim zemljama  izručivši ih prokletstvu! A ti, ti li ćeš se spasiti? 
\par 12 Jesu  li bogovi spasili narode što su ih uništili moji oci: Gozance, Harane, Resefce i Edence, u Tel Basaru? 
\par 13 Gdje je kralj hamatski, kralj arpadski, kralj Sefarvajima, Hene i Ive?'" 
\par 14 Ezekija primi pismo iz ruke poslanikove i pročita ga.  Zatim uđe u Dom Jahvin i razvi ga ondje pred Jahvom. 
\par 15 I pomoli  se Ezekija Jahvi ovako: "Jahve, Bože Izraelov, koji stoluješ  nad kerubima, ti si Bog jedini nad svim zemaljskim kraljevstvima, ti si stvorio nebo i zemlju. 
\par 16 Prikloni uho, Jahve, i počuj, otvori oči, Jahve, i vidi! Sanheribove čujder riječi koje poruči da izruga Boga živoga. 
\par 17 Istina je, o Jahve, asirski su kraljevi zatrli narode  i zemlje njihove; 
\par 18 pobacali im u oganj bogove; jer ne bijahu  bogovi to, već djela ruku ljudskih, od drveta i kamena; zato  ih i uništiše. 
\par 19 Ali sada, Jahve, Bože naš, izbavi nas iz ruke  njegove da spoznaju sva kraljevstva zemlje da si ti, Jahve, Bog  jedini." 
\par 20 Tada Izaija, sin Amosov, poruči Ezekiji: "Ovako veli  Jahve, Bog Izraelov: 'Uslišah molitvu koju mi uputi zbog Sanheriba, kralja asirskog.' 
\par 21 Evo riječi što je Jahve objavi protiv  njega:  Prezire te, ruga ti se djevica, kći sionska; za tobom maše glavom kći jeruzalemska. 
\par 22 Koga si grdio, hulio? Na koga si glasno vikao, ohol pogled dizao? Na Sveca Izraelova! 
\par 23 Po slugama si svojim vrijeđao Gospoda. Govorio si: s mnoštvom kola ja popeh se na vrh gora, na najviše vrhunce Libanona. Posjekoh mu cedre najviše i čemprese ponajljepše. Dosegoh mu vrh najviši i vrt njegov šumoviti. 
\par 24 Kopao sam i pio sam vode tuđe; stopalima tad isuših sve rijeke egipatske. 
\par 25 Čuješ li dobro? Odavna to sam snovao, od iskona smišljao, sada to ostvarujem: na tebi je da prometneš gradove tvrde u razvaline. 
\par 26 Stanovnici njini, nemoćni, prepadnuti i smeteni, bjehu kao trava u polju, kao mlado zelenilo, kao trava vrh krovova opaljena vjetrom istočnim. 
\par 27 Znam kad se dižeš i kad sjedaš, kad izlaziš i kad se vraćaš. 
\par 28 Jer bjesnio si na me i jer obijest tvoja do ušiju mi dođe, prsten ću ti provuć' kroz nozdrve, uzde stavit' u žvale, vratit ću te putem kojim si i došao! 
\par 29 A znak nek' ti bude ovo: Ove će se godine jesti što se samo okrÄunÄi, dogodine što samo uzraste, a treće godine sijte i žanjite, sadite vinograde, jedite im rod. 
\par 30 Preživjeli iz kuće Judine žilje će pustit' u dubinu, plodom rodit' u visinu. 
\par 31 Jer će iz Jeruzalema izići Ostatak, Sačuvani s gore Siona. Sve će to učinit' ljubomora Jahvina! 
\par 32 Zato ovo govori Jahve o kralju asirskom: 'U ovaj grad on ući neće, ovamo strijele svoje neće izmetati, k njemu neće ni štit okrenuti, niti oko njega nasipe kopati. 
\par 33 Vratit će se putem kojim je i došao, u grad ovaj neće ući' - Jahvina je riječ. 
\par 34 Grad ću ovaj štitit, spasiti ga, sebe radi i rad sluge svoga Davida." 
\par 35 Te iste noći iziđe Anđeo Jahvin i pobi u asirskom taboru  stotinu osamdeset i pet tisuća ljudi. Ujutro kad je valjalo ustati, gle, bijahu ondje sve sami mrtvaci. 
\par 36 Sanherib podiže tabor i ode. Vratio se u Ninivu. 
\par 37 Jednoga  dana, dok se klanjao u hramu svoga boga Nisroka, njegovi ga sinovi  Adramelek i Sareser ubiše mačem i pobjegoše u zemlju araratsku.  Na njegovo se mjesto zakralji sin mu Asar-Hadon. 


\chapter{20}

\par 1 U ono se vrijeme Ezekija razbolje nasmrt. Prorok Izaija, sin  Amosov, dođe mu i reče: "Ovako veli Jahve: 'Uredi kuću svoju  jer ćeš umrijeti; nećeš ozdraviti.'" 
\par 2 Ezekija se okrenu zidu  i ovako se pomoli Jahvi: 
\par 3 "Ah, Jahve! Sjeti se milostivo da  sam pred tobom hodio vjerno i poštena srca i da sam činio što  je dobro u tvojim očima." I Ezekija briznu u gorak plač. 
\par 4 Izaija još ne bijaše izišao iz središnjeg predvorja kad  mu je stigla riječ Jahvina: 
\par 5 "Vrati se i reci Ezekiji, glavaru  moga naroda. Ovako veli Jahve, Bog tvoga oca Davida: 'Uslišao  sam tvoju molitvu, vidio sam tvoje suze. Izliječit ću te; za  tri dana uzići ćeš u Dom Jahvin. 
\par 6 Dodat ću tvome vijeku još  petnaest godina. Izbavit ću tebe i ovaj grad iz ruku asirskoga  kralja; zakrilit ću ovaj grad radi sebe i sluge svoga Davida.'" 
\par 7 Izaija naloži: "Uzmite oblog od smokava, privijte mu ga  na čir i on će ozdraviti." 
\par 8 Ezekija upita Izaiju: "Po kojem ću znaku prepoznati da  će me Jahve izliječiti i da ću za tri dana uzići u Dom Jahvin?" 
\par 9 Izaija odgovori: "Evo ti znaka od Jahve da će učiniti što  je rekao: hoćeš li da se sjena pomakne za deset stupnjeva naprijed  ili da se vrati za deset stupnjeva?" 
\par 10 Ezekija odgovori: "Lako  je sjeni pomaknuti se deset stupnjeva naprijed! Ne! Neka se sjena  vrati natrag za deset stupnjeva!" 
\par 11 Prorok Izaija zazva Jahvu  i on učini da se sjena vrati za deset stupnjeva. Sišla je za  deset posljednjih stupnjeva na Ahazovu sunčaniku. 
\par 12 U to vrijeme posla babilonski kralj Merodak-Baladan,  sin Baladanov, pisma s darom Ezekiji, jer bijaše čuo da se razbolio  i ozdravio. 
\par 13 Ezekija se obradova tome i pokaza poslanicima  svoju riznicu - srebro, zlato, miomirise, mirisavo ulje - svoju  oružanu i sve što je bilo u skladištima. Nije bilo ničega u njegovu  dvoru i svemu njegovu gospodarstvu što im Ezekija nije pokazao. 
\par 14 Tada prorok Izaija dođe kralju Ezekiji i upita ga: "Što  su rekli ti ljudi i odakle su došli k tebi?" Ezekija odgovori:  "Došli su iz daleke zemlje, iz Babilona." 
\par 15 Izaija upita dalje:  "Što su vidjeli u tvojem dvoru?" Ezekija odgovori: "Vidjeli su  sve što je u mojem dvoru; nema u mojim skladištima ničega što  im nisam pokazao." 
\par 16 Tada Izaija reče Ezekiji: "Čuj riječ Jahvinu: 
\par 17 'Evo  dolaze dani kada će sve što je u tvojem dvoru, sve što su tvoji  oci nakupili do danas, biti odneseno u Babilon. Ništa neće ostati, ' kaže Jahve. 
\par 18 A od sinova što poteku od tebe, što ti se rode, neke će uzeti da budu uškopljeni dvorani u palači babilonskoga  kralja." 
\par 19 Ezekija odgovori Izaiji: "Povoljna je riječ koju  ti je Jahve objavio." A mislio je: "Zašto ne? Ako bude mira i  sigurnosti za moga života!" 
\par 20 Ostala povijest Ezekijina, svi njegovi pothvati i kako  je sagradio ribnjak i prorov da dovede vodu u grad, zar sve to  nije zapisano u knjizi Ljetopisa judejskih kraljeva? 
\par 21 Ezekija  je počinuo sa svojim ocima, a njegov sin Manaše zakralji se mjesto  njega. 


\chapter{21}

\par 1 Manašeu je bilo dvanaest godina kad se zakraljio. Pedeset  i pet godina kraljevao je u Jeruzalemu. Njegova se majka zvala  Hefsi-Bah. 
\par 2 Činio je što je zlo u Jahvinim očima, povodeći  se za gnusobama naroda što ih je Jahve protjerao pred Izraelovim  sinovima. 
\par 3 Obnovio je uzvišice što ih bijaše oborio otac mu  Ezekija, podigao je žrtvenik Baalu, načinio ašere kako bijaše  učinio izraelski kralj Ahab; i stao se klanjati svoj vojsci nebeskoj  i služiti joj. 
\par 4 Podigao je žrtvenike i u Domu Jahvinu, za koji  bijaše rekao Jahve: "U Jeruzalemu će prebivati moje Ime zauvijek." 
\par 5 Sagradio je žrtvenike svoj nebeskoj vojsci u oba predvorja  Doma Jahvina. 
\par 6 I sinove je svoje proveo kroz oganj. Vračao  je, gatao, stvorio bajače i opsjenare, učinio je premnogo zla  u očima Jahve i razjarivao ga. 
\par 7 Dao je načiniti lik Ašere i  posadio ga u Domu, za koji Jahve bijaše rekao Davidu i njegovu  sinu Salomonu: "U ovom Domu i u Jeruzalemu, koji sam izabrao  među svim izraelskim plemenima, postavit ću svoje Ime zauvijek. 
\par 8 Neću više dati da noga Izraelaca uzmakne iz zemlje koju sam  dao u baštinu njihovim očevima, samo ako budu držali i provodili  u djelo sve što sam im zapovjedio: Zakon što im ga je objavio  moj sluga Mojsije." 
\par 9 Ali oni nisu poslušali, Manaše ih je zaveo  te su radili još gore nego narodi što ih je Jahve iskorijenio  pred Izraelovim sinovima. 
\par 10 Tada je Jahve ovako govorio preko slugu svojih proroka: 
\par 11 "Zato što je judejski kralj Manaše činio te gnusobe, zato  što je učinio više zla nego što su prije njega radili Amorejci  i što je zaveo Judejce svojim idolima, 
\par 12 ovako veli Jahve,  Bog Izraelov: 'Evo, učinit ću da dođe nevolja na Jeruzalem i  Judeju, takva da će zazujati oba uha onima koji o njoj čuju. 
\par 13 Nategnut ću nad Jeruzalemom isto uže kao nad Samarijom, isto  mjerilo kao nad kućom Ahabovom; zbrisat ću Jeruzalem kao što  se briše zdjela pa se tad izvrne. 
\par 14 Odbacit ću ostatke svoje  baštine, predat ću ih u ruke njihovih neprijatelja; služit će  za plijen i grabež svim svojim neprijateljima 
\par 15 jer su činili  što je zlo u mojim očima jer su izazivali moj gnjev od dana kada  su njihovi oci izišli iz Egipta pa sve do danas.'" 
\par 16 I mnogo je nedužne krvi prolio Manaše, tako da se njome  napunio Jeruzalem od jednoga kraja do drugoga, da se i ne spominje  njegov grijeh kojim je zaveo Judu da čini što je zlo u očima  Jahvinim. 
\par 17 Ostala povijest Manašeova, njegova djela i grijesi koje  je počinio, zar sve to nije zapisano u knjizi Ljetopisa judejskih  kraljeva? 
\par 18 Manaše je počinuo kraj svojih otaca i sahranjen  je u vrtu svojeg dvora, u vrtu Uzinu. Sin mu Amon zakralji se  mjesto njega. 
\par 19 Amonu bijahu dvadeset i dvije godine kad je zavladao, a kraljevao je dvije godine u Jeruzalemu. Njegova se majka zvala  Mešulemet, kći Harusova, i bila je iz Jotbe. 
\par 20 On je činio  što je zlo u očima Jahvinim, kao što je činio njegov otac Manaše. 
\par 21 U svemu je slijedio put svoga oca, služio je idolima kojima  je služio i njegov otac i klanjao im se. 
\par 22 On je ostavio Jahvu, Boga svojih praotaca, i nije hodio putem Jahvinim. 
\par 23 Amonovi se časnici urotiše protiv njega i ubiše kralja  u dvoru. 
\par 24 Ali je prosti puk pobio sve one koji se bijahu urotili  protiv kralja Amona i na njegovo mjesto zakraljio sina mu Jošiju. 
\par 25 Ostala povijest Amonova i sve što je činio, zar sve to  nije zapisano u knjizi Ljetopisa judejskih kraljeva? 
\par 26 Pokopali  su ga u grobnicu njegova oca, u vrtu Uzinu, a njegov sin Jošija  zakralji se mjesto njega. 


\chapter{22}

\par 1 Jošiji je bilo osam godina kad se zakraljio. Kraljevao je  trideset i jednu godinu u Jeruzalemu. Mati mu se zvala Jedida, kći Adajina, i bila je iz Boskata. 
\par 2 Činio je što je pravo  u Jahvinim očima. U svemu je hodio putem svoga oca Davida, ne  skrećući ni desno ni lijevo. 
\par 3 Osamnaeste godine svoga kraljevanja Jošija posla svoga  tajnika Šafana, sina Asalijahina, sina Mešulamova, u Dom Jahvin  i reče mu: 
\par 4 "Idi velikom svećeniku Hilkiji da ti pripremi novac  koji je odnesen u Dom Jahvin i koji su čuvari praga sakupili  od naroda. 
\par 5 Neka ga uruči poslovođama postavljenim u Domu Jahvinu, a oni neka isplate radnike koji popravljaju Dom Jahvin, 
\par 6 drvodjelje, graditelje i zidare, i da se kupuje drvo i kamenje klesano što  je potrebno za popravak Doma. 
\par 7 Ali neka se ne traži od njih  račun za uručeni novac jer oni rade pošteno." 
\par 8 Veliki svećenik Hilkija reče tajniku Šafanu: "Našao sam  Knjigu Zakona u Domu Jahvinu." I Hilkija dade knjigu Šafanu,  koji ju je pročitao. 
\par 9 Tajnik Šafan dođe kralju te ga izvijesti:  "Tvoje sluge", reče on, "pokupile su novac koji se našao u Domu  i predale su ga poslovođama postavljenim u Domu Jahvinu." 
\par 10 Tada  tajnik Šafan obavijesti kralja: "Svećenik Hilkija dade mi jednu  knjigu." I Šafan je poče čitati pred kraljem. 
\par 11 Čuvši riječi Knjige Zakona, kralj razdrije haljine svoje. 
\par 12 I naredi svećeniku Hilkiji, Šafanovu sinu Ahikamu, Mikinu  sinu Akboru, tajniku Šafanu i kraljevu sluzi Asaji: 
\par 13 "Idite  i upitajte Jahvu o meni, i o narodu, i o svoj Judeji zbog ove  knjige što je nađena, jer je velika Jahvina jarost što se izlila  na nas zato što naši očevi nisu slušali riječi ove knjige, nisu  vršili što nam je u njoj napisano." 
\par 14 Svećenik Hilkija, Ahikam, Akbor, Šafan i Asaja odoše  proročici Huldi, ženi Šaluma, sina Tikvina, sina Harkasova, čuvara  odjeće; ona je živjela u Jeruzalemu, u novom gradu. Kad joj to  kazaše, 
\par 15 ona im reče: "Ovako veli Jahve, Bog Izraelov: 'Kažite  čovjeku koji vas je poslao k meni: 
\par 16 Ovako veli Jahve: Evo, dovest ću nesreću na ovaj grad i na njegove stanovnike, izvršit  ću sve što kaže knjiga koju je pročitao judejski kralj. 
\par 17 Jer  su me ostavili i prinose žrtve tuđim bogovima da bi me ljutili  svim djelima ruku svojih, planut će jarost moja na to mjesto  i neće se ugasiti.' 
\par 18 A judejskom kralju, koji vas je poslao  po Jahvin savjet, recite ovo: 'Ovako veli Jahve, Bog Izraelov:  Riječi si čuo. 
\par 19 Ali kako ti je omekšalo srce i jer si se ponizio  pred Jahvom čuvši što sam objavio tome gradu i njegovim stanovnicima, koje će pogoditi pustošenje i prokletstvo, i jer si razdro haljine  svoje i plakao preda mnom, zato sam te uslišio' - riječ je Jahvina. 
\par 20 'Evo, sjedinit ću te s ocima tvojim i s mirom ćeš leći u  grob da ne vidiš svojim očima svu nesreću koju ću svaliti na  ovo mjesto.'" Oni odnesoše taj odgovor kralju. 


\chapter{23}

\par 1 Tada kralj posla da se saberu kod njega sve judejske i jeruzalemske  starješine. 
\par 2 Kralj potom uzađe u Dom Jahvin s Judejcima, Jeruzalemcima, svećenicima i prorocima i sa svim narodom, od najmanjega do  najvećega. I pročita im sve riječi Knjige Saveza koja je nađena  u Domu Jahvinu. 
\par 3 Kralj, stojeći na svome mjestu, obnovi pred  Jahvom Savez da će slijediti Jahvu i držati se njegovih zapovijedi, pouka i uredaba svim srcem i svom dušom da bi ispunio sve stavke  toga Saveza zapisane u ovoj knjizi. Sav je narod stupio u Savez. 
\par 4 Kralj je zapovjedio velikom svećeniku Hilkiji, svećenicima  drugog reda i čuvarima hramskog praga da iz Svetišta Jahvina  iznesu sve bogoslužne predmete što bijahu načinjeni za Baala, za Ašeru i za svu nebesku vojsku. Odredio je da sve to spale  izvan Jeruzalema u poljima kidronskim, a pepeo je odnio u Betel. 
\par 5 Uklonio je lažne svećenike koje su judejski kraljevi postavili  da pale kad na uzvišicama, u gradovima judejskim i u okolici  Jeruzalema; i one koji su palili kad Baalu, suncu, mjesecu, zvijezdama  i svoj vojsci nebeskoj. 
\par 6 Izvan Jeruzalema iznio je iz Doma  Jahvina, u dolinu kidronsku, Ašeru i spalio ju je u dolini kidronskoj, satro u prah, a prah bacio na groblje sinova pučkih. 
\par 7 Razorio  je stanove posvećenih bludnica koji su bili u Domu Jahvinu i  u kojima su žene tkale haljine Ašeri. 
\par 8 Iz svih judejskih gradova doveo je svećenike i oskvrnuo  je uzvišice gdje su ti svećenici prinosili kad, od Gebe do Beer  Šebe. Zatim je srušio uzvišice pred vratima, one koje su bile  na ulazu vrata Jošue, upravitelja grada, nalijevo kad se prilazi  gradskim vratima. 
\par 9 Isto tako svećenici uzvišica nisu mogli  uzlaziti žrtveniku Jahvinu u Jeruzalemu, ali su jeli kruhove  bez kvasa među svojom braćom. 
\par 10 Oskvrnio je Tofet u dolini  Ben Hinom, kako nitko ne bi svoga sina ili kćerku provodio kroz  oganj u čast Moleku. 
\par 11 Razagnao je konje koje su judejski kraljevi  prinijeli suncu na ulazu u Dom Jahvin, kraj sobe dvoranina Netan  Meleka, koja se nalazila u blizini, i spalio je u ognju sunčana  kola. 
\par 12 Žrtvenike na krovu koje bijahu sagradili judejski kraljevi  i one koje je sagradio Manaše u oba predvorja Hrama Jahvina,  kralj je srušio, uklonio ih odatle i bacio njihov prah u dolinu  kidronsku. 
\par 13 Uzvišice koje su bile sučelice Jeruzalemu, na  južnom dijelu Maslinske gore, i koje je izraelski kralj Salomon  bio sagradio Aštarti, sramoti sidonskoj, Kemošu, sramoti moapskoj, i Milkomu, nakazi amonskoj - sve ih je kralj oskvrnio. 
\par 14 Razbio  je stupove, iskorijenio ašere i njihova je mjesta ispunio ljudskim  kostima. 
\par 15 Isto tako i žrtvenik u Betelu, uzvišicu koju je sagradio  Jeroboam, sin Nebatov, koji je naveo Izraela na grijeh, kralj  je srušio, oborio žrtvenik i tu uzvišicu, satro kamenje u prah, spalio ašere. 
\par 16 A kad se Jošija okrenuo i vidio grobove koji  bijahu ondje na gori, posla da se sakupe kosti iz onih grobova  i spali ih na žrtveniku. Tako ga je oskvrnuo, izvršavajući riječ  Jahvinu, koju je objavio čovjek Božji (dok je Jeroboam bio na  žrtveniku za vrijeme svečanosti). Okrenuvši se, Jošija baci oči  na grob čovjeka Božjeg koji je objavio sve to 
\par 17 i upita: "Kakav  je ono spomenik što ga vidim?" Ljudi iz grada odgovoriše mu:  "To je grob čovjeka Božjeg koji je došao iz Judeje i koji je  prorekao sve ovo što si ti učinio s betelskim žrtvenikom." 
\par 18 "Pustite  ga na miru", reče kralj, "i neka nitko ne dira njegove kosti."  Tako su ostale njegove kosti netaknute s kostima proroka koji  je došao iz Samarije. 
\par 19 Jošija je jednako razorio sve hramove uzvišica koje su  izraelski kraljevi sagradili po gradovima Samarije da bi srdili  Jahvu i učinio je s njima kao što je učinio u Betelu. 
\par 20 Sve  svećenike uzvišica poklao je na žrtvenicima; na njima je spalio  i ljudske kosti. Potom se vratio u Jeruzalem. 
\par 21 Kralj naredi svemu narodu: "Svetkujte Pashu u čast Jahve, Boga svoga, po običaju koji je zapisan u ovoj Knjizi Saveza." 
\par 22 Takva se Pasha nije svetkovala od vremena sudaca koji su  sudili Izraelu i za sve vrijeme kraljeva izraelskih i judejskih. 
\par 23 Samo je osamnaeste godine kraljevanja Jošijina svetkovana  takva Pasha u čast Jahve, u Jeruzalemu. 
\par 24 Osim toga, sve bajače i sve vračare, sve kućne bogove  i idole i sve sramote koje se mogu vidjeti u zemlji judejskoj  i Jeruzalemu - sve je to Jošija uklonio da izvrši riječi Zakona, zapisane u knjizi koju je našao Hilkija, svećenik Doma Jahvina. 
\par 25 Nije bilo prije njega takva kralja koji se obratio Jahvi  svim srcem svojim, svom dušom svojom i svom snagom svojom, u  svemu vjeran Zakonu Mojsijevu, a ni poslije njega nije mu bilo  ravna. 
\par 26 Ipak Jahve nije odustao od plamena svoga velikoga  gnjeva kojim je uskipio protiv Judejaca zbog svih izazova kojima  ga je Manaše ljutio. 
\par 27 Jahve je odlučio: "Maknut ću Judejce  ispred sebe kao što sam maknuo Izraela; odbacit ću ovaj grad  koji sam izabrao, Jeruzalem, i Dom o kojem rekoh: 'Tu će biti  Ime moje.'" 
\par 28 Ostala povijest Jošijina i sve što je učinio, zar sve  to nije zapisano u knjizi Ljetopisa judejskih kraljeva? 
\par 29 U  njegvo je vrijeme faraon Neko, egipatski kralj, krenuo protiv  asirskoga kralja na rijeci Eufratu. Kralj Jošija pošao je preda  nj, ali ga on ubi u Megidu, pri prvom susretu. 
\par 30 Sluge njegove  prenesoše mu tijelo kolima iz Megida, odvezoše ga u Jeruzalem  i sahraniše u njegovoj grobnici. Sav narod zemlje primi Joahaza, sina Jošijina; pomazaše ga i proglasiše kraljem namjesto njegova  oca. 
\par 31 Joahazu bijahu dvadeset i tri godine kad se zakraljio.  Kraljevao je tri mjeseca u Jeruzalemu. Njegova se majka zvala  Hamitah, kći Jeremije, i bila je iz Libne. 
\par 32 On je činio što  je zlo u očima Jahvinim, sve kao što su činili oci njegovi. 
\par 33 Faraon  Neko bacio ga je u okove u Ribli, na području Hamata, da ne vlada  u Jeruzalemu i udario je na zemlju danak od stotinu talenata  srebra i deset talenata zlata. 
\par 34 Faraon Neko postavio je za  kralja Elijakima, sina Jošijina, na mjesto njegova oca Jošije.  I ime mu je promijenio u Jojakim. A Joahaza je uzeo i odveo u  Egipat te on umrije ondje. 
\par 35 Jojakim je dao faraonu srebro  i zlato, ali je nametnuo zemlji porez da bi smogao svotu koju  je faraon zahtijevao. Svakome je nametnuo prema njegovu stanju, uzimao srebro i zlato koje je morao davati faraonu Neku. 
\par 36 Jojakimu je bilo dvadeset i pet godina kad je postao  kraljem i kraljevao je jedanaest godina u Jeruzalemu. Materi  mu je bilo ime Zebida, kći Pedajina, i bila je iz Rume. 
\par 37 On  je činio što je zlo u očima Jahvinim, sve kao što su činili i  oci njegovi. 


\chapter{24}

\par 1 U njegovo je vrijeme došao Nabukodonozor, kralj babilonski, i Jojakim mu je bio podložan tri godine, zatim se ponovno pobunio  protiv njega. 
\par 2 Ovaj pak posla protiv njega kaldejske pljačkaške  čete, aramejske, moapske i amonske, sve ih posla protiv Judeje  da je opustoše, potvrđujući riječ koju je Jahve bio objavio po  slugama svojim prorocima. 
\par 3 To se dogodilo Judeji prema prijetnji  Jahvinoj da će je istrijebiti ispred svoga lica zbog grijeha  Manašeovih: zbog svega što je Manaše učinio 
\par 4 i zbog nedužne  krvi koju je prolio, natopio Jeruzalem krvlju nedužnom. Jahve  nije htio oprostiti. 
\par 5 Ostala povijest Jojakimova i sve što je učinio, zar sve  to nije zapisano u knjizi Ljetopisa judejskih kraljeva? 
\par 6 Jojakim  je počinuo kraj svojih otaca, a njegov sin Jojakin zavlada mjesto  njega. 
\par 7 Egipatski kralj nije više izlazio iz zemlje, jer je babilonski  kralj osvojio od Egipatskog potoka do rijeke Eufrata sve što  je pripadalo egipatskom kralju. 
\par 8 Jojakinu je bilo osamnaest godina kad se zakraljio i kraljevao  je tri mjeseca u Jeruzalemu. Materi mu je bilo ime Nehušta, kći  Elnatana, i bila je iz Jeruzalema. 
\par 9 On je činio što je zlo  u očima Jahvinim, sve kao što je činio i njegov otac. 
\par 10 U ono vrijeme krenu ljudstvo babilonskog kralja Nabukodonozora  protiv Jeruzalema i grad je bio opkoljen. 
\par 11 Dođe i babilonski  kralj Nabukodonozor da napadne grad, dok ga je njegovo ljudstvo  opsjedalo. 
\par 12 Tada je judejski kralj Jojakin izišao pred babilonskoga  kralja: on, njegova majka, njegove sluge, njegove vojskovođe  i dvorani, a babilonski kralj zarobi ga - osme godine svoga kraljevanja. 
\par 13 On je odnio sve iz riznice Doma Jahvina i iz riznica kraljevskog  dvora i razbio je sve zlatne predmete koje je Salomon, kralj  Izraela, načinio za Svetište Jahvino. Tako se ispunila riječ  Jahvina. 
\par 14 Odveo je u progonstvo sav Jeruzalem, sve vojskovođe  i sve vrsne ratnike, oko deset tisuća prognanika, sa svim kovačima  i bravarima. Jedino je preostao najsiromašniji narod zemlje. 
\par 15 Odveo je Jojakina u Babilon; tako isto i kraljevu majku i  sve žene kraljeve, njegove dvorane, plemenitaše zemlje, sve ih  je odveo iz Jeruzalema u progonstvo u Babilon. 
\par 16 Sve sposobne  ljude, njih sedam tisuća na broju; kovače i bravare, tisuću na  broju; sve ljude sposobne za boj, sve ih je kralj babilonski  odveo u Babilon, u sužanjstvo. 
\par 17 Babilonski je kralj postavio  za kralja mjesto Jojakina njegova strica Mataniju, ali mu je  promijenio ime u Sidkija. 
\par 18 Sidkiji je bila dvadeset i jedna godina kad se zakraljio, a kraljevao je jedanaest godina u Jeruzalemu. Materi mu bijaše  ime Hamitala, kći Jeremije, i bila je iz Libne. 
\par 19 Činio je  što je zlo u očima Jahvinim, sve kao što je činio Jojakin. 
\par 20 To  je zadesilo Jeruzalem i Judu zbog gnjeva Jahvina; Jahve ih napokon  i odbaci ispred lica svoga. Sidkija se pobuni protiv babilonskog kralja. 



\chapter{25}

\par 1 Devete godine njegova kraljevanja, desetoga dana desetoga  mjeseca, krenu sam babilonski kralj Nabukodonozor sa svom svojom  vojskom na Jeruzalem. Utabori se pred gradom i opasa ga opkopom. 
\par 2 Grad osta opkoljen do jedanaeste godine Sidkijina kraljevanja. 
\par 3 Devetoga dana četvrtoga mjeseca, kad je u gradu zavladala  takva glad da priprosti puk nije imao ni kruha, 
\par 4 neprijatelj  provali u grad. Tada kralj i svi ratnici pobjegoše noću kroz  vrata između dva zida nad Kraljevim vrtom - Kaldejci bijahu opkolili  grad - i krenuše putem prema Arabi. 
\par 5 Kaldejske čete nagnuše  u potjeru za kraljem i sustigoše ga na Jerihonskim poljanama, a sva se njegova vojska razbježala. 
\par 6 Kaldejci uhvatiše kralja  i odvedoše ga u Riblu pred kralja babilonskog, koji mu izreče  presudu. 
\par 7 Sidkijine sinove pokla pred njegovim očima, Sidkiji  iskopa oči, okova ga verigama i odvede u Babilon. 
\par 8 Sedmoga dana petoga mjeseca - devetnaeste godine kraljevanja  Nabukodonozora, kralja babilonskog - uđe u Jeruzalem Nebuzaradan, zapovjednik kraljeve tjelesne straže i časnik babilonskog kralja. 
\par 9 On zapali Dom Jahvin, kraljevski dvor i sve kuće u Jeruzalemu. 
\par 10 Kaldejske čete, pod zapovjednikom kraljevske tjelesne straže, razoriše zidine koje su okruživale Jeruzalem. 
\par 11 Nebuzaradan, zapovjednik kraljeve tjelesne straže, odvede  u sužanjstvo ostatak naroda koji bijaše ostao u gradu, a tako  i prebjege babilonskom kralju i ostalu svjetinu. 
\par 12 Neke od  malih ljudi ostavi zapovjednik u zemlji kao vinogradare i ratare. 
\par 13 Kaldejci razbiše tučane stupove u Domu Jahvinu, podnožja  i mjedeno more koji su bili u Domu Jahvinu i tuč odniješe u Babilon. 
\par 14 Uzeše i lonce, lopate, noževe, posudice i uopće sav tučani  pribor koji se upotrebljavao za bogoslužja. 
\par 15 Zapovjednik uze  i kadionice i škropionice, uopće sve što bijaše od zlata i srebra, 
\par 16 dva stupa, jedno more i podnožja, što je Salomon dao izraditi  za Dom Jahvin. Nije moguće procijeniti koliko je tuča bilo u  svim tim predmetima. 
\par 17 Prvi stup bijaše visok osamnaest lakata, imao je glavicu od tuča, visoku pet lakata; obvijaše je oplet  i mogranji, sve od tuča. Takav je bio i drugi stup. 
\par 18 Zapovjednik straže odveo je svećeničkog poglavara Seraju, drugog svećenika, Sefaniju, i tri čuvara praga. 
\par 19 Iz grada  je odveo jednog dvoranina, vojničkog zapovjednika, pet ljudi  iz kraljeve pratnje koji se zatekoše u gradu, pisara zapovjednika  vojske koji je novačio puk i šezdeset pučana koji se također  zatekoše u gradu. 
\par 20 Zapovjednik kraljevske tjelesne straže  Nebuzaradan odvede ih pred kralja babilonskoga u Riblu. 
\par 21 I  kralj babilonski zapovjedi da ih pogube u Ribli, u zemlji hamatskoj.  Tako su judejski narod odveli s njegove rodne grude. 
\par 22 Narodu što je ostao u zemlji judejskoj i što ga je ostavio  babilonski kralj Nabukodonozor - postavio je ovaj za upravitelja  Gedaliju, sina Ahikamova, unuka Šafanova. 
\par 23 Svi vojni zapovjednici  i njihovi ljudi saznaše da je babilonski kralj postavio zemlji  za namjesnika Gedaliju i dođoše pred njega u Mispu: Netanijin  sin Jišmael; Kareahov sin Johanan; sin Tanhumeta iz Netofe, Seraja;  Maakatijev sin Jaazanija - oni i svi njihovi ljudi. 
\par 24 Gedalija  se zakle njima i njihovim ljudima i reče: "Ne bojte se služiti  Kaldejcima; ostanite u zemlji, budite podložni babilonskom kralju  i bit će vam dobro." 
\par 25 Ali sedmoga mjeseca Jišmael, sin Netanijin, unuk Elišamin, koji bijaše kraljevskog roda, i još deset ljudi s njim ubiše  Gedaliju te on umrije kao i svi Judejci i Kaldejci koji bijahu  s njim u Mispi. 
\par 26 Tada sav narod, od maloga do velikog, i svi zapovjednici  četa ustadoše i odoše u Egipat jer se bojahu Kaldejaca. 
\par 27 Trideset i sedme godine otkako je zasužnjen judejski  kralj Jojakin, dvadeset i sedmog dana dvanaestoga mjeseca, babilonski  kralj Evil Merodak u prvoj godini svoje vladavine pomilova judejskog  kralja Jojakina i pusti ga iz tamnice. 
\par 28 Ljubezno je s njime  razgovarao i stolicu mu postavio više nego ostalim kraljevima  koji bijahu s njim u Babilonu. 
\par 29 Jojakin je odložio svoje tamničke  haljine i jeo s kraljem za istim stolom svega svoga vijeka. 
\par 30 Do  kraja njegova života babilonski mu je kralj trajno, iz dana u  dan, davao uzdržavanje. 





\end{document}