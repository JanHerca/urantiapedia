\begin{document}

\title{Daniel}


\chapter{1}

\par 1 Treće godine kraljevanja Jojakima, kralja Judeje, dođe Nabukodonozor, kralj Babilona, na Jeruzalem te ga opsjede. 
\par 2 Gospodin mu dade  u ruke Jojakima, kralja judejskog, i dio predmeta iz Doma Božjega;  on ih dopremi u zemlju Šinear i pohrani predmete u riznici svojih  bogova. 
\par 3 Kralj naredi Ašfenazu, starješini svojih dvorjanika, da  dovede od Izraelaca nekoliko dječaka kraljevskoga ili velikaškog  roda: 
\par 4 neka budu bez nedostatka, lijepi, vrsni u svakoj mudrosti, dobro poučeni i bistri, prikladni da stoje na kraljevu dvoru;  Ašfenaz neka ih nauči pismu i jeziku Kaldejaca. 
\par 5 Kralj im odredi  dnevni obrok od kraljevskih jela i od vina sa svoga stola. Neka  se odgajaju tri godine, a poslije toga imali bi stajati pred  kraljem. 
\par 6 Među njima bijahu Judejci: Daniel, Hananija, Mišael  i Azarja. 
\par 7 Dvorjanički starješina nadjene im imena: Daniel  će se zvati Baltazar, Hananija Šadrak, Mišael Mešak, Azarja Abed  Nego. 
\par 8 Daniel je u srcu odlučio da se neće okaljati kraljevim  jelima i vinom s njegova stola, pa zamoli dvorjaničkog starješinu  da ga poštedi te se ne okalja. 
\par 9 Bog dade Danielu te nađe dobrohotnost  i smilovanje u dvorjaničkog starješine. 
\par 10 Starješina reče Danielu:  "Bojim se svoga gospodara kralja; on vam je odredio jelo i pilo, pa ako vidi da su vam lica mršavija nego u drugih dječaka vaše  dobi, ja ću zbog vas biti kriv pred kraljem." 
\par 11 Tada reče Daniel  čuvaru koga bijaše dvorjanički starješina odredio Danielu, Hananiji, Mišaelu i Azarji: 
\par 12 "Molim te, pokušaj sa svojim slugama deset  dana: neka nam se daje povrće za jelo i voda za pilo. 
\par 13 Vidjet  ćeš onda kakvi ćemo biti mi a kakvi dječaci koji jedu od kraljevih  jela, pa učini sa svojim slugama po onome što budeš vidio." 
\par 14 On  pristade i stavi ih na kušnju deset dana. 
\par 15 A nakon deset dana  oni bijahu ljepši i ugojeniji nego svi dječaci koji jeđahu od  kraljevih jela. 
\par 16 Od tada čuvar dokinu njihova jela i obrok  vina što su imali piti te im davaše povrća. 
\par 17 Ovoj četvorici dječaka dade Bog znanje i razumijevanje  svih knjiga i mudrosti. Daniel razumijevaše viđenja i sne. 
\par 18 Pošto  se navršilo vrijeme određeno od kralja da mu ih dovedu, dvorjanički  starješina uvede ih pred Nabukodonozora. 
\par 19 Kralj razgovaraše  s njima i među svima ne nađe se nijedan kao Daniel, Hananija, Mišael i Azarja. I tako oni ostadoše pred kraljem. 
\par 20 I u svemu  mudrom i umnom o čemu ih ispitivaše kralj nađe da su deset puta  vrsniji od svih čarobnika i gatalaca što ih bijaše u svem njegovu  kraljevstvu. 
\par 21 Daniel ostade ondje do prve godine kralja Kira. 


\chapter{2}

\par 1 Druge godine Nabukodonozorova kraljevanja usni Nabukodonozor  sanje: njegov se duh zbog toga uznemiri, a san ga ostavi. 
\par 2 Kralj  naredi da se pozovu čarobnici i gataoci, zaklinjači i zvjezdari  da protumače kralju njegove sanje. 
\par 3 Dođoše dakle te stadoše  pred kralja. Kralj im reče: "Usnih jednu sanju i moj se duh uznemiri  od želje da razumijem sanju." 
\par 4 Kaldejci odgovoriše kralju (aramejski): "O kralju, živ bio dovijeka! Pripovjedi svoju sanju slugama  svojim, a mi ćemo ti otkriti njezino značenje." 
\par 5 Kralj odgovori  i reče zvjezdarima: "Moja je odluka neopoziva: ako mi ne kažete  što sam snio i što san znači, bit ćete rastrgani u komade, a  vaše će kuće postati smetišta. 
\par 6 No ako mi otkrijete moju sanju  i njezino značenje, dobit ćete od mene darove i poklone i velike  časti. Otkrijte mi dakle što sam snio i što san znači." 
\par 7 Oni  opet odgovoriše: "Neka kralj rekne svoju sanju slugama svojim, a mi ćemo mu otkriti njezino značenje." 
\par 8 A kralj: "Dobro ja  znam da želite dobiti na vremenu jer znate da je moja odluka  neopoziva. 
\par 9 Ako mi ne kažete što sam snio, znači da me namjeravate  obmanjivati varavim riječima i izmišljotinama dok nekako ne prođe  vrijeme. Stoga, recite mi moj san, pa ću znati da li mi možete  kazati i njegovo značenje!" 
\par 10 Zvjezdari odgovoriše pred kraljem: "Nema na svijetu čovjeka  koji bi takvo što mogao otkriti kralju. I stoga nijedan kralj, ma kako velik i moćan, takvo što ne traži od čarobnika, gataoca  ili zvjezdara. 
\par 11 Što tražiš, kralju, teško je, i nema ga tko  bi to mogao otkriti kralju osim bogova, koji ne borave među smrtnicima." 
\par 12 Tada se kralj silno razgnjevi i razbjesni te naredi da  se pogube svi mudraci babilonski. 
\par 13 Pošto je objavljena naredba  da se ubiju mudraci, potražiše i Daniela i njegove drugove da  ih pogube. 
\par 14 No Daniel se mudrim i umnim riječima obrati na  Arjoka, zapovjednika kraljevskih straža, koji bijaše izišao da  pogubi mudrace babilonske. 
\par 15 On reče Arjoku, zapovjedniku kraljevu:  "Zašto je kralj izdao tako strogu naredbu?" Arjok pripovjedi  Danielu, 
\par 16 a Daniel otiđe kralju i zamoli da mu dade vremena  te će kralju otkriti što san znači. 
\par 17 Daniel uđe u svoju kuću te sve kaza Hananiji, Mišaelu  i Azarji, svojim drugovima, 
\par 18 da mole milosrđe u Boga Nebeskoga  radi te tajne, da Daniel i njegovi drugovi ne poginu s drugim  mudracima babilonskim. 
\par 19 I objavi se tajna Danielu u noćnom  viđenju. A Daniel blagoslovi Boga Nebeskoga. 
\par 20 Daniel prihvati  riječ i reče: "Bilo ime Božje blagoslovljeno odvijeka dovijeka, njegova je mudrost i sila. 
\par 21 On mijenja doba i vremena, ruši i postavlja kraljeve, daje mudrost mudrima a znanje pronicavima. 
\par 22 On otkriva dubine i tajne, zna što je u tminama i svjetlost prebiva u njega. 
\par 23 Tebe, o Bože otaca mojih, slavim i hvalim što si mi dao mudrost i jakost! Evo, objavio si mi ono što smo te molili, objavio si nam što kralj traži." 
\par 24 Daniel ode k Arjoku, kome bijaše kralj naredio da smakne  mudrace babilonske. Uđe i reče mu: "Ne ubijaj mudraca babilonskih!  Odvedi me kralju, pa ću mu otkriti što san znači." 
\par 25 Arjok  žurno odvede Daniela kralju i reče: "Našao sam među izgnanicima  judejskim čovjeka koji će kralju kazati što san znači." 
\par 26 Kralj  reče Danielu (koji se zvaše Baltazar): "Jesi li kadar kazati  mi san koji sam usnio i što znači?" 
\par 27 Daniel odgovori pred kraljem: "Tajnu koju istražuje kralj  ne mogahu kralju otkriti mudraci, čarobnici, gataoci i zaklinjači; 
\par 28 ali ima na nebu Bog koji objavljuje tajne i on je saopćio  kralju Nabukodonozoru ono što će biti na svršetku dana. Evo tvoje  sanje i onoga što ti se prividjelo na postelji: 
\par 29 O kralju, na tvojoj ti postelji dođoše misli o tome što  će se dogoditi kasnije, a Otkrivatelj tajna saopćio ti je ono  što će biti. 
\par 30 Iako nemam mudrosti više nego ostali smrtnici, ta mi je tajna objavljena samo zato da njezino značenje saopćim  kralju i da upoznaš misli svoga srca. 
\par 31 Ti si, o kralju, imao viđenje: gle, kip, golem kip, vrlo  blistav, stajaše pred tobom, strašan za oči. 
\par 32 Tome kipu glava  bijaše od čistog zlata, prsa i ruke od srebra, trbuh i bedra  od mjedi, 
\par 33 gnjati od željeza, a stopala dijelom od željeza, dijelom od gline. 
\par 34 Ti si promatrao: iznenada se odvali kamen  a da ga ne dodirnu ruka, pa udari u kip, u stopala od željeza  i gline te ih razbi. 
\par 35 Tada se smrvi najednom željezo i glina, mjed, srebro i zlato, i sve postade kao pljeva na gumnu ljeti  i vjetar sve odnese bez traga. A kamen koji bijaše u kip udario  postade veliko brdo te napuni svu zemlju. 
\par 36 To bijaše sanja;  a njezino ćemo značenje reći pred kraljem." 
\par 37 "Ti, o kralju, kralju kraljeva, komu Bog Nebeski dade  kraljevstvo, silu moć i slavu - 
\par 38 i u čije je ruke stavio,  gdje god se našli, sinove ljudske, životinje poljske, ptice nebeske  i postavio te gospodarom nad svim time - ti si glava od zlata. 
\par 39 Poslije tebe ustat će drugo kraljevstvo, slabije od tvoga, pa treće, od mjedi, koje će gospodariti svom zemljom. 
\par 40 A  četvrto kraljevstvo bit će tvrdo poput željeza, poput željeza  koje sve satire i mrvi; kao željezo koje razbija, skršit će i  razbit sva ona kraljevstva. 
\par 41 Stopala koja si vidio, dijelom  glina a dijelom željezo, jesu podijeljeno kraljevstvo; imat će  nešto od čvrstoće željeza prema onome što si vidio željezo izmiješano  s glinom. 
\par 42 Prsti stopala, dijelom željezo a dijelom glina:  kraljevstvo će biti dijelom čvrsto a dijelom krhko. 
\par 43 A što  si vidio željezo izmiješano s glinom: oni će se miješati ljudskim  sjemenom, ali se neće držati zajedno, kao što se ni željezo ne  da pomiješati s glinom. 
\par 44 U vrijeme ovih kraljeva Bog Nebeski  podići će kraljevstvo koje neće nikada propasti i neće prijeći  na neki drugi narod. Ono će razbiti i uništiti sva ona kraljevstva, a samo će stajati dovijeka - 
\par 45 kao što si vidio da se kamen  s brijega odvalio a da ga ne dodirnu ruka te smrvio željezo,  mjed, glinu, srebro i zlato. Veliki je Bog saopćio kralju što  se ima dogoditi. Sanja je istinita, a tumačenje joj pouzdano." 
\par 46 Nato kralj Nabukodonozor pade ničice i pokloni se pred  Danielom. Naredi da mu prinesu dar i kad. 
\par 47 I reče kralj Danielu:  "Zaista, vaš je bog Bog nad bogovima i gospodar nad kraljevima, Otkrivatelj tajna, kad si mogao otkriti ovu tajnu." 
\par 48 Kralj  uzvisi Daniela i dariva ga mnogim blistavim darovima. Postavi  ga upraviteljem sve pokrajine babilonske i starješinom svih mudraca  babilonskih. 
\par 49 Daniel zamoli kralja da odredi za upravitelje  pokrajine babilonske Šadraka, Mešaka i Abed Nega, a Daniel ostade  na kraljevu dvoru. 


\chapter{3}

\par 1 Kralj Nabukodonozor odredi da se načini zlatni kip, visok šezdeset lakata i širok šest, i da ga postave u ravnici Duri, u pokrajini babilonskoj. 
\par 2 Kralj Nabukodonozor pozva satrape, namjesnike, upravitelje, savjetnike, rizničare, suce i zakonoznance i sve namjesnike pokrajina da dođu na posvetu kipa što ga podiže kralj Nabukodonozor. 
\par 3 Tada se sakupiše satrapi, namjesnici, upravitelji, savjetnici, rizničari, suci i zakonoznanci i svi namjesnici pokrajinske vlasti na posvetu kipa što ga podiže kralj Nabukodonozor. I stadoše pred kip što podiže Nabukodonozor. 
\par 4 Glasnik proglasi: "O narodi, plemena i jezici, evo što vam se naređuje: 
\par 5 u času kad začujete zvuke roga, frule, citre, sambuke, psaltira, gajda i svakovrsnih drugih glazbala, bacite se na tlo i poklonite se zlatnome kipu što ga podiže kralj Nabukodonozor! 
\par 6 Tko se ne baci na tlo i ne pokloni, bit će smjesta bačen u peć užarenu." 
\par 7 Zato, čim začuše zvuk roga, frule, citre, sambuke, psaltira, gajda i svakovrsnih drugih glazbala, baciše se na tlo svi narodi, plemena i jezici klanjajući se zlatnome kipu što ga podiže kralj Nabukodonozor. 
\par 8 Uto dođoše neki Kaldejci i optužiše Judejce. 
\par 9 Rekoše kralju Nabukodonozoru: "O kralju, živ bio dovijeka! 
\par 10 Ti si, kralju, naredio svakom čovjeku koji začuje zvuke roga, frule, citre, sambuke, psaltira, gajda i svakovrsnih drugih glazbala da se baci na tlo i da se pokloni zlatnome kipu; 
\par 11 a tko se ne baci na tlo i ne pokloni, da bude bačen u peć užarenu. 
\par 12 A evo, ovdje su Judejci koje si postavio za upravitelje pokrajine babilonske: Šadrak, Mešak i Abed Nego. Ti ljudi ne mare za te, o kralju; oni ne štuju tvojih bogova i nisu se poklonili zlatnome kipu što si ga podigao." 
\par 13 Nabukodonozor, bijesan i gnjevan, pozva Šadraka, Mešaka i Abed Nega. Odmah ih dovedoše pred kralja. 
\par 14 A Nabukodonozor im reče: "Je li istina, Šadrače, Mešače i Abed Nego, da vi ne štujete mojih bogova i da se ne klanjate zlatnome kipu što ga podigoh? 
\par 15 Jeste li voljni, čim začujete zvuk roga, frule, citre, sambuke, psaltira, gajda i svakovrsnih drugih glazbala, baciti se na tlo i pokloniti se kipu što ga načinih? Ako li mu se ne poklonite, bit ćete smjesta bačeni u peć užarenu; i koji je taj bog koji bi vas izbavio iz ruke moje?" 
\par 16 Šadrak, Mešak i Abed Nego odgovoriše kralju Nabukodonozoru: "Ne treba da ti odgovorimo na to. 
\par 17 Bog naš, kome služimo, može nas izbaviti iz užarene peći i od ruke tvoje, kralju; on će nas i izbaviti. 
\par 18 No ako toga i ne učini, znaj, o kralju: mi nećemo služiti tvojemu bogu niti ćemo se pokloniti kipu što si ga podigao." 
\par 19 Na te riječi kralj Nabukodonozor uskipje bijesom, a lice mu se iznakazi na Šadraka, Mešaka i Abed Nega. 
\par 20 On naredi da se peć ugrije sedam puta jače no inače i jakim ljudima iz svoje vojske zapovjedi da svežu Šadraka, Mešaka i Abed Nega i bace u peć punu žarkoga ognja. 
\par 21 Svezaše ih, dakle, i u plaštevima, obući i kapama baciše u zažarenu peć. 
\par 22 Kako kraljeva zapovijed bijaše žurna a peć preko mjere užarena, plamen ubi ljude koji su bacali Šadraka, Mešaka i Abed Nega. 
\par 23 A tri čovjeka - Šadrak, Mešak i Abed Nego - padoše svezani u zažarenu peć. 
\par 24 Tada se kralj Nabukodonozor zaprepasti i brzo ustade. Zapita svoje savjetnike: "Nismo li bacili ova tri čovjeka svezana u oganj?" Oni odgovoriše: "Jesmo, kralju!" 
\par 25 On reče: "Ali ja vidim četiri čovjeka, odriješeni šeću po vatri i ništa im se zlo ne događa; četvrti je sličan sinu Božjemu." 
\par 26 Nabukodonozor priđe vratima užarene peći i viknu: "Šadrače, Mešače i Abed Nego, sluge Boga Višnjega, iziđite i dođite ovamo!" Tada iziđoše iz ognja Šadrak, Mešak i Abed Nego. 
\par 27 Sakupiše se satrapi, starješine, upravitelji i kraljevi savjetnici da vide te ljude: oganj ne bijaše naudio njihovu tijelu, kosa im na glavi neopaljena, plaštevi im neoštećeni, nikakav se zadah ognja ne bijaše uhvatio njih. 
\par 28 Nabukodonozor viknu: "Blagoslovljen bio Bog Šadrakov, Mešakov i Abed Negov, koji je poslao svog anđela i izbavio svoje sluge, one koji se uzdahu u njega te se ne pokoriše kraljevoj naredbi, već radije predadoše svoje tijelo ognju negoli da štuju ili se klanjaju drugome osim svome Bogu! 
\par 29 Naređujem dakle: O narodi, plemena i jezici, svatko između vas tko bi pogrdio Boga Šadrakova, Mešakova i Abed Negova neka bude raskomadan, a njegova kuća pretvorena u smetlište, jer nema boga koji bi mogao izbaviti kao ovaj." 
\par 30 Tada kralj uzvisi Šadraka, Mešaka i Abed Nega na visoke položaje u pokrajini babilonskoj. [31]  Nabukodonozor, kralj, svim plemenima, narodima i jezicima po svoj zemlji: Obilovali mirom! [32]  Svidjelo mi se obznaniti vam znakove i čudesa što ih na meni učini Bog Svevišnji. [33]  Znakovi njegovi kako su veliki! Čudesa njegova kako silna! Kraljevstvo njegovo - kraljevstvo vječno! Vlast njegova za sva pokoljenja! 


\chapter{4}

\par 1 Ja, Nabukodonozor, življah mirno u svojoj kući i sretno u svojoj palači, 
\par 2 kad vidjeh sanju koja me uplašila. Utvare i viđenja što su mi se na mom ležaju vrzla po glavi uznemiriše me. 
\par 3 I naredih: neka mi pozovu sve mudrace babilonske da mi kažu što sanja znači. 
\par 4 Dođoše gataoci, čarobnici, zvjezdari i tumači znakova: ja im rekoh svoju sanju, a oni mi ne znadoše reći njezino značenje. 
\par 5 Tada dođe preda me Daniel, koji je nazvan Baltazar prema imenu moga boga, i u komu prebiva duh Boga Svetoga. Ja mu pripovjedih svoju sanju: 
\par 6 "Baltazare, starješino gatalaca, znam da u tebi prebiva duh Boga Svetoga i da ti nijedna tajna nije preteška: evo sanje što je imah: daj mi njezino značenje. 
\par 7 Evo viđenja što mi se na postelji vrzlo po glavi: Pogledam, kad evo jedno stablo usred zemlje vrlo veliko. 
\par 8 Stablo poraste, postade snažno, visina mu doseže nebo, vidjelo se s krajeva zemlje. 
\par 9 Krošnja mu bijaše lijepa, plodovi obilni; na njemu je bilo hrane za sve, u njegovoj sjeni počivaše zvijerje poljsko, na njegovim se granama gnijezdile ptice nebeske i svako se tijelo hranilo od njega. 
\par 10 Ja promatrah viđenja što su mi se na mojoj postelji vrzla po glavi kad, evo, Stražar, Svetac, silazi s neba, 
\par 11 silnim glasom viče: 'Posijecite stablo, okrešite mu grane, počupajte mu lišće, pobacajte plodove! Neka se životinje razbjegnu ispod njega i ptice s grana njegovih! 
\par 12 U zemlji ostavite panj i korijenje u gvozdenim i mjedenim okovima, u travi poljskoj! Neka ga pere rosa nebeska, i travu zemaljsku neka dijeli sa zvijerjem poljskim! 
\par 13 Neka mu se promijeni srce čovječje, srce životinjsko nek' mu se dade! Sedam vremena neka prođe nad njim! 
\par 14 Tako su presudili Stražari, tako su odlučili Sveci, da sve živo upozna kako Svevišnji ima vlast nad kraljevstvom ljudskim: on ga daje kome hoće i postavlja nad njim najnižega od ljudi!' 
\par 15 Ovo je sanja što je vidjeh ja, kralj Nabukodonozor. A ti, Baltazare, reci mi njezino značenje, jer mi nijedan od mudraca moga kraljevstva to ne može reći; ti možeš, jer u tebi je duh Boga Svetoga." 
\par 16 Tada se Daniel, nazvan Baltazar, načas smete i prestraši u svojim mislima. Kralj reče: "Baltazare, ne daj se zbuniti ovom sanjom i njezinim značenjem!" Baltazar odgovori: "Gospodaru moj, ova sanja neka bude tvojim dušmanima i njezino značenje tvojim mrziteljima! 
\par 17 Stablo koje si vidio, veliko i snažno, koje seže sve do neba i vidi se po svoj zemlji, 
\par 18 krošnje lijepe i plodova obilnih na kojem bijaše hrane za sve i pod kojim počiva zvijerje poljsko, a na njegovim se granama gnijezde ptice nebeske: 
\par 19 to si ti, o kralju, koji si velik i moćan, veličina ti se povećala i dosegla do neba, a tvoja vlast do krajeva zemlje. 
\par 20 A što je vidio kralj kako Stražar, Svetac, silazi s neba te govori: 'Posijecite stablo, raskomadajte ga, no njegov panj i korijenje ostavite u zemlji, u gvozdenim i mjedenim okovima, u travi poljskoj, neka ga pere rosa nebeska i dio neka mu bude sa zvijerjem poljskim dok ne prođe sedam vremena nad njim' - 
\par 21 ovo je značenje, o kralju, odluka Svevišnjega što će se ispuniti na mom gospodaru kralju: 
\par 22 Izagnat će te iz društva ljudi i sa životinjama ćeš poljskim boraviti; hranit ćeš se travom kao goveda, tebe će prati rosa nebeska; sedam će vremena proći nad tobom dok ne upoznaš da Svevišnji ima vlast nad kraljevstvom ljudskim i da ga daje kome on hoće. 
\par 23 A što se reklo 'Ostavite panj i korijenje stabla' - tvoje će se kraljevstvo obnoviti čim spoznaš da Nebesa imaju svu vlast. 
\par 24 Zato, kralju, neka ti bude mio moj savjet: iskupi svoje grijehe pravednim djelima i svoja bezakonja milosrđem prema siromasima, da bi ti potrajala sreća." 
\par 25 Sve se to dogodi kralju Nabukodonozoru. 
\par 26 Dvanaest mjeseci kasnije, šetajući babilonskim kraljevskim dvorom, 
\par 27 kralj govoraše: "Nije li to taj veliki Babilon što ga ja sagradih da mi bude kraljevskom prijestolnicom - snagom svoje moći, na slavu svoga veličanstva?" 
\par 28 Još bijahu te riječi u ustima njegovim kad s neba dođe glas: "Tebi se objavljuje, kralju Nabukodonozore! Kraljevstvo ti se oduzelo; 
\par 29 bit ćeš izagnan iz društva ljudi, sa životinjama ćeš poljskim boraviti; hranit ćeš se travom kao goveda, i sedam će vremena proći nad tobom dok ne spoznaš da Svevišnji ima vlast nad kraljevstvom ljudskim, i da ga on daje kome hoće." 
\par 30 I smjesta se riječ izvrši na Nabukodonozoru: bi izagnan iz društva ljudi, jeđaše travu kao goveda, prala ga rosa nebeska; vlasi mu narastoše poput orlova perja, a njegovi nokti kao ptičje pandže. 
\par 31 "Pošto se navršiše određeni dani, ja, Nabukodonozor, podigoh oči prema nebu, razum mi se vrati, tada blagoslovih Svevišnjega hvaleći i uzvisujući onoga koji živi dovijeka: njegovo je kraljevstvo - kraljevstvo vječno, njegova je vlast za sva pokoljenja. 
\par 32 Stanovnici zemlje - upravo kao da ih i nema: po svojoj volji postupa on s vojskom nebeskom i sa žiteljima zemaljskim. Nitko ne može zaustaviti njegovu ruku ili mu kazati: 'Što to radiš?' 
\par 33 U isti čas razum mi se vrati, i na slavu moje kraljevske časti vrati mi se veličanstvo i sjaj; moji me savjetnici i velikaši potražiše, bih uspostavljen u kraljevsku čast i moja veličina još poraste. 
\par 34 Sada ja, Nabukodonozor, hvalim, uzvisujem i slavim Kralja nebeskoga, čija su sva djela istina, svi putovi pravda i koji može poniziti one koji hode u oholosti." 
\par 35 
\par 36 
\par 37 


\chapter{5}

\par 1 Kralj Baltazar priredi veliku gozbu tisući svojih velikaša;  s njima je pio vino. 
\par 2 Opijen vinom, Baltazar naredi da se donese  zlatno i srebrno suđe koje njegov otac Nabukodonozor bijaše oteo  iz jeruzalemskog Svetišta, pa da iz njega pije kralj, njegovi  velikaši, njegove žene i suložnice. 
\par 3 Donesoše dakle zlatno  i srebrno suđe oteto iz Božjega doma u Jeruzalemu i stadoše piti  iz njega kralj i njegovi velikaši, njegove žene i suložnice. 
\par 4 Pili su vino i slavili svoje bogove od zlata i srebra, mjedi  i željeza, drva i kamena. 
\par 5 Iznenada se pojaviše prsti čovječje ruke koji stadoše  pisati, nasuprot velikom svijećnjaku, po okrečenu zidu kraljevskog  dvora, i kralj vidje dlan ruke koja pisaše. 
\par 6 Kralj problijedje, misli ga uznemiriše, zglobovi njegovih kukova popustiše i koljena  mu stadoše udarati jedno o drugo. 
\par 7 Glasno dozva čarobnike,  zvjezdare i gataoce. I reče kralj mudracima babilonskim: "Tko  pročita ovo pismo i otkrije mi njegov smisao, bit će obučen u  grimiz, oko vrata nosit će zlatan lanac i bit će treći u kraljevstvu." 
\par 8 Pristupe svi mudraci kraljevi, ali ne mogoše pročitati pismo  niti mu otkriti značenje. 
\par 9 Kralj se Baltazar zbog toga silno  uplaši, problijedje, a njegovi velikaši ostadoše zbunjeni. 
\par 10 Kraljica, čuvši riječi kralja i velikaša, uđe u gozbenu  dvoranu i reče: "Kralju, živ bio dovijeka! Neka se tvoje misli  ne uznemiruju i tvoje lice neka ne blijedi! 
\par 11 Ima u tvome kraljevstvu  čovjek u kome prebiva duh Boga Svetoga. Još za vremena tvoga  oca nađe se u njemu svjetlo, razum i mudrost slična mudrosti  bogova. I zato ga kralj Nabukodonozor, otac tvoj, imenova starješinom  čarobnika, gatalaca, zvjezdara i mudraca. 
\par 12 Budući da se u  tom Danielu - koga kralj bijaše nazvao Baltazarom - našao duh  izvanredan, znanje, bistrina, vještina da tumači sanje, da rješava  zagonetke i da razrješuje teškoće, pozovi stoga Daniela i on  će ti kazati značenje." 
\par 13 Dovedoše Daniela pred kralja, a kralj ga upita: "Jesi  li ti Daniel, jedan od izgnanika judejskih koje dovede iz Judeje  kralj moj otac? 
\par 14 Čujem da duh Božji prebiva na tebi i da je  u tebi svjetlo, razum i mudrost izvanredna. 
\par 15 Dovedoše mi mudrace  i čarobnike da pročitaju ovo pismo i da mi reknu njegovo značenje, ali oni nisu kadri otkriti mi njegov smisao. 
\par 16 A čujem da  si ti kadar dati tumačenja i da razrješuješ teškoće. Ako si dakle  kadar pročitati ovo pismo i reći mi njegovo značenje, bit ćeš  odjeven u grimiz i nosit ćeš zlatan lanac oko vrata i bit ćeš  treći u kraljevstvu." 
\par 17 Daniel prihvati riječ i odgovori kralju: "Tvoji darovi  neka ti ostanu, i svoje poklone daj drugima! A ja ću pročitati  ovo pismo kralju i kazat ću mu njegovo značenje. 
\par 18 O kralju, Bog je Svevišnji dao kraljevstvo, veličinu, veličanstvo i slavu  Nabukodonozoru, ocu tvome. 
\par 19 Zbog veličine koju mu bijaše dao  drhtahu od straha pred njim narodi, plemena i jezici: on ubijaše  po svojoj volji, ostavljaše na životu koga je htio, uzdizaše  koga je htio, ponizivaše koga je htio. 
\par 20 No kad mu se srce  uzdiglo i duh uzobijestio do drskosti, tada bi oboren sa svoga  kraljevskog prijestolja i slava mu bijaše oduzeta. 
\par 21 Bi izagnan  iz ljudskog društva i srce mu posta slično životinjskom: prebivaše  s divljim magarcima; poput goveda jeđaše travu; nebeska je rosa  prala njegovo tijelo, dok ne spozna da Svevišnji Bog ima vlast  nad kraljevstvom ljudskim i stavlja mu na čelo onoga koga on  hoće. 
\par 22 No ti, Baltazare, sine njegov, nisi ponizio srce svoje, iako si znao sve ovo: 
\par 23 ti si se podigao protiv Gospoda Nebeskoga, dao si da ti donesu suđe iz njegova Doma i pili ste vino iz  njega ti, tvoji velikaši, tvoje žene i tvoje suložnice, hvaleći  bogove od zlata i srebra, od mjedi i željeza, od drva i kamena, koji ne vide, ne čuju niti razumiju, a nisi dao slavu Bogu koji  u svojoj ruci drži dah tvoj i sve tvoje putove. 
\par 24 I zato on  posla ovu ruku koja napisa ovo pismo." 
\par 25 "A evo što je napisano: Mene, Mene, Tekel, Parsin. 
\par 26 A  te riječi znače: Mene: izmjerio je Bog tvoje kraljevstvo i učinio  mu kraj; 
\par 27 Tekel: bio si vagnut na tezulji i nađen si prelagan; 
\par 28 Parsin: razdijeljeno je tvoje kraljevstvo i predano Medijcima  i Perzijancima." 
\par 29 Tada Baltazar naredi da Daniela obuku u grimiz, da mu  oko vrata objese zlatan lanac i da ga proglase trećim u kraljevstvu. 
\par 30 Iste te noći kaldejski kralj Baltazar bi ubijen. 
\par 31 (6:1) A Darije Medijac preuze kraljevstvo, star već šezdeset i dvije  godine. 


\chapter{6}

\par 1 Svidjelo se Dariju da postavi nad svojim kraljevstvom  stotinu i dva(6:2) deset satrapa da budu nad svim kraljevstvom. 
\par 2 (6:3) Njima  na čelo stavi tri pročelnika - Daniel bijaše jedan od njih -  kojima će satrapi polagati račun da se ne bi dosađivalo kralju. 
\par 3 (6:4) Daniel se toliko isticaše svojim izvanrednim duhom iznad pročelnika  i satrapa te kralj mišljaše da ga postavi nad svim kraljevstvom. 
\par 4 (6:5) Tada pročelnici i satrapi stadoše tražiti povod, štogod  oko državne uprave, zbog čega bi mogli optužiti Daniela; ali  ne mogoše na njemu naći ništa takvo, ništa zbog čega bi ga prekorili, jer bijaše vjeran, na njemu ni nemara ni ogrešenja. 
\par 5 (6:6) Ti ljudi  rekoše tada: "Nećemo naći nikakva povoda protiv Daniela, osim  da nađemo nešto protiv njega u zakonu njegova Boga." 
\par 6 (6:7) Tada pročelnici i satrapi navališe na kralja te mu rekoše:  "O kralju Darije, živ bio dovijeka! 
\par 7 (6:8) Svi pročelnici kraljevstva, predstojnici i satrapi, savjetnici i namjesnici složiše se u  tome da bi trebalo da kralj izda naredbu i zabranu: svaki onaj  koji bi u roku od trideset dana upravio molbu bilo na kojega  boga ili čovjeka, osim na tebe, o kralju, bit će bačen u lavsku  jamu. 
\par 8 (6:9) O kralju, potvrdi tu zabranu i potpiši naredbu da bude  neopoziva prema nepromjenljivom medijsko-perzijskom zakonu!" 
\par 9 (6:10) Nato kralj Darije potpisa pismo i zabranu. 
\par 10 (6:11) Saznavši Daniel da je spis potpisan, otiđe u svoju kuću.  Prozori gornje sobe bijahu otvoreni prema Jeruzalemu. Tu je on  tri puta na dan padao na koljena blagoslivljajući, moleći i hvaleći  Boga, kako je uvijek činio. 
\par 11 (6:12) Oni ljudi nahrupiše i nađoše  Daniela gdje moli i zaziva svoga Boga. 
\par 12 (6:13) Tada odoše i pred  kraljem se pozvaše na kraljevsku zabranu: "Zar ti nisi potpisao  zabranu prema kojoj će svaki onaj koji bi u vremenu od trideset  dana upravio molbu na nekoga boga ili čovjeka, osim na tebe,  o kralju, biti bačen u lavsku jamu?" Kralj odgovori: "Tako je  odlučeno po nepromjenljivom medijsko-perzijskom zakonu." 
\par 13 (6:14) Tada  rekoše kralju: "Daniel, onaj od izgnanika judejskih, ne mari  za tebe, o kralju, ni za tvoju zabranu koju si potpisao: tri  puta na dan obavlja svoju molitvu." 
\par 14 (6:15) Čuvši te riječi, kralj se vrlo ražalosti i odluči spasiti  Daniela. Sve do sunčeva zalaza nastojaše da ga spasi. 
\par 15 (6:16) Ali  oni ljudi navališe na kralja govoreći: "Znaj, o kralju, da prema  medijsko-perzijskom zakonu nijedna kraljevska zabrana ili odluka  ne može biti opozvana!" 
\par 16 (6:17) Tada kralj naredi da dovedu Daniela i da ga bace u lavsku  jamu. Kralj reče Danielu: "Bog tvoj, kome tako postojano služiš, neka te izbavi." 
\par 17 (6:18) Donesoše kamen i staviše ga jami na otvor.  Kralj ga zapečati prstenom svojim i prstenom svojih velikaša, da se ništa ne mijenja za Daniela. 
\par 18 (6:19) Kralj se vrati u svoj dvor i provede noć ne okusivši  jela i ne dopustivši da mu dovedu suložnice. Nije mogao usnuti. 
\par 19 (6:20) Kralj ustade u ranu zoru, kad se danilo, i pođe brzo k lavskoj  jami. 
\par 20 (6:21) Kad se primače blizu, viknu žalosnim glasom Danielu:  "Daniele, slugo Boga živoga, je li te Bog, kome postojano služiš, mogao izbaviti od lavova?" 
\par 21 (6:22) Daniel odgovori: "O kralju, živ  bio dovijeka! 
\par 22 (6:23) Moj je Bog poslao svog Anđela; zatvorio je  ralje lavovima te mi ne naudiše, jer sam nedužan pred njim. Pa  i pred tobom, o kralju, ja sam bez krivice." 
\par 23 (6:24) Kralj se vrlo  obradova i naredi da Daniela izvade iz jame. Izvadiše Daniela  iz jame neozlijeđena, jer se bijaše uzdao u svoga Boga. 
\par 24 (6:25) Kralj  zapovjedi da dovedu one ljude koji bijahu optužili Daniela i  da ih bace u lavsku jamu - njih, njihove žene i njihovu djecu:  i prije nego dodirnuše tlo, lavovi ih zgrabiše i smrviše im kosti. 
\par 25 (6:26) Nato kralj Darije napisa svim plemenima, narodima i jezicima  što stanuju po svoj zemlji: "Obilovali mirom! 
\par 26 (6:27) Evo naredbe  koju donosim: u svemu mojem kraljevstvu neka se ljudi boje i  neka dršću pred Bogom Danielovim: On je Bog živi, on ostaje dovijeka! Njegovo kraljevstvo neće propasti, njegovoj vlasti nema kraja! 
\par 27 (6:28) On izbavlja i spasava, čini znake i čudesa na nebesima i na zemlji! On je spasio Daniela iz šapa lavljih!" 
\par 28 (6:29) Daniel bijaše sretan za vladanja Darija i za vladanja Kira  Perzijanca. 


\chapter{7}

\par 1 Prve godine Baltazara, kralja babilonskoga, usni Daniel san:  utvare mu se na postelji vrzle glavom. Sažeto zapisa što je usnio. 
\par 2 Kazivaše ovako: Noću u viđenju pogledah, kad eno: četiri vjetra nebeska uzbibaše  veliko more. 
\par 3 Četiri goleme nemani iziđoše iz mora, svaka drukčija.  Prva bijaše kao lav, a krila joj orlovska. 
\par 4 Dok je promatrah, krila joj se iščupaše, diže se ona sa zemlje i uspravi na noge  kao čovjek, i bijaše joj dano srce čovječje. 
\par 5 Kad eno druga  neman: gle, sasvim drukčija: kao medvjed, s jedne strane uspravljena, tri joj rebra u raljama, među zubima. I bijaše joj rečeno: "Ustani, nažderi se mesa!" 
\par 6 Gledah dalje, i evo: treća neman kao leopard, na leđima joj četiri ptičja krila: imaše četiri glave, i dana  joj je moć. 
\par 7 Zatim, u noćnim viđenjima, pogledah, kad eno:  četvrta neman, strahovita, užasna, izvanredno snažna: imaše velike  gvozdene zube; ona žderaše, mrvljaše, a što preostade, gazila  je nogama. Razlikovala se od prijašnjih nemani i imaše deset  rogova. 
\par 8 Promatrah joj rogove, i gle: među njima poraste jedan  mali rog; i pred tim se rogom iščupaše tri prijašnja roga. I  gle, na tome rogu oči kao oči čovječje i usta koja govorahu velike  hule. 
\par 9 Gledao sam: Prijestolja bjehu postavljena i Pradavni sjede. Odijelo mu bijelo poput snijega; vlasi na glavi kao čista vuna. Njegovo prijestolje kao plamenovi ognjeni i točkovi kao žarki oganj. 
\par 10 Rijeka ognjena tekla, izvirala ispred njega. Tisuću tisuća služahu njemu, mirijade stajahu pred njim. Sud sjede, knjige se otvoriše. 
\par 11 Ja gledah tada, zbog buke velikih hula što ih govoraše  rog, i dok gledah, neman bi ubijena, njezino tijelo raskomadano  i predano ognju. 
\par 12 Ostalim nemanima vlast bi oduzeta, ali im  duljina života bi na jedno vrijeme i rok. 
\par 13 Gledah u noćnim viđenjima i gle, na oblacima nebeskim dolazi kao Sin čovječji. On se približi Pradavnome i dovedu ga k njemu. 
\par 14 Njemu bi predana vlast, čast i kraljevstvo, da mu služe svi narodi, plemena i jezici. Vlast njegova vlast je vječna i nikada neće proći, kraljevstvo njegovo neće propasti. 
\par 15 Meni, Danielu, smete se zbog toga sav duh, viđenja mi  se vrzoše glavom, svega me prestraviše. 
\par 16 Pristupih jednome  od onih koji stajahu ondje i zamolih ga da mi rekne istinu o  svemu tome. On mi odgovori i kaza mi značenje: 
\par 17 "One četiri goleme nemani jesu četiri kralja koji će  se dići na zemlji. 
\par 18 Ali će od njih kraljevstvo preuzeti Sveci  Svevišnjega i oni će ga posjedovati za vijeke vjekova." 
\par 19 Zaželjeh tada saznati istinu o četvrtoj nemani, onoj  koja se razlikovaše od svih drugih, bila izvanredno strašna,  imala gvozdene zube i mjedene pandže i koja je žderala i mrvila  i nogama gazila što preostajaše; 
\par 20 i o deset rogova što bijahu  na njezinoj glavi, i o drugom rogu koji poraste dok tri prva  otpadoše - o rogu koji imaše oči i usta što govorahu velike hule  i koji bijaše veći nego drugi rogovi. 
\par 21 I gledao sam kako ovaj  rog ratuje protiv Svetaca te ih nadvladava, 
\par 22 dok ne dođe Pradavni, koji dosudi pravdu Svecima Svevišnjega, i dok ne dođe vrijeme  kad Sveci zaposjedoše kraljevstvo. 
\par 23 On reče: "Četvrta neman bit će četvrto kraljevstvo na zemlji, različito od svih kraljevstava. Progutat će svu zemlju, zgazit' je i smrviti. 
\par 24 A deset rogova: Od ovoga kraljevstva nastat će deset kraljeva, a iza njih će se podići jedan drugi različit od onih prvih - i oborit će tri kralja. 
\par 25 On će huliti na Svevišnjega, zatirati Svece Svevišnjega; pomišljat će da promijeni blagdane i Zakon, i Sveci će biti predani u njegove ruke na jedno vrijeme i dva vremena i polovinu vremena. 
\par 26 Tada će sjesti Sud, vlast mu oduzeti, razoriti, sasvim uništiti. 
\par 27 A kraljevstvo, vlast i veličanstvo pod svim nebesima dat će se puku Svetaca Svevišnjega. Kraljevstvo njegovo kraljevstvo je vječno, i sve vlasti služit će mu i pokoravati se njemu." 
\par 28 Ovdje se završava izvještaj. Ja, Daniel, bijah vrlo potresen  u svojim mislima i lice mi problijedje, ali sve ovo sačuvah u  srcu svojemu. 


\chapter{8}

\par 1 Treće godine kralja Baltazara ukaza se meni, Danielu, viđenje  poslije onoga koje mi se ukazalo prije. 
\par 2 Gledah viđenje, i  dok gledah, nađoh se u Šušanu, čvrstu gradu u pokrajini Elamu;  i u viđenju se vidjeh na rijeci Ulaju. 
\par 3 Podigoh oči, i gle:  ovan stajaše kraj rijeke. Imaše dva roga: oba roga visoka, no  jedan viši nego drugi, a onaj viši narastao poslije. 
\par 4 Gledah  kako ovan bode na zapad, na sjever i jug. Nijedna mu se životinja  ne mogaše oprijeti, ništa mu ne mogaše izbjeći. Činio je što  je htio, osilio se. 
\par 5 Dok sam promatrao, gle: jarac dolazi sa zapada povrh sve  zemlje, ne dodirujući tla; jarac imaše silan rog među očima. 
\par 6 Približi se dvorogom ovnu kojega bijah vidio gdje stoji kraj  rijeke i potrča na njega u svoj žestini svoje snage. 
\par 7 Vidjeh  kako se približi ovnu: bijesno udari na ovna i slomi mu oba roga, a ovan nije imao snage da mu se opre; obori ga jarac na zemlju  i stade ga nogama gaziti; nikoga ne bijaše da spasi ovna. 
\par 8 Jarac  osili veoma, ali kad osili, veliki se rog slomi, a na njegovu  mjestu izrastoše četiri velika roga prema četiri vjetra nebeska. 
\par 9 Iz jednoga od njih izbi malen rog, ali taj silno poraste  prema jugu i istoku, prema Divoti. 
\par 10 On poraste sve do Nebeske  vojske, obori na zemlju neke iz Vojske i od zvijezda pa ih zgazi  nogama. 
\par 11 Poraste sve do Zapovjednika Vojske, oduze mu svagdašnju  žrtvu i razori mu njegovo Sveto mjesto. 
\par 12 Vojska se digla na  žrtvu svagdašnju zbog opačine, na zemlju oborila istinu i uspje  u svemu što činjaše. 
\par 13 Tada čuh gdje jedan Svetac govori, a drugi Svetac upita  onoga koji govoraše: "Dokle će trajati ovo viđenje o svagdašnjoj  žrtvi i o opačini što pustoši i gazi Svetište i Vojsku?" 
\par 14 Odgovori:  "Još dvije tisuće i tri stotine večeri i jutara; tada će Svetište  biti očišćeno." 
\par 15 Kad sam ja, Daniel, gledajući ovo viđenje, tražio da  ga razumijem, gle, preda me stade kao neki čovjek. 
\par 16 Začuh  glas čovječji gdje viče preko Ulaja: "Gabriele, objasni mu to  viđenje!" 
\par 17 On pođe onamo gdje stajah i kad mi se približi, strah me obuze i padoh na lice. On mi reče: "Sine čovječji,  razumij: jer ovo je viđenje za vrijeme posljednje." 
\par 18 On još  govoraše, a ja se onesvijestih, padoh na zemlju. On me dotače  i uspravi na mom mjestu. 
\par 19 I reče: "Evo, kazat ću ti što će  doći na kraju gnjeva, najavljeni svršetak. 
\par 20 Ovan što si ga  vidio - njegova dva roga - to su kraljevi Medije i Perzije. 
\par 21 Rutavi  jarac jest kralj Grčke; veliki rog među njegovim očima jest prvi  kralj; 
\par 22 slomljeni rog i četiri roga što izbiše na njegovu  mjestu, to su četiri kraljevstva što će izići iz njegova naroda, ali neće imati njegovu moć. 
\par 23 I potkraj njihova kraljevanja, kad bezakonici navrše mjeru, ustat će kralj, drzak i lukav. 
\par 24 Njegova će moć porasti, ali ne svojom snagom; nesmiljeno će pustošiti, uspijevat će u svojim pothvatima, zatirat' junake i narod Svetaca. 
\par 25 Zbog njegove lukavosti prijevara će uspijevati u njegovoj ruci. On će se uznijeti u svome srcu, iz čista mira upropastit će mnoge. Suprotstavit će se Knezu nad knezovima, ali će - ne rukom - biti skršen. 
\par 26 Viđenje o večerima i jutrima o kojem je bilo govora istinito je; no ti ga zapečati, jer je za daleke dane." 
\par 27 Tada ja, Daniel, obnemogoh i bijah bolestan više dana.  Zatim ustadoh da vršim kraljevske poslove. Bijah smeten zbog  viđenja, no nitko to nije dokučio. 


\chapter{9}

\par 1 Prve godine Darija, sina Artakserksova, iz roda Medijaca, koji  vladaše kraljevstvom kaldejskim, 
\par 2 prve dakle godine njegova  kraljevanja, ja, Daniel, istraživah u Pismima broj godina koje  se - prema riječi koju Jahve uputi proroku Jeremiji - imaju ispuniti  nad ruševinama Jeruzalema: sedamdeset godina. 
\par 3 Ja obratih svoje  lice prema Gospodinu Bogu nastojeći moliti se i zaklinjati u  postu, kostrijeti i pepelu. 
\par 4 Ja se moljah Jahvi, Bogu svome, priznavajući: "Ah, Gospodine moj, Bože veliki i strahoviti, koji čuvaš  Savez i naklonost onima koji tebe ljube i čuvaju zapovijedi tvoje! 
\par 5 Mi sagriješismo, mi bezakonje počinismo, zlo učinismo, odmetnusmo  se i udaljismo od zapovijedi i naredaba tvojih. 
\par 6 Nismo slušali  sluge tvoje, proroke koji govorahu u tvoje ime našim kraljevima, našim knezovima, našim očevima, svemu puku zemlje. 
\par 7 U tebe  je, Gospodine, pravednost, a u nas stid na obrazu, kao u ovaj  dan, u nas Judejaca, Jeruzalemaca, svega Izraela, blizu i daleko, u svim zemljama kuda si ih rastjerao zbog nevjernosti kojom  ti se iznevjeriše. 
\par 8 Jahve, stid na obraz nama, našim kraljevima, našim knezovima, našim očevima, jer sagriješismo protiv tebe! 
\par 9 U Gospoda je Boga našega smilovanje i oproštenje jer smo se  odmetnuli od njega 
\par 10 i nismo slušali glas Jahve, Boga našega, da slijedimo njegove zakone što nam ih dade po svojim slugama, prorocima. 
\par 11 Sav je Izrael prestupio Zakon tvoj, odmetnuo se ne slušajući  tvoj glas. Zato se na nas izlila kletva i prokletstvo, kako je  zapisano u Zakonu Mojsija, sluge Božjega - jer sagriješismo protiv  Njega. 
\par 12 Izvršio je prijetnje kojima je zaprijetio nama i sucima  koji su nam sudili: svalio je na nas tešku nesreću te se ne dogodi  pod nebom što se dogodi u Jeruzalemu. 
\par 13 Sva ova nesreća, kao  što je zapisano u Zakonu Mojsijevu, došla je na nas, a mi nismo  umilostivili lice Jahve, Boga svojega: nismo se obratili od svojih  bezakonja pa da prionemo uz istinu tvoju. 
\par 14 Jahve je bdio nad  nesrećom, on je dovede na nas. Jer je pravedan Jahve, Bog naš, u svim djelima koja učini, a mi nismo slušali glas njegov. 
\par 15 A sada, Gospode, Bože naš, koji si moćnom svojom rukom  izveo narod svoj iz zemlje egipatske - i time sebi stekao ime  koje traje do danas: mi sagriješismo, mi zlo učinismo. 
\par 16 Gospode, po svoj pravednosti svojoj odvrati svoj gnjev i svoju jarost  od Jeruzalema, grada svojega, Svete gore svoje, jer zbog naših  grijeha i zlodjela naših otaca Jeruzalem i tvoj narod ruglo su  svima koji nas okružuju." 
\par 17 "A sada poslušaj, o Bože naš, molitvu sluge svoga i usrdne  molbe njegove. Neka tvoje lice zasja nad svetištem tvojim opustošenim  - zbog tebe, Gospode! 
\par 18 Prikloni uho svoje, Bože moj, i slušaj!  Otvori oči te pogledaj našu pustoš i grad koji se tvojim zove  imenom! Jer mi te ne molimo zbog svoje pravednosti, već zbog  velikih smilovanja tvojih. 
\par 19 Gospode, čuj! Gospode, oprosti!  Gospode, poslušaj i čini! Ne oklijevaj - zbog sebe, Bože moj, jer se tvojim imenom zove grad tvoj i narod tvoj!" 
\par 20 Ja sam još govorio, moleći se i priznavajući grijehe  svoje i grijehe svog naroda Izraela i usrdno zaklinjući Jahvu, svoga Boga, za Svetu goru Boga svoga. 
\par 21 Dok sam dakle ja još  govorio moleći se, onaj čovjek Gabriel, koga vidjeh na početku  viđenja, doletje u brzu letu, dotače me se u vrijeme večernjeg  prinosa 
\par 22 i pouči me: "Daniele, evo me: dođoh da te poučim. 
\par 23 Od početka tvoje molitve izišla je riječ, i ja dođoh da ti  je navijestim. Ti si miljenik. Pazi dobro na riječ, razumij viđenje." 
\par 24 "Sedamdeset je sedmica određeno tvom narodu i tvom svetom gradu da se dokrajči opačina, da se stavi pečat grijehu, da se zadovolji za bezakonje, da se uvede vječna pravednost, da se stavi pečat viđenju i prorocima, da se pomaže Sveti nad svetima. 
\par 25 Znaj i razumij: Od časa kad izađe riječ 'Neka se vrate i neka opet sagrade Jeruzalem' pa do Kneza Pomazanika: sedam sedmica, a onda šezdeset i dvije sedmice, i bit će opet sagrađeni trg i opkop, i to u teško vrijeme. 
\par 26 A poslije šezdeset i dvije sedmice bit će Pomazanik pogubljen, ali ne za sebe. Narod jednog kneza koji će doći razorit će Grad i Svetište: svršetak im je u propasti, a do svršetka rat i određena pustošenja. 
\par 27 I sklopit će savez s mnogima za jednu sedmicu: a u polovici sedmice prestat će žrtva i prinos: na vrhu Hrama bit će grozota pustoši sve do svršetka, dok se određeno pustošenje ne obori na pustošnika." 


\chapter{10}

\par 1 Treće godine Kira, kralja perzijskoga, Danielu, prozvanome  Baltazar, bi objavljena riječ - riječ istinita: velik rat. On  je nastojao razumjeti riječ, i razumijevanje bi mu dano u viđenju. 
\par 2 U te dane ja, Daniel, žalovao sam tri sedmice: 
\par 3 nisam  jeo tečnih jela; meso ni vino nije ulazilo u moja usta i nisam  se mazao uljem dok ne prođoše te tri sedmice. 
\par 4 Dvadeset i četvrtoga  dana prvog mjeseca bijah na obali velike rijeke Tigrisa; 
\par 5 podigoh  oči da vidim, i gle: Čovjek odjeven u lanene haljine, oko pasa mu pojas od zlata ofirskoga, 
\par 6 tijelo mu poput krizolita, lice kao munja, oči kao baklje ognjene, ruke i noge poput mjedi uglađene, zvuk riječi njegovih kao žamor mnoštva. 
\par 7 Jedini ja, Daniel, gledah ovo viđenje, ljudi koji bijahu  sa mnom ne vidješe ga, ali ih spopade silan strah te pobjegoše  da se sakriju. 
\par 8 Ostadoh sam gledajući to veliko viđenje; onemoćah, lice mi problijedje, iznakazi se, snaga me ostavi. 
\par 9 Začuh glas njegovih riječi, i kad razabrah glas, onesvijestih  se i padoh licem na zemlju. 
\par 10 I gle: ruka me dotače i pomože  mi da se uprem na koljena i na dlanove. 
\par 11 On mi reče: "Daniele, miljeniče, pripazi na riječi koje ću ti kazati! Ustani, jer  ja sam evo k tebi poslan." To reče, a ja ustadoh dršćući. 
\par 12 I kaza mi: "Ne boj se, Daniele, jer od prvoga dana kad  si odlučio da se poniziš pred svojim Bogom da bi razumio, tvoje  su riječi uslišane i ja sam došao zbog tvojih riječi. 
\par 13 Knez  kraljevstva perzijskoga protivio mi se dvadeset i jedan dan,  ali Mihael, jedan od prvih Knezova, dođe mi u pomoć. Ostavih  ga nasuprot Knezu perzijskome, 
\par 14 a ja dođoh da ti kažem što  će zadesiti tvoj narod na svršetku dana. Jer još će jedno viđenje  biti za one dane." 
\par 15 Pošto mi to reče, ja oborih pogled na zemlju, bez riječi. 
\par 16 I gle: onaj, sličan sinu čovječjem dotače se mojih usana.  Otvorih usta da govorim te rekoh onome koji stajaše preda mnom:  "Gospodaru moj, zbog ovog viđenja obuzeše me tjeskobe i onemoćah. 
\par 17 I kako će sluga Gospodina svoga govoriti s Gospodinom kad  posve onemoćah i dah me ostavi?" 
\par 18 Tada me se opet dotače onaj  što bijaše kao čovjek te me okrijepi. 
\par 19 On reče: "Ne boj se, miljeniče! Mir tebi! Budi jak! Ohrabri se!" I dok mi to govoraše, ja se ohrabrih pa rekoh: "Govori, Gospodine, jer si me ohrabrio!" 
\par 20 Tada će on: "Znaš li zašto sam došao k tebi? Sad ću se  vratiti da se borim protiv Kneza Perzije; a čim svršim, doći  će Knez Grčke. 
\par 21 Ali ću ti prije otkriti što je zapisano u  Knjizi istine. Nema nikoga tko bi se sa mnom protiv njih borio, osim Mihaela, Kneza vašega, 


\chapter{11}

\par 1 Moje potpore i moga okrilja. 
\par 2 A sada ću ti otkriti istinu. Evo: još će tri kralja ustati za Perziju: četvrti će biti  bogatiji od svih ostalih, pa kad se zbog svoga bogatstva osili, sve će podići protiv kraljevstva grčkoga. 
\par 3 Ustat će junački kralj, vladat će silnom moću i činiti  što ga bude volja. 
\par 4 A čim se ustane, njegovo će se kraljevstvo  raspasti i bit će razdijeljeno na četiri vjetra nebeska, ali  ne među njegove potomke; i neće više biti tako moćno kao za njegove  vladavine, jer će njegovo kraljevstvo biti razoreno i predano  drugima, a ne njima. 
\par 5 Kralj će Juga postati moćan; jedan će od njegovih zapovjednika  biti moćniji od njega i zavladat će većom moću nego što je njegova. 
\par 6 Nekoliko godina kasnije oni će se udružiti, a kći kralja  Juga doći će kralju Sjevera da sklope ugovor. Ali ona tim neće  sačuvati snagu svoje mišice i njezino se potomstvo neće održati:  bit će predana ona, i njezina pratnja, i njezino dijete, i njezin  pomagač u tim vremenima. 
\par 7 No jedan će se izdanak njezina korijena podići na njezino  mjesto, navalit će na vojsku, prodrijet će u tvrđavu kralja Sjevera, postupati s njima po miloj volji i pobijediti ih. 
\par 8 Pa i njihove  bogove, njihove kipove i njihovo dragocjeno suđe, srebrno i zlatno, odnijet će kao plijen u Egipat. Nekoliko godina bit će jači  od kralja Sjevera, 
\par 9 koji će onda prodrijeti u kraljevstvo kralja  Juga, odakle će se vratiti u svoju zemlju. 
\par 10 Ali će se onda  njegovi sinovi naoružati, skupit će silnu vojsku, odlučno će  navaliti i poput poplave proći, zatim će se opet zametnuti rat  sve do njegove utvrde. 
\par 11 Tada će se kralj Juga razgnjeviti  i zavojštiti na kralja Sjevera; podići će silnu vojsku i nadvladati  vojsku njegovu. 
\par 12 Mnoštvo će biti uništeno, a on će se zbog  toga uzoholiti; pobit će desetke tisuća, ali se neće održati: 
\par 13 kralj će Sjevera opet dići vojsku veću nego prije, i poslije  nekoliko godina navalit će s velikom, dobro opremljenom vojskom. 
\par 14 U to vrijeme mnogi će se podići protiv kralja Juga; ustat  će i nasilnici iz tvog naroda da se ispuni viđenje, ali će propasti. 
\par 15 Doći će kralj Sjevera: podići će nasipe da zauzme jedan  utvrđeni grad. Mišice Juga neće odoljeti, pa ni izabrane čete  neće imati snage da se odupru. 
\par 16 Onaj će navaliti protiv njega  i učinit će s njime kako mu se prohtije - nitko mu se neće oprijeti:  zaustavit će se u Divoti, uništenje je u njegovim rukama. 
\par 17 Čvrsto  odlučivši da se pošto-poto domogne svega njegova kraljevstva, sklopit će s njim ugovor dajući mu jednu kćer za ženu da ga  upropasti, ali mu neće uspjeti, neće se to zbiti. 
\par 18 Zatim će  se okrenuti prema otocima i mnoge će osvojiti, ali će jedan zapovjednik  dokrajčiti tu sramotu, sramotu mu sramotom vratiti. 
\par 19 Brže će nagnuti prema utvrdama svoje zemlje, ali će posrnuti, pasti, više ga neće biti. 
\par 20 Na njegovo će mjesto doći jedan  koji će u diku kraljevstva poslati poreznika, ali će u kratko  vrijeme poginuti bez gnjeva i boja. 
\par 21 Na njegovo će se mjesto uzdići nitkov kome ne pripada  kraljevska čast. Ali on će iznenada doći i spletkama se domoći  kraljevstva. 
\par 22 Pred njim će biti preplavljene i skršene navalne snage  i sam knez Saveza. 
\par 23 Unatoč sporazumu s njime, izdajnički će  navaliti i svladati ga s malo ljudi. 
\par 24 Iznenada će upasti u  bogate pokrajine i postupat će kako nisu postupali njegovi očevi  ni očevi njegovih otaca, rasipajući među svoje plijen, pljačku  i bogatstvo, smišljat će osnove protiv tvrdih gradova, ali samo  za neko vrijeme. 
\par 25 Pokrenut će, s velikom vojskom, svoju snagu i hrabrost  protiv kralja Juga. Kralj Juga krenut će u rat s mnogom i moćnom  vojskom, ali neće izdržati, jer će se protiv njega skovati spletke. 
\par 26 I oni koji jeđahu za njegovim stolom skršit će ga: njegova  će vojska biti uništena i mnogi će posječeni popadati. 
\par 27 Oba će kralja smišljati zlo; sjedeći za istim stolom, govorit će laži jedan drugome: ali neće uspjeti, jer je svršetak  odložen do određenog vremena. 
\par 28 Vratit će se on u svoju zemlju  s velikim blagom; srcem protiv svetoga Saveza, učinit će svoje  i vratiti se u svoju zemlju. 
\par 29 U određeno vrijeme opet će krenuti protiv Juga, ali sada  neće biti kao prvi put. 
\par 30 Kitimski će brodovi navaliti na njega, i on će se uplašiti. Vratit će se, bjesnjeti protiv svetoga  Saveza i opet će se sporazumjeti s onima koji napustiše sveti  Savez. 
\par 31 Čete će njegove doći i oskvrnuti svetište-tvrđu, dokinut'  svagdašnju žrtvu i ondje postaviti grozotu pustoši. 
\par 32 Svojim  će spletkama navesti na otpad one koji se ogrešuju o Savez, ali  ljudi koji ljube Boga ostat će postojani i vršit će svoje. 
\par 33 Umnici  u narodu poučavat će mnoštvo, ali će ih jedno vrijeme zatirati  mačem i ognjem, izgnanstvom i pljačkanjem. 
\par 34 Dok ih budu zatirali, samo će im nekolicina pomagati, a mnogi će im se pridružiti  prijevarno. 
\par 35 Od umnika neki će pasti, da se prokušaju, probrani, čisti do vremena svršetka, jer još nije došlo određeno vrijeme. 
\par 36 Kralj će raditi što god mu se prohtije, uznoseći i uzdižući  sebe iznad svih bogova: protiv Boga nad bogovima govorit će hule  i uspijevat će dok se gnjev ne navrši - jer ono što je određeno, to će se ispuniti. 
\par 37 Neće mariti za bogove svojih otaca ni  za Miljenika ženÄa niti za kojega drugog boga: samog će sebe  izdizati iznad sviju. 
\par 38 Mjesto njih častit će boga tvrđava, boga koga nisu poznavali njegovi očevi, častiti ga zlatom i  srebrom, dragim kamenjem i drugim dragocjenostima. 
\par 39 Navalit  će na tvrđave gradova pomoću stranog boga: one koji njega priznaju  obasut će počastima i dat će im vlast nad mnoštvom i dijelit  će im zemlju za nagradu. 
\par 40 U vrijeme svršetka kralj će se Juga zaratiti s njime;  kralj će Sjevera navaliti na nj svojim kolima, svojim konjanicima  i svojim mnogim brodovima. Provalit će u zemlje i proći njima  poput poplave. 
\par 41 Prodrijet će u Divotu i mnogi će pasti. Njegovim  će rukama izmaći Edom i Moab i glavnina sinova Amonovih. 
\par 42 Pružit će svoju ruku za zemljama: Egipat mu neće izmaći. 
\par 43 On će se domoći zlatnog i srebrnog blaga i svih dragocjenosti  Egipta. Pratit će ga Libijci i Etiopljani. 
\par 44 Ali će ga uznemiriti  vijesti s istoka i sa sjevera te će poći vrlo gnjevan da uništi  i zatre mnoštvo. 
\par 45 Postavit će svoje dvorske šatore između  mora i Svete gore Divote. Ali će i njemu doći kraj, i nitko mu  neće pomoći. 



\chapter{12}

\par 1 U ono će vrijeme ustati Mihael, veliki knez koji štiti sinove  tvog naroda. Bit će to vrijeme tjeskobe kakve ne bijaše otkako  je ljudi pa do toga vremena. U ono vrijeme tvoj će se narod spasiti  - svi koji se nađu zapisani u Knjizi. 
\par 2 Tada će se probuditi mnogi koji snivaju u prahu zemljinu;  jedni za vječni život, drugi za sramotu, za vječnu gadost. 
\par 3 Umnici  će blistati kao sjajni nebeski svod, i koji su mnoge učili pravednosti, kao zvijezde navijeke, u svu vječnost. 
\par 4 A ti, Daniele, drži u tajnosti ove riječi i zapečati ovu  knjigu do vremena svršetka! Mnogi će tumarati, i bezakonja će  rasti." 
\par 5 Ja, Daniel, pogledah, kad eno: druga dvojica stajahu jedan  s jedne, drugi s druge strane rijeke. 
\par 6 Jedan upita čovjeka  odjevena u lanene haljine koji stajaše iznad voda rijeke: "Kada  će doći kraj tim čudesima?" 
\par 7 Začuh čovjeka odjevena u lanene  haljine, koji stajaše iznad voda rijeke; on podiže k nebu desnicu  i ljevicu, kunući se Vječno živim. "Nakon jednog vremena, dva  vremena i pola vremena - kada dođe kraj rasulu snage svetoga  naroda - sve će se to svršiti." 
\par 8 Ja slušah, ali ne razumjeh, pa upitah: "Gospodaru, kako  će to svršiti?" 
\par 9 On reče: "Idi, Daniele, ove su riječi tajne  i zapečaćene do vremena svršetka. 
\par 10 Mnogi će se očistiti, ubijeliti  i prokušati; a bezbožnici će i dalje biti bezbožni; bezbožnici  se neće urazumjeti, a umnici će razumjeti. 
\par 11 Od časa kad bude  dokinuta svagdašnja žrtva i postavljena grozota pustoši: tisuću  dvjesta i devedeset dana. 
\par 12 Blago onomu koji dočeka i dosegne  tisuću trista trideset i pet dana! 
\par 13 A ti idi i otpočini; ustat  ćeš da primiš svoju baštinu na kraju dana." 




\end{document}