\begin{document}

\title{Jošua}


\chapter{1}

\par 1 Poslije smrti Mojsija, sluge Jahvina, reče Jahve Jošui, sinu  Nunovu, pomoćniku Mojsijevu: 
\par 2 "Moj je sluga Mojsije umro; zato  sada ustani, prijeđi preko toga Jordana, ti i sav taj narod,  u zemlju koju dajem sinovima Izraelovim. 
\par 3 Svako mjesto na koje  stupi vaša noga dajem vam, kao što obećah Mojsiju. 
\par 4 Od pustinje  i od Libanona pa do Velike rijeke, rijeke Eufrata, i sve do Velikog  mora na sunčanom zapadu - sve će to biti vaše područje. 
\par 5 Nitko  neće odoljeti pred tobom u sve dane tvog života; ja ću biti s  tobom, kao što sam bio s Mojsijem, i nikada te neću napustiti  niti ću te ostaviti. 
\par 6 Budi odvažan i hrabar jer ćeš ti uvesti narod ovaj da  primi u baštinu zemlju za koju se zakleh ocima njihovim da ću  im je dati. 
\par 7 Samo budi odvažan i hrabar da sve učiniš vjerno  prema naredbama koje ti je dao Mojsije, sluga moj. Ne skreći  od toga ni desno ni lijevo da bi ti bilo sretno sve što poduzmeš. 
\par 8 Neka knjiga Zakona bude na ustima tvojim: razmišljaj o njoj  danju i noću, kako bi vjerno držao sve što je u njoj napisano:  samo ćeš tada biti sretan i uspjet ćeš u pothvatima. Nisam li  ti zapovjedio: 
\par 9 odvaži se i budi hrabar? Ne boj se i ne strahuj, jer kuda god pođeš, s tobom je Jahve, Bog tvoj." 
\par 10 Tada zapovijedi Jošua glavarima narodnim: 
\par 11 "Prođite  kroz tabor i proglasite puku ovu zapovijed: 'Spremite sebi brašnenice  jer ćete za tri dana prijeći preko Jordana da biste primili u  posjed zemlju koju vam Jahve, Bog vaš, daje u baštinu.'" 
\par 12 Zatim reče Jošua plemenu Rubenovu i Gadovu i polovini  plemena Manašeova: 
\par 13 "Sjetite se onoga što vam je zapovjedio  Mojsije, sluga Jahvin, kada vam je rekao: 'Jahve, Bog vaš, hoće  da počinete i daje vam ovu zemlju. 
\par 14 Vaše žene, djeca i stada  mogu ostati u zemlji koju vam je dao Mojsije s onu stranu Jordana.  Vi pak ratnici, za boj spremni, morate naoružani poći pred svojom  braćom da im pomognete, 
\par 15 sve dok Jahve ne dade da počinu i  vaša braća, kao i vi, i dok ne zaposjednu zemlju koju im daje  Jahve, Bog vaš. Tada se možete vratiti u zemlju koja vam pripada  i koju vam je dao Jahvin sluga Mojsije, na drugoj strani Jordana, prema istoku sunca.'" 
\par 16 Oni odgovore Jošui: "Sve što nam zapovjediš, učinit ćemo, i kuda nas god pošalješ, poći ćemo. 
\par 17 Kao što smo slušali  Mojsija, tako ćemo se pokoravati i tebi. Samo neka Jahve, Bog  tvoj, bude s tobom kao što bijaše s Mojsijem! 
\par 18 Tko se god  usprotivi tvome glasu i ne posluša tvojih riječi u svemu što  mu zapovjediš neka bude pogubljen. Samo ti budi odvažan i hrabar!" 


\chapter{2}

\par 1 Jošua, sin Nunov, posla potajno iz Šitima dvojicu uhoda s nalogom:  "Idite, izvidite područje, osobito Jerihon." Oni odu i stignu  u kuću bludnice koja se zvala Rahaba i ondje prenoće. 
\par 2 To bude  javljeno kralju jerihonskom: "Evo, stigoše noćas ovamo neki ljudi  od sinova Izraelovih da izvide zemlju." 
\par 3 Tada kralj jerihonski  poruči Rahabi: "Izvedi ljude koji su došli k tebi, koji su ušli  u tvoj dom, jer su došli uhoditi svu zemlju." 
\par 4 Ali žena uze  ona dva čovjeka, sakri ih i reče: "Istina, ti su ljudi došli  k meni, ali ja nisam znala odakle su. 
\par 5 Kada se u sumrak zatvarahu  gradska vrata, oni odoše i ja ne znam kamo su krenuli. Požurite  za njima jer ih još možete stići." 
\par 6 A ona ih bijaše izvela  na krov i sakrila pod netrveni lan što ga je ondje razastrla. 
\par 7 I požure se ljudi u potjeru za njima, prema Jordanu, sve do  prijelaza preko rijeke; a kad je potjera izišla, zatvore se za  njima gradska vrata. 
\par 8 Dok još oni gore ne bijahu zaspali, popne se Rahaba k  njima na krov 
\par 9 i reče im: "Znam da vam je Jahve dao ovu zemlju, jer nas je sve uhvatio strah od vas i prezaju od vas svi žitelji  ovoga kraja. 
\par 10 Jer čusmo kako je Jahve isušio vodu Crvenoga  mora pred vama kada ste izašli iz Egipta, i ono što ste učinili  dvojici kraljeva amorejskih s druge strane Jordana, Sihonu i  Ogu, koje pogubiste. 
\par 11 Kad smo čuli sve to, zastalo nam srce  i nitko da smogne snage da vam se suprotstavi jer Jahve, Bog  vaš - on je Bog gore na nebesima i dolje na zemlji. 
\par 12 Zakunite  mi se, dakle, Jahvom da ćete i vi učiniti milost domu oca moga, kao što i ja učinih milost vama, i dajte mi pouzdan znak 
\par 13 da  ćete ostaviti na životu moga oca i moju majku, braću moju i sestre  moje i sve njihovo i da ćete nas izbaviti od smrti." 
\par 14 Odgovoriše joj ljudi: "Životom svojim jamčimo za vas, samo ako nas ne izdate. Kad nam Jahve dade zemlju, iskazat ćemo  ti milost i vjernost." 
\par 15 Rahaba ih zatim spusti po konopu kroz prozor jer joj  je kuća bila uz bedem i ona je do bedema stanovala. 
\par 16 Još im  reče: "Pođite prema gori da vas potjera ne nađe i krijte se ondje  tri dana dok se progonitelji ne vrate, a onda idite svojim putem." 
\par 17 Ljudi joj odgovore: "Evo, ovako ćemo ti ispuniti zakletvu  kojom si nas zaklela: 
\par 18 kad uđemo u zemlju, posluži se ovim  znakom: priveži ovu crvenu vrpcu za prozor kroz koji nas spuštaš  i sakupi kod sebe, u kući, svoga oca, i svoju majku, i svoju  braću, i svu svoju rodbinu. 
\par 19 Tko god od vas stupi van preko  praga tvoje kuće, krv njegova na glavu njegovu: nije krivnja  na nama - sam je krivac svojoj smrti; a tko ostane s tobom u  kući, krv njegova neka padne na glave naše - mi ćemo biti krivci  ako ga se tko rukom dotakne. 
\par 20 Ako pak izdaš ovu našu stvar, slobodni smo od zakletve kojom si nas zaklela." 
\par 21 A ona odgovori: "Neka bude kako rekoste!" Tada ih pusti  i oni odoše, a ona zaveza na prozor crvenu vrpcu. 
\par 22 Oni odoše i dođoše u goru i ondje ostadoše tri dana dok  se ne vrati potjera; tražila ih je potjera na svim putovima,  ali ih nije nigdje našla. 
\par 23 Tada se vrate i one dvije uhode:  siđu s gore, prijeđu preko rijeke i dođu k Jošui, sinu Nunovu, te ga izvijeste o svemu što im se dogodilo. 
\par 24 I rekoše Jošui:  "Jahve nam je svu tu krajinu predao u ruke; sve je njezine stanovnike  uhvatio strah pred nama." 


\chapter{3}

\par 1 Urani Jošua i sa svim sinovima Izraelovim krene od Šitima.  I stignu do Jordana pa ondje prije prelaza prenoće. 
\par 2 Poslije  tri dana prođu starješine kroz tabor i zapovjede puku: 
\par 3 "Čim  ugledate Kovčeg saveza Jahve, Boga vašega, i svećenike levite  koji ga nose, krenite svi sa svoga mjesta i pođite za njim. 
\par 4 Tako  ćete znati put kojim vam je ići, jer tim putem još nikada niste  išli. Ali između vas i Kovčega neka bude razmak do dvije tisuće  lakata. I da mu se niste približili." 
\par 5 A Jošua zapovjedi narodu: "Posvetite se za sutra, jer  će sutra Jahve učiniti čudesa među vama." 
\par 6 A svećenicima Jošua  zapovjedi: "Dignite Kovčeg saveza i nosite ga pred narodom."  I digoše Kovčeg saveza i poniješe ga pred narodom. 
\par 7 Jahve reče Jošui: "Danas te počinjem uzvisivati pred očima  svega Izraela, neka znaju da sam s tobom kao što bijah s Mojsijem. 
\par 8 Ti pak zapovjedi svećenicima koji nose Kovčeg saveza:' Kada  stignete do voda jordanskih, u Jordanu se samom zaustavite.'" 
\par 9 Tada reče Jošua Izraelcima: "Priđite i čujte riječi Jahve, Boga svojega." 
\par 10 I reče Jošua: "Po ovomu ćete spoznati da  je među vama Bog živi: on će goniti ispred vas Kanaance, Hetite, Hivijce, Perižane, Girgašane, Amorejce i Jebusejce. 
\par 11 Evo, Kovčeg saveza Gospodara sve zemlje proći će pred vama preko  Jordana. 
\par 12 Izaberite odmah dvanaest ljudi iz plemena Izraelovih, po jednoga iz svakoga plemena. 
\par 13 Čim stopala svećenika koji  nose Kovčeg Jahve, Gospodara sve zemlje, stupe u Jordan, razdijelit  će se voda Jordana, i ona što teče odozgo ustavit će se kao nasip.'" 
\par 14 Kad je narod krenuo iz svojih šatora da prijeđe preko  Jordana, ponesu svećenici Kovčeg saveza pred njim. 
\par 15 A kad  su nosači Kovčega stigli do Jordana i kada su svećenici koji  su nosili Kovčeg zagazili u vodu na obali - a bilo je vrijeme  žetve kad se Jordan prelijeva preko svojih obala - 
\par 16 voda što  je tekla odozgo daleko se, poput nasipa, ustavila kod grada Adame, koji se nalazi kraj Sartana; a voda što je otjecala dolje u  Arabsko ili Slano more sasvim je otekla i narod je prelazio prema  Jerihonu. 
\par 17 Svećenici koji su nosili Kovčeg saveza Jahvina  stajahu na suhu usred Jordana i prelažaše Izrael po suhu sve  dok sav narod ne prijeđe preko rijeke. 


\chapter{4}

\par 1 Pošto je sav narod prešao preko Jordana, reče Jahve Jošui: 
\par 2 "Izaberite iz naroda dvanaest ljudi, od svakoga plemena po  jednoga, 
\par 3 i zapovjedite im: 'Dignite odavde, iz sredine Jordana  - s mjesta gdje stoje noge svećenika - dvanaest kamenova koje  ćete ponijeti sa sobom i položiti na mjestu gdje budete noćas  prenoćili.'" 
\par 4 Tada pozva Jošua dvanaest ljudi koje je bio izabrao između  sinova Izraelovih, iz svakoga plemena po jednoga čovjeka, 
\par 5 i  reče im: "Idite pred Kovčeg Jahve, Boga svoga, u sredinu Jordana, i neka svaki donese na svojim ramenima po jedan kamen prema  broju plemena Izraelovih. 
\par 6 To će biti na spomen među vama.  Kad vas jednoga dana budu pitala vaša djeca: 'Što vam znače ovi  kamenovi?' 
\par 7 reći ćete im: 'Voda se Jordana razdijelila pred  Kovčegom saveza Jahvina kad je prelazio preko Jordana.' I ovo  će kamenje biti vječni spomen sinovima Izraelovim." 
\par 8 Izraelci učine kako im je zapovjedio Jošua, uzmu dvanaest  kamenova iz sredine Jordana, prema broju plemena Izraelovih,  kako je Jahve naredio Jošui: prenesu ih do svoga noćišta i polože  ondje. 
\par 9 Zatim Jošua postavi usred Jordana dvanaest kamenova  na mjesta gdje su stajale noge svećenika koji su nosili Kovčeg  saveza. Ondje stoje i danas. 
\par 10 Svećenici koji su nosili Kovčeg saveza stajali su usred  Jordana, sve dok se nije izvršilo sve što je Jahve zapovjedio  Jošui da narod izvrši, sasvim onako kao što Mojsije bijaše naredio  Jošui. A narod je žurno prelazio. 
\par 11 Pošto je sav narod prešao, prijeđu i svećenici s Kovčegom saveza Jahvina i krenu pred narodom. 
\par 12 Tada sinovi Rubenovi i sinovi Gadovi i polovina plemena Manašeova  u bojnoj opremi stanu na čelo sinova Izraelovih, kao što im bijaše  zapovjedio Mojsije. 
\par 13 Oko četrdeset tisuća naoružanih ljudi  prešlo je pred Jahvom da se bori na Jerihonskim poljanama. 
\par 14 Toga  dana uzvisi Jahve Jošuu pred svim Izraelom i svi ga se bojahu, kao nekoć Mojsija, u sve dane njegova života. 
\par 15 Jahve reče Jošui: 
\par 16 "Zapovjedi svećenicima koji nose  Kovčeg saveza neka izađu iz Jordana." 
\par 17 Tada Jošua zapovjedi  svećenicima: "Izađite iz Jordana!" 
\par 18 A čim su svećenici koji  su nosili Kovčeg saveza Jahvina izašli isred Jordana i stali  nogama na suho, vrate se vode Jordana na svoje mjesto i poteku  kao i prije preko svojih obala. 
\par 19 Narod je izašao iz Jordana desetog dana prvoga mjeseca.  Tada se utaborio u Gilgalu, istočno od Jerihona. 
\par 20 A onih dvanaest  kamenova što su ih uzeli sa sobom iz Jordana, Jošua postavi u  Gilgalu. 
\par 21 Tada reče Izraelcima: "Ako potomci vaši upitaju  jednoga dana svoje očeve: 'Što znači ovo kamenje?' - 
\par 22 vi ih  poučite ovako: 'Izrael je ovdje po suhu prešao preko Jordana 
\par 23 jer je Jahve, Bog vaš, osušio pred vama vodu Jordana dok  ne prijeđoste, kao što je učinio Jahve, Bog vaš, s Morem crvenim  kad ga je osušio pred nama dok ne prijeđosmo. 
\par 24 A sve to, da  bi znali svi narodi zemlje koliko je moćna ruka Jahvina, i vi  sami da se svagda bojite Jahve, Boga svoga.'" 


\chapter{5}

\par 1 Pošto su čuli svi kraljevi amorejski na zapadnoj strani Jordana  i svi kraljevi kanaanski koji bijahu uz more da je Jahve osušio  Jordan pred Izraelcima dok ne prijeđoše, zastade im srce i nestade  im junaštva pred Izraelcima. 
\par 2 U to vrijeme Jahve reče Jošui: "Načini sebi kamene noževe  i ponovo obreži Izraelce." 
\par 3 Jošua načini sebi kamene noževe  i obreza Izraelce na brežuljku Aralotu. 
\par 4 A evo zašto ih je Jošua obrezao: sve ljudstvo što je izišlo  iz Egipta, sve što mogaše nositi oružje, pomrlo je na putu kroz  pustinju. 
\par 5 Svi oni bijahu obrezani, ali nije bio obrezan nitko  koji se rodio na putu kroz pustinju, poslije izlaska iz Egipta, 
\par 6 jer su četrdeset godina Izraelci lutali pustinjom dok ne  pomriješe svi za oružje sposobni koji bijahu izišli iz Egipta;  nisu slušali glasa Jahvina te im se Jahve zakleo da njihove oči  neće vidjeti zemlju koju je obećao njihovim ocima - zemlju u  kojoj teče mlijeko i med. 
\par 7 Na njihovo je mjesto podigao sinove  njihove i njih je Jošua obrezao: nisu bili obrezani jer se na  putu nije obrezivalo. 
\par 8 Kad je bio obrezan sav narod, počivali  su u taboru sve dok nisu ozdravili. 
\par 9 Tada reče Jahve Jošui:  "Danas skidoh s vas sramotu egipatsku." I prozva se ono mjesto  Gilgal sve do naših dana. 
\par 10 Izraelci se, dakle, utaboriše u Gilgalu i ondje na Jerihonskim  poljanama proslaviše Pashu uvečer četrnaestoga dana u mjesecu. 
\par 11 A sutradan poslije Pashe, upravo toga dana, blagovali su  od uroda one zemlje: beskvasna kruha i pržena zrnja. 
\par 12 I mÓana  je prestala padati čim su počeli jesti plodove zemlje. Tako Izraelci  nisu više imali mane, nego su se te godine hranili plodovima  zemlje kanaanske. 
\par 13 Kad se Jošua približio gradu Jerihonu, podiže oči i ugleda  čovjeka kako pred njim stoji s isukanim mačem u ruci. Jošua mu  pristupi i upita ga: "Jesi li ti s nama ili s našim neprijateljima?" 
\par 14 A on odgovori: "Ne, ja sam vođa vojske Jahvine i upravo sam  došao ..." Tada Jošua pade ničice, pokloni mu se i reče: "Što  zapovijedaš Gospodaru, sluzi svome?" 
\par 15 A vođa vojske Jahvine  odgovori Jošui: "Skini obuću s nogu svojih, jer je sveto mjesto  na kojem stojiš." I Jošua učini tako. 


\chapter{6}

\par 1 A Jerihon stajaše silno utvrđen i zatvoren pred sinovima Izraelovim.  Nitko nije izlazio niti je tko ulazio. 
\par 2 Tada Jahve reče Jošui:  "Evo, predajem ti u ruke Jerihon i kralja njegova s ratnicima. 
\par 3 Svi vi ratnici obiđite oko grada jedanput na dan. Tako činite  šest dana. 
\par 4 A sedam svećenika neka nose pred Kovčegom sedam  truba od ovnujskih rogova. Sedmoga dana obiđite sedam puta oko  grada, a svećenici neka trube u trublje. 
\par 5 Pa kad otežući zatrube  u rog ovnujski, neka sav narod, čim čuje glas trube, podigne  silnu bojnu viku. I srušit će se gradski bedemi, a narod neka  tada ulazi svaki odande gdje se nađe." 
\par 6 Jošua, sin Nunov, pozva k sebi svećenike i reče im: "Uzmite  Kovčeg saveza, a sedam svećenika neka ponese sedam truba od rogova  ovnujskih pred Kovčegom Jahvinim." 
\par 7 A narodu reče: "Pođite  i obiđite oko grada, a ratnici neka idu pred Kovčegom Jahvinim." 
\par 8 I bi kako je Jošua zapovjedio narodu. Pođe sedam svećenika  noseći trube od rogova ovnujskih: trubili su u rogove, a Kovčeg  Jahvin iđaše za njima. 
\par 9 Ratnici pođoše pred svećenicima koji  su trubili u trube, a zalaznica krenu za Kovčegom. Stupali su  tako dok se glas truba razlijegao. 
\par 10 A narodu bijaše zapovjedio Jošua govoreći: "Ne vičite  i ne dajte glasa od sebe i nijedna riječ neka se ne čuje iz vaših  usta dok vam ne kažem: 'Vičite!' Tada neka odjekne bojna vika." 
\par 11 I naredi da Kovčeg Jahvin obiđe jednom oko grada pa da  se vrate u tabor i ondje prenoće. 
\par 12 Sutradan urani Jošua, a  svećenici ponesu Kovčeg saveza. 
\par 13 A sedam svećenika koji su  nosili sedam truba od rogova ovnujskih pođu pred Kovčegom Jahvinim.  Idući trubili su u trube, ratnici iđahu pred njima, a zalaznica  pak za Kovčegom Jahvinim dok su trube odjekivale. 
\par 14 Tako i drugog dana obiđu jednom oko grada pa se vrate  natrag u tabor. Tako su činili šest dana. 
\par 15 A sedmoga dana  zorom ustanu i obiđu oko grada istim onakvim redom sedam puta.  Samo su toga dana obišli oko grada sedam puta. 
\par 16 Za sedmog obilaska snažno zatrube svećenici u rogove, a Jošua reče narodu: "Kličite bojne poklike jer vam je Jahve  predao grad! 
\par 17 Grad neka bude 'herem' Jahvi - uklet i predan  uništenju sa svime što je u njemu. Samo bludnica Rahaba da ostane  živa i svi koji budu s njom u kući, jer je ona sakrila uhode  koje smo poslali. 
\par 18 A čuvajte se svega ukletog u gradu da i  sami ne budete prokleti što ste uzeli ukleto, jer biste time  navukli prokletstvo na tabor i unesrećili ga. 
\par 19 Zato sve srebro  i zlato, sve bakreno i željezno posuđe neka bude posvećeno Jahvi  i pohranjeno u riznicu." 
\par 20 Tada povika narod i odjeknuše trube. Kada se zaori glas  truba i bojni povici naroda, padoše bedemi i narod prodrije u  grad, svatko odande gdje se našao, i osvojiše ga. 
\par 21 I tada  izvršiše kletvu ništeći oštricom mača sve što bijaše u gradu:  muško i žensko, staro i mlado, volove, ovce i magarad. 
\par 22 A onoj dvojici što su uhodili zemlju reče Jošua: "Idite  u kuću one bludnice pa izvedite ženu sa svima njezinima, kako  joj se zakleste." 
\par 23 I mladi ljudi, uhode, odoše te izvedoše  Rahabu, njezina oca i njezinu majku, braću i svu rodbinu. Izvedoše  sve njezine i smjestiše ih izvan izraelskog tabora. 
\par 24 Spališe grad i sve što bijaše u njemu: uzeše samo srebro, zlato, tučano i željezno posuđe i staviše u riznicu Doma Jahvina. 
\par 25 Ali bludnicu Rahabu, svu njenu obitelj i sve njihovo poštedi  Jošua. Ona ostade među Izraelcima sve do danas, jer je sakrila  glasnike koje je poslao Jošua da uhode Jerihon. 
\par 26 Tada izreče Jošua ovu kletvu: "Proklet bio pred licem Jahve čovjek koji pokuša da ponovo gradi  Jerihon: gradio mu temelje na svom prvencu, podizao mu vrata  na svome mezimcu!" 
\par 27 Jahve je bio s Jošuom te se pronio glas o njemu po svoj  zemlji. 


\chapter{7}

\par 1 Ali se sinovi Izraelovi teško ogriješiše o "herem", jer je  Akan, sin Karmija, sina Zabdijeva, sina Zerahova, od plemena  Judina, uzeo od ukletih stvari, i Jahve se razgnjevi na sinove  Izraelove. 
\par 2 Jošua pak posla ljude iz Jerihona u Aj, koji leži istočno  od Betela, i reče im: "Uziđite onamo, izvidite kraj!" Ljudi odoše  te izvidješe Aj. 
\par 3 Vrativši se k Jošui, rekoše mu: "Ne treba  da onamo uzlazi sav narod; dvije do tri tisuće ljudi neka idu  da osvoje Aj. Ne muči onamo sav narod, jer je ondje malo svijeta." 
\par 4 Pođe onamo oko tri tisuće ljudi od svega naroda, ali su  morali pobjeći pred onima iz Aja. 
\par 5 Ajani pobiše oko trideset  i šest ljudi i tjerali su ih ispred svojih vrata do Šebarima:  pobili su ih na strmini. Klonu tada srce narodu kao da mu je  voda u žilama. 
\par 6 Razdrije Jošua haljine svoje i baci se ničice pred Kovčegom  Jahvinim, i ostade tako do večeri, on i starješine u Izraelu, posuvši glave pepelom. 
\par 7 Tada reče Jošua: "Jao, Gospode Jahve, zašto si preveo ovaj narod preko Jordana? Da nas predaš u ruke  Amorejaca da nas pobiju? Kamo sreće da smo stali s onu stranu  Jordana! 
\par 8 Oprosti, Gospode! Što drugo da rečem kad je Izrael  okrenuo leđa pred svojim neprijateljima? 
\par 9 Ako to čuju Kanaanci  i ostali žitelji zemlje, udružit će se protiv nas da zbrišu ime  naše sa zemlje. Što ćeš, dakle, učiniti za veliko ime svoje?" 
\par 10 A Jahve odgovori Jošui: "Ustani! Zašto si pao ničice? 
\par 11 Izrael je sagriješio: prekršili su Savez kojim sam ih vezao.  Zaista, uzeše ukleto, porobiše, posakrivaše i prisvojiše. 
\par 12 I  zato Izraelci ne mogu izdržati pred svojim neprijateljima, okreću  leđa pred protivnicima jer su postali ukleti. Ja ne mogu više  biti s vama ako iz svoje sredine ne maknete proklete. 
\par 13 Ustani!  Sazovi narod na posvećenje i reci mu: Posvetite se za sutra,  jer ovako govori Jahve, Bog Izraelov: 'Kletva je u tebi, Izraele;  i nećeš izdržati pred svojim neprijateljima sve dok ne odstranite  kletvu iz svoje sredine.' 
\par 14 Zato sutra zorom pristupite pleme  za plemenom; iz plemena koje odredi Jahve prići će rod za rodom, a onda iz roda koji označi Jahve pristupit će obitelj po obitelj, a iz obitelji koju označi Jahve pristupit će čovjek za čovjekom. 
\par 15 I tko se tada nađe s ukletom stvari, neka se spali on i sve  što mu pripada, jer je prekršio Savez Jahvin i osramotio Izraela." 
\par 16 Urani Jošua ujutro i pozva Izraela po plemenima; pristupiše  i otkri se pleme Judino. 
\par 17 Potom pristupi rod za rodom iz plemena  Judina i pronađe se rod Zerahov. Pristupiše obitelji roda Zerahova, domaćin jedan za drugim, i pronađoše obitelj Zabdijevu. 
\par 18 Naposljetku  naredi Jošua da pristupi obitelj Zabdijeva, muškarac jedan za  drugim, i pronašao se Akan, sin Karmija, sina Zabdijeva, sina  Zerahova, od plemena Judina. 
\par 19 Tada reče Jošua Akanu: "Sine moj, daj slavu Jahvi, Bogu  Izraelovu, i priznaj mu što si učinio; objasni što si učinio  i nemoj mi ništa tajiti." 
\par 20 Akan reče Jošui: "Zaista, ja sagriješih  Jahvi, Bogu Izraelovu, i evo što sam učinio: 
\par 21 vidjeh u plijenu  lijep šinearski plašt, dvije stotine srebrnjaka i zlatnu šipku  vrijednu pedeset srebrnjaka, pa se polakomih i uzeh sebi. Eno  je sve zakopano usred moga šatora, a srebro je odozdo." 
\par 22 Tada uputi Jošua poslanike, koji otrčaše u šator. I gle, sve bijaše zakopano u šatoru, a odozdo srebro. 
\par 23 Uzmu sve  iz šatora i donesu Jošui i starješinama Izraelovim i prostriješe  sve pred Jahvu. 
\par 24 Tada uze Jošua Akana, sina Zerahova, i srebro, plašt  i zlatnu šipku, i sve sinove i kćeri njegove, volove njegove  i magarad, i ovce, šator njegov i sve što bijaše njegovo te ga  izvede u dolinu Akor. Pratio ih sav Izrael. 
\par 25 Reče Jošua: "Kako si ti nas unesrećio, tako danas tebe  unesrećio Jahve!" I kamenova ga sav Izrael. 
\par 26 Potom navališe na njega gomilu kamenja, koje stoji do  danas. Tako se Jahve ublaži od svoga žestoka gnjeva. Zbog toga  se događaja prozva ono mjesto dolina Akor i tako se zove do danas. 


\chapter{8}

\par 1 Tada reče Jahve Jošui: "Ne boj se i ne strahuj! Uzmi sa sobom  sve ratnike, ustani i navali na Aj. Gle, predajem ti u ruke ajskoga  kralja, njegov narod, grad i zemlju njegovu. 
\par 2 Učini s Ajem  i s njegovim kraljem kao što si učinio s Jerihonom i njegovim  kraljem; ali vam je slobodno da prigrabite plijen iz njega i  njegovu stoku. Postavi gradu zasjedu s leđa." 
\par 3 Spremi se Jošua da navali na Aj i svi ratnici s njime.  Izabrao je trideset tisuća junaka i poslao ih noću; 
\par 4 dade im  zapovijed: "Pazite! Poći ćete u zasjedu gradu s leđa, ali da  ne budete predaleko od grada i budite svi spremni. 
\par 5 A ja i  sav narod koji me prati primaknut ćemo se gradu; i kada ljudi  iz Aja izađu pred nas, mi ćemo kao i prije pobjeći ispred njih. 
\par 6 Oni će onda navaliti za nama dok ih ne odvedemo od grada jer  će misliti: 'Bježe ispred nas kao i prije.' 
\par 7 Tada provalite  iz zasjede i zauzmite grad: Jahve, Bog vaš, predat će vam ga  u ruke. 
\par 8 Kad jednom osvojite grad, spalite ga ognjem. Učinite  to po Jahvinoj zapovijedi. Pazite, to vam zapovjedih." 
\par 9 Jošua ih posla i oni odoše u zasjedu te se smjestiše između  Betela i Aja, gradu sa zapada. A Jošua provede noć među narodom. 
\par 10 Uranivši, Jošua ujutro prebroja narod i pođe sa starješinama  Izraelovim pred narodom na Aj. 
\par 11 Svi ratnici krenu s njim i  kad se primaknu gradu, utabore se Aju sa sjevera, tako da je  između njih i mjesta bila ravnica. 
\par 12 Jošua uze oko pet tisuća  ljudi i namjesti zasjedu između Betela i Aja, gradu sa zapadne  strane. 
\par 13 A narod se smjesti u tabor, koji je bio na sjeveru  grada, dok je njegova zalaznica bila na zapadu grada. Jošua opet  provede noć usred naroda. 
\par 14 Kad je sve to vidio ajski kralj, požuri se te izađe on  i sav njegov narod niz obronak prema Arabi u boj protiv Izraela.  A nisu ni slutili da je iza grada namještena zasjeda. 
\par 15 Tada  Jošua i sav Izrael nagnu bježati kao da su ih pobijedili. I bježali  su putem prema pustinji. 
\par 16 Ajani nato pozvaše sve iz grada  i dadoše se za njima u potjeru te, goneći Jošuu, odvoje se od  grada. 
\par 17 I ne ostade nitko u Aju i Betelu da nije pošao za  Izraelcima. Ostavili su grad otvoren i gonili Izraelce. 
\par 18 Tada reče Jahve Jošui: "Zamahni kopljem što ti je u ruci  prema Aju: predajem ti ga u ruke." I podiže Jošua koplje što  mu bješe u ruci i zamahnu prema gradu. 
\par 19 I tek što je podigao  ruku, dignu se ljudi iz zasjede i potrče prema gradu, osvoje  ga i umah ga ognjem zapale. 
\par 20 Kada se oni iz Aja obazreše, imadoše što vidjeti: dim  se dizao iz grada prema nebu. I nitko od njih nije imao kuda  uteći ni tamo ni amo. Tada se narod koji je bježao prema pustinji  okrenuo prema progoniteljima. 
\par 21 Vidjevši Jošua i sav Izrael  da je zasjeda zauzela grad i da se diže dim iz grada, vrate se  i udare na ljude iz Aja. 
\par 22 Njihovi su im izašli u susret iz  grada, i tako se oni iz Aja nađoše posred Izraelaca, opkoljeni  i s jedne i s druge strane: biše pobijeni tako te ni jedan ne  ostade živ niti uteče. 
\par 23 A kralja Aja uhvatiše živa i dovedoše  ga Jošui. 
\par 24 Kad su Izraelci pobili sve stanovnike Aja na otvorenu  polju i u pustinji, kuda su ih gonili, i kada svi padoše od mača, vratiše se Izraelci u Aj i sasjekoše mačem sve što bješe u njemu. 
\par 25 Bilo je dvanaest tisuća onih koji su izginuli toga dana,  ljudi i žena - sav Aj. 
\par 26 Jošua nije spuštao ruke kojom bijaše zamahnuo kopljem  sve dok nisu poubijani svi stanovnici Aja. 
\par 27 Samo stoku i plijen  iz onoga grada razdijele među sobom Izraelci, kao što je Jahve  zapovjedio Jošui. 
\par 28 Jošua spali Aj i učini ga za sve vijeke ruševinom, pustim  mjestom do danas. 
\par 29 Kralja ajskoga objesi o drvo do večeri.  O zapadu sunčanom zapovjedi Jošua te skinuše truplo s drveta, baciše ga pred gradska vrata i nabacaše na nj veliku gomilu  kamenja, koja stoji i danas. 
\par 30 Tada podiže Jošua žrtvenik Jahvi, Bogu Izraelovu, na  gori Ebalu, 
\par 31 kao što je zapovjedio Mojsije, sluga Jahvin,  svim sinovima Izraelovim i kako je napisano u Mojsijevoj knjizi  Zakona: žrtvenik od grubog kamena, neklesanog željezom. Na njemu  bi prinesena Jahvi žrtva paljenica i pričesnica. 
\par 32 Tu na kamenju Jošua prepiše Zakon Mojsijev koji bješe  napisan za sinove Izraelove. 
\par 33 I sav Izrael i njegove starješine, glavari narodni i suci, došljaci i domaći, stanu s obje strane  Kovčega prema svećenicima i levitima koji su nosili Kovčeg saveza  Jahvina - polovina prema gori Gerizimu, a polovina prema gori  Ebalu - da bi se blagoslovio puk Izraelov prema obredu koji zapovjedi  Mojsije. 
\par 34 Tada pročita Jošua svaku riječ Zakona, blagoslov  i prokletstvo, sve kako je napisano u knjizi Zakona. 
\par 35 Nije Jošua propustio nijedne Mojsijeve naredbe, nego  ih je sve pročitao pred saborom svih Izraelaca, pred ženama,  djecom i došljacima koji su išli s njima. 


\chapter{9}

\par 1 O tim su događajima čuli svi kraljevi s onu stranu Jordana  - u Gorju, u Šefeli i duž čitave obale Velikoga mora sve do Libanona:  Hetiti, Amorejci, Kanaanci, Perižani, Hivijci, Jebusejci - 
\par 2 pa  se svi udružiše da složno udare protiv Jošue i Izraela. 
\par 3 A stanovnici Gibeona, poučeni onim što Jošua učini Jerihonu  i Aju, 
\par 4 dosjete se lukavstvu. Uzmu hiniti da su putnici: bace  na svoje magarce stare vreće i vinske mješine, poderane i zakrpane. 
\par 5 Obuli su na noge rabljenu i pokrpanu obuću i vrgli na se staru  odjeću. Sav kruh što su ga ponijeli na put bijaše suh i razdrobljen. 
\par 6 Stigoše Jošui u gilgalski tabor i rekoše njemu i ljudima  Izraelcima: "Dolazimo iz daleke zemlje, sklopite savez s nama." 
\par 7 Ali ljudi Izraelci kažu tim Hivijcima: "Tko zna ne živite  li možda među nama? Kako ćemo, dakle, sklopiti savez s vama?" 
\par 8 A oni odgovore Jošui: "Tvoje smo sluge!" Jošua ih upita: "Tko  ste i odakle dolazite?" 
\par 9 Odgovore: "Daleka je zemlja iz koje  dolaze tvoje sluge u ime Jahve, Boga tvojega: čuli smo za slavu  njegovu i za sve što je učinio u Egiptu 
\par 10 i za ono što je učinio  dvojici kraljeva amorejskih koji su vladali s onu stranu Jordana  - Sihonu, kralju hešbonskom, i Ogu, kralju bašanskom u Aštarotu. 
\par 11 Tada nam rekoše naše starješine i svi u našoj zemlji: 'Opskrbite  se hranom za put, pođite im u susret i recite im: Vaše smo sluge, sklopite dakle savez s nama.' 
\par 12 Evo našega kruha: vruć smo  ponijeli na put od kuća svojih kada smo krenuli k vama, a sada  je, evo, suh i razdrobljen. 
\par 13 A ovo su vinski mjehovi: nove  smo ih nalili, pa su se, evo, već poderali; i haljine naše i  obuća već su trošni od dalekog puta." 
\par 14 I povjerovaše im ljudi po putnoj opskrbi, ne pitajući  Jahvu što će im reći. 
\par 15 Jošua uglavi s njima mir i sklopi savez  s njima da će ih poštedjeti. I glavari se na to zakunu. 
\par 16 A poslije tri dana, pošto su sklopili s njima savez,  saznalo se da su im susjedi i da žive usred Izraela. 
\par 17 Tada  krenu Izraelci iz tabora i stignu u njihove gradove, a to su  bili Gibeon, Kefira, Beerot i Kirjat Jearim. 
\par 18 Ali ih nisu  napali sinovi Izraelovi, jer su im se glavari zajednice zakleli  Jahvom, Bogom Izraelovim. Ali sva zajednica poče rogoboriti protiv  glavara. 
\par 19 Tada svi glavari rekoše zajednici: "Mi smo im se zakleli  Jahvom, Bogom Izraelovim, i zato ih ne smijemo dirati. 
\par 20 Evo  što ćemo: pustimo ih da žive, kako nas ne bi stigla srdžba zbog  zakletve kojom smo se zakleli." 
\par 21 Još dometnuše glavari: "Neka  žive i neka budu drvosječe i vodonoše svoj zajednici." Sva zajednica  prihvati što rekoše glavari. 
\par 22 Jošua pozva Gibeonce i reče im: "Zašto nas prevariste  govoreći: 'Vrlo smo daleko od vas', kad eto živite usred nas? 
\par 23 Zato će sada na vama biti kletva i nikada neće nestati među  vama ropstva: bit ćete drvosječe i vodonoše za Dom Boga moga." 
\par 24 Oni odgovore Jošui: "Sa svih strana dolazili su glasovi  nama, slugama tvojim, kako je Jahve, Bog tvoj, odredio Mojsiju, sluzi svomu, da će vam dati svu zemlju i da će istrijebiti ispred  vas sve stanovnike ove zemlje; silno smo se uplašili od vas za  svoje živote i zato smo učinili ovo. 
\par 25 I sada smo, evo, u tvojim  rukama: učini s nama što misliš da je dobro i pravo." 
\par 26 A on im je učinio ovako: izbavio ih iz ruku sinova Izraelovih  te ih nisu pobili. 
\par 27 I od toga dana naredi im Jošua da sijeku  drva i nose vodu, sve do danas, za zajednicu i za žrtvenik Jahvin  na mjestu koje se god izabere. 


\chapter{10}

\par 1 A kad ču jeruzalemski kralj Adoni-Sedek da je Jošua zauzeo  Aj i da ga je izručio "heremu", kletom uništenju, kao što je  učinio s Jerihonom i njegovim kraljem, i da su stanovnici Gibeona  učinili mir s Izraelom i uključili se među njih, 
\par 2 vrlo se uplaši, jer je Gibeon bio značajan kao kakav kraljevski grad, veći od  Aja, a svi žitelji njegovi bijahu ratnici. 
\par 3 Zato jeruzalemski  kralj Adoni-Sedek poruči Hohamu, kralju hebronskom, Piramu, kralju  jarmutskom, Jafiji, kralju lakiškom, i Debiru, kralju eglonskom: 
\par 4 "Dođite k meni i pomozite mi da udarimo na Gibeon, jer je  učinio mir s Jošuom i Izraelcima!" 
\par 5 Udruži se tada pet kraljeva  amorejskih: kralj jeruzalemski, kralj hebronski, kralj jarmutski, kralj lakiški i kralj eglonski; krenu oni i sva njihova vojska, opsjednu grad Gibeon i počnu ga napadati. 
\par 6 Tada Gibeonci poručiše Jošui u tabor u Gilgalu: "Ne napuštaj  svojih slugu, nego se požuri k nama da nas izbaviš i da nam pomogneš, jer su se protiv nas udružili svi amorejski kraljevi koji žive  u Gorju." 
\par 7 I pođe Jošua iz Gilgala, a s njim i svi ratnici, sve vrsni junaci. 
\par 8 A Jahve reče Jošui: "Ne boj se! Ja sam  ih predao u tvoje ruke i nijedan od njih neće se održati pred  tobom." 
\par 9 I udari na njih Jošua iznenadno, pošto je svu noć  išao od Gilgala. 
\par 10 I smete ih Jahve pred Izraelcima, koji ih teško poraziše  kod Gibeona i potjeraše prema strmini kojom se uzlazi u Bet-Horon.  Tukli su ih sve do Azeke i do Makede. 
\par 11 A dok su bježali pred  Izraelom uz bethoronsku strminu, bacao je Jahve s neba na njih  tuču kamenja sve do Azeke te su ginuli. I poginulo ih je više  od tuče kamene nego što su ih pobili sinovi Izraelovi svojim  mačevima. 
\par 12 Onoga dana kada Jahve predade Amorejce sinovima  Izraelovim, obrati se Jošua Jahvi i poviče pred Izraelcima: "Stani, sunce, iznad Gibeona, i mjeseče, iznad dola Ajalona!" 
\par 13 I stade sunce i zaustavi se mjesec sve dok se nije narod  osvetio neprijateljima svojim. Ne piše li to u knjizi Pravednika?  I stade sunce nasred neba i nije se nagnulo k zapadu gotovo cio  dan. 
\par 14 Nije bilo takva dana ni prije ni poslije da bi se Jahve  odazvao glasu čovječjem. Tako je Jahve vojevao za Izraela. 
\par 15 Potom  se vrati Jošua i sav Izrael s njim u tabor gilgalski. 
\par 16 A onih pet kraljeva uteče i sakri se u pećinu kod Makede. 
\par 17 Javiše Jošui: "Otkriveno je pet kraljeva sakrivenih u pećini  kod Makede." 
\par 18 A Jošua reče: "Navalite veliko kamenje pećini  na otvor i postavite ljude pred nju da je čuvaju. 
\par 19 A vi se  drugi ne zadržavajte, nego tjerajte svoje neprijatelje i tucite  ih s leđa; ne dajte im da uđu u svoje gradove, jer ih Jahve,  Bog vaš, predade u vaše ruke." 
\par 20 A kad Jošua i sinovi Izraelovi okončaše bitku teškim  pokoljem - utekla im je samo nekolicina preživjelih u tvrde gradove  - 
\par 21 vrati se narod zdrav i čitav k Jošui u tabor u Makedi.  I nitko više ni da pisne protiv sinova Izraelovih. 
\par 22 Tada reče Jošua: "Otvorite ulaz u pećinu i odande mi  izvedite onih pet kraljeva." 
\par 23 I učine tako, izvedu k njemu  iz pećine onih pet kraljeva: kralja jeruzalemskoga, kralja hebronskoga, kralja jarmutskoga, kralja lakiškog i kralja eglonskog. 
\par 24 A  kad ih izvedoše, pozva Jošua sve Izraelce i reče vojskovođama  koji su ga pratili: "Priđite i stanite svojim nogama na vratove  ovih kraljeva." Oni pristupe i stanu im svojim nogama na vratove. 
\par 25 Reče Jošua: "Ne bojte se i ne plašite se! Hrabri budite i  odlučni, jer će tako Jahve učiniti sa svim vašim neprijateljima  s kojima se budete borili." 
\par 26 Potom Jošua naredi da ih pogube  i objese na pet stabala; i visjeli su ondje do večeri. 
\par 27 A o zalasku sunčanom zapovjedi Jošua te ih skidoše s  drveća i baciše u istu onu pećinu u koju se bijahu sklonili te  na otvor navališe golemo kamenje, koje je i danas ondje. 
\par 28 Istoga dana zauze Jošua Makedu: udari na grad oštricom  mača i pogubi kralja njegova i sve živo u gradu izruči "heremu", kletom uništenju, ne puštajući da itko utekne. I učini s kraljem  makedskim kao što je učinio s kraljem jerihonskim. 
\par 29 Ode zatim Jošua i sav Izrael iz Makede u Libnu i udari  na nju. 
\par 30 I nju Jahve i njena kralja predade u ruke Izraelu, koji oštricom mača pobi sve živo u njoj; ne poštedje nikoga, a s kraljem Libne učini što i s kraljem jerihonskim. 
\par 31 Potom ode Jošua i svi Izraelci iz Libne u Lakiš, opsjede  ga i napade. 
\par 32 Jahve predade Lakiš u ruke Izraela, koji ga  osvoji sutradan: pobiše oštricom mača sve živo u njemu, onako  kao što su učinili s Libnom. 
\par 33 Tada ustade Horam, kralj Gezera, da pomogne Lakišu, ali Jošua porazi njega i njegov narod tako  te nitko ne preživje. 
\par 34 Jošua krenu zatim sa svim Izraelcima od Lakiša na Eglon.  Opsjedoše grad i napadoše ga. 
\par 35 Osvojiše ga još istoga dana  i pobiše sve oštricom mača. Sve živo izručiše kletom uništenju, kako su učinili s Lakišem. 
\par 36 Onda Jošua sa svim Izraelom krenu od Eglona na Hebron  i napade ga. 
\par 37 Osvojiše ga i pobiše sve oštricom mača, kralja  i stanovništvo u svim mjestima koja mu pripadaju, ne poštedjevši  nikoga. Učini s njime kao s Eglonom. Grad sa svim svojim stanovništvom  bi izručen kletom uništenju. 
\par 38 Napokon krenu Jošua i sav Izrael  s njim na Debir i napadoše ga. 
\par 39 Osvojiše ga i razoriše; kralja  njegova i žitelje okolnih mjesta pobiše oštricom mača. Kletom  uništenju izručiše sve njegovo stanovništvo. Ne poštedješe nikoga.  I učini Jošua s Debirom i njegovim kraljem kao što je učinio  s Hebronom i njegovim kraljem, s Libnom i njezinim kraljem. 
\par 40 Tako je Jošua zauzeo sav onaj kraj: Gorje i Negeb, Šefelu  i Visočje - i sve njihove kraljeve. Ne ostavi preživjelih, već  izruči kletom uništenju sve što je disalo, kako je zapovjedio  Jahve, Bog Izraelov. 
\par 41 I pobi ih Jošua sve od Kadeš Barnee  do Gaze i sav kraj Gošen do Gibeona. 
\par 42 Sve tamošnje kraljeve  i zemlje njihove zauze Jošua ujedanput, jer se za Izraela borio  Jahve, Bog Izraelov. 
\par 43 Naposljetku se Jošua i sav Izrael vratiše  u tabor u Gilgalu. 


\chapter{11}

\par 1 Kad je sve to čuo Jabin, kralj od Hasora, obavijesti Jobaba, kralja od Madona, i kralja od Šimrona, i kralja od Akšafa, 
\par 2 i  kraljeve na sjeveru, u Gorju, i u Arabi južno od Kinereta, i  u Šefeli, i na uzvišicama Dora prema moru; 
\par 3 Kanaance na istoku  i zapadu, Amorejce, Hetite, Perižane i Jebusejce u planinama, Hivijce pod Hermonom u zemlji Mispi. 
\par 4 Svi oni izađu sa svim  svojim četama, s mnoštvom što ga bijaše kao pijeska na obali  morskoj i s mnogim konjima i kolima. 
\par 5 Udruže se, dakle, svi ti kraljevi i utabore se zajedno  na vodama Meroma da se bore protiv Izraela. 
\par 6 Tada Jahve reče  Jošui: "Ne boj se njih, jer ću sutra u ovo doba učiniti te će  svi biti pobijeni pred Izraelom; konje njihove osakati, a bojna  im kola ognjem spali." 
\par 7 Jošua povede na njih sve svoje ratnike, iznenada ih napade  na vodama Meroma i udari na njih. 
\par 8 I Jahve ih dade u ruke Izraelcima  te ih oni pobiše i protjeraše sve do Velikog Sidona i do Misrefot  Majima i do ravnice Mispe na istoku; i poraziše ih tako te nitko  ne preživje. 
\par 9 Jošua učini kako mu je Jahve zapovjedio: konje  im osakati, a kola im ognjem spali. 
\par 10 U to se vrijeme vrati Jošua i zauze Hasor, a njegova  kralja pogubi mačem. Hasor je nekoć bio glavni grad svima tim  kraljevstvima. 
\par 11 Pobili su sve oštricom mača, izvršujući "herem", kletvu. Ne ostade ništa živo, a Hasor spališe ognjem. 
\par 12 Sve  gradove onih kraljeva pokori Jošua i pobi kraljeve oštricom mača, izvršujući "herem", kletvu, kao što je bio zapovjedio Mojsije, sluga Jahvin. 
\par 13 Od ostalih gradova koji se dizahu na svojim  brežuljcima Izraelci nisu spalili ni jednoga, osim Hasora, koji  spali Jošua. 
\par 14 Sav plijen iz tih gradova i stoku razdijeliše sinovi  Izraelovi među sobom, a sve ljude pobiše oštricom mača, istrijebiše  ih i ni žive duše ne ostade. 
\par 15 Sve što Jahve bijaše zapovjedio svome sluzi Mojsiju,  zapovjedio je Mojsije Jošui, a Jošua sve izvršio, ne izostavivši  ništa od svega što Jahve bijaše zapovjedio Mojsiju. 
\par 16 Tako  je Jošua zauzeo svu zemlju: Gorje, sav Negeb i svu zemlju Gošen, Šefelu, Arabu, Izraelsko gorje i njegove brežuljke, 
\par 17 od gore  Halaka, koja se diže prema Seiru, pa do Baal Gada, u ravnici  libanonskoj pod gorom Hermonom; zarobio je sve njihove kraljeve, pobio ih i pogubio. 
\par 18 Dugo je vremena ratovao Jošua s tim  kraljevima. 
\par 19 Nije bilo ni jednoga grada koji je sklopio mir  s sinovima Izraelovim, osim Hivijaca, koji su živjeli u Gibeonu:  sve ih zauzeše ratom. 
\par 20 Jahve im bijaše otvrdnuo srca te su  izašli u boj protiv Izraela i pali pod "herem", kletvu bez smilovanja, da budu istrijebljeni, kako je to Jahve bio zapovjedio Mojsiju. 
\par 21 U ono vrijeme dođe Jošua i istrijebi Anakovce iz Gorja, iz Hebrona, iz Debira, iz Anaba, iz svega gorja Judina i iz  svega gorja Izraelova: predade ih "heremu", uništenju, njih i  sve njihove gradove. 
\par 22 Tako ne ostade nijedan Anakovac u svoj  zemlji sinova Izraelovih, osim u Gazi, u Gatu i Ašdodu. 
\par 23 Jošua  zauze svu zemlju, kao što je Jahve bio rekao Mojsiju, i dade  je u baštinu Izraelu podijelivši je po plemenima. I konačno zemlja počinu od rata. 


\chapter{12}

\par 1 Ovo su zemaljski kraljevi što su ih pobijedili Izraelci i  osvojili njihovu zemlju s onu stranu Jordana k istoku, od potoka  Arnona do gore Hermona, sa svom Arabom na istoku: 
\par 2 Sihon, kralj  amorejski, koji je stolovao u Hešbonu; njegovo se kraljevstvo  protezalo od Aroera, koji leži na rubu doline potoka Arnona,  sredinom doline i polovinom Gileada pa do potoka Jaboka, gdje  je graničilo s Amoncima; 
\par 3 i na istoku mu bila Araba do Keneretskog  mora s jedne strane i sve do Arabskog ili Slanog mora prema Bet  Haješimotu; i dalje na jugu do obronaka Pisge. 
\par 4 Međašio s njime Og, kralj bašanski, jedan od posljednjih  Refaimaca; stolovao je u Aštarotu i Edreju. 
\par 5 A vladao je gorom  Hermonom i Salkom, čitavim Bašanom sve do gešurske i maakadske  međe te drugom polovinom Gileada sve do granice Sihona, kralja  hešbonskoga. 
\par 6 Mojsije, sluga Jahvin, i sinovi Izraelovi sve  su ih pobili i predao je Mojsije, sluga Jahvin, tu zemlju u posjed  plemenu Rubenovu i Gadovu plemenu te polovini plemena Manašeova. 
\par 7 A ovo su zemaljski kraljevi što su ih pobijedili Jošua  i sinovi Izraelovi s ovu stranu Jordana k zapadu, od Baal Gada  u libanonskoj ravnici pa do gore Halaka, koja se diže prema Seiru, a tu je zemlju Jošua dao u baštinu plemenima Izraelovim prema  njihovim dijelovima, 
\par 8 u Gorju, u Šefeli, u Arabi i po obroncima, u Pustinji te u Negebu: zemlju hetitsku, amorejsku i kanaansku, perižansku, hivijsku i jebusejsku: 
\par 9 jerihonski kralj, jedan; kralj Aja kod Betela, jedan; 
\par 10 jeruzalemski kralj, jedan; hebronski kralj, jedan; 
\par 11 jarmutski kralj, jedan; lakiški kralj, jedan; 
\par 12 eglonski kralj, jedan; gezerski kralj, jedan; 
\par 13 debirski kralj, jedan; gederski kralj, jedan; 
\par 14 hormski kralj, jedan; aradski kralj, jedan; 
\par 15 kralj Libne, jedan; adulamski kralj, jedan; 
\par 16 makedski kralj, jedan; betelski kralj, jedan; 
\par 17 kralj Tapuaha, jedan; heferski kralj, jedan; 
\par 18 afečki kralj, jedan; šaronski kralj, jedan; 
\par 19 madonski kralj, jedan; hasorski kralj, jedan; 
\par 20 šimron-meronski kralj, jedan; ahšafski kralj, jedan; 
\par 21 tanaački kralj, jedan; megidski kralj, jedan; 
\par 22 kedeški kralj, jedan; kralj Jokneama na Karmelu, jedan; 
\par 23 dorski kralj u pokrajini dorskoj, jedan; gojski kralj u Gilgalu, jedan; 
\par 24 tirški kralj, jedan. U svemu trideset i jedan kralj. 


\chapter{13}

\par 1 Kad je Jošua ostario i odmakao u svojim godinama, reče mu  Jahve: "Već si star i vremešan, a ostalo je mnogo zemlje da se  osvoji. 
\par 2 Evo područja što još preostaju: sve pokrajine filistejske  i sva zemlja gešurska; 
\par 3 od Šihora, što je pred Egiptom, sve  do granice Ekrona na sjeveru, a računa se kao područje Kanaanaca;  pet kneževina filistejskih: Gaza, Ašdod, Aškelon, Git i Ekron;  zatim Avijci 
\par 4 na jugu. Sva zemlja kanaanska od Are koja pripada  Sidoncima, pa do Afeka i do međe amorejske; 
\par 5 onda zemlja Giblijaca  i sav Libanon prema istoku, od Baal Gada u podnožju gore Hermona  do Lebo Hamata. 
\par 6 Sve stanovnike gorja, od Libanona do Misrefota na zapadu  - sve Sidonce otjerat ću ispred sinova Izraelovih. Samo razdijeli  Izraelu zemlju u baštinu, kao što sam ti zapovjedio. 
\par 7 Razdijeli, dakle, tu zemlju u baštinu među devet plemena i polovinu plemena  Manašeova." 
\par 8 Druga polovina plemena Manašeova, a s njome pleme Rubenovo  i Gadovo, primiše svoju baštinu koju im je predao Mojsije preko  Jordana, na istoku. Mojsije, sluga Jahvin, dodijelio im je ovako: 
\par 9 od Aroera, koji se nalazi uz obalu potoka Arnona, i od grada  usred doline, svu visoravan od Medebe do Dibona; 
\par 10 sve gradove  Sihona, kralja amorejskoga, koji je vladao u Hešbonu, sve do  međe sinova Amonovih; 
\par 11 i Gilead, i krajinu gešursku i maakansku  sa svom gorom Hermonom, i sav Bašan do Salke; 
\par 12 a u Bašanu  sve kraljevstvo Oga, koji je vladao u Aštarotu i Edreju i bio  posljednji potomak Refaima. Mojsije ih je pobijedio i protjerao. 
\par 13 Ali sinovi Izraelovi nisu protjerali Gešurce i Maakance,  pa tako ostadoše Gešurci i Maakanci usred Izraela sve do današnjega  dana. 
\par 14 Samo plemenu Levijevu ne dade baštine: Jahve, Bog Izraelov, njegova je baština, kao što je rekao. 
\par 15 Mojsije dade plemenu sinova Rubenovih dijelove po njihovim  porodicama. 
\par 16 Primili su zemlju od Aroera, koji leži uz obalu  potoka Arnona, i od grada koji je u sredini doline i svu visoravan  kod Medebe; 
\par 17 Hešbon sa svim njegovim gradovima koji leže na  visoravni: Dibon, Bamot Baal, Bet Baal Meon; 
\par 18 Jahas, Kedemot, Mefaat; 
\par 19 Kirjatajim, Sibmu i Seret Hašahar na gori iznad  doline; 
\par 20 Bet Peor, Ašdot Hapisgu, Bet Haješimot; 
\par 21 sve gradove  na visoravni i sve kraljevstvo Sihona, amorejskog kralja, koji  je vladao u Hešbonu. Mojsije ga je pobijedio kao i knezove midjanske:  Avija, Rekema, Sura, Hura, Reba, podanike Sihonove, koji su živjeli  u toj zemlji; 
\par 22 i vrača Bileama, sina Beorova, ubili su sinovi  Izraelovi oštricom mača s ostalim žrtvama. 
\par 23 Međa sinova Rubenovih  bijaše Jordan. To je bila baština sinova Rubenovih po njihovim  porodicama: gradovi i sela njihova. 
\par 24 Onda dade Mojsije plemenu Gadovu, sinovima Gadovim, dijelove  po porodicama njihovim. 
\par 25 Primili su u posjed: Jazer i sve  gradove gileadske, polovinu zemlje sinova Amonovih sve do Aroera, nasuprot Rabi, 
\par 26 i od Hešbona do Ramat Hamispe i Betonima, i od Mahanajima do pokrajine Lo-Debar; 
\par 27 a u dolini: Bet Haram, Bet Nimru, Sukot i Safon, to jest ostatak kraljevstva Sihona, kralja hešbonskoga; Jordan s obalom sve do kraja Kineretskoga  mora, na istočnoj strani Jordana. 
\par 28 To je baština sinova Gadovih, po njihovim porodicama, gradovi i sela njihova. 
\par 29 Mojsije je dao dio polovini plemena Manašeova po njegovim  porodicama. 
\par 30 Dobili su u posjed zemlju od Mahanajima, sav  Bašan, sve kraljevstvo Oga, kralja bašanskoga, i sva Sela Jairova  što su u Bašanu - šezdeset gradova. 
\par 31 A polovina Gileada, Aštarot  i Edrej, gradovi kraljevstva Ogova u Bašanu, pripali su sinovima  Makira, sina Manašeova, i to polovini sinova Makirovih po njihovim  porodicama. 
\par 32 Tako je Mojsije bio podijelio baštine na Moapskim poljanama, s druge strane Jordana, istočno od Jerihona. 
\par 33 Levijevu plemenu  ne dade Mojsije baštine: Jahve, Bog Izraelov, njihova je baština, kao što im je sam rekao. 


\chapter{14}

\par 1 Ovo je što su dobili u baštinu sinovi Izraelovi u zemlji kanaanskoj  - što su im razdijelili u baštinu svećenik Eleazar i Jošua, sin  Nunov, i glavari porodica izraelskih plemena. 
\par 2 Ždrijebom su  razdijelili baštinu, kao što je Jahve odredio preko Mojsija,  među devet plemena i polovinu desetoga plemena. 
\par 3 Mojsije je  odredio baštinu dvama plemenima i polovini desetog plemena s  onu stranu Jordana, a levitima nije dao baštine među njima. 
\par 4 Jer  bijahu dva plemena sinova Josipovih: Manašeovo i Efrajimovo.  A levitima nisu dali dijela u zemlji nego gradove za prebivanje  i pašnjake za njihovu stoku i za blago njihovo. 
\par 5 Kako je Jahve  zapovjedio Mojsiju, tako su učinili sinovi Izraelovi pri diobi  zemlje. 
\par 6 Sinovi Judini pristupe k Jošui u Gilgalu, a Kaleb, sin  Jefuneov, Kenižanin, reče mu: "Ti znaš što je Jahve rekao Mojsiju, čovjeku Božjem, za mene i za tebe u Kadeš Barnei. 
\par 7 Bilo mi  je četrdeset godina kad me posla Mojsije, sluga Jahvin, iz Kadeš  Barnee da uhodim zemlju. I donio sam mu izvješće kako sam najbolje  znao. 
\par 8 Braća koja su pošla sa mnom uplašila su srce naroda, ali sam ja vršio volju Jahve, Boga svojega. 
\par 9 I onoga se dana  zakle Mojsije: 'Zemlja kojom je stupala noga tvoja pripast će  tebi i sinovima tvojim u vječnu baštinu, jer si vršio volju Jahve, Boga mojega.' 
\par 10 I vidiš, Jahve me sačuvao u životu, kao što  je rekao. Već je prošlo četrdeset i pet godina kako je Jahve  to obećao Mojsiju, dok je Izrael još išao pustinjom; sada mi  je osamdeset i pet godina, 
\par 11 ali sam još i danas snažan kao  što sam bio onoga dana kad me Mojsije poslao kao uhodu. Kao nekoć, i sada je moja snaga u meni, za borbu, da odem i da se vratim. 
\par 12 Daj mi sada ovo gorje, koje mi je Jahve obećao onoga dana.  Sam si čuo onoga dana. Ondje su Anakovci, a i gradovi su im veliki  i tvrdi. Ako je Jahve sa mnom, protjerat ću ih, kako je to obećao  Jahve." 
\par 13 Tada ga Jošua blagoslovi i dade Kalebu, sinu Jefuneovu, Hebron u baštinu. 
\par 14 Hebron je pripao u baštinu Kalebu, sinu  Jefuneovu, Kenižaninu, sve do danas, jer je Kaleb vršio volju  Jahve, Boga Izraelova. 
\par 15 Hebron se prije zvao Kirjat Arba;  a Arba bijaše velik čovjek među Anakovcima. I počinu zemlja od rata. 


\chapter{15}

\par 1 Dio što je pripao plemenu sinova Judinih, po njihovim porodicama, bijaše prema granici edomskoj, na jug do Sinske pustinje, na  krajnjem jugu. 
\par 2 A južna im međa išla od kraja Slanoga mora  od zaljeva što je na jugu; 
\par 3 izlazila je onda južno od Akrabimskog  uspona, pružala se preko Sina i uzlazila južno od Kadeš Barnee, prelazila Hesron, penjala se k Adari i odatle okretala prema  Karkai, 
\par 4 potom prelazila Asmon i dopirala do Potoka egipatskog  i najposlije izbijala na more. To vam je južna međa. 
\par 5 Na istoku  je međa bila: Slano more do ušća Jordana. Sjeverna je međa počinjala  od Slanog mora kod ušća Jordana. 
\par 6 Odatle je međa uzlazila u  Bet-Hoglu, tekla sjeverno uz Bet-Arabu, išla gore na Kamen Bohana, sina Rubenova. 
\par 7 Međa se zatim dizala od Akorske doline prema  Debiru, okretala na sjever prema Gelilotu, koji leži naprama  Adumimskom usponu, južno od Potoka; dalje je međa prolazila prema  vodama En-Šemeša te izlazila kod En-Rogela. 
\par 8 Odatle se preko  doline Ben-Hinom s juga dizala k Jebusejskom obronku, to jest  k Jeruzalemu. Potom se uspinjala na vrh gore koja prema zapadu  gleda na dolinu Hinon i leži na sjevernom kraju doline Refaima. 
\par 9 S vrha te gore zavijala je međa na izvor Neftoah te izlazila  prema gradovima u gori Efronu da zatim okrene k Baali, to jest  Kirjat Jearimu. 
\par 10 Od Baale međa je okretala na zapad prema  gori Seiru i onda, prolazeći sjeverno od gore Jearima, to jest  Kesalona, spuštala se u Bet-Šemeš te išla k Timni. 
\par 11 Dalje  je međa tekla k sjevernom obronku Ekrona, okretala prema Šikronu, prelazila visove Baale, pružala se do Jabneela da konačno izbije  na more. 
\par 12 Zapadna je međa Veliko more s obalom. To su bile  zemlje sinova Judinih, unaokolo, po porodicama njihovim. 
\par 13 Kaleb, sin Jefuneov, primi dio među sinovima Judinim, kako je Jahve naredio Jošui. Dao mu je Kirjat Arbu, glavni grad  sinova Anakovih - Hebron. 
\par 14 Kaleb protjera odatle tri sina  Anakova: Šešaja, Ahimana i Talmaja, potomke Anakove. 
\par 15 Odatle  krenu na stanovnike Debira, koji se nekoć zvao Kirjat Sefer. 
\par 16 Tada reče Kaleb: "Tko pokori i zauzme Kirjat Sefer, dat ću  mu svoju kćer Aksu za ženu." 
\par 17 Zauze ga Otniel, sin Kenaza, brata Kalebova; i dade mu Kaleb svoju kćer Aksu za ženu. 
\par 18 Kad  je prišla mužu, on je nagovori da u svoga oca zatraži polje.  Ona siđe s magarca, a Kaleb je upita: "Šta hoćeš?" 
\par 19 Ona odgovori:  "Daj mi blagoslov! Kad si mi dao kraj u Negebu, daj mi i koji  izvor vode." I on joj dade Gornje i Donje izvore. 
\par 20 To je bila baština plemena sinova Judinih po porodicama  njihovim. 
\par 21 Međašni su gradovi plemena sinova Judinih, duž edomske  međe prema jugu, bili: Kabseel, Eder, Jagur; 
\par 22 Kina, Dimona, Adada; 
\par 23 Kedeš, Hasor Jitnan; 
\par 24 Zif, Telem, Bealot; 
\par 25 Novi  Hasor, Kirjat Hesron (to jest Hasor); 
\par 26 Amam, Šema, Molada; 
\par 27 Hasar Gada, Hešmon, Bet-Pelet; 
\par 28 Hasar Šual, Beer Šeba  s pripadnim područjima; 
\par 29 Baala, Ijim, Esem; 
\par 30 Eltolad, Kesil, Horma; 
\par 31 Siklag, Madmana, Sansana; 
\par 32 Lebaot, Šelhim, En  Rimon: svega dvadeset i devet gradova s njihovim selima. 
\par 33 U Dolini: Eštaol, Sora, Ašna; 
\par 34 Zanoah, En Ganim, Tapuah, Haenam; 
\par 35 Jarmut, Adulam, Soko, Azeka; 
\par 36 Šaarajim, Aditajim, Hagedera i Gederotajim: četrnaest gradova s njihovim selima. 
\par 37 Senan, Hadaša, Migdal-Gad; 
\par 38 Dilean, Hamispe, Jokteel; 
\par 39 Lakiš, Boskat, Eglon; 
\par 40 Kabon, Lahmas, Kitliš; 
\par 41 Gederot, Bet-Dagon, Naama, Makeda: šesnaest gradova s njihovim selima. 
\par 42 Libna, Eter, Ašan; 
\par 43 Jiftah, Ašna, Nesib; 
\par 44 Keila, Akzib i Mareša: devet gradova s njihovim selima. 
\par 45 Ekron s naseljima i selima njegovim; 
\par 46 od Ekrona pa  do Mora, sve što se nalazi pokraj Ašdoda, s njihovim selima; 
\par 47 Ašdod s naseljima i selima njegovim, Gaza s naseljima i selima  njegovim do Egipatskog potoka i Velikog mora, koje je međa. 
\par 48 A u Gori: Šamir, Jatir, Soko; 
\par 49 Dana, Kirjat Sefer  (to je Debir); 
\par 50 Anab, Eštemoa, Anim; 
\par 51 Gošen, Holon, Gilo:  jedanaest gradova s njihovim selima. 
\par 52 Arab, Duma, Ešean; 
\par 53 Janum, Bet-Tapuah, Afeka, 
\par 54 Humta, Kirjat Arba (to jest Hebron), Sior: devet gradova s njihovim  selima. 
\par 55 Maon, Karmel, Zif, Juta; 
\par 56 Jizreel, Jokdeam, Zanoah; 
\par 57 Hakajin, Gibea, Timna: deset gradova s njihovim selima. 
\par 58 Halhul, Bet-Sur, Gedor; 
\par 59 Maarat, Bet-Anot, Eltekon:  šest gradova s njihovim selima. Tekoa, Efrata (to jest Betlehem), Peor, Etan, Kulon, Tatam, Sores, Karem, Galim, Beter, Manah: jedanaest gradova s njihovim  selima. 
\par 60 Kirjat Baal (to jest Kirjat Jearim) i Haraba: dva grada  s njihovim selima. 
\par 61 U pustinji: Bet Haaraba, Midin, Sekaka; 
\par 62 Hanibšan, Slani grad i En-Gedi: šest gradova s njihovim selima. 
\par 63 A Jebusejce koji su živjeli u Jeruzalemu nisu mogli protjerati  sinovi Judini. Tako su ostali sa sinovima Judinim u Jeruzalemu  sve do danas. 


\chapter{16}

\par 1 Sinovima Josipovim pripao je ždrijebom posjed: od Jordana  kod Jerihona, od Jerihonskih voda na istok, pa pustinjom k Betelskoj  gori; 
\par 2 od Betel-Luza međa se nastavljala područjem Arkijaca  do Atarota. 
\par 3 Potom se spuštala na zapad do jafletske međe,  sve do Donjeg Bet-Horona i do Gezera, odakle je izlazila na more. 
\par 4 To je bila baština Josipovih sinova: Manašea i Efrajima. 
\par 5 Područje sinova Efrajimovih po njihovim porodicama bilo  je ovo: međa baštine njihove prema istoku išla je od Atrot Adara  pa do Gornjega Bet-Horona. 
\par 6 Odatle se pružala do mora ... (išla  na) Mikmetat na sjeveru i zavijala dalje na istok prema Taanat  Šilu i prolazila s istočne strane do Janoaha. 
\par 7 Od Janoaha spuštala  se u Atarot i Naarat i onda, dotičući se Jerihona, udarala na  Jordan. 
\par 8 Od Tapuaha išla je ta međa prema zapadu do potoka  Kane te izbijala na more. To je bila baština plemena sinova Efrajimovih  po njihovim porodicama. 
\par 9 A Efrajimovi su sinovi imali sve te  gradove s njihovim selima i još odvojene gradove usred baštine  sinova Manašeovih. 
\par 10 Ali nisu uspjeli otjerati Kanaanaca koji  su živjeli u Gezeru. Tako su Kanaanci ostali među sinovima Efrajimovim  do danas, ali im bijaše nametnuta tlaka. 


\chapter{17}

\par 1 Ždrijebom je dopao i dio plemenu Manašeovu, jer je Manaše  bio prvenac Josipov. Makiru, prvencu Manašeovu, ocu Gileadovu  - bijaše on ratnik bez premca - pripade Gilead i Bašan. 
\par 2 Dobili  su svoj dio i ostali sinovi Manašeovi po svojim porodicama: sinovi  Abiezerovi, sinovi Helekovi, sinovi Asrielovi; sinovi Šekemovi, sinovi Heferovi i sinovi Šemidini. To su muški potomci Manašea, sina Josipova, po svojim porodicama. 
\par 3 A Selofhad, sin Hefera, sina Gileada, sina Makira, sina Manašeova, nije imao sinova  nego samo kćeri. Evo im imena: Mahla, Noa, Hogla, Milka i Tirsa. 
\par 4 One dođoše pred svećenika Eleazara i pred Jošuu, sina Nunova, i pred glavare govoreći: "Jahve je zapovjedio Mojsiju da se  i nama dade baština među našom braćom." I dadoše im po Jahvinoj  zapovijedi baštinu među braćom njihova oca. 
\par 5 Tako je dopalo  Manašeu deset dijelova, povrh gileadske i bašanske zemlje, koje  su s onu stranu Jordana. 
\par 6 Kćeri Manašeove dobiše baštinu među  njegovim sinovima, a zemlja gileadska pripala je drugim sinovima  Manašeovim. 
\par 7 Međa je Manašeova išla od Ašera do Mikmetata, koji leži  nasuprot Šekemu, a zatim zavijala desno prema Jašibu na izvoru  Tapuahu. 
\par 8 Pokrajina Tapuah pripadaše Manašeu, ali sam Tapuah  na međi Manašeovoj pripadaše sinovima Efrajimovim. 
\par 9 Međa je  silazila do potoka Kane. Južno od potoka bili su i ovi gradovi  što su Efrajimovim sinovima pripadali između Manašeovih gradova;  a zemlja se Manašeova nalazila na sjeveru i izbijala na more. 
\par 10 Područje s juga pripadalo je Efrajimu, na sjeveru Manašeu, a more im bi međa; na sjeveru su graničili s Ašerom, a s Jisakarom  na istoku. 
\par 11 Manašeu pripadahu u Jisakaru i Ašeru: Bet-Šean  sa svojim selima, Jibleam sa svojim selima, stanovnici Dora sa  svojim selima, stanovnici En-Dora sa svojim selima, stanovnici  Taanaka sa svojim selima, stanovnici Megida sa svojim selima;  dakle: tri područja. 
\par 12 Ali Manašeovi sinovi nisu mogli osvojiti  te gradove i zato su Kanaanci ostali u tom kraju. 
\par 13 Ali kad  su ojačali sinovi Izraelovi, nametnuše Kanaancima tlaku, ali  ih nisu uspjeli protjerati. 
\par 14 Obrate se tada Josipovi sinovi Jošui i upitaju: "Zašto  si nam dao u baštinu prema jednom ždrijebu, samo jedan dio, kad  smo mnogobrojni i Jahve nas dosad blagoslivljao?" 
\par 15 Jošua im  odgovori: "Kad ste narod mnogobrojan, pođite u šumu i krčite  ondje sebi zemlje u periškoj i refaimskoj krajini, ako vam je  pretijesna gora Efrajimova." 
\par 16 A sinovi Josipovi rekoše: "Gora  nam ova neće biti dosta, a svi Kanaanci koji žive u ravnici imaju  željezna kola, oni što su u Bet-Šeanu i selima njegovim i oni  koji su u dolini jizreelskoj." 
\par 17 Tada odgovori Jošua domu Josipovu, i Efrajimu i Manašeu:  "Vi ste brojan narod i imate silnu snagu. Zato nećeš dobiti samo  jedan ždrijeb: 
\par 18 neka gora bude tvoja. Ako je šumovita, iskrči  je pa će obronci biti posjed doma tvoga. Istjerat ćeš sigurno  Kanaance ako i imaju željezna kola, ako i jesu jaki." 


\chapter{18}

\par 1 Sabrala se zajednica sinova Izraelovih u Šilo, i ondje razapeše  Šator sastanka. Sva im se zemlja pokorila. 
\par 2 Ali ostade među  sinovima Izraelovim još sedam plemena koja nisu primila svoje  baštine. 
\par 3 Tada im reče Jošua: "Dokle ćete oklijevati da pođete  i zaposjednete zemlju koju vam je dao Jahve, Bog vaših otaca? 
\par 4 Izaberite po tri čovjeka iz svakoga plemena, a ja ću ih poslati  da popišu svu zemlju za diobu. Kad se vrate k meni, 
\par 5 razdijelit  ću zemlju na sedam dijelova. Neka Juda ostane na svome području  na jugu, a Josipov dom neka ostane u svome kraju na sjeveru. 
\par 6 A vi raspišite zemlju na sedam dijelova i donesite mi amo  da bacim ždrijeb za vas ovdje pred Jahvom, Bogom našim. 
\par 7 Leviti  neće imati dijela među vama jer je svećeništvo Jahvino njihova  baština; a Gad, Ruben i polovina plemena Manašeova primili su  svoju baštinu na istočnoj strani Jordana - onu koju im je dao  Mojsije, sluga Jahvin." 
\par 8 Spreme se ti ljudi na put, a Jošua zapovjedi onima koji  su pošli popisati zemlju: "Idite i obiđite svu zemlju i opišite  je, pa se onda vratite k meni da bacim ždrijeb ovdje pred Jahvom  u Šilu." 
\par 9 Odoše oni ljudi, prođoše zemljom i u knjigu popisaše  sve gradove u sedam dijelova, pa se vratiše k Jošui u tabor u  Šilu. 
\par 10 A Jošua baci za njih ždrijeb u Šilu pred Jahvom i ondje  razdijeli Jošua zemlju sinovima Izraelovim po njihovim dijelovima  plemenskim. 
\par 11 I pade ždrijeb na pleme sinova Benjaminovih po njihovim  porodicama: utvrdi se da je njihov dio između dijela sinova Judinih  i sinova Josipovih. 
\par 12 Sjeverna im se međa protezala od Jordana  te išla uza sjeverni obronak Jerihona, uspinjala se sa zapada  na goru i završavala se u pustinji Bet-Avenu. 
\par 13 Odatle je išla  k Luzu, k južnom obronku Luza, to jest Betela; spuštala se zatim  u Atrot-Adar, kraj brda koje je južno od Donjeg Bet-Horona. 
\par 14 Međa  se dalje savijala i okretala sa zapada prema jugu, od gore koja  se diže nasuprot Bet-Horonu s juga, i svršavala se kod Kirjat  Baala, danas Kirjat Jearima, grada sinova Judinih. To je zapadna  strana. 
\par 15 Južna se strana počinjala od granice Kirjat Jearima, pa se pružala na zapad k vrelu Neftoahu; 
\par 16 potom se spuštala  međa do kraja gore koja je prema dolini Ben-Hinomu, na sjeveru  refaimske nizine, silazila zatim u dolinu Hinom uz Jebusejski  obronak i dosegla do izvora Rogela. 
\par 17 Zatim se savijala od  sjevera te izlazila na En-Šemeš i doticala Gelilot, koji se diže  prema Adumimskom usponu, i silazila na Kamen Bohana, sina Rubenova. 
\par 18 Prolazila je zatim obronkom sa sjeverne strane prema Bet-Haarabi  i silazila do Arabe. 
\par 19 Dalje je tekla međa uz obronak Bet-Hogle  prema sjeveru i svršavala se na sjevernom Jeziku Slanog mora, do južnog kraja Jordana. To je južna međa. 
\par 20 Jordan je pak  bio međa s istočne strane. To je baština sinova Benjaminovih, s njihovim međama unaokolo po porodicama njihovim. 
\par 21 Gradovi plemena sinova Benjaminovih po porodicama njihovim  jesu: Jerihon, Bet-Hogla, Emek Kesis; 
\par 22 Bet-Haaraba, Samarajim, Betel; 
\par 23 Avim, Para, Ofra; 
\par 24 Kefar Haamona, Ofni i Gaba:  dvanaest gradova s njihovim selima. 
\par 25 Gibeon, Rama, Beerot; 
\par 26 Mispe, Kefira i Mosa; 
\par 27 Rekem, Jirpeel, Tarala; 
\par 28 Sela Haelef, Jebus (to je Jeruzalem), Gibat  i Kirjat: četrnaest gradova s njihovim selima. To je baština  sinova Benjaminovih po porodicama njihovim. 


\chapter{19}

\par 1 Drugi ždrijeb izađe za Šimuna, za pleme sinova Šimunovih po  porodicama njihovim: njihova je baština bila usred sinova Judinih. 
\par 2 Dodijeljena im je kao baština: Beer Šeba, Šeba, Molada; 
\par 3 Hasar  Šual, Bala, Esem; 
\par 4 Eltolad, Betul, Horma, 
\par 5 Siklag, Bet-Hamarkabot, Hasar Susa, 
\par 6 Bet-Lebaot i Šaruhen: trinaest gradova i njihova  sela. 
\par 7 Ajin, Rimon, Eter i Ašan: četiri grada s njihovim selima. 
\par 8 I sva naselja što su oko tih gradova, do Baalat Beera, Ramat  Negeba. To je baština plemena sinova Šimunovih po porodicama  njihovim. 
\par 9 Baština je sinova Šimunovih bila od dijela sinova  Judinih, jer dio dodijeljen sinovima Judinim bijaše za njih prevelik.  Zato su sinovi Šimunovi dobili svoju baštinu usred njihova područja. 
\par 10 Treći ždrijeb izađe za sinove Zebulunove po porodicama  njihovim: njihovo je područje sezalo do Sarida, 
\par 11 odakle im  se međa na zapadu penjala do Marale, doticala Dabešet i dopirala  do potoka koji je nasuprot Jokneamu. 
\par 12 Od Sarida je međa okretala  prema istoku, sve do međe Kislot Tabora, odakle je izlazila do  Dabrata i uspinjala se do Jafije. 
\par 13 A odatle je išla opet prema  istoku, na Git Hefer i na Ita Kasin, izlazila na Rimon i vraćala  se do Nee. 
\par 14 Onda je okretala sa sjevera oko Hanatona i završavala  se u dolini Jiftah-Elu. 
\par 15 Pa Katat, Nahalal, Šimron, Jidalu  i Betlehem: dvanaest gradova s njihovim selima. 
\par 16 To je bila  baština sinova Zebulunovih po porodicama njihovim: ti gradovi  s njihovim selima. 
\par 17 Četvrti je ždrijeb izašao za Jisakara, za sinove Jisakarove  po njihovim porodicama. 
\par 18 A posjed im je bio: Jizreel, Hakesulot, Šunem; 
\par 19 Hafarajim, Šion, Anaharat; 
\par 20 Harabit, Kišjon, Ebes; 
\par 21 Remet i En-Ganim, En-Hada i Bet-Pases. 
\par 22 Potom međa dotiče  Tabor, Šahasimu i Bet-Šemeš i izlazi na Jordan: šesnaest gradova  s njihovim selima. 
\par 23 To je baština plemena sinova Jisakarovih  po porodicama njihovim: ti gradovi s njihovim selima. 
\par 24 Peti ždrijeb iziđe za pleme sinova Ašerovih po njihovim  porodicama. 
\par 25 Njihova je zemlja bila: Helkat, Hali, Beten,  Akšaf, 
\par 26 Alamelek, Amad, Mišal. Na zapadu je međa doticala  Karmel i Šihor Libnat. 
\par 27 Zatim je okretala prema sunčanom istoku  do Bet-Dagona i doticala se Zebuluna i doline Jiftahela sa sjevera;  protezala se dalje Bet-Haemekom i Neielom i dosezala slijeva  Kabul, 
\par 28 pa Abdon, Rehob, Hamon i Kanu sve do Velikog Sidona. 
\par 29 Međa je tada zavijala prema Rami i do tvrdoga grada Tira  te je okretala prema Hosi i izlazila na more. Obuhvaćala je Mehaleb, Akzib, 
\par 30 Ako, Afek i Rehob: dvadeset i dva grada s njihovim  selima. 
\par 31 To je baština plemena sinova Ašerovih po porodicama  njihovim: ti gradovi i njihova sela. 
\par 32 Šesti ždrijeb izađe za sinove Naftalijeve po njihovim  porodicama. 
\par 33 Njihova međa ide od Helefa i od Hrasta u Saananimu, od Adami Hanekeba i Jabneela do Lakuma i izbija na Jordan. 
\par 34 Potom  međa okreće na zapad k Aznot Taboru i pruža se odande prema Hukoku;  na jugu se dotiče Zebuluna, na zapadu Ašera, na istoku Jordana. 
\par 35 Utvrđeni gradovi bijahu Hasidim, Ser, Hamat, Rakat, Kineret; 
\par 36 Adama, Rama, Hasor, 
\par 37 Kedeš, Edrej, En-Hasor; 
\par 38 Jiron, Migdal-El, Horem, Bet-Anat, Bet-Šemeš: devetnaest gradova s  njihovim selima. 
\par 39 To je baština plemena Naftalijevih sinova  po porodicama njihovim: ti gradovi i njihova sela. 
\par 40 Izađe sedmi ždrijeb za pleme sinova Danovih po porodicama  njihovim. 
\par 41 Područje baštine njihove bilo je: Sora, Eštaol, Ir Šemeš, 
\par 42 Šaalabin, Ajalon, Jitla, 
\par 43 Elon, Timna, Ekron, 
\par 44 Elteke, Gibeton, Baalat, 
\par 45 Jehud, Bene-Berak, Gat-Rimon, 
\par 46 Me-Hajarkon i Harakon s područjem prema Jafi. 
\par 47 Ali područje sinova Danovih bilo je za njih pretijesno;  zato udare Danovi sinovi na Lešem, osvoje ga i sve pobiju oštricom  mača; zaposjednu grad, nastane se u njemu i Lešem prozovu Dan, po imenu Dana, oca svoga. 
\par 48 To je baština plemena sinova Danovih  po porodicama njihovim: ti im gradovi i sela njihova. 
\par 49 Kada završe diobu zemlje ždrijebom i utvrde njezine međe, dadu Izraelci Jošui, sinu Nunovu, baštinu u svojoj sredini. 
\par 50 Po zapovijedi Jahvinoj dali su mu grad koji je sebi želio:  Timnat-Serah u Efrajimovoj gori; on utvrdi taj grad i nastani  se u njemu. 
\par 51 To su baštine koje su svećenik Eleazar i Jošua, sin Nunov, i glavari izraelskih plemena podijelili ždrijebom među plemena  izraelska u Šilu, pred Jahvom, na vratima Šatora sastanka. Tako  je zavšena razdioba zemlje. 


\chapter{20}

\par 1 Jahve reče Jošui: 
\par 2 "Kaži sinovima Izraelovim i reci im:  'Odredite sebi gradove-utočišta za koje sam vam govorio preko  Mojsija, 
\par 3 da bi onamo mogao pobjeći ubojica koji nehotice ubije  koga i da vam budu utočišta od krvnoga osvetnika. 
\par 4 Ako ubojica  utekne u koji od tih gradova, neka stane pred gradska vrata i  neka starješinama toga grada iznese svoju stvar. Oni neka ga  prime u svoj grad i odrede mu mjesto gdje će prebivati među njima. 
\par 5 Ako ga krvni osvetnik progoni, ne smiju izručiti ubojicu u  njegove ruke: tÓa nehotice je ubio svoga bližnjega, a ne iz mržnje. 
\par 6 Ubojica neka ostane u tom gradu sve dok ne stupi pred sud  zajednice ili do smrti velikoga svećenika koji bude u ono vrijeme.  Tada neka se ubojica vrati i neka ode u svoj grad i svome domu  - u grad iz kojega je utekao.'" 
\par 7 I posvete Kedeš u Galileji, u Naftalijevoj gori; Šekem  u Efrajimovoj gori; Kirjat-Arbu, to jest Hebron, u Judinoj gori. 
\par 8 S druge strane Jordana, istočno od Jerihona, odrede Beser  u pustinji, u ravnici plemena Rubenova, i Ramot u Gileadu od  plemena Gadova, i Golan u Bašanu od plemena Manašeova. 
\par 9 To  su bili gradovi određeni svim Izraelcima i došljacima koji borave  među njima: ovamo je mogao uteći svaki koji nehotice drugoga  ubije, a da sam ne pogine od osvetničke ruke dok ne izađe na  sud, pred zajednicu. 


\chapter{21}

\par 1 Pođoše tada glavari levitskih obitelji k svećeniku Eleazaru  i Jošui, sinu Nunovu, i plemenskim glavarima Izraela. 
\par 2 I rekoše im u Šilu, u zemlji kanaanskoj: "Jahve je zapovjedio  preko Mojsija da nam se dadu gradovi gdje ćemo živjeti i pašnjaci  oko njih za našu stoku." 
\par 3 Izraelci dadoše levitima od svoje  baštine, po zapovijedi Jahvinoj, ove gradove s njihovim pašnjacima. 
\par 4 Iziđe, dakle, ždrijeb za porodice Kehatove: levitima, potomcima  svećenika Arona, pripade trinaest gradova od plemena Judina,  Šimunova i Benjaminova; 
\par 5 ostalim sinovima Kehatovim pripalo  je ždrijebom po porodicama deset gradova od plemena Efrajimova  i Danova i od polovine plemena Manašeova. 
\par 6 Sinovi Geršonovi  dobiše po porodicama trinaest gradova od plemena Jisakarova,  Ašerova i Naftalijeva i od polovine plemena Manašeova u Bašanu. 
\par 7 Merarijevim sinovima po njihovim porodicama pripalo je dvanaest  gradova od plemena Rubenova, Gadova i Zebulunova. 
\par 8 Tako Izraelci ždrijebom dodijeliše levitima te gradove  s pašnjacima, kako bijaše zapovjedio Jahve preko Mojsija. 
\par 9 Od plemena sinova Judinih i od plemena sinova Šimunovih  dodijeljeni su bili ovi gradovi koji se poimence navode: 
\par 10 sinovima  Aronovim u levitskim porodicama Kehatovim, jer je prvi ždrijeb  bio za njih, 
\par 11 pripade Kirjat-Arba, glavni grad Anakovaca,  to jest Hebron, u Judinoj gori, s pašnjacima unaokolo. 
\par 12 Ali  polja oko toga grada sa selima unaokolo bila su već dana u baštinu  Kalebu, sinu Jefuneovu. 
\par 13 Sinovima svećenika Arona pripade  grad-utočište Hebron s pašnjacima i Libna s pašnjacima; 
\par 14 Jatir  s pašnjacima, Eštemoa s pašnjacima, 
\par 15 Holon s pašnjacima, Debir  s pašnjacima, 
\par 16 Ašan s pašnjacima, Juta s pašnjacima, Bet-Šemeš  s pašnjacima. Dakle, devet gradova od ona dva plemena. 
\par 17 Od  plemena Benjaminova: Gibeon s pašnjacima, Geba s pašnjacima, 
\par 18 Anatot s pašnjacima, Almon s pašnjacima. Dakle, četiri grada. 
\par 19 Tako su svećenici, sinovi Aronovi, dobili svega trinaest  gradova s njihovim pašnjacima. 
\par 20 Ostalim levitima u porodicama sinova Kehatovih ždrijebom  su pripali gradovi plemena Efrajimova. 
\par 21 Dali su im grad-utočište  Šekem s pašnjacima njegovim na Efrajimovoj gori, zatim Gezer  s pašnjacima, 
\par 22 Kibsajim s pašnjacima, Bet-Horon s pašnjacima.  Dakle, četiri grada. 
\par 23 Od plemena Danova dobili su: Elteku  s pašnjacima i Gibeton s pašnjacima, 
\par 24 Ajalon s pašnjacima  i Gat-Rimon s pašnjacima. Dakle, četiri grada. 
\par 25 Od polovine  plemena Manašeova: Tanak s pašnjacima i Jibleam s pašnjacima.  Dakle, dva grada. 
\par 26 U svemu: deset su gradova s pašnjacima  dobile porodice ostalih sinova Kehatovih. 
\par 27 Geršonovim sinovima, porodicama levitskim, dadoše od  polovine plemena Manašeova grad-utočište Golan u Bašanu i Aštarot  s njihovim pašnjacima. Dakle, dva grada. 
\par 28 Od plemena Jisakarova:  Kišon s pašnjacima, Dabrat s pašnjacima, 
\par 29 Jarmut s pašnjacima  i En-Ganim s pašnjacima. Dakle, četiri grada. 
\par 30 Od plemena  Ašerova: Mišal s pašnjacima, Abdon s pašnjacima, 
\par 31 Helkat s  pašnjacima i Rehob s pašnjacima. Dakle, četiri grada. 
\par 32 Od  plemena Naftalijeva: grad-utočište Kedeš u Galileji s pašnjacima, Hamot Dor s pašnjacima i Kartan s pašnjacima. Dakle, tri grada. 
\par 33 Svega Geršonovih gradova po porodicama njihovim bijaše trinaest  gradova s pašnjacima. 
\par 34 Porodicama sinova Merarijevih, preostalim levitima, dali  su od plemena Zebulunova: Jokneam s pašnjacima, Kartu s pašnjacima, 
\par 35 Rimon s pašnjacima, Nahalal s pašnjacima. Dakle, četiri  grada. 
\par 36 S onu stranu Jordana od plemena Rubenova dadoše im  grad-utočište Beser s pašnjacima na pustinjskoj visoravni, Jahas  s pašnjacima, 
\par 37 Kedemot s pašnjacima, Mefaat s pašnjacima.  Dakle, četiri grada. 
\par 38 Od plemena Gadova: grad-utočište Ramot  u Gileadu s pašnjacima, Mahanajim s pašnjacima, 
\par 39 Hešbon s  pašnjacima, Jazer s pašnjacima. Dakle, četiri grada. 
\par 40 U svemu  bijaše dodijeljeno ždrijebom porodicama sinova Merarijevih, preostalim  levitima, dvanaest gradova. 
\par 41 Tako usred baštine sinova Izraelovih bijaše četrdeset  i osam levitskih gradova s pašnjacima. 
\par 42 Svaki je taj grad  imao pašnjake unaokolo. Tako je bilo sa svima spomenutim gradovima. 
\par 43 Tako je Jahve predao Izraelcima svu zemlju za koju se  zakleo da će je dati ocima njihovim. Primili su je u posjed i  nastanili se u njoj. 
\par 44 I dade im Jahve da otpočinu u miru na  svim međama, kako se bijaše zakleo njihovim ocima. Nitko im od  njihovih neprijatelja ne bijaše kadar odoljeti. Sve im je njihove  neprijatelje predao Jahve u ruke. 
\par 45 Od svih obećanja što ih  je Jahve dao domu Izraelovu nijedno ne osta neispunjeno. Sve  se ispunilo. 


\chapter{22}

\par 1 Tada sazove Jošua sinove Rubenove i Gadove i polovinu plemena  Manašeova 
\par 2 i reče im: "Izvršili ste sve što vam je Mojsije, sluga Jahvin, zapovjedio i poslušali ste me u svemu što sam  vam zapovjedio. 
\par 3 Niste ostavili svoje braće unatoč dugom vojevanju  do današnjega dana i vršili ste vjerno zapovijedi Jahve, Boga  svojega. 
\par 4 Sada je Jahve, Bog vaš, dao mir braći vašoj, kako  im bijaše obećao. Vratite se sada u svoje šatore, u zemlju koju  vam je dao Mojsije, sluga Jahvin, u baštinu s onu stranu Jordana. 
\par 5 Samo pazite da vršite zapovijedi i Zakon što vam ga dade Mojsije, sluga Jahvin: da ljubite Jahvu, Boga svojega, da uvijek idete  putovima njegovim, da čuvate zapovijedi njegove, da se držite  uz njega i da mu služite svim srcem i svom dušom." 
\par 6 I blagoslovi  ih Jošua i otpusti, a oni se zatim vrate u svoje šatore. 
\par 7 Mojsije bijaše jednoj polovini plemena Manašeova dao baštinu  u Bašanu; a drugoj polovini dade je Jošua usred njihove braće  zapadno od Jordana. Otpuštajući ih u njihove šatore, Jošua ih  blagoslovi. 
\par 8 I reče im: "Vratite se u svoje šatore s velikim  blagom i s mnogom stokom, sa srebrom, zlatom, tučem, željezom  i haljinama u izobilju i podijelite plijen od neprijatelja svojih  s braćom svojom." 
\par 9 Vratiše se sinovi Rubenovi i sinovi Gadovi i polovina  plemena Manašeova; odoše od sinova Izraelovih iz Šila u zemlji  kanaanskoj da krenu u zemlju gileadsku, na svoju baštinu koju  su zaposjeli, kako im je zapovjedio Jahve preko Mojsija. 
\par 10 Kad  su stigli do jordanskog područja u zemlji kanaanskoj, podigoše  sinovi Rubenovi, sinovi Gadovi i polovina plemena Manašeova žrtvenik  na Jordanu, žrtvenik velik, izdaleka se vidio. 
\par 11 Čuli Izraelci  gdje se govori: "Evo, sinovi Rubenovi, sinovi Gadovi i polovina  plemena Manašeova podigoše žrtvenik prema zemlji kanaanskoj,  kod Jordana, na izraelskoj strani." 
\par 12 Na to se skupi sva zajednica sinova Izraelovih u Šilu  da pođu u boj na njih. 
\par 13 Izraelci sinovima Rubenovim, sinovima Gadovim i polovini  plemena Manašeova u gileadsku zemlju poslaše Pinhasa, sina svećenika  Eleazara, 
\par 14 i s njime deset knezova, po jednoga rodovskog glavara  od svakoga plemena Izraelova, a svaki je od njih bio glavar obitelji  među tisućama porodica Izraelovih. 
\par 15 I kad oni dođoše k sinovima  Rubenovim, sinovima Gadovim i polovini plemena Manašeova u zemlju  gileadsku, rekoše im: 
\par 16 "Evo što veli sva zajednica Jahvina:  'Što znači nevjera koju činite protiv Jahve, Boga Izraelova?  Zašto se odvrgoste danas od Jahve i, podigavši žrtvenik, zašto  se bunite protiv Jahve? 
\par 17 Zar vam nije dosta zločina iz Peora, od kojega se nismo očistili do dana današnjega i zbog kojega  je došao pomor na zajednicu Jahvinu? 
\par 18 Ako se danas odvraćate  od Jahve i bunite se danas protiv njega, neće li se sutra izliti  njegov gnjev na svu zajednicu Izraelovu? 
\par 19 Ili vam je možda  zemlja vaše baštine nečista? Onda prijeđite u zemlju baštine  Jahvine, u kojoj je Jahvino Prebivalište, i prebivajte među nama.  Ali se ne bunite protiv Jahve i ne bunite se protiv nas dižući  sebi žrtvenik mimo žrtvenik Jahve, Boga našega. 
\par 20 Nije li se  Akan, Zerahov sin, sam ogriješio o 'herem' te se oborila srdžba  na svu zajednicu Izraelovu? Zar nije umro zbog krivice svoje?'" 
\par 21 Tada odgovoriše sinovi Rubenovi, sinovi Gadovi i polovina  plemena Manašeova govoreći plemenskim glavarima Izraelovim: 
\par 22 "Bog, Bog Jahve, Bog nad bogovima, Jahve zna i neka zna Izrael: ako  je to bila pobuna ili nevjernost prema Jahvi, neka nam uskrati  svoju pomoć danas; 
\par 23 ako smo podigli žrtvenik da se odvrgnemo  od Jahve i da prinosimo žrtve paljenice, prinosnice i žrtve pričesnice, neka nam onda sudi Jahve! 
\par 24 Učinismo to od brige i skrbi i  rekosmo: 'Jednoga će dana sinovi vaši reći našima: Što vam je  zajedničko s Jahvom, Bogom Izraelovim? 
\par 25 Zar nije, sinovi Rubenovi  i sinovi Gadovi, postavio Jahve između vas i nas među našu -  Jordan? Vi nemate dijela s Jahvom.' I tako bi sinovi vaši mogli  učiniti da se sinovi naši odvrate te ne štuju Jahvu. 
\par 26 Zato  smo rekli: 'Podignimo žrtvenik, ali ne za žrtve paljenice niti  za klanice, 
\par 27 nego da bude svjedočanstvo između nas i vas,  među potomcima našim, da želimo služiti Jahvi paljenicama, klanicama  i pričesnicama. Tako da ne mognu jednom vaši sinovi reći našima:  Nemate dijela s Jahvom.' 
\par 28 Ako bi kada tako rekli nama i potomcima  našim, mogli bismo odgovoriti: 'Pogledajte slog žrtvenika Jahvina  što su ga podigli oci naši ne za žrtve paljenice ni klanice,  nego za svjedočanstvo između nas i vas.' 
\par 29 Nije nam ni na kraj  pameti pomisao da se bunimo protiv Jahve i da se odvraćamo od  njega dižući žrtvenik za žrtve paljenice, prinosnice i klanice, mimo žrtvenik Jahve, Boga našega, koji je pred njegovim Prebivalištem!" 
\par 30 Kad svećenik Pinhas, knezovi zbora i glavari izraelskih  plemena koji su bili s njim čuše riječi koje im rekoše sinovi  Gadovi, sinovi Rubenovi i sinovi Manašeovi, umiriše se. 
\par 31 Tada  svećenik Pinhas, sin Eleazarov, odgovori sinovima Rubenovim,  sinovima Gadovim i sinovima Manašeovim: "Spoznali smo sada da  je Jahve među nama, jer mu se niste iznevjerili: tako ste sačuvali  sinove Izraelove od kazne Jahvine." 
\par 32 Svećenik Pinhas, sin Eleazarov, i knezovi odoše od sinova  Rubenovih i sinova Gadovih i vratiše se iz zemlje gileadske u  kanaansku k sinovima Izraelovim i kazaše im odgovor. 
\par 33 Izraelovim  sinovima bijaše drag taj odgovor: hvalili su Boga i odustali  su od nauma da udare na njih i da opustoše zemlju u kojoj su  živjeli sinovi Rubenovi i sinovi Gadovi. 
\par 34 Sinovi Rubenovi  i sinovi Gadovi nazvali su žrtvenik "Ed" - "Svjedočanstvo", jer  rekoše: "To je svjedočanstvo među nama: Jahve je Bog." 


\chapter{23}

\par 1 Proteklo je mnogo dana kako je Jahve dao Izraelu da otpočine  od svih neprijatelja unaokolo. I Jošua bijaše ostario, zašao  u godine. 
\par 2 Dozva zato Jošua sve Izraelce, starješine, glavare, suce i upravitelje njihove i reče im: "Ostario sam i odmakao  u godinama. 
\par 3 Vi ste bili svjedoci svega što je Jahve, Bog vaš, pred vašim očima učinio svim narodima radi vas: Jahve, Bog vaš, borio se za vas. 
\par 4 Vidite, razdijelio sam ždrijebom u baštinu  vašim plemenima sve narode koji su ostali i sve one narode koje  sam istrijebio od Jordana do Velikog mora na zapadu. 
\par 5 Jahve, Bog vaš, sam će ih goniti ispred vas i otjerat će ih ispred  vas i zaposjest ćete njihovu zemlju, kao što vam je obećao Jahve, Bog vaš. 
\par 6 Budite, dakle, postojani i sve čvršći u tome da čuvate  i vršite sve što je napisano u Knjizi zakona Mojsijeva i da ne  odstupite od toga ni desno ni lijevo. 
\par 7 Ne miješajte se s tim  narodima koji ostadoše među vama; i ne spominjite imena njihovih  bogova niti se kunite njima; nemojte im služiti i ne klanjajte  se njima. 
\par 8 Nego se držite Jahve, Boga svoga, kako ste činili  do danas. 
\par 9 Jahve je protjerao ispred vas velike i moćne narode  i nitko se nije do danas mogao održati pred vama. 
\par 10 Jedan je  od vas tjerao pred sobom tisuću, jer se Jahve, Bog vaš, borio  za vas, kao što vam je obećao. 
\par 11 Brižno pazite da ljubite Jahvu, Boga svojega, jer se radi o vašem životu. 
\par 12 Jer ako se odmetnete i prionete uz ostatak onih naroda  koji preostaše među vama i s njima se povežete tazbinom i pomiješate  se s njima i oni s vama, 
\par 13 znajte dobro da će Jahve, Bog vaš, prestati goniti te narode ispred vas; oni će vam postati zamka  i mreža, bit će bič bokovima vašim i trnje očima vašim, sve dok  se ne iselite iz ove dobre zemlje koju vam dade Jahve, Bog vaš. 
\par 14 Evo, ja krećem danas na put kojim je svima poći. Spoznajte  i priznajte svim srcem svojim i svom dušom svojom: ni jedno od  svih obećanja koja vam je dao Jahve, Bog vaš, nije ostalo neispunjeno. 
\par 15 I kao što vam se ispunilo svako obećanje što vam ga je dao  Jahve, Bog vaš, tako će Jahve ispuniti i svaku prijetnju dok  vas ne izbriše s lica ove dobre zemlje koju vam je dao Jahve, Bog vaš. 
\par 16 Ako prekršite Savez koji je Jahve, Bog vaš, sklopio  s vama; ako budete služili drugim bogovima i klanjali se njima, buknut će gnjev Jahvin na vas i nestat će vas ubrzo iz dobre  zemlje koju vam je Jahve dao." 



\chapter{24}

\par 1 Jošua potom sabra sva plemena Izraelova u Šekem; i sazva starješine  Izraelove, glavare, suce i upravitelje njihove i oni stadoše  pred Bogom. 
\par 2 Tada reče Jošua svemu narodu: "Ovako veli Jahve, Bog Izraelov: 'Nekoć su oci vaši, Terah, otac Abrahamov i Nahorov, živjeli s onu stranu Rijeke i služili  drugim bogovima. 
\par 3 Ali sam ja uzeo oca vašega Abrahama s one  strane Rijeke i proveo ga kroza svu zemlju kanaansku, umnožio  mu potomstvo i dao mu Izaka. 
\par 4 Izaku dadoh Jakova i Ezava. Ezavu  sam dao goru Seir u posjed. Jakov i sinovi njegovi otišli su  u Egipat. 
\par 5 Tada sam poslao Mojsija i Arona i udario sam Egipat  kaznama koje sam učinio u njemu i tada sam vas izveo. 
\par 6 Izveo  sam oce vaše iz Egipta i stigli su na more; Egipćani su progonili  vaše oce bojnim kolima i konjanicima sve do Mora crvenoga. 
\par 7 Zavapili  su tada Jahvi i on je razvukao gustu maglu između njih i Egipćana  i naveo ih u more koje ih je prekrilo. Vidjeli ste svojim očima  što sam učinio Egipćanima; zatim ste ostali dugo vremena u pustinji. 
\par 8 Nato sam vas uveo u zemlju Amorejaca, koji žive s onu stranu  Jordana. Zaratiše s vama i ja ih dadoh u vaše ruke; uzeli ste  u baštinu zemlju njihovu jer sam ih ja ispred vas uništio. 
\par 9 Tada  se digao moapski kralj Balak, sin Siporov, da ratuje s Izraelom  i on pozva Bileama, sina Beorova, da vas prokune. 
\par 10 Ali ja  ne htjedoh poslušati Bileama: morade vas on i blagosloviti, i  spasih vas iz njegove ruke. 
\par 11 Onda ste prešli preko Jordana  i došli u Jerihon, ali su glavari Jerihona poveli rat protiv  vas - kao i Amorejci, Perižani, Kanaanci, Hetiti, Girgašani,  Hivijci i Jebusejci - ali sam ih ja predao u vaše ruke. 
\par 12 Pred  vama sam poslao stršljene koji su ispred vas tjerali dva kralja  amorejska: nemaš što zahvaliti svome maču ni svome luku. 
\par 13 Dao  sam vam zemlju za koju se niste trudili i gradove koje niste  gradili i u njima se nastaniste; i vinograde vam dadoh i maslinike  koje niste sadili, a danas vas hrane.' 
\par 14 I zato se sada bojte Jahve i služite mu savršeno i vjerno!  Uklonite bogove kojima su služili oci vaši s onu stranu Rijeke  i u Egiptu i služite Jahvi! 
\par 15 Međutim, ako vam se ne sviđa  služiti Jahvi, onda danas izaberite kome ćete služiti: možda  bogovima kojima su služili vaši oci s onu stranu Rijeke ili bogovima  Amorejaca u čijoj zemlji sada prebivate. Ja i moj dom služit  ćemo Jahvi." 
\par 16 Narod odgovori: "Daleko neka je od nas da ostavimo Jahvu  a služimo drugim bogovima. 
\par 17 Jahve, Bog naš, izveo je nas i  naše oce iz Egipta, iz doma robovanja, i on je pred našim očima  učinio velika čudesa i čuvao nas cijelim putem kojim smo išli  i među svim narodima kroz koje smo prolazili. 
\par 18 Još više: Jahve  je ispred nas protjerao sve narode i Amorejce, koji su živjeli  u ovoj zemlji. I mi ćemo služiti Jahvi jer je on Bog naš." 
\par 19 Tada reče Jošua narodu: "Vi ne možete služiti Jahvi,  jer je on Bog sveti, Bog ljubomorni, koji ne može podnijeti vaših  prijestupa ni vaših grijeha. 
\par 20 Ako ostavite Jahvu da biste  služili tuđim bogovima, okrenut će se protiv vas i uništit će  vas, pošto vam je bio dobro činio." 
\par 21 A narod odgovori Jošui: "Ne, mi ćemo služiti Jahvi!" 
\par 22 Na to će Jošua narodu: "Sami ste protiv sebe svjedoci  da ste izabrali Jahvu da mu služite." Odgovoriše mu: "Svjedoci  smo." 
\par 23 "Maknite, dakle, tuđe bogove koji su među vama i priklonite  svoja srca Jahvi, Bogu Izraelovu." 
\par 24 Odgovori narod Jošui:  "Služit ćemo Jahvi, Bogu svojemu, i glas ćemo njegov slušati." 
\par 25 Tako sklopi Jošua toga dana Savez s narodom i utvrdi  mu uredbu i zakon. Bilo je to u Šekemu. 
\par 26 Jošua upisa te riječi  u Knjigu zakona Božjega. Zatim uze velik kamen i stavi ga ondje  pod hrast koji bijaše u svetištu Jahvinu. 
\par 27 Zatim reče Jošua  svemu narodu: "Gle, ovaj kamen neka nam bude svjedokom jer je  čuo riječi što ih je govorio Jahve; on će biti svjedok da ne  zatajite Boga svoga." 
\par 28 Tada Jošua otpusti narod, svakoga na  njegovu baštinu. 
\par 29 Poslije ovih događaja umrije Jošua, sin Nunov, sluga  Jahvin, u dobi od sto deset godina. 
\par 30 Sahraniše ga u kraju  što ga je baštinio u Timnat Serahu, u Efrajimovoj gori, sjeverno  od gore Gaaša. 
\par 31 Izrael je služio Jahvi svega vijeka Jošuina  i svega vijeka starješina koje su Jošuu nadživjele i vidjele  sva djela što ih je Jahve učinio Izraelu. 
\par 32 Kosti Josipove, koje su sinovi Izraelovi sa sobom donijeli  iz Egipta, pokopali su u Šekemu, na zemljištu koje Jakov bijaše  kupio od sinova Hamora, oca Šekemova, za stotinu srebrnjaka i  koje je pripalo u baštinu sinova Josipovih. 
\par 33 Umrije i Eleazar, sin Aronov, i pokopaše ga u Gibei, koja je pripadala njegovu  sinu Pinhasu a nalazila se u Efrajimovoj gori. 





\end{document}