\begin{document}

\title{Tužaljke}


\chapter{1}

\par 1 Kako osamljena sjedi prijestolnica, nekoć naroda puna; postade kao udovica, nekoć velika među narodima. Vladarica nad pokrajinama, na tlaku sad ide. 
\par 2 Noći provodi gorko plačući, suzama pokriva obraze. Nikog nema da je utješi, od svih koji su je ljubili. Svi je prijatelji iznevjeriše i postaše joj neprijatelji. 
\par 3 Izagnan je Juda, u nevolji je i u progonstvu teškom. Sad živi među poganima, ne nalazi počinka. Svi ga gonitelji sustižu u tjesnacima. 
\par 4 Putovi sionski tuguju jer nitko ne dolazi na svetkovine. Sva su vrata razvaljena, svećenici uzdišu, ucviljene su djevice njegove, a on je pun gorčine. 
\par 5 Tlačitelji njegovi sada gospodare, neprijatelji likuju: Jahve ga ucvili zbog grijeha njegovih premnogih. Djeca mu otišla u izgnanstvo pred tlačiteljem. 
\par 6 Povukla se od Kćeri sionske sva slava njezina. Knezovi joj postadoše k'o ovnovi koji paše ne nalaze; nemoćni vrludaju ispred goniča. 
\par 7 Jeruzalem se spominje danÄa bijede i lutanja, kad mu narod dušmanu u ruke pade a nitko mu pomoći ne pruži. Tlačitelji ga gledahu smijući se njegovoj propasti. 
\par 8 Teško sagriješi Jeruzalem, postade kao nečistoća ženina. Svi što ga štovahu, sada ga preziru: jer vidješe golotinju njegovu. On samo plače i natrag se okreće. 
\par 9 Skuti su mu uprljani, nije ni sanjao što ga čeka. Duboko je pao, a nikog da ga tješi. "Pogledaj, Jahve, moju nevolju: jer neprijatelj likuje." 
\par 10 Neprijatelj poseže rukom za svim dragocjenostima njegovim. Gledao je gdje pogani provaljuju u njegovo Svetište, oni kojima si zabranio i pristup u svoj zbor. 
\par 11 Sav narod njegov jeca, tražeći kruha; svi daju dragulje za hranu da bi ponovo živnuli. Evo, Jahve, pogledaj kako sam prezren. 
\par 12 Svi vi što putem prolazite, pogledajte i vidite ima li boli kakva je bol kojom sam ja pogođen, kojom me Jahve udari u dan žestokog gnjeva svoga! 
\par 13 S visine pusti oganj, utjera ga u kosti moje. Pred noge mrežu mi razape i tako me nauznak obori; ucvili me, ožalosti za sva vremena. 
\par 14 Natovario me mojim grijesima, rukom ih svojom pritegnuo; na vrat mi ih navalio, snagu mi oduzeo. Predao me Gospod u ruke njihove, ne mogu se uspraviti. 
\par 15 Sve junake iz moje sredine Gospod odbaci: digao je zbor protiv mene da uništi uzdanicu moju. U tijesku izgazi Gospod mene, djevicu, kćerku Judinu. 
\par 16 Zato moram plakati, oči mi suze liju, jer daleko je od mene moj tješitelj da mi duh povrati. Sinovi su moji poraženi, odveć silan bijaše neprijatelj. 
\par 17 Sion pruža ruke: nema mu tješitelja. Jahve je protiv Jakova sa svih strana pozvao tlačitelje; i tako Jeruzalem postade među njima strašilo. 
\par 18 Jahve, on je pravedan; jer riječi se njegovoj protivih. Oh, čujte, narodi svi, gledajte moju bol: djevice moje, moji mladići, svi odoše u izgnanstvo! 
\par 19 Pozvah sve ljubavnike svoje, ali me oni prevariše. Moji svećenici i starješine pogiboše u gradu tražeći hrane da bi ponovo živnuli. 
\par 20 Pogledaj, Jahve, u kakvoj sam tjeskobi, moja utroba strepi, srce mi se u grudima grči jer bijah opako prkosan! Vani mač pokosi moje sinove, a unutra - smrt. 
\par 21 Čuj kako stenjem: nema mi tješitelja! Svi neprijatelji čuju za moju nesreću i likuju što si to učinio! Daj da dođe dan što si ga objavio, da njima bude kao meni. 
\par 22 Neka se pokaže sva njina zloća pred licem tvojim, a onda postupaj s njima kao što si sa mnom postupio za sve grijehe moje! Jer samo uzdišem, a srce moje tuguje. 


\chapter{2}

\par 1 Kako mrakom zastrije Gospod u svom gnjevu Kćer sionsku. S neba na zemlju baci slavu Izraelovu! I ne sjeti se podnožja svoga u dan gnjeva svojega! 
\par 2 Bez milosti Gospod satrije sve stanove Jakovljeve, u svom gnjevu razori tvrđave kćeri Judine; sa zemljom je sravnio i prokleo kraljevstvo i njegove knezove. 
\par 3 U rasplamtjelom gnjevu svojem razbi svu snagu Izraelovu, povuče svoju desnicu pred neprijateljem; u Jakovu raspiri plamen ognjeni koji sve uokolo proždire. 
\par 4 Nategao je luk k'o neprijatelj, kao dušman ispružio desnicu, ubijajući sve što mu drago bijaše. Na šator Kćeri sionske sasu k'o oganj gnjev svoj jarosni. 
\par 5 K'o neprijatelj Gospod bijaše: razorio je Izraela, razorio sve dvore njegove, porušio njegove utvrde, umnožio kćeri Judinoj uzdisaje i jecaje. 
\par 6 Kao vrtu razvali mu sjenicu, razori mjesto sastanka. Baci Jahve u zaborav svetkovine i subote na Sionu; u gnjevu svojem prezre kralja i svećenika. 
\par 7 Svoj oltar je Gospod odbacio, zgadilo mu se Svetište njegovo. U ruke neprijatelja je predao bedeme svoje i dvorove. Bučili su u Domu Jahvinu, kao u dan blagdanji. 
\par 8 Jahve naumi razvaliti zidove Kćeri sionske. Nape uže mjerničko, ne ustegnu ruku od rušenja. Predziđe, zidine zavi u tugu: oronuše zajedno. 
\par 9 Vrata njina utonuše u zemlju, on im je razbio zasune; kralj i knezovi su među pucima, Zakona nema! Ni u prorokÄa više se ne nalaze viđenja Jahvina. 
\par 10 Starješine Kćeri sionske na zemlji sjede i šute, posiplju glavu prašinom, kostrijet pripasuju. K zemlji glave obaraju djevice jeruzalemske. 
\par 11 Iščilješe mi oči od suza, utroba moja ustreptala, jetra mi se na zemlju prosula zbog sloma kćeri naroda mojega, jer djeca i dojenčad umiru po trgovima Grada. 
\par 12 Govore majkama svojim: "Gdje je žito i vino?" dok obamiru kao ranjeni po trgovima Grada, dok ispuštaju dušu svoju na grudima matera svojih. 
\par 13 S čime da te prispodobim? Na koga si nalik, Kćeri jeruzalemska? S kime da te usporedim, kako utješim, djevice, Kćeri sionska? Jer kao more tvoja je nesreća neizmjerna. Tko će te iscijeliti? 
\par 14 Viđenja tvojih proroka bijahu varka i laž, oni nisu objavili krivnju tvoju da te od izgnanstva odvrate. Varali su te utvarama lažnim i zamamnim. 
\par 15 Nad tobom plješću rukama svi koji putem prolaze, zvižde i vrte glavom zbog Kćeri jeruzalemske: "Je li to grad na glasu ljepotom, radost svemu svijetu?" 
\par 16 Na tebe otvaraju usta svi neprijatelji tvoji, zvižde, škrguću zubima i govore: "Proždrijesmo je! To je dan za kojim čeznusmo, doživjesmo, vidjesmo!" 
\par 17 Jahve izvrši naum svoj, održa svoju riječ koju naredi u davnim danima: nemilice te razorio. Neprijatelj likuje zbog tebe, tvoj protivnik rog svoj podiže. 
\par 18 U sav glas viči Gospodu, jecaj, Kćeri sionska! Neka k'o potok teku tvoje suze danju i noću. Ne daj počinka sebi, neka se zjenica oka tvoga ne odmori. 
\par 19 Ustani, viči noću za svake promjene straže. K'o vodu izlij srce pred licem Gospodnjim, k njemu podiži ruke i traži milost za svoju nejačad koja od glada obamire po uglovima ulica. 
\par 20 Pogledaj, Jahve, i vidi kome si to učinio. Zar žene da jedu porod svoj, djecu što njišu u naručju? Zar moradoše biti poklani u Svetištu Gospodnjem svećenici i proroci? 
\par 21 U uličnoj prašini leže djeca i starci; moje djevice i moji mladići od mača padoše. Ti ih pomori u dan gnjeva svojega, ti ih pokla nemilice. 
\par 22 Ti si, kao na dan svečani, sa svih strana sazvao užase moje. U dan gnjeva Jahvina nitko nije preživio, nitko se nije spasio. One koje sam odnjihala i odgojila neprijatelj moj sve je istrijebio. 


\chapter{3}

\par 1 Ja sam čovjek što upozna bijedu pod šibom gnjeva njegova. 
\par 2 Mene je odveo i natjerao da hodam u tmini i bez svjetlosti. 
\par 3 I upravo mene bije i udara bez prestanka njegova ruka. 
\par 4 Iscijedio je moje meso, kožu moju, polomio kosti moje. 
\par 5 Načinio mi jaram, glavu obrubio tegobama. 
\par 6 Pustio me da živim u tminama kao mrtvaci vječiti. 
\par 7 Zazidao me, i ja ne mogu izaći, otežao je moje okove. 
\par 8 Kada sam vikao i zapomagao, molitvu je moju odbijao. 
\par 9 Zazidao mi ceste tesanim kamenom, zakrčio je putove moje. 
\par 10 Meni on bijaše medvjed koji vreba, lav u zasjedi. 
\par 11 U bespuća me vodio, razdirao, ostavljao me da umirem. 
\par 12 Napinjao je luk svoj i gađao me kao metu za svoje strelice. 
\par 13 U slabine mi sasuo strelice, sinove svoga tobolca. 
\par 14 Postao sam smiješan svome narodu, rugalica svakidašnja. 
\par 15 Gorčinom me hranio, pelinom me napajao. 
\par 16 Puštao me da zube kršim kamen grizući, zakapao me u pepeo. 
\par 17 Duši je mojoj oduzet mir i više ne znam što je sreća! 
\par 18 Rekoh: Dotrajao je život moj i nada koja mi od Jahve dolazi. 
\par 19 Spomeni se bijede moje i stradanja, pelina i otrova! 
\par 20 Bez prestanka na to misli i sahne duša u meni. 
\par 21 To nosim u srcu i gojim nadu u sebi. 
\par 22 Dobrota Jahvina nije nestala, milosrđe njegovo nije presušilo. 
\par 23 Oni se obnavljaju svako jutro: tvoja je vjernost velika! 
\par 24 "Jahve je dio moj", veli mi duša, "i zato se u nj pouzdajem." 
\par 25 Dobar je Jahve onom koji se u nj pouzdaje, duši koja ga traži. 
\par 26 Dobro je u miru čekati spasenje Jahvino! 
\par 27 Dobro je čovjeku da nosi jaram za svoje mladosti. 
\par 28 Neka sjedi u samoći i šuti, jer mu On to nametnu; 
\par 29 neka usne priljubi uz prašinu, možda još ima nade! 
\par 30 Neka pruži obraz onome koji ga bije, neka se zasiti porugom. 
\par 31 Jer Gospod ne odbacuje nikoga zauvijek: 
\par 32 jer ako i rastuži, on se smiluje po svojoj velikoj ljubavi. 
\par 33 Jer samo nerado on ponižava i rascvili sinove čovjeka. 
\par 34 Kad se gaze nogama svi zemaljski sužnjevi, 
\par 35 kad se izvrće pravica čovjeku pred licem Svevišnjeg, 
\par 36 kad se krivica nanosi čovjeku u parnici, zar Gospod ne vidi? 
\par 37 Tko je rekao nešto i zbilo se? Nije li Gospod to zapovjedio? 
\par 38 Ne dolazi li iz usta Svevišnjega i dobro i zlo? 
\par 39 Na što se tuže živi ljudi? Svatko na svoj grijeh. 
\par 40 Ispitajmo, pretražimo pute svoje i vratimo se Jahvi. 
\par 41 Dignimo svoje srce i ruke svoje k Bogu koji je na nebesima. 
\par 42 Da, mi smo se odmetali, bili nepokorni, a ti, ti nisi praštao! 
\par 43 Obastrt gnjevom svojim, gonio si nas, ubijao i nisi štedio. 
\par 44 Oblakom si se obastro da molitva ne prodre do tebe. 
\par 45 Načinio si od nas smeće i odmet među narodima. 
\par 46 Razjapili usta na nas svi neprijatelji naši. 
\par 47 Užas i jama bila nam sudbina, propast i zator! 
\par 48 Potoci suza teku iz očiju mojih zbog propasti Kćeri naroda mojega. 
\par 49 Moje oči liju suze bez prestanka, jer prestanka nema 
\par 50 dok ne pogleda i ne vidi Jahve s nebesa. 
\par 51 Moje mi oko bol zadaje zbog kćeri svih mojega grada. 
\par 52 Uporno me k'o pticu progone svi što me mrze, a bez razloga. 
\par 53 U jamu baciše moj život i zatrpaše je kamenjem. 
\par 54 Voda mi dođe preko glave, rekoh sam sebi: "Pogiboh!" 
\par 55 I tada zazvah ime tvoje, Jahve, iz najdublje jame. 
\par 56 Ti oču moj glas: "Ne začepljuj uši svoje na vapaje moje." 
\par 57 Bliz meni bijaše u dan vapaja mog, govoraše: "Ne boj se!" 
\par 58 Ti si, Gospode, izborio pravdu za dušu moju, ti si život moj izbavio. 
\par 59 Ti, Jahve, vidje kako me tlače, dosudi mi pravdu. 
\par 60 Ti vidje svu osvetu njinu, sve podvale protiv mene. 
\par 61 Čuo si, Jahve, podrugivanje njihovo, sve podvale protiv mene. 
\par 62 Usne protivnika mojih i misli njine protiv mene su cio dan. 
\par 63 Kad sjede, kad ustaju, pogledaj samo: ja sam im pjesma-rugalica. 
\par 64 Vrati im, Jahve, milo za drago, po djelu ruku njihovih. 
\par 65 Učini da srca im otvrdnu, udari ih prokletstvom svojim. 
\par 66 Goni ih gnjevno i sve ih istrijebi pod nebesima svojim, Jahve! 


\chapter{4}

\par 1 Jao, potamnje zlato, to suho zlato! Sveto se kamenje prosu na uglovima svih ulica. 
\par 2 Sinovi sionski, nekoć cijenjeni kao najčišće zlato, ah, sada ih cijene kao sudove glinske, kao djelo ruku lončarevih! 
\par 3 Čak i šakali pružaju dojke i doje mladunčad, ali kćeri naroda moga postaše okrutne kao nojevi u pustinji. 
\par 4 Jezik dojenčeta za nepce se lijepi od žeđi. Djeca vape za kruhom, a nikog da im ga pruži. 
\par 5 Oni što se nekoć sladiše biranim jelima ginu po ulicama; nekoć nošeni u grimizu, sada se valjaju po buništu. 
\par 6 Veći bijaše zločin Kćeri naroda moga od grijeha Sodome, što u tren oka bi razorena, a ničija se ruka ne diže na nju. 
\par 7 Njeni mladići bijahu nekoć čišći od snijega, bjelji od mlijeka, od koralja rumenija bijahu im tijela, lice glatko k'o safir. 
\par 8 Sad im je obraz crnji od čađe, ne prepoznaju se više na ulici. Koža im se lijepi za kosti, suha kao drvo. 
\par 9 Kako su sretni oni što ih mač probode, sretniji od onih koje pomori glad; koji padaju, iscrpljeni, jer im nedostaju plodovi zemljini. 
\par 10 Žene, tako nježne, kuhaše djecu svoju, njima se hraniše za propasti Kćeri naroda moga. 
\par 11 Jahve je utolio svoj bijes, izlio jarosnu srdžbu svoju, na Sionu raspirio požar što sažiže i same temelje njegove. 
\par 12 Nisu vjerovali kraljevi zemaljski ni svekoliko stanovništvo zemlje da će ugnjetač i neprijatelj ući na vrata jeruzalemska - 
\par 13 zbog grijeha svojih prorokÄa, zbog bezakonja svećenikÄa koji usred grada prolijevahu krv pravednikÄa! 
\par 14 K'o slijepi teturahu ulicama, omašteni krvlju, te nitko nije smio da se takne odjeće njihove. 
\par 15 "Natrag, nečisti!" - viču im. "Natrag! Ne dirajte!" I tada pobjegoše poganima, al' ne smjedoše ondje ostati. 
\par 16 Raspršilo ih lice Jahvino, on ih više nije gledao. Ne poštuju više svećenikÄa, ne sažaljuju staraca. 
\par 17 Već nam oči iščilješe iščekujući pomoć, ali uzalud; s kula naših zureć' u daljinu očekivasmo narod koji nas ne može spasiti. 
\par 18 Vrebaju nam na korake da ne hodamo po trgovima svojim. Bliži nam se kraj, navršili nam se dani, naš konac dolazi. 
\par 19 Naši gonitelji bijahu brži od orlova na nebu; u planini nas ganjahu, u pustinji dočekivahu u zasjedi. 
\par 20 Naš životni dah, Jahvin pomazanik, pade u njihove jame - on za koga govorasmo: "U sjeni njegovoj živjet ćemo među narodima." 
\par 21 Raduj se i veseli se, Kćeri edomska, ti koja živiš u zemlji Usu: doći će i do tebe čaša, opit ćeš se i razgoliti. 
\par 22 Tvoj grijeh je iskupljen, Kćeri sionska, neće te više u izgnanstvo voditi. Kaznit će opačinu tvoju, Kćeri edomska, razotkriti grijehe tvoje. 



\chapter{5}

\par 1 Spomeni se, Jahve, što nas je snašlo, pogledaj, vidi sramotu našu! 
\par 2 Baština naša pade u ruke strancima, domovi naši pripadoše tuđincima. 
\par 3 Siročad smo: oca nemamo, majke su nam kao udovice. 
\par 4 Vodu što pijemo plaćamo novcem, i za drvo valja nam platiti. 
\par 5 Jaram nam je o vratu, gone nas, iscrpljeni smo, ne daju nam predahnuti. 
\par 6 Pružamo ruke k Egiptu i Asiriji da se kruha nasitimo. 
\par 7 Oci naši zgriješiše i više ih nema, a mi nosimo krivice njihove. 
\par 8 Robovi nama zapovijedaju, a nitko da nas izbavi iz ruku njihovih. 
\par 9 Kruh svoj donosimo izlažući život maču u pustinji. 
\par 10 Koža nam gori kao peć užarena, ognjicom od plamena gladi. 
\par 11 Oskvrnuli su žene na Sionu i djevice u gradovima judejskim. 
\par 12 Svojim su rukama vješali knezove, ni lica staračka nisu poštivali. 
\par 13 Mladići su nosili žrvnjeve, djeca padala pod bremenom drva. 
\par 14 Starci su ostavili vrata, mladići više ne sviraju na lirama. 
\par 15 Radosti nesta iz naših srdaca, naš ples se pretvori u tugovanje. 
\par 16 Pao je vijenac s naše glave, jao nama što zgriješismo! 
\par 17 Evo zašto nam srce boluje, evo zašto nam oči se zastiru: 
\par 18 zato što Gora sionska opustje i po njoj se šuljaju šakali. 
\par 19 Ali ti, Jahve, ostaješ zauvijek, tvoj je prijesto od koljena do koljena. 
\par 20 Zašto da nas zaboraviš zauvijek, da nas ostaviš za mnoge dane? 
\par 21 Vrati nas k sebi, Jahve, obratit ćemo se, obnovi dane naše kao što nekoć bijahu. 
\par 22 Il' nas hoćeš sasvim zabaciti i na nas se beskrajno srditi? 





\end{document}