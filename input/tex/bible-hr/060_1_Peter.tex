\begin{document}

\title{1 Petrova}


\chapter{1}

\par 1 Petar, apostol Isusa Krista: putnicima Raseljeništva u Pontu, Galaciji, Kapadociji, Aziji i Bitiniji, 
\par 2 po predznanju Boga  Oca, posvećenjem Duha izabranima da budu poslušni te poškropljeni  krvlju Isusa Krista. Punina vam milosti i mira! 
\par 3 Blagoslovljen Bog i Otac Gospodina našega Isusa Krista koji nas po velikom milosrđu svojemu uskrsnućem Isusa Krista od mrtvih nanovo rodi za životnu nadu, 
\par 4 za baštinu neraspadljivu, neokaljanu i neuvelu, pohranjenu na nebesima za vas, 
\par 5 vas koje snaga Božja po vjeri čuva za spasenje, spremno da se objavi u posljednje vrijeme. 
\par 6 Zbog toga se radujte, makar se sada možda trebalo malo  i žalostiti zbog različitih kušnja: 
\par 7 da prokušanost vaše vjere  - dragocjenija od propadljivog zlata, koje se ipak u vatri kuša  - stekne hvalu, slavu i čast o Objavljenju Isusa Krista. 
\par 8 Njega  vi ljubite iako ga ne vidjeste; u njega, iako ga još ne gledate, vjerujete te klikćete od radosti neizrecive i proslavljene 
\par 9 što  postigoste svrhu svoje vjere: spasenje duša. 
\par 10 To su spasenje istraživali i pronicali proroci koji prorokovahu  o milosti vama namijenjenoj. 
\par 11 Pronicali su na koje ili kakvo  je vrijeme smjerao Duh Kristov u njima koji je unaprijed svjedočio  o Kristovim patnjama te slavama što su nakon njih imale doći: 
\par 12 bi im objavljeno da ne sebi nego vama poslužuju ono što vam  sada u Duhu Svetom s neba poslanom navijestiše vaši blagovjesnici, a nada što se i anđeli žude nadviti. 
\par 13 Zato opašite bokove pameti svoje, trijezni budite i savršeno  se pouzdajte u milost koju vam donosi Objavljenje Isusa Krista. 
\par 14 Kao poslušna djeca ne supriličujte se prijašnjim požudama  iz doba neznanja. 
\par 15 Naprotiv, kao što je svet Onaj koji vas  pozva, i vi budite sveti u svemu življenju. 
\par 16 Ta pisano je:  Budite sveti jer sam ja svet. 
\par 17 Ako dakle Ocem nazivate njega koji nepristrano svakoga  po djelu sudi, vrijeme svoga proputovanja proživite u bogobojaznosti. 
\par 18 Ta znate da od svog ispraznog načina života, što vam ga oci  namriješe, niste otkupljeni nečim raspadljivim, srebrom ili zlatom, 
\par 19 nego dragocjenom krvlju Krista, Jaganjca nevina i bez mane. 
\par 20 On bijaše doduše predviđen prije postanka svijeta, ali se  očitova na kraju vremena radi vas 
\par 21 koji po njemu vjerujete  u Boga koji ga uskrisi od mrtvih te mu dade slavu da vjera vaša  i nada bude u Bogu. 
\par 22 Pošto ste posluhom istini očistili duše svoje za nehinjeno  bratoljublje, od srca žarko ljubite jedni druge. 
\par 23 Ta nanovo  ste rođeni, ne iz sjemena raspadljiva nego neraspadljiva: riječju  Boga koji živi i ostaje. 
\par 24 Doista, svako je tijelo kao trava, sva mu slava ko cvijet poljski: sahne trava, vene cvijet, 
\par 25 ali Riječ Gospodnja ostaje dovijeka. Ta pak riječ jest evanđelje koje vam je naviješteno. 


\chapter{2}

\par 1 Odložite dakle svaku zloću i svaku prijevaru, himbe i zavisti  i sva klevetanja. 
\par 2 Kao novorođenčad žudite za duhovnim, nepatvorenim  mlijekom da po njemu uzrastete za spasenje, 
\par 3 ako ste doista  okusili kako je dobar Gospodin. 
\par 4 Pristupite k njemu, Kamenu živomu što ga, istina, ljudi odbaciše, ali je u očima Božjim izabran, dragocjen, 
\par 5 pa  se kao živo kamenje ugrađujte u duhovni Dom za sveto svećenstvo  da prinosite žrtve duhovne, ugodne Bogu po Isusu Kristu. 
\par 6 Stoga stoji u Pismu: Evo postavljam na Sionu kamen odabrani, dragocjeni kamen ugaoni: Tko u nj vjeruje, ne, neće se postidjeti. 
\par 7 Vama dakle koji vjerujete - čast! A onima koji ne vjeruju  - kamen koji odbaciše graditelji postade kamen zaglavni 
\par 8 i  kamen spoticanja, stijena posrtanja; oni se o nj spotiču, neposlušni Riječi, za što su i određeni. 
\par 9 A vi ste rod izabrani, kraljevsko svećenstvo, sveti  puk, narod stečeni da naviještate silna djela Onoga koji  vas iz tame pozva k divnom svjetlu svojemu; 
\par 10 vi, nekoć Ne-narod,  a sada Narod Božji; vi Ne-mili, a sada Mili. 
\par 11 Ljubljeni! Zaklinjem vas da se kao pridošlice i putnici  klonite putenih požuda koje vojuju protiv duše; 
\par 12 življenje  vaše među poganima neka bude uzorno da upravo onim za što vas  sada potvaraju kao zločince, pošto promotre vaša dobra djela, proslave Boga u dan pohoda. 
\par 13 Pokoravajte se svakoj ljudskoj ustanovi radi Gospodina:  bilo kralju kao vrhovniku, 
\par 14 bilo upraviteljima jer ih on šalje  da kazne zločince, a pohvale one koji dobro čine. 
\par 15 Doista, ovo je Božja volja: da čineći dobro ušutkate neznanje bezumnika. 
\par 16 Kao slobodni ljudi - ali ne kao oni kojima je sloboda tek  pokrivalom zloće, već kao Božje sluge - 
\par 17 sve poštujte, bratstvo  ljubite, Boga se bojte, kralja častite! 
\par 18 Sluge, budite sa svim poštovanjem pokorni gospodarima, ne samo dobrima i blagima nego i naopakima. 
\par 19 To je uistinu  milost ako tko radi savjesti, radi Boga podnosi nevolje trpeći  nepravedno. 
\par 20 Kakve li slave doista ako za grijehe udarani  strpljivo podnosite? No ako dobro čineći trpite pa strpljivo  podnosite, to je Bogu milo. 
\par 21 Ta na to ste pozvani jer i Krist je trpio za vas i ostavio  vam primjer da idete stopama njegovim. 
\par 22 On koji grijeha ne učini nit mu usta prijevaru izustiše; 
\par 23 on koji na uvredu nije uvredom uzvraćao i mučen nije prijetio, prepuštajući to Sucu pravednom; 
\par 24 on koji u tijelu svom grijehe naše ponese na drvo da umrijevši grijesima pravednosti živimo; on čijom se modricom izliječiste. 
\par 25 Doista, poput ovaca lutaste, ali se sada obratiste k pastiru i čuvaru duša svojih. 


\chapter{3}

\par 1 Tako i vi, žene, pokoravajte se svojim muževima: ako su neki  od njih možda neposlušni Riječi, da i bez riječi budu pridobiveni  življenjem vas žena, 
\par 2 pošto promotre vaše bogoljubno i čisto  življenje. 
\par 3 Vaš nakit neka ne bude izvanjski - pletenje kose, kićenje zlatom ili oblačenje haljina. 
\par 4 Nego: čovjek skrovita  srca, neprolazne ljepote, blaga i smirena duha. To je pred Bogom  dragocjeno. 
\par 5 Tako su se doista i nekoć svete žene, zaufane  u Boga, resile: pokoravale su se muževima. 
\par 6 Sara se tako pokori  Abrahamu te ga nazva gospodarom. Njezina ste djeca ako činite  dobro ne bojeći se nikakva zastrašivanja. 
\par 7 Tako i vi, muževi, obazrivo živite sa svojim ženama, kao  sa slabijim spolom, te im iskazujte čast kao subaštinicima milosti  Života da ne spriječite svojih molitava. 
\par 8 Napokon, budite svi jednodušni, puni suosjećanja i bratske  ljubavi, milosrdni, ponizni! 
\par 9 Ne vraćajte zlo za zlo ni uvredu  za uvredu! Naprotiv, blagoslivljajte jer ste na to i pozvani  da baštinite blagoslov! 
\par 10 Doista, tko želi ljubiti život i naužit se dana sretnih, nek suspregne jezik oda zla i usne od riječi prijevarnih; 
\par 11 zla nek se kloni, a čini dobro, mir neka traži i za njim ide: 
\par 12 jer oči Gospodnje gledaju pravedne, uši mu slušaju vapaje njihove, a lice se Gospodnje okreće protiv zločinaca. 
\par 13 Pa tko da vam naudi ako revnujete za dobro? 
\par 14 Nego, morali i trpjeti zbog svoje pravednosti, blago vama! No ne  bojte se njihova zastrašivanja i ne plašite se! 
\par 15 Naprotiv, Gospodin - Krist neka vam bude svet, u srcima  vašim, te budite uvijek spremni na odgovor svakomu koji od vas  zatraži obrazloženje nade koja je u vama, 
\par 16 ali blago i s poštovanjem, dobre savjesti da oni koji ozloglašuju vaš dobar život u Kristu, upravo onim budu postiđeni za što vas potvaraju. 
\par 17 Ta uspješnije  je trpjeti, ako je to Božja volja, čineći dobro, nego čineći  zlo. 
\par 18 Doista, i Krist jednom za grijehe umrije, pravedan za nepravedne, da vas privede k Bogu - ubijen doduše u tijelu, ali oživljen u duhu. 
\par 19 U njemu otiđe i propovijedati duhovima u tamnici 
\par 20 koji bijahu nekoć nepokorni, kad ih ono Božja strpljivost iščekivaše, u vrijeme Noino, dok se gradila korablja u kojoj nekolicina, to jest osam duša, bi spašena vodom. 
\par 21 Njezin protulik, krštenje - ne odlaganje tjelesne nečistoće, nego molitva za dobru savjest upravljena Bogu - i vas sada spasava po uskrsnuću Isusa Krista 
\par 22 koji, uzašavši na nebo, jest zdesna Bogu, pošto mu bijahu pokoreni anđeli, vlasti i sile. 


\chapter{4}

\par 1 Dakle, budući da je Krist trpio u tijelu, i vi se oboružajte  istim mišljenjem - jer tko trpi u tijelu okanio se grijeha - 
\par 2 da vrijeme što vam u tijelu još preostaje proživite ne više  po ljudskim požudama nego po Božjoj volji. 
\par 3 Dosta je uistinu  što ste u prošlom vremenu vršili volju pogana, hodeći u razvratnostima, požudama, pijančevanjima, pijankama, opijanjima i bezakoničkim  idolopoklonstvima. 
\par 4 Stoga se čude što se ne slijevate u tu  istu rijeku raskalašenosti te proklinju. 
\par 5 Polagat će oni račun  Onomu tko je već spreman suditi žive i mrtve. 
\par 6 Zato je i mrtvima  naviješteno evanđelje da osuđeni doduše po ljudsku, u tijelu, žive po Božju - u duhu. 
\par 7 Približio se svršetak svega! Osvijestite se i otrijeznite  za molitvu! 
\par 8 Prije svega imajte žarku ljubav jedni prema drugima  jer ljubav pokriva mnoštvo grijeha! 
\par 9 Gostoljubivo primajte  jedni druge bez mrmljanja! 
\par 10 Jedni druge poslužujte - svatko  po primljenom daru - kao dobri upravitelji različitih Božjih  milosti! 
\par 11 Govori li tko? Neka govori kao riječi Božje! Poslužuje  li tko? Neka poslužuje kao snagom koju daje Bog da se u svemu  slavi Bog po Isusu Kristu, komu slava i vlast u vijeke vijekova!  Amen. 
\par 12 Ljubljeni! Ne čudite se požaru što bukti među vama da  vas iskuša, kao da vam se događa štogod neobično! 
\par 13 Naprotiv, radujte se kao zajedničari Kristovih patnja da i o Objavljenju  njegove slave mognete radosno klicati. 
\par 14 Pogrđuju li vas zbog  imena Kristova, blago vama, jer Duh Slave, Duh Božji u vama počiva. 
\par 15 Tek neka nitko od vas ne trpi kao ubojica, ili kradljivac, ili zločinac, ili makar i kao nametljivac; 
\par 16 ako li kao kršćanin, neka se ne stidi, nego slavi Boga zbog tog imena. 
\par 17 Ta vrijeme  je da započne Sud - od doma Božjega. No ako već od vas započinje, kakav je onda svršetak onih što nisu poslušni Božjem evanđelju? 
\par 18 I ako se pravednik jedva spasava, opak i grešnik gdje da se pojavi? 
\par 19 Stoga oni koji po volji Božjoj trpe, neka dobrim djelima  povjere duše svoje vjernom Stvoritelju. 



\chapter{5}

\par 1 Starješine dakle među vama opominjem, ja sustarješina i svjedok  Kristovih patnja, a zato i zajedničar slave koja se ima očitovati: 
\par 2 pasite povjereno vam stado Božje, nadgledajte ga - ne prisilno, nego dragovoljno, po Božju; ne radi prljava dobitka, nego oduševljeno; 
\par 3 i ne kao gospodari Baštine nego kao uzori stada. 
\par 4 Pa kad  se pojavi Natpastir, primit ćete neuveli vijenac slave. 
\par 5 Tako i vi, mladići, podložite se starješinama; svi se  jedni prema drugima pripašite poniznošću jer Bog se oholima protivi, a poniznima daruje milost. 
\par 6 Ponizite se dakle pod snažnom rukom Božjom da vas uzvisi u  pravo vrijeme. 
\par 7 Svu svoju brigu povjerite njemu jer on se brine za  vas. 
\par 8 Otrijeznite se! Bdijte! Protivnik vaš, đavao, kao ričući  lav obilazi tražeći koga da proždre. 
\par 9 Oprite mu se stameni  u vjeri znajući da takve iste patnje podnose vaša braća po svijetu. 
\par 10 A Bog svake milosti, koji vas pozva na vječnu slavu u Kristu, on će vas, pošto malo potrpite, usavršiti, učvrstiti, ojačati, utvrditi. 
\par 11 Njemu vlast u vijeke vjekova! Amen. 
\par 12 Pišem vam ukratko, po Silvanu, koga smatram bratom vjernim, da vas ohrabrim i posvjedočim kako je ovo istinska milost Božja.  Nje se držite! 
\par 13 Pozdravlja vas suizabranica u Babilonu i Marko, sin moj. 
\par 14 Pozdravite jedni druge cjelovom ljubavi! Mir svima vama koji ste u Kristu! 




\end{document}