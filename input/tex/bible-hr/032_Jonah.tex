\begin{document}

\title{Jona}


\chapter{1}

\par 1 Riječ Jahvina dođe Joni, sinu Amitajevu: 
\par 2 "Ustani," reče  mu, "idi u Ninivu, grad veliki, i propovijedaj u njemu, jer se  zloća njihova popela do mene." 
\par 3 A Jona ustade da pobjegne u Taršiš, daleko od Jahve. Siđe  u Jafu i nađe lađu što je plovila u Taršiš. Plati vozarinu i  ukrca se da otplovi s njima u Taršiš, daleko od Jahve. 
\par 4 Ali  Jahve podiže na moru silan vjetar i nastade nevrijeme veliko  na moru te mišljahu da će se lađa razbiti. 
\par 5 Uplašiše se mornari;  svaki zazva svoga boga, i da bi je olakšali, stadoše iz lađe  bacati tovar što bijaše u njoj. Jona pak bijaše sišao na dno  lađe, legao i zaspao tvrdim snom. 
\par 6 Zapovjednik lađe pristupi  mu i reče: "Što spavaš kao zaklan? Ustaj i prizivlji Boga svojega!  Možda će nas se sjetiti Bog taj da ne poginemo." 
\par 7 Potom rekoše  jedni drugima: "Hajde da bacimo ždrijeb da vidimo od koga nam  dođe ovo zlo." Baciše ždrijeb i pade ždrijeb na Jonu. 
\par 8 Oni mu onda rekoše:  "Kaži nam: zbog koga nas ovo zlo snađe, kojim se poslom baviš, odakle dolaziš, iz koje si zemlje i od kojega naroda?" 
\par 9 On  im odgovori: "Ja sam Hebrej, i štujem Jahvu, Boga nebeskoga,  koji stvori more i zemlju." 
\par 10 Ljudi se uplašiše veoma i rekoše mu: "Što si to učinio!"  Jer bijahu doznali da on bježi od Jahve - sam im je to pripovjedio. 
\par 11 Oni ga zapitaše: "Što da učinimo s tobom da nam se more smiri?"  Jer se more sve bješnje dizalo. 
\par 12 On im odgovori: "Uzmite me i bacite u more, pa će vam  se more smiriti, jer znam da se zbog mene diglo na vas ovo veliko  nevrijeme." 
\par 13 Ljudi uzeše veslati ne bi li se primakli kopnu, jer se  more sve bješnje dizalo protiv njih. 
\par 14 Tad zazvaše Jahvu i  rekoše: "Ah, Jahve, ne daj da poginemo zbog života ovoga čovjeka  i ne svali na nas krv nevinu, jer ti si Jahve: činiš kako ti  je milo." 
\par 15 I uzevši Jonu, baciše ga u more - i more presta  bjesnjeti. 
\par 16 Tada velik strah Jahvin obuze ljude te prinesoše žrtvu  Jahvi i učiniše zavjete. 
\par 17 (2:1) Jahve zapovjedi velikoj ribi da proguta Jonu. Tri dana i tri  noći ostade Jona u ribljoj utrobi. 


\chapter{2}

\par 1 (2:2) Iz utrobe riblje stade  Jona moliti Jahvu, Boga svojega. 
\par 2 (2:3) On reče: "Iz nevolje svoje zavapih Jahvi, i on me usliša; iz utrobe Podzemlja zazvah, i ti si mi čuo glas. 
\par 3 (2:4) Ti me baci moru u dubine, i voda me opteče. Sve poplave tvoje i valovi oboriše se na me. 
\par 4 (2:5) Pomislih: odbačen sam ispred očiju tvojih. Al' ipak oči upirem svetom Hramu tvojem. 
\par 5 (2:6) Vode me do grla okružiše, bezdan me opkoli. Trave mi glavu omotaše, 
\par 6 (2:7) siđoh do korijena planina. Nada mnom se zatvoriše zauvijek zasuni zemljini. Al' ti iz jame izvadi život moj, o Jahve, Bože moj. 
\par 7 (2:8) Samo što ne izdahnuh kad se spomenuh Jahve, i molitva se moja k tebi vinula, prema svetom Hramu tvojemu. 
\par 8 (2:9) Oni koji štuju isprazna ništavila milost svoju ostavljaju. 
\par 9 (2:10) A ja ću ti s pjesmom zahvalnicom žrtvu prinijeti. Što se zavjetovah, ispunit ću. Spasenje je od Gospoda." 
\par 10 (2:11) Tada Jahve zapovjedi ribi i ona izbljuva Jonu na obalu. 


\chapter{3}

\par 1 Riječ Jahvina dođe Joni drugi put: 
\par 2 "Ustani," reče mu, "idi  u Ninivu, grad veliki, propovijedaj u njemu što ću ti reći." 
\par 3 Jona ustade i ode u Ninivu, kako mu Jahve zapovjedi. Niniva  bijaše grad velik do Boga - tri dana hoda. 
\par 4 Jona prođe gradom  dan hoda, propovijedajući: "Još četrdeset dana i Niniva će biti  razorena." 
\par 5 Ninivljani povjerovaše Bogu; oglasiše post i obukoše  se u kostrijet, svi od najvećega do najmanjega. 
\par 6 Glas doprije do kralja ninivskoga: on ustade s prijestolja, skide plašt sa sebe, odjenu se u kostrijet i sjede u pepeo. 
\par 7 Tada se po odredbi kralja i njegovih velikaša oglasi i objavi  u Ninivi: "Ljudi i stoka, goveda i ovce da ne okuse ništa, ni  da pasu, ni da vodu piju. 
\par 8 Nego i ljudi i stoka da se pokriju  kostrijeću, da glasno Boga zazivlju i da se obrati svatko sa  svojega zlog puta i nepravde koju je činio. 
\par 9 Tko zna, možda  će se povratiti Bog, smilovati se i odustati od ljutoga svog  gnjeva da ne izginemo?" 
\par 10 Bog vidje što su činili: da se obratiše od svojega zlog  puta. I sažali se Bog zbog nesreće kojom im bijaše zaprijetio  i ne učini. 



\chapter{4}

\par 1 Joni bi veoma krivo i rasrdi se. 
\par 2 I ovako se pomoli Jahvi:  "Ah, Jahve, nisam li ja to slutio dok još u svojoj zemlji bijah?  Zato sam htio prije pobjeći u Taršiš; jer znao sam da si ti Bog  milostiv i milosrdan, spor na gnjev i bogat milosrđem i da se  nad nesrećom brzo sažališ. 
\par 3 Sada, Jahve, uzmi moj život, jer  mi je bolje umrijeti nego živjeti." 
\par 4 Jahve odgovori: "Srdiš  li se ti s pravom?" 
\par 5 Jona iziđe iz grada i sjede s istoka gradu; načini ondje  kolibu i sjede pod njom u hlad da vidi što će biti od grada. 
\par 6 A Jahve Bog učini da izraste bršljan nad Jonom i pruži sjenu  njegovoj glavi te da ga izliječi od zlovolje. Jona se bršljanu  veoma obradova. 
\par 7 Ali sutradan, u osvit zore, Bog zapovjedi  crvu da podgrize bršljan, i on usahnu. 
\par 8 Kad je ogranulo sunce, posla Bog vruć istočni vjetar; sunce je palilo glavu Joninu  te je sasvim klonuo. Poželje umrijeti i reče: "Bolje mi je umrijeti  nego živjeti." 
\par 9 Bog upita Jonu: "Srdiš li se s pravom zbog bršljana?"  On odgovori: "Da, s pravom sam ljut nasmrt." 
\par 10 Jahve mu reče:  "Tebi je žao bršljana oko kojega se nisi trudio, nego je u jednu  noć nikao i u jednu noć usahnuo. 
\par 11 A meni da ne bude žao Ninive, grada velikoga, u kojem ima više od sto i dvadeset tisuća ljudi  koji ne znaju razlikovati desno i lijevo, a uz to i mnogo životinja!" 



\end{document}