\begin{document}

\title{Izlazak}


\chapter{1}

\par 1 Ovo su imena Izraelovih sinova koji su s Jakovom sišli u Egipat, svaki sa svojim domom: 
\par 2 Ruben, Šimun, Levi i Juda; 
\par 3 Jisakar, Zebulun i Benjamin; 
\par 4 Dan i Naftali; Gad i Ašer. 
\par 5 U svemu  Jakovljevih potomaka bijaše sedamdeset duša. A Josip je već bio  u Egiptu. 
\par 6 I umre Josip, a pomru i sva njegova braća i sav onaj naraštaj. 
\par 7 Ali su Izraelci bili rodni, namnožili se i silno ojačali,  tako da su napučili zemlju. 
\par 8 Uto u Egiptu zavlada novi kralj koji nije poznavao Josipa. 
\par 9 I reče on svome puku: "Eto, sinovi su Izraelovi postali narod  brojan i moćniji od nas. 
\par 10 Hajde, postupimo mudro s njima:  spriječimo im porast, da se u slučaju rata ne pridruže našim  neprijateljima, da ne udare na nas i napokon ne odu iz zemlje." 
\par 11 I postaviše nad njima nadglednike da ih tlače teškim  radovima. Tako su faraonu sagradili gradove-skladišta: Pitom  i Ramses. 
\par 12 Ali što su ih više tlačili, oni se još više množili, napredovali i širili se, tako da su Egipćani strahovali od Izraelaca. 
\par 13 I Egipćani se okrutno obore na Izraelce. 
\par 14 Ogorčavali  su im život teškim radovima: pravljenjem meljte i opeke, različitim  poljskim poslovima i svakovrsnim naporima koje im nemilosrdno  nametahu. 
\par 15 Egipatski se kralj obrati i na hebrejske babice, od kojih  jednoj bijaše ime Šifra, a drugoj Pua, pa im naredi: 
\par 16 "Kad  u porodu pomažete Hebrejkama, dobro pogledajte oba kamena sjedala:  ako je muško dijete, ubijte ga; ako je žensko, neka živi. 
\par 17 Ali  su se babice bojale Boga i nisu činile kako im je naredio egipatski  kralj, nego su ostavljale na životu mušku djecu. 
\par 18 Stoga egipatski  kralj pozove babice pa im rekne: "Zašto ste tako radile i na  životu ostavljale mušku djecu?" 
\par 19 Nato babice odgovore faraonu:  "Hebrejke nisu kao egipatske žene. One su životne. Prije nego  babica dođe k njima, one već rode." 
\par 20 Bog je to babicama za  dobro primio. Narod se množio i silno porastao. 
\par 21 A kako su  se babice bojale Boga, on ih obdari potomstvom. 
\par 22 Onda faraon izda naredbu svemu svome narodu: "Svako muško  dijete koje se rodi Hebrejima bacite u Rijeku! Na životu ostavite  samo žensku djecu." 


\chapter{2}

\par 1 Neki čovjek od Levijeva koljena ode i oženi se djevojkom Levijkom. 
\par 2 Žena zače i rodi sina. Vidjevši kako je krasan, krila ga je  tri mjeseca. 
\par 3 Kad ga nije mogla više sakrivati, nabavi košaricu  od papirusove trstike, oblijepi je smolom i paklinom, u nju stavi  dijete i položi ga u trstiku na obali Rijeke. 
\par 4 Njegova sestra  stane podalje da vidi što će s njime biti. 
\par 5 Faraonova kći siđe k Rijeci da se kupa, dok su njezine  sluškinje šetale uz obalu Rijeke. Opazi ona košaricu u trstici, pa pošalje sluškinju da je donese. 
\par 6 Otvori je i pogleda, a  to u njoj dijete! Muško čedo. Plakalo je. Njoj se sažali na nj.  "Bit će to hebrejsko dijete", reče. 
\par 7 Onda njegova sestra rekne  faraonovoj kćeri: "Hoćeš li da ti potražim dojilju među Hebrejkama  da ti dijete doji?" 
\par 8 "Idi!" - odgovori joj faraonova kći. Tako  djevojka ode i pozove djetetovu majku. 
\par 9 "Uzmi ovo dijete",  rekne joj faraonova kći, "i odgoji mi ga, a ja ću te plaćati."  Tako žena uzme dijete i othrani ga. 
\par 10 Kad je dijete odraslo, ona ga odvede faraonovoj kćeri, koja ga posini. Nadjene mu ime  Mojsije, "jer sam ga", reče, "iz vode izvadila". 
\par 11 Jednog dana, kad je Mojsije već odrastao, dođe među svoj  narod i vidje njegove muke. Spazi tada kako neki Egipćanin tuče  jednoga Hebrejca - brata njegova. 
\par 12 Okrene se tamo-amo i, vidjevši  da nikoga nema, ubije Egipćanina i zatrpa ga u pijesak. 
\par 13 Izađe  on i sutradan te zateče dva Hebrejca kako se tuku. "Zašto tučeš  svoga druga?" - rekne napadaču. 
\par 14 Ovaj odvrati: "Tko te postavi  za starješinu i suca našega? Kaniš li ubiti i mene kako si ubio  onog Egipćanina?" Mojsije se uplaši pa će u sebi: "Tako! Ipak  se saznalo." 
\par 15 Kad je faraon to dočuo, htjede Mojsija pogubiti.  Zato Mojsije pobjegne od faraona i skloni se u midjansku zemlju.  Ondje sjedne kraj nekog studenca. 
\par 16 Midjanski je svećenik imao sedam kćeri. Dođu one da zahvate  vode i naliju pojila, da napoje stado svoga oca. 
\par 17 Ali dođu  i pastiri te ih potjeraju. Mojsije ustane, obrani ih i stado  im napoji. 
\par 18 Kad su se vratile svome ocu Reuelu, on ih zapita:  "Kako ste se danas tako brzo vratile?" 
\par 19 One odgovore: "Neki  Egipćanin obrani nas od pastira i još nam zahvati vode i stado  nam napoji." 
\par 20 "Gdje je?" - zapita on svoje kćeri. "Zašto ste  ostavile toga čovjeka? Pozovite ga na objed." 
\par 21 Mojsije pristane  da ostane kod toga čovjeka. On oženi Mojsija svojom kćeri Siporom. 
\par 22 A kad ona rodi sina, on mu nadjene ime Geršon, "jer sam", reče, "stranac u tuđoj zemlji". 
\par 23 Poslije mnogo vremena umre egipatski kralj. Izraelci  su još stenjali u ropstvu. Vapili su, a njihov vapaj za pomoć  sred ropstva uzlazio je k Bogu. 
\par 24 Bog je čuo njihovo zapomaganje  i sjetio se svoga Saveza s Abrahamom, Izakom i Jakovom. 
\par 25 I  pogleda Bog na Izraelce i zauze se za njih. 


\chapter{3}

\par 1 Mojsije pasao ovce svoga tasta Jitra, midjanskoga svećenika.  Goneći tako stado po pustari, dođe do Horeba, brda Božjega. 
\par 2 Anđeo  mu se Jahvin ukaže u rasplamtjeloj vatri iz jednog grma. On se  zagleda: grm sav u plamenu, a ipak ne izgara. 
\par 3 "Hajde da priđem, " reče Mojsije, "i promotrim ovaj čudni prizor: zašto grm ne  sagorijeva." 
\par 4 Kad je Jahve vidio kako prilazi da razmotri,  iz grma ga Bog zovne: "Mojsije! Mojsije!" "Evo me!" - javi se. 
\par 5 "Ne prilazi ovamo!" - reče. "Izuj obuću s nogu! Jer mjesto  na kojem stojiš sveto je tlo. 
\par 6 Ja sam", nastavi, "Bog tvoga  oca; Bog Abrahamov, Bog Izakov, Bog Jakovljev." Mojsije zakloni  lice: bojao se u Boga gledati. 
\par 7 "Vidio sam jade svoga naroda u Egiptu", nastavi Jahve, "i čuo mu tužbu na tlačitelje njegove. Znane su mi muke njegove. 
\par 8 Zato sam sišao da ga izbavim iz šaka egipatskih i odvedem  ga iz te zemlje u dobru i prostranu zemlju - u zemlju kojom teče  med i mlijeko: u postojbinu Kanaanaca, Hetita, Amorejaca, Perižana, Hivijaca i Jebusejaca. 
\par 9 Vapaji sinova Izraelovih dopriješe  do mene. I sam vidjeh kako ih Egipćani tlače. 
\par 10 Zato, hajde!  Ja te šaljem faraonu da izbaviš narod moj, Izraelce, iz Egipta." 
\par 11 "Tko sam ja da se uputim faraonu", odgovori Mojsije Bogu, "i izvedem Izraelce iz Egipta!" 
\par 12 "Ja ću biti s tobom", nastavi.  "I ovo će ti biti znak da sam te ja poslao: kad izvedeš narod  iz Egipta, Bogu ćete iskazati štovanje na ovome brdu." 
\par 13 Nato Mojsije reče Bogu: "Ako dođem k Izraelcima pa im  kažem: 'Bog otaca vaših poslao me k vama', i oni me zapitaju:  'Kako mu je ime?' - što ću im odgovoriti?" 
\par 14 "Ja sam koji  jesam", reče Bog Mojsiju. Onda nastavi: "Ovako kaži Izraelcima:  'Ja jesam' posla me k vama." 
\par 15 Dalje je Bog Mojsiju  rekao: "Kaži Izraelcima ovako: 'Jahve, Bog vaših otaca, Bog Abrahamov, Bog Izakov i Bog Jakovljev, poslao me k vama.'  To mi je ime dovijeka, tako će me zvati od koljena do koljena." 
\par 16 "Idi, skupi starješine Izraelaca pa im kaži: 'Jahve,  Bog otaca - Bog Abrahamov, Izakov i Jakovljev - objavio mi se  i rekao mi: Pohodio sam vas i razabrao što vam se čini u Egiptu. 
\par 17 Odlučio sam vas izvesti iz egipatske bijede u zemlju Kanaanaca, Hetita, Amorejaca, Perižana, Hivijaca i Jebusejaca - u zemlju  kojom teče med i mlijeko!' 
\par 18 Oni će te poslušati. Onda pođi  sa starješinama Izraelaca k egipatskom kralju i reci mu: 'Objavio  nam se Jahve, Bog Hebreja. Pusti nas da odemo tri dana hoda u  pustinju, da ondje prinesemo žrtvu Jahvi, Bogu svojemu.' 
\par 19 Znam  ja da vas egipatski kralj neće pustiti ako ne bude natjeran teškom  šakom. 
\par 20 Zato ću ja pružiti svoju šaku i pritisnuti Egipat  svakovrsnim čudesima što ću ih u njemu izvesti. Poslije će vas  pustiti. 
\par 21 Dobro ću raspoložiti Egipćane prema ovome narodu, pa kad pođete, nećete poći praznih ruku. 
\par 22 Svaka će žena zatražiti  od svoje susjede i stanarke u svojoj kući nakita srebrnog i zlatnog  i odjeće. To stavite na svoje sinove i kćeri. Tako ćete oplijeniti  Egipćane." 


\chapter{4}

\par 1 Mojsije uzvrati: "Ali ako mi ne povjeruju i ne poslušaju me, nego mi reknu: 'Jahve ti se nije objavio?'" 
\par 2 "Što ti je to  u ruci?" - zapita ga Jahve. "Štap", odgovori. 
\par 3 "Baci ga na  zemlju!" - naredi mu Jahve. On ga baci na zemlju, a štap se pretvori  u zmiju. Mojsije pred njom uzmače. 
\par 4 Onda Jahve reče Mojsiju:  "Pruži ruku i uhvati je za rep." I on seže rukom i uhvati je  za rep, a ona opet postade štap u njegovoj ruci. 
\par 5 "Tako moraju  vjerovati da se Jahve, Bog njihovih otaca, Bog Abrahamov, Bog  Izakov i Bog Jakovljev, tebi objavio." 
\par 6 Još mu Jahve rekne: "Uvuci ruku u njedra." On uvuče ruku  u njedra. Kad ju je izvukao, gle - ruka mu gubava, bijela kao  snijeg. 
\par 7 "Stavi opet ruku u njedra!" - naredi mu Jahve. On  opet ruku u njedra. Kad ju je iz njedara izvukao, gle - opet  je bila kao i ostali dio tijela. 
\par 8 "Ako ti ne povjeruju i ne  prihvate poruku prvoga znamenja, povjerovat će poruci drugoga  znamenja. 
\par 9 A ako ih oba ova znamenja ne uvjere pa ti ne povjeruju, zahvati vode iz Rijeke i prolij je po suhu. Voda što je budeš  iz Rijeke uzeo na suhu će se u krv pretvoriti." 
\par 10 "Oprosti, Gospodine!" - nastavi Mojsije Jahvi. "Ja nikad  nisam bio čovjek rječit; ni prije ni sada kad govoriš svome sluzi.  Ja sam u govoru spor, a na jeziku težak." 
\par 11 "Tko je dao čovjeku  usta?" - reče mu Jahve. "Tko ga čini nijemim i gluhim; tko li  mu vid daje ili ga osljepljuje? Zar to nisam ja, Jahve! 
\par 12 Idi, dakle! Ja ću biti s tobom kad budeš govorio i kazivat ću ti  što ćeš govoriti." 
\par 13 "Oprosti, Gospodine", opet će Mojsije, "ne bi li poslao  koga drugoga!" 
\par 14 Razljuti se Jahve na Mojsija i reče: "Zar Aron, Levijevac, nije tvoj brat? Znam da je on vrlo rječit. Evo, baš ti izlazi  u susret. Kad te vidi, obradovat će se u srcu. 
\par 15 Ti govori  njemu i u njegova usta stavljaj riječi. Ja ću biti i s tobom  i s njime dok budete govorili; kazivat ću obojici što ćete raditi. 
\par 16 Neka on mjesto tebe govori narodu. Tako, on će tebi biti  mjesto usta, a ti ćeš njemu biti mjesto Boga. 
\par 17 Uzmi ovaj štap  u ruku. Njim izvodi znamenja." 
\par 18 Zatim se Mojsije vrati svome tastu Jitru te mu reče:  "Pusti me da se vratim braći u Egipat da vidim jesu li još na  životu." "Pođi u miru!" - reče Jitro Mojsiju. 
\par 19 I Jahve reče Mojsiju u Midjanu: "Vrati se u Egipat, jer  su pomrli svi ljudi koji su tražili tvoj život." 
\par 20 Tako Mojsije  posadi na magarca svoju ženu i sinove i ode u zemlju egipatsku.  A u ruku Mojsije uze Božji štap. 
\par 21 Jahve opet reče Mojsiju:  "Kad se vratiš u Egipat, pobrini se da pred faraonom izvedeš  sva čudesa za koja sam ti dao moć, premda ću ja tvrdim učiniti  njegovo srce, tako te neće pustiti narod da ode. 
\par 22 Tada reci  faraonu: 'Ovako kaže Jahve: Izrael je moj prvorođenac. 
\par 23 Tražim  od tebe da mi pustiš sina da mi iskaže štovanje. Ako odbiješ  da ga pustiš, ja ću ubiti tvoga prvorođenca.'" 
\par 24 Kad se na putu Mojsije zaustavi da prenoći, navali na  nj Jahve da ga ubije. 
\par 25 Ali Sipora pograbi oštar kremen, obreza  svoga sina i kožicom se dotakne Mojsijevih nogu: "Zaista si mi  ti krvav muž", reče. 
\par 26 I Jahve ga pusti. Ona je to zbog obrezanja  rekla "krvav muž". 
\par 27 Onda rekne Jahve Aronu: "Zaputi se prema pustinji, u  susret Mojsiju!" On ode i s njim se sastane na Božjem brdu. Poljubi  ga. 
\par 28 Mojsije pripovjedi Aronu sve što mu je Jahve povjerio  i sva znamenja koja mu je naredio da ih učini. 
\par 29 Sad odu Mojsije  i Aron i skupe sve starješine Izraelaca. 
\par 30 Aron izloži sve  što je Jahve govorio Mojsiju, a Mojsije izvede znamenja naočigled  naroda. 
\par 31 Narod je bio uvjeren, i pošto čuše da je Jahve pohodio  Izraelce i pogledao na njihove jade, popadaše ničice i pokloniše  se. 


\chapter{5}

\par 1 Poslije toga odu Mojsije i Aron pa reknu faraonu: "Ovako veli  Jahve, Bog Izraelov: 'Pusti narod moj da ode i u moju čast slavi  svetkovinu.'" 
\par 2 "Tko je taj Jahve da ga ja poslušam", odvrati  faraon, "i pustim Izraelce? Ja toga Jahvu ne znam niti ću pustiti  Izraelce." 
\par 3 "Bog Hebreja objavio nam se", rekoše. "Zato nas  pusti da odemo tri dana hoda u pustinju i prinesemo žrtvu Jahvi, Bogu svome, da se na nas ne obori pomorom ili mačem." 
\par 4 Nato  im odvrati egipatski kralj: "Mojsije i Arone, zašto odvraćate  svijet od njegovih dužnosti? Idite na svoj posao. 
\par 5 Sad kad  se svjetina tako umnožila", nastavi faraon, "vi biste ih od posla  odvratili?" 
\par 6 Istoga dana izda faraon naredbu nadglednicima i bilježnicima: 
\par 7 "Ne pribavljajte više ovome narodu slame kao do sada. Neka  idu sami i sebi je skupljaju. 
\par 8 A zahtijevajte od njih istu  količinu opeke koju su pravili i dosad. Ne smanjujte je! Lijenčine  su. Zato viču: 'Hajdemo prinijeti žrtvu Bogu svome!' 
\par 9 Navalite  poslove na taj svijet: neka rade, da ne obraćaju pažnje klevetama!" 
\par 10 Sad dođu nadglednici naroda i njegovi bilježnici te svijetu  objave: "Ovako poručuje faraon: 'Neću vam više nabavljati slame. 
\par 11 Vi sami morate ići i tražiti je gdje god je možete naći.  Ali zato neću smanjiti vaš posao.'" 
\par 12 Stoga se narod raziđe  po svoj zemlji egipatskoj da skuplja strnjiku namjesto slame. 
\par 13 A nadglednici ih gonili: "Morate svakoga dana svršiti jednako  posla kao i onda dok ste slamu dobivali." 
\par 14 A bilježnike koje  faraonovi nadglednici bijahu postavili nad Izraelcima tukli su  i korili: "Zašto niste ni jučer ni danas napravili opeke koliko  i prije?" 
\par 15 Onda bilježnici Izraelaca odu i potuže se faraonu: "Zašto  ovako postupaš sa svojim slugama? 
\par 16 Tvoje sluge više ne dobivaju  slame, a ipak se od nas traži: napravite opeku? Čak i tuku tvoje  sluge, a kriv je tvoj narod!" 
\par 17 "Lijenčine ste vi! Lijenčine!"  - odgovori faraon. "Stoga i kažete: 'Hajdemo da prinesemo žrtvu  Jahvi!' 
\par 18 Nosite se na posao! Slama vam se neće davati, ali  morate praviti određene količine opeke." 
\par 19 Bilježnici Izraelaca nađu se na muci zbog naredbe: "Svakodnevnu  količinu opeke ne smijete smanjiti!" 
\par 20 Otišavši od faraona, naiđu na Mojsija i Arona, koji su ih čekali. 
\par 21 "Neka vas Jahve  ima na oku i sudi vam!" - dobace im. "Omrazili ste nas kod faraona  i njegovih dvorana; dali ste im mač u ruke da nas pobiju." 
\par 22 Mojsije  se vrati Jahvi i reče: "Zašto, Gospodine, nanosiš štetu svome  puku? Zašto si me poslao? 
\par 23 Otkad sam ja stupio pred faraona  i progovorio mu u tvoje ime, on još gore postupa s ovim narodom.  A ti ništa ne poduzimaš da izbaviš svoj narod." 


\chapter{6}

\par 1 Jahve reče Mojsiju: "Naskoro ćeš vidjeti kako ću ja s faraonom!  Pod jakom rukom pustit će ih da odu; pod jakom rukom sam će ih  iz svoje zemlje istjerati." 
\par 2 Još reče Bog Mojsiju: "Ja sam  Jahve. 
\par 3 Abrahamu, Izaku i Jakovu objavljivao sam se kao El  Šadaj. Ali njima se nisam očitovao pod svojim imenom - Jahve. 
\par 4 I sklopio sam svoj Savez s njima da ću im dati kanaansku zemlju, zemlju gdje su živjeli kao pridošlice. 
\par 5 A sada, pošto sam  čuo uzdisaje Izraelaca koje Egipćani drže u ropstvu, sjetih se  svoga Saveza. 
\par 6 Kaži, dakle, Izraelcima da sam ja Jahve; da  ću vas izbaviti od tereta što su vam ga Egipćani nametnuli. Oslobodit  ću vas od ropstva u kojem vas drže; izbavit ću vas udarajući  jako i kažnjavajući strogo. 
\par 7 Za svoj ću vas narod uzeti i bit  ću vašim Bogom. Tada ćete znati da sam vas ja, Jahve, vaš Bog, izbavio od egipatske tlake. 
\par 8 Dovest ću vas u zemlju za koju  sam se zakleo da ću je dati Abrahamu, Izaku i Jakovu i dat ću  vam je u baštinu, ja, Jahve." 
\par 9 Mojsije to kazivaše Izraelcima, ali ga ne htjedoše slušati: duhovi su im bili pomućeni od teškoga  ropstva. 
\par 10 Onda Jahve reče Mojsiju: 
\par 11 "Idi i reci faraonu, kralju  egipatskome, da otpusti Izraelce iz svoje zemlje." 
\par 12 Mojsije  prozbori Jahvi: "Kad me Izraelci nisu slušali, kako će me, spora  u govoru, saslušati faraon!" 
\par 13 Ali je Jahve govorio Mojsiju  i Aronu i slao ih sad k Izraelcima, a sad k faraonu, kralju egipatskome, da pusti Izraelce iz Egipta. 
\par 14 Ovo su glave njihovih domova. Sinovi Izraelova prvorođenca Rubena: Henok, Palu, Hesron  i Karmi. To su obitelji potekle od Rubena. 
\par 15 A sinovi Šimunovi: Jemuel, Jamin, Ohad, Jakin, Sohar  i Šaul, sin Kanaanke. To su obitelji potekle od Šimuna. 
\par 16 Ovo su imena Levijevih sinova s njihovim potomstvom:  Geršon, Kehat i Merari. Levi je živio sto trideset i sedam godina. 
\par 17 Sinovi su Geršonovi: Libni i Šimi sa svojim obiteljima. 
\par 18 Sinovi su Kehatovi: Amram, Jishar, Hebron i Uziel. Kehat  je živio sto trideset i tri godine. 
\par 19 Merarijevi su sinovi: Mahli i Muši. To su Levijeve obitelji  s njihovim potomcima. 
\par 20 Amram se oženi svojom tetkom Jokebedom, koja mu rodi  Arona i Mojsija. Amram je živio sto trideset i sedam godina. 
\par 21 Sinovi Jisharovi bijahu: Korah, Nefeg i Zikri. 
\par 22 A sinovi su Uzielovi: Mišael, Elsafan i Sitri. 
\par 23 Aron se oženi Elišebom, kćerkom Aminadabovom, a sestrom  Nahšonovom, koja mu rodi: Nadaba, Abihua, Eleazara i Itamara. 
\par 24 Korahovi su sinovi: Asir, Elkana i Abiasaf. To su Korahovi  potomci. 
\par 25 Aronov sin Eleazar oženi se jednom Putielovom kćeri,  koja mu rodi Pinhasa. To su glave Levijevih domova prema njihovim koljenima. 
\par 26 To je onaj Aron i Mojsije kojima je Jahve zapovjedio  da izvedu Izraelce iz Egipta po njihovim četama. 
\par 27 To su oni  isti, Mojsije i Aron, koji su govorili faraonu, kralju egipatskome, da pusti Izraelce iz Egipta. 
\par 28 U dan kad je Jahve govorio s Mojsijem u egipatskoj zemlji, 
\par 29 rekao mu je: "Ja sam Jahve. Izvijesti faraona, egipatskoga  kralja, o svemu što ti kažem." 
\par 30 Mojsije se pred Jahvom ispričavao:  "Spor sam ja u govoru. Kako će me faraon poslušati?" 


\chapter{7}

\par 1 Mojsiju je Jahve odgovorio: "Vidi! Faraonu ću te nametnuti  kao božanstvo; tvoj brat Aron bit će tvoj prorok. 
\par 2 Ti kazuj  sve što ti naređujem, a tvoj brat Aron neka faraonu ponovi da  pusti Izraelce te odu iz njegove zemlje. 
\par 3 Ja ću učiniti da  otvrdne srce faraonu i umnožit ću znakove i čudesa u zemlji egipatskoj. 
\par 4 Kako vas faraon neće poslušati, ja ću staviti svoju ruku na  Egipat: strašno kažnjavajući, izbavit ću svoje čete, narod svoj, Izraelce, iz egipatske zemlje. 
\par 5 Kad pružim svoju ruku na Egipat  i izvedem Izraelce iz njihove sredine, tada će Egipćani spoznati  da sam ja Jahve." 
\par 6 Mojsije i Aron poslušaše: kako im je Jahve naredio, upravo  tako učiniše. 
\par 7 Mojsiju je bilo osamdeset, a Aronu osamdeset  i tri godine kad su faraonu postavili svoje zahtjeve. 
\par 8 Još doda Jahve Mojsiju i Aronu: 
\par 9 "Kad faraon zatraži  od vas da izvedete kakvo znamenje, ti reci Aronu da uzme svoj  štap i baci ga pred faraona, a štap će se pretvoriti u zmiju." 
\par 10 Dođu Mojsije i Aron pred faraona i učine kako im je Jahve  naredio. Aron baci pred faraona i njegove službenike svoj štap, koji se pretvori u zmiju. 
\par 11 Zovne faraon mudrace i vračare.  I zaista, egipatski vračari svojim vračanjem učine isto: 
\par 12 svaki  baci svoj štap, koji se pretvori u zmiju. Ali Aronov štap proguta  njihove štapove. 
\par 13 Faraon bijaše tvrdokorna srca: ne htjede  poslušati Mojsija i Arona, kako je Jahve i kazao. 
\par 14 Tada Jahve reče Mojsiju: "Faraonovo je srce okorjelo;  odbija da pusti narod. 
\par 15 Ujutro pođi k faraonu. Kad izađe k  vodi, stani preda nj na obali Rijeke. Uzmi u ruku štap što se  bio u zmiju pretvorio. 
\par 16 Reci mu: 'Jahve, Bog Hebreja, poslao  me k tebi s porukom da pustiš moj narod da mi iskaže štovanje  u pustinji. Ali sve dosad ti nisi poslušao. 
\par 17 Ovako Jahve poručuje:  Ovim ćeš spoznati da sam ja Jahve. Gledaj! Štapom koji imam u  ruci mlatnut ću po vodi u Rijeci i pretvorit će se u krv. 
\par 18 Ribe  će u Rijeci pocrkati; Rijeka će se usmrdjeti, i grstit će se  Egipćanima piti vodu iz Rijeke.'" 
\par 19 Još Jahve reče Mojsiju: "Reci Aronu da uzme svoj štap  i pruži svoju ruku povrh egipatskih voda: njihovih rijeka, njihovih  prokopa, njihovih jezeraca, svih njihovih vodenih stjecišta,  da se pretvore u krv; po svoj zemlji egipatskoj neka je krv,  čak i u drvenim i kamenim posudama." 
\par 20 Mojsije i Aron učiniše  kako im je Jahve naredio. Podiže Aron svoj štap i naočigled faraona  i njegovih službenika mlatnu po vodi u Rijeci. Sva se voda u  Rijeci prometnu u krv. 
\par 21 Ribe u Rijeci pocrkaše; Rijeka se  usmrdje, tako da Egipćani nisu mogli piti vodu iz Rijeke; krv  bijaše po svoj zemlji egipatskoj. 
\par 22 Ali egipatski vračari svojim  vračanjem učiniše isto. Tako faraon ostade tvrdokorna srca: nije  htio poslušati Mojsija i Arona, kako je Jahve i kazao. 
\par 23 Faraon  se okrenu i ode u svoj dvor, ne uzimajući ni to k srcu. 
\par 24 Svi  su Egipćani počeli kopati oko Rijeke tražeći pitke vode jer nisu  mogli piti vode iz Rijeke. 
\par 25 Kad je prošlo sedam dana kako je Jahve udario po Rijeci, 


\chapter{8}

\par 1 (7:26) Opet Jahve reče Mojsiju: "Pođi k faraonu i reci mu: 'Ovako  govori Jahve: Pusti moj narod da ode i štovanje mi iskaže. 
\par 2 (7:27) Ako  odbiješ da ih pustiš, svu ću ti zemlju kazniti žabama. 
\par 3 (7:28) Rijeka  će vrvjeti žabama. One će izići i prodrijeti u tvoj dvor, u ložnicu, u tvoju postelju, u kuće tvojih službenika i tvoga naroda, pod  sačeve i naćve tvoje. 
\par 4 (7:29) Po tebi, po tvome narodu i svim tvojim  službenicima skakat će žabe.'" 
\par 5 (8:1) Onda Jahve reče Mojsiju: "Reci Aronu neka ispruži svoju ruku  sa štapom povrh rijeka, prokopa i jezeraca i učini da žabe navale  na egipatsku zemlju." 
\par 6 (8:2) Aron pruži svoju ruku povrh egipatskih  voda, i žabe iziđoše i prekriše zemlju egipatsku. 
\par 7 (8:3) Ali i vračari  učiniše tako svojim vračanjem, te žabe navališe na egipatsku  zemlju. 
\par 8 (8:4) Zovne sad faraon Mojsija i Arona i rekne: "Molite Jahvu  da ukloni žabe od mene i moga puka, a ja ću pustiti narod da  prinese žrtvu Jahvi." 
\par 9 (8:5) Mojsije uzvrati faraonu: "Dostoj se  odrediti mi kad hoćeš da molim za te, za tvoje službenike i za  tvoj narod da se žabe odstrane od tebe i tvojih domova i ostanu  samo u Rijeci." 
\par 10 (8:6) "Sutra", reče. "Neka bude kako kažeš", odvrati  Mojsije, "da znaš kako nitko nije kao Jahve, Bog naš. 
\par 11 (8:7) Žabe  će otići od tebe, od tvojih službenika i tvoga naroda; ostat  će samo u Rijeci." 
\par 12 (8:8) Mojsije i Aron odu od faraona, a onda Mojsije zazva Jahvu  zbog žaba kojima je kaznio faraona. 
\par 13 (8:9) I Jahve usliša Mojsija, te žabe pocrkaju po kućama, dvorištima i njivama. 
\par 14 (8:10) Na hrpe  su ih zgrtali, zemlja se njima usmrdjela. 
\par 15 (8:11) Kad je faraon vidio  da je nastupilo olakšanje, srce mu otvrdnu te ne posluša Mojsija  i Arona, kako je Jahve i kazao. 
\par 16 (8:12) Onda će opet Jahve Mojsiju: "Reci Aronu neka zamahne  svojim štapom i udari po prahu na tlu neka se pretvori u komarce  po svoj zemlji egipatskoj." 
\par 17 (8:13) I učine tako: zamahne Aron rukom  i štapom te udari po prahu na tlu. Komarci navale na ljude i  životinje. Sav prah na tlu pretvori se u komarce po svoj zemlji  egipatskoj. 
\par 18 (8:14) Vračari pokušaše da svojim vračanjem stvore komarce, ali nisu mogli. Ljudi i životinje postanu plijenom komaraca. 
\par 19 (8:15) Tada vračari reknu faraonu: "To je prst Božji!" Ali je faraonovo  srce bilo okorjelo, pa nije poslušao Mojsija i Arona, kako je  Jahve i kazao. 
\par 20 (8:16) Onda Jahve reče Mojsiju: "Podrani ujutro, iziđi pred  faraona kad krene k vodi, i reci mu: 'Ovako poručuje Jahve: Pusti  moj narod da ode i da mi štovanje iskaže. 
\par 21 (8:17) Ako ne pustiš moga  naroda, pripustit ću obade na te, na tvoje službenike, na tvoj  puk i tvoje domove. Egipatski domovi i samo tlo na kojem stoje  vrvjet će od obada. 
\par 22 (8:18) Ali ću toga dana izuzeti gošenski kraj, u kojem živi moj narod, te se ondje obadi neće pojaviti, tako  da znaš da sam ja Jahve u središtu zemlje. 
\par 23 (8:19) Tu ću razliku  napraviti između svoga i tvoga naroda. To će znamenje biti sutra.'" 
\par 24 (8:20) I učini Jahve tako. Rojevi obada nalete u faraonov dvor,  na domove njegovih službenika i po svoj zemlji egipatskoj. Zemlja  nastrada od obada. 
\par 25 (8:21) Sad faraon pozove Mojsija i Arona pa im rekne: "Idite, prinesite žrtvu svome Bogu, ali u ovoj zemlji." 
\par 26 (8:22) "Ne dolikuje  da tako učinimo", odgovori Mojsije. "Žrtve koje mi prinosimo  Jahvi, Bogu svome, za Egipćane su svetogrđe. Kad bismo, dakle, na njihove oči prinosili žrtve koje su Egipćanima svetogrdne, zar nas ne bi kamenovali? 
\par 27 (8:23) Zato moramo u pustinju tri dana  hoda te prinijeti žrtvu Jahvi, Bogu svome, kako nam je zapovjedio." 
\par 28 (8:24) "Pustit ću vas da odete u pustinju", odgovori faraon, "i  prinesete žrtvu Jahvi, svome Bogu, ali ne odlazite predaleko.  Molite za me!" 
\par 29 (8:25) Nato odvrati Mojsije: "Čim odem od tebe, zazvat  ću Jahvu da sutra nestane obada s faraona, njegovih službenika  i njegova puka. Ali neka faraon više ne vara! Neka pusti narod  da ide i prinese žrtvu Jahvi." 
\par 30 (8:26) Tako Mojsije ode od faraona i pomoli se Jahvi. 
\par 31 (8:27) I  Jahve učini kako je Mojsije tražio: s faraona, s njegovih službenika  i s njegova puka nestane obada - ni jedan jedini nije ostao. 
\par 32 (8:28) Ali opet ukruti faraon srce svoje i ne dopusti narodu da  ode. 


\chapter{9}

\par 1 Tada Jahve reče Mojsiju: "Idi k faraonu i reci mu: 'Ovako poručuje  Jahve, Bog Hebreja: Pusti moj narod da ode i da mi štovanje iskaže. 
\par 2 Ako ga ne pustiš, nego ga i dalje budeš zadržavao, 
\par 3 ruka  Jahvina udarit će strašnim pomorom po tvome blagu što je u polju:  po konjima, magaradi, devama, krupnoj i sitnoj stoci. 
\par 4 Razlikovat  će Jahve stoku Izraelaca od stoke Egipćana, tako da ništa što  pripada Izraelcima neće stradati.'" 
\par 5 Jahve je odredio i vrijeme, rekavši: "Sutra će Jahve izvesti ovo u zemlji." 
\par 6 Sutradan  Jahve tako i učini. Sva stoka Egipćana ugine, a od stoke Izraelaca  nije uginulo ni jedno grlo. 
\par 7 Faraon je istraživao i uvjerio  se da od izraelske stoke nije uginulo ni jedno grlo. Ali je srce  faraonovo ipak otvrdlo i nije pustio naroda. 
\par 8 Reče Jahve Mojsiju i Aronu: "Zagrabite pune pregršti pepela  iz peći, pa neka ga Mojsije pred faraonovim očima baci prema  nebu. 
\par 9 Od toga će nastati sitna prašina po svoj zemlji egipatskoj, i na ljudima će i na životinjama izazivati otekline i stvarati  čireve s kraja na kraj Egipta." 
\par 10 Tako oni uzeše pepela iz  peći i dođoše pred faraona. Onda Mojsije rasu pepeo prema nebu, a otekline s čirevima prekriše ljude i životinje. 
\par 11 Ni čarobnjaci  se nisu mogli pojaviti pred Mojsijem, jer su i čarobnjaci, kao  i ostali Egipćani, bili prekriveni čirevima. 
\par 12 Ali je Jahve  otvrdnuo srce faraonu, pa on ne posluša Mojsija i Arona, kako  je Jahve Mojsiju i rekao. 
\par 13 Tada Jahve reče Mojsiju: "Podrani ujutro, iziđi pred  faraona i reci mu: 'Ovako poručuje Jahve, Bog Hebreja: Pusti  narod da ode i da mi štovanje iskaže. 
\par 14 Ako ih ne pustiš, sva  zla svoja navalit ću ovaj put na te, na tvoje službenike i tvoj  puk, tako da spoznaš da nema nikoga na svoj zemlji kao što sam  ja. 
\par 15 Da sam ruku svoju spustio i udario tebe i tvoj puk pomorom, nestalo bi te sa zemlje. 
\par 16 Poštedio sam te da ti pokažem svoju  moć i da se hvali moje ime po svoj zemlji. 
\par 17 Ali se ti previše  uzdižeš nad mojim narodom i priječiš mu da ode. 
\par 18 Sutra u ovo  doba pustit ću tuču tako strašnu kakve u Egiptu još nije bilo  otkad je postao do sada. 
\par 19 Zato naredi da pod krov utjeraju  tvoje blago i sve što je vani, na otvorenu. Sve što se nađe u  polju, bilo čovjek bilo živinče, ne bude li uvedeno unutra, poginut  će kad tuča zaspe po njima.'" 
\par 20 Faraonovi službenici, koji su se pobojali Jahvina govora, utjeraju svoje sluge i svoje blago unutra. 
\par 21 Oni koji nisu  marili za Jahvinu prijetnju ostave vani i svoje sluge i stoku. 
\par 22 Onda rekne Jahve Mojsiju: "Pruži ruku prema nebu da udari  tuča po svoj zemlji egipatskoj: po ljudima, životinjama i svemu  bilju u zemlji egipatskoj." 
\par 23 Mojsije diže svoj štap prema nebu. Jahve zagrmje i pusti  tuču i munje sastavi sa zemljom. Sipao je Jahve tuču po zemlji  Egipćana. 
\par 24 Tuča je mlatila, kroz nju munje parale. Strahota  se takva nije oborila na zemlju egipatsku otkako su ljudi u njoj. 
\par 25 Tuča pobi po svem Egiptu sve što je ostalo vani, ljude i  životinje; uništi sve bilje po poljima i sva stabla poljska polomi. 
\par 26 Samo u gošenskom kraju, gdje su živjeli Izraelci, nije bilo  tuče. 
\par 27 Faraon posla po Mojsija i Arona pa im reče: "Ovaj put  priznajem da sam kriv. Jahve ima pravo, a ja i moj narod krivo. 
\par 28 Molite Jahvu da ustavi gromove i tuču, a ja ću vas pustiti  da idete. Nećete više dugo ostati." 
\par 29 "Kad iziđem iz grada", reče mu Mojsije, "dići ću ruke prema Jahvi, pa će gromovi prestati, a ni tuče više neće biti, tako da znaš da zemlja pripada Jahvi. 
\par 30 Ali ni ti ni tvoji dvorani, znam ja, još se ne bojite Boga  Jahve." 
\par 31 I tako propade lan i ječam: jer ječam bijaše u klasu, a lan u cvatu. 
\par 32 Pšenica i raž nisu nastradali jer su ozima  žita. 
\par 33 Otišavši od faraona, Mojsije iziđe iz grada i podigne  ruke prema Jahvi. Prestane grmljavina i tuča, a ni kiša više  nije padala na zemlju. 
\par 34 Kad je faraon vidio da je prestala  grmljavina, tuča i kiša, opet padne u grijeh: i on i njegovi  službenici opet otvrdnu srcem. 
\par 35 Otvrdnu srce faraonu i ne  pusti on Izraelce, kako je Jahve i prorekao preko Mojsija. 


\chapter{10}

\par 1 Reče Jahve Mojsiju: "Idi k faraonu. Učinio sam da njemu i  njegovim službenicima otvrdne srce da izvedem svoja znamenja  među njima; 
\par 2 da možeš pripovjedati svome sinu i svome unuku  što sam učinio Egipćanima i kakva sam znamenja izvodio među njima, kako biste znali da sam ja Jahve." 
\par 3 Tako Mojsije i Aron odu  k faraonu i kažu mu: "Ovako poručuje Jahve, Bog Hebreja: 'Dokle  ćeš odbijati da se preda mnom poniziš? Pusti moj narod da mi  iskaže štovanje. 
\par 4 Jer ako ne pustiš moga naroda, sutra ću navesti  skakavce na tvoju zemlju. 
\par 5 Tako će prekriti površinu da se  zemlja od njih neće vidjeti. Pojest će ono što vam je iza tuče  ostalo; i ogolit će vam sva stabla što po polju rastu. 
\par 6 Ispunit  će ti sav dvor, kuće tvojih službenika i domove svih ostalih  Egipćana - takvo što ne vidješe ni tvoji očevi ni očevi tvojih  očeva u ovoj zemlji od svojih vremena do danas.'" Okrene se i  ode od faraona. 
\par 7 "Dokle će nam ovaj čovjek biti stupica?" -  rekoše faraonu njegovi službenici. - "Pusti te ljude neka idu  i iskažu štovanje Jahvi, svome Bogu! Zar ne vidiš kako Egipat  srlja u propast?" 
\par 8 Dovedu Mojsija i Arona natrag k faraonu, a on im reče:  "Idite! Iskažite štovanje Jahvi, svome Bogu! A tko će sve ići?" 
\par 9 "Svi idemo", odgovori Mojsije, "i mlado i staro. Odlazimo  sa svojim sinovima i svojim kćerima; sa svojom krupnom i sitnom  stokom, jer moramo održati svečanost Jahvi." 
\par 10 "Jahve bio s  vama isto kao što i ja pustio da s vama pođu i djeca!" - odgovori  im. "Očito se vidi da vam nakana nije čista. 
\par 11 Nećemo tako!  Nego muškarci neka odu i štovanje iskažu Jahvi. To ste i tražili."  I otjeraju ih od faraona. 
\par 12 Tada reče Jahve Mojsiju: "Pruži ruku povrh zemlje egipatske  da navale skakavci na egipatsku zemlju i pojedu sve bilje što  još ostade nakon tuče!" 
\par 13 Tako Mojsije podigne svoj štap povrh  egipatske zemlje, a Jahve navrati istočni vjetar po zemlji; puhao  je toga cijelog dana i cijele noći. A kad je jutro svanulo, vjetar  nanio skakavce. 
\par 14 Oni se razlete po svoj egipatskoj zemlji i padnu po svim  krajevima Egipta u silnoj gustoći: toliko ih mnoštvo nikad prije  nije bilo niti će kada biti. 
\par 15 Pokriju sve tlo, tako da se  od njih zacrnjelo. Pojedu sve bilje u polju i sve plodove sa  stabala što su bili ostali iza tuče. Ništa se više nije zelenjelo:  ni stabla ni poljska trava u svem Egiptu. 
\par 16 Brže-bolje dozva faraon Mojsija i Arona pa im reče: "Sagriješio  sam protiv Jahve, vašega Boga, i vas! 
\par 17 Oprostite mi uvredu  još samo ovaj put i molite Jahvu, Boga svoga, da samo otkloni  od mene ovaj smrtonosni bič!" 
\par 18 Kad je Mojsije otišao od faraona, zazva Jahvu 
\par 19 i Jahve promijeni vjetar u veoma jak zapadnjak, koji pothvati skakavce i odnese prema Crvenome moru. Ni jedan  jedini skakavac nije ostao ni u kojem kraju Egipta. 
\par 20 Ali je  Jahve otvrdnuo srce faraonu i ne pusti on Izraelaca. 
\par 21 "Pruži ruku prema nebu", rekne Jahve Mojsiju, "pa neka  se tmina spusti na egipatsku zemlju, tmina koja će se moći opipati." 
\par 22 Mojsije pruži ruku prema nebu i spusti se gusta tmina na  svu zemlju egipatsku: tri je dana trajala. 
\par 23 Tri dana nisu  ljudi jedan drugoga mogli vidjeti i nitko se sa svoga mjesta  nije micao. A u mjestima gdje su Izraelci živjeli sjala svjetlost. 
\par 24 Pozva onda faraon Mojsija i reče: "Idi i štovanje iskaži  Jahvi! Ali vaša stoka, krupna i sitna, neka ostane ovdje. Vaša  djeca neka idu s vama!" 
\par 25 "Ti nas sam moraš opskrbiti prinosima  i žrtvama paljenicama koje ćemo prinijeti Jahvi, Bogu svojemu", odgovori Mojsije. 
\par 26 "Zato ćemo sa sobom potjerati i svoja  stada. Ni papak neće ostati ovdje. Od njih nam valja izabrati  za žrtvovanje Jahvi, Bogu našemu, a ne znamo, dok onamo ne stignemo, što moramo Jahvi prinijeti." 
\par 27 Jahve otvrdne faraonu srce i on ne pristane da odu. 
\par 28 "Odlazi!"  - vikne faraon na Mojsija. "I da mi više na oči ne dolaziš! Onoga  dana kad mi se opet pojaviš na oči, zaglavit ćeš!" 
\par 29 "Dobro  si kazao!" - uzvrati Mojsije. "Lica tvoga više neću vidjeti!" 


\chapter{11}

\par 1 "Još ću samo jednom nedaćom udariti faraona i Egipat", reče  Jahve Mojsiju. "Poslije toga pustit će vas odavde. I više: sam  će vas odavde potjerati. 
\par 2 Kaži svijetu neka svaki čovjek ište  od svoga susjeda i svaka žena od svoje susjede srebrnih i zlatnih  dragocjenosti." 
\par 3 Jahve učini te Egipćani bijahu naklonjeni  narodu. Sam Mojsije postane vrlo uvažen u egipatskoj zemlji,  u očima faraonovih službenika i u očima naroda. 
\par 4 A onda Mojsije navijesti: "Ovako poručuje Jahve: 'O ponoći  proći ću Egiptom. 
\par 5 Svaki će prvorođenac u egipatskoj zemlji  umrijeti, od prvorođenca faraonova, koji bi imao sjediti na njegovu  prijestolju, do prvorođenca ropkinje koja se nalazi uz mlinski  kamen; a uginut će i sve prvine od stoke. 
\par 6 U svoj će zemlji  egipatskoj nastati veliki jauk, kakva nije bilo niti će kad poslije  biti. 
\par 7 Među Izraelcima ni pas neće zalajati na živo stvorenje:  ni na čovjeka ni na životinju.' Po tome ćete znati da Jahve luči  Izraelca od Egipćanina. 
\par 8 Onda će svi ovi tvoji dvorani k meni  doći, preda me se baciti i vikati: Nosi se i ti i sav puk koji  za tobom ide! Poslije toga ću otići." I gnjevan ode od faraona. 
\par 9 "Neće vas faraon poslušati", reče Jahve Mojsiju, "a to  da bi se umnožila moja znamenja u zemlji egipatskoj." 
\par 10 Mojsije i Aron izveli su sva ta znamenja pred faraonom, ali je Jahve okorio srce faraonu, tako te on nije puštao Izraelaca  da odu iz njegove zemlje. 


\chapter{12}

\par 1 Jahve reče Mojsiju i Aronu u zemlji egipatskoj: 
\par 2 "Ovaj mjesec  neka vam bude početak mjesecima; neka vam bude prvi mjesec u  godini. 
\par 3 Ovo objavite svoj zajednici izraelskoj: Desetog dana  ovoga mjeseca neka svatko za obitelj pribavi jedno živinče. Tako, jedno na obitelj. 
\par 4 Ako je obitelj premalena da ga potroši, neka se ona priključi svome susjedu, najbližoj kući, prema broju  osoba. Podijelite živinče prema tome koliko koja osoba može pojesti. 
\par 5 Živinče neka bude bez mane, od jedne godine i muško. Možete  izabrati bilo janje bilo kozle. 
\par 6 Čuvajte ga do četrnaestoga  dana ovoga mjeseca. A onda neka ga sva izraelska zajednica zakolje  kad se spusti suton. 
\par 7 Neka uzmu krvi i poškrope oba dovratnika  i nadvratnik kuće u kojoj se bude blagovalo. 
\par 8 Meso, pečeno  na vatri, neka se pojede te iste noći sa beskvasnim kruhom i  gorkim zeljem. 
\par 9 Da ništa sirovo ili na vodi skuhano od njega  niste jeli, nego na vatri pečeno: s glavom, nogama i ponutricom. 
\par 10 Ništa od njega ne smijete ostaviti za sutradan: što bi god  do jutra ostalo, morate na vatri spaliti. 
\par 11 A ovako ga blagujte:  opasanih bokova, s obućom na nogama i sa štapom u ruci. Jedite  ga žurno: to je Jahvina pasha. 
\par 12 Te ću, naime, noći ja proći  egipatskom zemljom i pobiti sve prvorođence u zemlji egipatskoj  - i čovjeka i životinju. Ja, Jahve, kaznit ću i sva egipatska  božanstva. 
\par 13 Krv neka označuje kuće u kojima vi budete. Gdje  god spazim krv, proći ću vas; tako ćete vi izbjeći biču zatornomu  kad se oborim na zemlju egipatsku." 
\par 14 "Taj dan neka vam bude spomen-dan. Slavite ga kao blagdan  u čast Jahvi. Svetkujte ga po trajnoj uredbi od koljena do koljena. 
\par 15 Sedam dana jedite beskvasan kruh. Prvoga već dana uklonite  kvasac iz svojih kuća. Jer, tko bi god od prvoga do sedmoga dana  jeo ukvasan kruh, taj se ima iskorijeniti između Izraelaca. 
\par 16 Prvoga  dana držite sveto zborovanje, a tako i sedmoga dana. Nikakva  posla tih dana nemojte raditi. Jedino jelo, što kome treba, možete  pripraviti. 
\par 17 Držite blagdan beskvasnog kruha! Toga sam, naime, dana izveo vaše čete iz zemlje egipatske. Držite zato taj dan  kao blagdan od koljena do koljena: to je vječna naredba. 
\par 18 Od  večeri četrnaestoga dana prvoga mjeseca pa do večeri dvadeset  prvoga dana toga mjeseca jedite beskvasan kruh. 
\par 19 Sedam dana  ne smije biti kvasca u vašim domovima. Tko bi god jeo bilo što  ukvasano, taj neka se ukloni iz izraelske zajednice, bio stranac  ili domorodac. 
\par 20 Ništa ukvasano ne smijete jesti: u svim svojim  prebivalištima jedite nekvasan kruh." 
\par 21 Zatim sazva Mojsije sve starješine Izraelaca te im reče:  "Idite i pribavite janje za svoje obitelji i žrtvujte Pashu. 
\par 22 Onda uzmite kitu izopa, zamočite je u krv što je u zdjeli  i poškropite krvlju iz zdjele nadvratnik i oba dovratnika. Neka  nitko ne izlazi preko kućnih vrata do jutra. 
\par 23 Kad Jahve bude  prolazio da pobije Egipćane, zapazit će krv na nadvratniku i  na oba dovratnika, pa će mimoići ta vrata i neće dopustiti da  Zatornik uđe u vaše kuće da hara. 
\par 24 Ovu uredbu držite u svim  vremenima kao zakon za se i djecu svoju. 
\par 25 I kad dođete u zemlju  koju će vam Jahve dati kako je obećao, vršite ovaj obred. 
\par 26 Kad  vas vaša djeca zapitaju: Što vam taj obred označuje? 
\par 27 odgovorite  im: Ovo je pashalna žrtva u čast Jahvi koji je prolazio mimo  kuće Izraelaca kad je usmrćivao Egipćane, a naše kuće pošteđivao."  Tada narod popada ničice i pokloni se. 
\par 28 Potom Izraelci odu i poslušaju: kako je Jahve Mojsiju  i Aronu naredio, tako i učine. 
\par 29 U ponoći Jahve pobije sve prvorođence po zemlji egipatskoj:  od prvorođenca faraonova, koji je imao sjediti na prijestolju, do prvorođenca sužnja u tamnici, a tako i sve prvine od stoke. 
\par 30 Noću ustane faraon, on pa svi njegovi dvorani i svi Egipćani, jer se strašan jauk razlijegao Egiptom: ne bijaše kuće u kojoj  nije ležao mrtvac. 
\par 31 Faraon pozva u noći Mojsija i Arona te  im reče: "Ustajte i odlazite od moga naroda i vi i vaši Izraelci!  Idite! Odajte štovanje Jahvi, kako ste tražili. 
\par 32 Pokupite  svoju i sitnu i krupnu stoku, kako ste zahtijevali: idite pa  i mene blagoslovite!" 
\par 33 Egipćani nagonili narod da brže ide  iz zemlje, "jer izgibosmo svi", govorahu oni. 
\par 34 Tako narod  ponese svoje još neukislo tijesto; naćve, uvijene u ogrtače,  ponesoše na ramenima. 
\par 35 I učiniše Izraelci kako im je Mojsije bio rekao: zatražiše  od Egipćana srebrnine, i zlatnine, i odjeće. 
\par 36 Jahve je učinio  te Egipćani bijahu naklonjeni narodu pa davahu. Tako su Egipćane  oplijenili. 
\par 37 Pođu tako Izraelci iz Ramsesa prema Sukotu. Bilo je oko  šest stotina tisuća pješaka, osim žena i djece. 
\par 38 A mnogo i  drugoga svijeta pođe s njima, i mnoga stoka, krupna i sitna. 
\par 39 Ispeku beskvasne prevrte od tijesta što su ga iz Egipta ponijeli:  nije se bilo ukvasalo. A kako su bili tjerani iz Egipta, nisu  mogli odgađati, i tako nisu sebi spremili poputninu. 
\par 40 Vrijeme  što su ga Izraelci proveli u Egiptu iznosilo je četiri stotine  i trideset godina. 
\par 41 I kad se navrši četiri stotine i trideset  godina - točno onoga dana - sve čete Jahvine iziđoše iz zemlje  egipatske. 
\par 42 Ona noć koju je Jahve probdio da njih izbavi iz  Egipta, odonda je svima Izraelcima, u sve naraštaje njihove,  noć bdjenja u čast Jahvi. 
\par 43 Reče Jahve Mojsiju i Aronu: "Neka je ovo pravilo za pashalnu  žrtvu: ni jedan stranac ne smije od nje jesti! 
\par 44 Svaki rob, kupljen novcem i obrezan, može je jesti. 
\par 45 Ni gost ni najamnik  ne smiju je jesti! 
\par 46 Blagujte je u jednoj te istoj kući; iz  kuće ne smijete iznositi mesa niti na žrtvi smijete koju kost  slomiti. 
\par 47 Sva zajednica Izraelaca neka je prikazuje! 
\par 48 Ako  bi stranac koji među vama boravi htio svetkovati Pashu u čast  Jahvi, svi se njegovi muški moraju obrezati. Tek tada neka pristupi  i slavi je, jer je tada kao i domorodac zemlje. Ali neobrezani  ne smije od nje jesti. 
\par 49 Neka vrijedi isto pravilo za domoroca  i pridošlicu koji među vama boravi." 
\par 50 Svi Izraelci poslušaju:  kako je Jahve naredio Mojsiju i Aronu, tako su i učinili. 
\par 51 Toga  istog dana izbavio je Jahve Izraelce u njihovim četama iz zemlje  egipatske. 


\chapter{13}

\par 1 Jahve reče Mojsiju: 
\par 2 "Meni posvetite svakoga prvorođenca!  Prvenci materina krila kod Izraelaca, i od ljudi i od životinja, meni pripadaju!" 
\par 3 A onda Mojsije reče narodu: "Sjećajte se ovoga dana u  koji ste izbavljeni iz Egipta, iz kuće ropstva, jer vas Jahve  izbavi odande svojom jakom mišicom. Ukvasani kruh neka se ne  jede! 
\par 4 Ovoga dana mjeseca Abiba vaše je izbavljenje. 
\par 5 Stoga:  kad te Jahve uvede u zemlju Kanaanaca, Hetita, Amorejaca, Hivijaca  i Jebusejaca, za koju se zakleo tvojim precima da će ti je dati  - zemlju kojom teče med i mlijeko - ovoga mjeseca obavi ovakav  obred: 
\par 6 sedam dana jedi nekvasan kruh, a sedmoga dana neka  se slavi svetkovina u čast Jahvi. 
\par 7 Sedam dana neka se jede  nekvasan kruh; ukvasanog kruha neka ne bude kod tebe; i neka  se nigdje ne vidi kvasac na tvome području. 
\par 8 Svome sinu toga  dana objasni: to je za ono što mi je Jahve učinio kad sam se  iz Egipta izbavio. 
\par 9 Neka ti bude kao znak na tvojoj ruci i  kao opomena na tvome čelu: da Jahvin zakon bude uvijek na tvojim  ustima. Jer te rukom jakom Jahve izbavio iz Egipta. 
\par 10 Ovaj  propis vršite svake godine u određeno vrijeme." 
\par 11 "A kada te Jahve dovede u zemlju Kanaanaca - kako vam  se zakle, tebi i tvojim ocima - i kada ti je preda, 
\par 12 ustupajte  Jahvi prvorođence materinjega krila, a tako i sve prvine što  ih tvoja stoka dade - svako muško pripada Jahvi! 
\par 13 Svaku prvinu  magaradi otkupi janjetom ili jaretom. Ako je ne otkupiš, slomi  joj vrat. A svakoga prvorođenca između svoje djece otkupi. 
\par 14 Kad  te sin tvoj sutra zapita: Što znači to? - odgovori mu: Rukom  jakom izvede nas Jahve iz Egipta, iz kuće ropstva. 
\par 15 Kako je  faraon postao tvrdokoran pa nas nije htio pustiti, Jahve je poubijao  sve prvorođence u zemlji egipatskoj: prvorođence ljudi i prvine  stoke. Eto zato Jahvi žrtvujem svaku mušku prvinu materinjega  krila, a svakoga prvorođenca od svojih sinova otkupljujem. 
\par 16 Neka  ti to bude kao znak na tvojoj ruci i kao znamenje posred čela  da nas je rukom jakom Jahve izbavio iz Egipta." 
\par 17 Kad je faraon dopustio da narod ode, Bog ih nije poveo  prema filistejskoj zemlji, iako je onuda bilo najbliže. Bog je, naime, rekao: "Mogao bi se narod predomisliti i vratiti u Egipat  kad vidi ratovanje." 
\par 18 Stoga Bog povede narod zaobilaznim putem, kroz pustinju prema Crvenome moru. Izraelci su napustili zemlju  egipatsku naoružani od glave do pete. 
\par 19 Mojsije ponese sa sobom  Josipove kosti. Jer Josip bijaše zakleo Izraelce riječima: "Bog  će se zacijelo za vas zauzeti. Tada i moje kosti odavde ponesite  sa sobom!" 
\par 20 Krenuvši iz Sukota, utabore se u Etamu, na kraju pustinje. 
\par 21 Jahve je išao pred njima, danju u stupu od oblaka da im put  pokazuje, a noću u stupu od ognja da im svijetli. Tako su mogli  putovati i danju i noću. 
\par 22 I nije ispred naroda nestajao stup  od oblaka danju ni stup od ognja noću. 


\chapter{14}

\par 1 Jahve reče Mojsiju: 
\par 2 "Reci Izraelcima da se vrate i utabore  pred Pi-Hahirotom, između Migdola i mora, nasuprot Baal-Sefonu.  Utaborite se nasuprot ovome mjestu, uz more. 
\par 3 Faraon će reći:  'Izraelci lutaju krajem tamo-amo; pustinja ih zatvorila.' 
\par 4 Ja  ću otvrdnuti faraonu srce, i on će za njima poći u potjeru. Ali  ja ću se proslaviti nad faraonom i svom njegovom vojskom. Tako  će Egipćani spoznati da sam ja Jahve." Izraelci tako učine. 
\par 5 Kad su egipatskom kralju kazali da je narod pobjegao,  faraon i njegovi dvorani predomisliše se o narodu. "Što ovo učinismo!"  - rekoše. "Pustismo Izraelce i više nam neće služiti." 
\par 6 Zato  opremi faraon svoja kola i povede svoju vojsku. 
\par 7 Uze šest stotina  svojih kola sve poizbor i ostala kola po Egiptu. I u svima bijahu  štitonoše. 
\par 8 Jahve otvrdnu srce faraonu, kralju egipatskom,  te on krenu u potjeru za Izraelcima, koji su otišli uzdignute  pesnice. 
\par 9 Egipćani, dakle, pođu za njima u potjeru. I dok su  Izraelci taborovali uz more, blizu Pi-Hahirota nasuprot Baal-Sefonu, stignu ih svi faraonovi konji pod kolima, njegovi konjanici  i njegovi ratnici. 
\par 10 Kako se faraon približavao, Izraelci pogledaju  i opaze da su Egipćani za njima u potjeri, pa ih obuzme velik  strah. I poviču Izraelci Jahvi: 
\par 11 "Zar nije bilo grobova u  Egiptu", reknu Mojsiju, "pa si nas izveo da pomremo u pustinji?  Kakvu si nam uslugu učinio što si nas izveo iz Egipta! 
\par 12 Zar  ti nismo rekli baš ovo u Egiptu: Pusti nas! Služit ćemo Egipćane!  Bolje nam je i njih služiti nego u pustinji poginuti." 
\par 13 "Ne  bojte se!" - reče Mojsije narodu. "Stojte čvrsto pa ćete vidjeti  što će vam Jahve učiniti da vas danas spasi: Egipćane koje danas  vidite nikad više nećete vidjeti. 
\par 14 Jahve će se boriti za vas.  Budite mirni!" 
\par 15 "Zašto zapomažete prema meni?" - reče Jahve Mojsiju.  "Reci Izraelcima da krenu na put. 
\par 16 A ti podigni svoj štap, ispruži svoju ruku nad morem i razdijeli ga nadvoje da Izraelci  mogu proći posred mora po suhu. 
\par 17 Ja ću otvrdnuti srce Egipćana, i oni će poći za njima, a ja ću se onda proslaviti nad faraonom  i njegovim ratnicima, njegovim kolima i konjanicima. 
\par 18 Neka  znaju Egipćani da sam ja Jahve kad se proslavim nad faraonom, njegovim kolima i njegovim konjanicima." 
\par 19 Anđeo Božji, koji je išao na čelu izraelskih četa, promijeni  mjesto i stupi im za leđa. A i stup od oblaka pomakne se ispred  njih i stade im za leđa. 
\par 20 Smjesti se između vojske egipatske  i vojske izraelske te postade onima oblak taman, a ovima rasvjetljivaše  noć, tako te ne mogoše jedni drugima prići cijele noći. 
\par 21 Mojsije  je držao ruku ispruženu nad morem dok je Jahve svu noć na stranu  valjao vode jakim istočnim vjetrom i more posušio. Kad su se  vode razdvojile, 
\par 22 Izraelci siđoše u more na osušeno dno, a  vode stajahu kao bedem njima nadesno i nalijevo. 
\par 23 Egipćani:  svi faraonovi konji, kola i konjanici, nagnu za njima u more, u potjeru. 
\par 24 Za jutarnje straže pogleda Jahve iz stupa od  ognja i oblaka na egipatsku vojsku i u njoj stvori zbrku. 
\par 25 Zakoči  točkove njihovih kola da su se jedva naprijed micali. "Bježimo  od Izraelaca!" - poviču Egipćani, "jer Jahve se za njih bori  protiv Egipćana!" Tada će Jahve Mojsiju: 
\par 26 "Pruži ruku nad  more da se vode vrate na Egipćane, na njihova kola i konjanike." 
\par 27 Mojsije pruži ruku nad more i u cik zore more se vrati u  svoje korito. Kako su Egipćani, bježeći, jurili prema moru, Jahve  ih strmoglavi usred voda. 
\par 28 Tako vode, slijevajući se natrag, potope kola, konjanike i svu vojsku faraonovu koja bijaše pošla  u potjeru za Izraelcima - u more. I ne ostade od njih ni jedan  jedini. 
\par 29 A Izraelci išli suhim posred mora, vode im stale  kao zid zdesna i slijeva. 
\par 30 Tako Jahve u onaj dan izbavi Izraela  iz šaka egipatskih, i vidje Izrael pomorene Egipćane na morskome  žalu. 
\par 31 Osvjedoči se Izrael i o silnoj moći koju Jahve pokaza  nad Egipćanima. Narod se poboja Jahve i povjerova Jahvi i njegovu  sluzi Mojsiju. 


\chapter{15}

\par 1 Tada Mojsije s Izraelcima zapjeva ovu pjesmu Jahvi u slavu: "U čast Jahvi zapjevat ću, jer se slavom proslavio! Konja s konjanikom u more je survao. 
\par 2 Moja je snaga, moja pjesma - Jahve jer je mojim postao izbaviteljem. On je Bog moj, njega ja ću slaviti, on je Bog oca moga, njega ću veličati. 
\par 3 Jahve je ratnik hrabar, Jahve je ime njegovo. 
\par 4 Kola faraonova i vojsku mu u more baci; cvijet njegovih štitonoša More crveno proguta. 
\par 5 Valovi ih prekriše; poput kamena u morske potonuše dubine. 
\par 6 Desnica tvoja, Jahve, snagom se prodiči; desnica tvoja, Jahve, raskomada dušmana. 
\par 7 Veličanstvom svojim obaraš ti protivnike; puštaš svoj gnjev i on ih k'o slamu proždire. 
\par 8 Od daha iz tvojih nosnica vode narastoše, valovi se u bedem uzdigoše, u srcu mora dubine se stvrdnuše. 
\par 9 Mislio je neprijatelj: 'Gonit ću ih, stići, plijen ću podijelit', duša će moja sita ga biti; trgnut ću mač, uništit' ih rukom svojom.' 
\par 10 A ti dahom svojim dahnu, more se nad njima sklopi; k'o olovo potonuše silnoj vodi u bezdane. 
\par 11 Tko je kao ti, Jahve među bogovima, tko kao ti sija u svetosti, u djelima strašan, divan u čudima? 
\par 12 Desnicu si pružio i zemlja ih proguta! 
\par 13 Milošću svojom vodio si ovaj narod, tobom otkupljen, k svetom tvom Stanu snagom si ga svojom upravio. 
\par 14 Kada to čuše, prodrhtaše narodi; Filistejce muke spopadoše. 
\par 15 Užas je srvao edomske glavare, trepet je obuzeo moapske knezove i tresu se svi koji žive u Kanaanu. 
\par 16 Strah i prepast na njih se obaraju; snaga tvoje ruke skamenila ih je dok narod tvoj, Jahve, ne prođe, dok ne prođe narod tvoj koji si otkupio. 
\par 17 Dovest ćeš ih i posaditi na gori svoje baštine, na mjestu koje ti, Jahve, svojim učini Boravištem, Svetištem, o Jahve, tvojom rukom sazidanim. 
\par 18 Vazda i dovijeka Jahve će kraljevati." 
\par 19 Kad su faraonovi konji, njegova kola i konjanici sašli  u more, Jahve je na njih povratio morske vode pošto su Izraelci  prošli posred mora po suhu. 
\par 20 Tada Aronova sestra, proročica  Mirjam, uze bubanj u ruku, a sve žene pridruže joj se s bubnjem  u ruci i plešući. 
\par 21 Mirjam je začinjala pjesmu: "Zapjevajte Jahvi jer se slavom proslavio! Konja s konjanikom u more je survao." 
\par 22 Pokrene Mojsije Izraelce od Crvenog mora i pođu na put  kroz pustinju Šur. Tri su dana putovali pustinjom, a vode nisu  našli. 
\par 23 Dođu k Mari, ali nisu mogli piti vode kod Mare jer  je bila gorka. Stoga se i zove Mara. 
\par 24 Narod je mrmljao na  Mojsija i govorio: "Što ćemo piti?" 
\par 25 A on zazva Jahvu. Jahve  mu pokaže neko drvo. Baci on to drvo u vodu i voda postane slatka. Tu im Jahve postavi zakon i pravo i tu ih stavi u kušnju. 
\par 26 Zatim reče: "Budeš li zdušno slušao glas Jahve, Boga svoga, vršeći što je pravo u njegovim očima; budeš li pružao svoje  uho njegovim zapovijedima i držao njegove zakone, nikakvih bolesti  koje sam pustio na Egipćane na vas neću puštati. Jer ja sam Jahve  koji dajem zdravlje." 
\par 27 Zatim stignu u Elim, gdje je bilo dvanaest izvora i sedamdeset  palma. Tu se, uz vodu, utabore. 


\chapter{16}

\par 1 Potom krenu iz Elima, i sva izraelska zajednica dođe u pustinju  Sin, koja je između Elima i Sinaja, petnaestoga dana drugoga  mjeseca nakon odlaska iz zemlje egipatske. 
\par 2 U pustinji sva  izraelska zajednica počne mrmljati protiv Mojsija i Arona. 
\par 3 "Oh, da smo pomrli od ruke Jahvine u zemlji egipatskoj kad smo sjedili  kod lonaca s mesom i jeli kruha do mile volje!" - rekoše im.  "Izveli ste nas u ovu pustinju da sve ovo mnoštvo gladom pomorite!" 
\par 4 Tada reče Jahve Mojsiju: "Učinit ću da vam daždi kruh s neba.  Neka narod ide i skuplja svaki dan koliko mu za dan treba. Tako  ću ih kušati i vidjeti hoće li se držati moga zakona ili neće. 
\par 5 A šestoga dana, kad spreme što su nakupili, bit će dvaput  onoliko koliko su skupljali za svaki dan." 
\par 6 Onda Mojsije i Aron progovore svim Izraelcima: "Večeras  ćete poznati da vas je Jahve izveo iz zemlje egipatske, 
\par 7 a  ujutro ćete vidjeti svojim očima Jahvinu slavu, jer vas je čuo  Jahve kako ste protiv njega mrmljali. Što smo mi da protiv nas  mrmljate? 
\par 8 Večeras će vam Jahve dati mesa da jedete", nastavi  Mojsije, "a ujutro kruha do mile volje, jer je Jahve čuo vaše  mrmljanje protiv njega. Što smo mi? Vi ne mrmljate protiv nas  nego protiv Jahve." 
\par 9 Poslije toga rekne Mojsije Aronu: "Reci svoj izraelskoj  zajednici: 'Skupite se pred Jahvu, jer je čuo vaše mrmljanje!'" 
\par 10 I dok je Aron svoj izraelskoj zajednici govorio, oni se okrenu  prema pustinji, i gle! u oblaku pojavi se Jahvina slava. 
\par 11 Onda  se Jahve oglasi Mojsiju i reče mu: 
\par 12 "Čuo sam mrmljanje Izraelaca.  Ovako im reci: 'Večeras ćete jesti meso, a ujutro ćete se nasititi  kruha. Tada ćete poznati da sam ja Jahve, Bog vaš.'" 
\par 13 I doista!  Navečer se pojave prepelice i prekriju tabor. A ujutro obilna  rosa sve orosila oko tabora. 
\par 14 Kad se prevlaka rose digla,  površinom pustinje ležao tanak sloj, nešto poput pahuljica, kao  da se slana uhvatila po zemlji. 
\par 15 Kad su Izraelci to vidjeli, pitali su jedan drugoga: "Što je to?" Jer nisu znali što je.  Onda im Mojsije reče: "To je kruh koji vam je Jahve pribavio  za hranu. 
\par 16 A ovo je zapovijed koju je Jahve izdao: 'Nakupite  koliko kome treba za jelo - jedan gomer po osobi, svatko prema  broju članova koji su mu u šatoru.'" 
\par 17 Izraelci tako uradiše. Neki nakupe više, neki manje. 
\par 18 Kad su izmjerili na gomer, pokaza se da nije ništa preteklo  onome koji bijaše nakupio mnogo, a niti je nedostajalo onome  koji bijaše nakupio manje: svatko je nakupio koliko mu je trebalo  za jelo. 
\par 19 "Neka nitko ne ostavlja ništa za ujutro!" - rekne im  Mojsije. 
\par 20 Ali oni nisu poslušali Mojsija; neki ostave i za  sutra. A to im se ucrva i usmrdje. Mojsije se na njih razljuti. 
\par 21 Tako su skupljali svako jutro koliko je kome trebalo za jelo.  I kad bi sunce ogrijalo, mÓana bi se rastopila. 
\par 22 Onda šestoga dana nakupiše dvostruku količinu hrane -  po dva gomera na svakoga. Kad su starješine zajednice došle da  izvijeste Mojsija, 
\par 23 on im reče: "Ovo je zapovijed Jahvina:  Sutra je dan potpunog odmora, subota Jahvi posvećena. Ispecite  što želite peći; skuhajte što želite kuhati. Sve što vam preteče  ostavite za sutra." 
\par 24 Ostave to oni za sutra, kako je Mojsije  naredio, i niti se usmrdjelo niti su se crvi pojavili. 
\par 25 "Jedite  to danas", reče im Mojsije, "jer je ovaj dan subota u čast Jahve;  danas nećete naći mÓane na polju. 
\par 26 Šest je dana skupljajte, a sedmoga, u subotu, neće je biti." 
\par 27 Bijaše nekih koji su  i sedmoga dana išli da je nakupe, ali ništa ne nađoše. 
\par 28 Zato  Jahve reče Mojsiju: "Dokle ćete odbijati da se pokorite mojim  zapovijedima i mojim zakonima? 
\par 29 Pogledajte! Zato što vam je  Jahve dao subotu, daje vam hrane šestoga dana za dva dana. Neka  svatko stoji gdje jest; neka nitko u sedmi dan ne izlazi iz svoga  stana." 
\par 30 Tako se sedmoga dana narod odmarao. 
\par 31 Dom je Izraelov tu hranu prozvao mÓanom. Bijaše kao zrno  korijandra; bijela, a imala je ukus medenog kolačića. 
\par 32 Onda rekne Mojsije: "Ovo je zapovijed koju je izdao Jahve:  Napunite tim jedan gomer i čuvajte ga za svoje potomke da vide  hranu kojom sam vas hranio u pustinji kad sam vas izbavio iz  zemlje egipatske." 
\par 33 I naredi Mojsije Aronu: "Uzmi jednu posudu;  stavi u nju cio gomer mane, a onda je položi pred Jahvu da se  sačuva za vaše potomke." 
\par 34 Kako je Jahve naredio Mojsiju, Aron  je stavi pred Svjedočanstvo na čuvanje. 
\par 35 Izraelci su se hranili manom četrdeset godina, sve dok  nisu došli u naseljenu zemlju: jeli su manu do dolaska na granicu  zemlje kanaanske. 
\par 36 Gomer je deseti dio efe. 


\chapter{17}

\par 1 Sva izraelska zajednica po Jahvinoj zapovijedi krene dalje  iz pustinje Sina. Utabore se kod Refidima. Tu nije bilo vode  da narod pije. 
\par 2 Zato narod zapodjene prepirku s Mojsijem. Vikali  su: "Daj nam vode da pijemo!" A Mojsije im odgovori: "Zašto se  sa mnom prepirete? Zašto kušate Jahvu?" 
\par 3 Ali je narod žeđao  za vodom, pa je mrmljao na Mojsija i govorio: "Zašto si nas iz  Egipta izveo? Zar da nas žeđom pomoriš, nas, našu djecu i našu  stoku?" 
\par 4 "Što ću s ovim narodom!" - zazivao je Mojsije Jahvu. "Još  malo pa će me kamenovati." 
\par 5 "Istupi pred narod!" - rekne Jahve  Mojsiju. "Uzmi sa sobom nekoliko izraelskih starješina; uzmi  u ruku štap kojim si udario Rijeku i pođi. 
\par 6 A ja ću stajati  pred tobom ondje, na pećini na Horebu. Udari po pećini: iz nje  će poteći voda, pa neka se narod napije." Mojsije učini tako  naočigled izraelskih starješina. 
\par 7 Mjesto prozovu Masa i Meriba, zbog toga što su se Izraelci prepirali i kušali Jahvu govoreći:  "Je li Jahve među nama ili nije?" 
\par 8 Uto dođu Amalečani i zarate s Izraelcima kod Refidima. 
\par 9 A Mojsije reče Jošui: "Odaberi momčad pa pođi i zapodjeni  borbu s Amalečanima. Ja ću sutra stati na vrh brda, sa štapom  Božjim u ruci." 
\par 10 Jošua učini kako mu je Mojsije rekao te zađe  u borbu s Amalečanima, a Mojsije, Aron i Hur uzađoše na vrh brda. 
\par 11 I dok bi Mojsije držao ruke uzdignute, Izraelci bi nadjačavali;  a kad bi ruke spustio, nadjačavali bi Amalečani. 
\par 12 Ali Mojsiju  ruke napokon klonu. Zato uzeše kamen, staviše ga poda nj i on  sjede, dok mu Aron i Hur, jedan s jedne, a drugi s druge strane, držahu ruke, tako da mu izdržaše do sunčanog zalaska. 
\par 13 I  Jošua oštricom mača svlada Amaleka i njegov narod. 
\par 14 Onda Jahve  reče Mojsiju: "Zapiši ovo u knjigu na sjećanje i utuvi u uši  Jošui da ću ja spomen na Amalečane sasvim izbrisati pod nebom!" 
\par 15 Podiže zatim Mojsije žrtvenik i nazva ga : Jahve mi je stijeg! 
\par 16 "Jer", reče, "Jahvin stijeg u ruku! Jahvin je boj protiv  Amalečana od naraštaja do naraštaja!" 


\chapter{18}

\par 1 A Jitro, midjanski svećenik, tast Mojsijev, ču sve što učini  Bog Mojsiju i svemu izraelskom narodu i kako Jahve izbavi Izraelce  iz Egipta. 
\par 2 Tada tast Mojsijev Jitro povede Siporu, Mojsijevu  ženu - koju Mojsije bijaše otpustio - 
\par 3 i oba njezina sina.  Jednomu je bilo ime Geršon, a to će reći: "Bijah došljak u tuđoj  zemlji." 
\par 4 Drugi se zvao Eliezer, to jest: "Bog oca moga bio  mi je u pomoći i spasio me od faraonova mača." 
\par 5 Tako Mojsijev  tast Jitro povede k Mojsiju u pustinju, gdje se Mojsije bio utaborio  na Božjem brdu, njegove sinove i njegovu ženu. 
\par 6 Poruči on Mojsiju:  "Ja, tvoj tast Jitro, dolazim k tebi s tvojom ženom i s oba njezina  sina." 
\par 7 Izađe Mojsije u susret svome tastu; duboko mu se nakloni  i zagrli ga. Pošto su se upitali za zdravlje, uđu pod šator. 
\par 8 Mojsije je onda pripovijedao svome tastu o svemu što je  Jahve učinio faraonu i Egipćanima zbog Izraelaca; o svim nezgodama  što su ih snašle na putu, ali ih je Jahve od njih izbavio. 
\par 9 Jitro  se radovao svemu dobru koje je Jahve učinio Izraelcima i što  ih je oslobodio od egipatskih šaka. 
\par 10 "Neka je hvaljen Jahve  koji vas je izbavio od egipatskih šaka i od šaka faraonovih", reče Jitro. 
\par 11 "Sada znam da je Jahve veći od svih bogova jer  je izbavio narod ispod egipatske vlasti kad su s njim okrutno  postupali." 
\par 12 Zatim Jitro, Mojsijev tast, prinese Bogu žrtvu paljenicu  i prinos. Uto dođe Aron i sve izraelske starješine da s Mojsijevim  tastom blaguju gozbu pred Bogom. 
\par 13 Sutradan Mojsije sjede da kroji pravdu narodu. Narod je oko  njega stajao od jutra do mraka. 
\par 14 Vidjevši Mojsijev tast sav  trud što ga on za narod čini, rekne mu: "Što to imaš toliko s  narodom? I zašto ti sam sjediš, a sav narod stoji oko tebe od  jutra do mraka?" 
\par 15 "Narod dolazi k meni", odgovori Mojsije, "da se s Bogom posavjetuje. 
\par 16 Kad zađu u prepirku, dođu k  meni. Ja onda rasudim između jednoga i drugoga; izložim im Božje  zakone i odredbe." 
\par 17 "Nije dobro kako radiš", odgovori Mojsiju  tast. 
\par 18 "I ti i taj narod s tobom potpuno ćete se iscrpsti.  Taj je posao za te pretežak; sam ga ne možeš obavljati. 
\par 19 Poslušaj  me. Svjetovat ću te, i Bog će biti s tobom! Ti zastupaj narod  pred Bogom; podastiri Bogu njihove razmirice. 
\par 20 Poučavaj ih  o zakonima i odredbama; svraćaj ih na put kojim moraju ići, upućuj  ih na djela koja moraju vršiti. 
\par 21 Onda proberi između svega  puka ljude sposobne, bogobojazne i pouzdane, koji mrze mito,  te ih postavi za glavare puku: tisućnike, stotnike, pedesetnike  i desetnike. 
\par 22 Neka sude narodu u svako doba. Sve veće slučajeve  neka preda te iznose, a u manjima neka sami rasuđuju. Olakšaj  sebi breme: neka ga oni s tobom nose. 
\par 23 Ako tako uradiš - i  Bog ti to odobri - moći ćeš izdržati, a sav ovaj narod odlazit  će kući u miru." 
\par 24 Mojsije posluša savjet svoga tasta i učini sve kako ga  svjetova. 
\par 25 Probere Mojsije sposobnih ljudi od svih Izraelaca  pa ih postavi za glavare narodu: tisućnike, stotnike, pedesetnike  i desetnike. 
\par 26 Oni su sudili narodu u svako doba. Teže slučajeve  iznosili bi Mojsiju, a sve manje rješavali sami. 
\par 27 Zatim Mojsije otpusti svoga tasta i on ode u svoju zemlju. 


\chapter{19}

\par 1 Tri mjeseca nakon izlaska iz zemlje egipatske, istoga dana, stignu Izraelci u Sinajsku pustinju. 
\par 2 Idući od Refidima, dođu  u Sinajsku pustinju i utabore se u pustinji. Postave Izraelci  tabor tu pred brdom, 
\par 3 a Mojsije se popne k Bogu. Jahve ga zovne  s brda pa mu rekne: "Ovako kaži domu Jakovljevu, proglasi djeci Izraelovoj: 
\par 4 'Vi  ste vidjeli što sam učinio Egipćanima; kako sam vas nosio na  orlovskim krilima i k sebi vas doveo. 
\par 5 Stoga, budete li mi  se vjerno pokoravali i držali moj Savez, vi ćete mi biti predraga  svojina mimo sve narode - tÓa moj je sav svijet! - 
\par 6 vi ćete  mi biti kraljevstvo svećenika, narod svet.' Tim riječima oslovi  Izraelce." 
\par 7 Mojsije se vrati i sazva narodne starješine te im izloži  sve što mu je Jahve naredio. 
\par 8 A sav narod uzvrati jednoglasno:  "Vršit ćemo sve što je Jahve naredio." Onda Mojsije prenese odgovor  naroda Jahvi. 
\par 9 Nato Jahve reče Mojsiju: "Ja ću, evo, doći k tebi u gustom  oblaku da narod čuje kad budem s tobom govorio i da ti zauvijek  vjeruje."  Tako je Mojsije prenio Jahvi odgovor naroda. 
\par 10 "Pođi k narodu", reče Jahve Mojsiju, "i posvećuj ga danas  i sutra. Neka opere svoju odjeću; 
\par 11 neka bude gotov prekosutra, jer će prekosutra sići Jahve na brdo Sinaj naočigled svega puka. 
\par 12 Postavi naokolo granicu za narod i izdaj naredbu: 'Pripazite  da se na brdo ne penjete; da se ni podnožja ne dotičete! Tko  se god brda dotakne, smrt će ga snaći. 
\par 13 Nikakva ruka neka  ga se ne dotakne, nego neka bude kamenjem zasut ili strijelom  ustrijeljen: bio čovjek ili živinče, neka na životu ne ostane.'  Na otegnuti zvuk trube neka se na brdo penju." 
\par 14 Mojsije siđe s brda k narodu i poče posvećivati narod.  Oni operu svoju odjeću. 
\par 15 "Budite gotovi za prekosutra!" -  rekne Mojsije narodu. "Ne primičite se ženi!" 
\par 16 A prekosutra, u osvit dana, prolomi se grmljavina, munje  zasijevaše, a gust se oblak nadvi nad brdo. Gromko zaječa truba, zadrhta sav puk koji bijaše u taboru. 
\par 17 Mojsije povede puk  iz tabora u susret Bogu. Stadoše na podnožju brda. 
\par 18 Brdo Sinaj  zavilo se u dim jer je Jahve u obliku ognja sišao na nj. Dizao  se dim kao dim iz peći. Sve se brdo silno treslo. 
\par 19 Zvuk trube  bivao sve jači. Mojsije je govorio, a Bog mu grmljavinom odgovarao. 
\par 20 Jahve siđe na Sinajsko brdo, na vrhunac, i pozva Jahve Mojsija  na vrhunac brda. Mojsije se uspe. 
\par 21 Sad Jahve reče Mojsiju: "Siđi i opomeni narod da ne bi  provalio prema Jahvi da ga vidi. Mnogo bi ih poginulo. 
\par 22 I  sami svećenici, koji dolaze blizu Jahvi, moraju se očistiti,  da ih Jahve ne uništi." 
\par 23 "Narod se ne može popeti na brdo  Sinaj", odgovori Mojsije Jahvi, "jer si nas sam ti opomenuo:  'Postavi granice naokolo brda i proglasi ga svetim.'" 
\par 24 "Siđi  pa se opet popni zajedno s Aronom", odgovori mu Jahve. "Ali neka  svećenici i narod ne navaljuju da se popnu prema Jahvi da ne  izginu." 
\par 25 Mojsije siđe k narodu i sve mu kaza. 


\chapter{20}

\par 1 Onda Bog izgovori sve ove riječi: 
\par 2 "Ja sam Jahve, Bog tvoj, koji sam te izveo iz zemlje egipatske, iz kuće ropstva. 
\par 3 Nemoj imati drugih bogova uz mene. 
\par 4 Ne pravi sebi lika ni obličja bilo čega što je gore na  nebu, ili dolje na zemlji, ili u vodama pod zemljom. 
\par 5 Ne klanjaj  im se niti im služi. Jer ja, Jahve, Bog tvoj, Bog sam ljubomoran.  Kažnjavam grijeh otaca - onih koji me mrze - na djeci do trećeg  i četvrtog koljena, 
\par 6 a iskazujem milosrđe tisućama koji me  ljube i vrše moje zapovijedi. 
\par 7 Ne uzimaj uzalud imena Jahve, Boga svoga, jer Jahve ne  oprašta onome koji uzalud izgovara ime njegovo. 
\par 8 Sjeti se da svetkuješ dan subotni. 
\par 9 Šest dana radi i  obavljaj sav svoj posao. 
\par 10 A sedmoga je dana subota, počinak  posvećen Jahvi, Bogu tvojemu. Tada nikakva posla nemoj raditi:  ni ti, ni sin tvoj, ni kći tvoja, ni sluga tvoj, ni sluškinja  tvoja, ni živina tvoja, niti došljak koji se nađe unutar tvojih  vrata. 
\par 11 Ta i Jahve je šest dana stvarao nebo, zemlju i more  i sve što je u njima, a sedmoga je dana počinuo. Stoga je Jahve  blagoslovio i posvetio dan subotni. 
\par 12 Poštuj oca svoga i majku svoju da imadneš dug život na  zemlji koju ti da Jahve, Bog tvoj. 
\par 13 Ne ubij! 
\par 14 Ne učini preljuba! 
\par 15 Ne ukradi! 
\par 16 Ne svjedoči lažno na bližnjega svoga! 
\par 17 Ne poželi kuće bližnjega svoga! Ne poželi žene bližnjega  svoga; ni sluge njegova, ni sluškinje njegove, ni vola njegova, ni magarca njegova, niti išta što je bližnjega tvoga!" 
\par 18 Sav je puk bio svjedok grmljavine i sijevanja, svi čuše  zvuk trube i vidješe kako se brdo dimi: gledali su i tresli se  i stajali podalje. 
\par 19 Onda rekoše Mojsiju: "Ti nam govori, a  mi ćemo slušati. Neka nam Bog ne govori, da ne pomremo!" 
\par 20 "Ne  bojte se", reče Mojsije narodu. "Bog je došao da vas samo iskuša;  da strah pred njim ostane s vama te da ne griješite." 
\par 21 Narod  ostane podalje, a Mojsije pristupi gustom oblaku gdje se Bog  nalazio. 
\par 22 "Ovako reci Izraelcima", progovori Jahve Mojsiju. "Sami  ste vidjeli da sam s vama govorio s neba. 
\par 23 Ne pravite uza  me kumira od srebra niti sebi pravite kumira od zlata. 
\par 24 Načini  mi žrtvenik od zemlje i na njemu mi prinosi svoje žrtve paljenice  i žrtve pričesnice, svoju sitnu i svoju krupnu stoku. Na svakome  mjestu koje odredim da se moje ime spominje ja ću doći k tebi  da te blagoslovim. 
\par 25 Ako mi budeš gradio kameni žrtvenik, nemoj  ga graditi od klesanoga kamena, jer čim na nj spustiš svoje dlijeto, oskvrnit ćeš ga. 
\par 26 Ne uzlazi na moj žrtvenik po stepenicama, da se ne pokaže na njemu golotinja tvoja." 


\chapter{21}

\par 1 "Ovo su propisi koje treba da im izložiš: 
\par 2 Kad za roba kupiš  jednoga Hebreja, neka služi šest godina. Sedme godine neka ode, bez otkupnine, slobodan. 
\par 3 Ako dođe sam, neka sam i ode; ako  li je oženjen, neka s njim ide i njegova žena. 
\par 4 Ako mu gospodar  nabavi ženu, pa mu ona rodi bilo sinova bilo kćeri, i žena i  njezina djeca neka pripadnu njezinu gospodaru, a on neka ide  sam. 
\par 5 Ali ako rob otvoreno izjavi: 'Volim svoga gospodara,  svoju ženu i svoju djecu, neću da budem slobodan', 
\par 6 neka ga  onda njegov gospodar dovede k Bogu. Kad ga dovede k vratima ili  dovratku, neka mu gospodar šilom probuši uho i neka mu trajno  ostane u službi. 
\par 7 Kad čovjek proda svoju kćer za ropkinju,  neka se ona ne oslobađa kao i muški robovi. 
\par 8 Ako se ne svidi  svome gospodaru, koji ju je sebi bio odredio, neka joj dopusti  da se otkupi. Nema prava prodati je strancima kad joj nije bio  vjeran. 
\par 9 A ako je odredi svome sinu, neka s njome postupa kao  i sa kćeri. 
\par 10 Ako se oženi drugom, ne smije prvoj uskraćivati  hrane, odjeće ili njezinih bračnih prava. 
\par 11 Ne bude li joj  činio ovo troje, neka je slobodna da ode bez otkupnine." 
\par 12 "Tko god udari čovjeka pa ga usmrti, neka se smrću kazni. 
\par 13 Ali ako to ne učini hotimično, nego Bog pripusti da padne  u njegovu šaku, odredit ću ti mjesto kamo može pobjeći. 
\par 14 Tko  hotimično navali na svoga bližnjega te ga podmuklo ubije, odvuci  ga i s moga žrtvenika da se pogubi. 
\par 15 Tko udari svoga oca ili  svoju majku, neka se kazni smrću. 
\par 16 Tko otme čovjeka - bilo  da ga proda, bilo da ga u svojoj vlasti zadrži - neka se kazni  smrću. 
\par 17 Tko prokune svoga oca ili svoju majku, neka se kazni  smrću." 
\par 18 "Ako se ljudi posvade, pa jedan od njih udari drugoga  kamenom ili šakom, ali ovaj ne pogine nego padne u postelju, 
\par 19 ali poslije ustane i mogne izlaziti, makar i sa štapom, onda  onome koji ga je udario neka je oprošteno, samo neka mu plati  njegov gubitak vremena i pribavi mu posvemašnje izlječenje. 
\par 20 Ako tko udari batinom svoga roba ili svoju ropkinju te  umru pod njegovom šakom, mora snositi osvetu. 
\par 21 Ali ako rob  preživi dan-dva, neka se osveta ne provodi, jer je rob njegovo  vlasništvo. 
\par 22 Ako se ljudi pobiju i udare trudnu ženu te ona pobaci, ali druge štete ne bude, onda onaj koji ju je udario neka plati  odštetu koju zatraži njezin muž. On neka plati kako suci odrede. 
\par 23 Bude li drugog zla, neka je kazna: život za život, 
\par 24 oko  za oko, zub za zub, ruka za ruku, noga za nogu, 
\par 25 opeklina  za opeklinu, rana za ranu, modrica za modricu. 
\par 26 Udari li tko svoga roba ili svoju ropkinju u oko i upropasti  ga, neka ga oslobodi zbog oka. 
\par 27 Ako izbije zub svome robu  - ili svojoj ropkinji - neka ga oslobodi zbog zuba." 
\par 28 "Kad goveče ubode čovjeka ili ženu pa ih usmrti, neka  se kamenjem kamenuje. Njegovo se meso tada ne smije pojesti,  a vlasniku njegovu neka je oprošteno. 
\par 29 Ali ako je to goveče  i prije bolo, a njegov vlasnik, iako opominjan, nije ga čuvao, pa ono usmrti čovjeka ili ženu, neka se to goveče kamenuje;  a i njegov se vlasnik ima pogubiti. 
\par 30 Ako se vlasniku označi  otkupna cijena da svoj život iskupi, neka plati koliko mu se  odredi. 
\par 31 Ubode li goveče dječaka ili djevojčicu, neka se prema  njemu postupi isto prema ovome pravilu. 
\par 32 Ako ubode roba ili  ropkinju, neka vlasnik isplati njihovu gospodaru trideset srebrnih  šekela, a goveče neka se kamenuje. 
\par 33 Kad tko ostavi bunar otvoren, ili tko iskopa bunar a  ne pokrije ga, pa u nj upadne goveče ili magare, vlasnik bunara  ima dati naknadu: 
\par 34 neka isplati vlasniku u novcu, a uginula  životinja neka njemu pripadne. 
\par 35 Kad nečije goveče ubode goveče  drugome te ono ugine, onda neka prodaju živo goveče, a dobiveni  novac neka podijele; i uginulo goveče neka među sebe podijele. 
\par 36 Ali ako se zna da je to goveče i prije bolo, a njegov ga  gospodar nije čuvao, onda mora nadoknaditi goveče za goveče,  dok će uginulo živinče biti njegovo." 


\chapter{22}

\par 1 (21:37) "Tko ukrade goveče ili marvinče od sitne stoke, pa bilo  da ga zakolje, bilo da ga proda, onda za jedno goveče neka se  vrati petero goveda, a za malo marvinče četvero marvinčadi. 
\par 2 (22:1) Ako se lopov zateče gdje probija zid, pa mu se zada smrtan  udarac, njegovu krv ne treba osvećivati. 
\par 3 (22:2) No ako je već izišlo  sunce, njegovu krv treba osvetiti. Lopov mora štetu nadoknaditi.  Ako nema ništa, njega za njegovu krađu treba prodati. 
\par 4 (22:3) Nađe  li se ukradeno živinče živo u njegovu vlasništvu - goveče, magare  ili koja glava sitne stoke - treba da ga plati dvostruko." 
\par 5 (22:4) "Tko opustoši njivu ili vinograd pustivši svoju stoku  da obrsti tuđe, neka nadoknadi onim što najbolje nađe na svojoj  njivi i u svome vinogradu. 
\par 6 (22:5) Tko zapali vatru pa ona zahvati drač te izgori žito u  snopu, u klasu ili na njivi, onaj tko je vatru zapalio mora štetu  nadoknaditi. 
\par 7 (22:6) Kad tko položi kod znanca novac ili stvari na čuvanje, pa budu pokradene iz njegove kuće, ako se lopov pronađe, mora  dvostruko platiti. 
\par 8 (22:7) Ako se lopov ne pronađe, vlasnik kuće neka  se primakne k Bogu, da se dokaže kako on nije spustio svoje ruke  na dobra svoga bližnjega. 
\par 9 (22:8) Za svaki prekršaj pronevjere - radilo se o govečetu, magaretu, sitnoj stoci, odjeći ili bilo kojoj izgubljenoj stvari za koju  se ustvrdi: to je ono! - treba spor iznijeti pred Boga. Onaj  koga Bog proglasi krivim neka plati dvostruko drugome. 
\par 10 (22:9) Kad tko povjeri svome susjedu magare, goveče, glavu sitne  stoke ili bilo kakvo živinče, pa ono ugine, osakati se ili ga  tko odvede a da ne bude svjedoka, 
\par 11 (22:10) zakletva pred Jahvom neka  odluči među obojicom je li čuvar posegao za dobrom svoga bližnjega  ili nije. Neka je vlasniku to dovoljno, a čuvar nije dužan da  nadoknađuje. 
\par 12 (22:11) Nađe li se da je on ukrao, mora štetu nadoknaditi. 
\par 13 (22:12) Ako ga zvijer razdere, neka ga donese za dokaz, tako da za  razderano ne daje odštete. 
\par 14 (22:13) Kad tko posudi živinu na izor  od svoga susjeda, pa se ona osakati ili ugine dok joj vlasnik  nije bio s njom, neka plati odštetu. 
\par 15 (22:14) Je li vlasnik bio s  njom, odštete mu ne daje; ali ako je bila unajmljena na izor, neka dođe po svoju nadnicu." 
\par 16 (22:15) "Ako tko zavede djevojku koja nije zaručena i s njom  legne, neka za nju dadne ženidbenu procjenu i uzme je za ženu. 
\par 17 (22:16) Ako njezin otac odbije da mu je dadne, zavodnik mora odmjeriti  srebra u vrijednosti ženidbene procjene za djevojku. 
\par 18 (22:17) Ne dopuštaj da vračarica živi! 
\par 19 (22:18) Tko bi god sa živinom legao, treba ga kazniti smrću. 
\par 20 (22:19) Tko bi prinosio žrtve kojemu kumiru - osim Jahvi jedinom  - neka bude izručen prokletstvu, potpuno uništen. 
\par 21 (22:20) Ne tlači pridošlicu niti mu nanosi nepravde, jer ste  i sami bili pridošlice u zemlji egipatskoj. 
\par 22 (22:21) Ne cvilite udovice  i siročeta! 
\par 23 (22:22) Ako ih ucviliš i oni zavape k meni, sigurno ću  njihove vapaje uslišati. 
\par 24 (22:23) Moj će se gnjev raspaliti i mačem  ću vas pogubiti. Tako će vam žene ostati udovice a djeca siročad. 
\par 25 (22:24) Ako uzajmiš novca kome od moga naroda, siromahu koji  je kod tebe, ne postupaj prema njemu kao lihvar! Ne nameći mu  kamata! 
\par 26 (22:25) Uzmeš li svome susjedu ogrtač u zalog, moraš mu ga vratiti  prije zalaza sunca. 
\par 27 (22:26) TÓa to mu je jedini pokrivač kojim omata  svoje tijelo i u kojem može leći. Ako k meni zavapi, uslišat  ću ga jer sam ja milostiv! 
\par 28 (22:27) Ne huli Boga i ne psuj glavara svoga naroda. 
\par 29 (22:28) Ne oklijevaj s prinosima od svoga obilja s gumna i od  svoga mladog vina! Meni daj prvorođenca od svojih sinova. 
\par 30 (22:29) Isto  učini sa svojim govedima i sitnom stokom: sedam dana neka ostane  sa svojom majkom, a osmoga dana da si ga meni dao! 
\par 31 (22:30) Budite narod meni posvećen! Zato nemojte jesti mesa od  životinje koju je rastrgala zvjerad nego je bacite paščadi!" 


\chapter{23}

\par 1 "Nemojte davati lažne izjave! Ne pomaži zlikovcu svjedočeći  krivo! 
\par 2 Ne povodi se za mnoštvom da činiš zlo; niti svjedoči  u parnici stajući na stranu većine protiv pravde. 
\par 3 Ne smiješ  biti pristran prema siromahu u njegovoj parnici. 
\par 4 Kad nabasaš na zalutalo goveče ili magare svoga neprijatelja, moraš mu ga natrag dovesti. 
\par 5 Ako opaziš magarca onoga koji  te mrzi kako je pao pod svojim tovarom, nemoj ga ostaviti: zajedno  s njegovim gospodarom moraš mu pomoći da se digne. 
\par 6 Ne krnji prava svome siromahu u njegovoj parnici. 
\par 7 Stoj  daleko od lažne optužbe; ne ubijaj nedužna i pravedna, jer ja  zlikovcu ne praštam. 
\par 8 Ne primaj mita, jer mito zasljepljuje  i one koji najjasnije gledaju i upropašćuje pravo pravednika. 
\par 9 Ne ugnjetavaj pridošlicu! TÓa znate kako je pridošlici;  i sami ste bili pridošlice u zemlji egipatskoj." 
\par 10 "Šest godina zasijavaj svoju zemlju i njezine plodove  pobiri, 
\par 11 a sedme je godine pusti da počiva neobrađena. Neka  se s nje hrani sirotinja tvoga naroda, a što njoj ostane, neka  pojede poljska živina. Radi tako i sa svojim vinogradom i svojim  maslinikom. 
\par 12 Šest dana obavljaj svoj posao, ali sedmoga dana od posala  odustani, da ti otpočine vo i magarac i da odahne sin tvoje sluškinje  i pridošlica. 
\par 13 Pripazite na sve što sam vam rekao. Ne spominjite imena  drugih bogova. Neka se to i ne čuje iz tvojih usta." 
\par 14 "Triput na godinu održavaj u moju čast svetkovinu. 
\par 15 Slavi  Blagdan beskvasnoga kruha. U određeno vrijeme u mjesecu Abibu  - jer si u njemu iz Egipta izišao - sedam dana jedi beskvasan  kruh, kako sam ti naredio. Neka nitko ne stupa preda me praznih  ruku! 
\par 16 Onda slavi Blagdan žetve - prvina što ih donose polja  koja zasijavaš. Zatim Blagdan berbe na koncu godine, kad s polja  pokupiš plodove svoga truda. 
\par 17 Triput na godinu neka svi tvoji  muški stupe pred Gospodara Jahvu. 
\par 18 Krv žrtve koju u moju čast žrtvuješ nemoj prinositi s  ukvasanim kruhom; salo od žrtve prinesene na moju svetkovinu  ne ostavljaj za sutradan. 
\par 19 Donosi u kuću Jahve, svoga Boga, najbolje prvine sa svoje  zemlje.  Ne kuhaj kozleta u mlijeku njegove majke." 
\par 20 "Šaljem, evo, svog anđela pred tobom da te čuva na putu  i dovede te u mjesto koje sam priredio. 
\par 21 Poštuj ga i slušaj!  Ne buni se protiv njega, jer vam neće opraštati prekršaje: tÓa  moje je ime u njemu. 
\par 22 Ako mu se budeš vjerno pokoravao i budeš  vršio sve što sam naredio, ja ću biti neprijatelj tvojim neprijateljima  i protivnik tvojim protivnicima. 
\par 23 Anđeo će moj ići pred tobom  i dovesti te do Amorejaca, Hetita, Perižana, Kanaanaca, Hivijaca  i Jebusejaca da ih uništim. 
\par 24 Nemoj se klanjati njihovim kumirima  niti im iskazuj štovanje; ne postupaj kako oni rade nego njihove  kumire poruši i stupove im porazbijaj. 
\par 25 Iskazujte štovanje  Jahvi, Bogu svome, pa ću blagoslivati tvoj kruh i tvoju vodu  i uklanjati od tebe bolest. 
\par 26 U tvojoj zemlji neće biti pometkinje;  ja ću učiniti punim broj tvojih dana. 
\par 27 Pred tobom ću odaslati stravu svoju; u metež ću baciti  sav svijet među koji dospiješ i učinit ću da svi tvoji neprijatelji  bježe pred tobom. 
\par 28 Stršljene ću pred tobom odašiljati da ispred  tebe tjeraju u bijeg Hivijce, Kanaance i Hetite. 
\par 29 Neću ih  otjerati ispred tebe u jednoj godini, da zemlja ne opusti i divlje  se životinje ne razmnože na tvoju štetu. 
\par 30 Tjerat ću ih ispred  tebe malo-pomalo dok ti potomstvo ne odraste, tako da zemlju  zaposjedneš. 
\par 31 Postavit ću ti granicu: od Crvenoga do Filistejskoga  mora, od pustinje pa do Rijeke. Predat ću, naime, stanovništvo  zemlje u tvoje šake, a ti ga ispred sebe tjeraj. 
\par 32 Ne pravi  savez ni s njima ni s njihovim kumirima. 
\par 33 Neka ne ostanu u  tvojoj zemlji da te ne navode na grijeh protiv mene. Ako bi štovao  njihove kumire, to bi ti bila stupica." 


\chapter{24}

\par 1 Potom reče Mojsiju: "Uzađi k Jahvi - ti, Aron, Nadab i Abihu  i sedamdeset izraelskih starješina. Poklonite se izdaljega! 
\par 2 Neka  se sam Mojsije primakne k Jahvi! Oni neka se ne primiču, a puk  neka se s njim ne penje." 
\par 3 Dođe Mojsije i kaza narodu sve riječi Jahvine i sve odredbe.  A sav puk odgovori u jedan glas: "Sve riječi što ih Jahve reče, vršit ćemo." 
\par 4 Tada Mojsije popiše sve riječi Jahvine. A ujutro  podrani te podigne žrtvenik na podnožju brda i dvanaest stupova  za dvanaest plemena Izraelovih. 
\par 5 Zatim naloži mladim Izraelcima  da prinesu žrtve paljenice i da žrtvuju Jahvi junce kao žrtve  pričesnice. 
\par 6 Mojsije uhvati krv; polovinu krvi ulije u posude, a polovinu izlije po žrtveniku. 
\par 7 Prihvati zatim Knjigu Saveza  pa je narodu glasno pročita, a narod uzvrati: "Sve što je Jahve  rekao, izvršit ćemo i poslušat ćemo." 
\par 8 Mojsije potom uzme krvi  te poškropi narod govoreći: "Ovo je krv Saveza koji je Jahve  s vama uspostavio na temelju svih ovih riječi." 
\par 9 Onda se uspne Mojsije s Aronom, Nadabom i Abihuom i sa  sedamdeset starješina Izraelovih. 
\par 10 Oni vidješe Boga Izraelova:  podnožje njegovim nogama kao da je bilo od dragoga kamena safira, sjajem nalik na samo nebo. 
\par 11 Ni ruke svoje nije pružio na  izabranike Izraelaca: slobodno su Boga motrili i jeli i pili. 
\par 12 Onda Jahve reče Mojsiju: "Popni se k meni na brdo i pričekaj  ondje. Dat ću ti kamene ploče sa zakonom i zapovijedima koje  sam za njihovu pouku napisao." 
\par 13 Ustane Mojsije i njegov pomoćnik  Jošua te se Mojsije popne na brdo Božje. 
\par 14 A starješinama reče:  "Čekajte nas ovdje dok se ne vratimo. Eto je s vama Aron i Hur.  Tko imadne kakvu razmiricu, neka se obrati na njih." 
\par 15 Zatim Mojsije uzađe na brdo, a onda oblak prekri brdo. 
\par 16 Slava se Jahvina nastani na Sinajskom brdu i oblak ga obavijaše  šest dana. Sedmoga dana zovne Jahve Mojsija isred oblaka. 
\par 17 Slava  Jahvina na vrhuncu brda bijaše očima Izraelaca kao vatra koja  sažiže. Mojsije zađe u oblak i uspne se na brdo. 
\par 18 Četrdeset  dana i četrdeset noći boravio je Mojsije na brdu. 


\chapter{25}

\par 1 Jahve reče Mojsiju: 
\par 2 "Reci Izraelcima da me darivaju, a  vi primajte darove u moju čast od svakoga koji daje od srca. 
\par 3 A primajte ove darove: zlato, srebro i tuč; 
\par 4 ljubičasto, crveno i tamnocrveno predivo i prepredeni lan; 
\par 5 učinjene ovnujske  kože, pa fine kože; bagremovo drvo; 
\par 6 ulje za svjetlo; mirodije  za ulje pomazanja i miomirisno kađenje; 
\par 7 oniks i drugo drago  kamenje koje će se umetnuti u oplećak i naprsnik. 
\par 8 Neka mi  sagrade Svetište da mogu boraviti među njima. 
\par 9 Pri gradnji  Prebivališta i svega u njemu postupi točno prema uzorku koji  ti pokažem." 
\par 10 "Od bagremova drva neka naprave Kovčeg: dva i po lakta  dug, lakat i po širok i lakat i po visok. 
\par 11 Okuj ga čistim  zlatom, okuj ga izvana i iznutra; a oko njega stavi naokolo završni  pojas od zlata. 
\par 12 Salij za nj četiri zlatna koluta; prikuj  ih za četiri njegove noge; dva koluta s jedne strane, a dva s  druge. 
\par 13 Od bagremova drva napravi i motke te ih u zlato okuj. 
\par 14 Onda provuci motke kroz kolutove sa strana Kovčega da se  na njima Kovčeg nosi. 
\par 15 Neka motke ostanu u kolutima Kovčega;  neka se iz njih ne izvlače. 
\par 16 Svjedočanstvo koje ću ti predati  - u Kovčeg položi." 
\par 17 "Pomirilište napravi također od čistoga zlata. Neka bude  dugo dva i pol lakta, a široko lakat i pol. 
\par 18 Skuj i dva kerubina  od zlata za oba kraja Pomirilišta. 
\par 19 Napravi jednoga kerubina  za jedan kraj, a drugoga kerubina za drugi kraj. Pričvrsti ih  na oba kraja Pomirilišta da s njim sačinjavaju jedan komad. 
\par 20 Kerubini  neka dignu svoja krila uvis tako da svojim krilima zaklanjaju  Pomirilište. Neka budu licem okrenuti jedan prema drugome, ali  tako da lica kerubina gledaju u Pomirilište. 
\par 21 Stavi na Kovčeg  Pomirilište, a u Kovčeg položi ploče Svjedočanstva što ću ti  ih dati. 
\par 22 Tu ću se ja s tobom sastajati i ozgo ću ti, iznad  Pomirilišta - između ona dva kerubina što su na Kovčegu ploča  Svjedočanstva - saopćavati sve zapovijedi namijenjene Izraelcima." 
\par 23 "Napravi od bagremova drva stol dva lakta dug, lakat  širok, a lakat i pol visok. 
\par 24 U čisto ga zlato obloži i načini  mu naokolo završni pojas od zlata. 
\par 25 Naokolo mu načini obrub, podlanicu širok, a onda po obrubu stavi završni pojas od zlata. 
\par 26 Nadalje, uspravi mu četiri koluta od zlata pa mu ih pričvrsti  na njegova četiri nožna ugla. 
\par 27 Neka su kolutovi tik pod obrubom  da služe kao kvake motkama za nošenje stola. 
\par 28 Motke napravi  od bagremova drva i u zlato ih okuj. O njima će se stol nositi. 
\par 29 Za nj onda napravi: zdjele, varjače, vrčeve i pehare za izlijevanje  prinosa. Načini ih od čistoga zlata. 
\par 30 Na stol svagda stavljaj  pred moje lice prineseni kruh." 
\par 31 "Načini svijećnjak od čistoga zlata. Svijećnjak neka  bude skovan. Njegovo podnožje, njegov stalak, njegove čaše, čaške  i latice - sve neka bude od jednoga komada. 
\par 32 Šest krakova  neka mu izbija sa strana: tri kraka s jedne strane stalka, a  tri kraka s druge strane stalka. 
\par 33 Na jednome kraku neka budu  tri čaše u obliku bademova cvijeta, svaka s čaškom i laticama.  Tako za svih šest krakova što budu izbijali iz stalka svijećnjaka. 
\par 34 Na samome svijećnjaku neka budu četiri čaše u obliku bademova  cvijeta, svaka s čaškom i laticama. 
\par 35 Čaška ispod dva kraka, sačinjavajući jedan komad s njime; onda čaška ispod druga dva  kraka, od jednoga komada s njime, pa čaška ispod dva posljednja  kraka, od jednoga komada s njime. Tako za svih šest krakova što  iz stalka budu izbijali. 
\par 36 Njihove čaške i njihovi krakovi  sačinjavat će jedan komad s njim - sve skovano u jednome komadu  od čistoga zlata. 
\par 37 Napravi i sedam svjetiljaka za njih. Svjetiljke  neka tako budu postavljene da osvjetljuju prostor sprijeda. 
\par 38 Usekači  i pepeljare za njih neka su od čistoga zlata. 
\par 39 Upotrijebi  talenat čistoga zlata za svijećnjak i sav njegov pribor. 
\par 40 Pazi!  Načini ih prema uzorku koji ti je na brdu pokazan." 


\chapter{26}

\par 1 "Prebivalište načini od deset zavjesa: od ljubičastog, crvenog  i tamnocrvenog prediva i prepredenog lana. Na njima neka budu  vezeni likovi kerubina - djelo umjetnika. 
\par 2 Dužina svake zavjese  neka je dvadeset i osam lakata, neka joj je širina četiri lakta.  Sve zavjese neka su iste mjere. 
\par 3 Pet zavjesa neka su sastavljene  jedna s drugom, a drugih pet zavjesa opet jedna s drugom. 
\par 4 Napravi  petlje od ljubičaste vune pri rubu krajnje zavjese u sastavljenom  komadu. 
\par 5 Napravi pedeset petlji na rubu jednoga sastavljenog  komada od zavjesa, a pedeset pri rubu drugoga. Neka su petlje  načinjene jedna spram druge. 
\par 6 Onda napravi pedeset kopča od  zlata. Zavjese zatim kopčama sastavi jednu s drugom. Tako će  Prebivalište biti jedna cjelina. 
\par 7 Načini zatim zavjese od kostrijeti za Šator povrh Prebivališta.  Načini ih jedanaest. 
\par 8 Neka duljina svake zavjese bude trideset  lakata, a širina svake zavjese četiri lakta. Tih jedanaest zavjesa  neka bude iste mjere. 
\par 9 Sastavi pet zavjesa napose, a onda opet  drugih šest zavjesa napose. Šestu zavjesu podvostruči na pročelju  Šatora. 
\par 10 Ušij pedeset petlji na rubu jednoga sastavljenog  komada od zavjesa, a pedeset na rubu drugoga. 
\par 11 Izradi pedeset  kopča od tuča, zapni kopče za petlje da sastaviš Šator u cjelinu. 
\par 12 A kako će zavjese od Šatora pretjecati, neka se polovina  zavjesa što preostane spušta na zadnjem dijelu Prebivališta. 
\par 13 Od onoga što preteče na dužini šatorskih zavjesa neka po  jedan lakat visi na obje strane svetoga Šatora da ga zaklanja. 
\par 14 Napokon napravi Šatoru pokrov od učinjenih i u crveno  obojenih ovnujskih koža, a povrh njega pokrov od finih koža. 
\par 15 Trenice što će nauzgor stajati za Prebivalište napravi  od bagremova drva. 
\par 16 Svaka trenica neka bude deset lakata duga, a lakat i pol široka. 
\par 17 Svaka trenica neka ima dva klina da  je uspravno drže. Tako napravi na svakoj trenici za Prebivalište. 
\par 18 Trenice za Prebivalište postavi: dvadeset trenica s juga, prema podnevu; 
\par 19 onda pod dvadeset trenica napravi četrdeset  podnožja od srebra, dva podnožja pod prvu trenicu za njezina  dva klina, i tako redom, dva podnožja za dva klina svake slijedeće  trenice. 
\par 20 Za drugu stranu Prebivališta, sa sjevera: dvadeset  trenica 
\par 21 i četrdeset srebrnih podnožja, dva podnožja za dva  klina prve trenice, i tako redom, dva podnožja za svaku trenicu. 
\par 22 Na stražnjoj strani Prebivališta, sa zapada, postavi šest  trenica. 
\par 23 Napravi i dvije trenice za stražnje uglove Prebivališta. 
\par 24 Neka budu rastavljene pri dnu, ali na vrhu kod prvoga koluta  neka budu sastavljene. Neka tako obadvije prave dva ugla. 
\par 25 Neka  dakle bude osam trenica s njihovim srebrnim podnožjima: šesnaest  podnožja, dva podnožja pod prvom trenicom, a dva opet podnožja  pod svakom slijedećom trenicom. 
\par 26 Nadalje napravi priječnice od bagremova drva: pet njih  za trenice s jedne strane Prebivališta, 
\par 27 a pet priječnica  s druge strane Prebivališta; onda pet priječnica za trenice Prebivališta  straga prema zapadu. 
\par 28 Srednja priječnica neka ide sredinom  trenica s jednoga kraja na drugi. 
\par 29 Trenice obloži zlatom,  a i kolutove za njih, kroz koje će se priječnice provlačiti,  načini od zlata. Priječnice onda obloži zlatom. 
\par 30 Tako, dakle, podigni Prebivalište prema nacrtu koji ti je pokazan na brdu." 
\par 31 "Napravi zavjesu od ljubičastog, crvenog i tamnocrvenog  prediva i prepredenog lana. Neka su na njoj izvezeni kerubini. 
\par 32 Objesi je na četiri stupa od bagremova drva, zlatom obložena, s kopčama od zlata, a na četiri podnožja od srebra. 
\par 33 Objesi  zavjesu za kvake. Onda unesi Kovčeg svjedočanstva tu za zavjesu.  Neka ti tako zavjesa odjeljuje Svetište od Svetišta nad svetištima. 
\par 34 Stavi Pomirilište na Kovčeg svjedočanstva u Svetinji nad  svetinjama. 
\par 35 Postavi zatim stol van pred zavjesu, a svijećnjak  na južnu stranu Prebivališta, prema stolu. Stol stavi na sjevernu  stranu. 
\par 36 A na ulazu u Šator napravi zastorak od ljubičastog, crvenog i tamnocrvenog prediva i prepredenog lana - vezom izvezen. 
\par 37 Za zastorak isteši pet stupčića od bagrenova drva pa ih obloži  zlatom. Kopče za njih neka budu od zlata. Salij za njih pet podnožja  od tuča." 


\chapter{27}

\par 1 "Načini žrtvenik od bagremova drva, pet lakata dug, pet lakata  širok - prava četvorina - i tri lakta visok. 
\par 2 Na njegova četiri  ugla načini rogove. Neka mu rogovi budu u jednome komadu s njim.  I tučem ga okuj. 
\par 3 Dalje, načini za žrtvenik posude za zgrtanje  otpadaka: strugače, kotliće, viljuške i kadionike. Sve potrepštine  za žrtvenik načini od tuča. 
\par 4 Onda načini za nj rešetku od tuča, u obliku mrežice, 
\par 5 a na četiri ugla mrežicu ispod izbočine  žrtvenika, tako da zahvati do sredine žrtvenika. 
\par 6 Napravi zatim  motke za žrtvenik, motke od bagremova drva, pa ih tučem okuj. 
\par 7 Neka se motke provuku kroz kolutove, tako da dođu na obje  strane žrtvenika kad se nosi. 
\par 8 Načini ga šuplja, od dasaka:  kako ti je pokazano na brdu, onako neka je i napravljen." 
\par 9 "Napravi i dvorište Prebivališta. Na južnoj strani napravi  zavjese od prepredenog lana, sto lakata u dužinu s te strane. 
\par 10 Njihovih dvadeset stupova neka stoji na dvadeset podnožja  od tuča i neka imaju kopče i šipke od srebra. 
\par 11 Isto tako za  sjevernu stranu načini plahte sto lakata duge. Njihovih dvadeset  stupova i dvadeset podnožja od tuča, ali kopče i šipke neka su  od srebra. 
\par 12 Širini dvorišta sa zapadne strane trebat će zavjese  pedeset lakata duge, sa deset stupova i deset podnožja. 
\par 13 Širina  dvorišta prema istočnoj strani neka bude pedeset lakata. 
\par 14 Nadalje, zavjese s jedne strane vrata neka su petnaest lakata duge, sa  svoja tri stupa i njihova tri podnožja. 
\par 15 A s druge strane  neka su zavjese opet petnaest lakata, sa svoja tri stupa i njihova  tri podnožja. 
\par 16 Za dvorišni ulaz: vezen zastor od dvadeset  lakata, od ljubičastog, crvenog i tamnocrvenog prediva i prepredenog  lana; i sa svoja četiri stupa i njihova četiri podnožja. 
\par 17 Svi  stupovi naokolo dvorišta neka su povezani srebrnim šipkama. Neka  su im kopče od srebra, a podnožja od tuča. 
\par 18 Neka je dvorište  u duljinu sto lakata, u širinu pedeset, a u visinu pet lakata.  Neka su mu plahte od prepredenog lana, podnožja od tuča. 
\par 19 Sve  potrepštine u Prebivalištu za opću upotrebu i svi njegovi kočići, a tako i kočići u dvorištu, neka su od tuča." 
\par 20 "Nadalje, naredi Izraelcima da ti za svjetlo donose čistoga  ulja od istupanih maslina, tako da svjetlo neprestano gori. 
\par 21 Aron  i njegovi sinovi neka ga postavljaju u Šator sastanka izvan zavjese  što zaklanja Svjedočanstvo da gori pred Jahvom od večeri do jutra.  Neka je to neopoziva naredba za izraelske naraštaje." 


\chapter{28}

\par 1 "A onda dovedi k sebi između Izraelaca svoga brata Arona zajedno  s njegovim sinovima: Nadabom, Abihuom, Eleazarom i Itamarom da  mi služe kao svećenici. 
\par 2 Napravi svome bratu Aronu sveto ruho  na čast i ukras. 
\par 3 Obrati se svim vještacima koje sam obdario  mudrošću neka naprave haljine Aronu da bi se posvetio i vršio  svećeničku službu u moju čast. 
\par 4 Neka ovu odjeću naprave: naprsnik, oplećak, ogrtač, košulju resama obrubljenu, mitru i pas; neka  naprave svetu odjeću za tvoga brata Arona i njegove sinove da  mi služe kao svećenici. 
\par 5 Stoga neka oni primaju zlato, ljubičasto, crveno i tamnocrveno predivo i prepredeni lan." 
\par 6 "Oplećak neka naprave od zlata, od ljubičastog, crvenog  i tamnocrvenog prediva i od prepredenog lana - vješto izrađen. 
\par 7 Neka na njemu budu dvije poramenice, pričvršćene za njegove  krajeve. 
\par 8 Tkanica što bude na njemu neka je napravljena kao  i on: od zlata, od ljubičastog, crvenog i tamnocrvenog prediva  i od prepredenog lana, a neka s njim sačinjava jedan komad. 
\par 9 Zatim  uzmi dva draga kamena oniksa i u njih ureži imena Izraelovih  sinova: 
\par 10 šest njihovih imena na jednome dragom kamenu, a preostalih  šest imena na drugome dragom kamenu, prema njihovu rođenju. 
\par 11 Kao  što rezbar dragulja urezuje pečate na prstene, tako ti ureži  imena Izraelovih sinova. Oko njih navezi zlatan obrub, 
\par 12 pa  pričvrsti oba draga kamena za poramenice oplećka da budu spomen-dragulji  na Izraelove sinove. Tako neka Aron nosi njihova imena o svoja  dva ramena pred Jahvom da ih se sjeća. 
\par 13 Načini zlatne okvire 
\par 14 i dva lančića od čistoga zlata. Načini ih kao zasukane uzice  i onda zasukane lančiće pričvrsti za okvire." 
\par 15 "Naprsnik za presuđivanje izradi umjetnički; izvedi to  kao i posao na oplećku: od zlata, od ljubičastog, crvenog i tamnocrvenog  prediva i od prepredenog lana. 
\par 16 Neka bude četvorinast i dvostruk;  jedan pedalj neka mu je duljina, a pedalj širina. 
\par 17 Na njemu  poredaj četiri reda dragulja. U prvome redu neka bude: rubin, topaz i alem; 
\par 18 u drugome redu: smaragd, safir i ametist; 
\par 19 u trećem redu: hijacint, ahat i ledac; 
\par 20 a u četvrtom redu:  krizolit, oniks i jaspis. Neka budu ukovani u zlatne okvire. 
\par 21 Tih dragulja neka bude dvanaest, koliko i imena Izraelovih  sinova. Neka budu urezani kao i pečati na prstenju, svaki s imenom  jednoga od dvanaest plemena. 
\par 22 Napravi za naprsnik lančiće  od čistoga zlata, zasukane kao uzice. 
\par 23 Zatim napravi za naprsnik  dva kolutića od zlata i pričvrsti ih na dva gornja ugla naprsnika. 
\par 24 Onda priveži dvije zlatne uzice za ta dva kolutića koja budu  pričvršćena za uglove naprsnika. 
\par 25 Druga dva kraja uzica priveži  za dva okvira. Sad ih tako pričvrsti za poramenice oplećka sprijeda. 
\par 26 Napravi dva kolutića od zlata pa ih pričvrsti za dva donja  ugla naprsnika, uz rub iznutra koji je okrenut prema oplećku. 
\par 27 Napravi još dva kolutića od zlata i pričvrsti ih za donji, prednji kraj poramenice oplećka, uz njegov šav povrh tkanice  oplećka. 
\par 28 Neka se naprsnik sveže za svoje kolutiće s kolutićima  oplećka vrpcom od modroga grimiza, tako da naprsnik stoji iznad  tkanice i da se ne može odvajati od oplećka. 
\par 29 Neka tako Aron, kada god ulazi u Svetište, na svome srcu nosi imena sinova Izraelovih  na naprsniku za presuđivanje da ih uvijek doziva u sjećanje pred  Jahvom. 
\par 30 U naprsnik za presuđivanje neka se stave i 'Urim'  i 'Tumim' da i oni budu na Aronovu srcu kad bude dolazio pred  Jahvu. Tako neka Aron uvijek na svom srcu pred Jahvom nosi presudu  sinova Izraelovih." 
\par 31 "Ogrtač za oplećak sav napravi od ljubičastog prediva. 
\par 32 Prorez za glavu na njemu neka bude na sredini. Rub naokolo  proreza neka bude opšiven kao ovratnik na oklopu, tako da se  ogrtač ne podere. 
\par 33 Na njegovu rubu sve naokolo načini šipke  od ljubičastog, crvenog i tamnocrvenog prediva, a između njih  zvonca od zlata naokolo; 
\par 34 zlatno zvonce pa šipak, zlatno zvonce  pa šipak naokolo ogrtača uz rub. 
\par 35 Neka budu na Aronu dok vrši  službu, da se čuje kad ulazi u Svetište pred Jahvu i kad izlazi;  tako neće umrijeti." 
\par 36 "Napravi potom jednu ploču od čistoga zlata i na njoj  ureži, kao što se urezuje na pečatnom prstenu: 'Jahvi posvećen'. 
\par 37 Za mitru je priveži modrom vrpcom da stoji s pročelja mitre. 
\par 38 Neka stoji na Aronovu čelu. Tako neka Aron na se preuzme  nedostatke koji bi mogli okaljati sve svete prinose što ih Izraelci  posvećuju. Neka uvijek stoji na njegovu čelu da za njih stječe  blagonaklonost Jahvinu. 
\par 39 Košulju s resama napravi od lana, od lana napravi i mitru, a pas vezom izvezi. 
\par 40 I za Aronove sinove napravi haljine, pasove i turbane, njima na čast i ukras. 
\par 41 U njih odjeni svoga brata Arona i  njegove sinove; onda ih pomaži, ispuni im ruke vlašću i posveti  ih da mi služe kao svećenici. 
\par 42 Napravi za njih gaćice od lana  da im pokriju golo tijelo. Neka sežu od bedara do stegna. 
\par 43 Neka  ih nosi Aron i njegovi sinovi kad ulaze u Šator sastanka ili  kad se primiču žrtveniku za službu u Svetištu da ne navuku na  se krivnju i umru. To neka bude vječna naredba za nj i za njegovo  potomstvo poslije njega." 


\chapter{29}

\par 1 "Ovo je obred koji ćeš obaviti na njima da ih posvetiš za  moje svećenike: Uzmi jednog junca i dva ovna bez mane; 
\par 2 onda  beskvasnoga kruha, beskvasnih kolača zamiješenih u ulju i beskvasnih  prevrta uljem namazanih. Napravi ih od bijeloga pšeničnog brašna. 
\par 3 Naslaži ih u košaricu i u košarici prinesi ih s juncem i oba  ovna." 
\par 4 "Dovedi Arona i njegove sinove k ulazu u Šator sastanka  i operi ih u vodi. 
\par 5 Zatim uzmi odijelo i obuci Arona u košulju;  stavi na nj ogrtač oplećka, oplećak i naprsnik i opaši ga tkanicom  oplećka. 
\par 6 Ustakni mu mitru na glavu; na mitru stavi sveti vijenac. 
\par 7 Uzmi zatim ulja za pomazanje; izlij na njegovu glavu i pomaži  ga. 
\par 8 Onda dovedi njegove sinove; obuci ih u košulje; 
\par 9 opaši  ih u pasove i obvij im turbane. Svećeništvo neka im pripada vječnom  uredbom. Tako posveti Arona i njegove sinove!" 
\par 10 "Dovedi zatim junca pred Šator sastanka, pa neka Aron  i njegovi sinovi stave ruke juncu na glavu. 
\par 11 Onda pred Jahvom, na ulazu u Šator sastanka, junca zakolji. 
\par 12 Uzmi junčeve krvi  i svojim je prstom stavi na rogove žrtvenika. Ostatak krvi izlij  podno žrtvenika. 
\par 13 Uzmi sav loj oko droba, privjesak na jetri  i oba bubrega s lojem oko njih, pa spali na žrtveniku. 
\par 14 Meso  od junca, njegovu kožu i njegovu nečist spali na vatri izvan  taborišta. To je žrtva okajnica. 
\par 15 Poslije toga uzmi jednoga ovna, pa neka Aron i njegovi  sinovi stave na njegovu glavu svoje ruke. 
\par 16 Onda ovna zakolji, uhvati mu krvi i zapljusni njome žrtvenik sa svih strana. 
\par 17 Isijeci  zatim ovna u komade, operi mu drobinu i noge i položi ih na njegove  ostale dijelove i glavu. 
\par 18 I onda cijeloga ovna spali na žrtveniku.  Žrtva je to paljenica u čast Jahvi, miris ugodan, žrtva ognjena. 
\par 19 Uzmi onda drugoga ovna, pa neka Aron i njegovi sinovi  stave svoje ruke ovnu na glavu. 
\par 20 Sad ovna zakolji; uzmi mu  krvi i njome namaži resicu desnoga Aronova uha, resicu desnog  uha njegovim sinovima, palac na njihovoj desnoj ruci pa palac  na njihovoj desnoj nozi. Ostatkom krvi zapljusni žrtvenik naokolo. 
\par 21 Uzmi onda krvi što je ostala na žrtveniku i ulja za pomazanje  i poškropi Arona i njegovo odijelo, njegove sinove i njihova  odijela. Tako će biti posvećen on i njegovo odijelo, njegovi  sinovi i odijela njegovih sinova." 
\par 22 "Poslije toga uzmi s ovna loj, pretili rep, loj oko droba, privjesak s jetre, oba bubrega i loj oko njih; desno pleće -  jer je to ovan prinesen za svećeničko posvećenje - 
\par 23 zatim  jedan okrugli kruh, jedan kolač na ulju i jednu prevrtu iz košarice  beskvasnoga kruha što je pred Jahvom. 
\par 24 Sve to stavi na ruke  Arona i njegovih sinova i prinesi žrtvu prikaznicu pred Jahvom. 
\par 25 Uzmi ih onda s njihovih ruku i spali na žrtveniku, povrh  žrtve paljenice, da bude Jahvi na ugodan miris. To je paljena  žrtva u čast Jahvi. 
\par 26 Zatim uzmi grudi ovna prinesena za Aronovo posvećenje  i prinesi ih kao žrtvu prikaznicu pred Jahvom. Neka to bude tvoj  dio. 
\par 27 Posveti grudi što su bile prinesene kao žrtva prikaznica  i pleće što je bilo prineseno kao žrtva podizanica od ovna prinesena  za posvećenje Arona i njegovih sinova. 
\par 28 Neka to bude pristojba  Aronu i njegovim potomcima od Izraelaca za sva vremena. TÓa to  je ujam koji će Izraelci davati od svojih pričesnica - ujam koji  Jahvi pripada. 
\par 29 Aronova posvećena odijela neka pripadnu njegovim sinovima  poslije njega da u njima budu pomazani i posvećeni. 
\par 30 Sin koji  postane svećenikom mjesto njega, kad uđe u Šator sastanka da  vrši službu u Svetištu, neka ih nosi sedam dana." 
\par 31 "Uzmi onda ovna za posvećenje i skuhaj njegovo meso na  posvećenome mjestu. 
\par 32 Aron i njegovi sinovi neka blaguju meso  od toga ovna i kruh iz košarice na ulazu u Šator sastanka. 
\par 33 Neka  jedu od onoga što je poslužilo za njihovo očišćenje, da im se  ruke ispune vlašću i da budu posvećeni. Nijedan svjetovnjak neka  ne jede od toga jer je posvećeno. 
\par 34 Ako bi ostalo što mesa  od svećeničkog posvećenja ili što od onoga kruha do ujutro, spali  na vatri. Ne smije se pojesti jer je posvećeno." 
\par 35 "Točno tako učini Aronu i njegovim sinovima kako sam  ti naredio. Posvećuj ih sedam dana. 
\par 36 Svakoga dana prinesi  jednoga junca kao žrtvu okajnicu - za pomirenje. I prinesi žrtvu  okajnicu za pomirenje oltara, zatim ga pomaži da bude posvećen. 
\par 37 Sedam dana prinosi žrtvu pomirnicu za žrtvenik i posvećuj  ga. Tako će žrtvenik postati presvet, i sve što se žrtvenika  dotakne bit će posvećeno." 
\par 38 "A ovo treba da prinosiš na žrtveniku: dva janjca godinu  dana stara, svaki dan bez prijekida. 
\par 39 Jedno janje žrtvuj ujutro, a drugo uvečer. 
\par 40 Prinesi s prvim janjetom jednu desetinu  efe bijeloga brašna zamiješena u četvrtini hina istupanog ulja  i žrtvu ljevanicu od četvrtine hina vina. 
\par 41 Drugo janje prinesi  u suton. S njim prinesi žrtvu prinosnicu s njezinom žrtvom ljevanicom  kao i izjutra - na ugodan miris, žrtvu u čast Jahvi paljenu. 
\par 42 Neka to bude trajna žrtva paljenica od koljena do koljena  - na ulazu u Šator sastanka, pred Jahvom. Tu ću se ja s tobom  sastajati da ti govorim. 
\par 43 I s Izraelcima ću se tu sastajati, i moja će ih slava posvećivati. 
\par 44 Ja ću posvetiti Šator sastanka  i žrtvenik; posvetit ću Arona i njegove sinove da mi služe kao  svećenici. 
\par 45 Ja ću prebivati među Izraelcima i biti njihov  Bog. 
\par 46 Upoznat će oni tada da sam to ja, Jahve, Bog njihov  koji ih je izbavio iz zemlje egipatske da prebivam među njima  - ja, Jahve, Bog njihov." 


\chapter{30}

\par 1 "Napravi i žrtvenik za paljenje tamjana; napravi ga od bagremova  drva. 
\par 2 Neka bude lakat dug, lakat širok, u pravokut, i dva  lakta visok. Neka mu roščići budu od jednoga komada s njim. 
\par 3 Obloži  mu u čisto zlato: njegovu gornju plohu, njegove strane naokolo  i njegove roščiće. Načini mu zlatan završni pojas naokolo. 
\par 4 Načini  mu dva zlatna koluta. Pričvrsti mu ih s dviju suprotnih strana  ispod završnog pojasa. Kroz njih će se provlačiti motke za nošenje. 
\par 5 Motke načini od bagremova drva i zlatom ih obloži. 
\par 6 Postavi  žrtvenik pred zavjesu što zastire Kovčeg Svjedočanstva - nasuprot  Pomirilištu nad Svjedočanstvom - gdje ću se ja s tobom sastajati. 
\par 7 Neka na njemu Aron pali miomirisni tamjan svako jutro kad  priprema svjetla; 
\par 8 neka ga Aron opet pali u suton kad svjetla  zapaljuje, da to bude svagdašnje kadiono prinošenje pred Jahvom  u sve vaše naraštaje. 
\par 9 Ne prinosi na njemu ni neposvećenoga  tamjana, ni paljenice, ni prinosnice, ni ljevanice! 
\par 10 Jednom  u godini neka Aron obavi obred pomirenja na njegovim roščićima.  Krvlju žrtve koja se prinosi za grijeh, jednom na godinu, neka  obavi obred pomirenja za žrtvenik. Tako činite u sve naraštaje.  Jer oltar je presveta svetinja Jahvina." 
\par 11 Nadalje Jahve reče Mojsiju: 
\par 12 "Kad budeš pravio popis  Izraelaca prilikom novačenja, neka svatko da Jahvi otkupninu  za se kad se upiše, da ih kakvo zlo ne snađe zbog novačenja. 
\par 13 Tko god potpada pod novačenje, ovoliko neka dadne: pola šekela  - prema hramskom šekelu, gdje je dvadeset gera u šekelu. To pola  šekela neka bude kao prinos Jahvi. 
\par 14 Tko god potpada pod novačenje, od dvadeset godina starosti pa naviše, neka dadne prinos Jahvi. 
\par 15 Bogataš neka ne plaća više niti siromah manje od pola šekela  kad daju prinos Jahvi kao otkup za se. 
\par 16 Uzimaj otkupni novac  od Izraelaca i određuj ga za potrebe Šatora sastanka. Neka to  bude Jahvi na spomen da se sjeća Izraelaca i da im bude milostiv." 
\par 17 Reče Jahve Mojsiju: 
\par 18 "Napravi umivaonik od tuča i  podnožje od tuča za umivanje. Postavi ga između Šatora sastanka  i žrtvenika. Nalij u nj vode 
\par 19 pa neka Aron i njegovi sinovi  peru svoje ruke i noge vodom iz njega. 
\par 20 Kad moradnu ulaziti  u Šator sastanka, ili kad se moradnu primicati žrtveniku za službu  da spaljuju žrtve u čast Jahvi paljene, neka se vodom operu da  ne poginu. 
\par 21 Neka operu ruke svoje i noge svoje da izbjegnu  smrti: to je trajna naredba Aronu i njegovim potomcima u sve  naraštaje." 
\par 22 Još reče Jahve Mojsiju: 
\par 23 "Nabavi najboljih mirodija:  pet stotina šekela smirne samotoka, pola te težine - dvjesta  pedeset - mirisavog cimeta, dvjesta pedeset mirisave trstike, 
\par 24 pet stotina - prema hramskom šekelu - lovorike i jedan hin  maslinova ulja. 
\par 25 Od toga napravi posvećeno ulje za pomazanje;  da bude smjesa kao da ju je pravio pomastar. Neka to bude posvećeno  ulje za pomazanje. 
\par 26 Time onda pomaži: Šator sastanka i Kovčeg  Svjedočanstva; 
\par 27 stol i sav njegov pribor; svijećnjak i sav  njegov pribor; žrtvenik kadioni; 
\par 28 žrtvenik za žrtve paljenice  i sav njegov pribor; umivaonik i njegov stalak: 
\par 29 posveti ih, i oni će tako postati posvećeni; i što god ih se dotakne, posvećeno  će postati. 
\par 30 Pomaži Arona i njegove sinove i posveti ih meni  za svećenike. 
\par 31 Onda kaži Izraelcima ovako: 'Ovo je moje posvećeno  ulje za pomazanje od koljena do koljena. 
\par 32 Ne smije se polijevati  po tijelu običnoga čovjeka; ne smijete praviti drugoga ovakva  sastava! To je posvećeno i neka vam bude sveto! 
\par 33 Tko god takvo  napravi, ili tko ga stavi na kojeg svjetovnjaka, neka se odstrani  od svog naroda!'" 
\par 34 Jahve još reče Mojsiju: "Nabavi mirodija: natafe, šeheleta  i helebene. Od ovih mirodija i čistoga tamjana, 
\par 35 sve u jednakim  dijelovima, napravi tamjan za kađenje, smjesu mirodija kakvu  pravi pomastar, opranu, čistu, svetu. 
\par 36 Od toga nešto smrvi  u prah i jedan dio stavi pred Svjedočanstvo, u Šator sastanka, gdje ću se ja s tobom sastajati. Držite ovu mirodiju presvetom! 
\par 37 A miomiris koji napraviš prema ovome sastavu za svoju upotrebu  ne smijete praviti. To drži za svetinju Jahvi! 
\par 38 Tko sebi napravi  što takvo da mu miriše, neka se iskorijeni iz svoga naroda." 


\chapter{31}

\par 1 Jahve reče Mojsiju: 
\par 2 "Pozvao sam, gledaj, po imenu Besalela, sina Urijeva, od koljena Hurova iz plemena Judina. 
\par 3 Napunio  sam ga duhom Božjim koji mu je dao umješnost, razumijevanje i  sposobnost za svakovrsne poslove: 
\par 4 da zamišlja nacrte za radove  od zlata, srebra i tuča; 
\par 5 za rezanje dragulja, za umetanje;  za rezbarije u drvu i poslove svakakve. 
\par 6 Dodao sam još Oholiaba, sina Ahisamakova iz Danova plemena; vještinom sam obdario sve  sposobne ljude da mognu napraviti sve što sam ti naredio: 
\par 7 Šator  sastanka, Kovčeg Svjedočanstva, povrh njega Pomirilište i sav  namještaj Šatora; 
\par 8 stol i sav njegov pribor, čisti svijećnjak  sa svim njegovim priborom; 
\par 9 kadioni žrtvenik, žrtvenik za žrtve  paljenice i njegov pribor, onda umivaonik i njegovo podnožje; 
\par 10 odijela za službu, posvećena odijela za svećenika Arona i  odijela za njegove sinove, za njihovu svećeničku službu; 
\par 11 pa  ulje za pomazanje i miomirisni tamjan za Svetište. Sve neka načine  kako sam ti naredio." 
\par 12 Jahve opet reče Mojsiju: 
\par 13 "Reci Izraelcima: Subote  moje morate održavati, jer subota je znak između mene i vas od  naraštaja do naraštaja, da budete svjesni da vas ja, Jahve, posvećujem. 
\par 14 Držite, dakle, subotu, jer je ona za vas sveta. Tko je oskvrne  neka se pogubi; tko bude u njoj radio ikakav posao neka se odstrani  iz svoga naroda. 
\par 15 Šest dana neka se vrše poslovi, ali sedmi  dan neka bude dan posvemašnjeg odmora, Jahvi posvećen. Tko bi  u dan subotni obavljao kakav posao neka se pogubi. 
\par 16 Stoga  neka Izraelci drže subotu - svetkujući je od naraštaja do naraštaja  - kao vječni savez. 
\par 17 Neka je ona znak, zauvijek, između mene  i Izraelaca. TÓa Jahve je za šest dana sazdao nebo i zemlju,  a sedmoga je dana prestao raditi i odahnuo." 
\par 18 Kad Jahve svrši svoj razgovor s Mojsijem na Sinajskom  brdu, dade mu dvije ploče Svjedočanstva, ploče kamene, ispisane  prstom Božjim. 


\chapter{32}

\par 1 A narod, videći gdje Mojsije dugo ne silazi s brda, okupi  se oko Arona pa mu rekne: "Ustaj! Napravi nam boga, pa neka on  pred nama ide! Ne znamo što se dogodi s tim čovjekom Mojsijem  koji nas izvede iz zemlje egipatske." 
\par 2 "Poskidajte zlatne naušnice  što vise o ušima vaših žena, vaših sinova i vaših kćeri", odgovori  im Aron, "pa ih meni donesite." 
\par 3 Sav svijet skine zlatne naušnice  što ih je o ušima imao i donese Aronu. 
\par 4 Primivši zlato iz njihovih  ruku, rastopi kovinu u kalupu i načini saliveno tele. A oni poviču:  "Ovo je tvoj bog, Izraele, koji te izveo iz zemlje egipatske." 
\par 5 Vidjevši to Aron, sagradi pred njim žrtvenik a onda najavi:  "Sutra neka se priredi svečanost u čast Jahvi!" 
\par 6 Sutradan rano ustanu i prinesu žrtve paljenice i donesu  žrtve pričesnice. Onda svijet posjeda da jede i pije. Poslije  toga ustade da se zabavlja. 
\par 7 "Požuri se dolje!" - progovori Jahve Mojsiju. "Narod tvoj, koji si izveo iz zemlje egipatske, pošao je naopako. 
\par 8 Brzo  su zašli s puta koji sam im odredio. Napravili su sebi tele od  rastopljene kovine, preda nj pali ničice i žrtve mu prinijeli  uz poklike: 'Ovo je tvoj bog, Izraele, koji te izveo iz zemlje  egipatske!' 
\par 9 Dobro vidim", reče dalje Jahve Mojsiju, "da je  ovaj narod tvrde šije. 
\par 10 Pusti sada neka se moj gnjev na njih  raspali da ih istrijebim. Onda ću od tebe razviti velik narod." 
\par 11 Mojsije pak zapomagao pred Jahvom, Bogom svojim, i govorio:  "O Jahve! Čemu da gnjevom plamtiš na svoj narod koji si izbavio  iz zemlje egipatske silom velikom i rukom jakom! 
\par 12 Zašto bi  Egipćani morali reći: 'U zloj ih je namjeri i odveo, tako da  ih smakne u brdinama i izbriše s lica zemlje!' Smiri svoj gnjev  i ljutinu; odustani od zla svome narodu! 
\par 13 Sjeti se Abrahama, Izaka i Izraela, slugu svojih, kojima si se samim sobom zakleo  i obećao im: 'Razmnožit ću vaše potomstvo kao zvijezde na nebu  i svu zemlju ovu što sam obećao dat ću vašem potomstvu i ona  će zavazda biti njihova baština.'" 
\par 14 I Jahve odustane da na  svoj narod svali nesreću kojom mu bijaše zaprijetio. 
\par 15 Mojsije se okrene i siđe s brda. U rukama su mu bile  dvije ploče Svjedočanstva, ploče ispisane na objema plohama;  ispisane i s jedne i s druge strane. 
\par 16 Ploče su bile djelo  Božje; pismo je bilo pismo Božje u pločama urezano. 
\par 17 A Jošua ču viku naroda koji je bučio pa reče Mojsiju:  "Bojna vika u taboru!" 
\par 18 Mojsije mu odgovori: "Niti viču pobjednici, niti tuže pobijeđeni: tu ja samo pjesmu čujem." 
\par 19 Čim se približi taboru te opazi tele i kako igraju, razgnjevi  se Mojsije. Baci iz ruku ploče i razbije ih na podnožju brda. 
\par 20 Pograbi tele koje bijahu napravili, spali ga ognjem i u prah  satre. Onda prah razbaca po vodi i natjera Izraelce da je piju. 
\par 21 "Što ti je ovaj puk učinio", reče Mojsije Aronu, "da si tako  velik grijeh na nj svalio?" 
\par 22 "Neka se moj gospodar srdžbom  ne raspaljuje", odgovori Aron. "Sam dobro znaš kako je ovaj narod  na zlo sklon. 
\par 23 Rekoše mi: 'Napravi nam boga pa neka pred nama  ide! Ne znamo što se dogodi s tim čovjekom Mojsijem koji nas  izbavi iz zemlje egipatske.' 
\par 24 Na to im ja rekoh: 'Tko ima  zlata, neka ga skine!' Tako mi ga dadoše, a ja ga bacih u vatru  te izađe ovo tele." 
\par 25 Kad je Mojsije vidio kako je narod postao razuzdan -  tÓa Aron ih je pustio da padnu u idolopoklonstvo među svojim  neprijateljima - 
\par 26 stade na taborskim vratima i povika: "Tko  je za Jahvu, k meni!" Svi se sinovi Levijevi okupe oko njega. 
\par 27 On im reče: "Ovako govori Jahve, Bog Izraela: 'Neka svatko  pripaše mač o bedro i pođe taborom od vrata do vrata pa neka  ubije tko svoga brata, tko svoga prijatelja, tko svoga susjeda.'" 
\par 28 Sinovi Levijevi izvršiše Mojsijev nalog, i toga dana pade  naroda oko tri tisuće ljudi. 
\par 29 "Danas ste se posvetili Jahvi  za službu", reče Mojsije, "tko uz cijenu svoga sina, tko uz cijenu  svoga brata, tako da vam danas daje blagoslov." 
\par 30 Sutradan reče Mojsije narodu: "Težak ste grijeh počinili.  Ipak ću se Jahvi popeti. Možda za vaš grijeh oproštenje pribavim." 
\par 31 Mojsije se vrati Jahvi pa reče: "Jao! Narod onaj težak je  grijeh počinio napravivši sebi boga od zlata. 
\par 32 Ipak im taj  grijeh oprosti... Ako nećeš, onda i mene izbriši iz svoje knjige  koju si napisao." 
\par 33 Nato Jahve odgovori Mojsiju: "Onoga koji  je protiv mene sagriješio izbrisat ću iz svoje knjige. 
\par 34 Nego, idi sad! Povedi narod kamo sam ti rekao. Anđeo će moj pred tobom  ići. Ali u dan kad ih pohodim, zbog njihova ću ih grijeha kazniti." 
\par 35 Udari Jahve po narodu pomorom zbog teleta što im ga Aron  načini. 


\chapter{33}

\par 1 Jahve reče Mojsiju: "Idi! Putuj odavde, ti i narod koji si  izveo iz zemlje egipatske, u zemlju za koju sam se zakleo Abrahamu, Izaku i Jakovu da ću je dati njihovim potomcima. 
\par 2 Pred tobom  ću poslati anđela; istjerat ću Kanaance, Amorejce, Hetite, Perižane, Hivijce i Jebusejce. 
\par 3 Idite u zemlju kojom teče mlijeko i  med. Ja s vama neću poći - jer ste narod tvrde šije - da vas  putem ne istrijebim." 
\par 4 Kad narod ču ove oštre riječi, poče  tugovati. I nitko više ne stavi na se svoga nakita. 
\par 5 Jer reče  Jahve Mojsiju: "Kaži Izraelcima: 'Vi ste narod tvrde šije. Kad  bih ja s vama išao samo čas, uništio bih vas. Stoga skinite svoj  nakit, a ja ću vidjeti što ću s vama učiniti.'" 
\par 6 Tako su od  brda Horeba Izraelci bili bez nakita. 
\par 7 Mojsije uze Šator i razape ga izvan tabora, daleko od  tabora. I nazva ga Šator sastanka. Tko bi se god htio obratiti  Jahvi, pošao bi k Šatoru sastanka, koji se nalazio izvan tabora. 
\par 8 Kad bi god Mojsije pošao u Šator, sav bi se narod digao; svatko  bi stajao kod ulaza u svoj šator i gledao za Mojsijem dok ne  bi ušao u Šator. 
\par 9 A kad bi Mojsije ušao u Šator, stup bi se  oblaka spustio i ostajao na ulazu u Šator dok je Jahve s Mojsijem  razgovarao. 
\par 10 Videći kako stup oblaka stoji na ulazu Šatora, sav bi se narod tada dizao i svatko bi se duboko klanjao na  vratima svoga šatora. 
\par 11 Tako bi Jahve razgovarao s Mojsijem  licem u lice, kao što čovjek govori s prijateljem. Mojsije bi  se poslije vratio u tabor, ali se njegov pomoćnik Jošua, sin  Nunov, mlađarac, iz Šatora ne bi micao. 
\par 12 Mojsije oslovi Jahvu: "Vidi, ti si meni rekao: 'Povedi  ovaj narod', ali mi nisi objavio koga ćeš sa mnom poslati. Još  si mi rekao: 'Znam te po imenu, i ti uživaš moju blagonaklonost.' 
\par 13 Stoga, ako uživam tvoju blagonaklonost, objavi mi svoje putove  da te shvatim i da dalje uživam tvoju blagonaklonost. Promisli  također da je ova svjetina tvoj narod." 
\par 14 "Ja ću osobno s tobom  poći", odgovori Jahve, "i počinak ti priuštiti." 
\par 15 "Ako ti  ne pođeš", nadoda Mojsije, "odavde nas i ne izvodi. 
\par 16 TÓa kako  će se znati da uživamo tvoju naklonost, ja i tvoj narod? Po tome  što ideš s nama. Time ćemo se samo razlikovati ja i tvoj narod  među svim narodima koji su na licu zemlje." 
\par 17 "I ovo što si  zatražio, učinit ću", odgovori Jahve Mojsiju. "TÓa ti uživaš  moju blagonaklonost jer te po imenu poznajem." 
\par 18 "Pokaži mi svoju slavu", zamoli Mojsije. 
\par 19 "Dopustit  ću da ispred tebe prođe sav moj sjaj", odgovori, "i pred tobom  ću izustiti svoje ime Jahve. Bit ću milostiv kome hoću da milostiv  budem; smilovat ću se komu hoću da se smilujem. 
\par 20 A ti", doda, "moga lica ne možeš vidjeti, jer ne može čovjek mene vidjeti  i na životu ostati. 
\par 21 Evo mjesta ovdje uza me", nastavi Jahve.  "Stani na pećinu! 
\par 22 Dok moja slava bude prolazila, stavit ću  te u pukotinu pećine i svojom te rukom zakloniti dok ne prođem. 
\par 23 Onda ću ja svoju ruku maknuti, pa ćeš me s leđa vidjeti.  Ali se lice moje ne može vidjeti." 


\chapter{34}

\par 1 Reče Jahve Mojsiju: "Okleši dvije kamene ploče kao i prijašnje  pa ću ja na ploče napisati riječi koje su bile na prvim pločama  što si ih razbio. 
\par 2 Budi gotov do jutra. Onda, ujutro, popni  se na brdo Sinaj i ondje ćeš, navrh brda, stupiti preda me. 
\par 3 Nitko  drugi neka se s tobom ne penje; neka se nitko nigdje na brdu  ne pokaže. Neka ni ovce ni goveda ne pasu podno brda." 
\par 4 Mojsije  okleše dvije kamene ploče kao i prijašnje; rano jutrom ustane  i popne se na Sinajsko brdo, uzevši u ruke dvije kamene ploče, kako mu je Jahve naredio. 
\par 5 Jahve se spusti u liku oblaka, a on stade preda nj i zazva  Ime: "Jahve!" 
\par 6 Jahve prođe ispred njega te se javi: "Jahve!  Jahve! Bog milosrdan i milostiv, spor na srdžbu, bogat ljubavlju  i vjernošću, 
\par 7 iskazuje milost tisućama, podnosi opačinu, grijeh  i prijestup, ali krivca nekažnjena ne ostavlja nego kažnjava  opačinu otaca na djeci - čak na unučadi do trećega i četvrtog  koljena." 
\par 8 Mojsije smjesta pade na zemlju i pokloni se. 
\par 9 Onda reče:  "Gospodine moj! Ako sam stekao blagonaklonost u tvojim očima, onda, o Gospodine, pođi s nama! Premda je narod tvrde šije,  oprosti naše grijehe i naše opačine i primi nas za svoju baštinu!" 
\par 10 "Dobro", odgovori, "sklapam Savez. Pred cijelim tvojim  pukom činit ću čudesa kakva se nisu događala ni u kojoj zemlji, ni u kojem narodu. Sav narod koji te okružuje vidjet će što  može Jahve, jer ono što ću s tobom učiniti bit će strašno. 
\par 11 Vrši, dakle, što ti danas nalažem! Gle, protjerat će ispred tebe Amorejce, Kanaance, Hetite, Perižane, Hivijce i Jebusejce. 
\par 12 Čuvaj se  da ne praviš saveza sa stanovnicima zemlje u koju ideš; da ne  budu zamkom u tvojoj sredini. 
\par 13 Nego porušite njihove žrtvenike, oborite njihove stupove, počupajte im ašere! 
\par 14 Jer ne smiješ  se klanjati drugome bogu. TÓa Jahve - ime mu je Ljubomorni -  Bog je ljubomoran. 
\par 15 Ne pravi saveza sa stanovnicima one zemlje  da te oni, kad se odaju bludnosti sa svojim bogovima i žrtve  im budu prinosili, ne bi pozivali, a ti pristao da jedeš od prinesene  žrtve; 
\par 16 da ne bi uzimao njihove djevojke za žene svojim sinovima, da one - odajući se bludništvu sa svojim bogovima - ne bi za  sobom povele i tvoje sinove. 
\par 17 Ne pravi sebi livenih bogova! 
\par 18 Drži Blagdan beskvasnoga kruha - jedući beskvasni kruh  sedam dana, kako sam ti naredio - u određeno vrijeme u mjesecu  Abibu, jer si u mjesecu Abibu izišao iz Egipta. 
\par 19 Svako prvorođenče materinjega krila meni pripada: svako  muško, svaki prvenac tvoga i sitnoga i krupnoga blaga. 
\par 20 Prvenca  od magarice otkupi jednim grlom sitne stoke. Ako ga ne otkupiš, moraš mu šijom zavrnuti. A sve prvorođence od svojih sinova  otkupljuj.  Neka nitko preda me ne stupa praznih ruku! 
\par 21 Šest dana radi, a sedmoga od poslova odustani, sve ako  je u doba oranja ili u vrijeme žetve. 
\par 22 Svetkuj Blagdan sedmica - prvine pšenične žetve - i Blagdan  berbe na prekretu godine. 
\par 23 Triput na godinu neka se svi muškarci pojave pred Gospodinom  Jahvom, Bogom Izraelovim. 
\par 24 Jer ću protjerati narode ispred  tebe i proširiti tvoje međe te nitko neće hlepiti za tvojom zemljom  kad triput u godini budeš uzlazio da se pokažeš pred Jahvom,  Bogom svojim. 
\par 25 Od žrtve koju mi namjenjuješ ne prinosi krvi ni s čim  ukvasanim; niti ostavljaj žrtve prinesene na blagdan Pashe da  prenoći do jutra. 
\par 26 U kuću Jahve, Boga svoga, donosi najbolje prvine plodova  sa svoje zemlje. Ne kuhaj kozleta u mlijeku njegove majke. 
\par 27 Zapiši ove riječi", reče Jahve Mojsiju, "jer su one temelji  na kojima sam s tobom i s Izraelom sklopio Savez." 
\par 28 Mojsije ostade ondje s Jahvom četrdeset dana i četrdeset  noći. Niti je kruha jeo niti je vode pio. Tada je na ploče ispisao  riječi Saveza - Deset zapovijedi. 
\par 29 Napokon Mojsije siđe sa Sinajskog brda. Silazeći s brda, nosio je u rukama ploče Svjedočanstva. Nije ni znao da iz njegova  lica, zbog razgovora s Jahvom, izbija svjetlost. 
\par 30 Kad su Aron  i svi Izraelci vidjeli kako iz Mojsijeva lica izbija svjetlost, ne usudiše se k njemu pristupiti. 
\par 31 Onda ih Mojsije zovnu.  Tada k njemu dođoše Aron i sve starješine zajednice. I Mojsije  razgovaraše s njima. 
\par 32 Poslije k njemu dođoše i svi Izraelci, pa im on priopći sve što mu je naložio Jahve na Sinajskom brdu. 
\par 33 Kad je Mojsije završio razgovor s njima, prevuče preko svoga  lica koprenu. 
\par 34 Kad bi god Mojsije ulazio pred Jahvu da s njim  razgovara, koprenu bi skinuo dok opet ne bi izišao. Kad bi izlazio  da Izraelcima kaže što mu je naređeno, 
\par 35 Izraelci bi vidjeli  kako iz Mojsijeva lica izbija svjetlost. Tada bi Mojsije opet  prevukao koprenu preko lica dok ne uđe da s Jahvom govori. 


\chapter{35}

\par 1 Mojsije sazva svu zajednicu sinova Izraelovih pa im reče:  "Ovo vam je Jahve naložio da činite: 
\par 2 Neka se posao obavlja  šest dana. Sedmi dan neka vam bude sveti dan, dan potpunog počinka  u čast Jahvi. Tko bi radio u taj dan neka se kazni smrću. 
\par 3 Na  subotnji dan ni vatre ne ložite po svojim stanovima." 
\par 4 Nadalje Mojsije reče svoj zajednici izraelskoj: "Ovo je  Jahve naredio: 
\par 5 Među sobom pokupite prinos Jahvi! Tko god je  plemenita srca neka Jahvi donese prinos: zlata, srebra i tuča; 
\par 6 ljubičastog, crvenog i tamnocrvenog prediva i prepredenog  lana; 
\par 7 učinjenih ovnujskih koža, onda finih koža, bagremova  drva, 
\par 8 ulja za svjetlo, mirodija za ulje pomazanja i miomirisni  tamjan; 
\par 9 oniksa i drugoga dragog kamenja za umetanje u oplećak  i naprsnik. 
\par 10 A svi koji su među vama vješti neka dođu praviti što  je Jahve naredio: 
\par 11 Prebivalište, njegov Šator i pokrov; njegove  kuke i trenice, njegove priječnice i stupce; njegova podnožja; 
\par 12 njegov Kovčeg i motke; Pomirilište pa zavjesu za zaklon; 
\par 13 stol s njegovim motkama i sve njegove potrepštine; prinesene  hljebove; 
\par 14 svijećnjak za svjetlo, njegov pribor i njegove  svijeće, onda ulje za svjetlo; 
\par 15 kadioni žrtvenik i njegove  motke; ulje za pomazanje i miomirisni tamjan; zastorak na ulazu  u Prebivalište; 
\par 16 žrtvenik za žrtve paljenice s njegovom rešetkom  od tuča; motke za nj i sav njegov pribor; umivaonik i njegov  stalak; 
\par 17 zastore za dvorište; stupce i njihova podnožja, pa  zastor na ulazu u dvorište; 
\par 18 kočiće za Prebivalište i kočiće  za dvorište s njihovim uzicama; 
\par 19 svečano ruho za vršenje službe  u Svetištu - posvećena odijela za svećenika Arona i odijela za  svećeničku službu njegovih sinova." 
\par 20 Nato se sva izraelska zajednica povuče ispred Mojsija. 
\par 21 A onda svatko koga je srce vuklo i duša poticala dođe noseći  svoj prinos u čast Jahvi za gradnju Šatora sastanka, za svaku  službu u njemu i za posvećena odijela. 
\par 22 Strčaše se muževi  i žene: svi koje je srce vuklo donesoše zapinjača, naušnica,  prstenja, narukvica, ogrlica i svakovrsna zlatnog nakita; svi  koji bijahu zavjetovali kakvu zlatninu u čast Jahvi. 
\par 23 Svi  kod kojih se našlo ljubičastog, crvenog i tamnocrvenog prediva  i prepredenog lana, učinjenih ovnujskih koža ili finih koža donesoše  svoje. 
\par 24 Nadalje, svatko tko je mogao dati kakav dar u srebru  ili tuču donese to kao prinos u čast Jahvi. Svatko u koga se  našlo bagremova drva za upotrebu u bilo kojem poslu, donese ga. 
\par 25 Sve žene koje su bile vješte prele su svojim rukama i donosile  što bijahu oprele: ljubičastog, crvenog i tamnocrvenog prediva  i prepredenog lana. 
\par 26 Sve opet žene koje je njihovo srce poticalo  zbog njihove vještine prele su kostrijet. 
\par 27 Glavari su donosili  oniksa i drugoga dragog kamenja za umetanje u oplećak i naprsnik; 
\par 28 pa mirodije i ulje za svjetlo, ulje za pomazanje i miomirisni  tamjan. 
\par 29 I tako Izraelci - svi ljudi i sve žene koje je srce  poticalo da pridonesu bilo što poslu koji je Jahve po Mojsiju  naredio da se izvrši - donesoše to kao dragovoljan prinos u čast  Jahvi. 
\par 30 Potom reče Mojsije Izraelcima: "Vidite! Jahve je po imenu  pozvao Besalela, sina Urijeva, od koljena Hurova a iz plemena  Judina. 
\par 31 Njega je napunio duhom Božjim, dao mu umješnost,  sposobnost i razumijevanje u svim poslovima: 
\par 32 da zamišlja  nacrte i da radove izvodi od zlata, srebra i tuča; 
\par 33 da reže  dragulje za umetanje; da urezuje u drvetu i da umješno radi svaki  posao. 
\par 34 Njemu i Oholiabu, sinu Ahisamakovu, od plemena Danova, udijeli i sposobnost da poučavaju druge. 
\par 35 Obdari ih umještvom  u svakom poslu rezbarskom, krojačkom, veziljskom i tkalačkom;  oni tkaju tkanine od ljubičastog, crvenog i tamnocrvenog prediva  i prepredenog lana, sposobni su u svakom poslu i vješti u nacrtima. 


\chapter{36}

\par 1 Stoga neka Besalel, Oholiab i svi vještaci koje je Jahve obdario  vještinom i sposobnošću da vješto izvedu sve poslove oko podizanja  Svetišta obave sve kako je Jahve naredio." 
\par 2 Mojsije onda pozva Besalela, Oholiaba i sve one što ih  je Jahve obdario vještinom; sve koje je srce poticalo da se prihvate  posla i izvedu ga. 
\par 3 Oni preuzmu od Mojsija sve prinose koje  Izraelci bijahu donijeli da se izvedu poslovi oko podizanja Svetišta.  Ali kako su oni i dalje donosili prinose jutro za jutrom, 
\par 4 svi  majstori koji su gradili Svetište dođu - svaki s posla na kojem  je radio - 
\par 5 i reknu Mojsiju: "Svijet donosi mnogo više nego  što je potrebno za izvođenje posla koji nam je Jahve naredio  da izvedemo." 
\par 6 Zato Mojsije izda naredbu koju po taboru proglase:  "Neka više nijedan čovjek ni žena ne donosi nikakva priloga za  Svetište!" Tako ustave narod te nije donosio novih darova. 
\par 7 Što  imahu bijaše dosta da se izvede sve djelo; i još je pretjecalo. 
\par 8 I tako najvještiji ljudi među radnicima naprave Prebivalište.  Načine ga od deset zavjesa od prepredenog lana i ljubičastog, crvenog i tamnocrvenog prediva. Na njima bijahu izvezeni likovi  kerubina. 
\par 9 Dužina je svake zavjese iznosila dvadeset osam lakata, a širina svake zavjese četiri lakta. Sve su zavjese bile iste  mjere. 
\par 10 Pet zavjesa sastave jednu s drugom, a pet drugih zavjesa  sastave opet jednu s drugom. 
\par 11 Na rubu posljednje od zavjesa  sastavljenih u jedno načine petlje od modre vune; jednako ih  načine i na rubu posljednje zavjese drugoga dijela; 
\par 12 načine  pedeset petlji na jednome komadu, a pedeset na rubu zavjese drugoga  komada. Petlje su stajale jedna spram druge. 
\par 13 Onda naprave  pedeset zlatnih kopča pa sastave zavjese kopčama jednu s drugom.  Tako je Prebivalište bilo kao jedna cjelina. 
\par 14 Zatim za Šator  povrh Prebivališta načine zavjese od kostrijeti; načine ih jedanaest. 
\par 15 Dužina svake zavjese bila je trideset lakata, a širina četiri  lakta. Tih jedanaest zavjesa bilo je iste mjere. 
\par 16 Sastave  pet zavjesa za se, a šest drugih opet za se. 
\par 17 Naprave pedeset  petlji na rubu zavjese jednoga komada, a pedeset načine na rubu  drugoga komada. 
\par 18 Načine i pedeset kopča od tuča da sastave  Šator zajedno, tako da bude jedna cjelina. 
\par 19 Zatim naprave  pokrov za Šator od učinjenih ovnujskih koža, a povrh njega drugi, od finih koža. 
\par 20 Trenice nauzgor za Prebivalište izrade od bagremova drva. 
\par 21 Duljina je svake trenice bila deset lakata, a širina lakat  i pol. 
\par 22 Svaka je trenica imala dva klina da je drže uspravno.  To su napravili na svakoj trenici za Šator. 
\par 23 Trenice za Prebivalište  napravili su ovako: dvadeset trenica za južnu stranu; 
\par 24 napravili  su četrdeset podnožja od srebra pod dvadeset trenica - dva podnožja  pod prvu trenicu za njezina dva klina i dva podnožja pod svaku  slijedeću trenicu za njezina dva klina. 
\par 25 Za drugu, sjevernu, stranu Prebivališta naprave dvadeset trenica 
\par 26 i za njih četrdeset  podnožja od srebra - dva podnožja pod prvu trenicu, a po dva  podnožja pod svaku slijedeću trenicu. 
\par 27 Prebivalištu straga, prema zapadu, napraviše šest trenica. 
\par 28 Naprave i dvije trenice  za uglove Prebivališta straga. 
\par 29 Pri dnu su bile rastavljene, ali su se pri vrhu, kod prvoga koluta, sastajale. Tako su ih  obje postavili za dva ugla. 
\par 30 Bilo je osam trenica s njihovim  podnožjima od srebra: šesnaest podnožja, pod svakom trenicom  dva. 
\par 31 Načine priječnice od bagremova drva: pet njih za trenice  s jedne strane Prebivališta, 
\par 32 a pet opet priječnica za trenice  s druge strane Prebivališta te pet priječnica za trenice Prebivalištu  straga, prema zapadu. 
\par 33 Onda načine središnju priječnicu što  je prolazila sredinom trenica s kraja na kraj. 
\par 34 Trenice oblože  zlatom, a njihove kolutove, kroz koje su priječnice bile provučene, načine od zlata. I priječnice oblože zlatom. 
\par 35 Naprave zavjesu od ljubičastog, crvenog i tamnocrvenog  prediva i prepredenog lana; načine je s izvezenim kerubinima. 
\par 36 Za nju naprave četiri stupa od bagremova drva i oblože ih  zlatom. Kuke su im bile od zlata, a saliju im i četiri podnožja  od srebra. 
\par 37 Na ulazu u Šator naprave zavjesu od ljubičastog, crvenog i tamnocrvenog prediva i prepredenog lana, umjetnički  protkanu, i za nju pet stupčića s njihovim kukama. Vrhove stupčića  i njihove šipke oblože zlatom, dok im pet podnožja naprave od  tuča. 
\par 38 


\chapter{37}

\par 1 Besalel napravi Kovčeg od bagremova drva, dug dva i pol lakta, širok lakat i pol, a lakat i pol visok. 
\par 2 Iznutra ga i izvana  okuje čistim zlatom. Naokolo mu napravi zlatan završni pojas. 
\par 3 I salije mu četiri koluta na njegova četiri ugla: dva koluta  s jedne strane, a dva koluta s njegove druge strane. 
\par 4 Napravi  i motke od bagremova drva i u zlato ih okuje; 
\par 5 onda provuče  motke kroz kolutove Kovčegu sa strane za nošenje Kovčega. 
\par 6 Zatim  napravi Pomirilište od čistoga zlata, dva i pol lakta dugo, a  lakat i pol široko. 
\par 7 Napravi i dva kerubina od kovanoga zlata, na dva kraja Pomirilišta: 
\par 8 jednoga kerubina na jednome kraju, a drugoga kerubina na drugome kraju. Kerubine na oba kraja načini  u jednome komadu s Pomirilištem. 
\par 9 Kerubini imali uzdignuta  i raširena krila, zaklanjali njima Pomirilište. Bili su licem  okrenuti jedan prema drugome, tako da su im lica gledala u Pomirilište. 
\par 10 Od bagremova drva načini stol, dva lakta dug, lakat širok, a lakat i pol visok. 
\par 11 Obloži ga čistim zlatom i od zlata  mu naokolo načini završni pojas. 
\par 12 I načini mu obrub unaokolo, podlanicu širok. A za obrub naokolo načini zlatan završni pojas. 
\par 13 Salije mu četiri zlatna koluta. Kolutove onda pričvrsti za  njegova četiri nožna ugla. 
\par 14 Kolutovi su bili tik pod obrubom, kao kvake za motke, da se stol može nositi. 
\par 15 Motke za nošenje  stola načinio je od bagremova drva i zlatom ih obložio. 
\par 16 A  pribor što se držao na stolu - njegove zdjele, varjače, vrčeve  i pehare za izlijevanje prinosa - napravio je od čistoga zlata. 
\par 17 Od čistoga zlata načini i svijećnjak. Svijećnjak - njegovo  podnožje i stalak - skova. Njegove čaše - čaške i latice - bile  su u jednome komadu s njim. 
\par 18 Šest je krakova izbijalo s njegovih  strana: tri kraka svijećnjaka s jedne strane, a tri kraka svijećnjaka  s druge strane. 
\par 19 Na jednome kraku bile su tri čaše u obliku  bademova cvijeta, svaka sa svojom čaškom i laticama. Na drugome  opet kraku bile su tri čaše u obliku bademova cvijeta, svaka  s čaškom i laticama. Tako je bilo na svih šest krakova što izbijahu  iz svijećnjaka. 
\par 20 Na samome svijećnjaku bile su četiri čaše  u obliku bademova cvijeta, svaka s čaškom i laticama: 
\par 21 čaška, u jednom komadu s njim, pod prva dva kraka; pa konačno čaška, u jednom komadu s njim, pod zadnja dva kraka. Tako na svih šest  krakova što su iz njega izbijali. 
\par 22 Njihove čaške i njihove  peteljke bile su u jednom komadu s njim; sve to od čistoga kovanog  zlata. 
\par 23 A od čistoga zlata napravi mu i sedam svjetiljaka, usekače i pepeljare. 
\par 24 Svijećnjak i sav njegov pribor načini  od jednoga talenta čistoga zlata. 
\par 25 Kadioni je žrtvenik napravio od bagremova drva, lakat  dug, lakat širok - u četvorinu - a dva lakta visok. Roščići su  mu bili u jednom komadu s njim. 
\par 26 Obloži mu čistim zlatom plohu, strane naokolo i njegove roščiće. Načini mu naokolo završni  pojas od zlata. 
\par 27 Na njemu načini i dva zlatna koluta na oprečnim  stranama, ispod završnog pojasa, da služe motkama za kvake kad  se na njima nosi. 
\par 28 Motke načini od bagremova drva pa ih obloži  zlatom. 
\par 29 Onda pripravi posvećeno ulje za pomazanje i čisti  kad mirisni, onako kako ga pravi pomastar. 


\chapter{38}

\par 1 Od bagremova drva napravi žrtvenik za žrtve paljenice, pet  lakata dug, pet lakata širok - u četvorinu - a tri lakta visok. 
\par 2 Na njegova četiri ugla načini mu četiri roga. Rogovi su bili  u jednom komadu s njim. Onda ga obloži tučem. 
\par 3 A načini i sav  pribor za žrtvenik: lonce, strugače, kotliće, viljuške i kadionike;  sav mu je ovaj pribor načinio od tuča. 
\par 4 Za žrtvenik zatim načini  rešetku u obliku mrežice od tuča ispod izbočine; zahvaćala mu  je do sredine. 
\par 5 Salije četiri koluta na četiri ugla tučane  rešetke da služe kao kvake za motke. 
\par 6 Motke načini od bagremova  drva pa ih obloži tučem. 
\par 7 Onda provuče motke kroz kolutove  na objema stranama žrtvenika da se na njima nosi. Napravio ga  je šuplja - od dasaka. 
\par 8 A zatim, od zrcala žena koje su posluživale na vratima  Šatora sastanka, načini tučani umivaonik i tučani stalak za nj. 
\par 9 Onda načini dvorište. Na južnoj strani dvorišta bijahu  zavjese od prepredenog lana, stotinu lakata duge. 
\par 10 Njihovih  dvadeset stupova sa dvadeset podnožja bilo je od tuča, dok su  kuke na stupovima i njihove šipke bile od srebra. 
\par 11 Od stotinu  lakata bile su zavjese i sa sjeverne strane. Njihovih dvadeset  stupova sa dvadeset podnožja bilo je od tuča, dok su kuke na  stupovima i njihove šipke bile od srebra. 
\par 12 Sa zapadne strane  bijahu zavjese od pedeset lakata, sa deset stupova i deset njihovih  podnožja. Kuke su na stupovima i njihove šipke bile od srebra. 
\par 13 Sprijeda, s istoka, zavjese od pedeset lakata. 
\par 14 S jedne  strane vrata zavjese su bile petnaest lakata, sa tri stupca i  njihova tri podnožja. 
\par 15 Tako i s druge strane - dakle, na obje  strane dvorišnih vrata - bile su zavjese od petnaest lakata,  sa tri stupca i njihova tri podnožja. 
\par 16 Sve su zavjese oko  dvorišta bile od prepredenog lana. 
\par 17 Podnožja za stupove bila  su od tuča, a kuke na stupovima i njihove šipke od srebra. Vrhovi  stupova bili su srebrom obloženi. Sve šipke na dvorišnim stupovima  bijahu od srebra. 
\par 18 Zavjesa na dvorišnim vratima - izvezena  - bila je od ljubičastog, crvenog i tamnocrvenog prediva i prepredenog  lana. Dvadeset je lakata bila duga; visoka, po širini, pet lakata  kao i dvorišne zavjese. 
\par 19 Bila su četiri njihova stupa sa četiri  podnožja od tuča. Kuke na stupovima bile su od srebra. Vrhovi  stupova bili su srebrom obloženi, a njihove šipke bile su srebrne. 
\par 20 Svi kočići unutar Prebivališta bili su od tuča. 
\par 21 To je popis stvari za Prebivalište - Prebivalište Svjedočanstva, koji je sastavljen na zapovijed Mojsijevu trudom levita pod  vodstvom Itamara, sina svećenika Arona. 
\par 22 Besalel, Urijev sin, iz koljena Hurova od plemena Judina  napravio je sve što je Jahve Mojsiju naredio. 
\par 23 S njim je bio  Oholiab, sin Ahisamakov, iz plemena Danova, rezbar, krojač i  vezilac za ljubičasto, crveno i tamnocrveno predivo i prepredeni  lan. 
\par 24 Sve zlato što je utrošeno u radove oko Svetišta - zlato  posvećeno prinosom - iznosilo je: dvadeset i devet talenata i  sedam stotina trideset šekela u hramskim šekelima. 
\par 25 A srebro, sabrano prigodom upisivanja zajednice - 
\par 26 to jest beku po  glavi, odnosno pola šekela prema hramskom šekelu, od svakoga  koji je bio upisan, od dvadeset godina pa naprijed - iznosilo  je: stotinu talenata i tisuću sedam stotina sedamdeset i pet  šekela u hramskim šekelima. Bilo je upisanih: šest stotina tri  tisuće i petsto pedeset. 
\par 27 Stotinu talenata srebra otišlo je  za salijevanje podnožja Svetištu i zavjesi: sto podnožja od sto  talenata - talenat za podnožje. 
\par 28 A od tisuću sedam stotina  sedamdeset i pet šekela načinio je kuke za stupove, obložio njihove  vrhove i napravio šipke za njih. 
\par 29 Tuč od žrtve prikaznice  iznosio je sedamdeset talenata i dvije tisuće četiri stotine  šekela. 
\par 30 Od njega je načinio: podnožja za ulaz u Šator sastanka, žrtvenik od tuča s njegovom tučanom rešetkom i sav pribor za  žrtvenik; 
\par 31 dalje, podnožja oko dvorišta, podnožja za dvorišni  ulaz; sve kočiće za Prebivalište i sve kočiće oko dvorišta. 


\chapter{39}

\par 1 Od ljubičastog, crvenog i tamnocrvenog prediva naprave lijepo  izrađeno ruho za službu u Svetištu; naprave svetu odjeću Aronu, kako je Jahve naredio Mojsiju. 
\par 2 Oplećak naprave od zlata, ljubičastog, crvenog i tamnocrvenog  prediva i prepredenog lana. 
\par 3 Skuju zlatne pločice, a onda ih  na niti izrežu da ih vještački uvezu u ljubičasto, crveno i tamnocrveno  predivo i prepredeni lan. 
\par 4 Za oplećak naprave poramenice koje  su bile s njim sastavljene na svoja dva kraja; 
\par 5 tkanica što  je na njemu stajala bila je napravljena od zlata, ljubičastog, crvenog i tamnocrvenog prediva i prepredenog lana kao i on,  i u jednome komadu s njim, kako je Jahve naredio Mojsiju. 
\par 6 Kamenje  oniksa optoče obrubom od zlata. Na njima su, kao što se režu  pečati, bila urezana imena Izraelovih sinova. 
\par 7 Njih stave na  poramenice oplećka da budu spomen-kamenje sinovima Izraelovim, kako je Jahve naredio Mojsiju. 
\par 8 I naprsnik izrade radovima vještaka kao i oplećak: od  zlata, ljubičastog, crvenog i tamnocrvenog prediva i prepredenog  lana. 
\par 9 Naprsnik načiniše četverouglast, dvostruk; bio je pedalj  dug, pedalj širok, a predvostručen. 
\par 10 Umetnu u nj četiri reda  dragulja. Prvi red bijaše od rubina, topaza i alema; 
\par 11 drugi  red od smaragda, safira i ametista; 
\par 12 treći red od hijacinta, ahata i leca; 
\par 13 a četvrti red od krizolita, oniksa i jaspisa.  Sve je bilo zlatom obrubljeno. 
\par 14 Na kamenima su bila imena  Izraelovih sinova. Na broj ih je bilo dvanaest, kao i njihovih  imena. Bila su urezana kao i pečati - svaki kamen s imenom jednoga  od dvanaest plemena. 
\par 15 Za naprsnik naprave lančiće od čistoga  zlata kao zasukane uzice. 
\par 16 Naprave zatim dva zlatna okvira  i dva zlatna kolutića pa pričvrste oba kolutića za dva gornja  ugla naprsnika. 
\par 17 Sad privežu ovdje zasukane uzice od zlata  za dva kolutića što su bila pričvšćena za uglove naprsnika. 
\par 18 Druga  dva kraja zasukanih uzica pričvrste za dva okvira. Tako ih povežu  za poramenice oplećka sprijeda. 
\par 19 Potom načine dva zlatna kolutića  pa ih pričvrste za dva kraja naprsnika uz nutarnji rub, okrenut  prema oplećku. 
\par 20 Još naprave dva zlatna kolutića te ih pričvrste  za donji, prednji kraj poramenice oplećka, pokraj mjesta gdje  se veže, povrh tkanice oplećka. 
\par 21 Svežu kolutiće naprsnika  s kolutićima oplećka modrom vrpcom, tako da naprsnik stoji nad  tkanicom oplećka i da se s oplećka ne mogne odvojiti, kako je  Jahve Mojsiju naredio. 
\par 22 Naprave i ogrtač za oplećak, sav satkan od ljubičastog  prediva. 
\par 23 U sredini je ogrtača bio prorez kao otvor na oklopu, prorez naokolo opšiven, da se ogrtač ne podere. 
\par 24 O donjem  rubu ogrtača načine šipke od ljubičastog, crvenog i tamnocrvenog  prediva i prepredenog lana. 
\par 25 A načine i zvonca od čistog zlata, pa zvonca privežu među šipke; sve naokolo donjeg ruba ogrtača  između šipaka: 
\par 26 zvonce pa šipak, zvonce pa šipak okolo donjeg  ruba ogrtača za vršenje službe, kako je Jahve naredio Mojsiju. 
\par 27 Zatim od otkanog lana načine košulje Aronu i njegovim  sinovima; 
\par 28 a naprave i mitru od lana i kape od lana; platnene  gaće načine od prepredenog lana. 
\par 29 I pasovi su bili od prepredenog  lana i od ljubičastog, crvenog i tamnocrvenog prediva, iglama  izvezeni, kako je Jahve Mojsiju naredio. 
\par 30 Načine i ploču, sveti vijenac, od čistoga zlata i na  njoj urežu natpis kako se urezuje na pečatnome prstenu: "Posvećen  Jahvi." 
\par 31 Za nju privežu modru vrpcu da je mogu svezati na  vrhu mitre, kako je Jahve naredio Mojsiju. 
\par 32 Tako su bili završeni svi radovi na Prebivalištu, Šatoru  sastanka. Izraelci su sve načinili onako kako je Jahve Mojsiju  naredio da načine. 
\par 33 Onda donesu Mojsiju Prebivalište, Šator i sav njegov  pribor: njegove kuke, njegove trenice, njegove priječnice, njegove  stupove i njegova podnožja; 
\par 34 pokrov od učinjenih ovnujskih  koža, pokrov od finih koža, zavjesu za zaklon; 
\par 35 Kovčeg svjedočanstva  s njegovim motkama i Pomirilištem; 
\par 36 stol i sav njegov pribor, prinesene hljebove, 
\par 37 svijećnjak od čistoga zlata s njegovim  svijećama - svijeće već u red stavljene - i sav njegov pribor  i ulje za svjetlo; 
\par 38 zlatni žrtvenik, ulje za pomazanje, miomirisni  tamjan i zavjesu za ulaz Šatora; 
\par 39 žrtvenik od tuča s tučanom  rešetkom; njegove motke i sav njegov pribor; umivaonik i njegov  stalak; 
\par 40 zavjese za dvorište; njihove stupove i njihova podnožja, zavjesu za dvorišni ulaz, njegova užeta i njihove kočiće - sav  pribor za službu u Prebivalištu, za Šator sastanka; 
\par 41 lijepo  izrađeno ruho za službu u Svetištu - svetu odjeću za svećenika  Arona i odijela za svećeničku službu njegovih sinova. 
\par 42 Upravo  kako je Jahve Mojsiju naredio, tako su Izraelci sav posao obavili. 
\par 43 Mojsije pregleda sve radove i utvrdi da su ih dovršili:  kako je Jahve naredio, onako su ih i napravili. I Mojsije ih  blagoslovi. 



\chapter{40}

\par 1 Reče Jahve Mojsiju: 
\par 2 "Na prvi dan prvoga mjeseca podigni  Prebivalište, Šator sastanka. 
\par 3 Ondje postavi Kovčeg svjedočanstva, onda Kovčeg zakloni zavjesom. 
\par 4 Potom unesi stol i što na nj  spada poredaj; unesi i svijećnjak i svijeće mu pripremi. 
\par 5 A  zlatni žrtvenik za kađenje postavi pred Kovčeg svjedočanstva.  Onda objesi zastor nad ulazom u Prebivalište. 
\par 6 Stavi žrtvenik  za žrtve paljenice pred ulaz Prebivališta, Šatora sastanka. 
\par 7 Između  Šatora sastanka i žrtvenika smjesti umivaonik i u nj nalij vode. 
\par 8 Naokolo napravi dvorište i objesi zastor nad dvorišnim ulazom. 
\par 9 Zatim uzmi ulja za pomazanje pa pomaži Prebivalište i sve  što je u njemu; posveti ga i sav njegov pribor, pa će svetim  postati. 
\par 10 Pomaži potom žrtvenik za žrtve paljenice i sav njegov  pribor; posveti žrtvenik i presvetim će žrtvenik postati. 
\par 11 Pomaži  umivaonik s njegovim stalkom: posveti ga! 
\par 12 Dovedi zatim Arona  i njegove sinove na ulaz Šatora sastanka pa ih operi vodom. 
\par 13 Stavi  onda na Arona posvećenu odjeću; pomaži ga i posveti da mi služi  kao svećenik. 
\par 14 Dovedi i njegove sinove, na njih stavi košulje 
\par 15 i pomaži ih, kako si pomazao i njihova oca, da mi služe kao  svećenici. Njihovo pomazanje neka ih uvede u vječno svećenstvo  u sve njihove naraštaje." 
\par 16 Tako Mojsije učini. Kako mu je Jahve naredio, sve je  tako i učinio. 
\par 17 Prvoga dana prvoga mjeseca druge godine Prebivalište  bi podignuto. 
\par 18 Ovako Mojsije namjesti Prebivalište: razmjesti  njegova podnožja, onda uspravi njegove trenice, zatim postavi  priječnice i podiže stupove. 
\par 19 Zatim raspne Šator nad Prebivalište, a povrh njega stavi pokrov Šatora, kako je Jahve Mojsiju naredio. 
\par 20 Uze onda Svjedočanstvo i stavi ga u Kovčeg; na Kovčeg postavi  motke; onda stavi Pomirilište ozgo na Kovčeg. 
\par 21 Potom unese  Kovčeg u Prebivalište; objesi zavjesu za zaklon. Tako zastre  Kovčeg svjedočanstva, kako je Jahve i naredio Mojsiju. 
\par 22 Zatim  postavi stol u Šator sastanka, Prebivalištu sa sjeverne strane, ali izvan zavjese. 
\par 23 Po njemu poreda kruhove pred Jahvom,  kako je Jahve naredio Mojsiju. 
\par 24 Onda smjesti svijećnjak u  Šator sastanka naprama stolu, na južnoj strani Prebivališta. 
\par 25 I postavi svjetiljke pred Jahvom, kako je Jahve naredio Mojsiju. 
\par 26 Zlatni žrtvenik smjesti u Šator sastanka, pred zavjesu. 
\par 27 Na  njemu zapali miomirisnog tamjana, kako je Jahve naredio Mojsiju. 
\par 28 Poslije toga stavi zavjesu na ulaz u Prebivalište. 
\par 29 Kod  ulaza u Prebivalište, u Šator sastanka, postavi žrtvenik za žrtve  paljenice. Na njemu prinese žrtvu paljenicu i žrtvu od žita,  kako je Jahve naredio Mojsiju. 
\par 30 Između Šatora sastanka i žrtvenika  smjesti umivaonik pa u nj ulije vode za pranje. 
\par 31 Iz njega  su Mojsije, Aron i njegovi sinovi prali svoje ruke i svoje noge. 
\par 32 A prali su se kad su ulazili u Šator sastanka i kad su pristupali  k žrtveniku, kako je Jahve naredio Mojsiju. 
\par 33 Napokon Mojsije  napravi dvorište oko Prebivališta i žrtvenika i postavi zavjesu  na dvorišnim vratima. Tako Mojsije završi taj posao. 
\par 34 A onda oblak prekri Šator sastanka i slava Jahvina ispuni  Prebivalište. 
\par 35 Mojsije nije mogao ući u Šator sastanka zbog  oblaka koji je na njemu stajao i slave Jahvine koja je ispunjala  Prebivalište. 
\par 36 Sve vrijeme njihova putovanja, kad god bi se oblak digao  s Prebivališta, Izraelci bi krenuli; 
\par 37 ali ako se oblak ne  bi digao, ni oni ne bi na put polazili sve do dana dok se ne  bi digao. 
\par 38 Jer sve vrijeme njihova putovanja oblak Jahvin  danju stajaše nad Prebivalištem, a noću bi se u oblaku pojavila  vatra vidljiva svemu domu Izraelovu. 




\end{document}