\begin{document}

\title{1 Samuelova}


\chapter{1}

\par 1 Bio jedan čovjek iz Ramatajima, Sufovac iz Efrajimove gore, po imenu Elkana, sin Jerohama, sina Elihua, sina Tohua, sina  Sufova, Efrajimljanin. 
\par 2 Imao je dvije žene: ime jednoj bijaše  Ana, a drugoj bijaše ime Penina. Penina je imala djece, a Ana  ih nije imala. 
\par 3 Taj je čovjek svake godine uzlazio iz svoga  grada da se pokloni i prinese žrtvu Jahvi Sebaotu u Šilu. Ondje  su bila dva sina Elijeva, Hofni i Pinhas, kao svećenici Jahvini. 
\par 4 Jednoga dana Elkana prinese žrtvu. On je obično svojoj  ženi Penini i svim njezinim sinovima i kćerima davao više žrtvenih  dijelova, 
\par 5 a Ani je davao samo jedan dio, premda je više ljubio  Anu, ali Jahve joj ne bijaše dao od srca poroda. 
\par 6 Uz to joj  je suparnica njezina zanovijetala da je ponizi što joj Jahve  ne bijaše dao od srca poroda. 
\par 7 Tako je bivalo svake godine  kad god bi polazili u Dom Jahvin: Penina je zanovijetala Ani.  Ana je stoga plakala i nije htjela jesti. 
\par 8 Tada joj reče Elkana, njezin muž: "Zašto plačeš, Ana? I zašto ne jedeš? Zašto ti je  srce rastuženo? Nisam li ti ja vredniji nego deset sinova?" 
\par 9 Ali Ana ustade, pošto su jeli i pili u sobi, i stupi pred  Jahvu - a svećenik Eli sjeđaše na stolici na pragu svetišta Jahvina. 
\par 10 I ojađena u duši pomoli se Ana Jahvi, plačući gorko. 
\par 11 I  zavjetova se ovako: "Jahve Sebaote! Ako pogledaš na nevolju službenice  svoje i opomeneš se mene i ne zaboraviš službenice svoje te dadeš  službenici svojoj muško čedo, ja ću ga darovati Jahvi za sve  dane njegova života i britva neće prijeći preko glave njegove." 
\par 12 Tako se ona dugo molila pred Jahvom, a Eli je motrio  usta njezina. 
\par 13 Ana govoraše u srcu; samo se usne njezine micahu, a glas joj se nije čuo. Zato Eli pomisli da je pijana. 
\par 14 I  reče joj Eli: "Dokle ćeš biti pijana? Otrijezni se od vina što  je u tebi!" 
\par 15 Ali Ana odgovori i reče: "Nisam pijana, gospodaru, nego  sam velika nesretnica. Nisam pila ni vina ni opojna pića nego  izlijevam dušu svoju pred Jahvom. 
\par 16 Ne sudi službenicu svoju  kao ženu nevaljalu, jer sam od preteške tuge i žalosti tako dugo  molila." 
\par 17 Tada joj Eli odgovori ovako: "Pođi u miru! A Bog Izraelov  neka ti ispuni molitvu kojom si ga molila." 
\par 18 A ona reče: "Neka službenica tvoja nađe milost u očima  tvojim!" I žena ode svojim putem: jela je i lice joj nije više  bilo tužno kao i prije. 
\par 19 Sutradan uraniše i pokloniše se Jahvi, a onda se vratiše  i dođoše svojoj kući u Ramu. Elkana pozna Anu, ženu svoju, a  Jahve je se opomenu. 
\par 20 Ana zatrudnje i, kad bi vrijeme, rodi  sina koga nazva imenom Samuel, "jer sam ga", reče, "izmolila  od Jahve". 
\par 21 Poslije godine dana uziđe njezin muž Elkana sa svim domom  svojim da prinese Jahvi godišnju žrtvu i da izvrši zavjet. 
\par 22 Ali  Ana ne pođe s njim jer reče svome mužu: "Neću poći dok se dijete  ne odbije od prsiju, a onda ću ga odvesti da se pokaže pred Jahvom  i da ostane ondje zauvijek." 
\par 23 I odgovori joj Elkana, njezin muž: "Čini kako misliš  da je dobro; ostani dok ga ne odbiješ od prsiju; samo neka ti  Jahve ispuni tvoju želju!" I žena osta kod kuće dojeći sina svoga dok ga nije odbila  od prsiju. 
\par 24 Čim ga je odbila od prsiju, povede ga sa sobom uzevši  uz to trogodišnjeg junca, efu brašna i mijeh vina; i uvede ga  u Dom Jahvin u Šilu. A dječak je bio još vrlo mlad. 
\par 25 Tada  zaklaše junca, a majka dječakova pristupi k Eliju. 
\par 26 I reče  Ana: "Dopusti, gospodaru! Tako ti života tvoga, gospodaru, ja  sam ona žena koja je stajala ovdje kraj tebe moleći se Jahvi. 
\par 27 Molila sam za ovo dijete, i Jahve mi je uslišio prošnju kojom  sam ga prosila. 
\par 28 Zato i ja njega ustupam Jahvi za sve dane  njegova života: ta isprošen je od Jahve." I pokloniše se ondje  Jahvi. 


\chapter{2}

\par 1 Nato se Ana pomoli ovako: "Kliče srce moje u Jahvi, raste snaga moja po Bogu mom. Šire mi se usta na dušmane moje, jer se radujem pomoći tvojoj. 
\par 2 Nitko nije svet kao što je Jahve (jer nema nikoga osim tebe), i nema hridi kao što Bog je naš. 
\par 3 Ne govorite mnogo hvastavih riječi, neka ne izlazi drskost iz usta vaših, jer Jahve je sveznajući Bog, pravo on prosuđuje djela. 
\par 4 Lomi se luk junacima, nemoćni se snagom opasuju. 
\par 5 Nekoć siti sad se za kruh muče, nekoć gladni ne gladuju više. Nerotkinja rađa sedam puta, majka brojne djece svježinu izgubi. 
\par 6 Jahve daje smrt i život, ruši u Šeol i odande diže. 
\par 7 Jahve čini uboga i bogata, obara čovjeka i uzvisuje. 
\par 8 Diže slabića iz prašine, iz bunjišta izvlači uboga, da ih posadi s knezovima i da im odredi počasna mjesta. Jer Jahvini su stupovi zemlje, na njih je stavio ovaj svijet. 
\par 9 Korake čuva svojih vjernika, zlikovce stiže propast u mraku (svojom snagom čovjek ne stječe pobjede). 
\par 10 Koji se protive Jahvi, padaju, Svevišnji grmi s nebesa. Jahve sudi međama zemlje, daje silu svojemu kralju, uzdiže snagu pomazanika svoga." 
\par 11 Potom se Ana vrati u Ramu, a dječak ostade da služi Jahvi  pod okom svećenika Elija. 
\par 12 A Elijevi sinovi bijahu nevaljali ljudi jer nisu marili  za Jahvu 
\par 13 ni za prava svećenika nasuprot narodu: kad bi tko  prinosio žrtvu, došao bi sluga svećenikov, dok se meso još kuhalo, s trorogom vilicom u ruci 
\par 14 i zabadao njom u kotlić ili u  lonac, u tavu ili u zdjelu, i što god bi se nabolo na vilicu, uzimao je svećenik sebi. Tako su činili svim Izraelcima što  su dolazili onamo, u Šilo. 
\par 15 Tako i prije nego bi se spalilo  salo, došao bi sluga svećenikov i rekao čovjeku koji je prinosio  žrtvu: "Daj mi mesa da ispečem svećeniku! On neće od tebe kuhana  mesa nego samo sirovo." 
\par 16 Ako bi mu čovjek tada rekao: "Neka  se najprije spali salo, a onda uzmi što ti duša želi", on bi  odgovorio: "Ne, nego daj odmah! Ako ne daš, uzet ću silom." 
\par 17 Grijeh  je mladića bio vrlo velik pred Jahvom, jer su ljudi prezirali  žrtvu koja se prinosila Jahvi. 
\par 18 A Samuel služaše pred Jahvom, još dijete u oplećku lanenom. 
\par 19 Mati bi mu njegova napravila dolamicu i donosila mu je svake  godine kad bi dolazila s mužem svojim da prinese godišnju žrtvu. 
\par 20 A Eli bi blagoslovio Elkanu i njegovu ženu govoreći: "Neka  ti Jahve dade poroda od te žene na uzdarje za dar što ga je dala  Jahvi." Nato bi se vraćali svojoj kući. 
\par 21 Jahve pohodi Anu  i ona zatrudnje i rodi još tri sina i dvije kćeri. A mladi je  Samuel rastao pred Jahvom. 
\par 22 Eli je bio već vrlo star, ali je ipak čuo sve što su  njegovi sinovi činili svemu Izraelu. 
\par 23 I on im reče: "Zašto  radite takvo što da o tome moram slušati od svega ovog naroda? 
\par 24 Nemojte tako, sinovi moji! Nisu dobri glasovi što ih čujem  ... Sablažnjujete narod Jahvin. 
\par 25 Ako čovjek zgriješi čovjeku, Bog će prosuditi. Ali ako čovjek zgriješi Jahvi, tko će se zauzeti  za njega?" Ali sinovi ne poslušaše glasa oca svojega, jer je  Jahve odlučio da ih pogubi. 
\par 26 A mladi je Samuel sve više rastao u dobi i mudrosti i  pred Jahvom i pred ljudima. 
\par 27 Uto dođe jedan Božji čovjek k Eliju i reče mu: "Ovako  govori Jahve: 'Nisam li se jasno objavio domu oca tvojega kad  su bili u Egiptu, robovi u kući faraonovoj? 
\par 28 Odabrao sam ih  između svih plemena Izraelovih da mi budu svećenici, da se uspinju  na moj žrtvenik, da prinose žrtve paljenice i da nose oplećak  preda mnom: i dao sam domu oca tvojega sve paljene žrtve sinova  Izraelovih. 
\par 29 Zašto gledaš zavidnim okom žrtvu i prinos što  sam ih odredio za svoj Dom? I zašto paziš sinove svoje više nego  mene, toveći ih najboljim dijelovima svih žrtvenih prinosa naroda  moga Izraela? 
\par 30 Zato sam - riječ je Jahve, Boga Izraelova -  rekao doduše da će dom tvoj i dom oca tvojega stupati preda mnom  dovijeka, ali sada - riječ je Jahvina - neka je to daleko od  mene! Jer ja častim one koji mene časte, a koji mene preziru, bit će osramoćeni. 
\par 31 Gle, dolaze dani kad ću odsjeći mišicu  tvoju i mišicu doma oca tvojega, tako da više neće biti starca  u tvom domu. 
\par 32 Ti ćeš kivnim okom gledati na sve dobro kojim  ću obasuti Izraela, i nikada više neće biti starca u tvom domu. 
\par 33 Zadržat ću ipak nekoga od tvojih kod oltara svoga, samo zato  da mu sahnu oči i vene duša njegova, ali sve mnoštvo doma tvoga  poginut će od ljudskoga mača. 
\par 34 Znak će ti biti ono što će  stići oba tvoja sina, Hofnija i Pinhasa: obojica će poginuti  istoga dana. 
\par 35 Ja ću sebi podići vjerna svećenika koji će raditi  po mom srcu i po mojoj želji i njemu ću sazdati trajan dom, i  on će svagda stupati pred pomazanikom mojim. 
\par 36 A koji god ostane  od tvoga doma, dolazit će da mu se pokloni i da izmoli srebrn  novčić ili hljeb kruha i kazat će: 'Molim te, primi me u kakvu  god službu svećeničku, da imam zalogaj kruha.'" 


\chapter{3}

\par 1 Mladi je Samuel služio Jahvi pod nadzorom Elijevim; u ono vrijeme  Jahve je izrijetka govorio ljudima, a viđenja nisu bila česta. 
\par 2 No jednoga je dana Eli ležao u svojoj sobi - oči su njegove  počele slabiti te više nije mogao vidjeti - 
\par 3 svijećnjak Božji  još ne bijaše ugašen i Samuel je spavao u svetištu Jahvinu, ondje  gdje je bio Kovčeg Božji. 
\par 4 I Jahve zovnu: "Samuele! Samuele!"  A on odgovori: "Evo me!" 
\par 5 I otrča k Eliju i reče: "Evo me!  Ti si me zvao!" A Eli reče: "Ja te nisam zvao. Vrati se i spavaj!"  I on ode i leže. 
\par 6 I Jahve opet zovnu: "Samuele! Samuele!" Samuel usta, ode  k Eliju i reče: "Evo me! Ti si me zvao!" A Eli odgovori: "Ja  te nisam zvao, sine! Vrati se i spavaj!" 
\par 7 Samuel još nije poznavao  Jahve i još mu nikada ne bijaše objavljena riječ Jahvina. 
\par 8 I Jahve zovnu Samuela po treći put. On usta, ode k Eliju  i reče: "Evo me! Ti si me zvao!" Sada Eli razumje da je Jahve  zvao dječaka. 
\par 9 Zato reče Samuelu: "Idi i lezi; a ako te zovne, ti reci: 'Govori, sluga tvoj sluša.'" I Samuel ode i leže na  svoje mjesto. 
\par 10 I dođe Jahve i stade i zovnu kao prije: "Samuele! Samuele!"  A Samuel odgovori: "Govori, sluga tvoj sluša." 
\par 11 Tada Jahve reče Samuelu: "Evo, učinit ću nešto u Izraelu  da će oba uha zujati svakome koji čuje. 
\par 12 U onaj ću dan ispuniti  na Eliju sve što sam rekao za kuću njegovu, od početka do kraja. 
\par 13 Ti ćeš mu objaviti da osuđujem kuću njegovu dovijeka; on  je znao da njegovi sinovi hule na Boga, a nije ih obuzdao. 
\par 14 Zato  - kunem se domu Elijevu - neće oprati krivicu Elijeva doma nikakve  žrtve ni prinosi dovijeka." 
\par 15 Samuel je spavao do jutra, a onda otvori vrata Doma Jahvina.  Samuel se bojao kazati viđenje Eliju. 
\par 16 Ali Eli zovnu Samuela  govoreći: "Samuele, sine!" A on odgovori: "Evo me!" 
\par 17 I on  upita: "Kakva je riječ koju ti reče? Nemoj mi zatajiti ništa!  Tako ti Bog učinio zlo i dodao ti drugo ako mi zatajiš nešto  od onoga što ti je kazao." 
\par 18 Nato mu Samuel pripovjedi sve  i ništa ne zataji od njega. A Eli reče: "On je Jahve, neka čini  što je dobro u očima njegovim!" 
\par 19 Samuel je rastao, a Jahve je bio s njim i nije pustio  da ijedna od njegovih riječi padne na zemlju. 
\par 20 Sav Izrael, od Dana do Beer Šebe, spozna da je Samuel postavljen za proroka  Jahvina. 
\par 21 Jahve se i dalje javljao u Šilu, jer se objavljivao  Samuelu, [4:1] i riječ se Samuelova obraćala svemu Izraelu. (Eli  je bio vrlo star, a njegovi su sinovi ustrajali u svome opakom  postupku pred Jahvom.) 


\chapter{4}

\par 1 U ono vrijeme skupiše Filistejci vojsku protiv Izraela. Izraelci  iziđoše pred njih da se pobiju i utaboriše se kod Eben Haezera, dok su Filistejci udarili tabor kod Afeka. 
\par 2 Filistejci se  svrstaše u bojni red protiv Izraela i nasta žestoka bitka. Izrael  podleže Filistejcima: oko četiri tisuće ljudi pogibe na bojištu, na otvorenu polju. 
\par 3 Kad se narod vratio u tabor, rekoše starješine Izraelove:  "Zašto je Jahve dopustio da nas Filistejci danas pobijede? Pođimo  u Šilo po Kovčeg saveza Jahvina neka dođe u našu sredinu i spasi  nas iz ruku naših neprijatelja." 
\par 4 Narod posla ljude u Šilo i donesoše odande Kovčeg saveza  Jahve nad vojskama, koji stoluje nad kerubinima; oba sina Elijeva, Hofni i Pinhas, dođoše kao pratioci Kovčega. 
\par 5 Kad je Kovčeg  Jahvin stigao u tabor, sav Izrael podiže gromki poklik, od kojega  odjeknu zemlja. 
\par 6 Filistejci čuše taj gromki poklik i zapitaše: "Što znači  taj gromki poklik u taboru Hebreja?" I shvatiše da je Kovčeg  Jahvin stigao u njihov tabor. 
\par 7 Tada Filistejce obuze strah  jer su govorili: "Bog je došao u tabor!" I povikaše: "Jao nama!  Tako nije bilo dosad. 
\par 8 Jao nama! Tko će nas izbaviti iz ruke  toga silnog Boga? To je onaj koji je udario Egipat svakojakim  nevoljama. 
\par 9 Ohrabrite se i budite junaci, Filistejci, da ne  postanete robovi Hebrejima kao što su oni bili robovi vama; budite  junaci i borite se!" 
\par 10 Tada Filistejci zametnuše bitku, Izraelci biše potučeni  i pobjegoše svaki u svoj šator. Poraz je bio silan, jer je trideset  tisuća pješaka poginulo na izraelskoj strani. 
\par 11 I Kovčeg Božji  bi otet, i oba sina Elijeva poginuše, Hofni i Pinhas. 
\par 12 Jedan Benjaminovac otrča iz bojnih redova i stiže u Šilo  još istoga dana, razderanih haljina i glave posute prašinom. 
\par 13 Kad je stigao, Eli je sjedio na svojoj stolici, pokraj vrata, pazeći na cestu, jer mu je srce strepilo za Kovčeg Božji. Taj  dakle čovjek dođe da gradu donese glas, i nasta silna vika po  svem gradu. 
\par 14 Kad je Eli čuo viku, upita: "Kakva je to velika  vika?" Čovjek se požuri i dođe da obavijesti Elija. - 
\par 15 A Eliju  bijaše devedeset i osam godina, oči mu bijahu ukočene te ništa  više nije vidio. - 
\par 16 Čovjek reče Eliju: "Dolazim s bojišta, danas sam utekao iz boja." Tada starac zapita: "Što se dogodilo, sine?" 
\par 17 Glasnik odgovori: "Izrael je pobjegao pred Filistejcima, bio je to težak poraz za narod i još su oba tvoja sina poginula  i Kovčeg je Božji otet!" 
\par 18 Kad je spomenuo Kovčeg Božji, pade Eli sa stolice nauznak  kraj vrata, slomi vrat i umrije, jer je bio star čovjek i težak.  Bio je sudac u Izraelu četrdeset godina. 
\par 19 Njegova snaha, žena Pinhasova, bijaše trudna i pred porodom.  Kad je čula vijest da je otet Kovčeg Božji i da je umro njezin  svekar i poginuo njezin muž, savila se i rodila jer su je najednom  uhvatili trudovi. 
\par 20 Kako je bila na samrti, rekoše joj žene  koje stajahu oko nje: "Budi bez brige jer si rodila sina!" Ali  ona ne odgovori niti obrati misli na to. 
\par 21 Djetetu nadjenu  ime Ikabod govoreći: "Otišla je slava od Izraela." Time je mislila  na oteti Kovčeg Božji i na svoga svekra i svoga muža. 
\par 22 Zato  reče: "Otišla je slava od Izraela" jer je otet Kovčeg Božji. 


\chapter{5}

\par 1 Kad su Filistejci osvojili Kovčeg Božji, prenesoše ga iz Eben  Haezera u Ašdod. 
\par 2 Nato Filistejci uzeše Kovčeg Božji, unesoše  ga u hram Dagonov i smjestiše pokraj Dagona. 
\par 3 Sutradan ujutro, kad su žitelji Ašdoda došli u hram Dagonov, gle, Dagon ležaše  ničice na zemlji pred Kovčegom Jahvinim. Oni digoše Dagona i  metnuše ga natrag na njegovo mjesto. 
\par 4 Ali kad su ujutro uranili, gle, Dagon opet ležaše ničice na zemlji pred Kovčegom Jahvinim;  glava Dagona i obje njegove ruke ležahu odsječene na pragu: na  mjestu je stajao samo Dagonov trup. 
\par 5 Zato Dagonovi svećenici  i svi koji ulaze u Dagonov hram ne staju nogom na prag Dagonov  u Ašdodu sve do današnjeg dana. 
\par 6 Tada ruka Jahvina teško pritisnu žitelje Ašdoda i natjera  ih u silan strah: udari ih čirevima, Ašdod i njegovo područje. 
\par 7 Kad su ljudi u Ašdodu vidjeli što se dogodilo, rekoše: "Kovčeg  Boga Izraelova ne smije ostati kod nas jer se ruka njegova ispriječila  protiv nas i protiv našega boga Dagona." 
\par 8 Oni sazvaše i okupiše  sve knezove filistejske k sebi i rekoše: "Što da radimo s Kovčegom  Boga Izraelova?" A oni odgovoriše: "U Gat neka se prenese Kovčeg  Boga Izraelova." I prenesoše Kovčeg Boga Izraelova onamo. 
\par 9 Ali kad su ga prenijeli, ruka se Jahvina spusti na grad  i nasta silna strava: udari građane, od najmanjega do najvećega, tako da im se pojaviše čirevi. 
\par 10 Oni tada poslaše Kovčeg Božji  u Ekron. Ali kad je Kovčeg Božji stigao u Ekron, povikaše Ekronjani:  "Donesoše Kovčeg Boga Izraelova k meni da pomori mene i sav moj  narod!" 
\par 11 Zato sazvaše i okupiše sve knezove filistejske i  rekoše: "Pošaljite natrag Kovčeg Boga Izraelova, neka se vrati  na svoje mjesto da ne pomori mene i moj narod!" Jer vladaše smrtna  strava u svemu gradu, toliko ondje bijaše pritisnula ruka Božja. 
\par 12 Ljudi koji nisu pomrli bili su udareni čirevima i bolni se  vapaj grada dizao do neba. 


\chapter{6}

\par 1 Kovčeg Jahvin bijaše sedam mjeseci u zemlji Filistejaca. 
\par 2 Tada  Filistejci sazvaše svećenike i vrače i zapitaše ih: "Što da radimo  s Kovčegom Jahvinim? Poučite nas kako da ga pošaljemo natrag  na njegovo mjesto." 
\par 3 Oni odgovoriše: "Ako hoćete vratiti Kovčeg  Boga Izraelova, ne šaljite ga natrag prazna nego uza nj pošaljite  i naknadnicu. Tada ćete se izliječiti i znat ćete zašto se njegova  ruka nije okrenula od vas." 
\par 4 Oni zapitaše: "Kakvu naknadnicu  treba da mu pošaljemo?" Oni odgovoriše: "Prema broju filistejskih  knezova, pet zlatnih čireva i pet zlatnih štakora, jer je ista  nevolja na vama i na vašim knezovima. 
\par 5 Načinite, dakle, likove  svojih čireva i likove svojh štakora, koji vam zatiru zemlju, i dajte slavu Bogu Izraelovu. Možda će dignuti ruku svoju od  vas, od vaših bogova i od vaše zemlje. 
\par 6 Zašto hoćete da vam  srce otvrdne kao što je bilo otvrdnulo Egipćanima i faraonu?  Kad ih je Bog pritisnuo, nisu li ih onda pustili da odu? 
\par 7 Pripremite  sada jedna nova kola i uzmite dvije krave dojilice koje još nisu  nosile jarma: upregnite krave u kola, a njihovu telad odvedite  natrag u staju. 
\par 8 Tada ćete uzeti Kovčeg Jahvin i staviti ga  na kola. Zlatne predmete koje mu prinosite kao žrtvu naknadnicu  stavit ćete u kovčežić kraj njega i tako neka pođe. 
\par 9 Zatim  gledajte: ako krene prema svome kraju, put Bet Šemeša, onda je  sigurno da nam je on zadao ovo veliko zlo; ako li ne krene tako, znat ćemo da nas nije udarila njegova ruka, nego da nam se to  dogodilo slučajno." 
\par 10 Ljudi učiniše tako: uzeše dvije krave dojilice i upregoše  ih u kola, a njihovu telad zadržaše u staji. 
\par 11 Kovčeg Jahvin  staviše na kola i kovčežić sa zlatnim štakorima i s likovima  svojih čireva. 
\par 12 Krave udariše ravno cestom prema Bet Šemešu i jednako  su išle istim putem, mukale su idući, a nisu skretale ni desno  ni lijevo. Filistejski knezovi pratili su ih do granice Bet Šemeša. 
\par 13 Stanovnici Bet Šemeša upravo su bili zabavljeni žetvom  pšenice u dolini. Digavši oči, ugledaše Kovčeg i potrčaše mu  s veseljem u susret. 
\par 14 Kad su kola stigla na polje Jošue iz  Bet Šemeša, zaustaviše se. Ondje bijaše velik kamen. Tada iscijepaše  drvo od kola i prinesoše krave kao žrtvu paljenicu Jahvi. 
\par 15 Leviti  bijahu skinuli Kovčeg Jahvin i kovčežić što je bio kraj njega  i u kojem su bili zlatni predmeti i sve bijahu stavili na onaj  veliki kamen. Stanovnici Bet Šemeša prinosili su toga dana žrtve  paljenice i klali žrtve klanice Jahvi. 
\par 16 Kad je to vidjelo  pet filistejskih knezova, vratiše se u Ekron isti dan. 
\par 17 A  ovo je pet zlatnih čireva što su ih Filistejci poslali kao žrtvu  naknadnicu Jahvi: za Ašdod jedan, za Gazu jedan, za Aškelon jedan, za Gat jedan, za Ekron jedan. 
\par 18 A zlatnih je štakora bilo  toliko koliko svih gradova filistejskih, u svih pet kneževina, od utvrđenih gradova do otvorenih sela. Svjedok je veliki kamen  na koji su položili Kovčeg Jahvin i koji još i danas stoji na  polju Jošue iz Bet Šemeša. 
\par 19 Sinovi Jekonijini nisu se radovali  sa stanovnicima Bet Šemeša kad su vidjeli Kovčeg Jahvin. Zato  je Jahve pobio sedamdeset ljudi među njima. Narod je tugovao  zbog toga što ga je Jahve tako teško iskušao. 
\par 20 Tada ljudi u Bet Šemešu rekoše: "Tko bi mogao opstati  pred Jahvom, ovim Svetim Bogom? Kome će otići sada od nas?" 
\par 21 I  poslaše poslanike stanovnicima Kirjat Jearima i poručiše im:  "Filistejci su vratili Kovčeg Jahvin. Dođite i odnesite ga sebi." 


\chapter{7}

\par 1 Tada dođoše ljudi iz Kirjat Jearima i odnesoše Kovčeg Jahvin  sebi. Unesoše ga u kuću Abinadabovu, na uzvišici, i posvetiše  njegova sina Eleazara da čuva Kovčeg Jahvin. 
\par 2 Od dana kad je Kovčeg bio postavljen u Kirjat Jearimu, prođe mnogo vremena - dvadeset godina - i sav je dom Izraelov  uzdisao za Jahvom. 
\par 3 Tada Samuel progovori svemu domu Izraelovu  ovako: "Ako se od svega srca svoga vraćate Jahvi, uklonite iz  svoje sredine tuđe bogove, baale i aštarte, i upravite srce svoje  Jahvi i njemu jedinome služite. Tada će vas on izbaviti iz ruke  Filistejaca." 
\par 4 Sinovi Izraelovi ukloniše nato baale i aštarte  i služahu jedinome Jahvi. 
\par 5 Samuel tada zapovjedi: "Skupite sve sinove Izraelove u  Mispu da se pomolim Jahvi za vas." 
\par 6 Oni se dakle skupiše u  Mispi; ondje su grabili vodu i izlijevali je pred Jahvom. I postili  su onaj dan i priznavali: "Sagriješili smo Jahvi!" I Samuel je  sudio sinovima Izraelovim u Mispi. 
\par 7 Kad su Filistejci čuli da su se sinovi Izraelovi skupili  u Mispi, krenu filistejski knezovi da napadnu na Izraela. Kad  to vidješe sinovi Izraelovi, uplašiše se Filistejaca. 
\par 8 I zamoliše  sinovi Izraelovi Samuela: "Ne prestaj vapiti za nas Jahvi, Bogu  našemu, da nas izbavi iz ruke Filistejaca." 
\par 9 Samuel uze jedno janje sisanče i prinese ga Jahvi kao  žrtvu paljenicu i glasno se pomoli Jahvi za Izraela, i Jahve  ga usliša. 
\par 10 Dok je Samuel prinosio žrtvu paljenicu, Filistejci  su došli da udare na Izraela, ali Jahve toga dana zagrmi silnom  grmljavinom na Filistejce i tako ih prestraši i smete da su podlegli  Izraelu. 
\par 11 Ratnici izraelski iziđoše iz Mispe i potjeraše Filistejce, tukući ih sve do ispod Bet Kara. 
\par 12 A Samuel uze jedan kamen  i postavi ga između Mispe i Ješane i nazva ga imenom Eben Haezer  govoreći: "Dovde nam je Jahve pomogao." 
\par 13 Tako su Filistejci bili poniženi i nikada više ne navališe  na zemlju Izraelovu, a ruka je Jahvina pritiskivala Filistejce  svega vijeka Samuelova. 
\par 14 I gradove koje Filistejci bijahu  zauzeli od Izraela vratiše se njemu, od Ekrona do Gata, i Izrael  oslobodi njihovo područje iz ruke filistejske. I bio je mir između  Izraela i Amorejaca. 
\par 15 Samuel je bio sudac u Izraelu svega svoga vijeka. 
\par 16 Svake  je godine obilazio Betel, Gilgal i Mispu i u svim je tim mjestima  sudio Izraelu. 
\par 17 Zatim se vraćao u Ramu, jer je ondje imao  svoju kuću i ondje je sudio Izraelu. Ondje je podigao i žrtvenik  Jahvi. 


\chapter{8}

\par 1 Kad je Samuel ostario, postavio je svoje sinove za suce u Izraelu. 
\par 2 Njegov prvorođenac zvao se Joel, a drugi sin Abija; oni su  bili suci u Beer Šebi. 
\par 3 Ali sinovi nisu išli stopama očevim:  gledali su na svoj dobitak, primali mito i izvrtali pravicu. 
\par 4 Tada se skupiše sve starješine izraelske i dođoše k Samuelu  u Ramu. 
\par 5 I rekoše mu: "Eto, ti si ostario, a tvoji sinovi ne  idu tvojim stopama. Postavi nam, dakle, kralja da nam vlada,  kao što je to kod svih naroda." 
\par 6 Ali Samuelu nije bilo drago  što su rekli: "Daj nam kralja da nam vlada!" Zato se Samuel pomoli  Jahvi. 
\par 7 A Jahve reče Samuelu: "Poslušaj glas naroda u svemu što  od tebe traži, jer nisu odbacili tebe, nego su odbacili mene, ne želeći da ja kraljujem nad njima. 
\par 8 Sve što su činili meni  od onoga dana kad sam ih izveo iz Egipta pa do današnjega dana  - ostavili su mene i služili tuđim bogovima - tako oni čine i  tebi. 
\par 9 Sada, dakle, poslušaj njihov zahtjev, ali ih svečano  opomeni i pouči o pravima kralja koji će vladati nad njima." 
\par 10 Samuel ponovi sve Jahvine riječi narodu koji je od njega  tražio kralja. 
\par 11 I reče: "Ovo će biti pravo kralja koji će  kraljevati nad vama: uzimat će vaše sinove da mu služe kod bojnih  kola i kod konja i oni će trčati pred njegovim bojnim kolima. 
\par 12 Postavljat će ih za tisućnike i pedesetnike; orat će oni  njegovu zemlju, žeti njegovu žetvu, izrađivati mu bojno oružje  i opremu za njegova bojna kola. 
\par 13 Uzimat će kralj vaše kćeri  da mu priređuju mirisne pomasti, da mu kuhaju i peku. 
\par 14 Uzimat  će najbolja vaša polja, vaše vinograde i vaše maslinike i poklanjat  će ih svojim dvoranima. 
\par 15 Uzimat će desetinu od vaših usjeva  i vaših vinograda i davat će je svojim dvoranima i svojim službenicima. 
\par 16 Uzimat će vaše sluge i vaše sluškinje, vaše najljepše volove  i magarce i upotrebljavat će ih za svoj posao. 
\par 17 Uzimat će  desetinu od vaše sitne stoke, a vi sami postat ćete mu robovi. 
\par 18 I kad jednoga dana budete vapili za pomoć zbog kralja koga  ste sami izabrali, Jahve vas neće uslišati u onaj dan!" 
\par 19 Narod nije htio poslušati Samuelova glasa nego reče:  "Ne! Hoćemo da kralj vlada nad nama! 
\par 20 Tako ćemo i mi biti  kao svi narodi: sudit će nam naš kralj, bit će nam vođa i vodit  će naše ratove." 
\par 21 Kad je Samuel čuo što narod govori, kaza sve Jahvi. 
\par 22 A  Jahve reče Samuelu: "Poslušaj njihovu želju i postavi im kralja!" Tada Samuel reče Izraelcima: "Vratite se svaki u svoj grad!" 


\chapter{9}

\par 1 Živio u ono vrijeme jedan čovjek u Benjaminovu plemenu po imenu  Kiš, sin Abiela, sina Serora, sina Bekorata, sina Afijahova;  bio je iz plemena Benjaminova, čovjek imućan. 
\par 2 Imao je sina  po imenu Šaula, koji je bio mlad i lijep. Među sinovima Izraelovim  nije bilo ljepšega čovjeka od njega: za glavu bijaše viši od  svega naroda. 
\par 3 Uto se Kišu, Šaulovu ocu, izgubilo nekoliko  magarica, pa Kiš reče svome sinu Šaulu: "Uzmi sa sobom jednoga  momka pa ustani i idi tražiti magarice!" 
\par 4 I prođoše oni Efrajimovu  goru i prođoše zemlju Šališu, ali ne nađoše ništa; prođoše zemlju  Šaalim, ali magarica ne bijaše ondje; prođoše i zemlju Benjaminovu, ali ne nađoše ništa. 
\par 5 Kad su došli u zemlju Suf, reče Šaul momku koji ga je  pratio: "Hajde, vratimo se da se ne bi otac okanio magarica i  zabrinuo se za nas!" 
\par 6 A on mu odgovori: "Eno, u onom ondje  gradu živi čovjek Božji; to je vrlo ugledan čovjek: što god rekne, sve se zacijelo ispunja. Pođimo, dakle, k njemu, možda će nas  uputiti u ono zbog čega smo pošli na put." 
\par 7 A Šaul reče svome  momku: "Ako zaista pođemo onamo, što ćemo ponijeti čovjeku? Kruha  je nestalo u našim torbama, nemamo dara da ponesemo čovjeku Božjem.  Što mu možemo dati?" 
\par 8 A momak opet progovori i reče Šaulu:  "Gle, imam u ruci četvrt šekela srebra: dat ću ga Božjem čovjeku  da nas uputi kamo bismo išli." 
\par 9 Nekoć se u Izraelu, kad bi išli pitati Boga za savjet, govorilo:  "Hajde, pođimo k vidiocu!" Jer koga danas zovu prorokom nekoć  se zvao vidjelac. - 
\par 10 Šaul odvrati svome momku:  "Dobro veliš. Hajdemo!" I krenuše u grad gdje je živio čovjek  Božji. 
\par 11 Kad su se penjali usponom prema gradu, sretoše djevojke  koje su izašle da zahvate vode. I zapitaše ih: "Je li gore vidjelac?"  - 
\par 12 One im odgovore ovako: "Jest, vidjelac  je pred vama. Upravo je stigao u grad, jer danas narod ima žrtvu  na uzvišici. 
\par 13 Čim uđete u grad, naći ćete ga još prije nego  što se popne na uzvišicu da sudjeluje na žrtvenoj gozbi. Narod  neće jesti dok on ne dođe, jer on mora blagosloviti žrtvu, a  onda će tek uzvanici jesti. Zato idite odmah gore, jer ćete ga  sada još naći." 
\par 14 Oni otiđoše gore u grad. Kad su ulazili na vrata, Samuel  ih susrete polazeći na uzvišicu. 
\par 15 A dan prije nego što je Šaul došao bijaše Jahve objavio  Samuelu: 
\par 16 "Sutra u ovo doba poslat ću k tebi čovjeka iz Benjaminove  zemlje. Ti ćeš ga pomazati za kneza nad mojim narodom Izraelom.  On će izbaviti moj narod iz ruke filistejske. Vidio sam nevolju  svoga naroda i njegov je vapaj dopro do mene." 
\par 17 A kad je Samuel  ugledao Šaula, Jahve mu progovori: "Evo ti čovjeka za koga ti  rekoh: 'Taj će vladati nad mojim narodom.'" 
\par 18 Šaul pristupi  Samuelu na vratima i reče: "Daj mi kaži gdje je vidiočeva kuća." 
\par 19 A Samuel odgovori Šaulu: "Ja sam vidjelac. Pođi preda  mnom na uzvišicu, danas ćete sa mnom jesti. Sutra ću te ujutro  otpustiti i sve ću ti kazati što ti je na srcu. 
\par 20 A za magarice  koje su ti se izgubile prije tri dana ne uznemiruj se jer su  se našle. Uostalom, kome pripada sve što je najdragocjenije u  Izraelu? Zar ne tebi i svemu domu tvoga oca?" 
\par 21 A Šaul odgovori ovako: "Nisam li ja od Benjaminova plemena, najmanjega plemena Izraelova? A moj rod nije li najneznatniji  između svih rodova Benjaminova plemena? Zašto mi, dakle, govoriš  takve riječi?" 
\par 22 Samuel uze Šaula i njegova momka, odvede ih u sobu i  dade im mjesto u pročelju među uzvanicima, kojih je bilo tridesetak. 
\par 23 Zatim Samuel reče kuharu: "Donesi dio koji ti dadoh i za  koji ti rekoh da ga staviš na stranu." 
\par 24 Kuhar uze but, donese ga i stavi pred Šaula, a Samuel  mu reče: "Evo, pred tobom je ono što je sačuvano za tebe. Jedi, jer to ti je sačuvano baš za ovu zgodu." Tako je u onaj dan Šaul jeo sa Samuelom. 
\par 25 Potom odande siđoše u grad. Ondje prostriješe Šaulu na  krovu. 
\par 26 I on leže na počinak. Čim je svanula zora, Samuel zovnu Šaula (na krovu) govoreći:  "Ustani da te otpustim!" Kad je Šaul ustao, izađoše obojica,  on i Samuel. 
\par 27 Kad su došli na kraj grada, reče Samuel Šaulu: "Kaži  momku neka pođe naprijed pred nama! A ti stani sada da ti objavim  riječ Božju." 


\chapter{10}

\par 1 Tada Samuel uze uljanicu s uljem te je izli na glavu Šaulu;  zatim ga poljubi i reče: "Ovim te Jahve pomazao za kneza nad svojim narodom Izraelom.  Ti ćeš vladati nad narodom Jahvinim i izbavit ćeš ga iz ruke  njegovih neprijatelja unaokolo. I evo ti znaka da te Jahve pomazao  za kneza nad svojom baštinom. 
\par 2 Kad odeš sada od mene, naći  ćeš dva čovjeka kod Rahelina groba, na granici zemlje Benjaminove, u Selsahu. Oni će ti reći: 'Našle su se magarice koje si pošao  tražiti; i gle, tvoj je otac zaboravio na magarice, a zabrinut  je za vas i govori: Što da učinim za svoga sina?' 
\par 3 A kad odeš  odande dalje i dođeš do Taborskog Hrasta, srest ćeš ondje tri  čovjeka koja će ići gore k Bogu u Betel. Jedan će nositi tri  jareta, drugi tri okrugla kruha, a treći mijeh vina. 
\par 4 Oni će  te pozdraviti i dat će ti dva kruha, a ti ih primi iz njihove  ruke. 
\par 5 Poslije toga doći ćeš u Gibeu Božju (gdje se nalazi  filistejski stup). Kad uđeš u grad, namjerit ćeš se na povorku  proroka koji će silaziti s uzvišice, a pred njima harfe, bubnjevi, frule i citre; oni će biti u proročkom zanosu. 
\par 6 Tada će na  te sići duh Jahvin te ćeš pasti u proročki zanos s njima i promijenit  ćeš se u drugog čovjeka. 
\par 7 A kad ti se ispune ti znakovi, onda  čini kako ti se prilika pruži jer je Bog s tobom. 
\par 8 Zatim ćeš  sići preda mnom u Gilgal i ja ću sići k tebi da prinesem žrtve  paljenice i žrtve pričesnice. Sedam dana čekaj dok ne dođem k  tebi i ne poučim te što ćeš činiti." 
\par 9 Čim je Šaul okrenuo leđa da ode od Samuela, Bog mu promijeni  srce i svi se oni znakovi ispuniše u onaj dan. 
\par 10 Kad su, naime, došli u Gibeu, gle, dođe mu u susret povorka proroka i duh Božji  siđe na njega te on pade u proročki zanos usred njih. 
\par 11 I kad  su ga svi koji ga poznavahu otprije vidjeli gdje prorokuje s  prorocima, počeše govoriti jedan drugome: "Što se to dogodilo  sa sinom Kiševim? Zar je Šaul među prorocima?" 
\par 12 A jedan od  njih odvrati i reče: "A tko je njihov otac?" Otuda je nastala  poslovica: "Zar je i Šaul među prorocima?" 
\par 13 Kad je prošao njegov zanos, Šaul se vrati kući. 
\par 14 A  Šaulov stric upita njega i njegova momka: "Kamo ste išli?" A  Šaul odgovori: "Da tražimo magarice; a kad smo vidjeli da ih  nema, otišli smo k Samuelu." 
\par 15 A njegov ga stric zamoli: "Pripovijedaj  mi što vam je rekao Samuel." 
\par 16 A Šaul odgovori svome stricu:  "Rekao nam je da su se našle magarice." Ali mu ništa ne reče  o kraljevskoj časti koju mu je prorekao Samuel. 
\par 17 Poslije toga Samuel sazva narod pred Jahvu u Mispu 
\par 18 i  reče sinovima Izraelovim: "Ovako govori Jahve: 'Ja sam izveo  Izraela iz Egipta i izbavio sam vas iz egipatske ruke i iz ruke  svih kraljevstava koja su vas tlačila. 
\par 19 A vi ste danas odbacili  svoga Boga, onoga koji vas je izbavljao od svih vaših zala i  svih vaših nevolja i rekli ste mu: 'Ne, nego postavi kralja nad  nama!' Zato sada stanite pred Jahvom po svojim plemenima i rodovima.'" 
\par 20 Potom Samuel privede sva plemena Izraelova i ždrijeb  pade na pleme Benjaminovo. 
\par 21 Zatim privede pleme Benjaminovo  po rodovima i ždrijeb pade na Matrijev rod; a kad privede Matrijev  rod, čovjeka po čovjeka, ždrijeb pade na Šaula, sina Kiševa;  ali kad ga potražiše, na nađoše ga. 
\par 22 Tada još jednom upitaše Jahvu: "Je li taj čovjek došao  ovamo?" A Jahve odgovori: "Eno ga, sakrio se za tovarom." 
\par 23 Otrčaše  i dovedoše ga odande; a kad je stao usred naroda, bijaše glavom  i ramenima viši od sviju. 
\par 24 Tada Samuel reče svemu narodu:  "Vidite li koga je izabrao Jahve? Nema mu ravna u svemu narodu."  I sav narod uze klicati i vikati: "Živio kralj!" 
\par 25 Nato Samuel objavi narodu kraljevsko pravo i zapisa ga  u knjigu koju položi pred Jahvu. Najposlije Samuel otpusti sav  narod da ide svaki svojoj kući. 
\par 26 Šaul se također vrati kući u Gibeu, a s njim pođoše junaci  kojima je Bog taknuo srce. 
\par 27 Ali neke ništarije rekoše: "Kako  će nas taj spasiti?" I prezreše ga i ne donesoše mu nikakva dara. 


\chapter{11}

\par 1 Otprilike poslije mjesec dana dođe Amonac Nahaš i utabori  se kod Jabeša Gileadskog. Svi Jabešani poručiše Nahašu: "Sklopi  savez s nama pa ćemo ti se pokoriti." 
\par 2 Ali im Amonac Nahaš  odgovori: "Ovako ću sklopiti savez s vama: svakome ću od vas  iskopati desno oko, i tako ću učiniti sramotu svemu Izraelu." 
\par 3 A jabeške mu starješine rekoše: "Ostavi nam sedam dana da  pošaljemo glasnike u sve krajeve Izraelove, pa ako se ne nađe  nitko da nas izbavi, predat ćemo se tebi." 
\par 4 I dođoše poslanici u Šaulovu Gibeu te izložiše sve narodu  da čuje. Tada sav narod zaplaka iza glasa. 
\par 5 A gle, Šaul je upravo išao za govedima iz polja pa upita:  "Što je ljudima te plaču?" I pripovjediše mu što su rekli Jabešani. 
\par 6 Kad je Šaul čuo te riječi, duh Jahvin siđe na njega i silan  gnjev uskipje u njemu. 
\par 7 I uze on dva vola, isiječe ih i komade  razasla po poslanicima u sve krajeve Izraelove i poruči: "Tko  ne pođe za Šaulom, ovako će biti s njegovim govedima." I strah  Božji obuze ljude te pođoše kao jedan čovjek. 
\par 8 Šaul ih izbroji  u Bezeku: i bijaše sinova Izraelovih tri stotine tisuća, a Judinih  ljudi trideset tisuća. 
\par 9 Zatim reče poslanicima koji bijahu  došli: "Ovako recite Jabešanima u Gileadu: sutra, kad sunce pripeče, stići će vam pomoć." Kad su se poslanici vratili, javiše sve to Jabešanima i oni  se obradovaše. 
\par 10 I poručiše Nahašu: "Sutra ćemo izaći k vama, pa učinite s nama što vam bude drago." 
\par 11 Sutradan Šaul razdijeli narod u tri čete, koje provališe  u tabor o jutarnjoj straži i tukoše Amonce do najveće dnevne  žege; a što preživje, rasprša se da ni dvojica ne ostaše zajedno. 
\par 12 Tada narod reče Samuelu: "Tko je onaj što je govorio:  'Zar će Šaul kraljevati nad nama?' Dajte te ljude da ih pogubimo!" 
\par 13 Ali Šaul odgovori: "Neka se ne pogubi u ovaj dan nitko, jer  je danas Jahve izvojevao pobjedu u Izraelu." 
\par 14 Tada Samuel  reče narodu: "Hajdemo u Gilgal da ondje potvrdimo kraljevstvo." 
\par 15 I sav narod krenu u Gilgal i ondje postaviše Šaula za  kralja pred Jahvom, u Gilgalu. Ondje žrtvovaše pred Jahvom žrtve  pričesnice i ondje je Šaul sa svim Izraelcima slavio slavlje. 


\chapter{12}

\par 1 Tada Samuel reče svemu Izraelu: "Evo, ispunio sam vašu želju  u svemu što ste od mene tražili i postavih kralja nad vama. 
\par 2 I  od sada će kralj ići pred vama. A ja sam ostario i osijedio i  moji sinovi eto su među vama. Ja sam išao pred vama od svoje  mladosti pa do današnjega dana. 
\par 3 Evo me! Posvjedočite protiv  mene pred Jahvom i pred njegovim pomazanikom: kome sam oteo vola  i kome sam oteo magarca? Koga sam prevario? Koga sam tlačio?  Od koga sam primio mito da bih zažmirio na jedno oko? Ja ću vam  sve natrag vratiti." 
\par 4 A oni odgovoriše: "Nisi nas prevario, nisi nas tlačio, nisi ni od koga primio ništa." 
\par 5 Još im reče: "Svjedok je Jahve protiv vas i svjedok je  njegov pomazanik u ovaj dan da niste našli ništa u mojoj ruci."  A oni odgovoriše: "Tako je!" 
\par 6 Tada Samuel reče narodu: "Jest, svjedok je Jahve koji  je postavio Mojsija i Arona i koji je izveo vaše oce iz Egipta. 
\par 7 Stanite sada ovamo da probesjedim s vama pred Jahvom i da  vas podsjetim na sva velika djela koja je učinio Jahve vama i  vašim ocima. 
\par 8 Kad je Jakov došao u Egipat, Egipćani su ih pritisnuli, a vaši su oci vapili Jahvi za pomoć. I Jahve posla Mojsija i  Arona, koji izvedoše oce vaše iz Egipta i naseliše ih na ovome  mjestu. 
\par 9 Ali oni zaboraviše Jahvu, Boga svoga, i on ih predade  u ruke Siseri, vojvodi hasorske vojske, i u ruke Filistejaca, i u ruke moapskome kralju, koji su vojevali na njih. 
\par 10 I opet  su vapili Jahvi za pomoć govoreći: 'Zgriješili smo jer smo ostavili  Jahvu i uzeli služiti baalima i aštartama; izbavi nas sada iz  ruku naših neprijatelja pa ćemo ti služiti!' 
\par 11 I Jahve posla  Jerubaala i Baraka, Jiftaha i Samuela te vas izbavi iz ruku vaših  neprijatelja unaokolo, tako da ste mogli živjeti bez straha. 
\par 12 Ali kad vidjeste Nahaša, kralja amonskoga, kako ide na  vas, rekoste mi: 'Ne, nego kralj neka vlada nad nama!' Pa ipak  je vaš kralj Jahve, vaš Bog! 
\par 13 I eto vam sada kralja koga ste  izabrali! Eto, Jahve je postavio kralja nad vama. 
\par 14 Ako se  budete bojali Jahve i njemu budete služili, ako budete slušali  njegov glas i ne budete se protivili njegovim zapovijedima, slijedit  ćete Jahvu, Boga svoga, vi i kralj koji kraljuje nad vama. 
\par 15 Ako  li ne budete slušali Jahvina glasa, ako se budete protivili njegovim  zapovijedima, tada će se ruka Jahvina spustiti na vas i na vašega  kralja da vas uništi. 
\par 16 Sada još jednom pristupite i vidite veliki znak koji  će Jahve učiniti pred vašim očima. 
\par 17 Nije li sada pšenična  žetva? Ali ja ću zazvati Jahvu i on će poslati gromove i kišu.  I jasno ćete razabrati kako je veliko zlo koje ste učinili pred  Jahvom tražeći sebi kralja." 
\par 18 Tada Samuel zazva Jahvu i Jahve posla gromove i kišu  u onaj dan i sav se narod vrlo poboja Jahve i Samuela. 
\par 19 I  sav narod reče Samuelu: "Moli se Jahvi, svome Bogu, za svoje  sluge da ne pomremo, jer smo svim svojim grijesima dodali zlo  tražeći sebi kralja." 
\par 20 Ali Samuel reče narodu: "Ne bojte se! Vi ste, doduše, učinili sve ovo zlo, ali sada ne ostavljajte više Jahvu, nego  služite Jahvi svim svojim srcem. 
\par 21 Ne priklanjajte se više  ništavim idolima koji vam ništa ne koriste, ništa vam ne pomažu  jer su samo ništavila. 
\par 22 A Jahve neće odbaciti svoga naroda, radi velikog imena svoga, jer se Jahve udostojao da vas učini  svojim narodom. 
\par 23 A od mene neka je daleko da zgriješim Jahvi  prestajući moliti za vas i upućivati vas na dobar i pošten put. 
\par 24 Samo se bojte Jahve i njemu iskreno služite svim svojim srcem;  jer, pogledajte kako se velikim očitovao među vama. 
\par 25 Ako li  budete činili zlo, propast ćete vi i vaš kralj." 


\chapter{13}

\par 1 Šaulu je bilo ... godina kad je postao kralj, a kraljevao  je ... i dvije godine nad Izraelom. 
\par 2 Šaul izabra sebi tri tisuće  Izraelaca: dvije tisuće od njih bijahu sa Šaulom u Mikmasu i  u Betelskoj gori, a jedna tisuća bijaše s Jonatanom u Benjaminovoj  Gebi. Ostali je narod Šaul otpustio svakoga u njegov šator. 
\par 3 Jonatan sruši filistejski stup koji je stajao u Gibei  i Filistejci saznaše da su se Hebreji pobunili. Šaul zapovjedi  te zatrubiše u rog po svoj zemlji 
\par 4 i sav Izrael doznade novost:  "Šaul je srušio filistejski stup, Izrael se omrazio Filistejcima!"  I narod se poče skupljati oko Šaula u Gilgalu. 
\par 5 A Filistejci se skupiše da vojuju na Izraela: tri tisuće  bojnih kola, šest tisuća konja, a mnoštvo naroda kao pijeska  na morskoj obali. I utaboriše se kod Mikmasa, istočno od Bet  Avena. 
\par 6 Kad su Izraelci vidjeli da su u nevolji i da je narod  pritisnut od neprijatelja, posakrivaše se u pećine, jame, kamenjake, jarke i čatrnje. 
\par 7 Neki su prešli i preko gazova Jordana u  zemlju Gadovu i Gileadovu. Šaul je još bio u Gilgalu, a sav je narod oko njega drhtao  od straha. 
\par 8 On pričeka sedam dana kako mu je odredio Samuel;  ali kad Samuel nije došao u Gilgal, narod se stade razilaziti  od Šaula. 
\par 9 Tada reče Šaul: "Donesite mi žrtvu paljenicu i žrtve  pričesnice!" I prinese žrtvu paljenicu. 
\par 10 I upravo je završavao  žrtvu paljenicu, kad eto Samuela; Šaul mu iziđe u susret da ga  pozdravi. 
\par 11 Samuel ga upita: "Što si učinio?" A Šaul odgovori: "Kad  sam vidio da se narod razilazi od mene, a ti da ne dolaziš do  određenoga dana, a Filistejci se skupili u Mikmasu, 
\par 12 pomislio  sam: sad će udariti Filistejci na me u Gilgalu, a ja neću stići  molitvom ublažiti Jahvu! Zato se odvažih i prinesoh žrtvu paljenicu." 
\par 13 Samuel tada reče Šaulu: "Ludo si radio! Da si održao  zapovijed koju ti je dao Jahve, tvoj Bog, Jahve bi učvrstio tvoje  kraljevstvo nad Izraelom dovijeka. 
\par 14 A sada se tvoje kraljevstvo  neće trajno održati: Jahve je potražio sebi čovjeka po svom srcu  i odredio ga za kneza nad svojim narodom, jer ti nisi održao  što ti je Jahve zapovjedio." 
\par 15 Nato Samuel ustade i ode iz  Gilgala svojim putem. Što je naroda ostalo, pođe za Šaulom u susret ratnicima.  Kad su došli iz Gilgala u Gebu Benjaminovu, Šaul pobroji narod  koji je ostao uza nj i bijaše ga oko šest stotina ljudi. 
\par 16 Šaul i sin mu Jonatan s ljudima što bijahu s njima zaposjeli  su Benjaminovu Gebu, a Filistejci se utaborili u Mikmasu. 
\par 17 Tada  iz filistejskog tabora izađe četa pljačkaša u tri odjela: jedan  odio udari prema Ofri u zemlju šualsku; 
\par 18 drugi odio krenu  prema Bet Horonu, a treći odio udari prema Gebi koja se uz Dolinu  hijena diže nad pustinjom. 
\par 19 A po svoj zemlji Izraelovoj nije bilo kovača, jer su  Filistejci rekli: "Treba sve učiniti da Hebreji ne bi pravili  sebi mačeva i kopalja." 
\par 20 Zato su svi Izraelci išli k Filistejcima  ako je tko htio da prekuje svoj raonik ili motiku, svoju sjekiru  ili ostan za volove. 
\par 21 A cijena je bila dvije trećine šekela  za raonike i motike, jedna trećina za oštrenje sjekire i za nasađivanje  ostana. 
\par 22 Tako se dogodilo da na dan bitke kod Mikmasa nitko  od svega naroda koji bijaše sa Šaulom i Jonatanom nije imao ni  mača ni koplja u ruci; samo ih imahu Šaul i njegov sin Jonatan. 
\par 23 A dotle jedna straža filistejska bijaše izišla prema  klancu kod Mikmasa. 


\chapter{14}

\par 1 Jednoga dana Šaulov sin Jonatan reče svome momku štitonoši:  "Hajde da prijeđemo do filistejske straže koja je ondje prijeko."  Svome ocu nije ništa o tom javio. 
\par 2 Šaul je sjedio na međi Gebe, pod šipkom koji je stajao kraj gumna; a bilo je s njim oko šest  stotina ljudi. 
\par 3 A Ahija, sin Ahituba, brata Ikaboda, sina Pinhasa, sina Elija, svećenika Jahvina u Šilu, nosio je u to vrijeme  oplećak. Narod nije primijetio da je Jonatan otišao. 
\par 4 U sredini klanca kuda je Jonatan htio prijeći da dođe  do filistejske straže bila je litica s jedne strane i litica  s druge strane. Jedna se zvala Boses, a druga Sene. 
\par 5 Prva je  litica stajala na sjeveru nasuprot Mikmasu, a druga na jugu nasuprot  Gebi. 
\par 6 Jonatan reče svome štitonoši: "Hajde da prijeđemo do straže  onih neobrezanika. Možda će Jahve učiniti nešto za nas, jer ništa  ne priječi Jahvu da udijeli pobjedu - bilo mnogo ljudi ili malo." 
\par 7 A štitonoša mu odgovori: "Čini sve na što te srce tvoje  potiče. Ja ću s tobom, moje je srce kao tvoje srce." 
\par 8 Jonatan mu reče: "Evo, prijeći ćemo k tim ljudima i pokazat  ćemo im se. 
\par 9 Ako nam reknu ovako: 'Ne mičite se dok ne dođemo  do vas', tada ćemo se ustaviti na mjestu i nećemo se uspinjati  k njima. 
\par 10 Ako li nam reknu ovako: 'Uspnite se k nama', tada  ćemo se uspeti, jer ih je Jahve predao nama u ruke. To će nam  biti znak." 
\par 11 Kad su se obojica pokazala filistejskoj straži, rekoše  Filistejci: "Gle, Hebreji su počeli izlaziti iz rupa u koje su  se skrili." 
\par 12 I stražari doviknuše Jonatanu i njegovu štitonoši:  "Uspnite se k nama da vas nešto naučimo!" A Jonatan reče svome štitonoši: "Penji se za mnom, jer ih  je Jahve predao u ruke Izraelove." 
\par 13 Jonatan se poče penjati  pomažući se rukama i nogama, a za njim njegov štitonoša. Filistejci  su padali pred Jonatanom, a njegov ih je štitonoša ubijao za  njim. 
\par 14 U tome prvom pokolju što ga učiniše Jonatan i njegov  štitonoša pade dvadesetak ljudi na otprilike pola jutra izoranog  polja. 
\par 15 Tada se proširi strah po taboru i po polju, a i stražare  i četu pljačkaša obuze strava; i zemlja zadrhta i bijaše to silan  strah Božji. 
\par 16 A Šaulovi stražari u Benjaminovoj Gebi opaziše  da se mnoštvo u taboru uskomešalo na sve strane. 
\par 17 I Šaul reče  ljudima koji su bili s njim: "Prozovite ljude i vidite tko je  otišao od nas." A kad prozvaše, gle, ne bijaše Jonatana i njegova  štitonoše! 
\par 18 Tada Šaul reče Ahiji: "Primakni oplećak! Posavjetuj se  s Jahvom!" On je, naime, tada nosio oplećak pred sinovima Izraelovim. 
\par 19 Ali dok je Šaul govorio sa svećenikom, bivala je buka u filistejskom  taboru sve veća, pa Šaul reče svećeniku: "Povuci ruku!" 
\par 20 Nato  Šaul i sav narod što je bio s njim krenuše zajedno na mjesto  boja, i gle, ondje bijahu isukali mačeve jedni na druge i velika  pomutnja vladaše među njima. 
\par 21 A oni Hebreji koji su već poodavno  bili u službi Filistejaca i sada pošli s njima na vojsku, odmetnuše  se od njih i pristadoše uz Izraelce koji bijahu sa Šaulom i Jonatanom. 
\par 22 I svi Izraelci koji se bijahu sakrili u Efrajimovoj gori, čuvši da Filistejci bježe, nagrnuše za njima u boj. 
\par 23 Tako je Jahve udijelio pobjedu Izraelu u onaj dan, a  boj se raširio sve do preko Bet Horona. 
\par 24 Izraelci su onog dana bili vrlo izmoreni, jer je Šaul  izrekao nad narodom ovu zakletvu: "Proklet bio čovjek koji okusi  hrane prije večeri, prije nego što se osvetim svojim neprijateljima!"  Tako sav narod ne okusi hrane toga dana. 
\par 25 Ali je ondje bilo medenoga saća na površini zemlje. 
\par 26 Kad  je narod došao onamo, vidje gdje teče med, ali nitko ne prinese  ruke k ustima, jer se narod bojao zakletve. 
\par 27 Samo Jonatan, koji nije čuo kad je njegov otac zakleo narod, primače vrh štapa  koji mu bijaše u ruci i zamoči ga u medeno saće, zatim prinese  ruku k ustima; i odmah mu se zasvijetliše oči. 
\par 28 Tada jedan  iz naroda progovori i reče mu: "Tvoj je otac zakleo narod govoreći:  'Proklet bio onaj koji okusi hrane danas!'" 
\par 29 A Jonatan odgovori:  "Moj otac svaljuje nesreću na zemlju. Gledajte kako su mi se  zasvijetlile oči jer sam okusio malo toga meda. 
\par 30 Što bi tek  bilo da je narod slobodno jeo od plijena koji je zadobio od neprijatelja?  Ne bi li filistejski poraz bio još veći?" 
\par 31 Onoga dana potukoše Filistejce od Mikmasa sve do Ajalona, a narod je bio na kraju svojih snaga. 
\par 32 Tada se narod baci  na plijen, nahvata sitne stoke, goveda i teladi i poče ih klati  na goloj zemlji i jesti meso s krvlju. 
\par 33 I javiše to Šaulu  govoreći: "Gle, narod griješi Jahvi jedući meso s krvlju!" A  on reče: "Iznevjeriste se! Dovaljajte mi ovamo velik kamen!" 
\par 34 Zatim reče: "Zađite među narod i recite svima neka svaki  dovede k meni svoga vola ili ovcu; ovdje ćete ih klati i jesti, a nećete griješiti Jahvi jedući meso s krvlju." Tako sav narod još iste noći dovede što je tko imao i to  su ondje klali. 
\par 35 A Šaul podiže žrtvenik Jahvi; bijaše to prvi  žrtvenik koji je podigao Jahvi. 
\par 36 Nato reče Šaul: "Pođimo još noćas u potjeru za Filistejcima  i plijenimo ih dok ne svane jutro! Nećemo im ostaviti nijednoga  čovjeka!" A narod mu odgovori: "Čini sve što misliš da je dobro!"  Ali svećenik reče: "Pristupimo ovdje k Bogu!" 
\par 37 I Šaul upita  Boga: "Moram li poći u potjeru za Filistejcima? Hoćeš li ih predati  u ruke Izraelu?" Ali mu ne odgovori u onaj dan. 
\par 38 Zato Šaul reče: "Pristupite ovamo, svi narodni glavari!  Ispitajte i vidite u čemu je bio današnji prestupak. 
\par 39 Jer, živoga mi Jahve, koji daje pobjedu Izraelu, ako se nađe krivnja  ma i na mome sinu Jonatanu, mora umrijeti!" Ali nitko se iz naroda  ne usudi odgovoriti Šaulu. 
\par 40 Šaul onda reče svemu Izraelu: "Vi stanite na jednu stranu, a ja i moj sin Jonatan stat ćemo na drugu stranu." A narod odgovori  Šaulu: "Čini ono što misliš da je dobro!" 
\par 41 Tada se Šaul pomoli:  "Jahve, Bože Izraelov, zašto nisi danas odgovorio svome sluzi?  Ako je krivnja na meni ili na mome sinu Jonatanu, Jahve, Bože  Izraelov, daj Urim; ako li je krivnja na tvom narodu Izraelu, daj Tumim." I ždrijeb pade na Šaula i Jonatana, a narod izađe  slobodan. 
\par 42 Šaul nastavi: "Bacite ždrijeb između mene i moga sina  Jonatana!" I ždrijeb pade na Jonatana. 
\par 43 Tada Šaul reče Jonatanu: "Priznaj mi što si učinio!"  Jonatan odgovori: "Ja sam samo okusio malo meda vrškom štapa  koji mi bijaše u ruci. Evo me, spreman sam umrijeti!" 
\par 44 Šaul  odgovori: "Tako mi Bog učinio zlo i dodao mi drugo ako doista  ne umreš, Jonatane!" 
\par 45 Ali narod reče Šaulu: "Zar da umre Jonatan, koji je izvojevao  ovu veliku pobjedu u Izraelu? Ne smije to biti! Živoga nam Jahve, nijedna vlas neće pasti s njegove glave na zemlju jer je on  s Bogom izvršio ovo djelo danas!" Tako ga narod izbavi te Jonatan  ne pogibe. 
\par 46 Šaul odusta od potjere za Filistejcima, a Filistejci  se vratiše u svoj kraj. 
\par 47 Kad je Šaul učvrstio svoju kraljevsku vlast nad Izraelom, okrenu ratovati protiv svih svojih neprijatelja unaokolo: protiv  Moaba, protiv Amonaca, protiv Edoma, protiv Bet Rehoba, protiv  kralja Sobe i protiv Filistejaca; kuda god bi se okrenuo, svuda  bi pobjeđivao. 
\par 48 Dao je mnogo dokaza svoje hrabrosti, potukao  je Amalečane i izbavio Izraela iz ruku onih koji su ga pljačkali. 
\par 49 Šaulovi sinovi bijahu Jonatan, Išjo i Malki-Šua, a od  njegovih dviju kćeri starija se zvala Meraba, a mlađa Mikala. 
\par 50 Šaulova se žena zvala Ahinoama, a bila je kći Ahimaasova.  Vojvoda njegove vojske zvao se Abner, a bio je sin Nera, Šaulova  strica. 
\par 51 Jer Kiš, Šaulov otac, i Ner, Abnerov otac, bijahu  sinovi Abielovi. 
\par 52 Žestok se rat vodio protiv Filistejaca svega Šaulova  vijeka. Koga bi god hrabra ili bojovna čovjeka Šaul vidio, svakoga  bi uzimao u svoju službu. 


\chapter{15}

\par 1 Jednom Samuel reče Šaulu: "Mene je Jahve poslao da te pomažem  za kralja nad njegovim narodom Izraelom. Poslušaj, dakle, riječi  Jahvine. 
\par 2 Ovako govori Jahve nad vojskama: 'Odlučio sam osvetiti  ono što je Amalek učinio Izraelu zatvarajući mu put kad je izlazio  iz Egipta. 
\par 3 Sada idi i udari na Amaleka, izvrši "herem", kleto  uništenje, na njemu i na svemu što posjeduje; ne štedi ga, pobij  muškarce i žene, djecu i dojenčad, goveda i ovce, deve i magarce!'" 
\par 4 Šaul sazva narod te ih izbroji u Telamu: bijaše ih dvije  stotine tisuća pješaka (i deset tisuća Judejaca). 
\par 5 Šaul dođe  do amalečkoga grada i postavi zasjedu u dolini potoka. 
\par 6 Potom  Šaul poruči Kenijcima: "Otiđite i odvojite se od Amalečana da  vas ne bih istrijebio zajedno s njima, jer ste bili skloni svim  Izraelcima kad su izlazili iz Egipta." I Kenijci se odvojiše  od Amalečana. 
\par 7 Šaul potuče Amalečane od Havile pa sve do Šura, koji leži  pred Egiptom. 
\par 8 I živa uhvati Agaga, amalečkog kralja, a sav  narod zatre oštricom mača, izvršujući "herem", kleto uništenje. 
\par 9 Ali Šaul i narod poštedješe Agaga i najbolje ovce i najbolja  goveda, ugojenu stoku i jaganjce i sve što je bilo dobro. Na  svemu tome ne htjedoše izvršiti "herem"; nego što je god od stoke  bilo bez cijene i vrijednosti, na tom izvršiše "herem". 
\par 10 Zato dođe riječ Jahvina Samuelu ovako: 
\par 11 "Kajem se  što sam Šaula postavio za kralja: okrenuo se od mene i nije izvršio  mojih zapovijedi." Samuel se ražalosti i svu je noć vapio Jahvi. 
\par 12 U rano jutro krenu Samuel da potraži Šaula. I javiše  Samuelu ovako: "Šaul je otišao u Karmel, i gle, podigao je ondje  sebi spomenik; zatim je otišao dalje i sišao u Gilgal." 
\par 13 Kad  je Samuel došao k Šaulu, reče mu Šaul: "Blagoslovljen da si od  Jahve! Izvršio sam Jahvinu zapovijed." 
\par 14 Ali Samuel upita:  "Kakvo je to ovčje blejanje što dopire do mojih ušiju i mukanje  goveda koje čujem?" 
\par 15 A Šaul odgovori: "Dognali su ih od Amalečana, jer je narod poštedio najbolje ovce i najbolja goveda da ih  žrtvuje Jahvi, tvome Bogu. Na svemu drugome izvršili smo 'herem'." 
\par 16 A Samuel reče Šaulu: "Stani da ti kažem što mi je noćas  objavio Jahve." A on reče: "Govori!" 
\par 17 Tada će Samuel: "Koliko  god si malen sam u svojim očima, ipak si postao glavar Izraelovih  plemena. Jahve te pomazao za kralja nad Izraelom. 
\par 18 Jahve te  poslao na vojni pohod i zapovjedio ti: 'Idi, izvrši 'herem' na  tim grešnicima, na Amalečanima, vojuj na njih do istrebljenja.' 
\par 19 Zašto nisi poslušao riječi Jahvine? Zašto si se bacio na  plijen i učinio ono što je zlo u Jahvinim očima?" 
\par 20 Šaul odgovori Samuelu: "Ja sam poslušao riječ Jahvinu:  poduzeo sam pohod kamo me poslao, doveo sam Agaga, amalečkoga  kralja, i izvršio 'herem' na Amalečanima. 
\par 21 Ali je narod od  plijena uzeo ovaca i goveda, i to najbolje na čemu se imao izvršiti  'herem', da žrtvuje Jahvi, tvome Bogu, u Gilgalu." 
\par 22 A Samuel odvrati: "Jesu li Jahvi milije paljenice i klanice nego poslušnost njegovu glasu? Znaj, poslušnost je vrednija od najbolje žrtve, pokornost je bolja od ovnujske pretiline. 
\par 23 Nepokornost je kao grijeh čaranja, samovolja je kao zločin s idolima. Ti si odbacio riječ Jahvinu, zato je Jahve odbacio tebe da  ne budeš više kralj!" 
\par 24 Tada Šaul odvrati Samuelu: "Zgriješio sam što sam prekršio  Jahvinu zapovijed i tvoje naredbe. Bojao sam se naroda i popustio  njegovu zahtjevu. 
\par 25 A sada mi oprosti moj grijeh i vrati se  sa mnom da se poklonim Jahvi." 
\par 26 Ali Samuel odgovori Šaulu:  "Neću se vratiti s tobom: ti si odbacio Jahvinu riječ, zato je  Jahve odbacio tebe da ne budeš više kralj nad Izraelom." 
\par 27 Kad se Samuel okrenuo da ode, Šaul čvrsto uhvati skut  njegova plašta, ali se skut otkide. 
\par 28 Tada mu reče Samuel: "Danas ti je Jahve otkinuo kraljevstvo  nad Izraelom i dao ga tvome susjedu, koji je bolji od tebe."  - 
\par 29 Ipak, Slava Izraelova ne laže i ne kaje se, jer nije čovjek  da bi se kajao. - 
\par 30 Šaul reče: "Sagriješio sam; ali mi sada  učini čast pred starješinama moga naroda i pred Izraelom i vrati  se sa mnom da se poklonim Jahvi, tvome Bogu." 
\par 31 I Samuel se  vrati sa Šaulom i Šaul se pokloni Jahvi. 
\par 32 Potom zapovjedi Samuel: "Dovedite k meni Agaga, amalečkoga  kralja!" I Agag dođe k njemu opirući se i reče: "Zaista, smrt  je gorka!" 
\par 33 Samuel mu odvrati: "Kao što je tvoj mač mnogim  ženama oteo djecu, tako će među ženama tvoja majka ostati bez  djeteta!" I Samuel posiječe Agaga pred Jahvom u Gilgalu. 
\par 34 Potom Samuel ode u Ramu, a Šaul se vrati svojoj kući  u Šaulovu Gibeu. 
\par 35 I Samuel nije više vidio Šaula do svoga  smrtnog dana. Samuel je tugovao zbog Šaula, ali se Jahve pokajao  što je Šaula postavio za kralja nad Izraelom. 


\chapter{16}

\par 1 Jahve reče Samuelu: "Dokle ćeš tugovati zbog Šaula, kad sam  ga ja odbacio da ne kraljuje više nad Izraelom? Napuni uljem  svoj rog i pođi na put! Ja te šaljem Betlehemcu Jišaju, jer sam  između njegovih sinova izabrao sebi kralja." 
\par 2 A Samuel reče:  "Kako bih mogao ići onamo? Šaul će to čuti i ubit će me!" Ali  mu Jahve odgovori: "Uzmi sa sobom junicu pa reci: 'Došao sam  da žrtvujem Jahvi!' 
\par 3 I pozovi Jišaja na žrtvu, a ja ću te sam  poučiti što ćeš činiti: pomazat ćeš onoga koga ti kažem." 
\par 4 Samuel učini kako mu je zapovjedio Jahve. Kad je došao  u Betlehem, gradske mu starješine dršćući dođu u susret i zapitaju:  "Znači li tvoj dolazak dobro?" 
\par 5 Samuel odgovori: "Da, dobro!  Došao sam da žrtvujem Jahvi. Očistite se i dođite sa mnom na  žrtvu!" Potom očisti Jišaja i njegove sinove i pozva ih na žrtvu. 
\par 6 Kad su došli i kad je Samuel vidio Eliaba, reče u sebi:  "Jamačno, evo pred Jahvom stoji njegov pomazanik!" 
\par 7 Ali Jahve  reče Samuelu: "Ne gledaj na njegovu vanjštinu ni na njegov visoki  stas, jer sam ga odbacio. Bog ne gleda kao što gleda čovjek:  čovjek gleda na oči, a Jahve gleda što je u srcu." 
\par 8 Zatim Jišaj  dozva Abinadaba i dovede ga pred Samuela. A on reče: "Ni ovoga  Jahve nije izabrao." 
\par 9 Tada Jišaj dovede Šamu, ali Samuel reče:  "Ni ovoga Jahve nije izabrao." 
\par 10 Tako Jišaj dovede sedam svojih  sinova pred Samuela, ali Samuel reče Jišaju: "Jahve nije izabrao  nijednoga od ovih." 
\par 11 Potom zapita Jišaja: "Jesu li to svi tvoji sinovi?" A  on odgovori: "Ostao je još najmlađi, on je na paši, za stadom."  Tada Samuel reče Jišaju: "Pošalji po njega, jer nećemo sjedati  za stol dok on ne dođe." 
\par 12 Jišaj posla po njega: bio je to  rumen momak, lijepih očiju i krasna stasa. I Jahve reče Samuelu:  "Ustani, pomaži ga: taj je!" 
\par 13 Samuel uze rog s uljem i pomaza  ga usred njegove braće. Duh Jahvin obuze Davida od onoga dana.  A Samuel krenu na put i ode u Ramu. 
\par 14 Duh Jahvin bijaše odstupio od Šaula, a jedan zao duh, od Jahve, stao ga je salijetati. 
\par 15 Tada rekoše Šaulu sluge  njegove: "Evo, zao duh Božji salijeće te. 
\par 16 Zato neka naš gospodar  zapovjedi, pa će sluge tvoje potražiti čovjeka koji zna udarati  u harfu: kad te napadne zao duh Božji, neka onaj udara u harfu  pa će ti biti bolje." 
\par 17 Šaul reče svojim slugama: "Nađite mi čovjeka koji umije  vješto udarati u harfu i dovedite ga k meni!" 
\par 18 Jedan od njegovih  slugu odgovori i reče: "Ja sam vidio jednog sina Betlehemca Jišaja:  on umije udarati u harfu, hrabar je junak i čovjek ratnik, vješt  je govornik, krasna je stasa i Jahve je s njim." 
\par 19 Tada Šaul posla glasnike k Jišaju i poruči mu: "Pošalji  mi svoga sina Davida (koji je kod stada)!" 
\par 20 A Jišaj uze pet  hljebova, mijeh vina i jedno jare i posla Šaulu po svome sinu  Davidu. 
\par 21 Tako David dođe k Šaulu i stupi u njegovu službu.  I Šaul ga veoma zavolje i David posta njegov štitonoša. 
\par 22 Potom  Šaul posla k Jišaju i poruči mu: "Neka David ostane kod mene  u službi, jer je stekao moju naklonost." 
\par 23 I kad god bi Božji  duh napao Šaula, David bi uzeo harfu i svirao; tada bi Šaulu  odlanulo i bilo bi mu bolje, a zao bi duh odlazio od njega. 


\chapter{17}

\par 1 Filistejci skupiše svoje čete za rat i sastaše se kod Soka  u Judeji. Tabor udariše između Soka i Azeke kod Efes Damima. 
\par 2 A Šaul i Izraelci skupiše se i utaboriše u Terebintskoj dolini, i svrstaše se za boj protiv Filistejaca. 
\par 3 Filistejci su stajali  na gori s jedne strane, Izraelci na gori s druge strane, a dolina  bila među njima. 
\par 4 Iz filistejskih redova izađe jedan izazivač. Zvao se Golijat, a bio je iz Gata. Visok bijaše šest lakata i jedan pedalj. 
\par 5 Na  glavi je imao mjedenu kacigu, obučen je bio u ljuskav oklop,  a oklop mu težak pet tisuća mjedenih šekela. 
\par 6 Na nogama je  imao mjedene nogavice, a na ramenima mjedenu sulicu. 
\par 7 Kopljača  njegova koplja bila je kao tkalačko vratilo, a šiljak koplja  težak šest stotina željeznih šekela. Pred njim je stupao štitonoša. 
\par 8 On se postavi pred izraelske bojne redove i dovikne im:  "Što ste izašli da se svrstate za bitku? Nisam li ja Filistejac, a vi Šaulove sluge? Izaberite između sebe jednoga čovjeka pa  neka siđe k meni! 
\par 9 Ako pobijedi u borbi sa mnom i pogubi me, mi ćemo biti vaše sluge. Ako li ja pobijedim njega i pogubim  ga, onda ćete vi biti naše sluge i nama ćete robovati." 
\par 10 Još  je Filistejac rekao: "Ja sam danas izazvao Izraelove bojne redove.  Dajte mi čovjeka da se ogledamo u dvoboju!" 
\par 11 Kad je Šaul i  sav Izrael čuo što je rekao Filistejac, obuze ih strah i drhat. 
\par 12 David je bio sin nekoga Efraćanina iz Betlehema u Judeji;  taj se zvao Jišaj, a imao je osam sinova. Taj je čovjek u Šaulovo  vrijeme bio star i odmakao u godinama. 
\par 13 Tri najstarija Jišajeva  sina bijahu otišla u rat za Šaulom; a ta trojica njegovih sinova  koji bijahu otišli u rat zvahu se: najstariji Eliab, drugi Abinadab, a treći Šama. 
\par 14 David bijaše najmlađi. A tri najstarija bijahu  otišla za Šaulom. - 
\par 15 David je odlazio k Šaulu i vraćao se  iz njegove službe da pase stada svoga oca u Betlehemu. 
\par 16 A  Filistejac izlazio svakoga jutra i večeri i postavljao se tako  četrdeset dana. - 
\par 17 A Jišaj reče svome sinu Davidu: "Uzmi za  svoju braću ovu efu prženoga žita i ovih deset hljebova i odnesi  brže svojoj braći u tabor. 
\par 18 A ovih deset sireva odnesi njihovu  tisućniku. Propitaj se za zdravlje svoje braće i donesi od njih  znak da si izvršio nalog! 
\par 19 Oni su sa Šaulom i svim Izraelom  u Terebintskoj dolini: vojuju protiv Filistejaca." 
\par 20 David ustade u rano jutro, ostavi stado jednom čuvaru, spremi se i ode kako mu bijaše zapovjedio Jišaj. U tabor je  stigao kad je vojska izlazila u bojni red i dizala bojni poklik. 
\par 21 Izraelci i Filistejci svrstaše se u bojni red jedni prema  drugima. 
\par 22 David ostavi svoje stvari čuvaru opreme pa otrča  u bojni red. Došavši, zapita svoju braću za zdravlje. 
\par 23 Dok je s njima govorio, gle, onaj izazivač (zvao se Golijat, Filistejac iz Gata) iziđe iz filistejskih bojnih redova i ponovi  iste riječi kao prije. I David ih je čuo. 
\par 24 A čim su Izraelci  ugledali toga čovjeka, pobjegoše svi daleko od njega i strah  ih uhvati. 
\par 25 Neki Izraelac reče: "Jeste li vidjeli onoga čovjeka  što je izišao? A izišao je da izaziva Izraela. Tko njega pogubi, kralj će mu dati silno blago i dat će mu svoju kćer i oslobodit  će od poreza njegov očinski dom u Izraelu." 
\par 26 Tada David zapita ljude koji stajahu oko njega: "Što  će to dobiti čovjek koji ubije toga Filistejca i skine sramotu  s Izraela? I tko je taj neobrezani Filistejac da izaziva bojne  redove živoga Boga?" 
\par 27 A narod mu odgovori istim riječima kao  prije: "Eto to će dobiti čovjek koji ga pogubi." 
\par 28 A kad je Eliab, njegov najstariji brat, čuo kako se razgovara  s ljudima, usplamtje gnjevom na Davida pa mu reče: "A što si  ti došao ovamo? Kome si ostavio ono malo ovaca u pustinji? Znam  ja tvoju drskost i zlobu tvoga srca: došao si da vidiš bitku!" 
\par 29 A David odgovori: "A što sam učinio? Zar se ne smije ni riječ  reći?" 
\par 30 Tada se okrene od njega k drugome i zapita istim riječima  kao prije. Narod mu odgovori isto kao prvi put. 
\par 31 Kad su ljudi  čuli što je govorio David, jave to Šaulu, a on ga pozva preda  se. 
\par 32 David reče Šaulu: "Neka nikome ne klone srce zbog onoga  čovjeka! Tvoj će sluga izaći i borit će se s tim Filistejcem." 
\par 33 Ali Šaul odvrati Davidu: "Ne možeš ti izaći na toga Filistejca  da se boriš s njim jer si ti još dijete, a on ratnik od svoje  mladosti." 
\par 34 Ali David odgovori Šaulu: "Tvoj je sluga čuvao ovce svome  ocu, pa kad bi došao lav ili medvjed te uhvatio ovcu iz stada, 
\par 35 ja bih potrčao za njim, udario ga i istrgao mu ovcu iz ralja.  A ako bi se on digao na me, uhvatio bih ga za grivu i udarao  ga dok ga ne bih ubio. 
\par 36 I lava je i medvjeda tvoj sluga ubio, pa će i taj neobrezani Filistejac proći kao jedan od njih jer  je izazvao bojne čete Boga živoga." 
\par 37 David još dometne: "Jahve  koji me izbavio iz lavlje pandže i medvjeđe šape izbavit će me  i iz ruku toga Filistejca." Tada Šaul reče Davidu: "Idi i Jahve  neka bude s tobom!" 
\par 38 Šaul obuče Davida u svoju ratnu odoru, na glavu mu ustače  mjedenu kacigu i stavi mu oklop. 
\par 39 Pripasa Davidu svoj mač  preko odore, ali David uzalud pokuša hodati, jer ne bijaše navikao, pa reče Šaulu: "Ne mogu hodati u tome jer nisam navikao." Zato  sve skinu sa sebe. 
\par 40 David uze svoj štap u ruku, izabra u potoku pet glatkih  kamenova i metnu ih u svoju pastirsku torbu, koja mu je služila  kao torba za praćku, te s praćkom u ruci pođe prema Filistejcu. 
\par 41 A Filistejac se sve bliže primicao Davidu, dok je njegov  štitonoša stupao pred njim. 
\par 42 A kad Filistejac pogleda i vidje  Davida, prezre ga s njegove mladosti - bijaše David mladić, rumen, lijepa lica. 
\par 43 Zato Filistejac reče Davidu: "Zar sam ja pseto  te ideš na me sa štapovima?" I uze proklinjati Davida svojim  bogovima. 
\par 44 Zatim Filistejac reče Davidu: "Dođi k meni da dam  tvoje meso pticama nebeskim i zvijerima zemaljskim!" 
\par 45 A David odgovori Filistejcu: "Ti ideš na me mačem, kopljem  i sulicom, a ja idem na te u ime Jahve Sebaota, Boga Izraelovih  četa koje si ti izazvao. 
\par 46 Danas će te Jahve predati u moju  ruku, ja ću te ubiti, skinut ću tvoju glavu i još ću danas tvoje  mrtvo tijelo i mrtva tjelesa filistejske vojske dati pticama  nebeskim i zvijerima zemaljskim. Sva će zemlja znati da ima Bog  u Izraelu. 
\par 47 I sav će ovaj zbor znati da Jahve ne daje pobjedu  mačem ni kopljem, jer je Jahve gospodar bitke i on vas predaje  u naše ruke." 
\par 48 Kad se Filistejac približio i pošao prema Davidu, izađe  David iz bojnih redova i krenu pred Filistejca. 
\par 49 David segnu  rukom u torbu, izvadi iz nje kamen i hitnu ga iz praćke. I pogodi  Filistejca u čelo; kamen mu se zabi u čelo i on pade ničice na  zemlju. 
\par 50 Tako je David praćkom i kamenom nadjačao Filistejca:  udario je Filistejca i ubio ga, a nije imao mača u ruci. 
\par 51 Zato  David potrča i stade na Filistejca, zgrabi njegov mač, izvuče  ga iz korica i pogubi Filistejca odsjekavši mu glavu. Kad Filistejci vidješe kako pogibe njihov junak, nagnuše  u bijeg. 
\par 52 Tada ustadoše Izraelci i Judejci, digoše bojnu viku  i potjeraše Filistejce do opkopa oko Gata i do gradskih vrata  Ekrona; filistejski mrtvaci pokriše put od Šaarajima sve do Gata  i do Ekrona. 
\par 53 Nato se Izraelci vratiše iz te žestoke potjere  za Filistejcima i opljačkaše njihov tabor. 
\par 54 A David uze Filistejčevu  glavu i odnese je u Jeruzalem, a oružje njegovo položi u svoj  šator. 
\par 55 Kad je Šaul vidio Davida gdje izlazi pred Filistejca, upitao je svoga vojvodu Abnera: "Čiji je sin taj mladić, Abnere?"  A Abner je odgovorio: "Tako mi tvoga života, kralju, ne znam!" 
\par 56 A kralj mu reče: "Raspitaj se čiji je sin taj mladić!" 
\par 57 A kad se David vratio pošto je pogubio Filistejca, uze  ga Abner i dovede ga pred Šaula, a u ruci David još držaše Filistejčevu  glavu. 
\par 58 Šaul ga upita: "Čiji si ti sin, momče?" A David odgovori:  "Sin sam tvoga sluge Betlehemca Jišaja." 


\chapter{18}

\par 1 Kad je David završio razgovor sa Šaulom, Jonatanova se duša  prikloni Davidovoj duši i Jonatan ga zavolje kao samoga sebe. 
\par 2 Šaul zadrža Davida onoga istog dana kod sebe i nije mu dao  da se vrati kući svoga oca. 
\par 3 I Jonatan sklopi savez s Davidom  jer ga je ljubio kao samoga sebe. 
\par 4 I skide Jonatan plašt koji  je imao na sebi i dade ga Davidu; tako i svoju odoru, čak i svoj  mač, svoj luk i svoj pojas. 
\par 5 Na svim svojim pohodima, kamo  ga je god slao Šaul, David bijaše sretne ruke i Šaul ga postavi  na čelo svojim ratnicima; omilje on svemu narodu, pa i Šaulovim  dvoranima. 
\par 6 Za njihova povratka, kad se David vraćao ubivši Filistejca, izađoše žene iz svih gradova Izraelovih u susret kralju Šaulu  veselo kličući, pjevajući i plešući uza zvuke bubnjeva i cimbala. 
\par 7 Žene su plešući pjevale: "Pobi Šaul svoje tisuće, David na desetke tisuća." 
\par 8 Šaul se vrlo ražestio, nije mu bila draga ta pjesma. Zato  reče: "Davidu su dale desetke tisuća, a meni samo tisuće! Još  mu samo treba kraljevstvo!" 
\par 9 I od toga dana Šaul poprijeko  gledaše Davida. 
\par 10 Sutradan zao duh Božji napade Šaula, tako da je bjesnio  po kući. David je rukom udarao u harfu kao drugih dana, a Šaul  je u ruci imao koplje. 
\par 11 I Šaul baci koplje govoreći u sebi:  "Sad ću pribiti Davida uza zid!" Ali mu se David izmače dva puta. 
\par 12 Šaul se poče bojati Davida, jer je Jahve bio s njim a  od Šaula je odstupio. 
\par 13 Zato ga Šaul ukloni iz svoje blizine  i postavi ga za tisućnika: on je izlazio i vraćao se na čelu  naroda. 
\par 14 David je imao uspjeha na svim svojim putovima jer  Jahve bijaše s njim. 
\par 15 Kad je Šaul vidio da David ima mnogo  uspjeha, obuze ga strah od njega. 
\par 16 Ali svemu Izraelu i Judi  omilje David jer ih je on vodio na svim njihovim putovima. 
\par 17 Šaul reče Davidu: "Evo svoju najstariju kćer Merabu dat  ću ti za ženu, samo mi budi hrabar i vodi Jahvine bojeve!" Mišljaše  Šaul: "Neću da padne od moje ruke, nego filistejska ruka neka  se digne na njega!" 
\par 18 A David odgovori: "Tko sam ja i što znači  moj život, što li kuća oca mojega u Izraelu da budem kraljev  zet?" 
\par 19 I kad dođe vrijeme da Šaulova kći Meraba pođe za Davida, dadoše je za ženu Adrielu iz Mehole. 
\par 20 Ali je Davida ljubila Šaulova kći Mikala; kad su to javili  Šaulu, bilo mu je pravo. 
\par 21 Reče on u sebi: "Dat ću mu je, ali  će mu ona biti zamka i ruka filistejska dići će se na njega."  (Šaul je po drugi put rekao Davidu: "Danas ćeš mi biti zet.") 
\par 22 Tada Šaul zapovjedi svojim slugama ovako: "Razgovarajte se  s Davidom tajno i recite mu: 'Gle, omilio si kralju i svi te  njegovi dvorani vole; zato budi kraljev zet.'" 
\par 23 I Šaulove  sluge ponoviše te riječi Davidu, ali im David odgovori: "Zar  je u vašim očima malenkost postati kraljev zet? Ja sam samo siromah  i mali čovjek!" 
\par 24 Šaulove sluge dojaviše to Šaulu govoreći:  "Evo riječi što ih je rekao David." 
\par 25 A Šaul odgovori: "Ovako  recite Davidu: 'Kralj ne traži nikakva ženidbenog dara nego samo  sto filistejskih obrezaka da se osveti kraljevim neprijateljima.'"  Šaul mišljaše da će tako Davida gurnuti u ruke Filistejcima. 
\par 26 Šaulove sluge dojaviše te riječi Davidu, a njemu bijaše  po volji da postane kraljev zet. Još prije nego što je isteklo  vrijeme, 
\par 27 spremi se David i krenu sa svojim ljudima te ubi  Filistejcima dvije stotine ljudi; i donese njihove obreske i  predade ih kralju na broj da bi postao njegov zet. Tada mu Šaul  dade svoju kćer Mikalu za ženu. 
\par 28 Šaul je jasno vidio da je Jahve s Davidom i da ga ljubi  sav dom Izraelov. 
\par 29 I Šaul se još većma poboja Davida i posta  neprijatelj Davidu zauvijek. 
\par 30 A filistejski su knezovi izlazili  u boj, ali koliko su god puta izlazili, David je imao više uspjeha  nego svi Šaulovi dvorani; i tako ime njegovo posta vrlo slavno. 


\chapter{19}

\par 1 Šaul razloži svome sinu Jonatanu i svim svojim dvoranima svoju  namjeru da ubije Davida. Ali Jonatan, Šaulov sin, vrlo je volio  Davida. 
\par 2 I Jonatan to javi Davidu ovako: "Moj otac Šaul kani  te ubiti. Budi, dakle, na oprezu sutra ujutro, ostani u skrovištu  i pritaji se. 
\par 3 A ja ću izaći i stajat ću pokraj svoga oca u  polju gdje ti budeš i govorit ću za tebe sa svojim ocem. Kad  saznam kako je, javit ću ti." 
\par 4 Jonatan pohvali Davida svome ocu Šaulu i reče mu ovako:  "Neka se kralj ne ogriješi o svoga slugu Davida jer se on nije  ništa ogriješio o tebe; naprotiv, ono što je radio bilo je od  velike koristi za tebe. 
\par 5 On je stavio život svoj na kocku,  ubio je Filistejca i Jahve je pribavio veliku pobjedu svemu Izraelu:  vidio si i radovao se. Zašto bi se, dakle, ogriješio o nevinu  krv ubijajući Davida bez razloga?" 
\par 6 Šaul posluša Jonatanove  riječi i zakle se: "Živoga mi Jahve, David neće umrijeti!" 
\par 7 Tada  Jonatan dozva Davida i kaza mu sve te riječi. Zatim Jonatan dovede  Davida k Šaulu i David opet dobi službu koju je imao prije. 
\par 8 Kad je rat i opet buknuo, iziđe David na bojište da se  bori s Filistejcima; i porazi ih tako da su pobjegli pred njim. 
\par 9 Tada zao duh Jahvin obuze Šaula: kad je sjedio u svojoj kući, s kopljem u ruci, a David rukom udarao u harfu, 
\par 10 Šaul pokuša  da svojim kopljem pribode Davida uza zid, ali on izmakne Šaulovu  udarcu te se koplje zabode u zid. David pobježe i spasi se. 
\par 11 Iste noći Šaul posla glasnike da nadziru Davidovu kuću  jer je htio da ubije Davida u rano jutro. Ali Davidova žena Mikala  javi to Davidu govoreći: "Ako noćas ne umakneš na sigurno mjesto, sutra ćeš biti mrtav!" 
\par 12 Tada Mikala spusti Davida kroz prozor.  On ode i spasi se bijegom. 
\par 13 A Mikala uze idol, položi ga u postelju, stavi mu oko  glave kozju dlaku i pokri ga pokrivačem. 
\par 14 Kad je Šaul poslao  glasnike da uhvate Davida, ona im reče: "Bolestan je." 
\par 15 Ali  Šaul vrati glasnike natrag da vide Davida i zapovjedi im: "Donesite  ga k meni u postelji da ga ubijem!" 
\par 16 A kad su glasnici ušli, gle: u postelji bješe idol, s kozjom dlakom oko glave! 
\par 17 Tada  Šaul reče Mikali: "Zašto si me tako prevarila i pustila moga  neprijatelja da pobjegne i da se spasi?" A Mikala odgovori Šaulu:  "On mi je rekao: 'Pusti me da odem, ili ću te ubiti!'" 
\par 18 Tako je David pobjegao i spasio se. I ode on k Samuelu  u Ramu i javi mu sve što mu je učinio Šaul. Potom odoše on i  Samuel i nastaniše se u Najotu. 
\par 19 A Šaulu javiše ovako: "Eno  Davida u Najotu u Rami." 
\par 20 Tada Šaul posla glasnike da uhvate  Davida. Kad su oni vidjeli zbor proroka u proročkom zanosu, a  Samuela im na čelu, obuze Božji duh i Šaulove glasnike te i oni  padoše u proročki zanos. 
\par 21 Kad su to javili Šaulu, on posla  druge glasnike, ali i oni padoše u proročki zanos. Potom Šaul  posla i treće glasnike, ali i oni padoše u proročki zanos. 
\par 22 Tada Šaul krenu sam u Ramu i kad dođe do velikog bunara  kod Sekua, zapita: "Gdje su Samuel i David?" I odgovoriše mu:  "Eno ih u Najotu u Rami." 
\par 23 On odmah pođe prema Najotu u Rami.  Ali i njega obuze duh Božji te je išao u proročkom zanosu sve  dok nije došao u Najot u Rami. 
\par 24 Tu i on svuče svoje haljine  jer i njega obuze zanos pred Samuelom; zatim je legao gol i ostao  tako cio onaj dan i svu noć. Tako je nastala uzrečica: "Zar je i Šaul među prorocima?" 


\chapter{20}

\par 1 David pobježe iz Najota u Rami i dođe k Jonatanu te mu reče:  "Što sam učinio? Kakva je bila moja krivica i što sam zgriješio  tvome ocu da traži moj život?" 
\par 2 A on mu odgovori: "Daleko od  tebe ta misao! Ti nećeš poginuti. Eto, moj otac ne poduzima ništa, bilo veliko ili ne bilo, a da to meni ne otkrije. Zašto bi,  dakle, moj otac krio od mene upravo to? Neće to biti!" 
\par 3 Ali  se David zakle i reče: "Tvoj otac dobro zna da sam ja stekao  tvoju naklonost, pa misli: 'Ne treba da Jonatan išta zna o tome, da ne bude žalostan.' Ali živoga mi Jahve i života mi tvoga, ima samo jedan korak između mene i smrti." 
\par 4 Tada Jonatan upita Davida: "Što želiš da učinim za tebe?" 
\par 5 A David odgovori Jonatanu: "Evo, sutra je mladi mjesec i ja  bih morao jesti s kraljem za stolom; ali me ti pusti da odem, da se sakrijem u polju do večera. 
\par 6 Ako tvoj otac opazi da  me nema, reći ćeš mu ovako: 'David me uporno molio da ga pustim  da skokne u svoj grad Betlehem, jer se ondje slavi godišnja žrtva  za svu njegovu obitelj.' 
\par 7 Ako on rekne: 'Dobro!', tvoj je sluga  spašen. Ako li plane gnjevom, znaj da je čvrsto naumio da me  pogubi. 
\par 8 Iskaži, dakle, milost svome sluzi kad si slugu svoga  uveo sa sobom u savez Jahvin. Ali ako ima kakva krivica na meni, ubij me sam; zašto bi me vodio k svome ocu?" 
\par 9 A Jonatan mu odgovori: "Daleko od tebe ta misao! Kad bih  ja pouzdano znao da je moj otac čvrsto naumio da na tebe svali  nesreću, zar ti ja ne bih dojavio?" 
\par 10 David upita Jonatana: "A tko će mi javiti ako ti tvoj  otac odgovori što zlo?" 
\par 11 Jonatan odgovori Davidu: "Hodi, izađimo u polje!" I izađu  obojica u polje. 
\par 12 Tada Jonatan reče Davidu: "Jahve, Bog Izraelov, neka mi bude svjedok! Ja ću iskušati svoga oca sutra u ovo doba.  Ako bude dobro po Davida, a ja ne pošaljem k tebi da te obavijestim, 
\par 13 neka Jahve učini to zlo Jonatanu i neka mu doda drugo zlo!  Ako li mome ocu bude drago da ti učini zlo, javit ću ti i pustit  ću te da odeš u miru; i Jahve neka bude s tobom kao što je bio  s mojim ocem! 
\par 14 Ako ja još budem živ, moći ćeš mi iskazati  milosrđe Jahvino; ako li umrem, 
\par 15 ne uskrati svoje dobrote  mome domu dovijeka! Kad Jahve redom iskorijeni Davidove neprijatelje  s lica zemlje, 
\par 16 neka ime Jonatanovo ne iščezne s domom Šaulovim, inače će Jahve tražiti o tome račun od Davida." 
\par 17 Tada se Jonatan još jednom zakune Davidu ljubavlju svojom, jer ga je ljubio svom ljubavlju duše svoje. 
\par 18 Potom reče Jonatan Davidu: "Sutra je mladi mjesec i opazit  će se da te nema, jer će tvoje mjesto biti prazno. 
\par 19 Prekosutra  će se još očitije vidjeti da te nema, a ti dođi na mjesto gdje  si se bio sakrio u dan onoga događaja i sjedni kraj onoga humka  što ga znaš. 
\par 20 A ja ću prekosutra izmetati strijele na onu  stranu kao da gađam onamo. 
\par 21 A onda ću poslati momka i reći  mu: 'Idi! Nađi strijelu!' Ako onda doviknem momku: 'Pazi, strijela  je ovamo bliže od tebe, donesi je!' - ti onda dođi, jer je za  tebe dobro i nema nikakve opasnosti, tako mi Jahve živoga! 
\par 22 Ako  li doviknem momku: 'Pazi, strijela je onamo dalje od tebe!' -  ti onda otiđi, jer te Jahve šalje odavde. 
\par 23 A za ovaj dogovor  što smo ga ugovorili ja i ti neka je Jahve svjedok između mene  i tebe dovijeka!" 
\par 24 Potom se David sakri u polju. Kad je došao mlađak, kralj  je sjeo za stol da jede. 
\par 25 Kralj sjede na svoje obično mjesto, na mjesto uza zid, Jonatan se smjesti sučelice njemu, Abner  sjede kraj Šaula, a Davidovo mjesto osta prazno. 
\par 26 Ali Šaul  ne reče ništa onaj dan jer mišljaše: "Dogodilo mu se štogod,  bit će da nije čist." 
\par 27 Sutradan iza mladog mjeseca, drugi dan u mjesecu, opet  Davidovo mjesto osta prazno, i Šaul upita svoga sina Jonatana:  "Zašto Jišajev sin nije došao na objed ni jučer ni danas?" 
\par 28 A Jonatan odgovori Šaulu: "David me uporno molio da ga  pustim da ide u Betlehem. 
\par 29 Rekao mi je: 'Pusti me da idem  jer slavimo obiteljsku žrtvu u mom gradu i moja su me braća pozvala  da dođem. Ako sam, dakle, stekao tvoju naklonost, daj mi dopust  da pohodim svoju braću.' Eto, zato ga nema kod kraljeva stola." 
\par 30 Tada Šaul planu gnjevom na Jonatana i reče mu: "Izrode  i propalico! Misliš da ne znam da si u savezu s Jišajevim sinom, na sramotu svoju i na sramotu majčinu krilu! 
\par 31 Jer dokle god  bude živ na zemlji Jišajev sin, nećeš biti siguran ni ti ni tvoje  kraljevstvo. Zato sad pošalji po njega i dovedi ga k meni jer  je osuđen na smrt." 
\par 32 A Jonatan odvrati svome ocu Šaulu i reče mu: "Zašto on  mora umrijeti? Što je učinio?" 
\par 33 Tada Šaul izmetnu koplje na  sina da ga probode. Jonatan vidje da je njegov otac odlučio da  ubije Davida. 
\par 34 Jonatan ustade od stola sav jarostan i nije  jeo ništa toga drugog dana u mjesecu jer se zabrinuo za Davida  što ga je njegov otac pogrdio. 
\par 35 Sutradan ujutro izađe Jonatan u polje prema dogovoru  s Davidom; s njim je išao mlad momak. 
\par 36 I on reče svome momku:  "Ti ćeš otrčati i naći strijele koje ću sada izmetnuti." I momak  otrča, a Jonatan odape strijelu tako da je preletjela preko njega. 
\par 37 Kad je momak došao do mjesta gdje je bila strijela koju je  izbacio Jonatan, viknu Jonatan za momkom: "Nije li strijela onamo  dalje od tebe?" 
\par 38 Još Jonatan viknu za momkom: "Brže! Požuri  se! Ne stoj!" Jonatanov momak diže strijelu i donese je svome  gospodaru. 
\par 39 Momak nije ništa opazio, samo su Jonatan i David  znali o čemu se radi. 
\par 40 Nato Jonatan preda oružje momku i reče mu: "Idi i odnesi  to u grad!" 
\par 41 Kad je momak otišao, David iziđe iza humka, pade  ničice na zemlju i pokloni se tri puta. Potom se izljubiše i  plakahu zajedno dok se nisu isplakali. 
\par 42 Zatim Jonatan reče  Davidu: "Idi u miru! Što smo se obojica zakleli Jahvinim imenom, neka Jahve bude svjedok između mene i tebe, između moga potomstva  i tvoga potomstva dovijeka!" (21:1) Nato David usta i ode, a Jonatan se vrati u grad. 


\chapter{21}

\par 1 (21:2) David dođe u Nob k svećeniku Ahimeleku. Ovaj dršćući pođe  u susret Davidu i upita ga: "Zašto si sam i nema nikoga s tobom?" 
\par 2 (21:3) A David odgovori svećeniku Ahimeleku: "Kralj mi je dao nalog  i rekao mi: 'Nitko neka ništa ne dozna zašto te šaljem i što  sam ti zapovjedio!' A momke sam poslao da me dočekaju na tom  i tom mjestu. 
\par 3 (21:4) A sada, ako imaš pri ruci pet hljebova, daj  mi ih, ili što god se nađe!" 
\par 4 (21:5) A svećenik odgovori Davidu: "Nemam  pri ruci običnoga kruha nego samo svetoga kruha; ali samo ako  su se tvoji momci uzdržali od žena." 
\par 5 (21:6) David odgovori svećeniku ovako: "Sasvim pouzdano! Žene  su nam bile uskraćene, kao uvijek kad izlazimo na vojni pohod, i tijela su u momaka čista. Iako je ovo običan put, uistinu  su danas čisti tijelom." 
\par 6 (21:7) Tada mu svećenik dade svetoga kruha, jer nije bilo drugoga kruha ondje osim žrtvenoga, onoga koji  se uklanjao ispred Jahve da se zamijeni toplim kruhom u dan kad  se uzima. 
\par 7 (21:8) Ondje je istoga dana bio jedan od Šaulovih slugu, zadržao  se pred Jahvom; zvao se Doeg Edomac, a bio je nadglednik Šaulovih  pastira. 
\par 8 (21:9) David upita Ahimeleka: "A nemaš li ovdje pri ruci kakvo  koplje ili mač? Nisam uzeo sa sobom ni svoga mača ni svoga oružja, jer je kraljev nalog bio hitan." 
\par 9 (21:10) A svećenik mu odgovori:  "Ovdje je mač Filistejca Golijata, onoga koga si ubio u Terebintskoj  dolini; zamotan je u plašt i položen iza oplećka; ako ga hoćeš  uzeti, uzmi ga samo, jer drugoga osim njega nema ovdje." A David  odvrati: "Takva više nema, daj mi ga!" 
\par 10 (21:11) Potom David ustade i pobježe onaj dan daleko od Šaula  i dođe Akišu, kralju Gata. 
\par 11 (21:12) A dvorani Akiševi rekoše svome  kralju: "Nije li to David, kralj zemlje? To je onaj o kome su  plešući pjevali: 'Pobi Šaul svoje tisuće, David na desetke tisuća.'" 
\par 12 (21:13) David se zamisli o tim riječima i silno se uplaši gatskoga  kralja Akiša. 
\par 13 (21:14) Tada se David poče pretvarati pred njima kao  da je umobolan i vladati se kao luđak u njihovim rukama: bubnjao  je po vratima i puštao da mu teče slina niz bradu. 
\par 14 (21:15) Tada Akiš reče svojim dvoranima: "Vidite dobro da je  čovjek lud! Zašto ga dovodite k meni? 
\par 15 (21:16) Zar nemam dosta budala  te mi dovodite ovoga da mi dosađuje svojim ludilom? Zar će taj  ući u moju kuću?" 


\chapter{22}

\par 1 David ode odande i skloni se u spilju Adulam. A kad su to  čula njegova braća i sva njegova obitelj, dođoše onamo da mu  se priključe. 
\par 2 Osim toga skupiše se oko njega svi koji bijahu  u nevolji, svi zaduženi, svi nezadovoljni, i on im posta vođom.  A bijaše ih oko njega do četiri stotine ljudi. 
\par 3 Odande ode David u Mispu u zemlji moapskoj i reče kralju  moapskome: "Dopusti da se moj otac i moja mati sklonu kod vas  dok ne vidim što će Bog učiniti sa mnom." 
\par 4 I ostavi ih kod  kralja moapskoga i oni ostadoše kod njega sve dok David bijaše  u skrovištu. 
\par 5 Ali prorok Gad reče Davidu: "Nemoj ostati u svome skrovištu, nego idi i zađi u zemlju Judinu." I David ode i zađe u Heretsku  šumu. 
\par 6 Šaul doznade da se pojavio David s ljudima koji bijahu  s njim. Šaul je upravo bio u Gibei; sjedio je pod tamariskom  na uzvišici, s kopljem u ruci, a oko njega stajali svi njegovi  dvorani. 
\par 7 I reče Šaul svojim dvoranima koji stajahu oko njega:  "Poslušajte me, sinovi Benjaminovi! Hoće li vam i Jišajev sin  svima darovati njive i vinograde? Hoće li vas sve postaviti za  tisućnike i stotnike? 
\par 8 A zašto ste se onda svi urotili protiv  mene? Nema nikoga da mi dojavi kad moj sin sklapa savez s Jišajevim  sinom, nema nikoga među vama da me požali i da mi otkrije kako  je moj sin podjario moga slugu na me, kao što se događa danas." 
\par 9 Tada progovori Doeg Edomac, koji je stajao među Šaulovim  dvoranima, i reče: "Ja sam vidio Jišajeva sina kad je došao u  Nob k Ahimeleku, Ahitubovu sinu. 
\par 10 Ovaj je zatražio za njega  savjet od Jahve i dao mu hrane i predao mu mač Filistejca Golijata." 
\par 11 Šaul nato zapovjedi da pozovu svećenika Ahimeleka, Ahitubova  sina, i svu njegovu obitelj, svećenike u Nobu. I dođoše svi pred  kralja. 
\par 12 Tada reče Šaul: "Čuj me, Ahitubov sine!" A on odgovori:  "Evo me, gospodaru!" 
\par 13 A Šaul ga upita: "Zašto ste se urotili  protiv mene, ti i Jišajev sin? Ti si mu dao kruha i mač i tražio  si za njega savjet od Boga da se digne protiv mene kao neprijatelj, kao što se danas događa." 
\par 14 Ahimelek odgovori kralju: "A tko je među svim tvojim  slugama ravan Davidu, tako vjeran, uz to kraljev zet, glavar  tvoje tjelesne straže, čovjek koji je poštovan u tvojoj kući? 
\par 15 Zar sam danas prvi put tražio za njega savjet od Boga? Daleko  od mene svaka druga misao! Neka kralj ništa ne okrivljuje svoga  sluge i sve njegove obitelji, jer sluga njegov nije znao od svega  toga ništa!" 
\par 16 Ali kralj odvrati: "Ti ćeš umrijeti, Ahimeleče, ti i  sva tvoja obitelj!" 
\par 17 I kralj zapovjedi glasonošama koji stajahu oko njega:  "Pristupite i pogubite svećenike Jahvine jer su i oni pomogli  Davidu: znali su da je na bijegu, a nisu mi to dojavili." Ali  kraljevi stražari ne htjedoše dići ruke na Jahvine svećenike  da ih smaknu. 
\par 18 Tada kralj zapovjedi Doegu: "Pristupi ti i  smakni svećenike!" Doeg Edomac pristupi i smaknu svećenike: on  pogubi u onaj dan osamdeset i pet ljudi koji su nosili laneni  oplećak. 
\par 19 I Nob, svećenički grad, pohara oštricom mača, pobivši  muškarce i žene, djecu i dojenčad, goveda, magarce i ovce. 
\par 20 Izbavio se samo jedan sin Ahimeleka, Ahitubova sina,  po imenu Ebjatar i pobjegao k Davidu. 
\par 21 Ebjatar javi Davidu  da je Šaul poklao Jahvine svećenike. 
\par 22 A David odvrati Ebjataru:  "Ja sam već onoga dana kad ondje bijaše Doeg Edomac znao da će  on zacijelo javiti to Šaulu! Ja sam kriv za živote tvoga očinskog  doma. 
\par 23 Ostani kod mene, ne boj se: tko bude tražio tvoj život, tražit će moj. Kod mene ćeš biti dobro čuvan." 


\chapter{23}

\par 1 Javiše onda Davidu: "Filistejci opsjedaju Keilu i pljačkaju  gumna." 
\par 2 David tada upita Jahvu: "Treba li da idem na Filistejce  i hoću li ih potući?" A Jahve odgovori Davidu: "Idi, potući ćeš  Filistejce i oslobodit ćeš Keilu." 
\par 3 Ali rekoše Davidu ljudi  njegovi: "Gle, mi smo već ovdje, u Judi, u neprestanom strahu;  što će tek biti ako odemo u Keilu protiv filistejskih četa!" 
\par 4 Zato David još jednom upita Jahvu, a Jahve mu odgovori ovako:  "Ustani i siđi u Keilu jer ću predati Filistejce u tvoje ruke!" 
\par 5 David onda krenu sa svojim ljudima u Keilu, udari na Filistejce, otjera njihovu stoku i zada im težak poraz. Tako je David oslobodio  građane Keile. - 
\par 6 Kad je ono Ebjatar, Ahimelekov sin, pobjegao  k Davidu, on je došao u Keilu noseći u ruci oplećak. 
\par 7 Kad su Šaulu javili da je David ušao u Keilu, reče Šaul:  "Bog ga je predao u moje ruke jer se sam uhvatio u zamku kad  je ušao u grad s vratima i prijevornicama." 
\par 8 I Šaul sazva sav  narod na oružje da ide na Keilu i da opkoli Davida i njegove  ljude. 
\par 9 Kad je David doznao da mu Šaul snuje zlo, reče svećeniku  Ebjataru: "Donesi oplećak!" 
\par 10 Nato se David pomoli: "Jahve, Bože Izraelov, tvoj je sluga čuo da Šaul sprema navalu na Keilu  da razori grad zbog mene. 
\par 11 Hoće li Šaul doći kao što je tvoj  sluga čuo? Jahve, Bože Izraelov, odgovori svome sluzi!" A Jahve  odgovori: "Doći će!" 
\par 12 David opet upita: "Hoće li me prvaci  Keile predati, mene i moje ljude, u Šaulove ruke?" A Jahve odgovori:  "Predat će vas!" 
\par 13 Tada David ustade sa svojim ljudima, bijaše  ih oko šest stotina; iziđoše iz Keile te lutahu kojekuda. A kad  su Šaulu javili da je David utekao iz Keile, odusta od vojnog  pohoda. 
\par 14 David se skloni u pustinju u gorska skloništa; nastani  se na gori u pustinji Zifu. Šaul ga je neprestano tražio, ali  ga Bog ne predade u njegove ruke. 
\par 15 David se bojao što je Šaul izišao na vojnu da napadne  na njegov život. Zato je David ostao u pustinji Zifu, u Horši. 
\par 16 Tada Šaulov sin Jonatan krenu na put i dođe k Davidu  u Horšu i ohrabri ga u ime Božje. 
\par 17 Reče mu: "Ne boj se, jer  te neće stići ruka moga oca Šaula. Ti ćeš kraljevati nad Izraelom, a ja ću biti drugi do tebe; i moj otac Šaul zna to dobro." 
\par 18 I  sklopiše njih dvojica savez pred Jahvom. David osta u Horši,  a Jonatan ode svojoj kući. 
\par 19 Jednoga dana dođoše Zifejci k Šaulu u Gibeu i javiše  mu: "David se krije kod nas u gorskim skloništima u Horši, na  brdu Hakili, što je južno od Ješimona. 
\par 20 Sada, kralju, kad  god zaželiš sići, siđi, a naše je da ga predamo u ruke kralju." 
\par 21 A Šaul odgovori: "Blagoslovio vas Jahve što ste me požalili! 
\par 22 Idite, dakle, raspitajte se još i dobro razvidite mjesto  kamo ga donesu njegovi hitri koraci; rekli su mi da je vrlo lukav. 
\par 23 Zato pretražite sve rupe u koje se zavlači, pa se vratite  k meni kad budete pouzdano znali. Tada ću ja poći s vama, pa  ako bude gdje u zemlji, ići ću za njegovim tragom po svim Judinim  rodovima." 
\par 24 Tada krenuše na put i odoše u Zif, pred Šaulom. David  je sa svojim ljudima bio u pustinji Maonu u Arabi, južno od Ješimona. 
\par 25 Potom i Šaul pođe sa svojim ljudima da traži Davida. Kad  su to javili Davidu, siđe on u klanac koji leži u pustinji Maonu.  Šaul to doznade i krenu u potjeru za Davidom u pustinju Maon. 
\par 26 Šaul je sa svojim ljudima išao jednom stranom planine, a  David sa svojim ljudima drugom stranom planine. David se silno  žurio da umakne Šaulu. Kad je Šaul sa svojim ljudima htio prijeći  na drugu stranu da opkoli Davida i njegove ljude i da ih pohvata, 
\par 27 dođe glasnik Šaulu s porukom: "Dođi brže, Filistejci provališe  u zemlju!" 
\par 28 Tada Šaul odusta od potjere za Davidom i okrenu  se protiv Filistejaca. Zato se prozvalo ono mjesto "Klanac razlaza". 
\par 29 (24:1) David se odande uspe i nastani u engadskim gorskim skloništima. 


\chapter{24}

\par 1 (24:2) Kad se Šaul vratio iz potjere za Filistejcima, javiše mu ovo:  "David je u Engadskoj pustinji!" 
\par 2 (24:3) Tada Šaul uze tri tisuće  odabranih ljudi iz svega Izraela i pođe da traži Davida i njegove  ljude na istok od Litica divokoza. 
\par 3 (24:4) Idući dođe k ovčjim torovima  pokraj puta; ondje bijaše pećina i Šaul uđe da čučne; a David  je sa svojim ljudima sjedio u dnu pećine. 
\par 4 (24:5) I rekoše Davidu  ljudi njegovi: "Evo dana za koji ti je rekao Jahve: 'Ja ću predati  tvoga neprijatelja u tvoje ruke, postupaj s njim kako ti se mili!'"  A David ustade i neprimjetno odsiječe skut od Šaulova plašta. 
\par 5 (24:6) Ali poslije zapeče Davida savjest što je odsjekao skut od  Šaulova plašta, 
\par 6 (24:7) pa reče svojim ljudima: "Očuvao me Jahve da  takvo što učinim svome gospodaru, da dignem ruku na njega, jer  je pomazanik Jahvin." 
\par 7 (24:8) I David oštrim riječima ukori svoje  ljude i ne dopusti im da ustanu na Šaula. A Šaul izađe iz pećine i pođe svojim putem. 
\par 8 (24:9) Zatim ustade  David, iziđe iz pećine i vikne za Šaulom: "Gospodaru kralju!"  A kad se Šaul obazre, David se baci ničice na zemlju i pokloni  mu se. 
\par 9 (24:10) Tada David reče Šaulu: "Zašto slušaš ljude koji ti  govore da David snuje tebi propast? 
\par 10 (24:11) Gle, upravo u ovaj dan  tvoje su oči mogle vidjeti da te Jahve predao danas u moje ruke  u ovoj pećini. Rekoše mi da te ubijem, ali te poštedjeh i rekoh:  'Neću dići svoje ruke na svoga gospodara, jer je Jahvin pomazanik.' 
\par 11 (24:12) O, moj oče, pogledaj i vidi skut od svoga plašta u mojoj  ruci: odsjekao sam skut od tvoga plašta, a tebe nisam ubio; spoznaj  i vidi da u mojoj ruci nema ni zlobe ni opačine. Ja nisam zgriješio  protiv tebe, a ti vrebaš na moj život da mi ga uzmeš! 
\par 12 (24:13) Jahve  neka sudi između mene i tebe, Jahve neka me osveti na tebi, ali  se moja ruka neće dići na tebe. 
\par 13 (24:14) Kako kaže stara poslovica:  od nepravednika dolazi nepravda, i zato se moja ruka neće dići  protiv tebe. 
\par 14 (24:15) Za kim je izišao izraelski kralj? Za kim ideš  u potjeru? Za mrtvim psom, za običnom buhom! 
\par 15 (24:16) Jahve neka bude  sudac, on neka sudi između mene i tebe, neka ispita i brani moju  stvar i neka mi pribavi pravdu: neka me izbavi iz tvoje ruke!" 
\par 16 (24:17) Kad je David izgovorio te riječi Šaulu, odvrati Šaul:  "Je li to tvoj glas, sine Davide?" I Šaul glasno zaplaka. 
\par 17 (24:18) Zatim  reče Davidu: "Pravedniji si od mene jer ti si meni učinio dobro, a ja sam tebi učinio zlo. 
\par 18 (24:19) A danas si okrunio svoju dobrotu  prema meni, jer me Jahve predao u tvoje ruke, a ti me nisi ubio. 
\par 19 (24:20) Kad se čovjek namjeri na svoga neprijatelja, pušta li ga  da ide mirno svojim putem? Neka ti Jahve naplati za ono dobro  što si mi danas učinio! 
\par 20 (24:21) Sada pouzdano znam da ćeš zacijelo  biti kralj i da će se kraljevstvo nad Izraelom trajno održati  u tvojoj ruci. 
\par 21 (24:22) Zato mi se sada zakuni Jahvom da nećeš zatrti  moga potomstva poslije mene i da nećeš izbrisati moga imena iz  moga očinskoga doma!" 
\par 22 (24:23) David se zakle Šaulu, Šaul ode svojoj kući, a David se  sa svojim ljudima vrati u gorska skloništa. 


\chapter{25}

\par 1 Uto umrije Samuel. Sav se Izrael skupi i oplaka ga naričući  za njim; i pokopaše ga u njegovu zavičaju u Rami. A David usta  i siđe u pustinju Paran. 
\par 2 U Maonu živio čovjek koji je imao svoje gospodarstvo u  Karmelu; bio je to vrlo bogat čovjek, imao je tri tisuće ovaca  i tisuću koza. Upravo je tada strigao svoje ovce u Karmelu. 
\par 3 Taj  se čovjek zvao Nabal, a njegova žena Abigajila. Žena je bila  mudra i vrlo lijepa, a čovjek surov i opak: bio je Kalebovac. 
\par 4 David je u pustinji čuo da Nabal striže svoje ovce. 
\par 5 Stoga  posla deset momaka naloživši im: "Idite gore u Karmel, otiđite  k Nabalu i pozdravite ga u moje ime. 
\par 6 I recite ovako mome bratu:  'Mir tebi, mir tvome domu, mir svemu što imaš! 
\par 7 Sada, čujem, strižeš ovce. A tvoji su pastiri bili kod nas, nismo ih dirali, ništa im nije nestalo dokle god su bili u Karmelu. 
\par 8 Pitaj  svoje sluge i kazat će ti. Zato neka ovi momci nađu milost pred  tobom, jer smo došli u svečan dan. Podaj svojim slugama i svome  sinu Davidu što ti se nađe pri ruci.'" 
\par 9 Dođoše momci Davidovi i ponoviše Nabalu u Davidovo ime  sve ove riječi, a onda pričekaše. 
\par 10 Ali Nabal odgovori Davidovim  slugama ovako: "Tko je David, tko je Jišajev sin? Danas ima mnogo  slugu koji su pobjegli od svojih gospodara. 
\par 11 Zar da uzmem  svoj kruh, svoju vodu, svoju stoku koju sam poklao za svoje strigače  pa da to poklonim ljudima o kojima ne znam ni odakle su?" 
\par 12 Davidovi se momci okrenuše i vratiše se svojim putem.  Kad su se vratili, javiše sve ove riječi Davidu. 
\par 13 A David  reče svojim ljudima: "Pripašite svaki svoj mač!" I pripasaše  svaki svoj mač, i David pripasa svoj, i oko četiri stotine ljudi  krenu za Davidom, dok ih dvije stotine osta kod tovara. 
\par 14 A ženi Nabalovoj, Abigajili, javio jedan od Nabalovih  slugu ovo: "Eto, David je poslao iz pustinje glasnike da pozdrave  našega gospodara, a on ih potjerao. 
\par 15 A ti su ljudi bili vrlo  dobri prema nama: nisu nas dirali, ništa nismo izgubili dokle  god smo bili u njihovoj blizini kad smo bili u polju. 
\par 16 Noću  i danju bili su nam kao bedem u sve vrijeme dok smo bili s njima  pasući stada. 
\par 17 Razmisli sada i vidi što ćeš učiniti, jer je  gotova pogibija našem gospodaru i svemu njegovu domu; a on je  opak čovjek komu se ne može ništa kazati." 
\par 18 Abigajila brzo uze dvije stotine hljebova, dva mijeha  vina, pet zgotovljenih ovaca, pet mjera pržena žita, sto grozdova  suhoga grožđa, dvije stotine smokovih kolača i sve to natovari  na magarce. 
\par 19 I zapovjedi svojim slugama: "Idite preda mnom, a ja ću za vama." Svome mužu Nabalu nije kazala ništa. 
\par 20 Dok je, jašući na magarcu, silazila iza gorskog zavoja, David je sa svojim ljudima silazio nasuprot njoj, tako da se  ona susrela s njima. 
\par 21 A David je upravo mislio: "Uzalud sam, dakle, zaštićivao u pustinji sve što je taj čovjek imao i ništa  mu nije nestalo od svega što je posjedovao! Sada mi vraća zlo  za dobro! 
\par 22 Neka Bog učini Davidu ovo zlo i neka mu doda drugo  ako Nabalu do zore od svega što ima ostavim i ono što mokri uza  zid!" 
\par 23 Kad je Abigajila ugledala Davida, brzo sjaha s magarca  i pade pred Davida ničice, poklonivši se do zemlje. 
\par 24 Bacivši  mu se tako pred noge, reče: "Gospodaru, neka na mene padne krivica!  Dopusti da službenica tvoja progovori tvojim ušima i udostoj  se poslušati riječi službenice svoje! 
\par 25 Neka moj gospodar ne  gleda na toga opakog čovjeka, na Nabala, jer on s pravom nosi  svoje ime: zove se Luda i ludost je s njim. A ja, službenica  tvoja, nisam vidjela momaka koje je poslao moj gospodar. 
\par 26 Zato  sada, gospodaru, živoga mi Jahve, i tako živ bio ti, i tako ti  Jahve koji te očuvao da ne svališ na se krvnu krivicu i da ne  pribaviš sebi pravdu svojom rukom: neka prođu kao Nabal tvoji  neprijatelji i oni koji snuju zlo mome gospodaru! 
\par 27 A ovaj  dar, što ga evo tvoja službenica nosi svome gospodaru, neka se  dade momcima koji idu za mojim gospodarom na njegovim putovima. 
\par 28 Oprosti službenici svojoj njezinu krivnju! Zacijelo će Jahve  osnovati trajan dom mome gospodaru, jer moj gospodar bije Jahvine  bojeve i za svega tvoga života neće se naći zlo na tebi. 
\par 29 Ako  se tko digne da te progoni i da ti radi o glavi, neka život moga  gospodara bude pohranjen u škrinji života kod Jahve, tvoga Boga, a život tvojih neprijatelja neka on baci kao iz praćke. 
\par 30 I  kad Jahve učini mome gospodaru svako dobro koje ti je obećao  i kad te odredi da budeš knezom nad Izraelom, 
\par 31 onda neka ne  bude na smutnju ni na grižnju savjesti mome gospodaru da je ni  za što prolio krv i da je sebi pribavio pravdu svojoj rukom.  I kad Jahve učini dobro mome gospodaru, sjeti se tada službenice  svoje!" 
\par 32 David odgovori Abigajili: "Neka je blagoslovljen Jahve, Bog Izraelov, koji te danas poslao meni u susret! 
\par 33 Neka je  blagoslovljena tvoja mudrost i blagoslovljena bila ti što si  me danas zadržala da ne svalim na se krvnu krivicu i da ne pribavim  sebi pravdu svojom rukom. 
\par 34 Ali, tako mi živog Jahve, Boga  Izraelova, koji nije dopustio da ti učinim zlo: da mi nisi tako  brzo izišla u susret, zaista ne bi Nabalu do jutra ostalo ni  ono što uza zid mokri!" 
\par 35 Nato David primi iz njezine ruke što mu bijaše donijela  i reče joj: "Vrati se s mirom svojoj kući. Gle, uslišao sam tvoj  glas i obazreo se na tebe." 
\par 36 Kad se Abigajila vratila k Nabalu, on je upravo imao  gozbu u kući, pravu kraljevsku gozbu: Nabal bijaše veseo i sasvim  pijan; zato mu ona ne reče ništa dok nije svanulo jutro. 
\par 37 A  ujutro, kad se Nabal otrijeznio, pripovjedi mu njegova žena sve  što se dogodilo, a njemu obamrije srce u grudima i on osta kao  da se skamenio. 
\par 38 A desetak dana poslije toga Jahve udari Nabala  te umrije. 
\par 39 Kad David ču da je umro Nabal, reče: "Neka je blagoslovljen  Jahve, koji mi je ispravio nepravdu što mi je učini Nabal; i  Jahve je očuvao svoga slugu da ne učini zla, a svalio je Nabalovu  zloću na njegovu glavu!" Potom David posla poruku Abigajili da će je uzeti za ženu. 
\par 40 Davidove sluge dođoše k Abigajili u Karmel i rekoše joj:  "David nas je poslao k tebi da te uzme sebi za ženu." 
\par 41 A ona  ustade, pokloni se do zemlje i reče: "Evo službenice tvoje koja  je spremna da bude robinja i da pere noge slugama svoga gospodara!" 
\par 42 Potom Abigajila brzo ustade i zajaha na magarca, a za njom  pođe pet njezinih dvorkinja. Tako je otišla za Davidovim poslanicima  i postala njegovom ženom. 
\par 43 I Ahinoamom iz Jizreela bijaše se oženio David i obje  mu bjehu žene. 
\par 44 Jer Šaul bijaše svoju kćer Mikalu, Davidovu  ženu, dao Paltiju, sinu Lajiša iz Galima. 


\chapter{26}

\par 1 Ljudi iz Zifa dođoše Šaulu i javiše mu: "David se krije na  Hakilskom brdu, nasuprot Ješimonu." 
\par 2 Šaul tada krenu na put  i siđe u pustinju Zif, a s njim tri tisuće izabranih Izraelaca, da traži Davida u pustinji Zifu. 
\par 3 Šaul se utabori podno Hakilskog  brda, koje je nasuprot Ješimonu, kraj puta. David, koji je boravio  u pustinji, opazi da je Šaul došao onamo da ga progoni. 
\par 4 Zato  David posla uhode i sazna da je Šaul zaista došao. 
\par 5 David se  podiže i dođe do mjesta gdje se Šaul bio utaborio. Tu David ugleda  mjesto gdje su spavali Šaul i Abner, sin Nerov, njegov vojvoda:  Šaul je spavao usred tabora, a vojska ležala u krugu oko njega. 
\par 6 David se obrati Hetitu Ahimeleku i Abišaju, sinu Sarvijinu  a bratu Joabovu, i reče im: "Tko će sa mnom u tabor sve do Šaula?"  A Abišaj odgovori: "Ja ću s tobom." 
\par 7 I tako David i Abišaj  dopriješe noću do vojske: i gle, Šaul ležaše i spavaše u taboru, a koplje mu kod uzglavlja zabodeno u zemlju. Abner i vojnici  ležahu oko njega. 
\par 8 Tada Abišaj reče Davidu: "Danas ti je Bog predao tvoga  neprijatelja u tvoje ruke; zato sada dopusti da ga njegovim vlastitim  kopljem pribodem za zemlju, jednim jedinim udarcem, drugoga mi  neće trebati." 
\par 9 Ali David odgovori Abišaju: "Nemoj ga ubijati!  Jer tko će dignuti ruku svoju na Jahvina pomazanika i ostati  nekažnjen?" 
\par 10 Još nastavi David: "Živoga mi Jahve, i udarit  će ga Jahve, bilo da će mu doći njegov dan da umre, bilo da će  otići u boj i poginuti. 
\par 11 Ne dao mi Jahve da dignem ruku na  pomazanika Jahvina! Nego uzmi sada koplje što mu je kod uzglavlja  i vrč za vodu, pa hajdemo!" 
\par 12 I uze David koplje i vrč za vodu  što su bili kod Šaulova uzglavlja i oni odoše: nitko nije ništa  vidio ni opazio, nitko se nije probudio, nego su svi spavali  jer bijaše na njih pao dubok san od Jahve. 
\par 13 David prijeđe na drugu stranu i stade na vrh gore u nekoj  daljini, tako da je među njima bio velik prostor. 
\par 14 Tada viknu  vojsci i Abneru, Nerovu sinu, ovako: "Zar se nećeš odazvati,  Abnere?" A Abner se odazva i upita: "Tko si ti što uznemiruješ  kralja?" 
\par 15 A David odgovori Abneru: "Nisi li ti junak? I tko  ti je ravan u Izraelu? Pa zašto onda nisi čuvao kralja, svoga  gospodara? Jedan je od ratnika sišao do vas da ubije kralja,  tvoga gospodara. 
\par 16 Nije lijepo to što si učinio. Tako mi živog  Jahve, zaslužili ste smrt što niste čuvali svoga gospodara, pomazanika  Jahvina. Pogledaj sada gdje je kraljevo koplje i gdje je vrč  za vodu što mu bijaše do uzglavlja!" 
\par 17 Tada Šaul poznade Davidov glas i upita: "Je li to tvoj  glas, sine Davide?" A David odgovori: "Jest, kralju gospodaru!" 
\par 18 I nastavi: "Zašto moj gospodar progoni svoga slugu? Što sam  učinio? Kakva je krivica u mojoj ruci? 
\par 19 Zato neka se sada  moj gospodar i kralj udostoji poslušati riječi svoga sluge: ako  te Jahve diže protiv mene, neka se prinosnicom ublaži; ako li  to čine sinovi ljudski, neka su prokleti pred Jahvom jer su me  izagnali, tako da ne mogu imati udjela u baštini Jahvinoj, kao  da su mi govorili: 'Idi, služi tuđim bogovima!' 
\par 20 Zato neka  ne padne moja krv na zemlju daleko od Jahvina lica. Jer kralj  je Izraelov izišao u lov na moj život, kao kad tko goni jarebicu  po planini." 
\par 21 Tada Šaul reče: "Zgriješio sam! Vrati mi se, sine Davide, neću ti više činiti zla, kad je danas moj život u očima tvojim  bio tako drag. Jest, ludo sam radio i teško sam pogriješio!" 
\par 22 A David odgovori: "Evo kraljeva koplja, neka dođe jedan od  momaka i neka ga uzme! 
\par 23 A Jahve će vratiti svakome po njegovoj  pravdi i po njegovoj vjernosti: danas te Jahve bijaše predao  u moje ruke, ali nisam htio dići ruke svoje na pomazanika Jahvina. 
\par 24 I gle, kako je danas tvoj život bio drag u mojim očima, tako  neka moj život bude drag u Jahvinim očima! I neka me Jahve izbavi  iz svake nevolje!" 
\par 25 A Šaul doviknu Davidu: "Budi mi blagoslovljen, sine Davide!  Zacijelo ćeš izvršiti svoje djelo i uspjet ćeš!" Potom David  ode svojim putem, a Šaul se vrati svojoj kući. 


\chapter{27}

\par 1 David reče u sebi: "Ipak ću jednoga dana poginuti od Šaulove  ruke. Zato nema ništa bolje za me nego da se spasim u zemlju  Filistejaca. Tada će Šaul odustati da me dalje traži po svim  krajevima Izraelovim i izbavit ću se iz njegove ruke." 
\par 2 David  se dakle podiže i prijeđe, sa šest stotina ljudi koje je imao, k Akišu, sinu Maokovu, kralju Gata. 
\par 3 David se nastani kod  Akiša u Gatu, on i njegovi ljudi, svaki sa svojom obitelji, a  David sa svoje dvije žene, Ahinoamom Jizreelkom i Abigajilom, Nabalovom ženom iz Karmela. 
\par 4 Kad je Šaul doznao da je David  pobjegao u Gat, nije ga više progonio. 
\par 5 David reče Akišu: "Ako sam našao milost u tvojim očima, neka mi dadu mjesto u jednom gradu u zemlji da se nastanim u  njemu. Zašto da tvoj sluga stanuje kod tebe u kraljevskom gradu?" 
\par 6 Akiš mu još istoga dana dade Siklag. Stoga Siklag pripada  do današnjega dana kraljevima Jude. 
\par 7 I osta David u filistejskoj  zemlji godinu dana i četiri mjeseca. 
\par 8 David je sa svojim ljudima izlazio da pljačka Gešurce, Girzijce i Amalečane, jer su to bili stanovnici zemlje od Telama  preko Šura sve do egipatske zemlje. 
\par 9 David je pustošio zemlju  ne ostavljajući na životu ni čovjeka ni žene, otimao je ovce  i goveda, magarce, deve i haljine i vraćao se da sve to donese  Akišu. 
\par 10 Akiš bi ga pitao: "Gdje ste danas pljačkali?" A David  bi odgovorio da su pljačkali u Negebu Judinu ili u Negebu Jerahmeelskom  ili u Negebu Kenijskom. 
\par 11 David nije ostavljao na životu ni  čovjeka ni žene da ih dovede u Gat jer mišljaše: "Mogli bi nas  optužiti i reći: 'Tako je David radio.'" Takav je imao običaj  za sve vrijeme dok je boravio u filistejskoj zemlji. 
\par 12 Akiš je vjerovao Davidu i govorio u sebi: "Baš se omrazio  svome narodu, Izraelu! Zato će mi biti sluga dovijeka!" 


\chapter{28}

\par 1 U ono vrijeme Filistejci skupiše svoje čete za rat protiv  Izraela. I Akiš reče Davidu: "Znaj da ćeš ići sa mnom na vojsku, ti i tvoji ljudi!" 
\par 2 A David odgovori Akišu: "Dobro! Sad ćeš  vidjeti što će učiniti tvoj sluga!" A Akiš odvrati Davidu: "Dobro!  Zato ću te postaviti da budeš mojim čuvarom zauvijek." 
\par 3 Samuel bijaše umro, a sav ga Izrael bijaše oplakao naričući  za njim. Ukopali su ga u njegovu gradu Rami. A Šaul bijaše istjerao  iz zemlje sve zazivače duhova i vračeve. 
\par 4 Dok su se Filistejci skupljali te došli i utaborili se  kod Šunema, Šaul skupi sve Izraelce te se utabori na Gilboi. 
\par 5 Kad Šaul ugleda filistejski tabor, uplaši se i srce mu snažno  zadrhta. 
\par 6 Šaul upita za savjet Jahvu, ali mu Jahve ne dade  odgovora - ni u snima, ni po Urimu, ni preko proroka. 
\par 7 Zato  Šaul reče svojim slugama: "Potražite mi ženu koja zaziva duhove  da odem k njoj i upitam je." A sluge mu odgovoriše: "Evo, u En  Doru ima žena koja zaziva duhove." 
\par 8 Tada se Šaul preruši, obuče druge haljine i otputi se  sa dva čovjeka. I dođe noću k onoj ženi i reče joj: "Daj mi vračaj  pomoću duha i dozovi mi onoga koga ti reknem." 
\par 9 A žena mu odgovori:  "Ta ti znaš što je učinio Šaul i kako je istrijebio iz zemlje  zazivače duhova i vračeve. Zašto postavljaš zamke mome životu  da me pogubiš?" 
\par 10 A Šaul joj se zakle Jahvom govoreći: "Tako  mi živog Jahve, nećeš biti ništa kriva za ovo!" 
\par 11 Tada žena  zapita: "Koga da ti dozovem?" A on odgovori: "Dozovi mi Samuela!" 
\par 12 Kad žena ugleda Samuela, povika iza glasa, a onda reče  Šaulu: "Zašto si me prevario? Ta ti si Šaul!" 
\par 13 A kralj joj  odvrati: "Ne boj se! Nego što vidiš?" A žena odgovori Šaulu:  "Vidim nešto božansko što se diže iz zemlje." 
\par 14 Šaul je upita:  "Kakva je obličja?" A ona odgovori: "Izlazi starac, ogrnut plaštem."  Tada Šaul spozna da je to Samuel, pa pade licem do zemlje i pokloni  se. 
\par 15 Samuel upita Šaula: "Zašto si pomutio moj mir dozivajući  me gore?" A Šaul odgovori: "U velikoj sam nevolji jer su Filistejci  zavojštili na me, a Bog se okrenuo od mene i ne odgovara mi više  ni preko proroka ni u snima. Zato sam dozvao tebe da me poučiš  što da činim." 
\par 16 A Samuel odvrati: "Zašto mene pitaš kad se  Jahve odvratio od tebe i postao ti neprijateljem? 
\par 17 Jahve ti  je učinio kako ti je kazao preko mene: istrgao je kraljevstvo  iz tvoje ruke i dao ga tvome suparniku, Davidu, 
\par 18 jer nisi  poslušao riječi Jahvinih i jer nisi izvršio njegova žestokog  gnjeva na Amaleku: stoga ti je Jahve danas ovako učinio. 
\par 19 Jahve  će predati, zajedno s tobom, i Izraela u filistejske ruke. Sutra  ćeš sa svojim sinovima biti sa mnom, a i tabor izraelski Jahve  će predati u filistejske ruke." 
\par 20 Šaul se užasnu i pade na zemlju kako je dug. Spopade  ga silan strah od Samuelovih riječi. I ponestade mu snage, jer  nije ništa jeo cijeli dan i cijelu noć. 
\par 21 Kad ona žena dođe  k Šaulu i opazi kako je sav zaplašen, reče mu: "Gle, tvoja je  službenica poslušala tvoju riječ, stavila sam svoj život na kocku  i poslušala tvoje zapovijedi koje si mi naložio. 
\par 22 Zato sada  poslušaj i ti riječi službenice svoje: dopusti da ti pružim zalogaj  kruha; jedi da ti se vrati snaga te uzmogneš poći svojim putem." 
\par 23 Ali on ne htjede nego reče: "Neću jesti!" Ali kad ga  zaokupiše njegove sluge, zajedno sa ženom, posluša ih, ustade  sa zemlje i sjede na postelju. 
\par 24 Žena je imala kod kuće tele  u tovu. Brzo ga zakla, zatim uze brašna, umijesi ga i napeče  beskvasnoga kruha. 
\par 25 Potom stavi sve pred Šaula i njegove ljude.  Pošto su jeli, ustadoše i još iste noći krenuše natrag. 


\chapter{29}

\par 1 Filistejci skupiše sve svoje čete u Afeku, a Izraelci se utaboriše  kod izvora u Jizreelu. 
\par 2 Filistejski su knezovi prolazili sa  svojim stotinama i tisućama, a David i njegovi ljudi išli su  sasvim na kraju s Akišem. 
\par 3 Filistejski knezovi zapitaše: "Što  hoće ti Hebreji ovdje?" A Akiš odgovori filistejskim knezovima:  "Pa ovo je David, sluga izraelskoga kralja Šaula! Već je godinu-dvije  kod mene, ali nisam našao na njemu ništa sumnjivo od onoga dana  kad je prebjegao k meni pa do današnjega dana." 
\par 4 Ali filistejski  knezovi planuše na njega i rekoše mu: "Pošalji toga čovjeka natrag, neka se vrati na mjesto koje si mu označio. Neka ne ide s nama  u boj, da se ne okrene protiv nas u boju! Čime bi se on opet  umilio svome gospodaru ako ne glavama ovih naših ljudi? 
\par 5 To  je onaj isti David o kome se pjevalo igrajući: 'Pobi Šaul svoje tisuće, David na desetke tisuća!'" 
\par 6 Tada Akiš dozva Davida i reče mu: "Živoga mi Jahve, ti  si pošten i meni bi drago bilo da me pratiš u pokretima moje  vojske, jer nisam našao nikakva zla na tebi od onoga dana kad  si došao k meni pa do današnjega dana. Ali nisi drag u očima  knezova. 
\par 7 Zato se sada vrati i otiđi s mirom kući da ne ozlovoljiš  filistejske knezove!" 
\par 8 David odvrati Akišu: "Ta što sam učinio i što si zamjerio  svome sluzi od onoga dana kad sam stupio u tvoju službu pa do  današnjega dana da ne mogu ići da se bijem s neprijateljima svoga  gospodara kralja?" 
\par 9 A Akiš odgovori Davidu: "Ti znaš da si  mi drag kao Božji anđeo, ali su filistejski knezovi rekli: 'Neka  ne ide s nama u boj!' 
\par 10 Zato ustanite rano ujutro, ti i sluge  tvoga gospodara koji su došli s tobom, i otiđite na mjesto koje  sam vam označio. I nemoj gajiti u svom srcu nikakve mržnje jer  si mi mio. Ustat ćete, dakle, u rano jutro, čim svane, i otići  ćete!" 
\par 11 Tako David sa svojim ljudima ustade rano i krenu odmah  ujutro i vrati se u filistejsku zemlju, a Filistejci odoše u  Jizreel. 


\chapter{30}

\par 1 Kad je David sa svojim ljudima treći dan stigao u Siklag,  a to Amalečani bijahu navalili na Negeb i na Siklag; opljačkali  su Siklag i ognjem ga spalili. 
\par 2 Zarobili su žene i sve koji  su bili ondje, malo i veliko. Nisu ubili nikoga, nego su samo  odveli roblje i otišli svojim putem. 
\par 3 Kad je, dakle, David  sa svojim ljudima došao u grad, vidješe da je grad spaljen, a  njihove žene, njihovi sinovi i njihove kćeri odvedeni u ropstvo. 
\par 4 Tada David i ljudi koji bijahu s njim podigoše glas i plakahu  dok im nije ponestalo snage za plač. 
\par 5 I obje Davidove žene  bijahu odvedene u ropstvo - Ahinoama Jizreelka i Abigajila, Nabalova  žena iz Karmela. 
\par 6 David se našao u velikoj nevolji jer su ljudi počeli govoriti  da će ga kamenovati, budući da su svi bili ogorčeni, svaki zbog  svojih sinova i zbog svojih kćeri. Ali se David ohrabri u Jahvi, svome Bogu. 
\par 7 David reče svećeniku Ebjataru, Ahimelekovu sinu:  "Donesi mi ovamo oplećak!" I Ebjatar donese Davidu oplećak. 
\par 8 Tada  David upita Jahvu za savjet govoreći: "Hoću li u potjeru za onim  razbojnicima i hoću li ih stići?" A on mu odgovori: "Idi u potjeru  jer ćeš ih zacijelo stići i zarobljenike ćeš izbaviti." 
\par 9 I pođe David sa šest stotina ljudi koji bijahu s njim  i dođoše do potoka Besora. 
\par 10 Odavde David sa četiri stotine  ljudi nastavi potjeru, a ostadoše dvije stotine ljudi što bijahu  tako umorni da nisu mogli prijeći preko potoka Besora. 
\par 11 U polju naiđoše na nekog Egipćanina. Dovedoše ga k Davidu, dadoše mu kruha da jede i vode da pije. 
\par 12 Dadoše mu grudu  smokava i dva grozda suhoga grožđa. Kad je to pojeo, vratio mu  se život, jer tri dana i tri noći ne bijaše ništa jeo i ništa  pio. 
\par 13 Tada ga David upita: "Čiji si ti i odakle si?" A on  odgovori: "Ja sam Egipćanin, sluga jednog Amalečanina. Moj me  gospodar ostavio jer sam se razbolio prije tri dana. 
\par 14 Bili  smo provalili u Negeb Keretski i Negeb Judejski, i u Negeb Kalebov, a Siklag smo zapalili ognjem." 
\par 15 David ga upita: "Hoćeš li me odvesti k toj razbojničkoj  družbi?" A on odgovori: "Zakuni mi se Bogom da me nećeš pogubiti  i da me nećeš predati u ruke mome gospodaru, pa ću te odvesti  k njima!" 
\par 16 On ga, dakle, odvede, i gle, oni se bijahu razasuli po  svem onom kraju, jedući, pijući i slaveći slavlje zbog svega  velikog plijena što su ga oteli iz zemlje filistejske i iz zemlje  Judine. 
\par 17 I David ih poče biti i tukao ih je od zore do mraka, izvršujući na njima "herem", kleto uništenje. Nitko od njih  nije izmakao, osim četiri stotine momaka, koji zajahaše na deve  i pobjegoše. 
\par 18 Tako je David izbavio sve što su bili oteli  Amalečani; i obje svoje žene izbavi David. 
\par 19 I ništa im nije  nestalo, od najmanjih stvari do najvećih, od plijena sve do sinova  i kćeri, sve što im bijaše oteto: sve je vratio David. 
\par 20 Tada  uzeše sve ovce i goveda, dotjeraše ih pred njega vičući: "Ovo  je plijen Davidov!" 
\par 21 Kad je David došao k onim dvjema stotinama ljudi koji  bijahu sustali te ne mogahu ići za Davidom i koje on bijaše ostavio  kod potoka Besora, iziđoše oni u susret Davidu i četi njegovoj:  približivši se Davidu i četi, pozdraviše ih. 
\par 22 Tada progovoriše svi zlobnici i ništarije između ljudi  koji su išli s Davidom i rekoše: "Budući da nisu išli s nama, ne dajmo im ništa od plijena koji smo izbavili, nego samo svakome  njegovu ženu i njegovu djecu, neka ih povedu sa sobom i neka  idu!" 
\par 23 Ali David reče: "Ne činite tako, braćo moja, poslije  onoga što nam je dao Jahve: on nas je čuvao i predao nam u ruke  razbojničku družbu koja bijaše izišla protiv nas. 
\par 24 Ta tko  će vas poslušati u tome? Jer kakav je dio onome koji ide u boj, takav je dio onome koji ostaje kod tovara. Jednak dio neka imaju  svi." 
\par 25 Tako ostade od onoga dana unapredak. David to učini uredbom  i zakonom u Izraelu sve do današnjeg dana. 
\par 26 Kad je David došao u Siklag, posla dio plijena starješinama  Jude, po pojedinim njihovim gradovima, s porukom: "Evo za vas  dar od plijena Jahvinih neprijatelja!" 
\par 27 Onima u Betulu, onima  u Rami u Negebu i onima u Jatiru; 
\par 28 onima u Aroeru, onima u  Sifmotu i onima u Eštemoi; 
\par 29 onima u Karmelu, onima u jerahmeelskim  gradovima i onima u kenijskim gradovima; 
\par 30 onima u Hormi, onima  u Bor Ašanu i onima u Eteru; 
\par 31 onima u Hebronu i u svim onima  mjestima u koja je dolazio David sa svojim ljudima. 



\chapter{31}

\par 1 Filistejci su zavojštili na Izraelce, a Izraelci su pobjegli  pred njima i padali pobijeni po gori Gilboi. 
\par 2 Filistejci stisnuše  Šaula i njegove sinove i pogubiše Šaulove sinove Jonatana, Abinadaba  i Malki-Šuu. 
\par 3 Boj je postao žešći oko Šaula. Iznenadiše ga  strijelci s lukovima i on pade teško ranjen od strijelaca. 
\par 4 Šaul tada reče svome štitonoši: "Izvuci svoj mač i probodi  me da ne dođu ti neobrezanici i ne narugaju mi se." Ali se njegov  štitonoša prestravi i ne htjede toga učiniti. Zato Šaul uze mač  i baci se na nj. 
\par 5 Kad je štitonoša vidio da je Šaul umro, baci  se i on na svoj mač i umrije s njim. 
\par 6 Tako onoga dana pogiboše  zajedno Šaul, njegova tri sina, njegov štitonoša i svi njegovi  ljudi. 
\par 7 Kad Izraelci koji bijahu na drugoj strani doline i na  drugoj strani Jordana vidješe da su sinovi Izraelovi pobjegli  i da je poginuo Šaul sa sinovima, ostaviše svoje gradove te se  razbježaše. Filistejci dođoše i nastaniše se u njima. 
\par 8 Kad su sutradan došli Filistejci da oplijene pobijeđene, nađoše Šaula i njegova tri sina gdje leže na gori Gilboi. 
\par 9 Oni  mu odsjekoše glavu i skidoše s njega oružje, koje poslaše po  svoj filistejskoj zemlji naokolo, javljajući veselu vijest svojim  idolima i narodu. 
\par 10 Potom oružje metnuše u Aštartin hram, a  Šaulovo mrtvo tijelo pribiše na zid grada Bet Šana. 
\par 11 Ali kad oni u Jabešu Gileadskom čuše što su Filistejci  učinili od Šaula, 
\par 12 ustadoše svi hrabri ljudi i, pošto su hodili  svu noć, uzeše Šaulovo mrtvo tijelo i tjelesa njegovih sinova  sa zida grada Bet Šana pa ih donesoše u Jabeš i ondje spališe. 
\par 13 Potom uzeše njihove kosti i ukopaše ih pod tamarisom u Jabešu  i postiše sedam dana. 





\end{document}