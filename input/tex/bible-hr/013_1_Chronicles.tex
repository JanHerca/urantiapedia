\begin{document}

\title{1 Ljetopisa}


\chapter{1}

\par 1 Adam, Šet, Enoš, 
\par 2 Kenan, Mahalalel, Jared, 
\par 3 Henok, Metušalah, Lamek, 
\par 4 Noa, Šem, Ham i Jafet. 
\par 5 Sinovi Jafetovi: Gomer, Magog, Madaj, Javan, Tubal, Mešek  i Tiras. 
\par 6 Gomerovi sinovi: Aškenaz, Rifat i Togarma. 
\par 7 Javanovi  sinovi: Eliša, Taršiš, Kitijci i Dodanci. 
\par 8 Hamovi sinovi: Kuš, Misrajim, Put i Kanaan. 
\par 9 Kuševi sinovi: Seba, Havila, Sabta, Rama i Sabteka; Ramini  sinovi: Šeba i Dedan. 
\par 10 Kušu se rodi Nimrod, koji bijaše prvi  vlastodržac na zemlji. 
\par 11 Od Misrajima potekli su Ludijci, Anamijci, Lehabijci, Neftuhijci, 
\par 12 Patrušani i Kasluhijci, od kojih su potekli  Filistejci i Kaftorci. 
\par 13 Kanaan rodi Sidona, svog prvenca,  i Heta, 
\par 14 i Jebusejce, Amorejce, Girgašane, 
\par 15 Hivijce, Arkijce, Sinijce, 
\par 16 Arvadijce, Semarijce i Hamatijce. 
\par 17 Šemovi sinovi: Elam, Ašur, Arpakšad, Lud i Aram. Aramovi  sinovi: Us, Hul, Geter i Mešek. 
\par 18 Arpakšadu se rodi Šelah,  Šelahu se rodi Eber. 
\par 19 Eberu se rodiše dva sina: jednom bješe  ime Peleg, jer se za njegova doba razdijelila zemlja. Njegovu  je bratu bilo ime Joktan. 
\par 20 Od Joktana se rodiše Almodad, Šelef, Hasarmavet, Jerah, 
\par 21 Hadoram, Uzal, Dikla, 
\par 22 Obal, Abimael, Šeba, 
\par 23 Ofir, Havila i Jobab. Svi su to sinovi Joktanovi. 
\par 24 Šem, Arpakšad, Šalah, 
\par 25 Eber, Peleg, Rau, 
\par 26 Serug, Nahor, Tarah, 
\par 27 Abram, to jest Abraham. 
\par 28 Abrahamovi sinovi:  Izak i Jišmael. 
\par 29 Ovo je njihovo rodoslovlje: Jišmaelov prvenac Nebajot, zatim Kedar, Adbeel, Mibsam, 
\par 30 Mišma, Duma, Masa, Hadad, Tema, 
\par 31 Jetur, Nafiš i Kedma. To su Jišmaelovi  sinovi. 
\par 32 Sinovi Keture, Abrahamove inoče: ona rodi Zimrana, Jokšana, Medana, Midjana, Jišbaka i Šuaha. Sinovi Jokšanovi jesu: Šeba  i Dedan. 
\par 33 Midjanovi su sinovi bili: Efa, Efer, Henok, Abida  i Eldaa. Svi su oni bili Keturini sinovi. 
\par 34 Abrahamu se rodi Izak; Izakovi su sinovi bili: Ezav i  Izrael. 
\par 35 Ezavovi su sinovi bili: Elifaz, Reuel, Jeuš, Jalam  i Korah. 
\par 36 Elifazovi su sinovi bili: Teman, Omar, Sefi, Gatan, Kenaz, Timna i Amalek. 
\par 37 Reuelovi su sinovi bili: Nahat, Zerah, Šama i Miza. 
\par 38 Seirovi su sinovi bili: Lotan, Šobal, Sibeon, Ana, Dišon, Eser i Dišan. 
\par 39 Lotanovi su sinovi bili: Hori i Homam; Lotanova  je sestra bila Timna. 
\par 40 Šobalovi su sinovi bili: Alvan, Manahat, Ebal, Šefi i Onam. Sibeonovi su sinovi bili: Aja i Ana. 
\par 41 Anin  je sin bio Dišon, a Dišonovi su sinovi bili: Hamram, Ešban, Jitran  i Keran. 
\par 42 Eserovi su sinovi bili: Bilhan, Zaavan i Jaakan.  Dišonovi su sinovi bili Us i Aran. 
\par 43 Evo kraljeva koji su kraljevali u zemlji edomskoj prije  nego je zavladao kralj sinova Izraelovih: Bela, sin Beorov; gradu  mu je bilo ime Dinhaba. 
\par 44 Kad je umro Bela, na njegovo se mjesto  zakraljio Jobab, sin Zareha iz Bosre. 
\par 45 Kad je umro Jobab,  zakraljio se na njegovo mjesto Hušam iz temanske zemlje. 
\par 46 Kad  je umro Hušam, zakraljio se na njegovo mjesto Bedadov sin Hadad, koji je potukao Midjance na Moapskom polju; gradu mu je bilo  ime Avit. 
\par 47 Kad je umro Hadad, zakraljio se na njegovo mjesto  Samla iz Masreke. 
\par 48 Kad je umro Samla, zakraljio se na njegovo  mjesto Šaul iz Rehobota na Rijeci. 
\par 49 Kad umrije Šaul, zavlada  Baal Hanan, Akborov sin. 
\par 50 Kad je umro Baal Hanan, zavladao  je Hadad; gradu mu je bilo ime Pai. Žena mu se zvala Mehetabela.  Bila je kći Matredova iz Me Zahaba. 
\par 51 Kad je umro Hadad, nastali su knezovi u Edomu: knez Timna, knez Alva, knez Jetet, 
\par 52 knez Oholibama, knez Ela, knez Pinon, 
\par 53 knez Kenaz, knez Teman, knez Mibsar, 
\par 54 knez Magdiel i  knez Iram. To su bili knezovi edomski. 


\chapter{2}

\par 1 Evo Izraelovih sinova: Ruben, Šimun, Levi, Juda, Jisakar i  Zebulun, 
\par 2 Dan, Josip i Benjamin, Naftali, Gad i Ašer. 
\par 3 Judini sinovi: Er, Onan i Šela. Ta mu je tri rodila Kanaanka, Šuina kći. Ali Er, Judin prvenac, bijaše nevaljao u Jahvinim  očima i Jahve ga pogubi. 
\par 4 Njegova nevjesta Tamara rodi mu Peresa  i Zeraha. U svemu je bilo pet Judinih sinova. 
\par 5 Peresovi su sinovi bili: Hesron i Hamul. 
\par 6 Zerahovi su sinovi bili: Zimri, Etan, Heman, Kalkol i  Dara; u svemu pet. 
\par 7 Karmijevi su sinovi bili: Akar, koji je nanio zlo Izraelu  prekršivši kletvu. 
\par 8 A sinovi Etanovi: Azarja. 
\par 9 Hesronovi sinovi koji mu se rodiše bijahu: Jerahmeel,  Ram i Kelubaj. 
\par 10 Ram rodi Aminadaba, a Aminadab rodi Nahšona, kneza Judinih  sinova. 
\par 11 Nahšon rodi Salmu, Salma rodi Boaza. 
\par 12 Boaz rodi  Obeda, Obed rodi Jišaja. 
\par 13 Jišaj rodi prvenca Eliaba, drugog  Abinadaba, trećeg Šimu, 
\par 14 četvrtog Netanela, petog Radaja, 
\par 15 šestog Osema, sedmoga Davida. 
\par 16 Sestre im: Sarviju i Abigajilu.  Sarvijini su sinovi bili: Abišaj, Joab, Asahel, trojica. 
\par 17 Abigajila  je rodila Amasu, Amasin je otac bio Jišmaelac Jeter. 
\par 18 Hesronov sin Kaleb imao je sa ženom Azubom i s Jeriotom  sinove, a sinovi su mu bili: Ješer, Šobab i Ardon. 
\par 19 Kad je  umrla Azuba, uze Kaleb za ženu Efratu i ona mu rodi Hura. 
\par 20 Hur  rodi Urija, a Uri rodi Besalelu. 
\par 21 Potom Hesron uze kćer Makira, oca Gileadova; uzeo ju  je za ženu kad mu je bilo šezdeset godina i ona mu rodi Seguba. 
\par 22 Segub rodi Jaira, koji je imao dvadeset i tri grada u gileadskoj  zemlji. 
\par 23 Ali su Gešurci i Aramejci osvojili od njih Jairova  Sela, Kenat sa selima, šezdeset gradova. To su sve osvojili sinovi  Makira, oca Gileadova. 
\par 24 Kad je umro Hesron u Kaleb Efrati, Hesronova žena Abija  rodi mu Ašhura, oca Tekoina. 
\par 25 Sinovi Jerahmeela, Hesronova prvenca, bili su: prvenac  Ram, pa Buna, Oren, Osem i Ahija. 
\par 26 Jerahmeel je imao i drugu  ženu, zvala se Atara, koja je bila Onamova majka. 
\par 27 Sinovi Rama, Jerahmeelova prvenca, bili su Maas, Jamin  i Eker. 
\par 28 Onamovi su sinovi bili: Šamaj i Jada; Šamajevi sinovi:  Nadab i Abišur. 
\par 29 Abišurova se žena zvala Abihajla, koja mu  rodi Ahbana i Molida. 
\par 30 Nadabovi su sinovi bili: Seled i Afajim, ali je Seled umro bez djece. 
\par 31 Sinovi su Afajimovi bili: Jiši;  Jišijevi sinovi: Šešan; Šešanov sin Ahlaj. 
\par 32 Sinovi Jade, brata  Šamajeva, bili su: Jeter i Jonatan; ali je Jeter umro bez djece. 
\par 33 Jonatanovi su sinovi bili: Pelet i Zaza. To su bili Jerahmeelovi sinovi. 
\par 34 Šešan nije imao sinova, nego kćeri; ali je Šešan imao  slugu Egipćanina po imenu Jarhu. 
\par 35 Zato je Šešan dao kćer sluzi  Jarhi za ženu i ona mu rodi Ataja. 
\par 36 Ataj rodi Natana, Natan  rodi Zabada. 
\par 37 Zabad rodi Eflala, Eflal rodi Obeda; 
\par 38 Obed  rodi Jehua, Jehu rodi Azarju, 
\par 39 Azarja rodi Helesa, Heles rodi  Elasu; 
\par 40 Elasa rodi Sismaja, Sismaj rodi Šaluma, 
\par 41 Šalum  rodi Jekamju, Jekamja rodi Elišamu. 
\par 42 Sinovi Kaleba, Jerahmeelova brata, bili su: njegov prvenac  Meša, otac Zifov; sinovi oca Mareše bili su: Hebron. 
\par 43 Hebronovi  sinovi Korah, Tapuah, Rekem i Šema. 
\par 44 Šema rodi Rahama, oca  Jorkoamova, a Rekem rodi Šamaja. 
\par 45 Šamajev je sin bio Maon, a Maon je bio otac Bet-Sarov. 
\par 46 Efa, Kalebova inoča, rodila je Harana, Mosu i Gazeza;  Haran je rodio Gazeza. 
\par 47 Johdajevi su sinovi bili: Regem, Jotam, Gešan, Felet, Efa i Šaaf. 
\par 48 Maaka, Kalebova inoča, rodila je Šebera i Tirhanu. 
\par 49 Rodila  je Šaafa, oca Madmanina, Ševu, oca Makbenina, i oca Gibina; a  Kalebova je kći bila Aksa. 
\par 50 To su bili sinovi Kalebovi. A sinovi Hura, Efratina prvenca: Šobal, otac Kirjat Jearimov, 
\par 51 Salma, otac Betlehemov, i Haref, otac Bet-Gaderov. 
\par 52 A  sinovi Šobala, oca Kirjat Jearimova: Reaja, polovica Manahaćana. 
\par 53 Kirjatjearimske su porodice bile: Jitrani, Pućani, Šumaćani  i Mišrani; od njih su potekli Soraćani i Eštaoljani. 
\par 54 Salmini su sinovi: Betlehem, Netofaćani, Atrot, Bet Joab  i polovina Manahaćana, Saraćani. 
\par 55 Književničke obitelji koje  su živjele u Jabesu bile su: Tiraćani, Simeaćani, Sukaćani. To  su Kinejci koji su potekli od Hamata, oca Rekabova doma. 


\chapter{3}

\par 1 Ovo su Davidovi sinovi koji mu se rodiše u Hebronu: prvenac  Amnon od Jizreelke Ahinoame, drugi Daniel od Karmelke Abigajle, 
\par 2 treći Abšalom, sin Maake, kćeri Talmaja, gešurskoga kralja, četvrti Adonija, sin Hagitin, 
\par 3 peti Šefatja od Abitale, šesti  Jitream od njegove žene Egle. 
\par 4 Šest mu se sinova rodilo u Hebronu, gdje je kraljevao sedam godina i šest mjeseci; a trideset je  i tri godine kraljevao u Jeruzalemu. 
\par 5 Ovi mu se sinovi rodiše  u Jeruzalemu: Šimeja, Šobab, Natan i Salomon; četiri od Bat-Šebe, kćeri Amielove. 
\par 6 Jibhar, Elišama, Elifalet, 
\par 7 Nogah, Nefeg, Jafija, 
\par 8 Elišama, Elijada, Elifelet: devet. 
\par 9 Sve Davidovi sinovi osim inočkih sinova i njihove sestre  Tamare. 
\par 10 Salomonov je sin bio Roboam, njegov sin Abija, njegov  sin Asa, njegov sin Jošafat, 
\par 11 njegov sin Joram, njegov sin  Ahazja, njegov sin Joaš, 
\par 12 njegov sin Amasja, njegov sin Azarja, njegov sin Jotam, 
\par 13 njegov sin Ahaz, njegov sin Ezekija, njegov  sin Manaše, 
\par 14 njegov sin Amon, njegov sin Jošija. 
\par 15 Jošijini  su sinovi bili: prvenac Johanan, drugi Jojakim, treći Sidkija, četvrti Šalum. 
\par 16 Jojakimovi su sinovi bili: Jekonija, njegov  sin, i Sidkija, njegov sin. 
\par 17 Sinovi sužnja Jekonje bili su Šealtiel, njegov sin, 
\par 18 Malkiram, Pedaja, Šenasar, Jekamja, Jošama i Nebadja. 
\par 19 Pedajini su  sinovi bili: Zerubabel i Šimej; Zerubabelovi sinovi: Mešulam  i Hananija, i sestra im Šelomita. 
\par 20 Hašuba, Ohel, Berekja,  Hasadja i Jušab-Hesed, njih petorica. 
\par 21 Hananijini su sinovi  bili: Pelatja i Jišaja, Refajini sinovi, Arnanov sin, Obadjin  sin, Šekanijin sin. 
\par 22 Šekanijini su sinovi bili: Šemaja, a  Šemajini su sinovi bili: Hatuš, Jigal, Barijah, Nearja i Šafat, šestorica. 
\par 23 Nearjini su sinovi bili Elijoenaj, Ezekija i  Azrikam, trojica. 
\par 24 Elijoenajevi su sinovi bili Hodavja, Elijašib, Felaja, Akub, Johanan, Delaja i Anani, sedmorica. 


\chapter{4}

\par 1 Judini su sinovi bili: Peres, Hesron, Karmi, Hur i Šobal. 
\par 2 Šobalov sin Reaja rodi Jahata, a Jahat rodi Ahumaja i  Lahada. To su soratski rodovi. 
\par 3 Ovo su sinovi od oca Etama: Jizreel, Jišma i Jidbaš, a  njihovoj je sestri bilo ime Haslelponija. 
\par 4 Fenuel je bio otac Gedoru, a Ezer je bio Hušin otac. To su bili sinovi Hura, prvenca Efrate, oca Betlehema. 
\par 5 A  otac Tekoe Ašhur imao je dvije žene, Helu i Naaru. 
\par 6 Naara mu je rodila Ahuzama, Hefera, Temnance i Ahaštarce.  To su Naarini sinovi. 
\par 7 Helini su sinovi bili: Seret, Sohar i Etnan. 
\par 8 Kos rodi Anuba i Hasobebu i porodice Harumova sina Aharhela. 
\par 9 Jabes je bio izvrsniji među braćom i mati mu je nadjela ime  Jabes govoreći: "Rodila sam ga s bolom." 
\par 10 Jabes je prizvao  Izraelova Boga govoreći: "Ako me odista blagoslivljaš, raširi  moje područje, neka bude tvoja ruka uza me i sačuvaj me oda zla, tako da se ne mučim!" Ispuni mu Bog za što ga je molio. 
\par 11 Šuhin brat Kelub rodi Mehira; on je bio Eštonov otac. 
\par 12 Od Eštona poteče Bet Rafa, Paseah i Tehina, otac Ir Nahaša.  To su Rekini ljudi. 
\par 13 A Kenazovi su sinovi bili: Otniel i Seraja. Otnielovi  sinovi: Hatat i Meonotaj. 
\par 14 Meonotaj rodi Ofru; Šeraja rodi  Joaba, oca onih što žive u Dolini rukotvoraca, jer bijahu rukotvorci. 
\par 15 Sinovi Jefuneova sina Kaleba bili su: Ir, Ela i Naam;  Elin je sin bio Kenaz. 
\par 16 Jehalelelovi su sinovi bili Zif, Zifa, Tirja i Asrael. 
\par 17 Ezrini sinovi: Jeter, Mered, Efer i Jalon; Jeter rodi  Mirjamu, Šamaja i Jišboha, Eštemoina oca. 
\par 18 Njegova žena Judejka  rodila je Jereda, Gedorova oca, Hebera, Sokova oca, i Jekutiela, Zanoahova oca. To su bili sinovi Bitje, faraonove kćeri koju  je za ženu uzeo Mered. 
\par 19 Sinovi Hodijine žene, sestre Nahama, Keilina oca, bili  su: Šimun, otac Jomama Garmijca, i Eštemoa Maakaćanin. 
\par 20 Šimunovi su sinovi bili: Amnon, Rina, Ben-Hanan i Tilon.  Išijevi sinovi: Zohet i Ben-Zohet. 
\par 21 Sinovi Judina sina Šele bili su: Er, Lekin otac, Lada, Marešin otac, i obitelji platnarske kuće u Bet Ašbeji; 
\par 22 Jokim  i ljudi iz Kozebe Joaš i Saraf, koji su vladali nad Moabom i  vratili se u Betlehem. Ali su to stari događaji. 
\par 23 To su bili  lončari koji su živjeli u Netajimu i u Gederi kod kralja i bili  su ondje zaposleni u njega. 
\par 24 Šimunovi su sinovi bili Nemuel, Jamin, Jarib, Zerah i  Šaul. 
\par 25 Njegov je sin bio Šalum, a njegov je sin Mibsam, njegov  sin Mišma. 
\par 26 Mišmini su sinovi bili: Hamuel, sin mu, i njegov  sin Zakur i njegov sin Šimej. 
\par 27 Šimej je imao šesnaest sinova  i šest kćeri; njegova braća nisu imala mnogo sinova, i sve njihove  porodice nije bilo tako mnogo kao Judinih sinova. 
\par 28 Živjeli  su u Beer Šebi, Moladi i Hasar Šualu, 
\par 29 u Bilhi, u Esemu, u  Toladu, 
\par 30 u Betuelu, u Hormi, u Siklagu, 
\par 31 u Bet Markabotu, u Hasar Susimu, u Bet Biriju i u Šaarajimu. To su bili njihovi  gradovi do Davidova kraljevanja. 
\par 32 A njihova su naselja bila:  Etam i Ajin, Rimon, Token i Ašan, pet gradova. 
\par 33 I sva njihova  naselja što su bila oko tih gradova do Baala. To su bili njihovi  stanovi i njihovi plemenski popisi. 
\par 34 Mešobad, Jamlek i Amasjin sin Joša, 
\par 35 Joel i Jehu,  sin Jošibje, sina Serajina, sina Asielova, 
\par 36 Elijoenaj, Jaakoba, Ješohaja, Asaja, Adiel, Jesimiel i Benaja, 
\par 37 Ziza, sin Šifija, sina Alonova, sina Jedajeva, sina Šimrijeva, sina Šemajina. 
\par 38 Ti su imenovani bili starješine svojim rodovima i njihove  su se porodice veoma umnožile. 
\par 39 Zato su otišli do mjesta kako  se ide u Gedor do istočne strane doline da traže pašu stoci. 
\par 40 Našli su obilatu i dobru pašu i prostranu, sigurnu i mirnu  zemlju. Budući da su ondje prije živjeli Hamovi potomci, 
\par 41 Šimunovci, poimence popisani, navališe za vremena judejskoga kralja Ezekije  te razbiše njihove šatore i njihove zaklone koji se nađoše ondje.  Baciše na njih kletvu, koja traje do današnjega dana, i nastaniše  se na njihovo mjesto jer su ondje bili pašnjaci za njihovu stoku. 
\par 42 Onda su neki među onima što su pripadali Šimunovim sinovima, njih pet stotina, izbili na planinu Seir, na čelu s Felatjom, Nearjom, Refajom i Uzielom, Išijevim sinovima. 
\par 43 Oni pobiše  ostatak koji se spasio između Amalečana i naseliše se ondje do  današnjega dana. 


\chapter{5}

\par 1 Sinovi Izraelova prvenca Rubena. On je doista bio prvenac;  ali kad je oskvrnuo očevu postelju, njegovo je pravo prvorodstva  bilo dano sinovima Izraelova sina Josipa, ali im nije bilo upisano  u rodovnik, 
\par 2 jer je Juda nadvladao među braćom, a od njega  se rodio knez. Ipak je pravo prvorodstva pripalo Josipu. 
\par 3 Sinovi  Izraelova prvenca Rubena bili su Henok, Falu, Hesron i Karmi. 
\par 4 Joelovi sinovi: njegov sin Šemaja, njegov sin Gog, njegov  sin Šimej, 
\par 5 njegov sin Mika, njegov sin Reaja, njegov sin Baal, 
\par 6 njegov sin Beera, koga je odveo u sužanjstvo asirski kralj  Tiglat Pileser; on je bio poglavar Rubenova plemena. 
\par 7 Njegovoj braći, po obiteljima, kad su se zapisali u plemenski  rodovnik po naraštajima, bio je poglavar Jeiel, Zaharija, 
\par 8 Bela, sin Azazov, sin Šemin, sin Joelov. On je živio u Aroeru i do Neba i Baal Meona. 
\par 9 Prema istoku  njegova se zemlja prostirala do ulaza u pustinju, od rijeke Eufrata, jer mu se stoka umnožila u gileadskoj zemlji. 
\par 10 Za Šaulovih vremena vojevali su s Hagrijcima koji su  izginuli od njihove ruke; tako su se naselili u njihove šatore  po svemu istočnom području od Gileada. 
\par 11 Gadovi su sinovi živjeli blizu njih u bašanskoj zemlji  do Salke. 
\par 12 Poglavar je bio Joel, a drugi Šafan, pa Janaj i  Šafat u Bašanu. 
\par 13 Njihova su braća po svojim rodovima bila: Mihael, Mešulam, Šeba, Joraj, Jakan, Zija, Eber, sedmorica. 
\par 14 To su bili sinovi Abihajila, sina Hurija, sina Jaroaha, sina Gileada, sina Mihaela, sina Ješišaja, sina Jahdona, sina  Buza. 
\par 15 Ahi, sin Abdiela, Gunijeva sina, bio je poglavar njihova  roda. 
\par 16 Živjeli su u Gileadu i u njihovim zaseocima te po svim  šaronskim pašnjacima do njihovih krajeva. 
\par 17 Svi su bili zapisani  u plemenski rodovnik za vremena judejskoga kralja Jotama i za  vremena izraelskoga kralja Jeroboama. 
\par 18 Rubenovih i Gadovih sinova, i polovine Manašeova plemena, hrabrih junaka koji su nosili štit i mač te zapinjali luk i  bili vješti boju, bijaše četrdeset tisuća sedam stotina i šezdeset  vojnika. 
\par 19 Ratovali su protiv Hagrijaca, Iturejaca, Nafišejaca  i Nodabejaca. 
\par 20 U boju su zavapili k Bogu i on ih je uslišao  jer su se pouzdali u nj: potpomognuti su protiv neprijatelja  te su im predani u ruke Hagrijci sa svim njihovim saveznicima. 
\par 21 Zaplijenili su njihovu stoku - pedeset tisuća deva, dvije  stotine i pedeset tisuća ovaca i koza i dvije tisuće magaraca  - i odveli u ropstvo sto tisuća ljudi. 
\par 22 Pobijenih je mnogo  palo, jer je taj boj bio od Boga; onda se naseliše na njihovo  mjesto do sužanjstva. 
\par 23 Sinovi polovine Manašeova plemena nastanili su se u toj  zemlji od Bašana do Baal Hermona i Šenira i do Hermonske gore.  I bijahu se umnožili. 
\par 24 Ovo su bili poglavari njihovih rodova: Efer, Jiši, Eliel, Azriel, Jeremija, Hodavja i Jahdiel, hrabri junaci i ugledni  muževi: poglavari u svojim rodovima. 
\par 25 Ali kad su se iznevjerili Bogu svojih otaca i odali se  preljubu s bogovima naroda one zemlje koje je Bog iskorijenio  pred njima, 
\par 26 probudio je Izraelov Bog neprijateljstvo asirskoga  kralja Pula i neprijateljstvo asirskoga kralja Tiglat Pilesera.  Oni su odveli u sužanjstvo Rubenovo i Gadovo pleme i polovinu  Manašeova plemena. Doveli su ih u Helah, Habor i Haru i na Gozansku  rijeku do današnjega dana. 


\chapter{6}

\par 1 (5:27) Levijevi su sinovi bili Geršom, Kehat i Merari. 
\par 2 (5:28) Kehatovi  sinovi: Amram, Jishar, Hebron i Uziel. 
\par 3 (5:29) Amramovi sinovi: Aron, Mojsije i Mirjama. Aronovi sinovi: Nabad i Abihu, Eleazar i  Itamar. 
\par 4 (5:30) Eleazar rodi Pinhasa, Pinhas rodi Abišuu; 
\par 5 (5:31) Abišua  rodi Bukija, Buki rodi Uziju; 
\par 6 (5:32) Uzija rodi Zerahju; Zerahja  rodi Merajota. 
\par 7 (5:33) Merajot rodi Amarju; Amarja rodi Ahituba; 
\par 8 (5:34) Ahitub rodi Sadoka; Sadok rodi Ahimaasa; 
\par 9 (5:35) Ahimaas rodi  Azarju, Azarja rodi Johanana; 
\par 10 (5:36) Johanan rodi Azarju, koji je  bio svećenik u Hramu što ga je sagradio Salomon u Jeruzalemu. 
\par 11 (5:37) Azarja rodi Amarju; Amarja rodi Ahituba; 
\par 12 (5:38) Ahitub rodi  Sadoka, Sadok rodi Šaluma; 
\par 13 (5:39) Šalum rodi Hilkiju, Hilkija rodi  Azarju; 
\par 14 (5:40) Azarja rodi Seraju; Seraja rodi Josadaka. 
\par 15 (5:41) Josadak  je otišao kad je Jahve odveo u sužanjstvo Judu i Jeruzalem Nabukodonozorovom  rukom. 
\par 16 (6:1) Levijevi su sinovi bili Geršom, Kehat i Merari. 
\par 17 (6:2) Evo imena  Geršomovih sinova: Libni i Šimej. 
\par 18 (6:3) Kehatovi su sinovi bili:  Amram, Jishar, Hebron i Uziel. 
\par 19 (6:4) Merarijevi sinovi: Mahli i  Muši. Ovo su rodovi Levijevaca po svojim ocima. 
\par 20 (6:5) Od Geršoma: sin mu Libni, njegov sin Jahat, njegov sin  Zima, 
\par 21 (6:6) njegov sin Joah, njegov sin Ido, njegov sin Zerah, njegov  sin Jeatraj. 
\par 22 (6:7) Kehatovi sinovi: sin mu Aminadab, njegov sin Korah, njegov  sin Asir, 
\par 23 (6:8) njegov sin Elkana, njegov sin Ebjasaf, njegov sin  Asir; 
\par 24 (6:9) njegov sin Tahat, njegov sin Uriel, njegov sin Uzija, njegov sin Šaul. 
\par 25 (6:10) Elkanini sinovi: Amasaj i Ahimot; 
\par 26 (6:11) njegov  sin Elkana, njegov sin Sufaj, njegov sin Nahat; 
\par 27 (6:12) njegov sin  Eliab, njegov sin Jeroham, njegov sin Elkana. Elkanini sinovi: 
\par 28 (6:13) Samuel, njegov prvenac, drugi Abija. 
\par 29 (6:14) Merarijevi sinovi: Mahli, njegov sin Libni, njegov sin  Šimej, njegov sin Uza, 
\par 30 (6:15) njegov sin Šima, njegov sin Hagija, njegov sin Asaja. 
\par 31 (6:16) Ovo su oni koje je postavio David da se brinu za pjevanje  u Domu Jahvinu kad je Kovčeg ondje našao svoje počivalište; 
\par 32 (6:17) oni  koji su služili pred Prebivalištem, Šatorom sastanka, pjevajući, dok nije Salomon sagradio Dom Jahvin u Jeruzalemu i koji su  obavljali službu po propisanom redoslijedu. 
\par 33 (6:18) Evo onih što  su obavljali službu i njihovih sinova: od Kehatovih sinova: pjevač  Heman, sin Joela, sina Samuela, 
\par 34 (6:19) sina Elkane, sina Jerohama, sina Eliela, sina Toaha, 
\par 35 (6:20) sina Sifa, sina Elkane, sina Mahata, sina Amasaja, 
\par 36 (6:21) sina Elkane, sina Joela, sina Azarje, sina  Sefanije, 
\par 37 (6:22) sina Tahata, sina Asira, sina Abjasafa, sina Koraha, 
\par 38 (6:23) sina Jishara, sina Kehata, sina Levija, sina Izraelova. 
\par 39 (6:24) Brat mu Asaf stajao je s desne strane; Asaf je bio sin  Berekje, sina Šime, 
\par 40 (6:25) sina Mihaela, sina Baaseja, sina Malkije, 
\par 41 (6:26) sina Etnija, sina Zeraha, sina Adaje, 
\par 42 (6:27) sina Etana, sina  Zime, sina Šimeja, 
\par 43 (6:28) sina Jahata, sina Geršoma, sina Levijeva. 
\par 44 (6:29) Merarijevi sinovi, njihova braća, stajala su mu s lijeve  strane: Etan, sin Kušija, sina Abdija, sina Maluka, 
\par 45 (6:30) sina  Hašabje, sina Amasje, sina Hilkije, 
\par 46 (6:31) sina Amsija, sina Banija, sina Šomera, 
\par 47 (6:32) sina Mahlija, sina Mušija, sina Merarija, sina  Levijeva. 
\par 48 (6:33) Njihova braća leviti bili su postavljeni za svu službu  u svetom Prebivalištu, u Domu Božjem. 
\par 49 (6:34) Aron i njegovi sinovi  prinosili su kad na žrtveniku za paljenice i na kadionom žrtveniku, obavljajući sav posao u Svetinji nad svetinjama i izvršujući  obred pomirenja nad Izraelom, prema svemu što je zapovjedio Božji  sluga Mojsije. 
\par 50 (6:35) Ovo su Aronovi sinovi: sin mu Eleazar, njegov sin Pinhas, njegov sin Abišua, 
\par 51 (6:36) njegov sin Buki, njegov sin Uzi, njegov  sin Zerahja, 
\par 52 (6:37) njegov sin Merajot, njegov sin Amarja, njegov  sin Ahitub, 
\par 53 (6:38) njegov sin Sadok, njegov sin Ahimaas. 
\par 54 (6:39) Ovo su im boravišta po naseljima u njihovu području: Aronovim sinovima od Kehatove obitelji - jer na njih je pao  ždrijeb - 
\par 55 (6:40) dali su Hebron u judejskoj zemlji s pašnjacima  oko njega. 
\par 56 (6:41) Gradsko polje i njegova sela dali su Jefuneovu  sinu Kalebu. 
\par 57 (6:42) Dali su, dakle, Aronovim sinovima gradove-utočišta  Hebron i Libnu s pašnjacima, Jatir i Eštemou s pašnjacima, 
\par 58 (6:43) Hilez  s pašnjacima, Debir s pašnjacima, 
\par 59 (6:44) Ašan s pašnjacima i Bet  Šemeš s pašnjacima. 
\par 60 (6:45) Od Benjaminova plemena: Gebu s pašnjacima, Alemet s pašnjacima i Anatot s pašnjacima; dakle trinaest gradova  po njihovim rodovima. 
\par 61 (6:46) Ostalim Kehatovim sinovima prema plemenskim rodovima  pripalo je ždrijebom deset gradova od polovine Manašeova plemena. 
\par 62 (6:47) Geršomovim sinovima po njihovim rodovima pripalo je od Jisakarova  plemena, od Ašerova plemena, od Naftalijeva plemena i od Manašeova  plemena u Bašanu trinaest gradova. 
\par 63 (6:48) Merarijevim sinovima po  njihovim rodovima pripalo je ždrijebom od Rubenova plemena, od  Gadova plemena i od Zebulunova plemena dvanaest gradova. 
\par 64 (6:49) Tako  su Izraelovi sinovi dali levitima te gradove s pašnjacima. 
\par 65 (6:50) Dali su ždrijebom od plemena Judinih sinova, od plemena  Šimunovih sinova i od plemena Benjaminovih sinova te gradove  koje su spomenuli poimence. 
\par 66 (6:51) Onima koji su bili od rodova Kehatovih sinova te dobili  ždrijebom gradove od Efrajimova plemena 
\par 67 (6:52) dali su kao gradove-utočišta  Šekem s pašnjacima u Efrajimovoj gori i Gezer s pašnjacima, 
\par 68 (6:53) Jokmeam  s pašnjacima, Bet Horon s pašnjacima, 
\par 69 (6:54) Ajalon s pašnjacima  i Gat-Rimon s pašnjacima. 
\par 70 (6:55) Od polovine Manašeova plemena dali  su rodovima ostalih Kehatovih sinova: Aner s pašnjacima i Bileam  s pašnjacima. 
\par 71 (6:56) Geršomovim sinovima dali su od rodova polovine Manašeova  plemena Golan u Bašanu s pašnjacima i Aštarot s pašnjacima. 
\par 72 (6:57) Od  Jisakarova plemena Kedeš s pašnjacima, Dobrat s pašnjacima, 
\par 73 (6:58) Ramot  s pašnjacima i Anem s pašnjacima. 
\par 74 (6:59) Od Ašerova plemena Mašal  s pašnjacima, Abdon s pašnjacima, 
\par 75 (6:60) Hukok s pašnjacima i Rehob  s pašnjacima. 
\par 76 (6:61) Od Naftalijeva plemena Kedeš u Galileji s pašnjacima, Hamon s pašnjacima i Kirjatajim s pašnjacima. 
\par 77 (6:62) Ostalim Merarijevim sinovima dali su od Zebulunova plemena  Rimon s pašnjacima i Tabor s pašnjacima. 
\par 78 (6:63) S onu stranu Jordana, prema Jerihonu, na istočnoj strani Jordana, dali su im od Rubenova  plemena Beser u pustinji s pašnjacima, Jahsu s pašnjacima, 
\par 79 (6:64) Kedemot  s pašnjacima i Mefaat s pašnjacima. 
\par 80 (6:65) Od Gadova plemena Ramot  u Gileadu s pašnjacima, Mahanajim s pašnjacima, 
\par 81 (6:66) Hešbon s  pašnjacima i Jazer s pašnjacima. 


\chapter{7}

\par 1 Jisakarovi su sinovi bili Tola i Fua, Jašub i Šimron, njih  četvorica. 
\par 2 Tolini sinovi: Uzi, Refaja, Jeriel, Jahmaj, Jibsam i Samuel, glavari obitelji od Tole, hrabri junaci svrstani po srodstvu;  bilo ih je na broju za Davidova vremena dvadeset i dvije tisuće  i šest stotina. 
\par 3 Uzijevi sinovi: Jizrahja; Jizrahjini sinovi: Mihael, Obadja, Joel i Jišija, u svemu pet glavara. 
\par 4 S njima je po obiteljima  srodnih bilo u vojnim četama za rat trideset i šest tisuća ljudi, jer su imali mnogo žena i sinova. 
\par 5 Njihove braće po svim Jisakarovim  rodovima, hrabrih junaka, bilo je svega osamdeset i sedam tisuća  i svi su bili popisani u plemenskim rodovnicima. 
\par 6 Benjaminovi sinovi: Bela, Beker i Jediael, njih trojica. 
\par 7 Belini sinovi: Esbon, Uzi, Uziel, Jerimot i Iri, pet obiteljskih  glavara, hrabrih junaka; u plemenskom popisu bilo je zapisanih  dvadeset dvije tisuće i trideset četiri. 
\par 8 Bekerovi sinovi: Zimra, Joaš, Eliezer, Elijoenaj, Omri, Jerimot, Abija, Anatot i Alamet, svi Bekerovi sinovi. 
\par 9 U plemenskom  popisu po koljenima, po obiteljskim glavarima, hrabrih junaka, bilo je zapisano dvadeset tisuća i dvije stotine. 
\par 10 Jediaelovi sinovi: Bilhan, Bilhanovi sinovi: Jeuš, Benjamin, Ahud, Kenaana, Zetan, Taršiš i Ahišahar. 
\par 11 Svih Jediaelovih sinova po obiteljskim glavarima, hrabrih  junaka, bilo je sedamnaest tisuća i dvije stotine, sve za rat  sposobnih. 
\par 12 Šupim i Hupim. Sinovi Irovi: Hušim; njegov sin Aher. 
\par 13 Naftalijevi sinovi: Jahasiel, Guni, Jeser i Šalum. Bilhini  sinovi. 
\par 14 Manašeovi sinovi: Asriel, koga je rodila Manašeova inoča  Aramejka; ona je rodila i Makira, Gileadova oca. 
\par 15 Makir je  oženio Hupima i Šupima; sestra mu se zvala Maaka; ime drugome  bilo je Selofhad, a Selofhad je imao kćeri. 
\par 16 Makirova žena Maaka rodila je sina, komu je nadjela ime  Pereš. Bratu mu je dala ime Šareš, a njegovi su sinovi bili Ulam  i Rakem. 
\par 17 Ulamovi sinovi: Bedan. To su sinovi Gileada, sina Makira, Manašeova sina. 
\par 18 Njegova sestra Hamoleketa rodila je Išhoda, Abiezera  i Mahlu. 
\par 19 Šemidini su sinovi bili: Ahjan, Šekem, Likhi i Aniam. 
\par 20 Efrajimovi sinovi: Šutelah, njegov sin Bered, njegov  sin Tahat, njegov sin Elada, njegov sin Tahat, 
\par 21 njegov sin  Zabad, njegov sin Šutelah, Ezer i Elad. Njih su ubili gatski građani, rođeni u zemlji, jer su sišli  da im otmu stoku. 
\par 22 Zato je njihov otac Efrajim tugovao dugo  vremena, a braća su mu odlazila da ga tješe. 
\par 23 Onda je ušao  k svojoj ženi i ona je zatrudnjela i rodila sina, a on mu nadjenu  ime Berija, jer se nesreća dogodila u njegovoj kući. 
\par 24 Kći  mu je bila Šeera, koja je sagradila Donji i Gornji Bet Horon  i Uzen Šeeru. 
\par 25 Sin mu je bio Refah i Rešef, njegov sin Telah, njegov  sin Tahan, 
\par 26 njegov sin Ladan, njegov sin Amihud, njegov sin  Elišama, 
\par 27 njegov sin Nun, njegov sin Jošua. 
\par 28 Njihov posjed i njihova naselja bili su Betel i njegova  sela, s istoka Naaran, sa zapada Gazer i njegova sela, Šekem  i njegova sela do Gaze s njezinim selima. 
\par 29 U rukama Manašeovih  sinova bio je Bet Šean sa svojim selima, Tanak sa svojim selima, Megido sa svojim selima, Dor sa svojim selima. U njima su živjeli  sinovi Izraelova sina Josipa. 
\par 30 Ašerovi su sinovi bili: Jimna, Jišva, Jišvi i Berija, i njihova sestra Seraha. 
\par 31 Berijini sinovi: Heber i Malkiel; on je bio Birzajitov  otac. 
\par 32 Heber postade otac Jafletu, Šomeru, Hotamu i njihovoj  sestri Šui. 
\par 33 Jafletovi su sinovi bili: Pasak, Bimhal i Ašvat; to su  bili Jafletovi sinovi. 
\par 34 A sinovi njegova brata Šomera: Rohga, Huba i Aram. 
\par 35 Sinovi njegova brata Helema: Sofah, Jimna, Šeleš i Amal. 
\par 36 Sofahovi sinovi: Suah, Harnefer, Šual, Beri, Jimra, 
\par 37 Beser, Hod, Šama, Šilša, Jitran i Bera. 
\par 38 Jeterovi sinovi: Jefune, Fispa i Ara. 
\par 39 Ulini sinovi: Arah, Haniel i Risja. 
\par 40 Svi su oni bili Ašerovi sinovi, obiteljski glavari, probrani  hrabri junaci, glavari među knezovima; kad su bili popisani,  bilo ih je dvadeset i šest tisuća ljudi u bojnim četama. 


\chapter{8}

\par 1 Benjamin rodi prvenca Belu, drugog Ašbela, trećeg Ahraba, 
\par 2 četvrtog  Nohu i petog Rafu. 
\par 3 Belini su sinovi bili: Adar, Gera, Ehudov  otac, 
\par 4 Abišua, Naaman, Ahoah, 
\par 5 Gera, Šefufan i Huram. 
\par 6 Oni su bili Ehudovi sinovi i bili su obiteljski glavari  onima koji su živjeli u Gebi, odakle su ih odveli u sužanjstvo  u Manahat; 
\par 7 Naaman, Ahija i Gera; on ih je vodio u sužanjstvo  i rodio Uzu i Ahihuda. 
\par 8 Šaharajim, pošto je otpustio žene Hušimu i Baru, dobio  je sinove u Moapskom polju: 
\par 9 sa svojom ženom Hodešom imao je  sinove Jobaba, Sibju, Mešu, Malkama, 
\par 10 Jeusa, Sakju i Mirmu;  to su bili njegovi sinovi, obiteljski glavari. 
\par 11 S Hušimom je rodio Abituba i Elpaala. 
\par 12 Elpaalovi su  sinovi bili: Eber, Mišam i Šamed; on je sagradio Ono i Lod s  njihovim selima. 
\par 13 Zatim Berija i Šema. Oni su bili obiteljski glavari onima  koji su živjeli u Ajalonu i istjerali su gatske stanovnike. 
\par 14 Njegov brat: Šešak. Jeremot, 
\par 15 Zabadja, Arad i Eder, 
\par 16 Mihael, Jišpa i Joha  bili su Berijini sinovi. 
\par 17 Zebadja, Mešulam, Hizki, Haber, 
\par 18 Jišmeraj, Jizlia  i Jobab bili su Elpaalovi sinovi. 
\par 19 Jakim, Zikri, Zabdi, 
\par 20 Elijoenaj, Siltaj, Eliel, 
\par 21 Adaja, Beraja i Šimrat bili su Šimijevi sinovi. 
\par 22 Jišpan, Eber, Eliel, 
\par 23 Abdon, Zikri, Hanan, 
\par 24 Hananija, Elam, Antotija, 
\par 25 Jifdeja, Fenuel bili su Šešakovi sinovi. 
\par 26 Šamšeraj, Šeharja, Atalija, 
\par 27 Jaarešja, Elija i Zikri  bili su Jerohamovi sinovi. 
\par 28 To su bili glavari obitelji svrstanih po koljenima. Živjeli  su u Jeruzalemu. 
\par 29 U Gibeonu su živjeli: praotac Gibeon, čija se žena zvala  Maaka. 
\par 30 Njegov je sin prvenac bio Abdon, pa Sur, Kiš, Baal, Nadab, 
\par 31 Gedor, Ahjo, Zaker, 
\par 32 i Miklot, koji je rodio Šimu;  pa su i oni živjeli kod svoje braće u Jeruzalemu, sa svojom braćom. 
\par 33 Ner rodi Kiša, a Kiš rodi Šaula, Šaul rodi Jonatana,  Malki-Šua, Abinadaba, Ešbaala, 
\par 34 Jonatanov je sin bio Merib  Baal; Merib Baal rodi Miku. 
\par 35 Mikini su sinovi bili: Piton, Melek, Tarea i Ahaz. 
\par 36 Ahaz rodi Joadu; Joada rodi Alemeta, Azmaveta i Zimrija; Zimri rodi Mosu. 
\par 37 Mosa rodi Biniju, čiji  je sin bio Rafa, a njegov sin Elasa, njegov sin Asel. 
\par 38 Asel  je imao šest sinova, kojima su imena: Azrikam, njegov prvenac, Bokru, Jišmael, Šearja, Obadja i Hanan; svi su oni bili Aselovi  sinovi. 
\par 39 Sinovi njegova brata Ešeka bili su: Ulam, prvenac mu, drugi Jehuš, treći Elifelet. 
\par 40 Ulamovi su sinovi bili hrabri  junaci koji su zapinjali luk i imali mnogo sinova i unuka, sto  pedeset.  Svi su oni bili od Benjaminovih sinova. 


\chapter{9}

\par 1 Svi su Izraelci bili upisani u plemenskim rodovnicima, a zapisani  su i u Knjizi izraelskih kraljeva. A Judejci su zbog nevjere  bili odvedeni u sužanjstvo u Babilon. 
\par 2 Prvi su stanovnici na  svojem posjedu i u svojim gradovima bili Izraelci, svećenici, leviti i netinci. 
\par 3 U Jeruzalemu su živjeli ljudi od Judinih  sinova, od Benjaminovih sinova, od Efrajimovih i Manašeovih sinova, i to: 
\par 4 Utaj, sin Amihuda, sina Omrija, sina Imrija, sina Banija, od sinova Judina sina Peresa. 
\par 5 Od Šilonaca: Asaja, prvenac, sa svojim sinovima. 
\par 6 Od Zarehovih sinova: Jeuel i njegova  braća, šest stotina i devedeset. 
\par 7 Od Benjaminovih sinova Salu, sin Mešulama, sina Hodavje, Hasenuina sina; 
\par 8 Ibneja, Jerohamov sin, i Ela, sin Uzije,  Mokrijeva sina, i Mešulam, sin Šefatje, sina Reuela, Ibnijina  sina. 
\par 9 Imali su po svojim rodovima devet stotina pedeset i  šestero braće. Svi su oni bili glavari, svaki svoga roda. 
\par 10 Od svećenika: Jedaja, Jojarib i Jakin, 
\par 11 Azarja, sin  Hilkije, sina Mešulama, sina Sadoka, sina Merajota, Ahitubova  sina, predstojnik Doma Božjeg. 
\par 12 Adaja, sin Jerohama, sina  Pašhura, Malkijina sina, Masaj, sin Adiela, sina Jahzere, sina  Mešulama, sina Mešilemita, Imerova sina. 
\par 13 Njihove braće, glava  obitelji, boraca što su obavljali službu u Domu Božjem, bilo  je tisuću sedam stotina i šezdeset. 
\par 14 Od levita Šemaja, sin Hašuba, sin Azrikama, Hašabjina  sina, između Merarijevih sinova; 
\par 15 Bakbakar, Hereš, Galal i  Matanija, sin Mike, sina Zikrija, Asafova sina; 
\par 16 Obadja, sin  Šemaje, sina Galala, Jedutunova sina, i Berekja, sin Ase, Elkanina  sina, koji je živio u Netofatskim selima. 
\par 17 Vratari: Šalum, Akub, Talmon i Ahiman, i njihova braća;  Šalum je bio poglavar, 
\par 18 i dosad je bio na kraljevskim vratima  prema istoku. Oni su bili vratari po četama levita. 
\par 19 Šalum, sin Korea, sina Abjasafa, Korahova sina, sa svojom braćom Korahovcima  iz njihove obitelji, bili su odgovorni za bogoslužje; oni su  čuvali pragove Šatora, dok su njihovi oci čuvali ulaz u Jahvin  tabor. 
\par 20 Eleazarov sin Pinhas bio je predstojnik nad njima  nekada (Jahve bio s njim!). 
\par 21 Mešelemjin sin Zaharija bio je  vratar na vratima Šatora sastanka. 
\par 22 Svih izabranih vratara  pragova bilo je dvjesta i dvanaest. Bili su upisani u rodovnike  u svojim selima. Postavili su ih u službu David i vidjelac Samuel  zbog njihove vjernosti. 
\par 23 Oni i njihovi sinovi čuvali su stražu  na vratima Doma Jahvina, Doma Šatora. 
\par 24 Vratari su stajali  na četiri strane: na istoku, na zapadu, na sjeveru i na jugu. 
\par 25 Njihova braća po selima dolazila su od vremena do vremena  da im se pridruže po sedam dana. 
\par 26 Samo su četiri vratarska  predstojnika bila neprestano u službi. Bili su leviti, postavljeni  nad sobama i nad riznicama Božjega Doma. 
\par 27 Noćivali su oko  Božjega Doma jer im je bila dužnost da stražare i da otključavaju  svako jutro. 
\par 28 Neki su od njih bili odgovorni za bogoslužno posuđe.  Prebrojavali su ga kad bi ga unosili i kad bi ga iznosili. 
\par 29 Neki  su se od njih brinuli za pokućstvo, sve posvećene stvari, fino  brašno, vino, ulje, tamjan i miomirise; 
\par 30 a neki od svećeničkih  sinova miješali su pomast od miomirisa. 
\par 31 Matitja, jedan od levita, prvenac Šaluma Korahovca, brinuo  se za stvari koje se peku na tavi. 
\par 32 Neki od Kehatovaca, njihove  braće, bili su odgovorni za kruhove što se postavljaju svake  subote. 
\par 33 Oni su bili i pjevači, glavari levitskih obitelji. Kad  su bili slobodni, živjeli su u hramskih sobama, jer su dan i  noć bili na dužnosti. 
\par 34 To su bili glavari levitskih obitelji prema svom srodstvu.  Ti su poglavari živjeli u Jeruzalemu. 
\par 35 U Gibeonu su živjeli: Gibeonov otac Jeiel, čijoj je ženi  bilo ime Maaka. 
\par 36 Sin mu je prvenac bio Abdon, pa Sur, Kiš, Baal, Ner, Nadab, 
\par 37 Gedor, Ahjo, Zaharija i Miklot. 
\par 38 Miklot  rodi Šimeama. I oni su živjeli u Jeruzalemu, naprama svojoj braći. 
\par 39 Ner rodi Kiša; a Kiš rodi Šaula; Šaul rodi Jonatana,  Malki-Šuu, Abinadaba i Ešbaala. 
\par 40 Jonatanov je sin bio Merib  Baal. Merib Baal rodi Miku. 
\par 41 Mikini su sinovi bili: Piton, Melek, Tahrea i Ahaz. 
\par 42 Ahaz rodi Jaru; Jara rodi Alemeta, Azmaveta i Zimrija; Zimri rodi Mosu. 
\par 43 Mosa rodi Binu; njegov  je sin bio Rafaja, njegov sin Elasa, njegov sin Asel. 
\par 44 Asel  je imao šest sinova, kojima su imena: Azrikam, Bokru, Jišmael, Šearja, Obadja i Hanan; to su Aselovi sinovi. 


\chapter{10}

\par 1 Filistejci su zavojštili na Izraelce. Izraelci su pobjegli  pred njima i padali pobijeni po gori Gilboi. 
\par 2 Filistejci stisnuše  Šaula i njegove sinove i pogubiše Šaulove sinove Jonatana, Abinadaba  i Malki-Šuu. 
\par 3 Boj je postao žešći oko Šaula. Iznenadiše ga  strijelci s lukovima i on pade ranjen od strijelaca. 
\par 4 Tada  Šaul reče svome štitonoši: "Izvuci svoj mač i probodi me da ne  dođu ti neobrezanci i ne narugaju mi se." Ali se njegov štitonoša  prestravi i ne htjede toga učiniti. Zato Šaul uze mač i baci  se na nj. 
\par 5 Kad je štitonoša vidio da je Šaul umro, baci se  i on na svoj mač i umrije s njim. 
\par 6 Tako onog dana pogiboše  zajedno Šaul, njegova tri sina i sav njegov dom. 
\par 7 Kad su svi  Izraelci koji su bili u dolini vidjeli da su sinovi Izraelovi  pobjegli i da je poginuo Šaul sa sinovima, ostavili su svoje  gradove i razbježali se. Filistejci dođoše i nastaniše se u njima. 
\par 8 Kad su sutradan došli Filistejci da oplijene pobijene, našli su Šaula s njegovim sinovima gdje leže mrtvi na gori Gilboi. 
\par 9 Svukavši ga, uzeše mu glavu i oružje te poslaše po filistejskoj  zemlji unaokolo javljajući veselu vijest svojim idolima i narodu. 
\par 10 Potom su oružje metnuli u hram svoga boga, a lubanju mu izložili  u Dagonovu hramu. 
\par 11 Kad su čuli svi Jabeš-Gileađani što su Filistejci učinili  od Šaula, 
\par 12 ustali su svi hrabri ljudi i uzeli Šaulovo mrtvo  tijelo i tjelesa njegovih sinova i, donijevši ih u Jabeš, pokopali  su njihove kosti pod tamarisom u Jabešu; i postiše sedam dana. 
\par 13 Tako je poginuo Šaul za svoju nevjeru kojom se iznevjerio  Jahvi: nije držao Jahvine zapovijedi i povrh toga je pitao za  savjet bajačicu, 
\par 14 a nije pitao Jahvu; zato ga je ubio i prenio  kraljevstvo na Jišajeva sina Davida. 


\chapter{11}

\par 1 Tada se sabraše svi Izraelci k Davidu u Hebron i rekoše: "Evo, mi smo od tvoje kosti i tvojeg mesa. 
\par 2 Još prije, dok je Šaul  bio kralj, ti si upravljao svim pokretima Izraela; Jahve, tvoj  Bog, rekao ti je: 'Ti ćeš pasti moj izraelski narod i ti ćeš  biti knez nad mojim narodom Izraelom.'" 
\par 3 Tako dođoše sve izraelske  starješine kralju u Hebron, a kralj David s njima sklopi savez  u Hebronu pred Jahvom i pomazaše Davida za kralja nad Izraelom, kako bijaše Jahve rekao Samuelu. 
\par 4 Onda je otišao David sa svim Izraelom na Jeruzalem, a  to je Jebus, jer su ondje bili Jebusejci i živjeli su u onoj  zemlji. 
\par 5 Ali su Jebusejci poručili Davidu: "Nećeš ući ovamo!"  Ipak David osvoji Sionsku tvrđavu, to jest Davidov grad. 
\par 6 Jer  je David rekao: "Tko prvi porazi Jebusejce, bit će vrhovni vojvoda  i knez." Prvi se popeo Sarvijin sin Joab i postao vojvoda. 
\par 7 Tada  se David nastanio u toj tvrđavi; zato su je prozvali Davidovim  gradom. 
\par 8 Sazidao je tada grad unaokolo, od Milona do ograde, a Joab je obnovio ostali dio grada. 
\par 9 David je postajao sve  silniji, jer je Jahve nad vojskama bio s njim. 
\par 10 Evo vojvoda Davidovim junacima koji su junački radili  uza nj za njegovo kraljevstvo sa svim Izraelom da ga po Jahvinoj  riječi zakralje nad Izraelom. 
\par 11 Evo popisa Davidovih junaka:  Hakmonijev sin Jašobam, glavar nad tridesetoricom; on je mahnuo  svojim kopljem na tri stotine i pobio ih odjednom. 
\par 12 Za njim  Dodonov sin Eleazar, Ahošanin, jedan između tri junaka. 
\par 13 On  je bio s Davidom u Pas Damimu, kad su se Filistejci skupili na  boj, a ondje je bilo polje puno ječma; kad je narod počeo bježati  ispred Filistejaca, 
\par 14 oni su stali usred toga polja i obranili  ga pobivši Filistejce. Tako im Jahve dade veliku pobjedu. 
\par 15 Trojica su između tridesetorice jednom sišla do hridi  k Davidu u Adulamsku pećinu kad su filistejske čete stajale u  taboru u Refaimskoj dolini. 
\par 16 David je tada bio u svojoj kuli, a filistejska je posada tada bila u Betlehemu. 
\par 17 David uzdahnu:  "O kad bi me tko napojio vodom iz betlehemskoga studenca što  je kod vrata!" 
\par 18 Tada ta trojica prodriješe kroz filistejski  tabor i, zahvativši vode iz betlehemskoga studenca što je kod  vrata, donesoše je i dadoše Davidu. Ali je David ne htjede piti  nego je proli kao ljevanicu Jahvi 
\par 19 govoreći: "Ne dao mi moj  Bog da to učinim! Zar da pijem krv ovih ljudi? TÓa izlažući život  pogibli donijeli su vode." I nije htio piti. To su, eto, učinila  ta tri junaka. 
\par 20 Abišaj, Joabov brat, bio je vojvoda nad tridesetoricom;  on je vitlao kopljem na tri stotine, pobio ih i proslavio se  među tridesetoricom. 
\par 21 Bio je među trojicom ugledniji od druge  dvojice i bio im vojvoda, ali prve trojice nije dostigao. 
\par 22 Jojadin sin Benaja, junak iz Kabseela, bogat junačkim  djelima, ubio je dva sina Ariela iz Moaba; on je jednoga snježnog  dana sišao i ubio lava usred jame. 
\par 23 Ubio je i nekog Egipćanina, čovjeka od pet lakata. Egipćanin je imao u ruci koplje kao tkalačko  vratilo, a on je izišao preda nj sa štapom i, istrgavši Egipćaninu  koplje iz ruke, ubio ga njegovim kopljem. 
\par 24 To je učinio Jojadin  sin Benaja i proslavio se imenom među ona tri junaka. 
\par 25 Bio  je najznamenitiji među tridesetoricom, ali one prve trojice nije  dostigao. David ga postavi za zapovjednika svoje tjelesne straže. 
\par 26 Hrabri su junaci bili: Joabov brat Asahel, Dodonov sin  Elhanan iz Betlehema, 
\par 27 Haroranin Šamot, Pelonjanin Heles; 
\par 28 Akešov sin Ira, Tekoanin, Abiezer Anatoćanin; 
\par 29 Sibkaj  Hušaćanin, Ilaj Ahošanin; 
\par 30 Mahraj Netofaćanin, Baanin sin  Heled, Netofaćanin; 
\par 31 Ribajev sin Itaj iz Gibeata sinova Benjaminovih, Benaja Piratonjanin; 
\par 32 Huraj iz Gaaških potoka, Abiel Arbaćanin; 
\par 33 Azmavet Bahurimljanin, Eljahba Šaalbonjanin. 
\par 34 Sinovi Hašema  Gizonjanina: Sagejin sin Jonatan, Hararanin; 
\par 35 Sakarov sin  Ahiam, Hararanin, Urov sin Elipal; 
\par 36 Hefer Mekeranin, Ahija  Pelonjanin; 
\par 37 Hesro Karmelac, Ezbajev sin Naaraj; 
\par 38 Natanov  brat Joel, Hagrijev sin Mibhar; 
\par 39 Amonac Selek, Beroćanin Nahraj, štitonoša Sarvijina sina Joaba; 
\par 40 Ira Jitranin, Gareb Jitranin; 
\par 41 Urija Hetit, Ahlajev sin Zabad; 
\par 42 Šizin sin Adina, Rubenovac, vojvoda Rubenova plemena, i s njime tridesetorica. 
\par 43 Maakin  sin Hanan i Jošafat Mitnjanin. 
\par 44 Uzija Aštaroćanin, Šama i  Jeiel, sinovi Aroerca Hotama; 
\par 45 Šimrijev sin Jediael i njegov  brat Joha Tišanin. 
\par 46 Mahavac Eliel i Elnaamovi sinovi Jeribaj  i Jošavja i Moabac Jitma; 
\par 47 Eliel i Obed i Mesobajanin Jaasiel. 


\chapter{12}

\par 1 Evo onih što dođoše k Davidu u Siklag dok se još uklanjao  od Kiševa sina Šaula i bili su mu među junacima pomagači u boju; 
\par 2 umjeli su rukovati lukom i desnicom i ljevicom i znali se  služiti kamenjem i strijelama. Između Šaulove braće, Benjaminovaca: 
\par 3 vojvoda Ahiezer i  Joaš, sinovi Gibeanca Šemaje, pa Jeziel i Pelet, Azmavetovi sinovi, i Beraka i Jehu Anatoćanin; 
\par 4 Gibeonac Išmaja, junak među tridesetoricom  i nad tridesetoricom, (12:5) Jeremija, Jahaziel, Johanan i Jozabad  Gederoćanin; 
\par 5 (12:6) Eluzaj, Jerimot, Bealja, Šemarja i Šefatja Harufejac; 
\par 6 (12:7) Elkana, Jišija, Azarel, Joezer i Jašobam Korhinjani, 
\par 7 (12:8) Joel  i Zebadja, sinovi Jerohama Gedorca. 
\par 8 (12:9) Neki su Gadovci prešli k Davidu u tvrđavu u pustinju,  hrabri junaci, ratnici vješti boju, naoružani štitom i kopljem;  lica im bijahu kao lavovska, a brzi bijahu kao gazele po gorama: 
\par 9 (12:10) vojvoda Ezer, drugi Obadja, treći Eliab; 
\par 10 (12:11) četvrti Mišmana, peti Jeremija, 
\par 11 (12:12) šesti Ataj, sedmi Eliel; 
\par 12 (12:13) osmi Johanan, deveti Elzabad, 
\par 13 (12:14) deseti Jeremija, jedanaesti Makbanaj. 
\par 14 (12:15) To  su bile od Gadovih sinova vojne starješine, najmanji nad stotinom, a najveći nad tisućom. 
\par 15 (12:16) To su oni koji su prvoga mjeseca  prešli preko Jordana kad se razlio preko svih svojih obala i  koji su rastjerali sve stanovnike iz dubokih dolina na istok  i na zapad. 
\par 16 (12:17) Došli su i od Benjaminovih i Judinih sinova k Davidu  u tvrđavu. 
\par 17 (12:18) David je izašao pred njih i, progovorivši, rekao  im: "Ako dolazite s mirom k meni da mi pomognete, moje će se  srce ujediniti s vama; ako li ste došli da me izdate mojim neprijateljima, neka Bog naših otaca vidi i neka osudi, jer nema nepravde na  mojim rukama!" 
\par 18 (12:19) Tada duh obuze Amasaja, vojvodu nad tridesetoricom, i  on reče: "Tebi, Davide! S tobom, sine Jišajev, mir! Mir s tobom, mir s onim tko ti pomaže, jer tvoj pomoćnik jest tvoj Bog!" Tako ih je David primio i postavio ih među vojvode nad četama. 
\par 19 (12:20) Od Manašeovih su sinova neki prešli k Davidu kad je išao  s Filistejcima na Šaula u boj, ali im nije pomogao, jer su ga  filistejski knezovi, dobro promislivši, otpustili govoreći: "Mogao  bi prijeći k svome gospodaru Šaulu, a to bi nas stajalo glava." 
\par 20 (12:21) Kad se, dakle, vraćao u Siklag, prešli su k njemu od Manašeova  plemena: Adna, Jozabad, Jedael, Mihael, Jozabad, Elihu i Siltaj, glavari tisućnici u Manašeovu plemenu. 
\par 21 (12:22) Oni su pomagali Davidu  protiv razbojničkih četa jer su svi bili hrabri junaci te su  postali zapovjednici u njegovoj vojsci. 
\par 22 (12:23) Iz dana u dan odista  su dolazili k Davidu da mu pomažu, sve dok njegov tabor ne postade  divovski, kao Božji tabor. 
\par 23 (12:24) Evo broja ljudi naoružanih za rat koji su došli k Davidu  u Hebron da Šaulovo kraljevstvo prenesu na nj po Jahvinoj zapovijedi: 
\par 24 (12:25) Judinih sinova, koji su nosili štit i koplje, šest tisuća  i osam stotina naoružanih za rat. 
\par 25 (12:26) Od Šimunovih sinova, hrabrih junaka za rat, sedam tisuća  i sto. 
\par 26 (12:27) Od Levijevih sinova četiri tisuće i šest stotina. 
\par 27 (12:28) Tako i Jojada, poglavar Aronovim potomcima, i s njim  tri tisuće i sedam stotina; 
\par 28 (12:29) i mladi Sadok, hrabar junak,  i od njegova roda dvadeset i dva kneza. 
\par 29 (12:30) A od Benjaminovih  sinova, Šaulove braće, tri tisuće, jer ih je dotad najveći dio  još ostao vjeran Šaulovoj kući. 
\par 30 (12:31) Efrajimovih sinova dvadeset tisuća i osam stotina, sve  hrabrih junaka, ljudi na glasu u svojim porodicama. 
\par 31 (12:32) Od polovine Manašeova plemena osamnaest tisuća, poimence  spomenutih, da dođu da zakralje Davida. 
\par 32 (12:33) Od Jisakarovih sinova, koji su umjeli proniknuti svoje  vrijeme i spoznati što treba da učini Izrael; njihovih poglavara  dvije stotine. Sva su im njihova braća bila podložna. 
\par 33 (12:34) Od Zebulunovih sinova, sposobnih za rat i naoružanih  za boj svakojakim bojnim oružjem, pedeset tisuća, koji su se  odvažna srca vrstali u bojne redove. 
\par 34 (12:35) Od Naftalijeva plemena tisuću knezova i s njima trideset  i sedam tisuća ljudi sa štitovima i kopljima; 
\par 35 (12:36) od Danova plemena  dvadeset i osam tisuća i šest stotina naoružanih za boj, 
\par 36 (12:37) a  od Ašerova plemena četrdeset tisuća sposobnih za vojsku i za  boj opremljenih. 
\par 37 (12:38) Od onih s onu stranu Jordana, od Rubenova, od Gadova i od polovine Manašeova plemena, sto i dvadeset tisuća  ljudi sa svakojakim ratnim oružjem. 
\par 38 (12:39) Svi ti vojnici, svrstani u bojne redove, dođoše poštena  srca u Hebron da zakralje Davida nad svim Izraelom; i svi su  ostali Izraelci bili jednodušni da Davida postave za kralja. 
\par 39 (12:40) Proveli su s Davidom tri dana, jedući i pijući. Braća sve  spremiše za njih. 
\par 40 (12:41) Njihovi su najbliži susjedi, sve do Jisakara, Zebuluna i Naftalija, donosili hranu na magarcima, devama i  mazgama, a na volovima jela: brašna, smokvenih kolača, suha grožđa, vina, ulja, krupne i sitne stoke izobila, jer je bilo veselje  u Izraelu. 


\chapter{13}

\par 1 David je vijećao s tisućnicima, stotnicima i sa svim vođama. 
\par 2 I reče on svemu zboru Izraelovu: "Ako vam je pravo te ako  je naš Bog Jahve odlučio tako, poslat ćemo glasnike k svojoj  ostaloj braći u svim izraelskim zemljama, a tako i svećenicima  s njima i levitima po gradovima pašnjaka njihovih, da se ujedine  s nama. 
\par 3 Prenijet ćemo k sebi Kovčeg svoga Boga, jer ga nismo  doista tražili za Šaulovih dana." 
\par 4 Sav zbor odluči da se tako  učini, jer je to bilo pravo u očima svega naroda. 
\par 5 Tako je  David sabrao sav narod Izraelov od Egipatskoga Šihora pa do Ulaza  u Hamat da donesu Kovčeg Božji iz Kirjat Jearima. 
\par 6 Pošao je  David sa svim Izraelom u Baalu, u Kirjat Jearim, koji je u Judi, da odande ponesu Kovčeg Božji nazvan imenom Jahve, koji stoluje  nad kerubinima. 
\par 7 Povezli su Kovčeg Božji na novim kolima iz  Abinadabove kuće; a Uza i Ahjo upravljali su kolima. 
\par 8 David  i sav Izrael igrali su pred Bogom iz sve snage pjevajući uza  zvuke citara, harfa, bubnjeva, cimbala i truba. 
\par 9 Kad su došli  do Kidonova gumna, posegnu Uza rukom da pridrži Kovčeg jer ga  volovi umalo ne prevrnuše. 
\par 10 Ali se Jahve razgnjevio na Uzu  i udario ga zato što je pružio ruku prema Kovčegu. Umro je ondje  pred Bogom. 
\par 11 Davidu bijaše žao što je Jahve onako udario Uzu  i on prozva ono mjesto Peres Uza, kako se zove i dan-danas. 
\par 12 Toga  se dana David uplaši Boga i reče: "Kako ću donijeti k sebi Kovčeg  Božji?" 
\par 13 Nije dao svratiti Kovčega k sebi u Davidov grad nego  ga skloni u kuću Obed-Edoma Gitejca. 
\par 14 I ostade Kovčeg Božji  kod Obed-Edomove obitelji, u njegovoj kući, tri mjeseca. Jahve  stoga blagoslovi Obed-Edomovu kuću i sve što je imao. 


\chapter{14}

\par 1 Tirski kralj Hiram posla k Davidu izaslanstvo i cedrovih drva, zidara i tesara da mu grade dvor. 
\par 2 Tada David spozna da ga  je Jahve potvrdio za kralja nad Izraelom i da je uzvisio njegovo  kraljevstvo radi svojega izraelskog naroda. 
\par 3 David je uzeo još žena u Jeruzalemu i imao još sinova  i kćeri. 
\par 4 Evo imena djece koja mu se rodiše u Jeruzalemu: Šamua, Šobab, Natan, Salomon, 
\par 5 Jibhar, Elišua, Elpalet, 
\par 6 Nogah, Nefeg, Jafija, 
\par 7 Elišama, Beeljada i Elifelet. 
\par 8 Kad su Filistejci čuli da su Davida pomazali za kralja  nad svim Izraelom, iziđoše svi da se dočepaju Davida. David,  čuvši to, iziđe pred njih. 
\par 9 Filistejci dođoše i raširiše se  po Refaimskoj dolini. 
\par 10 Tada David upita Boga: "Mogu li napasti  Filistejce? Hoćeš li ih predati meni u ruke?" Jahve mu odgovori:  "Napadni, jer ću ih predati tebi u ruke!" 
\par 11 Tada krenuše u  Baal Perasim i David ih ondje pobi. David reče: "Bog je prodro  među moje neprijatelje mojom rukom, kao što voda prodire." Stoga  se ono mjesto prozvalo Baal Perasim. 
\par 12 Ostavili su ondje svoje  bogove; a David zapovjedi da ih spale. 
\par 13 Opet se Filistejci raširiše po onoj dolini. 
\par 14 David  opet upita Boga, a Bog mu odgovori: "Ne idi za njima nego ih  opkoli i navali na njih s protivne strane Bekaima. 
\par 15 Pa kad  začuješ topot koraka po bekaimskim vrhovima, onda izađi u boj, jer će tada ići Bog pred tobom da pobije filistejsku vojsku." 
\par 16 David učini kako mu je zapovjedio Bog; i pobili su filistejsku  vojsku od Gibeona do Gezera. 
\par 17 Davidovo se ime pročulo po svim zemljama, a Jahve uli  strah od njega svim narodima. 


\chapter{15}

\par 1 Onda je David sazidao dvore u Davidovu gradu, pripravio mjesto  za Kovčeg Božji i razapeo mu Šator. 
\par 2 Potom je rekao David:  "Ne smije nositi Kovčeg Božji nitko osim levita, jer je njih  izabrao Jahve da nose Kovčeg Jahvin i da mu služe dovijeka." 
\par 3 David je sakupio sav Izrael u Jeruzalem da prenesu Kovčeg  Jahvin gore na njegovo mjesto koje mu bijaše pripravio. 
\par 4 Skupio  je David i Aronove sinove i levite. 
\par 5 Od Kehatovih sinova: kneza  Uriela i sto dvadeset njegove braće; 
\par 6 od Merarijevih sinova:  kneza Asaju i dvjesta dvadeset njegove braće; 
\par 7 od Geršomovih  sinova: kneza Joela i sto trideset njegove braće. 
\par 8 Od Elisafanovih  sinova: kneza Šemaju i dvjesta njegove braće. 
\par 9 Od Hebronovih  sinova: kneza Eliela i osamdeset njegove braće; 
\par 10 od Uzielovih  sinova: kneza Aminadaba i sto dvanaest njegove braće. 
\par 11 Tada David pozva svećenike Sadoka i Ebjatara i levite  Uriela, Asaju, Joela, Šemaju, Eliela i Aminadaba, 
\par 12 pa im reče:  "Vi ste glavari levitskih porodica; posvetite sebe i svoju braću  da prenesete gore Kovčeg Jahve, Izraelova Boga, na mjesto koje  sam mu pripravio. 
\par 13 Jer nas je pobio Jahve, Bog naš, zato što  prvi put vi niste bili nazočni i što ga nismo tražili onako kako  je trebalo." 
\par 14 Posvetiše se tada svećenici i leviti da prenesu  gore Kovčeg Jahve, Izraelova Boga. 
\par 15 Levitski su sinovi ponijeli  Božji Kovčeg, na svojim ramenima, o motkama, kako je zapovjedio  Mojsije po Jahvinoj riječi. 
\par 16 Tada David reče levitskim knezovima da između svoje braće  postave pjevače s glazbalima, s harfama, citrama i cimbalima  da se čuje i da gromko odjekuje radosno pjevanje. 
\par 17 Leviti su postavili Joelova sina Hemana, a od njegove  braće Berekjina sina Asafa, i od njihove braće, Merarijevih sinova, Kušajina sina Etana. 
\par 18 S njima njihovu braću drugoga reda:  Zahariju, sina Jaazielova, Šemiramota, Jehiela, Unija, Eliaba, Benaju, Maaseju, Matitju, Eliflehua, Mikneju, Obed Edoma i Jeiela, vratare. 
\par 19 A pjevači, Heman, Asaf i Etan gromko su udarali  u mjedene cimbale. 
\par 20 A Zaharija, Uziel, Šemiramot, Jehiel,  Uni, Eliab, Maaseja i Benaja u harfe s visokim zvucima; 
\par 21 a  Matitja, Eliflehu, Mikneja, Obed Edom, Jeiel i Azazja u citre, u osminskoj pratnji. 
\par 22 Kenanja, knez onih levita koji su nosili  Kovčeg, upravljao je prenošenjem jer je bio vješt u tome. 
\par 23 Berekja  i Elkana bili su vratari kod Kovčega. 
\par 24 Šebanija, Jošafat,  Netanel, Amasaj, Zaharija, Benaja i Eliezer, svećenici, trubili  su u trube pred Božjim Kovčegom; Obed Edom i Jehija bili su vratari  kod Kovčega. 
\par 25 Tako je David s izraelskim starješinama i tisućnicima  radosno išao prenoseći gore Kovčeg saveza Jahvina iz Obed-Edomove  kuće. 
\par 26 Kad je Bog pomogao levitima koji su nosili Kovčeg saveza  Jahvina, žrtvovali su sedam junaca i sedam ovnova. 
\par 27 David  bijaše ogrnut plaštem od tanka platna, a tako i svi leviti što  su nosili Kovčeg, kao i pjevači i Kenanija koji je upravljao  pjevačima. David je imao na sebi lanen oplećak. 
\par 28 Tako je sav  Izrael prenosio gore Kovčeg saveza Jahvina, radosno kličući uz  jeku rogova, truba i cimbala, igrajući uza zvuke harfe i citre. 
\par 29 Kad je Kovčeg saveza Jahvina ulazio u Davidov grad, Šaulova  kći Mikala, gledajući s prozora, vidje kralja Davida kako skače  i igra i prezre ga ona u svom srcu. 


\chapter{16}

\par 1 Tada unesoše Kovčeg Božji i postaviše ga usred šatora koji  mu bijaše razapeo David. Onda su prinijeli paljenice i pričesnice  pred Bogom. 
\par 2 Pošto je prinio paljenice i pričesnice, David  blagoslovi narod Jahvinim imenom. 
\par 3 Onda razdijeli svim Izraelcima, ljudima i ženama, svakome po jedan okrugao kruh, komad mesa  i kolač od suhoga grožđa. 
\par 4 Onda je postavio pred Jahvinim Kovčegom službenike među  levitima da uznose, slave i hvale Jahvu, Boga Izraelova, i to: 
\par 5 poglavara Asafa, a drugoga za njim Zahariju, zatim Jeiela, Šemiramota, Jehiela, Matitju, Eliaba, Benaju, Obed Edoma i Jeiela  s harfama i citrama; Asaf je udarao u cimbale. 
\par 6 Svećenici Benaja  i Jahaziel bili su bez prijekida s trubama pred Kovčegom saveza  Jahvina. 
\par 7 Toga dana povjeri David prvi put Asafu i njegovoj  braći da slave Jahvu ovom pohvalnicom: 
\par 8 "Hvalite Jahvu, prizivajte mu ime; navješćujte među narodima djela njegova! 
\par 9 Pjevajte mu, svirajte mu, propovijedajte sva njegova čudesa! 
\par 10 Dičite se svetim imenom njegovim, neka se raduje srce onih što traže Jahvu! 
\par 11 Tražite Jahvu i njegovu snagu, tražite svagda njegovo lice! 
\par 12 Sjetite se čudesa koja učini, njegovih čuda i sudova usta njegovih. 
\par 13 Izraelov rod njegov je sluga, sinovi Jakovljevi njegovi izabranici. 
\par 14 On je Jahve, Bog naš; po svoj su zemlji njegovi sudovi! 
\par 15 Sjećajte se uvijek njegova Saveza, Riječi koju objavi tisući naraštaja; 
\par 16 Saveza koji sklopi s Abrahamom i njegove zakletve Izaku. 
\par 17 Ustanovi je kao zakon Jakovu, Izraelu vječni Savez. 
\par 18 Govoreći 'Tebi ću dati kanaansku zemlju kao dio u baštinu vašu, 
\par 19 kad vas još bješe malo na broju, vrlo malo, i kad bjeste pridošlice u njoj.' 
\par 20 Išli su od naroda do naroda, iz jednoga kraljevstva k drugom narodu. 
\par 21 Ne dopusti nikom da ih tlači, kažnjavaše zbog njih kraljeve: 
\par 22 'Ne dirajte u moje pomazanike, ne činite zla mojim prorocima!' 
\par 23 Pjevaj Jahvi, sva zemljo, Navješćujte iz dana u dan spasenje njegovo! 
\par 24 Kazujte poganima njegovu slavu, svim narodima čudesa njegova. 
\par 25 Velik je Jahve, hvale predostojan, strašniji od svih bogova. 
\par 26 Ništavni su svi bozi naroda. Jahve stvori nebesa. 
\par 27 Slava je i veličanstvo pred njim, sila i radost u Svetištu njegovu. 
\par 28 Dajte Jahvi, narodna plemena, dajte Jahvi slavu i silu! 
\par 29 Dajte Jahvi slavu imena njegova, nosite prinose i dolazite pred njegovo lice! Poklonite se Jahvi u sjaju svetosti njegove! 
\par 30 Strepi pred njim, zemljo sva! Učvrstio je svemir da se ne poljulja. 
\par 31 Neka se vesele nebesa i neka klikće zemlja; neka se govori među poganima: 'Jahve kraljuje!' 
\par 32 Neka huči more i što je u njemu; nek' se raduje polje i što je na njemu! 
\par 33 Neka klikće šumsko drveće pred Jahvom, jer dolazi da sudi zemlji. 
\par 34 Slavite Jahvu jer je dobar, jer je vječna ljubav njegova. 
\par 35 I recite: 'Spasi nas, o Bože, Spasitelju naš, i saberi nas i izbavi nas od bezbožnih naroda, da slavimo tvoje sveto ime, da se ponosimo tvojom slavom. 
\par 36 Blagoslovljen Jahve, Bog Izraelov, od vijeka do vijeka!' Sav narod neka kaže: 'Amen! Aleluja!'" 
\par 37 I ondje pred Kovčegom saveza Jahvina ostaviše Asafa i  njegovu braću da služe pred Kovčegom bez prestanka, koliko treba  iz dana u dan; 
\par 38 i Obed-Edoma s njegovom braćom, njih šezdeset  i osam, i Obed-Edoma, Jedutunova sina, i Hosu, da budu vratari; 
\par 39 a svećenika Sadoka s njegovom braćom svećenicima pred Jahvinim  Prebivalištem na uzvišici u Gibeonu 
\par 40 da prinose paljenice  Jahvi na žrtveniku za paljenice bez prestanka, jutrom i večerom, i da vrše sve što je napisano u Zakonu koji je Jahve odredio  Izraelu; 
\par 41 s njima Hemana i Jedutuna i ostale izabrane, koji  su bili poimence spomenuti, da slave Jahvu, "jer je vječna njegova  ljubav"; 
\par 42 i to Hemana i Jedutuna da trube u trube i udaraju  u cimbale i druga glazbala Bogu na čast; a Jedutunove sinove  da budu vratari. 
\par 43 Tada se razišao sav narod, svatko svojoj kući; a David  se vratio da blagoslovi svoj dvor. 


\chapter{17}

\par 1 Kad se David nastanio u dvoru, rekao je proroku Natanu: "Pogledaj!  Ja, evo, stojim u dvoru od cedrovine, a Kovčeg saveza Jahvina  pod zavjesama!" 
\par 2 Natan odgovori Davidu: "Što ti je god na srcu, čini, jer je Bog s tobom." 
\par 3 Ali još iste noći dođe Natanu ova Božja riječ: 
\par 4 "Idi i reci mome sluzi Davidu: 'Ovako govori Jahve: Ti  mi nećeš sagraditi kuće da prebivam u njoj. 
\par 5 Nisam nikad prebivao  u kući otkako sam izveo Izraela iz Egipta pa do današnjega dana, nego sam išao od šatora do šatora i od prebivališta do prebivališta. 
\par 6 Dok sam hodio sa svim Izraelom, jesam li ijednu riječ rekao  nekom od Izraelovih sudaca, kojima sam zapovjedio da budu pastiri  mojem narodu, i kazao: Zašto mi ne sagradite kuću od cedrovine?' 
\par 7 Zato sad ovo reci mome sluzi Davidu: 'Ovako govori Jahve nad  vojskama: Ja sam te doveo s pašnjaka, od ovaca i koza, da budeš  knez nad mojim izraelskim narodom. 
\par 8 Bio sam s tobom kuda si  god išao, iskorijenio sam sve tvoje neprijatelje pred tobom.  Ja ću ti pribaviti veliko ime, kao što je velikaško ime na zemlji. 
\par 9 Odredit ću prebivalište svome izraelskom narodu i posadit  ću ga da živi na svojem mjestu i da ne luta više naokolo niti  da ga zlikovci muče kao prije, 
\par 10 onda kad sam odredio suce  nad svojim izraelskim narodom. Pokorit ću sve tvoje neprijatelje  i učinit ću te velikim. Jahve će ti podići dom. 
\par 11 Jer kad se  ispune tvoji životni dani i dođe vrijeme da počineš kod otaca, podići ću tvoga potomka nakon tebe, koji će biti između tvojih  sinova, i utvrdit ću njegovo kraljevstvo. 
\par 12 On će mi sagraditi  dom, a ja ću utvrditi njegovo prijestolje zauvijek. 
\par 13 Ja ću  njemu biti otac, a on će meni biti sin: svoje naklonosti neću  odvratiti od njega, kao što sam je odvratio od tvoga prethodnika. 
\par 14 Utvrdit ću ga u svojem domu i u svom kraljevstvu zauvijek, i prijestolje će mu čvrsto stajati zasvagda.'" 
\par 15 Natan prenese Davidu sve te riječi i cijelo viđenje. 
\par 16 Tada kralj David dođe i stade pred Jahvu i reče: "Tko sam ja, o Bože Jahve, i što je moj dom te si me doveo  dovde? 
\par 17 Pa i to je bilo premalo u tvojim očima, o Bože, nego  si dao obećanja domu svoga sluge i za daleku budućnost i pogledao  si na me kako se gleda na ugledna čovjeka, o Bože Jahve! 
\par 18 Pa  što da ti još David govori o slavi tvoga sluge; tÓa ti poznaješ  svoga slugu! 
\par 19 Jahve, radi svoga sluge i po svome srcu učinio  si sve ovo veliko djelo, obznanivši ove veličajnosti. 
\par 20 Jahve, nema takvoga kakav si ti, niti ima Boga osim tebe, po svemu  što smo ušima svojim čuli. 
\par 21 Postoji li ijedan narod na zemlji  kao tvoj izraelski narod, radi kojega je Bog išao da ga izbavi  sebi za narod, da tako stečeš sebi ime velikim i strašnim čudesima, izgoneći krivobožačka plemena pred svojim narodom koji si otkupio  iz Egipta? 
\par 22 Tako si učinio svoj izraelski narod svojim narodom  zauvijek, a ti si mu, Jahve, postao Bogom. 
\par 23 Zato sada, Jahve, neka bude čvrsta dovijeka riječ koju si dao svome sluzi i njegovu  domu i učini kako si obrekao. 
\par 24 Neka bude čvrsta, da se veliča  tvoje ime zauvijek i da se govori: Jahve nad vojskama, Izraelov  Bog, jest Bog nad Izraelom, a dom tvoga sluge Davida neka stoji  čvrsto pred tobom. 
\par 25 Jer si ti, moj Bože, javio uhu svoga sluge  da ćeš mu podići dom, zato je tvoj sluga smogao hrabrosti da  se pomoli pred tobom. 
\par 26 Uistinu, Jahve, ti si Bog i ti si ovo  lijepo obećanje dao svome sluzi. 
\par 27 Zato se sada udostoj blagosloviti  dom svoga sluge da ostane dovijeka pred tobom, jer kad ti, Jahve, blagosloviš, bit će blagoslovljen zasvagda." 


\chapter{18}

\par 1 Poslije toga David porazi Filistejce i pokori ih te ote Gat  s njegovim selima iz filistejskih ruku. 
\par 2 Porazio je i Moapce  i oni postadoše Davidovi podanici koji su mu donosili danak. 
\par 3 David je porazio i Hadadezera, sopskoga kralja u Hamatu, kad je izišao da utvrdi svoju vlast do rijeke Eufrata. 
\par 4 David  zarobi od njega tisuću bojnih kola, sedam tisuća konjanika i  dvadeset tisuća pješaka; ispresijecao je petne žile svim konjima  od bojnih kola, ostavio ih je samo stotinu. 
\par 5 Damaščanski su  Aramejci bili došli u pomoć Hadadezeru, sopskome kralju, ali  je David pobio među Aramejcima dvadeset i dvije tisuće ljudi. 
\par 6 Postavio je namjesnike u Damaščanskom Aramu. Tako Aramejci  postadoše Davidovi podanici i moradoše mu plaćati danak. Jahve  je davao pobjedu Davidu kuda je god išao. 
\par 7 David zaplijeni  zlatne štitove što ih imahu Hadadezerove sluge i donese ih u  Jeruzalem. 
\par 8 I iz Hadadezerovih gradova Tibhata i Kuna odnio  je silni tuč od kojega je Salomon načinio mjedeno more, stupove  i tučano posuđe. 
\par 9 Kad je čuo hamatski kralj Tou da je David porazio svu  vojsku Hadadezera, sopskoga kralja, 
\par 10 posla svoga sina Hadorama  kralju Davidu da ga pozdravi i da mu čestita što je vojevao protiv  Hadadezera i porazio ga, jer je Tou bio u ratu s Hadadezerom;  i da mu odnese svakojakih zlatnih, srebrnih i tučanih predmeta. 
\par 11 I njih je kralj David posvetio Jahvi sa srebrom i zlatom  što ga bijaše uzeo od svih naroda, od Edomaca, Moabaca, Amonaca, Filistejaca i Amalečana. 
\par 12 Sarvijin sin Abišaj pobio je osamnaest  tisuća Edomaca u Slanoj dolini. 
\par 13 David je postavio namjesnike  po Edomu. Tako su svi Edomci postali Davidove sluge. I kuda je  god David išao, Jahve mu davaše pobjedu. 
\par 14 David kraljevaše nad svim Izraelom čineći pravo i pravicu  svemu svome narodu. 
\par 15 Sarvijin je sin Joab bio zapovjednik  vojske; Ahiludov sin Jošafat bijaše tajni savjetnik. 
\par 16 Ahitubov sin Sadok i Ahimelekov sin Ebjatar bili su svećenici, Šavša pisar. 
\par 17 Jojadin sin Benaja bio je nad Kerećanima i  Pelećanima, a Davidovi su sinovi bili prvi do kralja. 


\chapter{19}

\par 1 Poslije toga umrije Nahaš, kralj Amonaca, i zakralji mu se  sin na njegovo mjesto. 
\par 2 David reče u sebi: "Iskazat ću ljubav  Nahaševu sinu Hanunu jer je i njegov otac iskazao milost meni."  David uputi poslanike da mu izraze sućut zbog smrti njegova oca.  Kad su Davidove sluge došle u zemlju Amonaca k Hanunu da mu izraze  sućut, 
\par 3 rekoše knezovi Amonaca Hanunu: "Zar misliš da je David  poslao ljude da ti izraze sućut zato što bi htio iskazati čast  tvome ocu? Nisu li zato došle njegove sluge k tebi da razvide, istraže i uhode zemlju?" 
\par 4 Tada Hanun pograbi Davidove sluge  i obrija ih, podreza im haljine dopola, do zadnjice, i posla  ih natrag! 
\par 5 Kad su to javili Davidu, posla on čovjeka pred  njih, jer su bili vrlo osramoćeni, i poruči im: "Ostanite u Jerihonu  dok vam ne naraste brada pa se onda vratite." 
\par 6 Kad su Amonovi sinovi vidjeli da su se omrazili s Davidom, poslao je Hanun s Amonovim sinovima tisuću srebrnih talenata  da za plaću najme bojnih kola i konjanika iz Aram Naharajima, iz Aram Maake i iz Soba. 
\par 7 Najmili su za plaću trideset i dvije  tisuće bojnih kola, i kralja Maake s njegovim narodom te su oni  došli i utaborili se pred Medebom. Amonovi su se sinovi skupili  iz svojih gradova i došli u boj. 
\par 8 Kad je to čuo David, poslao  je Joaba sa svom svojom junačkom vojskom. 
\par 9 Amonovi sinovi iziđoše  i svrstaše se u bojni red pred gradskim vratima; a kraljevi koji  su došli stajali su zasebno na polju. 
\par 10 Vidjevši postavljene  bojne redove prema sebi, sprijeda i straga, Joab probra najvrsnije  među Izraelcima i svrsta ih prema Aramejcima. 
\par 11 Ostalu vojsku  predade bratu Abišaju da je svrsta prema Amoncima. 
\par 12 I reče  mu: "Ako Aramejci budu jači od mene, onda ti meni priskoči u  pomoć; ako li Amonci budu jači od tebe, ja ću tebi pohrliti u  pomoć. 
\par 13 Budi hrabar i junački se držimo radi naroda i radi  gradova svoga Boga; a Jahve neka učini što je dobro u njegovim  očima." 
\par 14 Tada se Joab i vojska koja je bila s njim počeše  primicati da udare na Aramejce, ali oni pobjegoše pred njima. 
\par 15 Kad su Amonci vidjeli da su Aramejci pobjegli, umakoše i  oni ispred njegova brata Abišaja i povukoše se u grad. Tada se  Joab vrati u Jeruzalem. 
\par 16 A Aramejci, vidjevši gdje su ih potukli Izraelci, uputili  su poslanike i doveli Aramejce što su s onu stranu Rijeke, na  čelu sa Šofakom, vojvodom Hadadezerove vojske. 
\par 17 Pošto su to javili Davidu, on skupi sve Izraelce i, prešavši  preko Jordana, primače se Aramejcima i svrsta se prema njima;  kad se David svrstao prema Aramejcima u bojni red, oni zametnuše  s njime boj. 
\par 18 Ali Aramejci udariše u bijeg ispred Izraelaca  i David im pobi sedam tisuća konja od bojnih kola i četrdeset  tisuća pješaka; pogubio je i vojvodu Šofaka. 
\par 19 Kad Hadadezerove  sluge vidješe da ih je razbio Izrael, sklopiše mir s Davidom  i počeše mu služiti. A Aramejci se više nisu usuđivali pomagati  Amoncima. 


\chapter{20}

\par 1 Slijedeće godine, u doba kad kraljevi izlaze u rat, izvede  Joab vojsku i poče pustošiti zemlju amonsku. Došavši, opsjeo  je Rabu; David bijaše ostao u Jeruzalemu. Joab je osvojio Rabu  i razorio je. 
\par 2 Tada je David uzeo njihovu kralju s glave krunu  i vidio da je teška jedan zlatni talenat, a na njoj je bilo drago  kamenje. Stavili su je na glavu Davidu, koji je iz grada odnio  vrlo velik plijen. 
\par 3 Narod koji bijaše u gradu izvede van i  stavi ga da radi pilama, gvozdenim pijucima i sjekirama. Tako  je David učinio svim gradovima Amonovih sinova. Potom se vratio  sa svim narodom u Jeruzalem. 
\par 4 Poslije toga opet izbi rat s Filistejcima u Gezeru; tada  je Hušanin Sibkaj pogubio Sipaja, koji je bio od Refaimovih potomaka;  i bili su pokoreni. 
\par 5 Uz to je nastao i rat s Filistejcima, u kojem je Jairov  sin Elhanan pogubio Lahmija, brata Golijata Gitejca, koji je  imao kopljaču kao tkalačko vratilo. 
\par 6 Potom opet izbi rat u  Gatu, gdje je bio neki čovjek visoka rasta: imaše taj na svakoj  ruci i nozi po šest prstiju, dakle dvadeset i četiri; i on bijaše  Rafin potomak. 
\par 7 Kad je počeo ružiti Izraela, ubi ga Jonatan, sin Davidova brata Šimeja. 
\par 8 To su bili Rafini potomci u Gatu  koji su izginuli od Davidove ruke i od ruke njegovih slugu. 


\chapter{21}

\par 1 Tada Satan ustade na Izraela i potače Davida da izbroji Izraelce. 
\par 2 Kralj reče Joabu i narodnim knezovima: "Idite, izbrojte Izraelce  od Beer Šebe pa do Dana, onda se vratite i kažite mi koliko ih  je na broju." 
\par 3 Joab reče: "Neka Jahve dade svome narodu još  sto puta ovoliko koliko ga je sada! Nisu li, moj gospodaru kralju, svi oni sluge mome gospodaru? Zašto traži to moj gospodar? Zašto  da bude na krivicu Izraelu?" 
\par 4 Ali kraljeva riječ bijaše jača od Joabove. Tako je Joab  otišao i počeo obilaziti sav Izrael, a onda se, najposlije, vrati  u Jeruzalem. 
\par 5 Joab dade Davidu popis naroda; Izraelaca bijaše  milijun i sto tisuća ljudi vičnih maču, a Judejaca četiri stotine  i sedamdeset tisuća vičnih maču. 
\par 6 Ali nije pobrojio među njima  ni Levijeva ni Benjaminova plemena, jer je Joabu bila odvratna  kraljeva zapovijed. 
\par 7 Bilo je to mrsko i u Božjim očima, pa Bog udari Izraela. 
\par 8 David reče Bogu: "Veoma sam sagriješio što sam to učinio.  Ali oprosti krivicu svome sluzi jer sam vrlo ludo radio!" 
\par 9 Jahve  reče Davidovu vidiocu Gadu: 
\par 10 "Idi i kaži Davidu: 'Ovako veli  Jahve: Troje stavljam preda te; izaberi sebi jedno od toga da  ti učinim!'" 
\par 11 Došavši k Davidu, Gad mu reče: "Ovako veli Jahve: 'Biraj  sebi 
\par 12 ili glad za tri godine, ili da tri mjeseca bježiš pred  neprijateljima i mač tvojih neprijatelja da te stiže, ili da  tri dana Jahvin mač i kuga bude na zemlji i Jahvin anđeo da ubija  po svim izraelskim krajevima.' Sada promisli i gledaj što da  odgovorim onome koji me poslao!" 
\par 13 David reče Gadu: "Na velikoj sam muci! Ah, neka padnem  u Jahvine ruke, jer je veliko njegovo milosrđe, a u ljudske ruke  da ne zapadnem!" 
\par 14 Tako je Jahve poslao kugu na Izraela te pomrije sedamdeset  tisuća Izraelaca. 
\par 15 Bog je poslao anđela na Jeruzalem da ga  istrebljuje; a kad je počeo istrebljivati, pogledao je Jahve  i sažalilo mu se zbog zla, pa je rekao anđelu zatorniku: "Dosta  je sada, spusti ruku!" Jahvin je anđeo stajao kraj gumna Jebusejca Ornana. 
\par 16 David, podigavši oči, vidje Jahvina anđela kako stoji između zemlje  i neba držeći u ruci isukan mač koji je podigao na Jeruzalem, i on pade ničice sa starješinama obučenim u kostrijet. 
\par 17 David  reče Bogu: "Nisam li ja zapovjedio da se izbroji narod? Ja sam, dakle, onaj koji sam sagriješio i grdno zlo načinio, a što učiniše  te ovce? Jahve, Bože moj, neka tvoja ruka dođe na me i na moju  obitelj, a ne na taj narod da ga pomori!" 
\par 18 Tada Jahvin anđeo reče Gadu da kaže Davidu neka uziđe  i neka podigne žrtvenik Jahvi na gumnu Jebusejca Ornana. 
\par 19 David  je otišao po riječi koju mu je Gad rekao u Jahvino ime. 
\par 20 A  Ornan, okrenuvši se, opazi anđela, a njegova se četiri sina sakriše.  Ornan je vrhao pšenicu. 
\par 21 Uto dođe David do Ornana, a on, pogledavši  i opazivši Davida, dođe s gumna i pokloni se Davidu licem do  zemlje. 
\par 22 Tada David reče Ornanu: "Daj mi to gumno da sagradim  na njemu žrtvenik Jahvi; za potpunu cijenu daj mi ga da bi prestao  pomor u narodu!" 
\par 23 Ornan odgovori Davidu: "Neka ga uzme i neka  čini moj gospodar kralj što je dobro u njegovim očima; evo, dajem  ti goveda za paljenice, i mlatilice za drva, i pšenicu za prinosnicu;  sve ti to poklanjam." 
\par 24 Kralj David reče Ornanu: "Ne, nego hoću da kupim u tebe  i da platim, jer neću da prinosim Jahvi što je tvoje, da prinosim  paljenice koje su mi poklonjene." 
\par 25 I David dade Ornanu za ono mjesto šest stotina zlatnih  šekela na mjeru. 
\par 26 Tada sagradi ondje žrtvenik Jahvi i prinese  paljenice i pričesnice; a kad je prizvao Jahvu, on ga usliša  spustivši oganj s neba na žrtvenik za paljenice. 
\par 27 Jahve zapovjedi  anđelu da vrati mač u korice. 
\par 28 U ono vrijeme, vidjevši da  ga je Jahve uslišio na gumnu Jebusejca Ornana, David poče prinositi  žrtve ondje. 
\par 29 Jahvino prebivalište, koje je napravio Mojsije  u pustinji, i žrtvenik za paljenice bio je u to vrijeme na uzvisini  u Gibeonu. 
\par 30 David nije mogao ići k njemu da traži Boga jer  ga je bio spopao strah od mača Jahvina anđela. 


\chapter{22}

\par 1 Zato David reče: "Ovo je Dom Jahve i ovo je žrtvenik za paljenice  Izraelu!" 
\par 2 David zapovjedi da se skupe stranci koji su bili u izraelskoj  zemlji i odredi klesare da propisno klešu kamenje za gradnju  Doma Božjeg. 
\par 3 David je pripravio mnogo željeza za čavle na  vratnim krilima i za kvačice; i bez mjere mnogo tuča. 
\par 4 Mnogo  cedrovine, jer su Sidonci i Tirci dovozili mnogo cedrovih drva  Davidu. 
\par 5 Jer David mišljaše: "Moj je sin Salomon mlad i nježan, a Dom koji treba graditi Jahvi mora biti veličanstven, na slavu  i čast po svim zemljama. Hajde da mu sve pripravim." I David  je pripravio mnogo toga prije svoje smrti. 
\par 6 Potom dozva sina  Salomona i zapovjedi mu da sagradi Dom Jahvi, Bogu Izraelovu. 
\par 7 Još David reče Salomonu: "Sine! Bio sam nakanio u srcu  da sagradim Dom imenu Jahve, svoga Boga. 
\par 8 Ali mi je došla Jahvina  riječ: 'Mnogo si krvi prolio i velike si ratove vodio; nećeš  ti graditi Doma mome imenu jer si mnogo krvi prolio na zemlju  preda mnom. 
\par 9 Gle, rodit će ti se sin; on će biti miroljubac  i dat ću mu mir od svih njegovih neprijatelja odasvud unaokolo;  ime će mu biti Salomon. Mir i pokoj dat ću Izraelu za njegova  vremena. 
\par 10 On će sagraditi Dom mome imenu, on će mi biti sin, a ja ću njemu biti otac i utvrdit ću njegovo kraljevsko prijestolje  nad Izraelom zauvijek.' 
\par 11 Sada, moj sine, neka bude Jahve s  tobom da izvršiš i sagradiš Dom Jahve, svoga Boga, kao što je  rekao za te. 
\par 12 Samo neka ti Jahve poda razum i mudrost kad  te postavi nad Izraelom zato da se držiš Zakona Jahve, svoga  Boga! 
\par 13 Bit ćeš sretan budeš li brižno vršio uredbe i zakone  koje je Jahve preko Mojsija dao Izraelu. Budi junak i hrabar, ne boj se i ne plaši se! 
\par 14 Ja sam, evo, svojim trudom pripravio  za Dom Jahvin sto tisuća zlatnih talenata i milijun srebrnih  talenata, a tuča i željeza bez mjere, jer ga je tako mnogo. Pripravio  sam i drva i kamenja, a i ti dodaj nešto k tomu. 
\par 15 Imaš mnogo  valjanih radnika, klesara, zidara, tesara i svakovrsnih vještaka  u svakom umijeću; 
\par 16 zlatu, srebru, tuču i željezu nema mjere;  idi, dakle, i gradi, i neka Jahve bude s tobom!" 
\par 17 Tada David zapovjedi svim izraelskim knezovima da pomažu  njegovu sinu Salomonu: 
\par 18 "Nije li s vama Jahve, Bog vaš, koji  vam je dao mir odasvud unaokolo jer je predao u moje ruke stanovnike  ove zemlje i zemlja je pokorena pred Jahvom i pred njegovim narodom. 
\par 19 Sada, dakle, pregnite svojim srcem i svojom dušom da tražite  Jahvu, svoga Boga; idite i gradite Svetište Bogu Jahvi, unesite  Kovčeg saveza Jahvina i Božje sveto posuđe u Dom koji će se sagraditi  Jahvinu imenu!" 


\chapter{23}

\par 1 Ostarjevši i nauživši se dana, postavi David svoga sina Salomona  kraljem nad Izraelom. 
\par 2 Potom skupi sve izraelske knezove, svećenike  i levite. 
\par 3 On izbroji levite od trideset godina naviše, i bilo ih  je po muškim glavama trideset i osam tisuća. 
\par 4 Između njih bilo  je dvadeset i četiri tisuće onih koji su upravljali poslom oko  Jahvina Doma, a šest tisuća nadzornika i sudaca, 
\par 5 četiri tisuće  vratara i četiri tisuće onih koji su hvalili Jahvu uz glazbala  što ih je napravio za hvalu. 
\par 6 David ih razdijeli na redove  po Levijevim sinovima: Geršonu, Kehatu i Merariju. 
\par 7 Od Geršonova su koljena bili: Ladan i Šimej. 
\par 8 Ladanovi  sinovi: poglavari Jehiel, Zetam i Joel, njih trojica. 
\par 9 Šimejevi  sinovi: Šelomit, Haziel i Haram, njih trojica; to su poglavari  Ladanovih obitelji. 
\par 10 Šimejevi sinovi: Jahat, Zina, Jeuš i  Berija. Ta su četvorica Šimejevi sinovi. 
\par 11 Jahat je bio poglavar, a drugi Ziza; a Jeuš i Berija nisu imali mnogo djece, zato su  se brojili u jednu obitelj, u jedan razred. 
\par 12 Kehatovi sinovi: Amram, Jishar, Hebron i Uziel, četvorica. 
\par 13 Amramovi sinovi: Aron i Mojsije. Aron je bio određen da posvećuje  Svetinju nad svetinjama; on i njegovi sinovi dovijeka da kade  pred Jahvom, da mu služe i da blagoslivljaju u njegovo ime dovijeka. 
\par 14 Mojsije je bio Božji čovjek. Njegovi se sinovi broje u Levijevo  pleme. 
\par 15 Mojsijevi su sinovi Geršom i Eliezer. 
\par 16 Geršomovi  sinovi: poglavar Šebuel. 
\par 17 Eliezerovi su sinovi bili: poglavar  Rehabja. Eliezer nije imao drugih sinova, nego su se Rehabjini  sinovi vrlo namnožili. 
\par 18 Jisharovi sinovi: poglavar Šelomit. 
\par 19 Hebronovi sinovi: poglavar Jerija, drugi Amarja, treći Jahaziel, četvrti Jekamam. 
\par 20 Uzielovi sinovi: poglavar Mika, drugi Ješija. 
\par 21 Merarijevi sinovi: Mahli i Muši. Mahlijevi sinovi: Eleazar  i Kiš. 
\par 22 Eleazar je umro nemajući sinova, nego samo kćeri,  koje su sebi uzeli za žene njihovi rođaci, Kiševi sinovi. 
\par 23 Mušijevi  sinovi: Mahli, Eder i Jerimot, trojica. 
\par 24 To su bili Levijevi sinovi po obiteljima, poglavari porodica, koji su bili popisani poimence; oni su radili posao za službu  Jahvina Doma u dobi od dvadeset godina naviše. 
\par 25 David je rekao: "Jahve, Izraelov Bog, dao je mir svojem  narodu i živjet će u Jeruzalemu zauvijek. 
\par 26 Zato ni leviti  neće više nositi Prebivališta ni svakovrsnog pribora za njegovu  službu." 
\par 27 Po posljednjim Davidovim riječima, bili su izbrojeni  Levijevi sinovi od dvadeset godina naviše. 
\par 28 Bili su određeni  da budu kraj Aronovih sinova u službi u Jahvinu Domu, u predvorjima  i u dvoranama, da čiste sve svete stvari, da rade u službi oko  Jahvina Doma, 
\par 29 oko prinesenih hljebova, oko sitnog brašna  za prinos, oko beskvasnih kolača pripravljenih na tavi i u ulju  zamiješenih i oko mjera za sadržaj i dužinu; 
\par 30 da pristupaju  svakoga jutra, da slave i hvale Jahvu; tako i večerom. 
\par 31 A  kad se god prinose paljenice Jahvi, subotom, za mlađaka i na  blagdane, da dolaze prema svom broju, po svom redu, svagdje pred  Jahvu. 
\par 32 I da vrše što treba vršiti u Šatoru sastanka, službu  u Svetištu i službu za svoju braću, Aronove sinove, u službi  oko Jahvina Doma. 


\chapter{24}

\par 1 Aronovi su sinovi imali svoje redove. Sinovi Aronovi bili  su: Nadab, Abihu, Eleazar i Itamar. 
\par 2 Ali su Nadab i Abihu umrli  prije oca i nisu imali djece; zato su svećeničku službu vršili  Eleazar i Itamar. 
\par 3 David je razdijelio na redove njih i Sadoka, od Eleazarovih sinova, i Ahimeleka, od Itamarovih sinova, po  njihovu redu u njihovoj službi. 
\par 4 Ali se u Eleazarovih sinova  našlo više muških poglavara nego u Itamarovih sinova, pa kad  ih podijeliše, od Eleazarovih je sinova bilo šesnaest porodičnih  poglavara, a od Itamarovih sinova samo osam porodičnih poglavara. 
\par 5 Zato su ih razdijelili ždrebovima, jedne i druge, jer su posvećeni  knezovi i Božji knezovi bili i od Eleazarovih sinova i od Itamarovih  sinova. 
\par 6 Popisao ih je Netanelov sin Šemaja, pisar od Levijeva  plemena, pred kraljem, knezovima, svećenikom Sadokom, Ebjatarovim  sinom Ahimelekom, pred poglavarima porodica među svećenicima  i levitima, uzevši po jednu porodicu za Eleazara, a po jednu  opet za Itamara. 
\par 7 Prvi je ždrijeb pao na Jojariba, drugi na Jedaju, 
\par 8 treći  na Harima, četvrti na Seorima, 
\par 9 peti na Malkiju, šesti na Mijamina, 
\par 10 sedmi na Hakosa, osmi na Abiju, 
\par 11 deveti na Ješuu, deseti  na Šekaniju, 
\par 12 jedanaesti na Elijašiba, dvanaesti na Jakima, 
\par 13 trinaesti na Hupu, četrnaesti na Ješebaba, 
\par 14 petnaesti  na Bilgu, šesnaesti na Imera, 
\par 15 sedamnaesti na Hezira, osamnaesti  na Hapisesa, 
\par 16 devetnaesti na Petahju, dvadeseti na Ezekiela, 
\par 17 dvadeset i prvi na Jakina, dvadeset i drugi na Gamula, 
\par 18 dvadeset  i treći na Delaju, dvadeset i četvrti na Maazju. 
\par 19 To je njihov red u službi kojim treba da idu u Jahvin  Dom, po svom pravilu, primljenu od oca im Arona, kako mu je zapovjedio  Jahve, Bog Izraelov. 
\par 20 Od ostalih Levijevih sinova bio je od Amramovih sinova  Šubael; od Šubaelovih sinova Jehdeja; 
\par 21 od Rehabje, od Rehabjinih  sinova poglavar Jišija; 
\par 22 od Jisharovaca Šelomot; od Šelomotovih  sinova Jahat. 
\par 23 Od Jerijinih sinova: drugi Amarja, treći Jahaziel, četvrti Jekaman. 
\par 24 Od sinova Uzielovih Mika; od Mikinih sinova  Šamir; 
\par 25 Mikin brat Jišija; od Jišijinih sinova Zaharija; 
\par 26 Merarijevi  sinovi: Mahli i Muši; sinovi Jaazije, njegova sina. 
\par 27 Merarijevi  sinovi po Jaaziji, njegovu sinu: Šoham, Zakur i Ibri; 
\par 28 po  Mahliju Eleazar, koji nije imao djece; 
\par 29 po Kišu, Kišovi sinovi, Jerahmeel. 
\par 30 Mušijevi sinovi: Mahli, Eder i Jerimot. To su bili levitski sinovi po svojim porodicama. 
\par 31 I oni  su bacali ždrebove kao njihovi rođaci, Aronovi sinovi, pred kraljem  Davidom, Sadokom, Ahimelekom i porodičnim poglavarima među svećenicima  i levitima, i to jednako glavar obitelji kao i njegov najmlađi  brat. 


\chapter{25}

\par 1 David je s vojničkim zapovjednicima izabrao za službu Asafove, Hemanove i Jedutunove sinove koji će zanosno pjevati hvalu uz  citre, harfe i cimbale; između njih su bili izbrojeni ljudi za  posao u svojoj službi: 
\par 2 od Asafovih sinova: Zakur, Josip, Netanija  i Asarela; Asafovi sinovi pod upravom Asafa, koji je zanosno  pjevao hvalu po kraljevoj uredbi. 
\par 3 Od Jedutuna: Jedutunovih šest sinova: Gedalija, Sori,  Ješaja, Šimej, Hašabja i Matitja pod upravom svog oca Jedutuna  koji je zanosno pjevao hvalu uz citru slaveći i hvaleći Jahvu. 
\par 4 Od Hemana: Hemanovi sinovi: Bukija, Matanija, Uziel, Šebuel, Jerimot, Hananija, Hanani, Eliata, Gidalti, Romamti-Ezer, Jošbekaša, Maloti, Hotir, Mahaziot. 
\par 5 Svi su oni bili sinovi kraljeva  vidioca Hemana koji je objavljivao Božje stvari da uzvisi njegovu  moć; a Bog je dao Hemanu četrnaest sinova i tri kćeri. 
\par 6 Svi  su oni pod vodstvom svoga oca Asafa te Jedutuna i Hemana pjevali  u Jahvinu Domu uz cimbale, harfe i citre za službu u Božjem Domu, po kraljevoj uredbi. 
\par 7 Bilo ih je, s njihovom braćom, uvježbanih  u pjevanju Jahvinih pjesama, dvjesta osamdeset i osam, sve samih  vještaka. 
\par 8 Bacili su ždrebove za svoju službenu dužnost, najmanji  isto kao i najveći, učitelj kao i učenik. 
\par 9 Prvi je ždrijeb  pao na Asafovca Josipa, drugi na Gedaliju s njegovom braćom i  sinovima, njih dvanaest, 
\par 10 treći na Zakura s njegovim sinovima  i braćom, njih dvanaest; 
\par 11 četvrti na Jisrija s njegovim sinovima  i braćom, njih dvanaest; 
\par 12 peti na Netaniju s njegovim sinovima  i braćom, njih dvanaest; 
\par 13 šesti na Bukiju s njegovim sinovima  i braćom, njih dvanaest, 
\par 14 sedmi na Isarelu s njegovim sinovima  i braćom, njih dvanaest; 
\par 15 osmi na Ješaja s njegovim sinovima  i braćom, njih dvanaest; 
\par 16 deveti na Mataniju s njegovim sinovima  i braćom, njih dvanaest; 
\par 17 deseti na Šimeja s njegovim sinovima  i braćom, njih dvanaest; 
\par 18 jedanaesti na Azarela s njegovim  sinovima i braćom, njih dvanaest; 
\par 19 dvanaesti na Hašabju s  njegovim sinovima i braćom, njih dvanaest; 
\par 20 trinaesti na Šubaela  s njegovim sinovima i braćom, njih dvanaest; 
\par 21 četrnaesti na  Matitju s njegovim sinovima i braćom, njih dvanaest; 
\par 22 petnaesti  na Jeremota s njegovim sinovima i braćom, njih dvanaest; 
\par 23 šesnaesti  na Hananiju s njegovim sinovima i braćom, njih dvanaest; 
\par 24 sedamnaesti  na Jošbekaša s njegovim sinovima i braćom, njih dvanaest; 
\par 25 osamnaesti  na Hananija s njegovim sinovima i braćom, njih dvanaest; 
\par 26 devetnaesti  na Malotija s njegovim sinovima i braćom, njih dvanaest; 
\par 27 dvadeseti  na Elijatu s njegovim sinovima i braćom, njih dvanaest; 
\par 28 dvadeset  i prvi na Hotira s njegovim sinovima i braćom, njih dvanaest; 
\par 29 dvadeset i drugi na Gidaltija s njegovim sinovima i braćom, njih dvanaest; 
\par 30 dvadeset i treći na Mahaziota s njegovim  sinovima i braćom, njih dvanaest; 
\par 31 dvadeset i četvrti na Romamti-Ezera  s njegovim sinovima i braćom, njih dvanaest. 


\chapter{26}

\par 1 Vratarski su redovi bili: od Korahovaca: Korahov sin Mešelemja  između Asafovih sinova; 
\par 2 a Mešelemjini sinovi: prvenac Zaharija, drugi Jediael, treći Zebadja, četvrti Jatniel; 
\par 3 peti Elam, šesti Johanan, sedmi Elijoenaj. 
\par 4 Sinovi Obed-Edomovi: prvenac Šemaja, drugi Jozabad, treći  Joah, četvrti Sakar, a peti Netanel, 
\par 5 šesti Amiel, sedmi Jisakar, osmi Peuletaj, jer ga je blagoslovio Bog, 
\par 6 a njegovu su se  sinu Šemaji rodili sinovi koji su bili poglavari u porodici jer  bijahu hrabri junaci. 
\par 7 Šemajini su sinovi bili: Otni, Rafael, Obed, Elzabad sa svojom braćom, vrsnim ljudima, Elihu i Semakja. 
\par 8 Svi su oni bili od Obed-Edomovih sinova, oni i njihovi sinovi  i njihova braća, vrsni ljudi, sposobni za službu; bilo ih je  šezdeset i dva od Obed-Edoma. 
\par 9 Mešelemjinih sinova i braće, vrsnih ljudi, bilo je osamnaest. 
\par 10 Hosini sinovi od Merarijevih sinova: poglavar Šimri, iako  nije bio prvenac, njegov ga je otac postavio za poglavara; 
\par 11 drugi  Hilkija, treći Tebalija, četvrti Zaharija; Hosinih svih sinova  i braće bilo je trinaest. 
\par 12 Ovo su vratarski redovi. Glavari ovih junaka bili su, kao i njihova braća, čuvari u službi Jahvina Doma. 
\par 13 Bacali  su ždrebove, najmanji kao i najveći, po obiteljima za svaka pojedina  vrata. 
\par 14 Ždrijeb na istok pao je Šelemji; njegov sin Zaharija  bio je mudar savjetnik. Kad su bacili ždrebove, dopao mu je ždrijeb  na sjever, 
\par 15 Obed-Edomu na jug; a njegovim sinovima na spremište; 
\par 16 Šufimu i Hosi na zapad Šaleketskim vratima, na putu koji  vodi k usponu; straža je bila do straže. 
\par 17 S istoka šest levita, sa sjevera četiri na dan, s juga četiri na dan; a kod spremišta  po dva. 
\par 18 Na hramskoj prigradnji, sa zapada, četiri na usponu, dva kod prigradnje. 
\par 19 To su vratarski redovi među Korahovim  i Merarijevim sinovima. 
\par 20 Leviti, njihova braća, bili su: Ahija nad blagom Božjega  Doma i nad blagom posvećenih stvari. 
\par 21 Ladanovi sinovi, Geršonovci po Ladanu, poglavari obitelji  Ladana Geršonovca, bili su Jehielovci. 
\par 22 Jehielovci Zetam i  brat mu Joel bili su nadstojnici nad blagom Jahvina Doma. 
\par 23 Od Amramovaca, Jisharovaca, Hebronovaca i Uzielovaca  bili su: 
\par 24 Šebuel, sin Mojsijeva sina Geršoma, nadstojnik nad blagom. 
\par 25 Njegova braća po Eliezeru: Rehabja, sin mu, njegov sin Izaija, njegov sin Joram, njegov sin Zikri, njegov sin Šelomit. 
\par 26 Taj  je Šelomit sa svojom braćom bio odgovoran za sve blago od posvećenih  stvari koje je posvetio kralj David s porodičnim poglavarima, s tisućnicima, stotnicima i vojnim zapovjednicima. 
\par 27 Posvetili  su dio ratnog plijena da se bolje ojača Jahvin Dom. 
\par 28 Što je  god bio posvetio vidjelac Samuel, Kišev sin Šaul, Nerov sin Abner  i Sarvijin sin Joab, sve posvećeno, bilo je pod nadzorom Šelomita  i njegove braće. 
\par 29 Jisharovci Kenanija i njegovi sinovi bili su nad svjetovnim  poslovima kao nadzornici i suci u Izraelu. 
\par 30 Hebronovci Hašabja i njegova braća, tisuću i sedam stotina  vrsnih ljudi, upravljali su Izraelom s ovu stranu Jordana na  zapadu u svakom Jahvinu poslu i u kraljevskoj službi. 
\par 31 Poglavar  Hebronovaca bio je Jerija. Četrdesete godine Davidova kraljevanja  potražili su obiteljska rodoslovlja Hebronovaca i našlo se među  njima vrsnih ljudi u Gileadskom Jazeru. 
\par 32 Njegove braće, vrsnih  ljudi, bilo je dvije tisuće i sedam stotina porodičnih poglavara;  kralj David postavio ih je nad Rubenovim i Gadovim plemenom i  nad polovinom Manašeova plemena za sve Božje poslove i za kraljevske  poslove. 


\chapter{27}

\par 1 Izraelovi sinovi po svome broju. Poglavari porodica, tisućnici, stotnici i nadzornici služili su kralju u svakom poslu. U redovima  su dolazili i odlazili od mjeseca do mjeseca, u svim godišnjim  mjesecima; svaki je red imao dvadeset i četiri tisuće ljudi. 
\par 2 Nad prvim je redom, prvoga mjeseca, bio Zabdielov sin  Jašobam. U svom je redu imao dvadeset i četiri tisuće ljudi. 
\par 3 Pripadao je Peresovim sinovima i bio zapovjednik svih vojvoda  u vojsci prvoga mjeseca. 
\par 4 Nad redom drugoga mjeseca bio je Ahošanin Dodaj, a predstojnik  u njegovu redu bio je Mikelot. U svom je redu imao dvadeset i  četiri tisuće ljudi. 
\par 5 Vojvoda treće vojske, trećega mjeseca, bio je sin svećenika  Jojade, poglavar Benaja. U svom je redu imao dvadeset i četiri  tisuće. 
\par 6 Taj je Benaja bio junak među tridesetoricom i nad  tridesetoricom i u njegovu je redu bio sin mu Amizabad. 
\par 7 Četvrti, četvrtoga mjeseca, bio je Joabov brat Asahel, a za njim sin mu Zebadja. U svom je redu imao dvadeset i četiri  tisuće. 
\par 8 Peti, petoga mjeseca, bio je vojvoda Jizrahanin Šamhut.  U svom je redu imao dvadeset i četiri tisuće. 
\par 9 Šesti, šestoga  mjeseca, bio je Ikešov sin Ira, Tekoanac. U svom je redu imao  dvadeset i četiri tisuće. 
\par 10 Sedmi, sedmoga mjeseca, bio je  Pelonjanin Heles od Efrajimovih sinova. U svom je redu imao dvadeset  i četiri tisuće. 
\par 11 Osmi, osmoga mjeseca, bio je Hušaćanin Sibkaj, Zarhijevac. U svom je redu imao dvadeset i četiri tisuće. 
\par 12 Deveti, devetoga mjeseca, bio je Anatoćanin Abiezer, Benjaminovac. U  svom je redu imao dvadeset i četiri tisuće. 
\par 13 Deseti, desetoga  mjeseca, bio je Netofaćanin Mahraj, Zarhijevac. U svom je redu  imao dvadeset i četiri tisuće. 
\par 14 Jedanaesti, jedanaestoga mjeseca, bio je Piratonjanin Benaja, Efrajimovac. U svom je redu imao  dvadeset i četiri tisuće. 
\par 15 Dvanaesti, dvanaestoga mjeseca, bio je Netofaćanin Heldaj, Otnielovac. U svom je redu imao dvadeset  i četiri tisuće. 
\par 16 Nad Izraelovim plemenima bili su knezovi: nad Rubenovim  Zikrijev sin knez Eliezer; nad Šimunovim Maakin sin Šefatja; 
\par 17 nad Levijevim Kemuelov sin Hašabja; nad Aronovim Sadok; 
\par 18 nad  Judinim Elihu od Davidove braće; nad Jisakarovim Mihaelov sin  Omri; 
\par 19 nad Zebulunovim Obadjin sin Jišmaja; nad Naftalijevim  Azrielov sin Jerimot. 
\par 20 Nad Efrajimovim sinovima Azazjin sin  Hošea; nad polovinom Manašeova plemena Pedajin sin Joel, 
\par 21 nad  drugom polovinom Manašeova plemena u Gileadu Zaharijin sin Jido;  nad Benjaminom Abnerov sin Jaasiel; 
\par 22 nad Danom Jerohamov sin  Azarel. To su bili knezovi izraelskih plemena. 
\par 23 Ali David nije dao izbrojiti onih kojima bijaše dvadeset  godina i manje, jer Jahve bijaše rekao da će umnožiti Izraelce  kao nebeske zvijezde. 
\par 24 Sarvijin je sin Joab počeo vršiti popis, ali ga nije dovršio. Stoga je Srdžba došla na Izrael i zato  taj broj nije bio primljen u brojčani izvještaj Ljetopisa kralja  Davida. 
\par 25 Nadstojnik nad kraljevim blagom bio je Adielov sin Azmavet, a nadstojnik za blago u zemlji, u gradovima, selima i tvrđavama, bio je Uzijin sin Jonatan. 
\par 26 Nadstojnik nad poljskim radnicima  koji su obrađivali zemlju bio je Kelubov sin Ezri. 
\par 27 Nadstojnik  nad vinogradarima Ramaćanin Šimej. Nadstojnik nad vinogradrskim  klijetima bio je Šifmejac Zabdi. 
\par 28 Nadstojnik nad maslinama  i dudovima što su po Šefeli bio je Gederac Hanan; nadstojnik  nad skladištima ulja Joaš. 
\par 29 Nadstojnik nad govedima što su  pasla u Šaronu bio je Šaronac Šitraj. Nadstojnik nad krupnom  stokom u dolinama bio je Edlajev sin Šafat. 
\par 30 Nadstojnik nad  devama Jišmaelac Obil. Nadstojnik nad magaricama Meronoćanin  Jehdeja. 
\par 31 Nadstojnik nad ovcama i kozama Hagrijac Jaziz. Svi  su oni bili nadstojnici nad imanjem kralja Davida. 
\par 32 Savjetnik je bio Davidov stric Jonatan, mudar čovjek;  bio je i književnik; a Hakmonijev sin Jehiel bio je s kraljevim  sinovima. 
\par 33 Ahitofel je bio kraljev savjetnik, Arkijac Hušaj  kraljev prijatelj. 
\par 34 Ahitofela su naslijedili Benajin sin Jojada  i Ebjatar, kraljev je vojvoda bio Joab. 


\chapter{28}

\par 1 David sakupi u Jeruzalem sve izraelske knezove, plemenske  knezove i poglavare od redova koji su služili kralja, tisućnike, stotnike i nadstojnike nad svim imanjem i blagom kraljevim i  blagom njegovih sinova, zajedno s dvoranima i junacima i svim  hrabrim vojnicima. 
\par 2 Ustavši na noge, kralj David reče: "Čujte me, moja braćo i moj narode! Ja sam bio namislio u  svom srcu da sagradim dom gdje bi počivao Kovčeg saveza Jahvina  i da bude podnožje nogama našega Boga te sam pripravio što treba  za gradnju. 
\par 3 Ali mi je Bog rekao: 'Nećeš ti sagraditi Doma  mome imenu jer si ratnik i prolijevao si krv.' 
\par 4 Jahve, Izraelov Bog, izabrao je mene od sveg moga roda  da budem kralj nad Izraelom zauvijek; jer je Judu izabrao za  kneza, a iz Judina doma dom moga oca; između sinova moga oca  bilo mu je drago da mene postavi kraljem nad svim Izraelom. 
\par 5 Tako  je između mojih sinova, jer mi je mnogo sinova dao Jahve, izabrao  moga sina Salomona da sjedi na prijestolju Jahvina kraljevstva  nad Izraelom. 
\par 6 I rekao mi je: 'Tvoj sin Salomon sagradit će  meni Dom i moja predvorja; jer sam njega izabrao sebi za sina  i ja ću mu biti otac. 
\par 7 Utvrdit ću njegovo kraljevstvo zauvijek  ako bude postojano vršio moje zapovijedi i moje zakone kao danas.' 
\par 8 Sada, dakle, pred očima sveg Izraela, Jahvina zbora, i  pred svojim Bogom, koji nas sluša, velim: držite i tražite sve  zapovijedi Jahve, svoga Boga, da biste zadržali u posjedu ovu  dobru zemlju i ostavili je u baštinu svojim sinovima nakon sebe  dovijeka. 
\par 9 A ti, sine moj Salomone, poznaj Boga, svoga oca, i služi  mu čitavim srcem i spremnom dušom, jer Jahve ispituje sva srca  i zna sve misli i namjere; ako ga budeš tražio, dat će ti se  da ga nađeš; ako li ga ostaviš, odbacit će te zauvijek. 
\par 10 Uvidi  sada da te Jahve izabrao da gradiš Dom za Svetište, budi junak  i radi!" 
\par 11 Tada David predade sinu Salomonu uzorak trijema, njegovih  kuća, riznica, gornjih soba, ćelija i doma Pomirilišta; 
\par 12 uzorak  svega što bijaše smislio u duhu za predvorja Jahvina Doma, za  sve sobe unaokolo, za riznicu Doma Božjega, za riznicu posvećenih  stvari, 
\par 13 za svećeničke i levitske redove, za svaki posao u  službi oko Doma Jahvina: 
\par 14 zlato u šipkama, zlato potrebno  za sve zlatno posuđe ove ili one službe; srebro u šipkama potrebno  za sve srebrno posuđe, za sve posuđe ove ili one službe; 
\par 15 šipke  za zlatne svijećnjake sa zlatnim svjetiljkama, prema težini svakoga  svijećnjaka i njegovih svjetiljaka, i za srebrne svijećnjake  prema težini svakoga svijećnjaka i njegovih svjetiljaka i prema  namjeni svakog svijećnjaka; 
\par 16 zlato u šipkama za stolove na  kojima će stajati prineseni hljebovi, za svaki stol; srebro za  srebrne stolove, 
\par 17 za viljuške i kotliće, za čaše od čista  zlata, za zlatne pehare, zlato u šipkama za svaki pehar; za srebrne  pehare, srebro u šipkama za svaki pehar, 
\par 18 za kadioni žrtvenik  žeženoga zlata u šipkama; za uzorak od kola sa zlatnim kerubinima  koji će raširenim krilima zaklanjati Jahvin Kovčeg. 
\par 19 Sve to  u skladu s onim što Jahve napisa vlastitom rukom da bi razjasnio  cijelo djelo za koje on pribavi uzorak. 
\par 20 Tada David reče svome sinu Salomonu: "Budi junak, hrabar, i radi! Ne boj se i ne plaši se, jer će Jahve, Bog, moj Bog, biti s tobom! Neće te napustiti niti te ostaviti dok ne svršiš  sav posao za službu oko Jahvina Doma. 
\par 21 Evo svećeničkih i levitskih  redova za svaku službu u Božjem Domu; imaš uza se za svaki posao  svakovrsnih ljudi, spremnih i vještih svakoj službi, knezovi  i sav narod pod tvojim su zapovjedništvom." 



\chapter{29}

\par 1 Kralj David reče svemu zboru: "Bog je izabrao moga sina Salomona, mlado i nježno momče, a ovo je velik posao, jer neće biti za  čovjeka dvor nego za Boga Jahvu. 
\par 2 Pripremio sam, koliko sam  mogao, za Dom svoga Boga zlata za zlatne stvari i srebra za srebrne, tuča za tučane, željeza za željezne, drva za drvene; oniksova  kamenja i dragulja za ukivanje, dragulja za ukras i šarenih dragulja, svakojakoga dragog kamenja i izobila mramora. 
\par 3 Iz ljubavi  prema Bogu dajem još i svoga zlata i srebra za Dom svoga Boga, osim svega što sam pripravio za sveti Dom. 
\par 4 Tri tisuće zlatnih  talenata ofirskoga zlata i sedam tisuća talenata čistoga srebra  da se oblože zidovi prostorija. 
\par 5 Zlato za zlatne stvari, a  srebro za srebrne i za svako djelo umjetničkih ruku. Bi li danas  još tko htio dragovoljno što priložiti svojom rukom Jahvi?" 
\par 6 Tada su dragovoljno priložili knezovi obitelji i knezovi  izraelskih plemena, tisućnici, stotnici i nadstojnici nad kraljevskim  poslovima. 
\par 7 Dali su za službu u Božjem Domu zlata pet tisuća  talenata i deset tisuća zlatnih darika, srebra deset tisuća talenata, tuča osamnaest tisuća talenata, željeza sto tisuća talenata. 
\par 8 U koga se god našlo dragulja, svi su darivali u riznicu Jahvina  Doma na ruke Jehiela Geršonovca. 
\par 9 Narod se veselio što su dragovoljno  prilagali, jer su prilagali iskrena srca Jahvi; i kralj David  radovao se od srca. 
\par 10 Potom David blagoslovi Jahvu pred svim zborom. I reče  David: "Blagoslovljen da si, Jahve, Bože našeg oca Izraela, od  vijeka do vijeka! 
\par 11 Tvoja je, Jahve, veličina, sila, slava, sjaj i veličanstvo, jer je tvoje sve što je na nebu i na zemlji;  tvoje je, Jahve, kraljevstvo i ti si uzvišen povrh svega, Poglavar  svega! 
\par 12 Od tebe je bogatstvo i slava, ti vladaš nad svim,  u tvojoj je ruci sila i moć, u tvojoj je vlasti da učiniš velikim  i jakim sve. 
\par 13 I slavimo te, Bože naš, i hvalimo tvoje dično  ime. 
\par 14 Tko sam ja i što je moj narod da bismo imali snage ovoliko  prinijeti tebi dragovoljno? Od tebe je sve, i iz tvojih ruku  primivši, dali smo tebi! 
\par 15 Pridošlice smo pred tobom, naseljenici  kao svi naši očevi; naši dani na zemlji prolaze kao sjena i nema  nade. 
\par 16 Jahve, Bože naš, sve ovo mnogo blago koje smo pripravili  za gradnju Doma tebi, tvome svetom imenu, iz tvoje je ruke i  sve je tvoje! 
\par 17 Ali znam, o Bože moj, da ti iskušavaš srca  i da ljubiš iskrenost; ja sam iskrena srca dragovoljno prinio  sve ovo i s radošću sam gledao tvoj narod koji je ovdje kako  ti dragovoljno prinosi. 
\par 18 Jahve, Bože naših otaca Abrahama, Izaka i Jakova, sačuvaj dovijeka u srcu svoga naroda tu misao  i namjeru i upravi njegovo srce k sebi! 
\par 19 A mome sinu Salomonu  daj pošteno srce da bi se držao tvojih zapovijedi, tvojih odredaba  i tvojih uredaba, da bi vršio sve i da bi sagradio dvor za koji  sam sve spremio!" 
\par 20 Tada David reče svemu zboru: "Blagoslovite sada Jahvu, svoga Boga!" I sav je zbor blagoslovio Jahvu, Boga svojih otaca, i, pavši ničice, poklonio se Jahvi i kralju. 
\par 21 Žrtvovali su  Jahvi klanice i prinijeli Jahvi paljenice sutradan: tisuću junaca, tisuću ovnova, tisuću jaganjaca s njihovim ljevanicama, mnogo  drugih žrtava za sav Izrael. 
\par 22 Jeli su i pili pred Jahvom onoga  dana vrlo se radujući. Zakraljili su po drugi put Davidova sina  Salomona i pomazali ga po Jahvinoj volji za kneza, a Sadoka za  svećenika. 
\par 23 Tako je Salomon sjeo na Jahvino prijestolje da  kraljuje namjesto svoga oca Davida. Bio je sretan i slušao ga  je sav Izrael. 
\par 24 Svi su knezovi i junaci i svi sinovi kralja  Davida pružili ruku kralju Salomonu i svečano mu obećali pokornost. 
\par 25 Jahve je vrlo uzvisio Salomona pred očima sveg Izraela i  dao njegovu kraljevstvu veličanstvo kakvo nijedan kralj prije  njega nije imao u Izraelu. 
\par 26 Tako je Jišajev sin David kraljevao nad svim Izraelom. 
\par 27 Nad Izraelom je kraljevao četrdeset godina; u Hebronu je  kraljevao sedam godina, u Jeruzalemu je kraljevao trideset i  tri godine. 
\par 28 Umro je u lijepoj starosti, nauživši se života, bogatstva i slave. Na njegovo se mjesto zakraljio sin mu Salomon. 
\par 29 Djela kralja Davida, od prvog do posljednjeg, zapisana su  u povijesti vidioca Samuela, u povijesti proroka Natana i u povijesti  vidioca Gada, 
\par 30 sa svim njegovim kraljevanjem, njegovim junaštvom  i događajima što prijeđoše preko njega i Izraela i svih drugih  kraljevstava zemaljskih. 





\end{document}