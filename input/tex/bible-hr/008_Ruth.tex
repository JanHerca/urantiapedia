\begin{document}

\title{Ruta}


\chapter{1}

\par 1 U ono vrijeme kada su vladali suci nastala glad u zemlji, pa  iz Betlehema Judina jedan čovjek ode sa svojom ženom i sa svoja  dva sina da se naseli na Moapskim poljanama. 
\par 2 Taj se čovjek  zvao Elimelek, žena mu Noemi, a dva njegova sina: Mahlon i Kiljon;  svi bijahu Efraćani iz Betlehema Judina. Stigoše na Moapske poljane  i tu se nastaniše. 
\par 3 Tada Elimelek, Noemin muž, umrije, i ona  osta sama sa svoja dva sina. 
\par 4 Oni se oženiše Moapkama; jedna  se zvala Orpa, a druga Ruta. I tu proboraviše deset godina. 
\par 5 Onda  umriješe i Mahlon i Kiljon, i tako Noemi osta i bez svoja dva  sina i bez svoga muža. 
\par 6 Tada se ona diže sa svojim snahama da ode s Moapskih poljana  jer je čula na Moapskim poljanama da je Jahve pohodio narod svoj  i dao mu kruha. 
\par 7 Ode, dakle, ona iz mjesta gdje je živjela, a s njome i njezine snahe; krenuše na put da se vrate u zemlju  Judinu. 
\par 8 Noemi tada reče svojim dvjema snahama: "Vratite se  svaka domu majke svoje! Neka vam Jahve bude milostiv kao što  vi bijaste pokojnicima i meni. 
\par 9 Neka vam Jahve udijeli da obje  nađete mir, svaka u domu svoga muža!" I poljubi ih, a one briznuše u plač. 
\par 10 I rekoše joj: "Ne!  Mi ćemo s tobom, tvome narodu." 
\par 11 Ali im reče Noemi: "Vratite se natrag, kćeri moje! Zašto  biste išle sa mnom? Zar ću još imati sinova u utrobi svojoj da  vam budu muževi? 
\par 12 Vratite se natrag, kćeri moje, idite samo!  Odviše sam stara, nisam za udaju. Pa i kad bih rekla: 'Imam nade  da se udam još noćas i da rodim sinove' - 
\par 13 zar biste mogle  čekati da odrastu i zar biste radi njih ostale neudate? Ne, kćeri  moje, tuga bi moja bila veća od vaše, jer se ruka Jahvina digla  na me." 
\par 14 One i opet zaplakaše i zajecaše. Orpa poljubi svoju svekrvu  i vrati se, a Ruta ostade s njom. 
\par 15 Noemi joj reče: "Eto vidiš, jetrva se tvoja vratila narodu  svome i bogu svome: vrati se i ti za jetrvom svojom!" 
\par 16 A Ruta joj odgovori: "Nemoj me tjerati da te ostavim  i da odem od tebe: jer kamo ti ideš, idem i ja i gdje se ti nastaniš, nastanit ću se i ja; tvoj narod moj je narod i tvoj Bog moj  je Bog. 
\par 17 Gdje ti umreš, umrijet ću i ja, gdje tebe pokopaju, pokopat će i mene. Neka mi Jahve uzvrati svakim zlom i nevoljom  ako me što drugo, osim smrti, rastavi od tebe." 
\par 18 Videći gdje je tvrdo naumila da ide s njom, prestade  je odvraćati. 
\par 19 Tako su zajedno išle dok ne dođoše u Betlehem. A kad  dođoše u Betlehem, sav se grad uzbudi zbog njih. "Ma je li ovo  Noemi?" - pitahu žene. 
\par 20 A ona im odgovaraše: "Ne zovite me  više Noemi nego me zovite Mara; jer me Šadaj gorčinom ispunio! 
\par 21 Odavde sam otišla punih ruku, a sad me Jahve vraća bez igdje  ičega. Zašto me zovete Noemi kad Jahve posvjedoči protiv mene  i Svemogući me u tugu zavi?" 
\par 22 Tako se vrati Noemi s Rutom Moapkom, snahom svojom, s  Moapskih poljana. Stigle su u Betlehem baš kad je počela žetva  ječma. 


\chapter{2}

\par 1 Noemi imaše rođaka po mužu, čovjeka vrlo imućna, iz porodice  Elimelekove: zvao se Boaz. 
\par 2 Tada Ruta Moapka reče Noemi: "Htjela  bih ići u polje pabirčiti klasje za onim u koga nađem milost."  Ona joj odgovori: "Hajde, kćeri moja!" 
\par 3 I ode, dođe u polje te poče pabirčiti za žeteocima. A  sreća je dovede u polje koje pripadaše Boazu, iz roda Elimelekova. 
\par 4 I gle, dođe Boaz iz Betlehema. "Jahve bio s vama!" - pozdravi  on žeteoce. A oni mu odgovoriše: "Jahve te blagoslovio!" 
\par 5 Boaz  će nato momku koji je nadzirao žeteoce: "Čija je ona mlada žena?" 
\par 6 A momak koji bijaše nad žeteocima odgovori: "Ono je mlada  Moapka što je došla prateći Noemi s Moapskih poljana. 
\par 7 Pitala  je: 'Smijem li pabirčiti i kupiti klasje između snopova za žeteocima?'  I došla je, eto, i ostala od ranog jutra sve dosad; i samo je  malo ušla u kuću." 
\par 8 Onda Boaz reče Ruti: "Čuj me, kćeri moja, ne idi pabirčiti  u drugoga nego se drži mojih njiva i mojih poslenika. 
\par 9 Pazi  na kojoj njivi oni žanju, pa idi za njima. A naredio sam momcima  da te nitko ne dira. Kad ožedniš, idi k posudama i pij što moje  sluge zahitaju." 
\par 10 Ona tada pade ničice, pokloni se do zemlje i reče: "Čime  sam stekla toliku milost u očima tvojim da mi posvećuješ pažnju  kad sam tuđinka?" 
\par 11 Boaz joj odgovori: "Čuo sam što si sve učinila za svoju  svekrvu poslije smrti svoga muža; kako si ostavila oca svoga, majku svoju i zavičaj svoj te došla u narod kojega do jučer  ili prekjučer nisi poznavala. 
\par 12 Neka ti Jahve plati sve što  si učinila i neka ti udijeli pravu nagradu Jahve, Bog Izraelov, kad si došla da se pod krila njegova skloniš!" 
\par 13 Ona preuze: "Kad bih mogla uvijek nalaziti milost u tvojim  očima, gospodaru, jer si me utješio i milostivo progovorio sluškinji  svojoj, ako i nisam kao jedna od tvojih sluškinja." 
\par 14 Kad bijaše vrijeme ručku, Boaz joj reče: "Hodi ovamo, jedi ovog kruha i umoči svoj zalogaj u ocat!" Ona sjede pokraj žetelaca, a on stavi pred nju prženih zrna.  Jela je i nasitila se i još joj preteče. 
\par 15 Kad je ustala da pabirči dalje, Boaz zapovjedi svojim  slugama: "I među snopljem neka ona pabirči, a vi joj nemojte  zanovijetati. 
\par 16 Nego navlaš ispuštajte klasove iz svojih rukoveti  i ostavljajte joj neka kÓupi i nemojte je koriti!" 
\par 17 I tako je pabirčila sve do večeri, pa onda ovrše ono  što je napabirčila: bijaše otprilike jedna efa ječma. 
\par 18 Uze ona svoje i dođe u grad, a svekrva vidje koliko je  napabirčila. Tada Ruta izvadi i dade joj što joj bijaše preteklo  pošto se nasitila. 
\par 19 Svekrva je upita: "Gdje si pabirčila danas? Gdje si radila?  Neka je blagoslovljen onaj koji je pogledao na te!" Onda ona pripovjedi svekrvi kod koga je radila i reče: "Čovjek  u koga sam danas radila zove se Boaz." 
\par 20 Tada će Noemi svojoj snasi: "Neka Jahve blagoslovi onoga  koji ne uskraćuje dobrote svoje ni živima ni mrtvima!" I dometnu  Noemi: "Taj je čovjek naš rod; jedan od naših skrbnika." 
\par 21 Ruta Moapka pripovjedi dalje: "Još mi reče: 'Drži se  mojih poslenika dokle ne požanju sve moje!'" 
\par 22 Noemi nato reče Ruti, snasi svojoj: "Dobro je, kćeri  moja, idi za njegovim poslenicima da ti ne bude neprilike na  kojoj drugoj njivi." 
\par 23 I tako se držala poslenika Boazovih i pabirčila dokle  ne požeše i ječam i pšenicu. I živjela je kod svekrve svoje. 


\chapter{3}

\par 1 Onda će joj Noemi, svekrva njezina: "Kćeri moja, da ti potražim  mirno mjesto gdje bi mogla biti sretna? 
\par 2 Vidiš, Boaz, s čijim  si se poslenicima našla, naš je rođak. Evo, on će noćas vijati  ječam na gumnu. 
\par 3 Umij se ti i namaži, lijepo se odjeni pa idi  na gumno. Ne daj da te prepozna prije nego što se najede i napije. 
\par 4 Kad bude lijegao, dobro pazi gdje će leći; pa kad legne, otiđi  onamo, podigni mu pokrivač s nogu i lezi ondje! Tada će ti on  reći što ti je činiti." 
\par 5 Ona joj odgovori: "Učinit ću sve kako mi kažeš." 
\par 6 I siđe na gumno i učini sve kako joj je svekrva naredila. 
\par 7 A Boaz, pošto je jeo i pio i tako se odobrovoljio, ode i leže  kraj stoga. Onda ona priđe polako, otkri mu noge i leže. 
\par 8 Kad bijaše oko ponoći, trže se čovjek i obrnu se, i gle:  žena leži do njegovih nogu. 
\par 9 "Tko si?" - upita on, a ona odgovori: "Ja sam Ruta, sluškinja  tvoja. Raširi skut svoje haljine na sluškinju svoju jer si mi  skrbnik." 
\par 10 "Blagoslovio te Jahve, kćeri moja!" - dočeka on. "Ovaj  drugi tvoj čin milosti još je vredniji od prvoga, jer se nisi  trudila da slijediš mlade poslenike, bili oni bogati ili siromašni. 
\par 11 I zato se, kćeri moja, sada ne plaši: učinit ću ti sve što  zatražiš, jer sva vrata moga naroda znaju da si čestita žena. 
\par 12 Jest, uistinu sam ti skrbnik; ali postoji još bliži od mene. 
\par 13 Ostani noćas; ako te sutra ujutro on kao skrbnik htjedne  uzeti, dobro, neka te uzme; a ne htjedne li, uzet ću te ja, tako  mi Jahve! Spavaj do jutra." 
\par 14 I spavaše ona do njegovih nogu do jutra. On ustade prije  nego što mogaše čovjek čovjeka razaznati jer mišljaše: "Ne treba  da znaju da je žena bila na gumnu." 
\par 15 I kaza joj: "Daj ogrtač  što je na tebi i drži ga dobro." Ona ga pridrža, a on joj nasu  šest mjerica ječma i naprti joj. I ode ona u grad. 
\par 16 Kad je stigla, upita je svekrva: "Što je s tobom, kćeri  moja?" A ona joj pripovjedi sve što je učinio za nju. 
\par 17 I nadoveza:  "Ovih šest mjerica ječma dade mi kazujući: 'Ne smiješ se vratiti  svekrvi praznih ruku.'" 
\par 18 Nato će joj Noemi: "Budi mirna, kćeri moja, dok ne vidiš  što će biti: jer neće on imati spokoja dok sve još danas ne dokrajči." 



\chapter{4}

\par 1 Boaz potom iziđe na gradska vrata i sjede ondje. I gle, naiđe  onaj skrbnik o kome je govorio. I dozva ga Boaz: "Ej, hodi ovamo  i sjedni!" Onaj dođe i sjede. 
\par 2 Onda Boaz uze deset ljudi između starješina gradskih i  reče: "Posjedajte ovdje!" I posjedaše. 
\par 3 Zatim reče skrbniku: "Noemi, koja se vratila s Moapskih  polja, htjela bi prodati ono zemlje našega brata Elimeleka. 
\par 4 Zato  sam odlučio da se s tobom razgovorim i predložim ti: otkupi njivu  pred ovima koji sjede ovdje i pred starješinama moga naroda.  Ako je kaniš otkupiti, onda otkupi; ako ne kaniš, kaži mi da  znam. Jer prije tebe nema nitko pravo na otkup; ja sam na redu  tek iza tebe." A onaj reče: "Hoću, otkupit ću je." 
\par 5 Onda kaza Boaz: "Kad uzmeš zemlju iz ruke Noemi, treba  da uzmeš i Rutu Moapku, pokojnikovu ženu, da se pokojniku sačuva  ime na baštini." 
\par 6 Ali skrbnik reče: "E, onda ne mogu biti otkupnik, da ne  raspem svoje baštine. Otkupi ti po svome skrbničkom pravu jer  ja ne mogu." 
\par 7 A bijaše od starine običaj u Izraelu: da se čemu potkrijepi  valjanost otkupa ili zamjene, čovjek bi izuo sandalu i dao je  drugome. To bijaše svjedočanstvo u Izraelu. 
\par 8 Tako dakle i onaj  skrbnik reče Boazu: "Otkupi ti!" te izu sandalu i dade mu je. 
\par 9 Tada Boaz kaza starješinama i svemu narodu: "Vi ste danas  svjedoci da ja otkupljujem iz ruke Noemine sve ono što je bilo  Elimelekovo, sve što je bilo Kiljonovo i Mahlonovo. 
\par 10 Uz to  uzimam za ženu Rutu Moapku, ženu Mahlonovu, da bi se sačuvalo  ime pokojnikovo na baštini i da se ime njegovo ne bi zatrlo među  braćom njegovom i nestalo s vrata zavičaja njegova. Vi ste danas  tome svjedoci." 
\par 11 Sav narod koji se nalazio na vratima gradskim i starješine  rekoše: "Svjedoci smo! Dao Jahve da žena koja ulazi u dom tvoj  bude kao Rahela i Lea, koje su obje podigle kuću Izraelovu! Obogati  se u Efrati, a prodiči u Betlehemu! 
\par 12 Neka tvoja kuća, po potomstvu  koje će ti dati Jahve od ove mlade žene, bude kao kuća Peresa, koga Judi rodi Tamara!" 
\par 13 Tako Boaz uze Rutu i ona posta žena njegova. Uđe on k  njoj i Jahve joj dade te ona zatrudnje i rodi sina. 
\par 14 Onda žene rekoše Noemi: "Blagoslovljen bio Jahve koji  ti danas nije uskratio skrbnika! I prodičio njegovo ime u Izraelu! 
\par 15 On će biti tvoja utjeha i potpora starosti tvojoj; jer ga  rodi snaha tvoja koja te ljubi i koja ti vrijedi više od sedam  sinova." 
\par 16 Noemi uze dječaka, metnu ga sebi na krilo i bi mu odgojiteljicom. 
\par 17 Susjede mu nadjenuše ime govoreći: "Noemi se rodio sin!"  I prozvaše ga Obed; on je otac Jišaja, oca Davidova. 
\par 18 A ovo je rodoslovlje Peresovo: 
\par 19 Peres imade sina Hesrona, Hesron Rama, Ram Aminadaba, 
\par 20 Aminadab Nahšona, Nahšon Salmona, 
\par 21 Salmon Boaza, Boaz Obeda, 
\par 22 Obed Jišaja, a Jišaj Davida. 





\end{document}