\begin{document}

\title{1 Korinčanima}


\chapter{1}

\par 1 Pavao, po Božjoj volji pozvan za apostola Krista Isusa, i brat  Sosten 
\par 2 Crkvi Božjoj u Korintu - posvećenima u Kristu Isusu, pozvanicima, svetima, sa svima što na bilo kojemu mjestu prizivlju  ime Isusa Krista, Gospodina našega, njihova i našega. 
\par 3 Milost vam i mir od Boga, Oca našega, i Gospodina Isusa  Krista! 
\par 4 Zahvaljujem Bogu svojemu svagda za vas zbog milosti Božje  koja vam je dana u Kristu Isusu: 
\par 5 u njemu se obogatiste u svemu  - u svakoj riječi i svakom spoznanju. 
\par 6 Kako li se svjedočanstvo  o Kristu utvrdilo u vama 
\par 7 te ne oskudijevate ni na jednom daru  čekajući Objavljenje Gospodina našega Isusa Krista! 
\par 8 On će  vas učiniti i postojanima do kraja, besprigovornima u Dan Gospodina  našega Isusa Krista. 
\par 9 Vjeran je Bog koji vas pozva u zajedništvo  Sina svojega Isusa Krista, Gospodina našega. 
\par 10 Zaklinjem vas, braćo, imenom Gospodina našega Isusa Krista:  svi budite iste misli; neka ne bude među vama razdora, nego budite  savršeno istog osjećanja i istog mišljenja. 
\par 11 Jer Klojini mi, braćo moja, o vama rekoše da među vama ima svađa. 
\par 12 Mislim  to što svaki od vas govori: "Ja sam Pavlov", "A ja Apolonov", "A ja Kefin", "A ja Kristov". 
\par 13 Zar je Krist razdijeljen?  Zar je Pavao raspet za vas? Ili ste u Pavlovo ime kršteni? 
\par 14 Hvala  Bogu što ne krstih nikoga od vas, osim Krispa i Gaja; 
\par 15 da  ne bi tko rekao da ste u moje ime kršteni. 
\par 16 A da, krstih i  Stefanin dom. Inače ne znam krstih li koga drugoga. 
\par 17 Jer ne posla me Krist krstiti, nego navješćivati evanđelje, i to ne mudrošću besjede, da se ne obeskrijepi križ Kristov. 
\par 18 Uistinu, besjeda o križu ludost je onima koji propadaju, a nama spašenicima sila je Božja. 
\par 19 Ta pisano je: Upropastit  ću mudrost mudrih, i odbacit ću umnost umnih. 
\par 20 Gdje je mudrac?  Gdje je književnik? Gdje je istraživač ovoga svijeta? Zar  ne izludi Bog mudrost svijeta? 
\par 21 Doista, kad svijet u mudrosti  Božjoj Boga ne upozna mudrošću, svidjelo se Bogu ludošću propovijedanja  spasiti vjernike. 
\par 22 Jer i Židovi znake ištu i Grci mudrost  traže, 
\par 23 a mi propovijedamo Krista raspetoga: Židovima sablazan, poganima ludost, 
\par 24 pozvanima pak - i Židovima i Grcima - Krista, Božju snagu i Božju mudrost. 
\par 25 Jer ludo Božje mudrije je od  ljudi i slabo Božje jače je od ljudi. 
\par 26 Ta gledajte, braćo, sebe, pozvane: nema mnogo mudrih po tijelu, nema mnogo snažnih, nema mnogo plemenitih. 
\par 27 Nego lude svijeta izabra Bog da posrami  mudre, i slabe svijeta izabra Bog da posrami jake; 
\par 28 i neplemenite  svijeta i prezrene izabra Bog, i ono što nije, da uništi ono  što jest, 
\par 29 da se nijedan smrtnik ne bi hvalio pred Bogom. 
\par 30 Od njega je da vi jeste u Kristu Isusu, koji nama posta mudrost  od Boga, i pravednost, i posvećenje, i otkupljenje, 
\par 31 da bude  kako je pisano: Tko se hvali, u Gospodu neka se hvali. 


\chapter{2}

\par 1 I ja kada dođoh k vama, braćo, ne dođoh s uzvišenom besjedom  ili mudrošću navješćivati vam svjedočanstvo Božje 
\par 2 jer ne htjedoh  među vama znati što drugo osim Isusa Krista, i to raspetoga. 
\par 3 I ja priđoh k vama slab, u strahu i u veliku drhtanju. 
\par 4 I  besjeda moja i propovijedanje moje ne bijaše u uvjerljivim riječima  mudrosti, nego u pokazivanju Duha i snage 
\par 5 da se vjera vaša  ne temelji na mudrosti ljudskoj nego na snazi Božjoj. 
\par 6 Mudrost doduše navješćujemo među zrelima, ali ne mudrost  ovoga svijeta, ni knezova ovoga svijeta koji propadaju, 
\par 7 nego  navješćujemo Mudrost Božju, u Otajstvu, sakrivenu; onu koju predodredi  Bog prije vjekova za slavu našu, 
\par 8 a koje nijedan od knezova  ovoga svijeta nije upoznao. Jer da su je upoznali, ne bi Gospodina  slave razapeli. 
\par 9 Nego, kako je pisano: Što oko ne vidje, i uho ne ču, i u srce čovječje ne uđe, to pripravi Bog onima  koji ga ljube. 
\par 10 A nama to Bog objavi po Duhu jer Duh sve  proniče, i dubine Božje. 
\par 11 Jer tko od ljudi zna što je u čovjeku  osim duha čovječjega u njemu? Tako i što je u Bogu, nitko ne  zna osim Duha Božjega. 
\par 12 A mi, mi ne primismo duha svijeta, nego Duha koji je od Boga da znamo čime nas je obdario Bog. 
\par 13 To i navješćujemo, ne naučenim riječima čovječje mudrosti, nego naukom Duha izlažući duhovno duhovnima. 
\par 14 Naravan čovjek  ne prima što je od Duha Božjega; njemu je to ludost i ne može  spoznati jer po Duhu valja prosuđivati. 
\par 15 Duhovan pak prosuđuje  sve, a njega nitko ne prosuđuje. 
\par 16 Jer tko spozna misao  Gospodnju, tko da ga pouči? A mi imamo misao Kristovu. 


\chapter{3}

\par 1 I ja, braćo, nisam mogao govoriti vama kao duhovnima, nego  kao tjelesnima, kao nejačadi u Kristu. 
\par 2 Mlijekom vas napojih, ne jelom: još ne mogoste, a ni sada još ne možete 
\par 3 jer još  ste tjelesni. Doista, dok je među vama zavist i prepiranje, zar  niste tjelesni, zar po ljudsku ne postupate? 
\par 4 Jer kad jedan  govori: "Ja sam Pavlov", a drugi: "Ja Apolonov", niste li odveć  ljudi? 
\par 5 Ta što je Apolon? Što je Pavao? Poslužitelji po kojima  povjerovaste - kako već komu Gospodin dade. 
\par 6 Ja zasadih, Apolon  zali, ali Bog dade rasti. 
\par 7 Tako niti je što onaj tko sadi ni  onaj tko zalijeva, nego Bog koji daje rasti. 
\par 8 Tko sadi i tko  zalijeva, jedno su; a svaki će po svome trudu primiti plaću. 
\par 9 Jer Božji smo suradnici: Božja ste njiva, Božja građevina. 
\par 10 Po milosti Božjoj koja mi je dana ja kao mudri graditelj  postavih temelj, a drugi naziđuje; ali svaki neka pazi kako naziđuje. 
\par 11 Jer nitko ne može postaviti drugoga temelja osim onoga koji  je postavljen, a taj je Isus Krist. 
\par 12 Naziđuje li tko na ovom  temelju zlatom, srebrom, dragim kamenjem, drvom, sijenom, slamom  - 
\par 13 svačije će djelo izići na svjetlo. Onaj će Dan pokazati  jer će se u ognju očitovati. I kakvo je čije djelo, oganj će  iskušati. 
\par 14 Ostane li djelo, primit će plaću onaj tko ga je  nazidao. 
\par 15 Izgori li čije djelo, taj će štetovati; ipak, on  će se sam spasiti, ali kao kroz oganj. 
\par 16 Ne znate li? Hram ste Božji i Duh Božji prebiva u vama. 
\par 17 Ako tko upropašćuje hram Božji, upropastit će njega Bog.  Jer hram je Božji svet, a to ste vi. 
\par 18 Nitko neka se ne vara. Ako tko misli da je mudar među  vama na ovome svijetu, neka bude lud da bude mudar. 
\par 19 Jer mudrost  ovoga svijeta ludost je pred Bogom. Ta pisano je: On hvata  mudre u njihovu lukavstvu. 
\par 20 I opet: Gospodin poznaje  namisli mudrih, one su isprazne. 
\par 21 Zato neka se nitko ne  hvasta ljudima jer sve je vaše. 
\par 22 Bio Pavao, ili Apolon, ili  Kefa, bio svijet, ili život, ili smrt, ili sadašnje, ili buduće:  sve je vaše, 
\par 23 vi Kristovi, a Krist Božji. 


\chapter{4}

\par 1 Tako, neka nas svatko smatra službenicima Kristovim i upraviteljima  otajstava Božjih. 
\par 2 A od upravitelja iziskuje se napokon da  budu vjerni. 
\par 3 Meni pak nije nimalo do toga da me sudite vi  ili bilo koji ljudski sud; a ni ja sam sebe ne sudim. 
\par 4 Doista, ničega sebi nisam svjestan, no time nisam opravdan: moj je sudac  Gospodin. 
\par 5 Zato ne sudite ništa prije vremena dok ne dođe Gospodin  koji će iznijeti na vidjelo što je sakriveno u tami i razotkriti  nakane srdaca. I tada će svatko primiti pohvalu od Boga. 
\par 6 Time, braćo, smjerah na sebe i Apolona radi vas: da na  nama naučite onu "Ne preko onoga što je pisano" te se ne nadimate  jednim protiv drugoga. 
\par 7 Ta tko tebi daje prednost? Što imaš  da nisi primio? Ako si primio, što se hvastaš kao da nisi primio? 
\par 8 Već ste siti, već se obogatiste, bez nas se zakraljiste! Kamo  sreće da se zakraljiste da i mi s vama zajedno kraljujemo! 
\par 9 Jer  Bog je, čini mi se, nas apostole prikazao posljednje, kao na  smrt osuđene, jer postali smo prizor svijetu, i anđelima, i ljudima  - 
\par 10 mi ludi poradi Krista, vi mudri u Kristu; mi slabi, vi  jaki; vi čašćeni, mi prezreni; 
\par 11 sve do ovoga časa i gladujemo, i žeđamo, i goli smo, i pljuskaju nas, i beskućnici smo, 
\par 12 i  patimo se radeći svojim rukama. Proklinjani blagoslivljamo, proganjani  ustrajavamo, 
\par 13 pogrđivani tješimo. Kao smeće svijeta postasmo, svačiji izmet sve do sada. 
\par 14 Ne pišem ovoga da vas postidim, nego da vas kao ljubljenu  svoju djecu urazumim. 
\par 15 Jer da imate u Kristu i deset tisuća  učitelja, ipak ne biste imali više otaca. Ta u Kristu Isusu po  evanđelju ja vas rodih! 
\par 16 Zaklinjem vas, dakle: nasljedovatelji  moji budite. 
\par 17 Zato upravo poslah k vama Timoteja, koji mi  je dijete ljubljeno i vjerno u Gospodinu, da vas podsjeti na  naputke moje, u Kristu, kako posvuda u svakoj crkvi učim. 
\par 18 Neki  se uzniješe kao da ja neću doći k vama. 
\par 19 Ipak, eto me ubrzo  k vama, ako Gospodin htjedne, i rasudit ću ne riječi onih nadutih, nego krepost. 
\par 20 Ta nije u riječi kraljevstvo Božje, nego u  kreposti. 
\par 21 Što želite? Da k vama dođem sa šibom ili s ljubavlju  i duhom blagosti? 


\chapter{5}

\par 1 Općenito se čuje o bludnosti među vama, i to takvoj bludnosti  kakve nema ni među poganima: da netko ima očevu ženu. 
\par 2 I vi  mi se uznijeli, mjesto da žalujete pa da se iskorijeni iz vaše  sredine onaj koji takvo djelo počini. 
\par 3 A ja, i nenazočan tijelom, ali nazočan duhom, već sam presudio kao nazočan onoga koji je  takvo što počinio. 
\par 4 Pošto se u ime Gospodina našega Isusa Krista  okupite vi i moj duh, snagom Gospodina našega Isusa, 
\par 5 neka  se takav preda Sotoni na propast tijela da bi se spasio duh u  Dan Gospodina Isusa. 
\par 6 Ne valja vam hvastanje! Zar ne znate da malo kvasca sve  tijesto ukvasa? 
\par 7 Očistite stari kvasac da budete novo tijesto, kao što i jeste beskvasni jer već je žrtvovana Pasha naša, Krist. 
\par 8 Zato svetkujmo, ne sa starim kvascem ni s kvascem zloće i  pakosti, nego s beskvasnim kruhovima čistoće i istine. 
\par 9 Napisah vam u poslanici da se ne miješate s bludnicima  - 
\par 10 ne općenito s bludnicima ovoga svijeta, ili lakomcima,  ili razbojnicima, ili idolopoklonicima jer biste inače morali  iz svijeta izići. 
\par 11 Napisah vam zapravo da se ne miješate s  nazovibratom koji bi bio bludnik, ili lakomac, ili idolopoklonik, ili pogrđivač, ili pijanica, ili razbojnik. S takvim ni za stol! 
\par 12 Što spada na me suditi one vani? Ne sudite li vi one koji  su unutra? 
\par 13 A one vani sudit će Bog. Iskorijenite opakoga  iz svoje sredine. 


\chapter{6}

\par 1 Tko bi se od vas u sporu s drugim usudio parničiti se pred  nepravednima, a ne pred svetima? 
\par 2 Ili zar ne znate da će sveti  suditi svijet? Pa ako ćete vi suditi svijet, zar niste vrijedni  suditi sitnice? 
\par 3 Ne znate li da ćemo suditi anđele, kamo li  ne ono svagdanje? 
\par 4 A vi, kad imate sporove o svagdanjem, sucima  postavljate one do kojih Crkva ništa ne drži! 
\par 5 Vama na sramotu  govorim. Tako? Zar nema među vama ni jednoga mudra koji bi mogao  rasuditi među braćom? 
\par 6 Nego brat se s bratom parniči, i to  pred nevjernicima? 
\par 7 Zapravo, već vam je to nedostatak što se  parničite među sobom. Zašto radije ne trpite nepravdu? Zašto  se radije ne pustite oplijeniti? 
\par 8 Nego vi činite nepravdu i  plijenite, i to braću. 
\par 9 Ili zar ne znate da nepravednici neće  baštiniti kraljevstva Božjega? Ne varajte se! Ni bludnici, ni  idolopoklonici, ni preljubnici, ni mekoputnici, ni muškoložnici, 
\par 10 ni kradljivci, ni lakomci, ni pijanice, ni psovači, ni razbojnici  neće baštiniti kraljevstva Božjega. 
\par 11 To evo, bijahu neki od  vas, ali oprali ste se, ali posvetili ste se, ali opravdali ste  se u imenu Gospodina našega Isusa Krista i u Duhu Boga našega. 
\par 12 "Sve mi je dopušteno!" Ali - sve ne koristi. "Sve mi  je dopušteno!" Ali - neću da mnome išta vlada. 
\par 13 "Jela trbuhu, a trbuh jelima; Bog će i jedno i drugo uništiti." Ali ne tijelo  bludnosti, nego Gospodinu, i Gospodin tijelu! 
\par 14 Ta Bog koji  je Gospodina uskrisio i nas će uskrisiti snagom njegovom. 
\par 15 Ne  znate li da su tijela vaša udovi Kristovi? Hoću li dakle uzeti  udove Kristove i učiniti ih udovima bludničinim? Nipošto! 
\par 16 Ili  zar ne znate: tko uz bludnicu prione, jedno je tijelo? Jer veli  se: Bit će njih dvoje jedno tijelo. 
\par 17 A tko prione uz  Gospodina, jedan je duh. 
\par 18 Bježite od bludnosti! Svaki grijeh  koji učini čovjek, izvan tijela je, a bludnik griješi protiv  svojega tijela. 
\par 19 Ili zar ne znate? Tijelo vaše hram je Duha  Svetoga koji je u vama, koga imate od Boga, te niste svoji. 
\par 20 Jer  kupljeni ste otkupninom. Proslavite dakle Boga u tijelu svojem! 


\chapter{7}

\par 1 Sada o onome što ste mi pisali. Dobro je čovjeku ne dotaći  ženu. 
\par 2 Ipak, zbog bludnosti, neka svaki ima svoju ženu i svaka  neka ima svoga muža. 
\par 3 Muž neka vrši dužnost prema ženi, a tako  i žena prema mužu. 
\par 4 Žena nije gospodar svoga tijela, nego muž, a tako ni muž nije gospodar svoga tijela, nego žena. 
\par 5 Ne uskraćujte  se jedno drugome, osim po dogovoru, povremeno, da se posvetite  molitvi pa se opet združite da vas Sotona ne bi napastovao zbog  vaše neizdržljivosti. 
\par 6 Ali to velim kao dopuštenje, ne kao  zapovijed. 
\par 7 A htio bih da svi ljudi budu kao i ja; ali svatko  ima svoj dar od Boga, ovaj ovako, onaj onako. 
\par 8 Neoženjenima pak i udovicama velim: dobro im je ako ostanu  kao i ja. 
\par 9 Ako li se ne mogu uzdržati, neka se žene, udaju.  Jer bolje je ženiti se negoli izgarati. 
\par 10 A oženjenima zapovijedam, ne ja, nego Gospodin: žena  neka se od muža ne rastavlja - 
\par 11 ako se ipak rastavi, neka  ostane neudana ili neka se s mužem pomiri - i muž neka ne otpušta  žene. 
\par 12 Ostalima pak velim - ja, ne Gospodin: ima li koji brat  ženu nevjernicu i ona privoli stanovati s njime, neka je ne otpušta. 
\par 13 I žena koja ima muža nevjernika te on privoli stanovati s  njome, neka ne otpušta muža. 
\par 14 Ta muž nevjernik posvećen je  ženom i žena nevjernica posvećena je bratom. Inače bi djeca vaša  bila nečista, a ovako - sveta su. 
\par 15 Ako li se nevjernik hoće  rastaviti, neka se rastavi; brat ili sestra u takvim prilikama  nisu vezani: ta na mir nas je pozvao Bog. 
\par 16 Jer što znaš, ženo, hoćeš li spasiti muža? Ili što znaš, mužu, hoćeš li spasiti  ženu? 
\par 17 U drugome svatko neka živi kako mu je Gospodin dodijelio, kako ga je Bog pozvao. Tako određujem po svim crkvama. 
\par 18 Je  li tko pozvan kao obrezan, neka ne prepravlja obrezanja. Ako  je pozvan kao neobrezan, neka se ne obrezuje. 
\par 19 Obrezanje nije  ništa i neobrezanje nije ništa, nego - držanje Božjih zapovijedi. 
\par 20 Svatko neka ostane u onom zvanju u koje je pozvan. 
\par 21 Jesi  li pozvan kao rob? Ne brini! Nego, ako i možeš postati slobodan, radije se okoristi. 
\par 22 Jer tko je u Gospodinu pozvan kao rob, slobodnjak je  Gospodnji. Tako i tko je pozvan kao slobodnjak, rob je Kristov. 
\par 23 Otkupninom ste kupljeni: ne budite robovi ljudima. 
\par 24 Svatko  u čemu je pozvan, braćo, u tome neka i ostane pred Bogom. 
\par 25 O djevicama nemam zapovijedi, nego dajem savjet kao čovjek  po milosrđu Gospodnjem vrijedan povjerenja. 
\par 26 Smatram dakle:  dobro je to zbog sadašnje nevolje, dobro je čovjeku tako biti. 
\par 27 Jesi li vezan za ženu? Ne traži rastave. Jesi li slobodan  od žene? Ne traži žene. 
\par 28 Ali ako se i oženiš, nisi sagriješio;  i djevica ako se uda, nije sagriješila. Ali takvi će imati tjelesnu  nevolju, a ja bih vas rado poštedio. 
\par 29 Ovo hoću reći, braćo: Vrijeme je kratko. Odsele i koji  imaju žene, neka budu kao da ih nemaju; 
\par 30 i koji plaču, kao  da ne plaču; i koji se vesele, kao da se ne vesele; i koji kupuju, kao da ne posjeduju; 
\par 31 i koji uživaju ovaj svijet, kao da  ga ne uživaju, jer - prolazi obličje ovoga svijeta. 
\par 32 A rado  bih da budete bezbrižni. Neoženjen se brine za Gospodnje, kako  da ugodi Gospodinu. 
\par 33 A oženjen se brine za svjetovno, kako  da ugodi ženi, 
\par 34 pa je razdijeljen. I žena neudana i djevica  brine se za Gospodnje, da bude sveta i tijelom i duhom; a udana  se brine za svjetovno, kako da ugodi mužu. 
\par 35 Ovo pak govorim  vama na korist, ne da vam postavim zamku, nego da primjerno i  nesmetano budete privrženi Gospodinu. 
\par 36 Misli li tko da je  nepriličan prema svojoj djevici kad je preživotan i s njome mora  biti, neka čini što je nakanio, ne griješi: neka se uzmu. 
\par 37 Tko  je pak nepokolebljivo stalan u srcu te nema potrebe, a u vlasti  mu je volja pa to odluči u svom srcu - čuvati svoju djevicu -  dobro čini. 
\par 38 Tako, tko se oženi svojom djevicom, dobro čini, a tko se ne oženi, bolje čini. 
\par 39 Žena je vezana dokle živi  muž njezin. Umre li muž, slobodna je: neka se uda za koga hoće, samo u Gospodinu. 
\par 40 Bit će ipak blaženija ostane li onako, po mojem savjetu. A mislim da i ja imam Duha Božjega. 


\chapter{8}

\par 1 U pogledu mesa žrtvovana idolima, znamo, svi posjedujemo znanje.  Ali znanje nadima, a ljubav izgrađuje. 
\par 2 Ako tko misli da što  zna, još ne zna kako treba znati. 
\par 3 A ljubi li tko Boga, Bog  ga poznaje. 
\par 4 Dakle, u pogledu blagovanja mesa žrtvovana idolima, znamo: nema idola na svijetu i nema Boga do Jednoga. 
\par 5 Jer  sve kad bi i bilo nazovibogova ili na nebu ili na zemlji - kao  što ima mnogo "bogova" i mnogo "gospodara"! - 
\par 6 nama je jedan  Bog, Otac, od koga je sve, a mi za njega; i jedan Gospodin, Isus  Krist, po kome je sve, i mi po njemu. 
\par 7 Ali nemaju svi toga znanja. Neki, navikli na idole, još  jedu meso kao idolima žrtvovano i njihova se savjest kalja jer  je nejaka. 
\par 8 A k Bogu nas ne privodi jelo. Niti što gubimo ako  ne jedemo; niti što dobivamo ako jedemo. 
\par 9 A pazite da ne bi  možda ta vaša sloboda bila spoticaj nejakima. 
\par 10 Jer vidi li  tko tebe koji imaš znanje za stolom u hramu idolskomu, neće li  se njegova savjest, jer je nejaka, "izgraditi" da jede žrtvovano  idolima? 
\par 11 I s tvoga znanja propada nejaki, brat za kojega  je Krist umro. 
\par 12 Tako griješeći protiv braće i ranjavajući  njihovu nejaku savjest, protiv Krista griješite. 
\par 13 Zato ako  jelo sablažnjava brata moga, ne, neću jesti mesa dovijeka da  brata svoga ne sablaznim. 


\chapter{9}

\par 1 Nisam li ja slobodan? Nisam li apostol? Nisam li vidio Isusa, Gospodina našega? Niste li vi djelo moje u Gospodinu? 
\par 2 Ako  drugima nisam apostol, vama svakako jesam. Ta vi ste pečat mojega  apostolstva u Gospodinu. 
\par 3 Moj odgovor mojim tužiteljima jest ovo: 
\par 4 Zar nemamo  prava jesti i piti? 
\par 5 Zar nemamo prava ženu vjernicu voditi  sa sobom kao i drugi apostoli i braća Gospodnja i Kefa? 
\par 6 Ili  samo ja i Barnaba nemamo prava ne raditi? 
\par 7 Tko ikada vojuje  o svojem trošku? Tko sadi vinograd pa roda njegova ne jede? Ili  tko pase stado pa od mlijeka stada ne jede? 
\par 8 Zar to govorim  po ljudsku? Ne kaže li to i Zakon? 
\par 9 Jer u Mojsijevu zakonu  piše: Ne zavezuj usta volu koji vrši! Zar je Bogu do volova? 
\par 10 Ne govori li on baš radi nas? Doista, radi nas je napisano, jer tko ore, u nadi treba da ore; i tko vrši, u nadi da će dobiti  dio. 
\par 11 Ako smo mi vama sijali dobra duhovna, veliko li je nešto  ako vam požanjemo tjelesna? 
\par 12 Ako drugi sudjeluju u vašim dobrima, zašto ne bismo mi mogli još većma. Ali nismo se poslužili tim  pravom, nego sve teglimo da ne bismo postavili kakvu zapreku  evanđelju Kristovu? 
\par 13 Ne znate li: koji obavljaju svetinje, od svetišta se hrane; i koji žrtveniku služe, sa žrtvenikom  dijele? 
\par 14 Tako je i Gospodin onima koji evanđelje navješćuju  odredio od evanđelja živjeti. 
\par 15 No ja se ničim od toga nisam poslužio. A i ne napisah  toga da bi se tako postupilo prema meni. Radije umrijeti, nego...  Te mi slave nitko neće oduzeti! 
\par 16 Jer što navješćujem evanđelje, nije mi na hvalu, ta dužnost mi je. Doista, jao meni ako evanđelja  ne navješćujem. 
\par 17 Jer ako to činim iz vlastite pobude, ide  me plaća; ako li ne iz vlastite pobude - služba je to koja mi  je povjerena. 
\par 18 Koja mi je dakle plaća? Da propovijedajući  pružam evanđelje besplatno ne služeći se svojim pravom u evanđelju. 
\par 19 Jer premda slobodan od sviju, sam sebe svima učinih slugom  da ih što više steknem. 
\par 20 Bijah Židovima Židov da Židove steknem;  onima pod Zakonom, kao da sam pod Zakonom - premda ja nisam pod  Zakonom - da one pod Zakonom steknem; 
\par 21 onima bez Zakona, kao  da sam bez zakona - premda nisam bez Božjega zakona, nego u Kristovu  zakonu - da steknem one bez Zakona; 
\par 22 bijah nejakima nejak  da nejake steknem. Svima bijah sve da pošto-poto neke spasim. 
\par 23 A sve činim poradi evanđelja da bih i ja bio suzajedničar  u njemu. 
\par 24 Ne znate li: trkači u trkalištu svi doduše trče, ali  jedan prima nagradu? Tako trčite da dobijete. 
\par 25 Svaki natjecatelj  sve moguće izdržava; oni da dobiju raspadljiv vijenac, mi neraspadljiv. 
\par 26 Ja dakle tako trčim - ne kao besciljno, tako udaram šakom  - ne kao da mlatim vjetar, 
\par 27 nego krotim svoje tijelo i zarobljavam  da sam ne budem isključen pošto sam drugima propovijedao. 


\chapter{10}

\par 1 Jer ne bih, braćo, htio da budete u neznanju: oci naši svi  bijahu pod oblakom, i svi prijeđoše kroz more, 
\par 2 i svi su se  na Mojsija krstili u oblaku i u moru, 
\par 3 i svi su isto duhovno  jelo jeli, 
\par 4 i svi su isto duhovno piće pili. A pili su iz duhovne  stijene koja ih je pratila; stijena bijaše Krist. 
\par 5 Ali većina  njih nije bila po volji Bogu: ta poubijani su po pustinji. 
\par 6 To  bijahu pralikovi naši: da ne žudimo za zlima kao što su žudjeli  oni. 
\par 7 I ne budite idolopoklonici kao neki od njih, kako je  pisano: Posjeda narod da jede i pije pa ustadoše da igraju. 
\par 8 I ne podajimo se bludu kao što se neki od njih bludu podaše  i padoše u jednom danu dvadeset i tri tisuće. 
\par 9 I ne iskušavajmo  Gospodina kao što su ga neki od njih iskušavali te od zmija izginuli. 
\par 10 I ne mrmljajte kao što neki od njih mrmljahu te izgiboše  od Zatornika. 
\par 11 Sve se to, kao pralik, događalo njima, a napisano  je za upozorenje nama, koje su zapala posljednja vremena. 
\par 12 Tko  dakle misli da stoji, neka pazi da ne padne. 
\par 13 Nije vas zahvatila  druga kušnja osim ljudske. Ta vjeran je Bog: neće pustiti da  budete kušani preko svojih sila, nego će s kušnjom dati i ishod  da možete izdržati. 
\par 14 Zato, ljubljeni moji, bježite od idolopoklonstva. 
\par 15 Kao  razumnima velim: sudite sami što govorim. 
\par 16 Čaša blagoslovna  koju blagoslivljamo nije li zajedništvo krvi Kristove? Kruh koji  lomimo nije li zajedništvo tijela Kristova? 
\par 17 Budući da je  jedan kruh, jedno smo tijelo mi mnogi; ta svi smo dionici jednoga  kruha. 
\par 18 Gledajte Izraela po tijelu! Koji blaguju žrtve nisu  li zajedničari žrtvenika? 
\par 19 Što dakle hoću reći? Idolska žrtva  da je nešto? Ili idol da je nešto? 
\par 20 Naprotiv, da pogani vrazima  žrtvuju, ne Bogu. A neću da budete zajedničari vražji. 
\par 21 Ne  možete piti čašu Gospodnju i čašu vražju. Ne možete biti sudionici  stola Gospodnjega i stola vražjega. 
\par 22 Ili da izazivamo ljubomor  Gospodnji? Zar smo jači od njega? 
\par 23 "Sve je slobodno!" Ali - sve ne koristi. "Sve je dopušteno!"  Ali - sve ne saziđuje. 
\par 24 Nitko neka ne traži svoje, nego dobro  drugoga. 
\par 25 Sve što se prodaje na tržnici, jedite ništa ne ispitujući  poradi savjesti. 
\par 26 Ta Gospodnja je zemlja i sve na njoj! 
\par 27 Pozove li vas koji nevjernik i želite se odazvati, jedite  što vam se ponudi ništa ne ispitujući poradi savjesti. 
\par 28 Ako  vam tko reče: "To je žrtvovano", ne jedite poradi onoga koji  vas je upozorio, i savjesti. 
\par 29 Savjesti mislim, ne svoje, nego  onoga drugoga. Ta zašto da moju slobodu druga savjest sudi? 
\par 30 Ako  sa zahvalom sudjelujem, zašto da me grde zbog onoga za što zahvaljujem? 
\par 31 Dakle, ili jeli, ili pili, ili drugo što činili, sve na slavu  Božju činite. 
\par 32 Ne budite na sablazan ni Židovima, ni Grcima, ni Crkvi Božjoj, 
\par 33 kao što i ja svima u svemu ugađam ne tražeći  svoju korist, nego što koristi mnogima na spasenje. 


\chapter{11}

\par 1 Nasljedovatelji moji budite, kao što sam i ja Kristov. 
\par 2 Hvalim  vas što me se u svemu sjećate i držite se predaja kako vam predadoh. 
\par 3 Ali htio bih da znate: svakomu je mužu glava Krist, glava  ženi muž, a glava Kristu Bog. 
\par 4 Svaki muž koji se moli ili prorokuje  pokrivene glave sramoti glavu svoju. 
\par 5 Svaka pak žena koja se  moli ili prorokuje gologlava sramoti glavu svoju. Ta to je isto  kao da je obrijana. 
\par 6 Jer ako se žena ne pokriva, neka se šiša;  ako li je pak ružno ženi šišati se ili brijati, neka se pokrije. 
\par 7 A muž ne mora pokrivati glave, ta slika je i slava Božja;  a žena je slava muževa. 
\par 8 Jer nije muž od žene, nego žena od  muža. 
\par 9 I nije stvoren muž radi žene, nego žena radi muža. 
\par 10 Zato  žena treba da ima "vlast" na glavi poradi anđela. 
\par 11 Ipak, u  Gospodinu - ni žena bez muža, ni muž bez žene! 
\par 12 Jer kao što  je žena od muža, tako je i muž po ženi; a sve je od Boga. 
\par 13 Sami  sudite dolikuje li da se žena gologlava Bogu moli? 
\par 14 Ne uči  li nas i sama narav da je mužu sramota ako goji kosu? 
\par 15 A ženi  je dika ako je goji jer kosa joj je dana mjesto prijevjesa. 
\par 16 Ako  je kome do prepirke, takva običaja mi nemamo, a ni Crkve Božje. 
\par 17 Kad već dajem ta upozorenja, ne mogu pohvaliti što se  ne sastajete na bolje, nego na gore. 
\par 18 Ponajprije čujem, djelomično  i vjerujem: kad se okupite na Sastanak, da su među vama razdori. 
\par 19 Treba doista da i podjela bude među vama da se očituju prokušani  među vama. 
\par 20 Kad se dakle tako zajedno sastajete, to nije blagovanje  Gospodnje večere: 
\par 21 ta svatko se pri blagovanju prihvati svoje  večere te jedan gladuje, a drugi se opija. 
\par 22 Zar nemate kuća  da jedete i pijete? Ili Crkvu Božju prezirete i postiđujete one  koji nemaju? Što da vam kažem? Da vas pohvalim? U tom vas ne  hvalim. 
\par 23 Doista, ja od Gospodina primih što vama predadoh: Gospodin  Isus one noći kad bijaše predan uze kruh, 
\par 24 zahvalivši razlomi  i reče: "Ovo je tijelo moje - za vas. Ovo činite meni na spomen." 
\par 25 Tako i čašu po večeri govoreći: "Ova čaša novi je Savez  u mojoj krvi. Ovo činite kad god pijete, meni na spomen." 
\par 26 Doista, kad god jedete ovaj kruh i pijete čašu, smrt Gospodnju  navješćujete dok on ne dođe. 
\par 27 Stoga, tko god jede kruh ili  pije čašu Gospodnju nedostojno, bit će krivac tijela i krvi Gospodnje. 
\par 28 Neka se dakle svatko ispita pa tada od kruha jede i iz čaše  pije. 
\par 29 Jer tko jede i pije, sud sebi jede i pije ako ne razlikuje  Tijela. 
\par 30 Zato su među vama mnogi nejaki i nemoćni, i spavaju  mnogi. 
\par 31 Jer kad bismo sami sebe sudili, ne bismo bili suđeni. 
\par 32 A kad nas sudi Gospodin, odgaja nas da ne budemo sa svijetom  osuđeni. 
\par 33 Zato, braćo moja, kad se sastajete na blagovanje, pričekajte jedni druge. 
\par 34 Je li tko gladan, kod kuće neka  jede da se ne sastajete na osudu. Drugo ću urediti kada dođem. 


\chapter{12}

\par 1 O darima Duha ne bih, braćo, htio da budete u neznanju. 
\par 2 Znate  kako ste se dok bijaste pogani, zavedeni, zanosili nijemim idolima. 
\par 3 Zato vam obznanjujem: nitko tko u Duhu Božjem govori ne kaže:  "Prokletstvo Isusu". I nitko ne može reći: "Gospodin Isus" osim  u Duhu Svetom. 
\par 4 Različiti su dari, a isti Duh; 
\par 5 i različite  službe, a isti Gospodin; 
\par 6 i različita djelovanja, a isti Bog  koji čini sve u svima. 
\par 7 A svakomu se daje očitovanje Duha na  korist. 
\par 8 Doista, jednomu se po Duhu daje riječ mudrosti, drugomu  riječ spoznanja po tom istom Duhu; 
\par 9 drugomu vjera u tom istom  Duhu, drugomu dari liječenja u tom jednom Duhu; 
\par 10 drugomu čudotvorstva, drugomu prorokovanje, drugomu razlučivanje duhova, drugomu različiti  jezici, drugomu tumačenje jezika. 
\par 11 A sve to djeluje jedan  te isti Duh dijeleći svakomu napose kako hoće. 
\par 12 Doista, kao što je tijelo jedno te ima mnogo udova, a  svi udovi tijela iako mnogi, jedno su tijelo - tako i Krist. 
\par 13 Ta u jednom Duhu svi smo u jedno tijelo kršteni, bilo Židovi, bilo Grci, bilo robovi, bilo slobodni. I svi smo jednim Duhom  napojeni. 
\par 14 Ta ni tijelo nije jedan ud, nego mnogi. 
\par 15 Rekne  li noga: "Nisam ruka, nisam od tijela", zar zbog toga nije od  tijela? 
\par 16 I rekne li uho: "Nisam oko, nisam od tijela", zar  zbog toga nije od tijela? 
\par 17 Kad bi sve tijelo bilo oko, gdje  bi bio sluh? Kad bi sve bilo sluh, gdje bi bio njuh? 
\par 18 A ovako, Bog je rasporedio udove, svaki od njih u tijelu, kako je htio. 
\par 19 Kad bi svi bili jedan ud, gdje bio bilo tijelo? 
\par 20 A ovako, mnogi udovi - jedno tijelo! 
\par 21 Ne može oko reći ruci: "Ne trebam  te", ili pak glava nogama: "Ne trebam vas." 
\par 22 Naprotiv, mnogo  su potrebniji udovi tijela koji izgledaju slabiji. 
\par 23 A udove  koje smatramo nečasnijima, okružujemo većom čašću. I s nepristojnima  se pristojnije postupa, 
\par 24 a pristojni toga ne trebaju. Nego, Bog je tako sastavio tijelo da je posljednjem udu dao izobilniju  čast 
\par 25 da ne bude razdora u tijelu, nego da se udovi jednako  brinu jedni za druge. 
\par 26 I ako trpi jedan ud, trpe zajedno svi  udovi; ako li se slavi jedan ud, raduju se zajedno svi udovi. 
\par 27 A vi ste tijelo Kristovo i, pojedinačno, udovi. 
\par 28 I  neke postavi Bog u Crkvi: prvo za apostole, drugo za proroke, treće za učitelje; onda čudesa, onda dari liječenja; zbrinjavanja, upravljanja, razni jezici. 
\par 29 Zar su svi apostoli? Zar svi  proroci? Zar svi učitelji? Zar svi čudotvorci? 
\par 30 Zar svi imaju  dare liječenja? Zar svi govore jezike? Zar svi tumače? 
\par 31 Čeznite za višim darima! A evo vam puta najizvrsnijega! 


\chapter{13}

\par 1 Kad bih sve jezike ljudske govorio i anđeoske, a ljubavi ne bih imao, bio bih mjed što ječi ili cimbal što zveči. 
\par 2 Kad bih imao dar prorokovanja i znao sva otajstva i sve spoznanje; i kad bih imao svu vjeru da bih i gore premještao, a ljubavi ne bih imao - ništa sam! 
\par 3 I kad bih razdao sav svoj imutak i kad bih predao tijelo svoje da se sažeže, a ljubavi ne bih imao - ništa mi ne bi koristilo. 
\par 4 Ljubav je velikodušna, dobrostiva je ljubav, ne zavidi, ljubav se ne hvasta, ne nadima se; 
\par 5 nije nepristojna, ne traži svoje, nije razdražljiva, ne pamti zlo; 
\par 6 ne raduje se nepravdi, a raduje se istini; 
\par 7 sve pokriva, sve vjeruje, svemu se nada, sve podnosi. 
\par 8 Ljubav nikad ne prestaje. Prorokovanja? Uminut će. Jezici? Umuknut će. Spoznanje? Uminut će. 
\par 9 Jer djelomično je naše spoznanje, i djelomično prorokovanje. 
\par 10 A kada dođe ono savršeno, uminut će ovo djelomično. 
\par 11 Kad bijah nejače, govorah kao nejače, mišljah kao nejače, rasuđivah kao nejače. A kad postadoh zreo čovjek, odbacih ono nejačko. 
\par 12 Doista, sada gledamo kroza zrcalo, u zagonetki, a tada - licem u lice! Sada spoznajem djelomično, a tada ću spoznati savršeno, kao što sam i spoznat! 
\par 13 A sada: ostaju vjera, ufanje i ljubav - to troje - ali najveća je među njima ljubav. 


\chapter{14}

\par 1 Težite za ljubavlju, čeznite za darima Duha, a najvećma da  prorokujete. 
\par 2 Jer tko govori drugim jezikom, ne govori ljudima  nego Bogu: nitko ga ne razumije jer Duhom govori stvari tajanstvene. 
\par 3 Tko pak prorokuje, ljudima govori: izgrađuje, hrabri, tješi. 
\par 4 Tko govori drugim jezikom, sam sebe izgrađuje, a tko prorokuje, Crkvu izgrađuje. 
\par 5 A htio bih da vi svi govorite drugim jezicima, ali većma da prorokujete. Jer veći je tko prorokuje, negoli  tko govori drugim jezicima, osim ako protumači Crkvi radi izgrađivanja. 
\par 6 A sada, braćo, kad bih došao k vama govoreći drugim jezicima, što bi vam koristilo kad vam ne bih priopćio bilo otkrivenje, bilo spoznanje, bilo proroštvo, bilo nauk? 
\par 7 Ako neživa glazbala, svirala ili citra, ne daju razgovijetna glasa, kako će se razabrati  što se to izvodi na svirali ili citri? 
\par 8 Ili ako trublja daje  nejasan glas, tko će se spremiti na boj? 
\par 9 Tako i vi, ako jezikom  ne budete jasno zborili, kako će se razabrati što se govori?  Govorit ćete u vjetar. 
\par 10 Toliko, recimo, ima na svijetu vrsta  glasova i - nijedan bez značenja. 
\par 11 Ako dakle ne znam značenja  glasa, bit ću sugovorniku tuđinac, a sugovornik tuđinac meni. 
\par 12 Tako i vi, budući da čeznete za darima Duha, nastojte njima  obilovati radi izgrađivanja Crkve. 
\par 13 Stoga tko govori drugim jezikom, neka se moli da može  protumačiti. 
\par 14 Jer ako se drugim jezikom molim, moj se duh  moli, ali um je moj neplodan. 
\par 15 Što dakle? Molit ću se duhom, molit ću se i umom; pjevat ću hvalospjeve duhom, ali pjevat  ću ih i umom. 
\par 16 Jer ako Boga blagoslivljaš duhom, kako će neupućen  reći "Amen" na tvoju zahvalnicu? Ne zna što govoriš. 
\par 17 Ti doduše  lijepo zahvaljuješ, ali se drugi ne izgrađuje. 
\par 18 Hvala Bogu, ja govorim drugim jezicima većma nego svi vi. 
\par 19 Ali draže  mi je u Crkvi reći pet riječi po svojoj pameti, da i druge poučim, negoli deset tisuća riječi drugim jezikom. 
\par 20 Braćo, ne budite djeca pameću, nego nejačad pakošću,  a zreli pameću! 
\par 21 U Zakonu je pisano: Drugim jezicima i  drugim usnama govorit ću ovomu narodu pa me ni tako neće  poslušati, govori Gospodin. 
\par 22 Tako drugi jezici nisu znak  vjernicima, nego nevjernicima; a prorokovanje vjernicima, ne  nevjernicima. 
\par 23 Ako se dakle skupi sva Crkva zajedno i svi  govore drugim jezicima, a uđu neupućeni ili nevjernici, neće  li reći da mahnitate? 
\par 24 Ako pak svi prorokuju, a uđe koji nevjernik  ili neupućen, sve ga prekorava, sve ga osuđuje. 
\par 25 Tajne se  njegova srca očituju te će pasti ničice i pokloniti se Bogu priznajući:  Zaista, Bog je u vama. 
\par 26 Što dakle braćo? Kad se skupite te poneki ima hvalospjev, poneki ima nauk, ima otkrivenje, ima jezik, ima tumačenje -  sve neka bude radi izgrađivanja. 
\par 27 Ako tko govori drugim jezikom  - dvojica, najviše trojica, i to jedan za drugim - jedan neka  tumači; 
\par 28 ako pak ne bi bilo tumača, neka šuti u Crkvi, neka  govori sam sebi i Bogu. 
\par 29 Od proroka pak neka govore dvojica  ili trojica, drugi neka rasuđuju. 
\par 30 Ali ako drugomu uza nj  bude što objavljeno, prvi neka šuti. 
\par 31 A možete jedan po jedan  svi prorokovati da svi budu poučeni i svi ohrabreni. 
\par 32 Proročki  su duhovi prorocima podložni 
\par 33 jer Bog nije Bog nesklada, nego  Bog mira. Kao u svim Crkvama svetih, žene na Sastancima neka šute. 
\par 34 Nije im dopušteno govoriti, nego neka budu podložne, kako  i Zakon govori. 
\par 35 Žele li što saznati, neka kod kuće pitaju  svoje muževe jer ružno je da žena govori na Sastanku. 
\par 36 Ili  zar je riječ Božja od vas proizašla, zar je samo k vama došla? 
\par 37 Smatra li tko da je prorok ili duhom obdaren, neka zna:  što vam pišem, Gospodnja je zapovijed. 
\par 38 Tko to ne prizna,  ne priznaje se. 
\par 39 Zato, braćo moja, težite prorokovati i ne  priječite da se govori drugim jezicima! 
\par 40 A sve neka bude dostojno  i uredno. 


\chapter{15}

\par 1 Dozivljem vam, braćo, u pamet evanđelje koje vam navijestih, koje primiste, u kome stojite, 
\par 2 po kojem se spasavate, ako  držite što sam vam navijestio; osim ako uzalud povjerovaste. 
\par 3 Doista, predadoh vam ponajprije što i primih: Krist umrije  za grijehe naše po Pismima; 
\par 4 bi pokopan i uskrišen treći dan  po Pismima; 
\par 5 ukaza se Kefi, zatim dvanaestorici. 
\par 6 Potom se  ukaza braći, kojih bijaše više od pet stotina zajedno; većina  ih još i sada živi, a neki usnuše. 
\par 7 Zatim se ukaza Jakovu,  onda svim apostolima. 
\par 8 Najposlije, kao nedonoščetu, ukaza se  i meni. 
\par 9 Da, ja sam najmanji među apostolima i nisam dostojan zvati  se apostolom jer sam progonio Crkvu Božju. 
\par 10 Ali milošću Božjom  jesam što jesam i njegova milost prema meni ne bijaše zaludna;  štoviše, trudio sam se više nego svi oni - ali ne ja, nego milost  Božja sa mnom. 
\par 11 Ili dakle ja ili oni: tako propovijedamo, tako vjerujete. 
\par 12 No ako se propovijeda da je Krist od mrtvih uskrsnuo, kako neki među vama govore da nema uskrsnuća mrtvih? 
\par 13 Ako  nema uskrsnuća mrtvih, ni Krist nije uskrsnuo. 
\par 14 Ako pak Krist  nije uskrsnuo, uzalud je doista propovijedanje naše, uzalud i  vjera vaša. 
\par 15 Zatekli bismo se i kao lažni svjedoci Božji što  posvjedočismo protiv Boga: da je uskrisio Krista, kojega nije  uskrisio, ako doista mrtvi ne uskršavaju. 
\par 16 Jer ako mrtvi ne  uskršavaju, ni Krist nije uskrsnuo. 
\par 17 A ako Krist nije uskrsnuo, uzaludna je vjera vaša, još ste u grijesima. 
\par 18 Onda i oni  koji usnuše u Kristu, propadoše. 
\par 19 Ako se samo u ovom životu  u Krista ufamo, najbjedniji smo od svih ljudi. 
\par 20 Ali sada: Krist uskrsnu od mrtvih, prvina usnulih! 
\par 21 Doista  po čovjeku smrt, po Čovjeku i uskrsnuće od mrtvih! 
\par 22 Jer kao  što u Adamu svi umiru, tako će i u Kristu svi biti oživljeni. 
\par 23 Ali svatko u svom redu: prvina Krist, a zatim koji su Kristovi, o njegovu Dolasku; 
\par 24 potom - svršetak, kad preda kraljevstvo  Bogu i Ocu, pošto obeskrijepi svako Vrhovništvo, svaku Vlast  i Silu. 
\par 25 Doista, on treba da kraljuje dok ne podloži sve  neprijatelje pod noge svoje. 
\par 26 Kao posljednji neprijatelj  bit će obeskrijepljena Smrt 
\par 27 jer sve podloži nogama njegovim.  A kad veli: Sve je podloženo, jasno - sve osim Onoga koji  mu je sve podložio. 
\par 28 I kad mu sve bude podloženo, tada će  se i on sam, Sin, podložiti Onomu koji je njemu sve podložio  da Bog bude sve u svemu. 
\par 29 Što onda čine oni koji se krste za mrtve? Ako mrtvi uopće  ne uskršavaju, što se krste za njih? 
\par 30 Što se onda i mi svaki  čas izlažemo pogiblima? 
\par 31 Dan za danom umirem, tako mi slave  vaše, braćo, koju imam u Kristu Isusu, Gospodinu našem! 
\par 32 Ako  sam se po ljudsku borio sa zvijerima u Efezu, kakva mi korist?  Ako mrtvi ne uskršavaju, jedimo i pijmo jer sutra nam je umrijeti. 
\par 33 Ne varajte se: "Zli razgovori kvare dobre običaje." 
\par 34 Otrijeznite  se kako valja i ne griješite jer neki, na sramotu vam kažem,  ne znaju za Boga. 
\par 35 Ali reći će netko: Kako uskršavaju mrtvi? I s kakvim  li će tijelom doći? 
\par 36 Bezumniče! Što siješ, ne oživljuje ako  ne umre. 
\par 37 I što siješ, ne siješ tijelo buduće, već golo zrno, pšenice - recimo - ili čega drugoga. 
\par 38 A Bog mu daje tijelo  kakvo hoće, i to svakom sjemenu svoje tijelo. 
\par 39 Nije svako tijelo isto tijelo; drugo je tijelo čovječje, drugo tijelo stoke, drugo tijelo ptičje, a drugo riblje. 
\par 40 Ima  tjelesa nebeskih i tjelesa zemaljskih, ali drugi je sjaj nebeskih, a drugi zemaljskih. 
\par 41 Drugi je sjaj sunca, drugi sjaj mjeseca  i drugi sjaj zvijezda; jer zvijezda se od zvijezde razlikuje  u sjaju. 
\par 42 Tako i uskrsnuće mrtvih: sije se u raspadljivosti, uskršava u neraspadljivosti; 
\par 43 sije se u sramoti, uskršava  u slavi; sije se u slabosti, uskršava u snazi; 
\par 44 sije se tijelo  naravno, uskršava tijelo duhovno. Ako ima tijelo naravno, ima i duhovno. 
\par 45 Tako je i pisano:  Prvi čovjek, Adam, postade živa duša, posljednji  Adam - duh životvorni. 
\par 46 Ali ne bi najprije duhovno, nego naravno  pa onda duhovno. 
\par 47 Prvi je čovjek od zemlje, zemljan; drugi  čovjek - s neba. 
\par 48 Kakav je zemljani takvi su i zemljani, a  kakav je nebeski takvi su i nebeski. 
\par 49 I kao što smo nosili  sliku zemljanoga, nosit ćemo i sliku nebeskoga. 
\par 50 A ovo, braćo, tvrdim: tijelo i krv ne mogu baštiniti  kraljevstva Božjega i raspadljivost ne baštini neraspadljivosti. 
\par 51 Evo otajstvo vam kazujem: svi doduše nećemo usnuti, ali svi  ćemo se izmijeniti. 
\par 52 Odjednom, u tren oka, na posljednju trublju  - jer zatrubit će - i mrtvi će uskrsnuti neraspadljivi i mi ćemo  se izmijeniti. 
\par 53 Jer ovo raspadljivo treba da se obuče u neraspadljivost  i ovo smrtno da se obuče u besmrtnost. 
\par 54 A kad se ovo raspadljivo obuče u neraspadljivost i ovo  smrtno obuče u besmrtnost, tada će se obistiniti riječ napisana:  Pobjeda iskapi smrt. 
\par 55 Gdje je, smrti, pobjeda tvoja? Gdje  je, smrti, žalac tvoj? 
\par 56 Žalac je smrti grijeh, snaga je  grijeha Zakon. 
\par 57 A hvala Bogu koji nam daje pobjedu po Gospodinu  našem Isusu Kristu! 
\par 58 Tako, braćo moja ljubljena, budite postojani, nepokolebljivi, i obilujte svagda u djelu Gospodnjem znajući da trud vaš nije  neplodan u Gospodinu. 



\chapter{16}

\par 1 U pogledu sabiranja za svete, i vi činite kako odredih crkvama  galacijskim. 
\par 2 Svakoga prvog dana u tjednu neka svaki od vas  kod sebe na stranu stavlja i skuplja što uzmogne da se ne sabire  istom kada dođem. 
\par 3 A kada dođem, poslat ću s preporučnicom  one koje odaberete da odnesu vašu ljubav u Jeruzalem. 
\par 4 Bude  li vrijedno da i ja pođem, poći će sa mnom. 
\par 5 A k vama ću doći kad prođem Makedoniju; Makedonijom ću  samo proći, 
\par 6 a kod vas ću se možda zadržati ili čak zimovati  da me otpratite kamo god pođem. 
\par 7 Ne bih vas doista htio tek  na prolazu vidjeti jer se nadam neko vrijeme proboraviti kod  vas, dopusti li Gospodin. 
\par 8 U Efezu ću ostati do Pedesetnice 
\par 9 jer vrata mi se otvoriše velika i uspješna, a protivnika mnogo. 
\par 10 Ako dođe Timotej, gledajte da bude kod vas bez bojazni jer  radi djelo Gospodnje kao i ja. 
\par 11 Neka ga dakle nitko ne prezre.  A ispratite ga u miru da dođe k meni jer ga s braćom iščekujem. 
\par 12 A što se tiče brata Apolona: mnogo sam ga nagovarao da ode  k vama s braćom. I nikako mu ne bijaše s voljom da sada dođe, no doći će kad mu bude zgodno. 
\par 13 Bdijte postojani u vjeri, muževni budite, čvrsti. 
\par 14 Sve  vaše neka bude u ljubavi! 
\par 15 Zaklinjem vas, braćo - znate dom Stefanin, da je prvina  Ahaje i da se posvetiše posluživanju svetih - 
\par 16 da se i vi  pokoravate takvima i svakomu tko surađuje i trudi se. 
\par 17 Radujem  se s dolaska Stefanina i Fortunatova i Ahajikova jer oni nadoknadiše  vašu nenazočnost: 
\par 18 umiriše duh moj i vaš. Cijenite dakle takve. 
\par 19 Pozdravljaju vas crkve azijske. Pozdravljaju vas mnogo  u Gospodinu Akvila i Priska zajedno s Crkvom u njihovu domu. 
\par 20 Pozdravljaju vas sva braća. Pozdravite jedni druge cjelovom  svetim. 
\par 21 Pozdrav mojom rukom, Pavlovom. 
\par 22 Ako tko ne ljubi Gospodina, neka bude proklet. Marana  tha! 
\par 23 Milost Gospodina Isusa s vama! 
\par 24 Ljubav moja sa svima vama u Kristu Isusu! 




\end{document}