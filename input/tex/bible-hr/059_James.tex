\begin{document}

\title{Jakovljeva}


\chapter{1}

\par 1 Jakov, sluga Boga i Gospodina Isusa Krista: dvanaestorma plemena  Raseljeništva pozdrav. 
\par 2 Pravom radošću smatrajte, braćo moja, kad upadnete u razne  kušnje 
\par 3 znajući da prokušanost vaše vjere rađa postojanošću. 
\par 4 Ali neka postojanost bude na djelu savršena da budete savršeni  i potpuni, bez ikakva nedostataka. 
\par 5 Nedostaje li komu od vas mudrosti, neka ište od Boga,  koji svima daje rado i bez negodovanja, i dat će mu se. 
\par 6 Ali  neka ište s vjerom, bez ikakva kolebanja. Jer kolebljivac je  sličan morskom valovlju, uzburkanu i gonjenu. 
\par 7 Neka takav ne  misli da će primiti što od Gospodina - 
\par 8 čovjek duše dvoumne, nepostojan na svim putovima svojim. 
\par 9 Neka se brat niska soja ponosi svojim uzvišenjem, 
\par 10 a  bogataš svojim poniženjem. Ta proći će kao cvijet trave: 
\par 11 sunce ogranu žarko te usahnu trava i cvijet njezin  uvenu; dražest mu lica propade. Tako će i bogataš na stazama  svojim usahnuti. 
\par 12 Blago čovjeku koji trpi kušnju: prokušan, primit će vijenac  života koji je Gospodin obećao onima što ga ljube. 
\par 13 Neka nitko u napasti ne rekne: "Bog me napastuje." Ta  Bog ne može biti napastovan na zlo, i ne napastuje nikoga. 
\par 14 Nego  svakoga napastuje njegova požuda koja ga privlači i mami. 
\par 15 Požuda  zatim, zatrudnjevši, rađa grijehom, a grijeh izvršen rađa smrću. 
\par 16 Ne varajte se, braćo moja ljubljena! 
\par 17 Svaki dobar dar, svaki savršen poklon odozgor je, silazi od Oca svjetlila u kome nema promjene ni sjene od mijene. 
\par 18 Po svom naumu on nas porodi riječju Istine da budemo prvina neka njegovih stvorova. 
\par 19 Znajte, braćo moja ljubljena! Svatko neka bude brz da  sluša, spor da govori, spor na srdžbu. 
\par 20 Jer srdžba čovjekova  ne čini pravde Božje. 
\par 21 Zato odložite svaku prljavštinu i preostalu  zloću i sa svom krotkošću primite usađenu riječ koja ima moć  spasiti duše vaše. 
\par 22 Budite vršitelji riječi, a ne samo slušatelji, zavaravajući sami sebe. 
\par 23 Jer ako je tko slušatelj riječi, a ne i izvršitelj, sličan je čovjeku koji motri svoje rođeno  lice u zrcalu: 
\par 24 promotri se, ode i odmah zaboravi kakav bijaše. 
\par 25 A koji se oglÄedÄa u savršenom zakonu slobode i uza nj prione, ne kao zaboravan slušatelj nego djelotvoran izvršitelj, blažen  će biti u svem djelovanju svome. 
\par 26 Smatra li se tko bogoljubnim, a ne obuzdava svoga jezika, nego zavarava srce svoje, isprazna je njegova bogoljubnost. 
\par 27 Bogoljubnost čista i neokaljana jest: zauzimati se za sirote  i udovice u njihovoj nevolji, čuvati se neokaljanim od ovoga  svijeta. 


\chapter{2}

\par 1 Braćo moja, vjeru Gospodina našega Isusa Krista slavnoga ne  miješajte s pristranošću! 
\par 2 Dođe li na vaš sastanak čovjek sa  zlatnim prstenjem, u sjajnoj odjeći, a dođe i siromah u bijednoj  odjeći 
\par 3 i vi se zagledate u onoga što nosi sjajnu odjeću te  reknete: "Ti lijepo ovdje sjedni!", a siromahu reknete: "Ti stani  - ili sjedni - ondje, podno podnožja moga!", 
\par 4 niste li u sebi  pristrano sudili te postali suci što naopako sude? 
\par 5 Čujte, braćo moja ljubljena: nije li Bog one koji su svijetu  siromašni izabrao da budu bogataši u vjeri i baštinici Kraljevstva  što ga je obećao onima koji ga ljube? 
\par 6 A vi prezreste siromaha!  Ne tlače li vas upravo bogataši? Ne vuku li vas baš oni na sudove? 
\par 7 Ne psuju li oni lijepo Ime na vas zazvano? 
\par 8 Ako doista izvršujete  kraljevski zakon po Pismu: Ljubi bližnjega svoga kao sebe  samoga, dobro činite; 
\par 9 ako li ste pristrani, grijeh činite  i Zakon vas osuđuje kao prijestupnike. 
\par 10 Ta tko sav Zakon uščuva, a u jednome samo posrne, postao je krivac svega. 
\par 11 Jer tko  reče: Ne čini preljuba, reče i: Ne ubij. Ako dakle  i ne činiš preljuba, a ubiješ, postao si prijestupnik Zakona. 
\par 12 Tako govorite i tako činite kao oni koji imaju biti suđeni  po zakonu slobode. 
\par 13 Jer nemilosrdan je sud onomu tko ne čini  milosrđa; a milosrđe likuje nad sudom. 
\par 14 Što koristi, braćo moja, ako tko rekne da ima vjeru,  a djela nema? Može li ga vjera spasiti? 
\par 15 Ako su koji brat  ili sestra goli i bez hrane svagdanje 
\par 16 pa im tkogod od vas  rekne: "Hajdete u miru, grijte se i sitite", a ne dadnete im  što je potrebno za tijelo, koja korist? 
\par 17 Tako i vjera: ako  nema djela, mrtva je u sebi. 
\par 18 Inače, mogao bi tko reći: "Ti imaš vjeru, a ja imam djela.  Pokaži mi svoju vjeru bez djela, a ja ću tebi djelima pokazati  svoju vjeru. 
\par 19 Ti vjeruješ da je jedan Bog? Dobro činiš! I  đavli vjeruju, i dršću." 
\par 20 Hoćeš li spoznati, šuplja glavo, da je vjera bez djela  jalova? 
\par 21 Zar se Abraham, otac naš, ne opravda djelima, kad  na žrtvenik prinese Izaka, sina svoga? 
\par 22 Vidiš: vjera je surađivala  s djelima njegovim i djelima se vjera usavršila 
\par 23 te se ispunilo  Pismo koje veli: Povjerova Abraham Bogu i uračuna mu se u  pravednost pa prijatelj Božji posta. 
\par 24 Gledajte: čovjek  se opravdava djelima, a ne samom vjerom. 
\par 25 Ne opravda li se  slično, djelima, i Rahaba bludnica kad primi glasnike i drugim  ih putom izvede? 
\par 26 Jer kao što je tijelo bez duha mrtvo, tako  je i vjera bez djela mrtva. 


\chapter{3}

\par 1 Neka vas, braćo moja, ne bude mnogo učitelja! Ta znate: bit  ćemo strože suđeni. 
\par 2 Doista, svi mnogo griješimo. Ako tko u  govoru ne griješi, savršen je čovjek, vrstan zauzdati i cijelo  tijelo. 
\par 3 Ubacimo li uzde u usta konjima da ih sebi upokorimo, upravljamo i cijelim tijelom njihovim. 
\par 4 Evo i lađa: tolike  su i silni ih vjetrovi gone, a neznatno ih kormilo upravlja kamo  kormilarova volja hoće. 
\par 5 Tako i jezik: malen je ud, a velikim  se može ponositi. Evo: kolicna vatra koliku šumu zapali! 
\par 6 I jezik je vatra, svijet nepravda jezik je među našim udovima, kalja cijelo tijelo  te, zapaljen od pakla, zapaljuje kotač života. 
\par 7 Doista, sav  rod zvijeri i ptica, gmazova i morskih životinja dade se ukrotiti, i rod ih je ljudski ukrotio, 
\par 8 a jezik - zlo nemirno, pun otrova  smrtonosnog - nitko od ljudi ne može ukrotiti. 
\par 9 Njime blagoslivljamo Gospodina i Oca, njime i proklinjemo  ljude na sliku Božju stvorene: 
\par 10 iz istih usta izlazi blagoslov  i prokletstvo. Ne smije se, braćo moja, tako događati! 
\par 11 Zar  vrelo na isti otvor šiklja slatko i gorko? 
\par 12 Može li, braćo  moja, smokva roditi maslinama ili trs smokvama? Ni slan izvor  ne može dati slatke vode. 
\par 13 Je li tko mudar i razborit među vama? Neka dobrim življenjem  pokaže svoja djela u mudroj blagosti. 
\par 14 Ako u srcu imate gorku  zavist i svadljivost, ne uznosite se i ne lažite protiv istine! 
\par 15 Nije to mudrost koja odozgor silazi, nego zemaljska, ljudska, đavolska. 
\par 16 Ta gdje je zavist i svadljivost, ondje je nered  i svako zlo djelo. 
\par 17 A mudrost odozgor ponajprije čista je, zatim mirotvorna, milostiva, poučljiva, puna milosrđa i dobrih  plodova, postojana, nehinjena. 
\par 18 Plod se pak pravednosti u  miru sije onima koji tvore mir. 


\chapter{4}

\par 1 Odakle ratovi, odakle borbe među vama? Zar ne odavde: od pohota  što vojuju u udovima vašim? 
\par 2 Žudite, a nemate; ubijate i hlepite, a ne možete postići; borite se i ratujete. Nemate jer ne ištete. 
\par 3 Ištete, a ne primate jer rđavo ištete: da u pohotama svojim  potratite. 
\par 4 Preljubnici! Ne znate li da je prijateljstvo sa svijetom  neprijateljstvo prema Bogu? Tko god dakle hoće da bude prijatelj  svijeta, promeće se u neprijatelja Božjega. 
\par 5 Ili mislite da  Pismo uzalud veli: Ljubomorno čezne za duhom što ga nastani u  nama? 
\par 6 A daje on i veću milost. Zato govori: Bog se oholima protivi, a poniznima daje milost. 
\par 7 Podložite se dakle Bogu! Oduprite se đavlu i pobjeći će od vas! 
\par 8 Približite se Bogu i on će se približiti vama! Očistite ruke, grešnici! Očistite srca, dvoličnjaci! 
\par 9 Zakukajte, protužite, proplačite! Smijeh vaš nek se u plač obrati i radost u žalost! 
\par 10 Ponizite se pred Gospodinom i on će vas uzvisiti! 
\par 11 Ne ogovarajte, braćo, jedni druge! Tko ogovara ili sudi  brata svoga, ogovara i sudi Zakon. A sudiš li Zakon, nisi vršitelj  nego sudac Zakona. 
\par 12 Jedan je Zakonodavac i Sudac: Onaj koji  može spasiti i pogubiti. A tko si ti da sudiš bližnjega? 
\par 13 De sada, vi što govorite: "Danas ili sutra otići ćemo  u taj i taj grad, provesti ondje godinu, trgovati i zaraditi", 
\par 14 a ne znate što će sutra biti. Ta što je vaš život? Dašak  ste što se načas pojavi i zatim nestane! 
\par 15 Umjesto da govorite:  "Htjedne li Gospodin, živjet ćemo i učiniti ovo ili ono", 
\par 16 vi  se razmećete svojim hvastanjima! Svako je takvo hvastanje opako. 
\par 17 Znati dakle dobro činiti, a ne činiti - grijeh je. 



\chapter{5}

\par 1 De sada, bogataši, proplačite i zakukajte zbog nevolja koje  će vas zadesiti! 
\par 2 Bogatstvo vam istrunu, haljine vaše postadoše  hrana moljcima, 
\par 3 zlato vam i srebro zarđa i rđa će njihova  biti svjedočanstvo protiv vas te će kao vatra izjesti tijela  vaša! Zgrnuste blago u posljednje dane! 
\par 4 Evo: plaća kosaca  vaših njiva - koju im uskratiste - viče i vapaji žetelaca dopriješe  do ušiju Gospoda nad Vojskama. 
\par 5 Raskošno ste na zemlji i razvratno  živjeli, utoviste srca svoja za dan klanja! 
\par 6 Osudiste i ubiste  pravednika: on vam se ne suprotstavlja! 
\par 7 Strpite se dakle, braćo, do Dolaska Gospodnjega! Evo:  ratar iščekuje dragocjeni urod zemlje, strpljiv je s njime dok  ne dobije kišu ranu i kasnu. 
\par 8 Strpite se i vi, očvrsnite  srca jer se Dolazak Gospodnji približio! 
\par 9 Ne tužite se jedni  na druge da ne budete osuđeni! Evo: sudac stoji pred vratima! 
\par 10 Za uzor strpljivosti i podnošenja zala uzmite, braćo, proroke  koji su govorili u ime Gospodnje. 
\par 11 Eto: blaženima nazivamo  one koji ustrajaše. Za postojanost Jobovu čuste i nakanu Gospodnju  vidjeste jer milostiv je Gospodin i milosrdan! 
\par 12 Prije svega, braćo moja, ne zaklinjite se ni nebom ni  zemljom, ni ikojom drugom zakletvom. Vaše "da" neka bude "da", i "ne" - "ne", da ne padnete pod sud. 
\par 13 Pati li tko među vama? Neka moli! Je li tko radostan?  Neka pjeva hvalospjeve! 
\par 14 Boluje li tko među vama? Neka dozove  starješine Crkve! Oni neka mole nad njim mažući ga uljem u ime  Gospodnje 
\par 15 pa će molitva vjere spasiti nemoćnika; Gospodin  će ga podići, i ako je sagriješio, oprostit će mu se. 
\par 16 Ispovijedajte  dakle jedni drugima grijehe i molite jedni za druge da ozdravite!  Mnogo može žarka molitva pravednikova. 
\par 17 Ilija bijaše čovjek  baš kao i mi; usrdno se pomoli da ne bude kiše i kiše nije bilo  na zemlji tri godine i šest mjeseci. 
\par 18 Zatim se ponovno pomoli  te nebo dade kišu i zemlja iznese urod svoj. 
\par 19 Braćo moja, odluta li tko od vas od istine pa ga tkogod  vrati, 
\par 20 znajte: tko vrati grešnika s lutalačkog puta njegova, spasit će dušu njegovu od smrti i pokriti mnoštvo grijeha. 




\end{document}