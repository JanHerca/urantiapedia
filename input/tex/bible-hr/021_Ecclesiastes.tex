\begin{document}

\title{Propovjednik}


\chapter{1}

\par 1 Misli Propovjednika, Davidova sina, kralja u Jeruzalemu. 
\par 2 Ispraznost nad ispraznošću, veli Propovjednik, ispraznost  nad ispraznošću, sve je ispraznost! 
\par 3 Kakva je korist čovjeku od svega truda njegova kojim se  trudi pod suncem? 
\par 4 Jedan naraštaj odlazi, drugi dolazi, a zemlja uvijek ostaje. 
\par 5 Sunce izlazi, sunce zalazi i onda hiti svojem mjestu odakle  izlazi. 
\par 6 Vjetar puše na jug i okreće se na sjever, kovitla sad  ovamo sad onamo i vraća se u novom vrtlogu. 
\par 7 Sve rijeke teku u more i more se ne prepunja; odakle teku  rijeke, onamo se vraćaju da ponovo počnu svoj tok. 
\par 8 Sve je mučno. Nitko ne može reći da se oči nisu do sita  nagledale i uši dovoljno naslušale. 
\par 9 Što je bilo, opet će biti, i što se činilo, opet će se  činiti, i nema ništa novo pod suncem. 
\par 10 Ima li išta o čemu bi se moglo reći: "Gle, ovo je novo!"  Sve je već davno prije nas postojalo. 
\par 11 Samo, od prošlosti ne ostade ni spomena, kao što ni u  budućnosti neće biti sjećanja na ono što će poslije doći. 
\par 12 Ja, Propovjednik, bijah kralj nad Izraelom u Jeruzalemu. 
\par 13 I trudih se da mudrošću istražim i dokučim sve što biva  pod nebom; o, kako mučnu zadaću zadade Bog sinovima ljudskim. 
\par 14 Vidjeh sve što se čini pod suncem: kakve li ispraznosti  i puste tlapnje! 
\par 15 Što je krivo, ne može se ispraviti; čega nema, izbrojiti se ne može. 
\par 16 Rekoh onda sam sebi: "Gle, stekao sam veću mudrost nego  bilo tko od mojih prethodnika u Jeruzalemu. Duh moj sabrao je  golemu mudrost i znanje." 
\par 17 Mudrost pomnjivo proučih, a tako  i glupost i ludost, ali sam spoznao da je to pusta tlapnja. 
\par 18 Mnogo mudrosti - mnogo jada; što više znanja, to više boli. 


\chapter{2}

\par 1 Tad rekoh u srcu svom: "Daj da okušam užitak i vidim što naslada  pruža" - ali gle: i to je ispraznost! 
\par 2 O smijehu rekoh: "Ludost je"; o užitku: "Čemu valja?" 
\par 3 Odlučih tijelo krijepiti vinom, a srce posvetiti mudrosti;  poželjeh prigrliti i ludost, kako bih spoznao usrećava li ljude  ono što pod nebesima čine za izbrojenih dana svojega života. 
\par 4 Učinih velika djela: sazidah sebi palače, zasadih vinograde; 
\par 5 uredih perivoje i voćnjake, nasadih u njima voćke svakojake. 
\par 6 Načinih jezera da plodna sadišta natapam. 
\par 7 Nakupovah robova i robinja, imadoh i sluge domaće, a tako  i stada krupne i sitne stoke više no itko prije mene u Jeruzalemu. 
\par 8 Nagomilah srebro i zlato i blago kraljeva i pokrajina, nabavih pjevače i pjevačice i svakoga raskošja ljudskog, sve  škrinju na škrinji. 
\par 9 I postadoh tako velik, veći no bilo tko prije mene u Jeruzalemu;  a nije me ni mudrost moja ostavila. 
\par 10 I što god su mi oči poželjele, nisam im uskratio, niti  branih srcu svojemu kakva veselja, već se srce moje veselilo  svakom trudu mojem, i takva bi nagrada svakom mojem naporu. 
\par 11 A onda razmotrih sva svoja djela, sve napore što uložih  da do njih dođem - i gle, sve je to opet ispraznost i pusta tlapnja!  I ništa nema valjano pod suncem. 
\par 12 Okrenuh zatim misao svoju mudrosti, gluposti i ludosti.  Što, na priliku, čini kraljev nasljednik? Ono što je već učinjeno. 
\par 13 I spoznadoh da je bolja mudrost od ludosti, kao što je  svjetlost bolja od tame. 
\par 14 Mudracu su oči u glavi, a bezumnik luta u tami. Ali također znam da obojicu stiže ista kob. 
\par 15 Zato rekoh u sebi: "Kakva je sudbina luđaku, takva je  i meni. Čemu onda žudjeti za mudrošću?" I rekoh u srcu: "I to  je ispraznost!" 
\par 16 Jer trajna spomena nema ni mudru ni bezumniku: obojicu  će poslije nekog vremena prekriti zaborav! I, jao, mudrac mora  umrijeti kao i bezumnik. 
\par 17 Omrznuh život, jer mi se učini mučnim sve što se zbiva  pod suncem: sve je ispraznost i pusta tlapnja. 
\par 18 Zamrzih sve za što sam se pod suncem trudio i što sad  ostavljam svome nasljedniku. 
\par 19 Tko zna hoće li on biti mudar ili lud? Pa ipak on će  biti gospodar sve moje muke u koju uložih sav svoj napor i mudrost  pod suncem. I to je ispraznost. 
\par 20 I stao sam srcem očajavati zbog velikog napora kojim  sam se trudio pod suncem. 
\par 21 Jer čovjek se trudi mudro i umješno i uspješno, pa sve  to mora ostaviti u baštinu drugomu koji se oko toga nije uopće  trudio. I to je ispraznost i velika nevolja. 
\par 22 Jer što on dobiva za sav svoj napor i trud koji je pod  suncem podnio? 
\par 23 Jer svi su njegovi dani doista mukotrpni, poslovi mu  puni brige; čak ni noću ne miruje srce njegovo. I to je ispraznost. 
\par 24 Nema čovjeku druge sreće već jesti i piti i biti zadovoljan  svojim poslom. I to je, vidim, dar Božje ruke. 
\par 25 Jer tko bi mogao jesti, tko li nezadovoljan biti, osim  po njemu. 
\par 26 Mudrost, spoznaju, radost on daruje onom tko mu  je po volji, a grešniku nameće zadaću da sabira i skuplja za  onoga tko je po volji Bogu. I to je ispraznost i pusta tlapnja. 


\chapter{3}

\par 1 Sve ima svoje doba i svaki posao pod nebom svoje vrijeme. 
\par 2 Vrijeme rađanja i vrijeme umiranja; vrijeme sađenja i vrijeme čupanja posađenog. 
\par 3 Vrijeme ubijanja i vrijeme liječenja; vrijeme rušenja i vrijeme građenja. 
\par 4 Vrijeme plača i vrijeme smijeha; vrijeme tugovanja i vrijeme plesanja. 
\par 5 Vrijeme bacanja kamenja i vrijeme sabiranja kamenja; vrijeme grljenja i vrijeme kad se ostavlja grljenje. 
\par 6 Vrijeme traženja i vrijeme gubljenja; vrijeme čuvanja i vrijeme odbacivanja. 
\par 7 Vrijeme deranja i vrijeme š§ijenja; vrijeme šutnje i vrijeme govorenja. 
\par 8 Vrijeme ljubljenja i vrijeme mržnje; vrijeme rata i vrijeme mira. 
\par 9 Koja je posleniku korist od njegovih napora? 
\par 10 Razmišljam o mučnoj zadaći što je Bog zadade sinovima  ljudskim. 
\par 11 Sve što on čini prikladno je u svoje vrijeme; ali iako  je dopustio čovjeku uvid u vjekove, čovjek ne može dokučiti djela  koja Boga čini od početka do kraja. 
\par 12 Znam da nije druge sreće čovjeku osim da se veseli i  čini dobro za svojega života. 
\par 13 I kad čovjek jede i pije i uživa u svojem radu, i to  je Božji dar. 
\par 14 I znam da sve što Bog čini, čini za stalno. Tome se ništa  dodati ne može niti mu se može oduzeti; a Bog čini tako da ga  se boje. 
\par 15 Što biva, već bijaše, i što će biti, već je bilo; a Bog  obnavlja što je prohujalo. 
\par 16 Još vidim kako pod suncem umjesto pravice vlada nepravda  i umjesto pravednika zločinac. 
\par 17 Zato rekoh u sebi: "Bog će suditi i pravedniku i zločincu, jer ovdje ima vrijeme za svaku namjeru i čin." 
\par 18 Još rekoh u sebi: "Ljudi se ponašaju tako da Bog može  pokazati kakvi su uistinu, da su jedni drugima poput zvijeri." 
\par 19 Jer zaista, kob ljudi i zvijeri jedna je te ista. Kako  ginu oni, tako ginu i one; i dišu jednakim dahom, i čovjek ničim  ne nadmašuje zvijer, jer sve je ispraznost. 
\par 20 I jedni i drugi odlaze na isto mjesto; svi su postali  od praha i u prah se vraćaju. 
\par 21 Tko zna da li dah ljudski uzlazi gore, a dah zvijeri  silazi dolje k zemlji? 
\par 22 Uviđam da čovjeku druge sreće nema osim radosti u svome  djelu, jer to je ljudska sudbina. A tko će ga dovesti do toga  da dozna što će biti poslije njega? 


\chapter{4}

\par 1 Opet stadoh promatrati sva nasilja koja se čine pod suncem, i gle, suze potlačenih, i nikog nema da ih utre; i nasilje iz  tlačiteljske ruke, a zaštitnika niotkuda. 
\par 2 Zato sretnima smatram mrtve koji su već pomrli; sretniji  su od živih što još žive. 
\par 3 A od obojih je sretniji onaj koji  još nije postao, koji nije vidio zlih djela što se čine pod suncem. 
\par 4 Nadalje iskusih da svaki napor i svaki uspjeh pribavlja  čovjeku zavist njegova bližnjeg. I to je ispraznost i pusta tlapnja. 
\par 5 Bezumnik prekriži ruke i izjeda sebe. 
\par 6 Bolja je puna šaka u miru nego obje pregršti mučna rada i puste tlapnje. 
\par 7 I još jednu opazih ispraznost pod suncem: 
\par 8 Čovjek potpun samac - bez sina, bez brata, i opet nema  kraja njegovu trudu; oči mu se ne mogu nasititi blaga; a ne misli:  za koga se mučim i uskraćujem dobro sebi? I to je ispraznost  i zla briga. 
\par 9 Bolje je dvojici nego jednome, jer imaju bolju plaću za  svoj trud. 
\par 10 Padne li jedan, drugi će ga podići; a teško jednomu!  Ako padne, nema nikoga da ga podigne. 
\par 11 Pa ako se i spava udvoje, toplije je; a kako će se samac  zagrijati? 
\par 12 I ako tko udari na jednoga, dvojica će mu se oprijeti;  i trostruko se uže ne kida brzo. 
\par 13 Bolji je mladić siromašan a mudar nego kralj star a lud, koji više ne zna za savjet. 
\par 14 Jer mladić može izići iz tamnice i postati kraljem, iako  se rodio kao prosjak u svom kraljevstvu. 
\par 15 Opazih kako svi koji žive i hode pod suncem pristaju  uz mladića, uz nastupnika koji ga naslijedi. 
\par 16 On stupa na čelo bezbrojnih podanika i kasniji se naraštaji  ne mogahu zbog njega radovati. I to je zacijelo ispraznost i  pusta tlapnja. 


\chapter{5}

\par 1 (4:17) Kad odlaziš u Božji dom, pazi na korake svoje. Priđi  da mogneš čuti - žrtva je valjanija nego prinos luđaka, jer oni  i ne znaju da čine zlo. 
\par 2 (5:1) Ne nagli ustima svojim i ne žuri se s riječima pred Bogom,  jer je Bog na nebu, a ti si na zemlji; zato štedi svoje riječi. 
\par 3 (5:2) San dolazi od mnogih briga, a lud govor od mnoštva riječi. 
\par 4 (5:3) Kad zavjetuješ štogod Bogu, odmah to izvrši, jer njemu  nisu mili bezumnici. Zato ispuni svaki svoj zavjet. 
\par 5 (5:4) Bolje je ne zavjetovati, nego zavjetovati a ne izvršiti  zavjeta. 
\par 6 (5:5) Ne daj ustima svojim da te navode na grijeh i ne reci  kasnije pred anđelom da je bilo nehotice. Zašto pružati Bogu  priliku da se srdi na riječ tvoju i uništi djelo tvojih ruku? 
\par 7 (5:6) Koliko sanja, toliko i ispraznosti; mnogo riječi - isprazna tlapnja. Zato boj se Boga. 
\par 8 (5:7) Ako vidiš gdje tlače siromaha i gaze pravo i pravicu u  zemlji, ne čudi se tomu, jer nad visokim straži viši, a nad njim  najviši. 
\par 9 (5:8) Korist zemlje je nada sve; i kralj ovisi o zemljištu. 
\par 10 (5:9) Tko novce ljubi, nikad ih dosta nema; tko bogatstvo ljubi, nikad mu dosta probitka. I to je ispraznost. 
\par 11 (5:10) Gdje je mnogo bogatstva, mnogo je i gotovana, pa kakva je korist od toga gospodaru, osim što ga očima gleda? 
\par 12 (5:11) Sladak je dan radiši, jeo malo ili mnogo, dok bogatstvo  ne da bogatašu zaspati. 
\par 13 (5:12) I vidjeh teško zlo pod suncem: skupljeno blago što je  na propast svojemu vlasniku. 
\par 14 (5:13) Jer zlom nezgodom propadne takvo bogatstvo te sinu što  mu se rodi ne ostane ništa. 
\par 15 (5:14) Gol je izašao iz utrobe majke svoje i tako će gol i otići  kakav je i došao; ništa nema od svega svojeg truda da ponese. 
\par 16 (5:15) I to je teško zlo što tako odlazi kako je i došao; pa  kakva mu je korist što se u vjetar mučio. 
\par 17 (5:16) Sve svoje dane živi u tami, nevolji, brizi, bolesti i  srdžbi. 
\par 18 (5:17) Ovo, stoga, zaključujem: prava je sreća čovjeku jesti  i piti i biti zadovoljan sa svim svojim trudom kojim se muči  pod suncem za kratka vijeka koji mu je dao Bog, jer takva mu  je sudbina dosuđena. 
\par 19 (5:18) Pa ako je čovjeku Bog dao bogatstvo i imanje da ih uživa  i bude zadovoljan svojim djelom - i to je dar od Boga. 
\par 20 (5:19) Jer tada barem ne misli mnogo na dane svog života, kad  mu Bog daje da mu se srce veseli. 


\chapter{6}

\par 1 I vidjeh: ima još jedno zlo pod suncem i teško tišti ljude. 
\par 2 Nekomu Bog udijeli bogatstvo i blago i počasti te ima  sve što mu duša poželi, ali mu ne udijeli da to i uživa, nego  uživa tuđinac. To je ispraznost i grdna nevolja. 
\par 3 I velim: bolje je nedonošče od onoga koji bi rodio stotinu  djece i živio mnogo godina, a sam se ne bi naužio dobra niti  bi imao pogreba; 
\par 4 jer je nedonošče uzalud došlo i u tamu otišlo i ime mu je tamom pokriveno; 
\par 5 sunca čak ne vidje niti spozna - a spokojnije je od onoga. 
\par 6 Pa kad bi takav živio i dvije tisuće godina, a svojeg  dobra ne bi uživao, zar ne odlaze obojica jednako na isto mjesto? 
\par 7 Čovjek se trudi samo da bi jeo, a želudac njegov nikad da se nasiti. 
\par 8 Jer po čemu je mudrac bolji od luđaka i što reći o siromahu  koji se umije držati pred ljudima? 
\par 9 Bolje je očima vidjeti nego duhom lutati. I to je ispraznost  i pusta tlapnja. 
\par 10 Što je već bilo, ime ima; i zna se što je čovjek; i on  se ne može parbiti s jačim od sebe. 
\par 11 Što više riječi, to veća ispraznost svega, i koja je  od toga korist čovjeku? 
\par 12 Tko zna što je dobro čovjeku u životu njegovu, za ono  malo dana koje tako isprazno živi, koji mu prolaze kao sjena?  Tko će kazati čovjeku što će biti poslije njega pod suncem? 


\chapter{7}

\par 1 Bolji je dobar glas nego skupocjeno ulje, i smrtni dan nego dan rođenja. 
\par 2 Bolje je ići u kuću gdje je žalost nego u kuću gdje je gozba, jer ondje je kraj svakoga čovjeka, i tko je živ, nek' primi k srcu! 
\par 3 Bolji je jad nego smijeh, jer pod žalosnim licem srce je radosno. 
\par 4 Srce je mudrih ljudi u kući žalosti, a srce bezumnih u kući veselja. 
\par 5 Bolje je poslušati ukor mudra čovjeka negoli slušati hvalospjev luđaka. 
\par 6 Jer kao prasak trnja ispod kotla, takav je smijeh luđaka, i to je ispraznost. 
\par 7 Jer smijeh od mudraca čini luđaka i veselje kvari srce. 
\par 8 Bolji je svršetak stvari nego njezin početak i bolja je strpljivost od oholosti. 
\par 9 Ne nagli u srdžbu, jer srdžba počiva u srcu luđaka. 
\par 10 Ne pitaj zašto su negdašnja vremena bila bolja od ovih, jer to nije mudro pitanje. 
\par 11 Mudrost je dragocjena baština i probitak onima na koje  sunce sja. 
\par 12 Jer kao što je novac zaštita, tako je i mudrost; a prednost  je mudrosti u tome što izbavlja onoga tko je ima. 
\par 13 Pogledaj djela Božja; tko može ispraviti što je on iskrivio? 
\par 14 U sretan dan uživaj sreću, a u zao dan razmišljaj: Bog  je stvorio jedno kao i drugo - da čovjek ne otkrije ništa od  svoje budućnosti. 
\par 15 Svašta vidjeh u svojemu ništavnom životu: pravednik propada  unatoč svojoj pravednosti, a bezbožnik i dalje živi unatoč svojoj  bezbožnosti. 
\par 16 Ne budi prepravedan i ne budi premudar; zašto da se uništavaš? 
\par 17 Ne budi preopak i ne budi lud; zašto bi umro prije vremena? 
\par 18 Dobro je da držiš jedno, ali ni drugo ne puštaj iz ruke, jer tko se boji Boga, izbavlja se od svega. 
\par 19 Mudrost mudraca veću moć daje gradu nego deset mogućnika. 
\par 20 Na zemlji nema pravednika koji, čineći dobro, ne bi nikad  sagriješio. 
\par 21 I još jedno: nemoj se obazirati na govorkanje; čut ćeš  možda da te sluga tvoj proklinjao; 
\par 22 a zna tvoje srce kako  si i ti često druge proklinjao. 
\par 23 Sve sam to mudrošću iskušao. Mislio sam da sam mudar, ali mi je mudrost bila nedokučiva. 
\par 24 Ono što jest, daleko je i duboko, tako duboko - tko da  i pronađe? 
\par 25 I još jednom pokušah istražiti i shvatiti mudrost i smisao, da spoznam opačinu kao ludost, a ludost kao bezumlje. 
\par 26 Otkrih da ima nešto gorče od smrti - žena, ona je zamka, srce joj je mreža, a ruke okovi; tko je Bogu drag, izmiče joj, a grešnik je njezin sužanj. 
\par 27 Eto, to sam sve u svemu otkrio, veli Propovjednik. 
\par 28 I još sam tražio, ali bez uspjeha. Nađoh čovjeka - jednog od tisuću, a žene ne nađoh među svima nijedne. 
\par 29 Otkrih ovo: Bog stvori čovjeka jednostavnim, a on snuje  nebrojene spletke. 


\chapter{8}

\par 1 Tko je kao mudrac? Tko još umije tumačiti stvari? Mudrost čovjeku razvedruje lice i mijenja njegov namršteni lik. 
\par 2 Zato velim: slušaj kraljevu zapovijed zbog Božje zakletve. 
\par 3 Ne nagli da je prekršiš: ne budi tvrdoglav kad razlog  nije dobar, jer on čini kako mu odgovara. 
\par 4 Jer kraljeva je riječ najjača, i tko ga smije pitati:  "Što činiš?" 
\par 5 Tko se drži zapovijedi, ne poznaje nevolju, i mudrac zna  za vrijeme i sud. 
\par 6 Jer postoji vrijeme i sud za sve, i čovjeka veoma tereti  nedjelo njegovo 
\par 7 jer on ne zna što će biti; a tko mu može kazati kad će  što biti? 
\par 8 Vjetar nitko ne može svladati, niti gospodariti nad danom  smrtnim, niti ima odgode u ratu; niti opačina izbavlja onoga  koji je čini. 
\par 9 Sve ovo vidjeh pazeći na sve što se čini pod suncem, kad  čovjek vlada nad čovjekom na njegovu nesreću. 
\par 10 Dalje vidjeh kako opake nose na groblje, i ljudi iz svetog  mjesta izlaze da ih slave zbog toga što su tako činili. I to  je ispraznost. 
\par 11 Kad nema brze osude za zlo djelo, ljudsko je srce sklono  činiti zlo. 
\par 12 I grešnik koji čini zlo i sto puta, dugo živi. Ja ipak  znam da će biti sretni oni koji se boje Boga jer ga se boje. 
\par 13 Ali opak čovjek neće biti sretan i neće produljivati  svoje dane ni kao sjena jer se ne boji Boga. 
\par 14 Ali je na zemlji ispraznost te pravednike stiže sudbina  opakih, a opake sudbina pravednika. Velim: i to je ispraznost. 
\par 15 Zato slavim veselje, jer nema čovjeku sreće pod suncem  nego u jelu, pilu i nasladi. I to neka ga prati u njegovoj muci  za života koji mu Bog dade pod suncem. 
\par 16 Poslije svih napora da dokučim mudrost, pokušah spoznati  što se radi na zemlji. Uistinu, čovjek ne nalazi spokojstva ni  danju ni noću. 
\par 17 Promatram cjelokupno djelo Božje: i odista - nitko ne  može dokučiti ono što se zbiva pod suncem. Jer ma koliko se čovjek  trudio da otkrije, nikad ne može otkriti. Pa ni mudrac to ne  može otkriti, iako misli da zna. 


\chapter{9}

\par 1 Razmišljah o svemu tome i shvatih kako su i pravednici i mudraci, sa djelima svojim, u Božjoj ruci; i čovjek ne razumije ni ljubavi  ni mržnje, i njemu su obje ispraznost. 
\par 2 Svima je ista kob, pravednomu kao i opakom, čistomu i  nečistomu, onomu koji žrtvuje kao i onomu koji ne žrtvuje; jednako  dobru kao i grešniku, onomu koji se zaklinje kao i onomu koji  se boji zakletve. 
\par 3 Najgore je od svega što biva pod suncem ovo: ista je kob  svima, ljudsko je srce puno zla, ludost je u srcima ljudi dok  žive, a potom se pridružuju mrtvima. 
\par 4 Jer onaj tko je među živima, ima nade: i živ pas više  vrijedi nego mrtav lav. 
\par 5 Živi barem znaju da će umrijeti, a mrtvi ne znaju ništa  niti imaju više nagrade, jer se zaboravlja i spomen na njih. 
\par 6 Davno je nestalo i njihove ljubavi, i mržnje, i zavisti, i više nemaju udjela ni u čem što biva pod suncem. 
\par 7 Zato s radošću jedi svoj kruh i vesela srca pij svoje vino, jer se Bogu već prije svidjelo tvoje djelo. 
\par 8 U svako doba nosi haljine bijele i ulja nek' ne ponestane na tvojoj glavi. 
\par 9 Uživaj život sa ženom koju ljubiš u sve dane svojega ispraznog  vijeka koji ti Bog daje pod suncem, jer to je tvoj udio u životu  i u trudu kojim se trudiš pod suncem. 
\par 10 I što god nakaniš učiniti, učini dok možeš, jer nema  ni djela, ni umovanja, ni spoznaje, ni mudrosti u Podzemlju u  koje ideš. 
\par 11 Osim toga, vidjeh pod suncem: ne dobivaju trku hitri, ni boj hrabri; nema kruha za mudraca, ni bogatstva za razumne, ni milosti za učene, jer vrijeme i kob sve ih dostiže. 
\par 12 Čovjek ne zna svoga časa: kao ribe ulovljene u podmukloj  mreži, i kao ptice u zamku uhvaćene, tako se hvataju sinovi ljudski  u vrijeme nevolje koja ih iznenada spopada. 
\par 13 Još vidjeh pod suncem i ovu "mudrost" koja mi se učini  velikom: 
\par 14 Bi jedan malen grad i u njem malo ljudi, a na nj udari  velik kralj, opkoli ga i podiže oko njega velike opsadne tornjeve. 
\par 15 Ali se u njemu nađe čovjek siromah mudar koji spasi grad  svojom mudrošću, a poslije se nitko nije sjećao toga čovjeka. 
\par 16 Ipak ja velim: bolja je mudrost nego jakost, ali se ne  cijeni mudrost siromaha i ne slušaju njegove riječi. 
\par 17 Blage se riječi mudraca bolje čuju nego vika zapovjednika  nad luđacima. 
\par 18 Mudrost više vrijedi nego bojno oružje, ali jedan jedini  grešnik pokvari mnogo dobra. 


\chapter{10}

\par 1 Uginula muha usmrdi mirisno ulje, a i malo ludosti jače je  od mudrosti i časti. 
\par 2 Mudrac kroči pravim putem, a luđak krivim. 
\par 3 Dovoljno je da luđak pođe putem: kako razbora nema, svakomu  pokazuje da je lud. 
\par 4 Ako se na te digne vladaočev gnjev, ne ostavljaj svoga  mjesta, jer blagost sprečava velike grijehe. 
\par 5 Ima zlo što ga vidjeh pod suncem kao prestupak koji dolazi  od vladaoca: 
\par 6 ludost se podiže na najviša mjesta, a veliki zauzimaju  niske položaje. 
\par 7 Vidjeh sluge na konjima, a knezove gdje idu pješice kao  sluge. 
\par 8 Tko jamu kopa, u nju pada; i tko ruši zid, ujeda ga zmija. 
\par 9 Tko lomi kamenje, ono ga ranjava; tko cijepa drva, može nastradati. 
\par 10 Kad zatupi željezo i oštrica mu nije nabrušena, tada  treba više snage; a nagrada mudrosti je uspjeh. 
\par 11 Ako zmija ujede prije čaranja, ništa onda opčaratelj  ne koristi. 
\par 12 Pune su miline riječi iz usta mudraca, a bezumnika upropašćuju njegove usne. 
\par 13 On počinje svoje besjede ludošću i svršava ih potpunim  bezumljem. 
\par 14 Luđak previše govori: čovjek ne poznaje budućnost, i  tko mu može kazati što će poslije njega biti? 
\par 15 Luđake mori njihov trud; tko ne zna puta, ne može u grad. 
\par 16 Jao tebi, zemljo, kad ti je kralj premlad i knezovi se  već ujutro goste. 
\par 17 Blago tebi, zemljo, kad ti je kralj plemenit i knezovi  ti u svoje vrijeme blaguju da se okrijepe, a ne da se opiju. 
\par 18 S lijenosti se ugiblju grede, zbog nebrige prokišnjava kuća. 
\par 19 Ali su gozbe radi zabave i vino uveseljava život, a novci  pribavljaju sve. 
\par 20 Ni u svojoj misli ne kuni kralja, ni u svojoj ložnici ne kuni bogataša, jer će ptice odnijeti glas i kleveta lako okrilati. 


\chapter{11}

\par 1 Baci kruh svoj na vodu i naći ćeš ga poslije mnogo vremena. 
\par 2 Podijeli sedmorici ili osmorici, jer ne znaš kakvo će  zlo zadesiti zemlju. 
\par 3 Kad se oblaci napune kišom, prosiplju je na zemlju, a  padne li drvo na jug ili na sjever, svejedno: gdje padne, ondje  i ostaje. 
\par 4 Tko pazi na vjetar, ne sije, i tko gleda na oblake, ne žanje. 
\par 5 Kao što ne znaš koji je put vjetru ni kako postaju kosti  u utrobi trudne žene, tako ne znaš ni djela Boga koji sve tvori. 
\par 6 Ujutro sij svoje sjeme, a navečer nek' ti ruka ne počiva. Jer ne znaš da li će biti bolje ovo ili ono, ili će oboje biti  jednako dobro. 
\par 7 Ljupka je svjetlost i ugodno je očima vidjeti sunce. 
\par 8 Ali ako čovjek živi i mnogo godina, neka se uvijek veseli, a neka se sjeti da će tamnih dana biti mnogo. Ispraznost je  sve što će doći. 
\par 9 Zato se raduj, mladiću, za svoje mladosti, i veseli se u danima svoga mladenaštva; idi putovima svoga srca i slijedi želje svojih očiju; ali znaj da će ti za sve to suditi Bog. 
\par 10 Ukloni dakle jad iz svoga srca i udalji bol od svojega tijela. Ali je isprazna i mladost i doba tamnih kosa. 



\chapter{12}

\par 1 I sjećaj se svoga Stvoritelja u danima svoje mladosti prije  nego dođu zli dani i prispiju godine kad ćeš reći: "Ne mile mi  se." 
\par 2 Prije nego potamni sunce i svjetlost, mjesec i zvijezde, i vrate se oblaci iza kiše. 
\par 3 U dan kad zadrhte čuvari kuće i pognu se junaci, i dosađuju se mlinarice jer ih je premalo, i potamne oni koji gledaju kroz prozore; 
\par 4 kad se zatvore ulična vrata, oslabi šum mlina, kad utihne pjev ptice i zamru zvuci pjesme. 
\par 5 Kad je put uzbrdo muka i svaki izlazak prijetnja; a badem je u cvatu, i skakavac ne skače više, i koprov plot puca, jer čovjek ide u svoj vječni dom! A narikače već se kreću ulicama. 
\par 6 Prije nego se prekine srebrna vrpca i zlatna se svjetiljka razbije i razlupa se vrč na izvoru i slomi točak na bunaru; 
\par 7 i vrati se prah u zemlju kao što je iz nje i došao, a duh  se vrati Bogu koji ga je dao. 
\par 8 Ispraznost nad ispraznostima, veli Propovjednik, sve je  ispraznost. 
\par 9 A osim toga što je sam Propovjednik bio mudar, on je i  narod učio mudrosti te je odmjerio, ispitao i sastavio mnogo  mudrih izreka. 
\par 10 Ujedno se Propovjednik trudio pronaći prikladne  riječi i izravno izraziti istinu. 
\par 11 Besjede su mudrih ljudi kao ostani i kao pobodeni kolci:  pastir se njima služi na dobro svojega stada. 
\par 12 I na kraju, sine moj, znaj da je neizmjerno mnogo truda  potrebno da se napiše knjiga i da mnogo učenje umara tijelo. 
\par 13 Čujmo svemu završnu riječ: "Boj se Boga, izvršuj njegove  zapovijedi, jer - to je sav čovjek." 
\par 14 Jer sva će skrivena  djela, bila dobra ili zla, Bog izvesti na sud. 





\end{document}