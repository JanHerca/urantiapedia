\begin{document}

\title{Rimljanima}


\chapter{1}

\par 1 Pavao,  sluga Krista Isusa, pozvan za apostola, odlučen za evanđelje Božje - 
\par 2 koje Bog unaprijed obećavaše po svojim prorocima u Pismima svetim 
\par 3 o Sinu svome, potomku Davidovu po tijelu, 
\par 4 postavljenu Sinom Božjim, u snazi, po Duhu posvetitelju uskrsnućem od mrtvih, o Isusu Kristu, Gospodinu našemu, 
\par 5 po komu primismo milost i apostolstvo da na slavu imena njegova k poslušnosti vjere privodimo sve pogane 
\par 6 među kojima ste i vi pozvanici Isusa Krista: 
\par 7 svima u Rimu, miljenicima Božjim, pozvanicima, svetima. Milost vam i mir od Boga, Oca našega, i Gospodina Isusa Krista. 
\par 8 Ponajprije zahvaljujem Bogu mojemu po Isusu Kristu za  sve vas: što se vaša vjera navješćuje po svem svijetu. 
\par 9 Doista, svjedok mi je Bog - komu duhom svojim služim u evanđelju Sina  njegova - da vas se 
\par 10 u svojim molitvama neprekidno spominjem  i uvijek molim ne bi li mi se već jednom s voljom Božjom nekako  posrećilo doći k vama. 
\par 11 Jer čeznem vidjeti vas da vam predam  nešto dara duhovnoga te se ojačate, zapravo - 
\par 12 da se zajedno  s vama ohrabrim zajedničkom vjerom, vašom i mojom. 
\par 13 A ne bih  htio, braćo, da ne znate: često sam bio nakanio doći k vama -  i sve dosad bio spriječen - da i među vama uberem koji plod kao  i među drugim narodima. 
\par 14 Dužnik sam Grcima i barbarima, mudracima  i neznalicama. 
\par 15 Odatle moja nakana da i vama u Rimu navijestim  evanđelje. 
\par 16 Ne stidim se, uistinu, evanđelja: ono je snaga Božja  na spasenje svakomu tko vjeruje - Židovu najprije, pa Grku. 
\par 17 Jer  pravednost se Božja od vjere k vjeri u njemu otkriva kao što  je pisano: Pravednik će od vjere živjeti. 
\par 18 Otkriva se doista s neba gnjev Božji na svaku bezbožnost  i nepravednost ljudi koji istinu sputavaju nepravednošću. 
\par 19 Jer  što se o Bogu može spoznati, očito im je: Bog im očitova. 
\par 20 Uistinu, ono nevidljivo njegovo, vječna njegova moć i božanstvo, onamo  od stvaranja svijeta, umom se po djelima razabire tako da nemaju  isprike. 
\par 21 Jer premda upoznaše Boga, ne iskazaše mu kao Bogu  ni slavu ni zahvalnost, nego ishlapiše u mozganjima svojim te  se pomrači bezumno srce njihovo. 
\par 22 Gradeći se mudrima, poludješe  i 
\par 23 zamijeniše slavu neraspadljivog Boga likom,  obličjem raspadljiva čovjeka, i ptica, i četveronožaca, i gmazova. 
\par 24 Zato ih je Bog po pohotama srdaca njihovih predao nečistoći  te sami obeščašćuju svoja tijela, 
\par 25 oni što su Istinu - Boga  zamijenili lažju, častili i štovali stvorenje umjesto Stvoritelja, koji je blagoslovljen u vjekove. Amen. 
\par 26 Stoga ih je Bog predao sramotnim strastima: njihove žene  zamijeniše naravno općenje protunaravnim, 
\par 27 a tako su i muškarci  napustili naravno općenje sa ženom i raspalili se pohotom jedni  za drugima te muškarci s muškarcima sramotno čine i sami na sebi  primaju zasluženu plaću svoga zastranjenja. 
\par 28 I kako nisu smatrali vrijednim držati se spoznaje Boga, predade ih Bog nevaljanu umu te čine što ne dolikuje, 
\par 29 puni  svake nepravde, pakosti, lakomosti, zloće; puni zavisti, ubojstva, svađe, prijevare, zlonamjernosti; došaptavači, 
\par 30 klevetnici, mrzitelji Boga, drznici, oholice, preuzetnici, izmišljači zala, roditeljima neposlušni, 
\par 31 nerazumni, nevjerni, bešćutni, nemilosrdni. 
\par 32 Znaju za odredbu Božju - da smrt zaslužuju koji takvo što  čine - a oni ne samo da to čine nego i povlađuju onima koji čine. 


\chapter{2}

\par 1 Zato nemaš isprike, čovječe koji sudiš, tko god ti bio. Jer  time što drugoga sudiš, sebe osuđuješ: ta to isto činiš ti što  sudiš. 
\par 2 Znamo pak: Bog po istini sudi one koji takvo što čine. 
\par 3 Misliš li da ćeš izbjeći sudu Božjemu, ti čovječe što sudiš  one koji takvo što čine, a sam to isto činiš? 
\par 4 Ili prezireš  bogatstvo dobrote, strpljivosti i velikodušnosti njegove ne shvaćajući  da te dobrota Božja k obraćenju privodi? 
\par 5 Tvrdokornošću svojom  i srcem koje neće obraćenja zgrćeš na se gnjev za Dan gnjeva  i objavljenja pravedna suda Boga 
\par 6 koji će uzvratiti svakom  po djelima: 
\par 7 onima koji postojanošću u dobrim djelima ištu  slavu, čast i neraspadljivost - život vječni; 
\par 8 buntovnicima  pak i nepokornima istini, a pokornima nepravdi - gnjev i srdžba! 
\par 9 Nevolja i tjeskoba na svaku dušu čovječju koja čini zlo, na  Židova najprije, pa na Grka; 
\par 10 a slava, čast i mir svakomu  koji čini dobro, Židovu najprije, pa Grku! 
\par 11 Ta u Boga nema  pristranosti. 
\par 12 Uistinu koji bez Zakona sagriješiše, bez Zakona će i  propasti; i koji pod Zakonom sagriješiše, po Zakonu će biti suđeni. 
\par 13 Ne, pred Bogom nisu pravedni slušatelji Zakona, nego - izvršitelji  će Zakona biti opravdani. 
\par 14 Ta kad se god pogani, koji nemaju  Zakona, po naravi drže Zakona, i nemajući Zakona, oni su sami  sebi Zakon: 
\par 15 pokazuju da je ono što Zakon nalaže upisano u  srcima njihovim. O tom svjedoči i njihova savjest, a i prosuđivanja  kojima se među sobom optužuju ili brane. 
\par 16 To će se očitovati  na Dan u koji će, po mojem evanđelju, Bog po Isusu Kristu suditi  ono što je skriveno u ljudima. 
\par 17 Ako pak ti koji se Židovom nazivaš, koji mirno počivaš  na Zakonu i dičiš se Bogom, 
\par 18 koji poznaješ Volju i iz Zakona  poučen razlučuješ što je bolje 
\par 19 te si uvjeren da si vođa slijepih, svjetlo onih u tami, 
\par 20 odgojitelj nerazumnih, učitelj nejačadi  jer u Zakonu imaš oličenje znanja i istine; 
\par 21 ti, dakle, koji  drugoga učiš, sam sebe ne učiš! Ti koji propovijedaš da se ne  krade, kradeš! 
\par 22 Ti koji zabranjuješ preljub, preljub počinjaš!  Ti komu su odvratni kumiri, pljačkaš hramove! 
\par 23 Ti koji se  Zakonom dičiš, kršenjem toga Zakona Boga obeščašćuješ! 
\par 24 Doista, kako je pisano, ime se Božje zbog vas huli među narodima. 
\par 25 Da, obrezanje koristi ako vršiš Zakon; ako pak kršiš  Zakon, obrezanje tvoje postalo je neobrezanje. 
\par 26 Ako dakle  neobrezani opslužuje propise Zakona, neće li se njegovo neobrezanje  smatrati obrezanjem? 
\par 27 I onaj koji je podrijetlom neobrezanik, a ispunja Zakon, sudit će tebi koji si, uza sve slovo i obrezanje, prijestupnik Zakona. 
\par 28 Ta nije Židov tko je Židov naizvana  i nije obrezanje ono izvana, 
\par 29 na tijelu, nego pravi je Židov  u nutrini i pravo je obrezanje u srcu, po duhu, ne po slovu.  Pohvala mu nije od ljudi, nego od Boga. 


\chapter{3}

\par 1 Koja je dakle prednost Židova? Ili kakva korist od obrezanja? 
\par 2 Velika u svakom pogledu. Ponajprije: povjerena su im obećanja  Božja. 
\par 3 Da, a što ako su se neki iznevjerili? Neće li njihova  nevjernost obeskrijepiti vjernost Božju? 
\par 4 Nipošto! Nego neka  Bog bude istinit, a svaki čovjek lažac, kao što je pisano: Da pravedan budeš po obećanjima svojim i pobijediš kada te sudili budu. 
\par 5 Ako pak naša nepravednost ističe Božju pravednost, što  ćemo na to reći? Nije li onda - po ljudsku govorim - nepravedan  Bog koji daje maha gnjevu? 
\par 6 Nipošto! Ta kako će inače Bog suditi  svijet? 
\par 7 Ako je, doista, istina Božja po mojoj lažljivosti  obilno zasjala njemu na slavu, zašto da ja još budem suđen kao  grešnik? 
\par 8 I zar da ne "činimo zlo da dođe dobro", kako nas  kleveću i kako neki kažu da govorimo? Sud ih pravedni čeka! 
\par 9 Što dakle? Imamo li prednost? Ne baš! Jer upravo optužismo  sve, i Židove i Grke, da su pod grijehom, 
\par 10 kao što je pisano: Nema pravedna ni jednoga, 
\par 11 nema razumna, nema ga tko bi Boga tražio. 
\par 12 Svi skrenuše, svi se zajedno pokvariše, nitko da čini dobro - nijednoga nema. 
\par 13 Grob otvoren grlo je njihovo, jezikom lažno laskaju, pod usnama im je otrov ljutičin, 
\par 14 usta im puna kletve i grkosti; 
\par 15 noge im hitre da krv proliju, 
\par 16 razvaline i nevolja na njinim su putima, 
\par 17 put mira oni ne poznaju, 
\par 18 straha Božjega nemaju pred očima. 
\par 19 A znamo: što god Zakon veli, govori onima pod Zakonom, da svaka usta umuknu i sav svijet bude krivac pred Bogom. 
\par 20 Zato  se po djelima Zakona nitko neće opravdati pred njim. Uistinu, po Zakonu - samo spoznaja grijeha! 
\par 21 Sada se pak izvan Zakona očitovala pravednost Božja,  posvjedočena Zakonom i Prorocima, 
\par 22 pravednost Božja po vjeri  Isusa Krista, prema svima koji vjeruju. Ne, nema razlike! 
\par 23 Svi  su zaista sagriješili i potrebna im je slava Božja; 
\par 24 opravdani  su besplatno, njegovom milošću po otkupljenju u Kristu Isusu. 
\par 25 Njega je Bog izložio da krvlju svojom bude Pomirilište po  vjeri. Htio je tako očitovati svoju pravednost kojom je u svojoj  božanskoj strpljivosti propuštao dotadašnje grijehe; 
\par 26 htio  je očitovati svoju pravednost u sadašnje vrijeme - da bude pravedan  i da opravdava onoga koji je od vjere Isusove. 
\par 27 Gdje je dakle hvastanje? Isključeno je. Po kojem zakonu?  Po zakonu djela? Ne, nego po zakonu vjere. 
\par 28 Smatramo zaista  da se čovjek opravdava vjerom bez djela Zakona. 
\par 29 Ili je Bog  samo Bog Židova? Nije li i pogana? Da, i pogana. 
\par 30 Jer jedan  je Bog: on će opravdati obrezane zbog vjere i neobrezane po vjeri. 
\par 31 Obeskrepljujemo li dakle Zakon po vjeri? Nipošto! Naprotiv, Zakon utvrđujemo. 


\chapter{4}

\par 1 Što ćemo dakle reći? Što je Abraham, otac naš, našao po tijelu? 
\par 2 Doista, ako je Abraham po djelima opravdan, ima se čime dičiti  - ali ne pred Bogom. 
\par 3 Ta što veli Pismo? Povjerova Abraham  Bogu i uračuna mu se u pravednost. 
\par 4 Onomu tko radi ne računa  se plaća kao milost, nego kao dug. 
\par 5 Onomu tko ne radi, a vjeruje  u Onoga koji opravdava bezbožnika, vjera se uračunava u pravednost, 
\par 6 kao što i David blaženim proglašuje čovjeka kojemu Bog uračunava  pravednost bez djela: 
\par 7 Blaženi oni kojima je zločin otpušten, kojima je grijeh pokriven! 
\par 8 Blago čovjeku komu Gospodin ne ubraja krivnju. 
\par 9 Ide li dakle ovo blaženstvo samo obrezane ili i neobrezane?  Ta velimo: Vjera se Abrahamu uračuna u pravednost. 
\par 10 A  kako mu se uračuna? Već obrezanu ili još neobrezanu? Ne obrezanu, nego neobrezanu! 
\par 11 I znak obrezanja primi kao pečat pravednosti  koju je po vjeri zadobio još neobrezan, da bude ocem svih vjernika:  neobrezanih - te im se uračuna pravednost - 
\par 12 i ocem obrezanih, ne onih koji su samo obrezani, nego onih koji uz to idu stopama  vjere još neobrezana oca našeg Abrahama. 
\par 13 Doista, obećanje da će biti baštinik svijeta nije Abrahamu  ili njegovu potomstvu dano na temelju nekog zakona, nego na temelju  pravednosti vjere. 
\par 14 Jer ako su baštinici oni iz Zakona, prazna  je vjera, jalovo obećanje. 
\par 15 Ta Zakon rađa gnjev; gdje pak  nema Zakona, nema ni prekršaja. 
\par 16 Zato - zbog vjere da bude  po milosti to obećanje zajamčeno svemu potomstvu, ne potomstvu  samo po Zakonu, nego i po vjeri Abrahama, koji je otac svih nas  - 
\par 17 kao što je pisano: Ocem mnoštva naroda ja te postavljam  - pred Onim komu povjerova, pred Bogom koji oživljuje mrtve i  zove da bude ono što nije. 
\par 18 U nadi protiv svake nade povjerova Abraham da postane  ocem naroda mnogih po onom što je rečeno: Toliko će  biti tvoje potomstvo. 
\par 19 Nepokolebljivom vjerom promotri  on tijelo svoje već obamrlo - bilo mu je blizu sto godina - i  obamrlost krila Sarina. 
\par 20 Ali pred Božjim obećanjem nije nevjeran  dvoumio, nego se vjerom ojačao davši slavu Bogu, 
\par 21 posve uvjeren  da on može učiniti što je obećao. 
\par 22 Zato mu se i uračuna  u pravednost. 
\par 23 Ali nije samo za nj napisano: Uračuna mu se, 
\par 24 nego  i za nas kojima se ima uračunati, nama što vjerujemo u Onoga  koji od mrtvih uskrisi Isusa, Gospodina našega, 
\par 25 koji je predan  za opačine naše i uskrišen radi našeg opravdanja. 


\chapter{5}

\par 1 Opravdani dakle vjerom, u miru smo s Bogom po Gospodinu našem  Isusu Kristu. 
\par 2 Po njemu imamo u vjeri i pristup u ovu milost  u kojoj stojimo i dičimo se nadom slave Božje. 
\par 3 I ne samo to!  Mi se dičimo i u nevoljama jer znamo: nevolja rađa postojanošću, 
\par 4 postojanost prokušanošću, prokušanost nadom. 
\par 5 Nada pak  ne postiđuje. Ta ljubav je Božja razlivena u srcima našim po  Duhu Svetom koji nam je dan! 
\par 6 Doista, dok mi još bijasmo nemoćni, Krist je, već u to vrijeme, za nas bezbožnike umro. 
\par 7 Zbilja, jedva bi tko za pravedna umro; možda bi se za dobra tko i odvažio  umrijeti. 
\par 8 A Bog pokaza ljubav svoju prema nama ovako: dok  još bijasmo grešnici, Krist za nas umrije. 
\par 9 Koliko li ćemo  se više sada, pošto smo opravdani krvlju njegovom, spasiti po  njemu od srdžbe? 
\par 10 Doista, ako se s Bogom pomirismo po smrti  Sina njegova dok još bijasmo neprijatelji, mnogo ćemo se više, pomireni, spasiti životom njegovim. 
\par 11 I ne samo to! Dičimo  se u Bogu po Gospodinu našemu Isusu Kristu po kojem zadobismo  pomirenje. 
\par 12 Zbog toga, kao što po jednom Čovjeku uđe u svijet  grijeh i po grijehu smrt, i time što svi sagriješiše, na sve ljude prijeđe smrt... 
\par 13 Doista, do Zakona bilo je grijeha  u svijetu, ali se grijeh ne ubraja kad nema zakona. 
\par 14 Da, ali  smrt je od Adama do Mojsija doista kraljevala i nad onima koji  ne sagriješiše prekršajem sličnim kao Adam, koji je pralik Onoga  koji ima doći. 
\par 15 Ali s darom nije kao s grijehom. Jer ako su grijehom  jednoga mnogi umrli, mnogo se obilatije na sve razlila milost  Božja, milost darovana u jednom čovjeku, Isusu Kristu. 
\par 16 I  dar - to nije kao kad je ono jedan sagriješio: jer presuda nakon  jednoga grijeha posta osudom, a dar nakon mnogih grijeha - opravdanjem. 
\par 17 Uistinu, ako grijehom jednoga smrt zakraljeva - po jednome, mnogo će više oni koji primaju izobilje milosti i dara pravednosti  kraljevati u životu - po Jednome, Isusu Kristu. 
\par 18 Dakle, grijeh jednoga - svim ljudima na osudu, tako i  pravednost Jednoga - svim ljudima na opravdanje, na život! 
\par 19 Doista, kao što su neposluhom jednoga čovjeka mnogi postali grešnici  tako će i posluhom Jednoga mnogi postati pravednici. 
\par 20 A zakon nadođe da se umnoži grijeh. Ali gdje se umnožio  grijeh, nadmoćno izobilova milost: 
\par 21 kao što grijeh zakraljeva  smrću, da tako i milost kraljuje pravednošću za život vječni  po Isusu Kristu Gospodinu našemu. 


\chapter{6}

\par 1 Što ćemo dakle reći? Da ostanemo u grijehu da milost izobiluje? 
\par 2 Nipošto! Jednom umrli grijehu, kako da još živimo u njemu? 
\par 3 Ili zar ne znate: koji smo god kršteni u Krista Isusa, u smrt  smo njegovu kršteni. 
\par 4 Krštenjem smo dakle zajedno s njime ukopani  u smrt da kao što Krist slavom Očevom bi uskrišen od mrtvih,  i mi tako hodimo u novosti života. 
\par 5 Ako smo doista s njime srasli po sličnosti smrti njegovoj, očito ćemo srasti i po sličnosti njegovu uskrsnuću. 
\par 6 Ovo znamo:  naš je stari čovjek zajedno s njim raspet da onemoća ovo grešno  tijelo te više ne robujemo grijehu. 
\par 7 Ta tko umre, opravdan  je od grijeha. 
\par 8 Pa ako umrijesmo s Kristom, vjerujemo da ćemo i živjeti  zajedno s njime. 
\par 9 Znamo doista: Krist uskrišen od mrtvih, više  ne umire, smrt njime više ne gospoduje. 
\par 10 Što umrije, umrije  grijehu jednom zauvijek; a što živi, živi Bogu. 
\par 11 Tako i vi:  smatrajte sebe mrtvima grijehu, a živima Bogu u Kristu Isusu! 
\par 12 Neka dakle ne kraljuje grijeh u vašem smrtnom tijelu  da slušate njegove požude; 
\par 13 i ne predajite grijehu udova svojih  za oružje nepravde, nego sebe, od mrtvih oživjele, predajte Bogu  i udove svoje dajte Bogu za oružje pravednosti. 
\par 14 Valjda grijeh  neće vama gospodovati! Ta niste pod Zakonom nego pod milošću! 
\par 15 Što dakle? Da griješimo jer nismo pod Zakonom nego pod  milošću? Nipošto! 
\par 16 Ne znate li: ako se komu predate za robove, na poslušnost, robovi ste onoga koga slušate: bilo grijeha -  na smrt, bilo poslušnosti - na pravednost. 
\par 17 Bijaste robovi  grijeha, ali ste, hvala Bogu, od srca poslušali ono pravilo nauka  kojemu ste povjereni; 
\par 18 da, oslobođeni grijeha, postadoste  sluge pravednosti. 
\par 19 Po ljudsku govorim zbog vaše ljudske slabosti:  kao što nekoć predadoste udove svoje za robove nečistoći i bezakonju  - do bezakonja, tako sada predajte udove svoje za robove pravednosti  - do posvećenja. 
\par 20 Uistinu, kad bijaste robovi grijeha, "slobodni"  bijaste od pravednosti. 
\par 21 Pa kakav ste plod onda imali? Onoga  se sada stidite jer svršetak je tomu - smrt. 
\par 22 Sada pak pošto  ste oslobođeni grijeha i postali sluge Božje, imate plod svoj  za posvećenje, a svršetak - život vječni. 
\par 23 Jer plaća je grijeha  smrt, a dar Božji jest život vječni u Kristu Isusu, Gospodinu  našem. 


\chapter{7}

\par 1 Ili zar ne znate, braćo - poznavaocima zakona govorim - da  zakon gospodari čovjekom samo za vrijeme njegova života. 
\par 2 Doista, udana je žena vezana zakonom dok joj muž živi; umre li joj muž, riješena je zakona o mužu. 
\par 3 Dakle: dok joj muž živi, zvat  će se, očito, preljubnicom pođe li za drugoga. Ako li joj pak  muž umre, slobodna je od zakona te nije preljubnica pođe li za  drugoga. 
\par 4 Tako, braćo moja, i vi po tijelu Kristovu umrijeste Zakonu  da pripadnete drugomu, Onomu koji je od mrtvih uskrišen, te plodove  donosimo Bogu. 
\par 5 Doista, dok bijasmo u tijelu, grešne su strasti, Zakonom izazvane, djelovale u našim udovima te smrti donosile  plodove; 
\par 6 sada pak umrijevši onomu što nas je sputavalo, riješeni  smo Zakona te služimo u novosti Duha, a ne u stareži slova. 
\par 7 Što ćemo dakle reći? Je li Zakon grijeh? Nipošto! Nego:  grijeha ne spoznah doli po Zakonu jer za požudu ne bih znao da  Zakon nije govorio: Ne poželi! 
\par 8 A grijeh je, uhvativši  priliku, po zapovijedi u meni prouzročio svakovrsnu požudu. Ta  bez zakona grijeh je mrtav. 
\par 9 Da, ja sam nekoć živio bez zakona.  Ali kad je došla zapovijed, grijeh oživje. 
\par 10 Ja pak umrijeh  i ustanovi se: zapovijed dana za život bi mi na smrt. 
\par 11 Doista  grijeh, uhvativši priliku, zapovijeđu me zavede, njome me i ubi. 
\par 12 Tako: Zakon je svet, i zapovijed je sveta, i pravedna, i  dobra. 
\par 13 Pa zar se to dobro meni u smrt prometnu? Nipošto! Nego:  grijeh, da se grijehom očituje, po tom dobru uzrokuje mi smrt  - da grijeh po zapovijedi postane najvećim grešnikom. 
\par 14 Zakon je, znamo, duhovan; ja sam pak tjelesan, prodan  pod grijeh. 
\par 15 Zbilja ne razumijem što radim: ta ne činim ono  što bih htio, nego što mrzim - to činim. 
\par 16 Ako li pak činim  što ne bih htio, slažem se sa Zakonom, priznajem da je dobar. 
\par 17 Onda to ne činim više ja, nego grijeh koji prebiva u meni. 
\par 18 Doista znam da dobro ne prebiva u meni, to jest u mojem tijelu.  Uistinu: htjeti mi ide, ali ne i činiti dobro. 
\par 19 Ta ne činim  dobro koje bih htio, nego zlo koje ne bih htio - to činim. 
\par 20 Ako  li pak činim ono što ne bih htio, nipošto to ne radim ja, nego  grijeh koji prebiva u meni. 
\par 21 Nalazim dakle ovaj zakon: kad bih htio činiti dobro,  nameće mi se zlo. 
\par 22 Po nutarnjem čovjeku s užitkom se slažem  sa Zakonom Božjim, 
\par 23 ali opažam u svojim udovima drugi zakon  koji vojuje protiv zakona uma moga i zarobljuje me zakonom grijeha  koji je u mojim udovima. 
\par 24 Jadan li sam ja čovjek! Tko će me istrgnuti iz ovoga tijela  smrtonosnoga? 
\par 25 Hvala Bogu po Isusu Kristu Gospodinu našem! Ja, dakle, umom ja služim zakonu Božjemu, a tijelom zakonu grijeha. 


\chapter{8}

\par 1 Nikakve dakle sada osude onima koji su u Kristu Isusu! 
\par 2 Ta  zakon Duha života u Kristu Isusu oslobodi me zakona grijeha i  smrti. 
\par 3 Uistinu, što je bilo nemoguće Zakonu, jer je zbog tijela  onemoćao, Bog je učinio: poslavši Sina svoga u obličju grešnoga  tijela i s obzirom na grijeh, osudi grijeh u tijelu 
\par 4 da se  pravednost Zakona ispuni u nama koji ne živimo po tijelu nego  po Duhu. 
\par 5 Da, oni koji žive po tijelu, teže za onim što je tjelesno;  a koji po Duhu, za onim što je Duhovo: 
\par 6 težnja je tijela smrt, a težnja Duha život i mir. 
\par 7 Jer težnja je tijela protivna  Bogu: zakonu se Božjemu ne podvrgava, a i ne može. 
\par 8 Oni pak  koji su u tijelu, ne mogu se Bogu svidjeti. 
\par 9 A vi niste u tijelu, nego u Duhu, ako Duh Božji prebiva u vama. A nema li tko Duha  Kristova, taj nije njegov. 
\par 10 I ako je Krist u vama, tijelo  je doduše mrtvo zbog grijeha, ali Duh je život zbog pravednosti. 
\par 11 Ako li Duh Onoga koji uskrisi Isusa od mrtvih prebiva u vama, Onaj koji uskrisi Krista od mrtvih oživit će i smrtna tijela  vaša po Duhu svome koji prebiva u vama. 
\par 12 Dakle, braćo, dužnici smo, ali ne tijelu da po tijelu  živimo! 
\par 13 Jer ako po tijelu živite, umrijeti vam je, ako li  pak Duhom usmrćujete tjelesna djela, živjet ćete. 
\par 14 Svi koje vodi Duh Božji sinovi su Božji. 
\par 15 Ta ne primiste  duh robovanja da se opet bojite, nego primiste Duha posinstva  u kojem kličemo: "Abba! Oče!" 
\par 16 Sam Duh susvjedok je s našim  duhom da smo djeca Božja; 
\par 17 ako pak djeca, onda i baštinici, baštinici Božji, a subaštinici Kristovi, kada doista s njime  zajedno trpimo, da se zajedno s njime i proslavimo. 
\par 18 Smatram, uistinu: sve patnje sadašnjega vremena nisu  ništa prema budućoj slavi koja se ima očitovati u nama. 
\par 19 Doista, stvorenje sa svom žudnjom iščekuje ovo objavljenje sinova Božjih: 
\par 20 stvorenje je uistinu podvrgnuto ispraznosti - ne po svojoj  volji, nego zbog onoga koji ga podvrgnu - ali u nadi. 
\par 21 Jer  i stvorenje će se osloboditi robovanja pokvarljivosti da sudjeluje  u slobodi i slavi djece Božje. 
\par 22 Jer znamo: sve stvorenje zajedno  uzdiše i muči se u porođajnim bolima sve do sada. 
\par 23 Ali ne  samo ono! I mi koji imamo prvine Duha, i mi u sebi uzdišemo iščekujući  posinstvo, otkupljenje svoga tijela. 
\par 24 Ta u nadi smo spašeni!  Nada pak koja se vidi nije nada. Jer što tko gleda, kako da se  tomu i nada? 
\par 25 Nadamo li se pak onomu čega ne gledamo, postojano  to iščekujemo. 
\par 26 Tako i Duh potpomaže našu nemoć. Doista ne znamo što  da molimo kako valja, ali se sam Duh za nas zauzima neizrecivim  uzdasima. 
\par 27 A Onaj koji proniče srca zna koja je želja Duha  - da se on po Božju zauzima za svete. 
\par 28 Znamo pak da Bog u svemu na dobro surađuje s onima koji  ga ljube, s onima koji su odlukom njegovom pozvani. 
\par 29 Jer koje  predvidje, te i predodredi da budu suobličeni slici Sina njegova  te da on bude prvorođenac među mnogom braćom. 
\par 30 Koje pak predodredi, te i pozva; koje pozva, te i opravda; koje opravda, te i proslavi. 
\par 31 Što ćemo dakle na to reći? Ako je Bog za nas, tko će  protiv nas? 
\par 32 Ta on ni svojega Sina nije poštedio, nego ga  je za sve nas predao! Kako nam onda s njime neće sve darovati? 
\par 33 Tko će optužiti izabranike Božje? Bog opravdava! 
\par 34 Tko  će osuditi? Krist Isus umrije, štoviše i uskrsnu, on je i zdesna  Bogu - on se baš zauzima za nas! 
\par 35 Tko će nas rastaviti od  ljubavi Kristove? Nevolja? Tjeskoba? Progonstvo? Glad? Golotinja?  Pogibao? Mač? 
\par 36 Kao što je pisano: Poradi tebe ubijaju nas dan za danom i mi smo im ko ovce za klanje. 
\par 37 U svemu tome nadmoćno pobjeđujemo po onome koji nas uzljubi. 
\par 38 Uvjeren sam doista: ni smrt ni život, ni anđeli ni vlasti, ni sadašnjost ni budućnost, ni sile, 
\par 39 ni dubina ni visina, ni ikoji drugi stvor neće nas moći rastaviti od ljubavi Božje  u Kristu Isusu Gospodinu našem. 


\chapter{9}

\par 1 Istinu govorim u Kristu, ne lažem; susvjedok mi je savjest  moja u Duhu Svetom: 
\par 2 silna mi je tuga i neprekidna bol u srcu. 
\par 3 Da, htio bih ja sam proklet biti, odvojen od Krista, za braću  svoju, sunarodnjake svoje po tijelu. 
\par 4 Oni su Izraelci, njihovo  je posinstvo, i Slava, i Savezi, i zakonodavstvo, i bogoštovlje, i obećanja; 
\par 5 njihovi su i oci, od njih je, po tijelu, i Krist, koji je iznad svega, Bog blagoslovljen u vjekove. Amen. 
\par 6 Ali ne kao da se izjalovila riječ Božja. Jer nisu Izrael  svi koji potječu od Izraela; 
\par 7 i nisu svi djeca Abrahamova zato  što su njegovo potomstvo, nego po Izaku će ti se nazivati  potomstvo; 
\par 8 to jest: djeca tijela nisu i djeca Božja, nego  - djeca obećanja računaju se u potomstvo. 
\par 9 Evo doista riječi  obećanja: U ovo ću doba doći i Sara će imati sina. 
\par 10 Ali  ne samo to! I Rebeka je s jednim, s Izakom, ocem našim, zanijela. 
\par 11 Pa kad još blizanci ne bijahu rođeni niti učiniše što dobro  ili zlo - da bi trajnom ostala odluka Božja o izabranju: 
\par 12 ne  po djelima, nego po onome tko poziva - rečeno joj je: Stariji  će služiti mlađemu, 
\par 13 kako je pisano: Jakova sam zavolio, a Ezav mi omrznu. 
\par 14 Što ćemo dakle reći? Možda da u Boga ima nepravde? Nipošto! 
\par 15 Ta Mojsiju veli: Smilovat ću se komu hoću da se smilujem;  sažalit ću se nad kim hoću da se sažalim. 
\par 16 Nije dakle  do onoga koji hoće ni do onoga koji trči, nego do Boga koji se  smiluje. 
\par 17 Jer Pismo veli faraonu: Zato te upravo podigoh  da na tebi pokažem svoju moć i da se razglasi ime moje po svoj  zemlji. 
\par 18 Tako dakle: smiluje se komu hoće, a otvrdnjuje  koga hoće. 
\par 19 Da, reći ćeš mi: Što se onda još tuži? Ta tko se to volji  njegovoj odupro? 
\par 20 Čovječe, tko si ti zapravo da se pravdaš  s Bogom? Zar da djelo rekne tvorcu: "Što si me ovakvim  načinio?" 
\par 21 Ili zar lončar nema vlasti nad glinom  da od istoga tijesta načini posudu sad časnu, sad nečasnu. 
\par 22 A  što ako je Bog, hoteći očitovati gnjev i obznaniti svoju moć  u silnoj strpljivosti podnosio posude gnjeva, dozrele za propast, 
\par 23 da obznani bogatstvo slave svoje na posudama milosrđa, koje  unaprijed pripravi za slavu, 
\par 24 na nama koje pozva ne samo između  Židova nego i između pogana? 
\par 25 Tako i u Hošeji veli: Ne-narod moj prozvat ću narodom mojim i Neljubljenu ljubljenom. 
\par 26 Na mjestu gdje im je rečeno: Vi niste moj narod prozvat će se sinovi Boga živoga. 
\par 27 Izaija pak proglasuje o Izraelu: Zaista, sinova će  Izraelovih brojem biti kao pijeska morskog - Ostatak će se spasiti; 
\par 28 jer riječ će ispuniti i uskoro izvršiti Gospodin na zemlji. 
\par 29 Tako je Izaija i prorekao: Da nam Gospodin nad Vojskama ne ostavi sjeme, ko Sodoma bismo bili i Gomori nalik. 
\par 30 Što ćemo dakle reći? Da pogani koji nisu tražili pravednosti  stekoše pravednost, pravednost po vjeri. 
\par 31 Izrael pak koji  je tražio neki zakon pravednosti, nije do zakona dopro. 
\par 32 Zašto?  Jer nije tražio po vjeri, nego kao po djelima. Spotakoše se o  kamen spoticanja, 
\par 33 kao što je pisano: Evo postavljam na Sionu kamen spoticanja i stijenu posrtanja. Ali tko u nj vjeruje, neće se postidjeti. 


\chapter{10}

\par 1 Braćo! Želja je srca moga i molitva Bogu za njih: da se spase. 
\par 2 Svjedočim doista za njih: imaju revnosti Božje, ali ne u pravom  razumijevanju. 
\par 3 Ne priznajući, doista, Božje pravednosti i  tražeći uspostaviti svoju, pravednosti se Božjoj ne podložiše. 
\par 4 Jer dovršetak je Zakona Krist - na opravdanje svakomu tko  vjeruje. 
\par 5 Da, Mojsije piše o onoj pravednosti iz Zakona: Tko  je vrši, naći će život u njoj. 
\par 6 A pravednost iz vjere ovako  veli: Nemoj reći u srcu svom: Tko će se popeti na nebo  - to jest Krista svesti? 
\par 7 Ili: Tko će sići u bezdan  - to jest izvesti Krista od mrtvih? 
\par 8 Nego što veli? Blizu  ti je Riječ, u ustima tvojim i u srcu tvome - to jest Riječ  vjere koju propovijedamo. 
\par 9 Jer ako ustima ispovijedaš da je Isus Gospodin, i srcem  vjeruješ da ga je Bog uskrisio od mrtvih, bit ćeš spašen. 
\par 10 Doista, srcem vjerovati opravdava, a ustima ispovijedati spasava. 
\par 11 Jer  veli Pismo: Tko god u nj vjeruje, neće se postidjeti. 
\par 12 Nema uistinu razlike između Židova i Grka jer jedan je  Gospodin sviju, bogat prema svima koji ga prizivlju. 
\par 13 Jer:  Tko god prizove ime Gospodnje, bit će spašen. 
\par 14 Ali kako da prizovu onoga u koga ne povjerovaše? A kako  da povjeruju u onoga koga nisu čuli? Kako pak da čuju bez propovjednika? 
\par 15 A kako propovijedati bez poslanja? Tako je pisano: Kako li su ljupke noge onih koji donose blagovijest dobra. 
\par 16 Ali nisu svi poslušali blagovijesti - evanđelja! Zaista, Izaija veli: Gospodine, tko povjerova našoj poruci? 
\par 17 Dakle: vjera po poruci, a poruka riječju Kristovom. 
\par 18 Nego pitam: Zar nisu čuli? Dapače! Po svoj zemlji razliježe se jeka, riječi njihove sve do nakraj svijeta. 
\par 19 Onda pitam: Zar Izrael nije shvatio? Najprije Mojsije veli: Ja ću vas na ljubomor izazvati pukom ništavnim, razdražit ću vas glupim nekim narodom. 
\par 20 Izaija pak hrabro veli: Nađoše me koji me ne tražahu, objavih se onima koji me ne pitahu. 
\par 21 A Izraelu veli: Cio dan pružah ruku narodu nepokornom i buntovnom. 


\chapter{11}

\par 1 Pitam dakle: Zar je Bog odbacio narod svoj? Nipošto?  Ta i ja sam Izraelac, iz potomstva Abrahamova, plemena Benjaminova. 
\par 2 Nije Bog odbacio naroda svojega koga predvidje. Ili  zar ne znate što veli Pismo, ono o Iliji - kako se tuži Bogu  na Izraela: 
\par 3 Gospode, proroke tvoje pobiše, žrtvenike tvoje  porušiše; ja ostadoh sam i još mi o glavi rade. 
\par 4 Pa što  mu veli Božji glas? Ostavih sebi sedam tisuća ljudi  koji ne prignuše koljena pred Baalom. 
\par 5 Tako dakle i u sadašnje  vrijeme postoji Ostatak po milosnom izboru. 
\par 6 Ako pak po milosti, nije po djelima; inače milost nije više milost! 
\par 7 Što dakle? Što Izrael ište, to nije postigao, ali izabrani  postigoše. Ostali pak otvrdnuše, 
\par 8 kao što je pisano: Dade im Bog duh obamrlosti, oči da ne vide, uši da ne čuju sve do dana današnjega. 
\par 9 I David veli: Nek im stol pred njima bude zamkom, i mrežom, i stupicom, i plaćom. 
\par 10 Nek im potamne oči da ne vide i leđa im zauvijek pogni! 
\par 11 Pitam dakle: jesu li posrnuli da propadnu? Nipošto! Naprotiv:  po njihovu posrtaju spasenje poganima da se tako oni, Židovi, izazovu na ljubomor. 
\par 12 Pa ako je njihov posrtaj bogatstvo  za svijet, i njihovo smanjenje bogatstvo za pogane, koliko li  će više to biti njihov puni broj? 
\par 13 Vama pak, poganima, velim: ja kao apostol pogana službu  svoju proslavljam 
\par 14 ne bih li na ljubomor izazvao njih, tijelo  svoje, i spasio neke od njih. 
\par 15 Jer ako je njihovo odbačenje  izmirenje svijeta, što li će biti njihovo prihvaćanje ako ne  oživljenje od mrtvih? 
\par 16 Ako li su prvine svete, sveto je i tijesto; ako li je  korijen svet, svete su i grane. 
\par 17 Pa ako su neke grane odlomljene, a ti, divlja maslina, pricijepljen umjesto njih, postao suzajedničar  korijena, sočnosti masline, 
\par 18 ne uznosi se nad grane. Ako li  se hoćeš uznositi - ne nosiš ti korijena, nego korijen tebe. 
\par 19 Reći ćeš na to: grane su odlomljene da se ja pricijepim. 
\par 20 Dobro! Oni su zbog nevjere odlomljeni, a ti po vjeri stojiš.  Ne uznosi se, nego strahuj! 
\par 21 Jer ako Bog ne poštedje prirodnih  grana, ni tebe neće poštedjeti. 
\par 22 Promotri dakle dobrotu i  strogost Božju: strogost na palima, a dobrotu Božju na sebi ako  ostaneš u toj dobroti, inače ćeš i ti biti odsječen. 
\par 23 A i  oni, ako ne ostanu u nevjeri, bit će pricijepljeni; ta moćan  je Bog da ih opet pricijepi. 
\par 24 Doista, ako si ti, po naravi  divlja maslina, odsječen pa mimo narav pricijepljen na pitomu  maslinu, koliko li će lakše oni po naravi biti pricijepljeni  na vlastitu maslinu! 
\par 25 Jer ne bih htio, braćo, da budete sami po sebi pametni, a da ne znate ovo otajstvo: djelomično je otvrdnuće zadesilo  Izraela dok punina pogana ne uđe. 
\par 26 I tako će se cio Izrael  spasiti, kako je pisano: Doći će sa Siona Otkupitelj, odvratit će bezbožnost od Jakova. 
\par 27 I to će biti moj Savez s njima, kad uklonim grijehe njihove. 
\par 28 U pogledu evanđelja oni su, istina, protivnici poradi  vas, ali u pogledu izabranja oni su ljubimci poradi otaca. 
\par 29 Ta  neopozivi su dari i poziv Božji! 
\par 30 Doista, kao što vi nekoć bijaste neposlušni Bogu, a sada  po njihovoj neposlušnosti zadobiste milosrđe 
\par 31 tako i oni sada  po milosrđu vama iskazanu postadoše neposlušni da i oni sada  zadobiju milosrđe. 
\par 32 Jer Bog je sve zatvorio u neposlušnost  da se svima smiluje. 
\par 33 O dubino bogatstva, i mudrosti, i spoznanja Božjega!  Kako li su nedokučivi sudovi i neistraživi putovi njegovi! 
\par 34 Doista, tko spozna misao Gospodnju, tko li mu bi savjetnikom? 
\par 35 Ili: tko ga darom preteče da bi mu se uzvratiti moralo? 
\par 36 Jer sve je od njega i po njemu i za njega! Njemu slava  u vjekove! Amen. 


\chapter{12}

\par 1 Zaklinjem vas, braćo, milosrđem Božjim: prikažite svoja tijela  za žrtvu živu, svetu, Bogu milu - kao svoje duhovno bogoslužje. 
\par 2 Ne suobličujte se ovomu svijetu, nego se preobrazujte obnavljanjem  svoje pameti da mognete razabirati što je volja Božja, što li  je dobro, Bogu milo, savršeno. 
\par 3 Da, po milosti koja mi je dana svakomu između vas velim:  ne precjenjujte se više no što se treba cijeniti, nego cijenite  se razumno, kako je već komu Bog odmjerio mjeru vjere. 
\par 4 Jer  kao što u jednom tijelu imamo mnogo udova, a nemaju svi isto  djelovanje, 
\par 5 tako smo i mi, mnogi, jedno tijelo u Kristu, a  pojedinci udovi jedan drugomu. 
\par 6 Dare pak imamo različite po  milosti koja nam je dana: je li to prorokovanje - neka je primjereno  vjeri; 
\par 7 je li služenje - neka je u služenju; je li poučavanje  - u poučavanju; 
\par 8 je li hrabrenje - u hrabrenju; tko dijeli, neka je darežljiv; tko je predstojnik - revan; tko iskazuje  milosrđe - radostan! 
\par 9 Ljubav nehinjena! Zazirite oda zla, prianjajte uz dobro! 
\par 10 Srdačno se ljubite pravim bratoljubljem! Pretječite jedni  druge poštovanjem! 
\par 11 U revnosti budite hitri, u duhu gorljivi, Gospodinu služite! 
\par 12 U nadi budite radosni, u nevolji strpljivi, u molitvi postojani! 
\par 13 Pritječite u pomoć svetima u nuždi, gajite gostoljubivost! 
\par 14 Blagoslivljajte svoje progonitelje, blagoslivljajte,  a ne proklinjite! 
\par 15 Radujte se s radosnima, plačite sa zaplakanima! 
\par 16 Budite istomišljenici među sobom! Neka vas ne zanosi što  je visoko, nego privlači što je ponizno. Ne umišljajte si da  ste mudri! 
\par 17 Nikome zlo za zlo ne vraćajte; zauzimajte se  za dobro pred svim ljudima! 
\par 18 Ako je moguće, koliko je  do vas, u miru budite sa svim ljudima! 
\par 19 Ne osvećujte se, ljubljeni, nego dajte mjesta Božjem  gnjevu. Ta pisano je: Moja je odmazda, ja ću je vratiti,  veli Gospodin. 
\par 20 Naprotiv: Ako je gladan neprijatelj tvoj, nahrani ga, i ako je žedan, napoj ga! Činiš li tako, ugljevlje mu ražareno zgrćeš na glavu. 
\par 21 Ne daj se pobijediti zlom, nego dobrim svladavaj zlo. 


\chapter{13}

\par 1 Svaka duša neka se podlaže vlastima nad sobom. Jer nema vlasti  doli od Boga: koje postoje, od Boga su postavljene. 
\par 2 Stoga  tko se suprotstavlja vlasti, Božjoj se odredbi protivi; koji  se pak protive, sami će na se navući osudu. 
\par 3 Vladari doista  nisu strah i trepet zbog dobra, nego zbog zla djela. Hoćeš li  se ne bojati vlasti? Dobro čini pa ćeš imati pohvalu od nje! 
\par 4 Ta Božji je ona poslužitelj - tebi na dobro. Ako li zlo činiš, strahuj! Ne nosi uzalud mača! Božji je ona poslužitelj: gnjev  njegov iskaljuje na onome koji zlo čini. 
\par 5 Treba se stoga podlagati, ne samo zbog gnjeva nego i zbog savjesti. 
\par 6 Zato i poreze plaćate:  ta službenici su Božji oni koji se time bave. 
\par 7 Dajte svakomu  što mu pripada: komu porez - porez, komu carina - carina, komu  poštovanje - poštovanje, komu čast - čast. 
\par 8 Nikomu ništa ne dugujte, osim da jedni druge ljubite.  Jer tko drugoga ljubi, ispunio je Zakon. 
\par 9 Uistinu: Ne čini  preljuba! Ne ubij! Ne ukradi! Ne poželi! i ima li koja druga  zapovijed, sažeta je u ovoj riječi: Ljubi svoga bližnjega  kao sebe samoga. 
\par 10 Ljubav bližnjemu zla ne čini. Punina  dakle Zakona jest ljubav. 
\par 11 To tim više što shvaćate ovaj čas: vrijeme je već da  se oda sna prenemo jer nam je sada spasenje bliže nego kad povjerovasmo. 
\par 12 Noć poodmače, dan se približi! Odložimo dakle djela tame  i zaodjenimo se oružjem svjetlosti. 
\par 13 Kao po danu pristojno  hodimo, ne u pijankama i pijančevanjima, ne u priležništvima  i razvratnostima, ne u svađi i ljubomoru, 
\par 14 nego zaodjenite  se Gospodinom Isusom Kristom i, u brizi za tijelo, ne pogodujte  požudama. 


\chapter{14}

\par 1 Slaboga u vjeri prigrlite, ali ne da se prepirete o mišljenjima. 
\par 2 Netko vjeruje da smije sve jesti, slabi opet jede samo povrće. 
\par 3 Tko jede, neka ne prezire onoga tko ne jede; tko pak ne jede, neka ne sudi onoga tko jede. Ta Bog ga je prigrlio. 
\par 4 Tko si  ti da sudiš tuđega slugu? Svojemu Gospodaru i stoji i pada! A  stajat će jer moćan je Gospodin da ga podrži. 
\par 5 Netko razlikuje dan od dana, nekomu je opet svaki dan  jednak. Samo nek je svatko posve uvjeren u svoje mišljenje. 
\par 6 Tko  na dan misli, poradi Gospodina misli; i tko jede, poradi Gospodina  jede: zahvaljuje Bogu. I tko ne jede, poradi Gospodina ne jede  i - zahvaljuje Bogu. 
\par 7 Jer nitko od nas sebi ne živi, nitko sebi ne umire. 
\par 8 Doista, ako živimo, Gospodinu živimo, i ako umiremo, Gospodinu umiremo.  Živimo li dakle ili umiremo - Gospodinovi smo. 
\par 9 Ta Krist zato  umrije i oživje da gospodar bude i mrtvima i živima. 
\par 10 A ti, što sudiš brata svoga? Ili ti, što prezireš brata svoga? Ta  svi ćemo stati pred sudište Božje. 
\par 11 Jer pisano je: Života mi moga, govori Gospodin, prignut će se preda mnom svako koljeno i svaki će jezik priznati Boga. 
\par 12 Svaki će dakle od nas za sebe Bogu dati račun. 
\par 13 Dakle, ne sudimo više jedan drugoga, nego radije sudite  o tome da ne valja postavljati bratu stupice ili spoticala. 
\par 14 Znam  i uvjeren sam u Gospodinu: ništa samo od sebe nije nečisto. Samo  je onomu nečisto tko to smatra nečistim. 
\par 15 Doista, ako je poradi  hrane tvoj brat ražalošćen, već nisi na putu ljubavi. Ne upropašćuj  tom svojom hranom onoga za koga je Krist umro! 
\par 16 Nemojte da se pogrđuje vaše dobro! 
\par 17 Ta kraljevstvo  Božje nije jelo ili piće, nego pravednost, mir i radost u Duhu  Svetome. 
\par 18 Da, tko tako Kristu služi, mio je Bogu i cijene  ga ljudi. 
\par 19 Nastojmo stoga promicati mir i uzajamno izgrađivanje! 
\par 20 Ne razaraj djela Božjega poradi hrane! Sve je, istina, čisto, ali je zlo za onoga tko na sablazan jede. 
\par 21 Dobro je ne jesti  mesa i ne piti vina i ne uzimati ništa o što se tvoj brat spotiče. 
\par 22 Ti imaš uvjerenje. Za sebe ga imaj pred Bogom. Blago  onomu tko samoga sebe ne osuđuje u onom na što se odlučuje! 
\par 23 Jede  li tko dvoumeći, osudio se jer ne radi iz uvjerenja. A sve što  nije iz uvjerenja, grijeh je. 


\chapter{15}

\par 1 Mi jaki treba da nosimo slabosti slabih, a ne da sebi ugađamo. 
\par 2 Svaki od nas neka ugađa bližnjemu na dobro, na izgrađivanje. 
\par 3 Ta ni Krist nije sebi ugađao, nego kao što je pisano: Poruge  onih koji se rugaju tebi padoše na me. 
\par 4 Uistinu, što je nekoć napisano, nama je za pouku napisano  da po postojanosti i utjesi Pisama imamo nadu. 
\par 5 A Bog postojanosti  i utjehe dao vam da međusobno budete složni po Kristu Isusu 
\par 6 te  jednodušno, iz jednoga grla, slavite Boga i Oca Gospodina našega  Isusa Krista. 
\par 7 Prigrljujte jedni druge kao što je Krist prigrlio vas  na slavu Božju. 
\par 8 Krist je, velim, postao poslužitelj obrezanika  za istinu Božju da ispuni obećanja dana ocima, 
\par 9 a pogani da  za milosrđe proslave Boga, kao što je pisano: Zato ću te slaviti među pucima i psalam pjevati tvome imenu. 
\par 10 I još veli: Kličite, puci, s njegovim narodom. 
\par 11 I još: Hvalite, svi puci, Gospodina, slavili ga svi narodi! 
\par 12 Izaija opet veli: Pojavit će se Jišajev izdanak, dignut da vlada narodima, u njemu je nada narodima. 
\par 13 A Bog nade napunio vas svakom radošću i mirom u vjeri da  izobilujete u nadi snagom Duha Svetoga. 
\par 14 Ja sam, braćo moja, uvjeren: vi ste i sami puni čestitosti, ispunjeni svakim znanjem, sposobni jedni druge urazumljivati. 
\par 15 Ipak vam djelomično smionije napisah da vas na poznato nekako  podsjetim poradi milosti koja mi je dana od Boga - 
\par 16 da budem  bogoslužnik Krista Isusa među poganima, svećenik evanđelja Božjega  te prinos pogana postane ugodan, posvećen Duhom Svetim. 
\par 17 Imam se dakle čime dičiti u Kristu Isusu s obzirom na  ono što je Božje. 
\par 18 / 
\par 19 Jer ne bih se usudio govoriti o nečemu  što Krist riječju i djelom, snagom znamenja i čudesa, snagom  Duha nije po meni učinio da k poslušnosti privede pogane. Tako  sam od Jeruzalema pa uokolo sve do Ilirika pronio evanđelje Kristovo, 
\par 20 i to tako da sam se trsio navješćivati evanđelje ne gdje  se već spominjao Krist - da ne bih gradio na temeljima drugih  - 
\par 21 nego, kako je pisano: Vidjet će ga oni kojima nije naviješten, shvatiti oni koji za nj nisu čuli. 
\par 22 Time sam ponajčešće i bio spriječen doći k vama. 
\par 23 Sad  mi pak više nema mjesta u ovim krajevima, a živa mi je želja, ima već mnogo godina, doći k vama 
\par 24 kad pođem u Španjolsku.  Nadam se doista da ću vas na proputovanju posjetiti i da ćete  me onamo otpraviti pošto mi se najprije bar donekle ispuni želja  biti s vama. 
\par 25 Ali sad idem u Jeruzalem da poslužim svetima. 
\par 26 Makedonija  i Ahaja odlučiše očitovati neko zajedništvo prema siromašnim  svetima u Jeruzalemu. 
\par 27 Da, odlučiše, a i dužnici su im. Jer  ako su pogani postali sudionicima njihovih duhovnih dobara, dužni  su im u tjelesnima poslužiti. 
\par 28 Pošto dakle to obavim - ovaj im plod zapečaćen uručim  - uputit ću se u Španjolsku i usput k vama. 
\par 29 A kada dođem  k vama, doći ću, znam, s puninom blagoslova Kristova. 
\par 30 Ali zaklinjem vas, braćo, Gospodinom Isusom Kristom i  ljubavlju Duha: suborci mi budite u molitvama Bogu upravljenima  za me, 
\par 31 da umaknem onim nevjernima u Judeji i da moja pomoć  Jeruzalemu bude po volji svetima 
\par 32 te s Božjom voljom radosno  dođem k vama i s vama zajedno odahnem. 
\par 33 Bog mira sa svima vama! Amen. 



\chapter{16}

\par 1 Preporučujem vam Febu, sestru našu, poslužiteljicu Crkve u  Kenhreji: 
\par 2 primite je u Gospodinu kako dolikuje svetima i priskočite  joj u pomoć u svemu što od vas ustreba jer je i ona bila zaštitnicom  mnogima i meni samomu. 
\par 3 Pozdravite Prisku i Akvilu, suradnike moje u Kristu Isusu. 
\par 4 Oni su za moj život podmetnuli svoj vrat; zahvaljujem im ne  samo ja nego i sve Crkve pogana. 
\par 5 Pozdravite i Crkvu u njihovoj  kući.  Pozdravite ljubljenog mi Epeneta koji je prvina Azije za  Krista. 
\par 6 Pozdravite Mariju koja se mnogo trudila za vas. 
\par 7 Pozdravite  Andronika i Juniju, rođake i suuznike moje; oni su ugledni među  apostolima i prije mene bili su u Kristu. 
\par 8 Pozdravite Amplijata, ljubljenoga moga u Gospodinu. 
\par 9 Pozdravite Urbana, suradnika  moga u Kristu, i ljubljenog mi Staha. 
\par 10 Pozdravite Apela, prokušanoga  u Kristu. Pozdravite Aristobulove. 
\par 11 Pozdravite Herodiona,  rođaka moga. Pozdravite Narcisove koji su u Gospodinu. 
\par 12 Pozdravite  Trifenu i Trifozu koje se trude u Gospodinu. Pozdravite ljubljenu  Persidu koja se mnogo trudila u Gospodinu. 
\par 13 Pozdravite Rufa, izabranika u Gospodinu, i majku njegovu i moju. 
\par 14 Pozdravite  Asinkrita, Flegonta, Herma, Patrobu, Hermu i braću koja su s  njima. 
\par 15 Pozdravite Filologa i Juliju, Nereja i njegovu sestru, i Olimpu, i sve svete koji su s njima. 
\par 16 Pozdravite jedni druge cjelovom svetim. Pozdravljaju vas  sve Crkve Kristove. 
\par 17 Zaklinjem vas, braćo, čuvajte se onih koji siju razdore  i sablazni mimo nauk u kojem ste poučeni, i klonite ih se. 
\par 18 Jer  takvi ne služe Gospodinu našemu Kristu, nego svom trbuhu te lijepim  i laskavim riječima zavode srca nedužnih. 
\par 19 Doista, vaša je  poslušnost doprla do sviju. Zbog vas se dakle radujem i htio  bih da budete mudri za dobro, a bezazleni za zlo. 
\par 20 Bog mira  satrt će ubrzo Sotonu pod vašim nogama. Milost Gospodina Isusa  s vama! 
\par 21 Pozdravlja vas Timotej, suradnik moj, i Lucije, Jason  i Sosipater, rođaci moji. 
\par 22 Pozdravljam vas u Gospodinu ja, Tercije, koji napisah ovu  poslanicu. 
\par 23 Pozdravlja vas Gaj, gostoprimac moj i cijele Crkve. Pozdravlja  vas Erast, gradski blagajnik, i brat Kvart. 
\par 24 # 
\par 25 Onomu koji vas može učvrstiti - po mojem evanđelju i  propovijedanju Isusa Krista, po objavljenju Otajstva prešućenog  drevnim vremenima, 
\par 26 a sada očitovanog i po proročkim pismima  odredbom vječnoga Boga svim narodima obznanjenog za poslušnost, vjeru - 
\par 27 jedinomu Mudromu, Bogu, po Isusu Kristu: Njemu slava  u vijeke! Amen. 





\end{document}