\begin{document}

\title{Titu}


\chapter{1}

\par 1 Pavao, sluga Božji i apostol Isusa Krista poradi vjere izabranika  Božjih i spoznanja istine usmjerene k pobožnosti 
\par 2 u nadi života  vječnoga što ga, prije vremena vjekovječnih, obeća Bog, On koji  ne laže, 
\par 3 a u svoje doba očitova riječ svoju u propovijedanju  koje je meni povjereno po odredbi Spasitelja našega, Boga: 
\par 4 Titu, pravomu sinu po zajedničkoj vjeri, milost i mir od Boga i Krista  Isusa, Spasitelja našega! 
\par 5 Poradi toga ostavih te na Kreti da urediš preostalo te  po gradovima postaviš starješine kako sam ti ja odredio: 
\par 6 je  li tko besprigovoran, jedne žene muž, jesu li mu djeca vjernici  i ne pod optužbom raskalašenosti ili nepokorna... 
\par 7 Jer nadstojnik  kao Božji upravitelj treba da bude besprigovoran: ne samoživ, ne jedljiv, ne vinu sklon, ni nasilju, ni prljavu dobitku, 
\par 8 nego  gostoljubiv, ljubitelj dobra, razuman, pravedan, svet, uzdržljiv, 
\par 9 priljubljen uz vjerodostojnu riječ nauka da može i hrabriti  u zdravom nauku i uvjeravati protivnike. 
\par 10 Jer mnogi su nepokorni, praznorječni i zavodnici, ponajpače  oni iz obrezanja. 
\par 11 Njima treba začepiti usta jer cijele domove  prevraćaju naučavajući što ne bi smjeli, i to poradi prljava  dobitka. 
\par 12 Reče netko od njih, njihov vlastiti prorok: "Krećani  uvijek lašci, opake zvijeri, trbusi dangubni." 
\par 13 Svjedočanstvo je to istinito. Zato ih karaj oštro da  budu zdravi u vjeri, 
\par 14 da ne prianjaju uza židovske bajke i  propise ljudi koji se odvraćaju od istine. 
\par 15 Sve je čisto čistima; okaljanima pak i nevjernima ništa  čisto, nego su im okaljani i razum i savjest. 
\par 16 Ispovijedaju  da Boga poznaju, ali djelima ga niječu - odvratni, neposlušni  i za koje god dobro djelo nepodesni. 


\chapter{2}

\par 1 Ti, naprotiv, govori što se priliči zdravu nauku: 
\par 2 starci  da budu trijezni, ozbiljni, razumni, zdrave vjere, ljubavi, postojanosti; 
\par 3 starice isto tako - vladanja kakvo dolikuje svetima: ne klevetnice, ne ropkinje mnogog vina, nego učiteljice dobra 
\par 4 da urazumljuju  mlađe neka ljube svoje muževe, djecu, 
\par 5 neka budu razumne, čiste, kućevne, dobre, podložne svojim muževima da se riječ Božja ne  bi pogrđivala. 
\par 6 Mladiće isto tako potiči da budu razumni. 
\par 7 U svemu se  pokaži uzorom dobrih djela: u poučavanju - nepokvarljivost, ozbiljnost, 
\par 8 riječ zdrava, besprigovorna da se onaj nasuprot postidi nemajući  o nama reći ništa zlo. 
\par 9 Robovi neka se svojim gospodarima u svemu podlažu, ugađaju  im, ne proturječe, 
\par 10 ne pronevjeruju, nego neka im iskazuju  svaku dobru vjernost da u svemu budu ures nauku Spasitelja našega, Boga. 
\par 11 Pojavila se doista milost Božja, spasiteljica svih ljudi; 
\par 12 odgojila nas da se odreknemo bezbožnosti i svjetovnih požuda  te razumno, pravedno i pobožno živimo u sadašnjem svijetu, 
\par 13 iščekujući  blaženu nadu i pojavak slave velikoga Boga i Spasitelja našega  Isusa Krista. 
\par 14 On sebe dade za nas da nas otkupi od svakoga  bezakonja i očisti sebi Narod izabrani koji revnuje oko dobrih  djela. 
\par 15 To govori, zapovijedaj, karaj sa svom vlašću. Nitko neka  te ne prezire. 



\chapter{3}

\par 1 Podsjećaj ih da se podlažu poglavarstvima, vlastima, da slušaju, da budu spremni na svako dobro djelo, 
\par 2 nikoga da ne pogrđuju, da budu neratoborni, popustljivi, da očituju svaku blagost prema  svim ljudima. 
\par 3 Jer i mi nekoć bijasmo nerazumni, nepokorni, lutalice, robovi raznih požuda i naslada, živjeli smo u zlu  i zavisti, odvratni bili, mrzili jedni druge. 
\par 4 Ali kad se pojavila dobrostivost i čovjekoljublje Spasitelja  našega, Boga, 
\par 5 on nas spasi ne po djelima što ih u pravednosti  mi učinismo, nego po svojem milosrđu: kupelji novoga rođenja  i obnavljanja po Duhu Svetom 
\par 6 koga bogato izli na nas po Isusu  Kristu, Spasitelju našemu, 
\par 7 da opravdani njegovom milošću budemo, po nadi, baštinici života vječnoga. 
\par 8 Vjerodostojna je ovo riječ i hoću da to uporno tvrdiš  te da oni koji su povjerovali Bogu uznastoje prednjačiti dobrim  djelima. To je dobro i korisno ljudima. 
\par 9 A ludih se raspra, i rodoslovlja, i svađa, i sukoba zakonskih  kloni: beskorisni su i isprazni. 
\par 10 S krivovjercem nakon prvoga  i drugog upozorenja prekini 
\par 11 znajući da je izopačen i da griješi:  on sam sebe osuđuje. 
\par 12 Kad pošaljem k tebi Artemu ili Tihika, požuri se k meni  u Nikopol jer sam odlučio ondje prezimiti. 
\par 13 Zenu, pravnika, i Apolona brižljivo opremi da im ništa ne ponestane. 
\par 14 A i  naši neka se uče prednjačiti dobrim djelima u životnim potrebama  da ne budu neplodni. 
\par 15 Pozdravljaju te svi koji su sa mnom. Pozdravi one koji nas ljube u vjeri. Milost sa svima vama! 




\end{document}