\begin{document}

\title{Suci}


\chapter{1}

\par 1 Poslije smrti Jošuine upitaše Izraelci Jahvu: "Tko će od nas  prvi poći na Kanaance da se protiv njih bori?" 
\par 2 A Jahve odgovori:  "Neka Juda prvi pođe; u njegove ruke stavljam zemlju." 
\par 3 Tada  Juda reče svome bratu Šimunu: "Pođi sa mnom u zemlju koja mi  je dosuđena u baštinu; borit ćemo se protiv Kanaanaca, a potom  ću se ja uza te boriti na tvojoj zemlji." I Šimun ode s njim. 
\par 4 Ode Juda i Jahve im predade u ruke Kanaance i Perižane te  pobiše u Bezeku deset tisuća ljudi. 
\par 5 U Bezeku zatekoše Adoni-Sedeka, udariše na nj i poraziše Kanaance i Perižane. 
\par 6 Kad je Adoni-Sedek  nagnuo u bijeg, gonili su ga, uhvatili ga i odsjekli mu palce  na rukama i nogama. 
\par 7 Tada reče Adoni-Sedek: "Sedamdeset kraljeva  odsječenih palaca na rukama i na nogama kupilo je mrvice pod  mojim stolom. Kako sam činio, tako mi Bog vraća." Odveli su ga  u Jeruzalem i ondje je umro. 
\par 8 Zatim Judini sinovi udariše na  Jeruzalem, osvojiše ga, posjekoše mačem žitelje i spališe grad. 
\par 9 Poslije toga krenuše Judini sinovi da se bore protiv Kanaanaca  koji su živjeli u Gorju, Negebu i u Šefeli. 
\par 10 Onda Juda ode  na Kanaance koji su živjeli u Hebronu - Hebronu bijaše nekoć  ime Kirjat Arba - i ondje potuče Šešaja, Ahimana i Talmaja. 
\par 11 Odatle  krenu na stanovnike Debira, koji se nekoć zvao Kirjat Sefer. 
\par 12 Tada reče Kaleb: "Tko pokori i zauzme Kirjat Sefer, dat ću  mu svoju kćer Aksu za ženu." 
\par 13 Zauze ga Otniel, sin Kenaza, mlađeg brata Kalebova, i Kaleb mu dade svoju kćer Aksu za ženu. 
\par 14 Kad je prišla mužu, on je nagovori da u svoga oca ište polje.  Siđe ona s magarca, a Kaleb je upita: "Što hoćeš?" 
\par 15 Ona mu  odgovori: "Daj mi blagoslov! Kad si mi dao kraj u Negebu, daj  mi onda i koji izvor vode." I Kaleb joj dade Gornje i Donje izvore. 
\par 16 Sinovi Hobaba Kenijca, tasta Mojsijeva, odoše iz Palmova  grada s Judinim sinovima u Judinu pustinju, koja je u Negebu, na jugu od Arada. Tu se nastaniše među Amalečanima. 
\par 17 Potom  ode Juda s bratom Šimunom i pobiše Kanaance koji su živjeli u  Sefatu i grad izručiše "heremu", prokletstvu. Zbog toga se grad  prozva Horma. 
\par 18 Ali Juda nije uspio zauzeti Gaze s njenim područjem, ni Aškelona s njegovim područjem, ni Ekrona s njegovim područjem. 
\par 19 Jahve bijaše s njim te on osvoji gorje, ali ne mogaše potjerati  onih u nizini jer imahu željezna kola. 
\par 20 Kao što bijaše odredio  Mojsije, dadoše Hebron Kalebu, koji iz njega otjera tri sina  Anakova. 
\par 21 A Benjaminovi sinovi ne uspješe otjerati Jebusejaca  koji su živjeli u Jeruzalemu i tako Jebusejci ostadoše u Jeruzalemu  s Benjaminovim sinovima do dana današnjega. 
\par 22 Krenu i pleme Josipovo na Betel i Jahve bijaše s njima. 
\par 23 I pleme Josipovo uze izviđati Betel. Grad se nekoć zvao Luz. 
\par 24 Uhode opaziše čovjeka gdje izlazi iz grada i rekoše mu: "Pokaži  nam kuda se može u grad, pa ćemo ti biti milostivi." 
\par 25 On im  pokaza kuda mogu u grad. I sve u gradu isjekoše mačem, a onoga  čovjeka sa svom njegovom obitelji pustiše da ode. 
\par 26 Čovjek  je otišao u zemlju Hetita i ondje sagradio grad i prozvao ga  Luz. Tako se zove još i danas. 
\par 27 Manaše nije osvojio Bet-Šeana i njegovih sela ni Tanaka  i njegovih sela. Nije potjerao ni stanovnika iz Dora i njegovih  sela, ni stanovnika Jibleama i njegovih sela, ni stanovnika Megida  i njegovih sela. Tako su Kanaanci ostali i živjeli u toj zemlji. 
\par 28 Kad je Izrael ojačao, nametnuo je Kanaancima tlaku, ali ih  nije mogao otjerati. 
\par 29 Ni Efrajim nije otjerao Kanaanaca koji  su živjeli u Gezeru, tako te su Kanaanci tu živjeli među njima. 
\par 30 Zebulun nije otjerao stanovnika Kitrona ni stanovnika Nahalola.  Tako su Kanaanci ostali usred Zebulunovih sinova, ali im bijaše  nametnuta tlaka. 
\par 31 Ni Ašer nije otjerao stanovnika Akona, ni  stanovnika Sidona, ni onih iz Mahalaba, Akziba, Helbe, Afika  i Rehoba. 
\par 32 Ašerovci su ostali tako među Kanaancima, stanovnicima  te zemlje, jer ih nisu otjerali. 
\par 33 Naftali nije otjerao stanovnika  Bet-Šemeša i Bet-Anata, nego je živio među Kanaancima koji su  nastavali tu zemlju, ali je stanovnicima Bet-Šemeša i Bet-Anata  nametnuta tlaka. 
\par 34 Amorejci su potisnuli Danove sinove u goru  i nisu ih puštali da siđu u ravnicu. 
\par 35 Amorejci su se zadržali  u Har-Heresu, Ajalonu i Šaalbimu, ali kad je ruka Josipova doma  ojačala, bila im je nametnuta tlaka. 
\par 36 Područje Edomaca pruža se od Akrabimskog uspona do Stijene  pa naviše. 


\chapter{2}

\par 1 Anđeo Jahvin dođe iz Gilgala u Bokim i reče: "Izveo sam vas  iz Egipta i doveo vas u zemlju koju sam vam obećao zaklevši se  ocima vašim. Rekao sam: 'Neću raskinuti Saveza svog s vama dovijeka. 
\par 2 A vi ne sklapajte saveza sa stanovnicima ove zemlje; nego  rušite njihove žrtvenike!' Ali vi niste poslušali moga glasa.  Što ste učinili? 
\par 3 Zato vam kažem: neću ih odagnati pred vama.  Nego, oni će vas tlačiti i bogovi njihovi bit će vam zamkom." 
\par 4 Kad Anđeo Jahvin izreče te riječi svim Izraelcima, narod zakuka  i zaplaka. 
\par 5 I tako prozvaše ono mjesto Bokim i ondje prinesoše  žrtve Jahvi. 
\par 6 Tada Jošua otpusti narod i raziđoše se Izraelci svaki  na svoju baštinu da zaposjednu zemlju. 
\par 7 Narod je služio Jahvi  svega vijeka Jošuina i svega vijeka starješina koje su nadživjele  Jošuu i vidjele sva velika djela što ih je Jahve učinio Izraelu. 
\par 8 Jošua, sin Nunov, sluga Jahvin, umrije u dobi od sto deset  godina. 
\par 9 Sahraniše ga u kraju što ga je baštinio u Timnat Heresu, u Efrajimovoj gori, sjeverno od planine Gaaša. 
\par 10 A kada se  sav onaj naraštaj pridružio svojim ocima, naslijedi ga drugi  naraštaj koji nije mario za Jahvu ni za djela što ih je učinio  Izraelu. 
\par 11 Tada su sinovi Izraelovi počeli činiti ono što Jahvi  nije po volji i služili su baalima. 
\par 12 Ostaviše Jahvu, Boga  otaca svojih, koji ih je izveo iz zemlje egipatske, i pođoše  za drugim bogovima između bogova okolnih naroda. Klanjahu im  se, razgnjeviše Jahvu. 
\par 13 Otpali su od Jahve da bi služili Baalu  i Aštarti. 
\par 14 Zato Jahve izli gnjev svoj na Izraela: prepusti  ih pljačkašima da ih plijene, izruči ih neprijateljima uokolo, tako te se ne mogoše oduprijeti. 
\par 15 Što bi god počeli, ruka  se Jahvina okretala protiv njih na njihovu nesreću, kao što im  je Jahve rekao i kao što im se zakleo. I tako zapadoše u veliku  nevolju. 
\par 16 Tada im Jahve stade podizati suce da ih izbavljaju iz  ruku onih koji su ih pljačkali. 
\par 17 Ali oni ni svojih sudaca  nisu slušali, nego se iznevjeriše s drugim bogovima te im se  klanjahu. Brzo su zašli s puta kojim su išli oci njihovi slušajući  Jahvine zapovijedi; oni nisu činili tako. 
\par 18 Kada im je podizao  suce, Jahve bijaše sa svakim sucem te ih izbavljaše iz ruku njihovih  neprijatelja za svega vijeka sučeva, jer se sažalilo Jahvi koliko  su uzdisali pod jarmom onih koji su ih ugnjetavali. 
\par 19 A kada  bi sudac umro, oni bi opet zapadali u veću pokvarenost nego njihovi  oci. Išli su za drugim bogovima, služili im i klanjali im se, ne odustajući od svojih opakih djela i postupaka. 
\par 20 Tada Jahve planu gnjevom na Izraela i reče: "Kad je taj  narod pogazio Savez kojim sam obvezao njihove očeve i nije poslušao  glasa moga, 
\par 21 ni ja odsad neću pred njim potjerati ni jednoga  između naroda što ih je Jošua po svojoj smrti ostavio", 
\par 22 da  bi njima stavio na kušnju Izraela: hoće li se ili neće držati  Jahvinih putova kao što ih se držahu oci njihovi. 
\par 23 Zato Jahve  bijaše ostavio te narode i nije ih odmah izagnao ni predao Jošui  u ruke. 


\chapter{3}

\par 1 Ovo su narodi koje je Jahve pustio da ostanu kako bi njima  iskušavao sinove Izraelove, sve one koji ne iskusiše ratova kanaanskih. 
\par 2 Bijaše to samo na korist pokoljenjima sinova Izraelovih da  nauče vještinu  ratovanja - barem oni koji nisu iskusili prijašnjih  ratova: 
\par 3 ostade pet knezova filistejskih i svi Kanaanci, Sidonci  i Hivijci koji su živjeli na gori Libanonu od gore Baal-Hermona  do ulaza u Hamat. 
\par 4 Oni su poslužili da se iskuša Izrael: da  bi se vidjelo hoće li se držati zapovijedi što ih je Jahve preko  Mojsija dao njihovim ocima. 
\par 5 Tako su Izraelci prebivali usred  Kanaanaca, Hetita, Amorejaca, Perižana, Hivijaca i Jebusejaca; 
\par 6 ženili se njihovim kćerima i davali svoje kćeri njihovim sinovima  i služili njihovim bogovima. 
\par 7 I činili su Izraelci ono što Jahvi nije bilo po volji.  Zaboravili su Jahvu, svoga Boga, da bi služili baalima i aštartama. 
\par 8 Tada Jahve planu gnjevom na Izraela i dade ih u ruke Kušanu  Rišatajimu, kralju edomskom; i služiše Kušanu Rišatajimu osam  godina. 
\par 9 Tad Izraelci zavapiše Jahvi i Jahve im podiže izbavitelja, Otniela, sina Kenaza, mlađega brata Kalebova, da ih oslobodi. 
\par 10 Duh Jahvin siđe na nj i on posta sucem Izraelu. I povede  Izraela u boj. Jahve mu preda u ruke Kušana Rišatajima, kralja  edomskog, i on pobijedi Kušana Rišatajima. 
\par 11 Zemlja je otad  bila u miru četrdeset godina. Poslije smrti Otniela, sina Kenazova, 
\par 12 Izraelci su počeli  opet činiti što je zlo u očima Jahvinim. Zato Jahve dade Eglonu, kralju moapskom, moć nad Izraelom, jer su činili što je zlo  pred Jahvom. 
\par 13 Eglon se ujedini sa sinovima Amonovim i Amalekovim, pođe na Izraela, potuče ga i osvoji Palmov grad. 
\par 14 Izraelci  su služili moapskom kralju Eglonu osamnaest godina. 
\par 15 Tada  Izraelci zavapiše Jahvi i Jahve im podiže izbavitelja - Ehuda, sina Gere iz Benjaminova plemena, čovjeka koji bijaše ljevak.  I poslaše ga Izraelci da im odnese danak Eglonu, kralju moapskom. 
\par 16 A Ehud načini sebi bodež sa dvije oštrice, lakat dug, i pripasa  ga pod haljine uz desno bedro. 
\par 17 I odnese danak Eglonu, kralju  moapskom. Eglon bijaše vrlo debeo. 
\par 18 Predavši danak, Ehud ode s ljudima koji bijahu donijeli  danak. 
\par 19 Ali kada je došao do idola u blizini Gilgala, vrati  se i reče: "Imam ti, kralju, reći jednu tajnu!" Kralj mu odvrati:  "Tiho!" I svi koji su uza nj bili izađu. 
\par 20 Ehud uđe. Kralj je sjedio u hladovitoj gornjoj sobi;  bio je sam. Ehud mu reče: "Imam, kralju, za tebe riječ od Boga!"  On odmah usta s prijestolja. 
\par 21 Tad Ehud lijevom rukom trgnu  bodež s desnog bedra i satjera mu ga u trbuh. 
\par 22 Za oštricom  uđe sav držak i salo se sklopi za oštricom, jer Ehud nije mogao  izvući oštricu iz trbuha. Nečist je izlazila odande. 
\par 23 Ehud  je otišao kroz trijem; za sobom je zatvorio vrata gornje sobe  i zaključao ih. 
\par 24 Kada je on otišao, vrate se sluge da pogledaju. Kako  vrata gornje sobe bijahu zaključana, rekoše: "Bit će da je otišao  na stranu, u klijet do hladovite sobe." 
\par 25 Čekali su ga dugo, u nedoumici, jer on nije otvarao vrata gornje sobe. Naposljetku  uzeše ključ i otvoriše: gospodar im ležao na tlu, mrtav. 
\par 26 Dok su oni čekali, Ehud je pobjegao, prošao već idole  i sklonio se u Seiru. 
\par 27 Čim dođe u zemlju Izraelovu, zasvira  u rog na Efrajimovoj gori; i siđoše Izraelci s njim s gore, a  on im stajaše na čelu. 
\par 28 I reče im: "Pođite za mnom! Jahve  vam je u ruke predao Moapce, vaše neprijatelje." Oni krenuše  za njim, zatvoriše Moapcima put preko gazova Jordana i ne dadoše  nikome prijeko. 
\par 29 Pobili su u to vrijeme oko deset tisuća Moabaca, sve kršnih i hrabrih ljudi, i nijedan im nije umakao. 
\par 30 Toga  su dana Moapci potpali pod ruku Izraelovu i zemlja bijaše mirna  osamdeset godina. 
\par 31 Poslije njega bijaše Šamgar, sin Anatov. On je pobio  šest stotina Filistejaca ostanom volujskim. Tako je i on spasio  Izraela. 


\chapter{4}

\par 1 Poslije smrti Ehudove Izraelci su opet stali činiti što Jahvi  nije po volji 
\par 2 i Jahve ih predade u ruke Jabinu, kanaanskom  kralju koji je vladao u Hasoru. Vojskovođa vojsci njegovoj bijaše  Sisera, koji je živio u Harošetu Poganskom. 
\par 3 Tad Izraelci zavapiše  Jahvi. Jer Jabin imaše devet stotina željeznih bojnih kola i  teško je tlačio Izraelce dvadeset godina. 
\par 4 U to vrijeme Izraelu je sudila proročica Debora, žena  Lapidotova. 
\par 5 Živjela je pod Deborinom palmom između Rame i  Betela u Efrajimovoj gori i k njoj su dolazili Izraelci da presuđuje  u njihovim sporovima. 
\par 6 Ona dozva Baraka, sina Abinoamova, iz  Naftalijeva Kedeša i reče mu: "Evo što ti Jahve, Bog Izraelov, zapovijeda: 'Idi, kreni na goru Tabor i uzmi sa sobom deset  tisuća ljudi između Naftalijevih i Zebulunovih sinova. 
\par 7 Ja  ću k tebi na Kišonski potok privući Siseru, vojskovođu Jabinove  vojske, s njegovim bojnim kolima i svim ratnicima te ću ga predati  u tvoje ruke.'" 
\par 8 Barak joj odgovori: "Ako ti pođeš sa mnom, ići ću; ako  li ne pođeš sa mnom, ne idem." 
\par 9 "Idem s tobom", reče mu ona, "ali na putu kojim ćeš poći slava neće tebi pripasti jer će  Jahve ženi predati u ruke Siseru." Tada Debora ustane i pođe  s Barakom u Kedeš. 
\par 10 Onamo je Barak pozvao Zebuluna i Naftalija.  Deset tisuća ljudi pođe za njim, a išla je s njim i Debora. 
\par 11 Heber Kenijac bijaše se odvojio od Kajina, jednoga od  sinova Hababa, tasta Mojsijeva; razapeo je svoj šator kod Hrasta  u Saananimu, nedaleko od Kedeša. 
\par 12 Javiše Siseri da je Barak, sin Abinoamov, izašao na goru  Tabor. 
\par 13 Nato Sisera sabra sva svoja kola, devet stotina željeznih  kola, i sve ljude koje je doveo od Harošeta Poganskog do Kišonskog  potoka. 
\par 14 Debora reče Baraku: "Ustani, evo dana kada će Jahve  predati Siseru u tvoje ruke! Sam Jahve ide pred tobom!" I Barak  siđe s gore Tabora sa deset tisuća ljudi za sobom. 
\par 15 Jahve  zastraši Siseru, sva njegova kola i čitavu njegovu vojsku, koja  naže u bijeg pred mačem Barakovim. Sisera siđe sa svojih kola  i pobježe pješice. 
\par 16 Barak je gonio kola i vojsku sve do Harošeta  Poganskog. Sva je Siserina vojska pala od oštrog mača i nijedan  čovjek nije umakao. 
\par 17 Sisera je dotle bježao pješice prema šatoru Jaele, žene  Hebera Kenijca, jer između Jabina, kralja hasorskog, i kuće Hebera  Kenijca bijaše mir. 
\par 18 Jaela iziđe Siseri u susret i reče mu:  "Zaustavi se, gospodaru, svrati se k meni. Ne boj se ničega!"  On svrati k njoj pod šator, a ona ga pokri pokrivačem. 
\par 19 On  joj reče: "Daj mi malo vode jer sam žedan." Ona otvori mijeh  s mlijekom, napoji ga i opet ga pokri. 
\par 20 "Stani na ulazu u  šator", reče joj on, "pa ako tko naiđe i zapita te: 'Ima li tu  koga?' ti odgovori: 'Nema!'" 
\par 21 A Jaela, žena Heberova, uze  šatorski klin i čekić u ruke, tiho mu se približi i zabi mu klin  kroza sljepoočice tako da se zario u zemlju. On od iscrpljenosti  bijaše tvrdo zaspao i tako umrije. 
\par 22 I gle, dođe Barak progoneći  Siseru. Jaela iziđe preda nj i reče mu: "Dođi da ti pokažem čovjeka  koga tražiš." On uđe k njoj, i gle - Sisera ležaše mrtav, s klinom  u sljepoočici. 
\par 23 Tako je Bog u onaj dan ponizio Jabina, kralja kanaanskog, pred Izraelcima. 
\par 24 Ruka Izraelaca postajaše sve teža Jabinu, kralju kanaanskom, dok ga nije napokon zatrla. 


\chapter{5}

\par 1 Toga dana Debora i Barak, sin Abinoamov, zapjevaše ovu pjesmu: 
\par 2 Ratoborno rasuše kose borci izraelski i dragovoljno krenu narod: blagoslivljajte Jahvu! 
\par 3 Čujte, o kraljevi! Poslušajte, knezovi! Jahvi ja pjesmu pjevam, Jahvu, Boga Izraelova, ja slavim. 
\par 4 Sa Seira kad si silazio, Jahve, pobjednički kad si kročio iz polja edomskih, sva se zemlja tresla, lila se nebesa, oblaci curkom daždjeli. 
\par 5 Brda se tresla pred tobom, o Jahve, Jahve, Bože Izraelov! 
\par 6 U dane Šamgara, sina Anatova, u dane Jaele opustješe putovi; i oni koji su putovali, obilažahu naokolo. 
\par 7 Pusta bijahu sela izraelska dok ne ustadoh ja, Debora, dok ne ustadoh kao majka Izraelu. 
\par 8 Tuđe bogove sebi izabraše, i zato im rat stade pred vrata. Za pet gradova ne bi nijednog štita! Nijednog kralja za četrdeset tisuća u Izraelu! 
\par 9 Srce moje kuca za vođe izraelske, za narod što dragovoljno u boj kreće! Blagoslivljajte Jahvu! 
\par 10 Vi koji na bijelim jašete magaricama, na sagovima sjedeći, i vi koji hodite putovima, pjevajte, 
\par 11 uz povike razdraganih pastira kod pojila. Neka se slave dobročinstva Jahvina i vladavina njegova Izraelom! I narod Jahvin siđe na vrata. 
\par 12 Probudi se, Deboro, ustani! Ustani, pjesmu zapjevaj! Hrabro! Ustani, Barače, vodi u roblje porobljivače svoje, sine Abinoamov! 
\par 13 Tad siđe na vrata Izrael, narod Jahvin pohrli junački. 
\par 14 Iz Efrajima potekoše u dolinu, za njima stiže među čete tvoje Benjamin. Iz Makira stupaju glavari, iz Zebuluna oni što nose štap zapovjednički. 
\par 15 Knezovi Jisakarovi s Deborom bjehu, a Naftali pođe s Barakom, pohrli da ga stigne u dolini. Kod Rubenovih potoka dugo se savjetuju. 
\par 16 Zašto si ostao u torovima da slušaš sred stada svirku frule? Kod Rubenovih potoka dugo se savjetuju. 
\par 17 Gilead osta s onu stranu Jordana. A zašto je Dan na stranim lađama? Zašto na obali mora Ašer sjedi, mirno prebiva u svojim zaljevima? 
\par 18 Zebulun je narod što prkosi smrti s Naftalijem, na visoravnima. 
\par 19 Došli su kraljevi, boj zametnuli, boj bili kraljevi kanaanski, u Tanaku, na vodi megidskoj, al' ni mrve srebra ne dobiše. 
\par 20 Sa nebeskih staza vojevahu, vojevahu zvijezde prot' Siseri. 
\par 21 Sve otplavi potok Kišon, potok Kišon pradavni. Gazi čvrsto, moja dušo! 
\par 22 Topot silan odjekuje: jure borci na konjima! 
\par 23 "Proklinjite Meroz," Anđeo će Jahvin, "proklinjite žitelje njegove što Jahvi nisu u pomoć pritekli, u pomoć Jahvi s junacima." 
\par 24 Blagoslovljena među ženama bila Jaela, žena Hebera Kenijca, među ženama šatora nek' je slavljena! 
\par 25 On vode zaiska, mlijeka mu ona dade, u zdjelu dragocjenu nali mu povlake. 
\par 26 Rukom lijevom za klinom segnu, a desnom za čekićem kovačkim. Udari Siseru, glavu mu razmrska, probode mu, razbi sljepoočicu. 
\par 27 Do nogu pade joj, sruši se, leže, do nogu pade joj, sruši se; i gdje pade, mrtav osta. 
\par 28 Kroz prozor motri Siserina mati, kroz prozor motri, na rešetku jÓada: "Dugo mu se kola ne vraćaju: što im je zapreg tako spor?" 
\par 29 Najmudrija zbori joj dvorkinja, sebi samoj ona odgovara: 
\par 30 "Plijen su našli pa ga dijele: po djevojku na ratnika, po djevojku i po dvije, halju-dvije za Siseru, vezen rubac za moj vrat!" 
\par 31 Tako neka ginu, Jahve, svi neprijatelji tvoji! A oni koji te ljube nek budu kao sunce kada se diže u svojemu sjaju! I zemlja bijaše mirna četrdeset godina. 


\chapter{6}

\par 1 Opet su Izraelci činili što je zlo u Jahvinim očima; i Jahve  ih predade u ruke Midjancima za sedam godina. 
\par 2 Teška bijaše  ruka Midjanaca nad Izraelom. Da bi izmakli Midjancima, Izraelci  se sklanjahu u gorske pukotine, spilje i skrovišta. 
\par 3 I kada  bi Izraelci posijali, dolazili bi na njih Midjanci i Amalečani  i sinovi Istoka. 
\par 4 Utaborivši se na njihovoj zemlji, uništavali  bi rod zemlje sve do Gaze. Ne ostavljahu Izraelu ništa da se  prehrani, ni ovce ni koze, ni vola ni magarca, 
\par 5 jer dolažahu  sa svojim stadima i svojim šatorima u takvu mnoštvu kao skakavci;  ne bijaše broja njima ni njihovim devama; preplavili bi zemlju, opustošili je. 
\par 6 Tako su Midjanci bacili Izraela u veliku bijedu  te Izraelci zavapiše Jahvi. 
\par 7 Kad su Izraelci zavapili Jahvi zbog Midjanaca, 
\par 8 Jahve  posla Izraelcima proroka koji im reče: "Ovako kaže Jahve, Bog  Izraelov: 'Ja sam vas izveo iz Egipta, izbavio vas iz kuće ropstva. 
\par 9 Ja sam vas oslobodio od ruke Egipćana i od ruke svih vaših  tlačitelja. Protjerao sam ih pred vama, dao vam njihovu zemlju 
\par 10 i rekao vam: Ja sam Jahve, Bog vaš. Ne štujte bogova Amorejaca  u kojih zemlji živite. Ali vi ne poslušaste moga glasa.'" 
\par 11 Anđeo Jahvin dođe i sjede pod hrast kod Ofre koji pripadaše  Joašu Abiezerovu. Njegov sin Gideon vrhao je pšenicu na tijesku  da bi je sačuvao od Midjanaca. 
\par 12 I ukaza mu se Anđeo Jahvin  i reče mu: "Jahve s tobom, hrabri junače!" 
\par 13 Gideon mu odgovori:  "Oh, gospodaru, ako je Jahve s nama, zašto nas sve ovo snađe?  Gdje su sva ona čudesa njegova o kojima nam pripovijedahu oci  naši govoreći: 'Nije li nas Jahve iz Egipta izveo?' A sada nas  je Jahve ostavio, predao nas u ruke Midjancima." 
\par 14 Jahve se tad okrenu prema njemu i reče mu: "Idi s tom  snagom u sebi i izbavit ćeš Izraela iz ruke Midjanaca. Ne šaljem  li te ja?" 
\par 15 "Ali, gospodaru", odgovori mu Gideon, "kako ću  izbaviti Izraela? Moj je rod najmanji u Manašeovu plemenu, a  ja sam posljednji u kući svoga oca." 
\par 16 Jahve mu reče: "Ja ću  biti s tobom te ćeš pobijediti Midjance kao jednoga." 
\par 17 Gideon  mu reče: "Ako sam našao milost u tvojim očima, daj mi znak da  ti govoriš sa mnom. 
\par 18 Nemoj otići odavde dok se ne vratim s  darom i stavim ga preda te." A on odgovori: "Ostat ću dok se  ne vratiš." 
\par 19 Gideon ode, zgotovi jare i od efe brašna načini beskvasne  hljebove, stavi meso u košaricu i juhu u lonac pa donese sve  to pod hrast. 
\par 20 Anđeo Jahvin reče mu: "Uzmi meso i beskvasne  hljebove, stavi ih na tu stijenu, a juhu prolij." On učini tako. 
\par 21 Anđeo Jahvin tad uze štap što ga je držao i vrhom dotaknu  meso i beskvasne hljebove. Oganj planu iz stijene, spali meso  i beskvasne hljebove. Anđeo Jahvin nato iščeze pred njegovim  očima. 
\par 22 Tad Gideon vidje da je to bio Anđeo Jahvin i reče:  "Jao, Jahve, Gospode! Anđela Jahvina vidjeh licem u lice!" 
\par 23 Jahve  mu odgovori: "Mir s tobom! Ne boj se, nećeš umrijeti!" 
\par 24 Gideon podiže na tome mjestu žrtvenik Jahvi i nazva ga  "Jahve-Mir". Žrtvenik još i danas stoji u Ofri Abiezerovoj. 
\par 25 Iste noći Jahve reče Gideonu: "U svojega oca uzmi utovljena  junca, junca od sedam godina, i razori Baalov žrtvenik i posijeci  gaj pokraj njega. 
\par 26 Potom podigni žrtvenik Jahvi, Bogu svome, na vrhu te gorske stijene i dobro ga uredi. Uzmi junca i prinesi  paljenicu na drvima Ašere što ih u gaju nasiječeš." 
\par 27 Tada  Gideon uze deset ljudi između svojih slugu i učini kako mu je  zapovjedio Jahve. Ali kako se bojao svoje obitelji i građana, učini to noću. 
\par 28 Kad su građani sutradan poranili, a to razoren  Baalov žrtvenik i gaj posječen pored njega, a junac žrtvovan  kao paljenica na novom oltaru. 
\par 29 I pitahu jedni druge: "Tko  je to učinio?" Ispitaše, istražiše pa rekoše: "Gideon, Joašev  sin, učini to." 
\par 30 Tada građani rekoše Joašu: "Izvedi sina da  umre jer je razorio Baalov žrtvenik i posjekao gaj pored njega." 
\par 31 Joaš odgovori svima koji stajahu oko njega: "Zar ćete vi  braniti Baala? Zar ćete ga vi spasavati? Tko brani Baala, bit  će pogubljen prije sutrašnjeg dana. Ako je on bog, neka se sam  brani od Gideona što mu je razorio žrtvenik." 
\par 32 Toga dana prozvali  su Gideona Jerubaal jer se govorilo: "Neka sam Baal s njim obračuna  što mu je srušio žrtvenik." 
\par 33 Svi Midjanci, Amalečani i sinovi Istoka bijahu se sakupili  i, prešavši Jordan, utaborili se u Jizreelskoj ravnici. 
\par 34 Duh  Jahvin obuze Gideona i on zasvira u rog, a Abiezerov rod stade  iza njega. 
\par 35 Posla on glasnike po svem plemenu Manašeovu te  i oni stadoše iza njega. Posla glasnike i u pleme Ašerovo, Zebulunovo  i Naftalijevo te im i oni krenuše u susret. 
\par 36 Gideon reče Bogu: "Ako zaista hoćeš osloboditi Izraela  mojom rukom, kao što si obećao, 
\par 37 evo ću metnuti ovčje runo  na gumno: ako bude rose samo na runu, a zemlja ostane suha, tada  ću znati da ćeš mojom rukom izbaviti Izraela, kao što si obećao." 
\par 38 I bi tako. Gideon urani sutradan te iscijedi rosu iz  runa - punu zdjelu vode. 
\par 39 Opet Gideon reče Bogu: "Ne razgnjevi se na me što ti  progovaram još jednom. Dopusti mi da još ovaj put pokušam s runom:  neka samo runo bude suho, a neka po svoj zemlji bude rosa!" 
\par 40 I Bog one noći učini tako: samo je runo ostalo suho,  a po svoj zemlji pala rosa. 


\chapter{7}

\par 1 Urani Jerubaal, to jest Gideon, i sav narod bijaše s njim i  utabori se kod En-Haroda; a tabor Midjanaca nalazio se na sjeveru  od njegova, podno brijega More, u dolini. 
\par 2 Tada Jahve reče  Gideonu: "Previše je naroda s tobom a da bih  predao Midjance  u njegove ruke. Izrael bi se mogao pohvaliti i reći: 'Vlastita  me ruka izbavila.' 
\par 3 Zato oglasi da narod čuje: 'Tko se boji  i strahuje, neka se vrati.'" Gideon ih iskuša. Dvadeset i dvije  tisuće ljudi iz naroda vrati se, a ostade ih deset tisuća. 
\par 4 Jahve reče Gideonu: "Još je previše naroda. Povedi ih  na vodu i ondje ću ih iskušati. Za koga ti kažem: 'Neka ide s  tobom', taj će s tobom ići. A za koga ti kažem: 'Neka ne ide  s tobom', taj neće ići." 
\par 5 Gideon povede narod na vodu i Jahve  mu reče: "Koji bude laptao vodu jezikom kao što lapće pas, stavi  ga na stranu. Koji klekne da pije, odvoji ga na drugu stranu." 
\par 6 Onih koji su laptali vodu jezikom - prinoseći vodu rukom k  ustima - bijaše tri stotine, a sav je ostali narod kleknuo da  pije. 
\par 7 Tad Jahve reče Gideonu: "Sa one tri stotine ljudi koji  su laptali vodu ja ću vas izbaviti i predat ću Midjance u vaše  ruke. Svi drugi neka se vrate svaki svojoj kući." 
\par 8 Gideon tad  naloži narodu da mu preda opskrbu i rogove, a onda otpusti Izraelce  da ide svaki svome šatoru; zadrža samo one tri stotine. A midjanski se tabor prostirao niže njega u dolini. 
\par 9 One noći reče mu Jahve: "Ustani, navali na tabor, jer  ti ga predajem u ruke. 
\par 10 Ako se bojiš napasti, siđi najprije  u tabor s Purom, momkom svojim; 
\par 11 slušaj što govore; ohrabrit  ćeš se i napast ćeš na tabor." On siđe sa svojim momkom Purom  do prvih taborskih straža. 
\par 12 Midjanci, Amalečani i svi sinovi Istoka pali po dolini, brojni kao skakavci; njihovim devama ne bijaše broja, kao pijesku  na obali mora. 
\par 13 Kad je Gideon došao, a to jedan baš pripovijedaše  svome drugu što je sanjao: "Usnuo sam kako se pogača ječmenog  kruha kotrlja u midjanski tabor: dokotrlja se do jednog šatora  i pogodi, a šator pade, prevrnu se." 
\par 14 A drug mu odgovori:  "Nije to drugo nego mač Gideona, Joaševa sina, Izraelca. Bog  mu je predao u ruke Midjance i sav tabor." 
\par 15 Kada je Gideon čuo kako je onaj pripovjedio san i kako  ga je drugi protumačio, baci se ničice, vrati se onda u tabor  Izraelov i povika: "Ustajte, jer vam je Jahve predao u ruke tabor  midjanski!" 
\par 16 Gideon tad podijeli svoje tri stotine ljudi u tri čete.  Svakome čovjeku dade u ruke rog, prazan vrč i luč u vrču: 
\par 17 "Gledajte  mene", reče im, "i činite što i ja! Kada dođem na rub tabora, činite što budem i ja činio! 
\par 18 Kad zatrubim u rog ja i svi  koji su sa mnom, tada i vi zasvirajte u rog oko sveg tabora i  vičite: 'Za Jahvu i Gideona!'" 
\par 19 Gideon i stotina ljudi što ga je pratila dođoše na rub  tabora pri početku ponoćne straže; tek što su postavili straže, oni zatrubiše u rogove i razbiše vrčeve koje su imali u ruci. 
\par 20 Tako tri čete zasviraše u rogove i razbiše vrčeve; lijevom  rukom držahu luči, a desnom rogove da trube i udariše vikati:  "Za Jahvu i Gideona!" 
\par 21 I svaki stajaše nepomično na svome  mjestu uokrug tabora. Tada se probudi sav tabor i Midjanci vičući  nagoše u bijeg. 
\par 22 Dok su one tri stotine trubile u rogove, učini Jahve  te oni u taboru okrenuše mač jedan na drugoga. I sva se vojska  razbježa do Bet-Hašita, prema Sartanu, do Abel-Meholske obale  kod Tabata. 
\par 23 A Izraelci iz plemena Naftalijeva, Ašerova i iz svega  plemena Manašeova sabraše se i pognaše Midjance. 
\par 24 Gideon posla  glasnike po svoj Efrajimovoj gori da govore: "Siđite pred Midjance  i zauzmite prije njih sve gazove do Bet-Bara i Jordana." Svi  se ljudi od plemena Efrajimova odazvaše i zauzeše gazove voda  do Bet-Bara i Jordana. 
\par 25 I uhvatiše dva midjanska kneza, Oreba  i Zeeba; Oreba ubiše na Orebovoj stijeni, a Zeeba kod Zeebova  tijeska. Progonili su Midjance i donijeli Gideonu preko Jordana  glavu Orebovu i Zeebovu. 


\chapter{8}

\par 1 Tada Efrajimovi ljudi rekoše Gideonu: "Kako si postupio prema  nama: nisi nas pozvao kada si pošao u boj protiv Midjanaca?"  I žestoko mu prigovoriše. 
\par 2 On im odgovori: "Pa što sam ja učinio  kad se usporedim s vama? Nije li Efrajimovo pabirčenje bolje  od Abiezerove berbe? 
\par 3 U vaše je ruke Jahve predao knezove midjanske, Oreba i Zeeba. Može li se usporediti moje djelo s onim što ste  vi učinili?" Na te riječi utiša se njihova srdžba prema njemu. 
\par 4 Kad je Gideon došao do Jordana, prijeđe ga, ali i on i  tri stotine ljudi s njim bijahu iznemogli i gladni. 
\par 5 Stoga  reče ljudima iz Sukota: "Dajte kruha ljudima koji idu za mnom, iznemogli su. Ja gonim Zebaha i Salmunu, kraljeve midjanske." 
\par 6 Ali mu sukotski glavari odgovoriše: "Zar je Zebahova i Salmunina  šaka već u tvojoj ruci da dademo kruha tvojoj vojsci?" 
\par 7 Gideon  im reče: "Dobro! Kad mi Jahve preda u ruke Zebaha i Salmunu,  iskidat ću vam meso trnjem i dračem pustinjskim." 
\par 8 Odatle ode u Penuel i zatraži isto od Penuelaca, a oni  mu odgovore kao što su mu odgovorili i Sukoćani. 
\par 9 On zaprijeti  i Penuelcima: "Kad se vratim kao pobjednik, porušit ću ovu kulu." 
\par 10 Zebah i Salmuna bijahu u Karkoru i vojska njihova s njima, oko petnaest tisuća ljudi, što ih god osta od vojske sinova  Istoka; sto dvadeset tisuća ratnika bijaše palo. 
\par 11 Gideon pođe  putem kojim prolaze oni što žive pod šatorima, istočno od Nobaha  i Jogbohe, te potuče vojsku kad stajaše bezbrižna. 
\par 12 Zebah  i Salmuna pobjegoše. On ih pogna i uhvati dva kralja midjanska, Zebaha i Salmunu. A vojsku im svu uništi. 
\par 13 Poslije bitke Gideon, sin Joašev, vrati se preko Hareške  uzvisine. 
\par 14 I uhvati nekog momka iz Sukota te ga uze ispitivati;  a on mu popisa imena sukotskih knezova i starješina, sedamdeset  i sedam ljudi. 
\par 15 Potom Gideon ode Sukoćanima i reče: "Evo Zebaha  i Salmune zbog kojih ste mi se rugali govoreći: 'Je li Zebahova  i Salmunina šaka već u tvojoj ruci pa da dademo kruha tvojim  iznemoglim ljudima?'" 
\par 16 I uhvati starješine gradske, nabra  pustinjskog trnja i drača da ih oćute leđa Sukoćana. 
\par 17 Poruši  Penuelsku kulu i pobi građane. 
\par 18 Onda reče Zebahu i Salmuni: "Kakvi bijahu ljudi koje  pobiste na Taboru?" "Bili su nalik na te", odgovoriše. "Svaki  bijaše kao kraljev sin." 
\par 19 "To su bila moja braća, sinovi moje  matere", reče Gideon. "Tako mi Jahve, da ste ih ostavili na životu, ne bih vas ubio." 
\par 20 Potom zapovjedi svom prvencu Jeteru: "Ustani, pogubi ih!" Ali dječak ne izvuče mača: bojao se, bijaše još  mlad. 
\par 21 Tada rekoše Zebah i Salmuna: "Ustani ti i navali na  nas, jer kakav je čovjek, onakva mu i snaga." I ustavši, Gideon  pogubi Zebaha i Salmunu i uze mjesečiće što su visjeli o vratu  njihovih deva. 
\par 22 Izraelci rekoše Gideonu: "Vladaj nad nama, ti, sin tvoj  i unuk tvoj, jer si nas ti izbavio iz ruku Midjanaca." 
\par 23 Ali  im Gideon odgovori: "Ne, neću ja vladati nad vama, a ni moj sin;  Jahve će biti vaš vladar." 
\par 24 Još im reče Gideon: "Jedno samo  od vas tražim: da mi svaki dade prsten od svog plijena." Pobijeđeni  su nosili zlatne prstenove jer bijahu Jišmaelci. 
\par 25 "Vrlo rado", odgovore oni. On nato razastrije svoj plašt, a svaki od njih  baci od svog plijena po prsten. 
\par 26 Težina zlatnih prestenova  što ih je zaiskao iznosila je tisuću i sedam stotina zlatnih  šekela, osim mjesečića, naušnica i skrletnih haljina koje su  nosili midjanski kraljevi i osim lančića što bijahu oko vrata  njihovih deva. 
\par 27 Gideon načini od toga efod i postavi ga u  svome gradu Ofri. I sav Izrael udari za njim u nevjeru i bijaše  to zamka Gideonu i njegovu domu. 
\par 28 Tako su Midjanci bili poniženi pred Izraelcima. Više  ne dizahu glave i zemlja bi mirna četrdeset godina, koliko još  potraja vijek Gideonov. 
\par 29 Jerubaal, sin Joašev, otišao je i  živio u svojoj kući. 
\par 30 Gideon je imao sedamdeset sinova koji  su potekli od njega jer je imao mnogo žena. 
\par 31 Njegova inoča  koja je živjela u Šekemu rodi mu sina komu nadjenu ime Abimelek. 
\par 32 Gideon, sin Joašev, umrije u dubokoj starosti; sahraniše  ga u grobu njegova oca Joaša u Abiezerovoj Ofri. 
\par 33 Po Gideonovoj smrti Izraelci okrenuše u preljub s baalima  te postaviše sebi za boga Baal-Berita. 
\par 34 Izraelci se nisu više  sjećali Jahve, svoga Boga, koji ih je izbavio iz ruku svih njihovih  neprijatelja unaokolo. 
\par 35 I nisu iskazivali zahvalnost domu  Jerubaala Gideona za dobro što ga je učinio Izraelu. 


\chapter{9}

\par 1 Abimelek, sin Jerubaalov, otiđe u Šekem k braći svoje matere  i reče njima i svemu rodu kuće svoje majke: 
\par 2 "Upitajte sve  šekemske građane: što vam je bolje - da nad vama vlada sedamdeset  ljudi, svi sinovi Jerubaalovi, ili jedan čovjek? Sjetite se da  sam ja od vašeg mesa i vaših kostiju!" 
\par 3 To braća njegove matere  prenesoše ostalim šekemskim građanima i njihovo se srce prikloni  Abimeleku jer govorahu: "Naš je brat!" 
\par 4 I dadoše mu sedamdeset  šekela srebra iz hrama Baal-Beritova; time Abimelek unajmi klatež  i pustolove koji pođoše za njim. 
\par 5 Onda dođe u kuću svoga oca  u Ofri i pobi svoju braću, sinove Jerubaalove, sedamdeset ljudi, na jednom kamenu. Izmakao mu je samo Jotam, najmlađi sin Jerubaalov  jer se bijaše sakrio. 
\par 6 Tada se skupiše svi šekemski građani  i sav Bet-Milo te postaviše Abimeleka za kralja kod hrasta koji  stoji u Šekemu. 
\par 7 Kada su to dojavili Jotamu, ode on, stade na vrh gore  Gerizima i povika im na sav glas: "Čujte me, uglednici šekemski, tako vas čuo Bog! 
\par 8 Jednom se zaputila stabla da pomažu kralja koji će vladati nad njima. Pa rekoše maslini: 'Budi nam kraljem!' 
\par 9 Odgovori im maslina: 'Zar da se svog ulja odreknem što je na čast bozima i ljudima da bih vladala nad drugim drvećem?' 
\par 10 Tad rekoše stabla smokvi: 'Dođi, budi nam kraljem!' 
\par 11 Odgovori im smokva: 'Zar da se odreknem slatkoće i krasnoga ploda svog da bih vladala nad drugim drvećem?' 
\par 12 Tad rekoše stabla lozi: 'Dođi, budi nam kraljem!' 
\par 13 Odgovori im loza: 'Zar da se odreknem vina što veseli bogove i ljude da bih vladala nad drugim drvećem?' 
\par 14 Sva stabla rekoše tad glogu: 'Dođi, budi nam kraljem!' 
\par 15 A glog odgovori stablima: 'Ako me doista hoćete pomazat' za kralja, u sjenu se moju sklonite. Ako nećete, iz gloga će oganj planuti i sažeći cedrove libanonske!' 
\par 16 Sada, jeste li vjerno i čestito učinili kad ste izabrali  Abimeleka za kralja? Jeste li se dobro ponijeli prema Jerubaalu  i njegovoj kući? Jeste li mu uzvratili za djela što ih za vas  učini? 
\par 17 Moj se otac za vas borio izloživši svoj život te vas  izbavio iz ruku Midjanaca, 
\par 18 a vi danas ustaste protiv kuće  moga oca, pobiste njegove sinove, sedamdeset ljudi na istom kamenu, i nad građanima Šekema učiniste  kraljem Abimeleka, sina njegove  robinje, zato što je vaš brat! 
\par 19 Ako ste vjerno i pošteno danas  radili prema Jerubaalu i prema njegovoj kući, radujte se s Abimelekom, a on neka se raduje s vama! 
\par 20 Ako niste, neka oganj iziđe  iz Abimeleka i sažeže građane Šekema i Bet-Mila i neka iziđe  oganj iz građana Šekema i Bet-Mila i sažeže Abimeleka!" 
\par 21 Onda Jotam pobježe, skloni se i dođe u Beer, i ondje  ostade, jer se bojao svoga brata Abimeleka. 
\par 22 Abimelek je vladao nad Izraelom tri godine. 
\par 23 Tada  Bog posla duh razdora među Abimeleka i šekemske građane i šekemski  se građani pobuniše protiv Abimeleka. 
\par 24 Bijaše to zato da bi  se osvetio zločin počinjen nad sedamdeset Jerubaalovih sinova  i da bi njihova krv pala na njihova brata Abimeleka, koji ih  ubi, i na građane Šekema, koji mu pomogoše da ubije braću. 
\par 25 Hoteći  mu napakostiti, šekemski su građani postavili zasjede po vrhovima  planina i pljačkali svakoga tko bi prošao mimo njih onim putem.  Javiše to Abimeleku. 
\par 26 Gaal, sin Ebedov, dođe sa svojom braćom i nastani se  u Šekemu; a šekemski se građani pouzdaše u njega. 
\par 27 Otišavši  u polje, trgali su u svojim vinogradima grožđe i gazili ga, a  onda udarili u veselje; ušli su u hram svoga boga, jeli su, pili  i proklinjali Abimeleka. 
\par 28 A Gaal, Ebedov sin, povika: "Tko  je Abimelek da mu služimo? Zar ne bi trebalo da Jerubaalov sin  i Zebul, njegov namjesnik, služe ljude Hamora, Šekemova oca?  Zašto da mi njemu služimo? 
\par 29 O, kad bih imao ovaj narod u svojoj  ruci, protjerao bih Abimeleka i rekao mu: 'Pojačaj svoju vojsku  i iziđi u boj!'" 
\par 30 A kad Zebul, gradski načelnik, doznade što  je govorio Gaal, sin Ebedov, razgnjevi se. 
\par 31 Posla glasnike  Abimeleku u Arumu i poruči mu: "Evo, Gaal, sin Ebedov, došao  u Šekem sa svojom braćom i bune građane protiv tebe. 
\par 32 Zato  ustani noću, ti i narod što je s tobom, i stani u zasjedu u polju. 
\par 33 A ujutro, kad ograne sunce, digni se i udari na grad. Kada  Gaal i njegovi ljudi iziđu preda te, ti učini s njima što ti  prilike posavjetuju." 
\par 34 Abimelek usta noću sa svim svojim ljudima i stade u zasjedu  oko Šekema u četiri čete. 
\par 35 Kada je Gaal, sin Ebedov, izišao  pred gradska vrata i zaustavio se, Abimelek i njegovi ljudi ustaše  iz zasjede. 
\par 36 Gaal ugleda ljude i reče Zebulu: "Eno silaze ljudi s  gorskih vrhova." "Od sjena gorskih vrhova", odgovori mu Zebul, "čine ti se ljudi." 
\par 37 Opet progovori Gaal: "Eno silaze ljudi  s visa zvana Zemljin pupak, a četa jedna dolazi putem od Čarobnjačkog  hrasta." 
\par 38 Tad mu reče Zebul: "Gdje ti je sada jezik? Pa ti  si govorio: 'Tko je Abimelek da mu služimo?' Nisu li ondje ljudi  koje si prezirao? Iziđi sada i pobij se s Abimelekom." 
\par 39 I Gaal iziđe na čelu šekemskih građana i pobi se s Abimelekom. 
\par 40 Abimelek potjera Gaala i on pobježe pred njim; i mnogi njegovi  ljudi padoše mrtvi prije nego što su i došli do vrata. 
\par 41 Abimelek  se tada vrati u Arumu, a Zebul potjera Gaala i njegovu braću  i nije im više dao da ostanu u Šekemu. 
\par 42 Sutradan je narod izišao u polje i javiše to Abimeleku. 
\par 43 On uze svoju vojsku, podijeli je u tri čete i stade u zasjedu  u polju. Kad bi vidio gdje ljudi izlaze iz grada, nasrnuo bi  na njih i pobio ih. 
\par 44 Dok je Abimelek sa svojom četom udarao  kod gradskih vrata, druge se dvije čete baciše na one koji bijahu  u polju i tako ih pobiše. 
\par 45 Čitav je dan Abimelek opsjedao  grad. Zauzevši ga, poubija sve stanovništvo, razori grad i posu  sol po njemu. 
\par 46 Kad su to čuli gospodari Migdal Šekema, uđoše svi u tvrdi  prostor hrama El-Berita. 
\par 47 Kada je Abimelek doznao da su se  svi građani Migdal Šekema ondje sakupili, 
\par 48 popne se na Salmonsku  goru sa svom vojskom svojom. Uzevši u ruke sjekiru, odsječe granu  od drveta, podiže je i metnu sebi na rame. A ljudima zapovjedi:  "Što vidjeste da sam ja učinio, učinite brzo i vi." 
\par 49 I svi  ljudi odsjekoše sebi po granu, a onda krenuše za Abimelekom,  nabacaše granje na utvrdu i zapališe ga nad onima koji su se  ondje nalazili. Tako izgiboše svi žitelji Migdal Šekema, oko  tisuću ljudi i žena. 
\par 50 Potom Abimelek krenu na Tebes, opsjede ga i osvoji. 
\par 51 Bijaše  ondje usred grada kula kamo su se sklonili svi ljudi i žene i  svi uglednici gradski. Zatvorivši za sobom vrata, popeše se kuli  na krov. 
\par 52 Abimelek dođe do kule i napade je. Dok je prilazio  vratima kule da je zapali, 
\par 53 neka žena baci mu žrvanj na glavu  i razbi mu lubanju. 
\par 54 On brzo pozva svoga momka koji mu je  nosio oružje i reče mu: "Trgni mač i ubij me da se ne govori  o meni: 'Žena ga je ubila.'" Njegov ga momak probode te on umrije. 
\par 55 Kad su Izraelci vidjeli da je Abimelek mrtav, svi se vratiše  svojim kućama. 
\par 56 Tako je Bog svalio na Abimeleka zlo koje je on učinio  svome ocu pobivši sedamdesetero svoje braće. 
\par 57 I sve zlo Šekemaca  Bog svali na njihove glave i tako ih stiže kletva Jotama, sina  Jerubaalova. 


\chapter{10}

\par 1 Poslije Abimeleka ustao je Tola, sin Pue, sina Dodova, da  izbavi Izraela. On bijaše iz Jisakarova plemena, a živio je u  Šamiru, u Efrajimovoj gori. 
\par 2 Bio je sudac Izraelu dvadeset  i tri godine, a kad je umro, pokopali su ga u Šamiru. 
\par 3 Poslije njega ustao je Jair Gileađanin, koji je bio sudac  Izraelu dvadeset i dvije godine. 
\par 4 Imao je trideset sinova koji  su jahali na tridesetero magaradi i imali trideset gradova što  se do dana današnjega zovu Sela Jairova, a nalaze se u gileadskoj  zemlji. 
\par 5 Kad umrije Jair, pokopaše ga u Kamonu. 
\par 6 Izraelci su opet stali činiti ono što Jahvi nije po volji.  Služili su baalima i aštartama, aramejskim bogovima i sidonskim  bogovima, bogovima Moabaca, bogovima Amonaca i bogovima Filistejaca.  A Jahvu su napustili i nisu mu više služili. 
\par 7 Tad planu Jahve  gnjevom i predade ih u ruke Filistejcima i Amoncima. 
\par 8 Oni su  od tada osamnaest godina satirali i tlačili Izraelce - sve Izraelce  koji življahu s onu stranu Jordana, u zemlji amorejskoj, koja  je u Gileadu. 
\par 9 Potom su Amonci prešli Jordan da zavojšte i  na Judu, Benjamina i na Efrajima te se Izrael nađe u velikoj  nevolji. 
\par 10 Tada zavapiše Izraelci Jahvi govoreći: "Griješili smo  prema tebi jer smo ostavili Jahvu, svoga Boga, da bismo služili  baalima." 
\par 11 A Jahve odgovori Izraelcima: "Nisu li vas tlačili  Egipćani i Amorejci, Amonci i Filistejci, 
\par 12 Sidonci, Amalečani  i Midjanci? Ali kad ste zavapili prema meni, nisam li vas izbavio  iz njihovih ruku? 
\par 13 Ali vi ostaviste mene i uzeste služiti  drugim bogovima. Zbog toga vas neću više izbavljati. 
\par 14 Idite  i vapite za pomoć onim bogovima koje ste izabrali! Neka vas oni  izbave iz vaše nevolje!" 
\par 15 Izraelci odgovoriše Jahvi: "Sagriješili  smo! Čini s nama što ti drago, samo nas danas izbavi!" 
\par 16 I  odstraniše tuđe bogove i počeše opet služiti Jahvi. A Jahve više  ne mogaše trpjeti da Izraelci pate. 
\par 17 Kada su se Amonci sabrali i utaborili u Gileadu, skupiše  se i Izraelci i utaboriše se u Mispi. 
\par 18 Tada narod i knezovi  gileadski rekoše jedni drugima: "Koji čovjek povede boj protiv  Amonaca, neka bude poglavar svima koji žive u Gileadu." 


\chapter{11}

\par 1 Gileađanin Jiftah bijaše hrabar ratnik. Rodila ga bludnica, a otac mu bijaše Gilead. 
\par 2 Ali je Gileadu i njegova žena rodila  sinove, pa kada su sinovi te žene odrasli, otjeraše Jiftaha govoreći  mu: "Nećeš dobiti baštine od našeg oca jer si sin strane žene." 
\par 3 Jiftah zato pobježe od svoje braće i naseli se u zemlji Tobu.  Ondje se oko njega okupila hrpa beskućnika koji su s njim pljačkali. 
\par 4 Poslije nekog vremena Amonci zavojštiše na Izraela. 
\par 5 Kada  su Amonci napali Izraela, krenuše gileadske starješine da trže  Jiftaha u zemlji Tobu. 
\par 6 "Hodi", rekoše mu, "budi nam vojvoda da ratujemo protiv  Amonaca." 
\par 7 Ali Jiftah odgovori gileadskim starješinama: "Niste  li me vi mrzili i otjerali iz kuće moga oca? Zašto sada dolazite  k meni kada ste u nevolji?" 
\par 8 Gileadske starješine rekoše Jiftahu:  "Zato smo sada došli tebi: pođi s nama, povedi rat protiv Amonaca  i bit ćeš poglavar nama i svima u Gileadu." 
\par 9 Jiftah upita gileadske  starješine: "Ako me odvedete natrag da ratujem protiv Amonaca  te ako ih Jahve meni preda, hoću li biti vaš poglavar?" 
\par 10 "Jahve  neka bude svjedokom među nama", odgovore Jiftahu gradske starješine.  "Jao nama ako ne učinimo kako si rekao!" 
\par 11 I Jiftah ode sa starješinama Gileada. Narod ga postavi  sebi za poglavara i vojvodu; a Jiftah je ponovio sve svoje uvjete  pred Jahvom u Mispi. 
\par 12 Jiftah posla onda poslanike kralju Amonaca s porukom:  "Što ima između tebe i mene da si došao ratovati protiv moje  zemlje?" 
\par 13 Kralj Amonaca odgovori Jiftahovim poslanicima: "U  vrijeme kada je izlazio iz Egipta, Izrael ja zaposjeo moju zemlju  od Arnona do Jaboka i Jordana. Zato mi je sada dragovoljno vrati!" 
\par 14 Jiftah nanovo pošalje glasnike kralju Amonaca 
\par 15 i poruči  mu:  "Ovako govori Jiftah: Nije Izrael zaposjeo ni moapsku ni  amonsku zemlju, 
\par 16 nego je, izišavši iz Egipta, Izrael prešao  pustinjom do Crvenog mora i došao u Kadeš. 
\par 17 Tada je poslao  Izrael poslanike edomskom kralju s molbom: 'Htio bih proći kroz  tvoju zemlju!' Ali ga edomski kralj ne posluša. Poslao ih je  i moapskom kralju, ali ni on ne htjede, te Izrael ostade u Kadešu. 
\par 18 Onda je preko pustinje zaobišao edomsku i moapsku zemlju  i došao na istok od moapske zemlje. Narod se utaborio s one strane  Arnona ne prelazeći granice Moaba, jer Arnon bijaše moapska međa. 
\par 19 Izrael posla zatim poslanike Sihonu, amorejskom kralju, koji  je vladao u Hešbonu, i poruči mu: 'Pusti nas da prođemo kroz  tvoju zemlju do mjesta koje nam je određeno.' 
\par 20 Ali Sihon ne  dopusti Izraelu da prođe preko njegova područja, nego skupi svu  svoju vojsku koja bijaše utaborena u Jahasu i zametnu boj s Izraelom. 
\par 21 Jahve, Bog Izraelov, predade Sihona i svu njegovu vojsku  u ruke Izraelu, koji ih porazi, te Izrael zaposjede svu zemlju  Amorejaca koji nastavahu to područje. 
\par 22 Zaposjeo je tako svu  zemlju Amorejaca od Arnona do Jaboka i od pustinje do Jordana. 
\par 23 I sada kad je Jahve, Bog Izraelov, protjerao Amorejce pred  svojim narodom Izraelom, ti bi nas htio odagnati? 
\par 24 Zar ne  posjeduješ sve što je tvoj bog Kemoš bio oteo starim posjednicima?  Tako i sve ono što je Jahve, naš Bog, oteo starim posjednicima, mi sada posjedujemo! 
\par 25 Po čemu si ti bolji od moapskog kralja  Balaka, sina Siporova? Je li se i on sporio s Izraelom? Je li  on ratovao protiv njega? 
\par 26 Kada se Izrael nastanio u Hešbonu  i u njegovim selima, u Aroeru i u njegovim selima, a tako i po  svim gradovima na obali Jordana - evo, već tri stotine godina  - zašto ih tada niste oteli? 
\par 27 Nisam ja tebi skrivio nego ti  meni činiš krivo ratujući protiv mene. Neka Jahve, Sudac, danas  presudi između sinova Izraelovih i sinova Amonovih." 
\par 28 Ali kralj Amonaca ne posluša riječi što mu ih je poručio  Jiftah. 
\par 29 Duh Jahvin siđe na Jiftaha te on pođe kroz Gileadovo  i Manašeovo pleme, prođe kroz gileadsku Mispu, a od gileadske  Mispe dođe iza Amonaca. 
\par 30 I Jiftah se zavjetova Jahvi: "Ako  mi predaš u ruke Amonce, 
\par 31 tko prvi iziđe na vrata moje kuće  u susret meni kada se budem vraćao kao pobjednik iz boja s Amoncima  bit će Jahvin i njega ću prinijeti kao paljenicu." 
\par 32 Jiftah krenu protiv Amonaca da ih napadne i Jahve ih  izruči u njegove ruke. 
\par 33 I porazi ih Jiftah od Aroera do blizu  Minita - u dvadeset gradova - i sve do Abel Keramima. Bijaše  to njihov veliki poraz; i Amonci bijahu poniženi pred Izraelom. 
\par 34 Kada se Jiftah vratio kući u Mispu, gle, iziđe mu u susret  kći plešući uza zvuke bubnjeva. Bijaše mu ona jedinica, osim  nje nije imao ni sina ni kćeri. 
\par 35 Ugledavši je, razdrije svoje haljine i zakuka: "Jao,  kćeri moja, u veliku me tugu bacaš! Zar mi baš ti moraš donijeti  nesreću! Zavjetovah se Jahvi i ne mogu zavjeta poreći." 
\par 36 Ona mu odgovori: "Oče moj, ako si učinio zavjet Jahvi, učini sa mnom kako si se zavjetovao, jer ti je Jahve dao da  se osvetiš Amoncima, svojim neprijateljima." 
\par 37 Onda zamoli  svog oca: "Ispuni mi ovu molbu: pusti me da budem slobodna dva  mjeseca; lutat ću po gorama sa svojim drugama i oplakivati svoje  djevičanstvo." 
\par 38 "Idi", reče joj on i pusti je na dva mjeseca. Ona ode  sa svojim drugama i oplakivaše na gorama svoje djevičanstvo. 
\par 39 Kada su prošla dva mjeseca, ona se vrati ocu i on izvrši  na njoj zavjet što ga bijaše učinio. I nikada nije upoznala čovjeka.  Otada je potekao običaj u Izraelu 
\par 40 da svake godine odlaze  Izraelove kćeri i oplakuju kćer Jiftaha Gileađanina četiri dana  na godinu. 


\chapter{12}

\par 1 Uto se skupiše ljudi od Efrajimova plemena, prijeđoše Jordan  put Safona i rekoše Jiftahu. "Zašto si išao u boj protiv Amonaca  a nas nisi pozvao da idemo s tobom? Spalit ćemo ti kuću i tebe!" 
\par 2 Jiftah im odgovori: "Imali smo veliku parbu, ja i moj  narod, i Amonci su nas teško tlačili. Pozvao sam vas u pomoć, ali me niste izbavili iz njihovih ruku. 
\par 3 Videći da mi nitko  ne pritječe u pomoć, stavih svoj život na kocku, odoh sam na  Amonce, i Jahve mi ih predade u ruke. Zašto ste, dakle, pošli  danas da ratujete protiv mene?" 
\par 4 Tada skupi Jiftah sve Gileađane i udari na Efrajima. Gileađani  potukoše Efrajima, jer su ovi govorili: "Vi ste, Gileađani, Efrajimovi  bjegunci koji ste živjeli usred Efrajima i Manašea." 
\par 5 Zatim  Gileađani presjekoše Efrajimu jordanske gazove, i kada bi koji  bjegunac Efrajimov rekao: "Pustite me da prijeđem", Gileađani  bi ga pitali: "Jesi li Efrajimovac?" A kada bi on odgovorio:  "Nisam", 
\par 6 oni bi mu kazali: "Hajde reci: Šibolet!" On bi rekao:  "Sibolet" jer nije mogao dobro izgovoriti. Oni bi ga tada uhvatili  i pogubili na jordanskim plićacima. Tako je poginulo četrdeset  i dvije tisuće ljudi iz Efrajimova plemena. 
\par 7 Jiftah je sudio Izraelu šest godina. A kada je Gileađanin  Jiftah umro, pokopaše ga u njegovu gradu, u Gileadu. 
\par 8 Poslije njega sudac u Izraelu bijaše Ibsan iz Betlehema. 
\par 9 On je imao trideset sinova i trideset kćeri, koje je poudao  iz kuće, a trideset je snaha doveo izvana svojim sinovima. On  je sudio Izraelu sedam godina. 
\par 10 Zatim umrije Ibsan i pokopaše  ga u Betlehemu. 
\par 11 Poslije njega sudac u Izraelu bijaše Elon Zebulunac.  On je sudio Izraelu deset godina. 
\par 12 Zatim umrije Zebulunac  Elon i pokopaše ga u Ajalonu u zemlji Zebulunovoj. 
\par 13 Poslije njega sudac u Izraelu bijaše Abdon, sin Hilela  iz Pireatona. 
\par 14 On je imao četrdeset sinova i trideset unuka  koji su jahali na sedamdesetero magaradi. On je sudio Izraelu  osam godina. 
\par 15 Zatim umrije Abdon, sin Hilela iz Pireatona, i pokopaše ga u Pireatonu u Efrajimovoj gori, u zemlji Šaalimu. 


\chapter{13}

\par 1 Izraelci su opet okrenuli da čine ono što Jahvi nije po volji  i Jahve ih predade u ruke Filistejcima za čerdeset godina. 
\par 2 A bijaše neki čovjek iz Sore, od Danova plemena, po imenu  Manoah. Žena mu bila nerotkinja i nije imala djece. 
\par 3 Toj se  ženi ukaza Anđeo Jahvin i reče joj: "Ti si neplodna i nisi rađala. 
\par 4 Ali se odsad pazi: da ne piješ ni vina ni žestoka pića i da  ne jedeš ništa nečisto. 
\par 5 Jer, zatrudnjet ćeš, evo, i rodit  ćeš sina. I neka mu britva ne prijeđe po glavi, jer će od majčine  utrobe dijete biti Bogu posvećeno - bit će nazirej Božji i on  će početi izbavljati Izraela iz ruke Filistejaca." 
\par 6 Žena ode i kaza mužu: "Božji čovjek došao k meni, lice  mu kao u Božjeg anđela, puno dostojanstva. Nisam ga upitala odakle  je došao, niti mi on kaza svog imena. 
\par 7 Ali mi je rekao: 'Ti  ćeš začeti i roditi sina. Ne pij odsad ni vina ni žestoka pića  i ne jedi ništa nečisto jer će ti dijete biti nazirej Božji od  majčine utrobe do smrti.'" 
\par 8 Tada se Manoah pomoli Jahvi i reče: "Molim te, Gospode, neka Božji čovjek koga si jednom poslao dođe još jednom k nama  i pouči nas što ćemo činiti s djetetom kad se rodi!" 
\par 9 Jahve  usliši Manoaha i Anđeo Jahvin dođe opet k ženi dok je sjedila  u polju. Manoah, muž njezin, ne bijaše kraj nje. 
\par 10 Žena brzo  otrča da obavijesti muža i reče mu: "Gle, ukazao mi se čovjek  koji mi je došao onog dana." 
\par 11 Manoah ustade, pođe za ženom i kada dođe k čovjeku, upita  ga: "Jesi li ti onaj što je govorio s ovom ženom?" A on odgovori:  "Jesam." 
\par 12 "Kada se ispuni ono što si rekao", opet će Manoah, "po kojim propisima i kako treba postupati s djetetom?" 
\par 13 Anđeo  Jahvin odgovori Manoahu: "Neka se žena čuva svega što sam joj  zabranio. 
\par 14 Neka ne uživa ništa što dolazi od vinove loze,  neka ne pije ni vina ni žestoka pića, neka ne jede ništa nečisto  i neka se drži svega što sam joj zapovjedio." 
\par 15 Tada reče Manoah Anđelu Jahvinu: "Rado bismo te ustavili  i pogostili jaretom." 
\par 16 Anđeo Jahvin nato će Manoahu: "Sve  da me i ustaviš, ja ne bih jeo tvoga jela; nego ako želiš žrtvovati  paljenicu, prinesi je Jahvi." Manoah, ne znajući da je to Anđeo Jahvin, 
\par 17 reče tada Anđelu  Jahvinu: "Kako ti je ime, da te možemo častiti kada se ispuni  što si obećao." 
\par 18 Anđeo Jahvin odgovori mu: "Zašto pitaš za  moje ime? Ono je tajanstveno." 
\par 19 Manoah nato uze jare i prinos te ga na stijeni kao paljenicu  žrtvova Jahvi koji čini tajanstvene stvari. 
\par 20 Kada se poče  dizati plamen sa žrtvenika k nebu, podiže se Anđeo Jahvin u tome  plamenu. Kad to vidješe Manoah i njegova žena, padoše ničice. 
\par 21 Anđeo Jahvin nije se više ukazivao Manoahu i njegovoj ženi.  Manoah tada shvati da je to Anđeo Jahvin. 
\par 22 "Zacijelo ćemo umrijeti", reče ženi, "jer smo vidjeli  Boga." 
\par 23 "Da nas je htio usmrtiti", odgovori mu žena, "ne bi  iz naše ruke primio paljenice ni prinosa i ne bi nam dao da sve  to vidimo niti da takvo što čujemo." 
\par 24 Žena rodi sina i nadjenu mu ime Samson. Dijete odraste  i Jahve ga blagoslovi. 
\par 25 I Jahvin duh bijaše s njim u Danovu  taboru, između Sore i Eštaola. 


\chapter{14}

\par 1 I siđe Samson u Timnu i ugleda ondje djevojku među filistejskim  kćerima. 
\par 2 Vrativši se, povjeri to ocu i majci: "Opazio sam  u Timni", reče on, "djevojku među filistejskim kćerima: oženite  me njome." 
\par 3 Otac i mati rekoše: "Zar nema djevojaka među kćerima  tvoga plemena i u svemu našem narodu da moraš uzeti ženu između  neobrezanih Filistejaca?" Ali Samson odgovori ocu: "Oženi me  njome jer mi ona omilje." 
\par 4 Otac mu i majka nisu znali da je  to od Jahve, koji je tražio zadjevicu s Filistejcima jer Filistejci  u ono doba vladahu Izraelom. 
\par 5 Samson siđe tako u Timnu i kad dođe do timnjanskih vinograda, gle - odjednom preda nj iskoči mladi lav ričući. 
\par 6 Duh Jahvin  zahvati Samsona, i on goloruk raskida lava kao što se raskida  jare; ali ne reče ni ocu ni majci što je učinio. 
\par 7 Došavši,  razgovori se s djevojkom i ona mu omilje. 
\par 8 Poslije nekog vremena, kada se vratio da je odvede, Samson skrenu da vidi mrtvog lava, a to u mrtvom lavu roj pčela i med. 
\par 9 On uze meda u ruke i  jeo ga je idući putem. Kada se vratio k ocu i majci, dade ga  i njima te i oni jedoše; ali im ne reče da ga je uzeo iz mrtvog  lava. 
\par 10 Zatim ode ženi i ondje prirediše gozbu Samsonu; trajala  je sedam dana, jer tako običavahu mladi ljudi. 
\par 11 Ali kako ga  se bojahu, izabraše trideset svadbenih drugova da budu uza nj. 
\par 12 Tad im reče Samson: "Hajde da vam zadam zagonetku. Ako  je odgonetnete za sedam svadbenih dana, dat ću vam trideset truba  finog platna i trideset svečanih haljina. 
\par 13 Ali ako je ne mognete  odgonetnuti, vi ćete meni dati trideset truba platna i trideset  svečanih haljina." "Zadaj nam zagonetku", odgovore mu oni, "mi te slušamo." 
\par 14 A on im reče: "Od onog koji jede izišlo je jelo, od jakoga izišlo je slatko." Ali za tri dana nisu mogli odgonetnuti zagonetke. 
\par 15 Četvrtoga  dana rekoše Samsonovoj ženi: "Izvuci od muža na prijevaru rješenje  zagonetke, ili ćemo spaliti i tebe i očev ti dom! Zar ste nas  ovamo pozvali da nas oplijenite?" 
\par 16 Tada žena, uplakana, obisnu  Samsonu oko vrata govoreći: "Ti mene samo mrziš i ne ljubiš me.  Zadao si zagonetku sinovima moga naroda, a meni je nisi objasnio."  On joj odgovori: "Nisam je objasnio ni ocu ni majci, a tebi da  je kažem?" 
\par 17 Ona mu plakaše oko vrata sedam dana, koliko je  trajala gozba. Sedmoga dana on joj kaza odgonetku: toliko je  na nj navaljivala. I ona je odade sinovima svoga naroda. 
\par 18 Sedmoga dana, prije nego je zašlo sunce, ljudi iz toga  grada rekoše Samsonu: "Što ima slađe od meda i što ima jače od lava?" A on im odgovori: "Da niste s mojom junicom orali, ne biste zagonetke pogodili." 
\par 19 Tada duh Jahvin dođe na njega, te on siđe u Aškelon i  ondje pobi trideset ljudi, uze im odjeću i dade svečane haljine  onima koji su odgonetnuli zagonetku, a onda se sav gnjevan vrati  očevoj kući. 
\par 20 A Samsonovu ženu dadoše drugu koji mu bijaše  svadbeni pratilac. 


\chapter{15}

\par 1 Poslije nekog vremena, o žetvi pšenice, Samson dođe da pohodi  svoju ženu, donijevši joj kozle i reče: "Želim ući k svojoj ženi  u ložnicu." Ali mu tast ne dopusti. 
\par 2 "Mislio sam," reče mu  on, "da si je zamrzio, pa sam je dao tvome drugu. Ali zar njezina  mlađa sestra nije ljepša od nje? Uzmi je namjesto one!" 
\par 3 Samson  mu odgovori: "Ovaj put neću biti krivac Filistejcima kad im učinim  zlo." 
\par 4 I ode Samson, ulovi tri stotine lisica, uze luči i,  okrenuvši rep prema repu, stavi jednu luč među dva repa. 
\par 5 Tad  zapali luči, pusti lisice u filistejska polja i popali im snopove, i nepokošeno žito, i vinograde, i maslinike. 
\par 6 Filistejci zapitaše: "Tko je to učinio?" Odgovoriše im:  "Samson, Timnjaninov zet, zato što mu tast oduze ženu i dade  je njegovu drugu." Tad Filistejci odoše i spališe onu ženu i  njenu obitelj. 
\par 7 "Kad ste to učinili", reče im Samson, "neću mirovati dok  vam se ne osvetim." 
\par 8 I sve ih izudara uzduž i poprijeko i žestoko  ih porazi. Poslije toga ode u spilju Etamske stijene i ondje  se nastani. 
\par 9 Tad Filistejci krenuše, utaboriše se u Judi i raširiše  do Lehija. 
\par 10 "Zašto ste pošli na nas?" - upitaše ih Judejci.  A oni im odgovoriše: "Pošli smo da svežemo Samsona i da mu učinimo  kako je on učinio nama." 
\par 11 Tri tisuće Judejaca odoše tada k  spilji Etamske stijene i rekoše Samsonu: "Zar ne znaš da Filistejci  nama gospodare? Zašto si nam onda to učinio?" On im odgovori:  "Kako oni meni, tako ja njima!" A oni mu rekoše: 
\par 12 "Dođosmo  da te svežemo i predamo u ruke Filistejaca." "Zakunite mi se", reče im, "da me nećete ubiti." 
\par 13 "Ne", odgovoriše mu, "mi  ćemo te samo svezati i predati u njihove ruke, ali te zacijelo  ne želimo pogubiti." Onda ga svezaše sa dva nova užeta i odvedoše  iz spilje. 
\par 14 Kad ga dovedoše u Lehi i kad Filistejci, vičući od radosti, pojuriše na nj, duh Jahvin zahvati ga i užeta na njegovim rukama  postadoše kao laneni konci, spaljeni ognjem, i spadoše mu s ruku. 
\par 15 Spazivši još sirovu magareću čeljust, pruži on ruku, uze  onu čeljust i pobi njome tisuću ljudi. 
\par 16 Tad reče Samson: "Magarećom čeljusti gomile prebih, Magarećom čeljusti tisuću pobih." 
\par 17 Rekavši to, baci čeljust iz ruke. Zato odonda ono mjesto  zovu Ramat Lehi. 
\par 18 Kako bijaše jako ožednio, zavapi Jahvi govoreći:  "Ti si izvojštio ovu veliku pobjedu rukama svoga sluge, a zar  sada moram umrijeti od žeđi i pasti u ruke neobrezanima?" 
\par 19 Tad  Jahve rasiječe udubinu što je kod Lehija i voda poteče iz nje.  Samson se napi i vrati mu se snaga, oživje mu duh. Zato su onom  izvoru dali ime En Hakore, a postoji još i danas u Lehiju. 
\par 20 Samson  bijaše sudac u Izraelu za vrijeme filistejske vladavine dvadeset  godina. 


\chapter{16}

\par 1 Odatle ode Samson u Gazu; ondje vidje neku bludnicu i uđe  k njoj. 
\par 2 Žiteljima Gaze javiše: "Samson je došao ovamo!" Opkoliše  ga i vrebahu ga svu noć na gradskim vratima. Svu noć bijahu mirni.  "Pričekajmo do zore", mišljahu, "pa ćemo ga ubiti." 
\par 3 Ali je  Samson ležao samo do ponoći, a o ponoći ustade, dohvati gradska  vrata s oba dovratnika, iščupa ih zajedno s prijevornicom, metnu  ih na ramena i odnese na vrh gore koja je nasuprot Hebronu i  položi ih ondje. 
\par 4 Poslije toga zamilova on neku ženu iz doline Soreka po  imenu Delilu. 
\par 5 Filistejski knezovi dođoše k njoj i rekoše joj:  "Zavedi ga i doznaj gdje stoji njegova velika snaga, kako bismo  ga mogli svladati pa da ga svežemo i učinimo nemoćnim. A dat  će ti svaki od nas po tisuću i sto srebrnih šekela." 
\par 6 Delila upita Samsona: "Kaži mi gdje stoji tvoja velika  snaga i čime bi se mogao svezati i svladati." 
\par 7 Samson joj odgovori:  "Da me svežu sa sedam svježih još neosušenih žila od luka, onemoćao  bih i postao kao običan čovjek." 
\par 8 Filistejski knezovi donesu Delili sedam svježih još neosušenih  žila i ona ga veza njima. 
\par 9 Kod nje u sobi bijaše zasjeda i  ona viknu: "Samsone, eto Filistejaca na te!" On pokida žile kao  što se prekine kučina kad se primakne ognju. I tako ne doznadoše  za tajnu njegove snage. 
\par 10 Tad reče Delila Samsonu: "Prevario si me i slagao mi.  Ali mi sada kaži čime bi te trebalo vezati." 
\par 11 On joj odgovori:  "Da me dobro svežu novim još neupotrijebljenim užetima, onemoćao  bih i postao kao običan čovjek." 
\par 12 Tada Delila uze nova užeta, sveza ga njima i viknu mu: "Samsone, eto Filistejaca na te!"  Kod nje u sobi bijaše zasjeda, ali on prekide užeta na rukama  kao da su konci. 
\par 13 Tada Delila reče Samsonu: "Varaš me svejednako i lažeš  mi. Kaži mi napokon čime bi te trebalo vezati." On joj odgovori:  "Da otkaš sedam pramenova moje kose na tkalačkom stanu i da ih  zaglaviš klinom, onemoćao bih i postao kao običan čovjek." 
\par 14 Ona  ga uspava i otka sedam pramenova njegove kose na tkalačkom stanu, zabi klin i viknu mu: "Eto Filistejaca na te, Samsone!" On se  probudi i istrgne i klin i tkalački stan. I nije otkrila tajnu  njegove snage. 
\par 15 Delila mu reče: "Kako možeš reći da me ljubiš kad tvoje  srce nije sa mnom? Triput si me već prevario i nisi mi kazao  gdje je tvoja velika snaga." 
\par 16 Kako mu je svakog dana dodijavala  molbama i mučila ga, njemu već dozlogrdje. 
\par 17 I otvori joj cijelo  svoje srce: "Nikada britva nije prešla po mojoj glavi jer sam  od majčine utrobe nazirej Božji. Da me obriju, sva bi me snaga  ostavila, onemoćao bih i postao bih kao običan čovjek." 
\par 18 Delila tad shvati da joj je otvorio cijelo svoje srce;  pozva filistejske knezove i reče im: "Dođite sada jer mi je otvorio  cijelo svoje srce." I filistejski knezovi dođoše k njoj i donesoše  sa sobom novac. 
\par 19 Uspavavši Samsona na svojim koljenima, ona  dozva čovjeka te mu obrija s glave sedam pramenova kose. Tako  on poče slabiti i ostavi ga snaga. 
\par 20 Kad ona povika: "Samsone, eto Filistejaca na te!" on se probudi i pomisli: "Izvući ću  se kao i uvijek i oslobodit ću se." Ali nije znao da se Jahve  od njega okrenuo. 
\par 21 Filistejci ga uhvatiše, iskopaše mu oči  i odvedoše ga u Gazu. Okovaše ga dvostrukim mjedenim lancem te  je okretao mlin u tamnici. 
\par 22 Ali kosa gdje mu je obrijaše počne opet rasti. 
\par 23 A  knezovi se filistejski skupiše da prinesu veliku žrtvu svome  bogu Dagonu i da se provesele. Govorahu oni: "Bog naš predade nam u ruke Samsona, našeg neprijatelja." 
\par 24 A narod, vidjevši ga, uze hvaliti svoga boga i klicati  u njegovu čast govoreći: "Bog naš predade nam u ruke Samsona, našeg neprijatelja, koji nam je zemlju pustošio i tolike naše usmrtio." 
\par 25 Kad im se srce razigralo, povikaše: "Dovedite Samsona  da nas zabavlja!" I dovedoše iz tamnice Samsona i on igraše pred  njima; a onda ga postaviše među stupove. 
\par 26 Samson tada reče  dječaku koji ga je vodio za ruku: "Vodi me i pomozi mi da opipam  stupove na kojima počiva zdanje da se naslonim na njih." 
\par 27 A  kuća bijaše puna ljudi i žena. Bijahu tu i svi filistejski knezovi, a na krovu tri tisuće ljudi koji su gledali kako Samson igra. 
\par 28 Samson zavapi Jahvi: "Gospodine Jahve, spomeni me se  i samo mi još sada podaj snagu da se Filistejcima odjednom osvetim  za oba oka." 
\par 29 I Samson napipa dva srednja stupa na kojima  počivaše zdanje, oprije se o njih, desnom o jedan, a lijevom  o drugi, 
\par 30 i viknu: "Neka poginem s Filistejcima!" Nato uprije  iz sve snage i sruši zdanje na knezove i na sav narod koji se  ondje nalazio. Više ih ubi umirući nego što ih pobi za života. 
\par 31 Poslije dođoše njegova braća i sva kuća njegova oca, uzeše  ga i odnesoše i pokopaše ga između Sore i Eštaola, u grobu Manoaha, oca njegova. On je sudio Izraelu dvadeset godina. 


\chapter{17}

\par 1 Bijaše u Efrajimovoj gori čovjek po imenu Mikajehu. 
\par 2 On  reče majci: "Tisuću i sto srebrnih šekela što su ti ukradeni  i zbog kojih si izustila kletvu - uši su je moje čule - taj je  novac kod mene, ja sam ga uzeo." Mati mu odgovori: "Jahve te  blagoslovio, sine moj!" 
\par 3 I Mikajehu vrati joj tisuću i sto  srebrnih šekela. A mati mu njegova reče: "Te sam novce posvetila  Jahvi iz svoje ruke za tebe, sine moj, da se izdjela za to rezan  ili ljeven idol. I evo, za to ih dajem." 
\par 4 Majka uze dvije stotine srebrnih šekela i dade ih zlataru.  On načini od njih rezani i ljeveni idol koji postaviše u Mikajehuovoj  novoj kući. 
\par 5 On mu sagradi svetište, zatim načini efod i terafe  te posveti jednoga od svojih sinova da mu bude svećenik. 
\par 6 U  to vrijeme u Izraelu nije bilo kralja i svatko je radio po miloj  volji. 
\par 7 Bijaše neki mladić iz Betlehema u Judi, iz Judina plemena;  bio je levit i boravio je ondje kao došljak. 
\par 8 Taj čovjek ode  iz grada Betlehema u Judi da se nastani na kakvu prikladnu mjestu  kao došljak. Putujući, dođe u Efrajimovu goru do Mikine kuće. 
\par 9 Mika ga upita: "Odakle dolaziš?" "Ja sam levit iz Judina Betlehema", odgovori mu on, "i putujem da se negdje nastanim." 
\par 10 "Ostani  kod mene", reče mu Mika, "i budi mi ocem i svećenikom, a ja ću  ti davati deset srebrnih šekela na godinu, haljine i hranu."  I levit uđe. 
\par 11 Levit je pristao da ostane u njega, i mladić mu bijaše  kao jedan od sinova. 
\par 12 Mika posveti levita za svećenika; mladić  je postao njegovim svećenikom i živio je u Mikinoj kući. 
\par 13 "Sad  znam", reče Mika, "da će mi Jahve učiniti dobro kad imam levita  za svećenika." 


\chapter{18}

\par 1 U ono vrijeme ne bijaše kralja u Izraelu. Tada je Danovo pleme  tražilo zemljište gdje da se naseli, jer mu do toga dana nije  dopalo zemljište među Izraelovim plemenima. 
\par 2 Zato poslaše Danovci  petoricu ljudi iz svoga plemena, ljude osobito hrabre iz Sore  i Eštaola, da izvide i upoznaju zemlju. I rekoše im: "Idite,  istražite zemlju." I oni dođoše u Efrajimovu goru, do Mikine  kuće, i ondje zanoćiše. 
\par 3 Kako bijahu blizu Mikine kuće, poznaše  glas mladog levita; svratiše se onamo te ga upitaše: "Tko te  doveo ovamo? Što tu radiš? I što ćeš tu?" 
\par 4 A on im odgovori: "Mika je učinio sa mnom tako i tako.  On me najmio, a ja mu služim kao svećenik." 
\par 5 "Upitaj Boga", kazaše mu, "da znamo hoće li nam uspjeti put koji smo poduzeli." 
\par 6 "Idite u miru", odgovori im svećenik, "put na koji ste pošli  po volji je Jahvi." 
\par 7 Tada odoše ona petorica i stigoše u Lajiš.  I vidješe da narod koji prebiva u njemu živi bez straha - po  običaju Sidonaca - bezbrižno i mirno; imaju svega što rodi zemlja, daleko su od Sidonaca i nemaju nikakvih odnosa s Aramejcima. 
\par 8 Kad se vratiše svojoj braći u Sori i Eštaolu, braća ih  upitaše: "Što ste doznali?" 
\par 9 Oni odgovoriše: "Na noge! Navalimo  na njih! Zemlja koju smo vidjeli vrlo je dobra. O vi, lijenčine!  Ne oklijevajte navaliti da osvojite tu zemlju. 
\par 10 Kada dođete, naći ćete ondje bezbrižan narod. Zemlja je prostrana. Bog je  predao u vaše ruke mjesto koje ne oskudijeva ni u čemu što rodi  zemlja!" 
\par 11 Tako je odande krenulo šest stotina naoružanih ljudi  iz Danova plemena iz Sore i Eštaola. 
\par 12 Krenuli su i utaborili  se u Kirjat Jearimu u Judi. Zato se to mjesto naziva do današnjeg  dana Danovim taborom, a nalazi se na zapadu od Kirjat Jearima. 
\par 13 Odatle se zaputiše u Efrajimovu goru i dođoše do Mikine kuće. 
\par 14 A ona petorica što bijahu išla izviđati zemlju rekoše svojoj  braći: "Znate li da u ovim kućama imaju efod, terafe i ljeveni  idol? Sada pazite što ćete raditi." 
\par 15 Skrenuvši, oni uđoše  u kuću mladog levita, u Mikinu kuću, i pozdraviše ga. 
\par 16 I dok  je šest stotina naoružanih ljudi od Danovih sinova stajalo pred  vratima, 
\par 17 ona petorica što su išla izviđati zemlju uđoše,  uzeše efod, terafe i ljeveni idol, a svećenik stajaše na pragu  pokraj šest stotina naoružanih ljudi. 
\par 18 Kad su ušli u Mikinu kuću i uzeli efod, terafe, rezani  i ljeveni idol, svećenik im reče: "Što to radite?" 
\par 19 "Šuti", odgovoriše mu. "Stavi ruku na usta i hajde s nama. Bit ćeš nam  otac i svećenik. Zar ti je bolje biti svećenikom u kući jednog  čovjeka nego da budeš svećenikom jednog plemena i roda u Izraelu?" 
\par 20 Svećenik se obradova; uze on efod, terafe i rezani i  ljeveni idol te ode s ljudima. 
\par 21 Vrativši se na put kojim su krenuli, odoše pustivši naprijed  žene i djecu, stoku i dragocjenosti. 
\par 22 Bijahu već daleko od  Mikine kuće, kad gle - ljudi što življahu u susjednim kućama, blizu Mikine, uzbunili se i krenuli u potjeru za Danovcima. 
\par 23 Kada počeše vikati za Danovim sinovima, oni se obazreše i  rekoše Miki: "Što ti je? Što ste se skupili?" 
\par 24 On odgovori:  "Uzeli ste moga boga koga sam sebi načinio i svećenika te odlazite.  A što ostaje meni? I još mi kažete: 'Što ti je?'" 
\par 25 Danovci  mu odgovore: "Da te više nismo čuli! Jer bi gnjevni ljudi mogli  udariti na vas te bi upropastio sebe i svoju kuću!" 
\par 26 Danovci  odoše dalje, a Mika, videći da su jači od njega, okrenu se i  vrati kući. 
\par 27 I tako, uzevši boga što ga je načinio Mika i svećenika  koga je najmio da mu služi, Danovci navališe na Lajiš, na mirne  i spokojne ljude, te ih posjekoše oštrim mačem i spališe grad. 
\par 28 Nikoga ne bijaše da pomogne Lajišanima, jer bijahu daleko  od Sidona i ne imahu nikakvih odnosa s Aramejcima, a osim toga  grad bijaše u dolini koja se pruža prema Bet-Rehobu. Potom su  opet sagradili grad i nastanili se u njemu. 
\par 29 I nazvaše ga  Dan, po imenu svoga pretka Dana, koji se rodio Izraelu. A prije  se grad zvao Lajiš. 
\par 30 I Danovci namjestiše sebi rezani i ljeveni idol. A Jonatan, sin Geršona, sina Mojsijeva, a zatim njegovi sinovi, bijahu  svećenici Danova plemena do dana kada je narod bio odveden u  izgnanstvo. 
\par 31 I stajaše im onaj rezani i ljeveni idol što ga  je Mika načinio, i ostade ondje za sve vrijeme dokle Dom Božji  bijaše u Šilu. 


\chapter{19}

\par 1 U ono vrijeme kad u Izraelu još ne bijaše kralja, živio neki  čovjek, levit, kao došljak na kraju Efrajimove gore. Uzeo on  za inoču ženu iz Judina Betlehema. 
\par 2 Rasrdivši se jednom, njegova  ga inoča ostavi i vrati se u očevu kuću u Judin Betlehem i bila  je ondje neko vrijeme, kakva četiri mjeseca. 
\par 3 Njen muž ode  k njoj da je urazumi i dovede natrag; imao je sa sobom slugu  i dva magarca. Dok je prilazio kući oca mlade žene, opazi ga  tast i veselo mu iziđe u susret. 
\par 4 Tast, otac mlade žene, zadrži  ga tri dana kod sebe te su jeli, pili i noćivali. 
\par 5 Četvrtoga  dana uraniše; levit se spremao da ide, kad otac mlade žene reče  zetu: "Okrijepi se zalogajem kruha, pa onda idite." 
\par 6 I tako  sjedoše te su obojica jela i pila, a onda otac mlade žene reče  čovjeku: "Hajde, ostani još noćas i proveseli se!" 
\par 7 A kad čovjek  ustade da pođe, tast uze navaljivati na njega te on još jednom  ondje prenoći. 
\par 8 Petoga dana levit urani da krene, ali mu otac  mlade žene reče: "Okrijepi se najprije!" Tako su proveli vrijeme  jedući zajedno dok se nije nagnuo dan. 
\par 9 Muž ustade da ide,  s inočom i slugom, kad mu tast, otac mlade žene, reče: "Evo se  dan nagnuo k večeru. Prenoći još ovdje i proveseli se, pa sutra  uranite na put i vratite se svom šatoru." 
\par 10 Ali čovjek ne htjede  prenoćiti nego ustade i krenu. Tako je došao do pred Jebus, to  jest Jeruzalem. S njim su bila dva osamarena magarca, inoča i  sluga. 
\par 11 Kad su bili blizu Jeruzalema, dan se već jako nagnuo, pa sluga reče svome gospodaru: "Hajde da se svratimo u taj jebusejski  grad da tu prenoćimo." 
\par 12 Ali mu gospodar odgovori: "Nećemo  se svraćati u grad tuđinaca koji nisu Izraelci, nego ćemo ići  do Gibee." 
\par 13 Još reče sluzi: "Hajde, požurimo se da stignemo  u koje od tih mjesta gdje ćemo prenoćiti, u Gibeu ili Ramu." 
\par 14 I prođoše, nastavljajući put. Kad su stigli pred Benjaminovu  Gibeu, sunce je zapadalo. 
\par 15 Oni skrenuše onamo da prenoće u  Gibei. Ušavši, levit sjede na gradskom trgu, ali ne bijaše nikoga  da ih primi u kuću da prenoće. 
\par 16 I dođe neki starac koji se predvečer vraćao s posla u  polju. Bijaše to čovjek iz Efrajimove gore; življaše u Gibei  kao došljak, a svi žitelji toga mjesta bijahu Benjaminovci. 
\par 17 Podigavši  oči, ugleda putnika na gradskom trgu: "Odakle dolaziš i kamo  ćeš?" - upita ga starac. 
\par 18 A on mu odgovori: "Idemo od Judina  Betlehema, na kraj Efrajimove gore. Ja sam odande. Išao sam u  Judin Betlehem i vraćam se kući, ali nema nikoga da me primi  k sebi u kuću. 
\par 19 Imam i slame i krme za svoje magarce, a i  kruha i vina za sebe, za svoju ženu i za momka koji prati mene, tvoga slugu. Imamo svega dosta." 
\par 20 "Mir s tobom i dobro mi  došao", odgovori starac. "Moja je briga što ti je potrebno, samo  nemoj noćiti na trgu." 
\par 21 I uvede ga u svoju kuću i baci krme  magarcima. Putnici su oprali noge, a onda jeli i pili. 
\par 22 Dok su se oni krijepili, gle, neki građani, opaki ljudi, okružiše kuću i, lupajući svom snagom o vrata, rekoše starcu, gospodaru kuće: "Izvedi toga čovjeka što je ušao u tvoju kuću  da ga se namilujemo." 
\par 23 Tad iziđe domaćin iz kuće i reče im:  "Ne, braćo moja, ne činite zla. Taj je čovjek ušao u moju kuću, zato ne činite bezakonja. 
\par 24 Evo, moja je kći djevica, prepustit  ću vam je. Činite od nje što vam drago, ali ovom čovjeku ne činite  bezakonja." 
\par 25 Ljudi ga ne htjedoše poslušati. Tad onaj čovjek  uze inoču te im je izvede. Oni su je silovali i zlostavljali  svu noć do jutra, a kad je zora zabijeljela, pustiše je. 
\par 26 Pred zoru žena dođe i pade na ulaz kuće onog čovjeka  gdje je bio njen gospodar i ležala je ondje dok se nije razdanilo. 
\par 27 Njen je gospodar ujutro ustao, otvorio kućna vrata te izišao  da nastavi put, kad spazi ženu, svoju inoču, kako leži na kućnim  vratima s rukama na pragu. 
\par 28 "Ustani, idemo!" - reče joj. Ali  ne bijaše odgovora. Onda je uze, natovari na magarca i krenu  na put da se vrati kući. 
\par 29 Kada je došao kući, trže nož i uze  mrtvo tijelo inočino, rasiječe ga, ud po ud, na dvanaest dijelova  te ih razasla u sve krajeve Izraela. 
\par 30 I tko god vidje reče:  "Ovakvo što se nije dogodilo od dana kada su Izraelci izašli  iz Egipta do današnjeg dana. Valja o tome promisliti, vijećati  i govoriti." 


\chapter{20}

\par 1 Tada iziđe sav Izrael i sabra se sva zajednica kao jedan čovjek, od Dana do Beer Šebe i do gileadske zemlje, kod Jahve u Mispi. 
\par 2 Glavari svega naroda, svih Izraelovih plemena, dođoše na zbor  Božjeg naroda, četiri stotine tisuća pješaka vičnih maču. 
\par 3 A  Benjaminovci doznaše da su Izraelovi sinovi uzišli u Mispu. Sinovi  Izraelovi zapitaše tada: "Kažite nam kako se dogodio zločin!" 
\par 4 Levit, muž ubijene žene, uze riječ: "Došao sam s inočom u  Benjaminovu Gibeu da prenoćim. 
\par 5 A građani Gibee ustadoše na  mene i noću opkoliše kuću u kojoj sam bio; mene su htjeli ubiti, a moju su inoču silovali tako da je umrla. 
\par 6 Zato sam uzeo  mrtvu inoču, rasjekao je u komade i razaslao je u sve krajeve  Izraelove baštine, jer su počinili sramotno djelo u Izraelu. 
\par 7 Izraelci, evo vas svih ovdje. Posavjetujte se i ovdje stvorite  odluku." 
\par 8 Sav narod ustade kao jedan čovjek govoreći: "Neka se nitko  od nas ne vraća svome šatoru, neka nitko ne ide svojoj kući! 
\par 9 Nego da sada ovo učinimo Gibei: bacit ćemo ždrijeb; 
\par 10 i  uzet ćemo iz svih Izraelovih plemena po deset ljudi od stotine, po stotinu od tisuće i po tisuću od deset tisuća: oni će nositi  hranu vojsci, onima koji će krenuti da kazne Benjaminovu Gibeu  za sramotu što ju je počinila u Izraelu." 
\par 11 I sabraše se svi  Izraelci protiv onoga grada, udruženi kao jedan čovjek. 
\par 12 Tada Izraelova plemena razaslaše poslanike po svemu Benjaminovu  plemenu s porukom: "Kakav se to zločin dogodio među vama? 
\par 13 Sada  izručite one opake ljude što su u Gibei da ih smaknemo te iskorijenimo  zlo iz Izraela!" Ali Benjaminovci ne htjedoše poslušati svoje  braće Izraelaca. 
\par 14 Benjaminovci se skupiše u Gibeu iz svojih gradova da  se pobiju s Izraelcima. 
\par 15 A Benjaminovaca koji su došli iz  raznih gradova nabrojiše toga dana dvadeset i šest tisuća ljudi  vičnih maču, bez stanovnika Gibee. 
\par 16 Od svega toga naroda bijaše  sedam stotina vrsnih ljudi, koji su bili ljevaci, i svaki je  taj gađao kamenom iz praćke navlas točno, ne promašujući cilja. 
\par 17 A bijaše Izraelaca, osim sinova Benjaminovih, četiri stotine  tisuća, sve ljudi vičnih maču i sve samih ratnika. 
\par 18 I sinovi  Izraelovi, ustavši, pođoše u Betel da se posavjetuju s Bogom:  "Tko će od nas prvi u boj protiv Benjaminovaca?" - zapitaše Izraelci.  A Jahve odgovori: "Neka Juda pođe prvi." 
\par 19 Izjutra krenuše Izraelci te se utaboriše pred Gibeom. 
\par 20 Krenuvši u boj protiv Benjaminovaca, svrstaše se u bojni  red pred Gibeom. 
\par 21 A Benjaminovci iziđoše iz Gibee i pobiše  toga dana Izraelu dvadeset i dvije tisuće ljudi, koji ostadoše  na onome polju. 
\par 22 (20:23) Tada se vojska Izraelovih sinova ohrabri  i nanovo svrsta u bojni red na istome mjestu gdje se svrstala  prvog dana. 
\par 23 (20:22) Izraelci odoše i plakahu pred Jahvom sve  do večeri, a onda upitaše Jahvu govoreći: "Moramo li opet izići  u boj protiv sinova svoga brata Benjamina?" A Jahve im odgovori:  "Pođite na njega!" 
\par 24 Drugoga se dana Izraelci približiše Benjaminovcima, 
\par 25 ali toga drugog dana Benjamin iziđe iz Gibee pred njih i  pobi Izraelcima još osamnaest tisuće ljudi, koji ostadoše na  onome polju - sve sami poizbor ratnici, vični maču. 
\par 26 Tada  svi Izraelci i sav narod odoše u Betel te plakahu i stajahu ondje  pred Jahvom; cio su dan postili do večeri, prinosili paljenice  i žrtve pomirnice pred Jahvom. 
\par 27 I tad opet Izraelci upitaše  Jahvu, jer se u ono vrijeme Kovčeg saveza Božjega nalazio na  tome mjestu, 
\par 28 i Pinhas, sin Aronova sina Eleazara, posluživaše  ga. Oni upitaše: "Moramo li opet izići u boj protiv sinova našega  brata Benjamina?" A Jahve im odgovori: "Pođite, jer ću ih sutra  predati u vaše ruke." 
\par 29 Tad Izrael postavi čete u zasjedu oko Gibee. 
\par 30 Trećega  dana pođoše Izraelci protiv Benjaminovaca i svrstaše se u bojne  redove pred Gibeom, kao i prije. 
\par 31 Benjaminovci iziđoše na  njih, a oni ih odmamiše daleko od grada. Kao i prije, ubijahu  Benjaminovci neke po putovima, od kojih jedan ide u Betel, a  drugi u Gibeu; ubiše tako oko trideset Izraelaca. 
\par 32 I govorahu  Benjaminovci: "Evo ih tučemo kao i prvi put." A Izraelci rekoše:  "Bježimo dok ih ne odmamimo na otvorene putove, daleko od grada!" 
\par 33 Tada se glavnina Izraelove vojske pomakne sa svoga položaja  i svrsta se u bojni red kod Baal Tamara, a zasjeda Izraelova  iziđe iz svog skrovišta zapadno od Gibee. 
\par 34 Deset tisuća vrsnih  ljudi izabranih iz sveg Izraela sleže se prema Gibei. Boj bijaše  žestok. Benjaminovci nisu ni slutili da će ih zadesiti zlo. 
\par 35 I  Jahve potuče Benjamina pred Izraelom toga dana te Izraelci pobiše  Benjaminu dvadeset i pet tisuća i sto ljudi vičnih maču. 
\par 36 Benjaminovci  vidješe da su pobijeđeni. Ljudi Izraelci bijahu se povukli sa svojih bojnih položaja  pred Benjaminom uzdajući se u zasjedu što su je postavili oko  Gibee. 
\par 37 A oni koji bijahu u zasjedi navališe brže na Gibeu  i, ušavši u nju, posjekoše oštrim mačem sve stanovništvo. 
\par 38 Izraelovi  se ljudi bijahu dogovorili s onima u zasjedi da ovi podignu iz  grada stup dima kao znak: 
\par 39 tada bi se Izraelovi ljudi povukli  iz boja. Benjamin poče ubijati Izraelce i posiječe im tridesetak  ljudi. "Doista, padaju pred nama kao u prijašnjem boju." 
\par 40 A  kada se znak, stup dima, počeo dizati iz grada, obazre se Benjamin  i vidje kako se plamen iz svega grada diže prema nebu. 
\par 41 Tada  se Izraelovi ljudi okrenuše, a Benjaminovce obuze užas jer vidješe  da ih je zadesilo zlo. 
\par 42 I pobjegoše ispred Izraelaca prema pustinji, ali im ratnici  bijahu za petama, a oni što su dolazili iz grada ubijahu ih s  leđa. 
\par 43 Tako su opkolili Benjamina i, goneći ga bez predaha, uništiše ga pred Gibeom na istočnoj strani. 
\par 44 I palo je Benjaminu  osmnaest tisuća ljudi, sve samih vrsnih junaka. 
\par 45 Preživjeli  se okrenuše i pobjegoše u pustinju prema Rimonskoj stijeni. Sijekući  po cestama, Izraelci pobiše još pet tisuća ljudi; a onda pognaše  Benjamina do Gideoma i pobiše još dvije tisuće ljudi. 
\par 46 Toga  dana palo je Benjaminovaca dvadeset tisuća ljudi vičnih maču, sve samih vrsnih junaka. 
\par 47 Šest stotina ljudi pobjeglo je  u pustinju prema Rimonskoj stijeni. 
\par 48 Izraelovi se ljudi vratiše  potom Benjaminovcima, posjekoše oštrim mačem muškarce u gradovima, stoku i što se god našlo; i sve gradove na koje su naišli u  Benjaminu popališe ognjem. 



\chapter{21}

\par 1 Izraelovi se ljudi bijahu ovako zakleli u Mispi: "Nitko od  nas neće dati svoju kćer za ženu Benjaminovu sinu." 
\par 2 I ode  narod u Betel i ostade ondje pred Bogom do večeri, naričući i  jecajući. 
\par 3 Govorili su: "Zašto se, o Jahve, Bože Izraelov,  ova nesreća morala dogoditi da Izraelu danas nestane jednog plemena?" 
\par 4 Sutradan uraniše ljudi i sagradiše ondje žrtvenik; prinesoše  paljenice i žrtve zahvalnice. 
\par 5 Tad zapitaše Izraelci: "Ima  li koga među svim plemenima Izraelovim da nije došao na zbor  Jahvi?" Jer su se svečano zakleli da će pogubiti onoga tko ne  dođe u Mispu k Jahvi. 
\par 6 Izraelcima se sada sažalilo na brata Benjamina te rekoše:  "Danas je otkinuto jedno pleme od Izraela. 
\par 7 Kako ćemo dati  žene onima koji su preostali kad se zaklesmo Jahvom da im nećemo  dati svojih kćeri za žene?" 
\par 8 Zato zapitaše: "Ima li koga među Izraelovim plemenima  da nije došao k Jahvi u Mispu?" I pronađe se da nije došao u  tabor, na zbor, nitko od žitelja Jabeša u Gileadu. 
\par 9 Jer kada  se narod prebrojio, ondje ne bijaše nikoga od žitelja Jabeša  u Gileadu. 
\par 10 Zato zajednica posla onamo dvanaest tisuća hrabrih  ljudi i zapovjedi im: "Idite i posijecite oštrim mačem stanovnike  Jabeša u Gileadu, zajedno sa ženama i djecom. 
\par 11 Evo što ćete  učiniti: izručit ćete prokletstvu sve muškarce i sve žene što  su dijelile postelju sa čovjekom, ali ćete sačuvati život djevicama."  Tako i učiniše. 
\par 12 I našli su među stanovnicima Jabeša u Gileadu  četiri stotine mladih djevojaka koje nisu dijelile postelje s  čovjekom i doveli su ih u tabor u Šilu, koji je u Kanaanu. 
\par 13 Sva zajednica posla tada poslanike Benjaminovcima koji  bijahu na Rimonskoj stijeni: objaviše im mir. 
\par 14 Tako se oporavi  Benjamin. Dadoše im one među ženama iz Jabeša u Gileadu koje  su ostavili na životu, ali ih ne bijaše dovoljno za sve. 
\par 15 Narodu se sažalio Benjamin što je Jahve načinio prazninu  među Izraelovim plemenima. 
\par 16 "Kako ćemo naći žene onima što  su ostali", rekoše starješine zbora, "kad su Benjaminu istrijebljene  žene?" 
\par 17 Rekoše još: "Kako sačuvati ostatak Benjaminu da se  ne zatre jedno pleme iz Izraela? 
\par 18 A ne možemo im dati svoje  kćeri za žene." Jer se bijahu zakleli rekavši: "Proklet bio onaj  koji dade ženu Benjaminu!" 
\par 19 "Ali", rekoše, "svake se godine slavi u Šilu Jahvina  svetkovina." Grad se nalazi na sjeveru od Betela, istočno od  ceste koja vodi iz Betela u Šekem i južno od Lebone. 
\par 20 I zato  svjetovaše Benjaminovce: "Idite u zasjedu po vinogradima. 
\par 21 Pazite, pa kada djevojke iz Šila iziđu da plešu u kolu, vi iskočite  iz vinograda, otmite svaki sebi ženu između šilskih kćeri pa  otiđite u Benjaminovu zemlju. 
\par 22 A kad njihovi očevi ili njihova  braća dođu da se prituže na vas, mi ćemo im reći: 'Oprostite  im što je svaki uzeo po ženu kao u ratu; vi im ih niste dali, pa je tako krivnja na vama.'" 
\par 23 Benjaminovci učiniše tako i od djevojaka koje oteše uzeše  onoliki broj žena koliko bijaše njih. Onda odoše svaki na svoju  baštinu, sagradiše opet gradove i naseliše se u njima. 
\par 24 Izraelci se tada raziđoše, svaki u svoje pleme i u svoj  rod, i svaki se odande vrati na svoju baštinu. 
\par 25 U to vrijeme ne bijaše kralja u Izraelu i svatko je živio  kako mu se činilo da je pravo. 





\end{document}