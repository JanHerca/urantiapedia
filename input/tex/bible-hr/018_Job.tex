\begin{document}

\title{Job}


\chapter{1}

\par 1 Bijaše nekoć u zemlji Usu čovjek po imenu Job. Bio je to čovjek  neporočan i pravedan: bojao se Boga i klonio zla. 
\par 2 Rodilo mu  se sedam sinova i tri kćeri. 
\par 3 Imao je sedam tisuća ovaca, tri  tisuće deva, pet stotina jarmova goveda, pet stotina magarica  i veoma mnogo služinčadi. Čovjek taj bijaše najugledniji među  svim istočnjacima. 
\par 4 Sinovi su njegovi običavali naizmjence  priređivati gozbe kod jednoga od njih, svaki u svoj dan, te su  pozivali svoje tri sestre da jedu i piju s njima. 
\par 5 A kad bi  se izredali s gozbama, Job bi ih pozvao na očišćenje. Uranio  bi izjutra i prinio paljenice za svakog od njih; mislio je: "Tko  zna nisu li mi sinovi griješili i u srcu Boga hulili!" Tako je  Job svagda činio. 
\par 6 Jednoga dana dođu sinovi Božji da stanu pred Jahvu, a  među njima pristupi i Satan. 
\par 7 Jahve tad upita Satana: "Odakle  dolaziš?" - "Evo prođoh zemljom i obiđoh je", odgovori on. 
\par 8 Nato  će Jahve: "Nisi li zapazio slugu moga Joba? Njemu na zemlji nema  ravna. Čovjek je to neporočan i pravedan, boji se Boga i kloni  zla!" 
\par 9 A Satan odgovori Jahvi: "Zar se Job uzalud Boga boji? 
\par 10 Zar nisi ogradio njega, kuću mu i sav posjed njegov? Blagoslovio  si djelo njegovih ruku, stoka mu se namnožila po zemlji. 
\par 11 Ali  pruži jednom ruku i dirni mu u dobra: u lice će te prokleti!" 
\par 12 "Neka ti bude! - reče Jahve Satanu. - Sa svime što ima radi  što ti drago; samo ruku svoju na nj ne diži." I Satan ode ispred  lica Jahvina. 
\par 13 Jednoga dana, dok su Jobovi sinovi i kćeri jeli i pili  vino u kući najstarijeg brata, 
\par 14 dođe glasnik k Jobu i reče:  "Tvoji su volovi orali a magarice pokraj njih pasle, 
\par 15 kad  iznenada Sabejci navališe na njih i oteše ih, pobivši momke oštrim  mačem. Jedini ja utekoh da ti ovo javim." 
\par 16 Dok je on još to  govorio, dođe drugi i reče: "Oganj Božji udari s neba, spali  tvoje ovce i pastire te ih proždrije. Jedini ja utekoh da ti  javim." 
\par 17 Dok je još govorio, dođe treći i reče: "Kaldejci  navališe sa tri čete na tvoje deve i oteše ih, pobivši momke  oštrim mačem. Jedini ja utekoh da ti javim." 
\par 18 Dok je ovaj  još govorio, dođe četvrti i reče: "Tvoji su sinovi i kćeri jeli  i pili vino u kući najstarijeg brata. 
\par 19 I gle, vjetar se silan  diže iz pustinje, udari na sva četiri ugla kuće, obori je na  djecu te ona zaglaviše. Jedini ja utekoh da ti javim." 
\par 20 Tad ustade Job, razdrije haljinu na sebi, obrija glavu  pa ničice pade na zemlju, pokloni se 
\par 21 i reče: "Go iziđoh iz krila majčina, go ću se onamo i vratiti. Jahve dao, Jahve oduzeo! Blagoslovljeno ime Jahvino!" 
\par 22 Uza sve to, nije sagriješio Job niti je kakvu ludost  protiv Boga izustio. 


\chapter{2}

\par 1 Jednoga dana dođu opet sinovi Božji da stanu pred Jahvu, a  među njima pristupi i Satan. 
\par 2 Jahve tad upita Satana: "Odakle  dolaziš?" - "Evo prođoh zemljom i obiđoh je", odgovori on. 
\par 3 Nato  će Jahve: "Nisi li zapazio slugu moga Joba? Njemu na zemlji nema  ravna. Čovjek je to neporočan i pravedan: boji se Boga i kloni  zla! On je još postojan u neporočnosti, pa si me uzalud izazvao  da ga upropastim." 
\par 4 A Satan odvrati: "Koža za kožu! Sve što  čovjek ima dat će za život. 
\par 5 Ali pruži ruku, dotakni se kosti  njegove i mesa: u lice će te prokleti!" 
\par 6 "Neka ti bude! - reče  Jahve Satanu. - U tvojoj je ruci; život mu samo sačuvaj!" 
\par 7 I  Satan ode ispred lica Jahvina. On udari Joba zlim prištem od tabana do tjemena. 
\par 8 Job uze  crijep da se struže njime i sjede u pepeo. 
\par 9 Tada mu njegova  žena reče: "Zar si još postojan u neporočnosti? Prokuni Boga  i umri!" 
\par 10 Job joj odgovori: "Brbljaš kao luđakinja! Kad od  Boga primamo dobro, zar da onda i zlo ne primimo?" U svemu tome  Job nije sagriješio svojim usnama. 
\par 11 U to čuše tri Jobova prijatelja za sve nevolje koje ga  zadesiše; svaki se zaputi iz svoga kraja - Elifaz iz Temana,  Bildad iz Šuaha, Sofar iz Naama - i odlučiše da odu zajedno ožaliti  ga i utješiti. 
\par 12 A kad su izdaleka upravili oči na njega, nisu  ga prepoznali. Tad udariše u plač; svaki razdrije svoju haljinu  i prosu prah po glavi. 
\par 13 Potom sjedoše kraj njega na zemlju  i ostadoše tako sedam dana i sedam noći. Nijedan mu ne progovori  ni riječi, jer vidješe da je velika njegova bol. 


\chapter{3}

\par 1 Napokon otvori Job usta i prokle dan svoj; 
\par 2 poče svoju besjedu  i reče: 
\par 3 "O, ne bilo dana kad sam se rodio i noći što javi: 'Začeo se dječak!' 
\par 4 U crnu tminu dan taj nek se prometne! S visina se njega Bog ne spominjao, svjetlost sunčeva ne svijetlila mu više! 
\par 5 Mrak i sjena smrtna o nj se otimali, posvema ga tmina gusta prekrila, pomrčine dnevne stravom ga morile! 
\par 6 O, da bi ga tama svega presvojila, nek' se ne dodaje danima godine, nek' ne ulazi u brojenje mjeseci! 
\par 7 A noć ona bila žalosna dovijeka, ne čulo se u njoj radosno klicanje! 
\par 8 Prokleli je oni štono dan proklinju i Levijatana probudit' su kadri! 
\par 9 Pomrčale zvijezde njezina svanuća, zaludu se ona vidjelu nadala, i zorinih vjeđa ne gledala nigda! 
\par 10 Što mi od utrobe ne zatvori vrata da sakrije muku od mojih očiju! 
\par 11 Što nisam mrtav od krila materina, što ne izdahnuh izlazeć' iz utrobe? 
\par 12 Čemu su me dva koljena prihvatila i dojke dvije da me nejaka podoje? 
\par 13 U miru bih vječnom počivao sada, spavao bih, pokoj svoj bih uživao 
\par 14 s kraljevima i savjetnicima zemlje koji su sebi pogradili grobnice, 
\par 15 ili s knezovima, zlatom bogatima, što su kuće svoje srebrom napunili. 
\par 16 Ne bih bio - k'o nedonošče zakopano, k'o novorođenče što svjetla ne vidje. 
\par 17 Zlikovci se više ne obijeste ondje, iznemogli tamo nalaze počinka. 
\par 18 Sužnjeve na miru tamo ostavljaju: ne slušaju više poviku stražara. 
\par 19 Malen ondje leži zajedno s velikim, rob je slobodan od gospodara svoga. 
\par 20 Čemu darovati svjetlo nesretniku i život ljudima zagorčene duše 
\par 21 koji smrt ištu, a ona ne dolazi, i kao za blagom za njome kopaju? 
\par 22 Grobnom bi se humku oni radovali, klicali od sreće kad bi grob svoj našli. 
\par 23 Što će to čovjeku kom je put sakriven, koga je Bog sa svih strana zapriječio? 
\par 24 Zato videć' hranu, uzdahnuti moram, k'o voda se moji razlijevaju krici. 
\par 25 Obistinjuje se moje strahovanje, snalazi me, evo, čega god se bojah. 
\par 26 Pokoja ni mira meni više nema, u mukama mojim nikad mi počinka." 


\chapter{4}

\par 1 Tad prozbori Elifaz Temanac i reče: 
\par 2 "Možeš li podnijeti da ti progovorim? Ali tko se može uzdržat' od riječi! 
\par 3 Eto, mnoge ljude ti si poučio, okrijepio si iznemogle mišice; 
\par 4 riječju svojom klonule si pridizao, ojačavao si koljena klecava. 
\par 5 A kad tebe stiže, klonuo si duhom, na tebe kad pade, čitav si se smeo! 
\par 6 Zar pobožnost tvoja nadu ti ne daje, neporočnost tvoja životu ufanje? 
\par 7 TÓa sjeti se: nevin - propade li kada? Kada su zatrti bili pravednici? 
\par 8 Iz iskustva zborim: nesrećom tko ore i nevolju sije, nju će i požeti. 
\par 9 Od daha Božjega oni pogibaju, na gnjevni mu disaj nestaju sa zemlje. 
\par 10 Rika lavlja, urlik leopardov krše se k'o zubi u lavića. 
\par 11 Lav ugiba jer mu nesta plijena, rasuli se mladi lavičini. 
\par 12 Tajna riječ se meni objavila, šapat njen je uho moje čulo. 
\par 13 Noću, kada snovi duh obuzmu i san dubok kad na ljude pada, 
\par 14 strah i trepet mene su svladali, kosti moje žestoko se stresle. 
\par 15 Dah mi neki preko lica prođe, digoše se dlake na mom tijelu. 
\par 16 Stajao je netko - lica mu ne poznah - ali likom bješe pred očima mojim. Posvuda tišina; uto začuh šapat: 
\par 17 'Zar je smrtnik koji pred Bogom pravedan? Zar je čovjek čist pred svojim Stvoriteljem? 
\par 18 Ni slugama svojim više ne vjeruje, i anđele svoje za grijeh okrivljuje - 
\par 19 kako ne bi onda goste stanova glinenih kojima je temelj u prahu zemaljskom. Gle, kao moljce njih sveudilj satiru: 
\par 20 od jutra do mraka u prah pretvore, nestaju zasvagda - nitko i ne vidi. 
\par 21 Iščupan je kolčić njihova šatora, pogibaju skoro, mudrost ne spoznavši.' 


\chapter{5}

\par 1 Ded zazivlji! Zar će ti se tko odazvat'?  Kojem li se svecu misliš sad obratit'? 
\par 2 Doista, budalu njegov bijes ubija, luđaka će sasvim skončat ljubomora. 
\par 3 Bezumnika vidjeh kako korijen pušta, al' prokletstvo skoro na kuću mu pade. 
\par 4 Njegovi su sinci daleko od spasa, njih nezaštićene na Vratima tlače. 
\par 5 Ljetinu njihovu pojedoše gladni, sam Bog ju je njima oteo iz usta, a žedni hlepe za njihovim dobrima. 
\par 6 Ne, opačina ne izbija iz zemlje, nit' nevolja iz tla može nići sama, 
\par 7 nego čovjek rađa muku i nevolju kao što let orlov teži u visinu. 
\par 8 Al' ja bih se ipak Bogu utekao i pred njime stvar bih svoju razložio. 
\par 9 Nedokučiva on djela silna stvori, čudesa koja se izbrojit' ne mogu. 
\par 10 On kišom rosi po svem licu zemljinu i vodu šalje da nam polja natapa. 
\par 11 Da bi ponižene visoko digao, da bi ojađene srećom obdario, 
\par 12 redom ruši ono što lukavci smisle, što god započeli, on im izjalovi. 
\par 13 On hvata mudre u njihovu lukavstvu, naume spletkara obraća u ništa. 
\par 14 Posred bijela dana zapadnu u tamu, pipaju u podne kao usred noći. 
\par 15 On iz njinih ralja izbavlja jadnika, iz silničkih ruku diže siromaha. 
\par 16 Tako se pokaže nada nevoljniku, i nepravda mora zatvoriti usta. 
\par 17 Da, blago čovjeku koga Bog odbaci! Stoga ti ne prezri karanje Svesilnog! 
\par 18 On ranjava, ali i ranu povija, udara i svojom zacjeljuje rukom. 
\par 19 Iz šest će nevolja tebe izbaviti, ni u sedmoj zlo te dotaknuti neće. 
\par 20 U gladi, od smrti on će te spasiti, a u ratu, oštru će te otet maču. 
\par 21 Biču zla jezika uklonit će tebe, ispred otimača bez straha ćeš biti. 
\par 22 Suši i studeni ti ćeš se smijati i od divljih zvijeri strahovati nećeš. 
\par 23 Sklopit' ti ćeš savez s kamenjem na njivi, zvjerad divlja s tobom u miru će biti. 
\par 24 U šatoru svome mir ćeš uživati, dom svoj kad pohodiš netaknut će stajat. 
\par 25 Koljeno ćeš svoje gledat' gdje se množi i potomstvo gdje ti kao trava raste. 
\par 26 U grob ti ćeš leći kada budeš zreo, kao što se žito snosi kad dozori. 
\par 27 Sve motrismo ovo: istina je živa! zato sve za dobro svoje ti poslušaj." 


\chapter{6}

\par 1 A Job progovori i reče: 
\par 2 "O, kad bi se jad moj izmjeriti mog'o, a nevolje moje stavit' na tezulju! 
\par 3 Teže one jesu od sveg pijeska morskog, i stoga mi riječi zastraniti znaju. 
\par 4 Strijele Svesilnoga u mojem su mesu, ljuti otrov njihov ispija mi dušu, Božje se strahote oborile na me. 
\par 5 TÓa, kraj svježe trave njače li magarac, muče li goveče kraj punih jasala? 
\par 6 Zar hranu bljutavu jedemo bez soli? Zar kakove slasti ima u bjelancu? 
\par 7 Al' ono što mi se gadilo dotaći, to mi je sada sva hrana u bolesti. 
\par 8 O, da bi se molba moja uslišala, da mi Bog ispuni ono čem se nadam! 
\par 9 O, kada bi me Bog uništiti htio, kada bi mahnuo rukom da me satre! 
\par 10 Za mene bi prava utjeha to bila, klicati bih mog'o u mukama teškim što se ne protivljah odluci Svetoga. 
\par 11 Zar snage imam da mogu čekati? Radi kakve svrhe da ja duže živim? 
\par 12 Zar je snaga moja k'o snaga kamena, zar je tijelo moje od mjedi liveno? 
\par 13 Na što se u sebi osloniti mogu? Zar mi svaka pomoć nije uskraćena? 
\par 14 Tko odbija milost bližnjemu svojemu, prezreo je strah od Boga Svesilnoga. 
\par 15 Kao potok me iznevjeriše braća, kao bujice zimske svoje korito. 
\par 16 Od leda mutne vode im se nadimlju, 'bujaju od snijega što se topit' stao; 
\par 17 al u doba sušno naskoro presahnu, od žege ishlape tada iz korita. 
\par 18 Karavane zbog njih skreću sa putova, u pustinju zađu i u njoj se gube. 
\par 19 Karavane temske očima ih traže, putnici iz Šebe nadaju se njima. 
\par 20 A kad do njih dođu, nađu se u čudu, jer su se u nadi svojoj prevarili. 
\par 21 U ovom ste času i vi meni takvi: vidjeste strahotu pa se preplašiste. 
\par 22 Rekoh li vam možda: 'Darujte mi štogod, poklonite nešto od svojega blaga; 
\par 23 iz šake dušmanske izbavite mene, oslobodite me silnikova jarma?' 
\par 24 Vi me poučite, pa ću ušutjeti, u čem je moj prijestup, pokažite meni. 
\par 25 O, kako su snažne besjede iskrene! Al' kamo to vaši smjeraju prijekori? 
\par 26 Mislite li možda prekoriti riječi? TÓa u vjetar ide govor očajnikov! 
\par 27 Nad sirotom kocku zar biste bacali i sa prijateljem trgovali svojim? 
\par 28 U oči me sada dobro pogledajte, paz'te neću li vam slagati u lice. 
\par 29 Povucite riječ! Kakve li nepravde! Povucite riječ, neporočan ja sam! 
\par 30 Zar pakosti ima na usnama mojim? Zar nesreću svaku okusio nisam? 


\chapter{7}

\par 1 Nije l' vojska život čovjekov na zemlji?  Ne provodi l' dane poput najamnika? 
\par 2 Kao što trudan rob za hladom žudi, poput nadničara štono plaću čeka, 
\par 3 mjeseci jada tako me zapadoše i noći su mučne meni dosuđene. 
\par 4 Liježuć' mislim svagda: 'Kada ću ustati?' A dižuć se: 'Kada večer dočekati!' I tako se kinjim sve dok se ne smrkne. 
\par 5 PÓut moju crvi i blato odjenuše, koža na meni puca i raščinja se. 
\par 6 Dani moji brže od čunka prođoše, promakoše hitro bez ikakve nade. 
\par 7 Spomeni se: život moj je samo lahor i oči mi neće više vidjet' sreće! 
\par 8 Prijateljsko oko neće me gledati; pogled svoj u mene upro si te sahnem. 
\par 9 Kao što se oblak gubi i raspline, tko u Šeol siđe, više ne izlazi. 
\par 10 Domu svome natrag ne vraća se nikad, njegovo ga mjesto više ne poznaje. 
\par 11 Ustima ja svojim stoga branit' neću, u tjeskobi duha govorit ću sada, u gorčini duše ja ću zajecati. 
\par 12 Zar sam more ili neman morska, pa si stražu nada mnom stavio? 
\par 13 Kažem li: 'Na logu ću se smirit', ležaj će mi olakšati muke', 
\par 14 snovima me prestravljuješ tada, prepadaš me viđenjima mučnim. 
\par 15 Kamo sreće da mi se zadavit'! Smrt mi je od patnja mojih draža. 
\par 16 Ja ginem i vječno živjet neću; pusti me, tek dah su dani moji! 
\par 17 Što je čovjek da ga toliko ti cijeniš, da je srcu tvojem tako prirastao 
\par 18 i svakoga jutra da njega pohodiš i svakoga trena da ga iskušavaš? 
\par 19 Kada ćeš svoj pogled skinuti sa mene i dati mi barem pljuvačku progutat'? 
\par 20 Ako sam zgriješio, što učinih tebi, o ti koji pomno nadzireš čovjeka? Zašto si k'o metu mene ti uzeo, zbog čega sam tebi na teret postao? 
\par 21 Zar prijestupa moga ne možeš podnijeti i ne možeš prijeći preko krivnje moje? Jer, malo će proći i u prah ću leći, ti ćeš me tražiti, al' me biti neće." 


\chapter{8}

\par 1 Bildan iz Šuaha progovori tad i reče: 
\par 2 "Dokad ćeš jošte govoriti tako, dokle će ti riječ kao vihor biti? 
\par 3 TÓa zar može Bog pravo pogaziti, može li pravdu izvrnut' Svesilni? 
\par 4 Ako mu djeca tvoja sagriješiše, preda ih zato bezakonju njinu. 
\par 5 Al' ako Boga potražiš iskreno i od Svesilnog milost ti izmoliš; 
\par 6 ako li budeš čist i neporočan, odsad će svagda on nad tobom bdjeti i obnovit će kuću pravedniku. 
\par 7 Bit će malena tvoja sreća prošla prema budućoj što te očekuje. 
\par 8 No pitaj samo prošle naraštaje, na mudrost pređa njihovih pripazi. 
\par 9 Od jučer mi smo i ništa ne znamo, poput sjene su na zemlji nam dani. 
\par 10 Oni će te poučit' i reći ti, iz srca će svog izvući besjede: 
\par 11 'Izvan močvare zar će rogoz nići? Zar će bez vode trstika narasti? 
\par 12 Zeleni se sva, al' i nekošena usahne prije svake druge trave. 
\par 13 To je kob svakog tko Boga zaboravi; tako propada nada bezbožnika: 
\par 14 Nit je tanana njegovo uzdanje, a ufanje mu kuća paukova. 
\par 15 Nasloni li se, ona mu se ljulja, prihvati li se, ona mu se ruši. 
\par 16 Zeleni se i sav na suncu buja, vrt su mu cio mladice prerasle. 
\par 17 Svojim korijenjem krš je isprepleo te život crpe iz živa kamena. 
\par 18 A kad ga s mjesta njegova istrgnu, ono ga niječe: 'Nikada te ne vidjeh!' 
\par 19 I evo gdje na putu sada trune dok drugo bilje već niče iz zemlje. 
\par 20 Ne, Bog neće odbacit' neporočne, niti će rukom poduprijet' opake. 
\par 21 Smijeh će ti opet ispuniti usta, s usana će odjeknuti klicanje. 
\par 22 Dušmane će ti odjenut' sramota i šatora će nestat' zlikovačkog.'" 


\chapter{9}

\par 1 Job progovori i reče: 
\par 2 "Zaista, dobro ja znadem da je tako: kako da pred Bogom čovjek ima pravo? 
\par 3 Ako bi se tkogod htio prÓeti s njime, odvratio mu ne bi ni jednom od tisuću. 
\par 4 Srcem on je mudar, a snagom svesilan, i tko bi se njemu nekažnjeno opro? 
\par 5 On brda premješta, a ona to ne znaju, u jarosti svojoj on ih preokreće. 
\par 6 Pokreće on zemlju sa njezina mjesta, iz temelja njene potresa stupove. 
\par 7 Kad zaprijeti suncu, ono se ne rađa, on pečatom svojim i zvijezde pečati. 
\par 8 Jedini on je nebesa razapeo i pučinom morskom samo on hodao. 
\par 9 Stvorio je Medvjede i Oriona, Vlašiće i zvijezđa na južnome nebu. 
\par 10 Tvorac on je djela silnih, nepojmljivih čudesa koja se izbrojit' ne mogu. 
\par 11 Ide pored mene, a ja ga ne vidim; evo, on prolazi - ja ga ne opažam. 
\par 12 Ugrabi li što, tko će mu to priječit, i tko ga pitat smije: 'Što si učinio?' 
\par 13 Bog silni srdžbu svoju ne opoziva: pred njim poniču saveznici Rahaba. 
\par 14 Pa kako onda da njemu odgovorim, koju riječ da protiv njega izaberem? 
\par 15 I da sam u pravu, odvratio ne bih, u suca svojega milost bih molio. 
\par 16 A kad bi se na zov moj i odazvao, vjerovao ne bih da on glas moj sluša. 
\par 17 Jer, za dlaku jednu on mene satire, bez razloga moje rane umnožava. 
\par 18 Ni časa jednoga predahnut' mi ne da, nego mene svakom gorčinom napaja! 
\par 19 Ako je na snagu - tÓa on je najjači! Ako je na pravdu - tko će njega na sud? 
\par 20 Da sam i prav, usta bi me osudila, da sam i nevin, zlim bi me proglasila. 
\par 21 A jesam li nevin? Ni sam ne znam više, moj je život meni sasvim omrzao! 
\par 22 Jer, to je svejedno; i zato ja kažem: nevina i grešnika on dokončava. 
\par 23 I bič smrtni kad bi odjednom ubijo ... ali on se ruga nevolji nevinih. 
\par 24 U zemlji predanoj u šake zlikovaca, on oči sucima njezinim zastire. Ako on to nije, tko je drugi onda? 
\par 25 Od skoroteče su brži moji dani, bježe daleko, nigdje dobra ne videć.' 
\par 26 K'o čamci od rogoza hitro promiču, k'o orao na plijen kada se zaleti. 
\par 27 Kažem li: zaboravit ću jadikovku, razvedrit ću lice i veseo biti, 
\par 28 od mojih me muka groza obuzima, jer znadem da me ti ne držiš nevinim. 
\par 29 Ako li sam grešan, tÓa čemu onda da zalud mučim sebe. 
\par 30 Kad bih i sniježnicom sebe ja isprao, kad bih i lugom ruke svoje umio, 
\par 31 u veću bi me nečist opet gurnuo, i moje bi me se gnušale haljine! 
\par 32 Nije čovjek k'o ja da se s njime pravdam i na sud da idem s njim se parničiti. 
\par 33 Niti kakva suca ima među nama da ruke svoje stavi na nas dvojicu, 
\par 34 da šibu njegovu od mene odmakne, da užas njegov mene više ne plaši! 
\par 35 Govorit ću ipak bez ikakva straha, jer ja nisam takav u svojim očima! 


\chapter{10}

\par 1 Kad mi je duši život omrznuo, nek' mi tužaljka poteče slobodno, zborit ću u gorčini duše svoje. 
\par 2 Reću ću Bogu: Nemoj me osudit! Kaži mi zašto se na me obaraš. 
\par 3 TÓa što od toga imaš da me tlačiš, da djelo ruku svojih zabacuješ, da pomažeš namjerama opakih? 
\par 4 Jesu li u tebe oči tjelesne? Zar ti vidiš kao što čovjek vidi? 
\par 5 Zar su ti dani k'o dani smrtnika a kao ljudski vijek tvoje godine? 
\par 6 Zbog čega krivnju moju istražuješ i grijehe moje hoćeš razotkriti, 
\par 7 kad znadeš dobro da sam nedužan, da ruci tvojoj izmaknut ne mogu? 
\par 8 Tvoje me ruke sazdaše, stvoriše, zašto da me sada opet raščiniš! 
\par 9 Sjeti se, k'o glinu si me sazdao i u prah ćeš me ponovo vratiti. 
\par 10 Nisi li mene k'o mlijeko ulio i učinio da se k'o sir zgrušam? 
\par 11 Kožom si me i mesom odjenuo, kostima si me spleo i žilama. 
\par 12 S milošću si mi život darovao, brižljivo si nad mojim bdio dahom. 
\par 13 Al' u svom srcu ovo si sakrio, znam da je tvoja to bila namjera: 
\par 14 da paziš budno hoću li zgriješiti i da mi grijeh ne prođe nekažnjeno. 
\par 15 Ako sam grešan, onda teško meni, ako li sam prav, glavu ne smijem dići - shrvan sramotom, nesrećom napojen! 
\par 16 Ispravim li se, k'o lav me nagoniš, snagu svoju okušavaš na meni, 
\par 17 optužbe nove na mene podižeš, jarošću većom na mene usplamtiš i sa svježim se četama obaraš. 
\par 18 Iz utrobe što si me izvukao? O, što ne umrijeh: vidjeli me ne bi, 
\par 19 bio bih k'o da me ni bilo nije, iz utrobe u grob bi me stavili. 
\par 20 Mog su života dani tako kratki! Pusti me da se još malo veselim 
\par 21 prije nego ću na put bez povratka, u zemlju tame, zemlju sjene smrtne, 
\par 22 u zemlju tmine guste i meteža, gdje je svjetlost slična noći najcrnjoj." 


\chapter{11}

\par 1 Sofar iz Naama progovori tad i reče: 
\par 2 "Zar na riječi mnoge da se ne odvrati? Zar će se brbljavac još i opravdati? 
\par 3 Zar će tvoje trice ušutkati ljude, zar će ruganje ostat' neizrugano? 
\par 4 Rekao si: 'Nauk moj je neporočan, u očima tvojim čist sam i bez ljage.' 
\par 5 Ali kada bi Bog htio progovorit' i otvorit usta da ti odgovori 
\par 6 kada bi ti tajne mudrosti otkrio koje um nijedan ne može doumit', znao bi da ti za grijehe račun ište. 
\par 7 Možeš li dubine Božje proniknuti, dokučiti savršenstvo Svesilnoga? 
\par 8 Od neba je više: što još da učiniš? Od Šeola dublje: što još da mudruješ? 
\par 9 Duže je od zemlje - šire je od mora! 
\par 10 Ako se povuče, ako te pograbi, ako na sud preda, tko će mu braniti? 
\par 11 Jer on u čovjeku prozire prijevaru, vidi opačinu ako i ne gleda. 
\par 12 Čovjek se bezuman obraća k pameti i divlji magarac uzdi se pokori. 
\par 13 Ako li srce svoje ti uspraviš i ruke svoje pružiš prema njemu, 
\par 14 ako li zloću iz ruku odbaciš i u šatoru svom ne daš zlu stana, 
\par 15 čisto ćeš čelo moći tad podići, čvrst ćeš biti i bojati se nećeš. 
\par 16 Svojih se kušnja nećeš sjećat' više kao ni vode koja je protekla. 
\par 17 Jasnije će tvoj život sjat' no podne, tmina će se obratit' u svanuće. 
\par 18 U uzdanju svom živjet ćeš sigurno i zaštićen počivat ćeš u miru. 
\par 19 Kad legneš, nitko te buniti neće; mnogi će tvoju tražiti naklonost. 
\par 20 A zlikovcima ugasnut će oči, neće im više biti utočišta: izdahnut', bit će jedina im nada." 


\chapter{12}

\par 1 Job progovori i reče: 
\par 2 "Uistinu, vi ste cvijet naroda, sa vama će izumrijeti mudrost. 
\par 3 Al' i ja znam k'o i vi misliti, ni u čemu od vas gori nisam: tko za stvari takve ne bi znao? 
\par 4 Prijateljima sam svojim ja na podsmijeh što zazivam Boga da mi odgovori! Na podsmijeh ja sam - pravednik neporočan! 
\par 5 Prezirat' je nesretnika - sretni misle, udariti treba onog što posrće! 
\par 6 Dotle su na miru šatori pljačkaša, izazivači Boga žive bezbrižno kao da Boga u šaci svojoj drže! 
\par 7 Ali pitaj zvijeri, i poučit će te; ptice nebeske pitaj, i razjasnit će ti. 
\par 8 Gušteri zemlje to će ti protumačit', ribe u moru ispripovjedit će ti. 
\par 9 Od stvorenja sviju, koje ne bi znalo da je sve to Božja ruka učinila?! 
\par 10 U ruci mu leži život svakog bića i dah životvorni svakog ljudskog tijela. 
\par 11 Zar uhom mi ne sudimo besjedu k'o što kušamo nepcem okus jela? 
\par 12 Sjedine mudrost donose čovjeku, a s vijekom dugim umnost mu dolazi. 
\par 13 Ali u Njemu mudrost je i snaga, u Njemu savjet je i sva razumnost. 
\par 14 Što razgradi, sagradit neće nitko, kog zatvori, nitko ne oslobađa. 
\par 15 Ustavi li vodu, suša nastaje; pusti li je, svu zemlju ispremetne. 
\par 16 Jer u njemu je snaga i sva mudrost, njegov je prevareni i varalica. 
\par 17 On savjetnike lišava razbora, suce pametne udara bezumljem. 
\par 18 On otpasuje pojas kraljevima i užetom im vezuje bokove. 
\par 19 On bosonoge tjera svećenike i mogućnike sa vlasti obara. 
\par 20 On diže riječ iz usta rječitima i starcima pravo rasuđivanje. 
\par 21 On sasiplje prezir po plemićima i junacima bedra raspasuje. 
\par 22 On dubinama razotkriva tmine i sjenu smrtnu na svjetlo izvodi. 
\par 23 On diže narod pa ga uništava, umnoži ga a potom iskorijeni. 
\par 24 On zaluđuje vladare naroda te po bespuću lutaju pustinjskom 
\par 25 i pipaju u tmini bez svjetlosti glavinjajući poput pijanaca. 


\chapter{13}

\par 1 Očima svojim sve to ja vidjeh, ušima svojim čuh i razumjeh. 
\par 2 Sve što vi znate znadem to i ja, ni u čemu od vas gori nisam. 
\par 3 Zato, zborit' moram sa Svesilnim, pred Bogom svoj razlog izložiti. 
\par 4 Jer, kovači laži vi ste pravi, i svi ste vi zaludni liječnici! 
\par 5 Kada biste bar znali šutjeti, mudrost biste svoju pokazali! 
\par 6 Dokaze mi ipak poslušajte, razlog mojih usana počujte. 
\par 7 Zar zbog Boga govorite laži, zar zbog njega riječi te prijevarne? 
\par 8 Zar biste pristrano branit' htjeli Boga, zar biste mu htjeli biti odvjetnici? 
\par 9 Zar bi dobro bilo da vas on ispita? Zar biste ga obmanuli k'o čovjeka? 
\par 10 Kaznom preteškom on bi vas pokarao poradi potajne vaše pristranosti. 
\par 11 Zar vas veličanstvo njegovo ne plaši i zar vas od njega užas ne spopada? 
\par 12 Razlozi su vam od pepela izreke, obrana je vaša obrana od blata. 
\par 13 Umuknite sada! Dajte da govorim, pa neka me poslije snađe što mu drago. 
\par 14 Zar da meso svoje sam kidam zubima? Da svojom rukom život upropašćujem? 
\par 15 On me ubit' može: nade druge nemam već da pred njim svoje držanje opravdam. 
\par 16 I to je već zalog mojega spasenja, jer bezbožnik preda nj ne može stupiti. 
\par 17 Pažljivo mi riječi poslušajte, nek' vam prodre u uši besjeda. 
\par 18 Gle: ja sam pripremio parnicu, jer u svoje sam pravo uvjeren. 
\par 19 Tko se sa mnom hoće parničiti? - Umuknut ću potom te izdahnut'. 
\par 20 Dvije mi molbe samo ne uskrati da se od tvog lica ne sakrivam: 
\par 21 digni s mene tešku svoju ruku i užasom svojim ne straši me. 
\par 22 Tada me pitaj, a ja ću odgovarat'; ili ja da pitam, ti da odgovaraš. 
\par 23 Koliko počinih prijestupa i grijeha? Prekršaj mi moj pokaži i krivicu. 
\par 24 Zašto lice svoje kriješ sad od mene, zašto u meni vidiš neprijatelja? 
\par 25 Zašto strahom mučiš list vjetrom progonjen, zašto se na suhu obaraš slamčicu? 
\par 26 O ti, koji mi gorke pišeš presude i teretiš mene grijesima mladosti, 
\par 27 koji si mi noge u klade sapeo i koji bdiš nad svakim mojim korakom i tragove stopa mojih ispituješ! 
\par 28 Život mi se k'o trulo drvo raspada, k'o haljina što je moljci izjedaju! 


\chapter{14}

\par 1 Čovjek koga je žena rodila kratka je vijeka i pun nevolja. 
\par 2 K'o cvijet je nikao i vene već, poput sjene bježi ne zastajuć'. 
\par 3 Na takva, zar, ti oči otvaraš i preda se na sud ga izvodiš? 
\par 4 Tko će čisto izvuć' iz nečista? Nitko! 
\par 5 Pa kad su njegovi dani odbrojeni, kad mu broj mjeseci o tebi ovisi, kad mu granicu stavljaš neprijelaznu, 
\par 6 skini s njega pogled da počinut' može, poput najamnika da svoj dan uživa. 
\par 7 TÓa ni drvu nije nada sva propala, posječeno, ono opet prozeleni i mladice nove iz njega izbiju. 
\par 8 Ako mu korijen i ostari u zemlji, ako mu se panj i sasuši u prahu, 
\par 9 oćutjevši vodu, ono će propupat' i pustiti grane kao stablo novo. 
\par 10 Al' kad čovjek umre, ostaje pokošen, kad smrtnik izdahne, gdje li je on tada? 
\par 11 Može sva voda iz mora ispariti i presahnut' rijeke, isušit posvema', 
\par 12 al' čovjek kad legne, ne ustaje više, dok nebesa bude, neće se podići, od sna se svojega probuditi neće. 
\par 13 O, kad bi me htio skriti u Šeolu, zakloniti me dok srdžba ti ne mine, dÓati mi rok kad ćeš me se spomenuti, 
\par 14 - jer, kad umre čovjek, zar uskrsnut' može? - čekao bih te sve dane vojske svoje dok ne bi došao da mi smjenu dadeš. 
\par 15 Zvao bi me, a ja bih se odazvao: zaželio si se djela svojih ruku. 
\par 16 A sad nad svakim mojim vrebaš korakom, nijednog mi grijeha nećeš oprostiti, 
\par 17 u vreći si prijestup moj zapečatio i krivicu moju svu si zapisao. 
\par 18 Vaj! K'o što se jednom uruši planina, k'o što se hridina s mjesta svog odvali, 
\par 19 k'o što voda kamen s vremenom istroši, a pljusak bujicom zemlju svu sapere, tako uništavaš nadu u čovjeku. 
\par 20 Oborio si ga - on ode za svagda, nagrđena lica, otjeran, odbačen. 
\par 21 Djecu mu poštuju - o tom ništa ne zna; ako su prezrena - o tom ne razmišlja. 
\par 22 On jedino pati zbog svojega tijela, on jedino tuži zbog svojeg života." 


\chapter{15}

\par 1 Elifaz Temanac progovori tad i reče: 
\par 2 "Zar šupljom naukom odgovara mudrac i vjetrom istočnim trbuh napuhuje? 
\par 3 Zar on sebe brani riječima ispraznim, besjedama koje ničem ne koriste? 
\par 4 Još više ti činiš: ništiš strah od Boga, pred njegovim licem pribranost ukidaš. 
\par 5 Tvoje riječi krivicu tvoju odaju, poslužio si se jezikom lukavih, 
\par 6 vlastita te usta osuđuju, ne ja, protiv tebe same ti usne svjedoče. 
\par 7 Zar si prvi čovjek koji se rodio? Zar si na svijet prije bregova došao? 
\par 8 Zar si tajne Božje ti prisluškivao i mudrost čitavu za se prisvojio? 
\par 9 Što ti znadeš, a da i mi ne znamo, što ti razumiješ, a da to ne shvaćamo? 
\par 10 Ima među nama i sijedih i starih kojima je više ljeta no tvom ocu. 
\par 11 Zar su ti utjehe Božje premalene i blage riječi upućene tebi? 
\par 12 Što te srce tvoje tako slijepo goni i što tako divlje prevrćeš očima 
\par 13 kad proti Bogu jarost svoju okrećeš, a iz usta takve riječi ti izlaze! 
\par 14 Što je čovjek da bi čist mogao biti? Zar je itko rođen od žene pravedan? 
\par 15 Gle, ni u svece se On ne pouzdava, oku njegovu ni nebesa čista nisu, 
\par 16 a kamoli to biće gadno i buntovno, čovjek što k'o vodu pije opačinu! 
\par 17 Mene sad poslušaj, poučit' te hoću, što god sam vidjeh, ispričat' ti želim, 
\par 18 i ono što naučavahu mudraci ne tajeć' što su primili od pređa 
\par 19 kojima je zemlja ova bila dana kamo tuđin nije nikada stupio. 
\par 20 Zlikovac se muči cijelog svoga vijeka, nasilniku već su ljeta odbrojena. 
\par 21 Krik strave svagda mu u ušima ječi, dok miruje, na njeg baca se razbojnik. 
\par 22 Ne nada se da će izbjeći tminama i znade dobro da je maču namijenjen, 
\par 23 strvinaru da je kao plijen obećan. On znade da mu se dan propasti bliži. 
\par 24 Nemir i tjeskoba na njeg navaljuju, k'o kralj spreman na boj na nj se obaraju. 
\par 25 On je protiv Boga podizao ruku, usuđivao se prkosit' Svesilnom 
\par 26 Ohola je čela na njega srljao, iza štita debela dobro zaklonjen. 
\par 27 Lice mu bijaše obloženo salom a bokovi pretilinom otežali. 
\par 28 Razrušene je zaposjeo gradove i kućišta nastanio napuštena. Srušit će se ono što za sebe sazda; 
\par 29 cvasti mu neće, već rasuti se blago, sjena mu se neće po zemlji širiti. 
\par 30 On se tami više izmaknuti neće, opržit će oganj njegove mladice, u dahu plamenih usta nestat će ga. 
\par 31 U taštinu svoju neka se ne uzda, jer će mu ispraznost biti svom nagradom. 
\par 32 Prije vremena će svenut' mu mladice, grane mu se nikad neće zazelenjet'. 
\par 33 Kao loza, grozd će stresat' svoj nezreo, poput masline pobacit će cvatove. 
\par 34 Da, bezbožničko je jalovo koljeno, i vatra proždire šator podmitljivca. 
\par 35 Koji zlom zanesu, rađaju nesreću i prijevaru nose u utrobi svojoj." 


\chapter{16}

\par 1 Job progovori i reče: 
\par 2 "Koliko se takvih naslušah besjeda, kako ste mi svi vi mučni tješioci! 
\par 3 Ima li kraja tim riječima ispraznim? Što te goni da mi tako odgovaraš? 
\par 4 I ja bih mogao k'o vi govoriti da vam je duša na mjestu duše moje; i ja bih vas mog'o zasuti riječima i nad sudbom vašom tako kimat' glavom; 
\par 5 i ja bih mogao ustima vas hrabrit', i ne bih žalio trud svojih usana. 
\par 6 Al' ako govorim, patnja se ne blaži, ako li zašutim, zar će me minuti? 
\par 7 Zlopakost me sada shrvala posvema, čitava se rulja oborila na me. 
\par 8 Ustao je proti meni da svjedoči i u lice mi se baca klevetama. 
\par 9 Jarošću me svojom razdire i goni, škrgućuć' zubima obara se na me. Moji protivnici sijeku me očima, 
\par 10 prijeteći, na mene usta razvaljuju, po obrazima me sramotno ćuškaju, u čoporu svi tad navaljuju na me. 
\par 11 Da, zloćudnicima Bog me predao, u ruke opakih on me izručio. 
\par 12 Mirno življah dok On ne zadrma mnome, za šiju me ščepa da bi me slomio. 
\par 13 Uze me za biljeg i strijelama osu, nemilosrdno mi bubrege probode i mojom žuči zemlju žednu natopi. 
\par 14 Na tijelu mi ranu do rane otvara, kao bijesan ratnik nasrće na mene. 
\par 15 Tijelo sam golo u kostrijet zašio, zario sam čelo svoje u prašinu. 
\par 16 Zapalilo mi se sve lice od suza, sjena tamna preko vjeđa mi je pala. 
\par 17 A nema nasilja na rukama mojim, molitva je moja bila uvijek čista. 
\par 18 O zemljo, krvi moje nemoj sakriti i kriku mom ne daj nigdje da počine. 
\par 19 Odsad na nebu imam ja svjedoka, u visini gore moj stoji branitelj. 
\par 20 Moja vika moj je odvjetnik kod Boga dok se ispred njega suze moje liju: 
\par 21 o, da me obrani u parbi mojoj s Bogom ko što smrtnik brani svojega bližnjega. 
\par 22 No životu mom su odbrojena ljeta, na put bez povratka meni je krenuti. 


\chapter{17}

\par 1 Daha mi nestaje, gasnu moji dani  i za mene već se skupljaju grobari. 
\par 2 Rugači su evo mene dohvatili, od uvreda oka sklopiti ne mogu. 
\par 3 Stoga me zaštiti i budi mi jamcem kad mi nitko u dlan neće da udari. 
\par 4 Jer, srca si njina lišio razuma i dopustiti im nećeš da opstanu. 
\par 5 K'o taj što imanje dijeli drugovima, a djeci njegovoj dotle oči gasnu, 
\par 6 narodima svim sam na ruglo postao, onaj kom u lice svatko pljunut' može. 
\par 7 Od tuge vid mi se muti u očima, poput sjene moji udovi postaju. 
\par 8 Začudit će se zbog toga pravednici, na bezbožnika će planuti čestiti; 
\par 9 neporočni će na svom ustrajat' putu, čovjek čistih ruku ojačat će još više. 
\par 10 Hajde, svi vi, nećete li opet počet', tÓa među vama ja mudra ne nalazim! 
\par 11 Minuli su dani, propale zamisli, želje srca moga izjalovile se. 
\par 12 'U noći najcrnjoj, dan se približava; blizu je već svjetlo što tminu izgoni.' 
\par 13 A meni je nada otići u Šeol i prostrijeti sebi ležaj u mrklini. 
\par 14 Dovikujem grobu: 'Oče moj rođeni!' a crve pozdravljam: 'Mati moja, sestro!' 
\par 15 Ali gdje za mene ima jošte nade? Sreću moju tko će ikada vidjeti? 
\par 16 Hoće li u Šeol ona sa mnom sići da u prahu zajedno otpočinemo?" 


\chapter{18}

\par 1 Bildad iz Šuaha progovori tad i reče: 
\par 2 "Kada kaniš obuzdat' svoje besjede? Opameti se sad da razgovaramo! 
\par 3 Zašto nas držiš za stoku nerazumnu, zar smo životinje u tvojim očima? 
\par 4 O ti, koji se od jarosti razdireš, hoćeš li da zemlja zbog tebe opusti da iz svoga mjesta iskoče pećine? 
\par 5 Al' ugasit će se svjetlost opakoga, i neće mu sjati plamen na ognjištu. 
\par 6 Potamnjet će svjetlo u njegovu šatoru i nad njime će se utrnut' svjetiljka. 
\par 7 Krepki mu koraci postaju sputani, o vlastite on se spotiče namjere. 
\par 8 Jer njegove noge vode ga u zamku, i evo ga gdje već korača po mreži. 
\par 9 Tanka mu je zamka nogu uhvatila, i evo, užeta čvrsto ga pritežu. 
\par 10 Njega vreba omča skrivena na zemlji, njega čeka klopka putem kojim hodi. 
\par 11 Odasvuda strahovi ga prepadaju, ustopice sveudilj ga proganjaju. 
\par 12 Glad je požderala svu snagu njegovu, nesreća je uvijek o njegovu boku. 
\par 13 Boleština kobna kožu mu razjeda, prvenac mu smrti nagriza udove. 
\par 14 Njega izvlače iz šatora njegova da bi ga odveli vladaru strahota. 
\par 15 U njegovu stanu tuđinac stanuje, po njegovu domu prosipaju sumpor. 
\par 16 Odozdo se suši njegovo korijenje, a odozgo grane sve mu redom sahnu. 
\par 17 Spomen će se njegov zatrti na zemlji, njegovo se ime s lica zemlje briše. 
\par 18 Iz svjetlosti njega u tminu tjeraju, izagnat' ga hoće iz kruga zemaljskog. 
\par 19 U rodu mu nema roda ni poroda, nit' preživjela na njegovu ognjištu. 
\par 20 Sudba je njegova Zapad osupnula, i čitav je Istok obuzela strepnja. 
\par 21 Evo, takav usud snalazi zlikovca i dom onog koji ne priznaje Boga." 


\chapter{19}

\par 1 Job progovori i reče: 
\par 2 "TÓa dokle ćete mučit' dušu moju, dokle ćete me riječima satirat'? 
\par 3 Već deseti put pogrdiste mene i stid vas nije što me zlostavljate. 
\par 4 Pa ako sam zastranio doista, na meni moja zabluda ostaje. 
\par 5 Mislite li da ste me nadjačali i krivnju moju da ste dokazali? 
\par 6 Znajte: Bog je to mene pritisnuo i svojom me je on stegnuo mrežom. 
\par 7 Vičem: 'Nasilje!' - nema odgovora; vapijem - ali za me pravde nema. 
\par 8 Sa svih strana put mi je zagradio, sve staze moje u tminu zavio. 
\par 9 Slavu je moju sa mene skinuo, sa moje glave strgnuo je krunu. 
\par 10 Podsijeca me odasvud te nestajem; k'o drvo, nadu mi je iščupao. 
\par 11 Raspalio se gnjev njegov na mene i svojim me drži neprijateljem. 
\par 12 U bojnom redu pristižu mu čete, putove proti meni nasipaju, odasvud moj opkoljavaju šator. 
\par 13 Od mene su se udaljila braća, otuđili se moji poznanici. 
\par 14 Nestade bližnjih mojih i znanaca, gosti doma mog zaboraviše me. 
\par 15 Sluškinjama sam svojim kao stranac, neznanac sam u njihovim očima. 
\par 16 Slugu zovnem, a on ne odgovara i za milost ga moram zaklinjati. 
\par 17 Mojoj je ženi dah moj omrznuo, gadim se djeci vlastite utrobe. 
\par 18 I deranima na prezir tek služim, ako se dignem, rugaju se meni. 
\par 19 Pouzdanicima sam svojim mrzak, protiv mene su oni koje ljubljah. 
\par 20 Kosti mi se za kožu prilijepiše, osta mi jedva koža oko zuba. 
\par 21 Smilujte mi se, prijatelji moji, jer Božja me je ruka udarila. 
\par 22 Zašto da me k'o Bog sam progonite, zar se niste moga nasitili mesa? 
\par 23 O, kad bi se riječi moje zapisale i kad bi se u mjed tvrdu urezale; 
\par 24 kad bi se željeznim dlijetom i olovom u spomen vječan u stijenu uklesale! 
\par 25 Ja znadem dobro: moj Izbavitelj živi i posljednji će on nad zemljom ustati. 
\par 26 A kad se probudim, k sebi će me dići: iz svoje ću puti tad vidjeti Boga. 
\par 27 Njega ja ću kao svojega gledati, i očima mojim neće biti stranac: za njime srce mi čezne u grudima. 
\par 28 Kad kažete: 'Kako ćemo ga goniti? Koji ćemo razlog protiv njega naći?', 
\par 29 mača tad se bojte: grijehu mač je kazna. Saznat ćete tada da imade suda!" 


\chapter{20}

\par 1 Sofar iz Naamata progovori tad i reče: 
\par 2 "Misli me tjeraju da ti odgovorim, i zato u meni vri to uzbuđenje 
\par 3 dok slušam ukore koji me sramote, al' odgovor mudar um će moj već naći. 
\par 4 Zar tebi nije od davnine poznato, otkad je čovjek na zemlju stavljen bio, 
\par 5 da je kratka vijeka radost opakoga, da kao tren prođe sreća bezbožnička. 
\par 6 Pa ako stasom i do neba naraste, ako mu se glava dotakne oblaka, 
\par 7 poput utvare on zauvijek nestaje; koji ga vidješe kažu: 'Gdje je sad on?' 
\par 8 Kao san bez traga on se rasplinjuje, nestaje ga kao priviđenja noćnog. 
\par 9 Nijedno ga oko više gledat neće, niti će ga mjesto njegovo vidjeti 
\par 10 Njegovu će djecu gonit' siromasi: rukama će svojim vraćati oteto. 
\par 11 Kosti su njegove bujale mladošću; gle, zajedno s njome pokošen je sada. 
\par 12 Zlo bijaše slatko njegovim ustima te ga je pod svojim jezikom skrivao; 
\par 13 sladio se pazeć' da ga ne proguta i pod nepcem svojim zadržavao ga. 
\par 14 Ali hrana ta mu trune u utrobi, otrovom zmijskim u crijevima postaje. 
\par 15 Blago progutano mora izbljuvati. Bog će ga istjerat' njemu iz utrobe. 
\par 16 Iz zmijine glave otrov je sisao: sada umire od jezika gujina. 
\par 17 Potoke ulja on gledat' više neće, ni vidjet' gdje rijekom med i mlijeko teku. 
\par 18 Vratit će dobitak ne okusivši ga, neće uživat' u plodu trgovine. 
\par 19 Jer je sirotinju gnjeo i tlačio, otimao kuće koje ne sazida, 
\par 20 jer ne bješe kraja požudi njegovoj, njegova ga blaga neće izbaviti. 
\par 21 Jer mu proždrljivost ništa ne poštedi, ni sreća njegova dugo trajat neće. 
\par 22 Sred izobilja u škripcu će se naći, svom će snagom na nj se oboriti bijeda. 
\par 23 I dok hranom bude trbuh svoj punio, Bog će na nj pustiti jarost svoga gnjeva, sasut' dažd strelica na meso njegovo. 
\par 24 Ako i izmakne gvozdenom oružju, luk će mjedeni njega prostrijeliti. 
\par 25 Strijelu bi izvuk'o, al' mu probi leđa, a šiljak blistavi viri mu iz žuči. Kamo god krenuo, strepnje ga vrebaju, 
\par 26 na njega tmine sve tajom očekuju. Vatra ga ništi, ni od kog zapaljena, i proždire sve pod njegovim šatorom. 
\par 27 Gle, nebo krivicu njegovu otkriva i čitava zemlja na njega se diže. 
\par 28 Njegovu će kuću raznijeti poplava, otplaviti je u dan Božje jarosti. 
\par 29 Takvu sudbinu Bog priprema zlikovcu i takvu baštinu on mu dosuđuje." 


\chapter{21}

\par 1 Job progovori i reče: 
\par 2 "Slušajte, slušajte dobro što ću reći, utjehu mi takvu barem udijelite. 
\par 3 Otrpite da riječ jednu ja izrečem, kad završim, tad se rugajte slobodno. 
\par 4 Zar protiv čovjeka dižem ja optužbu? Kako da strpljenje onda ne izgubim? 
\par 5 Pogledajte na me: užas će vas spopast', rukom ćete svoja zakloniti usta; 
\par 6 pomislim li na to, prestravim se i sam i čitavim svojim tad protrnem tijelom. 
\par 7 Zašto na životu ostaju zlikovci i, što su stariji, moćniji bivaju? 
\par 8 Potomstvo njihovo s njima napreduje a izdanci im se množe pred očima. 
\par 9 Strah nikakav kuće njihove ne mori i šiba ih Božja ostavlja na miru. 
\par 10 Njihovi bikovi plode pouzdano, krave im se tele i ne jalove se. 
\par 11 K'o jagnjad djeca im slobodno skakuću, veselo igraju njihovi sinovi. 
\par 12 Oni pjevaju uz harfe i bubnjeve i vesele se uz zvukove svirale. 
\par 13 Dane svoje završavaju u sreći, u Podzemlje oni silaze spokojno. 
\par 14 A govorili su Bogu: 'Ostavi nas, ne želimo znati za tvoje putove! 
\par 15 TÓa tko je Svesilni da njemu služimo i kakva nam korist da ga zazivamo?' 
\par 16 Zar svoju sreću u ruci ne imahu, makar do Njega ne drže ništa oni? 
\par 17 Zar se luč opakog kada ugasila? Zar se na njega oborila nesreća? Zar mu u gnjevu svom On skroji sudbinu? 
\par 18 Zar je kao slama na vjetru postao, kao pljeva koju vihor svud raznosi? 
\par 19 Hoće l' ga kaznit' Bog u njegovoj djeci? Ne, njega nek' kazni da sam to osjeti! 
\par 20 Vlastitim očima nek' rasap svoj vidi, neka se napije srdžbe Svesilnoga! 
\par 21 TÓa što poslije smrti on za dom svoj mari kad će se presjeć' niz njegovih mjeseci? 
\par 22 Ali tko će Boga učiti mudrosti, njega koji sudi najvišim bićima? 
\par 23 Jedan umire u punom blagostanju, bez briga ikakvih, u potpunom miru, 
\par 24 bokova od pretiline otežalih i kostiju sočne moždine prepunih. 
\par 25 A drugi umire s gorčinom u duši, nikad nikakve ne okusivši sreće. 
\par 26 Obojica leže zajedno u prahu, crvi ih jednako prekrivaju oba. 
\par 27 O, znam dobro kakve vaše su namjere, kakve zlosti protiv mene vi snujete. 
\par 28 Jer pitate: 'Gdje je kuća plemićeva, šator u kojem stanovahu opaki?' 
\par 29 Niste li na cesti putnike pitali, zar njihovo svjedočanstvo ne primate: 
\par 30 'Opaki je u dan nesreće pošteđen i u dan Božje jarosti veseo je.' 
\par 31 Al' na postupcima tko će mu predbacit' i tko će mu vratit' što je počinio? 
\par 32 A kad ga na kraju na groblje odnesu, na grobni mu humak postavljaju stražu. 
\par 33 Lake su mu grude zemlje u dolini dok za njime ide čitavo pučanstvo. 
\par 34 O, kako su vaše utjehe isprazne! Kakva su prijevara vaši odgovori!" 


\chapter{22}

\par 1 Elifaz Temanac progovori tad i reče: 
\par 2 "Zar Bogu koristan može biti čovjek? TÓa tko je mudar, sebi samom koristi. 
\par 3 Zar je Svesilnom milost što si pravedan i zar mu je dobit što si neporočan? 
\par 4 Ili te zbog tvoje pobožnosti kara i zato se hoće s tobom parničiti? 
\par 5 Nije l' to zbog zloće tvoje prevelike i zbog bezakonja kojim broja nema? 
\par 6 Od braće si brao nizašto zaloge i s golih si ljudi svlačio haljine; 
\par 7 ti nisi žednoga vodom napojio, uskraćivao si kruh izgladnjelima; 
\par 8 otimao si od siromaha zemlju da bi na njoj svog nastanio ljubimca; 
\par 9 puštao si praznih ruku udovice i siročadi si satirao ruku. 
\par 10 Eto zašto tebe mreže sad sapinju, zašto te strahovi muče iznenadni. 
\par 11 Svjetlost ti mrak posta i ništa ne vidiš, vode su duboke tebe potopile. 
\par 12 Zar Bog nije u visini nebeskoj i zar zvijezdama tjeme on ne vidi? 
\par 13 Ali ti kažeš: 'Što Bog može znati? Kroz oblak tmasti zar što razabire? 
\par 14 Oblaci pogled njegov zaklanjaju, i rubom kruga on hoda nebeskog.' 
\par 15 TÓa kaniš li se drevnog držat' puta kojim su išli ljudi nepravedni? 
\par 16 Prije vremena nestadoše oni, bujica im je temelje raznijela. 
\par 17 Zborahu Bogu: 'Nas se ti ostavi! Što nam Svesilni učiniti može?' 
\par 18 A on im je dom punio dobrima makar do njega ne držahu ništa. 
\par 19 Videć' im propast, klikću pravednici, neporočni se njima izruguju: 
\par 20 'Gle, propadoše protivnici naši, što od njih osta, vatra im proždrije!" 
\par 21 S Bogom ti se sprijatelji i pomiri, i vraćena će ti opet biti sreća. 
\par 22 Ded prihvati Zakon iz njegovih usta, u srce svoje riječ njegovu usadi. 
\par 23 Ako se raskajan vratiš Svesilnome i nepravdu iz svog šatora odstraniš, 
\par 24 tad ćeš odbaciti zlato u prašinu i ofirsko blago u šljunak potočni. 
\par 25 Svesilni će postat' tvoje suho zlato, on će biti tvoje gomile srebrene. 
\par 26 Da, Svesilni bit će tvoje radovanje, i lice ćeš k Bogu dizati slobodno. 
\par 27 Molit ćeš mu se, i uslišat će tebe, ispunit ćeš što si mu zavjetovao. 
\par 28 Što god poduzeo, sve će ti uspjeti, i putove će ti obasjavat' svjetlost. 
\par 29 Jer, on ponizuje ponos oholima, dok u pomoć smjernim očima pritječe. 
\par 30 Iz nevolje on izbavlja nevinoga; i tebe će spasit' tvoje čiste ruke." 


\chapter{23}

\par 1 Job progovori i reče: 
\par 2 "Zar mi je i danas tužaljka buntovna? Teška mu ruka iz mene vapaj budi: 
\par 3 o, kada bih znao kako ću ga naći, do njegova kako doprijeti prijestolja, 
\par 4 pred njim parnicu bih svoju razložio, iz mojih bi usta navrli dokazi. 
\par 5 Rad bih znati što bi meni odvratio i razumjeti riječ što bi je rekao! 
\par 6 Zar mu treba snage velike za raspru? Ne, dosta bi bilo da me on sasluša. 
\par 7 U protivniku bi vidio pravedna, i parnica moja tad bi pobijedila. 
\par 8 Na istok krenem li, naći ga ne mogu; pođem li na zapad, ne razabirem ga. 
\par 9 Ištem na sjeveru, al' ga ne opažam; nevidljiv je ako se k jugu okrenem. 
\par 10 Pa ipak, on dobro zna put kojim kročim! Neka me kuša: čist k'o zlato ću izići! 
\par 11 Noga mi se stopa njegovih držala, putem sam njegovim išao ne skrećuć'; 
\par 12 slušao sam nalog njegovih usana, pohranih mu riječi u grudima svojim. 
\par 13 Al' htjedne li štogod, tko će ga odvratit'? Što zaželi dušom, to će ispuniti. 
\par 14 Izvršit će što je dosudio meni, kao i sve drugo što je odlučio! 
\par 15 Zbog toga pred njime sav ustravljen ja sam, i što više mislim, jače strah me hvata. 
\par 16 U komade Bog mi je srce smrvio, užasom me svega prožeo Svesilni, 
\par 17 premda nisam ni u tminama propao, ni u mraku što je lice moje zastro. 


\chapter{24}

\par 1 Zašto Svesilni ne promatra vremena,  a dane njegove ne vide mu vjernici? 
\par 2 Bezbožnici pomiču granice, otimaju stado i pasu ga. 
\par 3 Sirotama odvode magarca, udovi u zalog vola dižu. 
\par 4 Siromahe tjeraju sa puta; skrivaju se ubogari zemlje. 
\par 5 K'o magarci divlji u pustinji zarana idu da plijen ugrabe: pustinja im hrani mališane. 
\par 6 Po tuđem polju oni pabirče, paljetkuju vinograd opakog. 
\par 7 Goli noće, nemaju haljine, ni pokrivača protiv studeni. 
\par 8 Oni kisnu na planinskom pljusku; bez skloništa uz hrid se zbijaju. 
\par 9 Otkidaju od sise sirotu, ubogom u zalog dijete grabe. 
\par 10 Goli hode, nemaju haljina; izgladnjeli, tuđe snoplje nose. 
\par 11 Oni mlina za ulje nemaju; ožednjeli, gaze u kacama. 
\par 12 Samrtnici hropću iz gradova, ranjenici u pomoć zazivlju. Al' na sve to Bog se oglušuje. 
\par 13 Ima onih koji mrze svjetlost: ne priznaju njezinih putova niti se staza drže njezinih. 
\par 14 Za mraka se diže ubojica, kolje ubogog i siromaha. U gluhoj se noći lopov skiće [16a] i u tmini provaljuje kuće. 
\par 15 Sumrak žudi oko preljubnika: 'Nitko me vidjet neće', kaže on i zastire velom svoje lice. 
\par 16 [16b]Za vidjela oni se skrivaju, oni neće da za svjetlost znaju. 
\par 17 Zora im je kao sjena smrtna: kad zarudi, silan strah ih hvata. 
\par 18 Prije nego svane, on već hitro bježi kloneći se puta preko vinograda. Njegova su dobra prokleta u zemlji. 
\par 19 K'o što vrućina i žega snijeg upija, tako i Podzemlje proždire grešnike. 
\par 20 Zaboravilo ga krilo što ga rodi, ime se njegovo više ne spominje: poput stabla zgromljena je opačina. 
\par 21 Ženu nerotkinju on je zlostavljao, udovici nije učinio dobra. 
\par 22 Al' Onaj što snažno hvata nasilnike, ustaje, a njima sva se nada gasi. 
\par 23 Dade mu sigurnost, i on se pouzda; okom je njegove nadzirao staze. 
\par 24 Dignu se za kratko, a onda nestanu, ruše se i kao svi drugi istrunu, posječeni kao glave klasovima." 
\par 25 Nije li tako? Tko će me u laž utjerat'? Tko moje riječi poništiti može?" 


\chapter{25}

\par 1 Bildad iz Šuaha progovori tad i reče: 
\par 2 "Gospodstvo i strah u njegovoj su ruci i on stvara mir u svojim visinama. 
\par 3 Zar se njemu čete izbrojiti mogu i svjetlo njegovo nad kim ne izlazi? 
\par 4 Pa kako da čovjek prav bude pred Bogom i od žene rođen kako da čist bude? 
\par 5 Eto, i mjesec pred njime sjaj svoj gubi, njegovim očima zvijezde nisu čiste. 
\par 6 Što reći onda o čovjeku, tom crvu, o sinu čovjekovu, crviću jadnom? 


\chapter{26}

\par 1 Job progovori i reče: 
\par 2 "Kako dobro znadeš pomoći nemoćnom i mišicu iznemoglu poduprijeti! 
\par 3 Kako dobar savjet daješ neukome; baš si preveliku mudrost pokazao. 
\par 4 Kome li si ove uputio riječi i koji duh je iz tebe govorio?" 
\par 5 Pred Bogom mrtvi pod zemljom dolje strepe, vode morske dršću i nemani njine. 
\par 6 Pred njegovim okom otkriven zja Šeol i bezdan smrti nema vela na sebi. 
\par 7 On povrh praznine Sjever razapinje, on drži zemlju o ništa obješenu. 
\par 8 On zatvara vodu u svoje oblake, a oblaci se pod njome ne prodiru. 
\par 9 On zastire puno lice mjesečevo razastirući svoj oblak preko njega. 
\par 10 On je na vodi označio kružnicu gdje prestaje svjetlost i tmine počinju. 
\par 11 Svodu se nebeskom potresu stupovi i premru od straha kada on zaprijeti. 
\par 12 Svojom je snagom on ukrotio more i neman Rahaba smrvio mudrošću. 
\par 13 Nebesa je svojim razbistrio dahom, a ruka mu je brzu zmiju probola. 
\par 14 Sve to samo djelić je djela njegovih, od kojih tek slabu jeku mi čujemo. Ali tko će shvatit' grom njegove moći?" 


\chapter{27}

\par 1 Job nastavi svoju besjedu i reče: 
\par 2 "Živoga mi Boga što mi pravdu krati i Svesilnog koji dušu mi zagorča: 
\par 3 sve dok duha moga bude još u meni, dok mi dah Božji u nosnicama bude, 
\par 4 usne moje neće izustiti zloću niti će laž kakva doći na moj jezik. 
\par 5 Daleko od mene da vam dadem pravo, nedužnost svoju do zadnjeg daha branim. 
\par 6 Pravde svoje ja se držim, ne puštam je; zbog mojih me dana srce korit' neće. 
\par 7 Neka mi dušmana kob opakog snađe, a mog protivnika udes bezbožnikov! 
\par 8 Čemu se nadati može kad vapije i kada uzdiže k Bogu dušu svoju? 
\par 9 Hoće li čuti Bog njegove krikove kada se na njega obori nevolja? 
\par 10 Zar će se radovat' on u Svesilnome, zar će Boga svakog časa zazivati? 
\par 11 Ali Božju ruku ja ću vam pokazat' i neću vam sakrit namjere Svesilnog. 
\par 12 Eto, sve ste sami mogli to vidjeti, što se onda u ispraznosti gubite?" 
\par 13 "Ovu sudbu Bog dosuđuje opakom, ovo baštini silnik od Svemogućeg. 
\par 14 Ima li sinova mnogo, mač ih čeka, a porod mu neće imat' dosta kruha. 
\par 15 Smrt će sahranit' preživjele njegove i udovice ih oplakivat neće. 
\par 16 Ako i srebra k'o praha nagomila, ako i nakupi haljina k'o blata, 
\par 17 nek' ih skuplja, odjenut će ih pravednik, ljudi će nedužni podijeliti srebro. 
\par 18 Od paučine je kuću sagradio, kolibicu kakvu sebi diže čuvar: 
\par 19 bogat je legao, al' po posljednji put; kad oči otvori, ničeg više nema. 
\par 20 Usred bijela dana strava ga spopada, noću ga oluja zgrabi i odnese. 
\par 21 Istočni ga vjetar digne i odvuče, daleko ga baca od njegova mjesta. 
\par 22 Bez milosti njime vitla on posvuda, dok mu ovaj kuša umaći iz ruke. 
\par 23 Rukama plješću nad njegovom propašću i zvižde na njega kamo god došao. 


\chapter{28}

\par 1 "Da, srebro ima svoja nalazišta, a zlato mjesta gdje se pročišćava. 
\par 2 Ruda željezna iz zemlje se vadi, a iz rudače rastaljene bakar. 
\par 3 Ljudi tami postavljaju granice i kopaju do najvećih dubina za kamenom u mraku zakopanim. 
\par 4 Čeljad iz tuđine rovove dube do kojih ljudska ne dopire noga, visi njišuć' se, daleko od ljudi. 
\par 5 Krilo zemlje iz kojeg kruh nam niče kao od vatre sve je razrovano. 
\par 6 Stijene njene safira su skrovišta, prašina zlatna krije se u njima. 
\par 7 Tih putova ne znaju grabljivice, jastrebovo ih oko ne opaža. 
\par 8 Zvijeri divlje njima nisu kročile niti je kada lav njima prošao. 
\par 9 Ali na kamen diže čovjek ruku te iz korijena prevraća planine. 
\par 10 U kamenu prokopava prolaze, oko mu sve dragocjeno opaža. 
\par 11 Žilama vode on tok zaustavlja; stvari skrivene nosi na vidjelo. 
\par 12 Ali otkuda nam Mudrost dolazi? Na kojemu mjestu Razum prebiva? 
\par 13 Čovjek njezina ne poznaje puta, u zemlji živih nisu je otkrili. 
\par 14 Bezdan govori: 'U meni je nema!' a more: 'Ne nalazi se kod mene!' 
\par 15 Zlatom se čistim kupiti ne može, ni cijenu njenu srebrom odmjeriti; 
\par 16 ne mjeri se ona zlatom ofirskim, ni oniksom skupim pa ni safirom. 
\par 17 Sa zlatom, staklom ne poređuje se, nit' se daje za sud od suha zlata. 
\par 18 Čemu spominjat' prozirac, koralje, bolje je steći Mudrost no biserje. 
\par 19 Što je prema njoj topaz etiopski? Ni čistim zlatom ne procjenjuje se. 
\par 20 Ali otkuda nam Mudrost dolazi? Na kojemu mjestu Razum prebiva? 
\par 21 Sakrivena je očima svih živih; ona izmiče pticama nebeskim. 
\par 22 Propast paklena i Smrt izjavljuju: 'Za slavu njenu mi smo samo čuli.' 
\par 23 Jedino je Bog put njen proniknuo, on jedini znade gdje se nalazi. 
\par 24 Jer pogledom granice zemlje hvata i opaža sve pod svodom nebeskim. 
\par 25 Kad htjede vjetru odredit težinu i mjerilom svu vodu izmjeriti, 
\par 26 kad je zakone daždu nametnuo i oblacima gromovnim putove, 
\par 27 tad ju je vidio te izmjerio, učvrstio i do dna ispitao. 
\par 28 A potom je rekao čovjeku: Strah Gospodnji - eto što je mudrost; 'Zla se kloni' - to ti je razumnost." 


\chapter{29}

\par 1 Job nastavi svoju besjedu i reče: 
\par 2 "O, da mi je prošle proživjet' mjesece, dane one kad je Bog nada mnom bdio, 
\par 3 kad mi je nad glavom njegov sjao žižak a kroz mrak me svjetlo njegovo vodilo, 
\par 4 kao u dane mojih zrelih jeseni kad s mojim stanom Bog prijateljevaše, 
\par 5 kada uz mene još bijaše Svesilni i moji me okruživahu dječaci, 
\par 6 kada mi se noge u mlijeku kupahu, a potokom ulja ključaše mi kamen! 
\par 7 Kada sam na vrata gradska izlazio i svoju stolicu postavljao na trg, 
\par 8 vidjevši me, sklanjali bi se mladići, starci bi ustavši stojeći ostali. 
\par 9 Razgovor bi prekidali uglednici i usta bi svoja rukom zatvarali. 
\par 10 Glavarima glas bi sasvim utihnuo, za nepce bi im se zalijepio jezik. 
\par 11 Tko god me slušao, blaženim me zvao, hvalilo me oko kad bi me vidjelo. 
\par 12 Jer, izbavljah bijednog kada je kukao i sirotu ostavljenu bez pomoći. 
\par 13 Na meni bješe blagoslov izgubljenih, srcu udovice ja veselje vraćah. 
\par 14 Pravdom se ja kao haljinom odjenuh, nepristranost bje mi plaštem i povezom. 
\par 15 Bjeh oči slijepcu i bjeh noge bogalju, 
\par 16 otac ubogima, zastupnik strancima. 
\par 17 Kršio sam zube čovjeku opaku, plijen sam čupao iz njegovih čeljusti. 
\par 18 Govorah: 'U svom ću izdahnuti gnijezdu, k'o palma, bezbrojne proživjevši dane.' 
\par 19 Korijenje se moje sve do vode pruža, na granama mojim odmara se rosa. 
\par 20 Pomlađivat će se svagda slava moja i luk će mi se obnavljati u ruci.' 
\par 21 Slušali su željno što ću im kazati i šutjeli da od mene savjet čuju. 
\par 22 Na riječi mi ne bi ništa dometali i besjede su mi daždile po njima. 
\par 23 Za mnom žudjeli su oni k'o za kišom, otvarali usta k'o za pljuskom ljetnim. 
\par 24 Osmijeh moj bijaše njima ohrabrenje; pazili su na vedrinu moga lica. 
\par 25 Njima ja sam izabirao putove, kao poglavar ja sam ih predvodio, kao kralj među svojim kad je četama kao onaj koji tješi ojađene. 


\chapter{30}

\par 1 "A sada, gle, podruguju se mnome  ljudi po ljetima mlađi od mene kojih oce ne bih bio metnuo ni s ovčarskim psima stada svojega. 
\par 2 Ta što će mi jakost ruku njihovih kad im muževna ponestane snaga ispijena glađu i oskudicom. 
\par 3 Glodali su u pustinji korijenje i čestar opustjelih ruševina. 
\par 4 Lobodu su i s grmlja lišće brali, kao kruh jeli korijenje žukino. 
\par 5 Od društva ljudskog oni su prognani, za njima viču k'o za lopovima. 
\par 6 Živjeli su po strašnim jarugama, po spiljama i u raspuklinama. 
\par 7 Urlik im se iz šikarja dizao; po trnjacima ležahu stisnuti. 
\par 8 Sinovi bezvrijednih, soj bezimenih, bičevima su iz zemlje prognani. 
\par 9 Rugalicom sam postao takvima i njima sada služim kao priča! 
\par 10 Gnušaju me se i bježe od mene, ne ustežu se pljunut' mi u lice. 
\par 11 I jer On luk mi slomi i satrije me, iz usta svojih izbaciše uzdu. 
\par 12 S desne moje strane rulja ustaje, noge moje u bijeg oni tjeraju, put propasti prema meni nasiplju. 
\par 13 Stazu mi ruše da bi me satrli, napadaju i ne brani im nitko, 
\par 14 prolomom oni širokim naviru i kotrljaju se poput oluje. 
\par 15 Strahote sve se okreću na mene, mojeg ugleda kao vjetra nesta, poput oblaka iščeznu spasenje. 
\par 16 Duša se moja rasipa u meni, dani nevolje na me se srušili. 
\par 17 Noću probada bolest kosti moje, ne počivaju boli što me glođu. 
\par 18 Muka mi je i halju nagrdila i stegla me k'o ovratnik odjeće. 
\par 19 U blato me je oborila dolje, gle, postao sam k'o prah i pepeo. 
\par 20 K Tebi vičem, al' Ti ne odgovaraš; pred Tobom stojim, al' Ti i ne mariš. 
\par 21 Prema meni postao si okrutan; rukom preteškom na me se obaraš. 
\par 22 U vihor me dižeš, nosiš me njime, u vrtlogu me olujnom kovitlaš. 
\par 23 Da, znadem da si me smrti predao, saborištu zajedničkom svih živih. 
\par 24 Al' ne pruža li ruku utopljenik, ne viče li kad padne u nevolju? 
\par 25 Ne zaplakah li nad nevoljnicima, ne sažalje mi duša siromaha? 
\par 26 Sreći se nadah, a dođe nesreća; svjetlost čekah, a gle, zavi me tama. 
\par 27 Utroba vri u meni bez prestanka, svaki dan nove patnje mi donosi. 
\par 28 Smrknut idem, al' nitko me ne tješi; ustajem u zboru - da bih kriknuo. 
\par 29 Sa šakalima sam se zbratimio i nojevima postao sam drugom. 
\par 30 Na meni sva je koža pocrnjela, i kosti mi je sažgala ognjica. 
\par 31 Tužaljka mi je ugodila harfu, svirala mi glas narikača ima. 


\chapter{31}

\par 1 Sa svojim očima savez sam sklopio  da pogledat neću nijednu djevicu. 
\par 2 A što mi je Bog odozgo dosudio, kakva mi je baština od Svesilnoga? 
\par 3 TÓa nije li nesreća za opakoga, a nevolja za one koji zlo čine? 
\par 4 Ne proniče li on sve moje putove, ne prebraja li on sve moje korake? 
\par 5 Zar sam ikad u društvu laži hodio, zar mi je noga k prijevari hitjela? 
\par 6 Nek' me na ispravnoj mjeri Bog izmjeri pa će uvidjeti neporočnost moju! 
\par 7 Ako mi je korak s puta kad zašao, ako mi se srce za okom povelo, ako mi je ljaga ruke okaljala, 
\par 8 neka drugi jede što sam posijao, neka sve moje iskorijene izdanke! 
\par 9 Ako mi zavede srce žena neka, ako za vratima svog bližnjeg kad vrebah, 
\par 10 neka moja žena drugom mlin okreće, neka s drugim svoju podijeli postelju! 
\par 11 Djelo bestidno time bih počinio, zločin kojem pravda treba da presudi, 
\par 12 užego vatru što žeže do Propasti i što bi svu moju sažgala ljetinu. 
\par 13 Ako kada prezreh pravo sluge svoga il' služavke, sa mnom kad su se parbili, 
\par 14 što ću učiniti kada Bog ustane? Što ću odvratit' kad račun zatraži? 
\par 15 Zar nas oba on ne stvori u utrobi i jednako sazda u krilu majčinu? 
\par 16 Ogluših li se na molbe siromaha ili rasplakah oči udovičine? 
\par 17 Jesam li kada sam svoj jeo zalogaj a da ga nisam sa sirotom dijelio? 
\par 18 TÓa od mladosti k'o otac sam mu bio, vodio sam ga od krila materina! 
\par 19 Zar sam beskućnika vidio bez odjeće ili siromaha kog bez pokrivača 
\par 20 a da mu bedra ne blagosloviše mene kad se runom mojih ovaca ogrija? 
\par 21 Ako sam ruku na nevina podigao znajuć' da mi je na vratima branitelj, 
\par 22 nek' se rame moje od pleća odvali i neka mi ruka od lakta otpadne! 
\par 23 Jer strahote Božje na mene bi pale, njegovu ne bih odolio veličanstvu. 
\par 24 Zar sam u zlato pouzdanje stavio i rekao zlatu: 'Sigurnosti moja!' 
\par 25 Zar sam se veliku blagu radovao, bogatstvima koja su mi stekle ruke? 
\par 26 Zar se, gledajući sunce kako blista i kako mjesec sjajni nebom putuje, 
\par 27 moje srce dalo potajno zavesti da bih rukom njima poljubac poslao? 
\par 28 Grijeh bi to bio što za sudom vapije, jer Boga višnjega bih se odrekao. 
\par 29 Zar se obradovah nevolji dušmana i likovah kad ga je zlo zadesilo, 
\par 30 ja koji ne dadoh griješiti jeziku, proklinjući ga i želeći da umre? 
\par 31 Ne govorahu li ljudi mog šatora: 'TÓa koga nije on mesom nasitio'? 
\par 32 Nikad nije stranac vani noćivao, putniku sam svoja otvarao vrata. 
\par 33 Zar sam grijehe svoje ljudima tajio, zar sam u grudima skrivao krivicu 
\par 34 jer sam se plašio govorkanja mnoštva i strahovao od prezira plemenskog te sam mučao ne prelazeć' svoga praga? 
\par 35 O, kad bi koga bilo da mene sasluša! Posljednju sam svoju riječ ja izrekao: na Svesilnom je sad da mi odgovori! Nek' mi optužnicu napiše protivnik, 
\par 36 i ja ću je nosit' na svome ramenu, čelo ću njome k'o krunom uresit'. 
\par 37 Dat ću mu račun o svojim koracima i poput kneza pred njega ću stupiti." 
\par 38 Ako je na me zemlja moja vikala, ako su s njom brazde njezine plakale; 
\par 39 ako sam plodove jeo ne plativši i ako sam joj ojadio ratare, 
\par 40 [40a] neka mjesto žita po njoj niče korov, a mjesto ječma nek' posvud kukolj raste! [40b] Konac riječi Jobovih. 


\chapter{32}

\par 1 Ona tri čovjeka prestadoše Jobu odgovarati, jer je on sebe  smatrao nevinim. 
\par 2 Nato se rasrdi Elihu, sin Barakeelov, iz  Buza, od plemena Ramova: planu gnjevom na Joba zato što je sebe  držao pravednim pred Bogom; 
\par 3 a planu gnjevom i na tri njegova  prijatelja jer nisu više našli ništa što bi odgovorili te su  tako Boga osudili. 
\par 4 Dok su oni govorili s Jobom, Elihu je šutio, jer su oni bili stariji od njega. 
\par 5 Ali kad vidje da ona tri  čovjeka nisu više imala odgovora u ustima, planu od srdžbe. 
\par 6 I  progovorivši, Elihu, sin Barakeelov, iz Buza, reče: "Po godinama svojim još mlad sam ja, a u duboku vi ste ušli starost; bojažljivo se zato ja ustezah znanje svoje pokazati pred vama. 
\par 7 Mišljah u sebi: 'Govorit će starost, mnoge godine pokazat će mudrost.' 
\par 8 Uistinu, dah neki u ljudima, duh Svesilnog mudrim čini čovjeka. 
\par 9 Dob poodmakla ne daje mudrosti a niti starost pravednosti uči. 
\par 10 Zato vas molim, poslušajte mene da vam i ja znanje svoje izložim. 
\par 11 S pažnjom sam vaše besjede pratio i razloge sam vaše saslušao dok ste tražili što ćete kazati. 
\par 12 Na vama moja sva bijaše pažnja, al' ne bi nikog da Joba pobije ni da mu od vas tko riječ opovrgne. 
\par 13 Nemojte reći: 'Na mudrost smo naišli! Bog će ga pobit jer čovjek ne može.' 
\par 14 Nije meni on besjedu upravio: odvratit mu neću vašim riječima. 
\par 15 Poraženi, otpovrgnut ne mogu, riječi zapeše u grlu njihovu. 
\par 16 Čekao sam! Al', gle, oni ne zbore. Umukoše, ni riječ više da kažu! 
\par 17 Na meni je da progovorim sada, znanje ću svoje i ja izložiti. 
\par 18 Riječi mnoge u meni naviru dok iznutra moj duh mene nagoni. 
\par 19 Gle, nutrina mi je k'o mošt zatvoren, k'o nova će se raspući mješina. 
\par 20 Da mi odlane, govorit ću stoga, otvorit ću usne i odvratit' vama. 
\par 21 Nijednoj strani priklonit se neću niti laskat ja namjeravam kome. 
\par 22 Laskati ja ne umijem nikako, jer smjesta bi me Tvorac moj smaknuo. 


\chapter{33}

\par 1 Čuj dakle, Jobe, što ću ti kazati, prikloni uho mojim besjedama. 
\par 2 Evo, usta sam svoja otvorio, a jezik riječi pod nepcem mi stvara. 
\par 3 Iskreno će ti zborit' srce moje, usne će čistu izreći istinu. 
\par 4 TÓa i mene je duh Božji stvorio, dah Svesilnoga oživio mene. 
\par 5 Ako uzmogneš, ti me opovrgni; spremi se da se suprotstaviš meni! 
\par 6 Gle, kao i ti, i ja sam pred Bogom, kao i ti, od gline bjeh načinjen; 
\par 7 zato ja strahom tebe motrit' neću, ruka te moja neće pritisnuti. 
\par 8 Dakle, na moje uši rekao si - posve sam jasno tvoje čuo riječi: 
\par 9 'Nedužan sam i bez ikakva grijeha; prav sam i nema krivice na meni. 
\par 10 Al' On izlike protiv mene traži i za svojeg me drži dušmanina. 
\par 11 Noge je moje u klade metnuo, nad svakim mojim on pazi korakom.' 
\par 12 Ovdje, kažem ti, u pravu ti nisi, jer s Bogom čovjek mjerit' se ne može. 
\par 13 Pa zašto s njime zamećeš prepirku što ti na svaku riječ ne odgovara? 
\par 14 Bog zbori nama jednom i dva puta, al' čovjek na to pažnju ne obraća. 
\par 15 U snovima, u viđenjima noćnim, kada san dubok ovlada ljudima i na ležaju dok tvrdo snivaju, 
\par 16 tad on govori na uho čovjeku i utvarama plaši ga jezivim 
\par 17 da ga od djela njegovih odvrati, da u čovjeku obori oholost, 
\par 18 da dušu njegovu spasi od jame i život mu od puta u Podzemlje. 
\par 19 Bolešću on ga kara na ležaju kad mu se kosti tresu bez prestanka, 
\par 20 kad se kruh gadi njegovu životu i ponajbolje jelo duši njegovoj; 
\par 21 kada mu tijelo gine naočigled i vide mu se kosti ogoljele, 
\par 22 kad mu se duša približava jami a život njegov boravištu mrtvih. 
\par 23 Ako se uza nj nađe tad anđeo, posrednik jedan između tisuću, da čovjeka na dužnost opomene, 
\par 24 pa se sažali nad njim i pomoli: 'Izbavi ga da u jamu ne ide; za život njegov nađoh otkupninu! 
\par 25 Neka mu tijelo procvate mladošću, nek' se vrati u dane mladenačke!' 
\par 26 Vapije k Bogu i Bog ga usliša: radosno On ga pogleda u lice; vrati čovjeku pravednost njegovu. 
\par 27 Tada čovjek pred ljudima zapjeva: 'Griješio sam i pravo izvrtao, ali mi Bog zlom nije uzvratio. 
\par 28 On mi je dušu spasio od jame i život mi se veseli svjetlosti.' 
\par 29 Gle, sve to Bog je spreman učiniti do dva i do tri puta za čovjeka: 
\par 30 da dušu njegovu spasi od jame i da mu život svjetlošću obasja. 
\par 31 Pazi dÓe, Jobe, dobro me poslušaj; šuti, jer nisam sve još izrekao. 
\par 32 Ako riječi još imaš, odvrati mi, zbori - rado bih opravdao tebe. 
\par 33 Ako li nemaš, poslušaj me samo: pazi, rad bih te poučit' mudrosti." 


\chapter{34}

\par 1 Elihu nastavi svoju besjedu i reče: 
\par 2 "I vi, mudraci, čujte što ću reći, vi, ljudi umni, poslušajte mene, 
\par 3 jer uši nam prosuđuju besjede isto kao što nepce hranu kuša. 
\par 4 Zajedno ispitajmo što je pravo i razmislimo skupa što je dobro. 
\par 5 Job je utvrdio: 'Ja sam pravedan, ali Bog meni pravdu uskraćuje. 
\par 6 U pravu sam, a lašcem prave mene, nasmrt prostrijeljen, a bez krivnje svoje!' 
\par 7 Zar gdje čovjeka ima poput Joba koji porugu pije kao vodu, 
\par 8 sa zlikovcima koji skupa hodi i s opakima isti dijeli put? 
\par 9 On tvrdi: 'Kakva korist je čovjeku od tog što Bogu ugoditi želi?' 
\par 10 Stoga me čujte, vi ljudi pametni! Od Boga zlo je veoma daleko i nepravednost od Svemogućega, 
\par 11 te on čovjeku plaća po djelima, daje svakom po njegovu vladanju. 
\par 12 Odista, Bog zla nikada ne čini, niti Svesilni kad izvrće pravo. 
\par 13 TÓa tko je njemu povjerio zemlju i vasioni svijet tko je stvorio? 
\par 14 Kad bi on dah svoj u se povukao, kad bi čitav svoj duh k sebi vratio, 
\par 15 sva bića bi odjednom izdahnula i u prah bi se pretvorio čovjek. 
\par 16 Ako razuma imaš, slušaj ovo, prikloni uho glasu riječi mojih. 
\par 17 Može li vladat' koji mrzi pravo? Najpravednijeg hoćeš li osudit'? - 
\par 18 Onog koji kaže kralju: 'Nitkove!' a odličniku govori: 'Zlikovče!' 
\par 19 Koji nije spram knezovima pristran i jednak mu je ubog i mogućnik, jer oni su djelo ruku njegovih? 
\par 20 Zaglave za tren, usred gluhe noći: komešaju se narodi, prolaze; ni od čije ruke moćni padaju. 
\par 21 Jer, on nadzire pute čovjekove, pazi nad svakim njegovim korakom. 
\par 22 Nema toga mraka niti crne tmine gdje bi se mogli zlikovci sakriti. 
\par 23 Bog nikome unaprijed ne kaže kada će na sud pred njega stupiti. 
\par 24 Bez saslušanja on satire jake i stavlja druge na njihovo mjesto. 
\par 25 TÓa odveć dobro poznaje im djela! Sred noći on ih obara i gazi. 
\par 26 Ćuškom ih bije zbog zloće njihove na mjestu gdje ih svi vidjeti mogu. 
\par 27 Jer prestadoše za njime hoditi, zanemariše putove njegove 
\par 28 goneć uboge da vape do njega i potlačene da k njemu leleču. 
\par 29 Al' miruje li, tko da njega gane? Zastre li lice, tko ga vidjet' može? 
\par 30 Nad pucima bdi k'o i nad čovjekom da ne zavlada tko narod zavodi. 
\par 31 Kada bezbožnik Bogu svome kaže: 'Zavedoše me, više griješit neću. 
\par 32 Ne uviđam li, ti me sad pouči, i ako sam kad nepravdu činio, ubuduće ja činiti je neću!' 
\par 33 Misliš da Bog mora njega kazniti, dok ti zamisli njegove prezireš? Al' kada ti odlučuješ, a ne ja, mudrost nam svoju istresi dÓe sada! 
\par 34 Svi ljudi umni sa mnom će se složit' i svatko razuman koji čuje mene: 
\par 35 Nepromišljeno Job je govorio, u riječima mu neima mudrosti. 
\par 36 Stoga, nek' se Job dokraja iskuša, jer odgovara poput zlikovaca; 
\par 37 a svom grijehu još pobunu domeće, među nama on plješće dlanovima i hule svoje na Boga gomila." 


\chapter{35}

\par 1 Elihu nastavi svoju besjedu i reče: 
\par 2 "Zar ti misliš da pravo svoje braniš, da pravednost pred Bogom dokazuješ, 
\par 3 kada mu kažeš: 'Što ti je to važno, i ako griješim, što ti činim time?' 
\par 4 Na sve to ja ću odgovorit' tebi i prijateljima tvojim ujedno. 
\par 5 Po nebu se obazri i promatraj! Gledaj oblake: od tebe su viši! 
\par 6 Ako griješiš, što si mu uradio, prijestupom svojim što si mu zadao? 
\par 7 Ako si prav, što si dodao njemu i što iz ruke tvoje on dobiva? 
\par 8 Opakost tvoja tebi slične pogađa i pravda tvoja čovjeku koristi. 
\par 9 Ali kad ispod teškog stenju jarma, kad vapiju na nasilje moćnika, 
\par 10 nitko ne kaže: 'Gdje je Bog, moj tvorac, koji noć pjesmom veselom ispunja, 
\par 11 umnijim nas od zvijeri zemskih čini i mudrijima od ptica nebeskih?' 
\par 12 Tad vapiju, al' on ne odgovara poradi oholosti zlikovaca. 
\par 13 Ali kako je isprazno tvrditi da Bog njihove ne čuje vapaje, da pogled na njih ne svraća Svesilni! 
\par 14 A kamoli tek kada ti govoriš: 'On ne vidi mene, parnica moja pred njime stoji, a ja na nj još čekam.' 
\par 15 Ili: 'Njegova srdžba ne kažnjava, nimalo on za prijestupe ne mari.' 
\par 16 Isprazno tada otvara Job usta i besjede gomila nerazumne." 


\chapter{36}

\par 1 Elihu nastavi i reče: 
\par 2 "Strpi se malo, pa ću te poučit', jer još nisam sve rekao za Boga. 
\par 3 Izdaleka ću svoje iznijet' znanje da Stvoritelja svojega opravdam. 
\par 4 Zaista, za laž ne znaju mi riječi, uza te je čovjek znanjem savršen. 
\par 5 Gle, Bog je silan, ali ne prezire, silan je snagom razuma svojega. 
\par 6 Opakome on živjeti ne daje, nevoljnicima pravicu pribavlja. 
\par 7 S pravednika on očiju ne skida, na prijestolje ih diže uz kraljeve da bi dovijeka bili uzvišeni. 
\par 8 Ako su negvam' oni okovani i užetima nevolje sputani, 
\par 9 djela njihova on im napominje, kazuje im grijeh njine oholosti. 
\par 10 Tad im otvara uho k opomeni i poziva ih da se zla okane. 
\par 11 Poslušaju li te mu se pokore, dani im završavaju u sreći, u užicima godine njihove. 
\par 12 Ne slušaju li, od koplja umiru, zaglave, sami ne znajući kako. 
\par 13 A srca opaka mržnju njeguju, ne ištu pomoć kad ih on okuje; 
\par 14 u cvatu svoga dječaštva umiru i venu poput hramskih milosnika. 
\par 15 Nevoljnog on bijedom njegovom spasava i u nesreći otvara mu oči: 
\par 16 izbavit će te iz ždrijela tjeskobe k prostranstvima bezgraničnim izvesti, k prepunu stolu mesa pretiloga. 
\par 17 Ako sudio nisi opakima, ako si pravo krnjio siroti, 
\par 18 nek' te obilje odsad ne zavede i nek' te dar prebogat ne iskvari. 
\par 19 Nek' ti je gavan k'o čovjek bez zlata, a čovjek jake ruke poput slaba. 
\par 20 Ne goni one koji su ti tuđi da rodbinu na njino mjesto staviš. 
\par 21 Pazi se da u nepravdu ne skreneš, jer zbog nje snađe tebe iskušenje. 
\par 22 Gle, uzvišen je Bog u svojoj snazi! Zar učitelja ima poput njega? 
\par 23 Tko je njemu put njegov odredio? Tko će mu reći: 'Radio si krivo'? 
\par 24 Spomeni se veličati mu djelo što ga pjesmama ljudi opjevaše. 
\par 25 S udivljenjem svijet čitav ga promatra, divi se čovjek, pa ma izdaleka. 
\par 26 Veći je Bog no što pojmit' možemo, nedokučiv je broj ljeta njegovih! 
\par 27 U visini on skuplja kapi vode te dažd u paru i maglu pretvara. 
\par 28 Pljuskovi tada pljušte iz oblaka, po mnoštvu ljudskom dažde obilato. 
\par 29 Tko li će shvatit' širenje oblaka, tutnjavu strašnu njegovih šatora? 
\par 30 Gle, on nad sobom razastire svjetlost i dno morsko on vodama pokriva. 
\par 31 Pomoću njih on podiže narode, u izobilju hranom ih dariva. 
\par 32 On munju drži objema rukama i kazuje joj kamo će zgoditi. 
\par 33 Glasom gromovnim sebe navješćuje, stiže s gnjevom da zgromi opačinu. 


\chapter{37}

\par 1 Da, od toga i moje srce drhti i s mjesta svoga iskočiti hoće. 
\par 2 Čujte, čujte gromor glasa njegova, tutnjavu što mu iz usta izlazi. 
\par 3 Gle, munja lijeće preko cijelog neba - i sijevne blijesak s kraja na kraj zemlje - 
\par 4 iza nje silan jedan glas se ori: to On gromori glasom veličajnim. Munje mu lete, nitko ih ne priječi, tek što mu je glas jednom odjeknuo. 
\par 5 Da, Bog gromori glasom veličajnim, djela velebna, neshvatljiva stvara. 
\par 6 Kad snijegu kaže: 'Zasniježi po zemlji!' i pljuskovima: 'Zapljuštite silno!' 
\par 7 svakom čovjeku zapečati ruke da svi njegovo upoznaju djelo. 
\par 8 U brlog se tad zvijeri sve uvuku i na svojem se šćućure ležaju. 
\par 9 S južne se strane podiže oluja, a studen vjetri sjeverni donose. 
\par 10 Već led od daha Božjega nastaje i vodena se kruti površina. 
\par 11 I opet vodom puni on oblake, i sijevat' stanu oblaci munjama; 
\par 12 kruže posvuda po volji njegovoj, što im naloži, to će izvršiti na licu cijelog kruga zemaljskoga. 
\par 13 Šalje ih - ili da kazni narode, ili da ih milosrđem obdari. 
\par 14 Poslušaj ovo, Jobe, umiri se i promotri djela Božja čudesna. 
\par 15 Znaš li kako Bog njima zapovijeda, kako munju iz oblaka svog pušta? 
\par 16 Znaš li o čem vise gore oblaci? Čudesna to su znanja savršenog. 
\par 17 Kako ti gore od žege haljine u južnom vjetru kad zemlja obamre? 
\par 18 Zar si nebesa s njim ti razapeo, čvrsta poput ogledala livenog? 
\par 19 DÓe naputi me što da mu kažemo: zbog tmine se ne snalazimo više. 
\par 20 Zar ćeš mu reći: 'Hoću govoriti'? Ili na propast vlastitu pristati? 
\par 21 Tko, dakle, može u svjetlost gledati na nebesima što se sja blistavo kada oblake rastjeraju vjetri? 
\par 22 Sa sjevera k'o zlato je bljesnulo: veličanstvom strašnim Bog se odjenu! 
\par 23 Da, Svesilnog doseći ne možemo, neizmjeran je u moći i sudu, velik u pravdi, nikog on ne tlači. 
\par 24 Zato ljudi svi neka ga se boje! Na mudrost oholu on i ne gleda!" 


\chapter{38}

\par 1 Nato Jahve odgovori Jobu iz oluje i reče: 
\par 2 "Tko je taj koji riječima bezumnim zamračuje božanski promisao? 
\par 3 Bokove svoje opaši k'o junak: ja ću te pitat', a ti me pouči. 
\par 4 Gdje si bio kad zemlju utemeljih? Kazuj, ako ti je znanje sigurno. 
\par 5 Znaš li tko joj je mjere odredio i nad njom uže mjerničko napeo? 
\par 6 Na čemu joj počivaju temelji? Tko joj postavi kamen ugaoni 
\par 7 dok su klicale zvijezde jutarnje i Božji uzvikivali dvorjani? 
\par 8 Tko li zatvori more vratnicama kad je navrlo iz krila majčina; 
\par 9 kad ga oblakom k'o haljom odjenuh i k'o pelenam' ovih maglom gustom; 
\par 10 kad sam njegovu odredio među, vrata stavio sa prijevornicama? 
\par 11 Dotle, ne dalje, rekao sam njemu, tu nek' se lomi ponos tvog valovlja! 
\par 12 Zar si ikad zapovjedio jutru, zar si kazao zori mjesto njeno, 
\par 13 da poduhvati zemlju za rubove i da iz nje sve bezbožnike strese; 
\par 14 da je pretvori u glinu pečatnu i oboji je k'o kakvu haljinu. 
\par 15 Ona uzima svjetlost zlikovcima i pesnicu im lomi uzdignutu. 
\par 16 Zar si ti prodro do izvora morskih, po dnu bezdana zar si kad hodio? 
\par 17 Zar su ti vrata smrti pokazali; vidje li dveri kraja mrtvih sjena? 
\par 18 Zar si prostranstvo zemlje uočio? Govori, ako ti je znano sve to. 
\par 19 Koji putovi u dom svjetla vode, na kojem mjestu prebivaju tmine, 
\par 20 da ih odvedeš u njine krajeve, da im put k stanu njihovu pokažeš? 
\par 21 Ti znadeš to, tÓa davno ti se rodi, tvojih dana broj veoma je velik! 
\par 22 Zar si stigao do riznica snijega i zar si tuče spremišta vidio 
\par 23 što ih pričuvah za dane nevolje, za vrijeme boja krvava i rata? 
\par 24 Kojim li se putem dijeli munja kada iskre po svoj zemlji prosipa? 
\par 25 Tko li je jaz iskopao povodnju, tko prokrčio pute grmljavini 
\par 26 da bi daždjelo na kraj nenastanjen, na pustinju gdje žive duše nema, 
\par 27 da bi neplodnu napojio pustoš, da bi u stepi trava izniknula? 
\par 28 Ima li kiša svoga roditelja? Tko je taj koji kapi rose rađa? 
\par 29 Iz čijeg li mraz izlazi krila, tko slanu stvara što s nebesa pada? 
\par 30 Kako čvrsnu vode poput kamena i led se hvata površja bezdana? 
\par 31 Možeš li lancem vezati Vlašiće i razdriješiti spone Orionu, 
\par 32 u pravo vrijeme izvesti Danicu, vodit' Medvjeda s njegovim mladima? 
\par 33 Zar poznaješ ti zakone nebeske pa da njima moć na zemlji dodijeliš? 
\par 34 Zar doviknuti možeš oblacima pa da pljuskovi tebe poslušaju? 
\par 35 Zar na zapovijed tvoju munje lijeću i tebi zar se odazivlju: 'Evo nas'? 
\par 36 Tko je mudrost darovao ibisu, tko li je pamet ulio u pijetla? 
\par 37 Tko to mudro prebrojava oblake i tko nebeske izlijeva mjehove 
\par 38 dok se zemlja u tijesto ne zgusne i dok se grude njezine ne slijepe? 
\par 39 Zar ćeš ti plijen uloviti lavici ili ćeš glad utažit' lavićima 
\par 40 na leglu svojem dok gladni čekaju i vrebaju na žrtvu iz zaklona? 
\par 41 Tko hranu gavranovima pribavlja kad Bogu ptići njegovi cijuču i naokolo oblijeću bez hrane? 


\chapter{39}

\par 1 Znaš li kako se legu divokoze? Vidje li kako se mlade košute? 
\par 2 Izbroji li koliko nose mjeseci, znaš li u koje doba se omlade? 
\par 3 Sagnuvši se, polegu lanad svoju i breme usred pustinje odlažu, 
\par 4 a kad im porod ojača, poraste, ostave ga i ne vraćaju mu se. 
\par 5 Tko dade divljem magarcu slobodu i tko to oglav skinu njemu s glave? 
\par 6 U zavičaj mu dadoh ja pustinju i polja slana da ondje živuje. 
\par 7 Buci gradova on se podruguje i ne sluša goničevih povika. 
\par 8 Luta brdima, svojim pašnjacima, u potrazi za zeleni svakakvom. 
\par 9 Možeš li slugom učinit' bivola, zadržat' ga noć jednu za jaslama? 
\par 10 Možeš li njega za brazdu prikovat' da ralo vuče po docima tvojim? 
\par 11 Možeš li se osloniti na njega jer je njegova snaga prevelika i prepustit' mu težak svoj posao? 
\par 12 Misliš li tebi da će se vratiti i na gumno ti dotjerati žito? 
\par 13 Krilima svojim noj trepće radosno, iako krila oskudnih i perja. 
\par 14 On svoja jaja na zemlji ostavlja, povjerava ih pijesku da ih grije, 
\par 15 ne mareć' što ih zgazit' može noga ili nekakva divlja zvijer zgnječiti. 
\par 16 S nojićima k'o s tuđima postupa; što mu je trud zaludu, on ne mari. 
\par 17 Jer Bog je njega lišio pameti, nije mu dao nikakva razbora. 
\par 18 Ali kada na let krila raširi, tada se ruga konju i konjaniku. 
\par 19 Zar si ti konja obdario snagom zar si mu ti vrat grivom ukrasio? 
\par 20 Zar ti činiš da skače k'o skakavac, da u strah svakog nagoni hrzanjem? 
\par 21 Kopitom zemlju veselo raskapa, neustrašivo srlja na oružje. 
\par 22 Strahu se ruga, ničeg se ne boji, ni pred mačem uzmaknuti neće. 
\par 23 Na sapima mu zvekeće tobolac, koplje sijeva i ubojna sulica. 
\par 24 Bijesan i nestrpljiv guta prostore; kad rog zasvira, tko će ga zadržat': 
\par 25 na svaki zvuk roga on zarže: Ha! Izdaleka on ljuti boj već njuši, viku bojnu i poklič vojskovođa. 
\par 26 Zar po promislu tvojem lijeće soko i prema jugu krila svoja širi? 
\par 27 Zar se na nalog tvoj diže orao i vrh timora gnijezdo sebi vije? 
\par 28 Na litici on stanuje i noćÄi, na grebenima vrleti visokih. 
\par 29 Odatle na plijen netremice vreba, oči njegove vide nadaleko. 
\par 30 Krvlju se hrane njegovi orlići; gdje je ubijenih, tamo je i on." 


\chapter{40}

\par 1 I Jahve se obrati Jobu i reče mu: 
\par 2 "Zar će se s Jakim preti još kudilac? Tužitelj Božji nek' sam odgovori!" 
\par 3 A Job odgovori Jahvi i reče: 
\par 4 "Odveć sam malen: što da odgovorim? Rukom ću svoja zatisnuti usta. 
\par 5 Riječ rekoh - neću više započeti; rekoh dvije - al' neću nastaviti." 
\par 6 Nato Jahve odgovori Jobu iz oluje i reče: 
\par 7 "Bokove svoje opaši k'o junak, ja ću te pitat', a ti me pouči. 
\par 8 Zar bi i moj sud pogaziti htio, okrivio me da sebe opravdaš? 
\par 9 Zar ti mišica snagu Božju ima, zar glasom grmjet' možeš poput njega? 
\par 10 Ogrni se sjajem i veličanstvom, dostojanstvom se odjeni i slavom. 
\par 11 Plani dÓe bijesom ognja jarosnoga, pogledom jednim snizi oholnika. 
\par 12 Ponositoga pogledaj, slomi ga, na mjestu satri svakoga zlikovca. 
\par 13 U zemlju sve njih zajedno zakopaj, u mračnu ih pozatvaraj tamnicu. 
\par 14 Tada ću i ja tebi odat' hvalu što si se svojom desnicom spasio. 
\par 15 A sada, dÓe promotri Behemota! Travom se hrani poput govečeta, 
\par 16 u bedrima je, gle, snaga njegova, a krepkost mu u mišićju trbušnom. 
\par 17 Poput cedra rep podignut ukruti, sva su mu stegna ispreplele žile. 
\par 18 Mjedene cijevi kosti su njegove, zglobovi mu od željeza kvrge. 
\par 19 Prvenac on je Božjega stvaranja; mačem ga je naoružao tvorac. 
\par 20 Gore mu danak u hrani donose i sve zvijerje što po njima se igra. 
\par 21 Pod lotosom on zavaljen počiva, guštik močvarni i glib kriju ga. 
\par 22 Sjenu mu pravi lotosovo lišće, pod vrbama on hladuje potočnim. 
\par 23 Nabuja li rijeka, on ne strahuje: nimalo njega ne bi zabrinulo da mu u žvale i sav Jordan jurne. 
\par 24 Tko bi za oči uhvatio njega i tko bi mu nos sulicom probio? 


\chapter{41}

\par 1 (40:25) Zar loviš Levijatana udicom? Zar ćeš mu jezik zažvalit' užetom? 
\par 2 (40:26) Zar mu nozdrve trskom probost' možeš ili mu kukom probiti vilicu? 
\par 3 (40:27) Hoće li te on preklinjat' za milost, hoće li s tobom blago govoriti? 
\par 4 (40:28) I zar će s tobom savez on sklopiti da sveg života tebi sluga bude? 
\par 5 (40:29) Hoćeš li se s njim k'o s pticom poigrat' i vezat' ga da kćeri razveseliš? 
\par 6 (40:30) Hoće li se za nj cjenkati ribari, među sobom podijelit' ga trgovci? 
\par 7 (40:31) Možeš li kopljem njemu kožu izbost ili glavu mu probiti ostima? 
\par 8 (40:32) Podigni de ruku svoju na njega: za boj se spremi - bit će ti posljednji! 
\par 9 (41:1) Zalud je nadu u njega gojiti, na pogled njegov čovjek već pogiba. 
\par 10 (41:2) Junaka nema da njega razdraži, tko će mu se u lice suprotstavit'? 
\par 11 (41:3) Tko se sukobi s njim i živ ostade? Pod nebesima tog čovjeka nema! 
\par 12 (41:4) Prešutjet neću njegove udove, ni silnu snagu, ni ljepotu stasa. 
\par 13 (41:5) Tko mu smije razodjenut' odjeću, tko li kroz dvostruk prodrijeti mu oklop? 
\par 14 (41:6) Tko će mu ralje rastvorit' dvokrilne kad strah vlada oko zubi njegovih? 
\par 15 (41:7) Hrbat mu je od ljuskavih štitova, zapečaćenih pečatom kamenim. 
\par 16 (41:8) Jedni uz druge tako se sljubiše da među njima dah ne bi prošao. 
\par 17 (41:9) Tako su čvrsto slijepljeni zajedno: priljubljeni, razdvojit' se ne mogu. 
\par 18 (41:10) Kad kihne, svjetlost iz njega zapršti, poput zorinih vjeđa oči su mu. 
\par 19 (41:11) Zublje plamsaju iz njegovih ralja, iskre ognjene iz njih se prosiplju. 
\par 20 (41:12) Iz nozdrva mu sukljaju dimovi kao iz kotla što kipi na vatri. 
\par 21 (41:13) Dah bi njegov zapalio ugljevlje, jer mu iz ralja plamenovi suču. 
\par 22 (41:14) U šiji leži sva snaga njegova, a ispred njega užas se prostire. 
\par 23 (41:15) Kad se ispravi, zastrepe valovi i prema morskoj uzmiču pučini. 
\par 24 (41:16) Poput pećine srce mu je tvrdo, poput mlinskoga kamena otporno. 
\par 25 (41:17) Pregibi tusta mesa srasli su mu, čvrsti su kao da su saliveni. 
\par 26 (41:18) Zgodi li ga mač, od njeg se odbije, tako i koplje, sulica i strijela. 
\par 27 (41:19) Poput slame je za njega željezo, mjed je k'o drvo iscrvotočeno. 
\par 28 (41:20) On ne uzmiče od strelice s luka, stijenje iz praćke na nj k'o pljeva pada. 
\par 29 (41:21) K'o slamčica je toljaga za njega, koplju se smije kad zazviždi nad njim. 
\par 30 (41:22) Crepovlje oštro ima na trbuhu i blato njime ore k'o drljačom. 
\par 31 (41:23) Pod njim vrtlog sav k'o lonac uskipi, uspjeni more k'o pomast u kotlu. 
\par 32 (41:24) Za sobom svijetlu ostavlja on brazdu, regbi, bijelo runo bezdan prekriva. 
\par 33 (41:25) Ništa slično na zemlji ne postoji i niti je tko tako neustrašiv. 
\par 34 (41:26) I na najviše on s visoka gleda, kralj je svakome, i najponosnijim." 



\chapter{42}

\par 1 A Job ovako odgovori Jahvi: 
\par 2 "Ja znadem, moć je tvoja bezgranična: što god naumiš, to izvesti možeš. 
\par 3 Tko je taj koji riječima bezumnim zamračuje božanski promisao? Govorah stoga, ali ne razumjeh, o čudesima meni neshvatljivim. 
\par 4 O, poslušaj me, pusti me da zborim: ja ću te pitat', a ti me pouči. 
\par 5 Po čuvenju tek poznavah te dosad, ali sada te oči moje vidješe. 
\par 6 Sve riječi svoje zato ja poričem i kajem se u prahu i pepelu." 
\par 7 Kada Jahve izgovori Jobu ove riječi, reče on Elifazu Temancu:  "Ti i tvoja dva prijatelja raspalili ste gnjev moj jer niste  o meni onako pravo govorili kao moj sluga Job. 
\par 8 Zato uzmite  sada sedam junaca i sedam ovnova i pođite k mome sluzi Jobu,  pa prinesite za sebe paljenicu, a sluga moj Job molit će se za  vas. Imat ću obzira prema njemu i neću vam učiniti ništa nažao  zato što niste o meni onako pravo govorili kao moj sluga Job." 
\par 9 Tada odoše Elifaz iz Temana, Bildad iz Šuaha i Sofar iz  Naamata i učiniše kako im je Jahve zapovjedio. I Jahve se obazre  na Joba. 
\par 10 I Jahve vrati Joba u prijašnje stanje jer se založio  za svoje prijatelje, pa mu još udvostruči ono što je posjedovao. 
\par 11 Tad se vratiše Jobu sva njegova braća, i sve njegove sestre, i svi prijašnji znanci te su jeli s njim kruh u njegovoj kući, žaleći ga i tješeći zbog svih nevolja što ih Jahve bijaše na  nj poslao. Svaki mu darova po jedan srebrnik i po jedan zlatan  prsten. 
\par 12 Jahve blagoslovi novo Jobovo stanje još više negoli  prijašnje. Blago mu je brojilo četrnaest tisuća ovaca, šest tisuća  deva, tisuću jarmova volova i tisuću magarica. 
\par 13 Imao je sedam  sinova i tri kćeri. 
\par 14 Prvoj nadjenu ime Jemima, drugoj Kasija, a trećoj Keren-Hapuk. 
\par 15 U svem onom kraju ne bijaše žena tako  lijepih kao Jobove kćeri. I otac im dade jednaku baštinu kao  i njihovoj braći. 
\par 16 Poslije toga Job doživje dob od sto četrdeset godina  i vidje djecu svoju i djecu svoje djece do četvrtog koljena.  Potom umrije Job, star, nauživši se života. 
\par 17 




\end{document}