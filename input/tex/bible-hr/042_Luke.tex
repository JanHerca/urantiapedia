\begin{document}

\title{Luka}


\chapter{1}

\par 1 Kad već mnogi poduzeše sastaviti izvješće o događajima koji  se ispuniše među nama - 
\par 2 kako nam to predadoše oni koji od  početka bijahu očevici i sluge Riječi - 
\par 3 pošto sam sve, od  početka, pomno ispitao, naumih i ja tebi, vrli Teofile, sve po  redu napisati 
\par 4 da se tako osvjedočiš o pouzdanosti svega u  čemu si poučen. 
\par 5 U dane Heroda, kralja judejskoga, bijaše neki svećenik  imenom Zaharija iz razreda Abijina. Žena mu bijaše od kćeri Aronovih, a ime joj Elizabeta. 
\par 6 Oboje bijahu pravedni pred Bogom: živjeli  su besprijekorno po svim zapovijedima i odredbama Gospodnjim. 
\par 7 No nisu imali djeteta jer Elizabeta bijaše nerotkinja, a oboje  već poodmakle dobi. 
\par 8 Dok je Zaharija jednom po redu svoga razreda obavljao  svećeničku službu pred Bogom, 
\par 9 ždrijebom ga zapade po bogoslužnom  običaju da uđe u Svetište Gospodnje i prinese kad. 
\par 10 Za vrijeme  kađenice sve je ono mnoštvo naroda vani molilo. 
\par 11 A njemu se ukaza anđeo Gospodnji. Stajao je s desne strane  kadionoga žrtvenika. 
\par 12 Ugledavši ga, Zaharija se prepade i  strah ga spopade. 
\par 13 No anđeo mu reče: "Ne boj se, Zaharija!  Uslišana ti je molitva: žena će ti Elizabeta roditi sina. Nadjenut  ćeš mu ime Ivan. 
\par 14 Bit će ti radost i veselje i rođenje će  njegovo mnoge obradovati. 
\par 15 Bit će doista velik pred Gospodinom.  Ni vina ni drugoga opojnog pića neće piti. Duha Svetoga  bit će pun već od majčine utrobe. 
\par 16 Mnoge će sinove Izraelove  obratiti Gospodinu, Bogu njihovu. 
\par 17 Ići će pred njim u duhu  i sili Ilijinoj da obrati srce otaca k sinovima i nepokorne  k razumnosti pravednih te spremi Gospodinu narod pripravan." 
\par 18 Nato Zaharija reče anđelu: "Po čemu ću ja to razaznati.  Ta star sam i žena mi poodmakle dobi." 
\par 19 Anđeo mu odgovori:  "Ja sam Gabriel koji stojim pred Bogom. Poslan sam da govorim  s tobom i da ti donesem ovu radosnu poruku. 
\par 20 I evo, budući  da nisi povjerovao mojim riječima, koje će se ispuniti u svoje  vrijeme, zanijemjet ćeš i nećeš moći govoriti do dana dok se  to ne zbude." 
\par 21 Narod je iščekivao Zahariju i čudio se što se toliko  zadržao u Svetištu. 
\par 22 Kad je napokon izašao, nije im mogao  ništa reći pa zaključiše da je u Svetištu imao viđenje. Nastojao  im se doduše izraziti znakovima, ali osta nijem. 
\par 23 Kad se navršiše dani njegove službe, otiđe kući. 
\par 24 Nakon  tih dana zatrudnje Elizabeta, njegova žena. Krila se pet mjeseci  govoreći: 
\par 25 "Evo, to mi je učinio Gospodin u dane kad mu se  svidje skinuti s mene sramotu među ljudima." 
\par 26 U šestome mjesecu posla Bog anđela Gabriela u galilejski  grad imenom Nazaret 
\par 27 k djevici zaručenoj s mužem koji se zvao  Josip iz doma Davidova; a djevica se zvala Marija. 
\par 28 Anđeo  uđe k njoj i reče: "Zdravo, milosti puna! Gospodin s tobom!" 
\par 29 Na tu se riječ ona smete i stade razmišljati kakav bi to  bio pozdrav. 
\par 30 No anđeo joj reče: "Ne boj se, Marijo! Ta našla  si milost u Boga. 
\par 31 Evo, začet ćeš i roditi sina i nadjenut  ćeš mu ime Isus. 
\par 32 On će biti velik i zvat će se Sin Svevišnjega.  Njemu će Gospodin Bog dati prijestolje Davida, oca njegova, 
\par 33 i kraljevat će nad domom Jakovljevim uvijeke  i njegovu kraljevstvu neće biti kraja." 
\par 34 Nato će Marija anđelu: "Kako će to biti kad ja muža ne  poznajem?" 
\par 35 Anđeo joj odgovori: "Duh Sveti sići će na te i  sila će te Svevišnjega osjeniti. Zato će to čedo i biti sveto, Sin Božji. 
\par 36 A evo tvoje rođakinje Elizabete: i ona u starosti  svojoj zače sina. I njoj, nerotkinjom prozvanoj, ovo je već šesti  mjesec. 
\par 37 Ta Bogu ništa nije nemoguće!" 
\par 38 Nato Marija reče: "Evo službenice Gospodnje, neka mi  bude po tvojoj riječi!" I anđeo otiđe od nje. 
\par 39 Tih dana usta Marija i pohiti u Gorje, u grad Judin. 
\par 40 Uđe u Zaharijinu kuću i pozdravi Elizabetu. 
\par 41 Čim Elizabeta  začu Marijin pozdrav, zaigra joj čedo u utrobi. I napuni se Elizabeta  Duha Svetoga 
\par 42 i povika iz svega glasa: "Blagoslovljena ti  među ženama i blagoslovljen plod utrobe tvoje! 
\par 43 Ta otkuda  meni da mi dođe majka Gospodina mojega? 
\par 44 Gledaj samo! Tek  što mi do ušiju doprije glas pozdrava tvojega, zaigra mi od radosti  čedo u utrobi. 
\par 45 Blažena ti što povjerova da će se ispuniti  što ti je rečeno od Gospodina!" 
\par 46 Tada Marija reče: "Veliča duša moja Gospodina, 
\par 47 klikće duh moj u Bogu, mome Spasitelju, 
\par 48 što pogleda na neznatnost službenice svoje: odsad će me, evo, svi naraštaji zvati blaženom. 
\par 49 Jer velika mi djela učini Svesilni, sveto je ime njegovo! 
\par 50 Od koljena do koljena dobrota je njegova nad onima što se njega boje. 
\par 51 Iskaza snagu mišice svoje, rasprši oholice umišljene. 
\par 52 Silne zbaci s prijestolja, a uzvisi neznatne. 
\par 53 Gladne napuni dobrima, a bogate otpusti prazne. 
\par 54 Prihvati Izraela, slugu svoga, kako obeća ocima našim: 
\par 55 spomenuti se dobrote svoje prema Abrahamu i potomstvu njegovu dovijeka." 
\par 56 Marija osta s Elizabetom oko tri mjeseca, a onda se vrati  kući. 
\par 57 Elizabeti se međutim navršilo vrijeme da rodi. I porodi  sina. 
\par 58 Kad su njezini susjedi i rođaci čuli da joj Gospodin  obilno iskaza dobrotu, radovahu se s njome. 
\par 59 Osmoga se dana okupe da obrežu dječaka. Htjedoše ga prozvati  imenom njegova oca - Zaharija, 
\par 60 no mati se njegova usprotivi:  "Nipošto, nego zvat će se Ivan!" 
\par 61 Rekoše joj na to: "Ta nikoga  nema od tvoje rodbine koji bi se tako zvao." 
\par 62 Tada znakovima  upitaju oca kojim ga imenom želi prozvati. 
\par 63 On zaiska pločicu  i napisa "Ivan mu je ime!" Svi se začude, 
\par 64 a njemu se umah  otvoriše usta i jezik te progovori blagoslivljajući Boga. 
\par 65 Strah obuze sve njihove susjede, a po svem su se Gorju  judejskom razglašavali svi ti događaji. 
\par 66 I koji su god čuli, razmišljahu o tome pitajući se: "Što li će biti od ovoga djeteta?"  Uistinu, ruka Gospodnja bijaše s njime. 
\par 67 A Zaharija, otac njegov, napuni se Duha Svetoga i stade  prorokovati: 
\par 68 "Blagoslovljen Gospodin Bog Izraelov, što pohodi i otkupi narod svoj! 
\par 69 Podiže nam snagu spasenja u domu Davida, sluge svojega, 
\par 70 kao što obeća na usta svetih proroka svojih odvijeka: 
\par 71 spasiti nas od neprijatelja naših i od ruke sviju koji nas mrze; 
\par 72 iskazati dobrotu ocima našim i sjetiti se svetog Saveza svojega, 
\par 73 zakletve kojom se zakle Abrahamu, ocu našemu: da će nam dati 
\par 74 te mu, izbavljeni iz ruku neprijatelja, služimo bez straha 
\par 75 u svetosti i pravednosti pred njim u sve dane svoje. 
\par 76 A ti, dijete, prorok ćeš se Svevišnjega zvati jer ćeš ići pred Gospodinom da mu pripraviš putove, 
\par 77 da pružiš spoznaju spasenja narodu njegovu po otpuštenju grijeha njihovih, 
\par 78 darom premilosrdnog srca Boga našega po kojem će nas pohoditi Mlado sunce s visine 
\par 79 da obasja one što sjede u tmini i sjeni smrtnoj, da upravi noge naše na put mira." 
\par 80 Dječak je međutim rastao i duhom jačao. Boravio je u  pustinji sve do dana svoga javnog nastupa pred Izraelom. 


\chapter{2}

\par 1 U one dane izađe naredba cara Augusta da se provede popis svega  svijeta. 
\par 2 Bijaše to prvi popis izvršen za Kvirinijeva upravljanja  Sirijom. 
\par 3 Svi su išli na popis, svaki u svoj grad. 
\par 4 Tako  i Josip, budući da je bio iz doma i loze Davidove, uziđe iz Galileje, iz grada Nazareta, u Judeju - u grad Davidov, koji se zove Betlehem  - 
\par 5 da se podvrgne popisu zajedno sa svojom zaručnicom Marijom  koja bijaše trudna. 
\par 6 I dok se bili ondje, navršilo joj se vrijeme da rodi. 
\par 7 I porodi sina svoga, prvorođenca, povi ga i položi u jasle  jer za njih nije bilo mjesta u svratištu. 
\par 8 A u tom kraju bijahu pastiri: pod vedrim su nebom čuvali  noćnu stražu kod svojih stada. 
\par 9 Anđeo im Gospodnji pristupi  i slava ih Gospodnja obasja! Silno se prestrašiše. 
\par 10 No anđeo  im reče: "Ne bojte se! Evo javljam vam blagovijest, veliku radost  za sav narod! 
\par 11 Danas vam se u gradu Davidovu rodio Spasitelj  - Krist, Gospodin. 
\par 12 I evo vam znaka: naći ćete novorođenče  povijeno gdje leži u jaslama." 
\par 13 I odjednom se anđelu pridruži  silna nebeska vojska hvaleći Boga i govoreći: 
\par 14 "Slava na visinama Bogu, a na zemlji mir ljudima, miljenicima njegovim!" 
\par 15 Čim anđeli otiđoše od njih na nebo, pastiri stanu poticati  jedni druge: "Hajdemo dakle do Betlehema. Pogledajmo što se to  dogodilo, događaj koji nam obznani Gospodin." 
\par 16 I pohite te  pronađu Mariju, Josipa i novorođenče gdje leži u jaslama. 
\par 17 Pošto  sve pogledaše, ispripovijediše što im bijaše rečeno o tom djetetu. 
\par 18 A svi koji su to čuli divili se tome što su im pripovijedali  pastiri. 
\par 19 Marija u sebi pohranjivaše sve te događaje i prebiraše  ih u svome srcu. 
\par 20 Pastiri se zatim vratiše slaveći i hvaleći Boga za sve  što su čuli i vidjeli kako im je bilo rečeno. 
\par 21 Kad se navršilo osam dana da bude obrezan, nadjenuše  mu ime Isus, kako ga je bio prozvao anđeo prije njegova začeća. 
\par 22 Kad se zatim po Mojsijevu Zakonu navršiše dani njihova  čišćenja, poniješe ga u Jeruzalem da ga prikažu Gospodinu  - 
\par 23 kao što piše u Zakonu Gospodnjem: Svako muško prvorođenče  neka se posveti Gospodinu! - 
\par 24 i da prinesu žrtvu kako  je rečeno u Zakonu Gospodnjem: dvije grlice ili dva golubića. 
\par 25 Živio tada u Jeruzalemu čovjek po imenu Šimun. Taj čovjek, pravedan i bogobojazan, iščekivaše Utjehu Izraelovu i Duh Sveti  bijaše na njemu. 
\par 26 Objavio mu Duh Sveti da neće vidjeti smrti  dok ne vidi Pomazanika Gospodnjega. 
\par 27 Ponukan od Duha, dođe  u Hram. I kad roditelji uniješe dijete Isusa da obave što o njemu  propisuje Zakon, 
\par 28 primi ga on u naručje, blagoslovi Boga i  reče: 
\par 29 "Sad otpuštaš slugu svojega, Gospodaru, po riječi svojoj, u miru! 
\par 30 Ta vidješe oči moje spasenje tvoje, 
\par 31 koje si pripravio pred licem svih naroda: 
\par 32 svjetlost na prosvjetljenje naroda, slavu puka svoga izraelskoga." 
\par 33 Otac njegov i majka divili se što se to o njemu govori. 
\par 34 Šimun ih blagoslovi i reče Mariji, majci njegovoj: "Ovaj  je evo postavljen na propast i uzdignuće mnogima u Izraelu i  za znak osporavan - 
\par 35 a i tebi će samoj mač probosti dušu -  da se razotkriju namisli mnogih srdaca!" 
\par 36 A bijaše neka proročica Ana, kći Penuelova, iz plemena  Ašerova, žena veoma odmakla u godinama. Nakon djevojaštva živjela  je s mužem sedam godina, 
\par 37 a sama kao udovica do osamdeset  i četvrte. Nije napuštala Hrama, nego je postovima i molitvama  danju i noću služila Bogu. 
\par 38 Upravo u taj čas nadođe. Hvalila  je Boga i svima koji iščekivahu otkupljenje Jeruzalema pripovijedala  o djetetu. 
\par 39 Kad obaviše sve po Zakonu Gospodnjem, vratiše se u Galileju, u svoj grad Nazaret. 
\par 40 A dijete je raslo, jačalo i napunjalo  se mudrosti i milost je Božja bila na njemu. 
\par 41 Njegovi su roditelji svake godine o blagdanu Pashe išli  u Jeruzalem. 
\par 42 Kad mu bijaše dvanaest godina, uziđoše po običaju  blagdanskom. 
\par 43 Kad su minuli ti dani, vraćahu se oni, a dječak  Isus osta u Jeruzalemu, a da nisu znali njegovi roditelji. 
\par 44 Uvjereni  da je među suputnicima, odoše dan hoda, a onda ga stanu tražiti  među rodbinom i znancima. 
\par 45 I kad ga ne nađu, vrate se u Jeruzalem  tražeći ga. 
\par 46 Nakon tri dana nađoše ga u Hramu gdje sjedi posred učitelja, sluša ih i pita. 
\par 47 Svi koji ga slušahu bijahu zaneseni razumnošću  i odgovorima njegovim. 
\par 48 Kad ga ugledaše, zapanjiše se, a majka  mu njegova reče: "Sinko, zašto si nam to učinio? Gle, otac tvoj  i ja žalosni smo te tražili." 
\par 49 A on im reče: "Zašto ste me  tražili? Niste li znali da mi je biti u onome što je Oca mojega?" 
\par 50 Oni ne razumješe riječi koju im reče. 
\par 51 I siđe s njima, dođe u Nazaret i bijaše im poslušan.  A majka je njegova brižno čuvala sve ove uspomene u svom srcu. 
\par 52 A Isus napredovaše u mudrosti, dobi i milosti kod  Boga i ljudi. 


\chapter{3}

\par 1 Petnaeste godine vladanja cara Tiberija, dok je upravitelj  Judeje bio Poncije Pilat, tetrarh Galileje Herod, a njegov brat  Filip tetrarh Itureje i zemlje trahonitidske, i Lizanije tetrarh  Abilene, 
\par 2 za velikog svećenika Ane i Kajfe, dođe riječ Božja  Ivanu, sinu Zaharijinu, u pustinji. 
\par 3 On obiđe svu okolicu jordansku  propovijedajući obraćeničko krštenje na otpuštenje grijeha 
\par 4 kao  što je pisano u Knjizi besjeda Izaije proroka: Glas viče u pustinji: Pripravite put Gospodinu, poravnite mu staze! 
\par 5 Svaka dolina neka se ispuni, svaka gora i brežuljak neka se slegne! Što je krivudavo, neka se izravna, a hrapavi putovi neka se izglade! 
\par 6 I svako će tijelo vidjeti spasenje Božje. 
\par 7 Govoraše dakle mnoštvu koje je dolazilo da se krsti: "Leglo  gujinje! Tko vas samo upozori da bježite od skore srdžbe? 
\par 8 Donosite  dakle plodove dostojne obraćenja. I nemojte početi u sebi govoriti:  'Imamo oca Abrahama!' Jer, kažem vam: Bog iz ovog kamenja može  podići djecu Abrahamovu. 
\par 9 Već je sjekira položena na korijen  stablima: svako dakle stablo koje ne donosi dobra roda siječe  se i u oganj baca." 
\par 10 Pitalo ga mnoštvo: "Što nam je dakle činiti?" 
\par 11 On  im odgovaraše: "Tko ima dvije haljine, neka podijeli s onim koji  nema. U koga ima hrane, neka učini isto tako." 
\par 12 Dođoše krstiti  se i carinici pa ga pitahu: "Učitelju, što nam je činiti?" 
\par 13 Reče  im: "Ne utjerujte više nego što vam je određeno." 
\par 14 Pitahu  ga i vojnici: "A nama, što je nama činiti?" I reče im: "Nikome  ne činite nasilja, nikoga krivo ne prijavljujte i budite zadovoljni  svojom plaćom." 
\par 15 Narod bijaše u iščekivanju i svi se u srcu pitahu o Ivanu  nije li on možda Krist. 
\par 16 Zato im Ivan svima reče: "Ja vas, istina, vodom krstim. Ali dolazi jači od mene. Ja nisam dostojan  odriješiti mu remenje na obući. On će vas krstiti Duhom Svetim  i ognjem. 
\par 17 U ruci mu vijača da pročisti gumno svoje i sabere  žito u žitnicu svoju, a pljevu će spaliti ognjem neugasivim." 
\par 18 I mnogim je drugim pobudama Ivan narodu navješćivao evanđelje. 
\par 19 A Heroda je tetrarha Ivan prekorio zbog Herodijade, žene  njegova brata i zbog svih njegovih zlodjela. 
\par 20 Svemu tome nadoda  Herod još i ovo: zatvori Ivana u tamnicu. 
\par 21 Kad se krstio sav narod, krstio se i Isus. I dok se molio, rastvori se nebo, 
\par 22 siđe na nj Duh Sveti u tjelesnom obličju, poput goluba, a glas se s neba zaori: "Ti si Sin moj, Ljubljeni!  U tebi mi sva milina!" 
\par 23 Kad je Isus nastupio, bilo mu je oko trideset godina. Bijaše - kako se smatralo - sin Josipov, Elijev, 
\par 24 Matatov, Levijev, Malkijev, Janajev, Josipov. 
\par 25 Matatijin, Amosov,  Naumov, Heslijev, Nagajev, 
\par 26 Mahatov, Matatijin, Šimijev, Josehov, Jodin, 
\par 27 Johananov, Resin, Zerubabelov, Šealtielov, Nerijev, 
\par 28 Malkijev, Adijev, Kosamov, Elmadamov, Erov, 
\par 29 Jošuin, Eliezerov, Jorimov, Matatov, Levijev, 
\par 30 Šimunov, Judin, Josipov, Jonamov, Elijakimov, 
\par 31 Melejin, Menin, Matatin, Natanov, Davidov, 
\par 32 Jišajev, Obedov, Boazov, Salin, Nahšonov, 
\par 33 Aminadabov, Adminov, Arnijev, Hesronov, Peresov, Judin, 
\par 34 Jakovljev, Izakov, Abrahamov, Terahov, Nahorov, 
\par 35 Serugov, Reuov, Pelegov, Eberov, Šelahov, 
\par 36 Kenanov, Arpakšadov, Šemov, Noin, Lamekov, 
\par 37 Metušalahov, Henokov, Jeredov, Mahalalelov, Kenanov, 
\par 38 Enošev, Šetov, Adamov, Božji. 


\chapter{4}

\par 1 Isus se, pun Duha Svetoga, vratio s Jordana i Duh ga četrdeset  dana vodio pustinjom, 
\par 2 gdje ga je iskušavao đavao. Tih dana  nije ništa jeo, te kad oni istekoše, ogladnje. 
\par 3 A đavao mu  reče: "Ako si Sin Božji, reci ovom kamenu da postane kruhom." 
\par 4 Isus mu odgovori: "Pisano je: Ne živi čovjek samo o kruhu." 
\par 5 I povede ga đavao na visoko, pokaza mu odjednom sva kraljevstva  zemlje 
\par 6 i reče mu: "Tebi ću dati svu ovu vlast i slavu njihovu  jer meni je dana i komu hoću, dajem je. 
\par 7 Ako se dakle pokloniš  preda mnom, sve je tvoje." 
\par 8 Isus mu odgovori: "Pisano je: Klanjaj se Gospodinu, Bogu svomu, i njemu jedinomu služi!" 
\par 9 Povede ga u Jeruzalem i postavi na vrh Hrama i reče mu:  "Ako si Sin Božji, baci se odavde dolje! 
\par 10 Ta pisamo je: Anđelima će svojim zapovjediti za tebe da te čuvaju. 
\par 11 I: Na rukama će te nositi da se gdje nogom ne spotakneš o kamen." 
\par 12 Odgovori mu Isus: "Rečeno je: Ne iskušavaj Gospodina, Boga svojega!" 
\par 13 Pošto iscrpi sve kušnje, đavao se udalji od njega do  druge prilike. 
\par 14 A Isus se u snazi Duha vrati u Galileju te glas o njemu  puče po svoj okolici. 
\par 15 I slavljen od sviju, naučavaše po njihovim  sinagogama. 
\par 16 I dođe u Nazaret, gdje bijaše othranjen. I uđe po svom  običaju na dan subotni u sinagogu te ustane čitati. 
\par 17 Pruže  mu Knjigu proroka Izaije. On razvije knjigu i nađe mjesto gdje  stoji napisano: 
\par 18 Duh Gospodnji na meni je jer me pomaza! On me posla blagovjesnikom biti siromasima, proglasiti sužnjima oslobođenje, vid slijepima, na slobodu pustiti potlačene, 
\par 19 proglasiti godinu milosti Gospodnje. 
\par 20 Tada savi knjigu, vrati je poslužitelju i sjede. Oči  sviju u sinagogi bijahu uprte u njega. 
\par 21 On im progovori: "Danas  se ispunilo ovo Pismo što vam još odzvanja u ušima." 
\par 22 I svi  su mu povlađivali i divili se milini riječi koje su tekle iz  njegovih usta. Govorahu: "Nije li ovo sin Josipov?" 
\par 23 A on im reče: "Zacijelo ćete mi reći onu prispodobu:  Liječniče, izliječi sam sebe! Što smo čuli da se dogodilo u Kafarnaumu, učini i ovdje, u svom zavičaju." 
\par 24 I nastavi: "Zaista, kažem  vam, nijedan prorok nije dobro došao u svom zavičaju. 
\par 25 Uistinu, kažem vam, mnogo bijaše udovica u Izraelu u dane Ilijine kad  se na tri godine i šest mjeseci zatvorilo nebo pa zavladala velika  glad po svoj zemlji. 
\par 26 I ni k jednoj od njih nije bio poslan  Ilija doli k ženi udovici u Sarfati sidonskoj. 
\par 27 I mnogo bijaše  gubavaca u Izraelu za proroka Elizeja. I nijedan se od njih ne  očisti doli Naaman Sirac." 
\par 28 Čuvši to, svi se u sinagogi napune gnjevom, 
\par 29 ustanu, izbace ga iz grada i odvedu na rub brijega na kojem je sagrađen  njihov grad da ga strmoglave. 
\par 30 No on prođe između njih i ode. 
\par 31 I siđe u Kafarnaum, grad galilejski. I poučavaše ih subotom 
\par 32 te bijahu zaneseni njegovim naukom jer silna bijaše riječ  njegova. 
\par 33 A zatekao se u sinagogi čovjek s duhom nečistoga đavla.  On povika u sav glas: 
\par 34 "Hej, što ti imaš s nama, Isuse Nazarećanine?  Došao si da nas uništiš! Znam ja tko si ti: Svetac Božji." 
\par 35 Isus  mu zaprijeti: "Umukni i iziđi iz njega!" Nato đavao čovjeka obori  u sredinu te iziđe iz njega ne naudiv mu ništa. 
\par 36 I nasta opće  zaprepaštenje te se među sobom razgovarahu: "Kakve li riječi!  S vlašću i snagom zapovijeda nečistim dusima te izlaze!" 
\par 37 I  glas se o njemu širio po svim okolnim mjestima. 
\par 38 Ustavši iz sinagoge, uđe u kuću Šimunovu. A Šimunovu  je punicu mučila velika ognjica. I zamole ga za nju. 
\par 39 On se  nadvi nad nju, zaprijeti ognjici i ona je pusti. I odmah ustade  i posluživaše im. 
\par 40 O zalazu sunca svi koji su imali bolesnike od raznih  bolesti dovedoše ih k njemu. A on bi na svakoga od njih stavljao  ruke i ozdravljao ih. 
\par 41 A iz mnogih su izlazili i zlodusi vičući:  "Ti si Sin Božji!" On im se prijetio i nije im dao govoriti jer  su znali da je on Krist. 
\par 42 Kad osvanu dan, iziđe i pođe na samotno mjesto. I mnoštvo  ga tražilo. Dođoše k njemu i zadržavahu ga da ne ode od njih. 
\par 43 A on im reče: "I drugim gradovima treba da navješćujem evanđelje  o kraljevstvu Božjem. Jer za to sam poslan." 
\par 44 I naučavaše  po sinagogama judejskim. 


\chapter{5}

\par 1 Dok se jednom oko njega gurao narod da čuje riječ Božju, stajaše  on pokraj Genezaretskog jezera. 
\par 2 Spazi dvije lađe gdje stoje  uz obalu; ribari bili izašli iz njih i ispirali mreže. 
\par 3 Uđe  u jednu od tih lađa; bila je Šimunova pa zamoli Šimuna da malo  otisne od kraja. Sjedne te iz lađe poučavaše mnoštvo. 
\par 4 Kada dovrši pouku, reče Šimunu: "Izvezi na pučinu i bacite  mreže za lov." 
\par 5 Odgovori Šimun: "Učitelju, svu smo se noć trudili  i ništa ne ulovismo, ali na tvoju riječ bacit ću mreže." 
\par 6 Učiniše  tako te uhvatiše veoma mnogo riba; mreže im se gotovo razdirale. 
\par 7 Mahnuše drugovima na drugoj lađi da im dođu pomoći. Oni dođoše  i napuniše obje lađe, umalo im ne potonuše. 
\par 8 Vidjevši to, Šimun Petar pade do nogu Isusovih govoreći:  "Idi od mene! Grešan sam čovjek, Gospodine!" 
\par 9 Zbog lovine riba  što ih uloviše bijaše se zapanjio on i svi koji bijahu s njime, 
\par 10 a tako i Jakov i Ivan, Zebedejevi sinovi, drugovi Šimunovi.  Isus reče Šimunu: "Ne boj se! Odsada ćeš loviti ljude!" 
\par 11 Oni  izvukoše lađe na kopno, ostaviše sve i pođoše za njim. 
\par 12 I dok bijaše u jednom gradu, gle čovjeka puna gube! Ugledavši  Isusa, padne ničice i zamoli ga: "Gospodine, ako hoćeš, možeš  me očistiti." 
\par 13 Isus pruži ruku i dotakne ga se govoreći: "Hoću, očisti se!" I odmah nesta gube s njega. 
\par 14 I zapovjedi mu:  "Nikome ni riječi, nego otiđi, pokaži se svećeniku i prinesi  za svoje očišćenje kako propisa Mojsije, njima za svjedočanstvo." 
\par 15 Glas se o njemu sve više širio i silan svijet grnuo k  njemu da ga sluša i da ozdravi od svojih bolesti. 
\par 16 A on se  sklanjao na samotna mjesta da moli. 
\par 17 I jednog je dana on naučavao. A sjeđahu ondje farizeji  i učitelji Zakona koji bijahu došli iz svih galilejskih i judejskih  sela i Jeruzalema. A sila ga je Gospodnja nukala da liječi. 
\par 18 I gle, ljudi doniješe na nosiljci čovjeka koji bijaše  uzet. Tražili su da ga unesu i stave preda nj. 
\par 19 Budući da  zbog mnoštva nisu našli kuda bi ga unijeli, popnu se na krov  te ga između crepova s nosiljkom spuste u sredinu pred Isusa. 
\par 20 Vidjevši njihovu vjeru reče on: "Čovječe, otpušteni su ti  grijesi!" 
\par 21 Pismoznanci i farizeji počeše mudrovati: "Tko je  ovaj što huli? Tko može grijehe otpuštati doli Bog jedini?" 
\par 22 Proniknuvši  njihovo mudrovanje, upita ih Isus: "Što mudrujete u sebi? 
\par 23 Što  je lakše? Reći: 'Otpušteni su ti grijesi' ili reći: 'Ustani i  hodi?' 
\par 24 Ali da znate: Vlastan je Sin Čovječji na zemlji otpuštati  grijehe!" I reče uzetomu: "Tebi zapovijedam: ustani, uzmi nosiljku  i idi kući!" 
\par 25 I on odmah usta pred njima, uze na čemu ležaše  i ode kući slaveći Boga. 
\par 26 A sve obuze zanos te su slavili Boga i puni straha govorili:  "Danas vidjesmo nešto neviđeno!" 
\par 27 Nakon toga iziđe i ugleda carinika imenom Levija gdje  sjedi u carinarnici. I reče mu: "Pođi za mnom!" 
\par 28 On sve ostavi, usta i pođe za njim. 
\par 29 I Levi mu u svojoj kući priredi veliku gozbu. A s njime  bijaše za stolom veliko mnoštvo carinika i drugih. 
\par 30 Farizeji  i pismoznanci njihovi negodovahu i govorahu njegovim učenicima:  "Zašto s carinicima i grešnicima jedete i pijete?" 
\par 31 Isus im  odgovori: "Ne treba zdravima liječnika, nego bolesnima. 
\par 32 Nisam  došao zvati pravedne, nego grešnike na obraćenje." 
\par 33 A oni mu rekoše: "Učenici Ivanovi, a tako i farizejski, počesto poste i obavljaju molitve, tvoji pak jedu i piju." 
\par 34 Reče  im Isus: "Ne možete svatove prisiliti da poste dok je zaručnik  s njima. 
\par 35 Doći će već dani: kad im se ugrabi zaručnik, tada  će postiti, u one dane!" 
\par 36 A kazivao im je i prispodobu: "Nitko neće otparati krpe  s novog odijela da je stavi na staro odijelo. Inače će i novo  rasparati, a starom neće pristajati krpa s novoga." 
\par 37 "I nitko ne ulijeva novo vino u stare mješine. Inače  će novo vino proderati mješine pa će se i ono proliti i mješine  će propasti. 
\par 38 Nego, novo vino neka se ulijeva u nove mješine!" 
\par 39 "I nitko pijuć staro, ne zaželi novoga. Ta veli se: 'Valja  staro!'" 


\chapter{6}

\par 1 Jedne je subote prolazio kroz usjeve. Učenici su njegovi trgali  klasje, trli ga rukama i jeli. 
\par 2 A neki farizeji rekoše: "Zašto  činite što subotom nije dopušteno?" 
\par 3 Odgovori im Isus: "Zar  niste čitali što učini David kad ogladnje on i njegovi pratioci? 
\par 4 Kako uđe u Dom Božji, uze, pojede i svojim pratiocima dade  prinesene kruhove kojih ne smije jesti nitko, nego samo svećenici?" 
\par 5 I govoraše im: "Sin Čovječji gospodar je subote!" 
\par 6 Druge subote uđe u sinagogu i stane naučavati. Bio je  ondje čovjek kome desnica bijaše usahla. 
\par 7 Pismoznanci i farizeji  vrebahu na nj da li subotom liječi kako bi našli u čemu da ga  optuže. 
\par 8 A on je znao njihove namjere pa reče čovjeku s usahlom  rukom: "Ustani i stani na sredinu!" On usta i stade. 
\par 9 A Isus  im reče: "Pitam ja vas: je li subotom dopušteno činiti dobro  ili činiti zlo? Život spasiti ili upropastiti?" 
\par 10 Sve ih ošinu  pogledom pa reče čovjeku: "Ispruži ruku!" On učini tako - i ruka  mu zdrava. 
\par 11 A oni se, izbezumljeni, počnu dogovarati što da  poduzmu protiv Isusa. 
\par 12 Onih dana iziđe na goru da se pomoli. I provede noć moleći  se Bogu. 
\par 13 Kad se razdanilo, dozva k sebi učenike te između  njih izabra dvanaestoricu, koje prozva apostolima: 
\par 14 Šimuna, koga prozva Petrom, i Andriju, brata njegova, i Jakova, i Ivana, i Filipa, i Bartolomeja, 
\par 15 i Mateja, i Tomu, i Jakova Alfejeva, i Šimuna zvanoga Revnitelj, 
\par 16 i Judu Jakovljeva, i Judu Iškariotskoga, koji posta izdajica. 
\par 17 Isus siđe s njima i zaustavi se na ravnu. I silno mnoštvo  njegovih učenika i silno mnoštvo naroda iz cijele Judeje i Jeruzalema, iz primorja tirskog i sidonskog 
\par 18 nagrnuše da ga slušaju i  da ozdrave od svojih bolesti. I ozdravljali su oni koje su mučili  nečisti dusi. 
\par 19 Sve je to mnoštvo tražilo da ga se dotakne  jer je snaga izlazila iz njega i sve ozdravljala. 
\par 20 On podigne oči prema učenicima i govoraše: "Blago vama, siromasi: vaše je kraljevstvo Božje! 
\par 21 Blago vama koji sada gladujete: vi ćete se nasititi! Blago vama koji sada plačete: vi ćete se smijati! 
\par 22 Blago vama kad vas zamrze ljudi i kad vas izopće i pogrde te izbace ime vaše kao zločinačko zbog Sina Čovječjega! 
\par 23 Radujte se u dan onaj i poskakujte: evo, plaća vaša velika je na nebu. Ta jednako su činili prorocima oci njihovi!" 
\par 24 "Ali jao vama, bogataši: imate svoju utjehu! 
\par 25 Jao vama koji ste sada siti: gladovat ćete! Jao vama koji se sada smijete: jadikovat ćete i plakati! 
\par 26 Jao vama kad vas svi budu hvalili! Ta tako su činili lažnim prorocima oci njihovi." 
\par 27 "Nego, velim vama koji slušate: Ljubite svoje neprijatelje, dobro činite svojim mrziteljima, 
\par 28 blagoslivljajte one koji vas proklinju, molite za one koji vas zlostavljaju." 
\par 29 "Onomu tko te udari po jednom obrazu pruži i drugi, i  onomu tko ti otima gornju haljinu ne krati ni donje. 
\par 30 Svakomu  tko od tebe ište daji, a od onoga tko tvoje otima ne potražuj." 
\par 31 "I kako želite da ljudi vama čine, tako činite i vi njima." 
\par 32 "Ako ljubite one koji vas ljube, kakvo li vam uzdarje?  Ta i grešnici ljube ljubitelje svoje. 
\par 33 Jednako tako, ako dobro  činite svojim dobročiniteljima, kakvo li vam uzdarje? I grešnici  to isto čine. 
\par 34 Ako pozajmljujete samo onima od kojih se nadate  dobiti, kakvo li vam uzdarje? I grešnici grešnicima pozajmljuju  da im se jednako vrati." 
\par 35 "Nego, ljubite neprijatelje svoje. Činite dobro i pozajmljujte  ne nadajuć se odatle ničemu. I bit će vam plaća velika, i bit  ćete sinovi Svevišnjega jer je on dobrostiv i prema nezahvalnicima  i prema opakima." 
\par 36 "Budite milosrdni kao što je Otac vaš milosrdan." 
\par 37 "Ne sudite i nećete biti suđeni. Ne osuđujte i nećete  biti osuđeni. Praštajte i oprostit će vam se. 
\par 38 Dajite i dat  će vam se: mjera dobra, nabijena, natresena, preobilna dat će  se u krilo vaše jer mjerom kojom mjerite vama će se zauzvrat  mjeriti." 
\par 39 A kaza im i prispodobu: "Može li slijepac slijepca voditi?  Neće li obojica u jamu upasti? 
\par 40 Nije učenik nad učiteljem.  Pa i tko je posve doučen, bit će samo kao njegov učitelj." 
\par 41 "Što gledaš trun u oku brata svojega, a brvna u oku svome  ne opažaš? 
\par 42 Kako možeš kazati bratu svomu: 'Brate, de da izvadim  trun koji ti je u oku', a sam u svom oku brvna ne vidiš? Licemjere!  Izvadi najprije brvno iz oka svoga pa ćeš onda dobro vidjeti  izvaditi trun što je u oku bratovu." 
\par 43 "Nema dobra stabla koje bi rađalo nevaljalim plodom niti  stabla nevaljala koje bi rađalo dobrim plodom. 
\par 44 Ta svako se  stablo po svom plodu poznaje. S trnja se ne beru smokve niti  se s gloga grožđe trga." 
\par 45 "Dobar čovjek iz dobra blaga srca svojega iznosi dobro, a zao iz zla iznosi zlo. Ta iz obilja srca usta mu govore." 
\par 46 "Što me zovete 'Gospodine, Gospodine!', a ne činite što  zapovijedam? 
\par 47 Tko god dolazi k meni te sluša moje riječi i  vrši ih, pokazat ću vam kome je sličan: 
\par 48 sličan je čovjeku  koji gradi kuću pa iskopa u dubinu i postavi temelj na kamen.  A kad bude poplava, nahrupi bujica na tu kuću, ali je ne može  uzdrmati jer je dobro sagrađena. 
\par 49 A koji čuje i ne izvrši, sličan je čovjeku koji sagradi kuću na tlu bez temelja; nahrupi  na nju bujica i umah se sruši te bude od te kuće razvalina velika." 


\chapter{7}

\par 1 Pošto dovrši sve te svoje besjede narodu, uđe u Kafarnaum. 
\par 2 Nekomu satniku bijaše bolestan sluga, samo što ne izdahnu, a bijaše mu veoma drag. 
\par 3 Kad je satnik čuo za Isusa, posla  k njemu starješine židovske moleći ga da dođe i ozdravi mu slugu. 
\par 4 Kad oni dođoše Isusu, usrdno ga moljahu: "Dostojan je da mu  to učiniš 
\par 5 jer voli naš narod, i sinagogu nam je sagradio." 
\par 6 Isus se uputi s njima. I kad bijaše već kući nadomak, posla  satnik prijatelje s porukom: "Gospodine, ne muči se. Nisam dostojan  da uđeš pod krov moj. 
\par 7 Zato se i ne smatrah dostojnim doći  k tebi. Nego - reci riječ da ozdravi sluga moj. 
\par 8 Ta i ja, premda  sam vlasti podređen, imam pod sobom vojnike pa reknem jednomu:  'Idi' - i ode, drugomu: 'Dođi' - i dođe, a sluzi svomu: 'Učini  to' - i učini." 
\par 9 Čuvši to, zadivi mu se Isus pa se okrenu mnoštvu koje  je išlo za njim i reče: "Kažem vam, ni u Izraelu na nađoh tolike  vjere." 
\par 10 Kad se oni koji su bili poslani vratiše kući, nađoše  slugu zdrava. 
\par 11 Nakon toga uputi se Isus u grad zvani Nain. Pratili ga  njegovi učenici i silan svijet. 
\par 12 Kad se približi gradskim  vratima, gle, upravo su iznosili mrtvaca, sina jedinca u majke, majke udovice. Pratilo ju mnogo naroda iz grada. 
\par 13 Kad je  Gospodin ugleda, sažali mu se nad njom i reče joj: "Ne plači!" 
\par 14 Pristupi zatim, dotače se nosila; nosioci stadoše, a on reče:  "Mladiću, kažem ti, ustani!" 
\par 15 I mrtvac se podiže i progovori, a on ga dade njegovoj majci. 
\par 16 Sve obuze strah te slavljahu  Boga govoreći: "Prorok velik usta među nama! Pohodi Bog narod  svoj!" 
\par 17 I proširi se taj glas o njemu po svoj Judeji i po  svoj okolici. 
\par 18 Sve to dojaviše Ivanu njegovi učenici. On dozva dvojicu  svojih učenika 
\par 19 i posla ih Gospodinu da ga pitaju: "Jesi li  ti Onaj koji ima doći ili drugoga da čekamo?" 
\par 20 Došavši k njemu, rekoše ti ljudi: "Ivan Krstitelj posla nas k tebi da pitamo:  'Jesi li ti Onaj koji ima doći ili drugoga da čekamo?'" 
\par 21 Upravo u taj čas Isus je ozdravio mnoge od bolesti, muka  i zlih duhova i mnoge je slijepe podario vidom. 
\par 22 Tada im odgovori:  "Pođite i javite Ivanu što ste vidjeli i čuli: Slijepi progledaju,  hromi hode, gubavi se čiste, gluhi čuju, mrtvi ustaju, siromasima  se navješćuje evanđelje. 
\par 23 I blago onom tko se ne sablazni  o mene." 
\par 24 Kad Ivanovi glasnici odoše, poče Isus govoriti mnoštvu  o Ivanu: "Što ste izašli u pustinju gledati? Trsku koju vjetar  ljulja? 
\par 25 Ili što ste izašli vidjeti: Čovjeka u mekušaste haljine  odjevena? Eno, oni u sjajnoj odjeći i raskošju po kraljevskim  su dvorima. 
\par 26 Ili što ste izašli vidjeti? Proroka? Uistinu, kažem vam, i više nego proroka! 
\par 27 On je onaj o kome je pisano: Evo, šaljem glasnika svoga pred licem tvojim da pripravi put pred tobom. 
\par 28 Kažem vam: među rođenima od žene nitko nije veći od Ivana.  A ipak, i najmanji u kraljevstvu Božjem veći je od njega." 
\par 29 Sav narod koji ga je slušao, pa i carinici, uvidješe  pravednost Božju: pokrstiše se Ivanovim krstom. 
\par 30 Naprotiv, farizeji i zakonoznanci ometoše što je Bog s njima naumio jer  ne htjedoše da ih Ivan krsti. 
\par 31 "S kime dakle da prispodobim ljude ovog naraštaja? Komu  su nalik? 
\par 32 Nalik su djeci što sjede na trgu pa jedni drugima  po poslovici dovikuju: 'Zasvirasmo vam i ne zaigraste! Zakukasmo  i ne zaplakaste!' 
\par 33 Doista, došao je Ivan Krstitelj. Nije kruha  jeo ni vina pio, a velite: 'Ðavla ima!' 
\par 34 Došao je Sin Čovječji  koji jede i pije, a govorite: 'Evo izjelice i vinopije, prijatelja  carinika i grešnika!' 
\par 35 Ali opravda se Mudrost pred svom djecom  svojom." 
\par 36 Neki farizej pozva Isusa da bi blagovao s njime. On uđe  u kuću farizejevu i priđe stolu. 
\par 37 Kad eto neke žene koja bijaše  grešnica u gradu. Dozna da je Isus za stolom u farizejevoj kući  pa ponese alabastrenu posudicu pomasti 
\par 38 i stade odostrag kod  njegovih nogu. Sva zaplakana poče mu suzama kvasiti noge: kosom  ih glave svoje otirala, cjelivala i mazala pomašću. 
\par 39 Kad to vidje farizej koji ga pozva, pomisli: "Kad bi  ovaj bio Prorok, znao bi tko i kakva je to žena koja ga se dotiče:  da je grešnica." 
\par 40 A Isus, da mu odgovori, reče: "Šimune, imam  ti nešto reći." A on će: "Učitelju, reci!" A on: 
\par 41 "Neki vjerovnik  imao dva dužnika. Jedan mu dugovaše pet stotina denara, drugi  pedeset. 
\par 42 Budući da nisu imali odakle vratiti, otpusti obojici.  Koji će ga dakle od njih više ljubiti?" 
\par 43 Šimun odgovori: "Predmnijevam, onaj kojemu je više otpustio." Reče mu Isus: "Pravo si prosudio." 
\par 44 I okrenut ženi reče Šimunu: "Vidiš li ovu ženu? Uđoh ti u  kuću, nisi mi vodom noge polio, a ona mi suzama noge oblila i  kosom ih svojom otrla. 
\par 45 Poljupca mi nisi dao, a ona, otkako  uđe, ne presta mi noge cjelivati. 
\par 46 Uljem mi glave nisi pomazao, a ona mi pomašću noge pomaza. 
\par 47 Stoga, kažem ti, oprošteni  su joj grijesi mnogi jer ljubljaše mnogo. Komu se malo oprašta, malo ljubi." 
\par 48 A ženi reče: "Oprošteni su ti grijesi." 
\par 49 Sustolnici  počeli nato među sobom govoriti: "Tko je ovaj da i grijehe oprašta?" 
\par 50 A on reče ženi: "Vjera te tvoja spasila! Idi u miru!" 


\chapter{8}

\par 1 Zatim zareda obilaziti gradom i selom propovijedajući i navješćujući  evanđelje o kraljevstvu Božjemu. Bila su s njim dvanaestorica 
\par 2 i neke žene koje bijahu izliječene od zlih duhova i bolesti:  Marija zvana Magdalena, iz koje bijaše izagnao sedam đavola; 
\par 3 zatim Ivana, žena Herodova upravitelja Huze; Suzana i mnoge  druge. One su im posluživale od svojih dobara. 
\par 4 Kad se skupio silan svijet te iz svakoga grada nagrnuše  k njemu, prozbori u prispodobi: 
\par 5 "Iziđe sijač sijati sjeme. Dok je sijao, jedno pade uz  put, bi pogaženo i ptice ga nebeske pozobaše. 
\par 6 Drugo pade na  kamen i, tek što je izniklo, osuši se jer ne imaše vlage. 
\par 7 Drugo  opet pade među trnje i trnje ga preraste i uguši. 
\par 8 Drugo napokon  pade u dobru zemlju, nikne i urodi stostrukim plodom." Rekavši  to, povika: "Tko ima uši da čuje, neka čuje!" 
\par 9 Upitaše ga učenici kakva bi to bila prispodoba. 
\par 10 A  on im reče: "Vama je dano znati otajstva kraljevstva Božjega, a ostalima u prispodobama - da gledajući ne vide i slušajući ne razumiju." 
\par 11 "A ovo je prispodoba: Sjeme je Riječ Božja. 
\par 12 Oni uz  put slušatelji su. Zatim dolazi đavao i odnosi Riječ iz srca  njihova da ne bi povjerovali i spasili se. 
\par 13 A na kamenu -  to su oni koji kad čuju, s radošću prime Riječ, ali korijena  nemaju: ti neko vrijeme vjeruju, a u vrijeme kušnje otpadnu. 
\par 14 A što pade u trnje - to su oni koji poslušaju, ali poneseni  brigama, bogatstvom i nasladama života, uguše se i ne dorode  roda. 
\par 15 Ono pak u dobroj zemlji - to su oni koji u plemenitu  i dobru srcu slušaju Riječ, zadrže je i donose rod u ustrajnosti." 
\par 16 "Nitko ne užiže svjetiljke da je pokrije posudom ili  stavi pod postelju, nego je stavlja na svijećnjak da oni koji  ulaze vide svjetlost. 
\par 17 Ta ništa nije tajno što se neće očitovati;  ništa skriveno što se neće saznati i na vidjelo doći." 
\par 18 "Pazite dakle kako slušate. Doista, onomu tko ima dat  će se, a onomu tko nema oduzet će se i ono što misli da ima." 
\par 19 A majka i braća njegova htjedoše k njemu, ali ne mogoše  do njega zbog mnoštva. 
\par 20 Javiše mu: "Majka tvoja i braća tvoja  stoje vani i žele te vidjeti." 
\par 21 A on im odgovori: "Majka moja, braća moja - ovi su koji riječ Božju slušaju i vrše." 
\par 22 Jednoga dana uđe u lađu on i učenici njegovi. I reče  im: "Prijeđimo na onu stranu jezera." I otisnuše se. 
\par 23 Dok  su plovili, on zaspa. I spusti se oluja na jezero. Voda stane  nadirati te bijahu u pogibli. 
\par 24 Oni pristupiše i probudiše  ga govoreći: "Učitelju, učitelju, propadosmo!" On se probudi, zaprijeti vjetru i valovlju; i oni se smire te nasta utiha. 
\par 25 A on će im: "Gdje vam je vjera?" A oni se prestrašeni u čudu  zapitkivahu: "Tko li je ovaj da i vjetrovima zapovijeda i vodi, i pokoravaju mu se?" 
\par 26 Doploviše u gergezenski kraj koji je nasuprot Galileji. 
\par 27 Čim iziđe na kopno, eto mu iz grada u susret nekog čovjeka  koji imaše zloduhe. Već dugo vremena nije se uopće odijevao niti  stanovao u kući, nego po grobnicama. 
\par 28 Kad opazi Isusa, zastenja, pade ničice preda nj i u sav glas povika: "Što ti imaš sa  mnom, Isuse, Sine Boga Svevišnjega? Molim te, ne muči me!" 
\par 29 Jer bijaše zapovjedio nečistom duhu da iziđe iz toga čovjeka.  Da, dugo ga je već vremena držao u vlasti i makar su ga lancima  vezali i u verigama čuvali, on bi raskidao spone i zloduh bi  ga odagnao u pustinju. 
\par 30 Isus ga nato upita: "Kako ti je ime?"  On reče: "Legija", jer u nj uđoše mnogi zlodusi. 
\par 31 I zaklinjahu  ga da im ne naredi vratiti se u Bezdan. 
\par 32 A ondje u gori paslo je poveliko krdo svinja. Zaklinjahu  ga dakle da im dopusti ući u njih. I on im dopusti. 
\par 33 Tada  zlodusi iziđoše iz čovjeka i uđoše u svinje. Krdo jurnu niz obronak  u jezero i podavi se. 
\par 34 Vidjevši što se dogodilo, svinjari pobjegoše i razglasiše  gradom i selima. 
\par 35 A ljudi iziđoše vidjeti što se dogodilo.  Dođoše Isusu i nađoše čovjeka iz kojega bijahu izašli zlodusi  gdje do nogu Isusovih sjedi, obučen i zdrave pameti. I prestraše  se. 
\par 36 A očevici im ispripovijediše kako je opsjednuti ozdravio. 
\par 37 I zamoli ga sve ono mnoštvo iz okolice gergezenske da  ode od njih jer ih strah velik spopade. On uđe u lađu i vrati se. 
\par 38 A moljaše ga čovjek iz koga iziđoše zlodusi da može ostati  s njim, ali ga on otpusti govoreći: 
\par 39 "Vrati se kući i pripovijedaj  što ti učini Bog." On ode razglašujući po svem gradu što mu učini  Isus. 
\par 40 Na povratku Isusa dočeka mnoštvo jer su ga svi željno  iščekivali. 
\par 41 I gle, dođe čovjek, ime mu Jair, koji bijaše  predstojnik sinagoge. Baci se Isusu pred noge i stane ga moliti  da dođe u njegovu kuću. 
\par 42 Imaše kćer jedinicu, otprilike od  dvanaest godina, koja umiraše. Dok je onamo išao, mnoštvo ga  guralo odasvud. 
\par 43 A neka žena koja je već dvanaest godina bolovala od krvarenja, sve svoje imanje potrošila na liječnike i nitko je nije mogao  izliječiti, 
\par 44 priđe odostrag i dotaknu se skuta njegove haljine  i umah joj se zaustavi krvarenje. 
\par 45 I reče Isus: "Tko me se to dotaknu?" Svi se branili,  a Petar će: "Učitelju, mnoštvo te gura i pritišće." 
\par 46 A Isus:  "Netko me se dotaknuo. Osjetio sam kako snaga izlazi iz mene." 
\par 47 A žena, vidjevši da se ne može kriti, sva u strahu pristupi  i baci se preda nj te pred svim narodom ispripovjedi zašto ga  se dotakla i kako je umah ozdravila. 
\par 48 A on joj reče: "Kćeri, vjera te tvoja spasila. Idi u miru!" 
\par 49 Dok je on još govorio, eto jednog od nadstojnikovih s  porukom: "Umrla ti kći, ne muči više Učitelja." 
\par 50 Čuo to Isus  pa mu reče: "Ne boj se! Samo vjeruj i ona će se spasiti!" 
\par 51 Uđe u kuću, ali nikomu ne dopusti da s njim uđe osim  Petra, Ivana, Jakova i djetetova oca i majke. 
\par 52 A svi plakahu  i žalovahu za njom. A on im reče: "Ne plačite! Nije umrla, nego  spava!" 
\par 53 No oni mu se podsmjehivahu znajući da je umrla. 
\par 54 On  je uhvati za ruku i povika: "Dijete, ustani!" 
\par 55 I povrati joj  se duh i umah ustade, a on naredi da joj dadu jesti. 
\par 56 Njezini  se roditelji začudiše, a on zapovjedi da nikome ne reknu što  se dogodilo. 


\chapter{9}

\par 1 Sazva dvanaestoricu i dade im moć i vlast nad svim zlodusima  i da liječe bolesti. 
\par 2 I posla ih propovijedati kraljevstvo  Božje i liječiti bolesnike. 
\par 3 I reče im: "Ništa ne uzimajte  na put: ni štapa, ni torbe, ni kruha, ni srebra! I da niste imali  više od dvije haljine! 
\par 4 U koju god kuću uđete, ondje ostanite  pa odande dalje pođite. 
\par 5 Gdje vas ne prime, iziđite iz toga  grada i stresite prašinu s nogu za svjedočanstvo protiv njih." 
\par 6 Oni krenuše: obilazili su po selima, navješćivali evanđelje  i liječili posvuda. 
\par 7 Dočuo Herod tetrarh sve što se događa te se nađe u nedoumici  jer su neki govorili: "Ivan uskrsnu od mrtvih"; 
\par 8 drugi: "Pojavio  se Ilija"; treći opet: "Ustao je neki od drevnih proroka." 
\par 9 A  Herod reče: "Ivanu ja odrubih glavu. Tko je onda ovaj o kom toliko  čujem?" I tražio je priliku da ga vidi. 
\par 10 Apostoli se vrate i ispripovjede što su učinili. Isus  ih povede sa sobom i povuče se nasamo u grad zvani Betsaida. 
\par 11 Saznalo to mnoštvo po pođe za njim. On ih primi te im govoraše  o kraljevstvu Božjem i ozdravljaše sve koji su trebali ozdravljenja. 
\par 12 Dan bijaše na izmaku. Pristupe dakle dvanaestorica pa  mu reknu: "Otpusti svijet, neka pođu po okolnim selima i zaseocima  da se sklone i nađu jela jer smo ovdje u pustu kraju." 
\par 13 A  on im reče: "Podajte im vi jesti!" Oni rekoše: "Nemamo više od  pet kruhova i dvije ribe, osim da odemo kupiti hrane za sav ovaj  narod." 
\par 14 A bijaše oko pet tisuća muškaraca. Nato će on svojim učenicima: "Posjedajte ih po skupinama, otprilike  po pedeset." 
\par 15 I učine tako: sve ih posjedaju. 
\par 16 A on uze  pet kruhova i dvije ribe, pogleda na nebo, blagoslovi ih i razlomi  pa davaše učenicima da posluže mnoštvo. 
\par 17 Jeli su i svi se nasitili. I od preteklih ulomaka nakupilo  se dvanaest košara. 
\par 18 Dok je jednom u osami molio, bijahu s njim samo njegovi  učenici. On ih upita: "Što govori svijet, tko sam ja?" 
\par 19 Oni  odgovoriše: "Da si Ivan Krstitelj, drugi: da si Ilija, treći  opet: da neki od drevnih proroka usta." 
\par 20 A on im reče: "A  vi, što vi kažete, tko sam ja?" Petar prihvati i reče: "Krist  - Pomazanik Božji!" 
\par 21 A on im zaprijeti da toga nikomu ne kazuju. 
\par 22 Reče: "Treba da Sin Čovječji mnogo pretrpi, da ga starješine, glavari svećenički i pismoznanci odbace, da bude ubijen i treći  dan da uskrsne." 
\par 23 A govoraše svima: "Hoće li tko za mnom, neka se odrekne  samoga sebe, neka danomice uzima križ svoj i neka ide za mnom. 
\par 24 Tko hoće život svoj spasiti, izgubit će ga; a tko izgubi  život svoj poradi mene, taj će ga spasiti. 
\par 25 Ta što koristi  čovjeku ako sav svijet zadobije, a sebe samoga izgubi ili sebi  naudi?" 
\par 26 "Doista, tko se zastidi mene i mojih riječi, toga će  se i Sin Čovječji stidjeti kada dođe u slavi svojoj i Očevoj  i svetih anđela." 
\par 27 "A kažem vam uistinu: neki od nazočnih neće okusiti smrti  dok ne vide kraljevstva Božjega." 
\par 28 Jedno osam dana nakon tih besjeda povede Isus sa sobom  Petra, Ivana i Jakova te uziđe na goru da se pomoli. 
\par 29 I dok  se molio, izgled mu se lica izmijeni, a odjeća sjajem zablista. 
\par 30 I gle, dva čovjeka razgovarahu s njime. Bijahu to Mojsije  i Ilija. 
\par 31 Ukazali se u slavi i razgovarali s njime o njegovu  Izlasku, što se doskora imao ispuniti u Jeruzalemu. 
\par 32 No Petra  i njegove drugove bijaše svladao san. Kad se probudiše, ugledaše  njegovu slavu i dva čovjeka koji stajahu uza nj. 
\par 33 I dok su  oni odlazili od njega, reče Petar Isusu: "Učitelju, dobro nam  je ovdje biti. Načinimo tri sjenice: jednu tebi, jednu Mojsiju, jednu Iliji." Nije znao što govori. 
\par 34 Dok je on to govorio, pojavi se oblak i zasjeni ih. Ušavši  u oblak, oni se prestrašiše. 
\par 35 A glas se začu iz oblaka: "Ovo  je Sin moj, Izabranik! Njega slušajte!" 
\par 36 I upravo kad  se začu glas, osta Isus sam. Oni su šutjeli i nikomu onih dana  nisu kazivali što su vidjeli. 
\par 37 A kad su sutradan sišli s gore, pohiti mu u susret silan  svijet. 
\par 38 I gle, čovjek neki iz mnoštva povika: "Učitelju,  molim te pogledaj mi sina: jedinac mi je, 
\par 39 a gle, duh ga spopada  te on odmah udari u kriku; trza njime i on se pjeni te jedva  da od njega odstupi dok ga nije posve satro. 
\par 40 Molio sam tvoje  učenike da ga izagnaju, ali ne mogoše." 
\par 41 Isus odvrati: "O rode nevjerni i opaki, dokle mi je biti  s vama i podnositi vas? Dovedi ovamo svoga sina!" 
\par 42 I dok je  prilazio, obori ga zloduh i potrese. A Isus zaprijeti nečistom  duhu te izliječi dječaka i preda ga njegovu ocu. 
\par 43 Svi se zapanjiše  zbog veličanstva Božjega. Dok su se svi divili svemu što je činio, reče on učenicima: 
\par 44 "Uzmite k srcu ove riječi: Sin Čovječji doista ima biti predan  ljudima u ruke." 
\par 45 Ali oni nerazumješe te besjede, bijaše im  skrivena te ne shvatiše, a bojahu se upitati ga o tome. 
\par 46 U njima se porodi misao tko bi od njih bio najveći. 
\par 47 Znajući  tu misao njihova srca, uzme Isus dijete, postavi ga uza se 
\par 48 i  reče im: "Tko god primi ovo dijete u moje ime, mene prima. A  tko mene prima, prima onoga koji me je poslao. Doista, tko je  najmanji među vama svima, taj je velik!" 
\par 49 Prihvati Ivan i reče: "Učitelju, vidjesmo jednoga koji  u tvoje ime izgoni zloduhe. Mi smo mu branili, jer ne ide za  nama." 
\par 50 Reče mu Isus: "Ne branite! Ta tko nije protiv vas, za vas je!" 
\par 51 Kad su se navršili dani da bude uznesen, krenu Isus sa  svom odlučnošću prema Jeruzalemu. 
\par 52 I posla glasnike pred sobom.  Oni odoše i uđoše u neko samarijansko selo da mu priprave mjesto. 
\par 53 No ondje ga ne primiše jer je bio na putu u Jeruzalem. 
\par 54 Kada  to vidješe učenici Jakov i Ivan, rekoše: "Gospodine, hoćeš li  da kažemo neka oganj siđe s neba i uništi ih?" 
\par 55 No  on se okrenu i prekori ih. 
\par 56 I odoše u drugo selo. 
\par 57 Dok su išli putom, reče mu netko: "Za tobom ću kamo god  ti pošao." 
\par 58 Reče mu Isus: "Lisice imaju jazbine, ptice nebeske  gnijezda, a Sin Čovječji nema gdje bi glavu naslonio." 
\par 59 Drugomu nekom reče: "Pođi za mnom!" A on će mu: "Dopusti  mi da prije odem i pokopam oca." 
\par 60 Reče mu: "Pusti neka mrtvi  pokapaju svoje mrtve, a ti idi i navješćuj kraljevstvo Božje." 
\par 61 I neki drugi reče: "Za tobom ću, Gospodine, ali dopusti  mi da se prije oprostim sa svojim ukućanima." 
\par 62 Reče mu Isus:  "Nitko tko stavi ruku na plug pa se obazire natrag, nije prikladan  za kraljevstvo Božje." 


\chapter{10}

\par 1 Nakon toga odredi Gospodin drugih sedamdesetdvojicu učenika  i posla ih po dva pred sobom u svaki grad i u svako mjesto kamo  je kanio doći. 
\par 2 Govorio im je: "Žetva je velika, ali radnika  malo. Molite dakle gospodara žetve da radnike pošalje u žetvu  svoju. 
\par 3 Idite! Evo, šaljem vas kao janjce među vukove. 
\par 4 Ne  nosite sa sobom ni kese, ni torbe, ni obuće. I nikoga putem ne  pozdravljajte. 
\par 5 U koju god kuću uđete, najprije recite: 'Mir kući ovoj!' 
\par 6 Bude li tko ondje prijatelj mira, počinut će na njemu mir  vaš. Ako li ne, vratit će se na vas. 
\par 7 U toj kući ostanite,  jedite i pijte što se kod njih nađe. Ta vrijedan je radnik plaće  svoje. Ne prelazite iz kuće u kuću." 
\par 8 "Kad u koji grad uđete pa vas prime, jedite što vam se  ponudi 
\par 9 i liječite bolesnike koji su u njemu. I kazujte im:  'Približilo vam se kraljevstvo Božje!' 
\par 10 A kad u neki grad  uđete pa vas ne prime, iziđite na njegove ulice i recite: 
\par 11 'I  prašinu vašega grada, koja nam se nogu uhvatila, stresamo vam  sa sebe! Ipak znajte ovo: Približilo se kraljevstvo Božje!' 
\par 12 Kažem  vam: Sodomcima će u onaj dan biti lakše negoli tomu gradu." 
\par 13 "Jao tebi, Korozaine! Jao tebi, Betsaido! Da su se u  Tiru i Sidonu zbila čudesa koja su se dogodila u vama, odavna  bi već, sjedeć u kostrijeti i pepelu, činili pokoru. 
\par 14 Ali  Tiru i Sidonu bit će na Sudu lakše negoli vama. 
\par 15 I ti Kafarnaume! Zar ćeš se do neba uzvisiti? Do u Podzemlje ćeš se strovaliti. 
\par 16 Tko vas sluša, mene sluša; tko vas prezire, mene prezire.  A tko mene prezire, prezire onoga koji mene posla." 
\par 17 Vratiše se zatim sedamdesetdvojica radosni govoreći:  "Gospodine, i zlodusi nam se pokoravaju na tvoje ime!" 
\par 18 A  on im reče: "Promatrah Sotonu kako poput munje s neba pade. 
\par 19 Evo, dao sam vam vlast da gazite po zmijama i štipavcima i  po svoj sili neprijateljevoj i ništa vam neće naškoditi. 
\par 20 Ali  ne radujte se što vam se duhovi pokoravaju, nego radujte se što  su vam imena zapisana na nebesima." 
\par 21 U taj isti čas uskliknu Isus u Duhu Svetom: "Slavim te, Oče, Gospodaru neba i zemlje, što si ovo sakrio od mudrih i  umnih, a objavio malenima. Da, Oče! Tako se tebi svidjelo. 
\par 22 Sve  mi preda Otac moj i nitko ne zna tko je Sin - doli Otac; niti  tko je Otac - doli Sin i onaj kome Sin hoće da objavi." 
\par 23 Tada se okrene učenicima pa im nasamo reče: "Blago očima  koje gledaju što vi gledate! 
\par 24 Kažem vam: mnogi su proroci  i kraljevi htjeli vidjeti što vi gledate, ali nisu vidjeli; i  čuti što vi slušate, ali nisu čuli!" 
\par 25 I gle, neki zakonoznanac usta i, da ga iskuša, upita:  "Učitelju, što mi je činiti da život vječni baštinim?" 
\par 26 A  on mu reče: "U Zakonu što piše? Kako čitaš?" 
\par 27 Odgovori mu  onaj: Ljubi Gospodina Boga svojega iz svega srca svoga, i  svom dušom svojom, i svom snagom svojom, i svim umom svojim;  i svoga bližnjega kao sebe samoga!" 
\par 28 Reče mu na to  Isus: "Pravo si odgovorio. To čini i živjet ćeš." 
\par 29 Ali hoteći se opravdati, reče on Isusu: "A tko je moj  bližnji?" 
\par 30 Isus prihvati i reče: "Čovjek neki silazio iz Jeruzalema u Jerihon. Upao među razbojnike  koji ga svukoše i izraniše pa odoše ostavivši ga polumrtva. 
\par 31 Slučajno  je onim putem silazio neki svećenik, vidje ga i zaobiđe. 
\par 32 A  tako i levit: prolazeći onuda, vidje ga i zaobiđe. 
\par 33 Neki Samarijanac  putujući dođe do njega, vidje ga, sažali se 
\par 34 pa mu pristupi  i povije rane zalivši ih uljem i vinom. Zatim ga posadi na svoje  živinče, odvede ga u gostinjac i pobrinu se za nj. 
\par 35 Sutradan  izvadi dva denara, dade ih gostioničaru i reče: 'Pobrini se za  njega. Ako što više potrošiš, isplatit ću ti kad se budem vraćao.'" 
\par 36 "Što ti se čini, koji je od ove trojice bio bližnji onomu  koji je upao među razbojnike?" 
\par 37 On odgovori: "Onaj koji mu  iskaza milosrđe." Nato mu reče Isus: "Idi pa i ti čini tako!" 
\par 38 Dok su oni tako putovali, uđe on u jedno selo. Žena neka, imenom Marta, primi ga u kuću. 
\par 39 Imala je sestru koja se zvala  Marija. Ona sjede do nogu Gospodinovih i slušaše riječ njegovu. 
\par 40 A Marta bijaše sva zauzeta posluživanjem pa pristupi i reče:  "Gospodine, zar ne mariš što me sestra samu ostavila posluživati?  Reci joj dakle da mi pomogne." 
\par 41 Odgovori joj Gospodin: "Marta, Marta! Brineš se i uznemiruješ za mnogo, 
\par 42 a jedno je potrebno.  Marija je uistinu izabrala bolji dio, koji joj se neće oduzeti." 


\chapter{11}

\par 1 Jednom je Isus na nekome mjestu molio. Čim presta, reče mu  jedan od učenika: "Gospodine, nauči nas moliti kao što je i Ivan  naučio svoje učenike." 
\par 2 On im reče: "Kad molite, govorite: 'Oče! Sveti se ime tvoje! Dođi kraljevstvo tvoje! 
\par 3 Kruh naš svagdanji daji nam svaki dan! 
\par 4 I otpusti nam grijehe naše: ta i mi otpuštamo svakom dužniku svojem! I ne uvedi nas u napast!'" 
\par 5 I reče im: "Tko to od vas ima ovakva prijatelja? Pođe  k njemu o ponoći i rekne mu: 'Prijatelju, posudi mi tri kruha. 
\par 6 Prijatelj mi se s puta svratio te nemam što staviti preda  nj!' 
\par 7 A onaj mu iznutra odgovori: 'Ne dosađuj mi! Vrata su  već zatvorena, a dječica sa mnom u postelji. Ne mogu ustati da  ti dadnem...' 
\par 8 Kažem vam: ako i ne ustane da mu dadne zato  što mu je prijatelj, ustat će i dati mu što god treba zbog njegove  bezočnosti." 
\par 9 "I ja vama kažem: Ištite i dat će vam se! Tražite i naći  ćete! Kucajte i otvorit će vam se! 
\par 10 Doista, tko god ište,  prima; i tko traži, nalazi; i onomu tko kuca, otvorit će se." 
\par 11 "A koji je to otac među vama: kad ga sin zaište ribu, zar će mu mjesto ribe zmiju dati? 
\par 12 Ili kad zaište jaje, zar  će mu dati štipavca? 
\par 13 Ako dakle vi, iako zli, znate dobrim  darima darivati djecu svoju, koliko li će više Otac s neba obdariti  Duhom Svetim one koji ga zaištu!" 
\par 14 I istjerivaše đavla koji bijaše nijem. Kad iziđe đavao, progovori njemak. I mnoštvo se divilo. 
\par 15 A neki od njih rekoše:  "Po Beelzebulu, poglavici đavolskom, izgoni đavle!" 
\par 16 A drugi  su iskušavajući ga, tražili od njega kakav znak s neba. 
\par 17 Ali  on, znajući njihove misli, reče im: "Svako kraljevstvo u sebi  razdijeljeno opustjet će i kuća će na kuću pasti. 
\par 18 Ako je  dakle Sotona u sebi razdijeljen, kako će opstati kraljevstvo  njegovo? Jer vi kažete da ja po Beelzebulu izgonim đavle. 
\par 19 Ako  dakle ja po Beelzebulu izgonim đavle, po kome ih vaši sinovi  izgone? Zato će vam oni biti suci. 
\par 20 Ali ako ja prstom Božjim  izgonim đavle, zbilja je došlo k vama kraljevstvo Božje." 
\par 21 "Dokle god jaki i naoružani čuva svoj stan, u miru je  sav njegov posjed. 
\par 22 Ali ako dođe jači od njega, svlada ga  i otme mu sve njegovo oružje u koje se uzdao, a plijen razdijeli." 
\par 23 "Tko nije sa mnom, protiv mene je. I tko sa mnom ne sabire, rasipa." 
\par 24 "Kad nečisti duh iziđe iz čovjeka, luta bezvodnim mjestima  tražeći spokoja. Kad ga ne nađe, rekne: 'Vratit ću se u kuću  odakle iziđoh.' 
\par 25 Došavši, nađe je pometenu i uređenu. 
\par 26 Tada  ode i uzme sa sobom sedam drugih duhova, gorih od sebe, te uđu  i nastane se ondje. Na kraju bude onomu čovjeku gore nego na  početku." 
\par 27 Dok je on to govorio, povika neka žena iz mnoštva: "Blažena  utroba koja te nosila i prsi koje si sisao!" 
\par 28 On odgovori:  "Još blaženiji oni koji slušaju riječ Božju i čuvaju je!" 
\par 29 Kad je nagrnulo mnoštvo, poče im Isus govoriti: "Naraštaj  ovaj naraštaj je opak. Znak traži, ali mu se znak neće dati doli  znak Jonin. 
\par 30 Doista, kao što je Jona bio znak Ninivljanima, tako će biti i Sin Čovječji ovomu naraštaju." 
\par 31 "Kraljica će Juga ustati na Sudu s ljudima ovog naraštaja  i osuditi ih jer je s krajeva zemlje došla čuti mudrost Salomonovu, a evo ovdje i više od Salomona! 
\par 32 Ninivljani će ustati na  Sudu s ovim naraštajem i osuditi ga jer se obratiše na propovijed  Joninu, a evo ovdje i više od Jone!" 
\par 33 "Nitko ne užiže svjetiljku da je stavi u zakutak ili  pod posudu, nego na svijećnjak da oni koji ulaze vide svjetlost. 
\par 34 Oko je svjetiljka tvomu tijelu. Kad ti je oko bistro, sve  ti je tijelo svijetlo. A kad je ono nevaljalo, i tijelo ti je  tamno. 
\par 35 Pazi dakle da svjetlost koja je u tebi ne bude tamna. 
\par 36 Ako ti dakle sve tijelo bude svijetlo, bez djelića tame,  bit će posve svijetlo, kao kad te svjetiljka svojim sjajem rasvjetljuje." 
\par 37 Dok je on govorio, pozva ga neki farizej k sebi na objed.  On uđe i priđe k stolu. 
\par 38 Vidjevši to, farizej se začudi što  se Isus prije objeda ne opra. 
\par 39 A Gospodin mu reče: "Da, vi  farizeji čistite vanjštinu čaše u zdjele, a nutrina vam je puna  grabeža i pakosti. 
\par 40 Bezumnici! Nije li onaj koji načini vanjštinu  načinio i nutrinu. 
\par 41 Nego, dajte za milostinju ono iznutra  i gle - sve vam je čisto." 
\par 42 "Ali jao vama, farizeji! Namirujete desetinu od metvice  i rutvice i svake vrste povrća, a ne marite za pravednost i ljubav  Božju. Ovo je trebalo činiti, a ono ne zanemariti." 
\par 43 "Jao vama farizeji! Volite prvo sjedalo u sinagogama  i pozdrave na trgovima. 
\par 44 Jao vama! Vi ste kao nezamjetljivi  grobovi po kojima ljudi ne znajući hode." 
\par 45 Nato će neki zakonoznanac: "Učitelju, tako govoreći i  nas vrijeđaš." 
\par 46 A on reče: "Jao i vama, zakonoznanci! Tovarite  na ljude terete nepodnosive, a sami ni da ih se jednim prstom  dotaknete." 
\par 47 "Jao vama! Podižete spomenike prorocima, a vaši ih oci  ubiše. 
\par 48 Zato ste svjedoci i sumišljenici djela svojih otaca:  oni ih ubiše, a vi spomenike podižete! 
\par 49 Zbog toga i kaza Mudrost  Božja: 'Poslat ću k njima proroke i apostole. Neke će poubijati  i prognati - 
\par 50 da se od ovog naraštaja zatraži krv svih proroka  prolivena od postanka svijeta, 
\par 51 od krvi Abelove do krvi Zaharije, koji je pogubljen između žrtvenika i svetišta.' Da, kažem vam, tražit će se od ovoga naraštaja!" 
\par 52 "Jao vama, zakonoznanci! Uzeste ključ znanja: sami ne  uđoste, a spriječiste one koji htjedoše ući." 
\par 53 Kad Isus izađe odande, stadoše pismoznanci i farizeji  žestoko na nj navaljivati i postavljati mu mnoga pitanja 
\par 54 vrebajući  na nj, ne bi li štogod ulovili iz njegovih usta. 


\chapter{12}

\par 1 Kad se uto skupilo mnoštvo, tisuće i tisuće, te su jedni druge  gazili, poče Isus govoriti najprije svojim učenicima: "Čuvajte  se kvasca farizejskoga, to jest licemjerja. 
\par 2 Ništa nije skriveno  što se neće otkriti ni tajno što se neće saznati. 
\par 3 Naprotiv, sve što u tami rekoste, na svjetlu će se čuti; i što ste po  skrovištima u uho šaptali, propovijedat će se po krovovima." 
\par 4 "A kažem vama, prijateljima svojim: ne bojte se onih koji  ubijaju tijelo, a nakon toga nemaju više što učiniti. 
\par 5 Pokazat  ću vam koga vam se bojati: onoga se bojte koji pošto ubije, ima  moć baciti u pakao. Da, velim vam, njega se bojte! 
\par 6 Ne prodaje  li se pet vrapčića za dva novčića? Pa ipak ni jednoga od njih  Bog ne zaboravlja. 
\par 7 A vama su i vlasi na glavi sve izbrojene.  Ne bojte se! Vredniji ste nego mnogo vrabaca!" 
\par 8 "A kažem vam: tko se god prizna mojim pred ljudima, i  Sin Čovječji priznat će se njegovim pred anđelima Božjim. 
\par 9 A  tko mene zaniječe pred ljudima, bit će zanijekan pred anđelima  Božjim." 
\par 10 "I tko god rekne riječ na Sina Čovječjega, oprostit će  mu se. Ali tko pohuli protiv Duha Svetoga, neće mu se oprostiti." 
\par 11 "Nadalje, kad vas budu dovodili pred sinagoge i poglavarstva  i vlasti, ne budite zabrinuti kako ćete se ili čime braniti,  što li reći! 
\par 12 Ta Duh Sveti poučit će vas u taj čas što valja  reći." 
\par 13 Tada mu netko iz mnoštva reče: "Učitelju, reci mome bratu  da podijeli sa mnom baštinu." 
\par 14 Nato mu on reče: "Čovječe,  tko me postavio sucem ili djeliocem nad vama?" 
\par 15 I dometnu  im: "Klonite se i čuvajte svake pohlepe: koliko god netko obilovao, život mu nije u onom što posjeduje." 
\par 16 Kaza im i prispodobu: "Nekomu bogatu čovjeku obilno urodi  zemlja 
\par 17 pa u sebi razmišljaše: 'Što da učinim? Nemam gdje  skupiti svoju ljetinu.' 
\par 18 I reče: 'Evo što ću učiniti! Srušit  ću svoje žitnice i podignuti veće pa ću ondje zgrnuti sve žito  i dobra svoja. 
\par 19 Tada ću reći duši svojoj: dušo, evo imaš u  zalihi mnogo dobara za godine mnoge. Počivaj, jedi, pij, uživaj!' 
\par 20 Ali Bog mu reče: 'Bezumniče! Već noćas duša će se tvoja zaiskati  od tebe! A što si pripravio, čije će biti?' 
\par 21 Tako biva s onim  koji sebi zgrće blago, a ne bogati se u Bogu." 
\par 22 Zatim reče svojim učenicima: "Zato vam kažem: ne budite  zabrinuti za život: što ćete jesti; ni za tijelo: u što ćete  se obući. 
\par 23 Ta život je vredniji od jela i tijelo od odijela. 
\par 24 Promotrite gavrane! Ne siju niti žanju, nemaju spremišta  ni žitnice, pa ipak ih Bog hrani. Koliko li ste vi vredniji od  ptica! 
\par 25 A tko od vas zabrinutošću može svojemu stasu dodati  lakat? 
\par 26 Ako dakle ni ono najmanje ne možete, što ste onda  za ostalo zabrinuti? 
\par 27 Promotrite ljiljane, kako niti predu  niti tkaju, a kažem vam: ni Salomon se u svoj svojoj slavi ne  zaodjenu kao jedan od njih. 
\par 28 Pa ako travu koja je danas u  polju, a sutra se u peć baca Bog tako odijeva, koliko li će više  vas, malovjerni!" 
\par 29 "Zato i vi: ne tražite što ćete jesti, što piti. Ne uznemirujte  se! 
\par 30 Ta sve to traže pogani ovoga svijeta. Otac vaš zna da  vam je sve to potrebno. 
\par 31 Nego, tražite kraljevstvo njegovo, a to će vam se nadodati!" 
\par 32 "Ne boj se, stado malo: svidjelo se Ocu vašemu dati vam  Kraljevstvo." 
\par 33 "Prodajte što god imate i dajte za milostinju! Načinite  sebi kese koje ne stare, blago nepropadljivo na nebesima, kamo  se kradljivac ne približava i gdje moljac ne rastače. 
\par 34 Doista, gdje vam je blago, ondje će vam i srce biti." 
\par 35 "Neka vam bokovi budu opasani i svjetiljke upaljene, 
\par 36 a vi slični ljudima što čekaju gospodara kad se vraća sa  svadbe da mu odmah otvore čim stigne i pokuca. 
\par 37 Blago onim  slugama koje gospodar, kada dođe, nađe budne! Zaista, kažem vam, pripasat će se, posaditi ih za stol pa će pristupiti i posluživati  ih. 
\par 38 Pa dođe li o drugoj ili o trećoj straži i nađe ih tako, blago njima!" 
\par 39 "A ovo znajte: kad bi domaćin znao u koji čas kradljivac  dolazi, ne bi dao prokopati kuće. 
\par 40 I vi budite pripravni jer  u čas kad i ne mislite Sin Čovječji dolazi." 
\par 41 Nato će Petar: "Gospodine, govoriš li tu prispodobu samo  za nas ili i za sve?" 
\par 42 Reče Gospodin: "Tko li je onaj vjerni  i razumni upravitelj što će ga gospodar postaviti nad svojom  poslugom da im u pravo vrijeme daje obrok? 
\par 43 Blago onome sluzi  kojega gospodar kada dođe, nađe da tako radi. 
\par 44 Uistinu, kažem  vam, postavit će ga nad svim imanjem svojim." 
\par 45 "No rekne li taj sluga u srcu: 'Okasnit će gospodar moj'  pa stane tući sluge i sluškinje, jesti, piti i opijati se, 
\par 46 doći  će gospodar toga sluge u dan u koji mu se ne nada i u čas u koji  i ne sluti; rasjeći će ga i dodijeliti mu udes među nevjernicima." 
\par 47 "I onaj sluga što je znao volju gospodara svoga, a nije  bio spreman ili nije učinio po volji njegovoj, dobit će mnogo  udaraca. 
\par 48 A onaj koji nije znao, ali je učinio što zaslužuje  udarce, dobit će malo udaraca. Kome je god mnogo dano, od njega  će se mnogo iskati. Kome je mnogo povjereno, više će se od njega  iskati." 
\par 49 "Oganj dođoh baciti na zemlju pa što hoću ako je već  planuo! 
\par 50 Ali krstom mi se krstiti i kakve li muke za me dok  se to ne izvrši!" 
\par 51 "Mislite li da sam došao mir dati na zemlji? Nipošto, kažem vam, nego razdjeljenje. 
\par 52 Ta bit će odsada petorica  u jednoj kući razdijeljena: razdijelit će se trojica protiv dvojice  i dvojica protiv trojice - 
\par 53 otac protiv sina i sin protiv  oca, mati protiv kćeri i kći protiv matere, svekrva  protiv snahe i snaha protiv svekrve." 
\par 54 Zatim je govorio mnoštvu: "Kad opazite da se oblak diže  na zapadu, odmah kažete: 'Kiša će!' I bude tako. 
\par 55 Kad zapuše  južnjak, kažete: 'Bit će vrućine!' I bude. 
\par 56 Licemjeri! Lice  zemlje i neba umijete rasuditi, kako onda ovo vrijeme ne rasuđujete?" 
\par 57 "Zašto sami od sebe ne sudite što je pravo? 
\par 58 Kad s  protivnikom ideš glavaru, na putu sve uloži da ga se oslobodiš  pa te ne odvuče k sucu. Sudac će te predati izvršitelju, a izvršitelj  baciti u tamnicu. 
\par 59 Kažem ti: nećeš izići odande dok ne isplatiš  do posljednjega novčića." 


\chapter{13}

\par 1 Upravo u taj čas dođoše neki te mu javiše što se dogodilo  s Galilejcima kojih je krv Pilat pomiješao s krvlju njihovih  žrtava. 
\par 2 Isus im odgovori: "Mislite li da ti Galilejci, jer  tako postradaše, bijahu grešniji od drugih Galilejaca? 
\par 3 Nipošto, kažem vam, nego ako se ne obratite, svi ćete slično propasti! 
\par 4 Ili onih osamnaest na koje se srušila kula u Siloamu i ubila  ih, zar mislite da su oni bili veći dužnici od svih Jeruzalemaca? 
\par 5 Nipošto, kažem vam, nego ako se ne obratite, svi ćete tako  propasti." 
\par 6 Nato im pripovjedi ovu prispodobu: "Imao netko smokvu  zasađenu u svom vinogradu. Dođe tražeć ploda na njoj i ne nađe 
\par 7 pa reče vinogradaru: 'Evo, već tri godine dolazim i tražim  ploda na ovoj smokvi i ne nalazim. Posijeci je. Zašto da iscrpljuje  zemlju?' 
\par 8 A on mu odgovori: 'Gospodaru, ostavi je još ove godine  dok je ne okopam i ne pognojim. 
\par 9 Možda će ubuduće ipak uroditi.  Ako li ne, posjeći ćeš je.'" 
\par 10 Jedne je subote naučavao u nekoj sinagogi. 
\par 11 Kad eto  žene koja je osamnaest godina imala duha bolesti. Bila je zgrbljena  i nikako se nije mogla uspraviti. 
\par 12 Kad je Isus opazi, dozva  je i reče joj: "Ženo, oslobođena si svoje bolesti!" 
\par 13 I položi  na nju ruke, a ona se umah uspravi i poče slaviti Boga. 
\par 14 Nadstojnik sinagoge - ozlovoljen što je Isus u subotu  izliječio - govoraše mnoštvu: "Šest je dana u koje treba raditi!  U te dakle dane dolazite i liječite se, a ne u dan subotni!" 
\par 15 Odgovori mu Gospodin: "Licemjeri! Ne driješi li svaki od  vas u subotu svoga vola ili magarca od jasala da ga vodi na vodu? 
\par 16 Nije li dakle i ovu kćer Abrahamovu, koju Sotona sveza evo  osamnaest je već godina, trebalo odriješiti od tih spona u dan  subotni?" 
\par 17 Na te njegove riječi postidješe se svi protivnici njegovi, a sav se narod radovaše zbog svega čime se on proslavio. 
\par 18 Govoraše dakle: "Čemu je slično kraljevstvo Božje? Čemu  da ga prispodobim? 
\par 19 Ono je kao kad čovjek uze gorušičino zrno  i baci ga u svoj vrt. Uzraste i razvi se u stablo te mu se ptice  nebeske gnijezde po granama." 
\par 20 I opet im reče: "Čemu da prispodobim kraljevstvo Božje? 
\par 21 Ono je kao kad žena uze kvasac i zamijesi ga u tri mjere  brašna dok sve ne uskisne." 
\par 22 Putujući tako u Jeruzalem, prolazio je i naučavao gradovima  i selima. 
\par 23 Reče mu tada netko: "Gospodine, je li malo onih  koji se spasavaju?" A on im reče: 
\par 24 "Borite se da uđete na  uska vrata jer mnogi će, velim vam, tražiti da uđu, ali neće  moći." 
\par 25 "Kada gospodar kuće ustane i zaključa vrata, a vi stojeći  vani počnete kucati na vrata: 'Gospodine, otvori nam!', on će  vam odgovoriti: 'Ne znam vas odakle ste!' 
\par 26 Tada ćete početi  govoriti: 'Pa mi smo s tobom jeli i pili, po našim si trgovima  naučavao!' 
\par 27 A on će vam reći: 'Kažem vam: ne znam odakle ste.  Odstupite od mene, svi zlotvori!'" 
\par 28 "Ondje će biti plač i škrgut zubi kad ugledate Abrahama  i Izaka i Jakova i sve proroke u kraljevstvu Božjem, a sebe vani, izbačene. 
\par 29 I doći će s istoka i zapada, sa sjevera i juga  i sjesti za stol u kraljevstvu Božjem. 
\par 30 Evo, ima posljednjih  koji će biti prvi, ima i prvih koji će biti posljednji." 
\par 31 U taj čas pristupe neki farizeji i reknu mu: "Otiđi,  otputuj odavde jer te Herod hoće ubiti." 
\par 32 A on će njima: "Idite  i kažite toj lisici: 'Evo, izgonim đavle i liječim danas i sutra, a treći dan dovršujem. 
\par 33 Ali danas, sutra i prekosutra moram  nastaviti put jer ne priliči da prorok pogine izvan Jeruzalema.'" 
\par 34 "Jeruzaleme, Jeruzaleme, koji ubijaš proroke i kamenuješ  one što su tebi poslani! Koliko li puta htjedoh skupiti djecu  tvoju kao kvočka piliće pod krila i ne htjedoste! 
\par 35 Evo, napuštena  vam kuća. A kažem vam, nećete me vidjeti dok ne dođe čas te reknete:  "Blagoslovljen Onaj koji dolazi u ime Gospodnje!" 


\chapter{14}

\par 1 Jedne subote dođe on u kuću nekoga prvaka farizejskog na objed.  A oni ga vrebahu. 
\par 2 Kad evo: pred njim neki čovjek koji je imao  vodenu bolest. 
\par 3 Nato Isus upita zakonoznance i farizeje: "Je  li dopušteno subotom liječiti ili nije?" 
\par 4 A oni mukom ponikoše.  On ga dotaknu, izliječi i otpusti. 
\par 5 A njima reče: "Ako komu  od vas sin ili vol padne u bunar, neće li ga brže bolje izvući  i u dan subotni?" 
\par 6 I ne mogoše mu na to odgovoriti. 
\par 7 Promatrajući kako uzvanici biraju prva mjesta, kaza im  prispodobu: 
\par 8 "Kada te tko pozove na svadbu, ne sjedaj na prvo  mjesto da ne bi možda bio pozvan koji časniji od tebe, 
\par 9 te  ne dođe onaj koji je pozvao tebe i njega i ne rekne ti: 'Ustupi  mjesto ovome.' Tada ćeš, postiđen, morati zauzeti posljednje  mjesto. 
\par 10 Nego kad budeš pozvan, idi i sjedni na posljednje  mjesto pa, kada dođe onaj koji te pozvao, da ti rekne: 'Prijatelju, pomakni se naviše!' Bit će ti to tada na čast pred svim sustolnicima, 
\par 11 jer - svaki koji se uzvisuje, bit će ponižen, a koji se  ponizuje, bit će uzvišen." 
\par 12 A i onome koji ga pozva, kaza: "Kad priređuješ objed  ili večeru, ne pozivaj svojih prijatelja, ni braće, ni rodbine, ni bogatih susjeda, da ne bi možda i oni tebe pozvali i tako  ti uzvratili. 
\par 13 Nego kad priređuješ gozbu, pozovi siromahe, sakate, hrome, slijepe. 
\par 14 Blago tebi jer oni ti nemaju čime  uzvratiti. Uzvratit će ti se doista o uskrsnuću pravednih." 
\par 15 Kad je to čuo jedan od sustolnika, reče mu: "Blago onome  koji bude blagovao u kraljevstvu Božjem!" 
\par 16 A on mu reče: "Čovjek neki priredi veliku večeru i pozva mnoge. 
\par 17 I posla  slugu u vrijeme večere da rekne uzvanicima: 'Dođite! Već je pripravljeno!' 
\par 18 A oni se odreda počeli ispričavati. Prvi mu reče: 'Njivu  sam kupio i valja mi poći pogledati je. Molim te, ispričaj me.' 
\par 19 Drugi reče: 'Kupio sam pet jarmova volova pa idem okušati  ih. Molim te, ispričaj me.' 
\par 20 Treći reče: 'Oženio sam se i  zato ne mogu doći.'" 
\par 21 "Sluga se vrati i javi to domaćinu. Tada domaćin, gnjevan, reče sluzi: 'Iziđi brzo na trgove gradske i ulice pa dovedi  ovamo prosjake, sakate, slijepe i hrome.' 
\par 22 I sluga reče: 'Gospodaru, učinjeno je što si naredio i još ima mjesta.' 
\par 23 Reče gospodar  sluzi: 'Iziđi na putove i među ograde i prisili neka uđu da mi  se napuni kuća.' 
\par 24 A kažem vam: nijedan od onih pozvanih neće  okusiti moje večere." 
\par 25 S njim je zajedno putovalo silno mnoštvo. On se okrene  i reče im: 
\par 26 "Dođe li tko k meni, a ne mrzi svog oca i majku, ženu i djecu, braću i sestre, pa i sam svoj život, ne može biti  moj učenik! 
\par 27 I tko ne nosi svoga križa i ne ide za mnom, ne  može biti moj učenik!" 
\par 28 "Tko od vas, nakan graditi kulu, neće prije sjesti i  proračunati troškove ima li čime dovršiti: 
\par 29 da ga ne bi -  pošto već postavi temelj, a ne mogne dovršiti - počeli ismjehivati  svi koji to vide: 
\par 30 'Ovaj čovjek poče graditi, a ne može dovršiti!' 
\par 31 Ili koji kralj kad polazi da se zarati s drugim kraljem,  neće prije sjesti i promisliti da li s deset tisuća može presresti  onoga koji na nj dolazi s dvadeset tisuća? 
\par 32 Ako ne može, dok  je onaj još daleko, poslat će poslanstvo da zaište mir." 
\par 33 "Tako dakle nijedan od vas koji se ne odrekne svega što  posjeduje, ne može biti moj učenik." 
\par 34 "Dobra je sol. Ali ako i sol obljutavi, čime će se ona  začiniti? 
\par 35 Nije prikladna ni za zemlju ni za gnojište. Van  se baca. Tko ima uši da čuje, neka čuje!" 


\chapter{15}

\par 1 Okupljahu se oko njega svi carinici i grešnici da ga slušaju. 
\par 2 Stoga farizeji i pismoznanci mrmljahu: "Ovaj prima grešnike, i blaguje s njima." 
\par 3 Nato im Isus kaza ovu prispodobu: 
\par 4 "Tko to od vas, ako  ima sto ovaca pa izgubi jednu od njih, ne ostavi onih devedeset  i devet u pustinji te pođe za izgubljenom dok je ne nađe? 
\par 5 A  kad je nađe, stavi je na ramena sav radostan 
\par 6 pa došavši kući, sazove prijatelje i susjede i rekne im: 'Radujte se sa mnom!  Nađoh ovcu svoju izgubljenu.' 
\par 7 Kažem vam, tako će na nebu biti  veća radost zbog jednog obraćena grešnika nego li zbog devedeset  i devet pravednika kojima ne treba obraćenja." 
\par 8 "Ili koja to žena, ima li deset drahma pa izgubi jednu  drahmu, ne zapali svjetiljku, pomete kuću i brižljivo pretraži  dok je ne nađe? 
\par 9 A kad je nađe, pozove prijateljice i susjede  pa će im: 'Radujte se sa mnom! Nađoh drahmu što je bijah izgubila.' 
\par 10 Tako, kažem vam, biva radost pred anđelima Božjim zbog jednog  obraćena grešnika." 
\par 11 I nastavi: "Čovjek neki imao dva sina. 
\par 12 Mlađi reče  ocu: 'Oče, daj mi dio dobara koji mi pripada.' I razdijeli im  imanje. 
\par 13 Nakon nekoliko dana mlađi sin pokupi sve, otputova  u daleku zemlju i ondje potrati svoja dobra živeći razvratno." 
\par 14 "Kad sve potroši, nasta ljuta glad u onoj zemlji te on  poče oskudijevati. 
\par 15 Ode i pribi se kod jednoga žitelja u onoj  zemlji. On ga posla na svoja polja pasti svinje. 
\par 16 Želio se  nasititi rogačima što su ih jele svinje, ali mu ih nitko nije  davao." 
\par 17 "Došavši k sebi, reče: 'Koliki najamnici oca moga imaju  kruha napretek, a ja ovdje umirem od gladi! 
\par 18 Ustat ću, poći  svomu ocu i reći mu: 'Oče, sagriješih protiv Neba i pred tobom! 
\par 19 Nisam više dostojan zvati se sinom tvojim. Primi me kao jednog  od svojih najamnika.'" 
\par 20 "Usta i pođe svom ocu. Dok je još bio daleko, njegov  ga otac ugleda, ganu se, potrča, pade mu oko vrata i izljubi  ga. 
\par 21 A sin će mu: 'Oče! Sagriješih protiv Neba i pred tobom!  Nisam više dostojan zvati se sinom tvojim.' 
\par 22 A otac reče slugama:  'Brzo iznesite haljinu najljepšu i obucite ga! Stavite mu prsten  na ruku i obuću na noge! 
\par 23 Tele ugojeno dovedite i zakoljite, pa da se pogostimo i proveselimo 
\par 24 jer sin mi ovaj bijaše  mrtav i oživje, izgubljen bijaše i nađe se!' I stadoše se veseliti." 
\par 25 "A stariji mu sin bijaše u polju. Kad se na povratku  približio kući, začu svirku i igru 
\par 26 pa dozva jednoga slugu  da se raspita što je to. 
\par 27 A ovaj će mu: 'Došao tvoj brat pa  otac tvoj zakla tele ugojeno što sina zdrava dočeka.' 
\par 28 A on  se rasrdi i ne htjede ući. Otac tada iziđe i stane ga nagovarati. 
\par 29 A on će ocu: 'Evo toliko ti godina služim i nikada ne prestupih  tvoju zapovijed, a nikad mi ni jareta nisi dao da se s prijateljima  proveselim. 
\par 30 A kada dođe ovaj sin tvoj koji s bludnicama proždrije  tvoje imanje, ti mu zakla ugojeno tele.' 
\par 31 Nato će mu otac:  'Sinko, ti si uvijek sa mnom i sve moje - tvoje je. 
\par 32 No trebalo  se veseliti i radovati jer ovaj brat tvoj bijaše mrtav i oživje, izgubljen i nađe se!'" 


\chapter{16}

\par 1 Govoraše i svojim učenicima: "Bijaše neki bogat čovjek koji  je imao upravitelja. 
\par 2 Ovaj je bio optužen pred njim kao da  mu rasipa imanje. On ga pozva pa mu reče: 'Što to čujem o tebi?  Položi račun o svom upravljanju jer više ne možeš biti upravitelj!' 
\par 3 Nato upravitelj reče u sebi: 'Što da učinim kad mi gospodar  moj oduzima upravu? Kopati? Nemam snage. Prositi? Stidim se. 
\par 4 Znam što ću da me prime u svoje kuće kad budem maknut s uprave.'" 
\par 5 "I pozva dužnike svoga gospodara, jednog po jednog. Upita  prvoga: 'Koliko duguješ gospodaru mojemu?' On reče: 'Sto bata  ulja.' 
\par 6 A on će mu: 'Uzmi svoju zadužnicu, sjedni brzo, napiši  pedeset.' 
\par 7 Zatim reče drugomu: 'A ti, koliko ti duguješ?' On  odgovori: 'Sto kora pšenice.' Kaže mu: 'Uzmi svoju zadužnicu  i napiši osamdeset.'" 
\par 8 "I pohvali gospodar nepoštenog upravitelja što snalažljivo  postupi jer sinovi su ovoga svijeta snalažljiviji prema svojima  od sinova svjetlosti." 
\par 9 "I ja vama kažem: napravite sebi prijatelje od nepoštena  bogatstva pa kad ga nestane da vas prime u vječne šatore." 
\par 10 "Tko je vjeran u najmanjem, i u najvećem je vjeran; a  tko je u najmanjem nepošten, i u najvećem je nepošten. 
\par 11 Ako  dakle ne bijaste vjerni u nepoštenom bogatstvu, tko li će vam  istinsko povjeriti? 
\par 12 I ako u tuđem ne bijaste vjerni, tko  li će vam vaše dati?" 
\par 13 "Nijedan sluga ne može služiti dvojici gospodara. Ili  će jednoga mrziti, a drugoga ljubiti; ili će uz jednoga prianjati, a drugoga prezirati. Ne možete služiti Bogu i bogatstvu." 
\par 14 Sve su to slušali farizeji, srebroljupci, i rugali mu  se. 
\par 15 On im reče: "Vi se pravite pravedni pred ljudima, ali  zna Bog srca vaša. Jer što je ljudima uzvišeno, odvratnost je  pred Bogom." 
\par 16 "Zakon i Proroci do Ivana su, a otada se navješćuje kraljevstvo  Božje i svatko se u nj silom probija. 
\par 17 Lakše će nebo i zemlja  proći, negoli propasti i jedan potezić Zakona." 
\par 18 "Tko god otpusti svoju ženu pa se oženi drugom, čini  preljub. I tko se god oženi otpuštenom, čini preljub." 
\par 19 "Bijaše neki bogataš. Odijevao se u grimiz i tanani lan  i danomice se sjajno gostio. 
\par 20 A neki siromah, imenom Lazar, ležao je sav u čirevima pred njegovim vratima 
\par 21 i priželjkivao  nasititi se onim što je padalo s bogataševa stola. Čak su i psi  dolazili i lizali mu čireve." 
\par 22 "Kad umrije siromah, odnesoše ga anđeli u krilo Abrahamovo.  Umrije i bogataš te bude pokopan. 
\par 23 Tada u teškim mukama u  paklu, podiže svoje oči te izdaleka ugleda Abrahama i u krilu  mu Lazara 
\par 24 pa zavapi: 'Oče Abrahame, smiluj mi se i pošalji  Lazara da umoči vršak svoga prsta u vodu i rashladi mi jezik  jer se strašno mučim u ovom plamenu.' 
\par 25 Reče nato Abraham:  'Sinko! Sjeti se da si za života primio dobra svoja, a tako i  Lazar zla. Sada se on ovdje tješi, a ti se mučiš. 
\par 26 K tome  između nas i vas zjapi provalija golema te koji bi i htjeli prijeći  odavde k vama, ne mogu, a ni odatle k nama prijelaza nema.'" 
\par 27 "Nato će bogataš: 'Molim te onda, oče, pošalji Lazara  u kuću oca moga. 
\par 28 Imam petero braće pa neka im posvjedoči  da i oni ne dođu u ovo mjesto muka.' 
\par 29 Kaže Abraham: 'Imaju Mojsija i Proroke! Njih neka poslušaju!' 
\par 30 A on će: 'O ne, oče Abrahame! Nego dođe li tko od mrtvih  k njima, obratit će se.' 
\par 31 Reče mu: 'Ako ne slušaju Mojsija  i Proroka, neće povjerovati sve da i od mrtvih tko ustane.'" 


\chapter{17}

\par 1 I reče svojim učenicima: "Nije moguće da ne dođu sablazni, no jao onome po kom dolaze; 
\par 2 je li s mlinskim kamenom o vratu  strovaljen u more, korisnije mu je, nego da sablazni jednoga  od ovih malenih. 
\par 3 Čuvajte se!" "Pogriješi li tvoj brat, prekori ga; ako se obrati, oprosti  mu. 
\par 4 Pa ako se sedam puta na dan ogriješi o tebe i sedam se  puta obrati tebi govoreći: 'Žao mi je!', oprosti mu." 
\par 5 Apostoli zamole Gospodina: "Umnoži nam vjeru!" 
\par 6 Gospodin  im odvrati: "Da imate vjere koliko je zrno gorušičino, rekli  biste ovom dudu: 'Iščupaj se s korijenom i presadi se u more!'  I on bi vas poslušao." 
\par 7 "Tko će to od vas reći sluzi svomu, oraču ili pastiru, koji se vrati s polja: 'Dođi brzo i sjedni za stol?' 
\par 8 Neće  li mu naprotiv reći: 'Pripravi što ću večerati pa se pripaši  i poslužuj mi dok jedem i pijem; potom ćeš ti jesti i piti?' 
\par 9 Zar duguje zahvalnost sluzi jer je izvršio što mu je naređeno? 
\par 10 Tako i vi: kad izvršite sve što vam je naređeno, recite:  'Sluge smo beskorisne! Učinismo što smo bili dužni učiniti!'" 
\par 11 Dok je tako putovao u Jeruzalem, prolazio je između Samarije  i Galileje. 
\par 12 Kad je ulazio u neko selo, eto mu u susret deset  gubavaca. Zaustave se podaleko 
\par 13 i zavape: "Isuse, Učitelju, smiluj nam se!" 
\par 14 Kad ih Isus ugleda, reče im: "Idite, pokažite  se svećenicima!" I dok su išli, očistiše se. 
\par 15 Jedan od  njih vidjevši da je ozdravio, vrati se slaveći Boga u sav glas. 
\par 16 Baci se ničice k Isusovim nogama zahvaljujući mu. A to bijaše  neki Samarijanac. 
\par 17 Nato Isus primijeti: "Zar se ne očistiše  desetorica? 
\par 18 A gdje su ona devetorica? Ne nađe li se nijedan  koji bi se vratio i podao slavu Bogu, osim ovoga tuđinca?" 
\par 19 A  njemu reče: "Ustani! Idi! Tvoja te vjera spasila!" 
\par 20 Upitaju ga farizeji: "Kad će doći kraljevstvo Božje?"  Odgovori im: "Kraljevstvo Božje ne dolazi primjetljivo. 
\par 21 Niti  će se moći kazati: 'Evo ga ovdje!' ili: 'Eno ga ondje!' Ta evo  - kraljevstvo je Božje među vama!" 
\par 22 Zatim reče učenicima: "Doći će dani kad ćete zaželjeti  vidjeti i jedan dan Sina Čovječjega, ali ga nećete vidjeti. 
\par 23 Govorit  će vam: 'Eno ga ondje, evo ovdje!' Ne odlazite i ne pomamite  se! 
\par 24 Jer kao što munja sijevne na jednom kraju obzorja i odbljesne  na drugom, tako će biti i sa Sinom Čovječjim u Dan njegov. 
\par 25 No  prije treba da on mnogo pretrpi i da ga ovaj naraštaj odbaci." 
\par 26 "I kao što bijaše u dane Noine, tako će biti i u dane  Sina Čovječjega: 
\par 27 jeli su, pili, ženili se i udavali do dana  kad Noa uđe u korablju. I dođe potop i sve uništi. 
\par 28 Slično  kao što bijaše u dane Lotove: jeli su, pili, kupovali, prodavali, sadili, gradili. 
\par 29 A onog dana kad Lot iziđe iz Sodome, zapljušti  s neba oganj i sumpor i sve uništi. 
\par 30 Tako će isto biti  u dan kad se Sin Čovječji objavi." 
\par 31 "U onaj dan tko bude na krovu, a stvari mu u kući, neka  ne siđe da ih uzme. 
\par 32 I tko bude u polju, neka se ne okreće  natrag. Sjetite se žene Lotove! 
\par 33 Tko god bude nastojao  život svoj sačuvati, izgubit će ga; a tko ga izgubi, živa će  ga sačuvati." 
\par 34 "Kažem vam, one će noći biti dvojica u jednoj postelji:  jedan će se uzeti, drugi ostaviti. 
\par 35 Dvije će mljeti zajedno:  jedna će se uzeti, druga ostaviti." 
\par 36 # 
\par 37 Upitaše ga na  to: "Gdje to, Gospodine?" A on im reče: "Gdje bude trupla, ondje  će se okupljati i orlovi." 


\chapter{18}

\par 1 Kaza im i prispodobu kako valja svagda moliti i nikada ne  sustati: 
\par 2 "U nekom gradu bio sudac. Boga se nije bojao, za  ljude nije mario. 
\par 3 U tom gradu bijaše i neka udovica. Dolazila  k njemu i molila: 'Obrani me od mog tužitelja!' 
\par 4 No on ne htjede  zadugo. Napokon reče u sebi: 'Iako se Boga ne bojim nit za ljude  marim, ipak, 
\par 5 jer mi udovica ova dodijava, obranit ću je da  vječno ne dolazi mučiti me.'" 
\par 6 Nato reče Gospodin: "Čujte što govori nepravedni sudac! 
\par 7 Neće li onda Bog obraniti svoje izabrane koji dan i noć vape  k njemu sve ako i odgađa stvar njihovu? 
\par 8 Kažem vam, ustat će  žurno na njihovu obranu. Ali kad Sin Čovječji dođe, hoće li naći  vjere na zemlji?" 
\par 9 Nekima pak koji se pouzdavahu u sebe da su pravednici, a druge potcjenjivahu, reče zatim ovu prispodobu: 
\par 10 "Dva čovjeka  uziđoše u Hram pomoliti se: jedan farizej, drugi carinik. 
\par 11 Farizej  se uspravan ovako u sebi molio: 'Bože, hvala ti što nisam kao  ostali ljudi: grabežljivci, nepravednici, preljubnici ili - kao  ovaj carinik.' 
\par 12 Postim dvaput u tjednu, dajem desetinu od  svega što steknem.' 
\par 13 A carinik, stojeći izdaleka, ne usudi  se ni očiju podignuti k nebu, nego se udaraše u prsa govoreći:  'Bože milostiv budi meni grešniku!' 
\par 14 Kažem vam: ovaj siđe  opravdan kući svojoj, a ne onaj! Svaki koji se uzvisuje, bit  će ponižen; a koji se ponizuje, bit će uzvišen." 
\par 15 A donosili mu i dojenčad da ih se dotakne. Vidjevši to, učenici im branili. 
\par 16 A Isus ih dozva i reče: "Pustite dječicu  neka dolaze k meni i ne priječite im jer takvih je kraljevstvo  Božje." 
\par 17 "Zaista, kažem vam, tko ne primi kraljevstva Božjega  kao dijete, ne, u nj neće ući." 
\par 18 I upita ga neki uglednik: "Učitelju dobri, što mi je  činiti da baštinim život vječni?" 
\par 19 Reče mu Isus: "Što me zoveš  dobrim? Nitko nije dobar, doli Bog jedini. 
\par 20 Zapovijedi znaš: Ne čini preljuba! Ne ubij! Ne ukradi! Ne svjedoči lažno! Poštuj oca svoga i majku!" 
\par 21 A onaj će: "Sve sam to čuvao od mladosti." 
\par 22 Čuvši  to, Isus mu reče: "Još ti jedno preostaje: sve što imaš prodaj  i razdaj siromasima pa ćeš imati blago na nebu. A onda dođi i  idi za mnom." 
\par 23 Kad je on to čuo, ražalosti se jer bijaše silno  bogat. 
\par 24 Vidjevši ga, reče Isus: "Kako li je teško imućnicima  u kraljevstvo Božje! 
\par 25 Lakše je devi kroz uši iglene nego bogatašu  u kraljevstvo Božje." 
\par 26 Koji su to čuli, rekoše: "Pa tko se onda može spasiti?" 
\par 27 A on će: "Što je nemoguće ljudima, moguće je Bogu." 
\par 28 Nato reče Petar: "Evo, mi ostavismo svoje i pođosmo za  tobom." 
\par 29 Isus će im: "Zaista, kažem vam, nema ga tko bi ostavio  kuću, ili ženu, ili braću, ili roditelje, ili djecu poradi kraljevstva  Božjega, 
\par 30 a da ne bi primio mnogostruko već u ovom vremenu, i u budućem vijeku život vječni." 
\par 31 I uzevši sa sobom dvanaestoricu, reče im: "Evo uzlazimo  u Jeruzalem i na Sinu Čovječjem ispunit će se sve što su napisali  proroci: 
\par 32 doista, bit će predan poganima, izrugan, zlostavljan  i popljuvan; 
\par 33 i pošto ga izbičuju, ubit će ga, ali on će treći  dan ustati." 
\par 34 No oni ništa od toga ne razumješe. Te im riječi bijahu  skrivene i ne shvaćahu što bijaše rečeno. 
\par 35 A kad se približavao Jerihonu, neki slijepac sjedio kraj  puta i prosio. 
\par 36 Čuvši gdje mnoštvo prolazi, raspitivao se  što je to. 
\par 37 Rekoše mu: "Isus Nazarećanin prolazi." 
\par 38 Tada  povika: "Isuse, Sine Davidov, smiluj mi se!" 
\par 39 Oni ga sprijeda  ušutkivali, ali on je još jače vikao: "Sine Davidov, smiluj mi  se!" 
\par 40 Isus se zaustavi i zapovjedi da ga dovedu k njemu. Kad  se on približi, upita ga: 
\par 41 "Što hoćeš da ti učinim?" A on  će: "Gospodine, da progledam." 
\par 42 Isus će mu: "Progledaj! Vjera  te tvoja spasila." 
\par 43 I umah progleda i uputi se za njim slaveći  Boga. I sav narod koji to vidje dade hvalu Bogu. 


\chapter{19}

\par 1 I uđe u Jerihon. Dok je njime prolazio, 
\par 2 eto čovjeka imenom  Zakej. Bijaše on nadcarinik, i to bogat. 
\par 3 Želio je vidjeti  tko je to Isus, ali ne mogaše od mnoštva jer je bio niska stasa. 
\par 4 Potrča naprijed, pope se na smokvu da ga vidi jer je onuda  imao proći. 
\par 5 Kad Isus dođe na to mjesto, pogleda gore i reče  mu: "Zakeju, žurno siđi! Danas mi je proboraviti u tvojoj kući." 
\par 6 On žurno siđe i primi ga sav radostan. 
\par 7 A svi koji to vidješe  stadoše mrmljati: "Čovjeku se grešniku svratio!" 
\par 8 A Zakej usta  i reče Gospodinu: "Evo, Gospodine, polovicu svog imanja dajem  siromasima! I ako sam koga u čemu prevario, vraćam četverostruko." 
\par 9 Reče mu na to Isus: "Danas je došlo spasenje ovoj kući jer  i on je sin Abrahamov! 
\par 10 Ta Sin Čovječji dođe potražiti  i spasiti izgubljeno!" 
\par 11 Kako su oni to slušali, dometnu on prispodobu - zato  što bijaše nadomak Jeruzalemu i oni mislili da će se umah pojaviti  kraljevstvo Božje. 
\par 12 Reče dakle: "Neki je ugledan čovjek imao otputovati u daleku zemlju da  primi svoje kraljevstvo pa da se vrati. 
\par 13 Dozva svojih deset  slugu, dade im deset mna i reče: 'Trgujte dok ne dođem.' 
\par 14 A  njegovi ga građani mrzili te poslaše za njim poslanstvo s porukom:  'Nećemo da se ovaj zakralji nad nama.'" 
\par 15 "Kad je on primio kraljevstvo i vratio se, naredi da  mu dozovu one sluge kojima je predao novac da sazna što su zaradili." 
\par 16 "Pristupi prvi i reče: 'Gospodaru, tvoja mna donije deset  mna.' 
\par 17 Reče mu: 'Valjaš, slugo dobri! U najmanjem si bio vjeran, vladaj nad deset gradova!' 
\par 18 Dođe i drugi govoreći: 'Mna je  tvoja, gospodaru, donijela pet mna.' 
\par 19 I tomu reče: 'I ti budi  nad pet gradova!'" 
\par 20 "Treći, opet dođe govoreći: 'Gospodaru, evo ti tvoje  mne. Držao sam je pohranjenu u rupcu. 
\par 21 Bojao sam te se jer  si čovjek strog: uzimaš što nisi pohranio, žanješ što nisi posijao.'" 
\par 22 "Kaže mu: 'Iz tvojih te usta sudim, zli slugo! Znao si  da sam čovjek strog, da uzimam što nisam pohranio i žanjem što  nisam posijao? 
\par 23 Zašto onda nisi uložio moj novac u novčarnicu?  Ja bih ga po povratku podigao s dobitkom.' 
\par 24 Nato reče nazočnima:  'Uzmite od njega mnu i dajte onomu koji ih ima deset.' 
\par 25 Rekoše  mu: 'Gospodaru, ta već ima deset mna!' 
\par 26 Kažem vam: svakomu  koji ima još će se dati, a od onoga koji nema oduzet će se i  ono što ima. 
\par 27 A moje neprijatelje - one koji me ne htjedoše  za kralja - dovedite ovamo i smaknite ih pred mojim očima!'" 
\par 28 Rekavši to, nastavi put uzlazeći u Jeruzalem. 
\par 29 Kad se približi Betfagi i Betaniji, uz goru koja se zove  Maslinska, posla dvojicu učenika 
\par 30 govoreći: "Hajdete u selo  pred vama. Čim uđete u nj, naći ćete privezano magare koje još  nitko nije zajahao. Odriješite ga i dovedite. 
\par 31 Upita li vas  tko: 'Zašto driješite?', ovako recite: 'Gospodinu treba.'" 
\par 32 Oni koji bijahu poslani otiđoše i nađoše kako im bijaše  rekao. 
\par 33 I dok su driješili magare, rekoše im gospodari: "Što  driješite magare?" 
\par 34 Oni odgovore: "Gospodinu treba." 
\par 35 I  dovedoše ga Isusu i staviše svoje haljine na magare te posjednuše  Isusa. 
\par 36 I kuda bi on prolazio, prostirali bi po putu svoje haljine. 
\par 37 A kad se već bio približio obronku Maslinske gore, sve ono  mnoštvo učenika, puno radosti, poče iza glasa hvaliti Boga za  sva silna djela što ih vidješe: 
\par 38 "Blagoslovljen Kralj, Onaj koji dolazi u ime Gospodnje! Na nebu mir! Slava  na visinama!" 
\par 39 Nato mu neki farizeji iz mnoštva rekoše: "Učitelju, prekori  svoje učenike." 
\par 40 On odgovori: "Kažem vam, ako ovi ušute, kamenje  će vikati!" 
\par 41 Kad se približi i ugleda grad, zaplaka nad njim 
\par 42 govoreći:  "O kad bi i ti u ovaj dan spoznao što je za tvoj mir! 
\par 43 Ali  sada je sakriveno tvojim očima. Doći će dani na tebe kad će te  neprijatelji tvoji opkoliti opkopom, okružit će te i pritijesniti  odasvud. 
\par 44 Smrskat će o zemlju tebe i djecu tvoju  u tebi. I neće ostaviti u tebi ni kamena na kamenu zbog toga  što nisi upoznao časa svoga pohođenja." 
\par 45 Ušavši u Hram, stane izgoniti prodavače. 
\par 46 Kaže im:  "Pisano je: Dom moj bit će Dom molitve, a vi od njega  načiniste pećinu razbojničku!" 
\par 47 I danomice naučavaše u Hramu. A glavari su svećenički i pismoznanci tražili kako da ga  pogube, a tako i prvaci narodni, 
\par 48 ali nikako naći što da učine  jer je sav narod visio o njegovoj riječi. 


\chapter{20}

\par 1 Jednog dana dok je naučavao narod u Hramu i naviještao evanđelje, ispriječe se glavari svećenički i pismoznanci sa starješinama 
\par 2 pa mu dobace: "Reci nam kojom vlašću to činiš ili tko ti dade  tu vlast?" 
\par 3 On odgovori: "Upitat ću i ja vas. Recite mi: 
\par 4 krst  Ivanov bijaše li od Neba ili od ljudi?" 
\par 5 A oni smišljahu među  sobom: "Reknemo li 'od Neba', odvratit će 'Zašto mu ne povjerovaste?' 
\par 6 A reknemo li 'od ljudi', sav će nas narod kamenovati. Ta uvjeren  je da je Ivan prorok." 
\par 7 I odgovore da ne znaju odakle. 
\par 8 I  Isus reče njima: "Ni ja vama neću kazati kojom vlašću ovo činim." 
\par 9 Zatim uze narodu kazivati ovu prispodobu: "Čovjek posadi  vinograd, iznajmi ga vinogradarima i otputova na dulje vrijeme." 
\par 10 "Kada dođe doba, posla slugu vinogradarima da mu dadnu  od uroda vinogradskoga. No vinogradari ga istukoše i otposlaše  praznih ruku. 
\par 11 Nato on posla drugoga slugu. Ali oni i toga  istukoše, izružiše i otposlaše praznih ruku. 
\par 12 Posla i trećega.  A oni i njega izraniše i izbaciše." 
\par 13 "Nato reče gospodar vinograda: 'Što da učinim? Poslat  ću im sina svoga ljubljenoga. Njega će valjda poštovati.' 
\par 14 Ali  kada ga vinogradari ugledaju, stanu među sobom umovati: 'Ovo  je baštinik. Ubijmo ga da baština bude naša.' 
\par 15 Izbaciše ga  iz vinograda i ubiše." "Što će dakle učiniti s njima gospodar vinograda? 
\par 16 Doći će  i pogubiti te vinogradare i dati vinograd drugima." Koji ga slušahu  rekoše: "Bože sačuvaj!" 
\par 17 A on ih ošinu pogledom i reče: "A  što ono piše: Kamen što ga odbaciše graditelji postade kamen zaglavni? 
\par 18 Tko god padne na taj kamen, smrskat će se, a na koga  on padne, satrt će ga." 
\par 19 Pismoznanci i glavari svećenički gledahu da istog časa  stave ruke na nj, ali se pobojaše naroda. Dobro razumješe da  o njima kaza tu prispodobu. 
\par 20 Vrebajući na nj, poslaše uhode koji su se pravili pravednicima  da ga uhvate u riječi pa da ga predaju oblasti i vlasti upraviteljevoj. 
\par 21 Upitaše ga dakle: "Učitelju, znamo da pravo zboriš i naučavaš  te nisi pristran, nego po istini učiš putu Božjem. 
\par 22 Je li  nam dopušteno dati porez caru ili nije?" 
\par 23 Proničući njihovu  lukavost, reče im: 
\par 24 "Pokažite mi denar." "Čiju ima sliku i  natpis?" 
\par 25 A oni će: "Carevu." On im reče: "Stoga dajte caru  carevo, a Bogu Božje." 
\par 26 I ne mogoše ga uhvatiti u riječi pred narodom, nego umuknuše  zadivljeni njegovim odgovorom. 
\par 27 Pristupe mu neki od saduceja, koji niječu uskrsnuće.  Upitaše ga: 
\par 28 "Učitelju! Mojsije nam napisa: Umre li bez  djece čiji brat koji imaše ženu, neka njegov brat uzme  tu ženu te podigne porod bratu svomu. 
\par 29 Bijaše tako sedmero  braće. Prvi se oženi i umrije bez djece. 
\par 30 Drugi uze njegovu  ženu, 
\par 31 onda treći; i tako redom sva sedmorica pomriješe ne  ostavivši djece. 
\par 32 Naposljetku umrije i žena. 
\par 33 Kojemu će  dakle od njih ta žena pripasti o uskrsnuću? Jer sedmorica su  je imala za ženu." 
\par 34 Reče im Isus: "Djeca se ovog svijeta žene i udaju. 
\par 35 No  oni koji se nađoše dostojni onog svijeta i uskrsnuća od mrtvih  niti se žene niti udaju. 
\par 36 Zaista, ni umrijeti više ne mogu:  anđelima su jednaki i sinovi su Božji jer su sinovi uskrsnuća." 
\par 37 "A da mrtvi ustaju, naznači i Mojsije kad u odlomku o  grmu Gospodina zove Bogom Abrahamovim, Bogom Izakovim i Bogom  Jakovljevim. 
\par 38 A nije on Bog mrtvih, nego živih. Ta svi  njemu žive!" 
\par 39 Neki pismoznanci primijete: "Učitelju! Dobro si rekao!" 
\par 40 I nisu se više usuđivali upitati ga bilo što. 
\par 41 A on im reče: "Kako kažu da je Krist sin Davidov? 
\par 42 Ta  sam David veli u Knjizi psalama: Reče Gospod Gospodinu mojemu: 'Sjedi mi zdesna 
\par 43 dok ne položim neprijatelje tvoje za podnožje nogama tvojim!' 
\par 44 David ga dakle naziva Gospodinom. Kako mu je onda sin?" 
\par 45 I pred svim narodom reče svojim učenicima: 
\par 46 "Čuvajte  se pismoznanaca, koji rado hodaju u dugim haljinama, vole pozdrave  na trgovima, prva sjedala u sinagogama i pročelja na gozbama, 
\par 47 proždiru kuće udovičke, još pod izlikom dugih molitava.  Stići će ih to oštrija osuda." 


\chapter{21}

\par 1 Pogleda i vidje kako bogataši bacaju u riznicu svoje darove. 
\par 2 A ugleda i neku ubogu udovicu kako baca onamo dva novčića. 
\par 3 I reče: "Uistinu, kažem vam: ova je sirota udovica ubacila  više od sviju. 
\par 4 Svi su oni zapravo među darove ubacili od svog  suviška, a ona je od svoje sirotinje ubacila sav žitak što ga  imaše." 
\par 5 I dok su neki razgovarali o Hramu, kako ga resi divno  kamenje i zavjetni darovi, reče: 
\par 6 "Doći će dani u kojima se  od ovoga što motrite neće ostaviti ni kamen na kamenu nerazvaljen." 
\par 7 Upitaše ga: "Učitelju, a kada će to biti? I na koji se  znak to ima dogoditi?" 
\par 8 A on reče: "Pazite, ne dajte se zavesti.  Mnogi će doista doći u moje ime i govoriti: 'Ja sam' i: 'Vrijeme  se približilo!' Ne idite za njima. 
\par 9 A kad čujete za ratove  i pobune, ne prestrašite se. Doista treba da se to prije  dogodi, ali to još nije odmah svršetak." 
\par 10 Tada im kaza: "Narod će ustati protiv naroda i kraljevstvo  protiv kraljevstva. 
\par 11 I bit će velikih potresa i po raznim  mjestima gladi i pošasti; bit će strahota i velikih znakova s  neba." 
\par 12 "No prije svega toga podignut će na vas ruke i progoniti  vas, predavati vas u sinagoge i tamnice. Vući će vas pred kraljeve  i upravitelje zbog imena mojega. 
\par 13 Zadesit će vas to radi svjedočenja." 
\par 14 "Stoga uzmite k srcu: nemojte unaprijed smišljati obranu! 
\par 15 Ta ja ću vam dati usta i mudrost kojoj se neće moći suprotstaviti  niti oduprijeti nijedan vaš protivnik. 
\par 16 A predavat će vas  čak i vaši roditelji i braća, rođaci i prijatelji. Neke će od  vas i ubiti." 
\par 17 "Svi će vas zamrziti zbog imena mojega. 
\par 18 Ali ni vlas  vam s glave neće propasti. 
\par 19 Svojom ćete se postojanošću spasiti." 
\par 20 "Kad ugledate da vojska opkoljuje Jeruzalem, tada znajte:  približilo se njegovo opustošenje. 
\par 21 Koji se tada zateknu u  Judeji, neka bježe u gore; a koji u Gradu, neka ga napuste; koji  pak po poljima, neka se u nj ne vraćaju 
\par 22 jer to su dani  odmazde, da se ispuni sve što je pisano." 
\par 23 "Jao trudnicama i dojiljama u one dane jer bit će jad  velik na zemlji i gnjev nad ovim narodom. 
\par 24 Padat će od oštrice  mača, odvodit će ih kao roblje po svim narodima. I Jeruzalem  će gaziti pogani sve dok se ne navrše vremena pogana." 
\par 25 "I bit će znaci na suncu, mjesecu i zvijezdama, a na  zemlji bezizlazna tjeskoba naroda zbog huke mora i valovlja. 
\par 26 Izdisat će ljudi od straha i iščekivanja onoga što prijeti  svijetu. Doista, sile će se nebeske poljuljati. 
\par 27 Tada  će ugledati Sina Čovječjega gdje dolazi u oblaku s velikom  moći i slavom. 
\par 28 Kad se sve to stane zbivati, uspravite se  i podignite glave jer se približuje vaše otkupljenje." 
\par 29 I reče im prispodobu: "Pogledajte smokvu i sva stabla. 
\par 30 Kad već propupaju, i sami vidite i znate: blizu je već ljeto. 
\par 31 Tako i vi kad vidite da se to zbiva, znajte: blizu je kraljevstvo  Božje. 
\par 32 Zaista, kažem vam, ne, neće uminuti naraštaj ovaj  dok se sve ne zbude. 
\par 33 Nebo će i zemlja uminuti, ali moje riječi  ne, neće uminuti." 
\par 34 "Pazite na se da vam srca ne otežaju u proždrljivosti, pijanstvu i u životnim brigama te vas iznenada ne zatekne onaj  Dan 
\par 35 jer će kao zamka nadoći na sve žitelje po svoj zemlji." 
\par 36 "Stoga budni budite i u svako doba molite da uzmognete  umaći svemu tomu što se ima zbiti i stati pred Sina Čovječjega." 
\par 37 Danju je učio u Hramu, a noću bi izlazio i noćio na gori  zvanoj Maslinska. 
\par 38 A sav bi narod rano hrlio k njemu u Hram  da ga sluša. 


\chapter{22}

\par 1 Bližio se Blagdan beskvasnih kruhova zvan Pasha. 
\par 2 Glavari  svećenički i pismoznanci tražili su kako da Isusa smaknu jer  se bojahu naroda. 
\par 3 A Sotona uđe u Judu zvanog Iškariotski koji bijaše iz  broja dvanaestorice. 
\par 4 On ode i ugovori s glavarima svećeničkim  i zapovjednicima kako da im ga preda. 
\par 5 Oni se povesele i ugovore  da će mu dati novca. 
\par 6 On pristade. Otada je tražio priliku  da im ga preda mimo naroda. 
\par 7 Kada dođe Dan beskvasnih kruhova, u koji je trebalo žrtvovati  pashu, 
\par 8 posla Isus Petra i Ivana i reče: "Hajdete, pripravite  nam da blagujemo pashu." 
\par 9 Rekoše mu: "Gdje hoćeš da pripravimo?" 
\par 10 On im reče: "Evo, čim uđete u grad, namjerit ćete se na čovjeka  koji nosi krčag vode. Pođite za njim u kuću u koju uniđe 
\par 11 i  recite domaćinu te kuće: 'Učitelj veli: Gdje je svratište u kojem  bih blagovao pashu sa svojim učenicima?' 
\par 12 I on će vam pokazati  na katu veliko blagovalište prostrto: ondje pripravite." 
\par 13 Oni  odu, nađu kako im je rekao i priprave pashu. 
\par 14 Kada dođe čas, sjede Isus za stol i apostoli s njim. 
\par 15 I reče im: "Svom sam dušom čeznuo ovu pashu blagovati s vama  prije svoje muke. 
\par 16 Jer kažem vam, neću je više blagovati dok  se ona ne završi u kraljevstvu Božjem." 
\par 17 I uze čašu, zahvali i reče: "Uzmite je i razdijelite  među sobom. 
\par 18 Jer kažem vam, ne, neću više piti od roda trsova  dok kraljevstvo Božje ne dođe." 
\par 19 I uze kruh, zahvali, razlomi i dade im govoreći: "Ovo  je tijelo moje koje se za vas predaje. Ovo činite meni na spomen." 
\par 20 Tako i čašu, pošto večeraše, govoreći: "Ova čaša novi  je Savez u mojoj krvi koja se za vas prolijeva." 
\par 21 "A evo, ruka mog izdajice sa mnom je na stolu. 
\par 22 Sin  Čovječji, istina, ide kako je određeno, ali jao čovjeku onomu  koji ga predaje." 
\par 23 I oni se počeše ispitivati tko bi od njih mogao takvo što  učiniti. 
\par 24 Uto nasta među njima prepirka tko bi od njih bio najveći. 
\par 25 A on im reče: "Kraljevi gospoduju svojim narodima i vlastodršci  nazivaju sebe dobrotvorima. 
\par 26 Vi nemojte tako! Naprotiv, najveći  među vama neka bude kao najmlađi; i predstojnik kao poslužitelj. 
\par 27 Ta tko je veći? Koji je za stolom ili koji poslužuje? Zar  ne onaj koji je za stolom? A ja sam posred vas kao onaj koji  poslužuje." 
\par 28 "Da, vi ste sa mnom ustrajali u mojim kušnjama. 
\par 29 Ja  vam stoga u baštinu predajem kraljevstvo što ga je meni predao  moj Otac: 
\par 30 da jedete i pijete za mojim stolom u kraljevstvu  mojemu i sjedite na prijestoljima sudeći dvanaest plemena Izraelovih." 
\par 31 "Šimune, Šimune, evo Sotona zaiska da vas prorešeta kao  pšenicu. 
\par 32 Ali ja sam molio za tebe da ne malakše tvoja vjera.  Pa kad k sebi dođeš, učvrsti svoju braću." 
\par 33 Petar mu reče: "Gospodine, s tobom sam spreman i u tamnicu  i u smrt." 
\par 34 A Isus će mu: "Kažem ti, Petre, neće se danas  oglasiti pijetao dok triput ne zatajiš da me poznaš." 
\par 35 I reče: "Kad sam vas poslao bez kese i bez torbe i bez  sandala, je li vam što nedostajalo?" Oni odgovore: "Ništa." 
\par 36 Nato  će im: "No sada tko ima kesu, neka je uzme! Isto tako i torbu!  A koji nema, neka proda svoju haljinu i neka kupi sebi mač 
\par 37 jer  kažem vam, ono što je napisano treba se ispuniti na meni: Među  zlikovce bi ubrojen. Uistinu, sve što se odnosi na mene ispunja  se." 
\par 38 Oni mu rekoše: "Gospodine, evo ovdje dva mača!" Reče  im: "Dosta je!" 
\par 39 Tada iziđe te se po običaju zaputi na Maslinsku goru.  Za njim pođoše i njegovi učenici. 
\par 40 Kada dođe onamo, reče im:  "Molite da ne padnete u napast!" 
\par 41 I otrgnu se od njih koliko bi se kamenom dobacilo, pade  na koljena pa se molio: 
\par 42 "Oče! Ako hoćeš, otkloni ovu čašu  od mene. Ali ne moja volja, nego tvoja neka bude!" 
\par 43 A ukaza mu se anđeo s neba koji ga ohrabri. A kad je  bio u smrtnoj muci, usrdnije se molio. 
\par 44 I bijaše znoj njegov  kao kaplje krvi koje su padale na zemlju. 
\par 45 Usta od molitve, dođe učenicima i nađe ih snene od žalosti 
\par 46 pa im reče: "Što spavate? Ustanite! Molite da ne padnete  u napast!" 
\par 47 Dok je on još govorio, eto svjetine, a pred njom jedan  od dvanaestorice, zvani Juda. On se približi Isusu da ga poljubi. 
\par 48 Isus mu reče: "Juda, poljupcem Sina Čovječjeg predaješ?" 
\par 49 A oni oko njega, vidjevši što se zbiva, rekoše: "Gospodine, da udarimo mačem?" 
\par 50 I jedan od njih udari slugu velikoga  svećenika i odsiječe mu desno uho. 
\par 51 Isus odgovori: "Pustite!  Dosta!" Onda se dotače uha i zacijeli ga. 
\par 52 Nato Isus reče onima koji se digoše na nj, glavarima  svećeničkim, zapovjednicima hramskim i starješinama: "Kao na  razbojnika iziđoste s mačevima i toljagama! 
\par 53 Danomice bijah  s vama u Hramu i ne digoste ruke na me. No ovo je vaš čas i vlast  Tmina." 
\par 54 Uhvatiše ga dakle, odvedoše i uvedoše u dom velikoga  svećenika. Petar je išao za njim izdaleka. 
\par 55 A posred dvorišta  naložiše vatru i posjedaše uokolo. Među njih sjedne Petar. 
\par 56 Ugleda ga neka sluškinja gdje sjedi kraj vatre, oštro  ga pogleda i reče: "I ovaj bijaše s njim!" 
\par 57 A on zanijeka:  "Ne znam ga, ženo!" 
\par 58 Malo zatim opazi ga netko drugi i reče:  "I ti si od njih!" A Petar reče: "Čovječe, nisam!" 
\par 59 I nakon  otprilike jedne ure drugi neki navaljivaše: "Doista, i ovaj bijaše  s njim! Ta Galilejac je!" 
\par 60 A Petar će: "Čovječe, ne znam što  govoriš!"  I umah, dok je on još govorio, oglasi se pijetao. 
\par 61 Gospodin  se obazre i upre pogled u Petra, a Petar se spomenu riječi Gospodinove, kako mu ono reče: "Prije nego se danas pijetao oglasi, zatajit  ćeš me tri puta." 
\par 62 I iziđe te gorko zaplaka. 
\par 63 A ljudi koji su Isusa čuvali udarajući ga poigravali  se njime 
\par 64 i zastirući mu lice, zapitkivali ga: "Proreci tko  te udario!" 
\par 65 I mnogim se drugim pogrdama nabacivali na nj. 
\par 66 A kad se razdanilo, sabra se starješinstvo narodno, glavari  svećenički i pismoznanci te ga dovedoše pred svoje Vijeće 
\par 67 i  rekoše: "Ako si ti Krist, reci nam!" A on će im: "Ako vam reknem, nećete vjerovati; 
\par 68 ako vas zapitam, nećete odgovoriti. 
\par 69 No  od sada će Sin Čovječji sjedjeti zdesna Sile Božje." 
\par 70 Nato svi rekoše: "Ti si, dakle, Sin Božji!" On im reče: "Vi  velite! Ja jesam!" 
\par 71 Nato će oni: "Što nam još svjedočanstvo  treba? Ta sami smo čuli iz njegovih usta!" 


\chapter{23}

\par 1 I ustade sva ona svjetina. Odvedoše ga Pilatu 
\par 2 i stadoše  ga optuživati: "Ovoga nađosmo kako zavodi naš narod i brani davati  caru porez te za sebe tvrdi da je Krist, kralj." 
\par 3 Pilat ga upita: "Ti li si kralj židovski?" On mu odgovori:  "Ti kažeš!" 
\par 4 Tada Pilat reče glavarima svećeničkim i svjetini:  "Nikakve krivnje ne nalazim na ovom čovjeku!" 
\par 5 No oni navaljivahu:  "Buni narod naučavajući po svoj Judeji, počevši od Galileje pa  dovde!" 
\par 6 Čuvši to, Pilat propita da li je taj čovjek Galilejac. 
\par 7 Saznavši da je iz oblasti Herodove, posla ga Herodu, koji  i sam bijaše onih dana u Jeruzalemu. 
\par 8 A kad Herod ugleda Isusa, veoma se obradova jer ga je  već odavna želo vidjeti zbog onoga što je o njemu slušao te se  nadao od njega vidjeti koje čudo. 
\par 9 Postavljao mu je mnoga pitanja, ali mu Isus uopće nije odgovarao. 
\par 10 A stajahu ondje i glavari  svećenički i pismoznanci optužujući ga žestoko. 
\par 11 Herod ga  zajedno sa svojom vojskom prezre i ismija: obuče ga u bijelu  haljinu i posla natrag Pilatu. 
\par 12 Onoga se dana Herod i Pilat  sprijateljiše, jer prije bijahu neprijatelji. 
\par 13 A Pilat dade sazvati glavare svećeničke, vijećnike i  narod 
\par 14 te im reče: "Doveli ste mi ovoga čovjeka kao da buni  narod. Ja ga evo ispitah pred vama pa ne nađoh na njemu ni jedne  krivice za koju ga optužujete. 
\par 15 A ni Herod jer ga posla natrag  nama. Evo, on nije počinio ništa čime bi zaslužio smrt. 
\par 16 Kaznit  ću ga dakle i pustiti." 
\par 17 # 
\par 18 I povikaše svi uglas: "Smakni ovoga, a pusti nam Barabu!" 
\par 19 A taj bijaše bačen u tamnicu zbog neke pobune u gradu i ubojstva. 
\par 20 Pilat im stoga ponovno progovori hoteći osloboditi Isusa. 
\par 21 Ali oni vikahu: "Raspni, raspni ga!" 
\par 22 On im treći put  reče: "Ta što je on zla učinio? Ne nađoh na njemu smrtne krivice.  Kaznit ću ga dakle i pustiti." 
\par 23 Ali oni navaljivahu iza glasa  ištući da se razapne. I vika im bivala sve jača. 
\par 24 Pilat presudi da im bude što ištu. 
\par 25 Pusti onoga koji  zbog pobune i ubojstva bijaše bačen u tamnicu, koga su iskali, a Isusa preda njima na volju. 
\par 26 Kad ga odvedoše, uhvatiše nekog Šimuna Cirenca koji je  dolazio s polja i stave na nj križ da ga nosi za Isusom. 
\par 27 Za njim je išlo silno mnoštvo svijeta, napose žena, koje  su plakale i naricale za njim. 
\par 28 Isus se okrenu prema njima  pa im reče: "Kćeri Jeruzalemske, ne plačite nada mnom, nego plačite  nad sobom i nad djecom svojom. 
\par 29 Jer evo idu dani kad će se  govoriti: 'Blago nerotkinjama, utrobama koje ne rodiše i sisama  koje ne dojiše.' 
\par 30 Tad će početi govoriti gorama: 'Padnite  na nas!' i bregovima: 'Pokrijte nas!' 
\par 31 Jer ako se tako  postupa sa zelenim stablom, što li će biti sa suhim?" 
\par 32 A vodili su i drugu dvojicu, zločince, da ih s njime pogube. 
\par 33 I kada dođoše na mjesto zvano Lubanja, ondje razapeše  njega i te zločince, jednoga zdesna, drugoga slijeva. 
\par 34 A Isus je govorio: "Oče, oprosti im, ne znaju što čine!"  I razdijeliše među se haljine njegove bacivši kocke. 
\par 35 Stajao je ondje narod i promatrao. A podrugivali  se i glavari govoreći: "Druge je spasio, neka spasi sam sebe  ako je on Krist Božji, Izabranik!" 
\par 36 Izrugivali ga i vojnici, prilazili mu i nudili ga octom 
\par 37 govoreći: "Ako si ti kralj židovski, spasi sam sebe!" 
\par 38 A bijaše i natpis ponad njega: "Ovo je kralj židovski." 
\par 39 Jedan ga je od obješenih zločinaca pogrđivao: "Nisi li  ti Krist? Spasi sebe i nas!" 
\par 40 A drugi ovoga prekoravaše: "Zar  se ne bojiš Boga ni ti, koji si pod istom osudom? 
\par 41 Ali mi  po pravdi jer primamo što smo djelima zaslužili, a on - on ništa  opako ne učini." 
\par 42 Onda reče: "Isuse, sjeti me se kada dođeš  u kraljevstvo svoje." 
\par 43 A on će mu: "Zaista ti kažem: danas  ćeš biti sa mnom u raju!" 
\par 44 Bijaše već oko šeste ure kad nasta tama po svoj zemlji  - sve do ure devete, 
\par 45 jer sunce pomrča, a hramska se zavjesa  razdrije po sredini. 
\par 46 I povika Isus iza glasa: "Oče, u ruke tvoje predajem duh  svoj!" To rekavši, izdahnu. 
\par 47 Kad satnik vidje što se zbiva, stane slaviti Boga: "Zbilja, čovjek ovaj bijaše pravednik!" 
\par 48 I kad je sav svijet koji  se zgrnuo na taj prizor vidio što se zbiva, vraćao se bijući  se u prsa. 
\par 49 Stajahu podalje i gledahu to svi znanci njegovi  i žene koje su za njim išle iz Galileje. 
\par 50 I dođe čovjek imenom Josip, vijećnik, čovjek čestit i  pravedan; 
\par 51 on ne privoli njihovoj odluci i postupku. Bijaše  iz Arimateje, grada judejskoga i iščekivaše kraljevstvo Božje. 
\par 52 Taj dakle pristupi Pilatu i zaiska tijelo Isusovo. 
\par 53 Zatim ga skinu, povi u platno i položi u grob isklesan u  koji još ne bijaše nitko položen. 
\par 54 Bijaše dan Priprave; subota je svitala. 
\par 55 A pratile  to žene koje su s Isusom došle iz Galileje: motrile grob i kako  je položeno tijelo njegovo. 
\par 56 Zatim se vrate i priprave miomirise  i pomasti. U subotu mirovahu po propisu. 



\chapter{24}

\par 1 Prvoga dana u tjednu, veoma rano, dođoše one na grob s miomirisima  što ih pripraviše. 
\par 2 Kamen nađoše otkotrljan od groba. 
\par 3 Uđoše, ali ne nađoše tijela Gospodina Isusa. 
\par 4 I dok su stajale zbunjene  nad tim, gle, dva čovjeka u blistavoj odjeći stadoše do njih. 
\par 5 Zastrašene obore lica k zemlji, a oni će im: "Što tražite  Živoga među mrtvima? 
\par 6 Nije ovdje, nego uskrsnu! Sjetite se  kako vam je govorio dok je još bio u Galileji: 
\par 7 'Treba da Sin  Čovječji bude predan u ruke grešnika, i raspet, i treći dan da  ustane.'" 
\par 8 I sjetiše se one riječi njegovih, 
\par 9 vratiše se  s groba te javiše sve to jedanaestorici i svima drugima. 
\par 10 A bile su to: Marija Magdalena, Ivana i Marija Jakovljeva.  I ostale zajedno s njima govorahu to apostolima, 
\par 11 ali njima  se te riječi pričiniše kao tlapnja, te im ne vjerovahu. 
\par 12 A Petar usta i potrča na grob. Sagnuvši se, opazi samo  povoje. I vrati se kući čudeći se tome što se zbilo. 
\par 13 I gle, dvojica su od njih toga istog dana putovala u  selo koje se zove Emaus, udaljeno od Jeruzalema šezdeset stadija. 
\par 14 Razgovarahu međusobno o svemu što se dogodilo. 
\par 15 I dok  su tako razgovarali i raspravljali, približi im se Isus i pođe  s njima. 
\par 16 Ali prepoznati ga - bijaše uskraćeno njihovim očima. 
\par 17 On ih upita: "Što to putem pretresate među sobom?" Oni se  snuždeni zaustave 
\par 18 te mu jedan od njih, imenom Kleofa, odgovori:  "Zar si ti jedini stranac u Jeruzalemu te ne znaš što se u njemu  dogodilo ovih dana?" 
\par 19 A on će: "Što to?" Odgovore mu: "Pa ono s Isusom Nazarećaninom, koji bijaše prorok - silan na djelu i na riječi pred Bogom i  svim narodom: 
\par 20 kako su ga glavari svećenički i vijećnici naši  predali da bude osuđen na smrt te ga razapeli. 
\par 21 A mi se nadasmo  da je on onaj koji ima otkupiti Izraela. Ali osim svega toga  ovo je već treći dan što se to dogodilo. 
\par 22 A zbuniše nas i  žene neke od naših: u praskozorje bijahu na grobu, 
\par 23 ali nisu  našle njegova tijela pa dođoše te rekoše da su im se ukazali  anđeli koji su rekli da je on živ. 
\par 24 Odoše nato i neki naši  na grob i nađoše kako žene rekoše, ali njega ne vidješe." 
\par 25 A  on će im: "O bezumni i srca spora da vjerujete što god su proroci  navijestili! 
\par 26 Nije li trebalo da Krist sve to pretrpi te uđe  u svoju slavu?" 
\par 27 Počevši tada od Mojsija i svih proroka, protumači  im što u svim Pismima ima o njemu. 
\par 28 Uto se približe selu kamo  su išli, a on kao da htjede dalje. 
\par 29 No oni navaljivahu: "Ostani  s nama jer zamalo će večer i dan je na izmaku!" I uniđe da ostane  s njima. 
\par 30 Dok bijaše s njima za stolom, uze kruh, izreče blagoslov, razlomi te im davaše. 
\par 31 Uto im se otvore oči te ga prepoznaše, a on im iščeznu s očiju. 
\par 32 Tada rekoše jedan drugome: "Nije  li gorjelo srce u nama dok nam je putem govorio, dok nam je otkrivao  Pisma?" 
\par 33 U isti se čas digoše i vratiše u Jeruzalem. Nađoše  okupljenu jedanaestoricu i one koji bijahu s njima. 
\par 34 Oni im  rekoše: "Doista uskrsnu Gospodin i ukaza se Šimunu!" 
\par 35 Nato  oni pripovjede ono s puta i kako ga prepoznaše u lomljenju kruha. 
\par 36 Dok su oni o tom razgovarali, stane Isus posred njih  i reče im: "Mir vama!" 
\par 37 Oni, zbunjeni i prestrašeni, pomisliše  da vide duha. 
\par 38 Reče im Isus: "Zašto se prepadoste? Zašto vam  sumnje obuzimaju srce? 
\par 39 Pogledajte ruke moje i noge! Ta ja  sam! Opipajte me i vidite jer duh tijela ni kostiju nema kao  što vidite da ja imam." 
\par 40 Rekavši to, pokaza im ruke i noge. 
\par 41 I dok oni od radosti  još nisu vjerovali, nego se čudom čudili, on im reče: "Imate  li ovdje što za jelo?" 
\par 42 Oni mu pruže komad pečene ribe. 
\par 43 On  uzme i pred njima pojede. 
\par 44 Nato im reče: "To je ono što sam vam govorio dok sam  još bio s vama: treba da se ispuni sve što je u Mojsijevu Zakonu, u Prorocima i Psalmima o meni napisano." 
\par 45 Tada im otvori  pamet da razumiju Pisma 
\par 46 te im reče: "Ovako je pisano: 'Krist  će trpjeti i treći dan ustati od mrtvih, 
\par 47 i u njegovo će se  ime propovijedati obraćenje i otpuštenje grijeha po svim narodima  počevši od Jeruzalema.' 
\par 48 Vi ste tomu svjedoci. 
\par 49 I evo,  ja šaljem na vas Obećanje Oca svojega. Ostanite zato u gradu  dok se ne obučete u Silu odozgor." 
\par 50 Zatim ih izvede do Betanije, podiže ruke pa ih blagoslovi. 
\par 51 I dok ih blagoslivljaše, rasta se od njih i uznesen bi na  nebo. 
\par 52 Oni mu se ničice poklone pa se s velikom radosti vrate  u Jeruzalem 
\par 53 te sve vrijeme u Hramu blagoslivljahu Boga. 




\end{document}