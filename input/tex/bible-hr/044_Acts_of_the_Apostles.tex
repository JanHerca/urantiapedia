\begin{document}

\title{Djela apostolska}


\chapter{1}

\par 1 Prvu sam knjigu, Teofile, sastavio o svemu što je Isus činio  i učio 
\par 2 do dana kad je uznesen pošto je dao upute apostolima  koje je izabrao po Duhu Svetome. 
\par 3 Njima je poslije svoje muke mnogim dokazima pokazao da  je živ, četrdeset im se dana ukazivao i govorio o kraljevstvu  Božjem. 
\par 4 I dok je jednom s njima blagovao, zapovjedi im da ne napuštaju  Jeruzalema, nego neka čekaju Obećanje Očevo "koje čuste od mene: 
\par 5 Ivan je krstio vodom, a vi ćete naskoro nakon ovih dana biti  kršteni Duhom Svetim." 
\par 6 Nato ga sabrani upitaše: "Gospodine, hoćeš li u ovo vrijeme  Izraelu opet uspostaviti kraljevstvo?" 
\par 7 On im odgovori: "Nije  vaše znati vremena i zgode koje je Otac podredio svojoj vlasti. 
\par 8 Nego primit ćete snagu Duha Svetoga koji će sići na vas i  bit ćete mi svjedoci u Jeruzalemu, po svoj Judeji i Samariji  i sve do kraja zemlje." 
\par 9 Kada to reče, bi uzdignut njima naočigled i oblak ga ote  njihovim očima. 
\par 10 I dok su netremice gledali kako on odlazi  na nebo, gle, dva čovjeka stadoše kraj njih u bijeloj odjeći 
\par 11 i rekoše im: "Galilejci, što stojite i gledate u nebo? Ovaj  Isus koji je od vas uznesen na nebo isto će tako doći kao što  ste vidjeli da odlazi na nebo." 
\par 12 Onda se vratiše u Jeruzalem s brda zvanoga Maslinsko, koje je blizu Jeruzalema, udaljeno jedan subotni hod. 
\par 13 I  pošto uđu u grad, uspnu se u gornju sobu gdje su boravili: Petar  i Ivan i Jakov i Andrija, Filip i Toma, Bartolomej i Matej, Jakov  Alfejev i Šimun Revnitelj i Juda Jakovljev - 
\par 14 svi oni bijahu  jednodušno postojani u molitvi sa ženama, i Marijom, majkom Isusovom, i braćom njegovom. 
\par 15 U one dane ustade Petar među braćom - a bijaše sakupljenog  naroda oko sto i dvadeset duša - i reče: 
\par 16 "Braćo! Trebalo  je da se ispuni Pismo što ga na usta Davidova proreče Duh Sveti  o Judi koji bijaše vođa onih što uhvatiše Isusa. 
\par 17 A Juda se  ubrajao među nas i imao udio u ovoj službi. 
\par 18 On, eto, steče  predio cijenom nepravednosti pa se stropošta, raspuče po sredini  i razli mu se sva utroba. 
\par 19 I svim je Jeruzalemcima znano da  se onaj predio njihovim jezikom zove Akeldama, to jest Predio  smrti. 
\par 20 Pisano je doista u Knjizi psalama: Njegova kuća nek opusti, nek ne bude stanovnika u njoj! Njegovo nadgledništvo nek dobije drugi! 
\par 21 Jedan dakle od ovih ljudi što bijahu s nama za sve vrijeme  što je među nama živio Gospodin Isus - 
\par 22 počevši od krštenja  Ivanova pa sve do dana kad bi uzet od nas - treba da bude svjedokom  njegova uskrsnuća. 
\par 23 I postaviše dvojicu: Josipa koji se zvao Barsaba a prozvao  se Just, i Matiju. 
\par 24 Onda se pomoliše: "Ti, Gospodine, poznavaoče svih srdaca, pokaži koga si od  ove dvojice izabrao 
\par 25 da primi mjesto ove apostolske službe  kojoj se iznevjeri Juda da ode na svoje mjesto." 
\par 26 Onda baciše kocke i kocka pade na Matiju; tako bi pribrojen  jedanaestorici apostola. 


\chapter{2}

\par 1 Kad je napokon došao dan Pedesetnice, svi su bili zajedno na  istome mjestu. 
\par 2 I eto iznenada šuma s neba, kao kad se digne  silan vjetar. Ispuni svu kuću u kojoj su bili. 
\par 3 I pokažu im  se kao neki ognjeni razdijeljeni jezici te siđe po jedan na svakoga  od njih. 
\par 4 Svi se napuniše Duha Svetoga i počeše govoriti drugim  jezicima, kako im već Duh davaše zboriti. 
\par 5 A u Jeruzalemu su boravili Židovi, ljudi pobožni iz svakog  naroda pod nebom. 
\par 6 Pa kad nasta ona huka, strča se mnoštvo  i smetÄe jer ih je svatko čuo govoriti svojim jezikom. 
\par 7 Svi  su bili izvan sebe i divili se govoreći: "Gle! Nisu li svi ovi  što govore Galilejci? 
\par 8 Pa kako to da ih svatko od nas čuje  na svojem materinskom jeziku? 
\par 9 Parti, Međani, Elamljani, žitelji  Mezopotamije, Judeje i Kapadocije, Ponta i Azije, 
\par 10 Frigije  i Pamfilije, Egipta i krajeva libijskih oko Cirene, pridošlice  Rimljani, 
\par 11 Židovi i sljedbenici, Krećani i Arapi - svi ih  mi čujemo gdje našim jezicima razglašuju veličanstvena djela  Božja." 
\par 12 Svi su izvan sebe zbunjeno jedan drugog pitali: "Što  bi to moglo biti?" 
\par 13 Drugi su pak, podrugujući se, govorili:  "Slatkog su se vina ponapili!" 
\par 14 A Petar zajedno s jedanaestoricom ustade, podiže glas  i prozbori: "Židovi i svi što boravite u Jeruzalemu, ovo znajte  i riječi mi poslušajte: 
\par 15 Nisu ovi pijani, kako vi mislite  - ta istom je treća ura dana - 
\par 16 nego to je ono što je rečeno  po proroku Joelu: 
\par 17 "U posljednje dane, govori Bog: Izlit ću Duha svoga na svako tijelo i proricat će vaši sinovi i kćeri, vaši će mladići gledati viđenja, a starci vaši sne sanjati. 
\par 18 Čak ću i na sluge i sluškinje svoje izliti Duha svojeg u dane one i proricat će. 
\par 19 Pokazat ću čudesa na nebu gore  i znamenja na zemlji dolje, krv i oganj i stupove dima. 
\par 20 Sunce će se prometnut u tminu, a mjesec u krv prije nego svane Dan Gospodnji velik i slavan. 
\par 21 I tko god prizove ime Gospodnje bit će spašen." 
\par 22 "Izraelci, čujte ove riječi: Isusa Nazarećanina, čovjeka  kojega Bog pred vama potvrdi silnim djelima, čudesima i znamenjima  koja, kao što znate, po njemu učini među vama - 
\par 23 njega, predana  po odlučenu naumu i promislu Božjem, po rukama bezakonika razapeste  i pogubiste. 
\par 24 Ali Bog ga uskrisi oslobodivši ga grozote smrti  jer ne bijaše moguće da ona njime ovlada. 
\par 25 David doista za  nj kaže: Gospodin mi je svagda pred očima jer mi je zdesna da ne posrnem. 
\par 26 Stog mi se raduje srce i kliče jezik, pa i tijelo mi spokojno počiva. 
\par 27 Jer mi nećeš ostaviti dušu u Podzemlju ni dati da pravednik tvoj truleži ugleda. 
\par 28 Pokazat ćeš mi stazu života, ispuniti me radošću lica svoga. 
\par 29 Braćo, dopustite da vam otvoreno kažem: praotac je David  umro, pokopan je i eno mu među nama groba sve do današnjeg dana. 
\par 30 Ali kako je bio prorok i znao da mu se zakletvom zakle  Bog plod utrobe njegove posaditi na prijestolje njegovo, 
\par 31 unaprijed je vidio i navijestio uskrsnuće Kristovo: Nije ostavljen u Podzemlju niti mu tijelo truleži ugleda. 
\par 32 Toga Isusa uskrisi Bog! Svi smo mi tomu svjedoci. 
\par 33 Desnicom  dakle Božjom uzvišen, primio je od Oca Obećanje, Duha Svetoga, i izlio ga kako i sami gledate i slušate. 
\par 34 Ta David nije  bio uznesen na nebesa, a veli: Reče Gospodin Gospodinu mojemu: 'Sjedi mi zdesna' 
\par 35 dok ne položim neprijatelje tvoje za podnožje nogama tvojim! 
\par 36 Pouzdano dakle neka znade sav dom Izraelov da je toga  Isusa kojega vi razapeste Bog učinio i Gospodinom i Kristom." 
\par 37 Kad su to čuli, duboko potreseni rekoše Petru i drugim  apostolima: "Što nam je činiti, braćo?" 
\par 38 Petar će im: "Obratite  se i svatko od vas neka se krsti u ime Isusa Krista da vam se  oproste grijesi i primit ćete dar, Duha Svetoga. 
\par 39 Ta za vas  je ovo obećanje i za djecu vašu i za sve one izdaleka, koje  pozove Gospodin Bog naš." 
\par 40 I mnogim je drugim riječima još svjedočio i hrabrio ih:  "Spasite se od naraštaja ovog opakog!" 
\par 41 I oni prigrliše riječ  njegovu i krstiše se te im se u onaj dan pridruži oko tri tisuće  duša. 
\par 42 Bijahu postojani u nauku apostolskom, u zajedništvu,  lomljenju kruha i molitvama. 
\par 43 Strahopoštovanje obuzimaše svaku  dušu: apostoli su činili mnoga čudesa i znamenja. 
\par 44 Svi koji  prigrliše vjeru bijahu združeni i sve im bijaše zajedničko. 
\par 45 Sva  bi imanja i dobra prodali porazdijelili svima kako bi tko trebao. 
\par 46 Svaki bi dan jednodušno i postojano hrlili u Hram, u kućama  bi lomili kruh te u radosti i prostodušnosti srca zajednički  uzimali hranu 
\par 47 hvaleći Boga i uživajući naklonost svega naroda.  Gospodin je pak danomice zajednici pridruživao spasenike. 


\chapter{3}

\par 1 Petar i Ivan uzlazili su u Hram na devetu molitvenu uru. 
\par 2 Upravo  su donosili nekog čovjeka, hroma od majčine utrobe; njega bi  svaki dan postavljali kod hramskih vrata, zvanih Divna, da prosi  milostinju od onih koji ulaze u Hram. 
\par 3 On ugleda Petra i Ivana  upravo kad zakoračiše u Hram te zamoli milostinju. 
\par 4 Petar ga  zajedno s Ivanom prodorno pogleda i reče: "Pogledaj u nas!" 
\par 5 Dok  ih je molećivo motrio očekujući od njih nešto dobiti, 
\par 6 reče  mu Petar: "Srebra i zlata nema u mene, ali što imam - to ti dajem:  u ime Isusa Krista Nazarećanina hodaj!" 
\par 7 I uhvativši ga za  desnu ruku, pridiže ga: umah mu omoćaše noge i gležnjevi 
\par 8 pa  skoči, uspravi se, stane hodati te uđe s njima u Hram hodajući, poskakujući i hvaleći Boga. 
\par 9 Sav ga narod vidje kako hoda  i hvali Boga. 
\par 10 Razabraše da je to on - onaj koji je na Divnim  vratima Hrama prosio milostinju - i ostadoše zapanjeni i izvan  sebe zbog onoga što se s njim dogodilo. 
\par 11 Kako se pak on držao Petra i Ivana, sav se narod zapanjen  strča k njima u trijem zvani Salomonov. 
\par 12 Kada to vidje Petar, obrati se narodu: "Izraelci, što se ovomu čudite? Ili što nas  gledate kao da smo svojom snagom ili pobožnošću postigli da ovaj  prohoda? 
\par 13 Bog Abrahamov, Izakov i Jakovljev, Bog otaca  naših, proslavi slugu svoga, Isusa kojega vi predadoste i  kojega se odrekoste pred Pilatom kad već bijaše odlučio pustiti  ga. 
\par 14 Vi se odrekoste Sveca i Pravednika, a izmoliste da vam  se daruje ubojica. 
\par 15 Začetnika života ubiste. Ali Bog ga uskrisi  od mrtvih, čemu smo mi svjedoci." 
\par 16 "I po vjeri u njegovo ime, to je ime dalo snagu ovomu  kojega gledate i poznate: vjera u Njega vratila je ovomu potpuno  zdravlje naočigled vas sviju." 
\par 17 "I sada, braćo, znam da ste ono uradili iz neznanja kao  i glavari vaši. 
\par 18 Ali Bog tako ispuni što unaprijed navijesti  po ustima svih proroka: da će njegov Pomazanik trpjeti. 
\par 19 Pokajte  se dakle i obratite da se izbrišu grijesi vaši 
\par 20 pa od Gospodina  dođu vremena rashlade te on pošalje vama unaprijed namijenjenog  Pomazanika, Isusa." 
\par 21 Njega treba da nebo pridrži do vremena uspostave svega  što obeća Bog na usta svetih proroka svojih odvijeka." 
\par 22 "Mojsije tako reče: Proroka poput mene od vaše braće  podignut će vam Gospodin, Bog vaš. Njega slušajte u svemu što  vam god reče. 
\par 23 I svaka duša koja ne posluša toga proroka, neka se iskorijeni iz naroda." 
\par 24 "I svi Proroci koji su - od Samuela dalje - govorili, također su navijestili ove dane." 
\par 25 "Vi ste sinovi proroka i Saveza koji sklopi Bog s ocima  vašim govoreći Abrahamu: Potomstvom će se tvojim blagoslivljati  sva plemena zemlje. 
\par 26 Vama najprije podiže Bog Slugu svoga  i posla ga blagoslivljati vas da se svatko obrati od opačina  svojih." 


\chapter{4}

\par 1 Dok su oni još govorili narodu, priđu im svećenici, hramski  zapovjednik i saduceji, 
\par 2 ozlovoljeni što uče narod i navješćuju  - u Isusu - uskrsnuće od mrtvih; 
\par 3 pograbe ih i bace u tamnicu  do sutra jer već bijaše večer. 
\par 4 Ipak mnogi od onih koji su  čuli Riječ, povjerovaše te broj vjernika poraste nekako do pet  tisuća. 
\par 5 Sutradan se sastadoše u Jeruzalemu glavari, starješine  i pismoznanci - 
\par 6 i veliki svećenik Ana, i Kajfa, i Ivan, i  Aleksandar, i svi od roda velikosvećeničkoga. 
\par 7 Izvedoše apostole  preda se pa ih stadoše ispitivati: "Kojom snagom ili po kojem  imenu vi to učiniste?" 
\par 8 Onda Petar pun Duha Svetoga reče: "Glavari narodni i starješine! 
\par 9 Zar mi danas odgovaramo zbog dobra djela učinjena bolesnu  čovjeku? Po kome je ovaj spašen? 
\par 10 Neka bude znano svima vama  i svemu narodu Izraelovu: po imenu Isusa Krista Nazarećanina, kojega ste vi raspeli, a kojega Bog uskrisi od mrtvih! Po njemu  ovaj stoji pred vama zdrav! 
\par 11 On je onaj kamen koji  vi graditelji odbaciste, ali koji postade kamen zaglavni. 
\par 12 I nema ni u kome drugom spasenja. Nema uistinu pod nebom  drugoga imena dana ljudima po kojemu se možemo spasiti." 
\par 13 Kad vidješe neustrašivost Petrovu i Ivanovu, a znajući  da su to ljudi nepismeni i neuki, bijahu u čudu; znali su ih, da bijahu s Isusom, ali 
\par 14 videći gdje s njima stoji izliječeni  čovjek, nisu mogli ništa protusloviti. 
\par 15 Stoga zapovjediše  da izađu iz vijećnice pa stadoše raspravljati: 
\par 16 "Što ćemo  s tim ljudima? Ta učinili su očit znak, poznat svim Jeruzalemcima, ne možemo ga nijekati; 
\par 17 ali da se još više ne razglasi u  narod, zaprijetimo im da nikomu živom o tom Imenu više ne govore." 
\par 18 Pozvaše ih i zapovjediše im da podnipošto ne zbore niti  naučavaju u ime Isusovo. 
\par 19 Ali im Petar i Ivan odgovoriše:  "Sudite je li pred Bogom pravo slušati radije vas nego Boga. 
\par 20 Mi doista ne možemo ne govoriti što vidjesmo i čusmo." 
\par 21 Ali  oni ne našavši kako da ih kazne, opet im zaprijete pa ih otpuste  poradi naroda jer su svi slavili Boga zbog onoga što se dogodilo. 
\par 22 Jer čovjeku na kom se dogodi čudo ozdravljenja bijaše više  od četrdeset godina. 
\par 23 Otpušteni, odoše svojima i javiše što im rekoše veliki  svećenici i starješine. 
\par 24 Kad su oni to čuli, jednodušno podigoše  glas k Bogu i rekoše: "Gospodine, ti si stvorio nebo i zemlju i more i sve što je u njima! 
\par 25 Ti si na usta oca našega, sluge svoga Davida, po Duhu Svetom rekao: Zašto se bune narodi, zašto puci ludosti snuju? 
\par 26 Ustaju kraljevi zemaljski, Knezovi se rotÄe protiv Gospodina i protiv Pomazanika njegova. 
\par 27 RotÄe se, uistinu, u ovome gradu na svetog Slugu tvoga Isusa, kog pomaza, rotÄe se Herod i Poncije Pilat zajedno s narodima i pucima izraelskim 
\par 28 da učine što tvoja ruka i tvoja volja predodredi da se zbude. 
\par 29 I evo sada, Gospodine, promotri prijetnje njihove i daj slugama svojim sa svom smjelošću navješćivati riječ tvoju! 
\par 30 Pruži ruku svoju da bude ozdravljenja, znamenja i čudesa po imenu svetoga Sluge tvoga Isusa." 
\par 31 I pošto se pomoliše, potrese se mjesto gdje bijahu  sabrani, i svi se napuniše Duha Svetoga te stanu navješćivati  riječ Božju smjelo. 
\par 32 U mnoštva onih što prigrliše vjeru bijaše jedno srce  i jedna duša. I nijedan od njih nije svojim zvao ništa od onoga  što je imao, nego im sve bijaše zajedničko. 
\par 33 Apostoli pak  velikom silom davahu svjedočanstvo o uskrsnuću Gospodina Isusa  i svi uživahu veliku naklonost. 
\par 34 Doista, nitko među njima  nije oskudijevao jer koji bi god posjedovali zemljišta ili kuće, prodavali bi ih i utržak donosili 
\par 35 i stavljali pred noge  apostolima. A dijelilo se svakomu koliko je trebao. 
\par 36 A Josip, od apostola prozvan Barnaba, što znači Sin utjehe, levit, rodom Cipranin, 
\par 37 posjedovaše jednu njivu; proda je  pa donese novac i postavi pred noge apostolima. 


\chapter{5}

\par 1 Neki pak čovjek po imenu Ananija, zajedno sa svojom ženom Safirom  proda imanje 
\par 2 pa u dogovoru sa ženom odvoji nešto od utrška, a samo jedan dio donese i postavi pred noge apostolima. 
\par 3 Petar  mu reče: "Ananija, zašto ti Sotona ispuni srce te si slagao Duhu  Svetomu i odvojio od utrška imanja? 
\par 4 Da je ostalo neprodano, ne bi li tvoje ostalo; i jednoć prodano, nije li u tvojoj vlasti?  Zašto si se na takvo što odlučio? Nisi slagao ljudima, nego Bogu!" 
\par 5 Kako Ananija ču te riječi, sruši se i izdahnu. I silan strah  spopade sve koji su to čuli. 
\par 6 Nato ustanu mladići, poviju ga, iznesu i pokopaju. 
\par 7 Nakon otprilike tri sata uđe njegova žena ne znajući što  se dogodilo. 
\par 8 Petar joj reče: "Reci mi, jeste li za toliko  dali imanje?" Ona odgovori: "Da, za toliko." 
\par 9 A Petar će joj:  "Što vam bi da se složiste iskušati Duha Gospodnjega? Eto na  vratima nogu onih koji ti pokopaše muža! I tebe će iznijeti!" 
\par 10 Ona se umah sruši do njegovih nogu i izdahnu. Oni mladići  uđu, nađu je mrtvu, iznesu je i pokopaju uz muža. 
\par 11 I silan strah spopade cijelu Crkvu i sve koji su to čuli. 
\par 12 Po rukama se apostolskim događala mnoga znamenja i čudesa  u narodu. Svi su se jednodušno okupljali u Trijemu Salomonovu. 
\par 13 Nitko se drugi nije usuđivao pridružiti im se, ali ih je  narod veličao. 
\par 14 I sve se više povećavalo mnoštvo muževa i  žena što vjerovahu Gospodinu 
\par 15 tako da su na trgove iznosili  bolesnike i postavljali ih na ležaljkama i posteljama ne bi li, kad Petar bude prolazio, bar sjena njegova osjenila kojega od  njih. 
\par 16 A slijegalo bi se i mnoštvo iz gradova oko Jeruzalema:  donosili bi bolesnike i opsjednute od nečistih duhova, i svi  bi ozdravljali. 
\par 17 Onda se podiže veliki svećenik i sve njegove pristaše  - sljedba saducejska. 
\par 18 Puni zavisti, pohvataju apostole i  strpaju ih u javnu tamnicu. 
\par 19 Ali anđeo Gospodnji noću otvori  vrata tamnice, izvede ih i reče: 
\par 20 "Pođite i postojano u Hramu  navješćujte narodu sve riječi Života ovoga." 
\par 21 Poslušni, u  praskozorje su ušli u Hram te naučavali. Uto stiže veliki svećenik i njegove pristaše, sazovu Vijeće  i sve starješinstvo sinova Izraelovih pa pošalju u zatvor da  ih dovedu. 
\par 22 Kad stražari stigoše onamo, ne nađoše ih u tamnici  pa se vrate i jave: 
\par 23 "Zatvor smo našli sa svom pomnjom zatvoren  i čuvare na straži pred vratima, ali kad smo otvorili, nikoga  unutra ne nađosmo." 
\par 24 Kad su hramski zapovjednik i veliki svećenici čuli te  riječi, u nedoumici su se pitali što bi to moglo biti. 
\par 25 Nato  netko pristigne i dojavi im: "Eno, ljudi koje baciste u tamnicu, u Hramu stoje i uče narod." 
\par 26 Tada zapovjednik sa stražarima  ode te ih dovede - ne na silu jer se bojahu da ih narod ne kamenuje. 
\par 27 Dovedoše ih i privedoše pred Vijeće. Veliki ih svećenik  zapita: 
\par 28 "Nismo li vam strogo zabranili učiti u to Ime? A  vi ste eto napunili Jeruzalem svojim naukom i hoćete na nas navući  krv toga čovjeka." 
\par 29 Petar i apostoli odvrate: "Treba se većma pokoravati  Bogu negoli ljudima! 
\par 30 Bog otaca naših uskrisi Isusa kojega  vi smakoste objesivši ga na drvo. 
\par 31 Njega Bog desnicom  svojom uzvisi za Začetnika i Spasitelja da obraćenjem podari  Izraela i oproštenjem grijeha. 
\par 32 I mi smo svjedoci tih događaja  i Duh Sveti kojega dade Bog onima što mu se pokoravaju." 
\par 33 Nato se oni razgnjeviše i htjedoše ih ubiti. 
\par 34 Ali  ustade u Vijeću neki farizej imenom Gamaliel, zakonoznanac, kojega  je poštovao sav narod. On zapovjedi da ljude načas izvedu 
\par 35 pa  će vijećnicima: "Izraelci, dobro promislite što ćete s tim ljudima. 
\par 36 Ta prije nekog vremena podiže se Teuda tvrdeći da je netko, i uza nj prista oko četiri stotine ljudi. Bi smaknut i sve mu  se pristaše razbjegoše i netragom ih nesta. 
\par 37 Nakon toga se  u dane popisa podiže Juda Galilejac i odvuče narod za sobom.  I on propade i sve mu se pristaše raspršiše. 
\par 38 I sad evo kanite  se, velim vam, tih ljudi i otpustite ih. Jer ako je taj naum  ili to djelo od ljudi, propast će; 
\par 39 ako li je pak od Boga, nećete ga moći uništiti - da se i s Bogom u ratu ne nađete."  Poslušaju ga 
\par 40 pa dozovu apostole, išibaju ih, zapovjede im  da ne govore u ime Isusovo pa ih otpuste. 
\par 41 Oni pak odu ispred  Vijeća radosni što bijahu dostojni podnijeti pogrde za Ime. 
\par 42 I  svaki su dan u Hramu i po kućama neprestance učili i navješćivali  Krista, Isusa. 


\chapter{6}

\par 1 U one dane, kako se broj učenika množio, Židovi grčkog jezika  stadoše mrmljati protiv domaćih Židova što se u svagdanjem služenju  zanemaruju njihove udovice. 
\par 2 Dvanaestorica nato sazvaše mnoštvo  učenika i rekoše: "Nije pravo da mi napustimo riječ Božju da  bismo služili kod stolova. 
\par 3 De pronađite, braćo, između sebe  sedam muževa na dobru glasu, punih Duha i mudrosti. Njih ćemo  postaviti nad ovom službom, 
\par 4 a mi ćemo se posvetiti molitvi  i posluživanju Riječi." 
\par 5 Prijedlog se svidje svemu mnoštvu pa izabraše Stjepana, muža puna vjere i Duha Svetoga, zatim Filipa, Prohora, Nikanora, Timona, Parmenu te antiohijskog pridošlicu Nikolu. 
\par 6 Njih postave  pred apostole, a oni pomolivši se, polože na njih ruke. 
\par 7 I riječ je Božja rasla, uvelike se množio broj učenika u Jeruzalemu  i veliko je mnoštvo svećenika prihvaćalo vjeru. 
\par 8 Stjepan je pun milosti i snage činio velika čudesa i znamenja  u narodu. 
\par 9 Nato se digoše neki iz takozvane sinagoge Slobodnjaka, Cirenaca, Aleksandrinaca te onih iz Cilicije i Azije pa počeše  raspravljati sa Stjepanom, 
\par 10 ali nisu mogli odoljeti mudrosti  i Duhu kojim je govorio. 
\par 11 Onda podmetnuše neke ljude koji rekoše: "Čuli smo ga  govoriti pogrdne riječi protiv Mojsija i Boga." 
\par 12 Podjare i  narod, starješine i pismoznance pa priđu, zgrabe ga i odvuku  u Vijeće. 
\par 13 Ondje namjestiše lažne svjedoke koji rekoše: "Ovaj  čovjek neprestance govori protiv svetog Mjesta i Zakona. 
\par 14 Čuli  smo ga doista govoriti: 'Isus Nazarećanin razvalit će ovo Mjesto  i izmijeniti običaje koje nam predade Mojsije'." 
\par 15 A svi koji  su sjedili u Vijeću upriješe pogled u Stjepana te opaziše - lice  mu kao u anđela. 


\chapter{7}

\par 1 Veliki svećenik upita: "Je li to tako?" 
\par 2 Stjepan odgovori:  "Braćo i oci, čujte! Bog slave ukaza se ocu našemu Abrahamu dok  bijaše u Mezopotamiji, prije negoli se nastani u Haranu, 
\par 3 i  reče mu: Iziđi iz zemlje svoje, iz zavičaja svoga, hajde u zemlju koju ću ti pokazati. 
\par 4 On nato iziđe iz zemlje kaldejske i nastani se u Haranu.  Odande ga nakon smrti oca njegova Bog preseli u ovu zemlju u  kojoj vi sada boravite. 
\par 5 U njoj mu ne dade ni stope baštine,  nego obeća dati je u posjed njemu i potomstvu njegovu  nakon njega, premda još nije imao djeteta. 
\par 6 Bog  isto tako reče da će potomci njegovi biti pridošlice u zemlji  tuđoj, da će ih porobljavati i tlačiti četiri stotine  godina. 
\par 7 Ali narod kojemu budu robovali ja ću suditi, reče  Bog. A nakon toga izići će i iskazati mi štovanje na ovome  mjestu. 
\par 8 Dade mu i Savez obrezanja. Tako rodi Izaka  i obreza ga osmi dan, Izak Jakova, Jakov dvanaest rodozačetnika." 
\par 9 "Rodozačetnici pak, iz zavisti, Josipa predadoše u  Egipat. Ali Bog bijaše s njim 
\par 10 te ga izbavljaše iz svih  nevolja, podari ga naklonošću i mudrošću pred faraonom, kraljem egipatskim koji ga postavi za upravitelja nad  Egiptom i nad cijelim dvorom svojim. 
\par 11 Onda u cijeloj  zemlji egipatskoj i kanaanskoj nasta glad i nevolja velika:  oci naši ne mogahu naći hrane. 
\par 12 Kad Jakov doču da u Egiptu  ima žita, posla onamo najprije oce naše. 
\par 13 Drugi se put  Josip očitova braći svojoj pa faraon dozna za podrijetlo  Josipovo. 
\par 14 Josip tada posla po Jakova, oca svoga, i svu rodbinu, sedamdeset i pet duša. 
\par 15 Jakov tako siđe u Egipat. I umrije  on i oci naši. 
\par 16 Preneseni su u Sihem i položeni u grob  koji je Abraham za srebro kupio od sinova Hamorovih u Sihemu." 
\par 17 "Kako se bližilo vrijeme obećanja koje Bog obreče Abrahamu, rastao je u Egiptu narod i množio se 
\par 18 dok ondje  ne zavlada drugi kralj koji nije poznavao Josipa. 
\par 19 Lukav  prema rodu našemu, tlačio je on oce naše da bi djecu svoju izlagali  da ne ostanu na životu. 
\par 20 U taj se čas rodi Mojsije.  Bijaše božanski lijep. Tri je mjeseca hranjen u kući očinskoj, 
\par 21 a onda, kad je bio izložen, prigrli ga kći faraonova  i othrani sebi za sina. 
\par 22 Tako Mojsije, odgojen u svoj  mudrosti egipatskoj, bijaše silan na riječima i djelima." 
\par 23 "Kad mu bijaše četrdeset godina, ponuka ga srce da pohodi  braću svoju, sinove Izraelove. 
\par 24 I kad vidje kako je jednomu  nanesena nepravda, suprotstavi se i osveti zlostavljenoga ubivši  Egipćanina. 
\par 25 Mislio je da će braća njegova shvatiti kako  će im Bog po njegovoj ruci pružiti spasenje, ali oni ne shvatiše. 
\par 26 Sutradan se pojavi pred onima koji su se tukli te ih stade  nagovarati da se izmire: 'Ljudi, braća ste! Zašto zlostavljate  jedan drugoga?' 
\par 27 Ali ga onaj što je zlostavljao svoga bližnjega  odbi riječima: Tko te postavi glavarom i sucem nad nama? 
\par 28 Kaniš  li ubiti i mene kao što si jučer ubio onog Egipćanina? 
\par 29 Na  te riječi pobježe Mojsije i skloni se u zemlju midjansku,  gdje mu se rodiše dva sina." 
\par 30 "Nakon četrdeset godina ukaza mu se Anđeo u pustinji  brda Sinaja u rasplamtjeloj vatri jednoga grma. 
\par 31 Opazivši  to, zadivi se Mojsije viđenju. Dok je prilazio da bolje promotri, eto glasa Gospodnjega: 
\par 32 Ja sam Bog Otaca tvojih, Bog Abrahamov, Izakov i Jakovljev. Sav preplašen, Mojsije se ne usudi pogledati. 
\par 33 A Gospodin će mu: Izuj obuću s nogu! Jer mjesto na kojem  stojiš, sveto je tlo. 
\par 34 Vidio sam, vidio nevolju naroda svoga  u Egiptu i uzdisaj mu čuo pa siđoh izbaviti ga. I sad hajde!  Šaljem te u Egipat!" 
\par 35 "Toga Mojsija - kojega su se odrekli rekavši: Tko  te postavi glavarom i sucem? - toga im Bog kao glavara i  otkupitelja posla po Anđelu koji mu se ukaza u grmu. 
\par 36 On ih  izvede učinivši čudesa i znamenja u zemlji egipatskoj, u Crvenome  moru i u pustinji kroz četrdeset godina. 
\par 37 To je onaj Mojsije  koji reče sinovima Izraelovim: Proroka poput mene od vaše  braće podići će vam Bog. 
\par 38 To je onaj koji za skupa u pustinji  bijaše između Anđela što mu govoraše na brdu Sinaju i otaca naših;  onaj koji je primio riječi životne da ih nama preda. 
\par 39 Njemu  se ne htjedoše pokoriti oci naši, nego ga odbiše i u srcima se  svojima vratiše u Egipat 
\par 40 rekavši Aronu: 'Napravi  nam bogove koji će ići pred nama! Ta ne znamo što se dogodi s  tim Mojsijem koji nas izvede iz zemlje egipatske.' 
\par 41 Tele načiniše  u dane one, prinesoše žrtvu tom kumiru i veseljahu se  djelima ruku svojih. 
\par 42 Bog se pak odvrati i prepusti ih da  časte vojsku nebesku, kao što piše u Knjizi proročkoj: Prinosiste li mi žrtve i prinose četrdeset godina u pustinji, dome Izraelov? 
\par 43 Poprimiste šator Molohov i zvijezdu boga Refana - likove koje napraviste da biste im se klanjali. Odvest ću vas stoga u progonstvo onkraj Babilona!" 
\par 44 "Oci naši imahu u pustinji Šator svjedočanstva kako odredi  Onaj koji reče Mojsiju da se on načini po praliku koji  je vidio. 
\par 45 Taj su Šator preuzeli oci naši i pod Jošuom ga  unijeli u posjed s kojega Bog pred licem njihovim rastjera narode.  Tako bijaše sve do dana Davida, 
\par 46 koji je našao milost pred  Bogom te molio da nađe boravište Bogu Jakovljevu. 
\par 47 Istom  Salomon izgradi mu Dom. 
\par 48 Ali Svevišnji u rukotvorinama  ne prebiva, kao što veli prorok: 
\par 49 Nebesa su moje prijestolje, a zemlja podnožje nogama. Kakav dom da mi sagradite, govori Gospodin, i gdje da bude mjesto mog počinka? 
\par 50 Nije li ruka moja načinila sve to? 
\par 51 Tvrdovrati i neobrezanih srdaca i ušiju, vi se uvijek opirete Duhu Svetomu: kako oci vaši tako i vi! 
\par 52 Kojega  od proroka nisu progonili oci vaši? I pobiše one koji su unaprijed  navijestili dolazak Pravednika čiji ste vi sada izdajice i ubojice, 
\par 53 vi koji po anđeoskim uredbama primiste Zakon, ali ga se  niste držali." 
\par 54 Kad su to čuli, uskipješe u srcima i počeše škripati  zubima na njega. 
\par 55 Ali on, pun Duha Svetoga, uprije pogled  u nebo i ugleda slavu Božju i Isusa gdje stoji zdesna Bogu 
\par 56 pa  reče: "Evo vidim nebesa otvorena i Sina Čovječjega gdje stoji  zdesna Bogu." 
\par 57 Vičući iza glasa, oni zatisnuše uši i navališe  jednodušno na njega. 
\par 58 Izbaciše ga iz grada pa ga kamenovahu.  Svjedoci odložiše haljine do nogu mladića koji se zvao Savao. 
\par 59 I dok su ga kamenovali, Stjepan je zazivao: "Gospodine Isuse, primi duh moj!" 
\par 60 Onda se baci na koljena i povika iza glasa:  "Gospodine, ne uzmi im ovo za grijeh!" Kada to reče, usnu. 


\chapter{8}

\par 1 Savao je pristao da se Stjepan smakne. U onaj dan navali velik progon na Crkvu u Jeruzalemu. Svi  se osim apostola raspršiše po krajevima judejskim i samarijskim. 
\par 2 Bogobojazni su ljudi pokopali Stjepana i održali veliko žalovanje  za njim. 
\par 3 Savao je pak pustošio Crkvu: ulazio je u kuće, odvlačio  muževe i žene i predavao ih u tamnicu. 
\par 4 Oni dakle što su se raspršili obilazili su navješćujući  Riječ. 
\par 5 Filip tako siđe u grad samarijski i stade im propovijedati  Krista. 
\par 6 Mnoštvo je jednodušno prihvaćalo što je Filip govorio  slušajući ga i gledajući znamenja koja je činio. 
\par 7 Doista, iz  mnogih su opsjednutih izlazili nečisti duhovi vičući iza glasa, a ozdravljali su i mnogi uzeti i hromi. 
\par 8 Nasta tako velika  radost u onome gradu. 
\par 9 Čovjek se neki, imenom Šimun, u gradu već duže bavio čarobnjaštvom  i opčaravao narod tvrdeći da je neki veliki. 
\par 10 Priklanjalo  mu se sve, malo i veliko, te govorilo: "Ovaj je Snaga Božja,  zvana Velika." 
\par 11 A priklanjahu mu se jer ih je duže vremena  opčaravao svojim vradžbinama. 
\par 12 Ali kad povjerovaše Filipu  koji navješćivaše evanđelje o kraljevstvu Božjemu i o imenu Isusa  Krista, krštavahu se - muževi i žene. 
\par 13 Povjerova i Šimun te  se krsti i osta uz Filipa: bio je zanesen promatrajući znamenja  i čudesa koja su se događala. 
\par 14 Kad su apostoli u Jeruzalemu čuli da je Samarija prigrlila  riječ Božju, poslaše k njima Petra i Ivana. 
\par 15 Oni siđoše i  pomoliše se za njih da bi primili Duha Svetoga. 
\par 16 Jer još ni  na koga od njih ne bijaše sišao; bijahu samo kršteni u ime Gospodina  Isusa. 
\par 17 Tada polagahu ruke na njih i oni primahu Duha Svetoga. 
\par 18 Kad Šimun vidje da se polaganjem ruku apostolskih daje  Duh, ponudi apostolima novaca 
\par 19 govoreći: "Dajte i meni tu  moć da svatko na koga položim ruke primi Duha Svetoga." 
\par 20 Petar mu odvrati: "Novac tvoj zajedno s tobom propao  kad si mislio dar Božji novcima steći! 
\par 21 Nema tebi ovdje dijela  ni udjela jer tvoje srce nije pravo pred Bogom! 
\par 22 Obrati se  od te opakosti svoje i moli Gospodina ne bi li ti se kako oprostila  namisao srca tvoga. 
\par 23 Ta gledam te: žučju si gorak i nepravdom  okovan." 
\par 24 Šimun odgovori: "Molite i vi za me Gospodina da me ne snađe  ništa od toga što rekoste!" 
\par 25 Oni pak pošto posvjedočiše i dorekoše riječ Gospodnju, vratiše se u Jeruzalem navješćujući evanđelje mnogim selima  samarijskim. 
\par 26 Anđeo se Gospodnji obrati Filipu: "Ustani i pođi na jug  putom što iz Jeruzalema silazi u Gazu; on je pust." 
\par 27 On usta  i pođe. Odjednom eto nekog Etiopljanina, dvoranina, visokog dostojanstvenika  kandake, kraljice etiopske koji bijaše nad svom njezinom riznicom. 
\par 28 Vraćao se iz Jeruzalema, kamo je bio pošao pokloniti se;  sjeđaše na svojim kolima i čitaše proroka Izaiju. 
\par 29 Duh reče  Filipu: "Pođi i pridruži se tim kolima!" 
\par 30 Filip pritrča i  ču gdje onaj čita Izaiju proroka pa će mu: "Razumiješ li što  čitaš?" 
\par 31 On odvrati: "Kako bih mogao ako me tko ne uputi?"  Onda zamoli Filipa da se uspne i sjedne uza nj. 
\par 32 A čitao je  ovaj odlomak Pisma: Ko ovcu na klanje odvedoše ga, ko janje nijemo pred onim što ga striže on ne otvara svojih usta. 
\par 33 U poniženju sud mu je uskraćen. Naraštaj njegov tko da opiše? Da, uklonjen je sa zemlje život njegov. 
\par 34 Dvoranin se obrati Filipu pa će mu: "Molim te, o kome  to prorok govori? O sebi ili o kome drugom? 
\par 35 Filip prozbori  te mu, pošavši od toga Pisma, navijesti evanđelje: Isusa. 
\par 36 Putujući tako, stigoše do neke vode pa će dvoranin: "Evo  vode! Što priječi da se krstim?" 
\par 37 # 
\par 38 Zapovjedi da kola  stanu pa obojica, Filip i dvoranin, siđoše u vodu te ga Filip  krsti. 
\par 39 A kad iziđoše iz vode, Duh Gospodnji ugrabi Filipa  te ga dvoranin više ne vidje. On radosno nastavi svojim putom, 
\par 40 a Filip se nađe u Azotu. I kako je prolazio, navješćivaše  evanđelje svim gradovima dok ne stiže u Cezareju. 


\chapter{9}

\par 1 Savao pak, sveudilj zadahnut prijetnjom i pokoljem prema učenicima  Gospodnjim, pođe k velikomu svećeniku, 
\par 2 zaiska od njega pisma  za sinagoge u Damasku, da sve koje nađe od ovoga Puta, muževe  i žene, okovane dovede u Jeruzalem. 
\par 3 Kad se putujući približi Damasku, iznenada ga obasja svjetlost  s neba. 
\par 4 Sruši se na zemlju i začu glas što mu govoraše: "Savle, Savle, zašto me progoniš?" 
\par 5 On upita: "Tko si, Gospodine?" A on će: "Ja sam Isus kojega ti progoniš! 
\par 6 Nego ustani, uđi u grad i reći će ti se što ti je činiti." 
\par 7 Njegovi suputnici ostadoše bez riječi: čuli su doduše  glas, ali ne vidješe nikoga. 
\par 8 Savao usta sa zemlje. Otvorenih  očiju nije ništa vidio pa ga povedu za ruku i uvedu u Damask. 
\par 9 Tri dana nije vidio, nije jeo ni pio. 
\par 10 U Damasku bijaše neki učenik imenom Ananija. Njemu u viđenju  reče Gospodin: "Ananija!" On se odazva: "Evo me, Gospodine!" 
\par 11 A Gospodin će mu: "Ustani, pođi u ulicu zvanu Ravna i  u kući Judinoj potraži Taržanina imenom Savla. Eno, moli se; 
\par 12 i u viđenju vidje čovjeka imenom Ananiju gdje ulazi i polaže  na nj ruke da bi progledao." 
\par 13 Ananija odgovori: "Gospodine, od mnogih sam čuo o tom  čovjeku kolika je zla tvojim svetima učinio u Jeruzalemu. 
\par 14 On  ima od velikih svećenika i punomoć okovati sve koji prizivlju  ime tvoje." 
\par 15 Gospodin mu odvrati: "Pođi jer on mi je oruđe izabrano  da ponese ime moje pred narode i kraljeve i sinove Izraelove. 
\par 16 Ja ću mu uistinu pokazati koliko mu je za ime moje trpjeti." 
\par 17 Ananija ode, uđe u kuću, položi na nj ruke i reče: "Savle, brate! Gospodin, Isus koji ti se ukaza na putu kojim si išao, posla me da progledaš i napuniš se Duha Svetoga." 
\par 18 I odmah mu s očiju spade nešto kao ljuske te on progleda  pa usta, krsti se 
\par 19 i uzevši hrane, okrijepi se. Nekoliko dana provede s učenicima u Damasku. 
\par 20 te odmah  stade po sinagogama propovijedati Isusa, da je on Sin Božji. 
\par 21 Koji ga god slušahu, izvan sebe govorahu: "Nije li ovo onaj  koji je u Jeruzalemu istrebljivao sve koji Ime ovo prizivlju, pa i ovamo zato došao da ih okovane odvede pred velike svećenike?" 
\par 22 Savao pak, sve silniji, zbunjivaše Židove koji prebivahu  u Damasku dokazujući: "Ovo je Krist!" 
\par 23 Pošto je minulo podosta vremena, odluče Židovi pogubiti  ga, 
\par 24 ali Savao dozna za njihov naum. Nadzirahu i vrata danju  i noću da bi ga pogubili, 
\par 25 ali ga učenici noću uzeše i preko  zidina oprezno spustiše u košari. 
\par 26 Kad je Savao došao u Jeruzalem, gledao se pridružiti  učenicima, ali ga se svi bojahu: nisu vjerovali da je učenik. 
\par 27 Tada ga Barnaba uze i povede k apostolima te im pripovjedi  kako je Savao na putu vidio Gospodina koji mu je govorio i kako  je u Damasku smjelo propovijedao u ime Isusovo. 
\par 28 Od tada se s njima slobodno kretao po Jeruzalemu i smjelo  propovijedao u ime Gospodnje. 
\par 29 Govorio je i raspravljao sa  Židovima grčkog jezika pa i oni snovahu pogubiti ga. 
\par 30 Saznala to braća pa ga odvedoše u Cezareju i uputiše u Tarz. 
\par 31 Crkva je po svoj Judeji, Galileji i Samariji uživala  mir, izgrađivala se i napredovala u strahu Gospodnjem te rasla  utjehom Svetoga Duha. 
\par 32 Jednom Petar, obilazeći posvuda, siđe i k svetima u Lidi. 
\par 33 Ondje nađe nekog čovjeka imenom Eneja, koji je osam godina  ležao na postelji: bijaše uzet. 
\par 34 Reče mu Petar: "Eneja, ozdravlja  te Isus Krist! Ustani i prostri sam sebi!" On umah usta. 
\par 35 Vidješe  to svi žitelji Lide i Šarona te se obratiše Gospodinu. 
\par 36 U Jopi pak bijaše učenica imenom Tabita, što prevedeno  znači Košuta. Bijaše ona bogata dobrim djelima i milostinjama  što ih je dijelila. 
\par 37 Upravo u one dane obolje i umrije. Pošto  je operu, izlože je u gornjoj sobi. 
\par 38 A kako je Lida blizu  Jope, učenici čuše da je Petar ondje i poslaše k njemu dva čovjeka  s molbom: "Dođi k nama, ne oklijevaj!" 
\par 39 Petar usta i krenu s njima. Čim stiže, odvedoše ga u  gornju sobu. Okružiše ga sve udovice te mu plačući pokazivahu  haljine i odijela što ih je Košuta izrađivala dok je još bila  s njima. 
\par 40 Petar sve istjera van, kleknu, pomoli se pa se okrenu  prema tijelu i reče: "Tabita, ustani!" Ona otvori oči, pogleda  Petra i sjede. 
\par 41 On joj pruži ruku i pridiže je. Onda pozove  svete i udovice pa im je pokaza živu. 
\par 42 Dozna se za to po svoj  Jopi te mnogi povjerovaše u Gospodina. 
\par 43 Petar osta neko vrijeme u Jopi u nekog Šimuna kožara. 


\chapter{10}

\par 1 U Cezareji bijaše neki čovjek imenom Kornelije, satnik takozvane  italske čete, 
\par 2 pobožan i bogobojazan sa svim svojim domom.  Dijelio je mnoge milostinje narodu i bez prestanka se molio Bogu. 
\par 3 U viđenju negdje oko devete ure dana ugleda on jasno anđela  Božjega gdje dolazi k njemu i veli mu: "Kornelije!" 
\par 4 Zagleda  se u nj pa mu prestrašen reče: "Što je, Gospodine?" A on njemu:  "Molitve su tvoje i milostinje uzišle kao žrtva podsjetnica pred  Boga. 
\par 5 Zato sada pošalji ljude u Jopu i dozovi Šimuna koji  se zove Petar. 
\par 6 On je gost u nekog Šimuna kožara čija je kuća  uz more." 
\par 7 Čim ode anđeo koji mu je govorio, pozove on dvojicu  slugu i jednoga pobožna, privržena vojnika, 
\par 8 sve im ispripovjedi  i posla ih u Jopu. 
\par 9 Sutradan, dok su oni putovali i približavali se gradu, oko šeste ure uziđe Petar na krov moliti. 
\par 10 Ogladnje i zaželje  se jela. Dok mu pripremahu, pade on u zanos. 
\par 11 Gleda on nebo  rastvoreno i posudu neku poput velika platna: uleknuta s četiri  okrajka, silazi na zemlju. 
\par 12 U njoj bijahu svakovrsni četveronošci, gmazovi zemaljski i ptice nebeske. 
\par 13 I glas će mu neki: "Ustaj, Petre! Kolji i jedi!" 
\par 14 Petar odvrati: "Nipošto, Gospodine!  Ta nikad još ne okusih ništa okaljano i nečisto". 
\par 15 A glas  će mu opet, po drugi put: "Što Bog očisti, ti ne zovi okaljanim!" 
\par 16 To se ponovi do triput, a onda je posuda ponesena na nebo. 
\par 17 Dok se Petar dvoumio što bi imalo značiti viđenje koje  vidje, eto ljudi koje je poslao Kornelije: pošto se raspitaše  za Šimunovu kuću, pojave se na vratima, 
\par 18 zovnu te upitaju  je li ondje ugošćen neki Šimun, nazvan Petar. 
\par 19 Dok je Petar sveudilj razmišljao o viđenju, reče mu Duh:  "Evo, neka te trojica traže. 
\par 20 De ustani, siđi i pođi s njima  ne skanjujući se jer ja sam ih poslao." 
\par 21 Petar siđe k ljudima  i reče: "Evo me! Ja sam onaj kojega tražite! Zbog čega ste došli?" 
\par 22 Oni odgovore: "Satnik Kornelije, muž pravedan i bogobojazan, za kojega svjedoči sav narod židovski, primi od svetog anđela  naputak da te dozove u dom svoj i čuje od tebe riječi." 
\par 23 Tada  ih Petar pozva unutra i ugosti. Sutradan usta i krenu s njima; pratila ga neka braća iz Jope. 
\par 24 Drugi dan stiže u Cezareju. Kornelije ih je čekao sazvavši  rodbinu i prisne prijatelje. 
\par 25 Kad je Petar ulazio, pohrli  mu Kornelije u susret, padne mu k nogama i pokloni se. 
\par 26 Petar  ga pridigne govoreći: "Ustani! I ja sam čovjek." 
\par 27 I razgovarajući  s njime, uđe i nađe sabrane mnoge 
\par 28 te im reče: "Vi znate kako  je Židovu zabranjeno družiti se sa strancem ili k njemu ulaziti, ali meni Bog pokaza da nikoga ne zovem okaljanim ili nečistim. 
\par 29 Stoga, pozvan, i dođoh bez pogovora. Da čujemo dakle zbog  čega me pozvaste!" 
\par 30 Kornelije reče: "Prije četiri dana baš u ovo doba, o  devetoj uri, molio sam se u kući kad gle: čovjek neki u sjajnoj  odjeći stane preda me 
\par 31 i reče: 'Kornelije, uslišana ti je  molitva i milostinje su tvoje spomenute pred Bogom! 
\par 32 Pošalji  dakle u Jopu i dozovi Šimuna koji se zove Petar. On je gost u  kući Šimuna kožara uz more.' 
\par 33 Odmah sam dakle poslao k tebi, a ti si dobro učinio što si došao. Evo nas dakle sviju pred  Bogom da čujemo sve što ti zapovjedi Gospodin!" 
\par 34 Petar tada prozbori i reče: "Sad uistinu shvaćam da Bog  nije pristran, 
\par 35 nego - u svakom je narodu njemu mio onaj koji  ga se boji i čini pravdu. 
\par 36 Riječ posla sinovima Izraelovim  navješćujući im evanđelje: mir po Isusu Kristu; on je  Gospodar sviju. 
\par 37 Vi znate što se događalo po svoj Judeji,  počevši od Galileje, nakon krštenja koje je propovijedao Ivan: 
\par 38 kako Isusa iz Nazareta Bog pomaza Duhom Svetim i snagom, njega koji je, jer Bog bijaše s njime, prošao zemljom čineći  dobro i ozdravljajući sve kojima bijaše ovladao đavao." 
\par 39 "Mi smo svjedoci svega što on učini u zemlji judejskoj  i Jeruzalemu. I njega smakoše, objesivši ga na drvo! 
\par 40 Bog  ga uskrisi treći dan i dade mu da se očituje - 
\par 41 ne svemu narodu, nego svjedocima od Boga predodređenima - nama koji smo s njime  zajedno jeli i pili pošto uskrsnu od mrtvih." 
\par 42 "On nam i naloži propovijedati narodu i svjedočiti: Ovo  je onaj kojega Bog postavi sucem živih i mrtvih!" 
\par 43 "Za nj svjedoče svi proroci: da tko god u nj vjeruje, po imenu njegovu prima oproštenje grijeha." 
\par 44 Dok je Petar još govorio te riječi, siđe Duh Sveti na  sve koji su slušali tu besjedu. 
\par 45 A vjernici iz obrezanja,  koji dođoše zajedno s Petrom, začudiše se što se i na pogane  izlio dar Duha Svetoga. 
\par 46 Jer čuli su ih govoriti drugim jezicima  i veličati Boga. Tada Petar reče: 
\par 47 "Može li tko uskratiti vodu da se ne  krste ovi koji su primili Duha Svetoga kao i mi?" 
\par 48 I zapovjedi  da se krste u ime Isusa Krista. Tada ga zamole da ostane ondje  nekoliko dana. 


\chapter{11}

\par 1 Dočuli apostoli i braća po Judeji da i pogani primiše riječ  Božju 
\par 2 pa kad Petar uziđe u Jeruzalem, uzeše mu obrezanici  prigovarati: 
\par 3 "Ušao si, dobacivahu, k ljudima neobrezanima  i jeo s njima!" 
\par 4 Onda započe Petar te im izloži sve po redu: 
\par 5 "Molio sam se, reče, u Jopi kadli u zanosu ugledam viđenje:  posudu neku poput velika platna, uleknuta s četiri okrajka, gdje  silazi s neba i dolazi do mene. 
\par 6 Zagledah se, promotrih je  i vidjeh četvoronošce zemaljske, zvijeri i gmazove te ptice nebeske. 
\par 7 Začuh i glas koji mi govoraše: 'Ustaj, Petre! Kolji i jedi!' 
\par 8 Ja odvratih: 'Nipošto, Gospodine! Ta nikad mi još ništa okaljano  ili nečisto ne uđe u usta.' 
\par 9 A glas će s neba po drugi put:  'Što Bog očisti, ti ne zovi nečistim.' 
\par 10 To se ponovi do triput, a onda se sve opet povuče na nebo." 
\par 11 "I odmah se, evo, pred kućom u kojoj bijah pojaviše tri  čovjeka poslana iz Cezareje k meni. 
\par 12 A Duh mi reče da pođem  s njima ništa ne premišljajući. Sa mnom pođoše i ova šestorica  braće te uđosmo u kuću tog čovjeka. 
\par 13 On nam pripovjedi kako  je u svojoj kući vidio anđela koji je stao preda nj i rekao:  'Pošalji u Jopu i dozovi Šimuna nazvanog Petar; 
\par 14 on će ti  navijestiti riječi po kojima ćeš se spasiti ti i sav dom tvoj.'" 
\par 15 "I kad počeh govoriti, siđe na njih Duh Sveti kao ono  na nas u početku. 
\par 16 Sjetih se tada riječi Gospodnje: 'Ivan  je, govoraše on, krstio vodom, a vi ćete biti kršteni Duhom Svetim.' 
\par 17 Ako im je dakle Bog dao isti dar kao i nama koji povjerovasmo  u Gospodina Isusa Krista, tko sam ja da bih se smio oprijeti  Bogu?" 
\par 18 Kad su to čuli, umiriše se te stadoše slaviti Boga govoreći:  "Dakle i poganima Bog dade obraćenje na život!" 
\par 19 Oni dakle što ih rasprši nevolja nastala u povodu Stjepana  dopriješe do Fenicije, Cipra i Antiohije, nikomu ne propovijedajući  Riječi doli samo Židovima. 
\par 20 Neki su od njih bili Ciprani i  Cirenci. Kad stigoše u Antiohiju, propovijedahu i Grcima navješćujući  evanđelje: Gospodina, Isusa. 
\par 21 Ruka Gospodnja bijaše s njima  te velik broj ljudi povjerova i obrati se Gospodinu. 
\par 22 Vijest o tome doprije do Crkve u Jeruzalemu pa poslaše  Barnabu u Antiohiju. 
\par 23 Kad on stiže i vidje milost Božju, obradova  se te potaknu sve da u odlučnosti srca ostanu uz Gospodina. 
\par 24 Ta  bijaše to muž čestit, pun Duha Svetoga i vjere. Znatno se mnoštvo  prikloni Gospodinu. 
\par 25 Barnaba se zatim zaputi u Tarz potražiti Savla. 
\par 26 Kad  ga nađe, odvede ga u Antiohiju. Punu su se godinu dana sastajali  u toj Crkvi i poučavali poveće mnoštvo te se u Antiohiji učenici  najprije prozvaše kršćanima. 
\par 27 U one dane dođoše u Antiohiju neki proroci iz Jeruzalema. 
\par 28 Jedan od njih, imenom Agab, usta i po Duhu pretkaza da će  uskoro nastati velika glad po svem svijetu. Ona i nasta za Klaudija. 
\par 29 Stoga će svatko od učenika, odlučiše, koliko smogne poslati  da se posluži braći u Judeji. 
\par 30 To i učiniše te poslaše starješinama  po Barnabi i Savlu. 


\chapter{12}

\par 1 U to vrijeme uze Herod zlostavljati neke od Crkve. 
\par 2 Mačem  pogubi Jakova, brata Ivanova. 
\par 3 Kad vidje da je to drago Židovima, uhvati i Petra (bijahu upravo Dani beskvasnih kruhova). 
\par 4 Uhiti  ga, baci u tamnicu i dade da ga čuvaju četiri vojničke četverostraže, nakan izvesti ga nakon Pashe pred narod. 
\par 5 Petra su dakle čuvali  u tamnici, a Crkva se svesrdno moljaše Bogu za njega. 
\par 6 One noći kad ga je Herod kanio privesti, spavao je Petar  između dva vojnika, okovan dvojim verigama, a stražari pred vratima  čuvahu stražu. 
\par 7 Kad eto: pojavi se anđeo Gospodnji te svjetlost  obasja ćeliju. Anđeo udari Petra u rebra, probudi ga i reče:  "Ustaj brzo!" I spadoše mu verige s ruku. 
\par 8 Anđeo mu reče: "Opaši  se i priveži obuću!" On učini tako. Onda će mu anđeo: "Zaogrni  se i hajde za mnom!" 
\par 9 Petar izađe, pođe za njim, a nije znao  da je zbilja što se događa po anđelu: činilo mu se da gleda viđenje. 
\par 10 Prošavši prvu stražu, i drugu, dođoše do željeznih vrata  koja vode u grad. Ona im se sama otvore te oni izađu, prođu jednu  ulicu, a onda anđeo odjednom odstupi od njega. 
\par 11 Petar pak, došavši k sebi, reče: "Sad uistinu znam da je Gospodin poslao  anđela svoga i izbavio me iz Herodove ruke i od svega što je  očekivao židovski narod." 
\par 12 Kad je to uočio, zaputi se kući Marije, majke Ivana nazvanog  Marko. Ondje se mnogi bijahu sabrali i molili. 
\par 13 Kad Petar  pokuca na dvorišna vrata, dođe prisluhnuti sluškinja imenom Ruža. 
\par 14 Kad prepozna Petrov glas, od radosti i ne otvori vrata, nego  utrča i javi da je Petar pred vratima. 
\par 15 Oni joj rekoše: "Mahnitaš!" Ali je ona uporno tvrdila  da je tako. Nato će oni: "Bit će njegov anđeo!" 
\par 16 Petar nastavi  kucati. Kad napokon otvoriše i ugledaše ga, ostadoše izvan sebe. 
\par 17 On im rukom mahnu neka šute pa im pripovjedi kako ga Gospodin  izvede iz tamnice te dometnu: "Javite to Jakovu i braći!" Onda  izađe i ode u drugo mjesto. 
\par 18 Kad se razdani, nasta među vojnicima uzbuna nemalena  što li se s Petrom dogodilo. 
\par 19 Herod ga stade tražiti, a kad  ga ne nađe, sasluša stražare i naredi da se smaknu. Onda siđe  iz Judeje u Cezareju i ondje osta. 
\par 20 A bio je u žestoku sukobu s Tircima i Sidoncima. Oni  zajednički dođoše k njemu i pošto pridobiše kraljevskoga komornika  Blasta, zaiskaše mir, jer je njihova zemlja dobivala živež od  kraljeve. 
\par 21 U određeni dan sjede Herod odjeven u kraljevsko  ruho na prijestol i stade im govoriti. 
\par 22 Narod izvikivaše:  "Božji glas, a ne ljudski!" 
\par 23 Umah ga, zbog toga što ne dade  slavu Bogu, udari anđeo Gospodnji te on rascrvotočen izdahnu. 
\par 24 Riječ je pak Božja rasla i širila se. 
\par 25 Barnaba i Savao, pošto obaviše služenje u Jeruzalemu, vratiše se uzevši sa sobom  Ivana zvanog Marko. 


\chapter{13}

\par 1 U antiohijskoj je Crkvi bilo proroka i učitelja: Barnaba,  Šimun zvani Niger, Lucije Cirenac, Manahen, suothranjenik Heroda  četverovlasnika, i Savao. 
\par 2 Dok su jednom obavljali službu Božju  i postili, reče Duh Sveti: "De mi odlučite Barnabu i Savla za  djelo na koje sam ih pozvao." 
\par 3 Onda su postili, molili, položili  na njih ruke i otpustili ih. 
\par 4 Poslani od Svetoga Duha siđu u Seleuciju, a odande odjedre  na Cipar. 
\par 5 Kad se nađoše u Salamini, navješćivahu riječ Božju  u židovskim sinagogama. Imali su i Ivana za poslužitelja. 
\par 6 Pošto  pak prođoše sav otok do Pafa, nađoše nekog vračara, nazoviproroka, Židova, imenom Barjesu. 
\par 7 On bijaše uz namjesnika Sergija Pavla, čovjeka razborita. Sergije dozva Barnabu i Savla te zaiska čuti  riječ Božju, 
\par 8 ali im se usprotivi Elim, Vračar - tako mu se  ime prevodi - nastojeći odvratiti namjesnika od vjere. 
\par 9 Savao  pak, zvan i Pavao, pun Duha Svetoga, ošinu ga pogledom 
\par 10 i  reče: "Pun svake lukavosti i prevrtljivosti, sine đavolski, neprijatelju  svake pravednosti, zar nikako da prestaneš iskrivljavati ravne  putove Gospodnje? 
\par 11 Evo stoga sada ruke Gospodnje na tebi:  oslijepljet ćeš i neko vrijeme nećeš gledati sunca!" Odmah pade  na nj mrak i tama te on glavinjajući stade tražiti ruke vodilje. 
\par 12 Videći što se dogodilo, povjerova tada namjesnik, zanesen  naukom Gospodnjim. 
\par 13 Pošto se Pavao i oni oko njega otisnuše od Pafa, stigoše  u Pergu pamfilijsku. Ivan ih napusti te se vrati u Jeruzalem. 
\par 14 Oni pak krenuše iz Perge i stigoše u Antiohiju pizidijsku.  U dan subotni ušli su u sinagogu i sjeli. 
\par 15 Nakon čitanja Zakona  i Proroka pošalju nadstojnici sinagoge k njima: "Braćo, rekoše, ima li u vas koja riječ utjehe za narod, govorite!" 
\par 16 Nato usta Pavao, dadne rukom znak i reče: "Izraelci i  vi koji se Boga bojite, čujte! 
\par 17 Bog naroda ovoga, Izraela, izabra oce naše i uzdiže narod za boravka u zemlji egipatskoj  te ga ispruženom rukom izvede iz nje. 
\par 18 Oko četrdeset  ga je godina na rukama nosio u pustinji 
\par 19 pa pošto zatre  sedam naroda u zemlji kanaanskoj, ubaštini ga u zemlji njihovoj 
\par 20 za kakve četiri stotine i pedeset godina. Nakon toga dade  im suce - do Samuela proroka. 
\par 21 Onda zaiskaše kralja pa im  Bog za četrdeset godina dade Šaula, sina Kiševa, iz plemena Benjaminova. 
\par 22 Pošto svrgnu njega, podiže im za kralja Davida za kojega  posvjedoči: Nađoh Davida, sina Jišajeva, čovjeka po  svom srcu, koji će ispuniti sve moje želje. 
\par 23 Iz njegova  potomstva izvede Bog po svom obećanju Izraelu Spasitelja, Isusa. 
\par 24 Pred njegovim je dolaskom Ivan propovijedao krštenje obraćenja  svemu narodu izraelskomu. 
\par 25 A kad je Ivan dovršavao svoju trku, govorio je: 'Nisam ja onaj za koga me vi držite. Nego za mnom  evo dolazi onaj komu ja nisam dostojan odriješiti obuće na nogama.'" 
\par 26 "Braćo, sinovi roda Abrahamova, vi i oni koji se među  vama Boga boje, nama je upravljena ova Riječ spasenja. 
\par 27 Doista, žitelji Jeruzalema i glavari njihovi ne upoznaše njega ni riječi  proročkih što se čitaju svake subote pa ih, osudivši ga, ispuniše. 
\par 28 Premda ne nađoše nikakva razloga smrti, zatražiše od Pilata  da ga smakne. 
\par 29 Pošto pak izvršiše sve što je o njemu napisano, skinuše ga s drveta i položiše u grob. 
\par 30 Ali Bog ga uskrisi  od mrtvih. 
\par 31 On se mnogo dana ukazivao onima koji s njim bijahu  uzašli iz Galileje u Jeruzalem. Oni su sada njegovi svjedoci  pred narodom." 
\par 32 "I mi vam navješćujemo evanđelje: obećanje dano ocima 
\par 33 Bog je ispunio djeci, nama, uskrisivši Isusa, kao što je  i pisano u Psalmu drugom: Ti si Sin moj, danas te rodih. 
\par 34 Da ga pak uskrisi od mrtvih te se on više nikad neće vratiti  u trulež, rekao je ovime: Dat ću vama svetinje Davidove, pouzdane. 
\par 35 Zato i na drugome mjestu kaže: Nećeš dati da Svetac tvoj  ugleda truleži. 
\par 36 David doista, pošto u svom naraštaju  posluži volji Božjoj, preminu, pridruži se ocima svojim i vidje  trulež, 
\par 37 a Onaj koga Bog uskrisi ne vidje truleži. 
\par 38 / 
\par 39 Neka vam dakle braćo, znano bude: po Ovome vam se navješćuje  oproštenje grijeha! Po Ovome se tko god vjeruje, opravdava od  svega od čega se po Mojsijevu zakonu niste mogli opravdati! 
\par 40 Pazite  da se ne zbude što je rečeno u Prorocima: 
\par 41 Obazrite se, preziratelji, snebijte se i nestanite! Jer djelo činim u dane vaše, djelo u koje ne biste vjerovali da vam ga tko ispriča." 
\par 42 Na izlasku su ih molili da im iduće subote o tome govore. 
\par 43 A pošto se skup raspustio, mnogi Židovi i bogobojazne pridošlice  pođoše za Pavlom i Barnabom koji su ih nagovarali ustrajati u  milosti Božjoj. 
\par 44 Iduće se subote gotovo sav grad zgrnu čuti riječ Gospodnju. 
\par 45 Kad su Židovi ugledali mnoštvo, puni zavisti psujući suprotstavljali  su se onomu što je Pavao govorio. 
\par 46 Na to im Pavao i Barnaba  smjelo rekoše: "Trebalo je da se najprije vama navijesti riječ  Božja. Ali kad je odbacujete i sami sebe ne smatrate dostojnima  života vječnoga, obraćamo se evo poganima. 
\par 47 Jer ovako nam  je zapovjedio Gospodin: Postavih te za svjetlost poganima, da budeš na spasenje do nakraj zemlje. 
\par 48 Pogani koji su slušali radovali su se i slavili riječ  Gospodnju te povjerovaše oni koji bijahu određeni za život vječni. 
\par 49 Riječ se pak Gospodnja pronese po svoj onoj pokrajini. 
\par 50 Ali Židovi potakoše ugledne bogobojazne žene i prvake  gradske te zametnuše progon protiv Pavla i Barnabe pa ih izbaciše  iz svoga kraja. 
\par 51 Oni pak stresu prašinu s nogu protiv njih  pa odu u Ikonij. 
\par 52 A učenici su se ispunjali radošću i Duhom  Svetim. 


\chapter{14}

\par 1 U Ikoniju isto tako uđoše u židovsku sinagogu i govorahu tako  da povjerova veliko mnoštvo Židova i Grka. 
\par 2 Ali nepokorni Židovi  razdražiše i podjariše pogane protiv braće. 
\par 3 Oni se ipak zadržaše  duže vremena, smjeli u Gospodinu koji je svjedočio za Riječ milosti  svoje, davao da se po njihovim rukama događaju znamenja i čudesa. 
\par 4 Mnoštvo se gradsko podvoji: jedni bijahu za Židove, drugi  za apostole. 
\par 5 Pogani i Židovi sa svojim glavarima navališe  da zlostave i kamenuju apostole. 
\par 6 Kada to opaziše, prebjegoše  oni u likaonske gradove Listru i Derbu i okolicu. 
\par 7 Ondje su  navješćivali evanđelje. 
\par 8 U Listri je sjedio neki čovjek uzetih nogu, hrom od majčine  utrobe; nikada nije hodao. 
\par 9 Čuo je Pavla gdje govori. 
\par 10 Pavao  ga pronikne pogledom, vidje da ima vjeru u spasenje pa mu iza  glasa reče: "Uspravi se na noge!" On skoči i prohoda. 
\par 11 Kad mnoštvo ugleda što učini Pavao, povika likaonski:  "Bogovi u ljudskom obličju siđoše k nama!" 
\par 12 I nazvaše Barnabu  Zeusom, a Pavla Hermesom jer je Pavao vodio riječ. 
\par 13 A svećenik  Zeusa Predgradskoga dovede pred vrata bikove i vijence te u zajednici  s narodom htjede žrtvovati. 
\par 14 Kada su to dočuli apostoli Barnaba  i Pavao, razdriješe haljine i uletješe u narod vičući: 
\par 15 "Ljudi, što to radite? I mi smo smrtnici, baš kao i vi! Navješćujemo  vam da se od tih ispraznosti obratite k Bogu živomu koji stvori  nebo i zemlju, more i sve što je u njima. 
\par 16 On je u prošlim  naraštajima pustio da svi pogani pođu svojim putovima. 
\par 17 Ipak  ne ostavi sebe neposvjedočena: dobročinstva iskazuje, s neba  vam kišu daje i vremena plodonosna, napunja hranom i radošću  srca vaša." 
\par 18 I tako govoreći, jedva sklonuše mnoštvo da im  ne žrtvuje. 
\par 19 Uto iz Antiohije i Ikonija nadođu neki Židovi, pridobiju  svjetinu te kamenuju Pavla i odvuku ga izvan grada misleći da  je mrtav. 
\par 20 Kad ga pak okružiše učenici, usta on i uđe u grad.  Sutradan ode s Barnabom u Derbu. 
\par 21 Pošto navijestiše evanđelje tomu gradu i mnoge učiniše  učenicima, vratiše se u Listru, u Ikonij i u Antiohiju. 
\par 22 Učvršćivali  su duše učenika bodreći ih da ustraju u vjeri jer da nam je kroz  mnoge nevolje ući u kraljevstvo Božje. 
\par 23 Postavljali su im  po crkvama starješine te ih, nakon molitve i posta, povjeravahu  Gospodinu u kojega su povjerovali. 
\par 24 Pošto su prešli Pizidiju, stigoše u Pamfiliju. 
\par 25 U  Pergi navijestiše Riječ pa siđu u Ataliju. 
\par 26 Odande pak odjedriše  u Antiohiju, odakle ono bijahu povjereni milosti Božjoj za djelo  koje izvršiše. 
\par 27 Kada stigoše, sabraše Crkvu i pripovjediše što sve učini  Bog po njima: da i poganima otvori vrata vjere. 
\par 28 I proveli  su nemalo vremena s učenicima. 


\chapter{15}

\par 1 Uto neki siđoše iz Judeje i počeše učiti braću: "Ako se ne  obrežete po običaju Mojsijevu, ne možete se spasiti." 
\par 2 Kad  između njih te Pavla i Barnabe nasta prepirka i raspra nemalena, odrediše da Pavao i Barnaba i još neki drugi između njih uzađu  u Jeruzalem k apostolima i starješinama poradi tog pitanja. 
\par 3 Oni su dakle, ispraćeni od Crkve, prolazili kroz Feniciju  i Samariju pripovijedajući o obraćenju pogana i donoseći svoj  braći veliku radost. 
\par 4 Kada pak stigoše u Jeruzalem, primi ih  Crkva, apostoli i starješine. Ispripovjediše što sve Bog učini  po njima. 
\par 5 Onda ustanu neki od onih što iz farizejske sljedbe bijahu  prigrlili vjeru pa reknu: "Treba ih obrezati i zapovjediti im  da opslužuju Zakon Mojsijev." 
\par 6 Nato se apostoli i starješine sastanu da to razmotre. 
\par 7 Nakon duge raspre ustade Petar i reče im: "Braćo, vi znate  kako me Bog od najprvih dana između vas izabra da iz mojih usta  pogani čuju riječ evanđelja i uzvjeruju. 
\par 8 I Bog, Poznavatelj  srdaca, posvjedoči za njih: dade im Duha Svetoga kao i nama. 
\par 9 Nikakve razlike nije pravio između nas i njih: vjerom očisti  njihova srca. 
\par 10 Što dakle sada iskušavate Boga stavljajući  učenicima na vrat jaram kojeg ni oci naši ni mi nismo mogli nositi? 
\par 11 Vjerujemo, naprotiv: po milosti smo Gospodina Isusa spašeni, baš kao i oni." 
\par 12 Nato sve mnoštvo umuknu. Slušali su Barnabu i Pavla koji  pripovjedahu kolika je znamenja i čudesa Bog po njima učinio  među poganima. 
\par 13 Kad oni ušutješe, progovori Jakov: "Poslušajte me, braćo! 
\par 14 Šimun je izložio kako se Bog već na početku pobrinu između  pogana uzeti narod imenu svojemu. 
\par 15 S time su u skladu riječi  Proroka. Ovako je doista pisano: 
\par 16 Nakon toga vratit ću se i opet podići pali šator Davidov, iz ruševina ga podići, opet ga sazidati 
\par 17 da preostali ljudi potraže Gospodina i svi pogani na koje je zazvano ime moje, govori Gospodin, koji to 
\par 18 obznanjuje odvijeka. 
\par 19 Zato smatram da ne valja dodijavati onima koji se s poganstva  obraćaju k Bogu, 
\par 20 nego im poručiti da se uzdržavaju od mesa  okaljana idolima, od bludništva, od udavljenoga i od krvi. 
\par 21 Ta  Mojsije od pradavnih naraštaja ima po gradovima propovjednike  koji ga u sinagogama svake subote čitaju." 
\par 22 Tad apostoli i starješine zajedno sa svom Crkvom zaključe  izabrati neke muževe između sebe i poslati ih u Antiohiju s Pavlom  i Barnabom. Bijahu to Juda zvani Barsaba, i Sila, muževi vodeći  među braćom. 
\par 23 Po njima pošalju ovo pismo: "Apostoli i starješine, braća, braći iz poganstva po Antiohiji, Siriji i Ciliciji - pozdrav!" 
\par 24 "Budući da smo čuli kako vas neki od naših, ali bez našega  naloga, nekakvim izjavama smetoše i duše vam uznemiriše, 
\par 25 zaključismo  jednodušno izabrati neke muževe i poslati ih k vama zajedno s  našim ljubljenim Barnabom i Pavlom, 
\par 26 ljudima koji su svoje  živote izložili za ime Gospodina našega Isusa Krista. 
\par 27 Šaljemo  vam dakle Judu i Silu. Oni će vam i usmeno priopćiti to isto. 
\par 28 Zaključismo Duh Sveti i mi ne nametati vam nikakva tereta  osim onoga što je potrebno: 
\par 29 uzdržavati se od mesa žrtvovana  idolima, od krvi, od udavljenoga i od bludništva. Budete li se  toga držali, dobro ćete učiniti. Živjeli!" 
\par 30 Oni su se dakle oprostili i sišli u Antiohiju; sabrali  su mnoštvo i predali pismo. 
\par 31 Kad ga pročitaše, svi se obradovaše  zbog ohrabrenja. 
\par 32 Juda i Sila, i sami proroci, mnogim riječima  ohrabriše i utvrdiše braću. 
\par 33 Neko se vrijeme zadrže pa se  onda s mirom od braće vrate onima koji ih poslaše. 
\par 34 # 
\par 35 A  Pavao i Barnaba ostadoše u Antiohiji naučavajući i navješćujući  zajedno s mnogima drugima riječ Gospodnju. 
\par 36 Nakon nekog vremena reče Pavao Barnabi: "Vratimo se i  pohodimo braću po svim gradovima u kojima smo navješćivali riječ  Gospodnju, da vidimo kako su!" 
\par 37 Barnaba je htio povesti i  Ivana zvanog Marko. 
\par 38 Pavao pak nije smatrao uputnim sa sobom  voditi onoga koji se u Pamfiliji odvojio od njih te nije s njima  pošao na djelo. 
\par 39 Spopade ih takva ogorčenost da se raziđoše:  Barnaba povede Marka i otplovi na Cipar, 
\par 40 a Pavao sebi izabra  Silu pa od braće povjeren milosti Gospodnjoj 
\par 41 proputova Siriju  i Ciliciju, utvrđujući Crkve. 


\chapter{16}

\par 1 Stiže tako u Derbu i Listru. Ondje, gle, bijaše učenik neki  imenom Timotej, sin neke pokrštene Židovke i oca Grka. 
\par 2 Uživao  je dobar glas među braćom u Listri i Ikoniju. 
\par 3 Pavao htjede  da on pođe s njime pa ga uze i obreza zbog Židova koji bijahu  u onim mjestima. Jer svi su znali da mu je otac Grk. 
\par 4 I kako su prolazili gradovima, predavali su im za opsluživanje  odredbe koje su apostoli i starješine utvrdili u Jeruzalemu. 
\par 5 Tako se Crkve učvršćivahu u vjeri i broj im se danomice povećavao. 
\par 6 Prođoše Frigiju i galacijski kraj jer ih je Duh Sveti  spriječio propovijedati riječ u Aziji. 
\par 7 Kad su došli do Mizije, htjedoše u Bitiniju, ali im ne dopusti Duh Isusov. 
\par 8 Onda prođoše  Miziju i siđoše u Troadu. 
\par 9 Noću je Pavao imao viđenje: Makedonac neki stajaše i zaklinjaše  ga: "Prijeđi u Makedoniju i pomozi nam!" 
\par 10 Nakon viđenja nastojasmo  odmah otputovati u Makedoniju, uvjereni da nas Bog zove navješćivati  im evanđelje. 
\par 11 Otplovismo iz Troade i zaputismo se ravno u Samotraku  pa sutradan u Neapol, 
\par 12 a odande u naseobinu Filipe - grad  prvog dijela Makedonije. U tom se gradu zadržasmo nekoliko dana. 
\par 13 U dan subotni iziđosmo izvan gradskih vrata k rijeci, gdje  smo mislili da će biti bogomolja. Sjedosmo i stadosmo govoriti  okupljenim ženama. 
\par 14 Slušala je tako i neka bogobojazna žena  imenom Lidija, prodavačica grimiza iz grada Tijatire. Gospodin  joj otvori srce, te ona prihvati što je Pavao govorio. 
\par 15 Pošto  se pak krsti ona i njezin dom, zamoli: "Ako smatrate da sam vjerna  Gospodinu, uđite u moj dom i ostanite u njemu." I prisili nas. 
\par 16 Jednom nas na putu u bogomolju sretne neka ropkinja koja  je imala duha vračarskoga i gatajući donosila veliku dobit svojim  gospodarima. 
\par 17 Pošla je za Pavlom i za nama te vikala: "Ovi  su ljudi sluge Boga Svevišnjega; navješćuju vam put spasenja." 
\par 18 To je činila mnogo dana. Pavlu to napokon dodija pa se okrenu  i reče duhu: "Zapovijedam ti u ime Isusa Krista: iziđi iz nje!"  I iziđe toga časa. 
\par 19 Kad njezini gospodari vidješe da im nesta nade u dobit, pograbiše Pavla i Silu te ih odvukoše na trg pred glavare. 
\par 20 Privedoše  ih pretorima i rekoše: "Ovi ljudi uznemiruju naš grad. Židovi  su 
\par 21 te šire običaje kojih mi Rimljani ne smijemo ni prihvatiti  ni držati." 
\par 22 Nato svjetina nahrupi na njih, a pretori trgoše  s njih odijelo i zapovjediše da se išibaju. 
\par 23 Pošto ih izudaraše, bace ih u tamnicu i zapovjede tamničaru da ih pomno čuva. 
\par 24 Primivši  takvu zapovijed, uze ih on i baci u nutarnju tamnicu, a noge  im stavi u klade. 
\par 25 Oko ponoći su Pavao i Sila molili pjevajući hvalu Bogu, a uznici ih slušali. 
\par 26 Odjednom nasta potres velik te se poljuljaše  temelji zatvora, umah se otvoriše sva vrata, i svima spadoše  okovi. 
\par 27 Tamničar se prenu oda sna pa kad ugleda tamnička vrata  otvorena, trgnu mač i samo što se ne ubi misleći da su uznici  pobjegli. 
\par 28 Ali Pavao povika iza glasa: "Ne čini sebi nikakva  zla! Svi smo ovdje!" 
\par 29 Onaj nato zaiska svjetlo, uleti i dršćući baci se pred  Pavla i Silu; 
\par 30 izvede ih i upita: "Gospodo, što mi je činiti  da se spasim?" 
\par 31 Oni će mu: "Vjeruj u Gospodina Isusa i spasit  ćeš se - ti i dom tvoj!" 
\par 32 Onda navijestiše riječ Gospodnju  njemu i svima u domu njegovu. 
\par 33 Te iste noćne ure uze ih, opra im rane pa se odmah krsti  - on i svi njegovi. 
\par 34 Onda ih uvede u dom, prostre stol te  se zajedno sa svim domom obradova što je povjerovao Bogu. 
\par 35 Kad se razdani, poslaše pretori liktore s porukom: "Pusti  te ljude!" 
\par 36 Tamničar to priopći Pavlu: "Pretori, reče, poručiše  da vas pustim. Iziđite dakle sad i pođite u miru!" 
\par 37 Nato im  Pavao odvrati: "Javno su nas neosuđene išibali, nas rimske građane, i bacili u tamnicu. A sada da nas potajno izbace? Nipošto, nego  neka oni sami dođu i izvedu nas!" 
\par 38 Liktori to jave pretorima. Oni su se uplašili kada doznaše  da su Rimljani. 
\par 39 Zato dođu da ih nagovore pa ih izvedu i zamole  da odu iz grada. 
\par 40 Izišavši iz tamnice, oni pođu k Lidiji,  pogledaju i obodre braću pa odu. 


\chapter{17}

\par 1 Prošavši kroz Amfipol i Apoloniju, stigoše u Solun, gdje bijaše  židovska sinagoga. 
\par 2 Po običaju uđe Pavao onamo. Tri je subote  s njima raspravljao na temelju Pisama. 
\par 3 Tumačio je i izlagao:  "Trebalo je da Krist trpi i uskrsne od mrtvih. Taj Krist jest  Isus koga vam ja navješćujem." 
\par 4 Neki se od njih uvjeriše pa  se pridružiše Pavlu i Sili; tako i veliko mnoštvo bogobojaznih  Grka i nemalo uglednih žena. 
\par 5 Židove nato spopade zavist pa pridobiše neke opake uličnjake, potakoše ih i pobuniše grad te nahrupiše u kuću Jasonovu tražeći  da se Pavao i Sila izvedu pred narod. 
\par 6 Kako ih ne nađoše, odvukoše  Jasona i neke od braće pred gradske glavare vičući: "Evo i ovdje  onih koji pobuniše sav svijet. 
\par 7 Jason ih je ugostio. Svi oni  rade protiv carskih odredaba: tvrde da postoji drugi kralj -  Isus." 
\par 8 Time uzbuniše svjetinu i glavare koji su to čuli 
\par 9 te  oni od Jasona i ostalih uzeše jamčevinu pa ih pustiše. 
\par 10 Braća su pak brže-bolje noću odaslala Pavla i Silu u  Bereju. Kad su stigli, odoše u židovsku sinagogu. 
\par 11 Ovi su  Židovi bili plemenitiji od solunskih: primili su Riječ sa svom  spremnošću i danomice istraživali Pisma, da li je to tako. 
\par 12 Mnogi  od njih stoga povjerovaše, a tako i nemalo uglednih grčkih žena  i muževa. 
\par 13 Ali kad su solunski Židovi doznali da Pavao i u Bereji  navješćuje riječ Božju, odoše te i ondje podjariše i uzbuniše  svjetinu. 
\par 14 Braća tada brže-bolje uputiše Pavla k moru. Sila  pak i Timotej ostadoše ondje. 
\par 15 Pratioci dovedoše Pavla do  Atene pa se vratiše noseći Sili i Timoteju zapovijed da što prije  dođu k njemu. 
\par 16 Dok ih je u Ateni iščekivao, ogorči se Pavao u duši promatrajući  kako je grad pokumiren. 
\par 17 Međutim raspravljaše u sinagogi sa  Židovima i bogobojaznima, a na trgu svaki dan s onima koji bi  se ondje zatekli. 
\par 18 Dobacivahu mu i neki od epikurejskih i  stoičkih filozofa. Jedni su govorili: "Što bi htjela reći ta  čavka?" Drugi pak: "Navješćuje, čini se, neke tuđe bogove." Jer  navješćivaše Isusa i uskrsnuće. 
\par 19 Onda su ga uzeli i odveli na Areopag i upitali: "Bismo  li mogli znati kakav to nov nauk naučavaš? 
\par 20 Čudnovatim nam  nekim tvrdnjama uši puniš. Željeli bismo stoga znati što bi to  imalo biti." 
\par 21 Nijedan Atenjanin ni doseljeni stranac ni na što drugo  ne trati vrijeme nego na pripovijedanje i slušanje novosti. 
\par 22 Tada Pavao stade posred Areopaga i reče: "Atenjani! U  svemu ste, vidim, nekako veoma bogoljubni. 
\par 23 Doista, prolazeći  i promatrajući vaše svetinje nađoh i žrtvenik s natpisom: Nepoznatom  Bogu. Što dakle ne poznajete, a štujete, to vam ja navješćujem." 
\par 24 "Bog koji stvori svijet i sve na njemu,  on, neba i zemlje Gospodar, ne prebiva u rukotvorenim hramovima; 
\par 25 i ne poslužuju ga ljudske ruke, kao da bi što trebao, on  koji svima daje život, dah i - sve. 
\par 26 Od jednoga sazda cijeli  ljudski rod da prebiva po svem licu zemlje; ustanovi određena  vremena i međe prebivanja njihova 
\par 27 da traže Boga, ne bi li  ga kako napipali i našli. Ta nije daleko ni od koga od nas. 
\par 28 U  njemu doista živimo, mičemo se i jesmo, kao što i neki od vaših  pjesnika rekoše: "Njegov smo čak i rod!" 
\par 29 "Ako smo dakle rod Božji, ne smijemo smatrati da je božanstvo  slično zlatu, srebru ili kamenu, liku isklesanu umijećem i maštom  ljudskom." 
\par 30 "I ne obazirući se na vremena neznanja, nutka sada Bog  ljude da se svi i posvuda obrate 
\par 31 jer ustanovi Dan u koji  će suditi svijetu po pravdi, po Čovjeku kojega odredi, pred svima ovjerovi uskrisivši ga od mrtvih." 
\par 32 Kad čuše "uskrsnuće od mrtvih", jedni se stadoše rugati, a drugi rekoše: "Još ćemo te o tom slušati!" 
\par 33 Tako se Pavao  povuče od njih. 
\par 34 Neki ipak prionuše uza nj i povjerovaše;  među njima i Dionizije Areopagit, neka žena imenom Damara i drugi  s njima. 


\chapter{18}

\par 1 Nakon toga napusti Pavao Atenu i ode u Korint. 
\par 2 Ondje nađe  nekog Židova imenom Akvilu, rodom iz Ponta, koji netom bijaše  došao iz Italije sa svojom ženom Priscilom jer je Klaudije naredio  da svi Židovi napuste Rim. Pohodio ih je 
\par 3 i, kako bijahu istog  zanimanja, ostao kod njih i radio. Po zanimanju bijahu šatorari. 
\par 4 Svake je pak subote raspravljao u sinagogi i uvjeravao Židove  i Grke. 
\par 5 Kad iz Makedonije pristigoše Sila i Timotej, Pavao se  potpuno posveti Riječi svjedočeći Židovima da Isus jest Krist. 
\par 6 Kako se pak oni stadoše protiviti i huliti, otrese on haljine  i reče im: "Krv vaša na glave vaše! Ja sam nedužan. Od sada idem  k poganima." 
\par 7 I ode odande te prijeđe u kuću nekoga bogobojazna  čovjeka, imenom Ticija Justa, čija kuća bijaše tik do sinagoge. 
\par 8 A nadstojnik sinagoge Krisp povjerova Gospodinu zajedno sa  svim svojim domom. I mnogi od Korinćana koji su to slušali povjerovaše  i pokrstiše se. 
\par 9 Jedne noći reče Gospodin Pavlu u viđenju: "Ne boj se,  nego govori i ne daj se ušutkati! 
\par 10 Ta ja sam s tobom  i nitko se neće usuditi da ti naudi. Jer mnogo je naroda mojega  u ovome gradu." 
\par 11 Tako se zadrža godinu i šest mjeseci naučavajući  među njima riječ Božju. 
\par 12 Ali dok je Galion bio prokonzul Ahaje, navališe Židovi  jednodušno na Pavla, dovukoše ga u sudnicu 
\par 13 i rekoše: "Ovaj  potiče ljude da protiv zakona štuju Boga." 
\par 14 Pavao samo što  nije zaustio kadli Galion reče Židovima: "Da je posrijedi zločin  kakav ili nedjelo opako, saslušao bih vas, Židovi, kako je pravo; 
\par 15 je li pak raspra o riječi i imenima i o nekom vašem zakonu, proviđajte sami; u tome ja ne želim biti sudac." 
\par 16 I otpremi  ih iz sudnice. 
\par 17 A oni svi pograbiše nadstojnika sinagoge Sostena  i stadoše ga šibati pred sudnicom. Galion nije za to ništa mario. 
\par 18 Pavao osta još podosta vremena, a onda se oprosti s braćom  pa pošto se u Kenhreji ošiša jer imaše zavjet, zaplovi prema  Siriji, a s njime i Priscila i Akvila. 
\par 19 Stigoše u Efez. Tu  ih ostavi, a on uđe u sinagogu i stade raspravljati sa Židovima. 
\par 20 Oni ga zamole da ostanu duže vremena, ali on ne pristade, 
\par 21 nego se oprosti: "Još ću se, reče, vratiti k vama, bude  li Božja volja." I otplovi iz Efeza. 
\par 22 Kad stiže u Cezareju, uziđe pozdraviti Crkvu pa onda siđe  u Antiohiju. 
\par 23 Neko se vrijeme zadrža pa onda ode i zareda galacijskim  područjem i Frigijom utvrđujući sve učenike. 
\par 24 Uto neki Židov imenom Apolon, rodom Aleksandrijac, čovjek  rječit i upućen u Pisma, stiže u Efez. 
\par 25 On bijaše upućen u  Put Gospodnji pa je vatrene duše govorio i naučavao pomno o Isusu, premda je znao samo za Ivanovo krštenje. 
\par 26 Poče on tako smjelo  govoriti u sinagogi. Čuše ga Priscila i Akvila, uzeše ga k sebi  i pomnije mu izložiše Put Božji. 
\par 27 A kad je nakanio otići u Ahaju, ohrabriše ga braća i  napisaše učenicima da ga prime. Kad je stigao onamo, uvelike  je koristio vjernicima po milosti 
\par 28 jer je snažno pobijao Židove  javno pokazujući iz Pisama da Isus jest Krist. 


\chapter{19}

\par 1 Dok je Apolon bio u Korintu, Pavao, pošto prođe gornje krajeve, dođe u Efez, nađe neke učenike 
\par 2 pa ih upita: "Jeste li primili  Duha Svetoga kad ste povjerovali?" Oni će mu: "Ta ni čuli nismo  da ima Duh Sveti." 
\par 3 Nato će on: "Kako ste onda kršteni?" "Krštenjem  Ivanovim", odvrate oni. 
\par 4 Nato će Pavao: "Ivan je krstio krštenjem  obraćenja govoreći narodu da vjeruje u Onoga koji za njim dolazi, to jest u Isusa." 
\par 5 Čuvši to, krste se u ime Gospodina Isusa, 
\par 6 pa kad Pavao položi na njih ruke, dođe Duh Sveti na njih  te stanu govoriti drugim jezicima i prorokovati. 
\par 7 Bijaše u  svemu dvanaestak muževa. 
\par 8 Onda Pavao uđe u sinagogu te je tri mjeseca hrabro raspravljao  i uvjeravao o kraljevstvu Božjem. 
\par 9 Ali kako neki, okorjeli  i nepokorni, ocrnjivahu ovaj Put pred mnoštvom, odstupi od njih, odvoji učenike i danomice raspravljaše u školi nekog Tirana. 
\par 10 Trajalo je to dvije godine, tako da su svi azijski žitelji, Židovi i Grci, čuli riječ Božju. 
\par 11 Bog je pak činio čudesa nesvakidašnja po rukama Pavlovima 
\par 12 tako da bi na bolesnike stavljali rupce ili rublje s Pavlova  tijela pa bi s njih nestajalo bolesti i zli duhovi iz njih izlazili. 
\par 13 Zato i neki Židovi zaklinjaoci-potukači pokušaše zazvati  ime Gospodina Isusa nad one koji imahu zle duhove. Govorili su:  "Zaklinjem vas Isusom koga Pavao propovijeda." 
\par 14 To činjaše  sedam sinova nekog Skeve, židovskoga velikog svećenika. 
\par 15 Zli  im duh odvrati: "Isusa poznajem i Pavla znam, ali tko ste vi?" 
\par 16 I čovjek u kome bijaše zli duh, nasrnu na njih i nadjača  ih te oni goli i izranjeni pobjegoše iz one kuće. 
\par 17 Doznaše  to svi žitelji efeški, Židovi i Grci, pa ih sve obuze strah te  se stade veličati ime Gospodina Isusa. 
\par 18 Mnogi pak od onih koji su povjerovali dolazili su ispovijedati  i očitovati svoja djela. 
\par 19 I podosta onih koji su se bavili  praznovjerjem donosili su knjige i spaljivali ih pred svima.  Procijeniše ih te nađoše da vrijede pedeset tisuća srebrnjaka. 
\par 20 TAko se snagom Gospodnjom Riječ širila i jačala. 
\par 21 Pošto se to ispuni, naumi Pavao preko Makedonije i Ahaje  otići u Jeruzalem te reče: "Pošto budem ondje, trebat će da i  Rim vidim." 
\par 22 Onda posla u Makedoniju dvojicu svojih poslužitelja, Timoteja i Erasta, a on provede još neko vrijeme u Aziji. 
\par 23 Nekako u ono doba nasta nemalena pobuna protiv ovog Puta. 
\par 24 Neki srebrar, imenom Demetrije, izrađivao je srebrne hramiće  Artemidine i namicao obrtnicima nemalu dobit. 
\par 25 Skupi on njih  i sve koji su se bavili takvim poslom te im reče: "Ljudi, vi  znate, u ovom je umijeću naše blagostanje. 
\par 26 A vidite i čujete  da je taj Pavao ne samo u Efezu nego gotovo i u svoj Aziji uvjerio  i preokrenuo poveliko mnoštvo govoreći da nema bogova rukama  izdjeljanih. 
\par 27 Tako prijeti opasnost ne samo da na zao glas  dođe naše zanimanje, nego i to da se ništa neće držati do hrama  velike božice Artemide te će nestati veličanstva one koju štuje  sva Azija i sav svijet." 
\par 28 Čuvši to, razgnjeve se pa poviču: "Velika je Artemida  efeška!" 
\par 29 Sav se grad uskomeša; jednodušno nahrupe u kazalište  vukući sa sobom Makedonce Gaja i Aristarha, suputnike Pavlove. 
\par 30 Kad je Pavao htio među narod, ne dopustiše mu učenici. 
\par 31 Čak  i neki azijarsi, njegovi prijatelji, poslaše k njemu i zamoliše  da ne dolazi u kazalište. 
\par 32 Jedni su izvikivali jedno, drugi drugo jer je skup bio  uskomešan te mnogi nisu ni znali zašto su se strčali. 
\par 33 Neki  iz svjetine uputiše nekog Aleksandra jer su ga Židovi gurali  naprijed. Aleksandar pak mahnu rukom i htjede se obraniti pred  narodom. 
\par 34 Ali kada doznaše da je Židov, udarahu gotovo dva  sata svi u jedan glas: "Velika je Artemida efeška!" 
\par 35 Onda  tajnik umiri svjetinu pa reče: "Efežani! Tko to od ljudi ne zna  da je grad Efez čuvar hrama velike Artemide i kipa s neba palog? 
\par 36 Budući dakle da je to neporecivo, valja da budete mirni te  ništa brzopleto ne činite. 
\par 37 Doveli ste ove ljude, a nisu ni  svetokradice ni hulitelji naše božice. 
\par 38 Ako pak Demetrije  i njegovi obrtnici imaju protiv koga kakvu tužbu, sudovi se sastaju, a tu su i prokonzuli. Neka se tuže! 
\par 39 Ištete li pak što drugo, u zakonitu će se skupu riješiti. 
\par 40 Ta izlažemo se opasnosti  da za ovo današnje budemo optuženi s pobune jer nema nikakva  razloga kojim bismo mogli opravdati ovu strku." To rekavši, raspusti  skup. 
\par 41 - - - 


\chapter{20}

\par 1 Kad se sleže metež, posla Pavao po učenike, ohrabri ih, pozdravi  i otputova u Makedoniju. 
\par 2 Prešavši one krajeve, hrabreći braću  besjedom mnogom, dođe u Grčku 
\par 3 i provede ondje tri mjeseca.  Upravo kad je htio otploviti u Siriju, postaviše mu Židovi zasjedu  pa odluči vratiti se preko Makedonije. 
\par 4 Pratili su ga: Sopater  Pirov, Berejac, Solunjani Aristarh i Sekund, Gaj Derbanin, Timotej  i Azijci Tihik i Trofim. 
\par 5 Oni odoše prije te nas dočekaše u  Troadi. 
\par 6 Mi pak nakon dana Beskvasnih kruhova otplovismo iz  Filipa i nakon pet dana dođosmo k njima u Troadu gdje proboravismo  sedam dana. 
\par 7 U prvi dan tjedna, kad se sabrasmo lomiti kruh, Pavao  im govoraše i kako je sutradan kanio otputovati, probesjedi sve  do ponoći. 
\par 8 U gornjoj sobi gdje smo se sabrali bijaše dosta  svjetiljaka. 
\par 9 Na prozoru je sjedio neki mladić imenom Eutih.  Kako je Pavao dulje govorio, utone on u dubok san. Svladan snom, pade s trećeg kata dolje. Digoše ga mrtva. 
\par 10 Pavao siđe, nadnese  se nad dječaka, obujmi ga i reče: "Ne uznemirujte se! Duša je  još u njemu!" 
\par 11 Zatim se pope pa pošto razlomi kruh i blagova, dugo je još zborio, sve do zore. Tad otputova. 
\par 12 Mladića odvedoše  živa, neizmjerno utješeni. 
\par 13 Mi pak pođosmo naprijed lađom: otplovismo u As. Odande  smo imali povesti Pavla - tako je odredio kad se spremao poći  pješice. 
\par 14 Kad nam se u Asu pridruži, uzesmo ga i stigosmo  u Mitilenu. 
\par 15 Odande odjedrismo sutradan i stigosmo nadomak  Hija, prekosutra krenusmo u Sam, a idućeg dana stigosmo u Milet. 
\par 16 Jer Pavao je odlučio mimoići Efez da se ne bi zadržao u Aziji:  žurio se da, uzmogne li, na dan Pedesetnice bude u Jeruzalemu. 
\par 17 Ipak iz Mileta posla u Efez po starješine Crkve. 
\par 18 Kad  stigoše, reče im: "Vi znate kako sam se sve vrijeme, od prvog  dana kada stupih u Aziju, ponašao među vama: 
\par 19 služio sam Gospodinu  sa svom poniznošću u suzama i kušnjama koje me zadesiše zbog  zasjeda židovskih; 
\par 20 ništa korisno nisam propustio navijestiti  vam i naučiti vas - javno i po kućama; 
\par 21 upozoravao sam Židove  i Grke da se obrate k Bogu i da vjeruju u Gospodina našega Isusa." 
\par 22 "A sad, evo, okovan Duhom idem u Jeruzalem. Što će me  u njemu zadesiti, ne znam, 
\par 23 osim što mi Duh Sveti u svakom  gradu jamči da me čekaju okovi i nevolje. 
\par 24 Ali ni najmanje  mi nije do života, samo da dovršim trku svoju i službu koju primih  od Gospodina Isusa: svjedočiti za evanđelje milosti Božje." 
\par 25 "I sad, evo, znam: nećete više vidjeti lica moga, svi  vi posred kojih prođoh propovijedajući Kraljevstvo. 
\par 26 Zato  vam u ovaj dan današnji jamčim: čist sam od krvi sviju 
\par 27 jer  nisam propustio navijestiti vam ništa od svega nauma Božjega." 
\par 28 "Pazite na sebe i na sve stado u kojem vas Duh Sveti  postavi nadglednicima, da pasete Crkvu Božju koju steče krvlju  svojom." 
\par 29 "Ja znam da će nakon mog odlaska među vas uljesti vuci  okrutni koji ne štede stada, 
\par 30 a između vas će samih ustati  ljudi koji će iskrivljavati nauk da bi odvukli učenike za sobom. 
\par 31 Zato bdijte imajući na pameti da sam tri godine bez prestanka  noću i danju suze lijevajući urazumljivao svakoga od vas." 
\par 32 "I sada vas povjeravam Bogu i Riječi milosti njegove  koja je kadra izgraditi vas i dati vam baštinu među svima posvećenima." 
\par 33 "Ni za čijim srebrom, zlatom ili ruhom nisam hlepio. 
\par 34 Sami znate: za potrebe moje i onih koji su sa mnom zasluživale  su ove ruke. 
\par 35 U svemu vam pokazah: tako se trudeći treba se  zauzimati za nemoćne i na pameti imati riječi Gospodina Isusa  jer on reče: 'Blaženije je davati nego primati.'" 
\par 36 Kada to doreče, klekne te se zajedno sa svima njima pomoli. 
\par 37 Tad svi briznuše u velik plač, obisnuše Pavlu oko vrata i  stadoše ga cjelivati, 
\par 38 ražalošćeni nadasve riječju koju im  reče: da više neće vidjeti lica njegova. Zatim ga ispratiše na  lađu. 


\chapter{21}

\par 1 Pošto se otrgosmo od njih, zaplovismo. Jedreći ravno, stigosmo  na Kos, a sutradan na Rod pa odande u Pataru. 
\par 2 Kad nađosmo  lađu za Feniciju, popesmo se i otplovismo. 
\par 3 Kad bijasmo napomol  Cipru, ostavismo ga slijeva jedreći prema Siriji. Pristadosmo  u Tiru jer je ondje lađa imala iskrcati tovar. 
\par 4 Pronađosmo  učenike i ostadosmo ondje sedam dana. Oni po Duhu nagovarahu  Pavla da ne uzlazi u Jeruzalem. 
\par 5 Ali kad nam istekoše dani, ipak otputovasmo. Ispratiše nas svi, sa ženama i djecom, do  izvan grada. Na žalu klekosmo i pomolismo se. 
\par 6 Pozdravismo  se, popesmo se na lađu, a oni se vratiše kući. 
\par 7 Tako dovršismo plovidbu. Iz Tira stigosmo u Ptolemaidu.  Pozdravili smo braću i ostali jedan dan u njih. 
\par 8 Sutradan otputovasmo  i stigosmo u Cezareju. Uđosmo u kuću Filipa evanđelista, jednog  od sedmorice, i ostadosmo kod njega. 
\par 9 On je imao četiri kćeri  djevice koje su prorokovale. 
\par 10 Kako smo se zadržali mnogo dana, siđe iz Judeje neki  prorok imenom Agab, 
\par 11 dođe k nama, uze Pavlov pojas, sveza  sebi noge i ruke te reče: "Ovo govori Duh Sveti: Čovjeka čiji  je ovo pojas ovako će svezati Židovi u Jeruzalemu i predati u  ruke pogana." 
\par 12 Kada smo to čuli, stadosmo mi i mještani zaklinjati  Pavla da ne uzlazi u Jeruzalem. 
\par 13 Nato on odvrati: "Što plačete  i parate mi srce? Ta spreman sam ne samo biti svezan nego i umrijeti  u Jeruzalemu za ime Gospodina Isusa." 
\par 14 A kako se nije dao  nagovoriti, ušutjesmo rekavši: "Gospodnja budi volja!" 
\par 15 Nakon tih dana spremismo se i uzađosmo u Jeruzalem. 
\par 16 S  nama pođoše i učenici iz Cezareje pa nas odvedoše k nekomu Mnasonu  Cipraninu, starom učeniku, da u njega odsjednemo. 
\par 17 Kad stigosmo u Jeruzalem, primiše nas braća radosno. 
\par 18 Sutradan ode Pavao zajedno s nama k Jakovu. Nađoše se ondje  i sve starješine. 
\par 19 Pošto ih pozdravi, stade im potanko izlagati  što učini Bog među poganima po njegovoj službi. 
\par 20 Pošto su  ga oni poslušali, dadoše slavu Bogu pa mu rekoše: "Vidiš, brate:  deseci su tisuća Židova povjerovali i svi su revnitelji Zakona. 
\par 21 A o tebi im je dojavljeno da sve Židove koji su među poganima  upućuješ na otpad od Mojsija učeći ih da ne obrezuju djece i  ne žive po običajima. 
\par 22 Što dakle? Čut će svakako da si došao. 
\par 23 Učini stoga što ti kažemo. U nas su četiri čovjeka koji imaju  zavjet. 
\par 24 Njih uzmi, s njima se zajedno posveti, plati za njih  da se ošišaju pa će svi spoznati da nema ništa od onoga što im  je o tebi dojavljeno, nego da si i ti na pravu putu i da opslužuješ  Zakon. 
\par 25 A što se tiče pogana koji povjerovaše - poslali smo  što odlučismo: da se klone mesa žrtvovana idolima, krvi, udavljenoga  i bludništva." 
\par 26 Nato Pavao uze one ljude, sutradan se s njima zajedno  posveti, uđe u Hram, oglasi svršetak dana posvećenja nakon kojih  će se za svakoga od njih prinijeti prinos. 
\par 27 Kad se upravo navršavalo tih sedam dana, neki ga Židovi  iz Azije opaze u Hramu, uzbune sav narod pa podignu na nj ruke 
\par 28 vičući: "Izraelci, u pomoć! Evo čovjeka koji sve posvuda  poučava protiv naroda, Zakona i ovoga mjesta pa je još i Grke  uveo u Hram i oskvrnuo ovo sveto mjesto." 
\par 29 Jer prije su s  njime u Gradu vidjeli Trofima Efežanina i mislili da je Pavao  njega uveo u Hram. 
\par 30 Sav se grad uskomeša, nasta strka naroda. Pograbe Pavla  i odvuku ga izvan Hrama pa odmah pozatvaraju vrata. 
\par 31 Dok su  mu o glavi radili, dođe do tisućnika čete glas da je sav Jeruzalem  uzavreo. 
\par 32 On odmah uze vojnike i satnike pa otrča dolje k  njima. Oni pak kako ugledaše tisućnika i vojnike, prestadoše  udarati Pavla. 
\par 33 Onda se tisućnik približi, uhvati ga, zapovjedi  da ga okuju dvojim verigama pa stade ispitivati tko je i što  je učinio. 
\par 34 Iz svjetine su jedni izvikivali ovo, drugi ono.  Kako zbog graje nije mogao saznati ništa pouzdano, zapovjedi  da se odvede u vojarnu. 
\par 35 Kad se Pavao pojavi na stubama, morali su ga vojnici  nositi zbog silovitosti svjetine. 
\par 36 Jer mnoštvo je naroda išlo  za njima i vikalo: "Smakni ga!" 
\par 37 Upravo na ulazu u vojarnu reče Pavao tisućniku: "Smijem  li nešto reći?" On ga upita: "Zar znaš grčki? 
\par 38 Ti dakle nisi  onaj Egipćanin koji je prije nekoliko dana pobunio i u pustinju  odveo one četiri tisuće bodežara?" 
\par 39 Pavao odvrati: "Ja sam  Židov iz Tarza cilicijskoga, građanin grada znamenitoga. Molim  te, dopusti mi progovoriti narodu." 
\par 40 Kad mu on dopusti, Pavao  stojeći na stubama mahnu rukom narodu pa kad nasta velika tišina, prozbori hebrejskim jezikom: 


\chapter{22}

\par 1 "Braćo i oci, poslušajte što ću vam sad u svoju obranu reći." 
\par 2 Kad čuše da im govori hebrejskim jezikom, još većma utihnuše.  On nastavi: 
\par 3 "Ja sam Židov, rođen u Tarzu cilicijskom, ali  odrastao u ovom gradu, do nogu Gamalielovih odgojen točno po  otačkom Zakonu; bijah revnitelj Božji kao što ste svi vi još  danas. 
\par 4 Ovaj sam Put na smrt progonio, u okove bacao i predavao  u tamnice muževe i žene, 
\par 5 kako mi to može posvjedočiti i veliki  svećenik i sve starješinstvo. Od njih sam i pisma dobio za braću  u Damasku pa se zaputio da i one ondje okovane dovedem u Jeruzalem  da se kazne." 
\par 6 "Dok sam tako putovao i približavao se Damasku, s neba  me oko podneva iznenada obasja svjetlost velika. 
\par 7 Sruših se  na tlo i začuh glas što mi govoraše: 'Savle, Savle, zašto me  progoniš?' 
\par 8 Ja odgovorih: 'Tko si, Gospodine?' Reče mi: 'Ja  sam Isus Nazarećanin koga ti progoniš.' 
\par 9 Oni koji bijahu sa  mnom svjetlost doduše primijetiše, ali ne čuše glasa Onoga koji  mi govoraše. 
\par 10 Rekoh nato: 'Što mi je činiti, Gospodine?' Gospodin  će mi: 'Ustani, pođi u Damask i ondje će ti se reći što ti je  određeno učiniti.' 
\par 11 Kako od sjaja one svjetlosti obnevidjeh, pratioci me povedoše za ruku te stigoh u Damask." 
\par 12 "Neki Ananija, čovjek po Zakonu pobožan i na dobru glasu  u Židova ondje nastanjenih - 
\par 13 dođe k meni, pristupi mi i reče:  'Savle, brate, progledaj!' I ja se u taj čas zagledah u nj. 
\par 14 A  on će: 'Bog otaca naših predodredi te da upoznaš volju njegovu, da vidiš Pravednika i čuješ glas iz usta njegovih 
\par 15 jer bit  ćeš mu pred svim ljudima svjedokom onoga što si vidio i čuo. 
\par 16 I što sad oklijevaš? Ustani, krsti se i operi grijehe svoje, prizivljući Ime njegovo!'" 
\par 17 "Pošto se vratih u Jeruzalem, dok sam se jednom molio  u Hramu, padoh u zanos 
\par 18 i vidjeh Gospodina gdje mi govori:  'Pohiti, žurno izađi iz Jeruzalema jer neće primiti tvoga svjedočanstva  o meni.' 
\par 19 Ja rekoh: 'Gospodine, oni znaju da sam ja u tamnice  bacao i bičevao po sinagogama one koji vjeruju u te. 
\par 20 I dok  se prolijevala krv Stjepana, svjedoka tvoga, i ja sam ondje stajao  i odobravao te čuvao haljine onih koji ga ubijahu.' 
\par 21 Nato  mi reče: 'Pođi jer ću te poslati daleko k poganima!' 
\par 22 Slušali su ga sve do te riječi, a tada podigoše glas:  "Ukloni takva sa zemlje! Nije pravo da živi!" 
\par 23 Kako oni stadoše  bučiti, odbacivati haljine i vitlati prašinu u zrak, 
\par 24 zapovjedi  tisućnik da Pavla uvedu u vojarnu pa odredi da ga bičevima ispitaju  kako bi doznao zašto tako viču protiv njega. 
\par 25 Kad ga remenjem  rastegoše, reče on nazočnom satniku: "Rimskoga građanina, i još  neosuđena, smijete bičevati?" 
\par 26 Kad je to čuo satnik, priđe tisućniku i dojavi mu: "Što  si to nakanio? Ovaj je čovjek Rimljanin!" 
\par 27 Tisućnik tada priđe Pavlu pa mu reče: "Reci mi, jesi  li Rimljanin!" On odvrati: "Da." 
\par 28 Tisućnik dometnu: "Ja stekoh  to građanstvo za skupe novce." Pavao nato reče: "Ja sam se pak  s njim i rodio." 
\par 29 Brže stoga odstupe od njega oni koji su ga imali ispitivati.  I tisućnik se preplaši kad sazna da je Pavao Rimljanin, a on  ga bijaše okovao. 
\par 30 Sutradan pak kad je htio točno saznati za što ga Židovi  optužuju, odriješi ga pa zapovjedi da se sastanu veliki svećenici  i sve Vijeće te privede Pavla i postavi ga pred njih. 


\chapter{23}

\par 1 Pavao uprije pogled u Vijeće i reče: "Braćo, ja sam posve  mirne savjesti živio pred Bogom sve do dana današnjega." 
\par 2 Nato  veliki svećenik Ananija naredi onima što stajahu uza nj da ga  udare po ustima. 
\par 3 Onda mu Pavao reče: "Udarit će Bog tebe,  zide obijeljeni! Ti li sjediš da me po Zakonu sudiš, a protuzakonito  zapovijedaš da me biju?" 
\par 4 Oni što su ondje stajali rekoše nato:  "Zar velikog svećenika Božjega da pogrđuješ?" 
\par 5 Pavao odvrati:  "Nisam znao, braćo, da je veliki svećenik. Ta pisano je: Glavara  naroda svoga ne proklinji." 
\par 6 Pavao je znao da su oni dijelom saduceji, a dijelom farizeji  pa povika u Vijeću: "Braćo, ja sam farizej, sin farizeja. Sudi  mi se zbog nade, uskrsnuća mrtvih." 
\par 7 Tek što je on to rekao, nasta razmirica između farizeja  i saduceja i mnoštvo se razdijeli. 
\par 8 Jer saduceji vele da nema  uskrsnuća, ni anđela, ni duha, a farizeji sve to priznaju. 
\par 9 Nasta  velika graja te ustadoše neki pismoznanci farizejske stranke  i zaoštre boj govoreći: "Ništa zlo ne nalazimo na tom čovjeku!  A što ako mu je duh govorio, ili anđeo?" 
\par 10 Kad razmirica posta  još većom, poboja se tisućnik da Pavla ne rastrgaju pa zapovjedi  da vojska siđe, otme ga ispred njih i povede u vojarnu. 
\par 11 Iduće noći pristupi mu Gospodin i reče: "Hrabro samo!  Jer kao što si za me svjedočio u Jeruzalemu tako treba da i u  Rimu posvjedočiš!" 
\par 12 Kad osvanu dan, skovaše Židovi urotu i zakleše se da  neće ni jesti ni piti dok ne ubiju Pavla. 
\par 13 Bilo je više od  četrdeset onih koji su skovali tu zavjeru. 
\par 14 Oni odu k velikim  svećenicima i starješinama pa reknu: "Zakletvom se zaklesmo ništa  ne okusiti dok ne ubijemo Pavla. 
\par 15 Stoga vi sada zajedno s  Vijećem predočite tisućniku neka vam ga dovede kao da kanite  točnije razaznati njegov slučaj. A mi smo spremni pogubiti ga  prije negoli se i približi." 
\par 16 Ali sin Pavlove sestre doču za zavjeru, približi se i  uđe u vojarnu dojaviti Pavlu. 
\par 17 Pavao pak pozove jednog satnika  i reče mu: "Ovog mladića odvedi k tisućniku: ima mu nešto dojaviti." 
\par 18 On ga uze, odvede k tisućniku i reče mu: "Uznik me Pavao  pozva i zaiska da ovog mladića privedem k tebi; ima ti nešto  reći." 
\par 19 Tisućnik ga prihvati za ruku, povede nasamo pa ga  upita: "Što mi imaš dojaviti?" 
\par 20 "Židovi su se, reče on, dogovorili  da te zamole da im sutra Pavla dovedeš u Vijeće kao da se kane  točnije raspitati o njemu. 
\par 21 Ne vjeruj im! U zasjedi ga čeka  više od četrdeset onih koji se zakleše da neće jesti ni piti  dok ga ne smaknu. Već su spremni, samo čekaju tvoju privolu." 
\par 22 Tisućnik onda otpusti mladića i zapovjedi mu: "Nikomu ne  kazuj da si mi to dojavio." 
\par 23 Zatim dozva dva satnika i reče im: "Pripravite dvjesta  vojnika, sedamdeset konjanika i dvjesta strijelaca da nakon treće  noćne ure pođu u Cezareju. 
\par 24 Neka se pripravi živina na koju  će se posaditi Pavao te živ i zdrav dovesti k upravitelju Feliksu." 
\par 25 Napisa i pismo ovoga sadržaja: 
\par 26 "Klaudije Lizija vrlom  upravitelju Feliksu - pozdrav! 
\par 27 Ovoga čovjeka Židovi uhvatiše  i tek što ga ne smakoše kadli s vojskom pritrčah i istrgoh im  ga kada doznah da je Rimljanin. 
\par 28 Htjedoh saznati za što ga  okrivljuju pa ga dovedoh u njihovo Vijeće. 
\par 29 Utvrdih da ga  okrivljuju za nešto prijeporno u njihovu zakonu i da nema nikakve  krivnje kojom bi zaslužio smrt ili okove. 
\par 30 Kad mi pak dojaviše  da su protiv njega skovali zavjeru, poslah ga k tebi, a tužitelje  uputih neka se tebi obrate protiv njega." 
\par 31 Vojnici dakle, po primljenoj naredbi uzeše Pavla i odvedoše  ga noću u Antipatridu. 
\par 32 Sutradan ostave konjanike da s njime  pođu dalje, a oni se vratiše u vojarnu. 
\par 33 Kad konjanici stigoše  u Cezareju, uručiše upravitelju pismo i privedoše mu Pavla. 
\par 34 Pošto  upravitelj pročita pismo, zapita iz koje je pokrajine. Kad sazna  da je iz Cilicije: 
\par 35 "Saslušat ću te, reče, kad pristignu i  tužitelji tvoji." Onda zapovjedi čuvati ga u dvoru Herodovu. 


\chapter{24}

\par 1 Nakon pet dana siđe veliki svećenik Ananija s nekim starješinama  i odvjetnikom, nekim Tertulom te izniješe upravitelju tužbu protiv  Pavla. 
\par 2 Pošto dozvaše Pavla, poče ga Tertul optuživati: "Veliki  mir što ga po tebi, vrli Felikse, uživamo i boljitak što tvojom  providnošću narodu ovomu nastaje, 
\par 3 u svemu i posvuda primamo  sa svom zahvalnošću. 
\par 4 Ali, da ti dulje ne dodijavam, molim  te da nas u svojoj blagonaklonosti ukratko poslušaš. 
\par 5 Utvrdismo  da je ovaj čovjek kuga, da pokreće bune među svim Židovima po  svijetu, da je kolovođa nazaretske sljedbe, 
\par 6 da je čak i Hram  pokušao oskvrnuti pa ga uhitismo. 
\par 7 # 
\par 8 Od njega, ako ga  o svemu tomu ispitaš, možeš saznati za što ga mi optužujemo." 
\par 9 Podržaše ga i Židovi tvrdeći da je tako. 
\par 10 Nato Pavao odvrati pošto mu upravitelj kimnu da govori: "Kako znam da si već mnogo godina sudac narodu ovomu, mirne se  duše branim. 
\par 11 Ta možeš se osvjedočiti da nema više od dvanaest  dana otkad uzađoh u Jeruzalem da se poklonim. 
\par 12 A nisu me našli  ni u Hramu da s kim raspravljam ili bunu podižem, ni u sinagogama, ni po gradu. 
\par 13 I ne mogu ti dokazati ono za što me sada optužuju." 
\par 14 "Jamčim ti, naprotiv, ovo: Putom koji nazivaju sljedbom  služim otačkom Bogu vjerujući u sve što je u Zakonu i u Prorocima  napisano, 
\par 15 uzdajući se u Boga da će uskrsnuti pravednici i  nepravednici, što oni i sami očekuju. 
\par 16 Zato se i ja trudim  uvijek imati savjest besprijekornu pred Bogom i pred ljudima." 
\par 17 "Nakon više godina dođoh da donesem milostinju za svoj  narod i prinose; 
\par 18 dok sam ih prinosio, nađoše me posvećena  u Hramu, a ne sa svjetinom ni u metežu. 
\par 19 Ali neki Židovi iz  Azije - da, trebalo bi da se oni pojave pred tobom i optuže me  ako što imaju protiv mene. 
\par 20 Ili neka ovi sami kažu: koji su  zločin na meni našli kad sam stajao pred Vijećem, 
\par 21 osim možda  one jedne riječi koju doviknuh među njima stojeći: Zbog uskrsnuća  mrtvih sudi mi se danas pred vama!" 
\par 22 Nato Feliks, koji je točno znao sve o ovom Putu, odgodi  njihovu parnicu rekavši: "Kada dođe tisućnik Lizija, riješit  ću vaš spor." 
\par 23 Satniku pak naredi da se Pavao čuva, ali da  uživa olakšice i da se nikomu od njegovih ne brani posluživati  ga. 
\par 24 Nakon nekoliko dana stigne i Feliks sa svojom ženom Druzilom  koja bijaše Židovka; posla po Pavla i posluša ga o vjeri u Isusa  Krista. 
\par 25 Kad Pavao stade raspravljati o pravednosti, uzdržljivosti  i budućem Sudu, Feliks uplašen reče: "Zasad idi, a kad nađem  vremena, pozvat ću te." 
\par 26 Ujedno se nadao da će mu Pavao dati  novaca. Zato ga je češće pozivao i s njim razgovarao. 
\par 27 Nakon dvije godine dobi Feliks za nasljednika Porcija  Festa. Hoteći ugoditi Židovima, ostavi Feliks Pavla u okovima. 


\chapter{25}

\par 1 Fest dakle tri dana nakon dolaska u provinciju uziđe iz Cezareje  u Jeruzalem. 
\par 2 Veliki mu svećenici i prvaci židovski izniješe  tužbu protiv Pavla te ga zaklinjahu 
\par 3 ištući milost protiv Pavla:  da ga pošalje u Jeruzalem. Jer spremali su zasjedu da ga putom  smaknu. 
\par 4 Ali Fest odvrati kako Pavao treba da ostane zatvoren  u Cezareji, a i on da će uskoro onamo. 
\par 5 "Ovlašteni dakle među  vama, reče, neka sa mnom siđu pa ako na tom čovjeku ima krivnje, neka ga tuže." 
\par 6 Pošto se u njih zadrži najviše osam ili deset dana, siđe  u Cezareju. Sutradan sjede na sudačku stolicu i zapovjedi da  se dovede Pavao. 
\par 7 Kad se on pojavi, okružiše ga Židovi koji  su sišli iz Jeruzalema i izniješe protiv njega mnoge i teške  optužbe kojih ne mogahu dokazati. 
\par 8 Pavao se branio: "Ničim  se nisam ogriješio ni o židovski Zakon, ni o Hram, ni o cara." 
\par 9 Nato Fest hoteći ugoditi Židovima, odvrati Pavlu: "Hoćeš li  u Jeruzalem da ti se ondje za to sudi preda mnom?" 
\par 10 A Pavao  će: "Stojim pred sudom carevim, gdje treba da mi se sudi. Židovima  ništa ne skrivih, kao što i ti veoma dobro znaš. 
\par 11 Ako sam  pak doista što skrivio i učinio štogod što zavređuje smrt, ne  izmičem smrti; ako li pak ne stoji ono za što me ovi tuže, nitko  me ne može njima izručiti. Na cara se prizivljem!" 
\par 12 Tada se  Fest posavjetova s vijećem pa odgovori: "Na cara si se prizvao, pred cara ćeš ići!" 
\par 13 Nekoliko dana poslije dođu kralj Agripa i Berenika u  Cezareju da pozdrave Festa. 
\par 14 Kako se ondje zadržaše nekoliko  dana, izloži Fest kralju to o Pavlu: "Ima neki čovjek, reče,  što ga je Feliks ostavio uznikom. 
\par 15 Kad bijah u Jeruzalemu, izniješe veliki svećenici i starješine protiv njega tužbu i  zatražiše osudu. 
\par 16 Odgovorih im da u Rimljana nije običaj izručivati  kojega čovjeka prije negoli se, optužen, suoči s tužiteljima  i dobije prigodu da se brani od optužbe. 
\par 17 Pošto zajedno dođosmo  ovamo, bez ikakva odgađanja sjedoh ja sutradan na sudačku stolicu  i zapovjedih dovesti toga čovjeka. 
\par 18 Tužitelji ga okružiše, ali ne izniješe tužbe ni za jedno od zlodjela koja sam ja naslućivao, 
\par 19 nego su protiv njega imali nešto prijeporno o svojoj vjeri  i o nekom Isusu koji je umro, a Pavao tvrdi da je živ. 
\par 20 Ne  snalazeći se u takvoj raspravi, upitah bi li htio u Jeruzalem  da mu se ondje za to sudi. 
\par 21 Budući da se Pavao prizivom podložio  presudi njegova Veličanstva, zapovjedih da ga čuvaju dok ga ne  pošaljem caru." 
\par 22 Na to će Agripa Festu: "Htio bih i ja čuti  toga čovjeka." "Sutra ćeš ga, reče, čuti." 
\par 23 Sutradan dakle dođu Agripa i Berenika s velikim sjajem  te uđu u dvoranu zajedno s tisućnicima i najuglednijim gradskim  muževima. Kad na zapovijed Festovu dovedu Pavla, 
\par 24 reče Fest:  "Agripa, kralju, i vi svi ovdje s nama nazočni, gledajte ovoga  čovjeka! Zbog njega me sav narod židovski salijetao i u Jeruzalemu  i ovdje vičući da on ne smije više živjeti. 
\par 25 Ali ja nađoh  da nije učinio ništa čime bi zaslužio smrt pa kad se on sam prizvao  na njegovo Veličanstvo, odlučih poslati mu ga. 
\par 26 Ja nemam ništa  pouzdano o njemu napisati Gospodaru. Zato ga izvedoh pred vas, ponajpače preda te, kralju Agripa, da bih nakon ove istrage  imao što napisati. 
\par 27 Čini mi se doista besmislenim poslati  uznika, a ne naznačiti optužbu protiv njega." 


\chapter{26}

\par 1 Nato Agripa reče Pavlu: "Dopušta ti se o sebi govoriti." Pavao  ispruži ruku i stade se braniti: 
\par 2 "Smatram se sretnim što se u svemu za što me Židovi optužuju  mogu, evo, danas braniti pred tobom, kralju Agripa, 
\par 3 jer ti  najbolje poznaješ židovske običaje i zadjevice. Zato me, molim, velikodušno poslušaj." 
\par 4 "Dakle, život moj od najranije mladosti proveden u narodu  mojem, u Jeruzalemu, znaju svi Židovi. 
\par 5 Poznaju me odavna te  mogu, ako samo hoće, svjedočiti da sam po najstrožoj sljedbi  naše vjere živio kao farizej. 
\par 6 I sada stojim pred sudom zbog  nade u obećanje koje Bog dade ocima našim 
\par 7 i kojemu se dovinuti  nada dvanaest plemena naših, svesrdno noću i danju služeći Bogu.  Za tu me nadu, kralju, tuže Židovi. 
\par 8 Zašto nevjerojatnim smatrate  da Bog mrtve uskrisuje?" 
\par 9 "Pa i ja sam nekoć smatrao da mi se svim silama boriti  protiv imena Isusa Nazarećanina. 
\par 10 To sam i činio u Jeruzalemu:  mnoge sam svete, pošto od velikih svećenika dobih punomoć, u  tamnice zatvorio, dao svoj glas kad su ih ubijali 
\par 11 i po svim  ih sinagogama često mučenjem prisiljavao psovati i, prekomjerno  bijesan na njih, progonio sam ih čak i u tuđim gradovima." 
\par 12 "Radi toga pođoh u Damask s punomoći i ovlaštenjem velikih  svećenika 
\par 13 kadli u pol bijela dana na putu vidjeh, kralju, kako s neba svjetlost od sunca sjajnija obasja mene i moje suputnike. 
\par 14 Pošto popadasmo na zemlju, začuh glas što mi govoraše hebrejskim  jezikom: 'Savle, Savle, zašto me progoniš? Teško ti se protiv  ostana praćakati.' 
\par 15 Ja odvratih: 'Tko si, Gospodine?' Gospodin  će mi: 'Ja sam Isus koga ti progoniš! 
\par 16 Nego ustani, na  noge se jer zato ti se ukazah da te postavim za poslužitelja  i svjedoka onoga što si vidio i što ću ti pokazati. 
\par 17 Izbavit  ću te od naroda i od pogana kojima te šaljem 
\par 18 da im  otvoriš oči pa se obrate od tame k svjetlosti, od  vlasti Sotonine k Bogu te po vjeri u mene prime oproštenje grijeha  i baštinu među posvećenima.'" 
\par 19 "Otada, kralju Agripa, ne bijah neposlušan nebeskom viđenju. 
\par 20 Nego najprije onima u Damasku pa onda i u Jeruzalemu, svoj  zemlji židovskoj i poganima navješćivah da se pokaju i obrate  k Bogu i čine djela dostojna obraćenja. 
\par 21 Zbog toga me Židovi  uhvatiše u Hramu i pokušaše ubiti. 
\par 22 Ali s pomoću Božjom sve  do dana današnjega svjedočim, evo, malu i veliku, ne govoreći  ništa osim onoga što Proroci govorahu i Mojsije da se ima zbiti: 
\par 23 da će Krist trpjeti i da će on, prvouskrsli od mrtvih, svjetlost  navješćivati narodu i poganima." 
\par 24 Dok se on tako branio, Fest će mu u sav glas: "Mahnitaš, Pavle! Veliko ti znanje mozgom zavrnulo." 
\par 25 "Ne mahnitam,  vrli Feste, odvrati Pavao, nego riječi istine i razbora kazujem. 
\par 26 Ta znade za to kralj komu s pouzdanjem govorim. Ništa mu  od toga, uvjeren sam, nije nepoznato; jer nije se to dogodilo  u kakvu zakutku. 
\par 27 Vjeruješ li, kralju Agripa, Prorocima? Znam  da vjeruješ!" 
\par 28 Agripa će Pavlu: "Zamalo pa me uvjeri te kršćaninom  postah!" 
\par 29 Pavao pak: "Dao Bog te i za malo i za mnogo, ne  samo ti nego i svi koji me danas slušaju postali ovakvima kakav  sam ja, osim ovih okova!" 
\par 30 Nato usta kralj, upravitelj, Berenika i oni koji su s  njima zasjedali. 
\par 31 Udaljujući se govorili su među sobom: "Ovaj  čovjek ne čini ništa čime bi zaslužio smrt ili okove." 
\par 32 Agripa  pak reče Festu: "Ovaj bi čovjek mogao biti pušten da se nije  prizvao na cara." 


\chapter{27}

\par 1 Kad je odlučeno da odjedrimo u Italiju, predadoše i Pavla  i neke druge uznike satniku carske čete, imenom Juliju. 
\par 2 Popesmo  se na neku adramitsku lađu koja je imala ploviti u azijska mjesta  pa otplovismo. S nama je bio Aristarh Makedonac, Solunjanin. 
\par 3 Sutradan doplovismo u Sidon. Julije, koji je s Pavlom čovječno  postupao, dopusti mu poći k prijateljima da se pobrinu za nj. 
\par 4 Odande smo otplovili, jedrili uz Cipar - jer su nam vjetrovi  bili protivni - 
\par 5 pa preplovili more duž Cilicije i Pamfilije  i stigli u Miru licijsku. 
\par 6 Ondje satnik nađe neku aleksandrijsku  lađu za Italiju i ukrca nas na nju. 
\par 7 Više smo dana plovili sporo i jedva doprli do Knida. Kako  nam vjetar ne dade pristati, doplovismo pod Kretu kod Salmone 
\par 8 pa jedva jedvice ploveći uza nju, stigosmo na neko mjesto  zvano Dobra pristaništa, blizu kojega je grad Laseja. 
\par 9 Kad je nakon duljeg vremena plovidba već postala pogibeljna  jer je Post već bio izminuo, opominjaše Pavao: 
\par 10 "Ljudi, govorio  im je, vidim da će plovidba biti nezgodna i na veliku štetu ne  samo za tovar i lađu nego i za naše živote." 
\par 11 Ali je satnik  više vjerovao kormilaru i brodovlasniku negoli Pavlovim riječima. 
\par 12 A kako luka nije bila prikladna za zimovanje, većina je predlagala  da odande otplove ne bi li kako doprli do kretske luke Feniksa, što gleda prema jugozapadu i sjeverozapadu, pa ondje prezimili. 
\par 13 Uto duhne blagi južnjak i oni, misleći da bi mogli ostvariti  naum, digoše sidro i zaploviše tik uz Kretu. 
\par 14 Ali nedugo zatim  razbjesni se žestok vjetar zvan sjeveroistočnjak. 
\par 15 Zahvati  lađu te mu nije mogla odoljeti pa se prepustismo da nas nosi. 
\par 16 Prolazeći ispod nekog otočića zvanog Kauda, jedva uspjesmo  dohvatiti čamac. 
\par 17 Podigoše ga pa upotrijebiše snast da potpašu  lađu. Bojeći se pak da se ne nasuču u Sirti, spustiše prvenjaču.  Tako ih je nosilo. 
\par 18 Budući da nas je oluja silovito udarala, sutradan se riješiše tovara, 
\par 19 a treći dan svojim rukama izbaciše  brodsku opremu. 
\par 20 Kako se pak više dana nije pomaljalo ni sunce  ni zvijezde, a oluja bjesnjela nemalena, bila je već propala  svaka nada da ćemo se spasiti. 
\par 21 Ni jelo se već dugo nije. Onda usta Pavao posred njih  i reče: "Trebalo je, ljudi, poslušati me, ne se otiskivati od  Krete i izbjeći ovu nepogodu i štetu. 
\par 22 Sada vas pak opominjem:  razvedrite se jer ni živa duša između vas neće stradati, nego  samo lađa. 
\par 23 Noćas mi se ukaza anđeo Boga čiji sam i komu služim 
\par 24 te reče: 'Ne boj se, Pavle! Pred cara ti je stati i evo Bog  ti daruje sve koji plove s tobom.' 
\par 25 Zato razvedrite se, ljudi!  Vjerujem Bogu: bit će kako mi je rečeno. 
\par 26 Ali treba da se  nasučemo na neki otok." 
\par 27 Bijaše već četrnaesta noć što smo bili tamo-amo gonjani  po Jadranu kad oko ponoći naslutiše mornari da im se primiče  neka zemlja. 
\par 28 Bacivši olovnicu, nađoše dvadeset hvati dubine;  malo poslije baciše je opet i nađoše ih petnaest. 
\par 29 Kako se bojahu da ne naletimo na grebene, baciše s krme  četiri sidra iščekujući da se razdani. 
\par 30 Kad su mornari bili  naumili uteći iz lađe i počeli spuštati čamac u more pod izlikom  da s pramca kane spustiti sidra, 
\par 31 reče Pavao satniku i vojnicima:  "Ako ovi ne ostanu na lađi, vi se spasiti ne možete!" 
\par 32 Nato  vojnici presjekoše užad čamca i pustiše da padne. 
\par 33 Do pred svanuće nutkao je Pavao sve da uzmu hrane: "Četrnaesti  je danas dan, reče, što bez jela čekate, ništa ne okusivši. 
\par 34 Stoga  vas molim: založite nešto jer to je za vaš spas. Ta nikome od  vas ni vlas s glave neće propasti." 
\par 35 Rekavši to, uze kruh, pred svima zahvali Bogu, razlomi i stade jesti. 
\par 36 Svi se razvedre  te i oni uzmu hrane. 
\par 37 A svih nas je u lađi bilo dvjesta sedamdeset  i šest duša. 
\par 38 Jednom nasićeni, stanu rasterećivati lađu bacajući  žito u more. 
\par 39 Kad osvanu, mornari ne prepoznaše zemlje; razabraše  neki zaljev ravne obale pa odluče, bude li moguće, u nj zavesti  lađu. 
\par 40 Odriješe sidra i ostave ih u moru. Istodobno popuste  i spone kormila, razapnu prvenjaču prema vjetru pa udare k obali. 
\par 41 Ali naletješe na plićak i nasukaše brod: pramac, nasađen, osta nepomičan, a krmu razdiraše žestina valova. 
\par 42 Tada vojnici naumiše poubijati sužnje da ne bi koji isplivao  i pobjegao, 
\par 43 ali im satnik, hoteći spasiti Pavla, omete naum:  zapovjedi da oni koji znaju plivati najprvi poskaču i izađu na  kraj, 
\par 44 a ostali će, tko na daskama, tko na olupinama lađe.  Tako svi živi i zdravi prispješe na kopno. 



\chapter{28}

\par 1 Jednom spašeni, doznasmo da se otok zove Malta. 
\par 2 Urođenici  nam iskazivahu nesvakidašnje čovjekoljublje. Zapališe krijes  i okupiše nas oko njega jer je počela kiša i bilo zima. 
\par 3 Pavao  nakupi naramak granja i baci na krijes kadli zbog vrućine izađe  zmija i pripije mu se za ruku. 
\par 4 Kad su urođenici vidjeli gdje  mu životinja visi o ruci, govorili su među sobom: "Ovaj je čovjek  zacijelo ubojica: umakao je moru i Pravda mu ne da živjeti." 
\par 5 Ali on otrese životinju u vatru i ne bi mu ništa; 
\par 6 a oni  očekivahu da će oteći i umah se srušiti mrtav. Pošto su dugo  čekali i vidjeli da mu se ništa neobično nije dogodilo, promijeniše  mišljenje te stadoše govoriti da je bog. 
\par 7 U okolici onoga mjesta bilo je imanje prvaka otoka, imenom  Publija. On nas je primio i tri dana uljudno gostio. 
\par 8 A Publijeva  je oca uhvatila ognjica i srdobolja pa je ležao. Pavao uđe k  njemu, pomoli se, stavi na nj ruke i izliječi ga. 
\par 9 Nakon toga  su dolazili i drugi koji na otoku bijahu bolesni te ozdravljali. 
\par 10 Oni nas mnogim počastima počastiše i na odlasku nam priskrbiše  što je potrebno. 
\par 11 Nakon tri mjeseca otplovismo aleksandrijskom lađom koja  je prezimila na otoku i imala za znak Dioskure. 
\par 12 Doplovismo  u Sirakuzu i ostadosmo ondje tri dana. 
\par 13 Odande ploveći uz  obalu, stigosmo u Regij. Sutradan okrenu južnjak te za dva dana  stigosmo u Puteole. 
\par 14 Ondje nađosmo braću koja nas zamoliše  da ostanemo u njih sedam dana. Tako stigosmo u Rim. 
\par 15 Kada su tamošnja braća čula za nas, iziđoše nam u susret  do Apijeva trga i Triju gostionica. Kad ih Pavao ugleda, zahvali  Bogu i ohrabri se. 
\par 16 A kad uđosmo u Rim, Pavlu su dopustili  stanovati zasebno, zajedno s vojnikom koji ga je čuvao. 
\par 17 Nakon tri dana sazva on židovske prvake. Kad se sabraše, reče im: "Ja, braćo, ne učinih ništa protiv naroda ni običaja  otačkih, a ipak me okovana u Jeruzalemu predadoše u ruke Rimljana. 
\par 18 Oni me nakon istrage htjedoše pustiti jer nije na meni bilo  ništa čime bih bio zaslužio smrt. 
\par 19 Kako se Židovi tome opriješe, bio sam prisiljen prizvati se na cara; ne dakle stoga što bih  imao bilo za što tužiti svoj narod. 
\par 20 S toga dakle razloga  zamolih vidjeti vas i obratiti vam se jer zbog nade Izraelove  nosim ove verige." 
\par 21 Oni mu odvrate: "Mi o tebi nismo primili nikakva pisma  iz Judeje niti nam je tko od pristigle braće o tebi što zlo javio  ili rekao. 
\par 22 Nego htjeli bismo od tebe čuti što misliš jer  o toj sljedbi znamo samo da joj se posvuda proturječi." 
\par 23 Nato urekoše dan pa dođoše mnogi k njemu u stan. Izlagao  im je i svjedočio o kraljevstvu Božjemu te ih od jutra do večeri  iz Mojsijeva Zakona i Proroka uvjeravao o Isusu. 
\par 24 I jedne  uvjeriše njegove riječi, a drugi nisu vjerovali. 
\par 25 Nesložni  tako među sobom, stadoše se razilaziti kadli im Pavao reče još  jednu riječ: "Lijepo Duh Sveti po Izaiji proroku reče ocima vašim: 
\par 26 Idi k tomu narodu i reci mu: Slušat ćete, slušati - i nećete razumjeti; gledat ćete, gledati - i nećete vidjeti! 
\par 27 Jer usalilo se srce naroda ovoga: uši začepiše, oči zatvoriše da očima ne vide, ušima ne čuju, srcem ne razumiju te se ne obrate pa ih izliječim. 
\par 28 Neka vam je dakle svima znano: poganima je poslano ovo  spasenje Božje; oni će poslušati!" 
\par 29 # 
\par 30 Pavao osta pune dvije godine u svom unajmljenom stanu  gdje je primao sve koji su dolazili k njemu, 
\par 31 propovijedao  kraljevstvo Božje i naučavao o Gospodinu Isusu Kristu sa svom  slobodom, nesmetano. 




\end{document}