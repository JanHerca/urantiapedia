\begin{document}

\title{Izreke}


\chapter{1}

\par 1 Mudre izreke Salomona, sina Davidova, kralja izraelskog: 
\par 2 da se spozna mudrost i pouka, da se shvate razumne riječi; 
\par 3 da se primi umna pouka, pravda i pravica i nepristranost; 
\par 4 da se dade pamet neiskusnima, mladiću znanje i umijeće; 
\par 5 kad mudar čuje, da umnoži znanje, a razuman steče mudrije misli; 
\par 6 da razumije izreke i prispodobe, riječi mudraca i njihove zagonetke. 
\par 7 Strah je Gospodnji početak spoznaje, ali ludi preziru mudrost i pouku. 
\par 8 Poslušaj, sine moj, pouku oca svoga i ne odbacuj naputka svoje majke! 
\par 9 Jer će ti biti ljupki vijenac na glavi i ogrlica oko tvoga vrata. 
\par 10 Sine moj, ako te grešnici mame, ne pristaj; 
\par 11 ako bi rekli: "Hodi s nama, da vrebamo krv, čekamo u zasjedi nevina ni za što; 
\par 12 da ih progutamo žive kao carstvo smrti i cijele kao one koji silaze u grob; 
\par 13 naplijenit ćemo svakojaka blaga, napuniti svoje kuće plijenom; 
\par 14 bacat ćeš s nama svoj ždrijeb, svi ćemo zajedno imati jednu kesu." 
\par 15 Sine moj, ne idi s njima na put, makni nogu od njihove staze. 
\par 16 Jer na zlo trče svojim nogama i hite prolijevati krv. 
\par 17 Jer uzalud je razapinjati mrežu pred očima svima pticama. 
\par 18 A oni vrebaju vlastitu krv, postavljaju zasjedu svojemu životu. 
\par 19 Takva je sudba svih lakomih na ružan dobitak: on ih života stane. 
\par 20 Mudrost glasno uzvikuje na ulici, na trgovima diže svoj glas; 
\par 21 propovijeda po bučnim uglovima, na otvorenim gradskim vratima govori svoje riječi: 
\par 22 "Dokle ćete, vi glupi, ljubiti glupost i dokle će podsmjevačima biti milo podsmijevanje, i dokle će bezumnici mrziti znanje? 
\par 23 Poslušajte moju opomenu! Gle, svoj duh pred vas izlijevam, hoću vas poučiti svojim riječima. 
\par 24 Koliko sam vas zvala, a vi ste odbijali; pružala sam ruku, ali je nitko ne opazi. 
\par 25 Nego ste odbacili svaki moj savjet i niste poslušali moje opomene; 
\par 26 zato ću se i ja smijati vašoj propasti, rugat ću se kad vas obuzme tjeskoba: 
\par 27 kad navali na vas strah kao nevrijeme i zgrabi vas propast kao vihor, kad navali na vas nevolja i muka. 
\par 28 Tada će me zvati, ali se ja neću odazvati; tražit će me, ali me neće naći. 
\par 29 Jer su mrzili spoznaju i nisu izabrali Gospodnjeg straha 
\par 30 niti su poslušali moj savjet, nego su prezreli svaku moju opomenu. 
\par 31 Zato će jesti plod svojeg vladanja i nasititi se vlastitih savjeta. 
\par 32 Jer glupe će ubiti njihovo odbijanje, a nemar će upropastiti bezumne. 
\par 33 A tko sluša mene, bezbrižan ostaje i spokojno živi bez straha od zla." 


\chapter{2}

\par 1 Sine moj, ako primiš moje riječi i pohraniš u sebi moje zapovijedi, 
\par 2 i uhom svojim osluhneš mudrost i obratiš svoje srce razboru; 
\par 3 jest, ako prizoveš razum i zavapiš za razborom; 
\par 4 ako ga potražiš kao srebro i tragaš za njim kao za skrivenim blagom - 
\par 5 tada ćeš shvatiti strah Gospodnji i naći ćeš Božje znanje. 
\par 6 Jer Jahve daje mudrost, iz njegovih usta dolazi znanje i razboritost. 
\par 7 On pravednicima pruža svoju pomoć, štit je onih koji hode u bezazlenosti. 
\par 8 Jer on štiti staze pravde i čuva pute svojih pobožnika. 
\par 9 Tada ćeš shvatiti pravdu, pravicu, pravednost i sve staze dobra, 
\par 10 jer će mudrost ući u tvoje srce i spoznaja će obradovati tvoju dušu. 
\par 11 Oprez će paziti na te i razboritost će te čuvati: 
\par 12 da te izbavi od zla puta, od varava čovjeka, 
\par 13 od onih koji ostavljaju staze poštenja te idu mračnim putovima; 
\par 14 koji se vesele čineći zlo i likuju u opačinama zloće; 
\par 15 kojih su staze krive i koji su opaki na svojim putovima; 
\par 16 da te izbavi od preljubnice i od tuđinke koja laska riječima; 
\par 17 koja ostavlja prijatelja svoje mladosti i zaboravlja zavjet svoga Boga 
\par 18 jer joj kuća tone u smrt i njezini putovi vode mrtvima. 
\par 19 Tko god zalazi k njoj ne vraća se nikad i ne nalazi više putove života. 
\par 20 Zato idi putem čestitih i drži se staza pravedničkih! 
\par 21 Jer samo će pravedni nastavati zemlju i bezazleni će ostati na njoj. 
\par 22 A opake će zbrisati sa zemlje i bogohulnike iščupati iz nje. 


\chapter{3}

\par 1 Sine moj, ne zaboravljaj moje pouke,  i tvoje srce neka čuva moje zapovijedi, 
\par 2 jer će ti produljiti dane i životne godine i podariti spokojstvo. 
\par 3 Neka te ne ostavljaju dobrota i vjernost, objesi ih sebi oko vrata, upiši ih na ploču srca svoga. 
\par 4 Tako ćeš steći ugled i uspjeti pred Božjim i ljudskim očima. 
\par 5 Uzdaj se u Jahvu svim srcem i ne oslanjaj se na vlastiti razbor. 
\par 6 Misli na nj na svim svojim putovima i on će ispraviti tvoje staze. 
\par 7 Ne umišljaj da si mudar: boj se Jahve i kloni se zla. 
\par 8 To će biti lijek tvome tijelu i okrepa tvojim kostima. 
\par 9 Časti Jahvu svojim blagom i prvinama svega svojeg prirasta. 
\par 10 I tvoje će žitnice biti prepune i tvoje će se kace prelijevati novim vinom. 
\par 11 Sine moj, ne odbacuj Jahvine opomene i nemoj da ti omrzne njegov ukor. 
\par 12 Jer koga Jahve ljubi onoga i kori, kao otac sina koga voli. 
\par 13 Blago čovjeku koji je stekao mudrost i čovjeku koji je zadobio razboritost. 
\par 14 Jer bolje je steći nju nego steći srebro, i veći je dobitak ona i od zlata. 
\par 15 Skupocjenija je od bisera, i što je god tvojih dragocjenosti, s njome se porediti ne mogu; 
\par 16 dug joj je život u desnoj ruci, a u lijevoj bogatstvo i čast. 
\par 17 Njezini su putovi putovi miline i sve su njene staze pune spokoja. 
\par 18 Životno je drvo onima koji se nje drže i sretan je onaj tko je zadrži. 
\par 19 Jahve je mudrošću utemeljio zemlju i umom utvrdio nebesa; 
\par 20 njegovim su se znanjem otvorili bezdani i oblaci osuli rosom. 
\par 21 Sine moj, ne gubi to iz očiju, sačuvaj razbor i oprez. 
\par 22 I bit će život tvojoj duši i ures vratu tvome. 
\par 23 Bez straha ćeš tada kročiti svojim putem i noga ti se neće spoticati. 
\par 24 Kad legneš, nećeš se plašiti, i kad zaspiš, slatko ćeš snivati. 
\par 25 Ne boj se nenadne strahote ni nagle propasti kad stigne bezbožnike. 
\par 26 Jer će ti Jahve biti uzdanje i čuvat će nogu tvoju od zamke. 
\par 27 Ne uskrati dobročinstva potrebitim kad god to možeš učiniti. 
\par 28 Ne reci svome bližnjemu: "Idi i dođi opet, sjutra ću ti dati", kad možeš već sada. 
\par 29 Ne kuj zla svome bližnjemu dok on bez straha kod tebe boravi. 
\par 30 Ne pravdaj se ni s kim bez razloga ako ti nije učinio nikakva zla. 
\par 31 Nemoj zavidjeti nasilniku niti slijediti njegove pute, 
\par 32 jer su Jahvi mrski pokvarenjaci, a prisan je s pravednima. 
\par 33 Jahvino je prokletstvo na domu bezbožnika, a blagoslov u stanu pravednika. 
\par 34 S podsmjevačima on se podsmijeva, a poniznima dariva milost. 
\par 35 Mudri će baštiniti čast, a bezumnici snositi sramotu. 


\chapter{4}

\par 1 Slušajte, djeco, pouku očevu i pazite kako biste spoznali mudrost, 
\par 2 jer dobar vam nauk dajem: ne prezrite moga naputka. 
\par 3 I ja sam bio sin u svoga oca i nježan jedinac u svoje matere; 
\par 4 i mene je on učio i govorio mi: "Zadrži moje riječi u svojem srcu, poštuj moje zapovijedi i živjet ćeš. 
\par 5 Steci mudrost, steci razbor, ne smeći ih s uma i ne odstupi od riječi mojih usta. 
\par 6 Ne ostavljaj je i čuvat će te; ljubi je i obranit će te. 
\par 7 Početak je mudrosti: steci sebi mudrost i svim svojim imanjem steci razboritost. 
\par 8 Veličaj je i uzvisit će te; donijet će ti čast kad je prigrliš. 
\par 9 Stavit će ti ljupki vijenac na glavu, i obdarit će te krasnom krunom." 
\par 10 Poslušaj, sine moj, primi moje riječi i umnožit će se godine tvojeg života. 
\par 11 Poučih te putu mudrosti, navratih te na prave staze; 
\par 12 neće ti se zapletati koraci kad staneš hoditi; potrčiš li, nećeš posrnuti. 
\par 13 Čvrsto se drži pouke, ne puštaj je, čuvaj je, jer ona ti je život. 
\par 14 Ne idi stazom opakih i ne stupaj putem zlikovaca. 
\par 15 Ostavi ga, ne hodi njime; kloni ga se i zaobiđi ga. 
\par 16 Jer oni ne spavaju ako ne učine zla, i san im ne dolazi ako koga ne obore. 
\par 17 Jer jedu kruh opačine i piju vino nasilja. 
\par 18 A pravednička je staza kao svjetlost svanuća, koja je sve jasnija do potpunog dana. 
\par 19 A put je opakih kao mrkli mrak: ne znaju o što će se spotaknuti. 
\par 20 Sine moj, pazi na moje riječi, prigni uho svoje mojim besjedama. 
\par 21 Ne gubi ih nikad iz očiju, pohrani ih usred srca svoga. 
\par 22 Jer su život onima koji ih nalaze i ozdravljenje svemu tijelu njihovu. 
\par 23 A svrh svega, čuvaj svoje srce, jer iz njega izvire život. 
\par 24 Drži daleko od sebe lažna usta i udalji od sebe usne prijevarne. 
\par 25 Nek' tvoje oči gledaju u lice i neka ti je pogled uvijek prav. 
\par 26 Pazi na stazu kojom kročiš i neka ti svi putovi budu pouzdani. 
\par 27 Ne skreći ni desno ni lijevo, drži svoj korak daleko oda zla. 


\chapter{5}

\par 1 Sine moj, čuj moju mudrost, prigni uho mojoj razboritosti 
\par 2 da sačuvaš oprez, da ti usne zadrže znanje. 
\par 3 Jer s usana žene preljubnice kaplje med i nepce joj je glađe od ulja, 
\par 4 ali je ona naposljetku gorka kao pelin, oštra kao dvosjekli mač. 
\par 5 Njene noge silaze k smrti, a koraci vode u Podzemlje. 
\par 6 Ona ne pazi na put života, ne mari što su joj staze kolebljive. 
\par 7 Zato me sada poslušaj, sine, i ne odstupaj od riječi mojih usta. 
\par 8 Neka je put tvoj daleko od nje i ne približuj se vratima njezine kuće, 
\par 9 da drugima ne bi dao svoju slavu i okrutnima svoje godine; 
\par 10 da se ne bi tuđinci nasitili tvoga dobra i da tvoja zaslužba ne ode u tuđu kuću; 
\par 11 da ne ridaš na koncu kad ti nestane tijela i puti 
\par 12 i da ne kažeš: "Oh, kako sam mrzio pouku i kako mi je srce preziralo ukor! 
\par 13 I ne slušah glasa svojih učitelja, niti priklonih uho onima što me poučavahu. 
\par 14 I umalo ne zapadoh u svako zlo, usred zbora i zajednice!" 
\par 15 Pij vodu iz svoje nakapnice i onu što teče iz tvoga studenca. 
\par 16 Moraju li se tvoji izvori razlijevati i tvoji potoci teći ulicama? 
\par 17 Nego neka oni budu samo tvoji, a ne i tuđinaca koji su uza te. 
\par 18 Neka je blagoslovljen izvor tvoj i raduj se sa ženom svoje mladosti: 
\par 19 neka ti je kao mila košuta i ljupka gazela, neka te grudi njene opajaju u svako doba, njezina ljubav zatravljuje bez prestanka! 
\par 20 TÓa zašto bi se, sine moj, zanosio preljubnicom i grlio tuđinki njedra? 
\par 21 Jer pred Jahvinim su očima čovjekovi putovi i on motri sve njegove staze. 
\par 22 Opakoga će uhvatiti njegova zloća i sapet će ga užad njegovih grijeha. 
\par 23 Umrijet će jer nema pouke, propast će zbog svoje goleme gluposti. 


\chapter{6}

\par 1 Sine moj, kad jamčiš bližnjemu svojem  i daš svoju ruku drugome, 
\par 2 vezao si se vlastitim usnama, uhvatio se riječima svojih usta; 
\par 3 učini onda ovo, sine moj: oslobodi se! Jer si dopao u ruke bližnjemu svojemu; idi, baci se preda nj i salijeći bližnjega svoga. 
\par 4 Ne daj sna svojim očima ni drijema svojim vjeđama; 
\par 5 otmi se kao gazela iz mreže i kao ptica iz ruku ptičaru. 
\par 6 Idi k mravu, lijenčino, promatraj njegove pute i budi mudar: 
\par 7 on nema vođe, nadzornika, ni nadstojnika, 
\par 8 ljeti se sebi brine za hranu i prikuplja jelo u doba žetve. 
\par 9 A ti, dokle ćeš, lijenčino, spavati? Kad ćeš se dići oda sna svoga? 
\par 10 Još malo odspavaj, još malo odrijemaj, još malo podvij ruke za počinak 
\par 11 i doći će tvoje siromaštvo kao skitač i tvoja oskudica kao oružanik. 
\par 12 Nevaljalac i opak čovjek hodi s lažljivim ustima; 
\par 13 namiguje očima, lupka nogama, pokazuje prstima; 
\par 14 prijevare su mu u srcu, snuje zlo u svako doba, zameće svađe. 
\par 15 Zato će mu iznenada doći propast, i učas će se slomiti i neće mu biti lijeka. 
\par 16 Šest je stvari koje Gospod mrzi, a sedam ih je gnusoba njegovu biću: 
\par 17 ohole oči, lažljiv jezik, ruke koje prolijevaju krv nevinu, 
\par 18 srce koje smišlja grešne misli, noge koje hitaju na zlo, 
\par 19 lažan svjedok koji širi laži, i čovjek koji zameće svađe među braćom. 
\par 20 Sine moj, čuvaj zapovijedi oca svoga i ne odbacuj nauka matere svoje. 
\par 21 Priveži ih sebi na srce zauvijek, ovij ih oko svoga grla; 
\par 22 da te vode kada hodiš, da te čuvaju kada spavaš i da te razgovaraju kad se probudiš. 
\par 23 Jer je zapovijed svjetiljka, pouka je svjetlost, opomene stege put su života; 
\par 24 da te čuvaju od zle žene, od laskava jezika tuđinke. 
\par 25 Ne poželi u svom srcu njezine ljepote i ne daj da te osvoji trepavicama svojim, 
\par 26 jer bludnici dostaje i komad kruha, dok preljubnica lovi dragocjeni život. 
\par 27 Može li tko nositi oganj u njedrima a da mu se odjeća ne upali? 
\par 28 Može li tko hoditi po živom ugljevlju a svojih nogu da ne ožeže? 
\par 29 Tako biva onomu tko ide k ženi svoga bližnjega: neće ostati bez kazne tko god se nje dotakne. 
\par 30 Ne sramote li lupeža sve ako je krao da gladan utoli glad: 
\par 31 uhvaćen, on sedmerostruko vraća i plaća svim imanjem kuće svoje. 
\par 32 Nerazuman je, dakle, tko se upušta s preljubnicom; dušu svoju gubi koji tako čini. 
\par 33 Bruke i sramote dopada i rug mu se nikad ne briše. 
\par 34 Jer bijesna je ljubomornost u muža: on ne zna za milost u osvetni dan; 
\par 35 ne pristaje ni na kakav otkup i ne prima ma kolike mu darove dao. 


\chapter{7}

\par 1 Čuvaj, sine, riječi moje i pohrani moje zapovijedi kod sebe. 
\par 2 Čuvaj moje zapovijedi, i bit ćeš živ, i nauk moj kao zjenicu oka svoga. 
\par 3 Priveži ih sebi na prste, upiši ih na ploči srca svoga; 
\par 4 reci mudrosti: "Moja si sestra" i razboritost nazovi "sestričnom", 
\par 5 da te čuva od žene preljubnice, od tuđinke koja laskavo govori. 
\par 6 Kad bijah jednom na prozoru svoje kuće i gledah van kroz rešetku, 
\par 7 vidjeh među lakovjernima, opazih među momcima nerazumna mladića: 
\par 8 prolazio je ulicom kraj njezina ugla i koracao putem k njezinoj kući 
\par 9 u sumraku između dana i večeri kad se hvata noćna tmina; 
\par 10 i gle, susrete ga žena, bludno odjevena i s prijevarom u srcu. 
\par 11 Jogunasta bijaše i razuzdana, noge joj se nisu mogle u kući zadržati; 
\par 12 bila je čas na ulici, čas na trgovima i vrebala kod svakog ugla; 
\par 13 i uhvati ga i poljubi i reče mu bezobrazna lica: 
\par 14 "Bila sam dužna žrtvu pričesnicu, i danas izvrših svoj zavjet; 
\par 15 zato sam ti izašla u susret, da te tražim, i nađoh te. 
\par 16 Svoju sam postelju nastrla sagovima, vezenim pokrivačima misirskim; 
\par 17 svoj sam krevet namirisala smirnom, alojem i cimetom. 
\par 18 Hajde da se opijamo nasladom do jutra i da se radujemo užicima ljubavi. 
\par 19 Jer muža mi nema kod kuće: otišao je na dalek put; 
\par 20 uzeo je sa sobom novčani tobolac; a vratit će se kući tek o uštapu." 
\par 21 Tako ga zavede svojim vičnim nagovorom, odvuče ga svojim glatkim usnama. 
\par 22 I ludo on pođe za njom, kao što vol ide na klaonicu i kao što jelen zapleten u mrežu čeka 
\par 23 dok mu strijela ne probije jetra, i kao ptica što ulijeće u zamku, i ne znajući da će ga to života stajati. 
\par 24 Zato me, sine moj, poslušaj i čuj riječi mojih usta. 
\par 25 Nek' ti srce ne zastranjuje na njezine putove i ne lutaj po njezinim stazama. 
\par 26 Jer je mnoge smrtno ranila i oborila, i mnogo je onih što ih je pobila. 
\par 27 U Podzemlje vode putovi kroz njenu kuću, dolje u odaje smrti. 


\chapter{8}

\par 1 Ne propovijeda li mudrost i ne diže li razboritost svoj glas? 
\par 2 Navrh brda, uza cestu, na raskršćima stoji, 
\par 3 kod izlaza iz grada, kraj ulaznih vrata, ona glasno viče: 
\par 4 "Vama, o ljudi, propovijedam i upravljam svoj glas sinovima ljudskim. 
\par 5 Shvatite mudrost, vi neiskusni, a vi nerazumni, urazumite srce. 
\par 6 Slušajte, jer ću zboriti o važnim stvarima, i moje će usne otkriti što je pravo. 
\par 7 Jer moje nepce zbori istinu i zloća je mojim usnama mrska. 
\par 8 Sve su riječi mojih usta pravične, u njima nema ništa ni krivo ni prijetvorno. 
\par 9 Sve su one jasne razboritomu i pravedne onomu tko je stekao spoznaju. 
\par 10 Primajte radije moju pouku no srebro i znanje požudnije od zlata. 
\par 11 Jer mudrost je vrednija od biserja i nikakve se dragocjenosti ne mogu porediti s njom. 
\par 12 Ja, mudrost, boravim s razboritošću i posjedujem znanje umna djelovanja. 
\par 13 Strah Gospodnji mržnja je na zlo. Oholost, samodostatnost, put zloće i usta puna laži - to ja mrzim. 
\par 14 Moji su savjet i razboritost, ja sam razbor i moja je jakost. 
\par 15 Po meni kraljevi kraljuju i velikaši dijele pravdu. 
\par 16 Po meni knezuju knezovi i odličnici i svi suci zemaljski. 
\par 17 Ja ljubim one koji ljube mene i nalaze me koji me traže. 
\par 18 U mene je bogatstvo i slava, postojano dobro i pravednost. 
\par 19 Moj je plod bolji od čista i žežena zlata i moj je prihod bolji od čistoga srebra. 
\par 20 Ja kročim putem pravde, sred pravičnih staza, 
\par 21 da dadem dobra onima koji me ljube i napunim njihove riznice. 
\par 22 Jahve me stvori kao počelo svoga djela, kao najraniji od svojih čina, u pradoba; 
\par 23 oblikovana sam još od vječnosti, odiskona, prije nastanka zemlje. 
\par 24 Rodih se kad još nije bilo pradubina, dok nije bilo izvora obilnih voda. 
\par 25 Rodih se prije nego su utemeljene gore, prije brežuljaka. 
\par 26 Kad još ne bijaše načinio zemlje, ni poljana, ni početka zemaljskom prahu; 
\par 27 kad je stvarao nebesa, bila sam nazočna, kad je povlačio krug na licu bezdana. 
\par 28 Kad je u visini utvrđivao oblake i kad je odredio snagu izvoru pradubina; 
\par 29 kad je postavljao moru njegove granice da mu se vode ne preliju preko obala, kad je polagao temelje zemlji, 
\par 30 bila sam kraj njega, kao graditeljica, bila u radosti, iz dana u dan, igrajući pred njim sve vrijeme: 
\par 31 igrala sam po tlu njegove zemlje, i moja su radost djeca čovjekova. 
\par 32 Tako, djeco, poslušajte me, blago onima koji čuvaju moje putove. 
\par 33 Poslušajte pouku - da stečete mudrost i nemojte je odbaciti. 
\par 34 Blago čovjeku koji me sluša i bdi na mojim vratima svaki dan i koji čuva dovratnike moje. 
\par 35 Jer tko nalazi mene, nalazi život i stječe milost od Jahve. 
\par 36 A ako se ogriješi o mene, udi svojoj duši: svi koji mene mrze ljube smrt." 


\chapter{9}

\par 1 Mudrost je sazidala sebi kuću, i otesala sedam stupova. 
\par 2 Poklala je svoje klanice, pomiješala svoje vino i postavila svoj stol. 
\par 3 Poslala je svoje djevojke da objave svrh gradskih visina: 
\par 4 "Tko je neiskusan, neka se svrati ovamo!" A nerazumnima govori: 
\par 5 "Hodite, jedite od mojega kruha i pijte vina koje sam pomiješala. 
\par 6 Ostavite ludost, da biste živjeli, i hodite putem razboritosti." 
\par 7 Tko poučava podrugljivca, prima pogrdu, i tko prekorava opakoga, prima ljagu. 
\par 8 Ne kori podsmjevača, da te ne zamrzi; kori mudra, da te zavoli. 
\par 9 Pouči mudroga, i bit će još mudriji; uputi pravednoga, i uvećat će se njegovo znanje. 
\par 10 Gospodnji strah početak je mudrosti, a razboritost je spoznaja Presvetog. 
\par 11 "Po meni ti se umnožavaju dani i množe ti se godine života. 
\par 12 Ako si mudar, sebi si mudar; budeš li podsmjevač, sam ćeš snositi." 
\par 13 Gospođa ludost puna je strasti, prosta je i ne zna ništa. 
\par 14 I sjedi na vratima svoje kuće na stolici, u gradskim visinama, 
\par 15 te poziva one koji prolaze putem, koji ravno idu svojim stazama: 
\par 16 "Tko je neiskusan, neka se svrati ovamo!" I nerazumnomu govori: 
\par 17 "Kradena je voda slatka i ugodno je potajno jesti kruh." 
\par 18 A on ne zna da su Sjene ondje, da uzvanici njezini počivaju u Podzemlju. 


\chapter{10}

\par 1 Mudar sin veseli oca, a lud je sin žalost majci svojoj. 
\par 2 Ne koristi krivo stečeno blago, dok pravednost izbavlja od smrti. 
\par 3 Ne dopušta Jahve da gladuje duša pravednika, ali odbija pohlepu opakih. 
\par 4 Lijena ruka osiromašuje čovjeka, a marljiva ga obogaćuje. 
\par 5 Tko sabira ljeti, razuman je sin, a tko hrče o žetvi, navlači sramotu. 
\par 6 Blagoslovi su nad glavom pravedniku, a usta opakih kriju nasilje. 
\par 7 Pravednikov je spomen blagoslovljen, a opakom se ime proklinje. 
\par 8 Tko je mudra srca, prima zapovijedi, dok brbljava luda propada. 
\par 9 Tko nedužno živi, hodi bez straha, a tko ide krivim putovima, poznat će se. 
\par 10 Tko žmirka okom, zadaje tugu, a tko ludo zbori, propada. 
\par 11 Pravednikova su usta izvor života, a opakomu usta kriju nasilje. 
\par 12 Mržnja izaziva svađu, a ljubav pokriva sve pogreške. 
\par 13 Na usnama razumnoga nalazi se mudrost, a batina je za leđa nerazumna čovjeka. 
\par 14 Mudri kriju znanje, a luđakova su usta blizu propasti. 
\par 15 Blago je bogatomu tvrdi grad, a ubogima je propast njihovo siromaštvo. 
\par 16 Pravednik prirađuje za život, a opaki prirađuje za grijeh. 
\par 17 Tko se naputka drži, na putu je života, a zabluđuje tko se na ukor ne osvrće. 
\par 18 Lažljive usne kriju mržnju, a tko klevetu širi, bezuman je! 
\par 19 Obilje riječi ne biva bez grijeha, a tko zauzdava svoj jezik, razuman je. 
\par 20 Pravednikov je jezik odabrano srebro, a razum opakoga malo vrijedi. 
\par 21 Pravednikove su usne hrana mnogima, a luđaci umiru s ludosti svoje. 
\par 22 Gospodnji blagoslov obogaćuje i ne prati ga nikakva muka. 
\par 23 Bezumniku je radost učiniti sramotno djelo, a razumnu čovjeku biti mudar. 
\par 24 Čega se opaki boji, ono će ga stići, a pravednička se želja ispunjava. 
\par 25 Kad oluja prohuja, opakoga nestane, a pravednik ima temelj vječni. 
\par 26 Kakav je ocat zubima i dim očima, takav je ljenivac onima koji ga šalju. 
\par 27 Strah Gospodnji umnaža dane, a opakima se prekraćuju godine. 
\par 28 Pravedničko je ufanje puno radosti, a opakima je nada uprazno. 
\par 29 Gospodnji je put okrilje bezazlenu, a propast onima koji čine zlo. 
\par 30 Pravednik se neće nikad pokolebati, a opakih će nestati s lica zemlje. 
\par 31 Pravednikova usta rađaju mudrošću, a opak jezik čupa se s korijenom. 
\par 32 Pravednikove usne znaju što je milo, dok usta opakih poznaju zloću. 


\chapter{11}

\par 1 Lažna je mjera mrska Jahvi, a puna mjera mila mu je. 
\par 2 S ohološću dolazi sramota, a u smjernih je mudrost. 
\par 3 Pravednike vodi nevinost njihova, a bezbožnike upropašćuje njihova opačina. 
\par 4 Ne pomaže bogatstvo u dan Božje srdžbe, a pravednost izbavlja od smrti. 
\par 5 Nedužnomu pravda njegova put utire, a zao propada od svoje zloće. 
\par 6 Poštene izbavlja pravda njihova, a bezbožnici se hvataju u svoju lakomost. 
\par 7 Kad zao čovjek umre, nada propada i ufanje u imetak ruši se. 
\par 8 Pravednik se od tjeskobe izbavlja, a opaki dolazi na mjesto njegovo. 
\par 9 Bezbožnik ustima ubija svoga bližnjega, a pravednici se izbavljaju znanjem. 
\par 10 Sa sreće pravedničke grad se raduje i klikuje zbog propasti opakoga. 
\par 11 Blagoslovom pravednika grad se diže, a ustima opakih razara se. 
\par 12 Nerazumnik prezire svoga bližnjega, dok čovjek uman šuti. 
\par 13 Tko s klevetom hodi, otkriva tajnu, a čovjek pouzdana duha čuva se. 
\par 14 Gdje vodstva nema, narod propada, jer spasenje je u mnogim savjetnicima. 
\par 15 Veoma zlo prolazi tko jamči za drugoga, a bez straha je tko mrzi na jamstvo. 
\par 16 Ljupka žena stječe slavu, a krepki muževi bogatstvo. 
\par 17 Dobrostiv čovjek sam sebi dobro čini, a okrutnik muči vlastito tijelo. 
\par 18 Opak čovjek pribavlja isprazan dobitak, a tko sije pravdu, ima sigurnu nagradu. 
\par 19 Tko je čvrst u pravednosti, ide u život, a tko za zlom trči, na smrt mu je. 
\par 20 Mrski su Jahvi srcem opaki, a mili su mu životom savršeni. 
\par 21 Zaista, zao čovjek neće proći bez kazne, a rod će se pravednički izbaviti. 
\par 22 Zlatan je kolut na rilu svinjskom: žena lijepa, a bez razuma. 
\par 23 Pravednička je želja samo na sreću, a nada je opakih prolazna. 
\par 24 Tko dijeli obilato, sve više ima, a tko škrtari, sve je siromašniji. 
\par 25 Podašna duša nalazi okrepu, i tko napaja druge, sam će se napojiti. 
\par 26 Tko ne da žita, kune ga narod, a blagoslov je nad glavom onoga koji ga prodaje. 
\par 27 Tko traži dobro, nalazi milost, a tko za zlom ide, ono će ga snaći. 
\par 28 Tko se uzda u bogatstvo, propada, a pravednici uspijevaju kao zeleno lišće. 
\par 29 Tko vlastitu kuću zapusti, vjetar žanje, a luđak je sluga mudromu. 
\par 30 Plod je pravednikov drvo života, i mudrac je tko predobiva žive duše. 
\par 31 Ako se pravedniku plaća na zemlji, još će se više opakomu i grešniku. 


\chapter{12}

\par 1 Tko ljubi pouku, ljubi znanje, a tko mrzi ukor, lud je. 
\par 2 Dobar dobiva milost od Jahve, a podmukao osudu. 
\par 3 Zloćom se čovjek ne utvrđuje, a korijen se pravedniku ne pomiče. 
\par 4 Kreposna je žena vijenac mužu svojemu, a sramotna mu je kao gnjilež u kostima. 
\par 5 Pravedničke su misli pravične, spletke opakih prijevarne. 
\par 6 Riječi opakih pogubne su zamke, a pravedne izbavljaju usta njihova. 
\par 7 Opaki se ruše i nema ih više, a kuća pravednika ostaje. 
\par 8 Čovjek se hvali po oštrini svoga razuma, a prezire se tko je opak srcem. 
\par 9 Bolje je biti malen i imati samo jednog slugu nego se hvastati a nemati ni kruha. 
\par 10 Pravednik pazi i na život svog živinčeta, dok je opakomu srce okrutno. 
\par 11 Tko obrađuje svoju zemlju, sit je kruha, a tko trči za ništavilom, nerazuman je. 
\par 12 Čežnja je opakoga mreža od zala, a korijen pravednika daje ploda. 
\par 13 Opakomu je zamka grijeh njegovih usana, a pravednik se izbavlja od tjeskobe. 
\par 14 Od ploda svojih usta nasitit će se svatko obilno, a ono što je rukama učinio vratit će mu se. 
\par 15 Luđaku se čini pravim njegov put, a mudar čovjek sluša savjete. 
\par 16 Luđak odmah odaje svoj bijes, a pametan pokriva sramotu. 
\par 17 Tko govori istinu, otkriva što je pravo, a lažljiv svjedok prijevaru. 
\par 18 Nesmotren govori kao da mačem probada, a jezik je mudrih iscjeljenje. 
\par 19 Istinita usta traju dovijeka, a lažljiv jezik samo za čas. 
\par 20 Prijevara je u srcu onih koji snuju zlo, a veselje u onih koji dijele miroljubive savjete. 
\par 21 Pravednika ne stiže nikakva nevolja, a opaki u zlu grcaju. 
\par 22 Mrske su Jahvi usne lažljive, a mili su mu koji zbore istinu. 
\par 23 Promišljen čovjek prikriva svoje znanje, a srce bezumničko razglašuje svoju ludost. 
\par 24 Marljiva ruka vlada, a nemar vodi u podložnost. 
\par 25 Briga u srcu pritiskuje čovjeka, a blaga riječ veseli ga. 
\par 26 Pravednik vodi svojeg prijatelja, a opake zavodi njihov put. 
\par 27 Nemaran ne ulovi svoje lovine, a marljivost je čovjeku blago dragocjeno. 
\par 28 Na stazi pravice stoji život i na njezinu putu nema smrti. 


\chapter{13}

\par 1 Mudar sin sluša naputak očev, a podsmjevač ne sluša ukora. 
\par 2 Od ploda usta svojih uživa čovjek sreću, a srce je nevjernika puno nasilja. 
\par 3 Tko čuva usta svoja, čuva život svoj, a tko nesmotreno zbori, o glavu mu je. 
\par 4 Uzaludna je žudnja lijenčine, a ispunit će se želja marljivih. 
\par 5 Pravednik mrzi na lažljivu riječ, a opaki goji mržnju i sramotu. 
\par 6 Pravda čuva pobožna, a opake grijeh obara. 
\par 7 Netko se gradi bogatim, a ništa nema, netko se gradi siromašnim, a ima veliko bogatstvo. 
\par 8 Otkup života bogatstvo je čovjeku; a siromah ne sluša opomene. 
\par 9 Svjetlost pravednička blistavo sja, a svjetiljka opakih gasi se. 
\par 10 Oholost rađa samo svađu, a mudrost je u onih koji primaju savjet. 
\par 11 Naglo stečeno bogatstvo iščezava, a tko sabire pomalo, biva bogat. 
\par 12 Predugo očekivanje ubija srce, a ispunjena želja drvo je života. 
\par 13 Tko riječ prezire, taj propada, a tko poštiva zapovijedi, plaću dobiva. 
\par 14 Pouka mudračeva izvor je životni, ona izbavlja od zamke smrti. 
\par 15 Uvid u dobro pribavlja milost, a put bezbožnika hrapav je. 
\par 16 Svatko pametan djeluje promišljeno, a bezumnik se hvališe svojom ludošću. 
\par 17 Zao glasnik zapada u zlo, a vjeran poslanik donosi spasenje. 
\par 18 Siromaštvo i sramota onomu tko odbija pouku, a tko ukor prima, doći će do časti. 
\par 19 Slatka je duši ispunjena želja, a bezumnicima je mrsko kloniti se oda zla. 
\par 20 Druži se s mudrima, i postat ćeš mudar, a tko se drži bezumnika, postaje opak. 
\par 21 Grešnika progoni zlo, a dobro je nagrada pravednima. 
\par 22 Valjan čovjek ostavlja baštinu unucima, a bogatstvo se grešnikovo čuva pravedniku. 
\par 23 Izobilje je hrane na krčevini siromaškoj, a ima i tko propada s nepravde. 
\par 24 Tko štedi šibu, mrzi na sina svog, a tko ga ljubi, na vrijeme ga opominje. 
\par 25 Pravednik ima jela do sitosti, a trbuh opakih poznaje oskudicu. 


\chapter{14}

\par 1 Ženska mudrost sagradi kuću, a ludost je rukama razgrađuje. 
\par 2 Tko živi s poštenjem, boji se Jahve, a tko ide stranputicom, prezire ga. 
\par 3 U luđakovim je ustima šiba za oholost njegovu, a mudre štite vlastite usne. 
\par 4 Gdje nema volova, prazne su jasle, a obilna je žetva od snage bikove. 
\par 5 Istinit svjedok ne laže, a krivi svjedok širi laž. 
\par 6 Podsmjevač traži mudrost i ne nalazi je, a razumni lako dolazi do znanja. 
\par 7 Idi od čovjeka bezumna jer nećeš upoznati usne što zbore znanje. 
\par 8 Mudrost je pametna čovjeka u tom što pazi na svoj put, a bezumnička ludost prijevara je. 
\par 9 Luđacima je grijeh šala, a milost je Božja s poštenima. 
\par 10 Srce poznaje svoj jad, i veselje njegovo ne može dijeliti nitko drugi. 
\par 11 Dom opakih propast će, a šator će pravednika procvasti. 
\par 12 Neki se put učini čovjeku prav, a na koncu vodi k smrti. 
\par 13 I u smijehu srce osjeća bol, a poslije veselja dolazi tuga. 
\par 14 Otpadnik se siti svojim prestupcima, a dobar čovjek svojim radom. 
\par 15 Glupan vjeruje svakoj riječi, a pametan pazi na korak svoj. 
\par 16 Mudar se boji i oda zla se uklanja, a bezuman se raspaljuje i bez straha je. 
\par 17 Nagao čovjek čini ludosti, a razborit ih podnosi. 
\par 18 Glupaci baštine ludost, a mudre ovjenčava znanje. 
\par 19 Zli padaju ničice pred dobrima i opaki pred vratima pravednikovim. 
\par 20 I svom prijatelju mrzak je siromah, a bogataš ima mnogo ljubitelja. 
\par 21 Griješi tko prezire bližnjega svoga, a blago onomu tko je milostiv ubogima. 
\par 22 Koji snuju zlo, ne hode li stranputicom, a zar nisu dobrota i vjernost s onima koji snuju dobro? 
\par 23 U svakom trudu ima probitka, a pusto brbljanje samo je na siromaštvo. 
\par 24 Mudrima je vijenac bogatstvo njihovo, a bezumnima kruna - njihova ludost. 
\par 25 Istinit svjedok izbavlja duše, a tko laži širi, taj je varalica. 
\par 26 U strahu je Gospodnjem veliko pouzdanje i njegovim je sinovima utočište. 
\par 27 Strah Gospodnji izvor je života: on izbavlja od zamke smrti. 
\par 28 Mnoštvo je naroda ponos kralju, a bez puka knez propada. 
\par 29 Tko se teško srdi, velike je razboritosti, a nabusit duhom pokazuje ludost. 
\par 30 Mirno je srce život tijelu, a ljubomor je gnjilež u kostima. 
\par 31 Tko tlači siromaha huli na stvoritelja, a časti ga tko je milostiv ubogomu. 
\par 32 Opaki propada zbog vlastite pakosti, a pravednik i u samoj smrti nalazi utočište. 
\par 33 U razumnu srcu mudrost počiva, a što je u bezumnome, to se i pokaže. 
\par 34 Pravednost uzvisuje narod, a grijeh je sramota pucima. 
\par 35 Kralju je mio razborit sluga, a na sramotna se srdi. 


\chapter{15}

\par 1 Blag odgovor ublažava jarost, a riječ osorna uvećava srdžbu. 
\par 2 Jezik mudrih ljudi proslavlja znanje, a usta bezumnih prosipaju ludost. 
\par 3 Oči su Jahvine na svakome mjestu i budno motre i zle i dobre. 
\par 4 Blaga je besjeda drvo života, a pakosna je rana duhu. 
\par 5 Luđak prezire pouku oca svog, a tko ukor prima, pametno čini. 
\par 6 U pravednikovoj je kući mnogo blaga, a opaki zarađuje propast svoju. 
\par 7 Usne mudrih siju znanje, a srce je bezumnika nepostojano. 
\par 8 Žrtva opakog mrska je Jahvi, a mila mu je molitva pravednika. 
\par 9 Put opakih Jahvi je mrzak, a mio mu je onaj koji ide za pravicom. 
\par 10 Oštra kazna čeka onog tko ostavlja pravi put, a umrijet će tko mrzi ukor. 
\par 11 I Šeol i Abadon stoje pred Jahvom, a nekmoli srca sinova ljudskih. 
\par 12 Podsmjevač ne ljubi onog tko ga kori: on se ne druži s mudrima. 
\par 13 Veselo srce razvedrava lice, a bol u srcu tjeskoba je duhu. 
\par 14 Razumno srce traži znanje, a bezumnička se usta bave ludošću. 
\par 15 Svi su dani bijednikovi zli, a komu je srce sretno, na gozbi je bez prestanka. 
\par 16 Bolje je malo sa strahom Gospodnjim nego veliko blago i s njime nemir. 
\par 17 Bolji je obrok povrća gdje je ljubav nego od utovljena vola gdje je mržnja. 
\par 18 Gnjevljiv čovjek zameće svađu, a ustrpljiv utišava raspru. 
\par 19 Put je ljenivčev kao glogov trnjak, a utrta je staza pravednika. 
\par 20 Mudar sin veseli oca, a bezumnik prezire majku svoju. 
\par 21 Ludost je veselje nerazumnomu, a razuman čovjek pravo hodi. 
\par 22 Ne uspijevaju nakane kad nema vijećanja, a ostvaruju se gdje je mnogo savjetnika. 
\par 23 Čovjek se veseli odgovoru usta svojih, i riječ u pravo vrijeme - kako je ljupka! 
\par 24 Razumnu čovjeku put života ide gore, da izmakne carstvu smrti koje je dolje. 
\par 25 Jahve ruši kuću oholima, a postavlja među udovici. 
\par 26 Mrske su Jahvi zle misli, a dobrostive riječi mile su mu. 
\par 27 Tko se grabežu oda, razara svoj dom, a tko mrzi mito, živjet će. 
\par 28 Pravednikovo srce smišlja odgovor, a opakomu usta govore zlobom. 
\par 29 Daleko je Jahve od opakih, a uslišava molitvu pravednih. 
\par 30 Bistar pogled razveseli srce i radosna vijest oživi kosti. 
\par 31 Uho koje posluša spasonosan ukor prebiva među mudracima. 
\par 32 Tko odbaci pouku, prezire vlastitu dušu, a tko posluša ukor, stječe razboritost. 
\par 33 Strah je Gospodnji škola mudrosti, jer pred slavom ide poniznost. 


\chapter{16}

\par 1 Čovjek snuje u srcu, a od Jahve je što će jezik odgovoriti. 
\par 2 Čovjeku se svi njegovi putovi čine čisti, a Jahve ispituje duhove. 
\par 3 Prepusti Jahvi svoja djela, i tvoje će se namisli ostvariti. 
\par 4 Jahve je sve stvorio u svoju svrhu, pa i opakoga za dan zli. 
\par 5 Mrzak je Jahvi svatko ohola duha: takav zaista ne ostaje bez kazne. 
\par 6 Ljubavlju se i vjernošću pomiruje krivnja, i strahom se Gospodnjim uklanja zlo. 
\par 7 Kad su Jahvi mili putovi čovječji, i neprijatelje njegove miri s njim. 
\par 8 Bolje je malo s pravednošću nego veliki dohoci s nepravdom. 
\par 9 Srce čovječje smišlja svoj put, ali Jahve upravlja korake njegove. 
\par 10 Proročanstvo je na usnama kraljevim: u osudi se njegova usta neće ogriješiti. 
\par 11 Mjere i tezulje pripadaju Jahvi; njegovo su djelo i svi utezi. 
\par 12 Mrsko je kraljevima počiniti opačinu, jer se pravdom utvrđuje prijestolje. 
\par 13 Mile su kraljevima usne pravedne i oni ljube onog koji govori pravo. 
\par 14 Jarost je kraljeva vjesnik smrti ali je mudar čovjek ublaži. 
\par 15 U kraljevu je vedru licu život, i njegova je milost kao oblak s kišom proljetnom. 
\par 16 Probitačnije je steći mudrost nego zlato, i stjecati razbor dragocjenije je nego srebro. 
\par 17 Životni je put pravednih: kloniti se zla, i tko pazi na svoj put, čuva život svoj. 
\par 18 Pred slomom ide oholost i pred padom uznositost. 
\par 19 Bolje je biti krotak s poniznima nego dijeliti plijen s oholima. 
\par 20 Tko pazi na riječ, nalazi sreću, i tko se uzda u Jahvu, blago njemu. 
\par 21 Mudar srcem naziva se razumnim i prijazne usne uvećavaju znanje. 
\par 22 Izvor je životni razum onima koji ga imaju, a ludima je kazna njihova ludost. 
\par 23 Mudračev duh urazumljuje usta njegova, na usnama mu znanje umnožava. 
\par 24 Saće meda riječi su ljupke, slatke duši i lijek kostima. 
\par 25 Neki se put čini čovjeku prav, a na kraju vodi k smrti. 
\par 26 Radnikova glad radi za nj; jer ga tjeraju usta njegova. 
\par 27 Bezočnik pripravlja samo zlo i na usnama mu je oganj plameni. 
\par 28 Himben čovjek zameće svađu i klevetnik razdor među prijatelje. 
\par 29 Nasilnik zavodi bližnjega svoga i navodi ga na rđav put. 
\par 30 Tko očima namiguje, himbu smišlja, a tko usne stišće, već je smislio pakost. 
\par 31 Sijede su kose prekrasna kruna, nalaze se na putu pravednosti. 
\par 32 Tko se teško srdi, bolji je od junaka, i tko nad sobom vlada, bolji je od osvojitelja grada. 
\par 33 U krilo plašta baca se kocka, ali je od Jahve svaka odluka. 


\chapter{17}

\par 1 Bolji je zalogaj suha kruha s mirom nego sa svađom kuća puna žrtvene pečenke. 
\par 2 Razuman sluga vlada nad sinom sramotnim i s braćom će dijeliti baštinu. 
\par 3 Taljika je za srebro i peć za zlato, a srca iskušava Jahve sam. 
\par 4 Zločinac rado sluša usne prijevarne, i lažac spremno prisluškuje pogubnu jeziku. 
\par 5 Tko se ruga siromahu, podruguje se Stvoritelju njegovu, i tko se veseli nesreći, ne ostaje bez kazne. 
\par 6 Unuci su vijenac starcima, a sinovima ures oci njihovi. 
\par 7 Ne dolikuje budali uzvišena besjeda, a još manje odličniku usne lažljive. 
\par 8 Dar je čarobni kamen u očima onoga koji ga daje: kamo se god okrene, uspijeva. 
\par 9 Tko prikriva prijestup, traži ljubav, a tko glasinu širi, razgoni prijatelje. 
\par 10 Razumna se ukor jače doima nego bezumna stotina udaraca. 
\par 11 Opak čovjek ide samo za zlom, ali se okrutan glasnik šalje na nj. 
\par 12 Bolje je nabasati na medvjedicu kojoj ugrabiše mlade nego na bezumnika u njegovoj ludosti. 
\par 13 Tko dobro zlom uzvraća neće ukloniti nesreću od doma svojeg. 
\par 14 Zametnuti svađu isto je kao pustiti poplavu: stoga prije nego svađa izbije, udalji se! 
\par 15 Tko opravdava krivoga i tko osuđuje pravoga, obojica su mrski Jahvi. 
\par 16 Čemu novac u ruci bezumnomu? Da njime mudrost kupi, kad nema razbora! 
\par 17 Prijatelj ljubi u svako vrijeme, a u nevolji i bratom postaje. 
\par 18 Nerazuman čovjek daje ruku i jamči pred svojim bližnjim. 
\par 19 Grijeh ljubi tko ljubi svađu, i tko visoko diže svoja vrata, traži propast. 
\par 20 Opak srcem ne nalazi sreće, i komu je jezik zao, zapada u nesreću. 
\par 21 Tko rodi bezumna, na tugu mu je; a nije veseo ni otac budale. 
\par 22 Veselo je srce izvrstan lijek, a neveseo duh suši kosti. 
\par 23 Opaki prima dar iz njedara da bi iskrivio putove pravici. 
\par 24 Razuman ima mudrost pred sobom, a bezumniku su oči na kraj zemlje. 
\par 25 Briga je ocu bezuman sin i žalost roditeljki svojoj. 
\par 26 Ne valja kažnjavati pravednika, a nije pravo ni tući odličnike. 
\par 27 Tko usteže svoje riječi, razumije mudrost, i razuman je čovjek mirna duha. 
\par 28 I luđak se smatra mudrim kada šuti i razumnim kad susteže svoje usne. 


\chapter{18}

\par 1 Vlastitoj požudi popušta onaj tko zastranjuje, i svađa se usprkos svakom razboru. 
\par 2 Bezumnomu nije mio razum; stalo mu je dati srcu oduška. 
\par 3 Kad dolazi opačina, dolazi i prezir i bruka sa sramotom. 
\par 4 Duboke su vode riječi iz usta nečijih, izvor mudrosti bujica što se razlijeva. 
\par 5 Ne valja se obazirati na opaku osobu, da se pravedniku nanese nepravda na sudu. 
\par 6 Bezumnikove se usne upuštaju u svađu i njegova usta izazivlju udarce. 
\par 7 Bezumnomu su propast vlastita usta i usne su mu zamka životu. 
\par 8 Klevetnikove su riječi kao poslastice: spuštaju se u dno utrobe. 
\par 9 Tko je nemaran u svom poslu, brat je onomu koji rasipa. 
\par 10 Tvrda je kula ime Jahvino: njemu se pravednik utječe i nalazi utočišta. 
\par 11 Bogatstvo je bogatašu njegova tvrđava i kao visok zid u mašti njegovoj. 
\par 12 Pred slomom se oholi srce čovječje, a pred slavom ide poniznost. 
\par 13 Tko odgovara prije nego što sasluša, na ludost mu je i sramotu. 
\par 14 Kad je čovjek bolestan, njegov ga duh podiže, a ubijen duh tko će podići? 
\par 15 Razumno srce stječe znanje i uho mudrih traži znanje. 
\par 16 Dar čovjeku otvara put i vodi ga pred velikaše. 
\par 17 Prvi je pravedan u svojoj parnici, a kad dođe njegov protivnik, opovrgne ga. 
\par 18 Ždrijeb poravna svađe, pa i među moćnicima odlučuje. 
\par 19 Uvrijeđen brat jači je od tvrda grada i svađe su kao prijevornice na tvrđavi. 
\par 20 Svatko siti trbuh plodom usta svojih, nasićuje se rodom usana svojih. 
\par 21 Smrt i život u vlasti su jeziku, a tko ga miluje, jede od ploda njegova. 
\par 22 Tko je našao ženu, našao je sreću i stekao milost od Jahve. 
\par 23 Ponizno moleći govori siromah, a grubo odgovara bogataš. 
\par 24 Ima prijatelja koji vode u propast, a ima i prijatelja privrženijih od brata. 


\chapter{19}

\par 1 Bolji je siromah koji živi u nedužnosti nego čovjek opakih usana i k tomu bezuman. 
\par 2 Revnost bez razboritosti nije dobra, i tko brzo hoda, spotiče se. 
\par 3 Ludost čovjeku kvari život, a srce mu se ljuti na Jahvu! 
\par 4 Bogatstvo pribavlja mnoge prijatelje, a siromaha i njegov prijatelj ostavlja. 
\par 5 Lažljiv svjekok ne ostaje bez kazne, i tko širi laži, neće uteći. 
\par 6 Mnogi laskaju licu odličnikovu i svatko je prijatelj čovjeku darežljivu. 
\par 7 Na siromaha mrze sva braća njegova, još više se udaljuju od njega prijatelji njegovi: on hlepi za dobrim riječima, ali ih ne nalazi! 
\par 8 Tko stječe razboritost, ljubi sebe, a tko čuva razum, nalazi sreću. 
\par 9 Lažljiv svjedok ne ostaje bez kazne, i tko širi laži, propada. 
\par 10 Ne dolikuje bezumnomu živjeti raskošno, a još manje sluzi vlast nad knezovima. 
\par 11 Um čovjeka usteže od srdžbe, a čast mu je oprostiti krivicu. 
\par 12 Kraljev je gnjev kao rika lavlja, a njegova milost kao rosa bilju. 
\par 13 Nesreća je ocu svojemu bezuman sin, i neprestano prokišnjavanje svađe su ženine. 
\par 14 Kuća se i bogatstvo baštine od otaca, a od Jahve je žena razumna. 
\par 15 Lijenost navlači čovjeku dubok san i nemarna duša gladuje. 
\par 16 Tko se drži zapovijedi, čuva život svoj, a tko ne pazi putove svoje, umire. 
\par 17 Jahvi pozaima tko je siromahu milostiv, i on će mu platiti dobročinstvo. 
\par 18 Kažnjavaj sina svoga dok ima nade, ali ne idi za tim da ga ubiješ. 
\par 19 Tko je jarostan, plaća globu, i kad ga štediš, samo uvećavaš njegov gnjev. 
\par 20 Slušaj savjet i primaj pouku, kako bi naposljetku postao mudar. 
\par 21 Mnogo je namisli u srcu čovječjem, ali što Jahve naumi, to i bude. 
\par 22 Dražest je čovjekova u dobroti njegovoj, i bolji je siromah od lažljivca. 
\par 23 Strah Gospodnji daje život, i tko se njime ispuni, zlo ga ne pohodi. 
\par 24 Lijenčina umače ruku u zdjelu, ali je ustima svojim ne prinosi. 
\par 25 Udari podsmjehivača, i lud se opameti; ukori razumnog, i shvatit će znanje. 
\par 26 Sin je sramotan i pokvaren tko zlostavlja oca i odgoni majku. 
\par 27 Prestani, sine moj, slušati naputke koji odvode od riječi spoznaje! 
\par 28 Nevaljao se svjedok podruguje pravdi i usta opakih gutaju nepravdu. 
\par 29 Pripravljene su kazne podsmjevačima i udarci za leđa bezumnika. 


\chapter{20}

\par 1 Vino je podsmjevač, žestoko piće bukač, i tko se njima odaje neće steći mudrosti. 
\par 2 Kraljev je gnjev kao rika lavlja: tko ga izaziva, griješi protiv sebe samog. 
\par 3 Čast je čovjeku ustegnuti se od raspre, a tko je bezuman počinje svađu. 
\par 4 Lijenčina u jesen ne ore: u doba žetve on traži, i ništa nema. 
\par 5 Savjet je u srcu čovječjem voda duboka i razuman će je čovjek iscrpsti. 
\par 6 Mnogi se naziva dobrim čovjekom, ali tko će naći vjerna čovjeka? 
\par 7 Pravednik hodi u bezazlenosti svojoj: blago sinovima njegovim poslije njega! 
\par 8 Kralj koji sjedi na stolici sudačkoj istražuje svako zlo svojim očima. 
\par 9 Tko može reći: "Očistih srce svoje, oprah se od grijeha svoga?" 
\par 10 Dvojaki utezi i dvojaka mjera mrski su Jahvi podjednako. 
\par 11 I dijete se poznaje po onome što čini, je li čisto i pravedno djelo njegovo. 
\par 12 I uho koje čuje i oko koje vidi, oboje je Jahve načinio. 
\par 13 Ne ljubi sna, da ne osiromašiš; otvori oči svoje i nasitit ćeš se kruha. 
\par 14 "Loše, loše", govori kupac, a kad ode, hvali se dobrom kupovinom. 
\par 15 Ima zlata i mnogih bisera, ali su mudre usne najdragocjeniji nakit. 
\par 16 Uzmi haljinu onomu tko je jamčio za drugoga; oplijeni njega umjesto tuđinca. 
\par 17 Sladak je čovjeku kruh prijevare, ali mu se usta poslije napune pijeskom. 
\par 18 Naumi se provode savjetom: zato dobro razmisli pa vodi boj! 
\par 19 Tko okolo kleveće, otkriva tajne: zato se ne miješaj s onim komu su usne uvijek otvorene. 
\par 20 Tko kune oca svoga i majku svoju svjetiljka mu se gasi usred tmine. 
\par 21 Od početka brzo stečeno imanje na koncu nije blagoslovljeno. 
\par 22 Nemoj govoriti: "Osvetit ću se za zlo"; čekaj Jahvu, i on će te spasiti. 
\par 23 Mrski su Jahvi dvojaki utezi, i kriva mjera ne valja. 
\par 24 Od Jahve su koraci čovječji, i kako da čovjek razumije svoj put? 
\par 25 Zamka je čovjeku nesmotreno reći: "Ovo je sveto", a poslije promišljati što je zavjetovao. 
\par 26 Mudar kralj umije izlučiti opake i stavlja ih pod točkove. 
\par 27 Svjetiljka je Gospodnja duh čovječji: ona istražuje sve do dna utrobe. 
\par 28 Dobrota i vjernost čuvaju kralja, jer dobrotom utvrđuje prijestol svoj. 
\par 29 Ljepota je mladićima njihova snaga, a starcima je ures sijeda kosa. 
\par 30 Krvave masnice očiste zlo i udarci pročiste odaje utrobe. 


\chapter{21}

\par 1 Kraljevo je srce u ruci Jahve kao voda tekućica; vodi ga kuda god hoće. 
\par 2 Svaki je put čovjeku pravedan u vlastitim očima, a Jahve ispituje srca. 
\par 3 Da se vrši pravda i čini pravo, draže je Jahvi nego žrtva. 
\par 4 Ponosite oči i oholo srce i svjetiljka opakih - to je grijeh. 
\par 5 Namisli marljivoga samo su na korist, a nagloga samo na siromaštvo. 
\par 6 Blago stečeno jezikom lažljivim nestalna je ispraznost onih koji traže smrt. 
\par 7 Opake će odnijeti nasilje njihovo jer ne žele činiti pravice. 
\par 8 Zapleten je put zločinca, a pravo je djelo čista čovjeka. 
\par 9 Bolje je živjeti pod rubom krova nego u zajedničkoj kući sa ženom svadljivom. 
\par 10 Duša opakoga želi zlo: u njega nema samilosti ni za bližnjega. 
\par 11 Kad se podsmjevač kazni, neiskusan postaje mudar, a mudri iz pouke crpe znanje. 
\par 12 Na kuću opakoga pazi Svepravedni i opake strovaljuje u nesreću. 
\par 13 Tko zatvori uho svoje pred vikom siromaha, i sam će vikati, ali ga neće nitko uslišati. 
\par 14 Potajan dar utišava srdžbu, a poklon ispod ruke i žestoku jarost. 
\par 15 Sud pravičan radost je pravedniku a užas zločincima. 
\par 16 Čovjek koji skreće s puta razbora počivat će u zboru mrtvačkom. 
\par 17 Tko ljubi veselje, postaje siromah, i tko ljubi vino i mirisno ulje, ne obogati se. 
\par 18 Opak čovjek otkup je za pravednika, i bezbožnik stupa na mjesto pravednog. 
\par 19 Bolje je živjeti u pustinji nego sa ženom svadljivom i gnjevljivom. 
\par 20 Krasno je blago i ulje u stanu mudroga, a bezuman ih čovjek rasipa. 
\par 21 Tko teži za pravdom i dobrohotnošću, nalazi život i čast. 
\par 22 Mudrac nadvladava i grad pun ratnika i krši silu u koju su se uzdali. 
\par 23 Tko čuva usta i jezik svoj, čuva sebe od nevolje. 
\par 24 Drzovitom i oholici ime je "podsmjevač"; on sve radi s prekomjernom drskošću. 
\par 25 Lijenčinu ubija želja njegova jer mu ruke bježe od posla. 
\par 26 Opak po cio dan živo želi, a pravednik daje i ne škrtari. 
\par 27 Mrska je žrtva opakih, osobito kad se požudno prinosi. 
\par 28 Lažljiv svjedok propada, a čovjek koji sluša, opet će govoriti. 
\par 29 Opaki pokazuju drsko lice, a poštenjak učvršćuje put svoj. 
\par 30 Nema mudrosti i nema razuma i nema savjeta protiv Jahve. 
\par 31 Konj se oprema za dan boja, ali Jahve daje pobjedu. 


\chapter{22}

\par 1 Dobro je ime bolje od velika bogatstva,  i bolja je naklonost od srebra i zlata. 
\par 2 Bogataš se i siromah sreću: obojicu ih Jahve stvori. 
\par 3 Pametan čovjek vidi zlo i skrije se, a glupaci idu bezbrižno i trpe kaznu. 
\par 4 Nagrada je poniznosti strah Gospodnji, bogatstvo, čast i život. 
\par 5 Trnje i zamke su na putu varalici: tko čuva život svoj, daleko je od oboga. 
\par 6 Upućuj dijete prema njegovu putu, pa kad i ostari, neće odstupiti od njega. 
\par 7 Bogataš vlada nad siromasima, a dužnik je sluga vjerovniku. 
\par 8 Tko sije nepravdu, žanje nesreću, i šiba njegova gnjeva udarit će njega samog. 
\par 9 Milostivo se oko blagoslivlje, jer daje od svog kruha siromahu. 
\par 10 Otjeraj podsmjevača i prestat će svađe i nestat će nesloga i pogrda. 
\par 11 Jahve ljubi čisto srce, i tko je ljubeznih usana, kralj mu je prijatelj. 
\par 12 Pogled Jahvin čuva znanje, Jahve pomućuje riječi bezbožnika. 
\par 13 Lijenčina veli: "Lav je vani, nasred trga poginuo bih." 
\par 14 Duboka jama usta su preljubnice, i na koga se Jahve srdi, pada onamo. 
\par 15 Ludost prianja uza srce djetinje: šiba pouke otklanja je od njega. 
\par 16 Tko tlači siromaha, taj mu koristi; tko daje bogatašu, samo mu šteti. 
\par 17 Riječi mudraca: Prigni uho svoje i čuj riječi moje i upravi svoje srce mojem znanju, 
\par 18 jer milina je ako ih čuvaš u nutrini svojoj, i kad ti budu sve spremne na usnama tvojim. 
\par 19 Da bi uzdanje tvoje bilo u Jahvi, upućujem danas i tebe. 
\par 20 Napisah ti trideset što savjeta što pouka 
\par 21 da te poučim riječima istine, da uzmogneš pouzdanim riječima odgovoriti onomu tko te zapita. 
\par 22 Nemoj pljačkati siromaha zato što je siromah i ne gazi ubogoga na sudu. 
\par 23 Jer će Jahve parbiti parbu njihovu i otet će život onima koji ga njima otimlju. 
\par 24 Ne druži se sa srditim i ne idi s čovjekom jedljivim 
\par 25 da se ne bi privikao na staze njegove i namjestio zamku duši svojoj. 
\par 26 Ne budi među onima koji daju ruku, koji jamče za dugove: 
\par 27 ako nemaš čime nadoknaditi, zašto da ti oduzmu i postelju ispod tebe? 
\par 28 Ne pomiči prastare međe koju su postavili oci tvoji. 
\par 29 Jesi li vidio čovjeka vična poslu svom: takav ima pristup kraljevima i ne služi prostacima. 


\chapter{23}

\par 1 Kad sjedneš blagovati s moćnikom. dobro pazi što je pred tobom; 
\par 2 stavljaš nož sebi pod grlo ako si proždrljivac; 
\par 3 ne poželi slastica njegovih jer su jelo prijevarno. 
\par 4 Ne trudi se stjecati bogatstvo; okani se takve misli; 
\par 5 usmjeriš li oči prema njemu, njega već nema jer načini sebi krila kao orao i odleti u nebo. 
\par 6 Ne jedi jela zavidnikova, ne čezni za slasticama njegovim, 
\par 7 jer on je onakav kako u sebi misli: "Jedi i pij", veli ti, ali mu srce nije s tobom. 
\par 8 Zalogaj koji si pojeo izbljuvat ćeš, uzalud ćeš prosut' svoje ljupke riječi. 
\par 9 Pred bezumnikom nemoj govoriti jer prezire tvoje umne riječi. 
\par 10 Ne pomiči prastare međe i ne prodiri u polje siročadi, 
\par 11 jer je moćan njihov osvetnik: branit će njihovo pravo protiv tebe. 
\par 12 Obrati pouci srce svoje i uho svoje riječima mudrim. 
\par 13 Ne uskraćuj djetetu opomene, jer, udariš li ga šibom, neće umrijeti: 
\par 14 biješ ga šibom, ali mu dušu iz Podzemlja izbavljaš. 
\par 15 Sine moj, kad ti je mudro srce, i ja se od srca veselim; 
\par 16 i kliče sva nutrina moja kad ti usne govore što je pravo. 
\par 17 Neka ti srce ne zavidi grešnicima, nego neka ti uvijek bude u strahu Gospodnjem, 
\par 18 jer imat ćeš budućnost i tvoja nada neće propasti. 
\par 19 Slušaj, sine moj, i mudar budi i ravnim putem vodi srce svoje. 
\par 20 Ne druži se s vinopijama ni sa žderačima mesa, 
\par 21 jer pijanica i izjelica osiromaše i pospanac se oblači u krpe. 
\par 22 Slušaj svoga oca, svoga roditelja, i ne prezri majku kad ostari. 
\par 23 Pribavi istinu i ne prodaji je, steci mudrost, pouku i razbor. 
\par 24 Radovat će se otac pravednikov, i roditelj će se mudroga veseliti. 
\par 25 Neka se veseli otac tvoj i majka tvoja, i neka se raduje roditeljka tvoja. 
\par 26 Daj mi, sine moj, srce svoje, i neka oči tvoje raduju putovi moji. 
\par 27 Jer bludnica je jama duboka i tuđinka tijesan zdenac. 
\par 28 Ona i vreba u zasjedi kao lupež i uvećava broj bezbožnika među ljudima. 
\par 29 Komu: ah? komu: jao? komu: svađe? komu: uzdasi? komu: rane nizašto? komu: zamućene oči? 
\par 30 Onima što kasno sjede kod vina, koji su došli kušati vino začinjeno. 
\par 31 Ne gledaj na vino kad rujno iskri, kad se u čaši svjetlucavo prelijeva: pije se tako glatko, 
\par 32 a na kraju ujeda kao zmija i žaca kao guja ljutica. 
\par 33 Oči će ti gledati tlapnje i srce govoriti ludosti. 
\par 34 I bit će ti kao da ležiš na pučini morskoj ili kao da ležiš navrh jarbola. 
\par 35 "Izbiše me, ali me ne zabolje; istukoše me, ali ne osjetih; kad se otrijeznim, još ću tražiti." 


\chapter{24}

\par 1 Ne zavidi opakim ljudima niti želi da budeš s njima. 
\par 2 Jer im srce smišlja nasilje i usne govore o nedjelu. 
\par 3 Mudrošću se zida kuća i razborom utvrđuje, 
\par 4 i po znanju se pune klijeti svakim blagom dragocjenim i ljupkim. 
\par 5 Bolji je mudar od jakoga i čovjek razuman od silne ljudine. 
\par 6 Jer s promišljanjem se ide u boj i pobjeda je u mnoštvu savjetnika. 
\par 7 Previsoka je bezumnomu mudrost: zato na sudu ne otvara usta svojih! 
\par 8 Tko smišlja zlo zove se učitelj podmukli. 
\par 9 Ludost samo grijeh snuje, i podrugljivac je mrzak ljudima. 
\par 10 Kloneš li u dan bijede, bijedna je tvoja snaga. 
\par 11 Izbavi one koje vode u smrt; i spasavaj one koji posrćući idu na stratište. 
\par 12 Ako kažeš: "Nismo za to znali", ne razumije li onaj koji ispituje srca? I ne znade li onaj koji ti čuva dušu? I ne plaća li on svakomu po njegovim djelima? 
\par 13 Jedi med, sine moj, jer je dobar, i saće je slatko nepcu tvome. 
\par 14 Takva je, znaj, i mudrost tvojoj duši: ako je nađeš, našao si budućnost i nada tvoja neće propasti. 
\par 15 Ne postavljaj, opaki, zasjede stanu pravednikovu, ne čini nasilja boravištu njegovu; 
\par 16 jer padne li pravednik i sedam puta, on ustaje, a opaki propadaju u nesreći. 
\par 17 Ne veseli se kad padne neprijatelj tvoj i ne kliči srcem kada on posrće, 
\par 18 da ne bi vidio Jahve i za zlo uzeo i obratio srdžbu svoju od njega. 
\par 19 Nemoj se srditi zbog zločinaca, nemoj zavidjeti opakima, 
\par 20 jer zao čovjek nema budućnosti, svjetiljka opakih gasi se. 
\par 21 Boj se Jahve, sine moj, i kralja: i ne buni se ni protiv jednoga ni protiv drugoga. 
\par 22 Jer iznenada provaljuje nesreća njihova i tko zna kad će doći propast njihova. 
\par 23 I ovo je od mudraca: Ne valja biti pristran na sudu. 
\par 24 Tko opakomu veli: "Pravedan si", proklinju ga narodi i kunu puci; 
\par 25 a oni koji ga ukore nalaze zadovoljstvo, i na njih dolazi blagoslov sreće. 
\par 26 U usta ljubi tko odgovara pošteno. 
\par 27 Svrši svoj posao vani i uredi svoje polje, potom i kuću svoju zidaj. 
\par 28 Ne svjedoči lažno na bližnjega svoga: zar ćeš varati usnama svojim? 
\par 29 Ne reci: "Kako je on meni učinio, tako ću i ja njemu; platit ću tom čovjeku po djelu njegovu!" 
\par 30 Prolazio sam mimo polje nekog lijenčine i mimo vinograd nekog luđaka, 
\par 31 i gle, sve bijaše zaraslo u koprive, i sve pokrio čkalj, i kamena ograda porušena. 
\par 32 Vidjeh to i pohranih u srcu, promotrih i uzeh pouku: 
\par 33 "Još malo odspavaj, još malo odrijemaj, još malo podvij ruke za počinak, 
\par 34 i doći će tvoje siromaštvo kao skitač i oskudica kao oružanik!" 


\chapter{25}

\par 1 I ovo su mudre izreke Salomonove; sabrali ih ljudi Ezekije, kralja judejskog. 
\par 2 Slava je Božja sakrivati stvar, a slava kraljevska istraživati je. 
\par 3 Neistražljivo je nebo u visinu, zemlja u dubinu i srce kraljevsko. 
\par 4 Ukloni trosku od srebra, i uspjet će posao zlataru. 
\par 5 Ukloni opakoga ispred kralja, i utvrdit će se pravicom prijestol njegov. 
\par 6 Ne veličaj se pred kraljem i ne sjedaj na mjesto velikaško, 
\par 7 jer je bolje da ti se kaže: "Popni se gore" nego da te ponize pred odličnikom. 
\par 8 Što su ti oči vidjele ne iznosi prebrzo na raspru; jer što ćeš učiniti na koncu kad te opovrgne bližnji tvoj? 
\par 9 Kad si u parbi s bližnjim svojim, ne otkrivaj tuđe tajne, 
\par 10 da te ne izgrdi tko čuje i da ti se kleveta ne vrati. 
\par 11 Riječi kazane u pravo vrijeme zlatne su jabuke u srebrnim posudama. 
\par 12 Mudrac koji kori uhu je poslušnu zlatan prsten i ogrlica od tanka zlata. 
\par 13 Vjeran je glasnik onomu tko ga šalje kao ledena studen u doba žetve: on krijepi dušu svoga gospodara. 
\par 14 Tko se diči lažljivim darom, on je kao oblak i vjetar bez kiše. 
\par 15 Strpljivošću se ublažava sudac, mek jezik i kosti lomi. 
\par 16 Kad naiđeš na med, jedi umjereno, kako se ne bi prejeo i pojedeno izbljuvao. 
\par 17 Rijetko zalazi u kuću bližnjega svoga, da te se ne zasiti i ne zamrzi na te. 
\par 18 Čovjek koji svjedoči lažno na bližnjega svoga on je kao bojni malj i mač i oštra strijela. 
\par 19 Uzdanje u bezbožnika na dan nevolje - krnjav je zub i noga klecava. 
\par 20 Kao onaj koji skida haljinu u zimski dan ili ocat lije na ranu, takav je onaj tko pjeva pjesmu turobnu srcu. 
\par 21 Ako je gladan neprijatelj tvoj, nahrani ga kruhom, i ako je žedan, napoji ga vodom. 
\par 22 Jer mu zgrćeš ugljevlje na glavu i Jahve će ti platiti. 
\par 23 Sjeverni vjetar donosi dažd, a himben jezik srdito lice. 
\par 24 Bolje je stanovati pod rubom krova nego u zajedničkoj kući sa ženom svadljivom. 
\par 25 Kao studena voda žednu grlu, takva je dobra vijest iz zemlje daleke. 
\par 26 Kao zatrpan izvor i vrelo zamućeno, takav je pravednik koji kleca pred opakim. 
\par 27 Jesti mnogo meda nije dobro niti tražiti pretjerane časti. 
\par 28 Grad razvaljen i bez zidova - takav je čovjek koji nema vlasti nad sobom. 


\chapter{26}

\par 1 Kao snijeg ljeti ili kiša o žetvi, tako pristaju počasti bezumnomu. 
\par 2 Kao vrabac kad prhne i lastavica kad odleti, tako se i bezrazložna kletva ne ispunja. 
\par 3 Bič konju, uzda magarcu, a šiba leđima bezumnika. 
\par 4 Ne odgovaraj bezumniku po njegovoj ludosti, da mu i sam ne postaneš jednak. 
\par 5 Odgovori bezumniku po ludosti njegovoj, da se ne bi učinio sam sebi mudar. 
\par 6 Odsijeca noge sebi i gorčinu pije tko po bezumnom poruke šalje. 
\par 7 Klecava bedra u hromoga - mudra je izreka u ustima bezumničkim. 
\par 8 Kamen za praćku vezuje tko bezumnom iskazuje čast. 
\par 9 Trnovita grana u ruci pijanice: mudra izreka u ustima bezumnika. 
\par 10 Strijelac koji ranjava sve prolaznike: takav je onaj tko unajmljuje bezumnika. 
\par 11 Bezumnik se vraća svojoj ludosti kao što se pas vraća na svoju bljuvotinu. 
\par 12 Vidiš li čovjeka koji se sam sebi mudrim čini? Znaj, i od bezumnika ima više nade nego od njega! 
\par 13 Lijenčina veli: "Zvijer je na putu, i lav je na ulicama." 
\par 14 Kao što se vrata okreću na stožerima svojim, tako i lijenčina na postelji svojoj. 
\par 15 Lijenčina umače ruku u zdjelu, ali je ne može prinijeti ustima. 
\par 16 Lijenčina se čini sebi mudrijim od sedmorice koji umno odgovaraju. 
\par 17 Psa za uši hvata tko se, u prolazu, umiješa u raspru koja ga se ne tiče. 
\par 18 Kao bjesomučnik koji baca zublje, strelice i sije smrt, 
\par 19 takav je čovjek koji vara bližnjega svoga i veli: "Samo se našalih." 
\par 20 Kad nestane drva, oganj se gasi, i kad više nema klevetnika, prestaje svađa. 
\par 21 Ugljen je za žeravnicu i drvo za oganj, a svadljivac da raspaljuje svađu. 
\par 22 Klevetnikove su riječi kao slastice: spuštaju se u dno utrobe. 
\par 23 Srebrna gleđa preko zemljana suđa: laskave usne i opako srce. 
\par 24 Mrzitelj hini usnama svojim, a u sebi nosi prijevaru; 
\par 25 ne vjeruj mu kad ljupkim glasom govori, jer u srcu mu je sedam grdila; 
\par 26 ako himbom skriva mržnju, njegova će se opačina otkriti na zboru. 
\par 27 Tko jamu kopa, sam u nju pada, i tko kamen valja, na njega se prevaljuje. 
\par 28 Lažljiv jezik mrzi svoje žrtve, laskava usta propast spremaju. 


\chapter{27}

\par 1 Ne hvali se danom sutrašnjim jer ne znaš što danas može donijeti. 
\par 2 Neka te hvali drugi, a ne tvoja usta, tuđinac, a ne tvoje usne. 
\par 3 Težak je kamen i pijesak je težak, ali je od obojega teži bezumnikov bijes. 
\par 4 Jarost je okrutna i srdžba žestoka a tko će odoljeti ljubomoru? 
\par 5 Bolji je javni ukor nego lažna ljubav. 
\par 6 Čestiti su udarci prijateljevi, a lažni poljupci neprijateljevi. 
\par 7 Sito grlo prezire i med samotok, a gladnu je i sve gorko - slatko. 
\par 8 Kao ptica daleko od gnijezda svog, takav je čovjek daleko od svojeg zavičaja. 
\par 9 Kao što ulje i kad vesele srce, tako i slatkoća prijateljstva tješi dušu. 
\par 10 Ne ostavljaj prijatelja svoga ni prijatelja očeva i ne dolazi u kuću bratovu kad si u nesreći; bolji je susjed blizu nego brat daleko. 
\par 11 Budi mudar, sine moj, i obraduj mi srce da mogu odgovoriti onome koji me grdi. 
\par 12 Pametan čovjek opazi zlo i skrije se, a glupaci idu bezbrižno i trpe kaznu. 
\par 13 Uzmi haljinu onomu tko je jamčio za drugoga i oplijeni ga mjesto tuđinca. 
\par 14 Tko pozdravlja svoga prijatelja naglas, a rano ujutro, prima mu se blagoslov za kletvu. 
\par 15 Streha što prokišnjava za žestoke kiše i svadljiva žena - jedno su te isto. 
\par 16 Tko nju zaustavlja, zaustavlja vjetar i desnicom hvata ulje. 
\par 17 Željezo se željezom oštri i čovjek oštri jedan drugoga. 
\par 18 Tko čuva smokvu, jede od njena ploda, i tko čuva svoga gospodara, poštiva se. 
\par 19 Kao što se u vodi različito odražava lice od lica, tako i u srcu čovjek od čovjeka. 
\par 20 Carstvo Smrti i Propast ne mogu se zasititi, tako ni oči čovječje. 
\par 21 Taljika je za srebro i peć za zlato, a čovjek se poznaje po ustima koja ga hvale. 
\par 22 Da bezumnika stučeš tučkom u stupi, ne bi ga ostavila ludost njegova. 
\par 23 Brižno pazi na stoku svoju i srcem se brini o stadima, 
\par 24 jer blago ne traje dovijeka; i baštini li se kruna od koljena do koljena? 
\par 25 Kad trava nikne i zelen se pokaže i bilje se kupi planinsko, 
\par 26 tad su ti janjci za odijelo i jarci za kupovinu polja; 
\par 27 tad imaš izobilje kozjega mlijeka sebi za jelo, i za hranu kući svojoj i za prehranu sluškinjama svojim. 


\chapter{28}

\par 1 Opaki bježe i kad ih nitko ne progoni,  a pravednici su neustrašivi kao mladi lav. 
\par 2 Kad se u zemlji griješi, mnogi su joj knezovi, a s čovjekom razumnim i umnim uprava je postojana. 
\par 3 Čovjek opak koji tlači ubogoga - kiša je razorna poslije koje kruha nema. 
\par 4 Koji zapuštaju Zakon, veličaju opake, a koji se drže Zakona, protive im se. 
\par 5 Zli ljudi ne razumiju pravice, a koji traže Jahvu, razumiju sve. 
\par 6 Bolji je siromah koji živi bezazleno nego bogataš koji kroči krivim putem. 
\par 7 Tko se drži Zakona, razuman je sin, a tko se druži s izbjeglicama, sramoti oca svoga. 
\par 8 Tko umnožava bogatstvo svoje lihvom i pridom, skuplja ga onomu tko je milostiv ubogima. 
\par 9 Tko uklanja uho svoje da ne sluša Zakona, i molitva je njegova mrska. 
\par 10 Tko zavodi poštene na put zao, past će u jamu svoju, a pošteni će baštiniti sreću. 
\par 11 Bogat se čovjek čini sebi mudrim, ali će ga razuman siromah raskrinkati. 
\par 12 Velika je slava kad se raduju pravednici, a kad se dižu opaki, ljudi se kriju. 
\par 13 Tko skriva svoje grijehe, nema sreće, a tko ih ispovijeda i odriče ih se, milost nalazi. 
\par 14 Blago čovjeku uvijek bojaznu, jer čovjek okorjela srca zapada u nesreću. 
\par 15 Lav koji riče i gladan medvjed: takav je opak vladalac siromašnu narodu. 
\par 16 Nerazuman knez čini mnoga nasilja, a koji mrzi lakomost, dugo živi. 
\par 17 Onaj koga tišti krvna krivica, do groba bježi: ne zaustavljajte ga. 
\par 18 Spasava se tko živi pravedno, tko se koleba između dva puta, propada na jednom od njih. 
\par 19 Tko obrađuje svoju zemlju, nasitit će se kruha, a tko trči za tlapnjama, nasitit će se siromaštva. 
\par 20 Čestit čovjek stječe blagoslov, a tko hrli za bogatstvom, ne ostaje bez kazne. 
\par 21 Ne valja biti pristran na sudu, jer i za zalogaj kruha čovjek čini zlo. 
\par 22 Pohlepnik hrli za bogatstvom, a ne zna da će ga stići oskudica. 
\par 23 Tko kori čovjeka, nalazi poslije veću milost nego onaj koji laska jezikom. 
\par 24 Tko pljačka oca svoga i majku svoju i veli: "Nije grijeh", drug je razbojniku. 
\par 25 Lakomac zameće svađu, a tko se uzda u Jahvu, uspjet će. 
\par 26 Bezuman je tko se uzda u svoje srce, a spasava se tko živi mudro. 
\par 27 Tko daje siromahu, ne trpi oskudicu; a tko odvraća oči svoje, bit će proklet. 
\par 28 Kad se dižu opaki, ljudi se kriju, a kad propadaju, tad se množe pravednici. 


\chapter{29}

\par 1 Čovjek koji, po opomeni, ostaje tvrdoglav,  u tren će se slomiti, i neće mu biti spasa. 
\par 2 Narod se veseli kad se množe pravednici, a puk uzdiše kad zavlada opaki. 
\par 3 Čovjek koji ljubi mudrost, veseli oca svoga, a koji se druži s bludnicama, rasipa imetak. 
\par 4 Kralj pravicom održava državu, a ruši je čovjek koji nameće daće. 
\par 5 Čovjek koji laska bližnjemu svome razapinje mrežu stopama njegovim. 
\par 6 U grijehu je zamka zlu čovjeku, a pravednik likuje i veseli se. 
\par 7 Pravednik razumije pravo malenih, a opaki ne shvaća spoznaju. 
\par 8 Podsmjevači uzbunjuju grad, a mudri stišavaju srdžbu. 
\par 9 Kad se mudrac parbi s bezumnikom, il' se srdio, il' se smijao, svejednako mira nema. 
\par 10 Krvopije mrze poštenoga, a pravednici mu se za život brinu. 
\par 11 Bezumnik izlijeva sav svoj gnjev, a mudrac susteže svoju srdžbu. 
\par 12 Ako vladalac posluša riječ lažljivu, sve mu sluge postaju opake. 
\par 13 Siromah se i gulikoža susreću: Jahve obojici prosvjetljuje oči. 
\par 14 Kralj koji sudi siromasima po istini ima prijesto čvrst dovijeka. 
\par 15 Šiba i ukor podaruju mudrost, a razuzdan mladić sramoti majku svoju. 
\par 16 Kad se množe opaki, množi se i grijeh, ali pravednici promatraju propast njihovu. 
\par 17 Ukori sina svoga, i zadovoljit će te i dati radost duši tvojoj. 
\par 18 Kad objave nema, narod se razuzda, a blago onome tko se drži Zakona! 
\par 19 Samim se riječima sluga ne popravlja, jer se ne pokorava iako umom shvaća. 
\par 20 Jesi li vidio čovjeka brza na riječima? I bezumnik ima više nade nego on. 
\par 21 Tko mazi slugu svoga od djetinjstva bit će mu poslije neposlušan. 
\par 22 Gnjevljiv čovjek zameće svađu, a naprasit čovjek počini mnoge grijehe. 
\par 23 Oholost ponizuje čovjeka, a ponizan duhom postiže časti. 
\par 24 Tko s lupežom plijen dijeli, mrzi sebe samog: čuje proklinjanje i ništa ne otkriva. 
\par 25 Strah čovjeku postavlja zamku, a tko se uzda u Jahvu, nalazi okrilje. 
\par 26 Mnogi traže milost vladaočevu, ali Jahve dijeli pravdu svakome. 
\par 27 Nepravednik je mrzak pravednicima, a pravednik je mrzak opakima. 


\chapter{30}

\par 1 Riječi Agura, sina Jakeova, iz Mase; proročanstvo njegovo  za Itiela, za Itiela i Ukala. 
\par 2 Da, preglup sam da bih bio čovjek i nemam razbora čovječjeg. 
\par 3 Ne stekoh mudrosti i ne poznajem znanosti svetih! 
\par 4 Tko uzađe na nebo i siđe? Tko uhvati vjetar u šake svoje? Tko sabra vode u plašt svoj? Tko postavi krajeve zemaljske? Kako se zove i kako mu se zove sin? Znaš li? 
\par 5 Svaka je Božja riječ prokušana, štit onima koji se u nj uzdaju. 
\par 6 Ne dodaji ništa njegovim riječima, da te ne prekori i ne smatra lažljivim. 
\par 7 Za dvoje te molim, ne uskrati mi, dok ne umrem: 
\par 8 udalji od mene licemjernu i lažnu riječ; ne daj mi siromaštva ni bogatstva: hrani me kruhom mojim dostatnim; 
\par 9 inače bih, presitivši se, zatajio tebe i rekao: "Tko je Jahve?" Ili bih, osiromašivši, krao i oskvrnio ime Boga svojega. 
\par 10 Ne klevetaj sluge gospodaru njegovu, jer bi te mogao kleti i ti morao okajati. 
\par 11 Ima izrod koji kune oca svoga i ne blagoslivlje majke svoje! 
\par 12 Izrod koji za se misli da je čist, a od kala svojeg nije opran! 
\par 13 Izrod uznositih očiju koji visoko diže svoje trepavice! 
\par 14 Izrod komu su zubi mačevi i očnjaci noževi da proždiru nesretnike na zemlji i siromahe među ljudima! 
\par 15 Pijavica ima dvije kćeri: "Daj! Daj!" Postoje tri stvari nezasitne i četiri koje ne kažu: "Dosta!" 
\par 16 Carstvo smrti, jalova utroba, zemlja nikad gasna vode i vatra koja nikad ne kaže: "Dosta!" 
\par 17 Oko koje se ruga ocu i odriče posluh majci iskljuvat će potočni gavrani i izjesti mladi orlovi. 
\par 18 Troje mi je nedokučivo, a četvrto ne razumijem: 
\par 19 put orlov po nebu, put zmijin po stijeni, put lađin posred mora i put muškarčev djevojci. 
\par 20 Takav je put preljubnice: najede se, obriše usta i veli: "Nisam sagriješila." 
\par 21 Od troga se zemlja ljulja, a četvrtoga ne može podnijeti: 
\par 22 od roba kad postane kralj i kad se prostak kruha nasiti, 
\par 23 od puštenice kad se uda i sluškinje kad istisne svoju gospodaricu. 
\par 24 Četvero je maleno na zemlji, ali mudrije od mudraca: 
\par 25 mravi, nejaki stvorovi, koji sebi ljeti spremaju hranu; 
\par 26 jazavci, stvorovi bez moći, što u stijeni grade sebi stan; 
\par 27 skakavci, koji nemaju kralja, a svi idu u poretku; 
\par 28 gušter, što se rukama hvata, a prodire u kraljevske palače. 
\par 29 Troje ima lijep korak, a četvero lijepo hodi: 
\par 30 lav, junak među zvijerima, koji ni pred kim ne uzmiče; 
\par 31 pijetao što se odvažno šeće među kokošima; jarac koji vodi stado; i kralj sa svojom vojskom. 
\par 32 Ako si ludovao oholeći se ili to svjesno činio, stavi ruku na usta. 
\par 33 Kad se mlijeko metÄe, izlazi maslac; kad se nos pritisne, poteče krv; kad se srdžba potisne, dobiva se spor. 



\chapter{31}

\par 1 Riječ Lemuela, kralja Mase, kojima ga je učila majka njegova. 
\par 2 Ne, sine moj! Ne, sine srca mog! Ne, sine zavjeta mojih! 
\par 3 Ne daj snage svoje ženama ni putova svojih zatiračima kraljeva. 
\par 4 Nije za kraljeve, Lemuele, ne pristaje kraljevima vino piti, ni glavarima piće opojno, 
\par 5 da u piću ne zaborave zakona i prevrnu pravo nevoljnicima. 
\par 6 Dajte žestoko piće onomu koji će propasti i vino čovjeku komu je gorčina u duši: 
\par 7 on će piti i zaboraviti svoju bijedu i neće se više sjećati svoje nevolje. 
\par 8 Otvaraj usta svoja za nijemoga i za pravo sviju nesretnika što propadaju. 
\par 9 Otvaraj usta svoja, sudi pravedno i pribavi pravo siromahu i nevoljniku. 
\par 10 Tko će naći ženu vrsnu? Više vrijedi ona nego biserje. 
\par 11 Muževljevo se srce uzda u nju i blagom neće oskudijevati. 
\par 12 Ona mu čini dobro, a ne zlo, u sve dane vijeka svojeg. 
\par 13 Pribavlja vunu i lan i vješto radi rukama marnim. 
\par 14 Ona je kao lađa trgovačka: izdaleka donosi kruh svoj. 
\par 15 Još za noći ona ustaje, hrani svoje ukućane i određuje posao sluškinjama svojim. 
\par 16 Opazi li polje, kupi ga; plodom svojih ruku sadi vinograd. 
\par 17 Opasuje snagom bedra svoja i živo miče rukama. 
\par 18 Vidi kako joj posao napreduje: noću joj se ne gasi svjetiljka. 
\par 19 Rukama se maša preslice i prstima drži vreteno. 
\par 20 Siromahu dlan svoj otvara, ruke pruža nevoljnicima. 
\par 21 Ne boji se snijega za svoje ukućane, jer sva čeljad ima po dvoje haljine. 
\par 22 Sama sebi šije pokrivače, odijeva se lanom i purpurom. 
\par 23 Muž joj je slavan na Vratima, gdje sjedi sa starješinama zemaljskim. 
\par 24 Platno tka i prodaje ga i pojase daje trgovcu. 
\par 25 Odjevena je snagom i dostojanstvom, pa se smije danu budućem. 
\par 26 Svoja usta mudro otvara i pobožan joj je nauk na jeziku. 
\par 27 Na vladanje pazi ukućana i ne jede kruha besposlice. 
\par 28 Sinovi njezini podižu se i sretnom je nazivaju, i muž njezin hvali je: 
\par 29 "Mnoge su žene bile vrsne, ali ti ih sve nadmašuješ." 
\par 30 Lažna je ljupkost, tašta je ljepota: žena sa strahom Gospodnjim zaslužuje hvalu. 
\par 31 Plod joj dajte ruku njezinih i neka je na Vratima hvale djela njezina! 





\end{document}