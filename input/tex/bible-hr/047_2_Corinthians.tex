\begin{document}

\title{2 Korinčanima}


\chapter{1}

\par 1 Pavao, po volji Božjoj apostol Krista Isusa, i brat Timotej:  Crkvi Božjoj u Korintu sa svima svetima u svoj Ahaji. 
\par 2 Milost  vam i mir od Boga, Oca našega, i Gospodina Isusa Krista! 
\par 3 Blagoslovljen Bog i Otac Gospodina našega Isusa Krista, Otac milosrđa i Bog svake utjehe! 
\par 4 On nas tješi u svakoj našoj  nevolji da bismo i mi sve koji su u nevolji mogli tješiti onom  utjehom kojom nas same tješi Bog. 
\par 5 Jer kao što su obilate patnje  Kristove u nama, tako je po Kristu obilata i utjeha naša. 
\par 6 Bili  mi nevoljama pritisnuti za vašu je to utjehu i spasenje; bili  utješeni, za vašu je utjehu - djelotvornu: da strpljivo podnesete  iste patnje koje i mi podnosimo. 
\par 7 I tako je stamena nada naša  o vama jer znamo: kao što ste zajedničari patnja tako ste i utjehe. 
\par 8 Ne bismo doista htjeli, braćo, da ne znate za nevolju koja  nas je snašla u Aziji. Bijasmo prekomjerno, preko snage, opterećeni  te smo već strepili i za život. 
\par 9 Ali u sebi prihvatismo i smrtnu  osudu da se ne bismo uzdali u same sebe, nego u Boga koji uskrisuje  mrtve. 
\par 10 On nas je od takve smrti izbavio i izbavit će nas;  u njega se uzdamo, on će nas i dalje izbavljati. 
\par 11 A i vi ćete  nam pomagati molitvom da bi mnogi za nas zahvaljivali Bogu na  milosti koja nam je darovana posredovanjem mnogih. 
\par 12 A ovo je naša slava: svjedočansto naše savjesti da smo  u svijetu živjeli - osobito prema vama - u svetosti i iskrenosti  Božjoj, ne u mudrosti tjelesnoj, nego u Božjoj milosti. 
\par 13 Ta  i ne pišemo vam drugo doli ovo što čitate i razumijete; a nadam  se da ćete i do kraja razumjeti, 
\par 14 kao što nas djelomično i  razumjeste: da smo mi vaša slava kao i vi naša u Dan Gospodina  našega Isusa. 
\par 15 U tom uvjerenju namjeravao sam najprije doći k vama 
\par 16 i  preko vas prijeći u Makedoniju pa se opet, da biste imali i drugu  milost, iz Makedonije vratiti k vama da me vi otpratite u Judeju. 
\par 17 Pa jesam li možda bio lakomislen kad sam to namjeravao? Ili  što namjeravam, po tijelu namjeravam te je u mene istodobno "Da, da!" i "Ne, ne!"? 
\par 18 Bog je svjedok: naša riječ vama nije "Da!"  i "Ne!" 
\par 19 jer Sin Božji, Isus Krist, koga mi - ja i Silvan  i Timotej - vama navijestismo nije bio "Da!" i "Ne!" nego u njemu  bijaše "Da!". 
\par 20 Doista, sva obećanja Božja u njemu su "Da!".  I stoga po njemu i naš "Amen!" Bogu na slavu! 
\par 21 A Bog je onaj  koji nas zajedno s vama utvrđuje za Krista; on nas i pomaza, 
\par 22 on nas i zapečati i u srca naša dade zalog - Duha. 
\par 23 A ja prizivljem Boga za svjedoka: duše mi, da vas poštedim, nisam više dolazio u Korint. 
\par 24 Ta mi nismo gospodari vaše  vjere, nego suradnici vaše radosti. Ta u vjeri ste postojani. 


\chapter{2}

\par 1 Odlučih dakle u sebi da neću k vama opet sa žalošću. 
\par 2 Jer  ako ja vas ražalostim, a tko će mene obradovati ako ne onaj koga  ja žalostim? 
\par 3 Zato vam to i napisah da me, kada dođem, ne ražaloste  oni koji bi mi imali biti na radost. Uzdam se doista u sve vas, da je moja radost - radost svih vas. 
\par 4 Pisah vam uistinu uz  mnoge suze, iz velike nevolje i tjeskobe srca, ne da se ražalostite, nego da upoznate moju preveliku ljubav prema vama. 
\par 5 Ako me tko ražalostio, nije ražalostio mene, nego u neku  ruku - da ne pretjeram - sve vas. 
\par 6 Dosta je takvu ona kazna  od većine 
\par 7 pa ga vi radije pomilujte i utješite da ga pretjerana  žalost ne shrva. 
\par 8 Zato vas molim, iskažite mu ljubav. 
\par 9 Ta  zato vam i pisah da vidim jeste li prokušani, jeste li u svemu  poslušni. 
\par 10 Komu dakle vi što oprostite, tomu i ja; jer i ja, ako kome što oprostih, oprostih poradi vas - pred Kristom, 
\par 11 da  nas ne nadmudri Sotona. Ta znamo njegove namjere! 
\par 12 Kada dođoh u Troadu poradi evanđelja Kristova, premda  mi se otvoriše vrata u Gospodinu, 
\par 13 ne bijaše mi duši spokoja  što ne nađoh Tita, brata svoga; oprostih se stoga s njima i pođoh  u Makedoniju. 
\par 14 Ali hvala Bogu koji nas u Kristu uvijek proslavlja  te širi po nama na svakome mjestu miris svoga spoznanja. 
\par 15 Da, Kristov smo miomiris Bogu i među onima koji se spasavaju i među  onima koji propadaju: 
\par 16 ovima miris iz smrti za smrt, onima  miris iz života za život. A tko je za to podoban? 
\par 17 Uistinu, mi nismo kao mnogi koji trguju riječju Božjom, nego iskreno  - kao od Boga pred Bogom - u Kristu govorimo. 


\chapter{3}

\par 1 Počinjemo li opet sami sebe preporučivati? Ili trebamo li,  kao neki, preporučna pisma na vas ili od vas? 
\par 2 Vi ste pismo  naše, upisano u srcima vašim; znaju ga i čitaju svi ljudi. 
\par 3 Vi  ste, očito, pismo Kristovo kojemu mi poslužismo, napisano ne  crnilom, nego Duhom Boga živoga; ne na pločama kamenim,  nego na pločama od mesa, u srcima. 
\par 4 Takvo pouzdanje imamo po Kristu u Boga. 
\par 5 Ne kao da smo  sami sobom, kao od sebe, sposobni što pomisliti, nego naša je  sposobnost od Boga. 
\par 6 On nas osposobi za poslužitelje novoga  Saveza, ne slova, nego Duha; jer slovo ubija, a Duh oživljuje. 
\par 7 Pa ako je smrtonosna služba, slovima uklesana u kamenju, bila  tako slavna da sinovi Izraelovi nisu mogli pogledati u lice Mojsijevo  zbog prolazne slave lica njegova, 
\par 8 koliko li će slavnija biti  služba Duha. 
\par 9 Jer ako je služba osude bila slavna, mnogo je  slavnija služba pravednosti. 
\par 10 I zbilja, nije ni bilo proslavljeno  ono što je u toj mjeri proslavljeno, ako se usporedi s uzvišenijom  slavom. 
\par 11 Jer ako je ono prolazno bilo slavno, mnogo je slavnije  ovo što ostaje. 
\par 12 Imajući dakle takvo pouzdanje, nastupamo sa svom otvorenošću, 
\par 13 a ne kao Mojsije koji je stavljao prijevjes na lice da sinovi  Izraelovi ne vide svršetak prolaznoga. 
\par 14 Ali otvrdnu im pamet.  Doista, do dana današnjega zastire taj prijevjes čitanje Staroga  zavjeta: nije im otkriveno da je u Kristu prestao. 
\par 15 Naprotiv, kad god se čita Mojsije, do danas prijevjes zastire srce njihovo. 
\par 16 Ali kad se Izrael obrati Gospodinu, skinut će se prijevjes. 
\par 17 Gospodin je Duh, a gdje je Duh Gospodnji, ondje je sloboda. 
\par 18 A svi mi, koji otkrivenim licem odrazujemo slavu Gospodnju, po Duhu se Gospodnjem preobražavamo u istu sliku - iz slave  u slavu. 


\chapter{4}

\par 1 Zato, budući da po milosrđu imamo ovu službu, ne malakšemo. 
\par 2 Ali odrekosmo se sramotnoga prikrivanja: ne nastupamo lukavo  niti izopačujemo riječ Božju, nego se objavljivanjem istine preporučujemo  svakoj savjesti ljudskoj pred Bogom. 
\par 3 Ako je i zastrto evanđelje  naše, u onima je zastrto koji propadaju: 
\par 4 u onima kojima bog  ovoga svijeta oslijepi pameti nevjerničke da ne zasvijetli svjetlost  evanđelja slave Krista koji je slika Božja. 
\par 5 Jer ne propovijedamo  same sebe, nego Krista Isusa Gospodinom, a sebe slugama vašim  poradi Isusa. 
\par 6 Ta Bog koji reče: Neka iz tame svjetlost  zasine!, on zasvijetli u srcima našim da nam spoznanje slave  Božje zasvijetli na licu Kristovu. 
\par 7 To pak blago imamo u glinenim posudama da izvanredna ona  snaga bude očito Božja, a ne od nas. 
\par 8 U svemu pritisnuti, ali  ne pritiješnjeni; dvoumeći, ali ne zdvajajući; 
\par 9 progonjeni, ali ne napušteni; obarani, ali ne oboreni - 
\par 10 uvijek umiranje  Isusovo u tijelu pronosimo da se i život Isusov u tijelu našem  očituje. 
\par 11 Doista, mi se živi uvijek na smrt predajemo poradi  Isusa da se i život Isusov očituje u našem smrtnom tijelu. 
\par 12 Tako  smrt djeluje u nama, život u vama. 
\par 13 A budući da imamo isti  duh vjere kao što je pisano: Uzvjerovah, zato besjedim,  i mi vjerujemo pa zato i besjedimo. 
\par 14 Ta znamo: onaj koji je  uskrisio Gospodina Isusa i nas će s Isusom uskrisiti i zajedno  s vama uza se postaviti. 
\par 15 A sve je to za vas: da milost -  umnožena - zahvaljivanjem mnogih izobiluje Bogu na slavu. 
\par 16 Zato ne malakšemo. Naprotiv, ako se naš izvanji čovjek  i raspada, nutarnji se iz dana u dan obnavlja. 
\par 17 Ta ova malenkost  naše časovite nevolje donosi nam obilato, sve obilatije, breme  vječne slave 
\par 18 jer nama nije do vidljivog nego do nevidljivog:  ta vidljivo je privremeno, a nevidljivo - vječno. 


\chapter{5}

\par 1 Znamo doista: ako se razruši naš zemaljski dom, šator, imamo  zdanje od Boga, dom nerukotvoren, vječan na nebesima. 
\par 2 U ovome  doista stenjemo i čeznemo da se povrh njega zaodjenemo svojim  nebeskim obitavalištem; 
\par 3 dakako, ako se nađemo obučeni, ne  goli. 
\par 4 Da, i mi koji smo u ovom šatoru, stenjemo opterećeni  jer nećemo da budemo svučeni, nego da se još obučemo da život  iskapi što je smrtno. 
\par 5 A zato nas je sazdao Bog - on koji nam  dade zalog Duha. 
\par 6 Uvijek smo stoga puni pouzdanja makar i znamo:  naseljeni u tijelu, iseljeni smo od Gospodina. 
\par 7 Ta u vjeri  hodimo, ne u gledanju. 
\par 8 Da, puni smo pouzdanja i najradije  bismo se iselili iz tijela i naselili kod Gospodina. 
\par 9 Zato  se i trsimo da mu omilimo, bilo naseljeni, bilo iseljeni. 
\par 10 Jer  svima nam se pojaviti pred sudištem Kristovim da svaki dobije  što je kroz tijelo zaradio, bilo dobro, bilo zlo. 
\par 11 Prožeti dakle strahom Gospodnjim uvjeravamo ljude; razotkriveni  smo Bogu, a nadam se - i vašim savjestima. 
\par 12 Ne preporučujemo  vam opet sami sebe, nego vam dajemo prigodu ponositi se nama, da imate odgovor za one koji se diče licem, a ne srcem. 
\par 13 Doista, ako bijasmo "izvan sebe" - Bogu bijasmo; ako li "pri sebi" -  vama bijasmo. 
\par 14 Jer ljubav nas Kristova obuzima kad promatramo  ovo: jedan za sve umrije, svi dakle umriješe; 
\par 15 i za sve umrije  da oni koji žive ne žive više sebi, nego onomu koji za njih umrije  i uskrsnu. 
\par 16 Stoga mi od sada nikoga ne poznajemo po tijelu;  ako smo i poznavali po tijelu Krista, sada ga tako više ne poznajemo. 
\par 17 Dakle, je li tko u Kristu, nov je stvor. Staro uminu, novo, gle, nasta! 
\par 18 A sve je od Boga koji nas sa sobom pomiri po  Kristu i povjeri nam službu pomirenja. 
\par 19 Jer Bog je u Kristu  svijet sa sobom pomirio ne ubrajajući im opačina njihovih i polažući  u nas riječ pomirenja. 
\par 20 Kristovi smo dakle poslanici; Bog  vas po nama nagovara. Umjesto Krista zaklinjemo: dajte, pomirite  se s Bogom! 
\par 21 Njega koji ne okusi grijeha Bog za nas grijehom  učini da mi budemo pravednost Božja u njemu. 


\chapter{6}

\par 1 Kao suradnici opominjemo vas da ne primite uzalud milosti Božje. 
\par 2 Jer on veli: U vrijeme milosti usliših te i u dan spasa  pomogoh ti. Evo sad je vrijeme milosno, evo sad je vrijeme  spasa. 
\par 3 Ni u čemu ne dajemo nikakve sablazni da se ne kudi ova  služba, 
\par 4 nego se u svemu iskazujemo kao poslužitelji Božji:  velikom postojanošću u nevoljama, u potrebama, u tjeskobama, 
\par 5 pod udarcima, u tamnicama, u bunama, u naporima, u bdjenjima, u postovima, 
\par 6 u čistoći, u spoznanju, u velikodušnosti, u  dobroti, u Duhu Svetomu, u ljubavi nehinjenoj, 
\par 7 u riječi istinitoj, u snazi Božjoj; oružjem pravde zdesna i slijeva; 
\par 8 slavom i  sramotom; zlim i dobrim glasom; kao zavodnici, a istiniti; 
\par 9 kao  nepoznati, a poznati; kao umirući, a evo živimo; kao kažnjeni, a ne ubijeni; 
\par 10 kao žalosni, a uvijek radosni; kao siromašni, a mnoge obogaćujemo; kao oni koji ništa nemaju, a sve posjeduju. 
\par 11 Usta su naša otvorena vama, Korinćani, srce naše rašireno. 
\par 12 Nije vam tijesno u nama, ali je tijesno u vašim grudima. 
\par 13 Za uzdarje - kao djeci govorim - raširite se i vi. 
\par 14 Ne ujarmljujte se s nevjernicima. Ta što ima pravednost  s bezakonjem? Ili kakvo zajedništvo svjetlo s tamom? 
\par 15 Kakvu  slogu Krist s Belijarom? Ili kakav dio vjernik s nevjernikom? 
\par 16 Kakav sporazum hram Božji s idolima? Jer mi smo hram Boga  živoga, kao što reče Bog: Prebivat ću u njima i hoditi među njima; i bit ću Bog njihov, a oni narod moj. 
\par 17 Zato iziđite iz njihove sredine i odvojite se, govori Gospodin, i ništa nečisto ne dotičite i ja ću vas primiti. 
\par 18 I bit ću vam otac i vi ćete mi biti sinovi i kćeri, veli Gospodin Svemogući. 


\chapter{7}

\par 1 Dakle, budući da imamo ta obećanja, očistimo se, ljubljeni, od svake ljage tijela i duha te dovršimo posvećenje u strahu  Božjemu. 
\par 2 Shvatite nas! Nikomu nismo nanijeli nepravde, nikoga nismo  upropastili, nikoga zakinuli. 
\par 3 Ne govorim da osudim. Ta rekoh  već: u srcima ste našim te umiremo i živimo zajedno. 
\par 4 Veliko  je moje pouzdanje u vas, uvelike se vama ponosim. Pun sam utjehe, obilujem radošću uza svu nevolju našu. 
\par 5 Doista, i kada dođosmo u Makedoniju, nikakva spokoja nije  imalo tijelo naše, nego nevolje odasvud: izvana borbe, iznutra  strepnje. 
\par 6 Ali Bog, tješitelj poniznih, utješi nas dolaskom  Titovim. 
\par 7 Ne samo dolaskom njegovim, nego i utjehom kojom se  utješi zbog vas: obavijesti nas o vašoj čežnji, vašem jadikovanju, vašoj žarkoj ljubavi prema meni tako da se još većma obradovah. 
\par 8 Doista, ako sam vas i ožalostio onom poslanicom, nije  mi žao; ako mi i bijaše žao - vidim uistinu da vas je ta poslanica  makar i načas ožalostila - 
\par 9 sad se radujem, ne što ste se ožalostili, nego što ste se ožalostili na obraćenje. Jer ožalostili ste  se po Božju te zbog nas ni u čemu niste štetovali. 
\par 10 Jer žalost  po Božju rađa neopozivo spasonosnim obraćenjem, a žalost svjetovna  rađa smrću. 
\par 11 Gle, doista baš to što ste se po Božju ožalostili, kolikom gorljivošću urodi među vama, pa opravdavanjem, pa ogorčenjem, pa strahom, pa čežnjom, pa revnošću, pa kažnjavanjem. Svime  ste time pokazali da ste u onome nedužni. 
\par 12 Ako sam vam dakle  pisao, nisam to zbog uvreditelja ni zbog uvrijeđenoga, nego zbog  toga da vam se očituje vaša gorljivost za nas pred Bogom. 
\par 13 To  nas je utješilo. A povrh te naše utjehe još se mnogo više obradovasmo  zbog radosti Titove jer svi vi okrijepiste duh njegov. 
\par 14 Doista, ako sam mu se što vama pohvalio, ne postidjeh se, nego kao što  smo po istini vama govorili, tako je istina bila i pohvala naša  pred Titom. 
\par 15 I njegovo je srce prema vama još nježnije kad  se sjeti poslušnosti svih vas, kako ga sa strahom i trepetom  primiste. 
\par 16 Radujem se što se u svemu mogu pouzdati u vas. 


\chapter{8}

\par 1 Priopćujemo vam, braćo, milost Božju koja je dana crkvama makedonskim: 
\par 2 unatoč mnogim kušnjama i nevoljama izobilna njihova radost  i skrajnje siromaštvo preli se u bogatstvo darežljivosti. 
\par 3 Svjedočim  uistinu: oni su nas dragovoljno - po svojim mogućnostima i preko  mogućnosti - 
\par 4 veoma usrdno molili za milost zajedništva u ovom  posluživanju svetih. 
\par 5 I to ne samo kako se nadasmo, nego same  sebe predadoše najprvo Gospodinu, a onda nama, po volji Božjoj. 
\par 6 Zato zamolismo Tita da kao što je započeo, tako i dovrši među  vama i to djelo darežljivosti. 
\par 7 Stoga kao što se u svemu odlikujete - u vjeri, i riječi, i spoznanju, i svakoj gorljivosti, i u ljubavi svojoj prema  nama - odlikujte se i u ovoj darežljivosti. 
\par 8 Ne zapovijedam, nego gorljivošću drugih prokušavam istinitost  vaše ljubavi. 
\par 9 Ta poznate darežljivost Gospodina našega Isusa  Krista! Premda bogat, radi vas posta siromašan, da se vi njegovim  siromaštvom obogatite. 
\par 10 Time dajem samo savjet: to doista  dolikuje vama koji već prošle godine prvi to započeste, ne samo  činom nego i odlukom. 
\par 11 Sada dovršite to djelo da kao što spremno  odlučiste, tako prema mogućnostima i dovršite. 
\par 12 Jer ima li  spremnosti, mila je po onom što ima, a ne po onom čega nema. 
\par 13 Ne dakako: drugima olakšica, vama oskudica, nego - jednakost! 
\par 14 U sadašnjem trenutku vaš suvišak za njihovu oskudicu da jednom  njihov suvišak bude za vašu oskudicu - te bude jednakost, 
\par 15 kao  što je pisano: Nije ništa preteklo onome koji bijaše nakupio  mnogo, a niti je nedostajalo onome koji bijaše nakupio manje. 
\par 16 A hvala Bogu koji je stavio jednaku gorljivost za vas  u srce Titovo. 
\par 17 On je prihvatio i molbu, ali budući da je  veoma revan, otiđe k vama i dragovoljno. 
\par 18 S njime pak šaljemo  brata kojega s evanđelja slave sve crkve. 
\par 19 Štoviše, crkve  ga izabraše za našeg suputnika u ovom djelu darežljivosti kojemu  služimo - na slavu samoga Gospodina i na našu želju 
\par 20 kako  bismo izbjegli da nas tko ne prekori zbog ovog obilja kojim raspolažemo. 
\par 21 Doista, revno nastojimo oko dobra ne samo pred  Gospodinom nego i pred ljudima. 
\par 22 Šaljemo s njima  i našega brata koji je, kako smo u mnogome često iskusili, gorljiv, a sada je još mnogo gorljiviji zbog velikoga pouzdanja u vas. 
\par 23 A Tito? Moj je drug i suradnik za vas. A braća naša? Poslanici  su crkava, slava Kristova. 
\par 24 Pružite im dakle pred crkvama  dokaz svoje ljubavi i toga da se s pravom vama ponosimo. 


\chapter{9}

\par 1 A o posluživanju svetih suvišno je da vam pišem. 
\par 2 Ta poznajem  vašu spremnost s koje se vama ponosim pred Makedoncima: "Ahaja  je spremna od prošle godine." I vaša gorljivost potaknu mnoge. 
\par 3 Ipak šaljem braću da se u tome pogledu ne opovrgne što se  vama ponosimo; da budete spremni kao što sam tvrdio te se - 
\par 4 ako  sa mnom dođu Makedonci i nađu vas nespremne - ne osramotimo s  preuzetnosti mi, da ne kažemo vi. 
\par 5 Smatrao sam dakle potrebnim  zamoliti braću da unaprijed pođu k vama i da pripreme vaš još  prije obećani dar te bude pripravan - kao dar darežljivosti,  a ne škrtosti. 
\par 6 Ta eno: tko sije oskudno, oskudno će i žeti; a tko sije  obilato, obilato će i žeti. 
\par 7 Svatko neka dade kako je srcem  odlučio; ne sa žalošću ili na silu jer Bog ljubi vesela  darivatelja. 
\par 8 A Bog vas može obilato obdariti svakovrsnim  darom da u svemu svagda imate svega dovoljno za se i izobilno  za svako dobro djelo - 
\par 9 kao što je pisano: Rasipno dijeli, daje sirotinji, pravednost njegova ostaje dovijeka. 
\par 10 A onaj koji pribavlja sjeme sijaču i kruh za jelo,  pribavit će i umnožiti sjeme vaše i povećati plodove pravednosti  vaše. 
\par 11 Tako ćete se u svemu obogatiti za svakovrsnu darežljivost  koja se, našim posredovanjem, izvija u zahvalnicu Bogu. 
\par 12 Jer  ovo bogoslužno posluživanje ne samo da podmiruje oskudicu svetih  nego se i obilno prelijeva u mnoge zahvalnice Bogu. 
\par 13 Osvjedočeni  ovim posluživanjem, slave Boga zbog vašega pokornog ispovijedanja  evanđelja Kristova i zbog velikodušnog zajedništva prema njima  i prema svima. 
\par 14 A moleći se za vas, čeznu za vama zbog preobilne  milosti Božje na vama. 
\par 15 Hvala Bogu na njegovu neizrecivom  daru! 


\chapter{10}

\par 1 Ja, Pavao, osobno vas zaklinjem blagošću i obazrivošću Kristovom  - ja koji sam licem u lice među vama "skroman", a nenazočan prema  vama "odvažan" - 
\par 2 molim da, jednom nazočan, ne moram biti odvažan  smionošću kojom se kanim osmjeliti protiv nekih što smatraju  da mi po tijelu živimo. 
\par 3 Jer iako živimo u tijelu, ne vojujemo  po tijelu. 
\par 4 Ta oružje našega vojevanja nije tjelesno, nego  božanski snažno za rušenje utvrda. Obaramo mudrovanja 
\par 5 i svaku  oholost koja se podiže protiv spoznanja Boga i zarobljujemo svaki  um na pokornost Kristu; 
\par 6 i spremni smo kazniti svaku nepokornost  čim bude savršena vaša pokornost. 
\par 7 Gledajte što je očito! Ako je tko uvjeren da je "Kristov", neka sam ponovno promisli ovo: kako je on Kristov, tako smo  i mi. 
\par 8 Kad bih se doista i malo više pohvalio našom vlašću  - koju nam Gospodin dade za vaše izgrađivanje, a ne rušenje -  ne bih se morao stidjeti. 
\par 9 Samo da se ne bi činilo kao da vas  zastrašujem poslanicama! 
\par 10 Jer "poslanice su, kaže, stroge  i snažne, ali tjelesna nazočnost nemoćna i riječ bezvrijedna". 
\par 11 Takav neka promisli ovo: kakvi smo nenazočni riječju u poslanicama, takvi smo i nazočni djelom. 
\par 12 Ne usuđujemo se, doista, izjednačiti ili usporediti s  nekima koji sami sebe preporučuju, ali nisu razumni jer sami  sebe sobom mjere i sami sebe sa sobom uspoređuju. 
\par 13 Mi se pak  nećemo hvaliti u bezmjerje, nego po mjeri, mjerilu što nam ga  odmjeri Bog kao mjeru: doprijeti sve do vas. 
\par 14 Jer mi ne posežemo  preko svoga, kao da još nismo stigli do vas. Ta prvi doista doprijesmo  do vas s evanđeljem Kristovim. 
\par 15 Ne hvalimo se u bezmjerje, tuđim naporima. A nadamo se da ćemo s uzrastom vaše vjere među  vama i mi - po našem mjerilu - prerasti u izobilje: 
\par 16 i preko  vaših granica navijestiti evanđelje, a ne hvastati se onim što  je već učinjeno na tuđem području. 
\par 17 Tko se hvali, u Gospodinu  neka se hvali. 
\par 18 Ta nije prokušan tko sam sebe preporučuje, nego koga preporučuje Gospodin. 


\chapter{11}

\par 1 O kad biste podnijeli nešto malo bezumlja mojega! Da, podnesite  me! 
\par 2 Ljubomoran sam doista na vas Božjim ljubomorom: ta zaručih  vas s jednim mužem, kao čistu djevicu privedoh vas Kristu. 
\par 3 Ali  se bojim da se - kao što zmija zavede Evu svojom lukavštinom  - misli vaše ne pokvare i odmetnu od iskrenosti prema Kristu. 
\par 4 Uistinu, ako tko dođe i propovijeda drugog Isusa, kojega mi  nismo propovijedali - ili ako drugoga Duha primate, kojega niste  primili; ili drugo evanđelje, koje niste prigrlili - takva lijepo  podnosÄite. 
\par 5 Smatram, eto, da ni u čemu nisam manji od "nadapostola". 
\par 6 Jer ako sam i nevješt u govoru, nisam u znanju; naprotiv,  u svemu vam ga i pred svima očitovasmo. 
\par 7 Ili sam grijeh počinio što sam vam - ponizujući sebe da  se vi uzvisite - besplatno navješćivao Božje evanđelje? 
\par 8 Druge  sam crkve plijenio, od njih primao potporu da bih mogao vama  služiti. I dok bijah u vas, premda u oskudici, nikomu nisam bio  na teret. 
\par 9 U oskudici su mi pomogla braća koja dođoše iz Makedonije.  U svemu sam se čuvao da vam ne budem težak, a i čuvat ću se. 
\par 10 Istine mi Kristove u meni, ove mi hvale nitko neće oduzeti  u ahajskim krajevima. 
\par 11 Zašto? Jer vas ne ljubim? Bog znade! 
\par 12 A što činim, i dalje ću činiti da izbijem izliku onima  koji izliku traže ne bi li se s nama izjednačili u onom čime  se hvastaju. 
\par 13 Jer takvi su ljudi lažni apostoli, himbeni radnici, prerušuju se u apostole Kristove. 
\par 14 I nikakvo čudo! Ta sam  se Sotona prerušuje u anđela svjetla. 
\par 15 Ništa osobito dakle  ako se i službenici njegovi prerušuju u službenike pravednosti.  Svršetak će im biti po djelima njihovim. 
\par 16 Opet velim: da me tko ne bi smatrao bezumnim! Uostalom, primite me makar i kao bezumna da se i ja nešto malo pohvalim. 
\par 17 Što govorim, ne govorim po Gospodnju, nego kao u bezumlju, u ovoj hvalisavoj smionosti. 
\par 18 Budući da se mnogi hvale po  ljudsku, i ja ću se hvaliti. 
\par 19 Ta rado podnosÄite bezumne,  vi umni! 
\par 20 Da, podnosÄite ako vas tko zarobljava, ako vas tko  proždire, ako tko otima, ako se tko uznosi, ako vas tko po obrazu  bije. 
\par 21 Na sramotu govorim: bili smo, biva, slabi! Ipak, čime  se god tko osmjeljuje - u bezumlju govorim - osmjeljujem se i  ja! 
\par 22 Hebreji su? I ja sam! Izraelci su? I ja sam! Potomstvo  su Abrahamovo? I ja sam! 
\par 23 Poslužitelji su Kristovi? Kao mahnit  govorim: ja još više! U naporima - preobilno; u tamnicama - preobilno;  u batinama - prekomjerno; u smrtnim pogiblima - često. 
\par 24 Od  Židova primio sam pet puta po četrdeset manje jednu. 
\par 25 Triput  sam bio šiban, jednom kamenovan, triput doživio brodolom, jednu  noć i dan proveo sam u bezdanu. 
\par 26 Česta putovanja, pogibli  od rijeka, pogibli od razbojnika, pogibli od sunarodnjaka, pogibli  od pogana, pogibli u gradu, pogibli u pustinji, pogibli na moru, pogibli od lažne braće; 
\par 27 u trudu i naporu, često u nespavanju, u gladu i žeđi, često u postovima, u studeni i golotinji! 
\par 28 Osim  toga, uz drugo, salijetanje svakodnevno, briga za sve crkve. 
\par 29 Tko je slab, a ja da ne budem slab? Tko se sablažnjuje, a  ja da ne izgaram? 
\par 30 Treba li se hvaliti, svojom ću se slabošću  hvaliti. 
\par 31 Bog i Otac Gospodina Isusa, blagoslovljen u vijeke, zna da ne lažem. 
\par 32 U Damasku namjesnik kralja Arete čuvaše  grad damaščanski hoteći me uhvatiti. 
\par 33 Ali kroz prozor spustiše  me u košari preko zida te umakoh njegovim rukama. 


\chapter{12}

\par 1 Hvaliti se treba? Ne koristi doduše ali - dolazim na viđenje  i objave Gospodnje. 
\par 2 Znam čovjeka u Kristu: prije četrnaest  godina - da li u tijelu, ne znam; da li izvan tijela, ne znam, Bog zna - taj je bio ponesen do trećeg neba. 
\par 3 I znam da je  taj čovjek - da li u tijelu, da li izvan tijela, ne znam, Bog  zna - 
\par 4 bio ponesen u raj i čuo neizrecive riječi, kojih čovjek  ne smije govoriti. 
\par 5 Time ću se hvaliti, a samim se sobom neću  hvaliti osim slabostima svojim. 
\par 6 Uistinu, kad bih se i htio  hvaliti, ne bih bio bezuman; istinu bih govorio. Ali se uzdržavam  da ne bi tko mislio o meni više nego što vidi na meni ili što  čuje od mene. 
\par 7 I da se zbog uzvišenosti objava ne bih uzoholio, dan mi  je trn u tijelu, anđeo Sotonin, da me udara da se ne uzoholim. 
\par 8 Za to sam triput molio Gospodina, da odstupi od mene. A on  mi reče: 
\par 9 "Dosta ti je moja milost jer snaga se u slabosti  usavršuje." Najradije ću se dakle još više hvaliti svojim slabostima  da se nastani u meni snaga Kristova. 
\par 10 Zato uživam u slabostima, uvredama, poteškoćama, progonstvima, tjeskobama poradi Krista.  Jer kad sam slab, onda sam jak. 
\par 11 Postao sam bezuman! Vi me natjeraste. Ta trebalo je da  me vi preporučite jer ni u čemu nisam manji od "nadapostola", premda nisam ništa. 
\par 12 Znamenja apostolstva moga ostvarena  su među vama u posvemašnjoj postojanosti: znakovima i čudesima  i silnim djelima. 
\par 13 Ta u čemu ste to manji od drugih crkava, osim što vam ja nisam bio na teret? Oprostite mi ovu "nepravdu". 
\par 14 Evo, spremam se treći put doći k vama i neću vam biti na  teret jer ne ištem vaše, nego vas. Djeca doista nisu dužna stjecati  roditeljima, nego roditelji djeci. 
\par 15 A ja ću najradije trošiti  i istrošiti se za duše vaše. Ako vas više ljubim, zar da budem  manje ljubljen? 
\par 16 Ali neka! Ja vas nisam opterećivao, nego, "lukav" kako jesam, "na prijevaru vas uhvatih". 
\par 17 Da vas možda  nisam zakinuo po kome od onih koje poslah k vama? 
\par 18 Zamolio  sam Tita i poslao s njime brata. Da vas možda Tit nije u čemu  zakinuo? Zar nismo hodili u istom duhu? I istim stopama? 
\par 19 Odavna smatrate da se pred vama branimo. Pred Bogom u  Kristu govorimo: sve je to, ljubljeni, za vaše izgrađivanje. 
\par 20 Bojim se doista da vas kada dođem, možda neću naći kakve  bih htio i da ćete vi mene naći kakva ne biste htjeli: da ne  bi možda bilo svađa, zavisti, žestina, spletkarenja, klevetanja, došaptavanja, nadimanja, buna; 
\par 21 da me opet kada dođem, ne  bi ponizio Bog moj kod vas kako ne bih morao oplakivati mnoge  koji su prije sagriješili,  a nisu se pokajali za nečistoću i  bludnost i razvratnost koju počiniše. 



\chapter{13}

\par 1 Evo treći put idem k vama. Svaka presuda neka počiva na  iskazu dvojice ili trojice svjedoka. 
\par 2 Onima koji su prije  sagriješili i svima drugima rekoh već i opet - kao onda drugi  put nazočan, tako i sada nenazočan - unaprijed velim: ako opet  dođem, neću štedjeti. 
\par 3 Jer vi tražite dokaz da u meni govori  Krist koji prema vama nije nemoćan, nego je snažan među vama. 
\par 4 I raspet bi, istina, po slabosti, ali živi po snazi Božjoj.  I mi smo, istina, slabi u njemu, ali ćemo po snazi Božjoj živjeti  s njime za vas. 
\par 5 Same sebe ispitujte, jeste li u vjeri! Same sebe provjeravajte!  Zar ne spoznajete sami sebe: da je Isus Krist u vama? Inače niste  pravi. 
\par 6 A spoznat ćete, nadam se, da smo mi pravi. 
\par 7 Molimo  se Bogu da ne činite nikakva zla; ne da se mi pokažemo pravi, nego da vi dobro činite, pa izašli mi i kao nepravi. 
\par 8 Ta ništa  ne možemo protiv istine, nego samo za istinu. 
\par 9 Da, radujemo  se kad smo mi slabi, a vi jaki. Za to se i molimo, za vaše usavršavanje. 
\par 10 To vam nenazočan pišem zato da nazočan ne bih morao oštro  nastupiti vlašću koju mi Gospodin dade za izgrađivanje, a ne  za rušenje. 
\par 11 Uostalom, braćo, radujte se, usavršujte se, tješite se, složni budite, mir njegujte i Bog ljubavi i mira bit će s vama. 
\par 12 Pozdravite jedni druge svetim cjelovom. Pozdravljaju vas svi sveti. 
\par 13 Milost Gospodina Isusa Krista, ljubav Boga i zajedništvo  Duha Svetoga sa svima vama! 
\par 14 - - - 




\end{document}