\begin{document}

\title{Postanak}


\chapter{1}

\par 1 U početku stvori Bog nebo i zemlju. 
\par 2 Zemlja bijaše pusta  i prazna; tama se prostirala nad bezdanom i Duh Božji lebdio  je nad vodama. 
\par 3 I reče Bog: "Neka bude svjetlost!" I bi svjetlost. 
\par 4 I  vidje Bog da je svjetlost dobra; i rastavi Bog svjetlost od tame. 
\par 5 Svjetlost prozva Bog dan, a tamu prozva noć. Tako bude večer, pa jutro - dan prvi. 
\par 6 I reče Bog: "Neka bude svod posred voda da dijeli vode  od voda!" I bi tako. 
\par 7 Bog načini svod i vode pod svodom odijeli  od voda nad svodom. 
\par 8 A svod prozva Bog nebo. Tako bude večer, pa jutro - dan drugi. 
\par 9 I reče Bog: "Vode pod nebom neka se skupe na jedno mjesto  i neka se pokaže kopno!" I bi tako. 
\par 10 Kopno prozva Bog zemlja, a skupljene vode mora. I vidje Bog da je dobro. 
\par 11 I reče Bog: "Neka proklija zemlja zelenilom - travom  sjemenitom, stablima plodonosnim, koja, svako prema svojoj vrsti, na zemlji donose plod što u sebi nosi svoje sjeme. I bi tako. 
\par 12 I nikne iz zemlje zelena trava što se sjemeni, svaka prema  svojoj vrsti, i stabla koja rode plodovima što u sebi nose svoje  sjeme, svako prema svojoj vrsti. I vidje Bog da je dobro. 
\par 13 Tako  bude večer, pa jutro - dan treći. 
\par 14 I reče Bog: "Neka budu svjetlila na svodu nebeskom da  luče dan od noći, da budu znaci blagdanima, danima i godinama, 
\par 15 i neka svijetle na svodu nebeskom i rasvjetljuju zemlju!"  I bi tako. 
\par 16 I načini Bog dva velika svjetlila - veće da vlada  danom, manje da vlada noću - i zvijezde. 
\par 17 I Bog ih postavi  na svod nebeski da rasvjetljuju zemlju, 
\par 18 da vladaju danom  i noću i da rastavljaju svjetlost od tame. I vidje Bog da je  dobro. 
\par 19 Tako bude večer, pa jutro - dan četvrti. 
\par 20 I reče Bog: "Nek' povrvi vodom vreva živih stvorova,  i ptice nek' polete nad zemljom, svodom nebeskim!" I bi tako. 
\par 21 Stvori Bog morske grdosije i svakovrsne žive stvorove što  mile i vrve vodom i ptice krilate svake vrste. I vidje Bog da  je dobro. 
\par 22 I blagoslovi ih govoreći: "Plodite se i množite  i napunite vode morske! I ptice neka se namnože na zemlji!" 
\par 23 Tako  bude večer, pa jutro - dan peti. 
\par 24 I reče Bog: "Neka zemlja izvede živa bića, svako prema  svojoj vrsti: stoku, gmizavce i zvjerad svake vrste!" I bi tako. 
\par 25 I stvori Bog svakovrsnu zvjerad, stoku i gmizavce svake vrste.  I vidje Bog da je dobro. 
\par 26 I reče Bog: "Načinimo čovjeka na svoju sliku, sebi slična, da bude gospodar ribama morskim, pticama nebeskim i stoci -  svoj zemlji - i svim gmizavcima što puze po zemlji!" 
\par 27 Na svoju sliku stvori Bog čovjeka, na sliku Božju on ga stvori, muško i žensko stvori ih. 
\par 28 I blagoslovi ih Bog i reče im: "Plodite se, i množite, i napunite zemlju, i sebi je podložite! Vladajte ribama u moru  i pticama u zraku i svim živim stvorovima što puze po zemlji!" 
\par 29 I doda Bog: "Evo, dajem vam sve bilje što se sjemeni, po  svoj zemlji, i sva stabla plodonosna što u sebi nose svoje sjeme:  neka vam budu za hranu! 
\par 30 A zvijerima na zemlji i pticama u  zraku i gmizavcima što puze po zemlji u kojima je dah života  - neka je za hranu sve zeleno bilje!" I bi tako. 
\par 31 I vidje  Bog sve što je učinio, i bijaše veoma dobro. Tako bude večer, pa jutro - dan šesti. 



\chapter{2}



\par 1 Tako bude dovršeno nebo i zemlja sa svom svojom vojskom. 
\par 2 I  sedmoga dana Bog dovrši svoje djelo koje učini. I počinu u sedmi  dan od svega djela koje učini. 
\par 3 I blagoslovi Bog sedmi dan  i posveti, jer u taj dan počinu od svega djela svoga koje učini. 
\par 4 To je postanak neba i zemlje, tako su stvarani.  Kad je Jahve, Bog, sazdao nebo i zemlju, 
\par 5 još nije bilo  nikakva poljskoga grmlja po zemlji, još ne bijaše niklo nikakvo  poljsko bilje, jer Jahve, Bog, još ne pusti dažda na zemlju i  nije bilo čovjeka da zemlju obrađuje. 
\par 6 Ipak, voda je izvirala  iz zemlje i natapala svu površinu zemaljsku. 
\par 7 Jahve, Bog, napravi  čovjeka od praha zemaljskog i u nosnice mu udahne dah života.  Tako postane čovjek živa duša. 
\par 8 I Jahve, Bog, zasadi vrt na istoku, u Edenu, i u nj smjesti  čovjeka koga je napravio. 
\par 9 Tada Jahve, Bog, učini te iz zemlje  nikoše svakovrsna stabla - pogledu zamamljiva a dobra za hranu  - i stablo života, nasred vrta, i stablo spoznaje dobra i zla. 
\par 10 Rijeka je izvirala iz Edena da bi natapala vrt; odatle  se granala u četiri kraka. 
\par 11 Prvom je ime Pišon, a optječe  svom zemljom havilskom, u kojoj ima zlata. 
\par 12 Zlato je te zemlje  dobro, a ima ondje i bdelija i oniksa. 
\par 13 Drugoj je rijeci ime  Gihon, a optječe svu zemlju Kuš. 
\par 14 Treća je rijeka Tigris,  a teče na istok od Ašura; četvrta je Eufrat. 
\par 15 Jahve, Bog, uzme čovjeka i postavi ga u edenski vrt da  ga obrađuje i čuva. 
\par 16 Jahve, Bog, zapovjedi čovjeku: "Sa svakoga  stabla u vrtu slobodno jedi, 
\par 17 ali sa stabla spoznaje dobra  i zla da nisi jeo! U onaj dan u koji s njega okusiš, zacijelo  ćeš umrijeti!" 
\par 18 I reče Jahve, Bog: "Nije dobro da čovjek bude sam: načinit  ću mu pomoć kao što je on." 
\par 19 Tada Jahve, Bog, načini od zemlje  sve životinje u polju i sve ptice u zraku i predvede ih čovjeku  da vidi kako će koju nazvati, pa kako koje stvorenje čovjek prozove, da mu tako bude ime. 
\par 20 Čovjek nadjene imena svoj stoci, svim  pticama u zraku i životinjama u polju. No čovjeku se ne nađe  pomoć kao što je on. 
\par 21 Tada Jahve, Bog, pusti tvrd san na čovjeka  te on zaspa, pa mu izvadi jedno rebro, a mjesto zatvori mesom. 
\par 22 Od rebra što ga je uzeo čovjeku napravi Jahve, Bog, ženu  pa je dovede čovjeku. 
\par 23 Nato čovjek reče: "Gle, evo kosti od mojih kostiju, mesa od mesa mojega! Ženom neka se zove, od čovjeka kad je uzeta!" 
\par 24 Stoga će čovjek ostaviti oca i majku da prione uza svoju  ženu i bit će njih dvoje jedno tijelo. 
\par 25 A bijahu oboje goli - čovjek i njegova žena - ali ne  osjećahu stida. 



\chapter{3}



\par 1 Zmija bijaše lukavija od sve zvjeradi što je stvori Jahve,  Bog. Ona reče ženi: "Zar vam je Bog rekao da ne smijete jesti  ni s jednog drveta u vrtu?" 
\par 2 Žena odgovori zmiji: "Plodove  sa stabala u vrtu smijemo jesti. 
\par 3 Samo za plod stabla što je  nasred vrta rekao je Bog: 'Da ga niste jeli! I ne dirajte u nj, da ne umrete!'" 
\par 4 Nato će zmija ženi: "Ne, nećete umrijeti! 
\par 5 Nego, zna Bog: onog dana kad budete s njega jeli, otvorit  će vam se oči, i vi ćete biti kao bogovi koji razlučuju dobro  i zlo." 
\par 6 Vidje žena da je stablo dobro za jelo, za oči zamamljivo, a za mudrost poželjno: ubere ploda njegova i pojede. Dade i  svom mužu, koji bijaše s njom, pa je i on jeo. 
\par 7 Tada se obadvoma  otvore oči i upoznaju da su goli. Spletu smokova lišća i naprave  sebi pregače. 
\par 8 Uto čuju korak Jahve, Boga, koji je šetao vrtom za dnevnog  povjetarca. I sakriju se - čovjek i njegova žena - pred Jahvom, Bogom, među stabla u vrtu. 
\par 9 Jahve, Bog, zovne čovjeka: "Gdje  si?" - reče mu. 
\par 10 On odgovori: "Čuo sam tvoj korak po vrtu;  pobojah se jer sam go, pa se sakrih." 
\par 11 Nato mu reče: "Tko  ti kaza da si go? Ti si, dakle, jeo sa stabla s kojega sam ti  zabranio jesti?" 
\par 12 Čovjek odgovori: "Žena koju si stavio uza  me - ona mi je dala sa stabla pa sam jeo." 
\par 13 Jahve, Bog, reče  ženi: "Što si to učinila?" "Zmija me prevarila pa sam jela",  odgovori žena. 
\par 14 Nato Jahve, Bog, reče zmiji: "Kad si to učinila, prokleta bila među svim životinjama i svom zvjeradi divljom! Po trbuhu svome puzat ćeš i zemlju jesti sveg života svog! 
\par 15 Neprijateljstvo ja zamećem između tebe i žene, između roda tvojeg i roda njezina: on će ti glavu satirati, a ti ćeš mu vrebati petu." 
\par 16 A ženi reče: "Trudnoći tvojoj muke ću umnožit, u mukama djecu ćeš rađati. Žudnja će te mužu tjerati, a on će gospodariti nad tobom." 
\par 17 A čovjeku reče: "Jer si poslušao glas svoje žene te jeo  sa stabla s kojega sam ti zabranio jesti rekavši: S njega da  nisi jeo! - evo: Zemlja neka je zbog tebe prokleta: s trudom ćeš se od nje hraniti svega vijeka svog! 
\par 18 Rađat će ti trnjem i korovom, a hranit ćeš se poljskim raslinjem. 
\par 19 U znoju lica svoga kruh svoj ćeš jesti dokle se u zemlju ne vratiš: tÓa iz zemlje uzet si bio - prah si, u prah ćeš se i vratiti." 
\par 20 Svojoj ženi čovjek nadjene ime Eva, jer je majka svima  živima. 
\par 21 I načini Jahve, Bog, čovjeku i njegovoj ženi odjeću  od krzna pa ih odjenu. 
\par 22 Zatim reče Bog: "Evo, čovjek postade  kao jedan od nas - znajući dobro i zlo! Da ne bi sada pružio  ruku, ubrao sa stabla života pa pojeo i živio navijeke!" 
\par 23 Zato  ga Jahve, Bog, istjera iz vrta edenskoga da obrađuje zemlju iz  koje je i uzet. 
\par 24 Istjera, dakle, čovjeka i nastani ga istočno  od vrta edenskog, pa postavi kerubine i plameni mač koji se svjetlucao  - da straže nad stazom koja vodi k stablu života. 



\chapter{4}



\par 1 Čovjek pozna svoju ženu Evu, a ona zače i rodi Kajina, pa reče:  "Muško sam čedo stekla pomoću Jahve!" 
\par 2 Poslije rodi Abela,  brata Kajinova; Abel postane stočar, a Kajin zemljoradnik. 
\par 3 I  jednoga dana Kajin prinese Jahvi žrtvu od zemaljskih plodova. 
\par 4 A prinese i Abel od prvine svoje stoke, sve po izbor pretilinu.  Jahve milostivo pogleda na Abela i njegovu žrtvu, 
\par 5 a na Kajina  i žrtvu njegovu ni pogleda ne svrati. Stoga se Kajin veoma razljuti  i lice mu se namrgodi. 
\par 6 I Jahve reče Kajinu: "Zašto si ljut?  Zašto ti je lice namrgođeno? 
\par 7 Jer ako pravo radiš, vedrinom  odsijevaš. A ne radiš li pravo, grijeh ti je kao zvijer na pragu  što na te vreba; još mu se možeš oduprijeti." 
\par 8 Kajin pak reče  svome bratu Abelu: "Hajdemo van!" I našavši se na polju, Kajin  skoči na brata Abela te ga ubi. 
\par 9 Potom Jahve zapita Kajina: "Gdje ti je brat Abel?" "Ne  znam", odgovori. "Zar sam ja čuvar brata svoga?" 
\par 10 Jahve nastavi:  "Što si učinio? Slušaj! Krv brata tvoga iz zemlje k meni viče. 
\par 11 Stoga budi proklet na zemlji koja je rastvorila usta da proguta  s ruke tvoje krv brata tvoga! 
\par 12 Obrađivat ćeš zemlju, ali ti  više neće davati svoga roda. Vječni ćeš skitalica na zemlji biti!" 
\par 13 A Kajin reče Jahvi: "Kazna je moja odviše teška da se snosi. 
\par 14 Evo me tjeraš danas s plodnoga tla; moram se skrivati od  tvoga lica i biti vječni lutalac na zemlji - tko me god nađe, može me ubiti." 
\par 15 A Jahve mu reče: "Ne! Nego tko ubije Kajina, sedmerostruka osveta na njemu će se izvršiti!" I Jahve stavi  znak na Kajina, da ga tko, našavši ga, ne ubije. 
\par 16 Kajin ode  ispred lica Jahvina u zemlju Nod, istočno od Edena, i ondje se  nastani. 
\par 17 Kajin pozna svoju ženu te ona zače i rodi Henoka. Podigao  je grad i grad prozvao imenom svoga sina - Henok. 
\par 18 Henoku  se rodio Irad, a od Irada potekao Mehujael; od Mehujaela poteče  Metušael, od Metušaela Lamek. 
\par 19 Lamek uzme dvije žene. Jedna  se zvala Ada, a druga Sila. 
\par 20 Ada rodi Jabala, koji je postao  praocem onih što pod šatorima žive sa stokom. 
\par 21 Bratu mu bijaše  ime Jubal. On je praotac svih koji sviraju na liru i sviralu. 
\par 22 Sila rodi Tubal-Kajina, praoca onih koji kuju bakar i željezo.  Tubal-Kajinovoj sestri bijaše ime Naama. 
\par 23 Lamek prozbori svojim ženama: "Ada i Sila, glas moj poslušajte! Žene Lamekove, čujte mi besjedu: Čovjeka sam ubio jer me ranio i dijete jer me udarilo. 
\par 24 Ako će Kajin biti osvećen sedmerostruko, Lamek će sedamdeset i sedam puta!" 
\par 25 Adam pozna svoju ženu te ona rodi sina i nadjenu mu ime  Šet. Reče ona: "Bog mi dade drugo dijete mjesto Abela, koga ubi  Kajin." 
\par 26 Šetu se rodi sin, komu on nadjenu ime Enoš. Tada  se počelo zazivati ime Jahvino. 



\chapter{5}



\par 1 Ovo je povijest Adamova roda. Kad je Bog stvorio čovjeka, napravio ga je na priliku svoju; 
\par 2 stvorio je muško i žensko. A kad ih je stvorio, blagoslovi  ih i nazva - čovjek. 
\par 3 Kad je Adamu bilo sto i trideset godina, rodi mu se sin  njemu sličan, na njegovu sliku; nadjenu mu ime Šet. 
\par 4 Po rođenju  Šetovu Adam je živio osam stotina godina te mu se rodilo još  sinova i kćeri. 
\par 5 Adam poživje u svemu devet stotina i trideset  godina. Potom umrije. 
\par 6 Kad je Šetu bilo sto i pet godina, rodi mu se Enoš. 
\par 7 Po  rođenju Enoševu Šet je živio osam stotina i sedam godina te mu  se rodilo još sinova i kćeri. 
\par 8 Šet poživje u svemu devet stotina  i dvanaest godina. Potom umrije. 
\par 9 Kad je Enošu bilo devedeset godina, rodi mu se Kenan. 
\par 10 Po rođenju Kenanovu Enoš je živio osam stotina i petnaest  godina te mu se rodilo još sinova i kćeri. 
\par 11 Enoš poživje u  svemu devet stotina i pet godina. Potom umrije. 
\par 12 Kad je Kenanu bilo sedamdeset godina, rodi mu se Mahalalel. 
\par 13 Po rođenju Mahalalelovu Kenan je živio osam stotina i četrdeset  godina te mu se rodilo još sinova i kćeri. 
\par 14 Kenan poživje  u svemu devet stotina i deset godina. Potom umrije. 
\par 15 Kad je Mahalalelu bilo šezdeset i pet godina, rodi mu  se Jered. 
\par 16 Po rođenju Jeredovu Mahalalel je živio osam stotina  i trideset godina te mu se rodilo još sinova i kćeri. 
\par 17 Mahalalel  poživje u svemu osam stotina devedeset i pet godina. Potom umrije. 
\par 18 Kad je Jeredu bilo sto šezdeset i dvije godine, rodi  mu se Henok. 
\par 19 Po rođenju Henokovu Jered je živio osam stotina  godina te mu se rodilo još sinova i kćeri. 
\par 20 Jered poživje  u svemu devet stotina šezdeset i dvije godine. Potom umrije. 
\par 21 Kad je Henoku bilo šezdeset i pet godina, rodi mu se  Metušalah. 
\par 22 Henok je hodio s Bogom. Po rođenju Metušalahovu  Henok je živio trista godina te mu se rodilo još sinova i kćeri. 
\par 23 Henok poživje u svemu trista šezdeset i pet godina. 
\par 24 Henok  je hodio s Bogom, potom iščeznu; Bog ga uze. 
\par 25 Kad je Metušalahu bilo sto osamdeset i sedam godina,  rodi mu se Lamek. 
\par 26 Po rođenju Lamekovu Metušalah je živio  sedam stotina osamdeset i dvije godine te mu se rodilo još sinova  i kćeri. 
\par 27 Metušalah poživje u svemu devet stotina šezdeset  i devet godina. Potom umrije. 
\par 28 Kad su Lameku bile sto osamdeset i dvije godine, rodi  mu se sin. 
\par 29 Nadjene mu ime Noa, govoreći: "Ovaj će nam pribavljati, u trudu i naporu naših ruku, utjehu iz zemlje koju je Bog prokleo." 
\par 30 Po rođenju Noinu Lamek je živio pet stotina devedeset i pet  godina te mu se rodilo još sinova i kćeri. 
\par 31 Lamek poživje  u svemu sedam stotina sedamdeset i sedam godina. Potom umrije. 
\par 32 Pošto je Noa proživio pet stotina godina, rode mu se  Šem, Ham i Jafet. 



\chapter{6}



\par 1 Kad su se ljudi počeli širiti po zemlji i kćeri im se narodile, 
\par 2 opaze sinovi Božji da su kćeri ljudske pristale, pa ih uzimahu  sebi za žene koje su god htjeli. 
\par 3 Onda Jahve reče: "Neće moj  duh u čovjeku ostati dovijeka; čovjek je tjelesan, pa neka mu  vijek bude stotinu dvadeset godina." 
\par 4 U ona su vremena - a  i kasnije - na zemlji bili Nefili, kad su Božji sinovi općili  s ljudskim kćerima pa im one rađale djecu. To su oni od starine  po snazi glasoviti ljudi. 
\par 5 Vidje Jahve kako je čovjekova pokvarenost na zemlji velika  i kako je svaka pomisao u njegovoj pameti uvijek samo zloća. 
\par 6 Jahve se pokaja i u svom srcu ražalosti što je načinio čovjeka  na zemlji. 
\par 7 Reče Jahve: "Ljude koje sam stvorio izbrisat ću  s lica zemlje - od čovjeka do zvijeri, puzavce i ptice u zraku  - jer sam se pokajao što sam ih napravio." 
\par 8 Ali je Noa našao  milost u očima Jahvinim. 
\par 9 Ovo je povijest Noina: Noa je bio čovjek pravedan i neporočan u svom vremenu. S  Bogom je Noa hodio. 
\par 10 Tri su se sina rodila Noi: Šem, Ham i  Jafet. 
\par 11 U očima Božjim zemlja se bila iskvarila; nepravdom se  napunila. 
\par 12 I kad je Bog vidio kako se zemlja iskvarila - tÓa  svako se biće na zemlji izopačilo - 
\par 13 reče Bog Noi: "Odlučio  sam da bude kraj svim bićima jer se zemlja napunila opačinom;  i, evo, uništit ću ih zajedno sa zemljom. 
\par 14 Napravi sebi korablju  od smolastoga drveta; korablju načini s prijekletima i obloži  je iznutra i izvana paklinom. 
\par 15 A pravit ćeš je ovako: neka  korablja bude trista lakata u duljinu, pedeset u širinu, a trideset  lakata u visinu. 
\par 16 Na korablji načini otvor za svjetlo, završi  ga jedan lakat od vrha. Vrata na korablji načini sa strane; neka  ima donji, srednji i gornji kat. 
\par 17 Ja ću, evo, pustiti potop  - vode na zemlju - da izgine svako biće pod nebom, sve u čemu  ima dah života: sve na zemlji mora poginuti. 
\par 18 A s tobom ću  učiniti Savez; ti ćeš ući u korablju - ti i s tobom tvoji sinovi, tvoja žena i žene tvojih sinova. 
\par 19 A od svega što je živo  - od svih bića - uvedi u korablju od svakoga po dvoje da s tobom  preživi, i neka budu muško i žensko. 
\par 20 Od ptica prema njihovim  vrstama, od životinja prema njihovim vrstama i od svih stvorova  što po tlu puze prema njihovim vrstama: po dvoje od svega neka  uđe k tebi da preživi. 
\par 21 Sa sobom uzmi svega za jelo pa čuvaj  da bude hrane tebi i njima." 
\par 22 Noa učini tako. Sve kako mu je Bog naredio, tako je izvršio. 



\chapter{7}



\par 1 Onda Jahve reče Noi: "Uđi ti i sva tvoja obitelj u korablju, jer sam uvidio da si ti jedini preda mnom pravedan u ovom vremenu. 
\par 2 Uzmi sa sobom od svih čistih životinja po sedam parova: mužjaka  i njegovu ženku. 
\par 3 Isto tako od ptica nebeskih po sedam parova  - mužjaka i ženku - da im se sjeme sačuva na zemlji. 
\par 4 Jer ću  do sedam dana pustiti dažd po zemlji četrdeset dana i četrdeset  noći te ću istrijebiti s lica zemlje svako živo biće što sam  ga načinio." 
\par 5 Noa učini sve kako mu je Jahve naredio. 
\par 6 Noi bijaše šest stotina godina kad je potop došao na zemlju. 
\par 7 I pred vodama potopnim uđu s Noom u korablju njegovi sinovi, njegova žena i žene sinova njegovih. 
\par 8 Od čistih životinja  i od životinja koje nisu čiste, od ptica, od svega što zemljom  puzi, 
\par 9 uđe po dvoje - mužjak i ženka - u korablju s Noom, kako  je Bog naredio Noi. 
\par 10 A sedmoga dana zapljušte potopne vode  po zemlji. 
\par 11 U dan onaj - šestote godine Noina života, mjeseca drugog, dana u mjesecu sedamnaestog - navale svi izvori bezdana, rastvore se ustave nebeske. 
\par 12 I udari dažd na zemlju da pljušti četrdeset dana i četrdeset  noći. 
\par 13 Onog dana uđe u korablju Noa i njegovi sinovi: Šem, Ham i Jafet, Noina žena i tri žene Noinih sinova s njima; 
\par 14 oni, pa sve vrste životinja: stoka, gmizavci što po tlu gmižu, ptice  i svakovrsna krilata stvorenja, 
\par 15 uđu u korablju s Noom, po  dvoje od svih bića što u sebi imaju dah života. 
\par 16 Što uđe,  sve bijaše par, mužjak i ženka od svih bića, kako je Bog naredio  Noi.  Onda Jahve zatvori za njim vrata. 
\par 17 Pljusak je na zemlju padao četrdeset dana; vode sveudilj  rasle i korablju nosile: digla se visoko iznad zemlje. 
\par 18 Vode  su nad zemljom bujale i visoko rasle, a korablja plovila površinom. 
\par 19 Vode su sve silnije navaljivale i rasle nad zemljom, tako  te prekriše sva najviša brda pod nebom. 
\par 20 Petnaest lakata dizale  se vode povrh potonulih brda. 
\par 21 Izgiboše sva bića što se po  zemlji kreću: ptice, stoka, zvijeri, svi gmizavci i svi ljudi. 
\par 22 Sve što u svojim nosnicama imaše dah života - sve što bijaše  na kopnu - izgibe. 
\par 23 Istrijebi se svako biće s površja zemaljskog:  čovjek, životinje, gmizavci i ptice nebeske, sve se izbrisa sa  zemlje. Samo Noa ostade i oni što bijahu s njim u korablji. 
\par 24 Stotinu  pedeset dana vladahu vode zemljom. 



\chapter{8}



\par 1 Onda se Bog sjeti Noe, svih zvijeri i sve stoke što bijaše  s njim u korablji, pa pokrenu vjetar nad zemljom da uzbije vodu. 
\par 2 Zatvoriše se izvori bezdanu i ustave nebeske, i dažd s neba  prestade. 
\par 3 Polako se povlačile vode sa zemlje. Nakon stotinu  pedeset dana vode su jenjale, 
\par 4 a sedmoga mjeseca, sedamnaestog  dana u mjesecu korablja se zaustavi na brdima Ararata. 
\par 5 Vode  su neprestano opadale do desetog mjeseca, a prvoga dana desetog  mjeseca pokažu se brdski vrhunci. 
\par 6 Kad je izminulo četrdeset dana, Noa otvori prozor što  ga je načinio na korablji; 
\par 7 ispusti gavrana, a gavran svejednako  odlijetaše i dolijetaše dok se vode sa zemlje nisu isušile. 
\par 8 Zatim  ispusti golubicu da vidi je li voda nestala sa zemlje. 
\par 9 Ali  golubica ne nađe uporišta nogama te se vrati k njemu u korablju, jer voda još pokrivaše svu površinu; on pruži ruku, uhvati golubicu  te je unese k sebi u korablju. 
\par 10 Počeka još sedam dana pa opet  pusti golubicu iz korablje. 
\par 11 Prema večeri golubica se vrati  k njemu, i gle! u kljunu joj svjež maslinov list; tako je Noa  doznao da su opale vode sa zemlje. 
\par 12 Još počeka sedam dana  pa opet pusti golubicu: više mu se nije vratila. 
\par 13 Šest stotina prve godine Noina života, prvoga mjeseca, prvog dana u mjesecu uzmakoše vode sa zemlje. Noa skine pokrov s korablje i pogleda: površina okopnjela. 
\par 14 A drugoga mjeseca, sedamnaestog dana u mjesecu, zemlja  bijaše suha. 
\par 15 Tada Bog reče Noi: 
\par 16 "Iziđi iz korablje, ti, tvoja  žena, tvoji sinovi i žene tvojih sinova s tobom. 
\par 17 Sa sobom  izvedi sva živa bića, sva stvorenja što su s tobom: ptice, stoku  i sve gmizavce što zemljom puze; neka zemljom vrve, plode se  i na zemlji množe!" 
\par 18 I Noa iziđe, a s njime sinovi njegovi, žena njegova i žene sinova njegovih. 
\par 19 Sve životinje, svi  gmizavci, sve ptice - svi stvorovi što se zemljom miču - iziđu  iz korablje, vrsta za vrstom. 
\par 20 I podiže Noa žrtvenik Jahvi; uze od svih čistih životinja  i od svih čistih ptica i prinese na žrtveniku žrtve paljenice. 
\par 21 Jahve omirisa miris ugodni pa reče u sebi: "Nikad više neću  zemlju u propast strovaliti zbog čovjeka, tÓa čovječje su misli  opake od njegova početka; niti ću ikad više uništiti sva živa  stvorenja, kako sam učinio. 
\par 22 Sve dok zemlje bude, sjetve, žetve, studeni, vrućine, ljeta, zime, dani, noći nikada prestati neće." 



\chapter{9}



\par 1 Tada Bog blagoslovi Nou i njegove sinove i reče im: "Plodite  se i množite i zemlju napunite. 
\par 2 Neka vas se boje i od vas  strahuju sve životinje na zemlji, sve ptice u zraku, sve što  se po zemlji kreće i sve ribe u moru: u vaše su ruke predane. 
\par 3 Sve što se kreće i živi neka vam bude za hranu: sve vam dajem, kao što vam dadoh zeleno bilje. 
\par 4 Samo ne smijete jesti mesa  u kojem je još duša, to jest njegova krv. 
\par 5 A za vašu krv, za  vaš život, tražit ću obračun: tražit ću ga od svake životinje;  i od čovjeka za njegova druga tražit ću obračun za ljudski život. 
\par 6 Tko prolije krv čovjekovu, njegovu će krv čovjek proliti! Jer na sliku Božju stvoren je čovjek! 
\par 7 A vi, plodite se, i množite i zemlju napunite, i podložite  je sebi!" 
\par 8 Još reče Bog Noi i njegovim sinovima s njim: 
\par 9 "A ja, evo, sklapam svoj Savez s vama i s vašim potomstvom poslije  vas 
\par 10 i sa svim živim stvorovima što su s vama: s pticama,  sa stokom, sa zvijerima - sa svime što je s vama izišlo iz korablje  - sa svim živim stvorovima na zemlji. 
\par 11 Držat ću se ja svog  Saveza s vama te nikada više vode potopne neće uništiti živa  bića niti će ikad više potop zemlju opustošiti." 
\par 12 I reče Bog: "A ovo znamen je Saveza koji stavljam između sebe i vas i svih živih bića što su s vama, za naraštaje buduće: 
\par 13 Dugu svoju u oblak stavljam, da zalogom bude Savezu između mene i zemlje. 
\par 14 Kad oblake nad zemlju navučem i duga se u oblaku pokaže, 
\par 15 spomenut ću se Saveza svoga, Saveza između mene i vas i stvorenja svakoga živog: potopa više neće biti da uništi svako biće. 
\par 16 U oblaku kad se pojavi duga, ja ću je vidjeti i vjekovnog ću se sjećati Saveza između Boga i svake žive duše, svakog tijela na zemlji." 
\par 17 I reče Bog Noi: "To neka je znak Saveza koji sam postavio  između sebe i svih živih bića što su na zemlji." 
\par 18 Sinovi Noini, koji su iz korablje izišli, bijahu: Šem, Ham i Jafet. Ham je praotac Kanaanaca. 
\par 19 Ovo su trojica Noinih  sinova i od njih se sav svijet razgranao. 
\par 20 Noa, zemljoradnik, zasadio vinograd. 
\par 21 Napio se vina  i opio, pa se otkrio nasred šatora. 
\par 22 Ham, praotac Kanaanaca, opazi oca gola pa to kaza dvojici svoje braće vani. 
\par 23 Šem  i Jafet uzmu ogrtač, obojica ga prebace sebi preko ramena pa  njime, idući natraške, pokriju očevu golotinju. Lica im bijahu  okrenuta na drugu stranu, tako te ne vidješe oca gola. 
\par 24 Kad se Noa otrijeznio od vina i saznao što mu je učinio  najmlađi sin, reče: 
\par 25 "Neka je proklet Kanaanac, braći svojoj najniži sluga nek' bude!" 
\par 26 Onda nastavi: "Blagoslovljen Jahve, Šemov Bog, Kanaanac nek' mu je sluga! 
\par 27 Nek Bog raširi Jafeta da prebiva pod šatorima Šemovim, Kanaanac nek' mu je sluga!" 
\par 28 Poslije Potopa Noa poživje trista pedeset godina. 
\par 29 U  svemu poživje Noa devet stotina pedeset godina; potom umrije. 



\chapter{10}



\par 1 Ovo je povijest Noinih sinova: Šema, Hama i Jafeta, kojima  su se rodili sinovi poslije Potopa. 
\par 2 Sinovi su Jafetovi: Gomer, Magog, Madaj, Javan, Tubal, Mešak, Tiras. 
\par 3 A sinovi su Gomerovi: Aškenaz, Rifat i Togarma. 
\par 4 Javanovi su opet sinovi: Eliša, Taršiš, Kitijci i Dodanci. 
\par 5 Od njih su se razgranali narodi po otocima. To su Jafetovi sinovi prema svojim zemljama - svaki s vlastitim  jezikom - prema svojim plemenima i narodima. 
\par 6 Sinovi su Hamovi: Kuš i Misrajim, Put i Kanaan. 
\par 7 Kuševi  su: Seba, Havila, Sabta, Rama i Sabteka. Ramini su: Šeba i Dedan. 
\par 8 Od Kuša se rodio Nimrod, koji je postao prvi velmoža na  zemlji. 
\par 9 Voljom Jahve bio je silan lovac. Zato se veli: "Kao  Nimrod, silan lovac voljom Jahve." 
\par 10 Glavno uporište njegova  kraljevstva bili su: Babilon, Erek, Akad i Kalne, svi u zemlji  Šinearu. 
\par 11 Iz ove je zemlje došao Ašur. On je podigao Ninivu, Rehobot Ir, Kalah 
\par 12 i Resen između Ninive i Kalaha (to je  glavni grad). 
\par 13 Od Misrajima potekli su Ludijci, Anamijci, Lehabijci, Naftuhijci, 
\par 14 pa Patrušani, Kasluhijci i Kaftorci, od kojih  su potekli Filistejci. 
\par 15 Od Kanaana potječe Sidon, njegov prvenac, i Het. 
\par 16 Dalje:  Jebusejci, Amorejci, Girgašani, 
\par 17 Hivijci, Arkijci, Sinijci, 
\par 18 Arvađani, Semarjani i Hamaćani. Poslije se kanaanska plemena  razgranaše, 
\par 19 tako da se granica Kanaanaca protezala od Sidona  prema Geraru sve do Gaze pa prema Sodomi, Gomori, Admi i Sebojimu  sve do Leše. 
\par 20 To su sinovi Hamovi prema svojim plemenima i jezicima, po svojim zemljama i narodima. 
\par 21 A i Šemu - praocu svih sinova Eberovih i starijem bratu  Jafetovu - rodili se sinovi. 
\par 22 Šemovi su sinovi: Elam, Ašur, Arpakšad, Lud i Aram. 
\par 23 A Aramovi su sinovi: Us, Hul, Geter  i Maš. 
\par 24 Arpakšad rodi Šelaha, Šelah rodi Ebera. 
\par 25 Eberu  su se rodila dva sina: jednomu bješe ime Peleg, jer se za njegova  vijeka zemlja razdijelila. Njegovu je bratu bilo ime Joktan. 
\par 26 Od Joktana se rodiše: Almodad, Šelef, Hasarmavet, Jerah, 
\par 27 Hadoram, Uzal, Dikla, 
\par 28 Obal, Abimael, Šeba, 
\par 29 Ofir,  Havila i Jobab. Sve su to sinovi Joktanovi. 
\par 30 Njihova se naselja  protezahu od Meše sve do Sefara, brdovitih krajeva na istoku. 
\par 31 To su sinovi Šemovi prema svojim plemenima, jezicima  i zemljama, po svojim narodima. 
\par 32 To su rodovi Noinih sinova prema svojim lozama i narodima.  Od njih su se razgranali narodi po zemlji poslije Potopa. 



\chapter{11}



\par 1 Sva je zemlja imala jedan jezik i riječi iste. 
\par 2 Ali kako  su se ljudi selili s istoka, naiđu na jednu dolinu u zemlji Šinearu  i tu se nastane. 
\par 3 Jedan drugome reče: "Hajdemo praviti opeke  te ih peći da otvrdnu!" Opeke im bile mjesto kamena, a paklina  im služila za žbuku. 
\par 4 Onda rekoše: "Hajde da sebi podignemo  grad i toranj s vrhom do neba! Pribavimo sebi ime, da se ne raspršimo  po svoj zemlji!" 
\par 5 Jahve se spusti da vidi grad i toranj što su ga gradili  sinovi čovječji. 
\par 6 Jahve reče. "Zbilja su jedan narod, s jednim  jezikom za sve! Ovo je tek početak njihovih nastojanja. Sad im  ništa neće biti neostvarivo što god naume izvesti. 
\par 7 Hajde da  siđemo i jezik im pobrkamo, da jedan drugome govora ne razumije." 
\par 8 Tako ih Jahve rasu odande po svoj zemlji te ne sazidaše grada. 
\par 9 Stoga mu je ime Babel, jer je ondje Jahve pobrkao govor svima  u onom kraju i odande ih je Jahve raspršio po svoj zemlji. 
\par 10 Ovo su potomci Šemovi: Kad je Šemu bilo sto godina - dvije godine poslije Potopa  - rodi mu se Arpakšad. 
\par 11 Po rođenju Arpakšadovu Šem je živio  petsto godina te mu se rodilo još sinova i kćeri. 
\par 12 Kad je Arpakšadu bilo trideset i pet godina, rodi mu  se Šelah. 
\par 13 Po rođenju Šelahovu Arpakšad je živio četiri stotine  i tri godine te mu se rodilo još sinova i kćeri. 
\par 14 Kad je Šelahu bilo trideset godina, rodi mu se Eber. 
\par 15 Po rođenju Eberovu Šelah je živio četiri stotine i tri godine  te mu se rodilo još sinova i kćeri. 
\par 16 Kad su Eberu bile trideset i četiri godine, rodi mu se  Peleg. 
\par 17 Po rođenju Pelegovu Eber je živio četiri stotine i  trideset godina te mu se rodilo još sinova i kćeri. 
\par 18 Kad je Pelegu bilo trideset godina, rodi mu se Reu. 
\par 19 Po  rođenju Reuovu Peleg je živio dvjesta i devet godina te mu se  rodilo još sinova i kćeri. 
\par 20 Kad su Reuu bile trideset i dvije godine, rodi mu se  Serug. 
\par 21 Po rođenju Serugovu Reu je živio dvjesta i sedam godina  te mu se rodilo još sinova i kćeri. 
\par 22 Kad je Serugu bilo trideset godina, rodi mu se Nahor. 
\par 23 Po rođenju Nahorovu Serug je živio dvjesta godina te mu se  rodilo još sinova i kćeri. 
\par 24 Kad je Nahoru bilo dvadeset i devet godina, rodi mu se  Terah. 
\par 25 Po rođenju Terahovu Nahor je živio sto i devetnaest  godina te mu se rodilo još sinova i kćeri. 
\par 26 Kad je Terahu bilo sedamdeset godina, rode mu se: Abram, Nahor i Haran. 
\par 27 Ovo je povijest Terahova. Terahu se rodio Abram, Nahor  i Haran; a Haranu se rodio Lot. 
\par 28 Haran umrije za života svoga  oca Teraha, u svome rodnom kraju, u Uru Kaldejskom. 
\par 29 Abram  se i Nahor ožene. Abramovoj ženi bijaše ime Saraja, a Nahorovoj  Milka; ova je bila kći Harana, oca Milke i Jiske. 
\par 30 Saraja  bijaše nerotkinja - nije imala poroda. 
\par 31 Terah povede svoga sina Abrama, svog unuka Lota, sina  Haranova, svoju snahu Saraju, ženu svoga sina Abrama, pa se zaputi  s njima iz Ura Kaldejskoga u zemlju kanaansku. Kad stignu do  Harana, ondje se nastane. 
\par 32 Dob Terahova dosegnu dvjesta i pet godina; a onda Terah  umrije u Haranu. 



\chapter{12}



\par 1 Jahve reče Abramu: "Idi iz zemlje svoje, iz zavičaja i doma očinskog, u krajeve koje ću ti pokazati. 
\par 2 Velik ću narod od tebe učiniti, blagoslovit ću te, ime ću ti uzveličati, i sam ćeš biti blagoslov. 
\par 3 Blagoslivljat ću one koji te blagoslivljali budu, koji te budu kleli, njih ću proklinjati; sva plemena na zemlji tobom će se blagoslivljati." 
\par 4 Abram se zaputi kako mu je Jahve rekao. S njime krenu  i Lot. Abramu je bilo sedamdeset i pet godina kad je otišao iz  Harana. 
\par 5 Abram uze sa sobom svoju ženu Saraju, svoga bratića  Lota, svu imovinu što su je namakli i svu čeljad koju su stekli  u Haranu te svi pođu u zemlju kanaansku. Kad su stigli u Kanaan, 
\par 6 Abram prođe zemljom do mjesta Šekema - do hrasta More. Kanaanci  su onda bili u zemlji. 
\par 7 Jahve se javi Abramu pa mu reče: "Tvome  ću potomstvu dati ovu zemlju." Abram tu podigne žrtvenik Jahvi  koji mu se objavio. 
\par 8 Odatle prijeđe u brdoviti kraj, na istok  od Betela. Svoj šator postavi između Betela na zapadu i Aja na  istoku. Ondje podigne žrtvenik Jahvi i zazva ime Jahvino. 
\par 9 Od postaje do postaje Abram se pomicao prema Negebu. 
\par 10 Ali kad je zemljom zavladala glad, Abram se spusti u  Egipat da ondje proboravi, jer je velika glad harala zemljom. 
\par 11 Kad je bio na ulazu u Egipat, reče svojoj ženi Saraji: "Znam  da si lijepa žena. 
\par 12 Kad te Egipćani vide, reći će: 'To je  njegova žena', i mene će ubiti, a tebe na životu ostaviti. 
\par 13 Nego  reci da si mi sestra, tako da i meni bude zbog tebe dobro i da, iz obzira prema tebi, poštede moj život." 
\par 14 Zbilja, kad je Abram ušao u Egipat, Egipćani vide da  je žena veoma lijepa. 
\par 15 Vide je faraonovi dvorani pa je pohvale  faraonu i odvedu ženu na faraonov dvor. 
\par 16 Abramu pođe dobro  zbog nje; steče on stoke i goveda, magaraca, slugu i sluškinja, magarica i deva. 
\par 17 Ali Jahve udari faraona i njegov dom velikim nevoljama  zbog Abramove žene Saraje. 
\par 18 I faraon pozva Abrama pa reče:  "Što si mi to učinio? Zašto mi nisi kazao da je ona tvoja žena? 
\par 19 Zašto si rekao: 'Ona mi je sestra', pa je ja uzeh sebi za  ženu? A sad, evo ti žene; uzmi je i hajde!" 
\par 20 Faraon ga onda preda momcima, a oni ga otprave s njegovom  ženom i sa svime što bijaše njegovo. 



\chapter{13}



\par 1 Iz Egipta Abram ode gore u Negeb sa svojom ženom i sa svime  što je imao. I Lot bješe s njim. 
\par 2 Abram je bio veoma bogat  stokom, srebrom i zlatom. 
\par 3 Od postaje do postaje iz Negeba  išao je do Betela, 
\par 4 do mjesta na kojem je bio postavio šator, između Betela i Aja, gdje je prije podigao žrtvenik. Tu je Abram  zazivao ime Jahvino. 
\par 5 I Lot, koji iđaše s Abramom, imaše ovaca, goveda i šatora, 
\par 6 tako da ih kraj ne bi izdržavao kad bi zajedno ostali. Njihovo  je blago bilo veliko, te zajedno nisu mogli boraviti. 
\par 7 Svađa  je nastajala između pastira stoke Abramove i pastira stoke Lotove.  Tada su zemlju nastavali Kanaanci i Perižani. 
\par 8 Zato Abram reče  Lotu: "Neka ne bude svađe između mene i tebe, između pastira  mojih i tvojih - tÓa mi smo braća! 
\par 9 Nije li sva zemlja pred  tobom? Odvoji se od mene! Kreneš li ti nalijevo, ja ću nadesno;  ako ćeš ti nadesno, ja ću nalijevo." 
\par 10 Lot podiže oči i vidje kako je dobro posvuda natapana  sva Jordanska dolina, kao kakav vrt Jahvin, kao zemlja egipatska  prema Soaru. - Bilo je to prije nego što je Jahve uništio Sodomu  i Gomoru. - 
\par 11 Lot izabere za se svu Jordansku dolinu i ode  na istok. Tako se odijele jedan od drugoga. 
\par 12 Abram ostade  u kanaanskoj zemlji, dok je Lot živio po mjestima u dolini i  razapeo svoje šatore do Sodome. 
\par 13 A žitelji Sodome bijahu veoma  opaki, sami grešnici protiv Jahve. 
\par 14 Jahve reče Abramu, pošto se Lot od njega rastao: "Oči  svoje podigni i s mjesta na kojem si pogledaj prema sjeveru,  jugu, istoku i zapadu; 
\par 15 jer svu zemlju što je možeš vidjeti  dat ću tebi i tvome potomstvu zauvijek. 
\par 16 Potomstvo ću tvoje  učiniti kao prah na zemlji. Ako tko mogne prebrojiti prah zemlje, i tvoje će potomstvo moći prebrojiti. 
\par 17 Na noge! Prođi zemljom  uzduž i poprijeko jer ću je tebi predati." 
\par 18 Abram digne šatore i dođe pa se naseli kod hrasta Mamre, što je u Hebronu. Ondje podigne žrtvenik Jahvi. 



\chapter{14}



\par 1 Kad Amrafel bijaše kralj Šineara, Ariok kralj Elasara, Kedor-Laomer  kralj Elama, Tidal kralj Gojima, 
\par 2 povedoše oni rat protiv Bere, kralja Sodome, Birše, kralja Gomore, Šinaba, kralja Adme, Šemebera, kralja Sebojima, i protiv kralja u Beli, to jest Soaru. 
\par 3 I vojske se sliju u dolinu Sidim, gdje je danas Slano  more. 
\par 4 Dvanaest su godina služili Kedor-Laomera, ali trinaeste  godine dignu se na ustanak. 
\par 5 U četrnaestoj godini digne se  Kedor-Laomer i kraljevi koji su bili s njim te potuku Refaimce  u Ašterot Karnajimu, Zuzijce u Hamu, Emijce na ravnici Kirjatajimu, 
\par 6 Horijce u brdskom kraju Seiru, blizu El Parana, koji je uz  pustinju. 
\par 7 Onda se povuku natrag i stignu u En Mišpat, to jest  Kadeš, i pokore sve krajeve Amalečana i Amorejaca, koji su nastavali  Haseson Tamar. 
\par 8 Zatim istupi kralj Sodome, kralj Gomore, kralj  Adme, kralj Sebojima i kralj Bele, odnosno Soara, te zapodjenu  borbu protiv onih u dolini Sidimu: 
\par 9 Kedor-Laomera, kralja Elama, Tidala, kralja Gojima, Amrafela, kralja Šineara, Arioka, kralja  Elasara - četiri kralja protiv pet. 
\par 10 Dolina Sidim bila je puna provalija s paklinom, pa kraljevi  Sodome i Gomore, na bijegu, u njih poskaču, a ostali izmaknu  u planine. 
\par 11 Pobjednici pokupe sve blago po Sodomi i Gomori  i svu hranu pa odu. 
\par 12 Pograbe i Lota, Abramova bratića - i  on je živio u Sodomi - i njegovo blago pa otiđu. 
\par 13 A bjegunac neki - rođak Eškola i Anera, Abramovih saveznika  - donese vijest Abramu Hebrejcu dok je boravio kod hrasta Amorejske  Mamre. 
\par 14 Kad je Abram čuo da mu je bratić zarobljen, skupi  svoju momčad - rođenu u njegovu domu - njih trista osamnaest, pa pođe u potjeru do Dana. 
\par 15 Podijeli svoje momke u dvije  čete, napadne noću te one potuče. Progonio ih je do Hobe, sjeverno  od Damaska. 
\par 16 Povrati sve blago, svoga bratića Lota i njegovo  blago, žene i ostali svijet. 
\par 17 Pošto se vratio, porazivši Kedor-Laomera i kraljeve koji  su bili s njim, u susret mu, u dolinu Šave, to jest u Kraljev  dol, iziđe kralj Sodome. 
\par 18 A Melkisedek, kralj Šalema, iznese  kruha i vina. On je bio svećenik Boga Svevišnjega. 
\par 19 Blagoslovi  ga govoreći: "Od Boga Svevišnjega, Stvoritelja neba i zemlje, neka je Abramu blagoslov! 
\par 20 I Svevišnji Bog, što ti u ruke preda neprijatelje, hvaljen bio!" Abram mu dade desetinu od svega. 
\par 21 Tada kralj Sodome reče Abramu: "Meni daj ljude, a dobra  uzmi sebi!" 
\par 22 Abram odgovori kralju Sodome: "Ruku uzdižem pred  Jahvom, Svevišnjim Stvoriteljem neba i zemlje, 
\par 23 da neću uzeti  ni končića, ni remena od obuće, niti išta što je tvoje da ne  kažeš: na meni se Abram obogatio. 
\par 24 Ne, meni ništa, osim što  su moji momci upotrijebili; i dio za momčad što je sa mnom išla:  Aner, Eškol i Mamre, oni neka uzmu svoj dio." 



\chapter{15}



\par 1 Poslije tih događaja Jahve uputi Abramu riječ u ukazanju: "Ne boj se, Abrame, ja sam ti zaštita; a nagrada tvoja bit će vrlo velika!" 
\par 2 Abram odgovori: "Gospodine moj, Jahve, čemu mi tvoji darovi  kad ostajem bez poroda; kad je mojoj kući nasljednik Eliezer  Damaščanin? 
\par 3 Kako mi nisi dao potomstva - nastavi Abram - jedan  će, eto, od mojih ukućana postati moj baštinik." 
\par 4 Ali mu Jahve  opet uputi riječ: "Taj neće biti tvoj baštinik, nego će ti baštinik  biti tvoj potomak." 
\par 5 Izvede ga van i reče: "Pogledaj na nebo  i zvijezde prebroj ako ih možeš prebrojiti." A onda doda: "Toliko  će biti tvoje potomstvo." 
\par 6 Abram povjerova Jahvi, i on mu to  uračuna u pravednost. 
\par 7 Tada mu on reče: "Ja sam Jahve koji sam te odveo iz Ura  Kaldejskoga da ti predam ovu zemlju u posjed." 
\par 8 A on odvrati:  "Gospodine moj, Jahve, kako ću ja doznati da ću je zaposjesti?" 
\par 9 Odgovori mu: "Prinesi mi junicu od tri godine, kozu od tri  godine, ovna od tri godine, jednu grlicu i jednog golubića." 
\par 10 Sve mu to donese, rasiječe na pole i metnu sve pole jednu  prema drugoj; ptica nije rasijecao. 
\par 11 Ptice grabežljivice obarale  se na leševe, ali ih je Abram rastjerivao. 
\par 12 Kad je sunce bilo  pri zalazu, dubok san obuzme Abrama, a onda se na nj spusti gust  mrak pun jeze. 
\par 13 Tada Bog reče Abramu: "Dobro znaj da će tvoji  potomci biti stranci u tuđoj zemlji; robovat će i biti tlačeni  četiri stotine godina, 
\par 14 ali narodu kojem budu služili ja ću  suditi; i konačno će izići s velikim blagom. 
\par 15 A ti ćeš k ocima  svojim u miru poći, u sretnoj starosti bit ćeš sahranjen. 
\par 16 Oni  će se ovamo vratiti za četvrtog naraštaja, jer mjera se zlodjela  amorejskih još nije navršila." 
\par 17 Kad je sunce zašlo i pao gust mrak, pojavi se zadimljen  žeravnjak i goruća zublja te prođu između onih dijelova. 
\par 18 Toga  je dana Jahve sklopio Savez s Abramom rekavši: "Potomstvu tvojemu dajem zemlju ovu od Rijeke u Egiptu do Velike rijeke, rijeke Eufrata: 
\par 19 Kenijce, Kenižane, Kadmonce, 
\par 20 Hetite, Perižane, Refaimce, 
\par 21 Amorejce, Kanaance, Girgašane, Jebusejce." 



\chapter{16}



\par 1 Abramova žena Saraja nije mu rađala djece. A imaše ona sluškinju  Egipćanku - zvala se Hagara. 
\par 2 I reče Saraja Abramu: "Vidiš, Jahve me učinio nerotkinjom. Hajde k mojoj sluškinji, možda  ću imati djece." Abram posluša riječ Sarajinu. 
\par 3 Tako, pošto je Abram proboravio deset godina u zemlji  kanaanskoj, njegova žena Saraja uzme Hagaru, Egipćanku, sluškinju  svoju, pa je dade svome mužu Abramu za ženu. 
\par 4 Uđe on k Hagari  te ona zače. A kad je vidjela da je začela, s prezirom je gledala  na svoju gospodaricu. 
\par 5 Tada reče Saraja Abramu: "Nepravda što  se meni nanosi tvoja je krivnja! Prepustila sam svoju sluškinju  tvome zagrljaju, ali otkako opazi da je zanijela, s prezirom  na me gleda. Jahve sudio i meni i tebi!" 
\par 6 Nato Abram odvrati  Saraji: "Tvoja je sluškinja u tvojoj ruci: kako ti se čini da  je dobro, tako prema njoj postupi!" Saraja postupi prema njoj  tako loše da ona od nje pobježe. 
\par 7 Anđeo Jahvin nađe je kod izvora u pustinji - uz vrelo  što je na putu prema Šuru - 
\par 8 pa je zapita: "Hagaro, sluškinjo  Sarajina, odakle dolaziš i kamo ideš?" "Bježim, evo, od svoje  gospodarice Saraje", odgovori ona. 
\par 9 Nato joj anđeo Jahvin reče: "Vrati se svojoj gospodarici  i pokori joj se!" 
\par 10 Još joj reče anđeo Jahvin: "Tvoje ću potomstvo  silno umnožiti; od mnoštva se neće moći ni prebrojiti." 
\par 11 Dalje  joj je anđeo Jahvin rekao: "Gle, zanijela si i rodit ćeš sina. Nadjeni mu ime Jišmael, jer Jahve ču jad tvoj. 
\par 12 On će biti kao divlje magare: ruka će se njegova dizati na svakoga i svačija ruka na njega; i pred licem sve mu braće on će stan sebi podići." 
\par 13 A Jahvu koji joj govoraše nazva: "Ti si El Roi - Svevid  Bog", jer - reče ona - "vidjeh Boga i nakon viđenja - još živim!" 
\par 14 Stoga se taj zdenac zove Beer Lahaj Roi - Zdenac životvornog  Svevida, a eno ga između Kadeša i Bereda. 
\par 15 Rodi Hagara Abramu sina, a Abram sinu što mu ga rodi  Hagara nadjene ime Jišmael. 
\par 16 Abramu je bilo osamdeset i šest  godina kad mu je Hagara rodila Jišmaela. 



\chapter{17}



\par 1 Kad je Abramu bilo devedeset i devet godina, ukaza mu se Jahve  pa mu reče: "Ja sam El Šadaj - Bog Svesilni, Mojim hodi putem i neporočan budi. 
\par 2 A Savez svoj ja sklapam s tobom i silno ću te razmnožiti." 
\par 3 Abram pade ničice dok mu Bog govoraše dalje: 
\par 4 "A ovo je Savez moj s tobom: postat ćeš ocem mnogim narodima; 
\par 5 i nećeš se više zvati Abram - već Abraham će ti ime biti, jer naroda mnogih ocem ja te postavljam. 
\par 6 Silno ću te rodnim učiniti; narode ću iz tebe izvesti;  i kraljevi će od tebe izaći. 
\par 7 Savez svoj sklapam između sebe  i tebe i tvoga potomstva poslije tebe - Savez svoj za vjekove:  ja ću biti Bogom tvojim i tvoga potomstva poslije tebe. 
\par 8 Tebi  i tvome potomstvu poslije tebe dajem zemlju u kojoj boraviš kao  pridošlica - svu zemlju kanaansku - u vjekovni posjed; a ja ću  biti njihov Bog." 
\par 9 Još reče Bog Abrahamu: "A ti Savez čuvaj moj - ti i tvoje  potomstvo poslije tebe u sve vijeke. 
\par 10 A ovo je Savez moj s  tobom i tvojim potomstvom poslije tebe koji ćeš vršiti: svako  muško među vama neka bude obrezano. 
\par 11 Obrezujte se, i to neka  bude znak Saveza između mene i vas. 
\par 12 Svako muško među vama, kroz vaša pokoljenja, kad mu se navrši osam dana, neka bude  obrezano; i rob, rođen u vašem domu, i onaj što bude kupljen  od stranca, koji ne bude od vaše krvi. 
\par 13 Da, i rob rođen u  tvome domu ili za novac kupljen mora se obrezati! Tako će moj  Savez na vašem tijelu ostati vječnim Savezom. 
\par 14 Muško koje  se ne bi obrezalo neka se odstrani od svoga roda: takav je prekršio  moj Savez." 
\par 15 Još reče Bog Abrahamu: "Tvojoj ženi Saraji nije više  ime Saraja: Sara će joj ime biti. 
\par 16 Nju ću ja blagosloviti  i od nje ti dati sina; blagoslov ću na nju izliti te će se narodi  od nje razviti; kraljevi će narodima od nje poteći." 
\par 17 Abraham  pade ničice pa se nasmija i reče u sebi: "Onome komu je stotinu  godina, zar se može roditi dijete? Zar će Sara u devedesetoj  rod rađati!" 
\par 18 Abraham reče Bogu: "Neka tvojom milošću Jišmael  poživi!" 
\par 19 A Bog reče: "Ipak će ti tvoja žena Sara roditi sina;  nadjeni mu ime Izak. Savez svoj s njime ću sklopiti, Savez vječni  s njime i s njegovim potomstvom poslije njega. 
\par 20 I za Jišmaela  uslišah te. Evo ga blagoslivljam: rodnim ću ga učiniti i silno  ga razmnožiti; dvanaest će knezova od njega postati i u velik  će narod izrasti. 
\par 21 Ali ću držati svoj Savez s Izakom, koga  će ti roditi Sara dogodine u ovo doba." 
\par 22 Kad je završio razgovor  s njim, od Abrahama Bog se podiže. 
\par 23 Uzme zatim Abraham svoga sina Jišmaela i sve robove koji  su bili rođeni u njegovu domu i sve koje je kupio novcem - sve  muške ukućane - pa ih toga istog dana obreže, kako mu je Bog  rekao. 
\par 24 Abrahamu bijaše devedeset i devet godina kad se obrezao, 
\par 25 a njegovu sinu Jišmaelu bijaše trinaest godina kad ga obreza. 
\par 26 Tako su toga istog dana bili obrezani Abraham i njegov sin  Jišmael; 
\par 27 i svi muškarci njegova doma, rođeni u njegovoj kući  ili za novac kupljeni od stranca - svi s njim bijahu obrezani. 



\chapter{18}



\par 1 Jahve mu se ukaza kod hrasta Mamre dok je on sjedio na ulazu  u šator za dnevne žege. 
\par 2 Podigavši oči, opazi tri čovjeka gdje  stoje nedaleko od njega. Čim ih spazi, potrča s ulaza šatora  njima u susret. Pade ničice na zemlju 
\par 3 pa reče: "Gospodine  moj, ako sam stekao milost u tvojim očima, nemoj mimoići svoga  sluge! 
\par 4 Nek' se donese malo vode: operite noge i pod stablom  otpočinite. 
\par 5 Donijet ću kruha da se okrijepite prije nego pođete  dalje. TÓa k svome ste sluzi navratili." Oni odgovore: "Dobro, učini kako si rekao!" 
\par 6 Abraham se požuri u šator k Sari pa joj reče: "Brzo! Tri  mjerice najboljeg brašna! Zamijesi i prevrtu ispeci!" 
\par 7 Zatim  Abraham otrča govedima, uhvati tele, mlado i debelo, i dade ga  momku da ga brže zgotovi. 
\par 8 Poslije uzme masla, mlijeka i zgotovljeno  tele pa stavi pred njih, a sam stajaše pred njima, pod stablom, dok su blagovali. 
\par 9 "Gdje ti je žena Sara?" - zapitaju ga. "Eno je pod šatorom", odgovori. 
\par 10 Onda on reče: "Vratit ću se k tebi kad isteče  vrijeme trudnoće; a tvoja žena Sara imat će sina." Iza njega, na ulazu u šator, Sara je prisluškivala. 
\par 11 Abraham i Sara  bijahu u odmakloj dobi, ostarjeli. U Sare bijaše prestalo što  biva u žena. 
\par 12 Zato se u sebi Sara smijala i govorila: "Pošto  sam uvenula, sad da spoznam nasladu? A još mi je i gospodar star!" 
\par 13 Onda Jahve upita Abrahama: "A zašto se Sara smijala i govorila:  'Kako ću rod roditi ja starica?' 
\par 14 Zar je Jahvi išta nemoguće?  Navratit ću se k tebi kad isteče vrijeme trudnoće: Sara će imati  sina." 
\par 15 Sara se napravi nevještom govoreći: "Nisam se smijala."  Jer se prestrašila. Ali on reče: "Jesi, smijala si se!" 
\par 16 Ljudi ustanu i krenu put Sodome. Abraham pođe s njima  da ih isprati. 
\par 17 Jahve pomisli: "Zar da sakrivam od Abrahama  što ću učiniti 
\par 18 kad će od Abrahama nastati velik i brojan  narod te će se svi narodi zemlje njim blagoslivljati? 
\par 19 Njega  sam izlučio zato da pouči svoju djecu i svoju buduću obitelj  kako će hoditi putem Jahvinim, radeći što je dobro i pravedno, tako da Jahve mogne ostvariti što je Abrahamu obećao." 
\par 20 Onda  Jahve nastavi: "Velika je vika na Sodomu i Gomoru da je njihov  grijeh pretežak. 
\par 21 Idem dolje da vidim rade li zaista kako  veli tužba što je do mene stigla. Želim razvidjeti." 
\par 22 Odande ljudi krenu prema Sodomi, dok je Abraham još stajao  pred Jahvom. 
\par 23 Nato se Abraham primače bliže i reče: "Hoćeš  li iskorijeniti i nevinoga s krivim? 
\par 24 Možda ima pedeset nevinih  u gradu. Zar ćeš uništiti mjesto radije nego ga poštedjeti zbog  pedeset nevinih koji budu ondje? 
\par 25 Daleko to bilo od tebe da  ubijaš nevinoga kao i krivoga, tako da i nevini i krivi prođu  jednako! Daleko bilo od tebe! Zar da ni Sudac svega svijeta ne  radi pravo?" 
\par 26 "Ako nađem u gradu Sodomi pedeset nevinih",  odvrati Jahve, "zbog njih ću poštedjeti cijelo mjesto." 
\par 27 "Ja se, evo, usuđujem govoriti Gospodinu", opet progovori  Abraham. - "Ja, prah i pepeo! 
\par 28 Da slučajno bude nevinih pet  manje od pedeset, bi li uništio sav grad zbog tih pet?" "Neću  ga uništiti ako ih ondje nađem četrdeset i pet", odgovori. 
\par 29 "Ako  ih se ondje možda nađe samo četrdeset?" - opet će Abraham. "Neću  to učiniti zbog četrdesetorice", odgovori. 
\par 30 "Neka se Gospodin ne ljuti ako nastavim. Ako ih se ondje  nađe možda samo trideset?" - opet će on. "Neću to učiniti", odgovori, "ako ih ondje nađem samo trideset." 
\par 31 "Evo se opet usuđujem  govoriti Gospodinu", nastavi dalje. "Ako ih se slučajno ondje  nađe samo dvadeset?" "Neću ga uništiti", odgovori, "zbog dvadesetorice." 
\par 32 "Neka se Gospodin ne ljuti", on će opet, "ako rečem još samo  jednom: Ako ih je slučajno ondje samo deset?" "Neću ga uništiti  zbog njih deset", odgovori. 
\par 33 Kad je Jahve završio razgovor s Abrahamom, ode, a Abraham  se vrati u svoje mjesto. 



\chapter{19}



\par 1 Ona dva anđela stignu navečer u Sodomu dok je Lot sjedio na  vratima Sodome. Kad ih Lot ugleda, ustade i pođe im u susret.  Nakloni se licem do zemlje, 
\par 2 a onda im reče: "Molim, gospodo, svrnite u kuću svoga sluge da noć provedete i noge operete;  a onda možete na put rano." A oni rekoše: "Ne, noć ćemo provesti  na trgu." 
\par 3 Ali ih on uporno navraćaše, i oni se uvratiše k  njemu i uđoše u njegovu kuću. On ih ugosti, ispeče pogaču te  blagovaše. 
\par 4 Još ne bijahu legli na počinak, kad građani Sodome, mladi  i stari, sav narod do posljednjeg čovjeka, opkole kuću. 
\par 5 Zovnu  Lota pa mu reknu: "Gdje su ljudi što su noćas došli k tebi? Izvedi  nam ih da ih se namilujemo?" 
\par 6 Lot iziđe k njima na ulaz, a  za sobom zatvori vrata. 
\par 7 "Braćo moja," reče on, "molim vas, ne činite toga zla! 
\par 8 Imam, evo, dvije kćeri s kojima još čovjek  nije imao dodira: njih ću vam izvesti pa činite s njima što želite;  samo ovim ljudima nemojte ništa učiniti jer su došli pod sjenu  moga krova." 
\par 9 "Odstupi odatle!" - rekoše. - "Došao kao dotepenac, a za suca se već postavlja. Sad ćemo mi s tobom gore nego s  njima." I nasrnuše na jadnika Lota i navališe na vrata da ih  razbiju. 
\par 10 Ali ona dvojica pruže ruke van, povukoše Lota k  sebi u kuću i zatvore vrata; 
\par 11 a ljude pred vratima, mlade  i stare, zabliješte tako da nisu mogli naći vrata. 
\par 12 Onda ona dvojica upitaju Lota: "Koga još ovdje imaš:  sinove i kćeri, sve koje imaš u gradu iz mjesta izvedi! 
\par 13 Jer  mi ćemo zatrti ovo mjesto: vika je na njih pred Jahvom postala  tolika te nas Jahve posla da ga uništimo." 
\par 14 Iziđe Lot da to  kaže svojima budućim zetovima koji namjeravahu uzeti njegove  kćeri te reče: "Na noge! Odlazite iz ovog mjesta jer će Jahve  uništiti grad!" Ali je u očima svojih budućih zetova ispao kao  da zbija šalu. 
\par 15 Kako zora puče, anđeli navale na Lota govoreći: "Na noge!  Uzmi svoju ženu i svoje dvije kćeri koje su ovdje da ne budeš  zatrt kaznom grada!" 
\par 16 Ali on oklijevaše. Zato ga oni uzeše  za ruku, a tako i njegovu ženu i njegove dvije kćeri i - po smilovanju  Jahvinu nad njim - odvedoše ih i ostaviše izvan grada. 
\par 17 Kad  ih izvedoše u polje, jedan progovori: "Bježi da život spasiš!  Ne obaziri se niti se igdje u ravnici zaustavljaj! Bježi u brdo  da ne budeš zatrt!" 
\par 18 Ali Lot odvrati: "Nemoj, gospodine! 
\par 19 Nego  ako je tvoj sluga našao milost u tvojim očima - a toliko milosrđe  već si mi iskazao spasivši mi život - ja ne mogu pobjeći u brdo  a da me nesreća ne snađe i ne poginem. 
\par 20 Eno onamo grada; dosta  je blizu da u nj pobjegnem, a mjesto je tako malo. Daj da onamo  bježim - mjesto je zbilja maleno - daj da život spasim!" 
\par 21 Odgovori  mu: "Uslišat ću ti i tu molbu i neću zatrti grada o kojemu govoriš. 
\par 22 Brzo! Bježi onamo, jer ne mogu ništa činiti dok ti onamo  ne stigneš." Zato se onaj grad zove Soar. 
\par 23 Kako je sunce na zemlju izlazilo i Lot ulazio u Soar, 
\par 24 Jahve zapljušti s neba na Sodomu i Gomoru sumpornim ognjem 
\par 25 i uništi one gradove i svu onu ravnicu, sve žitelje gradske  i sve raslinstvo na zemlji. 
\par 26 A Lotova se žena obazre i pretvori  se u stup soli. 
\par 27 Sutradan u rano jutro Abraham se požuri na mjesto gdje  je stajao pred Jahvom, 
\par 28 upravi pogled prema Sodomi i Gomori  i svoj ravnici u daljini: i vidje kako se diže dim nad zemljom  kao dim kakve klačine. 
\par 29 Tako se Bog, dok je zatirao gradove u ravnici u kojima  je Lot boravio, sjetio Abrahama i uklonio Lota ispred propasti. 
\par 30 Lot se bojao boraviti u Soaru, pa sa svoje dvije kćeri  ode gore iz Soara i nastani se u brdu. On i njegove dvije kćeri  živjeli su u pećini. 
\par 31 Starija reče mlađoj: "Otac nam ostarje, a muža na zemlji nema da bude s nama, kako je običaj po svem  svijetu. 
\par 32 Hajdemo oca opiti vinom, pa s njime leći: tako ćemo  s ocem sačuvati potomstvo." 
\par 33 One noći opiju oca vinom, i starija  ode te legne sa svojim ocem, a on nije znao kad je legla ni kad  je ustala. 
\par 34 Sutradan starija reče mlađoj: "Sinoć sam, eto, ležala ja s našim ocem; napojimo ga vinom i noćas, pa idi ti  i s njim lezi: tako ćemo ocu sačuvati potomstvo." 
\par 35 Opiju oca  vinom i one noći te mlađa ode i s njim legne, a on nije znao  kad je legla ni kad je ustala. 
\par 36 Tako obje Lotove kćeri zanesu  s ocem. 
\par 37 Starija rodi sina i nadjenu mu ime Moab. On je praotac  današnjih Moabaca. 
\par 38 I mlađa rodi sina i nadjene mu ime Ben-Ami.  On je praotac današnjih Amonaca. 



\chapter{20}



\par 1 Odande Abraham krene u krajeve Negeba i nastani se između  Kadeša i Šura. Dok je boravio kao pridošlica u Geraru, 
\par 2 rekao  je Abraham za svoju ženu Saru da mu je sestra. I Abimelek, kralj  gerarski, uze Saru sebi. 
\par 3 Ali Bog dođe Abimeleku noću u snu  te mu reče: "Zbog žene koju si uzeo moraš umrijeti, jer je ona  žena udata." 
\par 4 A nije se Abimelek k njoj približavao. Zato reče:  "Gospodine, zar ćeš pravednika pogubiti? 
\par 5 Zar mi on nije rekao:  'Ona mi je sestra.' A ona mi je sama rekla: 'On je moj brat.'  Čiste sam savjesti i neokaljanih ruku ovo učinio." 
\par 6 Bog mu  odvrati u snu: "Znam da si to učinio čiste savjesti; i ja sam  te zadržavao da protiv mene ne griješiš; i nisam dopuštao da  je dotakneš. 
\par 7 Sada vrati čovjeku ženu njegovu; prorok je on;  molit će se za tebe da ostaneš na životu. Ako je ne vratiš, znaj  da ćeš umrijeti, ti i svi tvoji." 
\par 8 Rano ujutro Abimelek ustane, sazove sve svoje sluge i  kaže im sve što je bilo, a ljudi se veoma uplaše. 
\par 9 Potom Abimelek  dozva Abrahama te mu reče: "Što si nam učinio! Čime sam se ja  ogriješio prema tebi da izložiš mene i moje kraljevstvo velikoj  grehoti? Ponio si se prema meni kako ne valja. 
\par 10 Što si, dakle  na umu imao", upita dalje Abimelek, "kad si tako radio?" 
\par 11 Abraham  uzvrati: "Zbilja sam držao da nema Božjeg straha u ovome mjestu, pa će me ljudi ubiti zbog moje žene. 
\par 12 A onda, ona je uistinu  moja sestra: kći je moga oca, iako ne i moje majke, pa je pošla  za me. 
\par 13 A kad me Bog udaljio od doma očeva, rekoh joj: Ovu  mi uslugu učini: kamo god dođemo, reci o meni da sam ti brat." 
\par 14 Abimelek uzme ovaca i goveda, sluga i sluškinja pa ih  dade Abrahamu; vrati mu i njegovu ženu Saru. 
\par 15 Abimelek zatim  reče: "Evo, moja ti je zemlja otvorena. Nastani se gdje ti se  svidi!" 
\par 16 A Sari reče: "Evo tisuću srebrnika što ih dajem tvome  bratu: neka ti budu koprenom pred očima sviju što su s tobom.  Ti si svakako opravdana." 
\par 17 Abraham se pomoli Bogu, i Bog ozdravi  Abimeleka, njegovu ženu i njegove sluškinje, tako te opet mogahu  rađati. 
\par 18 Jer Jahve bijaše zbog Sare, Abrahamove žene, zatvorio  svaku utrobu u domu Abimelekovu. 



\chapter{21}



\par 1 Jahve se sjeti Sare kako je rekao i učini joj kako je obećao: 
\par 2 Sara zače i rodi Abrahamu sina u njegovoj starosti - u vrijeme  koje je Bog označio. 
\par 3 Abraham nadjene ime Izak svome sinu što  mu ga Sara rodi. 
\par 4 I poslije osam dana obreza Abraham svoga  sina Izaka, kako mu je Bog naredio. 
\par 5 Abrahamu bijaše stotinu  godina kad mu se rodio sin Izak. 
\par 6 Sara reče: "Dade mi Bog da se nasmijem, i tko god to čuje  nasmijat će mi se." 
\par 7 Još doda: "Tko bi ikad rekao Abrahamu: djecu će ti Sara dojiti! Ipak sina mu rodih u starosti". 
\par 8 Dijete je raslo i bilo od sise odbijeno. A u dan u koji  Izak bijaše od sise odbijen Abraham priredi veliku gozbu. 
\par 9 Jednom opazi Sara gdje se sin koga je Egipćanka Hagara  Abrahamu rodila igra s njezinim sinom Izakom, 
\par 10 pa reče Abrahamu:  "Otjeraj tu sluškinju i njezina sina, jer sin sluškinje ne smije  biti baštinik s mojim sinom - s Izakom!" 
\par 11 To je Abrahamu bilo nemilo, jer je i Jišmael bio njegov  sin. 
\par 12 Ali Bog reče Abrahamu: "Nemoj se uznemirivati zbog dječaka  i zbog svoje sluškinje; sve što ti kaže Sara poslušaj, jer će  Izakovo potomstvo tebi ovjekovječiti ime. 
\par 13 I od sina tvoje  sluškinje podići ću velik narod, jer je tvoj potomak." 
\par 14 Rano ujutro Abraham uze kruha i mješinicu vode pa dade  Hagari; stavi to na njezina ramena, zajedno s dječakom, te je  otpusti. Vrludala je amo-tamo po pustinji Beer Šebe. 
\par 15 Potrošivši  vodu iz mješinice, ostavi dijete pod jednim grmom, 
\par 16 a sama  ode i sjede nasuprot, daleko koliko luk može dobaciti. Govorila  je u sebi: "Neću da vidim kako dijete umire." Sjedeći tako, udari  u jecanje. 
\par 17 Bog ču plač dječaka te anđeo Božji zovne s neba Hagaru  i reče joj: "Što ti je, Hagaro? Ne boj se! Jer je Bog čuo plač  dječaka u njegovoj nevolji. 
\par 18 Na noge! Digni dječaka i utješi  ga, jer od njega ću podići velik narod." 
\par 19 Tada joj Bog otvori  oči pa ona opazi studenac. Ode i napuni vodom mješinicu pa napoji  dječaka. 
\par 20 Bog je bio s dječakom te je rastao i odrastao. Živio  je u pustinji te postao vješt u strijeljanju iz luka. 
\par 21 Dom  mu bijaše u pustinji Paranu; a njegova mu majka dobavi ženu iz  zemlje egipatske. 
\par 22 U to vrijeme Abimelek - koga je pratio Fikol, zapovjednik  njegove vojske - reče Abrahamu: "Bog je s tobom u svemu što radiš. 
\par 23 Stoga mi se ovdje i sada zakuni Bogom da nećeš varati ni  mene ni moju rodbinu i prijatelje nego da ćeš se prema meni i  prema zemlji u kojoj sad boraviš ponašati pošteno, kao što sam  se ja prema tebi ponio." 
\par 24 "Kunem se", odgovori Abraham. 
\par 25 Onda Abraham prekori Abimeleka zbog zdenca vode što su  ga Abimelekove sluge bile prisvojile. 
\par 26 A Abimelek reče: "Ne  znam tko je to učinio; ni ti me nisi o tome obavijestio, niti  sam ja o tome čuo, osim danas." 
\par 27 Abraham uzme ovaca i goveda  pa ih dade Abimeleku te njih dvojica sklope savez. 
\par 28 Potom  Abraham razluči napose sedam janjaca od stada. 
\par 29 Nato Abimelek  zapita Abrahama: "Što znači ovih sedam janjaca koje si na stranu  stavio?" 
\par 30 A on odgovori: "Primi ovih sedam janjaca iz moje  ruke da mi bude dokazom da sam ja iskopao ovaj zdenac." 
\par 31 Zato  se ono mjesto nazvalo Beer Šeba; jer se njih dvojica ondje zakleše. 
\par 32 Pošto su sklopili savez kod Beer Šebe, Abimelek i zapovjednik  njegove vojske Fikol odu i vrate se u zemlju Filistejaca. 
\par 33 Abraham  zasadi kod Beer Šebe tamarisku i ondje zazove ime Jahve - Boga  Vječnoga. 
\par 34 Dugo je vremena Abraham proveo u zemlji filistejskoj  kao pridošlica. 



\chapter{22}



\par 1 Poslije tih događaja Bog stavi Abrahama na kušnju. Zovnu ga:  "Abrahame!" On odgovori: "Evo me!" 
\par 2 Bog nastavi: "Uzmi svoga  sina, jedinca svoga Izaka koga ljubiš, i pođi u krajinu Moriju  pa ga ondje prinesi kao žrtvu paljenicu na brdu koje ću ti pokazati." 
\par 3 Ujutro Abraham podrani, osamari magarca, sa sobom povede  dvojicu svojih slugu i svog sina Izaka, pošto je prije nacijepao  drva za žrtvu paljenicu, i uputi se na mjesto koje mu je Bog  označio. 
\par 4 Treći dan Abraham podigne oči i opazi mjesto izdaleka. 
\par 5 Abraham onda reče slugama: "Vi ostanite ovdje uz magarca,  a ja i dječak odosmo gore da se poklonimo, pa ćemo se vratiti  k vama". 
\par 6 Abraham uzme drva za žrtvu paljenicu, stavi ih na  sina Izaka, a u svoju ruku uzme kremen i nož. Tako pođu obojica  zajedno. 
\par 7 Onda Izak reče svome ocu Abrahamu: "Oče!" "Evo me, sine!" - javi se on. "Evo kremena i drva," opet će sin, "ali  gdje je janje za žrtvu paljenicu?" 
\par 8 "Bog će već providjeti  janje za žrtvu paljenicu, sine moj!" - odgovori Abraham. I nastave  put. 
\par 9 Stignu na mjesto o kojemu je Bog govorio. Ondje Abraham  podigne žrtvenik, naslaže drva, sveže svog sina Izaka i položi  ga po drvima na žrtvenik. 
\par 10 Pruži sad Abraham ruku i uzme nož  da zakolje svog sina. 
\par 11 Uto ga zovne s neba anđeo Jahvin i  poviče: "Abrahame! Abrahame!" "Evo me!" - odgovori on. 
\par 12 "Ne  spuštaj ruku na dječaka", reče, "niti mu što čini! Sad, evo,  znam da se Boga bojiš, jer nisi uskratio ni svog sina, jedinca  svoga." 
\par 13 Podiže Abraham oči i pogleda, i gle - za njim ovan, rogovima se zapleo u grmu. Tako Abraham ode, uzme ovna i prinese  ga za žrtvu paljenicu mjesto svoga sina. 
\par 14 Onome mjestu Abraham  dade ime "Jahve proviđa". Zato se danas veli: "Na brdu Jahvina  proviđanja." 
\par 15 Anđeo Jahvin zovne Abrahama s neba drugi put 
\par 16 i reče:  "Kunem se samim sobom, izjavljuje Jahve: Kad si to učinio i nisi  mi uskratio svog jedinca sina, 
\par 17 svoj ću blagoslov na te izliti  i učiniti tvoje potomstvo brojnim poput zvijezda na nebu i pijeska  na obali morskoj! A tvoji će potomci osvajati vrata svojih neprijatelja. 
\par 18 Budući da si poslušao moju zapovijed, svi će se narodi zemlje  blagoslivljati tvojim potomstvom." 
\par 19 Zatim se Abraham vrati k svojim slugama pa se zajedno  upute u Beer Šebu. U Beer Šebi se Abraham nastani. 
\par 20 Poslije tih događaja obavijeste Abrahama: "I tvome bratu  Nahoru Milka je porodila djecu: 
\par 21 njegova prvorođenca Usa,  brata mu Buza i Kemuela - oca Aramova, 
\par 22 Keseda, Haza, Pildaša, Jidlafa i Betuela." 
\par 23 Betuel je bio otac Rebekin. Njih je  osam rodila Milka Nahoru, Abrahamovu bratu. 
\par 24 A i njegova suložnica, kojoj bijaše ime Reuma, rodila je Tebaha, Gahama, Tahaša i Maaku. 



\chapter{23}



\par 1 Duljina Sarina života bila je stotinu dvadeset i sedam godina. 
\par 2 Sara umrije u Kirjat Arbi, to jest u Hebronu, u zemlji kanaanskoj;  i Abraham uđe u žalost za Sarom i naricaše za njom. 
\par 3 Potom se Abraham digne ispred svoje pokojnice te prozbori  sinovima Hetovim: 
\par 4 "Premda sam ja među vama doseljeni stranac, prodajte mi zemljište za grob među vama, tako da mogu iznijeti  svoju pokojnicu i sahraniti je." 
\par 5 A sinovi Hetovi odgovore  Abrahamu: 
\par 6 "Gospodine, saslušaj nas! Ti si izabranik Božji  u našoj sredini. Pokopaj svoju pokojnicu u našem najbiranijem  grobu. Nitko ti od nas neće odbiti svoga groba da mogneš sahraniti  svoju pokojnicu." 
\par 7 Nato se Abraham diže pa se mještanima, sinovima  Hetovim, duboko pokloni 
\par 8 te im reče: "Ako se slažete da svoju  pokojnicu uklonim i sahranim, čujte me: zauzmite se za me kod  Efrona, sina Soharova, 
\par 9 da mi proda spilju Makpelu što njemu  pripada a nalazi se na kraju njegova posjeda; neka mi je za punu  cijenu, u vašoj nazočnosti, proda u vlasništvo za sahranjivanje." 
\par 10 A Efron je sjedio sa sinovima Hetovim. Potom Efron, Hetit, odgovori Abrahamu da ga čuju sinovi Hetovi svojim ušima - svi  koji su sjedili u vijeću onoga grada: 
\par 11 "Ne, moj gospodine!  Saslušaj mene! Ja tebi dajem poljanu i spilju što je na njoj;  darujem ti to pred sinovima svoga naroda. Sahrani svoju pokojnicu." 
\par 12 Abraham se duboko nakloni mještanima, 
\par 13 a onda progovori  Efronu da mještani čuju na svoje uši: "Ded me samo poslušaj!  Dajem ti cijenu za poljanu; primi je od mene da ondje mogu sahraniti  svoju pokojnicu!" 
\par 14 Efron odgovori Abrahamu: 
\par 15 "Čuj me, moj  gospodine: zemljište u vrijednosti od četiri stotine srebrnika, što je to tebi i meni! Sahrani, dakle, svoju pokojnicu!" 
\par 16 Abraham  se složi s Efronom; isplati Abraham Efronu novac što ga je spomenuo  tako da su na svoje uši čuli sinovi Hetovi - četiri stotine srebrnika  trgovačke mjere. 
\par 17 I tako Efronova poljana u Makpeli nasuprot Mamri - poljana, spilja i sva stabla što su bila na poljani - 
\par 18 prijeđe u vlasništvo  Abrahamovo u nazočnosti sinova Hetovih, sviju koji su sjedili  u vijeću svoga grada. 
\par 19 A onda Abraham sahrani svoju ženu Saru  u spilji na poljani Makpeli nasuprot Mamri - danas Hebronu -  u zemlji kanaanskoj. 
\par 20 Tako je poljana i spilja na njoj prešla  od sinova Hetovih u vlasništvo Abrahamovo za sahranjivanje. 



\chapter{24}



\par 1 Abraham bijaše već ostario, zašao u godine, Jahve je Abrahama  blagoslovio u svemu. 
\par 2 Abraham prozbori svome najstarijem sluzi u kući, pod čijom  je upravom bilo sve njegovo: "Stavi svoju ruku pod moje stegno 
\par 3 da te zakunem Jahvom, Bogom neba i Bogom zemlje, da mome sinu  nećeš nabaviti za ženu ni jednu od kćeri Kanaanaca, među kojima  boravim, 
\par 4 nego ćeš otići u moj rodni kraj i dobaviti ženu mom  sinu Izaku." 
\par 5 A sluga mu reče: "A što ako žena ne htjedne za  mnom ići u ovu zemlju? Hoću li ja onda odvesti tvoga sina u zemlju  iz koje si ti došao?" 
\par 6 Abraham mu odgovori: "Dobro pripazi  da onamo ne vodiš moga sina! 
\par 7 Jahve, Bog nebesa, koji me odveo  iz kuće moga oca i rodnog kraja i koji mi je pod zakletvom obećao:  'Tvome ću potomstvu dati ovu zemlju', pred tobom će poslati svog  anđela, i odande ćeš ti dovesti ženu mome sinu. 
\par 8 A ako žena  ne bude htjela za tobom poći, ti ćeš biti oslobođen od ove moje  zakletve; ali moga sina onamo ne vodi!" 
\par 9 Tako sluga stavi svoju  ruku pod stegno Abrahamu, svom gospodaru, te mu se zakune. 
\par 10 Sluga opremi deset gospodarevih deva, ponese sa sobom  svakog blaga svoga gospodara pa se zaputi u Aram Naharajim, u  Nahorov grad. 
\par 11 Pusti deve da poliježu izvan grada, pokraj  studenca. Bijaše večer, kad žene izlaze da crpu vodu. 
\par 12 Onda  reče: "Oh, Jahve, Bože moga gospodara Abrahama, molim te, iziđi  mi danas u susret i mome gospodaru Abrahamu milost iskaži! 
\par 13 Evo  me kraj studenca, a kćeri onih iz grada dolaze crpsti vodu; 
\par 14 pa  neka djevojka kojoj ja rečem: 'Molim te, spusti svoj vrč da se  napijem', a ona odgovori: 'Pij! I deve ću ti napojiti', bude  ona koju si odredio za svoga slugu Izaka. Tako ću saznati da  si iskazao milost mome gospodaru." 
\par 15 Tek što on izreče svoje, gle, dođe Rebeka, kći Betuelova;  taj Betuel bijaše sin Milke, žene Abrahamova brata Nahora. Dođe  ona s krčagom na ramenu. 
\par 16 Djevojka je bila krasna, djevica  koju muškarac nije dirnuo. Siđe ona k vrelu, napuni krčag i eto  je opet gore. 
\par 17 Sluga joj potrča u susret i reče: "Daj mi malo  vode iz svog vrča!" 
\par 18 "Pij, gospodine!" - odgovori ona. Brzo  spusti krčag na ruku i dade mu piti. 
\par 19 Kad je njega napojila, reče: "Nalit ću i tvojim devama da se napoje." 
\par 20 Izlivši brzo  krčag u korito, otrča natrag zdencu da ponovo zahvaća, i tako  nali svim njegovim devama. 
\par 21 Čovjek ju je šutke motrio ne bi  li saznao je li Jahve njegov put uspješno priveo kraju ili nije. 
\par 22 Kad su deve prestale piti, čovjek izvadi viticu od zlata, tešku pol šekela, i stavi je na njezine nosnice, a na ruke joj  stavi dvije zlatne narukvice, teške deset šekela. 
\par 23 Zatim reče:  "Kaži mi čija si kći. Ima li u kući tvoga oca mjesta za nas da  prenoćimo?" 
\par 24 Ona mu odgovori: "Ja sam kći Betuela, koga je  Milka rodila Nahoru." 
\par 25 Još mu doda: "Ima slame i p§iće kod  nas u obilju, a i mjesta za prenoćište." 
\par 26 Čovjek se onda duboko  nakloni te iskaže poštovanje Jahvi 
\par 27 i progovori: "Neka je  blagoslovljen Jahve, Bog moga gospodara Abrahama, što nije uskratio  svoju ljubav i svoju vjernost mome gospodaru. Mene je Jahve vodio  pravim putem, u kuću brata moga gospodara." 
\par 28 Djevojka otrča i sve ovo ispripovjedi u kući svoje majke. 
\par 29 A Rebeka imala brata komu bijaše ime Laban. Laban se požuri  van, k čovjeku kod studenca. 
\par 30 Čim je vidio nosnu viticu i  narukvice na rukama svoje sestre te čuo kako je njegova sestra  Rebeka rekla: "Ovako mi je čovjek govorio", on pođe onome koji  je još stajao kod deva na studencu. 
\par 31 Reče on: "Hajde unutra, blagoslovljeni od Jahve! Što stojiš vani kad sam ja spremio  kuću i mjesto za deve." 
\par 32 Tako čovjek uđe u kuću. Rastovare  deve i dadu im slame i p§iće, a njemu i ljudima koji su ga pratili  donesu vode da operu noge. 
\par 33 Ali kad su preda nj stavili hranu, reče: "Neću jesti dok ne kažem što imam kazati." A Laban mu  reče: "Onda kazuj!" 
\par 34 "Ja sam sluga Abrahamov", poče on. 
\par 35 "Jahve je uvelike  blagoslovio moga gospodara te je postao bogat. Nadavao mu je  ovaca i goveda, srebra i zlata, sluga i sluškinja, deva i magaradi. 
\par 36 Sara, žena moga gospodara, rodi mu sina pošto je ostarjela, i on mu ustupi sve svoje. 
\par 37 Potom mene moj gospodar zakune  rekavši: 'Nemoj uzeti za ženu mome sinu djevojku Kanaanku, u  zemlji u kojoj boravim kao stranac, 
\par 38 nego otiđi k obitelji  moga oca, k mojoj rodbini, da nađeš ženu mome sinu.' 
\par 39 A ja  rekoh svome gospodaru: 'A što ako žena za mnom ne pođe?' 
\par 40 On  mi odgovori: 'Jahve, pred čijim sam licem hodio, poslat će s  tobom svog anđela i tvoje će putovanje dovesti k cilju, a ti  ćeš naći ženu mome sinu od moje rodbine, od obitelji moga oca. 
\par 41 Jedino ćeš ovako biti oslobođen moje zakletve: ako dođeš  k mojoj rodbini, i oni te odbiju, od moje si zakletve oslobođen.' 
\par 42 Danas dođoh na studenac i rekoh: 'Jahve, Bože moga gospodara  Abrahama, ako si voljan da uspješno završim putovanje što sam  ga poduzeo, 
\par 43 ja, evo, stojim kraj studenca, a djevojka koja  dođe vodu crpsti i ja joj rečem: Daj mi da se napijem malo vode  iz tvog vrča! - 
\par 44 i koja mi kaže: Pij ti, a i tvojim ću devama  zahvatiti! - ona neka bude žena koju je Jahve odredio sinu moga  gospodara.' 
\par 45 Tek što sam ja završio govor u sebi, kad se,  evo, pojavi Rebeka s vrčem na ramenu; siđe k izvoru i zahvati.  Ja joj rekoh: 'Daj mi da se napijem!' 
\par 46 Ona brzo spusti vrč  i odvrati: 'Pij! A napojit ću i tvoje deve.' Tako sam se ja napio, a ona napoji i moje deve. 
\par 47 Pitao sam je: 'Čija si kći?' Odgovorila  je: 'Kći sam Betuela, koga je Nahoru rodila Milka.' Tada joj  stavim viticu na nos a narukvice na ruke. 
\par 48 Duboko se naklonim  i štovanje Jahvi iskažem te blagoslovim Jahvu, Boga gospodara  moga, koji me vodio pravim putem da uzmem kćer brata moga gospodara  njegovu sinu. 
\par 49 A sad, ako kanite iskazati ljubav i vjernost  mome gospodaru, recite mi; ako li ne, to mi kažite, tako da mogu  krenuti bilo desno bilo lijevo." 
\par 50 Tada odgovore Laban i Betuel: "Od Jahve to dolazi; mi  tu ne možemo reći ni da ni ne. 
\par 51 Rebeka je, eto, pred tobom:  uzmi je pa idi, neka bude ženom sinu tvoga gospodara, kako je  Jahve rekao." 
\par 52 Kad Abrahamov sluga ču njihov pristanak, do  zemlje se nakloni Jahvi. 
\par 53 Sluga zatim izvadi srebrnih i zlatnih  predmeta te haljina i dade ih Rebeki, a dade darova i njezinu  bratu i majci. 
\par 54 Tada jedoše i piše on i ljudi koji su bili s njim i provedoše  noć. Kad su ujutro ustali, on reče: "Pustite me da se vratim  svome gospodaru!" 
\par 55 A njezin brat i majka odgovore: "Neka djevojka  ostane s nama još desetak dana, pa poslije toga pođi!" 
\par 56 On  im reče: "Ne zadržavajte me kad je Jahve moje putovanje uspješno  kraju priveo. Pustite me da se vratim svome gospodaru!" 
\par 57 Oni  odgovore: "Pozovimo djevojku i upitajmo što ona misli!" 
\par 58 Dozovu  Rebeku pa je upitaju: "Hoćeš li poći s ovim čovjekom?" Ona odgovori:  "Hoću." 
\par 59 I tako otpreme svoju sestru Rebeku i njezinu dojilju  s Abrahamovim slugom i njegovim ljudima. 
\par 60 Blagoslove Rebeku  i reknu joj: "Sejo naša, budi mati nebrojenim tisućama, a dušmana svojih vrata potomci ti zaposjeli!" 
\par 61 Onda se diže Rebeka i njezine dvorkinje, zajahaše deve  te pođoše za čovjekom. Tako sluga preuze Rebeku i ode. 
\par 62 Izak se vratio iz blizine Beer Lahaj Roja; živio je,  naime, u kraju Negeba. 
\par 63 U predvečerje iziđe Izak da se poljem  prošeta; diže oči i ugleda deve gdje dolaze. 
\par 64 Kad Rebeka,  podigavši svoje oči, opazi Izaka, sjaha s deve 
\par 65 pa zapita  slugu: "Tko je onaj čovjek što poljem ide nama u susret?" A sluga  odgovori: "Ono je moj gospodar." Nato ona uze koprenu te se pokri. 
\par 66 Sluga ispriča Izaku sve što je učinio. 
\par 67 Tada Izak  uvede Rebeku u svoj šator i uze je sebi za ženu. U ljubavi prema  njoj Izak je nalazio utjehu nakon smrti svoje majke. 



\chapter{25}



\par 1 Abraham je sebi uzeo još jednu ženu; zvala se Ketura. 
\par 2 Ona  mu je rodila Zimrana, Jokšana, Medana, Midjana, Jišbaka i Šuaha. 
\par 3 A od Jokšana rodili se Šeba i Dedan. Dedanovi su potomci:  Ašurci, Letušci i Leumci. 
\par 4 Sinovi su Midjanovi: Efa, Efer,  Hanok, Abida i Eldaa. Sve su to potomci Keturini. 
\par 5 Abraham prenese sav svoj imutak na Izaka; 
\par 6 a sinovima  od svojih suložnica dade Abraham samo darove i još ih za svoga  života razašalje po istoku - daleko od svog sina Izaka - u Istočni  kraj. 
\par 7 Ovo je duljina Abrahamova života što ga je proživio:  stotinu sedamdeset i pet godina. 
\par 8 Zatim Abraham preminu, umrije  u sretnoj dobi - star i pun godina - te bi pridružen svojim precima. 
\par 9 Njegovi sinovi, Izak i Jišmael, sahrane ga u spilji Makpeli, na poljani Efrona, sina Hetita Sohara, nasuprot Mamri: 
\par 10 to  je poljana što ju je Abraham kupio od Hetovih sinova. Ondje je  sahranjen Abraham i njegova žena Sara. 
\par 11 Poslije Abrahamove  smrti Bog je blagoslivljao njegova sina Izaka. Izak je živio  blizu Beer Lahaj Roja. 
\par 12 Ovo je povijest Abrahamova sina Jišmaela, koga je Abrahamu  rodila Sarina sluškinja, Egipćanka Hagara. 
\par 13 A ovo su sinovi  Jišmaelovi, svaki po svom imenu i po svom rođenju: Jišmaelov  prvenac Nebajot, Kedar, Adbeel, Mibsam, 
\par 14 Mišma, Duma, Masa, 
\par 15 Hadad, Tema, Jetur, Nafiš i Kedma. 
\par 16 To su Jišmaelovi sinovi i to su njihova imena prema njihovim  naseljima i taborištima: dvanaest poglavica od isto toliko plemena. 
\par 17 A ovo je duljina Jišmaelova života: stotinu trideset i sedam  godina. Zatim izdahnu; umrije i bi pridružen svojim precima. 
\par 18 Potomstvo mu se naselilo od Havile do Šura, koji je na istok  Egiptu idući prema Ašuru. Nastaniše se nasuprot svojoj braći. 
\par 19 Ovo je povijest Abrahamova sina Izaka: Izak se rodio  od Abrahama. 
\par 20 Izaku je bilo četrdeset godina kad se oženio  Rebekom, kćerkom Aramejca Betuela iz Padan Arama, a sestrom Aramejca  Labana. 
\par 21 Izak se obrati Jahvi za svoju ženu jer je bila nerotkinja.  Jahve ga usliša te njegova žena Rebeka zače. 
\par 22 No djeca se  u njezinoj utrobi tako sudarala te ona uzviknu: "Ako je tako, zašto ću živjeti!" Ode, dakle, da se posavjetuje s Jahvom. 
\par 23 I  Jahve joj reče: "Dva su svijeta u utrobi tvojoj; dva će se naroda iz tvog krila odijeliti. Narod će nad narodom gospodovati, stariji će služiti mlađemu." 
\par 24 Došlo vrijeme da rodi, kad gle - blizanci u njezinoj  utrobi. 
\par 25 Pojavi se prvi. Bio je crven; sav runjav kao ogrtač.  Stoga mu nadjenuše ime Ezav. 
\par 26 Potom se pojavi njegov brat.  Rukom se držao Ezavu za petu. Zato mu nadjenuše ime Jakov. Izaku  je bilo šezdeset godina kad su oni rođeni. 
\par 27 Kad su dječaci odrasli, Ezav postane vješt lovac, čovjek  pustare. Jakov je bio čovjek krotak i boravio je u šatorima. 
\par 28 Izaku je Ezav bio draži jer je volio divljač, a Rebeka je  više voljela Jakova. 
\par 29 Jednom Jakov kuhaše jelo. Ezav stigne s polja, gladan. 
\par 30 Reče Ezav Jakovu: "Daj mi toga crvenog variva da pojedem  jer sam izgladnio." Stoga mu je ime Edom. 
\par 31 A Jakov odgovori:  "Ustupi mi prije svoje prvorodstvo!" 
\par 32 Ezav reče: "Evo me skoro  na smrti; što će mi prvorodstvo!" 
\par 33 Jakov nastavi: "Prije mi  se zakuni!" On mu se zakune, i tako proda Jakovu svoje prvorodstvo. 
\par 34 Tada Jakov dade Ezavu kruha i čorbe od sočivice. Jeo je i  pio, onda se digao i otišao. Tako Ezav pogazi svoje prvorodstvo. 



\chapter{26}



\par 1 U zemlji zavlada glad, različita od prijašnje što je bila  za vrijeme Abrahama, pa Izak ode Abimeleku, kralju Filistejaca, u Geraru. 
\par 2 Jahve mu se ukaza i reče: "Ne silazi u Egipat:  boravi u zemlji koju ću ti označiti. 
\par 3 U ovoj se zemlji nastani, ja ću s tobom biti i blagoslivljati te; tebi i tvome potomstvu  dat ću sve ove krajeve, da izvršim zakletvu kojom sam se zakleo  tvome ocu Abrahamu. 
\par 4 Tvoje ću potomstvo umnožiti kao zvijezde  na nebesima i tvome ću potomstvu predati sve ove krajeve, tako  da će se tvojim potomstvom blagoslivljati svi narodi zemlje; 
\par 5 a to zato što je Abraham slušao moj glas i pokoravao se mojim  zapovijedima, mojim zakonima i odredbama!" 
\par 6 Tako Izak ostane u Geraru. 
\par 7 Kad su ga mještani pitali  o njegovoj ženi, reče: "Ona mi je sestra." Bojao se reći: "Ona  mi je žena", misleći: "Mještani bi me mogli ubiti zbog Rebeke  jer je lijepa." 
\par 8 Kako su se ondje duže zadržali, kralj Filistejaca  Abimelek jednom pogleda kroz prozor i opazi kako Izak miluje  svoju ženu Rebeku. 
\par 9 Nato Abimelek pozove Izaka te reče: "Tako, ona ti je žena! Kako si mogao reći da ti je sestra?" Izak mu  odgovori: "Jer sam mislio da bih zbog nje mogao poginuti." 
\par 10 Abimelek  reče: "Zašto si nam to učinio? Umalo netko od ljudi nije legao  s tvojom ženom. Tako bi na nas svalio krivnju." 
\par 11 Onda Abimelek  izda naredbu svemu narodu: "Tko se god dotakne ovog čovjeka i  njegove žene, glavu će izgubiti." 
\par 12 Izak je sijao u onom kraju i one godine urodilo mu stostruko.  Jahve ga blagoslivljao 
\par 13 te je čovjek bivao sve bogatiji, dok  nije postao vrlo bogat. 
\par 14 Stekao je stada ovaca i goveda i  mnogu služinčad, tako da su mu Filistejci zavidjeli. 
\par 15 Zato  Filistejci zasuše sve bunare što su ih sluge njegova oca bile  iskopale - u vrijeme njegova oca Abrahama - i napuniše ih zemljom. 
\par 16 Onda Abimelek reče Izaku: "Idi od nas jer si postao mnogo  moćniji od nas!" 
\par 17 Tako Izak ode odande, postavi svoj šator u gerarskoj  dolini i nastani se ondje. 
\par 18 Izak opet iskopa bunare za vodu  što su bili iskopani u vrijeme njegova oca Abrahama, a Filistejci  ih bili zasuli poslije Abrahamove smrti. On ih je nazvao istim  imenima kojima ih je zvao i njegov otac. 
\par 19 Ali kad su Izakove  sluge, dok su u dolini kopale, ondje našle bunar sa živom vodom, 
\par 20 pastiri iz Gerara posvade se s Izakovim pastirima govoreći:  "Naša je voda!" Bunaru je dao ime Esek, jer su se oni s njim  svadili. 
\par 21 A kad su iskopali drugi bunar te se i zbog njega  svađali, nazva ga imenom Sitna. 
\par 22 Odatle se preseli pa iskopa  drugi bunar. Zbog njega se nisu svađali, pa ga nazove imenom  Rehobot i protumači: "Jer nam je Jahve dao prostor da se na zemlji  umnožimo." 
\par 23 Odande se popne u Beer Šebu. 
\par 24 Iste mu se noći ukaže Jahve i reče: "Ja sam Bog oca tvoga Abrahama. Ne boj se, ja sam s tobom! Blagoslovit ću te, potomke ti umnožit, zbog Abrahama, sluge svojega." 
\par 25 Izak tu podigne žrtvenik i zazove Jahvu po imenu; postavi  ondje svoj šator, a njegove sluge počnu kopati bunar. 
\par 26 Uto mu dođe Abimelek iz Gerara sa svojim savjetnikom  Ahuzatom i s Fikolom, zapovjednikom vojske. 
\par 27 Izak ih upita:  "Zašto ste došli k meni kad me mrzite i kad ste me otjerali od  sebe?" 
\par 28 Oni odgovore: "Jasno vidimo da je Jahve s tobom. Stoga  pomislismo: neka zakletva bude veza između nas i tebe. Daj da  s tobom sklopimo savez: 
\par 29 ti nama nećeš zla nanositi, kao što  mi tebe nismo zlostavljali, nego uvijek prema tebi lijepo postupali  i s mirom te otpustili. A blagoslov Jahvin bio nad tobom." 
\par 30 On  im priredi gozbu te su jeli i pili. 
\par 31 Rano ujutro jedni se drugima zakunu. Potom ih Izak otpusti  i oni od njega odu u miru. 
\par 32 Toga istog dana dođu Izakove sluge  i obavijeste ga o bunaru što su ga iskopali te mu reknu: "Našli  smo vodu." 
\par 33 On ga prozva Šiba. Zato je ime onom gradu do danas  - Beer Šeba. 
\par 34 Kad je Ezavu bilo četrdeset godina, uzme za ženu Juditu, kćer Hetita Beerija, i Basematu, kćer Hetita Elona. 
\par 35 One postadoše izvor ogorčenja Izaku i Rebeki. 



\chapter{27}



\par 1 Ostarje Izak, vid mu se očinji gasio. Zato zovne svoga starijeg  sina Ezava i reče mu: "Sine!" On mu odgovori: "Evo me!" 
\par 2 A  on nastavi: "Vidiš, ostario sam, a ne znam dana svoje smrti. 
\par 3 Zato uzmi svoju opremu, svoj tobolac i luk, pa idi u pustaru  i ulovi mi divljači. 
\par 4 Onda mi pripremi ukusan obrok, kako volim, te mi ga donesi da blagujem, pa da te mognem blagosloviti prije  nego umrem." 
\par 5 Rebeka je slušala dok je Izak govorio svome sinu Ezavu, i kad je Ezav otišao u pustaru da ulovi divljači svome ocu, 
\par 6 Rebeka reče svome sinu Jakovu: "Upravo sam čula kako tvoj  otac govori tvome bratu Ezavu: 
\par 7 'Donesi mi divljači te mi priredi  ukusan obrok da blagujem pa da te pred licem Jahvinim blagoslovim  prije nego umrem.' 
\par 8 A sad, sine moj, poslušaj me i učini kako  ti naredim. 
\par 9 Otiđi k stadu i odande mi donesi dva lijepa kozleta, a ja ću od njih prirediti ukusan obrok tvome ocu, kako on voli. 
\par 10 Onda ti donesi svome ocu da jede te tebe mogne blagosloviti  prije nego umre." 
\par 11 Ali Jakov odgovori svojoj majci Rebeki:  "E, ali moj je brat Ezav runjav, a ja sam bez dlaka! 
\par 12 Možda  me se moj otac dotakne te ću u njegovim očima ispasti varalicom  i na se svaliti prokletstvo, a ne blagoslov." 
\par 13 Ali njegova  mu majka odgovori: "Sine moj, tvoje prokletstvo neka padne na  mene! Samo ti mene poslušaj, otiđi i donesi!" 
\par 14 Ode on, nađe i donese svojoj majci, a njegova majka priredi  ukusan obrok, kako je njegov otac volio. 
\par 15 Potom Rebeka uzme  najljepše odijelo svoga starijeg sina Ezava što je u kući imala, pa u nj odjene svoga mlađeg sina Jakova. 
\par 16 U kožu kozleta  zamota mu ruke i goli dio vrata. 
\par 17 Stavi zatim ukusan obrok  i kruh što ga je pripravila na ruke svoga sina Jakova. 
\par 18 Ode on k ocu i reče: "Oče!" On odgovori: "Evo me. Koji  si ti moj sin?" 
\par 19 A Jakov odgovori svome ocu: "Ja sam Ezav, tvoj prvorođenac; učinio sam kako si mi rekao. Sad ustaj, sjedi  pa jedi moje lovine, da me onda mogneš blagosloviti." 
\par 20 Izak  upita svoga sina: "Kako si tako brzo uspio, sine moj?" On odgovori:  "Jer mi je Jahve, Bog tvoj, bio milostiv." 
\par 21 Potom Izak reče  Jakovu: "Primakni se, sine moj, da opipam jesi li ti zbilja moj  sin Ezav ili nisi." 
\par 22 Jakov se primakne k svome ocu Izaku,  koji ga opipa i reče: "Glas je Jakovljev, ali su ruke Ezavove." 
\par 23 Nije ga prepoznao jer su mu ruke bile runjave kao i ruke  njegova brata Ezava. Kad ga je htio blagosloviti, 
\par 24 upita još  jednom: "Jesi li ti zaista moj sin Ezav?" Odgovori on: "Jesam." 
\par 25 Potom reče Izak: "Stavi preda me da blagujem lovine svoga  sina pa da te blagoslovi duša moja." Jakov ga posluži pa je jeo.  Zatim mu donese i vina, pa je pio. 
\par 26 Poslije toga reče mu njegov  otac Izak: "Primakni se, sine moj, i poljubi me!" 
\par 27 Kad se  primače i poljubi ga, Izak osjeti miris njegove odjeće pa ga  blagoslovi:  "Gle, miris sina mog nalik je mirisu polja koje Jahve blagoslovi. 
\par 28 Neka ti Bog daje rosu s neba i rodnost zemlje: izobilje žita i mladoga vina. 
\par 29 Narodi ti služili, plemena ti se klanjala! Braćom svojom gospodari, nek sinci majke tvoje pred tobom padaju! Proklet bio tko tebe proklinje; blagoslovljen tko te blagoslivlje!" 
\par 30 Tek što se Jakov udaljio od svoga oca Izaka - pošto je  Izak podijelio blagoslov Jakovu - njegov brat Ezav dođe iz lova. 
\par 31 I on priredi ukusan obrok i donese ga svome ocu. I reče svome  ocu: "Ustani, oče moj, i blaguj od lovine svoga sina da me onda  mogneš blagosloviti!" 
\par 32 A njegov ga otac Izak zapita: "Tko  si ti?" On odgovori: "Ja sam tvoj prvorođenac Ezav!" 
\par 33 Izak  se silno prepadne: "Pa tko je onda bio onaj što je divljači ulovio  i meni već donio? Blagovao sam je prije nego si ti došao; onoga  sam blagoslovio i blagoslovljen će ostati." 
\par 34 Kad je Ezav čuo  riječi svoga oca, kriknu glasno i gorko zaplaka pa reče svome  ocu: "I mene blagoslovi, oče!" 
\par 35 A on odvrati: "Brat tvoj dođe  na prijevaru i odnese tvoj blagoslov." 
\par 36 "Zato valjda što mu  je ime Jakov, dvaput me već prevario", reče Ezav. "Oduzeo mi  prvorodstvo, a sad mi evo oduze i blagoslov." Onda doda: "Zar  za me nisi sačuvao nikakva blagoslova?" 
\par 37 Izak odgovori Ezavu:  "Njega sam već postavio za tvoga gospodara; njemu sam svu njegovu  braću predao za sluge; žitom sam ga i vinom opskrbio. A što sad  za te mogu učiniti, sine moj?" 
\par 38 Ezav odgovori svome ocu: "Zar  ti, oče, raspolažeš samo jednim blagoslovom? Blagoslovi i mene, oče moj!" Ezav jecaše na sav glas. 
\par 39 Tada otac njegov Izak  progovori i reče: "Daleko od plodna tla dom tvoj će biti, daleko od rose s neba. 
\par 40 Od mača svoga ćeš živjeti, brata svoga ćeš služiti. Ali jednom, kada se pobuniš, jaram ćeš njegov stresti sa svog vrata." 
\par 41 Ezav zamrzi Jakova zbog blagoslova kojim ga je otac njegov  blagoslovio pa reče u sebi: "Čim dođu dani žalosti za mojim ocem, ubit ću ja svoga brata Jakova." 
\par 42 Kada su Rebeki javili te  riječi što ih je izrekao njezin stariji sin Ezav, zovne ona svoga  mlađeg sina Jakova te mu reče: "Pazi! Brat ti se Ezav nosi mišlju  kako će te ubiti. 
\par 43 Ali ti, sine moj, poslušaj mene: odmah  bježi mome bratu Labanu u Haran. 
\par 44 Ostani kod njega neko vrijeme, dok bijes brata tvoga na te jenja, 
\par 45 dok se srdžba brata tvoga  odvrati od tebe te on zaboravi što si mu učinio. Ja ću onda po  te poslati i odande te dovesti. Zašto da vas obojicu izgubim  u jedan dan!" 
\par 46 Potom Rebeka reče Izaku: "Moj mi je život dosadio zbog  ovih žena Hetitkinja. Ako se i Jakov oženi kojom kao što su ove  urođenice, Hetitkinjom, što će mi onda život!" 



\chapter{28}



\par 1 Stoga Izak pozove Jakova, blagoslovi ga te mu naloži: "Nemoj  uzimati ženu od kanaanskih djevojaka. 
\par 2 Odmah se zaputi u Padan  Aram, u dom Betuela, oca svoje majke, pa odande sebi uzmi ženu, od kćeri Labana, brata svoje majke. 
\par 3 A Bog Svemožni, El-Šadaj, neka te blagoslovi i neka te učini rodnim i brojnim, tako da  postaneš mnoštvo naroda. 
\par 4 Neka protegne na te blagoslov Abrahamov, na te i na tvoje potomstvo, tako da zaposjedneš zemlju u kojoj  boraviš kao pridošlica, a koju je Bog predao Abrahamu!" 
\par 5 Tako  Izak otpremi Jakova, i on ode u Padan Aram Labanu, sinu Aramejca  Betuela, bratu Rebeke, majke Jakova i Ezava. 
\par 6 Kad je Ezav vidio kako je Izak blagoslovio Jakova kad  ga je otpremao u Padan Aram da odande sebi uzme ženu, naređujući  mu kad ga je blagoslivljao: "Ne smiješ uzeti ženu od kanaanskih  djevojaka", 
\par 7 i da je Jakov poslušao svoga oca i svoju majku  te otišao u Padan Aram, 
\par 8 Ezav shvati koliko su djevojke kanaanske  mrske njegovu ocu Izaku. 
\par 9 Stoga ode k Jišmaelu te se, uza žene  koje već imaše, oženi Mahalatom, kćerju Jišmaela, sina Abrahamova, a sestrom Nebajotovom. 
\par 10 Jakov ostavi Beer Šebu i zaputi se u Haran. 
\par 11 Stigne  u neko mjesto i tu prenoći, jer sunce bijaše već zašlo. Uzme  jedan kamen s onog mjesta, stavi ga pod glavu i na tom mjestu  legne. 
\par 12 I usne san: ljestve stoje na zemlji, a vrhom do neba  dopiru, i anđeli Božji po njima se penju i silaze. 
\par 13 Uza nj  je Jahve te mu govori: "Ja sam Jahve, Bog tvoga praoca Abrahama  i Bog Izakov. Zemlju na kojoj ležiš dat ću tebi i tvome potomstvu. 
\par 14 Tvojih će potomaka biti kao i praha na zemlji; raširit ćete  se na zapad, istok, sjever i jug; tobom će se i tvojim potomstvom  blagoslivljati svi narodi zemlje. 
\par 15 Dobro znaj: ja sam s tobom;  čuvat ću te kamo god pođeš te ću te dovesti natrag u ovu zemlju;  i neću te ostaviti dok ne izvršim što sam ti obećao." 
\par 16 Jakov se probudi od sna te reče: "Zaista se Jahve nalazi  na ovome mjestu, ali ja nisam znao!" 
\par 17 Potresen, uzviknu: "Kako  je strašno ovo mjesto! Zaista, ovo je kuća Božja, ovo su vrata  nebeska!" 
\par 18 Rano ujutro Jakov uzme onaj kamen što ga bijaše  stavio pod glavu, uspravi ga kao stup i po vrhu mu izlije ulja. 
\par 19 Ono mjesto on nazva Betel, dok je ime tome gradu prije bilo  Luz. 
\par 20 Tada učini zavjet: "Ako Bog ostane sa mnom i uščuva me  na ovom putu kojim idem, dade mi kruha da jedem i odijela da  se oblačim, 
\par 21 te se zdravo vratim kući svoga oca, Jahve će  biti moj Bog. 
\par 22 A ovaj kamen koji sam uspravio kao stup bit  će kuća Božja. A od svega što mi budeš davao za te ću odlagati  desetinu." 



\chapter{29}



\par 1 Jakov nastavi put i dođe u zemlju istočnu. 
\par 2 Najednom opazi  studenac u polju. Tri su stada ovaca oko njega plandovala, jer  se na tome studencu napajahu. Velik se kamen nalazio studencu  na otvoru. 
\par 3 Jedino kad bi se svi pastiri ondje skupili, mogli  bi odvaliti kamen s otvora i ovce napojiti; tada bi opet prevalili  kamen na njegovo mjesto, na otvor studenca. 
\par 4 "Odakle ste, braćo moja?" - zapita ih Jakov. "Iz Harana", odgovore. 
\par 5 "Poznajete li", pitaše ih dalje, "Nahorova sina  Labana?" "Poznajemo", odgovore. 
\par 6 "Je li zdravo?" - opet ih  upita. "Zdravo je; a evo mu dolazi kći Rahela sa stadom", odgovore. 
\par 7 "Još ima mnogo dana", nastavi on, "nije vrijeme spraćati blago.  Zašto ga ne napojite i ne otjerate na pašu?" 
\par 8 "Ne možemo dok  se ne skupe svi pastiri", odgovoriše, "da odvale kamen s otvora  studenca, tako da mognemo napojiti ovce." 
\par 9 Dok je on još s njima govorio, dođe Rahela s ovcama svoga  oca. Bila je, naime, pastirica. 
\par 10 Kako Jakov ugleda Rahelu, kćer Labana, brata svoje majke, sa stadom svoga ujaka Labana, Jakov se primače i odvali kamen s otvora studenca te napoji  stado svoga ujaka Labana. 
\par 11 Zatim Jakov poljubi Rahelu, a onda  briznu u plač. 
\par 12 Potom Jakov kaza Raheli da je on sestrić njezina  oca, sin Rebekin. Nato ona otrča i obavijesti oca. 
\par 13 Kad je  Laban čuo vijest o Jakovu, sinu svoje sestre, potrča mu u susret.  Zagrli ga i poljubi te dovede u svoju kuću. Ispriča Labanu sve  što mu se dogodilo. 
\par 14 A onda Laban reče. "Zbilja si ti moja  kost i moje meso!" Pošto je Jakov proboravio s Labanom mjesec dana, 
\par 15 Laban  reče Jakovu: "Zar ćeš me zato što si mi sestrić badava služiti!  Kaži mi koliko ćeš tražiti za najam?" 
\par 16 A Laban imaše dvije  kćeri. Starijoj bijaše ime Lea, a mlađoj Rahela. 
\par 17 Lea imala  slabe oči, a Rahela bila stasita i lijepa. 
\par 18 Kako je Jakov  volio Rahelu, reče: "Služit ću ti sedam godina za tvoju mlađu  kćer Rahelu." 
\par 19 Laban odvrati: "Bolje je da je tebi dam nego  kakvu strancu. Ostani sa mnom!" 
\par 20 Tako je Jakov služio za Rahelu sedam godina, ali mu se  učinile, zbog ljubavi prema njoj, kao nekoliko dana. 
\par 21 Poslije toga Jakov reče Labanu: "Daj mi moju ženu, jer  se moje vrijeme navršilo pa bih htio k njoj." 
\par 22 Laban sabra  sav svijet onog mjesta i priredi gozbu. 
\par 23 Ali navečer uzme  svoju kćer Leu pa nju uvede k Jakovu, i on priđe k njoj. 
\par 24 Laban  dade svoju sluškinju Zilpu svojoj kćeri Lei za sluškinju. 
\par 25 Kad  bi ujutro, a to, gle, Lea! Tada Jakov reče Labanu: "Zašto si  mi to učinio! Zar te ja nisam služio za Rahelu? Zašto si me prevario?" 
\par 26 Laban odgovori: "U našem mjestu nije običaj da se mlađa udaje  prije starije. 
\par 27 Završi s njom ovu ženidbenu sedmicu, a onda  ću ti dati i drugu, za drugih sedam godina službe kod mene."  Jakov pristane: navrši onu ženidbenu sedmicu. 
\par 28 Onda mu Laban  dade i svoju kćer Rahelu za ženu. 
\par 29 Laban dade svoju sluškinju  Bilhu svojoj kćeri Raheli za sluškinju. 
\par 30 Jakov nato priđe  Raheli. Rahelu je više volio nego Leu. I tako je služio Labana  još sedam godina. 
\par 31 Jahve je vidio da Lea nije voljena, te je učini plodnom, dok Rahela ostade nerotkinja. 
\par 32 Lea zače i rodi sina; nadjenu  mu ime Ruben, a to znači, kako je ona protumačila: "Jahve je  vidio moju nevolju i stoga će me sada muž moj ljubiti." 
\par 33 Opet  zače i rodi sina te izjavi: "Jahve je čuo da nisam voljena, stoga  mi je dao i ovoga." Zato mu nadjenu ime Šimun. 
\par 34 Opet zače  i rodi sina te izjavi: "Sad će se moj muž meni prikloniti: tri  sam mu sina rodila." Zato mu nadjenu ime Levi. 
\par 35 A kad je još  jednom začela i sina rodila, izjavi: "Ovaj put hvalit ću Jahvu."  Stoga sinu nadjenu ime Juda. Potom prestade rađati. 



\chapter{30}



\par 1 Vidjevši Rahela da Jakovu ne rađa djece, postade zavidna svojoj  sestri pa reče Jakovu: "Daj mi djecu! Inače ću svisnuti!" 
\par 2 Jakov  se razljuti na Rahelu te reče. "Zar sam ja namjesto Boga koji  ti je uskratio plod utrobe?" 
\par 3 A ona odgovori: "Evo moje sluškinje  Bilhe: uđi k njoj, pa neka rodi na mojim koljenima, da tako i  ja steknem djecu po njoj." 
\par 4 Dade mu dakle svoju sluškinju Bilhu  za ženu, i Jakov priđe k njoj. 
\par 5 Bilha zače te Jakovu rodi sina. 
\par 6 Tada Rahela reče: "Jahve mi je dosudio pravo. Uslišao je moj  glas i dao mi sina." Stoga mu nadjenu ime Dan. 
\par 7 Rahelina sluškinja  Bilha opet zače i rodi Jakovu drugoga sina. 
\par 8 Tada Rahela reče:  "Žestoko sam se borila sa sestrom, ali sam pobijedila." Tako  mu nadjenu ime Naftali. 
\par 9 A vidjevši Lea da je prestala rađati, uzme svoju sluškinju  Zilpu pa je dade Jakovu za ženu. 
\par 10 I kad je Leina sluškinja  Zilpa rodila Jakovu sina, 
\par 11 Lea uskliknu: "Koje sreće!" Tako  mu nadjenu ime Gad. 
\par 12 Leina sluškinja Zilpa rodi Jakovu i drugog  sina, 
\par 13 i Lea opet uskliknu: "Blago meni! Žene će me zvati  blaženom!" Tako mu nadjenu ime Ašer. 
\par 14 Jednoga dana, u vrijeme pšenične žetve, namjeri se Ruben  u polju na ljubavčice te ih donese svojoj majci Lei. I Rahela  reče Lei: "Daj mi od ljubavčica svoga sina!" 
\par 15 A Lea odgovori:  "Zar ti nije dosta što si mi oduzela muža pa još hoćeš da od  mene uzmeš i ljubavčice moga sina?" Rahela odgovori: "Pa dobro, neka s tobom noćas leži u zamjenu za ljubavčice tvog sina." 
\par 16 Kad je Jakov navečer stigao iz polja, Lea mu iziđe u susret  pa reče: "Treba da dođeš k meni, jer sam te unajmila za ljubavčice  moga sina." One je noći on s njom ležao. 
\par 17 Bog usliša Leu;  ona zače te Jakovu rodi petog sina. 
\par 18 Onda Lea reče: "Bog mi  je uzvratio nagradom što sam ustupila svoju sluškinju svome mužu."  Stoga sinu nadjenu ime Jisakar. 
\par 19 Lea opet zače i rodi Jakovu  šestoga sina. 
\par 20 Onda Lea reče: "Bog me obdari dragocjenim darom;  sada će mi moj muž dati darove: tÓa rodila sam mu šest sinova."  Tako mu nadjenu ime Zebulun. 
\par 21 Zatim rodi kćer te joj nadjenu  ime Dina. 
\par 22 Uto se Bog sjeti Rahele: Bog je usliša i otvori njezinu  utrobu. 
\par 23 Ona zače i rodi sina te reče: "Ukloni Bog moju sramotu!" 
\par 24 Nadjene mu ime Josip, rekavši: "Neka mi Jahve pridoda drugog  sina!" 
\par 25 Pošto je Rahela rodila Josipa, Jakov reče Labanu: "Pusti  me da idem u svoj zavičaj! 
\par 26 Daj mi moje žene za koje sam te  služio i moju djecu da mogu otići: tÓa dobro znaš kako sam te  služio." 
\par 27 A Laban mu odgovori: "Ne idi, ako si mi prijatelj.  Znam da me Jahve blagoslivljao zbog tebe." 
\par 28 I nadoda: "Odredi  plaću koju želiš od mene, i dat ću ti." 
\par 29 On mu odgovori: "Ti  dobro znaš što je moja služba značila za te i kako je tvome blagu  bilo sa mnom. 
\par 30 Malenkost što si je imao prije nego sam ja  došao povećala se vrlo mnogo, jer kuda god sam prolazio Jahve  te blagoslivljao na mojim koracima. A sad je vrijeme da poradim  i za svoj dom." 
\par 31 On upita: "Koliko da ti platim?" Jakov odgovori:  "Nemoj mi platiti ništa! Ako mi učiniš ovo, opet ću na pašu goniti  i čuvati tvoje stado. 
\par 32 Daj da prođem danas kroz tvoje stado  i od njega izlučim svaku garavu ovcu i svaku šarenu ili napruganu  kozu! Neka to bude moja plaća! 
\par 33 A ubuduće kad budeš svojim  očima provjeravao moju naplatu, moje će poštenje biti svjedok  za mene: nađe li se među mojim kozama ijedna koja ne bude šarena  ili naprugana, ili među ovcama koja ne bi bila garava, neka se  smatra ukradenom!" 
\par 34 Laban reče: "Dobro, neka bude kako si kazao." 
\par 35 Ali toga dana Laban izluči naprugane i šarene jarce i  sve riđaste i šarene koze - svaku koja je na sebi imala bijelo  - i sve garave ovce pa ih preda svojim sinovima. 
\par 36 I odande  gdje je Jakov pasao ostatak Labanova stada udalji se za koja  tri dana hoda. 
\par 37 A Jakov uzme zelenih mladica od topola, badema i platana;  na njima izreza bijele pruge, otkrivši bjeliku na mladicama. 
\par 38 Pruće tako isprugano postavi u korita, u pojila iz kojih  se stoka napajala. A kako se stoka parila kad je na vodu dolazila, 
\par 39 to su se jarci parili uz pruće, pa su koze kozile prugaste, riđaste i šarene kozliće. 
\par 40 Tako je i ovce Jakov bio izlučio  i glave im okrenuo prema prugastima ili posve garavima što su  bile u Labanovu stadu. Tako je za se namicao posebna stada koja  nije miješao s Labanovim stadima. 
\par 41 Osim toga, kad bi se god  dobro uzrasla stoka parila, Jakov bi stavio pruće u korita, baš  pred oči živine, tako da se pari pred prućem. 
\par 42 Ali ga pred  kržljavu marvu nije stavljao. Tako je kržljava zapadala Labana, a dobro razvijena Jakova. 
\par 43 Čovjek se tako silno obogatio, stekao mnogu stoku, sluge i sluškinje, deve i magarad. 



\chapter{31}



\par 1 Uto Jakov dozna kako Labanovi sinovi govore: "Sve dobro našega  oca uze Jakov; i od onoga što bi moralo pripasti našem ocu namaknuo  je sve ono bogatstvo." 
\par 2 A opazi Jakov i na Labanovu licu da  se on ne drži prema njemu kao prije. 
\par 3 Tada Jahve reče Jakovu:  "Vrati se u zemlju svojih otaca, u svoj zavičaj, i ja ću biti  s tobom!" 
\par 4 Jakov onda pozove Rahelu i Leu u polje, k svome  stadu, 
\par 5 pa im reče: "Ja vidim na licu vašega oca da se on ne  drži prema meni kao prije; ali Bog oca moga sa mnom je bio. 
\par 6 I  same znate da sam vašega oca služio koliko sam god mogao; 
\par 7 pa  ipak je vaš otac mene varao, deset mi je puta plaću mijenjao.  Ali Bog nije dopuštao da mi nanese štetu. 
\par 8 Ako bi on rekao:  'Svaka šarena neka bude tebi za naplatu', onda bi cijelo stado  mladilo šarene; ako bi opet rekao: 'Prugasti neka budu tebi za  plaću', onda bi cijelo stado mladilo prugaste. 
\par 9 Tako je Bog  uzimao blago od vašeg oca pa ga meni davao. 
\par 10 Jednom, kad se  stado oplođivalo, nenadano vidjeh u snu da su jarci u stadu,  dok su se parili, bili prugasti, mjestimično bijeli i šareni. 
\par 11 Još u snu anđeo Božji mene zovne: 'Jakove!' 'Evo me!' rekoh. 
\par 12 A on nastavi: 'Primijeti dobro da su jarci u stadu što se  pare prugasti, mjestimično bijeli i šareni. Ja sam, naime, vidio  sve što ti je Laban činio. 
\par 13 Ja sam Bog koji ti se ukazao u  Betelu, gdje si uljem pomazao stup i gdje si mi učinio zavjet.  Sad ustaj i idi iz ove zemlje; vrati se u svoj zavičaj!'" 
\par 14 Nato mu Rahela i Lea odgovore: "Zar još imamo baštinskog  dijela u svome očinskom domu? 
\par 15 Zar nas otac nije smatrao tuđinkama?  TÓa on je nas prodao, a onda je pojeo novac što ga je za nas  dobio! 
\par 16 Sve bogatstvo što je Bog oduzeo našem ocu zbilja je  naše i djece naše. Zato izvrši sve što ti je Bog rekao!" 
\par 17 Nato Jakov naprti na deve svoju djecu i svoje žene; 
\par 18 pred  sobom potjera sve svoje blago, sva svoja dobra što ih je stekao, stoku što ju je namaknuo u Padan Aramu: krenu u zemlju kanaansku, k svome ocu Izaku. 
\par 19 Laban bijaše otišao da striže svoje ovce, pa Rahela prisvoji  kućne kumire koji su pripadali njezinu ocu. 
\par 20 Jakov zavara  Aramejca Labana tako da nije ni slutio da će bježati. 
\par 21 I pobjegne  sa svim što je bilo njegovo. Ubrzo prijeđe Eufrat i upravi put  prema brdu Gileadu. 
\par 22 Trećeg dana obavijeste Labana da je Jakov pobjegao. 
\par 23 On  povede sa sobom svoje rođake te je za Jakovom išao u potjeru  sedam dana hoda; stiže ga na brdu Gileadu. 
\par 24 Ali se Bog ukaza  Aramejcu Labanu, noću u snu, te mu reče. "Pazi da protiv Jakova  ne poduzimlješ ništa, ni dobro ni zlo!" 
\par 25 Uto Laban stigne  Jakova. Jakov bijaše postavio svoj šator na Glavici, a Laban  se utabori na brdu Gileadu. 
\par 26 Onda Laban reče Jakovu: "Što si to htio zavaravajući  me i odvodeći mi kćeri kao zarobljenice na maču? 
\par 27 Zašto si  potajno pobjegao, u bludnju me zaveo i nisi me obavijestio? Otpratio  bih te s veseljem i pjesmom, uz bubnje i lire. 
\par 28 Nisi mi dopustio  ni da izljubim svoje kćeri i svoju unučad! Zbilja si ludo postupio. 
\par 29 U mojoj je ruci da s tobom loše postupim. Ali Bog tvoga oca  noćas mi reče: 'Pazi da protiv Jakova ne poduzmeš ništa, ni dobro  ni zlo!' 
\par 30 Sada dobro, otišao si jer si čeznuo za svojim očinskim  domom; ali zašto si mi kumire pokrao?" 
\par 31 Jakov odgovori Labanu: "Strepio sam od pomisli da bi  mi mogao silom oteti svoje kćeri. 
\par 32 A kumire svoje u koga nađeš, onaj neka pogine! Ovdje pred našom braćom kaži što je tvoga  pri meni i nosi!" Jakov nije znao da ih je Rahela prisvojila. 
\par 33 Tako Laban uđe u šator Jakovljev, pa u šator Lein, onda u  šator dviju sluškinja, ali ništa ne nađe. Izišavši iz Leina šatora, uđe u šator Rahelin. 
\par 34 A Rahela bijaše uzela kumire i stavila  ih u sjedalo svoje deve, a onda na njih sjela. Laban je premetao  po svemu šatoru, ali ih ne nađe. 
\par 35 Ona je, naime, rekla svome  ocu: "Neka se moj gospodar ne ljuti što ne mogu pred njim ustati  jer imam ono što je red kod žena." I tako je pretraživao, ali  kumira nije našao. 
\par 36 Sad se Jakov ražesti i zađe u prepirku s Labanom. Otvoreno  Jakov reče Labanu: "Kakvo je moje zlodjelo, koja li je moja krivnja  da me progoniš? 
\par 37 Eto si premetnuo sve moje stvari, pa kakav  si predmet našao od svega svog kućanstva? Položi ga tu pred moj  i svoj rod pa neka oni budu suci među nama dvojicom. 
\par 38 Za ovih  dvadeset godina što sam ih s tobom proveo ni tvoje ovce ni tvoje  koze nisu se jalovile niti sam ja jeo ovnova iz tvoga stada. 
\par 39 Ono što bi zvijer razdrla, tebi nisam donosio, nego bih od  svoga gubitak nadoknadio. Ti si to od mene tražio, bilo da je  nestalo danju ili da je nestalo noću. 
\par 40 Često sam danju skapavao  od žeđi, a obnoć od studeni. San je bježao od mojih očiju. 
\par 41 Od  ovih dvadeset godina što sam ih proveo u tvojoj kući četrnaest  sam ti godina služio za tvoje dvije kćeri, a šest godina za tvoju  stoku, jer si mi mijenjao zaradu deset puta. 
\par 42 Da sa mnom nije  bio Bog moga oca, Bog Abrahamov, Strah Izakov, otpravio bi me  praznih ruku. Ali je Bog gledao moju nevolju i trud mojih ruku  te je sinoć dosudio." 
\par 43 Nato Laban odgovori Jakovu: "Kćeri su moje kćeri; djeca  su moja djeca; stada su moja stada, sve što gledaš moje je. Ali  što danas mogu učiniti ovim svojim kćerima ili djeci koju su  rodile? 
\par 44 Pa dobro, hajde da ti i ja napravimo ugovor, tako  da bude svjedok između mene i tebe." 
\par 45 Nato Jakov uzme jedan kamen pa ga uspravi kao stup, 
\par 46 a  onda reče svojim ljudima: "Skupite kamenja!" Tako oni nakupe  kamenja i nabace gomilu. Tu su na gomili blagovali. 
\par 47 Laban  je nazva "Jegar sahaduta", a Jakov je nazva "Gal-ed". 
\par 48 Onda  Laban izjavi: "Neka ova gomila danas bude svjedok između mene  i tebe!" Stoga je nazvana Gal-ed, 
\par 49 ali i Mispa, jer je rekao.  "Neka Jahve bude na vidu i tebi i meni kad jedan drugog ne budemo  gledali. 
\par 50 Ako budeš loše postupao prema mojim kćerima, ili  ako uzmeš druge žene uz moje kćeri, sve da nitko drugi ne bude  s nama, znaj da će Bog biti svjedok između mene i tebe." 
\par 51 Potom Laban reče Jakovu: "Ovdje je, evo, gomila; ovdje  je stup koji sam uspravio između sebe i tebe: 
\par 52 ova gomila  i ovaj stup neka budu jamac da ja u zloj namjeri neću ići na  te iza ove gomile i da ti nećeš ići na me iza ove gomile i ovog  stupa. 
\par 53 Neka Bog Abrahamov i Bog Nahorov budu naši suci!"  Jakov se zakune Bogom - Strahom svoga oca Izaka. 
\par 54 Poslije  toga Jakov prinese žrtvu na Glavici i pozva svoje ljude da blaguju.  Poslije objeda proveli su noć na Glavici. 
\par 55 Ranim se jutrom Laban digne, izljubi svoje sinove i svoje  kćeri te ih blagoslovi; onda se zaputi natrag u svoje mjesto. 



\chapter{32}



\par 1 Jakov je putovao svojim putem, kad mu u susret izađu anđeli  Božji. 
\par 2 Kad ih Jakov opazi, reče: "Ovo je Božje taborište!"  Zato nazva ono mjesto Mahanajim. 
\par 3 Jakov pošalje pred sobom glasnike svome bratu Ezavu u  zemlju Seir, u Edomsku pustaru, 
\par 4 i naloži im: "Ovako ćete reći  mome gospodaru Ezavu: 'Sluga tvoj Jakov poručuje ti: Boravio  sam kod Labana i dosad se ondje zadržao. 
\par 5 Stekao sam goveda, magaradi, ovaca, sluga i sluškinja. Javljam to svome gospodaru, ne bih li našao naklonost u njegovim očima.'" 
\par 6 Glasnici se vrate Jakovu te mu reknu: "Bili smo kod tvoga  brata Ezava; on sam dolazi ti u susret sa četiri stotine momaka." 
\par 7 Jakov se silno uplaši. U zabrinutosti rastavi na dva tabora  ljude, stada, krda i deve što ih je sa sobom imao. 
\par 8 Računao  je: ako Ezav naiđe na jedan tabor i napadne ga, drugi bi se tabor  mogao spasiti. 
\par 9 Onda se Jakov pomoli: "O Bože oca moga Abrahama!  Bože oca moga Izaka! O Jahve, koji si mi naredio: 'Vrati se u  svoj rodni kraj, i ja ću ti biti dobrostiv!' 
\par 10 Nisam vrijedan  sve dobrote koju si tako postojano iskazivao svome sluzi. TÓa  samo sam sa svojim štapom nekoć prešao ovaj Jordan, a sad sam  narastao u dva tabora. 
\par 11 Izbavi me od šaka moga brata, od šaka  Ezavovih! Inače se bojim da bi mogao doći i umlatiti i mene,  i majke, i djecu. 
\par 12 Ti si rekao: 'Obilnim ću te dobrima obasipati  i tvoje potomstvo umnožiti poput pijeska u moru koji se ne da  prebrojiti zbog množine.'" 
\par 13 Ondje provede onu noć; a onda, od onog što je imao pri  ruci, pripravi dar svome bratu Ezavu: 
\par 14 dvjesta koza i dvadeset  jaraca, dvjesta ovaca i dvadeset ovnova; 
\par 15 trideset deva dojilica  s njihovim mladima; četrdeset krava i deset junaca; dvadeset  magarica i deset magaraca. 
\par 16 Stado po stado preda svojim slugama.  Onda reče svojim slugama: "Idite preda mnom, ali držite razmak  među stadima!" 
\par 17 A prvom izda naredbu rekavši: "Kad te sretne  moj brat Ezav pa te upita: 'Čiji si ti? Kamo ideš? Čije je ovo  pred tobom?' 
\par 18 odgovori: 'Tvoga sluge Jakova; ovo je dar koji  šalje svome gospodaru Ezavu; on je tamo za nama.'" 
\par 19 Tako je  naredio i drugome, pa trećemu i svima drugima koji su išli za  stadima: "Ovo i ovako reci Ezavu kad ga sretneš. 
\par 20 Još mu dodaj:  'A sluga tvoj Jakov i sam je za nama.'" Mislio je naime: "Ako  ga unaprijed udobrostivim darovima, a onda se s njim suočim,  možda će mi oprostiti." 
\par 21 Tako darovi krenu naprijed, dok je  on ostao one noći u taborištu. 
\par 22 One noći on ustane, uzme svoje obje žene, obje svoje  sluškinje i svoje jedanaestero djece te prijeđe Jabok preko gaza. 
\par 23 Prebacivši njih na drugu stranu toka, prebaci zatim i ostalo  što bijaše njegovo. 
\par 24 Jakov ostane sam. I neki se čovjek rvao s njim dok nije zora svanula. 
\par 25 Videći  da ga ne može svladati, ugane mu bedro pri zglobu, tako da se  Jakovu kuk iščašio dok su se rvali. 
\par 26 Potom reče: "Pusti me  jer zora sviće!" Ali on odgovori: "Neću te pustiti dok me ne  blagosloviš." 
\par 27 Nato ga onaj zapita: "Kako ti je ime?" Odgovori:  "Jakov." 
\par 28 Onaj reče. "Više se nećeš zvati Jakov nego Izrael, jer si se hrabro borio i s Bogom i s ljudima i nadvladao si." 
\par 29 Onda zapita Jakov: "Reci mi svoje ime!" Odgovori onaj: "Za  moje me ime ne smiješ pitati!" I tu ga blagoslovi. 
\par 30 Onom mjestu Jakov nadjene ime Penuel jer - reče - "Vidjeh  Boga licem u lice, i na životu ostadoh." 
\par 31 Sunce je nad njim  bilo ogranulo kad je prošao Penuel. Hramao je zbog kuka. 
\par 32 Zato  Izraelci do današnjeg dana ne jedu kukovnu tetivu što se nalazi  na bedrenom zglobu, budući da je Jakovljev bedreni zglob bio  iščašen u kukovnoj tetivi. 



\chapter{33}



\par 1 Jakov podiže oči i opazi gdje dolazi Ezav i s njime četiri  stotine ljudi. Onda on podijeli svoju djecu među Leu, Rahelu  i dvije sluškinje; 
\par 2 postavi sluškinje i njihovu djecu na čelo;  iza njih Leu i njezinu djecu; a Rahelu i Josipa straga. 
\par 3 Sam  prođe naprijed, nakloni se do zemlje sedam puta dok se ne primače  svome bratu. 
\par 4 Ezav mu potrča u susret. Zagrli ga padnuvši mu  oko vrata, poljubi ga i zaplaka. 
\par 5 Onda podiže oči i vidje žene  i djecu. "Tko su ovi s tobom?" - zapita. On odgovori: "Djeca  kojom je Bog obdario tvoga slugu." 
\par 6 Potom naprijed stupe sluškinje  sa svojom djecom te se duboko naklone. 
\par 7 Naprijed stupi i Lea  sa svojom djecom te se duboko nakloni. Najposlije stupe naprijed  Josip i Rahela te se duboko naklone. 
\par 8 Ezav upita: "Što kaniš sa svom ovom povorkom što sam je  sreo?" Odgovori: "Naći naklonost svoga gospodara." 
\par 9 Ezav odgovori:  "Ja imam dosta, brate moj. Neka ostane tebi što je tvoje." 
\par 10 A  Jakov reče: "Nemoj tako! Ako sam našao naklonost u tvojim očima, primi dar iz moje ruke; jer meni je, što si me ljubezno primio, kao da gledam lice Božje. 
\par 11 Zato prihvati moj dar što sam  ti ga donio; Bog mi je bio sklon te imam svega." Kako ga je uporno  nagovarao, Ezav prihvati. 
\par 12 "Pođimo na put", reče Ezav, "i ja ću s tobom putovati." 
\par 13 Ali mu on odvrati: "Zna moj gospodar da su djeca nejaka.  Osim toga, valja mi se brinuti o ovcama i kravama koje doje:  ako bi se tjerale prebrzo samo jednog dana, sve bi pocrkale. 
\par 14 Neka moj gospodar ide ispred svoga sluge, a ja ću ići polako, uz korak marve pred sobom i uz korak djece, dok ne stignem k  svome gospodaru u Seir." 
\par 15 Onda reče Ezav: "Da ti barem ostavim  nekoliko ljudi koji se sa mnom nalaze." Ali on odgovori: "Čemu  to? Neka ja samo nađem milost u očima svoga gospodara!" 
\par 16 Tako  se Ezav onog dana zaputi natrag u Seir, 
\par 17 dok je Jakov otišao  u Sukot, gdje sebi sagradi kuću, a svom blagu podigne staje.  Stoga je onom mjestu dano ime Sukot. 
\par 18 Došavši tako iz Padan Arama, Jakov sretno stigne u grad  Šekem, koji se nalazi u zemlji kanaanskoj, i postavi svoj šator  pred gradom. 
\par 19 A komad zemlje na kojoj je postavio svoj šator  kupi od sinova Hamora, Šekemova oca, za stotinu kesita. 
\par 20 Tu  podiže žrtvenik i nazva ga "El, Bog Izraelov". 



\chapter{34}



\par 1 Dina, kći koju je Lea rodila Jakovu, iziđe da posjeti neke  žene onoga kraja. 
\par 2 Opazi je Hivijac Šekem, sin Hamora, poglavice  kraja, pa je pograbi i na silu s njom leže. 
\par 3 Njegovo srce prione  za Dinu, Jakovljevu kćer, i on se u djevojku zaljubi. Nastojao  je pridobiti djevojčino srce. 
\par 4 Šekem je govorio i svom ocu  Hamoru: "Onu mi djevojku uzmi za ženu!" 
\par 5 Jakov sazna da je  Šekem obeščastio njegovu kćer Dinu. Ali kako su njegovi sinovi  bili uz blago na polju, Jakov nije poduzimao ništa dok oni ne  dođu. 
\par 6 Uto dođe k Jakovu Šekemov otac Hamor da se s njim sporazumije, 
\par 7 upravo kad su se Jakovljevi sinovi vraćali iz polja. Kad  su čuli vijest, ljudi su bili ojađeni i vrlo ljuti. Što je Šekem  učinio - legavši s Jakovljevom kćeri - u Izraelu je bila sramota.  To se nije smjelo trpjeti. 
\par 8 Hamor im reče. "Moj se sin Šekem  svom dušom zaljubio u vašu kćer. Dajte mu je za ženu! 
\par 9 Oprijateljite  se s nama: dajite nam svoje kćeri, a naše kćeri uzimajte sebi! 
\par 10 Tako možete živjeti među nama; zemlja je pred vama da se  naselite, u njoj se slobodno krećete i stječete imovinu!" 
\par 11 Potom  Šekem reče njezinu ocu i njezinoj braći: "Da nađem milost u vašim  očima, dat ću vam što zatražite. 
\par 12 Tražite od mene koliko hoćete:  sve što god zapitate dat ću, samo mi dajte djevojku za ženu." 
\par 13 Jakovljevi sinovi odgovore Šekemu i njegovu ocu Hamoru  - govorili su s prijevarom jer je obeščastio njihovu sestru Dinu  - 
\par 14 te im rekoše: "Ne možemo pristati da svoju sestru damo  čovjeku koji nije obrezan, jer bi to za nas bila sramota. 
\par 15 Jedino  ćemo je dati ako postanete kao i mi, ako obrežete sve svoje muškarce. 
\par 16 Onda vam možemo davati svoje kćeri i uzimati vaše sebi, s  vama se naseliti i biti jedan rod. 
\par 17 A ako ne pristajete na  obrezanje, uzet ćemo svoju kćer i otići." 
\par 18 Hamoru i Šekemu, Hamorovu sinu, njihov se zahtjev učini povoljan. 
\par 19 Mladić  nije časio da zahtjev izvrši, jer je čeznuo za Jakovljevom kćeri;  a bio je najuvaženiji od svih u očevu domu. 
\par 20 Tako Hamor i njegov sin Šekem dođu u svoje gradsko vijeće  i obrate se svojim sugrađanima ovako: 
\par 21 "Ovaj je svijet prijazan;  neka se među nama u zemlji nasele; neka se po njoj slobodno kreću;  ima dosta prostora u zemlji za njih; možemo uzimati njihove kćeri  sebi za žene, a njima davati svoje. 
\par 22 No ljudi će pristati  da među nama žive i s nama budu jedan rod samo ako se svi naši  muškarci obrežu kao što su oni obrezani. 
\par 23 Zar tako ne bi stoka  koju su stekli, sve njihovo blago - bilo naše? Pristanimo, pa  neka se među nama nasele!" 
\par 24 Svi odrasli muškarci koji imaju pravo izaći na gradska  vrata poslušaše Hamora i njegova sina Šekema, pa bude obrezan  svaki muškarac - svaki koji ima pravo izaći na gradska vrata. 
\par 25 A trećega dana, dok su oni još bili u bolovima, dva Jakovljeva  sina, Šimun i Levi, Dinina braća, pograbe svaki svoj mač i nesmetano  dođu u grad te poubijaju sve muškarce. 
\par 26 Sasijeku mačem Hamora  i njegova sina Šekema, uzmu Dinu iz Šekemove kuće i odu. 
\par 27 Ostali  Jakovljevi sinovi dođu na ubijene i opustoše grad što je njihova  sestra bila obeščašćena. 
\par 28 Što je bilo krupne i sitne stoke  i magaradi, u gradu i u polju, otjeraju; 
\par 29 opljačkaju sva njihova  dobra, a svu im djecu i žene - sve što je bilo po kućama - odvedu  u roblje. 
\par 30 Jakov reče Šimunu i Leviju: "Uveli ste me u nepriliku  omrazivši me stanovnicima zemlje, Kanaancima i Perižanima. Ako  se ujedine protiv mene i napadnu me, dok je nas ovako malo na  broj, istrijebit će me s mojim domom." 
\par 31 Oni odgovore: "Zar  da prema našoj sestri postupaju kao prema kakvoj bludnici?" 



\chapter{35}



\par 1 Bog reče Jakovu: "Ustani, idi gore u Betel te ondje ostani!  Načini ondje žrtvenik Bogu koji ti se objavio kad si bježao od  svoga brata Ezava!" 
\par 2 I Jakov reče svojoj obitelji i svima koji bijahu s njime:  "Odbacite tuđe kumire koji se nalaze u vašoj sredini; očistite  se i preobucite. 
\par 3 Idemo gore u Betel; ondje ću načiniti žrtvenik  Bogu, koji me uslišao kad sam bio u nevolji i sa mnom bio na  putu kojim sam hodio." 
\par 4 Oni predaju Jakovu sve tuđe kumire  što su ih imali i naušnice što su bile o njihovim ušima, pa ih  Jakov zakopa pod hrast kod Šekema. 
\par 5 Kad su se zaputili, strah  od Boga spopadne okolišna mjesta, tako da nisu išli u potjeru  za Jakovljevim sinovima. 
\par 6 Jakov stigne u Luz, to jest Betel, u zemlji kanaanskoj, i sav puk što je bio s njim. 
\par 7 Ondje sagradi žrtvenik i mjesto  nazva El Betel, jer mu se ondje Bog objavio kad on bježaše pred  svojim bratom Ezavom. 
\par 8 Tada umre Rebekina dojilja Debora te  je sahraniše pod Betelom, pod hrastom, koji se otad zove "Tužni  hrast". 
\par 9 Bog se opet objavi Jakovu kad je stigao iz Padan Arama, te ga blagoslovi. 
\par 10 Bog mu reče: "Ime ti je Jakov, ali se  odsad nećeš zvati Jakov nego će Izrael biti tvoje ime." Tako  ga prozva Izraelom. 
\par 11 Onda mu Bog reče: "Ja sam El Šadaj - Bog Svesilni! Budi  rodan i množi se! Od tebe poteći će narod, mnoštvo naroda, i  kraljevi iz tvog će izaći krila. 
\par 12 Zemlju što je dadoh Abrahamu  i Izaku tebi predajem; i potomstvu tvojem poslije tebe zemlju  ću ovu dati." 
\par 13 A onda Bog ode od njega gore. 
\par 14 Na mjestu gdje je Bog s njim govorio Jakov uspravi stup, stup od kamena; na njemu prinese žrtvu i izli ulja. 
\par 15 A mjesto  gdje mu je Bog govorio Jakov nazva Betel. 
\par 16 Potom odu iz Betela. Još bijaše malo puta do Efrate,  a Rahela se nađe pri porođaju. Napali je teški trudovi. 
\par 17 Kad  su joj porođajni bolovi bili najteži, reče joj babica: "Ne boj  se jer ti je i ovo sin!" 
\par 18 Kad se rastavljala s dušom - jer  umiraše Rahela - nadjenu sinu ime Ben Oni; ali ga otac prozva  Benjamin. 
\par 19 Tako umrije Rahela. Sahrane je na putu u Efratu, to jest Betlehem. 
\par 20 A na njezinu grobu Jakov podigne spomenik  - onaj što je na Rahelinu grobu do danas. 
\par 21 Izrael krenu dalje te razape svoj šator s onu stranu  Migdal-Edera. 
\par 22 Dok je Izrael boravio u onom kraju, ode Ruben  i legne s Bilhom, priležnicom svoga oca. Sazna za to Izrael. Izrael je imao dvanaest sinova. 
\par 23 S Leom: Rubena, koji  je Jakovljev prvorođenac, Šimuna, Levija, Judu, Jisakara i Zebuluna; 
\par 24 s Rahelom: Josipa i Benjamina; 
\par 25 s Bilhom, Rahelinom sluškinjom:  Dana i Naftalija; 
\par 26 sa Zilpom, sluškinjom Leinom: Gada i Ašera.  To su Jakovljevi sinovi što su mu se rodili u Padan Aramu. 
\par 27 Jakov dođe k svome ocu Izaku u Mamru u Kirjat Arbu, to  je Hebron - gdje su boravili Abraham i Izak kao pridošlice. 
\par 28 Kad  je Izaku bilo sto i osamdeset godina, umrije. 
\par 29 Izak izdahne  i umre, starac i godinama zasićen, te bude pridružen svojim precima.  Sahrane ga njegovi sinovi, Ezav i Jakov. 



\chapter{36}



\par 1 Ovo su potomci Ezava, koji se zvao i Edom. 
\par 2 Ezav je uzeo  sebi žene od kanaanskih djevojaka: Adu, kćer Hetita Elona; Oholibamu, kćer Ane, unuku Sibeona Horijca; 
\par 3 i Basematu, kćer Jišmaelovu, sestru Nebajotovu. 
\par 4 Ada Ezavu rodi Elifaza, a Basemata rodi  Reuela, 
\par 5 Oholibama rodi Jeuša, Jalama i Koraha. To su Ezavovi  sinovi koji se rodiše u zemlji kanaanskoj. 
\par 6 Ezav uzme svoje žene, svoje sinove, svoje kćeri, svu čeljad  svoga doma; svoju stoku - krupnu i sitnu; svu imovinu što ju  je namakao u zemlji kanaanskoj, pa ode u zemlju seirsku, daleko  od svog brata Jakova. 
\par 7 Njihov se, naime, posjed jako uvećao  te nisu mogli ostati zajedno: kraj u kojem su boravili nije ih  mogao izdržavati zbog njihova blaga. 
\par 8 Tako se Ezav - Edom nazvani  - naseli u brdskom kraju Seiru. 
\par 9 Ovo je, dakle, potomstvo Ezava, praoca Edomaca, u brdskom  kraju Seiru. 
\par 10 Ovo su imena Ezavovih sinova: Elifaz, sin Ezavove žene  Ade; Reuel, sin Ezavove žene Basemate. 
\par 11 Elifazovi su sinovi bili: Teman, Omar, Sefo, Gatam i  Kenaz. 
\par 12 Timna je bila inoča Ezavova sina Elifaza; ona je Elifazu  rodila Amaleka. To su potomci Ezavove žene Ade. 
\par 13 A ovo su sinovi Reuelovi: Nahat, Zerah, Šama i Miza.  Oni su bili sinovi Ezavove žene Basemate. 
\par 14 A ovo su opet sinovi Ezavove žene Oholibame, Anine kćeri, unuke Sibeonove; ona je Ezavu rodila Jeuša, Jalama i Koraha. 
\par 15 Ovo su rodovske glave Ezavovih potomaka. Potomci Ezavova prvorođenca Elifaza: knez Teman, knez Omar, knez Sefo, knez Kenaz, 
\par 16 knez Korah, knez Gatam i knez Amalek.  To su rodovski glavari Elifazovi u zemlji edomskoj; to su potomci  Adini. 
\par 17 A ovo su potomci Ezavova sina Reuela: knez Nahat,  knez Zerah, knez Šama i knez Miza. To su rodovski glavari Reuelovi  u zemlji edomskoj; to su potomci Ezavove žene Basemate. 
\par 18 A ovo su potomci Ezavove žene Oholibame: knez Jeuš, knez  Jalam i knez Korah. To su rodovski glavari Ezavove žene Oholibame, kćeri Anine. 
\par 19 To su bili sinovi Ezava-Edoma, njihovi knezovi. 
\par 20 A ovo su sinovi Seira Horijca, žitelji one zemlje: Lotan, Šobal, Sibeon, Ana, 
\par 21 Dišon, Eser i Dišan. To su koljenovići  Horijci, sinovi Seirovi, u zemlji edomskoj. 
\par 22 Lotanovi sinovi bili su: Hori i Hemam; a sestra Lotanova  bila je Timna. 
\par 23 Ovo su bili sinovi Šobalovi: Alvan, Manahat, Ebal, Šefo i Onam. 
\par 24 Sinovi Sibeonovi bijahu Aja i Ana. Ana  je onaj koji je našao vruća vrela u pustari dok je čuvao magarad  svoga oca Sibeona. 
\par 25 Ovo su bila djeca Ane: sin Dišon i Anina  kći Oholibama. 
\par 26 Ovo su bili sinovi Dišonovi: Hemdan, Ešban, Jitran i Keran. 
\par 27 Ovo su bili sinovi Eserovi: Bilhan, Zaavan  i Akan. 
\par 28 A sinovi Dišanovi bili su: Uz i Aran. 
\par 29 Ovo su knezovi Horijaca: knez Lotan, knez Šobal, knez  Sibeon, knez Ana, 
\par 30 knez Dišon, knez Eser i knez Dišan. To  su bili knezovi Horijaca, glavar za glavarom, u zemlji seirskoj. 
\par 31 Evo kraljeva koji su kraljevali u edomskoj zemlji prije  nego je zavladao kralj sinova Izraelovih. 
\par 32 Beorov sin Bela  vladao je u Edomu; njegov se grad zvao Dinhaba. 
\par 33 Kad je umro  Bela, na njegovo se mjesto zakraljio Jobab, sin Zeraha iz Bosre. 
\par 34 Kad je umro Jobab, zakraljio se na njegovo mjesto Hušam iz  temanske zemlje. 
\par 35 Kad je umro Hušam, zakraljio se na njegovo  mjesto Bedadov sin Hadad, koji je potukao Midjance na Moapskom  polju. Ime je njegovu gradu bilo Avit. 
\par 36 Kad je umro Hadad, zakraljio se na njegovo mjesto Samla iz Masreke. 
\par 37 Kad je  umro Samla, zakraljio se na njegovo mjesto Šaul iz Rehobota na  Rijeci. 
\par 38 Kad umrije Šaul, zavlada Baal Hanan, Akborov sin. 
\par 39 Kad je umro Baal Hanan, Akborov sin, vladaše Hadad. Ime je  njegovu gradu bilo Pai. Žena mu se zvala Mehetabela. Bila je  kći Matredova, iz Me Zahaba. 
\par 40 Ovo su imena Ezavovih knezova s njihovim nazivima po  rodovima i smještaju: knez Timna, knez Alva, knez Jetet, 
\par 41 knez  Oholibama, knez Ela, knez Pinon, 
\par 42 knez Kenaz, knez Teman,  knez Mibzar, 
\par 43 knez Magdiel i knez Iram. To su bili knezovi  edomski, prema njihovim naseljima u zemlji koju su zaposjeli.  To je Ezav, praotac Edomaca. 




\chapter{37}



\par 1 A Jakov se bijaše nastanio u zemlji gdje je njegov otac boravio  kao pridošlica - u zemlji kanaanskoj. 
\par 2 Evo nasljedstva Jakovljeva. Kao mladić, u dobi od sedamnaest godina, Josip je čuvao stada  sa svojom braćom, sinovima Bilhe i Zilpe, koje bijahu žene njegova  oca. Josip je ocu svome donosio zle glasove o njima. 
\par 3 Izrael je volio Josipa više nego ijednog svoga sina jer  je bio dijete njegove staračke dobi; i on mu napravi kićenu haljinu. 
\par 4 Kako njegova braća opaze da ga njihov otac voli više od svih  drugih svojih sinova, zamrze ga toliko da mu nisu mogli ni prijaznu  riječ progovoriti. 
\par 5 Jednom Josip usni san i kaza ga svojoj braći, a oni ga  zbog toga još više zamrze. 
\par 6 "Poslušajte", reče im, "san što  sam ga usnio! 
\par 7 Pomislite! Vezali smo nasred polja snopove,  kadli se najednom moj snop uspravi i stade uzgor. Uto se vaši  snopovi okupe okolo i duboko se poklone mom snopu." 
\par 8 Njegova  ga braća upitaše: "Kaniš li nad nama zakraljevati? Hoćeš li nam  biti gospodar?" I još ga više zamrze zbog njegova pričanja o  snovima. 
\par 9 Usni on još jedan san te ga ispriča svojoj braći:  "Još sam jedan san usnuo. Pazite! Sunce, mjesec i jedanaest zvijezda  duboko mi se klanjahu!" 
\par 10 Kad je to ispričao svome ocu, ukori  ga otac i reče mu: "Što znači taj san što si ga usnuo? Zar ćemo  doći ja, tvoja majka i tvoja braća pa ti se do zemlje klanjati?" 
\par 11 I dok su braća od zavisti bila ljuta na nj, njegov je otac  razmišljao o svemu. 
\par 12 Jednom njegova braća odu čuvati očeva stada blizu Šekema. 
\par 13 Izrael reče Josipu: "Tvoja braća čuvaju stada kod Šekema, pa hajde da te pošaljem k njima." On mu odgovori: "Dobro, idem." 
\par 14 Potom će mu otac: "Hajde i vidi kako su ti braća i stoka  pa mi javi." Tako ga otpremi iz doline Hebrona, i on stigne u  Šekem. 
\par 15 Neki čovjek nađe ga gdje luta poljem pa ga upita: "Što  tražiš?" 
\par 16 "Tražim braću", odgovori. "Možeš li mi kazati gdje  čuvaju stada?" 
\par 17 A čovjek reče: "Odavde su otišli. Čuo sam  ih gdje govore: 'Hajdemo u Dotan.'" Tako Josip ode za svojom  braćom i nađe ih u Dotanu. 
\par 18 Oni ga opaze izdaleka; prije nego im se približio, počnu  se dogovarati da ga ubiju. 
\par 19 I jedan drugom reče: "Eno stiže  onaj sanjar! 
\par 20 Hajde da ga sad ubijemo i bacimo u kakvu čatrnju!  Možemo kazati da ga je proždrla divlja zvijer. Vidjet ćemo što  će biti od njegovih snova!" 
\par 21 Ali kad je to čuo Ruben, pokuša da ga izbavi iz njihovih  šaka. I reče: "Nemojmo oduzimati njegova života! 
\par 22 Ne prolijevajte  krvi" - dalje je govorio Ruben. "Bacite ga u čatrnju u pustari;  ali ne dižite na nj ruke!" Htio ga je tako izbaviti iz njihovih  šaka i odvesti ocu. 
\par 23 Ali kad je Josip stigao braći, oni s  Josipa svuku njegovu haljinu, haljinu kićenu što je bila na njemu; 
\par 24 pograbe ga i bace u čatrnju. Čatrnja je bila prazna; nije  bilo u njoj vode. 
\par 25 Potom sjednu da ručaju. Kako podignu svoje oči, opaze povorku Jišmaelaca gdje dolazi  iz Gileada. Deve su im nosile mirodije, balzam i mirisavu smolu  da ih preprodaju u Egipat. 
\par 26 Tada reče Juda svojoj braći: "Što  ćemo postići ako ubijemo svog brata a krv njegovu sakrijemo? 
\par 27 Hajde da ga prodamo Jišmaelcima; ali ne dižimo na nj ruke.  TÓa on je naš brat, naše meso." Braća ga poslušaju. 
\par 28 Uto naiđu ljudi, midjanski trgovci. Braća izvuku Josipa  iz čatrnje i prodaju ga za dvadeset srebrnika Jišmaelcima, a  oni Josipa dovedu u Egipat. 
\par 29 Kad se Ruben vratio k čatrnji i vidio da Josipa nema  u čatrnji, razdere svoju odjeću. 
\par 30 A kad se vratio svojoj braći, povika: "Dječaka nema! Kamo ću ja sad?" 
\par 31 A oni uzmu Josipovu haljinu, zakolju jedno kozle i haljinu  zamoče u krv. 
\par 32 Kićenu haljinu otpreme ocu i poruče: "Ovo smo  našli; gledaj je li ovo haljina tvoga sina ili nije." 
\par 33 Prepozna  je on pa reče: "Haljina je moga sina! Divlja ga je zvijer rastrgla!  Na komade je Josip rastrgan!" 
\par 34 I razdere Jakov svoje haljine, stavi pokorničku kostrijet oko bokova i dugo vremena oplakivaše  svoga sina. 
\par 35 Svi su ga njegovi sinovi i sve njegove kćeri  nastojali utješiti, ali se on ne mogaše utješiti. Govorio je:  "Ne, sići ću k svome sinu u Šeol tugujući!" Tako ga je oplakivao  njegov otac. 
\par 36 A Midjanci ga prodaju u Egipat Potifaru, dvoraninu faraonovu, zapovjedniku straže. 



\chapter{38}



\par 1 Otprilike u to vrijeme Juda ode od svoje braće te okrenu nekom  Adulamcu komu ime bijaše Hira. 
\par 2 Tu Juda zapazi kćer jednog  Kanaanca - zvao se Šua - i njome se oženi. Priđe njoj 
\par 3 te ona  zače i rodi sina, komu dade ime Er. 
\par 4 Opet ona zače, rodi sina  i dade mu ime Onan. 
\par 5 Još jednog sina rodi te mu nadjene ime  Šela. Nalazila se u Kezibu kad je njega rodila. 
\par 6 Juda oženi svoga prvorođenca Era djevojkom kojoj bijaše  ime Tamara. 
\par 7 Ali Judin prvorođenac Er uvrijedi Jahvu i Jahve  ga pogubi. 
\par 8 Tada reče Juda Onanu: "Priđi k udovici svoga brata, izvrši prema njoj djeversku dužnost i tako očuvaj lozu svome  bratu!" 
\par 9 Ali Onan, znajući da se sjeme neće računati kao njegovo, ispuštaše ga na zemlju kad god bi prišao bratovoj udovici, tako  da ne dade potomstva svome bratu. 
\par 10 To što je činio uvrijedilo  je Jahvu, pa i njega pogubi. 
\par 11 Onda Juda reče svojoj nevjesti  Tamari: "Ostani kao udovica u domu svoga oca dok poodraste moj  sin Šela." Bojao se, naime, da bi i on mogao umrijeti kao i njegova  braća. I tako Tamara ode da živi u očevu domu. 
\par 12 Dugo vremena poslije toga umre Šuina kći, Judina žena.  Kad je prošlo vrijeme žalosti, Juda ode, zajedno sa svojim prijateljem  Adulamcem Hirom, u Timnu da striže svoje ovce. 
\par 13 Obavijeste  Tamaru: "Eno ti je svekar", rekoše joj, "na putu u Timnu da striže  ovce." 
\par 14 Ona svuče udovičko ruho, navuče koprenu i zamota se  pa sjede na ulazu u Enajim, što je na putu k Timni. Vidjela je, naime, da je Šela odrastao, ali nju još ne udaše za nj. 
\par 15 Kad je Juda opazi, pomisli da je bludnica, jer je bila  pokrila lice. 
\par 16 Svrati se on k njoj i reče: "Daj da ti priđem!"  Nije znao da mu je nevjesta. A ona odgovori: "Što ćeš mi dati  da uđeš k meni?" 
\par 17 "Spremit ću ti jedno kozle od svoga stada", odgovori. "Treba da ostaviš jamčevinu dok ga ne pošalješ." 
\par 18 A  on zapita: "Kakvu jamčevinu da ti ostavim?" Ona odgovori: "Svoj  pečatnjak o vrpci i štap što ti je u ruci." Dade joj jedno i  drugo, a onda priđe k njoj i ona po njem zače. 
\par 19 Potom ona  ustade i ode; skide sa sebe koprenu i opet se odjenu u svoje  udovičko ruho. 
\par 20 Uto Juda pošalje kozle po svom prijatelju Adulamcu da  iskupi jamčevinu iz ruku žene, ali je nije mogao naći. 
\par 21 Upita  ljude u mjestu: "Gdje je bludnica što se nalazila uz put u Enajim?"  Oni mu odgovore: "Ovdje nije nikad bilo bludnice." 
\par 22 Tako se  on vrati k Judi pa reče: "Nisam je mogao naći. Osim toga, ljudi  mi u mjestu rekoše da ondje nije nikad bilo bludnice." 
\par 23 Onda  reče Juda: "Da ne ostanemo za ruglo, neka ih drži! Slao sam joj, eto, ovo kozle, ali je ti nisi našao." 
\par 24 Otprilike poslije tri mjeseca donesoše vijest Judi: "Tvoja  nevjesta Tamara odala se bludništvu; čak je u bludničenju i začela."  "Izvedite je", naredi Juda, "pa neka se spali!" 
\par 25 Dok su je  izvodili, ona poruči svekru: "Začela sam po čovjeku čije je ovo."  Još doda: "Vidi čiji je ovaj pečatnjak o vrpci i ovaj štap!" 
\par 26 Juda ih prepozna pa reče: "Ona je pravednija nego ja, koji  joj nisam dao svoga sina Šelu." Ali više s njom nije imao posla. 
\par 27 Kad joj je došlo vrijeme da rodi, pokaže se da nosi blizance. 
\par 28 Dok je rađala, jedan od njih pruži ruku van. Nato babica  priveže za njegovu ruku crven konac govoreći: "Ovaj je izišao  prvi." 
\par 29 Ali baš tada on uvuče ruku te iziđe njegov brat. A  ona reče: "Kakav li proder napravi!" Stoga mu nadjenu ime Peres. 
\par 30 Poslije iziđe njegov brat koji je oko ruke imao crveni konac.  Njemu dadoše ime Zerah. 



\chapter{39}



\par 1 Josipa dovedoše u Egipat. Tu ga od Jišmaelaca koji su ga onamo  doveli kupi Egipćanin Potifar, dvoranin faraonov i zapovjednik  njegove tjelesne straže. 
\par 2 Jahve je bio s Josipom, zato je u  svemu imao sreću: Egipćanin ga uzme k sebi u kuću. 
\par 3 Vidje njegov  gospodar da je Jahve s njim i da svemu što mu ruka poduzme Jahve  daje uspjeh; 
\par 4 zavolje on Josipa, uze ga za dvoranina i postavi  ga za upravitelja svoga doma i povjeri mu sav svoj imetak. 
\par 5 I  otkad mu je povjerio upravu svoga doma i svega svog imetka, blagoslovi  Jahve dom Egipćaninov zbog Josipa: blagoslov Jahvin bijaše na  svemu što je imao - u kući i u polju. 
\par 6 I tako sve svoje prepusti  brizi Josipovoj te se više ni za što nije brinuo, osim za jelo  što je jeo. A Josip je bio mladić stasit i naočit. 
\par 7 Poslije nekog vremena žena njegova gospodara zagleda se  u Josipa i reče mu: "Legni sa mnom!" 
\par 8 On se oprije i reče ženi  svoga gospodara: "Gledaj! Otkako sam ja ovdje, moj se gospodar  ne brine ni za što u kući; sve što ima meni je povjerio. 
\par 9 On  u ovoj kući nema više vlasti negoli ja i ništa mi ne krati, osim  tebe, jer si njegova žena. Pa kako bih ja mogao učiniti tako  veliku opačinu i sagriješiti protiv Boga!" 
\par 10 Iako je Josipa  salijetala iz dana u dan, on nije pristajao da uz nju legne;  nije joj prilazio. 
\par 11 Jednog dana Josip uđe u kuću na posao. Kako nikog od  služinčadi nije bilo u kući, 
\par 12 ona ga uhvati za ogrtač i reče:  "Legni sa mnom!" Ali on ostavi svoj ogrtač u njezinoj ruci, otrže  se i pobježe van. 
\par 13 Vidjevši ona da je u njezinoj ruci ostavio  ogrtač i pobjegao van, 
\par 14 zovne svoje sluge te im reče: "Gledajte!  Trebalo je da nam dovede jednog Hebrejca da se s nama poigrava.  Taj k meni dođe da sa mnom legne, ali sam ja na sav glas zaviknula. 
\par 15 A čim je čuo kako vičem, ostavi svoj ogrtač pokraj mene i  pobježe van." 
\par 16 Uza se je držala njegov ogrtač dok mu je gospodar došao  kući. 
\par 17 Onda i njemu kaza istu priču: "Onaj sluga Hebrejac  koga si nam doveo dođe k meni da sa mnom ljubaka! 
\par 18 Ali čim  je čuo kako vičem, ostavi svoj ogrtač pokraj mene i pobježe van." 
\par 19 Kad je njegov gospodar čuo pripovijest svoje žene koja reče:  "Eto, tako sa mnom tvoj sluga", razgnjevi se. 
\par 20 Gospodar pograbi  Josipa i baci ga u tamnicu - tamo gdje su bili zatvoreni kraljevi  utamničenici. I osta u tamnici. 
\par 21 Ali je Jahve bio s njim, iskaza naklonost Josipu te on  nađe milost u očima upravitelja tamnice. 
\par 22 Tako upravitelj  tamnice preda u Josipove ruke sve utamničenike koji su se nalazili  u tamnici; i ondje se ništa nije radilo bez njega. 
\par 23 Budući  da je Jahve bio s njim, upravitelj tamnice nije nadgledao ništa  što je Josipu bilo povjereno: Jahve bijaše s njim, i što god  bi poduzeo, Jahve bi to okrunio uspjehom. 



\chapter{40}



\par 1 Poslije toga peharnik se i pekar egipatskog kralja ogriješe  o svoga gospodara, kralja egipatskog. 
\par 2 Faraon se razljuti na  svoja dva dvoranina, glavnog peharnika i glavnog pekara, 
\par 3 te  ih stavi u zatvor, u zgradu zapovjednika tjelesne straže - u  istu tamnicu gdje je i Josip bio zatvoren. 
\par 4 Zapovjednik tjelesne  straže odredi Josipa da ih poslužuje. Pošto su proveli u zatvoru neko vrijeme, 
\par 5 obojica njih  - peharnik i pekar egipatskog kralja, utamničenici - usnu san  jedne te iste noći. Svaki je usnuo svoj san; i svaki je san imao  svoje značenje. 
\par 6 Kad je Josip ujutro došao k njima, opazi da  su neraspoloženi. 
\par 7 Upita faraonove dvorane koji su bili s njim  u zatvoru u zgradi njegova gospodara: "Zašto ste danas tako potišteni?" 
\par 8 Odgovore mu: "Sne smo usnuli, ali nikog nema da nam ih protumači."  Josip im reče: "Zar tumačenje ne spada na Boga? Dajte, pričajte  mi!" 
\par 9 Onda je glavni peharnik ispripovjedio Josipu svoj san:  "Sanjao sam da je preda mnom lozov trs. 
\par 10 Na trsu bile tri  mladice. I tek što je propupao, procvjeta i na njegovim grozdovima  sazru bobe. 
\par 11 Kako sam u ruci držao faraonov pehar, uzmem grožđa, istiještim ga u faraonov pehar, a onda stavim pehar u faraonovu  ruku." 
\par 12 Josip mu reče: "Ovo ti je značenje: tri mladice tri  su dana. 
\par 13 Poslije tri dana faraon će te pomilovati i vratiti  na tvoje mjesto; opet ćeš stavljati pehar faraonu u ruku, kao  i prije, dok si mu bio peharnik. 
\par 14 Kada ti bude opet dobro, sjeti se da sam i ja bio s tobom, pa mi učini ovu uslugu: spomeni  me faraonu i pokušaj me izvesti iz ove kuće. 
\par 15 Jer, zbilja, bio sam silom odveden iz zemlje Hebreja; ni ovdje nisam ništa  skrivio, a baciše me u tamnicu." 
\par 16 Kad je glavni pekar vidio kako je Josip dao dobro tumačenje, reče mu: "Usnuh da su mi na glavi tri bijele košare. 
\par 17 U najgornjoj  bilo svakovrsna peciva što ga pekar pripravlja faraonu, ali su  ptice jele iz košare povrh moje glave." 
\par 18 Josip odgovori: "Ovo  je značenje: tri košare tri su dana. 
\par 19 Poslije tri dana faraon  će uzdići tvoju glavu i o drvo te objesiti te će ptice jesti  meso s tebe." 
\par 20 I zaista, trećega dana, kad je faraon priredio gozbu  za sve svoje službenike - bio mu je rođendan - iz sredine svojih  službenika izluči glavnog peharnika i glavnog pekara. 
\par 21 Vrati  glavnog peharnika u peharničku službu te je i dalje stavljao  pehar u faraonovu ruku, 
\par 22 a glavnog pekara objesi, kako je  Josip protumačio. 
\par 23 Ipak se glavni peharnik nije sjetio Josipa  - zaboravio je na nj. 



\chapter{41}



\par 1 Poslije dvije godine usnu faraon da stoji pokraj Nila. 
\par 2 Iz  Nila iziđe sedam krava, lijepih i debelih; pasle su po šašu. 
\par 3 Ali odmah poslije njih iz Nila iziđe sedam drugih krava, ružnih  i mršavih, te stanu uz one krave na obali Nila. 
\par 4 Ružne i mršave  krave požderu ono sedam lijepih i pretilih, i uto se faraon probudi. 
\par 5 Opet zaspi te usnu drugi san: sedam punih i jedrih klasova  izraste na jednoj stabljici. 
\par 6 Ali, eto, poslije njih uzraste  sedam klasova šturih, istočnjakom opaljenih. 
\par 7 Šturi klasovi  proždru sedam jedrih i punih klasova. I faraon se probudi, i  gle: bio je to san. 
\par 8 Ujutro faraon bijaše uznemiren u duši, pa pozva sve čarobnjake  i sve mudrace egipatske: ispriča im faraon svoje sne, ali mu  ih nitko nije mogao protumačiti. 
\par 9 Onda progovori faraonov glavni  peharnik: "Moram danas spomenuti jedan svoj propust. 
\par 10 Jednom, kad se faraon razljutio na svoje službenike, mene i glavnog  pekara stavio je u zatvor u zgradi glavnog upravitelja. 
\par 11 Usnusmo  san iste noći, i ja i on, ali je svaki od nas usnuo san drugog  značenja. 
\par 12 Onda je s nama bio neki mladi Hebrej, sluga zapovjednika  straže. Ispričasmo njemu svoje sne, a on nam ih protumači: kaza  svakom značenje njegova sna. 
\par 13 Kako nam ih je protumačio, tako  nam se i dogodilo: mene vratiše na moje mjesto, a onoga objesiše." 
\par 14 Faraon odmah pošalje po Josipa; izvuku ga brže-bolje  iz tamnice; ošišaju mu kosu, obuku novo odijelo i on stupi pred  faraona. 
\par 15 Onda faraon reče Josipu: "Usnuo sam san, a nitko  ga ne može protumačiti. Čuo sam o tebi da možeš protumačiti san  čim ga čuješ." 
\par 16 "Ništa ja ne mogu", odgovori Josip faraonu, "nego će Bog dati pravi odgovor faraonu." 
\par 17 Onda je faraon pripovjedao Josipu: "U svom snu stojim  na obali Nila. 
\par 18 I gle! Iz Nila iziđe sedam debelih i lijepih  krava. Pasle su po šašu. 
\par 19 Poslije njih izađe drugih sedam  krava. Bile su mršave, vrlo ružne i koštunjave. Još nikad ne  vidjeh onako ružnih krava u svoj zemlji egipatskoj! 
\par 20 I sedam  mršavih i ružnih krava proždru prvih sedam debelih krava. 
\par 21 Pa  iako su ih progutale, nije se vidjelo da im je što u trbuhu:  bile su ružne kao i prije. Uto se probudim. 
\par 22 Zatim sam u snu  vidio kako na jednoj stabljici uzraste sedam punih i lijepih  klasova. 
\par 23 Ali poslije njih uzraste sedam klasova zgrčenih, šturih, istočnjakom opaljenih. 
\par 24 I šturi klasovi proždru sedam  jedrih klasova. Kazao sam ovo i vračarima, ali nema nikoga da  mi razjasni." 
\par 25 Onda Josip reče faraonu: "Faraonov je san samo jedan:  Bog javlja faraonu što kani učiniti. 
\par 26 Sedam lijepih krava, to je sedam godina; sedam lijepih klasova opet je sedam godina.  Tako je samo jedan san. 
\par 27 Sedam mršavih i ružnih krava poslije  njih, a tako i sedam praznih, istočnjakom opaljenih klasova,  označuje sedam gladnih godina. 
\par 28 To je ono što sam već faraonu  rekao: Bog objavljuje faraonu što kani učiniti. 
\par 29 Dolazi, evo, sedam godina velikog obilja svoj zemlji egipatskoj. 
\par 30 A poslije  njih nastat će sedam gladnih godina, kada će se zaboraviti sve  obilje u zemlji egipatskoj. 
\par 31 Kako glad bude harala zemljom, neće se ni znati da je u zemlji bilo obilje - zbog gladi koja  će doći - jer će biti vrlo velika. 
\par 32 A što se faraonov san  ponovio, znači da se Bog na to zaista odlučio i da će to uskoro  provesti. 
\par 33 Zato neka faraon izabere sposobna i mudra čovjeka  te ga postavi nad zemljom egipatskom. 
\par 34 Nadalje, neka se faraon  pobrine da postavi nadglednika u zemlji koji će kÓupiti petinu  sve žetve u zemlji egipatskoj za sedam godina obilja. 
\par 35 Neka  skupljaju od svakog žita za sedam dobrih godina što dolaze; neka  s ovlaštenjem faraonovim sabiru žito za hranu i pohranjuju ga  po gradovima. 
\par 36 Neka zalihe služe za hranu u zemlji za sedam  godina gladi što će snaći zemlju egipatsku, tako da za gladi  zemlja ne propadne." 
\par 37 Svidje se odgovor faraonu i svim njegovim službenicima. 
\par 38 Zato faraon reče svojim službenicima: "Zar bismo mogli naći  drugoga kao što je on, čovjeka koji bi bio tako obdaren duhom  Božjim?" 
\par 39 A onda faraon reče Josipu: "Otkako je sve to Bog  tebi otkrio, nikoga nema sposobna i mudra kao što si ti. 
\par 40 Ti  ćeš biti upravitelj moga dvora: sav će se moj narod pokoravati  tvojim naredbama. Jedino prijestoljem ja ću biti veći od tebe. 
\par 41 Postavljam te, evo," reče faraon Josipu, "nad svom zemljom  egipatskom." 
\par 42 Poslije toga skine faraon sa svoje ruke pečatni  prsten i stavi ga Josipu na ruku. Zatim zaodjene Josipa odjećom  od najljepše tkanine, a o vrat mu objesi zlatan lanac. 
\par 43 Vozio  se on u kolima kao njegov zamjenik, a pred njim klicahu: "Abrek!  Na koljena!" Tako ga postavi nad svu zemlju egipatsku. 
\par 44 Još  faraon reče Josipu: "Premda sam ja faraon, neće nitko dići svoje  ruke ni noge bez tvog odobrenja u svoj zemlji egipatskoj." 
\par 45 Faraon  nazva Josipa "Safenat Paneah", a za ženu mu dade Asenatu, kćer  Poti-Fere, svećenika u Onu. I Josip postade poznat po zemlji  egipatskoj. 
\par 46 Josipu je bilo trideset godina kad je stupio u službu  faraona, kralja egipatskog. A otišavši Josip ispred faraona,  putovao je po svoj zemlji egipatskoj. 
\par 47 Za sedam rodnih godina  zemlja je rađala u obilju; 
\par 48 on je - u tih sedam godina što  ih je egipatska zemlja uživala - kÓupio od različite ljetine  i hranu pohranjivao u gradove, smještajući u svakom gradu urod  iz okolnih polja. 
\par 49 Tako Josip nagomila mnogo žita, kao pijeska  u moru, pa ga prestade i mjeriti jer mu mjere ne bijaše. 
\par 50 Dok još ne nasta gladna godina, Josip imade dva sina  koje mu rodi Asenata, kći Poti-Fere, svećenika u Onu. 
\par 51 Prvorođencu  Josip nadjenu ime Manaše, "jer Bog je", reče, "dao te sam zaboravio  svoje teškoće i svoj očinski dom." 
\par 52 Drugomu nadjenu ime Efrajim, "jer Bog me", reče, "učinio rodnim u zemlji moje nevolje." 
\par 53 Sedam godina obilja koje je uživala zemlja egipatska  dođe kraju, 
\par 54 a primače se sedam gladnih godina, kako je Josip  prorekao. U svim zemljama bijaše glad, a u svoj zemlji egipatskoj  bijaše kruha. 
\par 55 A kad je i sva zemlja egipatska osjetila glad, puk zavapi faraonu za kruh; a faraon reče Egipćanima: "Idite  k Josipu i što god vam rekne, činite!" 
\par 56 Kad se glad proširi  po svoj zemlji, Josip rastvori skladišta te je Egipćane opskrbljivao  žitom, jer je glad postala žestoka i u zemlji egipatskoj. 
\par 57 Sav  je svijet išao u Egipat k Josipu da kupuje žita, jer je strašna  glad vladala po svem svijetu. 



\chapter{42}



\par 1 Kad je Jakov čuo da u Egiptu ima žita, reče svojim sinovima:  "Što tu zurite jedan u drugoga? 
\par 2 Čujem da ima žita u Egiptu.  Otiđite dolje te nam ga odande nabavite da ostanemo na životu  i ne pomremo." 
\par 3 Tako desetero Josipove braće siđe da nabavi  žita iz Egipta. 
\par 4 Benjamina, Josipova pravog brata, Jakov ne  posla s ostalima. "Da ga ne bi zadesila kakva nesreća", govorio  je. 
\par 5 Među onima koji su išli nabavljati žito, jer u zemlji  kanaanskoj vladaše glad, bijahu i sinovi Izraelovi. 
\par 6 Josip je bio namjesnik u zemlji; on je dijelio žito svemu  svijetu. Dođu tako i Josipova braća i poklone mu se licem do  zemlje. 
\par 7 Josip prepozna braću čim ih ugleda, ali se prema njima  vladao kao stranac i oštro im govorio. Zapita ih: "Odakle dolazite?"  Odgovore: "Iz zemlje kanaanske došli smo da kupimo hrane." 
\par 8 Iako  je Josip prepoznao svoju braću, oni njega nisu prepoznali. 
\par 9 Josip  se sjeti snova što ih je o njima sanjao. I reče im: "Vi ste uhode!  Došli ste da izvidite slaba mjesta ove zemlje." 
\par 10 Oni mu odgovore:  "Ne, gospodaru! Tvoje su sluge došle da nabave hrane. 
\par 11 Svi  smo sinovi jednog oca; pošteni smo ljudi; sluge tvoje nikad nisu  bile uhode." 
\par 12 On će im opet: "Ne, nego ste došli da izvidite  slaba mjesta ove zemlje." 
\par 13 Nato oni uzvrate: "Nas, tvojih  slugu, bijaše dvanaestero braće - sinovi jednog oca, u zemlji  kanaanskoj; najmlađi je sad s ocem, a jednoga više nema." 
\par 14 No  Josip im dobaci: "Onako kako sam vam već rekao: vi ste uhode! 
\par 15 Ovako ću vas iskušati: odavde, tako mi faraona, nećete izići  ako vaš najmlađi brat ne dođe ovamo! 
\par 16 Pošaljite jednoga između  sebe da vam dovede brata, a vi ostali u zatvor! Tako ću iskušati  vaše riječi i vidjeti je li u vas istina ili nije. Inače, tako  mi faraona, vi ste uhode!" 
\par 17 Potom ih baci u zatvor na tri  dana. 
\par 18 Treći im dan reče Josip: "Izvršite to, i ostat ćete na  životu, jer sam ja čovjek bogobojazan. 
\par 19 Ako ste pošteni, neka  jedan od vas ostane u zatvoru, a vi ostali idite i nosite žito  svojim izgladnjelim domovima. 
\par 20 Poslije toga dovedite mi svoga  najmlađeg brata, tako da se obistine vaše riječi te da ne izginete."  Oni pristanu. 
\par 21 Zatim je jedan drugom govorio : "Jao nama!  Stiže nas kazna zbog našega brata; gledali smo njegovu muku dok  nas je molio za milost, ali ga nismo uslišali. Stoga nas je ova  nevolja snašla." 
\par 22 Ruben im odvrati: "Zar vam nisam govorio:  Ne ogrešujte se o mladića! Ali vi niste slušali. Sad se traži  račun za njegovu krv." 
\par 23 Nisu znali da ih Josip razumije, jer  su se s njim razgovarali preko tumača. 
\par 24 On se od njih udalji  te zaplaka. Opet se vrati i razgovaraše s njima. Onda izdvoji  Šimuna između njih i naredi da bude svezan na njihove oči. 
\par 25 Potom Josip zapovjedi da im vreće napune žitom; da svakome  njegov novac metnu u vreću i da im daju poputninu. Tako im učine. 
\par 26 Tada oni natovare žito na svoje magarce i krenu odande. 
\par 27 Kad  na prenoćištu jedan od njih otvori svoju vreću da nahrani magarca, opazi svoj novac ozgo u vreći. 
\par 28 "Moj je novac vraćen!" -  povika braći. - "Evo ga u mojoj vreći!" Zadrhta srce u njima.  Zgledaše se, uplašeni, i rekoše: "Što nam ovo Bog uradi!" 
\par 29 Došavši k svome ocu Jakovu u zemlju kanaansku, kazaše  mu sve što ih je snašlo. 
\par 30 "Čovjek koji je gospodar one zemlje", rekoše, "oštro nam je govorio i optužio nas kao uhode. 
\par 31 Pošteni  smo ljudi, kazasmo mu, i nikad nismo bili uhode. 
\par 32 Bilo nas  je dvanaestero braće, sinovi istog oca, ali jednoga više nema, dok se najmlađi sad nalazi s našim ocem u zemlji kanaanskoj. 
\par 33 Ali čovjek koji je gospodar one zemlje reče nam: 'Ovim ću  doznati da ste pošteni ljudi: ostavite jednoga brata kod mene, a vi ostali uzmite što vam treba za izgladnjele domove, pa idite. 
\par 34 Onda mi dovedite svoga najmlađeg brata, tako da znam da niste  uhode, nego pošteni ljudi. Poslije toga vratit ću vam vašeg brata, i vi ćete se moći slobodno kretati u ovoj zemlji.'" 
\par 35 Kako su praznili svoje vreće, svaki nađe u vreći svoju  kesu. Opazivši to, zapadoše u strah - i oni i njihov otac. 
\par 36 "Mene  vi ostavljate bez djece!" - reče im njihov otac. - "Josipa je  nestalo, Šimuna nema, a sad biste odveli i Benjamina. Sve se  to na me svaljuje!" 
\par 37 Onda Ruben reče svome ocu: "Ubij moja dva sina ako ti  ga ja natrag ne dovedem! Predaj ga u moje ruke, i ja ću ti ga  vratiti!" 
\par 38 "Moj sin neće s vama!" - uzvrati on. - "Njegov  je pravi brat već mrtav, a on je ostao sam. Ako bi ga na putu  na koji ćete poći snašla nesreća, u tuzi biste otpravili moju  sijedu glavu dolje u Šeol." 



\chapter{43}



\par 1 Strašna glad pritisla zemlju. 
\par 2 Kad su pojeli hranu koju  bijahu donijeli iz Egipta, njihov im otac reče: "Idite opet i  nabavite nam malo hrane." 
\par 3 Nato će mu Juda: "Onaj nam je čovjek  jasno rekao: 'Ne smijete preda me ako vaš brat ne bude s vama.' 
\par 4 Ako si, dakle, voljan s nama poslati našega brata, mi ćemo  otići dolje i kupit ćemo ti žita. 
\par 5 Ali ako njega ne pustiš  s nama, onda mi tamo i ne idemo, jer nam je onaj zaprijetio:  'Ne smijete preda me ako vaš brat ne bude s vama.'" 
\par 6 "Zašto  ste mi", zapita Izrael, "nanijeli jad rekavši onom čovjeku da  imate još jednoga brata?" 
\par 7 Oni odgovore: "Čovjek nas je neprestano  zapitkivao o nama i o našoj obitelji: 'Je li vam još živ otac?  Imate li još kojega brata?' Mi smo mu odgovarali na pitanja.  Kako smo mogli znati da će reći : 'Dovedite svoga brata!'" 
\par 8 Potom Juda reče svome ocu Izraelu: "Pusti dječaka sa mnom  pa da se dignemo i krenemo; tako ćemo preživjeti, a ne pomrijeti, i mi, i ti, i naša djeca. 
\par 9 Ja za nj jamčim; mene drži odgovornim  za nj. Ako ga tebi ne vratim i preda te ga ne dovedem, bit ću  ti kriv svega vijeka. 
\par 10 TÓa da nismo toliko oklijevali, mogli  smo se već i dvaput vratiti." 
\par 11 Njihov otac Izrael reče im: "Kad je tako, neka bude,  ali učinite ovo: metnite u torbe najbiranijih proizvoda ove zemlje  i ponesite na dar onom čovjeku: nešto balzama, nešto meda i mirodija, mirisne smole, pa lješnjaka i badema. 
\par 12 Sa sobom uzmite dvostruko  novaca, jer treba vratiti novac koji ste našli u grlima svojih  vreća. Možda je ono bila zabuna. 
\par 13 Uzmite svoga brata pa se  opet zaputite onom čovjeku. 
\par 14 Neka Bog Svemogući, El Šadaj, potakne onog čovjeka na milosrđe prema nama te vam pusti i drugoga  brata i Benjamina. A ja, moram li bez djece ostati, neka ostanem." 
\par 15 Uzmu ljudi darove; uzmu sa sobom dvostruko novaca, povedu  Benjamina te siđu u Egipat i stupe pred Josipa. 
\par 16 Kad Josip  ugleda s njima Benjamina, reče upravitelju svoga kućanstva: "Odvedi  ljude u kuću, zakolji jedno živinče i pripremi, jer će ovi ljudi  blagovati sa mnom o podne!" 
\par 17 Čovjek učini kako je Josip rekao  i povede ljude u Josipov dom. 
\par 18 Ljudi se pobojaše kad su bili povedeni u dom Josipov  te rekoše: "Zbog novca koji se našao u našim vrećama prvi put  vode nas unutra tako da nas napadnu i zajedno s našom magaradi  uzmu za robove." 
\par 19 Stoga se primaknu upravitelju Josipova doma  te mu, na ulazu u kuću, reknu: 
\par 20 "Oprosti, gospodaru! Mi smo  i prije jednom dolazili da nabavimo hrane; 
\par 21 i kad smo stigli  na prenoćište i otvorili svoje vreće, a to novac svakoga od nas  ozgo u njegovoj vreći, naš novac, ista svota. Sad smo ga donijeli  sa sobom. 
\par 22 A ponijeli smo i drugog novca da kupimo hrane.  Mi ne znamo tko nam je stavio novac u naše vreće." 
\par 23 "Budite  mirni", reče im on. "Ne bojte se! Bog vaš i Bog vašega oca stavio  je blago u vaše vreće. Vaš je novac k meni stigao." Potom im  izvede Šimuna. 
\par 24 Čovjek zatim uvede ljude u Josipovu kuću; dade im vode  da operu noge, a njihovoj magaradi baci p§iće. 
\par 25 Potom priprave  oni svoje darove za dolazak Josipov o podne, jer su čuli da će  ondje ručati. 
\par 26 Kad je Josip došao u kuću, dadu mu darove koje su sa  sobom donijeli i do zemlje mu se poklone. 
\par 27 Upita ih on za  zdravlje te će dalje: "A je li dobro vaš stari otac o kome ste  mi govorili? Je li još dobra zdravlja?" 
\par 28 "Sluga tvoj, otac  naš, dobro je i još je dobra zdravlja", odgovore i duboko se  naklone iskazujući poštovanje. 
\par 29 Podigavši svoje oči, Josip  opazi svoga brata Benjamina - sina svoje majke - te upita: "Je  li ovo vaš najmlađi brat o kome ste mi govorili?" Onda nastavi:  "Bog ti bio milostiv, sine moj!" 
\par 30 Josip se poslije toga požuri  van jer mu se srce uzbudilo zbog brata; bilo mu je da zaplače.  Uđe u jednu sobu i tu se isplaka. 
\par 31 Onda opere lice, ponovo  se javi i, svladavajući se, naredi: "Poslužite ručak!" 
\par 32 Staviše  njemu napose, njima napose, a napose opet Egipćanima koji su  s njim jeli. Egipćani ne bi mogli jesti s Hebrejima, jer bi to  Egipćanima bilo odvratno. 
\par 33 I kad posjedaše pred njim, najstariji  prema starosti svojoj, a najmlađi prema mladosti svojoj, samo  se zgledahu. 
\par 34 I naređivaše on da jela ispred njega nose njima, a obrok Benjaminov bijaše pet puta veći od svih ostalih. I pili  su i gostili se s njim. 



\chapter{44}



\par 1 Onda Josip naredi upravitelju svoga kućanstva: "Napuni vreće  ovih ljudi hranom koliko mogu ponijeti, a novac svakog stavi  u grlo njegove vreće. 
\par 2 A moj pehar - onaj od srebra - stavi  u grlo vreće najmlađega, zajedno s njegovim novcem za žito."  On učini kako mu je Josip naredio. 
\par 3 Kad je svanulo, otpreme ljude i njihove magarce. 
\par 4 Tek  što su izišli iz grada - nisu bili odmakli daleko - kad Josip  reče upravitelju svoga kućanstva: "Na noge! Pođi za onim ljudima!  Kad ih stigneš, kaži im: 'Zašto uzvraćate zlo za dobro? 
\par 5 Zar  iz onog pehara ne pije moj gospodar i ne čita iz njega proricanje?  Zlo ste učinili!'" 
\par 6 Stigavši ih, ponovi im te riječi. 
\par 7 Oni odgovore: "Zašto  nam gospodar govori tako? Daleko bilo od slugu tvojih da učine  takvo što! 
\par 8 Čak i novac koji smo našli u svojim vrećama donijeli  smo ti natrag iz zemlje kanaanske. Kako bismo onda mogli ukrasti  srebra ili zlata iz kuće tvoga gospodara! 
\par 9 Onaj u koga se od  tvojih slugu nađe, neka se usmrti, a mi drugi postat ćemo robovi  tvome gospodaru." 
\par 10 "Premda je ono što predlažeš pravo", preuzme  on, "ipak će samo onaj u koga se ukradeno pronađe biti moj rob, a ostali bit ćete slobodni." 
\par 11 Brže spustiše vreće na zemlju i svaki svoju otvori. 
\par 12 On  je pretraživao, počevši s najstarijim i završivši s najmlađim.  Pehar se nađe u Benjaminovoj vreći. 
\par 13 Nato oni razdru svoje  haljine; svaki ponovo natovari svoga magarca i vrate se u grad. 
\par 14 Kad su Juda i njegova braća ponovo stupili u Josipov  dom, još je on bio ondje. Bace se preda nj na zemlju. 
\par 15 Onda  im Josip reče: "Kakvo je to djelo što ste ga učinili? Zar ne  znate da se čovjek kao što sam ja bavi proricanjem?" 
\par 16 Nato  Juda odgovori: "Što bismo mogli reći svome gospodaru? Što možemo  kazati, čime li se opravdati? Bog je otkrio zlodjelo tvojih slugu.  Evo nas za robove svome gospodaru - jednako nas kao i onog u  koga se našao pehar." 
\par 17 "Daleko od mene da učinim tako!" -  odgovori. "Nego, onaj u koga se našao pehar bit će moj rob, a  vi drugi pođite mirno k svome ocu!" 
\par 18 Onda mu se Juda primače i reče: "Gospodaru moj, molim  te, dopusti sluzi svojem da rekne riječ ušima gospodara svojega  i neka se tvoja srdžba ne razlijeva na tvog slugu. TÓa ti si  ravan faraonu. 
\par 19 Pitao je moj gospodar svoje sluge: 'Imate  li oca ili još kojega brata?' 
\par 20 Svome smo gospodaru odgovorili:  'Imamo stara oca; on još ima jednog sina, rođena u njegovoj staračkoj  dobi. Taj je najmlađi. Njegov je pravi brat umro, tako da je  on jedini ostao od svoje majke. Njegov ga otac osobito voli.' 
\par 21 Potom si rekao svojim slugama: 'Dovedite mi ga ovamo da ga  vide moje oči?' 
\par 22 A mi smo odgovorili svome gospodaru: 'Dječak  ne može ostaviti oca; kad bi ga ostavio, njegov bi otac umro.' 
\par 23 Nato si rekao svojim slugama: 'Ako vaš najmlađi brat s vama  ne dođe ovamo, više ne smijete preda me.' 
\par 24 Kad smo se vratili  tvome sluzi, ocu mome, kazali smo mu riječi moga gospodara. 
\par 25 Naš  nam je otac rekao: 'Idite opet i nabavite nam malo hrane!' 
\par 26 Odgovorili  smo: 'Ne možemo onamo. Samo ako s nama pođe naš najmlađi brat, sići ćemo, jer ne smijemo pred onoga čovjeka ako ne bude s nama  naš najmlađi brat.' 
\par 27 Tvoj sluga, otac moj, odvrati nam: 'Kako  znate, žena mi je rodila dva sina. 
\par 28 Jedan je nestao, te sam  zaključio: sigurno je rastrgan! Odonda ga više nisam vidio. 
\par 29 Ako  i ovoga od mene odvedete pa ga kakva nesreća snađe, moju ćete  sijedu glavu s tugom strovaliti dolje u Šeol.' 
\par 30 Ako sad dođem  k tvome sluzi, ocu svome, a mladić - čiji je život tako povezan  s njegovim - ne bude s nama, 
\par 31 on će svisnuti kad vidi da dječaka  nema s nama; tako će tvoje sluge strovaliti u tuzi sijedu glavu  tvoga sluge, oca našega, dolje u Šeol. 
\par 32 Jer tvoj je sluga  zajamčio ocu svome za dječaka, rekavši: 'Ako ti ga ne vratim, bit ću kriv svome ocu svega vijeka.' 
\par 33 Zato, molim te, neka  tvoj sluga ostane kao rob mome gospodaru, a dječak neka ide natrag  s braćom. 
\par 34 Jer, kako mogu k svome ocu ako dječaka nema sa  mnom! Ne bih mogao gledati jad što bi snašao moga oca." 



\chapter{45}



\par 1 Josip se više nije mogo svladavati pred onima koji su ga okruživali  pa povika: "Neka svi odstupe!" Tako nitko nije ostao s Josipom  kad se očitovao svojoj braći. 
\par 2 Briznuo je u glasan plač, da  su ga i Egipćani mogli čuti. Doznalo se za to i na faraonovu  dvoru. 
\par 3 "Ja sam Josip", reče Josip svojoj braći. "Otac mi je,  dakle, još na životu!" Ali mu braća nisu mogla odgovoriti, toliko  se zapanjiše pred njim. 
\par 4 Onda će opet Josip svojoj braći: "Primaknite  se k meni!" Kad su se primakli, nastavi: "Ja sam Josip, vaš brat;  onaj koga ste prodali u Egipat. 
\par 5 Ali se nemojte uznemirivati  i prekoravati što ste me ovamo prodali; jer Bog je onaj koji  me pred vama poslao da vas održi u životu. 
\par 6 Dvije su već godine  što je glad došla na zemlju, a još pet godina neće biti ni oranja  ni žetve u zemlji. 
\par 7 Zato me Bog poslao pred vama da vam se  sačuva ostatak na zemlji te da vam život spasi velikim izbavljenjem. 
\par 8 Tako niste vi mene poslali ovamo nego Bog; on me postavio  faraonu za oca, gospodara nad svim njegovim domom i vladaocem  nad svom zemljom egipatskom. 
\par 9 Žurite se k mome ocu te mu recite: 'Ovo ti poručuje tvoj  sin Josip: Bog me postavio gospodarem nad svim Egiptom; siđi  k meni bez oklijevanja. 
\par 10 Nastanit ćeš se u kraju Gošenu. Tako  ćeš biti blizu mene: ti, tvoja djeca, tvoja unučad, tvoje ovce  i goveda i sve što je tvoje. 
\par 11 Ondje ću se za te brinuti, jer  će glad potrajati još pet godina. Tako nećeš oskudijevati ni  ti, ni tvoja obitelj, niti itko tvoj.' 
\par 12 Ta svojim očima možete  vidjeti, kao što vidi i moj brat Benjamin, da vam to moja usta  govore. 
\par 13 Pripovjedite ocu o mome visokom položaju u Egiptu  i sve što ste vidjeli; i brzo mi ovamo oca dovedite!" 
\par 14 Potom zagrli brata Benjamina te zaplaka; a plakao je  i Benjamin obisnuvši mu oko vrata. 
\par 15 Izljubi zatim svu svoju  braću, u naručju im se rasplaka. Poslije toga njegova braća zađu  s njim u razgovor. 
\par 16 Glas se pročuje u faraonovu dvoru: "Stigla Josipova braća!"  Bilo je to drago faraonu i njegovim dvoranima. 
\par 17 Onda faraon  reče Josipu: "Kaži svojoj braći neka učine ovo: 'Natovarite svoje  živine i odmah se uputite u zemlju kanaansku. 
\par 18 Uzmite svoga  oca i svoje obitelji i k meni dođite! Ja ću vam dati najbolju  zemlju u Egiptu te ćete uživati od obilja ove zemlje.' 
\par 19 A  naredi i ovo: 'Ovako učinite: Iz zemlje egipatske potjerajte  kola za svoju djecu i svoje žene, uzmite oca i dođite. 
\par 20 Neka  vam se oči ne rastužuju za vašim stvarima, jer sve što je u Egiptu  najbolje bit će vaše.'" 
\par 21 Sinovi Izraelovi tako učine. Po faraonovoj zapovijedi  Josip im dade kola i popudbinu. 
\par 22 Svakom od njih dade nove  haljine, a Benjaminu dade tri stotine srebrnika i petore haljine. 
\par 23 Isto tako pošalje svome ocu: deset magaraca natovarenih najboljim  plodovima egipatskim i deset magarica natovarenih žitom, kruhom  i namirnicama ocu za put. 
\par 24 Isprativši svoju braću na put,  reče im: "Nemojte se putem svađati!" 
\par 25 I tako oni odoše iz Egipta i stigoše u zemlju kanaansku, k svome ocu Jakovu. 
\par 26 Kad mu rekoše: "Josip je živ i čak vlada  nad svom zemljom egipatskom!", njegovo se srce skameni jer im  nije mogao vjerovati. 
\par 27 Ali kad mu ispripovjediše sve što im  je Josip rekao i kad vidje kola što ih je Josip poslao da ga  prevezu, duh njihova oca Jakova oživje. 
\par 28 "Dosta", reče Izrael.  "Sin moj Josip još je živ! Moram poći i vidjeti ga prije nego  umrem." 



\chapter{46}



\par 1 Tako Izrael krene na put sa svim što bijaše njegovo i stigne  u Beer Šebu te prinese žrtvu Bogu svoga oca Izaka. 
\par 2 U noćnom  viđenju zovne Bog Izraela: "Jakove! Jakove!" On odgovori: "Evo  me!" 
\par 3 "Ja sam Bog, Bog tvoga oca. Ne boj se sići u Egipat,  jer ću ondje od tebe proizvesti velik narod. 
\par 4 Ja ću sići u  Egipat s tobom i sam ću te vratiti ovamo; a Josip će ti svojom  rukom oči zaklopiti." 
\par 5 I Jakov krene iz Beer Šebe. Sinovi Izraelovi postave svoga  oca Jakova, svoju djecu i svoje žene u kola što ih je faraon  poslao da ga prevezu. 
\par 6 Uzmu sa sobom svoje blago i dobra što ih bijahu stekli  u zemlji kanaanskoj te stignu Jakov i sve njegovo potomstvo u  Egipat. 
\par 7 Sa sobom je u Egipat poveo svoje sinove i unuke, svoje  kćeri i kćeri svojih sinova, sve svoje potomstvo. 
\par 8 Ovo su imena Izraelaca - Jakova i njegovih potomaka -  koji su stigli u Egipat: Jakovljev prvorođenac Ruben. 
\par 9 Rubenovi  sinovi: Henok, Falu, Hesron i Karmi. 
\par 10 Sinovi Šimunovi: Jemuel, Jamin, Ohad, Jakin, Sohar i Šaul, sin Kanaanke. 
\par 11 Sinovi Levijevi:  Geršon, Kehat i Merari. 
\par 12 Sinovi Judini: Er, Onan, Šela, Peres  i Zerah. Er i Onan umrli su u zemlji kanaanskoj. Peresovi sinovi  bili su Hesron i Hamul. 
\par 13 Sinovi Jisakarovi: Tola, Fuva, Jašub  i Šimron. 
\par 14 Sinovi Zebulunovi: Sered, Elon i Jahleel. 
\par 15 To  su sinovi koje je Lea imala s Jakovom u Padan Aramu i još kćerka  Dina. U svemu je, dakle, imao sinova i kćeri trideset i troje. 
\par 16 Sinovi Gadovi: Sifjon, Hagi, Šuni, Esbon, Eri, Arodi  i Areli. 
\par 17 Sinovi Ašerovi: Jimna, Jišva, Jišvi, Berija i sestra  im Serah. Sinovi Berijini: Heber i Malkiel. 
\par 18 To su bili potomci  Zilpe, koju je Laban darovao svojoj kćeri Lei. Ona je tako rodila  Jakovu šesnaest duša. 
\par 19 Sinovi Jakovljeve žene Rahele: Josip i Benjamin. 
\par 20 Josipu  su se u egipatskoj zemlji rodili Manaše i Efrajim. Rodila mu  ih je kći onskog svećenika Poti-Fere. 
\par 21 Sinovi Benjaminovi:  Bela, Beker, Ašbel, Gera, Naaman, Ehi, Roš, Mupim, Hupim i Ard. 
\par 22 To su bili potomci Rahelini koje je rodila Jakovu - u svemu  njih četrnaest. 
\par 23 Danov je sin Hušim. 
\par 24 Sinovi Naftalijevi: Jahseel,  Guni, Jeser i Šilem. 
\par 25 To su bili potomci Bilhe, koju je Laban  dao svojoj kćeri Raheli. Ona je Jakovu rodila sedam potomaka. 
\par 26 Tako je sve Jakovljeve čeljadi što je od njega poteklo  i u Egipat doselilo - ne uključujući žena Jakovljevih sinova  - u svemu šezdeset i šest osoba. 
\par 27 I k tome dva sina Josipova  što su mu se rodila u Egiptu. Prema tome, sve čeljadi Jakovljeva  doma što se naseli u Egiptu bijaše sedamdeset duša. 
\par 28 Izrael posla Judu naprijed k Josipu da se pred njim pojavi  u Gošenu. Kad stignu u gošenski kraj, 
\par 29 Josip upregne svoja  kola i zaputi se u Gošen - u susret svome ocu Izraelu. Stupivši  preda nj, pade mu oko vrata i dugo je tako plakao. 
\par 30 Onda Izrael  reče Josipu: "Sada, pošto sam rođenim očima vidio da si još živ, mogu umrijeti." 
\par 31 Zatim Josip reče svojoj braći i očevoj obitelji: "Otići  ću i obavijestiti faraona; reći ću mu: 'Moja braća i obitelj  moga oca, koji su bili u zemlji kanaanskoj, došli su k meni. 
\par 32 Oni su ljudi pastiri, uvijek su se bavili stočarstvom; dotjerali  su sa sobom svoja stada i sve što im pripada.' 
\par 33 Tako, kad  vas faraon pozove i zapita: 'Čime se bavite?' 
\par 34 odgovorite:  'Ljudi smo, sluge tvoje, koji se od početka do sad bavimo stočarstvom;  i mi i naši preci', tako da se možete naseliti u gošenskom kraju.  Svi su, naime, pastiri Egipćanima mrski." 



\chapter{47}



\par 1 Ode, dakle, Josip te obavijesti faraona: "Moj otac i moja  braća stigoše sa svojim ovcama i govedima i sa svime što imaju  iz zemlje kanaanske, i eno ih u gošenskom kraju." 
\par 2 I uzevši  petoricu između svoje braće, uvede ih faraonu. 
\par 3 Onda faraon  zapita njegovu braću: "Čime se bavite?" Odgovore faraonu: "Tvoje  su sluge stočari, baš kao što su bili naši preci. 
\par 4 Došli smo  da potražimo kratak boravak u ovoj zemlji", rekoše faraonu, "jer  je nestalo paše za stada tvojih slugu, strašna glad pritište  kanaansku zemlju. Dopusti da se tvoje sluge nastane u gošenskom  kraju." 
\par 5 [5a] Faraon reče Josipu: [6b] "Neka se, dakle, nastane  u gošenskom kraju. A ako znaš da među njima ima prikladnih, postavi  ih za nadglednike moga osobnog blaga." [5b] Tako, kad Jakov i njegovi sinovi stigoše u Egipat i kad  faraon, kralj egipatski, to ču, reče Josipu: "Budući da su tvoj  otac i tvoja braća došli k tebi, 
\par 6 [6a] egipatska ti je zemlja na  raspolaganju: smjesti svoga oca i svoju braću u najboljem kraju." 
\par 7 Josip onda dovede svoga oca Jakova faraonu. Jakov blagoslovi  faraona. 
\par 8 A faraon upita Jakova: "Koliko ti je godina?" 
\par 9 Jakov  odgovori faraonu: "Godina moga lutalačkog življenja ima stotina  i trideset. Malo ih je i nesretne su bile godine moga života;  ne dostižu brojem godine življenja na zemlji mojih otaca." 
\par 10 Poslije  toga Jakov se oprosti s faraonom i ode od njega. 
\par 11 Tako Josip nastani svoga oca i svoju braću davši im u  vlasništvo najljepši kraj egipatske zemlje, u kraju Ramsesovu, kako je faraon naredio. 
\par 12 A Josip opskrbi hranom svoga oca, svoju braću i svu očevu  obitelj sve do najmanjega. 
\par 13 Nigdje nije bilo hrane jer je pritisla strašna glad:  izmuči ona i zemlju egipatsku i zemlju kanaansku. 
\par 14 Josip pobra  sav novac što se nalazio u zemlji egipatskoj i zemlji kanaanskoj  u zamjenu za žito koje se prodavalo i odnese novac u faraonov  dvor. 
\par 15 Kad je nestalo novca u zemlji egipatskoj i zemlji kanaanskoj, svi Egipćani dođu k Josipu te mu reknu: "Daj nam kruha! Zašto  da pomremo pred tvojim očima? Novca više nema." 
\par 16 Josip odgovori:  "Predajte svoju stoku pa ću vam dati žita u zamjenu za stoku  kad je novca nestalo." 
\par 17 Tako su oni dovodili svoju stoku Josipu, a Josip im davaše kruh u zamjenu za konje, za sitnu i krupnu  stoku i za magarad. Tako ih je one godine opskrbljivao kruhom  u zamjenu za sve njihovo blago. 
\par 18 Kad je ona godina prošla, dođu k njemu i druge godine  te mu reknu: "Ne možemo sakriti od svoga gospodara: novca je  nestalo, blaga su već ustupljena gospodaru; drugo ništa ne preostaje  da gospodaru ustupimo nego sebe i svoje oranice. 
\par 19 Zašto da  uništimo na tvoje oči i sebe i svoje zemlje? Uzmi i nas i naše  zemlje u zakup za kruh, i tako ćemo zajedno sa svojom zemljom  postati faraonovi kmetovi; daj sjemena da preživimo: da ne izginemo  i da nam oranice ne postanu pustoš!" 
\par 20 Tako Josip steče faraonu u posjed sve egipatske oranice, jer je svaki Egipćanin, kako ih pritisnu glad, prodao svoje  njive. Tako je zemlja postala faraonovo vlasništvo, 
\par 21 a narod  od jednog kraja Egipta do drugoga njegovim robljem. 
\par 22 Jedino  nije preuzeo svećeničkih imanja, jer je faraon davao svećenicima  određeni dio, i tako su živjeli od prihoda što im ga je faraon  davao. Stoga nisu prodali svojih imanja. 
\par 23 Onda Josip reče svijetu: "Budući da sam danas za faraona  prekupio i vas i vašu zemlju, evo vam sjeme pa zasijte zemlju. 
\par 24 A kad bude pobiranje ljetine, faraonu ćete davati jednu petinu, dok će četiri petine ostajati vama: za zasijavanje polja, za  hranu vama i onima koji su u vašim domovima i za hranu vašoj  djeci." 
\par 25 Oni odgovore: "Život si nam spasio! Mi smo zahvalni  svome gospodaru što možemo biti faraonovi robovi." 
\par 26 Tako Josip  napravi za Egipat zemljišni zakon koji i danas vrijedi: petina  pripada faraonu; jedino svećenička imanja nisu prešla faraonu. 
\par 27 Izraelci se nastaniše u zemlji egipatskoj, u kraju gošenskom;  u njem stekoše vlasništvo; bijahu rodni i broj im se veoma umnoži. 
\par 28 U zemlji egipatskoj poživje Jakov sedamnaest godina. Tako  je duljina Jakovljeva života iznosila sto četrdeset i sedam godina. 
\par 29 A kad se približi vrijeme Izraelu da umre, pozva svoga sina  Josipa te mu reče: "Ako mi želiš ugoditi, stavi svoju ruku pod  moje stegno kao jamstvo svoje odanosti meni: nemoj me sahraniti  u Egiptu! 
\par 30 Kad legnem dolje sa svojim ocima, prenesi me iz  Egipta gore i sahrani me u njihovu grobnicu!" "Učinit ću kako  si rekao", odgovori. 
\par 31 "Zakuni mi se!" - reče. I on mu se zakle.  Tada se Izrael duboko prignu na uzglavlju. 



\chapter{48}



\par 1 Poslije nekog vremena jave Josipu: "Eno ti je otac obolio."  Nato on uzme sa sobom svoja dva sina, Manašea i Efrajima. 
\par 2 Kad  Jakovu rekoše: "Evo ti je došao sin Josip", Izrael skupi svoje  snage i sjede na postelju. 
\par 3 Reče Jakov Josipu: "Bog Svemožni, El Šadaj, objavi mi se u Luzu, u zemlji kanaanskoj; blagoslov  mi dade, 
\par 4 a potom mi reče: 'Učinit ću te rodnim i mnogobrojnim, učinit ću da postaneš skup naroda, a tvome potomstvu poslije  tebe dat ću ovu zemlju u posjed zauvijek.' 
\par 5 Sad, oba tvoja  sina što su ti se rodila u zemlji egipatskoj, prije nego sam  ja stigao k tebi u Egipat, neka budu moji - Efrajim i Manaše  neka budu moji kao i Ruben i Šimun! 
\par 6 A djeca što su ti se rodila  poslije njih neka ostanu tvoja; a u svom nasljedstvu neka se  zovu po imenu svoje braće. 
\par 7 Kad sam se, naime, vraćao iz Padana, na moju žalost, tvoja  majka Rahela umrije na putovanju u kanaansku zemlju, tek u maloj  udaljenosti od Efrate. Sahranio sam je ondje uz put u Efratu, sadašnji Betlehem." 
\par 8 Opazivši Izrael Josipove sinove, zapita: "Tko su ovi?" 
\par 9 Josip odgovori svome ocu: "Sinovi su to moji koje mi je Bog  dao ovdje." "Dovedi mi ih da ih blagoslovim", reče. 
\par 10 Izraelu  oči oslabile od starosti, nije vidio. Zato mu privede sinove, a on ih poljubi i zagrli. 
\par 11 Potom Izrael reče Josipu: "Nisam  očekivao da ću još ikada vidjeti tvoje lice; kad, evo, Bog mi  dade da vidim i tvoje potomke." 
\par 12 Josip ih tada skine s njegovih  koljena i duboko se, sve do zemlje, nakloni. 
\par 13 Nato ih uze Josip obojicu - Efrajima svojom desnicom, Izraelu nalijevo, a Manašea svojom ljevicom, Izraelu nadesno  - te ih k njemu primače. 
\par 14 Ali Izrael ispruži svoju desnicu  i stavi je na Efrajimovu glavu, premda je bio mlađi, a svoju  ljevicu na glavu Manašeovu - tako je držao ruke unakrst - iako  je Manaše bio prvorođenac. 
\par 15 Tako je davao svoj blagoslov Josipu  govoreći:  "Bog, čijim su putovima hodili oci moji Abraham i Izak, Bog, koji mi je pastir bio otkako postah pa do danas, 
\par 16 anđeo koji me od svakog zla izbavljao - djecu ovu neka blagoslovi! Neka se ime moje i mojih pređa Abrahama i Izaka po njima spominje! U mnoštva se mnogobrojna po zemlji razmnožili!" 
\par 17 Kad je Josip vidio da je njegov otac položio desnicu  na Efrajimovu glavu, njegovim se očima to učini krivo; zato posegne  za rukom svoga oca da je pomakne s Efrajimove glave na glavu  Manašeovu. 
\par 18 "Ne tako, oče moj," reče Josip svome ocu, "jer  ovo je prvorođenac; zato stavi desnicu na njegovu glavu!" 
\par 19 Ali  njegov otac to odbije rekavši: "Znam ja, sine moj, znam; i od  njega će postati narod i bit će velik. Ali njegov mlađi brat  bit će veći od njega, a njegovo će potomstvo biti mnoštvo." 
\par 20 Onoga ih, dakle, dana blagoslovi rekavši: "Vama nek' se Izrael blagoslivlja govoreći: Kao što je Efrajimu i Manašeu, nek' i tebi Bog učini!" Tako stavi Efrajima pred Manašea. 
\par 21 Poslije Izrael reče Josipu: "Ja ću, evo, naskoro umrijeti;  no Bog će biti s vama i opet vas dovesti u zemlju vaših otaca. 
\par 22 A tebi ostavljam Šekem, nešto više nego tvojoj braći, što  sam ga svojim mačem i lukom osvojio od Amorejaca." 



\chapter{49}



\par 1 Jakov zatim sazva svoje sinove te reče: "Skupite se da vam  kažem što će vas snaći u kasnije vrijeme: 
\par 2 Okupite se, čujte, sinovi Jakovljevi, čujte oca svoga Izraela! 
\par 3 Ti Rubene, moj prvorođenče, snaga ti si moja, prvenac moje muškosti. Ističeš, se ponosom, snagom se ističeš, 
\par 4 no, poput vode nabujao, nećeš više imati prvenstva, jer na ležaj oca svog se pope, moj tad oskvrnu krevet. 
\par 5 Šimun i Levi braća su prava! Mačevi im oruđe nasilja. 
\par 6 Na njihova vijećanja ja ne silazio, u njihovim zborovima udjela ne imao! U srdžbi su svojoj ljude ubijali; u obijesti bikove sakatili. 
\par 7 Prokleta im srdžba, jer je prežestoka! Prokleta im obijest, jer je preokrutna! Razdijelit ću ih po Jakovu, Izraelom raspršiti. 
\par 8 Judo! Tvoja braća slavit će te; svagda ti je šaka na šiji dušmana, sinci oca tvoga tebi će se klanjat. 
\par 9 Judo, laviću mali! Plijenom si se, sine, udebljao; poput lava, poput lavice legao potrbuške! Tko bi ga dražiti smio? 
\par 10 Od Jude žezlo se kraljevsko, ni palica vladalačka od nogu njegovih udaljiti neće dok ne dođe onaj kome pripada - kome će se narodi pokoriti. 
\par 11 Svog magarca za lozu privezuje, mlado magarice svoje za čokot. U vinu on kupa svoju odjeću svoju halju u krvi od grožđa. 
\par 12 Oči su mu od vina mutne, zubi bjelji od mlijeka. 
\par 13 Zebulun će stanovati uz obalu morsku, luka spasa bit će brodarima, uz bok njegov Sidon će ležati. 
\par 14 Jisakar je koščat magarac polegao među ogradama. 
\par 15 Vidje da je odmor ugodan, a zemlja lijepa, te leđa svoja pod teret podmetnu i na tlaku pristade. 
\par 16 Dan će narod svoj suditi kao svako pleme Izraelovo. 
\par 17 Nek' Dan zmija bude na putu, guja pokraj staze što će konja za zglob ujesti, i njegov konjik nauznak će pasti. 
\par 18 U spas tvoj se, Jahve, uzdam! 
\par 19 Gada će pljačkat razbojnici, pljačkom će im za petama biti. 
\par 20 U Ašera bit će hrane, poslastica za kraljeve. 
\par 21 Naftali je košuta lakonoga koja krasnu lanad mladi. 
\par 22 Josip je stablo plodno, plodno stablo kraj izvora, grane svoje grana preko zida. 
\par 23 Strijelci njega saletjeli, strijeljali ga, opljačkali. 
\par 24 Ali luk mu čvrst ostaje, mišice mu ojačale, rukom Jakog Jakovljeva, imenom Pastira, Stijene Izraela, 
\par 25 Bogom, Ocem tvojim, koji ti pomaže, Svesilnim koji te blagoslivlje blagoslovom ozgo sa nebesa, blagoslovom ozdo iz dubina, blagoslovom iz svih prsa, iz svih utroba! 
\par 26 Blagoslovom klasja i cvjetova, blagoslovom drevnih brda, želja vječnih brežuljaka - nek' se oni spuste na Josipa, između braće posvećenog! 
\par 27 Benjamin je vuk grabežljivi, lovinu on jutrom jede, a navečer plijen dijeli." 
\par 28 Sve su to Izraelova plemena - dvanaest ih na broj - i  to im je otac rekao kad ih je blagoslivljao; svakoga je od njih  blagoslovio njegovim blagoslovom. 
\par 29 Poslije toga im dade ovu naredbu: "Naskoro ću se pridružiti  svojim precima. Sahranite me kraj mojih otaca, 
\par 30 u spilji što  se nalazi na polju Efrona, Hetita, u spilji na polju Makpeli, nasuprot Mamri, u zemlji kanaanskoj. To je ona koju je Abraham  kupio s poljem od Hetita Efrona za mjesto sahranjivanja. 
\par 31 Ondje  je sahranjen Abraham i njegova žena Sara; sahranjeni su ondje  Izak i njegova žena Rebeka; ondje sam ja sahranio Leu. 
\par 32 Polje  i spilja na njemu kupljeni su od Hetita." 
\par 33 Kad je Jakov tako naputio svoje sinove, povuče noge natrag  na postelju te izdahnu - pridruži se svojim precima. 



\chapter{50}

\par 1 Josip se baci na oca, suzama mu oblije lice, izljubi ga. 
\par 2 Poslije  toga Josip naredi liječnicima koji su se nalazili u njegovoj  službi da mu oca balzamiraju, i oni balzamiraše Izraela. 
\par 3 Trebalo  je četrdeset dana: toliko, naime, traje balzamiranje. Sedamdeset  su ga dana Egipćani oplakivali. 
\par 4 A kad je prošlo vrijeme oplakivanja, Josip reče onima u dvoru faraonovu: "Učinite mi milost i prenesite  faraonu ovo: 
\par 5 Moj me otac zakleo govoreći: 'Kad umrem, sahrani  me u grob koji sam sebi pripravio u zemlji kanaanskoj!' Dopusti  mi da odem gore i sahranim oca, a onda ću se vratiti." 
\par 6 Faraon  odgovori: "Otiđi gore i sahrani svoga oca kako si mu se zakleo." 
\par 7 Tako Josip ode da sahrani oca. S njim su pošli i svi faraonovi  službenici - odličnici njegova dvora i svi dostojanstvenici egipatske  zemlje; 
\par 8 sva Josipova obitelj, njegova braća i očeva porodica.  Jedino su u gošenskom kraju ostala njihova djeca, njihove ovce  i goveda. 
\par 9 S njim su išla i kola i konjanici: bila je to vrlo  duga povorka. 
\par 10 Stigavši u Goren Haatad, s onu stranu Jordana, održaše  ondje veliko i svečano naricanje. Josip održa sedmodnevnu žalost  za ocem. 
\par 11 Kad su stanovnici te zemlje, Kanaanci, vidjeli tugovanje  u Goren Haatadu, rekoše: "To ti je svečano naricanje Egipćana!"  Zato nazovu to mjesto Abel-Misrajim. Nalazi se s onu stranu Jordana. 
\par 12 Jakovljevi sinovi učine kako im je naredio otac: 
\par 13 odnesu  ga u zemlju kanaansku te ga sahrane u spilji na polju Makpeli  kod Mamre, polju što ga je Abraham kupio od Hetita Efrona za  sahranjivanje. 
\par 14 Pošto je sahranio svoga oca, Josip se vrati u Egipat  - on, njegova braća i svi koji su s njim išli da mu oca pokopaju. 
\par 15 Kad su Josipova braća vidjela da im je otac umro, rekoše:  "Što ako je Josip na nas ljut i pokuša uzvratiti nam za sve zlo  koje smo mi njemu nanijeli?" 
\par 16 Stoga poruče Josipu ovako: "Pred  svoju smrt tvoj je otac naredio: 
\par 17 'Ovako recite Josipu: Oprosti  braći svojoj zlo i grijeh što su onako okrutno prema tebi postupili.'  Oprosti, dakle, uvredu slugama Boga svoga oca!" Na te riječi  Josip brizne u plač. 
\par 18 Tada sama njegova braća dođu k njemu, bace se preda nj  te mu reknu: "Evo nas k tebi da budemo tvoji robovi!" 
\par 19 Josip  im odvrati: "Ne bojte se! TÓa zar sam ja namjesto Boga! 
\par 20 Osim  toga, iako ste vi namjeravali da meni naudite, Bog je bio ono  okrenuo na dobro: da učini što se danas zbiva - da spasi život  velikom narodu. 
\par 21 Zato se ne bojte! Ja ću se brinuti za vas  i za vašu djecu." Tako ih je smirio ljubeznim riječima. 
\par 22 Josip ostane u Egiptu zajedno s rodom svojim i očevim.  Poživje Josip stotinu i deset godina. 
\par 23 Tako je Josip gledao  Efrajimovu djecu do trećeg koljena; a rađala se djeca i Makiru, Manašeovu sinu, na Josipovim koljenima. 
\par 24 Napokon reče Josip  svojoj braći: "Ja ću, evo, naskoro umrijeti. Ali će se Bog, zacijelo, sjetiti vas i odvesti vas iz ove zemlje u zemlju što ju je pod  zakletvom obećao Abrahamu, Izaku i Jakovu." 
\par 25 Tada Josip zakune  Izraelove sinove: "Bog će se vas doista sjetiti, i tada ponesite  moje kosti odavde!" 
\par 26 Josip umrije kad mu bijaše sto i deset godina; balzamiraše  ga i u Egiptu položiše u lijes. 





\end{document}