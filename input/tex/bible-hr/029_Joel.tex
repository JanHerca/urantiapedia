\begin{document}

\title{Joel}


\chapter{1}

\par 1 Riječ Jahvina koja dođe Joelu, sinu Petuelovu. 
\par 2 Čujte ovo, starci, počujte, svi žitelji zemlje! Je li ovakvo što ikad bilo u vaše dane il' u dane vaših otaca? 
\par 3 Recite ovo svojim sinovima, vaši sinovi svojim sinovima, a njihovi sinovi potonjem koljenu. 
\par 4 Što ostavi šaška, proždrije skakavac, što ostavi skakavac, proždrije gusjenica, što ostavi gusjenica, proždrije ljupilac. 
\par 5 Probudite se, pijanice, i plačite! Sve vinopije, tužite za novim vinom: iz usta vam je oteto. 
\par 6 Jer prekri moju zemlju narod moćan i bezbrojan; zubi su mu kao zubi lavlji, očnjaci mu kao u lavice. 
\par 7 Opustoši mi lozu vinovu i polomi smokve moje; oguli ih i razbaca, grane su im pobijeljele. 
\par 8 Plačite k'o djevica odjevena u kostrijet za zaručnikom svojim. 
\par 9 Nestade prinosnice i ljevanice iz Doma Jahvina. Tuže svećenici, sluge Jahvine. 
\par 10 Opustošeno polje, zemlja poharana. Poharano žito, vino propade, presahnu ulje. 
\par 11 Tugujte, težaci, kukajte, vinogradari, za pšenicom i za ječmom, jer propade žetva poljska. 
\par 12 Loza usahnu, uvenu smokva, mogranj, palma i jabuka: svako se drvo poljsko sasuši. Da, nestade radosti između sinova ljudskih. 
\par 13 Svećenici, opašite kostrijet i tužite! Službenici žrtvenika, naričite! Dođite, prenoćite u kostrijeti, službenici Boga mojeg! Jer iz Doma Boga našeg nesta prinosnice i ljevanice! 
\par 14 Naredite sveti post, proglasite zbor svečani; starješine, saberite sve stanovnike zemlje u kuću Jahve, Boga svojeg. Zavapijte Jahvi: 
\par 15 "Jao dana!" Jer Jahvin dan je blizu i dolazi k'o pohara od Svevišnjeg. 
\par 16 Ne iščeznu li hrana pred našim očima? Nije li nestalo radosti i sreće iz Doma Boga našega? 
\par 17 Istrunu zrnje pod grudama; puste su žitnice, porušene spreme jer žita nesta. 
\par 18 Kako li stoka uzdiše! Krda goveda podivljala lutaju jer im nema paše. Čak i stada ovaca kaznu podnose. 
\par 19 Tebi, Jahve, vapijem: oganj popali pašnjake pustinjske, plamen sažga sva stabla poljska. 
\par 20 Čak i zvijeri čeznu za tobom, jer presušiše potoci, oganj popali pašnjake pustinjske. 


\chapter{2}

\par 1 Trubite u trubu na Sionu! Dižite uzbunu na svetoj mi gori! Neka svi stanovnici zemlje dršću, jer dolazi Jahvin dan. Da, on je blizu. 
\par 2 Dan pun mraka i tmine, dan oblačan i crn. K'o zora po gorama se prostire narod jak i mnogobrojan, kakva ne bje nikad prije, niti će ga igda biti do vremena najdaljih. 
\par 3 Pred njim oganj proždire, za njim plamen guta. Zemlja je k'o vrt rajski pred njim, a za njim pustinja tužna. Ništa mu ne umiče. 
\par 4 Nalik su na konje, jure poput konjanika. 
\par 5 Buče kao bojna kola, po gorskim vrhuncima skaču, pucketaju k'o plamen ognjeni kad strnjiku proždire, kao vojska jaka u bojnome redu. 
\par 6 Pred njima narodi dršću i svako lice problijedi. 
\par 7 Skaču k'o junaci, k'o ratnici se na zidove penju. Svaki ide pravo naprijed, ne odstupa od svog puta. 
\par 8 Ne tiskaju jedan drugog, već svak' ide svojom stazom. Padaju od strijela ne kidajuć' redova. 
\par 9 Na grad navaljuju, na zidine skaču, penju se na kuće i kroz okna ulaze poput lupeža. 
\par 10 Pred njima se zemlja trese, nebo podrhtava, sunce, mjesec mrčaju, zvijezdama se trne sjaj. 
\par 11 I Jahve glas svoj šalje pred vojsku svoju. I odista, tabor mu je silno velik, zapovijedi njegove moćan izvršitelj. Da, velik je Jahvin dan i vrlo strašan. Tko će ga podnijeti? 
\par 12 "Al' i sada - riječ je Jahvina - vratite se k meni svim srcem svojim posteć', plačuć' i kukajuć'." 
\par 13 Razderite srca, a ne halje svoje! Vratite se Jahvi, Bogu svome, jer on je nježnost sama i milosrđe, spor na ljutnju, a bogat dobrotom, on se nad zlom ražali. 
\par 14 Tko zna neće li se opet ražaliti, neće li blagoslov ostaviti za sobom! Prinose i ljevanice Jahvi, Bogu našemu! 
\par 15 Trubite u trubu na Sionu! Sveti post naredite, oglasite zbor svečani, 
\par 16 narod saberite, posvetite zbor. Saberite starce, sakupite djecu, čak i nejač na prsima. Neka ženik iziđe iz svadbene sobe a nevjesta iz odaje. 
\par 17 Između trijema i žrtvenika neka tuže svećenici, sluge Jahvine. Neka mole: "Smiluj se, Jahve, svojem narodu! Ne prepusti baštine svoje sramoti, poruzi naroda. Zašto da se kaže među narodima: Gdje im je Bog?" 
\par 18 Tad Jahve, ljubomoran na zemlju svoju, smilova se svom narodu. 
\par 19 Odgovori Jahve svojem narodu: "Šaljem vam, evo, žita, vina i ulja da se njime nasitite. Nikad više neću pustiti da budete na sramotu narodima. 
\par 20 Protjerat ću Sjevernjaka od vas daleko, odagnat ga u zemlju suhu i pustu, prethodnicu u Istočno more, zalaznicu u Zapadno more. Dići će se njegov smrad, dizat će se trulež njegova." (Jer učini stvari velike.) 
\par 21 O zemljo, ne boj se! Budi sretna, raduj se, jer Jahve učini djela velika. 
\par 22 Zvijeri poljske, ne bojte se; pašnjaci u pustinji opet se zelene, voćke daju rod, smokva i loza nose izobila. 
\par 23 Sinovi sionski, radujte se, u Jahvi se veselite, svojem Bogu; jer vam daje kišu jesensku u pravoj mjeri, izli na vas kišu, jesensku i proljetnu kišu kao nekoć. 
\par 24 Gumna će biti puna žita, kace će se prelijevati od vina i ulja. 
\par 25 "Nadoknadit ću vam godine koje izjedoše skakavac, gusjenica, ljupilac i šaška, silna vojska moja što je poslah na vas." 
\par 26 Jest ćete izobila, jest ćete do sita, slavit ćete ime Jahve, svojeg Boga, koji je s vama čudesno postupao. ("Moj se narod neće postidjeti nikad više.") 
\par 27 "Znat ćete da sam posred Izraela, da sam ja Jahve, vaš Bog, i nitko više. Moj se narod neće postidjeti nikad više." 
\par 28 (3:1) "Poslije ovoga izlit ću Duha svoga na svako tijelo, i proricat će vaši sinovi i kćeri, vaši će starci sanjati sne, a vaši mladići gledati viđenja. 
\par 29 (3:2) Čak ću i na sluge i sluškinje izliti Duha svojeg u dane one. 
\par 30 (3:3) Pokazat ću znamenja na nebu i zemlji, krv i oganj i stupove dima." 
\par 31 (3:4) Sunce će se prometnut' u tminu a mjesec u krv, prije nego svane Jahvin dan, velik i strašan. 
\par 32 (3:5) Svi što prizivaju ime Jahvino spašeni će biti, jer će na brdu Sionu i u Jeruzalemu biti spasenje, kao što Jahve reče, a među preživjelima oni koje Jahve pozove. 



\chapter{3}

\par 1 (4:1) "Jer, gle, u one dane i u ono vrijeme, kad okrenem udes Judeji i Jeruzalemu, 
\par 2 (4:2) sakupit ću sve narode i povesti ih u dolinu Jošafat. Ondje ću im suditi zbog Izraela, naroda mog i moje baštine, koju rastjeraše među narode i razdijeliše moju zemlju među se. 
\par 3 (4:3) Baciše ždrijeb za moj narod; davali su dječake za bludnice, djevojke prodavali za vino i pili." 
\par 4 (4:4) "I vi, Tire i Sidone, što hoćete od mene? I vi, filistejski kraljevi? Želite li mi se osvetiti? Ako se budete svetili meni, osveta će brzo na vaše glave. 
\par 5 (4:5) Na vas što mi oteste srebro i zlato, što odnesoste bogate mi riznice u svoje hramove, 
\par 6 (4:6) na vas koji prodavaste Grcima sinove Jude i Jeruzalema da biste ih otjerali od domovine njihove. 
\par 7 (4:7) Gle, ja ih kanim dići s mjesta gdje god ih prodaste, i učinit ću da vam zločin vaš padne na glave. 
\par 8 (4:8) Prodat ću vaše sinove i kćeri sinovima Judinim, a oni će ih prodat' Sabejcima, daleku narodu. Jer Jahve reče!" 
\par 9 (4:9) Razglasite ovo među narodima! Posvetite se za rat! Dižite junake! Naprijed, navalite, svi ratnici! 
\par 10 (4:10) Prekujte raonike u mačeve, kosire u koplja, nek' slabić kaže: "Junak sam!" 
\par 11 (4:11) Pohitajte i dođite, svi okolni narodi, i ondje se saberite! (Jahve, onamo pošalji svoje junake!) 
\par 12 (4:12) "Budite se, narodi, stupajte u Dolinu Jošafat, jer ću ondje sjesti da sudim svim okolnim narodima. 
\par 13 (4:13) Hvatajte se srpa: ljetina je zrela. Ustanite, siđite: tijesak je pun, prelijevaju se kace, jer je velika zloća njihova." 
\par 14 (4:14) Mnoštvo, mnoštvo u Dolini Odluke! Da, blizu je dan Jahvin u Dolini Odluke! 
\par 15 (4:15) Sunce i mjesec pomrčaše, zvijezde potamnješe. 
\par 16 (4:16) Jahve grmi sa Siona, glas diže iz Jeruzalema; nebo se i zemlja tresu. Ali je Jahve utočište svome narodu i zaštita sinovima Izraela. 
\par 17 (4:17) "Znat ćete tada da sam ja Jahve vaš Bog što stoluje na Sionu, svetoj gori svojoj. Jeruzalem će biti svetište, tuđinac više neće kroza nj proći." 
\par 18 (4:18) Kad dođe taj dan, kapat će gore moštom, iz bregova će brizgati mlijeko, kroza sva korita riječna u Judeji voda će proteći. Vrelo će šiknuti iz kuće Jahvine da natopi Dolinu sitimsku. 
\par 19 (4:19) Egipat će opustjeti, Edom će postati beživotna pustinja zbog nasilja učinjena sinovima Judinim, jer proliše krv nevinu u njihovoj zemlji. 
\par 20 (4:20) Judeja će dovijek biti naseljena i Jeruzalem u sva koljena. 
\par 21 (4:21) "Osvetit ću krv njihovu za koju se nisam još osvetio." Jahve će dići Dom svoj na Sionu. 





\end{document}