\begin{document}

\title{III João}


\chapter{1}

\par 1 O presbítero ao amado Gaio, a quem eu amo na verdade.
\par 2 Amado, acima de tudo, faço votos por tua prosperidade e saúde, assim como é próspera a tua alma.
\par 3 Pois fiquei sobremodo alegre pela vinda de irmãos e pelo seu testemunho da tua verdade, como tu andas na verdade.
\par 4 Não tenho maior alegria do que esta, a de ouvir que meus filhos andam na verdade.
\par 5 Amado, procedes fielmente naquilo que praticas para com os irmãos, e isto fazes mesmo quando são estrangeiros,
\par 6 os quais, perante a igreja, deram testemunho do teu amor. Bem farás encaminhando-os em sua jornada por modo digno de Deus;
\par 7 pois por causa do Nome foi que saíram, nada recebendo dos gentios.
\par 8 Portanto, devemos acolher esses irmãos, para nos tornarmos cooperadores da verdade.
\par 9 Escrevi alguma coisa à igreja; mas Diótrefes, que gosta de exercer a primazia entre eles, não nos dá acolhida.
\par 10 Por isso, se eu for aí, far-lhe-ei lembradas as obras que ele pratica, proferindo contra nós palavras maliciosas. E, não satisfeito com estas coisas, nem ele mesmo acolhe os irmãos, como impede os que querem recebê-los e os expulsa da igreja.
\par 11 Amado, não imites o que é mau, senão o que é bom. Aquele que pratica o bem procede de Deus; aquele que pratica o mal jamais viu a Deus.
\par 12 Quanto a Demétrio, todos lhe dão testemunho, até a própria verdade, e nós também damos testemunho; e sabes que o nosso testemunho é verdadeiro.
\par 13 Muitas coisas tinha que te escrever; todavia, não quis fazê-lo com tinta e pena,
\par 14 pois, em breve, espero ver-te. Então, conversaremos de viva voz.


\end{document}