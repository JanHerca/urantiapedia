\begin{document}

\title{Tito}


\chapter{1}

\par 1 Paulo, servo de Deus e apóstolo de Jesus Cristo, para promover a fé que é dos eleitos de Deus e o pleno conhecimento da verdade segundo a piedade,
\par 2 na esperança da vida eterna que o Deus que não pode mentir prometeu antes dos tempos eternos
\par 3 e, em tempos devidos, manifestou a sua palavra mediante a pregação que me foi confiada por mandato de Deus, nosso Salvador,
\par 4 a Tito, verdadeiro filho, segundo a fé comum, graça e paz, da parte de Deus Pai e de Cristo Jesus, nosso Salvador.
\par 5 Por esta causa, te deixei em Creta, para que pusesses em ordem as coisas restantes, bem como, em cada cidade, constituísses presbíteros, conforme te prescrevi:
\par 6 alguém que seja irrepreensível, marido de uma só mulher, que tenha filhos crentes que não são acusados de dissolução, nem são insubordinados.
\par 7 Porque é indispensável que o bispo seja irrepreensível como despenseiro de Deus, não arrogante, não irascível, não dado ao vinho, nem violento, nem cobiçoso de torpe ganância;
\par 8 antes, hospitaleiro, amigo do bem, sóbrio, justo, piedoso, que tenha domínio de si,
\par 9 apegado à palavra fiel, que é segundo a doutrina, de modo que tenha poder tanto para exortar pelo reto ensino como para convencer os que o contradizem.
\par 10 Porque existem muitos insubordinados, palradores frívolos e enganadores, especialmente os da circuncisão.
\par 11 É preciso fazê-los calar, porque andam pervertendo casas inteiras, ensinando o que não devem, por torpe ganância.
\par 12 Foi mesmo, dentre eles, um seu profeta, que disse: Cretenses, sempre mentirosos, feras terríveis, ventres preguiçosos.
\par 13 Tal testemunho é exato. Portanto, repreende-os severamente, para que sejam sadios na fé
\par 14 e não se ocupem com fábulas judaicas, nem com mandamentos de homens desviados da verdade.
\par 15 Todas as coisas são puras para os puros; todavia, para os impuros e descrentes, nada é puro. Porque tanto a mente como a consciência deles estão corrompidas.
\par 16 No tocante a Deus, professam conhecê-lo; entretanto, o negam por suas obras; é por isso que são abomináveis, desobedientes e reprovados para toda boa obra.

\chapter{2}

\par 1 Tu, porém, fala o que convém à sã doutrina.
\par 2 Quanto aos homens idosos, que sejam temperantes, respeitáveis, sensatos, sadios na fé, no amor e na constância.
\par 3 Quanto às mulheres idosas, semelhantemente, que sejam sérias em seu proceder, não caluniadoras, não escravizadas a muito vinho; sejam mestras do bem,
\par 4 a fim de instruírem as jovens recém-casadas a amarem ao marido e a seus filhos,
\par 5 a serem sensatas, honestas, boas donas de casa, bondosas, sujeitas ao marido, para que a palavra de Deus não seja difamada.
\par 6 Quanto aos moços, de igual modo, exorta-os para que, em todas as coisas, sejam criteriosos.
\par 7 Torna-te, pessoalmente, padrão de boas obras. No ensino, mostra integridade, reverência,
\par 8 linguagem sadia e irrepreensível, para que o adversário seja envergonhado, não tendo indignidade nenhuma que dizer a nosso respeito.
\par 9 Quanto aos servos, que sejam, em tudo, obedientes ao seu senhor, dando-lhe motivo de satisfação; não sejam respondões,
\par 10 não furtem; pelo contrário, dêem prova de toda a fidelidade, a fim de ornarem, em todas as coisas, a doutrina de Deus, nosso Salvador.
\par 11 Porquanto a graça de Deus se manifestou salvadora a todos os homens,
\par 12 educando-nos para que, renegadas a impiedade e as paixões mundanas, vivamos, no presente século, sensata, justa e piedosamente,
\par 13 aguardando a bendita esperança e a manifestação da glória do nosso grande Deus e Salvador Cristo Jesus,
\par 14 o qual a si mesmo se deu por nós, a fim de remir-nos de toda iniqüidade e purificar, para si mesmo, um povo exclusivamente seu, zeloso de boas obras.
\par 15 Dize estas coisas; exorta e repreende também com toda a autoridade. Ninguém te despreze.

\chapter{3}

\par 1 Lembra-lhes que se sujeitem aos que governam, às autoridades; sejam obedientes, estejam prontos para toda boa obra,
\par 2 não difamem a ninguém; nem sejam altercadores, mas cordatos, dando provas de toda cortesia, para com todos os homens.
\par 3 Pois nós também, outrora, éramos néscios, desobedientes, desgarrados, escravos de toda sorte de paixões e prazeres, vivendo em malícia e inveja, odiosos e odiando-nos uns aos outros.
\par 4 Quando, porém, se manifestou a benignidade de Deus, nosso Salvador, e o seu amor para com todos,
\par 5 não por obras de justiça praticadas por nós, mas segundo sua misericórdia, ele nos salvou mediante o lavar regenerador e renovador do Espírito Santo,
\par 6 que ele derramou sobre nós ricamente, por meio de Jesus Cristo, nosso Salvador,
\par 7 a fim de que, justificados por graça, nos tornemos seus herdeiros, segundo a esperança da vida eterna.
\par 8 Fiel é esta palavra, e quero que, no tocante a estas coisas, faças afirmação, confiadamente, para que os que têm crido em Deus sejam solícitos na prática de boas obras. Estas coisas são excelentes e proveitosas aos homens.
\par 9 Evita discussões insensatas, genealogias, contendas e debates sobre a lei; porque não têm utilidade e são fúteis.
\par 10 Evita o homem faccioso, depois de admoestá-lo primeira e segunda vez,
\par 11 pois sabes que tal pessoa está pervertida, e vive pecando, e por si mesma está condenada.
\par 12 Quando te enviar Ártemas ou Tíquico, apressa-te a vir até Nicópolis ao meu encontro. Estou resolvido a passar o inverno ali.
\par 13 Encaminha com diligência Zenas, o intérprete da lei, e Apolo, a fim de que não lhes falte coisa alguma.
\par 14 Agora, quanto aos nossos, que aprendam também a distinguir-se nas boas obras a favor dos necessitados, para não se tornarem infrutíferos.
\par 15 Todos os que se acham comigo te saúdam; saúda quantos nos amam na fé. A graça seja com todos vós.


\end{document}