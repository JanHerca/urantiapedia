\begin{document}

\title{Romanos}


\chapter{1}

\par 1 Paulo, servo de Jesus Cristo, chamado para ser apóstolo, separado para o evangelho de Deus,
\par 2 o qual foi por Deus, outrora, prometido por intermédio dos seus profetas nas Sagradas Escrituras,
\par 3 com respeito a seu Filho, o qual, segundo a carne, veio da descendência de Davi
\par 4 e foi designado Filho de Deus com poder, segundo o espírito de santidade pela ressurreição dos mortos, a saber, Jesus Cristo, nosso Senhor,
\par 5 por intermédio de quem viemos a receber graça e apostolado por amor do seu nome, para a obediência por fé, entre todos os gentios,
\par 6 de cujo número sois também vós, chamados para serdes de Jesus Cristo.
\par 7 A todos os amados de Deus, que estais em Roma, chamados para serdes santos, graça a vós outros e paz, da parte de Deus, nosso Pai, e do Senhor Jesus Cristo.
\par 8 Primeiramente, dou graças a meu Deus, mediante Jesus Cristo, no tocante a todos vós, porque, em todo o mundo, é proclamada a vossa fé.
\par 9 Porque Deus, a quem sirvo em meu espírito, no evangelho de seu Filho, é minha testemunha de como incessantemente faço menção de vós
\par 10 em todas as minhas orações, suplicando que, nalgum tempo, pela vontade de Deus, se me ofereça boa ocasião de visitar-vos.
\par 11 Porque muito desejo ver-vos, a fim de repartir convosco algum dom espiritual, para que sejais confirmados,
\par 12 isto é, para que, em vossa companhia, reciprocamente nos confortemos por intermédio da fé mútua, vossa e minha.
\par 13 Porque não quero, irmãos, que ignoreis que, muitas vezes, me propus ir ter convosco (no que tenho sido, até agora, impedido), para conseguir igualmente entre vós algum fruto, como também entre os outros gentios.
\par 14 Pois sou devedor tanto a gregos como a bárbaros, tanto a sábios como a ignorantes;
\par 15 por isso, quanto está em mim, estou pronto a anunciar o evangelho também a vós outros, em Roma.
\par 16 Pois não me envergonho do evangelho, porque é o poder de Deus para a salvação de todo aquele que crê, primeiro do judeu e também do grego;
\par 17 visto que a justiça de Deus se revela no evangelho, de fé em fé, como está escrito: O justo viverá por fé.
\par 18 A ira de Deus se revela do céu contra toda impiedade e perversão dos homens que detêm a verdade pela injustiça;
\par 19 porquanto o que de Deus se pode conhecer é manifesto entre eles, porque Deus lhes manifestou.
\par 20 Porque os atributos invisíveis de Deus, assim o seu eterno poder, como também a sua própria divindade, claramente se reconhecem, desde o princípio do mundo, sendo percebidos por meio das coisas que foram criadas. Tais homens são, por isso, indesculpáveis;
\par 21 porquanto, tendo conhecimento de Deus, não o glorificaram como Deus, nem lhe deram graças; antes, se tornaram nulos em seus próprios raciocínios, obscurecendo-se-lhes o coração insensato.
\par 22 Inculcando-se por sábios, tornaram-se loucos
\par 23 e mudaram a glória do Deus incorruptível em semelhança da imagem de homem corruptível, bem como de aves, quadrúpedes e répteis.
\par 24 Por isso, Deus entregou tais homens à imundícia, pelas concupiscências de seu próprio coração, para desonrarem o seu corpo entre si;
\par 25 pois eles mudaram a verdade de Deus em mentira, adorando e servindo a criatura em lugar do Criador, o qual é bendito eternamente. Amém!
\par 26 Por causa disso, os entregou Deus a paixões infames; porque até as mulheres mudaram o modo natural de suas relações íntimas por outro, contrário à natureza;
\par 27 semelhantemente, os homens também, deixando o contacto natural da mulher, se inflamaram mutuamente em sua sensualidade, cometendo torpeza, homens com homens, e recebendo, em si mesmos, a merecida punição do seu erro.
\par 28 E, por haverem desprezado o conhecimento de Deus, o próprio Deus os entregou a uma disposição mental reprovável, para praticarem coisas inconvenientes,
\par 29 cheios de toda injustiça, malícia, avareza e maldade; possuídos de inveja, homicídio, contenda, dolo e malignidade; sendo difamadores,
\par 30 caluniadores, aborrecidos de Deus, insolentes, soberbos, presunçosos, inventores de males, desobedientes aos pais,
\par 31 insensatos, pérfidos, sem afeição natural e sem misericórdia.
\par 32 Ora, conhecendo eles a sentença de Deus, de que são passíveis de morte os que tais coisas praticam, não somente as fazem, mas também aprovam os que assim procedem.

\chapter{2}

\par 1 Portanto, és indesculpável, ó homem, quando julgas, quem quer que sejas; porque, no que julgas a outro, a ti mesmo te condenas; pois praticas as próprias coisas que condenas.
\par 2 Bem sabemos que o juízo de Deus é segundo a verdade contra os que praticam tais coisas.
\par 3 Tu, ó homem, que condenas os que praticam tais coisas e fazes as mesmas, pensas que te livrarás do juízo de Deus?
\par 4 Ou desprezas a riqueza da sua bondade, e tolerância, e longanimidade, ignorando que a bondade de Deus é que te conduz ao arrependimento?
\par 5 Mas, segundo a tua dureza e coração impenitente, acumulas contra ti mesmo ira para o dia da ira e da revelação do justo juízo de Deus,
\par 6 que retribuirá a cada um segundo o seu procedimento:
\par 7 a vida eterna aos que, perseverando em fazer o bem, procuram glória, honra e incorruptibilidade;
\par 8 mas ira e indignação aos facciosos, que desobedecem à verdade e obedecem à injustiça.
\par 9 Tribulação e angústia virão sobre a alma de qualquer homem que faz o mal, ao judeu primeiro e também ao grego;
\par 10 glória, porém, e honra, e paz a todo aquele que pratica o bem, ao judeu primeiro e também ao grego.
\par 11 Porque para com Deus não há acepção de pessoas.
\par 12 Assim, pois, todos os que pecaram sem lei também sem lei perecerão; e todos os que com lei pecaram mediante lei serão julgados.
\par 13 Porque os simples ouvidores da lei não são justos diante de Deus, mas os que praticam a lei hão de ser justificados.
\par 14 Quando, pois, os gentios, que não têm lei, procedem, por natureza, de conformidade com a lei, não tendo lei, servem eles de lei para si mesmos.
\par 15 Estes mostram a norma da lei gravada no seu coração, testemunhando-lhes também a consciência e os seus pensamentos, mutuamente acusando-se ou defendendo-se,
\par 16 no dia em que Deus, por meio de Cristo Jesus, julgar os segredos dos homens, de conformidade com o meu evangelho.
\par 17 Se, porém, tu, que tens por sobrenome judeu, e repousas na lei, e te glorias em Deus;
\par 18 que conheces a sua vontade e aprovas as coisas excelentes, sendo instruído na lei;
\par 19 que estás persuadido de que és guia dos cegos, luz dos que se encontram em trevas,
\par 20 instrutor de ignorantes, mestre de crianças, tendo na lei a forma da sabedoria e da verdade;
\par 21 tu, pois, que ensinas a outrem, não te ensinas a ti mesmo? Tu, que pregas que não se deve furtar, furtas?
\par 22 Dizes que não se deve cometer adultério e o cometes? Abominas os ídolos e lhes roubas os templos?
\par 23 Tu, que te glorias na lei, desonras a Deus pela transgressão da lei?
\par 24 Pois, como está escrito, o nome de Deus é blasfemado entre os gentios por vossa causa.
\par 25 Porque a circuncisão tem valor se praticares a lei; se és, porém, transgressor da lei, a tua circuncisão já se tornou incircuncisão.
\par 26 Se, pois, a incircuncisão observa os preceitos da lei, não será ela, porventura, considerada como circuncisão?
\par 27 E, se aquele que é incircunciso por natureza cumpre a lei, certamente, ele te julgará a ti, que, não obstante a letra e a circuncisão, és transgressor da lei.
\par 28 Porque não é judeu quem o é apenas exteriormente, nem é circuncisão a que é somente na carne.
\par 29 Porém judeu é aquele que o é interiormente, e circuncisão, a que é do coração, no espírito, não segundo a letra, e cujo louvor não procede dos homens, mas de Deus.

\chapter{3}

\par 1 Qual é, pois, a vantagem do judeu? Ou qual a utilidade da circuncisão?
\par 2 Muita, sob todos os aspectos. Principalmente porque aos judeus foram confiados os oráculos de Deus.
\par 3 E daí? Se alguns não creram, a incredulidade deles virá desfazer a fidelidade de Deus?
\par 4 De maneira nenhuma! Seja Deus verdadeiro, e mentiroso, todo homem, segundo está escrito: Para seres justificado nas tuas palavras e venhas a vencer quando fores julgado.
\par 5 Mas, se a nossa injustiça traz a lume a justiça de Deus, que diremos? Porventura, será Deus injusto por aplicar a sua ira? (Falo como homem.)
\par 6 Certo que não. Do contrário, como julgará Deus o mundo?
\par 7 E, se por causa da minha mentira, fica em relevo a verdade de Deus para a sua glória, por que sou eu ainda condenado como pecador?
\par 8 E por que não dizemos, como alguns, caluniosamente, afirmam que o fazemos: Pratiquemos males para que venham bens? A condenação destes é justa.
\par 9 Que se conclui? Temos nós qualquer vantagem? Não, de forma nenhuma; pois já temos demonstrado que todos, tanto judeus como gregos, estão debaixo do pecado;
\par 10 como está escrito: Não há justo, nem um sequer,
\par 11 não há quem entenda, não há quem busque a Deus;
\par 12 todos se extraviaram, à uma se fizeram inúteis; não há quem faça o bem, não há nem um sequer.
\par 13 A garganta deles é sepulcro aberto; com a língua, urdem engano, veneno de víbora está nos seus lábios,
\par 14 a boca, eles a têm cheia de maldição e de amargura;
\par 15 são os seus pés velozes para derramar sangue,
\par 16 nos seus caminhos, há destruição e miséria;
\par 17 desconheceram o caminho da paz.
\par 18 Não há temor de Deus diante de seus olhos.
\par 19 Ora, sabemos que tudo o que a lei diz, aos que vivem na lei o diz para que se cale toda boca, e todo o mundo seja culpável perante Deus,
\par 20 visto que ninguém será justificado diante dele por obras da lei, em razão de que pela lei vem o pleno conhecimento do pecado.
\par 21 Mas agora, sem lei, se manifestou a justiça de Deus testemunhada pela lei e pelos profetas;
\par 22 justiça de Deus mediante a fé em Jesus Cristo, para todos [e sobre todos] os que crêem; porque não há distinção,
\par 23 pois todos pecaram e carecem da glória de Deus,
\par 24 sendo justificados gratuitamente, por sua graça, mediante a redenção que há em Cristo Jesus,
\par 25 a quem Deus propôs, no seu sangue, como propiciação, mediante a fé, para manifestar a sua justiça, por ter Deus, na sua tolerância, deixado impunes os pecados anteriormente cometidos;
\par 26 tendo em vista a manifestação da sua justiça no tempo presente, para ele mesmo ser justo e o justificador daquele que tem fé em Jesus.
\par 27 Onde, pois, a jactância? Foi de todo excluída. Por que lei? Das obras? Não; pelo contrário, pela lei da fé.
\par 28 Concluímos, pois, que o homem é justificado pela fé, independentemente das obras da lei.
\par 29 É, porventura, Deus somente dos judeus? Não o é também dos gentios? Sim, também dos gentios,
\par 30 visto que Deus é um só, o qual justificará, por fé, o circunciso e, mediante a fé, o incircunciso.
\par 31 Anulamos, pois, a lei pela fé? Não, de maneira nenhuma! Antes, confirmamos a lei.

\chapter{4}

\par 1 Que, pois, diremos ter alcançado Abraão, nosso pai segundo a carne?
\par 2 Porque, se Abraão foi justificado por obras, tem de que se gloriar, porém não diante de Deus.
\par 3 Pois que diz a Escritura? Abraão creu em Deus, e isso lhe foi imputado para justiça.
\par 4 Ora, ao que trabalha, o salário não é considerado como favor, e sim como dívida.
\par 5 Mas, ao que não trabalha, porém crê naquele que justifica o ímpio, a sua fé lhe é atribuída como justiça.
\par 6 E é assim também que Davi declara ser bem-aventurado o homem a quem Deus atribui justiça, independentemente de obras:
\par 7 Bem-aventurados aqueles cujas iniqüidades são perdoadas, e cujos pecados são cobertos;
\par 8 bem-aventurado o homem a quem o Senhor jamais imputará pecado.
\par 9 Vem, pois, esta bem-aventurança exclusivamente sobre os circuncisos ou também sobre os incircuncisos? Visto que dizemos: a fé foi imputada a Abraão para justiça.
\par 10 Como, pois, lhe foi atribuída? Estando ele já circuncidado ou ainda incircunciso? Não no regime da circuncisão, e sim quando incircunciso.
\par 11 E recebeu o sinal da circuncisão como selo da justiça da fé que teve quando ainda incircunciso; para vir a ser o pai de todos os que crêem, embora não circuncidados, a fim de que lhes fosse imputada a justiça,
\par 12 e pai da circuncisão, isto é, daqueles que não são apenas circuncisos, mas também andam nas pisadas da fé que teve Abraão, nosso pai, antes de ser circuncidado.
\par 13 Não foi por intermédio da lei que a Abraão ou a sua descendência coube a promessa de ser herdeiro do mundo, e sim mediante a justiça da fé.
\par 14 Pois, se os da lei é que são os herdeiros, anula-se a fé e cancela-se a promessa,
\par 15 porque a lei suscita a ira; mas onde não há lei, também não há transgressão.
\par 16 Essa é a razão por que provém da fé, para que seja segundo a graça, a fim de que seja firme a promessa para toda a descendência, não somente ao que está no regime da lei, mas também ao que é da fé que teve Abraão (porque Abraão é pai de todos nós,
\par 17 como está escrito: Por pai de muitas nações te constituí.), perante aquele no qual creu, o Deus que vivifica os mortos e chama à existência as coisas que não existem.
\par 18 Abraão, esperando contra a esperança, creu, para vir a ser pai de muitas nações, segundo lhe fora dito: Assim será a tua descendência.
\par 19 E, sem enfraquecer na fé, embora levasse em conta o seu próprio corpo amortecido, sendo já de cem anos, e a idade avançada de Sara,
\par 20 não duvidou, por incredulidade, da promessa de Deus; mas, pela fé, se fortaleceu, dando glória a Deus,
\par 21 estando plenamente convicto de que ele era poderoso para cumprir o que prometera.
\par 22 Pelo que isso lhe foi também imputado para justiça.
\par 23 E não somente por causa dele está escrito que lhe foi levado em conta,
\par 24 mas também por nossa causa, posto que a nós igualmente nos será imputado, a saber, a nós que cremos naquele que ressuscitou dentre os mortos a Jesus, nosso Senhor,
\par 25 o qual foi entregue por causa das nossas transgressões e ressuscitou por causa da nossa justificação.

\chapter{5}

\par 1 Justificados, pois, mediante a fé, temos paz com Deus por meio de nosso Senhor Jesus Cristo;
\par 2 por intermédio de quem obtivemos igualmente acesso, pela fé, a esta graça na qual estamos firmes; e gloriamo-nos na esperança da glória de Deus.
\par 3 E não somente isto, mas também nos gloriamos nas próprias tribulações, sabendo que a tribulação produz perseverança;
\par 4 e a perseverança, experiência; e a experiência, esperança.
\par 5 Ora, a esperança não confunde, porque o amor de Deus é derramado em nosso coração pelo Espírito Santo, que nos foi outorgado.
\par 6 Porque Cristo, quando nós ainda éramos fracos, morreu a seu tempo pelos ímpios.
\par 7 Dificilmente, alguém morreria por um justo; pois poderá ser que pelo bom alguém se anime a morrer.
\par 8 Mas Deus prova o seu próprio amor para conosco pelo fato de ter Cristo morrido por nós, sendo nós ainda pecadores.
\par 9 Logo, muito mais agora, sendo justificados pelo seu sangue, seremos por ele salvos da ira.
\par 10 Porque, se nós, quando inimigos, fomos reconciliados com Deus mediante a morte do seu Filho, muito mais, estando já reconciliados, seremos salvos pela sua vida;
\par 11 e não apenas isto, mas também nos gloriamos em Deus por nosso Senhor Jesus Cristo, por intermédio de quem recebemos, agora, a reconciliação.
\par 12 Portanto, assim como por um só homem entrou o pecado no mundo, e pelo pecado, a morte, assim também a morte passou a todos os homens, porque todos pecaram.
\par 13 Porque até ao regime da lei havia pecado no mundo, mas o pecado não é levado em conta quando não há lei.
\par 14 Entretanto, reinou a morte desde Adão até Moisés, mesmo sobre aqueles que não pecaram à semelhança da transgressão de Adão, o qual prefigurava aquele que havia de vir.
\par 15 Todavia, não é assim o dom gratuito como a ofensa; porque, se, pela ofensa de um só, morreram muitos, muito mais a graça de Deus e o dom pela graça de um só homem, Jesus Cristo, foram abundantes sobre muitos.
\par 16 O dom, entretanto, não é como no caso em que somente um pecou; porque o julgamento derivou de uma só ofensa, para a condenação; mas a graça transcorre de muitas ofensas, para a justificação.
\par 17 Se, pela ofensa de um e por meio de um só, reinou a morte, muito mais os que recebem a abundância da graça e o dom da justiça reinarão em vida por meio de um só, a saber, Jesus Cristo.
\par 18 Pois assim como, por uma só ofensa, veio o juízo sobre todos os homens para condenação, assim também, por um só ato de justiça, veio a graça sobre todos os homens para a justificação que dá vida.
\par 19 Porque, como, pela desobediência de um só homem, muitos se tornaram pecadores, assim também, por meio da obediência de um só, muitos se tornarão justos.
\par 20 Sobreveio a lei para que avultasse a ofensa; mas onde abundou o pecado, superabundou a graça,
\par 21 a fim de que, como o pecado reinou pela morte, assim também reinasse a graça pela justiça para a vida eterna, mediante Jesus Cristo, nosso Senhor.

\chapter{6}

\par 1 Que diremos, pois? Permaneceremos no pecado, para que seja a graça mais abundante?
\par 2 De modo nenhum! Como viveremos ainda no pecado, nós os que para ele morremos?
\par 3 Ou, porventura, ignorais que todos nós que fomos batizados em Cristo Jesus fomos batizados na sua morte?
\par 4 Fomos, pois, sepultados com ele na morte pelo batismo; para que, como Cristo foi ressuscitado dentre os mortos pela glória do Pai, assim também andemos nós em novidade de vida.
\par 5 Porque, se fomos unidos com ele na semelhança da sua morte, certamente, o seremos também na semelhança da sua ressurreição,
\par 6 sabendo isto: que foi crucificado com ele o nosso velho homem, para que o corpo do pecado seja destruído, e não sirvamos o pecado como escravos;
\par 7 porquanto quem morreu está justificado do pecado.
\par 8 Ora, se já morremos com Cristo, cremos que também com ele viveremos,
\par 9 sabedores de que, havendo Cristo ressuscitado dentre os mortos, já não morre; a morte já não tem domínio sobre ele.
\par 10 Pois, quanto a ter morrido, de uma vez para sempre morreu para o pecado; mas, quanto a viver, vive para Deus.
\par 11 Assim também vós considerai-vos mortos para o pecado, mas vivos para Deus, em Cristo Jesus.
\par 12 Não reine, portanto, o pecado em vosso corpo mortal, de maneira que obedeçais às suas paixões;
\par 13 nem ofereçais cada um os membros do seu corpo ao pecado, como instrumentos de iniqüidade; mas oferecei-vos a Deus, como ressurretos dentre os mortos, e os vossos membros, a Deus, como instrumentos de justiça.
\par 14 Porque o pecado não terá domínio sobre vós; pois não estais debaixo da lei, e sim da graça.
\par 15 E daí? Havemos de pecar porque não estamos debaixo da lei, e sim da graça? De modo nenhum!
\par 16 Não sabeis que daquele a quem vos ofereceis como servos para obediência, desse mesmo a quem obedeceis sois servos, seja do pecado para a morte ou da obediência para a justiça?
\par 17 Mas graças a Deus porque, outrora, escravos do pecado, contudo, viestes a obedecer de coração à forma de doutrina a que fostes entregues;
\par 18 e, uma vez libertados do pecado, fostes feitos servos da justiça.
\par 19 Falo como homem, por causa da fraqueza da vossa carne. Assim como oferecestes os vossos membros para a escravidão da impureza e da maldade para a maldade, assim oferecei, agora, os vossos membros para servirem à justiça para a santificação.
\par 20 Porque, quando éreis escravos do pecado, estáveis isentos em relação à justiça.
\par 21 Naquele tempo, que resultados colhestes? Somente as coisas de que, agora, vos envergonhais; porque o fim delas é morte.
\par 22 Agora, porém, libertados do pecado, transformados em servos de Deus, tendes o vosso fruto para a santificação e, por fim, a vida eterna;
\par 23 porque o salário do pecado é a morte, mas o dom gratuito de Deus é a vida eterna em Cristo Jesus, nosso Senhor.

\chapter{7}

\par 1 Porventura, ignorais, irmãos (pois falo aos que conhecem a lei), que a lei tem domínio sobre o homem toda a sua vida?
\par 2 Ora, a mulher casada está ligada pela lei ao marido, enquanto ele vive; mas, se o mesmo morrer, desobrigada ficará da lei conjugal.
\par 3 De sorte que será considerada adúltera se, vivendo ainda o marido, unir-se com outro homem; porém, se morrer o marido, estará livre da lei e não será adúltera se contrair novas núpcias.
\par 4 Assim, meus irmãos, também vós morrestes relativamente à lei, por meio do corpo de Cristo, para pertencerdes a outro, a saber, aquele que ressuscitou dentre os mortos, a fim de que frutifiquemos para Deus.
\par 5 Porque, quando vivíamos segundo a carne, as paixões pecaminosas postas em realce pela lei operavam em nossos membros, a fim de frutificarem para a morte.
\par 6 Agora, porém, libertados da lei, estamos mortos para aquilo a que estávamos sujeitos, de modo que servimos em novidade de espírito e não na caducidade da letra.
\par 7 Que diremos, pois? É a lei pecado? De modo nenhum! Mas eu não teria conhecido o pecado, senão por intermédio da lei; pois não teria eu conhecido a cobiça, se a lei não dissera: Não cobiçarás.
\par 8 Mas o pecado, tomando ocasião pelo mandamento, despertou em mim toda sorte de concupiscência; porque, sem lei, está morto o pecado.
\par 9 Outrora, sem a lei, eu vivia; mas, sobrevindo o preceito, reviveu o pecado, e eu morri.
\par 10 E o mandamento que me fora para vida, verifiquei que este mesmo se me tornou para morte.
\par 11 Porque o pecado, prevalecendo-se do mandamento, pelo mesmo mandamento, me enganou e me matou.
\par 12 Por conseguinte, a lei é santa; e o mandamento, santo, e justo, e bom.
\par 13 Acaso o bom se me tornou em morte? De modo nenhum! Pelo contrário, o pecado, para revelar-se como pecado, por meio de uma coisa boa, causou-me a morte, a fim de que, pelo mandamento, se mostrasse sobremaneira maligno.
\par 14 Porque bem sabemos que a lei é espiritual; eu, todavia, sou carnal, vendido à escravidão do pecado.
\par 15 Porque nem mesmo compreendo o meu próprio modo de agir, pois não faço o que prefiro, e sim o que detesto.
\par 16 Ora, se faço o que não quero, consinto com a lei, que é boa.
\par 17 Neste caso, quem faz isto já não sou eu, mas o pecado que habita em mim.
\par 18 Porque eu sei que em mim, isto é, na minha carne, não habita bem nenhum, pois o querer o bem está em mim; não, porém, o efetuá-lo.
\par 19 Porque não faço o bem que prefiro, mas o mal que não quero, esse faço.
\par 20 Mas, se eu faço o que não quero, já não sou eu quem o faz, e sim o pecado que habita em mim.
\par 21 Então, ao querer fazer o bem, encontro a lei de que o mal reside em mim.
\par 22 Porque, no tocante ao homem interior, tenho prazer na lei de Deus;
\par 23 mas vejo, nos meus membros, outra lei que, guerreando contra a lei da minha mente, me faz prisioneiro da lei do pecado que está nos meus membros.
\par 24 Desventurado homem que sou! Quem me livrará do corpo desta morte?
\par 25 Graças a Deus por Jesus Cristo, nosso Senhor. De maneira que eu, de mim mesmo, com a mente, sou escravo da lei de Deus, mas, segundo a carne, da lei do pecado.

\chapter{8}

\par 1 Agora, pois, já nenhuma condenação há para os que estão em Cristo Jesus.
\par 2 Porque a lei do Espírito da vida, em Cristo Jesus, te livrou da lei do pecado e da morte.
\par 3 Porquanto o que fora impossível à lei, no que estava enferma pela carne, isso fez Deus enviando o seu próprio Filho em semelhança de carne pecaminosa e no tocante ao pecado; e, com efeito, condenou Deus, na carne, o pecado,
\par 4 a fim de que o preceito da lei se cumprisse em nós, que não andamos segundo a carne, mas segundo o Espírito.
\par 5 Porque os que se inclinam para a carne cogitam das coisas da carne; mas os que se inclinam para o Espírito, das coisas do Espírito.
\par 6 Porque o pendor da carne dá para a morte, mas o do Espírito, para a vida e paz.
\par 7 Por isso, o pendor da carne é inimizade contra Deus, pois não está sujeito à lei de Deus, nem mesmo pode estar.
\par 8 Portanto, os que estão na carne não podem agradar a Deus.
\par 9 Vós, porém, não estais na carne, mas no Espírito, se, de fato, o Espírito de Deus habita em vós. E, se alguém não tem o Espírito de Cristo, esse tal não é dele.
\par 10 Se, porém, Cristo está em vós, o corpo, na verdade, está morto por causa do pecado, mas o espírito é vida, por causa da justiça.
\par 11 Se habita em vós o Espírito daquele que ressuscitou a Jesus dentre os mortos, esse mesmo que ressuscitou a Cristo Jesus dentre os mortos vivificará também o vosso corpo mortal, por meio do seu Espírito, que em vós habita.
\par 12 Assim, pois, irmãos, somos devedores, não à carne como se constrangidos a viver segundo a carne.
\par 13 Porque, se viverdes segundo a carne, caminhais para a morte; mas, se, pelo Espírito, mortificardes os feitos do corpo, certamente, vivereis.
\par 14 Pois todos os que são guiados pelo Espírito de Deus são filhos de Deus.
\par 15 Porque não recebestes o espírito de escravidão, para viverdes, outra vez, atemorizados, mas recebestes o espírito de adoção, baseados no qual clamamos: Aba, Pai.
\par 16 O próprio Espírito testifica com o nosso espírito que somos filhos de Deus.
\par 17 Ora, se somos filhos, somos também herdeiros, herdeiros de Deus e co-herdeiros com Cristo; se com ele sofremos, também com ele seremos glorificados.
\par 18 Porque para mim tenho por certo que os sofrimentos do tempo presente não podem ser comparados com a glória a ser revelada em nós.
\par 19 A ardente expectativa da criação aguarda a revelação dos filhos de Deus.
\par 20 Pois a criação está sujeita à vaidade, não voluntariamente, mas por causa daquele que a sujeitou,
\par 21 na esperança de que a própria criação será redimida do cativeiro da corrupção, para a liberdade da glória dos filhos de Deus.
\par 22 Porque sabemos que toda a criação, a um só tempo, geme e suporta angústias até agora.
\par 23 E não somente ela, mas também nós, que temos as primícias do Espírito, igualmente gememos em nosso íntimo, aguardando a adoção de filhos, a redenção do nosso corpo.
\par 24 Porque, na esperança, fomos salvos. Ora, esperança que se vê não é esperança; pois o que alguém vê, como o espera?
\par 25 Mas, se esperamos o que não vemos, com paciência o aguardamos.
\par 26 Também o Espírito, semelhantemente, nos assiste em nossa fraqueza; porque não sabemos orar como convém, mas o mesmo Espírito intercede por nós sobremaneira, com gemidos inexprimíveis.
\par 27 E aquele que sonda os corações sabe qual é a mente do Espírito, porque segundo a vontade de Deus é que ele intercede pelos santos.
\par 28 Sabemos que todas as coisas cooperam para o bem daqueles que amam a Deus, daqueles que são chamados segundo o seu propósito.
\par 29 Porquanto aos que de antemão conheceu, também os predestinou para serem conformes à imagem de seu Filho, a fim de que ele seja o primogênito entre muitos irmãos.
\par 30 E aos que predestinou, a esses também chamou; e aos que chamou, a esses também justificou; e aos que justificou, a esses também glorificou.
\par 31 Que diremos, pois, à vista destas coisas? Se Deus é por nós, quem será contra nós?
\par 32 Aquele que não poupou o seu próprio Filho, antes, por todos nós o entregou, porventura, não nos dará graciosamente com ele todas as coisas?
\par 33 Quem intentará acusação contra os eleitos de Deus? É Deus quem os justifica.
\par 34 Quem os condenará? É Cristo Jesus quem morreu ou, antes, quem ressuscitou, o qual está à direita de Deus e também intercede por nós.
\par 35 Quem nos separará do amor de Cristo? Será tribulação, ou angústia, ou perseguição, ou fome, ou nudez, ou perigo, ou espada?
\par 36 Como está escrito: Por amor de ti, somos entregues à morte o dia todo, fomos considerados como ovelhas para o matadouro.
\par 37 Em todas estas coisas, porém, somos mais que vencedores, por meio daquele que nos amou.
\par 38 Porque eu estou bem certo de que nem a morte, nem a vida, nem os anjos, nem os principados, nem as coisas do presente, nem do porvir, nem os poderes,
\par 39 nem a altura, nem a profundidade, nem qualquer outra criatura poderá separar-nos do amor de Deus, que está em Cristo Jesus, nosso Senhor.

\chapter{9}

\par 1 Digo a verdade em Cristo, não minto, testemunhando comigo, no Espírito Santo, a minha própria consciência:
\par 2 tenho grande tristeza e incessante dor no coração;
\par 3 porque eu mesmo desejaria ser anátema, separado de Cristo, por amor de meus irmãos, meus compatriotas, segundo a carne.
\par 4 São israelitas. Pertence-lhes a adoção e também a glória, as alianças, a legislação, o culto e as promessas;
\par 5 deles são os patriarcas, e também deles descende o Cristo, segundo a carne, o qual é sobre todos, Deus bendito para todo o sempre. Amém!
\par 6 E não pensemos que a palavra de Deus haja falhado, porque nem todos os de Israel são, de fato, israelitas;
\par 7 nem por serem descendentes de Abraão são todos seus filhos; mas: Em Isaque será chamada a tua descendência.
\par 8 Isto é, estes filhos de Deus não são propriamente os da carne, mas devem ser considerados como descendência os filhos da promessa.
\par 9 Porque a palavra da promessa é esta: Por esse tempo, virei, e Sara terá um filho.
\par 10 E não ela somente, mas também Rebeca, ao conceber de um só, Isaque, nosso pai.
\par 11 E ainda não eram os gêmeos nascidos, nem tinham praticado o bem ou o mal (para que o propósito de Deus, quanto à eleição, prevalecesse, não por obras, mas por aquele que chama),
\par 12 já fora dito a ela: O mais velho será servo do mais moço.
\par 13 Como está escrito: Amei Jacó, porém me aborreci de Esaú.
\par 14 Que diremos, pois? Há injustiça da parte de Deus? De modo nenhum!
\par 15 Pois ele diz a Moisés: Terei misericórdia de quem me aprouver ter misericórdia e compadecer-me-ei de quem me aprouver ter compaixão.
\par 16 Assim, pois, não depende de quem quer ou de quem corre, mas de usar Deus a sua misericórdia.
\par 17 Porque a Escritura diz a Faraó: Para isto mesmo te levantei, para mostrar em ti o meu poder e para que o meu nome seja anunciado por toda a terra.
\par 18 Logo, tem ele misericórdia de quem quer e também endurece a quem lhe apraz.
\par 19 Tu, porém, me dirás: De que se queixa ele ainda? Pois quem jamais resistiu à sua vontade?
\par 20 Quem és tu, ó homem, para discutires com Deus?! Porventura, pode o objeto perguntar a quem o fez: Por que me fizeste assim?
\par 21 Ou não tem o oleiro direito sobre a massa, para do mesmo barro fazer um vaso para honra e outro, para desonra?
\par 22 Que diremos, pois, se Deus, querendo mostrar a sua ira e dar a conhecer o seu poder, suportou com muita longanimidade os vasos de ira, preparados para a perdição,
\par 23 a fim de que também desse a conhecer as riquezas da sua glória em vasos de misericórdia, que para glória preparou de antemão,
\par 24 os quais somos nós, a quem também chamou, não só dentre os judeus, mas também dentre os gentios?
\par 25 Assim como também diz em Oséias: Chamarei povo meu ao que não era meu povo; e amada, à que não era amada;
\par 26 e no lugar em que se lhes disse: Vós não sois meu povo, ali mesmo serão chamados filhos do Deus vivo.
\par 27 Mas, relativamente a Israel, dele clama Isaías: Ainda que o número dos filhos de Israel seja como a areia do mar, o remanescente é que será salvo.
\par 28 Porque o Senhor cumprirá a sua palavra sobre a terra, cabalmente e em breve;
\par 29 como Isaías já disse: Se o Senhor dos Exércitos não nos tivesse deixado descendência, ter-nos-íamos tornado como Sodoma e semelhantes a Gomorra.
\par 30 Que diremos, pois? Que os gentios, que não buscavam a justificação, vieram a alcançá-la, todavia, a que decorre da fé;
\par 31 e Israel, que buscava a lei de justiça, não chegou a atingir essa lei.
\par 32 Por quê? Porque não decorreu da fé, e sim como que das obras. Tropeçaram na pedra de tropeço,
\par 33 como está escrito: Eis que ponho em Sião uma pedra de tropeço e rocha de escândalo, e aquele que nela crê não será confundido.

\chapter{10}

\par 1 Irmãos, a boa vontade do meu coração e a minha súplica a Deus a favor deles são para que sejam salvos.
\par 2 Porque lhes dou testemunho de que eles têm zelo por Deus, porém não com entendimento.
\par 3 Porquanto, desconhecendo a justiça de Deus e procurando estabelecer a sua própria, não se sujeitaram à que vem de Deus.
\par 4 Porque o fim da lei é Cristo, para justiça de todo aquele que crê.
\par 5 Ora, Moisés escreveu que o homem que praticar a justiça decorrente da lei viverá por ela.
\par 6 Mas a justiça decorrente da fé assim diz: Não perguntes em teu coração: Quem subirá ao céu?, isto é, para trazer do alto a Cristo;
\par 7 ou: Quem descerá ao abismo?, isto é, para levantar Cristo dentre os mortos.
\par 8 Porém que se diz? A palavra está perto de ti, na tua boca e no teu coração; isto é, a palavra da fé que pregamos.
\par 9 Se, com a tua boca, confessares Jesus como Senhor e, em teu coração, creres que Deus o ressuscitou dentre os mortos, serás salvo.
\par 10 Porque com o coração se crê para justiça e com a boca se confessa a respeito da salvação.
\par 11 Porquanto a Escritura diz: Todo aquele que nele crê não será confundido.
\par 12 Pois não há distinção entre judeu e grego, uma vez que o mesmo é o Senhor de todos, rico para com todos os que o invocam.
\par 13 Porque: Todo aquele que invocar o nome do Senhor será salvo.
\par 14 Como, porém, invocarão aquele em quem não creram? E como crerão naquele de quem nada ouviram? E como ouvirão, se não há quem pregue?
\par 15 E como pregarão, se não forem enviados? Como está escrito: Quão formosos são os pés dos que anunciam coisas boas!
\par 16 Mas nem todos obedeceram ao evangelho; pois Isaías diz: Senhor, quem acreditou na nossa pregação?
\par 17 E, assim, a fé vem pela pregação, e a pregação, pela palavra de Cristo.
\par 18 Mas pergunto: Porventura, não ouviram? Sim, por certo: Por toda a terra se fez ouvir a sua voz, e as suas palavras, até aos confins do mundo.
\par 19 Pergunto mais: Porventura, não terá chegado isso ao conhecimento de Israel? Moisés já dizia: Eu vos porei em ciúmes com um povo que não é nação, com gente insensata eu vos provocarei à ira.
\par 20 E Isaías a mais se atreve e diz: Fui achado pelos que não me procuravam, revelei-me aos que não perguntavam por mim.
\par 21 Quanto a Israel, porém, diz: Todo o dia estendi as mãos a um povo rebelde e contradizente.

\chapter{11}

\par 1 Pergunto, pois: terá Deus, porventura, rejeitado o seu povo? De modo nenhum! Porque eu também sou israelita da descendência de Abraão, da tribo de Benjamim.
\par 2 Deus não rejeitou o seu povo, a quem de antemão conheceu. Ou não sabeis o que a Escritura refere a respeito de Elias, como insta perante Deus contra Israel, dizendo:
\par 3 Senhor, mataram os teus profetas, arrasaram os teus altares, e só eu fiquei, e procuram tirar-me a vida.
\par 4 Que lhe disse, porém, a resposta divina? Reservei para mim sete mil homens, que não dobraram os joelhos diante de Baal.
\par 5 Assim, pois, também agora, no tempo de hoje, sobrevive um remanescente segundo a eleição da graça.
\par 6 E, se é pela graça, já não é pelas obras; do contrário, a graça já não é graça.
\par 7 Que diremos, pois? O que Israel busca, isso não conseguiu; mas a eleição o alcançou; e os mais foram endurecidos,
\par 8 como está escrito: Deus lhes deu espírito de entorpecimento, olhos para não ver e ouvidos para não ouvir, até ao dia de hoje.
\par 9 E diz Davi: Torne-se-lhes a mesa em laço e armadilha, em tropeço e punição;
\par 10 escureçam-se-lhes os olhos, para que não vejam, e fiquem para sempre encurvadas as suas costas.
\par 11 Pergunto, pois: porventura, tropeçaram para que caíssem? De modo nenhum! Mas, pela sua transgressão, veio a salvação aos gentios, para pô-los em ciúmes.
\par 12 Ora, se a transgressão deles redundou em riqueza para o mundo, e o seu abatimento, em riqueza para os gentios, quanto mais a sua plenitude!
\par 13 Dirijo-me a vós outros, que sois gentios! Visto, pois, que eu sou apóstolo dos gentios, glorifico o meu ministério,
\par 14 para ver se, de algum modo, posso incitar à emulação os do meu povo e salvar alguns deles.
\par 15 Porque, se o fato de terem sido eles rejeitados trouxe reconciliação ao mundo, que será o seu restabelecimento, senão vida dentre os mortos?
\par 16 E, se forem santas as primícias da massa, igualmente o será a sua totalidade; se for santa a raiz, também os ramos o serão.
\par 17 Se, porém, alguns dos ramos foram quebrados, e tu, sendo oliveira brava, foste enxertado em meio deles e te tornaste participante da raiz e da seiva da oliveira,
\par 18 não te glories contra os ramos; porém, se te gloriares, sabe que não és tu que sustentas a raiz, mas a raiz, a ti.
\par 19 Dirás, pois: Alguns ramos foram quebrados, para que eu fosse enxertado.
\par 20 Bem! Pela sua incredulidade, foram quebrados; tu, porém, mediante a fé, estás firme. Não te ensoberbeças, mas teme.
\par 21 Porque, se Deus não poupou os ramos naturais, também não te poupará.
\par 22 Considerai, pois, a bondade e a severidade de Deus: para com os que caíram, severidade; mas, para contigo, a bondade de Deus, se nela permaneceres; doutra sorte, também tu serás cortado.
\par 23 Eles também, se não permanecerem na incredulidade, serão enxertados; pois Deus é poderoso para os enxertar de novo.
\par 24 Pois, se foste cortado da que, por natureza, era oliveira brava e, contra a natureza, enxertado em boa oliveira, quanto mais não serão enxertados na sua própria oliveira aqueles que são ramos naturais!
\par 25 Porque não quero, irmãos, que ignoreis este mistério (para que não sejais presumidos em vós mesmos): que veio endurecimento em parte a Israel, até que haja entrado a plenitude dos gentios.
\par 26 E, assim, todo o Israel será salvo, como está escrito: Virá de Sião o Libertador e ele apartará de Jacó as impiedades.
\par 27 Esta é a minha aliança com eles, quando eu tirar os seus pecados.
\par 28 Quanto ao evangelho, são eles inimigos por vossa causa; quanto, porém, à eleição, amados por causa dos patriarcas;
\par 29 porque os dons e a vocação de Deus são irrevogáveis.
\par 30 Porque assim como vós também, outrora, fostes desobedientes a Deus, mas, agora, alcançastes misericórdia, à vista da desobediência deles,
\par 31 assim também estes, agora, foram desobedientes, para que, igualmente, eles alcancem misericórdia, à vista da que vos foi concedida.
\par 32 Porque Deus a todos encerrou na desobediência, a fim de usar de misericórdia para com todos.
\par 33 Ó profundidade da riqueza, tanto da sabedoria como do conhecimento de Deus! Quão insondáveis são os seus juízos, e quão inescrutáveis, os seus caminhos!
\par 34 Quem, pois, conheceu a mente do Senhor? Ou quem foi o seu conselheiro?
\par 35 Ou quem primeiro deu a ele para que lhe venha a ser restituído?
\par 36 Porque dele, e por meio dele, e para ele são todas as coisas. A ele, pois, a glória eternamente. Amém!

\chapter{12}

\par 1 Rogo-vos, pois, irmãos, pelas misericórdias de Deus, que apresenteis o vosso corpo por sacrifício vivo, santo e agradável a Deus, que é o vosso culto racional.
\par 2 E não vos conformeis com este século, mas transformai-vos pela renovação da vossa mente, para que experimenteis qual seja a boa, agradável e perfeita vontade de Deus.
\par 3 Porque, pela graça que me foi dada, digo a cada um dentre vós que não pense de si mesmo além do que convém; antes, pense com moderação, segundo a medida da fé que Deus repartiu a cada um.
\par 4 Porque assim como num só corpo temos muitos membros, mas nem todos os membros têm a mesma função,
\par 5 assim também nós, conquanto muitos, somos um só corpo em Cristo e membros uns dos outros,
\par 6 tendo, porém, diferentes dons segundo a graça que nos foi dada: se profecia, seja segundo a proporção da fé;
\par 7 se ministério, dediquemo-nos ao ministério; ou o que ensina esmere-se no fazê-lo;
\par 8 ou o que exorta faça-o com dedicação; o que contribui, com liberalidade; o que preside, com diligência; quem exerce misericórdia, com alegria.
\par 9 O amor seja sem hipocrisia. Detestai o mal, apegando-vos ao bem.
\par 10 Amai-vos cordialmente uns aos outros com amor fraternal, preferindo-vos em honra uns aos outros.
\par 11 No zelo, não sejais remissos; sede fervorosos de espírito, servindo ao Senhor;
\par 12 regozijai-vos na esperança, sede pacientes na tribulação, na oração, perseverantes;
\par 13 compartilhai as necessidades dos santos; praticai a hospitalidade;
\par 14 abençoai os que vos perseguem, abençoai e não amaldiçoeis.
\par 15 Alegrai-vos com os que se alegram e chorai com os que choram.
\par 16 Tende o mesmo sentimento uns para com os outros; em lugar de serdes orgulhosos, condescendei com o que é humilde; não sejais sábios aos vossos próprios olhos.
\par 17 Não torneis a ninguém mal por mal; esforçai-vos por fazer o bem perante todos os homens;
\par 18 se possível, quanto depender de vós, tende paz com todos os homens;
\par 19 não vos vingueis a vós mesmos, amados, mas dai lugar à ira; porque está escrito: A mim me pertence a vingança; eu é que retribuirei, diz o Senhor.
\par 20 Pelo contrário, se o teu inimigo tiver fome, dá-lhe de comer; se tiver sede, dá-lhe de beber; porque, fazendo isto, amontoarás brasas vivas sobre a sua cabeça.
\par 21 Não te deixes vencer do mal, mas vence o mal com o bem.

\chapter{13}

\par 1 Todo homem esteja sujeito às autoridades superiores; porque não há autoridade que não proceda de Deus; e as autoridades que existem foram por ele instituídas.
\par 2 De modo que aquele que se opõe à autoridade resiste à ordenação de Deus; e os que resistem trarão sobre si mesmos condenação.
\par 3 Porque os magistrados não são para temor, quando se faz o bem, e sim quando se faz o mal. Queres tu não temer a autoridade? Faze o bem e terás louvor dela,
\par 4 visto que a autoridade é ministro de Deus para teu bem. Entretanto, se fizeres o mal, teme; porque não é sem motivo que ela traz a espada; pois é ministro de Deus, vingador, para castigar o que pratica o mal.
\par 5 É necessário que lhe estejais sujeitos, não somente por causa do temor da punição, mas também por dever de consciência.
\par 6 Por esse motivo, também pagais tributos, porque são ministros de Deus, atendendo, constantemente, a este serviço.
\par 7 Pagai a todos o que lhes é devido: a quem tributo, tributo; a quem imposto, imposto; a quem respeito, respeito; a quem honra, honra.
\par 8 A ninguém fiqueis devendo coisa alguma, exceto o amor com que vos ameis uns aos outros; pois quem ama o próximo tem cumprido a lei.
\par 9 Pois isto: Não adulterarás, não matarás, não furtarás, não cobiçarás, e, se há qualquer outro mandamento, tudo nesta palavra se resume: Amarás o teu próximo como a ti mesmo.
\par 10 O amor não pratica o mal contra o próximo; de sorte que o cumprimento da lei é o amor.
\par 11 E digo isto a vós outros que conheceis o tempo: já é hora de vos despertardes do sono; porque a nossa salvação está, agora, mais perto do que quando no princípio cremos.
\par 12 Vai alta a noite, e vem chegando o dia. Deixemos, pois, as obras das trevas e revistamo-nos das armas da luz.
\par 13 Andemos dignamente, como em pleno dia, não em orgias e bebedices, não em impudicícias e dissoluções, não em contendas e ciúmes;
\par 14 mas revesti-vos do Senhor Jesus Cristo e nada disponhais para a carne no tocante às suas concupiscências.

\chapter{14}

\par 1 Acolhei ao que é débil na fé, não, porém, para discutir opiniões.
\par 2 Um crê que de tudo pode comer, mas o débil come legumes;
\par 3 quem come não despreze o que não come; e o que não come não julgue o que come, porque Deus o acolheu.
\par 4 Quem és tu que julgas o servo alheio? Para o seu próprio senhor está em pé ou cai; mas estará em pé, porque o Senhor é poderoso para o suster.
\par 5 Um faz diferença entre dia e dia; outro julga iguais todos os dias. Cada um tenha opinião bem definida em sua própria mente.
\par 6 Quem distingue entre dia e dia para o Senhor o faz; e quem come para o Senhor come, porque dá graças a Deus; e quem não come para o Senhor não come e dá graças a Deus.
\par 7 Porque nenhum de nós vive para si mesmo, nem morre para si.
\par 8 Porque, se vivemos, para o Senhor vivemos; se morremos, para o Senhor morremos. Quer, pois, vivamos ou morramos, somos do Senhor.
\par 9 Foi precisamente para esse fim que Cristo morreu e ressurgiu: para ser Senhor tanto de mortos como de vivos.
\par 10 Tu, porém, por que julgas teu irmão? E tu, por que desprezas o teu? Pois todos compareceremos perante o tribunal de Deus.
\par 11 Como está escrito: Por minha vida, diz o Senhor, diante de mim se dobrará todo joelho, e toda língua dará louvores a Deus.
\par 12 Assim, pois, cada um de nós dará contas de si mesmo a Deus.
\par 13 Não nos julguemos mais uns aos outros; pelo contrário, tomai o propósito de não pordes tropeço ou escândalo ao vosso irmão.
\par 14 Eu sei e estou persuadido, no Senhor Jesus, de que nenhuma coisa é de si mesma impura, salvo para aquele que assim a considera; para esse é impura.
\par 15 Se, por causa de comida, o teu irmão se entristece, já não andas segundo o amor fraternal. Por causa da tua comida, não faças perecer aquele a favor de quem Cristo morreu.
\par 16 Não seja, pois, vituperado o vosso bem.
\par 17 Porque o reino de Deus não é comida nem bebida, mas justiça, e paz, e alegria no Espírito Santo.
\par 18 Aquele que deste modo serve a Cristo é agradável a Deus e aprovado pelos homens.
\par 19 Assim, pois, seguimos as coisas da paz e também as da edificação de uns para com os outros.
\par 20 Não destruas a obra de Deus por causa da comida. Todas as coisas, na verdade, são limpas, mas é mau para o homem o comer com escândalo.
\par 21 É bom não comer carne, nem beber vinho, nem fazer qualquer outra coisa com que teu irmão venha a tropeçar [ou se ofender ou se enfraquecer].
\par 22 A fé que tens, tem-na para ti mesmo perante Deus. Bem-aventurado é aquele que não se condena naquilo que aprova.
\par 23 Mas aquele que tem dúvidas é condenado se comer, porque o que faz não provém de fé; e tudo o que não provém de fé é pecado.

\chapter{15}

\par 1 Ora, nós que somos fortes devemos suportar as debilidades dos fracos e não agradar-nos a nós mesmos.
\par 2 Portanto, cada um de nós agrade ao próximo no que é bom para edificação.
\par 3 Porque também Cristo não se agradou a si mesmo; antes, como está escrito: As injúrias dos que te ultrajavam caíram sobre mim.
\par 4 Pois tudo quanto, outrora, foi escrito para o nosso ensino foi escrito, a fim de que, pela paciência e pela consolação das Escrituras, tenhamos esperança.
\par 5 Ora, o Deus da paciência e da consolação vos conceda o mesmo sentir de uns para com os outros, segundo Cristo Jesus,
\par 6 para que concordemente e a uma voz glorifiqueis ao Deus e Pai de nosso Senhor Jesus Cristo.
\par 7 Portanto, acolhei-vos uns aos outros, como também Cristo nos acolheu para a glória de Deus.
\par 8 Digo, pois, que Cristo foi constituído ministro da circuncisão, em prol da verdade de Deus, para confirmar as promessas feitas aos nossos pais;
\par 9 e para que os gentios glorifiquem a Deus por causa da sua misericórdia, como está escrito: Por isso, eu te glorificarei entre os gentios e cantarei louvores ao teu nome.
\par 10 E também diz: Alegrai-vos, ó gentios, com o seu povo.
\par 11 E ainda: Louvai ao Senhor, vós todos os gentios, e todos os povos o louvem.
\par 12 Também Isaías diz: Haverá a raiz de Jessé, aquele que se levanta para governar os gentios; nele os gentios esperarão.
\par 13 E o Deus da esperança vos encha de todo o gozo e paz no vosso crer, para que sejais ricos de esperança no poder do Espírito Santo.
\par 14 E certo estou, meus irmãos, sim, eu mesmo, a vosso respeito, de que estais possuídos de bondade, cheios de todo o conhecimento, aptos para vos admoestardes uns aos outros.
\par 15 Entretanto, vos escrevi em parte mais ousadamente, como para vos trazer isto de novo à memória, por causa da graça que me foi outorgada por Deus,
\par 16 para que eu seja ministro de Cristo Jesus entre os gentios, no sagrado encargo de anunciar o evangelho de Deus, de modo que a oferta deles seja aceitável, uma vez santificada pelo Espírito Santo.
\par 17 Tenho, pois, motivo de gloriar-me em Cristo Jesus nas coisas concernentes a Deus.
\par 18 Porque não ousarei discorrer sobre coisa alguma, senão sobre aquelas que Cristo fez por meu intermédio, para conduzir os gentios à obediência, por palavra e por obras,
\par 19 por força de sinais e prodígios, pelo poder do Espírito Santo; de maneira que, desde Jerusalém e circunvizinhanças até ao Ilírico, tenho divulgado o evangelho de Cristo,
\par 20 esforçando-me, deste modo, por pregar o evangelho, não onde Cristo já fora anunciado, para não edificar sobre fundamento alheio;
\par 21 antes, como está escrito: Hão de vê-lo aqueles que não tiveram notícia dele, e compreendê-lo os que nada tinham ouvido a seu respeito.
\par 22 Essa foi a razão por que também, muitas vezes, me senti impedido de visitar-vos.
\par 23 Mas, agora, não tendo já campo de atividade nestas regiões e desejando há muito visitar-vos,
\par 24 penso em fazê-lo quando em viagem para a Espanha, pois espero que, de passagem, estarei convosco e que para lá seja por vós encaminhado, depois de haver primeiro desfrutado um pouco a vossa companhia.
\par 25 Mas, agora, estou de partida para Jerusalém, a serviço dos santos.
\par 26 Porque aprouve à Macedônia e à Acaia levantar uma coleta em benefício dos pobres dentre os santos que vivem em Jerusalém.
\par 27 Isto lhes pareceu bem, e mesmo lhes são devedores; porque, se os gentios têm sido participantes dos valores espirituais dos judeus, devem também servi-los com bens materiais.
\par 28 Tendo, pois, concluído isto e havendo-lhes consignado este fruto, passando por vós, irei à Espanha.
\par 29 E bem sei que, ao visitar-vos, irei na plenitude da bênção de Cristo.
\par 30 Rogo-vos, pois, irmãos, por nosso Senhor Jesus Cristo e também pelo amor do Espírito, que luteis juntamente comigo nas orações a Deus a meu favor,
\par 31 para que eu me veja livre dos rebeldes que vivem na Judéia, e que este meu serviço em Jerusalém seja bem aceito pelos santos;
\par 32 a fim de que, ao visitar-vos, pela vontade de Deus, chegue à vossa presença com alegria e possa recrear-me convosco.
\par 33 E o Deus da paz seja com todos vós. Amém!

\chapter{16}

\par 1 Recomendo-vos a nossa irmã Febe, que está servindo à igreja de Cencréia,
\par 2 para que a recebais no Senhor como convém aos santos e a ajudeis em tudo que de vós vier a precisar; porque tem sido protetora de muitos e de mim inclusive.
\par 3 Saudai Priscila e Áqüila, meus cooperadores em Cristo Jesus,
\par 4 os quais pela minha vida arriscaram a sua própria cabeça; e isto lhes agradeço, não somente eu, mas também todas as igrejas dos gentios;
\par 5 saudai igualmente a igreja que se reúne na casa deles. Saudai meu querido Epêneto, primícias da Ásia para Cristo.
\par 6 Saudai Maria, que muito trabalhou por vós.
\par 7 Saudai Andrônico e Júnias, meus parentes e companheiros de prisão, os quais são notáveis entre os apóstolos e estavam em Cristo antes de mim.
\par 8 Saudai Amplíato, meu dileto amigo no Senhor.
\par 9 Saudai Urbano, que é nosso cooperador em Cristo, e também meu amado Estáquis.
\par 10 Saudai Apeles, aprovado em Cristo. Saudai os da casa de Aristóbulo.
\par 11 Saudai meu parente Herodião. Saudai os da casa de Narciso, que estão no Senhor.
\par 12 Saudai Trifena e Trifosa, as quais trabalhavam no Senhor. Saudai a estimada Pérside, que também muito trabalhou no Senhor.
\par 13 Saudai Rufo, eleito no Senhor, e igualmente a sua mãe, que também tem sido mãe para mim.
\par 14 Saudai Asíncrito, Flegonte, Hermes, Pátrobas, Hermas e os irmãos que se reúnem com eles.
\par 15 Saudai Filólogo, Júlia, Nereu e sua irmã, Olimpas e todos os santos que se reúnem com eles.
\par 16 Saudai-vos uns aos outros com ósculo santo. Todas as igrejas de Cristo vos saúdam.
\par 17 Rogo-vos, irmãos, que noteis bem aqueles que provocam divisões e escândalos, em desacordo com a doutrina que aprendestes; afastai-vos deles,
\par 18 porque esses tais não servem a Cristo, nosso Senhor, e sim a seu próprio ventre; e, com suaves palavras e lisonjas, enganam o coração dos incautos.
\par 19 Pois a vossa obediência é conhecida por todos; por isso, me alegro a vosso respeito; e quero que sejais sábios para o bem e símplices para o mal.
\par 20 E o Deus da paz, em breve, esmagará debaixo dos vossos pés a Satanás. A graça de nosso Senhor Jesus seja convosco.
\par 21 Saúda-vos Timóteo, meu cooperador, e Lúcio, Jasom e Sosípatro, meus parentes.
\par 22 Eu, Tércio, que escrevi esta epístola, vos saúdo no Senhor.
\par 23 Saúda-vos Gaio, meu hospedeiro e de toda a igreja. Saúda-vos Erasto, tesoureiro da cidade, e o irmão Quarto.
\par 24 [A graça de nosso Senhor Jesus Cristo seja com todos vós. Amém!]
\par 25 Ora, àquele que é poderoso para vos confirmar segundo o meu evangelho e a pregação de Jesus Cristo, conforme a revelação do mistério guardado em silêncio nos tempos eternos,
\par 26 e que, agora, se tornou manifesto e foi dado a conhecer por meio das Escrituras proféticas, segundo o mandamento do Deus eterno, para a obediência por fé, entre todas as nações,
\par 27 ao Deus único e sábio seja dada glória, por meio de Jesus Cristo, pelos séculos dos séculos. Amém!


\end{document}