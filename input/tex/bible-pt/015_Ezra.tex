\begin{document}

\title{Esdras}


\chapter{1}

\par 1 No primeiro ano de Ciro, rei da Pérsia, para que se cumprisse a palavra do SENHOR, por boca de Jeremias, despertou o SENHOR o espírito de Ciro, rei da Pérsia, o qual fez passar pregão por todo o seu reino, como também por escrito, dizendo:
\par 2 Assim diz Ciro, rei da Pérsia: O SENHOR, Deus dos céus, me deu todos os reinos da terra e me encarregou de lhe edificar uma casa em Jerusalém de Judá.
\par 3 Quem dentre vós é, de todo o seu povo, seja seu Deus com ele, e suba a Jerusalém de Judá e edifique a Casa do SENHOR, Deus de Israel; ele é o Deus que habita em Jerusalém.
\par 4 Todo aquele que restar em alguns lugares em que habita, os homens desse lugar o ajudarão com prata, ouro, bens e gado, afora as dádivas voluntárias para a Casa de Deus, a qual está em Jerusalém.
\par 5 Então, se levantaram os cabeças de famílias de Judá e de Benjamim, e os sacerdotes, e os levitas, com todos aqueles cujo espírito Deus despertou, para subirem a edificar a Casa do SENHOR, a qual está em Jerusalém.
\par 6 Todos os que habitavam nos arredores os ajudaram com objetos de prata, com ouro, bens, gado e coisas preciosas, afora tudo o que, voluntariamente, se deu.
\par 7 Também o rei Ciro tirou os utensílios da Casa do SENHOR, os quais Nabucodonosor tinha trazido de Jerusalém e que tinha posto na casa de seus deuses.
\par 8 Tirou-os Ciro, rei da Pérsia, sob a direção do tesoureiro Mitredate, que os entregou contados a Sesbazar, príncipe de Judá.
\par 9 Eis o número deles: trinta bacias de ouro, mil bacias de prata, vinte e nove facas,
\par 10 trinta taças de ouro, quatrocentas e dez taças de prata de outra espécie e mil outros objetos.
\par 11 Todos os utensílios de ouro e de prata foram cinco mil e quatrocentos; todos estes levou Sesbazar, quando os do exílio subiram da Babilônia para Jerusalém.

\chapter{2}

\par 1 São estes os filhos da província que subiram do cativeiro, dentre os exilados que Nabucodonosor, rei da Babilônia, tinha levado para lá, e voltaram para Jerusalém e para Judá, cada um para a sua cidade,
\par 2 os quais vieram com Zorobabel, Jesua, Neemias, Seraías, Reelaías, Mordecai, Bilsã, Mispar, Bigvai, Reum e Baaná. Eis o número dos homens do povo de Israel:
\par 3 os filhos de Parós, dois mil cento e setenta e dois.
\par 4 Os filhos de Sefatias, trezentos e setenta e dois.
\par 5 Os filhos de Ará, setecentos e setenta e cinco.
\par 6 Os filhos de Paate-Moabe, dos filhos de Jesua-Joabe, dois mil oitocentos e doze.
\par 7 Os filhos de Elão, mil duzentos e cinqüenta e quatro.
\par 8 Os filhos de Zatu, novecentos e quarenta e cinco.
\par 9 Os filhos de Zacai, setecentos e sessenta.
\par 10 Os filhos de Bani, seiscentos e quarenta e dois.
\par 11 Os filhos de Bebai, seiscentos e vinte e três.
\par 12 Os filhos de Azgade, mil duzentos e vinte e dois.
\par 13 Os filhos de Adonicão, seiscentos e sessenta e seis.
\par 14 Os filhos de Bigvai, dois mil e cinqüenta e seis.
\par 15 Os filhos de Adim, quatrocentos e cinqüenta e quatro.
\par 16 Os filhos de Ater, da família de Ezequias, noventa e oito.
\par 17 Os filhos de Bezai, trezentos e vinte e três.
\par 18 Os filhos de Jora, cento e doze.
\par 19 Os filhos de Hasum, duzentos e vinte e três.
\par 20 Os filhos de Gibar, noventa e cinco.
\par 21 Os filhos de Belém, cento e vinte e três.
\par 22 Os homens de Netofa, cinqüenta e seis.
\par 23 Os homens de Anatote, cento e vinte e oito.
\par 24 Os filhos de Azmavete, quarenta e dois.
\par 25 Os filhos de Quiriate-Arim, Cefira e Beerote, setecentos e quarenta e três.
\par 26 Os filhos de Ramá e de Geba, seiscentos e vinte e um.
\par 27 Os homens de Micmás, cento e vinte e dois.
\par 28 Os homens de Betel e Ai, duzentos e vinte e três.
\par 29 Os filhos de Nebo, cinqüenta e dois.
\par 30 Os filhos de Magbis, cento e cinqüenta e seis.
\par 31 Os filhos do outro Elão, mil duzentos e cinqüenta e quatro.
\par 32 Os filhos de Harim, trezentos e vinte.
\par 33 Os filhos de Lode, Hadide e Ono, setecentos e vinte e cinco.
\par 34 Os filhos de Jericó, trezentos e quarenta e cinco.
\par 35 Os filhos de Senaá, três mil seiscentos e trinta.
\par 36 Os sacerdotes: os filhos de Jedaías, da casa de Jesua, novecentos e setenta e três.
\par 37 Os filhos de Imer, mil e cinqüenta e dois.
\par 38 Os filhos de Pasur, mil duzentos e quarenta e sete.
\par 39 Os filhos de Harim, mil e dezessete.
\par 40 Os levitas: os filhos de Jesua e Cadmiel, dos filhos de Hodavias, setenta e quatro.
\par 41 Os cantores: os filhos de Asafe, cento e vinte e oito.
\par 42 Os filhos dos porteiros: os filhos de Salum, os filhos de Ater, os filhos de Talmom, os filhos de Acube, os filhos de Hatita, os filhos de Sobai; ao todo, cento e trinta e nove.
\par 43 Os servidores do templo: os filhos de Zia, os filhos de Hasufa, os filhos de Tabaote,
\par 44 os filhos de Queros, os filhos de Sia, os filhos de Padom,
\par 45 os filhos de Lebana, os filhos de Hagaba, os filhos de Acube, os filhos de Hagabe,
\par 46 os filhos de Sanlai, os filhos de Hanã,
\par 47 os filhos de Gidel, os filhos de Gaar, os filhos de Reaías,
\par 48 os filhos de Rezim, os filhos de Necoda, os filhos de Gazão,
\par 49 os filhos de Uzá, os filhos de Paséia, os filhos de Besai,
\par 50 os filhos de Asná, os filhos dos meunitas, os filhos dos nefuseus,
\par 51 os filhos de Baquebuque, os filhos de Hacufa, os filhos de Harur,
\par 52 os filhos de Baslute, os filhos de Meída, os filhos de Harsa,
\par 53 os filhos de Barcos, os filhos de Sísera, os filhos de Temá,
\par 54 os filhos de Nesias, os filhos de Hatifa.
\par 55 Os filhos dos servos de Salomão: os filhos de Sotai, os filhos de Soferete, os filhos de Peruda,
\par 56 os filhos de Jaala, os filhos de Darcom, os filhos de Gidel,
\par 57 os filhos de Sefatias, os filhos de Hatil, os filhos de Poquerete-Hazebaim e os filhos de Ami.
\par 58 Todos os servidores do templo e os filhos dos servos de Salomão, trezentos e noventa e dois.
\par 59 Também estes subiram de Tel-Melá, Tel-Harsa, Querube, Adã e Imer, porém não puderam provar que as suas famílias e a sua linhagem eram de Israel:
\par 60 os filhos de Delaías, os filhos de Tobias, os filhos de Necoda, seiscentos e cinqüenta e dois.
\par 61 Também dos filhos dos sacerdotes: os filhos de Habaías, os filhos de Coz, os filhos de Barzilai, que se casara com uma das filhas de Barzilai, o gileadita, e que foi chamado do nome dele.
\par 62 Estes procuraram o seu registro nos livros genealógicos, porém o não acharam; pelo que foram tidos por imundos para o sacerdócio.
\par 63 O governador lhes disse que não comessem das coisas sagradas, até que se levantasse um sacerdote com Urim e Tumim.
\par 64 Toda esta congregação junta foi de quarenta e dois mil trezentos e sessenta,
\par 65 afora os seus servos e as suas servas, que foram sete mil trezentos e trinta e sete; e tinham duzentos cantores e cantoras.
\par 66 Os seus cavalos, setecentos e trinta e seis; os seus mulos, duzentos e quarenta e cinco;
\par 67 os seus camelos, quatrocentos e trinta e cinco; os jumentos, seis mil setecentos e vinte.
\par 68 Alguns dos cabeças de famílias, vindo à Casa do SENHOR, a qual está em Jerusalém, deram voluntárias ofertas para a Casa de Deus, para a restaurarem no seu lugar.
\par 69 Segundo os seus recursos, deram para o tesouro da obra, em ouro, sessenta e um mil daricos, e, em prata, cinco mil arráteis, e cem vestes sacerdotais.
\par 70 Os sacerdotes, os levitas e alguns do povo, tanto os cantores como os porteiros e os servidores do templo habitaram nas suas cidades, como também todo o Israel.

\chapter{3}

\par 1 Em chegando o sétimo mês, e estando os filhos de Israel já nas cidades, ajuntou-se o povo, como um só homem, em Jerusalém.
\par 2 Levantou-se Jesua, filho de Jozadaque, e seus irmãos, sacerdotes, e Zorobabel, filho de Sealtiel, e seus irmãos e edificaram o altar do Deus de Israel, para sobre ele oferecerem holocaustos, como está escrito na Lei de Moisés, homem de Deus.
\par 3 Firmaram o altar sobre as suas bases; e, ainda que estavam sob o terror dos povos de outras terras, ofereceram sobre ele holocaustos ao SENHOR, de manhã e à tarde.
\par 4 Celebraram a Festa dos Tabernáculos, como está escrito, e ofereceram holocaustos diários, segundo o número ordenado para cada dia;
\par 5 e, depois disto, o holocausto contínuo e os sacrifícios das Festas da Lua Nova e de todas as festas fixas do SENHOR, como também os dos que traziam ofertas voluntárias ao SENHOR.
\par 6 Desde o primeiro dia do sétimo mês, começaram a oferecer holocaustos ao SENHOR; porém ainda não estavam postos os fundamentos do templo do SENHOR.
\par 7 Deram, pois, o dinheiro aos pedreiros e aos carpinteiros, como também comida, bebida e azeite aos sidônios e tírios, para trazerem do Líbano madeira de cedro ao mar, para Jope, segundo a permissão que lhes tinha dado Ciro, rei da Pérsia.
\par 8 No segundo ano da sua vinda à Casa de Deus, em Jerusalém, no segundo mês, Zorobabel, filho de Sealtiel, e Jesua, filho de Jozadaque, e os outros seus irmãos, sacerdotes e levitas, e todos os que vieram do cativeiro a Jerusalém começaram a obra da Casa do SENHOR e constituíram levitas da idade de vinte anos para cima, para a superintenderem.
\par 9 Então, se apresentaram Jesua com seus filhos e seus irmãos, Cadmiel e seus filhos, os filhos de Judá, para juntamente vigiarem os que faziam a obra na Casa de Deus, bem como os filhos de Henadade, seus filhos e seus irmãos, os levitas.
\par 10 Quando os edificadores lançaram os alicerces do templo do SENHOR, apresentaram-se os sacerdotes, paramentados e com trombetas, e os levitas, filhos de Asafe, com címbalos, para louvarem o SENHOR, segundo as determinações de Davi, rei de Israel.
\par 11 Cantavam alternadamente, louvando e rendendo graças ao SENHOR, com estas palavras: Ele é bom, porque a sua misericórdia dura para sempre sobre Israel. E todo o povo jubilou com altas vozes, louvando ao SENHOR por se terem lançado os alicerces da sua casa.
\par 12 Porém muitos dos sacerdotes, e levitas, e cabeças de famílias, já idosos, que viram a primeira casa, choraram em alta voz quando à sua vista foram lançados os alicerces desta casa; muitos, no entanto, levantaram as vozes com gritos de alegria.
\par 13 De maneira que não se podiam discernir as vozes de alegria das vozes do choro do povo; pois o povo jubilava com tão grandes gritos, que as vozes se ouviam de mui longe.

\chapter{4}

\par 1 Ouvindo os adversários de Judá e Benjamim que os que voltaram do cativeiro edificavam o templo ao SENHOR, Deus de Israel,
\par 2 chegaram-se a Zorobabel e aos cabeças de famílias e lhes disseram: Deixai-nos edificar convosco, porque, como vós, buscaremos a vosso Deus; como também já lhe sacrificamos desde os dias de Esar-Hadom, rei da Assíria, que nos fez subir para aqui.
\par 3 Porém Zorobabel, Jesua e os outros cabeças de famílias lhes responderam: Nada tendes conosco na edificação da casa a nosso Deus; nós mesmos, sozinhos, a edificaremos ao SENHOR, Deus de Israel, como nos ordenou Ciro, rei da Pérsia.
\par 4 Então, as gentes da terra desanimaram o povo de Judá, inquietando-o no edificar;
\par 5 alugaram contra eles conselheiros para frustrarem o seu plano, todos os dias de Ciro, rei da Pérsia, até ao reinado de Dario, rei da Pérsia.
\par 6 No princípio do reinado de Assuero, escreveram uma acusação contra os habitantes de Judá e de Jerusalém.
\par 7 E, nos dias de Artaxerxes, rei da Pérsia, Bislão, Mitredate, Tabeel e os outros seus companheiros lhe escreveram; a carta estava escrita em caracteres aramaicos e na língua siríaca.
\par 8 Reum, o comandante, e Sinsai, o escrivão, escreveram contra Jerusalém uma carta ao rei Artaxerxes.
\par 9 Escreveu Reum, o comandante, e Sinsai, o escrivão, os outros seus companheiros: dinaítas, afarsaquitas, tarpelitas, afarsitas, arquevitas, babilônios, susanquitas, deavitas, elamitas
\par 10 e outros povos, que o grande e afamado Osnapar transportou e que fez habitar na cidade de Samaria, e os outros aquém do Eufrates.
\par 11 Eis o teor da carta endereçada ao rei Artaxerxes:
\par 12 Teus servos, os homens daquém do Eufrates e em tal tempo. Seja do conhecimento do rei que os judeus que subiram de ti vieram a nós a Jerusalém. Eles estão reedificando aquela rebelde e malvada cidade e vão restaurando os seus muros e reparando os seus fundamentos.
\par 13 Saiba ainda o rei que, se aquela cidade se reedificar, e os muros se restaurarem, eles não pagarão os direitos, os impostos e os pedágios e assim causarão prejuízos ao rei.
\par 14 Agora, pois, como somos assalariados do rei e não nos convém ver a desonra dele, por isso, mandamos dar-lhe aviso,
\par 15 a fim de que se busque no Livro das Crônicas de seus pais, e nele achará o rei e saberá que aquela cidade foi rebelde e danosa aos reis e às províncias e que nela tem havido rebeliões, desde tempos antigos; pelo que foi a cidade destruída.
\par 16 Nós, pois, fazemos notório ao rei que, se aquela cidade se reedificar, e os seus muros se restaurarem, sucederá que não terá a posse das terras deste lado do Eufrates.
\par 17 Então, respondeu o rei: A Reum, o comandante, a Sinsai, o escrivão, e a seus companheiros que habitam em Samaria, como aos restantes que estão além do Eufrates: Paz!
\par 18 A carta que nos enviastes foi distintamente lida na minha presença.
\par 19 Ordenando-o eu, buscaram e acharam que, de tempos antigos, aquela cidade se levantou contra os reis, e nela se têm feito rebeliões e motins.
\par 20 Também houve reis poderosos sobre Jerusalém, que dalém do Eufrates dominaram em todo lugar, e se lhes pagaram direitos, impostos e pedágios.
\par 21 Agora, pois, dai ordem a fim de que aqueles homens parem o trabalho e não se edifique aquela cidade, a não ser com autorização minha.
\par 22 Guardai-vos, não sejais remissos nestas coisas. Por que há de crescer o dano em prejuízo dos reis?
\par 23 Depois de lida a cópia da carta do rei Artaxerxes perante Reum, Sinsai, o escrivão, e seus companheiros, foram eles apressadamente a Jerusalém, aos judeus, e, de mão armada, os forçaram a parar com a obra.
\par 24 Cessou, pois, a obra da Casa de Deus, a qual estava em Jerusalém; e isso até ao segundo ano do reinado de Dario, rei da Pérsia.

\chapter{5}

\par 1 Ora, os profetas Ageu e Zacarias, filho de Ido, profetizaram aos judeus que estavam em Judá e em Jerusalém, em nome do Deus de Israel, cujo Espírito estava com eles.
\par 2 Então, se dispuseram Zorobabel, filho de Sealtiel, e Jesua, filho de Jozadaque, e começaram a edificar a Casa de Deus, a qual está em Jerusalém; e, com eles, os referidos profetas de Deus, que os ajudavam.
\par 3 Nesse tempo, veio a eles Tatenai, governador daquém do Eufrates, e Setar-Bozenai, e seus companheiros e assim lhes perguntaram: Quem vos deu ordem para reedificardes esta casa e restaurardes este muro?
\par 4 Perguntaram-lhes mais: E quais são os nomes dos homens que constroem este edifício?
\par 5 Porém os olhos de Deus estavam sobre os anciãos dos judeus, de maneira que não foram obrigados a parar, até que o assunto chegasse a Dario, e viesse resposta por carta sobre isso.
\par 6 Eis a cópia da carta que Tatenai, o governador daquém do Eufrates, com Setar-Bozenai e os seus companheiros, os afarsaquitas, que estavam deste lado do rio, enviaram ao rei Dario,
\par 7 na qual lhe deram uma relação escrita do modo seguinte: Ao rei Dario, toda a paz!
\par 8 Seja notório ao rei que nós fomos à província de Judá, à casa do grande Deus, a qual se edifica com grandes pedras; a madeira se está pondo nas paredes, e a obra se vai fazendo com diligência e se adianta nas suas mãos.
\par 9 Perguntamos aos anciãos e assim lhes dissemos: Quem vos deu ordem para reedificardes esta casa e restaurardes este muro?
\par 10 Demais disto, lhes perguntamos também pelo seu nome, para tos declararmos, para que te pudéssemos escrever os nomes dos homens que são entre eles os chefes.
\par 11 Esta foi a resposta que nos deram: Nós somos servos do Deus dos céus e da terra e reedificamos a casa que há muitos anos fora construída, a qual um grande rei de Israel edificou e a terminou.
\par 12 Mas, depois que nossos pais provocaram à ira o Deus dos céus, ele os entregou nas mãos de Nabucodonosor, rei da Babilônia, o caldeu, o qual destruiu esta casa e transportou o povo para a Babilônia.
\par 13 Porém Ciro, rei da Babilônia, no seu primeiro ano, deu ordem para que esta Casa de Deus se edificasse.
\par 14 Também os utensílios de ouro e de prata, da Casa de Deus, que Nabucodonosor levara do templo que estava em Jerusalém e os meteu no templo de Babilônia, o rei Ciro os tirou de lá, e foram dados a um homem cujo nome era Sesbazar, a quem nomeara governador
\par 15 e lhe disse: Toma estes utensílios, e vai, e leva-os ao templo de Jerusalém, e faze reedificar a Casa de Deus, no seu lugar.
\par 16 Então, veio o dito Sesbazar e lançou os fundamentos da Casa de Deus, a qual está em Jerusalém; e, daí para cá, se está edificando e ainda não está acabada.
\par 17 Agora, pois, se parece bem ao rei, que se busque nos arquivos reais, na Babilônia, se é verdade haver uma ordem do rei Ciro para edificar esta Casa de Deus, em Jerusalém; e sobre isto nos faça o rei saber a sua vontade.

\chapter{6}

\par 1 Então, o rei Dario deu ordem, e uma busca se fez nos arquivos reais da Babilônia, onde se guardavam os documentos.
\par 2 Em Acmetá, na fortaleza que está na província da Média, se achou um rolo, e nele estava escrito um memorial que dizia assim:
\par 3 O rei Ciro, no seu primeiro ano, baixou o seguinte decreto: Com respeito à Casa de Deus, em Jerusalém, deve ela edificar-se para ser um lugar em que se ofereçam sacrifícios; seus fundamentos serão firmes, a sua altura, de sessenta côvados, e a sua largura, de sessenta côvados, com três carreiras de grandes pedras e uma de madeira nova.
\par 4 A despesa se fará da casa do rei.
\par 5 Demais disto, os utensílios de ouro e de prata, da Casa de Deus, que Nabucodonosor tirara do templo que estava em Jerusalém, levando-os para a Babilônia, serão devolvidos para o templo que está em Jerusalém, cada utensílio para o seu lugar; serão recolocados na Casa de Deus.
\par 6 Agora, pois, Tatenai, governador dalém do Eufrates, Setar-Bozenai e seus companheiros, os afarsaquitas, que estais para além do rio, retirai-vos para longe dali.
\par 7 Não interrompais a obra desta Casa de Deus, para que o governador dos judeus e os seus anciãos reedifiquem a Casa de Deus no seu lugar.
\par 8 Também por mim se decreta o que haveis de fazer a estes anciãos dos judeus, para que reedifiquem esta Casa de Deus, a saber, que da tesouraria real, isto é, dos tributos dalém do rio, se pague, pontualmente, a despesa a estes homens, para que não se interrompa a obra.
\par 9 Também se lhes dê, dia após dia, sem falta, aquilo de que houverem mister: novilhos, carneiros e cordeiros, para holocausto ao Deus dos céus; trigo, sal, vinho e azeite, segundo a determinação dos sacerdotes que estão em Jerusalém;
\par 10 para que ofereçam sacrifícios de aroma agradável ao Deus dos céus e orem pela vida do rei e de seus filhos.
\par 11 Também por mim se decreta que todo homem que alterar este decreto, uma viga se arrancará da sua casa, e que seja ele levantado e pendurado nela; e que da sua casa se faça um monturo.
\par 12 O Deus, pois, que fez habitar ali o seu nome derribe a todos os reis e povos que estenderem a mão para alterar o decreto e para destruir esta Casa de Deus, a qual está em Jerusalém. Eu, Dario, baixei o decreto; que se execute com toda a pontualidade.
\par 13 Então, Tatenai, o governador daquém do Eufrates, Setar-Bozenai e os seus companheiros assim o fizeram pontualmente, segundo decretara o rei Dario.
\par 14 Os anciãos dos judeus iam edificando e prosperando em virtude do que profetizaram os profetas Ageu e Zacarias, filho de Ido. Edificaram a casa e a terminaram segundo o mandado do Deus de Israel e segundo o decreto de Ciro, de Dario e de Artaxerxes, rei da Pérsia.
\par 15 Acabou-se esta casa no dia terceiro do mês de adar, no sexto ano do reinado do rei Dario.
\par 16 Os filhos de Israel, os sacerdotes, os levitas e o restante dos exilados celebraram com regozijo a dedicação desta Casa de Deus.
\par 17 Para a dedicação desta Casa de Deus ofereceram cem novilhos, duzentos carneiros, quatrocentos cordeiros e doze cabritos, para oferta pelo pecado de todo o Israel, segundo o número das tribos de Israel.
\par 18 Estabeleceram os sacerdotes nos seus turnos e os levitas nas suas divisões, para o serviço de Deus em Jerusalém, segundo está escrito no Livro de Moisés.
\par 19 Os que vieram do cativeiro celebraram a Páscoa no dia catorze do primeiro mês;
\par 20 porque os sacerdotes e os levitas se tinham purificado como se fossem um só homem, e todos estavam limpos; mataram o cordeiro da Páscoa para todos os que vieram do cativeiro, para os sacerdotes, seus irmãos, e para si mesmos.
\par 21 Assim, comeram a Páscoa os filhos de Israel que tinham voltado do exílio e todos os que, unindo-se a eles, se haviam separado da imundícia dos gentios da terra, para buscarem o SENHOR, Deus de Israel.
\par 22 Celebraram a Festa dos Pães Asmos por sete dias, com regozijo, porque o SENHOR os tinha alegrado, mudando o coração do rei da Assíria a favor deles, para lhes fortalecer as mãos na obra da Casa de Deus, o Deus de Israel.

\chapter{7}

\par 1 Passadas estas coisas, no reinado de Artaxerxes, rei da Pérsia, Esdras, filho de Seraías, filho de Azarias, filho de Hilquias,
\par 2 filho de Salum, filho de Zadoque, filho de Aitube,
\par 3 filho de Amarias, filho de Azarias, filho de Meraiote,
\par 4 filho de Zeraías, filho de Uzi, filho de Buqui,
\par 5 filho de Abisua, filho de Finéias, filho de Eleazar, filho de Arão, o sumo sacerdote, este Esdras subiu da Babilônia.
\par 6 Ele era escriba versado na Lei de Moisés, dada pelo SENHOR, Deus de Israel; e, segundo a boa mão do SENHOR, seu Deus, que estava sobre ele, o rei lhe concedeu tudo quanto lhe pedira.
\par 7 Também subiram a Jerusalém alguns dos filhos de Israel, dos sacerdotes, dos levitas, dos cantores, dos porteiros e dos servidores do templo, no sétimo ano do rei Artaxerxes.
\par 8 Esdras chegou a Jerusalém no quinto mês, no sétimo ano deste rei;
\par 9 pois, no primeiro dia do primeiro mês, partiu da Babilônia e, no primeiro dia do quinto mês, chegou a Jerusalém, segundo a boa mão do seu Deus sobre ele.
\par 10 Porque Esdras tinha disposto o coração para buscar a Lei do SENHOR, e para a cumprir, e para ensinar em Israel os seus estatutos e os seus juízos.
\par 11 Esta é a cópia da carta que o rei Artaxerxes deu ao sacerdote Esdras, o escriba das palavras, dos mandamentos e dos estatutos do SENHOR sobre Israel:
\par 12 Artaxerxes, rei dos reis, ao sacerdote Esdras, escriba da Lei do Deus do céu: Paz perfeita!
\par 13 Por mim se decreta que, no meu reino, todo aquele do povo de Israel e dos seus sacerdotes e levitas que quiser ir contigo a Jerusalém, vá.
\par 14 Porquanto és mandado da parte do rei e dos seus sete conselheiros para fazeres inquirição a respeito de Judá e de Jerusalém, segundo a Lei do teu Deus, a qual está na tua mão;
\par 15 e para levares a prata e o ouro que o rei e os seus conselheiros, espontaneamente, ofereceram ao Deus de Israel, cuja habitação está em Jerusalém,
\par 16 bem assim a prata e o ouro que achares em toda a província da Babilônia, com as ofertas voluntárias do povo e dos sacerdotes, oferecidas, espontaneamente, para a casa de seu Deus, a qual está em Jerusalém.
\par 17 Portanto, diligentemente comprarás com este dinheiro novilhos, e carneiros, e cordeiros, e as suas ofertas de manjares, e as suas libações e as oferecerás sobre o altar da casa de teu Deus, a qual está em Jerusalém.
\par 18 Também o que a ti e a teus irmãos bem parecer fazerdes do resto da prata e do ouro, fazei-o, segundo a vontade do vosso Deus.
\par 19 E os utensílios que te foram dados para o serviço da casa de teu Deus, restitui-os perante o Deus de Jerusalém.
\par 20 E tudo mais que for necessário para a casa de teu Deus, que te convenha dar, dá-lo-ás da casa dos tesouros do rei.
\par 21 Eu mesmo, o rei Artaxerxes, decreto a todos os tesoureiros que estão dalém do Eufrates: tudo quanto vos pedir o sacerdote Esdras, escriba da Lei do Deus do céu, pontualmente se lhe faça;
\par 22 até cem talentos de prata, até cem coros de trigo, até cem batos de vinho, até cem batos de azeite e sal à vontade.
\par 23 Tudo quanto se ordenar, segundo o mandado do Deus do céu, exatamente se faça para a casa do Deus do céu; pois para que haveria grande ira sobre o reino do rei e de seus filhos?
\par 24 Também vos fazemos saber, acerca de todos os sacerdotes e levitas, cantores, porteiros, de todos os que servem nesta Casa de Deus, que não será lícito impor-lhes nem direitos, nem impostos, nem pedágios.
\par 25 Tu, Esdras, segundo a sabedoria do teu Deus, que possuis, nomeia magistrados e juízes que julguem a todo o povo que está dalém do Eufrates, a todos os que sabem as leis de teu Deus, e ao que não as sabe, que lhas façam saber.
\par 26 Todo aquele que não observar a lei do teu Deus e a lei do rei, seja condenado ou à morte, ou ao desterro, ou à confiscação de bens, ou à prisão.
\par 27 Bendito seja o SENHOR, Deus de nossos pais, que deste modo moveu o coração do rei para ornar a Casa do SENHOR, a qual está em Jerusalém;
\par 28 e que estendeu para mim a sua misericórdia perante o rei, os seus conselheiros e todos os seus príncipes poderosos. Assim, me animei, segundo a boa mão do SENHOR, meu Deus, sobre mim, e ajuntei de Israel alguns chefes para subirem comigo.

\chapter{8}

\par 1 São estes os cabeças de famílias, com as suas genealogias, os que subiram comigo da Babilônia, no reinado do rei Artaxerxes:
\par 2 dos filhos de Finéias, Gérson; dos filhos de Itamar, Daniel; dos filhos de Davi, Hatus;
\par 3 dos filhos de Secanias, dos filhos de Parós, Zacarias, e, com ele, foram registrados cento e cinqüenta homens.
\par 4 Dos filhos de Paate-Moabe, Elioenai, filho de Zeraías, e, com ele, duzentos homens.
\par 5 Dos filhos de Secanias, o filho de Jaaziel, e, com ele, trezentos homens.
\par 6 Dos filhos de Adim, Ebede, filho de Jônatas, e, com ele, cinqüenta homens.
\par 7 Dos filhos de Elão, Jesaías, filho de Atalias, e, com ele, setenta homens.
\par 8 Dos filhos de Sefatias, Zebadias, filho de Micael, e, com ele, oitenta homens.
\par 9 Dos filhos de Joabe, Obadias, filho de Jeiel, e, com ele, duzentos e dezoito homens.
\par 10 Dos filhos de Bani, Selomite, filho de Josifias, e, com ele, cento e sessenta homens.
\par 11 Dos filhos de Bebai, Zacarias, o filho de Bebai, e, com ele, vinte e oito homens.
\par 12 Dos filhos de Azgade, Joanã, o filho de Hacatã, e, com ele, cento e dez homens.
\par 13 Dos filhos de Adonicão, últimos a chegar, seus nomes eram estes: Elifelete, Jeiel e Semaías, e, com eles, sessenta homens.
\par 14 Dos filhos de Bigvai, Utai e Zabude, e, com eles, setenta homens.
\par 15 Ajuntei-os perto do rio que corre para Aava, onde ficamos acampados três dias. Passando revista ao povo e aos sacerdotes e não tendo achado nenhum dos filhos de Levi,
\par 16 enviei Eliézer, Ariel, Semaías, Elnatã, Jaribe, Elnatã, Natã, Zacarias e Mesulão, os chefes, como também a Joiaribe e a Elnatã, que eram sábios.
\par 17 Enviei-os a Ido, chefe em Casifia, e lhes dei expressamente as palavras que deveriam dizer a Ido e aos servidores do templo, seus irmãos, em Casifia, para nos trazerem ministros para a casa do nosso Deus.
\par 18 Trouxeram-nos, segundo a boa mão de Deus sobre nós, um homem sábio, dos filhos de Mali, filho de Levi, filho de Israel, a saber, Serebias, com os seus filhos e irmãos, dezoito;
\par 19 e a Hasabias e, com ele, Jesaías, dos filhos de Merari, com seus irmãos e os filhos deles, vinte;
\par 20 e dos servidores do templo, que Davi e os príncipes deram para o ministério dos levitas, duzentos e vinte, todos eles mencionados nominalmente.
\par 21 Então, apregoei ali um jejum junto ao rio Aava, para nos humilharmos perante o nosso Deus, para lhe pedirmos jornada feliz para nós, para nossos filhos e para tudo o que era nosso.
\par 22 Porque tive vergonha de pedir ao rei exército e cavaleiros para nos defenderem do inimigo no caminho, porquanto já lhe havíamos dito: A boa mão do nosso Deus é sobre todos os que o buscam, para o bem deles; mas a sua força e a sua ira, contra todos os que o abandonam.
\par 23 Nós, pois, jejuamos e pedimos isto ao nosso Deus, e ele nos atendeu.
\par 24 Então, separei doze dos principais, isto é, Serebias, Hasabias e dez dos seus irmãos.
\par 25 Pesei-lhes a prata, e o ouro, e os utensílios que eram a contribuição para a casa de nosso Deus, a qual ofereceram o rei, os seus conselheiros, os seus príncipes e todo o Israel que se achou ali.
\par 26 Entreguei-lhes nas mãos seiscentos e cinqüenta talentos de prata e, em objetos de prata, cem talentos, além de cem talentos de ouro;
\par 27 e vinte taças de ouro de mil daricos e dois objetos de lustroso e fino bronze, tão precioso como ouro.
\par 28 Disse-lhes: Vós sois santos ao SENHOR, e santos são estes objetos, como também esta prata e este ouro, oferta voluntária ao SENHOR, Deus de vossos pais.
\par 29 Vigiai-os e guardai-os até que os peseis na presença dos principais sacerdotes, e dos levitas, e dos cabeças de famílias de Israel, em Jerusalém, nas câmaras da Casa do SENHOR.
\par 30 Então, receberam os sacerdotes e os levitas o peso da prata, do ouro e dos objetos, para trazerem a Jerusalém, à casa de nosso Deus.
\par 31 Partimos do rio Aava, no dia doze do primeiro mês, a fim de irmos para Jerusalém; e a boa mão do nosso Deus estava sobre nós e livrou-nos das mãos dos inimigos e dos que nos armavam ciladas pelo caminho.
\par 32 Chegamos a Jerusalém e repousamos ali três dias.
\par 33 No quarto dia, pesamos, na casa do nosso Deus, a prata, o ouro, os objetos e os entregamos a Meremote, filho do sacerdote Urias; com ele estava Eleazar, filho de Finéias, e, com eles, Jozabade, filho de Jesua, e Noadias, filho de Binui, levitas;
\par 34 tudo foi contado e pesado, e o peso total, imediatamente registrado.
\par 35 Os exilados que vieram do cativeiro ofereceram holocaustos ao Deus de Israel, doze novilhos por todo o Israel, noventa e seis carneiros, setenta e sete cordeiros e, como oferta pelo pecado, doze bodes; tudo em holocausto ao SENHOR.
\par 36 Então, deram as ordens do rei aos seus sátrapas e aos governadores deste lado do Eufrates; e estes ajudaram o povo na reconstrução da Casa de Deus.

\chapter{9}

\par 1 Acabadas, pois, estas coisas, vieram ter comigo os príncipes, dizendo: O povo de Israel, e os sacerdotes, e os levitas não se separaram dos povos de outras terras com as suas abominações, isto é, dos cananeus, dos heteus, dos ferezeus, dos jebuseus, dos amonitas, dos moabitas, dos egípcios e dos amorreus,
\par 2 pois tomaram das suas filhas para si e para seus filhos, e, assim, se misturou a linhagem santa com os povos dessas terras, e até os príncipes e magistrados foram os primeiros nesta transgressão.
\par 3 Ouvindo eu tal coisa, rasguei as minhas vestes e o meu manto, e arranquei os cabelos da cabeça e da barba, e me assentei atônito.
\par 4 Então, se ajuntaram a mim todos os que tremiam das palavras do Deus de Israel, por causa da transgressão dos do cativeiro; porém eu permaneci assentado atônito até ao sacrifício da tarde.
\par 5 Na hora do sacrifício da tarde, levantei-me da minha humilhação, com as vestes e o manto já rasgados, me pus de joelhos, estendi as mãos para o SENHOR, meu Deus,
\par 6 e disse: Meu Deus! Estou confuso e envergonhado, para levantar a ti a face, meu Deus, porque as nossas iniqüidades se multiplicaram sobre a nossa cabeça, e a nossa culpa cresceu até aos céus.
\par 7 Desde os dias de nossos pais até hoje, estamos em grande culpa e, por causa das nossas iniqüidades, fomos entregues, nós, os nossos reis e os nossos sacerdotes, nas mãos dos reis de outras terras e sujeitos à espada, ao cativeiro, ao roubo e à ignomínia, como hoje se vê.
\par 8 Agora, por breve momento, se nos manifestou a graça da parte do SENHOR, nosso Deus, para nos deixar alguns que escapem e para dar-nos estabilidade no seu santo lugar; para nos alumiar os olhos, ó Deus nosso, e para nos dar um pouco de vida na nossa servidão;
\par 9 porque somos servos, porém, na nossa servidão, não nos desamparou o nosso Deus; antes, estendeu sobre nós a sua misericórdia, e achamos favor perante os reis da Pérsia, para nos reviver, para levantar a casa do nosso Deus, para restaurar as suas ruínas e para que nos desse um muro de segurança em Judá e em Jerusalém.
\par 10 Agora, ó nosso Deus, que diremos depois disto? Pois deixamos os teus mandamentos,
\par 11 que ordenaste por intermédio dos teus servos, os profetas, dizendo: A terra em que entrais para a possuir é terra imunda pela imundícia dos seus povos, pelas abominações com que, na sua corrupção, a encheram de uma extremidade à outra.
\par 12 Por isso, não dareis as vossas filhas a seus filhos, e suas filhas não tomareis para os vossos filhos, e jamais procurareis a paz e o bem desses povos; para que sejais fortes, e comais o melhor da terra, e a deixeis por herança a vossos filhos, para sempre.
\par 13 Depois de tudo o que nos tem sucedido por causa das nossas más obras e da nossa grande culpa, e vendo ainda que tu, ó nosso Deus, nos tens castigado menos do que merecem as nossas iniqüidades e ainda nos deste este restante que escapou,
\par 14 tornaremos a violar os teus mandamentos e a aparentar-nos com os povos destas abominações? Não te indignarias tu, assim, contra nós, até de todo nos consumires, até não haver restante nem alguém que escapasse?
\par 15 Ah! SENHOR, Deus de Israel, justo és, pois somos os restantes que escaparam, como hoje se vê. Eis que estamos diante de ti na nossa culpa, porque ninguém há que possa estar na tua presença por causa disto.

\chapter{10}

\par 1 Enquanto Esdras orava e fazia confissão, chorando prostrado diante da Casa de Deus, ajuntou-se a ele de Israel mui grande congregação de homens, de mulheres e de crianças; pois o povo chorava com grande choro.
\par 2 Então, Secanias, filho de Jeiel, um dos filhos de Elão, tomou a palavra e disse a Esdras: Nós temos transgredido contra o nosso Deus, casando com mulheres estrangeiras, dos povos de outras terras, mas, no tocante a isto, ainda há esperança para Israel.
\par 3 Agora, pois, façamos aliança com o nosso Deus, de que despediremos todas as mulheres e os seus filhos, segundo o conselho do Senhor e o dos que tremem ao mandado do nosso Deus; e faça-se segundo a Lei.
\par 4 Levanta-te, pois esta coisa é de tua incumbência, e nós seremos contigo; sê forte e age.
\par 5 Então, Esdras se levantou e ajuramentou os principais sacerdotes, os levitas e todo o Israel, de que fariam segundo esta palavra. E eles juraram.
\par 6 Esdras se retirou de diante da Casa de Deus, e entrou na câmara de Joanã, filho de Eliasibe, e lá não comeu pão, nem bebeu água, porque pranteava por causa da transgressão dos que tinham voltado do exílio.
\par 7 Fez-se passar pregão por Judá e Jerusalém a todos os que vieram do exílio, que deviam ajuntar-se em Jerusalém;
\par 8 e que, se alguém, em três dias, não viesse, segundo o conselho dos príncipes e dos anciãos, todos os seus bens seriam totalmente destruídos, e ele mesmo separado da congregação dos que voltaram do exílio.
\par 9 Então, todos os homens de Judá e Benjamim, em três dias, se ajuntaram em Jerusalém; no dia vinte do mês nono, todo o povo se assentou na praça da Casa de Deus, tremendo por causa desta coisa e por causa das grandes chuvas.
\par 10 Então, se levantou Esdras, o sacerdote, e lhes disse: Vós transgredistes casando-vos com mulheres estrangeiras, aumentando a culpa de Israel.
\par 11 Agora, pois, fazei confissão ao SENHOR, Deus de vossos pais, e fazei o que é do seu agrado; separai-vos dos povos de outras terras e das mulheres estrangeiras.
\par 12 Respondeu toda a congregação e disse em altas vozes: Assim seja; segundo as tuas palavras, assim nos convém fazer.
\par 13 Porém o povo é muito, e, sendo tempo de grandes chuvas, não podemos estar aqui de fora; e não é isto obra de um dia ou dois, pois somos muitos os que transgredimos nesta coisa.
\par 14 Ora, que os nossos príncipes decidam por toda a congregação, e que venham a eles em tempos determinados todos os que em nossas cidades casaram com mulheres estrangeiras, e com estes os anciãos de cada cidade, e os seus juízes, até que desviemos de nós o brasume da ira do nosso Deus, por esta coisa.
\par 15 No entanto, Jônatas, filho de Asael, e Jazeías, filho de Ticvá, se opuseram a esta coisa; e Mesulão e Sabetai, levita, os apoiaram.
\par 16 Assim o fizeram os que voltaram do exílio; então, Esdras, o sacerdote, elegeu nominalmente os homens cabeças de famílias, segundo a casa de seus pais, que se assentaram no dia primeiro do décimo mês, para inquirir nesta coisa;
\par 17 e o concluíram no dia primeiro do primeiro mês, a respeito de todos os homens que casaram com mulheres estrangeiras.
\par 18 Acharam-se dentre os filhos dos sacerdotes estes, que casaram com mulheres estrangeiras: dos filhos de Jesua, filho de Jozadaque, e de seus irmãos: Maaséias, Eliézer, Jaribe e Gedalias.
\par 19 Com um aperto de mão, prometeram despedir suas mulheres e, por serem culpados, ofereceram um carneiro do rebanho pela sua culpa.
\par 20 Dos filhos de Imer: Hanani e Zebadias.
\par 21 Dos filhos de Harim: Maaséias, Elias, Semaías, Jeiel e Uzias.
\par 22 Dos filhos de Pasur: Elioenai, Maaséias, Ismael, Natanael, Jozabade e Elasa.
\par 23 Dos levitas: Jozabade e Simei, Quelaías (este é Quelita), Petaías, Judá e Eliézer.
\par 24 Dos cantores: Eliasibe; dos porteiros: Salum, Telém e Uri.
\par 25 E de Israel: dos filhos de Parós: Ramias, Jezias, Malquias, Miamim, Eleazar, Malquias e Benaia.
\par 26 Dos filhos de Elão: Matanias, Zacarias, Jeiel, Abdi, Jerimote e Elias.
\par 27 Dos filhos de Zatu: Elioenai, Eliasibe, Matanias, Jerimote, Zabade e Aziza.
\par 28 Dos filhos de Bebai: Joanã, Hananias, Zabai e Atlai.
\par 29 Dos filhos de Bani: Mesulão, Maluque, Adaías, Jasube, Seal e Jerimote.
\par 30 Dos filhos de Paate-Moabe: Adna, Quelal, Benaia, Maaséias, Matanias, Bezalel, Binui e Manassés.
\par 31 Dos filhos de Harim: Eliézer, Issias, Malquias, Semaías, Simeão,
\par 32 Benjamim, Maluque e Semarias.
\par 33 Dos filhos de Hasum: Matenai, Matatá, Zabade, Elifelete, Jeremai, Manassés e Simei.
\par 34 Dos filhos de Bani: Maadai, Anrão, Uel,
\par 35 Benaia, Bedias, Queluí,
\par 36 Vanias, Meremote, Eliasibe,
\par 37 Matanias, Matenai, Jaasai,
\par 38 Bani, Binui, Simei,
\par 39 Selemias, Natã, Adaías,
\par 40 Macnadbai, Sasai, Sarai,
\par 41 Azarel, Selemias, Semarias,
\par 42 Salum, Amarias e José.
\par 43 Dos filhos de Nebo: Jeiel, Matitias, Zabade, Zebina, Jadai, Joel e Benaia.
\par 44 Todos estes haviam tomado mulheres estrangeiras, alguns dos quais tinham filhos destas mulheres.


\end{document}