\begin{document}

\title{I João}


\chapter{1}

\par 1 O que era desde o princípio, o que temos ouvido, o que temos visto com os nossos próprios olhos, o que contemplamos, e as nossas mãos apalparam, com respeito ao Verbo da vida
\par 2 (e a vida se manifestou, e nós a temos visto, e dela damos testemunho, e vo-la anunciamos, a vida eterna, a qual estava com o Pai e nos foi manifestada),
\par 3 o que temos visto e ouvido anunciamos também a vós outros, para que vós, igualmente, mantenhais comunhão conosco. Ora, a nossa comunhão é com o Pai e com seu Filho, Jesus Cristo.
\par 4 Estas coisas, pois, vos escrevemos para que a nossa alegria seja completa.
\par 5 Ora, a mensagem que, da parte dele, temos ouvido e vos anunciamos é esta: que Deus é luz, e não há nele treva nenhuma.
\par 6 Se dissermos que mantemos comunhão com ele e andarmos nas trevas, mentimos e não praticamos a verdade.
\par 7 Se, porém, andarmos na luz, como ele está na luz, mantemos comunhão uns com os outros, e o sangue de Jesus, seu Filho, nos purifica de todo pecado.
\par 8 Se dissermos que não temos pecado nenhum, a nós mesmos nos enganamos, e a verdade não está em nós.
\par 9 Se confessarmos os nossos pecados, ele é fiel e justo para nos perdoar os pecados e nos purificar de toda injustiça.
\par 10 Se dissermos que não temos cometido pecado, fazemo-lo mentiroso, e a sua palavra não está em nós.

\chapter{2}

\par 1 Filhinhos meus, estas coisas vos escrevo para que não pequeis. Se, todavia, alguém pecar, temos Advogado junto ao Pai, Jesus Cristo, o Justo;
\par 2 e ele é a propiciação pelos nossos pecados e não somente pelos nossos próprios, mas ainda pelos do mundo inteiro.
\par 3 Ora, sabemos que o temos conhecido por isto: se guardamos os seus mandamentos.
\par 4 Aquele que diz: Eu o conheço e não guarda os seus mandamentos é mentiroso, e nele não está a verdade.
\par 5 Aquele, entretanto, que guarda a sua palavra, nele, verdadeiramente, tem sido aperfeiçoado o amor de Deus. Nisto sabemos que estamos nele:
\par 6 aquele que diz que permanece nele, esse deve também andar assim como ele andou.
\par 7 Amados, não vos escrevo mandamento novo, senão mandamento antigo, o qual, desde o princípio, tivestes. Esse mandamento antigo é a palavra que ouvistes.
\par 8 Todavia, vos escrevo novo mandamento, aquilo que é verdadeiro nele e em vós, porque as trevas se vão dissipando, e a verdadeira luz já brilha.
\par 9 Aquele que diz estar na luz e odeia a seu irmão, até agora, está nas trevas.
\par 10 Aquele que ama a seu irmão permanece na luz, e nele não há nenhum tropeço.
\par 11 Aquele, porém, que odeia a seu irmão está nas trevas, e anda nas trevas, e não sabe para onde vai, porque as trevas lhe cegaram os olhos.
\par 12 Filhinhos, eu vos escrevo, porque os vossos pecados são perdoados, por causa do seu nome.
\par 13 Pais, eu vos escrevo, porque conheceis aquele que existe desde o princípio. Jovens, eu vos escrevo, porque tendes vencido o Maligno.
\par 14 Filhinhos, eu vos escrevi, porque conheceis o Pai. Pais, eu vos escrevi, porque conheceis aquele que existe desde o princípio. Jovens, eu vos escrevi, porque sois fortes, e a palavra de Deus permanece em vós, e tendes vencido o Maligno.
\par 15 Não ameis o mundo nem as coisas que há no mundo. Se alguém amar o mundo, o amor do Pai não está nele;
\par 16 porque tudo que há no mundo, a concupiscência da carne, a concupiscência dos olhos e a soberba da vida, não procede do Pai, mas procede do mundo.
\par 17 Ora, o mundo passa, bem como a sua concupiscência; aquele, porém, que faz a vontade de Deus permanece eternamente.
\par 18 Filhinhos, já é a última hora; e, como ouvistes que vem o anticristo, também, agora, muitos anticristos têm surgido; pelo que conhecemos que é a última hora.
\par 19 Eles saíram de nosso meio; entretanto, não eram dos nossos; porque, se tivessem sido dos nossos, teriam permanecido conosco; todavia, eles se foram para que ficasse manifesto que nenhum deles é dos nossos.
\par 20 E vós possuís unção que vem do Santo e todos tendes conhecimento.
\par 21 Não vos escrevi porque não saibais a verdade; antes, porque a sabeis, e porque mentira alguma jamais procede da verdade.
\par 22 Quem é o mentiroso, senão aquele que nega que Jesus é o Cristo? Este é o anticristo, o que nega o Pai e o Filho.
\par 23 Todo aquele que nega o Filho, esse não tem o Pai; aquele que confessa o Filho tem igualmente o Pai.
\par 24 Permaneça em vós o que ouvistes desde o princípio. Se em vós permanecer o que desde o princípio ouvistes, também permanecereis vós no Filho e no Pai.
\par 25 E esta é a promessa que ele mesmo nos fez, a vida eterna.
\par 26 Isto que vos acabo de escrever é acerca dos que vos procuram enganar.
\par 27 Quanto a vós outros, a unção que dele recebestes permanece em vós, e não tendes necessidade de que alguém vos ensine; mas, como a sua unção vos ensina a respeito de todas as coisas, e é verdadeira, e não é falsa, permanecei nele, como também ela vos ensinou.
\par 28 Filhinhos, agora, pois, permanecei nele, para que, quando ele se manifestar, tenhamos confiança e dele não nos afastemos envergonhados na sua vinda.
\par 29 Se sabeis que ele é justo, reconhecei também que todo aquele que pratica a justiça é nascido dele.

\chapter{3}

\par 1 Vede que grande amor nos tem concedido o Pai, a ponto de sermos chamados filhos de Deus; e, de fato, somos filhos de Deus. Por essa razão, o mundo não nos conhece, porquanto não o conheceu a ele mesmo.
\par 2 Amados, agora, somos filhos de Deus, e ainda não se manifestou o que haveremos de ser. Sabemos que, quando ele se manifestar, seremos semelhantes a ele, porque haveremos de vê-lo como ele é.
\par 3 E a si mesmo se purifica todo o que nele tem esta esperança, assim como ele é puro.
\par 4 Todo aquele que pratica o pecado também transgride a lei, porque o pecado é a transgressão da lei.
\par 5 Sabeis também que ele se manifestou para tirar os pecados, e nele não existe pecado.
\par 6 Todo aquele que permanece nele não vive pecando; todo aquele que vive pecando não o viu, nem o conheceu.
\par 7 Filhinhos, não vos deixeis enganar por ninguém; aquele que pratica a justiça é justo, assim como ele é justo.
\par 8 Aquele que pratica o pecado procede do diabo, porque o diabo vive pecando desde o princípio. Para isto se manifestou o Filho de Deus: para destruir as obras do diabo.
\par 9 Todo aquele que é nascido de Deus não vive na prática de pecado; pois o que permanece nele é a divina semente; ora, esse não pode viver pecando, porque é nascido de Deus.
\par 10 Nisto são manifestos os filhos de Deus e os filhos do diabo: todo aquele que não pratica justiça não procede de Deus, nem aquele que não ama a seu irmão.
\par 11 Porque a mensagem que ouvistes desde o princípio é esta: que nos amemos uns aos outros;
\par 12 não segundo Caim, que era do Maligno e assassinou a seu irmão; e por que o assassinou? Porque as suas obras eram más, e as de seu irmão, justas.
\par 13 Irmãos, não vos maravilheis se o mundo vos odeia.
\par 14 Nós sabemos que já passamos da morte para a vida, porque amamos os irmãos; aquele que não ama permanece na morte.
\par 15 Todo aquele que odeia a seu irmão é assassino; ora, vós sabeis que todo assassino não tem a vida eterna permanente em si.
\par 16 Nisto conhecemos o amor: que Cristo deu a sua vida por nós; e devemos dar nossa vida pelos irmãos.
\par 17 Ora, aquele que possuir recursos deste mundo, e vir a seu irmão padecer necessidade, e fechar-lhe o seu coração, como pode permanecer nele o amor de Deus?
\par 18 Filhinhos, não amemos de palavra, nem de língua, mas de fato e de verdade.
\par 19 E nisto conheceremos que somos da verdade, bem como, perante ele, tranqüilizaremos o nosso coração;
\par 20 pois, se o nosso coração nos acusar, certamente, Deus é maior do que o nosso coração e conhece todas as coisas.
\par 21 Amados, se o coração não nos acusar, temos confiança diante de Deus;
\par 22 e aquilo que pedimos dele recebemos, porque guardamos os seus mandamentos e fazemos diante dele o que lhe é agradável.
\par 23 Ora, o seu mandamento é este: que creiamos em o nome de seu Filho, Jesus Cristo, e nos amemos uns aos outros, segundo o mandamento que nos ordenou.
\par 24 E aquele que guarda os seus mandamentos permanece em Deus, e Deus, nele. E nisto conhecemos que ele permanece em nós, pelo Espírito que nos deu.

\chapter{4}

\par 1 Amados, não deis crédito a qualquer espírito; antes, provai os espíritos se procedem de Deus, porque muitos falsos profetas têm saído pelo mundo fora.
\par 2 Nisto reconheceis o Espírito de Deus: todo espírito que confessa que Jesus Cristo veio em carne é de Deus;
\par 3 e todo espírito que não confessa a Jesus não procede de Deus; pelo contrário, este é o espírito do anticristo, a respeito do qual tendes ouvido que vem e, presentemente, já está no mundo.
\par 4 Filhinhos, vós sois de Deus e tendes vencido os falsos profetas, porque maior é aquele que está em vós do que aquele que está no mundo.
\par 5 Eles procedem do mundo; por essa razão, falam da parte do mundo, e o mundo os ouve.
\par 6 Nós somos de Deus; aquele que conhece a Deus nos ouve; aquele que não é da parte de Deus não nos ouve. Nisto reconhecemos o espírito da verdade e o espírito do erro.
\par 7 Amados, amemo-nos uns aos outros, porque o amor procede de Deus; e todo aquele que ama é nascido de Deus e conhece a Deus.
\par 8 Aquele que não ama não conhece a Deus, pois Deus é amor.
\par 9 Nisto se manifestou o amor de Deus em nós: em haver Deus enviado o seu Filho unigênito ao mundo, para vivermos por meio dele.
\par 10 Nisto consiste o amor: não em que nós tenhamos amado a Deus, mas em que ele nos amou e enviou o seu Filho como propiciação pelos nossos pecados.
\par 11 Amados, se Deus de tal maneira nos amou, devemos nós também amar uns aos outros.
\par 12 Ninguém jamais viu a Deus; se amarmos uns aos outros, Deus permanece em nós, e o seu amor é, em nós, aperfeiçoado.
\par 13 Nisto conhecemos que permanecemos nele, e ele, em nós: em que nos deu do seu Espírito.
\par 14 E nós temos visto e testemunhamos que o Pai enviou o seu Filho como Salvador do mundo.
\par 15 Aquele que confessar que Jesus é o Filho de Deus, Deus permanece nele, e ele, em Deus.
\par 16 E nós conhecemos e cremos no amor que Deus tem por nós. Deus é amor, e aquele que permanece no amor permanece em Deus, e Deus, nele.
\par 17 Nisto é em nós aperfeiçoado o amor, para que, no Dia do Juízo, mantenhamos confiança; pois, segundo ele é, também nós somos neste mundo.
\par 18 No amor não existe medo; antes, o perfeito amor lança fora o medo. Ora, o medo produz tormento; logo, aquele que teme não é aperfeiçoado no amor.
\par 19 Nós amamos porque ele nos amou primeiro.
\par 20 Se alguém disser: Amo a Deus, e odiar a seu irmão, é mentiroso; pois aquele que não ama a seu irmão, a quem vê, não pode amar a Deus, a quem não vê.
\par 21 Ora, temos, da parte dele, este mandamento: que aquele que ama a Deus ame também a seu irmão.

\chapter{5}

\par 1 Todo aquele que crê que Jesus é o Cristo é nascido de Deus; e todo aquele que ama ao que o gerou também ama ao que dele é nascido.
\par 2 Nisto conhecemos que amamos os filhos de Deus: quando amamos a Deus e praticamos os seus mandamentos.
\par 3 Porque este é o amor de Deus: que guardemos os seus mandamentos; ora, os seus mandamentos não são penosos,
\par 4 porque todo o que é nascido de Deus vence o mundo; e esta é a vitória que vence o mundo: a nossa fé.
\par 5 Quem é o que vence o mundo, senão aquele que crê ser Jesus o Filho de Deus?
\par 6 Este é aquele que veio por meio de água e sangue, Jesus Cristo; não somente com água, mas também com a água e com o sangue. E o Espírito é o que dá testemunho, porque o Espírito é a verdade.
\par 7 Pois há três que dão testemunho [no céu: o Pai, a Palavra e o Espírito Santo; e estes três são um.
\par 8 E três são os que testificam na terra]: o Espírito, a água e o sangue, e os três são unânimes num só propósito.
\par 9 Se admitimos o testemunho dos homens, o testemunho de Deus é maior; ora, este é o testemunho de Deus, que ele dá acerca do seu Filho.
\par 10 Aquele que crê no Filho de Deus tem, em si, o testemunho. Aquele que não dá crédito a Deus o faz mentiroso, porque não crê no testemunho que Deus dá acerca do seu Filho.
\par 11 E o testemunho é este: que Deus nos deu a vida eterna; e esta vida está no seu Filho.
\par 12 Aquele que tem o Filho tem a vida; aquele que não tem o Filho de Deus não tem a vida.
\par 13 Estas coisas vos escrevi, a fim de saberdes que tendes a vida eterna, a vós outros que credes em o nome do Filho de Deus.
\par 14 E esta é a confiança que temos para com ele: que, se pedirmos alguma coisa segundo a sua vontade, ele nos ouve.
\par 15 E, se sabemos que ele nos ouve quanto ao que lhe pedimos, estamos certos de que obtemos os pedidos que lhe temos feito.
\par 16 Se alguém vir a seu irmão cometer pecado não para morte, pedirá, e Deus lhe dará vida, aos que não pecam para morte. Há pecado para morte, e por esse não digo que rogue.
\par 17 Toda injustiça é pecado, e há pecado não para morte.
\par 18 Sabemos que todo aquele que é nascido de Deus não vive em pecado; antes, Aquele que nasceu de Deus o guarda, e o Maligno não lhe toca.
\par 19 Sabemos que somos de Deus e que o mundo inteiro jaz no Maligno.
\par 20 Também sabemos que o Filho de Deus é vindo e nos tem dado entendimento para reconhecermos o verdadeiro; e estamos no verdadeiro, em seu Filho, Jesus Cristo. Este é o verdadeiro Deus e a vida eterna.
\par 21 Filhinhos, guardai-vos dos ídolos.


\end{document}