\begin{document}

\title{Ester}


\chapter{1}

\par 1 Nos dias de Assuero, o Assuero que reinou, desde a Índia até à Etiópia, sobre cento e vinte e sete províncias,
\par 2 naqueles dias, assentando-se o rei Assuero no trono do seu reino, que está na cidadela de Susã,
\par 3 no terceiro ano de seu reinado, deu um banquete a todos os seus príncipes e seus servos, no qual se representou o escol da Pérsia e Média, e os nobres e príncipes das províncias estavam perante ele.
\par 4 Então, mostrou as riquezas da glória do seu reino e o esplendor da sua excelente grandeza, por muitos dias, por cento e oitenta dias.
\par 5 Passados esses dias, deu o rei um banquete a todo o povo que se achava na cidadela de Susã, tanto para os maiores como para os menores, por sete dias, no pátio do jardim do palácio real.
\par 6 Havia tecido branco, linho fino e estofas de púrpura atados com cordões de linho e de púrpura a argolas de prata e a colunas de alabastro. A armação dos leitos era de ouro e de prata, sobre um pavimento de pórfiro, de mármore, de alabastro e de pedras preciosas.
\par 7 Dava-se-lhes de beber em vasos de ouro, vasos de várias espécies, e havia muito vinho real, graças à generosidade do rei.
\par 8 Bebiam sem constrangimento, como estava prescrito, pois o rei havia ordenado a todos os oficiais da sua casa que fizessem segundo a vontade de cada um.
\par 9 Também a rainha Vasti deu um banquete às mulheres na casa real do rei Assuero.
\par 10 Ao sétimo dia, estando já o coração do rei alegre do vinho, mandou a Meumã, Bizta, Harbona, Bigtá, Abagta, Zetar e Carcas, os sete eunucos que serviam na presença do rei Assuero,
\par 11 que introduzissem à presença do rei a rainha Vasti, com a coroa real, para mostrar aos povos e aos príncipes a formosura dela, pois era em extremo formosa.
\par 12 Porém a rainha Vasti recusou vir por intermédio dos eunucos, segundo a palavra do rei; pelo que o rei muito se enfureceu e se inflamou de ira.
\par 13 Então, o rei consultou os sábios que entendiam dos tempos (porque assim se tratavam os interesses do rei na presença de todos os que sabiam a lei e o direito;
\par 14 e os mais chegados a ele eram: Carsena, Setar, Admata, Társis, Meres, Marsena e Memucã, os sete príncipes dos persas e dos medos, que se avistavam pessoalmente com o rei e se assentavam como principais no reino)
\par 15 sobre o que se devia fazer, segundo a lei, à rainha Vasti, por não haver ela cumprido o mandado do rei Assuero, por intermédio dos eunucos.
\par 16 Então, disse Memucã na presença do rei e dos príncipes: A rainha Vasti não somente ofendeu ao rei, mas também a todos os príncipes e a todos os povos que há em todas as províncias do rei Assuero.
\par 17 Porque a notícia do que fez a rainha chegará a todas as mulheres, de modo que desprezarão a seu marido, quando ouvirem dizer: Mandou o rei Assuero que introduzissem à sua presença a rainha Vasti, porém ela não foi.
\par 18 Hoje mesmo, as princesas da Pérsia e da Média, ao ouvirem o que fez a rainha, dirão o mesmo a todos os príncipes do rei; e haverá daí muito desprezo e indignação.
\par 19 Se bem parecer ao rei, promulgue de sua parte um edito real, e que se inscreva nas leis dos persas e dos medos e não se revogue, que Vasti não entre jamais na presença do rei Assuero; e o rei dê o reino dela a outra que seja melhor do que ela.
\par 20 Quando for ouvido o mandado, que o rei decretar em todo o seu reino, vasto que é, todas as mulheres darão honra a seu marido, tanto ao mais importante como ao menos importante.
\par 21 O conselho pareceu bem tanto ao rei como aos príncipes; e fez o rei segundo a palavra de Memucã.
\par 22 Então, enviou cartas a todas as províncias do rei, a cada província segundo o seu modo de escrever e a cada povo segundo a sua língua: que cada homem fosse senhor em sua casa, e que se falasse a língua do seu povo.

\chapter{2}

\par 1 Passadas estas coisas, e apaziguado já o furor do rei Assuero, lembrou-se de Vasti, e do que ela fizera, e do que se tinha decretado contra ela.
\par 2 Então, disseram os jovens do rei, que lhe serviam: Tragam-se moças para o rei, virgens de boa aparência e formosura.
\par 3 Ponha o rei comissários em todas as províncias do seu reino, que reúnam todas as moças virgens, de boa aparência e formosura, na cidadela de Susã, na casa das mulheres, sob as vistas de Hegai, eunuco do rei, guarda das mulheres, e dêem-se-lhes os seus ungüentos.
\par 4 A moça que cair no agrado do rei, essa reine em lugar de Vasti. Com isto concordou o rei, e assim se fez.
\par 5 Ora, na cidadela de Susã havia certo homem judeu, benjamita, chamado Mordecai, filho de Jair, filho de Simei, filho de Quis,
\par 6 que fora transportado de Jerusalém com os exilados que foram deportados com Jeconias, rei de Judá, a quem Nabucodonosor, rei da Babilônia, havia transportado.
\par 7 Ele criara a Hadassa, que é Ester, filha de seu tio, a qual não tinha pai nem mãe; e era jovem bela, de boa aparência e formosura. Tendo-lhe morrido o pai e a mãe, Mordecai a tomara por filha.
\par 8 Em se divulgando, pois, o mandado do rei e a sua lei, ao serem ajuntadas muitas moças na cidadela de Susã, sob as vistas de Hegai, levaram também Ester à casa do rei, sob os cuidados de Hegai, guarda das mulheres.
\par 9 A moça lhe pareceu formosa e alcançou favor perante ele; pelo que se apressou em dar-lhe os ungüentos e os devidos alimentos, como também sete jovens escolhidas da casa do rei; e a fez passar com as suas jovens para os melhores aposentos da casa das mulheres.
\par 10 Ester não havia declarado o seu povo nem a sua linhagem, pois Mordecai lhe ordenara que o não declarasse.
\par 11 Passeava Mordecai todos os dias diante do átrio da casa das mulheres, para se informar de como passava Ester e do que lhe sucederia.
\par 12 Em chegando o prazo de cada moça vir ao rei Assuero, depois de tratada segundo as prescrições para as mulheres, por doze meses (porque assim se cumpriam os dias de seu embelezamento, seis meses com óleo de mirra e seis meses com especiarias e com os perfumes e ungüentos em uso entre as mulheres),
\par 13 então, é que vinha a jovem ao rei; a ela se dava o que desejasse para levar consigo da casa das mulheres para a casa do rei.
\par 14 À tarde, entrava e, pela manhã, tornava à segunda casa das mulheres, sob as vistas de Saasgaz, eunuco do rei, guarda das concubinas; não tornava mais ao rei, salvo se o rei a desejasse, e ela fosse chamada pelo nome.
\par 15 Ester, filha de Abiail, tio de Mordecai, que a tomara por filha, quando lhe chegou a vez de ir ao rei, nada pediu além do que disse Hegai, eunuco do rei, guarda das mulheres. E Ester alcançou favor de todos quantos a viam.
\par 16 Assim, foi levada Ester ao rei Assuero, à casa real, no décimo mês, que é o mês de tebete, no sétimo ano do seu reinado.
\par 17 O rei amou a Ester mais do que a todas as mulheres, e ela alcançou perante ele favor e benevolência mais do que todas as virgens; o rei pôs-lhe na cabeça a coroa real e a fez rainha em lugar de Vasti.
\par 18 Então, o rei deu um grande banquete a todos os seus príncipes e aos seus servos; era o banquete de Ester; concedeu alívio às províncias e fez presentes segundo a generosidade real.
\par 19 Quando, pela segunda vez, se reuniram as virgens, Mordecai estava assentado à porta do rei.
\par 20 Ester não havia declarado ainda a sua linhagem e o seu povo, como Mordecai lhe ordenara; porque Ester cumpria o mandado de Mordecai como quando a criava.
\par 21 Naqueles dias, estando Mordecai sentado à porta do rei, dois eunucos do rei, dos guardas da porta, Bigtã e Teres, sobremodo se indignaram e tramaram atentar contra o rei Assuero.
\par 22 Veio isso ao conhecimento de Mordecai, que o revelou à rainha Ester, e Ester o disse ao rei, em nome de Mordecai.
\par 23 Investigou-se o caso, e era fato; e ambos foram pendurados numa forca. Isso foi escrito no Livro das Crônicas, perante o rei.

\chapter{3}

\par 1 Depois destas coisas, o rei Assuero engrandeceu a Hamã, filho de Hamedata, agagita, e o exaltou, e lhe pôs o trono acima de todos os príncipes que estavam com ele.
\par 2 Todos os servos do rei, que estavam à porta do rei, se inclinavam e se prostravam perante Hamã; porque assim tinha ordenado o rei a respeito dele. Mordecai, porém, não se inclinava, nem se prostrava.
\par 3 Então, os servos do rei, que estavam à porta do rei, disseram a Mordecai: Por que transgrides as ordens do rei?
\par 4 Sucedeu, pois, que, dizendo-lhe eles isto, dia após dia, e não lhes dando ele ouvidos, o fizeram saber a Hamã, para ver se as palavras de Mordecai se manteriam de pé, porque ele lhes tinha declarado que era judeu.
\par 5 Vendo, pois, Hamã que Mordecai não se inclinava, nem se prostrava diante dele, encheu-se de furor.
\par 6 Porém teve como pouco, nos seus propósitos, o atentar apenas contra Mordecai, porque lhe haviam declarado de que povo era Mordecai; por isso, procurou Hamã destruir todos os judeus, povo de Mordecai, que havia em todo o reino de Assuero.
\par 7 No primeiro mês, que é o mês de nisã, no ano duodécimo do rei Assuero, se lançou o Pur, isto é, sortes, perante Hamã, dia a dia, mês a mês, até ao duodécimo, que é o mês de adar.
\par 8 Então, disse Hamã ao rei Assuero: Existe espalhado, disperso entre os povos em todas as províncias do teu reino, um povo cujas leis são diferentes das leis de todos os povos e que não cumpre as do rei; pelo que não convém ao rei tolerá-lo.
\par 9 Se bem parecer ao rei, decrete-se que sejam mortos, e, nas próprias mãos dos que executarem a obra, eu pesarei dez mil talentos de prata para que entrem nos tesouros do rei.
\par 10 Então, o rei tirou da mão o seu anel, deu-o a Hamã, filho de Hamedata, agagita, adversário dos judeus,
\par 11 e lhe disse: Essa prata seja tua, como também esse povo, para fazeres dele o que melhor for de teu agrado.
\par 12 Chamaram, pois, os secretários do rei, no dia treze do primeiro mês, e, segundo ordenou Hamã, tudo se escreveu aos sátrapas do rei, aos governadores de todas as províncias e aos príncipes de cada povo; a cada província no seu próprio modo de escrever e a cada povo na sua própria língua. Em nome do rei Assuero se escreveu, e com o anel do rei se selou.
\par 13 Enviaram-se as cartas, por intermédio dos correios, a todas as províncias do rei, para que se destruíssem, matassem e aniquilassem de vez a todos os judeus, moços e velhos, crianças e mulheres, em um só dia, no dia treze do duodécimo mês, que é o mês de adar, e que lhes saqueassem os bens.
\par 14 Tais cartas encerravam o traslado do decreto para que se proclamasse a lei em cada província; esse traslado foi enviado a todos os povos para que se preparassem para aquele dia.
\par 15 Os correios, pois, impelidos pela ordem do rei, partiram incontinenti, e a lei se proclamou na cidadela de Susã; o rei e Hamã se assentaram a beber, mas a cidade de Susã estava perplexa.

\chapter{4}

\par 1 Quando soube Mordecai tudo quanto se havia passado, rasgou as suas vestes, e se cobriu de pano de saco e de cinza, e, saindo pela cidade, clamou com grande e amargo clamor;
\par 2 e chegou até à porta do rei; porque ninguém vestido de pano de saco podia entrar pelas portas do rei.
\par 3 Em todas as províncias aonde chegava a palavra do rei e a sua lei, havia entre os judeus grande luto, com jejum, e choro, e lamentação; e muitos se deitavam em pano de saco e em cinza.
\par 4 Então, vieram as servas de Ester e os eunucos e fizeram-na saber, com o que a rainha muito se doeu; e mandou roupas para vestir a Mordecai e tirar-lhe o pano de saco; porém ele não as aceitou.
\par 5 Então, Ester chamou a Hataque, um dos eunucos do rei, que este lhe dera para a servir, e lhe ordenou que fosse a Mordecai para saber que era aquilo e o seu motivo.
\par 6 Saiu, pois, Hataque à praça da cidade para encontrar-se com Mordecai à porta do rei.
\par 7 Mordecai lhe fez saber tudo quanto lhe tinha sucedido; como também a quantia certa da prata que Hamã prometera pagar aos tesouros do rei pelo aniquilamento dos judeus.
\par 8 Também lhe deu o traslado do decreto escrito que se publicara em Susã para os destruir, para que o mostrasse a Ester e a fizesse saber, a fim de que fosse ter com o rei, e lhe pedisse misericórdia, e, na sua presença, lhe suplicasse pelo povo dela.
\par 9 Tornou, pois, Hataque e fez saber a Ester as palavras de Mordecai.
\par 10 Então, respondeu Ester a Hataque e mandou-lhe dizer a Mordecai:
\par 11 Todos os servos do rei e o povo das províncias do rei sabem que, para qualquer homem ou mulher que, sem ser chamado, entrar no pátio interior para avistar-se com o rei, não há senão uma sentença, a de morte, salvo se o rei estender para ele o cetro de ouro, para que viva; e eu, nestes trinta dias, não fui chamada para entrar ao rei.
\par 12 Fizeram saber a Mordecai as palavras de Ester.
\par 13 Então, lhes disse Mordecai que respondessem a Ester: Não imagines que, por estares na casa do rei, só tu escaparás entre todos os judeus.
\par 14 Porque, se de todo te calares agora, de outra parte se levantará para os judeus socorro e livramento, mas tu e a casa de teu pai perecereis; e quem sabe se para conjuntura como esta é que foste elevada a rainha?
\par 15 Então, disse Ester que respondessem a Mordecai:
\par 16 Vai, ajunta a todos os judeus que se acharem em Susã, e jejuai por mim, e não comais, nem bebais por três dias, nem de noite nem de dia; eu e as minhas servas também jejuaremos. Depois, irei ter com o rei, ainda que é contra a lei; se perecer, pereci.
\par 17 Então, se foi Mordecai e tudo fez segundo Ester lhe havia ordenado.

\chapter{5}

\par 1 Ao terceiro dia, Ester se aprontou com seus trajes reais e se pôs no pátio interior da casa do rei, defronte da residência do rei; o rei estava assentado no seu trono real fronteiro à porta da residência.
\par 2 Quando o rei viu a rainha Ester parada no pátio, alcançou ela favor perante ele; estendeu o rei para Ester o cetro de ouro que tinha na mão; Ester se chegou e tocou a ponta do cetro.
\par 3 Então, lhe disse o rei: Que é o que tens, rainha Ester, ou qual é a tua petição? Até metade do reino se te dará.
\par 4 Respondeu Ester: Se bem te parecer, venha o rei e Hamã, hoje, ao banquete que eu preparei ao rei.
\par 5 Então, disse o rei: Fazei apressar a Hamã, para que atendamos ao que Ester deseja. Vindo, pois, o rei e Hamã ao banquete que Ester havia preparado,
\par 6 disse o rei a Ester, no banquete do vinho: Qual é a tua petição? E se te dará. Que desejas? Cumprir-se-á, ainda que seja metade do reino.
\par 7 Então, respondeu Ester e disse: Minha petição e desejo são o seguinte:
\par 8 se achei favor perante o rei, e se bem parecer ao rei conceder-me a petição e cumprir o meu desejo, venha o rei com Hamã ao banquete que lhes hei de preparar amanhã, e, então, farei segundo o rei me concede.
\par 9 Então, saiu Hamã, naquele dia, alegre e de bom ânimo; quando viu, porém, Mordecai à porta do rei e que não se levantara, nem se movera diante dele, então, se encheu de furor contra Mordecai.
\par 10 Hamã, porém, se conteve e foi para casa; e mandou vir os seus amigos e a Zeres, sua mulher.
\par 11 Contou-lhes Hamã a glória das suas riquezas e a multidão de seus filhos, e tudo em que o rei o tinha engrandecido, e como o tinha exaltado sobre os príncipes e servos do rei.
\par 12 Disse mais Hamã: A própria rainha Ester a ninguém fez vir com o rei ao banquete que tinha preparado, senão a mim; e também para amanhã estou convidado por ela, juntamente com o rei.
\par 13 Porém tudo isto não me satisfaz, enquanto vir o judeu Mordecai assentado à porta do rei.
\par 14 Então, lhe disse Zeres, sua mulher, e todos os seus amigos: Faça-se uma forca de cinqüenta côvados de altura, e, pela manhã, dize ao rei que nela enforquem Mordecai; então, entra alegre com o rei ao banquete. A sugestão foi bem aceita por Hamã, que mandou levantar a forca.

\chapter{6}

\par 1 Naquela noite, o rei não pôde dormir; então, mandou trazer o Livro dos Feitos Memoráveis, e nele se leu diante do rei.
\par 2 Achou-se escrito que Mordecai é quem havia denunciado a Bigtã e a Teres, os dois eunucos do rei, guardas da porta, que tinham procurado matar o rei Assuero.
\par 3 Então, disse o rei: Que honras e distinções se deram a Mordecai por isso? Nada lhe foi conferido, responderam os servos do rei que o serviam.
\par 4 Perguntou o rei: Quem está no pátio? Ora, Hamã tinha entrado no pátio exterior da casa do rei, para dizer ao rei que se enforcasse a Mordecai na forca que ele, Hamã, lhe tinha preparado.
\par 5 Os servos do rei lhe disseram: Hamã está no pátio. Disse o rei que entrasse.
\par 6 Entrou Hamã. O rei lhe disse: Que se fará ao homem a quem o rei deseja honrar? Então, Hamã disse consigo mesmo: De quem se agradaria o rei mais do que de mim para honrá-lo?
\par 7 E respondeu ao rei: Quanto ao homem a quem agrada ao rei honrá-lo,
\par 8 tragam-se as vestes reais, que o rei costuma usar, e o cavalo em que o rei costuma andar montado, e tenha na cabeça a coroa real;
\par 9 entreguem-se as vestes e o cavalo às mãos dos mais nobres príncipes do rei, e vistam delas aquele a quem o rei deseja honrar; levem-no a cavalo pela praça da cidade e diante dele apregoem: Assim se faz ao homem a quem o rei deseja honrar.
\par 10 Então, disse o rei a Hamã: Apressa-te, toma as vestes e o cavalo, como disseste, e faze assim para com o judeu Mordecai, que está assentado à porta do rei; e não omitas coisa nenhuma de tudo quanto disseste.
\par 11 Hamã tomou as vestes e o cavalo, vestiu a Mordecai, e o levou a cavalo pela praça da cidade, e apregoou diante dele: Assim se faz ao homem a quem o rei deseja honrar.
\par 12 Depois disto, Mordecai voltou para a porta do rei; porém Hamã se retirou correndo para casa, angustiado e de cabeça coberta.
\par 13 Contou Hamã a Zeres, sua mulher, e a todos os seus amigos tudo quanto lhe tinha sucedido. Então, os seus sábios e Zeres, sua mulher, lhe disseram: Se Mordecai, perante o qual já começaste a cair, é da descendência dos judeus, não prevalecerás contra ele; antes, certamente, cairás diante dele.
\par 14 Falavam estes ainda com ele quando chegaram os eunucos do rei e apressadamente levaram Hamã ao banquete que Ester preparara.

\chapter{7}

\par 1 Veio, pois, o rei com Hamã, para beber com a rainha Ester.
\par 2 No segundo dia, durante o banquete do vinho, disse o rei a Ester: Qual é a tua petição, rainha Ester? E se te dará. Que desejas? Cumprir-se-á ainda que seja metade do reino.
\par 3 Então, respondeu a rainha Ester e disse: Se perante ti, ó rei, achei favor, e se bem parecer ao rei, dê-se-me por minha petição a minha vida, e, pelo meu desejo, a vida do meu povo.
\par 4 Porque fomos vendidos, eu e o meu povo, para nos destruírem, matarem e aniquilarem de vez; se ainda como servos e como servas nos tivessem vendido, calar-me-ia, porque o inimigo não merece que eu moleste o rei.
\par 5 Então, falou o rei Assuero e disse à rainha Ester: Quem é esse e onde está esse cujo coração o instigou a fazer assim?
\par 6 Respondeu Ester: O adversário e inimigo é este mau Hamã. Então, Hamã se perturbou perante o rei e a rainha.
\par 7 O rei, no seu furor, se levantou do banquete do vinho e passou para o jardim do palácio; Hamã, porém, ficou para rogar por sua vida à rainha Ester, pois viu que o mal contra ele já estava determinado pelo rei.
\par 8 Tornando o rei do jardim do palácio à casa do banquete do vinho, Hamã tinha caído sobre o divã em que se achava Ester. Então, disse o rei: Acaso, teria ele querido forçar a rainha perante mim, na minha casa? Tendo o rei dito estas palavras, cobriram o rosto de Hamã.
\par 9 Então, disse Harbona, um dos eunucos que serviam o rei: Eis que existe junto à casa de Hamã a forca de cinqüenta côvados de altura que ele preparou para Mordecai, que falara em defesa do rei. Então, disse o rei: Enforcai-o nela.
\par 10 Enforcaram, pois, Hamã na forca que ele tinha preparado para Mordecai. Então, o furor do rei se aplacou.

\chapter{8}

\par 1 Naquele mesmo dia, deu o rei Assuero à rainha Ester a casa de Hamã, inimigo dos judeus; e Mordecai veio perante o rei, porque Ester lhe fez saber que era seu parente.
\par 2 Tirou o rei o seu anel, que tinha tomado a Hamã, e o deu a Mordecai. E Ester pôs a Mordecai por superintendente da casa de Hamã.
\par 3 Falou mais Ester perante o rei e se lhe lançou aos pés; e, com lágrimas, lhe implorou que revogasse a maldade de Hamã, o agagita, e a trama que havia empreendido contra os judeus.
\par 4 Estendeu o rei para Ester o cetro de ouro. Então, ela se levantou, pôs-se de pé diante do rei
\par 5 e lhe disse: Se bem parecer ao rei, se eu achei favor perante ele, se esta coisa é reta diante do rei, e se nisto lhe agrado, escreva-se que se revoguem os decretos concebidos por Hamã, filho de Hamedata, o agagita, os quais ele escreveu para aniquilar os judeus que há em todas as províncias do rei.
\par 6 Pois como poderei ver o mal que sobrevirá ao meu povo? E como poderei ver a destruição da minha parentela?
\par 7 Então, disse o rei Assuero à rainha Ester e ao judeu Mordecai: Eis que dei a Ester a casa de Hamã, e a ele penduraram numa forca, porquanto intentara matar os judeus.
\par 8 Escrevei, pois, aos judeus, como bem vos parecer, em nome do rei, e selai-o com o anel do rei; porque os decretos feitos em nome do rei e que com o seu anel se selam não se podem revogar.
\par 9 Então, foram chamados, sem detença, os secretários do rei, aos vinte e três dias do mês de sivã, que é o terceiro mês. E, segundo tudo quanto ordenou Mordecai, se escreveu um edito para os judeus, para os sátrapas, para os governadores e para os príncipes das províncias que se estendem da Índia à Etiópia, cento e vinte e sete províncias, a cada uma no seu próprio modo de escrever, e a cada povo na sua própria língua; e também aos judeus segundo o seu próprio modo de escrever e a sua própria língua.
\par 10 Escreveu-se em nome do rei Assuero, e se selou com o anel do rei; as cartas foram enviadas por intermédio de correios montados em ginetes criados na coudelaria do rei.
\par 11 Nelas, o rei concedia aos judeus de cada cidade que se reunissem e se dispusessem para defender a sua vida, para destruir, matar e aniquilar de vez toda e qualquer força armada do povo da província que viessem contra eles, crianças e mulheres, e que se saqueassem os seus bens,
\par 12 num mesmo dia, em todas as províncias do rei Assuero, no dia treze do duodécimo mês, que é o mês de adar.
\par 13 A carta, que determinava a proclamação do edito em todas as províncias, foi enviada a todos os povos, para que os judeus se preparassem para aquele dia, para se vingarem dos seus inimigos.
\par 14 Os correios, montados em ginetes que se usavam no serviço do rei, saíram incontinenti, impelidos pela ordem do rei; e o edito foi publicado na cidadela de Susã.
\par 15 Então, Mordecai saiu da presença do rei com veste real azul-celeste e branco, como também com grande coroa de ouro e manto de linho fino e púrpura; e a cidade de Susã exultou e se alegrou.
\par 16 Para os judeus houve felicidade, alegria, regozijo e honra.
\par 17 Também em toda província e em toda cidade aonde chegava a palavra do rei e a sua ordem, havia entre os judeus alegria e regozijo, banquetes e festas; e muitos, dos povos da terra, se fizeram judeus, porque o temor dos judeus tinha caído sobre eles.

\chapter{9}

\par 1 No dia treze do duodécimo mês, que é o mês de adar, quando chegou a palavra do rei e a sua ordem para se executar, no dia em que os inimigos dos judeus contavam assenhorear-se deles, sucedeu o contrário, pois os judeus é que se assenhorearam dos que os odiavam;
\par 2 porque os judeus, nas suas cidades, em todas as províncias do rei Assuero, se ajuntaram para dar cabo daqueles que lhes procuravam o mal; e ninguém podia resistir-lhes, porque o terror que inspiravam caiu sobre todos aqueles povos.
\par 3 Todos os príncipes das províncias, e os sátrapas, e os governadores, e os oficiais do rei auxiliavam os judeus, porque tinha caído sobre eles o temor de Mordecai.
\par 4 Porque Mordecai era grande na casa do rei, e a sua fama crescia por todas as províncias; pois ele se ia tornando mais e mais poderoso.
\par 5 Feriram, pois, os judeus a todos os seus inimigos, a golpes de espada, com matança e destruição; e fizeram dos seus inimigos o que bem quiseram.
\par 6 Na cidadela de Susã, os judeus mataram e destruíram a quinhentos homens,
\par 7 como também a Parsandata, a Dalfom, a Aspata,
\par 8 a Porata, a Adalia, a Aridata,
\par 9 a Farmasta, a Arisai, a Aridai e a Vaizata,
\par 10 que eram os dez filhos de Hamã, filho de Hamedata, o inimigo dos judeus; porém no despojo não tocaram.
\par 11 No mesmo dia, foi comunicado ao rei o número dos mortos na cidadela de Susã.
\par 12 Disse o rei à rainha Ester: Na cidadela de Susã, mataram e destruíram os judeus a quinhentos homens e os dez filhos de Hamã; nas mais províncias do rei, que terão eles feito? Qual é, pois, a tua petição? E se te dará. Ou que é que desejas ainda? E se cumprirá.
\par 13 Então, disse Ester: Se bem parecer ao rei, conceda-se aos judeus que se acham em Susã que também façam, amanhã, segundo o edito de hoje e dependurem em forca os cadáveres dos dez filhos de Hamã.
\par 14 Então, disse o rei que assim se fizesse; publicou-se o edito em Susã, e dependuraram os cadáveres dos dez filhos de Hamã.
\par 15 Reuniram-se os judeus que se achavam em Susã também no dia catorze do mês de adar, e mataram, em Susã, a trezentos homens; porém no despojo não tocaram.
\par 16 Também os demais judeus que se achavam nas províncias do rei se reuniram, e se dispuseram para defender a vida, e tiveram sossego dos seus inimigos; e mataram a setenta e cinco mil dos que os odiavam; porém no despojo não tocaram.
\par 17 Sucedeu isto no dia treze do mês de adar; no dia catorze, descansaram e o fizeram dia de banquetes e de alegria.
\par 18 Os judeus, porém, que se achavam em Susã se ajuntaram nos dias treze e catorze do mesmo; e descansaram no dia quinze e o fizeram dia de banquetes e de alegria.
\par 19 Também os judeus das vilas que habitavam nas aldeias abertas fizeram do dia catorze do mês de adar dia de alegria e de banquetes e dia de festa e de mandarem porções dos banquetes uns aos outros.
\par 20 Mordecai escreveu estas coisas e enviou cartas a todos os judeus que se achavam em todas as províncias do rei Assuero, aos de perto e aos de longe,
\par 21 ordenando-lhes que comemorassem o dia catorze do mês de adar e o dia quinze do mesmo, todos os anos,
\par 22 como os dias em que os judeus tiveram sossego dos seus inimigos, e o mês que se lhes mudou de tristeza em alegria, e de luto em dia de festa; para que os fizessem dias de banquetes e de alegria, e de mandarem porções dos banquetes uns aos outros, e dádivas aos pobres.
\par 23 Assim, os judeus aceitaram como costume o que, naquele tempo, haviam feito pela primeira vez, segundo Mordecai lhes prescrevera;
\par 24 porque Hamã, filho de Hamedata, o agagita, inimigo de todos os judeus, tinha intentado destruir os judeus; e tinha lançado o Pur, isto é, sortes, para os assolar e destruir.
\par 25 Mas, tendo Ester ido perante o rei, ordenou ele por cartas que o seu mau intento, que assentara contra os judeus, recaísse contra a própria cabeça dele, pelo que enforcaram a ele e a seus filhos.
\par 26 Por isso, àqueles dias chamam Purim, do nome Pur. Daí, por causa de todas as palavras daquela carta, e do que testemunharam, e do que lhes havia sucedido,
\par 27 determinaram os judeus e tomaram sobre si, sobre a sua descendência e sobre todos os que se chegassem a eles que não se deixaria de comemorar estes dois dias segundo o que se escrevera deles e segundo o seu tempo marcado, todos os anos;
\par 28 e que estes dias seriam lembrados e comemorados geração após geração, por todas as famílias, em todas as províncias e em todas as cidades, e que estes dias de Purim jamais caducariam entre os judeus, e que a memória deles jamais se extinguiria entre os seus descendentes.
\par 29 Então, a rainha Ester, filha de Abiail, e o judeu Mordecai escreveram, com toda a autoridade, segunda vez, para confirmar a carta de Purim.
\par 30 Expediram cartas a todos os judeus, às cento e vinte e sete províncias do reino de Assuero, com palavras amigáveis e sinceras,
\par 31 para confirmar estes dias de Purim nos seus tempos determinados, como o judeu Mordecai e a rainha Ester lhes tinham estabelecido, e como eles mesmos já o tinham estabelecido sobre si e sobre a sua descendência, acerca do jejum e do seu lamento.
\par 32 E o mandado de Ester estabeleceu estas particularidades de Purim; e se escreveu no livro.

\chapter{10}

\par 1 Depois disto, o rei Assuero impôs tributo sobre a terra e sobre as terras do mar.
\par 2 Quanto aos mais atos do seu poder e do seu valor e ao relatório completo da grandeza de Mordecai, a quem o rei exaltou, porventura, não estão escritos no Livro da História dos Reis da Média e da Pérsia?
\par 3 Pois o judeu Mordecai foi o segundo depois do rei Assuero, e grande para com os judeus, e estimado pela multidão de seus irmãos, tendo procurado o bem-estar do seu povo e trabalhado pela prosperidade de todo o povo da sua raça.


\end{document}