\begin{document}

\title{Tiago}


\chapter{1}

\par 1 Tiago, servo de Deus e do Senhor Jesus Cristo, às doze tribos que se encontram na Dispersão, saudações.
\par 2 Meus irmãos, tende por motivo de toda alegria o passardes por várias provações,
\par 3 sabendo que a provação da vossa fé, uma vez confirmada, produz perseverança.
\par 4 Ora, a perseverança deve ter ação completa, para que sejais perfeitos e íntegros, em nada deficientes.
\par 5 Se, porém, algum de vós necessita de sabedoria, peça-a a Deus, que a todos dá liberalmente e nada lhes impropera; e ser-lhe-á concedida.
\par 6 Peça-a, porém, com fé, em nada duvidando; pois o que duvida é semelhante à onda do mar, impelida e agitada pelo vento.
\par 7 Não suponha esse homem que alcançará do Senhor alguma coisa;
\par 8 homem de ânimo dobre, inconstante em todos os seus caminhos.
\par 9 O irmão, porém, de condição humilde glorie-se na sua dignidade,
\par 10 e o rico, na sua insignificância, porque ele passará como a flor da erva.
\par 11 Porque o sol se levanta com seu ardente calor, e a erva seca, e a sua flor cai, e desaparece a formosura do seu aspecto; assim também se murchará o rico em seus caminhos.
\par 12 Bem-aventurado o homem que suporta, com perseverança, a provação; porque, depois de ter sido aprovado, receberá a coroa da vida, a qual o Senhor prometeu aos que o amam.
\par 13 Ninguém, ao ser tentado, diga: Sou tentado por Deus; porque Deus não pode ser tentado pelo mal e ele mesmo a ninguém tenta.
\par 14 Ao contrário, cada um é tentado pela sua própria cobiça, quando esta o atrai e seduz.
\par 15 Então, a cobiça, depois de haver concebido, dá à luz o pecado; e o pecado, uma vez consumado, gera a morte.
\par 16 Não vos enganeis, meus amados irmãos.
\par 17 Toda boa dádiva e todo dom perfeito são lá do alto, descendo do Pai das luzes, em quem não pode existir variação ou sombra de mudança.
\par 18 Pois, segundo o seu querer, ele nos gerou pela palavra da verdade, para que fôssemos como que primícias das suas criaturas.
\par 19 Sabeis estas coisas, meus amados irmãos. Todo homem, pois, seja pronto para ouvir, tardio para falar, tardio para se irar.
\par 20 Porque a ira do homem não produz a justiça de Deus.
\par 21 Portanto, despojando-vos de toda impureza e acúmulo de maldade, acolhei, com mansidão, a palavra em vós implantada, a qual é poderosa para salvar a vossa alma.
\par 22 Tornai-vos, pois, praticantes da palavra e não somente ouvintes, enganando-vos a vós mesmos.
\par 23 Porque, se alguém é ouvinte da palavra e não praticante, assemelha-se ao homem que contempla, num espelho, o seu rosto natural;
\par 24 pois a si mesmo se contempla, e se retira, e para logo se esquece de como era a sua aparência.
\par 25 Mas aquele que considera, atentamente, na lei perfeita, lei da liberdade, e nela persevera, não sendo ouvinte negligente, mas operoso praticante, esse será bem-aventurado no que realizar.
\par 26 Se alguém supõe ser religioso, deixando de refrear a língua, antes, enganando o próprio coração, a sua religião é vã.
\par 27 A religião pura e sem mácula, para com o nosso Deus e Pai, é esta: visitar os órfãos e as viúvas nas suas tribulações e a si mesmo guardar-se incontaminado do mundo.

\chapter{2}

\par 1 Meus irmãos, não tenhais a fé em nosso Senhor Jesus Cristo, Senhor da glória, em acepção de pessoas.
\par 2 Se, portanto, entrar na vossa sinagoga algum homem com anéis de ouro nos dedos, em trajos de luxo, e entrar também algum pobre andrajoso,
\par 3 e tratardes com deferência o que tem os trajos de luxo e lhe disserdes: Tu, assenta-te aqui em lugar de honra; e disserdes ao pobre: Tu, fica ali em pé ou assenta-te aqui abaixo do estrado dos meus pés,
\par 4 não fizestes distinção entre vós mesmos e não vos tornastes juízes tomados de perversos pensamentos?
\par 5 Ouvi, meus amados irmãos. Não escolheu Deus os que para o mundo são pobres, para serem ricos em fé e herdeiros do reino que ele prometeu aos que o amam?
\par 6 Entretanto, vós outros menosprezastes o pobre. Não são os ricos que vos oprimem e não são eles que vos arrastam para tribunais?
\par 7 Não são eles os que blasfemam o bom nome que sobre vós foi invocado?
\par 8 Se vós, contudo, observais a lei régia segundo a Escritura: Amarás o teu próximo como a ti mesmo, fazeis bem;
\par 9 se, todavia, fazeis acepção de pessoas, cometeis pecado, sendo argüidos pela lei como transgressores.
\par 10 Pois qualquer que guarda toda a lei, mas tropeça em um só ponto, se torna culpado de todos.
\par 11 Porquanto, aquele que disse: Não adulterarás também ordenou: Não matarás. Ora, se não adulteras, porém matas, vens a ser transgressor da lei.
\par 12 Falai de tal maneira e de tal maneira procedei como aqueles que hão de ser julgados pela lei da liberdade.
\par 13 Porque o juízo é sem misericórdia para com aquele que não usou de misericórdia. A misericórdia triunfa sobre o juízo.
\par 14 Meus irmãos, qual é o proveito, se alguém disser que tem fé, mas não tiver obras? Pode, acaso, semelhante fé salvá-lo?
\par 15 Se um irmão ou uma irmã estiverem carecidos de roupa e necessitados do alimento cotidiano,
\par 16 e qualquer dentre vós lhes disser: Ide em paz, aquecei-vos e fartai-vos, sem, contudo, lhes dar o necessário para o corpo, qual é o proveito disso?
\par 17 Assim, também a fé, se não tiver obras, por si só está morta.
\par 18 Mas alguém dirá: Tu tens fé, e eu tenho obras; mostra-me essa tua fé sem as obras, e eu, com as obras, te mostrarei a minha fé.
\par 19 Crês, tu, que Deus é um só? Fazes bem. Até os demônios crêem e tremem.
\par 20 Queres, pois, ficar certo, ó homem insensato, de que a fé sem as obras é inoperante?
\par 21 Não foi por obras que Abraão, o nosso pai, foi justificado, quando ofereceu sobre o altar o próprio filho, Isaque?
\par 22 Vês como a fé operava juntamente com as suas obras; com efeito, foi pelas obras que a fé se consumou,
\par 23 e se cumpriu a Escritura, a qual diz: Ora, Abraão creu em Deus, e isso lhe foi imputado para justiça; e: Foi chamado amigo de Deus.
\par 24 Verificais que uma pessoa é justificada por obras e não por fé somente.
\par 25 De igual modo, não foi também justificada por obras a meretriz Raabe, quando acolheu os emissários e os fez partir por outro caminho?
\par 26 Porque, assim como o corpo sem espírito é morto, assim também a fé sem obras é morta.

\chapter{3}

\par 1 Meus irmãos, não vos torneis, muitos de vós, mestres, sabendo que havemos de receber maior juízo.
\par 2 Porque todos tropeçamos em muitas coisas. Se alguém não tropeça no falar, é perfeito varão, capaz de refrear também todo o corpo.
\par 3 Ora, se pomos freio na boca dos cavalos, para nos obedecerem, também lhes dirigimos o corpo inteiro.
\par 4 Observai, igualmente, os navios que, sendo tão grandes e batidos de rijos ventos, por um pequeníssimo leme são dirigidos para onde queira o impulso do timoneiro.
\par 5 Assim, também a língua, pequeno órgão, se gaba de grandes coisas. Vede como uma fagulha põe em brasas tão grande selva!
\par 6 Ora, a língua é fogo; é mundo de iniqüidade; a língua está situada entre os membros de nosso corpo, e contamina o corpo inteiro, e não só põe em chamas toda a carreira da existência humana, como também é posta ela mesma em chamas pelo inferno.
\par 7 Pois toda espécie de feras, de aves, de répteis e de seres marinhos se doma e tem sido domada pelo gênero humano;
\par 8 a língua, porém, nenhum dos homens é capaz de domar; é mal incontido, carregado de veneno mortífero.
\par 9 Com ela, bendizemos ao Senhor e Pai; também, com ela, amaldiçoamos os homens, feitos à semelhança de Deus.
\par 10 De uma só boca procede bênção e maldição. Meus irmãos, não é conveniente que estas coisas sejam assim.
\par 11 Acaso, pode a fonte jorrar do mesmo lugar o que é doce e o que é amargoso?
\par 12 Acaso, meus irmãos, pode a figueira produzir azeitonas ou a videira, figos? Tampouco fonte de água salgada pode dar água doce.
\par 13 Quem entre vós é sábio e inteligente? Mostre em mansidão de sabedoria, mediante condigno proceder, as suas obras.
\par 14 Se, pelo contrário, tendes em vosso coração inveja amargurada e sentimento faccioso, nem vos glorieis disso, nem mintais contra a verdade.
\par 15 Esta não é a sabedoria que desce lá do alto; antes, é terrena, animal e demoníaca.
\par 16 Pois, onde há inveja e sentimento faccioso, aí há confusão e toda espécie de coisas ruins.
\par 17 A sabedoria, porém, lá do alto é, primeiramente, pura; depois, pacífica, indulgente, tratável, plena de misericórdia e de bons frutos, imparcial, sem fingimento.
\par 18 Ora, é em paz que se semeia o fruto da justiça, para os que promovem a paz.

\chapter{4}

\par 1 De onde procedem guerras e contendas que há entre vós? De onde, senão dos prazeres que militam na vossa carne?
\par 2 Cobiçais e nada tendes; matais, e invejais, e nada podeis obter; viveis a lutar e a fazer guerras. Nada tendes, porque não pedis;
\par 3 pedis e não recebeis, porque pedis mal, para esbanjardes em vossos prazeres.
\par 4 Infiéis, não compreendeis que a amizade do mundo é inimiga de Deus? Aquele, pois, que quiser ser amigo do mundo constitui-se inimigo de Deus.
\par 5 Ou supondes que em vão afirma a Escritura: É com ciúme que por nós anseia o Espírito, que ele fez habitar em nós?
\par 6 Antes, ele dá maior graça; pelo que diz: Deus resiste aos soberbos, mas dá graça aos humildes.
\par 7 Sujeitai-vos, portanto, a Deus; mas resisti ao diabo, e ele fugirá de vós.
\par 8 Chegai-vos a Deus, e ele se chegará a vós outros. Purificai as mãos, pecadores; e vós que sois de ânimo dobre, limpai o coração.
\par 9 Afligi-vos, lamentai e chorai. Converta-se o vosso riso em pranto, e a vossa alegria, em tristeza.
\par 10 Humilhai-vos na presença do Senhor, e ele vos exaltará.
\par 11 Irmãos, não faleis mal uns dos outros. Aquele que fala mal do irmão ou julga a seu irmão fala mal da lei e julga a lei; ora, se julgas a lei, não és observador da lei, mas juiz.
\par 12 Um só é Legislador e Juiz, aquele que pode salvar e fazer perecer; tu, porém, quem és, que julgas o próximo?
\par 13 Atendei, agora, vós que dizeis: Hoje ou amanhã, iremos para a cidade tal, e lá passaremos um ano, e negociaremos, e teremos lucros.
\par 14 Vós não sabeis o que sucederá amanhã. Que é a vossa vida? Sois, apenas, como neblina que aparece por instante e logo se dissipa.
\par 15 Em vez disso, devíeis dizer: Se o Senhor quiser, não só viveremos, como também faremos isto ou aquilo.
\par 16 Agora, entretanto, vos jactais das vossas arrogantes pretensões. Toda jactância semelhante a essa é maligna.
\par 17 Portanto, aquele que sabe que deve fazer o bem e não o faz nisso está pecando.

\chapter{5}

\par 1 Atendei, agora, ricos, chorai lamentando, por causa das vossas desventuras, que vos sobrevirão.
\par 2 As vossas riquezas estão corruptas, e as vossas roupagens, comidas de traça;
\par 3 o vosso ouro e a vossa prata foram gastos de ferrugens, e a sua ferrugem há de ser por testemunho contra vós mesmos e há de devorar, como fogo, as vossas carnes. Tesouros acumulastes nos últimos dias.
\par 4 Eis que o salário dos trabalhadores que ceifaram os vossos campos e que por vós foi retido com fraude está clamando; e os clamores dos ceifeiros penetraram até aos ouvidos do Senhor dos Exércitos.
\par 5 Tendes vivido regaladamente sobre a terra; tendes vivido nos prazeres; tendes engordado o vosso coração, em dia de matança;
\par 6 tendes condenado e matado o justo, sem que ele vos faça resistência.
\par 7 Sede, pois, irmãos, pacientes, até à vinda do Senhor. Eis que o lavrador aguarda com paciência o precioso fruto da terra, até receber as primeiras e as últimas chuvas.
\par 8 Sede vós também pacientes e fortalecei o vosso coração, pois a vinda do Senhor está próxima.
\par 9 Irmãos, não vos queixeis uns dos outros, para não serdes julgados. Eis que o juiz está às portas.
\par 10 Irmãos, tomai por modelo no sofrimento e na paciência os profetas, os quais falaram em nome do Senhor.
\par 11 Eis que temos por felizes os que perseveraram firmes. Tendes ouvido da paciência de Jó e vistes que fim o Senhor lhe deu; porque o Senhor é cheio de terna misericórdia e compassivo.
\par 12 Acima de tudo, porém, meus irmãos, não jureis nem pelo céu, nem pela terra, nem por qualquer outro voto; antes, seja o vosso sim sim, e o vosso não não, para não cairdes em juízo.
\par 13 Está alguém entre vós sofrendo? Faça oração. Está alguém alegre? Cante louvores.
\par 14 Está alguém entre vós doente? Chame os presbíteros da igreja, e estes façam oração sobre ele, ungindo-o com óleo, em nome do Senhor.
\par 15 E a oração da fé salvará o enfermo, e o Senhor o levantará; e, se houver cometido pecados, ser-lhe-ão perdoados.
\par 16 Confessai, pois, os vossos pecados uns aos outros e orai uns pelos outros, para serdes curados. Muito pode, por sua eficácia, a súplica do justo.
\par 17 Elias era homem semelhante a nós, sujeito aos mesmos sentimentos, e orou, com instância, para que não chovesse sobre a terra, e, por três anos e seis meses, não choveu.
\par 18 E orou, de novo, e o céu deu chuva, e a terra fez germinar seus frutos.
\par 19 Meus irmãos, se algum entre vós se desviar da verdade, e alguém o converter,
\par 20 sabei que aquele que converte o pecador do seu caminho errado salvará da morte a alma dele e cobrirá multidão de pecados.


\end{document}