\begin{document}

\title{II Pedro}


\chapter{1}

\par 1 Simão Pedro, servo e apóstolo de Jesus Cristo, aos que conosco obtiveram fé igualmente preciosa na justiça do nosso Deus e Salvador Jesus Cristo,
\par 2 graça e paz vos sejam multiplicadas, no pleno conhecimento de Deus e de Jesus, nosso Senhor.
\par 3 Visto como, pelo seu divino poder, nos têm sido doadas todas as coisas que conduzem à vida e à piedade, pelo conhecimento completo daquele que nos chamou para a sua própria glória e virtude,
\par 4 pelas quais nos têm sido doadas as suas preciosas e mui grandes promessas, para que por elas vos torneis co-participantes da natureza divina, livrando-vos da corrupção das paixões que há no mundo,
\par 5 por isso mesmo, vós, reunindo toda a vossa diligência, associai com a vossa fé a virtude; com a virtude, o conhecimento;
\par 6 com o conhecimento, o domínio próprio; com o domínio próprio, a perseverança; com a perseverança, a piedade;
\par 7 com a piedade, a fraternidade; com a fraternidade, o amor.
\par 8 Porque estas coisas, existindo em vós e em vós aumentando, fazem com que não sejais nem inativos, nem infrutuosos no pleno conhecimento de nosso Senhor Jesus Cristo.
\par 9 Pois aquele a quem estas coisas não estão presentes é cego, vendo só o que está perto, esquecido da purificação dos seus pecados de outrora.
\par 10 Por isso, irmãos, procurai, com diligência cada vez maior, confirmar a vossa vocação e eleição; porquanto, procedendo assim, não tropeçareis em tempo algum.
\par 11 Pois desta maneira é que vos será amplamente suprida a entrada no reino eterno de nosso Senhor e Salvador Jesus Cristo.
\par 12 Por esta razão, sempre estarei pronto para trazer-vos lembrados acerca destas coisas, embora estejais certos da verdade já presente convosco e nela confirmados.
\par 13 Também considero justo, enquanto estou neste tabernáculo, despertar-vos com essas lembranças,
\par 14 certo de que estou prestes a deixar o meu tabernáculo, como efetivamente nosso Senhor Jesus Cristo me revelou.
\par 15 Mas, de minha parte, esforçar-me-ei, diligentemente, por fazer que, a todo tempo, mesmo depois da minha partida, conserveis lembrança de tudo.
\par 16 Porque não vos demos a conhecer o poder e a vinda de nosso Senhor Jesus Cristo seguindo fábulas engenhosamente inventadas, mas nós mesmos fomos testemunhas oculares da sua majestade,
\par 17 pois ele recebeu, da parte de Deus Pai, honra e glória, quando pela Glória Excelsa lhe foi enviada a seguinte voz: Este é o meu Filho amado, em quem me comprazo.
\par 18 Ora, esta voz, vinda do céu, nós a ouvimos quando estávamos com ele no monte santo.
\par 19 Temos, assim, tanto mais confirmada a palavra profética, e fazeis bem em atendê-la, como a uma candeia que brilha em lugar tenebroso, até que o dia clareie e a estrela da alva nasça em vosso coração,
\par 20 sabendo, primeiramente, isto: que nenhuma profecia da Escritura provém de particular elucidação;
\par 21 porque nunca jamais qualquer profecia foi dada por vontade humana; entretanto, homens [santos] falaram da parte de Deus, movidos pelo Espírito Santo.

\chapter{2}

\par 1 Assim como, no meio do povo, surgiram falsos profetas, assim também haverá entre vós falsos mestres, os quais introduzirão, dissimuladamente, heresias destruidoras, até ao ponto de renegarem o Soberano Senhor que os resgatou, trazendo sobre si mesmos repentina destruição.
\par 2 E muitos seguirão as suas práticas libertinas, e, por causa deles, será infamado o caminho da verdade;
\par 3 também, movidos por avareza, farão comércio de vós, com palavras fictícias; para eles o juízo lavrado há longo tempo não tarda, e a sua destruição não dorme.
\par 4 Ora, se Deus não poupou anjos quando pecaram, antes, precipitando-os no inferno, os entregou a abismos de trevas, reservando-os para juízo;
\par 5 e não poupou o mundo antigo, mas preservou a Noé, pregador da justiça, e mais sete pessoas, quando fez vir o dilúvio sobre o mundo de ímpios;
\par 6 e, reduzindo a cinzas as cidades de Sodoma e Gomorra, ordenou-as à ruína completa, tendo-as posto como exemplo a quantos venham a viver impiamente;
\par 7 e livrou o justo Ló, afligido pelo procedimento libertino daqueles insubordinados
\par 8 (porque este justo, pelo que via e ouvia quando habitava entre eles, atormentava a sua alma justa, cada dia, por causa das obras iníquas daqueles),
\par 9 é porque o Senhor sabe livrar da provação os piedosos e reservar, sob castigo, os injustos para o Dia de Juízo,
\par 10 especialmente aqueles que, seguindo a carne, andam em imundas paixões e menosprezam qualquer governo. Atrevidos, arrogantes, não temem difamar autoridades superiores,
\par 11 ao passo que anjos, embora maiores em força e poder, não proferem contra elas juízo infamante na presença do Senhor.
\par 12 Esses, todavia, como brutos irracionais, naturalmente feitos para presa e destruição, falando mal daquilo em que são ignorantes, na sua destruição também hão de ser destruídos,
\par 13 recebendo injustiça por salário da injustiça que praticam. Considerando como prazer a sua luxúria carnal em pleno dia, quais nódoas e deformidades, eles se regalam nas suas próprias mistificações, enquanto banqueteiam junto convosco;
\par 14 tendo os olhos cheios de adultério e insaciáveis no pecado, engodando almas inconstantes, tendo coração exercitado na avareza, filhos malditos;
\par 15 abandonando o reto caminho, se extraviaram, seguindo pelo caminho de Balaão, filho de Beor, que amou o prêmio da injustiça
\par 16 (recebeu, porém, castigo da sua transgressão, a saber, um mudo animal de carga, falando com voz humana, refreou a insensatez do profeta).
\par 17 Esses tais são como fonte sem água, como névoas impelidas por temporal. Para eles está reservada a negridão das trevas;
\par 18 porquanto, proferindo palavras jactanciosas de vaidade, engodam com paixões carnais, por suas libertinagens, aqueles que estavam prestes a fugir dos que andam no erro,
\par 19 prometendo-lhes liberdade, quando eles mesmos são escravos da corrupção, pois aquele que é vencido fica escravo do vencedor.
\par 20 Portanto, se, depois de terem escapado das contaminações do mundo mediante o conhecimento do Senhor e Salvador Jesus Cristo, se deixam enredar de novo e são vencidos, tornou-se o seu último estado pior que o primeiro.
\par 21 Pois melhor lhes fora nunca tivessem conhecido o caminho da justiça do que, após conhecê-lo, volverem para trás, apartando-se do santo mandamento que lhes fora dado.
\par 22 Com eles aconteceu o que diz certo adágio verdadeiro: O cão voltou ao seu próprio vômito; e: A porca lavada voltou a revolver-se no lamaçal.

\chapter{3}

\par 1 Amados, esta é, agora, a segunda epístola que vos escrevo; em ambas, procuro despertar com lembranças a vossa mente esclarecida,
\par 2 para que vos recordeis das palavras que, anteriormente, foram ditas pelos santos profetas, bem como do mandamento do Senhor e Salvador, ensinado pelos vossos apóstolos,
\par 3 tendo em conta, antes de tudo, que, nos últimos dias, virão escarnecedores com os seus escárnios, andando segundo as próprias paixões
\par 4 e dizendo: Onde está a promessa da sua vinda? Porque, desde que os pais dormiram, todas as coisas permanecem como desde o princípio da criação.
\par 5 Porque, deliberadamente, esquecem que, de longo tempo, houve céus bem como terra, a qual surgiu da água e através da água pela palavra de Deus,
\par 6 pela qual veio a perecer o mundo daquele tempo, afogado em água.
\par 7 Ora, os céus que agora existem e a terra, pela mesma palavra, têm sido entesourados para fogo, estando reservados para o Dia do Juízo e destruição dos homens ímpios.
\par 8 Há, todavia, uma coisa, amados, que não deveis esquecer: que, para o Senhor, um dia é como mil anos, e mil anos, como um dia.
\par 9 Não retarda o Senhor a sua promessa, como alguns a julgam demorada; pelo contrário, ele é longânimo para convosco, não querendo que nenhum pereça, senão que todos cheguem ao arrependimento.
\par 10 Virá, entretanto, como ladrão, o Dia do Senhor, no qual os céus passarão com estrepitoso estrondo, e os elementos se desfarão abrasados; também a terra e as obras que nela existem serão atingidas.
\par 11 Visto que todas essas coisas hão de ser assim desfeitas, deveis ser tais como os que vivem em santo procedimento e piedade,
\par 12 esperando e apressando a vinda do Dia de Deus, por causa do qual os céus, incendiados, serão desfeitos, e os elementos abrasados se derreterão.
\par 13 Nós, porém, segundo a sua promessa, esperamos novos céus e nova terra, nos quais habita justiça.
\par 14 Por essa razão, pois, amados, esperando estas coisas, empenhai-vos por serdes achados por ele em paz, sem mácula e irrepreensíveis,
\par 15 e tende por salvação a longanimidade de nosso Senhor, como igualmente o nosso amado irmão Paulo vos escreveu, segundo a sabedoria que lhe foi dada,
\par 16 ao falar acerca destes assuntos, como, de fato, costuma fazer em todas as suas epístolas, nas quais há certas coisas difíceis de entender, que os ignorantes e instáveis deturpam, como também deturpam as demais Escrituras, para a própria destruição deles.
\par 17 Vós, pois, amados, prevenidos como estais de antemão, acautelai-vos; não suceda que, arrastados pelo erro desses insubordinados, descaiais da vossa própria firmeza;
\par 18 antes, crescei na graça e no conhecimento de nosso Senhor e Salvador Jesus Cristo. A ele seja a glória, tanto agora como no dia eterno.


\end{document}