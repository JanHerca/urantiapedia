\begin{document}

\title{Juízes}


\chapter{1}

\par 1 Depois da morte de Josué, os filhos de Israel consultaram o SENHOR, dizendo: Quem dentre nós, primeiro, subirá aos cananeus para pelejar contra eles?
\par 2 Respondeu o SENHOR: Judá subirá; eis que nas suas mãos entreguei a terra.
\par 3 Disse, pois, Judá a Simeão, seu irmão: Sobe comigo à herança que me caiu por sorte, e pelejemos contra os cananeus, e também eu subirei contigo à tua, que te caiu por sorte. E Simeão partiu com ele.
\par 4 Subiu Judá, e o SENHOR lhe entregou nas mãos os cananeus e os ferezeus; e feriram deles, em Bezeque, dez mil homens.
\par 5 Em Bezeque, encontraram Adoni-Bezeque e pelejaram contra ele; e feriram aos cananeus e aos ferezeus.
\par 6 Adoni-Bezeque, porém, fugiu; mas o perseguiram e, prendendo-o, lhe cortaram os polegares das mãos e dos pés.
\par 7 Então, disse Adoni-Bezeque: Setenta reis, a quem haviam sido cortados os polegares das mãos e dos pés, apanhavam migalhas debaixo da minha mesa; assim como eu fiz, assim Deus me pagou. E o levaram a Jerusalém, e morreu ali.
\par 8 Os filhos de Judá pelejaram contra Jerusalém e, tomando-a, passaram-na a fio de espada, pondo fogo à cidade.
\par 9 Depois, os filhos de Judá desceram a pelejar contra os cananeus que habitavam nas montanhas, no Neguebe e nas planícies.
\par 10 Partiu Judá contra os cananeus que habitavam em Hebrom, cujo nome, outrora, era Quiriate-Arba, e Judá feriu a Sesai, a Aimã e a Talmai.
\par 11 Dali partiu contra os moradores de Debir; e era, dantes, o nome de Debir Quiriate-Sefer.
\par 12 Disse Calebe: A quem derrotar Quiriate-Sefer e a tomar, darei minha filha Acsa por mulher.
\par 13 Tomou-a, pois, Otniel, filho de Quenaz, o irmão de Calebe, mais novo do que ele; e Calebe lhe deu sua filha Acsa por mulher.
\par 14 Esta, quando se foi a ele, insistiu com ele para que pedisse um campo ao pai dela; e ela apeou do jumento; então, Calebe lhe perguntou: Que desejas?
\par 15 Respondeu ela: Dá-me um presente; deste-me terra seca, dá-me também fontes de água. Então, Calebe lhe deu as fontes superiores e as fontes inferiores.
\par 16 Os filhos do queneu, sogro de Moisés, subiram, com os filhos de Judá, da cidade das Palmeiras ao deserto de Judá, que está ao sul de Arade; foram e habitaram com este povo.
\par 17 Foi-se, pois, Judá com Simeão, seu irmão, e feriram aos cananeus que habitavam em Zefate e totalmente a destruíram; por isso, lhe chamaram Horma.
\par 18 Tomou ainda Judá a Gaza, a Asquelom e a Ecrom com os seus respectivos territórios.
\par 19 Esteve o SENHOR com Judá, e este despovoou as montanhas; porém não expulsou os moradores do vale, porquanto tinham carros de ferro.
\par 20 E, como Moisés o dissera, deram Hebrom a Calebe, e este expulsou dali os três filhos de Anaque.
\par 21 Porém os filhos de Benjamim não expulsaram os jebuseus que habitavam em Jerusalém; antes, os jebuseus habitam com os filhos de Benjamim em Jerusalém, até ao dia de hoje.
\par 22 Subiu também a casa de José contra Betel, e o SENHOR era com eles.
\par 23 A casa de José enviou homens a espiar Betel, cujo nome, dantes, era Luz.
\par 24 Vendo os espias um homem que saía da cidade, lhe disseram: Mostra-nos a entrada da cidade, e usaremos de misericórdia para contigo.
\par 25 Mostrando-lhes ele a entrada da cidade, feriram a cidade a fio de espada; porém, àquele homem e a toda a sua família, deixaram ir.
\par 26 Então, se foi ele à terra dos heteus, e edificou uma cidade, e lhe chamou Luz; este é o seu nome até ao dia de hoje.
\par 27 Manassés não expulsou os habitantes de Bete-Seã, nem os de Taanaque, nem os de Dor, nem os de Ibleão, nem os de Megido, todas com suas respectivas aldeias; pelo que os cananeus lograram permanecer na mesma terra.
\par 28 Quando, porém, Israel se tornou mais forte, sujeitou os cananeus a trabalhos forçados e não os expulsou de todo.
\par 29 Efraim não expulsou os cananeus, habitantes de Gezer; antes, continuaram com ele em Gezer.
\par 30 Zebulom não expulsou os habitantes de Quitrom, nem os de Naalol; porém os cananeus continuaram com ele, sujeitos a trabalhos forçados.
\par 31 Aser não expulsou os habitantes de Aco, nem os de Sidom, os de Alabe, os de Aczibe, os de Helba, os de Afeca e os de Reobe;
\par 32 porém os aseritas continuaram no meio dos cananeus que habitavam na terra, porquanto os não expulsaram.
\par 33 Naftali não expulsou os habitantes de Bete-Semes, nem os de Bete-Anate; mas continuou no meio dos cananeus que habitavam na terra; porém os de Bete-Semes e Bete-Anate lhe foram sujeitos a trabalhos forçados.
\par 34 Os amorreus arredaram os filhos de Dã até às montanhas e não os deixavam descer ao vale.
\par 35 Porém os amorreus lograram habitar nas montanhas de Heres, em Aijalom e em Saalabim; contudo, a mão da casa de José prevaleceu, e foram sujeitos a trabalhos forçados.
\par 36 O limite dos amorreus foi desde a subida de Acrabim e desde Sela para cima.

\chapter{2}

\par 1 Subiu o Anjo do SENHOR de Gilgal a Boquim e disse: Do Egito vos fiz subir e vos trouxe à terra que, sob juramento, havia prometido a vossos pais. Eu disse: nunca invalidarei a minha aliança convosco.
\par 2 Vós, porém, não fareis aliança com os moradores desta terra; antes, derribareis os seus altares; contudo, não obedecestes à minha voz. Que é isso que fizestes?
\par 3 Pelo que também eu disse: não os expulsarei de diante de vós; antes, vos serão por adversários, e os seus deuses vos serão laços.
\par 4 Sucedeu que, falando o Anjo do SENHOR estas palavras a todos os filhos de Israel, levantou o povo a sua voz e chorou.
\par 5 Daí, chamarem a esse lugar Boquim; e sacrificaram ali ao SENHOR.
\par 6 Havendo Josué despedido o povo, foram-se os filhos de Israel, cada um à sua herança, para possuírem a terra.
\par 7 Serviu o povo ao SENHOR todos os dias de Josué e todos os dias dos anciãos que ainda sobreviveram por muito tempo depois de Josué e que viram todas as grandes obras feitas pelo SENHOR a Israel.
\par 8 Faleceu Josué, filho de Num, servo do SENHOR, com a idade de cento e dez anos;
\par 9 sepultaram-no no limite da sua herança, em Timnate-Heres, na região montanhosa de Efraim, ao norte do monte Gaás.
\par 10 Foi também congregada a seus pais toda aquela geração; e outra geração após eles se levantou, que não conhecia o SENHOR, nem tampouco as obras que fizera a Israel.
\par 11 Então, fizeram os filhos de Israel o que era mau perante o SENHOR; pois serviram aos baalins.
\par 12 Deixaram o SENHOR, Deus de seus pais, que os tirara da terra do Egito, e foram-se após outros deuses, dentre os deuses das gentes que havia ao redor deles, e os adoraram, e provocaram o SENHOR à ira.
\par 13 Porquanto deixaram o SENHOR e serviram a Baal e a Astarote.
\par 14 Pelo que a ira do SENHOR se acendeu contra Israel e os deu na mão dos espoliadores, que os pilharam; e os entregou na mão dos seus inimigos ao redor; e não mais puderam resistir a eles.
\par 15 Por onde quer que saíam, a mão do SENHOR era contra eles para seu mal, como o SENHOR lhes dissera e jurara; e estavam em grande aperto.
\par 16 Suscitou o SENHOR juízes, que os livraram da mão dos que os pilharam.
\par 17 Contudo, não obedeceram aos seus juízes; antes, se prostituíram após outros deuses e os adoraram. Depressa se desviaram do caminho por onde andaram seus pais na obediência dos mandamentos do SENHOR; e não fizeram como eles.
\par 18 Quando o SENHOR lhes suscitava juízes, o SENHOR era com o juiz e os livrava da mão dos seus inimigos, todos os dias daquele juiz; porquanto o SENHOR se compadecia deles ante os seus gemidos, por causa dos que os apertavam e oprimiam.
\par 19 Sucedia, porém, que, falecendo o juiz, reincidiam e se tornavam piores do que seus pais, seguindo após outros deuses, servindo-os e adorando-os eles; nada deixavam das suas obras, nem da obstinação dos seus caminhos.
\par 20 Pelo que a ira do SENHOR se acendeu contra Israel; e disse: Porquanto este povo transgrediu a minha aliança que eu ordenara a seus pais e não deu ouvidos à minha voz,
\par 21 também eu não expulsarei mais de diante dele nenhuma das nações que Josué deixou quando morreu;
\par 22 para, por elas, pôr Israel à prova, se guardará ou não o caminho do SENHOR, como seus pais o guardaram.
\par 23 Assim, o SENHOR deixou ficar aquelas nações e não as expulsou logo, nem as entregou na mão de Josué.

\chapter{3}

\par 1 São estas as nações que o SENHOR deixou para, por elas, provar a Israel, isto é, provar quantos em Israel não sabiam de todas as guerras de Canaã.
\par 2 Isso tão-somente para que as gerações dos filhos de Israel delas soubessem (para lhes ensinar a guerra), pelo menos as gerações que, dantes, não sabiam disso:
\par 3 cinco príncipes dos filisteus, e todos os cananeus, e sidônios, e heveus que habitavam as montanhas do Líbano, desde o monte de Baal-Hermom até à entrada de Hamate.
\par 4 Estes ficaram para, por eles, o SENHOR pôr Israel à prova, para saber se dariam ouvidos aos mandamentos que havia ordenado a seus pais por intermédio de Moisés.
\par 5 Habitando, pois, os filhos de Israel no meio dos cananeus, dos heteus, e amorreus, e ferezeus, e heveus, e jebuseus,
\par 6 tomaram de suas filhas para si por mulheres e deram as suas próprias aos filhos deles; e rendiam culto a seus deuses.
\par 7 Os filhos de Israel fizeram o que era mau perante o SENHOR e se esqueceram do SENHOR, seu Deus; e renderam culto aos baalins e ao poste-ídolo.
\par 8 Então, a ira do SENHOR se acendeu contra Israel, e ele os entregou nas mãos de Cusã-Risataim, rei da Mesopotâmia; e os filhos de Israel serviram a Cusã-Risataim oito anos.
\par 9 Clamaram ao SENHOR os filhos de Israel, e o SENHOR lhes suscitou libertador, que os libertou: Otniel, filho de Quenaz, que era irmão de Calebe e mais novo do que ele.
\par 10 Veio sobre ele o Espírito do SENHOR, e ele julgou a Israel; saiu à peleja, e o SENHOR lhe entregou nas mãos a Cusã-Risataim, rei da Mesopotâmia, contra o qual ele prevaleceu.
\par 11 Então, a terra ficou em paz durante quarenta anos. Otniel, filho de Quenaz, faleceu.
\par 12 Tornaram, então, os filhos de Israel a fazer o que era mau perante o SENHOR; mas o SENHOR deu poder a Eglom, rei dos moabitas, contra Israel, porquanto fizeram o que era mau perante o SENHOR.
\par 13 E ajuntou consigo os filhos de Amom e os amalequitas, e foi, e feriu a Israel; e apoderaram-se da cidade das Palmeiras.
\par 14 E os filhos de Israel serviram a Eglom, rei dos moabitas, dezoito anos.
\par 15 Então, os filhos de Israel clamaram ao SENHOR, e o SENHOR lhes suscitou libertador: Eúde, homem canhoto, filho de Gera, benjamita. Por intermédio dele, enviaram os filhos de Israel tributo a Eglom, rei dos moabitas.
\par 16 Eúde fez para si um punhal de dois gumes, do comprimento de um côvado; e cingiu-o debaixo das suas vestes, do lado direito.
\par 17 Levou o tributo a Eglom, rei dos moabitas; era Eglom homem gordo.
\par 18 Tendo entregado o tributo, despediu a gente que o trouxera e saiu com ela.
\par 19 Porém voltou do ponto em que estavam as imagens de escultura ao pé de Gilgal e disse ao rei: Tenho uma palavra secreta a dizer-te, ó rei. O rei disse: Cala-te. Então, todos os que lhe assistiam saíram de sua presença.
\par 20 Eúde entrou numa sala de verão, que o rei tinha só para si, onde estava assentado, e disse: Tenho a dizer-te uma palavra de Deus. E Eglom se levantou da cadeira.
\par 21 Então, Eúde, estendendo a mão esquerda, puxou o seu punhal do lado direito e lho cravou no ventre,
\par 22 de tal maneira que entrou também o cabo com a lâmina, e, porque não o retirou do ventre, a gordura se fechou sobre ele; e Eúde, saindo por um postigo,
\par 23 passou para o vestíbulo, depois de cerrar sobre ele as portas, trancando-as.
\par 24 Tendo saído, vieram os servos do rei e viram, e eis que as portas da sala de verão estavam trancadas; e disseram: Sem dúvida está ele aliviando o ventre na privada da sala de verão.
\par 25 Aborreceram-se de esperar; e, como não abria a porta da sala, tomaram da chave e a abriram; e eis seu senhor estendido morto em terra.
\par 26 Eúde escapou enquanto eles se demoravam e, tendo passado pelas imagens de escultura, foi para Seirá.
\par 27 Tendo ele chegado, tocou a trombeta nas montanhas de Efraim; e os filhos de Israel desceram com ele das montanhas, indo ele à frente.
\par 28 E lhes disse: Segui-me, porque o SENHOR entregou nas vossas mãos os vossos inimigos, os moabitas; e desceram após ele, e tomaram os vaus do Jordão contra os moabitas, e a nenhum deles deixaram passar.
\par 29 Naquele tempo, feriram dos moabitas uns dez mil homens, todos robustos e valentes; e não escapou nem sequer um.
\par 30 Assim, foi Moabe subjugado, naquele dia, sob o poder de Israel; e a terra ficou em paz oitenta anos.
\par 31 Depois dele, foi Sangar, filho de Anate, que feriu seiscentos homens dos filisteus com uma aguilhada de bois; e também ele libertou a Israel.

\chapter{4}

\par 1 Os filhos de Israel tornaram a fazer o que era mau perante o SENHOR, depois de falecer Eúde.
\par 2 Entregou-os o SENHOR nas mãos de Jabim, rei de Canaã, que reinava em Hazor. Sísera era o comandante do seu exército, o qual, então, habitava em Harosete-Hagoim.
\par 3 Clamaram os filhos de Israel ao SENHOR, porquanto Jabim tinha novecentos carros de ferro e, por vinte anos, oprimia duramente os filhos de Israel.
\par 4 Débora, profetisa, mulher de Lapidote, julgava a Israel naquele tempo.
\par 5 Ela atendia debaixo da palmeira de Débora, entre Ramá e Betel, na região montanhosa de Efraim; e os filhos de Israel subiam a ela a juízo.
\par 6 Mandou ela chamar a Baraque, filho de Abinoão, de Quedes de Naftali, e disse-lhe: Porventura, o SENHOR, Deus de Israel, não deu ordem, dizendo: Vai, e leva gente ao monte Tabor, e toma contigo dez mil homens dos filhos de Naftali e dos filhos de Zebulom?
\par 7 E farei ir a ti para o ribeiro Quisom a Sísera, comandante do exército de Jabim, com os seus carros e as suas tropas; e o darei nas tuas mãos.
\par 8 Então, lhe disse Baraque: Se fores comigo, irei; porém, se não fores comigo, não irei.
\par 9 Ela respondeu: Certamente, irei contigo, porém não será tua a honra da investida que empreendes; pois às mãos de uma mulher o SENHOR entregará a Sísera. E saiu Débora e se foi com Baraque para Quedes.
\par 10 Então, Baraque convocou a Zebulom e a Naftali em Quedes, e com ele subiram dez mil homens; e Débora também subiu com ele.
\par 11 Ora, Héber, queneu, se tinha apartado dos queneus, dos filhos de Hobabe, sogro de Moisés, e havia armado as suas tendas até ao carvalho de Zaananim, que está junto a Quedes.
\par 12 Anunciaram a Sísera que Baraque, filho de Abinoão, tinha subido ao monte Tabor.
\par 13 Sísera convocou todos os seus carros, novecentos carros de ferro, e todo o povo que estava com ele, de Harosete-Hagoim para o ribeiro Quisom.
\par 14 Então, disse Débora a Baraque: Dispõe-te, porque este é o dia em que o SENHOR entregou a Sísera nas tuas mãos; porventura, o SENHOR não saiu adiante de ti? Baraque, pois, desceu do monte Tabor, e dez mil homens, após ele.
\par 15 E o SENHOR derrotou a Sísera, e todos os seus carros, e a todo o seu exército a fio de espada, diante de Baraque; e Sísera saltou do carro e fugiu a pé.
\par 16 Mas Baraque perseguiu os carros e os exércitos até Harosete-Hagoim; e todo o exército de Sísera caiu a fio de espada, sem escapar nem sequer um.
\par 17 Porém Sísera fugiu a pé para a tenda de Jael, mulher de Héber, queneu; porquanto havia paz entre Jabim, rei de Hazor, e a casa de Héber, queneu.
\par 18 Saindo Jael ao encontro de Sísera, disse-lhe: Entra, senhor meu, entra na minha tenda, não temas. Retirou-se para a sua tenda, e ela pôs sobre ele uma coberta.
\par 19 Então, ele lhe disse: Dá-me, peço-te, de beber um pouco de água, porque tenho sede. Ela abriu um odre de leite, e deu-lhe de beber, e o cobriu.
\par 20 E ele lhe disse mais: Põe-te à porta da tenda; e há de ser que, se vier alguém e te perguntar: Há aqui alguém?, responde: Não.
\par 21 Então, Jael, mulher de Héber, tomou uma estaca da tenda, e lançou mão de um martelo, e foi-se mansamente a ele, e lhe cravou a estaca na fonte, de sorte que penetrou na terra, estando ele em profundo sono e mui exausto; e, assim, morreu.
\par 22 E eis que, perseguindo Baraque a Sísera, Jael lhe saiu ao encontro e lhe disse: Vem, e mostrar-te-ei o homem que procuras. Ele a seguiu; e eis que Sísera jazia morto, e a estaca na fonte.
\par 23 Assim, Deus, naquele dia, humilhou a Jabim, rei de Canaã, diante dos filhos de Israel.
\par 24 E cada vez mais a mão dos filhos de Israel prevalecia contra Jabim, rei de Canaã, até que o exterminaram.

\chapter{5}

\par 1 Naquele dia, cantaram Débora e Baraque, filho de Abinoão, dizendo:
\par 2 Desde que os chefes se puseram à frente de Israel, e o povo se ofereceu voluntariamente, bendizei ao SENHOR.
\par 3 Ouvi, reis, dai ouvidos, príncipes: eu, eu mesma cantarei ao SENHOR; salmodiarei ao SENHOR, Deus de Israel.
\par 4 Saindo tu, ó SENHOR, de Seir, marchando desde o campo de Edom, a terra estremeceu; os céus gotejaram, sim, até as nuvens gotejaram águas.
\par 5 Os montes vacilaram diante do SENHOR, e até o Sinai, diante do SENHOR, Deus de Israel.
\par 6 Nos dias de Sangar, filho de Anate, nos dias de Jael, cessaram as caravanas; e os viajantes tomavam desvios tortuosos.
\par 7 Ficaram desertas as aldeias em Israel, repousaram, até que eu, Débora, me levantei, levantei-me por mãe em Israel.
\par 8 Escolheram-se deuses novos; então, a guerra estava às portas; não se via escudo nem lança entre quarenta mil em Israel.
\par 9 Meu coração se inclina para os comandantes de Israel, que, voluntariamente, se ofereceram entre o povo; bendizei ao SENHOR.
\par 10 Vós, os que cavalgais jumentas brancas, que vos assentais em juízo e que andais pelo caminho, falai disto.
\par 11 À música dos distribuidores de água, lá entre os canais dos rebanhos, falai dos atos de justiça do SENHOR, das justiças a prol de suas aldeias em Israel. Então, o povo do SENHOR pôde descer ao seu lar.
\par 12 Desperta, Débora, desperta, desperta, acorda, entoa um cântico; levanta-te, Baraque, e leva presos os que te prenderam, tu, filho de Abinoão.
\par 13 Então, desceu o restante dos nobres, o povo do SENHOR em meu auxílio contra os poderosos.
\par 14 De Efraim, cujas raízes estão na antiga região de Amaleque, desceram guerreiros; depois de ti, ó Débora, seguiu Benjamim com seus povos; de Maquir desceram comandantes, e, de Zebulom, os que levam a vara de comando.
\par 15 Também os príncipes de Issacar foram com Débora; Issacar seguiu a Baraque, em cujas pegadas foi enviado para o vale. Entre as facções de Rúben houve grande discussão.
\par 16 Por que ficaste entre os currais para ouvires a flauta? Entre as facções de Rúben houve grande discussão.
\par 17 Gileade ficou dalém do Jordão, e Dã, por que se deteve junto a seus navios? Aser se assentou nas costas do mar e repousou nas suas baías.
\par 18 Zebulom é povo que expôs a sua vida à morte, como também Naftali, nas alturas do campo.
\par 19 Vieram reis e pelejaram; pelejaram os reis de Canaã em Taanaque, junto às águas de Megido; porém não levaram nenhum despojo de prata.
\par 20 Desde os céus pelejaram as estrelas contra Sísera, desde a sua órbita o fizeram.
\par 21 O ribeiro Quisom os arrastou, Quisom, o ribeiro das batalhas. Avante, ó minha alma, firme!
\par 22 Então, as unhas dos cavalos socavam pelo galopar, o galopar dos seus guerreiros.
\par 23 Amaldiçoai a Meroz, diz o Anjo do SENHOR, amaldiçoai duramente os seus moradores, porque não vieram em socorro do SENHOR, em socorro do SENHOR e seus heróis.
\par 24 Bendita seja sobre as mulheres Jael, mulher de Héber, o queneu; bendita seja sobre as mulheres que vivem em tendas.
\par 25 Água pediu ele, leite lhe deu ela; em taça de príncipes lhe ofereceu nata.
\par 26 À estaca estendeu a mão e, ao maço dos trabalhadores, a direita; e deu o golpe em Sísera, rachou-lhe a cabeça, furou e traspassou-lhe as fontes.
\par 27 Aos pés dela se encurvou, caiu e ficou estirado; a seus pés se encurvou e caiu; onde se encurvou, ali caiu morto.
\par 28 A mãe de Sísera olhava pela janela e exclamava pela grade: Por que tarda em vir o seu carro? Por que se demoram os passos dos seus cavalos?
\par 29 As mais sábias das suas damas respondem, e até ela a si mesma respondia:
\par 30 Porventura, não achariam e repartiriam despojos? Uma ou duas moças, a cada homem? Para Sísera, estofos de várias cores, estofos de várias cores de bordados; um ou dois estofos bordados, para o pescoço da esposa?
\par 31 Assim, ó SENHOR, pereçam todos os teus inimigos! Porém os que te amam brilham como o sol quando se levanta no seu esplendor. E a terra ficou em paz quarenta anos.

\chapter{6}

\par 1 Fizeram os filhos de Israel o que era mau perante o SENHOR; por isso, o SENHOR os entregou nas mãos dos midianitas por sete anos.
\par 2 Prevalecendo o domínio dos midianitas sobre Israel, fizeram estes para si, por causa dos midianitas, as covas que estão nos montes, e as cavernas, e as fortificações.
\par 3 Porque, cada vez que Israel semeava, os midianitas e os amalequitas, como também os povos do Oriente, subiam contra ele.
\par 4 E contra ele se acampavam, destruindo os produtos da terra até à vizinhança de Gaza, e não deixavam em Israel sustento algum, nem ovelhas, nem bois, nem jumentos.
\par 5 Pois subiam com os seus gados e tendas e vinham como gafanhotos, em tanta multidão, que não se podiam contar, nem a eles nem aos seus camelos; e entravam na terra para a destruir.
\par 6 Assim, Israel ficou muito debilitado com a presença dos midianitas; então, os filhos de Israel clamavam ao SENHOR.
\par 7 Tendo os filhos de Israel clamado ao SENHOR, por causa dos midianitas,
\par 8 o SENHOR lhes enviou um profeta, que lhes disse: Assim diz o SENHOR, Deus de Israel: Eu é que vos fiz subir do Egito e vos tirei da casa da servidão;
\par 9 e vos livrei da mão dos egípcios e da mão de todos quantos vos oprimiam; e os expulsei de diante de vós e vos dei a sua terra;
\par 10 e disse: Eu sou o SENHOR, vosso Deus; não temais os deuses dos amorreus, em cuja terra habitais; contudo, não destes ouvidos à minha voz.
\par 11 Então, veio o Anjo do SENHOR, e assentou-se debaixo do carvalho que está em Ofra, que pertencia a Joás, abiezrita; e Gideão, seu filho, estava malhando o trigo no lagar, para o pôr a salvo dos midianitas.
\par 12 Então, o Anjo do SENHOR lhe apareceu e lhe disse: O SENHOR é contigo, homem valente.
\par 13 Respondeu-lhe Gideão: Ai, senhor meu! Se o SENHOR é conosco, por que nos sobreveio tudo isto? E que é feito de todas as suas maravilhas que nossos pais nos contaram, dizendo: Não nos fez o SENHOR subir do Egito? Porém, agora, o SENHOR nos desamparou e nos entregou nas mãos dos midianitas.
\par 14 Então, se virou o SENHOR para ele e disse: Vai nessa tua força e livra Israel da mão dos midianitas; porventura, não te enviei eu?
\par 15 E ele lhe disse: Ai, Senhor meu! Com que livrarei Israel? Eis que a minha família é a mais pobre em Manassés, e eu, o menor na casa de meu pai.
\par 16 Tornou-lhe o SENHOR: Já que eu estou contigo, ferirás os midianitas como se fossem um só homem.
\par 17 Ele respondeu: Se, agora, achei mercê diante dos teus olhos, dá-me um sinal de que és tu, SENHOR, que me falas.
\par 18 Rogo-te que daqui não te apartes até que eu volte, e traga a minha oferta, e a deponha perante ti. Respondeu ele: Esperarei até que voltes.
\par 19 Entrou Gideão e preparou um cabrito e bolos asmos de um efa de farinha; a carne pôs num cesto, e o caldo, numa panela; e trouxe-lho até debaixo do carvalho e lho apresentou.
\par 20 Porém o Anjo de Deus lhe disse: Toma a carne e os bolos asmos, põe-nos sobre esta penha e derrama-lhes por cima o caldo. E assim o fez.
\par 21 Estendeu o Anjo do SENHOR a ponta do cajado que trazia na mão e tocou a carne e os bolos asmos; então, subiu fogo da penha e consumiu a carne e os bolos; e o Anjo do SENHOR desapareceu de sua presença.
\par 22 Viu Gideão que era o Anjo do SENHOR e disse: Ai de mim, SENHOR Deus! Pois vi o Anjo do SENHOR face a face.
\par 23 Porém o SENHOR lhe disse: Paz seja contigo! Não temas! Não morrerás!
\par 24 Então, Gideão edificou ali um altar ao SENHOR e lhe chamou de O SENHOR É Paz. Ainda até ao dia de hoje está o altar em Ofra, que pertence aos abiezritas.
\par 25 Naquela mesma noite, lhe disse o SENHOR: Toma um boi que pertence a teu pai, a saber, o segundo boi de sete anos, e derriba o altar de Baal que é de teu pai, e corta o poste-ídolo que está junto ao altar.
\par 26 Edifica ao SENHOR, teu Deus, um altar no cimo deste baluarte, em camadas de pedra, e toma o segundo boi, e o oferecerás em holocausto com a lenha do poste-ídolo que vieres a cortar.
\par 27 Então, Gideão tomou dez homens dentre os seus servos e fez como o SENHOR lhe dissera; temendo ele, porém, a casa de seu pai e os homens daquela cidade, não o fez de dia, mas de noite.
\par 28 Levantando-se, pois, de madrugada, os homens daquela cidade, eis que estava o altar de Baal derribado, e o poste-ídolo que estava junto dele, cortado; e o referido segundo boi fora oferecido no altar edificado.
\par 29 E uns aos outros diziam: Quem fez isto? E, perguntando e inquirindo, disseram: Gideão, o filho de Joás, fez esta coisa.
\par 30 Então, os homens daquela cidade disseram a Joás: Leva para fora o teu filho, para que morra; pois derribou o altar de Baal e cortou o poste-ídolo que estava junto dele.
\par 31 Porém Joás disse a todos os que se puseram contra ele: Contendereis vós por Baal? Livrá-lo-eis vós? Qualquer que por ele contender, ainda esta manhã, será morto. Se é deus, que por si mesmo contenda; pois derribaram o seu altar.
\par 32 Naquele dia, Gideão passou a ser chamado Jerubaal, porque foi dito: Baal contenda contra ele, pois ele derribou o seu altar.
\par 33 E todos os midianitas, e amalequitas, e povos do Oriente se ajuntaram, e passaram, e se acamparam no vale de Jezreel.
\par 34 Então, o Espírito do SENHOR revestiu a Gideão, o qual tocou a rebate, e os abiezritas se ajuntaram após dele.
\par 35 Enviou mensageiros por toda a tribo de Manassés, que também foi convocada para o seguir; enviou ainda mensageiros a Aser, e a Zebulom, e a Naftali, e saíram para encontrar-se com ele.
\par 36 Disse Gideão a Deus: Se hás de livrar a Israel por meu intermédio, como disseste,
\par 37 eis que eu porei uma porção de lã na eira; se o orvalho estiver somente nela, e seca a terra ao redor, então, conhecerei que hás de livrar Israel por meu intermédio, como disseste.
\par 38 E assim sucedeu, porque, ao outro dia, se levantou de madrugada e, apertando a lã, do orvalho dela espremeu uma taça cheia de água.
\par 39 Disse mais Gideão: Não se acenda contra mim a tua ira, se ainda falar só esta vez; rogo-te que mais esta vez faça eu a prova com a lã; que só a lã esteja seca, e na terra ao redor haja orvalho.
\par 40 E Deus assim o fez naquela noite, pois só a lã estava seca, e sobre a terra ao redor havia orvalho.

\chapter{7}

\par 1 Então, Jerubaal, que é Gideão, se levantou de madrugada, e todo o povo que com ele estava, e se acamparam junto à fonte de Harode, de maneira que o arraial dos midianitas lhe ficava para o norte, no vale, defronte do outeiro de Moré.
\par 2 Disse o SENHOR a Gideão: É demais o povo que está contigo, para eu entregar os midianitas nas suas mãos; Israel poderia se gloriar contra mim, dizendo: A minha própria mão me livrou.
\par 3 Apregoa, pois, aos ouvidos do povo, dizendo: Quem for tímido e medroso, volte e retire-se da região montanhosa de Gileade. Então, voltaram do povo vinte e dois mil, e dez mil ficaram.
\par 4 Disse mais o SENHOR a Gideão: Ainda há povo demais; faze-os descer às águas, e ali tos provarei; aquele de quem eu te disser: este irá contigo, esse contigo irá; porém todo aquele de quem eu te disser: este não irá contigo, esse não irá.
\par 5 Fez Gideão descer os homens às águas. Então, o SENHOR lhe disse: Todo que lamber a água com a língua, como faz o cão, esse porás à parte, como também a todo aquele que se abaixar de joelhos a beber.
\par 6 Foi o número dos que lamberam, levando a mão à boca, trezentos homens; e todo o restante do povo se abaixou de joelhos a beber a água.
\par 7 Então, disse o SENHOR a Gideão: Com estes trezentos homens que lamberam a água eu vos livrarei, e entregarei os midianitas nas tuas mãos; pelo que a outra gente toda que se retire, cada um para o seu lugar.
\par 8 Tomou o povo provisões nas mãos e as trombetas. Gideão enviou todos os homens de Israel cada um à sua tenda, porém os trezentos homens reteve consigo. Estava o arraial dos midianitas abaixo dele, no vale.
\par 9 Sucedeu que, naquela mesma noite, o SENHOR lhe disse: Levanta-te e desce contra o arraial, porque o entreguei nas tuas mãos.
\par 10 Se ainda temes atacar, desce tu com teu moço Pura ao arraial;
\par 11 e ouvirás o que dizem; depois, fortalecidas as tuas mãos, descerás contra o arraial. Então, desceu ele com seu moço Pura até à vanguarda do arraial.
\par 12 Os midianitas, os amalequitas e todos os povos do Oriente cobriam o vale como gafanhotos em multidão; e eram os seus camelos em multidão inumerável como a areia que há na praia do mar.
\par 13 Chegando, pois, Gideão, eis que certo homem estava contando um sonho ao seu companheiro e disse: Tive um sonho. Eis que um pão de cevada rodava contra o arraial dos midianitas e deu de encontro à tenda do comandante, de maneira que esta caiu, e se virou de cima para baixo, e ficou assim estendida.
\par 14 Respondeu-lhe o companheiro e disse: Não é isto outra coisa, senão a espada de Gideão, filho de Joás, homem israelita. Nas mãos dele entregou Deus os midianitas e todo este arraial.
\par 15 Tendo ouvido Gideão contar este sonho e o seu significado, adorou; e tornou ao arraial de Israel e disse: Levantai-vos, porque o SENHOR entregou o arraial dos midianitas nas vossas mãos.
\par 16 Então, repartiu os trezentos homens em três companhias e deu-lhes, a cada um nas suas mãos, trombetas e cântaros vazios, com tochas neles.
\par 17 E disse-lhes: Olhai para mim e fazei como eu fizer. Chegando eu às imediações do arraial, como fizer eu, assim fareis.
\par 18 Quando eu tocar a trombeta, e todos os que comigo estiverem, então, vós também tocareis a vossa ao redor de todo o arraial e direis: Pelo SENHOR e por Gideão!
\par 19 Chegou, pois, Gideão e os cem homens que com ele iam às imediações do arraial, ao princípio da vigília média, havendo-se pouco tempo antes trocado as guardas; e tocaram as trombetas e quebraram os cântaros que traziam nas mãos.
\par 20 Assim, tocaram as três companhias as trombetas e despedaçaram os cântaros; e seguravam na mão esquerda as tochas e na mão direita, as trombetas que tocavam; e exclamaram: Espada pelo SENHOR e por Gideão!
\par 21 E permaneceu cada um no seu lugar ao redor do arraial, que todo deitou a correr, e a gritar, e a fugir.
\par 22 Ao soar das trezentas trombetas, o SENHOR tornou a espada de um contra o outro, e isto em todo o arraial, que fugiu rumo de Zererá, até Bete-Sita, até ao limite de Abel-Meolá, acima de Tabate.
\par 23 Então, os homens de Israel, de Naftali e de Aser e de todo o Manassés foram convocados e perseguiram os midianitas.
\par 24 Gideão enviou mensageiros a todas as montanhas de Efraim, dizendo: Descei de encontro aos midianitas e impedi-lhes a passagem pelas águas do Jordão até Bete-Bara. Convocados, pois, todos os homens de Efraim, cortaram-lhes a passagem pelo Jordão, até Bete-Bara.
\par 25 E prenderam a dois príncipes dos midianitas, Orebe e Zeebe; mataram Orebe na penha de Orebe e Zeebe mataram no lagar de Zeebe. Perseguiram aos midianitas e trouxeram as cabeças de Orebe e de Zeebe a Gideão, dalém do Jordão.

\chapter{8}

\par 1 Então, os homens de Efraim disseram a Gideão: Que é isto que nos fizeste, que não nos chamaste quando foste pelejar contra os midianitas? E contenderam fortemente com ele.
\par 2 Porém ele lhes disse: Que mais fiz eu, agora, do que vós? Não são, porventura, os rabiscos de Efraim melhores do que a vindima de Abiezer?
\par 3 Deus entregou nas vossas mãos os príncipes dos midianitas, Orebe e Zeebe; que pude eu fazer comparável com o que fizestes? Então, com falar-lhes esta palavra, abrandou-se-lhes a ira para com ele.
\par 4 Vindo Gideão ao Jordão, passou com os trezentos homens que com ele estavam, cansados mas ainda perseguindo.
\par 5 E disse aos homens de Sucote: Dai, peço-vos, alguns pães para estes que me seguem, pois estão cansados, e eu vou ao encalço de Zeba e Salmuna, reis dos midianitas.
\par 6 Porém os príncipes de Sucote disseram: Porventura, tens já sob teu poder o punho de Zeba e de Salmuna, para que demos pão ao teu exército?
\par 7 Então, disse Gideão: Por isso, quando o SENHOR entregar nas minhas mãos Zeba e Salmuna, trilharei a vossa carne com os espinhos do deserto e com os abrolhos.
\par 8 Dali subiu a Penuel e de igual modo falou a seus homens; estes de Penuel lhe responderam como os homens de Sucote lhe haviam respondido.
\par 9 Pelo que também falou aos homens de Penuel, dizendo: Quando eu voltar em paz, derribarei esta torre.
\par 10 Estavam, pois, Zeba e Salmuna em Carcor, e os seus exércitos, com eles, uns quinze mil homens, todos os que restaram do exército de povos do Oriente; e os que caíram foram cento e vinte mil homens que puxavam da espada.
\par 11 Subiu Gideão pelo caminho dos nômades, ao oriente de Noba e Jogbeá, e feriu aquele exército, que se achava descuidado.
\par 12 Fugiram Zeba e Salmuna; porém ele os perseguiu, e prendeu os dois reis dos midianitas, Zeba e Salmuna, e desbaratou todo o exército.
\par 13 Voltando, pois, Gideão, filho de Joás, da peleja, pela subida de Heres,
\par 14 deteve a um moço de Sucote e lhe fez perguntas; o moço deu por escrito o nome dos príncipes e anciãos de Sucote, setenta e sete homens.
\par 15 Então, veio Gideão aos homens de Sucote e disse: Vedes aqui Zeba e Salmuna, a respeito dos quais motejastes de mim, dizendo: Porventura, tens tu já sob teu poder o punho de Zeba e Salmuna para que demos pão aos teus homens cansados?
\par 16 E tomou os anciãos da cidade, e espinhos do deserto, e abrolhos e, com eles, deu severa lição aos homens de Sucote.
\par 17 Derribou a torre de Penuel e matou os homens da cidade.
\par 18 Depois disse a Zeba e a Salmuna: Que homens eram os que matastes em Tabor? Responderam: Como tu és, assim eram eles; cada um se assemelhava a filho de rei.
\par 19 Então, disse ele: Eram meus irmãos, filhos de minha mãe. Tão certo como vive o SENHOR, se os tivésseis deixado com vida, eu não vos mataria a vós outros.
\par 20 E disse a Jéter, seu primogênito: Dispõe-te e mata-os. Porém o moço não arrancou da sua espada, porque temia, porquanto ainda era jovem.
\par 21 Então, disseram Zeba e Salmuna: Levanta-te e arremete contra nós, porque qual o homem, tal a sua valentia. Dispôs-se, pois, Gideão, e matou a Zeba e a Salmuna, e tomou os ornamentos em forma de meia-lua que estavam no pescoço dos seus camelos.
\par 22 Então, os homens de Israel disseram a Gideão: Domina sobre nós, tanto tu como teu filho e o filho de teu filho, porque nos livraste do poder dos midianitas.
\par 23 Porém Gideão lhes disse: Não dominarei sobre vós, nem tampouco meu filho dominará sobre vós; o SENHOR vos dominará.
\par 24 Disse-lhes mais Gideão: Um pedido vos farei: dai-me vós, cada um as argolas do seu despojo (porque tinham argolas de ouro, pois eram ismaelitas).
\par 25 Disseram eles: De bom grado as daremos. E estenderam uma capa, e cada um deles deitou ali uma argola do seu despojo.
\par 26 O peso das argolas de ouro que pediu foram mil e setecentos siclos de ouro (afora os ornamentos em forma de meia-lua, as arrecadas e as vestes de púrpura que traziam os reis dos midianitas, e afora os ornamentos que os camelos traziam ao pescoço).
\par 27 Desse peso fez Gideão uma estola sacerdotal e a pôs na sua cidade, em Ofra; e todo o Israel se prostituiu ali após ela; a estola veio a ser um laço a Gideão e à sua casa.
\par 28 Assim, foram abatidos os midianitas diante dos filhos de Israel e nunca mais levantaram a cabeça; e ficou a terra em paz durante quarenta anos nos dias de Gideão.
\par 29 Retirou-se Jerubaal, filho de Joás, e habitou em sua casa.
\par 30 Teve Gideão setenta filhos, todos provindos dele, porque tinha muitas mulheres.
\par 31 A sua concubina, que estava em Siquém, lhe deu também à luz um filho; e ele lhe pôs por nome Abimeleque.
\par 32 Faleceu Gideão, filho de Joás, em boa velhice; e foi sepultado no sepulcro de Joás, seu pai, em Ofra dos abiezritas.
\par 33 Morto Gideão, tornaram a prostituir-se os filhos de Israel após os baalins e puseram Baal-Berite por deus.
\par 34 Os filhos de Israel não se lembraram do SENHOR, seu Deus, que os livrara do poder de todos os seus inimigos ao redor;
\par 35 nem usaram de benevolência com a casa de Jerubaal, a saber, Gideão, segundo todo o bem que ele fizera a Israel.

\chapter{9}

\par 1 Abimeleque, filho de Jerubaal, foi-se a Siquém, aos irmãos de sua mãe, e falou-lhes e a toda a geração da casa do pai de sua mãe, dizendo:
\par 2 Falai, peço-vos, aos ouvidos de todos os cidadãos de Siquém: Que vos parece melhor: que setenta homens, todos os filhos de Jerubaal, dominem sobre vós ou que apenas um domine sobre vós? Lembrai-vos também de que sou osso vosso e carne vossa.
\par 3 Então, os irmãos de sua mãe falaram a todos os cidadãos de Siquém todas aquelas palavras; e o coração deles se inclinou a seguir Abimeleque, porque disseram: É nosso irmão.
\par 4 E deram-lhe setenta peças de prata, da casa de Baal-Berite, com as quais alugou Abimeleque uns homens levianos e atrevidos, que o seguiram.
\par 5 Foi à casa de seu pai, a Ofra, e matou seus irmãos, os filhos de Jerubaal, setenta homens, sobre uma pedra. Porém Jotão, filho menor de Jerubaal, ficou, porque se escondera.
\par 6 Então, se ajuntaram todos os cidadãos de Siquém e toda Bete-Milo; e foram e proclamaram Abimeleque rei, junto ao carvalho memorial que está perto de Siquém.
\par 7 Avisado disto, Jotão foi, e se pôs no cimo do monte Gerizim, e em alta voz clamou, e disse-lhes: Ouvi-me, cidadãos de Siquém, e Deus vos ouvirá a vós outros.
\par 8 Foram, certa vez, as árvores ungir para si um rei e disseram à oliveira: Reina sobre nós.
\par 9 Porém a oliveira lhes respondeu: Deixaria eu o meu óleo, que Deus e os homens em mim prezam, e iria pairar sobre as árvores?
\par 10 Então, disseram as árvores à figueira: Vem tu e reina sobre nós.
\par 11 Porém a figueira lhes respondeu: Deixaria eu a minha doçura, o meu bom fruto e iria pairar sobre as árvores?
\par 12 Então, disseram as árvores à videira: Vem tu e reina sobre nós.
\par 13 Porém a videira lhes respondeu: Deixaria eu o meu vinho, que agrada a Deus e aos homens, e iria pairar sobre as árvores?
\par 14 Então, todas as árvores disseram ao espinheiro: Vem tu e reina sobre nós.
\par 15 Respondeu o espinheiro às árvores: Se, deveras, me ungis rei sobre vós, vinde e refugiai-vos debaixo de minha sombra; mas, se não, saia do espinheiro fogo que consuma os cedros do Líbano.
\par 16 Agora, pois, se, deveras e sinceramente, procedestes, proclamando rei Abimeleque, e se bem vos portastes para com Jerubaal e para com a sua casa, e se com ele agistes segundo o merecimento dos seus feitos
\par 17 (porque meu pai pelejou por vós e, arriscando a vida, vos livrou das mãos dos midianitas;
\par 18 porém vós, hoje, vos levantastes contra a casa de meu pai e matastes seus filhos, setenta homens, sobre uma pedra; e a Abimeleque, filho de sua serva, fizestes reinar sobre os cidadãos de Siquém, porque é vosso irmão),
\par 19 se, deveras e sinceramente, procedestes, hoje, com Jerubaal e com a sua casa, alegrai-vos com Abimeleque, e também ele se alegre convosco.
\par 20 Mas, se não, saia fogo de Abimeleque e consuma os cidadãos de Siquém e Bete-Milo; e saia fogo dos cidadãos de Siquém e de Bete-Milo, que consuma a Abimeleque.
\par 21 Fugiu logo Jotão, e foi-se para Beer, e ali habitou por temer seu irmão Abimeleque.
\par 22 Havendo, pois, Abimeleque dominado três anos sobre Israel,
\par 23 suscitou Deus um espírito de aversão entre Abimeleque e os cidadãos de Siquém; e estes se houveram aleivosamente contra Abimeleque,
\par 24 para que a vingança da violência praticada contra os setenta filhos de Jerubaal viesse, e o seu sangue caísse sobre Abimeleque, seu irmão, que os matara, e sobre os cidadãos de Siquém, que contribuíram para que ele matasse seus próprios irmãos.
\par 25 Os cidadãos de Siquém puseram contra ele homens de emboscada sobre os cimos dos montes; e todo aquele que passava pelo caminho junto a eles, eles o assaltavam; e isto se contou a Abimeleque.
\par 26 Veio também Gaal, filho de Ebede, com seus irmãos, e se estabeleceram em Siquém; e os cidadãos de Siquém confiaram nele,
\par 27 e saíram ao campo, e vindimaram as suas vinhas, e pisaram as uvas, e fizeram festas, e foram à casa de seu deus, e comeram, e beberam, e amaldiçoaram Abimeleque.
\par 28 Disse Gaal, filho de Ebede: Quem é Abimeleque, e quem somos nós de Siquém, para que o sirvamos? Não é, porventura, filho de Jerubaal? E não é Zebul o seu oficial? Servi, antes, aos homens de Hamor, pai de Siquém. Mas nós, por que serviremos a ele?
\par 29 Quem dera estivesse este povo sob a minha mão, e eu expulsaria Abimeleque e lhe diria: Multiplica o teu exército e sai.
\par 30 Ouvindo Zebul, governador da cidade, as palavras de Gaal, filho de Ebede, se acendeu em ira;
\par 31 e enviou, astutamente, mensageiros a Abimeleque, dizendo: Eis que Gaal, filho de Ebede, e seus irmãos vieram a Siquém e alvoroçaram a cidade contra ti.
\par 32 Levanta-te, pois, de noite, tu e o povo que tiveres contigo, e ponde-vos de emboscada no campo.
\par 33 Levanta-te pela manhã, ao sair o sol, e dá de golpe sobre a cidade; saindo contra ti Gaal com a sua gente, procede com ele como puderes.
\par 34 Levantou-se, pois, de noite, Abimeleque e todo o povo que com ele estava, e se puseram de emboscada contra Siquém, em quatro grupos.
\par 35 Gaal, filho de Ebede, saiu e pôs-se à entrada da porta da cidade; com isto Abimeleque e todo o povo que com ele estava se levantaram das emboscadas.
\par 36 Vendo Gaal aquele povo, disse a Zebul: Eis que desce gente dos cimos dos montes. Zebul, ao contrário, lhe disse: As sombras dos montes vês por homens.
\par 37 Porém Gaal tornou ainda a falar e disse: Eis ali desce gente defronte de nós, e uma tropa vem do caminho do carvalho dos Adivinhadores.
\par 38 Então, lhe disse Zebul: Onde está, agora, a tua boca, com a qual dizias: Quem é Abimeleque, para que o sirvamos? Não é este, porventura, o povo que desprezaste? Sai, pois, e peleja contra ele.
\par 39 Saiu Gaal adiante dos cidadãos de Siquém e pelejou contra Abimeleque.
\par 40 Abimeleque o perseguiu; Gaal fugiu de diante dele, e muitos feridos caíram até a entrada da porta da cidade.
\par 41 Abimeleque ficou em Arumá. E Zebul expulsou a Gaal e seus irmãos, para que não habitassem em Siquém.
\par 42 No dia seguinte, saiu o povo ao campo; disto foi avisado Abimeleque,
\par 43 que tomou os seus homens, e os repartiu em três grupos, e os pôs de emboscada no campo. Olhando, viu que o povo saía da cidade; então, se levantou contra eles e os feriu.
\par 44 Abimeleque e o grupo que com ele estava romperam de improviso e tomaram posição à porta da cidade, enquanto os dois outros grupos deram de golpe sobre todos quantos estavam no campo e os destroçaram.
\par 45 Todo aquele dia pelejou Abimeleque contra a cidade e a tomou. Matou o povo que nela havia, assolou-a e a semeou de sal.
\par 46 Tendo ouvido isto todos os cidadãos da Torre de Siquém, entraram na fortaleza subterrânea, no templo de El-Berite.
\par 47 Contou-se a Abimeleque que todos os cidadãos da Torre de Siquém se haviam congregado.
\par 48 Então, subiu ele ao monte Salmom, ele e todo o seu povo; Abimeleque tomou de um machado, e cortou uma ramada de árvore, e a levantou, e pô-la ao ombro, e disse ao povo que com ele estava: O que me vistes fazer, fazei-o também vós, depressa.
\par 49 Assim, cada um de todo o povo cortou a sua ramada, e seguiram Abimeleque, e as puseram em cima da fortaleza subterrânea, e queimaram sobre todos os da Torre de Siquém, de maneira que morreram todos, uns mil homens e mulheres.
\par 50 Então, se foi Abimeleque a Tebes, e a sitiou, e a tomou.
\par 51 Havia, porém, no meio da cidade, uma torre forte; e todos os homens e mulheres, todos os moradores da cidade, se acolheram a ela, e fecharam após si as portas da torre, e subiram ao seu eirado.
\par 52 Abimeleque veio até à torre, pelejou contra ela e se chegou até à sua porta para a incendiar.
\par 53 Porém certa mulher lançou uma pedra superior de moinho sobre a cabeça de Abimeleque e lhe quebrou o crânio.
\par 54 Então, chamou logo ao moço, seu escudeiro, e lhe disse: Desembainha a tua espada e mata-me, para que não se diga de mim: Mulher o matou. O moço o atravessou, e ele morreu.
\par 55 Vendo, pois, os homens de Israel que Abimeleque já estava morto, foram-se, cada um para sua casa.
\par 56 Assim, Deus fez tornar sobre Abimeleque o mal que fizera a seu pai, por ter aquele matado os seus setenta irmãos.
\par 57 De igual modo, todo o mal dos homens de Siquém Deus fez cair sobre a cabeça deles. Assim, veio sobre eles a maldição de Jotão, filho de Jerubaal.

\chapter{10}

\par 1 Depois de Abimeleque, se levantou, para livrar Israel, Tola, filho de Puá, filho de Dodô, homem de Issacar; e habitava em Samir, na região montanhosa de Efraim.
\par 2 Julgou a Israel vinte e três anos, e morreu, e foi sepultado em Samir.
\par 3 Depois dele, se levantou Jair, gileadita, e julgou a Israel vinte e dois anos.
\par 4 Tinha este trinta filhos, que cavalgavam trinta jumentos; e tinham trinta cidades, a que chamavam Havote-Jair, até ao dia de hoje, as quais estão na terra de Gileade.
\par 5 Morreu Jair e foi sepultado em Camom.
\par 6 Tornaram os filhos de Israel a fazer o que era mau perante o SENHOR e serviram aos baalins, e a Astarote, e aos deuses da Síria, e aos de Sidom, de Moabe, dos filhos de Amom e dos filisteus; deixaram o SENHOR e não o serviram.
\par 7 Acendeu-se a ira do SENHOR contra Israel, e entregou-os nas mãos dos filisteus e nas mãos dos filhos de Amom,
\par 8 os quais, nesse mesmo ano, vexaram e oprimiram os filhos de Israel. Por dezoito anos, oprimiram a todos os filhos de Israel que estavam dalém do Jordão, na terra dos amorreus, que está em Gileade.
\par 9 Os filhos de Amom passaram o Jordão para pelejar também contra Judá, e contra Benjamim, e contra a casa de Efraim, de maneira que Israel se viu muito angustiado.
\par 10 Então, os filhos de Israel clamaram ao SENHOR, dizendo: Contra ti havemos pecado, porque deixamos o nosso Deus e servimos aos baalins.
\par 11 Porém o SENHOR disse aos filhos de Israel: Quando os egípcios, e os amorreus, e os filhos de Amom, e os filisteus,
\par 12 e os sidônios, e os amalequitas, e os maonitas vos oprimiam, e vós clamáveis a mim, não vos livrei eu das suas mãos?
\par 13 Contudo, vós me deixastes a mim e servistes a outros deuses, pelo que não vos livrarei mais.
\par 14 Ide e clamai aos deuses que escolhestes; eles que vos livrem no tempo do vosso aperto.
\par 15 Mas os filhos de Israel disseram ao SENHOR: Temos pecado; faze-nos tudo quanto te parecer bem; porém livra-nos ainda esta vez, te rogamos.
\par 16 E tiraram os deuses alheios do meio de si e serviram ao SENHOR; então, já não pôde ele reter a sua compaixão por causa da desgraça de Israel.
\par 17 Tendo sido convocados os filhos de Amom, acamparam-se em Gileade; mas os filhos de Israel se congregaram e se acamparam em Mispa.
\par 18 Então, o povo, aliás, os príncipes de Gileade, disseram uns aos outros: Quem será o homem que começará a pelejar contra os filhos de Amom? Será esse o cabeça de todos os moradores de Gileade.

\chapter{11}

\par 1 Era, então, Jefté, o gileadita, homem valente, porém filho de uma prostituta; Gileade gerara a Jefté.
\par 2 Também a mulher de Gileade lhe deu filhos, os quais, quando já grandes, expulsaram Jefté e lhe disseram: Não herdarás em casa de nosso pai, porque és filho doutra mulher.
\par 3 Então, Jefté fugiu da presença de seus irmãos e habitou na terra de Tobe; e homens levianos se ajuntaram com ele e com ele saíam.
\par 4 Passado algum tempo, pelejaram os filhos de Amom contra Israel.
\par 5 Quando pelejavam, foram os anciãos de Gileade buscar Jefté da terra de Tobe.
\par 6 E disseram a Jefté: Vem e sê nosso chefe, para que combatamos contra os filhos de Amom.
\par 7 Porém Jefté disse aos anciãos de Gileade: Porventura, não me aborrecestes a mim e não me expulsastes da casa de meu pai? Por que, pois, vindes a mim, agora, quando estais em aperto?
\par 8 Responderam os anciãos de Gileade a Jefté: Por isso mesmo, tornamos a ti. Vem, pois, conosco, e combate contra os filhos de Amom, e sê o nosso chefe sobre todos os moradores de Gileade.
\par 9 Então, Jefté perguntou aos anciãos de Gileade: Se me tornardes a levar para combater contra os filhos de Amom, e o SENHOR mos der a mim, então, eu vos serei por cabeça?
\par 10 Responderam os anciãos de Gileade a Jefté: O SENHOR será testemunha entre nós e nos castigará se não fizermos segundo a tua palavra.
\par 11 Então, Jefté foi com os anciãos de Gileade, e o povo o pôs por cabeça e chefe sobre si; e Jefté proferiu todas as suas palavras perante o SENHOR, em Mispa.
\par 12 Enviou Jefté mensageiros ao rei dos filhos de Amom, dizendo: Que há entre mim e ti que vieste a mim a pelejar contra a minha terra?
\par 13 Respondeu o rei dos filhos de Amom aos mensageiros de Jefté: É porque, subindo Israel do Egito, me tomou a terra desde Arnom até ao Jaboque e ainda até ao Jordão; restitui-ma, agora, pacificamente.
\par 14 Porém Jefté tornou a enviar mensageiros ao rei dos filhos de Amom,
\par 15 dizendo-lhe: Assim diz Jefté: Israel não tomou nem a terra dos moabitas nem a terra dos filhos de Amom;
\par 16 porque, subindo Israel do Egito, andou pelo deserto até ao mar Vermelho e chegou a Cades.
\par 17 Então, Israel enviou mensageiros ao rei dos edomitas, dizendo: Rogo-te que me deixes passar pela tua terra. Porém o rei dos edomitas não lhe deu ouvidos; a mesma coisa mandou Israel pedir ao rei dos moabitas, o qual também não lhe quis atender; e, assim, Israel ficou em Cades.
\par 18 Depois, andou pelo deserto, e rodeou a terra dos edomitas e a terra dos moabitas, e chegou ao oriente da terra destes, e se acampou além do Arnom; por isso, não entrou no território dos moabitas, porque Arnom é o limite deles.
\par 19 Mas Israel enviou mensageiros a Seom, rei dos amorreus, rei de Hesbom; e disse-lhe: Deixa-nos, peço-te, passar pela tua terra até ao meu lugar.
\par 20 Porém Seom, não confiando em Israel, recusou deixá-lo passar pelo seu território; pelo contrário, ajuntou todo o seu povo, e se acampou em Jaza, e pelejou contra Israel.
\par 21 O SENHOR, Deus de Israel, entregou Seom e todo o seu povo nas mãos de Israel, que os feriu; e Israel desapossou os amorreus das terras que habitavam.
\par 22 Tomou posse de todo o território dos amorreus, desde o Arnom até ao Jaboque e desde o deserto até ao Jordão.
\par 23 Assim, o SENHOR, Deus de Israel, desapossou os amorreus ante o seu povo de Israel. E pretendes tu ser dono desta terra?
\par 24 Não é certo que aquilo que Quemos, teu deus, te dá consideras como tua possessão? Assim, possuiremos nós o território de todos quantos o SENHOR, nosso Deus, expulsou de diante de nós.
\par 25 És tu melhor do que o filho de Zipor, Balaque, rei dos moabitas? Porventura, contendeu este, em algum tempo, com Israel ou pelejou alguma vez contra ele?
\par 26 Enquanto Israel habitou trezentos anos em Hesbom e nas suas vilas, e em Aroer e nas suas vilas, e em todas as cidades que estão ao longe do Arnom, por que vós, amonitas, não as recuperastes durante esse tempo?
\par 27 Não sou eu, portanto, quem pecou contra ti! Porém tu fazes mal em pelejar contra mim; o SENHOR, que é juiz, julgue hoje entre os filhos de Israel e os filhos de Amom.
\par 28 Porém o rei dos filhos de Amom não deu ouvidos à mensagem que Jefté lhe enviara.
\par 29 Então, o Espírito do SENHOR veio sobre Jefté; e, atravessando este por Gileade e Manassés, passou até Mispa de Gileade e de Mispa de Gileade passou contra os filhos de Amom.
\par 30 Fez Jefté um voto ao SENHOR e disse: Se, com efeito, me entregares os filhos de Amom nas minhas mãos,
\par 31 quem primeiro da porta da minha casa me sair ao encontro, voltando eu vitorioso dos filhos de Amom, esse será do SENHOR, e eu o oferecerei em holocausto.
\par 32 Assim, Jefté foi de encontro aos filhos de Amom, a combater contra eles; e o SENHOR os entregou nas mãos de Jefté.
\par 33 Este os derrotou desde Aroer até às proximidades de Minite (vinte cidades ao todo) e até Abel-Queramim; e foi mui grande a derrota. Assim, foram subjugados os filhos de Amom diante dos filhos de Israel.
\par 34 Vindo, pois, Jefté a Mispa, a sua casa, saiu-lhe a filha ao seu encontro, com adufes e com danças; e era ela filha única; não tinha ele outro filho nem filha.
\par 35 Quando a viu, rasgou as suas vestes e disse: Ah! Filha minha, tu me prostras por completo; tu passaste a ser a causa da minha calamidade, porquanto fiz voto ao SENHOR e não tornarei atrás.
\par 36 E ela lhe disse: Pai meu, fizeste voto ao SENHOR; faze, pois, de mim segundo o teu voto; pois o SENHOR te vingou dos teus inimigos, os filhos de Amom.
\par 37 Disse mais a seu pai: Concede-me isto: deixa-me por dois meses, para que eu vá, e desça pelos montes, e chore a minha virgindade, eu e as minhas companheiras.
\par 38 Consentiu ele: Vai. E deixou-a ir por dois meses; então, se foi ela com as suas companheiras e chorou a sua virgindade pelos montes.
\par 39 Ao fim dos dois meses, tornou ela para seu pai, o qual lhe fez segundo o voto por ele proferido; assim, ela jamais foi possuída por varão. Daqui veio o costume em Israel
\par 40 de as filhas de Israel saírem por quatro dias, de ano em ano, a cantar em memória da filha de Jefté, o gileadita.

\chapter{12}

\par 1 Então, foram convocados os homens de Efraim, e passaram para Zafom, e disseram a Jefté: Por que foste combater contra os filhos de Amom e não nos chamaste para ir contigo? Queimaremos a tua casa, estando tu dentro dela.
\par 2 E Jefté lhes disse: Eu e o meu povo tivemos grande contenda com os filhos de Amom; chamei-vos, e não me livrastes das suas mãos.
\par 3 Vendo eu que não me livráveis, arrisquei a minha vida e passei contra os filhos de Amom, e o SENHOR os entregou nas minhas mãos; por que, pois, subistes, hoje, contra mim, para me combaterdes?
\par 4 Ajuntou Jefté todos os homens de Gileade e pelejou contra Efraim; e os homens de Gileade feriram Efraim, porque este dissera: Fugitivos sois de Efraim, vós, gileaditas, que morais no meio de Efraim e Manassés.
\par 5 Porém os gileaditas tomaram os vaus do Jordão que conduzem a Efraim; de sorte que, quando qualquer fugitivo de Efraim dizia: Quero passar; então, os homens de Gileade lhe perguntavam: És tu efraimita? Se respondia: Não;
\par 6 então, lhe tornavam: Dize, pois, chibolete; quando dizia sibolete, não podendo exprimir bem a palavra, então, pegavam dele e o matavam nos vaus do Jordão. E caíram de Efraim, naquele tempo, quarenta e dois mil.
\par 7 Jefté, o gileadita, julgou a Israel seis anos; e morreu e foi sepultado numa das cidades de Gileade.
\par 8 Depois dele, julgou a Israel Ibsã, de Belém.
\par 9 Tinha este trinta filhos e trinta filhas; a estas, casou fora; e, de fora, trouxe trinta mulheres para seus filhos. Julgou a Israel sete anos.
\par 10 Então, faleceu Ibsã e foi sepultado em Belém.
\par 11 Depois dele, veio Elom, o zebulonita, que julgou a Israel dez anos.
\par 12 Faleceu Elom, o zebulonita, e foi sepultado em Aijalom, na terra de Zebulom.
\par 13 Depois dele, julgou a Israel Abdom, filho de Hilel, o piratonita.
\par 14 Tinha este quarenta filhos e trinta netos, que cavalgavam setenta jumentos. Julgou a Israel oito anos.
\par 15 Então, faleceu Abdom, filho de Hilel, o piratonita; e foi sepultado em Piratom, na terra de Efraim, na região montanhosa dos amalequitas.

\chapter{13}

\par 1 Tendo os filhos de Israel tornado a fazer o que era mau perante o SENHOR, este os entregou nas mãos dos filisteus por quarenta anos.
\par 2 Havia um homem de Zorá, da linhagem de Dã, chamado Manoá, cuja mulher era estéril e não tinha filhos.
\par 3 Apareceu o Anjo do SENHOR a esta mulher e lhe disse: Eis que és estéril e nunca tiveste filho; porém conceberás e darás à luz um filho.
\par 4 Agora, pois, guarda-te, não bebas vinho ou bebida forte, nem comas coisa imunda;
\par 5 porque eis que tu conceberás e darás à luz um filho sobre cuja cabeça não passará navalha; porquanto o menino será nazireu consagrado a Deus desde o ventre de sua mãe; e ele começará a livrar a Israel do poder dos filisteus.
\par 6 Então, a mulher foi a seu marido e lhe disse: Um homem de Deus veio a mim; sua aparência era semelhante à de um anjo de Deus, tremenda; não lhe perguntei donde era, nem ele me disse o seu nome.
\par 7 Porém me disse: Eis que tu conceberás e darás à luz um filho; agora, pois, não bebas vinho, nem bebida forte, nem comas coisa imunda; porque o menino será nazireu consagrado a Deus, desde o ventre materno até ao dia de sua morte.
\par 8 Então, Manoá orou ao SENHOR e disse: Ah! Senhor meu, rogo-te que o homem de Deus que enviaste venha outra vez e nos ensine o que devemos fazer ao menino que há de nascer.
\par 9 Deus ouviu a voz de Manoá, e o Anjo de Deus veio outra vez à mulher, quando esta se achava assentada no campo; porém não estava com ela seu marido Manoá.
\par 10 Apressou-se, pois, a mulher, e, correndo, noticiou-o a seu marido, e lhe disse: Eis que me apareceu aquele homem que viera a mim no outro dia.
\par 11 Então, se levantou Manoá, e seguiu a sua mulher, e, tendo chegado ao homem, lhe disse: És tu o que falaste a esta mulher? Ele respondeu: Eu sou.
\par 12 Então, disse Manoá: Quando se cumprirem as tuas palavras, qual será o modo de viver do menino e o seu serviço?
\par 13 Respondeu-lhe o Anjo do SENHOR: Guarde-se a mulher de tudo quanto eu lhe disse.
\par 14 De tudo quanto procede da videira não comerá, nem vinho nem bebida forte beberá, nem coisa imunda comerá; tudo quanto lhe tenho ordenado guardará.
\par 15 Então, Manoá disse ao Anjo do SENHOR: Permite-nos deter-te, e te prepararemos um cabrito.
\par 16 Porém o Anjo do SENHOR disse a Manoá: Ainda que me detenhas, não comerei de teu pão; e, se preparares holocausto, ao SENHOR o oferecerás. Porque não sabia Manoá que era o Anjo do SENHOR.
\par 17 Perguntou Manoá ao Anjo do SENHOR: Qual é o teu nome, para que, quando se cumprir a tua palavra, te honremos?
\par 18 Respondeu-lhe o Anjo do SENHOR e lhe disse: Por que perguntas assim pelo meu nome, que é maravilhoso?
\par 19 Tomou, pois, Manoá um cabrito e uma oferta de manjares e os apresentou sobre uma rocha ao SENHOR; e o Anjo do SENHOR se houve maravilhosamente. Manoá e sua mulher estavam observando.
\par 20 Sucedeu que, subindo para o céu a chama que saiu do altar, o Anjo do SENHOR subiu nela; o que vendo Manoá e sua mulher, caíram com o rosto em terra.
\par 21 Nunca mais apareceu o Anjo do SENHOR a Manoá, nem a sua mulher; então, Manoá ficou sabendo que era o Anjo do SENHOR.
\par 22 Disse Manoá a sua mulher: Certamente, morreremos, porque vimos a Deus.
\par 23 Porém sua mulher lhe disse: Se o SENHOR nos quisera matar, não aceitaria de nossas mãos o holocausto e a oferta de manjares, nem nos teria mostrado tudo isto, nem nos teria revelado tais coisas.
\par 24 Depois, deu a mulher à luz um filho e lhe chamou Sansão; o menino cresceu, e o SENHOR o abençoou.
\par 25 E o Espírito do SENHOR passou a incitá-lo em Maané-Dã, entre Zorá e Estaol.

\chapter{14}

\par 1 Desceu Sansão a Timna; vendo em Timna uma das filhas dos filisteus,
\par 2 subiu, e declarou-o a seu pai e a sua mãe, e disse: Vi uma mulher em Timna, das filhas dos filisteus; tomai-ma, pois, por esposa.
\par 3 Porém seu pai e sua mãe lhe disseram: Não há, porventura, mulher entre as filhas de teus irmãos ou entre todo o meu povo, para que vás tomar esposa dos filisteus, daqueles incircuncisos? Disse Sansão a seu pai: Toma-me esta, porque só desta me agrado.
\par 4 Mas seu pai e sua mãe não sabiam que isto vinha do SENHOR, pois este procurava ocasião contra os filisteus; porquanto, naquele tempo, os filisteus dominavam sobre Israel.
\par 5 Desceu, pois, com seu pai e sua mãe a Timna; e, chegando às vinhas de Timna, eis que um leão novo, bramando, lhe saiu ao encontro.
\par 6 Então, o Espírito do SENHOR de tal maneira se apossou dele, que ele o rasgou como quem rasga um cabrito, sem nada ter na mão; todavia, nem a seu pai nem a sua mãe deu a saber o que fizera.
\par 7 Desceu, e falou àquela mulher, e dela se agradou.
\par 8 Depois de alguns dias, voltou ele para a tomar; e, apartando-se do caminho para ver o corpo do leão morto, eis que, neste, havia um enxame de abelhas com mel.
\par 9 Tomou o favo nas mãos e se foi andando e comendo dele; e chegando a seu pai e a sua mãe, deu-lhes do mel, e comeram; porém não lhes deu a saber que do corpo do leão é que o tomara.
\par 10 Descendo, pois, seu pai à casa daquela mulher, fez Sansão ali um banquete; porque assim o costumavam fazer os moços.
\par 11 Sucedeu que, como o vissem, convidaram trinta companheiros para estarem com ele.
\par 12 Disse-lhes, pois, Sansão: Dar-vos-ei um enigma a decifrar; se, nos sete dias das bodas, mo declarardes e descobrirdes, dar-vos-ei trinta camisas e trinta vestes festivais;
\par 13 se mo não puderdes declarar, vós me dareis a mim as trinta camisas e as trinta vestes festivais. E eles lhe disseram: Dá-nos o teu enigma a decifrar, para que o ouçamos.
\par 14 Então, lhes disse: Do comedor saiu comida, e do forte saiu doçura. E, em três dias, não puderam decifrar o enigma.
\par 15 Ao sétimo dia, disseram à mulher de Sansão: Persuade a teu marido que nos declare o enigma, para que não queimemos a ti e a casa de teu pai. Convidastes-nos para vos apossardes do que é nosso, não é assim?
\par 16 A mulher de Sansão chorou diante dele e disse: Tão-somente me aborreces e não me amas; pois deste aos meus patrícios um enigma a decifrar e ainda não mo declaraste a mim. E ele lhe disse: Nem a meu pai nem a minha mãe o declarei e to declararia a ti?
\par 17 Ela chorava diante dele os sete dias em que celebravam as bodas; ao sétimo dia, lhe declarou, porquanto o importunava; então, ela declarou o enigma aos seus patrícios.
\par 18 Disseram, pois, a Sansão os homens daquela cidade, ao sétimo dia, antes de se pôr o sol: Que coisa há mais doce do que o mel e mais forte do que o leão? E ele lhes citou o provérbio: Se vós não lavrásseis com a minha novilha, nunca teríeis descoberto o meu enigma.
\par 19 Então, o Espírito do SENHOR de tal maneira se apossou dele, que desceu aos asquelonitas, matou deles trinta homens, despojou-os e as suas vestes festivais deu aos que declararam o enigma; porém acendeu-se a sua ira, e ele subiu à casa de seu pai.
\par 20 Ao companheiro de honra de Sansão foi dada por mulher a esposa deste.

\chapter{15}

\par 1 Passado algum tempo, nos dias da ceifa do trigo, Sansão, levando um cabrito, foi visitar a sua mulher, pois dizia: Entrarei na câmara de minha mulher. Porém o pai dela não o deixou entrar
\par 2 e lhe disse: Por certo, pensava eu que de todo a aborrecias, de sorte que a dei ao teu companheiro; porém não é mais formosa do que ela a irmã que é mais nova? Toma-a, pois, em seu lugar.
\par 3 Então, Sansão lhe respondeu: Desta feita sou inocente para com os filisteus, quando lhes fizer algum mal.
\par 4 E saiu e tomou trezentas raposas; e, tomando fachos, as virou cauda com cauda e lhes atou um facho no meio delas.
\par 5 Tendo ele chegado fogo aos tições, largou-as na seara dos filisteus e, assim, incendiou tanto os molhos como o cereal por ceifar, e as vinhas, e os olivais.
\par 6 Perguntaram os filisteus: Quem fez isto? Responderam: Sansão, o genro do timnita, porque lhe tomou a mulher e a deu a seu companheiro. Então, subiram os filisteus e queimaram a ela e o seu pai.
\par 7 Disse-lhes Sansão: Se assim procedeis, não desistirei enquanto não me vingar.
\par 8 E feriu-os com grande carnificina; e desceu e habitou na fenda da rocha de Etã.
\par 9 Então, os filisteus subiram, e acamparam-se contra Judá, e estenderam-se por Leí.
\par 10 Perguntaram-lhes os homens de Judá: Por que subistes contra nós? Responderam: Subimos para amarrar Sansão, para lhe fazer a ele como ele nos fez a nós.
\par 11 Então, três mil homens de Judá desceram até à fenda da rocha de Etã e disseram a Sansão: Não sabias tu que os filisteus dominam sobre nós? Por que, pois, nos fizeste isto? Ele lhes respondeu: Assim como me fizeram a mim, eu lhes fiz a eles.
\par 12 Descemos, replicaram eles, para te amarrar, para te entregar nas mãos dos filisteus. Sansão lhes disse: Jurai-me que vós mesmos não me acometereis.
\par 13 Eles lhe disseram: Não, mas somente te amarraremos e te entregaremos nas suas mãos; porém de maneira nenhuma te mataremos. E amarraram-no com duas cordas novas e fizeram-no subir da rocha.
\par 14 Chegando ele a Leí, os filisteus lhe saíram ao encontro, jubilando; porém o Espírito do SENHOR de tal maneira se apossou dele, que as cordas que tinha nos braços se tornaram como fios de linho queimados, e as suas amarraduras se desfizeram das suas mãos.
\par 15 Achou uma queixada de jumento, ainda fresca, à mão, e tomou-a, e feriu com ela mil homens.
\par 16 E disse: Com uma queixada de jumento um montão, outro montão; com uma queixada de jumento feri mil homens.
\par 17 Tendo ele acabado de falar, lançou da sua mão a queixada. Chamou-se aquele lugar Ramate-Leí.
\par 18 Sentindo grande sede, clamou ao SENHOR e disse: Por intermédio do teu servo deste esta grande salvação; morrerei eu, agora, de sede e cairei nas mãos destes incircuncisos?
\par 19 Então, o SENHOR fendeu a cavidade que estava em Leí, e dela saiu água; tendo Sansão bebido, recobrou alento e reviveu; daí chamar-se aquele lugar En-Hacoré até ao dia de hoje.
\par 20 Sansão julgou a Israel, nos dias dos filisteus, vinte anos.

\chapter{16}

\par 1 Sansão foi a Gaza, e viu ali uma prostituta, e coabitou com ela.
\par 2 Foi dito aos gazitas: Sansão chegou aqui. Cercaram-no, pois, e toda a noite o esperaram, às escondidas, na porta da cidade; e, toda a noite, estiveram em silêncio, pois diziam: Esperaremos até ao raiar do dia; então, daremos cabo dele.
\par 3 Porém Sansão esteve deitado até à meia-noite; então, se levantou, e pegou ambas as folhas da porta da cidade com suas ombreiras, e, juntamente com a tranca, as tomou, pondo-as sobre os ombros; e levou-as para cima, até ao cimo do monte que olha para Hebrom.
\par 4 Depois disto, aconteceu que se afeiçoou a uma mulher do vale de Soreque, a qual se chamava Dalila.
\par 5 Então, os príncipes dos filisteus subiram a ela e lhe disseram: Persuade-o e vê em que consiste a sua grande força e com que poderíamos dominá-lo e amarrá-lo, para assim o subjugarmos; e te daremos cada um mil e cem siclos de prata.
\par 6 Disse, pois, Dalila a Sansão: Declara-me, peço-te, em que consiste a tua grande força e com que poderias ser amarrado para te poderem subjugar.
\par 7 Respondeu-lhe Sansão: Se me amarrarem com sete tendões frescos, ainda não secos, então, me enfraquecerei, e serei como qualquer outro homem.
\par 8 Os príncipes dos filisteus trouxeram a Dalila sete tendões frescos, que ainda não estavam secos; e com os tendões ela o amarrou.
\par 9 Tinha ela no seu quarto interior homens escondidos. Então, ela lhe disse: Os filisteus vêm sobre ti, Sansão! Quebrou ele os tendões como se quebra o fio da estopa chamuscada; assim, não se soube em que lhe consistia a força.
\par 10 Disse Dalila a Sansão: Eis que zombaste de mim e me disseste mentiras; ora, declara-me, agora, com que poderias ser amarrado.
\par 11 Ele lhe disse: Se me amarrarem bem com cordas novas, com que se não tenha feito obra nenhuma, então, me enfraquecerei e serei como qualquer outro homem.
\par 12 Dalila tomou cordas novas, e o amarrou, e disse-lhe: Os filisteus vêm sobre ti, Sansão! Tinha ela no seu quarto interior homens escondidos. Ele as rebentou de seus braços como um fio.
\par 13 Disse Dalila a Sansão: Até agora, tens zombado de mim e me tens dito mentiras; declara-me, pois, agora: com que poderias ser amarrado? Ele lhe respondeu: Se teceres as sete tranças da minha cabeça com a urdidura da teia e se as firmares com pino de tear, então, me enfraquecerei e serei como qualquer outro homem. Enquanto ele dormia, tomou ela as sete tranças e as teceu com a urdidura da teia.
\par 14 E as fixou com um pino de tear e disse-lhe: Os filisteus vêm sobre ti, Sansão! Então, despertou do seu sono e arrancou o pino e a urdidura da teia.
\par 15 Então, ela lhe disse: Como dizes que me amas, se não está comigo o teu coração? Já três vezes zombaste de mim e ainda não me declaraste em que consiste a tua grande força.
\par 16 Importunando-o ela todos os dias com as suas palavras e molestando-o, apoderou-se da alma dele uma impaciência de matar.
\par 17 Descobriu-lhe todo o coração e lhe disse: Nunca subiu navalha à minha cabeça, porque sou nazireu de Deus, desde o ventre de minha mãe; se vier a ser rapado, ir-se-á de mim a minha força, e me enfraquecerei e serei como qualquer outro homem.
\par 18 Vendo, pois, Dalila que já ele lhe descobrira todo o coração, mandou chamar os príncipes dos filisteus, dizendo: Subi mais esta vez, porque, agora, me descobriu ele todo o coração. Então, os príncipes dos filisteus subiram a ter com ela e trouxeram com eles o dinheiro.
\par 19 Então, Dalila fez dormir Sansão nos joelhos dela e, tendo chamado um homem, mandou rapar-lhe as sete tranças da cabeça; passou ela a subjugá-lo; e retirou-se dele a sua força.
\par 20 E disse ela: Os filisteus vêm sobre ti, Sansão! Tendo ele despertado do seu sono, disse consigo mesmo: Sairei ainda esta vez como dantes e me livrarei; porque ele não sabia ainda que já o SENHOR se tinha retirado dele.
\par 21 Então, os filisteus pegaram nele, e lhe vazaram os olhos, e o fizeram descer a Gaza; amarraram-no com duas cadeias de bronze, e virava um moinho no cárcere.
\par 22 E o cabelo da sua cabeça, logo após ser rapado, começou a crescer de novo.
\par 23 Então, os príncipes dos filisteus se ajuntaram para oferecer grande sacrifício a seu deus Dagom e para se alegrarem; e diziam: Nosso deus nos entregou nas mãos a Sansão, nosso inimigo.
\par 24 Vendo-o o povo, louvavam ao seu deus, porque diziam: Nosso deus nos entregou nas mãos o nosso inimigo, e o que destruía a nossa terra, e o que multiplicava os nossos mortos.
\par 25 Alegrando-se-lhes o coração, disseram: Mandai vir Sansão, para que nos divirta. Trouxeram Sansão do cárcere, o qual os divertia. Quando o fizeram estar em pé entre as colunas,
\par 26 disse Sansão ao moço que o tinha pela mão: Deixa-me, para que apalpe as colunas em que se sustém a casa, para que me encoste a elas.
\par 27 Ora, a casa estava cheia de homens e mulheres, e também ali estavam todos os príncipes dos filisteus; e sobre o teto havia uns três mil homens e mulheres, que olhavam enquanto Sansão os divertia.
\par 28 Sansão clamou ao SENHOR e disse: SENHOR Deus, peço-te que te lembres de mim, e dá-me força só esta vez, ó Deus, para que me vingue dos filisteus, ao menos por um dos meus olhos.
\par 29 Abraçou-se, pois, Sansão com as duas colunas do meio, em que se sustinha a casa, e fez força sobre elas, com a mão direita em uma e com a esquerda na outra.
\par 30 E disse: Morra eu com os filisteus. E inclinou-se com força, e a casa caiu sobre os príncipes e sobre todo o povo que nela estava; e foram mais os que matou na sua morte do que os que matara na sua vida.
\par 31 Então, seus irmãos desceram, e toda a casa de seu pai, tomaram-no, subiram com ele e o sepultaram entre Zorá e Estaol, no sepulcro de Manoá, seu pai. Julgou ele a Israel vinte anos.

\chapter{17}

\par 1 Havia um homem da região montanhosa de Efraim cujo nome era Mica,
\par 2 o qual disse a sua mãe: Os mil e cem siclos de prata que te foram tirados, por cuja causa deitavas maldições e de que também me falaste, eis que esse dinheiro está comigo; eu o tomei. Então, lhe disse a mãe: Bendito do SENHOR seja meu filho!
\par 3 Assim, restituiu os mil e cem siclos de prata a sua mãe, que disse: De minha mão dedico este dinheiro ao SENHOR para meu filho, para fazer uma imagem de escultura e uma de fundição, de sorte que, agora, eu to devolvo.
\par 4 Porém ele restituiu o dinheiro a sua mãe, que tomou duzentos siclos de prata e os deu ao ourives, o qual fez deles uma imagem de escultura e uma de fundição; e a imagem esteve em casa de Mica.
\par 5 E, assim, este homem, Mica, veio a ter uma casa de deuses; fez uma estola sacerdotal e ídolos do lar e consagrou a um de seus filhos, para que lhe fosse por sacerdote.
\par 6 Naqueles dias, não havia rei em Israel; cada qual fazia o que achava mais reto.
\par 7 Havia um moço de Belém de Judá, da tribo de Judá, que era levita e se demorava ali.
\par 8 Esse homem partiu da cidade de Belém de Judá para ficar onde melhor lhe parecesse. Seguindo, pois, o seu caminho, chegou à região montanhosa de Efraim, até à casa de Mica.
\par 9 Perguntou-lhe Mica: Donde vens? Ele lhe respondeu: Sou levita de Belém de Judá e vou ficar onde melhor me parecer.
\par 10 Então, lhe disse Mica: Fica comigo e sê-me por pai e sacerdote; e cada ano te darei dez siclos de prata, o vestuário e o sustento. O levita entrou
\par 11 e consentiu em ficar com aquele homem; e o moço lhe foi como um de seus filhos.
\par 12 Consagrou Mica ao moço levita, que lhe passou a ser sacerdote; e ficou em casa de Mica.
\par 13 Então, disse Mica: Sei, agora, que o SENHOR me fará bem, porquanto tenho um levita por sacerdote.

\chapter{18}

\par 1 Naqueles dias, não havia rei em Israel, e a tribo dos danitas buscava para si herança em que habitar; porquanto, até àquele dia, entre as tribos de Israel, não lhe havia caído por sorte a herança.
\par 2 Enviaram os filhos de Dã cinco homens dentre todos os da sua tribo, homens valentes, de Zorá e de Estaol, a espiar e explorar a terra; e lhes disseram: Ide, explorai a terra. Chegaram à região montanhosa de Efraim, até à casa de Mica, e ali pernoitaram.
\par 3 Estando eles junto da casa de Mica, reconheceram a voz do moço, do levita; chegaram-se para lá e lhe disseram: Quem te trouxe para aqui? Que fazes aqui? E que é que tens aqui?
\par 4 Ele respondeu: Assim e assim me fez Mica; pois me assalariou, e eu lhe sirvo de sacerdote.
\par 5 Então, lhe disseram: Consulta a Deus, para que saibamos se prosperará o caminho que levamos.
\par 6 Disse-lhes o sacerdote: Ide em paz; o caminho que levais está sob as vistas do SENHOR.
\par 7 Partiram os cinco homens, e chegaram a Laís, e viram que o povo que havia nela estava seguro, segundo o costume dos sidônios, em paz e confiado. Nenhuma autoridade havia que, por qualquer coisa, o oprimisse; também estava longe dos sidônios e não tinha trato com nenhuma outra gente.
\par 8 Então, voltaram a seus irmãos, a Zorá e a Estaol; e estes lhes perguntaram: Que nos dizeis?
\par 9 Eles disseram: Disponde-vos e subamos contra eles; porque examinamos a terra, e eis que é muito boa. Estais aí parados? Não vos demoreis em sair para ocupardes a terra.
\par 10 Quando lá chegardes, achareis um povo confiado, e a terra é ampla; porque Deus vo-la entregou nas mãos; é um lugar em que não há falta de coisa alguma que há na terra.
\par 11 Então, partiram dali, da tribo dos danitas, de Zorá e de Estaol, seiscentos homens armados de suas armas de guerra.
\par 12 Subiram e acamparam-se em Quiriate-Jearim, em Judá; pelo que chamaram a este lugar Maané-Dã, até ao dia de hoje; está por detrás de Quiriate-Jearim.
\par 13 Dali, passaram à região montanhosa de Efraim e chegaram até à casa de Mica.
\par 14 Os cinco homens que foram espiar a terra de Laís disseram a seus irmãos: Sabeis vós que, naquelas casas, há uma estola sacerdotal, e ídolos do lar, e uma imagem de escultura, e uma de fundição? Vede, pois, o que haveis de fazer.
\par 15 Então, foram para lá, e chegaram à casa do moço, o levita, em casa de Mica, e o saudaram.
\par 16 Os seiscentos homens que eram dos filhos de Dã, armados de suas armas de guerra, ficaram à entrada da porta.
\par 17 Porém, subindo os cinco homens que foram espiar a terra, entraram e apanharam a imagem de escultura, a estola sacerdotal, os ídolos do lar e a imagem de fundição, ficando o sacerdote em pé à entrada da porta, com os seiscentos homens que estavam armados com as armas de guerra.
\par 18 Entrando eles, pois, na casa de Mica e tomando a imagem de escultura, a estola sacerdotal, os ídolos do lar e a imagem de fundição, disse-lhes o sacerdote: Que estais fazendo?
\par 19 Eles lhe disseram: Cala-te, e põe a mão na boca, e vem conosco, e sê-nos por pai e sacerdote. Ser-te-á melhor seres sacerdote da casa de um só homem do que seres sacerdote de uma tribo e de uma família em Israel?
\par 20 Então, se alegrou o coração do sacerdote, tomou a estola sacerdotal, os ídolos do lar e a imagem de escultura e entrou no meio do povo.
\par 21 Assim, viraram e, tendo posto diante de si os meninos, o gado e seus bens, partiram.
\par 22 Estando já longe da casa de Mica, reuniram-se os homens que estavam nas casas junto à dele e alcançaram os filhos de Dã.
\par 23 E clamaram após eles, os quais, voltando-se, disseram a Mica: Que tens, que convocaste esse povo?
\par 24 Respondeu-lhes: Os deuses que eu fiz me tomastes e também o sacerdote e vos fostes; que mais me resta? Como, pois, me perguntais: Que é o que tens?
\par 25 Porém os filhos de Dã lhe disseram: Não nos faças ouvir a tua voz, para que, porventura, homens de ânimo amargoso não se lancem sobre ti, e tu percas a tua vida e a vida dos da tua casa.
\par 26 Assim, prosseguiram o seu caminho os filhos de Dã; e Mica, vendo que eram mais fortes do que ele, voltou-se e tornou para sua casa.
\par 27 Levaram eles o que Mica havia feito e o sacerdote que tivera, e chegaram a Laís, a um povo em paz e confiado, e os feriram a fio de espada, e queimaram a cidade.
\par 28 Ninguém houve que os livrasse, porquanto estavam longe de Sidom e não tinham trato com ninguém; a cidade estava no vale junto a Bete-Reobe. Reedificaram a cidade, habitaram nela
\par 29 e lhe chamaram Dã, segundo o nome de Dã, seu pai, que nascera a Israel; porém, outrora, o nome desta cidade era Laís.
\par 30 Os filhos de Dã levantaram para si aquela imagem de escultura; e Jônatas, filho de Gérson, o filho de Manassés, ele e seus filhos foram sacerdotes da tribo dos danitas até ao dia do cativeiro do povo.
\par 31 Assim, pois, a imagem de escultura feita por Mica estabeleceram para si todos os dias que a Casa de Deus esteve em Siló.

\chapter{19}

\par 1 Naqueles dias, em que não havia rei em Israel, houve um homem levita, que, peregrinando nos longes da região montanhosa de Efraim, tomou para si uma concubina de Belém de Judá.
\par 2 Porém ela, aborrecendo-se dele, o deixou, tornou para a casa de seu pai, em Belém de Judá, e lá esteve os dias de uns quatro meses.
\par 3 Seu marido, tendo consigo o seu servo e dois jumentos, levantou-se e foi após ela para falar-lhe ao coração, a fim de tornar a trazê-la. Ela o fez entrar na casa de seu pai. Este, quando o viu, saiu alegre a recebê-lo.
\par 4 Seu sogro, o pai da moça, o deteve por três dias em sua companhia; comeram, beberam, e o casal se alojou ali.
\par 5 Ao quarto dia, madrugaram e se levantaram para partir; então, o pai da moça disse a seu genro: Fortalece-te com um bocado de pão, e, depois, partireis.
\par 6 Assentaram-se, pois, e comeram ambos juntos, e beberam; então, disse o pai da moça ao homem: Peço-te que ainda esta noite queiras passá-la aqui, e se alegre o teu coração.
\par 7 Contudo, o homem levantou-se para partir; porém o seu sogro, instando com ele, fê-lo pernoitar ali.
\par 8 Madrugando ele ao quinto dia para partir, disse o pai da moça: Fortalece-te, e detende-vos até ao declinar do dia; e ambos comeram juntos.
\par 9 Então, o homem se levantou para partir, ele, e a sua concubina, e o seu moço; e disse-lhe seu sogro, o pai da moça: Eis que já declina o dia, a tarde vem chegando; peço-te que passes aqui a noite; vai-se o dia acabando, passa aqui a noite, e que o teu coração se alegre; amanhã de madrugada, levantai-vos a caminhar e ide para a vossa casa.
\par 10 Porém o homem não quis passar ali a noite; mas levantou-se, e partiu, e veio até à altura de Jebus (que é Jerusalém), e com ele os dois jumentos albardados, como também a sua concubina.
\par 11 Estando, pois, já perto de Jebus e tendo-se adiantado o declinar-se do dia, disse o moço a seu senhor: Caminhai, agora, e retiremo-nos a esta cidade dos jebuseus e passemos ali a noite.
\par 12 Porém o seu senhor lhe disse: Não nos retiraremos a nenhuma cidade estranha, que não seja dos filhos de Israel, mas passemos até Gibeá.
\par 13 Disse mais a seu moço: Caminha, e cheguemos a um daqueles lugares e pernoitemos em Gibeá ou em Ramá.
\par 14 Passaram, pois, adiante e caminharam, e o sol se lhes pôs junto a Gibeá, que pertence a Benjamim.
\par 15 Retiraram-se para Gibeá, a fim de, nela, passarem a noite; entrando ele, assentou-se na praça da cidade, porque não houve quem os recolhesse em casa para ali pernoitarem.
\par 16 Eis que, ao anoitecer, vinha do seu trabalho no campo um homem velho; era este da região montanhosa de Efraim, mas morava em Gibeá; porém os habitantes do lugar eram benjamitas.
\par 17 Erguendo o velho os olhos, viu na praça da cidade este viajante e lhe perguntou: Para onde vais e donde vens?
\par 18 Ele lhe respondeu: Estamos viajando de Belém de Judá para os longes da região montanhosa de Efraim, donde sou; fui a Belém de Judá e, agora, estou de viagem para a Casa do SENHOR; e ninguém há que me recolha em casa,
\par 19 ainda que há palha e pasto para os nossos jumentos, e também pão e vinho para mim, e para a tua serva, e para o moço que vem com os teus servos; de coisa nenhuma há falta.
\par 20 Então, disse o velho: Paz seja contigo; tudo quanto te vier a faltar fique a meu cargo; tão-somente não passes a noite na praça.
\par 21 Levou-o para casa e deu pasto aos jumentos; e, tendo eles lavado os pés, comeram e beberam.
\par 22 Enquanto eles se alegravam, eis que os homens daquela cidade, filhos de Belial, cercaram a casa, batendo à porta; e falaram ao velho, senhor da casa, dizendo: Traze para fora o homem que entrou em tua casa, para que abusemos dele.
\par 23 O senhor da casa saiu a ter com eles e lhes disse: Não, irmãos meus, não façais semelhante mal; já que o homem está em minha casa, não façais tal loucura.
\par 24 Minha filha virgem e a concubina dele trarei para fora; humilhai-as e fazei delas o que melhor vos agrade; porém a este homem não façais semelhante loucura.
\par 25 Porém aqueles homens não o quiseram ouvir; então, ele pegou da concubina do levita e entregou a eles fora, e eles a forçaram e abusaram dela toda a noite até pela manhã; e, subindo a alva, a deixaram.
\par 26 Ao romper da manhã, vindo a mulher, caiu à porta da casa do homem, onde estava o seu senhor, e ali ficou até que se fez dia claro.
\par 27 Levantando-se pela manhã o seu senhor, abriu as portas da casa e, saindo a seguir o seu caminho, eis que a mulher, sua concubina, jazia à porta da casa, com as mãos sobre o limiar.
\par 28 Ele lhe disse: Levanta-te, e vamos; porém ela não respondeu; então, o homem a pôs sobre o jumento, dispôs-se e foi para sua casa.
\par 29 Chegando a casa, tomou de um cutelo e, pegando a concubina, a despedaçou por seus ossos em doze partes; e as enviou por todos os limites de Israel.
\par 30 Cada um que a isso presenciava aos outros dizia: Nunca tal se fez, nem se viu desde o dia em que os filhos de Israel subiram da terra do Egito até ao dia de hoje; ponderai nisso, considerai e falai.

\chapter{20}

\par 1 Saíram todos os filhos de Israel, e a congregação se ajuntou perante o SENHOR em Mispa, como se fora um só homem, desde Dã até Berseba, como também a terra de Gileade.
\par 2 Os príncipes de todo o povo e todas as tribos de Israel se apresentaram na congregação do povo de Deus. Havia quatrocentos mil homens de pé, que puxavam da espada.
\par 3 Ouviram os filhos de Benjamim que os filhos de Israel haviam subido a Mispa. Disseram os filhos de Israel: Contai-nos como sucedeu esta maldade.
\par 4 Então, respondeu o homem levita, marido da mulher que fora morta, e disse: Cheguei com a minha concubina a Gibeá, cidade de Benjamim, para passar a noite;
\par 5 os cidadãos de Gibeá se levantaram contra mim e, à noite, cercaram a casa em que eu estava; intentaram matar-me e violaram a minha concubina, de maneira que morreu.
\par 6 Então, peguei a minha concubina, e a fiz em pedaços, e os enviei por toda a terra da herança de Israel, porquanto fizeram vergonha e loucura em Israel.
\par 7 Eis que todos sois filhos de Israel; eia! Dai a vossa palavra e conselho neste caso.
\par 8 Então, todo o povo se levantou como um só homem, dizendo: Nenhum de nós voltará para sua tenda, nenhum de nós se retirará para casa.
\par 9 Porém isto é o que faremos a Gibeá: subiremos contra ela por sorte.
\par 10 Tomaremos dez homens de cem de todas as tribos de Israel, e cem de mil, e mil de dez mil, para providenciarem mantimento para o povo, a fim de que este, vindo a Gibeá de Benjamim, faça a ela conforme toda a loucura que tem feito em Israel.
\par 11 Assim, se ajuntaram contra esta cidade todos os homens de Israel, unidos como um só homem.
\par 12 As tribos de Israel enviaram homens por toda a tribo de Benjamim, para lhe dizerem: Que maldade é essa que se fez entre vós?
\par 13 Dai-nos, agora, os homens, filhos de Belial, que estão em Gibeá, para que os matemos e tiremos de Israel o mal; porém Benjamim não quis ouvir a voz de seus irmãos, os filhos de Israel.
\par 14 Antes, os filhos de Benjamim se ajuntaram, vindos das cidades em Gibeá, para saírem a pelejar contra os filhos de Israel.
\par 15 E contaram-se, naquele dia, os filhos de Benjamim vindos das cidades; eram vinte e seis mil homens que puxavam da espada, afora os moradores de Gibeá, de que se contavam setecentos homens escolhidos.
\par 16 Entre todo este povo havia setecentos homens escolhidos, canhotos, os quais atiravam com a funda uma pedra num cabelo e não erravam.
\par 17 Contaram-se dos homens de Israel, afora os de Benjamim, quatrocentos mil homens que puxavam da espada, e todos eles, homens de guerra.
\par 18 Levantaram-se os israelitas, subiram a Betel e consultaram a Deus, dizendo: Quem dentre nós subirá, primeiro, a pelejar contra Benjamim? Respondeu o SENHOR: Judá subirá primeiro.
\par 19 Levantaram-se, pois, os filhos de Israel pela manhã e acamparam-se contra Gibeá.
\par 20 Saíram os homens de Israel à peleja contra Benjamim; e, junto a Gibeá, se ordenaram contra ele.
\par 21 Então, os filhos de Benjamim saíram de Gibeá e derribaram por terra, naquele dia, vinte e dois mil homens de Israel.
\par 22 Porém se animou o povo dos homens de Israel e tornaram a ordenar-se para a peleja, no lugar onde, no primeiro dia, o tinham feito.
\par 23 Antes, subiram os filhos de Israel, e choraram perante o SENHOR até à tarde, e consultaram o SENHOR, dizendo: Tornaremos a pelejar contra os filhos de Benjamim, nosso irmão? Respondeu o SENHOR: Subi contra ele.
\par 24 Chegaram-se, pois, os filhos de Israel contra os filhos de Benjamim, no dia seguinte.
\par 25 Também os de Benjamim, no dia seguinte, saíram de Gibeá de encontro a eles e derribaram ainda por terra mais dezoito mil homens, todos dos que puxavam da espada.
\par 26 Então, todos os filhos de Israel, todo o povo, subiram, e vieram a Betel, e choraram, e estiveram ali perante o SENHOR, e jejuaram aquele dia até à tarde; e, perante o SENHOR, ofereceram holocaustos e ofertas pacíficas.
\par 27 E os filhos de Israel perguntaram ao SENHOR (porquanto a arca da Aliança de Deus estava ali naqueles dias;
\par 28 e Finéias, filho de Eleazar, filho de Arão, ministrava perante ela naqueles dias), dizendo: Tornaremos a sair ainda a pelejar contra os filhos de Benjamim, nosso irmão, ou desistiremos? Respondeu o SENHOR: Subi, que amanhã eu os entregarei nas vossas mãos.
\par 29 Então, Israel pôs emboscadas em redor de Gibeá.
\par 30 Ao terceiro dia, subiram os filhos de Israel contra os filhos de Benjamim e se ordenaram à peleja contra Gibeá, como das outras vezes.
\par 31 Então, os filhos de Benjamim saíram de encontro ao povo, e, deixando-se atrair para longe da cidade, começaram a ferir alguns do povo, e mataram, como das outras vezes, uns trinta dos homens de Israel, pelas estradas, das quais uma sobe para Betel, a outra, para Gibeá do Campo.
\par 32 Então, os filhos de Benjamim disseram: Vão derrotados diante de nós como dantes. Porém os filhos de Israel disseram: Fujamos e atraiamo-los da cidade para as estradas.
\par 33 Todos os homens de Israel se levantaram do seu lugar e se ordenaram para a peleja em Baal-Tamar; e a emboscada de Israel saiu do seu lugar, das vizinhanças de Geba.
\par 34 Dez mil homens escolhidos de todo o Israel vieram contra Gibeá, e a peleja se tornou renhida; porém eles não imaginavam que a calamidade lhes tocaria.
\par 35 Então, feriu o SENHOR a Benjamim diante de Israel; e mataram os filhos de Israel, naquele dia, vinte e cinco mil e cem homens de Benjamim, todos dos que puxavam da espada;
\par 36 assim, viram os filhos de Benjamim que estavam feridos. Os homens de Israel retiraram-se perante os benjamitas, porquanto estavam confiados na emboscada que haviam posto contra Gibeá.
\par 37 A emboscada se apressou, e acometeu a Gibeá, e de golpe feriu-a toda a fio de espada.
\par 38 Os homens de Israel tinham um sinal determinado com a emboscada, que era fazerem levantar da cidade uma grande nuvem de fumaça.
\par 39 Então, os homens de Israel deviam voltar à peleja. Começara Benjamim a ferir e havia já matado uns trinta entre os homens de Israel, porque diziam: Com efeito, já estão derrotados diante de nós, como na peleja anterior.
\par 40 Então, a nuvem de fumaça começou a levantar-se da cidade, como se fora uma coluna; virando-se Benjamim a olhar para trás de si, eis que toda a cidade subia em chamas para o céu.
\par 41 Viraram os homens de Israel, e os de Benjamim pasmaram, porque viram que a calamidade lhes tocaria.
\par 42 E viraram diante dos homens de Israel, para o caminho do deserto; porém a peleja os apertou; e os que vinham das cidades os destruíram no meio deles.
\par 43 Cercaram a Benjamim, seguiram-no e, onde repousava, ali o alcançavam, até diante de Gibeá, para o nascente do sol.
\par 44 Caíram de Benjamim dezoito mil homens, todos estes homens valentes.
\par 45 Então, viraram e fugiram para o deserto, à penha Rimom; e, na respiga, mataram ainda pelos caminhos uns cinco mil homens, e de perto os seguiram até Gidom, e feriram deles dois mil homens.
\par 46 Todos os que de Benjamim caíram, naquele dia, foram vinte e cinco mil homens que puxavam da espada, todos eles homens valentes.
\par 47 Porém seiscentos homens viraram e fugiram para o deserto, à penha Rimom, onde ficaram quatro meses.
\par 48 Os homens de Israel voltaram para os filhos de Benjamim e passaram a fio de espada tudo o que restou da cidade, tanto homens como animais, em suma, tudo o que encontraram; e também a todas as cidades que acharam puseram fogo.

\chapter{21}

\par 1 Ora, haviam jurado os homens de Israel em Mispa, dizendo: Nenhum de nós dará sua filha por mulher aos benjamitas.
\par 2 Veio o povo a Betel, e ali ficaram até à tarde diante de Deus, e levantaram a voz, e prantearam com grande pranto.
\par 3 Disseram: Ah! SENHOR, Deus de Israel, por que sucedeu isto em Israel, que, hoje, lhe falte uma tribo?
\par 4 Ao dia seguinte, o povo, pela manhã, se levantou e edificou ali um altar; e apresentaram holocaustos e ofertas pacíficas.
\par 5 Disseram os filhos de Israel: Quem de todas as tribos de Israel não subiu à assembléia do SENHOR? Porque se tinha feito um grande juramento acerca do que não viesse ao SENHOR a Mispa, que dizia: Será morto.
\par 6 Os filhos de Israel tiveram compaixão de seu irmão Benjamim e disseram: Foi, hoje, eliminada uma tribo de Israel.
\par 7 Como obteremos mulheres para os restantes deles, pois juramos, pelo SENHOR, que das nossas filhas não lhes daríamos por mulheres?
\par 8 E disseram: Há alguma das tribos de Israel que não tenha subido ao SENHOR a Mispa? E eis que ninguém de Jabes-Gileade viera ao acampamento, à assembléia.
\par 9 Quando se contou o povo, eis que nenhum dos moradores de Jabes-Gileade se achou ali.
\par 10 Por isso, a congregação enviou lá doze mil homens dos mais valentes e lhes ordenou, dizendo: Ide e, a fio de espada, feri os moradores de Jabes-Gileade, e as mulheres, e as crianças.
\par 11 Isto é o que haveis de fazer: a todo homem e a toda mulher que se houver deitado com homem destruireis.
\par 12 Acharam entre os moradores de Jabes-Gileade quatrocentas moças virgens, que não se deitaram com homem; e as trouxeram ao acampamento, a Siló, que está na terra de Canaã.
\par 13 Toda a congregação, pois, enviou mensageiros aos filhos de Benjamim que estavam na penha Rimom, e lhes proclamaram a paz.
\par 14 Nesse mesmo tempo, voltaram os benjamitas; e se lhes deram por mulheres as que foram conservadas com vida, das de Jabes-Gileade; porém estas ainda não lhes bastaram.
\par 15 Então, o povo teve compaixão de Benjamim, porquanto o SENHOR tinha feito brecha nas tribos de Israel.
\par 16 Disseram os anciãos da congregação: Como obteremos mulheres para os restantes ainda, pois foram exterminadas as mulheres dos benjamitas?
\par 17 Disseram mais: A herança dos que ficaram de resto não na deve perder Benjamim, visto que nenhuma tribo de Israel deve ser destruída.
\par 18 Porém nós não lhes poderemos dar mulheres de nossas filhas, porque os filhos de Israel juraram, dizendo: Maldito o que der mulher aos benjamitas.
\par 19 Então, disseram: Eis que, de ano em ano, há solenidade do SENHOR em Siló, que se celebra para o norte de Betel, do lado do nascente do sol, pelo caminho alto que sobe de Betel a Siquém e para o sul de Lebona.
\par 20 Ordenaram aos filhos de Benjamim, dizendo: Ide, e emboscai-vos nas vinhas,
\par 21 e olhai; e eis aí, saindo as filhas de Siló a dançar em rodas, saí vós das vinhas, e arrebatai, dentre elas, cada um sua mulher, e ide-vos à terra de Benjamim.
\par 22 Quando seus pais ou seus irmãos vierem queixar-se a nós, nós lhes diremos: por amor de nós, tende compaixão deles, pois, na guerra contra Jabes-Gileade, não obtivemos mulheres para cada um deles; e também não lhes destes, pois neste caso ficaríeis culpados.
\par 23 Assim fizeram os filhos de Benjamim e levaram mulheres conforme o número deles, das que arrebataram das rodas que dançavam; e foram-se, voltaram à sua herança, reedificaram as cidades e habitaram nelas.
\par 24 Então, os filhos de Israel também partiram dali, cada um para a sua tribo, para a sua família e para a sua herança.
\par 25 Naqueles dias, não havia rei em Israel; cada um fazia o que achava mais reto.


\end{document}