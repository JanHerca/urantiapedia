\begin{document}

\title{Neemias}


\chapter{1}

\par 1 As palavras de Neemias, filho de Hacalias. No mês de quisleu, no ano vigésimo, estando eu na cidadela de Susã,
\par 2 veio Hanani, um de meus irmãos, com alguns de Judá; então, lhes perguntei pelos judeus que escaparam e que não foram levados para o exílio e acerca de Jerusalém.
\par 3 Disseram-me: Os restantes, que não foram levados para o exílio e se acham lá na província, estão em grande miséria e desprezo; os muros de Jerusalém estão derribados, e as suas portas, queimadas.
\par 4 Tendo eu ouvido estas palavras, assentei-me, e chorei, e lamentei por alguns dias; e estive jejuando e orando perante o Deus dos céus.
\par 5 E disse: ah! SENHOR, Deus dos céus, Deus grande e temível, que guardas a aliança e a misericórdia para com aqueles que te amam e guardam os teus mandamentos!
\par 6 Estejam, pois, atentos os teus ouvidos, e os teus olhos, abertos, para acudires à oração do teu servo, que hoje faço à tua presença, dia e noite, pelos filhos de Israel, teus servos; e faço confissão pelos pecados dos filhos de Israel, os quais temos cometido contra ti; pois eu e a casa de meu pai temos pecado.
\par 7 Temos procedido de todo corruptamente contra ti, não temos guardado os mandamentos, nem os estatutos, nem os juízos que ordenaste a Moisés, teu servo.
\par 8 Lembra-te da palavra que ordenaste a Moisés, teu servo, dizendo: Se transgredirdes, eu vos espalharei por entre os povos;
\par 9 mas, se vos converterdes a mim, e guardardes os meus mandamentos, e os cumprirdes, então, ainda que os vossos rejeitados estejam pelas extremidades do céu, de lá os ajuntarei e os trarei para o lugar que tenho escolhido para ali fazer habitar o meu nome.
\par 10 Estes ainda são teus servos e o teu povo que resgataste com teu grande poder e com tua mão poderosa.
\par 11 Ah! Senhor, estejam, pois, atentos os teus ouvidos à oração do teu servo e à dos teus servos que se agradam de temer o teu nome; concede que seja bem sucedido hoje o teu servo e dá-lhe mercê perante este homem. Nesse tempo eu era copeiro do rei.

\chapter{2}

\par 1 No mês de nisã, no ano vigésimo do rei Artaxerxes, uma vez posto o vinho diante dele, eu o tomei para oferecer e lho dei; ora, eu nunca antes estivera triste diante dele.
\par 2 O rei me disse: Por que está triste o teu rosto, se não estás doente? Tem de ser tristeza do coração. Então, temi sobremaneira
\par 3 e lhe respondi: viva o rei para sempre! Como não me estaria triste o rosto se a cidade, onde estão os sepulcros de meus pais, está assolada e tem as portas consumidas pelo fogo?
\par 4 Disse-me o rei: Que me pedes agora? Então, orei ao Deus dos céus
\par 5 e disse ao rei: se é do agrado do rei, e se o teu servo acha mercê em tua presença, peço-te que me envies a Judá, à cidade dos sepulcros de meus pais, para que eu a reedifique.
\par 6 Então, o rei, estando a rainha assentada junto dele, me disse: Quanto durará a tua ausência? Quando voltarás? Aprouve ao rei enviar-me, e marquei certo prazo.
\par 7 E ainda disse ao rei: Se ao rei parece bem, dêem-se-me cartas para os governadores dalém do Eufrates, para que me permitam passar e entrar em Judá,
\par 8 como também carta para Asafe, guarda das matas do rei, para que me dê madeira para as vigas das portas da cidadela do templo, para os muros da cidade e para a casa em que deverei alojar-me. E o rei mas deu, porque a boa mão do meu Deus era comigo.
\par 9 Então, fui aos governadores dalém do Eufrates e lhes entreguei as cartas do rei; ora, o rei tinha enviado comigo oficiais do exército e cavaleiros.
\par 10 Disto ficaram sabendo Sambalate, o horonita, e Tobias, o servo amonita; e muito lhes desagradou que alguém viesse a procurar o bem dos filhos de Israel.
\par 11 Cheguei a Jerusalém, onde estive três dias.
\par 12 Então, à noite me levantei, e uns poucos homens, comigo; não declarei a ninguém o que o meu Deus me pusera no coração para eu fazer em Jerusalém. Não havia comigo animal algum, senão o que eu montava.
\par 13 De noite, saí pela Porta do Vale, para o lado da Fonte do Dragão e para a Porta do Monturo e contemplei os muros de Jerusalém, que estavam assolados, cujas portas tinham sido consumidas pelo fogo.
\par 14 Passei à Porta da Fonte e ao açude do rei; mas não havia lugar por onde passasse o animal que eu montava.
\par 15 Subi à noite pelo ribeiro e contemplei ainda os muros; voltei, entrei pela Porta do Vale e tornei para casa.
\par 16 Não sabiam os magistrados aonde eu fora nem o que fazia, pois até aqui não havia eu declarado coisa alguma, nem aos judeus, nem aos sacerdotes, nem aos nobres, nem aos magistrados, nem aos mais que faziam a obra.
\par 17 Então, lhes disse: Estais vendo a miséria em que estamos, Jerusalém assolada, e as suas portas, queimadas; vinde, pois, reedifiquemos os muros de Jerusalém e deixemos de ser opróbrio.
\par 18 E lhes declarei como a boa mão do meu Deus estivera comigo e também as palavras que o rei me falara. Então, disseram: Disponhamo-nos e edifiquemos. E fortaleceram as mãos para a boa obra.
\par 19 Porém Sambalate, o horonita, e Tobias, o servo amonita, e Gesém, o arábio, quando o souberam, zombaram de nós, e nos desprezaram, e disseram: Que é isso que fazeis? Quereis rebelar-vos contra o rei?
\par 20 Então, lhes respondi: o Deus dos céus é quem nos dará bom êxito; nós, seus servos, nos disporemos e reedificaremos; vós, todavia, não tendes parte, nem direito, nem memorial em Jerusalém.

\chapter{3}

\par 1 Então, se dispôs Eliasibe, o sumo sacerdote, com os sacerdotes, seus irmãos, e reedificaram a Porta das Ovelhas; consagraram-na, assentaram-lhe as portas e continuaram a reconstrução até à Torre dos Cem e à Torre de Hananel.
\par 2 Junto a ele edificaram os homens de Jericó; também, ao seu lado, edificou Zacur, filho de Inri.
\par 3 Os filhos de Hassenaá edificaram a Porta do Peixe; colocaram-lhe as vigas e lhe assentaram as portas com seus ferrolhos e trancas.
\par 4 Ao seu lado, reparou Meremote, filho de Urias, filho de Coz; junto deste reparou Mesulão, filho de Berequias, filho de Mesezabel, a cujo lado reparou Zadoque, filho de Baaná.
\par 5 Ao lado destes, repararam os tecoítas; os seus nobres, porém, não se sujeitaram ao serviço do seu senhor.
\par 6 Joiada, filho de Paséia, e Mesulão, filho de Besodias, repararam a Porta Velha; colocaram-lhe as vigas e lhe assentaram as portas com seus ferrolhos e trancas.
\par 7 Junto deles, trabalharam Melatias, gibeonita, e Jadom, meronotita, homens de Gibeão e de Mispa, que pertenciam ao domínio do governador de além do Eufrates.
\par 8 Ao seu lado, reparou Uziel, filho de Haraías, um dos ourives; junto dele, Hananias, um dos perfumistas; e restauraram Jerusalém até ao Muro Largo.
\par 9 Junto a estes, trabalhou Refaías, filho de Hur, maioral da metade de Jerusalém.
\par 10 Ao seu lado, reparou Jedaías, filho de Harumafe, defronte da sua casa; e, ao seu lado, reparou Hatus, filho de Hasabnéias.
\par 11 A outra parte reparou Malquias, filho de Harim, e Hassube, filho de Paate-Moabe, como também a Torre dos Fornos.
\par 12 Ao lado dele, reparou Salum, filho de Haloés, maioral da outra meia parte de Jerusalém, ele e suas filhas.
\par 13 A Porta do Vale, reparou-a Hanum e os moradores de Zanoa; edificaram-na e lhe assentaram as portas com seus ferrolhos e trancas e ainda mil côvados da muralha, até à Porta do Monturo.
\par 14 A Porta do Monturo, reparou-a Malquias, filho de Recabe, maioral do distrito de Bete-Haquerém; ele a edificou e lhe assentou as portas com seus ferrolhos e trancas.
\par 15 A Porta da Fonte, reparou-a Salum, filho de Col-Hozé, maioral do distrito de Mispa; ele a edificou, e a cobriu, e lhe assentou as portas com seus ferrolhos e trancas, e ainda o muro do açude de Selá, junto ao jardim do rei, até aos degraus que descem da Cidade de Davi.
\par 16 Depois dele, reparou Neemias, filho de Azbuque, maioral da metade do distrito de Bete-Zur, até defronte dos sepulcros de Davi, até ao açude artificial e até à casa dos heróis.
\par 17 Depois dele, repararam os levitas, Reum, filho de Bani, e, ao seu lado, Hasabias, maioral da metade do distrito de Queila.
\par 18 Depois dele, repararam seus irmãos: Bavai, filho de Henadade, maioral da metade do distrito de Queila;
\par 19 ao seu lado, reparou Ezer, filho de Jesua, maioral de Mispa, outra parte defronte da subida para a casa das armas, no ângulo do muro.
\par 20 Depois dele, reparou com grande ardor Baruque, filho de Zabai, outra porção, desde o ângulo do muro até à porta da casa de Eliasibe, o sumo sacerdote.
\par 21 Depois dele, reparou Meremote, filho de Urias, filho de Coz, outra porção, desde a porta da casa de Eliasibe até à extremidade da casa de Eliasibe.
\par 22 Depois dele, repararam os sacerdotes que habitavam na campina.
\par 23 Depois, repararam Benjamim e Hassube, defronte da sua casa; depois deles, reparou Azarias, filho de Maaséias, filho de Ananias, junto à sua casa.
\par 24 Depois dele, reparou Binui, filho de Henadade, outra porção, desde a casa de Azarias até ao ângulo e até à esquina.
\par 25 Palal, filho de Uzai, reparou defronte do ângulo e da torre que sai da casa real superior, que está junto ao pátio do cárcere; depois dele, reparou Pedaías, filho de Parós,
\par 26 e os servos do templo que habitavam em Ofel, até defronte da Porta das Águas, para o oriente, e até à torre alta.
\par 27 Depois, repararam os tecoítas outra porção, defronte da torre grande e alta, e até ao Muro de Ofel.
\par 28 Para cima da Porta dos Cavalos, repararam os sacerdotes, cada um defronte da sua casa.
\par 29 Depois deles, reparou Zadoque, filho de Imer, defronte de sua casa; e, depois dele, Semaías, filho de Secanias, guarda da Porta Oriental.
\par 30 Depois dele, reparou Hananias, filho de Selemias, e Hanum, o sexto filho de Zalafe, outra porção; depois deles, reparou Mesulão, filho de Berequias, defronte da sua morada.
\par 31 Depois dele, reparou Malquias, filho de um ourives, até à casa dos servos do templo e dos mercadores, defronte da Porta da Guarda, até ao eirado da esquina.
\par 32 Entre o eirado da esquina e a Porta das Ovelhas, repararam os ourives e os mercadores.

\chapter{4}

\par 1 Tendo Sambalate ouvido que edificávamos o muro, ardeu em ira, e se indignou muito, e escarneceu dos judeus.
\par 2 Então, falou na presença de seus irmãos e do exército de Samaria e disse: Que fazem estes fracos judeus? Permitir-se-lhes-á isso? Sacrificarão? Darão cabo da obra num só dia? Renascerão, acaso, dos montões de pó as pedras que foram queimadas?
\par 3 Estava com ele Tobias, o amonita, e disse: Ainda que edifiquem, vindo uma raposa, derribará o seu muro de pedra.
\par 4 Ouve, ó nosso Deus, pois estamos sendo desprezados; caia o seu opróbrio sobre a cabeça deles, e faze que sejam despojo numa terra de cativeiro.
\par 5 Não lhes encubras a iniqüidade, e não se risque de diante de ti o seu pecado, pois te provocaram à ira, na presença dos que edificavam.
\par 6 Assim, edificamos o muro, e todo o muro se fechou até a metade de sua altura; porque o povo tinha ânimo para trabalhar.
\par 7 Mas, ouvindo Sambalate e Tobias, os arábios, os amonitas e os asdoditas que a reparação dos muros de Jerusalém ia avante e que já se começavam a fechar-lhe as brechas, ficaram sobremodo irados.
\par 8 Ajuntaram-se todos de comum acordo para virem atacar Jerusalém e suscitar confusão ali.
\par 9 Porém nós oramos ao nosso Deus e, como proteção, pusemos guarda contra eles, de dia e de noite.
\par 10 Então, disse Judá: Já desfaleceram as forças dos carregadores, e os escombros são muitos; de maneira que não podemos edificar o muro.
\par 11 Disseram, porém, os nossos inimigos: Nada saberão disto, nem verão, até que entremos no meio deles e os matemos; assim, faremos cessar a obra.
\par 12 Quando os judeus que habitavam na vizinhança deles, dez vezes, nos disseram: De todos os lugares onde moram, subirão contra nós,
\par 13 então, pus o povo, por famílias, nos lugares baixos e abertos, por detrás do muro, com as suas espadas, e as suas lanças, e os seus arcos;
\par 14 inspecionei, dispus-me e disse aos nobres, aos magistrados e ao resto do povo: não os temais; lembrai-vos do Senhor, grande e temível, e pelejai pelos vossos irmãos, vossos filhos, vossas filhas, vossa mulher e vossa casa.
\par 15 E sucedeu que, ouvindo os nossos inimigos que já o sabíamos e que Deus tinha frustrado o desígnio deles, voltamos todos nós ao muro, cada um à sua obra.
\par 16 Daquele dia em diante, metade dos meus moços trabalhava na obra, e a outra metade empunhava lanças, escudos, arcos e couraças; e os chefes estavam por detrás de toda a casa de Judá;
\par 17 os carregadores, que por si mesmos tomavam as cargas, cada um com uma das mãos fazia a obra e com a outra segurava a arma.
\par 18 Os edificadores, cada um trazia a sua espada à cinta, e assim edificavam; o que tocava a trombeta estava junto de mim.
\par 19 Disse eu aos nobres, aos magistrados e ao resto do povo: Grande e extensa é a obra, e nós estamos no muro mui separados, longe uns dos outros.
\par 20 No lugar em que ouvirdes o som da trombeta, para ali acorrei a ter conosco; o nosso Deus pelejará por nós.
\par 21 Assim trabalhávamos na obra; e metade empunhava as lanças desde o raiar do dia até ao sair das estrelas.
\par 22 Também nesse mesmo tempo disse eu ao povo: Cada um com o seu moço fique em Jerusalém, para que de noite nos sirvam de guarda e de dia trabalhem.
\par 23 Nem eu, nem meus irmãos, nem meus moços, nem os homens da guarda que me seguiam largávamos as nossas vestes; cada um se deitava com as armas à sua direita.

\chapter{5}

\par 1 Foi grande, porém, o clamor do povo e de suas mulheres contra os judeus, seus irmãos.
\par 2 Porque havia os que diziam: Somos muitos, nós, nossos filhos e nossas filhas; que se nos dê trigo, para que comamos e vivamos.
\par 3 Também houve os que diziam: As nossas terras, as nossas vinhas e as nossas casas hipotecamos para tomarmos trigo nesta fome.
\par 4 Houve ainda os que diziam: Tomamos dinheiro emprestado até para o tributo do rei, sobre as nossas terras e as nossas vinhas.
\par 5 No entanto, nós somos da mesma carne como eles, e nossos filhos são tão bons como os deles; e eis que sujeitamos nossos filhos e nossas filhas para serem escravos, algumas de nossas filhas já estão reduzidas à escravidão. Não está em nosso poder evitá-lo; pois os nossos campos e as nossas vinhas já são de outros.
\par 6 Ouvindo eu, pois, o seu clamor e estas palavras, muito me aborreci.
\par 7 Depois de ter considerado comigo mesmo, repreendi os nobres e magistrados e lhes disse: Sois usurários, cada um para com seu irmão; e convoquei contra eles um grande ajuntamento.
\par 8 Disse-lhes: nós resgatamos os judeus, nossos irmãos, que foram vendidos às gentes, segundo nossas posses; e vós outra vez negociaríeis vossos irmãos, para que sejam vendidos a nós?
\par 9 Então, se calaram e não acharam o que responder. Disse mais: não é bom o que fazeis; porventura não devíeis andar no temor do nosso Deus, por causa do opróbrio dos gentios, os nossos inimigos?
\par 10 Também eu, meus irmãos e meus moços lhes demos dinheiro emprestado e trigo. Demos de mão a esse empréstimo.
\par 11 Restituí-lhes hoje, vos peço, as suas terras, as suas vinhas, os seus olivais e as suas casas, como também o centésimo do dinheiro, do trigo, do vinho e do azeite, que exigistes deles.
\par 12 Então, responderam: Restituir-lhes-emos e nada lhes pediremos; faremos assim como dizes. Então, chamei os sacerdotes e os fiz jurar que fariam segundo prometeram.
\par 13 Também sacudi o meu regaço e disse: Assim o faça Deus, sacuda de sua casa e de seu trabalho a todo homem que não cumprir esta promessa; seja sacudido e despojado. E toda a congregação respondeu: Amém! E louvaram o SENHOR; e o povo fez segundo a sua promessa.
\par 14 Também desde o dia em que fui nomeado seu governador na terra de Judá, desde o vigésimo ano até ao trigésimo segundo ano do rei Artaxerxes, doze anos, nem eu nem meus irmãos comemos o pão devido ao governador.
\par 15 Mas os primeiros governadores, que foram antes de mim, oprimiram o povo e lhe tomaram pão e vinho, além de quarenta siclos de prata; até os seus moços dominavam sobre o povo, porém eu assim não fiz, por causa do temor de Deus.
\par 16 Antes, também na obra deste muro fiz reparação, e terra nenhuma compramos; e todos os meus moços se ajuntaram ali para a obra.
\par 17 Também cento e cinqüenta homens dos judeus e dos magistrados e os que vinham a nós, dentre as gentes que estavam ao nosso redor, eram meus hóspedes.
\par 18 O que se preparava para cada dia era um boi e seis ovelhas escolhidas; também à minha custa eram preparadas aves e, de dez em dez dias, muito vinho de todas as espécies; nem por isso exigi o pão devido ao governador, porquanto a servidão deste povo era grande.
\par 19 Lembra-te de mim para meu bem, ó meu Deus, e de tudo quanto fiz a este povo.

\chapter{6}

\par 1 Tendo ouvido Sambalate, Tobias, Gesém, o arábio, e o resto dos nossos inimigos que eu tinha edificado o muro e que nele já não havia brecha nenhuma, ainda que até este tempo não tinha posto as portas nos portais,
\par 2 Sambalate e Gesém mandaram dizer-me: Vem, encontremo-nos, nas aldeias, no vale de Ono. Porém intentavam fazer-me mal.
\par 3 Enviei-lhes mensageiros a dizer: Estou fazendo grande obra, de modo que não poderei descer; por que cessaria a obra, enquanto eu a deixasse e fosse ter convosco?
\par 4 Quatro vezes me enviaram o mesmo pedido; eu, porém, lhes dei sempre a mesma resposta.
\par 5 Então, Sambalate me enviou pela quinta vez o seu moço, o qual trazia na mão uma carta aberta,
\par 6 do teor seguinte: Entre as gentes se ouviu, e Gesém diz que tu e os judeus intentais revoltar-vos; por isso, reedificas o muro, e, segundo se diz, queres ser o rei deles,
\par 7 e puseste profetas para falarem a teu respeito em Jerusalém, dizendo: Este é rei em Judá. Ora, o rei ouvirá isso, segundo essas palavras. Vem, pois, agora, e consultemos juntamente.
\par 8 Mandei dizer-lhe: De tudo o que dizes coisa nenhuma sucedeu; tu, do teu coração, é que o inventas.
\par 9 Porque todos eles procuravam atemorizar-nos, dizendo: As suas mãos largarão a obra, e não se efetuará. Agora, pois, ó Deus, fortalece as minhas mãos.
\par 10 Tendo eu ido à casa de Semaías, filho de Delaías, filho de Meetabel (que estava encerrado), disse ele: Vamos juntamente à Casa de Deus, ao meio do templo, e fechemos as portas do templo; porque virão matar-te; aliás, de noite virão matar-te.
\par 11 Porém eu disse: homem como eu fugiria? E quem há, como eu, que entre no templo para que viva? De maneira nenhuma entrarei.
\par 12 Então, percebi que não era Deus quem o enviara; tal profecia falou ele contra mim, porque Tobias e Sambalate o subornaram.
\par 13 Para isto o subornaram, para me atemorizar, e para que eu, assim, viesse a proceder e a pecar, para que tivessem motivo de me infamar e me vituperassem.
\par 14 Lembra-te, meu Deus, de Tobias e de Sambalate, no tocante a estas suas obras, e também da profetisa Noadia e dos mais profetas que procuraram atemorizar-me.
\par 15 Acabou-se, pois, o muro aos vinte e cinco dias do mês de elul, em cinqüenta e dois dias.
\par 16 Sucedeu que, ouvindo-o todos os nossos inimigos, temeram todos os gentios nossos circunvizinhos e decaíram muito no seu próprio conceito; porque reconheceram que por intervenção de nosso Deus é que fizemos esta obra.
\par 17 Também naqueles dias alguns nobres de Judá escreveram muitas cartas, que iam para Tobias, e cartas de Tobias vinham para eles.
\par 18 Pois muitos em Judá lhe eram ajuramentados porque era genro de Secanias, filho de Ará; e seu filho Joanã se casara com a filha de Mesulão, filho de Berequias.
\par 19 Também das suas boas ações falavam na minha presença, e as minhas palavras lhe levavam a ele; Tobias escrevia cartas para me atemorizar.

\chapter{7}

\par 1 Ora, uma vez reedificado o muro e assentadas as portas, estabelecidos os porteiros, os cantores e os levitas,
\par 2 eu nomeei Hanani, meu irmão, e Hananias, maioral do castelo, sobre Jerusalém. Hananias era homem fiel e temente a Deus, mais do que muitos outros.
\par 3 E lhes disse: não se abram as portas de Jerusalém até que o sol aqueça e, enquanto os guardas ainda estão ali, que se fechem as portas e se tranquem; ponham-se guardas dos moradores de Jerusalém, cada um no seu posto diante de sua casa.
\par 4 A cidade era espaçosa e grande, mas havia pouca gente nela, e as casas não estavam edificadas ainda.
\par 5 Então, o meu Deus me pôs no coração que ajuntasse os nobres, os magistrados e o povo, para registrar as genealogias. Achei o livro da genealogia dos que subiram primeiro, e nele estava escrito:
\par 6 São estes os filhos da província que subiram do cativeiro, dentre os exilados, que Nabucodonosor, rei da Babilônia, levara para o exílio e que voltaram para Jerusalém e para Judá, cada um para a sua cidade,
\par 7 os quais vieram com Zorobabel, Jesua, Neemias, Azarias, Raamias, Naamani, Mordecai, Bilsã, Misperete, Bigvai, Neum e Baaná. Este é o número dos homens do povo de Israel:
\par 8 foram os filhos de Parós, dois mil cento e setenta e dois.
\par 9 Os filhos de Sefatias, trezentos e setenta e dois.
\par 10 Os filhos de Ará, seiscentos e cinqüenta e dois.
\par 11 Os filhos de Paate-Moabe, dos filhos de Jesua e de Joabe, dois mil oitocentos e dezoito.
\par 12 Os filhos de Elão, mil duzentos e cinqüenta e quatro.
\par 13 Os filhos de Zatu, oitocentos e quarenta e cinco.
\par 14 Os filhos de Zacai, setecentos e sessenta.
\par 15 Os filhos de Binui, seiscentos e quarenta e oito.
\par 16 Os filhos de Bebai, seiscentos e vinte e oito.
\par 17 Os filhos de Azgade, dois mil trezentos e vinte e dois.
\par 18 Os filhos de Adonicão, seiscentos e sessenta e sete.
\par 19 Os filhos de Bigvai, dois mil e sessenta e sete.
\par 20 Os filhos de Adim, seiscentos e cinqüenta e cinco.
\par 21 Os filhos de Ater, da família de Ezequias, noventa e oito.
\par 22 Os filhos de Hasum, trezentos e vinte e oito.
\par 23 Os filhos de Besai, trezentos e vinte e quatro.
\par 24 Os filhos de Harife, cento e doze.
\par 25 Os filhos de Gibeão, noventa e cinco.
\par 26 Os homens de Belém e de Netofa, cento e oitenta e oito.
\par 27 Os homens de Anatote, cento e vinte e oito.
\par 28 Os homens de Bete-Azmavete, quarenta e dois.
\par 29 Os homens de Quiriate-Jearim, Cefira e Beerote, setecentos e quarenta e três.
\par 30 Os homens de Ramá e Geba, seiscentos e vinte e um.
\par 31 Os homens de Micmás, cento e vinte e dois.
\par 32 Os homens de Betel e Ai, cento e vinte e três.
\par 33 Os homens do outro Nebo, cinqüenta e dois.
\par 34 Os filhos do outro Elão, mil duzentos e cinqüenta e quatro.
\par 35 Os filhos de Harim, trezentos e vinte.
\par 36 Os filhos de Jericó, trezentos e quarenta e cinco.
\par 37 Os filhos de Lode, Hadide e Ono, setecentos e vinte e um.
\par 38 Os filhos de Senaá, três mil novecentos e trinta.
\par 39 Os sacerdotes: os filhos de Jedaías, da casa de Jesua, novecentos e setenta e três.
\par 40 Os filhos de Imer, mil e cinqüenta e dois.
\par 41 Os filhos de Pasur, mil duzentos e quarenta e sete.
\par 42 Os filhos de Harim, mil e dezessete.
\par 43 Os levitas: os filhos de Jesua, de Cadmiel, dos filhos de Hodeva, setenta e quatro.
\par 44 Os cantores: os filhos de Asafe, cento e quarenta e oito.
\par 45 Os porteiros: os filhos de Salum, os filhos de Ater, os filhos de Talmom, os filhos de Acube, os filhos de Hatita, os filhos de Sobai, cento e trinta e oito.
\par 46 Os servidores do templo: os filhos de Zia, os filhos de Hasufa, os filhos de Tabaote,
\par 47 os filhos de Queros, os filhos de Sia, os filhos de Padom,
\par 48 os filhos de Lebana, os filhos de Hagaba, os filhos de Salmai,
\par 49 os filhos de Hanã, os filhos de Gidel, os filhos de Gaar,
\par 50 os filhos de Reaías, os filhos de Rezim, os filhos de Necoda,
\par 51 os filhos de Gazão, os filhos de Uzá, os filhos de Paséia,
\par 52 os filhos de Besai, os filhos de Meunim, os filhos de Nefusesim,
\par 53 os filhos de Baquebuque, os filhos de Hacufa, os filhos de Harur,
\par 54 os filhos de Bazlite, os filhos de Meída, os filhos de Harsa,
\par 55 os filhos de Barcos, os filhos de Sísera, os filhos de Tama,
\par 56 os filhos de Nesias e os filhos de Hatifa.
\par 57 Os filhos dos servos de Salomão: os filhos de Sotai, os filhos de Soferete, os filhos de Perida,
\par 58 os filhos de Jaala, os filhos de Darcom, os filhos de Gidel,
\par 59 os filhos de Sefatias, os filhos de Hatil, os filhos de Poquerete-Hazebaim e os filhos de Amom.
\par 60 Todos os servidores do templo e os filhos dos servos de Salomão, trezentos e noventa e dois.
\par 61 Os seguintes subiram de Tel-Melá, Tel-Harsa, Querube, Adom e Imer, porém não puderam provar que as suas famílias e a sua linhagem eram de Israel:
\par 62 os filhos de Delaías, os filhos de Tobias, os filhos de Necoda, seiscentos e quarenta e dois.
\par 63 Dos sacerdotes: os filhos de Habaías, os filhos de Coz, os filhos de Barzilai, o qual se casou com uma das filhas de Barzilai, o gileadita, e que foi chamado pelo nome dele.
\par 64 Estes procuraram o seu registro nos livros genealógicos, porém o não acharam; pelo que foram tidos por imundos para o sacerdócio.
\par 65 O governador lhes disse que não comessem das coisas sagradas, até que se levantasse um sacerdote com Urim e Tumim.
\par 66 Toda esta congregação junta foi de quarenta e dois mil trezentos e sessenta,
\par 67 afora os seus servos e as suas servas, que foram sete mil trezentos e trinta e sete; e tinham duzentos e quarenta e cinco cantores e cantoras.
\par 68 Os seus cavalos, setecentos e trinta e seis; os seus mulos, duzentos e quarenta e cinco.
\par 69 Camelos, quatrocentos e trinta e cinco; jumentos, seis mil setecentos e vinte.
\par 70 Alguns dos cabeças das famílias contribuíram para a obra. O governador deu para o tesouro, em ouro, mil daricos, cinqüenta bacias e quinhentas e trinta vestes sacerdotais.
\par 71 E alguns mais dos cabeças das famílias deram para o tesouro da obra, em ouro, vinte mil daricos e, em prata, dois mil e duzentos arráteis.
\par 72 O que deu o restante do povo foi, em ouro, vinte mil daricos, e dois mil arráteis em prata, e sessenta e sete vestes sacerdotais.
\par 73 Os sacerdotes, os levitas, os porteiros, os cantores, alguns do povo, os servidores do templo e todo o Israel habitavam nas suas cidades.

\chapter{8}

\par 1 Em chegando o sétimo mês, e estando os filhos de Israel nas suas cidades, todo o povo se ajuntou como um só homem, na praça, diante da Porta das Águas; e disseram a Esdras, o escriba, que trouxesse o Livro da Lei de Moisés, que o SENHOR tinha prescrito a Israel.
\par 2 Esdras, o sacerdote, trouxe a Lei perante a congregação, tanto de homens como de mulheres e de todos os que eram capazes de entender o que ouviam. Era o primeiro dia do sétimo mês.
\par 3 E leu no livro, diante da praça, que está fronteira à Porta das Águas, desde a alva até ao meio-dia, perante homens e mulheres e os que podiam entender; e todo o povo tinha os ouvidos atentos ao Livro da Lei.
\par 4 Esdras, o escriba, estava num púlpito de madeira, que fizeram para aquele fim; estavam em pé junto a ele, à sua direita, Matitias, Sema, Anaías, Urias, Hilquias e Maaséias; e à sua esquerda, Pedaías, Misael, Malquias, Hasum, Hasbadana, Zacarias e Mesulão.
\par 5 Esdras abriu o livro à vista de todo o povo, porque estava acima dele; abrindo-o ele, todo o povo se pôs em pé.
\par 6 Esdras bendisse ao SENHOR, o grande Deus; e todo o povo respondeu: Amém! Amém! E, levantando as mãos; inclinaram-se e adoraram o SENHOR, com o rosto em terra.
\par 7 E Jesua, Bani, Serebias, Jamim, Acube, Sabetai, Hodias, Maaséias, Quelita, Azarias, Jozabade, Hanã, Pelaías e os levitas ensinavam o povo na Lei; e o povo estava no seu lugar.
\par 8 Leram no livro, na Lei de Deus, claramente, dando explicações, de maneira que entendessem o que se lia.
\par 9 Neemias, que era o governador, e Esdras, sacerdote e escriba, e os levitas que ensinavam todo o povo lhe disseram: Este dia é consagrado ao SENHOR, vosso Deus, pelo que não pranteeis, nem choreis. Porque todo o povo chorava, ouvindo as palavras da Lei.
\par 10 Disse-lhes mais: ide, comei carnes gordas, tomai bebidas doces e enviai porções aos que não têm nada preparado para si; porque este dia é consagrado ao nosso Senhor; portanto, não vos entristeçais, porque a alegria do SENHOR é a vossa força.
\par 11 Os levitas fizeram calar todo o povo, dizendo: Calai-vos, porque este dia é santo; e não estejais contristados.
\par 12 Então, todo o povo se foi a comer, a beber, a enviar porções e a regozijar-se grandemente, porque tinham entendido as palavras que lhes foram explicadas.
\par 13 No dia seguinte, ajuntaram-se a Esdras, o escriba, os cabeças das famílias de todo o povo, os sacerdotes e os levitas, e isto para atentarem nas palavras da Lei.
\par 14 Acharam escrito na Lei que o SENHOR ordenara por intermédio de Moisés que os filhos de Israel habitassem em cabanas, durante a festa do sétimo mês;
\par 15 que publicassem e fizessem passar pregão por todas as suas cidades e em Jerusalém, dizendo: Saí ao monte e trazei ramos de oliveiras, ramos de zambujeiros, ramos de murtas, ramos de palmeiras e ramos de árvores frondosas, para fazer cabanas, como está escrito.
\par 16 Saiu, pois, o povo, trouxeram os ramos e fizeram para si cabanas, cada um no seu terraço, e nos seus pátios, e nos átrios da Casa de Deus, e na praça da Porta das Águas, e na praça da Porta de Efraim.
\par 17 Toda a congregação dos que tinham voltado do cativeiro fez cabanas e nelas habitou; porque nunca fizeram assim os filhos de Israel, desde os dias de Josué, filho de Num, até àquele dia; e houve mui grande alegria.
\par 18 Dia após dia, leu Esdras no Livro da Lei de Deus, desde o primeiro dia até ao último; e celebraram a festa por sete dias; no oitavo dia, houve uma assembléia solene, segundo o prescrito.

\chapter{9}

\par 1 No dia vinte e quatro deste mês, se ajuntaram os filhos de Israel com jejum e pano de saco e traziam terra sobre si.
\par 2 Os da linhagem de Israel se apartaram de todos os estranhos, puseram-se em pé e fizeram confissão dos seus pecados e das iniqüidades de seus pais.
\par 3 Levantando-se no seu lugar, leram no Livro da Lei do SENHOR, seu Deus, uma quarta parte do dia; em outra quarta parte dele fizeram confissão e adoraram o SENHOR, seu Deus.
\par 4 Jesua, Bani, Cadmiel, Sebanias, Buni, Serebias, Bani e Quenani se puseram em pé no estrado dos levitas e clamaram em alta voz ao SENHOR, seu Deus.
\par 5 Os levitas Jesua, Cadmiel, Bani, Hasabnéias, Serebias, Hodias, Sebanias e Petaías disseram: Levantai-vos, bendizei ao SENHOR, vosso Deus, de eternidade em eternidade. Então, se disse: Bendito seja o nome da tua glória, que ultrapassa todo bendizer e louvor.
\par 6 Só tu és SENHOR, tu fizeste o céu, o céu dos céus e todo o seu exército, a terra e tudo quanto nela há, os mares e tudo quanto há neles; e tu os preservas a todos com vida, e o exército dos céus te adora.
\par 7 Tu és o SENHOR, o Deus que elegeste Abrão, e o tiraste de Ur dos caldeus, e lhe puseste por nome Abraão.
\par 8 Achaste o seu coração fiel perante ti e com ele fizeste aliança, para dares à sua descendência a terra dos cananeus, dos heteus, dos amorreus, dos ferezeus, dos jebuseus e dos girgaseus; e cumpriste as tuas promessas, porquanto és justo.
\par 9 Viste a aflição de nossos pais no Egito, e lhes ouviste o clamor junto ao mar Vermelho.
\par 10 Fizeste sinais e milagres contra Faraó e seus servos e contra todo o povo da sua terra, porque soubeste que os trataram com soberba; e, assim, adquiriste renome, como hoje se vê.
\par 11 Dividiste o mar perante eles, de maneira que o atravessaram em seco; lançaste os seus perseguidores nas profundezas, como uma pedra nas águas impetuosas.
\par 12 Guiaste-os, de dia, por uma coluna de nuvem e, de noite, por uma coluna de fogo, para lhes alumiar o caminho por onde haviam de ir.
\par 13 Desceste sobre o monte Sinai, do céu falaste com eles e lhes deste juízos retos, leis verdadeiras, estatutos e mandamentos bons.
\par 14 O teu santo sábado lhes fizeste conhecer; preceitos, estatutos e lei, por intermédio de Moisés, teu servo, lhes mandaste.
\par 15 Pão dos céus lhes deste na sua fome e água da rocha lhes fizeste brotar na sua sede; e lhes disseste que entrassem para possuírem a terra que, com mão levantada, lhes juraste dar.
\par 16 Porém eles, nossos pais, se houveram soberbamente, e endureceram a sua cerviz, e não deram ouvidos aos teus mandamentos.
\par 17 Recusaram ouvir-te e não se lembraram das tuas maravilhas, que lhes fizeste; endureceram a sua cerviz e na sua rebelião levantaram um chefe, com o propósito de voltarem para a sua servidão no Egito. Porém tu, ó Deus perdoador, clemente e misericordioso, tardio em irar-te e grande em bondade, tu não os desamparaste,
\par 18 ainda mesmo quando fizeram para si um bezerro de fundição e disseram: Este é o teu Deus, que te tirou do Egito; e cometeram grandes blasfêmias.
\par 19 Todavia, tu, pela multidão das tuas misericórdias, não os deixaste no deserto. A coluna de nuvem nunca se apartou deles de dia, para os guiar pelo caminho, nem a coluna de fogo de noite, para lhes alumiar o caminho por onde haviam de ir.
\par 20 E lhes concedeste o teu bom Espírito, para os ensinar; não lhes negaste para a boca o teu maná; e água lhes deste na sua sede.
\par 21 Desse modo os sustentaste quarenta anos no deserto, e nada lhes faltou; as suas vestes não envelheceram, e os seus pés não se incharam.
\par 22 Também lhes deste reinos e povos, que lhes repartiste em porções; assim, possuíram a terra de Seom, a saber, a terra do rei de Hesbom e a terra de Ogue, rei de Basã.
\par 23 Multiplicaste os seus filhos como as estrelas do céu e trouxeste-os à terra de que tinhas dito a seus pais que nela entrariam para a possuírem.
\par 24 Entraram os filhos e tomaram posse da terra; abateste perante eles os moradores da terra, os cananeus, e lhos entregaste nas mãos, como também os reis e os povos da terra, para fazerem deles segundo a sua vontade.
\par 25 Tomaram cidades fortificadas e terra fértil e possuíram casas cheias de toda sorte de coisas boas, cisternas cavadas, vinhas e olivais e árvores frutíferas em abundância; comeram, e se fartaram, e engordaram, e viveram em delícias, pela tua grande bondade.
\par 26 Ainda assim foram desobedientes e se revoltaram contra ti; viraram as costas à tua lei e mataram os teus profetas, que protestavam contra eles, para os fazerem voltar a ti; e cometeram grandes blasfêmias.
\par 27 Pelo que os entregaste nas mãos dos seus opressores, que os angustiaram; mas no tempo de sua angústia, clamando eles a ti, dos céus tu os ouviste; e, segundo a tua grande misericórdia, lhes deste libertadores que os salvaram das mãos dos que os oprimiam.
\par 28 Porém, quando se viam em descanso, tornavam a fazer o mal diante de ti; e tu os desamparavas nas mãos dos seus inimigos, para que dominassem sobre eles; mas, convertendo-se eles e clamando a ti, tu os ouviste dos céus e, segundo a tua misericórdia, os livraste muitas vezes.
\par 29 Testemunhaste contra eles, para que voltassem à tua lei; porém eles se houveram soberbamente e não deram ouvidos aos teus mandamentos, mas pecaram contra os teus juízos, pelo cumprimento dos quais o homem viverá; obstinadamente deram de ombros, endureceram a cerviz e não quiseram ouvir.
\par 30 No entanto, os aturaste por muitos anos e testemunhaste contra eles pelo teu Espírito, por intermédio dos teus profetas; porém eles não deram ouvidos; pelo que os entregaste nas mãos dos povos de outras terras.
\par 31 Mas, pela tua grande misericórdia, não acabaste com eles nem os desamparaste; porque tu és Deus clemente e misericordioso.
\par 32 Agora, pois, ó Deus nosso, ó Deus grande, poderoso e temível, que guardas a aliança e a misericórdia, não menosprezes toda a aflição que nos sobreveio, a nós, aos nossos reis, aos nossos príncipes, aos nossos sacerdotes, aos nossos profetas, aos nossos pais e a todo o teu povo, desde os dias dos reis da Assíria até ao dia de hoje.
\par 33 Porque tu és justo em tudo quanto tem vindo sobre nós; pois tu fielmente procedeste, e nós, perversamente.
\par 34 Os nossos reis, os nossos príncipes, os nossos sacerdotes e os nossos pais não guardaram a tua lei, nem deram ouvidos aos teus mandamentos e aos teus testemunhos, que testificaste contra eles.
\par 35 Pois eles no seu reino, na muita abundância de bens que lhes deste, na terra espaçosa e fértil que puseste diante deles não te serviram, nem se converteram de suas más obras.
\par 36 Eis que hoje somos servos; e até na terra que deste a nossos pais, para comerem o seu fruto e o seu bem, eis que somos servos nela.
\par 37 Seus abundantes produtos são para os reis que puseste sobre nós por causa dos nossos pecados; e, segundo a sua vontade, dominam sobre o nosso corpo e sobre o nosso gado; estamos em grande angústia.
\par 38 Por causa de tudo isso, estabelecemos aliança fiel e o escrevemos; e selaram-na os nossos príncipes, os nossos levitas e os nossos sacerdotes.

\chapter{10}

\par 1 Os que selaram foram: Neemias, o governador, filho de Hacalias, e Zedequias,
\par 2 Seraías, Azarias, Jeremias,
\par 3 Pasur, Amarias, Malquias,
\par 4 Hatus, Sebanias, Maluque,
\par 5 Harim, Meremote, Obadias,
\par 6 Daniel, Ginetom, Baruque,
\par 7 Mesulão, Abias, Miamim,
\par 8 Maazias, Bilgai, Semaías; estes eram os sacerdotes.
\par 9 E os levitas: Jesua, filho de Azanias, Binui, dos filhos de Henadade, Cadmiel
\par 10 e os irmãos deles: Sebanias, Hodias, Quelita, Pelaías, Hanã,
\par 11 Mica, Reobe, Hasabias,
\par 12 Zacur, Serebias, Sebanias,
\par 13 Hodias, Bani e Beninu.
\par 14 Os chefes do povo: Parós, Paate-Moabe, Elão, Zatu, Bani,
\par 15 Buni, Azgade, Bebai,
\par 16 Adonias, Bigvai, Adim,
\par 17 Ater, Ezequias, Azur,
\par 18 Hodias, Hasum, Besai,
\par 19 Harife, Anatote, Nebai,
\par 20 Magpias, Mesulão, Hezir,
\par 21 Mesezabel, Zadoque, Jadua,
\par 22 Pelatias, Hanã, Anaías,
\par 23 Oséias, Hananias, Hassube,
\par 24 Haloés, Pilha, Sobeque,
\par 25 Reum, Hasabna, Maaséias,
\par 26 Aías, Hanã, Anã,
\par 27 Maluque, Harim e Baaná.
\par 28 O resto do povo, os sacerdotes, os levitas, os porteiros, os cantores, os servidores do templo e todos os que se tinham separado dos povos de outras terras para a Lei de Deus, suas mulheres, seus filhos e suas filhas, todos os que tinham saber e entendimento,
\par 29 firmemente aderiram a seus irmãos; seus nobres convieram, numa imprecação e num juramento, de que andariam na Lei de Deus, que foi dada por intermédio de Moisés, servo de Deus, de que guardariam e cumpririam todos os mandamentos do SENHOR, nosso Deus, e os seus juízos e os seus estatutos;
\par 30 de que não dariam as suas filhas aos povos da terra, nem tomariam as filhas deles para os seus filhos;
\par 31 de que, trazendo os povos da terra no dia de sábado qualquer mercadoria e qualquer cereal para venderem, nada comprariam deles no sábado, nem no dia santificado; e de que, no ano sétimo, abririam mão da colheita e de toda e qualquer cobrança.
\par 32 Também sobre nós pusemos preceitos, impondo-nos cada ano a terça parte de um siclo para o serviço da casa do nosso Deus,
\par 33 e para os pães da proposição, e para a contínua oferta de manjares, e para o contínuo holocausto dos sábados e das Festas da Lua Nova, e para as festas fixas, e para as coisas sagradas, e para as ofertas pelo pecado, e para fazer expiação por Israel, e para toda a obra da casa do nosso Deus.
\par 34 Nós, os sacerdotes, os levitas e o povo deitamos sortes acerca da oferta da lenha que se havia de trazer à casa do nosso Deus, segundo as nossas famílias, a tempos determinados, de ano em ano, para se queimar sobre o altar do SENHOR, nosso Deus, como está escrito na Lei.
\par 35 E que também traríamos as primícias da nossa terra e todas as primícias de todas as árvores frutíferas, de ano em ano, à Casa do SENHOR;
\par 36 os primogênitos dos nossos filhos e os do nosso gado, como está escrito na Lei; e que os primogênitos das nossas manadas e das nossas ovelhas traríamos à casa do nosso Deus, aos sacerdotes que ministram nela.
\par 37 As primícias da nossa massa, as nossas ofertas, o fruto de toda árvore, o vinho e o azeite traríamos aos sacerdotes, às câmaras da casa do nosso Deus; os dízimos da nossa terra, aos levitas, pois a eles cumpre receber os dízimos em todas as cidades onde há lavoura.
\par 38 O sacerdote, filho de Arão, estaria com os levitas quando estes recebessem os dízimos, e os levitas trariam os dízimos dos dízimos à casa do nosso Deus, às câmaras da casa do tesouro.
\par 39 Porque àquelas câmaras os filhos de Israel e os filhos de Levi devem trazer ofertas do cereal, do vinho e do azeite; porquanto se acham ali os vasos do santuário, como também os sacerdotes que ministram, e os porteiros, e os cantores; e, assim, não desampararíamos a casa do nosso Deus.

\chapter{11}

\par 1 Os príncipes do povo habitaram em Jerusalém, mas o seu restante deitou sortes para trazer um de dez para que habitasse na santa cidade de Jerusalém; e as nove partes permaneceriam em outras cidades.
\par 2 O povo bendisse todos os homens que voluntariamente se ofereciam ainda para habitar em Jerusalém.
\par 3 São estes os chefes da província que habitaram em Jerusalém; porém nas cidades de Judá habitou cada um na sua possessão, nas suas cidades, a saber, Israel, os sacerdotes, os levitas, os servidores do templo e os filhos dos servos de Salomão.
\par 4 Habitaram, pois, em Jerusalém alguns dos filhos de Judá e dos filhos de Benjamim. Dos filhos de Judá: Ataías, filho de Uzias, filho de Zacarias, filho de Amarias, filho de Sefatias, filho de Maalalel, dos filhos de Perez;
\par 5 e Maaséias, filho de Baruque, filho de Col-Hozé, filho de Hazaías, filho de Adaías, filho de Joiaribe, filho de Zacarias, filho do silonita.
\par 6 Todos os filhos de Perez que habitaram em Jerusalém foram quatrocentos e sessenta e oito homens valentes.
\par 7 São estes os filhos de Benjamim: Salu, filho de Mesulão, filho de Joede, filho de Pedaías, filho de Colaías, filho de Maaséias, filho de Itiel, filho de Jesaías.
\par 8 Depois dele, Gabai e Salai; ao todo, novecentos e vinte e oito.
\par 9 Joel, filho de Zicri, superintendente deles; e Judá, filho de Senua, o segundo sobre a cidade.
\par 10 Dos sacerdotes: Jedaías, filho de Joiaribe, Jaquim,
\par 11 Seraías, filho de Hilquias, filho de Mesulão, filho de Zadoque, filho de Meraiote,
\par 12 filho de Aitube, príncipe da Casa de Deus, e os irmãos deles, que faziam o serviço do templo, oitocentos e vinte e dois; e Adaías, filho de Jeroão, filho de Pelalias, filho de Anzi, filho de Zacarias, filho de Pasur, filho de Malquias,
\par 13 e seus irmãos, cabeças de famílias, duzentos e quarenta e dois; e Amasai, filho de Azarel, filho de Azai, filho de Mesilemote, filho de Imer,
\par 14 e os irmãos deles, homens valentes, cento e vinte e oito; e, superintendente deles, Zabdiel, filho de Gedolim.
\par 15 Dos levitas: Semaías, filho de Hassube, filho de Azricão, filho de Hasabias, filho de Buni;
\par 16 Sabetai e Jozabade, dos cabeças dos levitas, que presidiam o serviço de fora da Casa de Deus;
\par 17 Matanias, filho de Mica, filho de Zabdi, filho de Asafe, o chefe, que dirigia os louvores nas orações, e Baquebuquias, o segundo de seus irmãos; depois, Abda, filho de Samua, filho de Galal, filho de Jedutum.
\par 18 Todos os levitas na santa cidade foram duzentos e oitenta e quatro.
\par 19 Dos porteiros: Acube, Talmom e os irmãos deles, os guardas das portas, cento e setenta e dois.
\par 20 O restante de Israel, dos sacerdotes e levitas se estabeleceu em todas as cidades de Judá, cada um na sua herança.
\par 21 Os servidores do templo habitaram em Ofel e estavam a cargo de Zia e Gispa.
\par 22 O superintendente dos levitas em Jerusalém era Uzi, filho de Bani, filho de Hasabias, filho de Matanias, filho de Mica, dos filhos de Asafe, que eram cantores ao serviço da Casa de Deus.
\par 23 Porque havia um mandado do rei a respeito deles e certo acordo com os cantores, concernente às obrigações de cada dia.
\par 24 Petaías, filho de Mesezabel, dos filhos de Zera, filho de Judá, estava à disposição do rei, em todos os negócios do povo.
\par 25 Quanto às aldeias, com os seus campos, alguns dos filhos de Judá habitaram em Quiriate-Arba e suas aldeias, em Dibom e suas aldeias, em Jecabzeel e suas aldeias,
\par 26 e em Jesua, em Moladá, em Bete-Palete,
\par 27 em Hazar-Sual, em Berseba e suas aldeias;
\par 28 em Ziclague, em Mecona e suas aldeias;
\par 29 em En-Rimom, em Zorá, em Jarmute;
\par 30 em Zanoa, em Adulão e nas aldeias delas; em Laquis e em seus campos, em Azeca e suas aldeias. Acamparam-se desde Berseba até ao vale de Hinom.
\par 31 Os filhos de Benjamim também se estabeleceram em Geba e daí em diante, em Micmás, Aia, Betel e suas aldeias;
\par 32 em Anatote, em Nobe, em Ananias,
\par 33 em Hazor, em Ramá, em Gitaim,
\par 34 em Hadide, em Zeboim, em Nebalate,
\par 35 em Lode e em Ono, no vale dos Artífices.
\par 36 Dos levitas, havia grupos tanto em Judá como em Benjamim.

\chapter{12}

\par 1 São estes os sacerdotes e levitas que subiram com Zorobabel, filho de Sealtiel, e com Jesua: Seraías, Jeremias, Esdras,
\par 2 Amarias, Maluque, Hatus,
\par 3 Secanias, Reum, Meremote,
\par 4 Ido, Ginetoi, Abias,
\par 5 Miamim, Maadias, Bilga,
\par 6 Semaías, Joiaribe, Jedaías,
\par 7 Salu, Amoque, Hilquias e Jedaías; estes foram os chefes dos sacerdotes e de seus irmãos, nos dias de Jesua.
\par 8 Também os levitas Jesua, Binui, Cadmiel, Serebias, Judá e Matanias; este e seus irmãos dirigiam os louvores.
\par 9 Baquebuquias e Uni, seus irmãos, estavam defronte deles, cada qual no seu mister.
\par 10 Jesua gerou a Joiaquim, Joiaquim gerou a Eliasibe, Eliasibe gerou a Joiada,
\par 11 Joiada gerou a Jônatas, e Jônatas gerou a Jadua.
\par 12 Nos dias de Joiaquim, foram sacerdotes, cabeças de famílias: de Seraías, Meraías; de Jeremias, Hananias;
\par 13 de Esdras, Mesulão; de Amarias, Joanã;
\par 14 de Maluqui, Jônatas; de Sebanias, José;
\par 15 de Harim, Adna; de Meraiote, Helcai;
\par 16 de Ido, Zacarias; de Ginetom, Mesulão;
\par 17 de Abias, Zicri; de Miniamim e de Moadias, Piltai;
\par 18 de Bilga, Samua; de Semaías, Jônatas;
\par 19 de Joiaribe, Matenai; de Jedaías, Uzi;
\par 20 de Salai, Calai; de Amoque, Héber;
\par 21 de Hilquias, Hasabias; de Jedaías, Netanel.
\par 22 Dos levitas, nos dias de Eliasibe, foram inscritos como cabeças de famílias Joiada, Joanã e Jadua, como também os sacerdotes, até ao reinado de Dario, o persa.
\par 23 Os filhos de Levi foram inscritos como cabeças de famílias no Livro das Crônicas, até aos dias de Joanã, filho de Eliasibe.
\par 24 Foram, pois, chefes dos levitas: Hasabias, Serebias e Jesua, filho de Cadmiel; os irmãos deles lhes estavam fronteiros para louvarem e darem graças, segundo o mandado de Davi, homem de Deus, coro contra coro.
\par 25 Matanias, Baquebuquias, Obadias, Mesulão, Talmom e Acube eram porteiros e faziam a guarda aos depósitos das portas.
\par 26 Estes viveram nos dias de Joiaquim, filho de Jesua, filho de Jozadaque, e nos dias de Neemias, o governador, e de Esdras, o sacerdote e escriba.
\par 27 Na dedicação dos muros de Jerusalém, procuraram aos levitas de todos os seus lugares, para fazê-los vir a fim de que fizessem a dedicação com alegria, louvores, canto, címbalos, alaúdes e harpas.
\par 28 Ajuntaram-se os filhos dos cantores, tanto da campina dos arredores de Jerusalém como das aldeias dos netofatitas,
\par 29 como também de Bete-Gilgal e dos campos de Geba e de Azmavete; porque os cantores tinham edificado para si aldeias nos arredores de Jerusalém.
\par 30 Purificaram-se os sacerdotes e os levitas, que também purificaram o povo e as portas e o muro.
\par 31 Então, fiz subir os príncipes de Judá sobre o muro e formei dois grandes coros em procissão, sendo um à mão direita sobre a muralha para o lado da Porta do Monturo.
\par 32 Após eles, ia Hosaías e a metade dos príncipes de Judá,
\par 33 Azarias, Esdras, Mesulão,
\par 34 Judá, Benjamim, Semaías e Jeremias;
\par 35 e dos filhos dos sacerdotes, com trombetas: Zacarias, filho de Jônatas, filho de Semaías, filho de Matanias, filho de Micaías, filho de Zacur, filho de Asafe,
\par 36 e seus irmãos, Semaías, Azarel, Milalai, Gilalai, Maai, Netanel, Judá e Hanani, com os instrumentos músicos de Davi, homem de Deus; Esdras, o escriba, ia adiante deles.
\par 37 À entrada da Porta da Fonte, subiram diretamente as escadas da Cidade de Davi, onde se eleva o muro por sobre a casa de Davi, até à Porta das Águas, do lado oriental.
\par 38 O segundo coro ia em frente, e eu, após ele; metade do povo ia por cima do muro, desde a Torre dos Fornos até ao Muro Largo;
\par 39 e desde a Porta de Efraim, passaram por cima da Porta Velha e da Porta do Peixe, pela Torre de Hananel, pela Torre dos Cem, até à Porta do Gado; e pararam à Porta da Guarda.
\par 40 Então, ambos os coros pararam na Casa de Deus, como também eu e a metade dos magistrados comigo.
\par 41 Os sacerdotes Eliaquim, Maaséias, Miniamim, Micaías, Elioenai, Zacarias e Hananias iam com trombetas,
\par 42 como também Maaséias, Semaías, Eleazar, Uzi, Joanã, Malquias, Elão e Ezer; e faziam-se ouvir os cantores sob a direção de Jezraías.
\par 43 No mesmo dia, ofereceram grandes sacrifícios e se alegraram; pois Deus os alegrara com grande alegria; também as mulheres e os meninos se alegraram, de modo que o júbilo de Jerusalém se ouviu até de longe.
\par 44 Ainda no mesmo dia, se nomearam homens para as câmaras dos tesouros, das ofertas, das primícias e dos dízimos, para ajuntarem nelas, das cidades, as porções designadas pela Lei para os sacerdotes e para os levitas; pois Judá estava alegre, porque os sacerdotes e os levitas ministravam ali;
\par 45 e executavam o serviço do seu Deus e o da purificação; como também os cantores e porteiros, segundo o mandado de Davi e de seu filho Salomão.
\par 46 Pois já outrora, nos dias de Davi e de Asafe, havia chefes dos cantores, cânticos de louvor e ações de graças a Deus.
\par 47 Todo o Israel, nos dias de Zorobabel e nos dias de Neemias, dava aos cantores e aos porteiros as porções de cada dia; e consagrava as coisas destinadas aos levitas, e os levitas, as destinadas aos filhos de Arão.

\chapter{13}

\par 1 Naquele dia, se leu para o povo no Livro de Moisés; achou-se escrito que os amonitas e os moabitas não entrassem jamais na congregação de Deus,
\par 2 porquanto não tinham saído ao encontro dos filhos de Israel com pão e água; antes, assalariaram contra eles Balaão para os amaldiçoar; mas o nosso Deus converteu a maldição em bênção.
\par 3 Ouvindo eles, o povo, esta lei, apartaram de Israel todo elemento misto.
\par 4 Ora, antes disto, Eliasibe, sacerdote, encarregado da câmara da casa do nosso Deus, se tinha aparentado com Tobias;
\par 5 e fizera para este uma câmara grande, onde dantes se depositavam as ofertas de manjares, o incenso, os utensílios e os dízimos dos cereais, do vinho e do azeite, que se ordenaram para os levitas, cantores e porteiros, como também contribuições para os sacerdotes.
\par 6 Mas, quando isso aconteceu, não estive em Jerusalém, porque no trigésimo segundo ano de Artaxerxes, rei da Babilônia, eu fora ter com ele; mas ao cabo de certo tempo pedi licença ao rei e voltei para Jerusalém.
\par 7 Então, soube do mal que Eliasibe fizera para beneficiar a Tobias, fazendo-lhe uma câmara nos pátios da Casa de Deus.
\par 8 Isso muito me indignou a tal ponto, que atirei todos os móveis da casa de Tobias fora da câmara.
\par 9 Então, ordenei que se purificassem as câmaras e tornei a trazer para ali os utensílios da Casa de Deus, com as ofertas de manjares e o incenso.
\par 10 Também soube que os quinhões dos levitas não se lhes davam, de maneira que os levitas e os cantores, que faziam o serviço, tinham fugido cada um para o seu campo.
\par 11 Então, contendi com os magistrados e disse: Por que se desamparou a Casa de Deus? Ajuntei os levitas e os cantores e os restituí a seus postos.
\par 12 Então, todo o Judá trouxe os dízimos dos cereais, do vinho e do azeite aos depósitos.
\par 13 Por tesoureiros dos depósitos pus Selemias, o sacerdote, Zadoque, o escrivão, e, dentre os levitas, Pedaías; como assistente deles, Hanã, filho de Zacur, filho de Matanias; porque foram achados fiéis, e se lhes encarregou que repartissem as porções para seus irmãos.
\par 14 Por isto, Deus meu, lembra-te de mim e não apagues as beneficências que eu fiz à casa de meu Deus e para o seu serviço.
\par 15 Naqueles dias, vi em Judá os que pisavam lagares ao sábado e traziam trigo que carregavam sobre jumentos; como também vinho, uvas e figos e toda sorte de cargas, que traziam a Jerusalém no dia de sábado; e protestei contra eles por venderem mantimentos neste dia.
\par 16 Também habitavam em Jerusalém tírios que traziam peixes e toda sorte de mercadorias, que no sábado vendiam aos filhos de Judá e em Jerusalém.
\par 17 Contendi com os nobres de Judá e lhes disse: Que mal é este que fazeis, profanando o dia de sábado?
\par 18 Acaso, não fizeram vossos pais assim, e não trouxe o nosso Deus todo este mal sobre nós e sobre esta cidade? E vós ainda trazeis ira maior sobre Israel, profanando o sábado.
\par 19 Dando já sombra as portas de Jerusalém antes do sábado, ordenei que se fechassem; e determinei que não se abrissem, senão após o sábado; às portas coloquei alguns dos meus moços, para que nenhuma carga entrasse no dia de sábado.
\par 20 Então, os negociantes e os vendedores de toda sorte de mercadorias pernoitaram fora de Jerusalém, uma ou duas vezes.
\par 21 Protestei, pois, contra eles e lhes disse: Por que passais a noite defronte do muro? Se outra vez o fizerdes, lançarei mão sobre vós. Daí em diante não tornaram a vir no sábado.
\par 22 Também mandei aos levitas que se purificassem e viessem guardar as portas, para santificar o dia de sábado. Também nisto, Deus meu, lembra-te de mim; e perdoa-me segundo a abundância da tua misericórdia.
\par 23 Vi também, naqueles dias, que judeus haviam casado com mulheres asdoditas, amonitas e moabitas.
\par 24 Seus filhos falavam meio asdodita e não sabiam falar judaico, mas a língua de seu respectivo povo.
\par 25 Contendi com eles, e os amaldiçoei, e espanquei alguns deles, e lhes arranquei os cabelos, e os conjurei por Deus, dizendo: Não dareis mais vossas filhas a seus filhos e não tomareis mais suas filhas, nem para vossos filhos nem para vós mesmos.
\par 26 Não pecou nisto Salomão, rei de Israel? Todavia, entre muitas nações não havia rei semelhante a ele, e ele era amado do seu Deus, e Deus o constituiu rei sobre todo o Israel. Não obstante isso, as mulheres estrangeiras o fizeram cair no pecado.
\par 27 Dar-vos-íamos nós ouvidos, para fazermos todo este grande mal, prevaricando contra o nosso Deus, casando com mulheres estrangeiras?
\par 28 Um dos filhos de Joiada, filho do sumo sacerdote Eliasibe, era genro de Sambalate, o horonita, pelo que o afugentei de mim.
\par 29 Lembra-te deles, Deus meu, pois contaminaram o sacerdócio, como também a aliança sacerdotal e levítica.
\par 30 Limpei-os, pois, de toda estrangeirice e designei o serviço dos sacerdotes e dos levitas, cada um no seu mister,
\par 31 como também o fornecimento de lenha em tempos determinados, bem como as primícias. Lembra-te de mim, Deus meu, para o meu bem.


\end{document}