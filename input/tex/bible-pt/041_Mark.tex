\begin{document}

\title{Marcos}


\chapter{1}

\par 1 Princípio do evangelho de Jesus Cristo, Filho de Deus.
\par 2 Conforme está escrito na profecia de Isaías: Eis aí envio diante da tua face o meu mensageiro, o qual preparará o teu caminho;
\par 3 voz do que clama no deserto: Preparai o caminho do Senhor, endireitai as suas veredas;
\par 4 apareceu João Batista no deserto, pregando batismo de arrependimento para remissão de pecados.
\par 5 Saíam a ter com ele toda a província da Judéia e todos os habitantes de Jerusalém; e, confessando os seus pecados, eram batizados por ele no rio Jordão.
\par 6 As vestes de João eram feitas de pêlos de camelo; ele trazia um cinto de couro e se alimentava de gafanhotos e mel silvestre.
\par 7 E pregava, dizendo: Após mim vem aquele que é mais poderoso do que eu, do qual não sou digno de, curvando-me, desatar-lhe as correias das sandálias.
\par 8 Eu vos tenho batizado com água; ele, porém, vos batizará com o Espírito Santo.
\par 9 Naqueles dias, veio Jesus de Nazaré da Galiléia e por João foi batizado no rio Jordão.
\par 10 Logo ao sair da água, viu os céus rasgarem-se e o Espírito descendo como pomba sobre ele.
\par 11 Então, foi ouvida uma voz dos céus: Tu és o meu Filho amado, em ti me comprazo.
\par 12 E logo o Espírito o impeliu para o deserto,
\par 13 onde permaneceu quarenta dias, sendo tentado por Satanás; estava com as feras, mas os anjos o serviam.
\par 14 Depois de João ter sido preso, foi Jesus para a Galiléia, pregando o evangelho de Deus,
\par 15 dizendo: O tempo está cumprido, e o reino de Deus está próximo; arrependei-vos e crede no evangelho.
\par 16 Caminhando junto ao mar da Galiléia, viu os irmãos Simão e André, que lançavam a rede ao mar, porque eram pescadores.
\par 17 Disse-lhes Jesus: Vinde após mim, e eu vos farei pescadores de homens.
\par 18 Então, eles deixaram imediatamente as redes e o seguiram.
\par 19 Pouco mais adiante, viu Tiago, filho de Zebedeu, e João, seu irmão, que estavam no barco consertando as redes.
\par 20 E logo os chamou. Deixando eles no barco a seu pai Zebedeu com os empregados, seguiram após Jesus.
\par 21 Depois, entraram em Cafarnaum, e, logo no sábado, foi ele ensinar na sinagoga.
\par 22 Maravilhavam-se da sua doutrina, porque os ensinava como quem tem autoridade e não como os escribas.
\par 23 Não tardou que aparecesse na sinagoga um homem possesso de espírito imundo, o qual bradou:
\par 24 Que temos nós contigo, Jesus Nazareno? Vieste para perder-nos? Bem sei quem és: o Santo de Deus!
\par 25 Mas Jesus o repreendeu, dizendo: Cala-te e sai desse homem.
\par 26 Então, o espírito imundo, agitando-o violentamente e bradando em alta voz, saiu dele.
\par 27 Todos se admiraram, a ponto de perguntarem entre si: Que vem a ser isto? Uma nova doutrina! Com autoridade ele ordena aos espíritos imundos, e eles lhe obedecem!
\par 28 Então, correu célere a fama de Jesus em todas as direções, por toda a circunvizinhança da Galiléia.
\par 29 E, saindo eles da sinagoga, foram, com Tiago e João, diretamente para a casa de Simão e André.
\par 30 A sogra de Simão achava-se acamada, com febre; e logo lhe falaram a respeito dela.
\par 31 Então, aproximando-se, tomou-a pela mão; e a febre a deixou, passando ela a servi-los.
\par 32 À tarde, ao cair do sol, trouxeram a Jesus todos os enfermos e endemoninhados.
\par 33 Toda a cidade estava reunida à porta.
\par 34 E ele curou muitos doentes de toda sorte de enfermidades; também expeliu muitos demônios, não lhes permitindo que falassem, porque sabiam quem ele era.
\par 35 Tendo-se levantado alta madrugada, saiu, foi para um lugar deserto e ali orava.
\par 36 Procuravam-no diligentemente Simão e os que com ele estavam.
\par 37 Tendo-o encontrado, lhe disseram: Todos te buscam.
\par 38 Jesus, porém, lhes disse: Vamos a outros lugares, às povoações vizinhas, a fim de que eu pregue também ali, pois para isso é que eu vim.
\par 39 Então, foi por toda a Galiléia, pregando nas sinagogas deles e expelindo os demônios.
\par 40 Aproximou-se dele um leproso rogando-lhe, de joelhos: Se quiseres, podes purificar-me.
\par 41 Jesus, profundamente compadecido, estendeu a mão, tocou-o e disse-lhe: Quero, fica limpo!
\par 42 No mesmo instante, lhe desapareceu a lepra, e ficou limpo.
\par 43 Fazendo-lhe, então, veemente advertência, logo o despediu
\par 44 e lhe disse: Olha, não digas nada a ninguém; mas vai, mostra-te ao sacerdote e oferece pela tua purificação o que Moisés determinou, para servir de testemunho ao povo.
\par 45 Mas, tendo ele saído, entrou a propalar muitas coisas e a divulgar a notícia, a ponto de não mais poder Jesus entrar publicamente em qualquer cidade, mas permanecia fora, em lugares ermos; e de toda parte vinham ter com ele.

\chapter{2}

\par 1 Dias depois, entrou Jesus de novo em Cafarnaum, e logo correu que ele estava em casa.
\par 2 Muitos afluíram para ali, tantos que nem mesmo junto à porta eles achavam lugar; e anunciava-lhes a palavra.
\par 3 Alguns foram ter com ele, conduzindo um paralítico, levado por quatro homens.
\par 4 E, não podendo aproximar-se dele, por causa da multidão, descobriram o eirado no ponto correspondente ao em que ele estava e, fazendo uma abertura, baixaram o leito em que jazia o doente.
\par 5 Vendo-lhes a fé, Jesus disse ao paralítico: Filho, os teus pecados estão perdoados.
\par 6 Mas alguns dos escribas estavam assentados ali e arrazoavam em seu coração:
\par 7 Por que fala ele deste modo? Isto é blasfêmia! Quem pode perdoar pecados, senão um, que é Deus?
\par 8 E Jesus, percebendo logo por seu espírito que eles assim arrazoavam, disse-lhes: Por que arrazoais sobre estas coisas em vosso coração?
\par 9 Qual é mais fácil? Dizer ao paralítico: Estão perdoados os teus pecados, ou dizer: Levanta-te, toma o teu leito e anda?
\par 10 Ora, para que saibais que o Filho do Homem tem sobre a terra autoridade para perdoar pecados -- disse ao paralítico:
\par 11 Eu te mando: Levanta-te, toma o teu leito e vai para tua casa.
\par 12 Então, ele se levantou e, no mesmo instante, tomando o leito, retirou-se à vista de todos, a ponto de se admirarem todos e darem glória a Deus, dizendo: Jamais vimos coisa assim!
\par 13 De novo, saiu Jesus para junto do mar, e toda a multidão vinha ao seu encontro, e ele os ensinava.
\par 14 Quando ia passando, viu a Levi, filho de Alfeu, sentado na coletoria e disse-lhe: Segue-me! Ele se levantou e o seguiu.
\par 15 Achando-se Jesus à mesa na casa de Levi, estavam juntamente com ele e com seus discípulos muitos publicanos e pecadores; porque estes eram em grande número e também o seguiam.
\par 16 Os escribas dos fariseus, vendo-o comer em companhia dos pecadores e publicanos, perguntavam aos discípulos dele: Por que come [e bebe] ele com os publicanos e pecadores?
\par 17 Tendo Jesus ouvido isto, respondeu-lhes: Os sãos não precisam de médico, e sim os doentes; não vim chamar justos, e sim pecadores.
\par 18 Ora, os discípulos de João e os fariseus estavam jejuando. Vieram alguns e lhe perguntaram: Por que motivo jejuam os discípulos de João e os dos fariseus, mas os teus discípulos não jejuam?
\par 19 Respondeu-lhes Jesus: Podem, porventura, jejuar os convidados para o casamento, enquanto o noivo está com eles? Durante o tempo em que estiver presente o noivo, não podem jejuar.
\par 20 Dias virão, contudo, em que lhes será tirado o noivo; e, nesse tempo, jejuarão.
\par 21 Ninguém costura remendo de pano novo em veste velha; porque o remendo novo tira parte da veste velha, e fica maior a rotura.
\par 22 Ninguém põe vinho novo em odres velhos; do contrário, o vinho romperá os odres; e tanto se perde o vinho como os odres. Mas põe-se vinho novo em odres novos.
\par 23 Ora, aconteceu atravessar Jesus, em dia de sábado, as searas, e os discípulos, ao passarem, colhiam espigas.
\par 24 Advertiram-no os fariseus: Vê! Por que fazem o que não é lícito aos sábados?
\par 25 Mas ele lhes respondeu: Nunca lestes o que fez Davi, quando se viu em necessidade e teve fome, ele e os seus companheiros?
\par 26 Como entrou na Casa de Deus, no tempo do sumo sacerdote Abiatar, e comeu os pães da proposição, os quais não é lícito comer, senão aos sacerdotes, e deu também aos que estavam com ele?
\par 27 E acrescentou: O sábado foi estabelecido por causa do homem, e não o homem por causa do sábado;
\par 28 de sorte que o Filho do Homem é senhor também do sábado.

\chapter{3}

\par 1 De novo, entrou Jesus na sinagoga e estava ali um homem que tinha ressequida uma das mãos.
\par 2 E estavam observando a Jesus para ver se o curaria em dia de sábado, a fim de o acusarem.
\par 3 E disse Jesus ao homem da mão ressequida: Vem para o meio!
\par 4 Então, lhes perguntou: É lícito nos sábados fazer o bem ou fazer o mal? Salvar a vida ou tirá-la? Mas eles ficaram em silêncio.
\par 5 Olhando-os ao redor, indignado e condoído com a dureza do seu coração, disse ao homem: Estende a mão. Estendeu-a, e a mão lhe foi restaurada.
\par 6 Retirando-se os fariseus, conspiravam logo com os herodianos, contra ele, em como lhe tirariam a vida.
\par 7 Retirou-se Jesus com os seus discípulos para os lados do mar. Seguia-o da Galiléia uma grande multidão. Também da Judéia,
\par 8 de Jerusalém, da Iduméia, dalém do Jordão e dos arredores de Tiro e de Sidom uma grande multidão, sabendo quantas coisas Jesus fazia, veio ter com ele.
\par 9 Então, recomendou a seus discípulos que sempre lhe tivessem pronto um barquinho, por causa da multidão, a fim de não o comprimirem.
\par 10 Pois curava a muitos, de modo que todos os que padeciam de qualquer enfermidade se arrojavam a ele para o tocar.
\par 11 Também os espíritos imundos, quando o viam, prostravam-se diante dele e exclamavam: Tu és o Filho de Deus!
\par 12 Mas Jesus lhes advertia severamente que o não expusessem à publicidade.
\par 13 Depois, subiu ao monte e chamou os que ele mesmo quis, e vieram para junto dele.
\par 14 Então, designou doze para estarem com ele e para os enviar a pregar
\par 15 e a exercer a autoridade de expelir demônios.
\par 16 Eis os doze que designou: Simão, a quem acrescentou o nome de Pedro;
\par 17 Tiago, filho de Zebedeu, e João, seu irmão, aos quais deu o nome de Boanerges, que quer dizer: filhos do trovão;
\par 18 André, Filipe, Bartolomeu, Mateus, Tomé, Tiago, filho de Alfeu, Tadeu, Simão, o Zelote,
\par 19 e Judas Iscariotes, que foi quem o traiu.
\par 20 Então, ele foi para casa. Não obstante, a multidão afluiu de novo, de tal modo que nem podiam comer.
\par 21 E, quando os parentes de Jesus ouviram isto, saíram para o prender; porque diziam: Está fora de si.
\par 22 Os escribas, que haviam descido de Jerusalém, diziam: Ele está possesso de Belzebu. E: É pelo maioral dos demônios que expele os demônios.
\par 23 Então, convocando-os Jesus, lhes disse, por meio de parábolas: Como pode Satanás expelir a Satanás?
\par 24 Se um reino estiver dividido contra si mesmo, tal reino não pode subsistir;
\par 25 se uma casa estiver dividida contra si mesma, tal casa não poderá subsistir.
\par 26 Se, pois, Satanás se levantou contra si mesmo e está dividido, não pode subsistir, mas perece.
\par 27 Ninguém pode entrar na casa do valente para roubar-lhe os bens, sem primeiro amarrá-lo; e só então lhe saqueará a casa.
\par 28 Em verdade vos digo que tudo será perdoado aos filhos dos homens: os pecados e as blasfêmias que proferirem.
\par 29 Mas aquele que blasfemar contra o Espírito Santo não tem perdão para sempre, visto que é réu de pecado eterno.
\par 30 Isto, porque diziam: Está possesso de um espírito imundo.
\par 31 Nisto, chegaram sua mãe e seus irmãos e, tendo ficado do lado de fora, mandaram chamá-lo.
\par 32 Muita gente estava assentada ao redor dele e lhe disseram: Olha, tua mãe, teus irmãos e irmãs estão lá fora à tua procura.
\par 33 Então, ele lhes respondeu, dizendo: Quem é minha mãe e meus irmãos?
\par 34 E, correndo o olhar pelos que estavam assentados ao redor, disse: Eis minha mãe e meus irmãos.
\par 35 Portanto, qualquer que fizer a vontade de Deus, esse é meu irmão, irmã e mãe.

\chapter{4}

\par 1 Voltou Jesus a ensinar à beira-mar. E reuniu-se numerosa multidão a ele, de modo que entrou num barco, onde se assentou, afastando-se da praia. E todo o povo estava à beira-mar, na praia.
\par 2 Assim, lhes ensinava muitas coisas por parábolas, no decorrer do seu doutrinamento.
\par 3 Ouvi: Eis que saiu o semeador a semear.
\par 4 E, ao semear, uma parte caiu à beira do caminho, e vieram as aves e a comeram.
\par 5 Outra caiu em solo rochoso, onde a terra era pouca, e logo nasceu, visto não ser profunda a terra.
\par 6 Saindo, porém, o sol, a queimou; e, porque não tinha raiz, secou-se.
\par 7 Outra parte caiu entre os espinhos; e os espinhos cresceram e a sufocaram, e não deu fruto.
\par 8 Outra, enfim, caiu em boa terra e deu fruto, que vingou e cresceu, produzindo a trinta, a sessenta e a cem por um.
\par 9 E acrescentou: Quem tem ouvidos para ouvir, ouça.
\par 10 Quando Jesus ficou só, os que estavam junto dele com os doze o interrogaram a respeito das parábolas.
\par 11 Ele lhes respondeu: A vós outros vos é dado conhecer o mistério do reino de Deus; mas, aos de fora, tudo se ensina por meio de parábolas,
\par 12 para que, vendo, vejam e não percebam; e, ouvindo, ouçam e não entendam; para que não venham a converter-se, e haja perdão para eles.
\par 13 Então, lhes perguntou: Não entendeis esta parábola e como compreendereis todas as parábolas?
\par 14 O semeador semeia a palavra.
\par 15 São estes os da beira do caminho, onde a palavra é semeada; e, enquanto a ouvem, logo vem Satanás e tira a palavra semeada neles.
\par 16 Semelhantemente, são estes os semeados em solo rochoso, os quais, ouvindo a palavra, logo a recebem com alegria.
\par 17 Mas eles não têm raiz em si mesmos, sendo, antes, de pouca duração; em lhes chegando a angústia ou a perseguição por causa da palavra, logo se escandalizam.
\par 18 Os outros, os semeados entre os espinhos, são os que ouvem a palavra,
\par 19 mas os cuidados do mundo, a fascinação da riqueza e as demais ambições, concorrendo, sufocam a palavra, ficando ela infrutífera.
\par 20 Os que foram semeados em boa terra são aqueles que ouvem a palavra e a recebem, frutificando a trinta, a sessenta e a cem por um.
\par 21 Também lhes disse: Vem, porventura, a candeia para ser posta debaixo do alqueire ou da cama? Não vem, antes, para ser colocada no velador?
\par 22 Pois nada está oculto, senão para ser manifesto; e nada se faz escondido, senão para ser revelado.
\par 23 Se alguém tem ouvidos para ouvir, ouça.
\par 24 Então, lhes disse: Atentai no que ouvis. Com a medida com que tiverdes medido vos medirão também, e ainda se vos acrescentará.
\par 25 Pois ao que tem se lhe dará; e, ao que não tem, até o que tem lhe será tirado.
\par 26 Disse ainda: O reino de Deus é assim como se um homem lançasse a semente à terra;
\par 27 depois, dormisse e se levantasse, de noite e de dia, e a semente germinasse e crescesse, não sabendo ele como.
\par 28 A terra por si mesma frutifica: primeiro a erva, depois, a espiga, e, por fim, o grão cheio na espiga.
\par 29 E, quando o fruto já está maduro, logo se lhe mete a foice, porque é chegada a ceifa.
\par 30 Disse mais: A que assemelharemos o reino de Deus? Ou com que parábola o apresentaremos?
\par 31 É como um grão de mostarda, que, quando semeado, é a menor de todas as sementes sobre a terra;
\par 32 mas, uma vez semeada, cresce e se torna maior do que todas as hortaliças e deita grandes ramos, a ponto de as aves do céu poderem aninhar-se à sua sombra.
\par 33 E com muitas parábolas semelhantes lhes expunha a palavra, conforme o permitia a capacidade dos ouvintes.
\par 34 E sem parábolas não lhes falava; tudo, porém, explicava em particular aos seus próprios discípulos.
\par 35 Naquele dia, sendo já tarde, disse-lhes Jesus: Passemos para a outra margem.
\par 36 E eles, despedindo a multidão, o levaram assim como estava, no barco; e outros barcos o seguiam.
\par 37 Ora, levantou-se grande temporal de vento, e as ondas se arremessavam contra o barco, de modo que o mesmo já estava a encher-se de água.
\par 38 E Jesus estava na popa, dormindo sobre o travesseiro; eles o despertaram e lhe disseram: Mestre, não te importa que pereçamos?
\par 39 E ele, despertando, repreendeu o vento e disse ao mar: Acalma-te, emudece! O vento se aquietou, e fez-se grande bonança.
\par 40 Então, lhes disse: Por que sois assim tímidos?! Como é que não tendes fé?
\par 41 E eles, possuídos de grande temor, diziam uns aos outros: Quem é este que até o vento e o mar lhe obedecem?

\chapter{5}

\par 1 Entrementes, chegaram à outra margem do mar, à terra dos gerasenos.
\par 2 Ao desembarcar, logo veio dos sepulcros, ao seu encontro, um homem possesso de espírito imundo,
\par 3 o qual vivia nos sepulcros, e nem mesmo com cadeias alguém podia prendê-lo;
\par 4 porque, tendo sido muitas vezes preso com grilhões e cadeias, as cadeias foram quebradas por ele, e os grilhões, despedaçados. E ninguém podia subjugá-lo.
\par 5 Andava sempre, de noite e de dia, clamando por entre os sepulcros e pelos montes, ferindo-se com pedras.
\par 6 Quando, de longe, viu Jesus, correu e o adorou,
\par 7 exclamando com alta voz: Que tenho eu contigo, Jesus, Filho do Deus Altíssimo? Conjuro-te por Deus que não me atormentes!
\par 8 Porque Jesus lhe dissera: Espírito imundo, sai desse homem!
\par 9 E perguntou-lhe: Qual é o teu nome? Respondeu ele: Legião é o meu nome, porque somos muitos.
\par 10 E rogou-lhe encarecidamente que os não mandasse para fora do país.
\par 11 Ora, pastava ali pelo monte uma grande manada de porcos.
\par 12 E os espíritos imundos rogaram a Jesus, dizendo: Manda-nos para os porcos, para que entremos neles.
\par 13 Jesus o permitiu. Então, saindo os espíritos imundos, entraram nos porcos; e a manada, que era cerca de dois mil, precipitou-se despenhadeiro abaixo, para dentro do mar, onde se afogaram.
\par 14 Os porqueiros fugiram e o anunciaram na cidade e pelos campos. Então, saiu o povo para ver o que sucedera.
\par 15 Indo ter com Jesus, viram o endemoninhado, o que tivera a legião, assentado, vestido, em perfeito juízo; e temeram.
\par 16 Os que haviam presenciado os fatos contaram-lhes o que acontecera ao endemoninhado e acerca dos porcos.
\par 17 E entraram a rogar-lhe que se retirasse da terra deles.
\par 18 Ao entrar Jesus no barco, suplicava-lhe o que fora endemoninhado que o deixasse estar com ele.
\par 19 Jesus, porém, não lho permitiu, mas ordenou-lhe: Vai para tua casa, para os teus. Anuncia-lhes tudo o que o Senhor te fez e como teve compaixão de ti.
\par 20 Então, ele foi e começou a proclamar em Decápolis tudo o que Jesus lhe fizera; e todos se admiravam.
\par 21 Tendo Jesus voltado no barco, para o outro lado, afluiu para ele grande multidão; e ele estava junto do mar.
\par 22 Eis que se chegou a ele um dos principais da sinagoga, chamado Jairo, e, vendo-o, prostrou-se a seus pés
\par 23 e insistentemente lhe suplicou: Minha filhinha está à morte; vem, impõe as mãos sobre ela, para que seja salva, e viverá.
\par 24 Jesus foi com ele. Grande multidão o seguia, comprimindo-o.
\par 25 Aconteceu que certa mulher, que, havia doze anos, vinha sofrendo de uma hemorragia
\par 26 e muito padecera à mão de vários médicos, tendo despendido tudo quanto possuía, sem, contudo, nada aproveitar, antes, pelo contrário, indo a pior,
\par 27 tendo ouvido a fama de Jesus, vindo por trás dele, por entre a multidão, tocou-lhe a veste.
\par 28 Porque, dizia: Se eu apenas lhe tocar as vestes, ficarei curada.
\par 29 E logo se lhe estancou a hemorragia, e sentiu no corpo estar curada do seu flagelo.
\par 30 Jesus, reconhecendo imediatamente que dele saíra poder, virando-se no meio da multidão, perguntou: Quem me tocou nas vestes?
\par 31 Responderam-lhe seus discípulos: Vês que a multidão te aperta e dizes: Quem me tocou?
\par 32 Ele, porém, olhava ao redor para ver quem fizera isto.
\par 33 Então, a mulher, atemorizada e tremendo, cônscia do que nela se operara, veio, prostrou-se diante dele e declarou-lhe toda a verdade.
\par 34 E ele lhe disse: Filha, a tua fé te salvou; vai-te em paz e fica livre do teu mal.
\par 35 Falava ele ainda, quando chegaram alguns da casa do chefe da sinagoga, a quem disseram: Tua filha já morreu; por que ainda incomodas o Mestre?
\par 36 Mas Jesus, sem acudir a tais palavras, disse ao chefe da sinagoga: Não temas, crê somente.
\par 37 Contudo, não permitiu que alguém o acompanhasse, senão Pedro e os irmãos Tiago e João.
\par 38 Chegando à casa do chefe da sinagoga, viu Jesus o alvoroço, os que choravam e os que pranteavam muito.
\par 39 Ao entrar, lhes disse: Por que estais em alvoroço e chorais? A criança não está morta, mas dorme.
\par 40 E riam-se dele. Tendo ele, porém, mandado sair a todos, tomou o pai e a mãe da criança e os que vieram com ele e entrou onde ela estava.
\par 41 Tomando-a pela mão, disse: Talitá cumi!, que quer dizer: Menina, eu te mando, levanta-te!
\par 42 Imediatamente, a menina se levantou e pôs-se a andar; pois tinha doze anos. Então, ficaram todos sobremaneira admirados.
\par 43 Mas Jesus ordenou-lhes expressamente que ninguém o soubesse; e mandou que dessem de comer à menina.

\chapter{6}

\par 1 Tendo Jesus partido dali, foi para a sua terra, e os seus discípulos o acompanharam.
\par 2 Chegando o sábado, passou a ensinar na sinagoga; e muitos, ouvindo-o, se maravilhavam, dizendo: Donde vêm a este estas coisas? Que sabedoria é esta que lhe foi dada? E como se fazem tais maravilhas por suas mãos?
\par 3 Não é este o carpinteiro, filho de Maria, irmão de Tiago, José, Judas e Simão? E não vivem aqui entre nós suas irmãs? E escandalizavam-se nele.
\par 4 Jesus, porém, lhes disse: Não há profeta sem honra, senão na sua terra, entre os seus parentes e na sua casa.
\par 5 Não pôde fazer ali nenhum milagre, senão curar uns poucos enfermos, impondo-lhes as mãos.
\par 6 Admirou-se da incredulidade deles. Contudo, percorria as aldeias circunvizinhas, a ensinar.
\par 7 Chamou Jesus os doze e passou a enviá-los de dois a dois, dando-lhes autoridade sobre os espíritos imundos.
\par 8 Ordenou-lhes que nada levassem para o caminho, exceto um bordão; nem pão, nem alforje, nem dinheiro;
\par 9 que fossem calçados de sandálias e não usassem duas túnicas.
\par 10 E recomendou-lhes: Quando entrardes nalguma casa, permanecei aí até vos retirardes do lugar.
\par 11 Se nalgum lugar não vos receberem nem vos ouvirem, ao sairdes dali, sacudi o pó dos pés, em testemunho contra eles.
\par 12 Então, saindo eles, pregavam ao povo que se arrependesse;
\par 13 expeliam muitos demônios e curavam numerosos enfermos, ungindo-os com óleo.
\par 14 Chegou isto aos ouvidos do rei Herodes, porque o nome de Jesus já se tornara notório; e alguns diziam: João Batista ressuscitou dentre os mortos, e, por isso, nele operam forças miraculosas.
\par 15 Outros diziam: É Elias; ainda outros: É profeta como um dos profetas.
\par 16 Herodes, porém, ouvindo isto, disse: É João, a quem eu mandei decapitar, que ressurgiu.
\par 17 Porque o mesmo Herodes, por causa de Herodias, mulher de seu irmão Filipe (porquanto Herodes se casara com ela), mandara prender a João e atá-lo no cárcere.
\par 18 Pois João lhe dizia: Não te é lícito possuir a mulher de teu irmão.
\par 19 E Herodias o odiava, querendo matá-lo, e não podia.
\par 20 Porque Herodes temia a João, sabendo que era homem justo e santo, e o tinha em segurança. E, quando o ouvia, ficava perplexo, escutando-o de boa mente.
\par 21 E, chegando um dia favorável, em que Herodes no seu aniversário natalício dera um banquete aos seus dignitários, aos oficiais militares e aos principais da Galiléia,
\par 22 entrou a filha de Herodias e, dançando, agradou a Herodes e aos seus convivas. Então, disse o rei à jovem: Pede-me o que quiseres, e eu to darei.
\par 23 E jurou-lhe: Se pedires mesmo que seja a metade do meu reino, eu ta darei.
\par 24 Saindo ela, perguntou a sua mãe: Que pedirei? Esta respondeu: A cabeça de João Batista.
\par 25 No mesmo instante, voltando apressadamente para junto do rei, disse: Quero que, sem demora, me dês num prato a cabeça de João Batista.
\par 26 Entristeceu-se profundamente o rei; mas, por causa do juramento e dos que estavam com ele à mesa, não lha quis negar.
\par 27 E, enviando logo o executor, mandou que lhe trouxessem a cabeça de João. Ele foi, e o decapitou no cárcere,
\par 28 e, trazendo a cabeça num prato, a entregou à jovem, e esta, por sua vez, a sua mãe.
\par 29 Os discípulos de João, logo que souberam disto, vieram, levaram-lhe o corpo e o depositaram no túmulo.
\par 30 Voltaram os apóstolos à presença de Jesus e lhe relataram tudo quanto haviam feito e ensinado.
\par 31 E ele lhes disse: Vinde repousar um pouco, à parte, num lugar deserto; porque eles não tinham tempo nem para comer, visto serem numerosos os que iam e vinham.
\par 32 Então, foram sós no barco para um lugar solitário.
\par 33 Muitos, porém, os viram partir e, reconhecendo-os, correram para lá, a pé, de todas as cidades, e chegaram antes deles.
\par 34 Ao desembarcar, viu Jesus uma grande multidão e compadeceu-se deles, porque eram como ovelhas que não têm pastor. E passou a ensinar-lhes muitas coisas.
\par 35 Em declinando a tarde, vieram os discípulos a Jesus e lhe disseram: É deserto este lugar, e já avançada a hora;
\par 36 despede-os para que, passando pelos campos ao redor e pelas aldeias, comprem para si o que comer.
\par 37 Porém ele lhes respondeu: Dai-lhes vós mesmos de comer. Disseram-lhe: Iremos comprar duzentos denários de pão para lhes dar de comer?
\par 38 E ele lhes disse: Quantos pães tendes? Ide ver! E, sabendo-o eles, responderam: Cinco pães e dois peixes.
\par 39 Então, Jesus lhes ordenou que todos se assentassem, em grupos, sobre a relva verde.
\par 40 E o fizeram, repartindo-se em grupos de cem em cem e de cinqüenta em cinqüenta.
\par 41 Tomando ele os cinco pães e os dois peixes, erguendo os olhos ao céu, os abençoou; e, partindo os pães, deu-os aos discípulos para que os distribuíssem; e por todos repartiu também os dois peixes.
\par 42 Todos comeram e se fartaram;
\par 43 e ainda recolheram doze cestos cheios de pedaços de pão e de peixe.
\par 44 Os que comeram dos pães eram cinco mil homens.
\par 45 Logo a seguir, compeliu Jesus os seus discípulos a embarcar e passar adiante para o outro lado, a Betsaida, enquanto ele despedia a multidão.
\par 46 E, tendo-os despedido, subiu ao monte para orar.
\par 47 Ao cair da tarde, estava o barco no meio do mar, e ele, sozinho em terra.
\par 48 E, vendo-os em dificuldade a remar, porque o vento lhes era contrário, por volta da quarta vigília da noite, veio ter com eles, andando por sobre o mar; e queria tomar-lhes a dianteira.
\par 49 Eles, porém, vendo-o andar sobre o mar, pensaram tratar-se de um fantasma e gritaram.
\par 50 Pois todos ficaram aterrados à vista dele. Mas logo lhes falou e disse: Tende bom ânimo! Sou eu. Não temais!
\par 51 E subiu para o barco para estar com eles, e o vento cessou. Ficaram entre si atônitos,
\par 52 porque não haviam compreendido o milagre dos pães; antes, o seu coração estava endurecido.
\par 53 Estando já no outro lado, chegaram a terra, em Genesaré, onde aportaram.
\par 54 Saindo eles do barco, logo o povo reconheceu Jesus;
\par 55 e, percorrendo toda aquela região, traziam em leitos os enfermos, para onde ouviam que ele estava.
\par 56 Onde quer que ele entrasse nas aldeias, cidades ou campos, punham os enfermos nas praças, rogando-lhe que os deixasse tocar ao menos na orla da sua veste; e quantos a tocavam saíam curados.

\chapter{7}

\par 1 Ora, reuniram-se a Jesus os fariseus e alguns escribas, vindos de Jerusalém.
\par 2 E, vendo que alguns dos discípulos dele comiam pão com as mãos impuras, isto é, por lavar
\par 3 (pois os fariseus e todos os judeus, observando a tradição dos anciãos, não comem sem lavar cuidadosamente as mãos;
\par 4 quando voltam da praça, não comem sem se aspergirem; e há muitas outras coisas que receberam para observar, como a lavagem de copos, jarros e vasos de metal [e camas]),
\par 5 interpelaram-no os fariseus e os escribas: Por que não andam os teus discípulos de conformidade com a tradição dos anciãos, mas comem com as mãos por lavar?
\par 6 Respondeu-lhes: Bem profetizou Isaías a respeito de vós, hipócritas, como está escrito: Este povo honra-me com os lábios, mas o seu coração está longe de mim.
\par 7 E em vão me adoram, ensinando doutrinas que são preceitos de homens.
\par 8 Negligenciando o mandamento de Deus, guardais a tradição dos homens.
\par 9 E disse-lhes ainda: Jeitosamente rejeitais o preceito de Deus para guardardes a vossa própria tradição.
\par 10 Pois Moisés disse: Honra a teu pai e a tua mãe; e: Quem maldisser a seu pai ou a sua mãe seja punido de morte.
\par 11 Vós, porém, dizeis: Se um homem disser a seu pai ou a sua mãe: Aquilo que poderias aproveitar de mim é Corbã, isto é, oferta para o Senhor,
\par 12 então, o dispensais de fazer qualquer coisa em favor de seu pai ou de sua mãe,
\par 13 invalidando a palavra de Deus pela vossa própria tradição, que vós mesmos transmitistes; e fazeis muitas outras coisas semelhantes.
\par 14 Convocando ele, de novo, a multidão, disse-lhes: Ouvi-me, todos, e entendei.
\par 15 Nada há fora do homem que, entrando nele, o possa contaminar; mas o que sai do homem é o que o contamina.
\par 16 [Se alguém tem ouvidos para ouvir, ouça.]
\par 17 Quando entrou em casa, deixando a multidão, os seus discípulos o interrogaram acerca da parábola.
\par 18 Então, lhes disse: Assim vós também não entendeis? Não compreendeis que tudo o que de fora entra no homem não o pode contaminar,
\par 19 porque não lhe entra no coração, mas no ventre, e sai para lugar escuso? E, assim, considerou ele puros todos os alimentos.
\par 20 E dizia: O que sai do homem, isso é o que o contamina.
\par 21 Porque de dentro, do coração dos homens, é que procedem os maus desígnios, a prostituição, os furtos, os homicídios, os adultérios,
\par 22 a avareza, as malícias, o dolo, a lascívia, a inveja, a blasfêmia, a soberba, a loucura.
\par 23 Ora, todos estes males vêm de dentro e contaminam o homem.
\par 24 Levantando-se, partiu dali para as terras de Tiro [e Sidom]. Tendo entrado numa casa, queria que ninguém o soubesse; no entanto, não pôde ocultar-se,
\par 25 porque uma mulher, cuja filhinha estava possessa de espírito imundo, tendo ouvido a respeito dele, veio e prostrou-se-lhe aos pés.
\par 26 Esta mulher era grega, de origem siro-fenícia, e rogava-lhe que expelisse de sua filha o demônio.
\par 27 Mas Jesus lhe disse: Deixa primeiro que se fartem os filhos, porque não é bom tomar o pão dos filhos e lançá-lo aos cachorrinhos.
\par 28 Ela, porém, lhe respondeu: Sim, Senhor; mas os cachorrinhos, debaixo da mesa, comem das migalhas das crianças.
\par 29 Então, lhe disse: Por causa desta palavra, podes ir; o demônio já saiu de tua filha.
\par 30 Voltando ela para casa, achou a menina sobre a cama, pois o demônio a deixara.
\par 31 De novo, se retirou das terras de Tiro e foi por Sidom até ao mar da Galiléia, através do território de Decápolis.
\par 32 Então, lhe trouxeram um surdo e gago e lhe suplicaram que impusesse as mãos sobre ele.
\par 33 Jesus, tirando-o da multidão, à parte, pôs-lhe os dedos nos ouvidos e lhe tocou a língua com saliva;
\par 34 depois, erguendo os olhos ao céu, suspirou e disse: Efatá!, que quer dizer: Abre-te!
\par 35 Abriram-se-lhe os ouvidos, e logo se lhe soltou o empecilho da língua, e falava desembaraçadamente.
\par 36 Mas lhes ordenou que a ninguém o dissessem; contudo, quanto mais recomendava, tanto mais eles o divulgavam.
\par 37 Maravilhavam-se sobremaneira, dizendo: Tudo ele tem feito esplendidamente bem; não somente faz ouvir os surdos, como falar os mudos.

\chapter{8}

\par 1 Naqueles dias, quando outra vez se reuniu grande multidão, e não tendo eles o que comer, chamou Jesus os discípulos e lhes disse:
\par 2 Tenho compaixão desta gente, porque há três dias que permanecem comigo e não têm o que comer.
\par 3 Se eu os despedir para suas casas, em jejum, desfalecerão pelo caminho; e alguns deles vieram de longe.
\par 4 Mas os seus discípulos lhe responderam: Donde poderá alguém fartá-los de pão neste deserto?
\par 5 E Jesus lhes perguntou: Quantos pães tendes? Responderam eles: Sete.
\par 6 Ordenou ao povo que se assentasse no chão. E, tomando os sete pães, partiu-os, após ter dado graças, e os deu a seus discípulos, para que estes os distribuíssem, repartindo entre o povo.
\par 7 Tinham também alguns peixinhos; e, abençoando-os, mandou que estes igualmente fossem distribuídos.
\par 8 Comeram e se fartaram; e dos pedaços restantes recolheram sete cestos.
\par 9 Eram cerca de quatro mil homens. Então, Jesus os despediu.
\par 10 Logo a seguir, tendo embarcado juntamente com seus discípulos, partiu para as regiões de Dalmanuta.
\par 11 E, saindo os fariseus, puseram-se a discutir com ele; e, tentando-o, pediram-lhe um sinal do céu.
\par 12 Jesus, porém, arrancou do íntimo do seu espírito um gemido e disse: Por que pede esta geração um sinal? Em verdade vos digo que a esta geração não se lhe dará sinal algum.
\par 13 E, deixando-os, tornou a embarcar e foi para o outro lado.
\par 14 Ora, aconteceu que eles se esqueceram de levar pães e, no barco, não tinham consigo senão um só.
\par 15 Preveniu-os Jesus, dizendo: Vede, guardai-vos do fermento dos fariseus e do fermento de Herodes.
\par 16 E eles discorriam entre si: É que não temos pão.
\par 17 Jesus, percebendo-o, lhes perguntou: Por que discorreis sobre o não terdes pão? Ainda não considerastes, nem compreendestes? Tendes o coração endurecido?
\par 18 Tendo olhos, não vedes? E, tendo ouvidos, não ouvis? Não vos lembrais
\par 19 de quando parti os cinco pães para os cinco mil, quantos cestos cheios de pedaços recolhestes? Responderam eles: Doze!
\par 20 E de quando parti os sete pães para os quatro mil, quantos cestos cheios de pedaços recolhestes? Responderam: Sete!
\par 21 Ao que lhes disse Jesus: Não compreendeis ainda?
\par 22 Então, chegaram a Betsaida; e lhe trouxeram um cego, rogando-lhe que o tocasse.
\par 23 Jesus, tomando o cego pela mão, levou-o para fora da aldeia e, aplicando-lhe saliva aos olhos e impondo-lhe as mãos, perguntou-lhe: Vês alguma coisa?
\par 24 Este, recobrando a vista, respondeu: Vejo os homens, porque como árvores os vejo, andando.
\par 25 Então, novamente lhe pôs as mãos nos olhos, e ele, passando a ver claramente, ficou restabelecido; e tudo distinguia de modo perfeito.
\par 26 E mandou-o Jesus embora para casa, recomendando-lhe: Não entres na aldeia.
\par 27 Então, Jesus e os seus discípulos partiram para as aldeias de Cesaréia de Filipe; e, no caminho, perguntou-lhes: Quem dizem os homens que sou eu?
\par 28 E responderam: João Batista; outros: Elias; mas outros: Algum dos profetas.
\par 29 Então, lhes perguntou: Mas vós, quem dizeis que eu sou? Respondendo, Pedro lhe disse: Tu és o Cristo.
\par 30 Advertiu-os Jesus de que a ninguém dissessem tal coisa a seu respeito.
\par 31 Então, começou ele a ensinar-lhes que era necessário que o Filho do Homem sofresse muitas coisas, fosse rejeitado pelos anciãos, pelos principais sacerdotes e pelos escribas, fosse morto e que, depois de três dias, ressuscitasse.
\par 32 E isto ele expunha claramente. Mas Pedro, chamando-o à parte, começou a reprová-lo.
\par 33 Jesus, porém, voltou-se e, fitando os seus discípulos, repreendeu a Pedro e disse: Arreda, Satanás! Porque não cogitas das coisas de Deus, e sim das dos homens.
\par 34 Então, convocando a multidão e juntamente os seus discípulos, disse-lhes: Se alguém quer vir após mim, a si mesmo se negue, tome a sua cruz e siga-me.
\par 35 Quem quiser, pois, salvar a sua vida perdê-la-á; e quem perder a vida por causa de mim e do evangelho salvá-la-á.
\par 36 Que aproveita ao homem ganhar o mundo inteiro e perder a sua alma?
\par 37 Que daria um homem em troca de sua alma?
\par 38 Porque qualquer que, nesta geração adúltera e pecadora, se envergonhar de mim e das minhas palavras, também o Filho do Homem se envergonhará dele, quando vier na glória de seu Pai com os santos anjos.

\chapter{9}

\par 1 Dizia-lhes ainda: Em verdade vos afirmo que, dos que aqui se encontram, alguns há que, de maneira nenhuma, passarão pela morte até que vejam ter chegado com poder o reino de Deus.
\par 2 Seis dias depois, tomou Jesus consigo a Pedro, Tiago e João e levou-os sós, à parte, a um alto monte. Foi transfigurado diante deles;
\par 3 as suas vestes tornaram-se resplandecentes e sobremodo brancas, como nenhum lavandeiro na terra as poderia alvejar.
\par 4 Apareceu-lhes Elias com Moisés, e estavam falando com Jesus.
\par 5 Então, Pedro, tomando a palavra, disse: Mestre, bom é estarmos aqui e que façamos três tendas: uma será tua, outra, para Moisés, e outra, para Elias.
\par 6 Pois não sabia o que dizer, por estarem eles aterrados.
\par 7 A seguir, veio uma nuvem que os envolveu; e dela uma voz dizia: Este é o meu Filho amado; a ele ouvi.
\par 8 E, de relance, olhando ao redor, a ninguém mais viram com eles, senão Jesus.
\par 9 Ao descerem do monte, ordenou-lhes Jesus que não divulgassem as coisas que tinham visto, até o dia em que o Filho do Homem ressuscitasse dentre os mortos.
\par 10 Eles guardaram a recomendação, perguntando uns aos outros que seria o ressuscitar dentre os mortos.
\par 11 E interrogaram-no, dizendo: Por que dizem os escribas ser necessário que Elias venha primeiro?
\par 12 Então, ele lhes disse: Elias, vindo primeiro, restaurará todas as coisas; como, pois, está escrito sobre o Filho do Homem que sofrerá muito e será aviltado?
\par 13 Eu, porém, vos digo que Elias já veio, e fizeram com ele tudo o que quiseram, como a seu respeito está escrito.
\par 14 Quando eles se aproximaram dos discípulos, viram numerosa multidão ao redor e que os escribas discutiam com eles.
\par 15 E logo toda a multidão, ao ver Jesus, tomada de surpresa, correu para ele e o saudava.
\par 16 Então, ele interpelou os escribas: Que é que discutíeis com eles?
\par 17 E um, dentre a multidão, respondeu: Mestre, trouxe-te o meu filho, possesso de um espírito mudo;
\par 18 e este, onde quer que o apanha, lança-o por terra, e ele espuma, rilha os dentes e vai definhando. Roguei a teus discípulos que o expelissem, e eles não puderam.
\par 19 Então, Jesus lhes disse: Ó geração incrédula, até quando estarei convosco? Até quando vos sofrerei? Trazei-mo.
\par 20 E trouxeram-lho; quando ele viu a Jesus, o espírito imediatamente o agitou com violência, e, caindo ele por terra, revolvia-se espumando.
\par 21 Perguntou Jesus ao pai do menino: Há quanto tempo isto lhe sucede? Desde a infância, respondeu;
\par 22 e muitas vezes o tem lançado no fogo e na água, para o matar; mas, se tu podes alguma coisa, tem compaixão de nós e ajuda-nos.
\par 23 Ao que lhe respondeu Jesus: Se podes! Tudo é possível ao que crê.
\par 24 E imediatamente o pai do menino exclamou [com lágrimas]: Eu creio! Ajuda-me na minha falta de fé!
\par 25 Vendo Jesus que a multidão concorria, repreendeu o espírito imundo, dizendo-lhe: Espírito mudo e surdo, eu te ordeno: Sai deste jovem e nunca mais tornes a ele.
\par 26 E ele, clamando e agitando-o muito, saiu, deixando-o como se estivesse morto, a ponto de muitos dizerem: Morreu.
\par 27 Mas Jesus, tomando-o pela mão, o ergueu, e ele se levantou.
\par 28 Quando entrou em casa, os seus discípulos lhe perguntaram em particular: Por que não pudemos nós expulsá-lo?
\par 29 Respondeu-lhes: Esta casta não pode sair senão por meio de oração [e jejum].
\par 30 E, tendo partido dali, passavam pela Galiléia, e não queria que ninguém o soubesse;
\par 31 porque ensinava os seus discípulos e lhes dizia: O Filho do Homem será entregue nas mãos dos homens, e o matarão; mas, três dias depois da sua morte, ressuscitará.
\par 32 Eles, contudo, não compreendiam isto e temiam interrogá-lo.
\par 33 Tendo eles partido para Cafarnaum, estando ele em casa, interrogou os discípulos: De que é que discorríeis pelo caminho?
\par 34 Mas eles guardaram silêncio; porque, pelo caminho, haviam discutido entre si sobre quem era o maior.
\par 35 E ele, assentando-se, chamou os doze e lhes disse: Se alguém quer ser o primeiro, será o último e servo de todos.
\par 36 Trazendo uma criança, colocou-a no meio deles e, tomando-a nos braços, disse-lhes:
\par 37 Qualquer que receber uma criança, tal como esta, em meu nome, a mim me recebe; e qualquer que a mim me receber, não recebe a mim, mas ao que me enviou.
\par 38 Disse-lhe João: Mestre, vimos um homem que, em teu nome, expelia demônios, o qual não nos segue; e nós lho proibimos, porque não seguia conosco.
\par 39 Mas Jesus respondeu: Não lho proibais; porque ninguém há que faça milagre em meu nome e, logo a seguir, possa falar mal de mim.
\par 40 Pois quem não é contra nós é por nós.
\par 41 Porquanto, aquele que vos der de beber um copo de água, em meu nome, porque sois de Cristo, em verdade vos digo que de modo algum perderá o seu galardão.
\par 42 E quem fizer tropeçar a um destes pequeninos crentes, melhor lhe fora que se lhe pendurasse ao pescoço uma grande pedra de moinho, e fosse lançado no mar.
\par 43 E, se tua mão te faz tropeçar, corta-a; pois é melhor entrares maneta na vida do que, tendo as duas mãos, ires para o inferno, para o fogo inextinguível
\par 44 [onde não lhes morre o verme, nem o fogo se apaga].
\par 45 E, se teu pé te faz tropeçar, corta-o; é melhor entrares na vida aleijado do que, tendo os dois pés, seres lançado no inferno
\par 46 [onde não lhes morre o verme, nem o fogo se apaga].
\par 47 E, se um dos teus olhos te faz tropeçar, arranca-o; é melhor entrares no reino de Deus com um só dos teus olhos do que, tendo os dois seres lançado no inferno,
\par 48 onde não lhes morre o verme, nem o fogo se apaga.
\par 49 Porque cada um será salgado com fogo.
\par 50 Bom é o sal; mas, se o sal vier a tornar-se insípido, como lhe restaurar o sabor? Tende sal em vós mesmos e paz uns com os outros.

\chapter{10}

\par 1 Levantando-se Jesus, foi dali para o território da Judéia, além do Jordão. E outra vez as multidões se reuniram junto a ele, e, de novo, ele as ensinava, segundo o seu costume.
\par 2 E, aproximando-se alguns fariseus, o experimentaram, perguntando-lhe: É lícito ao marido repudiar sua mulher?
\par 3 Ele lhes respondeu: Que vos ordenou Moisés?
\par 4 Tornaram eles: Moisés permitiu lavrar carta de divórcio e repudiar.
\par 5 Mas Jesus lhes disse: Por causa da dureza do vosso coração, ele vos deixou escrito esse mandamento;
\par 6 porém, desde o princípio da criação, Deus os fez homem e mulher.
\par 7 Por isso, deixará o homem a seu pai e mãe [e unir-se-á a sua mulher],
\par 8 e, com sua mulher, serão os dois uma só carne. De modo que já não são dois, mas uma só carne.
\par 9 Portanto, o que Deus ajuntou não separe o homem.
\par 10 Em casa, voltaram os discípulos a interrogá-lo sobre este assunto.
\par 11 E ele lhes disse: Quem repudiar sua mulher e casar com outra comete adultério contra aquela.
\par 12 E, se ela repudiar seu marido e casar com outro, comete adultério.
\par 13 Então, lhe trouxeram algumas crianças para que as tocasse, mas os discípulos os repreendiam.
\par 14 Jesus, porém, vendo isto, indignou-se e disse-lhes: Deixai vir a mim os pequeninos, não os embaraceis, porque dos tais é o reino de Deus.
\par 15 Em verdade vos digo: Quem não receber o reino de Deus como uma criança de maneira nenhuma entrará nele.
\par 16 Então, tomando-as nos braços e impondo-lhes as mãos, as abençoava.
\par 17 E, pondo-se Jesus a caminho, correu um homem ao seu encontro e, ajoelhando-se, perguntou-lhe: Bom Mestre, que farei para herdar a vida eterna?
\par 18 Respondeu-lhe Jesus: Por que me chamas bom? Ninguém é bom senão um, que é Deus.
\par 19 Sabes os mandamentos: Não matarás, não adulterarás, não furtarás, não dirás falso testemunho, não defraudarás ninguém, honra a teu pai e tua mãe.
\par 20 Então, ele respondeu: Mestre, tudo isso tenho observado desde a minha juventude.
\par 21 E Jesus, fitando-o, o amou e disse: Só uma coisa te falta: Vai, vende tudo o que tens, dá-o aos pobres e terás um tesouro no céu; então, vem e segue-me.
\par 22 Ele, porém, contrariado com esta palavra, retirou-se triste, porque era dono de muitas propriedades.
\par 23 Então, Jesus, olhando ao redor, disse aos seus discípulos: Quão dificilmente entrarão no reino de Deus os que têm riquezas!
\par 24 Os discípulos estranharam estas palavras; mas Jesus insistiu em dizer-lhes: Filhos, quão difícil é [para os que confiam nas riquezas] entrar no reino de Deus!
\par 25 É mais fácil passar um camelo pelo fundo de uma agulha do que entrar um rico no reino de Deus.
\par 26 Eles ficaram sobremodo maravilhados, dizendo entre si: Então, quem pode ser salvo?
\par 27 Jesus, porém, fitando neles o olhar, disse: Para os homens é impossível; contudo, não para Deus, porque para Deus tudo é possível.
\par 28 Então, Pedro começou a dizer-lhe: Eis que nós tudo deixamos e te seguimos.
\par 29 Tornou Jesus: Em verdade vos digo que ninguém há que tenha deixado casa, ou irmãos, ou irmãs, ou mãe, ou pai, ou filhos, ou campos por amor de mim e por amor do evangelho,
\par 30 que não receba, já no presente, o cêntuplo de casas, irmãos, irmãs, mães, filhos e campos, com perseguições; e, no mundo por vir, a vida eterna.
\par 31 Porém muitos primeiros serão últimos; e os últimos, primeiros.
\par 32 Estavam de caminho, subindo para Jerusalém, e Jesus ia adiante dos seus discípulos. Estes se admiravam e o seguiam tomados de apreensões. E Jesus, tornando a levar à parte os doze, passou a revelar-lhes as coisas que lhe deviam sobrevir, dizendo:
\par 33 Eis que subimos para Jerusalém, e o Filho do Homem será entregue aos principais sacerdotes e aos escribas; condená-lo-ão à morte e o entregarão aos gentios;
\par 34 hão de escarnecê-lo, cuspir nele, açoitá-lo e matá-lo; mas, depois de três dias, ressuscitará.
\par 35 Então, se aproximaram dele Tiago e João, filhos de Zebedeu, dizendo-lhe: Mestre, queremos que nos concedas o que te vamos pedir.
\par 36 E ele lhes perguntou: Que quereis que vos faça?
\par 37 Responderam-lhe: Permite-nos que, na tua glória, nos assentemos um à tua direita e o outro à tua esquerda.
\par 38 Mas Jesus lhes disse: Não sabeis o que pedis. Podeis vós beber o cálice que eu bebo ou receber o batismo com que eu sou batizado?
\par 39 Disseram-lhe: Podemos. Tornou-lhes Jesus: Bebereis o cálice que eu bebo e recebereis o batismo com que eu sou batizado;
\par 40 quanto, porém, ao assentar-se à minha direita ou à minha esquerda, não me compete concedê-lo; porque é para aqueles a quem está preparado.
\par 41 Ouvindo isto, indignaram-se os dez contra Tiago e João.
\par 42 Mas Jesus, chamando-os para junto de si, disse-lhes: Sabeis que os que são considerados governadores dos povos têm-nos sob seu domínio, e sobre eles os seus maiorais exercem autoridade.
\par 43 Mas entre vós não é assim; pelo contrário, quem quiser tornar-se grande entre vós, será esse o que vos sirva;
\par 44 e quem quiser ser o primeiro entre vós será servo de todos.
\par 45 Pois o próprio Filho do Homem não veio para ser servido, mas para servir e dar a sua vida em resgate por muitos.
\par 46 E foram para Jericó. Quando ele saía de Jericó, juntamente com os discípulos e numerosa multidão, Bartimeu, cego mendigo, filho de Timeu, estava assentado à beira do caminho
\par 47 e, ouvindo que era Jesus, o Nazareno, pôs-se a clamar: Jesus, Filho de Davi, tem compaixão de mim!
\par 48 E muitos o repreendiam, para que se calasse; mas ele cada vez gritava mais: Filho de Davi, tem misericórdia de mim!
\par 49 Parou Jesus e disse: Chamai-o. Chamaram, então, o cego, dizendo-lhe: Tem bom ânimo; levanta-te, ele te chama.
\par 50 Lançando de si a capa, levantou-se de um salto e foi ter com Jesus.
\par 51 Perguntou-lhe Jesus: Que queres que eu te faça? Respondeu o cego: Mestre, que eu torne a ver.
\par 52 Então, Jesus lhe disse: Vai, a tua fé te salvou. E imediatamente tornou a ver e seguia a Jesus estrada fora.

\chapter{11}

\par 1 Quando se aproximavam de Jerusalém, de Betfagé e Betânia, junto ao monte das Oliveiras, enviou Jesus dois dos seus discípulos
\par 2 e disse-lhes: Ide à aldeia que aí está diante de vós e, logo ao entrar, achareis preso um jumentinho, o qual ainda ninguém montou; desprendei-o e trazei-o.
\par 3 Se alguém vos perguntar: Por que fazeis isso? Respondei: O Senhor precisa dele e logo o mandará de volta para aqui.
\par 4 Então, foram e acharam o jumentinho preso, junto ao portão, do lado de fora, na rua, e o desprenderam.
\par 5 Alguns dos que ali estavam reclamaram: Que fazeis, soltando o jumentinho?
\par 6 Eles, porém, responderam conforme as instruções de Jesus; então, os deixaram ir.
\par 7 Levaram o jumentinho, sobre o qual puseram as suas vestes, e Jesus o montou.
\par 8 E muitos estendiam as suas vestes no caminho, e outros, ramos que haviam cortado dos campos.
\par 9 Tanto os que iam adiante dele como os que vinham depois clamavam: Hosana! Bendito o que vem em nome do Senhor!
\par 10 Bendito o reino que vem, o reino de Davi, nosso pai! Hosana, nas maiores alturas!
\par 11 E, quando entrou em Jerusalém, no templo, tendo observado tudo, como fosse já tarde, saiu para Betânia com os doze.
\par 12 No dia seguinte, quando saíram de Betânia, teve fome.
\par 13 E, vendo de longe uma figueira com folhas, foi ver se nela, porventura, acharia alguma coisa. Aproximando-se dela, nada achou, senão folhas; porque não era tempo de figos.
\par 14 Então, lhe disse Jesus: Nunca jamais coma alguém fruto de ti! E seus discípulos ouviram isto.
\par 15 E foram para Jerusalém. Entrando ele no templo, passou a expulsar os que ali vendiam e compravam; derribou as mesas dos cambistas e as cadeiras dos que vendiam pombas.
\par 16 Não permitia que alguém conduzisse qualquer utensílio pelo templo;
\par 17 também os ensinava e dizia: Não está escrito: A minha casa será chamada casa de oração para todas as nações? Vós, porém, a tendes transformado em covil de salteadores.
\par 18 E os principais sacerdotes e escribas ouviam estas coisas e procuravam um modo de lhe tirar a vida; pois o temiam, porque toda a multidão se maravilhava de sua doutrina.
\par 19 Em vindo a tarde, saíram da cidade.
\par 20 E, passando eles pela manhã, viram que a figueira secara desde a raiz.
\par 21 Então, Pedro, lembrando-se, falou: Mestre, eis que a figueira que amaldiçoaste secou.
\par 22 Ao que Jesus lhes disse: Tende fé em Deus;
\par 23 porque em verdade vos afirmo que, se alguém disser a este monte: Ergue-te e lança-te no mar, e não duvidar no seu coração, mas crer que se fará o que diz, assim será com ele.
\par 24 Por isso, vos digo que tudo quanto em oração pedirdes, crede que recebestes, e será assim convosco.
\par 25 E, quando estiverdes orando, se tendes alguma coisa contra alguém, perdoai, para que vosso Pai celestial vos perdoe as vossas ofensas.
\par 26 [Mas, se não perdoardes, também vosso Pai celestial não vos perdoará as vossas ofensas.]
\par 27 Então, regressaram para Jerusalém. E, andando ele pelo templo, vieram ao seu encontro os principais sacerdotes, os escribas e os anciãos
\par 28 e lhe perguntaram: Com que autoridade fazes estas coisas? Ou quem te deu tal autoridade para as fazeres?
\par 29 Jesus lhes respondeu: Eu vos farei uma pergunta; respondei-me, e eu vos direi com que autoridade faço estas coisas.
\par 30 O batismo de João era do céu ou dos homens? Respondei!
\par 31 E eles discorriam entre si: Se dissermos: Do céu, dirá: Então, por que não acreditastes nele?
\par 32 Se, porém, dissermos: dos homens, é de temer o povo. Porque todos consideravam a João como profeta.
\par 33 Então, responderam a Jesus: Não sabemos. E Jesus, por sua vez, lhes disse: Nem eu tampouco vos digo com que autoridade faço estas coisas.

\chapter{12}

\par 1 Depois, entrou Jesus a falar-lhes por parábola: Um homem plantou uma vinha, cercou-a de uma sebe, construiu um lagar, edificou uma torre, arrendou-a a uns lavradores e ausentou-se do país.
\par 2 No tempo da colheita, enviou um servo aos lavradores para que recebesse deles dos frutos da vinha;
\par 3 eles, porém, o agarraram, espancaram e o despacharam vazio.
\par 4 De novo, lhes enviou outro servo, e eles o esbordoaram na cabeça e o insultaram.
\par 5 Ainda outro lhes mandou, e a este mataram. Muitos outros lhes enviou, dos quais espancaram uns e mataram outros.
\par 6 Restava-lhe ainda um, seu filho amado; a este lhes enviou, por fim, dizendo: Respeitarão a meu filho.
\par 7 Mas os tais lavradores disseram entre si: Este é o herdeiro; ora, vamos, matemo-lo, e a herança será nossa.
\par 8 E, agarrando-o, mataram-no e o atiraram para fora da vinha.
\par 9 Que fará, pois, o dono da vinha? Virá, exterminará aqueles lavradores e passará a vinha a outros.
\par 10 Ainda não lestes esta Escritura: A pedra que os construtores rejeitaram, essa veio a ser a principal pedra, angular;
\par 11 isto procede do Senhor, e é maravilhoso aos nossos olhos?
\par 12 E procuravam prendê-lo, mas temiam o povo; porque compreenderam que contra eles proferira esta parábola. Então, desistindo, retiraram-se.
\par 13 E enviaram-lhe alguns dos fariseus e dos herodianos, para que o apanhassem em alguma palavra.
\par 14 Chegando, disseram-lhe: Mestre, sabemos que és verdadeiro e não te importas com quem quer que seja, porque não olhas a aparência dos homens; antes, segundo a verdade, ensinas o caminho de Deus; é lícito pagar tributo a César ou não? Devemos ou não devemos pagar?
\par 15 Mas Jesus, percebendo-lhes a hipocrisia, respondeu: Por que me experimentais? Trazei-me um denário para que eu o veja.
\par 16 E eles lho trouxeram. Perguntou-lhes: De quem é esta efígie e inscrição? Responderam: De César.
\par 17 Disse-lhes, então, Jesus: Dai a César o que é de César e a Deus o que é de Deus. E muito se admiraram dele.
\par 18 Então, os saduceus, que dizem não haver ressurreição, aproximaram-se dele e lhe perguntaram, dizendo:
\par 19 Mestre, Moisés nos deixou escrito que, se morrer o irmão de alguém e deixar mulher sem filhos, seu irmão a tome como esposa e suscite descendência a seu irmão.
\par 20 Ora, havia sete irmãos; o primeiro casou e morreu sem deixar descendência;
\par 21 o segundo desposou a viúva e morreu, também sem deixar descendência; e o terceiro, da mesma forma.
\par 22 E, assim, os sete não deixaram descendência. Por fim, depois de todos, morreu também a mulher.
\par 23 Na ressurreição, quando eles ressuscitarem, de qual deles será ela a esposa? Porque os sete a desposaram.
\par 24 Respondeu-lhes Jesus: Não provém o vosso erro de não conhecerdes as Escrituras, nem o poder de Deus?
\par 25 Pois, quando ressuscitarem de entre os mortos, nem casarão, nem se darão em casamento; porém, são como os anjos nos céus.
\par 26 Quanto à ressurreição dos mortos, não tendes lido no Livro de Moisés, no trecho referente à sarça, como Deus lhe falou: Eu sou o Deus de Abraão, o Deus de Isaque e o Deus de Jacó?
\par 27 Ora, ele não é Deus de mortos, e sim de vivos. Laborais em grande erro.
\par 28 Chegando um dos escribas, tendo ouvido a discussão entre eles, vendo como Jesus lhes houvera respondido bem, perguntou-lhe: Qual é o principal de todos os mandamentos?
\par 29 Respondeu Jesus: O principal é: Ouve, ó Israel, o Senhor, nosso Deus, é o único Senhor!
\par 30 Amarás, pois, o Senhor, teu Deus, de todo o teu coração, de toda a tua alma, de todo o teu entendimento e de toda a tua força.
\par 31 O segundo é: Amarás o teu próximo como a ti mesmo. Não há outro mandamento maior do que estes.
\par 32 Disse-lhe o escriba: Muito bem, Mestre, e com verdade disseste que ele é o único, e não há outro senão ele,
\par 33 e que amar a Deus de todo o coração e de todo o entendimento e de toda a força, e amar ao próximo como a si mesmo excede a todos os holocaustos e sacrifícios.
\par 34 Vendo Jesus que ele havia respondido sabiamente, declarou-lhe: Não estás longe do reino de Deus. E já ninguém mais ousava interrogá-lo.
\par 35 Jesus, ensinando no templo, perguntou: Como dizem os escribas que o Cristo é filho de Davi?
\par 36 O próprio Davi falou, pelo Espírito Santo: Disse o Senhor ao meu Senhor: Assenta-te à minha direita, até que eu ponha os teus inimigos debaixo dos teus pés.
\par 37 O mesmo Davi chama-lhe Senhor; como, pois, é ele seu filho? E a grande multidão o ouvia com prazer.
\par 38 E, ao ensinar, dizia ele: Guardai-vos dos escribas, que gostam de andar com vestes talares e das saudações nas praças;
\par 39 e das primeiras cadeiras nas sinagogas e dos primeiros lugares nos banquetes;
\par 40 os quais devoram as casas das viúvas e, para o justificar, fazem longas orações; estes sofrerão juízo muito mais severo.
\par 41 Assentado diante do gazofilácio, observava Jesus como o povo lançava ali o dinheiro. Ora, muitos ricos depositavam grandes quantias.
\par 42 Vindo, porém, uma viúva pobre, depositou duas pequenas moedas correspondentes a um quadrante.
\par 43 E, chamando os seus discípulos, disse-lhes: Em verdade vos digo que esta viúva pobre depositou no gazofilácio mais do que o fizeram todos os ofertantes.
\par 44 Porque todos eles ofertaram do que lhes sobrava; ela, porém, da sua pobreza deu tudo quanto possuía, todo o seu sustento.

\chapter{13}

\par 1 Ao sair Jesus do templo, disse-lhe um de seus discípulos: Mestre! Que pedras, que construções!
\par 2 Mas Jesus lhe disse: Vês estas grandes construções? Não ficará pedra sobre pedra, que não seja derribada.
\par 3 No monte das Oliveiras, defronte do templo, achava-se Jesus assentado, quando Pedro, Tiago, João e André lhe perguntaram em particular:
\par 4 Dize-nos quando sucederão estas coisas, e que sinal haverá quando todas elas estiverem para cumprir-se.
\par 5 Então, Jesus passou a dizer-lhes: Vede que ninguém vos engane.
\par 6 Muitos virão em meu nome, dizendo: Sou eu; e enganarão a muitos.
\par 7 Quando, porém, ouvirdes falar de guerras e rumores de guerras, não vos assusteis; é necessário assim acontecer, mas ainda não é o fim.
\par 8 Porque se levantará nação contra nação, e reino, contra reino. Haverá terremotos em vários lugares e também fomes. Estas coisas são o princípio das dores.
\par 9 Estai vós de sobreaviso, porque vos entregarão aos tribunais e às sinagogas; sereis açoitados, e vos farão comparecer à presença de governadores e reis, por minha causa, para lhes servir de testemunho.
\par 10 Mas é necessário que primeiro o evangelho seja pregado a todas as nações.
\par 11 Quando, pois, vos levarem e vos entregarem, não vos preocupeis com o que haveis de dizer, mas o que vos for concedido naquela hora, isso falai; porque não sois vós os que falais, mas o Espírito Santo.
\par 12 Um irmão entregará à morte outro irmão, e o pai, ao filho; filhos haverá que se levantarão contra os progenitores e os matarão.
\par 13 Sereis odiados de todos por causa do meu nome; aquele, porém, que perseverar até ao fim, esse será salvo.
\par 14 Quando, pois, virdes o abominável da desolação situado onde não deve estar (quem lê entenda), então, os que estiverem na Judéia fujam para os montes;
\par 15 quem estiver em cima, no eirado, não desça nem entre para tirar da sua casa alguma coisa;
\par 16 e o que estiver no campo não volte atrás para buscar a sua capa.
\par 17 Ai das que estiverem grávidas e das que amamentarem naqueles dias!
\par 18 Orai para que isso não suceda no inverno.
\par 19 Porque aqueles dias serão de tamanha tribulação como nunca houve desde o princípio do mundo, que Deus criou, até agora e nunca jamais haverá.
\par 20 Não tivesse o Senhor abreviado aqueles dias, e ninguém se salvaria; mas, por causa dos eleitos que ele escolheu, abreviou tais dias.
\par 21 Então, se alguém vos disser: Eis aqui o Cristo! Ou: Ei-lo ali! Não acrediteis;
\par 22 pois surgirão falsos cristos e falsos profetas, operando sinais e prodígios, para enganar, se possível, os próprios eleitos.
\par 23 Estai vós de sobreaviso; tudo vos tenho predito.
\par 24 Mas, naqueles dias, após a referida tribulação, o sol escurecerá, a lua não dará a sua claridade,
\par 25 as estrelas cairão do firmamento, e os poderes dos céus serão abalados.
\par 26 Então, verão o Filho do Homem vir nas nuvens, com grande poder e glória.
\par 27 E ele enviará os anjos e reunirá os seus escolhidos dos quatro ventos, da extremidade da terra até à extremidade do céu.
\par 28 Aprendei, pois, a parábola da figueira: quando já os seus ramos se renovam, e as folhas brotam, sabeis que está próximo o verão.
\par 29 Assim, também vós: quando virdes acontecer estas coisas, sabei que está próximo, às portas.
\par 30 Em verdade vos digo que não passará esta geração sem que tudo isto aconteça.
\par 31 Passará o céu e a terra, porém as minhas palavras não passarão.
\par 32 Mas a respeito daquele dia ou da hora ninguém sabe; nem os anjos no céu, nem o Filho, senão o Pai.
\par 33 Estai de sobreaviso, vigiai [e orai]; porque não sabeis quando será o tempo.
\par 34 É como um homem que, ausentando-se do país, deixa a sua casa, dá autoridade aos seus servos, a cada um a sua obrigação, e ao porteiro ordena que vigie.
\par 35 Vigiai, pois, porque não sabeis quando virá o dono da casa: se à tarde, se à meia-noite, se ao cantar do galo, se pela manhã;
\par 36 para que, vindo ele inesperadamente, não vos ache dormindo.
\par 37 O que, porém, vos digo, digo a todos: vigiai!

\chapter{14}

\par 1 Dali a dois dias, era a Páscoa e a Festa dos Pães Asmos; e os principais sacerdotes e os escribas procuravam como o prenderiam, à traição, e o matariam.
\par 2 Pois diziam: Não durante a festa, para que não haja tumulto entre o povo.
\par 3 Estando ele em Betânia, reclinado à mesa, em casa de Simão, o leproso, veio uma mulher trazendo um vaso de alabastro com preciosíssimo perfume de nardo puro; e, quebrando o alabastro, derramou o bálsamo sobre a cabeça de Jesus.
\par 4 Indignaram-se alguns entre si e diziam: Para que este desperdício de bálsamo?
\par 5 Porque este perfume poderia ser vendido por mais de trezentos denários e dar-se aos pobres. E murmuravam contra ela.
\par 6 Mas Jesus disse: Deixai-a; por que a molestais? Ela praticou boa ação para comigo.
\par 7 Porque os pobres, sempre os tendes convosco e, quando quiserdes, podeis fazer-lhes bem, mas a mim nem sempre me tendes.
\par 8 Ela fez o que pôde: antecipou-se a ungir-me para a sepultura.
\par 9 Em verdade vos digo: onde for pregado em todo o mundo o evangelho, será também contado o que ela fez, para memória sua.
\par 10 E Judas Iscariotes, um dos doze, foi ter com os principais sacerdotes, para lhes entregar Jesus.
\par 11 Eles, ouvindo-o, alegraram-se e lhe prometeram dinheiro; nesse meio tempo, buscava ele uma boa ocasião para o entregar.
\par 12 E, no primeiro dia da Festa dos Pães Asmos, quando se fazia o sacrifício do cordeiro pascal, disseram-lhe seus discípulos: Onde queres que vamos fazer os preparativos para comeres a Páscoa?
\par 13 Então, enviou dois dos seus discípulos, dizendo-lhes: Ide à cidade, e vos sairá ao encontro um homem trazendo um cântaro de água;
\par 14 segui-o e dizei ao dono da casa onde ele entrar que o Mestre pergunta: Onde é o meu aposento no qual hei de comer a Páscoa com os meus discípulos?
\par 15 E ele vos mostrará um espaçoso cenáculo mobilado e pronto; ali fazei os preparativos.
\par 16 Saíram, pois, os discípulos, foram à cidade e, achando tudo como Jesus lhes tinha dito, prepararam a Páscoa.
\par 17 Ao cair da tarde, foi com os doze.
\par 18 Quando estavam à mesa e comiam, disse Jesus: Em verdade vos digo que um dentre vós, o que come comigo, me trairá.
\par 19 E eles começaram a entristecer-se e a dizer-lhe, um após outro: Porventura, sou eu?
\par 20 Respondeu-lhes: É um dos doze, o que mete comigo a mão no prato.
\par 21 Pois o Filho do Homem vai, como está escrito a seu respeito; mas ai daquele por intermédio de quem o Filho do Homem está sendo traído! Melhor lhe fora não haver nascido!
\par 22 E, enquanto comiam, tomou Jesus um pão e, abençoando-o, o partiu e lhes deu, dizendo: Tomai, isto é o meu corpo.
\par 23 A seguir, tomou Jesus um cálice e, tendo dado graças, o deu aos seus discípulos; e todos beberam dele.
\par 24 Então, lhes disse: Isto é o meu sangue, o sangue da [nova] aliança, derramado em favor de muitos.
\par 25 Em verdade vos digo que jamais beberei do fruto da videira, até àquele dia em que o hei de beber, novo, no reino de Deus.
\par 26 Tendo cantado um hino, saíram para o monte das Oliveiras.
\par 27 Então, lhes disse Jesus: Todos vós vos escandalizareis, porque está escrito: Ferirei o pastor, e as ovelhas ficarão dispersas.
\par 28 Mas, depois da minha ressurreição, irei adiante de vós para a Galiléia.
\par 29 Disse-lhe Pedro: Ainda que todos se escandalizem, eu, jamais!
\par 30 Respondeu-lhe Jesus: Em verdade te digo que hoje, nesta noite, antes que duas vezes cante o galo, tu me negarás três vezes.
\par 31 Mas ele insistia com mais veemência: Ainda que me seja necessário morrer contigo, de nenhum modo te negarei. Assim disseram todos.
\par 32 Então, foram a um lugar chamado Getsêmani; ali chegados, disse Jesus a seus discípulos: Assentai-vos aqui, enquanto eu vou orar.
\par 33 E, levando consigo a Pedro, Tiago e João, começou a sentir-se tomado de pavor e de angústia.
\par 34 E lhes disse: A minha alma está profundamente triste até à morte; ficai aqui e vigiai.
\par 35 E, adiantando-se um pouco, prostrou-se em terra; e orava para que, se possível, lhe fosse poupada aquela hora.
\par 36 E dizia: Aba, Pai, tudo te é possível; passa de mim este cálice; contudo, não seja o que eu quero, e sim o que tu queres.
\par 37 Voltando, achou-os dormindo; e disse a Pedro: Simão, tu dormes? Não pudeste vigiar nem uma hora?
\par 38 Vigiai e orai, para que não entreis em tentação; o espírito, na verdade, está pronto, mas a carne é fraca.
\par 39 Retirando-se de novo, orou repetindo as mesmas palavras.
\par 40 Voltando, achou-os outra vez dormindo, porque os seus olhos estavam pesados; e não sabiam o que lhe responder.
\par 41 E veio pela terceira vez e disse-lhes: Ainda dormis e repousais! Basta! Chegou a hora; o Filho do Homem está sendo entregue nas mãos dos pecadores.
\par 42 Levantai-vos, vamos! Eis que o traidor se aproxima.
\par 43 E logo, falava ele ainda, quando chegou Judas, um dos doze, e com ele, vinda da parte dos principais sacerdotes, escribas e anciãos, uma turba com espadas e porretes.
\par 44 Ora, o traidor tinha-lhes dado esta senha: Aquele a quem eu beijar, é esse; prendei-o e levai-o com segurança.
\par 45 E, logo que chegou, aproximando-se, disse-lhe: Mestre! E o beijou.
\par 46 Então, lhe deitaram as mãos e o prenderam.
\par 47 Nisto, um dos circunstantes, sacando da espada, feriu o servo do sumo sacerdote e cortou-lhe a orelha.
\par 48 Disse-lhes Jesus: Saístes com espadas e porretes para prender-me, como a um salteador?
\par 49 Todos os dias eu estava convosco no templo, ensinando, e não me prendestes; contudo, é para que se cumpram as Escrituras.
\par 50 Então, deixando-o, todos fugiram.
\par 51 Seguia-o um jovem, coberto unicamente com um lençol, e lançaram-lhe a mão.
\par 52 Mas ele, largando o lençol, fugiu desnudo.
\par 53 E levaram Jesus ao sumo sacerdote, e reuniram-se todos os principais sacerdotes, os anciãos e os escribas.
\par 54 Pedro seguira-o de longe até ao interior do pátio do sumo sacerdote e estava assentado entre os serventuários, aquentando-se ao fogo.
\par 55 E os principais sacerdotes e todo o Sinédrio procuravam algum testemunho contra Jesus para o condenar à morte e não achavam.
\par 56 Pois muitos testemunhavam falsamente contra Jesus, mas os depoimentos não eram coerentes.
\par 57 E, levantando-se alguns, testificavam falsamente, dizendo:
\par 58 Nós o ouvimos declarar: Eu destruirei este santuário edificado por mãos humanas e, em três dias, construirei outro, não por mãos humanas.
\par 59 Nem assim o testemunho deles era coerente.
\par 60 Levantando-se o sumo sacerdote, no meio, perguntou a Jesus: Nada respondes ao que estes depõem contra ti?
\par 61 Ele, porém, guardou silêncio e nada respondeu. Tornou a interrogá-lo o sumo sacerdote e lhe disse: És tu o Cristo, o Filho do Deus Bendito?
\par 62 Jesus respondeu: Eu sou, e vereis o Filho do Homem assentado à direita do Todo-Poderoso e vindo com as nuvens do céu.
\par 63 Então, o sumo sacerdote rasgou as suas vestes e disse: Que mais necessidade temos de testemunhas?
\par 64 Ouvistes a blasfêmia; que vos parece? E todos o julgaram réu de morte.
\par 65 Puseram-se alguns a cuspir nele, a cobrir-lhe o rosto, a dar-lhe murros e a dizer-lhe: Profetiza! E os guardas o tomaram a bofetadas.
\par 66 Estando Pedro embaixo no pátio, veio uma das criadas do sumo sacerdote
\par 67 e, vendo a Pedro, que se aquentava, fixou-o e disse: Tu também estavas com Jesus, o Nazareno.
\par 68 Mas ele o negou, dizendo: Não o conheço, nem compreendo o que dizes. E saiu para o alpendre. [E o galo cantou.]
\par 69 E a criada, vendo-o, tornou a dizer aos circunstantes: Este é um deles.
\par 70 Mas ele outra vez o negou. E, pouco depois, os que ali estavam disseram a Pedro: Verdadeiramente, és um deles, porque também tu és galileu.
\par 71 Ele, porém, começou a praguejar e a jurar: Não conheço esse homem de quem falais!
\par 72 E logo cantou o galo pela segunda vez. Então, Pedro se lembrou da palavra que Jesus lhe dissera: Antes que duas vezes cante o galo, tu me negarás três vezes. E, caindo em si, desatou a chorar.

\chapter{15}

\par 1 Logo pela manhã, entraram em conselho os principais sacerdotes com os anciãos, os escribas e todo o Sinédrio; e, amarrando a Jesus, levaram-no e o entregaram a Pilatos.
\par 2 Pilatos o interrogou: És tu o rei dos judeus? Respondeu Jesus: Tu o dizes.
\par 3 Então, os principais sacerdotes o acusavam de muitas coisas.
\par 4 Tornou Pilatos a interrogá-lo: Nada respondes? Vê quantas acusações te fazem!
\par 5 Jesus, porém, não respondeu palavra, a ponto de Pilatos muito se admirar.
\par 6 Ora, por ocasião da festa, era costume soltar ao povo um dos presos, qualquer que eles pedissem.
\par 7 Havia um, chamado Barrabás, preso com amotinadores, os quais em um tumulto haviam cometido homicídio.
\par 8 Vindo a multidão, começou a pedir que lhes fizesse como de costume.
\par 9 E Pilatos lhes respondeu, dizendo: Quereis que eu vos solte o rei dos judeus?
\par 10 Pois ele bem percebia que por inveja os principais sacerdotes lho haviam entregado.
\par 11 Mas estes incitaram a multidão no sentido de que lhes soltasse, de preferência, Barrabás.
\par 12 Mas Pilatos lhes perguntou: Que farei, então, deste a quem chamais o rei dos judeus?
\par 13 Eles, porém, clamavam: Crucifica-o!
\par 14 Mas Pilatos lhes disse: Que mal fez ele? E eles gritavam cada vez mais: Crucifica-o!
\par 15 Então, Pilatos, querendo contentar a multidão, soltou-lhes Barrabás; e, após mandar açoitar a Jesus, entregou-o para ser crucificado.
\par 16 Então, os soldados o levaram para dentro do palácio, que é o pretório, e reuniram todo o destacamento.
\par 17 Vestiram-no de púrpura e, tecendo uma coroa de espinhos, lha puseram na cabeça.
\par 18 E o saudavam, dizendo: Salve, rei dos judeus!
\par 19 Davam-lhe na cabeça com um caniço, cuspiam nele e, pondo-se de joelhos, o adoravam.
\par 20 Depois de o terem escarnecido, despiram-lhe a púrpura e o vestiram com as suas próprias vestes. Então, conduziram Jesus para fora, com o fim de o crucificarem.
\par 21 E obrigaram a Simão Cireneu, que passava, vindo do campo, pai de Alexandre e de Rufo, a carregar-lhe a cruz.
\par 22 E levaram Jesus para o Gólgota, que quer dizer Lugar da Caveira.
\par 23 Deram-lhe a beber vinho com mirra; ele, porém, não tomou.
\par 24 Então, o crucificaram e repartiram entre si as vestes dele, lançando-lhes sorte, para ver o que levaria cada um.
\par 25 Era a hora terceira quando o crucificaram.
\par 26 E, por cima, estava, em epígrafe, a sua acusação: O REI DOS JUDEUS.
\par 27 Com ele crucificaram dois ladrões, um à sua direita, e outro à sua esquerda.
\par 28 [E cumpriu-se a Escritura que diz: Com malfeitores foi contado.]
\par 29 Os que iam passando, blasfemavam dele, meneando a cabeça e dizendo: Ah! Tu que destróis o santuário e, em três dias, o reedificas!
\par 30 Salva-te a ti mesmo, descendo da cruz!
\par 31 De igual modo, os principais sacerdotes com os escribas, escarnecendo, entre si diziam: Salvou os outros, a si mesmo não pode salvar-se;
\par 32 desça agora da cruz o Cristo, o rei de Israel, para que vejamos e creiamos. Também os que com ele foram crucificados o insultavam.
\par 33 Chegada a hora sexta, houve trevas sobre toda a terra até a hora nona.
\par 34 À hora nona, clamou Jesus em alta voz: Eloí, Eloí, lamá sabactâni? Que quer dizer: Deus meu, Deus meu, por que me desamparaste?
\par 35 Alguns dos que ali estavam, ouvindo isto, diziam: Vede, chama por Elias!
\par 36 E um deles correu a embeber uma esponja em vinagre e, pondo-a na ponta de um caniço, deu-lhe de beber, dizendo: Deixai, vejamos se Elias vem tirá-lo!
\par 37 Mas Jesus, dando um grande brado, expirou.
\par 38 E o véu do santuário rasgou-se em duas partes, de alto a baixo.
\par 39 O centurião que estava em frente dele, vendo que assim expirara, disse: Verdadeiramente, este homem era o Filho de Deus.
\par 40 Estavam também ali algumas mulheres, observando de longe; entre elas, Maria Madalena, Maria, mãe de Tiago, o menor, e de José, e Salomé;
\par 41 as quais, quando Jesus estava na Galiléia, o acompanhavam e serviam; e, além destas, muitas outras que haviam subido com ele para Jerusalém.
\par 42 Ao cair da tarde, por ser o dia da preparação, isto é, a véspera do sábado,
\par 43 vindo José de Arimatéia, ilustre membro do Sinédrio, que também esperava o reino de Deus, dirigiu-se resolutamente a Pilatos e pediu o corpo de Jesus.
\par 44 Mas Pilatos admirou-se de que ele já tivesse morrido. E, tendo chamado o centurião, perguntou-lhe se havia muito que morrera.
\par 45 Após certificar-se, pela informação do comandante, cedeu o corpo a José.
\par 46 Este, baixando o corpo da cruz, envolveu-o em um lençol que comprara e o depositou em um túmulo que tinha sido aberto numa rocha; e rolou uma pedra para a entrada do túmulo.
\par 47 Ora, Maria Madalena e Maria, mãe de José, observaram onde ele foi posto.

\chapter{16}

\par 1 Passado o sábado, Maria Madalena, Maria, mãe de Tiago, e Salomé, compraram aromas para irem embalsamá-lo.
\par 2 E, muito cedo, no primeiro dia da semana, ao despontar do sol, foram ao túmulo.
\par 3 Diziam umas às outras: Quem nos removerá a pedra da entrada do túmulo?
\par 4 E, olhando, viram que a pedra já estava removida; pois era muito grande.
\par 5 Entrando no túmulo, viram um jovem assentado ao lado direito, vestido de branco, e ficaram surpreendidas e atemorizadas.
\par 6 Ele, porém, lhes disse: Não vos atemorizeis; buscais a Jesus, o Nazareno, que foi crucificado; ele ressuscitou, não está mais aqui; vede o lugar onde o tinham posto.
\par 7 Mas ide, dizei a seus discípulos e a Pedro que ele vai adiante de vós para a Galiléia; lá o vereis, como ele vos disse.
\par 8 E, saindo elas, fugiram do sepulcro, porque estavam possuídas de temor e de assombro; e, de medo, nada disseram a ninguém.
\par 9 Havendo ele ressuscitado de manhã cedo no primeiro dia da semana, apareceu primeiro a Maria Madalena, da qual expelira sete demônios.
\par 10 E, partindo ela, foi anunciá-lo àqueles que, tendo sido companheiros de Jesus, se achavam tristes e choravam.
\par 11 Estes, ouvindo que ele vivia e que fora visto por ela, não acreditaram.
\par 12 Depois disto, manifestou-se em outra forma a dois deles que estavam de caminho para o campo.
\par 13 E, indo, eles o anunciaram aos demais, mas também a estes dois eles não deram crédito.
\par 14 Finalmente, apareceu Jesus aos onze, quando estavam à mesa, e censurou-lhes a incredulidade e dureza de coração, porque não deram crédito aos que o tinham visto já ressuscitado.
\par 15 E disse-lhes: Ide por todo o mundo e pregai o evangelho a toda criatura.
\par 16 Quem crer e for batizado será salvo; quem, porém, não crer será condenado.
\par 17 Estes sinais hão de acompanhar aqueles que crêem: em meu nome, expelirão demônios; falarão novas línguas;
\par 18 pegarão em serpentes; e, se alguma coisa mortífera beberem, não lhes fará mal; se impuserem as mãos sobre enfermos, eles ficarão curados.
\par 19 De fato, o Senhor Jesus, depois de lhes ter falado, foi recebido no céu e assentou-se à destra de Deus.
\par 20 E eles, tendo partido, pregaram em toda parte, cooperando com eles o Senhor e confirmando a palavra por meio de sinais, que se seguiam.


\end{document}