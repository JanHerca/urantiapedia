\begin{document}

\title{I Coríntios}


\chapter{1}

\par 1 Paulo, chamado pela vontade de Deus para ser apóstolo de Jesus Cristo, e o irmão Sóstenes,
\par 2 à igreja de Deus que está em Corinto, aos santificados em Cristo Jesus, chamados para ser santos, com todos os que em todo lugar invocam o nome de nosso Senhor Jesus Cristo, Senhor deles e nosso:
\par 3 graça a vós outros e paz, da parte de Deus, nosso Pai, e do Senhor Jesus Cristo.
\par 4 Sempre dou graças a [meu] Deus a vosso respeito, a propósito da sua graça, que vos foi dada em Cristo Jesus;
\par 5 porque, em tudo, fostes enriquecidos nele, em toda a palavra e em todo o conhecimento;
\par 6 assim como o testemunho de Cristo tem sido confirmado em vós,
\par 7 de maneira que não vos falte nenhum dom, aguardando vós a revelação de nosso Senhor Jesus Cristo,
\par 8 o qual também vos confirmará até ao fim, para serdes irrepreensíveis no Dia de nosso Senhor Jesus Cristo.
\par 9 Fiel é Deus, pelo qual fostes chamados à comunhão de seu Filho Jesus Cristo, nosso Senhor.
\par 10 Rogo-vos, irmãos, pelo nome de nosso Senhor Jesus Cristo, que faleis todos a mesma coisa e que não haja entre vós divisões; antes, sejais inteiramente unidos, na mesma disposição mental e no mesmo parecer.
\par 11 Pois a vosso respeito, meus irmãos, fui informado, pelos da casa de Cloe, de que há contendas entre vós.
\par 12 Refiro-me ao fato de cada um de vós dizer: Eu sou de Paulo, e eu, de Apolo, e eu, de Cefas, e eu, de Cristo.
\par 13 Acaso, Cristo está dividido? Foi Paulo crucificado em favor de vós ou fostes, porventura, batizados em nome de Paulo?
\par 14 Dou graças [a Deus] porque a nenhum de vós batizei, exceto Crispo e Gaio;
\par 15 para que ninguém diga que fostes batizados em meu nome.
\par 16 Batizei também a casa de Estéfanas; além destes, não me lembro se batizei algum outro.
\par 17 Porque não me enviou Cristo para batizar, mas para pregar o evangelho; não com sabedoria de palavra, para que se não anule a cruz de Cristo.
\par 18 Certamente, a palavra da cruz é loucura para os que se perdem, mas para nós, que somos salvos, poder de Deus.
\par 19 Pois está escrito: Destruirei a sabedoria dos sábios e aniquilarei a inteligência dos instruídos.
\par 20 Onde está o sábio? Onde, o escriba? Onde, o inquiridor deste século? Porventura, não tornou Deus louca a sabedoria do mundo?
\par 21 Visto como, na sabedoria de Deus, o mundo não o conheceu por sua própria sabedoria, aprouve a Deus salvar os que crêem pela loucura da pregação.
\par 22 Porque tanto os judeus pedem sinais, como os gregos buscam sabedoria;
\par 23 mas nós pregamos a Cristo crucificado, escândalo para os judeus, loucura para os gentios;
\par 24 mas para os que foram chamados, tanto judeus como gregos, pregamos a Cristo, poder de Deus e sabedoria de Deus.
\par 25 Porque a loucura de Deus é mais sábia do que os homens; e a fraqueza de Deus é mais forte do que os homens.
\par 26 Irmãos, reparai, pois, na vossa vocação; visto que não foram chamados muitos sábios segundo a carne, nem muitos poderosos, nem muitos de nobre nascimento;
\par 27 pelo contrário, Deus escolheu as coisas loucas do mundo para envergonhar os sábios e escolheu as coisas fracas do mundo para envergonhar as fortes;
\par 28 e Deus escolheu as coisas humildes do mundo, e as desprezadas, e aquelas que não são, para reduzir a nada as que são;
\par 29 a fim de que ninguém se vanglorie na presença de Deus.
\par 30 Mas vós sois dele, em Cristo Jesus, o qual se nos tornou, da parte de Deus, sabedoria, e justiça, e santificação, e redenção,
\par 31 para que, como está escrito: Aquele que se gloria, glorie-se no Senhor.

\chapter{2}

\par 1 Eu, irmãos, quando fui ter convosco, anunciando-vos o testemunho de Deus, não o fiz com ostentação de linguagem ou de sabedoria.
\par 2 Porque decidi nada saber entre vós, senão a Jesus Cristo e este crucificado.
\par 3 E foi em fraqueza, temor e grande tremor que eu estive entre vós.
\par 4 A minha palavra e a minha pregação não consistiram em linguagem persuasiva de sabedoria, mas em demonstração do Espírito e de poder,
\par 5 para que a vossa fé não se apoiasse em sabedoria humana, e sim no poder de Deus.
\par 6 Entretanto, expomos sabedoria entre os experimentados; não, porém, a sabedoria deste século, nem a dos poderosos desta época, que se reduzem a nada;
\par 7 mas falamos a sabedoria de Deus em mistério, outrora oculta, a qual Deus preordenou desde a eternidade para a nossa glória;
\par 8 sabedoria essa que nenhum dos poderosos deste século conheceu; porque, se a tivessem conhecido, jamais teriam crucificado o Senhor da glória;
\par 9 mas, como está escrito: Nem olhos viram, nem ouvidos ouviram, nem jamais penetrou em coração humano o que Deus tem preparado para aqueles que o amam.
\par 10 Mas Deus no-lo revelou pelo Espírito; porque o Espírito a todas as coisas perscruta, até mesmo as profundezas de Deus.
\par 11 Porque qual dos homens sabe as coisas do homem, senão o seu próprio espírito, que nele está? Assim, também as coisas de Deus, ninguém as conhece, senão o Espírito de Deus.
\par 12 Ora, nós não temos recebido o espírito do mundo, e sim o Espírito que vem de Deus, para que conheçamos o que por Deus nos foi dado gratuitamente.
\par 13 Disto também falamos, não em palavras ensinadas pela sabedoria humana, mas ensinadas pelo Espírito, conferindo coisas espirituais com espirituais.
\par 14 Ora, o homem natural não aceita as coisas do Espírito de Deus, porque lhe são loucura; e não pode entendê-las, porque elas se discernem espiritualmente.
\par 15 Porém o homem espiritual julga todas as coisas, mas ele mesmo não é julgado por ninguém.
\par 16 Pois quem conheceu a mente do Senhor, que o possa instruir? Nós, porém, temos a mente de Cristo.

\chapter{3}

\par 1 Eu, porém, irmãos, não vos pude falar como a espirituais, e sim como a carnais, como a crianças em Cristo.
\par 2 Leite vos dei a beber, não vos dei alimento sólido; porque ainda não podíeis suportá-lo. Nem ainda agora podeis, porque ainda sois carnais.
\par 3 Porquanto, havendo entre vós ciúmes e contendas, não é assim que sois carnais e andais segundo o homem?
\par 4 Quando, pois, alguém diz: Eu sou de Paulo, e outro: Eu, de Apolo, não é evidente que andais segundo os homens?
\par 5 Quem é Apolo? E quem é Paulo? Servos por meio de quem crestes, e isto conforme o Senhor concedeu a cada um.
\par 6 Eu plantei, Apolo regou; mas o crescimento veio de Deus.
\par 7 De modo que nem o que planta é alguma coisa, nem o que rega, mas Deus, que dá o crescimento.
\par 8 Ora, o que planta e o que rega são um; e cada um receberá o seu galardão, segundo o seu próprio trabalho.
\par 9 Porque de Deus somos cooperadores; lavoura de Deus, edifício de Deus sois vós.
\par 10 Segundo a graça de Deus que me foi dada, lancei o fundamento como prudente construtor; e outro edifica sobre ele. Porém cada um veja como edifica.
\par 11 Porque ninguém pode lançar outro fundamento, além do que foi posto, o qual é Jesus Cristo.
\par 12 Contudo, se o que alguém edifica sobre o fundamento é ouro, prata, pedras preciosas, madeira, feno, palha,
\par 13 manifesta se tornará a obra de cada um; pois o Dia a demonstrará, porque está sendo revelada pelo fogo; e qual seja a obra de cada um o próprio fogo o provará.
\par 14 Se permanecer a obra de alguém que sobre o fundamento edificou, esse receberá galardão;
\par 15 se a obra de alguém se queimar, sofrerá ele dano; mas esse mesmo será salvo, todavia, como que através do fogo.
\par 16 Não sabeis que sois santuário de Deus e que o Espírito de Deus habita em vós?
\par 17 Se alguém destruir o santuário de Deus, Deus o destruirá; porque o santuário de Deus, que sois vós, é sagrado.
\par 18 Ninguém se engane a si mesmo: se alguém dentre vós se tem por sábio neste século, faça-se estulto para se tornar sábio.
\par 19 Porque a sabedoria deste mundo é loucura diante de Deus; porquanto está escrito: Ele apanha os sábios na própria astúcia deles.
\par 20 E outra vez: O Senhor conhece os pensamentos dos sábios, que são pensamentos vãos.
\par 21 Portanto, ninguém se glorie nos homens; porque tudo é vosso:
\par 22 seja Paulo, seja Apolo, seja Cefas, seja o mundo, seja a vida, seja a morte, sejam as coisas presentes, sejam as futuras, tudo é vosso,
\par 23 e vós, de Cristo, e Cristo, de Deus.

\chapter{4}

\par 1 Assim, pois, importa que os homens nos considerem como ministros de Cristo e despenseiros dos mistérios de Deus.
\par 2 Ora, além disso, o que se requer dos despenseiros é que cada um deles seja encontrado fiel.
\par 3 Todavia, a mim mui pouco se me dá de ser julgado por vós ou por tribunal humano; nem eu tampouco julgo a mim mesmo.
\par 4 Porque de nada me argúi a consciência; contudo, nem por isso me dou por justificado, pois quem me julga é o Senhor.
\par 5 Portanto, nada julgueis antes do tempo, até que venha o Senhor, o qual não somente trará à plena luz as coisas ocultas das trevas, mas também manifestará os desígnios dos corações; e, então, cada um receberá o seu louvor da parte de Deus.
\par 6 Estas coisas, irmãos, apliquei-as figuradamente a mim mesmo e a Apolo, por vossa causa, para que por nosso exemplo aprendais isto: não ultrapasseis o que está escrito; a fim de que ninguém se ensoberbeça a favor de um em detrimento de outro.
\par 7 Pois quem é que te faz sobressair? E que tens tu que não tenhas recebido? E, se o recebeste, por que te vanglorias, como se o não tiveras recebido?
\par 8 Já estais fartos, já estais ricos; chegastes a reinar sem nós; sim, tomara reinásseis para que também nós viéssemos a reinar convosco.
\par 9 Porque a mim me parece que Deus nos pôs a nós, os apóstolos, em último lugar, como se fôssemos condenados à morte; porque nos tornamos espetáculo ao mundo, tanto a anjos, como a homens.
\par 10 Nós somos loucos por causa de Cristo, e vós, sábios em Cristo; nós, fracos, e vós, fortes; vós, nobres, e nós, desprezíveis.
\par 11 Até à presente hora, sofremos fome, e sede, e nudez; e somos esbofeteados, e não temos morada certa,
\par 12 e nos afadigamos, trabalhando com as nossas próprias mãos. Quando somos injuriados, bendizemos; quando perseguidos, suportamos;
\par 13 quando caluniados, procuramos conciliação; até agora, temos chegado a ser considerados lixo do mundo, escória de todos.
\par 14 Não vos escrevo estas coisas para vos envergonhar; pelo contrário, para vos admoestar como a filhos meus amados.
\par 15 Porque, ainda que tivésseis milhares de preceptores em Cristo, não teríeis, contudo, muitos pais; pois eu, pelo evangelho, vos gerei em Cristo Jesus.
\par 16 Admoesto-vos, portanto, a que sejais meus imitadores.
\par 17 Por esta causa, vos mandei Timóteo, que é meu filho amado e fiel no Senhor, o qual vos lembrará os meus caminhos em Cristo Jesus, como, por toda parte, ensino em cada igreja.
\par 18 Alguns se ensoberbeceram, como se eu não tivesse de ir ter convosco;
\par 19 mas, em breve, irei visitar-vos, se o Senhor quiser, e, então, conhecerei não a palavra, mas o poder dos ensoberbecidos.
\par 20 Porque o reino de Deus consiste não em palavra, mas em poder.
\par 21 Que preferis? Irei a vós outros com vara ou com amor e espírito de mansidão?

\chapter{5}

\par 1 Geralmente, se ouve que há entre vós imoralidade e imoralidade tal, como nem mesmo entre os gentios, isto é, haver quem se atreva a possuir a mulher de seu próprio pai.
\par 2 E, contudo, andais vós ensoberbecidos e não chegastes a lamentar, para que fosse tirado do vosso meio quem tamanho ultraje praticou?
\par 3 Eu, na verdade, ainda que ausente em pessoa, mas presente em espírito, já sentenciei, como se estivesse presente, que o autor de tal infâmia seja,
\par 4 em nome do Senhor Jesus, reunidos vós e o meu espírito, com o poder de Jesus, nosso Senhor,
\par 5 entregue a Satanás para a destruição da carne, a fim de que o espírito seja salvo no Dia do Senhor [Jesus].
\par 6 Não é boa a vossa jactância. Não sabeis que um pouco de fermento leveda a massa toda?
\par 7 Lançai fora o velho fermento, para que sejais nova massa, como sois, de fato, sem fermento. Pois também Cristo, nosso Cordeiro pascal, foi imolado.
\par 8 Por isso, celebremos a festa não com o velho fermento, nem com o fermento da maldade e da malícia, e sim com os asmos da sinceridade e da verdade.
\par 9 Já em carta vos escrevi que não vos associásseis com os impuros;
\par 10 refiro-me, com isto, não propriamente aos impuros deste mundo, ou aos avarentos, ou roubadores, ou idólatras; pois, neste caso, teríeis de sair do mundo.
\par 11 Mas, agora, vos escrevo que não vos associeis com alguém que, dizendo-se irmão, for impuro, ou avarento, ou idólatra, ou maldizente, ou beberrão, ou roubador; com esse tal, nem ainda comais.
\par 12 Pois com que direito haveria eu de julgar os de fora? Não julgais vós os de dentro?
\par 13 Os de fora, porém, Deus os julgará. Expulsai, pois, de entre vós o malfeitor.

\chapter{6}

\par 1 Aventura-se algum de vós, tendo questão contra outro, a submetê-lo a juízo perante os injustos e não perante os santos?
\par 2 Ou não sabeis que os santos hão de julgar o mundo? Ora, se o mundo deverá ser julgado por vós, sois, acaso, indignos de julgar as coisas mínimas?
\par 3 Não sabeis que havemos de julgar os próprios anjos? Quanto mais as coisas desta vida!
\par 4 Entretanto, vós, quando tendes a julgar negócios terrenos, constituís um tribunal daqueles que não têm nenhuma aceitação na igreja.
\par 5 Para vergonha vo-lo digo. Não há, porventura, nem ao menos um sábio entre vós, que possa julgar no meio da irmandade?
\par 6 Mas irá um irmão a juízo contra outro irmão, e isto perante incrédulos!
\par 7 O só existir entre vós demandas já é completa derrota para vós outros. Por que não sofreis, antes, a injustiça? Por que não sofreis, antes, o dano?
\par 8 Mas vós mesmos fazeis a injustiça e fazeis o dano, e isto aos próprios irmãos!
\par 9 Ou não sabeis que os injustos não herdarão o reino de Deus? Não vos enganeis: nem impuros, nem idólatras, nem adúlteros, nem efeminados, nem sodomitas,
\par 10 nem ladrões, nem avarentos, nem bêbados, nem maldizentes, nem roubadores herdarão o reino de Deus.
\par 11 Tais fostes alguns de vós; mas vós vos lavastes, mas fostes santificados, mas fostes justificados em o nome do Senhor Jesus Cristo e no Espírito do nosso Deus.
\par 12 Todas as coisas me são lícitas, mas nem todas convêm. Todas as coisas me são lícitas, mas eu não me deixarei dominar por nenhuma delas.
\par 13 Os alimentos são para o estômago, e o estômago, para os alimentos; mas Deus destruirá tanto estes como aquele. Porém o corpo não é para a impureza, mas, para o Senhor, e o Senhor, para o corpo.
\par 14 Deus ressuscitou o Senhor e também nos ressuscitará a nós pelo seu poder.
\par 15 Não sabeis que os vossos corpos são membros de Cristo? E eu, porventura, tomaria os membros de Cristo e os faria membros de meretriz? Absolutamente, não.
\par 16 Ou não sabeis que o homem que se une à prostituta forma um só corpo com ela? Porque, como se diz, serão os dois uma só carne.
\par 17 Mas aquele que se une ao Senhor é um espírito com ele.
\par 18 Fugi da impureza. Qualquer outro pecado que uma pessoa cometer é fora do corpo; mas aquele que pratica a imoralidade peca contra o próprio corpo.
\par 19 Acaso, não sabeis que o vosso corpo é santuário do Espírito Santo, que está em vós, o qual tendes da parte de Deus, e que não sois de vós mesmos?
\par 20 Porque fostes comprados por preço. Agora, pois, glorificai a Deus no vosso corpo.

\chapter{7}

\par 1 Quanto ao que me escrevestes, é bom que o homem não toque em mulher;
\par 2 mas, por causa da impureza, cada um tenha a sua própria esposa, e cada uma, o seu próprio marido.
\par 3 O marido conceda à esposa o que lhe é devido, e também, semelhantemente, a esposa, ao seu marido.
\par 4 A mulher não tem poder sobre o seu próprio corpo, e sim o marido; e também, semelhantemente, o marido não tem poder sobre o seu próprio corpo, e sim a mulher.
\par 5 Não vos priveis um ao outro, salvo talvez por mútuo consentimento, por algum tempo, para vos dedicardes à oração e, novamente, vos ajuntardes, para que Satanás não vos tente por causa da incontinência.
\par 6 E isto vos digo como concessão e não por mandamento.
\par 7 Quero que todos os homens sejam tais como também eu sou; no entanto, cada um tem de Deus o seu próprio dom; um, na verdade, de um modo; outro, de outro.
\par 8 E aos solteiros e viúvos digo que lhes seria bom se permanecessem no estado em que também eu vivo.
\par 9 Caso, porém, não se dominem, que se casem; porque é melhor casar do que viver abrasado.
\par 10 Ora, aos casados, ordeno, não eu, mas o Senhor, que a mulher não se separe do marido
\par 11 (se, porém, ela vier a separar-se, que não se case ou que se reconcilie com seu marido); e que o marido não se aparte de sua mulher.
\par 12 Aos mais digo eu, não o Senhor: se algum irmão tem mulher incrédula, e esta consente em morar com ele, não a abandone;
\par 13 e a mulher que tem marido incrédulo, e este consente em viver com ela, não deixe o marido.
\par 14 Porque o marido incrédulo é santificado no convívio da esposa, e a esposa incrédula é santificada no convívio do marido crente. Doutra sorte, os vossos filhos seriam impuros; porém, agora, são santos.
\par 15 Mas, se o descrente quiser apartar-se, que se aparte; em tais casos, não fica sujeito à servidão nem o irmão, nem a irmã; Deus vos tem chamado à paz.
\par 16 Pois, como sabes, ó mulher, se salvarás teu marido? Ou, como sabes, ó marido, se salvarás tua mulher?
\par 17 Ande cada um segundo o Senhor lhe tem distribuído, cada um conforme Deus o tem chamado. É assim que ordeno em todas as igrejas.
\par 18 Foi alguém chamado, estando circunciso? Não desfaça a circuncisão. Foi alguém chamado, estando incircunciso? Não se faça circuncidar.
\par 19 A circuncisão, em si, não é nada; a incircuncisão também nada é, mas o que vale é guardar as ordenanças de Deus.
\par 20 Cada um permaneça na vocação em que foi chamado.
\par 21 Foste chamado, sendo escravo? Não te preocupes com isso; mas, se ainda podes tornar-te livre, aproveita a oportunidade.
\par 22 Porque o que foi chamado no Senhor, sendo escravo, é liberto do Senhor; semelhantemente, o que foi chamado, sendo livre, é escravo de Cristo.
\par 23 Por preço fostes comprados; não vos torneis escravos de homens.
\par 24 Irmãos, cada um permaneça diante de Deus naquilo em que foi chamado.
\par 25 Com respeito às virgens, não tenho mandamento do Senhor; porém dou minha opinião, como tendo recebido do Senhor a misericórdia de ser fiel.
\par 26 Considero, por causa da angustiosa situação presente, ser bom para o homem permanecer assim como está.
\par 27 Estás casado? Não procures separar-te. Estás livre de mulher? Não procures casamento.
\par 28 Mas, se te casares, com isto não pecas; e também, se a virgem se casar, por isso não peca. Ainda assim, tais pessoas sofrerão angústia na carne, e eu quisera poupar-vos.
\par 29 Isto, porém, vos digo, irmãos: o tempo se abrevia; o que resta é que não só os casados sejam como se o não fossem;
\par 30 mas também os que choram, como se não chorassem; e os que se alegram, como se não se alegrassem; e os que compram, como se nada possuíssem;
\par 31 e os que se utilizam do mundo, como se dele não usassem; porque a aparência deste mundo passa.
\par 32 O que realmente eu quero é que estejais livres de preocupações. Quem não é casado cuida das coisas do Senhor, de como agradar ao Senhor;
\par 33 mas o que se casou cuida das coisas do mundo, de como agradar à esposa,
\par 34 e assim está dividido. Também a mulher, tanto a viúva como a virgem, cuida das coisas do Senhor, para ser santa, assim no corpo como no espírito; a que se casou, porém, se preocupa com as coisas do mundo, de como agradar ao marido.
\par 35 Digo isto em favor dos vossos próprios interesses; não que eu pretenda enredar-vos, mas somente para o que é decoroso e vos facilite o consagrar-vos, desimpedidamente, ao Senhor.
\par 36 Entretanto, se alguém julga que trata sem decoro a sua filha, estando já a passar-lhe a flor da idade, e as circunstâncias o exigem, faça o que quiser. Não peca; que se casem.
\par 37 Todavia, o que está firme em seu coração, não tendo necessidade, mas domínio sobre o seu próprio arbítrio, e isto bem firmado no seu ânimo, para conservar virgem a sua filha, bem fará.
\par 38 E, assim, quem casa a sua filha virgem faz bem; quem não a casa faz melhor.
\par 39 A mulher está ligada enquanto vive o marido; contudo, se falecer o marido, fica livre para casar com quem quiser, mas somente no Senhor.
\par 40 Todavia, será mais feliz se permanecer viúva, segundo a minha opinião; e penso que também eu tenho o Espírito de Deus.

\chapter{8}

\par 1 No que se refere às coisas sacrificadas a ídolos, reconhecemos que todos somos senhores do saber. O saber ensoberbece, mas o amor edifica.
\par 2 Se alguém julga saber alguma coisa, com efeito, não aprendeu ainda como convém saber.
\par 3 Mas, se alguém ama a Deus, esse é conhecido por ele.
\par 4 No tocante à comida sacrificada a ídolos, sabemos que o ídolo, de si mesmo, nada é no mundo e que não há senão um só Deus.
\par 5 Porque, ainda que há também alguns que se chamem deuses, quer no céu ou sobre a terra, como há muitos deuses e muitos senhores,
\par 6 todavia, para nós há um só Deus, o Pai, de quem são todas as coisas e para quem existimos; e um só Senhor, Jesus Cristo, pelo qual são todas as coisas, e nós também, por ele.
\par 7 Entretanto, não há esse conhecimento em todos; porque alguns, por efeito da familiaridade até agora com o ídolo, ainda comem dessas coisas como a ele sacrificadas; e a consciência destes, por ser fraca, vem a contaminar-se.
\par 8 Não é a comida que nos recomendará a Deus, pois nada perderemos, se não comermos, e nada ganharemos, se comermos.
\par 9 Vede, porém, que esta vossa liberdade não venha, de algum modo, a ser tropeço para os fracos.
\par 10 Porque, se alguém te vir a ti, que és dotado de saber, à mesa, em templo de ídolo, não será a consciência do que é fraco induzida a participar de comidas sacrificadas a ídolos?
\par 11 E assim, por causa do teu saber, perece o irmão fraco, pelo qual Cristo morreu.
\par 12 E deste modo, pecando contra os irmãos, golpeando-lhes a consciência fraca, é contra Cristo que pecais.
\par 13 E, por isso, se a comida serve de escândalo a meu irmão, nunca mais comerei carne, para que não venha a escandalizá-lo.

\chapter{9}

\par 1 Não sou eu, porventura, livre? Não sou apóstolo? Não vi Jesus, nosso Senhor? Acaso, não sois fruto do meu trabalho no Senhor?
\par 2 Se não sou apóstolo para outrem, certamente, o sou para vós outros; porque vós sois o selo do meu apostolado no Senhor.
\par 3 A minha defesa perante os que me interpelam é esta:
\par 4 não temos nós o direito de comer e beber?
\par 5 E também o de fazer-nos acompanhar de uma mulher irmã, como fazem os demais apóstolos, e os irmãos do Senhor, e Cefas?
\par 6 Ou somente eu e Barnabé não temos direito de deixar de trabalhar?
\par 7 Quem jamais vai à guerra à sua própria custa? Quem planta a vinha e não come do seu fruto? Ou quem apascenta um rebanho e não se alimenta do leite do rebanho?
\par 8 Porventura, falo isto como homem ou não o diz também a lei?
\par 9 Porque na lei de Moisés está escrito: Não atarás a boca ao boi, quando pisa o trigo. Acaso, é com bois que Deus se preocupa?
\par 10 Ou é, seguramente, por nós que ele o diz? Certo que é por nós que está escrito; pois o que lavra cumpre fazê-lo com esperança; o que pisa o trigo faça-o na esperança de receber a parte que lhe é devida.
\par 11 Se nós vos semeamos as coisas espirituais, será muito recolhermos de vós bens materiais?
\par 12 Se outros participam desse direito sobre vós, não o temos nós em maior medida? Entretanto, não usamos desse direito; antes, suportamos tudo, para não criarmos qualquer obstáculo ao evangelho de Cristo.
\par 13 Não sabeis vós que os que prestam serviços sagrados do próprio templo se alimentam? E quem serve ao altar do altar tira o seu sustento?
\par 14 Assim ordenou também o Senhor aos que pregam o evangelho que vivam do evangelho;
\par 15 eu, porém, não me tenho servido de nenhuma destas coisas e não escrevo isto para que assim se faça comigo; porque melhor me fora morrer, antes que alguém me anule esta glória.
\par 16 Se anuncio o evangelho, não tenho de que me gloriar, pois sobre mim pesa essa obrigação; porque ai de mim se não pregar o evangelho!
\par 17 Se o faço de livre vontade, tenho galardão; mas, se constrangido, é, então, a responsabilidade de despenseiro que me está confiada.
\par 18 Nesse caso, qual é o meu galardão? É que, evangelizando, proponha, de graça, o evangelho, para não me valer do direito que ele me dá.
\par 19 Porque, sendo livre de todos, fiz-me escravo de todos, a fim de ganhar o maior número possível.
\par 20 Procedi, para com os judeus, como judeu, a fim de ganhar os judeus; para os que vivem sob o regime da lei, como se eu mesmo assim vivesse, para ganhar os que vivem debaixo da lei, embora não esteja eu debaixo da lei.
\par 21 Aos sem lei, como se eu mesmo o fosse, não estando sem lei para com Deus, mas debaixo da lei de Cristo, para ganhar os que vivem fora do regime da lei.
\par 22 Fiz-me fraco para com os fracos, com o fim de ganhar os fracos. Fiz-me tudo para com todos, com o fim de, por todos os modos, salvar alguns.
\par 23 Tudo faço por causa do evangelho, com o fim de me tornar cooperador com ele.
\par 24 Não sabeis vós que os que correm no estádio, todos, na verdade, correm, mas um só leva o prêmio? Correi de tal maneira que o alcanceis.
\par 25 Todo atleta em tudo se domina; aqueles, para alcançar uma coroa corruptível; nós, porém, a incorruptível.
\par 26 Assim corro também eu, não sem meta; assim luto, não como desferindo golpes no ar.
\par 27 Mas esmurro o meu corpo e o reduzo à escravidão, para que, tendo pregado a outros, não venha eu mesmo a ser desqualificado.

\chapter{10}

\par 1 Ora, irmãos, não quero que ignoreis que nossos pais estiveram todos sob a nuvem, e todos passaram pelo mar,
\par 2 tendo sido todos batizados, assim na nuvem como no mar, com respeito a Moisés.
\par 3 Todos eles comeram de um só manjar espiritual
\par 4 e beberam da mesma fonte espiritual; porque bebiam de uma pedra espiritual que os seguia. E a pedra era Cristo.
\par 5 Entretanto, Deus não se agradou da maioria deles, razão por que ficaram prostrados no deserto.
\par 6 Ora, estas coisas se tornaram exemplos para nós, a fim de que não cobicemos as coisas más, como eles cobiçaram.
\par 7 Não vos façais, pois, idólatras, como alguns deles; porquanto está escrito: O povo assentou-se para comer e beber e levantou-se para divertir-se.
\par 8 E não pratiquemos imoralidade, como alguns deles o fizeram, e caíram, num só dia, vinte e três mil.
\par 9 Não ponhamos o Senhor à prova, como alguns deles já fizeram e pereceram pelas mordeduras das serpentes.
\par 10 Nem murmureis, como alguns deles murmuraram e foram destruídos pelo exterminador.
\par 11 Estas coisas lhes sobrevieram como exemplos e foram escritas para advertência nossa, de nós outros sobre quem os fins dos séculos têm chegado.
\par 12 Aquele, pois, que pensa estar em pé veja que não caia.
\par 13 Não vos sobreveio tentação que não fosse humana; mas Deus é fiel e não permitirá que sejais tentados além das vossas forças; pelo contrário, juntamente com a tentação, vos proverá livramento, de sorte que a possais suportar.
\par 14 Portanto, meus amados, fugi da idolatria.
\par 15 Falo como a criteriosos; julgai vós mesmos o que digo.
\par 16 Porventura, o cálice da bênção que abençoamos não é a comunhão do sangue de Cristo? O pão que partimos não é a comunhão do corpo de Cristo?
\par 17 Porque nós, embora muitos, somos unicamente um pão, um só corpo; porque todos participamos do único pão.
\par 18 Considerai o Israel segundo a carne; não é certo que aqueles que se alimentam dos sacrifícios são participantes do altar?
\par 19 Que digo, pois? Que o sacrificado ao ídolo é alguma coisa? Ou que o próprio ídolo tem algum valor?
\par 20 Antes, digo que as coisas que eles sacrificam, é a demônios que as sacrificam e não a Deus; e eu não quero que vos torneis associados aos demônios.
\par 21 Não podeis beber o cálice do Senhor e o cálice dos demônios; não podeis ser participantes da mesa do Senhor e da mesa dos demônios.
\par 22 Ou provocaremos zelos no Senhor? Somos, acaso, mais fortes do que ele?
\par 23 Todas as coisas são lícitas, mas nem todas convêm; todas são lícitas, mas nem todas edificam.
\par 24 Ninguém busque o seu próprio interesse, e sim o de outrem.
\par 25 Comei de tudo o que se vende no mercado, sem nada perguntardes por motivo de consciência;
\par 26 porque do Senhor é a terra e a sua plenitude.
\par 27 Se algum dentre os incrédulos vos convidar, e quiserdes ir, comei de tudo o que for posto diante de vós, sem nada perguntardes por motivo de consciência.
\par 28 Porém, se alguém vos disser: Isto é coisa sacrificada a ídolo, não comais, por causa daquele que vos advertiu e por causa da consciência;
\par 29 consciência, digo, não a tua propriamente, mas a do outro. Pois por que há de ser julgada a minha liberdade pela consciência alheia?
\par 30 Se eu participo com ações de graças, por que hei de ser vituperado por causa daquilo por que dou graças?
\par 31 Portanto, quer comais, quer bebais ou façais outra coisa qualquer, fazei tudo para a glória de Deus.
\par 32 Não vos torneis causa de tropeço nem para judeus, nem para gentios, nem tampouco para a igreja de Deus,
\par 33 assim como também eu procuro, em tudo, ser agradável a todos, não buscando o meu próprio interesse, mas o de muitos, para que sejam salvos.

\chapter{11}

\par 1 Sede meus imitadores, como também eu sou de Cristo.
\par 2 De fato, eu vos louvo porque, em tudo, vos lembrais de mim e retendes as tradições assim como vo-las entreguei.
\par 3 Quero, entretanto, que saibais ser Cristo o cabeça de todo homem, e o homem, o cabeça da mulher, e Deus, o cabeça de Cristo.
\par 4 Todo homem que ora ou profetiza, tendo a cabeça coberta, desonra a sua própria cabeça.
\par 5 Toda mulher, porém, que ora ou profetiza com a cabeça sem véu desonra a sua própria cabeça, porque é como se a tivesse rapada.
\par 6 Portanto, se a mulher não usa véu, nesse caso, que rape o cabelo. Mas, se lhe é vergonhoso o tosquiar-se ou rapar-se, cumpre-lhe usar véu.
\par 7 Porque, na verdade, o homem não deve cobrir a cabeça, por ser ele imagem e glória de Deus, mas a mulher é glória do homem.
\par 8 Porque o homem não foi feito da mulher, e sim a mulher, do homem.
\par 9 Porque também o homem não foi criado por causa da mulher, e sim a mulher, por causa do homem.
\par 10 Portanto, deve a mulher, por causa dos anjos, trazer véu na cabeça, como sinal de autoridade.
\par 11 No Senhor, todavia, nem a mulher é independente do homem, nem o homem, independente da mulher.
\par 12 Porque, como provém a mulher do homem, assim também o homem é nascido da mulher; e tudo vem de Deus.
\par 13 Julgai entre vós mesmos: é próprio que a mulher ore a Deus sem trazer o véu?
\par 14 Ou não vos ensina a própria natureza ser desonroso para o homem usar cabelo comprido?
\par 15 E que, tratando-se da mulher, é para ela uma glória? Pois o cabelo lhe foi dado em lugar de mantilha.
\par 16 Contudo, se alguém quer ser contencioso, saiba que nós não temos tal costume, nem as igrejas de Deus.
\par 17 Nisto, porém, que vos prescrevo, não vos louvo, porquanto vos ajuntais não para melhor, e sim para pior.
\par 18 Porque, antes de tudo, estou informado haver divisões entre vós quando vos reunis na igreja; e eu, em parte, o creio.
\par 19 Porque até mesmo importa que haja partidos entre vós, para que também os aprovados se tornem conhecidos em vosso meio.
\par 20 Quando, pois, vos reunis no mesmo lugar, não é a ceia do Senhor que comeis.
\par 21 Porque, ao comerdes, cada um toma, antecipadamente, a sua própria ceia; e há quem tenha fome, ao passo que há também quem se embriague.
\par 22 Não tendes, porventura, casas onde comer e beber? Ou menosprezais a igreja de Deus e envergonhais os que nada têm? Que vos direi? Louvar-vos-ei? Nisto, certamente, não vos louvo.
\par 23 Porque eu recebi do Senhor o que também vos entreguei: que o Senhor Jesus, na noite em que foi traído, tomou o pão;
\par 24 e, tendo dado graças, o partiu e disse: Isto é o meu corpo, que é dado por vós; fazei isto em memória de mim.
\par 25 Por semelhante modo, depois de haver ceado, tomou também o cálice, dizendo: Este cálice é a nova aliança no meu sangue; fazei isto, todas as vezes que o beberdes, em memória de mim.
\par 26 Porque, todas as vezes que comerdes este pão e beberdes o cálice, anunciais a morte do Senhor, até que ele venha.
\par 27 Por isso, aquele que comer o pão ou beber o cálice do Senhor, indignamente, será réu do corpo e do sangue do Senhor.
\par 28 Examine-se, pois, o homem a si mesmo, e, assim, coma do pão, e beba do cálice;
\par 29 pois quem come e bebe sem discernir o corpo, come e bebe juízo para si.
\par 30 Eis a razão por que há entre vós muitos fracos e doentes e não poucos que dormem.
\par 31 Porque, se nos julgássemos a nós mesmos, não seríamos julgados.
\par 32 Mas, quando julgados, somos disciplinados pelo Senhor, para não sermos condenados com o mundo.
\par 33 Assim, pois, irmãos meus, quando vos reunis para comer, esperai uns pelos outros.
\par 34 Se alguém tem fome, coma em casa, a fim de não vos reunirdes para juízo. Quanto às demais coisas, eu as ordenarei quando for ter convosco.

\chapter{12}

\par 1 A respeito dos dons espirituais, não quero, irmãos, que sejais ignorantes.
\par 2 Sabeis que, outrora, quando éreis gentios, deixáveis conduzir-vos aos ídolos mudos, segundo éreis guiados.
\par 3 Por isso, vos faço compreender que ninguém que fala pelo Espírito de Deus afirma: Anátema, Jesus! Por outro lado, ninguém pode dizer: Senhor Jesus!, senão pelo Espírito Santo.
\par 4 Ora, os dons são diversos, mas o Espírito é o mesmo.
\par 5 E também há diversidade nos serviços, mas o Senhor é o mesmo.
\par 6 E há diversidade nas realizações, mas o mesmo Deus é quem opera tudo em todos.
\par 7 A manifestação do Espírito é concedida a cada um visando a um fim proveitoso.
\par 8 Porque a um é dada, mediante o Espírito, a palavra da sabedoria; e a outro, segundo o mesmo Espírito, a palavra do conhecimento;
\par 9 a outro, no mesmo Espírito, a fé; e a outro, no mesmo Espírito, dons de curar;
\par 10 a outro, operações de milagres; a outro, profecia; a outro, discernimento de espíritos; a um, variedade de línguas; e a outro, capacidade para interpretá-las.
\par 11 Mas um só e o mesmo Espírito realiza todas estas coisas, distribuindo-as, como lhe apraz, a cada um, individualmente.
\par 12 Porque, assim como o corpo é um e tem muitos membros, e todos os membros, sendo muitos, constituem um só corpo, assim também com respeito a Cristo.
\par 13 Pois, em um só Espírito, todos nós fomos batizados em um corpo, quer judeus, quer gregos, quer escravos, quer livres. E a todos nós foi dado beber de um só Espírito.
\par 14 Porque também o corpo não é um só membro, mas muitos.
\par 15 Se disser o pé: Porque não sou mão, não sou do corpo; nem por isso deixa de ser do corpo.
\par 16 Se o ouvido disser: Porque não sou olho, não sou do corpo; nem por isso deixa de o ser.
\par 17 Se todo o corpo fosse olho, onde estaria o ouvido? Se todo fosse ouvido, onde, o olfato?
\par 18 Mas Deus dispôs os membros, colocando cada um deles no corpo, como lhe aprouve.
\par 19 Se todos, porém, fossem um só membro, onde estaria o corpo?
\par 20 O certo é que há muitos membros, mas um só corpo.
\par 21 Não podem os olhos dizer à mão: Não precisamos de ti; nem ainda a cabeça, aos pés: Não preciso de vós.
\par 22 Pelo contrário, os membros do corpo que parecem ser mais fracos são necessários;
\par 23 e os que nos parecem menos dignos no corpo, a estes damos muito maior honra; também os que em nós não são decorosos revestimos de especial honra.
\par 24 Mas os nossos membros nobres não têm necessidade disso. Contudo, Deus coordenou o corpo, concedendo muito mais honra àquilo que menos tinha,
\par 25 para que não haja divisão no corpo; pelo contrário, cooperem os membros, com igual cuidado, em favor uns dos outros.
\par 26 De maneira que, se um membro sofre, todos sofrem com ele; e, se um deles é honrado, com ele todos se regozijam.
\par 27 Ora, vós sois corpo de Cristo; e, individualmente, membros desse corpo.
\par 28 A uns estabeleceu Deus na igreja, primeiramente, apóstolos; em segundo lugar, profetas; em terceiro lugar, mestres; depois, operadores de milagres; depois, dons de curar, socorros, governos, variedades de línguas.
\par 29 Porventura, são todos apóstolos? Ou, todos profetas? São todos mestres? Ou, operadores de milagres?
\par 30 Têm todos dons de curar? Falam todos em outras línguas? Interpretam-nas todos?
\par 31 Entretanto, procurai, com zelo, os melhores dons. E eu passo a mostrar-vos ainda um caminho sobremodo excelente.

\chapter{13}

\par 1 Ainda que eu fale as línguas dos homens e dos anjos, se não tiver amor, serei como o bronze que soa ou como o címbalo que retine.
\par 2 Ainda que eu tenha o dom de profetizar e conheça todos os mistérios e toda a ciência; ainda que eu tenha tamanha fé, a ponto de transportar montes, se não tiver amor, nada serei.
\par 3 E ainda que eu distribua todos os meus bens entre os pobres e ainda que entregue o meu próprio corpo para ser queimado, se não tiver amor, nada disso me aproveitará.
\par 4 O amor é paciente, é benigno; o amor não arde em ciúmes, não se ufana, não se ensoberbece,
\par 5 não se conduz inconvenientemente, não procura os seus interesses, não se exaspera, não se ressente do mal;
\par 6 não se alegra com a injustiça, mas regozija-se com a verdade;
\par 7 tudo sofre, tudo crê, tudo espera, tudo suporta.
\par 8 O amor jamais acaba; mas, havendo profecias, desaparecerão; havendo línguas, cessarão; havendo ciência, passará;
\par 9 porque, em parte, conhecemos e, em parte, profetizamos.
\par 10 Quando, porém, vier o que é perfeito, então, o que é em parte será aniquilado.
\par 11 Quando eu era menino, falava como menino, sentia como menino, pensava como menino; quando cheguei a ser homem, desisti das coisas próprias de menino.
\par 12 Porque, agora, vemos como em espelho, obscuramente; então, veremos face a face. Agora, conheço em parte; então, conhecerei como também sou conhecido.
\par 13 Agora, pois, permanecem a fé, a esperança e o amor, estes três; porém o maior destes é o amor.

\chapter{14}

\par 1 Segui o amor e procurai, com zelo, os dons espirituais, mas principalmente que profetizeis.
\par 2 Pois quem fala em outra língua não fala a homens, senão a Deus, visto que ninguém o entende, e em espírito fala mistérios.
\par 3 Mas o que profetiza fala aos homens, edificando, exortando e consolando.
\par 4 O que fala em outra língua a si mesmo se edifica, mas o que profetiza edifica a igreja.
\par 5 Eu quisera que vós todos falásseis em outras línguas; muito mais, porém, que profetizásseis; pois quem profetiza é superior ao que fala em outras línguas, salvo se as interpretar, para que a igreja receba edificação.
\par 6 Agora, porém, irmãos, se eu for ter convosco falando em outras línguas, em que vos aproveitarei, se vos não falar por meio de revelação, ou de ciência, ou de profecia, ou de doutrina?
\par 7 É assim que instrumentos inanimados, como a flauta ou a cítara, quando emitem sons, se não os derem bem distintos, como se reconhecerá o que se toca na flauta ou cítara?
\par 8 Pois também se a trombeta der som incerto, quem se preparará para a batalha?
\par 9 Assim, vós, se, com a língua, não disserdes palavra compreensível, como se entenderá o que dizeis? Porque estareis como se falásseis ao ar.
\par 10 Há, sem dúvida, muitos tipos de vozes no mundo; nenhum deles, contudo, sem sentido.
\par 11 Se eu, pois, ignorar a significação da voz, serei estrangeiro para aquele que fala; e ele, estrangeiro para mim.
\par 12 Assim, também vós, visto que desejais dons espirituais, procurai progredir, para a edificação da igreja.
\par 13 Pelo que, o que fala em outra língua deve orar para que a possa interpretar.
\par 14 Porque, se eu orar em outra língua, o meu espírito ora de fato, mas a minha mente fica infrutífera.
\par 15 Que farei, pois? Orarei com o espírito, mas também orarei com a mente; cantarei com o espírito, mas também cantarei com a mente.
\par 16 E, se tu bendisseres apenas em espírito, como dirá o indouto o amém depois da tua ação de graças? Visto que não entende o que dizes;
\par 17 porque tu, de fato, dás bem as graças, mas o outro não é edificado.
\par 18 Dou graças a Deus, porque falo em outras línguas mais do que todos vós.
\par 19 Contudo, prefiro falar na igreja cinco palavras com o meu entendimento, para instruir outros, a falar dez mil palavras em outra língua.
\par 20 Irmãos, não sejais meninos no juízo; na malícia, sim, sede crianças; quanto ao juízo, sede homens amadurecidos.
\par 21 Na lei está escrito: Falarei a este povo por homens de outras línguas e por lábios de outros povos, e nem assim me ouvirão, diz o Senhor.
\par 22 De sorte que as línguas constituem um sinal não para os crentes, mas para os incrédulos; mas a profecia não é para os incrédulos, e sim para os que crêem.
\par 23 Se, pois, toda a igreja se reunir no mesmo lugar, e todos se puserem a falar em outras línguas, no caso de entrarem indoutos ou incrédulos, não dirão, porventura, que estais loucos?
\par 24 Porém, se todos profetizarem, e entrar algum incrédulo ou indouto, é ele por todos convencido e por todos julgado;
\par 25 tornam-se-lhe manifestos os segredos do coração, e, assim, prostrando-se com a face em terra, adorará a Deus, testemunhando que Deus está, de fato, no meio de vós.
\par 26 Que fazer, pois, irmãos? Quando vos reunis, um tem salmo, outro, doutrina, este traz revelação, aquele, outra língua, e ainda outro, interpretação. Seja tudo feito para edificação.
\par 27 No caso de alguém falar em outra língua, que não sejam mais do que dois ou quando muito três, e isto sucessivamente, e haja quem interprete.
\par 28 Mas, não havendo intérprete, fique calado na igreja, falando consigo mesmo e com Deus.
\par 29 Tratando-se de profetas, falem apenas dois ou três, e os outros julguem.
\par 30 Se, porém, vier revelação a outrem que esteja assentado, cale-se o primeiro.
\par 31 Porque todos podereis profetizar, um após outro, para todos aprenderem e serem consolados.
\par 32 Os espíritos dos profetas estão sujeitos aos próprios profetas;
\par 33 porque Deus não é de confusão, e sim de paz. Como em todas as igrejas dos santos,
\par 34 conservem-se as mulheres caladas nas igrejas, porque não lhes é permitido falar; mas estejam submissas como também a lei o determina.
\par 35 Se, porém, querem aprender alguma coisa, interroguem, em casa, a seu próprio marido; porque para a mulher é vergonhoso falar na igreja.
\par 36 Porventura, a palavra de Deus se originou no meio de vós ou veio ela exclusivamente para vós outros?
\par 37 Se alguém se considera profeta ou espiritual, reconheça ser mandamento do Senhor o que vos escrevo.
\par 38 E, se alguém o ignorar, será ignorado.
\par 39 Portanto, meus irmãos, procurai com zelo o dom de profetizar e não proibais o falar em outras línguas.
\par 40 Tudo, porém, seja feito com decência e ordem.

\chapter{15}

\par 1 Irmãos, venho lembrar-vos o evangelho que vos anunciei, o qual recebestes e no qual ainda perseverais;
\par 2 por ele também sois salvos, se retiverdes a palavra tal como vo-la preguei, a menos que tenhais crido em vão.
\par 3 Antes de tudo, vos entreguei o que também recebi: que Cristo morreu pelos nossos pecados, segundo as Escrituras,
\par 4 e que foi sepultado e ressuscitou ao terceiro dia, segundo as Escrituras.
\par 5 E apareceu a Cefas e, depois, aos doze.
\par 6 Depois, foi visto por mais de quinhentos irmãos de uma só vez, dos quais a maioria sobrevive até agora; porém alguns já dormem.
\par 7 Depois, foi visto por Tiago, mais tarde, por todos os apóstolos
\par 8 e, afinal, depois de todos, foi visto também por mim, como por um nascido fora de tempo.
\par 9 Porque eu sou o menor dos apóstolos, que mesmo não sou digno de ser chamado apóstolo, pois persegui a igreja de Deus.
\par 10 Mas, pela graça de Deus, sou o que sou; e a sua graça, que me foi concedida, não se tornou vã; antes, trabalhei muito mais do que todos eles; todavia, não eu, mas a graça de Deus comigo.
\par 11 Portanto, seja eu ou sejam eles, assim pregamos e assim crestes.
\par 12 Ora, se é corrente pregar-se que Cristo ressuscitou dentre os mortos, como, pois, afirmam alguns dentre vós que não há ressurreição de mortos?
\par 13 E, se não há ressurreição de mortos, então, Cristo não ressuscitou.
\par 14 E, se Cristo não ressuscitou, é vã a nossa pregação, e vã, a vossa fé;
\par 15 e somos tidos por falsas testemunhas de Deus, porque temos asseverado contra Deus que ele ressuscitou a Cristo, ao qual ele não ressuscitou, se é certo que os mortos não ressuscitam.
\par 16 Porque, se os mortos não ressuscitam, também Cristo não ressuscitou.
\par 17 E, se Cristo não ressuscitou, é vã a vossa fé, e ainda permaneceis nos vossos pecados.
\par 18 E ainda mais: os que dormiram em Cristo pereceram.
\par 19 Se a nossa esperança em Cristo se limita apenas a esta vida, somos os mais infelizes de todos os homens.
\par 20 Mas, de fato, Cristo ressuscitou dentre os mortos, sendo ele as primícias dos que dormem.
\par 21 Visto que a morte veio por um homem, também por um homem veio a ressurreição dos mortos.
\par 22 Porque, assim como, em Adão, todos morrem, assim também todos serão vivificados em Cristo.
\par 23 Cada um, porém, por sua própria ordem: Cristo, as primícias; depois, os que são de Cristo, na sua vinda.
\par 24 E, então, virá o fim, quando ele entregar o reino ao Deus e Pai, quando houver destruído todo principado, bem como toda potestade e poder.
\par 25 Porque convém que ele reine até que haja posto todos os inimigos debaixo dos pés.
\par 26 O último inimigo a ser destruído é a morte.
\par 27 Porque todas as coisas sujeitou debaixo dos pés. E, quando diz que todas as coisas lhe estão sujeitas, certamente, exclui aquele que tudo lhe subordinou.
\par 28 Quando, porém, todas as coisas lhe estiverem sujeitas, então, o próprio Filho também se sujeitará àquele que todas as coisas lhe sujeitou, para que Deus seja tudo em todos.
\par 29 Doutra maneira, que farão os que se batizam por causa dos mortos? Se, absolutamente, os mortos não ressuscitam, por que se batizam por causa deles?
\par 30 E por que também nós nos expomos a perigos a toda hora?
\par 31 Dia após dia, morro! Eu o protesto, irmãos, pela glória que tenho em vós outros, em Cristo Jesus, nosso Senhor.
\par 32 Se, como homem, lutei em Éfeso com feras, que me aproveita isso? Se os mortos não ressuscitam, comamos e bebamos, que amanhã morreremos.
\par 33 Não vos enganeis: as más conversações corrompem os bons costumes.
\par 34 Tornai-vos à sobriedade, como é justo, e não pequeis; porque alguns ainda não têm conhecimento de Deus; isto digo para vergonha vossa.
\par 35 Mas alguém dirá: Como ressuscitam os mortos? E em que corpo vêm?
\par 36 Insensato! O que semeias não nasce, se primeiro não morrer;
\par 37 e, quando semeias, não semeias o corpo que há de ser, mas o simples grão, como de trigo ou de qualquer outra semente.
\par 38 Mas Deus lhe dá corpo como lhe aprouve dar e a cada uma das sementes, o seu corpo apropriado.
\par 39 Nem toda carne é a mesma; porém uma é a carne dos homens, outra, a dos animais, outra, a das aves, e outra, a dos peixes.
\par 40 Também há corpos celestiais e corpos terrestres; e, sem dúvida, uma é a glória dos celestiais, e outra, a dos terrestres.
\par 41 Uma é a glória do sol, outra, a glória da lua, e outra, a das estrelas; porque até entre estrela e estrela há diferenças de esplendor.
\par 42 Pois assim também é a ressurreição dos mortos. Semeia-se o corpo na corrupção, ressuscita na incorrupção. Semeia-se em desonra, ressuscita em glória.
\par 43 Semeia-se em fraqueza, ressuscita em poder.
\par 44 Semeia-se corpo natural, ressuscita corpo espiritual. Se há corpo natural, há também corpo espiritual.
\par 45 Pois assim está escrito: O primeiro homem, Adão, foi feito alma vivente. O último Adão, porém, é espírito vivificante.
\par 46 Mas não é primeiro o espiritual, e sim o natural; depois, o espiritual.
\par 47 O primeiro homem, formado da terra, é terreno; o segundo homem é do céu.
\par 48 Como foi o primeiro homem, o terreno, tais são também os demais homens terrenos; e, como é o homem celestial, tais também os celestiais.
\par 49 E, assim como trouxemos a imagem do que é terreno, devemos trazer também a imagem do celestial.
\par 50 Isto afirmo, irmãos, que a carne e o sangue não podem herdar o reino de Deus, nem a corrupção herdar a incorrupção.
\par 51 Eis que vos digo um mistério: nem todos dormiremos, mas transformados seremos todos,
\par 52 num momento, num abrir e fechar de olhos, ao ressoar da última trombeta. A trombeta soará, os mortos ressuscitarão incorruptíveis, e nós seremos transformados.
\par 53 Porque é necessário que este corpo corruptível se revista da incorruptibilidade, e que o corpo mortal se revista da imortalidade.
\par 54 E, quando este corpo corruptível se revestir de incorruptibilidade, e o que é mortal se revestir de imortalidade, então, se cumprirá a palavra que está escrita: Tragada foi a morte pela vitória.
\par 55 Onde está, ó morte, a tua vitória? Onde está, ó morte, o teu aguilhão?
\par 56 O aguilhão da morte é o pecado, e a força do pecado é a lei.
\par 57 Graças a Deus, que nos dá a vitória por intermédio de nosso Senhor Jesus Cristo.
\par 58 Portanto, meus amados irmãos, sede firmes, inabaláveis e sempre abundantes na obra do Senhor, sabendo que, no Senhor, o vosso trabalho não é vão.

\chapter{16}

\par 1 Quanto à coleta para os santos, fazei vós também como ordenei às igrejas da Galácia.
\par 2 No primeiro dia da semana, cada um de vós ponha de parte, em casa, conforme a sua prosperidade, e vá juntando, para que se não façam coletas quando eu for.
\par 3 E, quando tiver chegado, enviarei, com cartas, para levarem as vossas dádivas a Jerusalém, aqueles que aprovardes.
\par 4 Se convier que eu também vá, eles irão comigo.
\par 5 Irei ter convosco por ocasião da minha passagem pela Macedônia, porque devo percorrer a Macedônia.
\par 6 E bem pode ser que convosco me demore ou mesmo passe o inverno, para que me encaminheis nas viagens que eu tenha de fazer.
\par 7 Porque não quero, agora, ver-vos apenas de passagem, pois espero permanecer convosco algum tempo, se o Senhor o permitir.
\par 8 Ficarei, porém, em Éfeso até ao Pentecostes;
\par 9 porque uma porta grande e oportuna para o trabalho se me abriu; e há muitos adversários.
\par 10 E, se Timóteo for, vede que esteja sem receio entre vós, porque trabalha na obra do Senhor, como também eu;
\par 11 ninguém, pois, o despreze. Mas encaminhai-o em paz, para que venha ter comigo, visto que o espero com os irmãos.
\par 12 Acerca do irmão Apolo, muito lhe tenho recomendado que fosse ter convosco em companhia dos irmãos, mas de modo algum era a vontade dele ir agora; irá, porém, quando se lhe deparar boa oportunidade.
\par 13 Sede vigilantes, permanecei firmes na fé, portai-vos varonilmente, fortalecei-vos.
\par 14 Todos os vossos atos sejam feitos com amor.
\par 15 E agora, irmãos, eu vos peço o seguinte (sabeis que a casa de Estéfanas são as primícias da Acaia e que se consagraram ao serviço dos santos):
\par 16 que também vos sujeiteis a esses tais, como também a todo aquele que é cooperador e obreiro.
\par 17 Alegro-me com a vinda de Estéfanas, e de Fortunato, e de Acaico; porque estes supriram o que da vossa parte faltava.
\par 18 Porque trouxeram refrigério ao meu espírito e ao vosso. Reconhecei, pois, a homens como estes.
\par 19 As igrejas da Ásia vos saúdam. No Senhor, muito vos saúdam Áqüila e Priscila e, bem assim, a igreja que está na casa deles.
\par 20 Todos os irmãos vos saúdam. Saudai-vos uns aos outros com ósculo santo.
\par 21 A saudação, escrevo-a eu, Paulo, de próprio punho.
\par 22 Se alguém não ama o Senhor, seja anátema. Maranata!
\par 23 A graça do Senhor Jesus seja convosco.
\par 24 O meu amor seja com todos vós, em Cristo Jesus.


\end{document}