\begin{document}

\title{Jonas}


\chapter{1}

\par 1 Veio a palavra do SENHOR a Jonas, filho de Amitai, dizendo:
\par 2 Dispõe-te, vai à grande cidade de Nínive e clama contra ela, porque a sua malícia subiu até mim.
\par 3 Jonas se dispôs, mas para fugir da presença do SENHOR, para Társis; e, tendo descido a Jope, achou um navio que ia para Társis; pagou, pois, a sua passagem e embarcou nele, para ir com eles para Társis, para longe da presença do SENHOR.
\par 4 Mas o SENHOR lançou sobre o mar um forte vento, e fez-se no mar uma grande tempestade, e o navio estava a ponto de se despedaçar.
\par 5 Então, os marinheiros, cheios de medo, clamavam cada um ao seu deus e lançavam ao mar a carga que estava no navio, para o aliviarem do peso dela. Jonas, porém, havia descido ao porão e se deitado; e dormia profundamente.
\par 6 Chegou-se a ele o mestre do navio e lhe disse: Que se passa contigo? Agarrado no sono? Levanta-te, invoca o teu deus; talvez, assim, esse deus se lembre de nós, para que não pereçamos.
\par 7 E diziam uns aos outros: Vinde, e lancemos sortes, para que saibamos por causa de quem nos sobreveio este mal. E lançaram sortes, e a sorte caiu sobre Jonas.
\par 8 Então, lhe disseram: Declara-nos, agora, por causa de quem nos sobreveio este mal. Que ocupação é a tua? Donde vens? Qual a tua terra? E de que povo és tu?
\par 9 Ele lhes respondeu: Sou hebreu e temo ao SENHOR, o Deus do céu, que fez o mar e a terra.
\par 10 Então, os homens ficaram possuídos de grande temor e lhe disseram: Que é isto que fizeste! Pois sabiam os homens que ele fugia da presença do SENHOR, porque lho havia declarado.
\par 11 Disseram-lhe: Que te faremos, para que o mar se nos acalme? Porque o mar se ia tornando cada vez mais tempestuoso.
\par 12 Respondeu-lhes: Tomai-me e lançai-me ao mar, e o mar se aquietará, porque eu sei que, por minha causa, vos sobreveio esta grande tempestade.
\par 13 Entretanto, os homens remavam, esforçando-se por alcançar a terra, mas não podiam, porquanto o mar se ia tornando cada vez mais tempestuoso contra eles.
\par 14 Então, clamaram ao SENHOR e disseram: Ah! SENHOR! Rogamos-te que não pereçamos por causa da vida deste homem, e não faças cair sobre nós este sangue, quanto a nós, inocente; porque tu, SENHOR, fizeste como te aprouve.
\par 15 E levantaram a Jonas e o lançaram ao mar; e cessou o mar da sua fúria.
\par 16 Temeram, pois, estes homens em extremo ao SENHOR; e ofereceram sacrifícios ao SENHOR e fizeram votos.
\par 17 Deparou o SENHOR um grande peixe, para que tragasse a Jonas; e esteve Jonas três dias e três noites no ventre do peixe.

\chapter{2}

\par 1 Então, Jonas, do ventre do peixe, orou ao SENHOR, seu Deus,
\par 2 e disse: Na minha angústia, clamei ao SENHOR, e ele me respondeu; do ventre do abismo, gritei, e tu me ouviste a voz.
\par 3 Pois me lançaste no profundo, no coração dos mares, e a corrente das águas me cercou; todas as tuas ondas e as tuas vagas passaram por cima de mim.
\par 4 Então, eu disse: lançado estou de diante dos teus olhos; tornarei, porventura, a ver o teu santo templo?
\par 5 As águas me cercaram até à alma, o abismo me rodeou; e as algas se enrolaram na minha cabeça.
\par 6 Desci até aos fundamentos dos montes, desci até à terra, cujos ferrolhos se correram sobre mim, para sempre; contudo, fizeste subir da sepultura a minha vida, ó SENHOR, meu Deus!
\par 7 Quando, dentro de mim, desfalecia a minha alma, eu me lembrei do SENHOR; e subiu a ti a minha oração, no teu santo templo.
\par 8 Os que se entregam à idolatria vã abandonam aquele que lhes é misericordioso.
\par 9 Mas, com a voz do agradecimento, eu te oferecerei sacrifício; o que votei pagarei. Ao SENHOR pertence a salvação!
\par 10 Falou, pois, o SENHOR ao peixe, e este vomitou a Jonas na terra.

\chapter{3}

\par 1 Veio a palavra do SENHOR, segunda vez, a Jonas, dizendo:
\par 2 Dispõe-te, vai à grande cidade de Nínive e proclama contra ela a mensagem que eu te digo.
\par 3 Levantou-se, pois, Jonas e foi a Nínive, segundo a palavra do SENHOR. Ora, Nínive era cidade mui importante diante de Deus e de três dias para percorrê-la.
\par 4 Começou Jonas a percorrer a cidade caminho de um dia, e pregava, e dizia: Ainda quarenta dias, e Nínive será subvertida.
\par 5 Os ninivitas creram em Deus, e proclamaram um jejum, e vestiram-se de panos de saco, desde o maior até o menor.
\par 6 Chegou esta notícia ao rei de Nínive; ele levantou-se do seu trono, tirou de si as vestes reais, cobriu-se de pano de saco e assentou-se sobre cinza.
\par 7 E fez-se proclamar e divulgar em Nínive: Por mandado do rei e seus grandes, nem homens, nem animais, nem bois, nem ovelhas provem coisa alguma, nem os levem ao pasto, nem bebam água;
\par 8 mas sejam cobertos de pano de saco, tanto os homens como os animais, e clamarão fortemente a Deus; e se converterão, cada um do seu mau caminho e da violência que há nas suas mãos.
\par 9 Quem sabe se voltará Deus, e se arrependerá, e se apartará do furor da sua ira, de sorte que não pereçamos?
\par 10 Viu Deus o que fizeram, como se converteram do seu mau caminho; e Deus se arrependeu do mal que tinha dito lhes faria e não o fez.

\chapter{4}

\par 1 Com isso, desgostou-se Jonas extremamente e ficou irado.
\par 2 E orou ao SENHOR e disse: Ah! SENHOR! Não foi isso o que eu disse, estando ainda na minha terra? Por isso, me adiantei, fugindo para Társis, pois sabia que és Deus clemente, e misericordioso, e tardio em irar-se, e grande em benignidade, e que te arrependes do mal.
\par 3 Peço-te, pois, ó SENHOR, tira-me a vida, porque melhor me é morrer do que viver.
\par 4 E disse o SENHOR: É razoável essa tua ira?
\par 5 Então, Jonas saiu da cidade, e assentou-se ao oriente da mesma, e ali fez uma enramada, e repousou debaixo dela, à sombra, até ver o que aconteceria à cidade.
\par 6 Então, fez o SENHOR Deus nascer uma planta, que subiu por cima de Jonas, para que fizesse sombra sobre a sua cabeça, a fim de o livrar do seu desconforto. Jonas, pois, se alegrou em extremo por causa da planta.
\par 7 Mas Deus, no dia seguinte, ao subir da alva, enviou um verme, o qual feriu a planta, e esta se secou.
\par 8 Em nascendo o sol, Deus mandou um vento calmoso oriental; o sol bateu na cabeça de Jonas, de maneira que desfalecia, pelo que pediu para si a morte, dizendo: Melhor me é morrer do que viver!
\par 9 Então, perguntou Deus a Jonas: É razoável essa tua ira por causa da planta? Ele respondeu: É razoável a minha ira até à morte.
\par 10 Tornou o SENHOR: Tens compaixão da planta que te não custou trabalho, a qual não fizeste crescer, que numa noite nasceu e numa noite pereceu;
\par 11 e não hei de eu ter compaixão da grande cidade de Nínive, em que há mais de cento e vinte mil pessoas, que não sabem discernir entre a mão direita e a mão esquerda, e também muitos animais?


\end{document}