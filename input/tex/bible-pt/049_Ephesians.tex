\begin{document}

\title{Efésios}


\chapter{1}

\par 1 Paulo, apóstolo de Cristo Jesus por vontade de Deus, aos santos que vivem em Éfeso e fiéis em Cristo Jesus,
\par 2 graça a vós outros e paz, da parte de Deus, nosso Pai, e do Senhor Jesus Cristo.
\par 3 Bendito o Deus e Pai de nosso Senhor Jesus Cristo, que nos tem abençoado com toda sorte de bênção espiritual nas regiões celestiais em Cristo,
\par 4 assim como nos escolheu, nele, antes da fundação do mundo, para sermos santos e irrepreensíveis perante ele; e em amor
\par 5 nos predestinou para ele, para a adoção de filhos, por meio de Jesus Cristo, segundo o beneplácito de sua vontade,
\par 6 para louvor da glória de sua graça, que ele nos concedeu gratuitamente no Amado,
\par 7 no qual temos a redenção, pelo seu sangue, a remissão dos pecados, segundo a riqueza da sua graça,
\par 8 que Deus derramou abundantemente sobre nós em toda a sabedoria e prudência,
\par 9 desvendando-nos o mistério da sua vontade, segundo o seu beneplácito que propusera em Cristo,
\par 10 de fazer convergir nele, na dispensação da plenitude dos tempos, todas as coisas, tanto as do céu como as da terra;
\par 11 nele, digo, no qual fomos também feitos herança, predestinados segundo o propósito daquele que faz todas as coisas conforme o conselho da sua vontade,
\par 12 a fim de sermos para louvor da sua glória, nós, os que de antemão esperamos em Cristo;
\par 13 em quem também vós, depois que ouvistes a palavra da verdade, o evangelho da vossa salvação, tendo nele também crido, fostes selados com o Santo Espírito da promessa;
\par 14 o qual é o penhor da nossa herança, até ao resgate da sua propriedade, em louvor da sua glória.
\par 15 Por isso, também eu, tendo ouvido a fé que há entre vós no Senhor Jesus e o amor para com todos os santos,
\par 16 não cesso de dar graças por vós, fazendo menção de vós nas minhas orações,
\par 17 para que o Deus de nosso Senhor Jesus Cristo, o Pai da glória, vos conceda espírito de sabedoria e de revelação no pleno conhecimento dele,
\par 18 iluminados os olhos do vosso coração, para saberdes qual é a esperança do seu chamamento, qual a riqueza da glória da sua herança nos santos
\par 19 e qual a suprema grandeza do seu poder para com os que cremos, segundo a eficácia da força do seu poder;
\par 20 o qual exerceu ele em Cristo, ressuscitando-o dentre os mortos e fazendo-o sentar à sua direita nos lugares celestiais,
\par 21 acima de todo principado, e potestade, e poder, e domínio, e de todo nome que se possa referir não só no presente século, mas também no vindouro.
\par 22 E pôs todas as coisas debaixo dos pés e, para ser o cabeça sobre todas as coisas, o deu à igreja,
\par 23 a qual é o seu corpo, a plenitude daquele que a tudo enche em todas as coisas.

\chapter{2}

\par 1 Ele vos deu vida, estando vós mortos nos vossos delitos e pecados,
\par 2 nos quais andastes outrora, segundo o curso deste mundo, segundo o príncipe da potestade do ar, do espírito que agora atua nos filhos da desobediência;
\par 3 entre os quais também todos nós andamos outrora, segundo as inclinações da nossa carne, fazendo a vontade da carne e dos pensamentos; e éramos, por natureza, filhos da ira, como também os demais.
\par 4 Mas Deus, sendo rico em misericórdia, por causa do grande amor com que nos amou,
\par 5 e estando nós mortos em nossos delitos, nos deu vida juntamente com Cristo, -- pela graça sois salvos,
\par 6 e, juntamente com ele, nos ressuscitou, e nos fez assentar nos lugares celestiais em Cristo Jesus;
\par 7 para mostrar, nos séculos vindouros, a suprema riqueza da sua graça, em bondade para conosco, em Cristo Jesus.
\par 8 Porque pela graça sois salvos, mediante a fé; e isto não vem de vós; é dom de Deus;
\par 9 não de obras, para que ninguém se glorie.
\par 10 Pois somos feitura dele, criados em Cristo Jesus para boas obras, as quais Deus de antemão preparou para que andássemos nelas.
\par 11 Portanto, lembrai-vos de que, outrora, vós, gentios na carne, chamados incircuncisão por aqueles que se intitulam circuncisos, na carne, por mãos humanas,
\par 12 naquele tempo, estáveis sem Cristo, separados da comunidade de Israel e estranhos às alianças da promessa, não tendo esperança e sem Deus no mundo.
\par 13 Mas, agora, em Cristo Jesus, vós, que antes estáveis longe, fostes aproximados pelo sangue de Cristo.
\par 14 Porque ele é a nossa paz, o qual de ambos fez um; e, tendo derribado a parede da separação que estava no meio, a inimizade,
\par 15 aboliu, na sua carne, a lei dos mandamentos na forma de ordenanças, para que dos dois criasse, em si mesmo, um novo homem, fazendo a paz,
\par 16 e reconciliasse ambos em um só corpo com Deus, por intermédio da cruz, destruindo por ela a inimizade.
\par 17 E, vindo, evangelizou paz a vós outros que estáveis longe e paz também aos que estavam perto;
\par 18 porque, por ele, ambos temos acesso ao Pai em um Espírito.
\par 19 Assim, já não sois estrangeiros e peregrinos, mas concidadãos dos santos, e sois da família de Deus,
\par 20 edificados sobre o fundamento dos apóstolos e profetas, sendo ele mesmo, Cristo Jesus, a pedra angular;
\par 21 no qual todo o edifício, bem ajustado, cresce para santuário dedicado ao Senhor,
\par 22 no qual também vós juntamente estais sendo edificados para habitação de Deus no Espírito.

\chapter{3}

\par 1 Por esta causa eu, Paulo, sou o prisioneiro de Cristo Jesus, por amor de vós, gentios,
\par 2 se é que tendes ouvido a respeito da dispensação da graça de Deus a mim confiada para vós outros;
\par 3 pois, segundo uma revelação, me foi dado conhecer o mistério, conforme escrevi há pouco, resumidamente;
\par 4 pelo que, quando ledes, podeis compreender o meu discernimento do mistério de Cristo,
\par 5 o qual, em outras gerações, não foi dado a conhecer aos filhos dos homens, como, agora, foi revelado aos seus santos apóstolos e profetas, no Espírito,
\par 6 a saber, que os gentios são co-herdeiros, membros do mesmo corpo e co-participantes da promessa em Cristo Jesus por meio do evangelho;
\par 7 do qual fui constituído ministro conforme o dom da graça de Deus a mim concedida segundo a força operante do seu poder.
\par 8 A mim, o menor de todos os santos, me foi dada esta graça de pregar aos gentios o evangelho das insondáveis riquezas de Cristo
\par 9 e manifestar qual seja a dispensação do mistério, desde os séculos, oculto em Deus, que criou todas as coisas,
\par 10 para que, pela igreja, a multiforme sabedoria de Deus se torne conhecida, agora, dos principados e potestades nos lugares celestiais,
\par 11 segundo o eterno propósito que estabeleceu em Cristo Jesus, nosso Senhor,
\par 12 pelo qual temos ousadia e acesso com confiança, mediante a fé nele.
\par 13 Portanto, vos peço que não desfaleçais nas minhas tribulações por vós, pois nisso está a vossa glória.
\par 14 Por esta causa, me ponho de joelhos diante do Pai,
\par 15 de quem toma o nome toda família, tanto no céu como sobre a terra,
\par 16 para que, segundo a riqueza da sua glória, vos conceda que sejais fortalecidos com poder, mediante o seu Espírito no homem interior;
\par 17 e, assim, habite Cristo no vosso coração, pela fé, estando vós arraigados e alicerçados em amor,
\par 18 a fim de poderdes compreender, com todos os santos, qual é a largura, e o comprimento, e a altura, e a profundidade
\par 19 e conhecer o amor de Cristo, que excede todo entendimento, para que sejais tomados de toda a plenitude de Deus.
\par 20 Ora, àquele que é poderoso para fazer infinitamente mais do que tudo quanto pedimos ou pensamos, conforme o seu poder que opera em nós,
\par 21 a ele seja a glória, na igreja e em Cristo Jesus, por todas as gerações, para todo o sempre. Amém!

\chapter{4}

\par 1 Rogo-vos, pois, eu, o prisioneiro no Senhor, que andeis de modo digno da vocação a que fostes chamados,
\par 2 com toda a humildade e mansidão, com longanimidade, suportando-vos uns aos outros em amor,
\par 3 esforçando-vos diligentemente por preservar a unidade do Espírito no vínculo da paz;
\par 4 há somente um corpo e um Espírito, como também fostes chamados numa só esperança da vossa vocação;
\par 5 há um só Senhor, uma só fé, um só batismo;
\par 6 um só Deus e Pai de todos, o qual é sobre todos, age por meio de todos e está em todos.
\par 7 E a graça foi concedida a cada um de nós segundo a proporção do dom de Cristo.
\par 8 Por isso, diz: Quando ele subiu às alturas, levou cativo o cativeiro e concedeu dons aos homens.
\par 9 Ora, que quer dizer subiu, senão que também havia descido até às regiões inferiores da terra?
\par 10 Aquele que desceu é também o mesmo que subiu acima de todos os céus, para encher todas as coisas.
\par 11 E ele mesmo concedeu uns para apóstolos, outros para profetas, outros para evangelistas e outros para pastores e mestres,
\par 12 com vistas ao aperfeiçoamento dos santos para o desempenho do seu serviço, para a edificação do corpo de Cristo,
\par 13 até que todos cheguemos à unidade da fé e do pleno conhecimento do Filho de Deus, à perfeita varonilidade, à medida da estatura da plenitude de Cristo,
\par 14 para que não mais sejamos como meninos, agitados de um lado para outro e levados ao redor por todo vento de doutrina, pela artimanha dos homens, pela astúcia com que induzem ao erro.
\par 15 Mas, seguindo a verdade em amor, cresçamos em tudo naquele que é a cabeça, Cristo,
\par 16 de quem todo o corpo, bem ajustado e consolidado pelo auxílio de toda junta, segundo a justa cooperação de cada parte, efetua o seu próprio aumento para a edificação de si mesmo em amor.
\par 17 Isto, portanto, digo e no Senhor testifico que não mais andeis como também andam os gentios, na vaidade dos seus próprios pensamentos,
\par 18 obscurecidos de entendimento, alheios à vida de Deus por causa da ignorância em que vivem, pela dureza do seu coração,
\par 19 os quais, tendo-se tornado insensíveis, se entregaram à dissolução para, com avidez, cometerem toda sorte de impureza.
\par 20 Mas não foi assim que aprendestes a Cristo,
\par 21 se é que, de fato, o tendes ouvido e nele fostes instruídos, segundo é a verdade em Jesus,
\par 22 no sentido de que, quanto ao trato passado, vos despojeis do velho homem, que se corrompe segundo as concupiscências do engano,
\par 23 e vos renoveis no espírito do vosso entendimento,
\par 24 e vos revistais do novo homem, criado segundo Deus, em justiça e retidão procedentes da verdade.
\par 25 Por isso, deixando a mentira, fale cada um a verdade com o seu próximo, porque somos membros uns dos outros.
\par 26 Irai-vos e não pequeis; não se ponha o sol sobre a vossa ira,
\par 27 nem deis lugar ao diabo.
\par 28 Aquele que furtava não furte mais; antes, trabalhe, fazendo com as próprias mãos o que é bom, para que tenha com que acudir ao necessitado.
\par 29 Não saia da vossa boca nenhuma palavra torpe, e sim unicamente a que for boa para edificação, conforme a necessidade, e, assim, transmita graça aos que ouvem.
\par 30 E não entristeçais o Espírito de Deus, no qual fostes selados para o dia da redenção.
\par 31 Longe de vós, toda amargura, e cólera, e ira, e gritaria, e blasfêmias, e bem assim toda malícia.
\par 32 Antes, sede uns para com os outros benignos, compassivos, perdoando-vos uns aos outros, como também Deus, em Cristo, vos perdoou.

\chapter{5}

\par 1 Sede, pois, imitadores de Deus, como filhos amados;
\par 2 e andai em amor, como também Cristo nos amou e se entregou a si mesmo por nós, como oferta e sacrifício a Deus, em aroma suave.
\par 3 Mas a impudicícia e toda sorte de impurezas ou cobiça nem sequer se nomeiem entre vós, como convém a santos;
\par 4 nem conversação torpe, nem palavras vãs ou chocarrices, coisas essas inconvenientes; antes, pelo contrário, ações de graças.
\par 5 Sabei, pois, isto: nenhum incontinente, ou impuro, ou avarento, que é idólatra, tem herança no reino de Cristo e de Deus.
\par 6 Ninguém vos engane com palavras vãs; porque, por essas coisas, vem a ira de Deus sobre os filhos da desobediência.
\par 7 Portanto, não sejais participantes com eles.
\par 8 Pois, outrora, éreis trevas, porém, agora, sois luz no Senhor; andai como filhos da luz
\par 9 (porque o fruto da luz consiste em toda bondade, e justiça, e verdade),
\par 10 provando sempre o que é agradável ao Senhor.
\par 11 E não sejais cúmplices nas obras infrutíferas das trevas; antes, porém, reprovai-as.
\par 12 Porque o que eles fazem em oculto, o só referir é vergonha.
\par 13 Mas todas as coisas, quando reprovadas pela luz, se tornam manifestas; porque tudo que se manifesta é luz.
\par 14 Pelo que diz: Desperta, ó tu que dormes, levanta-te de entre os mortos, e Cristo te iluminará.
\par 15 Portanto, vede prudentemente como andais, não como néscios, e sim como sábios,
\par 16 remindo o tempo, porque os dias são maus.
\par 17 Por esta razão, não vos torneis insensatos, mas procurai compreender qual a vontade do Senhor.
\par 18 E não vos embriagueis com vinho, no qual há dissolução, mas enchei-vos do Espírito,
\par 19 falando entre vós com salmos, entoando e louvando de coração ao Senhor com hinos e cânticos espirituais,
\par 20 dando sempre graças por tudo a nosso Deus e Pai, em nome de nosso Senhor Jesus Cristo,
\par 21 sujeitando-vos uns aos outros no temor de Cristo.
\par 22 As mulheres sejam submissas ao seu próprio marido, como ao Senhor;
\par 23 porque o marido é o cabeça da mulher, como também Cristo é o cabeça da igreja, sendo este mesmo o salvador do corpo.
\par 24 Como, porém, a igreja está sujeita a Cristo, assim também as mulheres sejam em tudo submissas ao seu marido.
\par 25 Maridos, amai vossa mulher, como também Cristo amou a igreja e a si mesmo se entregou por ela,
\par 26 para que a santificasse, tendo-a purificado por meio da lavagem de água pela palavra,
\par 27 para a apresentar a si mesmo igreja gloriosa, sem mácula, nem ruga, nem coisa semelhante, porém santa e sem defeito.
\par 28 Assim também os maridos devem amar a sua mulher como ao próprio corpo. Quem ama a esposa a si mesmo se ama.
\par 29 Porque ninguém jamais odiou a própria carne; antes, a alimenta e dela cuida, como também Cristo o faz com a igreja;
\par 30 porque somos membros do seu corpo.
\par 31 Eis por que deixará o homem a seu pai e a sua mãe e se unirá à sua mulher, e se tornarão os dois uma só carne.
\par 32 Grande é este mistério, mas eu me refiro a Cristo e à igreja.
\par 33 Não obstante, vós, cada um de per si também ame a própria esposa como a si mesmo, e a esposa respeite ao marido.

\chapter{6}

\par 1 Filhos, obedecei a vossos pais no Senhor, pois isto é justo.
\par 2 Honra a teu pai e a tua mãe (que é o primeiro mandamento com promessa),
\par 3 para que te vá bem, e sejas de longa vida sobre a terra.
\par 4 E vós, pais, não provoqueis vossos filhos à ira, mas criai-os na disciplina e na admoestação do Senhor.
\par 5 Quanto a vós outros, servos, obedecei a vosso senhor segundo a carne com temor e tremor, na sinceridade do vosso coração, como a Cristo,
\par 6 não servindo à vista, como para agradar a homens, mas como servos de Cristo, fazendo, de coração, a vontade de Deus;
\par 7 servindo de boa vontade, como ao Senhor e não como a homens,
\par 8 certos de que cada um, se fizer alguma coisa boa, receberá isso outra vez do Senhor, quer seja servo, quer livre.
\par 9 E vós, senhores, de igual modo procedei para com eles, deixando as ameaças, sabendo que o Senhor, tanto deles como vosso, está nos céus e que para com ele não há acepção de pessoas.
\par 10 Quanto ao mais, sede fortalecidos no Senhor e na força do seu poder.
\par 11 Revesti-vos de toda a armadura de Deus, para poderdes ficar firmes contra as ciladas do diabo;
\par 12 porque a nossa luta não é contra o sangue e a carne, e sim contra os principados e potestades, contra os dominadores deste mundo tenebroso, contra as forças espirituais do mal, nas regiões celestes.
\par 13 Portanto, tomai toda a armadura de Deus, para que possais resistir no dia mau e, depois de terdes vencido tudo, permanecer inabaláveis.
\par 14 Estai, pois, firmes, cingindo-vos com a verdade e vestindo-vos da couraça da justiça.
\par 15 Calçai os pés com a preparação do evangelho da paz;
\par 16 embraçando sempre o escudo da fé, com o qual podereis apagar todos os dardos inflamados do Maligno.
\par 17 Tomai também o capacete da salvação e a espada do Espírito, que é a palavra de Deus;
\par 18 com toda oração e súplica, orando em todo tempo no Espírito e para isto vigiando com toda perseverança e súplica por todos os santos
\par 19 e também por mim; para que me seja dada, no abrir da minha boca, a palavra, para, com intrepidez, fazer conhecido o mistério do evangelho,
\par 20 pelo qual sou embaixador em cadeias, para que, em Cristo, eu seja ousado para falar, como me cumpre fazê-lo.
\par 21 E, para que saibais também a meu respeito e o que faço, de tudo vos informará Tíquico, o irmão amado e fiel ministro do Senhor.
\par 22 Foi para isso que eu vo-lo enviei, para que saibais a nosso respeito, e ele console o vosso coração.
\par 23 Paz seja com os irmãos e amor com fé, da parte de Deus Pai e do Senhor Jesus Cristo.
\par 24 A graça seja com todos os que amam sinceramente a nosso Senhor Jesus Cristo.


\end{document}