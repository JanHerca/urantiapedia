\begin{document}

\title{Mateus}


\chapter{1}

\par 1 Livro da genealogia de Jesus Cristo, filho de Davi, filho de Abraão.
\par 2 Abraão gerou a Isaque; Isaque, a Jacó; Jacó, a Judá e a seus irmãos;
\par 3 Judá gerou de Tamar a Perez e a Zera; Perez gerou a Esrom; Esrom, a Arão;
\par 4 Arão gerou a Aminadabe; Aminadabe, a Naassom; Naassom, a Salmom;
\par 5 Salmom gerou de Raabe a Boaz; este, de Rute, gerou a Obede; e Obede, a Jessé;
\par 6 Jessé gerou ao rei Davi; e o rei Davi, a Salomão, da que fora mulher de Urias;
\par 7 Salomão gerou a Roboão; Roboão, a Abias; Abias, a Asa;
\par 8 Asa gerou a Josafá; Josafá, a Jorão; Jorão, a Uzias;
\par 9 Uzias gerou a Jotão; Jotão, a Acaz; Acaz, a Ezequias;
\par 10 Ezequias gerou a Manassés; Manassés, a Amom; Amom, a Josias;
\par 11 Josias gerou a Jeconias e a seus irmãos, no tempo do exílio na Babilônia.
\par 12 Depois do exílio na Babilônia, Jeconias gerou a Salatiel; e Salatiel, a Zorobabel;
\par 13 Zorobabel gerou a Abiúde; Abiúde, a Eliaquim; Eliaquim, a Azor;
\par 14 Azor gerou a Sadoque; Sadoque, a Aquim; Aquim, a Eliúde;
\par 15 Eliúde gerou a Eleazar; Eleazar, a Matã; Matã, a Jacó.
\par 16 E Jacó gerou a José, marido de Maria, da qual nasceu Jesus, que se chama o Cristo.
\par 17 De sorte que todas as gerações, desde Abraão até Davi, são catorze; desde Davi até ao exílio na Babilônia, catorze; e desde o exílio na Babilônia até Cristo, catorze.
\par 18 Ora, o nascimento de Jesus Cristo foi assim: estando Maria, sua mãe, desposada com José, sem que tivessem antes coabitado, achou-se grávida pelo Espírito Santo.
\par 19 Mas José, seu esposo, sendo justo e não a querendo infamar, resolveu deixá-la secretamente.
\par 20 Enquanto ponderava nestas coisas, eis que lhe apareceu, em sonho, um anjo do Senhor, dizendo: José, filho de Davi, não temas receber Maria, tua mulher, porque o que nela foi gerado é do Espírito Santo.
\par 21 Ela dará à luz um filho e lhe porás o nome de Jesus, porque ele salvará o seu povo dos pecados deles.
\par 22 Ora, tudo isto aconteceu para que se cumprisse o que fora dito pelo Senhor por intermédio do profeta:
\par 23 Eis que a virgem conceberá e dará à luz um filho, e ele será chamado pelo nome de Emanuel (que quer dizer: Deus conosco).
\par 24 Despertado José do sono, fez como lhe ordenara o anjo do Senhor e recebeu sua mulher.
\par 25 Contudo, não a conheceu, enquanto ela não deu à luz um filho, a quem pôs o nome de Jesus.

\chapter{2}

\par 1 Tendo Jesus nascido em Belém da Judéia, em dias do rei Herodes, eis que vieram uns magos do Oriente a Jerusalém.
\par 2 E perguntavam: Onde está o recém-nascido Rei dos judeus? Porque vimos a sua estrela no Oriente e viemos para adorá-lo.
\par 3 Tendo ouvido isso, alarmou-se o rei Herodes, e, com ele, toda a Jerusalém;
\par 4 então, convocando todos os principais sacerdotes e escribas do povo, indagava deles onde o Cristo deveria nascer.
\par 5 Em Belém da Judéia, responderam eles, porque assim está escrito por intermédio do profeta:
\par 6 E tu, Belém, terra de Judá, não és de modo algum a menor entre as principais de Judá; porque de ti sairá o Guia que há de apascentar a meu povo, Israel.
\par 7 Com isto, Herodes, tendo chamado secretamente os magos, inquiriu deles com precisão quanto ao tempo em que a estrela aparecera.
\par 8 E, enviando-os a Belém, disse-lhes: Ide informar-vos cuidadosamente a respeito do menino; e, quando o tiverdes encontrado, avisai-me, para eu também ir adorá-lo.
\par 9 Depois de ouvirem o rei, partiram; e eis que a estrela que viram no Oriente os precedia, até que, chegando, parou sobre onde estava o menino.
\par 10 E, vendo eles a estrela, alegraram-se com grande e intenso júbilo.
\par 11 Entrando na casa, viram o menino com Maria, sua mãe. Prostrando-se, o adoraram; e, abrindo os seus tesouros, entregaram-lhe suas ofertas: ouro, incenso e mirra.
\par 12 Sendo por divina advertência prevenidos em sonho para não voltarem à presença de Herodes, regressaram por outro caminho a sua terra.
\par 13 Tendo eles partido, eis que apareceu um anjo do Senhor a José, em sonho, e disse: Dispõe-te, toma o menino e sua mãe, foge para o Egito e permanece lá até que eu te avise; porque Herodes há de procurar o menino para o matar.
\par 14 Dispondo-se ele, tomou de noite o menino e sua mãe e partiu para o Egito;
\par 15 e lá ficou até à morte de Herodes, para que se cumprisse o que fora dito pelo Senhor, por intermédio do profeta: Do Egito chamei o meu Filho.
\par 16 Vendo-se iludido pelos magos, enfureceu-se Herodes grandemente e mandou matar todos os meninos de Belém e de todos os seus arredores, de dois anos para baixo, conforme o tempo do qual com precisão se informara dos magos.
\par 17 Então, se cumpriu o que fora dito por intermédio do profeta Jeremias:
\par 18 Ouviu-se um clamor em Ramá, pranto, [choro] e grande lamento; era Raquel chorando por seus filhos e inconsolável porque não mais existem.
\par 19 Tendo Herodes morrido, eis que um anjo do Senhor apareceu em sonho a José, no Egito, e disse-lhe:
\par 20 Dispõe-te, toma o menino e sua mãe e vai para a terra de Israel; porque já morreram os que atentavam contra a vida do menino.
\par 21 Dispôs-se ele, tomou o menino e sua mãe e regressou para a terra de Israel.
\par 22 Tendo, porém, ouvido que Arquelau reinava na Judéia em lugar de seu pai Herodes, temeu ir para lá; e, por divina advertência prevenido em sonho, retirou-se para as regiões da Galiléia.
\par 23 E foi habitar numa cidade chamada Nazaré, para que se cumprisse o que fora dito por intermédio dos profetas: Ele será chamado Nazareno.

\chapter{3}

\par 1 Naqueles dias, apareceu João Batista pregando no deserto da Judéia e dizia:
\par 2 Arrependei-vos, porque está próximo o reino dos céus.
\par 3 Porque este é o referido por intermédio do profeta Isaías: Voz do que clama no deserto: Preparai o caminho do Senhor, endireitai as suas veredas.
\par 4 Usava João vestes de pêlos de camelo e um cinto de couro; a sua alimentação eram gafanhotos e mel silvestre.
\par 5 Então, saíam a ter com ele Jerusalém, toda a Judéia e toda a circunvizinhança do Jordão;
\par 6 e eram por ele batizados no rio Jordão, confessando os seus pecados.
\par 7 Vendo ele, porém, que muitos fariseus e saduceus vinham ao batismo, disse-lhes: Raça de víboras, quem vos induziu a fugir da ira vindoura?
\par 8 Produzi, pois, frutos dignos de arrependimento;
\par 9 e não comeceis a dizer entre vós mesmos: Temos por pai a Abraão; porque eu vos afirmo que destas pedras Deus pode suscitar filhos a Abraão.
\par 10 Já está posto o machado à raiz das árvores; toda árvore, pois, que não produz bom fruto é cortada e lançada ao fogo.
\par 11 Eu vos batizo com água, para arrependimento; mas aquele que vem depois de mim é mais poderoso do que eu, cujas sandálias não sou digno de levar. Ele vos batizará com o Espírito Santo e com fogo.
\par 12 A sua pá, ele a tem na mão e limpará completamente a sua eira; recolherá o seu trigo no celeiro, mas queimará a palha em fogo inextinguível.
\par 13 Por esse tempo, dirigiu-se Jesus da Galiléia para o Jordão, a fim de que João o batizasse.
\par 14 Ele, porém, o dissuadia, dizendo: Eu é que preciso ser batizado por ti, e tu vens a mim?
\par 15 Mas Jesus lhe respondeu: Deixa por enquanto, porque, assim, nos convém cumprir toda a justiça. Então, ele o admitiu.
\par 16 Batizado Jesus, saiu logo da água, e eis que se lhe abriram os céus, e viu o Espírito de Deus descendo como pomba, vindo sobre ele.
\par 17 E eis uma voz dos céus, que dizia: Este é o meu Filho amado, em quem me comprazo.

\chapter{4}

\par 1 A seguir, foi Jesus levado pelo Espírito ao deserto, para ser tentado pelo diabo.
\par 2 E, depois de jejuar quarenta dias e quarenta noites, teve fome.
\par 3 Então, o tentador, aproximando-se, lhe disse: Se és Filho de Deus, manda que estas pedras se transformem em pães.
\par 4 Jesus, porém, respondeu: Está escrito: Não só de pão viverá o homem, mas de toda palavra que procede da boca de Deus.
\par 5 Então, o diabo o levou à Cidade Santa, colocou-o sobre o pináculo do templo
\par 6 e lhe disse: Se és Filho de Deus, atira-te abaixo, porque está escrito: Aos seus anjos ordenará a teu respeito que te guardem; e: Eles te susterão nas suas mãos, para não tropeçares nalguma pedra.
\par 7 Respondeu-lhe Jesus: Também está escrito: Não tentarás o Senhor, teu Deus.
\par 8 Levou-o ainda o diabo a um monte muito alto, mostrou-lhe todos os reinos do mundo e a glória deles
\par 9 e lhe disse: Tudo isto te darei se, prostrado, me adorares.
\par 10 Então, Jesus lhe ordenou: Retira-te, Satanás, porque está escrito: Ao Senhor, teu Deus, adorarás, e só a ele darás culto.
\par 11 Com isto, o deixou o diabo, e eis que vieram anjos e o serviram.
\par 12 Ouvindo, porém, Jesus que João fora preso, retirou-se para a Galiléia;
\par 13 e, deixando Nazaré, foi morar em Cafarnaum, situada à beira-mar, nos confins de Zebulom e Naftali;
\par 14 para que se cumprisse o que fora dito por intermédio do profeta Isaías:
\par 15 Terra de Zebulom, terra de Naftali, caminho do mar, além do Jordão, Galiléia dos gentios!
\par 16 O povo que jazia em trevas viu grande luz, e aos que viviam na região e sombra da morte resplandeceu-lhes a luz.
\par 17 Daí por diante, passou Jesus a pregar e a dizer: Arrependei-vos, porque está próximo o reino dos céus.
\par 18 Caminhando junto ao mar da Galiléia, viu dois irmãos, Simão, chamado Pedro, e André, que lançavam as redes ao mar, porque eram pescadores.
\par 19 E disse-lhes: Vinde após mim, e eu vos farei pescadores de homens.
\par 20 Então, eles deixaram imediatamente as redes e o seguiram.
\par 21 Passando adiante, viu outros dois irmãos, Tiago, filho de Zebedeu, e João, seu irmão, que estavam no barco em companhia de seu pai, consertando as redes; e chamou-os.
\par 22 Então, eles, no mesmo instante, deixando o barco e seu pai, o seguiram.
\par 23 Percorria Jesus toda a Galiléia, ensinando nas sinagogas, pregando o evangelho do reino e curando toda sorte de doenças e enfermidades entre o povo.
\par 24 E a sua fama correu por toda a Síria; trouxeram-lhe, então, todos os doentes, acometidos de várias enfermidades e tormentos: endemoninhados, lunáticos e paralíticos. E ele os curou.
\par 25 E da Galiléia, Decápolis, Jerusalém, Judéia e dalém do Jordão numerosas multidões o seguiam.

\chapter{5}

\par 1 Vendo Jesus as multidões, subiu ao monte, e, como se assentasse, aproximaram-se os seus discípulos;
\par 2 e ele passou a ensiná-los, dizendo:
\par 3 Bem-aventurados os humildes de espírito, porque deles é o reino dos céus.
\par 4 Bem-aventurados os que choram, porque serão consolados.
\par 5 Bem-aventurados os mansos, porque herdarão a terra.
\par 6 Bem-aventurados os que têm fome e sede de justiça, porque serão fartos.
\par 7 Bem-aventurados os misericordiosos, porque alcançarão misericórdia.
\par 8 Bem-aventurados os limpos de coração, porque verão a Deus.
\par 9 Bem-aventurados os pacificadores, porque serão chamados filhos de Deus.
\par 10 Bem-aventurados os perseguidos por causa da justiça, porque deles é o reino dos céus.
\par 11 Bem-aventurados sois quando, por minha causa, vos injuriarem, e vos perseguirem, e, mentindo, disserem todo mal contra vós.
\par 12 Regozijai-vos e exultai, porque é grande o vosso galardão nos céus; pois assim perseguiram aos profetas que viveram antes de vós.
\par 13 Vós sois o sal da terra; ora, se o sal vier a ser insípido, como lhe restaurar o sabor? Para nada mais presta senão para, lançado fora, ser pisado pelos homens.
\par 14 Vós sois a luz do mundo. Não se pode esconder a cidade edificada sobre um monte;
\par 15 nem se acende uma candeia para colocá-la debaixo do alqueire, mas no velador, e alumia a todos os que se encontram na casa.
\par 16 Assim brilhe também a vossa luz diante dos homens, para que vejam as vossas boas obras e glorifiquem a vosso Pai que está nos céus.
\par 17 Não penseis que vim revogar a Lei ou os Profetas; não vim para revogar, vim para cumprir.
\par 18 Porque em verdade vos digo: até que o céu e a terra passem, nem um i ou um til jamais passará da Lei, até que tudo se cumpra.
\par 19 Aquele, pois, que violar um destes mandamentos, posto que dos menores, e assim ensinar aos homens, será considerado mínimo no reino dos céus; aquele, porém, que os observar e ensinar, esse será considerado grande no reino dos céus.
\par 20 Porque vos digo que, se a vossa justiça não exceder em muito a dos escribas e fariseus, jamais entrareis no reino dos céus.
\par 21 Ouvistes que foi dito aos antigos: Não matarás; e: Quem matar estará sujeito a julgamento.
\par 22 Eu, porém, vos digo que todo aquele que [sem motivo] se irar contra seu irmão estará sujeito a julgamento; e quem proferir um insulto a seu irmão estará sujeito a julgamento do tribunal; e quem lhe chamar: Tolo, estará sujeito ao inferno de fogo.
\par 23 Se, pois, ao trazeres ao altar a tua oferta, ali te lembrares de que teu irmão tem alguma coisa contra ti,
\par 24 deixa perante o altar a tua oferta, vai primeiro reconciliar-te com teu irmão; e, então, voltando, faze a tua oferta.
\par 25 Entra em acordo sem demora com o teu adversário, enquanto estás com ele a caminho, para que o adversário não te entregue ao juiz, o juiz, ao oficial de justiça, e sejas recolhido à prisão.
\par 26 Em verdade te digo que não sairás dali, enquanto não pagares o último centavo.
\par 27 Ouvistes que foi dito: Não adulterarás.
\par 28 Eu, porém, vos digo: qualquer que olhar para uma mulher com intenção impura, no coração, já adulterou com ela.
\par 29 Se o teu olho direito te faz tropeçar, arranca-o e lança-o de ti; pois te convém que se perca um dos teus membros, e não seja todo o teu corpo lançado no inferno.
\par 30 E, se a tua mão direita te faz tropeçar, corta-a e lança-a de ti; pois te convém que se perca um dos teus membros, e não vá todo o teu corpo para o inferno.
\par 31 Também foi dito: Aquele que repudiar sua mulher, dê-lhe carta de divórcio.
\par 32 Eu, porém, vos digo: qualquer que repudiar sua mulher, exceto em caso de relações sexuais ilícitas, a expõe a tornar-se adúltera; e aquele que casar com a repudiada comete adultério.
\par 33 Também ouvistes que foi dito aos antigos: Não jurarás falso, mas cumprirás rigorosamente para com o Senhor os teus juramentos.
\par 34 Eu, porém, vos digo: de modo algum jureis; nem pelo céu, por ser o trono de Deus;
\par 35 nem pela terra, por ser estrado de seus pés; nem por Jerusalém, por ser cidade do grande Rei;
\par 36 nem jures pela tua cabeça, porque não podes tornar um cabelo branco ou preto.
\par 37 Seja, porém, a tua palavra: Sim, sim; não, não. O que disto passar vem do maligno.
\par 38 Ouvistes que foi dito: Olho por olho, dente por dente.
\par 39 Eu, porém, vos digo: não resistais ao perverso; mas, a qualquer que te ferir na face direita, volta-lhe também a outra;
\par 40 e, ao que quer demandar contigo e tirar-te a túnica, deixa-lhe também a capa.
\par 41 Se alguém te obrigar a andar uma milha, vai com ele duas.
\par 42 Dá a quem te pede e não voltes as costas ao que deseja que lhe emprestes.
\par 43 Ouvistes que foi dito: Amarás o teu próximo e odiarás o teu inimigo.
\par 44 Eu, porém, vos digo: amai os vossos inimigos e orai pelos que vos perseguem;
\par 45 para que vos torneis filhos do vosso Pai celeste, porque ele faz nascer o seu sol sobre maus e bons e vir chuvas sobre justos e injustos.
\par 46 Porque, se amardes os que vos amam, que recompensa tendes? Não fazem os publicanos também o mesmo?
\par 47 E, se saudardes somente os vossos irmãos, que fazeis de mais? Não fazem os gentios também o mesmo?
\par 48 Portanto, sede vós perfeitos como perfeito é o vosso Pai celeste.

\chapter{6}

\par 1 Guardai-vos de exercer a vossa justiça diante dos homens, com o fim de serdes vistos por eles; doutra sorte, não tereis galardão junto de vosso Pai celeste.
\par 2 Quando, pois, deres esmola, não toques trombeta diante de ti, como fazem os hipócritas, nas sinagogas e nas ruas, para serem glorificados pelos homens. Em verdade vos digo que eles já receberam a recompensa.
\par 3 Tu, porém, ao dares a esmola, ignore a tua mão esquerda o que faz a tua mão direita;
\par 4 para que a tua esmola fique em secreto; e teu Pai, que vê em secreto, te recompensará.
\par 5 E, quando orardes, não sereis como os hipócritas; porque gostam de orar em pé nas sinagogas e nos cantos das praças, para serem vistos dos homens. Em verdade vos digo que eles já receberam a recompensa.
\par 6 Tu, porém, quando orares, entra no teu quarto e, fechada a porta, orarás a teu Pai, que está em secreto; e teu Pai, que vê em secreto, te recompensará.
\par 7 E, orando, não useis de vãs repetições, como os gentios; porque presumem que pelo seu muito falar serão ouvidos.
\par 8 Não vos assemelheis, pois, a eles; porque Deus, o vosso Pai, sabe o de que tendes necessidade, antes que lho peçais.
\par 9 Portanto, vós orareis assim: Pai nosso, que estás nos céus, santificado seja o teu nome;
\par 10 venha o teu reino; faça-se a tua vontade, assim na terra como no céu;
\par 11 o pão nosso de cada dia dá-nos hoje;
\par 12 e perdoa-nos as nossas dívidas, assim como nós temos perdoado aos nossos devedores;
\par 13 e não nos deixes cair em tentação; mas livra-nos do mal [pois teu é o reino, o poder e a glória para sempre. Amém]!
\par 14 Porque, se perdoardes aos homens as suas ofensas, também vosso Pai celeste vos perdoará;
\par 15 se, porém, não perdoardes aos homens [as suas ofensas], tampouco vosso Pai vos perdoará as vossas ofensas.
\par 16 Quando jejuardes, não vos mostreis contristados como os hipócritas; porque desfiguram o rosto com o fim de parecer aos homens que jejuam. Em verdade vos digo que eles já receberam a recompensa.
\par 17 Tu, porém, quando jejuares, unge a cabeça e lava o rosto,
\par 18 com o fim de não parecer aos homens que jejuas, e sim ao teu Pai, em secreto; e teu Pai, que vê em secreto, te recompensará.
\par 19 Não acumuleis para vós outros tesouros sobre a terra, onde a traça e a ferrugem corroem e onde ladrões escavam e roubam;
\par 20 mas ajuntai para vós outros tesouros no céu, onde traça nem ferrugem corrói, e onde ladrões não escavam, nem roubam;
\par 21 porque, onde está o teu tesouro, aí estará também o teu coração.
\par 22 São os olhos a lâmpada do corpo. Se os teus olhos forem bons, todo o teu corpo será luminoso;
\par 23 se, porém, os teus olhos forem maus, todo o teu corpo estará em trevas. Portanto, caso a luz que em ti há sejam trevas, que grandes trevas serão!
\par 24 Ninguém pode servir a dois senhores; porque ou há de aborrecer-se de um e amar ao outro, ou se devotará a um e desprezará ao outro. Não podeis servir a Deus e às riquezas.
\par 25 Por isso, vos digo: não andeis ansiosos pela vossa vida, quanto ao que haveis de comer ou beber; nem pelo vosso corpo, quanto ao que haveis de vestir. Não é a vida mais do que o alimento, e o corpo, mais do que as vestes?
\par 26 Observai as aves do céu: não semeiam, não colhem, nem ajuntam em celeiros; contudo, vosso Pai celeste as sustenta. Porventura, não valeis vós muito mais do que as aves?
\par 27 Qual de vós, por ansioso que esteja, pode acrescentar um côvado ao curso da sua vida?
\par 28 E por que andais ansiosos quanto ao vestuário? Considerai como crescem os lírios do campo: eles não trabalham, nem fiam.
\par 29 Eu, contudo, vos afirmo que nem Salomão, em toda a sua glória, se vestiu como qualquer deles.
\par 30 Ora, se Deus veste assim a erva do campo, que hoje existe e amanhã é lançada no forno, quanto mais a vós outros, homens de pequena fé?
\par 31 Portanto, não vos inquieteis, dizendo: Que comeremos? Que beberemos? Ou: Com que nos vestiremos?
\par 32 Porque os gentios é que procuram todas estas coisas; pois vosso Pai celeste sabe que necessitais de todas elas;
\par 33 buscai, pois, em primeiro lugar, o seu reino e a sua justiça, e todas estas coisas vos serão acrescentadas.
\par 34 Portanto, não vos inquieteis com o dia de amanhã, pois o amanhã trará os seus cuidados; basta ao dia o seu próprio mal.

\chapter{7}

\par 1 Não julgueis, para que não sejais julgados.
\par 2 Pois, com o critério com que julgardes, sereis julgados; e, com a medida com que tiverdes medido, vos medirão também.
\par 3 Por que vês tu o argueiro no olho de teu irmão, porém não reparas na trave que está no teu próprio?
\par 4 Ou como dirás a teu irmão: Deixa-me tirar o argueiro do teu olho, quando tens a trave no teu?
\par 5 Hipócrita! Tira primeiro a trave do teu olho e, então, verás claramente para tirar o argueiro do olho de teu irmão.
\par 6 Não deis aos cães o que é santo, nem lanceis ante os porcos as vossas pérolas, para que não as pisem com os pés e, voltando-se, vos dilacerem.
\par 7 Pedi, e dar-se-vos-á; buscai e achareis; batei, e abrir-se-vos-á.
\par 8 Pois todo o que pede recebe; o que busca encontra; e, a quem bate, abrir-se-lhe-á.
\par 9 Ou qual dentre vós é o homem que, se porventura o filho lhe pedir pão, lhe dará pedra?
\par 10 Ou, se lhe pedir um peixe, lhe dará uma cobra?
\par 11 Ora, se vós, que sois maus, sabeis dar boas dádivas aos vossos filhos, quanto mais vosso Pai, que está nos céus, dará boas coisas aos que lhe pedirem?
\par 12 Tudo quanto, pois, quereis que os homens vos façam, assim fazei-o vós também a eles; porque esta é a Lei e os Profetas.
\par 13 Entrai pela porta estreita (larga é a porta, e espaçoso, o caminho que conduz para a perdição, e são muitos os que entram por ela),
\par 14 porque estreita é a porta, e apertado, o caminho que conduz para a vida, e são poucos os que acertam com ela.
\par 15 Acautelai-vos dos falsos profetas, que se vos apresentam disfarçados em ovelhas, mas por dentro são lobos roubadores.
\par 16 Pelos seus frutos os conhecereis. Colhem-se, porventura, uvas dos espinheiros ou figos dos abrolhos?
\par 17 Assim, toda árvore boa produz bons frutos, porém a árvore má produz frutos maus.
\par 18 Não pode a árvore boa produzir frutos maus, nem a árvore má produzir frutos bons.
\par 19 Toda árvore que não produz bom fruto é cortada e lançada ao fogo.
\par 20 Assim, pois, pelos seus frutos os conhecereis.
\par 21 Nem todo o que me diz: Senhor, Senhor! entrará no reino dos céus, mas aquele que faz a vontade de meu Pai, que está nos céus.
\par 22 Muitos, naquele dia, hão de dizer-me: Senhor, Senhor! Porventura, não temos nós profetizado em teu nome, e em teu nome não expelimos demônios, e em teu nome não fizemos muitos milagres?
\par 23 Então, lhes direi explicitamente: nunca vos conheci. Apartai-vos de mim, os que praticais a iniqüidade.
\par 24 Todo aquele, pois, que ouve estas minhas palavras e as pratica será comparado a um homem prudente que edificou a sua casa sobre a rocha;
\par 25 e caiu a chuva, transbordaram os rios, sopraram os ventos e deram com ímpeto contra aquela casa, que não caiu, porque fora edificada sobre a rocha.
\par 26 E todo aquele que ouve estas minhas palavras e não as pratica será comparado a um homem insensato que edificou a sua casa sobre a areia;
\par 27 e caiu a chuva, transbordaram os rios, sopraram os ventos e deram com ímpeto contra aquela casa, e ela desabou, sendo grande a sua ruína.
\par 28 Quando Jesus acabou de proferir estas palavras, estavam as multidões maravilhadas da sua doutrina;
\par 29 porque ele as ensinava como quem tem autoridade e não como os escribas.

\chapter{8}

\par 1 Ora, descendo ele do monte, grandes multidões o seguiram.
\par 2 E eis que um leproso, tendo-se aproximado, adorou-o, dizendo: Senhor, se quiseres, podes purificar-me.
\par 3 E Jesus, estendendo a mão, tocou-lhe, dizendo: Quero, fica limpo! E imediatamente ele ficou limpo da sua lepra.
\par 4 Disse-lhe, então, Jesus: Olha, não o digas a ninguém, mas vai mostrar-te ao sacerdote e fazer a oferta que Moisés ordenou, para servir de testemunho ao povo.
\par 5 Tendo Jesus entrado em Cafarnaum, apresentou-se-lhe um centurião, implorando:
\par 6 Senhor, o meu criado jaz em casa, de cama, paralítico, sofrendo horrivelmente.
\par 7 Jesus lhe disse: Eu irei curá-lo.
\par 8 Mas o centurião respondeu: Senhor, não sou digno de que entres em minha casa; mas apenas manda com uma palavra, e o meu rapaz será curado.
\par 9 Pois também eu sou homem sujeito à autoridade, tenho soldados às minhas ordens e digo a este: vai, e ele vai; e a outro: vem, e ele vem; e ao meu servo: faze isto, e ele o faz.
\par 10 Ouvindo isto, admirou-se Jesus e disse aos que o seguiam: Em verdade vos afirmo que nem mesmo em Israel achei fé como esta.
\par 11 Digo-vos que muitos virão do Oriente e do Ocidente e tomarão lugares à mesa com Abraão, Isaque e Jacó no reino dos céus.
\par 12 Ao passo que os filhos do reino serão lançados para fora, nas trevas; ali haverá choro e ranger de dentes.
\par 13 Então, disse Jesus ao centurião: Vai-te, e seja feito conforme a tua fé. E, naquela mesma hora, o servo foi curado.
\par 14 Tendo Jesus chegado à casa de Pedro, viu a sogra deste acamada e ardendo em febre.
\par 15 Mas Jesus tomou-a pela mão, e a febre a deixou. Ela se levantou e passou a servi-lo.
\par 16 Chegada a tarde, trouxeram-lhe muitos endemoninhados; e ele meramente com a palavra expeliu os espíritos e curou todos os que estavam doentes;
\par 17 para que se cumprisse o que fora dito por intermédio do profeta Isaías: Ele mesmo tomou as nossas enfermidades e carregou com as nossas doenças.
\par 18 Vendo Jesus muita gente ao seu redor, ordenou que passassem para a outra margem.
\par 19 Então, aproximando-se dele um escriba, disse-lhe: Mestre, seguir-te-ei para onde quer que fores.
\par 20 Mas Jesus lhe respondeu: As raposas têm seus covis, e as aves do céu, ninhos; mas o Filho do Homem não tem onde reclinar a cabeça.
\par 21 E outro dos discípulos lhe disse: Senhor, permite-me ir primeiro sepultar meu pai.
\par 22 Replicou-lhe, porém, Jesus: Segue-me, e deixa aos mortos o sepultar os seus próprios mortos.
\par 23 Então, entrando ele no barco, seus discípulos o seguiram.
\par 24 E eis que sobreveio no mar uma grande tempestade, de sorte que o barco era varrido pelas ondas. Entretanto, Jesus dormia.
\par 25 Mas os discípulos vieram acordá-lo, clamando: Senhor, salva-nos! Perecemos!
\par 26 Perguntou-lhes, então, Jesus: Por que sois tímidos, homens de pequena fé? E, levantando-se, repreendeu os ventos e o mar; e fez-se grande bonança.
\par 27 E maravilharam-se os homens, dizendo: Quem é este que até os ventos e o mar lhe obedecem?
\par 28 Tendo ele chegado à outra margem, à terra dos gadarenos, vieram-lhe ao encontro dois endemoninhados, saindo dentre os sepulcros, e a tal ponto furiosos, que ninguém podia passar por aquele caminho.
\par 29 E eis que gritaram: Que temos nós contigo, ó Filho de Deus! Vieste aqui atormentar-nos antes de tempo?
\par 30 Ora, andava pastando, não longe deles, uma grande manada de porcos.
\par 31 Então, os demônios lhe rogavam: Se nos expeles, manda-nos para a manada de porcos.
\par 32 Pois ide, ordenou-lhes Jesus. E eles, saindo, passaram para os porcos; e eis que toda a manada se precipitou, despenhadeiro abaixo, para dentro do mar, e nas águas pereceram.
\par 33 Fugiram os porqueiros e, chegando à cidade, contaram todas estas coisas e o que acontecera aos endemoninhados.
\par 34 Então, a cidade toda saiu para encontrar-se com Jesus; e, vendo-o, lhe rogaram que se retirasse da terra deles.

\chapter{9}

\par 1 Entrando Jesus num barco, passou para o outro lado e foi para a sua própria cidade.
\par 2 E eis que lhe trouxeram um paralítico deitado num leito. Vendo-lhes a fé, Jesus disse ao paralítico: Tem bom ânimo, filho; estão perdoados os teus pecados.
\par 3 Mas alguns escribas diziam consigo: Este blasfema.
\par 4 Jesus, porém, conhecendo-lhes os pensamentos, disse: Por que cogitais o mal no vosso coração?
\par 5 Pois qual é mais fácil? Dizer: Estão perdoados os teus pecados, ou dizer: Levanta-te e anda?
\par 6 Ora, para que saibais que o Filho do Homem tem sobre a terra autoridade para perdoar pecados -- disse, então, ao paralítico: Levanta-te, toma o teu leito e vai para tua casa.
\par 7 E, levantando-se, partiu para sua casa.
\par 8 Vendo isto, as multidões, possuídas de temor, glorificaram a Deus, que dera tal autoridade aos homens.
\par 9 Partindo Jesus dali, viu um homem chamado Mateus sentado na coletoria e disse-lhe: Segue-me! Ele se levantou e o seguiu.
\par 10 E sucedeu que, estando ele em casa, à mesa, muitos publicanos e pecadores vieram e tomaram lugares com Jesus e seus discípulos.
\par 11 Ora, vendo isto, os fariseus perguntavam aos discípulos: Por que come o vosso Mestre com os publicanos e pecadores?
\par 12 Mas Jesus, ouvindo, disse: Os sãos não precisam de médico, e sim os doentes.
\par 13 Ide, porém, e aprendei o que significa: Misericórdia quero e não holocaustos; pois não vim chamar justos, e sim pecadores [ao arrependimento].
\par 14 Vieram, depois, os discípulos de João e lhe perguntaram: Por que jejuamos nós, e os fariseus [muitas vezes], e teus discípulos não jejuam?
\par 15 Respondeu-lhes Jesus: Podem, acaso, estar tristes os convidados para o casamento, enquanto o noivo está com eles? Dias virão, contudo, em que lhes será tirado o noivo, e nesses dias hão de jejuar.
\par 16 Ninguém põe remendo de pano novo em veste velha; porque o remendo tira parte da veste, e fica maior a rotura.
\par 17 Nem se põe vinho novo em odres velhos; do contrário, rompem-se os odres, derrama-se o vinho, e os odres se perdem. Mas põe-se vinho novo em odres novos, e ambos se conservam.
\par 18 Enquanto estas coisas lhes dizia, eis que um chefe, aproximando-se, o adorou e disse: Minha filha faleceu agora mesmo; mas vem, impõe a mão sobre ela, e viverá.
\par 19 E Jesus, levantando-se, o seguia, e também os seus discípulos.
\par 20 E eis que uma mulher, que durante doze anos vinha padecendo de uma hemorragia, veio por trás dele e lhe tocou na orla da veste;
\par 21 porque dizia consigo mesma: Se eu apenas lhe tocar a veste, ficarei curada.
\par 22 E Jesus, voltando-se e vendo-a, disse: Tem bom ânimo, filha, a tua fé te salvou. E, desde aquele instante, a mulher ficou sã.
\par 23 Tendo Jesus chegado à casa do chefe e vendo os tocadores de flauta e o povo em alvoroço, disse:
\par 24 Retirai-vos, porque não está morta a menina, mas dorme. E riam-se dele.
\par 25 Mas, afastado o povo, entrou Jesus, tomou a menina pela mão, e ela se levantou.
\par 26 E a fama deste acontecimento correu por toda aquela terra.
\par 27 Partindo Jesus dali, seguiram-no dois cegos, clamando: Tem compaixão de nós, Filho de Davi!
\par 28 Tendo ele entrado em casa, aproximaram-se os cegos, e Jesus lhes perguntou: Credes que eu posso fazer isso? Responderam-lhe: Sim, Senhor!
\par 29 Então, lhes tocou os olhos, dizendo: Faça-se-vos conforme a vossa fé.
\par 30 E abriram-se-lhes os olhos. Jesus, porém, os advertiu severamente, dizendo: Acautelai-vos de que ninguém o saiba.
\par 31 Saindo eles, porém, divulgaram-lhe a fama por toda aquela terra.
\par 32 Ao retirarem-se eles, foi-lhe trazido um mudo endemoninhado.
\par 33 E, expelido o demônio, falou o mudo; e as multidões se admiravam, dizendo: Jamais se viu tal coisa em Israel!
\par 34 Mas os fariseus murmuravam: Pelo maioral dos demônios é que expele os demônios.
\par 35 E percorria Jesus todas as cidades e povoados, ensinando nas sinagogas, pregando o evangelho do reino e curando toda sorte de doenças e enfermidades.
\par 36 Vendo ele as multidões, compadeceu-se delas, porque estavam aflitas e exaustas como ovelhas que não têm pastor.
\par 37 E, então, se dirigiu a seus discípulos: A seara, na verdade, é grande, mas os trabalhadores são poucos.
\par 38 Rogai, pois, ao Senhor da seara que mande trabalhadores para a sua seara.

\chapter{10}

\par 1 Tendo chamado os seus doze discípulos, deu-lhes Jesus autoridade sobre espíritos imundos para os expelir e para curar toda sorte de doenças e enfermidades.
\par 2 Ora, os nomes dos doze apóstolos são estes: primeiro, Simão, por sobrenome Pedro, e André, seu irmão; Tiago, filho de Zebedeu, e João, seu irmão;
\par 3 Filipe e Bartolomeu; Tomé e Mateus, o publicano; Tiago, filho de Alfeu, e Tadeu;
\par 4 Simão, o Zelote, e Judas Iscariotes, que foi quem o traiu.
\par 5 A estes doze enviou Jesus, dando-lhes as seguintes instruções: Não tomeis rumo aos gentios, nem entreis em cidade de samaritanos;
\par 6 mas, de preferência, procurai as ovelhas perdidas da casa de Israel;
\par 7 e, à medida que seguirdes, pregai que está próximo o reino dos céus.
\par 8 Curai enfermos, ressuscitai mortos, purificai leprosos, expeli demônios; de graça recebestes, de graça dai.
\par 9 Não vos provereis de ouro, nem de prata, nem de cobre nos vossos cintos;
\par 10 nem de alforje para o caminho, nem de duas túnicas, nem de sandálias, nem de bordão; porque digno é o trabalhador do seu alimento.
\par 11 E, em qualquer cidade ou povoado em que entrardes, indagai quem neles é digno; e aí ficai até vos retirardes.
\par 12 Ao entrardes na casa, saudai-a;
\par 13 se, com efeito, a casa for digna, venha sobre ela a vossa paz; se, porém, não o for, torne para vós outros a vossa paz.
\par 14 Se alguém não vos receber, nem ouvir as vossas palavras, ao sairdes daquela casa ou daquela cidade, sacudi o pó dos vossos pés.
\par 15 Em verdade vos digo que menos rigor haverá para Sodoma e Gomorra, no Dia do Juízo, do que para aquela cidade.
\par 16 Eis que eu vos envio como ovelhas para o meio de lobos; sede, portanto, prudentes como as serpentes e símplices como as pombas.
\par 17 E acautelai-vos dos homens; porque vos entregarão aos tribunais e vos açoitarão nas suas sinagogas;
\par 18 por minha causa sereis levados à presença de governadores e de reis, para lhes servir de testemunho, a eles e aos gentios.
\par 19 E, quando vos entregarem, não cuideis em como ou o que haveis de falar, porque, naquela hora, vos será concedido o que haveis de dizer,
\par 20 visto que não sois vós os que falais, mas o Espírito de vosso Pai é quem fala em vós.
\par 21 Um irmão entregará à morte outro irmão, e o pai, ao filho; filhos haverá que se levantarão contra os progenitores e os matarão.
\par 22 Sereis odiados de todos por causa do meu nome; aquele, porém, que perseverar até ao fim, esse será salvo.
\par 23 Quando, porém, vos perseguirem numa cidade, fugi para outra; porque em verdade vos digo que não acabareis de percorrer as cidades de Israel, até que venha o Filho do Homem.
\par 24 O discípulo não está acima do seu mestre, nem o servo, acima do seu senhor.
\par 25 Basta ao discípulo ser como o seu mestre, e ao servo, como o seu senhor. Se chamaram Belzebu ao dono da casa, quanto mais aos seus domésticos?
\par 26 Portanto, não os temais; pois nada há encoberto, que não venha a ser revelado; nem oculto, que não venha a ser conhecido.
\par 27 O que vos digo às escuras, dizei-o a plena luz; e o que se vos diz ao ouvido, proclamai-o dos eirados.
\par 28 Não temais os que matam o corpo e não podem matar a alma; temei, antes, aquele que pode fazer perecer no inferno tanto a alma como o corpo.
\par 29 Não se vendem dois pardais por um asse? E nenhum deles cairá em terra sem o consentimento de vosso Pai.
\par 30 E, quanto a vós outros, até os cabelos todos da cabeça estão contados.
\par 31 Não temais, pois! Bem mais valeis vós do que muitos pardais.
\par 32 Portanto, todo aquele que me confessar diante dos homens, também eu o confessarei diante de meu Pai, que está nos céus;
\par 33 mas aquele que me negar diante dos homens, também eu o negarei diante de meu Pai, que está nos céus.
\par 34 Não penseis que vim trazer paz à terra; não vim trazer paz, mas espada.
\par 35 Pois vim causar divisão entre o homem e seu pai; entre a filha e sua mãe e entre a nora e sua sogra.
\par 36 Assim, os inimigos do homem serão os da sua própria casa.
\par 37 Quem ama seu pai ou sua mãe mais do que a mim não é digno de mim; quem ama seu filho ou sua filha mais do que a mim não é digno de mim;
\par 38 e quem não toma a sua cruz e vem após mim não é digno de mim.
\par 39 Quem acha a sua vida perdê-la-á; quem, todavia, perde a vida por minha causa achá-la-á.
\par 40 Quem vos recebe a mim me recebe; e quem me recebe recebe aquele que me enviou.
\par 41 Quem recebe um profeta, no caráter de profeta, receberá o galardão de profeta; quem recebe um justo, no caráter de justo, receberá o galardão de justo.
\par 42 E quem der a beber, ainda que seja um copo de água fria, a um destes pequeninos, por ser este meu discípulo, em verdade vos digo que de modo algum perderá o seu galardão.

\chapter{11}

\par 1 Ora, tendo acabado Jesus de dar estas instruções a seus doze discípulos, partiu dali a ensinar e a pregar nas cidades deles.
\par 2 Quando João ouviu, no cárcere, falar das obras de Cristo, mandou por seus discípulos perguntar-lhe:
\par 3 És tu aquele que estava para vir ou havemos de esperar outro?
\par 4 E Jesus, respondendo, disse-lhes: Ide e anunciai a João o que estais ouvindo e vendo:
\par 5 os cegos vêem, os coxos andam, os leprosos são purificados, os surdos ouvem, os mortos são ressuscitados, e aos pobres está sendo pregado o evangelho.
\par 6 E bem-aventurado é aquele que não achar em mim motivo de tropeço.
\par 7 Então, em partindo eles, passou Jesus a dizer ao povo a respeito de João: Que saístes a ver no deserto? Um caniço agitado pelo vento?
\par 8 Sim, que saístes a ver? Um homem vestido de roupas finas? Ora, os que vestem roupas finas assistem nos palácios reais.
\par 9 Mas para que saístes? Para ver um profeta? Sim, eu vos digo, e muito mais que profeta.
\par 10 Este é de quem está escrito: Eis aí eu envio diante da tua face o meu mensageiro, o qual preparará o teu caminho diante de ti.
\par 11 Em verdade vos digo: entre os nascidos de mulher, ninguém apareceu maior do que João Batista; mas o menor no reino dos céus é maior do que ele.
\par 12 Desde os dias de João Batista até agora, o reino dos céus é tomado por esforço, e os que se esforçam se apoderam dele.
\par 13 Porque todos os Profetas e a Lei profetizaram até João.
\par 14 E, se o quereis reconhecer, ele mesmo é Elias, que estava para vir.
\par 15 Quem tem ouvidos [para ouvir], ouça.
\par 16 Mas a quem hei de comparar esta geração? É semelhante a meninos que, sentados nas praças, gritam aos companheiros:
\par 17 Nós vos tocamos flauta, e não dançastes; entoamos lamentações, e não pranteastes.
\par 18 Pois veio João, que não comia nem bebia, e dizem: Tem demônio!
\par 19 Veio o Filho do Homem, que come e bebe, e dizem: Eis aí um glutão e bebedor de vinho, amigo de publicanos e pecadores! Mas a sabedoria é justificada por suas obras.
\par 20 Passou, então, Jesus a increpar as cidades nas quais ele operara numerosos milagres, pelo fato de não se terem arrependido:
\par 21 Ai de ti, Corazim! Ai de ti, Betsaida! Porque, se em Tiro e em Sidom se tivessem operado os milagres que em vós se fizeram, há muito que elas se teriam arrependido com pano de saco e cinza.
\par 22 E, contudo, vos digo: no Dia do Juízo, haverá menos rigor para Tiro e Sidom do que para vós outras.
\par 23 Tu, Cafarnaum, elevar-te-ás, porventura, até ao céu? Descerás até ao inferno; porque, se em Sodoma se tivessem operado os milagres que em ti se fizeram, teria ela permanecido até ao dia de hoje.
\par 24 Digo-vos, porém, que menos rigor haverá, no Dia do Juízo, para com a terra de Sodoma do que para contigo.
\par 25 Por aquele tempo, exclamou Jesus: Graças te dou, ó Pai, Senhor do céu e da terra, porque ocultaste estas coisas aos sábios e instruídos e as revelaste aos pequeninos.
\par 26 Sim, ó Pai, porque assim foi do teu agrado.
\par 27 Tudo me foi entregue por meu Pai. Ninguém conhece o Filho, senão o Pai; e ninguém conhece o Pai, senão o Filho e aquele a quem o Filho o quiser revelar.
\par 28 Vinde a mim, todos os que estais cansados e sobrecarregados, e eu vos aliviarei.
\par 29 Tomai sobre vós o meu jugo e aprendei de mim, porque sou manso e humilde de coração; e achareis descanso para a vossa alma.
\par 30 Porque o meu jugo é suave, e o meu fardo é leve.

\chapter{12}

\par 1 Por aquele tempo, em dia de sábado, passou Jesus pelas searas. Ora, estando os seus discípulos com fome, entraram a colher espigas e a comer.
\par 2 Os fariseus, porém, vendo isso, disseram-lhe: Eis que os teus discípulos fazem o que não é lícito fazer em dia de sábado.
\par 3 Mas Jesus lhes disse: Não lestes o que fez Davi quando ele e seus companheiros tiveram fome?
\par 4 Como entrou na Casa de Deus, e comeram os pães da proposição, os quais não lhes era lícito comer, nem a ele nem aos que com ele estavam, mas exclusivamente aos sacerdotes?
\par 5 Ou não lestes na Lei que, aos sábados, os sacerdotes no templo violam o sábado e ficam sem culpa? Pois eu vos digo:
\par 6 aqui está quem é maior que o templo.
\par 7 Mas, se vós soubésseis o que significa: Misericórdia quero e não holocaustos, não teríeis condenado inocentes.
\par 8 Porque o Filho do Homem é senhor do sábado.
\par 9 Tendo Jesus partido dali, entrou na sinagoga deles.
\par 10 Achava-se ali um homem que tinha uma das mãos ressequida; e eles, então, com o intuito de acusá-lo, perguntaram a Jesus: É lícito curar no sábado?
\par 11 Ao que lhes respondeu: Qual dentre vós será o homem que, tendo uma ovelha, e, num sábado, esta cair numa cova, não fará todo o esforço, tirando-a dali?
\par 12 Ora, quanto mais vale um homem que uma ovelha? Logo, é lícito, nos sábados, fazer o bem.
\par 13 Então, disse ao homem: Estende a mão. Estendeu-a, e ela ficou sã como a outra.
\par 14 Retirando-se, porém, os fariseus, conspiravam contra ele, sobre como lhe tirariam a vida.
\par 15 Mas Jesus, sabendo disto, afastou-se dali. Muitos o seguiram, e a todos ele curou,
\par 16 advertindo-lhes, porém, que o não expusessem à publicidade,
\par 17 para se cumprir o que foi dito por intermédio do profeta Isaías:
\par 18 Eis aqui o meu servo, que escolhi, o meu amado, em quem a minha alma se compraz. Farei repousar sobre ele o meu Espírito, e ele anunciará juízo aos gentios.
\par 19 Não contenderá, nem gritará, nem alguém ouvirá nas praças a sua voz.
\par 20 Não esmagará a cana quebrada, nem apagará a torcida que fumega, até que faça vencedor o juízo.
\par 21 E, no seu nome, esperarão os gentios.
\par 22 Então, lhe trouxeram um endemoninhado, cego e mudo; e ele o curou, passando o mudo a falar e a ver.
\par 23 E toda a multidão se admirava e dizia: É este, porventura, o Filho de Davi?
\par 24 Mas os fariseus, ouvindo isto, murmuravam: Este não expele demônios senão pelo poder de Belzebu, maioral dos demônios.
\par 25 Jesus, porém, conhecendo-lhes os pensamentos, disse: Todo reino dividido contra si mesmo ficará deserto, e toda cidade ou casa dividida contra si mesma não subsistirá.
\par 26 Se Satanás expele a Satanás, dividido está contra si mesmo; como, pois, subsistirá o seu reino?
\par 27 E, se eu expulso demônios por Belzebu, por quem os expulsam vossos filhos? Por isso, eles mesmos serão os vossos juízes.
\par 28 Se, porém, eu expulso demônios pelo Espírito de Deus, certamente é chegado o reino de Deus sobre vós.
\par 29 Ou como pode alguém entrar na casa do valente e roubar-lhe os bens sem primeiro amarrá-lo? E, então, lhe saqueará a casa.
\par 30 Quem não é por mim é contra mim; e quem comigo não ajunta espalha.
\par 31 Por isso, vos declaro: todo pecado e blasfêmia serão perdoados aos homens; mas a blasfêmia contra o Espírito não será perdoada.
\par 32 Se alguém proferir alguma palavra contra o Filho do Homem, ser-lhe-á isso perdoado; mas, se alguém falar contra o Espírito Santo, não lhe será isso perdoado, nem neste mundo nem no porvir.
\par 33 Ou fazei a árvore boa e o seu fruto bom ou a árvore má e o seu fruto mau; porque pelo fruto se conhece a árvore.
\par 34 Raça de víboras, como podeis falar coisas boas, sendo maus? Porque a boca fala do que está cheio o coração.
\par 35 O homem bom tira do tesouro bom coisas boas; mas o homem mau do mau tesouro tira coisas más.
\par 36 Digo-vos que de toda palavra frívola que proferirem os homens, dela darão conta no Dia do Juízo;
\par 37 porque, pelas tuas palavras, serás justificado e, pelas tuas palavras, serás condenado.
\par 38 Então, alguns escribas e fariseus replicaram: Mestre, queremos ver de tua parte algum sinal.
\par 39 Ele, porém, respondeu: Uma geração má e adúltera pede um sinal; mas nenhum sinal lhe será dado, senão o do profeta Jonas.
\par 40 Porque assim como esteve Jonas três dias e três noites no ventre do grande peixe, assim o Filho do Homem estará três dias e três noites no coração da terra.
\par 41 Ninivitas se levantarão, no Juízo, com esta geração e a condenarão; porque se arrependeram com a pregação de Jonas. E eis aqui está quem é maior do que Jonas.
\par 42 A rainha do Sul se levantará, no Juízo, com esta geração e a condenará; porque veio dos confins da terra para ouvir a sabedoria de Salomão. E eis aqui está quem é maior do que Salomão.
\par 43 Quando o espírito imundo sai do homem, anda por lugares áridos procurando repouso, porém não encontra.
\par 44 Por isso, diz: Voltarei para minha casa donde saí. E, tendo voltado, a encontra vazia, varrida e ornamentada.
\par 45 Então, vai e leva consigo outros sete espíritos, piores do que ele, e, entrando, habitam ali; e o último estado daquele homem torna-se pior do que o primeiro. Assim também acontecerá a esta geração perversa.
\par 46 Falava ainda Jesus ao povo, e eis que sua mãe e seus irmãos estavam do lado de fora, procurando falar-lhe.
\par 47 E alguém lhe disse: Tua mãe e teus irmãos estão lá fora e querem falar-te.
\par 48 Porém ele respondeu ao que lhe trouxera o aviso: Quem é minha mãe e quem são meus irmãos?
\par 49 E, estendendo a mão para os discípulos, disse: Eis minha mãe e meus irmãos.
\par 50 Porque qualquer que fizer a vontade de meu Pai celeste, esse é meu irmão, irmã e mãe.

\chapter{13}

\par 1 Naquele mesmo dia, saindo Jesus de casa, assentou-se à beira-mar;
\par 2 e grandes multidões se reuniram perto dele, de modo que entrou num barco e se assentou; e toda a multidão estava em pé na praia.
\par 3 E de muitas coisas lhes falou por parábolas e dizia: Eis que o semeador saiu a semear.
\par 4 E, ao semear, uma parte caiu à beira do caminho, e, vindo as aves, a comeram.
\par 5 Outra parte caiu em solo rochoso, onde a terra era pouca, e logo nasceu, visto não ser profunda a terra.
\par 6 Saindo, porém, o sol, a queimou; e, porque não tinha raiz, secou-se.
\par 7 Outra caiu entre os espinhos, e os espinhos cresceram e a sufocaram.
\par 8 Outra, enfim, caiu em boa terra e deu fruto: a cem, a sessenta e a trinta por um.
\par 9 Quem tem ouvidos [para ouvir], ouça.
\par 10 Então, se aproximaram os discípulos e lhe perguntaram: Por que lhes falas por parábolas?
\par 11 Ao que respondeu: Porque a vós outros é dado conhecer os mistérios do reino dos céus, mas àqueles não lhes é isso concedido.
\par 12 Pois ao que tem se lhe dará, e terá em abundância; mas, ao que não tem, até o que tem lhe será tirado.
\par 13 Por isso, lhes falo por parábolas; porque, vendo, não vêem; e, ouvindo, não ouvem, nem entendem.
\par 14 De sorte que neles se cumpre a profecia de Isaías: Ouvireis com os ouvidos e de nenhum modo entendereis; vereis com os olhos e de nenhum modo percebereis.
\par 15 Porque o coração deste povo está endurecido, de mau grado ouviram com os ouvidos e fecharam os olhos; para não suceder que vejam com os olhos, ouçam com os ouvidos, entendam com o coração, se convertam e sejam por mim curados.
\par 16 Bem-aventurados, porém, os vossos olhos, porque vêem; e os vossos ouvidos, porque ouvem.
\par 17 Pois em verdade vos digo que muitos profetas e justos desejaram ver o que vedes e não viram; e ouvir o que ouvis e não ouviram.
\par 18 Atendei vós, pois, à parábola do semeador.
\par 19 A todos os que ouvem a palavra do reino e não a compreendem, vem o maligno e arrebata o que lhes foi semeado no coração. Este é o que foi semeado à beira do caminho.
\par 20 O que foi semeado em solo rochoso, esse é o que ouve a palavra e a recebe logo, com alegria;
\par 21 mas não tem raiz em si mesmo, sendo, antes, de pouca duração; em lhe chegando a angústia ou a perseguição por causa da palavra, logo se escandaliza.
\par 22 O que foi semeado entre os espinhos é o que ouve a palavra, porém os cuidados do mundo e a fascinação das riquezas sufocam a palavra, e fica infrutífera.
\par 23 Mas o que foi semeado em boa terra é o que ouve a palavra e a compreende; este frutifica e produz a cem, a sessenta e a trinta por um.
\par 24 Outra parábola lhes propôs, dizendo: O reino dos céus é semelhante a um homem que semeou boa semente no seu campo;
\par 25 mas, enquanto os homens dormiam, veio o inimigo dele, semeou o joio no meio do trigo e retirou-se.
\par 26 E, quando a erva cresceu e produziu fruto, apareceu também o joio.
\par 27 Então, vindo os servos do dono da casa, lhe disseram: Senhor, não semeaste boa semente no teu campo? Donde vem, pois, o joio?
\par 28 Ele, porém, lhes respondeu: Um inimigo fez isso. Mas os servos lhe perguntaram: Queres que vamos e arranquemos o joio?
\par 29 Não! Replicou ele, para que, ao separar o joio, não arranqueis também com ele o trigo.
\par 30 Deixai-os crescer juntos até à colheita, e, no tempo da colheita, direi aos ceifeiros: ajuntai primeiro o joio, atai-o em feixes para ser queimado; mas o trigo, recolhei-o no meu celeiro.
\par 31 Outra parábola lhes propôs, dizendo: O reino dos céus é semelhante a um grão de mostarda, que um homem tomou e plantou no seu campo;
\par 32 o qual é, na verdade, a menor de todas as sementes, e, crescida, é maior do que as hortaliças, e se faz árvore, de modo que as aves do céu vêm aninhar-se nos seus ramos.
\par 33 Disse-lhes outra parábola: O reino dos céus é semelhante ao fermento que uma mulher tomou e escondeu em três medidas de farinha, até ficar tudo levedado.
\par 34 Todas estas coisas disse Jesus às multidões por parábolas e sem parábolas nada lhes dizia;
\par 35 para que se cumprisse o que foi dito por intermédio do profeta: Abrirei em parábolas a minha boca; publicarei coisas ocultas desde a criação [do mundo].
\par 36 Então, despedindo as multidões, foi Jesus para casa. E, chegando-se a ele os seus discípulos, disseram: Explica-nos a parábola do joio do campo.
\par 37 E ele respondeu: O que semeia a boa semente é o Filho do Homem;
\par 38 o campo é o mundo; a boa semente são os filhos do reino; o joio são os filhos do maligno;
\par 39 o inimigo que o semeou é o diabo; a ceifa é a consumação do século, e os ceifeiros são os anjos.
\par 40 Pois, assim como o joio é colhido e lançado ao fogo, assim será na consumação do século.
\par 41 Mandará o Filho do Homem os seus anjos, que ajuntarão do seu reino todos os escândalos e os que praticam a iniqüidade
\par 42 e os lançarão na fornalha acesa; ali haverá choro e ranger de dentes.
\par 43 Então, os justos resplandecerão como o sol, no reino de seu Pai. Quem tem ouvidos [para ouvir], ouça.
\par 44 O reino dos céus é semelhante a um tesouro oculto no campo, o qual certo homem, tendo-o achado, escondeu. E, transbordante de alegria, vai, vende tudo o que tem e compra aquele campo.
\par 45 O reino dos céus é também semelhante a um que negocia e procura boas pérolas;
\par 46 e, tendo achado uma pérola de grande valor, vende tudo o que possui e a compra.
\par 47 O reino dos céus é ainda semelhante a uma rede que, lançada ao mar, recolhe peixes de toda espécie.
\par 48 E, quando já está cheia, os pescadores arrastam-na para a praia e, assentados, escolhem os bons para os cestos e os ruins deitam fora.
\par 49 Assim será na consumação do século: sairão os anjos, e separarão os maus dentre os justos,
\par 50 e os lançarão na fornalha acesa; ali haverá choro e ranger de dentes.
\par 51 Entendestes todas estas coisas? Responderam-lhe: Sim!
\par 52 Então, lhes disse: Por isso, todo escriba versado no reino dos céus é semelhante a um pai de família que tira do seu depósito coisas novas e coisas velhas.
\par 53 Tendo Jesus proferido estas parábolas, retirou-se dali.
\par 54 E, chegando à sua terra, ensinava-os na sinagoga, de tal sorte que se maravilhavam e diziam: Donde lhe vêm esta sabedoria e estes poderes miraculosos?
\par 55 Não é este o filho do carpinteiro? Não se chama sua mãe Maria, e seus irmãos, Tiago, José, Simão e Judas?
\par 56 Não vivem entre nós todas as suas irmãs? Donde lhe vem, pois, tudo isto?
\par 57 E escandalizavam-se nele. Jesus, porém, lhes disse: Não há profeta sem honra, senão na sua terra e na sua casa.
\par 58 E não fez ali muitos milagres, por causa da incredulidade deles.

\chapter{14}

\par 1 Por aquele tempo, ouviu o tetrarca Herodes a fama de Jesus
\par 2 e disse aos que o serviam: Este é João Batista; ele ressuscitou dos mortos, e, por isso, nele operam forças miraculosas.
\par 3 Porque Herodes, havendo prendido e atado a João, o metera no cárcere, por causa de Herodias, mulher de Filipe, seu irmão;
\par 4 pois João lhe dizia: Não te é lícito possuí-la.
\par 5 E, querendo matá-lo, temia o povo, porque o tinham como profeta.
\par 6 Ora, tendo chegado o dia natalício de Herodes, dançou a filha de Herodias diante de todos e agradou a Herodes.
\par 7 Pelo que prometeu, com juramento, dar-lhe o que pedisse.
\par 8 Então, ela, instigada por sua mãe, disse: Dá-me, aqui, num prato, a cabeça de João Batista.
\par 9 Entristeceu-se o rei, mas, por causa do juramento e dos que estavam com ele à mesa, determinou que lha dessem;
\par 10 e deu ordens e decapitou a João no cárcere.
\par 11 Foi trazida a cabeça num prato e dada à jovem, que a levou a sua mãe.
\par 12 Então, vieram os seus discípulos, levaram o corpo e o sepultaram; depois, foram e o anunciaram a Jesus.
\par 13 Jesus, ouvindo isto, retirou-se dali num barco, para um lugar deserto, à parte; sabendo-o as multidões, vieram das cidades seguindo-o por terra.
\par 14 Desembarcando, viu Jesus uma grande multidão, compadeceu-se dela e curou os seus enfermos.
\par 15 Ao cair da tarde, vieram os discípulos a Jesus e lhe disseram: O lugar é deserto, e vai adiantada a hora; despede, pois, as multidões para que, indo pelas aldeias, comprem para si o que comer.
\par 16 Jesus, porém, lhes disse: Não precisam retirar-se; dai-lhes, vós mesmos, de comer.
\par 17 Mas eles responderam: Não temos aqui senão cinco pães e dois peixes.
\par 18 Então, ele disse: Trazei-mos.
\par 19 E, tendo mandado que a multidão se assentasse sobre a relva, tomando os cinco pães e os dois peixes, erguendo os olhos ao céu, os abençoou. Depois, tendo partido os pães, deu-os aos discípulos, e estes, às multidões.
\par 20 Todos comeram e se fartaram; e dos pedaços que sobejaram recolheram ainda doze cestos cheios.
\par 21 E os que comeram foram cerca de cinco mil homens, além de mulheres e crianças.
\par 22 Logo a seguir, compeliu Jesus os discípulos a embarcar e passar adiante dele para o outro lado, enquanto ele despedia as multidões.
\par 23 E, despedidas as multidões, subiu ao monte, a fim de orar sozinho. Em caindo a tarde, lá estava ele, só.
\par 24 Entretanto, o barco já estava longe, a muitos estádios da terra, açoitado pelas ondas; porque o vento era contrário.
\par 25 Na quarta vigília da noite, foi Jesus ter com eles, andando por sobre o mar.
\par 26 E os discípulos, ao verem-no andando sobre as águas, ficaram aterrados e exclamaram: É um fantasma! E, tomados de medo, gritaram.
\par 27 Mas Jesus imediatamente lhes disse: Tende bom ânimo! Sou eu. Não temais!
\par 28 Respondendo-lhe Pedro, disse: Se és tu, Senhor, manda-me ir ter contigo, por sobre as águas.
\par 29 E ele disse: Vem! E Pedro, descendo do barco, andou por sobre as águas e foi ter com Jesus.
\par 30 Reparando, porém, na força do vento, teve medo; e, começando a submergir, gritou: Salva-me, Senhor!
\par 31 E, prontamente, Jesus, estendendo a mão, tomou-o e lhe disse: Homem de pequena fé, por que duvidaste?
\par 32 Subindo ambos para o barco, cessou o vento.
\par 33 E os que estavam no barco o adoraram, dizendo: Verdadeiramente és Filho de Deus!
\par 34 Então, estando já no outro lado, chegaram a terra, em Genesaré.
\par 35 Reconhecendo-o os homens daquela terra, mandaram avisar a toda a circunvizinhança e trouxeram-lhe todos os enfermos;
\par 36 e lhe rogavam que ao menos pudessem tocar na orla da sua veste. E todos os que tocaram ficaram sãos.

\chapter{15}

\par 1 Então, vieram de Jerusalém a Jesus alguns fariseus e escribas e perguntaram:
\par 2 Por que transgridem os teus discípulos a tradição dos anciãos? Pois não lavam as mãos, quando comem.
\par 3 Ele, porém, lhes respondeu: Por que transgredis vós também o mandamento de Deus, por causa da vossa tradição?
\par 4 Porque Deus ordenou: Honra a teu pai e a tua mãe; e: Quem maldisser a seu pai ou a sua mãe seja punido de morte.
\par 5 Mas vós dizeis: Se alguém disser a seu pai ou a sua mãe: É oferta ao Senhor aquilo que poderias aproveitar de mim;
\par 6 esse jamais honrará a seu pai ou a sua mãe. E, assim, invalidastes a palavra de Deus, por causa da vossa tradição.
\par 7 Hipócritas! Bem profetizou Isaías a vosso respeito, dizendo:
\par 8 Este povo honra-me com os lábios, mas o seu coração está longe de mim.
\par 9 E em vão me adoram, ensinando doutrinas que são preceitos de homens.
\par 10 E, tendo convocado a multidão, lhes disse: Ouvi e entendei:
\par 11 não é o que entra pela boca o que contamina o homem, mas o que sai da boca, isto, sim, contamina o homem.
\par 12 Então, aproximando-se dele os discípulos, disseram: Sabes que os fariseus, ouvindo a tua palavra, se escandalizaram?
\par 13 Ele, porém, respondeu: Toda planta que meu Pai celestial não plantou será arrancada.
\par 14 Deixai-os; são cegos, guias de cegos. Ora, se um cego guiar outro cego, cairão ambos no barranco.
\par 15 Então, lhe disse Pedro: Explica-nos a parábola.
\par 16 Jesus, porém, disse: Também vós não entendeis ainda?
\par 17 Não compreendeis que tudo o que entra pela boca desce para o ventre e, depois, é lançado em lugar escuso?
\par 18 Mas o que sai da boca vem do coração, e é isso que contamina o homem.
\par 19 Porque do coração procedem maus desígnios, homicídios, adultérios, prostituição, furtos, falsos testemunhos, blasfêmias.
\par 20 São estas as coisas que contaminam o homem; mas o comer sem lavar as mãos não o contamina.
\par 21 Partindo Jesus dali, retirou-se para os lados de Tiro e Sidom.
\par 22 E eis que uma mulher cananéia, que viera daquelas regiões, clamava: Senhor, Filho de Davi, tem compaixão de mim! Minha filha está horrivelmente endemoninhada.
\par 23 Ele, porém, não lhe respondeu palavra. E os seus discípulos, aproximando-se, rogaram-lhe: Despede-a, pois vem clamando atrás de nós.
\par 24 Mas Jesus respondeu: Não fui enviado senão às ovelhas perdidas da casa de Israel.
\par 25 Ela, porém, veio e o adorou, dizendo: Senhor, socorre-me!
\par 26 Então, ele, respondendo, disse: Não é bom tomar o pão dos filhos e lançá-lo aos cachorrinhos.
\par 27 Ela, contudo, replicou: Sim, Senhor, porém os cachorrinhos comem das migalhas que caem da mesa dos seus donos.
\par 28 Então, lhe disse Jesus: Ó mulher, grande é a tua fé! Faça-se contigo como queres. E, desde aquele momento, sua filha ficou sã.
\par 29 Partindo Jesus dali, foi para junto do mar da Galiléia; e, subindo ao monte, assentou-se ali.
\par 30 E vieram a ele muitas multidões trazendo consigo coxos, aleijados, cegos, mudos e outros muitos e os largaram junto aos pés de Jesus; e ele os curou.
\par 31 De modo que o povo se maravilhou ao ver que os mudos falavam, os aleijados recobravam saúde, os coxos andavam e os cegos viam. Então, glorificavam ao Deus de Israel.
\par 32 E, chamando Jesus os seus discípulos, disse: Tenho compaixão desta gente, porque há três dias que permanece comigo e não tem o que comer; e não quero despedi-la em jejum, para que não desfaleça pelo caminho.
\par 33 Mas os discípulos lhe disseram: Onde haverá neste deserto tantos pães para fartar tão grande multidão?
\par 34 Perguntou-lhes Jesus: Quantos pães tendes? Responderam: Sete e alguns peixinhos.
\par 35 Então, tendo mandado o povo assentar-se no chão,
\par 36 tomou os sete pães e os peixes, e, dando graças, partiu, e deu aos discípulos, e estes, ao povo.
\par 37 Todos comeram e se fartaram; e, do que sobejou, recolheram sete cestos cheios.
\par 38 Ora, os que comeram eram quatro mil homens, além de mulheres e crianças.
\par 39 E, tendo despedido as multidões, entrou Jesus no barco e foi para o território de Magadã.

\chapter{16}

\par 1 Aproximando-se os fariseus e os saduceus, tentando-o, pediram-lhe que lhes mostrasse um sinal vindo do céu.
\par 2 Ele, porém, lhes respondeu: Chegada a tarde, dizeis: Haverá bom tempo, porque o céu está avermelhado;
\par 3 e, pela manhã: Hoje, haverá tempestade, porque o céu está de um vermelho sombrio. Sabeis, na verdade, discernir o aspecto do céu e não podeis discernir os sinais dos tempos?
\par 4 Uma geração má e adúltera pede um sinal; e nenhum sinal lhe será dado, senão o de Jonas. E, deixando-os, retirou-se.
\par 5 Ora, tendo os discípulos passado para o outro lado, esqueceram-se de levar pão.
\par 6 E Jesus lhes disse: Vede e acautelai-vos do fermento dos fariseus e dos saduceus.
\par 7 Eles, porém, discorriam entre si, dizendo: É porque não trouxemos pão.
\par 8 Percebendo-o Jesus, disse: Por que discorreis entre vós, homens de pequena fé, sobre o não terdes pão?
\par 9 Não compreendeis ainda, nem vos lembrais dos cinco pães para cinco mil homens e de quantos cestos tomastes?
\par 10 Nem dos sete pães para os quatro mil e de quantos cestos tomastes?
\par 11 Como não compreendeis que não vos falei a respeito de pães? E sim: acautelai-vos do fermento dos fariseus e dos saduceus.
\par 12 Então, entenderam que não lhes dissera que se acautelassem do fermento de pães, mas da doutrina dos fariseus e dos saduceus.
\par 13 Indo Jesus para os lados de Cesaréia de Filipe, perguntou a seus discípulos: Quem diz o povo ser o Filho do Homem?
\par 14 E eles responderam: Uns dizem: João Batista; outros: Elias; e outros: Jeremias ou algum dos profetas.
\par 15 Mas vós, continuou ele, quem dizeis que eu sou?
\par 16 Respondendo Simão Pedro, disse: Tu és o Cristo, o Filho do Deus vivo.
\par 17 Então, Jesus lhe afirmou: Bem-aventurado és, Simão Barjonas, porque não foi carne e sangue que to revelaram, mas meu Pai, que está nos céus.
\par 18 Também eu te digo que tu és Pedro, e sobre esta pedra edificarei a minha igreja, e as portas do inferno não prevalecerão contra ela.
\par 19 Dar-te-ei as chaves do reino dos céus; o que ligares na terra terá sido ligado nos céus; e o que desligares na terra terá sido desligado nos céus.
\par 20 Então, advertiu os discípulos de que a ninguém dissessem ser ele o Cristo.
\par 21 Desde esse tempo, começou Jesus Cristo a mostrar a seus discípulos que lhe era necessário seguir para Jerusalém e sofrer muitas coisas dos anciãos, dos principais sacerdotes e dos escribas, ser morto e ressuscitado no terceiro dia.
\par 22 E Pedro, chamando-o à parte, começou a reprová-lo, dizendo: Tem compaixão de ti, Senhor; isso de modo algum te acontecerá.
\par 23 Mas Jesus, voltando-se, disse a Pedro: Arreda, Satanás! Tu és para mim pedra de tropeço, porque não cogitas das coisas de Deus, e sim das dos homens.
\par 24 Então, disse Jesus a seus discípulos: Se alguém quer vir após mim, a si mesmo se negue, tome a sua cruz e siga-me.
\par 25 Porquanto, quem quiser salvar a sua vida perdê-la-á; e quem perder a vida por minha causa achá-la-á.
\par 26 Pois que aproveitará o homem se ganhar o mundo inteiro e perder a sua alma? Ou que dará o homem em troca da sua alma?
\par 27 Porque o Filho do Homem há de vir na glória de seu Pai, com os seus anjos, e, então, retribuirá a cada um conforme as suas obras.
\par 28 Em verdade vos digo que alguns há, dos que aqui se encontram, que de maneira nenhuma passarão pela morte até que vejam vir o Filho do Homem no seu reino.

\chapter{17}

\par 1 Seis dias depois, tomou Jesus consigo a Pedro e aos irmãos Tiago e João e os levou, em particular, a um alto monte.
\par 2 E foi transfigurado diante deles; o seu rosto resplandecia como o sol, e as suas vestes tornaram-se brancas como a luz.
\par 3 E eis que lhes apareceram Moisés e Elias, falando com ele.
\par 4 Então, disse Pedro a Jesus: Senhor, bom é estarmos aqui; se queres, farei aqui três tendas; uma será tua, outra para Moisés, outra para Elias.
\par 5 Falava ele ainda, quando uma nuvem luminosa os envolveu; e eis, vindo da nuvem, uma voz que dizia: Este é o meu Filho amado, em quem me comprazo; a ele ouvi.
\par 6 Ouvindo-a os discípulos, caíram de bruços, tomados de grande medo.
\par 7 Aproximando-se deles, tocou-lhes Jesus, dizendo: Erguei-vos e não temais!
\par 8 Então, eles, levantando os olhos, a ninguém viram, senão Jesus.
\par 9 E, descendo eles do monte, ordenou-lhes Jesus: A ninguém conteis a visão, até que o Filho do Homem ressuscite dentre os mortos.
\par 10 Mas os discípulos o interrogaram: Por que dizem, pois, os escribas ser necessário que Elias venha primeiro?
\par 11 Então, Jesus respondeu: De fato, Elias virá e restaurará todas as coisas.
\par 12 Eu, porém, vos declaro que Elias já veio, e não o reconheceram; antes, fizeram com ele tudo quanto quiseram. Assim também o Filho do Homem há de padecer nas mãos deles.
\par 13 Então, os discípulos entenderam que lhes falara a respeito de João Batista.
\par 14 E, quando chegaram para junto da multidão, aproximou-se dele um homem, que se ajoelhou e disse:
\par 15 Senhor, compadece-te de meu filho, porque é lunático e sofre muito; pois muitas vezes cai no fogo e outras muitas, na água.
\par 16 Apresentei-o a teus discípulos, mas eles não puderam curá-lo.
\par 17 Jesus exclamou: Ó geração incrédula e perversa! Até quando estarei convosco? Até quando vos sofrerei? Trazei-me aqui o menino.
\par 18 E Jesus repreendeu o demônio, e este saiu do menino; e, desde aquela hora, ficou o menino curado.
\par 19 Então, os discípulos, aproximando-se de Jesus, perguntaram em particular: Por que motivo não pudemos nós expulsá-lo?
\par 20 E ele lhes respondeu: Por causa da pequenez da vossa fé. Pois em verdade vos digo que, se tiverdes fé como um grão de mostarda, direis a este monte: Passa daqui para acolá, e ele passará. Nada vos será impossível.
\par 21 [Mas esta casta não se expele senão por meio de oração e jejum.]
\par 22 Reunidos eles na Galiléia, disse-lhes Jesus: O Filho do Homem está para ser entregue nas mãos dos homens;
\par 23 e estes o matarão; mas, ao terceiro dia, ressuscitará. Então, os discípulos se entristeceram grandemente.
\par 24 Tendo eles chegado a Cafarnaum, dirigiram-se a Pedro os que cobravam o imposto das duas dracmas e perguntaram: Não paga o vosso Mestre as duas dracmas?
\par 25 Sim, respondeu ele. Ao entrar Pedro em casa, Jesus se lhe antecipou, dizendo: Simão, que te parece? De quem cobram os reis da terra impostos ou tributo: dos seus filhos ou dos estranhos?
\par 26 Respondendo Pedro: Dos estranhos, Jesus lhe disse: Logo, estão isentos os filhos.
\par 27 Mas, para que não os escandalizemos, vai ao mar, lança o anzol, e o primeiro peixe que fisgar, tira-o; e, abrindo-lhe a boca, acharás um estáter. Toma-o e entrega-lhes por mim e por ti.

\chapter{18}

\par 1 Naquela hora, aproximaram-se de Jesus os discípulos, perguntando: Quem é, porventura, o maior no reino dos céus?
\par 2 E Jesus, chamando uma criança, colocou-a no meio deles.
\par 3 E disse: Em verdade vos digo que, se não vos converterdes e não vos tornardes como crianças, de modo algum entrareis no reino dos céus.
\par 4 Portanto, aquele que se humilhar como esta criança, esse é o maior no reino dos céus.
\par 5 E quem receber uma criança, tal como esta, em meu nome, a mim me recebe.
\par 6 Qualquer, porém, que fizer tropeçar a um destes pequeninos que crêem em mim, melhor lhe fora que se lhe pendurasse ao pescoço uma grande pedra de moinho, e fosse afogado na profundeza do mar.
\par 7 Ai do mundo, por causa dos escândalos; porque é inevitável que venham escândalos, mas ai do homem pelo qual vem o escândalo!
\par 8 Portanto, se a tua mão ou o teu pé te faz tropeçar, corta-o e lança-o fora de ti; melhor é entrares na vida manco ou aleijado do que, tendo duas mãos ou dois pés, seres lançado no fogo eterno.
\par 9 Se um dos teus olhos te faz tropeçar, arranca-o e lança-o fora de ti; melhor é entrares na vida com um só dos teus olhos do que, tendo dois, seres lançado no inferno de fogo.
\par 10 Vede, não desprezeis a qualquer destes pequeninos; porque eu vos afirmo que os seus anjos nos céus vêem incessantemente a face de meu Pai celeste.
\par 11 [Porque o Filho do Homem veio salvar o que estava perdido.]
\par 12 Que vos parece? Se um homem tiver cem ovelhas, e uma delas se extraviar, não deixará ele nos montes as noventa e nove, indo procurar a que se extraviou?
\par 13 E, se porventura a encontra, em verdade vos digo que maior prazer sentirá por causa desta do que pelas noventa e nove que não se extraviaram.
\par 14 Assim, pois, não é da vontade de vosso Pai celeste que pereça um só destes pequeninos.
\par 15 Se teu irmão pecar [contra ti], vai argüi-lo entre ti e ele só. Se ele te ouvir, ganhaste a teu irmão.
\par 16 Se, porém, não te ouvir, toma ainda contigo uma ou duas pessoas, para que, pelo depoimento de duas ou três testemunhas, toda palavra se estabeleça.
\par 17 E, se ele não os atender, dize-o à igreja; e, se recusar ouvir também a igreja, considera-o como gentio e publicano.
\par 18 Em verdade vos digo que tudo o que ligardes na terra terá sido ligado nos céus, e tudo o que desligardes na terra terá sido desligado nos céus.
\par 19 Em verdade também vos digo que, se dois dentre vós, sobre a terra, concordarem a respeito de qualquer coisa que, porventura, pedirem, ser-lhes-á concedida por meu Pai, que está nos céus.
\par 20 Porque, onde estiverem dois ou três reunidos em meu nome, ali estou no meio deles.
\par 21 Então, Pedro, aproximando-se, lhe perguntou: Senhor, até quantas vezes meu irmão pecará contra mim, que eu lhe perdoe? Até sete vezes?
\par 22 Respondeu-lhe Jesus: Não te digo que até sete vezes, mas até setenta vezes sete.
\par 23 Por isso, o reino dos céus é semelhante a um rei que resolveu ajustar contas com os seus servos.
\par 24 E, passando a fazê-lo, trouxeram-lhe um que lhe devia dez mil talentos.
\par 25 Não tendo ele, porém, com que pagar, ordenou o senhor que fosse vendido ele, a mulher, os filhos e tudo quanto possuía e que a dívida fosse paga.
\par 26 Então, o servo, prostrando-se reverente, rogou: Sê paciente comigo, e tudo te pagarei.
\par 27 E o senhor daquele servo, compadecendo-se, mandou-o embora e perdoou-lhe a dívida.
\par 28 Saindo, porém, aquele servo, encontrou um dos seus conservos que lhe devia cem denários; e, agarrando-o, o sufocava, dizendo: Paga-me o que me deves.
\par 29 Então, o seu conservo, caindo-lhe aos pés, lhe implorava: Sê paciente comigo, e te pagarei.
\par 30 Ele, entretanto, não quis; antes, indo-se, o lançou na prisão, até que saldasse a dívida.
\par 31 Vendo os seus companheiros o que se havia passado, entristeceram-se muito e foram relatar ao seu senhor tudo que acontecera.
\par 32 Então, o seu senhor, chamando-o, lhe disse: Servo malvado, perdoei-te aquela dívida toda porque me suplicaste;
\par 33 não devias tu, igualmente, compadecer-te do teu conservo, como também eu me compadeci de ti?
\par 34 E, indignando-se, o seu senhor o entregou aos verdugos, até que lhe pagasse toda a dívida.
\par 35 Assim também meu Pai celeste vos fará, se do íntimo não perdoardes cada um a seu irmão.

\chapter{19}

\par 1 E aconteceu que, concluindo Jesus estas palavras, deixou a Galiléia e foi para o território da Judéia, além do Jordão.
\par 2 Seguiram-no muitas multidões, e curou-as ali.
\par 3 Vieram a ele alguns fariseus e o experimentavam, perguntando: É lícito ao marido repudiar a sua mulher por qualquer motivo?
\par 4 Então, respondeu ele: Não tendes lido que o Criador, desde o princípio, os fez homem e mulher
\par 5 e que disse: Por esta causa deixará o homem pai e mãe e se unirá a sua mulher, tornando-se os dois uma só carne?
\par 6 De modo que já não são mais dois, porém uma só carne. Portanto, o que Deus ajuntou não o separe o homem.
\par 7 Replicaram-lhe: Por que mandou, então, Moisés dar carta de divórcio e repudiar?
\par 8 Respondeu-lhes Jesus: Por causa da dureza do vosso coração é que Moisés vos permitiu repudiar vossa mulher; entretanto, não foi assim desde o princípio.
\par 9 Eu, porém, vos digo: quem repudiar sua mulher, não sendo por causa de relações sexuais ilícitas, e casar com outra comete adultério [e o que casar com a repudiada comete adultério].
\par 10 Disseram-lhe os discípulos: Se essa é a condição do homem relativamente à sua mulher, não convém casar.
\par 11 Jesus, porém, lhes respondeu: Nem todos são aptos para receber este conceito, mas apenas aqueles a quem é dado.
\par 12 Porque há eunucos de nascença; há outros a quem os homens fizeram tais; e há outros que a si mesmos se fizeram eunucos, por causa do reino dos céus. Quem é apto para o admitir admita.
\par 13 Trouxeram-lhe, então, algumas crianças, para que lhes impusesse as mãos e orasse; mas os discípulos os repreendiam.
\par 14 Jesus, porém, disse: Deixai os pequeninos, não os embaraceis de vir a mim, porque dos tais é o reino dos céus.
\par 15 E, tendo-lhes imposto as mãos, retirou-se dali.
\par 16 E eis que alguém, aproximando-se, lhe perguntou: Mestre, que farei eu de bom, para alcançar a vida eterna?
\par 17 Respondeu-lhe Jesus: Por que me perguntas acerca do que é bom? Bom só existe um. Se queres, porém, entrar na vida, guarda os mandamentos.
\par 18 E ele lhe perguntou: Quais? Respondeu Jesus: Não matarás, não adulterarás, não furtarás, não dirás falso testemunho;
\par 19 honra a teu pai e a tua mãe e amarás o teu próximo como a ti mesmo.
\par 20 Replicou-lhe o jovem: Tudo isso tenho observado; que me falta ainda?
\par 21 Disse-lhe Jesus: Se queres ser perfeito, vai, vende os teus bens, dá aos pobres e terás um tesouro no céu; depois, vem e segue-me.
\par 22 Tendo, porém, o jovem ouvido esta palavra, retirou-se triste, por ser dono de muitas propriedades.
\par 23 Então, disse Jesus a seus discípulos: Em verdade vos digo que um rico dificilmente entrará no reino dos céus.
\par 24 E ainda vos digo que é mais fácil passar um camelo pelo fundo de uma agulha do que entrar um rico no reino de Deus.
\par 25 Ouvindo isto, os discípulos ficaram grandemente maravilhados e disseram: Sendo assim, quem pode ser salvo?
\par 26 Jesus, fitando neles o olhar, disse-lhes: Isto é impossível aos homens, mas para Deus tudo é possível.
\par 27 Então, lhe falou Pedro: Eis que nós tudo deixamos e te seguimos; que será, pois, de nós?
\par 28 Jesus lhes respondeu: Em verdade vos digo que vós, os que me seguistes, quando, na regeneração, o Filho do Homem se assentar no trono da sua glória, também vos assentareis em doze tronos para julgar as doze tribos de Israel.
\par 29 E todo aquele que tiver deixado casas, ou irmãos, ou irmãs, ou pai, ou mãe [ou mulher], ou filhos, ou campos, por causa do meu nome, receberá muitas vezes mais e herdará a vida eterna.
\par 30 Porém muitos primeiros serão últimos; e os últimos, primeiros.

\chapter{20}

\par 1 Porque o reino dos céus é semelhante a um dono de casa que saiu de madrugada para assalariar trabalhadores para a sua vinha.
\par 2 E, tendo ajustado com os trabalhadores a um denário por dia, mandou-os para a vinha.
\par 3 Saindo pela terceira hora, viu, na praça, outros que estavam desocupados
\par 4 e disse-lhes: Ide vós também para a vinha, e vos darei o que for justo. Eles foram.
\par 5 Tendo saído outra vez, perto da hora sexta e da nona, procedeu da mesma forma,
\par 6 e, saindo por volta da hora undécima, encontrou outros que estavam desocupados e perguntou-lhes: Por que estivestes aqui desocupados o dia todo?
\par 7 Responderam-lhe: Porque ninguém nos contratou. Então, lhes disse ele: Ide também vós para a vinha.
\par 8 Ao cair da tarde, disse o senhor da vinha ao seu administrador: Chama os trabalhadores e paga-lhes o salário, começando pelos últimos, indo até aos primeiros.
\par 9 Vindo os da hora undécima, recebeu cada um deles um denário.
\par 10 Ao chegarem os primeiros, pensaram que receberiam mais; porém também estes receberam um denário cada um.
\par 11 Mas, tendo-o recebido, murmuravam contra o dono da casa,
\par 12 dizendo: Estes últimos trabalharam apenas uma hora; contudo, os igualaste a nós, que suportamos a fadiga e o calor do dia.
\par 13 Mas o proprietário, respondendo, disse a um deles: Amigo, não te faço injustiça; não combinaste comigo um denário?
\par 14 Toma o que é teu e vai-te; pois quero dar a este último tanto quanto a ti.
\par 15 Porventura, não me é lícito fazer o que quero do que é meu? Ou são maus os teus olhos porque eu sou bom?
\par 16 Assim, os últimos serão primeiros, e os primeiros serão últimos [porque muitos são chamados, mas poucos escolhidos].
\par 17 Estando Jesus para subir a Jerusalém, chamou à parte os doze e, em caminho, lhes disse:
\par 18 Eis que subimos para Jerusalém, e o Filho do Homem será entregue aos principais sacerdotes e aos escribas. Eles o condenarão à morte.
\par 19 E o entregarão aos gentios para ser escarnecido, açoitado e crucificado; mas, ao terceiro dia, ressurgirá.
\par 20 Então, se chegou a ele a mulher de Zebedeu, com seus filhos, e, adorando-o, pediu-lhe um favor.
\par 21 Perguntou-lhe ele: Que queres? Ela respondeu: Manda que, no teu reino, estes meus dois filhos se assentem, um à tua direita, e o outro à tua esquerda.
\par 22 Mas Jesus respondeu: Não sabeis o que pedis. Podeis vós beber o cálice que eu estou para beber? Responderam-lhe: Podemos.
\par 23 Então, lhes disse: Bebereis o meu cálice; mas o assentar-se à minha direita e à minha esquerda não me compete concedê-lo; é, porém, para aqueles a quem está preparado por meu Pai.
\par 24 Ora, ouvindo isto os dez, indignaram-se contra os dois irmãos.
\par 25 Então, Jesus, chamando-os, disse: Sabeis que os governadores dos povos os dominam e que os maiorais exercem autoridade sobre eles.
\par 26 Não é assim entre vós; pelo contrário, quem quiser tornar-se grande entre vós, será esse o que vos sirva;
\par 27 e quem quiser ser o primeiro entre vós será vosso servo;
\par 28 tal como o Filho do Homem, que não veio para ser servido, mas para servir e dar a sua vida em resgate por muitos.
\par 29 Saindo eles de Jericó, uma grande multidão o acompanhava.
\par 30 E eis que dois cegos, assentados à beira do caminho, tendo ouvido que Jesus passava, clamaram: Senhor, Filho de Davi, tem compaixão de nós!
\par 31 Mas a multidão os repreendia para que se calassem; eles, porém, gritavam cada vez mais: Senhor, Filho de Davi, tem misericórdia de nós!
\par 32 Então, parando Jesus, chamou-os e perguntou: Que quereis que eu vos faça?
\par 33 Responderam: Senhor, que se nos abram os olhos.
\par 34 Condoído, Jesus tocou-lhes os olhos, e imediatamente recuperaram a vista e o foram seguindo.

\chapter{21}

\par 1 Quando se aproximaram de Jerusalém e chegaram a Betfagé, ao monte das Oliveiras, enviou Jesus dois discípulos, dizendo-lhes:
\par 2 Ide à aldeia que aí está diante de vós e logo achareis presa uma jumenta e, com ela, um jumentinho. Desprendei-a e trazei-mos.
\par 3 E, se alguém vos disser alguma coisa, respondei-lhe que o Senhor precisa deles. E logo os enviará.
\par 4 Ora, isto aconteceu para se cumprir o que foi dito por intermédio do profeta:
\par 5 Dizei à filha de Sião: Eis aí te vem o teu Rei, humilde, montado em jumento, num jumentinho, cria de animal de carga.
\par 6 Indo os discípulos e tendo feito como Jesus lhes ordenara,
\par 7 trouxeram a jumenta e o jumentinho. Então, puseram em cima deles as suas vestes, e sobre elas Jesus montou.
\par 8 E a maior parte da multidão estendeu as suas vestes pelo caminho, e outros cortavam ramos de árvores, espalhando-os pela estrada.
\par 9 E as multidões, tanto as que o precediam como as que o seguiam, clamavam: Hosana ao Filho de Davi! Bendito o que vem em nome do Senhor! Hosana nas maiores alturas!
\par 10 E, entrando ele em Jerusalém, toda a cidade se alvoroçou, e perguntavam: Quem é este?
\par 11 E as multidões clamavam: Este é o profeta Jesus, de Nazaré da Galiléia!
\par 12 Tendo Jesus entrado no templo, expulsou todos os que ali vendiam e compravam; também derribou as mesas dos cambistas e as cadeiras dos que vendiam pombas.
\par 13 E disse-lhes: Está escrito: A minha casa será chamada casa de oração; vós, porém, a transformais em covil de salteadores.
\par 14 Vieram a ele, no templo, cegos e coxos, e ele os curou.
\par 15 Mas, vendo os principais sacerdotes e os escribas as maravilhas que Jesus fazia e os meninos clamando: Hosana ao Filho de Davi!, indignaram-se e perguntaram-lhe:
\par 16 Ouves o que estes estão dizendo? Respondeu-lhes Jesus: Sim; nunca lestes: Da boca de pequeninos e crianças de peito tiraste perfeito louvor?
\par 17 E, deixando-os, saiu da cidade para Betânia, onde pernoitou.
\par 18 Cedo de manhã, ao voltar para a cidade, teve fome;
\par 19 e, vendo uma figueira à beira do caminho, aproximou-se dela; e, não tendo achado senão folhas, disse-lhe: Nunca mais nasça fruto de ti! E a figueira secou imediatamente.
\par 20 Vendo isto os discípulos, admiraram-se e exclamaram: Como secou depressa a figueira!
\par 21 Jesus, porém, lhes respondeu: Em verdade vos digo que, se tiverdes fé e não duvidardes, não somente fareis o que foi feito à figueira, mas até mesmo, se a este monte disserdes: Ergue-te e lança-te no mar, tal sucederá;
\par 22 e tudo quanto pedirdes em oração, crendo, recebereis.
\par 23 Tendo Jesus chegado ao templo, estando já ensinando, acercaram-se dele os principais sacerdotes e os anciãos do povo, perguntando: Com que autoridade fazes estas coisas? E quem te deu essa autoridade?
\par 24 E Jesus lhes respondeu: Eu também vos farei uma pergunta; se me responderdes, também eu vos direi com que autoridade faço estas coisas.
\par 25 Donde era o batismo de João, do céu ou dos homens? E discorriam entre si: Se dissermos: do céu, ele nos dirá: Então, por que não acreditastes nele?
\par 26 E, se dissermos: dos homens, é para temer o povo, porque todos consideram João como profeta.
\par 27 Então, responderam a Jesus: Não sabemos. E ele, por sua vez: Nem eu vos digo com que autoridade faço estas coisas.
\par 28 E que vos parece? Um homem tinha dois filhos. Chegando-se ao primeiro, disse: Filho, vai hoje trabalhar na vinha.
\par 29 Ele respondeu: Sim, senhor; porém não foi.
\par 30 Dirigindo-se ao segundo, disse-lhe a mesma coisa. Mas este respondeu: Não quero; depois, arrependido, foi.
\par 31 Qual dos dois fez a vontade do pai? Disseram: O segundo. Declarou-lhes Jesus: Em verdade vos digo que publicanos e meretrizes vos precedem no reino de Deus.
\par 32 Porque João veio a vós outros no caminho da justiça, e não acreditastes nele; ao passo que publicanos e meretrizes creram. Vós, porém, mesmo vendo isto, não vos arrependestes, afinal, para acreditardes nele.
\par 33 Atentai noutra parábola. Havia um homem, dono de casa, que plantou uma vinha. Cercou-a de uma sebe, construiu nela um lagar, edificou-lhe uma torre e arrendou-a a uns lavradores. Depois, se ausentou do país.
\par 34 Ao tempo da colheita, enviou os seus servos aos lavradores, para receber os frutos que lhe tocavam.
\par 35 E os lavradores, agarrando os servos, espancaram a um, mataram a outro e a outro apedrejaram.
\par 36 Enviou ainda outros servos em maior número; e trataram-nos da mesma sorte.
\par 37 E, por último, enviou-lhes o seu próprio filho, dizendo: A meu filho respeitarão.
\par 38 Mas os lavradores, vendo o filho, disseram entre si: Este é o herdeiro; ora, vamos, matemo-lo e apoderemo-nos da sua herança.
\par 39 E, agarrando-o, lançaram-no fora da vinha e o mataram.
\par 40 Quando, pois, vier o senhor da vinha, que fará àqueles lavradores?
\par 41 Responderam-lhe: Fará perecer horrivelmente a estes malvados e arrendará a vinha a outros lavradores que lhe remetam os frutos nos seus devidos tempos.
\par 42 Perguntou-lhes Jesus: Nunca lestes nas Escrituras: A pedra que os construtores rejeitaram, essa veio a ser a principal pedra, angular; isto procede do Senhor e é maravilhoso aos nossos olhos?
\par 43 Portanto, vos digo que o reino de Deus vos será tirado e será entregue a um povo que lhe produza os respectivos frutos.
\par 44 Todo o que cair sobre esta pedra ficará em pedaços; e aquele sobre quem ela cair ficará reduzido a pó.
\par 45 Os principais sacerdotes e os fariseus, ouvindo estas parábolas, entenderam que era a respeito deles que Jesus falava;
\par 46 e, conquanto buscassem prendê-lo, temeram as multidões, porque estas o consideravam como profeta.

\chapter{22}

\par 1 De novo, entrou Jesus a falar por parábolas, dizendo-lhes:
\par 2 O reino dos céus é semelhante a um rei que celebrou as bodas de seu filho.
\par 3 Então, enviou os seus servos a chamar os convidados para as bodas; mas estes não quiseram vir.
\par 4 Enviou ainda outros servos, com esta ordem: Dizei aos convidados: Eis que já preparei o meu banquete; os meus bois e cevados já foram abatidos, e tudo está pronto; vinde para as bodas.
\par 5 Eles, porém, não se importaram e se foram, um para o seu campo, outro para o seu negócio;
\par 6 e os outros, agarrando os servos, os maltrataram e mataram.
\par 7 O rei ficou irado e, enviando as suas tropas, exterminou aqueles assassinos e lhes incendiou a cidade.
\par 8 Então, disse aos seus servos: Está pronta a festa, mas os convidados não eram dignos.
\par 9 Ide, pois, para as encruzilhadas dos caminhos e convidai para as bodas a quantos encontrardes.
\par 10 E, saindo aqueles servos pelas estradas, reuniram todos os que encontraram, maus e bons; e a sala do banquete ficou repleta de convidados.
\par 11 Entrando, porém, o rei para ver os que estavam à mesa, notou ali um homem que não trazia veste nupcial
\par 12 e perguntou-lhe: Amigo, como entraste aqui sem veste nupcial? E ele emudeceu.
\par 13 Então, ordenou o rei aos serventes: Amarrai-o de pés e mãos e lançai-o para fora, nas trevas; ali haverá choro e ranger de dentes.
\par 14 Porque muitos são chamados, mas poucos, escolhidos.
\par 15 Então, retirando-se os fariseus, consultaram entre si como o surpreenderiam em alguma palavra.
\par 16 E enviaram-lhe discípulos, juntamente com os herodianos, para dizer-lhe: Mestre, sabemos que és verdadeiro e que ensinas o caminho de Deus, de acordo com a verdade, sem te importares com quem quer que seja, porque não olhas a aparência dos homens.
\par 17 Dize-nos, pois: que te parece? É lícito pagar tributo a César ou não?
\par 18 Jesus, porém, conhecendo-lhes a malícia, respondeu: Por que me experimentais, hipócritas?
\par 19 Mostrai-me a moeda do tributo. Trouxeram-lhe um denário.
\par 20 E ele lhes perguntou: De quem é esta efígie e inscrição?
\par 21 Responderam: De César. Então, lhes disse: Dai, pois, a César o que é de César e a Deus o que é de Deus.
\par 22 Ouvindo isto, se admiraram e, deixando-o, foram-se.
\par 23 Naquele dia, aproximaram-se dele alguns saduceus, que dizem não haver ressurreição, e lhe perguntaram:
\par 24 Mestre, Moisés disse: Se alguém morrer, não tendo filhos, seu irmão casará com a viúva e suscitará descendência ao falecido.
\par 25 Ora, havia entre nós sete irmãos. O primeiro, tendo casado, morreu e, não tendo descendência, deixou sua mulher a seu irmão;
\par 26 o mesmo sucedeu com o segundo, com o terceiro, até ao sétimo;
\par 27 depois de todos eles, morreu também a mulher.
\par 28 Portanto, na ressurreição, de qual dos sete será ela esposa? Porque todos a desposaram.
\par 29 Respondeu-lhes Jesus: Errais, não conhecendo as Escrituras nem o poder de Deus.
\par 30 Porque, na ressurreição, nem casam, nem se dão em casamento; são, porém, como os anjos no céu.
\par 31 E, quanto à ressurreição dos mortos, não tendes lido o que Deus vos declarou:
\par 32 Eu sou o Deus de Abraão, o Deus de Isaque e o Deus de Jacó? Ele não é Deus de mortos, e sim de vivos.
\par 33 Ouvindo isto, as multidões se maravilhavam da sua doutrina.
\par 34 Entretanto, os fariseus, sabendo que ele fizera calar os saduceus, reuniram-se em conselho.
\par 35 E um deles, intérprete da Lei, experimentando-o, lhe perguntou:
\par 36 Mestre, qual é o grande mandamento na Lei?
\par 37 Respondeu-lhe Jesus: Amarás o Senhor, teu Deus, de todo o teu coração, de toda a tua alma e de todo o teu entendimento.
\par 38 Este é o grande e primeiro mandamento.
\par 39 O segundo, semelhante a este, é: Amarás o teu próximo como a ti mesmo.
\par 40 Destes dois mandamentos dependem toda a Lei e os Profetas.
\par 41 Reunidos os fariseus, interrogou-os Jesus:
\par 42 Que pensais vós do Cristo? De quem é filho? Responderam-lhe eles: De Davi.
\par 43 Replicou-lhes Jesus: Como, pois, Davi, pelo Espírito, chama-lhe Senhor, dizendo:
\par 44 Disse o Senhor ao meu Senhor: Assenta-te à minha direita, até que eu ponha os teus inimigos debaixo dos teus pés?
\par 45 Se Davi, pois, lhe chama Senhor, como é ele seu filho?
\par 46 E ninguém lhe podia responder palavra, nem ousou alguém, a partir daquele dia, fazer-lhe perguntas.

\chapter{23}

\par 1 Então, falou Jesus às multidões e aos seus discípulos:
\par 2 Na cadeira de Moisés, se assentaram os escribas e os fariseus.
\par 3 Fazei e guardai, pois, tudo quanto eles vos disserem, porém não os imiteis nas suas obras; porque dizem e não fazem.
\par 4 Atam fardos pesados [e difíceis de carregar] e os põem sobre os ombros dos homens; entretanto, eles mesmos nem com o dedo querem movê-los.
\par 5 Praticam, porém, todas as suas obras com o fim de serem vistos dos homens; pois alargam os seus filactérios e alongam as suas franjas.
\par 6 Amam o primeiro lugar nos banquetes e as primeiras cadeiras nas sinagogas,
\par 7 as saudações nas praças e o serem chamados mestres pelos homens.
\par 8 Vós, porém, não sereis chamados mestres, porque um só é vosso Mestre, e vós todos sois irmãos.
\par 9 A ninguém sobre a terra chameis vosso pai; porque só um é vosso Pai, aquele que está nos céus.
\par 10 Nem sereis chamados guias, porque um só é vosso Guia, o Cristo.
\par 11 Mas o maior dentre vós será vosso servo.
\par 12 Quem a si mesmo se exaltar será humilhado; e quem a si mesmo se humilhar será exaltado.
\par 13 Ai de vós, escribas e fariseus, hipócritas, porque fechais o reino dos céus diante dos homens; pois vós não entrais, nem deixais entrar os que estão entrando!
\par 14 [Ai de vós, escribas e fariseus, hipócritas, porque devorais as casas das viúvas e, para o justificar, fazeis longas orações; por isso, sofrereis juízo muito mais severo!]
\par 15 Ai de vós, escribas e fariseus, hipócritas, porque rodeais o mar e a terra para fazer um prosélito; e, uma vez feito, o tornais filho do inferno duas vezes mais do que vós!
\par 16 Ai de vós, guias cegos, que dizeis: Quem jurar pelo santuário, isso é nada; mas, se alguém jurar pelo ouro do santuário, fica obrigado pelo que jurou!
\par 17 Insensatos e cegos! Pois qual é maior: o ouro ou o santuário que santifica o ouro?
\par 18 E dizeis: Quem jurar pelo altar, isso é nada; quem, porém, jurar pela oferta que está sobre o altar fica obrigado pelo que jurou.
\par 19 Cegos! Pois qual é maior: a oferta ou o altar que santifica a oferta?
\par 20 Portanto, quem jurar pelo altar jura por ele e por tudo o que sobre ele está.
\par 21 Quem jurar pelo santuário jura por ele e por aquele que nele habita;
\par 22 e quem jurar pelo céu jura pelo trono de Deus e por aquele que no trono está sentado.
\par 23 Ai de vós, escribas e fariseus, hipócritas, porque dais o dízimo da hortelã, do endro e do cominho e tendes negligenciado os preceitos mais importantes da Lei: a justiça, a misericórdia e a fé; devíeis, porém, fazer estas coisas, sem omitir aquelas!
\par 24 Guias cegos, que coais o mosquito e engolis o camelo!
\par 25 Ai de vós, escribas e fariseus, hipócritas, porque limpais o exterior do copo e do prato, mas estes, por dentro, estão cheios de rapina e intemperança!
\par 26 Fariseu cego, limpa primeiro o interior do copo, para que também o seu exterior fique limpo!
\par 27 Ai de vós, escribas e fariseus, hipócritas, porque sois semelhantes aos sepulcros caiados, que, por fora, se mostram belos, mas interiormente estão cheios de ossos de mortos e de toda imundícia!
\par 28 Assim também vós exteriormente pareceis justos aos homens, mas, por dentro, estais cheios de hipocrisia e de iniqüidade.
\par 29 Ai de vós, escribas e fariseus, hipócritas, porque edificais os sepulcros dos profetas, adornais os túmulos dos justos
\par 30 e dizeis: Se tivéssemos vivido nos dias de nossos pais, não teríamos sido seus cúmplices no sangue dos profetas!
\par 31 Assim, contra vós mesmos, testificais que sois filhos dos que mataram os profetas.
\par 32 Enchei vós, pois, a medida de vossos pais.
\par 33 Serpentes, raça de víboras! Como escapareis da condenação do inferno?
\par 34 Por isso, eis que eu vos envio profetas, sábios e escribas. A uns matareis e crucificareis; a outros açoitareis nas vossas sinagogas e perseguireis de cidade em cidade;
\par 35 para que sobre vós recaia todo o sangue justo derramado sobre a terra, desde o sangue do justo Abel até ao sangue de Zacarias, filho de Baraquias, a quem matastes entre o santuário e o altar.
\par 36 Em verdade vos digo que todas estas coisas hão de vir sobre a presente geração.
\par 37 Jerusalém, Jerusalém, que matas os profetas e apedrejas os que te foram enviados! Quantas vezes quis eu reunir os teus filhos, como a galinha ajunta os seus pintinhos debaixo das asas, e vós não o quisestes!
\par 38 Eis que a vossa casa vos ficará deserta.
\par 39 Declaro-vos, pois, que, desde agora, já não me vereis, até que venhais a dizer: Bendito o que vem em nome do Senhor!

\chapter{24}

\par 1 Tendo Jesus saído do templo, ia-se retirando, quando se aproximaram dele os seus discípulos para lhe mostrar as construções do templo.
\par 2 Ele, porém, lhes disse: Não vedes tudo isto? Em verdade vos digo que não ficará aqui pedra sobre pedra que não seja derribada.
\par 3 No monte das Oliveiras, achava-se Jesus assentado, quando se aproximaram dele os discípulos, em particular, e lhe pediram: Dize-nos quando sucederão estas coisas e que sinal haverá da tua vinda e da consumação do século.
\par 4 E ele lhes respondeu: Vede que ninguém vos engane.
\par 5 Porque virão muitos em meu nome, dizendo: Eu sou o Cristo, e enganarão a muitos.
\par 6 E, certamente, ouvireis falar de guerras e rumores de guerras; vede, não vos assusteis, porque é necessário assim acontecer, mas ainda não é o fim.
\par 7 Porquanto se levantará nação contra nação, reino contra reino, e haverá fomes e terremotos em vários lugares;
\par 8 porém tudo isto é o princípio das dores.
\par 9 Então, sereis atribulados, e vos matarão. Sereis odiados de todas as nações, por causa do meu nome.
\par 10 Nesse tempo, muitos hão de se escandalizar, trair e odiar uns aos outros;
\par 11 levantar-se-ão muitos falsos profetas e enganarão a muitos.
\par 12 E, por se multiplicar a iniqüidade, o amor se esfriará de quase todos.
\par 13 Aquele, porém, que perseverar até o fim, esse será salvo.
\par 14 E será pregado este evangelho do reino por todo o mundo, para testemunho a todas as nações. Então, virá o fim.
\par 15 Quando, pois, virdes o abominável da desolação de que falou o profeta Daniel, no lugar santo (quem lê entenda),
\par 16 então, os que estiverem na Judéia fujam para os montes;
\par 17 quem estiver sobre o eirado não desça a tirar de casa alguma coisa;
\par 18 e quem estiver no campo não volte atrás para buscar a sua capa.
\par 19 Ai das que estiverem grávidas e das que amamentarem naqueles dias!
\par 20 Orai para que a vossa fuga não se dê no inverno, nem no sábado;
\par 21 porque nesse tempo haverá grande tribulação, como desde o princípio do mundo até agora não tem havido e nem haverá jamais.
\par 22 Não tivessem aqueles dias sido abreviados, ninguém seria salvo; mas, por causa dos escolhidos, tais dias serão abreviados.
\par 23 Então, se alguém vos disser: Eis aqui o Cristo! Ou: Ei-lo ali! Não acrediteis;
\par 24 porque surgirão falsos cristos e falsos profetas operando grandes sinais e prodígios para enganar, se possível, os próprios eleitos.
\par 25 Vede que vo-lo tenho predito.
\par 26 Portanto, se vos disserem: Eis que ele está no deserto!, não saiais. Ou: Ei-lo no interior da casa!, não acrediteis.
\par 27 Porque, assim como o relâmpago sai do oriente e se mostra até no ocidente, assim há de ser a vinda do Filho do Homem.
\par 28 Onde estiver o cadáver, aí se ajuntarão os abutres.
\par 29 Logo em seguida à tribulação daqueles dias, o sol escurecerá, a lua não dará a sua claridade, as estrelas cairão do firmamento, e os poderes dos céus serão abalados.
\par 30 Então, aparecerá no céu o sinal do Filho do Homem; todos os povos da terra se lamentarão e verão o Filho do Homem vindo sobre as nuvens do céu, com poder e muita glória.
\par 31 E ele enviará os seus anjos, com grande clangor de trombeta, os quais reunirão os seus escolhidos, dos quatro ventos, de uma a outra extremidade dos céus.
\par 32 Aprendei, pois, a parábola da figueira: quando já os seus ramos se renovam e as folhas brotam, sabeis que está próximo o verão.
\par 33 Assim também vós: quando virdes todas estas coisas, sabei que está próximo, às portas.
\par 34 Em verdade vos digo que não passará esta geração sem que tudo isto aconteça.
\par 35 Passará o céu e a terra, porém as minhas palavras não passarão.
\par 36 Mas a respeito daquele dia e hora ninguém sabe, nem os anjos dos céus, nem o Filho, senão o Pai.
\par 37 Pois assim como foi nos dias de Noé, também será a vinda do Filho do Homem.
\par 38 Porquanto, assim como nos dias anteriores ao dilúvio comiam e bebiam, casavam e davam-se em casamento, até ao dia em que Noé entrou na arca,
\par 39 e não o perceberam, senão quando veio o dilúvio e os levou a todos, assim será também a vinda do Filho do Homem.
\par 40 Então, dois estarão no campo, um será tomado, e deixado o outro;
\par 41 duas estarão trabalhando num moinho, uma será tomada, e deixada a outra.
\par 42 Portanto, vigiai, porque não sabeis em que dia vem o vosso Senhor.
\par 43 Mas considerai isto: se o pai de família soubesse a que hora viria o ladrão, vigiaria e não deixaria que fosse arrombada a sua casa.
\par 44 Por isso, ficai também vós apercebidos; porque, à hora em que não cuidais, o Filho do Homem virá.
\par 45 Quem é, pois, o servo fiel e prudente, a quem o senhor confiou os seus conservos para dar-lhes o sustento a seu tempo?
\par 46 Bem-aventurado aquele servo a quem seu senhor, quando vier, achar fazendo assim.
\par 47 Em verdade vos digo que lhe confiará todos os seus bens.
\par 48 Mas, se aquele servo, sendo mau, disser consigo mesmo: Meu senhor demora-se,
\par 49 e passar a espancar os seus companheiros e a comer e beber com ébrios,
\par 50 virá o senhor daquele servo em dia em que não o espera e em hora que não sabe
\par 51 e castigá-lo-á, lançando-lhe a sorte com os hipócritas; ali haverá choro e ranger de dentes.

\chapter{25}

\par 1 Então, o reino dos céus será semelhante a dez virgens que, tomando as suas lâmpadas, saíram a encontrar-se com o noivo.
\par 2 Cinco dentre elas eram néscias, e cinco, prudentes.
\par 3 As néscias, ao tomarem as suas lâmpadas, não levaram azeite consigo;
\par 4 no entanto, as prudentes, além das lâmpadas, levaram azeite nas vasilhas.
\par 5 E, tardando o noivo, foram todas tomadas de sono e adormeceram.
\par 6 Mas, à meia-noite, ouviu-se um grito: Eis o noivo! Saí ao seu encontro!
\par 7 Então, se levantaram todas aquelas virgens e prepararam as suas lâmpadas.
\par 8 E as néscias disseram às prudentes: Dai-nos do vosso azeite, porque as nossas lâmpadas estão-se apagando.
\par 9 Mas as prudentes responderam: Não, para que não nos falte a nós e a vós outras! Ide, antes, aos que o vendem e comprai-o.
\par 10 E, saindo elas para comprar, chegou o noivo, e as que estavam apercebidas entraram com ele para as bodas; e fechou-se a porta.
\par 11 Mais tarde, chegaram as virgens néscias, clamando: Senhor, senhor, abre-nos a porta!
\par 12 Mas ele respondeu: Em verdade vos digo que não vos conheço.
\par 13 Vigiai, pois, porque não sabeis o dia nem a hora.
\par 14 Pois será como um homem que, ausentando-se do país, chamou os seus servos e lhes confiou os seus bens.
\par 15 A um deu cinco talentos, a outro, dois e a outro, um, a cada um segundo a sua própria capacidade; e, então, partiu.
\par 16 O que recebera cinco talentos saiu imediatamente a negociar com eles e ganhou outros cinco.
\par 17 Do mesmo modo, o que recebera dois ganhou outros dois.
\par 18 Mas o que recebera um, saindo, abriu uma cova e escondeu o dinheiro do seu senhor.
\par 19 Depois de muito tempo, voltou o senhor daqueles servos e ajustou contas com eles.
\par 20 Então, aproximando-se o que recebera cinco talentos, entregou outros cinco, dizendo: Senhor, confiaste-me cinco talentos; eis aqui outros cinco talentos que ganhei.
\par 21 Disse-lhe o senhor: Muito bem, servo bom e fiel; foste fiel no pouco, sobre o muito te colocarei; entra no gozo do teu senhor.
\par 22 E, aproximando-se também o que recebera dois talentos, disse: Senhor, dois talentos me confiaste; aqui tens outros dois que ganhei.
\par 23 Disse-lhe o senhor: Muito bem, servo bom e fiel; foste fiel no pouco, sobre o muito te colocarei; entra no gozo do teu senhor.
\par 24 Chegando, por fim, o que recebera um talento, disse: Senhor, sabendo que és homem severo, que ceifas onde não semeaste e ajuntas onde não espalhaste,
\par 25 receoso, escondi na terra o teu talento; aqui tens o que é teu.
\par 26 Respondeu-lhe, porém, o senhor: Servo mau e negligente, sabias que ceifo onde não semeei e ajunto onde não espalhei?
\par 27 Cumpria, portanto, que entregasses o meu dinheiro aos banqueiros, e eu, ao voltar, receberia com juros o que é meu.
\par 28 Tirai-lhe, pois, o talento e dai-o ao que tem dez.
\par 29 Porque a todo o que tem se lhe dará, e terá em abundância; mas ao que não tem, até o que tem lhe será tirado.
\par 30 E o servo inútil, lançai-o para fora, nas trevas. Ali haverá choro e ranger de dentes.
\par 31 Quando vier o Filho do Homem na sua majestade e todos os anjos com ele, então, se assentará no trono da sua glória;
\par 32 e todas as nações serão reunidas em sua presença, e ele separará uns dos outros, como o pastor separa dos cabritos as ovelhas;
\par 33 e porá as ovelhas à sua direita, mas os cabritos, à esquerda;
\par 34 então, dirá o Rei aos que estiverem à sua direita: Vinde, benditos de meu Pai! Entrai na posse do reino que vos está preparado desde a fundação do mundo.
\par 35 Porque tive fome, e me destes de comer; tive sede, e me destes de beber; era forasteiro, e me hospedastes;
\par 36 estava nu, e me vestistes; enfermo, e me visitastes; preso, e fostes ver-me.
\par 37 Então, perguntarão os justos: Senhor, quando foi que te vimos com fome e te demos de comer? Ou com sede e te demos de beber?
\par 38 E quando te vimos forasteiro e te hospedamos? Ou nu e te vestimos?
\par 39 E quando te vimos enfermo ou preso e te fomos visitar?
\par 40 O Rei, respondendo, lhes dirá: Em verdade vos afirmo que, sempre que o fizestes a um destes meus pequeninos irmãos, a mim o fizestes.
\par 41 Então, o Rei dirá também aos que estiverem à sua esquerda: Apartai-vos de mim, malditos, para o fogo eterno, preparado para o diabo e seus anjos.
\par 42 Porque tive fome, e não me destes de comer; tive sede, e não me destes de beber;
\par 43 sendo forasteiro, não me hospedastes; estando nu, não me vestistes; achando-me enfermo e preso, não fostes ver-me.
\par 44 E eles lhe perguntarão: Senhor, quando foi que te vimos com fome, com sede, forasteiro, nu, enfermo ou preso e não te assistimos?
\par 45 Então, lhes responderá: Em verdade vos digo que, sempre que o deixastes de fazer a um destes mais pequeninos, a mim o deixastes de fazer.
\par 46 E irão estes para o castigo eterno, porém os justos, para a vida eterna.

\chapter{26}

\par 1 Tendo Jesus acabado todos estes ensinamentos, disse a seus discípulos:
\par 2 Sabeis que, daqui a dois dias, celebrar-se-á a Páscoa; e o Filho do Homem será entregue para ser crucificado.
\par 3 Então, os principais sacerdotes e os anciãos do povo se reuniram no palácio do sumo sacerdote, chamado Caifás;
\par 4 e deliberaram prender Jesus, à traição, e matá-lo.
\par 5 Mas diziam: Não durante a festa, para que não haja tumulto entre o povo.
\par 6 Ora, estando Jesus em Betânia, em casa de Simão, o leproso,
\par 7 aproximou-se dele uma mulher, trazendo um vaso de alabastro cheio de precioso bálsamo, que lhe derramou sobre a cabeça, estando ele à mesa.
\par 8 Vendo isto, indignaram-se os discípulos e disseram: Para que este desperdício?
\par 9 Pois este perfume podia ser vendido por muito dinheiro e dar-se aos pobres.
\par 10 Mas Jesus, sabendo disto, disse-lhes: Por que molestais esta mulher? Ela praticou boa ação para comigo.
\par 11 Porque os pobres, sempre os tendes convosco, mas a mim nem sempre me tendes;
\par 12 pois, derramando este perfume sobre o meu corpo, ela o fez para o meu sepultamento.
\par 13 Em verdade vos digo: Onde for pregado em todo o mundo este evangelho, será também contado o que ela fez, para memória sua.
\par 14 Então, um dos doze, chamado Judas Iscariotes, indo ter com os principais sacerdotes, propôs:
\par 15 Que me quereis dar, e eu vo-lo entregarei? E pagaram-lhe trinta moedas de prata.
\par 16 E, desse momento em diante, buscava ele uma boa ocasião para o entregar.
\par 17 No primeiro dia da Festa dos Pães Asmos, vieram os discípulos a Jesus e lhe perguntaram: Onde queres que te façamos os preparativos para comeres a Páscoa?
\par 18 E ele lhes respondeu: Ide à cidade ter com certo homem e dizei-lhe: O Mestre manda dizer: O meu tempo está próximo; em tua casa celebrarei a Páscoa com os meus discípulos.
\par 19 E eles fizeram como Jesus lhes ordenara e prepararam a Páscoa.
\par 20 Chegada a tarde, pôs-se ele à mesa com os doze discípulos.
\par 21 E, enquanto comiam, declarou Jesus: Em verdade vos digo que um dentre vós me trairá.
\par 22 E eles, muitíssimo contristados, começaram um por um a perguntar-lhe: Porventura, sou eu, Senhor?
\par 23 E ele respondeu: O que mete comigo a mão no prato, esse me trairá.
\par 24 O Filho do Homem vai, como está escrito a seu respeito, mas ai daquele por intermédio de quem o Filho do Homem está sendo traído! Melhor lhe fora não haver nascido!
\par 25 Então, Judas, que o traía, perguntou: Acaso, sou eu, Mestre? Respondeu-lhe Jesus: Tu o disseste.
\par 26 Enquanto comiam, tomou Jesus um pão, e, abençoando-o, o partiu, e o deu aos discípulos, dizendo: Tomai, comei; isto é o meu corpo.
\par 27 A seguir, tomou um cálice e, tendo dado graças, o deu aos discípulos, dizendo: Bebei dele todos;
\par 28 porque isto é o meu sangue, o sangue da [nova] aliança, derramado em favor de muitos, para remissão de pecados.
\par 29 E digo-vos que, desta hora em diante, não beberei deste fruto da videira, até aquele dia em que o hei de beber, novo, convosco no reino de meu Pai.
\par 30 E, tendo cantado um hino, saíram para o monte das Oliveiras.
\par 31 Então, Jesus lhes disse: Esta noite, todos vós vos escandalizareis comigo; porque está escrito: Ferirei o pastor, e as ovelhas do rebanho ficarão dispersas.
\par 32 Mas, depois da minha ressurreição, irei adiante de vós para a Galiléia.
\par 33 Disse-lhe Pedro: Ainda que venhas a ser um tropeço para todos, nunca o serás para mim.
\par 34 Replicou-lhe Jesus: Em verdade te digo que, nesta mesma noite, antes que o galo cante, tu me negarás três vezes.
\par 35 Disse-lhe Pedro: Ainda que me seja necessário morrer contigo, de nenhum modo te negarei. E todos os discípulos disseram o mesmo.
\par 36 Em seguida, foi Jesus com eles a um lugar chamado Getsêmani e disse a seus discípulos: Assentai-vos aqui, enquanto eu vou ali orar;
\par 37 e, levando consigo a Pedro e aos dois filhos de Zebedeu, começou a entristecer-se e a angustiar-se.
\par 38 Então, lhes disse: A minha alma está profundamente triste até à morte; ficai aqui e vigiai comigo.
\par 39 Adiantando-se um pouco, prostrou-se sobre o seu rosto, orando e dizendo: Meu Pai, se possível, passe de mim este cálice! Todavia, não seja como eu quero, e sim como tu queres.
\par 40 E, voltando para os discípulos, achou-os dormindo; e disse a Pedro: Então, nem uma hora pudestes vós vigiar comigo?
\par 41 Vigiai e orai, para que não entreis em tentação; o espírito, na verdade, está pronto, mas a carne é fraca.
\par 42 Tornando a retirar-se, orou de novo, dizendo: Meu Pai, se não é possível passar de mim este cálice sem que eu o beba, faça-se a tua vontade.
\par 43 E, voltando, achou-os outra vez dormindo; porque os seus olhos estavam pesados.
\par 44 Deixando-os novamente, foi orar pela terceira vez, repetindo as mesmas palavras.
\par 45 Então, voltou para os discípulos e lhes disse: Ainda dormis e repousais! Eis que é chegada a hora, e o Filho do Homem está sendo entregue nas mãos de pecadores.
\par 46 Levantai-vos, vamos! Eis que o traidor se aproxima.
\par 47 Falava ele ainda, e eis que chegou Judas, um dos doze, e, com ele, grande turba com espadas e porretes, vinda da parte dos principais sacerdotes e dos anciãos do povo.
\par 48 Ora, o traidor lhes tinha dado este sinal: Aquele a quem eu beijar, é esse; prendei-o.
\par 49 E logo, aproximando-se de Jesus, lhe disse: Salve, Mestre! E o beijou.
\par 50 Jesus, porém, lhe disse: Amigo, para que vieste? Nisto, aproximando-se eles, deitaram as mãos em Jesus e o prenderam.
\par 51 E eis que um dos que estavam com Jesus, estendendo a mão, sacou da espada e, golpeando o servo do sumo sacerdote, cortou-lhe a orelha.
\par 52 Então, Jesus lhe disse: Embainha a tua espada; pois todos os que lançam mão da espada à espada perecerão.
\par 53 Acaso, pensas que não posso rogar a meu Pai, e ele me mandaria neste momento mais de doze legiões de anjos?
\par 54 Como, pois, se cumpririam as Escrituras, segundo as quais assim deve suceder?
\par 55 Naquele momento, disse Jesus às multidões: Saístes com espadas e porretes para prender-me, como a um salteador? Todos os dias, no templo, eu me assentava [convosco] ensinando, e não me prendestes.
\par 56 Tudo isto, porém, aconteceu para que se cumprissem as Escrituras dos profetas. Então, os discípulos todos, deixando-o, fugiram.
\par 57 E os que prenderam Jesus o levaram à casa de Caifás, o sumo sacerdote, onde se haviam reunido os escribas e os anciãos.
\par 58 Mas Pedro o seguia de longe até ao pátio do sumo sacerdote e, tendo entrado, assentou-se entre os serventuários, para ver o fim.
\par 59 Ora, os principais sacerdotes e todo o Sinédrio procuravam algum testemunho falso contra Jesus, a fim de o condenarem à morte.
\par 60 E não acharam, apesar de se terem apresentado muitas testemunhas falsas. Mas, afinal, compareceram duas, afirmando:
\par 61 Este disse: Posso destruir o santuário de Deus e reedificá-lo em três dias.
\par 62 E, levantando-se o sumo sacerdote, perguntou a Jesus: Nada respondes ao que estes depõem contra ti?
\par 63 Jesus, porém, guardou silêncio. E o sumo sacerdote lhe disse: Eu te conjuro pelo Deus vivo que nos digas se tu és o Cristo, o Filho de Deus.
\par 64 Respondeu-lhe Jesus: Tu o disseste; entretanto, eu vos declaro que, desde agora, vereis o Filho do Homem assentado à direita do Todo-Poderoso e vindo sobre as nuvens do céu.
\par 65 Então, o sumo sacerdote rasgou as suas vestes, dizendo: Blasfemou! Que necessidade mais temos de testemunhas? Eis que ouvistes agora a blasfêmia!
\par 66 Que vos parece? Responderam eles: É réu de morte.
\par 67 Então, uns cuspiram-lhe no rosto e lhe davam murros, e outros o esbofeteavam, dizendo:
\par 68 Profetiza-nos, ó Cristo, quem é que te bateu!
\par 69 Ora, estava Pedro assentado fora no pátio; e, aproximando-se uma criada, lhe disse: Também tu estavas com Jesus, o galileu.
\par 70 Ele, porém, o negou diante de todos, dizendo: Não sei o que dizes.
\par 71 E, saindo para o alpendre, foi ele visto por outra criada, a qual disse aos que ali estavam: Este também estava com Jesus, o Nazareno.
\par 72 E ele negou outra vez, com juramento: Não conheço tal homem.
\par 73 Logo depois, aproximando-se os que ali estavam, disseram a Pedro: Verdadeiramente, és também um deles, porque o teu modo de falar o denuncia.
\par 74 Então, começou ele a praguejar e a jurar: Não conheço esse homem! E imediatamente cantou o galo.
\par 75 Então, Pedro se lembrou da palavra que Jesus lhe dissera: Antes que o galo cante, tu me negarás três vezes. E, saindo dali, chorou amargamente.

\chapter{27}

\par 1 Ao romper o dia, todos os principais sacerdotes e os anciãos do povo entraram em conselho contra Jesus, para o matarem;
\par 2 e, amarrando-o, levaram-no e o entregaram ao governador Pilatos.
\par 3 Então, Judas, o que o traiu, vendo que Jesus fora condenado, tocado de remorso, devolveu as trinta moedas de prata aos principais sacerdotes e aos anciãos, dizendo:
\par 4 Pequei, traindo sangue inocente. Eles, porém, responderam: Que nos importa? Isso é contigo.
\par 5 Então, Judas, atirando para o santuário as moedas de prata, retirou-se e foi enforcar-se.
\par 6 E os principais sacerdotes, tomando as moedas, disseram: Não é lícito deitá-las no cofre das ofertas, porque é preço de sangue.
\par 7 E, tendo deliberado, compraram com elas o campo do oleiro, para cemitério de forasteiros.
\par 8 Por isso, aquele campo tem sido chamado, até ao dia de hoje, Campo de Sangue.
\par 9 Então, se cumpriu o que foi dito por intermédio do profeta Jeremias: Tomaram as trinta moedas de prata, preço em que foi estimado aquele a quem alguns dos filhos de Israel avaliaram;
\par 10 e as deram pelo campo do oleiro, assim como me ordenou o Senhor.
\par 11 Jesus estava em pé ante o governador; e este o interrogou, dizendo: És tu o rei dos judeus? Respondeu-lhe Jesus: Tu o dizes.
\par 12 E, sendo acusado pelos principais sacerdotes e pelos anciãos, nada respondeu.
\par 13 Então, lhe perguntou Pilatos: Não ouves quantas acusações te fazem?
\par 14 Jesus não respondeu nem uma palavra, vindo com isto a admirar-se grandemente o governador.
\par 15 Ora, por ocasião da festa, costumava o governador soltar ao povo um dos presos, conforme eles quisessem.
\par 16 Naquela ocasião, tinham eles um preso muito conhecido, chamado Barrabás.
\par 17 Estando, pois, o povo reunido, perguntou-lhes Pilatos: A quem quereis que eu vos solte, a Barrabás ou a Jesus, chamado Cristo?
\par 18 Porque sabia que, por inveja, o tinham entregado.
\par 19 E, estando ele no tribunal, sua mulher mandou dizer-lhe: Não te envolvas com esse justo; porque hoje, em sonho, muito sofri por seu respeito.
\par 20 Mas os principais sacerdotes e os anciãos persuadiram o povo a que pedisse Barrabás e fizesse morrer Jesus.
\par 21 De novo, perguntou-lhes o governador: Qual dos dois quereis que eu vos solte? Responderam eles: Barrabás!
\par 22 Replicou-lhes Pilatos: Que farei, então, de Jesus, chamado Cristo? Seja crucificado! Responderam todos.
\par 23 Que mal fez ele? Perguntou Pilatos. Porém cada vez clamavam mais: Seja crucificado!
\par 24 Vendo Pilatos que nada conseguia, antes, pelo contrário, aumentava o tumulto, mandando vir água, lavou as mãos perante o povo, dizendo: Estou inocente do sangue deste [justo]; fique o caso convosco!
\par 25 E o povo todo respondeu: Caia sobre nós o seu sangue e sobre nossos filhos!
\par 26 Então, Pilatos lhes soltou Barrabás; e, após haver açoitado a Jesus, entregou-o para ser crucificado.
\par 27 Logo a seguir, os soldados do governador, levando Jesus para o pretório, reuniram em torno dele toda a coorte.
\par 28 Despojando-o das vestes, cobriram-no com um manto escarlate;
\par 29 tecendo uma coroa de espinhos, puseram-lha na cabeça e, na mão direita, um caniço; e, ajoelhando-se diante dele, o escarneciam, dizendo: Salve, rei dos judeus!
\par 30 E, cuspindo nele, tomaram o caniço e davam-lhe com ele na cabeça.
\par 31 Depois de o terem escarnecido, despiram-lhe o manto e o vestiram com as suas próprias vestes. Em seguida, o levaram para ser crucificado.
\par 32 Ao saírem, encontraram um cireneu, chamado Simão, a quem obrigaram a carregar-lhe a cruz.
\par 33 E, chegando a um lugar chamado Gólgota, que significa Lugar da Caveira,
\par 34 deram-lhe a beber vinho com fel; mas ele, provando-o, não o quis beber.
\par 35 Depois de o crucificarem, repartiram entre si as suas vestes, tirando a sorte.
\par 36 E, assentados ali, o guardavam.
\par 37 Por cima da sua cabeça puseram escrita a sua acusação: ESTE É JESUS, O REI DOS JUDEUS.
\par 38 E foram crucificados com ele dois ladrões, um à sua direita, e outro à sua esquerda.
\par 39 Os que iam passando blasfemavam dele, meneando a cabeça e dizendo:
\par 40 Ó tu que destróis o santuário e em três dias o reedificas! Salva-te a ti mesmo, se és Filho de Deus, e desce da cruz!
\par 41 De igual modo, os principais sacerdotes, com os escribas e anciãos, escarnecendo, diziam:
\par 42 Salvou os outros, a si mesmo não pode salvar-se. É rei de Israel! Desça da cruz, e creremos nele.
\par 43 Confiou em Deus; pois venha livrá-lo agora, se, de fato, lhe quer bem; porque disse: Sou Filho de Deus.
\par 44 E os mesmos impropérios lhe diziam também os ladrões que haviam sido crucificados com ele.
\par 45 Desde a hora sexta até à hora nona, houve trevas sobre toda a terra.
\par 46 Por volta da hora nona, clamou Jesus em alta voz, dizendo: Eli, Eli, lamá sabactâni? O que quer dizer: Deus meu, Deus meu, por que me desamparaste?
\par 47 E alguns dos que ali estavam, ouvindo isto, diziam: Ele chama por Elias.
\par 48 E, logo, um deles correu a buscar uma esponja e, tendo-a embebido de vinagre e colocado na ponta de um caniço, deu-lhe a beber.
\par 49 Os outros, porém, diziam: Deixa, vejamos se Elias vem salvá-lo.
\par 50 E Jesus, clamando outra vez com grande voz, entregou o espírito.
\par 51 Eis que o véu do santuário se rasgou em duas partes de alto a baixo; tremeu a terra, fenderam-se as rochas;
\par 52 abriram-se os sepulcros, e muitos corpos de santos, que dormiam, ressuscitaram;
\par 53 e, saindo dos sepulcros depois da ressurreição de Jesus, entraram na cidade santa e apareceram a muitos.
\par 54 O centurião e os que com ele guardavam a Jesus, vendo o terremoto e tudo o que se passava, ficaram possuídos de grande temor e disseram: Verdadeiramente este era Filho de Deus.
\par 55 Estavam ali muitas mulheres, observando de longe; eram as que vinham seguindo a Jesus desde a Galiléia, para o servirem;
\par 56 entre elas estavam Maria Madalena, Maria, mãe de Tiago e de José, e a mulher de Zebedeu.
\par 57 Caindo a tarde, veio um homem rico de Arimatéia, chamado José, que era também discípulo de Jesus.
\par 58 Este foi ter com Pilatos e lhe pediu o corpo de Jesus. Então, Pilatos mandou que lho fosse entregue.
\par 59 E José, tomando o corpo, envolveu-o num pano limpo de linho
\par 60 e o depositou no seu túmulo novo, que fizera abrir na rocha; e, rolando uma grande pedra para a entrada do sepulcro, se retirou.
\par 61 Achavam-se ali, sentadas em frente da sepultura, Maria Madalena e a outra Maria.
\par 62 No dia seguinte, que é o dia depois da preparação, reuniram-se os principais sacerdotes e os fariseus e, dirigindo-se a Pilatos,
\par 63 disseram-lhe: Senhor, lembramo-nos de que aquele embusteiro, enquanto vivia, disse: Depois de três dias ressuscitarei.
\par 64 Ordena, pois, que o sepulcro seja guardado com segurança até ao terceiro dia, para não suceder que, vindo os discípulos, o roubem e depois digam ao povo: Ressuscitou dos mortos; e será o último embuste pior que o primeiro.
\par 65 Disse-lhes Pilatos: Aí tendes uma escolta; ide e guardai o sepulcro como bem vos parecer.
\par 66 Indo eles, montaram guarda ao sepulcro, selando a pedra e deixando ali a escolta.

\chapter{28}

\par 1 No findar do sábado, ao entrar o primeiro dia da semana, Maria Madalena e a outra Maria foram ver o sepulcro.
\par 2 E eis que houve um grande terremoto; porque um anjo do Senhor desceu do céu, chegou-se, removeu a pedra e assentou-se sobre ela.
\par 3 O seu aspecto era como um relâmpago, e a sua veste, alva como a neve.
\par 4 E os guardas tremeram espavoridos e ficaram como se estivessem mortos.
\par 5 Mas o anjo, dirigindo-se às mulheres, disse: Não temais; porque sei que buscais Jesus, que foi crucificado.
\par 6 Ele não está aqui; ressuscitou, como tinha dito. Vinde ver onde ele jazia.
\par 7 Ide, pois, depressa e dizei aos seus discípulos que ele ressuscitou dos mortos e vai adiante de vós para a Galiléia; ali o vereis. É como vos digo!
\par 8 E, retirando-se elas apressadamente do sepulcro, tomadas de medo e grande alegria, correram a anunciá-lo aos discípulos.
\par 9 E eis que Jesus veio ao encontro delas e disse: Salve! E elas, aproximando-se, abraçaram-lhe os pés e o adoraram.
\par 10 Então, Jesus lhes disse: Não temais! Ide avisar a meus irmãos que se dirijam à Galiléia e lá me verão.
\par 11 E, indo elas, eis que alguns da guarda foram à cidade e contaram aos principais sacerdotes tudo o que sucedera.
\par 12 Reunindo-se eles em conselho com os anciãos, deram grande soma de dinheiro aos soldados,
\par 13 recomendando-lhes que dissessem: Vieram de noite os discípulos dele e o roubaram enquanto dormíamos.
\par 14 Caso isto chegue ao conhecimento do governador, nós o persuadiremos e vos poremos em segurança.
\par 15 Eles, recebendo o dinheiro, fizeram como estavam instruídos. Esta versão divulgou-se entre os judeus até ao dia de hoje.
\par 16 Seguiram os onze discípulos para a Galiléia, para o monte que Jesus lhes designara.
\par 17 E, quando o viram, o adoraram; mas alguns duvidaram.
\par 18 Jesus, aproximando-se, falou-lhes, dizendo: Toda a autoridade me foi dada no céu e na terra.
\par 19 Ide, portanto, fazei discípulos de todas as nações, batizando-os em nome do Pai, e do Filho, e do Espírito Santo;
\par 20 ensinando-os a guardar todas as coisas que vos tenho ordenado. E eis que estou convosco todos os dias até à consumação do século.


\end{document}