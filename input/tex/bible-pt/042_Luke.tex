\begin{document}

\title{Lucas}


\chapter{1}

\par 1 Visto que muitos houve que empreenderam uma narração coordenada dos fatos que entre nós se realizaram,
\par 2 conforme nos transmitiram os que desde o princípio foram deles testemunhas oculares e ministros da palavra,
\par 3 igualmente a mim me pareceu bem, depois de acurada investigação de tudo desde sua origem, dar-te por escrito, excelentíssimo Teófilo, uma exposição em ordem,
\par 4 para que tenhas plena certeza das verdades em que foste instruído.
\par 5 Nos dias de Herodes, rei da Judéia, houve um sacerdote chamado Zacarias, do turno de Abias. Sua mulher era das filhas de Arão e se chamava Isabel.
\par 6 Ambos eram justos diante de Deus, vivendo irrepreensivelmente em todos os preceitos e mandamentos do Senhor.
\par 7 E não tinham filhos, porque Isabel era estéril, sendo eles avançados em dias.
\par 8 Ora, aconteceu que, exercendo ele diante de Deus o sacerdócio na ordem do seu turno, coube-lhe por sorte,
\par 9 segundo o costume sacerdotal, entrar no santuário do Senhor para queimar o incenso;
\par 10 e, durante esse tempo, toda a multidão do povo permanecia da parte de fora, orando.
\par 11 E eis que lhe apareceu um anjo do Senhor, em pé, à direita do altar do incenso.
\par 12 Vendo-o, Zacarias turbou-se, e apoderou-se dele o temor.
\par 13 Disse-lhe, porém, o anjo: Zacarias, não temas, porque a tua oração foi ouvida; e Isabel, tua mulher, te dará à luz um filho, a quem darás o nome de João.
\par 14 Em ti haverá prazer e alegria, e muitos se regozijarão com o seu nascimento.
\par 15 Pois ele será grande diante do Senhor, não beberá vinho nem bebida forte e será cheio do Espírito Santo, já do ventre materno.
\par 16 E converterá muitos dos filhos de Israel ao Senhor, seu Deus.
\par 17 E irá adiante do Senhor no espírito e poder de Elias, para converter o coração dos pais aos filhos, converter os desobedientes à prudência dos justos e habilitar para o Senhor um povo preparado.
\par 18 Então, perguntou Zacarias ao anjo: Como saberei isto? Pois eu sou velho, e minha mulher, avançada em dias.
\par 19 Respondeu-lhe o anjo: Eu sou Gabriel, que assisto diante de Deus, e fui enviado para falar-te e trazer-te estas boas-novas.
\par 20 Todavia, ficarás mudo e não poderás falar até ao dia em que estas coisas venham a realizar-se; porquanto não acreditaste nas minhas palavras, as quais, a seu tempo, se cumprirão.
\par 21 O povo estava esperando a Zacarias e admirava-se de que tanto se demorasse no santuário.
\par 22 Mas, saindo ele, não lhes podia falar; então, entenderam que tivera uma visão no santuário. E expressava-se por acenos e permanecia mudo.
\par 23 Sucedeu que, terminados os dias de seu ministério, voltou para casa.
\par 24 Passados esses dias, Isabel, sua mulher, concebeu e ocultou-se por cinco meses, dizendo:
\par 25 Assim me fez o Senhor, contemplando-me, para anular o meu opróbrio perante os homens.
\par 26 No sexto mês, foi o anjo Gabriel enviado, da parte de Deus, para uma cidade da Galiléia, chamada Nazaré,
\par 27 a uma virgem desposada com certo homem da casa de Davi, cujo nome era José; a virgem chamava-se Maria.
\par 28 E, entrando o anjo aonde ela estava, disse: Alegra-te, muito favorecida! O Senhor é contigo.
\par 29 Ela, porém, ao ouvir esta palavra, perturbou-se muito e pôs-se a pensar no que significaria esta saudação.
\par 30 Mas o anjo lhe disse: Maria, não temas; porque achaste graça diante de Deus.
\par 31 Eis que conceberás e darás à luz um filho, a quem chamarás pelo nome de Jesus.
\par 32 Este será grande e será chamado Filho do Altíssimo; Deus, o Senhor, lhe dará o trono de Davi, seu pai;
\par 33 ele reinará para sempre sobre a casa de Jacó, e o seu reinado não terá fim.
\par 34 Então, disse Maria ao anjo: Como será isto, pois não tenho relação com homem algum?
\par 35 Respondeu-lhe o anjo: Descerá sobre ti o Espírito Santo, e o poder do Altíssimo te envolverá com a sua sombra; por isso, também o ente santo que há de nascer será chamado Filho de Deus.
\par 36 E Isabel, tua parenta, igualmente concebeu um filho na sua velhice, sendo este já o sexto mês para aquela que diziam ser estéril.
\par 37 Porque para Deus não haverá impossíveis em todas as suas promessas.
\par 38 Então, disse Maria: Aqui está a serva do Senhor; que se cumpra em mim conforme a tua palavra. E o anjo se ausentou dela.
\par 39 Naqueles dias, dispondo-se Maria, foi apressadamente à região montanhosa, a uma cidade de Judá,
\par 40 entrou na casa de Zacarias e saudou Isabel.
\par 41 Ouvindo esta a saudação de Maria, a criança lhe estremeceu no ventre; então, Isabel ficou possuída do Espírito Santo.
\par 42 E exclamou em alta voz: Bendita és tu entre as mulheres, e bendito o fruto do teu ventre!
\par 43 E de onde me provém que me venha visitar a mãe do meu Senhor?
\par 44 Pois, logo que me chegou aos ouvidos a voz da tua saudação, a criança estremeceu de alegria dentro de mim.
\par 45 Bem-aventurada a que creu, porque serão cumpridas as palavras que lhe foram ditas da parte do Senhor.
\par 46 Então, disse Maria: A minha alma engrandece ao Senhor,
\par 47 e o meu espírito se alegrou em Deus, meu Salvador,
\par 48 porque contemplou na humildade da sua serva. Pois, desde agora, todas as gerações me considerarão bem-aventurada,
\par 49 porque o Poderoso me fez grandes coisas. Santo é o seu nome.
\par 50 A sua misericórdia vai de geração em geração sobre os que o temem.
\par 51 Agiu com o seu braço valorosamente; dispersou os que, no coração, alimentavam pensamentos soberbos.
\par 52 Derribou do seu trono os poderosos e exaltou os humildes.
\par 53 Encheu de bens os famintos e despediu vazios os ricos.
\par 54 Amparou a Israel, seu servo, a fim de lembrar-se da sua misericórdia
\par 55 a favor de Abraão e de sua descendência, para sempre, como prometera aos nossos pais.
\par 56 Maria permaneceu cerca de três meses com Isabel e voltou para casa.
\par 57 A Isabel cumpriu-se o tempo de dar à luz, e teve um filho.
\par 58 Ouviram os seus vizinhos e parentes que o Senhor usara de grande misericórdia para com ela e participaram do seu regozijo.
\par 59 Sucedeu que, no oitavo dia, foram circuncidar o menino e queriam dar-lhe o nome de seu pai, Zacarias.
\par 60 De modo nenhum! Respondeu sua mãe. Pelo contrário, ele deve ser chamado João.
\par 61 Disseram-lhe: Ninguém há na tua parentela que tenha este nome.
\par 62 E perguntaram, por acenos, ao pai do menino que nome queria que lhe dessem.
\par 63 Então, pedindo ele uma tabuinha, escreveu: João é o seu nome. E todos se admiraram.
\par 64 Imediatamente, a boca se lhe abriu, e, desimpedida a língua, falava louvando a Deus.
\par 65 Sucedeu que todos os seus vizinhos ficaram possuídos de temor, e por toda a região montanhosa da Judéia foram divulgadas estas coisas.
\par 66 Todos os que as ouviram guardavam-nas no coração, dizendo: Que virá a ser, pois, este menino? E a mão do Senhor estava com ele.
\par 67 Zacarias, seu pai, cheio do Espírito Santo, profetizou, dizendo:
\par 68 Bendito seja o Senhor, Deus de Israel, porque visitou e redimiu o seu povo,
\par 69 e nos suscitou plena e poderosa salvação na casa de Davi, seu servo,
\par 70 como prometera, desde a antiguidade, por boca dos seus santos profetas,
\par 71 para nos libertar dos nossos inimigos e das mãos de todos os que nos odeiam;
\par 72 para usar de misericórdia com os nossos pais e lembrar-se da sua santa aliança
\par 73 e do juramento que fez a Abraão, o nosso pai,
\par 74 de conceder-nos que, livres das mãos de inimigos, o adorássemos sem temor,
\par 75 em santidade e justiça perante ele, todos os nossos dias.
\par 76 Tu, menino, serás chamado profeta do Altíssimo, porque precederás o Senhor, preparando-lhe os caminhos,
\par 77 para dar ao seu povo conhecimento da salvação, no redimi-lo dos seus pecados,
\par 78 graças à entranhável misericórdia de nosso Deus, pela qual nos visitará o sol nascente das alturas,
\par 79 para alumiar os que jazem nas trevas e na sombra da morte, e dirigir os nossos pés pelo caminho da paz.
\par 80 O menino crescia e se fortalecia em espírito. E viveu nos desertos até ao dia em que havia de manifestar-se a Israel.

\chapter{2}

\par 1 Naqueles dias, foi publicado um decreto de César Augusto, convocando toda a população do império para recensear-se.
\par 2 Este, o primeiro recenseamento, foi feito quando Quirino era governador da Síria.
\par 3 Todos iam alistar-se, cada um à sua própria cidade.
\par 4 José também subiu da Galiléia, da cidade de Nazaré, para a Judéia, à cidade de Davi, chamada Belém, por ser ele da casa e família de Davi,
\par 5 a fim de alistar-se com Maria, sua esposa, que estava grávida.
\par 6 Estando eles ali, aconteceu completarem-se-lhe os dias,
\par 7 e ela deu à luz o seu filho primogênito, enfaixou-o e o deitou numa manjedoura, porque não havia lugar para eles na hospedaria.
\par 8 Havia, naquela mesma região, pastores que viviam nos campos e guardavam o seu rebanho durante as vigílias da noite.
\par 9 E um anjo do Senhor desceu aonde eles estavam, e a glória do Senhor brilhou ao redor deles; e ficaram tomados de grande temor.
\par 10 O anjo, porém, lhes disse: Não temais; eis aqui vos trago boa-nova de grande alegria, que o será para todo o povo:
\par 11 é que hoje vos nasceu, na cidade de Davi, o Salvador, que é Cristo, o Senhor.
\par 12 E isto vos servirá de sinal: encontrareis uma criança envolta em faixas e deitada em manjedoura.
\par 13 E, subitamente, apareceu com o anjo uma multidão da milícia celestial, louvando a Deus e dizendo:
\par 14 Glória a Deus nas maiores alturas, e paz na terra entre os homens, a quem ele quer bem.
\par 15 E, ausentando-se deles os anjos para o céu, diziam os pastores uns aos outros: Vamos até Belém e vejamos os acontecimentos que o Senhor nos deu a conhecer.
\par 16 Foram apressadamente e acharam Maria e José e a criança deitada na manjedoura.
\par 17 E, vendo-o, divulgaram o que lhes tinha sido dito a respeito deste menino.
\par 18 Todos os que ouviram se admiraram das coisas referidas pelos pastores.
\par 19 Maria, porém, guardava todas estas palavras, meditando-as no coração.
\par 20 Voltaram, então, os pastores glorificando e louvando a Deus por tudo o que tinham ouvido e visto, como lhes fora anunciado.
\par 21 Completados oito dias para ser circuncidado o menino, deram-lhe o nome de JESUS, como lhe chamara o anjo, antes de ser concebido.
\par 22 Passados os dias da purificação deles segundo a Lei de Moisés, levaram-no a Jerusalém para o apresentarem ao Senhor,
\par 23 conforme o que está escrito na Lei do Senhor: Todo primogênito ao Senhor será consagrado;
\par 24 e para oferecer um sacrifício, segundo o que está escrito na referida Lei: Um par de rolas ou dois pombinhos.
\par 25 Havia em Jerusalém um homem chamado Simeão; homem este justo e piedoso que esperava a consolação de Israel; e o Espírito Santo estava sobre ele.
\par 26 Revelara-lhe o Espírito Santo que não passaria pela morte antes de ver o Cristo do Senhor.
\par 27 Movido pelo Espírito, foi ao templo; e, quando os pais trouxeram o menino Jesus para fazerem com ele o que a Lei ordenava,
\par 28 Simeão o tomou nos braços e louvou a Deus, dizendo:
\par 29 Agora, Senhor, podes despedir em paz o teu servo, segundo a tua palavra;
\par 30 porque os meus olhos já viram a tua salvação,
\par 31 a qual preparaste diante de todos os povos:
\par 32 luz para revelação aos gentios, e para glória do teu povo de Israel.
\par 33 E estavam o pai e a mãe do menino admirados do que dele se dizia.
\par 34 Simeão os abençoou e disse a Maria, mãe do menino: Eis que este menino está destinado tanto para ruína como para levantamento de muitos em Israel e para ser alvo de contradição
\par 35 (também uma espada traspassará a tua própria alma), para que se manifestem os pensamentos de muitos corações.
\par 36 Havia uma profetisa, chamada Ana, filha de Fanuel, da tribo de Aser, avançada em dias, que vivera com seu marido sete anos desde que se casara
\par 37 e que era viúva de oitenta e quatro anos. Esta não deixava o templo, mas adorava noite e dia em jejuns e orações.
\par 38 E, chegando naquela hora, dava graças a Deus e falava a respeito do menino a todos os que esperavam a redenção de Jerusalém.
\par 39 Cumpridas todas as ordenanças segundo a Lei do Senhor, voltaram para a Galiléia, para a sua cidade de Nazaré.
\par 40 Crescia o menino e se fortalecia, enchendo-se de sabedoria; e a graça de Deus estava sobre ele.
\par 41 Ora, anualmente iam seus pais a Jerusalém, para a Festa da Páscoa.
\par 42 Quando ele atingiu os doze anos, subiram a Jerusalém, segundo o costume da festa.
\par 43 Terminados os dias da festa, ao regressarem, permaneceu o menino Jesus em Jerusalém, sem que seus pais o soubessem.
\par 44 Pensando, porém, estar ele entre os companheiros de viagem, foram caminho de um dia e, então, passaram a procurá-lo entre os parentes e os conhecidos;
\par 45 e, não o tendo encontrado, voltaram a Jerusalém à sua procura.
\par 46 Três dias depois, o acharam no templo, assentado no meio dos doutores, ouvindo-os e interrogando-os.
\par 47 E todos os que o ouviam muito se admiravam da sua inteligência e das suas respostas.
\par 48 Logo que seus pais o viram, ficaram maravilhados; e sua mãe lhe disse: Filho, por que fizeste assim conosco? Teu pai e eu, aflitos, estamos à tua procura.
\par 49 Ele lhes respondeu: Por que me procuráveis? Não sabíeis que me cumpria estar na casa de meu Pai?
\par 50 Não compreenderam, porém, as palavras que lhes dissera.
\par 51 E desceu com eles para Nazaré; e era-lhes submisso. Sua mãe, porém, guardava todas estas coisas no coração.
\par 52 E crescia Jesus em sabedoria, estatura e graça, diante de Deus e dos homens.

\chapter{3}

\par 1 No décimo quinto ano do reinado de Tibério César, sendo Pôncio Pilatos governador da Judéia, Herodes, tetrarca da Galiléia, seu irmão Filipe, tetrarca da região da Ituréia e Traconites, e Lisânias, tetrarca de Abilene,
\par 2 sendo sumos sacerdotes Anás e Caifás, veio a palavra de Deus a João, filho de Zacarias, no deserto.
\par 3 Ele percorreu toda a circunvizinhança do Jordão, pregando batismo de arrependimento para remissão de pecados,
\par 4 conforme está escrito no livro das palavras do profeta Isaías: Voz do que clama no deserto: Preparai o caminho do Senhor, endireitai as suas veredas.
\par 5 Todo vale será aterrado, e nivelados todos os montes e outeiros; os caminhos tortuosos serão retificados, e os escabrosos, aplanados;
\par 6 e toda carne verá a salvação de Deus.
\par 7 Dizia ele, pois, às multidões que saíam para serem batizadas: Raça de víboras, quem vos induziu a fugir da ira vindoura?
\par 8 Produzi, pois, frutos dignos de arrependimento e não comeceis a dizer entre vós mesmos: Temos por pai a Abraão; porque eu vos afirmo que destas pedras Deus pode suscitar filhos a Abraão.
\par 9 E também já está posto o machado à raiz das árvores; toda árvore, pois, que não produz bom fruto é cortada e lançada ao fogo.
\par 10 Então, as multidões o interrogavam, dizendo: Que havemos, pois, de fazer?
\par 11 Respondeu-lhes: Quem tiver duas túnicas, reparta com quem não tem; e quem tiver comida, faça o mesmo.
\par 12 Foram também publicanos para serem batizados e perguntaram-lhe: Mestre, que havemos de fazer?
\par 13 Respondeu-lhes: Não cobreis mais do que o estipulado.
\par 14 Também soldados lhe perguntaram: E nós, que faremos? E ele lhes disse: A ninguém maltrateis, não deis denúncia falsa e contentai-vos com o vosso soldo.
\par 15 Estando o povo na expectativa, e discorrendo todos no seu íntimo a respeito de João, se não seria ele, porventura, o próprio Cristo,
\par 16 disse João a todos: Eu, na verdade, vos batizo com água, mas vem o que é mais poderoso do que eu, do qual não sou digno de desatar-lhe as correias das sandálias; ele vos batizará com o Espírito Santo e com fogo.
\par 17 A sua pá, ele a tem na mão, para limpar completamente a sua eira e recolher o trigo no seu celeiro; porém queimará a palha em fogo inextinguível.
\par 18 Assim, pois, com muitas outras exortações anunciava o evangelho ao povo;
\par 19 mas Herodes, o tetrarca, sendo repreendido por ele, por causa de Herodias, mulher de seu irmão, e por todas as maldades que o mesmo Herodes havia feito,
\par 20 acrescentou ainda sobre todas a de lançar João no cárcere.
\par 21 E aconteceu que, ao ser todo o povo batizado, também o foi Jesus; e, estando ele a orar, o céu se abriu,
\par 22 e o Espírito Santo desceu sobre ele em forma corpórea como pomba; e ouviu-se uma voz do céu: Tu és o meu Filho amado, em ti me comprazo.
\par 23 Ora, tinha Jesus cerca de trinta anos ao começar o seu ministério. Era, como se cuidava, filho de José, filho de Eli;
\par 24 Eli, filho de Matate, Matate, filho de Levi, Levi, filho de Melqui, este, filho de Janai, filho de José;
\par 25 José, filho de Matatias, Matatias, filho de Amós, Amós, filho de Naum, este, filho de Esli, filho de Nagai;
\par 26 Nagai, filho de Maate, Maate, filho de Matatias, Matatias, filho de Semei, este, filho de José, filho de Jodá;
\par 27 Jodá, filho de Joanã, Joanã, filho de Resa, Resa, filho de Zorobabel, este, de Salatiel, filho de Neri;
\par 28 Neri, filho de Melqui, Melqui, filho de Adi, Adi, filho de Cosã, este, de Elmadã, filho de Er;
\par 29 Er, filho de Josué, Josué, filho de Eliézer, Eliézer, filho de Jorim, este, de Matate, filho de Levi;
\par 30 Levi, filho de Simeão, Simeão, filho de Judá, Judá, filho de José, este, filho de Jonã, filho de Eliaquim;
\par 31 Eliaquim, filho de Meleá, Meleá, filho de Mená, Mená, filho de Matatá, este, filho de Natã, filho de Davi;
\par 32 Davi, filho de Jessé, Jessé, filho de Obede, Obede, filho de Boaz, este, filho de Salá, filho de Naassom;
\par 33 Naassom, filho de Aminadabe, Aminadabe, filho de Admim, Admim, filho de Arni, Arni, filho de Esrom, este, filho de Perez, filho de Judá;
\par 34 Judá, filho de Jacó, Jacó, filho de Isaque, Isaque, filho de Abraão, este, filho de Tera, filho de Naor;
\par 35 Naor, filho de Serugue, Serugue, filho de Ragaú, Ragaú, filho de Faleque, este, filho de Éber, filho de Salá;
\par 36 Salá, filho de Cainã, Cainã, filho de Arfaxade, Arfaxade, filho de Sem, este, filho de Noé, filho de Lameque;
\par 37 Lameque, filho de Metusalém, Metusalém, filho de Enoque, Enoque, filho de Jarede, este, filho de Maalalel, filho de Cainã;
\par 38 Cainã, filho de Enos, Enos, filho de Sete, e este, filho de Adão, filho de Deus.

\chapter{4}

\par 1 Jesus, cheio do Espírito Santo, voltou do Jordão e foi guiado pelo mesmo Espírito, no deserto,
\par 2 durante quarenta dias, sendo tentado pelo diabo. Nada comeu naqueles dias, ao fim dos quais teve fome.
\par 3 Disse-lhe, então, o diabo: Se és o Filho de Deus, manda que esta pedra se transforme em pão.
\par 4 Mas Jesus lhe respondeu: Está escrito: Não só de pão viverá o homem.
\par 5 E, elevando-o, mostrou-lhe, num momento, todos os reinos do mundo.
\par 6 Disse-lhe o diabo: Dar-te-ei toda esta autoridade e a glória destes reinos, porque ela me foi entregue, e a dou a quem eu quiser.
\par 7 Portanto, se prostrado me adorares, toda será tua.
\par 8 Mas Jesus lhe respondeu: Está escrito: Ao Senhor, teu Deus, adorarás e só a ele darás culto.
\par 9 Então, o levou a Jerusalém, e o colocou sobre o pináculo do templo, e disse: Se és o Filho de Deus, atira-te daqui abaixo;
\par 10 porque está escrito: Aos seus anjos ordenará a teu respeito que te guardem;
\par 11 e: Eles te susterão nas suas mãos, para não tropeçares nalguma pedra.
\par 12 Respondeu-lhe Jesus: Dito está: Não tentarás o Senhor, teu Deus.
\par 13 Passadas que foram as tentações de toda sorte, apartou-se dele o diabo, até momento oportuno.
\par 14 Então, Jesus, no poder do Espírito, regressou para a Galiléia, e a sua fama correu por toda a circunvizinhança.
\par 15 E ensinava nas sinagogas, sendo glorificado por todos.
\par 16 Indo para Nazaré, onde fora criado, entrou, num sábado, na sinagoga, segundo o seu costume, e levantou-se para ler.
\par 17 Então, lhe deram o livro do profeta Isaías, e, abrindo o livro, achou o lugar onde estava escrito:
\par 18 O Espírito do Senhor está sobre mim, pelo que me ungiu para evangelizar os pobres; enviou-me para proclamar libertação aos cativos e restauração da vista aos cegos, para pôr em liberdade os oprimidos,
\par 19 e apregoar o ano aceitável do Senhor.
\par 20 Tendo fechado o livro, devolveu-o ao assistente e sentou-se; e todos na sinagoga tinham os olhos fitos nele.
\par 21 Então, passou Jesus a dizer-lhes: Hoje, se cumpriu a Escritura que acabais de ouvir.
\par 22 Todos lhe davam testemunho, e se maravilhavam das palavras de graça que lhe saíam dos lábios, e perguntavam: Não é este o filho de José?
\par 23 Disse-lhes Jesus: Sem dúvida, citar-me-eis este provérbio: Médico, cura-te a ti mesmo; tudo o que ouvimos ter-se dado em Cafarnaum, faze-o também aqui na tua terra.
\par 24 E prosseguiu: De fato, vos afirmo que nenhum profeta é bem recebido na sua própria terra.
\par 25 Na verdade vos digo que muitas viúvas havia em Israel no tempo de Elias, quando o céu se fechou por três anos e seis meses, reinando grande fome em toda a terra;
\par 26 e a nenhuma delas foi Elias enviado, senão a uma viúva de Sarepta de Sidom.
\par 27 Havia também muitos leprosos em Israel nos dias do profeta Eliseu, e nenhum deles foi purificado, senão Naamã, o siro.
\par 28 Todos na sinagoga, ouvindo estas coisas, se encheram de ira.
\par 29 E, levantando-se, expulsaram-no da cidade e o levaram até ao cimo do monte sobre o qual estava edificada, para, de lá, o precipitarem abaixo.
\par 30 Jesus, porém, passando por entre eles, retirou-se.
\par 31 E desceu a Cafarnaum, cidade da Galiléia, e os ensinava no sábado.
\par 32 E muito se maravilhavam da sua doutrina, porque a sua palavra era com autoridade.
\par 33 Achava-se na sinagoga um homem possesso de um espírito de demônio imundo, e bradou em alta voz:
\par 34 Ah! Que temos nós contigo, Jesus Nazareno? Vieste para perder-nos? Bem sei quem és: o Santo de Deus!
\par 35 Mas Jesus o repreendeu, dizendo: Cala-te e sai deste homem. O demônio, depois de o ter lançado por terra no meio de todos, saiu dele sem lhe fazer mal.
\par 36 Todos ficaram grandemente admirados e comentavam entre si, dizendo: Que palavra é esta, pois, com autoridade e poder, ordena aos espíritos imundos, e eles saem?
\par 37 E a sua fama corria por todos os lugares da circunvizinhança.
\par 38 Deixando ele a sinagoga, foi para a casa de Simão. Ora, a sogra de Simão achava-se enferma, com febre muito alta; e rogaram-lhe por ela.
\par 39 Inclinando-se ele para ela, repreendeu a febre, e esta a deixou; e logo se levantou, passando a servi-los.
\par 40 Ao pôr-do-sol, todos os que tinham enfermos de diferentes moléstias lhos traziam; e ele os curava, impondo as mãos sobre cada um.
\par 41 Também de muitos saíam demônios, gritando e dizendo: Tu és o Filho de Deus! Ele, porém, os repreendia para que não falassem, pois sabiam ser ele o Cristo.
\par 42 Sendo dia, saiu e foi para um lugar deserto; as multidões o procuravam, e foram até junto dele, e instavam para que não os deixasse.
\par 43 Ele, porém, lhes disse: É necessário que eu anuncie o evangelho do reino de Deus também às outras cidades, pois para isso é que fui enviado.
\par 44 E pregava nas sinagogas da Judéia.

\chapter{5}

\par 1 Aconteceu que, ao apertá-lo a multidão para ouvir a palavra de Deus, estava ele junto ao lago de Genesaré;
\par 2 e viu dois barcos junto à praia do lago; mas os pescadores, havendo desembarcado, lavavam as redes.
\par 3 Entrando em um dos barcos, que era o de Simão, pediu-lhe que o afastasse um pouco da praia; e, assentando-se, ensinava do barco as multidões.
\par 4 Quando acabou de falar, disse a Simão: Faze-te ao largo, e lançai as vossas redes para pescar.
\par 5 Respondeu-lhe Simão: Mestre, havendo trabalhado toda a noite, nada apanhamos, mas sob a tua palavra lançarei as redes.
\par 6 Isto fazendo, apanharam grande quantidade de peixes; e rompiam-se-lhes as redes.
\par 7 Então, fizeram sinais aos companheiros do outro barco, para que fossem ajudá-los. E foram e encheram ambos os barcos, a ponto de quase irem a pique.
\par 8 Vendo isto, Simão Pedro prostrou-se aos pés de Jesus, dizendo: Senhor, retira-te de mim, porque sou pecador.
\par 9 Pois, à vista da pesca que fizeram, a admiração se apoderou dele e de todos os seus companheiros,
\par 10 bem como de Tiago e João, filhos de Zebedeu, que eram seus sócios. Disse Jesus a Simão: Não temas; doravante serás pescador de homens.
\par 11 E, arrastando eles os barcos sobre a praia, deixando tudo, o seguiram.
\par 12 Aconteceu que, estando ele numa das cidades, veio à sua presença um homem coberto de lepra; ao ver a Jesus, prostrando-se com o rosto em terra, suplicou-lhe: Senhor, se quiseres, podes purificar-me.
\par 13 E ele, estendendo a mão, tocou-lhe, dizendo: Quero, fica limpo! E, no mesmo instante, lhe desapareceu a lepra.
\par 14 Ordenou-lhe Jesus que a ninguém o dissesse, mas vai, disse, mostra-te ao sacerdote e oferece, pela tua purificação, o sacrifício que Moisés determinou, para servir de testemunho ao povo.
\par 15 Porém o que se dizia a seu respeito cada vez mais se divulgava, e grandes multidões afluíam para o ouvirem e serem curadas de suas enfermidades.
\par 16 Ele, porém, se retirava para lugares solitários e orava.
\par 17 Ora, aconteceu que, num daqueles dias, estava ele ensinando, e achavam-se ali assentados fariseus e mestres da Lei, vindos de todas as aldeias da Galiléia, da Judéia e de Jerusalém. E o poder do Senhor estava com ele para curar.
\par 18 Vieram, então, uns homens trazendo em um leito um paralítico; e procuravam introduzi-lo e pô-lo diante de Jesus.
\par 19 E, não achando por onde introduzi-lo por causa da multidão, subindo ao eirado, o desceram no leito, por entre os ladrilhos, para o meio, diante de Jesus.
\par 20 Vendo-lhes a fé, Jesus disse ao paralítico: Homem, estão perdoados os teus pecados.
\par 21 E os escribas e fariseus arrazoavam, dizendo: Quem é este que diz blasfêmias? Quem pode perdoar pecados, senão Deus?
\par 22 Jesus, porém, conhecendo-lhes os pensamentos, disse-lhes: Que arrazoais em vosso coração?
\par 23 Qual é mais fácil, dizer: Estão perdoados os teus pecados ou: Levanta-te e anda?
\par 24 Mas, para que saibais que o Filho do Homem tem sobre a terra autoridade para perdoar pecados -- disse ao paralítico: Eu te ordeno: Levanta-te, toma o teu leito e vai para casa.
\par 25 Imediatamente, se levantou diante deles e, tomando o leito em que permanecera deitado, voltou para casa, glorificando a Deus.
\par 26 Todos ficaram atônitos, davam glória a Deus e, possuídos de temor, diziam: Hoje, vimos prodígios.
\par 27 Passadas estas coisas, saindo, viu um publicano, chamado Levi, assentado na coletoria, e disse-lhe: Segue-me!
\par 28 Ele se levantou e, deixando tudo, o seguiu.
\par 29 Então, lhe ofereceu Levi um grande banquete em sua casa; e numerosos publicanos e outros estavam com eles à mesa.
\par 30 Os fariseus e seus escribas murmuravam contra os discípulos de Jesus, perguntando: Por que comeis e bebeis com os publicanos e pecadores?
\par 31 Respondeu-lhes Jesus: Os sãos não precisam de médico, e sim os doentes.
\par 32 Não vim chamar justos, e sim pecadores, ao arrependimento.
\par 33 Disseram-lhe eles: Os discípulos de João e bem assim os dos fariseus freqüentemente jejuam e fazem orações; os teus, entretanto, comem e bebem.
\par 34 Jesus, porém, lhes disse: Podeis fazer jejuar os convidados para o casamento, enquanto está com eles o noivo?
\par 35 Dias virão, contudo, em que lhes será tirado o noivo; naqueles dias, sim, jejuarão.
\par 36 Também lhes disse uma parábola: Ninguém tira um pedaço de veste nova e o põe em veste velha; pois rasgará a nova, e o remendo da nova não se ajustará à velha.
\par 37 E ninguém põe vinho novo em odres velhos, pois o vinho novo romperá os odres; entornar-se-á o vinho, e os odres se estragarão.
\par 38 Pelo contrário, vinho novo deve ser posto em odres novos [e ambos se conservam].
\par 39 E ninguém, tendo bebido o vinho velho, prefere o novo; porque diz: O velho é excelente.

\chapter{6}

\par 1 Aconteceu que, num sábado, passando Jesus pelas searas, os seus discípulos colhiam e comiam espigas, debulhando-as com as mãos.
\par 2 E alguns dos fariseus lhes disseram: Por que fazeis o que não é lícito aos sábados?
\par 3 Respondeu-lhes Jesus: Nem ao menos tendes lido o que fez Davi, quando teve fome, ele e seus companheiros?
\par 4 Como entrou na casa de Deus, tomou, e comeu os pães da proposição, e os deu aos que com ele estavam, pães que não lhes era lícito comer, mas exclusivamente aos sacerdotes?
\par 5 E acrescentou-lhes: O Filho do Homem é senhor do sábado.
\par 6 Sucedeu que, em outro sábado, entrou ele na sinagoga e ensinava. Ora, achava-se ali um homem cuja mão direita estava ressequida.
\par 7 Os escribas e os fariseus observavam-no, procurando ver se ele faria uma cura no sábado, a fim de acharem de que o acusar.
\par 8 Mas ele, conhecendo-lhes os pensamentos, disse ao homem da mão ressequida: Levanta-te e vem para o meio; e ele, levantando-se, permaneceu de pé.
\par 9 Então, disse Jesus a eles: Que vos parece? É lícito, no sábado, fazer o bem ou o mal? Salvar a vida ou deixá-la perecer?
\par 10 E, fitando todos ao redor, disse ao homem: Estende a mão. Ele assim o fez, e a mão lhe foi restaurada.
\par 11 Mas eles se encheram de furor e discutiam entre si quanto ao que fariam a Jesus.
\par 12 Naqueles dias, retirou-se para o monte, a fim de orar, e passou a noite orando a Deus.
\par 13 E, quando amanheceu, chamou a si os seus discípulos e escolheu doze dentre eles, aos quais deu também o nome de apóstolos:
\par 14 Simão, a quem acrescentou o nome de Pedro, e André, seu irmão; Tiago e João; Filipe e Bartolomeu;
\par 15 Mateus e Tomé; Tiago, filho de Alfeu, e Simão, chamado Zelote;
\par 16 Judas, filho de Tiago, e Judas Iscariotes, que se tornou traidor.
\par 17 E, descendo com eles, parou numa planura onde se encontravam muitos discípulos seus e grande multidão do povo, de toda a Judéia, de Jerusalém e do litoral de Tiro e de Sidom,
\par 18 que vieram para o ouvirem e serem curados de suas enfermidades; também os atormentados por espíritos imundos eram curados.
\par 19 E todos da multidão procuravam tocá-lo, porque dele saía poder; e curava todos.
\par 20 Então, olhando ele para os seus discípulos, disse-lhes: Bem-aventurados vós, os pobres, porque vosso é o reino de Deus.
\par 21 Bem-aventurados vós, os que agora tendes fome, porque sereis fartos. Bem-aventurados vós, os que agora chorais, porque haveis de rir.
\par 22 Bem-aventurados sois quando os homens vos odiarem e quando vos expulsarem da sua companhia, vos injuriarem e rejeitarem o vosso nome como indigno, por causa do Filho do Homem.
\par 23 Regozijai-vos naquele dia e exultai, porque grande é o vosso galardão no céu; pois dessa forma procederam seus pais com os profetas.
\par 24 Mas ai de vós, os ricos! Porque tendes a vossa consolação.
\par 25 Ai de vós, os que estais agora fartos! Porque vireis a ter fome. Ai de vós, os que agora rides! Porque haveis de lamentar e chorar.
\par 26 Ai de vós, quando todos vos louvarem! Porque assim procederam seus pais com os falsos profetas.
\par 27 Digo-vos, porém, a vós outros que me ouvis: amai os vossos inimigos, fazei o bem aos que vos odeiam;
\par 28 bendizei aos que vos maldizem, orai pelos que vos caluniam.
\par 29 Ao que te bate numa face, oferece-lhe também a outra; e, ao que tirar a tua capa, deixa-o levar também a túnica;
\par 30 dá a todo o que te pede; e, se alguém levar o que é teu, não entres em demanda.
\par 31 Como quereis que os homens vos façam, assim fazei-o vós também a eles.
\par 32 Se amais os que vos amam, qual é a vossa recompensa? Porque até os pecadores amam aos que os amam.
\par 33 Se fizerdes o bem aos que vos fazem o bem, qual é a vossa recompensa? Até os pecadores fazem isso.
\par 34 E, se emprestais àqueles de quem esperais receber, qual é a vossa recompensa? Também os pecadores emprestam aos pecadores, para receberem outro tanto.
\par 35 Amai, porém, os vossos inimigos, fazei o bem e emprestai, sem esperar nenhuma paga; será grande o vosso galardão, e sereis filhos do Altíssimo. Pois ele é benigno até para com os ingratos e maus.
\par 36 Sede misericordiosos, como também é misericordioso vosso Pai.
\par 37 Não julgueis e não sereis julgados; não condeneis e não sereis condenados; perdoai e sereis perdoados;
\par 38 dai, e dar-se-vos-á; boa medida, recalcada, sacudida, transbordante, generosamente vos darão; porque com a medida com que tiverdes medido vos medirão também.
\par 39 Propôs-lhes também uma parábola: Pode, porventura, um cego guiar a outro cego? Não cairão ambos no barranco?
\par 40 O discípulo não está acima do seu mestre; todo aquele, porém, que for bem instruído será como o seu mestre.
\par 41 Por que vês tu o argueiro no olho de teu irmão, porém não reparas na trave que está no teu próprio?
\par 42 Como poderás dizer a teu irmão: Deixa, irmão, que eu tire o argueiro do teu olho, não vendo tu mesmo a trave que está no teu? Hipócrita, tira primeiro a trave do teu olho e, então, verás claramente para tirar o argueiro que está no olho de teu irmão.
\par 43 Não há árvore boa que dê mau fruto; nem tampouco árvore má que dê bom fruto.
\par 44 Porquanto cada árvore é conhecida pelo seu próprio fruto. Porque não se colhem figos de espinheiros, nem dos abrolhos se vindimam uvas.
\par 45 O homem bom do bom tesouro do coração tira o bem, e o mau do mau tesouro tira o mal; porque a boca fala do que está cheio o coração.
\par 46 Por que me chamais Senhor, Senhor, e não fazeis o que vos mando?
\par 47 Todo aquele que vem a mim, e ouve as minhas palavras, e as pratica, eu vos mostrarei a quem é semelhante.
\par 48 É semelhante a um homem que, edificando uma casa, cavou, abriu profunda vala e lançou o alicerce sobre a rocha; e, vindo a enchente, arrojou-se o rio contra aquela casa e não a pôde abalar, por ter sido bem construída.
\par 49 Mas o que ouve e não pratica é semelhante a um homem que edificou uma casa sobre a terra sem alicerces, e, arrojando-se o rio contra ela, logo desabou; e aconteceu que foi grande a ruína daquela casa.

\chapter{7}

\par 1 Tendo Jesus concluído todas as suas palavras dirigidas ao povo, entrou em Cafarnaum.
\par 2 E o servo de um centurião, a quem este muito estimava, estava doente, quase à morte.
\par 3 Tendo ouvido falar a respeito de Jesus, enviou-lhe alguns anciãos dos judeus, pedindo-lhe que viesse curar o seu servo.
\par 4 Estes, chegando-se a Jesus, com instância lhe suplicaram, dizendo: Ele é digno de que lhe faças isto;
\par 5 porque é amigo do nosso povo, e ele mesmo nos edificou a sinagoga.
\par 6 Então, Jesus foi com eles. E, já perto da casa, o centurião enviou-lhe amigos para lhe dizer: Senhor, não te incomodes, porque não sou digno de que entres em minha casa.
\par 7 Por isso, eu mesmo não me julguei digno de ir ter contigo; porém manda com uma palavra, e o meu rapaz será curado.
\par 8 Porque também eu sou homem sujeito à autoridade, e tenho soldados às minhas ordens, e digo a este: vai, e ele vai; e a outro: vem, e ele vem; e ao meu servo: faze isto, e ele o faz.
\par 9 Ouvidas estas palavras, admirou-se Jesus dele e, voltando-se para o povo que o acompanhava, disse: Afirmo-vos que nem mesmo em Israel achei fé como esta.
\par 10 E, voltando para casa os que foram enviados, encontraram curado o servo.
\par 11 Em dia subseqüente, dirigia-se Jesus a uma cidade chamada Naim, e iam com ele os seus discípulos e numerosa multidão.
\par 12 Como se aproximasse da porta da cidade, eis que saía o enterro do filho único de uma viúva; e grande multidão da cidade ia com ela.
\par 13 Vendo-a, o Senhor se compadeceu dela e lhe disse: Não chores!
\par 14 Chegando-se, tocou o esquife e, parando os que o conduziam, disse: Jovem, eu te mando: levanta-te!
\par 15 Sentou-se o que estivera morto e passou a falar; e Jesus o restituiu a sua mãe.
\par 16 Todos ficaram possuídos de temor e glorificavam a Deus, dizendo: Grande profeta se levantou entre nós; e: Deus visitou o seu povo.
\par 17 Esta notícia a respeito dele divulgou-se por toda a Judéia e por toda a circunvizinhança.
\par 18 Todas estas coisas foram referidas a João pelos seus discípulos. E João, chamando dois deles,
\par 19 enviou-os ao Senhor para perguntar: És tu aquele que estava para vir ou havemos de esperar outro?
\par 20 Quando os homens chegaram junto dele, disseram: João Batista enviou-nos para te perguntar: És tu aquele que estava para vir ou esperaremos outro?
\par 21 Naquela mesma hora, curou Jesus muitos de moléstias, e de flagelos, e de espíritos malignos; e deu vista a muitos cegos.
\par 22 Então, Jesus lhes respondeu: Ide e anunciai a João o que vistes e ouvistes: os cegos vêem, os coxos andam, os leprosos são purificados, os surdos ouvem, os mortos são ressuscitados, e aos pobres, anuncia-se-lhes o evangelho.
\par 23 E bem-aventurado é aquele que não achar em mim motivo de tropeço.
\par 24 Tendo-se retirado os mensageiros, passou Jesus a dizer ao povo a respeito de João: Que saístes a ver no deserto? Um caniço agitado pelo vento?
\par 25 Que saístes a ver? Um homem vestido de roupas finas? Os que se vestem bem e vivem no luxo assistem nos palácios dos reis.
\par 26 Sim, que saístes a ver? Um profeta? Sim, eu vos digo, e muito mais que profeta.
\par 27 Este é aquele de quem está escrito: Eis aí envio diante da tua face o meu mensageiro, o qual preparará o teu caminho diante de ti.
\par 28 E eu vos digo: entre os nascidos de mulher, ninguém é maior do que João; mas o menor no reino de Deus é maior do que ele.
\par 29 Todo o povo que o ouviu e até os publicanos reconheceram a justiça de Deus, tendo sido batizados com o batismo de João;
\par 30 mas os fariseus e os intérpretes da Lei rejeitaram, quanto a si mesmos, o desígnio de Deus, não tendo sido batizados por ele.
\par 31 A que, pois, compararei os homens da presente geração, e a que são eles semelhantes?
\par 32 São semelhantes a meninos que, sentados na praça, gritam uns para os outros: Nós vos tocamos flauta, e não dançastes; entoamos lamentações, e não chorastes.
\par 33 Pois veio João Batista, não comendo pão, nem bebendo vinho, e dizeis: Tem demônio!
\par 34 Veio o Filho do Homem, comendo e bebendo, e dizeis: Eis aí um glutão e bebedor de vinho, amigo de publicanos e pecadores!
\par 35 Mas a sabedoria é justificada por todos os seus filhos.
\par 36 Convidou-o um dos fariseus para que fosse jantar com ele. Jesus, entrando na casa do fariseu, tomou lugar à mesa.
\par 37 E eis que uma mulher da cidade, pecadora, sabendo que ele estava à mesa na casa do fariseu, levou um vaso de alabastro com ungüento;
\par 38 e, estando por detrás, aos seus pés, chorando, regava-os com suas lágrimas e os enxugava com os próprios cabelos; e beijava-lhe os pés e os ungia com o ungüento.
\par 39 Ao ver isto, o fariseu que o convidara disse consigo mesmo: Se este fora profeta, bem saberia quem e qual é a mulher que lhe tocou, porque é pecadora.
\par 40 Dirigiu-se Jesus ao fariseu e lhe disse: Simão, uma coisa tenho a dizer-te. Ele respondeu: Dize-a, Mestre.
\par 41 Certo credor tinha dois devedores: um lhe devia quinhentos denários, e o outro, cinqüenta.
\par 42 Não tendo nenhum dos dois com que pagar, perdoou-lhes a ambos. Qual deles, portanto, o amará mais?
\par 43 Respondeu-lhe Simão: Suponho que aquele a quem mais perdoou. Replicou-lhe: Julgaste bem.
\par 44 E, voltando-se para a mulher, disse a Simão: Vês esta mulher? Entrei em tua casa, e não me deste água para os pés; esta, porém, regou os meus pés com lágrimas e os enxugou com os seus cabelos.
\par 45 Não me deste ósculo; ela, entretanto, desde que entrei não cessa de me beijar os pés.
\par 46 Não me ungiste a cabeça com óleo, mas esta, com bálsamo, ungiu os meus pés.
\par 47 Por isso, te digo: perdoados lhe são os seus muitos pecados, porque ela muito amou; mas aquele a quem pouco se perdoa, pouco ama.
\par 48 Então, disse à mulher: Perdoados são os teus pecados.
\par 49 Os que estavam com ele à mesa começaram a dizer entre si: Quem é este que até perdoa pecados?
\par 50 Mas Jesus disse à mulher: A tua fé te salvou; vai-te em paz.

\chapter{8}

\par 1 Aconteceu, depois disto, que andava Jesus de cidade em cidade e de aldeia em aldeia, pregando e anunciando o evangelho do reino de Deus, e os doze iam com ele,
\par 2 e também algumas mulheres que haviam sido curadas de espíritos malignos e de enfermidades: Maria, chamada Madalena, da qual saíram sete demônios;
\par 3 e Joana, mulher de Cuza, procurador de Herodes, Suzana e muitas outras, as quais lhe prestavam assistência com os seus bens.
\par 4 Afluindo uma grande multidão e vindo ter com ele gente de todas as cidades, disse Jesus por parábola:
\par 5 Eis que o semeador saiu a semear. E, ao semear, uma parte caiu à beira do caminho; foi pisada, e as aves do céu a comeram.
\par 6 Outra caiu sobre a pedra; e, tendo crescido, secou por falta de umidade.
\par 7 Outra caiu no meio dos espinhos; e estes, ao crescerem com ela, a sufocaram.
\par 8 Outra, afinal, caiu em boa terra; cresceu e produziu a cento por um. Dizendo isto, clamou: Quem tem ouvidos para ouvir, ouça.
\par 9 E os seus discípulos o interrogaram, dizendo: Que parábola é esta?
\par 10 Respondeu-lhes Jesus: A vós outros é dado conhecer os mistérios do reino de Deus; aos demais, fala-se por parábolas, para que, vendo, não vejam; e, ouvindo, não entendam.
\par 11 Este é o sentido da parábola: a semente é a palavra de Deus.
\par 12 A que caiu à beira do caminho são os que a ouviram; vem, a seguir, o diabo e arrebata-lhes do coração a palavra, para não suceder que, crendo, sejam salvos.
\par 13 A que caiu sobre a pedra são os que, ouvindo a palavra, a recebem com alegria; estes não têm raiz, crêem apenas por algum tempo e, na hora da provação, se desviam.
\par 14 A que caiu entre espinhos são os que ouviram e, no decorrer dos dias, foram sufocados com os cuidados, riquezas e deleites da vida; os seus frutos não chegam a amadurecer.
\par 15 A que caiu na boa terra são os que, tendo ouvido de bom e reto coração, retêm a palavra; estes frutificam com perseverança.
\par 16 Ninguém, depois de acender uma candeia, a cobre com um vaso ou a põe debaixo de uma cama; pelo contrário, coloca-a sobre um velador, a fim de que os que entram vejam a luz.
\par 17 Nada há oculto, que não haja de manifestar-se, nem escondido, que não venha a ser conhecido e revelado.
\par 18 Vede, pois, como ouvis; porque ao que tiver, se lhe dará; e ao que não tiver, até aquilo que julga ter lhe será tirado.
\par 19 Vieram ter com ele sua mãe e seus irmãos e não podiam aproximar-se por causa da concorrência de povo.
\par 20 E lhe comunicaram: Tua mãe e teus irmãos estão lá fora e querem ver-te.
\par 21 Ele, porém, lhes respondeu: Minha mãe e meus irmãos são aqueles que ouvem a palavra de Deus e a praticam.
\par 22 Aconteceu que, num daqueles dias, entrou ele num barco em companhia dos seus discípulos e disse-lhes: Passemos para a outra margem do lago; e partiram.
\par 23 Enquanto navegavam, ele adormeceu. E sobreveio uma tempestade de vento no lago, correndo eles o perigo de soçobrar.
\par 24 Chegando-se a ele, despertaram-no dizendo: Mestre, Mestre, estamos perecendo! Despertando-se Jesus, repreendeu o vento e a fúria da água. Tudo cessou, e veio a bonança.
\par 25 Então, lhes disse: Onde está a vossa fé? Eles, possuídos de temor e admiração, diziam uns aos outros: Quem é este que até aos ventos e às ondas repreende, e lhe obedecem?
\par 26 Então, rumaram para a terra dos gerasenos, fronteira da Galiléia.
\par 27 Logo ao desembarcar, veio da cidade ao seu encontro um homem possesso de demônios que, havia muito, não se vestia, nem habitava em casa alguma, porém vivia nos sepulcros.
\par 28 E, quando viu a Jesus, prostrou-se diante dele, exclamando e dizendo em alta voz: Que tenho eu contigo, Jesus, Filho do Deus Altíssimo? Rogo-te que não me atormentes.
\par 29 Porque Jesus ordenara ao espírito imundo que saísse do homem, pois muitas vezes se apoderara dele. E, embora procurassem conservá-lo preso com cadeias e grilhões, tudo despedaçava e era impelido pelo demônio para o deserto.
\par 30 Perguntou-lhe Jesus: Qual é o teu nome? Respondeu ele: Legião, porque tinham entrado nele muitos demônios.
\par 31 Rogavam-lhe que não os mandasse sair para o abismo.
\par 32 Ora, andava ali, pastando no monte, uma grande manada de porcos; rogaram-lhe que lhes permitisse entrar naqueles porcos. E Jesus o permitiu.
\par 33 Tendo os demônios saído do homem, entraram nos porcos, e a manada precipitou-se despenhadeiro abaixo, para dentro do lago, e se afogou.
\par 34 Os porqueiros, vendo o que acontecera, fugiram e foram anunciá-lo na cidade e pelos campos.
\par 35 Então, saiu o povo para ver o que se passara, e foram ter com Jesus. De fato, acharam o homem de quem saíram os demônios, vestido, em perfeito juízo, assentado aos pés de Jesus; e ficaram dominados de terror.
\par 36 E algumas pessoas que tinham presenciado os fatos contaram-lhes também como fora salvo o endemoninhado.
\par 37 Todo o povo da circunvizinhança dos gerasenos rogou-lhe que se retirasse deles, pois estavam possuídos de grande medo. E Jesus, tomando de novo o barco, voltou.
\par 38 O homem de quem tinham saído os demônios rogou-lhe que o deixasse estar com ele; Jesus, porém, o despediu, dizendo:
\par 39 Volta para casa e conta aos teus tudo o que Deus fez por ti. Então, foi ele anunciando por toda a cidade todas as coisas que Jesus lhe tinha feito.
\par 40 Ao regressar Jesus, a multidão o recebeu com alegria, porque todos o estavam esperando.
\par 41 Eis que veio um homem chamado Jairo, que era chefe da sinagoga, e, prostrando-se aos pés de Jesus, lhe suplicou que chegasse até a sua casa.
\par 42 Pois tinha uma filha única de uns doze anos, que estava à morte. Enquanto ele ia, as multidões o apertavam.
\par 43 Certa mulher que, havia doze anos, vinha sofrendo de uma hemorragia, e a quem ninguém tinha podido curar [e que gastara com os médicos todos os seus haveres],
\par 44 veio por trás dele e lhe tocou na orla da veste, e logo se lhe estancou a hemorragia.
\par 45 Mas Jesus disse: Quem me tocou? Como todos negassem, Pedro [com seus companheiros] disse: Mestre, as multidões te apertam e te oprimem [e dizes: Quem me tocou?].
\par 46 Contudo, Jesus insistiu: Alguém me tocou, porque senti que de mim saiu poder.
\par 47 Vendo a mulher que não podia ocultar-se, aproximou-se trêmula e, prostrando-se diante dele, declarou, à vista de todo o povo, a causa por que lhe havia tocado e como imediatamente fora curada.
\par 48 Então, lhe disse: Filha, a tua fé te salvou; vai-te em paz.
\par 49 Falava ele ainda, quando veio uma pessoa da casa do chefe da sinagoga, dizendo: Tua filha já está morta, não incomodes mais o Mestre.
\par 50 Mas Jesus, ouvindo isto, lhe disse: Não temas, crê somente, e ela será salva.
\par 51 Tendo chegado à casa, a ninguém permitiu que entrasse com ele, senão Pedro, João, Tiago e bem assim o pai e a mãe da menina.
\par 52 E todos choravam e a pranteavam. Mas ele disse: Não choreis; ela não está morta, mas dorme.
\par 53 E riam-se dele, porque sabiam que ela estava morta.
\par 54 Entretanto, ele, tomando-a pela mão, disse-lhe, em voz alta: Menina, levanta-te!
\par 55 Voltou-lhe o espírito, ela imediatamente se levantou, e ele mandou que lhe dessem de comer.
\par 56 Seus pais ficaram maravilhados, mas ele lhes advertiu que a ninguém contassem o que havia acontecido.

\chapter{9}

\par 1 Tendo Jesus convocado os doze, deu-lhes poder e autoridade sobre todos os demônios, e para efetuarem curas.
\par 2 Também os enviou a pregar o reino de Deus e a curar os enfermos.
\par 3 E disse-lhes: Nada leveis para o caminho: nem bordão, nem alforje, nem pão, nem dinheiro; nem deveis ter duas túnicas.
\par 4 Na casa em que entrardes, ali permanecei e dali saireis.
\par 5 E onde quer que não vos receberem, ao sairdes daquela cidade, sacudi o pó dos vossos pés em testemunho contra eles.
\par 6 Então, saindo, percorriam todas as aldeias, anunciando o evangelho e efetuando curas por toda parte.
\par 7 Ora, o tetrarca Herodes soube de tudo o que se passava e ficou perplexo, porque alguns diziam: João ressuscitou dentre os mortos;
\par 8 outros: Elias apareceu; e outros: Ressurgiu um dos antigos profetas.
\par 9 Herodes, porém, disse: Eu mandei decapitar a João; quem é, pois, este a respeito do qual tenho ouvido tais coisas? E se esforçava por vê-lo.
\par 10 Ao regressarem, os apóstolos relataram a Jesus tudo o que tinham feito. E, levando-os consigo, retirou-se à parte para uma cidade chamada Betsaida.
\par 11 Mas as multidões, ao saberem, seguiram-no. Acolhendo-as, falava-lhes a respeito do reino de Deus e socorria os que tinham necessidade de cura.
\par 12 Mas o dia começava a declinar. Então, se aproximaram os doze e lhe disseram: Despede a multidão, para que, indo às aldeias e campos circunvizinhos, se hospedem e achem alimento; pois estamos aqui em lugar deserto.
\par 13 Ele, porém, lhes disse: Dai-lhes vós mesmos de comer. Responderam eles: Não temos mais que cinco pães e dois peixes, salvo se nós mesmos formos comprar comida para todo este povo.
\par 14 Porque estavam ali cerca de cinco mil homens. Então, disse aos seus discípulos: Fazei-os sentar-se em grupos de cinqüenta.
\par 15 Eles atenderam, acomodando a todos.
\par 16 E, tomando os cinco pães e os dois peixes, erguendo os olhos para o céu, os abençoou, partiu e deu aos discípulos para que os distribuíssem entre o povo.
\par 17 Todos comeram e se fartaram; e dos pedaços que ainda sobejaram foram recolhidos doze cestos.
\par 18 Estando ele orando à parte, achavam-se presentes os discípulos, a quem perguntou: Quem dizem as multidões que sou eu?
\par 19 Responderam eles: João Batista, mas outros, Elias; e ainda outros dizem que ressurgiu um dos antigos profetas.
\par 20 Mas vós, perguntou ele, quem dizeis que eu sou? Então, falou Pedro e disse: És o Cristo de Deus.
\par 21 Ele, porém, advertindo-os, mandou que a ninguém declarassem tal coisa,
\par 22 dizendo: É necessário que o Filho do Homem sofra muitas coisas, seja rejeitado pelos anciãos, pelos principais sacerdotes e pelos escribas; seja morto e, no terceiro dia, ressuscite.
\par 23 Dizia a todos: Se alguém quer vir após mim, a si mesmo se negue, dia a dia tome a sua cruz e siga-me.
\par 24 Pois quem quiser salvar a sua vida perdê-la-á; quem perder a vida por minha causa, esse a salvará.
\par 25 Que aproveita ao homem ganhar o mundo inteiro, se vier a perder-se ou a causar dano a si mesmo?
\par 26 Porque qualquer que de mim e das minhas palavras se envergonhar, dele se envergonhará o Filho do Homem, quando vier na sua glória e na do Pai e dos santos anjos.
\par 27 Verdadeiramente, vos digo: alguns há dos que aqui se encontram que, de maneira nenhuma, passarão pela morte até que vejam o reino de Deus.
\par 28 Cerca de oito dias depois de proferidas estas palavras, tomando consigo a Pedro, João e Tiago, subiu ao monte com o propósito de orar.
\par 29 E aconteceu que, enquanto ele orava, a aparência do seu rosto se transfigurou e suas vestes resplandeceram de brancura.
\par 30 Eis que dois varões falavam com ele: Moisés e Elias,
\par 31 os quais apareceram em glória e falavam da sua partida, que ele estava para cumprir em Jerusalém.
\par 32 Pedro e seus companheiros achavam-se premidos de sono; mas, conservando-se acordados, viram a sua glória e os dois varões que com ele estavam.
\par 33 Ao se retirarem estes de Jesus, disse-lhe Pedro: Mestre, bom é estarmos aqui; então, façamos três tendas: uma será tua, outra, de Moisés, e outra, de Elias, não sabendo, porém, o que dizia.
\par 34 Enquanto assim falava, veio uma nuvem e os envolveu; e encheram-se de medo ao entrarem na nuvem.
\par 35 E dela veio uma voz, dizendo: Este é o meu Filho, o meu eleito; a ele ouvi.
\par 36 Depois daquela voz, achou-se Jesus sozinho. Eles calaram-se e, naqueles dias, a ninguém contaram coisa alguma do que tinham visto.
\par 37 No dia seguinte, ao descerem eles do monte, veio ao encontro de Jesus grande multidão.
\par 38 E eis que, dentre a multidão, surgiu um homem, dizendo em alta voz: Mestre, suplico-te que vejas meu filho, porque é o único;
\par 39 um espírito se apodera dele, e, de repente, o menino grita, e o espírito o atira por terra, convulsiona-o até espumar; e dificilmente o deixa, depois de o ter quebrantado.
\par 40 Roguei aos teus discípulos que o expelissem, mas eles não puderam.
\par 41 Respondeu Jesus: Ó geração incrédula e perversa! Até quando estarei convosco e vos sofrerei? Traze o teu filho.
\par 42 Quando se ia aproximando, o demônio o atirou no chão e o convulsionou; mas Jesus repreendeu o espírito imundo, curou o menino e o entregou a seu pai.
\par 43 E todos ficaram maravilhados ante a majestade de Deus. Como todos se maravilhassem de quanto Jesus fazia, disse aos seus discípulos:
\par 44 Fixai nos vossos ouvidos as seguintes palavras: o Filho do Homem está para ser entregue nas mãos dos homens.
\par 45 Eles, porém, não entendiam isto, e foi-lhes encoberto para que o não compreendessem; e temiam interrogá-lo a este respeito.
\par 46 Levantou-se entre eles uma discussão sobre qual deles seria o maior.
\par 47 Mas Jesus, sabendo o que se lhes passava no coração, tomou uma criança, colocou-a junto a si
\par 48 e lhes disse: Quem receber esta criança em meu nome a mim me recebe; e quem receber a mim recebe aquele que me enviou; porque aquele que entre vós for o menor de todos, esse é que é grande.
\par 49 Falou João e disse: Mestre, vimos certo homem que, em teu nome, expelia demônios e lho proibimos, porque não segue conosco.
\par 50 Mas Jesus lhe disse: Não proibais; pois quem não é contra vós outros é por vós.
\par 51 E aconteceu que, ao se completarem os dias em que devia ele ser assunto ao céu, manifestou, no semblante, a intrépida resolução de ir para Jerusalém
\par 52 e enviou mensageiros que o antecedessem. Indo eles, entraram numa aldeia de samaritanos para lhe preparar pousada.
\par 53 Mas não o receberam, porque o aspecto dele era de quem, decisivamente, ia para Jerusalém.
\par 54 Vendo isto, os discípulos Tiago e João perguntaram: Senhor, queres que mandemos descer fogo do céu para os consumir?
\par 55 Jesus, porém, voltando-se os repreendeu [e disse: Vós não sabeis de que espírito sois].
\par 56 [Pois o Filho do Homem não veio para destruir as almas dos homens, mas para salvá-las.] E seguiram para outra aldeia.
\par 57 Indo eles caminho fora, alguém lhe disse: Seguir-te-ei para onde quer que fores.
\par 58 Mas Jesus lhe respondeu: As raposas têm seus covis, e as aves do céu, ninhos; mas o Filho do Homem não tem onde reclinar a cabeça.
\par 59 A outro disse Jesus: Segue-me! Ele, porém, respondeu: Permite-me ir primeiro sepultar meu pai.
\par 60 Mas Jesus insistiu: Deixa aos mortos o sepultar os seus próprios mortos. Tu, porém, vai e prega o reino de Deus.
\par 61 Outro lhe disse: Seguir-te-ei, Senhor; mas deixa-me primeiro despedir-me dos de casa.
\par 62 Mas Jesus lhe replicou: Ninguém que, tendo posto a mão no arado, olha para trás é apto para o reino de Deus.

\chapter{10}

\par 1 Depois disto, o Senhor designou outros setenta; e os enviou de dois em dois, para que o precedessem em cada cidade e lugar aonde ele estava para ir.
\par 2 E lhes fez a seguinte advertência: A seara é grande, mas os trabalhadores são poucos. Rogai, pois, ao Senhor da seara que mande trabalhadores para a sua seara.
\par 3 Ide! Eis que eu vos envio como cordeiros para o meio de lobos.
\par 4 Não leveis bolsa, nem alforje, nem sandálias; e a ninguém saudeis pelo caminho.
\par 5 Ao entrardes numa casa, dizei antes de tudo: Paz seja nesta casa!
\par 6 Se houver ali um filho da paz, repousará sobre ele a vossa paz; se não houver, ela voltará sobre vós.
\par 7 Permanecei na mesma casa, comendo e bebendo do que eles tiverem; porque digno é o trabalhador do seu salário. Não andeis a mudar de casa em casa.
\par 8 Quando entrardes numa cidade e ali vos receberem, comei do que vos for oferecido.
\par 9 Curai os enfermos que nela houver e anunciai-lhes: A vós outros está próximo o reino de Deus.
\par 10 Quando, porém, entrardes numa cidade e não vos receberem, saí pelas ruas e clamai:
\par 11 Até o pó da vossa cidade, que se nos pegou aos pés, sacudimos contra vós outros. Não obstante, sabei que está próximo o reino de Deus.
\par 12 Digo-vos que, naquele dia, haverá menos rigor para Sodoma do que para aquela cidade.
\par 13 Ai de ti, Corazim! Ai de ti, Betsaida! Porque, se em Tiro e em Sidom, se tivessem operado os milagres que em vós se fizeram, há muito que elas se teriam arrependido, assentadas em pano de saco e cinza.
\par 14 Contudo, no Juízo, haverá menos rigor para Tiro e Sidom do que para vós outras.
\par 15 Tu, Cafarnaum, elevar-te-ás, porventura, até ao céu? Descerás até ao inferno.
\par 16 Quem vos der ouvidos ouve-me a mim; e quem vos rejeitar a mim me rejeita; quem, porém, me rejeitar rejeita aquele que me enviou.
\par 17 Então, regressaram os setenta, possuídos de alegria, dizendo: Senhor, os próprios demônios se nos submetem pelo teu nome!
\par 18 Mas ele lhes disse: Eu via Satanás caindo do céu como um relâmpago.
\par 19 Eis aí vos dei autoridade para pisardes serpentes e escorpiões e sobre todo o poder do inimigo, e nada, absolutamente, vos causará dano.
\par 20 Não obstante, alegrai-vos, não porque os espíritos se vos submetem, e sim porque o vosso nome está arrolado nos céus.
\par 21 Naquela hora, exultou Jesus no Espírito Santo e exclamou: Graças te dou, ó Pai, Senhor do céu e da terra, porque ocultaste estas coisas aos sábios e instruídos e as revelaste aos pequeninos. Sim, ó Pai, porque assim foi do teu agrado.
\par 22 Tudo me foi entregue por meu Pai. Ninguém sabe quem é o Filho, senão o Pai; e também ninguém sabe quem é o Pai, senão o Filho, e aquele a quem o Filho o quiser revelar.
\par 23 E, voltando-se para os seus discípulos, disse-lhes particularmente: Bem-aventurados os olhos que vêem as coisas que vós vedes.
\par 24 Pois eu vos afirmo que muitos profetas e reis quiseram ver o que vedes e não viram; e ouvir o que ouvis e não o ouviram.
\par 25 E eis que certo homem, intérprete da Lei, se levantou com o intuito de pôr Jesus à prova e disse-lhe: Mestre, que farei para herdar a vida eterna?
\par 26 Então, Jesus lhe perguntou: Que está escrito na Lei? Como interpretas?
\par 27 A isto ele respondeu: Amarás o Senhor, teu Deus, de todo o teu coração, de toda a tua alma, de todas as tuas forças e de todo o teu entendimento; e: Amarás o teu próximo como a ti mesmo.
\par 28 Então, Jesus lhe disse: Respondeste corretamente; faze isto e viverás.
\par 29 Ele, porém, querendo justificar-se, perguntou a Jesus: Quem é o meu próximo?
\par 30 Jesus prosseguiu, dizendo: Certo homem descia de Jerusalém para Jericó e veio a cair em mãos de salteadores, os quais, depois de tudo lhe roubarem e lhe causarem muitos ferimentos, retiraram-se, deixando-o semimorto.
\par 31 Casualmente, descia um sacerdote por aquele mesmo caminho e, vendo-o, passou de largo.
\par 32 Semelhantemente, um levita descia por aquele lugar e, vendo-o, também passou de largo.
\par 33 Certo samaritano, que seguia o seu caminho, passou-lhe perto e, vendo-o, compadeceu-se dele.
\par 34 E, chegando-se, pensou-lhe os ferimentos, aplicando-lhes óleo e vinho; e, colocando-o sobre o seu próprio animal, levou-o para uma hospedaria e tratou dele.
\par 35 No dia seguinte, tirou dois denários e os entregou ao hospedeiro, dizendo: Cuida deste homem, e, se alguma coisa gastares a mais, eu to indenizarei quando voltar.
\par 36 Qual destes três te parece ter sido o próximo do homem que caiu nas mãos dos salteadores?
\par 37 Respondeu-lhe o intérprete da Lei: O que usou de misericórdia para com ele. Então, lhe disse: Vai e procede tu de igual modo.
\par 38 Indo eles de caminho, entrou Jesus num povoado. E certa mulher, chamada Marta, hospedou-o na sua casa.
\par 39 Tinha ela uma irmã, chamada Maria, e esta quedava-se assentada aos pés do Senhor a ouvir-lhe os ensinamentos.
\par 40 Marta agitava-se de um lado para outro, ocupada em muitos serviços. Então, se aproximou de Jesus e disse: Senhor, não te importas de que minha irmã tenha deixado que eu fique a servir sozinha? Ordena-lhe, pois, que venha ajudar-me.
\par 41 Respondeu-lhe o Senhor: Marta! Marta! Andas inquieta e te preocupas com muitas coisas.
\par 42 Entretanto, pouco é necessário ou mesmo uma só coisa; Maria, pois, escolheu a boa parte, e esta não lhe será tirada.

\chapter{11}

\par 1 De uma feita, estava Jesus orando em certo lugar; quando terminou, um dos seus discípulos lhe pediu: Senhor, ensina-nos a orar como também João ensinou aos seus discípulos.
\par 2 Então, ele os ensinou: Quando orardes, dizei: Pai, santificado seja o teu nome; venha o teu reino;
\par 3 o pão nosso cotidiano dá-nos de dia em dia;
\par 4 perdoa-nos os nossos pecados, pois também nós perdoamos a todo o que nos deve; e não nos deixes cair em tentação.
\par 5 Disse-lhes ainda Jesus: Qual dentre vós, tendo um amigo, e este for procurá-lo à meia-noite e lhe disser: Amigo, empresta-me três pães,
\par 6 pois um meu amigo, chegando de viagem, procurou-me, e eu nada tenho que lhe oferecer.
\par 7 E o outro lhe responda lá de dentro, dizendo: Não me importunes; a porta já está fechada, e os meus filhos comigo também já estão deitados. Não posso levantar-me para tos dar;
\par 8 digo-vos que, se não se levantar para dar-lhos por ser seu amigo, todavia, o fará por causa da importunação e lhe dará tudo o de que tiver necessidade.
\par 9 Por isso, vos digo: Pedi, e dar-se-vos-á; buscai, e achareis; batei, e abrir-se-vos-á.
\par 10 Pois todo o que pede recebe; o que busca encontra; e a quem bate, abrir-se-lhe-á.
\par 11 Qual dentre vós é o pai que, se o filho lhe pedir [pão, lhe dará uma pedra? Ou se pedir] um peixe, lhe dará em lugar de peixe uma cobra?
\par 12 Ou, se lhe pedir um ovo lhe dará um escorpião?
\par 13 Ora, se vós, que sois maus, sabeis dar boas dádivas aos vossos filhos, quanto mais o Pai celestial dará o Espírito Santo àqueles que lho pedirem?
\par 14 De outra feita, estava Jesus expelindo um demônio que era mudo. E aconteceu que, ao sair o demônio, o mudo passou a falar; e as multidões se admiravam.
\par 15 Mas alguns dentre eles diziam: Ora, ele expele os demônios pelo poder de Belzebu, o maioral dos demônios.
\par 16 E outros, tentando-o, pediam dele um sinal do céu.
\par 17 E, sabendo ele o que se lhes passava pelo espírito, disse-lhes: Todo reino dividido contra si mesmo ficará deserto, e casa sobre casa cairá.
\par 18 Se também Satanás estiver dividido contra si mesmo, como subsistirá o seu reino? Isto, porque dizeis que eu expulso os demônios por Belzebu.
\par 19 E, se eu expulso os demônios por Belzebu, por quem os expulsam vossos filhos? Por isso, eles mesmos serão os vossos juízes.
\par 20 Se, porém, eu expulso os demônios pelo dedo de Deus, certamente, é chegado o reino de Deus sobre vós.
\par 21 Quando o valente, bem armado, guarda a sua própria casa, ficam em segurança todos os seus bens.
\par 22 Sobrevindo, porém, um mais valente do que ele, vence-o, tira-lhe a armadura em que confiava e lhe divide os despojos.
\par 23 Quem não é por mim é contra mim; e quem comigo não ajunta espalha.
\par 24 Quando o espírito imundo sai do homem, anda por lugares áridos, procurando repouso; e, não o achando, diz: Voltarei para minha casa, donde saí.
\par 25 E, tendo voltado, a encontra varrida e ornamentada.
\par 26 Então, vai e leva consigo outros sete espíritos, piores do que ele, e, entrando, habitam ali; e o último estado daquele homem se torna pior do que o primeiro.
\par 27 Ora, aconteceu que, ao dizer Jesus estas palavras, uma mulher, que estava entre a multidão, exclamou e disse-lhe: Bem-aventurada aquela que te concebeu, e os seios que te amamentaram!
\par 28 Ele, porém, respondeu: Antes, bem-aventurados são os que ouvem a palavra de Deus e a guardam!
\par 29 Como afluíssem as multidões, passou Jesus a dizer: Esta é geração perversa! Pede sinal; mas nenhum sinal lhe será dado, senão o de Jonas.
\par 30 Porque, assim como Jonas foi sinal para os ninivitas, o Filho do Homem o será para esta geração.
\par 31 A rainha do Sul se levantará, no Juízo, com os homens desta geração e os condenará; porque veio dos confins da terra para ouvir a sabedoria de Salomão. E eis aqui está quem é maior do que Salomão.
\par 32 Ninivitas se levantarão, no Juízo, com esta geração e a condenarão; porque se arrependeram com a pregação de Jonas. E eis aqui está quem é maior do que Jonas.
\par 33 Ninguém, depois de acender uma candeia, a põe em lugar escondido, nem debaixo do alqueire, mas no velador, a fim de que os que entram vejam a luz.
\par 34 São os teus olhos a lâmpada do teu corpo; se os teus olhos forem bons, todo o teu corpo será luminoso; mas, se forem maus, o teu corpo ficará em trevas.
\par 35 Repara, pois, que a luz que há em ti não sejam trevas.
\par 36 Se, portanto, todo o teu corpo for luminoso, sem ter qualquer parte em trevas, será todo resplandecente como a candeia quando te ilumina em plena luz.
\par 37 Ao falar Jesus estas palavras, um fariseu o convidou para ir comer com ele; então, entrando, tomou lugar à mesa.
\par 38 O fariseu, porém, admirou-se ao ver que Jesus não se lavara primeiro, antes de comer.
\par 39 O Senhor, porém, lhe disse: Vós, fariseus, limpais o exterior do copo e do prato; mas o vosso interior está cheio de rapina e perversidade.
\par 40 Insensatos! Quem fez o exterior não é o mesmo que fez o interior?
\par 41 Antes, dai esmola do que tiverdes, e tudo vos será limpo.
\par 42 Mas ai de vós, fariseus! Porque dais o dízimo da hortelã, da arruda e de todas as hortaliças e desprezais a justiça e o amor de Deus; devíeis, porém, fazer estas coisas, sem omitir aquelas.
\par 43 Ai de vós, fariseus! Porque gostais da primeira cadeira nas sinagogas e das saudações nas praças.
\par 44 Ai de vós que sois como as sepulturas invisíveis, sobre as quais os homens passam sem o saber!
\par 45 Então, respondendo um dos intérpretes da Lei, disse a Jesus: Mestre, dizendo estas coisas, também nos ofendes a nós outros!
\par 46 Mas ele respondeu: Ai de vós também, intérpretes da Lei! Porque sobrecarregais os homens com fardos superiores às suas forças, mas vós mesmos nem com um dedo os tocais.
\par 47 Ai de vós! Porque edificais os túmulos dos profetas que vossos pais assassinaram.
\par 48 Assim, sois testemunhas e aprovais com cumplicidade as obras dos vossos pais; porque eles mataram os profetas, e vós lhes edificais os túmulos.
\par 49 Por isso, também disse a sabedoria de Deus: Enviar-lhes-ei profetas e apóstolos, e a alguns deles matarão e a outros perseguirão,
\par 50 para que desta geração se peçam contas do sangue dos profetas, derramado desde a fundação do mundo;
\par 51 desde o sangue de Abel até ao de Zacarias, que foi assassinado entre o altar e a casa de Deus. Sim, eu vos afirmo, contas serão pedidas a esta geração.
\par 52 Ai de vós, intérpretes da Lei! Porque tomastes a chave da ciência; contudo, vós mesmos não entrastes e impedistes os que estavam entrando.
\par 53 Saindo Jesus dali, passaram os escribas e fariseus a argüi-lo com veemência, procurando confundi-lo a respeito de muitos assuntos,
\par 54 com o intuito de tirar das suas próprias palavras motivos para o acusar.

\chapter{12}

\par 1 Posto que miríades de pessoas se aglomeraram, a ponto de uns aos outros se atropelarem, passou Jesus a dizer, antes de tudo, aos seus discípulos: Acautelai-vos do fermento dos fariseus, que é a hipocrisia.
\par 2 Nada há encoberto que não venha a ser revelado; e oculto que não venha a ser conhecido.
\par 3 Porque tudo o que dissestes às escuras será ouvido em plena luz; e o que dissestes aos ouvidos no interior da casa será proclamado dos eirados.
\par 4 Digo-vos, pois, amigos meus: não temais os que matam o corpo e, depois disso, nada mais podem fazer.
\par 5 Eu, porém, vos mostrarei a quem deveis temer: temei aquele que, depois de matar, tem poder para lançar no inferno. Sim, digo-vos, a esse deveis temer.
\par 6 Não se vendem cinco pardais por dois asses? Entretanto, nenhum deles está em esquecimento diante de Deus.
\par 7 Até os cabelos da vossa cabeça estão todos contados. Não temais! Bem mais valeis do que muitos pardais.
\par 8 Digo-vos ainda: todo aquele que me confessar diante dos homens, também o Filho do Homem o confessará diante dos anjos de Deus;
\par 9 mas o que me negar diante dos homens será negado diante dos anjos de Deus.
\par 10 Todo aquele que proferir uma palavra contra o Filho do Homem, isso lhe será perdoado; mas, para o que blasfemar contra o Espírito Santo, não haverá perdão.
\par 11 Quando vos levarem às sinagogas e perante os governadores e as autoridades, não vos preocupeis quanto ao modo por que respondereis, nem quanto às coisas que tiverdes de falar.
\par 12 Porque o Espírito Santo vos ensinará, naquela mesma hora, as coisas que deveis dizer.
\par 13 Nesse ponto, um homem que estava no meio da multidão lhe falou: Mestre, ordena a meu irmão que reparta comigo a herança.
\par 14 Mas Jesus lhe respondeu: Homem, quem me constituiu juiz ou partidor entre vós?
\par 15 Então, lhes recomendou: Tende cuidado e guardai-vos de toda e qualquer avareza; porque a vida de um homem não consiste na abundância dos bens que ele possui.
\par 16 E lhes proferiu ainda uma parábola, dizendo: O campo de um homem rico produziu com abundância.
\par 17 E arrazoava consigo mesmo, dizendo: Que farei, pois não tenho onde recolher os meus frutos?
\par 18 E disse: Farei isto: destruirei os meus celeiros, reconstruí-los-ei maiores e aí recolherei todo o meu produto e todos os meus bens.
\par 19 Então, direi à minha alma: tens em depósito muitos bens para muitos anos; descansa, come, bebe e regala-te.
\par 20 Mas Deus lhe disse: Louco, esta noite te pedirão a tua alma; e o que tens preparado, para quem será?
\par 21 Assim é o que entesoura para si mesmo e não é rico para com Deus.
\par 22 A seguir, dirigiu-se Jesus a seus discípulos, dizendo: Por isso, eu vos advirto: não andeis ansiosos pela vossa vida, quanto ao que haveis de comer, nem pelo vosso corpo, quanto ao que haveis de vestir.
\par 23 Porque a vida é mais do que o alimento, e o corpo, mais do que as vestes.
\par 24 Observai os corvos, os quais não semeiam, nem ceifam, não têm despensa nem celeiros; todavia, Deus os sustenta. Quanto mais valeis do que as aves!
\par 25 Qual de vós, por ansioso que esteja, pode acrescentar um côvado ao curso da sua vida?
\par 26 Se, portanto, nada podeis fazer quanto às coisas mínimas, por que andais ansiosos pelas outras?
\par 27 Observai os lírios; eles não fiam, nem tecem. Eu, contudo, vos afirmo que nem Salomão, em toda a sua glória, se vestiu como qualquer deles.
\par 28 Ora, se Deus veste assim a erva que hoje está no campo e amanhã é lançada no forno, quanto mais tratando-se de vós, homens de pequena fé!
\par 29 Não andeis, pois, a indagar o que haveis de comer ou beber e não vos entregueis a inquietações.
\par 30 Porque os gentios de todo o mundo é que procuram estas coisas; mas vosso Pai sabe que necessitais delas.
\par 31 Buscai, antes de tudo, o seu reino, e estas coisas vos serão acrescentadas.
\par 32 Não temais, ó pequenino rebanho; porque vosso Pai se agradou em dar-vos o seu reino.
\par 33 Vendei os vossos bens e dai esmola; fazei para vós outros bolsas que não desgastem, tesouro inextinguível nos céus, onde não chega o ladrão, nem a traça consome,
\par 34 porque, onde está o vosso tesouro, aí estará também o vosso coração.
\par 35 Cingido esteja o vosso corpo, e acesas, as vossas candeias.
\par 36 Sede vós semelhantes a homens que esperam pelo seu senhor, ao voltar ele das festas de casamento; para que, quando vier e bater à porta, logo lha abram.
\par 37 Bem-aventurados aqueles servos a quem o senhor, quando vier, os encontre vigilantes; em verdade vos afirmo que ele há de cingir-se, dar-lhes lugar à mesa e, aproximando-se, os servirá.
\par 38 Quer ele venha na segunda vigília, quer na terceira, bem-aventurados serão eles, se assim os achar.
\par 39 Sabei, porém, isto: se o pai de família soubesse a que hora havia de vir o ladrão, [vigiaria e] não deixaria arrombar a sua casa.
\par 40 Ficai também vós apercebidos, porque, à hora em que não cuidais, o Filho do Homem virá.
\par 41 Então, Pedro perguntou: Senhor, proferes esta parábola para nós ou também para todos?
\par 42 Disse o Senhor: Quem é, pois, o mordomo fiel e prudente, a quem o senhor confiará os seus conservos para dar-lhes o sustento a seu tempo?
\par 43 Bem-aventurado aquele servo a quem seu senhor, quando vier, achar fazendo assim.
\par 44 Verdadeiramente, vos digo que lhe confiará todos os seus bens.
\par 45 Mas, se aquele servo disser consigo mesmo: Meu senhor tarda em vir, e passar a espancar os criados e as criadas, a comer, a beber e a embriagar-se,
\par 46 virá o senhor daquele servo, em dia em que não o espera e em hora que não sabe, e castigá-lo-á, lançando-lhe a sorte com os infiéis.
\par 47 Aquele servo, porém, que conheceu a vontade de seu senhor e não se aprontou, nem fez segundo a sua vontade será punido com muitos açoites.
\par 48 Aquele, porém, que não soube a vontade do seu senhor e fez coisas dignas de reprovação levará poucos açoites. Mas àquele a quem muito foi dado, muito lhe será exigido; e àquele a quem muito se confia, muito mais lhe pedirão.
\par 49 Eu vim para lançar fogo sobre a terra e bem quisera que já estivesse a arder.
\par 50 Tenho, porém, um batismo com o qual hei de ser batizado; e quanto me angustio até que o mesmo se realize!
\par 51 Supondes que vim para dar paz à terra? Não, eu vo-lo afirmo; antes, divisão.
\par 52 Porque, daqui em diante, estarão cinco divididos numa casa: três contra dois, e dois contra três.
\par 53 Estarão divididos: pai contra filho, filho contra pai; mãe contra filha, filha contra mãe; sogra contra nora, e nora contra sogra.
\par 54 Disse também às multidões: Quando vedes aparecer uma nuvem no poente, logo dizeis que vem chuva, e assim acontece;
\par 55 e, quando vedes soprar o vento sul, dizeis que haverá calor, e assim acontece.
\par 56 Hipócritas, sabeis interpretar o aspecto da terra e do céu e, entretanto, não sabeis discernir esta época?
\par 57 E por que não julgais também por vós mesmos o que é justo?
\par 58 Quando fores com o teu adversário ao magistrado, esforça-te para te livrares desse adversário no caminho; para que não suceda que ele te arraste ao juiz, o juiz te entregue ao meirinho e o meirinho te recolha à prisão.
\par 59 Digo-te que não sairás dali enquanto não pagares o último centavo.

\chapter{13}

\par 1 Naquela mesma ocasião, chegando alguns, falavam a Jesus a respeito dos galileus cujo sangue Pilatos misturara com os sacrifícios que os mesmos realizavam.
\par 2 Ele, porém, lhes disse: Pensais que esses galileus eram mais pecadores do que todos os outros galileus, por terem padecido estas coisas?
\par 3 Não eram, eu vo-lo afirmo; se, porém, não vos arrependerdes, todos igualmente perecereis.
\par 4 Ou cuidais que aqueles dezoito sobre os quais desabou a torre de Siloé e os matou eram mais culpados que todos os outros habitantes de Jerusalém?
\par 5 Não eram, eu vo-lo afirmo; mas, se não vos arrependerdes, todos igualmente perecereis.
\par 6 Então, Jesus proferiu a seguinte parábola: Certo homem tinha uma figueira plantada na sua vinha e, vindo procurar fruto nela, não achou.
\par 7 Pelo que disse ao viticultor: Há três anos venho procurar fruto nesta figueira e não acho; podes cortá-la; para que está ela ainda ocupando inutilmente a terra?
\par 8 Ele, porém, respondeu: Senhor, deixa-a ainda este ano, até que eu escave ao redor dela e lhe ponha estrume.
\par 9 Se vier a dar fruto, bem está; se não, mandarás cortá-la.
\par 10 Ora, ensinava Jesus no sábado numa das sinagogas.
\par 11 E veio ali uma mulher possessa de um espírito de enfermidade, havia já dezoito anos; andava ela encurvada, sem de modo algum poder endireitar-se.
\par 12 Vendo-a Jesus, chamou-a e disse-lhe: Mulher, estás livre da tua enfermidade;
\par 13 e, impondo-lhe as mãos, ela imediatamente se endireitou e dava glória a Deus.
\par 14 O chefe da sinagoga, indignado de ver que Jesus curava no sábado, disse à multidão: Seis dias há em que se deve trabalhar; vinde, pois, nesses dias para serdes curados e não no sábado.
\par 15 Disse-lhe, porém, o Senhor: Hipócritas, cada um de vós não desprende da manjedoura, no sábado, o seu boi ou o seu jumento, para levá-lo a beber?
\par 16 Por que motivo não se devia livrar deste cativeiro, em dia de sábado, esta filha de Abraão, a quem Satanás trazia presa há dezoito anos?
\par 17 Tendo ele dito estas palavras, todos os seus adversários se envergonharam. Entretanto, o povo se alegrava por todos os gloriosos feitos que Jesus realizava.
\par 18 E dizia: A que é semelhante o reino de Deus, e a que o compararei?
\par 19 É semelhante a um grão de mostarda que um homem plantou na sua horta; e cresceu e fez-se árvore; e as aves do céu aninharam-se nos seus ramos.
\par 20 Disse mais: A que compararei o reino de Deus?
\par 21 É semelhante ao fermento que uma mulher tomou e escondeu em três medidas de farinha, até ficar tudo levedado.
\par 22 Passava Jesus por cidades e aldeias, ensinando e caminhando para Jerusalém.
\par 23 E alguém lhe perguntou: Senhor, são poucos os que são salvos?
\par 24 Respondeu-lhes: Esforçai-vos por entrar pela porta estreita, pois eu vos digo que muitos procurarão entrar e não poderão.
\par 25 Quando o dono da casa se tiver levantado e fechado a porta, e vós, do lado de fora, começardes a bater, dizendo: Senhor, abre-nos a porta, ele vos responderá: Não sei donde sois.
\par 26 Então, direis: Comíamos e bebíamos na tua presença, e ensinavas em nossas ruas.
\par 27 Mas ele vos dirá: Não sei donde vós sois; apartai-vos de mim, vós todos os que praticais iniqüidades.
\par 28 Ali haverá choro e ranger de dentes, quando virdes, no reino de Deus, Abraão, Isaque, Jacó e todos os profetas, mas vós, lançados fora.
\par 29 Muitos virão do Oriente e do Ocidente, do Norte e do Sul e tomarão lugares à mesa no reino de Deus.
\par 30 Contudo, há últimos que virão a ser primeiros, e primeiros que serão últimos.
\par 31 Naquela mesma hora, alguns fariseus vieram para dizer-lhe: Retira-te e vai-te daqui, porque Herodes quer matar-te.
\par 32 Ele, porém, lhes respondeu: Ide dizer a essa raposa que, hoje e amanhã, expulso demônios e curo enfermos e, no terceiro dia, terminarei.
\par 33 Importa, contudo, caminhar hoje, amanhã e depois, porque não se espera que um profeta morra fora de Jerusalém.
\par 34 Jerusalém, Jerusalém, que matas os profetas e apedrejas os que te foram enviados! Quantas vezes quis eu reunir teus filhos como a galinha ajunta os do seu próprio ninho debaixo das asas, e vós não o quisestes!
\par 35 Eis que a vossa casa vos ficará deserta. E em verdade vos digo que não mais me vereis até que venhais a dizer: Bendito o que vem em nome do Senhor!

\chapter{14}

\par 1 Aconteceu que, ao entrar ele num sábado na casa de um dos principais fariseus para comer pão, eis que o estavam observando.
\par 2 Ora, diante dele se achava um homem hidrópico.
\par 3 Então, Jesus, dirigindo-se aos intérpretes da Lei e aos fariseus, perguntou-lhes: É ou não é lícito curar no sábado?
\par 4 Eles, porém, nada disseram. E, tomando-o, o curou e o despediu.
\par 5 A seguir, lhes perguntou: Qual de vós, se o filho ou o boi cair num poço, não o tirará logo, mesmo em dia de sábado?
\par 6 A isto nada puderam responder.
\par 7 Reparando como os convidados escolhiam os primeiros lugares, propôs-lhes uma parábola:
\par 8 Quando por alguém fores convidado para um casamento, não procures o primeiro lugar; para não suceder que, havendo um convidado mais digno do que tu,
\par 9 vindo aquele que te convidou e também a ele, te diga: Dá o lugar a este. Então, irás, envergonhado, ocupar o último lugar.
\par 10 Pelo contrário, quando fores convidado, vai tomar o último lugar; para que, quando vier o que te convidou, te diga: Amigo, senta-te mais para cima. Ser-te-á isto uma honra diante de todos os mais convivas.
\par 11 Pois todo o que se exalta será humilhado; e o que se humilha será exaltado.
\par 12 Disse também ao que o havia convidado: Quando deres um jantar ou uma ceia, não convides os teus amigos, nem teus irmãos, nem teus parentes, nem vizinhos ricos; para não suceder que eles, por sua vez, te convidem e sejas recompensado.
\par 13 Antes, ao dares um banquete, convida os pobres, os aleijados, os coxos e os cegos;
\par 14 e serás bem-aventurado, pelo fato de não terem eles com que recompensar-te; a tua recompensa, porém, tu a receberás na ressurreição dos justos.
\par 15 Ora, ouvindo tais palavras, um dos que estavam com ele à mesa, disse-lhe: Bem-aventurado aquele que comer pão no reino de Deus.
\par 16 Ele, porém, respondeu: Certo homem deu uma grande ceia e convidou muitos.
\par 17 À hora da ceia, enviou o seu servo para avisar aos convidados: Vinde, porque tudo já está preparado.
\par 18 Não obstante, todos, à uma, começaram a escusar-se. Disse o primeiro: Comprei um campo e preciso ir vê-lo; rogo-te que me tenhas por escusado.
\par 19 Outro disse: Comprei cinco juntas de bois e vou experimentá-las; rogo-te que me tenhas por escusado.
\par 20 E outro disse: Casei-me e, por isso, não posso ir.
\par 21 Voltando o servo, tudo contou ao seu senhor. Então, irado, o dono da casa disse ao seu servo: Sai depressa para as ruas e becos da cidade e traze para aqui os pobres, os aleijados, os cegos e os coxos.
\par 22 Depois, lhe disse o servo: Senhor, feito está como mandaste, e ainda há lugar.
\par 23 Respondeu-lhe o senhor: Sai pelos caminhos e atalhos e obriga a todos a entrar, para que fique cheia a minha casa.
\par 24 Porque vos declaro que nenhum daqueles homens que foram convidados provará a minha ceia.
\par 25 Grandes multidões o acompanhavam, e ele, voltando-se, lhes disse:
\par 26 Se alguém vem a mim e não aborrece a seu pai, e mãe, e mulher, e filhos, e irmãos, e irmãs e ainda a sua própria vida, não pode ser meu discípulo.
\par 27 E qualquer que não tomar a sua cruz e vier após mim não pode ser meu discípulo.
\par 28 Pois qual de vós, pretendendo construir uma torre, não se assenta primeiro para calcular a despesa e verificar se tem os meios para a concluir?
\par 29 Para não suceder que, tendo lançado os alicerces e não a podendo acabar, todos os que a virem zombem dele,
\par 30 dizendo: Este homem começou a construir e não pôde acabar.
\par 31 Ou qual é o rei que, indo para combater outro rei, não se assenta primeiro para calcular se com dez mil homens poderá enfrentar o que vem contra ele com vinte mil?
\par 32 Caso contrário, estando o outro ainda longe, envia-lhe uma embaixada, pedindo condições de paz.
\par 33 Assim, pois, todo aquele que dentre vós não renuncia a tudo quanto tem não pode ser meu discípulo.
\par 34 O sal é certamente bom; caso, porém, se torne insípido, como restaurar-lhe o sabor?
\par 35 Nem presta para a terra, nem mesmo para o monturo; lançam-no fora. Quem tem ouvidos para ouvir, ouça.

\chapter{15}

\par 1 Aproximavam-se de Jesus todos os publicanos e pecadores para o ouvir.
\par 2 E murmuravam os fariseus e os escribas, dizendo: Este recebe pecadores e come com eles.
\par 3 Então, lhes propôs Jesus esta parábola:
\par 4 Qual, dentre vós, é o homem que, possuindo cem ovelhas e perdendo uma delas, não deixa no deserto as noventa e nove e vai em busca da que se perdeu, até encontrá-la?
\par 5 Achando-a, põe-na sobre os ombros, cheio de júbilo.
\par 6 E, indo para casa, reúne os amigos e vizinhos, dizendo-lhes: Alegrai-vos comigo, porque já achei a minha ovelha perdida.
\par 7 Digo-vos que, assim, haverá maior júbilo no céu por um pecador que se arrepende do que por noventa e nove justos que não necessitam de arrependimento.
\par 8 Ou qual é a mulher que, tendo dez dracmas, se perder uma, não acende a candeia, varre a casa e a procura diligentemente até encontrá-la?
\par 9 E, tendo-a achado, reúne as amigas e vizinhas, dizendo: Alegrai-vos comigo, porque achei a dracma que eu tinha perdido.
\par 10 Eu vos afirmo que, de igual modo, há júbilo diante dos anjos de Deus por um pecador que se arrepende.
\par 11 Continuou: Certo homem tinha dois filhos;
\par 12 o mais moço deles disse ao pai: Pai, dá-me a parte dos bens que me cabe. E ele lhes repartiu os haveres.
\par 13 Passados não muitos dias, o filho mais moço, ajuntando tudo o que era seu, partiu para uma terra distante e lá dissipou todos os seus bens, vivendo dissolutamente.
\par 14 Depois de ter consumido tudo, sobreveio àquele país uma grande fome, e ele começou a passar necessidade.
\par 15 Então, ele foi e se agregou a um dos cidadãos daquela terra, e este o mandou para os seus campos a guardar porcos.
\par 16 Ali, desejava ele fartar-se das alfarrobas que os porcos comiam; mas ninguém lhe dava nada.
\par 17 Então, caindo em si, disse: Quantos trabalhadores de meu pai têm pão com fartura, e eu aqui morro de fome!
\par 18 Levantar-me-ei, e irei ter com o meu pai, e lhe direi: Pai, pequei contra o céu e diante de ti;
\par 19 já não sou digno de ser chamado teu filho; trata-me como um dos teus trabalhadores.
\par 20 E, levantando-se, foi para seu pai. Vinha ele ainda longe, quando seu pai o avistou, e, compadecido dele, correndo, o abraçou, e beijou.
\par 21 E o filho lhe disse: Pai, pequei contra o céu e diante de ti; já não sou digno de ser chamado teu filho.
\par 22 O pai, porém, disse aos seus servos: Trazei depressa a melhor roupa, vesti-o, ponde-lhe um anel no dedo e sandálias nos pés;
\par 23 trazei também e matai o novilho cevado. Comamos e regozijemo-nos,
\par 24 porque este meu filho estava morto e reviveu, estava perdido e foi achado. E começaram a regozijar-se.
\par 25 Ora, o filho mais velho estivera no campo; e, quando voltava, ao aproximar-se da casa, ouviu a música e as danças.
\par 26 Chamou um dos criados e perguntou-lhe que era aquilo.
\par 27 E ele informou: Veio teu irmão, e teu pai mandou matar o novilho cevado, porque o recuperou com saúde.
\par 28 Ele se indignou e não queria entrar; saindo, porém, o pai, procurava conciliá-lo.
\par 29 Mas ele respondeu a seu pai: Há tantos anos que te sirvo sem jamais transgredir uma ordem tua, e nunca me deste um cabrito sequer para alegrar-me com os meus amigos;
\par 30 vindo, porém, esse teu filho, que desperdiçou os teus bens com meretrizes, tu mandaste matar para ele o novilho cevado.
\par 31 Então, lhe respondeu o pai: Meu filho, tu sempre estás comigo; tudo o que é meu é teu.
\par 32 Entretanto, era preciso que nos regozijássemos e nos alegrássemos, porque esse teu irmão estava morto e reviveu, estava perdido e foi achado.

\chapter{16}

\par 1 Disse Jesus também aos discípulos: Havia um homem rico que tinha um administrador; e este lhe foi denunciado como quem estava a defraudar os seus bens.
\par 2 Então, mandando-o chamar, lhe disse: Que é isto que ouço a teu respeito? Presta contas da tua administração, porque já não podes mais continuar nela.
\par 3 Disse o administrador consigo mesmo: Que farei, pois o meu senhor me tira a administração? Trabalhar na terra não posso; também de mendigar tenho vergonha.
\par 4 Eu sei o que farei, para que, quando for demitido da administração, me recebam em suas casas.
\par 5 Tendo chamado cada um dos devedores do seu senhor, disse ao primeiro: Quanto deves ao meu patrão?
\par 6 Respondeu ele: Cem cados de azeite. Então, disse: Toma a tua conta, assenta-te depressa e escreve cinqüenta.
\par 7 Depois, perguntou a outro: Tu, quanto deves? Respondeu ele: Cem coros de trigo. Disse-lhe: Toma a tua conta e escreve oitenta.
\par 8 E elogiou o senhor o administrador infiel porque se houvera atiladamente, porque os filhos do mundo são mais hábeis na sua própria geração do que os filhos da luz.
\par 9 E eu vos recomendo: das riquezas de origem iníqua fazei amigos; para que, quando aquelas vos faltarem, esses amigos vos recebam nos tabernáculos eternos.
\par 10 Quem é fiel no pouco também é fiel no muito; e quem é injusto no pouco também é injusto no muito.
\par 11 Se, pois, não vos tornastes fiéis na aplicação das riquezas de origem injusta, quem vos confiará a verdadeira riqueza?
\par 12 Se não vos tornastes fiéis na aplicação do alheio, quem vos dará o que é vosso?
\par 13 Ninguém pode servir a dois senhores; porque ou há de aborrecer-se de um e amar ao outro ou se devotará a um e desprezará ao outro. Não podeis servir a Deus e às riquezas.
\par 14 Os fariseus, que eram avarentos, ouviam tudo isto e o ridiculizavam.
\par 15 Mas Jesus lhes disse: Vós sois os que vos justificais a vós mesmos diante dos homens, mas Deus conhece o vosso coração; pois aquilo que é elevado entre homens é abominação diante de Deus.
\par 16 A Lei e os Profetas vigoraram até João; desde esse tempo, vem sendo anunciado o evangelho do reino de Deus, e todo homem se esforça por entrar nele.
\par 17 E é mais fácil passar o céu e a terra do que cair um til sequer da Lei.
\par 18 Quem repudiar sua mulher e casar com outra comete adultério; e aquele que casa com a mulher repudiada pelo marido também comete adultério.
\par 19 Ora, havia certo homem rico que se vestia de púrpura e de linho finíssimo e que, todos os dias, se regalava esplendidamente.
\par 20 Havia também certo mendigo, chamado Lázaro, coberto de chagas, que jazia à porta daquele;
\par 21 e desejava alimentar-se das migalhas que caíam da mesa do rico; e até os cães vinham lamber-lhe as úlceras.
\par 22 Aconteceu morrer o mendigo e ser levado pelos anjos para o seio de Abraão; morreu também o rico e foi sepultado.
\par 23 No inferno, estando em tormentos, levantou os olhos e viu ao longe a Abraão e Lázaro no seu seio.
\par 24 Então, clamando, disse: Pai Abraão, tem misericórdia de mim! E manda a Lázaro que molhe em água a ponta do dedo e me refresque a língua, porque estou atormentado nesta chama.
\par 25 Disse, porém, Abraão: Filho, lembra-te de que recebeste os teus bens em tua vida, e Lázaro igualmente, os males; agora, porém, aqui, ele está consolado; tu, em tormentos.
\par 26 E, além de tudo, está posto um grande abismo entre nós e vós, de sorte que os que querem passar daqui para vós outros não podem, nem os de lá passar para nós.
\par 27 Então, replicou: Pai, eu te imploro que o mandes à minha casa paterna,
\par 28 porque tenho cinco irmãos; para que lhes dê testemunho, a fim de não virem também para este lugar de tormento.
\par 29 Respondeu Abraão: Eles têm Moisés e os Profetas; ouçam-nos.
\par 30 Mas ele insistiu: Não, pai Abraão; se alguém dentre os mortos for ter com eles, arrepender-se-ão.
\par 31 Abraão, porém, lhe respondeu: Se não ouvem a Moisés e aos Profetas, tampouco se deixarão persuadir, ainda que ressuscite alguém dentre os mortos.

\chapter{17}

\par 1 Disse Jesus a seus discípulos: É inevitável que venham escândalos, mas ai do homem pelo qual eles vêm!
\par 2 Melhor fora que se lhe pendurasse ao pescoço uma pedra de moinho, e fosse atirado no mar, do que fazer tropeçar a um destes pequeninos.
\par 3 Acautelai-vos. Se teu irmão pecar contra ti, repreende-o; se ele se arrepender, perdoa-lhe.
\par 4 Se, por sete vezes no dia, pecar contra ti e, sete vezes, vier ter contigo, dizendo: Estou arrependido, perdoa-lhe.
\par 5 Então, disseram os apóstolos ao Senhor: Aumenta-nos a fé.
\par 6 Respondeu-lhes o Senhor: Se tiverdes fé como um grão de mostarda, direis a esta amoreira: Arranca-te e transplanta-te no mar; e ela vos obedecerá.
\par 7 Qual de vós, tendo um servo ocupado na lavoura ou em guardar o gado, lhe dirá quando ele voltar do campo: Vem já e põe-te à mesa?
\par 8 E que, antes, não lhe diga: Prepara-me a ceia, cinge-te e serve-me, enquanto eu como e bebo; depois, comerás tu e beberás?
\par 9 Porventura, terá de agradecer ao servo porque este fez o que lhe havia ordenado?
\par 10 Assim também vós, depois de haverdes feito quanto vos foi ordenado, dizei: Somos servos inúteis, porque fizemos apenas o que devíamos fazer.
\par 11 De caminho para Jerusalém, passava Jesus pelo meio de Samaria e da Galiléia.
\par 12 Ao entrar numa aldeia, saíram-lhe ao encontro dez leprosos,
\par 13 que ficaram de longe e lhe gritaram, dizendo: Jesus, Mestre, compadece-te de nós!
\par 14 Ao vê-los, disse-lhes Jesus: Ide e mostrai-vos aos sacerdotes. Aconteceu que, indo eles, foram purificados.
\par 15 Um dos dez, vendo que fora curado, voltou, dando glória a Deus em alta voz,
\par 16 e prostrou-se com o rosto em terra aos pés de Jesus, agradecendo-lhe; e este era samaritano.
\par 17 Então, Jesus lhe perguntou: Não eram dez os que foram curados? Onde estão os nove?
\par 18 Não houve, porventura, quem voltasse para dar glória a Deus, senão este estrangeiro?
\par 19 E disse-lhe: Levanta-te e vai; a tua fé te salvou.
\par 20 Interrogado pelos fariseus sobre quando viria o reino de Deus, Jesus lhes respondeu: Não vem o reino de Deus com visível aparência.
\par 21 Nem dirão: Ei-lo aqui! Ou: Lá está! Porque o reino de Deus está dentro de vós.
\par 22 A seguir, dirigiu-se aos discípulos: Virá o tempo em que desejareis ver um dos dias do Filho do Homem e não o vereis.
\par 23 E vos dirão: Ei-lo aqui! Ou: Lá está! Não vades nem os sigais;
\par 24 porque assim como o relâmpago, fuzilando, brilha de uma à outra extremidade do céu, assim será, no seu dia, o Filho do Homem.
\par 25 Mas importa que primeiro ele padeça muitas coisas e seja rejeitado por esta geração.
\par 26 Assim como foi nos dias de Noé, será também nos dias do Filho do Homem:
\par 27 comiam, bebiam, casavam e davam-se em casamento, até ao dia em que Noé entrou na arca, e veio o dilúvio e destruiu a todos.
\par 28 O mesmo aconteceu nos dias de Ló: comiam, bebiam, compravam, vendiam, plantavam e edificavam;
\par 29 mas, no dia em que Ló saiu de Sodoma, choveu do céu fogo e enxofre e destruiu a todos.
\par 30 Assim será no dia em que o Filho do Homem se manifestar.
\par 31 Naquele dia, quem estiver no eirado e tiver os seus bens em casa não desça para tirá-los; e de igual modo quem estiver no campo não volte para trás.
\par 32 Lembrai-vos da mulher de Ló.
\par 33 Quem quiser preservar a sua vida perdê-la-á; e quem a perder de fato a salvará.
\par 34 Digo-vos que, naquela noite, dois estarão numa cama; um será tomado, e deixado o outro;
\par 35 duas mulheres estarão juntas moendo; uma será tomada, e deixada a outra.
\par 36 [Dois estarão no campo; um será tomado, e o outro, deixado.]
\par 37 Então, lhe perguntaram: Onde será isso, Senhor? Respondeu-lhes: Onde estiver o corpo, aí se ajuntarão também os abutres.

\chapter{18}

\par 1 Disse-lhes Jesus uma parábola sobre o dever de orar sempre e nunca esmorecer:
\par 2 Havia em certa cidade um juiz que não temia a Deus, nem respeitava homem algum.
\par 3 Havia também, naquela mesma cidade, uma viúva que vinha ter com ele, dizendo: Julga a minha causa contra o meu adversário.
\par 4 Ele, por algum tempo, não a quis atender; mas, depois, disse consigo: Bem que eu não temo a Deus, nem respeito a homem algum;
\par 5 todavia, como esta viúva me importuna, julgarei a sua causa, para não suceder que, por fim, venha a molestar-me.
\par 6 Então, disse o Senhor: Considerai no que diz este juiz iníquo.
\par 7 Não fará Deus justiça aos seus escolhidos, que a ele clamam dia e noite, embora pareça demorado em defendê-los?
\par 8 Digo-vos que, depressa, lhes fará justiça. Contudo, quando vier o Filho do Homem, achará, porventura, fé na terra?
\par 9 Propôs também esta parábola a alguns que confiavam em si mesmos, por se considerarem justos, e desprezavam os outros:
\par 10 Dois homens subiram ao templo com o propósito de orar: um, fariseu, e o outro, publicano.
\par 11 O fariseu, posto em pé, orava de si para si mesmo, desta forma: Ó Deus, graças te dou porque não sou como os demais homens, roubadores, injustos e adúlteros, nem ainda como este publicano;
\par 12 jejuo duas vezes por semana e dou o dízimo de tudo quanto ganho.
\par 13 O publicano, estando em pé, longe, não ousava nem ainda levantar os olhos ao céu, mas batia no peito, dizendo: Ó Deus, sê propício a mim, pecador!
\par 14 Digo-vos que este desceu justificado para sua casa, e não aquele; porque todo o que se exalta será humilhado; mas o que se humilha será exaltado.
\par 15 Traziam-lhe também as crianças, para que as tocasse; e os discípulos, vendo, os repreendiam.
\par 16 Jesus, porém, chamando-as para junto de si, ordenou: Deixai vir a mim os pequeninos e não os embaraceis, porque dos tais é o reino de Deus.
\par 17 Em verdade vos digo: Quem não receber o reino de Deus como uma criança de maneira alguma entrará nele.
\par 18 Certo homem de posição perguntou-lhe: Bom Mestre, que farei para herdar a vida eterna?
\par 19 Respondeu-lhe Jesus: Por que me chamas bom? Ninguém é bom, senão um, que é Deus.
\par 20 Sabes os mandamentos: Não adulterarás, não matarás, não furtarás, não dirás falso testemunho, honra a teu pai e a tua mãe.
\par 21 Replicou ele: Tudo isso tenho observado desde a minha juventude.
\par 22 Ouvindo-o Jesus, disse-lhe: Uma coisa ainda te falta: vende tudo o que tens, dá-o aos pobres e terás um tesouro nos céus; depois, vem e segue-me.
\par 23 Mas, ouvindo ele estas palavras, ficou muito triste, porque era riquíssimo.
\par 24 E Jesus, vendo-o assim triste, disse: Quão dificilmente entrarão no reino de Deus os que têm riquezas!
\par 25 Porque é mais fácil passar um camelo pelo fundo de uma agulha do que entrar um rico no reino de Deus.
\par 26 E os que ouviram disseram: Sendo assim, quem pode ser salvo?
\par 27 Mas ele respondeu: Os impossíveis dos homens são possíveis para Deus.
\par 28 E disse Pedro: Eis que nós deixamos nossa casa e te seguimos.
\par 29 Respondeu-lhes Jesus: Em verdade vos digo que ninguém há que tenha deixado casa, ou mulher, ou irmãos, ou pais, ou filhos, por causa do reino de Deus,
\par 30 que não receba, no presente, muitas vezes mais e, no mundo por vir, a vida eterna.
\par 31 Tomando consigo os doze, disse-lhes Jesus: Eis que subimos para Jerusalém, e vai cumprir-se ali tudo quanto está escrito por intermédio dos profetas, no tocante ao Filho do Homem;
\par 32 pois será ele entregue aos gentios, escarnecido, ultrajado e cuspido;
\par 33 e, depois de o açoitarem, tirar-lhe-ão a vida; mas, ao terceiro dia, ressuscitará.
\par 34 Eles, porém, nada compreenderam acerca destas coisas; e o sentido destas palavras era-lhes encoberto, de sorte que não percebiam o que ele dizia.
\par 35 Aconteceu que, ao aproximar-se ele de Jericó, estava um cego assentado à beira do caminho, pedindo esmolas.
\par 36 E, ouvindo o tropel da multidão que passava, perguntou o que era aquilo.
\par 37 Anunciaram-lhe que passava Jesus, o Nazareno.
\par 38 Então, ele clamou: Jesus, Filho de Davi, tem compaixão de mim!
\par 39 E os que iam na frente o repreendiam para que se calasse; ele, porém, cada vez gritava mais: Filho de Davi, tem misericórdia de mim!
\par 40 Então, parou Jesus e mandou que lho trouxessem. E, tendo ele chegado, perguntou-lhe:
\par 41 Que queres que eu te faça? Respondeu ele: Senhor, que eu torne a ver.
\par 42 Então, Jesus lhe disse: Recupera a tua vista; a tua fé te salvou.
\par 43 Imediatamente, tornou a ver e seguia-o glorificando a Deus. Também todo o povo, vendo isto, dava louvores a Deus.

\chapter{19}

\par 1 Entrando em Jericó, atravessava Jesus a cidade.
\par 2 Eis que um homem, chamado Zaqueu, maioral dos publicanos e rico,
\par 3 procurava ver quem era Jesus, mas não podia, por causa da multidão, por ser ele de pequena estatura.
\par 4 Então, correndo adiante, subiu a um sicômoro a fim de vê-lo, porque por ali havia de passar.
\par 5 Quando Jesus chegou àquele lugar, olhando para cima, disse-lhe: Zaqueu, desce depressa, pois me convém ficar hoje em tua casa.
\par 6 Ele desceu a toda a pressa e o recebeu com alegria.
\par 7 Todos os que viram isto murmuravam, dizendo que ele se hospedara com homem pecador.
\par 8 Entrementes, Zaqueu se levantou e disse ao Senhor: Senhor, resolvo dar aos pobres a metade dos meus bens; e, se nalguma coisa tenho defraudado alguém, restituo quatro vezes mais.
\par 9 Então, Jesus lhe disse: Hoje, houve salvação nesta casa, pois que também este é filho de Abraão.
\par 10 Porque o Filho do Homem veio buscar e salvar o perdido.
\par 11 Ouvindo eles estas coisas, Jesus propôs uma parábola, visto estar perto de Jerusalém e lhes parecer que o reino de Deus havia de manifestar-se imediatamente.
\par 12 Então, disse: Certo homem nobre partiu para uma terra distante, com o fim de tomar posse de um reino e voltar.
\par 13 Chamou dez servos seus, confiou-lhes dez minas e disse-lhes: Negociai até que eu volte.
\par 14 Mas os seus concidadãos o odiavam e enviaram após ele uma embaixada, dizendo: Não queremos que este reine sobre nós.
\par 15 Quando ele voltou, depois de haver tomado posse do reino, mandou chamar os servos a quem dera o dinheiro, a fim de saber que negócio cada um teria conseguido.
\par 16 Compareceu o primeiro e disse: Senhor, a tua mina rendeu dez.
\par 17 Respondeu-lhe o senhor: Muito bem, servo bom; porque foste fiel no pouco, terás autoridade sobre dez cidades.
\par 18 Veio o segundo, dizendo: Senhor, a tua mina rendeu cinco.
\par 19 A este disse: Terás autoridade sobre cinco cidades.
\par 20 Veio, então, outro, dizendo: Eis aqui, senhor, a tua mina, que eu guardei embrulhada num lenço.
\par 21 Pois tive medo de ti, que és homem rigoroso; tiras o que não puseste e ceifas o que não semeaste.
\par 22 Respondeu-lhe: Servo mau, por tua própria boca te condenarei. Sabias que eu sou homem rigoroso, que tiro o que não pus e ceifo o que não semeei;
\par 23 por que não puseste o meu dinheiro no banco? E, então, na minha vinda, o receberia com juros.
\par 24 E disse aos que o assistiam: Tirai-lhe a mina e dai-a ao que tem as dez.
\par 25 Eles ponderaram: Senhor, ele já tem dez.
\par 26 Pois eu vos declaro: a todo o que tem dar-se-lhe-á; mas ao que não tem, o que tem lhe será tirado.
\par 27 Quanto, porém, a esses meus inimigos, que não quiseram que eu reinasse sobre eles, trazei-os aqui e executai-os na minha presença.
\par 28 E, dito isto, prosseguia Jesus subindo para Jerusalém.
\par 29 Ora, aconteceu que, ao aproximar-se de Betfagé e de Betânia, junto ao monte das Oliveiras, enviou dois de seus discípulos,
\par 30 dizendo-lhes: Ide à aldeia fronteira e ali, ao entrardes, achareis preso um jumentinho que jamais homem algum montou; soltai-o e trazei-o.
\par 31 Se alguém vos perguntar: Por que o soltais? Respondereis assim: Porque o Senhor precisa dele.
\par 32 E, indo os que foram mandados, acharam segundo lhes dissera Jesus.
\par 33 Quando eles estavam soltando o jumentinho, seus donos lhes disseram: Por que o soltais?
\par 34 Responderam: Porque o Senhor precisa dele.
\par 35 Então, o trouxeram e, pondo as suas vestes sobre ele, ajudaram Jesus a montar.
\par 36 Indo ele, estendiam no caminho as suas vestes.
\par 37 E, quando se aproximava da descida do monte das Oliveiras, toda a multidão dos discípulos passou, jubilosa, a louvar a Deus em alta voz, por todos os milagres que tinham visto,
\par 38 dizendo: Bendito é o Rei que vem em nome do Senhor! Paz no céu e glória nas maiores alturas!
\par 39 Ora, alguns dos fariseus lhe disseram em meio à multidão: Mestre, repreende os teus discípulos!
\par 40 Mas ele lhes respondeu: Asseguro-vos que, se eles se calarem, as próprias pedras clamarão.
\par 41 Quando ia chegando, vendo a cidade, chorou
\par 42 e dizia: Ah! Se conheceras por ti mesma, ainda hoje, o que é devido à paz! Mas isto está agora oculto aos teus olhos.
\par 43 Pois sobre ti virão dias em que os teus inimigos te cercarão de trincheiras e, por todos os lados, te apertarão o cerco;
\par 44 e te arrasarão e aos teus filhos dentro de ti; não deixarão em ti pedra sobre pedra, porque não reconheceste a oportunidade da tua visitação.
\par 45 Depois, entrando no templo, expulsou os que ali vendiam,
\par 46 dizendo-lhes: Está escrito: A minha casa será casa de oração. Mas vós a transformastes em covil de salteadores.
\par 47 Diariamente, Jesus ensinava no templo; mas os principais sacerdotes, os escribas e os maiorais do povo procuravam eliminá-lo;
\par 48 contudo, não atinavam em como fazê-lo, porque todo o povo, ao ouvi-lo, ficava dominado por ele.

\chapter{20}

\par 1 Aconteceu que, num daqueles dias, estando Jesus a ensinar o povo no templo e a evangelizar, sobrevieram os principais sacerdotes e os escribas, juntamente com os anciãos,
\par 2 e o argüiram nestes termos: Dize-nos: com que autoridade fazes estas coisas? Ou quem te deu esta autoridade?
\par 3 Respondeu-lhes: Também eu vos farei uma pergunta; dizei-me:
\par 4 o batismo de João era dos céus ou dos homens?
\par 5 Então, eles arrazoavam entre si: Se dissermos: do céu, ele dirá: Por que não acreditastes nele?
\par 6 Mas, se dissermos: dos homens, o povo todo nos apedrejará; porque está convicto de ser João um profeta.
\par 7 Por fim, responderam que não sabiam.
\par 8 Então, Jesus lhes replicou: Pois nem eu vos digo com que autoridade faço estas coisas.
\par 9 A seguir, passou Jesus a proferir ao povo esta parábola: Certo homem plantou uma vinha, arrendou-a a lavradores e ausentou-se do país por prazo considerável.
\par 10 No devido tempo, mandou um servo aos lavradores para que lhe dessem do fruto da vinha; os lavradores, porém, depois de o espancarem, o despacharam vazio.
\par 11 Em vista disso, enviou-lhes outro servo; mas eles também a este espancaram e, depois de o ultrajarem, o despacharam vazio.
\par 12 Mandou ainda um terceiro; também a este, depois de o ferirem, expulsaram.
\par 13 Então, disse o dono da vinha: Que farei? Enviarei o meu filho amado; talvez o respeitem.
\par 14 Vendo-o, porém, os lavradores, arrazoavam entre si, dizendo: Este é o herdeiro; matemo-lo, para que a herança venha a ser nossa.
\par 15 E, lançando-o fora da vinha, o mataram. Que lhes fará, pois, o dono da vinha?
\par 16 Virá, exterminará aqueles lavradores e passará a vinha a outros. Ao ouvirem isto, disseram: Tal não aconteça!
\par 17 Mas Jesus, fitando-os, disse: Que quer dizer, pois, o que está escrito: A pedra que os construtores rejeitaram, esta veio a ser a principal pedra, angular?
\par 18 Todo o que cair sobre esta pedra ficará em pedaços; e aquele sobre quem ela cair ficará reduzido a pó.
\par 19 Naquela mesma hora, os escribas e os principais sacerdotes procuravam lançar-lhe as mãos, pois perceberam que, em referência a eles, dissera esta parábola; mas temiam o povo.
\par 20 Observando-o, subornaram emissários que se fingiam de justos para verem se o apanhavam em alguma palavra, a fim de entregá-lo à jurisdição e à autoridade do governador.
\par 21 Então, o consultaram, dizendo: Mestre, sabemos que falas e ensinas retamente e não te deixas levar de respeitos humanos, porém ensinas o caminho de Deus segundo a verdade;
\par 22 é lícito pagar tributo a César ou não?
\par 23 Mas Jesus, percebendo-lhes o ardil, respondeu:
\par 24 Mostrai-me um denário. De quem é a efígie e a inscrição? Prontamente disseram: De César. Então, lhes recomendou Jesus:
\par 25 Dai, pois, a César o que é de César e a Deus o que é de Deus.
\par 26 Não puderam apanhá-lo em palavra alguma diante do povo; e, admirados da sua resposta, calaram-se.
\par 27 Chegando alguns dos saduceus, homens que dizem não haver ressurreição,
\par 28 perguntaram-lhe: Mestre, Moisés nos deixou escrito que, se morrer o irmão de alguém, sendo aquele casado e não deixando filhos, seu irmão deve casar com a viúva e suscitar descendência ao falecido.
\par 29 Ora, havia sete irmãos: o primeiro casou e morreu sem filhos;
\par 30 o segundo e o terceiro também desposaram a viúva;
\par 31 igualmente os sete não tiveram filhos e morreram.
\par 32 Por fim, morreu também a mulher.
\par 33 Esta mulher, pois, no dia da ressurreição, de qual deles será esposa? Porque os sete a desposaram.
\par 34 Então, lhes acrescentou Jesus: Os filhos deste mundo casam-se e dão-se em casamento;
\par 35 mas os que são havidos por dignos de alcançar a era vindoura e a ressurreição dentre os mortos não casam, nem se dão em casamento.
\par 36 Pois não podem mais morrer, porque são iguais aos anjos e são filhos de Deus, sendo filhos da ressurreição.
\par 37 E que os mortos hão de ressuscitar, Moisés o indicou no trecho referente à sarça, quando chama ao Senhor o Deus de Abraão, o Deus de Isaque e o Deus de Jacó.
\par 38 Ora, Deus não é Deus de mortos, e sim de vivos; porque para ele todos vivem.
\par 39 Então, disseram alguns dos escribas: Mestre, respondeste bem!
\par 40 Dali por diante, não ousaram mais interrogá-lo.
\par 41 Mas Jesus lhes perguntou: Como podem dizer que o Cristo é filho de Davi?
\par 42 Visto como o próprio Davi afirma no livro dos Salmos: Disse o Senhor ao meu Senhor: Assenta-te à minha direita,
\par 43 até que eu ponha os teus inimigos por estrado dos teus pés.
\par 44 Assim, pois, Davi lhe chama Senhor, e como pode ser ele seu filho?
\par 45 Ouvindo-o todo o povo, recomendou Jesus a seus discípulos:
\par 46 Guardai-vos dos escribas, que gostam de andar com vestes talares e muito apreciam as saudações nas praças, as primeiras cadeiras nas sinagogas e os primeiros lugares nos banquetes;
\par 47 os quais devoram as casas das viúvas e, para o justificar, fazem longas orações; estes sofrerão juízo muito mais severo.

\chapter{21}

\par 1 Estando Jesus a observar, viu os ricos lançarem suas ofertas no gazofilácio.
\par 2 Viu também certa viúva pobre lançar ali duas pequenas moedas;
\par 3 e disse: Verdadeiramente, vos digo que esta viúva pobre deu mais do que todos.
\par 4 Porque todos estes deram como oferta daquilo que lhes sobrava; esta, porém, da sua pobreza deu tudo o que possuía, todo o seu sustento.
\par 5 Falavam alguns a respeito do templo, como estava ornado de belas pedras e de dádivas;
\par 6 então, disse Jesus: Vedes estas coisas? Dias virão em que não ficará pedra sobre pedra que não seja derribada.
\par 7 Perguntaram-lhe: Mestre, quando sucederá isto? E que sinal haverá de quando estas coisas estiverem para se cumprir?
\par 8 Respondeu ele: Vede que não sejais enganados; porque muitos virão em meu nome, dizendo: Sou eu! E também: Chegou a hora! Não os sigais.
\par 9 Quando ouvirdes falar de guerras e revoluções, não vos assusteis; pois é necessário que primeiro aconteçam estas coisas, mas o fim não será logo.
\par 10 Então, lhes disse: Levantar-se-á nação contra nação, e reino, contra reino;
\par 11 haverá grandes terremotos, epidemias e fome em vários lugares, coisas espantosas e também grandes sinais do céu.
\par 12 Antes, porém, de todas estas coisas, lançarão mão de vós e vos perseguirão, entregando-vos às sinagogas e aos cárceres, levando-vos à presença de reis e governadores, por causa do meu nome;
\par 13 e isto vos acontecerá para que deis testemunho.
\par 14 Assentai, pois, em vosso coração de não vos preocupardes com o que haveis de responder;
\par 15 porque eu vos darei boca e sabedoria a que não poderão resistir, nem contradizer todos quantos se vos opuserem.
\par 16 E sereis entregues até por vossos pais, irmãos, parentes e amigos; e matarão alguns dentre vós.
\par 17 De todos sereis odiados por causa do meu nome.
\par 18 Contudo, não se perderá um só fio de cabelo da vossa cabeça.
\par 19 É na vossa perseverança que ganhareis a vossa alma.
\par 20 Quando, porém, virdes Jerusalém sitiada de exércitos, sabei que está próxima a sua devastação.
\par 21 Então, os que estiverem na Judéia, fujam para os montes; os que se encontrarem dentro da cidade, retirem-se; e os que estiverem nos campos, não entrem nela.
\par 22 Porque estes dias são de vingança, para se cumprir tudo o que está escrito.
\par 23 Ai das que estiverem grávidas e das que amamentarem naqueles dias! Porque haverá grande aflição na terra e ira contra este povo.
\par 24 Cairão a fio de espada e serão levados cativos para todas as nações; e, até que os tempos dos gentios se completem, Jerusalém será pisada por eles.
\par 25 Haverá sinais no sol, na lua e nas estrelas; sobre a terra, angústia entre as nações em perplexidade por causa do bramido do mar e das ondas;
\par 26 haverá homens que desmaiarão de terror e pela expectativa das coisas que sobrevirão ao mundo; pois os poderes dos céus serão abalados.
\par 27 Então, se verá o Filho do Homem vindo numa nuvem, com poder e grande glória.
\par 28 Ora, ao começarem estas coisas a suceder, exultai e erguei a vossa cabeça; porque a vossa redenção se aproxima.
\par 29 Ainda lhes propôs uma parábola, dizendo: Vede a figueira e todas as árvores.
\par 30 Quando começam a brotar, vendo-o, sabeis, por vós mesmos, que o verão está próximo.
\par 31 Assim também, quando virdes acontecerem estas coisas, sabei que está próximo o reino de Deus.
\par 32 Em verdade vos digo que não passará esta geração, sem que tudo isto aconteça.
\par 33 Passará o céu e a terra, porém as minhas palavras não passarão.
\par 34 Acautelai-vos por vós mesmos, para que nunca vos suceda que o vosso coração fique sobrecarregado com as conseqüências da orgia, da embriaguez e das preocupações deste mundo, e para que aquele dia não venha sobre vós repentinamente, como um laço.
\par 35 Pois há de sobrevir a todos os que vivem sobre a face de toda a terra.
\par 36 Vigiai, pois, a todo tempo, orando, para que possais escapar de todas estas coisas que têm de suceder e estar em pé na presença do Filho do Homem.
\par 37 Jesus ensinava todos os dias no templo, mas à noite, saindo, ia pousar no monte chamado das Oliveiras.
\par 38 E todo o povo madrugava para ir ter com ele no templo, a fim de ouvi-lo.

\chapter{22}

\par 1 Estava próxima a Festa dos Pães Asmos, chamada Páscoa.
\par 2 Preocupavam-se os principais sacerdotes e os escribas em como tirar a vida a Jesus; porque temiam o povo.
\par 3 Ora, Satanás entrou em Judas, chamado Iscariotes, que era um dos doze.
\par 4 Este foi entender-se com os principais sacerdotes e os capitães sobre como lhes entregaria a Jesus;
\par 5 então, eles se alegraram e combinaram em lhe dar dinheiro.
\par 6 Judas concordou e buscava uma boa ocasião de lho entregar sem tumulto.
\par 7 Chegou o dia da Festa dos Pães Asmos, em que importava comemorar a Páscoa.
\par 8 Jesus, pois, enviou Pedro e João, dizendo: Ide preparar-nos a Páscoa para que a comamos.
\par 9 Eles lhe perguntaram: Onde queres que a preparemos?
\par 10 Então, lhes explicou Jesus: Ao entrardes na cidade, encontrareis um homem com um cântaro de água; segui-o até à casa em que ele entrar
\par 11 e dizei ao dono da casa: O Mestre manda perguntar-te: Onde é o aposento no qual hei de comer a Páscoa com os meus discípulos?
\par 12 Ele vos mostrará um espaçoso cenáculo mobilado; ali fazei os preparativos.
\par 13 E, indo, tudo encontraram como Jesus lhes dissera e prepararam a Páscoa.
\par 14 Chegada a hora, pôs-se Jesus à mesa, e com ele os apóstolos.
\par 15 E disse-lhes: Tenho desejado ansiosamente comer convosco esta Páscoa, antes do meu sofrimento.
\par 16 Pois vos digo que nunca mais a comerei, até que ela se cumpra no reino de Deus.
\par 17 E, tomando um cálice, havendo dado graças, disse: Recebei e reparti entre vós;
\par 18 pois vos digo que, de agora em diante, não mais beberei do fruto da videira, até que venha o reino de Deus.
\par 19 E, tomando um pão, tendo dado graças, o partiu e lhes deu, dizendo: Isto é o meu corpo oferecido por vós; fazei isto em memória de mim.
\par 20 Semelhantemente, depois de cear, tomou o cálice, dizendo: Este é o cálice da nova aliança no meu sangue derramado em favor de vós.
\par 21 Todavia, a mão do traidor está comigo à mesa.
\par 22 Porque o Filho do Homem, na verdade, vai segundo o que está determinado, mas ai daquele por intermédio de quem ele está sendo traído!
\par 23 Então, começaram a indagar entre si quem seria, dentre eles, o que estava para fazer isto.
\par 24 Suscitaram também entre si uma discussão sobre qual deles parecia ser o maior.
\par 25 Mas Jesus lhes disse: Os reis dos povos dominam sobre eles, e os que exercem autoridade são chamados benfeitores.
\par 26 Mas vós não sois assim; pelo contrário, o maior entre vós seja como o menor; e aquele que dirige seja como o que serve.
\par 27 Pois qual é maior: quem está à mesa ou quem serve? Porventura, não é quem está à mesa? Pois, no meio de vós, eu sou como quem serve.
\par 28 Vós sois os que tendes permanecido comigo nas minhas tentações.
\par 29 Assim como meu Pai me confiou um reino, eu vo-lo confio,
\par 30 para que comais e bebais à minha mesa no meu reino; e vos assentareis em tronos para julgar as doze tribos de Israel.
\par 31 Simão, Simão, eis que Satanás vos reclamou para vos peneirar como trigo!
\par 32 Eu, porém, roguei por ti, para que a tua fé não desfaleça; tu, pois, quando te converteres, fortalece os teus irmãos.
\par 33 Ele, porém, respondeu: Senhor, estou pronto a ir contigo, tanto para a prisão como para a morte.
\par 34 Mas Jesus lhe disse: Afirmo-te, Pedro, que, hoje, três vezes negarás que me conheces, antes que o galo cante.
\par 35 A seguir, Jesus lhes perguntou: Quando vos mandei sem bolsa, sem alforje e sem sandálias, faltou-vos, porventura, alguma coisa? Nada, disseram eles.
\par 36 Então, lhes disse: Agora, porém, quem tem bolsa, tome-a, como também o alforje; e o que não tem espada, venda a sua capa e compre uma.
\par 37 Pois vos digo que importa que se cumpra em mim o que está escrito: Ele foi contado com os malfeitores. Porque o que a mim se refere está sendo cumprido.
\par 38 Então, lhe disseram: Senhor, eis aqui duas espadas! Respondeu-lhes: Basta!
\par 39 E, saindo, foi, como de costume, para o monte das Oliveiras; e os discípulos o acompanharam.
\par 40 Chegando ao lugar escolhido, Jesus lhes disse: Orai, para que não entreis em tentação.
\par 41 Ele, por sua vez, se afastou, cerca de um tiro de pedra, e, de joelhos, orava,
\par 42 dizendo: Pai, se queres, passa de mim este cálice; contudo, não se faça a minha vontade, e sim a tua.
\par 43 [Então, lhe apareceu um anjo do céu que o confortava.
\par 44 E, estando em agonia, orava mais intensamente. E aconteceu que o seu suor se tornou como gotas de sangue caindo sobre a terra.]
\par 45 Levantando-se da oração, foi ter com os discípulos, e os achou dormindo de tristeza,
\par 46 e disse-lhes: Por que estais dormindo? Levantai-vos e orai, para que não entreis em tentação.
\par 47 Falava ele ainda, quando chegou uma multidão; e um dos doze, o chamado Judas, que vinha à frente deles, aproximou-se de Jesus para o beijar.
\par 48 Jesus, porém, lhe disse: Judas, com um beijo trais o Filho do Homem?
\par 49 Os que estavam ao redor dele, vendo o que ia suceder, perguntaram: Senhor, feriremos à espada?
\par 50 Um deles feriu o servo do sumo sacerdote e cortou-lhe a orelha direita.
\par 51 Mas Jesus acudiu, dizendo: Deixai, basta. E, tocando-lhe a orelha, o curou.
\par 52 Então, dirigindo-se Jesus aos principais sacerdotes, capitães do templo e anciãos que vieram prendê-lo, disse: Saístes com espadas e porretes como para deter um salteador?
\par 53 Diariamente, estando eu convosco no templo, não pusestes as mãos sobre mim. Esta, porém, é a vossa hora e o poder das trevas.
\par 54 Então, prendendo-o, o levaram e o introduziram na casa do sumo sacerdote. Pedro seguia de longe.
\par 55 E, quando acenderam fogo no meio do pátio e juntos se assentaram, Pedro tomou lugar entre eles.
\par 56 Entrementes, uma criada, vendo-o assentado perto do fogo, fitando-o, disse: Este também estava com ele.
\par 57 Mas Pedro negava, dizendo: Mulher, não o conheço.
\par 58 Pouco depois, vendo-o outro, disse: Também tu és dos tais. Pedro, porém, protestava: Homem, não sou.
\par 59 E, tendo passado cerca de uma hora, outro afirmava, dizendo: Também este, verdadeiramente, estava com ele, porque também é galileu.
\par 60 Mas Pedro insistia: Homem, não compreendo o que dizes. E logo, estando ele ainda a falar, cantou o galo.
\par 61 Então, voltando-se o Senhor, fixou os olhos em Pedro, e Pedro se lembrou da palavra do Senhor, como lhe dissera: Hoje, três vezes me negarás, antes de cantar o galo.
\par 62 Então, Pedro, saindo dali, chorou amargamente.
\par 63 Os que detinham Jesus zombavam dele, davam-lhe pancadas e,
\par 64 vendando-lhe os olhos, diziam: Profetiza-nos: quem é que te bateu?
\par 65 E muitas outras coisas diziam contra ele, blasfemando.
\par 66 Logo que amanheceu, reuniu-se a assembléia dos anciãos do povo, tanto os principais sacerdotes como os escribas, e o conduziram ao Sinédrio, onde lhe disseram:
\par 67 Se tu és o Cristo, dize-nos. Então, Jesus lhes respondeu: Se vo-lo disser, não o acreditareis;
\par 68 também, se vos perguntar, de nenhum modo me respondereis.
\par 69 Desde agora, estará sentado o Filho do Homem à direita do Todo-Poderoso Deus.
\par 70 Então, disseram todos: Logo, tu és o Filho de Deus? E ele lhes respondeu: Vós dizeis que eu sou.
\par 71 Clamaram, pois: Que necessidade mais temos de testemunho? Porque nós mesmos o ouvimos da sua própria boca.

\chapter{23}

\par 1 Levantando-se toda a assembléia, levaram Jesus a Pilatos.
\par 2 E ali passaram a acusá-lo, dizendo: Encontramos este homem pervertendo a nossa nação, vedando pagar tributo a César e afirmando ser ele o Cristo, o Rei.
\par 3 Então, lhe perguntou Pilatos: És tu o rei dos judeus? Respondeu Jesus: Tu o dizes.
\par 4 Disse Pilatos aos principais sacerdotes e às multidões: Não vejo neste homem crime algum.
\par 5 Insistiam, porém, cada vez mais, dizendo: Ele alvoroça o povo, ensinando por toda a Judéia, desde a Galiléia, onde começou, até aqui.
\par 6 Tendo Pilatos ouvido isto, perguntou se aquele homem era galileu.
\par 7 Ao saber que era da jurisdição de Herodes, estando este, naqueles dias, em Jerusalém, lho remeteu.
\par 8 Herodes, vendo a Jesus, sobremaneira se alegrou, pois havia muito queria vê-lo, por ter ouvido falar a seu respeito; esperava também vê-lo fazer algum sinal.
\par 9 E de muitos modos o interrogava; Jesus, porém, nada lhe respondia.
\par 10 Os principais sacerdotes e os escribas ali presentes o acusavam com grande veemência.
\par 11 Mas Herodes, juntamente com os da sua guarda, tratou-o com desprezo, e, escarnecendo dele, fê-lo vestir-se de um manto aparatoso, e o devolveu a Pilatos.
\par 12 Naquele mesmo dia, Herodes e Pilatos se reconciliaram, pois, antes, viviam inimizados um com o outro.
\par 13 Então, reunindo Pilatos os principais sacerdotes, as autoridades e o povo,
\par 14 disse-lhes: Apresentastes-me este homem como agitador do povo; mas, tendo-o interrogado na vossa presença, nada verifiquei contra ele dos crimes de que o acusais.
\par 15 Nem tampouco Herodes, pois no-lo tornou a enviar. É, pois, claro que nada contra ele se verificou digno de morte.
\par 16 Portanto, após castigá-lo, soltá-lo-ei.
\par 17 [E era-lhe forçoso soltar-lhes um detento por ocasião da festa.]
\par 18 Toda a multidão, porém, gritava: Fora com este! Solta-nos Barrabás!
\par 19 Barrabás estava no cárcere por causa de uma sedição na cidade e também por homicídio.
\par 20 Desejando Pilatos soltar a Jesus, insistiu ainda.
\par 21 Eles, porém, mais gritavam: Crucifica-o! Crucifica-o!
\par 22 Então, pela terceira vez, lhes perguntou: Que mal fez este? De fato, nada achei contra ele para condená-lo à morte; portanto, depois de o castigar, soltá-lo-ei.
\par 23 Mas eles instavam com grandes gritos, pedindo que fosse crucificado. E o seu clamor prevaleceu.
\par 24 Então, Pilatos decidiu atender-lhes o pedido.
\par 25 Soltou aquele que estava encarcerado por causa da sedição e do homicídio, a quem eles pediam; e, quanto a Jesus, entregou-o à vontade deles.
\par 26 E, como o conduzissem, constrangendo um cireneu, chamado Simão, que vinha do campo, puseram-lhe a cruz sobre os ombros, para que a levasse após Jesus.
\par 27 Seguia-o numerosa multidão de povo, e também mulheres que batiam no peito e o lamentavam.
\par 28 Porém Jesus, voltando-se para elas, disse: Filhas de Jerusalém, não choreis por mim; chorai, antes, por vós mesmas e por vossos filhos!
\par 29 Porque dias virão em que se dirá: Bem-aventuradas as estéreis, que não geraram, nem amamentaram.
\par 30 Nesses dias, dirão aos montes: Caí sobre nós! E aos outeiros: Cobri-nos!
\par 31 Porque, se em lenho verde fazem isto, que será no lenho seco?
\par 32 E também eram levados outros dois, que eram malfeitores, para serem executados com ele.
\par 33 Quando chegaram ao lugar chamado Calvário, ali o crucificaram, bem como aos malfeitores, um à direita, outro à esquerda.
\par 34 Contudo, Jesus dizia: Pai, perdoa-lhes, porque não sabem o que fazem. Então, repartindo as vestes dele, lançaram sortes.
\par 35 O povo estava ali e a tudo observava. Também as autoridades zombavam e diziam: Salvou os outros; a si mesmo se salve, se é, de fato, o Cristo de Deus, o escolhido.
\par 36 Igualmente os soldados o escarneciam e, aproximando-se, trouxeram-lhe vinagre, dizendo:
\par 37 Se tu és o rei dos judeus, salva-te a ti mesmo.
\par 38 Também sobre ele estava esta epígrafe [em letras gregas, romanas e hebraicas]: ESTE É O REI DOS JUDEUS.
\par 39 Um dos malfeitores crucificados blasfemava contra ele, dizendo: Não és tu o Cristo? Salva-te a ti mesmo e a nós também.
\par 40 Respondendo-lhe, porém, o outro, repreendeu-o, dizendo: Nem ao menos temes a Deus, estando sob igual sentença?
\par 41 Nós, na verdade, com justiça, porque recebemos o castigo que os nossos atos merecem; mas este nenhum mal fez.
\par 42 E acrescentou: Jesus, lembra-te de mim quando vieres no teu reino.
\par 43 Jesus lhe respondeu: Em verdade te digo que hoje estarás comigo no paraíso.
\par 44 Já era quase a hora sexta, e, escurecendo-se o sol, houve trevas sobre toda a terra até à hora nona.
\par 45 E rasgou-se pelo meio o véu do santuário.
\par 46 Então, Jesus clamou em alta voz: Pai, nas tuas mãos entrego o meu espírito! E, dito isto, expirou.
\par 47 Vendo o centurião o que tinha acontecido, deu glória a Deus, dizendo: Verdadeiramente, este homem era justo.
\par 48 E todas as multidões reunidas para este espetáculo, vendo o que havia acontecido, retiraram-se a lamentar, batendo nos peitos.
\par 49 Entretanto, todos os conhecidos de Jesus e as mulheres que o tinham seguido desde a Galiléia permaneceram a contemplar de longe estas coisas.
\par 50 E eis que certo homem, chamado José, membro do Sinédrio, homem bom e justo
\par 51 (que não tinha concordado com o desígnio e ação dos outros), natural de Arimatéia, cidade dos judeus, e que esperava o reino de Deus,
\par 52 tendo procurado a Pilatos, pediu-lhe o corpo de Jesus,
\par 53 e, tirando-o do madeiro, envolveu-o num lençol de linho, e o depositou num túmulo aberto em rocha, onde ainda ninguém havia sido sepultado.
\par 54 Era o dia da preparação, e começava o sábado.
\par 55 As mulheres que tinham vindo da Galiléia com Jesus, seguindo, viram o túmulo e como o corpo fora ali depositado.
\par 56 Então, se retiraram para preparar aromas e bálsamos. E, no sábado, descansaram, segundo o mandamento.

\chapter{24}

\par 1 Mas, no primeiro dia da semana, alta madrugada, foram elas ao túmulo, levando os aromas que haviam preparado.
\par 2 E encontraram a pedra removida do sepulcro;
\par 3 mas, ao entrarem, não acharam o corpo do Senhor Jesus.
\par 4 Aconteceu que, perplexas a esse respeito, apareceram-lhes dois varões com vestes resplandecentes.
\par 5 Estando elas possuídas de temor, baixando os olhos para o chão, eles lhes falaram: Por que buscais entre os mortos ao que vive?
\par 6 Ele não está aqui, mas ressuscitou. Lembrai-vos de como vos preveniu, estando ainda na Galiléia,
\par 7 quando disse: Importa que o Filho do Homem seja entregue nas mãos de pecadores, e seja crucificado, e ressuscite no terceiro dia.
\par 8 Então, se lembraram das suas palavras.
\par 9 E, voltando do túmulo, anunciaram todas estas coisas aos onze e a todos os mais que com eles estavam.
\par 10 Eram Maria Madalena, Joana e Maria, mãe de Tiago; também as demais que estavam com elas confirmaram estas coisas aos apóstolos.
\par 11 Tais palavras lhes pareciam um como delírio, e não acreditaram nelas.
\par 12 Pedro, porém, levantando-se, correu ao sepulcro. E, abaixando-se, nada mais viu, senão os lençóis de linho; e retirou-se para casa, maravilhado do que havia acontecido.
\par 13 Naquele mesmo dia, dois deles estavam de caminho para uma aldeia chamada Emaús, distante de Jerusalém sessenta estádios.
\par 14 E iam conversando a respeito de todas as coisas sucedidas.
\par 15 Aconteceu que, enquanto conversavam e discutiam, o próprio Jesus se aproximou e ia com eles.
\par 16 Os seus olhos, porém, estavam como que impedidos de o reconhecer.
\par 17 Então, lhes perguntou Jesus: Que é isso que vos preocupa e de que ides tratando à medida que caminhais? E eles pararam entristecidos.
\par 18 Um, porém, chamado Cleopas, respondeu, dizendo: És o único, porventura, que, tendo estado em Jerusalém, ignoras as ocorrências destes últimos dias?
\par 19 Ele lhes perguntou: Quais? E explicaram: O que aconteceu a Jesus, o Nazareno, que era varão profeta, poderoso em obras e palavras, diante de Deus e de todo o povo,
\par 20 e como os principais sacerdotes e as nossas autoridades o entregaram para ser condenado à morte e o crucificaram.
\par 21 Ora, nós esperávamos que fosse ele quem havia de redimir a Israel; mas, depois de tudo isto, é já este o terceiro dia desde que tais coisas sucederam.
\par 22 É verdade também que algumas mulheres, das que conosco estavam, nos surpreenderam, tendo ido de madrugada ao túmulo;
\par 23 e, não achando o corpo de Jesus, voltaram dizendo terem tido uma visão de anjos, os quais afirmam que ele vive.
\par 24 De fato, alguns dos nossos foram ao sepulcro e verificaram a exatidão do que disseram as mulheres; mas não o viram.
\par 25 Então, lhes disse Jesus: Ó néscios e tardos de coração para crer tudo o que os profetas disseram!
\par 26 Porventura, não convinha que o Cristo padecesse e entrasse na sua glória?
\par 27 E, começando por Moisés, discorrendo por todos os Profetas, expunha-lhes o que a seu respeito constava em todas as Escrituras.
\par 28 Quando se aproximavam da aldeia para onde iam, fez ele menção de passar adiante.
\par 29 Mas eles o constrangeram, dizendo: Fica conosco, porque é tarde, e o dia já declina. E entrou para ficar com eles.
\par 30 E aconteceu que, quando estavam à mesa, tomando ele o pão, abençoou-o e, tendo-o partido, lhes deu;
\par 31 então, se lhes abriram os olhos, e o reconheceram; mas ele desapareceu da presença deles.
\par 32 E disseram um ao outro: Porventura, não nos ardia o coração, quando ele, pelo caminho, nos falava, quando nos expunha as Escrituras?
\par 33 E, na mesma hora, levantando-se, voltaram para Jerusalém, onde acharam reunidos os onze e outros com eles,
\par 34 os quais diziam: O Senhor ressuscitou e já apareceu a Simão!
\par 35 Então, os dois contaram o que lhes acontecera no caminho e como fora por eles reconhecido no partir do pão.
\par 36 Falavam ainda estas coisas quando Jesus apareceu no meio deles e lhes disse: Paz seja convosco!
\par 37 Eles, porém, surpresos e atemorizados, acreditavam estarem vendo um espírito.
\par 38 Mas ele lhes disse: Por que estais perturbados? E por que sobem dúvidas ao vosso coração?
\par 39 Vede as minhas mãos e os meus pés, que sou eu mesmo; apalpai-me e verificai, porque um espírito não tem carne nem ossos, como vedes que eu tenho.
\par 40 Dizendo isto, mostrou-lhes as mãos e os pés.
\par 41 E, por não acreditarem eles ainda, por causa da alegria, e estando admirados, Jesus lhes disse: Tendes aqui alguma coisa que comer?
\par 42 Então, lhe apresentaram um pedaço de peixe assado [e um favo de mel].
\par 43 E ele comeu na presença deles.
\par 44 A seguir, Jesus lhes disse: São estas as palavras que eu vos falei, estando ainda convosco: importava se cumprisse tudo o que de mim está escrito na Lei de Moisés, nos Profetas e nos Salmos.
\par 45 Então, lhes abriu o entendimento para compreenderem as Escrituras;
\par 46 e lhes disse: Assim está escrito que o Cristo havia de padecer e ressuscitar dentre os mortos no terceiro dia
\par 47 e que em seu nome se pregasse arrependimento para remissão de pecados a todas as nações, começando de Jerusalém.
\par 48 Vós sois testemunhas destas coisas.
\par 49 Eis que envio sobre vós a promessa de meu Pai; permanecei, pois, na cidade, até que do alto sejais revestidos de poder.
\par 50 Então, os levou para Betânia e, erguendo as mãos, os abençoou.
\par 51 Aconteceu que, enquanto os abençoava, ia-se retirando deles, sendo elevado para o céu.
\par 52 Então, eles, adorando-o, voltaram para Jerusalém, tomados de grande júbilo;
\par 53 e estavam sempre no templo, louvando a Deus.


\end{document}