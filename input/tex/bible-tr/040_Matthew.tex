\begin{document}

\title{Matta}


\chapter{1}

\par 1 Ibrahim oglu, Davut oglu Isa Mesih'in soy kaydi söyledir: Ibrahim Ishak'in babasiydi, Ishak Yakup'un babasiydi, Yakup Yahuda ve kardeslerinin babasiydi,
\par 3 Yahuda, Tamar'dan dogan Peres'le Zerah'in babasiydi, Peres Hesron'un babasiydi, Hesron Ram'in babasiydi,
\par 4 Ram Amminadav'in babasiydi, Amminadav Nahson'un babasiydi, Nahson Salmon'un babasiydi,
\par 5 Salmon, Rahav'dan dogan Boaz'in babasiydi, Boaz, Rut'tan dogan Ovet'in babasiydi, Ovet Isay'in babasiydi,
\par 6 Isay Kral Davut'un babasiydi, Davut, Uriya'nin karisindan dogan Süleyman'in babasiydi,
\par 7 Süleyman Rehavam'in babasiydi, Rehavam Aviya'nin babasiydi, Aviya Asa'nin babasiydi,
\par 8 Asa Yehosafat'in babasiydi, Yehosafat Yoram'in babasiydi, Yoram Uzziya'nin babasiydi,
\par 9 Uzziya Yotam'in babasiydi, Yotam Ahaz'in babasiydi, Ahaz Hizkiya'nin babasiydi,
\par 10 Hizkiya Manasse'nin babasiydi, Manasse Amon'un babasiydi, Amon Yosiya'nin babasiydi,
\par 11 Yosiya, Babil sürgünü* sirasinda dogan Yehoyakin'le kardeslerinin babasiydi,
\par 12 Yehoyakin, Babil sürgününden sonra dogan Sealtiel'in babasiydi, Sealtiel Zerubbabil'in babasiydi,
\par 13 Zerubbabil Avihut'un babasiydi, Avihut Elyakim'in babasiydi, Elyakim Azor'un babasiydi,
\par 14 Azor Sadok'un babasiydi, Sadok Ahim'in babasiydi, Ahim Elihut'un babasiydi,
\par 15 Elihut Elazar'in babasiydi, Elazar Mattan'in babasiydi, Mattan Yakup'un babasiydi,
\par 16 Yakup Meryem'in kocasi Yusuf'un babasiydi. Meryem'den Mesih* diye taninan Isa dogdu.
\par 17 Buna göre, Ibrahim'den Davut'a kadar toplam on dört kusak, Davut'tan Babil sürgününe kadar on dört kusak, Babil sürgününden Mesih'e kadar on dört kusak vardir.
\par 18 Isa Mesih'in dogumu söyle oldu: Annesi Meryem, Yusuf'la nisanliydi. Ama birlikte olmalarindan önce Meryem'in Kutsal Ruh'tan gebe oldugu anlasildi.
\par 19 Nisanlisi Yusuf, dogru bir adam oldugu ve onu herkesin önünde utandirmak istemedigi için ondan sessizce ayrilmak niyetindeydi.
\par 20 Ama böyle düsünmesi üzerine Rab'bin bir melegi rüyada ona görünerek söyle dedi: "Davut oglu Yusuf, Meryem'i kendine es olarak almaktan korkma. Çünkü onun rahminde olusan, Kutsal Ruh'tandir.
\par 21 Meryem bir ogul doguracak. Adini Isa koyacaksin. Çünkü halkini günahlarindan O kurtaracak."
\par 22 Bütün bunlar, Rab'bin peygamber araciligiyla bildirdigi su söz yerine gelsin diye oldu:
\par 23 "Iste, kiz gebe kalip bir ogul doguracak; adini Immanuel koyacaklar." Immanuel, Tanri bizimle demektir.
\par 24 Yusuf uyaninca Rab'bin meleginin buyruguna uydu ve Meryem'i es olarak yanina aldi.
\par 25 Ama oglunu doguruncaya dek Yusuf ona dokunmadi. Dogan çocugun adini Isa koydu.

\chapter{2}

\par 1 Isa'nin Kral Hirodes* devrinde Yahudiye'nin Beytlehem Kenti'nde dogmasindan sonra bazi yildizbilimciler dogudan Yerusalim'e* gelip söyle dediler: "Yahudiler'in Krali olarak dogan çocuk nerede? Doguda O'nun yildizini gördük ve O'na tapinmaya geldik."
\par 3 Kral Hirodes bunu duyunca kendisi de bütün Yerusalim halki da tedirgin oldu.
\par 4 Bütün baskâhinleri ve halkin din bilginlerini* toplayarak onlara Mesih'in nerede dogacagini sordu.
\par 5 "Yahudiye'nin Beytlehem Kenti'nde" dediler. "Çünkü peygamber araciligiyla söyle yazilmistir:
\par 6 'Ey sen, Yahuda'daki Beytlehem, Yahuda önderleri arasinda hiç de en önemsizi degilsin! Çünkü halkim Israil'i güdecek önder Senden çikacak.'"
\par 7 Bunun üzerine Hirodes yildizbilimcileri gizlice çagirip onlardan yildizin göründügü ani tam olarak ögrendi.
\par 8 "Gidin, çocugu dikkatle arayin, bulunca bana haber verin, ben de gelip O'na tapinayim" diyerek onlari Beytlehem'e gönderdi.
\par 9 Yildizbilimciler, krali dinledikten sonra yola çiktilar. Doguda görmüs olduklari yildiz onlara yol gösteriyordu, çocugun bulundugu yerin üzerine varinca durdu.
\par 10 Yildizi gördüklerinde olaganüstü bir sevinç duydular.
\par 11 Eve girip çocugu annesi Meryem'le birlikte görünce yere kapanarak O'na tapindilar. Hazinelerini açip O'na armagan olarak altin, günnük ve mür* sundular.
\par 12 Sonra gördükleri bir düste Hirodes'in yanina dönmemeleri için uyarilinca ülkelerine baska yoldan döndüler.
\par 13 Yildizbilimciler gittikten sonra Rab'bin bir melegi Yusuf'a rüyada görünerek, "Kalk!" dedi, "Çocukla annesini al, Misir'a kaç. Ben sana haber verinceye dek orada kal. Çünkü Hirodes öldürmek için çocugu aratacak."
\par 14 Böylece Yusuf kalkti, ayni gece çocukla annesini alip Misir'a dogru yola çikti.
\par 15 Hirodes'in ölümüne dek orada kaldi. Bu, Rab'bin peygamber araciligiyla bildirdigi su söz yerine gelsin diye oldu: "Oglumu Misir'dan çagirdim."
\par 16 Hirodes, yildizbilimciler tarafindan aldatildigini anlayinca çok öfkelendi. Onlardan ögrendigi vakti göz önüne alarak Beytlehem ve bütün yöresinde bulunan iki ve iki yasindan küçük erkek çocuklarin hepsini öldürttü.
\par 17 Böylelikle Peygamber Yeremya araciligiyla bildirilen su söz yerine gelmis oldu:
\par 18 "Rama'da bir ses duyuldu, Aglayis ve aci feryat sesleri! Çocuklari için aglayan Rahel Avutulmak istemiyor. Çünkü onlar yok artik!"
\par 19 Hirodes öldükten sonra, Rab'bin bir melegi Misir'da Yusuf'a rüyada görünerek, "Kalk!" dedi, "Çocukla annesini al, Israil'e dön. Çünkü çocugun canina kiymak isteyenler öldü."
\par 21 Bunun üzerine Yusuf kalkti, çocukla annesini alip Israil'e döndü.
\par 22 Ama Yahudiye'de Hirodes'in yerine oglu Arhelas'in kral oldugunu duyunca oraya gitmekten korktu. Rüyada uyarilinca Celile bölgesine gitti.
\par 23 Oraya varinca Nasira denen kente yerlesti. Bu, peygamberler araciligiyla bildirilen, "O'na Nasirali denecektir" sözü yerine gelsin diye oldu.

\chapter{3}

\par 1 O günlerde Vaftizci Yahya Yahudiye Çölü'nde ortaya çikti. Su çagriyi yapiyordu: "Tövbe edin! Göklerin Egemenligi yaklasmistir."
\par 3 Nitekim Peygamber Yesaya araciligiyla sözü edilen kisi Yahya'dir. Yesaya söyle demisti: "Çölde haykiran, 'Rab'bin yolunu hazirlayin, Geçecegi patikalari düzleyin' diye sesleniyor."
\par 4 Yahya'nin deve tüyünden giysisi, belinde deri kusagi vardi. Yedigi, çekirge ve yaban baliydi.
\par 5 Yerusalim, bütün Yahudiye ve Seria yöresinin halki ona geliyor, günahlarini itiraf ediyor, onun tarafindan Seria Irmagi'nda vaftiz ediliyordu.
\par 7 Ne var ki, birçok Ferisi'yle Saduki'nin* vaftiz olmak için kendisine geldigini gören Yahya onlara söyle seslendi: "Ey engerekler soyu! Gelecek gazaptan kaçmak için sizi kim uyardi?
\par 8 Bundan böyle tövbeye yarasir meyveler verin.
\par 9 Kendi kendinize, 'Biz Ibrahim'in soyundaniz' diye düsünmeyin. Ben size sunu söyleyeyim: Tanri, Ibrahim'e su taslardan da çocuk yaratabilir.
\par 10 Balta agaçlarin köküne dayanmis bile. Iyi meyve vermeyen her agaç kesilip atese atilir.
\par 11 Gerçi ben sizi tövbe için suyla vaftiz ediyorum, ama benden sonra gelen benden daha güçlüdür. Ben O'nun çariklarini çikarmaya bile layik degilim. O sizi Kutsal Ruh'la ve atesle vaftiz edecek.
\par 12 Yabasi elindedir. Harman yerini temizleyecek, bugdayini toplayip ambara yigacak, samani ise sönmeyen ateste yakacak."
\par 13 Bu sirada Isa, Yahya tarafindan vaftiz edilmek üzere Celile'den Seria Irmagi'na, Yahya'nin yanina geldi.
\par 14 Ne var ki Yahya, "Benim senin tarafindan vaftiz edilmem gerekirken sen mi bana geliyorsun?" diyerek O'na engel olmak istedi.
\par 15 Isa ona su karsiligi verdi: "Simdilik buna razi ol! Çünkü dogru olan her seyi bu sekilde yerine getirmemiz gerekir." O zaman Yahya O'nun dedigine razi oldu.
\par 16 Isa vaftiz olur olmaz sudan çikti. O anda gökler açildi ve Isa, Tanri'nin Ruhu'nun güvercin gibi inip üzerine kondugunu gördü.
\par 17 Göklerden gelen bir ses, "Sevgili Oglum budur, O'ndan hosnudum" dedi.

\chapter{4}

\par 1 Bundan sonra Isa, Iblis tarafindan denenmek üzere Ruh araciligiyla çöle götürüldü.
\par 2 Isa kirk gün kirk gece oruç* tuttuktan sonra acikti.
\par 3 O zaman Ayartici yaklasip, "Tanri'nin Oglu'ysan, söyle su taslar ekmek olsun" dedi.
\par 4 Isa ona su karsiligi verdi: "'Insan yalniz ekmekle yasamaz, Tanri'nin agzindan çikan her sözle yasar' diye yazilmistir."
\par 5 Sonra Iblis O'nu kutsal kente* götürdü. Tapinagin tepesine çikarip, "Tanri'nin Oglu'ysan, kendini asagi at" dedi, "Çünkü söyle yazilmistir: 'Tanri, senin için meleklerine buyruk verecek.' 'Ayagin bir tasa çarpmasin diye Seni elleri üzerinde tasiyacaklar.'"
\par 7 Isa Iblis'e su karsiligi verdi: "'Tanrin Rab'bi denemeyeceksin' diye de yazilmistir."
\par 8 Iblis bu kez Isa'yi çok yüksek bir daga çikardi. O'na bütün görkemiyle dünya ülkelerini göstererek,
\par 9 "Yere kapanip bana taparsan, bütün bunlari sana verecegim" dedi.
\par 10 Isa ona söyle karsilik verdi: "Çekil git, Seytan! 'Tanrin Rab'be tapacak, yalniz O'na kulluk edeceksin' diye yazilmistir."
\par 11 Bunun üzerine Iblis Isa'yi birakip gitti. Melekler gelip Isa'ya hizmet ettiler.
\par 12 Isa, Yahya'nin tutuklandigini duyunca Celile'ye döndü.
\par 13 Nasira'dan ayrilarak Zevulun ve Naftali yöresinde, Celile Gölü kiyisinda bulunan Kefarnahum'a yerlesti.
\par 14 Bu, Peygamber Yesaya araciligiyla bildirilen su söz yerine gelsin diye oldu: "Zevulun ve Naftali bölgeleri, Seria Irmagi'nin ötesinde, Deniz Yolu'nda, Uluslarin yasadigi Celile!
\par 16 Karanlikta yasayan halk, Büyük bir isik gördü. Ölümün gölgeledigi diyarda Yasayanlara isik dogdu."
\par 17 O günden sonra Isa su çagrida bulunmaya basladi: "Tövbe edin! Çünkü Göklerin Egemenligi yaklasti."
\par 18 Isa, Celile Gölü'nün kiyisinda yürürken Petrus diye de anilan Simun'la kardesi Andreas'i gördü. Balikçi olan bu iki kardes göle ag atiyorlardi.
\par 19 Onlara, "Ardimdan gelin" dedi, "Sizleri insan tutan balikçilar yapacagim."
\par 20 Onlar da hemen aglarini birakip O'nun ardindan gittiler.
\par 21 Isa daha ileri gidince baska iki kardesi, Zebedi'nin ogullari Yakup'la Yuhanna'yi gördü. Babalari Zebedi'yle birlikte teknede aglarini onariyorlardi. Onlari da çagirdi.
\par 22 Hemen tekneyi ve babalarini birakip Isa'nin ardindan gittiler.
\par 23 Isa, Celile bölgesinin her tarafini dolasti. Buralardaki havralarda ögretiyor, göksel egemenligin Müjdesi'ni duyuruyor, halk arasinda rastlanan her hastaligi, her illeti iyilestiriyordu.
\par 24 Ünü bütün Suriye'ye yayilmisti. Türlü hastaliklara yakalanmis bütün hastalari, aci çekenleri, cinlileri, saralilari, felçlileri O'na getirdiler; hepsini iyilestirdi.
\par 25 Celile, Dekapolis, Yerusalim, Yahudiye ve Seria Irmagi'nin karsi yakasindan gelen büyük kalabaliklar O'nun ardindan gidiyordu.

\chapter{5}

\par 1 Isa kalabaliklari görünce daga çikti. Oturunca ögrencileri yanina geldi.
\par 2 Isa konusmaya baslayip onlara sunlari ögretti:
\par 3 "Ne mutlu ruhta yoksul olanlara! Çünkü Göklerin Egemenligi onlarindir.
\par 4 Ne mutlu yasli olanlara! Çünkü onlar teselli edilecekler.
\par 5 Ne mutlu yumusak huylu olanlara! Çünkü onlar yeryüzünü miras alacaklar.
\par 6 Ne mutlu dogruluga acikip susayanlara! Çünkü onlar doyurulacaklar.
\par 7 Ne mutlu merhametli olanlara! Çünkü onlar merhamet bulacaklar.
\par 8 Ne mutlu yüregi temiz olanlara! Çünkü onlar Tanri'yi görecekler.
\par 9 Ne mutlu barisi saglayanlara! Çünkü onlara Tanri ogullari denecek.
\par 10 Ne mutlu dogruluk ugruna zulüm görenlere! Çünkü Göklerin Egemenligi onlarindir.
\par 11 "Benim yüzümden insanlar size sövüp zulmettikleri, yalan yere size karsi her türlü kötü sözü söyledikleri zaman ne mutlu size!
\par 12 Sevinin, sevinçle cosun! Çünkü göklerdeki ödülünüz büyüktür. Sizden önce yasayan peygamberlere de böyle zulmettiler."
\par 13 "Yeryüzünün tuzu sizsiniz. Ama tuz tadini yitirirse, bir daha ona nasil tuz tadi verilebilir? Artik disari atilip ayak altinda çignenmekten baska ise yaramaz.
\par 14 "Dünyanin isigi sizsiniz. Tepeye kurulan kent gizlenemez.
\par 15 Kimse kandil yakip tahil ölçeginin altina koymaz. Tersine, kandillige koyar; evdekilerin hepsine isik saglar.
\par 16 Sizin isiginiz insanlarin önünde öyle parlasin ki, iyi islerinizi görerek göklerdeki Babaniz'i yüceltsinler!"
\par 17 "Kutsal Yasa'yi* ya da peygamberlerin sözlerini geçersiz kilmak için geldigimi sanmayin. Ben geçersiz kilmaya degil, tamamlamaya geldim.
\par 18 Size dogrusunu söyleyeyim, yer ve gök ortadan kalkmadan, her sey gerçeklesmeden, Kutsal Yasa'dan ufacik bir harf ya da bir nokta bile yok olmayacak.
\par 19 Bu nedenle, bu buyruklarin en küçügünden birini kim çigner ve baskalarina öyle ögretirse, Göklerin Egemenligi'nde en küçük sayilacak. Ama bu buyruklari kim yerine getirir ve baskalarina ögretirse, Göklerin Egemenligi'nde büyük sayilacak.
\par 20 Size sunu söyleyeyim: Dogrulugunuz din bilginleriyle* Ferisiler'inkini* asmadikça, Göklerin Egemenligi'ne asla giremezsiniz!"
\par 21 "Atalarimiza, 'Adam öldürmeyeceksin. Öldüren yargilanacak' dendigini duydunuz.
\par 22 Ama ben size diyorum ki, kardesine öfkelenen herkes yargilanacaktir. Kim kardesine asagilayici bir söz söylerse, Yüksek Kurul'da* yargilanacaktir. Kim kardesine ahmak derse, cehennem atesini hak edecektir.
\par 23 Bu yüzden, sunakta adak sunarken kardesinin sana karsi bir sikâyeti oldugunu animsarsan, adagini orada, sunagin önünde birak, git önce kardesinle baris; sonra gelip adagini sun.
\par 25 Senden davaci olanla daha yoldayken çabucak anlas. Yoksa o seni yargica, yargiç da gardiyana teslim edebilir; sonunda da hapse atilabilirsin.
\par 26 Sana dogrusunu söyleyeyim, borcunun son kurusunu ödemeden oradan asla çikamazsin."
\par 27 "'Zina etmeyeceksin' dendigini duydunuz.
\par 28 Ama ben size diyorum ki, bir kadina sehvetle bakan her adam, yüreginde o kadinla zina etmis olur.
\par 29 Eger sag gözün günah islemene neden olursa, onu çikar at. Çünkü vücudunun bir üyesinin yok olmasi, bütün vücudunun cehenneme atilmasindan iyidir.
\par 30 Eger sag elin günah islemene neden olursa, onu kes at. Çünkü vücudunun bir üyesinin yok olmasi, bütün vücudunun cehenneme gitmesinden iyidir.
\par 31 "'Kim karisini bosarsa ona bosanma belgesi versin' denmistir.
\par 32 Ama ben size diyorum ki, karisini fuhus disinda bir nedenle bosayan onu zinaya itmis olur. Bosanmis bir kadinla evlenen de zina etmis olur."
\par 33 "Yine atalarimiza, 'Yalan yere ant içmeyeceksin, ama Rab'bin önünde içtigin antlari yerine getireceksin' dendigini duydunuz.
\par 34 Oysa ben size diyorum ki, hiç ant içmeyin: Ne gök üzerine, çünkü orasi Tanri'nin tahtidir; ne yer üzerine, çünkü orasi O'nun ayak taburesidir; ne de Yerusalim üzerine, çünkü orasi Büyük Kral'in kentidir.
\par 36 Basinizin üzerine de ant içmeyin. Çünkü saçinizin tek telini ak ya da kara edemezsiniz.
\par 37 'Evet'iniz evet, 'hayir'iniz hayir olsun. Bundan fazlasi Seytan'dandir."
\par 38 "'Göze göz, dise dis' dendigini duydunuz.
\par 39 Ama ben size diyorum ki, kötüye karsi direnmeyin. Sag yanaginiza bir tokat atana öbür yanaginizi da çevirin.
\par 40 Size karsi davaci olup mintaninizi almak isteyene abanizi da verin.
\par 41 Sizi bin adim yol yürümeye zorlayanla iki bin adim yürüyün.
\par 42 Sizden bir sey dileyene verin, sizden ödünç isteyeni geri çevirmeyin."
\par 43 "'Komsunu seveceksin, düsmanindan nefret edeceksin' dendigini duydunuz.
\par 44 Ama ben size diyorum ki, düsmanlarinizi sevin, size zulmedenler için dua edin.
\par 45 Öyle ki, göklerdeki Babaniz'in ogullari olasiniz. Çünkü O, günesini hem kötülerin hem iyilerin üzerine dogdurur; yagmurunu hem dogrularin hem egrilerin üzerine yagdirir.
\par 46 Eger yalniz sizi sevenleri severseniz, ne ödülünüz olur? Vergi görevlileri* de öyle yapmiyor mu?
\par 47 Yalniz kardeslerinize selam verirseniz, fazladan ne yapmis olursunuz? Putperestler de öyle yapmiyor mu?
\par 48 Bu nedenle, göksel Babaniz yetkin oldugu gibi, siz de yetkin olun."

\chapter{6}

\par 1 "Dogrulugunuzu insanlarin gözü önünde gösteris amaciyla sergilemekten kaçinin. Yoksa göklerdeki Babaniz'dan ödül alamazsiniz.
\par 2 "Bu nedenle, birisine sadaka verirken bunu borazan çaldirarak ilan etmeyin. Ikiyüzlüler, insanlarin övgüsünü kazanmak için havralarda ve sokaklarda böyle yaparlar. Size dogrusunu söyleyeyim, onlar ödüllerini almislardir.
\par 3 Siz sadaka verirken, sol eliniz sag elinizin ne yaptigini bilmesin.
\par 4 Öyle ki, verdiginiz sadaka gizli kalsin. Gizlice yapilani gören Babaniz sizi ödüllendirecektir."
\par 5 "Dua ettiginiz zaman ikiyüzlüler gibi olmayin. Onlar, herkes kendilerini görsün diye havralarda ve caddelerin köse baslarinda dikilip dua etmekten zevk alirlar. Size dogrusunu söyleyeyim, onlar ödüllerini almislardir.
\par 6 Ama siz dua edeceginiz zaman iç odaniza çekilip kapiyi örtün ve gizlide olan Babaniz'a dua edin. Gizlilik içinde yapilani gören Babaniz sizi ödüllendirecektir.
\par 7 Dua ettiginizde, putperestler gibi bos sözler tekrarlayip durmayin. Onlar söz kalabaligiyla seslerini duyurabileceklerini sanirlar.
\par 8 Siz onlara benzemeyin! Çünkü Babaniz nelere gereksinmeniz oldugunu siz daha O'ndan dilemeden önce bilir.
\par 9 "Bunun için siz söyle dua edin: 'Göklerdeki Babamiz, Adin kutsal kilinsin.
\par 10 Egemenligin gelsin. Gökte oldugu gibi, yeryüzünde de Senin istedigin olsun.
\par 11 Bugün bize gündelik ekmegimizi ver.
\par 12 Bize karsi suç isleyenleri bagisladigimiz gibi, Sen de bizim suçlarimizi bagisla.
\par 13 Ayartilmamiza izin verme. Bizi kötü olandan kurtar. Çünkü egemenlik, güç ve yücelik Sonsuzlara dek senindir! Amin'.
\par 14 "Baskalarinin suçlarini bagislarsaniz, göksel Babaniz da sizin suçlarinizi bagislar.
\par 15 Ama siz baskalarinin suçlarini bagislamazsaniz, Babaniz da sizin suçlarinizi bagislamaz."
\par 16 "Oruç* tuttugunuz zaman, ikiyüzlüler gibi surat asmayin. Onlar oruç tuttuklarini insanlara belli etmek için kendilerine perisan bir görünüm verirler. Size dogrusunu söyleyeyim, onlar ödüllerini almislardir.
\par 17 Siz oruç tuttugunuz zaman, basiniza yag sürüp yüzünüzü yikayin.
\par 18 Öyle ki, insanlara degil, gizlide olan Babaniz'a oruçlu görünesiniz. Gizlilik içinde yapilani gören Babaniz sizi ödüllendirecektir."
\par 19 "Yeryüzünde kendinize hazineler biriktirmeyin. Burada güve ve pas onlari yiyip bitirir, hirsizlar da girip çalarlar.
\par 20 Bunun yerine kendinize gökte hazineler biriktirin. Orada ne güve ne pas onlari yiyip bitirir, ne de hirsizlar girip çalar.
\par 21 Hazineniz neredeyse, yüreginiz de orada olacaktir.
\par 22 "Bedenin isigi gözdür. Gözünüz saglamsa, bütün bedeniniz aydinlik olur.
\par 23 Gözünüz bozuksa, bütün bedeniniz karanlik olur. Buna göre, içinizdeki 'isik' karanliksa, ne korkunçtur o karanlik!
\par 24 "Hiç kimse iki efendiye kulluk edemez. Ya birinden nefret edip öbürünü sever, ya da birine baglanip öbürünü hor görür. Siz hem Tanri'ya, hem de paraya kulluk edemezsiniz."
\par 25 "Bu nedenle size sunu söylüyorum: 'Ne yiyip ne içecegiz?' diye caniniz için, 'Ne giyecegiz?' diye bedeniniz için kaygilanmayin. Can yiyecekten, beden de giyecekten daha önemli degil mi?
\par 26 Gökte uçan kuslara bakin! Ne eker, ne biçer, ne de ambarlarda yiyecek biriktirirler. Göksel Babaniz yine de onlari doyurur. Siz onlardan çok daha degerli degil misiniz?
\par 27 Hangi biriniz kaygilanmakla ömrünü bir anlik uzatabilir?
\par 28 Giyecek konusunda neden kaygilaniyorsunuz? Kir zambaklarinin nasil büyüdügüne bakin! Ne çalisirlar, ne de iplik egirirler.
\par 29 Ama size sunu söyleyeyim, bütün görkemine karsin Süleyman bile bunlardan biri gibi giyinmis degildi.
\par 30 Bugün var olup yarin ocaga atilacak olan kir otunu böyle giydiren Tanri'nin sizi de giydirecegi çok daha kesin degil mi, ey kit imanlilar?
\par 31 "Öyleyse, 'Ne yiyecegiz?' 'Ne içecegiz?' ya da 'Ne giyecegiz?' diyerek kaygilanmayin.
\par 32 Uluslar hep bu seylerin pesinden giderler. Oysa göksel Babaniz bütün bunlara gereksinmeniz oldugunu bilir.
\par 33 Siz öncelikle O'nun egemenliginin ve dogrulugunun ardindan gidin, o zaman size bütün bunlar da verilecektir.
\par 34 O halde yarin için kaygilanmayin. Yarinin kaygisi yarinin olsun. Her günün derdi kendine yeter."

\chapter{7}

\par 1 "Baskasini yargilamayin ki, siz de yargilanmayasiniz.
\par 2 Çünkü nasil yargilarsaniz öyle yargilanacaksiniz. Hangi ölçekle verirseniz, ayni ölçekle alacaksiniz.
\par 3 Sen neden kardesinin gözündeki çöpü görürsün de kendi gözündeki mertegi farketmezsin?
\par 4 Kendi gözünde mertek varken kardesine nasil, 'Izin ver, gözündeki çöpü çikarayim' dersin?
\par 5 Seni ikiyüzlü! Önce kendi gözündeki mertegi çikar, o zaman kardesinin gözündeki çöpü çikarmak için daha iyi görürsün.
\par 6 "Kutsal olani köpeklere vermeyin. Incilerinizi domuzlarin önüne atmayin. Yoksa bunlari ayaklariyla çignedikten sonra dönüp sizi parçalayabilirler."
\par 7 "Dileyin, size verilecek; arayin, bulacaksiniz; kapiyi çalin, size açilacaktir.
\par 8 Çünkü her dileyen alir, arayan bulur, kapi çalana açilir.
\par 9 Hanginiz kendisinden ekmek isteyen ogluna tas verir?
\par 10 Ya da balik isterse yilan verir?
\par 11 Sizler kötü yürekli oldugunuz halde çocuklariniza güzel armaganlar vermeyi biliyorsaniz, göklerdeki Babaniz'in, kendisinden dileyenlere güzel armaganlar verecegi çok daha kesin degil mi?
\par 12 "Insanlarin size nasil davranmasini istiyorsaniz, siz de onlara öyle davranin. Çünkü Kutsal Yasa'nin ve peygamberlerin söyledigi budur."
\par 13 "Dar kapidan girin. Çünkü yikima götüren kapi genis ve yol enlidir. Bu kapidan girenler çoktur.
\par 14 Oysa yasama götüren kapi dar, yol da çetindir. Bu yolu bulanlar azdir."
\par 15 "Sahte peygamberlerden sakinin! Onlar size kuzu postuna bürünerek yaklasirlar, ama özde yirtici kurtlardir.
\par 16 Onlari meyvelerinden taniyacaksiniz. Dikenli bitkilerden üzüm, devedikenlerinden incir toplanabilir mi?
\par 17 Bunun gibi, her iyi agaç iyi meyve verir, kötü agaç ise kötü meyve verir.
\par 18 Iyi agaç kötü meyve, kötü agaç da iyi meyve veremez.
\par 19 Iyi meyve vermeyen her agaç kesilip atese atilir.
\par 20 Böylece sahte peygamberleri meyvelerinden taniyacaksiniz.
\par 21 "Bana, 'Ya Rab, ya Rab!' diye seslenen herkes Göklerin Egemenligi'ne girmeyecek. Ancak göklerdeki Babam'in istegini yerine getiren girecektir.
\par 22 O gün birçoklari bana diyecek ki, 'Ya Rab, ya Rab! Biz senin adinla peygamberlik etmedik mi? Senin adinla cinler kovmadik mi? Senin adinla birçok mucize yapmadik mi?'
\par 23 O zaman ben de onlara açikça, 'Sizi hiç tanimadim, uzak durun benden, ey kötülük yapanlar!' diyecegim."
\par 24 "Iste bu sözlerimi duyup uygulayan herkes, evini kaya üzerine kuran akilli adama benzer.
\par 25 Yagmur yagar, seller basar, yeller eser, eve saldirir; ama ev yikilmaz. Çünkü kaya üzerine kurulmustur.
\par 26 Bu sözlerimi duyup da uygulamayan herkes, evini kum üzerine kuran budala adama benzer.
\par 27 Yagmur yagar, seller basar, yeller eser, evi sarsar. Ev yikilir; yikilisi da korkunç olur."
\par 28 Isa konusmasini bitirince, halk O'nun ögretisine sasip kaldi.
\par 29 Çünkü onlara kendi din bilginleri gibi degil, yetkili biri gibi ögretiyordu.

\chapter{8}

\par 1 Isa dagdan inince büyük bir kalabalik O'nun ardindan gitti.
\par 2 Bu sirada cüzamli* bir adam yaklasip, "Ya Rab, istersen beni temiz kilabilirsin" diyerek O'nun ayaklarina kapandi.
\par 3 Isa elini uzatip adama dokundu, "Isterim, temiz ol!" dedi. Adam aninda cüzamdan temizlendi.
\par 4 Sonra Isa adama, "Sakin kimseye bir sey söyleme!" dedi. "Git, kâhine* görün ve cüzamdan temizlendigini herkese kanitlamak için Musa'nin buyurdugu sunuyu sun."
\par 5 Isa Kefarnahum'a varinca bir yüzbasi O'na gelip, "Ya Rab" diye yalvardi, "Usagim felç oldu, evde yatiyor; korkunç aci çekiyor."
\par 7 Isa, "Gelip onu iyilestirecegim" dedi.
\par 8 Ama yüzbasi, "Ya Rab, evime girmene layik degilim" dedi, "Yeter ki bir söz söyle, usagim iyilesir.
\par 9 Ben de buyruk altinda bir adamim, benim de buyrugumda askerlerim var. Birine, 'Git' derim, gider; ötekine, 'Gel' derim, gelir; köleme, 'Sunu yap' derim, yapar."
\par 10 Isa, duydugu bu sözlere hayran kaldi. Ardindan gelenlere, "Size dogrusunu söyleyeyim" dedi, "Ben Israil'de böyle imani olan birini görmedim.
\par 11 Size sunu söyleyeyim, dogudan ve batidan birçok insan gelecek, Göklerin Egemenligi'nde Ibrahim'le, Ishak'la ve Yakup'la birlikte sofraya oturacaklar.
\par 12 Ama bu egemenligin asil mirasçilari disaridaki karanliga atilacak. Orada aglayis ve dis gicirtisi olacak."
\par 13 Sonra Isa yüzbasiya, "Git, inandigin gibi olsun" dedi. Ve usak o anda iyilesti.
\par 14 Isa Petrus'un evine geldiginde onun kaynanasinin atesler içinde yattigini gördü.
\par 15 Eline dokununca kadinin atesi düstü. Kadin kalkip Isa'ya hizmet etmeye basladi.
\par 16 Aksam olunca birçok cinliyi kendisine getirdiler. Isa onlardaki kötü ruhlari tek sözle kovdu, hastalarin hepsini iyilestirdi.
\par 17 Bu, Peygamber Yesaya araciligiyla bildirilen su söz yerine gelsin diye oldu: "Zayifliklarimizi O kaldirdi, Hastaliklarimizi O üstlendi."
\par 18 Isa, çevresindeki kalabaligi görünce gölün karsi yakasina geçilmesini buyurdu.
\par 19 O sirada din bilginlerinden* biri O'na yaklasip, "Ögretmenim" dedi, "Nereye gidersen, senin ardindan gelecegim."
\par 20 Isa ona, "Tilkilerin ini, kuslarin yuvasi var, ama Insanoglu'nun* basini yaslayacak bir yeri yok" dedi.
\par 21 Baska bir ögrencisi Isa'ya, "Ya Rab, izin ver, önce gidip babami gömeyim" dedi.
\par 22 Isa ona, "Ardimdan gel" dedi. "Birak ölüleri, kendi ölülerini kendileri gömsün."
\par 23 Isa tekneye binince, ardindan ögrencileri de bindi.
\par 24 Gölde ansizin büyük bir firtina koptu. Öyle ki, dalgalar teknenin üzerinden asiyordu. Isa bu arada uyuyordu.
\par 25 Ögrenciler gidip O'nu uyandirarak, "Ya Rab, kurtar bizi, yoksa ölecegiz!" dediler.
\par 26 Isa, "Neden korkuyorsunuz, ey kit imanlilar?" dedi. Sonra kalkip rüzgari ve gölü azarladi. Ortalik sütliman oldu.
\par 27 Hepsi hayret içinde kaldi. "Bu nasil bir adam ki, rüzgar da göl de O'nun sözünü dinliyor?" dediler.
\par 28 Isa gölün karsi yakasinda Gadaralilar'in memleketine vardi. Orada O'nu mezarlik magaralardan çikan iki cinli karsiladi. Bunlar öyle tehlikeliydi ki, kimse o yoldan geçemiyordu.
\par 29 Isa'ya, "Ey Tanri'nin Oglu, bizden ne istiyorsun?" diye bagirdilar. "Buraya, vaktinden önce bize iskence etmek için mi geldin?"
\par 30 Onlardan uzakta otlayan büyük bir domuz sürüsü vardi.
\par 31 Cinler Isa'ya, "Bizi kovacaksan, su domuz sürüsüne gönder" diye yalvardilar.
\par 32 Isa onlara, "Gidin!" dedi. Cinler de adamlardan çikip domuzlarin içine girdiler. O anda bütün sürü dik yamaçtan asagi kosusarak göle atlayip boguldu.
\par 33 Domuzlari güdenler kaçip kente gittiler. Cinli adamlarla ilgili haberler dahil, olup bitenlerin hepsini anlattilar.
\par 34 Bunun üzerine bütün kent halki Isa'yi karsilamaya çikti. O'nu görünce bölgelerinden ayrilmasi için yalvardilar.

\chapter{9}

\par 1 Isa tekneye binip karsi kiyiya geçti ve kendi kentine gitti.
\par 2 Kendisine, yatak üzerinde felçli bir adam getirdiler. Isa onlarin imanini görünce felçliye, "Cesur ol, oglum, günahlarin bagislandi" dedi.
\par 3 Bunun üzerine bazi din bilginleri içlerinden, "Bu adam Tanri'ya küfrediyor!" dediler.
\par 4 Onlarin ne düsündüklerini bilen Isa dedi ki, "Yüreginizde neden kötü düsüncelere yer veriyorsunuz?
\par 5 Hangisi daha kolay? 'Günahlarin bagislandi' demek mi, yoksa 'Kalk, yürü' demek mi?
\par 6 Ne var ki, Insanoglu'nun* yeryüzünde günahlari bagislama yetkisine sahipoldugunu bilesiniz diye..." Sonra felçliye, "Kalk, yatagini topla, evine git!" dedi.
\par 7 Adam da kalkip evine gitti.
\par 8 Halk bunu görünce korkuya kapildi. Insana böyle bir yetki veren Tanri'yi yücelttiler.
\par 9 Isa oradan geçerken, vergi toplama yerinde oturan birini gördü. Matta adindaki bu adama, "Ardimdan gel" dedi. Adam da kalkip Isa'nin ardindan gitti.
\par 10 Sonra Isa, Matta'nin evinde sofrada otururken, birçok vergi görevlisiyle* günahkâr gelip O'nunla ve ögrencileriyle birlikte sofraya oturdu.
\par 11 Bunu gören Ferisiler, Isa'nin ögrencilerine, "Sizin ögretmeniniz neden vergi görevlileri ve günahkârlarla birlikte yemek yiyor?" diye sordular.
\par 12 Isa bunu duyunca söyle dedi: "Saglamlarin degil, hastalarin hekime ihtiyaci var.
\par 13 Gidin de, 'Ben kurban degil, merhamet isterim' sözünün anlamini ögrenin. Çünkü ben dogru kisileri degil, günahkârlari çagirmaya geldim."
\par 14 Bu arada Yahya'nin ögrencileri gelip Isa'ya, "Neden biz ve Ferisiler oruç tutuyoruz da senin ögrencilerin tutmuyor?" diye sordular.
\par 15 Isa söyle karsilik verdi: "Güvey aralarindayken, davetliler yas tutar mi? Ama güveyin aralarindan alinacagi günler gelecek, o zaman oruç tutacaklar.
\par 16 Hiç kimse eski giysiyi yeni kumas parçasiyla yamamaz. Çünkü yeni kumas çeker, giysiden kopar, yirtik daha beter olur.
\par 17 Hiç kimse yeni sarabi eski tulumlara doldurmaz. Yoksa tulumlar patlar; hem sarap dökülür, hem de tulumlar mahvolur. Yeni sarap yeni tulumlara konur, böylece her ikisi de korunmus olur."
\par 18 Isa onlara bu sözleri söylerken bir havra yöneticisi gelip O'nun önünde yere kapanarak, "Kizim az önce öldü. Ama sen gelip elini onun üzerine koyarsan, dirilecek" dedi.
\par 19 Isa kalkip ögrencileriyle birlikte adamin ardindan gitti.
\par 20 Tam o sirada, on iki yildir kanamasi olan bir kadin Isa'nin arkasindan yetisip giysisinin etegine dokundu.
\par 21 Içinden, "Giysisine bir dokunsam kurtulurum" diyordu.
\par 22 Isa arkasina dönüp onu görünce, "Cesur ol, kizim! Imanin seni kurtardi" dedi. Ve kadin o anda iyilesti.
\par 23 Isa, yöneticinin evine varip kaval çalanlarla gürültülü kalabaligi görünce, "Çekilin!" dedi. "Kiz ölmedi, uyuyor." Onlar ise kendisiyle alay ettiler.
\par 25 Kalabalik disari çikarilinca Isa içeri girip kizin elini tuttu, kiz ayaga kalkti.
\par 26 Bu haber bütün bölgeye yayildi.
\par 27 Isa oradan ayrilirken iki kör, "Ey Davut Oglu, halimize aci!" diye feryat ederek O'nun ardindan gittiler.
\par 28 Isa eve girince körler yanina geldi. Onlara, "Istediginizi yapabilecegime inaniyor musunuz?" diye sordu. Körler, "Inaniyoruz, ya Rab!" dediler.
\par 29 Bunun üzerine Isa körlerin gözlerine dokunarak, "Imaniniza göre olsun" dedi.
\par 30 Ve adamlarin gözleri açildi Isa, "Sakin kimse bunu bilmesin" diyerek onlari siki siki uyardi.
\par 31 Onlar ise çikip Isa'yla ilgili haberi bütün bölgeye yaydilar.
\par 32 Adamlar çikarken Isa'ya dilsiz bir cinli getirdiler.
\par 33 Cin kovulunca adamin dili çözüldü. Halk hayret içinde, "Israil'de böylesi hiç görülmemistir" diyordu.
\par 34 Ferisiler ise, "Cinleri, cinlerin önderinin gücüyle kovuyor" diyorlardi.
\par 35 Isa bütün kent ve köyleri dolasarak havralarda ögretiyor, göksel egemenligin Müjdesi'ni duyuruyor, her hastaligi, her illeti iyilestiriyordu.
\par 36 Kalabaliklari görünce onlara acidi. Çünkü çobansiz koyunlar gibi saskin ve perisandilar.
\par 37 O zaman Isa ögrencilerine, "Ürün bol, ama isçi az" dedi,
\par 38 "Bu nedenle ürünün sahibi Rab'be yalvarin, ürününü kaldiracak isçiler göndersin."

\chapter{10}

\par 1 Isa on iki ögrencisini yanina çagirip onlara kötü ruhlar üzerinde yetki verdi. Böylece kötü ruhlari kovacak, her hastaligi, her illeti iyilestireceklerdi.
\par 2 Bu on iki elçinin adlari söyle: Birincisi Petrus adiyla bilinen Simun, onun kardesi Andreas, Zebedi'nin ogullari Yakup ve Yuhanna, Filipus ve Bartalmay, Tomas ve vergi görevlisi Matta, Alfay oglu Yakup ve Taday, Yurtsever* Simun ve Isa'ya ihanet eden Yahuda Iskariot.
\par 5 Isa Onikiler'i su buyrukla halkin arasina gönderdi: "Öteki uluslarin arasina girmeyin. Samiriyeliler'in kentlerine de ugramayin.
\par 6 Bunun yerine, Israil halkinin yitik koyunlarina gidin.
\par 7 Gittiginiz her yerde Göklerin Egemenligi'nin yaklastigini duyurun.
\par 8 Hastalari iyilestirin, ölüleri diriltin, cüzamlilari* temiz kilin, cinleri kovun. Karsiliksiz aldiniz, karsiliksiz verin.
\par 9 Kusaginiza altin, gümüs, ya da bakir para koymayin.
\par 10 Yolculuk için ne torba, ne yedek mintan, ne çarik, ne de degnek alin. Çünkü isçi yiyecegini hak eder.
\par 11 Hangi kent ya da köye girerseniz, orada saygideger birini arayin ve ayrilincaya dek onunla kalin.
\par 12 Onun evine girerken, evdekilere esenlik dileyin.
\par 13 Eger evdekiler buna layiksa, dilediginiz esenlik üzerlerinde kalsin; layik degillerse, size geri dönsün.
\par 14 Sizi kabul etmez, sözlerinizi dinlemezlerse o evden ya da kentten ayrilirken, ayaklarinizin tozunu silkin.
\par 15 Size dogrusunu söyleyeyim, yargi günü o kentin hali Sodom'la Gomora bölgesinin halinden beter olacaktir."
\par 16 "Iste, sizi koyunlar gibi kurtlarin arasina gönderiyorum. Yilan gibi zeki, güvercin gibi saf olun.
\par 17 Insanlardan sakinin. Çünkü sizi mahkemelere verecek, havralarinda kamçilayacaklar.
\par 18 Benden ötürü valilerin, krallarin önüne çikarilacak, böylece onlara ve uluslara taniklik edeceksiniz.
\par 19 Sizleri mahkemeye verdiklerinde, neyi nasil söyleyeceginizi düsünerek kaygilanmayin. Ne söyleyeceginiz o anda size bildirilecek.
\par 20 Çünkü konusan siz degil, araciliginizla konusan Babaniz'in Ruhu olacak.
\par 21 "Kardes kardesi, baba çocugunu ölüme teslim edecek. Çocuklar anne babaya baskaldirip onlari öldürtecek.
\par 22 :22 Benim adimdan ötürü herkes sizden nefret edecek. Ama sonuna kadar dayanan kurtulacaktir.
\par 23 Bir kentte size zulmettikleri zaman ötekine kaçin. Size dogrusunu söyleyeyim, Insanoglu* gelinceye dek Israil'in bütün kentlerini dolasmis olmayacaksiniz.
\par 24 "Ögrenci ögretmeninden, köle efendisinden üstün degildir.
\par 25 Ögrencinin ögretmeni gibi, kölenin de efendisi gibi olmasi yeterlidir. Insanlar evin efendisine Baalzevul* derlerse, ev halkina neler demezler!"
\par 26 "Bu yüzden onlardan korkmayin. Çünkü örtülü olup da açiga çikarilmayacak, gizli olup da bilinmeyecek hiçbir sey yoktur.
\par 27 Size karanlikta söylediklerimi, siz gün isiginda söyleyin. Kulaginiza fisildanani, damlardan duyurun.
\par 28 Bedeni öldüren, ama cani öldüremeyenlerden korkmayin. Cani da bedeni de cehennemde mahvedebilen Tanri'dan korkun.
\par 29 Iki serçe bir metelige satilmiyor mu? Ama Babaniz'in izni olmadan bunlardan bir teki bile yere düsmez.
\par 30 Size gelince, basinizdaki bütün saçlar bile sayilidir.
\par 31 Onun için korkmayin, siz birçok serçeden daha degerlisiniz.
\par 32 "Insanlarin önünde beni açikça kabul eden herkesi, ben de göklerdeki Babam'in önünde açikça kabul edecegim.
\par 33 Insanlarin önünde beni inkâr edeni, ben de göklerdeki Babam'in önünde inkâr edecegim."
\par 34 "Yeryüzüne baris getirmeye geldigimi sanmayin! Baris degil, kiliç getirmeye geldim.
\par 35 Çünkü ben babayla ogulun, anneyle kizin, gelinle kaynananin arasina ayrilik sokmaya geldim.
\par 36 'Insanin düsmani kendi ev halki olacak.'
\par 37 Annesini ya da babasini beni sevdiginden çok seven bana layik degildir. Oglunu ya da kizini beni sevdiginden çok seven bana layik degildir.
\par 38 Çarmihini yüklenip ardimdan gelmeyen bana layik degildir.
\par 39 Canini kurtaran onu yitirecek. Canini benim ugruma yitiren ise onu kurtaracaktir.
\par 40 "Sizi kabul eden beni kabul etmis olur. Beni kabul eden de beni göndereni kabul etmis olur.
\par 41 Bir peygamberi peygamber oldugu için kabul eden, peygambere yarasan bir ödül alacaktir. Dogru birini dogru oldugu için kabul eden, dogru kisiye yarasan bir ödül alacaktir.
\par 42 Bu siradan kisilerden birine, ögrencim oldugu için bir bardak soguk su bile veren, size dogrusunu söyleyeyim, ödülsüz kalmayacaktir."

\chapter{11}

\par 1 Isa, on iki ögrencisine bu buyruklari verdikten sonra onlarin kentlerinde ögretmek ve Tanri sözünü duyurmak üzere oradan ayrildi.
\par 2 Tutukevinde bulunan Yahya, Mesih'in yaptigi isleri duyunca, O'na gönderdigi ögrencileri araciligiyla sunu sordu: "Gelecek Olan sen misin, yoksa baskasini mi bekleyelim?"
\par 5 Körlerin gözleri açiliyor, kötürümler yürüyor, cüzamlilar temiz kiliniyor, sagirlar isitiyor, ölüler diriliyor ve Müjde yoksullara duyuruluyor.
\par 6 Benden ötürü sendeleyip düsmeyene ne mutlu!"
\par 7 Yahya'nin ögrencileri ayrilirken Isa halka Yahya'dan söz etmeye basladi. "Çöle ne görmeye gittiniz?" dedi. "Rüzgarda sallanan bir kamis mi?
\par 8 Söyleyin, ne görmeye gittiniz? Pahali giysiler giymis bir adam mi? Oysa pahali giysi giyenler, kral saraylarinda bulunur.
\par 9 Öyleyse ne görmeye gittiniz? Bir peygamber mi? Evet! Size sunu söyleyeyim, gördügünüz kisi peygamberden de üstündür.
\par 10 'Iste, habercimi senin önünden gönderiyorum; O önden gidip senin yolunu hazirlayacak' diye yazilmis olan sözler onunla ilgilidir.
\par 11 Size dogrusunu söyleyeyim, kadindan doganlar arasinda Vaftizci Yahya'dan daha üstün biri çikmamistir. Bununla birlikte, Göklerin Egemenligi'nde en küçük olan ondan üstündür.
\par 12 Vaftizci Yahya'nin ortaya çiktigi günden bu yana Göklerin Egemenligi zorlaniyor, zorlu kisiler onu ele geçirmeye çalisiyor.
\par 13 Yahya'ya dek bütün peygamberlerle Kutsal Yasa, olacaklari önceden bildirdiler.
\par 14 Eger bunu kabul etmek isterseniz, gelecek olan Ilyas odur.
\par 15 Kulagi olan, isitsin!
\par 16 "Bu kusagin insanlarini neye benzeteyim? Çarsi meydanlarinda oturup arkadaslarina, 'Size kaval çaldik, oynamadiniz; Agit yaktik, dövünmediniz' diye seslenen çocuklara benziyorlar.
\par 18 Yahya geldigi zaman oruç tutup içkiden kaçindi, ona 'cinli' diyorlar.
\par 19 Insanoglu* geldigi zaman yiyip içti. Bu kez de diyorlar ki, 'Su obur ve ayyas adama bakin! Vergi görevlileri* ve günahkârlarla dost oldu!' Ne var ki bilgelik, ortaya koydugu islerle dogrulanir."
\par 20 Sonra Isa, mucizelerinin çogunu yapmis oldugu kentleri, tövbe etmedikleri için söyle azarlamaya basladi: "Vay haline, ey Horazin! Vay haline, ey Beytsayda! Sizlerde yapilan mucizeler Sur ve Sayda'da yapilmis olsaydi, çoktan çul* kusanip kül içinde oturarak tövbe etmis olurlardi.
\par 22 Size sunu söyleyeyim, yargi günü sizin haliniz Sur ve Sayda'nin halinden beter olacaktir!
\par 23 Ya sen, ey Kefarnahum, göge mi çikarilacaksin? Hayir, ölüler diyarina indirileceksin! Çünkü sende yapilan mucizeler Sodom'da yapilmis olsaydi, bugüne dek ayakta kalirdi.
\par 24 Sana sunu söyleyeyim, yargi günü senin halin Sodom bölgesinin halinden beter olacaktir!"
\par 25 Isa bundan sonra söyle dedi: "Baba, yerin ve gögün Rabbi! Bu gerçekleri bilge ve akilli kisilerden gizleyip küçük çocuklara açtigin için sana sükrederim.
\par 26 Evet Baba, senin istegin buydu.
\par 27 "Babam her seyi bana teslim etti. Ogul'u, Baba'dan baska kimse tanimaz. Baba'yi da Ogul'dan ve Ogul'un O'nu tanitmak istedigi kisilerden baskasi tanimaz.
\par 28 "Ey bütün yorgunlar ve yükü agir olanlar! Bana gelin, ben size rahat veririm.
\par 29 Boyundurugumu yüklenin, benden ögrenin. Çünkü ben yumusak huylu, alçakgönüllüyüm. Böylece canlariniz rahata kavusur.
\par 30 Boyundurugumu tasimak kolay, yüküm hafiftir."

\chapter{12}

\par 1 O siralarda, bir Sabat Günü* Isa ekinler arasindan geçiyordu. Ögrencileri acikinca basaklari koparip yemeye basladilar.
\par 2 Bunu gören Ferisiler Isa'ya, "Bak, ögrencilerin Sabat Günü yasak olani yapiyor" dediler.
\par 3 Isa onlara, "Davut'la yanindakiler acikinca Davut'un ne yaptigini okumadiniz mi?" diye sordu.
\par 4 "Tanri'nin evine girdi, kendisinin ve yanindakilerin yemesi yasak olan, ancak kâhinlerin yiyebilecegi adak ekmeklerini* yedi.
\par 5 Ayrica kâhinlerin her hafta tapinakta Sabat Günü'yle ilgili buyrugu çignedikleri halde suçlu sayilmadiklarini Kutsal Yasa'da okumadiniz mi?
\par 6 Size sunu söyleyeyim, burada tapinaktan daha üstün bir sey var.
\par 7 Eger siz, 'Ben kurban degil, merhamet isterim' sözünün anlamini bilseydiniz, suçsuzlari yargilamazdiniz.
\par 8 Çünkü Insanoglu* Sabat Günü'nün de Rabbi'dir."
\par 9 Isa oradan ayrilip onlarin havrasina gitti.
\par 10 Orada eli sakat bir adam vardi. Isa'yi suçlamak amaciyla kendisine, "Sabat Günü bir hastayi iyilestirmek Kutsal Yasa'ya uygun mudur?" diye sordular.
\par 11 Isa onlara su karsiligi verdi: "Hanginizin bir koyunu olur da Sabat Günü çukura düserse onu tutup çikarmaz?
\par 12 Insan koyundan çok daha degerlidir! Demek ki, Sabat Günü iyilik yapmak Yasa'ya uygundur."
\par 13 Sonra adama, "Elini uzat" dedi. Adam elini uzatti. Eli öteki gibi yine sapasaglam oluverdi.
\par 14 Bunun üzerine Ferisiler disari çiktilar, Isa'yi yok etmek için anlastilar.
\par 15 Isa bunu bildigi için oradan ayrildi. Birçok kisi ardindan gitti. Isa hepsini iyilestirdi.
\par 16 Kim oldugunu açiklamamalari için onlari uyardi.
\par 17 Bu, Peygamber Yesaya araciligiyla bildirilen su söz yerine gelsin diye oldu: "Iste Kulum, O'nu ben seçtim. Gönlümün hosnut oldugu sevgili Kulum O'dur. Ruhum'u O'nun üzerine koyacagim, O da adaleti uluslara bildirecek.
\par 19 Çekisip bagirmayacak, Sokaklarda kimse O'nun sesini duymayacak.
\par 20 Ezilmis kamisi kirmayacak, Tüten fitili söndürmeyecek, Ve sonunda adaleti zafere ulastiracak.
\par 21 Uluslar da O'nun adina umut baglayacak."
\par 22 Daha sonra Isa'ya kör ve dilsiz bir cinli getirdiler. Isa adami iyilestirdi. Adam konusmaya, görmeye basladi.
\par 23 Bütün kalabalik sasirip kaldi. "Bu, Davut'un Oglu* olabilir mi?" diye soruyorlardi.
\par 24 Ferisiler bunu duyunca, "Bu adam cinleri, ancak cinlerin önderi Baalzevul'un* gücüyle kovuyor" dediler.
\par 25 Onlarin ne düsündügünü bilen Isa söyle dedi: "Kendi içinde bölünen ülke yikilir. Kendi içinde bölünen kent ya da ev ayakta kalamaz.
\par 26 Eger Seytan Seytan'i kovarsa, kendi içinde bölünmüs demektir. Bu durumda onun egemenligi nasil ayakta kalabilir?
\par 27 Eger ben cinleri Baalzevul'un gücüyle kovuyorsam, sizin adamlariniz kimin gücüyle kovuyor? Bu durumda sizi kendi adamlariniz yargilayacak.
\par 28 Ama ben cinleri Tanri'nin Ruhu'yla kovuyorsam, Tanri'nin Egemenligi üzerinize gelmis demektir.
\par 29 "Bir kimse güçlü adamin evine girip malini nasil çalabilir? Ancak onu bagladiktan sonra evini soyabilir.
\par 30 "Benden yana olmayan bana karsidir. Benimle birlikte toplamayan dagitiyor demektir.
\par 31 Bunun için size diyorum ki, insanlarin isledigi her günah, ettigi her küfür bagislanacak; ama Ruh'a edilen küfür bagislanmayacaktir.
\par 32 Insanoglu'na* karsi bir söz söyleyen, bagislanacak; ama Kutsal Ruh'a karsi bir söz söyleyen, ne bu çagda, ne de gelecek çagda bagislanacaktir.
\par 33 "Ya agaci iyi, meyvesini de iyi sayin; ya da agaci kötü, meyvesini de kötü sayin. Çünkü her agaç meyvesinden taninir.
\par 34 Sizi engerekler soyu! Kötü olan sizler nasil iyi sözler söyleyebilirsiniz? Çünkü agiz yürekten tasani söyler.
\par 35 Iyi insan içindeki iyilik hazinesinden iyilik, kötü insan içindeki kötülük hazinesinden kötülük çikarir.
\par 36 Size sunu söyleyeyim, insanlar söyledikleri her bos söz için yargi günü hesap verecekler.
\par 37 Kendi sözlerinizle aklanacak, yine kendi sözlerinizle suçlu çikarilacaksiniz."
\par 38 Bunun üzerine bazi din bilginleri ve Ferisiler, "Ögretmenimiz, senden dogaüstü bir belirti görmek istiyoruz" dediler.
\par 39 Isa onlara su karsiligi verdi: "Kötü ve vefasiz kusak bir belirti istiyor! Ama ona Peygamber Yunus'un belirtisinden baska bir belirti gösterilmeyecektir.
\par 40 Yunus, nasil üç gün üç gece o koca baligin karninda kaldiysa, Insanoglu* da üç gün üç gece yerin bagrinda kalacaktir.
\par 41 Ninova halki, yargi günü bu kusakla birlikte kalkip bu kusagi yargilayacak. Çünkü Ninovalilar, Yunus'un çagrisi üzerine tövbe ettiler. Bakin, Yunus'tan daha üstün olan buradadir.
\par 42 Güney Kraliçesi, yargi günü bu kusakla birlikte kalkip bu kusagi yargilayacak. Çünkü kraliçe, Süleyman'in bilgece sözlerini dinlemek için dünyanin ta öbür ucundan gelmisti. Bakin, Süleyman'dan daha üstün olan buradadir.
\par 43 "Kötü ruh insandan çikinca kurak yerlerde dolanip huzur arar, ama bulamaz.
\par 44 O zaman, 'Çiktigim eve, kendi evime döneyim' der. Eve gelince orayi bombos, süpürülmüs, düzeltilmis bulur.
\par 45 Bunun üzerine gider, yanina kendisinden kötü yedi ruh daha alir ve eve girip yerlesirler. Böylece o kisinin son durumu ilkinden beter olur. Bu kötü kusagin basina gelecek olan da budur."
\par 46 Isa daha halka konusurken, annesiyle kardesleri geldi. Disarida durmus, O'nunla konusmak istiyorlardi.
\par 47 Birisi Isa'ya, "Bak, annenle kardeslerin disarida duruyor, seninle görüsmek istiyorlar" dedi.
\par 48 Isa, kendisiyle konusana, "Kimdir annem, kimdir kardeslerim?" karsiligini verdi.
\par 49 Eliyle ögrencilerini göstererek, "Iste annem, iste kardeslerim!" dedi.
\par 50 "Göklerdeki Babam'in istegini kim yerine getirirse, kardesim, kizkardesim ve annem odur."

\chapter{13}

\par 1 Ayni gün Isa evden çikti, gidip göl kiyisinda oturdu.
\par 2 Çevresinde büyük bir kalabalik toplandi. Bu yüzden Isa tekneye binip oturdu. Bütün kalabalik kiyida duruyordu.
\par 3 Isa onlara benzetmelerle birçok sey anlatti. "Bakin" dedi, "Ekincinin biri tohum ekmeye çikti.
\par 4 Ektigi tohumlardan kimi yol kenarina düstü. Kuslar gelip bunlari yedi.
\par 5 Kimi, topragi az, kayalik yerlere düstü; toprak derin olmadigindan hemen filizlendi.
\par 6 Ne var ki, günes dogunca kavruldular, kök salamadiklari için kuruyup gittiler.
\par 7 Kimi, dikenler arasina düstü. Dikenler büyüdü, filizleri bogdu.
\par 8 Kimi ise iyi topraga düstü. Bazisi yüz, bazisi altmis, bazisi da otuz kat ürün verdi.
\par 9 Kulagi olan isitsin!"
\par 10 Ögrencileri gelip Isa'ya, "Halka neden benzetmelerle konusuyorsun?" diye sordular.
\par 11 Isa söyle yanitladi: "Göklerin Egemenligi'nin sirlarini bilme ayricaligi size verildi, ama onlara verilmedi.
\par 12 Çünkü kimde varsa, ona daha çok verilecek, bolluga kavusturulacak. Ama kimde yoksa, elindeki de alinacak.
\par 13 Onlara benzetmelerle konusmamin nedeni budur. Çünkü, 'Gördükleri halde görmezler, Duyduklari halde duymaz ve anlamazlar.'
\par 14 "Böylece Yesaya'nin peygamberlik sözü onlar için gerçeklesmis oldu: 'Duyacak duyacak, ama hiç anlamayacaksiniz, Bakacak bakacak, ama hiç görmeyeceksiniz!
\par 15 Çünkü bu halkin yüregi duygusuzlasti, Kulaklari agirlasti. Gözlerini kapadilar. Öyle ki, gözleri görmesin, Kulaklari duymasin, yürekleri anlamasin Ve bana dönmesinler. Dönselerdi, onlari iyilestirirdim.'
\par 16 "Ama ne mutlu size ki, gözleriniz görüyor, kulaklariniz isitiyor!
\par 17 Size dogrusunu söyleyeyim, nice peygamberler, nice dogru kisiler sizin gördüklerinizi görmek istediler, ama göremediler. Sizin isittiklerinizi isitmek istediler, ama isitemediler.
\par 18 "Simdi ekinciyle ilgili benzetmeyi siz dinleyin.
\par 19 Kim göksel egemenlikle ilgili sözü isitir de anlamazsa, kötü olan* gelir, onun yüregine ekileni söker götürür. Yol kenarina ekilen tohum iste budur.
\par 20 Kayalik yerlere ekilen ise isittigi sözü hemen sevinçle kabul eden, ama kök salamadigi için ancak bir süre dayanan kisidir. Böyle biri Tanri sözünden ötürü sikinti ya da zulme ugrayinca hemen sendeleyip düser.
\par 22 Dikenler arasinda ekilen de sudur: Sözü isitir, ama dünyasal kaygilar ve zenginligin aldaticiligi sözü bogar ve ürün vermesini engeller.
\par 23 Iyi topraga ekilen tohum ise, sözü isitip anlayan birine benzer. Böylesi elbette ürün verir, kimi yüz, kimi altmis, kimi de otuz kat."
\par 24 Isa onlara baska bir benzetme anlatti: "Göklerin Egemenligi, tarlasina iyi tohum eken adama benzer" dedi.
\par 25 "Herkes uyurken, adamin düsmani geldi, bugdayin arasina delice ekip gitti.
\par 26 Ekin gelisip basak salinca, deliceler de göründü.
\par 27 "Mal sahibinin köleleri gelip ona söyle dediler: 'Efendimiz, sen tarlana iyi tohum ekmedin mi? Bu deliceler nereden çikti?'
\par 28 "Mal sahibi, 'Bunu bir düsman yapmistir' dedi. "'Gidip deliceleri toplamamizi ister misin?' diye sordu köleler.
\par 29 "'Hayir' dedi adam. 'Deliceleri toplarken belki bugdayi da sökersiniz.
\par 30 Birakin biçim vaktine dek birlikte büyüsünler. Biçim vakti orakçilara, önce deliceleri toplayin diyecegim, yakmak için demet yapin. Bugdayi ise toplayip ambarima koyun.'"
\par 31 Isa onlara bir benzetme daha anlatti: "Göklerin Egemenligi, bir adamin tarlasina ektigi hardal tanesine benzer" dedi.
\par 32 "Hardal tohumlarin en küçügü oldugu halde, gelisince bahçe bitkilerinin boyunu asar, agaç olur. Böylece kuslar gelip dallarinda barinir."
\par 33 Isa onlara baska bir benzetme anlatti: "Göklerin Egemenligi, bir kadinin üç ölçek una karistirdigi mayaya benzer. Sonunda bütün hamur kabarir."
\par 34 Isa bütün bunlari halka benzetmelerle anlatti. Benzetme kullanmadan onlara hiçbir sey anlatmazdi.
\par 35 Bu, peygamber araciligiyla bildirilen su söz yerine gelsin diye oldu: "Agzimi benzetmeler anlatarak açacagim, Dünyanin kurulusundan beri Gizli kalmis sirlari dile getirecegim."
\par 36 Bundan sonra Isa halktan ayrilip eve gitti. Ögrencileri yanina gelip, "Tarladaki delicelerle ilgili benzetmeyi bize açikla" dediler.
\par 37 Isa, "Iyi tohumu eken, Insanoglu'dur*" diye karsilik verdi.
\par 38 "Tarla ise dünyadir. Iyi tohum, göksel egemenligin ogullari, deliceler de kötü olanin* ogullaridir.
\par 39 Deliceleri eken düsman, Iblis'tir. Biçim vakti, çagin sonu; orakçilar ise meleklerdir.
\par 40 "Deliceler nasil toplanip yakilirsa, çagin sonunda da böyle olacaktir.
\par 41 Insanoglu meleklerini gönderecek, onlar da insanlari günaha düsüren her seyi, kötülük yapan herkesi O'nun egemenliginden toplayip kizgin firina atacaklar. Orada aglayis ve dis gicirtisi olacaktir.
\par 43 Dogru kisiler o zaman Babalari'nin egemenliginde günes gibi parlayacaklar. Kulagi olan isitsin!"
\par 44 "Göklerin Egemenligi, tarlada sakli bir defineye benzer. Onu bulan yeniden sakladi, sevinçle kosup gitti, varini yogunu satip tarlayi satin aldi.
\par 45 "Yine Göklerin Egemenligi, güzel inciler arayan bir tüccara benzer.
\par 46 Tüccar, çok degerli bir inci bulunca gitti, varini yogunu satip o inciyi satin aldi."
\par 47 "Yine Göklerin Egemenligi, denize atilan ve her çesit baligi toplayan aga benzer.
\par 48 Ag dolunca onu kiyiya çekerler. Oturup ise yarayan baliklari kaplara koyar, yaramayanlari atarlar.
\par 49 Çagin sonunda da böyle olacak. Melekler gelecek, kötü kisileri dogrularin arasindan ayirip kizgin firina atacaklar. Orada aglayis ve dis gicirtisi olacaktir."
\par 51 Isa, "Bütün bunlari anladiniz mi?" diye sordu. "Evet" karsiligini verdiler.
\par 52 O da onlara, "Iste böylece Göklerin Egemenligi için egitilmis her din bilgini, hazinesinden hem yeni hem eski degerler çikaran bir mal sahibine benzer" dedi.
\par 53 Isa bütün bu benzetmeleri anlattiktan sonra oradan ayrildi.
\par 54 Kendi memleketine gitti ve oradaki havrada halka ögretmeye basladi. Halk sasip kalmisti. "Adamin bu bilgeligi ve mucizeler yaratan gücü nereden geliyor?" diyorlardi.
\par 55 "Marangozun oglu degil mi bu? Annesinin adi Meryem degil mi? Yakup, Yusuf, Simun ve Yahuda O'nun kardesleri degil mi?
\par 56 Kizkardeslerinin hepsi aramizda yasamiyor mu? O halde O'nun bütün bu yaptiklari nereden geliyor?"
\par 57 Ve gücenip O'nu reddettiler. Ama Isa onlara söyle dedi: "Bir peygamber, kendi memleketinden ve evinden baska yerde hor görülmez."
\par 58 Imansizliklari yüzünden Isa orada pek fazla mucize yapmadi.

\chapter{14}

\par 1 O günlerde Isa'yla ilgili haberleri duyan bölge krali* Hirodes, adamlarina, "Bu, Vaftizci Yahya'dir" dedi. "Ölümden dirildi. Olaganüstü güçlerin onda etkin olmasinin nedeni budur."
\par 3 Hirodes, kardesi Filipus'un karisi Hirodiya yüzünden Yahya'yi tutuklatmis, baglatip zindana attirmisti.
\par 4 Çünkü Yahya Hirodes'e, "O kadinla evlenmen Kutsal Yasa'ya aykiridir" demisti.
\par 5 Hirodes Yahya'yi öldürtmek istemis, ama halktan korkmustu. Çünkü halk Yahya'yi peygamber sayiyordu.
\par 6 Hirodes'in dogum günü senligi sirasinda Hirodiya'nin kizi ortaya çikip dans etti. Bu, Hirodes'in öyle hosuna gitti ki, ant içerek kiza ne dilerse verecegini söyledi.
\par 8 Kiz, annesinin kiskirtmasiyla, "Bana simdi, bir tepsi üzerinde Vaftizci Yahya'nin basini ver" dedi.
\par 9 Kral buna çok üzüldüyse de, konuklarinin önünde içtigi anttan ötürü bu dilegin yerine getirilmesini buyurdu.
\par 10 Adam gönderip zindanda Yahya'nin basini kestirdi.
\par 11 Kesik bas tepsiyle getirilip kiza verildi, kiz da bunu annesine götürdü.
\par 12 Yahya'nin ögrencileri gelip cesedi aldilar ve gömdüler. Sonra gidip Isa'ya haber verdiler.
\par 13 Isa bunu duyunca, tek basina tenha bir yere çekilmek üzere bir tekneyle oradan ayrildi. Bunu ögrenen halk, kentlerden çikip O'nu yaya olarak izledi.
\par 14 Isa tekneden inince büyük bir kalabalikla karsilasti. Onlara acidi ve hasta olanlarini iyilestirdi.
\par 15 Aksama dogru ögrencileri yanina gelip, "Burasi issiz bir yer" dediler, "Vakit de geç oldu. Halki saliver de köylere gidip kendilerine yiyecek alsinlar."
\par 16 Isa, "Gitmelerine gerek yok, onlara siz yiyecek verin" dedi.
\par 17 Ögrenciler, "Burada bes ekmekle iki baliktan baska bir seyimiz yok ki" dediler.
\par 18 Isa, "Onlari buraya, bana getirin" dedi.
\par 19 Halka çayira oturmalarini buyurduktan sonra, bes ekmekle iki baligi aldi, gözlerini göge kaldirarak sükretti; sonra ekmekleri bölüp ögrencilerine verdi, onlar da halka dagittilar.
\par 20 Herkes yiyip doydu. Artakalan parçalardan on iki sepet dolusu topladilar.
\par 21 Yemek yiyenlerin sayisi, kadin ve çocuklar hariç, yaklasik bes bin erkekti.
\par 22 Bundan hemen sonra Isa ögrencilerine, tekneye binip kendisinden önce karsi yakaya geçmelerini buyurdu. Bu arada halki evlerine gönderecekti.
\par 23 Halki gönderdikten sonra dua etmek için tek basina daga çikti. Aksam olurken orada yalnizdi.
\par 24 O sirada tekne kiyidan bir hayli uzakta dalgalarla bogusuyordu. Çünkü rüzgar karsi yönden esiyordu.
\par 25 Sabaha karsi Isa, gölün üstünde yürüyerek onlara yaklasti.
\par 26 Ögrenciler, O'nun gölün üstünde yürüdügünü görünce dehsete kapildilar. "Bu bir hayalet!" diyerek korkuyla bagristilar.
\par 27 Ama Isa hemen onlara seslenerek, "Cesur olun, benim, korkmayin!" dedi.
\par 28 Petrus buna karsilik, "Ya Rab" dedi, "Eger sen isen, buyruk ver suyun üstünden yürüyerek sana geleyim."
\par 29 Isa, "Gel!" dedi. Petrus da tekneden indi, suyun üstünden yürüyerek Isa'ya yaklasti.
\par 30 Ama rüzgarin ne kadar güçlü estigini görünce korktu, batmaya basladi. "Ya Rab, beni kurtar!" diye bagirdi.
\par 31 Isa hemen elini uzatip onu tuttu. Ona, "Ey kit imanli, neden kusku duydun?" dedi.
\par 32 Onlar tekneye bindikten sonra rüzgar dindi.
\par 33 Teknedekiler, "Sen gerçekten Tanri'nin Oglu'sun" diyerek O'na tapindilar.
\par 34 Gölü asip Ginnesar'da karaya çiktilar.
\par 35 Yöre halki Isa'yi taniyinca çevreye haber saldi. Bütün hastalari O'na getirdiler.
\par 36 Giysisinin etegine bir dokunsak diye yalvariyorlardi. Dokunanlarin hepsi iyilesti.

\chapter{15}

\par 1 Bu sirada Yerusalim'den bazi Ferisiler ve din bilginleri Isa'ya gelip, "Ögrencilerin neden atalarimizin töresini çigniyor?" diye sordular, "Yemekten önce ellerini yikamiyorlar."
\par 3 Isa onlara su karsiligi verdi: "Ya siz, neden töreniz ugruna Tanri buyrugunu çigniyorsunuz?
\par 4 Çünkü Tanri söyle buyurdu: 'Annene babana saygi göstereceksin'; 'Annesine ya da babasina söven kesinlikle öldürülecektir.'
\par 5 Ama siz, 'Her kim anne ya da babasina, benden alacagin bütün yardim Tanri'ya adanmistir derse, artik babasina saygi göstermek zorunda degildir' diyorsunuz. Böylelikle, töreniz ugruna Tanri'nin sözünü geçersiz kilmis oluyorsunuz.
\par 7 Ey ikiyüzlüler! Yesaya'nin sizinle ilgili su peygamberlik sözü ne kadar yerindedir: 'Bu halk dudaklariyla beni sayar, Ama yürekleri benden uzak.
\par 9 Bana bosuna taparlar. Çünkü ögrettikleri, sadece insan buyruklaridir.'"
\par 10 Isa, halki yanina çagirip onlara, "Dinleyin ve sunu belleyin" dedi.
\par 11 "Agizdan giren sey insani kirletmez. Insani kirleten agizdan çikandir."
\par 12 Bu sirada ögrencileri O'na gelip, "Biliyor musun?" dediler, "Ferisiler bu sözü duyunca gücendiler."
\par 13 Isa su karsiligi verdi: "Göksel Babam'in dikmedigi her fidan kökünden sökülecektir.
\par 14 Birakin onlari; onlar körlerin kör kilavuzlaridir. Eger kör köre kilavuzluk ederse, ikisi de çukura düser."
\par 15 Petrus, "Bu benzetmeyi bize açikla" dedi.
\par 16 "Siz de mi hâlâ anlamiyorsunuz?" diye sordu Isa.
\par 17 "Agza giren her seyin mideye indigini, oradan da helaya atildigini bilmiyor musunuz?
\par 18 Ne var ki agizdan çikan, yürekten kaynaklanir. Insani kirleten de budur.
\par 19 Çünkü kötü düsünceler, cinayet, zina, fuhus, hirsizlik, yalan yere taniklik ve iftira hep yürekten kaynaklanir.
\par 20 Insani kirleten bunlardir. Yikanmamis ellerle yemek yemek insani kirletmez."
\par 21 Isa oradan ayrilip Sur ve Sayda bölgesine geçti.
\par 22 O yöreden Kenanli bir kadin Isa'ya gelip, "Ya Rab, ey Davut Oglu, halime aci! Kizim cine tutuldu, çok kötü durumda" diye feryat etti.
\par 23 Isa kadina hiçbir karsilik vermedi. Ögrencileri yaklasip, "Sal sunu, gitsin!" diye rica ettiler. "Arkamizdan bagirip duruyor."
\par 24 Isa, "Ben yalniz Israil halkinin kaybolmus koyunlarina gönderildim" diye yanitladi.
\par 25 Kadin ise yaklasip, "Ya Rab, bana yardim et!" diyerek O'nun önünde yere kapandi.
\par 26 Isa ona, "Çocuklarin ekmegini alip köpeklere atmak dogru degildir" dedi.
\par 27 Kadin, "Haklisin, ya Rab" dedi. "Ama köpekler de efendilerinin sofrasindan düsen kirintilari yer."
\par 28 O zaman Isa ona su karsiligi verdi: "Ey kadin, imanin büyük! Diledigin gibi olsun." Ve kadinin kizi o saatte iyilesti.
\par 29 Isa oradan ayrildi, Celile Gölü'nün kiyisindan geçerek daga çikip oturdu.
\par 30 Yanina büyük bir kalabalik geldi. Beraberlerinde kötürüm, kör, çolak, dilsiz ve daha birçok hasta getirdiler. Hastalari O'nun ayaklarinin dibine biraktilar. O da onlari iyilestirdi.
\par 31 Halk, dilsizlerin konustugunu, çolaklarin iyilestigini, körlerin gördügünü, kötürümlerin yürüdügünü görünce sasti ve Israil'in Tanrisi'ni yüceltti.
\par 32 Isa ögrencilerini yanina çagirip, "Halka aciyorum" dedi. "Üç gündür yanimdalar, yiyecek hiçbir seyleri yok. Onlari aç aç evlerine göndermek istemiyorum, yolda bayilabilirler."
\par 33 Ögrenciler kendisine, "Böyle issiz bir yerde bu kadar kalabaligi doyuracak ekmegi nereden bulalim?" dediler.
\par 34 Isa, "Kaç ekmeginiz var?" diye sordu. "Yedi ekmekle birkaç küçük baligimiz var" dediler.
\par 35 Bunun üzerine Isa, halka yere oturmalarini buyurdu.
\par 36 Yedi ekmekle baliklari aldi, sükredip bunlari böldü, ögrencilerine verdi. Onlar da halka dagittilar.
\par 37 Herkes yiyip doydu. Artakalan parçalardan yedi küfe dolusu topladilar.
\par 38 Yemek yiyenlerin sayisi, kadin ve çocuklar hariç, dört bin erkekti.
\par 39 Isa, halki evlerine gönderdikten sonra tekneye binip Magadan bölgesine geçti.

\chapter{16}

\par 1 Ferisiler'le Sadukiler* Isa'nin yanina geldiler. O'nu denemek amaciyla kendilerine gökten bir belirti göstermesini istediler.
\par 2 Isa onlara su karsiligi verdi: "Aksam, 'Gökyüzü kizil olduguna göre hava iyi olacak' dersiniz.
\par 3 Sabah, 'Bugün gök kizil ve bulutlu, hava bozacak' dersiniz. Gökyüzünün görünümünü yorumlayabiliyorsunuz da, zamanin belirtilerini yorumlayamiyor musunuz?
\par 4 Kötü ve vefasiz kusak bir belirti istiyor! Ama ona Yunus'un belirtisinden baska bir belirti gösterilmeyecek." Sonra Isa onlari birakip gitti.
\par 5 Ögrenciler gölün karsi yakasina geçerken ekmek almayi unutmuslardi.
\par 6 Isa onlara, "Dikkatli olun, Ferisiler'in ve Sadukiler'in mayasindan kaçinin!" dedi.
\par 7 Onlar ise kendi aralarinda tartisarak, "Ekmek almadigimiz için böyle diyor" dediler.
\par 8 Bunun farkinda olan Isa söyle dedi: "Ey kit imanlilar! Ekmeginiz yok diye niçin tartisiyorsunuz?
\par 9 Hâlâ anlamiyor musunuz? Bes ekmekle bes bin kisinin doydugunu, kaç sepet dolusu yemek fazlasi topladiginizi hatirlamiyor musunuz? Yedi ekmekle dört bin kisinin doydugunu, kaç küfe dolusu yemek fazlasi topladiginizi hatirlamiyor musunuz?
\par 11 Ben size, 'Ferisiler'in ve Sadukiler'in mayasindan kaçinin' derken, ekmekten söz etmedigimi nasil olur da anlamazsiniz?"
\par 12 Ekmek mayasindan degil de, Ferisiler'le Sadukiler'in ögretisinden kaçinin dedigini o zaman anladilar.
\par 13 Isa, Filipus Sezariyesi bölgesine geldiginde ögrencilerine sunu sordu: "Halk, Insanoglu'nun* kim oldugunu söylüyor?"
\par 14 Ögrencileri su karsiligi verdiler: "Kimi Vaftizci Yahya, kimi Ilyas, kimi de Yeremya ya da peygamberlerden biridir diyor."
\par 15 Isa onlara, "Siz ne dersiniz" dedi, "Sizce ben kimim?"
\par 16 Simun Petrus, "Sen, yasayan Tanri'nin Oglu Mesih'sin*" yanitini verdi.
\par 17 Isa ona, "Ne mutlu sana, Yunus oglu Simun!" dedi. "Bu sirri sana açan insan degil, göklerdeki Babam'dir.
\par 18 Ben de sana sunu söyleyeyim, sen Petrus'sun ve ben kilisemi* bu kayanin üzerine kuracagim. Ölüler diyarinin kapilari ona karsi direnemeyecek.
\par 19 Göklerin Egemenligi'nin anahtarlarini sana verecegim. Yeryüzünde baglayacagin her sey göklerde de baglanmis olacak; yeryüzünde çözecegin her sey göklerde de çözülmüs olacak."
\par 20 Bu sözlerden sonra Isa, kendisinin Mesih oldugunu kimseye söylememeleri için ögrencilerini uyardi.
\par 21 Bundan sonra Isa, kendisinin Yerusalim'e gitmesi, ileri gelenler, baskâhinler ve din bilginlerinin elinden çok aci çekmesi, öldürülmesi ve üçüncü gün dirilmesi gerektigini ögrencilerine anlatmaya basladi.
\par 22 Bunun üzerine Petrus O'nu bir kenara çekip azarlamaya basladi. "Tanri korusun, ya Rab! Senin basina asla böyle bir sey gelmeyecek!" dedi.
\par 23 Ama Isa Petrus'a dönüp, "Çekil önümden, Seytan!" dedi, "Bana engel oluyorsun. Düsüncelerin Tanri'ya degil, insana özgüdür."
\par 24 Sonra Isa, ögrencilerine sunlari söyledi: "Ardimdan gelmek isteyen kendini inkâr etsin, çarmihini yüklenip beni izlesin.
\par 25 Canini kurtarmak isteyen onu yitirecek, canini benim ugruma yitiren ise onu kurtaracaktir.
\par 26 Insan bütün dünyayi kazanip da canindan olursa, bunun kendisine ne yarari olur? Insan kendi canina karsilik ne verebilir?
\par 27 Insanoglu*, Babasi'nin görkemi içinde melekleriyle gelecek ve herkese, yaptiginin karsiligini verecektir.
\par 28 Size dogrusunu söyleyeyim, burada bulunanlar arasinda, Insanoglu'nun kendi egemenligi içinde gelisini görmeden ölümü tatmayacak olanlar var."

\chapter{17}

\par 1 Alti gün sonra Isa, yanina yalniz Petrus, Yakup ve Yakup'un kardesi Yuhanna'yi alarak yüksek bir daga çikti.
\par 2 Onlarin gözü önünde Isa'nin görünümü degisti. Yüzü günes gibi parladi, giysileri isik gibi bembeyaz oldu.
\par 3 O anda Musa'yla Ilyas ögrencilere göründü. Isa'yla konusuyorlardi.
\par 4 Petrus Isa'ya, "Ya Rab" dedi, "Burada bulunmamiz ne iyi oldu! Istersen burada üç çardak kurayim: Biri sana, biri Musa'ya, biri de Ilyas'a."
\par 5 Petrus daha konusurken parlak bir bulut onlara gölge saldi. Buluttan gelen bir ses, "Sevgili Oglum budur, O'ndan hosnudum. O'nu dinleyin!" dedi.
\par 6 Ögrenciler bunu isitince, dehset içinde yüzüstü yere kapandilar.
\par 7 Isa gelip onlara dokundu, "Kalkin, korkmayin!" dedi.
\par 8 Baslarini kaldirinca Isa'dan baska kimseyi göremediler.
\par 9 Dagdan inerlerken Isa onlara, "Insanoglu* ölümden dirilmeden, gördüklerinizi kimseye söylemeyin" diye buyurdu.
\par 10 Ögrencileri O'na sunu sordular: "Peki, din bilginleri* neden önce Ilyas'in gelmesi gerektigini söylüyorlar?"
\par 11 Isa, "Ilyas gerçekten gelecek ve her seyi yeniden düzene koyacak" diye yanitladi.
\par 12 "Size sunu söyleyeyim, Ilyas zaten geldi, ama onu tanimadilar, ona yapmadiklarini birakmadilar. Ayni sekilde Insanoglu da onlarin elinden aci çekecektir."
\par 13 O zaman ögrenciler Isa'nin kendilerine Vaftizci Yahya'dan söz ettigini anladilar.
\par 14 Kalabaligin yanina vardiklarinda bir adam Isa'ya yaklasip önünde diz çöktü.
\par 15 "Ya Rab" dedi, "Oglumun haline aci! Sarasi var, çok aci çekiyor. Sik sik atese, suya düsüyor.
\par 16 Onu senin ögrencilerine getirdim, ama iyilestiremediler."
\par 17 Isa, "Ey imansiz ve sapmis kusak!" dedi. "Sizinle daha ne kadar kalacagim? Size daha ne kadar katlanacagim? Çocugu buraya, bana getirin."
\par 18 Isa cini azarlayinca, cin çocuktan çikti, çocuk o anda iyilesti.
\par 19 Sonra ögrenciler tek baslarina Isa'ya gelip, "Biz cini neden kovamadik?" diye sordular.
\par 20 Isa, "Imaniniz kit oldugu için" karsiligini verdi. "Size dogrusunu söyleyeyim, bir hardal tanesi kadar imaniniz olsa su daga, 'Buradan suraya göç' derseniz, göçer; sizin için imkânsiz bir sey olmayacaktir."
\par 22 Celile'de bir araya geldiklerinde Isa onlara, "Insanoglu*, insanlarin eline teslim edilecek ve öldürülecek, ama üçüncü gün dirilecek" dedi. Ögrenciler buna çok kederlendiler.
\par 24 Kefarnahum'a geldiklerinde, iki dirhemlik tapinak vergisini toplayanlar Petrus'a gelip, "Ögretmeniniz tapinak vergisini ödemiyor mu?" diye sordular.
\par 25 Petrus, "Ödüyor" dedi. Petrus eve gelince, daha kendisi bir sey söylemeden Isa ona, "Simun, ne dersin?" dedi. "Dünya krallari gümrük ya da vergiyi kimlerden alir? Kendi ogullarindan mi, yabancilardan mi?"
\par 26 Petrus'un, "Yabancilardan" demesi üzerine Isa, "O halde ogullar muaftir" dedi.
\par 27 "Ama vergi toplayanlari gücendirmeyelim. Göle gidip oltani at. Tuttugun ilk baligi çikar, onun agzini aç, dört dirhemlik bir akçe bulacaksin. Parayi al, ikimizin vergisi olarak onlara ver."

\chapter{18}

\par 1 Bu sirada ögrencileri Isa'ya yaklasip, "Göklerin Egemenligi'nde en büyük kimdir?" diye sordular.
\par 2 Isa, yanina küçük bir çocuk çagirdi, onu orta yere dikip söyle dedi: "Size dogrusunu söyleyeyim, yolunuzdan dönüp küçük çocuklar gibi olmazsaniz, Göklerin Egemenligi'ne asla giremezsiniz.
\par 4 Kim bu çocuk gibi alçakgönüllü olursa, Göklerin Egemenligi'nde en büyük odur.
\par 5 Böyle bir çocugu benim adim ugruna kabul eden, beni kabul etmis olur.
\par 6 "Ama kim bana iman eden bu küçüklerden birini günaha düsürürse, boynuna kocaman bir degirmen tasi asilip denizin dibine atilmasi kendisi için daha iyi olur.
\par 7 Insani günaha düsüren tuzaklardan ötürü vay dünyanin haline! Böyle tuzaklarin olmasi kaçinilmazdir. Ama bu tuzaklara aracilik eden kisinin vay haline!
\par 8 "Eger elin ya da ayagin günah islemene neden olursa, onu kesip at. Tek el, tek ayakla yasama kavusman, iki elle, iki ayakla sönmez atese atilmandan iyidir.
\par 9 Eger gözün günah islemene neden olursa, onu çikar at. Tek gözle yasama kavusman, iki gözle cehennem atesine atilmandan iyidir.
\par 10 "Bu küçüklerden birini bile hor görmekten sakinin! Size sunu söyleyeyim, onlarin göklerdeki melekleri, göklerdeki Babam'in yüzünü her zaman görürler."
\par 12 "Siz ne dersiniz? Bir adamin yüz koyunu olsa ve bunlardan biri yolunu sasirsa, doksan dokuzunu daglarda birakip yolunu sasirani aramaya gitmez mi?
\par 13 Size dogrusunu söyleyeyim, eger onu bulursa, yolunu sasirmamis doksan dokuz koyun için sevindiginden daha çok onun için sevinir.
\par 14 Bunun gibi, göklerdeki Babaniz da bu küçüklerden hiçbirinin kaybolmasini istemez."
\par 15 "Eger kardesin sana karsi günah islerse, ona git, suçunu kendisine göster. Her sey yalniz ikinizin arasinda kalsin. Kardesin seni dinlerse, onu kazanmis olursun.
\par 16 Ama dinlemezse, yanina bir ya da iki kisi daha al ki, söylenen her sey iki ya da üç tanigin sözüyle dogrulansin.
\par 17 Onlari da dinlemezse, durumu inanlilar topluluguna* bildir. Toplulugu da dinlemezse, onu putperest ya da vergi görevlisi* say.
\par 18 "Size dogrusunu söyleyeyim, yeryüzünde baglayacaginiz her sey gökte de baglanmis olacak. Yeryüzünde çözeceginiz her sey gökte de çözülmüs olacak.
\par 19 Yine size sunu söyleyeyim, yeryüzünde aranizdan iki kisi, dileyecekleri herhangi bir sey için anlasirlarsa, göklerdeki Babam dileklerini yerine getirir.
\par 20 Nerede iki ya da üç kisi benim adimla toplanirsa, ben de orada, aralarindayim."
\par 21 Bunun üzerine Petrus Isa'ya gelip, "Ya Rab" dedi, "Kardesim bana karsi kaç kez günah islerse onu bagislamaliyim? Yedi kez mi?"
\par 22 Isa, "Yedi kez degil" dedi. "Yetmis kere yedi kez derim sana.
\par 23 Söyle ki, Göklerin Egemenligi, köleleriyle hesaplasmak isteyen bir krala benzer.
\par 24 Kral hesap görmeye basladiginda kendisine, borcu on bin talanti bulan bir köle getirildi.
\par 25 Kölenin ödeme gücü olmadigindan efendisi onun, karisinin, çocuklarinin ve bütün malinin satilip borcun ödenmesini buyurdu.
\par 26 Köle yere kapanip efendisine, 'Ne olur, sabret! Bütün borcumu ödeyecegim' dedi.
\par 27 Efendisi köleye acidi, borcunu bagislayip onu saliverdi.
\par 28 "Ama köle çikip gitti, kendisine yüz dinar borcu olan baska bir köleye rastladi. Onu yakalayip, 'Borcunu öde' diyerek bogazina sarildi.
\par 29 Bu köle yüzüstü yere kapandi, 'Ne olur, sabret! Borcumu ödeyecegim' diye yalvardi.
\par 30 Ama ilk köle bunu reddetti. Gitti, borcunu ödeyinceye dek adami zindana kapatti.
\par 31 Öteki köleler, olanlari görünce çok üzüldüler. Efendilerine gidip bütün olup bitenleri anlattilar.
\par 32 "Bunun üzerine efendisi köleyi yanina çagirdi. 'Ey kötü köle!' dedi. 'Bana yalvardigin için bütün borcunu bagisladim.
\par 33 Benim sana acidigim gibi, senin de köle arkadasina aciman gerekmez miydi?'
\par 34 Bu öfkeyle efendisi, bütün borcunu ödeyinceye dek onu iskencecilere teslim etti.
\par 35 "Eger her biriniz kardesini gönülden bagislamazsa, göksel Babam da size öyle davranacaktir."

\chapter{19}

\par 1 Isa konusmasini bitirdikten sonra Celile'den ayrilip Yahudiye'nin* Seria Irmagi'nin karsi yakasindaki topraklarina geçti.
\par 2 Büyük halk topluluklari da O'nun ardindan gitti. Hasta olanlari orada iyilestirdi.
\par 3 Isa'nin yanina gelen bazi Ferisiler*, O'nu denemek amaciyla sunu sordular: "Bir adamin, herhangi bir nedenle karisini bosamasi Kutsal Yasa'ya uygun mudur?"
\par 4 Isa su karsiligi verdi: "Kutsal Yazilar'i okumadiniz mi? Yaradan baslangiçtan 'Insanlari erkek ve disi olarak yaratti' ve söyle dedi: 'Bu nedenle adam annesini babasini birakip karisina baglanacak, ikisi tek beden olacak.'
\par 6 Söyle ki, onlar artik iki degil, tek bedendir. O halde Tanri'nin birlestirdigini, insan ayirmasin."
\par 7 Ferisiler Isa'ya, "Öyleyse" dediler, "Musa neden erkegin bosanma belgesi verip karisini bosayabilecegini söyledi?"
\par 8 Isa onlara, "Inatçi oldugunuz için Musa karilarinizi bosamaniza izin verdi" dedi. "Baslangiçta bu böyle degildi.
\par 9 Ben size sunu söyleyeyim, karisini fuhustan baska bir nedenle bosayip baskasiyla evlenen, zina etmis olur. Bosanan kadinla evlenen de zina etmis olur."
\par 10 Ögrenciler Isa'ya, "Eger erkekle karisi arasindaki iliski buysa, hiç evlenmemek daha iyi!" dediler.
\par 11 Isa onlara, "Herkes bu sözü kabul edemez, ancak Tanri'nin güç verdigi kisiler kabul edebilir" dedi.
\par 12 "Çünkü kimisi dogustan hadimdir, kimisi insanlar tarafindan hadim edilir, kimisi de Göklerin Egemenligi ugruna kendini hadim sayar. Bunu kabul edebilen etsin!"
\par 13 O sirada bazilari küçük çocuklari Isa'nin yanina getirdiler; ellerini onlarin üzerine koyup dua etmesini istediler. Ögrenciler onlari azarlayinca Isa, "Birakin çocuklari" dedi. "Bana gelmelerine engel olmayin! Çünkü Göklerin Egemenligi böylelerinindir."
\par 15 Ellerini onlarin üzerine koyduktan sonra oradan ayrildi.
\par 16 Adamin biri Isa'ya gelip, "Ögretmenim, sonsuz yasama kavusmak için nasil bir iyilik yapmaliyim?" diye sordu.
\par 17 Isa, "Bana neden iyilik hakkinda soru soruyorsun?" dedi. "Iyi olan yalniz biri var. Yasama kavusmak istiyorsan, O'nun buyruklarini yerine getir."
\par 18 "Hangi buyruklari?" diye sordu adam. Isa su karsiligi verdi: "'Adam öldürmeyeceksin, zina etmeyeceksin, çalmayacaksin, yalan yere taniklik etmeyeceksin, annene babana saygi göstereceksin' ve 'Komsunu kendin gibi seveceksin.'"
\par 20 Genç adam, "Bunlarin hepsini yerine getirdim" dedi, "Daha ne eksigim var?"
\par 21 Isa ona, "Eger eksiksiz olmak istiyorsan, git, varini yogunu sat, parasini yoksullara ver; böylece göklerde hazinen olur. Sonra gel, beni izle" dedi.
\par 22 Genç adam bu sözleri isitince üzüntü içinde oradan uzaklasti. Çünkü çok mali vardi.
\par 23 Isa ögrencilerine, "Size dogrusunu söyleyeyim" dedi, "Zengin kisi Göklerin Egemenligi'ne zor girecek.
\par 24 Yine sunu söyleyeyim ki, devenin igne deliginden geçmesi, zenginin Tanri Egemenligi'ne girmesinden daha kolaydir."
\par 25 Bunu isiten ögrenciler büsbütün sasirdilar, "Öyleyse kim kurtulabilir?" diye sordular.
\par 26 Isa onlara bakarak, "Insanlar için bu imkânsiz, ama Tanri için her sey mümkündür" dedi.
\par 27 Bunun üzerine Petrus O'na, "Bak" dedi, "Biz her seyi birakip senin ardindan geldik, kazancimiz ne olacak?"
\par 28 Isa onlara, "Size dogrusunu söyleyeyim" dedi, "Her sey yenilendiginde, Insanoglu* görkemli tahtina oturdugunda, siz, evet ardimdan gelen sizler, on iki tahta oturup Israil'in on iki oymagini yargilayacaksiniz.
\par 29 Benim adim ugruna evlerini, kardeslerini, anne ya da babasini, çocuklarini ya da topraklarini birakan herkes, bunlarin yüz katini elde edecek ve sonsuz yasami miras alacak.
\par 30 Ne var ki, birincilerin birçogu sonuncu, sonuncularin birçogu da birinci olacak."

\chapter{20}

\par 1 "Göklerin Egemenligi, sabah erkenden baginda çalisacak isçi aramaya çikan toprak sahibine benzer.
\par 2 Adam, isçilerle günlügü bir dinara anlasip onlari bagina gönderdi.
\par 3 "Saat* dokuza dogru tekrar disari çikti, çarsi meydaninda bos duran baska adamlar gördü.
\par 4 Onlara, 'Siz de baga gidip çalisin. Hakkiniz neyse, veririm' dedi, onlar da baga gittiler. "Ögleyin ve saat üçe dogru yine çikip ayni seyi yapti.
\par 6 Saat bese dogru çikinca, orada duran baska isçiler gördü. Onlara, 'Neden bütün gün burada bos duruyorsunuz?' diye sordu.
\par 7 "'Kimse bize is vermedi ki' dediler. "Onlara, 'Siz de baga gidin, çalisin' dedi.
\par 8 "Aksam olunca, bagin sahibi kâhyasina, 'Isçileri çagir' dedi. 'Sonuncudan baslayarak ilkine kadar, hepsine ücretlerini ver.'
\par 9 "Saat bese dogru ise baslayanlar gelip kâhyadan birer dinar aldilar.
\par 10 Ilk baslayanlar gelince daha çok alacaklarini sandilar, ama onlara da birer dinar verildi.
\par 11 Paralarini alinca bag sahibine söylenmeye basladilar:
\par 12 'En son çalisanlar yalniz bir saat çalisti' dediler. 'Ama onlari günün yükünü ve sicagini çeken bizlerle bir tuttun!'
\par 13 "Bag sahibi onlardan birine söyle karsilik verdi: 'Arkadas, sana haksizlik etmiyorum ki! Seninle bir dinara anlasmadik mi?
\par 14 Hakkini al, git! Sana verdigimi sonuncuya da vermek istiyorum.
\par 15 Kendi paramla istedigimi yapmaya hakkim yok mu? Yoksa cömertligimi kiskaniyor musun?'
\par 16 "Iste böylece sonuncular birinci, birinciler de sonuncu olacak."
\par 17 Isa Yerusalim'e giderken, yolda on iki ögrencisini bir yana çekip onlara özel olarak sunu söyledi: "Simdi Yerusalim'e gidiyoruz. Insanoglu*, baskâhinlerin ve din bilginlerinin eline teslim edilecek, onlar da O'nu ölüm cezasina çarptiracaklar.
\par 19 O'nunla alay etmeleri, kamçilayip çarmiha germeleri için O'nu öteki uluslara teslim edecekler. Ne var ki O, üçüncü gün dirilecek."
\par 20 O sirada Zebedi ogullarinin annesi ogullariyla birlikte Isa'ya yaklasti. Önünde yere kapanarak kendisinden bir dilegi oldugunu söyledi.
\par 21 Isa kadina, "Ne istiyorsun?" diye sordu. Kadin, "Buyruk ver, senin egemenliginde bu iki oglumdan biri saginda, biri solunda otursun" dedi.
\par 22 "Siz ne dilediginizi bilmiyorsunuz" diye karsilik verdi Isa. "Benim içecegim kâseden* siz içebilir misiniz?" "Evet, içebiliriz" dediler.
\par 23 Isa onlara, "Elbette benim kâsemden içeceksiniz" dedi, "Ama sagimda ya da solumda oturmaniza izin vermek benim elimde degil. Babam bu yerleri belirli kisiler için hazirlamistir."
\par 24 Bunu isiten on ögrenci iki kardese kizdilar.
\par 25 Ama Isa onlari yanina çagirip söyle dedi: "Bilirsiniz ki, uluslarin önderleri onlara egemen kesilir, ileri gelenleri de agirliklarini hissettirirler.
\par 26 Sizin aranizda böyle olmayacak. Aranizda büyük olmak isteyen, ötekilerin hizmetkâri olsun.
\par 27 Aranizda birinci olmak isteyen, ötekilerin kulu olsun.
\par 28 Nitekim Insanoglu*, hizmet edilmeye degil, hizmet etmeye ve canini birçoklari için fidye olarak vermeye geldi."
\par 29 Eriha'dan ayrilirlarken büyük bir kalabalik Isa'nin ardindan gitti.
\par 30 Yol kenarinda oturan iki kör, Isa'nin oradan geçmekte oldugunu duyunca, "Ya Rab, ey Davut Oglu*, halimize aci!" diye bagirdilar.
\par 31 Kalabalik onlari azarlayarak susturmak istediyse de onlar, "Ya Rab, ey Davut Oglu, halimize aci!" diyerek daha çok bagirdilar.
\par 32 Isa durup onlari çagirdi. "Sizin için ne yapmami istiyorsunuz?" diye sordu.
\par 33 Onlar da, "Ya Rab, gözlerimiz açilsin" dediler.
\par 34 Isa onlara acidi, gözlerine dokundu. O anda yeniden görmeye basladilar ve O'nun ardindan gittiler.

\chapter{21}

\par 1 Yerusalim'e yaklasip Zeytin Dagi'nin yamacindaki Beytfaci Köyü'ne geldiklerinde Isa, iki ögrencisini önden gönderdi. Onlara, "Karsinizdaki köye gidin" dedi, "Hemen orada bagli bir disi esek ve yaninda bir sipa bulacaksiniz. Onlari çözüp bana getirin.
\par 3 Size bir sey diyen olursa, 'Rab'bin bunlara ihtiyaci var, hemen geri gönderecek' dersiniz."
\par 4 Bu olay, peygamber araciligiyla bildirilen su söz yerine gelsin diye oldu:
\par 5 "Siyon* kizina deyin ki, 'Iste, alçakgönüllü Kralin, Esege, evet sipaya, Esek yavrusuna binmis Sana geliyor.'"
\par 6 Ögrenciler gidip Isa'nin kendilerine buyurdugu gibi yaptilar.
\par 7 Esekle sipayi getirip üzerlerine giysilerini yaydilar, Isa sipaya bindi.
\par 8 Halkin büyük bir bölümü giysilerini yolun üzerine serdi. Bazilari da agaçlardan dal kesip yola seriyordu.
\par 9 Önden giden ve arkadan gelen kalabaliklar söyle bagiriyorlardi: "Davut Oglu'na hozana*! Rab'bin adiyla gelene övgüler olsun, En yücelerde hozana!"
\par 10 Isa Yerusalim'e girdigi zaman bütün kent, "Bu kimdir?" diyerek çalkandi.
\par 11 Kalabaliklar, "Bu, Celile'nin Nasira Kenti'nden Peygamber Isa'dir" diyordu.
\par 12 Isa, tapinagin avlusuna girerek oradaki bütün alici ve saticilari disari kovdu. Para bozanlarin* masalarini, güvercin satanlarin sehpalarini devirdi.
\par 13 Onlara söyle dedi: "'Evime dua evi denecek' diye yazilmistir. Ama siz onu haydut inine çevirdiniz!"
\par 14 Isa tapinaktayken kendisine gelen kör ve kötürümleri iyilestirdi.
\par 15 Ne var ki, baskâhinlerle din bilginleri, O'nun yarattigi harikalari ve tapinakta, "Davut Oglu'na hozana!" diye bagiran çocuklari görünce öfkelendiler.
\par 16 Isa'ya, "Bunlarin ne söyledigini duyuyor musun?" diye sordular. "Duyuyorum" dedi Isa. "Siz su sözü hiç okumadiniz mi? 'Küçük çocuklarin ve emziktekilerin dudaklarindan kendine övgüler döktürdün.'"
\par 17 Isa onlari birakip kentten çikti. Beytanya'ya dönüp geceyi orada geçirdi.
\par 18 Isa sabah erkenden kente dönerken acikmisti.
\par 19 Yol kenarinda gördügü bir incir agacina yaklasti. Agaçta yapraktan baska bir sey bulamayinca agaca, "Artik sonsuza dek sende meyve yetismesin!" dedi. Incir agaci o anda kurudu.
\par 20 Ögrenciler bunu görünce saskina döndüler. "Incir agaci birdenbire nasil kurudu?" diye sordular.
\par 21 Isa onlara su karsiligi verdi: "Size dogrusunu söyleyeyim, eger imaniniz olur da kusku duymazsaniz, yalniz incir agacina olani yapmakla kalmazsiniz; su daga, 'Kalk, denize atil' derseniz, dediginiz olacaktir.
\par 22 Imanla dua ederseniz, dilediginiz her seyi alirsiniz."
\par 23 Isa tapinaga girmis ögretiyordu. Bu sirada baskâhinler ve halkin ileri gelenleri O'nun yanina gelerek, "Bunlari hangi yetkiyle yapiyorsun, bu yetkiyi sana kim verdi?" diye sordular.
\par 24 Isa onlara su karsiligi verdi: "Ben de size bir soru soracagim. Bana yanit verirseniz, ben de size bunlari hangi yetkiyle yaptigimi söylerim.
\par 25 Yahya'nin vaftiz etme yetkisi nereden geldi, Tanri'dan mi, insanlardan mi?" Bunu aralarinda söyle tartismaya basladilar: "'Tanri'dan' dersek, bize, 'Öyleyse ona niçin inanmadiniz?' diyecek.
\par 26 Yok eger 'Insanlardan' dersek... Halkin tepkisinden korkuyoruz. Çünkü herkes Yahya'yi peygamber sayiyor."
\par 27 Isa'ya, "Bilmiyoruz" diye yanit verdiler. Isa, "Ben de size bunlari hangi yetkiyle yaptigimi söylemeyecegim" dedi.
\par 28 "Ama suna ne dersiniz? Bir adamin iki oglu vardi. Adam birincisine gidip, 'Oglum, git bugün bagda çalis' dedi.
\par 29 "Oglu, 'Gitmem!' dedi. Ama sonra pisman olup gitti.
\par 30 "Adam ikinci ogluna gidip ayni seyi söyledi. O, 'Olur, efendim' dedi, ama gitmedi.
\par 31 "Ikisinden hangisi babasinin istegini yerine getirmis oldu?" "Birincisi" diye karsilik verdiler. Isa da onlara, "Size dogrusunu söyleyeyim, vergi görevlileriyle fahiseler, Tanri'nin Egemenligi'ne sizden önce giriyorlar" dedi.
\par 32 "Yahya size dogruluk yolunu göstermeye geldi, ona inanmadiniz. Oysa vergi görevlileriyle fahiseler ona inandilar. Siz bunu gördükten sonra bile pisman olup ona inanmadiniz."
\par 33 "Bir benzetme daha dinleyin: Toprak sahibi bir adam, bag dikti, çevresini çitle çevirdi, üzüm sikma çukuru kazdi, bir de bekçi kulesi yapti. Sonra bagi bagcilara kiralayip yolculuga çikti.
\par 34 Bagbozumu yaklasinca, üründen kendisine düseni almalari için kölelerini bagcilara yolladi.
\par 35 Bagcilar adamin kölelerini yakaladi, birini dövdü, birini öldürdü, ötekini de tasladi.
\par 36 Bag sahibi bu kez ilkinden daha çok sayida köle yolladi. Bagcilar bunlara da ayni seyi yaptilar.
\par 37 Sonunda bag sahibi, 'Oglumu sayarlar' diyerek bagcilara onu yolladi.
\par 38 "Ama bagcilar adamin oglunu görünce birbirlerine, 'Mirasçi bu; gelin, onu öldürüp mirasina konalim' dediler.
\par 39 Böylece onu yakaladilar, bagdan atip öldürdüler.
\par 40 Bu durumda bagin sahibi geldigi zaman bagcilara ne yapacak?"
\par 41 Isa'ya su karsiligi verdiler: "Bu korkunç adamlari korkunç bir sekilde yok edecek; bagi da, ürününü kendisine zamaninda verecek olan baska bagcilara kiralayacak."
\par 42 Isa onlara sunu sordu: "Kutsal Yazilar'da su sözleri hiç okumadiniz mi? 'Yapicilarin reddettigi tas, Iste kösenin bas tasi oldu. Rab'bin isidir bu, Gözümüzde harika bir is!'
\par 43 "Bu nedenle size sunu söyleyeyim, Tanri'nin Egemenligi sizden alinacak ve bunun ürünlerini yetistiren bir ulusa verilecek.
\par 44 "Bu tasin üzerine düsen, paramparça olacak; tas da kimin üzerine düserse, onu ezip toz edecek."
\par 45 Baskâhinler ve Ferisiler, Isa'nin anlattigi benzetmeleri duyunca bunlari kendileri için söyledigini anladilar.
\par 46 O'nu tutuklamak istedilerse de, halkin tepkisinden korktular. Çünkü halk, O'nu peygamber sayiyordu.

\chapter{22}

\par 1 Isa söz alip onlara yine benzetmelerle söyle seslendi: "Göklerin Egemenligi, oglu için dügün söleni hazirlayan bir krala benzer.
\par 3 Kral sölene davet ettiklerini çagirmak üzere kölelerini gönderdi, ama davetliler gelmek istemedi.
\par 4 "Kral yine baska kölelerini gönderirken onlara dedi ki, 'Davetlilere sunu söyleyin: Bakin, ben ziyafetimi hazirladim. Sigirlarim, besili hayvanlarim kesildi. Her sey hazir, buyrun sölene!'
\par 5 "Ama davetliler aldirmadilar. Biri tarlasina, biri ticaretine gitti.
\par 6 Öbürleri de kralin kölelerini yakalayip hirpaladilar ve öldürdüler.
\par 7 Kral öfkelendi. Ordularini gönderip o katilleri yok etti, kentlerini atese verdi.
\par 8 "Sonra kölelerine söyle dedi: 'Dügün söleni hazir, ama çagirdiklarim buna layik degilmis.
\par 9 Gidin yol kavsaklarina, kimi bulursaniz dügüne çagirin.'
\par 10 Böylece köleler yollara döküldü, iyi kötü kimi buldularsa, hepsini topladilar. Dügün yeri konuklarla doldu.
\par 11 "Kral konuklari görmeye geldiginde, orada dügün giysisi giymemis bir adam gördü.
\par 12 Ona, 'Arkadas, dügün giysisi giymeden buraya nasil girdin?' diye sorunca, adamin dili tutuldu.
\par 13 "O zaman kral, usaklarina, 'Sunun ellerini ayaklarini baglayin, disariya, karanliga atin!' dedi. 'Orada aglayis ve dis gicirtisi olacaktir.'
\par 14 "Çünkü çagrilanlar çok, ama seçilenler azdir."
\par 15 Bunun üzerine Ferisiler çikip gittiler. Isa'yi, kendi söyleyecegi sözlerle tuzaga düsürmek amaciyla düzen kurdular.
\par 16 Hirodes* yanlilariyla birlikte gönderdikleri kendi ögrencileri Isa'ya gelip, "Ögretmenimiz" dediler, "Senin dürüst biri oldugunu, Tanri yolunu dürüstçe ögrettigini, kimseyi kayirmadigini biliyoruz. Çünkü insanlar arasinda ayrim yapmazsin.
\par 17 Peki, söyle bize, sence Sezar'a* vergi vermek Kutsal Yasa'ya uygun mu degil mi?"
\par 18 Isa onlarin kötü niyetlerini bildiginden, "Ey ikiyüzlüler!" dedi. "Beni neden deniyorsunuz?
\par 19 Vergi öderken kullandiginiz parayi gösterin bana!" O'na bir dinar getirdiler.
\par 20 Isa, "Bu resim, bu yazi kimin?" diye sordu.
\par 21 "Sezar'in" dediler. O zaman Isa, "Öyleyse Sezar'in hakkini Sezar'a, Tanri'nin hakkini Tanri'ya verin" dedi.
\par 22 Bu sözleri duyunca sastilar, Isa'yi birakip gittiler.
\par 23 Ölümden sonra dirilis olmadigini söyleyen Sadukiler*, ayni gün Isa'ya gelip sunu sordular: "Ögretmenimiz, Musa söyle buyurmustur: 'Eger bir adam çocuk sahibi olmadan ölürse, kardesi onun karisini alsin, soyunu sürdürsün.'
\par 25 Aramizda yedi kardes vardi. Ilki evlendi ve öldü. Çocugu olmadigindan karisini kardesine birakti.
\par 26 Ikincisi, üçüncüsü, yedincisine kadar hepsine ayni sey oldu.
\par 27 Hepsinden sonra kadin da öldü.
\par 28 Buna göre dirilis günü kadin bu yedi kardesten hangisinin karisi olacak? Çünkü hepsi de onunla evlendi."
\par 29 Isa onlara, "Siz Kutsal Yazilar'i ve Tanri'nin gücünü bilmediginiz için yaniliyorsunuz" diye karsilik verdi.
\par 30 "Dirilisten sonra insanlar ne evlenir, ne de evlendirilir, gökteki melekler gibidirler.
\par 31 Ölülerin dirilmesi konusuna gelince, Tanri'nin size bildirdigi su sözü okumadiniz mi?
\par 32 'Ben Ibrahim'in Tanrisi, Ishak'in Tanrisi ve Yakup'un Tanrisi'yim' diyor. Tanri ölülerin degil, dirilerin Tanrisi'dir."
\par 33 Bunlari isiten halk, O'nun ögretisine sasip kaldi.
\par 34 Ferisiler, Isa'nin Sadukiler'i susturdugunu duyunca bir araya toplandilar.
\par 35 Onlardan biri, bir Kutsal Yasa uzmani, Isa'yi denemek amaciyla O'na sunu sordu: "Ögretmenim, Kutsal Yasa'da en önemli buyruk hangisidir?"
\par 37 Isa ona su karsiligi verdi: "'Tanrin Rab'bi bütün yüreginle, bütün caninla ve bütün aklinla seveceksin.'
\par 38 Iste ilk ve en önemli buyruk budur.
\par 39 Ilkine benzeyen ikinci buyruk da sudur: 'Komsunu kendin gibi seveceksin.'
\par 40 Kutsal Yasa'nin tümü ve peygamberlerin sözleri bu iki buyruga dayanir."
\par 41 Ferisiler toplu haldeyken Isa onlara sunu sordu: "Mesih'le* ilgili olarak ne düsünüyorsunuz? O kimin ogludur?" Onlar da, "Davut'un Oglu" dediler.
\par 43 Isa söyle dedi: "O halde nasil oluyor da Davut, Ruh'tan esinlenerek O'ndan 'Rab' diye söz ediyor? Söyle diyor Davut:
\par 44 'Rab Rabbim'e dedi ki, Ben düsmanlarini Ayaklarinin altina serinceye dek Sagimda otur.'
\par 45 Davut O'ndan Rab diye söz ettigine göre, O nasil Davut'un Oglu olur?"
\par 46 Isa'ya hiç kimse karsilik veremedi. O günden sonra artik kimse de O'na bir sey sormaya cesaret edemedi.

\chapter{23}

\par 1 Bundan sonra Isa halka ve ögrencilerine söyle seslendi: "Din bilginleri* ve Ferisiler* Musa'nin kürsüsünde otururlar.
\par 3 Bu nedenle size söylediklerinin tümünü yapin ve yerine getirin, ama onlarin yaptiklarini yapmayin. Çünkü söyledikleri seyleri kendileri yapmazlar.
\par 4 Agir ve tasinmasi güç yükleri baglayip baskalarinin sirtina yüklerler, kendileriyse bu yükleri tasimak için parmaklarini bile oynatmak istemezler.
\par 5 "Yaptiklarinin tümünü gösteris için yaparlar. Örnegin, hamaillerini büyük, giysilerinin püsküllerini uzun yaparlar.
\par 6 Sölenlerde basköseye, havralarda en seçkin yerlere kurulmaya bayilirlar.
\par 7 Meydanlarda selamlanmaktan ve insanlarin kendilerini 'Rabbî*' diye çagirmalarindan zevk duyarlar.
\par 8 "Kimse sizi 'Rabbî' diye çagirmasin. Çünkü sizin tek ögretmeniniz var ve hepiniz kardessiniz.
\par 9 Yeryüzünde kimseye 'Baba' demeyin. Çünkü tek Babaniz var, O da göksel Baba'dir.
\par 10 Kimse sizi 'Önder' diye çagirmasin. Çünkü tek önderiniz var, O da Mesih'tir.
\par 11 Aranizda en üstün olan, ötekilerin hizmetkâri olsun.
\par 12 Kendini yücelten alçaltilacak, kendini alçaltan yüceltilecektir.
\par 13 "Vay halinize ey din bilginleri ve Ferisiler, ikiyüzlüler! Göklerin Egemenligi'nin kapisini insanlarin yüzüne kapiyorsunuz; ne kendiniz içeri giriyor, ne de girmek isteyenleri birakiyorsunuz!
\par 15 "Vay halinize ey din bilginleri ve Ferisiler, ikiyüzlüler! Tek bir kisiyi dininize döndürmek için denizleri, kitalari dolasirsiniz. Dininize döneni de kendinizden iki kat cehennemlik yaparsiniz.
\par 16 "Vay halinize kör kilavuzlar! Diyorsunuz ki, 'Tapinak üzerine ant içenin andi sayilmaz, ama tapinaktaki altin üzerine ant içen, andini yerine getirmek zorundadir.'
\par 17 Budalalar, körler! Hangisi daha önemli, altin mi, altini kutsal kilan tapinak mi?
\par 18 Yine diyorsunuz ki, 'Sunak üzerine ant içenin andi sayilmaz, ama sunaktaki adagin üzerine ant içen, andini yerine getirmek zorundadir.'
\par 19 Ey körler! Hangisi daha önemli, adak mi, adagi kutsal kilan sunak mi?
\par 20 Öyleyse sunak üzerine ant içen, hem sunagin hem de sunaktaki her seyin üzerine ant içmis olur.
\par 21 Tapinak üzerine ant içen de hem tapinak, hem de tapinakta yasayan Tanri üzerine ant içmis olur.
\par 22 Gök üzerine ant içen, Tanri'nin tahti ve tahtta oturanin üzerine ant içmis olur.
\par 23 "Vay halinize ey din bilginleri ve Ferisiler, ikiyüzlüler! Siz nanenin, dereotunun ve kimyonun ondaligini verirsiniz de, Kutsal Yasa'nin daha önemli konularini -adaleti, merhameti, sadakati- ihmal edersiniz. Ondalik vermeyi ihmal etmeden asil bunlari yerine getirmeniz gerekirdi.
\par 24 Ey kör kilavuzlar! Küçük sinegi süzer ayirir, ama deveyi yutarsiniz!
\par 25 "Vay halinize ey din bilginleri ve Ferisiler, ikiyüzlüler! Bardagin ve çanagin disini temizlersiniz, oysa bunlarin içi açgözlülük ve taskinlikla doludur.
\par 26 Ey kör Ferisi! Sen önce bardagin ve çanagin içini temizle ki, distan da temiz olsunlar.
\par 27 "Vay halinize ey din bilginleri ve Ferisiler, ikiyüzlüler! Siz distan güzel görünen, ama içi ölü kemikleri ve her türlü pislikle dolu badanali mezarlara benzersiniz.
\par 28 Distan insanlara dogru görünürsünüz, ama içte ikiyüzlülük ve kötülükle dolusunuz.
\par 29 "Vay halinize ey din bilginleri ve Ferisiler, ikiyüzlüler! Peygamberlerin mezarlarini yapar, dogru kisilerin anitlarini donatirsiniz.
\par 30 'Atalarimizin yasadigi günlerde yasasaydik, onlarla birlikte peygamberlerin kanina girmezdik' diyorsunuz.
\par 31 Böylece, peygamberleri öldürenlerin torunlari oldugunuza kendiniz taniklik ediyorsunuz.
\par 32 Haydi, atalarinizin baslattigi isi bitirin!
\par 33 "Sizi yilanlar, engerekler soyu! Cehennem cezasindan nasil kaçacaksiniz?
\par 34 Iste bunun için size peygamberler, bilge kisiler ve din bilginleri gönderiyorum. Bunlardan kimini öldürecek, çarmiha gereceksiniz. Kimini havralarinizda kamçilayacak, kentten kente kovalayacaksiniz.
\par 35 Böylelikle, dogru kisi olan Habil'in kanindan, tapinakla sunak arasinda öldürdügünüz Berekya oglu Zekeriya'nin kanina kadar, yeryüzünde akitilan her dogru kisinin kanindan sorumlu tutulacaksiniz.
\par 36 Size dogrusunu söyleyeyim, bunlarin hepsinden bu kusak sorumlu tutulacaktir.
\par 37 "Ey Yerusalim! Peygamberleri öldüren, kendisine gönderilenleri taslayan Yerusalim! Tavugun civcivlerini kanatlari altina topladigi gibi ben de kaç kez senin çocuklarini toplamak istedim, ama siz istemediniz.
\par 38 Bakin, eviniz issiz birakilacak!
\par 39 Size sunu söyleyeyim: 'Rab'bin adiyla gelene övgüler olsun!' diyeceginiz zamana dek beni bir daha görmeyeceksiniz."

\chapter{24}

\par 1 Isa tapinaktan çikip giderken, ögrencileri, tapinagin binalarini O'na göstermek için yanina geldiler.
\par 2 Isa onlara, "Bütün bunlari görüyor musunuz?" dedi. "Size dogrusunu söyleyeyim, burada tas üstünde tas kalmayacak, hepsi yikilacak!"
\par 3 Isa, Zeytin Dagi'nda otururken ögrencileri yalniz olarak yanina geldiler. "Söyle bize" dediler, "Bu dediklerin ne zaman olacak, senin gelisini ve çagin bitimini gösteren belirti ne olacak?"
\par 4 Isa onlara su karsiligi verdi: "Sakin kimse sizi saptirmasin!
\par 5 Birçoklari, 'Mesih* benim' diyerek benim adimla gelip birçok kisiyi aldatacaklar.
\par 6 Savas gürültüleri, savas haberleri duyacaksiniz. Sakin korkmayin! Bunlarin olmasi gerek, ama bu daha son demek degildir.
\par 7 Ulus ulusa, devlet devlete savas açacak; yer yer kitliklar, depremler olacak.
\par 8 Bütün bunlar, dogum sancilarinin baslangicidir.
\par 9 "O zaman sizi sikintiya sokacak, öldürecekler. Benim adimdan ötürü bütün uluslar sizden nefret edecek.
\par 10 O zaman birçok kisi imandan sapacak, birbirlerini ele verecek ve birbirlerinden nefret edecekler.
\par 11 Birçok sahte peygamber türeyecek ve bunlar birçok kisiyi saptiracak.
\par 12 Kötülüklerin çogalmasindan ötürü birçoklarinin sevgisi soguyacak.
\par 13 Ama sonuna kadar dayanan kurtulacaktir.
\par 14 Göksel egemenligin bu Müjdesi bütün uluslara taniklik olmak üzere dünyanin her yerinde duyurulacak. Iste o zaman son gelecektir.
\par 15 "Peygamber Daniel'in sözünü ettigi yikici igrenç seyin* kutsal yerde dikildigini gördügünüz zaman -okuyan anlasin- Yahudiye'de bulunanlar daglara kaçsin.
\par 17 Damda olan, evindeki esyalarini almak için asagi inmesin.
\par 18 Tarlada olan, abasini almak için geri dönmesin.
\par 19 O günlerde gebe olan, çocuk emziren kadinlarin vay haline!
\par 20 Dua edin ki, kaçisiniz kisa ya da Sabat Günü'ne* rastlamasin.
\par 21 Çünkü o günlerde öyle korkunç bir sikinti olacak ki, dünyanin baslangicindan bu yana böylesi olmamis, bundan sonra da olmayacaktir.
\par 22 O günler kisaltilmamis olsaydi, hiç kimse kurtulamazdi. Ama seçilmis olanlar ugruna o günler kisaltilacak.
\par 23 Eger o zaman biri size, 'Iste Mesih burada', ya da 'Iste surada' derse, inanmayin.
\par 24 Çünkü sahte mesihler, sahte peygamberler türeyecek; bunlar büyük belirtiler ve harikalar yapacaklar. Öyle ki, ellerinden gelse, seçilmis olanlari bile saptiracaklar.
\par 25 Iste size önceden söylüyorum.
\par 26 "Bunun için size, 'Iste Mesih çölde' derlerse gitmeyin. 'Bakin, iç odalarda' derlerse inanmayin.
\par 27 Çünkü Insanoglu'nun* gelisi, doguda çakip batiya kadar her taraftan görülen simsek gibi olacaktir.
\par 28 "Les neredeyse, akbabalar oraya üsüsecek.
\par 29 "O günlerin sikintisindan hemen sonra, 'Günes kararacak, Ay isik vermez olacak, Yildizlar gökten düsecek, Göksel güçler sarsilacak.'
\par 30 "O zaman Insanoglu'nun belirtisi gökte görünecek. Yeryüzündeki bütün halklar aglayip dövünecek, Insanoglu'nun gökteki bulutlar üzerinde büyük güç ve görkemle geldigini görecekler.
\par 31 Kendisi güçlü bir borazan sesiyle meleklerini gönderecek. Melekler O'nun seçtiklerini gögün bir ucundan öbür ucuna dek, dünyanin dört bucagindan toplayacaklar.
\par 32 "Incir agacindan ders alin! Dallari filizlenip yapraklari sürünce, yaz mevsiminin yakin oldugunu anlarsiniz.
\par 33 Ayni sekilde, bütün bunlarin gerçeklestigini gördügünüzde bilin ki, Insanoglu yakindir, kapidadir.
\par 34 Size dogrusunu söyleyeyim, bütün bunlar olmadan bu kusak ortadan kalkmayacak.
\par 35 Yer ve gök ortadan kalkacak, ama benim sözlerim asla ortadan kalkmayacaktir."
\par 36 "O günü ve saati, ne gökteki melekler, ne de Ogul bilir; Baba'dan baska kimse bilmez.
\par 37 Nuh'un günlerinde nasil olduysa, Insanoglu'nun* gelisinde de öyle olacak.
\par 38 Nuh'un gemiye bindigi güne dek, tufandan önceki günlerde insanlar yiyip içiyor, evlenip evlendiriliyorlardi.
\par 39 Tufan gelinceye, hepsini süpürüp götürünceye dek baslarina geleceklerden habersizdiler. Insanoglu'nun gelisi de öyle olacak.
\par 40 O gün tarlada bulunan iki kisiden biri alinacak, biri birakilacak.
\par 41 Degirmende bugday ögüten iki kadindan biri alinacak, biri birakilacak.
\par 42 "Bunun için uyanik kalin. Çünkü Rabbiniz'in gelecegi günü bilemezsiniz.
\par 43 Ama sunu bilin ki, ev sahibi, hirsizin gece hangi saatte gelecegini bilse, uyanik kalir, evinin soyulmasina firsat vermez.
\par 44 Bunun için siz de hazir olun! Çünkü Insanoglu beklemediginiz saatte gelecektir.
\par 45 "Efendinin, hizmetkârlarina vaktinde yiyecek vermek için baslarina atadigi güvenilir ve akilli köle kimdir?
\par 46 Efendisi eve döndügünde isinin basinda bulacagi o köleye ne mutlu!
\par 47 Size dogrusunu söyleyeyim, efendisi onu bütün malinin üzerinde yetkili kilacak.
\par 48 Ama o köle kötü olur da içinden, 'Efendim gecikiyor' der ve öteki köleleri dövmeye baslarsa, sarhoslarla birlikte yiyip içerse, efendisi, onun beklemedigi günde, ummadigi saatte gelecek, onu siddetle cezalandirip ikiyüzlülerle bir tutacak. Orada aglayis ve dis gicirtisi olacaktir."

\chapter{25}

\par 1 "O zaman Göklerin Egemenligi, kandillerini alip güveyi karsilamaya çikan on kiza benzeyecek.
\par 2 Bunlarin besi akilli, besi akilsizdi.
\par 3 Akilsizlar yanlarina kandillerini aldilar, ama yag almadilar.
\par 4 Akillilar ise, kandilleriyle birlikte kaplar içinde yag da aldilar.
\par 5 Güvey gecikince hepsini uyku basti, dalip uyudular.
\par 6 "Gece yarisi bir ses yankilandi: 'Iste güvey geliyor, onu karsilamaya çikin!'
\par 7 Bunun üzerine kizlarin hepsi kalkip kandillerini tazelediler.
\par 8 "Akilsizlar akillilara, 'Kandillerimiz sönüyor, bize yag verin!' dediler.
\par 9 "Akillilar, 'Olmaz! Hem bize hem size yetmeyebilir. En iyisi saticilara gidin, kendinize yag alin' dediler.
\par 10 "Ne var ki, onlar yag satin almaya giderlerken güvey geldi. Hazirlikli olan kizlar, onunla birlikte dügün sölenine girdiler ve kapi kapandi.
\par 11 "Daha sonra gelen öbür kizlar, 'Efendimiz, efendimiz, aç kapiyi bize!' dediler.
\par 12 "Güvey ise, 'Size dogrusunu söyleyeyim, sizi tanimiyorum' dedi.
\par 13 "Bu nedenle uyanik kalin. Çünkü o günü ve o saati bilemezsiniz."
\par 14 "Göksel egemenlik, yolculuga çikan bir adamin kölelerini çagirip malini onlara emanet etmesine benzer.
\par 15 "Adam, her birinin yetenegine göre, birine bes, birine iki, birine de bir talant vererek yola çikti.
\par 16 Bes talant alan, hemen gidip bu parayi isletti ve bes talant daha kazandi.
\par 17 Iki talant alan da iki talant daha kazandi.
\par 18 Bir talant alan ise gidip topragi kazdi ve efendisinin parasini sakladi.
\par 19 "Uzun zaman sonra bu kölelerin efendisi döndü, onlarla hesaplasmaya oturdu.
\par 20 Bes talant alan gelip bes talant daha getirdi, 'Efendimiz' dedi, 'Bana bes talant emanet etmistin; bak, bes talant daha kazandim.'
\par 21 "Efendisi ona, 'Aferin, iyi ve güvenilir köle!' dedi. 'Sen küçük islerde güvenilir oldugunu gösterdin, ben de seni büyük islerin basina geçirecegim. Gel, efendinin senligine katil!'
\par 22 "Iki talant alan da geldi, 'Efendimiz' dedi, 'Bana iki talant emanet etmistin; bak, iki talant daha kazandim.'
\par 23 "Efendisi ona, 'Aferin, iyi ve güvenilir köle!' dedi. 'Sen küçük islerde güvenilir oldugunu gösterdin, ben de seni büyük islerin basina geçirecegim. Gel, efendinin senligine katil!'
\par 24 "Sonra bir talant alan geldi, 'Efendimiz' dedi, 'Senin sert bir adam oldugunu biliyordum. Ekmedigin yerden biçer, harman savurmadigin yerden devsirirsin.
\par 25 Bu nedenle korktum, gidip senin verdigin talanti topraga gömdüm. Iste, al parani!'
\par 26 "Efendisi ona su karsiligi verdi: 'Kötü ve tembel köle! Ekmedigim yerden biçtigimi, harman savurmadigim yerden devsirdigimi bildigine göre parami faize vermeliydin. Ben de geldigimde onu faiziyle geri alirdim...
\par 28 Haydi, elindeki talanti alin, on talanti olana verin!
\par 29 Çünkü kimde varsa, ona daha çok verilecek ve o bolluk içinde olacak. Ama kimde yoksa, kendisinde olan da elinden alinacak.
\par 30 Su yararsiz köleyi disariya, karanliga atin. Orada aglayis ve dis gicirtisi olacaktir.'"
\par 31 "Insanoglu* kendi görkemi içinde bütün melekleriyle birlikte gelince, görkemli tahtina oturacak.
\par 32 Uluslarin hepsi O'nun önünde toplanacak, O da koyunlari keçilerden ayiran bir çoban gibi, insanlari birbirinden ayiracak.
\par 33 Koyunlari sagina, keçileri soluna alacak.
\par 34 "O zaman Kral, sagindaki kisilere, 'Sizler, Babam'in kutsadiklari, gelin!' diyecek. 'Dünya kuruldugundan beri sizin için hazirlanmis olan egemenligi miras alin!
\par 35 Çünkü acikmistim, bana yiyecek verdiniz; susamistim, bana içecek verdiniz; yabanciydim, beni içeri aldiniz.
\par 36 Çiplaktim, beni giydirdiniz; hastaydim, benimle ilgilendiniz; zindandaydim, yanima geldiniz.'
\par 37 "O vakit dogru kisiler O'na su karsiligi verecek: 'Ya Rab, seni ne zaman aç görüp doyurduk, susuz görüp su verdik?
\par 38 Ne zaman seni yabanci görüp içeri aldik, ya da çiplak görüp giydirdik?
\par 39 Seni ne zaman hasta ya da zindanda görüp yanina geldik?'
\par 40 "Kral da onlari söyle yanitlayacak: 'Size dogrusunu söyleyeyim, bu en basit kardeslerimden biri için yaptiginizi, benim için yapmis oldunuz.'
\par 41 "Sonra solundakilere söyle diyecek: 'Ey lanetliler, çekilin önümden! Iblis'le melekleri için hazirlanmis sönmez atese gidin!
\par 42 Çünkü acikmistim, bana yiyecek vermediniz; susamistim, bana içecek vermediniz; yabanciydim, beni içeri almadiniz; çiplaktim, beni giydirmediniz; hastaydim, zindandaydim, benimle ilgilenmediniz.'
\par 44 "O vakit onlar da söyle karsilik verecekler: 'Ya Rab, seni ne zaman aç, susuz, yabanci, çiplak, hasta ya da zindanda gördük de yardim etmedik?'
\par 45 "Kral da onlara su yaniti verecek: 'Size dogrusunu söyleyeyim, mademki bu en basit kardeslerimden biri için bunu yapmadiniz, benim için de yapmamis oldunuz.'
\par 46 "Bunlar sonsuz azaba, dogrular ise sonsuz yasama gidecekler."

\chapter{26}

\par 1 Isa bütün bunlari anlattiktan sonra ögrencilerine, "Iki gün sonra Fisih Bayrami* oldugunu biliyorsunuz" dedi, "Insanoglu* çarmiha gerilmek üzere ele verilecek."
\par 3 Bu sirada baskâhinlerle halkin ileri gelenleri, Kayafa adindaki baskâhinin sarayinda toplandilar.
\par 4 Isa'yi hileyle tutuklayip öldürmek için düzen kurdular.
\par 5 Ama, "Bayramda olmasin ki, halk arasinda kargasalik çikmasin" diyorlardi.
\par 6 Isa Beytanya'da cüzamli* Simun'un evindeyken, yanina bir kadin geldi. Kadin kaymaktasindan bir kap içinde çok degerli, güzel kokulu yag getirmisti. Isa sofrada otururken, kadin yagi O'nun basina döktü.
\par 8 Ögrenciler bunu görünce kizdilar. "Nedir bu savurganlik?" dediler.
\par 9 "Bu yag pahaliya satilabilir, parasi yoksullara verilebilirdi."
\par 10 Söylenenleri farkeden Isa, ögrencilerine, "Kadini neden üzüyorsunuz?" dedi. "Benim için güzel bir sey yapti.
\par 11 Yoksullar her zaman aranizdadir, ama ben her zaman aranizda olmayacagim.
\par 12 Kadin bu güzel kokulu yagi, beni gömülmeye hazirlamak için bedenimin üzerine bosaltti.
\par 13 Size dogrusunu söyleyeyim, bu Müjde dünyanin neresinde duyurulursa, bu kadinin yaptigi da onun anilmasi için anlatilacak."
\par 14 O sirada Onikiler'den* biri -adi Yahuda Iskariot olani baskâhinlere giderek, "O'nu ele verirsem bana ne verirsiniz?" dedi. Otuz gümüs tartip ona verdiler.
\par 16 Yahuda o andan itibaren Isa'yi ele vermek için firsat kollamaya basladi.
\par 17 Mayasiz Ekmek Bayrami'nin* ilk günü ögrenciler Isa'nin yanina gelerek, "Fisih* yemegini yemen için nerede hazirlik yapmamizi istersin?" diye sordular.
\par 18 Isa onlara, "Kente varip o adamin evine gidin" dedi. "Ona söyle deyin: 'Ögretmen diyor ki, zamanim yaklasti. Fisih Bayrami'ni, ögrencilerimle birlikte senin evinde kutlayacagim.'"
\par 19 Ögrenciler, Isa'nin buyrugunu yerine getirerek Fisih yemegi için hazirlik yaptilar.
\par 20 Aksam olunca Isa on iki ögrencisiyle yemege oturdu.
\par 21 Yemek yerlerken, "Size dogrusunu söyleyeyim, sizden biri bana ihanet edecek" dedi.
\par 22 Bu söz onlari kedere bogdu. Teker teker, "Ya Rab, beni demek istemedin ya?" diye sormaya basladilar.
\par 23 O da, "Bana ihanet edecek olan" dedi, "Elindeki ekmegi benimle birlikte sahana batirandir.
\par 24 Insanoglu*, kendisi için yazilmis oldugu gibi gidiyor, ama Insanoglu'na ihanet edenin vay haline! O adam hiç dogmamis olsaydi, kendisi için daha iyi olurdu."
\par 25 O'na ihanet edecek olan Yahuda, "Rabbî*, yoksa beni mi demek istedin?" diye sordu. Isa ona, "Söyledigin gibidir" karsiligini verdi.
\par 26 Yemek sirasinda Isa eline ekmek aldi, sükredip ekmegi böldü ve ögrencilerine verdi. "Alin, yiyin" dedi, "Bu benim bedenimdir."
\par 27 Sonra bir kâse alip sükretti ve bunu ögrencilerine vererek, "Hepiniz bundan için" dedi.
\par 28 "Çünkü bu benim kanimdir, günahlarin bagislanmasi için birçoklari ugruna akitilan antlasma kanidir.
\par 29 Size sunu söyleyeyim, Babam'in egemenliginde sizinle birlikte tazesini içecegim o güne dek, asmanin bu ürününden bir daha içmeyecegim."
\par 30 Ilahi söyledikten sonra disari çikip Zeytin Dagi'na dogru gittiler.
\par 31 Bu arada Isa ögrencilerine, "Bu gece hepiniz benden ötürü sendeleyip düseceksiniz" dedi. "Çünkü söyle yazilmistir: 'Çobani vuracagim, Sürüdeki koyunlar darmadagin olacak.'
\par 32 Ama ben dirildikten sonra sizden önce Celile'ye gidecegim."
\par 33 Petrus O'na, "Herkes senden ötürü sendeleyip düsse de ben asla düsmem" dedi.
\par 34 "Sana dogrusunu söyleyeyim" dedi Isa, "Bu gece horoz ötmeden beni üç kez inkâr edeceksin."
\par 35 Petrus, "Seninle birlikte ölmem gerekse bile seni asla inkâr etmem" dedi. Ögrencilerin hepsi de ayni seyi söyledi.
\par 36 Sonra Isa ögrencileriyle birlikte Getsemani denen yere geldi. Ögrencilerine, "Ben suraya gidip dua edecegim, siz burada oturun" dedi.
\par 37 Petrus ile Zebedi'nin iki oglunu yanina aldi. Kederlenmeye, agir bir sikinti duymaya baslamisti.
\par 38 Onlara, "Ölesiye kederliyim" dedi. "Burada kalin, benimle birlikte uyanik durun."
\par 39 Biraz ilerledi, yüzüstü yere kapanip dua etmeye basladi. "Baba" dedi, "Mümkünse bu kâse* benden uzaklastirilsin. Yine de benim degil, senin istedigin olsun."
\par 40 Ögrencilerin yanina döndügünde onlari uyumus buldu. Petrus'a, "Demek ki benimle birlikte bir saat uyanik kalamadiniz!" dedi.
\par 41 "Uyanik durup dua edin ki, ayartilmayasiniz. Ruh isteklidir, ama beden güçsüzdür."
\par 42 Isa ikinci kez uzaklasip dua etti. "Baba" dedi, "Eger ben içmeden bu kâsenin uzaklastirilmasi mümkün degilse, senin istedigin olsun."
\par 43 Geri geldiginde ögrencilerini yine uyumus buldu. Onlarin göz kapaklarina agirlik çökmüstü.
\par 44 Onlari birakip tekrar uzaklasti, yine ayni sözlerle üçüncü kez dua etti.
\par 45 Sonra ögrencilerin yanina dönerek, "Hâlâ uyuyor, dinleniyor musunuz?" dedi. "Iste saat yaklasti, Insanoglu* günahkârlarin eline veriliyor.
\par 46 Kalkin, gidelim. Iste bana ihanet eden geldi!"
\par 47 Isa daha konusurken, Onikiler'den* biri olan Yahuda geldi. Yaninda, baskâhinlerle halkin ileri gelenleri tarafindan gönderilmis kiliçli sopali büyük bir kalabalik vardi.
\par 48 Isa'ya ihanet eden Yahuda, "Kimi öpersem, Isa O'dur, O'nu tutuklayin" diye onlarla sözlesmisti.
\par 49 Dosdogru Isa'ya gidip, "Selam, Rabbî*!" diyerek O'nu öptü.
\par 50 Isa, "Arkadas, ne yapacaksan yap!" dedi. Bunun üzerine adamlar yaklasti, Isa'yi yakalayip tutukladilar.
\par 51 Isa'yla birlikte olanlardan biri, ani bir hareketle kilicini çekti, baskâhinin kölesine vurup kulagini uçurdu.
\par 52 O zaman Isa ona, "Kilicini yerine koy!" dedi. "Kiliç çekenlerin hepsi kiliçla ölecek.
\par 53 Yoksa Babam'dan yardim isteyemez miyim saniyorsun? Istesem, hemen su an bana on iki tümenden* fazla melek gönderir.
\par 54 Ama böyle olmasi gerektigini bildiren Kutsal Yazilar o zaman nasil yerine gelir?"
\par 55 Bundan sonra Isa kalabaliga dönüp söyle seslendi: "Niçin bir haydutmusum gibi beni kiliç ve sopalarla yakalamaya geldiniz? Her gün tapinakta oturup ögretiyordum, beni tutuklamadiniz.
\par 56 Ama bütün bunlar, peygamberlerin yazdiklari yerine gelsin diye oldu." O zaman ögrencilerin hepsi O'nu birakip kaçti.
\par 57 Isa'yi tutuklayanlar, O'nu baskâhin Kayafa'ya götürdüler. Din bilginleriyle ileri gelenler de orada toplanmislardi.
\par 58 Petrus, Isa'yi uzaktan, ta baskâhinin avlusuna kadar izledi. Sonucu görmek için içeri girip nöbetçilerin yanina oturdu.
\par 59 Baskâhinlerle Yüksek Kurul'un* öteki üyeleri, Isa'yi ölüm cezasina çarptirmak için kendisine karsi yalanci taniklar ariyorlardi.
\par 60 Ortaya birçok yalanci tanik çiktigi halde, aradiklarini bulamadilar. Sonunda ortaya çikan iki kisi söyle dedi: "Bu adam, 'Ben Tanri'nin Tapinagi'ni yikip üç günde yeniden kurabilirim' dedi."
\par 62 Baskâhin ayaga kalkip Isa'ya, "Hiç yanit vermeyecek misin?" dedi. "Nedir bunlarin sana karsi ettigi bu tanikliklar?"
\par 63 Isa susmaya devam etti. Baskâhin ise O'na, "Yasayan Tanri adina ant içmeni buyuruyorum, söyle bize, Tanri'nin Oglu Mesih* sen misin?" dedi.
\par 64 Isa, "Söyledigin gibidir" karsiligini verdi. "Üstelik size sunu söyleyeyim, bundan sonra Insanoglu'nun*, Kudretli Olan'in saginda oturdugunu ve gögün bulutlari üzerinde geldigini göreceksiniz."
\par 65 Bunun üzerine baskâhin giysilerini yirtarak, "Tanri'ya küfretti!" dedi. "Artik taniklara ne ihtiyacimiz var? Iste küfürü isittiniz.
\par 66 Buna ne diyorsunuz?" "Ölümü hak etti!" diye karsilik verdiler.
\par 67 Bunun üzerine Isa'nin yüzüne tükürüp O'nu yumrukladilar. Bazilari da O'nu tokatlayip, "Ey Mesih, peygamberligini göster bakalim, sana vuran kim?" dediler.
\par 69 Petrus ise disarida, avluda oturuyordu. Bir hizmetçi kiz yanina gelip, "Sen de Celileli Isa'yla birlikteydin" dedi.
\par 70 Ama Petrus bunu herkesin önünde inkâr ederek, "Neden söz ettigini anlamiyorum" dedi.
\par 71 Sonra avlu kapisinin önüne çikti. Onu gören baska bir hizmetçi kiz orada bulunanlara, "Bu adam Nasirali Isa'yla birlikteydi" dedi.
\par 72 Petrus ant içerek, "Ben o adami tanimiyorum" diye yine inkâr etti.
\par 73 Orada duranlar az sonra Petrus'a yaklasip, "Gerçekten sen de onlardansin. Konusman seni ele veriyor" dediler.
\par 74 Petrus kendine lanet okuyup ant içerek, "O adami tanimiyorum!" dedi. Tam o anda horoz öttü.
\par 75 Petrus, Isa'nin, "Horoz ötmeden beni üç kez inkâr edeceksin" dedigini hatirladi ve disari çikip aci aci agladi.

\chapter{27}

\par 1 Sabah olunca bütün baskâhinlerle halkin ileri gelenleri, Isa'yi ölüm cezasina çarptirmak konusunda anlastilar.
\par 2 O'nu bagladilar ve götürüp Vali Pilatus'a teslim ettiler.
\par 3 Isa'ya ihanet eden Yahuda, O'nun mahkûm edildigini görünce yaptigina pisman oldu. Otuz gümüsü baskâhinlere ve ileri gelenlere geri götürdü.
\par 4 "Ben suçsuz birini ele vermekle günah isledim" dedi. Onlar ise, "Bundan bize ne? Onu sen düsün" dediler.
\par 5 Yahuda paralari tapinagin içine firlatarak oradan ayrildi, gidip kendini asti.
\par 6 Paralari toplayan baskâhinler, "Kan bedeli olan bu paralari tapinagin hazinesine koymak dogru olmaz" dediler.
\par 7 Kendi aralarinda anlasarak bu parayla yabancilar için mezarlik yapmak üzere Çömlekçi Tarlasi'ni satin aldilar.
\par 8 Bunun için bu tarlaya bugüne dek "Kan Tarlasi" denilmistir.
\par 9 Böylece Peygamber Yeremya araciligiyla bildirilen su söz yerine gelmis oldu: "Israilogullari'ndan kimilerinin O'na biçtikleri degerin karsiligi olan Otuz gümüsü aldilar; Rab'bin bana buyurdugu gibi, Çömlekçi Tarlasi'ni satin almak için harcadilar."
\par 11 Isa valinin önüne çikarildi. Vali O'na, "Sen Yahudiler'in Krali misin?" diye sordu. Isa, "Söyledigin gibidir" dedi.
\par 12 Baskâhinlerle ileri gelenler O'nu suçlayinca hiç karsilik vermedi.
\par 13 Pilatus O'na, "Senin aleyhinde yaptiklari bunca tanikligi duymuyor musun?" dedi.
\par 14 Isa tek konuda bile ona yanit vermedi. Vali buna çok sasti.
\par 15 Her Fisih Bayrami'nda* vali, halkin istedigi bir tutukluyu salivermeyi adet edinmisti.
\par 16 O günlerde Barabba adinda ünlü bir tutuklu vardi.
\par 17 Halk bir araya toplandiginda, Pilatus onlara, "Sizin için kimi salivermemi istersiniz, Barabba'yi mi, Mesih* denen Isa'yi mi?" diye sordu.
\par 18 Isa'yi kiskançliktan ötürü kendisine teslim ettiklerini biliyordu.
\par 19 Pilatus yargi kürsüsünde otururken karisi ona, "O dogru adama dokunma. Dün gece rüyamda O'nun yüzünden çok sikinti çektim" diye haber gönderdi.
\par 20 Baskâhinler ve ileri gelenler ise, Barabba'nin saliverilmesini ve Isa'nin öldürülmesini istesinler diye halki kiskirttilar.
\par 21 Vali onlara sunu sordu: "Sizin için hangisini salivermemi istersiniz?" "Barabba'yi" dediler.
\par 22 Pilatus, "Öyleyse Mesih denen Isa'yi ne yapayim?" diye sordu. Hep bir agizdan, "Çarmiha gerilsin!" dediler.
\par 23 Pilatus, "O ne kötülük yapti ki?" diye sordu. Onlar ise daha yüksek sesle, "Çarmiha gerilsin!" diye bagrisip durdular.
\par 24 Pilatus, elinden bir sey gelmedigini, tersine, bir kargasaligin basladigini görünce su aldi, kalabaligin önünde ellerini yikayip söyle dedi: "Bu adamin kanindan ben sorumlu degilim. Bu ise siz bakin!"
\par 25 Bütün halk su karsiligi verdi: "O'nun kaninin sorumlulugu bizim ve çocuklarimizin üzerinde olsun!"
\par 26 Bunun üzerine Pilatus onlar için Barabba'yi saliverdi. Isa'yi ise kamçilattiktan sonra çarmiha gerilmek üzere askerlere teslim etti.
\par 27 Sonra valinin askerleri Isa'yi vali konagina götürüp bütün taburu basina topladilar.
\par 28 O'nu soyup üzerine kirmizi bir kaftan geçirdiler.
\par 29 Dikenlerden bir taç örüp basina koydular, sag eline de bir kamis tutturdular. Önünde diz çöküp, "Selam, ey Yahudiler'in Krali!" diyerek O'nunla alay ettiler.
\par 30 Üzerine tükürdüler, kamisi alip basina vurdular.
\par 31 O'nunla böyle alay ettikten sonra kaftani üzerinden çikarip kendi giysilerini giydirdiler ve çarmiha germeye götürdüler.
\par 32 Disari çiktiklarinda Simun adinda Kireneli bir adama rastladilar. Isa'nin çarmihini ona zorla tasittilar.
\par 33 Golgota, yani Kafatasi denilen yere vardiklarinda içmesi için Isa'ya ödle karisik sarap verdiler. Isa bunu tadinca içmek istemedi.
\par 35 Askerler O'nu çarmiha gerdikten sonra kura çekerek giysilerini aralarinda paylastilar.
\par 36 Sonra oturup yaninda nöbet tuttular.
\par 37 Basinin üzerine, BU, YAHUDILER'IN KRALI ISA'DIR diye yazan bir suç yaftasi astilar.
\par 38 Isa'yla birlikte, biri saginda öbürü solunda olmak üzere iki haydut da çarmiha gerildi.
\par 39 Oradan geçenler baslarini sallayip Isa'ya sövüyor, "Hani sen tapinagi yikip üç günde yeniden kuracaktin? Haydi, kurtar kendini! Tanri'nin Oglu'ysan çarmihtan in!" diyorlardi.
\par 41 Baskâhinler, din bilginleri ve ileri gelenler de ayni sekilde O'nunla alay ederek, "Baskalarini kurtardi, kendini kurtaramiyor" diyorlardi. "Israil'in Krali imis! Simdi çarmihtan asagi insin de O'na iman edelim.
\par 43 Tanri'ya güveniyordu; Tanri O'nu seviyorsa, kurtarsin bakalim! Çünkü, 'Ben Tanri'nin Oglu'yum' demisti."
\par 44 Isa'yla birlikte çarmiha gerilen haydutlar da O'na ayni sekilde hakaret ettiler.
\par 45 Ögleyin on ikiden üçe kadar bütün ülkenin üzerine karanlik çöktü.
\par 46 Saat* üçe dogru Isa yüksek sesle, "Eli, Eli, lema sevaktani?" yani, "Tanrim, Tanrim, beni neden terk ettin?" diye bagirdi.
\par 47 Orada duranlardan bazilari bunu isitince, "Bu adam Ilyas'i çagiriyor" dediler.
\par 48 Içlerinden biri hemen kosup bir sünger getirdi, eksi saraba batirip bir kamisin ucuna takarak Isa'ya içirdi.
\par 49 Öbürleri ise, "Dur bakalim, Ilyas gelip O'nu kurtaracak mi?" dediler.
\par 50 Isa, yüksek sesle bir kez daha bagirdi ve ruhunu teslim etti.
\par 51 O anda tapinaktaki perde* yukaridan asagiya yirtilarak ikiye bölündü. Yer sarsildi, kayalar yarildi.
\par 52 Mezarlar* açildi, ölmüs olan birçok kutsal kisinin cesetleri dirildi.
\par 53 Bunlar mezarlarindan çikip Isa'nin dirilisinden sonra kutsal kente* girdiler ve birçok kimseye göründüler.
\par 54 Isa'yi bekleyen yüzbasi ve beraberindeki askerler, depremi ve öbür olaylari görünce dehsete kapildilar, "Bu gerçekten Tanri'nin Oglu'ydu!" dediler.
\par 55 Orada, olup bitenleri uzaktan izleyen birçok kadin vardi. Bunlar, Celile'den Isa'nin ardindan gelip O'na hizmet etmislerdi.
\par 56 Aralarinda Mecdelli Meryem, Yakup ile Yusuf'un annesi Meryem ve Zebedi ogullarinin annesi de vardi.
\par 57 Aksama dogru Yusuf adinda zengin bir Aramatyali geldi. O da Isa'nin bir ögrencisiydi.
\par 58 Pilatus'a gidip Isa'nin cesedini istedi. Pilatus da cesedin ona verilmesini buyurdu.
\par 59 Yusuf cesedi aldi, temiz keten beze sardi, kayaya oydurdugu kendi yeni mezarina yatirdi. Mezarin girisine büyük bir tas yuvarlayip oradan ayrildi.
\par 61 Mecdelli Meryem ile öteki Meryem ise orada, mezarin karsisinda oturuyorlardi.
\par 62 Ertesi gün, yani Hazirlik Günü'nden* sonraki gün, baskâhinlerle Ferisiler Pilatus'un önünde toplanarak, "Efendimiz" dediler, "O aldaticinin, daha yasarken, 'Ben öldükten üç gün sonra dirilecegim' dedigini hatirliyoruz.
\par 64 Onun için buyruk ver de üçüncü güne dek mezari güvenlik altina alsinlar. Yoksa ögrencileri gelir, cesedini çalar ve halka, 'Ölümden dirildi' derler. Son aldatmaca ilkinden beter olur."
\par 65 Pilatus onlara, "Yaniniza asker alin, gidip mezari dilediginiz gibi güvenlik altina alin" dedi.
\par 66 Onlar da askerlerle birlikte gittiler, tasi mühürleyip mezari güvenlik altina aldilar.

\chapter{28}

\par 1 Sabat Günü'nü* izleyen haftanin ilk günü*, tan yeri agarirken, Mecdelli Meryem ile öbür Meryem mezari* görmeye gittiler.
\par 2 Ansizin büyük bir deprem oldu. Rab'bin bir melegi gökten indi ve mezara gidip tasi bir yana yuvarlayarak üzerine oturdu.
\par 3 Görünüsü simsek gibi, giysileri ise kar gibi bembeyazdi.
\par 4 Nöbetçiler korkudan titremeye basladilar, sonra ölü gibi yere yikildilar.
\par 5 Melek kadinlara söyle seslendi: "Korkmayin! Çarmiha gerilen Isa'yi aradiginizi biliyorum.
\par 6 O burada yok; söylemis oldugu gibi dirildi. Gelin, O'nun yattigi yeri görün.
\par 7 Çabuk gidin, ögrencilerine söyle deyin: 'Isa ölümden dirildi. Sizden önce Celile'ye gidiyor, kendisini orada göreceksiniz.' Iste ben size söylemis bulunuyorum."
\par 8 Kadinlar korku ve büyük sevinç içinde hemen mezardan uzaklastilar; kosarak Isa'nin ögrencilerine haber vermeye gittiler.
\par 9 Isa ansizin karsilarina çikti, "Selam!" dedi. Yaklasip Isa'nin ayaklarina sarilarak O'na tapindilar.
\par 10 O zaman Isa, "Korkmayin!" dedi. "Gidip kardeslerime haber verin, Celile'ye gitsinler, beni orada görecekler."
\par 11 Kadinlar daha yoldayken nöbetçi askerlerden bazilari kente giderek olup bitenleri baskâhinlere bildirdiler.
\par 12 Baskâhinler ileri gelenlerle birlikte toplanip birbirlerine danistiktan sonra askerlere yüklü para vererek dediler ki, "Siz söyle diyeceksiniz: 'Ögrencileri geceleyin geldi, biz uyurken O'nun cesedini çalip götürdüler.'
\par 14 Eger bu haber valinin kulagina gidecek olursa biz onu yatistirir, size bir zarar gelmesini önleriz."
\par 15 Böylece askerler parayi aldilar ve kendilerine söylendigi gibi yaptilar. Bu söylenti Yahudiler arasinda bugün de yaygindir.
\par 16 On bir ögrenci Celile'ye, Isa'nin kendilerine bildirdigi daga gittiler.
\par 17 Isa'yi gördükleri zaman O'na tapindilar. Ama bazilari kusku içindeydi.
\par 18 Isa yanlarina gelip kendilerine sunlari söyledi: "Gökte ve yeryüzünde bütün yetki bana verildi.
\par 19 Bu nedenle gidin, bütün uluslari ögrencilerim olarak yetistirin; onlari Baba, Ogul ve Kutsal Ruh'un adiyla vaftiz edin;
\par 20 size buyurdugum her seye uymayi onlara ögretin. Iste ben, dünyanin sonuna dek her an sizinle birlikteyim."


\end{document}