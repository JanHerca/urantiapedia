\begin{document}

\title{Romalılar}


\chapter{1}

\par 1 Isa Mesih'in kulu, Tanri'nin Müjdesi'ni yaymak üzere seçilip elçi olmaya çagrilan ben Pavlus'tan selam!
\par 2 Tanri, Oglu Rabbimiz Isa Mesih'le ilgili bu Müjde'yi peygamberleri araciligiyla Kutsal Yazilar'da önceden vaat etti. Rabbimiz Isa Mesih beden açisindan Davut'un soyundandir; kutsallik ruhu açisindan ise ölümden dirilmekle Tanri'nin Oglu oldugu kudretle ilan edildi.
\par 5 Her ulustan insanin iman edip söz dinlemesini saglamak için Mesih araciligiyla ve O'nun adi ugruna Tanri lütfuna ve elçilik görevine sahip olduk.
\par 6 Isa Mesih'in çagrilmislari olan sizler de bu uluslardansiniz.
\par 7 Tanri'nin Roma'da bulunan, kutsal olmaya çagrilan bütün sevdiklerine, Babamiz Tanri'dan ve Rab Isa Mesih'ten size lütuf ve esenlik olsun.
\par 8 Ilkin hepiniz için Isa Mesih araciligiyla Tanrim'a sükrediyorum. Çünkü imaniniz bütün dünyada duyuruluyor.
\par 9 Oglu'nun Müjdesi'ni yaymakta bütün varligimla kulluk ettigim Tanri, sizi durmadan, her zaman dualarimda andigima taniktir. Tanri'nin istegiyle sonunda bir yol bulup yaniniza gelmek için dua ediyorum.
\par 11 Çünkü ruhça pekismeniz için size ruhsal bir armagan ulastirmak üzere sizi görmeyi çok istiyorum.
\par 12 Yani, ben aranizdayken karsilikli olarak birbirimizin imaniyla cesaret buluruz demek istiyorum.
\par 13 Kardesler, öteki uluslar arasinda oldugu gibi, çalismalarimin sizin aranizda da ürün vermesi için yaniniza gelmeyi birçok kez amaçladigimi, ama simdiye dek hep engellendigimi bilmenizi istiyorum.
\par 14 Grekler'e* ve Grek olmayanlara, bilgelere ve bilgisizlere karsi sorumlulugum var.
\par 15 Bu nedenle Roma'da bulunan sizlere de Müjde'yi elimden geldigince bildirmek için sabirsizlaniyorum.
\par 16 Çünkü Müjde'den utanmiyorum. Müjde iman eden herkesin -önce Yahudiler'in, sonra Yahudi olmayanlarin- kurtulusu için Tanri gücüdür.
\par 17 Tanri'nin insani akladigi, Müjde'de açiklanir. Aklanma yalniz imanla olur. Yazilmis oldugu gibi, "Imanla aklanan yasayacaktir."
\par 18 Haksizlikla gerçege engel olan insanlarin bütün tanrisizligina ve haksizligina karsi Tanri'nin gazabi gökten açikça gösterilmektedir.
\par 19 Çünkü Tanri'ya iliskin bilinen ne varsa, gözlerinin önündedir; Tanri hepsini gözlerinin önüne sermistir.
\par 20 Tanri'nin görünmeyen nitelikleri -sonsuz gücü ve Tanriligi- dünya yaratilali beri O'nun yaptiklariyla anlasilmakta, açikça görülmektedir. Bu nedenle özürleri yoktur.
\par 21 Tanri'yi bildikleri halde O'nu Tanri olarak yüceltmediler, O'na sükretmediler. Tersine, düsüncelerinde budalaliga düstüler; anlayissiz yüreklerini karanlik bürüdü.
\par 22 Akilli olduklarini ileri sürerken akilsiz olup çiktilar.
\par 23 Ölümsüz Tanri'nin yüceligi yerine ölümlü insana, kuslara, dört ayaklilara, sürüngenlere benzeyen putlari yeglediler.
\par 24 Bu yüzden Tanri, birbirlerinin bedenlerini asagilasinlar diye, onlari yüreklerinin tutkulari içinde ahlaksizliga teslim etti.
\par 25 Tanri'yla ilgili gerçegin yerine yalani koydular. Yaradan'in yerine yaratiga tapip kulluk ettiler. Oysa Tanri sonsuza dek övülmeye layiktir! Amin.
\par 26 Iste böylece Tanri onlari utanç verici tutkulara teslim etti. Kadinlari bile dogal iliski yerine dogal olmayani yeglediler.
\par 27 Ayni sekilde erkekler de kadinla dogal iliskilerini birakip birbirleri için sehvetle yanip tutustular. Erkekler erkeklerle utanç verici iliskilere girdiler ve kendi bedenlerinde sapikliklarina yarasan karsiligi aldilar.
\par 28 Tanri'yi tanimakta yarar görmedikleri için Tanri onlari yararsiz düsüncelere, yakisiksiz davranislara teslim etti.
\par 29 Her türlü haksizlik, kötülük, açgözlülük ve kinle doldular. Kiskançlik, öldürme hirsi, çekisme, hile, kötü niyetle doludurlar.
\par 30 Dedikoducu, yerici, Tanri'dan nefret eden, küstah, kibirli, övüngen, kötülük üreten, anne baba sözü dinlemeyen, anlayissiz, sözünde durmaz, sevgiden yoksun, acimasiz insanlardir.
\par 32 Böyle davrananlarin ölümü hak ettigine iliskin Tanri buyrugunu bildikleri halde, bunlari yalniz yapmakla kalmaz, yapanlari da onaylarlar.

\chapter{2}

\par 1 Bu nedenle sen, ey baskasini yargilayan insan, kim olursan ol, özrün yoktur. Baskasini yargiladigin konuda kendini mahkûm ediyorsun. Çünkü ey yargilayan sen, ayni seyleri yapiyorsun.
\par 2 Böyle davrananlari Tanri'nin hakli olarak yargiladigini biliriz.
\par 3 Bu gibi seyleri yapanlari yargilayan, ama aynisini yapan ey insan, Tanri'nin yargisindan kaçabilecegini mi saniyorsun?
\par 4 Tanri'nin sinirsiz iyiligini, hosgörüsünü, sabrini hor mu görüyorsun? O'nun iyiliginin seni tövbeye yönelttigini bilmiyor musun?
\par 5 Inatçiligin ve tövbesiz yüregin yüzünden Tanri'nin adil yargisinin açiklanacagi gazap günü için kendine karsi gazap biriktiriyorsun.
\par 6 Tanri "herkese, yaptiklarinin karsiligini verecektir."
\par 7 Sürekli iyilik ederek yücelik, sayginlik, ölümsüzlük arayanlara sonsuz yasam verecek.
\par 8 Bencillerin, gerçege uymayip haksizlik pesinden gidenlerin üzerineyse gazap ve öfke yagdiracak.
\par 9 Kötülük eden herkese -önce Yahudi'ye, sonra Yahudi olmayana- sikinti ve elem verecek; iyilik eden herkese -yine önce Yahudi'ye, sonra Yahudi olmayana- yücelik, sayginlik, esenlik verecektir.
\par 11 Çünkü Tanri insanlar arasinda ayrim yapmaz.
\par 12 Kutsal Yasa'yi* bilmeden günah isleyenler Yasa olmadan da mahvolacaklar. Yasa'yi bilerek günah isleyenlerse Yasa'yla yargilanacaklar.
\par 13 Çünkü Tanri katinda aklanacak olanlar Yasa'yi isitenler degil, yerine getirenlerdir.
\par 14 Kutsal Yasa'dan yoksun uluslar Yasa'nin gereklerini kendiliklerinden yaptikça, Yasa'dan habersiz olsalar bile kendi yasalarini koymus olurlar.
\par 15 Böylelikle Kutsal Yasa'nin gerektirdiklerinin yüreklerinde yazili oldugunu gösterirler. Vicdanlari buna taniklik eder. Düsünceleriyse onlari ya suçlar ya da savunur.
\par 16 Yaydigim Müjde'ye göre Tanri'nin, insanlari gizlice yaptiklari seylerden ötürü Isa Mesih araciligiyla yargilayacagi gün böyle olacaktir.
\par 17 Ya sen? Kendine Yahudi diyor, Kutsal Yasa'ya dayanip Tanri'yla övünüyorsun.
\par 18 Tanri'nin istegini biliyorsun. En üstün degerleri ayirt etmeyi Yasa'dan ögrenmissin.
\par 19 Kutsal Yasa'da bilginin ve gerçegin özüne kavusmus olarak körlerin kilavuzu, karanlikta kalanlarin isigi, akilsizlarin egiticisi, çocuklarin ögretmeni olduguna inanmissin.
\par 21 Öyleyse baskasina ögretirken, kendine de ögretmez misin? Çalmamayi ögütlerken, çalar misin?
\par 22 "Zina etmeyin" derken, zina eder misin? Putlardan tiksinirken, tapinaklari yagmalar misin?
\par 23 Kutsal Yasa'yla övünürken, Yasa'ya karsi gelerek Tanri'yi asagilar misin?
\par 24 Nitekim söyle yazilmistir: "Sizin yüzünüzden uluslar arasinda Tanri'nin adina küfrediliyor."
\par 25 Kutsal Yasa'yi yerine getirirsen, sünnetin elbet yarari vardir. Ama Yasa'ya karsi gelirsen, sünnetli olmanin hiçbir anlami kalmaz.
\par 26 Bu nedenle, sünnetsizler* Yasa'nin buyruklarina uyarsa, sünnetli sayilmayacak mi?
\par 27 Sen Kutsal Yazilar'a ve sünnete sahip oldugun halde Yasa'yi çignersen, bedence sünnetli olmayan ama Yasa'ya uyan kisi seni yargilamayacak mi?
\par 28 Çünkü ne distan Yahudi olan gerçek Yahudi'dir, ne de görünüste, bedensel olan sünnet gerçek sünnettir.
\par 29 Ancak içten Yahudi olan Yahudi'dir. Sünnet de yürekle ilgilidir; yazili yasanin degil, Ruh'un isidir. Içten Yahudi olan kisi, insanlarin degil, Tanri'nin övgüsünü kazanir.

\chapter{3}

\par 1 Öyleyse Yahudi'nin ne üstünlügü var? Sünnetin yarari nedir?
\par 2 Her yönden çoktur. Ilk olarak, Tanri'nin sözleri Yahudiler'e emanet edilmistir.
\par 3 Peki, kimi Yahudiler güvenilmez çikmissa ne olur? Onlarin güvenilmezligi Tanri'nin güvenilirligini ortadan kaldirir mi?
\par 4 Kesinlikle hayir! Herkes yalanci olsa bile, Tanri'nin dogruyu söyledigi bilinmelidir. Yazilmis oldugu gibi: "Öyle ki, sözlerinde dogru çikasin Ve yargilandiginda davayi kazanasin."
\par 5 Ama bizim haksizligimiz Tanri'nin adil oldugunu ortaya çikariyorsa, ne diyelim? Insanlarin diliyle konusuyorum: Gazapla cezalandiran Tanri haksiz mi?
\par 6 Kesinlikle hayir! Öyle olsa Tanri dünyayi nasil yargilayacak?
\par 7 Ama Tanri'nin her zaman dogruyu söyledigi benim yalanimla yüceligi için daha açik sekilde ortaya çikmissa, ben niçin yine bir günahkâr olarak yargilaniyorum?
\par 8 Bazilarinin bizi kötüleyerek, söyledigimizi ileri sürdügü gibi niçin, "Kötülük yapalim da bundan iyilik çiksin" demeyelim? Böylelerinin yargilanmasi yerindedir.
\par 9 Simdi ne diyelim? Biz Yahudiler öteki uluslardan üstün müyüz? Elbette degiliz. Ister Yahudi ister Grek* olsun, daha önce herkesi günahin denetiminde olmakla suçladik.
\par 10 Yazilmis oldugu gibi: "Dogru kimse yok, tek kisi bile yok.
\par 11 Anlayan kimse yok, Tanri'yi arayan yok.
\par 12 Hepsi sapti, Tümü yararsiz oldu. Iyilik eden yok, tek kisi bile!"
\par 13 "Agizlari açik birer mezardir. Dilleriyle aldatirlar." "Engerek zehiri var dudaklarinin altinda."
\par 14 "Agizlari lanet ve aci sözle doludur."
\par 15 "Ayaklari kan dökmeye segirtir.
\par 16 Yikim ve dert var yollarinda.
\par 17 Esenlik yolunu da bilmezler."
\par 18 "Tanri korkusu yoktur onlarda."
\par 19 Kutsal Yasa'da söylenenlerin her agiz kapansin, bütün dünya Tanri'ya hesap versin diye Yasa'nin yönetimi altindakilere söylendigini biliyoruz.
\par 20 Bu nedenle Yasa'nin gereklerini yapmakla hiç kimse Tanri katinda aklanmayacaktir. Çünkü Yasa sayesinde günahin bilincine varilir.
\par 21 Ama simdi Yasa'dan bagimsiz olarak Tanri'nin insani nasil aklayacagi açiklandi. Yasa ve peygamberler buna taniklik ediyor.
\par 22 Tanri insanlari Isa Mesih'e olan imanlariyla aklar. Bunu, iman eden herkes için yapar. Hiç ayrim yoktur.
\par 23 Çünkü herkes günah isledi ve Tanri'nin yüceliginden yoksun kaldi.
\par 24 Insanlar Isa Mesih'te olan kurtulusla, Tanri'nin lütfuyla, karsiliksiz olarak aklanirlar.
\par 25 Tanri Mesih'i, kaniyla günahlari bagislatan ve imanla benimsenen kurban olarak sundu. Böylece adaletini gösterdi. Çünkü sabredip daha önce islenmis günahlari cezasiz birakti. Bunu, adil kalmak ve Isa'ya iman edeni aklamak için simdiki zamanda kendi adaletini göstermek amaciyla yapti.
\par 27 Öyleyse neyle övünebiliriz? Hiçbir seyle! Hangi ilkeye dayanarak? Yasa'yi yerine getirme ilkesine mi? Hayir, iman ilkesine.
\par 28 Çünkü insanin, Yasa'nin gereklerini yaparak degil, iman ederek aklandigi kanisindayiz.
\par 29 Yoksa Tanri yalniz Yahudiler'in Tanrisi mi? Öteki uluslarin da Tanrisi degil mi? Elbet öteki uluslarin da Tanrisi'dir.
\par 30 Çünkü sünnetlileri* imanlari sayesinde, sünnetsizleri* de ayni imanla aklayacak olan Tanri tektir.
\par 31 Öyleyse biz iman araciligiyla Kutsal Yasa'yi geçersiz mi kiliyoruz? Hayir, tam tersine, Yasa'yi dogruluyoruz.

\chapter{4}

\par 1 Su halde soyumuzun atasi Ibrahim'in durumu için ne diyelim?
\par 2 Eger Ibrahim yaptigi iyi islerden dolayi aklandiysa, övünmeye hakki vardir; ama Tanri'nin önünde degil.
\par 3 Kutsal Yazi ne diyor? "Ibrahim Tanri'ya iman etti, böylece aklanmis sayildi."
\par 4 Çalisana verilen ücret lütuf degil, hak sayilir.
\par 5 Ancak çalismayan, ama tanrisizi aklayana iman eden kisi imani sayesinde aklanmis sayilir.
\par 6 Nitekim, iyi islerine bakmaksizin Tanri'nin aklanmis saydigi kisinin mutlulugunu Davut da söyle anlatir:
\par 7 "Ne mutlu suçlari bagislanmis, Günahlari örtülmüs olanlara!
\par 8 Günahi Rab tarafindan sayilmayana ne mutlu!"
\par 9 Bu mutluluk yalniz sünnetliler* için mi, yoksa ayni zamanda sünnetsizler* için midir? Diyoruz ki, "Ibrahim, imani sayesinde aklanmis sayildi."
\par 10 Hangi durumda aklanmis sayildi? Sünnet olduktan sonra mi, sünnetsizken mi? Sünnetliyken degil, sünnetsizken...
\par 11 Ibrahim daha sünnetsizken imanla aklandiginin kaniti olarak sünnet isaretini aldi. Öyle ki, sünnetsiz olduklari halde iman edenlerin hepsinin babasi olsun, böylece onlar da aklanmis sayilsin.
\par 12 Böylelikle atamiz Ibrahim, yalniz sünnetli olmakla kalmayan, ama kendisi sünnetsizken sahip oldugu imanin izinden yürüyen sünnetlilerin de babasi oldu.
\par 13 Çünkü Ibrahim'e ve soyuna dünyanin mirasçisi olma vaadi Kutsal Yasa yoluyla degil, imandan gelen aklanma yoluyla verildi.
\par 14 Eger Yasa'ya bagli olanlar mirasçi olursa, iman bos ve vaat geçersizdir.
\par 15 Yasa, Tanri'nin gazabina yol açar. Ama yasanin olmadigi yerde yasaya karsi gelmek de söz konusu degildir.
\par 16 Bu nedenle vaat, Tanri'nin lütfuna dayanmak ve Ibrahim'in bütün soyu için güvence altina alinmak üzere imana bagli kilinmistir. Ibrahim'in soyu yalniz Kutsal Yasa'ya bagli olanlar degil, ayni zamanda Ibrahim'in imanina sahip olanlardir. "Seni birçok ulusun babasi yaptim" diye yazilmis oldugu gibi Ibrahim, iman ettigi Tanri'nin -ölülere yasam veren, var olmayani buyruguyla var eden Tanri'nin- gözünde hepimizin babasidir.
\par 18 Ibrahim umutsuz bir durumdayken birçok ulusun babasi olacagina umutla iman etti. "Senin soyun böyle olacak" sözüne güveniyordu.
\par 19 Yüz yasina yaklasmisken, ölü denebilecek bedenini ve Sara'nin ölü rahmini düsündügünde imani zayiflamadi.
\par 20 Imansizlik edip Tanri'nin vaadinden kuskulanmadi; tersine, imani güçlendi ve Tanri'yi yüceltti.
\par 21 Tanri'nin vaadini yerine getirecek güçte olduguna tümüyle güvendi.
\par 22 Bunun için de aklanmis sayildi.
\par 23 "Aklanmis sayildi" sözü, yalniz onun için degil, aklanmis sayilacak olan bizler -Rabbimiz Isa'yi ölümden dirilten Tanri'ya iman eden bizler- için de yazildi.
\par 25 Isa suçlarimiz için ölüme teslim edildi ve aklanmamiz için diriltildi.

\chapter{5}

\par 1 Böylece imanla aklandigimiza göre, Rabbimiz Isa Mesih sayesinde Tanri'yla barismis oluyoruz.
\par 2 Içinde bulundugumuz bu lütfa Mesih araciligiyla, imanla kavustuk ve Tanri'nin yüceligine erismek umuduyla övünüyoruz.
\par 3 Yalniz bununla degil, sikintilarla da övünüyoruz. Çünkü biliyoruz ki, sikinti dayanma gücünü, dayanma gücü Tanri'nin begenisini, Tanri'nin begenisi de umudu yaratir.
\par 5 Umut düs kirikligina ugratmaz. Çünkü bize verilen Kutsal Ruh araciligiyla Tanri'nin sevgisi yüreklerimize dökülmüstür.
\par 6 Evet, biz daha çaresizken Mesih belirlenen zamanda tanrisizlar için öldü.
\par 7 Bir kimse dogru insan için güç ölür, ama iyi insan için belki biri ölmeyi göze alabilir.
\par 8 Tanri ise bizi sevdigini sununla kanitliyor: Biz daha günahkârken, Mesih bizim için öldü.
\par 9 Böylece simdi O'nun kaniyla aklandigimiza göre, O'nun araciligiyla Tanri'nin gazabindan kurtulacagimiz çok daha kesindir.
\par 10 Çünkü biz Tanri'nin düsmanlariyken Oglu'nun ölümü sayesinde O'nunla baristiksa, barismis olarak Oglu'nun yasamiyla kurtulacagimiz çok daha kesindir.
\par 11 Yalniz bu kadar da degil, bizi simdi Tanri'yla baristirmis olan Rabbimiz Isa Mesih araciligiyla, Tanri'nin kendisiyle de övünüyoruz.
\par 12 Günah bir insan araciligiyla, ölüm de günah araciligiyla dünyaya girdi. Böylece ölüm bütün insanlara yayildi. Çünkü hepsi günah isledi.
\par 13 Kutsal Yasa'dan önce de dünyada günah vardi; ama yasa olmayinca günahin hesabi tutulmaz.
\par 14 Oysa ölüm Adem'den Musa'ya dek, gelecek Kisi'nin örnegi olan Adem'in suçuna benzer bir günah islememis olanlar üzerinde de egemendi.
\par 15 Ne var ki, Tanri'nin armagani Adem'in suçu gibi degildir. Çünkü bir kisinin suçu yüzünden birçoklari öldüyse, Tanri'nin lütfu ve bir tek adamin, yani Isa Mesih'in lütfuyla verilen bagis birçoklari yararina daha da çogaldi.
\par 16 Tanri'nin bagisi o tek adamin günahinin sonucu gibi degildir. Tek suçtan sonra verilen yargi mahkûmiyet getirdi; oysa birçok suçtan sonra verilen armagan aklanmayi sagladi.
\par 17 Çünkü ölüm bir tek adamin suçu yüzünden o tek adam araciligiyla egemenlik sürdüyse, Tanri'nin bol lütfunu ve aklanma bagisini alanlarin bir tek adam, yani Isa Mesih sayesinde yasamda egemenlik sürecekleri çok daha kesindir.
\par 18 Iste, tek bir suçun bütün insanlarin mahkûmiyetine yol açtigi gibi, bir dogruluk eylemi de bütün insanlara yasam veren aklanmayi sagladi.
\par 19 Çünkü bir adamin sözdinlemezligi yüzünden nasil birçogu günahkâr kilindiysa, bir adamin söz dinlemesiyle birçogu da dogru kilinacaktir.
\par 20 Kutsal Yasa suç çogalsin diye araya girdi; ama günahin çogaldigi yerde Tanri'nin lütfu daha da çogaldi.
\par 21 Öyle ki, günah nasil ölüm yoluyla egemenlik sürdüyse, Tanri'nin lütfu da Rabbimiz Isa Mesih araciligiyla sonsuz yasam vermek üzere dogrulukla egemenlik sürsün.

\chapter{6}

\par 1 Öyleyse ne diyelim? Lütuf çogalsin diye günah islemeye devam mi edelim?
\par 2 Kesinlikle hayir! Günah karsisinda ölmüs olan bizler artik nasil günah içinde yasariz?
\par 3 Mesih Isa'ya vaftiz* edildigimizde, hepimizin O'nun ölümüne vaftiz edildigimizi bilmez misiniz?
\par 4 Baba'nin yüceligi sayesinde Mesih nasil ölümden dirildiyse, biz de yeni bir yasam sürmek üzere vaftiz yoluyla O'nunla birlikte ölüme gömüldük.
\par 5 Eger O'nunkine benzer bir ölümde O'nunla birlestiysek, O'nunkine benzer bir diriliste de O'nunla birlesecegiz.
\par 6 Artik günaha kölelik etmeyelim diye, günahli varligimizin ortadan kaldirilmasi için eski yaradilisimizin Mesih'le birlikte çarmiha gerildigini biliriz.
\par 7 Çünkü ölmüs kisi günahtan özgür kilinmistir.
\par 8 Mesih'le birlikte ölmüssek, O'nunla birlikte yasayacagimiza da inaniyoruz.
\par 9 Çünkü Mesih'in ölümden dirilmis oldugunu ve bir daha ölmeyecegini, ölümün artik O'nun üzerinde egemenlik sürmeyecegini biliyoruz.
\par 10 O'nun ölümü günaha karsilik ilk ve son ölüm olmustur. Sürmekte oldugu yasami ise Tanri için sürmektedir.
\par 11 Siz de böylece kendinizi günah karsisinda ölü, Mesih Isa'da Tanri karsisinda diri sayin.
\par 12 Bu nedenle bedenin tutkularina uymamak için günahin ölümlü bedenlerinizde egemenlik sürmesine izin vermeyin.
\par 13 Bedeninizin üyelerini haksizliga araç ederek günaha sunmayin. Ölümden dirilenler gibi kendinizi Tanri'ya adayin; bedeninizin üyelerini dogruluk araçlari olarak Tanri'ya sunun.
\par 14 Günah size egemen olmayacaktir. Çünkü Kutsal Yasa'nin yönetimi altinda degil, Tanri'nin lütfu altindasiniz.
\par 15 Öyleyse ne diyelim? Yasa'nin yönetimi altinda degil de, Tanri'nin lütfu altinda oldugumuz için günah mi isleyelim? Kesinlikle hayir!
\par 16 Söz dinleyen köleler gibi kendinizi kime teslim ederseniz, sözünü dinlediginiz kisinin köleleri oldugunuzu bilmez misiniz? Ya ölüme götüren günahin ya da dogruluga götüren sözdinlerligin kölelerisiniz.
\par 17 Ama sükürler olsun Tanri'ya! Eskiden günahin köleleri olan sizler, adandiginiz ögretinin özüne yürekten baglandiniz.
\par 18 Günahtan özgür kilinarak dogrulugun köleleri oldunuz.
\par 19 Doganizin güçsüzlügü yüzünden insan ölçülerine göre konusuyorum. Bedeninizin üyelerini ahlaksizliga ve kötülük yapmak üzere kötülüge nasil köle olarak sundunuzsa, simdi de bu üyelerinizi kutsal olmak üzere dogruluga köle olarak sunun.
\par 20 Sizler günahin kölesiyken dogruluktan özgürdünüz.
\par 21 Simdi utandiginiz seylerden o zaman ne kazanciniz oldu? Onlarin sonucu ölümdür.
\par 22 Ama simdi günahtan özgür kilinip Tanri'nin kullari oldugunuza göre, kazanciniz kutsallasma ve bunun sonucu olan sonsuz yasamdir.
\par 23 Çünkü günahin ücreti ölüm, Tanri'nin armagani ise Rabbimiz Mesih Isa'da sonsuz yasamdir.

\chapter{7}

\par 1 Bilmez misiniz ki, ey kardesler -Kutsal Yasa'yi bilenlere söylüyorum- Yasa insana ancak yasadigi sürece egemendir?
\par 2 Örnegin, evli kadin, kocasi yasadikça yasayla ona baglidir; kocasi ölürse, onu kocasina baglayan yasadan özgür olur.
\par 3 Buna göre kadin, kocasi yasarken baska bir erkekle iliski kurarsa, zina etmis sayilir. Ama kocasi ölürse, kadin yasadan özgür olur. Söyle ki, baska bir erkege varirsa, zina etmis olmaz.
\par 4 Ayni sekilde kardeslerim, siz de bir baskasina -ölümden dirilmis olan Mesih'e- varmak üzere Mesih'in bedeni araciligiyla Kutsal Yasa karsisinda öldünüz. Bu da Tanri'nin hizmetinde verimli olmamiz içindir.
\par 5 Çünkü biz benligin denetimindeyken, Yasa'nin kiskirttigi günah tutkulari bedenimizin üyelerinde etkindi. Bunun sonucu olarak ölüme götüren meyveler verdik.
\par 6 Simdiyse biz, daha önce tutsagi oldugumuz Yasa karsisinda öldügümüz için Yasa'dan özgür kilindik. Öyle ki, yazili yasanin eski yolunda degil, Ruh'un yeni yolunda kulluk edelim.
\par 7 Öyleyse ne diyelim? Kutsal Yasa günah mi oldu? Kesinlikle hayir! Ama Yasa olmasaydi, günahin ne oldugunu bilemezdim. Yasa, "Göz dikmeyeceksin" demeseydi, baskasinin malina göz dikmenin ne oldugunu bilemezdim.
\par 8 Ne var ki günah, bu buyrugun verdigi firsatla içimde her türlü açgözlülügü üretti. Çünkü Kutsal Yasa olmadikça günah ölüdür.
\par 9 Bir zamanlar, Yasa'nin bilincinde degilken diriydim. Ama buyrugun bilincine vardigimda günah dirildi, bense öldüm. Buyruk da bana yasam getirecegine, ölüm getirdi.
\par 11 Çünkü günah buyrugun verdigi firsatla beni aldatti, buyruk araciligiyla beni öldürdü.
\par 12 Iste böyle, Yasa gerçekten kutsaldir. Buyruk da kutsal, dogru ve iyidir.
\par 13 Öyleyse, iyi olan bana ölüm mü getirdi? Kesinlikle hayir! Ama günah, günah olarak taninsin diye, iyi olanin araciligiyla bana ölüm getiriyordu. Öyle ki, buyruk araciligiyla günahin ne denli günahli oldugu anlasilsin.
\par 14 Yasa'nin ruhsal oldugunu biliriz. Bense benligin denetimindeyim, köle gibi günaha satilmisim.
\par 15 Ne yaptigimi anlamiyorum. Çünkü istedigimi yapmiyorum; nefret ettigim ne ise, onu yapiyorum.
\par 16 Ama istemedigimi yaparsam, Yasa'nin iyi oldugunu kabul etmis olurum.
\par 17 Öyleyse bunu artik ben degil, içimde yasayan günah yapiyor.
\par 18 Içimde, yani benligimde iyi bir sey bulunmadigini biliyorum. Içimde iyiyi yapmaya istek var, ama güç yok.
\par 19 Istedigim iyi seyi yapmiyorum, istemedigim kötü seyi yapiyorum.
\par 20 Istemedigimi yapiyorsam, bunu yapan artik ben degil, içimde yasayan günahtir.
\par 21 Bundan su kurali çikariyorum: Ben iyi olani yapmak isterken, karsimda hep kötülük vardir.
\par 22 Iç varligimda Tanri'nin Yasasi'ndan zevk aliyorum.
\par 23 Ama bedenimin üyelerinde bambaska bir yasa görüyorum. Bu da aklimin onayladigi yasaya karsi savasiyor ve beni bedenimin üyelerindeki günah yasasina tutsak ediyor.
\par 24 Ne zavalli insanim! Ölüme götüren bu bedenden beni kim kurtaracak?
\par 25 Rabbimiz Isa Mesih araciligiyla Tanri'ya sükürler olsun! Sonuç olarak ben aklimla Tanri'nin Yasasi'na, ama benligimle günahin yasasina kulluk ediyorum.

\chapter{8}

\par 1 Böylece Mesih Isa'ya ait olanlara artik hiçbir mahkûmiyet yoktur.
\par 2 Çünkü yasam veren Ruh'un yasasi, Mesih Isa sayesinde beni günahin ve ölümün yasasindan özgür kildi.
\par 3 Insan benliginden ötürü güçsüz olan Kutsal Yasa'nin yapamadigini Tanri yapti. Öz Oglu'nu günahli insan benzerliginde günah sunusu* olarak gönderip günahi insan benliginde yargiladi.
\par 4 Öyle ki, Yasa'nin geregi, benlige göre degil, Ruh'a göre yasayan bizlerde yerine gelsin.
\par 5 Benlige uyanlar benlikle ilgili, Ruh'a uyanlarsa Ruh'la ilgili isleri düsünürler.
\par 6 Benlige dayanan düsünce ölüm, Ruh'a dayanan düsünceyse yasam ve esenliktir.
\par 7 Çünkü benlige dayanan düsünce Tanri'ya düsmandir; Tanri'nin Yasasi'na boyun egmez, egemez de...
\par 8 Benligin denetiminde olanlar Tanri'yi hosnut edemezler.
\par 9 Ne var ki, Tanri'nin Ruhu içinizde yasiyorsa, benligin degil, Ruh'un denetimindesiniz. Ama içinde Mesih'in Ruhu olmayan kisi Mesih'in degildir.
\par 10 Eger Mesih içinizdeyse, bedeniniz günah yüzünden ölü olmakla birlikte, aklanmis oldugunuz için ruhunuz diridir.
\par 11 Mesih Isa'yi ölümden dirilten Tanri'nin Ruhu içinizde yasiyorsa, Mesih'i ölümden dirilten Tanri, içinizde yasayan Ruhu'yla ölümlü bedenlerinize de yasam verecektir.
\par 12 Öyleyse kardeslerim, borçluyuz ama, benlige göre yasamak için benlige borçlu degiliz.
\par 13 Çünkü benlige göre yasarsaniz öleceksiniz; ama bedenin kötü islerini Ruh'la öldürürseniz yasayacaksiniz.
\par 14 Tanri'nin Ruhu'yla yönetilenlerin hepsi Tanri'nin ogullaridir.
\par 15 Çünkü sizi yeniden korkuya sürükleyecek kölelik ruhunu almadiniz, ogulluk ruhunu aldiniz. Bu ruhla, "Abba, Baba!" diye sesleniriz.
\par 16 Ruh'un kendisi, bizim ruhumuzla birlikte, Tanri'nin çocuklari oldugumuza taniklik eder.
\par 17 Eger Tanri'nin çocuklariysak, ayni zamanda mirasçiyiz. Mesih'le birlikte yüceltilmek üzere Mesih'le birlikte aci çekiyorsak, Tanri'nin mirasçilariyiz, Mesih'le ortak mirasçilariz.
\par 18 Kanim su ki, bu anin acilari, gözümüzün önüne serilecek yücelikle karsilastirilmaya degmez.
\par 19 Yaratilis, Tanri çocuklarinin ortaya çikmasini büyük özlemle bekliyor.
\par 20 Çünkü yaratilis amaçsizliga teslim edildi. Bu da yaratilisin istegiyle degil, onu amaçsizliga teslim eden Tanri'nin istegiyle oldu. Çünkü yaratilisin, yozlasmaya köle olmaktan kurtarilip Tanri çocuklarinin yüce özgürlügüne kavusturulmasi umudu vardi.
\par 22 Bütün yaratilisin su ana dek birlikte inleyip dogum agrisi çektigini biliyoruz.
\par 23 Yalniz yaratilis degil, biz de -evet Ruh'un turfandasina sahip olan bizler de- evlatliga alinmayi, yani bedenlerimizin kurtulmasini özlemle bekleyerek içimizden inliyoruz.
\par 24 Çünkü bu umutla kurtulduk. Ama görülen umut, umut degildir. Gördügü seyi kim umut eder?
\par 25 Oysa görmedigimize umut baglarsak, sabirla bekleyebiliriz.
\par 26 Bunun gibi, Ruh da güçsüzlügümüzde bize yardim eder. Ne için dua etmemiz gerektigini bilmeyiz, ama Ruh'un kendisi, sözle anlatilamaz iniltilerle bizim için aracilik eder.
\par 27 Yürekleri arastiran Tanri, Ruh'un düsüncesinin ne oldugunu bilir. Çünkü Ruh, Tanri'nin istegi uyarinca kutsallar için aracilik eder.
\par 28 Tanri'nin, kendisini sevenlerle, amaci uyarinca çagrilmis olanlarla birlikte her durumda iyilik için etkin oldugunu biliriz.
\par 29 Çünkü Tanri önceden bildigi kisileri Oglu'nun benzerligine dönüstürmek üzere önceden belirledi. Öyle ki, Ogul birçok kardes arasinda ilk dogan olsun.
\par 30 Tanri önceden belirledigi kisileri çagirdi, çagirdiklarini akladi ve akladiklarini yüceltti.
\par 31 Öyleyse buna ne diyelim? Tanri bizden yanaysa, kim bize karsi olabilir?
\par 32 Öz Oglu'nu bile esirgemeyip O'nu hepimiz için ölüme teslim eden Tanri, O'nunla birlikte bize her seyi bagislamayacak mi?
\par 33 Tanri'nin seçtiklerini kim suçlayacak? Onlari aklayan Tanri'dir.
\par 34 Kim suçlu çikaracak? Ölmüs, üstelik dirilmis olan Mesih Isa, Tanri'nin sagindadir ve bizim için aracilik etmektedir.
\par 35 Mesih'in sevgisinden bizi kim ayirabilir? Sikinti mi, elem mi, zulüm mü, açlik mi, çiplaklik mi, tehlike mi, kiliç mi?
\par 36 Yazilmis oldugu gibi: "Senin ugruna bütün gün öldürülüyoruz, Kasaplik koyun sayiliyoruz."
\par 37 Ama bizi sevenin araciligiyla bu durumlarin hepsinde galiplerden üstünüz.
\par 38 Eminim ki, ne ölüm, ne yasam, ne melekler, ne yönetimler, ne simdiki ne gelecek zaman, ne güçler, ne yükseklik, ne derinlik, ne de yaratilmis baska bir sey bizi Rabbimiz Mesih Isa'da olan Tanri sevgisinden ayirmaya yetecektir.

\chapter{9}

\par 1 Mesih'e ait biri olarak gerçegi söylüyorum, yalan söylemiyorum. Vicdanim da söylediklerimi Kutsal Ruh araciligiyla dogruluyor.
\par 2 Yüregimde büyük bir keder, dinmeyen bir aci var.
\par 3 Kardeslerimin, soydaslarim olan Israilliler'in yerine ben kendim lanetlenip Mesih'ten uzaklastirilmayi dilerdim. Evlatliga kabul edilenler, Tanri'nin yüceligini görenler onlardir. Antlasmalar, buyrulan Kutsal Yasa, tapinma düzeni, vaatler onlarindir.
\par 5 Büyük atalar onlarin atalaridir. Mesih de bedence onlardandir. O her seyin üzerinde hüküm süren, sonsuza dek övülecek Tanri'dir! Amin.
\par 6 Tanri'nin sözü bosa çikti demek istemiyorum. Çünkü Israil soyundan gelenlerin hepsi Israilli sayilmaz.
\par 7 Ibrahim'in soyundan olsalar bile, hepsi onun çocuklari degildir. Ama, "Senin soyun Ishak'la sürecek" diye yazilmistir.
\par 8 Demek ki Tanri'nin çocuklari olagan yoldan dogan çocuklar degildir; Ibrahim'in soyu sayilanlar Tanri'nin vaadi uyarinca dogan çocuklardir.
\par 9 Çünkü vaat söyleydi: "Gelecek yil bu zamanda gelecegim ve Sara'nin bir oglu olacak."
\par 10 Ayrica Rebeka bir erkekten, atamiz Ishak'tan ikizlere gebe kalmisti.
\par 11 Çocuklar henüz dogmamis, iyi ya da kötü bir sey yapmamisken, Tanri Rebeka'ya, "Büyügü küçügüne kulluk edecek" dedi. Öyle ki, Tanri'nin seçim yapmaktaki amaci yapilan islere degil, kendi çagrisina dayanarak sürsün.
\par 13 Yazilmis oldugu gibi, "Yakup'u sevdim, Esav'dan ise nefret ettim."
\par 14 Öyleyse ne diyelim? Tanri adaletsizlik mi ediyor? Kesinlikle hayir!
\par 15 Çünkü Musa'ya söyle diyor: "Merhamet ettigime merhamet edecegim, Acidigima aciyacagim."
\par 16 Demek ki bu, insanin istegine ya da çabasina degil, Tanri'nin merhametine baglidir.
\par 17 Tanri Kutsal Yazi'da firavuna söyle diyor: "Gücümü senin araciliginla göstermek Ve adimi bütün dünyada duyurmak için Seni yükselttim."
\par 18 Demek ki Tanri diledigine merhamet eder, dilediginin yüregini nasirlastirir.
\par 19 Simdi bana, "Öyleyse Tanri insani neden hâlâ suçlu buluyor? O'nun istegine kim karsi durabilir?" diyeceksin.
\par 20 Ama, ey insan, sen kimsin ki Tanri'ya karsilik veriyorsun? "Kendisine biçim verilen, biçim verene, 'Beni niçin böyle yaptin' der mi?"
\par 21 Ya da çömlekçinin ayni kil yiginindan bir kabi onurlu is için, ötekini bayagi is için yapmaya hakki yok mu?
\par 22 Eger Tanri gazabini göstermek ve gücünü tanitmak isterken, gazabina hedef olup mahvolmaya hazirlananlara büyük sabirla katlandiysa, ne diyelim?
\par 23 Yüceltmek üzere önceden hazirlayip merhamet ettiklerine yüceliginin zenginligini göstermek için bunu yaptiysa, ne diyelim?
\par 24 Yalniz Yahudiler arasindan degil, öteki uluslar arasindan da çagirdigi bu insanlar biziz.
\par 25 Tanri Hosea Kitabi'nda söyle diyor: "Halkim olmayana halkim, Sevgili olmayana sevgili diyecegim."
\par 26 "Kendilerine, 'Siz halkim degilsiniz' denilen yerde, Yasayan Tanri'nin çocuklari diye adlandirilacaklar."
\par 27 Yesaya, Israil için söyle sesleniyor: "Israilogullari'nin sayisi Denizin kumu kadar çok olsa da, Ancak pek azi kurtulacak.
\par 28 Çünkü Rab yeryüzündeki yargilama isini Tez yapip bitirecek."
\par 29 Yesaya'nin önceden dedigi gibi: "Her Seye Egemen Rab Soyumuzu sürdürecek birkaç kisiyi Sag birakmamis olsaydi, Sodom gibi olur, Gomora'ya benzerdik."
\par 30 Öyleyse ne diyelim? Aklanma pesinde olmayan uluslar aklanmaya, imandan gelen aklanmaya kavustular.
\par 31 Aklanmak için Yasa'nin ardindan giden Israil ise Yasa'yi yerine getiremedi.
\par 32 Neden? Çünkü imanla degil, iyi islerle olurmus gibi aklanmaya çalistilar ve "sürçme tasi"nda sürçtüler.
\par 33 Yazilmis oldugu gibi: "Iste, Siyon'a* bir sürçme tasi, Bir tökezleme kayasi koyuyorum. O'na iman eden utandirilmayacak."

\chapter{10}

\par 1 Kardesler! Israilliler'in kurtulmasini yürekten özlüyor, bunun için Tanri'ya yalvariyorum.
\par 2 Onlara iliskin taniklik ederim ki, Tanri için gayretlidirler; ama bu bilinçli bir gayret degildir.
\par 3 Tanri'nin öngördügü dogrulugu anlamadiklari ve kendi dogruluklarini yerlestirmeye çalistiklari için Tanri'nin öngördügü dogruluga boyun egmediler.
\par 4 Oysa her iman edenin aklanmasi için Mesih, Kutsal Yasa'nin sonudur.
\par 5 Musa, Kutsal Yasa'ya dayanan dogrulukla ilgili söyle yaziyor: "Yasa'nin gereklerini yapan, onlar sayesinde yasayacaktir."
\par 6 Imana dayanan dogruluk ise söyle diyor: "Yüreginde, 'Göge -yani Mesih'i indirmeye- kim çikacak?' ya da, 'Dipsiz derinliklere -yani Mesih'i ölüler arasindan çikarmaya- kim inecek?' deme."
\par 8 Ne deniyor? "Tanri sözü sana yakindir, Agzinda ve yüregindedir." Iste duyurdugumuz iman sözü budur.
\par 9 Isa'nin Rab oldugunu agzinla açikça söyler ve Tanri'nin O'nu ölümden dirilttigine yürekten iman edersen, kurtulacaksin.
\par 10 Çünkü insan yürekten iman ederek aklanir, imanini agziyla açiklayarak kurtulur.
\par 11 Kutsal Yazi, "O'na iman eden utandirilmayacak" diyor.
\par 12 Çünkü Yahudi Grek* ayrimi yoktur, ayni Rab hepsinin Rabbi'dir. Kendisine yakaranlarin tümüne eliaçiktir.
\par 13 "Rab'be yakaran herkes kurtulacak."
\par 14 Ama iman etmedikleri kisiye nasil yakaracaklar? Duymadiklari kisiye nasil iman edecekler? Tanri sözünü yayan olmazsa, nasil duyacaklar?
\par 15 Sözü yaymaya gönderilmezlerse, sözü nasil yayacaklar? Yazilmis oldugu gibi: "Iyi haber müjdeleyenlerin ayaklari ne güzeldir!"
\par 16 Ne var ki, herkes Müjde'ye uymadi. Yesaya'nin dedigi gibi: "Ya Rab, verdigimiz habere kim inandi?"
\par 17 Demek ki iman, haberi duymakla, duymak da Mesih'le ilgili sözün yayilmasiyla olur.
\par 18 Ama soruyorum: Onlar duymadilar mi? Elbet duydular. "Sesleri bütün yeryüzüne, Sözleri dünyanin dört bucagina ulasti."
\par 19 Yine soruyorum: Israil anlamadi mi? Önce Musa, "Ben sizi ulus olmayanla kiskandiracagim, Anlayissiz bir ulusla sizi öfkelendirecegim" diyor.
\par 20 Sonra Yesaya cesaretle, "Aramayanlar beni buldu, Sormayanlara kendimi gösterdim" diyor.
\par 21 Öte yandan Israil için söyle diyor: "Söz dinlemeyen, asi bir halka Bütün gün ellerimi uzatip durdum."

\chapter{11}

\par 1 Öyleyse soruyorum: Tanri kendi halkindan yüz mü çevirdi? Kesinlikle hayir! Ben de Ibrahim soyundan, Benyamin oymagindan bir Israilli'yim.
\par 2 Tanri önceden bildigi kendi halkindan yüz çevirmedi. Yoksa Ilyas'la ilgili bölümde Kutsal Yazi'nin ne dedigini, Ilyas'in Tanri'ya nasil Israil'den yakindigini bilmez misiniz?
\par 3 "Ya Rab, senin peygamberlerini öldürdüler, senin sunaklarini yiktilar. Yalniz ben kaldim. Beni de öldürmeye çalisiyorlar."
\par 4 Tanri'nin ona verdigi yanit nedir? "Baal'in* önünde diz çökmemis yedi bin kisiyi kendime ayirdim."
\par 5 Ayni sekilde, simdiki dönemde de Tanri'nin lütfuyla seçilmis küçük bir topluluk vardir.
\par 6 Eger bu, lütufla olmussa, iyi islerle olmamis demektir. Yoksa lütuf artik lütuf olmaktan çikar!
\par 7 Sonuç ne? Israil aradigina kavusamadi, seçilmis olanlar ise kavustular. Geriye kalanlarinsa yürekleri nasirlastirildi.
\par 8 Yazilmis oldugu gibi: "Tanri onlara uyusukluk ruhu verdi; Bugüne dek görmeyen gözler, duymayan kulaklar verdi."
\par 9 Davut da söyle diyor: "Sofralari onlara tuzak, Kapan, tökez ve ceza olsun.
\par 10 Gözleri kararsin, göremesinler. Bellerini hep iki büklüm et!"
\par 11 Öyleyse soruyorum: Israilliler, bir daha kalkmamak üzere mi sendeleyip düstüler? Kesinlikle hayir! Ama onlarin suçu yüzünden öteki uluslara kurtulus verildi; öyle ki, Israilliler onlara imrensin.
\par 12 Eger Israilliler'in suçu dünyaya zenginlik, bozgunu uluslara zenginlik getirdiyse, bütünlügü çok daha büyük bir zenginlik getirecektir!
\par 13 Öteki uluslardan olan sizlere söylüyorum: Uluslara elçi olarak gönderildigim için görevimi yüce sayarim.
\par 14 Böylelikle belki soydaslarimi imrendirip bazilarini kurtaririm.
\par 15 Çünkü onlarin reddedilmesi dünyanin Tanri'yla barismasini sagladiysa, kabul dilmeleri ölümden yasama geçis degil de nedir?
\par 16 Hamurun ilk parçasi kutsalsa, tümü kutsaldir; kök kutsalsa, dallar da kutsaldir.
\par 17 Ama zeytin agacinin bazi dallari kesildiyse ve sen yabanil bir zeytin filiziyken onlarin yerine asilanip agacin semiz köküne ortak oldunsa, o dallara karsi övünme. Eger övünüyorsan, unutma ki, sen kökü tasimiyorsun, kök seni tasiyor.
\par 19 O zaman, "Ben asilanayim diye dallar kesildi" diyeceksin.
\par 20 Dogru, onlar imansizlik yüzünden kesildiler. Sense imanla yerinde duruyorsun. Böbürlenme, kork!
\par 21 Çünkü Tanri asil dallari esirgemediyse, seni de esirgemeyecektir.
\par 22 Onun için Tanri'nin iyiligini de sertligini de gör. O, düsenlere karsi serttir; ama O'nun iyiligine bagli kalirsan, sana iyi davranir. Yoksa sen de kesilip atilirsin!
\par 23 Imansizlikta direnmezlerse, Israilliler de öz agaca asilanacaklar. Çünkü Tanri'nin onlari eski yerlerine asilamaya gücü vardir.
\par 24 Eger sen dogal yapisi yabanil zeytin agacindan kesilip dogaya aykiri olarak cins zeytin agacina asilandinsa, asil dallarin öz zeytin agacina asilanacaklari çok daha kesindir!
\par 25 Kardesler, bilgiçlige kapilmamaniz için su sirdan habersiz kalmanizi istemem: Israilliler'den bir bölümünün yüregi, öteki uluslardan kurtulacaklarin sayisi tamamlanincaya dek duyarsiz kalacaktir.
\par 26 Sonunda bütün Israil kurtulacaktir. Yazilmis oldugu gibi: "Kurtarici Siyon'dan* gelecek, Yakup'un soyundan tanrisizligi uzaklastiracak.
\par 27 Onlarin günahlarini kaldiracagim zaman Kendileriyle yapacagim antlasma budur."
\par 28 Israilliler Müjde'yi reddederek sizin ugrunuza Tanri'ya düsman oldular; ama Tanri'nin seçimine göre, atalari sayesinde sevilmektedirler.
\par 29 Çünkü Tanri'nin armaganlari ve çagrisi geri alinamaz.
\par 30 Bir zamanlar Tanri'nin sözünü dinlemeyen sizler simdi Israilliler'in sözdinlemezliginin sonucu merhamete kavustunuz.
\par 31 Bunun gibi, Israilliler de, sizin kavustugunuz merhametle merhamete erismek için simdi söz dinlemez oldular.
\par 32 Çünkü Tanri, merhametini herkese göstermek için herkesi söz dinlemezligin tutsagi kildi.
\par 33 Tanri'nin zenginligi ne büyük, bilgeligi ve bilgisi ne derindir! O'nun yargilari ne denli akil ermez, yollari ne denli anlasilmazdir!
\par 34 "Rab'bin düsüncesini kim bilebildi? Ya da kim O'nun ögütçüsü olabildi?"
\par 35 "Kim Tanri'ya bir sey verdi ki, Karsiligini O'ndan isteyebilsin?"
\par 36 Her seyin kaynagi O'dur; her sey O'nun araciligiyla ve O'nun için var oldu. O'na sonsuza dek yücelik olsun! Amin.

\chapter{12}

\par 1 Öyleyse kardeslerim, Tanri'nin merhameti adina size yalvaririm: Bedenlerinizi diri, kutsal, Tanri'yi hosnut eden birer kurban olarak sunun. Ruhsal tapinmaniz budur.
\par 2 Bu çagin gidisine uymayin; bunun yerine, Tanri'nin iyi, begenilir ve yetkin isteginin ne oldugunu ayirt edebilmek için düsüncenizin yenilenmesiyle degisin.
\par 3 Tanri'nin bana bagisladigi lütufla hepinize söylüyorum: Kimse kendisine gereginden çok deger vermesin. Herkes Tanri'nin kendisine verdigi iman ölçüsüne göre düsüncelerinde sagduyulu olsun.
\par 4 Bir bedende ayri ayri islevleri olan çok sayida üyemiz oldugu gibi, çok sayida olan bizler de Mesih'te tek bir bedeniz ve birbirimizin üyeleriyiz.
\par 6 Tanri'nin bize bagisladigi lütfa göre, ayri ayri ruhsal armaganlarimiz vardir. Birinin armagani peygamberlikse, imani oraninda peygamberlik etsin.
\par 7 Hizmetse, hizmet etsin. Ögretmekse, ögretsin.
\par 8 Ögüt veren, ögütte bulunsun. Bagista bulunan, bunu cömertçe yapsin. Yöneten, gayretle yönetsin. Merhamet eden, bunu güler yüzle yapsin.
\par 9 Sevginiz ikiyüzlü olmasin. Kötülükten tiksinin, iyilige baglanin.
\par 10 Birbirinize kardeslik sevgisiyle bagli olun. Birbirinize saygi göstermekte yarisin.
\par 11 Gayretiniz eksilmesin. Ruhta atesli olun. Rab'be kulluk edin.
\par 12 Umudunuzla sevinin. Sikintiya dayanin. Kendinizi duaya verin.
\par 13 Ihtiyaç içinde olan kutsallara yardim edin. Konuksever olmayi amaç edinin.
\par 14 Size zulmedenler için iyilik dileyin. Iyilik dileyin, lanet etmeyin.
\par 15 Sevinenlerle sevinin, aglayanlarla aglayin.
\par 16 Birbirinizle ayni düsüncede olun. Böbürlenmeyin; tersine, hor görülenlerle arkadaslik edin. Bilgiçlik taslamayin.
\par 17 Kötülüge kötülükle karsilik vermeyin. Herkesin gözünde iyi olani yapmaya dikkat edin.
\par 18 Mümkünse, elinizden geldigince herkesle baris içinde yasayin.
\par 19 Sevgili kardesler, kimseden öç almayin; bunu Tanri'nin gazabina birakin. Çünkü söyle yazilmistir: "Rab diyor ki, 'Öç benimdir, ben karsilik verecegim.'"
\par 20 Ama, "Düsmanin acikmissa doyur, Susamissa su ver. Bunu yapmakla onu utanca bogarsin."
\par 21 Kötülüge yenilme, kötülügü iyilikle yen.

\chapter{13}

\par 1 Herkes, bastaki yönetime bagli olsun. Çünkü Tanri'dan olmayan yönetim yoktur. Var olanlar Tanri tarafindan kurulmustur.
\par 2 Bu nedenle, yönetime karsi direnen, Tanri buyruguna karsi gelmis olur. Karsi gelenler yargilanir.
\par 3 Iyilik edenler degil, kötülük edenler yöneticilerden korkmalidir. Yönetimden korkmamak ister misin, öyleyse iyi olani yap, yönetimin övgüsünü kazanirsin.
\par 4 Çünkü yönetim, senin iyiligin için Tanri'ya hizmet etmektedir. Ama kötü olani yaparsan, kork! Yönetim, kilici bos yere tasimiyor; kötülük yapanin üzerine Tanri'nin gazabini salan öç alici olarak Tanri'ya hizmet ediyor.
\par 5 Bunun için, yalniz Tanri'nin gazabi nedeniyle degil, vicdan nedeniyle de yönetime bagli olmak gerekir.
\par 6 Vergi ödemenizin nedeni de budur. Çünkü yöneticiler Tanri'nin bu amaç için gayretle çalisan hizmetkârlaridir.
\par 7 Herkese hakkini verin: Vergi hakki olana vergi, gümrük hakki olana gümrük, saygi hakki olana saygi, onur hakki olana onur verin.
\par 8 Birbirinizi sevmekten baska hiç kimseye bir sey borçlu olmayin. Çünkü baskalarini seven, Kutsal Yasa'yi yerine getirmis olur.
\par 9 "Zina etmeyeceksin, adam öldürmeyeceksin, çalmayacaksin, baskasinin malina göz dikmeyeceksin" buyruklari ve bundan baska ne buyruk varsa, su sözde özetlenmistir: "Komsunu kendin gibi seveceksin."
\par 10 Seven kisi komsusuna kötülük etmez. Bu nedenle sevmek Kutsal Yasa'yi yerine getirmektir.
\par 11 Bunu, yasadiginiz zamanin bilincinde olarak yapin. Artik sizin için uykudan uyanma saati gelmistir. Çünkü su anda kurtulusumuz ilk iman ettigimiz zamankinden daha yakindir.
\par 12 Gece ilerledi, gündüz yaklasti. Bunun için karanligin islerini üzerimizden atip isigin silahlarini kusanalim.
\par 13 Çilginca eglenceye ve sarhosluga, fuhsa ve sefahate, çekismeye ve kiskançliga kapilmayalim. Gün isiginda oldugu gibi, saygin bir yasam sürelim.
\par 14 Rab Isa Mesih'i kusanin. Benliginizin tutkularina uymayi düsünmeyin.

\chapter{14}

\par 1 Imani zayif olani araniza kabul edin, ama tartismali konulara girmeyin.
\par 2 Biri her seyi yiyebilecegine inanir; imani zayif olansa yalniz sebze yer.
\par 3 Her seyi yiyen, yemeyeni hor görmesin. Her seyi yemeyen, yiyeni yargilamasin. Çünkü Tanri onu kabul etmistir.
\par 4 Sen kimsin ki, baskasinin kulunu yargiliyorsun? Kulu hakli çikaran da haksiz çikaran da efendisidir. Kul hakli çikacaktir. Çünkü Rab'bin onu hakli çikarmaya gücü vardir.
\par 5 Kimi bir günü baska bir günden üstün sayar, kimi her günü bir sayar. Herkesin kendi görüsüne tam güveni olsun.
\par 6 Belli bir günü kutlayan, Rab için kutlar. Her seyi yiyen, Tanri'ya sükrederek Rab için yer. Bazi seyleri yemeyen de Rab için yemez ve Tanri'ya sükreder.
\par 7 Hiçbirimiz kendimiz için yasamayiz, hiçbirimiz de kendimiz için ölmeyiz.
\par 8 Yasarsak Rab için yasariz; ölürsek Rab için ölürüz. Öyleyse, yasasak da ölsek de Rab'be aitiz.
\par 9 Mesih hem ölülerin hem yasayanlarin Rabbi olmak üzere ölüp dirildi.
\par 10 Sen neden kardesini yargiliyorsun? Ya sen, kardesini neden küçümsüyorsun? Tanri'nin yargi kürsüsü önüne hepimiz çikacagiz.
\par 11 Yazilmis oldugu gibi: "Rab söyle diyor: 'Varligim hakki için her diz önümde çökecek, Her dil Tanri oldugumu açikça söyleyecek.'"
\par 12 Böylece her birimiz kendi adina Tanri'ya hesap verecektir.
\par 13 Onun için, artik birbirimizi yargilamayalim. Bunun yerine, hiçbir kardesin yoluna sürçme ya da tökezleme tasi koymamaya kararli olun.
\par 14 Rab Isa'ya ait biri olarak kesinlikle biliyorum ki, hiçbir sey kendiliginden murdar* degildir. Ama bir seyi murdar sayan için o sey murdardir.
\par 15 Yedigin bir sey yüzünden kardesin incinmisse, artik sevgi yolunda yürümüyorsun demektir. Mesih'in, ugruna öldügü kardesini yediklerinle mahvetme!
\par 16 Size göre iyi olanin kötülenmesine firsat vermeyin.
\par 17 Çünkü Tanri'nin Egemenligi*, yiyecek içecek sorunu degil, dogruluk, esenlik ve Kutsal Ruh'ta sevinçtir.
\par 18 Mesih'e bu yolda hizmet eden, Tanri'yi hosnut eder, insanlarin da begenisini kazanir.
\par 19 Öyleyse kendimizi esenlik getiren ve karsilikli gelismemizi saglayan islere verelim.
\par 20 Yiyecek ugruna Tanri'nin isini bozma! Her yiyecek temizdir, ama yedikleriyle baskasinin sürçmesine yol açan kisi kötülük etmis olur.
\par 21 Et yememen, sarap içmemen, kardesinin sürçmesine yol açacak bir sey yapmaman iyidir.
\par 22 Bu konulardaki inancini Tanri'nin önünde kendine sakla. Onayladigi seyden ötürü kendini yargilamayan kisi ne mutludur!
\par 23 Ama bir yiyecekten kuskulanan kisi onu yerse yargilanir; çünkü imanla yemiyor. Imana dayanmayan her sey günahtir.

\chapter{15}

\par 1 Imani güçlü olan bizler, kendimizi hosnut etmeye degil, güçsüzlerin zayifliklarini yüklenmeye borçluyuz.
\par 2 Her birimiz komsusunu ruhça gelistirmek için komsusunun iyiligini gözeterek onu hosnut etsin.
\par 3 Çünkü Mesih bile kendini hosnut etmeye çalismadi. Yazilmis oldugu gibi: "Sana edilen hakaretlere ben ugradim."
\par 4 Önceden ne yazildiysa, bize ögretmek için, sabirla ve Kutsal Yazilar'in verdigi cesaretle umudumuz olsun diye yazildi.
\par 5 Sabir ve cesaret kaynagi olan Tanri'nin, sizleri Mesih Isa'nin istegine uygun olarak ayni düsüncede birlestirmesini dilerim.
\par 6 Öyle ki, Rabbimiz Isa Mesih'in Tanrisi'ni ve Babasi'ni birlik içinde hep bir agizdan yüceltesiniz.
\par 7 Bu nedenle, Mesih sizi kabul ettigi gibi, Tanri'nin yüceligi için birbirinizi kabul edin.
\par 8 Çünkü diyorum ki Mesih, Tanri'nin güvenilir oldugunu göstermek için Yahudiler'in hizmetkâri oldu. Öyle ki, atalarimiza verilen sözler dogrulansin ve öteki uluslar merhameti için Tanri'yi yüceltsin. Yazilmis oldugu gibi: "Bunun için uluslar arasinda sana sükredecegim, Adini ilahilerle övecegim."
\par 10 Yine deniyor ki, "Ey uluslar, O'nun halkiyla birlikte sevinin!" Ve, "Ey bütün uluslar, Rab'be övgüler sunun! Ey bütün halklar, O'nu yüceltin!"
\par 12 Yesaya da söyle diyor: "Isay'in Kökü ortaya çikacak, Uluslara egemen olmak üzere yükselecek. Uluslar O'na umut baglayacak."
\par 13 Umut kaynagi olan Tanri, Kutsal Ruh'un gücüyle umutla dolup tasmaniz için iman yasaminizda sizleri tam bir sevinç ve esenlikle doldursun.
\par 14 Size gelince, kardeslerim, iyilikle dolu, her bilgiyle donanmis oldugunuzdan ben eminim. Ayrica, birbirinize ögüt verebilecek durumdasiniz.
\par 15 Yine de Tanri'nin bana bagisladigi lütufla bazi noktalari yeniden animsatmak için size yazma cesaretini gösterdim.
\par 16 Ben Tanri'nin lütfuyla uluslar yararina Mesih Isa'nin hizmetkâri oldum. Tanri'nin Müjdesi'ni bir kâhin* olarak yaymaktayim. Öyle ki uluslar, Kutsal Ruh'la kutsal kilinarak Tanri'yi hosnut eden bir sunu olsun.
\par 17 Bunun için Mesih Isa'ya ait biri olarak Tanri'ya verdigim hizmetle övünebilirim.
\par 18 Uluslarin söz dinlemesi için Mesih'in benim araciligimla, sözle ve eylemle, mucizeler ve harikalar yaratan güçle, Kutsal Ruh'un gücüyle yaptiklarindan baska seyden söz etmeye cesaret edemem. Yerusalim'den* baslayip Illirikum bölgesine kadar dolasarak Mesih'in Müjdesi'ni her yerde duyurdum.
\par 20 Bir baskasinin attigi temel üzerine insa etmemek için Müjde'yi Mesih'in adinin duyulmadigi yerlerde yaymayi amaç edindim.
\par 21 Yazilmis oldugu gibi: "O'ndan habersiz olanlar görecekler. Duymamis olanlar anlayacaklar."
\par 22 Iste bu yüzden yaniniza gelmem kaç kez engellendi.
\par 23 Simdiyse bu yörelerde artik yapacagim bir sey kalmadigindan, yillardir da yaniniza gelmeyi arzuladigimdan, Ispanya'ya giderken size ugrarim. Yol üzerinde sizi görüp bir süre arkadasliginiza doyduktan sonra beni oraya ugurlayacaginizi umarim.
\par 25 Ama simdi kutsallara bir hizmet için Yerusalim'e gidiyorum.
\par 26 Çünkü Makedonya ve Ahaya'da bulunanlar, Yerusalim'deki kutsallar arasinda yoksul olanlar için yardim toplamayi uygun gördüler.
\par 27 Evet, uygun gördüler. Gerçekte onlara yardim borçlular. Uluslar, onlarin ruhsal bereketlerine ortak olduklarina göre, maddesel bereketlerle onlara hizmet etmeye borçlular.
\par 28 Bu isi bitirip saglanan yardimi onlara ulastirdiktan sonra size ugrayacagim, sonra da Ispanya'ya gidecegim.
\par 29 Yaniniza geldigimde, Mesih'in bereketinin doluluguyla gelecegimi biliyorum.
\par 30 Kardesler, Rabbimiz Isa Mesih ve Ruh'un sevgisi adina size yalvariyorum, benim için Tanri'ya dua ederek ugrasima katilin.
\par 31 Yahudiye'deki imansizlardan kurtulmam için ve Yerusalim'e olan hizmetimin kutsallarca kabul edilmesi için dua edin.
\par 32 Öyle ki, Tanri'nin istegiyle sevinçle yaniniza gelip sizlerle gönlümü ferahlatayim.
\par 33 Esenlik veren Tanri hepinizle birlikte olsun! Amin.

\chapter{16}

\par 1 Kenhere'deki kilisenin* görevlisi olan kizkardesimiz Fibi'yi size salik veririm.
\par 2 Kutsallara yarasir biçimde onu Rab'bin adina kabul edin. Herhangi bir ihtiyaci olursa, kendisine yardim edin. Çünkü o, ben dahil, birçoklarina destek saglamistir.
\par 3 Mesih Isa yolunda emektaslarim olan Priska ve Akvila'ya selam edin.
\par 4 Onlar benim ugruma yasamlarini tehlikeye attilar. Yalniz ben degil, öteki uluslarin* bütün kiliseleri de onlara minnettardir.
\par 5 Onlarin evindeki inanlilar topluluguna* da selam söyleyin. Asya Ili'nden* Mesih'e ilk iman eden sevgili kardesim Epenetus'a selam edin.
\par 6 Sizin için çok çalismis olan Meryem'e selam söyleyin.
\par 7 Mesih'in elçileri arasinda taninmis ve benden önce Mesih'e inanmis olan soydaslarim ve hapishane arkadaslarim Andronikus'la Yunya'ya selam edin.
\par 8 Rab'be ait olan sevgili kardesim Ampliatus'a selam söyleyin.
\par 9 Mesih yolunda emektasimiz olan Urbanus'a ve sevgili kardesim Stakis'e selam edin.
\par 10 Mesih'in begenisini kazanmis olan Apellis'e selam söyleyin. Aristobulus'un ev halkindan olanlara selam edin.
\par 11 Soydasim Herodion'a selam söyleyin. Narkis'in ev halkindan Rab'be ait olanlara selam söyleyin.
\par 12 Rab'bin hizmetinde çalisan Trifena'yla Trifosa'ya selam edin. Rab'bin hizmetinde çok çalismis olan sevgili Persis'e selam söyleyin.
\par 13 Rab'bin seçkin kulu olan Rufus'a ve bana da annelik etmis olan annesine selam edin.
\par 14 Asinkritus, Flegon, Hermes, Patrovas, Hermas ve yanlarindaki kardeslere selam edin.
\par 15 Filologus'la Yulya'ya, Nereus'la kizkardesine, Olimpas'la yanlarindaki bütün kutsallara selam edin.
\par 16 Birbirinizi kutsal öpüsle selamlayin. Mesih'in bütün kiliseleri size selam ederler.
\par 17 Kardesler, size yalvaririm, aldiginiz ögretiye karsi gelerek ayriliklara ve sapmalara neden olanlara dikkat edin, onlardan sakinin.
\par 18 Böyle kisiler Rabbimiz Mesih'e degil, kendi midelerine kulluk ediyorlar. Saf kisilerin yüreklerini kulagi oksayan tatli sözlerle aldatiyorlar.
\par 19 Sözdinlerliginizi herkes duydu, bu nedenle sizin adiniza seviniyorum. Iyilik konusunda bilge, kötülük konusunda deneyimsiz olmanizi isterim.
\par 20 Esenlik veren Tanri çok geçmeden Seytan'i ayaklarinizin altinda ezecektir. Rabbimiz Isa'nin lütfu sizinle birlikte olsun.
\par 21 Emektasim Timoteos, soydaslarimdan Lukius, Yason ve Sosipater size selam ederler.
\par 22 Mektubu yaziya geçiren ben Tertius, Rab'be ait biri olarak size selamlarimi gönderirim.
\par 23 Bana ve bütün inanlilar topluluguna konukseverlik eden Gayus size selam eder. Kent haznedari Erastus'un ve Kuartus kardesin size selamlari var.
\par 25 Tanri, duyurdugum Müjde ve Isa Mesih'le ilgili bildiri uyarinca, sonsuz çaglardan beri sakli tutulan sirri açiklayan vahiy uyarinca sizi ruhça pekistirecek güçtedir.
\par 26 O sir simdi aydinliga çikarilmis ve öncesiz Tanri'nin buyruguna göre peygamberlerin yazilari araciligiyla bütün uluslarin iman ederek söz dinlemesi için bildirilmistir.
\par 27 Bilge olan tek Tanri'ya Isa Mesih araciligiyla sonsuza dek yücelik olsun! Amin.


\end{document}