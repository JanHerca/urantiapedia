\begin{document}

\title{Ezgiler Ezgisi}


\chapter{1}

\par 1 Süleyman'in Ezgiler Ezgisi. [Kiz]
\par 2 Beni dudaklariyla öptükçe öpsün! Çünkü askin saraptan daha tatli.
\par 3 Ne güzel kokuyor sürdügün esans, Dökülmüs esans sanki adin, Kizlar bu yüzden seviyor seni.
\par 4 Al götür beni, haydi kosalim! Kral beni odasina götürsün. [Kizin Arkadaslari] Seninle cosup seviniriz, Askini saraptan çok överiz. [Kiz] Ne kadar haklilar seni sevmekte!
\par 5 Esmerim ben, ama güzelim, Ey Yerusalim kizlari! Kedar'in çadirlari gibi, Süleyman'in çadir bezleri gibi kara.
\par 6 Bakmayin esmer olduguma, Günes karartti beni. Çünkü kizdilar bana erkek kardeslerim, Baglara bakmakla görevlendirdiler. Ama kendi bagima bakmadim.
\par 7 Ey sevgilim, söyle bana, sürünü nerede otlatiyorsun, Ögleyin nerede yatiriyorsun? Neden arkadaslarinin sürüleri yaninda Yüzünü örten bir kadin durumuna düseyim? [Kizin Arkadaslari]
\par 8 Ey güzeller güzeli, Bilmiyorsan, Sürünün izine çik, Çobanlarin çadirlari yaninda Oglaklarini otlat. [Erkek]
\par 9 Firavunun arabalarina kosulu kisraga benzetiyorum seni, askim benim!
\par 10 Yanaklarin süslerle, Boynun gerdanliklarla ne güzel!
\par 11 Sana gümüs dügmelerle altin süsler yapacagiz. [Kiz]
\par 12 Kral divandayken, Hintsümbülümün güzel kokusu yayildi.
\par 13 Memelerim arasinda yatan Mür* dolu bir kesedir benim için sevgilim;
\par 14 Eyn-Gedi baglarinda Bir demet kina çiçegidir benim için sevgilim. [Erkek]
\par 15 Ah, ne güzelsin, askim, ah, ne güzel! Gözlerin tipki birer güvercin! [Kiz]
\par 16 Ne yakisiklisin, sevgilim, ah, ne çekici! Yesilliktir yatagimiz. [Erkek]
\par 17 Sedir agaçlaridir evimizin kirisleri, Tavanimizin tahtalari ardiçlar. [Kiz]

\chapter{2}

\par 1 Ben Saron çigdemiyim, Vadilerin zambagiyim. [Erkek]
\par 2 Dikenlerin arasinda bir zambaga benzer Kizlarin arasinda askim. [Kiz]
\par 3 Orman agaçlari arasinda bir elma agacina benzer Delikanlilarin arasinda sevgilim. Onun gölgesinde oturmaktan zevk alirim, Tadi damagimda kalir meyvesinin.
\par 4 Ziyafet evine götürdü beni, Üzerimdeki sancagi askti.
\par 5 Güçlendirin beni üzüm pestiliyle, Canlandirin elmayla, Çünkü ask hastasiyim ben.
\par 6 Sol eli basimin altinda, Sag eli sarsin beni.
\par 7 Disi ceylanlar, Yabanil disi geyikler üstüne Ant içiriyorum size, ey Yerusalim kizlari! Askimi ayiltmayasiniz, uyandirmayasiniz diye, Gönlü hos olana dek.
\par 8 Iste! Sevgilimin sesi! Daglarin üzerinden sekerek, Tepelerin üzerinden siçrayarak geliyor.
\par 9 Sevgilim ceylana benzer, sanki bir geyik yavrusu. Bakin, duvarimizin ardinda duruyor, Pencerelerden bakiyor, Kafeslerden seyrediyor.
\par 10 Sevgilim söyle dedi: "Kalk, gel askim, güzelim.
\par 11 Bak, kis geçti, Yagmurlarin ardi kesildi,
\par 12 Çiçekler açti, Sarki mevsimi geldi, Kumrular ötüsmeye basladi beldemizde.
\par 13 Incir agaci ilk meyvesini verdi, Yeseren asmalar mis gibi kokular saçmakta. Kalk, gel askim, güzelim." [Erkek]
\par 14 Kaya kovuklarinda, Uçurum kenarlarinda gizlenen güvercinim! Boyunu bosunu göster bana, Sesini duyur; Çünkü sesin tatli, boyun bosun güzeldir.
\par 15 Yakalayin tilkileri bizim için, Baglari bozan küçük tilkileri; Çünkü baglarimiz yeserdi. [Kiz]
\par 16 Sevgilim benimdir, ben de onun, Zambaklar arasinda gezinir durur.
\par 17 Ey sevgilim, gün serinleyip gölgeler uzayana dek, Engebeli daglar üzerinde bir ceylan gibi, Geyik yavrusu gibi ol!

\chapter{3}

\par 1 Gece boyunca yatagimda Sevgilimi aradim, Aradim, ama bulamadim.
\par 2 "Kalkip kenti dolasayim, Sokaklarda, meydanlarda sevgilimi arayayim" dedim, Aradim, ama bulamadim.
\par 3 Kenti dolasan bekçiler buldu beni, "Sevgilimi gördünüz mü?" diye sordum.
\par 4 Onlardan ayrilir ayrilmaz Sevgilimi buldum. Tuttum onu, birakmadim; Annemin evine, Beni doguran kadinin odasina götürünceye dek.
\par 5 Disi ceylanlar, Yabanil disi geyikler üstüne Ant içiriyorum size, ey Yerusalim kizlari! Askimi ayiltmayasiniz, uyandirmayasiniz diye, Gönlü hos olana dek.
\par 6 Kimdir bu kirdan çikan, Bir duman sütunu gibi, Tüccarin türlü türlü baharatiyla, Mür* ve günnükle tütsülenmis?
\par 7 Iste Süleyman'in tahtirevani! Israilli yigitlerden Altmis kisi eslik ediyor ona.
\par 8 Hepsi kiliç kusanmis, egitilmis savasçi. Gecenin tehlikelerine karsi, Hepsinin kilici belinde.
\par 9 Kral Süleyman tahtirevani Lübnan agaçlarindan yapti.
\par 10 Direklerini gümüsten, Tabanini altindan yapti. Koltugu mor kumasla kapliydi. Içini sevgiyle dösemisti Yerusalim kizlari.
\par 11 Disari çikin, ey Siyon kizlari! Dügününde, mutlu gününde Annesinin verdigi taci giymis Kral Süleyman'i görün. [Erkek]

\chapter{4}

\par 1 Ah, ne güzelsin, askim, ah, ne güzel! Peçenin ardindaki gözlerin güvercinler gibi. Siyah saçlarin Gilat Dagi'nin yamaçlarindan inen Keçi sürüsü sanki.
\par 2 Yeni kirkilip yikanmis, Sudan çikmis koyun sürüsü gibi dislerin, Hepsinin ikizi var. Yavrusunu yitiren yok aralarinda.
\par 3 Al kurdele gibi dudaklarin, Agzin ne güzel! Peçenin ardindaki yanaklarin Nar parçasi sanki.
\par 4 Boynun Davut'un kulesi gibi, Kakma taslarla yapilmis, Üzerine bin kalkan asilmis, Hepsi de birer yigit kalkani.
\par 5 Sanki bir çift geyik yavrusu memelerin Zambaklar arasinda otlayan Ikiz ceylan yavrusu.
\par 6 Gün serinleyip gölgeler uzayinca, Mür* dagina, Günnük tepesine gidecegim.
\par 7 Tepeden tirnaga güzelsin, askim, Hiç kusurun yok.
\par 8 Benimle gel Lübnan'dan, yavuklum, Benimle gel Lübnan'dan! Amana dorugundan, Senir ve Hermon doruklarindan, Aslanlarin inlerinden, Parslarin daglarindan geç.
\par 9 Çaldin gönlümü kizkardesim, yavuklum, Bir bakisinla, Gerdanliginin tek zinciriyle çaldin gönlümü!
\par 10 Askin ne güzel, kizkardesim, yavuklum, Saraptan çok daha tatli; Esansinin kokusu her türlü baharattan güzel!
\par 11 Ey yavuklum, bal damlar dudaklarindan, Bal ve süt var dilinin altinda, Lübnan'in kokusu geliyor giysilerinden!
\par 12 Kapali bahçesin sen, kizkardesim, yavuklum, Kapali bir kaynak, mühürlü bir pinar.
\par 13 Fidanlarin nar bahçesidir; Seçme meyvelerle, Kina ve hintsümbülüyle,
\par 14 Hintsümbülü ve safranla, Güzel kokulu kamis ve tarçinla, her türlü günnük agaciyla, Mür ve ödle, her türlü seçme baharatla.
\par 15 Sen bir bahçe pinarisin, Bir taze su kuyusu, Lübnan'dan akan bir dere. [Kiz]
\par 16 Uyan, ey kuzey rüzgari, Sen de gel, ey güney rüzgari! Bahçemde es de güzel kokusu saçilsin. Sevgilim bahçesine gelsin, seçme meyvelerini yesin! [Erkek]

\chapter{5}

\par 1 Bahçeme girdim, kizkardesim, yavuklum, Mürümü* topladim baharatimla, Gümecimi, balimi yedim, Sarabimi, sütümü içtim. [Kizin Arkadaslari] Yiyin, için, ey dostlar! Mest olun asktan, ey sevgililer! [Kiz]
\par 2 Ben uyuyordum ama yüregim uyanikti. Dinleyin! Sevgilim kapiyi vuruyor. "Aç bana, kizkardesim, askim, essiz güvercinim! Sirilsiklam oldu basim çiyden, Kaküllerim gecenin neminden."
\par 3 Entarimi çikardim, Yine giyinmeli miyim? Ayaklarimi yikadim, Yine kirletmeli miyim?
\par 4 Kapi deliginden uzatti elini sevgilim, Ask duygularim kabardi onun için.
\par 5 Kalktim, sevgilime kapiyi açayim diye, Mür* elimden damladi, Parmaklarimdan akti Sürgü tokmaklari üzerine.
\par 6 Kapiyi açtim sevgilime, Ama sevgilim yoktu, gitmisti! Kendimden geçmisim o konusurken. Aradim onu, ama bulamadim, Seslendim, ama yanit vermedi.
\par 7 Kenti dolasan bekçiler buldu beni, Dövüp yaraladilar. Sur bekçileri alip götürdü salimi.
\par 8 Size ant içiriyorum, ey Yerusalim kizlari! Eger sevgilimi bulursaniz, Söyleyin ona, ask hastasiyim ben. [Kizin Arkadaslari]
\par 9 Farki ne sevgilinin öbürlerinden, Ey güzeller güzeli? Farki ne ki, bize böyle ant içiriyorsun? [Kiz]
\par 10 Sevgilimin teni pembe-beyaz, isil isil yaniyor! Göze çarpiyor on binler arasinda.
\par 11 Basi saf altin, Kakülleri kivir kivir, kuzgun gibi siyah.
\par 12 Akarsu kiyisindaki Güvercinler gibi gözleri; Sütle yikanmis, Yuvasindaki mücevher sanki.
\par 13 Yanaklari güzel kokulu tarhlar gibi, Nefis kokular saçiyor. Dudaklari zambak gibi, Mür* yagi damlatiyor.
\par 14 Elleri, üzerine sari yakut kakilmis altin çubuklar, Gövdesi laciverttasiyla süslenmis cilali fildisi.
\par 15 Mermer sütun bacaklari Saf altin ayakliklar üzerine kurulmus. Boyu bosu Lübnan daglari gibi, Lübnan'in sedir agaçlari gibi essiz.
\par 16 Agzi çok tatli, Tepeden tirnaga güzel. Iste böyledir sevgilim, böyledir yarim, ey Yerusalim kizlari! [Kizin Arkadaslari]

\chapter{6}

\par 1 Nereye gitti sevgilin, Ey güzeller güzeli, Ne yana yöneldi? Biz de onu arayalim seninle birlikte! [Kiz]
\par 2 Bahçesine indi sevgilim, Güzel kokulu tarhlara, Bahçede gezinmek, zambak toplamak için.
\par 3 Ben sevgilime aitim, sevgilim de bana, Gezinip duruyor zambaklar arasinda. [Erkek]
\par 4 Sevgilim, Tirsa kadar güzelsin, Yerusalim kadar sirin, Sancak açmis bir ordu kadar görkemli.
\par 5 Çevir gözlerini benden, Çünkü sasirtiyorlar beni. Gilat Dagi'nin yamaçlarindan inen Keçi sürüsünü andiriyor siyah saçlarin.
\par 6 Yeni yikanmis, sudan çikmis disi koyun sürüsü gibi dislerin, Hepsinin ikizi var; Yavrusunu yitiren yok aralarinda.
\par 7 Peçenin ardindaki yanaklarin Nar parçasi sanki.
\par 8 Altmis kraliçe, Seksen cariye, Sayisiz bakire kiz olabilir;
\par 9 Ama bir tanedir benim essiz güvercinim, Biricik kizidir annesinin, Gözbebegi kendisini doguranin. Kizlar sevgilimi görünce, "Ne mutlu ona!" dediler. Kraliçeler, cariyeler onu övdüler. [Kizin Arkadaslari]
\par 10 Kimdir bu kadin? Safak gibi beliren, Ay kadar güzel, Günes kadar parlak, Sancak açmis bir ordu kadar görkemli. [Kiz]
\par 11 Ceviz bahçesine indim, Yesermis vadiyi göreyim diye; Asma tomurcuk verdi mi, Narlar çiçek açti mi bakayim diye.
\par 12 Nasil oldu farkina varmadan, Tutkum bindirdi beni soylu halkimin savas arabalarina. [Kizin Arkadaslari]
\par 13 Dön, geri dön, ey Sulamli kiz, Dön, geri dön de seni seyredelim. [Erkek] Niçin Sulamli kizi seyretmek istiyorsunuz, Mahanayim oyununu seyredercesine?

\chapter{7}

\par 1 Ne güzel sandaletli ayaklarin, Ey soylu kiz! Mücevher gibi yuvarlak kalçalarin, Usta ellerin isi.
\par 2 Karisik sarabin hiç eksilmedigi Yuvarlak bir tas gibi göbegin. Zambaklarla kusanmis Bugday yigini gibi karnin.
\par 3 Sanki bir çift geyik yavrusu memelerin, Ikiz ceylan yavrusu.
\par 4 Fildisi kule gibi boynun. Bat-Rabim Kapisi yanindaki Hesbon havuzlari gibi gözlerin. Sam'a bakan Lübnan Kulesi gibi burnun.
\par 5 Karmel Dagi gibi duruyor basin, Piril piril mora çalar saçlarin. Kaküllerine tutsak oldu kral.
\par 6 Ne güzel, ne çekicidir ask! Zevkten zevke sürükler.
\par 7 Hurma agacina benziyor boyun, Salkim salkim memelerin.
\par 8 "Çikayim hurma agacina" dedim, "Tutayim meyveli dallarini." Üzüm salkimlari gibi olsun memelerin, Elma gibi koksun solugun,
\par 9 En iyi sarap gibi agzin. [Kiz] Sevgilimin dudaklarina, dislerine dogru kaysin.
\par 10 Ben sevgilime aitim, O da bana tutkun.
\par 11 Gel, sevgilim, kira çikalim, Köylerde geceleyelim.
\par 12 Baglara gidelim sabah erkenden, Bakalim, asma tomurcuk verdi mi? Dallari yeserdi mi, Narlar çiçek açti mi, Orada sevisecegim seninle.
\par 13 Mis gibi koku saçiyor adamotlari, Kapimizin yanibasinda Taze, kuru, Her çesit seçme meyve var. Senin için sakladim onlari, sevgilim.

\chapter{8}

\par 1 Keske kardesim olsaydin, Annemin memelerinden süt emmis. Disarida görünce öperdim seni, Kimse de kinamazdi beni.
\par 2 Önüne düser, Beni egiten Annemin evine götürürdüm seni; Sana baharatli sarapla Kendi narlarimin suyundan içirirdim.
\par 3 Sol eli basimin altinda, Sag eli sarsin beni.
\par 4 Ant içiriyorum size, ey Yerusalim kizlari! Askimi ayiltmayasiniz, uyandirmayasiniz diye, Gönlü hos olana dek. [Kizin Arkadaslari]
\par 5 Kim bu, Sevgilisine yaslanarak çölden çikan? [Kiz] Elma agaci altinda uyandirdim seni, Orada dogum sancilari çekti annen, Orada dogum sancilari çekip dogurdu seni.
\par 6 Beni yüreginin üzerine bir mühür gibi, Kolunun üzerine bir mühür gibi yerlestir. Çünkü sevgi ölüm kadar güçlü, Tutku ölüler diyari kadar katidir. Alev alev yanar, Yakip bitiren ates gibi.
\par 7 Sevgiyi engin sular söndüremez, Irmaklar süpürüp götüremez. Insan varini yogunu sevgi ugruna verse bile, Yine de hor görülür! [Kizin Arkadaslari]
\par 8 Küçük bir kizkardesimiz var, Daha memeleri çikmadi. Ne yapacagiz kizkardesimiz için, Söz kesilecegi gün?
\par 9 Eger o bir sursa, Üzerine gümüs mazgalli siper yapariz; Eger bir kapiysa, Sedir tahtalariyla onu kaplariz. [Kiz]
\par 10 Ben bir surum, memelerim de kuleler gibi, Böylece hosnut eden biri oldum onun gözünde.
\par 11 Süleyman'in bagi vardi Baal-Hamon'da, Kiraya verdi bagini; Her biri bin gümüs öderdi ürünü için.
\par 12 Benim bagim kendi emrimde, Bin gümüs senin olsun, ey Süleyman, Iki yüz gümüs de ürününe bakan kiracilarin. [Erkek]
\par 13 Ey sen, bahçelerde oturan kadin, Arkadaslar kulak veriyor sesine, Bana da duyur onu. [Kiz]
\par 14 Kos, sevgilim, Mis kokulu daglarin üzerinde bir ceylan gibi, Geyik yavrusu gibi ol!


\end{document}