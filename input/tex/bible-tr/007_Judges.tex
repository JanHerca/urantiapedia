\begin{document}

\title{Hakimler}


\chapter{1}

\par 1 Israilliler, Yesu'nun ölümünden sonra RAB'be, "Bizim için Kenanlilar'la savasmaya ilk kim gidecek?" diye sordular.
\par 2 RAB, "Yahuda oymagi gidecek" dedi, "Kenan ülkesini onun eline teslim ediyorum."
\par 3 Yahudaogullari, kardesleri Simonogullari'na, "Kenanlilar'la savasmak için payimiza düsen bölgeye bizimle birlikte gelin" dediler, "Sonra biz de payiniza düsen bölgeye sizinle geliriz." Böylece Simonogullari Yahudaogullari'yla birlikte gitti.
\par 4 Yahudaogullari saldiriya geçti. RAB Kenanlilar'la Perizliler'i ellerine teslim etti. Bezek'te onlardan on bin kisiyi öldürdüler.
\par 5 Adoni-Bezek'le orada karsilasip savasa tutustular, Kenanlilar'la Perizliler'i yenilgiye ugrattilar.
\par 6 Adoni-Bezek kaçti, ama pesine düsüp onu yakaladilar; elleriyle ayaklarinin basparmaklarini kestiler.
\par 7 O zaman Adoni-Bezek söyle dedi: "Elleriyle ayaklarinin basparmaklari kesilmis yetmis kral, soframdan düsen kirintilari toplayip yerdi. Tanri bana onlara yaptiklarimin karsiligini veriyor." Adoni-Bezek'i Yerusalim'e götürdüler; orada öldü.
\par 8 Yahudaogullari Yerusalim'e saldirip kenti aldilar; halki kiliçtan geçirerek kenti atese verdiler.
\par 9 Sonra daglik bölgede, Negev'de ve Sefela'da yasayan Kenanlilar'la savasmak üzere güneye yöneldiler.
\par 10 Eski adi Kiryat-Arba olan Hevron'da yasayan Kenanlilar'in üzerine yürüyerek Sesay, Ahiman ve Talmay'i yenilgiye ugrattilar.
\par 11 Oradan eski adi Kiryat-Sefer olan Devir Kenti halkinin üzerine yürüdüler.
\par 12 Kalev, "Kiryat-Sefer halkini yenip orayi ele geçirene kizim Aksa'yi es olarak verecegim" dedi.
\par 13 Kenti Kalev'in küçük kardesi Kenaz'in oglu Otniel ele geçirdi. Bunun üzerine Kalev kizi Aksa'yi ona es olarak verdi.
\par 14 Kiz Otniel'in yanina varinca, onu babasindan bir tarla istemeye zorladi. Kalev, eseginden inen kizina, "Bir istegin mi var?" diye sordu.
\par 15 Kiz, "Bana bir armagan ver" dedi, "Madem Negev'deki topraklari bana verdin, su kaynaklarini da ver." Böylece Kalev yukari ve asagi su kaynaklarini ona verdi.
\par 16 Musa'nin kayinbabasinin torunlari olan Kenliler, Yahudaogullari'yla birlikte Hurma Kenti'nden ayrilip Arat'in güneyindeki Yahuda Çölü'nde yasamaya gittiler.
\par 17 Bundan sonra Yahudaogullari, kardesleri Simonogullari'yla birlikte gidip Sefat Kenti'nde oturan Kenanlilar'i yenilgiye ugrattilar. Kenti tümüyle yiktilar ve oraya Horma adini verdiler.
\par 18 Yahudaogullari Gazze'yi, Askelon'u, Ekron'u ve bunlara bagli topraklari da ele geçirdiler.
\par 19 RAB Yahudaogullari'yla birlikteydi. Yahudaogullari daglik bölgeyi ele geçirdilerse de ovada yasayan halki kovamadilar. Çünkü bunlarin demirden savas arabalari vardi.
\par 20 Musa'nin sözü uyarinca Hevron'u Kalev'e verdiler. Kalev de Anak'in üç torununu oradan sürdü.
\par 21 Bununla birlikte Benyaminogullari Yerusalim'de yasayan Yevuslular'i kovmadilar. Yevuslular bugün de Yerusalim'de Benyaminogullari'yla birlikte yasiyorlar.
\par 22 Yusuf'un soyundan gelenler Beytel'in üzerine yürüdüler. RAB onlarla birlikteydi.
\par 23 Eski adi Luz olan Beytel Kenti hakkinda bilgi toplamak için gönderdikleri casuslar kentten çikan bir adam gördüler. Ona, "Kentin girisini bize gösterirsen, sana iyi davraniriz" dediler.
\par 25 Kentin girisini gösteren adamla ailesini serbest biraktilar, kent halkini ise kiliçtan geçirdiler.
\par 26 Adam Hitit* topraklarina göç ederek Luz adinda bir kent kurdu; kent bugün de bu adla aniliyor.
\par 27 Manasseogullari Beytsean, Taanak, Dor, Yivleam, Megiddo ve bunlarin çevre köylerindeki halki kovmadi. Çünkü Kenanlilar bu topraklarda kalmakta kararliydi.
\par 28 Israilliler Kenan halkini tümüyle kovmadilar; ama zamanla güçlenince onlari angaryasina çalistirdilar.
\par 29 Efrayimogullari Gezer'de yasayan Kenanlilar'i buradan sürmediler. Kenanlilar Gezer'de Israilliler'in arasinda yasadilar.
\par 30 Zevulun da Kitron ve Nahalol halklarini kovmadi. Israilliler arasinda yasayan bu Kenanlilar angarya isler yaptilar.
\par 31 Aserogullari'na gelince, onlar da Akko, Sayda, Ahlav, Akziv, Helba, Afek ve Rehov halklarini kovmadilar.
\par 32 Bu topraklardaki Kenanlilar'i kovmayip onlarla birlikte yasadilar.
\par 33 Naftali Beytsemes ve Beytanat halkini kovmadi. Buralarin halki olan Kenanlilar'la birlikte yasayip onlari angaryasina çalistirdi.
\par 34 Amorlular Danogullari'ni ovaya inmekten alikoyarak daglik bölgelerde tuttular.
\par 35 Amorlular Heres Dagi'nda, Ayalon'da ve Saalvim'de kalmakta kararliydilar. Yusuf'un torunlari güçlenince onlari angaryasina çalistirmaya basladilar.
\par 36 Amorlular'in siniri Akrep Geçidi'nden Sela'ya ve ötesine uzaniyordu.

\chapter{2}

\par 1 RAB'bin melegi Gilgal'dan Bokim'e gitti ve Israilliler'e söyle dedi: "Sizi Misir'dan çikarip atalariniza söz verdigim topraga getirdim. 'Sizinle yaptigim antlasmayi hiçbir zaman bozmayacagim dedim.
\par 2 Dedim ki, 'Bu topraklarda yasayanlarla antlasma yapmayin; sunaklarini yikin. Ama sözümü dinlemediniz. Bunu neden yaptiniz?
\par 3 Onun için simdi, 'Bu halklari önünüzden kovmayacagim; onlar bögrünüzde diken, ilahlari da size tuzak olacak diyorum."
\par 4 RAB'bin melegi sözlerini bitirince bütün Israil halki hiçkira hiçkira aglamaya basladi.
\par 5 Bu yüzden oraya Bokim adini verdiler ve orada RAB'be kurban sundular.
\par 6 Bundan sonra Yesu halki gönderdi. Israilliler paylarina düsen topraklari miras edinmek için yola çiktilar.
\par 7 Yesu yasadikça ve RAB'bin Israil için yaptigi büyük isleri görmüs olup Yesu'dan sonra sag kalan ileri gelenler durdukça halk RAB'be kulluk etti.
\par 8 RAB'bin kulu Nun oglu Yesu yüz on yasinda öldü.
\par 9 Onu Efrayim'in daglik bölgesindeki Gaas Dagi'nin kuzeyine, kendi mülkünün sinirlari içinde kalan Timnat-Heres'e gömdüler. Israilliler Rab'den Uzaklasiyor
\par 10 Bu kusaktan olanlarin hepsi ölüp atalarina kavustuktan sonra, RAB'bi tanimayan ve O'nun Israil için yaptiklarini bilmeyen yeni bir kusak yetisti.
\par 11 Israilliler RAB'bin gözünde kötü olani yaptilar, Baallar'a* taptilar.
\par 12 Kendilerini Misir'dan çikaran atalarinin Tanrisi RAB'bi terk ettiler. Çevrelerinde yasayan uluslarin degisik ilahlarina baglanip onlara taparak RAB'bi öfkelendirdiler.
\par 13 Çünkü RAB'bi terk edip Baal'a ve Astoretler'e* taptilar.
\par 14 Bunun üzerine RAB Israil'e öfkelendi. Onlari, her seylerini alan yagmacilarin eline teslim etti; artik karsi koyamadiklari çevredeki düsmanlarinin kölesi yapti.
\par 15 RAB söyledigi ve ant içtigi gibi, onlara karsi oldugundan, savasa her gittiklerinde yenilgiye ugradilar. Büyük sikinti içindeydiler.
\par 16 Sonra RAB onlari yagmacilarin elinden kurtaran Hak çikardi.
\par 17 Ama Hakini de dinlemediler. RAB'be vefasizlik ederek baska ilahlara taptilar. RAB'bin buyruklarini yerine getiren atalari gibi davranmadilar, onlarin izledigi yoldan çabucak saptilar.
\par 18 RAB onlar için ne zaman bir hakim çikardiysa, onunla birlikte oldu; hakim yasadigi sürece onlari düsmanlarinin elinden kurtardi. Baski ve zulüm altinda inledikleri zaman RAB onlara aciyordu.
\par 19 Ne var ki, Haki ölür ölmez yine baska ilahlara baglaniyor, onlara kulluk edip tapiyorlardi. Bu yolda atalarindan beter oldular. Yaptiklari kötülüklerden ve inatçiliktan vazgeçmediler.
\par 20 RAB bu yüzden Israil'e öfkelenerek söyle dedi: "Madem bu ulus atalarinin uymasini buyurdugum antlasmayi bozdu ve sözümü dinlemedi,
\par 21 ben de Yesu öldügünde bu topraklarda biraktigi uluslarin hiçbirini artik önlerinden kovmayacagim.
\par 22 Atalari gibi özenle RAB'bin yolundan gidip gitmeyeceklerini görmek için onlari bu uluslarla sinayacagim."
\par 23 RAB o uluslari hemen kovmamis, Yesu'nun eline teslim etmeyerek ülkelerinde kalmalarina izin vermisti.

\chapter{3}

\par 1 Kenan'daki savaslarin hiçbirine katilmamis olan Israilliler'i sinamak ve hiç savas deneyimi olmayan yeni kusaklara savas egitimi vermek için RAB'bin dokunmadigi uluslar sunlardir:
\par 3 Bes Filist Beyligi, bütün Kenanlilar, Saydalilar, Baal-Hermon Dagi'ndan Levo-Hamat'a kadar uzanan Lübnan daglarinda yasayan Hivliler.
\par 4 RAB Israilliler'i sinamak, Musa araciligiyla atalarina verdigi buyruklari yerine getirip getirmeyeceklerini görmek için bu uluslari ülkelerinde birakti.
\par 5 Böylece Israilliler Kenan, Hitit*, Amor, Periz, Hiv ve Yevus halklari arasinda yasadilar.
\par 6 Onlardan kiz aldilar, kizlarini onlarin ogullarina verdiler ve onlarin ilahlarina taptilar.
\par 7 RAB'bin gözünde kötü olani yapan Israilliler Tanrilari RAB'bi unutup Baallar'a* ve Asera* putlarina taptilar.
\par 8 Bunun üzerine RAB Israil'e öfkelendi ve onlari Aram- Naharayim Krali Kusan-Risatayim'in eline teslim etti. Israilliler sekiz yil Kusan-Risatayim'in boyundurugunda kaldilar.
\par 9 Ama RAB'be yakarmalari üzerine RAB onlara Otniel adinda bir kurtarici çikardi. Kalev'in küçük kardesi Kenaz'in oglu Otniel onlari kurtardi.
\par 10 RAB'bin Ruhu Otniel'in üzerine indi. Otniel Israilliler'i yönetti, onlar için savasti. RAB Aram-Naharayim Krali Kusan-Risatayim'i onun eline teslim etti. Artik Otniel ondan daha güçlüydü.
\par 11 Ülke Kenaz oglu Otniel'in ölümüne dek kirk yil baris içinde yasadi.
\par 12 Sonra Israilliler yine RAB'bin gözünde kötü olani yaptilar. RAB gözünde kötü olani yaptiklari için Moav Krali Eglon'u onlara karsi güçlendirdi.
\par 13 Kral Eglon Ammonlular'la Amalekliler'i kendi tarafina çekerek Israil'e saldirdi. Onlari bozguna ugratarak Hurma Kenti'ni ele geçirdi.
\par 14 Israilliler on sekiz yil Moav Krali Eglon'un boyundurugu altinda kaldilar.
\par 15 Ama RAB'be yakarmalari üzerine RAB onlar için Ehut adinda bir kurtarici çikardi. Benyaminli Gera'nin oglu Ehut solakti. Israilliler Ehut'un eliyle Moav Krali Eglon'a haraç gönderdiler.
\par 16 Ehut kendine bir arsin uzunlugunda iki agizli bir kama yapti ve bunu sag kalçasi üzerine, giysisinin altina sakladi.
\par 17 Varip haraci Moav Krali Eglon'a sundu. Eglon çok sisman bir adamdi.
\par 18 Ehut haraci sunduktan sonra, haraci tasimis olan adamlarini saliverdi.
\par 19 Ama kendisi Gilgal yakinindaki tas putlardan*fe* geri döndü. "Ey kral, sana gizli bir haberim var" dedi. Kral ona, "Sus" diyerek yanindaki adamlarin hepsini disari çikardi.
\par 20 Ehut, üst kattaki serin odasinda yalniz kalan krala yaklasarak, "Tanri'dan sana bir haber getirdim" deyince kral tahtindan kalkti.
\par 21 Ehut sol eliyle sag kalçasi üzerindeki kamayi çekti ve kralin karnina sapladi.
\par 22 Kamanin ucu kralin sirtindan çikti. Biçagin ardindan kabza da ete saplanmisti. Ehut kamayi çekmeyince kama kralin yagli karnina gömüldü.
\par 23 Ehut sofaya çikti, üst kattaki odanin kapisini ardindan çekip kilitledi.
\par 24 O çiktiktan sonra, geri gelen kralin hizmetkârlari üst kattaki odanin kapilarini kilitli buldular. Birbirlerine, "Su döküyor olmali" dediler.
\par 25 Uzun süre bekledilerse de kral odanin kapilarini açmadi. Bunun üzerine bir anahtar bulup kapiyi açtilar. Efendilerinin ölüsü yerde yatiyordu.
\par 26 Onlar beklerken Ehut kaçmis, tas putlari geçerek Seira'ya yönelmisti.
\par 27 Oraya varinca Efrayim'in daglik bölgesine çikip boru çaldi. Israilliler onunla birlikte daglardan indiler. Ehut önden gidiyordu.
\par 28 Onlara, "Beni izleyin" dedi, "RAB düsmanlarinizi, Moavlilar'i elinize teslim etti." Ehut'u izleyen Israilliler, Moav'a giden Seria geçitlerini tuttular, kimseyi geçirmediler.
\par 29 Moav'in güçlü yigitlerinden on bin kadarini vurup öldürdüler; hiç kurtulan olmadi.
\par 30 Moav o gün Israilliler'in boyunduruguna girdi. Ülke seksen yil baris içinde yasadi.
\par 31 Ehut'tan sonra Anat oglu Samgar basa geçti. Samgar Filistliler'den alti yüz kisiyi üvendireyle öldürerek Israilliler'i kurtardi.

\chapter{4}

\par 1 Ehut'un ölümünden sonra Israilliler yine RAB'bin gözünde kötü olani yaptilar.
\par 2 RAB de Israilliler'i Hasor'da egemenlik süren Kenanli kral Yavin'in eline teslim etti. Yavin'in Sisera adinda bir ordu komutani vardi; Haroset-Goyim'de yasardi.
\par 3 Dokuz yüz demir savas arabasina sahip olan Yavin, yirmi yildir Israilliler'i acimasizca eziyordu. Bu yüzden Israilliler RAB'be yakardilar.
\par 4 O sirada Israil'i Lappidot'un karisi Peygamber Debora yönetiyordu.
\par 5 Debora Efrayim'in daglik bölgesinde, Rama ile Beytel arasindaki hurma agacinin altinda oturur, kendisine gelen Israilliler'in davalarina bakardi.
\par 6 Debora bir gün adam gönderip Avinoam oglu Barak'i Kedes-Naftali'den çagirtti. Ona, "Israil'in Tanrisi RAB, yanina Naftali ve Zevulunogullari'ndan on bin kisi alip Tavor Dagi'na gitmeni buyuruyor" dedi,
\par 7 "RAB, 'Kral Yavin'in ordu komutani Sisera'yi, savas arabalarini ve ordusunu Kison Vadisi'ne, senin yanina çekip eline teslim edecegim diyor."
\par 8 Barak Debora'ya, "Eger benimle gelirsen giderim" dedi, "Benimle gelmezsen gitmem."
\par 9 Debora, "Seninle gelmesine gelirim, ama böyle bir yol tuttugun için onurlandirilmayacaksin" dedi, "Çünkü RAB Sisera'yi bir kadinin eline teslim etmis olacak." Böylece Debora kalkip Barak'la birlikte Kedes'e gitti.
\par 10 Barak Zevulun ve Naftali ogullarini Kedes'te topladi. Ardinda on bin kisi vardi. Debora da onunla birlikte gitti.
\par 11 Kenliler'den Hever, Musa'nin kayinbiraderi Hovav'in torunlarindan, yani Kenliler'den ayrilmis, çadirini Kedes yakininda Saanannim'deki mese agacinin yanina kurmustu.
\par 12 Avinoam oglu Barak'in Tavor Dagi'na çiktigini duyan Sisera,
\par 13 dokuz yüz demir arabasini ve yanindaki halki Haroset-Goyim'den çikarip Kison Vadisi'nde topladi.
\par 14 Debora Barak'a, "Haydi kalk! Çünkü RAB'bin Sisera'yi senin eline teslim ettigi gün bugündür" dedi, "RAB senin önünden gidiyor." Bunun üzerine Barak ardinda on bin kisiyle Tavor Dagi'ndan indi.
\par 15 RAB, Sisera'yi, savas arabalarini sürenleri ve ordusunu Barak'in önünde saskina çevirerek bozguna ugratti. Sisera savas arabasindan indi ve yaya olarak kaçti.
\par 16 Barak savas arabalarini ve orduyu Haroset-Goyim'e kadar kovaladi. Sisera'nin bütün ordusu kiliçtan geçirildi, tek kisi bile kurtulamadi.
\par 17 Yaya olarak kaçan Sisera ise Kenliler'den Hever'in karisi Yael'in çadirina sigindi. Çünkü Hasor Krali Yavin'le Kenliler'den Hever'in arasi iyiydi.
\par 18 Yael Sisera'yi karsilamaya çikti. Ona, "Korkma, efendim, gel çadirima sigin" dedi. Çadirina siginan Sisera'nin üzerine bir yorgan örttü.
\par 19 Sisera, "Susadim, lütfen biraz su ver de içeyim" dedi. Yael süt tulumunu açip ona içirdikten sonra üzerini yine örttü.
\par 20 Sisera kadina, "Çadirin kapisinda dur" dedi, "Biri gelir de çadirda kimse var mi diye sorarsa, yok de."
\par 21 Hever'in karisi Yael eline bir çadir kazigi ile tokmak aldi. Yorgunluktan derin bir uykuya dalmis olan Sisera'ya sessizce yaklasarak kazigi sakagina dayadi ve yere saplanincaya dek çakti. Sisera hemen öldü.
\par 22 Yael Sisera'yi kovalayan Barak'i karsilamaya çikti. "Gel, aradigin adami sana göstereyim" dedi. Barak kadini izledi ve sakagina kazik çakilmis Sisera'yi ölü buldu.
\par 23 Böylece Tanri o gün Kenanli kral Yavin'i Israilliler'in önünde bozguna ugratti.
\par 24 Giderek güçlenen Israilliler sonunda Kenanli kral Yavin'i ortadan kaldirdilar.

\chapter{5}

\par 1 Debora ile Avinoam oglu Barak o gün su ezgiyi söylediler:
\par 2 "Israil'in önderleri basi çekince, Halk gönüllü olarak savasinca RAB'be övgüler sunun.
\par 3 Dinleyin, ey krallar! Ey yönetenler, kulak verin! RAB'be ezgiler söyleyip Israil'in Tanrisi RAB'bi ilahilerle övecegim.
\par 4 Seir'den çiktiginda, ya RAB, Edom kirlarindan geçtiginde, Yer sarsildi, göklerden yagmur bosandi, Evet, bulutlar yagmur yagdirdi.
\par 5 Sina Dagi'nda olan RAB'bin, Israil'in Tanrisi RAB'bin önünde Daglar sarsildi.
\par 6 Anat oglu Samgar zamaninda, Yael zamaninda kervanlarin ardi kesildi. Yolcular sapa yollardan gider oldu.
\par 7 Bombostu Israil'in köyleri, Ben Israil'de ana olarak ortaya çikincaya dek, Ben Debora ortaya çikincaya dek Israil'in köyleri bombostu.
\par 8 Yeni ilahlar seçtikleri zaman Savas kentin kapilarina dayandi. Israil'deki kirk bin askerin elinde Ne kalkan ne de mizrak vardi.
\par 9 Yüregim Israil'i yönetenlerle Ve halkin arasindaki gönüllülerledir. RAB'be övgüler sunun!
\par 10 Ey semerleri pahali boz eseklere binenler, Ey yoldan yaya gidenler, dinleyin!
\par 11 Kuyu basindaki kalabaliklar RAB'bin zaferlerini, Israil savasçilarinin zaferlerini anlatiyorlar. Ardindan RAB'bin halki kent kapilarina Akin etmeye basladi.
\par 12 Uyan, uyan Debora, uyan uyan! Söyle, ezgiler söyle! Ey Avinoam oglu Barak, Kalk, götür tutsaklarini.
\par 13 Geriye kalanlar soylularin yanina geldi, RAB'bin halki yigitleriyle bana geldi.
\par 14 Amalek kökünden olanlar Efrayim'den geldi, Benyaminliler de seni izleyenlerin arasindaydi. Yöneticiler Makir'den, Basbug asasini tasiyanlar Zevulun'dan geldi.
\par 15 Debora'yla birlikteydi Issakar'in beyleri. Evet, Issakarogullari da Barak'in ardindan Hizla ovaya indi. Ama Ruben oymaginin bölükleri Büyük bir kararsizlik içindeydi.
\par 16 Sürülerine kaval çalan çobanlari Dinlemek için neden agillarda kaldilar? Evet, Ruben oymaginin bölükleri Büyük bir kararsizlik içindeydi.
\par 17 Gilatlilar Seria Irmagi'nin ötesinde kaldi, Dan oymagiysa gemilerde oyalandi. Aser oymagi deniz kiyisinda dinlendi, Koylarda yan gelip oturdu.
\par 18 Ama Zevulun ve Naftali halklari Tehlikeye attilar canlarini savas alaninda.
\par 19 Taanak'ta ve Megiddo sularinin kiyisinda Krallar gelip savastilar. Kenan krallari da savasti. Ancak ne gümüs ne ganimet aldilar.
\par 20 Yildizlar göklerden savasa katildi. Gögü bir bastan öbür basa geçerken, Sisera'ya karsi savasti.
\par 21 Kison Irmagi, o eski irmak, Süpürüp götürdü onlari. Yürü, ey ruhum, üzerlerine güçle yürü!
\par 22 O zaman atlar dörtnala kostu. Güçlü atlarin toynaklari Yerde izler birakti.
\par 23 RAB'bin melegi, 'Meroz Kenti'ni lanetleyin dedi, 'Halkina lanetler yagdirin. Çünkü RAB'bin yardimina, Zorbalara karsi RAB'bin yardimina kosmadilar.
\par 24 Kenliler'den Hever'in karisi Yael Kadinlar arasinda alabildigine kutsansin. Çadirlarda yasayan kadinlar arasinda Alabildigine kutsansin.
\par 25 Sisera su istedi, Yael ona süt verdi. Soylulara yarasir bir çanakla ayran sundu.
\par 26 Sol eline çadir kazigini, Sag eline isçi tokmagini aldi. Vurdu, Sisera'nin basini ezdi. Sakagina çakti kazigi, deldi geçirdi.
\par 27 Ayaklarinin dibine çöktü, Yere serildi Sisera. Düsüp yigildi Yael'in ayaklari dibine, Yigildigi yerde cansiz kaldi.
\par 28 Sisera'nin annesi parmakliklarin ardindan, Pencereden bakip feryat etti: 'Oglumun savas arabasi Neden bu kadar gecikti, Nal sesleri neden duyulmuyor?
\par 29 Bilge kadinlar onu yanitladilar. O da söyle düsündü:
\par 30 'Ganimeti bulmus, paylasiyor olmalilar. Her yigide bir ya da iki kiz, Sisera'ya ganimet olarak rengarenk giysiler, Evet, islemeli, rengarenk giysiler. Yagmacilarin boyunlari için Iki yani islemeli renkli giysiler, Hepsi ganimet.
\par 31 Ya RAB, bütün düsmanlarin böyle yok olsun. Seni sevenlerse, Bütün gücüyle dogan günes gibi olsunlar." Bundan sonra ülke kirk yil baris içinde yasadi.

\chapter{6}

\par 1 Israilliler yine RAB'bin gözünde kötü olani yaptilar. RAB de onlari yedi yil süreyle Midyanlilar'in eline teslim etti.
\par 2 Midyan boyundurugu Israilliler'e öyle agir geldi ki, daglarda kendilerine siginaklar, magaralar, kaleler yaptilar.
\par 3 Ekin ektikleri vakit, Midyanlilar, Amalekliler ve öbür dogulu halklar topraklarina girip
\par 4 ordugah kurarlardi. Gazze'ye dek ekinleri yok eder, koyun, sigir, esek gibi geçim kaynagi olan her seyi alirlardi.
\par 5 Hayvanlari ve çadirlariyla birlikte çekirge sürüsü gibi gelirlerdi. Adamlari, develeri saymak olanaksizdi. Yakip yikmak amaciyla topraklari isgal ederlerdi.
\par 6 Midyanlilar Israil'i öyle yoksul düsürdüler ki, Israilliler RAB'be yakarmaya basladilar.
\par 7 Israilliler Midyanlilar'dan ötürü RAB'be yakarinca,
\par 8 RAB onlara bir peygamber gönderdi. Peygamber onlara söyle dedi: "Israil'in Tanrisi RAB diyor ki, 'Sizi Misir'dan ben çikardim, köle oldugunuz ülkeden ben getirdim.
\par 9 Misirlilar'in elinden, size baski yapanlarin hepsinin elinden sizi ben kurtardim. Onlari önünüzden kovdum, topraklarini size verdim.
\par 10 Size dedim ki, Ben Tanriniz RAB'bim. Topraklarinda yasadiginiz Amorlular'in ilahlarina tapmayin. Ama sözümü dinlemediniz."
\par 11 RAB'bin melegi gelip Aviezerli Yoas'in Ofra Kenti'ndeki yabanil fistik agacinin altinda oturdu. Yoas'in oglu Gidyon, bugdayi Midyanlilar'dan kurtarmak için üzüm sikma çukurunda dövüyordu.
\par 12 RAB'bin melegi ona görünerek, "Ey yigit savasçi, RAB seninledir" dedi.
\par 13 Gidyon, "Ey Efendim, eger RAB bizimleyse bütün bunlar neden basimiza geldi?" diye karsilik verdi, "Atalarimiz RAB'bin bizi Misir'dan çikardigini söylemediler mi? Bize anlattiklari RAB'bin bütün o harikalari nerede? RAB bizi terk etti, Midyanlilar'in eline teslim etti."
\par 14 RAB Gidyon'a dönüp, "Kendi gücünle git, Israil'i Midyanlilar'in elinden kurtar" dedi, "Seni ben gönderiyorum."
\par 15 Gidyon, "Ey Efendim, ben Israil'i nasil kurtarabilirim?" diye karsilik verdi, "Ait oldugum boy Manasse oymaginin en zayif boyudur. Ben de ailemin en genç adamiyim."
\par 16 RAB, "Ben seninle olacagim" dedi, "Midyanlilar'i tek bir adami yener gibi bozguna ugratacaksin."
\par 17 Gidyon, "Benden hosnutsan, benimle konusanin sen olduguna dair bana bir belirti göster" dedi,
\par 18 "Lütfen gelip sana adagimi sununcaya, önüne koyuncaya dek buradan ayrilma." RAB, "Sen dönünceye dek kalirim" diye yanitladi.
\par 19 Gidyon eve gidip bir oglak kesti, bir efa undan mayasiz pide yapti. Eti sepete, et suyunu tencereye koydu; bunlari getirip yabanil fistik agacinin altinda melege sundu.
\par 20 Tanri'nin melegi, "Eti ve mayasiz pideleri al, su kayanin üzerine koy. Et suyunu ise dök" dedi. Gidyon söyleneni yapti.
\par 21 RAB'bin melegi elindeki degnegin ucuyla ete ve mayasiz pidelere dokununca kayadan ates fiskirdi. Ates eti ve mayasiz pideleri yakip kül etti. Sonra RAB'bin melegi gözden kayboldu.
\par 22 Gidyon, gördügü kisinin RAB'bin melegi oldugunu anlayinca, "Eyvah, Egemen RAB! Meleginin yüzünü gördüm" dedi.
\par 23 RAB ona, "Sana esenlik olsun. Korkma, ölmeyeceksin" dedi.
\par 24 Gidyon orada RAB için bir sunak yapti. Sunaga "RAB esenliktir" adini verdi. Sunak bugün de Aviezerliler'in Ofra Kenti'nde duruyor.
\par 25 Ayni gece RAB, Gidyon'a, "Babanin bogasini, yedi yasindaki ikinci bogayi al" dedi, "Sonra babanin Baal* için yaptirdigi sunagi yik. Sunagin yanindaki Asera* putunu kes.
\par 26 Tanrin RAB için bu höyügün üstünde uygun bir sunak yap. Ikinci bogayi al, kesecegin Asera putunun odunlariyla yakmalik sunu* olarak sun."
\par 27 Gidyon adamlarindan onunu yanina alarak RAB'bin kendisine buyurduklarini yerine getirdi. Ne var ki, ailesinden ve kent halkindan korktugu için bunu gündüz yerine gece yapti.
\par 28 Sabah erkenden kalkan kent halki, Baal'a ait sunagin yikildigini, yanindaki Asera putunun kesildigini, ikinci boganin yeni yapilan sunak üzerinde sunuldugunu gördü.
\par 29 Birbirlerine, "Bu isi kim yapti?" diye sordular. Arastirip sorusturduktan sonra, bu isi Yoas oglu Gidyon'un yaptigini anladilar.
\par 30 Bunun üzerine Yoas'a, "Oglunu disari çikar" dediler, "Ölmesi gerek. Çünkü Baal'in sunagini yikti, yanindaki Asera putunu kesti."
\par 31 Yoas çevresindeki öfkeli kalabaliga, "Baal'i savunmak size mi düstü?" dedi, "Siz mi onu kurtaracaksiniz? Onu savunan safak sökmeden ölecek. Baal tanriysa, birakin kendini savunsun. Yikilan sunak onun!"
\par 32 O gün Yoas, "Baal kendini savunsun, yikilan sunak onun sunagidir" diyerek Gidyon'a Yerubbaal adini verdi.
\par 33 Bu arada Midyanlilar, Amalekliler ve öbür dogulu halklar birleserek Seria Irmagi'ni geçtiler, gidip Yizreel Vadisi'nde ordugah kurdular.
\par 34 RAB'bin Ruhu Gidyon'u yönlendirmeye basladi. Gidyon borusunu çalinca Aviezerliler onun çevresinde toplandi.
\par 35 Gidyon bütün Manasse'ye ulaklar göndererek oranin halkini da topladi. Aser, Zevulun ve Naftali'ye de ulaklar gönderdi. Onlar da onu karsilamaya çiktilar.
\par 36 Gidyon Tanri'ya söyle seslendi: "Söz verdigin gibi Israil'i benim araciligimla kurtaracagin dogruysa,
\par 37 çiy yalnizca harman yerine koydugum yün yapaginin üzerine düssün, topraksa kuru kalsin. Böylece, söyledigin gibi Israil'i benim araciligimla kurtaracagini bilecegim."
\par 38 Ve öyle oldu. Ertesi gün erkenden kalkan Gidyon yapagiyi alip sikti. Yapagidan bir tas dolusu çiy süzüldü.
\par 39 Bunun üzerine Gidyon Tanri'ya söyle seslendi: "Bana kizma, bir istekte daha bulunmak istiyorum. Yapagiyla bir deneme daha yapmama izin ver. Lütfen bu kez yalnizca yapagi kuru kalsin, topraksa çiyle islansin."
\par 40 Tanri o gece Gidyon'un dedigini yapti. Yapagi kuru kaldi, topragin her yaniysa çiyle kaplandi.

\chapter{7}

\par 1 Yerubbaal -Gidyon- ile yanindaki halk erkenden kalkip Harot Pinari'nin basinda ordugah kurdular. Midyanlilar'in ordugahiysa onlarin kuzeyinde, More Tepesi'nin yanindaki vadideydi.
\par 2 RAB Gidyon'a söyle dedi: "Yaninda fazla adam var; Midyan'i onlarin eline teslim etmem. Yoksa Israilliler, 'Kendi gücümüzle kurtulduk diyerek bana karsi övünebilirler.
\par 3 Simdi halka sunu söyle: 'Korkudan titreyen dönsün, Gilat Dagi'ndan geri gitsin." Bunun üzerine halktan yirmi iki bin kisi döndü, on bin kisi orada kaldi.
\par 4 RAB Gidyon'a, "Adamlarin sayisi hâlâ fazla" dedi, "Kalanlari suyun basina götür, onlari orada senin için sinayayim. 'Bu seninle gidecek dedigim adam seninle gidecek; 'Bu seninle gitmeyecek dedigim gitmeyecek."
\par 5 Gidyon halki suyun basina götürdü. RAB Gidyon'a, "Köpek gibi diliyle su içenleri bir yana, su içmek için dizleri üzerine çökenleri öbür yana ayir" dedi.
\par 6 Ellerini agizlarina götürerek dilleriyle su içenlerin sayisi üç yüzü buldu. Geri kalanlarin hepsi su içmek için dizleri üzerine çöktüler.
\par 7 RAB Gidyon'a, "Sizi diliyle su içen üç yüz kisinin eliyle kurtaracagim" dedi, "Midyanlilar'i senin eline teslim edecegim. Öbürleri yerlerine dönsün."
\par 8 Gidyon yalniz üç yüz kisiyi alikoyarak geri kalan Israilliler'i çadirlarina gönderdi. Bu üç yüz kisi, gidenlerin kumanyalariyla borularini da aldilar. Midyanlilar'in ordugahi Gidyon'un asagisinda, vadideydi.
\par 9 RAB ayni gece Gidyon'a, "Kalk, ordugaha saldir" dedi, "Çünkü orayi senin eline teslim ediyorum.
\par 10 Ordugaha yalniz gitmekten korkuyorsan, usagin Pura'yi da yanina al.
\par 11 Midyanlilar'in söylediklerine kulak kabart. O zaman ordugahlarina saldirmaya cesaret bulursun." Böylece Gidyon usagi Pura ile ordugahin yanina kadar sokuldu.
\par 12 Midyanlilar, Amalekliler ve öbür dogulu halklar çekirge sürüsü gibi vadiye yayilmislardi. Kiyilarin kumu kadar çok, sayisiz develeri vardi.
\par 13 Gidyon ordugahin yanina vardiginda, adamlardan biri arkadasina gördügü düsü anlatiyordu. "Bir düs gördüm" diyordu, "Arpa unundan yapilmis bir somun ekmek, Midyan ordugahina dogru yuvarlanarak çadira kadar geldi, çadira çarpip onu devirdi, altüst etti. Çadir yerle bir oldu."
\par 14 Adamin arkadasi söyle karsilik verdi: "Bu, Israilli Yoas oglu Gidyon'un kilicindan baska bir sey degildir. Tanri Midyan'i ve bütün ordugahi onun eline teslim edecek."
\par 15 Gidyon düsü ve yorumunu duyunca Tanri'ya tapindi. Israil ordugahina döndü ve adamlarina, "Kalkin! RAB Midyan ordugahini elinize teslim etti" dedi.
\par 16 Sonra üç yüz adamini üç bölüge ayirdi. Hepsine borular, bos testiler ve testilerin içinde yakilmak üzere çiralar verdi.
\par 17 Onlara, "Gözünüz bende olsun" dedi, "Ben ne yaparsam siz de onu yapin. Ordugahin yanina vardigimda ne yaparsam siz de aynisini yapin.
\par 18 Ben ve yanimdakiler borularimizi çalinca, siz de ordugahin çevresinde durup borularinizi çalin ve, 'RAB için ve Gidyon için! diye bagirin."
\par 19 Gidyon ile yanindaki yüz kisi gece yarisindan az önce, nöbetçi degisiminden hemen sonra ordugahin yanina vardilar; borularini çalmaya baslayip ellerindeki testileri kirdilar.
\par 20 Üç bölük de borularini çalip testileri kirdi. Çalacaklari borulari sag ellerinde, çiralariysa sol ellerinde tutuyorlardi. "Yasasin RAB'bin ve Gidyon'un kilici!" diye bagirdilar.
\par 21 Onlar ordugahin çevresinde dururken, ordugahtakilerin hepsi kosusmaya, bagirip kaçismaya basladi.
\par 22 Üç yüz boru birden çalinca RAB ordugahtakilerin hepsini kiliçla birbirlerine saldirtti. Midyan ordusu Serera'ya dogru, Beytsitta'ya, Tabbat yakinindaki Avel-Mehola sinirina dek kaçti.
\par 23 Naftali, Aser ve bütün Manasse'den çagrilan Israilliler Midyanlilar'i kovalamaya basladilar.
\par 24 Gidyon, Efrayim'in daglik bölgesine gönderdigi ulaklar araciligiyla, "Inip Midyanlilar'a saldirin" dedi, "Önlerini kesmek için Seria Irmagi'nin Beytbara'ya kadar uzanan bölümünü tutun." Efrayimogullari Seria Irmagi'nin Beytbara'ya kadarki bölümünü ele geçirdiler.
\par 25 Midyanli iki önderi, Orev ile Zeev'i tutsak aldilar. Orev'i Orev Kayasi'nda, Zeev'i ise Zeev'in üzüm sikma çukurunda öldürerek Midyanlilar'i kovalamaya devam ettiler. Orev'le Zeev'in kesik baslarini Seria Irmagi'nin karsi yakasindan Gidyon'a getirdiler.

\chapter{8}

\par 1 Efrayimogullari Gidyon'a, "Midyanlilar'la savasmaya gittiginde bizi çagirmadin; bize neden böyle davrandin?" diyerek onu sert bir dille elestirdiler.
\par 2 Gidyon, "Sizin yaptiginizin yaninda benim yaptigim ne ki?" diye karsilik verdi, "Efrayim'in bagbozumundan artakalan üzümler, Aviezer'in bütün bagbozumu ürününden daha iyi degil mi?
\par 3 Tanri Midyan önderlerini, Orev'i ve Zeev'i elinize teslim etti. Sizin yaptiklariniza kiyasla ben ne yapabildim ki?" Gidyon'un bu sözleri onlarin öfkesini yatistirdi.
\par 4 Gidyon bitkin olmalarina karsin Midyanlilar'i kovalamayi sürdüren üç yüz adamiyla Seria Irmagi'na ulasip karsiya geçti.
\par 5 Sukkot'a vardiklarinda kent halkina, "Lütfen ardimdaki adamlara ekmek verin, bitkin haldeler" dedi, "Ben Midyan krallari Zevah ve Salmunna'yi kovaliyorum."
\par 6 Sukkot önderleri, "Zevah ile Salmunna'yi tutsak aldin mi ki, orduna ekmek verelim?" dediler.
\par 7 Gidyon, "Öyle olsun!" diye karsilik verdi, "RAB Zevah ile Salmunna'yi elime teslim edince, bedenlerinizi çöl dikenleriyle, çalilarla yaracagim."
\par 8 Gidyon oradan Penuel'e gitti ve oranin halkindan da ayni seyi istedi. Penuel halki da Sukkot halkinin verdigi yanitin aynisini verdi.
\par 9 Gidyon onlara, "Esenlik içinde döndügüm zaman bu kuleyi yikacagim" dedi.
\par 10 Zevah ile Salmunna dogulu halklarin ordularindan artakalan yaklasik on bes bin kisilik bir orduyla birlikte Karkor'daydilar. Eli kiliç tutan yüz yirmi bin savasçi ölmüstü.
\par 11 Gidyon Novah ve Yogboha'nin dogusundan, göçebelerin yolundan geçerek düsman ordugahina saldirdi. Adamlar hazirliksiz yakalandilar.
\par 12 Zevah ile Salmunna kaçtiysa da Gidyon peslerine düstü. Bu iki Midyan kralini, Zevah ile Salmunna'yi yakalayip bütün ordularini bozguna ugratti.
\par 13 Yoas oglu Gidyon Heres Geçidi yoluyla savastan döndü.
\par 14 Yolda Sukkot'tan genç bir adami yakalayip sorguya çekti. Adam Sukkot önderleriyle ileri gelenlerinin adlarini, toplam yetmis yedi kisinin adini yazip Gidyon'a verdi.
\par 15 Gidyon Sukkot'a gidip halka söyle dedi: "'Zevah ile Salmunna'yi tutsak aldin mi ki bitkin adamlarina ekmek verelim diyerek beni asagiladiniz. Iste Zevah ile Salmunna!"
\par 16 Sonra kentin ileri gelenlerini topladi; Sukkot halkini çöl dikenleriyle, çalilarla döverek cezalandirdi.
\par 17 Ardindan Penuel Kulesi'ni yikip kent halkini kiliçtan geçirdi.
\par 18 Sonra Zevah ile Salmunna'ya, "Tavor'da öldürdükleriniz nasil adamlardi?" diye sordu. "Tipki senin gibiydiler, hepsi kral ogullarina benziyordu" yanitini verdiler.
\par 19 Gidyon, "Onlar kardeslerimdi, öz annemin ogullariydi" dedi, "Yasayan RAB'bin adiyla ant içerim ki, onlari sag biraksaydiniz sizi öldürmezdim."
\par 20 Sonra büyük oglu Yeter'e, "Haydi, öldür onlari" dedi. Ne var ki, henüz genç olan Yeter korktu, kilicini çekmedi.
\par 21 Bunun üzerine Zevah ile Salmunna Gidyon'a, "Sen öldür bizi" dediler, "Erkegin isini ancak erkek yapar." Böylece Gidyon varip Zevah ile Salmunna'yi öldürdü. Develerinin boyunlarindaki hilal biçimi süsleri de aldi.
\par 22 Israilliler Gidyon'a, "Sen, oglun ve torunun bize önderlik edin" dediler. "Çünkü bizi Midyanlilar'in elinden sen kurtardin."
\par 23 Ama Gidyon, "Ben size önderlik etmem, oglum da etmez" diye karsilik verdi, "Size RAB önderlik edecek."
\par 24 Sonra, "Yalniz sizden bir dilegim var" diye sözünü sürdürdü, "Ele geçirdiginiz ganimetin içindeki küpeleri bana verin." -Ismaililer altin küpeler takarlardi.-
\par 25 Israilliler, "Seve seve veririz" diyerek yere bir üstlük serdiler. Herkes ele geçirdigi küpeleri üstlügün üzerine atti.
\par 26 Hilaller, kolyeler, Midyan krallarinin giydigi mor giysiler ve develerin boyunlarindan alinan zincirler disinda, Gidyon'un aldigi altin küpelerin agirligi bin yedi yüz sekel tuttu.
\par 27 Gidyon bu altindan bir efod* yaparak onu kendi kenti olan Ofra'ya yerlestirdi. Bütün Israilliler bu put yüzünden RAB'be vefasizlik ettiler. Böylece efod Gidyon ile ailesi için bir tuzak oldu.
\par 28 Israilliler'e yenilen Midyanlilar bir daha toparlanamadilar. Ülke Gidyon zamaninda kirk yil baris içinde yasadi.
\par 29 Yoas oglu Yerubbaal dönüp kendi evinde yasamini sürdürdü.
\par 30 Çok sayida kadinla evlendi ve yetmis oglu oldu.
\par 31 Ayrica Sekem'de bir cariyesi vardi. Bundan da bir oglu oldu, adini Avimelek koydu.
\par 32 Yoas oglu Gidyon iyice yaslanip öldü. Aviezerliler'e ait Ofra Kenti'nde, babasi Yoas'in mezarina gömüldü.
\par 33 Gidyon ölünce Israilliler yine RAB'be vefasizlik ettiler. Baallar'a* taptilar. Baal-Berit'i ilah edinerek
\par 34 kendilerini çevrelerindeki düsmanlarinin elinden kurtaran Tanrilari RAB'bi unuttular.
\par 35 Israil'e büyük iyilikler yapan Yerubbaal'in -Gidyon'un- ev halkina vefasizlik ettiler.

\chapter{9}

\par 1 Yerubbaal'in oglu Avimelek, dayilarinin bulundugu Sekem Kenti'ne giderek onlara ve annesinin boyundan gelen herkese söyle dedi:
\par 2 "Sekem halkina sunu duyurun: 'Sizin için hangisi daha iyi? Gidyon'un yetmis oglu tarafindan yönetilmek mi, yoksa bir kisi tarafindan yönetilmek mi? Unutmayin ki ben sizinle ayni etten, ayni kandanim."
\par 3 Dayilari Avimelek'in söylediklerini Sekem halkina ilettiler. Halkin yüregi Avimelek'ten yanaydi. "O bizim kardesimizdir" dediler.
\par 4 Ona Baal-Berit Tapinagi'ndan yetmis parça gümüs verdiler. Avimelek bu parayla kiraladigi belali serserileri pesine takti.
\par 5 Sonra Ofra'ya, babasinin evine dönüp kardeslerini, Yerubbaal'in yetmis oglunu bir tasin üzerinde kesip öldürdü. Yalniz Yerubbaal'in küçük oglu Yotam kaçip gizlendigi için sag kaldi.
\par 6 Sekem ve Beytmillo halklari toplanarak hep birlikte Sekem'de dikili tas mesesinin oldugu yere gittiler; Avimelek'i orada kral ilan ettiler.
\par 7 Olup biteni Yotam'a bildirdiklerinde Yotam Gerizim Dagi'nin tepesine çikip yüksek sesle halka söyle dedi: "Ey Sekem halki, beni dinleyin, Tanri da sizi dinleyecek.
\par 8 Bir gün agaçlar kendilerine bir kral meshetmek* istediler; zeytin agacina gidip, 'Gel kralimiz ol dediler.
\par 9 "Zeytin agaci, 'Ilahlari ve insanlari onurlandirmak için kullanilan yagimi birakip agaçlar üzerinde sallanmaya mi gideyim? diye yanitladi.
\par 10 "Bunun üzerine agaçlar incir agacina, 'Gel sen kralimiz ol dediler.
\par 11 "Incir agaci, 'Tatliligimi ve güzel meyvemi birakip agaçlar üzerinde sallanmaya mi gideyim? diye yanitladi.
\par 12 "Sonra agaçlar asmaya, 'Gel sen bizim kralimiz ol dediler.
\par 13 Asma, 'Ilahlarla insanlara zevk veren yeni sarabimi birakip agaçlar üzerinde sallanmaya mi gideyim? dedi.
\par 14 "Sonunda agaçlar karaçaliya, 'Gel sen kralimiz ol dediler.
\par 15 "Karaçali, 'Eger gerçekten beni kendinize kral meshetmek istiyorsaniz, gelin gölgeme siginin diye karsilik verdi, 'Eger siginmazsaniz, karaçalidan çikan ates Lübnan'in bütün sedir agaçlarini yakip kül edecektir.
\par 16 "Simdi siz Avimelek'i kral yapmakla içten ve dürüst davrandiginizi mi saniyorsunuz? Yerubbaal'la ailesine iyilik mi ettiniz? Ona hak ettigi gibi mi davrandiniz?
\par 17 Oysa babam sizi Midyanlilar'in elinden kurtarmak için canini tehlikeye atarak sizin için savasti.
\par 18 Ama bugün siz babamin ailesine karsi ayaklandiniz, yetmis oglunu bir tasin üzerinde kesip öldürdünüz. Cariyesinden dogan Avimelek kardesiniz oldugu için onu Sekem'e kral yaptiniz.
\par 19 Eger bugün Yerubbaal'la ailesine içten ve dürüst davrandiginiza inaniyorsaniz, Avimelek'le sevinin, o da sizinle sevinsin!
\par 20 Ama öyle degilse, dilerim, Avimelek ates olsun, Sekem ve Beytmillo halkini yakip kül etsin. Ya da Sekem ve Beytmillo halki ates olsun, Avimelek'i yakip kül etsin."
\par 21 Ardindan Yotam kardesi Avimelek'ten korktugu için kaçti, gidip Beer'e yerlesti.
\par 22 Avimelek Israil'i üç yil yönetti.
\par 23 Sonra Tanri Avimelek'le Sekem halkini birbirine düsürdü; halk Avimelek'e baskaldirdi.
\par 24 Tanri bunu Avimelek'i Yerubbaal'in yetmis ogluna yapilan zorbaligin aynisina ugratmak, kardeslerini öldüren Avimelek'ten ve onu bu kirima isteklendiren Sekem halkindan akittiklari kanin öcünü almak için yapti.
\par 25 Sekem halki dag baslarinda Avimelek'e pusu kurdu. Oradan geçen herkesi soyuyorlardi. Bu durum Avimelek'e bildirildi.
\par 26 Ebet oglu Gaal kardesleriyle birlikte gelip Sekem'e yerlesti. Sekem halki ona güvendi.
\par 27 Baglara çikip üzümleri topladiktan, ezip sarap yaptiktan sonra bir senlik düzenlediler. Ilahlarinin tapinagina gittiler; orada yiyip içerken Avimelek'e lanetler yagdirdilar.
\par 28 Ebet oglu Gaal kalkip söyle dedi: "Avimelek kim ki, biz Sekem halki ona hizmet edelim? Yerubbaal'in oglu degil mi o? Zevul da onun yardimcisi degil mi? Sekemliler'in babasi Hamor'un soyundan gelenlere hizmet edin. Neden Avimelek'e hizmet edelim?
\par 29 Keske bu halki ben yönetseydim! Avimelek'i uzaklastirir ve, 'Ordunu güçlendir de öyle ortaya çik! derdim."
\par 30 Kentin yöneticisi olan Zevul, Ebet oglu Gaal'in sözlerini duyunca öfkelendi.
\par 31 Avimelek'e gizlice gönderdigi ulaklar araciligiyla söyle dedi: "Ebet oglu Gaal ve kardesleri Sekem'e geldiler. Kenti sana karsi ayaklandiriyorlar.
\par 32 Gel, adamlarinla birlikte gece kirda pusuya yat.
\par 33 Sabah günes dogar dogmaz kalk, kenti bas. Gaal ile adamlari sana saldirdiginda onlara yapacagini yap."
\par 34 Böylece Avimelek'le adamlari gece kalkip dört bölük halinde Sekem yakininda pusuya yattilar.
\par 35 Ebet oglu Gaal çikip kentin giris kapisinda durunca, Avimelek'le yanindakiler pusu yerinden firladilar.
\par 36 Gelenleri gören Gaal, Zevul'a, "Daglarin tepesinden inip gelenlere bak!" dedi. Zevul, "Adam sandigin aslinda daglarin gölgesidir" diye karsilik verdi.
\par 37 Ama Gaal israr etti: "Bak, topraklarimizin ortasinda ilerleyenler var. Bir kismi da Falcilar Mesesi yolundan geliyor."
\par 38 Bunun üzerine Zevul, "'Avimelek kim ki, ona hizmet edelim diye övünen sen degil miydin?" dedi, "Küçümsedigin halk bu degil mi? Haydi simdi git, onlarla savas!"
\par 39 Sekem halkina öncülük eden Gaal, Avimelek'le savasa tutustu.
\par 40 Ama tutunamayip kaçmaya basladi. Avimelek ardina düstü. Kentin giris kapisina dek çok sayida ölü yerde yatiyordu.
\par 41 Avimelek Aruma'da kaldi. Zevul ise Gaal'i ve kardeslerini Sekem'den kovdu, kentte yasamalarina izin vermedi.
\par 42 Savasin ertesi günü Avimelek Sekemliler'in tarlalarina gittiklerini haber aldi.
\par 43 Adamlarini üç bölüge ayirip kirda pusuya yatti. Halkin kentten çiktigini görünce saldirip onlari öldürdü.
\par 44 Sonra yanindaki bölükle hizla ilerleyerek kentin giris kapisina dayandi. Öbür iki bölükse tarlalardakilere saldirip onlari öldürdü.
\par 45 Avimelek gün boyu kente karsi savasti; kenti ele geçirdikten sonra halkini kiliçtan geçirdi. Kenti yikip üstüne tuz serpti.
\par 46 Sekem Kulesi'ndeki halk olup biteni duyunca, El-Berit Tapinagi'nin kalesine sigindi.
\par 47 Onlarin Sekem Kulesi'nde toplandigini haber alan Avimelek,
\par 48 yanindaki halkla birlikte Salmon Dagi'na çikti. Eline bir balta alip agaçtan bir dal kesti, dali omuzuna atarak yanindakilere, "Ne yaptigimi gördünüz" dedi, "Çabuk olun, siz de benim gibi yapin."
\par 49 Böylece hepsi birer dal kesip Avimelek'i izledi. Dallari kalenin dibinde yigip atese verdiler. Sekem Kulesi'ndeki bin kadar kadin, erkek yanarak öldü.
\par 50 Bundan sonra Avimelek Teves üzerine yürüdü, kenti kusatip ele geçirdi.
\par 51 Kentin ortasinda saglam bir kule vardi. Kadin erkek bütün kent halki oraya sigindi. Kapilari kapayip kulenin damina çiktilar.
\par 52 Avimelek gelip kuleyi kusatti. Atese vermek için kapisina yaklastiginda,
\par 53 bir kadin degirmenin üst tasini Avimelek'in üzerine atip basini yardi.
\par 54 Avimelek hemen silahlarini tasiyan usagini çagirdi ve, "Kilicini çek, beni öldür" dedi, "Hiç kimse, 'Avimelek'i bir kadin öldürdü demesin." Usak kilicini Avimelek'e saplayip onu öldürdü.
\par 55 Avimelek'in öldügünü görünce Israilliler evlerine döndüler.
\par 56 Böylece Tanri yetmis kardesini öldürerek babasina büyük kötülük eden Avimelek'i cezalandirdi.
\par 57 Tanri Sekem halkini da yaptiklari kötülüklerden ötürü cezalandirdi. Yerubbaal'in oglu Yotam'in lanetine ugradilar.

\chapter{10}

\par 1 Avimelek'in ölümünden sonra Israil'i kurtarmak için Issakar oymagindan Dodo oglu Pua oglu Tola adinda bir adam ortaya çikti. Tola Efrayim'in daglik bölgesindeki Samir'de yasardi.
\par 2 Israil'i yirmi üç yil yönettikten sonra öldü, Samir'de gömüldü.
\par 3 Ondan sonra Gilatli Yair basa geçti. Yair Israil'i yirmi iki yil yönetti.
\par 4 Otuz oglu vardi. Bunlar otuz esege biner, otuz kenti yönetirlerdi. Gilat yöresindeki bu kentler bugün de Havvot-Yair diye aniliyor.
\par 5 Yair ölünce Kamon'da gömüldü.
\par 6 Israilliler yine RAB'bin gözünde kötü olani yaptilar; Baallar'a*, Astoretler'e*, Aram, Sayda, Moav, Ammon ve Filist ilahlarina kulluk ettiler. RAB'bi terk ettiler, O'na kulluk etmediler.
\par 7 Bu yüzden Israilliler'e öfkelenen RAB, onlari Filistliler'e ve Ammonlular'a tutsak etti.
\par 8 Bunlar o yildan baslayarak Israilliler'i baski altinda ezdiler; Seria Irmagi'nin ötesinde, Gilat'taki Amorlular ülkesinde yasayan bütün Israilliler'i on sekiz yil baski altinda tuttular.
\par 9 Ammonlular Yahuda, Benyamin ve Efrayim oymaklariyla savasmak için Seria Irmagi'nin ötesine geçtiler. Israil büyük sikinti içindeydi.
\par 10 Israilliler RAB'be, "Sana karsi günah isledik" diye seslendiler, "Seni, Tanrimiz'i terk edip Baallar'a kulluk ettik."
\par 11 RAB, "Sizi Misirlilar'dan, Amorlular'dan, Ammonlular'dan, Filistliler'den kurtaran ben degil miyim?" diye karsilik verdi,
\par 12 "Saydalilar, Amalekliler, Maonlular size baski yaptiklarinda bana yakardiniz, ben de sizi onlarin elinden kurtardim.
\par 13 Sizse beni terk ettiniz, baska ilahlara kulluk ettiniz. Bu yüzden sizi bir daha kurtarmayacagim.
\par 14 Gidin, seçtiginiz ilahlara yakarin; sikintiya düstügünüzde sizi onlar kurtarsin."
\par 15 Israilliler, "Günah isledik" dediler, "Bize ne istersen yap. Yalniz bugün bizi kurtar."
\par 16 Sonra aralarindaki yabanci putlari atip RAB'be tapindilar. RAB de onlarin daha fazla aci çekmesine dayanamadi.
\par 17 Ammonlular toplanip Gilat'ta ordugah kurunca Israilliler de toplanarak Mispa'da ordugah kurdular.
\par 18 Gilat halkinin önderleri birbirlerine, "Ammonlular'a karsi ilk saldiriyi baslatan kisi, bütün Gilat halkinin önderi olacak" dediler.

\chapter{11}

\par 1 Yiftah adinda yigit bir savasçi vardi. Bir fahisenin oglu olan Yiftah'in babasinin adi Gilat'ti.
\par 2 Gilat'in karisi da ona erkek çocuklar dogurmustu. Bu çocuklar büyüyünce Yiftah'i kovmuslardi. Ona, "Babamizin evinden miras almayacaksin. Çünkü sen baska bir kadinin oglusun" demislerdi.
\par 3 Yiftah kardeslerinden kaçip Tov yöresine yerlesti. Çevresinde toplanan serserilere önderlik etmeye basladi.
\par 4 Bir süre sonra Ammonlular Israilliler'e savas açti.
\par 5 Savas patlak verince Gilat ileri gelenleri Yiftah'i almak için Tov yöresine gittiler.
\par 6 Ona, "Gel, komutanimiz ol, Ammonlular'la savasalim" dediler.
\par 7 Yiftah, "Benden nefret eden, beni babamin evinden kovan siz degil miydiniz?" diye yanitladi, "Sikintiya düsünce neden bana geldiniz?"
\par 8 Gilat ileri gelenleri, "Sana basvuruyoruz; çünkü bizimle gelip Ammonlular'la savasmani, bize, Gilat halkina önderlik etmeni istiyoruz" dediler.
\par 9 Yiftah, "Ammonlular'la savasmak için beni götürürseniz, RAB de onlari elime teslim ederse, sizin önderiniz olacak miyim?" diye sordu.
\par 10 Gilat ileri gelenleri, "RAB aramizda tanik olsun, kesinlikle dedigin gibi yapacagiz" dediler.
\par 11 Böylece Yiftah Gilat ileri gelenleriyle birlikte gitti. Halk onu kendine önder ve komutan yapti. Yiftah bütün söylediklerini Mispa'da, RAB'bin önünde yineledi.
\par 12 Sonra Ammon Krali'na ulaklar göndererek, "Aramizda ne var ki, ülkeme saldirmaya kalkiyorsun?" dedi.
\par 13 Ammon Krali, Yiftah'in ulaklarina su karsiligi verdi: "Israilliler Misir'dan çiktiktan sonra Arnon Vadisi'nden Yabbuk ve Seria irmaklarina kadar uzanan topraklarimi aldilar. Simdi buralari bana savassiz geri ver."
\par 14 Yiftah yine Ammon Krali'na ulaklar göndererek
\par 15 söyle dedi: "Yiftah diyor ki, Israilliler ne Moav ülkesini, ne de Ammon topraklarini aldi.
\par 16 Misir'dan çiktiklari zaman Kizildeniz'e* kadar çölde yürüyerek Kades'e ulastilar.
\par 17 Sonra Edom Krali'na ulaklar göndererek, 'Lütfen topraklarindan geçmemize izin ver dediler. Edom Krali kulak asmadi. Israilliler Moav Krali'na da ulaklar gönderdi, ama o da izin vermedi. Bunun üzerine Kades'te kaldilar.
\par 18 "Çölü izleyerek Edom ile Moav topraklarinin çevresinden geçtiler; Moav bölgesinin dogusunda, Arnon Vadisi'nin öbür yakasinda konakladilar. Moav sinirindan içeri girmediler. Çünkü Arnon Vadisi sinirdi.
\par 19 "Sonra Hesbon'da egemenlik süren Amorlular'in Krali Sihon'a ulaklar göndererek, 'Ülkenden geçip topraklarimiza ulasmamiza izin ver diye rica ettiler.
\par 20 Ama Sihon Israilliler'in topraklarindan geçip gideceklerine inanmadi. Bu nedenle bütün halkini toplayip Yahesa'da ordugah kurdu ve Israilliler'le savasa tutustu.
\par 21 "Israil'in Tanrisi RAB, Sihon'u ve bütün halkini Israilliler'in eline teslim etti. Israilliler Amorlular'i yenip o yöredeki halkin bütün topraklarini ele geçirdiler.
\par 22 Arnon Vadisi'nden Yabbuk Irmagi'na, çölden Seria Irmagi'na kadar uzanan bütün Amor topraklarini ele geçirdiler.
\par 23 "Israil'in Tanrisi RAB Amorlular'i kendi halki Israil'in önünden kovduktan sonra, sen hangi hakla buralari geri istiyorsun?
\par 24 Ilahin Kemos sana bir yer verse oraya sahip çikmaz misin? Biz de Tanrimiz RAB'bin önümüzden kovdugu halkin topraklarini sahiplenecegiz.
\par 25 Sen Moav Krali Sippor oglu Balak'tan üstün müsün? O hiç Israilliler'le çekisti mi, hiç onlarla savasmaya kalkisti mi?
\par 26 Israilliler üç yüz yildir Hesbon'da, Aroer'de, bunlarin çevre köylerinde ve Arnon kiyisindaki bütün kentlerde yasarken neden buralari geri almaya çalismadiniz?
\par 27 Ben sana karsi suç islemedim. Ama sen benimle savasmaya kalkismakla bana haksizlik ediyorsun. Hakim olan RAB, Israilliler'le Ammonlular arasinda bugün hakemlik yapsin."
\par 28 Ne var ki Ammon Krali, Yiftah'in kendisine ilettigi bu sözlere kulak asmadi.
\par 29 RAB'bin Ruhu Yiftah'in üzerine indi. Yiftah, Gilat ve Manasse'den geçti, Gilat'taki Mispa'dan geçerek Ammonlular'a dogru ilerledi.
\par 30 RAB'bin önünde ant içerek söyle dedi: "Gerçekten Ammonlular'i elime teslim edersen,
\par 31 onlari yenip sag salim döndügümde beni karsilamak için evimin kapisindan ilk çikan, RAB'be adanacaktir. Onu yakmalik sunu* olarak sunacagim."
\par 32 Yiftah bundan sonra Ammonlular'la savasmaya gitti. RAB onlari Yiftah'in eline teslim etti.
\par 33 Yiftah, basta Avel-Keramim olmak üzere, Aroer'den Minnit'e kadar yirmi kenti yakip yikarak Ammonlular'a çok büyük kayiplar verdirdi. Böylece Ammonlular Israilliler'in boyunduruguna girdi.
\par 34 Yiftah Mispa'ya, kendi evine döndügünde, kizi tef çalip dans ederek onu karsilamaya çikti. Tek çocugu oydu, ondan baska ne oglu ne de kizi vardi.
\par 35 Yiftah, kizini görünce giysilerini yirtarak, "Eyvahlar olsun, kizim!" dedi, "Beni perisan ettin, umarsiz biraktin! Çünkü RAB'be verdigim sözden dönemem."
\par 36 Kiz, "Baba, RAB'be ant içtin" dedi, "Madem RAB düsmanlarin olan Ammonlular'dan senin öcünü aldi, agzindan ne çiktiysa bana öyle yap."
\par 37 Sonra ekledi: "Yalniz bir dilegim var: Beni iki ay serbest birak, gidip arkadaslarimla kirlarda gezineyim, kizligima aglayayim."
\par 38 Babasi, "Gidebilirsin" diyerek onu iki ay serbest birakti. Kiz arkadaslariyla birlikte kirlara çikip erdenligine agladi.
\par 39 Iki ay sonra babasinin yanina döndü. Babasi da içtigi andi yerine getirdi. Kiza erkek eli degmemisti. Bundan sonra Israil'de bir gelenek olustu.
\par 40 Israil kizlari her yil kirlara çikip Gilatli Yiftah'in kizi için dört gün yas tutar oldular.

\chapter{12}

\par 1 Efrayimli erkekler toplanip Safon'a geçtiler. Yiftah'a, "Ammonlular'la savasmaya gittiginde bizi neden çagirmadin?" dediler, "Seni de evini de yakacagiz."
\par 2 Yiftah, "Halkimla ben Ammonlular'a karsi amansiz bir savasa tutusmustuk" diye yanitladi, "Sizi çagirdim, ama gelip beni onlarin elinden kurtarmadiniz.
\par 3 Beni kurtarmak istemediginizi görünce canimi disime takip Ammonlular'a karsi harekete geçtim. Sonunda RAB onlari elime teslim etti. Neden bugün benimle savasmaya kalkisiyorsunuz?"
\par 4 Bundan sonra Yiftah Gilat erkeklerini toplayarak Efrayimogullari'yla savasa girdi. Gilatlilar Efrayimogullari'na saldirdilar. Çünkü Efrayimogullari onlara, "Ey Efrayim ve Manasse halklari arasinda yasayan Gilatlilar, siz Efrayim'den kaçan döneklersiniz!" demislerdi.
\par 5 Seria Irmagi'nin Efrayim'e yol veren geçitlerini tutan Gilatlilar, geçmek isteyen Efrayimli kaçaklara, "Efrayimli misin?" diye sorarlardi. Adamlar, "Hayir" derlerse,
\par 6 o zaman onlara, "'Sibbolet deyin bakalim" derlerdi. Adamlar "Sibbolet" derdi. Çünkü "Sibbolet" sözcügünü dogru söyleyemezlerdi. Bunun üzerine onlari yakalayip Seria Irmagi'nin geçitlerinde öldürürlerdi. O gün Efrayimliler'den kirk iki bin kisi öldürüldü.
\par 7 Gilatli Yiftah Israil'i alti yil yönetti. Ölünce Gilat kentlerinden birinde gömüldü.
\par 8 Ondan sonra Israil'in basina Beytlehemli Ivsan geçti.
\par 9 Ivsan'in otuz oglu, otuz kizi vardi. Ivsan kizlarini baska boylara verdi, ogullarina da baska boylardan kizlar aldi. Israil'i yedi yil yönetti.
\par 10 Ölünce Beytlehem'de gömüldü.
\par 11 Ondan sonra Israil'in basina Zevulun oymagindan Elon geçti. Elon Israil'i on yil yönetti.
\par 12 Ölünce Zevulun topraklarinda, Ayalon'da gömüldü.
\par 13 Onun ardindan Israil'in basina Piratonlu Hillel oglu Avdon geçti.
\par 14 Avdon'un kirk oglu, otuz torunu ve bunlarin bindigi yetmis esegi vardi. Israil'i sekiz yil yönetti.
\par 15 Piratonlu Hillel oglu Avdon ölünce Amalekliler'e ait daglik bölgenin Efrayim yöresindeki Piraton'da gömüldü.

\chapter{13}

\par 1 Israilliler yine RAB'bin gözünde kötü olani yaptilar. RAB de onlari kirk yil süreyle Filistliler'in boyunduruguna terk etti.
\par 2 Dan oymagindan Sorali bir adam vardi. Adi Manoah'ti. Karisi kisirdi ve hiç çocugu olmamisti.
\par 3 RAB'bin melegi kadina görünerek, "Kisir oldugun, çocuk dogurmadigin halde gebe kalip bir ogul doguracaksin" dedi,
\par 4 "Bundan böyle sarap ya da içki içmemeye dikkat et, murdar* bir sey yeme.
\par 5 Çünkü gebe kalip bir ogul doguracaksin. Onun basina ustura degmeyecek. Çünkü o daha rahmindeyken Tanri'ya adanmis* olacak. Israil'i Filistliler'in elinden kurtarmaya baslayacak olan odur."
\par 6 Kadin kocasina gidip, "Yanima bir Tanri adami geldi" dedi, "Tanri'nin melegine benzer görkemli bir görünüsü vardi. Nereden geldigini sormadim. Bana adini da söylemedi.
\par 7 Ama, 'Gebe kalip bir ogul doguracaksin dedi, 'Bundan böyle sarap ve içki içme, murdar bir sey yeme. Çünkü çocuk ana rahmine düstügü andan ölecegi güne dek Tanri'nin adanmisi olacak."
\par 8 Manoah RAB'be söyle yakardi: "Ya Rab, gönderdigin Tanri adaminin yine gelmesini, dogacak çocuk için ne yapmamiz gerektigini bize ögretmesini dilerim."
\par 9 Tanri Manoah'in yakarisini duydu. Kadin tarladayken Tanri'nin melegi yine ona göründü. Ne var ki, Manoah karisinin yaninda degildi.
\par 10 Kadin haber vermek için kosa kosa kocasina gitti. "Iste geçen gün yanima gelen adam yine bana göründü!" dedi.
\par 11 Manoah kalkip karisinin ardisira gitti. Adamin yanina varinca, "Karimla konusan adam sen misin?" diye sordu. Adam, "Evet, benim" dedi.
\par 12 Manoah, "Söylediklerin yerine geldiginde, çocugun yasami ve göreviyle ilgili yargi ne olacak?" diye sordu.
\par 13 RAB'bin melegi, "Karin kendisine söyledigim her seyden sakinsin" diye karsilik verdi,
\par 14 "Asmanin ürününden üretilen hiçbir sey yemesin, sarap ve içki içmesin. Murdar bir sey yemesin. Buyurduklarimin hepsini yerine getirsin."
\par 15 Manoah, "Seni alikoymak, onuruna bir oglak kesmek istiyoruz" dedi.
\par 16 RAB'bin melegi, "Beni alikoysan da hazirlayacagin yemegi yemem" dedi, "Yakmalik bir sunu* sunacaksan, RAB'be sunmalisin." Manoah onun RAB'bin melegi oldugunu anlamamisti.
\par 17 RAB'bin melegine, "Adin ne?" diye sordu, "Bilelim ki, söylediklerin yerine geldiginde seni onurlandiralim."
\par 18 RAB'bin melegi, "Adimi niçin soruyorsun?" dedi, "Adim tanimlanamaz."
\par 19 Manoah bir oglakla tahil sunusunu* aldi, bir kayanin üzerinde RAB'be sundu. O anda Manoah'la karisinin gözü önünde sasilacak seyler oldu:
\par 20 RAB'bin melegi sunaktan yükselen alevle birlikte göge yükseldi. Bunu gören Manoah'la karisi yüzüstü yere kapandilar.
\par 21 RAB'bin melegi Manoah'la karisina bir daha görünmeyince, Manoah onun RAB'bin melegi oldugunu anladi.
\par 22 Karisina, "Kesinlikle ölecegiz" dedi, "Çünkü Tanri'yi gördük."
\par 23 Karisi, "RAB bizi öldürmek isteseydi, yakmalik sunuyu ve tahil sunusunu kabul etmezdi" diye karsilik verdi, "Bütün bunlari bize göstermezdi. Bugün söylediklerini de isitmezdik."
\par 24 Ve kadin bir erkek çocuk dogurdu. Adini Simson koydu. Çocuk büyüyüp gelisti. RAB de onu kutsadi.
\par 25 RAB'bin Ruhu Sora ile Estaol arasinda, Mahane-Dan'da bulunan Simson'u yönlendirmeye basladi.

\chapter{14}

\par 1 Simson bir gün Timna'ya gitti. Orada Filistli bir kadin gördü.
\par 2 Geri dönünce annesiyle babasina, "Timna'da Filistli bir kadin gördüm" dedi, "Onu hemen bana es olarak alin."
\par 3 Annesiyle babasi, "Akrabalarinin ya da halkimizin kizlari arasinda kimse yok mu ki, sünnetsiz* Filistliler'den kiz almaya kalkiyorsun?" diye karsilik verdiler. Ama Simson babasina, "Bana o kadini al, ondan hoslaniyorum" dedi.
\par 4 Simson'un annesiyle babasi bunu isteyenin RAB oldugunu anlamadilar. Çünkü RAB o sirada Israilliler'e egemen olan Filistliler'e karsi firsat kolluyordu.
\par 5 Böylece Simson annesi ve babasiyla Timna'ya dogru yola koyuldu. Timna baglarina vardiklarinda, genç bir aslan kükreyerek Simson'un karsisina çikti.
\par 6 Simson üzerine inen RAB'bin Ruhu'yla güçlendi ve aslani bir oglak parçalar gibi çiplak elle parçaladi. Ama yaptigini ne annesine ne de babasina bildirdi.
\par 7 Sonra gidip kadinla konustu ve ondan çok hoslandi.
\par 8 Bir süre sonra kadinla evlenmek üzere yine Timna'ya giderken, aslanin lesini görmek için yoldan sapti. Bir ari sürüsünün aslanin lesini kovana çevirdigini gördü.
\par 9 Kovandaki bali avuçlarina doldurdu, yiye yiye oradan uzaklasti. Annesiyle babasinin yanina varinca baldan onlara da verdi, onlar da yedi. Ama bali aslanin lesinden aldigini söylemedi.
\par 10 Babasi kadini görmeye gidince, Simson da damat gelenegine uyarak orada bir sölen düzenledi.
\par 11 Filistliler onu görünce ona eslik etmek üzere otuz genç getirdiler.
\par 12 Simson onlara, "Size bir bilmece sorayim" dedi, "Sölenin yedi günü içinde kesin yaniti bulup bana bildirirseniz, size otuz keten mintan, otuz takim da üst giysi verecegim.
\par 13 Ama bilmeceyi çözemezseniz, o zaman da siz bana otuz keten mintanla otuz takim üst giysi vereceksiniz." Ona, "Seni dinliyoruz" dediler, "Söyle bakalim bilmeceni."
\par 14 Simson, "Yiyenden yiyecek, Güçlüden tatli çikti" dedi. Üç gün geçtiyse de bilmeceyi çözemediler.
\par 15 Dördüncü gün gençler Simson'un karisina, "Kocani kandir da bize bilmecenin yanitini versin" dediler, "Yoksa, seni de babanin evini de yakariz. Bizi soymak için mi buraya çagirdiniz?"
\par 16 Simson'un karisi aglayarak ona, "Benden nefret ediyorsun" dedi, "Beni sevmiyorsun. Soydaslarima bir bilmece sordun, yanitini bana söylemedin." Simson karisina, "Bak" dedi, "Anneme babama bile söylemedim, sana mi söyleyecegim?"
\par 17 Kadin sölen boyunca yedi gün aglayip durdu. Kadinin sürekli sikistirmasi üzerine Simson yedinci gün bilmecenin yanitini ona söyledi. Kadin da yaniti soydaslarina iletti.
\par 18 Yedinci gün, gün batmadan kentli gençler Simson'a geldiler. "Baldan tatli, Aslandan güçlü ne var?" dediler. Simson, "Düvemle çift sürmüs olmasaydiniz, bilmecemi çözemezdiniz" diye karsilik verdi.
\par 19 RAB'bin Ruhu üzerine inince güçlenen Simson Askelon'a gitti; otuz kisi vurup mallarini yagmaladi, giysilerini de bilmeceyi çözenlere verdi. Öfkeden kudurmus bir halde babasinin evine döndü.
\par 20 Simson'un karisi ise Simson'a eslik eden sagdica verildi.

\chapter{15}

\par 1 Bir süre sonra, bugday biçimi sirasinda Simson bir oglak alip karisini ziyarete gitti. "Karimin odasina girmek istiyorum" dedi. Ama kizin babasi Simson'un girmesine izin vermedi.
\par 2 "Ondan gerçekten nefret ettigini saniyordum" dedi, "Bu nedenle onu senin sagdicina verdim. Küçük kizkardesi ondan daha güzel degil mi? Ablasinin yerine onu al."
\par 3 Simson, "Bu kez Filistliler'e kötülük etsem de buna hakkim var" dedi.
\par 4 Kira çikip üç yüz çakal yakaladi. Sonra çakallari çifter çifter kuyruk kuyruga bagladi. Kuyruklarinin arasina da birer çira sikistirdi.
\par 5 Çiralari tutusturup çakallari Filistliler'in ekinlerinin arasina saliverdi. Böylece demetleri, ekinleri, baglari, zeytinlikleri yakti.
\par 6 Filistliler, "Bunu kim yapti?" dediler, "Yapsa yapsa, Timnali'nin damadi Simson yapmistir. Çünkü Timnali karisini elinden alip sagdicina verdi." Sonra gidip kadinla babasini yaktilar.
\par 7 Simson onlara, "Madem böyle yaptiniz, sizden öcümü almadan duramam" dedi.
\par 8 Onlara acimasizca saldirarak çogunu öldürdü, sonra Etam Kayaligi'na çekilip bir magaraya sigindi.
\par 9 Filistliler de gidip Yahuda'da ordugah kurdular, Lehi yöresine yayildilar.
\par 10 Yahudalilar, "Neden bizimle savasmaya geldiniz?" diye sorunca, Filistliler, "Simson'u yakalamaya geldik, bize yaptiginin aynisini ona yapmak için buradayiz" diye karsilik verdiler.
\par 11 Yahudalilar'dan üç bin kisi, Etam Kayaligi'ndaki magaraya giderek Simson'a, "Filistliler'in bize egemen olduklarini bilmiyor musun? Nedir bu bize yaptigin?" dediler. Simson, "Onlar bana ne yaptilarsa ben de onlara öyle yaptim" diye karsilik verdi.
\par 12 "Seni yakalayip Filistliler'e teslim etmek için geldik" dediler. Simson, "Beni öldürmeyeceginize ant için" dedi.
\par 13 Onlar da, "Olur, ama seni sikica baglayip onlara teslim edecegiz" dediler, "Söz veriyoruz, seni öldürmeyecegiz." Sonra onu iki yeni urganla baglayip magaradan çikardilar.
\par 14 Simson Lehi'ye yaklasinca, Filistliler bagirarak ona yöneldiler. RAB'bin Ruhu büyük bir güçle Simson'un üzerine indi. Simson'un kollarini saran urganlar yanan keten gibi dagildi, elindeki baglar çözüldü.
\par 15 Simson yeni ölmüs bir esegin çene kemigini eline alip bununla bin kisiyi öldürdü.
\par 16 Sonra söyle dedi: "Bir esegin çene kemigiyle, Iki esek yigini yaptim, Esegin çene kemigiyle bin kisiyi öldürdüm."
\par 17 Bunlari söyledikten sonra çene kemigini elinden atti. Oraya Ramat-Lehi adi verildi.
\par 18 Simson ölesiye susamisti. RAB'be söyle yakardi: "Kulunun eliyle büyük bir kurtulus sagladin. Ama simdi susuzluktan ölüp sünnetsizlerin* eline mi düsecegim?"
\par 19 Bunun üzerine Tanri Lehi'deki çukuru yardi. Çukurdan su fiskirdi. Simson suyu içince canlanip güçlendi. Suyun çiktigi yere Eyn-Hakkore adini verdi. Pinar bugün de Lehi'de duruyor.
\par 20 Simson Filistliler'in egemenligi sirasinda Israilliler'e yirmi yil önderlik yapti.

\chapter{16}

\par 1 Simson bir gün Gazze'ye gitti. Orada gördügü bir fahisenin evine girdi.
\par 2 Gazzeliler'e, "Simson buraya geldi" diye haber verilince çevreyi kusattilar. Bütün gece kentin kapisinda pusuya yattilar. "Gün agarinca onu öldürürüz" diyerek gece boyunca yerlerinden kimildamadilar.
\par 3 Simson gece yarisina dek yatti. Gece yarisi kalkti, kent kapisinin iki kanadiyla iki diregini tutup sürgüyle birlikte yerlerinden söktü. Hepsini omuzlayip Hevron'un karsisindaki tepeye çikardi.
\par 4 Bir süre sonra Simson Sorek Vadisi'nde yasayan Delila adinda bir kadina asik oldu.
\par 5 Filist beyleri kadina gelip, "Simson'un üstün gücünün kaynagi nedir, onu kandirip ögrenmeye bak" dediler, "Böylece belki onu baglar, etkisiz hale getirip yenebiliriz. Her birimiz sana bin yüzer parça gümüs verecegiz."
\par 6 Bunun üzerine Delila Simson'a, "Lütfen, söyle bana, bu üstün gücü nereden aliyorsun?" diye sordu, "Seni baglayip yenmek olasi mi?"
\par 7 Simson, "Beni kurumamis yedi taze sirimla baglarlarsa siradan bir adam gibi güçsüz olurum" dedi.
\par 8 Bunun üzerine Filist beyleri Delila'ya kurumamis yedi taze sirim getirdiler. Delila bunlarla Simson'u bagladi.
\par 9 Adamlari bitisik odada pusuya yatmisti. Delila, "Simson, Filistliler geldi!" dedi. Simson sirimlari ates degdiginde dagiliveren kendir lifleri gibi koparip atti. Gücünün sirrini vermemisti.
\par 10 Delila, "Beni kandirdin, bana yalan söyledin" dedi, "Lütfen söyle bana, seni neyle baglamali?"
\par 11 Simson, "Beni hiç kullanilmamis yeni urganla simsiki baglarlarsa siradan bir adam gibi güçsüz olurum" dedi.
\par 12 Böylece Delila yeni urgan alip Simson'u bagladi. Sonra, "Simson, Filistliler geldi!" dedi. Adamlar hâlâ bitisik odada pusu kurmus bekliyorlardi. Simson urganlari iplik koparir gibi koparip kollarindan siyirdi.
\par 13 Delila ona, "Simdiye kadar beni hep kandirdin, bana yalan söyledin" dedi, "Söyle bana, seni neyle baglamali?" Simson, "Basimdaki yedi örgüyü dokuma tezgahindaki kumasla birlikte dokuyup kazikla burarsan siradan bir adam gibi güçsüz olurum" dedi.
\par 14 Simson uyurken Delila onun basindaki yedi örgüyü dokuma tezgahindaki kumasla birlikte dokuyup kazikla burdu. Sonra, "Simson, Filistliler geldi!" dedi. Simson uykusundan uyandi, saçini tezgah kazigindan ve kumastan çekip kurtardi.
\par 15 Delila, "Bana güvenmiyorsan nasil olur da, 'Seni seviyorum diyorsun?" dedi, "Üç kezdir beni kandiriyorsun, üstün gücünün nereden geldigini söylemiyorsun."
\par 16 Bu sözlerle Simson'u sikistirip günlerce basini agritti. Sonunda Simson dayanamayip
\par 17 yüregini kadina tümüyle açti. "Basima hiç ustura degmedi" dedi, "Çünkü ben ana rahmindeyken Tanri'ya adanmisim*. Tiras olursam gücümü yitiririm. Siradan bir adam gibi güçsüz olurum."
\par 18 Delila Simson'un gerçegi söyledigini anlayinca haber gönderip Filist beylerini çagirtti. "Bir kez daha gelin" dedi, "Simson bana gerçegi söyledi." Kadinin yanina gelen Filist beyleri gümüsü de birlikte getirdiler.
\par 19 Delila Simson'u dizleri üzerinde uyuttuktan sonra adamlardan birini çagirtip basindaki yedi örgüyü kestirdi. Sonra alay ederek onu dürtüklemeye basladi. Çünkü Simson gücünü yitirmisti.
\par 20 Delila, "Simson, Filistliler geldi!" dedi. Simson uyandi ve, "Her zamanki gibi kalkip silkinirim" diye düsündü. RAB'bin kendisinden ayrildigini bilmiyordu.
\par 21 Filistliler onu yakalayip gözlerini oydular. Gazze'ye götürüp tunç* zincirlerle bagladilar, cezaevinde degirmen tasina kostular.
\par 22 Bu arada Simson'un kesilen saçlari uzamaya basladi.
\par 23 Filist beyleri ilahlari Dagon'un onuruna çok sayida kurban kesip eglenmek için toplandilar. "Ilahimiz, düsmanimiz Simson'u elimize teslim etti" dediler.
\par 24 Halk Simson'u görünce kendi ilahlarini övmeye basladi. "Ilahimiz ülkemizi yakip yikan, Birçogumuzu öldüren Düsmanimizi elimize teslim etti" diyorlardi.
\par 25 Iyice cosunca, "Simson'u getirin, bizi eglendirsin" dediler. Simson'u cezaevinden getirip oynatmaya basladilar, sonra sütunlarin arasinda durdurdular.
\par 26 Simson, elinden tutan gence, "Beni tapinagin damini tasiyan sütunlarin yanina götür de onlara yaslanayim" dedi.
\par 27 Tapinak erkeklerle, kadinlarla doluydu. Bütün Filist beyleri de oradaydi. Üç bin kadar kadin erkek Simson'un oynayisini damdan seyrediyordu.
\par 28 Simson RAB'be yakarmaya basladi: "Ey Egemen RAB, lütfen beni animsa. Ey Tanri, bir kez daha beni güçlendir; Filistliler'den bir vurusta iki gözümün öcünü alayim."
\par 29 Sonra tapinagin damini tasiyan iki ana sütunun ortasinda durup sag eliyle birini, sol eliyle ötekini kavradi.
\par 30 "Filistliler'le birlikte öleyim" diyerek bütün gücüyle sütunlara yüklendi. Tapinak Filist beylerinin ve bütün içindekilerin üzerine çöktü. Böylece Simson ölürken, yasami boyunca öldürdügünden daha çok insan öldürdü.
\par 31 Simson'un kardesleriyle babasi Manoah'in bütün ailesi onun ölüsünü almaya geldiler. Simson'u götürüp babasinin Sora ile Estaol arasindaki mezarina gömdüler. Simson Israil'i yirmi yil süreyle yönetmisti.

\chapter{17}

\par 1 Efrayim'in daglik bölgesinde Mika adinda bir adam vardi.
\par 2 Mika annesine, "Senden çalinan, lanetledigini duydugum bin yüz parça gümüs var ya, iste o gümüsler bende, onlari ben çaldim" dedi. Annesi, "RAB seni kutsasin, oglum!" dedi.
\par 3 Mika bin yüz parça gümüsü annesine geri verdi. Annesi, "Oglumun bir oyma put, bir de dökme put yaptirabilmesi için gümüsün tamamini RAB'be adiyorum" dedi, "Gümüsü sana geri veriyorum."
\par 4 Gümüsü Mika'dan geri alan kadin, iki yüz parçasini ayirip kuyumcuya verdi. Kuyumcu bundan bir oyma, bir de dökme put yapti. Putlar Mika'nin evine götürüldü.
\par 5 Mika'nin bir tapinma yeri vardi. Özel aile putlari ve bir efod* yaptirmis, ogullarindan birini de kâhinlige atamisti.
\par 6 O dönemde Israil'de kral yoktu. Herkes diledigini yapiyordu.
\par 7 Yahuda'nin Beytlehem Kenti'nde, Yahudali bir ailenin yaninda geçici olarak yasayan genç bir Levili vardi.
\par 8 Adam yerlesecek baska bir yer bulmak üzere Yahuda'nin Beytlehem Kenti'nden ayrildi. Efrayim'in daglik bölgesinden geçerken Mika'nin evine geldi.
\par 9 Mika, "Nereden geliyorsun?" diye sorunca adam, "Yahuda'nin Beytlehem Kenti'nden geliyorum, Levili'yim, yerlesecek yer ariyorum" dedi.
\par 10 Mika, "Benimle kal" dedi, "Bana danismanlik ve kâhinlik yap. Seni doyurur, yilda bir takim giysi, on parça da gümüs veririm." Levili kabul etti.
\par 11 Mika ile kalmaya razi oldu. Mika da ona oglu gibi davrandi.
\par 12 Genç Levili'yi kâhinlige atayarak evine aldi.
\par 13 Mika, "Simdi biliyorum ki, RAB bana iyi davranacak" dedi, "Çünkü bir Levili kâhinim* var."

\chapter{18}

\par 1 O dönemde Israil'de kral yoktu ve Dan oymagindan olanlar yerlesecek yer ariyorlardi. Çünkü Israil oymaklari arasinda kendilerine düsen payi henüz almamislardi.
\par 2 Böylece kendi boylarindan, Sora ve Estaol kentlerinden bes cesur savasçiyi topraklari arastirip bilgi toplamak üzere yola çikardilar. Onlara, "Gidin, topraklari arastirin" dediler. Adamlar Efrayim'in daglik bölgesinde bulunan Mika'nin evine gelip geceyi orada geçirdiler.
\par 3 Mika'nin evinin yanindayken genç Levili'nin sesini tanidilar. Eve yaklasarak ona, "Seni buraya kim getirdi? Burada ne yapiyorsun? Burada ne isin var?" diye sordular.
\par 4 Levili Mika'nin kendisi için yaptiklarini anlatti. "Bana verdigi ücrete karsilik ona kâhinlik ediyorum" dedi.
\par 5 Adamlar, "Lütfen Tanri'ya danis, bu yolculugumuz basarili olacak mi, bilelim" dediler.
\par 6 Kâhin, "Esenlikle gidin, Tanri yolculugunuzu onayliyor" diye yanitladi.
\par 7 Böylece bes adam yola çikip Layis'e vardilar. Kent halkinin Saydalilar gibi kaygidan uzak, esenlik ve güvenlik içinde yasadigini gördüler. Yörede onlara egemen olan, baski yapan kimse yoktu. Saydalilar'dan uzaktaydilar, baska kimseyle de iliskileri yoktu.
\par 8 Sonra adamlar Sora ve Estaol'a, soydaslarinin yanina döndüler. Soydaslari, "Ne ögrendiniz?" diye sordular.
\par 9 Adamlar, "Haydi, onlara saldiralim" dediler, "Ülkeyi gördük, topragi çok güzel. Ne duruyorsunuz? Gecikmeden gidip ülkeyi sahiplenin.
\par 10 Oraya vardiginizda halkin her seyden habersiz oldugunu göreceksiniz. Tanri'nin elinize teslim ettigi bu ülke çok genis; öyle bir yer ki, hiçbir eksigi yok."
\par 11 Bunun üzerine Dan oymagindan alti yüz kisi silahlarini kusanip Sora ve Estaol'dan yola çikti.
\par 12 Gidip Yahuda'nin Kiryat-Yearim Kenti yakininda ordugah kurdular. Bu nedenle Kiryat-Yearim'in batisindaki bu yer bugün de Mahane-Dan diye aniliyor.
\par 13 Buradan Efrayim'in daglik bölgesine geçip Mika'nin evine gittiler.
\par 14 Layis yöresini arastirmaya gitmis olan bes adam soydaslarina, "Bu evlerden birinde bir efod*, özel aile putlari, bir oyma, bir de dökme put oldugunu biliyor musunuz?" dediler, "Ne yapacaginiza siz karar verin."
\par 15 Bunun üzerine halk genç Levili'nin kaldigi Mika'nin evine yöneldi. Eve girip Levili'ye hal hatir sordular.
\par 16 Silahlarini kusanmis alti yüz Danli dis kapinin önüne yigilmisti.
\par 17 Yöreyi arastirmis olan bes adam içeri girip efodu, özel putlari, oyma ve dökme putlari aldilar. Kâhinle silah kusanmis alti yüz kisiyse dis kapinin önünde duruyordu.
\par 18 Adamlarin Mika'nin evine girip efodu, özel putlari, oyma ve dökme putlari aldigini gören kâhin, "Ne yapiyorsunuz?" diye sordu.
\par 19 Adamlar, "Sus, sesini çikarma" dediler, "Bizimle gel. Bize danismanlik ve kâhinlik yap. Bir adamin evinde kâhinlik etmek mi iyi, yoksa Israil'in bir boyuna, bir oymagina kâhinlik etmek mi?"
\par 20 Kâhinin yüregi sevinçle doldu. Efodu, özel putlari, oyma putu alip toplulugun ortasinda yürümeye basladi.
\par 21 Topluluk çocuklarini, hayvanlarini, degerli esyalarini alip yola çikti.
\par 22 Danogullari Mika'nin evinden biraz uzaklastiktan sonra, Mika'nin komsulari toplanip onlara yetistiler.
\par 23 Bagirip çagirmaya basladilar. Danogullari dönüp Mika'ya, "Ne oldu, neden adamlarini toplayip geldin?" dediler.
\par 24 Mika, "Kâhinimi, yaptirdigim putlari alip gittiniz" dedi, "Bana ne kaldi ki? Bir de, 'Ne oldu? diye soruyorsunuz."
\par 25 "Kes sesini!" dediler, "Yoksa öfkeli adamlarimiz saldirip seni de, aileni de öldürür."
\par 26 Sonra yollarina devam ettiler. Mika onlarin kendisinden daha güçlü oldugunu görünce dönüp evine gitti.
\par 27 Danogullari Mika'nin yaptirdigi putlari ve kâhini yanlarina alarak Layis üzerine yürüdüler. Barisçil ve her seyden habersiz olan kent halkini kiliçtan geçirip kenti atese verdiler.
\par 28 Beytrehov yakinindaki vadide bulunan Layis Kenti'nin yardimina gelen olmadi. Çünkü kent Sayda'dan uzakti, baska bir kentle de iliskisi yoktu. Danogullari kenti yeniden insa ederek oraya yerlestiler.
\par 29 Yakup'un oglu olan atalari Dan'in anisina kente Dan adini verdiler. Kentin eski adi Layis'ti.
\par 30 Oyma putu oraya diktiler. Musa oglu Gersom oglu Yonatan ile ogullari sürgüne kadar onlara kâhinlik ettiler.
\par 31 Tanri'nin Tapinagi Silo'da oldugu sürece Mika'nin yaptirdigi puta taptilar.

\chapter{19}

\par 1 Israil'in kralsiz oldugu o dönemde Efrayim'in daglik bölgesinin ücra yerinde yasayan bir Levili vardi. Adam Yahuda'nin
\par 2 Ama kadin onu baska erkeklerle aldatti. Sonra adami birakip Yahuda'ya, babasinin Beytlehem'deki evine döndü. Kadin dört ay orada kaldiktan sonra kocasi kalkip onun yanina gitti. Gönlünü hos edip onu geri getirmek istiyordu. Yaninda usagi ve iki de esek vardi. Kadin onu babasinin evine götürdü. Kayinbaba damadini görünce onu sevinçle karsiladi.
\par 4 Yaninda alikoydu. Adam onlarin evinde üç gün kaldi, onlarla birlikte yedi, içti ve orada geceledi.
\par 5 Dördüncü günün sabahi erkenden kalktilar. Kizin babasi gitmeye hazirlanan damadina, "Rahatina bak, bir lokma ekmek ye, sonra gidersiniz" dedi.
\par 6 Ikisi oturup birlikte yiyip içtiler. Kayinbaba, "Lütfen bu gece de kal, keyfine bak" dedi.
\par 7 Damat gitmek üzere ayaga kalkinca kayinbabasi israrla kalmasini istedi; damat da geceyi orada geçirdi.
\par 8 Besinci gün gitmek üzere erkenden kalkti. Kayinbaba, "Rahatina bak, bir seyler ye; ögleden sonra gidersiniz" dedi. Ikisi birlikte yemek yediler.
\par 9 Damat, cariyesi ve usagiyla birlikte gitmek için ayaga kalkinca, kayinbaba, "Bak, aksam oluyor, lütfen geceyi burada geçirin" dedi, "Gün batmak üzere. Geceyi burada geçirin, keyfinize bakin. Yarin erkenden kalkip yola çikar, evine gidersin."
\par 10 Ama adam orada gecelemek istemedi. Cariyesini alip palan vurulmus iki esekle yola çikti. Yevus'un -Yerusalim'in-karsisinda bir yere geldiler.
\par 11 Yevus'a yaklastiklarinda gün batmak üzereydi. Usak efendisine, "Yevuslular'in bu kentine girip geceyi orada geçirelim" dedi.
\par 12 Efendisi, "Israilliler'e ait olmayan yabanci bir kente girmeyecegiz" dedi, "Giva'ya gidecegiz."
\par 13 Sonra ekledi: "Haydi Giva'ya ya da Rama'ya ulasmaya çalisalim. Bunlardan birinde geceleriz."
\par 14 Böylece yollarina devam ettiler. Benyaminliler'in Giva Kenti'ne yaklastiklarinda günes batmisti.
\par 15 Geceyi geçirmek için Giva'ya giden yola saptilar. Varip kentin meydaninda konakladilar. Çünkü hiç kimse onlari evine almadi.
\par 16 Aksam saatlerinde yasli bir adam tarladaki isinden dönüyordu. Efrayim'in daglik bölgesindendi. Giva'da oturuyordu. Kent halki ise Benyaminli'ydi.
\par 17 Yasli adam kent meydanindaki yolculari görünce Levili'ye, "Nereden geliyor, nereye gidiyorsunuz?" diye sordu.
\par 18 Levili, "Yahuda'nin Beytlehem Kenti'nden geliyor, Efrayim'in daglik bölgesinde uzak bir yere gidiyoruz" dedi, "Ben oraliyim. Beytlehem'e gitmistim. Simdi RAB'bin evine dönüyorum. Ama kimse bizi evine almadi.
\par 19 Eseklerimiz için yem ve saman, kendim, cariyem ve usagim için ekmek ve sarap var. Hepimiz sana hizmet etmeye haziriz. Hiçbir eksigimiz yok."
\par 20 Yasli adam, "Gönlün rahat olsun" dedi, "Her ihtiyacini ben karsilayacagim. Geceyi meydanda geçirmeyin."
\par 21 Onlari evine götürdü, eseklerine yem verdi. Konuklar ayaklarini yikadiktan sonra yiyip içtiler.
\par 22 Onlar dinlenirken kentin serserileri evi kusatti. Kapiya var güçleriyle vurarak yasli ev sahibine, "Evine gelen o adami disari çikar, onunla yatalim" diye bagirdilar.
\par 23 Ev sahibi disariya çikip onlarin yanina gitti. "Hayir, kardeslerim, rica ediyorum böyle bir kötülük yapmayin" dedi, "Madem adam evime gelip konugum oldu, böyle bir alçaklik yapmayin.
\par 24 Bakin, daha erkek eli degmemis kizimla adamin cariyesi içerde. Onlari disari çikarayim, onlarla yatin, onlara dilediginizi yapin. Ama adama bu kötülügü yapmayin."
\par 25 Ne var ki, adamlar onu dinlemediler. Bunun üzerine Levili cariyesini zorla disari çikarip onlara teslim etti. Adamlar bütün gece, sabaha dek kadinla yattilar, onun irzina geçtiler. Safak sökerken onu saliverdiler.
\par 26 Kadin gün agarirken efendisinin kaldigi evin kapisina geldi, düsüp yere yigildi. Ortalik aydinlanincaya dek öylece kaldi.
\par 27 Sabahleyin kalkan adam, yoluna devam etmek üzere kapiyi açti. Elleri esigin üzerinde, yerde boylu boyunca yatan cariyesini görünce,
\par 28 kadina, "Kalk, gidelim" dedi. Kadin yanit vermedi. Bunun üzerine adam onu esege bindirip evine dogru yola çikti.
\par 29 Eve varinca eline bir biçak aldi, cariyesinin cesedini on iki parçaya bölüp Israil'in on iki oymagina dagitti.
\par 30 Bunu her gören, "Israilliler Misir'dan çiktigindan beri böyle bir sey olmamis, görülmemistir" dedi, "Düsünün tasinin, ne yapmamiz gerek, söyleyin."

\chapter{20}

\par 1 Gilat basta olmak üzere Dan'dan Beer-Seva'ya kadar, bütün Israil halki yola çikip Mispa'da, RAB'bin önünde tek beden gibi toplandi.
\par 2 Tanri halki Israil'in bütün oymak önderleri bu toplantida hazir bulundular. Eli kiliç tutan dört yüz bin yayaydilar.
\par 3 -Bu arada Benyaminogullari Israilliler'in Mispa'da toplandigini duydular.- Israilliler, "Anlatin bize, bu korkunç olay nasil oldu?" diye sordular.
\par 4 Öldürülen kadinin Levili kocasi söyle yanitladi: "Cariyemle birlikte geceyi geçirmek üzere Benyamin bölgesinin Giva Kenti'ne girdik.
\par 5 Giva'dan bazi adamlar gece beni öldürmeyi tasarlayarak gelip evi kusattilar. Cariyemin irzina geçtiler, ölümüne neden oldular.
\par 6 Onun ölüsünü alip parçaladim, her bir parçasini Israil'in mülk aldigi bir bölgeye gönderdim. Çünkü bu alçakça rezalet Israil'de islendi.
\par 7 Ey Israilliler! Iste hepiniz buradasiniz. Düsünceniz, karariniz nedir, söyleyin."
\par 8 Oradakilerin hepsi agiz birligi etmisçesine, "Bizden hiç kimse çadirina gitmeyecek, evine dönmeyecek" dediler,
\par 9 "Yapacagimiz su: Giva'ya kura ile saldiracagiz.
\par 10 Halka yiyecek saglamak için bütün Israil oymaklarindan nüfuslarina göre, her yüz kisiden on, bin kisiden yüz, on bin kisiden bin kisi seçecegiz. Bunlar Benyamin'in Giva Kenti'ne geldiklerinde kentlilerden Israil'de yaptiklari bu alçakligin öcünü alsinlar."
\par 11 Giva'ya karsi toplanmis olan Israilliler tam bir birlik içindeydi.
\par 12 Israil oymaklari, Benyamin oymagina adamlar göndererek, "Aranizda yapilan bu alçaklik nedir?" diye sordular,
\par 13 "Giva'daki o serserileri bize hemen teslim edin. Onlari öldürüp Israil'deki kötülügün kökünü kaziyalim." Ama Benyaminogullari Israilli kardeslerini dinlemediler.
\par 14 Israilliler'le savasmak üzere öbür kentlerden akin akin Giva'ya geldiler.
\par 15 Giva halkindan olan yedi yüz seçme adam disinda, öbür kentlerden gelen ve eli kiliç tutan Benyaminogullari'nin sayisi o gün yirmi alti bini buldu.
\par 16 Solak olan yedi yüz seçme adam da bunlarin arasindaydi. Hepsi de bir kili sapanla vuracak kadar iyi nisanciydi.
\par 17 Benyaminogullari'nin yanisira Israilliler de sayildi. Eli kiliç tutan dört yüz bin askerleri vardi. Hepsi de yaman savasçilardi.
\par 18 Beytel'e çikan Israilliler Tanri'ya, "Benyaminogullari'na karsi önce hangimiz savasacak?" diye sordular. RAB, "Önce Yahudaogullari savasacak" dedi.
\par 19 Israilliler sabah kalkip Giva'nin karsisinda ordugah kurdular.
\par 20 Benyaminogullari'yla savasmak üzere ilerleyip Giva'da savas düzenine girdiler.
\par 21 Giva'dan çikan Benyaminogullari, o gün Israilliler'den yirmi iki bin kisiyi yere serdiler.
\par 22 Ama Israilliler birbirlerini yüreklendirerek önceki gün savas düzenine girdikleri yerde mevzilendiler.
\par 23 Sonra Beytel'de RAB'bin önünde aksama dek agladilar. RAB'be, "Kardeslerimiz olan Benyaminogullari'yla yine savasmaya çikalim mi?" diye sordular. RAB, "Evet, onlarla savasin" dedi.
\par 24 Bunun üzerine Israilliler ikinci gün yine Benyaminogullari'na yaklastilar.
\par 25 Benyaminogullari da ayni gün Giva'dan onlarin üzerine yürüyerek on sekiz bin kisiyi daha yere serdiler. Ölenlerin hepsi eli kiliç tutan savasçilardi.
\par 26 Bütün Israilliler, bütün halk çekilip Beytel'e döndü. Orada, RAB'bin önünde durup agladilar, o gün aksama dek oruç* tuttular. RAB'be yakmalik sunular* ve esenlik sunulari* sundular.
\par 27 Tanri'nin Antlasma Sandigi* o sirada Beytel'deydi. Harun oglu Elazar oglu Pinehas o sirada sandigin önünde görev yapiyordu. Israilliler RAB'be, "Kardesimiz Benyaminogullari'yla savasmaya devam edelim mi, yoksa vaz mi geçelim?" diye sordular. RAB, "Savasin" dedi, "Çünkü onlari yarin elinize teslim edecegim."
\par 29 Israilliler dört bir yandan Giva'nin çevresinde pusuya yattilar.
\par 30 Üçüncü gün Benyaminogullari'na karsi harekete geçerek önceki gibi kentin karsisinda savas düzenine girdiler.
\par 31 Saldiriya geçen Benyaminogullari kentten epey uzaklastilar. Beytel'e ve Giva'ya giden ana yollarda, kirlarda önceki çarpismalarda oldugu gibi Israilliler'e kayiplar verdirmeye basladilar; otuz kadarini öldürdüler.
\par 32 "Geçen seferki gibi onlari yine bozguna ugratiyoruz" dediler. Israilliler ise birbirlerine, "Kaçalim da onlari kentten uzaga, ana yollara çekelim" diyerek bulunduklari yerden çikip Baal-Tamar'da savas düzenine girdiler. Giva'nin batisinda pusuya yatanlar da birden yerlerinden firladi.
\par 34 Böylece bütün Israil'den seçme on bin kisi Giva'ya cepheden saldirdi. Savas iyice kizismisti. Benyaminogullari baslarina gelecek felaketten habersizdi.
\par 35 RAB onlari Israil'in önünde bozguna ugratti. Israilliler o gün Benyaminogullari'ndan eli kiliç tutan yirmi bes bin yüz kisiyi öldürdüler.
\par 36 Benyaminogullari yenildiklerini anladilar. Israilliler onlarin geçmesine izin verdiler; çünkü Giva çevresinde pusuda yatanlara güveniyorlardi.
\par 37 Pusudakiler ansizin Giva'ya saldirdilar. Bütün kente dagilarak halki kiliçtan geçirdiler.
\par 38 Pusuya yatanlarla öbür Israilliler arasinda bir isaret kararlastirilmisti: Kenti atese verip büyük bir duman bulutu olusturacaklardi.
\par 39 O zaman savas alanindaki Israilliler birden geri dönecekti. Bu arada Benyaminogullari Israilliler'e kayiplar verdirmeye baslamis, otuz kadarini vurmuslardi. Daha önceki savasta oldugu gibi, Israilliler'i kesin bir bozguna ugrattiklarini sandilar.
\par 40 Ama dönüp kente baktiklarinda orada hortum gibi göge yükselen duman bulutunu gördüler. Yanan kentin dumani gögü kaplamisti.
\par 41 Israilliler'in döndügünü gören Benyaminogullari panige kapildi. Çünkü baslarina gelecek felaketi sezmislerdi.
\par 42 Israilliler'in önüsira kirlara dogru yöneldilerse de savastan kaçamadilar. Çesitli kentlerden çikagelen Israilliler onlari kusatip yok etti.
\par 43 Geri kalan Benyaminogullari'ni kovaladilar. Giva'nin dogusunda konakladiklari yere dek onlari yol boyunca vurup yere serdiler.
\par 44 Benyaminogullari'ndan on sekiz bin kisi vuruldu. Hepsi de yigit savasçilardi.
\par 45 Sag kalanlar dönüp kirlara, Rimmon Kayaligi'na dogru kaçmaya basladi. Israilliler yol boyunca bunlardan bes bin kisi daha öldürdü. Gidom'a kadar onlari adim adim izleyerek iki binini daha vurup yere serdiler.
\par 46 O gün Benyaminogullari'ndan öldürülenlerin toplam sayisi yirmi bes bin kisiyi buldu. Hepsi de eli kiliç tutan yigit savasçilardi.
\par 47 Kirlara kaçip Rimmon Kayaligi'na siginanlarin sayisi alti yüzdü. Kayalikta dört ay kaldilar.
\par 48 Israilliler Benyamin kentlerine döndüler; insanlari, hayvanlari ve oradaki bütün canlilari kiliçtan geçirdiler, rastladiklari bütün kentleri atese verdiler.

\chapter{21}

\par 1 Israilliler Mispa'da, "Bizden hiç kimse Benyaminogullari'na kiz vermeyecek" diye ant içmislerdi.
\par 2 Halk Beytel'e geldi. Aksama dek orada, Tanri'nin önünde oturup hiçkira hiçkira agladilar.
\par 3 "Ey Israil'in Tanrisi RAB!" dediler, "Bugün Israil'den bir oymagin eksilmesine yol açan böyle bir sey neden oldu?"
\par 4 Ertesi gün erkenden kalkip bir sunak yaptilar, orada yakmalik sunular* ve esenlik sunulari* sundular.
\par 5 Israilliler, "RAB'bin önüne çikmak üzere toplandigimizda Israil oymaklarindan bize kimler katilmadi?" diye sordular. Çünkü Mispa'da, RAB'bin önünde toplandiklarinda kendilerine katilmayanlarin kesinlikle öldürülecegine dair ant içmislerdi.
\par 6 Israilliler Benyaminli kardesleri için çok üzülüyorlardi. "Israil bugün bir oymagini yitirdi" dediler,
\par 7 "Sag kalanlara es olacak kizlari bulmak için ne yapsak? Çünkü kizlarimizdan hiçbirini onlara es olarak vermeyecegimize RAB'bin adina ant içtik."
\par 8 Sonra, "Mispa'ya, RAB'bin önüne Israil oymaklarindan kim çikmadi?" diye sordular. Böylece Yaves-Gilat'tan toplantiya, ordugaha kimsenin gelmedigi ortaya çikti.
\par 9 Çünkü gelenler sayildiginda Yaves-Gilat'tan kimsenin olmadigi anlasilmisti.
\par 10 Bunun üzerine topluluk Yaves-Gilat halkinin üzerine on iki bin yigit savasçi gönderdi. "Gidin, Yaves-Gilat halkini, kadin, çoluk çocuk demeden kiliçtan geçirin" dediler,
\par 11 "Yapacaginiz su: Her erkegi ve erkek eli degmis her kadini öldüreceksiniz."
\par 12 Yaves-Gilat halki arasinda erkek eli degmemis dört yüz kiz bulup Kenan topraklarinda bulunan Silo'daki ordugaha getirdiler.
\par 13 Ardindan bütün topluluk Rimmon Kayaligi'ndaki Benyaminogullari'na aracilar göndererek baris yapmayi önerdi.
\par 14 Bunun üzerine Benyaminogullari döndü. Topluluk Yaves-Gilat halkindan sag birakilan kizlari onlara es olarak verdi. Ama kizlarin sayisi Benyaminogullari için yine de yeterli degildi.
\par 15 Israil halki Benyaminogullari'nin durumuna çok üzülüyordu. Çünkü RAB Israil oymaklari arasinda birligi bozmustu.
\par 16 Toplulugun ileri gelenleri, "Benyaminogullari'nin kadinlari öldürüldügüne göre, kalan erkeklere es bulmak için ne yapsak?" diyorlardi,
\par 17 "Israil'den bir oymagin yok olup gitmemesi için sag kalan Benyaminogullari'nin mirasçilari olmali.
\par 18 Biz onlara kizlarimizdan es veremeyiz. Çünkü Benyaminogullari'na kiz veren her Israilli lanetlenecek diye ant içtik."
\par 19 Sonra, "Bakin, Silo'da her yil RAB adina bir sölen düzenleniyor" diye eklediler. Silo Beytel'in kuzeyinde, Beytel'den Sekem'e giden yolun dogusunda, Levona'nin güneyindedir.
\par 20 Böylece Benyaminogullari'na, "Gidip baglarda gizlenin" diye ögüt verdiler,
\par 21 "Gözünüzü açik tutun. Silolu kizlar dans etmeye kalkinca baglardan firlayip onlardan kendinize birer es kapin ve Benyamin topraklarina götürün.
\par 22 Kizlarin babalari ya da erkek kardesleri bize yakinmaya gelirse, 'Benyaminogullari'ni hatirimiz için bagislayin diyecegiz, 'Savasarak aldigimiz kizlar hepsine yetmedi. Siz de kendi kizlarinizi isteyerek vermediginize göre suçlu sayilmazsiniz."
\par 23 Benyaminogullari da böyle yaptilar. Kizlar dans ederken her erkek bir kiz kapip götürdü. Kendi topraklarina gittiler, kentlerini onarip yerlestiler.
\par 24 Ardindan Israilliler de oradan ayrilip kendi topraklarina, oymaklarina, ailelerine döndüler.
\par 25 O dönemde Israil'de kral yoktu. Herkes diledigini yapiyordu.


\end{document}