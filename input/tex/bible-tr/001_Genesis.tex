\begin{document}

\title{Yaratılış}


\chapter{1}

\par 1 Baslangiçta Tanri gögü ve yeri yaratti.
\par 2 Yer bostu, yeryüzü sekilleri yoktu; engin karanliklarla kapliydi. Tanri'nin Ruhu sularin üzerinde dalgalaniyordu.
\par 3 Tanri, "Isik olsun" diye buyurdu ve isik oldu.
\par 4 Tanri isigin iyi oldugunu gördü ve onu karanliktan ayirdi.
\par 5 Isiga "Gündüz", karanliga "Gece" adini verdi. Aksam oldu, sabah oldu ve ilk gün olustu.
\par 6 Tanri, "Sularin ortasinda bir kubbe olsun, sulari birbirinden ayirsin" diye buyurdu.
\par 7 Ve öyle oldu. Tanri gökkubbeyi yaratti. Kubbenin altindaki sulari üstündeki sulardan ayirdi.
\par 8 Kubbeye "Gök" adini verdi. Aksam oldu, sabah oldu ve ikinci gün olustu.
\par 9 Tanri, "Gögün altindaki sular bir yere toplansin, kuru toprak görünsün" diye buyurdu ve öyle oldu.
\par 10 Kuru alana "Kara", toplanan sulara "Deniz" adini verdi. Tanri bunun iyi oldugunu gördü.
\par 11 Tanri, "Yeryüzü bitkiler, tohum veren otlar, türüne göre tohumu meyvesinde bulunan meyve agaçlari üretsin" diye buyurdu ve öyle oldu.
\par 12 Yeryüzü bitkiler, türüne göre tohum veren otlar, tohumu meyvesinde bulunan meyve agaçlari yetistirdi. Tanri bunun iyi oldugunu gördü.
\par 13 Aksam oldu, sabah oldu ve üçüncü gün olustu.
\par 14 Tanri söyle buyurdu: "Gökkubbede gündüzü geceden ayiracak, yeryüzünü aydinlatacak isiklar olsun. Belirtileri, mevsimleri, günleri, yillari göstersin." Ve öyle oldu.
\par 15 (#1:14)
\par 16 Tanri büyügü gündüze, küçügü geceye egemen olacak iki büyük isigi ve yildizlari yaratti.
\par 17 Yeryüzünü aydinlatmak, gündüze ve geceye egemen olmak, isigi karanliktan ayirmak için onlari gökkubbeye yerlestirdi. Tanri bunun iyi oldugunu gördü.
\par 18 (#1:17)
\par 19 Aksam oldu, sabah oldu ve dördüncü gün olustu.
\par 20 Tanri, "Sular canli yaratiklarla dolup tassin, yeryüzünün üzerinde, gökte kuslar uçussun" diye buyurdu.
\par 21 Tanri büyük deniz canavarlarini, sularda kaynasan canlilari ve uçan çesitli varliklari yaratti. Bunun iyi oldugunu gördü.
\par 22 Tanri, "Verimli olun, çogalin, denizleri doldurun, yeryüzünde kuslar çogalsin" diyerek onlari kutsadi.
\par 23 Aksam oldu, sabah oldu ve besinci gün olustu.
\par 24 Tanri, "Yeryüzü çesit çesit canli yaratik, evcil ve yabanil hayvan, sürüngen türetsin" diye buyurdu. Ve öyle oldu.
\par 25 Tanri çesit çesit yabanil hayvan, evcil hayvan, sürüngen yaratti. Bunun iyi oldugunu gördü.
\par 26 Tanri, "Insani kendi suretimizde, kendimize benzer yaratalim" dedi, "Denizdeki baliklara, gökteki kuslara, evcil hayvanlara, sürüngenlere, yeryüzünün tümüne egemen olsun."
\par 27 Tanri insani kendi suretinde yaratti. Böylece insan Tanri suretinde yaratilmis oldu. Insanlari erkek ve disi olarak yaratti.
\par 28 Onlari kutsayarak, "Verimli olun, çogalin" dedi, "Yeryüzünü doldurun ve denetiminize alin; denizdeki baliklara, gökteki kuslara, yeryüzünde yasayan bütün canlilara egemen olun.
\par 29 Iste yeryüzünde tohum veren her otu, tohumu meyvesinde bulunan her meyve agacini size veriyorum. Bunlar size yiyecek olacak.
\par 30 Yabanil hayvanlara, gökteki kuslara, sürüngenlere -soluk alip veren bütün hayvanlara- yiyecek olarak yesil otlari veriyorum." Ve öyle oldu.
\par 31 Tanri yarattiklarina bakti ve her seyin çok iyi oldugunu gördü. Aksam oldu, sabah oldu ve altinci gün olustu.

\chapter{2}

\par 1 Gök ve yer bütün ögeleriyle tamamlandi.
\par 2 Yedinci güne gelindiginde Tanri yapmakta oldugu isi bitirdi. Yaptigi isten o gün dinlendi.
\par 3 Yedinci günü kutsadi. Onu kutsal bir gün olarak belirledi. Çünkü Tanri o gün yaptigi, Yarattigi bütün isi bitirip dinlendi.
\par 4 Gögün ve yerin Yar.atilis öyküsü: RAB Tanri gögü ve yeri Yarattiginda,
\par 5 yeryüzünde yabanil bir fidan, bir ot bile bitmemisti. Çünkü RAB Tanri henüz yeryüzüne yagmur göndermemisti. Topragi isleyecek insan da yoktu.
\par 6 Yerden yükselen buhar bütün topraklari suluyordu.
\par 7 RAB Tanri Adem'i topraktan Yaratti ve burnuna yasam solugunu üfledi. Böylece Adem yasayan varlik oldu.
\par 8 RAB Tanri doguda, Aden'de bir bahçe dikti. Yarattigi Adem'i oraya koydu.
\par 9 Bahçede iyi meyve veren türlü türlü güzel agaç yetistirdi. Bahçenin ortasinda yasam agaciyla iyiyle kötüyü bilme agaci vardi.
\par 10 Aden'den bir irmak doguyor, bahçeyi sulayip orada dört kola ayriliyordu.
\par 11 Ilk irmagin adi Pison'dur. Altin kaynaklari olan Havila sinirlari boyunca akar.
\par 12 Orada iyi altin, reçine ve oniks bulunur.
\par 13 Ikinci irmagin adi Gihon'dur, Kûs* sinirlari boyunca akar.
\par 14 Üçüncü irmagin adi Dicle'dir, Asur'un dogusundan akar. Dördüncü irmak ise Firat'tir.
\par 15 RAB Tanri Aden bahçesine bakmasi, onu islemesi için Adem'i oraya koydu.
\par 16 Ona, "Bahçede istedigin agacin meyvesini yiyebilirsin" diye buyurdu,
\par 17 "Ama iyiyle kötüyü bilme agacindan yeme. Çünkü ondan yedigin gün kesinlikle ölürsün."
\par 18 Sonra, "Adem'in yalniz kalmasi iyi degil" dedi, "Ona uygun bir Yardimci Yaratacagim."
\par 19 RAB Tanri yerdeki hayvanlarin, gökteki kuslarin tümünü topraktan Yar.atmisti. Onlara ne ad verecegini görmek için hepsini Adem'e getirdi. Adem her birine ne ad verdiyse, o canli o adla anildi.
\par 20 Adem bütün evcil ve yabanil hayvanlara, gökte uçan kuslara ad koydu. Ama kendisi için uygun bir Yardimci bulunmadi.
\par 21 RAB Tanri Adem'e derin bir uyku verdi. Adem uyurken, RAB Tanri onun kaburga kemiklerinden birini alip yerini etle kapadi.
\par 22 Adem'den aldigi kaburga kemiginden bir kadin Yar.atarak onu Adem'e getirdi.
\par 23 Adem, "Iste, bu benim kemiklerimden alinmis kemik, Etimden alinmis ettir" dedi, "Ona 'Kadin denilecek, Çünkü o adamdan alindi."
\par 24 Bu nedenle adam annesini babasini birakip karisina baglanacak, ikisi tek beden olacak.
\par 25 Adem de karisi da çiplaktilar, henüz utanç nedir bilmiyorlardi.

\chapter{3}

\par 1 RAB Tanri'nin yarattigi yabanil hayvanlarin en kurnazi yilandi. Yilan kadina, "Tanri gerçekten, 'Bahçedeki agaçlarin hiçbirinin meyvesini yemeyin' dedi mi?" diye sordu.
\par 2 Kadin, "Bahçedeki agaçlarin meyvelerinden yiyebiliriz" diye yanitladi,
\par 3 "Ama Tanri, 'Bahçenin ortasindaki agacin meyvesini yemeyin, ona dokunmayin; yoksa ölürsünüz' dedi."
\par 4 Yilan, "Kesinlikle ölmezsiniz" dedi,
\par 5 "Çünkü Tanri biliyor ki, o agacin meyvesini yediginizde gözleriniz açilacak, iyiyle kötüyü bilerek Tanri gibi olacaksiniz."
\par 6 Kadin agacin güzel, meyvesinin yemek için uygun ve bilgelik kazanmak için çekici oldugunu gördü. Meyveyi koparip yedi. Yanindaki kocasina verdi, o da yedi.
\par 7 Ikisinin de gözleri açildi. Çiplak olduklarini anladilar. Bu yüzden incir yapraklari dikip kendilerine önlük yaptilar.
\par 8 Derken, günün serinliginde bahçede yürüyen RAB Tanri'nin sesini duydular. O'ndan kaçip agaçlarin arasina gizlendiler.
\par 9 RAB Tanri Adem'e, "Neredesin?" diye seslendi.
\par 10 Adem, "Bahçede sesini duyunca korktum. Çünkü çiplaktim, bu yüzden gizlendim" dedi.
\par 11 RAB Tanri, "Çiplak oldugunu sana kim söyledi?" diye sordu, "Sana meyvesini yeme dedigim agaçtan mi yedin?"
\par 12 Adem, "Yanima koydugun kadin agacin meyvesini bana verdi, ben de yedim" diye yanitladi.
\par 13 RAB Tanri kadina, "Nedir bu yaptigin?" diye sordu. Kadin, "Yilan beni aldatti, o yüzden yedim" diye karsilik verdi.
\par 14 Bunun üzerine RAB Tanri yilana, "Bu yaptigindan ötürü Bütün evcil ve yabanil hayvanlarin En lanetlisi sen olacaksin" dedi, "Karninin üzerinde sürünecek, Yasamin boyunca toprak yiyeceksin.
\par 15 Seninle kadini, onun soyuyla senin soyunu Birbirinize düsman edecegim. Onun soyu senin basini ezecek, Sen onun topuguna saldiracaksin."
\par 16 RAB Tanri kadina, "Çocuk dogururken sana Çok aci çektirecegim" dedi, "Agri çekerek dogum yapacaksin. Kocana istek duyacaksin, Seni o yönetecek."
\par 17 RAB Tanri Adem'e, "Karinin sözünü dinledigin ve sana, Meyvesini yeme dedigim agaçtan yedigin için Toprak senin yüzünden lanetlendi" dedi, "Yasam boyu emek vermeden yiyecekbulamayacaksin.
\par 18 Toprak sana diken ve çali verecek, Yaban otu yiyeceksin.
\par 19 Topraga dönünceye dek Ekmegini alin teri dökerek kazanacaksin. Çünkü topraksin, topraktan yaratildin Ve yine topraga döneceksin."
\par 20 Adem karisina Havva adini verdi. Çünkü o bütün insanlarin annesiydi.
\par 21 RAB Tanri Adem'le karisi için deriden giysiler yapti, onlari giydirdi.
\par 22 Sonra, "Adem iyiyle kötüyü bilmekle bizlerden biri gibi oldu" dedi, "Artik yasam agacina uzanip meyve almasina, yiyip ölümsüz olmasina izin verilmemeli."
\par 23 Böylece RAB Tanri, yaratilmis oldugu topragi islemek üzere Adem'i Aden bahçesinden çikardi.
\par 24 Onu kovdu. Yasam agacinin yolunu denetlemek için de Aden bahçesinin dogusuna Keruvlar ve her yana dönen alevli bir kiliç yerlestirdi.

\chapter{4}

\par 1 Adem karisi Havva ile yatti. Havva hamile kaldi ve Kayin'i dogurdu. "RAB'bin yardimiyla bir ogul dünyaya getirdim" dedi.
\par 2 Daha sonra Kayin'in kardesi Habil'i dogurdu. Habil çoban oldu, Kayin ise çiftçi.
\par 3 Günler geçti. Bir gün Kayin topragin ürünlerinden RAB'be sunu getirdi.
\par 4 Habil de sürüsünde ilk dogan hayvanlardan bazilarini, özellikle de yaglarini getirdi. RAB Habil'i ve sunusunu kabul etti.
\par 5 Kayin'le sunusunu ise reddetti. Kayin çok öfkelendi, suratini asti.
\par 6 RAB Kayin'e, "Niçin öfkelendin?" diye sordu, "Niçin surat astin?
\par 7 Dogru olani yapsan, seni kabul etmez miyim? Ancak dogru olani yapmazsan, günah kapida pusuya yatmis, seni bekliyor. Ona egemen olmalisin."
\par 8 Kayin kardesi Habil'e, "Haydi, tarlaya gidelim dedi. Tarlada birlikteyken kardesine saldirip onu öldürdü.
\par 9 RAB Kayin'e, "Kardesin Habil nerede?" diye sordu. Kayin, "Bilmiyorum, kardesimin bekçisi miyim ben?" diye karsilik verdi.
\par 10 RAB, "Ne yaptin?" dedi, "Kardesinin kani topraktan bana sesleniyor.
\par 11 Artik döktügün kardes kanini içmek için agzini açan topragin laneti altindasin.
\par 12 Isledigin toprak bundan böyle sana ürün vermeyecek. Yeryüzünde aylak aylak dolasacaksin."
\par 13 Kayin, "Cezam kaldiramayacagim kadar agir" diye karsilik verdi,
\par 14 "Bugün beni bu topraklardan kovdun. Artik huzurundan uzak kalacak, yeryüzünde aylak aylak dolasacagim. Kim bulsa öldürecek beni."
\par 15 Bunun üzerine RAB, "Seni kim öldürürse, ondan yedi kez öç alinacak" dedi. Kimse bulup öldürmesin diye Kayin'in üzerine bir nisan koydu.
\par 16 Kayin RAB'bin huzurundan ayrildi. Aden bahçesinin dogusunda, Nod topraklarina yerlesti.
\par 17 Kayin karisiyla yatti. Karisi hamile kaldi ve Hanok'u dogurdu. Kayin o sirada bir kent kurmaktaydi. Kente oglu Hanok'un adini verdi.
\par 18 Hanok'tan Irat oldu. Irat'tan Mehuyael, Mehuyael'den Metusael, Metusael'den Lemek oldu.
\par 19 Lemek iki kadinla evlendi. Birinin adi Âda, öbürünün ise Silla'ydi.
\par 20 Âda Yaval'i dogurdu. Yaval sürü sahibi göçebelerin atasiydi.
\par 21 Kardesinin adi Yuval'di. Yuval lir ve ney çalanlarin atasiydi.
\par 22 Silla Tuval-Kayin'i dogurdu. Tuval-Kayin tunç* ve demirden çesitli kesici aletler yapardi. Tuval-Kayin'in kizkardesi Naama'ydi.
\par 23 Lemek karilarina söyle dedi: "Ey Âda ve Silla, beni dinleyin, Ey Lemek'in karilari, sözlerime kulak verin. Beni yaraladigi için Bir adam öldürdüm, Beni hirpaladigi için Bir genci öldürdüm.
\par 24 Kayin'in yedi kez öcü alinacaksa, Lemek'in yetmis yedi kez öcü alinmali."
\par 25 Adem karisiyla yine yatti. Havva bir erkek çocuk dogurdu. "Tanri Kayin'in öldürdügü Habil'in yerine bana baska bir ogul bagisladi" diyerek çocuga Sit adini verdi.
\par 26 Sit'in de bir oglu oldu, adini Enos koydu.

\chapter{5}

\par 1 Adem soyunun öyküsü: Tanri insani yarattiginda onu kendine benzer kildi.
\par 2 Onlari erkek ve disi olarak yaratti ve kutsadi. Yaratildiklari gün onlara "Insan" adini verdi.
\par 3 Adem 130 yasindayken kendi suretinde, kendisine benzer bir oglu oldu. Ona Sit adini verdi.
\par 4 Sit'in dogumundan sonra Adem 800 yil daha yasadi. Baska ogullari, kizlari oldu.
\par 5 Adem toplam 930 yil yasadiktan sonra öldü.
\par 6 Sit 105 yasindayken oglu Enos dogdu.
\par 7 Enos'un dogumundan sonra Sit 807 yil daha yasadi. Baska ogullari, kizlari oldu.
\par 8 Sit toplam 912 yil yasadiktan sonra öldü.
\par 9 Enos 90 yasindayken oglu Kenan dogdu.
\par 10 Kenan'in dogumundan sonra Enos 815 yil daha yasadi. Baska ogullari, kizlari oldu.
\par 11 Enos toplam 905 yil yasadiktan sonra öldü.
\par 12 Kenan 70 yasindayken oglu Mahalalel dogdu.
\par 13 Mahalalel'in dogumundan sonra Kenan 840 yil daha yasadi. Baska ogullari, kizlari oldu.
\par 14 Kenan toplam 910 yil yasadiktan sonra öldü.
\par 15 Mahalalel 65 yasindayken oglu Yeret dogdu.
\par 16 Yeret'in dogumundan sonra Mahalalel 830 yil daha yasadi. Baska ogullari, kizlari oldu.
\par 17 Mahalalel toplam 895 yil yasadiktan sonra öldü.
\par 18 Yeret 162 yasindayken oglu Hanok dogdu.
\par 19 Hanok'un dogumundan sonra Yeret 800 yil daha yasadi. Baska ogullari, kizlari oldu.
\par 20 Yeret toplam 962 yil yasadiktan sonra öldü.
\par 21 Hanok 65 yasindayken oglu Metuselah dogdu.
\par 22 Metuselah'in dogumundan sonra Hanok 300 yil Tanri yolunda yürüdü. Baska ogullari, kizlari oldu.
\par 23 Hanok toplam 365 yil yasadi.
\par 24 Tanri yolunda yürüdü, sonra ortadan kayboldu; çünkü Tanri onu yanina almisti.
\par 25 Metuselah 187 yasindayken oglu Lemek dogdu.
\par 26 Lemek'in dogumundan sonra Metuselah 782 yil daha yasadi. Baska ogullari, kizlari oldu.
\par 27 Metuselah toplam 969 yil yasadiktan sonra öldü.
\par 28 Lemek 182 yasindayken bir oglu oldu.
\par 29 "RAB'bin lanetledigi bu toprak yüzünden çektigimiz eziyeti, harcadigimiz emegi bu çocuk hafifletip bizi rahatlatacak" diyerek çocuga Nuh adini verdi.
\par 30 Nuh'un dogumundan sonra Lemek 595 yil daha yasadi. Baska ogullari, kizlari oldu.
\par 31 Lemek toplam 777 yil yasadiktan sonra öldü.
\par 32 Nuh 500 yil yasadiktan sonra Sam, Ham, Yafet adli ogullari dogdu.

\chapter{6}

\par 1 Yeryüzünde insanlar çogalmaya basladi, kizlar dogdu.
\par 2 Ilahi varliklar insan kizlarinin güzelligini görünce begendikleriyle evlendiler.
\par 3 RAB, "Ruhum insanda sonsuza dek kalmayacak, çünkü o ölümlüdür" dedi, "Insanin ömrü yüz yirmi yil olacak."
\par 4 Ilahi varliklarin insan kizlariyla evlenip çocuk sahibi olduklari günlerde ve daha sonra yeryüzünde Nefiller vardi. Bunlar eski çag kahramanlari, ünlü kisilerdi.
\par 5 RAB bakti, yeryüzünde insanin yaptigi kötülük çok, akli fikri hep kötülükte.
\par 6 Insani yarattigina pisman oldu. Yüregi sizladi.
\par 7 "Yarattigim insanlari, hayvanlari, sürüngenleri*, kuslari yeryüzünden silip atacagim" dedi, "Çünkü onlari yarattigima pisman oldum."
\par 8 Ama Nuh RAB'bin gözünde lütuf buldu.
\par 9 Nuh'un öyküsü söyledir: Nuh dogru bir insandi. Çagdaslari arasinda kusursuz biriydi. Tanri yolunda yürüdü.
\par 10 Üç oglu vardi: Sam, Ham, Yafet.
\par 11 Tanri'nin gözünde yeryüzü bozulmus, zorbalikla dolmustu.
\par 12 Tanri yeryüzüne bakti ve her seyin ne denli bozuldugunu gördü. Çünkü insanlar yoldan çikmisti.
\par 13 Tanri Nuh'a, "Insanliga son verecegim" dedi, "Çünkü onlar yüzünden yeryüzü zorbalikla doldu. Onlarla birlikte yeryüzünü de yok edecegim.
\par 14 Kendine gofer agacindan bir gemi yap. Içini disini ziftle, içeriye kamaralar yap.
\par 15 Gemiyi söyle yapacaksin: Uzunlugu üç yüz, genisligi elli, yüksekligi otuz arsin olacak.
\par 16 Pencere de yap, boyu yukariya dogru bir arsini bulsun. Kapiyi geminin yan tarafina koy. Alt, orta ve üst güverteler yap.
\par 17 Yeryüzüne tufan gönderecegim. Göklerin altinda soluk alan bütün canlilari yok edecegim. Yeryüzündeki her canli ölecek.
\par 18 Ama seninle bir antlasma yapacagim. Ogullarin, karin, gelinlerinle birlikte gemiye bin.
\par 19 Sag kalabilmeleri için her canli türünden bir erkek, bir disi olmak üzere birer çifti gemiye al.
\par 20 Çesit çesit kuslar, hayvanlar, sürüngenler sag kalmak için çifter çifter sana gelecekler.
\par 21 Yanina hem kendin, hem onlar için yenebilecek ne varsa al, ilerde yemek üzere depola."
\par 22 Nuh Tanri'nin bütün buyruklarini yerine getirdi.

\chapter{7}

\par 1 RAB Nuh'a, "Bütün ailenle birlikte gemiye bin" dedi, "Çünkü bu kusak içinde yalniz seni dogru buldum.
\par 2 Yeryüzünde soylari tükenmesin diye, yanina temiz sayilan hayvanlardan erkek ve disi olmak üzere yediser çift, kirli sayilan hayvanlardan birer çift, kuslardan yediser çift al.
\par 3 (#7:2)
\par 4 Çünkü yedi gün sonra yeryüzüne kirk gün kirk gece yagmur yagdiracagim. Yarattigim her canliyi yeryüzünden silip atacagim."
\par 5 Nuh RAB'bin bütün buyruklarini yerine getirdi.
\par 6 Yeryüzünde tufan koptugunda Nuh alti yüz yasindaydi.
\par 7 Nuh, ogullari, karisi, gelinleri tufandan kurtulmak için hep birlikte gemiye bindiler.
\par 8 Tanri'nin Nuh'a buyurdugu gibi temiz ve kirli sayilan her tür hayvan, kus ve sürüngenden* erkek ve disi olmak üzere birer çift Nuh'a gelip gemiye bindiler.
\par 9 (#7:8)
\par 10 Yedi gün sonra tufan koptu.
\par 11 Nuh alti yüz yasindayken, o yilin ikinci ayinin* on yedinci günü enginlerin bütün kaynaklari fiskirdi, göklerin kapaklari açildi.
\par 12 Yeryüzüne kirk gün kirk gece yagmur yagdi.
\par 13 Nuh, ogullari Sam, Ham, Yafet, Nuh'un karisiyla üç gelini tam o gün gemiye bindiler.
\par 14 Onlarla birlikte her tür hayvan -evcil hayvanlarin, sürüngenlerin, kuslarin, uçan yaratiklarin her türü- gemiye bindi.
\par 15 Soluk alan her tür canli çifter çifter Nuh'un yanina gelip gemiye bindi.
\par 16 Gemiye giren hayvanlar Tanri'nin Nuh'a buyurdugu gibi erkek ve disiydi. RAB Nuh'un ardindan kapiyi kapadi.
\par 17 Tufan kirk gün sürdü. Çogalan sular gemiyi yerden yukari kaldirdi.
\par 18 Sular yükseldi, çogaldikça çogaldi; gemi suyun üzerinde yüzmeye basladi.
\par 19 Sular öyle yükseldi ki, yeryüzündeki bütün yüksek daglar su altinda kaldi.
\par 20 Yükselen sular daglari on bes arsin asti.
\par 21 Yeryüzünde yasayan bütün canlilar yok oldu; kuslar, evcil ve yabanil hayvanlar, sürüngenler, insanlar, soluk alan bütün canlilar öldü.
\par 22 (#7:21)
\par 23 RAB insanlardan evcil hayvanlara, sürüngenlerden kuslara dek bütün canlilari yok etti, yeryüzündeki her sey silinip gitti. Yalniz Nuh'la gemidekiler kaldi.
\par 24 Sular yüz elli gün boyunca yeryüzünü kapladi.

\chapter{8}

\par 1 Sonra Tanri Nuh'u ve gemideki evcil ve yabanil hayvanlari animsadi. Yeryüzünde bir rüzgar estirdi, sular alçalmaya basladi.
\par 2 Enginlerin kaynaklari, göklerin kapaklari kapandi. Yagmur dindi.
\par 3 Sular yeryüzünden çekilmeye basladi. Yüz elli gün geçtikten sonra sular azaldi.
\par 4 Gemi yedinci ayin* on yedinci günü Ararat daglarina oturdu.
\par 5 Sular onuncu aya kadar sürekli azaldi. Onuncu ayin birinde daglarin dorugu göründü.
\par 6 Kirk gün sonra Nuh yapmis oldugu geminin penceresini açti.
\par 7 Kuzgunu disari gönderdi. Kuzgun sular kuruyuncaya kadar dönmedi, uçup durdu.
\par 8 Bunun üzerine Nuh sularin yeryüzünden çekilip çekilmedigini anlamak için güvercini gönderdi.
\par 9 Güvercin konacak bir yer bulamadi, çünkü her yer suyla kapliydi. Gemiye, Nuh'un yanina döndü. Nuh uzanip güvercini tuttu ve gemiye, yanina aldi.
\par 10 Yedi gün daha bekledi, sonra güvercini yine disari saldi.
\par 11 Güvercin gagasinda yeni kopmus bir zeytin yapragiyla aksamleyin geri döndü. O zaman Nuh sularin yeryüzünden çekilmis oldugunu anladi.
\par 12 Yedi gün daha bekledikten sonra güvercini yine gönderdi. Bu kez güvercin geri dönmedi.
\par 13 Nuh alti yüz bir yasindayken, birinci ayin birinde yeryüzündeki sular kurudu. Nuh geminin üstündeki kapagi kaldirinca topragin kurumus oldugunu gördü.
\par 14 Ikinci ayin yirmi yedinci günü toprak tümüyle kurumustu.
\par 15 Tanri Nuh'a, "Karin, ogullarin ve gelinlerinle birlikte gemiden çik" dedi,
\par 16 (#8:15)
\par 17 "Kendinle birlikte bütün canlilari, kuslari, hayvanlari, sürüngenleri* de çikar. Üresinler, verimli olsunlar, yeryüzünde çogalsinlar."
\par 18 Nuh karisi, ogullari ve gelinleriyle birlikte gemiden çikti.
\par 19 Bütün hayvanlar, sürüngenler, kuslar, yeryüzünde yasayan her tür canli da gemiyi terk etti.
\par 20 Nuh RAB'be bir sunak yapti. Orada bütün temiz sayilan hayvanlarla kuslardan yakmalik sunular* sundu.
\par 21 Güzel kokudan hosnut olan RAB içinden söyle dedi: "Insanlar yüzünden yeryüzünü bir daha lanetlemeyecegim. Çünkü insan yüregindeki egilimler çocuklugundan beri kötüdür. Simdi yaptigim gibibütün canlilari bir daha yok etmeyecegim.
\par 22 "Dünya durdukça Ekin ekmek, biçmek, Sicak, soguk, Yaz, kis, Gece, gündüz hep var olacaktir."

\chapter{9}

\par 1 Tanri, Nuh'u ve ogullarini kutsayarak, "Verimli olun, çogalip yeryüzünü doldurun" dedi,
\par 2 "Yerdeki hayvanlarin, gökteki kuslarin tümü sizden korkup ürkecek. Yeryüzündeki bütün canlilar, denizdeki bütün baliklar sizin yönetiminize verilmistir.
\par 3 Bütün canlilar size yiyecek olacak. Yesil bitkiler gibi, hepsini size veriyorum.
\par 4 "Yalniz kanli et yemeyeceksiniz, çünkü kan cani içerir.
\par 5 Sizin de kaniniz dökülürse, hakkinizi kesinlikle arayacagim. Her hayvandan hesabini soracagim. Her insandan, kardesinin canina kiyan herkesten hakkinizi arayacagim.
\par 6 "Kim insan kani dökerse, Kendi kani da insan tarafindan dökülecektir. Çünkü Tanri insani kendi suretinde yaratti.
\par 7 Verimli olun, çogalin. Yeryüzünde üreyin, artin."
\par 8 Tanri Nuh'a ve ogullarina söyle dedi:
\par 9 "Sizinle ve gelecek kusaklarinizla, sizinle birlikteki bütün canlilarla -kuslar, evcil ve yabanil hayvanlar, gemiden çikan bütün hayvanlarla- antlasmami sürdürmek istiyorum.
\par 10 (#9:9)
\par 11 Sizinle antlasmami sürdürecegim: Bir daha tufanla bütün canlilar yok olmayacak. Yeryüzünü yok eden tufan bir daha olmayacak."
\par 12 Tanri söyle sürdürdü konusmasini: "Sizinle ve bütün canlilarla kusaklar boyu sonsuza dek sürecek antlasmamin belirtisi su olacak:
\par 13 Yayimi bulutlara yerlestirecegim ve bu, yeryüzüyle aramdaki antlasmanin belirtisi olacak.
\par 14 Yeryüzüne ne zaman bulut göndersem, yayim bulutlarin arasinda ne zaman görünse,
\par 15 sizinle ve bütün canli varliklarla yaptigim antlasmayi animsayacagim: Canlilari yok edecek bir tufan bir daha olmayacak.
\par 16 Ne zaman bulutlarda yay görünse, ona bakip yeryüzünde yasayan bütün canlilarla yaptigim sonsuza dek geçerli antlasmayi animsayacagim."
\par 17 Tanri Nuh'a, "Kendimle yeryüzündeki bütün canlilar arasinda sürdürecegim antlasmanin belirtisi budur" dedi.
\par 18 Gemiden çikan Nuh'un ogullari Sam, Ham ve Yafet idi. Ham Kenan'in babasiydi.
\par 19 Nuh'un üç oglu bunlardi. Yeryüzüne yayilan bütün insanlar onlardan üredi.
\par 20 Nuh çiftçiydi, ilk bagi o dikti.
\par 21 Sarap içip sarhos oldu, çadirinin içinde çirilçiplak uzandi.
\par 22 Kenan'in babasi olan Ham babasinin çiplak oldugunu görünce disari çikip iki kardesine anlatti.
\par 23 Sam'la Yafet bir giysi alip omuzlarina attilar, geri geri yürüyerek çiplak babalarini örttüler. Babalarini çiplak görmemek için yüzlerini öbür yana çevirdiler.
\par 24 Nuh ayilinca küçük oglunun ne yaptigini anlayarak,
\par 25 söyle dedi: "Kenan'a lanet olsun, Köleler kölesi olsun kardeslerine.
\par 26 Övgüler olsun Sam'in Tanrisi RAB'be, Kenan Sam'a kul olsun.
\par 27 Tanri Yafet'e bolluk versin, Sam'in çadirlarinda yasasin, Kenan Yafet'e kul olsun."
\par 28 Nuh tufandan sonra üç yüz elli yil daha yasadi.
\par 29 Toplam dokuz yüz elli yil yasadiktan sonra öldü.

\chapter{10}

\par 1 Nuh'un ogullari Sam, Ham ve Yafet'in öyküsü sudur: Tufandan sonra bunlarin birçok oglu oldu.
\par 2 Yafet'in ogullari: Gomer, Magog, Meday, Yâvan, Tuval, Mesek, Tiras.
\par 3 Gomer'in ogullari: Askenaz, Rifat, Togarma.
\par 4 Yâvan'in ogullari: Elisa, Tarsis, Kittim, Rodanim.
\par 5 Kiyilarda yasayan insanlarin atalari bunlardir. Ülkelerinde çesitli dillere, uluslarinda çesitli boylara bölündüler.
\par 6 Ham'in ogullari: Kûs, Misrayim, Pût, Kenan.
\par 7 Kûs'un ogullari: Seva, Havila, Savta, Raama, Savteka. Raama'nin ogullari: Seva, Dedan.
\par 8 Kûs'un Nemrut adinda bir oglu oldu. Yigitligiyle yeryüzüne ün saldi.
\par 9 RAB'bin önünde yigit bir avciydi. "RAB'bin önünde Nemrut gibi yigit avci" sözü buradan gelir.
\par 10 Ilkin Sinar topraklarinda, Babil, Erek, Akat, Kalne kentlerinde krallik yapti.
\par 11 Sonra Asur'a giderek Ninova, Rehovot-Ir, Kalah kentlerini ve Ninova'yla önemli bir kent olan Kalah arasinda Resen'i kurdu.
\par 12 (#10:11)
\par 13 Misrayim Ludlular'in, Anamlilar'in, Lehavlilar'in, Naftuhlular'in, Patruslular'in, Filistliler'in atalari olan Kasluhlular'in ve Kaftorlular'in atasiydi.
\par 14 (#10:13)
\par 15 Kenan ilk oglu olan Sidon'un babasi ve Hititler'in*, Yevuslular'in, Amorlular'in, Girgaslilar'in, Hivliler'in, Arklilar'in, Sinliler'in, Arvatlilar'in, Semarlilar'in, Hamalilar'in atasiydi. Kenan boylari daha sonra dagildi.
\par 16 (#10:15)
\par 17 (#10:15)
\par 18 (#10:15)
\par 19 Kenan siniri Sayda'dan Gerar, Gazze, Sodom, Gomora, Adma ve Sevoyim'e dogru Lasa'ya kadar uzaniyordu.
\par 20 Ülkelerinde ve uluslarinda çesitli boylara ve dillere bölünen Hamogullari bunlardi.
\par 21 Yafet'in agabeyi olan Sam'in da çocuklari oldu. Sam bütün Ever soyunun atasiydi.
\par 22 Sam'in ogullari: Elam, Asur, Arpaksat, Lud, Aram.
\par 23 Aram'in ogullari: Ûs, Hul, Geter, Mas.
\par 24 Arpaksat Selah'in babasiydi. Selah'tan Ever oldu.
\par 25 Ever'in iki oglu oldu. Birinin adi Pelek'ti; çünkü yeryüzündeki insanlar onun yasadigi dönemde bölündü. Kardesinin adi Yoktan'di.
\par 26 Yoktan Almodat'in, Selef'in, Hasarmavet'in, Yerah'in, Hadoram'in, Uzal'in, Dikla'nin, Oval'in, Avimael'in, Seva'nin, Ofir'in, Havila'nin, Yovav'in atasiydi. Bunlarin hepsi Yoktan'in soyundandi.
\par 27 (#10:26)
\par 28 (#10:26)
\par 29 (#10:26)
\par 30 Doguda, Mesa'dan Sefar'a uzanan daglik bölgede yasarlardi.
\par 31 Ülkelerinde ve uluslarinda çesitli boylara ve dillere bölünen Samogullari bunlardi.
\par 32 Tufandan sonra kayda geçen, ulus ulus, boy boy yeryüzüne yayilan bütün bu insanlar Nuh'un soyundan gelmedir.

\chapter{11}

\par 1 Baslangiçta dünyadaki bütün insanlar ayni dili konusur, ayni sözleri kullanirlardi.
\par 2 Doguya göçerlerken Sinar bölgesinde bir ova bulup oraya yerlestiler.
\par 3 Birbirlerine, "Gelin, tugla yapip iyice pisirelim" dediler. Tas yerine tugla, harç yerine zift kullandilar.
\par 4 Sonra, "Kendimize bir kent kuralim" dediler, "Göklere erisecek bir kule dikip ün salalim. Böylece yeryüzüne dagilmayiz."
\par 5 RAB insanlarin yaptigi kentle kuleyi görmek için asagiya indi.
\par 6 "Tek bir halk olup ayni dili konusarak bunu yapmaya basladiklarina göre, düsündüklerini gerçeklestirecek, hiçbir engel tanimayacaklar" dedi,
\par 7 "Gelin, asagi inip dillerini karistiralim ki, birbirlerini anlamasinlar."
\par 8 Böylece RAB onlari yeryüzüne dagitarak kentin yapimini durdurdu.
\par 9 Bu nedenle kente Babil adi verildi. Çünkü RAB bütün insanlarin dilini orada karistirmis ve onlari yeryüzünün dört bucagina dagitmisti.
\par 10 Sam'in soyunun öyküsü: Tufandan iki yil sonra Sam 100 yasindayken oglu Arpaksat dogdu.
\par 11 Arpaksat'in dogumundan sonra Sam 500 yil daha yasadi. Baska ogullari, kizlari oldu.
\par 12 Arpaksat 35 yasindayken oglu Selah dogdu.
\par 13 Selah'in dogumundan sonra Arpaksat 403 yil daha yasadi. Baska ogullari, kizlari oldu.
\par 14 Selah 30 yasindayken oglu Ever dogdu.
\par 15 Ever'in dogumundan sonra Selah 403 yil daha yasadi. Baska ogullari, kizlari oldu.
\par 16 Ever 34 yasindayken oglu Pelek dogdu.
\par 17 Pelek'in dogumundan sonra Ever 430 yil daha yasadi. Baska ogullari, kizlari oldu.
\par 18 Pelek 30 yasindayken oglu Reu dogdu.
\par 19 Reu'nun dogumundan sonra Pelek 209 yil daha yasadi. Baska ogullari, kizlari oldu.
\par 20 Reu 32 yasindayken oglu Seruk dogdu.
\par 21 Seruk'un dogumundan sonra Reu 207 yil daha yasadi. Baska ogullari, kizlari oldu.
\par 22 Seruk 30 yasindayken oglu Nahor dogdu.
\par 23 Nahor'un dogumundan sonra Seruk 200 yil daha yasadi. Baska ogullari, kizlari oldu.
\par 24 Nahor 29 yasindayken oglu Terah dogdu.
\par 25 Terah'in dogumundan sonra Nahor 119 yil daha yasadi. Baska ogullari, kizlari oldu.
\par 26 Yetmis yasindan sonra Terah'in Avram, Nahor ve Haran adli ogullari oldu.
\par 27 Terah soyunun öyküsü: Terah Avram, Nahor ve Haran'in babasiydi. Haran'in Lut adli bir oglu oldu.
\par 28 Haran, babasi Terah henüz sagken, dogdugu ülkede, Kildaniler'in* Ur Kenti'nde öldü.
\par 29 Avram'la Nahor evlendiler. Avram'in karisinin adi Saray, Nahor'unkinin adi Milka'ydi. Milka Yiska'nin babasi Haran'in kiziydi.
\par 30 Saray kisirdi, çocugu olmuyordu.
\par 31 Terah, oglu Avram'i, Haran'in oglu olan torunu Lut'u ve Avram'in karisi olan gelini Saray'i yanina aldi. Kenan ülkesine gitmek üzere Kildaniler'in Ur Kenti'nden ayrildilar. Harran'a gidip oraya yerlestiler.
\par 32 Terah iki yüz bes yil yasadiktan sonra Harran'da öldü

\chapter{12}

\par 1 RAB Avram'a, "Ülkeni, akrabalarini, baba evini birak, sana gösterecegim ülkeye git" dedi,
\par 2 "Seni büyük bir ulus yapacagim, Seni kutsayacak, sana ün kazandiracagim, Bereket kaynagi olacaksin.
\par 3 Seni kutsayanlari kutsayacak, Seni lanetleyeni lanetleyecegim. Yeryüzündeki bütün halklar Senin araciliginla kutsanacak."
\par 4 Avram RAB'bin buyurdugu gibi yola çikti. Lut da onunla birlikte gitti. Avram Harran'dan ayrildigi zaman yetmis bes yasindaydi.
\par 5 Karisi Saray'i, yegeni Lut'u, Harran'da kazandiklari mallari, edindikleri usaklari yanina alip Kenan ülkesine dogru yola çikti. Oraya vardilar.
\par 6 Avram ülke boyunca Sekem'deki More mesesine kadar ilerledi. O günlerde orada Kenanlilar yasiyordu.
\par 7 RAB Avram'a görünerek, "Bu topraklari senin soyuna verecegim" dedi. Avram kendisine görünen RAB'be orada bir sunak yapti.
\par 8 Oradan Beytel'in dogusundaki daglik bölgeye dogru gitti. Çadirini batidaki Beytel'le dogudaki Ay Kenti'nin arasina kurdu. Orada RAB'be bir sunak yapip RAB'be yakardi.
\par 9 Sonra kona göçe Negev'e dogru ilerledi.
\par 10 Ülkedeki siddetli kitlik yüzünden Avram geçici bir süre için Misir'a gitti.
\par 11 Misir'a yaklastiklarinda karisi Saray'a, "Güzel bir kadin oldugunu biliyorum" dedi,
\par 12 "Olur ki Misirlilar seni görüp, 'Bu onun karisi' diyerek beni öldürür, seni sag birakirlar.
\par 13 Lütfen, 'Onun kizkardesiyim' de ki, senin hatirin için bana iyi davransinlar, canima dokunmasinlar."
\par 14 Avram Misir'a girince, Misirlilar karisinin çok güzel oldugunu farkettiler.
\par 15 Kadini gören firavunun adamlari, güzelligini firavuna övdüler. Kadin saraya alindi.
\par 16 Onun hatiri için firavun Avram'a iyi davrandi. Avram davar, sigir, erkek ve disi esek, erkek ve kadin köle, deve sahibi oldu.
\par 17 RAB Avram'in karisi Saray yüzünden firavunla ev halkinin basina korkunç felaketler getirdi.
\par 18 Firavun Avram'i çagirtarak, "Nedir bana bu yaptigin?" dedi, "Neden Saray'in karin oldugunu söylemedin?
\par 19 Niçin 'Saray kizkardesimdir' diyerek onunla evlenmeme izin verdin? Al karini, git!"
\par 20 Firavun Avram için adamlarina buyruk verdi. Böylece Avram'la karisini sahip oldugu her seyle birlikte gönderdiler.

\chapter{13}

\par 1 Avram, karisi ve sahip oldugu her seyle birlikte Misir'dan ayrilip Negev'e dogru gitti. Lut da onunla birlikteydi.
\par 2 Avram çok zengindi. Sürüleri, altinlari, gümüsleri vardi.
\par 3 Negev'den baslayip bir yerden öbürüne göçerek Beytel'e kadar gitti. Beytel'le Ay Kenti arasinda daha önce çadirini kurmus oldugu yere vardi.
\par 4 Önceden yapmis oldugu sunagin bulundugu yere gidip orada RAB'be yakardi.
\par 5 Avram'la birlikte göçen Lut'un da davarlari, sigirlari, çadirlari vardi.
\par 6 Mallari öyle çoktu ki, toprak birlikte yasamalarina elvermedi; yan yana yasayamadilar.
\par 7 Avram'in çobanlariyla Lut'un çobanlari arasinda kavga çikti. -O günlerde Kenanlilar'la Perizliler de orada yasiyorlardi.-
\par 8 Avram Lut'a, "Biz akrabayiz" dedi, "Bu yüzden aramizda da çobanlarimiz arasinda da kavga çikmasin.
\par 9 Bütün topraklar senin önünde. Gel, ayrilalim. Sen sola gidersen, ben saga gidecegim. Sen saga gidersen, ben sola gidecegim."
\par 10 Lut çevresine bakti. Seria Ovasi'nin tümü RAB'bin bahçesi gibi, Soar'a dogru giderken Misir topraklari gibiydi. Her yerde bol su vardi. RAB Sodom ve Gomora kentlerini yok etmeden önce ova böyleydi.
\par 11 Lut kendine Seria Ovasi'nin tümünü seçerek doguya dogru göçtü. Birbirlerinden ayrildilar.
\par 12 Avram Kenan topraklarinda kaldi. Lut ovadaki kentlerin arasina yerlesti, Sodom'a yakin bir yere çadir kurdu.
\par 13 Sodom halki çok kötüydü. RAB'be karsi büyük günah isliyordu.
\par 14 Lut Avram'dan ayrildiktan sonra, RAB Avram'a, "Bulundugun yerden kuzeye, güneye, doguya, batiya dikkatle bak" dedi,
\par 15 "Gördügün bütün topraklari sonsuza dek sana ve soyuna verecegim.
\par 16 Soyunu topragin tozu kadar çogaltacagim. Öyle ki, biri çikip da topragin tozunu sayabilirse, senin soyunu da sayabilecek.
\par 17 Kalk, sana verecegim topraklari boydan boya dolas."
\par 18 Avram çadirini söktü, gidip Hevron'daki Mamre meseligine yerlesti. Orada RAB'be bir sunak yapti.

\chapter{14}

\par 1 Bu arada Sinar Krali Amrafel, Ellasar Krali Aryok, Elam Krali Kedorlaomer ve Goyim Krali Tidal
\par 2 Sodom Krali Bera'ya, Gomora Krali Birsa'ya, Adma Krali Sinav'a, Sevoyim Krali Semever'e ve Bala -Soar- Krali'na karsi savas açti.
\par 3 Bu son bes kral bugün Lut Gölü olan Siddim Vadisi'nde güçlerini birlestirmisti.
\par 4 Bu krallar on iki yil Kedorlaomer'in egemenligi altinda yasamis, on üçüncü yil ona baskaldirmislardi.
\par 5 On dördüncü yil Kedorlaomer'le onu destekleyen öbür krallar gelip Asterot-Karnayim'de Refalilar'i, Ham'da Zuzlular'i, Save-Kiryatayim'de Emliler'i, çöl kenarindaki El-Paran'a kadar uzanan daglik Seir bölgesinde Horlular'i bozguna ugrattilar.
\par 6 (#14:5)
\par 7 Oradan geri dönüp Eyn-Mispat'a -Kades'e- gittiler. Amalekliler'in bütün topraklarini alarak Haseson-Tamar'da yasayan Amorlular'i bozguna ugrattilar.
\par 8 Bunun üzerine Sodom, Gomora, Adma, Sevoyim, Bala -Soar- krallari yola çikti. Bu bes kral dört krala -Elam Krali Kedorlaomer, Goyim Krali Tidal, Sinar Krali Amrafel, Ellasar Krali Aryok'a- karsi Siddim Vadisi'nde savas düzenine girdiler.
\par 9 (#14:8)
\par 10 Siddim Vadisi zift çukurlariyla doluydu. Sodom ve Gomora krallari kaçarken adamlarindan bazilari bu çukurlara düstü. Sag kalanlarsa daglara kaçti.
\par 11 Dört kral Sodom ve Gomora'nin bütün malini ve yiyecegini alip gitti.
\par 12 Avram'in yegeni Lut'la mallarini da götürdüler. Çünkü o da Sodom'da yasiyordu.
\par 13 Oradan kaçip kurtulan biri gelip Ibrani Avram'a durumu bildirdi. Avram Eskol'la Aner'in kardesi Amorlu Mamre'nin meseliginde yasiyordu. Bunlarin hepsi Avram'dan yanaydilar.
\par 14 Avram yegeni Lut'un tutsak alindigini duyunca, evinde dogup yetismis üç yüz on sekiz adamini yanina alarak dört krali Dan'a kadar kovaladi.
\par 15 Adamlarini gruplara ayirdi, gece saldirip onlari bozguna ugratarak Sam'in kuzeyindeki Hova'ya kadar kovaladi.
\par 16 Yagmalanan bütün mali, yegeni Lut'la mallarini, kadinlari ve halki geri getirdi.
\par 17 Avram Kedorlaomer'le onu destekleyen krallari bozguna ugratip dönünce, Sodom Krali onu karsilamak için Kral Vadisi olan Save Vadisi'ne gitti.
\par 18 Yüce Tanri'nin kâhini* olan Salem Krali Melkisedek ekmek ve sarap getirdi.
\par 19 Avram'i kutsayarak söyle dedi: "Yeri gögü yaratan yüce Tanri Avram'i kutsasin,
\par 20 Düsmanlarini onun eline teslim eden yüce Tanri'ya övgüler olsun." Bunun üzerine Avram her seyin ondaligini Melkisedek'e verdi.
\par 21 Sodom Krali Avram'a, "Adamlarimi bana ver, mallar sana kalsin" dedi.
\par 22 Avram Sodom Krali'na, "Yeri gögü yaratan yüce Tanri RAB'bin önünde sana ait hiçbir sey, bir iplik, bir çarik bagi bile almayacagima ant içerim" diye karsilik verdi, "Öyle ki, 'Avram'i zengin ettim' demeyesin.
\par 23 (#14:22)
\par 24 Yalniz, adamlarimin yedikleri bunun disinda. Bir de beni destekleyen Aner, Eskol ve Mamre paylarina düseni alsinlar."

\chapter{15}

\par 1 Bundan sonra RAB bir görümde Avram'a, "Korkma, Avram" diye seslendi, "Senin kalkanin benim. Ödülün çok büyük olacak."
\par 2 Avram, "Ey Egemen RAB, bana ne vereceksin?" dedi, "Çocuk sahibi olamadim. Evim Samli Eliezer'e kalacak.
\par 3 Bana çocuk vermedigin için evimdeki bir usak mirasçim olacak."
\par 4 RAB yine seslendi: "O mirasçin olmayacak, öz çocugun mirasçin olacak."
\par 5 Sonra Avram'i disari çikararak, "Göklere bak" dedi, "Yildizlari sayabilir misin? Iste, soyun o kadar çok olacak."
\par 6 Avram RAB'be iman etti, RAB bunu ona dogruluk saydi.
\par 7 Tanri Avram'a, "Bu topraklari sana miras olarak vermek için Kildaniler'in* Ur Kenti'nden seni çikaran RAB benim" dedi.
\par 8 Avram, "Ey Egemen RAB, bu topraklari miras alacagimi nasil bilecegim?" diye sordu.
\par 9 RAB, "Bana bir düve, bir keçi, bir de koç getir" dedi, "Hepsi üçer yasinda olsun. Bir de kumruyla güvercin yavrusu getir."
\par 10 Avram hepsini getirdi, ortadan kesip parçalari birbirine karsi dizdi. Yalniz kuslari kesmedi.
\par 11 Leslerin üzerine konan yirtici kuslari kovdu.
\par 12 Günes batarken Avram derin bir uykuya daldi. Üzerine dehset verici zifiri bir karanlik çöktü.
\par 13 RAB Avram'a söyle dedi: "Sunu iyi bil ki, senin soyun yabanci bir ülkede, gurbette yasayacak. Dört yüz yil kölelik edip baski görecek.
\par 14 Ama soyuna kölelik yaptiran ulusu cezalandiracagim. Sonra soyun oradan büyük mal varligiyla çikacak.
\par 15 Sen de esenlik içinde atalarina kavusacaksin. Ileri yasta ölüp gömüleceksin.
\par 16 Soyunun dördüncü kusagi buraya geri dönecek. Çünkü Amorlular'in yaptigi kötülükler henüz doruga varmadi."
\par 17 Günes batip karanlik çökünce, dumanli bir mangalla alevli bir mesale göründü ve kesilen hayvan parçalarinin arasindan geçti.
\par 18 O gün RAB Avram'la antlasma yaparak ona söyle dedi: "Misir Irmagi'ndan büyük Firat Irmagi'na kadar uzanan bu topraklari -Ken, Keniz, Kadmon, Hitit*, Periz, Refa, Amor, Kenan, Girgas ve Yevus topraklarini- senin soyuna verecegim."

\chapter{16}

\par 1 Karisi Saray Avram'a çocuk verememisti. Saray'in Hacer adinda Misirli bir cariyesi vardi.
\par 2 Saray Avram'a, "RAB çocuk sahibi olmami engelledi" dedi, "Lütfen, cariyemle yat. Belki bu yoldan bir çocuk sahibi olabilirim." Avram Saray'in sözünü dinledi.
\par 3 Saray Misirli cariyesi Hacer'i kocasi Avram'a kari olarak verdi. Bu olay Avram Kenan'da on yil yasadiktan sonra oldu.
\par 4 Avram Hacer'le yatti, Hacer hamile kaldi. Hacer hamile oldugunu anlayinca, hanimini küçük görmeye basladi.
\par 5 Saray Avram'a, "Bu haksizlik senin yüzünden basima geldi!" dedi, "Cariyemi koynuna soktum. Hamile oldugunu anlayinca beni küçük görmeye basladi. Ikimiz arasinda RAB karar versin."
\par 6 Avram, "Cariyen senin elinde" dedi, "Neyi uygun görürsen yap." Böylece Saray cariyesine sert davranmaya basladi. Hacer onun yanindan kaçti.
\par 7 RAB'bin melegi Hacer'i çölde bir pinarin, Sur yolundaki pinarin basinda buldu.
\par 8 Ona, "Saray'in cariyesi Hacer, nereden gelip nereye gidiyorsun?" diye sordu. Hacer, "Hanimim Saray'dan kaçiyorum" diye yanitladi.
\par 9 RAB'bin melegi, "Hanimina dön ve ona boyun eg" dedi,
\par 10 "Senin soyunu öyle çogaltacagim ki, kimse sayamayacak.
\par 11 "Iste hamilesin, bir oglun olacak, Adini Ismail koyacaksin. Çünkü RAB sikinti içindeki yakarisini isitti.
\par 12 Oglun yaban esegine benzer bir adam olacak, O herkese, herkes de ona karsi çikacak. Kardeslerinin hepsiyle çekisme içinde yasayacak."
\par 13 Hacer, "Beni gören Tanri'yi gerçekten gördüm mü?" diyerek kendisiyle konusan RAB'be "El-Roi" adini verdi.
\par 14 Bu yüzden Kades'le Beret arasindaki o kuyuya Beer-Lahay-Roi adi verildi.
\par 15 Hacer Avram'a bir erkek çocuk dogurdu. Avram çocugun adini Ismail koydu.
\par 16 Hacer Ismail'i dogurdugunda, Avram seksen alti yasindaydi.

\chapter{17}

\par 1 Avram doksan dokuz yasindayken RAB ona görünerek, "Ben Her Seye Gücü Yeten Tanri'yim" dedi, "Benim yolumda yürü, kusursuz ol.
\par 2 Seninle yaptigim antlasmayi sürdürecek, soyunu alabildigine çogaltacagim."
\par 3 Avram yüzüstü yere kapandi. Tanri,
\par 4 "Seninle yaptigim antlasma sudur" dedi, "Birçok ulusun babasi olacaksin.
\par 5 Artik adin Avram degil, Ibrahim olacak. Çünkü seni birçok ulusun babasi yapacagim.
\par 6 Seni çok verimli kilacagim. Soyundan uluslar dogacak, krallar çikacak.
\par 7 Antlasmami seninle ve soyunla kusaklar boyunca, sonsuza dek sürdürecegim. Senin, senden sonra da soyunun Tanrisi olacagim.
\par 8 Bir yabanci olarak yasadigin topraklari, bütün Kenan ülkesini sonsuza dek mülkünüz olmak üzere sana ve soyuna verecegim. Onlarin Tanrisi olacagim."
\par 9 Tanri Ibrahim'e, "Sen ve soyun kusaklar boyu antlasmama bagli kalmalisiniz" dedi,
\par 10 "Seninle ve soyunla yaptigim antlasmanin kosulu sudur: Aranizdaki erkeklerin hepsi sünnet edilecek.
\par 11 Sünnet olmalisiniz. Sünnet aramizdaki antlasmanin belirtisi olacak.
\par 12 Evinizde dogmus ya da soyunuzdan olmayan bir yabancidan satin alinmis köleler dahil sekiz günlük her erkek çocuk sünnet edilecek. Gelecek kusaklariniz boyunca sürecek bu.
\par 13 Evinizde dogan ya da satin aldiginiz her çocuk kesinlikle sünnet edilecek. Bedeninizdeki bu belirti sonsuza dek sürecek antlasmamin simgesi olacak.
\par 14 Sünnet edilmemis her erkek halkinin arasindan atilacak, çünkü antlasmami bozmus demektir."
\par 15 Tanri, "Karin Saray'a gelince, ona artik Saray demeyeceksin" dedi, "Bundan böyle onun adi Sara olacak.
\par 16 Onu kutsayacak, ondan sana bir ogul verecegim. Onu kutsayacagim, uluslarin anasi olacak. Halklarin krallari onun soyundan çikacak."
\par 17 Ibrahim yüzüstü yere kapandi ve güldü. Içinden, "Yüz yasinda bir adam çocuk sahibi olabilir mi?" dedi, "Doksan yasindaki Sara dogurabilir mi?"
\par 18 Sonra Tanri'ya, "Keske Ismail'i mirasçim kabul etseydin!" dedi.
\par 19 Tanri, "Hayir. Ama karin Sara sana bir ogul doguracak, adini Ishak koyacaksin" dedi, "Onunla ve soyuyla antlasmami sonsuza dek sürdürecegim.
\par 20 Ismail'e gelince, seni isittim. Onu kutsayacak, verimli kilacak, soyunu alabildigine çogaltacagim. On iki beyin babasi olacak. Soyunu büyük bir ulus yapacagim.
\par 21 Ancak antlasmami gelecek yil bu zaman Sara'nin doguracagi oglun Ishak'la sürdürecegim."
\par 22 Tanri Ibrahim'le konusmasini bitirince ondan ayrilip yukariya çekildi.
\par 23 Ibrahim evindeki bütün erkekleri -oglu Ismail'i, evinde doganlarin, satin aldigi usaklarin hepsini- Tanri'nin kendisine buyurdugu gibi o gün sünnet ettirdi.
\par 24 Ibrahim sünnet oldugunda doksan dokuz yasindaydi.
\par 25 Oglu Ismail on üç yasinda sünnet oldu.
\par 26 Ibrahim, oglu Ismail'le ayni gün sünnet edildi.
\par 27 Ibrahim'in evindeki bütün erkekler -evinde doganlar ve yabancilardan satin alinanlar- onunla birlikte sünnet oldu.

\chapter{18}

\par 1 Ibrahim günün sicak saatlerinde Mamre meseligindeki çadirinin önünde otururken, RAB kendisine göründü.
\par 2 Ibrahim karsisinda üç adamin durdugunu gördü. Onlari görür görmez karsilamaya kostu. Yere kapanarak birine,
\par 3 "Ey efendim, eger gözünde lütuf bulduysam, lütfen kulunun yanindan ayrilma" dedi,
\par 4 "Biraz su getirteyim, ayaklarinizi yikayin. Su agacin altinda dinlenin.
\par 5 Madem kulunuza konuk geldiniz, birakin size yiyecek bir seyler getireyim. Biraz dinlendikten sonra yolunuza devam edersiniz." Adamlar, "Peki, dedigin gibi olsun" dediler.
\par 6 Ibrahim hemen çadira, Sara'nin yanina gitti. Ona, "Hemen üç sea ince un al, yogurup pide yap" dedi.
\par 7 Ardindan sigirlara kostu. Körpe ve besili bir buzagi seçip usagina verdi. Usak buzagiyi hemen hazirladi.
\par 8 Ibrahim hazirlanan buzagiyi yogurt ve sütle birlikte götürüp konuklarinin önüne koydu. Onlar yerken o da yanlarinda, agacin altinda durdu.
\par 9 Konuklar, "Karin Sara nerede?" diye sordular. Ibrahim, "Çadirda" diye yanitladi.
\par 10 RAB, "Gelecek yil bu zamanda kesinlikle yanina dönecegim" dedi, "O zaman karin Sara'nin bir oglu olacak." Sara RAB'bin arkasinda, çadirin girisinde durmus, dinliyordu.
\par 11 Ibrahim'le Sara kocamislardi, yaslari hayli ileriydi. Sara âdetten kesilmisti.
\par 12 Için için gülerek, "Bu yastan sonra bu sevinci tadabilir miyim?" diye düsündü, "Üstelik efendim de yasli."
\par 13 RAB Ibrahim'e sordu: "Sara niçin, 'Bu yastan sonra gerçekten çocuk sahibi mi olacagim?' diyerek güldü?
\par 14 RAB için olanaksiz bir sey var mi? Belirlenen vakitte, gelecek yil bu zaman yanina döndügümde Sara'nin bir oglu olacak."
\par 15 Sara korktu, "Gülmedim" diyerek yalan söyledi. RAB, "Hayir, güldün" dedi.
\par 16 Adamlar oradan ayrilirken Sodom'a dogru baktilar. Ibrahim onlari yolcu etmek için yanlarinda yürüyordu.
\par 17 RAB, "Yapacagim seyi Ibrahim'den mi gizleyecegim?" dedi,
\par 18 "Kuskusuz Ibrahim'den büyük ve güçlü bir ulus türeyecek, yeryüzündeki bütün uluslar onun araciligiyla kutsanacak.
\par 19 Dogru ve adil olani yaparak yolumda yürümeyi ogullarina ve soyuna buyursun diye Ibrahim'i seçtim. Öyle ki, ona verdigim sözü yerine getireyim."
\par 20 Sonra Ibrahim'e, "Sodom ve Gomora büyük suçlama altinda" dedi, "Günahlari çok agir.
\par 21 Onun için inip bakacagim. Duydugum suçlamalar dogru mu, degil mi görecegim. Bunlari yapip yapmadiklarini anlayacagim."
\par 22 Adamlar oradan ayrilip Sodom'a dogru gittiler. Ama Ibrahim RAB'bin huzurunda kaldi.
\par 23 RAB'be yaklasarak, "Haksizla birlikte hakliyi da mi yok edeceksin?" diye sordu,
\par 24 "Kentte elli dogru kisi var diyelim. Orayi gerçekten yok edecek misin? Içindeki elli dogru kisinin hatiri için kenti bagislamayacak misin?
\par 25 Senden uzak olsun bu. Hakliyi, haksizi ayni kefeye koyarak haksizin yaninda hakliyi da öldürmek senden uzak olsun. Bütün dünyayi yargilayan adil olmali."
\par 26 RAB, "Eger Sodom'da elli dogru kisi bulursam, onlarin hatirina bütün kenti bagislayacagim" diye karsilik verdi.
\par 27 Ibrahim, "Ben toz ve külüm, bir hiçim" dedi, "Ama seninle konusma yürekliligini gösterecegim.
\par 28 Kirk bes dogru kisi var diyelim, bes kisi için bütün kenti yok mu edeceksin?" RAB, "Eger kentte kirk bes dogru kisi bulursam, orayi yok etmeyecegim" dedi.
\par 29 Ibrahim yine sordu: "Ya kirk kisi bulursan?" RAB, "O kirk kisinin hatiri için hiçbir sey yapmayacagim" diye yanitladi.
\par 30 Ibrahim, "Ya Rab, öfkelenme ama, otuz kisi var diyelim?" dedi. RAB, "Otuz kisi bulursam, kente dokunmayacagim" diye yanitladi.
\par 31 Ibrahim, "Ya Rab, lütfen konusma yürekliligimi bagisla" dedi, "Eger yirmi kisi bulursan?" RAB, "Yirmi kisinin hatiri için kenti yok etmeyecegim" diye yanitladi.
\par 32 Ibrahim, "Ya Rab, öfkelenme ama, bir kez daha konusacagim" dedi, "Eger on kisi bulursan?" RAB, "On kisinin hatiri için kenti yok etmeyecegim" diye yanitladi.
\par 33 RAB Ibrahim'le konusmasini bitirince oradan ayrildi, Ibrahim de çadirina döndü.

\chapter{19}

\par 1 Iki melek aksamleyin Sodom'a vardilar. Lut kentin kapisinda oturuyordu. Onlari görür görmez karsilamak için ayaga kalkti. Yere kapanarak,
\par 2 "Efendilerim" dedi, "Kulunuzun evine buyurun. Ayaklarinizi yikayin, geceyi bizde geçirin. Sonra erkenden kalkip yolunuza devam edersiniz." Melekler, "Olmaz" dediler, "Geceyi kent meydaninda geçirecegiz."
\par 3 Ama Lut çok diretti. Sonunda onunla birlikte evine gittiler. Lut onlara yemek hazirladi, mayasiz ekmek pisirdi. Yediler.
\par 4 Onlar yatmadan, kentin erkekleri -Sodom'un her mahallesinden genç yasli bütün erkekler- evi sardi.
\par 5 Lut'a seslenerek, "Bu gece sana gelen adamlar nerede?" diye sordular, "Getir onlari da yatalim."
\par 6 Lut disari çikti, arkasindan kapiyi kapadi.
\par 7 "Kardesler, lütfen bu kötülügü yapmayin" dedi,
\par 8 "Erkek yüzü görmemis iki kizim var. Size onlari getireyim, ne isterseniz yapin. Yeter ki, bu adamlara dokunmayin. Çünkü onlar konugumdur, çatimin altina geldiler."
\par 9 Adamlar, "Çekil önümüzden!" diye karsilik verdiler, "Adam buraya disardan geldi, simdi yargiçlik tasliyor! Sana daha beterini yapariz." Lut'u ite kaka kapiyi kirmaya davrandilar.
\par 10 Ama içerdeki adamlar uzanip Lut'u evin içine, yanlarina aldilar ve kapiyi kapadilar.
\par 11 Kapiya dayanan adamlari, büyük küçük hepsini kör ettiler. Öyle ki, adamlar kapiyi bulamaz oldu.
\par 12 Içerdeki iki adam Lut'a, "Senin burada baska kimin var?" diye sordular, "Ogullarini, kizlarini, damatlarini, kentte sana ait kim varsa hepsini disari çikar.
\par 13 Çünkü burayi yok edecegiz. RAB bu halk hakkinda birçok kötü suçlama duydu, kenti yok etmek için bizi gönderdi."
\par 14 Lut disari çikti ve kizlariyla evlenecek olan adamlara, "Hemen buradan uzaklasin!" dedi, "Çünkü RAB bu kenti yok etmek üzere." Ne var ki damat adaylari onun saka yaptigini sandilar.
\par 15 Tan agarirken melekler Lut'a, "Karinla iki kizini al, hemen buradan uzaklas" diye üstelediler, "Yoksa kent cezasini bulurken sen de canindan olursun."
\par 16 Lut agir davrandi, ama RAB ona acidi. Adamlar Lut'la karisinin ve iki kizinin elinden tutup onlari kentin disina çikardilar.
\par 17 Kent disina çikinca, adamlardan biri Lut'a, "Kaç, canini kurtar, arkana bakma" dedi, "Bu ovanin hiçbir yerinde durma. Daga kaç, yoksa ölür gidersin."
\par 18 Lut, "Aman, efendim!" diye karsilik verdi,
\par 19 "Ben kulunuzdan hosnut kaldiniz, canimi kurtarmakla bana büyük iyilik yaptiniz. Ama daga kaçamam. Çünkü felaket bana yetisir, ölürüm.
\par 20 Iste, surada kaçabilecegim yakin bir kent var, küçücük bir kent. Izin verin, oraya kaçip canimi kurtarayim. Zaten küçücük bir kent."
\par 21 Adamlardan biri, "Peki, dilegini kabul ediyorum" dedi, "O kenti yikmayacagim.
\par 22 Çabuk ol, hemen kaç! Çünkü sen oraya varmadan bir sey yapamam." Bu yüzden o kente Soar adi verildi.
\par 23 Lut Soar'a vardiginda günes dogmustu.
\par 24 RAB Sodom ve Gomora'nin üzerine gökten atesli kükürt yagdirdi.
\par 25 Bu kentleri, bütün ovayi, oradaki insanlarin hepsini ve bütün bitkileri yok etti.
\par 26 Ancak Lut'un pesisira gelen karisi dönüp geriye bakinca tuz kesildi.
\par 27 Ibrahim sabah erkenden kalkip önceki gün RAB'bin huzurunda durdugu yere gitti.
\par 28 Sodom ve Gomora'ya ve bütün ovaya bakti. Yerden, tüten bir ocak gibi duman yükseliyordu.
\par 29 Tanri ovadaki kentleri yok ederken Ibrahim'i animsamis ve Lut'un yasadigi kentleri yok ederken Lut'u bu felaketin disina çikarmisti.
\par 30 Lut Soar'da kalmaktan korkuyordu. Bu yüzden iki kiziyla kentten ayrilarak daga yerlesti, onlarla birlikte bir magarada yasamaya basladi.
\par 31 Büyük kizi küçügüne, "Babamiz yasli" dedi, "Dünya geleneklerine uygun biçimde burada bizimle yatabilecek bir erkek yok.
\par 32 Gel, babamiza sarap içirelim, soyumuzu yasatmak için onunla yatalim."
\par 33 O gece babalarina sarap içirdiler. Büyük kiz gidip babasiyla yatti. Ancak Lut yatip kalktiginin farkinda degildi.
\par 34 Ertesi gün büyük kiz küçügüne, "Dün gece babamla yattim" dedi, "Bu gece de ona sarap içirelim. Soyumuzu yasatmak için sen de onunla yat."
\par 35 O gece de babalarina sarap içirdiler ve küçük kiz babasiyla yatti. Ama Lut yatip kalktiginin farkinda degildi.
\par 36 Böylece Lut'un iki kizi da öz babalarindan hamile kaldilar.
\par 37 Büyük kiz bir erkek çocuk dogurdu, ona Moav adini verdi. Moav bugünkü Moavlilar'in atasidir.
\par 38 Küçük kizin da bir oglu oldu, adini Ben-Ammi koydu. O da bugünkü Ammonlular'in atasidir.

\chapter{20}

\par 1 Ibrahim Mamre'den Negev'e göçerek Kades ve Sur kentlerinin arasina yerlesti. Sonra geçici bir süre Gerar'da kaldi.
\par 2 Karisi Sara için, "Bu kadin kizkardesimdir" dedi. Bunun üzerine Gerar Krali Avimelek adam gönderip Sara'yi getirtti.
\par 3 Ama Tanri gece düsünde Avimelek'e görünerek, "Bu kadini aldigin için öleceksin" dedi, "Çünkü o evli bir kadin."
\par 4 Avimelek henüz Sara'ya dokunmamisti. "Ya RAB" dedi, "Suçsuz bir ulusu mu yok edeceksin?
\par 5 Ibrahim'in kendisi bana, 'Bu kadin kizkardesimdir' demedi mi? Kadin da Ibrahim için, 'O kardesimdir' dedi. Ben temiz vicdanla, suçsuz ellerimle yaptim bunu."
\par 6 Tanri, düsünde ona, "Bunu temiz vicdanla yaptigini biliyorum" diye yanitladi, "Ben de seni bu yüzden bana karsi günah islemekten alikoydum, kadina dokunmana izin vermedim.
\par 7 Simdi kadini kocasina geri ver. Çünkü o bir peygamberdir. Senin için dua eder, ölmezsin. Ama kadini geri vermezsen, sen de sana ait olan herkes de ölecek, bilesin."
\par 8 Avimelek sabah erkenden kalkti, bütün adamlarini çagirarak olup biteni anlatti. Adamlar dehsete düstü.
\par 9 Avimelek Ibrahim'i çagirtarak, "Ne yaptin bize?" dedi, "Sana ne haksizlik ettim ki, beni ve kralligimi bu büyük günaha sürükledin? Bana bu yaptigin yapilacak is degil."
\par 10 Sonra, "Amacin neydi, niçin yaptin bunu?" diye sordu.
\par 11 Ibrahim, "Çünkü burada hiç Tanri korkusu yok" diye yanitladi, "Karim yüzünden beni öldürebilirler diye düsündüm.
\par 12 Üstelik, Sara gerçekten kizkardesimdir. Babamiz bir, annemiz ayridir. Onunla evlendim.
\par 13 Tanri beni babamin evinden gurbete gönderdigi zaman karima, 'Bana sevgini söyle göstereceksin: Gidecegimiz her yerde kardesin oldugumu söyle' dedim."
\par 14 Avimelek Ibrahim'e karisi Sara'yi geri verdi. Bunun yanisira ona davar, sigir, köleler, cariyeler de verdi.
\par 15 Ibrahim'e, "Iste ülkem önünde, nereye istersen oraya yerles" dedi.
\par 16 Sara'ya da, "Kardesine bin parça gümüs veriyorum" dedi, "Yanindakilere karsi senin suçsuz oldugunu gösteren bir kanittir bu. Herkes suçsuz oldugunu bilsin."
\par 17 Ibrahim Tanri'ya dua etti ve Tanri Avimelek'le karisina, cariyelerine sifa verdi. Çocuk sahibi oldular.
\par 18 Çünkü Ibrahim'in karisi Sara yüzünden RAB Avimelek'in evindeki kadinlarin hamile kalmasini engellemisti.

\chapter{21}

\par 1 RAB verdigi söz uyarinca Sara'ya iyilik etti ve sözünü yerine getirdi.
\par 2 Sara hamile kaldi; Ibrahim'in yaslilik döneminde, tam Tanri'nin belirttigi zamanda ona bir erkek çocuk dogurdu.
\par 3 Ibrahim Sara'nin dogurdugu çocuga Ishak adini verdi.
\par 4 Tanri'nin kendisine buyurdugu gibi oglu Ishak'i sekiz günlükken sünnet etti.
\par 5 Ishak dogdugunda Ibrahim yüz yasindaydi.
\par 6 Sara, "Tanri yüzümü güldürdü" dedi, "Bunu duyan herkes benimle birlikte gülecek.
\par 7 Kim Ibrahim'e Sara çocuk emzirecek derdi? Bu yasinda ona bir ogul dogurdum."
\par 8 Çocuk büyüdü. Sütten kesildigi gün Ibrahim büyük bir sölen verdi.
\par 9 Ne var ki Sara, Misirli Hacer'in Ibrahim'den olma oglu Ismail'in alay ettigini görünce,
\par 10 Ibrahim'e, "Bu cariyeyle oglunu kov" dedi, "Bu cariyenin oglu, oglum Ishak'in mirasina ortak olmasin."
\par 11 Bu Ibrahim'i çok üzdü, çünkü Ismail de öz ogluydu.
\par 12 Ancak Tanri Ibrahim'e, "Oglunla cariyen için üzülme" dedi, "Sara ne derse, onu yap. Çünkü senin soyun Ishak'la sürecektir.
\par 13 Cariyenin oglundan da bir ulus yaratacagim, çünkü o da senin soyun."
\par 14 Ibrahim sabah erkenden kalkti, biraz yiyecek, bir tulum da su hazirlayip Hacer'in omuzuna atti, çocugunu da verip onu gönderdi. Hacer Beer-Seva Çölü'ne gitti, orada bir süre dolasti.
\par 15 Tulumdaki su tükenince, oglunu bir çalinin altina birakti.
\par 16 Yaklasik bir ok atimi uzaklasip, "Oglumun ölümünü görmeyeyim" diyerek onun karsisina oturup hiçkira hiçkira agladi.
\par 17 Tanri çocugun sesini duydu. Tanri'nin melegi göklerden Hacer'e, "Nen var, Hacer?" diye seslendi, "Korkma! Çünkü Tanri çocugun sesini duydu.
\par 18 Kalk, oglunu kaldir, elini tut. Onu büyük bir ulus yapacagim."
\par 19 Sonra Tanri Hacer'in gözlerini açti, Hacer bir kuyu gördü. Gidip tulumunu doldurdu, ogluna içirdi.
\par 20 Çocuk büyürken Tanri onunlaydi. Çocuk çölde yasadi ve okçu oldu.
\par 21 Paran Çölü'nde yasarken annesi ona Misirli bir kadin aldi.
\par 22 O sirada Avimelek'le ordusunun komutani Fikol Ibrahim'e, "Yaptigin her seyde Tanri seninle" dediler,
\par 23 "Onun için, Tanri'nin önünde bana, ogluma ve soyuma haksiz davranmayacagina ant iç. Bana ve konuk olarak yasadigin bu ülkeye, benim sana yaptigim gibi iyi davran."
\par 24 Ibrahim, "Ant içerim" dedi.
\par 25 Ibrahim Avimelek'e bir kuyuyu zorla ele geçiren adamlarindan yakindi.
\par 26 Avimelek, "Bunu kimin yaptigini bilmiyorum" diye yanitladi, "Sen de bana söylemedin, ilk kez duyuyorum."
\par 27 Daha sonra Ibrahim Avimelek'e davar ve sigir verdi. Böylece ikisi bir antlasma yaptilar.
\par 28 Ibrahim sürüsünden yedi disi kuzu ayirdi.
\par 29 Avimelek, "Bunun anlami ne, niçin bu yedi disi kuzuyu ayirdin?" diye sordu.
\par 30 Ibrahim, "Bu yedi disi kuzuyu benim elimden almalisin" diye yanitladi, "Kuyuyu benim açtigimin kaniti olsun."
\par 31 Bu yüzden oraya Beer-Seva adi verildi. Çünkü ikisi orada ant içmislerdi.
\par 32 Beer-Seva'da yapilan bu antlasmadan sonra Avimelek, ordusunun komutani Fikol'la birlikte Filist yöresine geri döndü.
\par 33 Ibrahim Beer-Seva'da bir ilgin agaci dikti; orada RAB'be, ölümsüz Tanri'ya yakardi.
\par 34 Filist yöresinde konuk olarak uzun süre yasadi.

\chapter{22}

\par 1 Daha sonra Tanri Ibrahim'i denedi. "Ibrahim!" diye seslendi. Ibrahim, "Buradayim!" dedi.
\par 2 Tanri, "Ishak'i, sevdigin biricik oglunu al, Moriya bölgesine git" dedi, "Orada sana gösterecegim bir dagda oglunu yakmalik sunu* olarak sun."
\par 3 Ibrahim sabah erkenden kalkti, esegine palan vurdu. Yanina usaklarindan ikisini ve oglu Ishak'i aldi. Yakmalik sunu için odun yardiktan sonra, Tanri'nin kendisine belirttigi yere dogru yola çikti.
\par 4 Üçüncü gün gidecegi yeri uzaktan gördü.
\par 5 Usaklarina, "Siz burada, esegin yaninda kalin" dedi, "Tapinmak için oglumla birlikte oraya gidip dönecegiz."
\par 6 Yakmalik sunu için yardigi odunlari oglu Ishak'a yükledi. Atesi ve biçagi kendisi aldi. Birlikte giderlerken Ishak Ibrahim'e, "Baba!" dedi. Ibrahim, "Evet, oglum!" diye yanitladi. Ishak, "Atesle odun burada, ama yakmalik sunu kuzusu nerede?" diye sordu.
\par 7 (#22:6)
\par 8 Ibrahim, "Oglum, yakmalik sunu için kuzuyu Tanri kendisi saglayacak" dedi. Ikisi birlikte yürümeye devam ettiler.
\par 9 Tanri'nin kendisine belirttigi yere varinca Ibrahim bir sunak yapti, üzerine odun dizdi. Oglu Ishak'i baglayip sunaktaki odunlarin üzerine yatirdi.
\par 10 Onu bogazlamak için uzanip biçagi aldi.
\par 11 Ama RAB'bin melegi göklerden, "Ibrahim, Ibrahim!" diye seslendi. Ibrahim, "Iste buradayim!" diye karsilik verdi.
\par 12 Melek, "Çocuga dokunma" dedi, "Ona hiçbir sey yapma. Simdi Tanri'dan korktugunu anladim, biricik oglunu benden esirgemedin."
\par 13 Ibrahim çevresine bakinca, boynuzlari sik çalilara takilmis bir koç gördü. Gidip koçu getirdi. Oglunun yerine onu yakmalik sunu olarak sundu.
\par 14 Oraya "RAB saglar" adini verdi. "RAB'bin daginda saglanacaktir" sözü bu yüzden bugün de söyleniyor.
\par 15 RAB'bin melegi göklerden Ibrahim'e ikinci kez seslendi:
\par 16 "RAB diyor ki, kendi üzerime ant içiyorum. Bunu yaptigin için, biricik oglunu esirgemedigin için
\par 17 seni fazlasiyla kutsayacagim; soyunu göklerin yildizlari, kiyilarin kumu kadar çogaltacagim. Soyun düsmanlarinin kentlerini mülk edinecek.
\par 18 Soyunun araciligiyla yeryüzündeki bütün uluslar kutsanacak. Çünkü sözümü dinledin."
\par 19 Sonra Ibrahim usaklarinin yanina döndü. Birlikte yola çikip Beer-Seva'ya gittiler. Ibrahim Beer-Seva'da kaldi.
\par 20 Bir süre sonra Ibrahim'e, "Milka, kardesin Nahor'a çocuklar dogurdu" diye haber verdiler,
\par 21 "Ilk oglu Ûs, kardesi Bûz, Kemuel -Aram'in babasi-
\par 22 Keset, Hazo, Pildas, Yidlaf, Betuel."
\par 23 Betuel Rebeka'nin babasi oldu. Bu sekiz çocugu Ibrahim'in kardesi Nahor'a Milka dogurdu.
\par 24 Reuma adindaki cariyesi de Nahor'a Tevah, Gaham, Tahas ve Maaka'yi dogurdu.

\chapter{23}

\par 1 Sara yüz yirmi yedi yil yasadi. Ömrü bu kadardi.
\par 2 Kenan ülkesinde, bugün Hevron denilen Kiryat-Arba'da öldü. Ibrahim yas tutmak, aglamak için Sara'nin ölüsünün basina gitti.
\par 3 Sonra karisinin ölüsünün basindan kalkip Hititler'e*,
\par 4 "Ben aranizda konuk ve yabanciyim" dedi, "Bana mezar yapabilecegim bir toprak satin. Ölümü kaldirip gömeyim."
\par 5 Hititler, "Efendim, bizi dinle" diye yanitladilar, "Sen aramizda güçlü bir beysin. Ölünü mezarlarimizin en iyisine göm. Ölünü gömmen için kimse senden mezarini esirgemez."
\par 6 (#23:5)
\par 7 Ibrahim, ülke halki olan Hititler'in önünde egilerek,
\par 8 "Eger ölümü gömmemi istiyorsaniz, benim için Sohar oglu Efron'a ricada bulunun" dedi,
\par 9 "Tarlasinin dibindeki Makpela Magarasi'ni bana satsin. Fiyati neyse huzurunuzda eksiksiz ödeyip orayi mezarlik yapacagim."
\par 10 Hititli Efron halkinin arasinda oturuyordu. Kent kapisinda toplanan herkesin duyacagi biçimde,
\par 11 "Hayir, efendim!" diye karsilik verdi, "Beni dinle, magarayla birlikte tarlayi da sana veriyorum. Halkimin huzurunda onu sana veriyorum. Ölünü göm."
\par 12 Ibrahim ülke halkinin önünde egildi.
\par 13 Herkesin duyacagi biçimde Efron'a, "Lütfen beni dinle" dedi, "Tarlanin parasini ödeyeyim. Parayi kabul et ki, ölümü oraya gömeyim."
\par 14 Efron, "Efendim, beni dinle" diye karsilik verdi, "Aramizda dört yüz sekel gümüsün sözü mü olur? Ölünü göm."
\par 15 (#23:14)
\par 16 Ibrahim Efron'un önerisini kabul etti. Efron'un Hititler'in önünde sözünü ettigi dört yüz sekel gümüsü tüccarlarin agirlik ölçülerine göre tartti.
\par 17 Böylece Efron'un Mamre yakininda Makpela'daki tarlasi, çevresindeki bütün agaçlarla ve içindeki magarayla birlikte, kent kapisinda toplanan Hititler'in huzurunda Ibrahim'in mülkü kabul edildi.
\par 18 (#23:17)
\par 19 Ibrahim karisi Sara'yi Kenan ülkesinde Mamre'ye -Hevron'a- yakin Makpela Tarlasi'ndaki magaraya gömdü.
\par 20 Hititler tarlayi içindeki magarayla birlikte Ibrahim'in mezarlik yeri olarak onayladilar.

\chapter{24}

\par 1 Ibrahim kocamis, iyice yaslanmisti. RAB onu her yönden kutsamisti.
\par 2 Ibrahim, evindeki en yasli ve her seyden sorumlu usagina, "Elini uylugumun altina koy" dedi,
\par 3 "Yerin gögün Tanrisi RAB'bin adiyla ant içmeni istiyorum. Aralarinda yasadigim Kenanlilar'dan ogluma kiz almayacaksin.
\par 4 Oglum Ishak'a kiz almak için benim ülkeme, akrabalarimin yanina gideceksin."
\par 5 Usak, "Ya kiz benimle bu ülkeye gelmek istemezse?" diye sordu, "O zaman oglunu geldigin ülkeye götüreyim mi?"
\par 6 Ibrahim, "Sakin oglumu oraya götürme!" dedi,
\par 7 "Beni baba ocagindan, dogdugum ülkeden getiren, 'Bu topraklari senin soyuna verecegim' diyerek ant içen Göklerin Tanrisi RAB senin önünden melegini gönderecek. Böylece oradan ogluma bir kiz alabileceksin.
\par 8 Eger kiz seninle gelmek istemezse, içtigin ant seni baglamaz. Yalniz, oglumu oraya götürme."
\par 9 Bunun üzerine usak elini efendisi Ibrahim'in uylugunun altina koyarak bu konuda ant içti.
\par 10 Sonra efendisinden on deve alarak en iyi esyalarla birlikte yola çikti; Aram-Naharayim'e, Nahor'un yasadigi kente gitti.
\par 11 Develerini kentin disindaki kuyunun yanina çöktürdü. Aksamüzeriydi, kadinlarin su almak için disari çikacaklari zamandi.
\par 12 Usak, "Ya RAB, efendim Ibrahim'in Tanrisi, yalvaririm bugün beni basarili kil" diye dua etti, "Efendim Ibrahim'e iyilik et.
\par 13 Iste, pinarin basinda bekliyorum. Kentin kizlari su almaya geliyorlar.
\par 14 Birine, 'Lütfen testini indir, biraz su içeyim' diyecegim. O da, 'Sen iç, ben de develerine içireyim' derse, bilecegim ki o kiz kulun Ishak için seçtigin kizdir. Böylece efendime iyilik ettigini anlayacagim."
\par 15 O duasini bitirmeden, Ibrahim'in kardesi Nahor'la karisi Milka'nin oglu Betuel'in kizi Rebeka, omuzunda su testisiyle disari çikti.
\par 16 Çok güzel bir genç kizdi. Ona erkek eli degmemisti. Pinara gitti, testisini doldurup geri döndü.
\par 17 Usak onu karsilamaya kostu, "Lütfen testinden biraz su ver, içeyim" dedi.
\par 18 Rebeka, "Iç, efendim" diyerek hemen testisini indirdi, içmesi için ona uzatti.
\par 19 Ona su verdikten sonra, "Develerin için de su çekeyim" dedi, "Kanincaya kadar içsinler."
\par 20 Çabucak suyu hayvanlarin teknesine bosaltti, yine su çekmek için kuyuya kostu. Adamin bütün develeri için su çekti.
\par 21 Adam RAB'bin yolunu açip açmadigini anlamak için sessizce genç kizi süzüyordu.
\par 22 Develer su içtikten sonra, adam bir beka agirliginda altin bir burun halkasiyla on sekel agirliginda iki altin bilezik çikardi.
\par 23 "Lütfen söyle, kimin kizisin sen?" diye sordu, "Babanin evinde geceyi geçirebilecegimiz bir yer var mi?"
\par 24 Kiz, "Milka'yla Nahor'un oglu Betuel'in kiziyim" diye karsilik verdi,
\par 25 "Bizde saman ve yem bol, geceyi geçirebileceginiz yer de var."
\par 26 Adam egilip RAB'be tapindi.
\par 27 "Efendim Ibrahim'in Tanrisi RAB'be övgüler olsun" dedi, "Sevgisini, sadakatini efendimden esirgemedi. Efendimin akrabalarinin evine giden yolu bana gösterdi."
\par 28 Kiz annesinin evine kosup olanlari anlatti.
\par 29 Rebeka'nin Lavan adinda bir kardesi vardi. Lavan pinarin basindaki adama dogru kostu.
\par 30 Kizkardesinin burnundaki halkayi, kollarindaki bilezikleri görmüstü. Rebeka adamin kendisine söylediklerini de anlatinca, Lavan adamin yanina gitti. Adam pinarin basinda, develerinin yaninda duruyordu.
\par 31 Lavan, "Eve buyur, ey RAB'bin kutsadigi adam" dedi, "Niçin disarida bekliyorsun? Senin için oda, develerin için yer hazirladim."
\par 32 Böylece adam eve girdi. Lavan develerin kolanlarini çözdü, onlara saman ve yem verdi. Adamla yanindakilere ayaklarini yikamalari için su getirdi.
\par 33 Önüne yemek konulunca, adam, "Niçin geldigimi anlatmadan yemek yemeyecegim" dedi. Lavan, "Öyleyse anlat" diye karsilik verdi.
\par 34 Adam, "Ben Ibrahim'in usagiyim" dedi,
\par 35 "RAB efendimi alabildigine kutsadi. Onu zengin etti. Ona davar, sigir, altin, gümüs, erkek ve kadin köleler, develer, esekler verdi.
\par 36 Karisi Sara ileri yasta efendime bir ogul dogurdu. Efendim sahip oldugu her seyi ogluna verdi.
\par 37 'Ülkelerinde yasadigim Kenanlilar'dan ogluma kiz almayacaksin. Ogluma kiz almak için babamin ailesine, akrabalarimin yanina gideceksin' diyerek bana ant içirdi.
\par 38 (#24:37)
\par 39 "Efendime, 'Ya kiz benimle gelmezse?' diye sordum.
\par 40 "Efendim, 'Yolunda yürüdügüm RAB melegini seninle gönderecek, yolunu açacak' dedi, 'Akrabalarimdan, babamin ailesinden ogluma bir kiz getireceksin.
\par 41 Içtigin anttan ancak akrabalarimin yanina vardiginda sana kizi vermezlerse, evet, ancak o zaman özgür olabilirsin.'
\par 42 "Bugün pinarin basina geldigimde söyle dua ettim: 'Ya RAB, efendim Ibrahim'in Tanrisi, yalvaririm yolumu aç.
\par 43 Iste pinarin basinda bekliyorum. Su almaya gelen kizlardan birine, lütfen testinden bana biraz su ver, içeyim, diyecegim.
\par 44 O da, sen iç, develerin için de su çekeyim derse, anlayacagim ki efendimin oglu için RAB'bin seçtigi kiz odur.'
\par 45 "Ben içimden dua ederken, Rebeka omuzunda su testisiyle disari çikti. Pinar basina gidip su aldi. Ona, 'Lütfen, biraz su ver, içeyim' dedim.
\par 46 "Rebeka hemen testisini omuzundan indirdi, 'Iç efendim' dedi, 'Ben de develerine içireyim.' Ben içtim. Develere de su verdi.
\par 47 "Ona, 'Kimin kizisin sen?' diye sordum. "'Milka'yla Nahor'un oglu Betuel'in kiziyim' dedi. "Bunun üzerine burnuna halka, kollarina bilezik taktim.
\par 48 Egilip RAB'be tapindim. Efendimin ogluna kardesinin torununu almak için bana dogru yolu gösteren efendim Ibrahim'in Tanrisi RAB'be övgüler sundum.
\par 49 Simdi efendime sevgi ve sadakat mi göstereceksiniz, yoksa olmaz mi diyeceksiniz, bana bildirin. Öyle ki, ben de ne yapacagima karar vereyim."
\par 50 Lavan'la Betuel, "Bu RAB'bin isi" diye karsilik verdiler, "Biz sana ne iyi, ne kötü diyebiliriz.
\par 51 Iste Rebeka burada. Al götür. RAB'bin buyurdugu gibi efendinin ogluna kari olsun."
\par 52 Ibrahim'in usagi bu sözleri duyunca, yere kapanarak RAB'be tapindi.
\par 53 Rebeka'ya altin, gümüs takimlar, giysiler, kardesiyle annesine de degerli esyalar çikarip verdi.
\par 54 Sonra yanindakilerle birlikte yedi, içti. Geceyi orada geçirdiler. Sabah kalkinca Ibrahim'in usagi, "Beni yolcu edin, efendime döneyim" dedi.
\par 55 Rebeka'nin kardesiyle annesi, "Birak kiz on gün kadar bizimle kalsin, sonra gidersin" diye karsilik verdiler.
\par 56 Adam, "Madem RAB yolumu açti, beni geciktirmeyin" dedi, "Izin verin, efendime döneyim."
\par 57 "Kizi çagirip ona soralim" dediler.
\par 58 Rebeka'yi çagirip, "Bu adamla gitmek istiyor musun?" diye sordular. Rebeka, "Istiyorum" dedi.
\par 59 Böylece Rebeka'yla dadisini, Ibrahim'in usagiyla adamlarini ugurlamaya çiktilar.
\par 60 Rebeka'yi söyle kutsadilar: "Ey kizkardesimiz, Binlerce, on binlerce kisiye analik et, Soyun düsmanlarinin kentlerini mülk edinsin."
\par 61 Rebeka'yla genç hizmetçileri hazirlanip develere binerek Ibrahim'in usagini izlediler. Usak Rebeka'yi alip oradan ayrildi.
\par 62 Ishak Beer-Lahay-Roi'den gelmisti. Çünkü Negev bölgesinde yasiyordu.
\par 63 Aksamüzeri düsünmek için tarlaya gitti. Basini kaldirdiginda develerin yaklastigini gördü.
\par 64 Rebeka Ishak'i görünce deveden indi,
\par 65 Ibrahim'in usagina, "Tarladan bizi karsilamaya gelen su adam kim?" diye sordu. Usak, "Efendim" diye karsilik verdi. Rebeka peçesini alip yüzünü örttü.
\par 66 Usak bütün yaptiklarini Ishak'a anlatti.
\par 67 Ishak Rebeka'yi annesi Sara'nin yasamis oldugu çadira götürüp onunla evlendi. Böylece Rebeka Ishak'in karisi oldu. Ishak onu sevdi. Annesinin ölümünden sonra onunla avunç buldu.

\chapter{25}

\par 1 Ibrahim bir kadinla daha evlendi. Kadinin adi Ketura'ydi.
\par 2 Ondan Zimran, Yoksan, Medan, Midyan, Yisbak, Suah adli çocuklari oldu.
\par 3 Yoksan'dan da Seva, Dedan oldu. Dedan soyundan Asurlular, Letuslular, Leumlular dogdu.
\par 4 Midyan'in Efa, Efer, Hanok, Avida, Eldaa adli ogullari oldu. Bunlarin hepsi Ketura'nin soyundandi.
\par 5 Ibrahim sahip oldugu her seyi Ishak'a birakti.
\par 6 Cariyelerinin ogullarina da armaganlar verdi. Kendisi sagken bu çocuklari oglu Ishak'tan uzaklastirip doguya gönderdi.
\par 7 Ibrahim yüz yetmis bes yil yasadi. Ömrü bu kadardi.
\par 8 Kocamis, yasama doymus, iyice yaslanmis olarak son solugunu verdi. Ölüp atalarina kavustu.
\par 9 Ogullari Ishak'la Ismail onu Hititli* Sohar oglu Efron'un tarlasinda Mamre'ye yakin Makpela Magarasi'na gömdüler.
\par 10 Ibrahim o tarlayi Hititler'den satin almisti. Böylece Ibrahim'le karisi Sara oraya gömüldüler.
\par 11 Tanri Ibrahim'in ölümünden sonra oglu Ishak'i kutsadi. Ishak Beer-Lahay-Roi'de yasiyordu.
\par 12 Sara'nin cariyesi Misirli Hacer'in Ibrahim'e dogurdugu Ismail'in öyküsü:
\par 13 Dogum sirasina göre Ismail'in ogullarinin adlari sunlardir: Ilk oglu Nevayot. Sonra Kedar, Adbeel, Mivsam,
\par 14 Misma, Duma, Massa,
\par 15 Hadat, Tema, Yetur, Nafis, Kedema gelir.
\par 16 Ismail'in ogullari olan bu on iki bey oymaklarin atalariydi. Köylerine, obalarina da bu adlari verdiler.
\par 17 Ismail yüz otuz yedi yil yasadiktan sonra son solugunu verdi. Ölüp halkina kavustu.
\par 18 Ismailogullari Asur'a dogru giderken Misir siniri yakininda, Havila ile Sur arasindaki bölgeye yerlestiler. Kardeslerinin yasadigi yerin dogusuna yerlesmislerdi.
\par 19 Ibrahim'in oglu Ishak'in öyküsü:
\par 20 Ishak Aramli Lavan'in kizkardesi, Paddan-Aramli Betuel'in kizi Rebeka'yla evlendiginde kirk yasindaydi.
\par 21 Ishak karisi için RAB'be yakardi, çünkü karisi kisirdi. RAB Ishak'in yakarisini yanitladi, Rebeka hamile kaldi.
\par 22 Çocuklar karninda itisiyordu. Rebeka, "Nedir bu basima gelen?" diyerek RAB'be danismaya gitti.
\par 23 RAB onu söyle yanitladi: "Rahminde iki ulus var, Senden iki ayri halk dogacak, Biri öbüründen güçlü olacak, Büyügü küçügüne hizmet edecek."
\par 24 Dogum vakti gelince, Rebeka'nin ikiz ogullari oldu.
\par 25 Ilk dogan oglu kipkirmizi ve tüylüydü; kirmizi bir cüppeyi andiriyordu. Adini Esav koydular.
\par 26 Sonra kardesi dogdu. Eliyle Esav'in topugunu tutuyordu. Bu yüzden Ishak ona Yakup adini verdi. Rebeka dogum yaptiginda Ishak altmis yasindaydi.
\par 27 Çocuklar büyüdü. Esav kirlari seven usta bir avci oldu. Yakup'sa hep çadirda oturan sakin bir adamdi.
\par 28 Ishak Esav'i daha çok severdi, çünkü onun getirdigi av etlerini yerdi. Rebeka ise Yakup'u severdi.
\par 29 Bir gün Yakup çorba pisirirken Esav avdan geldi. Aç ve bitkindi.
\par 30 Yakup'a, "Lütfen su kizil çorbadan biraz ver de içeyim. Aç ve bitkinim" dedi. Bu nedenle ona Edom adi da verildi.
\par 31 Yakup, "Önce sen ilk ogulluk hakkini bana ver" diye karsilik verdi.
\par 32 Esav, "Baksana, açliktan ölmek üzereyim" dedi, "Ilk ogulluk hakkinin bana ne yarari var?"
\par 33 Yakup, "Önce ant iç" dedi. Esav ant içerek ilk ogulluk hakkini Yakup'a satti.
\par 34 Yakup Esav'a ekmekle mercimek çorbasi verdi. Esav yiyip içtikten sonra kalkip gitti. Böylece Esav ilk ogulluk hakkini küçümsemis oldu.

\chapter{26}

\par 1 Ibrahim'in yasadigi dönemdeki kitliktan baska ülkede bir kitlik daha oldu. Ishak Gerar'a, Filist Krali Avimelek'in yanina gitti.
\par 2 RAB Ishak'a görünerek, "Misir'a gitme" dedi, "Sana söyleyecegim ülkeye yerles.
\par 3 Orada bir süre kal. Ben seninle olacak, seni kutsayacagim: Bütün bu topraklari sana ve soyuna verecegim. Baban Ibrahim'e ant içerek verdigim sözü yerine getirecegim.
\par 4 Soyunu gökteki yildizlar kadar çogaltacagim. Bu ülkelerin tümünü onlara verecegim. Yeryüzündeki bütün uluslar senin soyun araciligiyla kutsanacak.
\par 5 Çünkü Ibrahim sözümü dinledi. Uyarilarima, buyruklarima, kurallarima, yasalarima bagli kaldi."
\par 6 Böylece Ishak Gerar'da kaldi.
\par 7 Yöre halki karisiyla ilgili soru sorunca, "Kizkardesimdir" diyordu. Çünkü "Karimdir" demekten korkuyordu. Rebeka yüzünden yöre halki beni öldürebilir diye düsünüyordu. Çünkü Rebeka güzeldi.
\par 8 Ishak orada uzun zaman kaldi. Bir gün Filist Krali Avimelek, pencereden disari bakarken, Ishak'in karisi Rebeka'yi oksadigini gördü.
\par 9 Ishak'i çagirtarak, "Bu kadin gerçekte senin karin!" dedi, "Neden kizkardesin oldugunu söyledin?" Ishak, "Çünkü onun yüzünden canimdan olurum diye düsündüm" dedi.
\par 10 Avimelek, "Nedir bize bu yaptigin?" dedi, "Az kaldi halkimdan biri karinla yatacakti. Bize suç isletecektin."
\par 11 Sonra bütün halka, "Kim bu adama ya da karisina dokunursa, kesinlikle öldürülecek" diye buyruk verdi.
\par 12 Ishak o ülkede ekin ekti ve o yil ektiginin yüz katini biçti. RAB onu kutsamisti.
\par 13 Ishak bolluga kavustu. Varligi gittikçe büyüyordu. Çok zengin oldu.
\par 14 Sürülerle davar, sigir ve birçok usak sahibi oldu. Filistliler onu kiskanmaya basladilar.
\par 15 Babasi Ibrahim yasarken kölelerinin kazmis oldugu bütün kuyulari toprakla doldurup kapadilar.
\par 16 Avimelek Ishak'a, "Ülkemizden git" dedi, "Çünkü gücün bizim gücümüzü asti."
\par 17 Ishak oradan ayrildi. Gerar Vadisi'nde çadir kurup oraya yerlesti.
\par 18 Babasi Ibrahim yasarken kazilmis olan kuyulari yeniden açtirdi. Çünkü Filistliler Ibrahim'in ölümünden sonra o kuyulari kapamislardi. Kuyulara ayni adlari, babasinin vermis oldugu adlari verdi.
\par 19 Ishak'in köleleri vadide kuyu kazarken bir kaynak buldular.
\par 20 Gerar'in çobanlari, "Su bizim" diyerek Ishak'in çobanlariyla kavgaya tutustular. Ishak kendisiyle çekistikleri için kuyuya Esek adini verdi.
\par 21 Ishak'in köleleri baska bir kuyu kazdilar. Bu kuyu yüzünden de kavga çikinca Ishak kuyuya Sitna adini verdi.
\par 22 Oradan ayrilip baska bir yerde kuyu kazdirdi. Bu kuyu yüzünden kavga çikmadi. Bu nedenle Ishak ona Rehovot adini verdi. "RAB en sonunda bize rahatlik verdi" dedi, "Bu ülkede verimli olacagiz."
\par 23 Ishak oradan Beer-Seva'ya gitti.
\par 24 O gece RAB kendisine görünerek, "Ben baban Ibrahim'in Tanrisi'yim, korkma" dedi, "Seninle birlikteyim. Seni kutsayacak, kulum Ibrahim'in hatiri için soyunu çogaltacagim."
\par 25 Ishak orada bir sunak yaparak RAB'be yakardi. Çadirini oraya kurdu. Köleleri de orada bir kuyu kazdi.
\par 26 Avimelek, danismani Ahuzzat ve ordusunun komutani Fikol ile birlikte, Gerar'dan Ishak'in yanina gitti.
\par 27 Ishak onlara, "Niçin yanima geldiniz?" dedi, "Benden nefret ediyorsunuz. Üstelik beni ülkenizden kovdunuz."
\par 28 "Açikça gördük ki, RAB seninle" diye yanitladilar, "Onun için, aramizda ant olsun: Biz nasil sana dokunmadiksa, hep iyi davranarak seni esenlik içinde gönderdikse, sen de bize kötülük etme. Bu konuda seninle anlasalim. Sen simdi RAB'bin kutsadigi bir adamsin."
\par 29 (#26:28)
\par 30 Ishak onlara bir sölen verdi, yiyip içtiler.
\par 31 Sabah erkenden kalkip karsilikli ant içtiler. Sonra Ishak onlari yolcu etti. Esenlik içinde oradan ayrildilar.
\par 32 Ayni gün Ishak'in köleleri gelip kazdiklari kuyu hakkinda kendisine bilgi verdiler, "Su bulduk" dediler.
\par 33 Ishak kuyuya Siva adini verdi. Bu yüzden kent bugüne kadar Beer-Seva diye anilir.
\par 34 Esav kirk yasinda Hititli* Beeri'nin kizi Yudit ve Hititli Elon'un kizi Basemat'la evlendi.
\par 35 Bu kadinlar Ishak'la Rebeka'nin basina dert oldular.

\chapter{27}

\par 1 Ishak yaslanmis, gözleri görmez olmustu. Büyük oglu Esav'i çagirip, "Oglum!" dedi. Esav, "Efendim!" diye yanitladi.
\par 2 Ishak, "Artik yaslandim" dedi, "Ne zaman ölecegimi bilmiyorum.
\par 3 Silahlarini -ok kilifini, yayini- al, kirlara çikip benim için bir hayvan avla.
\par 4 Sevdigim lezzetli bir yemek yap, bana getir yiyeyim. Ölmeden önce seni kutsayayim."
\par 5 Ishak, oglu Esav'la konusurken Rebeka onlari dinliyordu. Esav avlanmak için kira çikinca,
\par 6 Rebeka oglu Yakup'a söyle dedi: "Dinle, babanin agabeyin Esav'a söylediklerini duydum.
\par 7 Baban ona, 'Bana bir hayvan avla getir' dedi, 'Lezzetli bir yemek yap, yiyeyim. Ölmeden önce seni RAB'bin huzurunda kutsayayim.'
\par 8 Bak oglum, sana söyleyeceklerimi iyi dinle:
\par 9 Git süründen bana iki seçme oglak getir. Onlarla babanin sevdigi lezzetli bir yemek yapayim.
\par 10 Yemesi için onu babana sen götüreceksin. Öyle ki, ölmeden önce seni kutsasin."
\par 11 Yakup, "Ama kardesim Esav'in bedeni killi, benimkiyse kilsiz" diye yanitladi,
\par 12 "Ya babam bana dokunursa? O zaman kendisini aldattigimi anlar. Kutsama yerine üzerime lanet getirmis olurum."
\par 13 Annesi, "Sana gelecek lanet bana gelsin, oglum" dedi, "Sen beni dinle, git oglaklari getir."
\par 14 Yakup gidip oglaklari annesine getirdi. Annesi babasinin sevdigi lezzetli bir yemek yapti.
\par 15 Büyük oglu Esav'in en güzel giysileri o anda evdeydi. Rebeka onlari küçük oglu Yakup'a giydirdi.
\par 16 Ellerinin üstünü, ensesinin kilsiz yerini oglak derisiyle kapladi.
\par 17 Yaptigi güzel yemekle ekmegi Yakup'un eline verdi.
\par 18 Yakup babasinin yanina varip, "Baba!" diye seslendi. Babasi, "Evet, kimsin sen?" dedi.
\par 19 Yakup, "Ben ilk oglun Esav'im" diye karsilik verdi, "Söyledigini yaptim. Lütfen kalk, otur da getirdigim av etini ye. Öyle ki, beni kutsayabilesin."
\par 20 Ishak, "Nasil böyle çabucak buldun, oglum?" dedi. Yakup, "Tanrin RAB bana yardim etti" diye yanitladi.
\par 21 Ishak, "Yaklas, oglum" dedi, "Sana dokunayim, gerçekten oglum Esav misin, degil misin anlayayim."
\par 22 Yakup babasina yaklasti. Babasi ona dokunarak, "Ses Yakup'un sesi, ama eller Esav'in elleri" dedi.
\par 23 Onu taniyamadi. Çünkü Yakup'un elleri agabeyi Esav'in elleri gibi killiydi. Ishak onu kutsamak üzereyken,
\par 24 bir daha sordu: "Sen gerçekten oglum Esav misin?" Yakup, "Evet!" diye yanitladi.
\par 25 Ishak, "Oglum, av etini getir yiyeyim de seni kutsayayim" dedi. Yakup önce yemegi, sonra sarabi getirdi. Ishak yedi, içti.
\par 26 "Yaklas da beni öp, oglum" dedi.
\par 27 Yakup yaklasip babasini öptü. Babasi onun giysilerini kokladi ve kendisini kutsayarak söyle dedi: "Iste oglumun kokusu Sanki RAB'bin kutsadigi kirlarin kokusu.
\par 28 Tanri sana göklerin çiyinden Ve yerin verimli topraklarindan Bol bugday ve yeni sarap versin.
\par 29 Halklar sana kulluk etsin, Uluslar boyun egsin. Kardeslerine egemen ol, Kardeslerin sana boyun egsin. Sana lanet edenlere lanet olsun, Seni kutsayanlar kutsansin."
\par 30 Ishak Yakup'u kutsadiktan ve Yakup babasinin yanindan ayrildiktan hemen sonra kardesi Esav avdan döndü.
\par 31 Esav da lezzetli bir yemek yaparak babasina götürdü. Ona, "Baba, kalk, getirdigim av etini ye" dedi, "Öyle ki, beni kutsayabilesin."
\par 32 Babasi, "Sen kimsin?" diye sordu. Esav, "Ben ilk oglun Esav'im" diye karsilik verdi.
\par 33 Ishak'i bir titreme sardi. Tir tir titreyerek, "Öyleyse daha önce avlanip bana yemek getiren kimdi?" diye sordu, "Sen gelmeden önce yemegimi yiyip onu kutsadim. Artik o kutsanmis oldu."
\par 34 Esav babasinin anlattiklarini duyunca, aci aci haykirdi. "Beni de kutsa, baba, beni de!" dedi.
\par 35 Ishak, "Kardesin gelip beni kandirdi" diye karsilik verdi, "Senin yerine o kutsandi."
\par 36 Esav, "Ona bosuna mi Yakup diyorlar?" dedi, "Iki kezdir beni aldatiyor. Önce ilk ogulluk hakkimi aldi. Simdi de benim yerime o kutsandi." Sonra, "Kutsamak için bana bir hak ayirmadin mi?" diye sordu.
\par 37 Ishak, "Onu sana egemen kildim" diye yanitladi, "Bütün kardeslerini onun hizmetine verdim. Onu bugday ve yeni sarapla besledim. Senin için ne yapabilirim ki, oglum?"
\par 38 Esav, "Sen yalniz bir kisiyi mi kutsayabilirsin baba?" dedi, "Beni de kutsa, baba, beni de!" Sonra hiçkira hiçkira aglamaya basladi.
\par 39 Babasi söyle yanitladi: "Göklerin çiyinden, Zengin topraklardan Uzak yasayacaksin.
\par 40 Kilicinla yasayacak, Kardesine hizmet edeceksin. Ama özgür olmak isteyince, Onun boyundurugunu kirip atacaksin."
\par 41 Babasi Yakup'u kutsadigi için Esav kardesi Yakup'a kin bagladi. "Nasil olsa babamin ölümü yaklasti" diye düsünüyordu, "O zaman kardesim Yakup'u öldürürüm."
\par 42 Büyük oglu Esav'in ne düsündügü Rebeka'ya bildirilince Rebeka küçük oglu Yakup'u çagirtti. "Bak, agabeyin Esav seni öldürmeyi düsünerek kendini avutuyor" dedi,
\par 43 "Beni dinle, oglum. Hemen Harran'a, kardesim Lavan'in yanina kaç.
\par 44 Agabeyinin öfkesi dinip sana kizginligi geçinceye, ona yaptigini unutuncaya kadar orada kal. Birini gönderir, seni getirtirim. Niçin bir günde ikinizden de yoksun kalayim?"
\par 45 (#27:44)
\par 46 Sonra Ishak'a, "Bu Hititli* kadinlar yüzünden canimdan bezdim" dedi, "Eger Yakup da bu ülkenin kizlariyla, Hitit kizlariyla evlenirse, nasil yasarim?"

\chapter{28}

\par 1 Ishak Yakup'u çagirdi, onu kutsayarak, "Kenanli kizlarla evlenme" diye buyurdu,
\par 2 "Hemen Paddan-Aram'a, annenin babasi Betuel'in evine git. Orada dayin Lavan'in kizlarindan biriyle evlen.
\par 3 Her Seye Gücü Yeten Tanri seni kutsasin, verimli kilsin, soyunu çogaltsin; soyundan halklar türesin.
\par 4 Ibrahim'i kutsadigi gibi seni ve soyunu da kutsasin. Öyle ki, Tanri'nin Ibrahim'e verdigi topraklara -üzerinde yabanci olarak yasadigin bu topraklara- sahip olasin."
\par 5 Ishak Yakup'u böyle ugurladi. Yakup Paddan-Aram'a, kendisinin ve Esav'in annesi Rebeka'nin kardesi Aramli Betuel oglu Lavan'in yanina gitmek üzere yola çikti.
\par 6 Esav Ishak'in Yakup'u kutsadigini, evlenmek üzere Paddan-Aram'a gönderdigini ögrendi. Ayrica Yakup'u kutsarken, babasinin, "Kenanli kizlarla evlenme" diye buyurdugunu, Yakup'un da annesiyle babasini dinleyip Paddan-Aram'a gittigini ögrendi.
\par 7 (#28:6)
\par 8 Böylece babasinin Kenanli kizlardan hoslanmadigini anladi.
\par 9 Ismail'in yanina gitti. Ibrahim oglu Ismail'in kizi, Nevayot'un kizkardesi Mahalat'la evlenerek onu karilarinin üzerine getirdi.
\par 10 Yakup Beer-Seva'dan ayrilarak Harran'a dogru yola çikti.
\par 11 Bir yere varip orada geceledi, çünkü günes batmisti. Oradaki taslardan birini alip basinin altina koyarak yatti.
\par 12 Düste yeryüzüne bir merdiven dikildigini, basinin göklere eristigini gördü. Tanri'nin melekleri merdivenden çikip iniyorlardi.
\par 13 RAB yanibasinda durup, "Atan Ibrahim'in, Ishak'in Tanrisi RAB benim" dedi, "Üzerinde yattigin topraklari sana ve soyuna verecegim.
\par 14 Yeryüzünün tozu kadar sayisiz bir soya sahip olacaksin. Doguya, batiya, kuzeye, güneye dogru yayilacaksiniz. Yeryüzündeki bütün halklar sen ve soyun araciligiyla kutsanacak.
\par 15 Seninle birlikteyim. Gidecegin her yerde seni koruyacak ve bu topraklara geri getirecegim. Verdigim sözü yerine getirinceye kadar senden ayrilmayacagim."
\par 16 Yakup uyaninca, "RAB burada, ama ben farkina varamadim" diye düsündü.
\par 17 Korktu ve, "Ne korkunç bir yer!" dedi, "Bu, Tanri'nin evinden baska bir yer olamaz. Burasi göklerin kapisi."
\par 18 Ertesi sabah erkenden kalkip basinin altina koydugu tasi anit olarak dikti, üzerine zeytinyagi döktü.
\par 19 Oraya Beytel adini verdi. Kentin önceki adi Luz'du.
\par 20 Sonra bir adak adayarak söyle dedi: "Tanri benimle olur, gittigim yolda beni korur, bana yiyecek, giyecek saglarsa,
\par 21 babamin evine esenlik içinde dönersem, RAB benim Tanrim olacak.
\par 22 Anit olarak diktigim bu tas Tanri'nin evi olacak. Bana verecegin her seyin ondaligini sana verecegim."

\chapter{29}

\par 1 Yakup yoluna devam ederek dogu halklarinin ülkesine vardi.
\par 2 Kirda bir kuyu gördü. Kuyunun yanibasinda üç davar sürüsü yatiyordu. Sürülere o kuyudan su verilirdi. Kuyunun agzinda büyük bir tas vardi.
\par 3 Bütün sürüler oraya toplaninca, çobanlar kuyunun agzindaki tasi yuvarlar, davarlarini suvardiktan sonra tasi yine yerine, kuyunun agzina koyarlardi.
\par 4 Yakup çobanlara, "Kardesler, nerelisiniz?" diye sordu. Çobanlar, "Harranli'yiz" diye yanitladilar.
\par 5 Yakup, "Nahor'un torunu Lavan'i taniyor musunuz?" diye sordu. "Taniyoruz" dediler.
\par 6 Yakup, "Iyi midir?" diye sordu. "Iyidir. Iste kizi Rahel davarlarla birlikte geliyor."
\par 7 Yakup, "Aksama daha çok var" dedi, "Sürülerin toplanma vakti degil. Davarlarinizi suvarin, götürüp otlatin."
\par 8 Çobanlar, "Bütün sürüler toplanmadan, kuyunun agzindaki tasi yuvarlamadan olmaz" dediler, "Ancak o zaman davarlari suvarabiliriz."
\par 9 Yakup onlarla konusurken Rahel babasinin davarlarini getirdi. Rahel çobanlik yapiyordu.
\par 10 Yakup dayisi Lavan'in kizi Rahel'i ve davarlari görünce, gidip kuyunun agzindaki tasi yuvarladi, dayisinin davarlarini suvardi.
\par 11 Rahel'i öperek hiçkira hiçkira agladi.
\par 12 Rahel'e baba tarafindan akraba olduklarini, Rebeka'nin oglu oldugunu anlatti. Rahel kosup babasina haber verdi.
\par 13 Lavan, yegeni Yakup'un geldigini duyunca, onu karsilamaya kostu. Ona sarilip öptü, evine getirdi. Yakup bütün olanlari Lavan'a anlatti.
\par 14 Lavan, "Sen benim etim, kemigimsin" dedi. Yakup Lavan'in yaninda bir ay kaldiktan sonra,
\par 15 Lavan, "Akrabamsin diye benim için bedava mi çalisacaksin?" dedi, "Söyle, ne kadar ücret istiyorsun?"
\par 16 Lavan'in iki kizi vardi. Büyügünün adi Lea, küçügünün adi Rahel'di.
\par 17 Lea'nin gözleri alimliydi, Rahel ise boyu bosu yerinde, güzel bir kizdi.
\par 18 Yakup Rahel'e asikti. Lavan'a, "Küçük kizin Rahel için sana yedi yil hizmet ederim" dedi.
\par 19 Lavan, "Onu sana vermek baskasina vermekten daha iyidir" dedi, "Yanimda kal."
\par 20 Yakup Rahel için yedi yil çalisti. Rahel'i sevdigi için, yedi yil ona birkaç gün gibi geldi.
\par 21 Lavan'a, "Zaman doldu, kizini ver, evleneyim" dedi.
\par 22 Lavan bütün yöre halkini toplayip bir sölen verdi.
\par 23 Gece kizi Lea'yi Yakup'a götürdü. Yakup onunla yatti.
\par 24 Lavan cariyesi Zilpa'yi kizi Lea'nin hizmetine verdi.
\par 25 Sabah olunca Yakup bir de bakti ki, yanindaki Lea! Lavan'a, "Nedir bana bu yaptigin?" dedi, "Ben Rahel için yaninda çalismadim mi? Niçin beni aldattin?"
\par 26 Lavan, "Bizim buralarda adettir. Büyük kiz dururken küçük kiz evlendirilmez" dedi,
\par 27 "Bu bir haftayi tamamla, Rahel'i de sana veririz. Yalniz ona karsilik yedi yil daha yanimda çalisacaksin."
\par 28 Yakup kabul etti. Lea'yla bir hafta geçirdi. Sonra Lavan kizi Rahel'i de ona verdi.
\par 29 Cariyesi Bilha'yi Rahel'in hizmetine verdi.
\par 30 Yakup Rahel'le de yatti. Onu Lea'dan çok sevdi. Lavan'a yedi yil daha hizmet etti.
\par 31 RAB Lea'nin sevilmedigini görünce, çocuk sahibi olmasini sagladi. Oysa Rahel kisirdi.
\par 32 Lea hamile kalip bir erkek çocuk dogurdu. Adini Ruben koydu. "Çünkü RAB mutsuzlugumu gördü" dedi, "Kuskusuz artik kocam beni sever."
\par 33 Yine hamile kaldi ve bir erkek çocuk daha dogurdu. "RAB sevilmedigimi duydugu için bana bu çocugu verdi" diyerek adini Simon koydu.
\par 34 Üçüncü kez hamile kalip bir daha erkek çocuk dogurdu. "Artik kocam bana baglanacak" dedi, "Çünkü ona üç erkek çocuk dogurdum." Onun için çocuga Levi adi verildi.
\par 35 Dördüncü kez hamile kaldi ve bir erkek çocuk daha dogurdu. "Bu kez RAB'be övgüler sunacagim" dedi. Onun için çocuga Yahuda adini verdi. Bir süre dogum yapmadi.

\chapter{30}

\par 1 Rahel Yakup'a çocuk doguramayinca, ablasini kiskanmaya basladi. Yakup'a, "Bana çocuk ver, yoksa ölecegim" dedi.
\par 2 Yakup Rahel'e öfkelendi. "Çocuk sahibi olmani Tanri engelliyor. Ben Tanri degilim ki!" diye karsilik verdi.
\par 3 Rahel, "Iste cariyem Bilha" dedi, "Onunla yat, benim için çocuk dogursun, ben de aile kurayim."
\par 4 Rahel cariyesi Bilha'yi es olarak kocasina verdi. Yakup onunla yatti.
\par 5 Bilha hamile kalip Yakup'a bir erkek çocuk dogurdu.
\par 6 Rahel, "Tanri beni hakli çikardi" dedi, "Yakarisimi duyup bana bir ogul verdi." Bu yüzden çocuga Dan adini verdi.
\par 7 Rahel'in cariyesi Bilha yine hamile kaldi ve Yakup'a ikinci bir ogul dogurdu.
\par 8 Rahel, "Ablama karsi büyük savasim verdim ve onu yendim" diyerek çocuga Naftali adini verdi.
\par 9 Lea artik dogum yapamadigini görünce, cariyesi Zilpa'yi Yakup'a es olarak verdi.
\par 10 Zilpa Yakup'a bir erkek çocuk dogurdu.
\par 11 Lea, "Ugurum!" diyerek çocuga Gad adini verdi.
\par 12 Lea'nin cariyesi Zilpa Yakup'a ikinci bir ogul dogurdu.
\par 13 Lea, "Mutluyum!" dedi, "Kadinlar bana 'Mutlu' diyecek." Ve çocuga Aser adini verdi.
\par 14 Ruben hasat mevsimi tarlaya gitti. Orada adamotu bulup annesi Lea'ya getirdi. Rahel Lea'ya, "Lütfen oglunun getirdigi adamotundan bana da ver" dedi.
\par 15 Lea, "Kocami aldigin yetmez mi? Bir de oglumun adamotunu mu istiyorsun?" diye karsilik verdi. Rahel, "Öyle olsun" dedi, "Oglunun adamotuna karsilik kocam bu gece seninle yatsin."
\par 16 Aksamleyin Yakup tarladan dönerken Lea onu karsilamaya gitti. Yakup'a, "Benimle yatacaksin" dedi, "Oglumun adamotuna karsilik bu gece benimsin." Yakup o gece onunla yatti.
\par 17 Tanri Lea'nin duasini isitti. Lea hamile kalip Yakup'a besinci oglunu dogurdu.
\par 18 "Cariyemi kocama verdigim için Tanri beni ödüllendirdi" diyerek çocuga Issakar adini verdi.
\par 19 Lea yine hamile kaldi ve Yakup'a altinci oglunu dogurdu.
\par 20 "Tanri bana iyi bir armagan verdi" dedi, "Artik kocam bana deger verir. Çünkü ona alti erkek çocuk dogurdum." Ve çocuga Zevulun adini verdi.
\par 21 Bir süre sonra Lea bir kiz dogurdu ve adini Dina koydu.
\par 22 Tanri Rahel'i animsadi, onun duasini isiterek çocuk sahibi olmasini sagladi.
\par 23 Rahel hamile kaldi ve bir erkek çocuk dogurdu. "Tanri utancimi kaldirdi. RAB bana bir ogul daha versin!" diyerek çocuga Yusuf adini verdi.
\par 24 (#30:23)
\par 25 Rahel Yusuf'u dogurduktan sonra Yakup Lavan'a, "Beni gönder, evime, topraklarima gideyim" dedi,
\par 26 "Hizmetime karsilik karilarimi, çocuklarimi ver de gideyim. Sana nasil hizmet ettigimi biliyorsun."
\par 27 Lavan, "Eger benden hosnutsan, burada kal" dedi, "Çünkü fala bakarak anladim ki, RAB senin sayende beni kutsadi.
\par 28 Alacagin neyse söyle, ödeyeyim."
\par 29 Yakup, "Sana nasil hizmet ettigimi, sürülerine nasil baktigimi biliyorsun" diye karsilik verdi,
\par 30 "Ben gelmeden önce malin azdi. Sayemde RAB seni kutsadi, malin gitgide artti. Ya kendi evim için ne zaman çalisacagim?"
\par 31 Lavan, "Sana ne vereyim?" diye sordu. Yakup, "Bana bir sey verme" diye yanitladi, "Eger su önerimi kabul edersen, yine sürünü güder, hayvanlarina bakarim:
\par 32 Bugün bütün sürülerini yoklayip noktali veya benekli koyunlari, kara kuzulari, benekli veya noktali keçileri ayirayim. Ücretim bu olsun.
\par 33 Ileride bana verdiklerini denetlemeye geldiginde, dürüst olup olmadigimi kolayca anlayabilirsin. Noktali ve benekli olmayan keçilerim, kara olmayan kuzularim varsa, onlari çalmisim demektir."
\par 34 Lavan, "Kabul, söyledigin gibi olsun" dedi.
\par 35 Ama o gün çizgili ve benekli tekeleri, noktali ve benekli keçileri, beyaz keçilerin hepsini, bütün kara kuzulari ayirip ogullarina teslim etti.
\par 36 Sonra Yakup'tan üç günlük yol kadar uzaklasti. Yakup Lavan'in kalan sürüsünü gütmeye devam etti.
\par 37 Yakup aselbent, badem, çinar agaçlarindan taze dallar kesti. Dallari soyarak beyaz çentikler açti.
\par 38 Soydugu çubuklari koyunlarin önüne, su içtikleri yalaklara koydu. Koyunlar su içmeye gelince çiftlesiyorlardi.
\par 39 Çubuklarin önünde çiftlesince çizgili, noktali, benekli yavrular doguruyorlardi.
\par 40 Yakup kuzulari ayirip sürülerin yüzünü Lavan'in çizgili, kara hayvanlarina döndürüyordu. Kendi sürülerini ayri tutuyor, Lavan'inkilerle karistirmiyordu.
\par 41 Sürüdeki güçlü hayvanlar kizisinca, Yakup çubuklari onlarin gözü önüne, yalaklara koyuyordu ki, çubuklarin yaninda çiftlessinler.
\par 42 Sürünün zayif hayvanlarinin önüneyse çubuk koymuyordu. Böylece zayif hayvanlari Lavan, güçlüleri Yakup aldi.
\par 43 Yakup alabildigine zenginlesti. Çok sayida sürü, erkek ve kadin köle, deve, esek sahibi oldu.

\chapter{31}

\par 1 Lavan'in ogullari, "Yakup babamizin sahip oldugu her seyi aldi" dediler, "Bütün varligini babamiza ait seylerden kazandi." Yakup bu sözleri duyunca,
\par 2 Lavan'in kendisine karsi tutumunun eskisi gibi olmadigini anladi.
\par 3 RAB Yakup'a, "Atalarinin topraklarina, akrabalarinin yanina dön" dedi, "Seninle olacagim."
\par 4 Bunun üzerine Yakup Rahel'le Lea'yi sürüsünün bulundugu kirlara çagirtti.
\par 5 Onlara, "Bakiyorum, babaniz bana eskisi gibi davranmiyor" dedi, "Ama babamin Tanrisi benimle birlikte.
\par 6 Var gücümle babaniza hizmet ettigimi bilirsiniz.
\par 7 Ne yazik ki, babaniz beni aldatti, ondan alacagimi on kez degistirdi. Ama Tanri bana kötülük etmesine izin vermedi.
\par 8 Lavan, 'Ücret olarak noktali hayvanlari al' deyince, bütün sürü noktali dogurdu. 'Ücret olarak çizgili olanlari al' deyince de bütün sürü çizgili dogurdu.
\par 9 Tanri babanizin hayvanlarini aldi, bana verdi.
\par 10 "Sürülerin çiftlestigi mevsimde bir düs gördüm. Çiftlesen tekeler çizgili, noktali, kirçildi.
\par 11 Düsümde Tanri'nin melegi bana, 'Yakup!' diye seslendi. 'Buyur' dedim.
\par 12 Bana, 'Bak, bütün çiftlesen tekeler çizgili, noktali ve kirçil' dedi, 'Çünkü Lavan'in sana yaptiklarinin hepsini gördüm.
\par 13 Ben Beytel'in Tanrisi'yim. Hani orada bana anit dikip meshetmis*, adak adamistin. Kalk, bu ülkeden git, dogdugun ülkeye dön.'"
\par 14 Rahel'le Lea, "Babamizin evinde hâlâ payimiz, mirasimiz var mi?" dediler,
\par 15 "Onun gözünde artik yabanci degil miyiz? Çünkü bizi satti. Bizim için ödenen bedelin hepsini yedi.
\par 16 Tanri'nin babamizdan aldigi varligin tümü bize ve çocuklarimiza aittir. Tanri sana ne dediyse öyle yap."
\par 17 Böylece Yakup çocuklarini, karilarini develere bindirdi.
\par 18 Bütün hayvanlari önüne katti; topladigi mallarla, Paddan-Aram'da kazandigi hayvanlarla birlikte Kenan ülkesine, babasi Ishak'in yanina gitmek üzere yola çikti.
\par 19 Lavan koyunlarini kirkmaya gidince, Rahel babasinin putlarini çaldi.
\par 20 Yakup da kaçacagini söylemeyerek Aramli Lavan'i kandirdi.
\par 21 Böylece kendisine ait her seyi alip kaçti. Firat Irmagi'ni geçip Gilat daglik bölgesine dogru gitti.
\par 22 Üçüncü gün Yakup'un kaçtigini Lavan'a bildirdiler.
\par 23 Lavan yakinlarini yanina alip Yakup'un pesine düstü. Yedi gün sonra Gilat daglik bölgesinde ona yetisti.
\par 24 O gece Tanri Aramli Lavan'in düsüne girerek ona, "Dikkatli ol!" dedi, "Yakup'a ne iyi, ne kötü bir sey söyle."
\par 25 Lavan Yakup'a yetisti. Yakup çadirini Gilat daglik bölgesine kurmustu. Lavan da yakinlariyla birlikte çadirini ayni yere kurdu.
\par 26 Yakup'a, "Nedir bu yaptigin?" dedi, "Beni aldattin. Kizlarimi alip savas tutsagi gibi götürdün.
\par 27 Neden gizlice kaçtin? Neden beni aldattin? Niçin bana söylemedin? Seni sevinçle, ezgilerle, tefle, lirle yolcu ederdim.
\par 28 Torunlarimla, kizlarimla öpüsüp vedalasmama izin vermedin. Aptallik ettin.
\par 29 Size kötülük yapacak güçteyim, ama babanin Tanrisi dün gece bana, 'Dikkatli ol!' dedi, 'Yakup'a ne iyi, ne kötü hiçbir sey söyleme.'
\par 30 Babanin evini çok özledigin için bizden ayrildin. Ama ilahlarimi niçin çaldin?"
\par 31 Yakup, "Korktum" diye karsilik verdi, "Kizlarini zorla elimden alirsin diye düsündüm.
\par 32 Ilahlarini kimde bulursan, o öldürülecektir. Yakinlarimizin önünde kendin ara, esyalarimin arasinda sana ait ne bulursan al." Yakup ilahlari Rahel'in çaldigini bilmiyordu.
\par 33 Lavan Yakup'un, Lea'nin ve iki cariyenin çadirina baktiysa da ilahlari bulamadi. Lea'nin çadirindan çikip Rahel'in çadirina girdi.
\par 34 Rahel çaldigi putlari devesinin semerine koymus, üzerine oturmustu. Lavan çadirini didik didik aradiysa da putlari bulamadi.
\par 35 Rahel babasina, "Efendim, huzurunda kalkamadigim için kizma, âdet görüyorum da" dedi. Lavan her yeri aradiysa da, putlari bulamadi.
\par 36 Yakup kendini tutamadi. Lavan'a çikisarak, "Suçum ne?" diye sordu, "Ne günah isledim ki böyle öfkeyle pesime takildin?
\par 37 Bütün esyalarimi aradin, kendine ait bir sey buldun mu? Varsa onu buraya, yakinlarimizin önüne koy. Onlar ikimiz hakkinda karar versinler.
\par 38 Yirmi yil yaninda kaldim. Koyunlarin, keçilerin hiç düsük yapmadi. Sürülerinin içinden bir tek koç yemedim.
\par 39 Yabanil hayvanlarin parçaladigini sana göstermedim, zararini ben çektim. Gece ya da gündüz çalinan her hayvanin karsiligini benden istedin.
\par 40 Öyle bir durumdaydim ki, gündüz sicak, gece kiragi yedi bitirdi beni. Gözüme uyku girmedi.
\par 41 Yirmi yil evinde böyle yasadim. Iki kizin için on dört yil, sürün için alti yil sana hizmet ettim. On kez alacagimi degistirdin.
\par 42 Babamin ve Ibrahim'in Tanrisi, Ishak'in taptigi Tanri benden yana olmasaydi, beni eli bos gönderecektin. Tanri çektigim zorlugu, verdigim emegi gördü ve dün gece seni uyardi."
\par 43 Lavan, "Kadinlar benim kizlarim, çocuklar benim çocuklarim, sürüler benim sürülerim" diye karsilik verdi, "Burada gördügün her sey bana ait. Kizlarima ya da dogurduklari çocuklara bugün ne yapabilirim ki?
\par 44 Gel anlasalim. Aramiza tanik koyalim."
\par 45 Yakup bir tas alip onu anit olarak dikti.
\par 46 Yakinlarina, "Tas toplayin" dedi. Adamlar topladiklari taslari bir yere yigdilar. Orada, yiginin yaninda yemek yediler.
\par 47 Lavan tas yiginina Yegar-Sahaduta, Yakup ise Galet adini verdi.
\par 48 Lavan, "Bu yigin bugün aramizda tanik olsun" dedi. Bu yüzden yigina Galet adi verildi.
\par 49 Mispa diye de anilir. Çünkü Lavan, "Birbirimizden uzak oldugumuz zaman RAB aramizda gözcülük etsin" dedi,
\par 50 "Eger kizlarima kötü davranir, baska kadinlarla evlenirsen, yanimizda kimse olmasa bile Tanri tanik olacaktir."
\par 51 Sonra, "Iste tas yigini, iste aramiza diktigim anit" dedi,
\par 52 "Bu yigin ve anit birer tanik olsun. Bu yiginin ötesine geçip sana kötülük etmeyecegim. Sen de bu yigini ve aniti geçip bana kötülük etmeyeceksin.
\par 53 Ibrahim'in, Nahor'un ve babalarinin Tanrisi aramizda yargiç olsun." Yakup babasi Ishak'in taptigi Tanri'nin adiyla ant içti.
\par 54 Sonra dagda kurban kesip yakinlarini yemege çagirdi. Yemegi yiyip geceyi dagda geçirdiler.
\par 55 Lavan sabah erkenden kalkti; torunlarini, kizlarini öpüp kutsadiktan sonra evine gitti.

\chapter{32}

\par 1 Yakup yoluna devam ederken, Tanri'nin melekleriyle karsilasti.
\par 2 Onlari görünce, "Tanri'nin ordugahi bu" diyerek oraya Mahanayim adini verdi.
\par 3 Yakup Edom topraklarinda, Seir ülkesinde yasayan agabeyi Esav'a önceden haberciler gönderdi.
\par 4 Onlara su buyrugu verdi: "Efendim Esav'a söyle deyin: Kulun Yakup diyor ki, 'Simdiye kadar Lavan'in yaninda konuk olarak kaldim.
\par 5 Öküzlere, eseklere, davarlara, erkek ve kadin kölelere sahip oldum. Efendimi hosnut etmek için önceden haber gönderiyorum.'"
\par 6 Haberciler geri dönüp Yakup'a, "Agabeyin Esav'in yanina gittik" dediler, "Dört yüz adamla seni karsilamaya geliyor."
\par 7 Yakup çok korktu, sikildi. Yanindaki adamlari, davarlari, sigirlari, develeri iki gruba ayirdi.
\par 8 "Esav gelir, bir gruba saldirirsa, hiç degilse öteki grup kurtulur" diye düsündü.
\par 9 Sonra söyle dua etti: "Ey atam Ibrahim'in, babam Ishak'in Tanrisi RAB! Bana, 'Ülkene, akrabalarinin yanina dön, seni basarili kilacagim' diye söz verdin.
\par 10 Bana gösterdigin bunca iyilige, güvene layik degilim. Seria Irmagi'ni geçtigimde degnegimden baska bir seyim yoktu. Simdi iki orduyla döndüm.
\par 11 Yalvaririm, beni agabeyim Esav'dan koru. Gelip bana, çocuklarla annelerine saldirmasindan korkuyorum.
\par 12 'Seni kesinlikle basarili kilacagim, soyunu denizin kumu gibi sayilamayacak kadar çogaltacagim' diye söz vermistin bana."
\par 13 Yakup geceyi orada geçirdi. Birlikte getirdigi hayvanlardan agabeyi Esav'a armagan olarak iki yüz keçi, yirmi teke, iki yüz koyun, yirmi koç, yavrulariyla birlikte otuz disi deve, kirk inek, on boga, yirmi disi, on erkek esek ayirdi.
\par 14 (#32:13)
\par 15 (#32:13)
\par 16 Bunlari ayri sürüler halinde kölelerine teslim ederek, "Önümden gidin, sürüler arasinda bosluk birakin" dedi.
\par 17 Birinci köleye buyruk verdi: "Agabeyim Esav'la karsilastiginda, 'Sahibin kim, nereye gidiyorsun? Önündeki bu hayvanlar kimin?' diye sorarsa,
\par 18 'Kulun Yakup'un' diyeceksin, 'Efendisi Esav'a armagan olarak gönderiyor. Kendisi de arkamizdan geliyor.'"
\par 19 Ikinci ve üçüncü köleye, sürülerin pesinden giden herkese ayni buyrugu verdi: "Esav'la karsilastiginizda ayni seyleri söyleyeceksiniz.
\par 20 'Kulun Yakup arkamizdan geliyor' diyeceksiniz." "Önden gönderecegim armaganla onu yatistirir, sonra kendisini görürüm. Belki beni bagislar" diye düsünüyordu.
\par 21 Böylece armagani önden gönderip geceyi konakladigi yerde geçirdi.
\par 22 Yakup o gece kalkti; iki karisini, iki cariyesini, on bir oglunu yanina alip Yabbuk Irmagi'nin sig yerinden karsiya geçti.
\par 23 Onlari geçirdikten sonra sahip oldugu her seyi de karsiya geçirdi.
\par 24 Böylece Yakup arkada yalniz kaldi. Bir adam gün agarincaya kadar onunla güresti.
\par 25 Yakup'u yenemeyecegini anlayinca, onun uyluk kemiginin basina çarpti. Öyle ki, güresirken Yakup'un uyluk kemigi çikti.
\par 26 Adam, "Birak beni, gün agariyor" dedi. Yakup, "Beni kutsamadikça seni birakmam" diye yanitladi.
\par 27 Adam, "Adin ne?" diye sordu. "Yakup."
\par 28 Adam, "Artik sana Yakup degil, Israil denecek" dedi, "Çünkü Tanri'yla, insanlarla güresip yendin."
\par 29 Yakup, "Lütfen adini söyler misin?" diye sordu. Ama adam, "Neden adimi soruyorsun?" dedi. Sonra Yakup'u kutsadi.
\par 30 Yakup, "Tanri'yla yüzyüze görüstüm, ama canim bagislandi" diyerek oraya Peniel adini verdi.
\par 31 Yakup Peniel'den ayrilirken günes dogdu. Uylugundan ötürü aksiyordu.
\par 32 Bu nedenle Israilliler bugün bile uyluk kemiginin üzerindeki siniri yemezler. Çünkü Yakup'un uyluk kemiginin basindaki sinire çarpilmisti.

\chapter{33}

\par 1 Yakup bakti, Esav dört yüz adamiyla birlikte geliyor. Çocuklari Lea'yla Rahel'e ve iki cariyeye teslim etti.
\par 2 Cariyelerle çocuklarini öne, Lea'yla çocuklarini arkaya, Rahel'le Yusuf'u da en arkaya dizdi.
\par 3 Kendisi hepsinin önüne geçti. Agabeyine yaklasirken yedi kez yere kapandi.
\par 4 Ne var ki Esav kosarak onu karsiladi, kucaklayip boynuna sarildi, öptü. Ikisi de aglamaya basladi.
\par 5 Esav kadinlarla çocuklara bakti. "Kim bu yanindakiler?" diye sordu. Yakup, "Tanri'nin kuluna lütfettigi çocuklar" dedi.
\par 6 Cariyelerle yanlarindaki çocuklar yaklasip egildiler.
\par 7 Ardindan Lea çocuklariyla birlikte yaklasip egildi. En son da Yusuf'la Rahel yaklasip egildi.
\par 8 Esav, "Karsilastigim öbür toplulugun anlami neydi?" diye sordu. Yakup, "Efendimi hosnut etmek için" diye yanitladi.
\par 9 Esav, "Benim yeterince malim var, kardesim" dedi, "Senin malin sana kalsin."
\par 10 Yakup, "Olmaz, eger sevgini kazandimsa, lütfen armaganimi kabul et" diye karsilik verdi, "Senin yüzünü görmek Tanri'nin yüzünü görmek gibi. Çünkü beni kabul ettin.
\par 11 Lütfen sana gönderdigim armagani al. Tanri bana öyle iyilik yapti ki, her seyim var." Armagani kabul ettirinceye kadar diretti.
\par 12 Esav, "Haydi yolumuza devam edelim" dedi, "Ben önünsira gidecegim."
\par 13 Yakup, "Efendim, bilirsin, çocuklar narindir" dedi, "Yanimdaki koyunlarin, sigirlarin yavrulari var. Hayvanlari bir gün daha yürümeye zorlarsak hepsi ölür.
\par 14 Efendim, lütfen sen kulunun önünden git. Ben hayvanlarla çocuklara ayak uydurarak yavas yavas gelecegim. Seir'de efendime yetisirim."
\par 15 Esav, "Yanimdaki adamlardan birkaçini yanina vereyim" dedi. Yakup, "Niçin?" diye sordu, "Ben yalnizca seni hosnut etmek istiyorum."
\par 16 Esav o gün Seir'e dönmek üzere yola koyuldu.
\par 17 Yakup'sa Sukkot'a gitti. Orada kendine ev, hayvanlarina barinaklar yapti. Bu yüzden oraya Sukkot adini verdi.
\par 18 Yakup güvenlik içinde Paddan-Aram'dan Kenan ülkesine, Sekem Kenti'ne vardi. Kentin yakininda konakladi.
\par 19 Çadirini kurdugu arsayi Sekem'in babasi Hamor'un ogullarindan yüz parça gümüse aldi.
\par 20 Orada bir sunak kurarak El-Elohe-Israil adini verdi.

\chapter{34}

\par 1 Lea'yla Yakup'un kizi Dina bir gün yöre kadinlarini ziyarete gitti.
\par 2 O bölgenin beyi Hivli Hamor'un oglu Sekem Dina'yi görünce tutup irzina geçti.
\par 3 Yakup'un kizina gönlünü kaptirdi. Dina'yi sevdi ve ona nazik davrandi.
\par 4 Babasi Hamor'a, "Bu kizi bana es olarak al" dedi.
\par 5 Yakup kizi Dina'nin kirletildigini duydugunda, ogullari kirda hayvanlarin basindaydi. Yakup onlar gelinceye kadar konusmadi.
\par 6 Bu arada Sekem'in babasi Hamor konusmak için Yakup'un yanina gitti.
\par 7 Yakup'un ogullari olayi duyar duymaz kirdan döndüler. Üzüntülü ve çok öfkeliydiler. Çünkü Sekem Yakup'un kiziyla yatarak Israil'in onurunu kirmisti. Böyle bir sey olmamaliydi.
\par 8 Hamor onlara, "Oglum Sekem'in gönlü kizinizda" dedi, "Lütfen onu ogluma es olarak verin.
\par 9 Bizimle akraba olun. Birbirimize kiz verip kiz alalim.
\par 10 Bizimle birlikte yasayin. Ülke önünüzde, nereye isterseniz yerlesin, ticaret yapin, mülk edinin."
\par 11 Sekem de Dina'nin babasiyla kardeslerine, "Bana bu iyiligi yapin, ne isterseniz veririm" dedi,
\par 12 "Ne kadar baslik ve armagan isterseniz isteyin, dilediginiz her seyi verecegim. Yeter ki, kizi bana es olarak verin."
\par 13 Kizkardesleri Dina'nin irzina geçildigi için, Yakup'un ogullari Sekem'le babasi Hamor'a aldatici bir yanit verdiler.
\par 14 "Olmaz, kizkardesimizi sünnetsiz* bir adama veremeyiz" dediler, "Bizim için utanç olur.
\par 15 Ancak su kosulla kabul ederiz: Bütün erkekleriniz bizim gibi sünnet olursa,
\par 16 birbirimize kiz verip kiz alabiliriz. Sizinle birlikte yasar, bir halk oluruz.
\par 17 Eger kabul etmez, sünnet olmazsaniz, kizimizi alir gideriz."
\par 18 Bu öneri Hamor'la oglu Sekem'e iyi göründü.
\par 19 Ailesinde en saygin kisi olan genç Sekem öneriyi yerine getirmekte gecikmedi. Çünkü Yakup'un kizina asikti.
\par 20 Hamor'la oglu Sekem durumu kent halkina bildirmek için kentin kapisina gittiler.
\par 21 "Bu adamlar bize dostluk gösteriyor" dediler, "Ülkemizde yasasinlar, ticaret yapsinlar. Topraklarimiz genis, onlara da yeter, bize de. Birbirimize kiz verip kiz alabiliriz.
\par 22 Yalniz, su kosulla bizimle birlesmeyi, birlikte yasamayi kabul ediyorlar: Bizim erkeklerin de kendileri gibi sünnet olmasini istiyorlar.
\par 23 Böylece bütün sürüleri, mallari, öbür hayvanlari da bizim olur, degil mi? Gelin onlarla anlasalim, bizimle birlikte yasasinlar."
\par 24 Kent kapisindan geçen herkes Hamor'la oglu Sekem'in söylediklerini kabul etti ve kentteki bütün erkekler sünnet oldu.
\par 25 Üçüncü gün erkekler daha sünnetin acisini çekerken, Yakup'un ogullarindan ikisi -Dina'nin kardesleri Simon'la Levi- kiliçlarini kusanip kusku uyandirmadan kente girip bütün erkekleri kiliçtan geçirdiler.
\par 26 Hamor'la oglu Sekem'i de öldürdüler. Dina'yi Sekem'in evinden alip gittiler.
\par 27 Sonra Yakup'un bütün ogullari cesetleri soyup kenti yagmaladilar. Çünkü kizkardeslerini kirletmislerdi.
\par 28 Kentteki ve kirdaki davarlari, sigirlari, esekleri ele geçirdiler.
\par 29 Bütün mallarini, çocuklarini, kadinlarini aldilar, evlerindeki her seyi yagmaladilar.
\par 30 Yakup, Simon'la Levi'ye, "Bu ülkede yasayan Kenanlilar'la Perizliler'i bana düsman ettiniz, basimi belaya soktunuz" dedi, "Sayica aziz. Eger birlesir, bana saldirirlarsa, ailemle birlikte yok olurum."
\par 31 Simon'la Levi, "Kizkardesimize bir fahise gibi mi davranmaliydi?" diye karsilik verdiler.

\chapter{35}

\par 1 Tanri Yakup'a, "Git, Beytel'e yerles" dedi, "Agabeyin Esav'dan kaçarken sana görünen Tanri'ya orada bir sunak yap."
\par 2 Yakup ailesine ve yanindakilere, "Yabanci ilahlarinizi atin" dedi, "Kendinizi arindirip giysilerinizi degistirin.
\par 3 Beytel'e gidelim. Sikinti çektigim günlerde yakarisimi duyan, gittigim her yerde benimle birlikte olan Tanri'ya orada bir sunak yapacagim."
\par 4 Böylece herkes yabanci ilahlarini, kulaklarindaki küpeleri Yakup'a verdi. Yakup bunlari Sekem yakinlarinda bir yabanil fistik agacinin altina gömdü.
\par 5 Sonra göçtüler. Çevre kentlerde yasayan halk peslerine düsmedi, çünkü hepsini Tanri korkusu sarmisti.
\par 6 Yakup adamlariyla birlikte Kenan ülkesindeki Luz -Beytel- Kenti'ne geldi.
\par 7 Bir sunak yaparak oraya El-Beytel adini verdi. Çünkü agabeyinden kaçarken Tanri orada kendisine görünmüstü.
\par 8 Rebeka'nin dadisi Debora ölünce Beytel'in güneyindeki mese agacinin altina gömüldü. Bu yüzden agaca Allon-Bakut adi verildi.
\par 9 Yakup Paddan-Aram'dan dönünce, Tanri ona yine görünerek onu kutsadi.
\par 10 "Sana Yakup diyorlar, ama bundan böyle adin Yakup degil, Israil olacak" diyerek onun adini Israil koydu.
\par 11 "Ben Her Seye Gücü Yeten Tanri'yim" dedi, "Verimli ol, çogal. Senden bir ulus ve uluslar toplulugu dogacak. Krallarin atasi olacaksin.
\par 12 Ibrahim'e, Ishak'a verdigim topraklari sana verecek, senden sonra da soyuna bagislayacagim."
\par 13 Sonra Tanri Yakup'tan ayrilarak onunla konustugu yerden yukari çekildi.
\par 14 Yakup Tanri'nin kendisiyle konustugu yere tas bir anit dikti. Üzerine dökmelik sunu ve zeytinyagi döktü.
\par 15 Oraya, Tanri'nin kendisiyle konustugu yere Beytel adini verdi.
\par 16 Sonra Beytel'den göçtüler. Efrat'a varmadan Rahel dogum yapti. Dogum yaparken çok sanci çekti.
\par 17 O sanci çekerken, ebesi, "Korkma!" dedi, "Bir oglun daha oluyor."
\par 18 Ama Rahel ölmek üzereydi. Can verirken oglunun adini Ben-Oni koydu. Babasi ise çocuga Benyamin adini verdi.
\par 19 Rahel öldü ve Efrat -Beytlehem- yolunda gömüldü.
\par 20 Yakup Rahel'in mezarina bir tas dikti. Bu mezar tasi bugüne kadar kaldi.
\par 21 Israil yine göçtü ve Eder Kulesi'nin ötesinde konakladi.
\par 22 Israil o bölgede yasarken Ruben babasinin cariyesi Bilha'yla yatti. Israil bunu duyunca çok kizdi.
\par 23 Yakup'un on iki oglu vardi. Lea'nin ogullari: Ruben -Yakup'un ilk oglu- Simon, Levi, Yahuda, Issakar, Zevulun.
\par 24 Rahel'in ogullari: Yusuf, Benyamin.
\par 25 Rahel'in cariyesi Bilha'nin ogullari: Dan, Naftali.
\par 26 Lea'nin cariyesi Zilpa'nin ogullari: Gad, Aser. Yakup'un Paddan-Aram'da dogan ogullari bunlardir.
\par 27 Yakup, Ishak'la Ibrahim'in de yabanci olarak kalmis oldugu bugün Hevron denen Kiryat-Arba yakinlarindaki Mamre'ye, babasi Ishak'in yanina gitti.
\par 28 Ishak yüz seksen yil yasadi.
\par 29 Kocamis, yasama doymus olarak son solugunu verdi. Ölüp halkina kavustu. Ogullari Esav'la Yakup onu gömdüler.

\chapter{36}

\par 1 Esav'in, yani Edom'un öyküsü:
\par 2 Esav su Kenanli kizlarla evlendi: Hititli* Elon'un kizi Âda; Hivli Sivon'un torunu, Âna'nin kizi Oholivama;
\par 3 Nevayot'un kizkardesi, Ismail'in kizi Basemat.
\par 4 Âda Esav'a Elifaz'i, Basemat Reuel'i,
\par 5 Oholivama Yeus, Yalam ve Korah'i dogurdu. Esav'in Kenan ülkesinde dogan ogullari bunlardi.
\par 6 Esav karilarini, ogullarini, kizlarini, evindeki bütün adamlarini, hayvanlarinin hepsini, Kenan ülkesinde kazandigi mallarin tümünü alip kardesi Yakup'tan ayrildi, baska bir ülkeye gitti.
\par 7 Birlikte yasayamayacak kadar çok mallari vardi. Yabanci olarak yasadiklari bu topraklar davarlarina yetmiyordu.
\par 8 Esav -Edom- Seir daglik bölgesine yerlesti.
\par 9 Seir daglik bölgesine yerlesen Edomlular'in atasi Esav'insoyu:
\par 10 Esav'in ogullarinin adlari sunlardir:Esav'in karilarindan Âda'nin oglu Elifaz, Basemat'in oglu Reuel.
\par 11 Elifaz'in ogullari:Teman, Omar, Sefo, Gatam, Kenaz.
\par 12 Timna Esav'in oglu Elifaz'in cariyesiydi. Elifaz'a Amalek'idogurdu. Bunlar Esav'in karisi Âda'nin torunlaridir.
\par 13 Reuel'in ogullari:Nahat, Zerah, Samma, Mizza.Bunlar Esav'in karisi Basemat'in torunlaridir.
\par 14 Sivon'un torunu ve Âna'nin kizi olan Esav'in karisiOholivama'nin Esav'a dogurdugu ogullar sunlardir:Yeus, Yalam, Korah.
\par 15 Esavogullari'nin boy beyleri sunlardir:Esav'in ilk oglu Elifaz'in ogullari:Teman, Omar, Sefo, Kenaz,
\par 16 Korah, Gatam, Amalek. Bunlar Edom ülkesinde Elifaz'insoyundan beylerdi ve Âda'nin torunlariydi.
\par 17 Esav oglu Reuel'in ogullari sunlardir:Nahat, Zerah, Samma, Mizza. Bunlar Edom ülkesinde Reuel'in soyundan gelen beylerdi ve Esav'in karisi Basemat'in torunlariydi.
\par 18 Esav'in karisi Oholivama'nin ogullari sunlardir:Yeus, Yalam, Korah. Bunlar Âna'nin kizi olan Esav'in karisi Oholivama'nin soyundan gelen beylerdi.
\par 19 Bunlarin hepsi Esav'in -Edom'un- ogullaridir. Yukardakilerde onlarin beyleridir.
\par 20 Ülkede yasayan Horlu Seir'in ogullari sunlardi: Lotan,Soval, Sivon, Âna,
\par 21 Dison, Eser, Disan. Seir'in Edom'da beylik eden Horlu ogullari bunlardi.
\par 22 Lotan'in ogullari: Hori, Hemam. Timna Lotan'in kizkardesiydi.
\par 23 Soval'in ogullari:Alvan, Manahat, Eval, Sefo, Onam.
\par 24 Sivon'un ogullari:Aya ve Âna. Babasi Sivon'un eseklerini güderken çölde sicak su kaynaklari bulan Âna'dir bu.
\par 25 Âna'nin çocuklari sunlardi:Dison ve Âna'nin kizi Oholivama.
\par 26 Dison'un ogullari sunlardi:Hemdan, Esban, Yitran, Keran.
\par 27 Eser'in ogullari sunlardi: Bilhan, Zaavan, Akan.
\par 28 Disan'in ogullari sunlardi:Ûs, Aran.
\par 29 Horlu boy beyleri sunlardi Lotan, Soval, Sivon, Âna,
\par 30 Dison, Eser, Disan. Seir ülkesindeki Horlu boy beyleri bunlardi.
\par 31 Israilliler'i yöneten bir kralin olmadigi dönemde, Edom'u su krallar yönetti:
\par 32 Beor oglu Bala Edom Krali oldu. Kentinin adi Dinhava'ydi.
\par 33 Bala ölünce, yerine Bosrali Zerah oglu Yovav geçti.
\par 34 Yovav ölünce, Temanlilar ülkesinden Husam kral oldu.
\par 35 Husam ölünce, Midyan'i Moav kirlarinda bozguna ugratan Bedat oglu Hadat kral oldu. Kentinin adi Avit'ti.
\par 36 Hadat ölünce, yerine Masrekali Samla geçti.
\par 37 Samla ölünce, yerine Rehovot-Hannaharli Saul geçti.
\par 38 Saul ölünce, yerine Akbor oglu Baal-Hanan geçti.
\par 39 Akbor oglu Baal-Hanan ölünce, yerine Hadat geçti. Kentinin adi Pau'ydu. Karisi, Me-Zahav kizi Matret'in kizi Mehetavel'di.
\par 40 Boylarina ve bölgelerine göre Esav'in soyundan gelen beylerin adlari sunlardi: Timna, Alva, Yetet,
\par 41 Oholivama, Ela, Pinon,
\par 42 Kenaz, Teman, Mivsar,
\par 43 Magdiel, Iram. Sahip olduklari ülkede yasadiklari yerlere adlarini veren Edom beyleri bunlardi. Edomlular'in atasi Esav'di.

\chapter{37}

\par 1 Yakup babasinin yabanci olarak kalmis oldugu Kenan ülkesinde yasadi.
\par 2 Yakup soyunun öyküsü: Yusuf on yedi yasinda bir gençti. Babasinin karilari Bilha ve Zilpa'dan olan üvey kardesleriyle birlikte sürü güdüyordu. Kardeslerinin yaptigi kötülükleri babasina ulastirirdi.
\par 3 Israil Yusuf'u öbür ogullarinin hepsinden çok severdi. Çünkü Yusuf onun yasliliginda dogmustu. Yusuf'a uzun, renkli bir giysi yaptirmisti.
\par 4 Yusuf'un kardesleri babalarinin onu kendilerinden çok sevdigini görünce, ondan nefret ettiler. Yusuf'a tatli söz söylemez oldular.
\par 5 Yusuf bir düs gördü. Bunu kardeslerine anlatinca, ondan daha çok nefret ettiler.
\par 6 Yusuf, "Lütfen gördügüm düsü dinleyin!" dedi,
\par 7 "Tarlada demet bagliyorduk. Ansizin benim demetim kalkip dikildi. Sizinkilerse, çevresine toplanip önünde egildiler."
\par 8 Kardesleri, "Basimiza kral mi olacaksin? Bizi sen mi yöneteceksin?" dediler. Düslerinden, söylediklerinden ötürü ondan büsbütün nefret ettiler.
\par 9 Yusuf bir düs daha görüp kardeslerine anlatti. "Dinleyin, bir düs daha gördüm" dedi, "Günes, ay ve on bir yildiz önümde egildiler."
\par 10 Yusuf babasiyla kardeslerine bu düsü anlatinca, babasi onu azarladi: "Ne biçim düs bu?" dedi, "Ben, annen, kardeslerin gelip önünde yere mi egilecegiz yani?"
\par 11 Kardesleri Yusuf'u kiskaniyordu, ama bu olay babasinin aklina takildi.
\par 12 Bir gün Yusuf'un kardesleri babalarinin sürüsünü gütmek için Sekem'e gittiler.
\par 13 Israil Yusuf'a, "Kardeslerin Sekem'de sürü güdüyorlar" dedi, "Gel seni de onlarin yanina göndereyim." Yusuf, "Hazirim" diye yanitladi.
\par 14 Babasi, "Git kardeslerine ve sürüye bak" dedi, "Her sey yolunda mi, degil mi, bana haber getir." Böylece onu Hevron Vadisi'nden gönderdi. Yusuf Sekem'e vardi.
\par 15 Kirda dolasirken bir adam onu görüp, "Ne ariyorsun?" diye sordu.
\par 16 Yusuf, "Kardeslerimi ariyorum" diye yanitladi, "Buralarda sürü güdüyorlar. Nerede olduklarini biliyor musun?"
\par 17 Adam, "Buradan ayrildilar" dedi, "'Dotan'a gidelim' dediklerini duydum." Böylece Yusuf kardeslerinin pesinden gitti ve Dotan'da onlari buldu.
\par 18 Kardesleri onu uzaktan gördüler. Yusuf yanlarina varmadan, onu öldürmek için düzen kurdular.
\par 19 Birbirlerine, "Iste düs hastasi geliyor" dediler,
\par 20 "Hadi onu öldürüp kuyulardan birine atalim. Yabanil bir hayvan yedi deriz. Bakalim o zaman düsleri ne olacak!"
\par 21 Ruben bunu duyunca Yusuf'u kurtarmaya çalisti: "Canina kiymayin" dedi,
\par 22 "Kan dökmeyin. Onu su issiz yerdeki kuyuya atin, ama kendisine dokunmayin." Amaci Yusuf'u kurtarip babasina geri götürmekti.
\par 23 Yusuf yanlarina varinca, kardesleri sirtindaki renkli uzun giysiyi çekip çikardilar
\par 24 ve onu susuz, bos bir kuyuya attilar.
\par 25 Yemek yemek için oturduklarinda, Gilat yönünden bir Ismaili kervaninin geldigini gördüler. Develeri kitre, pelesenk, laden yüklüydü. Misir'a gidiyorlardi.
\par 26 Yahuda, kardeslerine, "Kardesimizi öldürür, suçumuzu gizlersek ne kazaniriz?" dedi,
\par 27 "Gelin onu Ismaililer'e satalim. Böylece canina dokunmamis oluruz. Çünkü o kardesimizdir, ayni kani tasiyoruz." Kardesleri kabul etti.
\par 28 Midyanli tüccarlar oradan geçerken, kardesleri Yusuf'u kuyudan çekip çikardilar, yirmi gümüse Ismaililer'e sattilar. Ismaililer Yusuf'u Misir'a götürdüler.
\par 29 Kuyuya geri dönen Ruben Yusuf'u orada göremeyince üzüntüden giysilerini yirtti.
\par 30 Kardeslerinin yanina gidip, "Çocuk orada yok" dedi, "Ne yapacagim simdi ben?"
\par 31 Bunun üzerine bir teke keserek Yusuf'un renkli uzun giysisini kanina buladilar.
\par 32 Giysiyi babalarina götürerek, "Bunu bulduk" dediler, "Bak, bakalim, oglunun mu, degil mi?"
\par 33 Yakup giysiyi tanidi, "Evet, bu oglumun giysisi" dedi, "Onu yabanil bir hayvan yemis olmali. Yusuf'u parçalamis olsa gerek."
\par 34 Yakup üzüntüden giysilerini yirtti, beline çul sardi, oglu için uzun süre yas tuttu.
\par 35 Bütün ogullari, kizlari onu avutmaya çalistilarsa da o avunmak istemedi. "Oglumun yanina, ölüler diyarina yas tutarak gidecegim" diyerek oglu için aglamaya devam etti.
\par 36 Bu arada Midyanlilar da Yusuf'u Misir'da firavunun bir görevlisine, muhafiz birligi komutani Potifar'a sattilar.

\chapter{38}

\par 1 O siralarda Yahuda kardeslerinden ayrilarak Adullamli Hira adinda bir adamin yanina gitti.
\par 2 Orada Kenanli bir kizla karsilasti. Kizin babasinin adi Sua'ydi. Yahuda kizla evlendi.
\par 3 Kadin hamile kaldi ve bir erkek çocuk dogurdu. Yahuda ona Er adini verdi.
\par 4 Kadin yine hamile kaldi, bir erkek çocuk daha dogurdu, adini Onan koydu.
\par 5 Yine bir erkek çocuk dogurdu, adini Sela koydu. Sela dogdugu zaman Yahuda Keziv'deydi.
\par 6 Yahuda ilk oglu Er için bir kadin aldi. Kadinin adi Tamar'di.
\par 7 Yahuda'nin ilk oglu Er, RAB'bin gözünde kötüydü. Bu yüzden RAB onu öldürdü.
\par 8 Yahuda Onan'a, "Kardesinin karisiyla evlen" dedi, "Kayinbiraderlik görevini yap. Kardesinin soyunu sürdür."
\par 9 Ama Onan dogacak çocuklarin kendisine ait olmayacagini biliyordu. Bu yüzden ne zaman kardesinin karisiyla yatsa, kardesine soy yetistirmemek için menisini yere bosaltiyordu.
\par 10 Bu yaptigi RAB'bin gözünde kötüydü. Bu yüzden RAB onu da öldürdü.
\par 11 Bunun üzerine Yahuda, gelini Tamar'a, "Babanin evine dön" dedi, "Oglum Sela büyüyünceye kadar orada dul olarak yasa." Yahuda, "Sela da kardesleri gibi ölebilir" diye düsünüyordu. Böylece Tamar babasinin evine döndü.
\par 12 Uzun süre sonra Sua'nin kizi olan Yahuda'nin karisi öldü. Yahuda yasi bittikten sonra arkadasi Adullamli Hira'yla birlikte Timna'ya, sürüsünü kirkanlarin yanina gitti.
\par 13 Tamar'a, "Kayinbaban sürüsünü kirkmak için Timna'ya gidiyor" diye haber verdiler.
\par 14 Tamar üzerindeki dul giysilerini çikardi. Peçesini örttü, sarinip Timna yolu üzerindeki Enayim Kapisi'nda oturdu. Çünkü Sela büyüdügü halde onunla evlenmesine izin verilmedigini görmüstü.
\par 15 Yahuda onu görünce fahise sandi. Çünkü yüzü örtülüydü.
\par 16 Yolun kenarina, ona dogru segirterek, kendi gelini oldugunu bilmeden, "Hadi gel, seninle yatmak istiyorum" dedi. Tamar, "Seninle yatarsam, bana ne vereceksin?" diye sordu.
\par 17 Yahuda, "Sürümden sana bir oglak göndereyim" dedi. Tamar, "Oglagi gönderinceye kadar rehin olarak bana bir sey verebilir misin?" dedi.
\par 18 Yahuda, "Ne vereyim?" diye sordu. Tamar, "Mührünü, kaytanini ve elindeki degnegi" diye yanitladi. Yahuda bunlari verip onunla yatti. Tamar hamile kaldi.
\par 19 Gidip peçesini çikardi, yine dul giysilerini giydi.
\par 20 Bu arada Yahuda rehin biraktigi esyalari geri almak için Adullamli arkadasiyla kadina bir oglak gönderdi. Ne var ki arkadasi kadini bulamadi.
\par 21 O çevrede yasayanlara, "Enayim'de, yol kenarinda bir fahise vardi, nerede o?" diye sordu. "Burada öyle bir kadin yok" diye karsilik verdiler.
\par 22 Bunun üzerine Yahuda'nin yanina dönerek, "Kadini bulamadim" dedi, "O çevrede yasayanlar da 'Burada fahise yok' dediler."
\par 23 Yahuda, "Varsin esyalar onun olsun" dedi, "Kimseyi kendimize güldürmeyelim. Ben oglagi gönderdim, ama sen kadini bulamadin."
\par 24 Yaklasik üç ay sonra Yahuda'ya, "Gelinin Tamar zina etmis, su anda hamile" diye haber verdiler. Yahuda, "Onu disariya çikarip yakin" dedi.
\par 25 Tamar disari çikarilinca, kayinbabasina, "Ben bu esyalarin sahibinden hamile kaldim" diye haber gönderdi, "Lütfen sunlara bak. Bu mühür, kaytan, degnek kime ait?"
\par 26 Yahuda esyalari tanidi. "O benden daha dogru bir kisi" dedi, "Çünkü onu oglum Sela'ya almadim." Bir daha onunla yatmadi.
\par 27 Dogum vakti gelince Tamar'in rahminde ikiz oldugu anlasildi.
\par 28 Dogum yaparken ikizlerden biri elini disari çikardi. Ebe çocugun elini yakalayip bilegine kirmizi bir iplik bagladi, "Bu önce dogdu" dedi.
\par 29 Ne var ki, çocuk elini içeri çekti, o sirada da kardesi dogdu. Ebe, "Kendine böyle mi gedik açtin?" dedi. Bu yüzden çocuga Peres adi kondu.
\par 30 Sonra bilegine kirmizi iplik bagli kardesi dogdu. Ona da Zerah adi verildi.

\chapter{39}

\par 1 Ismaililer Yusuf'u Misir'a götürmüstü. Firavunun görevlisi, muhafiz birligi komutani Misirli Potifar onu Ismaililer'den satin almisti.
\par 2 RAB Yusuf'la birlikteydi ve onu basarili kiliyordu. Yusuf Misirli efendisinin evinde kaliyordu.
\par 3 Efendisi RAB'bin Yusuf'la birlikte oldugunu, yaptigi her iste onu basarili kildigini gördü.
\par 4 Yusuf'tan hosnut kalarak onu özel hizmetine aldi. Evinin ve sahip oldugu her seyin sorumlulugunu ona verdi.
\par 5 Yusuf'u evinin ve sahip oldugu her seyin sorumlusu atadigi andan itibaren RAB Yusuf sayesinde Potifar'in evini kutsadi. Evini, tarlasini, kendisine ait her seyi bereketli kildi.
\par 6 Potifar sahip oldugu her seyin sorumlulugunu Yusuf'a verdi; yedigi yemek disinda hiçbir seyle ilgilenmedi. Yusuf güzel yapili, yakisikliydi.
\par 7 Bir süre sonra efendisinin karisi ona göz koyarak, "Benimle yat" dedi.
\par 8 Ama Yusuf reddetti. "Ben burada oldugum için efendim evdeki hiçbir seyle ilgilenme geregini duymuyor" dedi, "Sahip oldugu her seyin yönetimini bana verdi.
\par 9 Bu evde ben de onun kadar yetkiliyim. Senin disinda hiçbir seyi benden esirgemedi. Sen onun karisisin. Nasil böyle bir kötülük yapar, Tanri'ya karsi günah islerim?"
\par 10 Potifar'in karisi her gün kendisiyle yatmasi ya da birlikte olmasi için direttiyse de, Yusuf onun istegini kabul etmedi.
\par 11 Bir gün Yusuf olagan islerini yapmak üzere eve gitti. Içerde ev halkindan hiç kimse yoktu.
\par 12 Potifar'in karisi Yusuf'un giysisini tutarak, "Benimle yat" dedi. Ama Yusuf giysisini onun elinde birakip evden disari kaçti.
\par 13 Kadin Yusuf'un giysisini birakip kaçtigini görünce,
\par 14 usaklarini çagirdi. "Bakin suna!" dedi, "Kocamin getirdigi bu Ibrani bizi rezil etti. Yanima geldi, benimle yatmak istedi. Ben de bagirdim.
\par 15 Bagirdigimi duyunca giysisini yanimda birakip disari kaçti."
\par 16 Efendisi eve gelinceye kadar Yusuf'un giysisini yaninda alikoydu.
\par 17 Ona da ayni seyleri anlatti: "Buraya getirdigin Ibrani köle yanima gelip beni asagilamak istedi.
\par 18 Ama ben bagirinca giysisini yanimda birakip kaçti."
\par 19 Karisinin, "Kölen bana böyle yapti" diyerek anlattiklarini duyunca, Yusuf'un efendisinin öfkesi tepesine çikti.
\par 20 Yusuf'u yakalayip zindana, kralin tutsaklarinin bagli oldugu yere atti. Ama Yusuf zindandayken
\par 21 RAB onunla birlikteydi. Ona iyilik etti. Zindancibasi Yusuf'tan hosnut kaldi.
\par 22 Bütün tutsaklarin yönetimini ona verdi. Zindanda olup biten her seyden Yusuf sorumluydu.
\par 23 Zindancibasi Yusuf'un sorumlu oldugu islerle hiç ilgilenmezdi. Çünkü RAB Yusuf'la birlikteydi ve yaptigi her iste onu basarili kiliyordu.

\chapter{40}

\par 1 Bir süre sonra Misir Krali'nin sakisiyle firincisi efendilerini gücendirdiler.
\par 2 Firavun bu iki görevlisine, bas sakiyle firincibasina öfkelendi.
\par 3 Onlari muhafiz birligi komutaninin evinde, Yusuf'un tutsak oldugu zindanda göz altina aldi.
\par 4 Muhafiz birligi komutani Yusuf'u onlarin hizmetine atadi. Bir süre zindanda kaldilar.
\par 5 Firavunun sakisiyle firincisi tutsak olduklari zindanda ayni gece birer düs gördüler. Düsleri farkli anlamlar tasiyordu.
\par 6 Sabah Yusuf yanlarina gittiginde, onlari tedirgin gördü.
\par 7 Efendisinin evinde, kendisiyle birlikte zindanda kalan firavunun görevlilerine, "Niçin suratiniz asik bugün?" diye sordu.
\par 8 "Düs gördük ama yorumlayacak kimse yok" dediler. Yusuf, "Yorum Tanri'ya özgü degil mi?" dedi, "Lütfen düsünüzü bana anlatin."
\par 9 Bas saki düsünü Yusuf'a anlatti: "Düsümde önümde bir asma gördüm.
\par 10 Üç çubugu vardi. Tomurcuklar açar açmaz çiçeklendi, salkim salkim üzüm verdi.
\par 11 Firavunun kâsesi elimdeydi. Üzümleri alip firavunun kâsesine siktim. Sonra kâseyi ona verdim."
\par 12 Yusuf, "Bu su anlama gelir" dedi, "Üç çubuk üç gün demektir.
\par 13 Üç gün içinde firavun seni zindandan çikaracak, yine eski görevine döneceksin. Geçmiste oldugu gibi yine ona sakilik yapacaksin.
\par 14 Ama her sey yolunda giderse, lütfen beni animsa. Bir iyilik yap, firavuna benden söz et. Çikar beni bu zindandan.
\par 15 Çünkü ben Ibrani ülkesinden zorla kaçirildim. Burada da zindana atilacak bir sey yapmadim."
\par 16 Firincibasi bu iyi yorumu duyunca, Yusuf'a, "Ben de bir düs gördüm" dedi, "Basimin üstünde üç sepet beyaz ekmek vardi.
\par 17 En üstteki sepette firavun için pisirilmis çesitli pastalar vardi. Kuslar basimin üstündeki sepetten pastalari yiyorlardi."
\par 18 Yusuf, "Bu su anlama gelir" dedi, "Üç sepet üç gün demektir.
\par 19 Üç gün içinde firavun seni zindandan çikarip agaca asacak. Kuslar etini yiyecekler."
\par 20 Üç gün sonra, firavun dogum gününde bütün görevlilerine bir sölen verdi. Görevlilerinin önünde bas sakisiyle firincibasini zindandan çikardi.
\par 21 Yusuf'un yaptigi yoruma uygun olarak bas sakisini eski görevine atadi. Bas saki firavuna sarap sunmaya basladi. Ama firavun firincibasini astirdi.
\par 22 (#40:21)
\par 23 Gelgelelim, bas saki Yusuf'u animsamadi, unuttu gitti.

\chapter{41}

\par 1 Tam iki yil sonra firavun bir düs gördü: Nil Irmagi'nin kiyisinda duruyordu.
\par 2 Irmaktan güzel ve semiz yedi inek çikti. Sazlar arasinda otlamaya basladilar.
\par 3 Sonra yedi çirkin ve ciliz inek çikti. Irmagin kiyisinda öbür ineklerin yaninda durdular.
\par 4 Çirkin ve ciliz inekler güzel ve semiz yedi inegi yiyince, firavun uyandi.
\par 5 Yine uykuya daldi, bu kez baska bir düs gördü: Bir sapta yedi güzel ve dolgun basak bitti.
\par 6 Sonra, ciliz ve dogu rüzgariyla kavrulmus yedi basak daha bitti.
\par 7 Ciliz basaklar, yedi güzel ve dolgun basagi yuttular. Firavun uyandi, düs gördügünü anladi.
\par 8 Sabah uyandiginda kaygiliydi. Bütün Misirli büyücüleri, bilgeleri çagirtti. Onlara gördügü düsleri anlatti. Ama hiçbiri firavunun düslerini yorumlayamadi.
\par 9 Bu arada bas saki firavuna, "Bugün suçumu itiraf etmeliyim" dedi,
\par 10 "Kullarina -bana ve firincibasina- öfkelenince bizi zindana, muhafiz birligi komutaninin evine kapattin.
\par 11 Bir gece ikimiz de düs gördük. Düslerimiz farkli anlamlar tasiyordu.
\par 12 Orada bizimle birlikte muhafiz birligi komutaninin kölesi Ibrani bir genç vardi. Gördügümüz düsleri ona anlattik. Bize bir bir yorumladi.
\par 13 Her sey onun yorumladigi gibi çikti: Ben görevime döndüm, firincibasiysa asildi."
\par 14 Firavun Yusuf'u çagirtti. Hemen onu zindandan çikardilar. Yusuf tiras olup giysilerini degistirdikten sonra firavunun huzuruna çikti.
\par 15 Firavun Yusuf'a, "Bir düs gördüm" dedi, "Ama kimse yorumlayamadi. Duydugun her düsü yorumlayabildigini isittim."
\par 16 Yusuf, "Ben yorumlayamam" dedi, "Firavuna en uygun yorumu Tanri yapacaktir."
\par 17 Firavun Yusuf'a anlatmaya basladi: "Düsümde bir irmak kiyisinda duruyordum.
\par 18 Irmaktan semiz ve güzel yedi inek çikti. Sazlar arasinda otlamaya basladilar.
\par 19 Sonra arik, çirkin, ciliz yedi inek daha çikti. Misir'da onlar kadar çirkin inek görmedim.
\par 20 Ciliz ve çirkin inekler ilk çikan yedi semiz inegi yedi.
\par 21 Ancak kötü görünüsleri degismedi. Sanki bir sey yememis gibi görünüyorlardi. Sonra uyandim.
\par 22 "Bir de düsümde bir sapta dolgun ve güzel yedi basak bittigini gördüm.
\par 23 Sonra solgun, ciliz, dogu rüzgarinin kavurdugu yedi basak daha bitti.
\par 24 Ciliz basaklar yedi güzel basagi yuttular. Büyücülere bunu anlattim. Ama hiçbiri yorumlayamadi."
\par 25 Yusuf, "Efendim, iki düs de ayni anlami tasiyor" dedi, "Tanri ne yapacagini sana bildirmis.
\par 26 Yedi güzel inek yedi yil demektir. Yedi güzel basak da yedi yildir. Ayni anlama geliyor.
\par 27 Daha sonra çikan yedi ciliz, çirkin inek ve dogu rüzgarinin kavurdugu yedi solgun basaksa yedi yil kitlik olacagi anlamina gelir.
\par 28 "Söyledigim gibi, Tanri ne yapacagini sana göstermis.
\par 29 Misir'da yedi yil bolluk olacak.
\par 30 Sonra yedi yil öyle bir kitlik olacak ki, bolluk yillari hiç animsanmayacak. Çünkü kitlik ülkeyi kasip kavuracak.
\par 31 Ardindan gelen kitlik bollugu unutturacak, çünkü çok siddetli olacak.
\par 32 Bu konuda iki kez düs görmenin anlami, Tanri'nin kesin kararini verdigini ve en kisa zamanda uygulayacagini gösteriyor.
\par 33 "Simdi firavunun akilli, bilgili bir adam bulup onu Misir'in basina getirmesi gerekir.
\par 34 Ülke çapinda adamlar görevlendirmeli, bunlar yedi bolluk yili boyunca ürünlerin beste birini toplamali.
\par 35 Gelecek verimli yillarin bütün yiyecegini toplasinlar, firavunun yönetimi altinda kentlerde depolayip korusunlar.
\par 36 Bu yiyecek, gelecek yedi kitlik yili boyunca Misir'da ihtiyat olarak kullanilacak, ülke kitliktan kirilmayacak."
\par 37 Bu öneri firavunla görevlilerine iyi göründü.
\par 38 Firavun görevlilerine, "Bu adam gibi Tanri Ruhu'na sahip birini bulabilir miyiz?" diye sordu.
\par 39 Sonra Yusuf'a, "Madem Tanri bütün bunlari sana açikladi, senden daha akillisi, bilgilisi yoktur" dedi,
\par 40 "Sarayimin yönetimini sana verecegim. Bütün halkim buyruklarina uyacak. Tahttan baska senden üstünlügüm olmayacak.
\par 41 Seni bütün Misir'a yönetici atiyorum."
\par 42 Sonra mührünü parmagindan çikarip Yusuf'un parmagina takti. Ona ince ketenden giysi giydirdi. Boynuna altin zincir takti.
\par 43 Onu kendi yardimcisinin arabasina bindirdi. Yusuf'un önünde, "Yol açin!" diye bagirdilar. Böylece firavun ona bütün Misir'in yönetimini verdi.
\par 44 Firavun Yusuf'a, "Firavun benim" dedi, "Ama Misir'da senden izinsiz kimse elini ayagini oynatmayacak."
\par 45 Yusuf'un adini Safenat-Paneah koydu. On Kenti'nin kâhini Potifera'nin kizi Asenat'i da ona kari olarak verdi. Yusuf ülkeyi boydan boya dolasti.
\par 46 Yusuf firavunun hizmetine girdiginde otuz yasindaydi. Firavunun huzurundan ayrildiktan sonra bütün Misir'i dolasti.
\par 47 Yedi bolluk yili boyunca toprak çok ürün verdi.
\par 48 Yusuf Misir'da yedi yil içinde yetisen bütün ürünleri toplayip kentlerde depoladi. Her kente o kentin çevresindeki tarlalarda yetisen ürünleri koydu.
\par 49 Denizin kumu kadar çok bugday depoladi; öyle ki, ölçmekten vazgeçti. Çünkü bugday ölçülemeyecek kadar çoktu.
\par 50 Kitlik yillari baslamadan, On Kenti'nin kâhini Potifera'nin kizi Asenat Yusuf'a iki erkek çocuk dogurdu.
\par 51 Yusuf ilk oglunun adini Manasse koydu. "Tanri bana bütün acilarimi ve babamin ailesini unutturdu" dedi.
\par 52 "Tanri sikinti çektigim ülkede beni verimli kildi" diyerek ikinci oglunun adini Efrayim koydu.
\par 53 Misir'da yedi bolluk yili sona erdi.
\par 54 Yusuf'un söylemis oldugu gibi yedi kitlik yili basgösterdi. Bütün ülkelerde kitlik vardi, ama Misir'in her yaninda yiyecek bulunuyordu.
\par 55 Misirlilar aç kalinca, yiyecek için firavuna yakardilar. Firavun, "Yusuf'a gidin" dedi, "O size ne derse öyle yapin."
\par 56 Kitlik bütün ülkeyi sarinca, Yusuf depolari açip Misirlilar'a bugday satmaya basladi. Çünkü kitlik Misir'i boydan boya kavuruyordu.
\par 57 Bütün ülkelerden insanlar da bugday satin almak için Misir'a, Yusuf'a geliyordu. Çünkü kitlik bütün dünyayi sarmisti ve siddetliydi.

\chapter{42}

\par 1 Yakup Misir'da bugday oldugunu ögrenince, ogullarina, "Neden birbirinize bakip duruyorsunuz?" dedi,
\par 2 "Misir'da bugday oldugunu duydum. Gidin, satin alin ki, yasayalim, yoksa ölecegiz."
\par 3 Böylece Yusuf'un on kardesi bugday almak için Misir'a gittiler.
\par 4 Ancak Yakup Yusuf'un kardesi Benyamin'i onlarla birlikte göndermedi, çünkü oglunun basina bir sey gelmesinden korkuyordu.
\par 5 Bugday satin almaya gelenler arasinda Israil'in ogullari da vardi. Çünkü Kenan ülkesinde de kitlik hüküm sürüyordu.
\par 6 Yusuf ülkenin yöneticisiydi, herkese o bugday satiyordu. Kardesleri gelip onun önünde yere kapandilar.
\par 7 Yusuf kardeslerini görünce tanidi. Ama onlara yabanci gibi davranarak sert konustu: "Nereden geliyorsunuz?" "Kenan ülkesinden" diye yanitladilar, "Yiyecek satin almaya geldik."
\par 8 Yusuf kardeslerini tanidiysa da kardesleri onu tanimadilar.
\par 9 Yusuf onlarla ilgili düslerini animsayarak, "Siz casussunuz" dedi, "Ülkenin zayif noktalarini ögrenmeye geldiniz."
\par 10 "Aman, efendim" diye karsilik verdiler, "Biz kullarin yalnizca yiyecek satin almaya geldik.
\par 11 Hepimiz ayni babanin çocuklariyiz. Biz kullarin dürüst insanlariz, casus degiliz."
\par 12 Yusuf, "Hayir!" dedi, "Siz ülkenin zayif noktalarini ögrenmeye geldiniz."
\par 13 Kardesleri, "Biz kullarin on iki kardesiz" dediler, "Hepimiz Kenan ülkesinde yasayan ayni babanin çocuklariyiz. En küçügümüz babamizin yaninda kaldi, biri de kayboldu."
\par 14 Yusuf, "Söyledigim gibi" dedi, "Casussunuz siz.
\par 15 Sizi sinayacagim. Firavunun basina ant içerim. Küçük kardesiniz de gelmedikçe, buradan ayrilamazsiniz.
\par 16 Aranizdan birini gönderin, kardesinizi getirsin. Geri kalanlariniz göz altina alinacak. Anlattiklariniz dogru mu, degil mi, sizi sinayacagiz. Degilse, firavunun basina ant içerim ki casussunuz."
\par 17 Üç gün onlari göz altinda tuttu.
\par 18 Üçüncü gün, "Bir kosulla caninizi bagislarim" dedi, "Ben Tanri'dan korkarim.
\par 19 Dürüst oldugunuzu kanitlamak için, içinizden biri göz altinda tutuldugunuz evde kalsin, ötekiler gidip aç kalan ailenize bugday götürsün.
\par 20 Sonra küçük kardesinizi bana getirin. Böylece anlattiklarinizin dogru olup olmadigi ortaya çikar, ölümden kurtulursunuz." Kabul ettiler.
\par 21 Birbirlerine, "Besbelli kardesimize yaptigimizin cezasini çekiyoruz" dediler, "Bize yalvardiginda nasil sikinti çektigini gördük, ama dinlemedik. Bu sikinti onun için basimiza geldi."
\par 22 Ruben, "Çocuga zarar vermeyin diye sizi uyarmadim mi?" dedi, "Ama dinlemediniz. Iste simdi kaninin hesabi soruluyor."
\par 23 Yusuf'un konustuklarini anladigini farketmediler, çünkü onunla çevirmen araciligiyla konusuyorlardi.
\par 24 Yusuf kardeslerinden ayrilip aglamaya basladi. Sonra dönüp onlarla konustu. Aralarindan Simon'u alarak ötekilerin gözleri önünde bagladi.
\par 25 Sonra torbalarina bugday doldurulmasini, paralarinin torbalarina geri konulmasini, yol için kendilerine azik verilmesini buyurdu. Bunlar yapildiktan sonra
\par 26 bugdaylari eseklerine yükleyip oradan ayrildilar.
\par 27 Konakladiklari yerde içlerinden biri esegine yem vermek için torbasini açinca parasini gördü. Para torbanin agzina konmustu.
\par 28 Kardeslerine, "Parami geri vermisler" diye seslendi, "Iste torbamda!" Yürekleri yerinden oynadi. Titreyerek birbirlerine, "Tanri'nin bize bu yaptigi nedir?" dediler.
\par 29 Kenan ülkesine, babalari Yakup'un yanina varinca, baslarina gelenleri ona anlattilar:
\par 30 "Misir'in yöneticisi bizimle sert konustu. Bize casusmusuz gibi davrandi.
\par 31 Ona, 'Biz dürüst insanlariz' dedik, 'Casus degiliz.
\par 32 Hepimiz ayni babanin çocuklariyiz. On iki kardesiz; biri kayboldu, en küçügü de Kenan ülkesinde, babamizin yaninda.'
\par 33 "Ülkenin yöneticisi, 'Dürüst oldugunuzu söyle anlayabilirim' dedi, 'Kardeslerinizden birini yanimda birakin, bugdayi alip aç kalan ailelerinize götürün.
\par 34 Küçük kardesinizi de bana getirin. O zaman casus olmadiginizi, dürüst insanlar oldugunuzu anlar, kardesinizi size geri veririm. Ülkede ticaret yapabilirsiniz.'"
\par 35 Torbalarini bosaltinca, hepsi para kesesini torbasinda buldu. Para keselerini görünce hem kendileri hem babalari korkuya kapildi.
\par 36 Yakup, "Beni çocuklarimdan yoksun birakiyorsunuz" dedi, "Yusuf yok, Simon yok. Simdi de Benyamin'i götürmek istiyorsunuz. Sikintiyi çeken hep benim."
\par 37 Ruben babasina, "Benyamin'i geri getirmezsem, iki oglumu öldür" dedi, "Onu bana teslim et, ben sana geri getirecegim."
\par 38 Ama Yakup, "Oglumu sizinle göndermeyecegim" dedi, "Çünkü kardesi öldü, yalniz o kaldi. Yolda ona bir zarar gelirse, bu aciyla ak saçli basimi ölüler diyarina götürürsünüz."

\chapter{43}

\par 1 Kenan ülkesinde kitlik siddetlenmisti.
\par 2 Misir'dan getirilen bugday tükenince Yakup, ogullarina, "Yine gidin, bize biraz yiyecek alin" dedi.
\par 3 Yahuda, "Adam bizi siki siki uyardi" diye karsilik verdi, "'Kardesiniz sizinle birlikte gelmezse, yüzümü göremezsiniz' dedi.
\par 4 Kardesimizi bizimle gönderirsen, gider sana yiyecek aliriz.
\par 5 Göndermezsen gitmeyiz. Çünkü o adam, 'Kardesinizi birlikte getirmezseniz, yüzümü göremezsiniz' dedi."
\par 6 Israil, "Niçin adama bir kardesiniz daha oldugunu söyleyerek bana bu kötülügü yaptiniz?" dedi.
\par 7 Söyle yanitladilar: "Adam, 'Babaniz hâlâ yasiyor mu? Baska kardesiniz var mi?' diye sordu. Bizimle ve akrabalarimizla ilgili öyle sorular sordu ki, yanit vermek zorunda kaldik. Kardesinizi getirin diyecegini nereden bilebilirdik?"
\par 8 Yahuda, babasi Israil'e, "Çocugu benimle gönder, gidelim" dedi, "Sen de biz de yavrularimiz da ölmez, yasariz.
\par 9 Ona ben kefil oluyorum. Beni sorumlu say. Eger onu geri getirmez, önüne çikarmazsam, ömrümce sana karsi suçlu sayilayim.
\par 10 Çünkü gecikmeseydik, simdiye dek iki kez gidip gelmis olurduk."
\par 11 Bunun üzerine Israil, "Öyleyse gidin" dedi, "Yalniz, torbalariniza bu ülkenin en iyi ürünlerinden biraz pelesenk, biraz bal, kitre, laden, fistik, badem koyun, Misir'in yöneticisine armagan olarak götürün.
\par 12 Yaniniza iki kat para alin. Torbalarinizin agzina konan parayi geri götürün. Belki bir yanlislik olmustur.
\par 13 Kardesinizi alip gidin, o adamin yanina dönün.
\par 14 Her Seye Gücü Yeten Tanri, adamin yüregine size karsi merhamet koysun da, adam öbür kardesinizle Benyamin'i size geri versin. Bana gelince, çocuklarimdan yoksun kalacaksam kalayim."
\par 15 Böylece kardesler yanlarina armaganlar, iki kat para ve Benyamin'i alarak hemen Misir'a gidip Yusuf'un huzuruna çiktilar.
\par 16 Yusuf Benyamin'i yanlarinda görünce, kâhyasina, "Bu adamlari eve götür" dedi, "Bir hayvan kesip hazirla. Çünkü öglen benimle birlikte yemek yiyecekler."
\par 17 Kâhya Yusuf'un buyurdugu gibi onlari Yusuf'un evine götürdü.
\par 18 Ne var ki kardesleri Yusuf'un evine götürüldükleri için korktular. "Ilk gelisimizde torbalarimiza konan para yüzünden götürülüyoruz galiba!" dediler, "Bize saldirip egemen olmak, bizi köle edip eseklerimizi almak istiyor."
\par 19 Yusuf'un kâhyasina yaklasip evin kapisinda onunla konustular:
\par 20 "Aman, efendim!" dediler, "Buraya ilk kez yiyecek satin almaya gelmistik.
\par 21 Konakladigimiz yerde torbalarimizi açinca, bir de baktik ki, paramiz eksiksiz olarak torbalarimizin agzina konmus. Onu size geri getirdik.
\par 22 Ayrica yeniden yiyecek almak için yanimiza baska para da aldik. Paralari torbalarimiza kimin koydugunu bilmiyoruz."
\par 23 Kâhya, "Merak etmeyin" dedi, "Korkmaniza gerek yok. Parayi Tanriniz, babanizin Tanrisi torbalariniza koydurmus. Ben paranizi aldim." Sonra Simon'u onlara getirdi.
\par 24 Kâhya onlari Yusuf'un evine götürüp ayaklarini yikamalari için su getirdi, eseklerine yem verdi.
\par 25 Kardesler öglene, Yusuf'un gelecegi saate kadar armaganlarini hazirladilar. Çünkü orada yemek yiyeceklerini duymuslardi.
\par 26 Yusuf eve gelince, getirdikleri armaganlari kendisine sunup önünde yere kapandilar.
\par 27 Yusuf hatirlarini sorduktan sonra, "Bana sözünü ettiginiz yasli babaniz iyi mi?" dedi, "Hâlâ yasiyor mu?"
\par 28 Kardesleri, "Babamiz kulun iyi" diye yanitladilar, "Hâlâ yasiyor." Sonra saygiyla egilip yere kapandilar.
\par 29 Yusuf göz gezdirirken kendisiyle ayni anneden olan kardesi Benyamin'i gördü. "Bana sözünü ettiginiz küçük kardesiniz bu mu?" dedi, "Tanri sana lütfetsin, oglum."
\par 30 Sonra hemen oradan ayrildi, çünkü kardesini görünce yüregi sizlamisti. Aglayacak bir yer aradi. Odasina girip orada agladi.
\par 31 Yüzünü yikadiktan sonra disari çikti. Kendisini toparlayarak, "Yemegi getirin" dedi.
\par 32 Yusuf'a ayri, kardeslerine ayri, Yusuf'la yemek yiyen Misirlilar'a ayri hizmet edildi. Çünkü Misirlilar Ibraniler'le birlikte yemek yemez, bunu igrenç sayarlardi.
\par 33 Kardesleri Yusuf'un önünde büyükten küçüge dogru yas sirasina göre oturdular. Saskin saskin birbirlerine baktilar.
\par 34 Yusuf'un masasindan onlara yemek dagitildi. Benyamin'in payi ötekilerden bes kat fazlaydi. Içtiler, birlikte hos vakit geçirdiler.

\chapter{44}

\par 1 Yusuf kâhyasina, "Bu adamlarin torbalarina tasiyabilecekleri kadar yiyecek doldur" diye buyurdu, "Her birinin parasini torbasinin agzina koy.
\par 2 En küçügünün torbasina benim gümüs kâsemi ve bugdayinin parasini koy." Kâhya Yusuf'un buyrugunu yerine getirdi.
\par 3 Sabah erkenden adamlar esekleriyle yolcu edildi.
\par 4 Onlar kentten pek uzaklasmamisti ki Yusuf kâhyasina, "Hemen o adamlarin pesine düs" dedi, "Onlara yetisince, 'Niçin iyilige karsi kötülük yaptiniz?' de,
\par 5 'Efendimin sarap içmek, fala bakmak için kullandigi kâse degil mi bu? Bunu yapmakla kötülük ettiniz.'"
\par 6 Kâhya onlara yetisip bu sözleri yineledi.
\par 7 Adamlar, "Efendim, neden böyle konusuyorsun?" dediler, "Bizden uzak olsun, biz kullarin böyle sey yapmayiz.
\par 8 Torbalarimizin agzinda buldugumuz paralari Kenan ülkesinden sana geri getirdik. Nasil efendinin evinden altin ya da gümüs çalariz?
\par 9 Kullarindan birinde çikarsa öldürülsün, geri kalanlar efendimin kölesi olsun."
\par 10 Kâhya, "Peki, dediginiz gibi olsun" dedi, "Kimde çikarsa kölem olacak, geri kalanlar suçsuz sayilacak."
\par 11 Hemen torbalarini indirip açtilar.
\par 12 Kâhya büyükten küçüge dogru hepsinin torbasini aradi. Kâse Benyamin'in torbasinda çikti.
\par 13 Kardesleri üzüntüden giysilerini yirttilar. Sonra torbalarini eseklerine yükleyip kente geri döndüler.
\par 14 Yahuda'yla kardesleri Yusuf'un evine geldiginde, Yusuf daha evdeydi. Önünde yere kapandilar.
\par 15 Yusuf, "Nedir bu yaptiginiz?" dedi, "Benim gibi birinin fala bakabilecegi akliniza gelmedi mi?"
\par 16 Yahuda, "Ne diyelim, efendim?" diye karsilik verdi, "Nasil anlatalim? Kendimizi nasil temize çikaralim? Tanri suçumuzu ortaya çikardi. Hepimiz köleniz artik, efendim; hem biz hem de kendisinde kâse bulunan kardesimiz."
\par 17 Yusuf, "Benden uzak olsun!" dedi, "Yalniz kendisinde kâse bulunan kölem olacak. Siz esenlikle babanizin yanina dönün."
\par 18 Yahuda yaklasip, "Efendim, lütfen izin ver konusayim" dedi, "Kuluna öfkelenme. Sen firavunla ayni yetkiye sahipsin.
\par 19 Efendim, biz kullarina sormustun: 'Babaniz ya da baska kardesiniz var mi?' diye.
\par 20 Biz de, 'Yasli bir babamiz ve onun yasliliginda dogan küçük bir kardesimiz var' demistik, 'O çocugun kardesi öldü, kendisi annesinin tek oglu. Babamiz onu çok sever.'
\par 21 "Sen de biz kullarina, 'O çocugu bana getirin, gözümle göreyim' demistin.
\par 22 Biz de, 'Çocuk babasindan ayrilamaz, ayrilirsa babasi ölür' diye karsilik vermistik.
\par 23 Sen de biz kullarina, 'Eger küçük kardesiniz sizinle gelmezse, yüzümü bir daha göremezsiniz' demistin.
\par 24 "Kulun babamizin yanina döndügümüzde, söylediklerini ona anlattik.
\par 25 Babamiz, 'Yine gidin, bize biraz yiyecek alin' dedi.
\par 26 Ama biz, 'Gidemeyiz' dedik, 'Ancak küçük kardesimiz bizimle gelirse gideriz. Küçük kardesimiz bizimle olmazsa o adamin yüzünü göremeyiz.'
\par 27 "Babam, biz kullarina, 'Biliyorsunuz, karim bana iki erkek çocuk dogurdu' dedi,
\par 28 'Biri yanimdan ayrildi. Besbelli bir hayvan parçaladi, bir daha göremedim onu.
\par 29 Bunu da götürürseniz ve ona bir zarar gelirse, bu aciyla ak saçli basimi ölüler diyarina götürürsünüz.'
\par 30 "Efendim, simdi babam kulunun yanina döndügümde çocuk yanimizda olmazsa, babam onu görmeyince ölür. Çünkü onu yasama baglayan bu çocuktur. Biz kullarin da aci içinde babamizin ak saçli basini ölüler diyarina indiririz.
\par 31 (#44:30)
\par 32 Ben kulun bu çocuga kefil oldum. Babama, 'Onu sana geri getirmezsem, ömrümce kendimi sana karsi suçlu sayarim' dedim.
\par 33 "Lütfen simdi çocugun yerine beni kölen kabul et. Çocuk kardesleriyle birlikte geri dönsün.
\par 34 O yanimda olmadan babamin yanina nasil dönerim? Babamin basina gelecek kötülüge dayanamam."

\chapter{45}

\par 1 Yusuf adamlarinin önünde kendini tutamayip, "Herkesi çikarin buradan!" diye bagirdi. Kendini kardeslerine tanittiginda yaninda kimse olmasin istiyordu.
\par 2 O kadar yüksek sesle agladi ki, Misirlilar aglayisini isitti. Bu haber firavunun ev halkina da ulasti.
\par 3 Yusuf kardeslerine, "Ben Yusuf'um!" dedi, "Babam yasiyor mu?" Kardesleri donup kaldi, yanit veremediler.
\par 4 Yusuf, "Lütfen bana yaklasin" dedi. Onlar yaklasinca Yusuf söyle devam etti: "Misir'a sattiginiz kardesiniz Yusuf benim.
\par 5 Beni buraya sattiginiz için üzülmeyin. Kendinizi suçlamayin. Tanri insanligi korumak için beni önden gönderdi.
\par 6 Çünkü iki yildir ülkede kitlik var, bes yil daha sürecek. Kimse çift süremeyecek, ekin biçemeyecek.
\par 7 Tanri yeryüzünde soyunuzu korumak ve harika biçimde caninizi kurtarmak için beni önünüzden gönderdi.
\par 8 Beni buraya gönderen siz degilsiniz, Tanri'dir. Beni firavunun basdanismani, sarayinin efendisi, bütün Misir ülkesinin yöneticisi yapti.
\par 9 Hemen babamin yanina gidin, ona oglun Yusuf söyle diyor deyin: 'Tanri beni Misir ülkesine yönetici yapti. Durma, yanima gel.
\par 10 Gosen bölgesine yerlesirsin; çocuklarin, torunlarin, davarlarin, sigirlarin ve sahip oldugun her seyle birlikte yakinimda olursun.
\par 11 Orada sana bakarim, çünkü kitlik bes yil daha sürecek. Yoksa sen de ailen ve sana bagli olan herkes de perisan olursunuz.'
\par 12 "Hepiniz gözlerinizle görüyorsunuz, kardesim Benyamin, sen de görüyorsun konusanin gerçekten ben oldugumu.
\par 13 Misir'da ne denli güçlü oldugumu ve bütün gördüklerinizi babama anlatin. Babami hemen buraya getirin."
\par 14 Sonra kardesi Benyamin'in boynuna sarilip agladi. Benyamin de aglayarak ona sarildi.
\par 15 Yusuf aglayarak bütün kardeslerini öptü. Sonra kardesleri onunla konusmaya basladi.
\par 16 Yusuf'un kardeslerinin geldigi haberi firavunun sarayina ulasinca, firavunla görevlileri hosnut oldu.
\par 17 Firavun Yusuf'a söyle dedi: "Kardeslerine de ki, 'Hayvanlarinizi yükleyip Kenan ülkesine gidin.
\par 18 Babanizi ve ailelerinizi buraya getirin. Size Misir'in en iyi topraklarini verecegim. Ülkenin kaymagini yiyeceksiniz.'
\par 19 Onlara ayrica söyle demeni de buyuruyorum: 'Çocuklarinizla karilariniz için Misir'dan arabalar alin, babanizla birlikte buraya gelin.
\par 20 Gözünüz arkada kalmasin, çünkü Misir'da en iyi ne varsa sizin olacak.'"
\par 21 Israil'in ogullari söyleneni yapti. Firavunun buyrugu üzerine Yusuf onlara araba ve yol için azik verdi.
\par 22 Hepsine birer kat yedek giysi, Benyamin'e ise üç yüz parça gümüsle bes kat yedek giysi verdi.
\par 23 Böylece babasina Misir'da en iyi ne varsa hepsiyle yüklü on esek, yolculuk için bugday, ekmek ve azik yüklü on disi esek gönderdi.
\par 24 Kardeslerini yolcu ederken onlara, "Yolda kavga etmeyin" dedi.
\par 25 Yusuf'un kardesleri Misir'dan ayrilip Kenan ülkesine, babalari Yakup'un yanina döndüler.
\par 26 Ona, "Yusuf yasiyor!" dediler, "Üstelik Misir'in yöneticisi olmus." Babalari donup kaldi, onlara inanmadi.
\par 27 Yusuf'un kendilerine bütün söylediklerini anlattilar. Kendisini Misir'a götürmek için Yusuf'un gönderdigi arabalari görünce, Yakup'un keyfi yerine geldi.
\par 28 "Tamam!" dedi, "Oglum Yusuf yasiyor. Ölmeden önce gidip onu görecegim."

\chapter{46}

\par 1 Israil sahip oldugu her seyle birlikte yola çikti. Beer-Seva'ya varinca, orada babasi Ishak'in Tanrisi'na kurbanlar kesti.
\par 2 O gece Tanri bir görümde Israil'e, "Yakup, Yakup!" diye Seslendi. Yakup, "Buradayim" diye yanitladi.
\par 3 Tanri, "Ben Tanri'yim, babanin Tanrisi" dedi, "Misir'a gitmekten çekinme. Soyunu orada büyük bir ulus yapacagim.
\par 4 Seninle birlikte Misir'a gelecek, soyunu bu ülkeye geri getirecegim. Senin gözlerini Yusuf'un elleri kapayacak."
\par 5 Yakup Beer-Seva'dan ayrildi. Ogullari Yakup'u -Israil'i- götürmek üzere firavunun gönderdigi arabalara onu, kendi çocuklariyla karilarini bindirdiler.
\par 6 Yakup, bütün ailesini -ogullarini, kizlarini, torunlarini- hayvanlarini ve Kenan ülkesinde kazandigi mallari yanina alarak Misir'a gitti.
\par 7 (#46:6)
\par 8 Israil'in Misir'a giden ogullarinin -Yakup'la ogullarinin- adlari sunlardir: Yakup'un ilk oglu Ruben.
\par 9 9 Ruben'in ogullari: Hanok, Pallu, Hesron, Karmi.
\par 10 Simon'un ogullari: Yemuel, Yamin, Ohat, Yakin, Sohar ve Kenanli bir kadinin oglu Saul.
\par 11 Levi'nin ogullari: Gerson, Kehat, Merari.
\par 12 Yahuda'nin ogullari: Er, Onan, Sela, Peres, Zerah. Ancak Er'le Onan Kenan ülkesinde ölmüstü. Peres'in ogullari: Hesron, Hamul.
\par 13 Issakar'in ogullari: Tola, Puvva, Yov, Simron.
\par 14 Zevulun'un ogullari: Seret, Elon, Yahleel.
\par 15 Bunlar Lea'nin Yakup'a dogurdugu ogullardir. Lea onlari ve kizi Dina'yi Paddan-Aram'da dogurmustu. Yakup'un bu ogullariyla kizlari toplam otuz üç kisiydi.
\par 16 Gad'in ogullari: Sifyon, Hagi, Suni, Esbon, Eri, Arodi, Areli.
\par 17 Aser'in çocuklari: Yimna, Yisva, Yisvi, Beria; kizkardesleri Serah. Beria'nin ogullari: Hever, Malkiel.
\par 18 Bunlar Lavan'in kizi Lea'ya verdigi Zilpa'nin Yakup'a dogurdugu çocuklardir. Toplam on alti kisiydiler.
\par 19 Yakup'un karisi Rahel'in ogullari: Yusuf, Benyamin.
\par 20 Yusuf'un Misir'da On Kenti kâhini Potifera'nin kizi Asenat'tan Manasse ve Efrayim adinda iki oglu oldu.
\par 21 Benyamin'in ogullari: Bala, Beker, Asbel, Gera, Naaman, Ehi, Ros, Muppim, Huppim, Ard.
\par 22 Bunlar Rahel'in Yakup'a dogurdugu çocuklardir. Toplam on dört kisiydiler.
\par 23 Dan'in oglu: Husim.
\par 24 Naftali'nin ogullari: Yahseel, Guni, Yeser, Sillem.
\par 25 Bunlar Lavan'in, kizi Rahel'e verdigi Bilha'nin Yakup'a dogurdugu çocuklardir. Toplam yedi kisiydiler.
\par 26 Ogullarinin karilari disinda Yakup'un soyundan gelen ve onunla birlikte Misir'a gidenler toplam altmis alti kisiydi. Bunlarin hepsi Yakup'tan olmustu.
\par 27 Yusuf'un Misir'da dogan iki ogluyla birlikte Misir'a göçen Yakup ailesi toplam yetmis kisiydi.
\par 28 Yakup Gosen yolunu göstermesi için Yahuda'yi önden Yusuf'a gönderdi. Onlar Gosen'e varinca,
\par 29 Yusuf arabasini hazirlayip babasi Israil'i karsilamak üzere Gosen'e gitti. Babasini görür görmez boynuna sarilip uzun uzun agladi.
\par 30 Israil Yusuf'a, "Yüzünü gördüm ya, artik ölsem de gam yemem" dedi, "Yasiyorsun!"
\par 31 Yusuf kardesleriyle babasinin ev halkina söyle dedi: "Gidip firavuna haber vereyim, 'Kenan ülkesinde yasayan kardeslerimle babamin ev halki yanima geldi' diyeyim.
\par 32 Çoban oldugunuzu, hayvancilik yaptiginizi, bu yüzden davarlarinizla sigirlarinizi ve her seyinizi birlikte getirdiginizi anlatayim.
\par 33 Firavun sizi çagirip da, 'Ne is yaparsiniz?' diye sorarsa,
\par 34 'Atalarimiz gibi biz de çocukluktan beri hayvancilik yapiyoruz' dersiniz. Öyle deyin ki, sizi Gosen bölgesine yerlestirsin. Çünkü Misirlilar çobanlardan igrenir."

\chapter{47}

\par 1 Yusuf gidip firavuna, "Babamla kardeslerim davarlari, sigirlari ve bütün esyalariyla Kenan ülkesinden geldiler" diye haber verdi, "Su anda Gosen bölgesindeler."
\par 2 Sonra kardeslerinden besini seçerek firavunun huzuruna çikardi.
\par 3 Firavun Yusuf'un kardeslerine, "Ne is yapiyorsunuz?" diye sordu. "Biz kullarin atalarimiz gibi çobaniz" diye yanitladilar,
\par 4 "Bu ülkeye geçici bir süre için geldik. Çünkü Kenan ülkesinde siddetli kitlik var. Davarlarimiz için otlak bulamiyoruz. Izin ver, Gosen bölgesine yerleselim."
\par 5 Firavun Yusuf'a, "Babanla kardeslerin yanina geldiler" dedi,
\par 6 "Misir ülkesi senin sayilir. Onlari ülkenin en iyi yerine yerlestir. Gosen bölgesine yerlessinler. Sence aralarinda becerikli olanlar varsa, davarlarima bakmakla görevlendir."
\par 7 Yusuf babasi Yakup'u getirip firavunun huzuruna çikardi. Yakup firavunu kutsadi.
\par 8 Firavun, Yakup'a, "Kaç yasindasin?" diye sordu.
\par 9 Yakup, "Gurbet yillarim yüz otuz yili buldu" diye yanitladi, "Ama yillar çabuk ve zorlu geçti. Atalarimin gurbet yillari kadar uzun sürmedi."
\par 10 Sonra firavunu kutsayip huzurundan ayrildi.
\par 11 Yusuf babasiyla kardeslerini Misir'a yerlestirdi; firavunun buyrugu uyarinca onlara ülkenin en iyi yerinde, Ramses bölgesinde mülk verdi.
\par 12 Ayrica babasiyla kardeslerine ve babasinin ev halkina, sahip olduklari çocuklarin sayisina göre yiyecek sagladi.
\par 13 Kitlik öyle siddetlendi ki, hiçbir ülkede yiyecek bulunmaz oldu. Misir ve Kenan ülkeleri kitliktan kiriliyordu.
\par 14 Yusuf sattigi bugdaya karsilik Misir ve Kenan'daki bütün paralari toplayip firavunun sarayina götürdü.
\par 15 Misir ve Kenan'da para tükenince Misirlilar Yusuf'a giderek, "Bize yiyecek ver" dediler, "Gözünün önünde ölelim mi? Paramiz bitti."
\par 16 Yusuf, "Paraniz bittiyse, davarlarinizi getirin" dedi, "Onlara karsilik size yiyecek vereyim."
\par 17 Böylece davarlarini Yusuf'a getirdiler. Yusuf atlara, davar ve sigir sürülerine, eseklere karsilik onlara yiyecek verdi. Bir yil boyunca hayvanlarina karsilik onlara yiyecek sagladi.
\par 18 O yil geçince, ikinci yil yine geldiler. Yusuf'a, "Efendim, gerçegi senden saklayacak degiliz" dediler, "Paramiz tükendi, davarlarimizi da sana verdik. Canimizdan ve topragimizdan baska verecek bir seyimiz kalmadi.
\par 19 Gözünün önünde ölelim mi? Topragimiz çöle mi dönsün? Canimiza ve topragimiza karsilik bize yiyecek sat. Topragimizla birlikte firavunun kölesi olalim. Bize tohum ver ki ölmeyelim, yasayalim; toprak da çöle dönmesin."
\par 20 Böylece Yusuf Misir'daki bütün topraklari firavun için satin aldi. Misirlilar'in hepsi tarlalarini sattilar, çünkü kitlik onlari buna zorluyordu. Topraklarin tümü firavunun oldu.
\par 21 Yusuf Misir'in bir ucundan öbür ucuna kadar bütün halki kölelestirdi.
\par 22 Yalniz kâhinlerin topragini satin almadi. Çünkü onlar firavundan aylik aliyor, firavunun bagladigi aylikla geçiniyorlardi. Bu yüzden topraklarini satmadilar.
\par 23 Yusuf halka, "Sizi de topraginizi da firavun için satin aldim" dedi, "Iste size tohum, topragi ekin.
\par 24 Ürün devsirdiginizde, beste birini firavuna vereceksiniz. Beste dördünü ise tohumluk olarak kullanacak ve ailelerinizle, çocuklarinizla yiyeceksiniz."
\par 25 "Canimizi kurtardin" diye karsilik verdiler, "Efendimizin gözünde lütuf bulalim. Firavunun kölesi oluruz."
\par 26 Yusuf ürünün beste birinin firavuna verilmesini Misir'da toprak yasasi yapti. Bu yasa bugün de yürürlüktedir. Yalniz kâhinlerin topragi firavuna verilmedi.
\par 27 Israil Misir'da Gosen bölgesine yerlesti. Orada mülk sahibi oldular, çogalip arttilar.
\par 28 Yakup Misir'da on yedi yil yasadi. Ömrü toplam yüz kirk yedi yil sürdü.
\par 29 Ölümü yaklasinca, oglu Yusuf'u çagirip, "Eger benden hosnut kaldinsa, lütfen elini uylugumun altina koy" dedi, "Bana sevgi ve sadakat gösterecegine söz ver. Lütfen beni Misir'da gömme.
\par 30 Atalarima kavustugum zaman beni Misir'dan çikarip onlarin yanina göm." Yusuf, "Dedigin gibi yapacagim" diye karsilik verdi.
\par 31 Israil, "Ant iç" dedi. Yusuf ant içti. Israil yataginin basi ucunda egilip RAB'be tapindi.

\chapter{48}

\par 1 Bir süre sonra, "Baban hasta" diye Yusuf'a haber geldi. Yusuf iki oglu Manasse'yle Efrayim'i yanina alip yola çikti.
\par 2 Yakup'a, "Oglun Yusuf geliyor" diye haber verdiler. Israil kendini toparlayip yataginda oturdu.
\par 3 Yusuf'a, "Her Seye Gücü Yeten Tanri Kenan ülkesinde, Luz'da bana görünerek beni kutsadi" dedi,
\par 4 "Bana, 'Seni verimli kilacak, çogaltacagim' dedi, 'Soyundan birçok ulus doguracagim. Senden sonra bu ülkeyi sonsuza dek mülk olarak senin soyuna verecegim.'
\par 5 "Ben Misir'a gelmeden önce burada dogan iki oglun benim sayilir. Efrayim'le Manasse benim için Ruben'le Simon gibidir.
\par 6 Onlardan sonra dogacak çocuklar senin olsun. Efrayim'le Manasse'den onlara miras geçecek.
\par 7 Ben Paddan'dan dönerken Rahel Kenan ülkesinde, Efrat'a varmadan yolda yanimda öldü. Çok üzüldüm, onu orada Efrat'a -Beytlehem'e- giden yolun kenarina gömdüm."
\par 8 Israil, Yusuf'un ogullarini görünce, "Bunlar kim?" diye sordu.
\par 9 Yusuf, "Ogullarim" diye yanitladi, "Tanri onlari bana Misir'da verdi." Israil, "Lütfen onlari yanima getir, kutsayayim" dedi.
\par 10 Israil'in gözleri yasliliktan zayiflamisti, göremiyordu. Yusuf ogullarini onun yanina götürdü. Babasi onlari öpüp kucakladi.
\par 11 Sonra Yusuf'a, "Senin yüzünü görecegimi hiç sanmiyordum" dedi, "Ama iste Tanri bana soyunu bile gösterdi."
\par 12 Yusuf ogullarini babasinin kucagindan alip onun önünde yere kapandi.
\par 13 Sonra Efrayim'i sagina alarak Israil'in sol eline, Manasse'yi soluna alarak Israil'in sag eline yaklastirdi.
\par 14 Israil ellerini çapraz olarak uzatti, sag elini küçük olan Efrayim'in, sol elini Manasse'nin basina koydu. Oysa ilkin Manasse dogmustu.
\par 15 Sonra Yusuf'u kutsayarak söyle dedi: "Atalarim Ibrahim'in, Ishak'in hizmet ettigi, Bugüne dek yasamim boyunca bana çobanlik eden Tanri,
\par 16 Beni bütün kötülüklerden kurtaran melek bu gençleri kutsasin! Adim ve atalarim Ibrahim'le Ishak'in adlari bu gençlerle yasasin! Yeryüzünde çogaldikça çogalsinlar."
\par 17 Yusuf, babasinin sag elini Efrayim'in basina koydugunu görünce, bundan hoslanmadi. Babasinin elini Efrayim'in basindan kaldirip Manasse'nin basina koymak istedi.
\par 18 "Baba, öyle degil" dedi, "Ilkin Manasse dogdu. Sag elini onun basina koy."
\par 19 Ancak babasi bunu istemedi. "Biliyorum oglum, biliyorum" dedi, "Manasse de büyük bir halk olacak. Ama küçük kardesi daha büyük bir halk olacak, soyundan birçok ulus dogacak."
\par 20 O gün onlari kutsayarak söyle dedi: "Israilliler, 'Tanri seni Efrayim ve Manasse gibi yapsin' Diyerek sizin adinizla kutsayacaklar." Böylece Yakup Efrayim'i Manasse'nin önüne geçirdi.
\par 21 Israil Yusuf'a, "Ben ölmek üzereyim" dedi, "Tanri sizinle olacak. Sizi atalarinizin topragina geri götürecek.
\par 22 Sana kardeslerinden bir pay fazla veriyorum; onu Amorlular'dan kilicimla, yayimla aldim."

\chapter{49}

\par 1 Yakup ogullarini çagirarak, "Yanima toplanin" dedi, "Gelecekte size neler olacagini anlatayim.
\par 2 "Yakupogullari, toplanin ve dinleyin, Babaniz Israil'e kulak verin.
\par 3 "Ruben, sen benim ilk oglum, gücümsün, Kudretimin ilk ürünüsün, Saygi ve güç bakimindan en üstünsün.
\par 4 Ama su gibi oynaksin, Üstün olmayacaksin artik. Çünkü babanin yatagina girip Onu kirlettin. Dösegimi rezil ettin.
\par 5 "Simon'la Levi kardestir, Kiliçlari siddet kusar.
\par 6 Gizli tasarilarina ortak olmam, Toplantilarina katilmam. Çünkü öfkelenince adam öldürdüler, Canlari istedikçe sigirlari sakatladilar.
\par 7 Lanet olsun öfkelerine, Çünkü siddetlidir. Lanet olsun gazaplarina, Çünkü zalimcedir. Onlari Yakup'ta bölecek Ve Israil'de dagitacagim.
\par 8 "Yahuda, kardeslerin seni övecek, Düsmanlarinin ensesinde olacak elin. Kardeslerin önünde egilecek.
\par 9 Yahuda bir aslan yavrusudur. Oglum benim! Avindan dönüp yere çömelir, Aslan gibi, disi bir aslan gibi yatarsin. Kim onu uyandirmaya cesaret edebilir?
\par 10 Sahibi gelene kadar Krallik asasi Yahuda'nin elinden çikmayacak, Yönetim hep onun soyunda kalacak, Uluslar onun sözünü dinleyecek.
\par 11 Esegini bir asmaya, Sipasini seçme bir dala baglayacak; Giysilerini sarapta, Kaftanini üzümün kizil kaninda yikayacak.
\par 12 Gözleri saraptan kizil, Disleri sütten beyaz olacak.
\par 13 "Zevulun deniz kiyisinda yasayacak, Liman olacak gemilere, Siniri Sayda'ya dek uzanacak.
\par 14 "Issakar semerler arasinda yatan güçlü esek gibidir;
\par 15 Ne zaman dinlenecek iyi bir yer, Hosuna giden bir ülke görse, Yüklenmek için sirtini eger, Angaryaya katlanir.
\par 16 "Dan kendi halkini yönetecek, Bir Israil oymagi gibi.
\par 17 Yol kenarinda bir yilan, Toprak yolda bir engerek olacak; Atin topuklarini isirip Atliyi sirtüstü düsüren bir engerek.
\par 18 "Ben senin kurtarisini bekliyorum, ya RAB.
\par 19 "Gad akincilarin saldirisina ugrayacak, Ama onlarin topuklarina saldiracak.
\par 20 "Zengin yemekler olacak Aser'de, Krallara yarasir lezzetli yiyecekler yetistirecek Aser.
\par 21 "Naftali saliverilmis geyige benzer, Sevimli yavrular dogurur.
\par 22 "Yusuf meyveli bir dal gibidir, Kaynak kiyisinda verimli bir dal gibi, Filizleri duvarlarin üzerinden asar.
\par 23 Okçular acimadan saldirdi ona. Düsmanca savurdular oklarini üzerine.
\par 24 Ama onun yayi saglam, Kollari esnek çikti; Yakup'un güçlü Tanrisi, Israil'in Kayasi, Çobani olan Tanri sayesinde.
\par 25 Sana yardim eden babanin Tanrisi'dir, Her ªeye Gücü Yeten Tanri'dir seni kutsayan. Yukaridaki göklerin Ve asagidaki denizlerin bereketiyle, Memelerin, rahimlerin bereketiyle O'dur seni kutsayan.
\par 26 Babanin kutsamalari ebedi daglarin nimetlerinden, Ebedi tepelerin bollugundan daha yücedir; Yusuf'un basi üzerinde, Kardesleri arasinda önder olanin üstünde olacak.
\par 27 "Benyamin aç kurda benzer; Sabah avini yer, Aksam ganimeti paylasir."
\par 28 Israil'in on iki oymagi bunlardir. Babalari onlari kutsarken bunlari söyledi. Her birini uygun biçimde kutsadi.
\par 29 Sonra Yakup ogullarina su buyruklari verdi: "Ben ölmek, halkima kavusmak üzereyim. Beni Kenan ülkesinde atalarimin yanina, Mamre yakinlarinda Hititli* Efron'un tarlasindaki magaraya, Makpela Tarlasi'ndaki magaraya gömün. Ibrahim o magarayi mezar yapmak üzere Hititli Efron'dan tarlasiyla birlikte satin almisti.
\par 30 (#49:29)
\par 31 Ibrahim'le karisi Sara, Ishak'la karisi Rebeka oraya gömüldüler. Lea'yi da ben oraya gömdüm.
\par 32 Tarla ile içindeki magara Hititler'den satin alindi."
\par 33 Yakup ogullarina verdigi buyruklari bitirince, ayaklarini yatagin içine çekti, son solugunu vererek halkina kavustu.

\chapter{50}

\par 1 Yusuf kendini babasinin üzerine atti, aglayarak onu öptü.
\par 2 Babasinin cesedini mumyalamalari için özel hekimlerine buyruk verdi. Hekimler Israil'i mumyaladilar.
\par 3 Bu is kirk gün sürdü. Mumyalama için bu süre gerekliydi. Misirlilar Israil için yetmis gün yas tuttu.
\par 4 Yas günleri geçince, Yusuf firavunun ev halkina, "Eger benden hosnut kaldinizsa, lütfen firavunla konusun" dedi,
\par 5 "Babam bana ant içirdi: 'Ölmek üzereyim. Beni Kenan ülkesinde kendim için kazdirdigim mezara gömeceksin' dedi. ªimdi lütfen firavuna bildirin, izin versin gideyim, babami gömüp geleyim."
\par 6 Firavun, "Git, babani göm, andini yerine getir" dedi.
\par 7 Böylece Yusuf babasini gömmeye gitti. Firavunun bütün görevlileri, sarayin ve Misir'in ileri gelenleri ona eslik etti.
\par 8 Yusuf'un bütün ailesi, kardesleri, babasinin ev halki da onunla birlikteydi. Yalniz çocuklari, davarlarla sigirlari Gosen'de biraktilar.
\par 9 Arabalarla atlilar da onlari izledi. Büyük bir alay olusturdular.
\par 10 ªeria Irmagi'nin dogusunda Atat Harmani'na varinca, yüksek sesle, aci aci agit yaktilar. Yusuf babasi için yedi gün yas tuttu.
\par 11 O bölgede yasayan Kenanlilar, Atat Harmani'ndaki yasi görünce, "Misirlilar ne kadar hüzünlü yas tutuyor!" dediler. Bu yüzden, ªeria Irmagi'nin dogusundaki bu yere Avel-Misrayim adi verildi.
\par 12 Yakup'un ogullari, babalarinin vermis oldugu buyrugu tam tamina yerine getirdiler.
\par 13 Onu Kenan ülkesine götürüp Mamre yakinlarinda Makpela Tarlasi'ndaki magaraya gömdüler. O magarayi mezar yapmak üzere tarlayla birlikte Hititli* Efron'dan Ibrahim satin almisti.
\par 14 Yusuf babasini gömdükten sonra, kendisi, kardesleri ve onunla birlikte babasini gömmeye gelenlerin hepsi Misir'a döndüler.
\par 15 Babalarinin ölümünden sonra Yusuf'un kardesleri, "Belki Yusuf bize kin besliyordur" dediler, "Ya ona yaptigimiz kötülüge karsilik bizden öç almaya kalkarsa?"
\par 16 Böylece Yusuf'a haber gönderdiler: "Babamiz ölmeden önce Yusuf'a söyle deyin diye buyurmustu: 'Kardeslerin sana kötülük yaptilar, lütfen onlarin suçunu, günahini bagisla.' Ne olur simdi günahimizi bagisla. Biz babanin Tanrisi'nin kullariyiz." Yusuf bu haberi alinca agladi.
\par 17 (#50:16)
\par 18 Bunun üzerine kardesleri gidip onun önünde yere kapanarak, "Senin köleniz" dediler.
\par 19 Yusuf, "Korkmayin" dedi, "Ben Tanri miyim?
\par 20 Siz bana kötülük düsündünüz, ama Tanri bugün oldugu gibi birçok halkin yasamini korumak için o kötülügü iyilige çevirdi.
\par 21 Korkmaniza gerek yok, size de çocuklariniza da bakacagim." Yüreklerine dokunacak güzel sözlerle onlara güven verdi.
\par 22 Yusuf'la babasinin ev halki Misir'a yerlestiler. Yusuf yüz on yil yasadi.
\par 23 Efrayim'in üç göbek çocuklarini gördü. Manasse'nin oglu Makir'in çocuklari onun elinde dogdu.
\par 24 Yusuf yakinlarina, "Ben ölmek üzereyim" dedi, "Ama Tanri kesinlikle size yardim edecek; sizi Ibrahim'e, Ishak'a, Yakup'a ant içerek söz verdigi topraklara götürecek."
\par 25 Sonra onlara ant içirerek, "Tanri kesinlikle size yardim edecek" dedi, "O zaman kemiklerimi buradan götürürsünüz."
\par 26 Yusuf yüz on yasinda öldü. Onu mumyalayip Misir'da bir tabuta koydular.


\end{document}