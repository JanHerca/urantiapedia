\begin{document}

\title{Daniel}


\chapter{1}

\par 1 Yahuda Krali Yehoyakim'in kralliginin üçüncü yilinda Babil Krali Nebukadnessar Yerusalim'in üzerine yürüyüp kenti kusatti.
\par 2 Rab, Yahuda Krali Yehoyakim'i ve Tanri'nin Tapinagi'ndaki bazi esyalari Nebukadnessar'in eline teslim etti. Nebukadnessar bunlari Sinar ülkesine götürüp kendi ilahinin tapinaginin hazinesine yerlestirdi.
\par 3 Kral Israilliler arasindan kral soyundan gelme ya da soylu bazi gençlerin seçilip saraya getirilmesi için saray görevlilerinin yöneticisi Aspenaz'a buyruk verdi. Bu gençler kusursuz, yakisikli, her konuda bilge, bilgili, ögrenmeye yetenekli, sarayda görev almaya uygun nitelikte kisiler olmaliydi. Aspenaz onlara Kildaniler'in* dilini ve yazisini ögretecekti.
\par 5 Kral bu gençler için kendi sofrasindan gündelik yiyecek ve sarap ayirdi. Üç yil egitildikten sonra gençler kralin önüne çikarilacaklardi.
\par 6 Seçilen gençler arasinda Yahudalilar'dan Daniel, Hananya, Misael ve Azarya da vardi.
\par 7 Saray görevlilerinin yöneticisi onlara yeni adlar koydu. Daniel'e Beltesassar, Hananya'ya Sadrak, Misael'e Mesak, Azarya'ya Abed-Nego adini verdi.
\par 8 Daniel dinsel açidan kendini kirletmemek için kralin onlara ayirdigi yemeklerden yemeyi de saraptan içmeyi de istemedi. Bu yoldan kendini kirletmemek için saray görevlilerinin yöneticisine ricada bulundu.
\par 9 Tanri saray görevlileri yöneticisinin Daniel'e sevgiyle, sevecenlikle davranmasini sagladi.
\par 10 Adam Daniel'e, "Yiyecek içecek payinizi ayiran efendimiz kraldan korkarim" dedi, "Eger yüzünüzü yasitiniz olan öbür gençlerin yüzünden daha solgun görürse, basimi tehlikeye sokmus olursunuz."
\par 11 Daniel, saray görevlileri yöneticisinin Hananya, Misael, Azarya ve kendisinin basina koydugu gözeticiye gidip, "Lütfen kullariniza on gün olanak taniyin" dedi, "Bu on gün içinde bize yemek için sebze, içmek için de su verilsin.
\par 13 Sonra yüzlerimizi kralin yemeklerini yiyen öbür gençlerin yüzleriyle kiyaslayin ve kullariniza gördügünüze göre davranin."
\par 14 Gözetici bu istegi kabul etti ve onlara on gün deneme firsati verdi.
\par 15 On gün sonra dört genç kralin yemeklerini yiyen öbür gençlerin hepsinden daha saglikli, daha iyi beslenmis görünüyordu.
\par 16 Böylece gözetici o günden sonra kralin gençler için ayirdigi yemekle sarabi kaldirdi ve onlara sebze vermeyi sürdürdü.
\par 17 Tanri bu dört gence her konuda bilgi, beceri, bilgelik verdi. Daniel her çesit görümü ve düsü yorumlayabiliyordu.
\par 18 Kralin belirledigi süre tamamlaninca, saray görevlileri yöneticisi gençleri Nebukadnessar'a götürdü.
\par 19 Kral onlarla görüstü; içlerinde Daniel, Hananya, Misael, Azarya gibisi yoktu. Bu yüzden kralin hizmetine onlar atandi.
\par 20 Kral bilgelik ve anlayisla ilgili konularda onlari sinadi ve dört genci ülkesindeki bütün sihirbazlardan, falcilardan on kat üstün buldu.
\par 21 Kral Kores'in kralliginin birinci yilina dek Daniel sarayda kaldi.

\chapter{2}

\par 1 Kralliginin ikinci yilinda Nebukadnessar bir düs gördü. Ruhu üzüntüyle sarsildi, uykusu kaçti.
\par 2 Düsünün ne oldugunu söylesinler diye sihirbazlari, falcilari, büyücüleri, yildizbilimcileri çagirtti. Hepsi gelip kralin önünde durdular.
\par 3 Kral, "Beni üzüntüyle sarsan bir düs gördüm. Ne anlama geldigini ögrenmek istiyorum" dedi.
\par 4 Yildizbilimciler Aramice, "Ey kral, sen çok yasa!" dediler, "Düsünü bu kullarina anlat ki, ne anlama geldigini söyleyelim."
\par 5 Kral, "Gördügüm düsü ve ne anlama geldigini bana açiklamazsaniz, kararim kesin, paramparça edileceksiniz" diye karsilik verdi, "Evleriniz de çöplüge çevrilecek.
\par 6 Ama düsü ve ne anlama geldigini açiklayabilirseniz, sizi büyük armaganlarla ödüllendirip onurlandiracagim. Onun için bana düsü ve ne anlama geldigini açiklayin."
\par 7 Onlar yine, "Ey kral, düsü bu kullarina anlat ki, ne anlama geldigini söyleyelim" dediler.
\par 8 Bunun üzerine kral, "Kararimin kesin oldugunu bildiginiz için zaman kazanmak istediginizi anliyorum" dedi,
\par 9 "Ama düsün ne oldugunu bana açiklamazsaniz, sizin için tek ceza vardir. Durumun degisecegini umarak bana yalan yanlis seyler söylemek için aranizda anlasmissiniz. Simdi bana düsün ne oldugunu söyleyin ki, ne anlama geldigini açiklayabileceginizi anlayayim."
\par 10 Yildizbilimciler, "Yeryüzünde senin bu istegini yerine getirecek tek kisi yoktur" diye yanitladilar, "Kaldi ki, büyük, güçlü hiçbir kral bir sihirbazdan, falcidan ya da yildizbilimciden böyle bir sey istememistir.
\par 11 Kralin istegini yerine getirmek güçtür. Insanlar arasinda yasamayan ilahlardan baska krala bunu açiklayabilecek kimse yoktur."
\par 12 Buna çok öfkelenen kral, Babil'deki bütün bilgelerin öldürülmesini buyurdu.
\par 13 Böylece hepsinin öldürülmesi için buyruk çikti. Daniel'le arkadaslarinin öldürülmesi için de adamlar gönderildi.
\par 14 Daniel Babil'in bilgelerini öldürmeye giden kralin muhafiz birligi komutani Aryok'la bilgece, akillica konustu.
\par 15 Aryok'a, "Kralin buyrugu neden bu denli sert?" diye sordu. Aryok durumu Daniel'e anlatti.
\par 16 Bunun üzerine Daniel krala gidip düsünün ne anlama geldigini söyleyebilmesi için zaman istedi.
\par 17 Sonra evine dönüp olup bitenleri arkadaslari Hananya'ya, Misael'e, Azarya'ya anlatti.
\par 18 Göklerin Tanrisi'na yakarmalarini istedi; öyle ki, Tanri onlara lütfedip bu gizi açiklasin ve kendisiyle arkadaslari Babil'in öbür bilgeleriyle birlikte öldürülmesinler.
\par 19 Gece giz bir görümde Daniel'e açiklandi. Bunun üzerine Daniel Göklerin Tanrisi'ni övdü.
\par 20 Söyle dedi: "Tanri'nin adina öncesizlikten sonsuzluga dek övgüler olsun! Bilgelik ve güç O'na özgüdür.
\par 21 O'dur zamanlari ve mevsimleri degistiren. Krallari tahttan indirir, tahta çikarir. Bilgelere bilgelik, Anlayislilara bilgi verir.
\par 22 Derin ve gizli seyleri ortaya çikarir, Karanlikta neler oldugunu bilir, Çevresi isikla kusatilmistir.
\par 23 Ey atalarimin Tanrisi, Sana sükreder, seni överim. Sen ki, bana bilgelik ve güç verdin, Senden istediklerimizi bana bildirdin Ve kralin düsünü bize açikladin."
\par 24 Daniel, kralin Babil'in bilgelerini öldürmeye atadigi Aryok'a giderek, "Babil'in bilgelerini yok etme" dedi, "Beni krala götür, düsünün ne anlama geldigini açiklayacagim."
\par 25 Aryok onu hemen krala götürdü ve, "Sürgündeki Yahudalilar arasinda kralin düsünü yorumlayabilecek birini buldum" dedi.
\par 26 Kral, öbür adi Beltesassar olan Daniel'e, "Gördügüm düsü ve ne anlama geldigini bana söyleyebilir misin?" diye sordu.
\par 27 Daniel söyle yanitladi: "Kralin açiklanmasini istedigi gizi ne bir bilge, ne falci, ne de sihirbaz açiklayabilir.
\par 28 Ama gökte gizleri açiklayan bir Tanri var. Gelecekte neler olacagini Kral Nebukadnessar'a O bildirmistir. Yataginda yatarken gördügün düs ve görümler sunlardir:
\par 29 "Sen, ey kral, yatarken gelecekle ilgili düsüncelere daldin, gizleri açan da neler olacagini sana bildirdi.
\par 30 Bana gelince, ey kral, öbür insanlardan daha bilge oldugum için degil, düsünün ne anlama geldigini bilesin, aklindan geçenleri anlayasin diye bu giz bana açiklandi.
\par 31 "Ey kral, düsünde önünde duran büyük bir heykel gördün. Çok büyük ve olaganüstü parlakti, görünüsü ürkütücüydü.
\par 32 Basi saf altindan, gögsüyle kollari gümüsten, karniyla kalçalari tunçtan*,
\par 33 bacaklari demirden, ayaklarinin bir kesimi demirden, bir kesimi kildendi.
\par 34 Sen bakiyordun ki, bir tas insan eli degmeden kesilip heykelin demirden, kilden ayaklarina çarparak onlari paramparça etti.
\par 35 Demir, kil, tunç, gümüs, altin ayni anda parçalandi; yazin harman yerindeki saman çöpleri gibi oldular. Derken bir rüzgar çikti, hiç iz birakmadan hepsini alip götürdü. Heykele çarpan tassa büyük bir dag oldu, bütün dünyayi doldurdu.
\par 36 "Gördügün düs buydu. Simdi de ne anlama geldigini sana açiklayalim.
\par 37 Sen, ey kral, krallarin kralisin. Göklerin Tanrisi sana egemenlik, güç, kudret, yücelik verdi.
\par 38 Insanogullarini, yabanil hayvanlari, gökte uçan kuslari senin eline teslim etti. Seni hepsine egemen kildi. Altindan bas sensin.
\par 39 Senden sonra senden daha asagi durumda baska bir krallik çikacak. Sonra bütün dünyada egemenlik sürecek tunçtan üçüncü bir krallik çikacak.
\par 40 Dördüncü krallik demir gibi güçlü olacak. Çünkü demir her seyi kirip ezer. Demir gibi tümünü kirip parçalayacak.
\par 41 Ayaklarla parmaklarin bir kesiminin çömlekçi kilinden, bir kesiminin demirden oldugunu gördün; yani bölünmüs bir krallik olacak bu. Öyleyken onda demirin gücü de bulunacak, çünkü demiri kille karisik gördün.
\par 42 Ayak parmaklarinin bir kesimi demirden, bir kesimi kilden oldugu gibi, kralligin da bir bölümü güçlü, bir bölümü zayif olacak.
\par 43 Demirin kille karisik oldugunu gördügüne göre halklar evlilik bagiyla birbirleriyle karisacaklar, ama demirin kille karismadigi gibi onlar da birbirine bagli kalmayacaklar.
\par 44 "Bu krallar döneminde Göklerin Tanrisi hiç yikilmayacak, baska halkin eline geçmeyecek bir krallik kuracak. Bu krallik önceki kralliklari ezip yok edecek, kendisiyse sonsuza dek sürecek.
\par 45 Insan eli degmeden dagdan kesilip gelen tasin demiri, tuncu, kili, gümüsü, altini parçaladigini gördün. Ulu Tanri bundan sonra neler olacagini krala açiklamistir. Düs gerçek, yorumu da güvenilirdir."
\par 46 Bunun üzerine Kral Nebukadnessar Daniel'in önünde yüzüstü yere kapandi. Ona bir sunu ve buhur sunulmasini buyurdu.
\par 47 Daniel'e, "Madem bu gizi açiklayabildin, Tanrin gerçekten tanrilarin Tanrisi, krallarin Efendisi" dedi, "Gizleri açan O'dur."
\par 48 Sonra Daniel'i yüksek bir göreve getirdi; ona birçok degerli armagan verdi. Onu Babil Ili'ne vali atadi, Babil'in bütün bilgelerinin baskani yapti.
\par 49 Daniel'in istegi üzerine Sadrak'i, Mesak'i, Abed-Nego'yu da Babil Ili'nde yüksek görevlere atadi. Daniel ise sarayda kaldi.

\chapter{3}

\par 1 Kral Nebukadnessar altin bir heykel yapti; boyu altmis, eni alti arsindi. Onu Babil Ili'nde, Dura Ovasi'na dikti.
\par 2 Satraplari*, kaymakamlari, valileri, danismanlari, haznedarlari, yargiçlari, güvenlik görevlilerini ve illerin bütün öbür yüksek memurlarini diktigi heykeli adama törenine çagirtti.
\par 3 Böylece satraplar, kaymakamlar, valiler, danismanlar, haznedarlar, yargiçlar, güvenlik görevlileri ve illerin bütün öbür yüksek memurlari Kral Nebukadnessar'in diktigi heykeli adama töreni için toplanarak heykelin önünde durdular.
\par 4 Sonra haberci yüksek sesle bagirdi: "Ey halklar, uluslar, her dilden insanlar, size söyle yapmaniz buyruluyor:
\par 5 Boru, ney, lir, kanun, arp, davul ve her çesit çalgi sesini duyar duymaz yere kapanip Kral Nebukadnessar'in dikmis oldugu altin heykele tapinacaksiniz.
\par 6 Her kim yere kapanip tapinmazsa hemen kizgin firina atilacaktir."
\par 7 Bu yüzden ne zaman boru, ney, lir, kanun, arp ve her çesit çalgi sesi duyulsa, bütün halklar, uluslar, her dilden insanlar yere kapanip Kral Nebukadnessar'in diktigi altin heykele tapindilar.
\par 8 Bunun üzerine bazi Kildaniler* yaklasip Yahudiler'i suçladilar.
\par 9 Kral Nebukadnessar'a, "Ey kral, sen çok yasa!" dediler,
\par 10 "Boru, ney, lir, kanun, arp, davul ve her çesit çalgi sesini duyan herkes yere kapanip altin heykele tapinacak; kim yere kapanip tapinmazsa kizgin firina atilacak diye bir buyruk çikardin, ey kral.
\par 12 Oysa Babil Ili'nde yüksek görevlere atadigin Sadrak, Mesak, Abed-Nego adinda bazi Yahudiler var. Bu adamlar seni saymadilar, ey kral. Senin ilahlarina kulluk etmiyor, diktigin altin heykele tapinmiyorlar."
\par 13 Büyük öfkeye kapilan Nebukadnessar, Sadrak'i, Mesak'i, Abed-Nego'yu çagirtti. Bu kisiler kralin yanina getirildiler.
\par 14 Nebukadnessar, "Ey Sadrak, Mesak, Abed-Nego, ilahlarima kulluk etmediginiz, diktigim altin heykele tapinmadiginiz dogru mu?" diye sordu,
\par 15 "Simdi boru, ney, lir, kanun, arp, davul ve her çesit çalgi sesini duyar duymaz yere kapanip yaptigim heykele tapinmaya hazirsaniz ne iyi! Ama ona tapinmazsaniz, hemen kizgin firina atilacaksiniz. O zaman bakalim hangi ilah sizi elimden kurtaracak?"
\par 16 Sadrak, Mesak, Abed-Nego, "Bu konuda kendimizi savunma geregini duymuyoruz" diye karsilik verdiler,
\par 17 "Kizgin firina atilsak bile, ey kral, kendisine kulluk ettigimiz Tanri bizi kizgin firindan kurtarabilir; senin elinden de bizi kurtaracaktir.
\par 18 Ama bizi kurtarmasa bile bil ki, ey kral, ilahlarina kulluk etmeyiz, diktigin altin heykele tapinmayiz."
\par 19 Nebukadnessar Sadrak, Mesak, Abed-Nego'ya çok öfkelendi; onlara karsi tutumu degisti. Firinin her zamankinden yedi kat daha çok isitilmasini buyurdu.
\par 20 Sonra ordusundaki bazi güçlü askerlere Sadrak'i, Mesak'i, Abed-Nego'yu baglayip kizgin firina atmalarini buyurdu.
\par 21 Böylece bu kisiler, salvarlari, kaftanlari, sariklari ve öbür giysileriyle birlikte baglanip kizgin firina atildilar.
\par 22 Kralin buyrugu çok siki, firin da çok isitilmis oldugundan, Sadrak'i, Mesak'i, Abed-Nego'yu götüren adamlari atesin alevleri yakip öldürdü.
\par 23 Üç adamsa -Sadrak, Mesak, Abed-Nego- bagli olarak kizgin firina düstüler.
\par 24 O zaman Kral Nebukadnessar saskinlik içinde birden ayaga kalkti. Danismanlarina, "Biz atesin içine bagli üç kisi atmadik mi?" diye sordu. Danismanlar, "Kuskusuz, ey kral" diye karsilik verdiler.
\par 25 Kral, "Ben dört kisi görüyorum" dedi, "Atesin içinde yürüyorlar, baglarindan çözülmüs, hiçbir zarara ugramamislar. Dördüncünün görünümü de bir ilahi varliga benziyor."
\par 26 Sonra kizgin firinin kapisina yaklasarak, "Ey Yüce Tanri'nin kullari Sadrak, Mesak, Abed-Nego, disari çikip buraya gelin!" diye seslendi. Bunun üzerine Sadrak, Mesak, Abed-Nego atesin içinden çiktilar.
\par 27 Satraplar*, kaymakamlar, valiler, kralin danismanlari onlarin çevresinde toplandilar. Adamlarin bedenlerinde atesin hiçbir etkisi olmadigini gördüler. Baslarindaki tek saç yanmamis, giysileri degismemis, atesin kokusu üzerlerine sinmemisti.
\par 28 Bunun üzerine Nebukadnessar, "Sadrak, Mesak ve Abed-Nego'nun Tanrisi'na övgüler olsun!" dedi, "Melegini gönderip kendisine güvenen kullarini kurtardi. Onlar buyruguma karsi geldiler, kendi Tanrilari'ndan baska bir ilaha kulluk edip tapinmamak için canlarini tehlikeye attilar.
\par 29 Iste buyuruyorum: Hangi halktan, ulustan ya da dilden olursa olsun, Sadrak, Mesak ve Abed-Nego'nun Tanrisi'ndan saygisizca söz eden herkes paramparça edilecek, evleri çöplüge çevrilecek. Çünkü böyle kurtarabilen baska bir tanri yoktur."
\par 30 Sonra Sadrak'i, Mesak'i, Abed-Nego'yu Babil Ili'nde daha yüksek görevlere atadi.

\chapter{4}

\par 1 Kral Nebukadnessar dünyadaki bütün halklara, uluslara ve her dilden insanlara su bildiriyi gönderdi: "Esenliginiz bol olsun!
\par 2 Yüce Tanri'nin benim için gerçeklestirdigi belirtileri ve sasilasi isleri size bildirmeyi uygun gördüm.
\par 3 "Belirtileri ne büyük! Sasilasi isleri ne yüce! Kralligi ebedi kralliktir, Egemenligi kusaklar boyu sürecek.
\par 4 "Ben, Nebukadnessar, evimde huzur, sarayimda gönenç içindeydim.
\par 5 Beni korkutan bir düs gördüm. Yatagimda yatarken düsüncelerimle görümlerim beni ürküttü.
\par 6 Düsün ne anlama geldigini açiklamalari için Babil'in bütün bilgelerinin yanima getirilmesini buyurdum.
\par 7 Sihirbazlar, yildizbilimciler, falcilar yanima gelince, gördügüm düsü onlara anlattimsa da ne anlama geldigini açiklayamadilar.
\par 8 Sonunda ilahimin adindan gelen Beltesassar adiyla çagrilan ve kendisinde kutsal ilahlarin ruhu bulunan Daniel yanima geldi. Gördügüm düsü ona anlattim.
\par 9 "Ona söyle dedim: Ey sihirbazlarin baskani Beltesassar, sende kutsal ilahlarin ruhu oldugunu, her gizi açiklayabilecegini biliyorum. Iste gördügüm düs: Ne anlama geldigini bana açikla.
\par 10 Yatarken gördügüm görümler sunlar: Dünyanin ortasinda çok yüksek bir agaç gördüm.
\par 11 Agaç büyüdü, güçlendi, boyu göklere eristi. Dünyanin dört bucagindan görülüyordu.
\par 12 Yapraklari güzeldi, herkese yetecek kadar bol meyvesi vardi. Yabanil hayvanlar gölgesinde bariniyor, gökte uçan kuslar dallarina tünüyordu. Her canli ondan besleniyordu.
\par 13 "Yatagimda yatarken gördügüm görümlerde gökten inen bir gözcü, kutsal bir varlik gördüm.
\par 14 Yüksek sesle, 'Agaci ve dallarini kesin, yapraklarini yolun, meyvesini atin diye bagirdi, 'Altinda barinan hayvanlarla dallarina tüneyen kuslar kaçsin.
\par 15 Ama köklerin bulundugu kütügü demirle, tunçla* çevreleyip yerde, otlarin içinde birakin. "'Gögün çiyiyle islansin, hayvanlarla birlikte yerdeki otlardan pay alsin.
\par 16 Ondaki insan yüregi degistirilsin, yerine hayvan yüregi verilsin. Üzerinden yedi vakit geçsin.
\par 17 Bu yargiyi gözcüler, karari kutsallar verdi. Öyle ki, her canli Yüce Olan'in insan kralliklari üzerinde egemenlik sürdügünü ve onlari diledigi kisiye, en hor görülen birine bile verebilecegini bilsin.
\par 18 "Iste ben Kral Nebukadnessar'in gördügü düs! Simdi, ey Beltesassar, bunun ne anlama geldigini söyle. Çünkü kralligimdaki bilgelerin hiçbiri bu düsün ne anlama geldigini bana açiklayamadi. Ama sen açiklayabilirsin, çünkü kutsal ilahlarin ruhu var sende."
\par 19 O zaman öbür adi Beltesassar olan Daniel bir süre saskin saskin durdu, düsünceleri onu ürküttü. Bunun üzerine kral, "Ey Beltesassar, bu düs de yorumu da seni ürkütmesin" dedi. Beltesassar, "Ey efendim, keske bu düs senden nefret edenlerin, yorumu da düsmanlarinin basina gelseydi!" diye karsilik verdi,
\par 20 "Büyüyen, güçlenen, boyu göklere erisen, dünyadaki herkesçe görülebilen bir agaç gördün.
\par 21 Yapraklari güzeldi, meyvesi herkese yetecek kadar boldu. Yabanil hayvanlar altinda barinir, gökte uçan kuslar dallarina tünerdi.
\par 22 Ey kral, o agaç sensin! Sen büyüdün, güçlendin. Büyüklügün giderek göklere eristi, egemenligin dünyanin dört bucagina yayildi.
\par 23 "Sen, ey kral, bir gözcünün, kutsal bir varligin gökten indigini gördün. 'Agaci kesip yok edin, ama köklerin bulundugu kütügü demirle, tunçla* çevreleyip yerde, otlarin içinde birakin. Gögün çiyiyle islansin; üzerinden yedi vakit geçinceye dek yabanil hayvanlarla birlikte pay alsin diyordu.
\par 24 "Ey efendim kral, düsün anlami ve Yüce Olan'in senin basina getirecegi yargi sudur:
\par 25 Insanlar arasindan kovulacak, yabanil hayvanlarla yasayacaksin; öküz gibi otla beslenecek, gögün çiyiyle islanacaksin. Yüce Olan'in insan kralliklari üzerinde egemenlik sürdügünü ve kralligi diledigi kisiye verdigini anlayincaya dek yedi vakit geçecek.
\par 26 Köklerin bulundugu kütügün birakilmasi için buyruk verildi. Bunun anlami su: Sen göklerin egemenlik sürdügünü anlayinca kralligin sana geri verilecek.
\par 27 Bu yüzden, ey kral, ögüdümü benimse: Dogru olani yaparak günahindan, düskünlere iyilik ederek suçlarindan vazgeç. Olur ya, gönencin uzun sürer."
\par 28 Bunlarin hepsi Kral Nebukadnessar'in basina geldi.
\par 29 On iki ay sonra kral Babil Sarayi'nin daminda geziniyordu.
\par 30 Kral, "Iste onurum ve yüceligim için üstün gücümle kralligimin baskenti olarak kurdugum büyük Babil!" dedi.
\par 31 Daha sözünü bitirmeden gökten bir ses duyuldu: "Ey Kral Nebukadnessar, krallik senden alindi.
\par 32 Insanlar arasindan kovulacak, yabanil hayvanlarla yasayacaksin. Öküz gibi otla besleneceksin. Yüce Olan'in insan kralliklari üzerinde egemenlik sürdügünü ve kralligi diledigi kisiye verdigini anlayincaya dek yedi vakit geçecek."
\par 33 Nebukadnessar'a iliskin bu söz hemen yerine geldi. Insanlar arasindan kovuldu. Öküz gibi otla beslendi. Bedeni gögün çiyiyle islandi. Saçi kartal tüyü, tirnaklari kus pençesi gibi uzadi.
\par 34 Belirlenen sürenin sonunda ben Nebukadnessar gözlerimi göge kaldirdim ve kendime geldim. Yüce Olan'i övdüm. Sonsuza dek Diri Olan'i onurlandirip yücelttim. O'nun egemenligi ebedi egemenliktir, Kralligi kusaklar boyu sürecek.
\par 35 Dünyada yasayanlar bir hiç sayilir. O gökteki güçlere de dünyada yasayanlara da Diledigini yapar. O'nun elini durduracak, O'na, "Ne yapiyorsun?" diyecek kimse yoktur.
\par 36 O anda aklim basima geldi. Kralligimin yüceligi için onurum ve görkemim bana geri verildi. Danismanlarimla soylu adamlarim beni aradilar. Kralligima kavustum, bana daha büyük yücelik verildi.
\par 37 Ben Nebukadnessar Göklerin Krali'na sükrederim. O'nu över, yüceltirim. Çünkü bütün yaptiklari gerçek, yollari dogrudur; kendini begenmisleri alçaltmaya gücü yeter.

\chapter{5}

\par 1 Kral Belsassar soylu adamlarindan bin kisiye büyük bir sölen verdi, onlarla sarap içti.
\par 2 Sarabini keyifle içerken, atasi Nebukadnessar'in Yerusalim'deki tapinaktan çikarip getirdigi altin ve gümüs kaplarin getirilmesini buyurdu. Öyle ki, kendisi, karilari, cariyeleri, soylu adamlari onlarla içsinler.
\par 3 Böylece Tanri'nin Yerusalim'deki tapinagindan alinan altin kaplar getirildi; kral, karilari, cariyeleri, soylu adamlari onlarla içtiler.
\par 4 Saraplarini içerken altindan, gümüsten, tunçtan*, demirden, agaçtan, tastan ilahlari övdüler.
\par 5 Ansizin bir insan elinin parmaklari belirdi, kandilligin yanindaki saray duvarinin sivasi üzerine yazmaya basladi. Kral yazan eli gördü,
\par 6 aklindan geçenler onu ürküttü, benzi soldu; eli ayagi tutmaz oldu, dizlerinin bagi çözüldü.
\par 7 Yüksek sesle Babil'in bilgelerini - falcilarla yildizbilimcileri - çagirtti. Onlara, "Bu yaziyi kim okuyup ne anlama geldigini bana açiklarsa, kendisine mor giysi giydirilip boynuna altin zincir takilacak ve ülkede üçüncü önder olacak" dedi.
\par 8 Kralin bütün bilgeleri geldiyse de yaziyi kimse okuyamadi, ne anlama geldigini de açiklayamadi.
\par 9 Bu yüzden Kral Belsassar daha da korktu, benzi büsbütün soldu. Soylu adamlariysa saskindi.
\par 10 Kralla soylularin seslerini duyan kraliçe sölen salonuna geldi. "Çok yasa, ey kral!" dedi, "Aklindan geçenler seni ürkütmesin, benzin solmasin!
\par 11 Ülkende kendisinde kutsal ilahlarin ruhu bulunan biri var. Atan Kral Nebukadnessar'in döneminde kavrayisa, sagduyuya, ilahlara özgü bilgelige sahip olmakla taninirdi. Atan Kral Nebukadnessar onu sihirbazlarin, yildizbilimcilerin, falcilarin baskanligina atadi.
\par 12 Kralin Beltesassar diye çagirdigi Daniel olaganüstü bir ruha, bilgiye, sagduyuya sahiptir. Üstelik düsleri yorumlama, bilmeceleri çözme, gizemleri açiklama yetenegi de vardir. Daniel'i çagirt, yazinin ne anlama geldigini o sana söyleyecektir."
\par 13 Böylece Daniel'i kralin önüne getirdiler. Kral, "Kral atamin Yahuda'dan getirdigi, Yahuda sürgünlerinden Daniel sen misin?" diye sordu,
\par 14 "Sende ilahlarin ruhu bulundugunu, kavrayis, sagduyu ve olaganüstü bilgelikle donanmis oldugunu duydum.
\par 15 Bu yaziyi okuyup ne anlama geldigini söylemeleri için bilgelerle falcilari çagirttim. Ama ne anlama geldigini açiklayamadilar.
\par 16 Senin yorum yapabildigini, gizemleri açiklayabildigini duydum. Bu yaziyi okur, ne anlama geldigini açiklayabilirsen, sana mor giysi giydirilip boynuna altin zincir takilacak; ülkede üçüncü önder olacaksin."
\par 17 Daniel, "Armaganlarin senin olsun, ödüllerini de bir baskasina ver" diye karsilik verdi, "Ama ben yine de yaziyi okuyup ne anlama geldigini sana açiklayacagim.
\par 18 "Ey kral, Yüce Tanri atan Nebukadnessar'a kralligi, büyüklügü, yüceligi, görkemi verdi.
\par 19 Tanri'nin sagladigi büyüklük yüzünden bütün halklar, uluslar, her dilden insan ondan korkup titredi. Diledigini öldürür, diledigini yasatirdi; diledigini yüceltir, diledigini alçaltirdi.
\par 20 Ne var ki, gurura kapilip saygisizlikta direnince krallik tahtindan indirildi, yüceligi kendisinden alindi.
\par 21 Insanlar arasindan kovuldu ve ona hayvan yüregi verildi. Yüce Tanri'nin insanlarin kralligi üzerinde egemenlik sürdügünü, onu diledigi kisiye verdigini anlayincaya dek yaban esekleri arasinda yasadi, öküz gibi otla beslendi, bedeni gögün çiyiyle islandi.
\par 22 "Ama ey sen, onun torunu Belsassar, bunlarin hepsini bildigin halde alçakgönüllülügü benimsemedin.
\par 23 Bunun yerine gögün Rab'bine karsi kendini yükselttin. O'nun tapinagindan aldiklari kaplari sana getirdiler. Sen, karilarin, cariyelerin, soylu adamlarin onlarla sarap içtiniz. Görmeyen, duymayan, anlamayan altindan, gümüsten, tunçtan, demirden, agaçtan, tastan ilahlari övdün. Solugunu elinde tutan, bütün yollarini gözeten Tanri'yi ise yüceltmedin.
\par 24 Bu yüzden Tanri o yaziyi yazan eli gönderdi.
\par 25 "Yazilan yazi sudur: MENE, MENE, TEKEL ve PARSIN.
\par 26 "Bu sözcüklerin anlami sudur: MENE: Tanri senin kralliginin günlerini saydi ve ona son verdi.
\par 27 TEKEL: Terazide tartildin ve eksik bulundun.
\par 28 PERES: Kralligin ikiye bölünerek Medler'le Persler'e verildi."
\par 29 Belsassar'in buyrugu üzerine Daniel'e mor giysi giydirilip boynuna altin zincir takildi ve ülkede üçüncü önder ilan edildi.
\par 30 Kildan* Krali Belsassar o gece öldürüldü.
\par 31 Altmis iki yasinda olan Medli Darius kralligi eline geçirdi.

\chapter{6}

\par 1 Darius bütün ülkeyi yönetecek yüz yirmi satrap* atamayi uygun gördü.
\par 2 Bunlarin basina da biri Daniel olmak üzere üç bakan atadi. Krala zarar gelmemesi için bakanlar satraplardan hesap soracaklardi.
\par 3 Kendisinde bulunan olaganüstü ruh sayesinde Daniel öbür bakanlarla satraplardan üstün oldugundan, kral onu bütün ülkenin basina atamayi tasarliyordu.
\par 4 Bunun üzerine öbür bakanlarla satraplar Daniel'i ülke yönetimi konusunda suçlamak için firsat kollamaya basladilar. Ancak ne suçlanacak bir yanini, ne de bir yanlisini buldular. Çünkü Daniel güvenilir biriydi. Kendisinde hiçbir eksiklik ya da yanlislik bulamadilar.
\par 5 Sonunda, "Daniel'i Tanrisi'nin Yasasi'yla ilgili bir konuda suçlayamazsak, bir suçlama nedeni bulamayacagiz" dediler.
\par 6 Bunun üzerine bakanlarla satraplar hep birlikte krala gidip, "Ey Kral Darius, çok yasa!" dediler,
\par 7 "Ülkenin bütün bakanlari, kaymakamlari, satraplari, danismanlari, valileri olarak kralin zorlu bir yasa çikarmasi üzerinde anlastik. Ey kral, kim otuz gün içinde senden baska bir insana ya da ilaha dua ederse, aslan çukuruna atilsin.
\par 8 Simdi, ey kral, yasagi koy; Medler'le Persler'in degismez yasasi uyarinca yaziyi imzala ki degistirilemesin."
\par 9 Böylece Kral Darius yasagi içeren yasayi imzaladi.
\par 10 Daniel yasanin imzalandigini ögrenince evine gitti. Yukari odasinin Yerusalim yönüne bakan pencereleri açikti. Daha önce yaptigi gibi her gün üç kez diz çöküp dua etti, Tanrisi'na övgüler sundu.
\par 11 Ona tuzak kuran adamlar hep birlikte oraya gittiklerinde, onu Tanrisi'na dua edip yalvarirken gördüler.
\par 12 Bunun üzerine krala gidip çikardigi yasayla ilgili sunlari söylediler: "Ey kral, kim otuz gün içinde senden baska bir insana ya da ilaha dua ederse, aslan çukuruna atilsin diye yasa imzalamadin mi?" Kral, "Medler'le Persler'in degismez yasasi uyarinca çikardigim yasa geçerlidir" diye karsilik verdi.
\par 13 Bunun üzerine, "Ey kral, Yahuda sürgünlerinden olan Daniel seni de imzaladigin yasayi da saymiyor; günde üç kez dua ediyor" dediler.
\par 14 Bunu duyan kral çok üzüldü, Daniel'i kurtarmayi kafasina koydu. Onu kurtarmak için günes batincaya dek ugrasti.
\par 15 O zaman adamlar toplu halde krala gidip, "Ey kral, Medler'le Persler'in yasasi uyarinca, kralin koydugu yasanin ya da yasagin degistirilemeyecegini bilmelisin" dediler.
\par 16 Bunun üzerine kral Daniel'i getirip aslan çukuruna atmalarini buyurdu. Daniel'e de, "Kendisine sürekli kulluk ettigin Tanrin seni kurtarsin!" dedi.
\par 17 Bir tas getirip çukurun agzina koydular. Daniel'le ilgili hiçbir sey degistirilmesin diye kral hem kendi mühür yüzügüyle, hem soylularin mühür yüzükleriyle tasi mühürledi.
\par 18 Sonra sarayina döndü; geceyi yemek yemeden, eglenmeden geçirdi; uykusu kaçti.
\par 19 Safak sökerken kalkip acele aslan çukuruna gitti.
\par 20 Çukura yaklasinca üzgün bir sesle, "Ey yasayan Tanri'nin kulu Daniel, kendisine sürekli kulluk ettigin Tanrin seni aslanlardan kurtarabildi mi?" diye haykirdi.
\par 21 Daniel, "Ey kral, sen çok yasa!" diye yanitladi,
\par 22 "Tanrim melegini gönderip aslanlarin agzini kapadi. Beni incitmediler. Çünkü Tanri'nin önünde suçsuz bulundum. Sana karsi da, ey kral, hiçbir yanlislik yapmadim."
\par 23 Kral buna çok sevindi, Daniel'i çukurdan çikarmalarini buyurdu. Daniel çukurdan çikarildi. Bedeninde hiçbir yara izi bulunmadi. Çünkü Tanrisi'na güvenmisti.
\par 24 Kralin buyrugu uyarinca, Daniel'i haksiz yere suçlayan adamlari, karilariyla, çocuklariyla birlikte getirip aslan çukuruna attilar. Daha çukurun dibine varmadan aslanlar onlari kapip kemiklerini kirdilar.
\par 25 Kral Darius dünyada yasayan bütün halklara, uluslara ve her dilden insanlara söyle yazdi: "Esenliginiz bol olsun!
\par 26 Kralligimda yasayan herkesin Daniel'in Tanrisi'ndan korkup titremesini buyuruyorum. O yasayan Tanri'dir, Sonsuza dek var olacak. Kralligi yikilmayacak, Egemenligi son bulmayacak.
\par 27 O kurtarir, O yasatir, Gökte de yerde de Belirtiler, sasilasi isler yapar. Daniel'i aslanlarin pençesinden kurtaran O'dur."
\par 28 Böylece Darius'un ve Persli Kores'in kralligi döneminde Daniel'in isleri iyi gitti.

\chapter{7}

\par 1 Babil Krali Belsassar'in kralliginin birinci yilinda, Daniel yataginda yatarken bir düs ve görümler gördü. Sonra düsünün özetini yazdi;
\par 2 söyle dedi: "Gece bir görümde gögün dört rüzgarinin büyük denize saldirdigini gördüm.
\par 3 Denizden birbirinden farkli dört büyük yaratik çikti.
\par 4 "Birinci yaratik aslana benziyordu, kartal kanatlari vardi. Ben bakarken kanatlari koparildi, yaratik yerden kaldirildi, insan gibi ayaklari üzerine durduruldu. Ona bir insan yüregi verildi.
\par 5 "Ikinci yaratik ayiya benziyordu. Bir yani üzerinde dogrulmustu. Agzinda, disleri arasinda üç kaburga kemigi vardi. Ona, 'Haydi kalk, yiyebildigin kadar et ye! dediler.
\par 6 "Sonra baktim, parsa benzer bir baska yaratik gördüm. Sirtinda dört kus kanadi vardi. Bu yaratigin dört basi vardi ve ona egemenlik verilmisti.
\par 7 "Bundan sonraki gece görümlerimde korkunç, ürkütücü, çok güçlü dördüncü bir yaratik gördüm. Büyük demir disleri vardi; yiyip parçaliyor, artakalani ayaklari altinda çigniyordu. Kendisinden önceki yaratiklara benzemiyordu. On boynuzu vardi.
\par 8 "Ben gözümü dikmis boynuzlara bakarken, onlarin arasindan daha küçük baska bir boynuz çikti. Ilk boynuzlardan üçü onun önünde söküldü. Bu boynuzun insan gözü gibi gözleri, böbürlenen bir agzi vardi.
\par 9 "Ben bakarken Tahtlar kuruldu, Eskiden beri var Olan yerine oturdu. Giysileri kar gibi beyaz, Basindaki saçlar yün gibi apakti. Tahti alev alev, Tekerlekleri kizgin ates gibiydi.
\par 10 Önünden atesten bir irmak çikip akiyordu. Binlerce binler O'na hizmet ediyordu; On binlerce on binler Önünde duruyordu. Mahkeme kuruldu, Kitaplar açildi.
\par 11 "Boynuzun söyledigi övüngen sözleri duyunca baktim, yaratik gözümün önünde öldürüldü, bedeni kizgin atese atildi, yok oldu.
\par 12 Öbür yaratiklara gelince, egemenlik onlardan alinmis, ancak belirli bir süre için yasamalarina izin verilmisti.
\par 13 "Gece görümlerimde insanogluna benzer birinin gögün bulutlariyla geldigini gördüm. Eskiden beri var Olan'in yanina dogru ilerledi, O'nun önüne getirildi.
\par 14 Ona egemenlik, yücelik ve krallik verildi. Bütün halklar, uluslar ve her dilden insan ona tapindi. Egemenligi hiç bitmeyecek sonsuz bir egemenlik, kralligi hiç yikilmayacak bir kralliktir."
\par 15 "Ben Daniel'e gelince, ruhum üzüntüyle sarsildi, gördügüm görümler beni ürküttü.
\par 16 Orada duranlardan birine yaklastim, bütün bunlarin gerçek anlamini açiklamasini istedim. "O da bana bunlarin ne anlama geldigini açikladi:
\par 17 'Bu dört büyük yaratik yeryüzünde ortaya çikacak dört kraldir.
\par 18 Ama Yüceler Yücesi'nin kutsallari kralligi alacak, sonsuza dek ellerinde tutacaklar. Evet, sonsuzlara dek.
\par 19 "Bundan sonra öbürlerinden farkli, çok korkunç, demirden disleri, tunçtan* tirnaklari olan, yiyip parçalayan, artakalani ayaklari altinda çigneyen dördüncü yaratigin ne anlama geldigini ögrenmek istedim.
\par 20 Bunun yanisira basindaki on boynuzdan sonra çikan öbür boynuzun ne oldugunu da ögrenmek istedim. Bu boynuzun önünden üç boynuz düsmüstü, sanki ötekilerden daha iriceydi. Gözleri ve böbürlenen bir agzi vardi.
\par 21 Ben baktigim sirada bu boynuz kutsallarla savasiyor ve onlari yeniyordu.
\par 22 Eskiden beri var Olan -Yüceler Yücesi- gelip kutsallarinin lehine yargi verene dek bu böyle sürdü. Kutsallarin kralligi alma zamani gelmisti.
\par 23 "Bana su açiklamayi yapti: 'Dördüncü yaratik yeryüzünde ortaya çikacak dördüncü kralliktir. Bütün öbür kralliklardan farkli olacak, bütün dünyayi yiyip bitirecek, çigneyip parçalayacak.
\par 24 On boynuz bu kralliktan çikacak on kraldir. Bunlardan sonra öncekilerden farkli bir baska kral ortaya çikip üç krali tahtlarindan indirecek.
\par 25 Yüceler Yücesi'ni kötüleyen sözler söyleyecek, O'nun kutsallarina baski yapacak. Belirlenen zamanlari, yasalari degistirmeyi amaçlayacak. Kutsallar üç buçuk yil için eline teslim edilecekler.
\par 26 "'Ama mahkeme kurulacak, onun egemenligine son verilecek, büsbütün yok edilecek.
\par 27 Göklerin altindaki kralliklara özgü krallik, egemenlik ve büyüklük kutsallara, Yüceler Yücesi'nin halkina verilecek. Bu halkin kralligi sonsuza dek sürecek, bütün uluslar ona kulluk edip sözünü dinleyecek.
\par 28 "Iste olayin gelisimi burada bitiyor. Ben Daniel'e gelince, düsüncelerim beni çok ürküttü, benzim soldu. Ama bu olayi içimde sakladim."

\chapter{8}

\par 1 Kral Belsassar'in kralliginin üçüncü yilinda, ben Daniel daha önce gördügüm görümden baska bir görüm gördüm.
\par 2 Görümde kendimi Elam Ili'ndeki Sus Kalesi'nde, Ulay Kanali'nin yaninda gördüm.
\par 3 Gözlerimi kaldirip bakinca kanal kiyisinda duran bir koç gördüm; iki uzun boynuzu vardi. Boynuzlardan daha geç çikani öbüründen daha uzundu.
\par 4 Koçun batiya, kuzeye, güneye dogru boynuz attigini gördüm. Hiçbir hayvan ona karsi koyamiyor, kimse onun elinden kurtaramiyordu. Koç diledigi gibi davrandi ve gitgide güçlendi.
\par 5 Ben bu olayi düsünürken, batidan ansizin gözleri arasinda çarpici bir boynuzu olan bir teke geldi. Yere basmadan bütün dünyayi asti.
\par 6 Güç ve öfkeyle, kanalin yaninda durdugunu gördügüm iki boynuzlu koça dogru kostu.
\par 7 Öfkeyle saldirdigini, koça vurup boynuzlarini kirdigini gördüm. Koçun tekeye karsi duracak gücü yoktu; teke koçu yere vurup çignedi. Koçu onun elinden kurtaracak kimse yoktu.
\par 8 Teke çok güçlendi, ama en güçlü oldugu sirada büyük boynuzu kirildi. Kirilan boynuzun yerine, gögün dört rüzgarina dogru çarpici dört boynuz çikti.
\par 9 Bu boynuzlarin birinden baska bir küçük boynuz çikti; güneye, doguya ve Güzel Ülke'ye dogru yayilarak çok güçlendi.
\par 10 Göklerin ordusuna erisinceye dek büyüdü. Gökteki ordudan ve yildizlardan bazilarini yeryüzüne düsürdü, ayaklari altina alip çignedi.
\par 11 Kendisini Gök Ordusu'nun Önderi'ne kadar yükseltti. Tanri'ya sunulan günlük sunu kaldirildi, tapinak terk edildi.
\par 12 Baskaldiri yüzünden günlük sunuya karsi çikildi. Gerçek ayak altinda çignendi. Küçük boynuz yaptigi her seyde basarili oldu.
\par 13 Sonra kutsal bir varligin konustugunu duydum. Baska kutsal bir varlik ona, "Bu görümde -günlük sunuyla, yikim getiren baskaldiriyla, kutsal yerin ve ordunun ayak altinda çignenmesiyle ilgili görümde- olanlar ne zamana dek sürecek?" diye sordu.
\par 14 Kutsal varlik bana, "2 300 aksam, sabah olacak, sonra kutsal yer yeniden düzene konulacak" dedi.
\par 15 Ben Daniel, gördügüm görümün ne anlama geldigini çözmeye çalisirken, insana benzer biri karsimda durdu.
\par 16 Bir insan sesinin Ulay Kanali'ndan, "Ey Cebrail, görümün ne anlama geldigini suna açikla" diye seslendigini duydum.
\par 17 Cebrail durdugum yere yaklasinca korkudan yere yigildim. Bana, "Ey insanoglu!" dedi, "Bu görümün sonla ilgili oldugunu anla."
\par 18 O benimle konusurken, yüzükoyun yere uzanmis, derin bir uykuya dalmisim. Dokunup beni ayaga kaldirdi.
\par 19 Bana, "Daha sonra Tanri'nin öfkesi sona erdiginde neler olacagini sana söyleyecegim" dedi, "Çünkü görüm sonun belirlenen zamaniyla ilgilidir.
\par 20 Gördügün iki boynuzlu koç Med ve Pers krallarini simgeler.
\par 21 Teke Grek Krali'dir; gözleri arasindaki büyük boynuz birinci kraldir.
\par 22 Kirilan boynuzun yerine çikan dört boynuz, ulusundan çikacak dört kralligi simgeliyor. Ama ilk kral kadar güçlü olmayacaklar.
\par 23 "Bu dört kralligin sonu yaklasip yapilan kötülükler doruga varinca, sert yüzlü ve aldatmada usta bir kral ortaya çikacak.
\par 24 Kendisinden gelmeyen büyük bir güce kavusacak. Sasirtici yikimlar yapacak, el attigi her iste basarili olacak. Güçlüleri ve kutsal halki yok edecek.
\par 25 Yapacagi isleri aldatarak basaracak, kendisini yükseltecek. Güvenlikte olan birçoklarini yok edecek, Önderler Önderi'ne karsi duracak. Ama kendisi insan eli degmeden yok edilecek.
\par 26 "Aksam ve sabahla ilgili sana bildirilen görüm gerçektir. Ama sen görümü gizli tut. Çünkü uzak bir gelecekle ilgilidir."
\par 27 Ben Daniel günlerce bitkin ve hasta kaldim. Sonra kalkip kralin islerini yapmayi sürdürdüm. Bu anlasilmasi güç görümden ötürü saskindim.

\chapter{9}

\par 1 Medli Ahasveros oglu Darius Kildan* Krali oldu. Kralliginin birinci yilinda ben Daniel, RAB'bin Peygamber Yeremya'ya bildirdigi sayinin - Yerusalim'in issiz kalacagi yillarin sayisinin - yetmis oldugunu Kutsal Yazilar'dan anladim.
\par 3 Bunun üzerine yüzümü Rab Tanri'ya çevirdim. Duayla, yakarisla, oruçla* O'na yalvardim; çul kusanip külde oturdum.
\par 4 RAB Tanrim'a dua edip günahlarimizi itiraf ettim. Söyle dedim:"Ya Rab, kendisini sevenlerle, buyruklarina uyanlarla yaptigi antlasmaya bagli kalan yüce ve görkemli Tanri!
\par 5 Buyruklarindan, ilkelerinden ayrilip günah, suç isledik, kötülük yaptik, baskaldirdik.
\par 6 Senin adina krallarimiza, önderlerimize, atalarimiza, ülkedeki bütün halka seslenen kullarin peygamberleri dinlemedik.
\par 7 "Sen adaletlisin, ya Rab! Sadakatsizligimiz yüzünden bizi uzak yakin ülkelere sürdün. Oralarda yasayan biz Yahudiler, Yerusalim halki, Israilliler bugün utanç içindeyiz.
\par 8 Evet, ya RAB, bizler, krallarimiz, önderlerimiz, atalarimiz sana karsi isledigimiz günah yüzünden utanç içindeyiz.
\par 9 Sana karsi geldigimiz halde, sen aciyan, bagislayan Tanrimiz Rab'sin.
\par 10 Tanrimiz RAB'bin sözüne kulak vermedik, kullari peygamberler araciligiyla bize verdigi yasalara uymadik.
\par 11 Bütün Israil halki yasani çignedi, sirtini sana dönüp seni dinlemek istemedi. "Bu yüzden Tanri kulu Musa'nin Yasasi'nda yazilan lanet basimiza yagdi, içilen ant yerine geldi. Çünkü sana karsi günah isledik.
\par 12 Üzerimize büyük yikim getirerek bizim ve bizi yöneten önderlerimiz için söyledigin sözleri yerine getirdin. Yerusalim'in basina gelen, gögün altindaki baska hiçbir kentin basina gelmemistir.
\par 13 Musa'nin Yasasi'nda yazildigi gibi, bütün bu yikimlar basimiza geldi. Buna karsin, ey Tanrimiz RAB, suçumuzdan dönüp senin gerçeklerine yönelerek lütfunu dilemedik.
\par 14 RAB üzerimize yikim göndermekten caymadi. Çünkü Tanrimiz RAB yaptigi her seyde adildir. Bizse O'nun sözüne kulak vermedik.
\par 15 "Ey Tanrimiz Rab, sen halkini Misir'dan güçlü elinle çikardin ve bugün oldugu gibi ün kazandin. Bizse günah isledik, kötülük yaptik.
\par 16 Ya Rab, dogru islerin uyarinca kentin Yerusalim'den, kutsal dagindan öfkeni, kizginligini kaldirmani dilerim. Günahlarimiz ve atalarimizin suçlari yüzünden Yerusalim de halkin da çevremizdekilerin tümüne alay konusu oldu.
\par 17 "Simdi, ey Tanrimiz, kulunun duasini, yakarisini isit. Adin ugruna, ya Rab, yüzünü viran tapinagina çevir.
\par 18 Ey Tanrim, kulak ver ve isit! Gözlerini aç, senin olan viran kenti gör. Dogrulugumuzdan degil, senin büyük merhametinden ötürü dilekte bulunuyoruz.
\par 19 Ya Rab, dinle! Ya Rab, bagisla! Isit ve davran, ya Rab! Ey Tanrim, adinin hatiri için gecikme! Çünkü kent ve halk senindir."
\par 20 Ben daha konusup dua ederken, günahimi ve halkim Israil'in günahini açikça kabul edip Tanrim'in kutsal dagi için Tanrim RAB'be dilekte bulunurken,
\par 21 daha dua ediyorken, önceden görümde gördügüm adam -Cebrail- aksam sunusu saatinde hizla uçarak yanima geldi.
\par 22 "Daniel, sana anlayis vermek için geldim" diye açikladi,
\par 23 "Sen Tanri'ya yalvarmaya baslar baslamaz, duan yanitlandi; bunu bildirmeye geldim. Çünkü sen çok sevilen birisin. Bu nedenle sözün anlamini kavra ve görümü anla:
\par 24 "Baskaldiriyi ortadan kaldirmak, günaha son vermek, suçu bagislatmak, sonsuza dek kalici dogrulugu saglamak, görüm ve peygamberligi mühürlemek, En Kutsal'i meshetmek* için senin halkina ve kutsal kentine* yetmis hafta kadar zaman saptanmistir.
\par 25 "Sunu bil ve anla: Yerusalim'i yeniden kurmak için buyrugun verilmesinden, meshedilmis* olan önderin gelisine dek yedi hafta geçecek. Altmis iki hafta içinde Yerusalim yeniden sokaklarla, hendeklerle kurulacak. Ancak bu sikintili zamanlarda olacak.
\par 26 Bu altmis iki hafta sonunda meshedilmis olan öldürülecek ve onu destekleyen olmayacak. Gelecek önderin halki, kenti ve kutsal yeri yerle bir edecek. Sonu tufanla olacak: Savas sona dek sürecek. Yikimlarin da olacagi kararlastirildi.
\par 27 Gelecek önder birçoklariyla bir haftalik saglam bir antlasma yapacak. Haftanin yarisi geçince, kurbani da sunuyu da kaldiracak. Kararlastirilan yikim basina gelinceye dek yok edici önder tapinagin üst bölümüne yikici igrenç seyler* yerlestirecek."

\chapter{10}

\par 1 Pers Krali Kores'in kralliginin üçüncü yilinda Beltesassar diye çagrilan Daniel'e bir giz açiklandi. Büyük bir savasla ilgili olan bu giz gerçekti. Daniel görümde kendisine açiklanan gizi anladi.
\par 2 O sirada ben Daniel üç haftadir yas tutuyordum.
\par 3 Üç hafta dolana dek agzima ne güzel bir yiyecek ya da et koydum, ne sarap içtim, ne de yag süründüm.
\par 4 Birinci ayin* yirmi dördüncü günü, Büyük Irmak'in, yani Dicle'nin kiyisindayken,
\par 5 gözlerimi kaldirip bakinca keten giysi giyinmis, beline Ufaz altinindan kemer kusanmis bir adam gördüm.
\par 6 Bedeni sari yakut gibiydi. Yüzü simsek gibi parliyordu. Gözleri alevli mesalelere benziyordu. Kollariyla bacaklari cilali tunç* gibi parliyor, sesi büyük bir kalabaligin çikardigi gürültüyü andiriyordu.
\par 7 Görümü yalniz ben Daniel gördüm. Yanimdakiler görmediler, ama dehsete düserek gizlenmek için kaçtilar.
\par 8 Böylece ben yalniz kaldim. Bu büyük görümü seyrederken gücüm tükendi, benzim büsbütün soldu, kendimi toparlayamadim.
\par 9 Sonra adamin sesini duyunca yüzüstü yere düsüp derin bir uykuya daldim.
\par 10 Derken bir el dokundu, titredim; beni dizlerimle ellerimin üzerine kaldirdi.
\par 11 Bana, "Ey Daniel, sen ki çok sevilen birisin!" dedi, "Ayaga kalk ve söyleyeceklerime iyi kulak ver. Çünkü sana gönderildim." O bunlari söyler söylemez titreyerek ayaga kalktim.
\par 12 "Korkma, ey Daniel!" diye devam etti, "Anlayisa erismeye ve kendini Tanrin'in önünde alçaltmaya karar verdigin gün duan isitildi. Iste bu yüzden geldim.
\par 13 Pers kralliginin önderi yirmi bir gün bana karsi durdu. Sonra bas önderlerden Mikail bana yardima geldi, çünkü orada, Pers krallarinin yaninda alikonulmustum.
\par 14 Son günlerde halkinin basina neler gelecegini sana açiklamak için geldim simdi, çünkü bu görüm gelecekle ilgilidir."
\par 15 O bunlari söyleyince, suskun suskun yere baktim.
\par 16 Derken insanogluna benzeyen biri dudaklarima dokundu. Ben de agzimi açip konusmaya basladim. Karsimda durana, "Ey efendim, bu görüm yüzünden aci çekiyorum, kendimi toparlayamiyorum" dedim,
\par 17 "Ben kulun nasil seninle konusayim? Gücüm tükendi, solugum kesildi."
\par 18 Insana benzeyen varlik yine dokunup beni güçlendirdi.
\par 19 "Ey çok sevilen adam, korkma!" dedi, "Esenlik olsun sana! Güçlü ol! Evet, güçlü ol!" O benimle konusunca güçlendim. "Konusmani sürdür, efendim, çünkü bana güç verdin" dedim.
\par 20 Bunun üzerine, "Sana neden geldigimi biliyor musun?" dedi, "Çok yakinda dönüp Pers önderiyle savasacagim. Ben gidince Grek önderi gelecek.
\par 21 Ama önce Gerçek Kitap'ta neler yazildigini sana bildirecegim. Onlara karsi önderiniz Mikail disinda bana yardim eden kimse yok.

\chapter{11}

\par 1 "Medli Darius'un kralliginin birinci yilinda Mikail'i destekleyip korumak için onun yaninda durdum."
\par 2 "Simdi sana gerçegi bildirecegim: Pers kralliginda üç kral daha ortaya çikacak. Ama dördüncü kral öbür üçünden daha zengin olacak. Zenginligi sayesinde elde edecegi güçle herkesi Grek ülkesine karsi kiskirtacak.
\par 3 Sonra güçlü bir kral çikacak. Büyük yetkiyle krallik edecek ve diledigi gibi davranacak.
\par 4 Ne var ki, o gücünün dorugundayken, kralligi darmadagin edilecek, gögün dört rüzgari gibi dört parçaya bölünecek. Krallik onun soyundan gelenlere geçmeyecek, yerine geçenlerin hiçbiri onun gibi egemenlik sürmeyecek. Kralligi yikilip baskalarina verilecek.
\par 5 "Güney Krali güçlenecek. Ancak komutanlarindan biri ondan daha çok güçlenecek ve kralligi büyük olacak.
\par 6 Birkaç yil sonra bu ikisi uzlasacak. Güney Krali yapilan uzlasmayi onaylamak için kizini Kuzey Krali'na es olarak verecek. Ama kiz gücünü koruyamayacak. Kralin ömrü de gücü de uzun sürmeyecek. Bu arada kizla babasi da, ona eslik edenlerle onu destekleyen de ele verilecek.
\par 7 "Babasinin yerine kizin ailesinden biri ortaya çikacak. Kuzey Krali'nin ordusuna saldirip kalesini alacak. Onlarla savasip yenecek.
\par 8 Onlarin ilahlarini, dökme putlarini, degerli altin ve gümüs kaplarini alip Misir'a götürecek. Kuzey Krali'ni birkaç yil rahat birakacak.
\par 9 Sonra Kuzey Krali gidip Güney Krali'nin ülkesine saldiracak, ardindan kendi ülkesine dönecek.
\par 10 Kuzey Krali'nin ogullari savasa hazirlanarak çok büyük bir ordu toplayacaklar. Ordu sel gibi tasacak, önüne geleni alip götürecek, gelip Güney Krali'nin kalesine dayanacak.
\par 11 "Güney Krali öfkeyle çikip Kuzey Krali'na karsi savasacak. Kuzey Krali büyük bir ordu topladigi halde, bu ordu Güney Krali'nin eline teslim edilecek.
\par 12 Bu büyük ordu yenilgiye ugrayinca Güney Krali gurura kapilacak. On binlerce insani öldürecek, ama zaferi uzun sürmeyecek.
\par 13 Çünkü Kuzey Krali öncekinden daha büyük bir ordu toplayacak ve birkaç yil sonra büyük, iyi donatilmis bir orduyla ülkeye dogru ilerleyecek.
\par 14 "Bu sirada birçoklari Güney Krali'na karsi çikacak. Senin halkindan bazi zorbalar da, görüm yerine gelsin diye ayaklanacak, ama yenilgiye ugrayacaklar.
\par 15 Sonra Kuzey Krali gelip toprak yigarak tepecikler yapacak ve surlu kenti ele geçirecek. Güney Krali'nin güçleri buna karsi duramayacak. En seçme askerlerinin bile karsi durmaya güçleri yetmeyecek.
\par 16 Kente saldiran Kuzey Krali diledigi gibi davranacak, kimse ona karsi duramayacak. Güzel Ülke'yi yönetecek, yikip yok etme yetkisi onun elinde olacak.
\par 17 Kralliginin bütün gücünü toplayip Güney Krali'nin üzerine yürümeyi amaçlayacak ve Güney Krali'yla bir antlasma yapacak. Ülkesini yerle bir etmek için kizini es olarak ona verecek. Ama tasarisi basarili olmayacak, ona yarar saglamayacak.
\par 18 Bundan sonra deniz kiyisindaki bölgelere yönelecek, birçoklarini ele geçirecek. Ne var ki, bir komutan onun saygisizliklarini sona erdirecek, saygisizliginin karsiligini verecek.
\par 19 Bunun üzerine Kuzey Krali kendi ülkesinin kalelerine yönelecek, ama tökezleyip düsecek. Bir daha da ortaya çikmayacak.
\par 20 "Yerine geçen kral, kralliginin yüceligi için zorla vergi toplayacak birini gönderecek. Ama birkaç gün içinde öfkesiz ve savassiz yok edilecek.
\par 21 "Yerine krallikla onurlandirilmamis degersiz biri geçecek. Halk güvenlik içindeyken, kurdugu düzenler sayesinde gelip kralligi ele geçirecek.
\par 22 Çok güçlü ordulari süpürüp yok edecek; antlasma önderi de yok edilecek.
\par 23 Onunla antlasma yaptiktan sonra hileye basvuracak. Az sayida insanla gittikçe güçlenecek.
\par 24 Beklenmedik bir anda ilin zengin bölgelerine saldirip babalarinin, atalarinin yapmadigi seyleri yapacak. Adamlarina yagma ve çapul mali, servetler dagitacak. Kalelere saldirmak için düzenler kuracak, ama bu uzun sürmeyecek.
\par 25 "Gücünü ve cesaretini toplayarak büyük bir orduyla Güney Krali'na karsi çikacak. Güney Krali da büyük ve çok güçlü bir orduyla savasacak. Ne var ki, kurulan düzenler yüzünden ona karsi duramayacak.
\par 26 Sofrasindan yiyenler Güney Krali'ni yikmaya çalisacaklar; ordusu dagilacak, birçoklari vurulup öldürülecek.
\par 27 Her iki kral da kötülük tasarlayacak. Ayni masada oturup birbirlerine yalan söyleyecekler. Ancak bu bir yarar saglamayacak. Çünkü son yine de belirlenen zamanda gelecek.
\par 28 Kuzey Krali büyük bir servetle ülkesine dönecek, ama amaci kutsal antlasmaya karsi gelmek olacak. Diledigini yaptiktan sonra ülkesine dönecek.
\par 29 "Belirlenen zamanda dönüp yine Güney'e saldiracak. Ancak bu kez sonuç öncekinden farkli olacak.
\par 30 Ona karsi koymak için Kittim'den gelen gemiler cesaretini kiracak. Geri dönecek ve öfkeyle kutsal antlasmaya karsi çikacak, kutsal antlasmayi birakanlari yine kayiracak.
\par 31 "Askerleri gidip tapinakla kaleyi kirletecek, günlük sunulari kaldirip yikici igrenç seyi* koyacaklar.
\par 32 Kuzey Krali antlasmayi bozanlari yaltaklanarak ayartacak, ama Tanrisi'ni taniyan halk var gücüyle ona karsi duracak.
\par 33 "Halkin arasindaki bilge kisiler birçoklarini egitecekler. Ama bir süre bu kisiler ya kiliçla öldürülecek, yakilacak, tutsak edilecek ya da mallarindan edilecekler.
\par 34 Yenilgiye ugrayinca biraz yardim görecekler. Içtenlikten uzak birçok kisi onlardan yana geçecek.
\par 35 Bilgelerden kimisi tökezleyecek; öyle ki, son gelinceye dek arinip temizlenebilsin, lekesiz duruma gelebilsinler. Çünkü son yine de belirlenen zamanda gelecek.
\par 36 "Kral diledigi gibi davranacak. Kendini bütün tanrilardan daha büyük, daha yüce gösterecek, tanrilarin Tanrisi'na karsi duyulmamis sözler söyleyecek. Tanri'nin öfkesi tamamlanincaya dek basarili olacak. Çünkü tasarlanan, yerine gelecektir.
\par 37 Kral hiçbir tanriya, atalarinin ilahlarina da kadinlarin baglandigina da ilgi göstermeyecek. Kendisini hepsinden üstün görecek.
\par 38 Bu ilahlarin yerine, kaleler ilahini yüceltecek. Atalarinin tanimadigi bu ilaha altin, gümüs, degerli taslar, pahali armaganlar sunup onu onurlandiracak.
\par 39 Bu yabanci ilahin yardimiyla en güçlü kalelere saldiracak; onu kabul edenleri alabildigine onurlandiracak, onlari birçoklarinin basina önder atayacak, ülkeyi ödül olarak onlar arasinda bölüstürecek.
\par 40 "Son gelince, Güney Krali Kuzey Krali'yla savasa tutusacak. Kuzey Krali savas arabalariyla, atlilarla, birçok gemilerle saldiracak. Her seyi süpürüp götüren sel gibi tasarak birçok ülkeden geçecek.
\par 41 Güzel Ülke'ye de girecek, birçok ülke yenilgiye ugrayacak. Ancak Edom, Moav ve Ammon önderleri onun elinden kurtulacak.
\par 42 Öbür ülkelere de saldiracak. Misir bile elinden kurtulmayacak.
\par 43 Altin ve gümüs hazinelerine, Misir'in bütün degerli esyalarina el koyacak. Luvlular'la Kûslular* onun ardinca yürüyecekler.
\par 44 Ne var ki, dogudan ve kuzeyden gelen haberler onu ürkütecek. Birçoklarini yikip yok etmek için büyük öfkeyle yola çikacak.
\par 45 Denizle güzel kutsal dag arasinda saray çadirlarini kuracak. Yine de yasami son bulacak ve ona yardim eden olmayacak."

\chapter{12}

\par 1 "O zaman senin halkini koruyan büyük önder Mikail görünecek. Ulusun olusumundan beri hiç görülmemis bir sikinti dönemi olacak. Bu dönemde halkin -adi kitapta yazili olanlar- kurtulacak.
\par 2 Yeryüzü topraginda uyuyanlarin birçogu uyanacak: Kimisi sonsuz yasama, kimisi utanca ve sonsuz igrençlige gönderilecek.
\par 3 Bilgeler gökkubbe gibi, birçoklarini dogruluga döndürenler yildizlar gibi sonsuza dek parlayacaklar.
\par 4 Ama sen, ey Daniel, son gelinceye dek bu sözleri sakla, kitabi mühürle. Bilgileri artsin diye birçoklari oraya buraya gidecek."
\par 5 Ben Daniel baktim, biri irmagin bu kiyisinda, öbürü öbür kiyisinda duran baska iki varlik gördüm.
\par 6 Içlerinden biri, irmagin sulari üzerinde duran keten giysili adama, "Bu sasirtici olaylarin son bulmasi ne kadar zaman alacak?" diye sordu.
\par 7 Irmagin sulari üzerinde duran keten giysili adamin sag ve sol elini göge kaldirarak sonsuza dek Diri Olan'in adiyla ant içip, "Üç buçuk yil alacak" dedigini duydum, "Kutsal halkin gücü tümüyle kirilinca, bütün bu olaylar son bulacak."
\par 8 Adamin söylediklerini duydumsa da anlamadim. Bunun için, "Ey efendim, bunlarin sonu ne olacak?" diye sordum.
\par 9 Söyle yanitladi: "Sen git, Daniel. Bu sözler son gelinceye dek saklanip mühürlenecek.
\par 10 Birçoklari kendilerini aritip temizlenecek, lekesiz duruma gelecek, ama kötüler kötülük etmeyi sürdürecek. Kötülerin hiçbiri anlamayacak, bilgeler anlayacak.
\par 11 "Günlük sununun kaldirilip yikici igrenç seyin* kondugu zamandan baslayarak 1 290 gün geçecek.
\par 12 Bekleyip 1 335 güne ulasana ne mutlu!
\par 13 "Sana gelince, ey Daniel, son gelinceye dek yoluna devam et. Rahatina kavusacak ve günlerin sonunda ödülünü almak için uyanacaksin."


\end{document}