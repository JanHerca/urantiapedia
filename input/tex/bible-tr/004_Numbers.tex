\begin{document}

\title{Çölde Sayım}


\chapter{1}

\par 1 Israilliler'in Misir'dan çikisinin ikinci yili, ikinci ayin* birinci günü RAB Sina Çölü'nde, Bulusma Çadiri'nda Musa'ya söyle seslendi:
\par 2 "Sen ve Harun Israil toplulugunun bütün boylariyla ailelerinin sayimini yapin. Bütün erkekleri bir bir sayip adlarini yazin. Israilliler'den savasabilecek durumda yirmi ve daha yukari yastaki bütün erkekleri sayip bölüklere ayirin.
\par 4 Size yardim etmek için yaninizda her oymaktan birer adam bulunsun; bu kisiler aile basi olmali.
\par 5 Size yardimci olacak adamlarin adlari sunlardir: "Ruben oymagindan: Sedeur oglu Elisur,
\par 6 Simon oymagindan: Surisadday oglu Selumiel,
\par 7 Yahuda oymagindan: Amminadav oglu Nahson,
\par 8 Issakar oymagindan: Suar oglu Netanel,
\par 9 Zevulun oymagindan: Helon oglu Eliav,
\par 10 Yusufogullari'ndan Efrayim oymagindan: Ammihut oglu Elisama, Manasse oymagindan: Pedahsur oglu Gamliel,
\par 11 Benyamin oymagindan: Gidoni oglu Avidan,
\par 12 Dan oymagindan: Ammisadday oglu Ahiezer,
\par 13 Aser oymagindan: Okran oglu Pagiel,
\par 14 Gad oymagindan: Deuel oglu Elyasaf,
\par 15 Naftali oymagindan: Enan oglu Ahira."
\par 16 Bunlar Israil toplulugundan atanmis adamlardi; atalarinin soyundan gelen oymak önderleri, Israil'in boy baslariydi.
\par 17 Musa'yla Harun adlari bildirilen bu adamlari getirttiler.
\par 18 RAB'bin buyrugu uyarinca ikinci ayin birinci günü bütün halki topladilar. Yirmi ve daha yukari yastakileri boylarina, ailelerine göre birer birer sayip adlarini yazdilar. Böylece Musa Sina Çölü'nde halkin sayimini yapti.
\par 20 Israil'in ilk oglu Ruben'in soyundan olanlar: Savasabilecek durumda yirmi ve daha yukari yastaki bütün erkekler bagli olduklari boy ve aileye göre adlariyla birer birer kayda geçirildi.
\par 21 Ruben oymagindan sayilanlar 46 500 kisiydi.
\par 22 Simon'un soyundan olanlar: Savasabilecek durumda yirmi ve daha yukari yastaki bütün erkekler bagli olduklari boy ve aileye göre adlariyla birer birer belirlenip kayda geçirildi.
\par 23 Simon oymagindan sayilanlar 59 300 kisiydi.
\par 24 Gad'in soyundan olanlar: Savasabilecek durumda yirmi ve daha yukari yastakiler bagli olduklari boy ve aileye göre adlariyla kayda geçirildi.
\par 25 Gad oymagindan sayilanlar 45 650 kisiydi.
\par 26 Yahuda'nin soyundan olanlar: Savasabilecek durumda yirmi ve daha yukari yastakiler bagli olduklari boy ve aileye göre adlariyla kayda geçirildi.
\par 27 Yahuda oymagindan sayilanlar 74 600 kisiydi.
\par 28 Issakar'in soyundan olanlar: Savasabilecek durumda yirmi ve daha yukari yastakiler bagli olduklari boy ve aileye göre adlariyla kayda geçirildi.
\par 29 Issakar oymagindan sayilanlar 54 400 kisiydi.
\par 30 Zevulun'un soyundan olanlar: Savasabilecek durumda yirmi ve daha yukari yastakiler bagli olduklari boy ve aileye göre adlariyla kayda geçirildi.
\par 31 Zevulun oymagindan sayilanlar 57 400 kisiydi.
\par 32 Yusufogullari'ndan, Efrayim soyundan olanlar: Savasabilecek durumda yirmi ve daha yukari yastakiler bagli olduklari boy ve aileye göre adlariyla kayda geçirildi.
\par 33 Efrayim oymagindan sayilanlar 40 500 kisiydi.
\par 34 Manasse'nin soyundan olanlar: Savasabilecek durumda yirmi ve daha yukari yastakiler bagli olduklari boy ve aileye göre adlariyla kayda geçirildi.
\par 35 Manasse oymagindan sayilanlar 32 200 kisiydi.
\par 36 Benyamin'in soyundan olanlar: Savasabilecek durumda yirmi ve daha yukari yastakiler bagli olduklari boy ve aileye göre adlariyla kayda geçirildi.
\par 37 Benyamin oymagindan sayilanlar 35 400 kisiydi.
\par 38 Dan'in soyundan olanlar: Savasabilecek durumda yirmi ve daha yukari yastakiler bagli olduklari boy ve aileye göre adlariyla kayda geçirildi.
\par 39 Dan oymagindan sayilanlar 62 700 kisiydi.
\par 40 Aser'in soyundan olanlar: Savasabilecek durumda yirmi ve daha yukari yastakiler bagli olduklari boy ve aileye göre adlariyla kayda geçirildi.
\par 41 Aser oymagindan sayilanlar 41 500 kisiydi.
\par 42 Naftali'nin soyundan olanlar: Savasabilecek durumda yirmi ve daha yukari yastakiler bagli olduklari boy ve aileye göre adlariyla kayda geçirildi.
\par 43 Naftali oymagindan sayilanlar 53 400 kisiydi.
\par 44 Musa, Harun ve Israil'in on iki önderi tarafindan sayilanlar bunlardi. Her önder bagli oldugu aileyi temsil ediyordu.
\par 45 Israil'de savasabilecek durumda yirmi ve daha yukari yastakilerin tümü bagli olduklari aileye göre sayildilar.
\par 46 Sayilanlarin toplami 603 550 kisiydi.
\par 47 Ne var ki, Levi oymagindan olanlar öbürleriyle birlikte sayilmadi.
\par 48 Çünkü RAB Musa'ya söyle demisti:
\par 49 "Ancak Levi oymagini sayma, öbür Israilliler arasinda yaptigin sayima onlari katma.
\par 50 Levililer'i Levha Sandigi'nin bulundugu konuttan, esyalardan ve konuta ait her seyden sorumlu kil. Konutu ve bütün esyalarini onlar tasisin; konutun bakimini onlar yapsin, çevresinde ordugah kursun.
\par 51 Konut tasinirken onu Levililer toplayacak; konaklanacagi zaman da onlar kuracak. Levililer disinda konuta yaklasan ölüm cezasina çarptirilacak.
\par 52 Israilliler çadirlarini bölükler halinde kuracaklar. Herkes kendi ordugahinda, kendi sancaginin altinda bulunacak.
\par 53 Ancak Israil toplulugunun RAB'bin öfkesine ugramamasi için Levililer Levha Sandigi'nin bulundugu konutun çevresinde konaklayacak ve konuta bekçilik edecekler."
\par 54 Israilliler bütün bunlari tam tamina RAB'bin Musa'ya buyurdugu gibi yaptilar.

\chapter{2}

\par 1 RAB Musa'yla Harun'a, "Israilliler sancaklarinin altinda, aile bayraklariyla Bulusma Çadiri'ndan biraz ötede çepeçevre konaklasin" dedi.
\par 3 Doguda, gündogusunda konaklayan bölükler Yahuda ordugahinin sancagina bagli olacak. Yahudaogullari'nin önderi Amminadav oglu Nahson olacak.
\par 4 Bölügünün sayisi 74 600 kisiydi.
\par 5 Issakar oymagi onlarin bitisiginde konaklayacak. Issakarogullari'nin önderi Suar oglu Netanel olacak.
\par 6 Bölügünün sayisi 54 400 kisiydi.
\par 7 Sonra Zevulun oymagi konaklayacak. Zevulunogullari'nin önderi Helon oglu Eliav olacak.
\par 8 Bölügünün sayisi 57 400 kisiydi.
\par 9 Yahuda ordugahina ayrilan bölüklerdeki adam sayisi 186 400 kisiydi. Yola ilk çikacak olanlar bunlardi.
\par 10 Güneyde Ruben ordugahinin sancagi dikilecek, Ruben'e bagli bölükler orada konaklayacak. Rubenogullari'nin önderi Sedeur oglu Elisur olacak.
\par 11 Bölügünün sayisi 46 500 kisiydi.
\par 12 Simon oymagi onlarin bitisiginde konaklayacak. Simonogullari'nin önderi Surisadday oglu Selumiel olacak.
\par 13 Bölügünün sayisi 59 300 kisiydi.
\par 14 Sonra Gad oymagi konaklayacak. Gadogullari'nin önderi Deuel oglu Elyasaf olacak.
\par 15 Bölügünün sayisi 45 650 kisiydi.
\par 16 Ruben ordugahina ayrilan bölüklerdeki adam sayisi 151 450 kisiydi. Ikinci sirada yola çikacak olanlar bunlardi.
\par 17 Bulusma Çadiri ve Levililer'in ordugahi göç sirasinda öbür ordugahlarin ortasinda yola çikacak. Herkes konakladigi düzende kendi sancagi altinda göç edecek.
\par 18 Batida Efrayim ordugahinin sancagi dikilecek, Efrayim'e bagli bölükler orada konaklayacak. Efrayimogullari'nin önderi Ammihut oglu Elisama olacak.
\par 19 Bölügünün sayisi 40 500 kisiydi.
\par 20 Manasse oymagi onlara bitisik olacak. Manasseogullari'nin önderi Pedahsur oglu Gamliel olacak.
\par 21 Bölügünün sayisi 32 200 kisiydi.
\par 22 Sonra Benyamin oymagi konaklayacak. Benyaminogullari'nin önderi Gidoni oglu Avidan olacak.
\par 23 Bölügünün sayisi 35 400 kisiydi.
\par 24 Efrayim ordugahina ayrilan bölüklerdeki adam sayisi 108 100 kisiydi. Üçüncü olarak bunlar yola çikacak.
\par 25 Kuzeyde Dan ordugahinin sancagi dikilecek, Dan'a bagli bölükler orada konaklayacak. Danogullari'nin önderi Ammisadday oglu Ahiezer olacak.
\par 26 Bölügünün sayisi 62 700 kisiydi.
\par 27 Aser oymagi onlarin bitisiginde konaklayacak. Aserogullari'nin önderi Okran oglu Pagiel olacak.
\par 28 Bölügünün sayisi 41 500 kisiydi.
\par 29 Sonra Naftali oymagi konaklayacak. Naftaliogullari'nin önderi Enan oglu Ahira olacak.
\par 30 Bölügünün sayisi 53 400 kisiydi.
\par 31 Dan ordugahina ayrilan adamlarin sayisi 157 600 kisiydi. Kendi sancaklari altinda en son onlar yola çikacak.
\par 32 Ailelerine göre sayilan Israilliler bunlardi. Ordugahlardaki bütün bölüklerin toplami 603 550 kisiydi.
\par 33 RAB'bin Musa'ya verdigi buyruk uyarinca Levililer öbür Israilliler'le birlikte sayilmadi.
\par 34 Böylece Israilliler RAB'bin Musa'ya buyurdugu gibi yaptilar. Sancaklari altinda ordugah kurdular. Göç ederken de herkes boyu ve ailesiyle birlikte yola çikti.

\chapter{3}

\par 1 RAB Sina Dagi'nda Musa'ya seslendigi sirada Harun'la Musa'nin çocuklari sunlardi.
\par 2 Harun'un ogullari: Ilk oglu Nadav, Avihu, Elazar, Itamar.
\par 3 Harun'un kâhin atanip meshedilen* ogullari bunlardi.
\par 4 Nadav'la Avihu Sina Çölü'nde RAB'bin önünde kurallara aykiri bir ates sunarken öldüler. Ogullari yoktu. Elazar'la Itamar babalari Harun'un yaninda kâhinlik ettiler.
\par 5 RAB Musa'ya söyle dedi:
\par 6 "Kâhin Harun'a yardim etmek üzere Levi oymagini çagirip görevlendir.
\par 7 Bulusma Çadiri'nda Harun'la bütün topluluk adina konutla ilgili hizmeti yürütsünler.
\par 8 Bulusma Çadiri'ndaki bütün esyalardan sorumlu olacak, Israilliler adina konuta iliskin hizmeti yerine getirecekler.
\par 9 Levililer'i Harun'la ogullarinin hizmetine ver. Israilliler arasinda tümüyle onun hizmetine ayrilanlar onlardir.
\par 10 Kâhinlik görevini sürdürmek üzere Harun'la ogullarini görevlendir. Kutsal yere onlardan baska her kim yaklasirsa öldürülecektir."
\par 11 RAB Musa'ya söyle dedi:
\par 12 "Iste Israilli kadinlarin dogurdugu ilk erkek çocuklarin yerine Israilliler arasindan Levililer'i seçtim. Onlar benim olacaktir.
\par 13 Çünkü bütün ilk doganlar benimdir. Misir'da ilk doganlarin hepsini yok ettigim gün, Israil'de insan olsun hayvan olsun bütün ilk doganlari kendime ayirdim. Onlar benim olacak. Ben RAB'bim."
\par 14 RAB Musa'ya Sina Çölü'nde söyle dedi:
\par 15 "Leviogullari'ni ailelerine ve boylarina göre say. Bir aylik ve daha yukari yastaki her erkegi sayacaksin."
\par 16 Böylece Musa RAB'bin buyrugu uyarinca onlari saydi.
\par 17 Levi'nin ogullarinin adlari sunlardi: Gerson, Kehat, Merari.
\par 18 Gerson'un boy basi olan ogullari sunlardi: Livni, Simi.
\par 19 Kehat'in boy basi olan ogullari: Amram, Yishar, Hevron, Uzziel.
\par 20 Merari'nin boy basi olan ogullari: Mahli, Musi. Ailelerine göre Levi boylari bunlardi.
\par 21 Livni ve Simi boylari Gerson soyundandi. Gerson boylari bunlardi.
\par 22 Bir aylik ve daha yukari yastaki bütün erkeklerin sayisi 7 500'dü.
\par 23 Gerson boylari batida konutun arkasinda konaklayacakti.
\par 24 Gerson'a bagli ailelerin önderi Lael oglu Elyasaf'ti.
\par 25 Bulusma Çadiri'nda Gersonogullari konuttan, çadirin örtüsünden, Bulusma Çadiri'nin girisindeki perdeden,
\par 26 konutla sunagi çevreleyen avlunun perdelerinden, avlunun girisindeki perdeyle iplerden ve bunlarin kullanimindan sorumlu olacakti.
\par 27 Amram, Yishar, Hevron ve Uzziel boylari Kehat soyundandi. Kehat boylari bunlardi.
\par 28 Bir aylik ve daha yukari yastaki bütün erkeklerin sayisi 8 600'dü. Bunlar kutsal yerden sorumluydular.
\par 29 Kehat boylari konutun güney yaninda konaklayacakti.
\par 30 Kehat boylarina bagli ailelerin önderi Uzziel oglu Elisafan'di.
\par 31 Kehatlilar Antlasma Sandigi'ndan*, masayla kandillikten, sunaklardan, kutsal yerin esyalarindan, perdeden ve bunlarin kullanimindan sorumlu olacakti.
\par 32 Levili önderlerin basi Kâhin Harun'un oglu Elazar olacak, kutsal yerden sorumlu olanlari o yönetecekti.
\par 33 Mahli ve Musi boylari Merari soyundandi. Merari boylari bunlardi.
\par 34 Bir aylik ve daha yukari yastaki bütün erkeklerin sayisi 6 200'dü.
\par 35 Merari boylarina bagli ailelerin önderi Avihayil oglu Suriel'di. Merari boylari konutun kuzey yaninda konaklayacakti.
\par 36 Merariogullari konutun çerçevelerinden, kirislerinden, direklerinden, tabanlarindan, takimlarindan ve bunlarin kullanimindan,
\par 37 konutu çevreleyen avlunun direkleriyle tabanlarindan, kaziklariyla iplerinden sorumlu olacakti.
\par 38 Musa, Harun ve ogullari konutun önünde, gündogusu yönünde Bulusma Çadiri'nin önünde konaklayacaklardi. Israilliler adina kutsal yerin bakimindan sorumlu olacaklardi. Kutsal yere baska her kim yaklasirsa öldürülecekti.
\par 39 RAB'bin buyrugu uyarinca Musa'yla Harun'un Levili boylardan saydiklari bir aylik ve daha yukari yastaki bütün erkeklerin sayisi 22 000'di.
\par 40 Sonra RAB Musa'ya, "Israilliler'in bir aylik ve daha yukari yastaki ilk dogan bütün erkeklerini say, adlarini yaz" dedi,
\par 41 "Israilliler'in bütün ilk doganlarinin yerine Levililer'i, yine Israilliler'in ilk dogan hayvanlarinin yerine Levililer'in hayvanlarini bana ayir. Ben RAB'bim."
\par 42 Böylece Musa, RAB'bin buyurdugu gibi, Israilliler arasinda ilk doganlarin tümünü saydi.
\par 43 Adlariyla yazilan, ilk dogan bir aylik ve daha yukari yastaki erkeklerin sayisi 22 273'tü.
\par 44 RAB Musa'ya söyle dedi:
\par 45 "Israilliler'in ilk doganlari yerine Levililer'i, hayvanlari yerine de Levililer'in hayvanlarini ayir. Levililer benim olacak. Ben RAB'bim.
\par 46 Israilliler'in ilk doganlarindan olup Levililer'in sayisini asan 273 kisi için bedel alacaksin.
\par 47 Adam basina beser sekel al. Kutsal yerin yirmi geradan olusan sekelini ölçü tut.
\par 48 Levililer'in sayisini asanlardan bedel olarak alinan parayi Harun'la ogullarina ver."
\par 49 Böylece Musa, bedeli Levililer olanlarin sayisini asan Israilliler'den bedel parasini aldi.
\par 50 Israilliler'in ilk doganlarindan 1 365 kutsal yerin sekeli aldi.
\par 51 RAB'bin sözü uyarinca, RAB'bin kendisine buyurdugu gibi, bedel parasini Harun'la ogullarina verdi.

\chapter{4}

\par 1 RAB Musa'yla Harun'a, "Levi oymaginda Kehatogullari'na bagli boy ve aileler arasinda sayim yapin" dedi,
\par 3 "Bulusma Çadiri'nda hizmet etmeye gelen otuz ile elli yas arasindaki adamlarin hepsini sayin.
\par 4 "Kehatogullari'nin Bulusma Çadiri'ndaki görevi sudur: En kutsal esyalari tasimak.
\par 5 Ordugah tasinacagi zaman Harun'la ogullari gelip bölme perdesini indirecekler ve Levha Sandigi'ni bununla örtecekler.
\par 6 Sonra üzerine deri bir örtü geçirecek, üstüne de salt lacivert bir bez serecek, siriklarini yerine koyacaklar.
\par 7 "Kutsal masanin üzerine lacivert bir bez serip üzerine tabaklari, sahanlari, taslari, dökmelik sunu testilerini koyacaklar. Masada sürekli ekmek bulunmasini saglayacaklar.
\par 8 Bunlarin üzerine kirmizi bir bez serip deri*fg* bir örtüyle örtecek, siriklarini yerine koyacaklar.
\par 9 "Isik veren kandilligi, kandillerini, fitil masalarini, tablalarini ve zeytinyagi için kullanilan kaplarini lacivert bir bezle örtecekler.
\par 10 Kandillikle takimlarini deri*fg* bir örtüye sarip sedyenin üzerine koyacaklar.
\par 11 "Altin sunagin üzerine lacivert bir bez serip üzerine deri*fg* bir örtü örtecek, siriklarini yerine koyacaklar.
\par 12 Kutsal yerde kullanilan bütün esyalari lacivert bir beze sarip deri*fg* bir örtüyle örtecek, sedyenin üzerine koyacaklar.
\par 13 Sunaktan yagi, külü kaldiracak, sunagi mor bir bezle örtecekler.
\par 14 Sonra üzerine hizmet için kullanilan bütün takimlari, ates kaplarini, büyük çatallari, kürekleri, çanaklari yerlestirip deri*fg* bir örtüyle örtecek, siriklarini yerine koyacaklar.
\par 15 "Ordugah baska yere tasinirken Harun'la ogullari kutsal yere ait bütün esyalari ve takimlari örtmeyi bitirdikten sonra, Kehatogullari onlari tasimaya gelecekler. Ölmemek için kutsal esyalara dokunmayacaklar. Bulusma Çadiri'ndaki bu esyalarin tasinmasi Kehatogullari'nin sorumlulugu altindadir.
\par 16 "Kâhin Harun oglu Elazar isik için zeytinyagindan, güzel kokulu buhurdan, günlük tahil sunusundan* ve mesh yagindan -konuttan ve içindeki her seyden- kutsal yerle esyalarindan sorumlu olacaktir."
\par 17 RAB Musa'yla Harun'a söyle dedi:
\par 18 "Kehat boylarinin Levililer'in arasindan yok olmasina yol açmayin.
\par 19 En kutsal esyalara yaklasinca ölmemeleri için söyle yapin: Harun'la ogullari kutsal yere girecek, her adami görecegi ise atayip ne tasiyacagini bildirecek.
\par 20 Ancak Kehatogullari içeri girip bir an bile kutsal esyalara bakmamali, yoksa ölürler."
\par 21 RAB Musa'ya söyle dedi:
\par 22 "Gersonogullari'ni da boylarina, ailelerine göre say.
\par 23 Bulusma Çadiri'ndaki islerde çalisabilecek otuz ile elli yas arasindaki bütün erkekleri say.
\par 24 "Hizmet etmek ve yük tasimak konusunda Gerson boylarinin sorumlulugu sudur:
\par 25 Konutun perdelerini, Bulusma Çadiri'ni ve örtüsünü, üzerindeki deri örtüyü, Bulusma Çadiri'nin girisindeki perdeyi, konutla sunagi çevreleyen avlunun perdelerini, girisindeki perdeyi, ipleri ve bu amaçla kullanilan bütün esyalari tasiyacaklar. Bu konuda gereken her seyi Gersonogullari yapacak.
\par 27 Yapacaklari bütün hizmetler -yük tasima ya da baska bir is- Harun'la ogullarinin yönetimi altinda yapilacaktir. Tasiyacaklari yükten Gersonogullari'ni sorumlu tutacaksiniz.
\par 28 Gerson boylarinin Bulusma Çadiri'yla ilgili görevi budur. Kâhin Harun oglu Itamar'in yönetimi altinda hizmet edecekler."
\par 29 "Boylarina ve ailelerine göre Merariogullari'ni say.
\par 30 Bulusma Çadiri'ndaki islerde çalisabilecek otuz ile elli yas arasindaki bütün erkekleri say.
\par 31 Bulusma Çadiri'nda hizmet ederken sunlari yapacaklar: Konutun çerçevelerini, kirislerini, direklerini, tabanlarini,
\par 32 çadiri çevreleyen avlunun direkleriyle tabanlarini, kaziklarini, iplerini ve bunlarin kullanimiyla ilgili bütün esyalari tasiyacaklar. Herkesi yapacagi belirli ise ata.
\par 33 Merari boylarinin Bulusma Çadiri'yla ilgili görevi budur. Kâhin Harun oglu Itamar'in yönetimi altinda hizmet edecekler."
\par 34 Musa, Harun ve toplulugun önderleri, boylarina ve ailelerine göre Kehatogullari'ni, Bulusma Çadiri'ndaki islerde çalisabilecek otuz ile elli yas arasindaki bütün erkekleri saydilar.
\par 36 Boylarina göre sayilanlar 2 750 kisiydi.
\par 37 Bulusma Çadiri'nda hizmet gören Kehat boylarindan sayilanlar bunlardi. RAB'bin Musa'ya verdigi buyruk uyarinca Musa'yla Harun onlari saydilar.
\par 38 Gersonogullari'ndan boylarina ve ailelerine göre Bulusma Çadiri'ndaki islerde çalisabilecek otuz ile elli yas arasindaki bütün erkekleri saydilar.
\par 40 Boylarina ve ailelerine göre sayilanlar 2 630 kisiydi.
\par 41 Bulusma Çadiri'nda hizmet gören Gerson boylarindan sayilanlar bunlardi. RAB'bin buyrugu uyarinca Musa'yla Harun onlari saydilar.
\par 42 Merari boylarina ve ailelerine göre Bulusma Çadiri'ndaki islerde çalisabilecek otuz ile elli yas arasindaki bütün erkekleri saydilar.
\par 44 Boylarina göre sayilanlar 3 200 kisiydi.
\par 45 Bulusma Çadiri'nda hizmet gören Merari boylarindan sayilanlar bunlardi. RAB'bin Musa'ya verdigi buyruk uyarinca Musa'yla Harun onlari saydilar.
\par 46 Böylece Musa, Harun ve Israil önderleri, boylarina ve ailelerine göre Levililer'i, Bulusma Çadiri'nda hizmet gören ve yük tasima isini yapan otuz ile elli yas arasindaki bütün erkekleri saydilar.
\par 48 Sayilanlar 8 580 kisiydi.
\par 49 RAB'bin Musa'ya verdigi buyruk uyarinca, herkes yapacagi hizmete ve tasiyacagi yüke göre atandi. Böylece RAB'bin Musa'ya verdigi buyruk uyarinca yazildilar.

\chapter{5}

\par 1 RAB Musa'ya söyle dedi:
\par 2 "Israil halkina de ki, deri hastaligi veya akintisi olan ya da ölüye dokundugundan kirli sayilan herkesi ordugahin disina çikarsinlar.
\par 3 Erkegi de kadini da çikaracaksiniz. Aralarinda yasadigim ordugahlarini kirletmemeleri için onlari ordugahin disina çikaracaksiniz."
\par 4 Israil halki denileni yaparak RAB'bin Musa'ya buyurdugu gibi onlari ordugahin disina çikardi.
\par 5 RAB Musa'ya söyle dedi:
\par 6 "Israil halkina de ki, `Bir erkek ya da kadin, insanin isleyebilecegi günahlardan birini isler, RAB'be ihanet ederse o kisi suçlu sayilir.
\par 7 Isledigi günahi itiraf etmeli. Karsiligini, beste birini üzerine ekleyerek suç isledigi kisiye ödeyecek.
\par 8 Eger kisinin islenen suçun karsiligini alacak bir yakini yoksa, suçun karsiligi RAB'be ait olacak. Günahin bagislanmasi için sunulan bagislamalik koçla birlikte suçun karsiligi da kâhine verilecek.
\par 9 Israil halkinin kâhine sundugu kutsal armaganlarin bagis kisimlari kâhinin olacak.
\par 10 Herkesin kendine ayirdigi sunular kendinin, ama kâhine verdikleri kâhinin olacaktir."
\par 11 RAB Musa'ya söyle dedi:
\par 12 "Israil halkina de ki, `Eger bir adamin karisi yoldan çikar, ona ihanet eder,
\par 13 baska bir adamla yatar, kirlendigi halde bu olayi kocasindan gizlerse ve tanik olmadigi için kadinin yaptigi ortaya çikmazsa,
\par 14 koca karisini kiskanir, ona karsi yüreginde kusku uyanirsa, kadin suçluysa ya da suçlu olmadigi halde kocasi onu kiskanir, ona karsi yüreginde kusku uyanirsa,
\par 15 adam karisini kâhine götürecek. Karisi için sunu olarak onda bir efa arpa unu alacak. Üzerine zeytinyagi dökmeyecek, günnük koymayacak. Çünkü bu kiskançlik sunusudur. Suçu animsatan animsatma sunusudur.
\par 16 "`Kâhin kadini öne çagirip RAB'bin önünde durmasini saglayacak.
\par 17 Sonra, toprak bir kabin içine kutsal su koyacak. Konutun kurulu oldugu yerden biraz toprak alip suya katacak.
\par 18 Kadini RAB'bin önünde durdurduktan sonra onun saçini açacak, animsatma sunusu, yani kiskançlik sunusunu eline verecek. Kendisi de lanet getiren aci suyu elinde tutacak.
\par 19 Sonra kadina ant içirtip söyle diyecek: Eger baska bir adam seninle yatmadiysa, kocanla evliyken yoldan çikip günah islemediysen, lanet getiren bu aci su sana zarar vermesin.
\par 20 Ama kocanla evliyken yoldan çikip baska biriyle yatarak günah islediysen
\par 21 -kâhin kadina lanet andi içirtip söyle diyecek- RAB sana eriyen kalça, sisen karin versin. RAB halkin arasinda seni lanetli ve igrenç duruma düsürsün.
\par 22 Lanet getiren bu su karnina girince karnini sisirsin, kalçani eritsin. "'O zaman kadin, Amin, amin, diyecek.
\par 23 "'Kâhin bu lanetleri bir kitaba yazip aci suda yikayacak.
\par 24 Lanet getiren aci suyu kadina içirecek. Su kadinin içine girince acilik verecek.
\par 25 Kâhin kadinin elinden kiskançlik sunusunu alacak, RAB'bin huzurunda salladiktan sonra sunaga getirecek.
\par 26 Kadinin anma payi olarak sunudan bir avuç alip sunakta yakacak. Sonra kadina suyu içirecek.
\par 27 Eger kadin kocasina ihanet etmis, kendini kirletmisse, lanet getiren suyu içince aci duyacak; karni sisip kalçasi eriyecek. Halki arasinda lanetli olacak.
\par 28 Ama kendini kirletmemisse, temizse, zarar görmeyecek, çocuk dogurabilecek.
\par 29 "'Kiskançlik yasasi budur. Bir kadin yoldan çikar, kocasiyla evliyken kendini kirletirse,
\par 30 ya da bir koca karisini kiskanir, ona karsi yüreginde kusku uyanirsa, kâhin kadini RAB'bin önünde durduracak, bu yasayi ona uygulayacak.
\par 31 Kocasi herhangi bir suçtan suçsuz sayilacak, kadinsa suçunun cezasini çekecek."

\chapter{6}

\par 1 RAB Musa'ya söyle dedi:
\par 2 "Israil halkina de ki, 'Eger bir erkek ya da kadin RAB'be adanmis kisi* olarak RAB'be özel bir adak adamak, kendini RAB'be adamak isterse,
\par 3 saraptan ya da herhangi bir içkiden kaçinacak, saraptan ya da baska içkilerden yapilmis sirke içmeyecek. Üzüm suyu da içmeyecek. Yas ya da kuru üzüm yemeyecek.
\par 4 RAB'be adanmisligi süresince çekirdekten kabuguna dek asmanin ürününden hiçbir sey yemeyecek.
\par 5 "'RAB'be adanmisligi süresince basina ustura degmeyecek. Kendini RAB'be adadigi günler tamamlanincaya dek kutsal olacak, saçini uzatacak.
\par 6 Kendini RAB'be adadigi günler boyunca ölüye yaklasmayacak.
\par 7 Annesi, babasi, erkek ya da kiz kardesi ölse bile onlar için kendini kirletmeyecek. Çünkü kendini Tanrisi'na adama simgesi basi üzerindedir.
\par 8 Adanmisligi süresince RAB için kutsal olacaktir.
\par 9 "'Eger ansizin yaninda biri ölür, adamis oldugu basini kirletirse, temizlendigi gün, yedinci gün saçini tiras edecek.
\par 10 Sekizinci gün Bulusma Çadiri'nin giris bölümünde kâhine iki kumru ya da iki güvercin sunacak.
\par 11 Kâhin birini günah sunusu*, öbürünü yakmalik sunu* olarak sunacak. Böylece ölünün yaninda bulunmakla kirlenen adam arinacak. O gün basini yeniden adayacak.
\par 12 Adanmisligi süresince kendini RAB'be yeniden adayacak. Suç sunusu* olarak bir yasinda bir erkek kuzu getirecek. Önceki günler sayilmayacak, çünkü adanmisligi kirlenmis sayildi.
\par 13 "'Adanmis kisi için yasa sudur: Kendini adamis oldugu günler tamamlaninca, Bulusma Çadiri'nin giris bölümüne getirilecek.
\par 14 Orada RAB'be sunularini sunacak: Yakmalik sunu için bir yasinda kusursuz bir erkek kuzu, günah sunusu olarak bir yasinda kusursuz bir disi kuzu, esenlik sunusu* için kusursuz bir koç,
\par 15 tahil sunusu* ve dökmelik sunularla birlikte bir sepet mayasiz ekmek, ince undan zeytinyagiyla yogrulmus pideler, üzerine yag sürülmüs mayasiz yufkalar.
\par 16 "'Kâhin bunlari günah sunusu ve yakmalik sunu olarak RAB'bin önünde sunacak.
\par 17 Esenlik kurbani olarak da koçu mayasiz ekmek sepetiyle birlikte RAB'be sunacak. Ilgili tahil ve dökmelik sunulari da sunacak.
\par 18 "'Sonra adanmis kisi Bulusma Çadiri'nin giris bölümünde adadigi saçini tiras edecek, saçini alip esenlik kurbaninin altindaki atese koyacak.
\par 19 "'Adanmis kisi adadigi saçini tiras ettikten sonra, kâhin koçun haslanmis budunu, sepetten alacagi mayasiz pide ve mayasiz yufkayla birlikte adanmis kisinin eline koyacak.
\par 20 Sonra sallamalik sunu olarak RAB'bin huzurunda bunlari sallayacak. Bunlar sallanan dös ve bagis olarak sunulan butla birlikte kutsaldir, kâhin için ayrilacaktir. Bundan sonra adanmis kisi sarap içebilir.
\par 21 "'Adak adayan adanmis kisiyle ilgili yasa budur. RAB'be sunacagi sunu adagi uyarinca olacak. Elinden geldigince daha fazlasini verebilir. Adanmislikla ilgili yasa uyarinca adadigi adagi yerine getirmelidir."
\par 22 RAB Musa'ya söyle dedi:
\par 23 "Harun'la ogullarina de ki, 'Israil halkini söyle kutsayacaksiniz. Onlara diyeceksiniz ki,
\par 24 RAB sizi kutsasin Ve korusun;
\par 25 RAB aydin yüzünü size göstersin Ve size lütfetsin;
\par 26 RAB yüzünü size çevirsin Ve size esenlik versin.
\par 27 "Böylece kâhinler Israil halkini adimi anarak kutsayacaklar. Ben de onlari kutsayacagim."

\chapter{7}

\par 1 Musa konutu bitirdigi gün onu meshetti*. Onu ve içindeki bütün esyalari, sunagi ve bütün takimlarini da meshedip adadi.
\par 2 Sonra Israil ileri gelenleri, sayilanlardan sorumlu olan aile ve oymak önderleri armaganlar sundular. RAB'be armagan olarak üstü kapali alti araba ve on iki öküz getirdiler: Her iki önder için bir araba, her önder için bir öküz. Bu armaganlari konutun önüne getirdiler.
\par 4 Sonra RAB Musa'ya, "Bunlari Bulusma Çadiri'ndaki hizmetlerde kullanilmak üzere onlardan alip yapacaklari ise göre Levililer'e ver" dedi.
\par 6 Musa arabalari, öküzleri alip Levililer'e verdi.
\par 7 Yapacaklari ise göre Gersonogullari'na iki arabayla dört öküz,
\par 8 Merariogullari'na da dört arabayla sekiz öküz verdi. Bunlar Kâhin Harun oglu Itamar'in sorumlulugu altinda yapildi.
\par 9 Kehatogullari'na ise bir sey vermedi. Çünkü onlarin görevi kutsal esyalari omuzlarinda tasimakti.
\par 10 Meshedilen sunagin adanmasi için önderler armaganlarini sunagin önüne getirdiler.
\par 11 Çünkü RAB Musa'ya, "Sunagin adanmasi için her gün bir önder kendi armaganini sunacak" demisti.
\par 12 Birinci gün Yahuda oymagindan Amminadav oglu Nahson armaganlarini sundu.
\par 13 Getirdigi armaganlar sunlardi: 130 kutsal yerin sekeli agirliginda gümüs bir tabak, yetmis sekel agirliginda gümüs bir çanak -ikisi de tahil sunusu* için zeytinyagiyla yogrulmus ince un doluydu-
\par 14 buhur dolu on sekel agirliginda altin bir tabak;
\par 15 yakmalik sunu* için bir boga, bir koç, bir yasinda bir erkek kuzu;
\par 16 günah sunusu* için bir teke;
\par 17 esenlik kurbani için iki sigir, bes koç, bes teke, bir yasinda bes kuzu. Amminadav oglu Nahson'un getirdigi armaganlar bunlardi.
\par 18 Ikinci gün Issakar oymagi önderi Suar oglu Netanel armaganlarini sundu.
\par 19 Getirdigi armaganlar sunlardi: 130 kutsal yerin sekeli agirliginda gümüs bir tabak, yetmis sekel agirliginda gümüs bir çanak -ikisi de tahil sunusu için zeytinyagiyla yogrulmus ince un doluydu-
\par 20 buhur dolu on sekel agirliginda altin bir tabak;
\par 21 yakmalik sunu için bir boga, bir koç, bir yasinda bir erkek kuzu;
\par 22 günah sunusu için bir teke;
\par 23 esenlik kurbani için iki sigir, bes koç, bes teke, bir yasinda bes kuzu. Suar oglu Netanel'in getirdigi armaganlar bunlardi.
\par 24 Üçüncü gün Zevulun oymagi önderi Helon oglu Eliav armaganlarini sundu.
\par 25 Getirdigi armaganlar sunlardi: 130 kutsal yerin sekeli agirliginda gümüs bir tabak, yetmis sekel agirliginda gümüs bir çanak -ikisi de tahil sunusu için zeytinyagiyla yogrulmus ince un doluydu-
\par 26 buhur dolu on sekel agirliginda altin bir tabak;
\par 27 yakmalik sunu için bir boga, bir koç, bir yasinda bir erkek kuzu;
\par 28 günah sunusu için bir teke;
\par 29 esenlik kurbani için iki sigir, bes koç, bes teke, bir yasinda bes kuzu. Helon oglu Eliav'in getirdigi armaganlar bunlardi.
\par 30 Dördüncü gün Ruben oymagi önderi Sedeur oglu Elisur armaganlarini sundu.
\par 31 Getirdigi armaganlar sunlardi: 130 kutsal yerin sekeli agirliginda gümüs bir tabak, yetmis sekel agirliginda gümüs bir çanak -ikisi de tahil sunusu için zeytinyagiyla yogrulmus ince un doluydu-
\par 32 buhur dolu on sekel agirliginda altin bir tabak;
\par 33 yakmalik sunu için bir boga, bir koç, bir yasinda bir erkek kuzu;
\par 34 günah sunusu için bir teke;
\par 35 esenlik kurbani için iki sigir, bes koç, bes teke, bir yasinda bes kuzu. Sedeur oglu Elisur'un getirdigi armaganlar bunlardi.
\par 36 Besinci gün Simon oymagi önderi Surisadday oglu Selumiel armaganlarini sundu.
\par 37 Getirdigi armaganlar sunlardi: 130 kutsal yerin sekeli agirliginda gümüs bir tabak, yetmis sekel agirliginda gümüs bir çanak -ikisi de tahil sunusu için zeytinyagiyla yogrulmus ince un doluydu-
\par 38 buhur dolu on sekel agirliginda altin bir tabak;
\par 39 yakmalik sunu için bir boga, bir koç, bir yasinda bir erkek kuzu;
\par 40 günah sunusu için bir teke;
\par 41 esenlik kurbani için iki sigir, bes koç, bes teke, bir yasinda bes kuzu. Surisadday oglu Selumiel'in getirdigi armaganlar bunlardi.
\par 42 Altinci gün Gad oymagi önderi Deuel oglu Elyasaf armaganlarini sundu.
\par 43 Getirdigi armaganlar sunlardi: 130 kutsal yerin sekeli agirliginda gümüs bir tabak, yetmis sekel agirliginda gümüs bir çanak -ikisi de tahil sunusu için zeytinyagiyla yogrulmus ince un doluydu-
\par 44 buhur dolu on sekel agirliginda altin bir tabak;
\par 45 yakmalik sunu için bir boga, bir koç, bir yasinda bir erkek kuzu;
\par 46 günah sunusu için bir teke;
\par 47 esenlik kurbani için iki sigir, bes koç, bes teke, bir yasinda bes kuzu. Deuel oglu Elyasaf'in getirdigi armaganlar bunlardi.
\par 48 Yedinci gün Efrayim oymagi önderi Ammihut oglu Elisama armaganlarini sundu.
\par 49 Getirdigi armaganlar sunlardi: 130 kutsal yerin sekeli agirliginda gümüs bir tabak, yetmis sekel agirliginda gümüs bir çanak -ikisi de tahil sunusu için zeytinyagiyla yogrulmus ince un doluydu-
\par 50 buhur dolu on sekel agirliginda altin bir tabak;
\par 51 yakmalik sunu için bir boga, bir koç, bir yasinda bir erkek kuzu;
\par 52 günah sunusu için bir teke;
\par 53 esenlik kurbani için iki sigir, bes koç, bes teke, bir yasinda bes kuzu. Ammihut oglu Elisama'nin getirdigi armaganlar bunlardi.
\par 54 Sekizinci gün Manasse oymagi önderi Pedahsur oglu Gamliel armaganlarini sundu.
\par 55 Getirdigi armaganlar sunlardi: 130 kutsal yerin sekeli agirliginda gümüs bir tabak, yetmis sekel agirliginda gümüs bir çanak -ikisi de tahil sunusu için zeytinyagiyla yogrulmus ince un doluydu-
\par 56 buhur dolu on sekel agirliginda altin bir tabak;
\par 57 yakmalik sunu için bir boga, bir koç, bir yasinda bir erkek kuzu;
\par 58 günah sunusu için bir teke;
\par 59 esenlik kurbani için iki sigir, bes koç, bes teke, bir yasinda bes kuzu. Pedahsur oglu Gamliel'in getirdigi armaganlar bunlardi.
\par 60 Dokuzuncu gün Benyamin oymagi önderi Gidoni oglu Avidan armaganlarini sundu.
\par 61 Getirdigi armaganlar sunlardi: 130 kutsal yerin sekeli agirliginda gümüs bir tabak, yetmis sekel agirliginda gümüs bir çanak -ikisi de tahil sunusu için zeytinyagiyla yogrulmus ince un doluydu-
\par 62 buhur dolu on sekel agirliginda altin bir tabak;
\par 63 yakmalik sunu için bir boga, bir koç, bir yasinda bir erkek kuzu;
\par 64 günah sunusu için bir teke;
\par 65 esenlik kurbani için iki sigir, bes koç, bes teke, bir yasinda bes kuzu. Gidoni oglu Avidan'in getirdigi armaganlar bunlardi.
\par 66 Onuncu gün Dan oymagi önderi Ammisadday oglu Ahiezer armaganlarini sundu.
\par 67 Getirdigi armaganlar sunlardi: 130 kutsal yerin sekeli agirliginda gümüs bir tabak, yetmis sekel agirliginda gümüs bir çanak -ikisi de tahil sunusu için zeytinyagiyla yogrulmus ince un doluydu-
\par 68 buhur dolu on sekel agirliginda altin bir tabak;
\par 69 yakmalik sunu için bir boga, bir koç, bir yasinda bir erkek kuzu;
\par 70 günah sunusu için bir teke;
\par 71 esenlik kurbani için iki sigir, bes koç, bes teke, bir yasinda bes kuzu. Ammisadday oglu Ahiezer'in getirdigi armaganlar bunlardi.
\par 72 On birinci gün Aser oymagi önderi Okran oglu Pagiel armaganlarini sundu.
\par 73 Getirdigi armaganlar sunlardi: 130 kutsal yerin sekeli agirliginda gümüs bir tabak, yetmis sekel agirliginda gümüs bir çanak -ikisi de tahil sunusu için zeytinyagiyla yogrulmus ince un doluydu-
\par 74 buhur dolu on sekel agirliginda altin bir tabak;
\par 75 yakmalik sunu için bir boga, bir koç, bir yasinda bir erkek kuzu;
\par 76 günah sunusu için bir teke;
\par 77 esenlik kurbani için iki sigir, bes koç, bes teke, bir yasinda bes kuzu. Okran oglu Pagiel'in getirdigi armaganlar bunlardi.
\par 78 On ikinci gün Naftali oymagi önderi Enan oglu Ahira armaganlarini sundu.
\par 79 Getirdigi armaganlar sunlardi: 130 kutsal yerin sekeli agirliginda gümüs bir tabak, yetmis sekel agirliginda gümüs bir çanak -ikisi de tahil sunusu için zeytinyagiyla yogrulmus ince un doluydu-
\par 80 buhur dolu on sekel agirliginda altin bir tabak;
\par 81 yakmalik sunu için bir boga, bir koç, bir yasinda bir erkek kuzu;
\par 82 günah sunusu için bir teke;
\par 83 esenlik kurbani için iki sigir, bes koç, bes teke, bir yasinda bes kuzu. Enan oglu Ahira'nin getirdigi armaganlar bunlardi.
\par 84 Sunak meshedildiginde Israil önderlerinin adanmasi için sundugu armaganlar sunlardi: On iki gümüs tabak, on iki gümüs çanak, on iki altin tabak;
\par 85 her gümüs tabagin agirligi 130 sekel, her çanagin agirligi yetmis sekeldi. Bütün gümüs esyalarin toplam agirligi 2 400 kutsal yerin sekeliydi.
\par 86 Buhur dolu on iki altin tabagin her birinin agirligi on kutsal yerin sekeliydi. Bütün altin tabaklarin toplam agirligi 120 sekeldi.
\par 87 Yakmalik sunu için tahil sunulariyla birlikte sunulan hayvanlarin sayisi on iki boga, on iki koç, bir yasinda on iki erkek kuzuydu; günah sunusu için de on iki teke sunuldu.
\par 88 Esenlik kurbani için sunulan hayvanlarin sayisi yirmi dört sigir, altmis koç, altmis teke, bir yasinda altmis kuzuydu. Sunak meshedildikten sonra, adanmasi için verilen armaganlar bunlardi.
\par 89 Musa RAB'le konusmak için Bulusma Çadiri'na girince, Levha Sandigi'nin Bagislanma Kapagi'nin üstündeki iki Keruv* arasindan kendisine seslenen sesi duydu. RAB Musa'yla bu sekilde konustu.

\chapter{8}

\par 1 RAB Musa'ya söyle dedi:
\par 2 "Harun'a de ki, yedi kandili kandilligin önünü aydinlatacak biçimde yerlestirsin."
\par 3 Harun söyleneni yapti. RAB'bin Musa'ya buyurdugu gibi, kandilleri kandilligin önüne yerlestirdi.
\par 4 Kandillik, ayagindan çiçek motiflerine dek dövme altindan, RAB'bin Musa'ya gösterdigi örnege göre yapildi.
\par 5 RAB Musa'ya söyle dedi:
\par 6 "Levililer'i Israilliler'in arasindan ayirip dinsel açidan arindir.
\par 7 Onlari arindirmak için söyle yapacaksin: Günahtan arindirma suyunu üzerlerine serp; bedenlerindeki bütün killari tiras etmelerini, giysilerini yikamalarini sagla. Böylece arinmis olurlar.
\par 8 Sonra bir boga ile tahil sunusu* için zeytinyagiyla yogrulmus ince un alsinlar; günah sunusu* için sen de baska bir boga alacaksin.
\par 9 Levililer'i Bulusma Çadiri'nin önüne getir, bütün Israil toplulugunu da topla.
\par 10 Levililer'i RAB'bin huzuruna getireceksin, Israilliler ellerini üzerlerine koyacaklar.
\par 11 Harun, RAB'bin hizmetini yapabilmeleri için, Israilliler'in arasindan adak olarak Levililer'i RAB'be adayacak.
\par 12 "Levililer ellerini bogalarin basina koyacaklar; günahlarini bagislatmak için bogalardan birini günah sunusu, öbürünü yakmalik sunu* olarak RAB'be sunacaksin.
\par 13 Levililer Harun'la ogullarinin önünde duracaklar. Onlari adak olarak RAB'be adayacaksin.
\par 14 Levililer'i öbür Israilliler'in arasindan bu sekilde ayiracaksin. Levililer benim olacak.
\par 15 "Sen onlari arindirip adak olarak adadiktan sonra, Levililer Bulusma Çadiri'ndaki hizmeti yerine getirmeye baslayacaklar.
\par 16 Çünkü Israilliler arasindan Levililer tümüyle bana verilmistir. Ilk doganlarin, Israilli kadinlarin dogurduklari ilk erkek çocuklarin yerine onlari kendime ayirdim.
\par 17 Israilliler arasinda ilk dogan insan ya da hayvan benimdir. Misir'da ilk doganlari yok ettigim gün, onlari kendime ayirdim.
\par 18 Israil'de ilk dogan erkek çocuklarin yerine Levililer'i seçtim.
\par 19 Israilliler kutsal yere yaklastiklarinda belaya ugramamalari için, onlarin adina Bulusma Çadiri'ndaki hizmeti yerine getirmek ve günahlarini bagislatmak üzere, onlarin arasindan Levililer'i Harun'la ogullarina armagan olarak verdim."
\par 20 Musa, Harun ve bütün Israil toplulugu Levililer için söyleneni yaptilar. Israilliler RAB'bin Musa'ya Levililer'le ilgili verdigi her buyrugu yerine getirdiler.
\par 21 Levililer kendilerini günahtan arindirip giysilerini yikadilar. Sonra Harun onlari RAB'bin huzurunda adak olarak adadi; onlari arindirmak için günahlarini bagislatti.
\par 22 Bundan sonra Levililer Harun'la ogullarinin sorumlulugu altinda Bulusma Çadiri'ndaki hizmetlerini yapmaya geldiler. RAB'bin Levililer'e iliskin Musa'ya verdigi buyruklari yerine getirdiler.
\par 23 RAB Musa'ya söyle dedi:
\par 24 "Levililer'le ilgili kural sudur: Yirmi bes ve daha yukari yasta olanlar Bulusma Çadiri'nda hizmet edecekler.
\par 25 Ancak elli yasina gelince yaptiklari hizmetten ayrilip bir daha çadirda çalismayacaklar.
\par 26 Bulusma Çadiri'nda görev yapan kardeslerine yardimci olacaklar, ama kendileri hizmet etmeyecekler. Levililer'in sorumluluklarini böyle düzenleyeceksin."

\chapter{9}

\par 1 Israilliler'in Misir'dan çikislarinin ikinci yilinin birinci ayinda* RAB Sina Çölü'nde Musa'ya söyle seslendi:
\par 2 "Israilliler Fisih* kurbanini belirlenen zamanda kessinler.
\par 3 Bütün kurallar, ilkeler uyarinca kurbani belirlenen zamanda, bu ayin on dördüncü gününün aksamüstü keseceksiniz."
\par 4 Böylece Musa Israilliler'e Fisih kurbanini kesmelerini söyledi.
\par 5 Onlar da Sina Çölü'nde birinci ayin on dördüncü gününün aksamüstü Fisih kurbanini kestiler. Her seyi RAB'bin Musa'ya buyurdugu gibi yaptilar.
\par 6 Ancak, ölüye dokunduklarindan kirli sayilan bazi kisiler o gün Fisih kurbanini kesemediler. Ayni gün Musa'yla Harun'a gelip
\par 7 Musa'ya, "Ölüye dokundugumuzdan kirli sayiliriz" dediler, "Ama öbür Israilliler'le birlikte belirlenen zamanda RAB'bin sunusunu sunmamiz neden engellensin?"
\par 8 Musa, "RAB'bin sizinle ilgili bana neler söyleyecegini duyuncaya dek bekleyin" dedi.
\par 9 RAB Musa'ya söyle dedi:
\par 10 "Israilliler'e de ki, 'Sizlerden ya da soyunuzdan ölüye dokundugu için kirli sayilan ya da uzak bir yolculukta bulunan biri RAB'bin Fisih kurbanini kesebilir.
\par 11 Ikinci ayin on dördüncü gününün aksamüstü Fisih kurbanini kesip mayasiz ekmek ve aci otlarla yiyecek.
\par 12 Sabaha dek kurbandan bir sey birakmayacak, kemiklerini kirmayacak. Fisih kurbanini bütün kurallari uyarinca kesmelidir.
\par 13 Ancak, temiz sayilan ve yolculukta olmayan biri Fisih kurbanini kesmeyi savsaklarsa, halkinin arasindan atilacaktir. Çünkü belirlenen zamanda RAB'bin sunusunu sunmamistir. Günahinin cezasini çekecektir.
\par 14 "'Aranizda yasayan bir yabanci RAB'bin Fisih kurbanini kesmek isterse, Fisih'in kurallari, ilkeleri uyarinca kesmelidir. Yerli ya da yabanci için ayni kurali uygulamalisiniz."
\par 15 Konut, yani Levha Sandigi'nin bulundugu çadir kuruldugu gün üstünü bulut kapladi. Konutun üstündeki bulut aksamdan sabaha dek atesi andirdi.
\par 16 Bu hep böyle sürüp gitti. Konutu kaplayan bulut gece atesi andiriyordu.
\par 17 Israilliler ancak bulut çadirin üzerinden kalkinca göçer, bulut nerede durursa orada konaklarlardi.
\par 18 RAB'bin buyrugu uyarinca göç eder, yine RAB'bin buyrugu uyarinca konaklarlardi. Bulut konutun üzerinde durdukça yerlerinden ayrilmazlardi.
\par 19 Bulut konutun üzerinde uzun süre durdugu zaman RAB'bin buyruguna uyar, yola çikmazlardi.
\par 20 Bazen bulut konutun üzerinde birkaç gün kalirdi. Halk da RAB'bin verdigi buyruga göre ya konakladigi yerde kalir ya da göç ederdi.
\par 21 Bazi günler bulut aksamdan sabaha dek kalir, sabah konutun üzerinden kalkar kalkmaz halk yola çikardi. Gece olsun, gündüz olsun, bulut konutun üzerinden kalkar kalkmaz halk yola çikardi.
\par 22 Bulut konutun üzerinde iki gün, bir ay ya da uzun süre kalsa bile, Israilliler konakladiklari yerde kalir, yola koyulmazlardi. Ama bulut kalkar kalkmaz yola çikarlardi.
\par 23 RAB'bin buyrugu uyarinca konaklar ya da yola çikarlardi. Böylece RAB'bin Musa araciligiyla verdigi buyruga uydular.

\chapter{10}

\par 1 RAB Musa'ya söyle dedi:
\par 2 "Dövme gümüsten iki borazan yapacaksin; bunlari toplulugu çagirmak ve halkin yola çikmasi için kullanacaksin.
\par 3 Iki borazan birden çalininca, bütün topluluk senin yaninda, Bulusma Çadiri'nin girisi önünde toplanacak.
\par 4 Yalniz biri çalinirsa, önderler, Israil'in oymak baslari senin yaninda toplanacak.
\par 5 Borazan kisa çalininca, doguda konaklayanlar yola çikacak.
\par 6 Ikinci kez kisa çalininca da güneyde konaklayanlar yola çikacak. Borazanin kisa çalinmasi oymaklarin yola çikmasi için bir isarettir.
\par 7 Toplulugu toplamak için de borazan çaldirt, ama kisa olmasin.
\par 8 "Borazanlari kâhin olan Harunogullari çalacak. Borazan çalinmasi sizler ve gelecek kusaklar için kalici bir kural olacak.
\par 9 Sizi sikistiran düsmana karsi ülkenizde savasa çiktiginizda, borazan çalin. O zaman Tanriniz RAB sizi animsayacak, sizi düsmanlarinizdan kurtaracak.
\par 10 Sevinçli oldugunuz günler -kutladiginiz bayramlar ve Yeni Ay Törenleri'nde- yakmalik sunular* ve esenlik kurbanlari üzerine borazan çalacaksiniz. Böylelikle Tanriniz'in önünde animsanmis olacaksiniz. Ben Tanriniz RAB'bim."
\par 11 Ikinci yilin ikinci ayinin* yirminci günü bulut Levha Sandigi'nin bulundugu konutun üzerinden kalkti.
\par 12 Israilliler de Sina Çölü'nden göç etmeye basladilar. Bulut Paran Çölü'nde durdu.
\par 13 Bu, RAB'bin Musa araciligiyla verdigi buyruk uyarinca ilk göç edisleriydi.
\par 14 Önce Yahuda sancagi bölükleriyle yola çikti. Yahuda bölügüne Amminadav oglu Nahson komuta ediyordu.
\par 15 Issakar oymaginin bölügüne Suar oglu Netanel,
\par 16 Zevulun oymaginin bölügüne de Helon oglu Eliav komuta ediyordu.
\par 17 Konut yere indirilince, onu tasiyan Gersonogullari'yla Merariogullari yola koyuldular.
\par 18 Sonra Ruben sancagi bölükleriyle yola çikti. Ruben bölügüne Sedeur oglu Elisur komuta ediyordu.
\par 19 Simon oymaginin bölügüne Surisadday oglu Selumiel,
\par 20 Gad oymaginin bölügüne de Deuel oglu Elyasaf komuta ediyordu.
\par 21 Kehatlilar kutsal esyalari tasiyarak yola koyuldular. Bunlar varmadan konut kurulmus olurdu.
\par 22 Efrayim sancagi bölükleriyle yola çikti. Efrayim bölügüne Ammihut oglu Elisama komuta ediyordu.
\par 23 Manasse oymaginin bölügüne Pedahsur oglu Gamliel,
\par 24 Benyamin oymaginin bölügüne de Gidoni oglu Avidan komuta ediyordu.
\par 25 En sonunda Dan sancagi ordunun artçi kolu olan bölükleriyle yola çikti. Dan bölügüne Ammisadday oglu Ahiezer komuta ediyordu.
\par 26 Aser oymaginin bölügüne Okran oglu Pagiel,
\par 27 Naftali oymaginin bölügüne de Enan oglu Ahira komuta ediyordu.
\par 28 Yola koyulduklarinda Israilli bölüklerin yürüyüs düzeni böyleydi.
\par 29 Musa, kayinbabasi Midyanli Reuel oglu Hovav'a, "RAB'bin, 'Size verecegim dedigi yere gidiyoruz" dedi, "Bizimle gel, sana iyi davraniriz. Çünkü RAB Israil'e iyilik edecegine söz verdi."
\par 30 Hovav, "Gelmem" diye yanitladi, "Ülkeme, akrabalarimin yanina dönecegim."
\par 31 Musa, "Lütfen bizi birakma" diye üsteledi, "Çünkü çölde konaklayacagimiz yerleri sen biliyorsun. Sen bize göz olabilirsin.
\par 32 Bizimle gelirsen, RAB'bin yapacagi bütün iyilikleri seninle paylasiriz."
\par 33 RAB'bin Dagi'ndan ayrilip üç günlük yol aldilar. Konaklayacaklari yeri bulmalari için RAB'bin Antlasma Sandigi* üç gün boyunca önleri sira gitti.
\par 34 Konakladiklari yerden ayrildiklarinda da RAB'bin bulutu gündüzün onlarin üzerinde duruyordu.
\par 35 Sandik yola çikinca Musa, "Ya RAB, kalk! Düsmanlarin dagilsin, Senden nefret edenler önünden kaçsin!" diyordu.
\par 36 Sandik konaklayinca da, "Ya RAB, binlerce, on binlerce Israilli'ye dön!" diyordu.

\chapter{11}

\par 1 Halk çektigi sikintilardan ötürü yakinmaya basladi. RAB bunu duyunca öfkelendi, aralarina atesini göndererek ordugahin kenarlarini yakip yok etti.
\par 2 Halk Musa'ya yalvardi. Musa RAB'be yakarinca ates söndü.
\par 3 Bu nedenle oraya Tavera adi verildi. Çünkü RAB'bin gönderdigi ates onlarin arasinda yanmisti.
\par 4 Derken, halkin arasindaki yabancilar baska yiyeceklere özlem duymaya basladilar. Israilliler de yine aglayarak, "Keske yiyecek biraz et olsaydi!" dediler,
\par 5 "Misir'da parasiz yedigimiz baliklari, salataliklari, karpuzlari, pirasalari, soganlari, sarmisaklari animsiyoruz.
\par 6 Simdiyse yemek yeme istegimizi yitirdik. Bu mandan* baska hiçbir sey gördügümüz yok."
\par 7 Man kisnis tohumuna benzerdi, görünüsü de reçine gibiydi.
\par 8 Halk çikip onu toplar, degirmende ögütür ya da havanda döverdi. Çömlekte haslayip pide yaparlardi. Tadi zeytinyaginda pisirilmis yiyeceklere benzerdi.
\par 9 Gece ordugaha çiy düserken, man da birlikte düserdi.
\par 10 Musa herkesin, her ailenin çadirinin önünde agladigini duydu. RAB buna çok öfkelendi. Musa da üzüldü.
\par 11 RAB'be, "Kuluna neden kötü davrandin?" dedi, "Seni hosnut etmeyen ne yaptim ki, bu halkin yükünü bana yüklüyorsun?
\par 12 Bütün bu halka ben mi gebe kaldim? Onlari ben mi dogurdum? Öyleyse neden emzikteki çocugu tasiyan bir dadi gibi, atalarina ant içerek söz verdigin ülkeye onlari kucagimda tasimami istiyorsun?
\par 13 Bütün bu halka verecek eti nereden bulayim? Bana, 'Bize yiyecek et ver diye sizlanip duruyorlar.
\par 14 Bu halki tek basima tasiyamam, bunca yükü kaldiramam.
\par 15 Bana böyle davranacaksan -eger gözünde lütuf bulduysam- lütfen beni hemen öldür de kendi yikimimi görmeyeyim."
\par 16 RAB Musa'ya, "Halk arasinda önder ve yönetici bildigin Israil ileri gelenlerinden yetmis kisi topla" dedi, "Onlari Bulusma Çadiri'na getir, yaninda dursunlar.
\par 17 Ben inip seninle orada konusacagim. Senin üzerindeki Ruh'tan alip onlara verecegim. Halkin yükünü tek basina tasimaman için sana yardim edecekler.
\par 18 "Halka de ki, 'Yarin için kendinizi kutsayin, et yiyeceksiniz. Keske yiyecek biraz et olsaydi, Misir'da durumumuz iyiydi diye agladiginizi RAB duydu. Simdi yemeniz için size et verecek.
\par 19 Yalniz bir gün, iki gün, bes, on ya da yirmi gün degil,
\par 20 bir ay boyunca, burnunuzdan gelinceye dek, tiksinene dek yiyeceksiniz. Çünkü aranizda olan RAB'bi reddettiniz. O'nun önünde, Misir'dan neden çiktik diyerek agladiniz."
\par 21 Musa, "Aralarinda bulundugum halkin 600 000'i yetiskin erkektir" diye karsilik verdi, "Oysa sen, 'Bu halka bir ay boyunca yemesi için et verecegim diyorsun.
\par 22 Bütün davarlar, sigirlar kesilse, onlari doyurur mu? Denizdeki bütün baliklar tutulsa, onlari doyurur mu?"
\par 23 RAB, "Elim kisaldi mi?" diye yanitladi, "Sana söylediklerimin yerine gelip gelmeyecegini simdi göreceksin."
\par 24 Böylece Musa disari çikip RAB'bin kendisine söylediklerini halka bildirdi. Halkin ileri gelenlerinden yetmis adam toplayip çadirin çevresine yerlestirdi.
\par 25 Sonra RAB bulutun içinde inip Musa'yla konustu. Musa'nin üzerindeki Ruh'tan alip yetmis ileri gelene verdi. Ruh'u alinca peygamberlik ettilerse de, daha sonra hiç peygamberlik etmediler.
\par 26 Eldat ve Medat adinda iki kisi ordugahta kalmisti. Seçilen yetmis kisi arasindaydilar ama çadira gitmemislerdi. Ruh üzerlerine konunca ordugahta peygamberlik ettiler.
\par 27 Bir genç kosup Musa'ya, "Eldat'la Medat ordugahta peygamberlik ediyor" diye haber verdi.
\par 28 Gençliginden beri Musa'nin yardimcisi olan Nun oglu Yesu, "Ey efendim Musa, onlara engel ol!" dedi.
\par 29 Ama Musa, "Sen benim adima mi kiskaniyorsun?" diye yanitladi, "Keske RAB'bin bütün halki peygamber olsa da RAB üzerlerine Ruhu'nu gönderse!"
\par 30 Sonra Musa'yla Israil'in ileri gelenleri ordugaha döndüler. Rab Bildircin Gönderiyor
\par 31 RAB denizden bildircin getiren bir rüzgar gönderdi. Rüzgar bildircinlari ordugahin her yönünden bir günlük yol kadar uzakliga, yerden iki arsin yükseklige indirdi.
\par 32 Halk bütün gün, bütün gece ve ertesi gün durmadan bildircin topladi. Kimse on homerden az toplamadi. Bildircinlari ordugahin çevresine serdiler.
\par 33 Et daha halkin disleri arasindayken, çignemeye vakit kalmadan RAB öfkelendi, onlari büyük bir yikimla cezalandirdi.
\par 34 Bu nedenle oraya Kivrot-Hattaava adi verildi. Baska yiyeceklere özlem duyanlari oraya gömdüler.
\par 35 Halk Kivrot-Hattaava'dan Haserot'a göç edip orada kaldi.

\chapter{12}

\par 1 Musa Kûslu* bir kadinla evlenmisti. Bundan dolayi Miryam'la Harun onu yerdiler.
\par 2 "RAB yalniz Musa araciligiyla mi konustu?" dediler, "Bizim araciligimizla da konusmadi mi?" RAB bu yakinmalari duydu.
\par 3 Musa yeryüzünde yasayan herkesten daha alçakgönüllüydü.
\par 4 RAB ansizin Musa, Harun ve Miryam'a, "Üçünüz Bulusma Çadiri'na gelin" dedi. Üçü de gittiler.
\par 5 RAB bulut sütununun içinde indi. Çadirin kapisinda durup Harun'la Miryam'i çagirdi. Ikisi ilerlerken
\par 6 RAB onlara seslendi: "Sözlerime kulak verin: Eger aranizda bir peygamber varsa, Ben RAB görümde kendimi ona tanitir, Onunla düste konusurum.
\par 7 Ama kulum Musa öyle degildir. O bütün evimde sadiktir.
\par 8 Onunla bilmecelerle degil, Açikça, yüzyüze konusurum. O RAB'bin suretini görüyor. Öyleyse kulum Musa'yi yermekten korkmadiniz mi?"
\par 9 RAB onlara öfkelenip oradan gitti.
\par 10 Bulut çadirin üzerinden ayrildiginda Miryam deri hastaligina yakalanmis, kar gibi bembeyaz olmustu. Harun Miryam'a bakti, deri hastaligina yakalandigini gördü.
\par 11 Musa'ya, "Ey efendim, lütfen akilsizca isledigimiz günahtan ötürü bizi cezalandirma" dedi,
\par 12 "Miryam etinin yarisi yenmis olarak ana rahminden çikan ölü bir bebege benzemesin."
\par 13 Musa RAB'be, "Ey Tanri, lütfen Miryam'i iyilestir!" diye yakardi.
\par 14 RAB, "Babasi onun yüzüne tükürseydi, yedi gün utanç içinde kalmayacak miydi?" diye karsilik verdi, "Onu yedi gün ordugahtan uzaklastirin, sonra geri getirilsin."
\par 15 Böylece Miryam yedi gün ordugahtan uzaklastirildi, o geri getirilene dek halk yola çikmadi.
\par 16 Bundan sonra halk Haserot'tan ayrilip Paran Çölü'nde konakladi.

\chapter{13}

\par 1 RAB Musa'ya, "Israil halkina verecegim Kenan ülkesini arastirmak için bazi adamlar gönder" dedi, "Atalarin her oymagindan bir önder gönder."
\par 3 Musa RAB'bin buyrugu uyarinca Paran Çölü'nden adamlari gönderdi. Hepsi Israil halkinin önderlerindendi.
\par 4 Adlari söyleydi: Ruben oymagindan Zakkur oglu Sammua;
\par 5 Simon oymagindan Hori oglu Safat;
\par 6 Yahuda oymagindan Yefunne oglu Kalev;
\par 7 Issakar oymagindan Yusuf oglu Yigal;
\par 8 Efrayim oymagindan Nun oglu Hosea;
\par 9 Benyamin oymagindan Rafu oglu Palti;
\par 10 Zevulun oymagindan Sodi oglu Gaddiel;
\par 11 Yusuf oymagindan -Manasse oymagindan- Susi oglu Gaddi;
\par 12 Dan oymagindan Gemalli oglu Ammiel;
\par 13 Aser oymagindan Mikael oglu Setur;
\par 14 Naftali oymagindan Vofsi oglu Nahbi;
\par 15 Gad oymagindan Maki oglu Geuel.
\par 16 Ülkeyi arastirmak üzere Musa'nin gönderdigi adamlar bunlardi. Musa Nun oglu Hosea'ya Yesu adini verdi.
\par 17 Musa, Kenan ülkesini arastirmak üzere onlari gönderirken, "Negev'e, daglik bölgeye gidin" dedi,
\par 18 "Nasil bir ülke oldugunu, orada yasayan halkin güçlü mü zayif mi, çok mu az mi oldugunu ögrenin.
\par 19 Yasadiklari ülke iyi mi kötü mü, kentleri nasil, surlu mu degil mi anlayin.
\par 20 Toprak nasil? Verimli mi, kiraç mi? Çevre agaçlik mi, degil mi? Elinizden geleni yapip orada yetisen meyvelerden getirin."Mevsim üzümün olgunlasmaya basladigi zamandi.
\par 21 Böylece adamlar yola çikip ülkeyi Zin Çölü'nden Levo-Hamat'a dogru Rehov'a dek arastirdilar.
\par 22 Negev'den geçip Anakogullari'ndan Ahiman, Sesay ve Talmay'in yasadigi Hevron'a vardilar. -Hevron Misir'daki Soan Kenti'nden yedi yil önce kurulmustu.-
\par 23 Eskol Vadisi'ne varinca, üzerinde bir salkim üzüm olan bir asma dali kestiler. Adamlardan ikisi dali bir sirikta tasidilar. Yanlarina nar, incir de aldilar.
\par 24 Israilliler'in kestigi üzüm salkimindan dolayi oraya Eskol Vadisi adi verildi.
\par 25 Kirk gün dolastiktan sonra adamlar ülkeyi arastirmaktan döndüler.
\par 26 Paran Çölü'ndeki Kades'e, Musa'yla Harun'un ve Israil toplulugunun yanina geldiler. Onlara ve bütün topluluga gördüklerini anlatip ülkenin ürünlerini gösterdiler.
\par 27 Musa'ya, "Bizi gönderdigin ülkeye gittik" dediler, "Gerçekten süt ve bal akiyor orada! Iste ülkenin ürünleri!
\par 28 Ancak orada yasayan halk güçlü, kentler de surlu ve çok büyük. Orada Anak soyundan gelen insanlari bile gördük.
\par 29 Amalekliler Negev'de; Hititler*, Yevuslular ve Amorlular daglik bölgede; Kenanlilar da denizin yaninda ve Seria Irmagi'nin kiyisinda yasiyor."
\par 30 Kalev, Musa'nin önünde halki susturup, "Oraya gidip ülkeyi ele geçirelim. Kesinlikle buna yetecek gücümüz var" dedi.
\par 31 Ne var ki, kendisiyle oraya giden adamlar, "Bu halka saldiramayiz, onlar bizden daha güçlü" dediler.
\par 32 Arastirdiklari ülke hakkinda Israilliler arasinda kötü haber yayarak, "Boydan boya arastirdigimiz ülke, içinde yasayanlari yiyip bitiren bir ülkedir" dediler, "Üstelik orada gördügümüz herkes uzun boyluydu.
\par 33 Nefiller'i, Nefiller'in soyundan gelen Anaklilar'i gördük. Onlarin yaninda kendimizi çekirge gibi hissettik, onlara da öyle göründük."

\chapter{14}

\par 1 O gece bütün topluluk yüksek sesle bagrisip agladi.
\par 2 Bütün Israil halki Musa'yla Harun'a karsi söylenmeye basladi. Onlara, "Keske Misir'da ya da bu çölde ölseydik!" dediler,
\par 3 "RAB neden bizi bu ülkeye götürüyor? Kiliçtan geçirilelim diye mi? Karilarimiz, çocuklarimiz tutsak edilecek. Misir'a dönmek bizim için daha iyi degil mi?"
\par 4 Sonra birbirlerine, "Kendimize bir önder seçip Misir'a dönelim" dediler.
\par 5 Bunun üzerine Musa'yla Harun Israil toplulugunun önünde yüzüstü yere kapandilar.
\par 6 Ülkeyi arastiranlardan Nun oglu Yesu'yla Yefunne oglu Kalev giysilerini yirttilar.
\par 7 Sonra bütün Israil topluluguna söyle dediler: "Içinden geçip arastirdigimiz ülke çok iyi bir ülkedir.
\par 8 Eger RAB bizden hosnut kalirsa, süt ve bal akan o ülkeye bizi götürecek ve orayi bize verecektir.
\par 9 Ancak RAB'be karsi gelmeyin. Orada yasayan halktan korkmayin. Onlari ekmek yer gibi yiyip bitirecegiz. Koruyuculari onlari birakip gitti. Ama RAB bizimledir. Onlardan korkmayin!"
\par 10 Topluluk onlari tasa tutmayi düsünürken, ansizin RAB'bin görkemi Bulusma Çadiri'nda bütün Israil halkina göründü.
\par 11 RAB Musa'ya söyle dedi: "Ne zamana dek bu halk bana saygisizlik edecek? Onlara gösterdigim bunca belirtiye karsin, ne zamana dek bana iman etmeyecekler?
\par 12 Onlari salgin hastalikla cezalandiracagim, mirastan yoksun birakacagim. Ama seni onlardan daha büyük, daha güçlü bir ulus kilacagim."
\par 13 Musa, "Misirlilar bunu duyacak" diye karsilik verdi, "Çünkü bu halki gücünle onlarin arasindan sen çikardin.
\par 14 Kenan topraklarinda yasayan halka bunu anlatacaklar. Ya RAB, bu halkin arasinda oldugunu, onlarla yüz yüze görüstügünü, bulutunun onlarin üzerinde durdugunu, gündüz bulut sütunu, gece ates sütunu içinde onlara yol gösterdigini duymuslar.
\par 15 Eger bu halki bir insanmis gibi yok edersen, senin ününü duymus olan bu uluslar, 'RAB ant içerek söz verdigi ülkeye bu halki götüremedigi için onlari çölde yok etti diyecekler.
\par 17 "Simdi gücünü göster, ya Rab. Demistin ki,
\par 18 'RAB tez öfkelenmez, sevgisi engindir, suçu ve isyani bagislar. Ancak suçluyu cezasiz birakmaz; babalarin isledigi suçun hesabini üçüncü, dördüncü kusak çocuklarindan sorar.
\par 19 Misir'dan çikislarindan bugüne dek bu halki nasil bagisladiysan, büyük sevgin uyarinca onlarin suçunu bagisla."
\par 20 RAB, "Dilegin üzerine onlari bagisladim" diye yanitladi,
\par 21 "Ne var ki, varligim ve yeryüzünü dolduran yüceligim adina ant içerim ki,
\par 22 yüceligimi, Misir'da ve çölde gösterdigim belirtileri görüp de beni on kez sinayan, sözümü dinlemeyen bu kisilerden hiçbiri
\par 23 atalarina ant içerek söz verdigim ülkeyi görmeyecek. Beni küçümseyenlerden hiçbiri orayi görmeyecek.
\par 24 Ama kulum Kalev'de baska bir ruh var, o bütün yüregiyle ardimca yürüdü. Arastirmak için gittigi ülkeye onu götürecegim, onun soyu orayi miras alacak.
\par 25 Amalekliler'le Kenanlilar ovada yasiyorlar. Siz yarin geri dönün, Kizildeniz* yolundan çöle gidin."
\par 26 RAB Musa'yla Harun'a da, "Bu kötü topluluk ne zamana dek bana söylenecek?" dedi, "Bana söylenen Israil halkinin yakinmalarini duydum.
\par 28 Onlara RAB söyle diyor de: 'Varligim adina ant içerim ki, söylediklerinizin aynisini size yapacagim:
\par 29 Cesetleriniz bu çöle serilecek. Bana söylenen, yirmi ve daha yukari yasta sayilan herkes çölde ölecek.
\par 30 Sizi yerlestirecegime ant içtigim ülkeye Yefunne oglu Kalev'le Nun oglu Yesu'dan baskasi girmeyecek.
\par 31 Ama tutsak edilecek dediginiz çocuklarinizi oraya, sizin reddettiginiz ülkeye götürecegim; orayi taniyacaklar.
\par 32 Size gelince, cesetleriniz bu çöle serilecek.
\par 33 Çocuklariniz, hepiniz ölünceye dek kirk yil çölde çobanlik edecek ve sizin sadakatsizliginiz yüzünden sikinti çekecekler.
\par 34 Ülkeyi arastirdiginiz günler kadar -kirk gün, her gün için bir yildan kirk yil- suçunuzun cezasini çekeceksiniz. Sizden yüz çevirdigimi bileceksiniz!
\par 35 Ben RAB söyledim; bana karsi toplanan bu kötü topluluga bunlari gerçekten yapacagim. Bu çölde yikima ugrayacak, burada ölecekler."
\par 36 Musa'nin ülkeyi arastirmak üzere gönderdigi adamlar geri dönüp ülke hakkinda kötü haber yayarak bütün toplulugun RAB'be söylenmesine neden oldular.
\par 37 Ülke hakkinda kötü haber yayan bu adamlar RAB'bin önünde ölümcül hastaliktan öldüler.
\par 38 Ülkeyi arastirmak üzere gidenlerden yalniz Nun oglu Yesu'yla Yefunne oglu Kalev sag kaldi.
\par 39 Musa bu sözleri Israil halkina bildirince, halk yasa büründü.
\par 40 Sabah erkenden kalkip dagin tepesine çiktilar. "Günah isledik" dediler, "Ama RAB'bin söz verdigi yere çikmaya haziriz."
\par 41 Bunun üzerine Musa, "Neden RAB'bin buyruguna karsi geliyorsunuz?" dedi, "Bunu basaramazsiniz.
\par 42 Savasa gitmeyin, çünkü RAB sizinle olmayacak. Düsmanlarinizin önünde yenilgiye ugrayacaksiniz.
\par 43 Amalekliler'le Kenanlilar sizinle orada karsilasacak ve sizi kiliçtan geçirecekler. Çünkü RAB'bin ardinca gitmekten vazgeçtiniz. RAB de sizinle olmayacak."
\par 44 Öyleyken, kendilerine güvenerek daglik bölgenin tepesine çiktilar. RAB'bin Antlasma Sandigi* da Musa da ordugahta kaldi.
\par 45 Daglik bölgede yasayan Amalekliler'le Kenanlilar üzerlerine saldirdilar, Horma Kenti'ne dek onlari kovalayip bozguna ugrattilar.

\chapter{15}

\par 1 RAB Musa'ya söyle dedi:
\par 2 "Israil halkina de ki, 'Yerlesmek için size verecegim ülkeye girince,
\par 3 RAB'bi hosnut eden bir koku yapmak için yakmalik sunu*, özel adak kurbani, gönülden verilen sunu ya da bayram sunusu gibi yakilan sunu olarak RAB'be sigir ya da davar sunacaksiniz.
\par 4 Sunu sunan kisi RAB'be tahil sunusu* olarak dörtte bir hin zeytinyagiyla yogrulmus onda bir efa ince un sunacak.
\par 5 Yakmalik sunu ya da kurban için, her kuzuya dökmelik sunu olarak dörtte bir hin sarap hazirla.
\par 6 "'Koç sunarken tahil sunusu olarak üçte bir hin zeytinyagiyla yogrulmus onda iki efa ince un hazirla.
\par 7 Dökmelik sunu olarak da üçte bir hin*fv* sarap sun. Bunlari RAB'bi hosnut eden koku olarak sunacaksin.
\par 8 RAB'be yakmalik sunu, özel adak kurbani ya da esenlik sunusu* olarak bir boga sundugunda,
\par 9 bogayla birlikte tahil sunusu olarak yarim hin zeytinyagiyla yogrulmus onda üç efa ince un sun.
\par 10 Ayrica dökmelik sunu olarak yarim hin sarap sun. Yakilan bu sunu RAB'bi hosnut eden bir koku olacak.
\par 11 Sigir, koç, davar -kuzu ya da keçi- böyle hazirlanacak.
\par 12 Kaç hayvan sunacaksan her biri için ayni seyleri yapacaksin.
\par 13 "'Her Israil yerlisi RAB'bi hosnut eden koku olarak yakilan bir sunu sunarken bunlari aynen yapmalidir.
\par 14 Kusaklar boyunca aranizda yasayan bir yabanci ya da yerli olmayan bir konuk, RAB'bi hosnut eden koku olarak yakilan bir sunu sunarken, sizin uyguladiginiz kurallari uygulamalidir.
\par 15 Sizin ve aranizda yasayan yabancilar için topluluk ayni kurallari uygulamalidir. Kusaklar boyunca kalici bir kural olacak bu. RAB'bin önünde siz nasilsaniz, aranizda yasayan yabanci da ayni olacak.
\par 16 Size de aranizda yasayan yabanciya da ayni yasalar ve kurallar uygulanacak."
\par 17 RAB Musa'ya söyle dedi:
\par 18 "Israil halkina de ki, 'Sizi götürecegim ülkeye girip
\par 19 o ülkenin ekmeginden yediginizde, bir kismini bana sunacaksiniz.
\par 20 Ilk tahilinizdan sunu olarak bir pide sunacaksiniz; bunu harmaninizdan bir sunu olarak sunacaksiniz.
\par 21 Ilk tahilinizdan yapilmis bu sunuyu kusaklar boyunca RAB'be sunacaksiniz."
\par 22 "'Eger bilmeden günah islediyseniz, RAB'bin Musa'ya verdigi buyruklardan herhangi birini -RAB'bin buyruk verdigi günden baslayarak Musa araciligiyla size ve gelecek kusaklara buyurdugu herhangi bir seyi- yerine getirmediyseniz
\par 24 ve bu günah bilmeden islendiyse, bütün topluluk RAB'bi hosnut eden koku sunmak için yakmalik sunu olarak istenilen tahil ve dökmelik sunuyla birlikte bir boga, günah sunusu olarak da bir teke sunacaktir.
\par 25 Kâhin bütün Israil toplulugunun günahini bagislatacak, halk bagislanacak. Çünkü bilmeyerek günah islediler. Isledikleri günah yüzünden RAB için yakilan sunu olarak sunularini ve günah sunularini sundular.
\par 26 Bütün Israil toplulugu da aranizda yasayan yabancilar da bagislanacaktir. Çünkü halk bilmeyerek bu günahi isledi.
\par 27 "'Eger biri bilmeden günah islerse, günah sunusu olarak bir yasinda bir disi keçi getirmeli.
\par 28 Kâhin RAB'bin önünde, bilmeden günah isleyen kisinin günahini bagislatacak. Bagislatma yapilinca kisi bagislanacak.
\par 29 Bilmeden günah isleyen Israil yerlisi için de aranizda yasayan yabanci için de ayni yasayi uygulayacaksiniz.
\par 30 "'Yerli ya da yabanci biri bilerek günah islerse, RAB'be saygisizlik etmistir. Bu kisi halkinin arasindan atilmali.
\par 31 RAB'bin sözünü küçümsemis, buyruklarina karsi gelmistir. Bu nedenle o kisi halkinin arasindan kesinlikle atilacak, suçunun cezasini çekecektir."
\par 32 Israilliler çöldeyken, Sabat Günü* odun toplayan birini buldular.
\par 33 Odun toplarken adami bulanlar onu Musa'yla Harun'un ve bütün toplulugun önüne getirdiler.
\par 34 Adama ne yapilacagi belirlenmediginden onu gözaltinda tuttular.
\par 35 Derken RAB Musa'ya, "O adam öldürülmeli. Bütün topluluk ordugahin disinda onu tasa tutsun" dedi.
\par 36 Böylece topluluk adami ordugahin disina çikardi. RAB'bin Musa'ya buyurdugu gibi, onu taslayarak öldürdüler.
\par 37 RAB Musa'ya söyle dedi:
\par 38 "Israil halkina de ki, 'Kusaklar boyunca giysinizin dört yanina püskül dikeceksiniz. Her püskülün üzerine lacivert bir kordon koyacaksiniz.
\par 39 Öyle ki, püskülleri gördükçe RAB'bin buyruklarini animsayasiniz. Böylelikle RAB'bin buyruklarina uyacak, yüreginizin, gözünüzün istekleri ardinca gitmeyecek, hainlik etmeyeceksiniz.
\par 40 Ta ki, bütün buyruklarimi animsayip tutasiniz ve Tanriniz
\par 41 Tanriniz olmak için sizi Misir'dan çikaran Tanriniz RAB benim. Tanriniz RAB benim."

\chapter{16}

\par 1 Levi oglu Kehat oglu Yishar oglu Korah, Ruben soyundan Eliavogullari'ndan Datan, Aviram ve Pelet oglu On toplulukça seçilen, taninmis iki yüz elli Israilli önderle birlikte Musa'ya baskaldirdi.
\par 3 Hep birlikte Musa'yla Harun'un yanina varip, "Çok ileri gittiniz!" dediler, "Bütün topluluk, toplulugun her bireyi kutsaldir ve RAB onlarin arasindadir. Öyleyse neden kendinizi RAB'bin toplulugundan üstün görüyorsunuz?"
\par 4 Bunu duyan Musa yüzüstü yere kapandi.
\par 5 Sonra Korah'la yandaslarina söyle dedi: "Sabah RAB kimin kendisine ait oldugunu, kimin kutsal oldugunu açiklayacak ve o kisiyi huzuruna çagiracak. RAB seçecegi kisiyi huzuruna çagiracak.
\par 6 Ey Korah ve yandaslari, kendinize buhurdanlar alin.
\par 7 Yarin RAB'bin huzurunda buhurdanlarinizin içine ates, atesin üstüne de buhur koyun. RAB'bin seçecegi kisi, kutsal olan kisidir. Ey Levililer, çok ileri gittiniz!"
\par 8 Musa Korah'la konusmasini söyle sürdürdü: "Ey Levililer, beni dinleyin!
\par 9 Israil'in Tanrisi sizi kendi huzuruna çikarmak için ayirdi. RAB'bin Konutu'nun hizmetini yapmaniz, toplulugun önünde durmaniz, onlara hizmet etmeniz için sizi Israil toplulugunun arasindan seçti. Sizi ve bütün Levili kardeslerinizi huzuruna çikardi. Bu yetmiyormus gibi kâhinligi de mi istiyorsunuz?
\par 11 Ey Korah, senin ve yandaslarinin böyle toplanmasi RAB'be karsi gelmektir. Harun kim ki, ona dil uzatiyorsunuz?"
\par 12 Sonra Musa Eliavogullari Datan'la Aviram'i çagirtti. Ama onlar, "Gelmeyecegiz" dediler,
\par 13 "Bizi çölde öldürtmek için süt ve bal akan ülkeden çikardin. Bu yetmiyormus gibi basimiza geçmek istiyorsun.
\par 14 Bizi süt ve bal akan ülkeye götürmedigin gibi mülk olarak bize tarlalar, baglar da vermedin. Bu adamlari kör mü saniyorsun? Hayir, gelmeyecegiz."
\par 15 Çok öfkelenen Musa RAB'be, "Onlarin sunularini önemseme. Onlardan bir esek bile almadim, üstelik hiçbirine de haksizlik etmedim" dedi.
\par 16 Sonra Korah'a, "Yarin sen ve bütün yandaslarin -sen de, onlar da- RAB'bin önünde bulunmak için gelin" dedi, "Harun da gelsin.
\par 17 Herkes kendi buhurdanini alip içine buhur koysun. Iki yüz elli kisi birer buhurdan alip RAB'bin önüne getirsin. Harun'la sen de buhurdanlarinizi getirin."
\par 18 Böylece herkes buhurdanini alip içine ates, atesin üstüne de buhur koydu. Sonra Musa ve Harun'la birlikte Bulusma Çadiri'nin giris bölümünde durdular.
\par 19 Korah bütün toplulugu Musa'yla Harun'un karsisinda Bulusma Çadiri'nin giris bölümünde toplayinca, RAB'bin görkemi bütün topluluga göründü.
\par 20 RAB, Musa'yla Harun'a, "Bu toplulugun arasindan ayrilin da onlari bir anda yok edeyim" dedi.
\par 22 Musa'yla Harun yüzüstü yere kapanarak, "Ey Tanri, bütün insan ruhlarinin Tanrisi!" dediler, "Bir kisi günah isledi diye bütün topluluga mi öfkeleneceksin?"
\par 23 RAB Musa'ya, "Topluluga söyle, Korah'in, Datan'in, Aviram'in çadirlarindan uzaklassinlar" dedi.
\par 25 Musa Datan'la Aviram'a gitti. Israil'in ileri gelenleri onu izledi.
\par 26 Toplulugu uyararak, "Bu kötü adamlarin çadirlarindan uzak durun!" dedi, "Onlarin hiçbir seyine dokunmayin. Yoksa onlarin günahlari yüzünden caninizdan olursunuz."
\par 27 Bunun üzerine topluluk Korah, Datan ve Aviram'in çadirlarindan uzaklasti. Datan'la Aviram çikip karilari, küçük büyük çocuklariyla birlikte çadirlarinin önünde durdular.
\par 28 Musa söyle dedi: "Bütün bunlari yapmam için RAB'bin beni gönderdigini, kendiligimden bir sey yapmadigimi suradan anlayacaksiniz:
\par 29 Eger bu adamlar herkes gibi dogal bir ölümle ölür, herkesin basina gelen bir olayla karsilasirlarsa, bilin ki beni RAB göndermemistir.
\par 30 Ama RAB yepyeni bir olay yaratirsa, yer yarilip onlari ve onlara ait olan her seyi yutarsa, ölüler diyarina diri diri inerlerse, bu adamlarin RAB'be saygisizlik ettiklerini anlayacaksiniz."
\par 31 Musa konusmasini bitirir bitirmez Korah, Datan ve Aviram'in altindaki yer yarildi.
\par 32 Yer yarildi, onlari, ailelerini, Korah'in adamlariyla mallarini yuttu.
\par 33 Sahip olduklari her seyle birlikte diri diri ölüler diyarina indiler. Yer onlarin üzerine kapandi. Toplulugun arasindan yok oldular.
\par 34 Çigliklarini duyan çevredeki Israilliler, "Yer bizi de yutmasin!" diyerek kaçistilar.
\par 35 RAB'bin gönderdigi ates buhur sunan iki yüz elli adami yakip yok etti.
\par 36 RAB Musa'ya söyle dedi:
\par 37 "Kâhin Harun oglu Elazar'a buhurdanlari atesin içinden çikarmasini, ates korlarini az öteye dagitmasini söyle. Çünkü buhurdanlar kutsaldir.
\par 38 Isledikleri günahtan ötürü öldürülen bu adamlarin buhurdanlarini levha haline getirip sunagi bunlarla kapla. Buhurdanlar RAB'be sunulduklari için kutsaldir. Bunlar Israilliler için bir uyari olsun."
\par 39 Böylece Kâhin Elazar, yanarak ölen adamlarin getirdigi tunç* buhurdanlari RAB'bin Musa araciligiyla kendisine söyledigi gibi alip döverek sunagi kaplamak için levha haline getirdi. Bu, Israilliler'e Harun'un soyundan gelenlerden baska hiç kimsenin RAB'bin önüne çikip buhur yakmamasi gerektigini animsatacakti. Yoksa o kisi Korah'la yandaslari gibi yok olacakti.
\par 41 Ertesi gün bütün Israil toplulugu Musa'yla Harun'a söylenmeye basladi. "RAB'bin halkini siz öldürdünüz" diyorlardi.
\par 42 Topluluk Musa'yla Harun'a karsi toplanip Bulusma Çadiri'na dogru yönelince, çadiri ansizin bulut kapladi ve RAB'bin görkemi göründü.
\par 43 Musa'yla Harun Bulusma Çadiri'nin önüne geldiler.
\par 44 RAB Musa'ya, "Bu toplulugun arasindan ayrilin da onlari birden yok edeyim" dedi. Musa'yla Harun yüzüstü yere kapandilar.
\par 46 Sonra Musa Harun'a, "Buhurdanini alip içine sunaktan ates koy, üstüne de buhur koy" dedi, "Günahlarini bagislatmak için hemen topluluga git. Çünkü RAB öfkesini yagdirdi. Öldürücü hastalik basladi."
\par 47 Harun Musa'nin dedigini yaparak buhurdanini alip toplulugun ortasina kostu. Halkin arasinda öldürücü hastalik baslamisti. Harun buhur sunarak toplulugun günahini bagislatti.
\par 48 O ölülerle dirilerin arasinda durunca, öldürücü hastalik da dindi.
\par 49 Korah olayinda ölenler disinda, öldürücü hastaliktan ölenlerin sayisi 14 700 kisiydi.
\par 50 Öldürücü hastalik dindiginden, Harun Musa'nin yanina, Bulusma Çadiri'nin giris bölümüne döndü.

\chapter{17}

\par 1 RAB Musa'ya söyle dedi:
\par 2 "Israil halkina her oymak önderi için bir tane olmak üzere on iki degnek getirmesini söyle. Her önderin adini kendi degneginin üzerine yaz.
\par 3 Levi oymaginin degnegi üzerine Harun'un adini yazacaksin. Her oymak önderi için bir degnek olacak.
\par 4 Degnekleri Bulusma Çadiri'nda sizinle bulustugum Levha Sandigi'nin önüne koy.
\par 5 Seçecegim kisinin degnegi filiz verecek. Israil halkinin sizden sürekli yakinmasina son verecegim."
\par 6 Musa Israil halkiyla konustu. Halkin önderleri, her oymak önderi için bir tane olmak üzere on iki degnek getirdiler. Harun'un degnegi de aralarindaydi.
\par 7 Musa degnekleri Levha Sandigi'nin bulundugu çadirda RAB'bin önüne koydu.
\par 8 Ertesi gün Musa Levha Sandigi'nin bulundugu çadira girdi. Bakti, Levi oymagini temsil eden Harun'un degnegi filiz vermis, tomurcuklanip çiçek açmis, badem yetistirmis.
\par 9 Musa bütün degnekleri RAB'bin önünden çikarip Israil halkina gösterdi. Halk degneklere bakti, her biri kendi degnegini aldi.
\par 10 RAB Musa'ya, "Baskaldiranlara bir uyari olsun diye Harun'un degnegini saklanmak üzere Levha Sandigi'nin önüne koy" dedi, "Onlarin benden yakinmalarina son vereceksin; öyle ki, ölmesinler."
\par 11 Musa RAB'bin buyrugu uyarinca davrandi.
\par 12 Israilliler Musa'ya, "Yok olacagiz! Ölecegiz! Hepimiz yok olacagiz!" dediler,
\par 13 "RAB'bin Konutu'na her yaklasan ölüyor. Hepimiz mi yok olacagiz?"

\chapter{18}

\par 1 RAB Harun'a, "Sen, ogullarin ve ailen kutsal yere iliskin suçtan sorumlu tutulacaksiniz" dedi, "Kâhinlik görevinizle ilgili suçtan da sen ve ogullarin sorumlu tutulacaksiniz.
\par 2 Sen ve ogullarin Levha Sandigi'nin bulundugu çadirin önünde hizmet ederken, ataniz Levi'nin oymagindan kardeslerinizin de size katilip yardim etmelerini saglayin.
\par 3 Senin sorumlulugun altinda çadirda hizmet etsinler. Ancak, siz de onlar da ölmeyesiniz diye kutsal yerin esyalarina ya da sunaga yaklasmasinlar.
\par 4 Seninle çalisacak ve Bulusma Çadiri'yla ilgili bütün hizmetlerden sorumlu olacaklar. Levililer disinda hiç kimse bulundugunuz yere yaklasmayacak.
\par 5 "Bundan sonra Israil halkina öfkelenmemem için kutsal yerin ve sunagin hizmetinden sizler sorumlu olacaksiniz.
\par 6 Ben Israilliler arasindan Levili kardeslerinizi size bir armagan olarak seçtim. Bulusma Çadiri'yla ilgili hizmeti yapmalari için onlar bana adanmistir.
\par 7 Ama sunaktaki ve perdenin ötesindeki kâhinlik görevini sen ve ogullarin üstleneceksiniz. Kâhinlik görevini size armagan olarak veriyorum. Sizden baska kutsal yere kim yaklasirsa öldürülecektir."
\par 8 RAB Harun'la konusmasini söyle sürdürdü: "Bana sunulan kutsal sunularin bagis kisimlarini sana veriyorum. Bunlari sonsuza dek pay olarak sana ve ogullarina veriyorum.
\par 9 Sunakta tümüyle yakilmayan, bana sunulan en kutsal sunulardan sunlar senin olacak: Tahil, suç ve günah sunulari*. En kutsal sunular senin ve ogullarinin olacak.
\par 10 Bunlari en kutsal sunu olarak yiyeceksin. Her erkek onlardan yiyebilir. Onlari kutsal sayacaksin.
\par 11 "Ayrica sunlar da senin olacak: Israilliler'in sundugu sallamalik sunularin bagis kisimlarini sonsuza dek pay olarak sana, ogullarina ve kizlarina veriyorum. Ailende dinsel açidan temiz olan herkes onlari yiyebilir.
\par 12 "RAB'be verdikleri ilk ürünleri -zeytinyaginin, yeni sarabin, tahilin en iyisini- sana veriyorum.
\par 13 Ülkede yetisen ilk ürünlerden RAB'be getirdiklerinin tümü senin olacak. Ailende dinsel açidan temiz olan herkes onlari yiyebilir.
\par 14 "Israil'de RAB'be kosulsuz adanan her sey senin olacak.
\par 15 Insan olsun hayvan olsun RAB'be adanan her rahmin ilk ürünü senin olacak. Ancak ilk dogan her çocuk ve kirli sayilan hayvanlarin her ilk dogani için kesinlikle bedel alacaksin.
\par 16 Ilk doganlar bir aylikken, kendi biçecegin deger uyarinca, yirmi geradan olusan kutsal yerin sekeline göre bes sekel*fb* gümüs bedel alacaksin.
\par 17 "Ancak sigirin, koyunun ya da keçinin ilk dogani için bedel almayacaksin. Onlar benim için ayrilmistir. Kanlarini sunagin üzerine dökeceksin, yaglarini RAB'bi hosnut eden koku olsun diye yakilan bir sunu olarak yakacaksin.
\par 18 Sallamalik sununun gögsü ve sag budu senin oldugu gibi eti de senin olacak.
\par 19 Israilliler'in bana sunduklari kutsal sunularin bagis kisimlarini sonsuza dek pay olarak sana, ogullarina ve kizlarina veriyorum. Senin ve soyun için bu RAB'bin önünde sonsuza dek sürecek bozulmaz bir antlasmadir."
\par 20 RAB Harun'la konusmasini söyle sürdürdü: "Onlarin ülkesinde mirasin olmayacak, aralarinda hiçbir payin olmayacak. Israilliler arasinda payin ve mirasin benim.
\par 21 "Bulusma Çadiri'yla ilgili yaptiklari hizmete karsilik, Israil'de toplanan bütün ondaliklari pay olarak Levililer'e veriyorum.
\par 22 Bundan böyle öbür Israilliler Bulusma Çadiri'na yaklasmamali. Yoksa günahlarinin bedelini canlariyla öderler.
\par 23 Bulusma Çadiri'yla ilgili hizmeti Levililer yapacak, çadira karsi islenen suçtan onlar sorumlu olacak. Gelecek kusaklariniz boyunca kalici bir kural olacak bu. Israilliler arasinda onlarin payi olmayacak.
\par 24 Bunun yerine Israilliler'in RAB'be armagan olarak verdigi ondaligi miras olarak Levililer'e veriyorum. Bu yüzden Levililer için, 'Israilliler arasinda onlarin mirasi olmayacak dedim."
\par 25 RAB Musa'ya söyle dedi:
\par 26 "Levililer'e de ki, 'Pay olarak size verdigim ondaliklari Israilliler'den alinca, aldiginiz ondaligin ondaligini RAB'be armagan olarak sunacaksiniz.
\par 27 Armaganiniz harmandan tahil ya da üzüm sikma çukurundan bir armagan sayilacaktir.
\par 28 Böylelikle siz de Israilliler'den aldiginiz bütün ondaliklardan RAB'be armagan sunacaksiniz. Bu ondaliklardan RAB'bin armaganini Kâhin Harun'a vereceksiniz.
\par 29 Aldiginiz bütün armaganlardan RAB için bir armagan ayiracaksiniz; hepsinin en iyisini, en kutsalini ayiracaksiniz.
\par 30 "Levililer'e söyle de: 'En iyisini sundugunuzda, geri kalani harman ya da asma ürünü olarak size sayilacaktir.
\par 31 Siz ve aileniz her yerde ondan yiyebilirsiniz. Bulusma Çadiri'nda yaptiginiz hizmete karsilik size verilen ücrettir bu.
\par 32 En iyisini sunarsaniz, bu konuda günah islememis olursunuz. Ölmemek için Israilliler'in sundugu kutsal sunulari kirletmeyeceksiniz."

\chapter{19}

\par 1 RAB Musa'yla Harun'a söyle dedi:
\par 2 "RAB'bin buyurdugu yasanin kurali sudur: Israilliler'e size kusursuz, özürsüz, boyunduruk takmamis kizil bir inek getirmelerini söyleyin.
\par 3 Inek Kâhin Elazar'a verilsin; ordugahin disina çikarilip onun önünde kesilecek.
\par 4 Kâhin Elazar parmagiyla kanindan alip yedi kez Bulusma Çadiri'nin önüne dogru serpecek.
\par 5 Sonra Elazar'in gözü önünde inek, derisi, eti, kani ve gübresiyle birlikte yakilacak.
\par 6 Kâhin biraz sedir agaci, mercanköskotu ve kirmizi iplik alip yanmakta olan inegin üzerine atacak.
\par 7 Sonra giysilerini yikayacak, yikanacak. Ancak o zaman ordugaha girebilir. Ama aksama dek kirli sayilacaktir.
\par 8 Inegi yakan kisi de giysilerini yikayacak, yikanacak. O da aksama dek kirli sayilacak.
\par 9 "Temiz sayilan bir kisi inegin külünü toplayip ordugahin disinda temiz sayilan bir yere koyacak. Israil toplulugu temizlenme suyu için bu külü saklayacak; bu, günahtan arinmak içindir.
\par 10 Inegin külünü toplayan adam giysilerini yikayacak, aksama dek kirli sayilacak. Bu kural hem Israilliler, hem de aralarinda yasayan yabancilar için kalici olacaktir.
\par 11 "Herhangi bir insan ölüsüne dokunan kisi yedi gün kirli sayilacaktir.
\par 12 Üçüncü ve yedinci gün temizlenme suyuyla kendini arindiracak, böylece paklanmis olacak. Üçüncü ve yedinci gün kendini arindirmazsa, paklanmis sayilmayacak.
\par 13 Herhangi bir insan ölüsüne dokunup da kendini arindirmayan kisi RAB'bin Konutu'nu kirletmis olur. O kisi Israil'den atilmali. Temizlenme suyu üzerine dökülmedigi için kirli sayilir, kirliligi üzerinde kalir.
\par 14 "Çadirda biri öldügü zaman uygulanacak kural sudur: Çadira giren ve çadirda bulunan herkes yedi gün kirli sayilacaktir.
\par 15 Kapagi iple baglanmamis, agzi açik her kap kirli sayilacaktir.
\par 16 "Kirda kiliçla öldürülmüs ya da dogal ölümle ölmüs birine, insan kemigine ya da mezara her dokunan yedi gün kirli sayilacaktir.
\par 17 "Kirli sayilan kisi için bir kabin içine yakilan günah sunusunun* külünden koyun, üstüne duru su dökeceksiniz.
\par 18 Temiz sayilan bir adam mercanköskotunu alip suya batiracak. Sonra çadirin, bütün esyalarin ve orada bulunanlarin üzerine serpecek. Kemige, öldürülmüs ya da dogal ölümle ölmüs kisiye ya da mezara dokunanin üzerine de suyu serpecek.
\par 19 Temiz sayilan adam, üçüncü ve yedinci gün kirli sayilanin üzerine suyu serpecek. Yedinci gün onu arindiracak. Arinan kisi giysilerini yikayacak, yikanacak ve aksam temiz sayilacak.
\par 20 Ancak, kirli sayilan biri kendini arindirmazsa toplulugun arasindan atilmali. Çünkü RAB'bin Tapinagi'ni kirletmistir. Temizlenme suyu üzerine dökülmedigi için kirli sayilir.
\par 21 Onlar için bu kural kalici olacaktir. Temizlenme suyunu serpen kisi de giysisini yikamali. Temizlenme suyuna dokunan kisi aksama dek kirli sayilacak.
\par 22 Kirli sayilan birinin dokundugu nesne kirli sayilir; o nesneye dokunan da aksama dek kirli sayilir."

\chapter{20}

\par 1 Israil toplulugu birinci ay* Zin Çölü'ne vardi, halk Kades'te konakladi. Miryam orada öldü ve gömüldü.
\par 2 Ancak topluluk için içecek su yoktu. Halk Musa'yla Harun'a karsi toplandi.
\par 3 Musa'ya, "Keske kardeslerimiz RAB'bin önünde öldügünde biz de ölseydik!" diye çikistilar,
\par 4 "RAB'bin toplulugunu neden bu çöle getirdiniz? Biz de hayvanlarimiz da ölelim diye mi?
\par 5 Neden bizi bu korkunç yere getirmek için Misir'dan çikardiniz? Ne tahil, ne incir, ne üzüm ne de nar var. Üstelik içecek su da yok!"
\par 6 Musa'yla Harun topluluktan ayrilip Bulusma Çadiri'nin giris bölümüne gittiler, yüzüstü yere kapandilar. RAB'bin görkemi onlara göründü.
\par 7 RAB Musa'ya, "Degnegi al" dedi, "Sen ve agabeyin Harun toplulugu toplayin. Halkin gözü önünde su fiskirmasi için kayaya buyruk verin. Onlar da hayvanlari da içsin diye kayadan onlara su çikaracaksiniz."
\par 9 Musa kendisine verilen buyruk uyarinca degnegi RAB'bin önünden aldi.
\par 10 Musa'yla Harun toplulugu kayanin önüne topladilar. Musa, "Ey siz, baskaldiranlar, beni dinleyin!" dedi, "Bu kayadan size su çikaralim mi?"
\par 11 Sonra kolunu kaldirip degnegiyle kayaya iki kez vurdu. Kayadan bol su fiskirdi, topluluk da hayvanlari da içti.
\par 12 RAB Musa'yla Harun'a, "Madem Israilliler'in gözü önünde benim kutsalligimi sayarak bana güvenmediniz" dedi, "Bu toplulugu kendilerine verecegim ülkeye de götürmeyeceksiniz."
\par 13 Bu sulara Meriva sulari denildi. Çünkü Israil halki orada RAB'be çikismis, RAB de aralarinda kutsalligini göstermisti.
\par 14 Musa Kades'ten Edom Krali'na ulaklarla su haberi gönderdi: "Kardesin Israil söyle diyor: 'Basimiza gelen güçlükleri biliyorsun.
\par 15 Atalarimiz Misir'a gitmisler. Orada uzun yillar yasadik. Misirlilar atalarimiza da bize de kötü davrandilar.
\par 16 Ama biz RAB'be yakarinca, yakarisimizi isitti. Bir melek gönderip bizi Misir'dan çikardi. "'Simdi senin sinirina yakin bir kent olan Kades'teyiz.
\par 17 Izin ver, ülkenden geçelim. Tarlalardan, baglardan geçmeyecegiz, hiçbir kuyudan da su içmeyecegiz. Sinirindan geçinceye dek, saga sola sapmadan Kral yolundan yolumuza devam edecegiz."
\par 18 Ama Edom Krali, "Ülkemden geçmeyeceksiniz!" diye yanitladi, "Geçmeye kalkisirsaniz kiliçla karsiniza çikarim."
\par 19 Israilliler, "Yol boyunca geçip gidecegiz" dediler, "Eger biz ya da hayvanlarimiz suyundan içersek karsiligini öderiz. Yürüyüp geçmek için senden izin istiyoruz, hepsi bu."
\par 20 Edom Krali yine, "Geçmeyeceksiniz!" yanitini verdi. Edomlular Israilliler'e saldirmak üzere kalabalik ve güçlü bir orduyla yola çiktilar.
\par 21 Edom Krali ülkesinden geçmelerine izin vermeyince, Israilliler dönüp ondan uzaklastilar.
\par 22 Israil toplulugu Kades'ten ayrilip Hor Dagi'na geldi.
\par 23 RAB, Edom sinirindaki Hor Dagi'nda Musa'yla Harun'a söyle dedi:
\par 24 "Harun ölüp atalarina kavusacak. Israil halkina verecegim ülkeye girmeyecek. Çünkü ikiniz Meriva sularinda verdigim buyruga karsi geldiniz.
\par 25 Harun'la oglu Elazar'i Hor Dagi'na çikar.
\par 26 Harun'un kâhinlik giysilerini üzerinden çikarip oglu Elazar'a giydir. Harun orada ölüp atalarina kavusacak."
\par 27 Musa RAB'bin buyurdugu gibi yapti. Bütün toplulugun gözü önünde Hor Dagi'na çiktilar.
\par 28 Musa Harun'un kâhinlik giysilerini üzerinden çikarip oglu Elazar'a giydirdi. Harun orada, dagin tepesinde öldü. Sonra Musa'yla Elazar dagdan indiler.
\par 29 Harun'un öldügünü ögrenince bütün Israil halki onun için otuz gün yas tuttu.

\chapter{21}

\par 1 Negev'de yasayan Kenanli Arat Krali, Israilliler'in Atarim yolundan geldigini duyunca, onlara saldirarak bazilarini tutsak aldi.
\par 2 Bunun üzerine Israilliler, "Eger bu halki tümüyle elimize teslim edersen, kentlerini büsbütün yok edecegiz" diyerek RAB'be adak adadilar.
\par 3 RAB Israilliler'in yalvarisini isitti ve Kenanlilar'i ellerine teslim etti. Israilliler onlari da kentlerini de büsbütün yok ettiler. Oraya Horma adi verildi.
\par 4 Edom ülkesinin çevresinden geçmek için Kizildeniz* yoluyla Hor Dagi'ndan ayrildilar. Ama yolda halk sabirsizlandi.
\par 5 Tanri'dan ve Musa'dan yakinarak, "Çölde ölelim diye mi bizi Misir'dan çikardiniz?" dediler, "Burada ne ekmek var, ne de su. Ayrica bu igrenç yiyecekten de tiksiniyoruz!"
\par 6 Bunun üzerine RAB halkin arasina zehirli yilanlar gönderdi. Yilanlar isirinca Israilliler'den birçok kisi öldü.
\par 7 Halk Musa'ya gelip, "RAB'den ve senden yakinmakla günah isledik. Yalvar da, RAB aramizdan yilanlari kaldirsin" dedi. Bunun üzerine Musa halk için yalvardi.
\par 8 RAB Musa'ya, "Bir yilan yap ve onu bir diregin üzerine koy. Isirilan herkes ona bakinca yasayacaktir" dedi.
\par 9 Böylece Musa tunç* bir yilan yaparak diregin üzerine koydu. Yilan tarafindan isirilan kisiler tunç yilana bakinca yasadi.
\par 10 Israil halki yola koyulup Ovot'ta konakladi.
\par 11 Sonra Ovot'tan ayrilip doguda Moav'a bakan çölde, Iye-Haavarim'de konakladi.
\par 12 Oradan da ayrilip Zeret Vadisi'nde konakladi.
\par 13 Oradan da ayrilip Amorlular'in sinirina dek uzanan çölde, Arnon Vadisi'nin karsi yakasinda konakladilar. Arnon Moav'la Amorlular'in ülkesi arasindaki Moav siniridir.
\par 14 RAB'bin Savaslari Kitabi'nda söyle yazilidir: Sufa topraklarinda Vahev Kenti, vadiler, Arnon Vadisi,
\par 15 Ar Kenti'ne dayanan ve Moav siniri boyunca uzanan vadilerin yamaçlari ..."
\par 16 Oradan RAB'bin Musa'ya, "Halki bir araya topla, onlara su verecegim" dedigi kuyuya, Beer'e dogru yol aldilar.
\par 17 O zaman Israilliler su ezgiyi söylediler: "Sularin fiskirsin, ey kuyu! Ezgi okuyun ona.
\par 18 O kuyu ki, onu önderlerle Halkin soylulari Asayla, degnekle kazdilar." Bundan sonra çölden Mattana'ya,
\par 19 Mattana'dan Nahaliel'e, Nahaliel'den Bamot'a,
\par 20 Bamot'tan Moav topraklarindaki vadiye, çöle bakan Pisga Dagi'nin eteklerine gittiler.
\par 21 Israilliler Amorlular'in Krali Sihon'a ulaklarla su haberi gönderdi:
\par 22 "Izin ver, ülkenden geçelim. Tarlalardan, baglardan geçmeyecegiz, hiçbir kuyudan su içmeyecegiz. Sinirindan geçinceye dek, Kral yolundan yolumuza devam edecegiz."
\par 23 Ne var ki Sihon, ülkesinden Israilliler'in geçmesine izin vermedi. Israilliler'le savasmak üzere bütün halkini toplayip çöle çikti. Yahesa'ya varinca, Israilliler'e saldirdi.
\par 24 Israilliler onu kiliçtan geçirip Arnon'dan Yabbuk'a, Ammonlular'in sinirina dek uzanan topraklarini aldilar. Az Kenti Ammon sinirini olusturuyordu.
\par 25 Israilliler Hesbon ve çevresindeki köylerle birlikte Amorlular'in bütün kentlerini ele geçirerek orada yasamaya basladilar.
\par 26 Hesbon Amorlular'in Krali Sihon'un kentiydi. Sihon eski Moav Krali'na karsi savasmis, Arnon'a dek uzanan topraklarini elinden almisti.
\par 27 Bunun için ozanlar söyle diyor: "Hesbon'a gelin, Sihon'un kenti yeniden kurulsun Ve saglamlastirilsin.
\par 28 Hesbon'dan ates, Sihon'un kentinden alev çikti; Moav'in Ar Kenti'ni, Arnon tepelerinin efendilerini yakip yok etti.
\par 29 Vay sana, ey Moav! Ilah Kemos'un halki, yok oldun! Kemos senin ogullarinin Amorlular'in Krali Sihon'a kaçmasini, Kizlarinin ona tutsak olmasini önleyemedi.
\par 30 Onlari bozguna ugrattik; Hesbon Divon'a dek yikima ugradi. Medeva'ya uzanan Nofah'a dek onlari yikima ugrattik."
\par 31 Böylece Israil halki Amorlular'in ülkesinde yasamaya basladi.
\par 32 Musa Yazer'i arastirmak için adamlar gönderdi. Sonra Israilliler Yazer çevresindeki köyleri ele geçirerek orada yasayan Ammonlular'i kovdular.
\par 33 Bundan sonra dönüp Basan'a dogru ilerlediler. Basan Krali Og'la ordusu onlarla savasmak için Edrei'de karsilarina çikti.
\par 34 RAB Musa'ya, "Ondan korkma!" dedi, "Çünkü onu da ordusuyla ülkesini de senin eline teslim ettim. Amorlular'in Hesbon'da yasayan Krali Sihon'a yaptiginin aynisini ona da yapacaksin."
\par 35 Böylece Israil halki kimseyi sag birakmadan Og'la ogullarini ve ordusunu yok etti, ülkeyi ele geçirdi.

\chapter{22}

\par 1 Israilliler yollarina devam ederek Moav ovalarinda, Seria Irmagi'nin dogusunda, Eriha karsisinda konakladilar.
\par 2 Sippor oglu Balak Israilliler'in Amorlular'a neler yaptigini duydu.
\par 3 Israil halki kalabalik oldugundan, Moavlilar onlardan korkarak yilgiya düstü.
\par 4 Midyan ileri gelenlerine, "Öküz kirda nasil otu yiyip tüketirse, bu topluluk da çevremizdeki her seyi yiyip bitirecek" dediler. O sirada Sippor oglu Balak Moav Krali'ydi.
\par 5 Balak, Beor oglu Balam'i çagirmak için ulaklar gönderdi. Balam Firat Irmagi kiyisinda, Amav ülkesindeki Petor'da yasiyordu. Balak söyle dedi: "Misir'dan çikip yeryüzünü kaplayan bir halk yanibasima yerlesti.
\par 6 Lütfen gel de benden daha güçlü olan bu halka benim için lanet oku. Olur ki, onlari yener, ülkeden kovariz. Çünkü senin kutsadigin kisinin kutsanacagini, lanetledigin kisinin lanetlenecegini biliyorum."
\par 7 Moav ve Midyan ileri gelenleri falcilik ücretini alip gittiler. Balam'a varinca Balak'in bildirisini ona ilettiler.
\par 8 Balam onlara, "Geceyi burada geçirin" dedi, "RAB'bin bana söyleyecekleri uyarinca size yanit verecegim." Bunun üzerine Moav önderleri geceyi Balam'in yaninda geçirdiler.
\par 9 Tanri Balam'a gelip, "Evinde kalan bu adamlar kim?" diye sordu.
\par 10 Balam Tanri'yi söyle yanitladi: "Sippor oglu Moav Krali Balak bana su bildiriyi gönderdi:
\par 11 'Misir'dan çikan halk yeryüzünü kapladi. Gel de benim için onlara lanet oku. Olur ki, onlarla savasmaya gücüm yeter, onlari kovarim."
\par 12 Ama Tanri Balam'a, "Onlarla gitme! Bu halka lanet okuma, onlar kutsanmis halktir" dedi.
\par 13 Sabah Balam kalkti, Balak'in önderlerine, "Ülkenize dönün. Çünkü RAB sizinle gelmeme izin vermiyor" dedi.
\par 14 Moav önderleri dönüp Balak'a, "Balam bizimle gelmedi" dediler.
\par 15 Bunun üzerine Balak ilk gidenlerden daha çok ve daha saygin baska önderler gönderdi.
\par 16 Balam'a gidip söyle dediler: "Sippor oglu Balak diyor ki, 'Lütfen yanima gelmene engel olan hiçbir seye izin verme.
\par 17 Çünkü seni fazlasiyla ödüllendirecegim, ne istersen yapacagim. Ne olur, gel, benim için bu halka lanet oku."
\par 18 Balam Balak'in ulaklarina su yaniti verdi: "Balak sarayini altinla, gümüsle doldurup bana verse bile, Tanrim RAB'bin buyrugundan öte küçük büyük hiçbir sey yapamam.
\par 19 Lütfen siz de bu gece burada kalin, RAB'bin bana baska bir diyecegi var mi ögreneyim."
\par 20 O gece Tanri Balam'a gelip, "Madem bu adamlar seni çagirmaya gelmis, onlarla git; ancak sana ne söylersem onu yap" dedi.
\par 21 Balam sabah kalkip esegine palan vurdu, Moav önderleriyle birlikte gitti.
\par 22 Tanri onun gidisine öfkelendi. RAB'bin melegi engel olmak için yoluna dikildi. Balam esegine binmisti, yaninda iki usagi vardi.
\par 23 Esek, yalin kiliç yolda durmakta olan RAB'bin melegini görünce, yoldan sapip tarlaya girdi. Balam yola döndürmek için esegi dövdü.
\par 24 RAB'bin melegi iki bagin arasinda iki yani duvarli dar bir yolda durdu.
\par 25 Esek RAB'bin melegini görünce duvara sikisti, Balam'in ayagini ezdi. Balam esegi yine dövdü.
\par 26 RAB'bin melegi ilerledi, saga sola dönüsü olmayan dar bir yerde durdu.
\par 27 Esek RAB'bin melegini görünce, Balam'in altinda yikildi. Balam öfkelendi, degnegiyle esegi dövdü.
\par 28 Bunun üzerine RAB esegi konusturdu. Esek Balam'a, "Sana ne yaptim ki, üç kez beni böyle dövdün?" diye sordu.
\par 29 Balam, "Benimle alay ediyorsun" diye yanitladi, "Elimde kiliç olsaydi, seni hemen öldürürdüm."
\par 30 Esek, "Bugüne dek hep üzerine bindigin esek degil miyim ben?" dedi, "Daha önce sana hiç böyle davrandim mi?" Balam, "Hayir" diye yanitladi.
\par 31 Bundan sonra RAB Balam'in gözlerini açti. Balam yalin kiliç yolda durmakta olan RAB'bin melegini gördü, egilip yüzüstü yere kapandi.
\par 32 RAB'bin melegi, "Neden üç kez esegini dövdün?" diye sordu, "Ben seni engellemeye geldim. Çünkü gittigin yol seni yikima götürüyor.
\par 33 Esek beni gördü, üç kez önümden sapti. Eger yoldan sapmasaydi, seni öldürür, onu sag birakirdim."
\par 34 Balam RAB'bin melegine, "Günah isledim" dedi, "Beni engellemek için yolda dikildigini anlamadim. Uygun görmüyorsan simdi evime döneyim."
\par 35 RAB'bin melegi, "Adamlarla git" dedi, "Ama yalniz sana söyleyeceklerimi söyleyeceksin." Böylece Balam Balak'in önderleriyle gitti.
\par 36 Balak Balam'in geldigini duyunca, onu karsilamak için Arnon kiyisinda, sinirin en uzak kösesindeki Moav Kenti'ne gitti.
\par 37 Balam'a, "Seni çagirmak için adam gönderdigimde neden gelmedin?" dedi, "Seni ödüllendirmeye gücüm yetmez mi?"
\par 38 Balam, "Iste simdi geldim" diye yanitladi, "Ama ne diyebilirim ki? Ancak Tanri'nin bana buyurduklarini söyleyecegim."
\par 39 Bundan sonra Balam Balak'la yola çikarak Kiryat-Husot'a gitti.
\par 40 Balak sigirlar, davarlar kurban etti, Balam'la yanindaki önderlere et gönderdi.
\par 41 Sabah Balak Balam'i Bamot-Baal'a çikardi. Balam oradan Israil halkinin bir kesimini görebildi.

\chapter{23}

\par 1 Balam Balak'a, "Burada benim için yedi sunak kur ve yedi bogayla yedi koç hazirla" dedi.
\par 2 Balak onun dedigini yapti. Balak'la Balam her sunagin üstünde birer bogayla koç sundular.
\par 3 Sonra Balam Balak'a, "Ben az öteye gidecegim, sen yakmalik sununun* yaninda bekle" dedi, "Olur ki, RAB karsima çikar. Bana ne açiklarsa, sana bildiririm." Sonra çiplak bir tepeye çikti.
\par 4 Tanri Balam'a göründü. Balam Tanri'ya, "Yedi sunak kurdum, her sunagin üstünde birer bogayla koç sundum" dedi.
\par 5 RAB Balam'a ne söylemesi gerektigini bildirerek, "Balak'a git, ona su haberi ilet" dedi.
\par 6 Böylece Balam Balak'in yanina döndü. Onun Moav önderleriyle birlikte yakmalik sunusunun yaninda durdugunu gördü.
\par 7 Sonra su bildiriyi iletti: "Balak beni Aram'dan, Moav Krali beni dogu daglarindan getirdi. 'Gel, benim için Yakup soyuna lanet oku dedi, 'Gel, Israil'in yikimini dile.
\par 8 Tanri'nin lanetlemedigini Ben nasil lanetlerim? RAB'bin yikimini istemedigi kisinin yikimini Ben nasil isteyebilirim?
\par 9 Kayalarin dorugundan görüyorum onlari, Tepelerden bakiyorum onlara. Tek basina yasayan, Uluslardan kendini soyutlayan Bir halk görüyorum.
\par 10 Kim Yakup soyunun tozunu Ve Israil'in dörtte birini sayabilir? Dogru kisilerin ölümüyle öleyim, Sonum onlarinki gibi olsun!"
\par 11 Balak Balam'a, "Bana ne yaptin?" dedi, "Düsmanlarima lanet okuyasin diye seni getirdim. Oysa sen onlari kutsadin!"
\par 12 Balam, "Ben ancak RAB'bin söylememi istedigi seyleri söylemeliyim" diye yanitladi.
\par 13 Bunun üzerine Balak, "Ne olur, benimle gel" dedi, "Onlari görebilecegin baska bir yere gidelim. Onlarin hepsini görmeyeceksin, bir kesimini göreceksin. Oradan onlara benim için lanet oku."
\par 14 Böylece Balak Balam'i Pisga Dagi'ndaki Gözcüler Yaylasi'na götürdü. Orada yedi sunak kurdu, her sunagin üstünde birer bogayla koç sundu.
\par 15 Balam Balak'a, "Az ötede RAB'be danisacagim, sen burada yakmalik sununun* yaninda bekle" dedi.
\par 16 RAB Balam'a göründü, ne söylemesi gerektigini bildirerek, "Balak'a git, ona su haberi ilet" dedi.
\par 17 Böylece Balam Balak'in yanina döndü, onun Moav önderleriyle birlikte yakmalik sunusunun yaninda durdugunu gördü. Balak, "RAB ne dedi?" diye sordu.
\par 18 Balam su bildiriyi iletti: "Ey Balak, uyan ve dinle; Ey Sippor oglu, bana kulak ver.
\par 19 Tanri insan degil ki, Yalan söylesin; Insan soyundan degil ki, Düsüncesini degistirsin. O söyler de yapmaz mi? Söz verir de yerine getirmez mi?
\par 20 Kutsamak için bana buyruk verildi; O kutsadi, ben degistiremem.
\par 21 Yakup soyunda suç bulunmadi, Ne de Israil'de kötülük. Tanrilari RAB aralarindadir, Aralarindaki kral olarak Adina sevinç çigliklari atiyorlar.
\par 22 Tanri onlari Misir'dan çikardi, O'nun yaban öküzü gibi gücü var.
\par 23 Yakup soyuna yapilan büyü tutmaz; Israil'e karsi falcilik etkili olmaz. Simdi Yakup ve Israil için, 'Tanri neler yapti! denecek.
\par 24 Iste halk bir disi aslan gibi uyaniyor. Avini yiyip bitirmedikçe, Öldürülenlerin kanini içmedikçe rahat etmeyen aslan gibi kalkiyor."
\par 25 Bunun üzerine Balak, "Onlara ne lanet oku, ne de onlari kutsa!" dedi.
\par 26 Balam, "RAB ne derse onu yapmaliyim dememis miydim sana?" diye yanitladi.
\par 27 Sonra Balak Balam'a, "Ne olur, gel, seni baska bir yere götüreyim" dedi, "Olur ki, Tanri oradan benim için onlara lanet okumana izin verir."
\par 28 Böylece Balam'i çöle bakan Peor Dagi'nin tepesine götürdü.
\par 29 Balam, "Burada benim için yedi sunak kurup yedi bogayla yedi koç hazirla" dedi.
\par 30 Balak onun dedigini yapti, her sunagin üstünde birer bogayla koç sundu.

\chapter{24}

\par 1 Balam, RAB'bin Israil halkini kutsamaktan hosnut oldugunu anlayinca, önceden yaptigi gibi gidip fala basvurmadi, yüzünü çöle çevirdi.
\par 2 Bakti, Israil'in oymak oymak yerlestigini gördü. Tanri'nin Ruhu onun üzerine inince,
\par 3 su bildiriyi iletti: "Beor oglu Balam, Gözü açilmis olan,
\par 4 Tanri'nin sözlerini duyan, Her Seye Gücü Yeten'in görümlerini gören, Yere kapanan, Tanri'nin gözlerini açtigi kisi bildiriyor:
\par 5 'Ey Yakup soyu, çadirlarin, Ey Israil, konutlarin ne güzel!
\par 6 Yayiliyorlar vadiler gibi, Irmak kiyisinda bahçeler gibi, RAB'bin diktigi öd agaçlari gibi, Su kiyisindaki sedir agaçlari gibi.
\par 7 Kovalarindan sular akacak, Tohumlari bol suyla sulanacak. Krallari Agak'tan büyük olacak, Kralligi yüceltilecek.
\par 8 Tanri onlari Misir'dan çikardi, O'nun yaban öküzü gibi gücü var. Düsmani olan uluslari yiyip bitirecek, Kemiklerini parçalayacak, Oklariyla onlari desecekler.
\par 9 Aslan gibi, disi aslan gibi Yere çömelir, yatarlar, Kim onlari uyandirmaya cesaret edebilir? Seni kutsayan kutsansin, Seni lanetleyen lanetlensin!"
\par 10 Balam'a öfkelenen Balak ellerini birbirine vurarak, "Düsmanlarima lanet okuyasin diye seni çagirdim" dedi, "Oysa üç kez onlari kutsadin.
\par 11 Haydi, hemen evine dön! Seni ödüllendirecegimi söylemistim. Ama RAB seni ödül almaktan yoksun birakti."
\par 12 Balam söyle karsilik verdi: "Bana gönderdigin ulaklara, 'Balak sarayini altinla, gümüsle doldurup bana verse bile, RAB'bin buyrugundan öte iyi kötü hiçbir sey yapamam. Ancak RAB ne derse onu söylerim dememis miydim?
\par 14 Iste simdi halkima dönüyorum. Gel, bu halkin gelecekte halkina neler yapacagini sana bildireyim."
\par 15 Sonra Balam su bildiriyi iletti: "Beor oglu Balam, Gözü açilmis olan,
\par 16 Tanri'nin sözlerini duyan, Yüceler Yücesi'nin bilgisine kavusan, Her Seye Gücü Yeten'in görümlerini gören, Yere kapanan, Tanri'nin gözlerini açtigi kisi bildiriyor:
\par 17 'Onu görüyorum, ama simdilik degil, Ona bakiyorum, ama yakindan degil. Yakup soyundan bir yildiz çikacak, Israil'den bir önder yükselecek. Moavlilar'in alinlarini, Setogullari'nin baslarini ezecek.
\par 18 Düsmani olan Edom ele geçirilecek, Evet, Seir alinacak, Ama Israil güçlenecek.
\par 19 Yakup soyundan gelen kisi önderlik edecek, Kentte sag kalanlari yok edecek."
\par 20 Balam Amalekliler'i görünce su bildiriyi iletti: "Amalek halki uluslar arasinda birinciydi, Ama sonu yikim olacak."
\par 21 Kenliler'i görünce de su bildiriyi iletti: "Yasadiginiz yer güvenli, Yuvaniz kayalarda kurulmus;
\par 22 Ama, ey Kenliler, Asurlular sizi tutsak edince, Yanip yok olacaksiniz."
\par 23 Balam bildirisini iletmeyi sürdürdü: "Ah, bunu yapan Tanri'ysa, Kim sag kalabilir?
\par 24 Kittim kiyilarindan gemiler gelecek, Asur'la Ever'i dize getirecekler, Kendileri de yikima ugrayacak."
\par 25 Bundan sonra Balam kalkip evine döndü, Balak da kendi yoluna gitti.

\chapter{25}

\par 1 Israilliler Sittim'de yasarken, erkekleri Moavli kadinlarla zina etmeye basladi.
\par 2 Bu kadinlar kendi ilahlarina kurban sunarken Israilliler'i de çagirdilar. Israil halki yiyeceklerden yedi ve onlarin ilahlarina tapti.
\par 3 Böylece Baal-Peor'a baglandilar. RAB bu yüzden onlara öfkelendi.
\par 4 Musa'ya, "Bu halkin bütün önderlerini gündüz benim önümde öldür" dedi, "Öyle ki, Israil halkina öfkem yatissin."
\par 5 Bunun üzerine Musa Israil yargiçlarina, "Her biriniz kendi adamlariniz arasinda Baal-Peor'a baglanmis olanlari öldürün" dedi.
\par 6 O sirada Israilli bir adam geldi, Musa'nin ve Bulusma Çadiri'nin girisinde aglayan Israil toplulugunun gözü önünde kardesine Midyanli bir kadin getirdi.
\par 7 Bunu gören Kâhin Harun oglu Elazar oglu Pinehas topluluktan ayrilip eline bir mizrak aldi.
\par 8 Israilli'nin ardina düserek çadira girdi ve mizragi ikisine birden sapladi. Mizrak hem Israilli'nin, hem de Midyanli kadinin karnini delip geçti. Böylece Israil'i yok eden hastalik dindi.
\par 9 Hastaliktan ölenlerin sayisi 24 000 kisiydi.
\par 10 RAB Musa'ya söyle dedi:
\par 11 "Kâhin Harun oglu Elazar oglu Pinehas Israil halkina öfkemin dinmesine neden oldu. Çünkü o, aralarinda benim adima büyük kiskançlik duydu. Bu yüzden onlari kiskançliktan büsbütün yok etmedim.
\par 12 Ona de ki, 'Onunla bir esenlik antlasmasi yapacagim.
\par 13 Kendisi ve soyundan gelenler için kalici bir kâhinlik antlasmasi olacak bu. Çünkü o Tanrisi için kiskançlik duydu ve Israil halkinin günahlarini bagislatti."
\par 14 Midyanli kadinla birlikte öldürülen Israilli, Simonogullari'nin bir aile basiydi ve adi Salu oglu Zimri'ydi.
\par 15 Öldürülen kadin ise Midyanli bir aile basi olan Sur'un kizi Kozbi'ydi.
\par 16 RAB Musa'ya, "Midyanlilar'i düsman say ve yok et" dedi,
\par 18 "Çünkü Peor olayinda ve bunun sonucunda ölümcül hastalik çiktigi gün öldürülen kizkardesleri Midyanli önderin kizi Kozbi olayinda kurduklari tuzaklarla sizi aldatarak düsmanca davrandilar."

\chapter{26}

\par 1 Ölümcül hastalik son bulunca RAB, Musa'yla Kâhin Harun oglu Elazar'a, "Israil toplulugunun ailelerine göre sayimini yapin" dedi, "Savasabilecek durumdaki yirmi ve daha yukari yastaki bütün erkekleri sayin."
\par 3 Bunun üzerine Musa'yla Kâhin Elazar, Seria Irmagi'nin yaninda, Eriha karsisindaki Moav ovalarinda halka, "RAB'bin Musa'ya verdigi buyruk uyarinca, yirmi ve daha yukari yastaki erkekleri sayin" dediler. Misir'dan çikan Israilliler sunlardi:
\par 5 Israil'in ilk dogani Ruben'in soyundan gelen Rubenogullari: Hanok soyundan Hanok boyu, Pallu soyundan Pallu boyu,
\par 6 Hesron soyundan Hesron boyu, Karmi soyundan Karmi boyu.
\par 7 Ruben boylari bunlardi, sayilari 43 730 kisiydi.
\par 8 Pallu'nun oglu Eliav,
\par 9 Eliav'in ogullari Nemuel, Datan ve Aviram'di. Bunlar toplulugun seçtigi, Musa'yla Harun'a, dolayisiyla RAB'be baskaldirarak Korah'in yandaslarina katilan Datan'la Aviram'di.
\par 10 Yer yarilip onlari Korah'la birlikte yutunca yok oldular. Ates Korah'in iki yüz elli yandasini yakip yok etti. Böylece baskalarina bir uyari oldular.
\par 11 Korah'in ogullari ise ölmedi.
\par 12 Boylarina göre Simonogullari sunlardi: Nemuel soyundan Nemuel boyu, Yamin soyundan Yamin boyu, Yakin soyundan Yakin boyu,
\par 13 Zerah soyundan Zerah boyu, Saul soyundan Saul boyu.
\par 14 Simon boylari bunlardi, sayilari 22 200 kisiydi.
\par 15 Boylarina göre Gadogullari sunlardi: Sefon soyundan Sefon boyu, Hagi soyundan Hagi boyu, Suni soyundan Suni boyu,
\par 16 Ozni soyundan Ozni boyu, Eri soyundan Eri boyu,
\par 17 Arot soyundan Arot boyu, Areli soyundan Areli boyu.
\par 18 Gad boylari bunlardi, sayilari 40 500 kisiydi.
\par 19 Yahuda'nin iki oglu Er'le Onan Kenan ülkesinde ölmüslerdi.
\par 20 Boylarina göre Yahudaogullari sunlardi: Sela soyundan Sela boyu, Peres soyundan Peres boyu, Zerah soyundan Zerah boyu.
\par 21 Peres soyundan gelenler sunlardi: Hesron soyundan Hesron boyu, Hamul soyundan Hamul boyu.
\par 22 Yahuda boylari bunlardi, sayilari 76 500 kisiydi.
\par 23 Boylarina göre Issakarogullari sunlardi: Tola soyundan Tola boyu, Puvva soyundan Puvva boyu,
\par 24 Yasuv soyundan Yasuv boyu, Simron soyundan Simron boyu.
\par 25 Issakar boylari bunlardi, sayilari 64 300 kisiydi.
\par 26 Boylarina göre Zevulunogullari sunlardi: Seret soyundan Seret boyu, Elon soyundan Elon boyu, Yahleel soyundan Yahleel boyu.
\par 27 Zevulun boylari bunlardi, sayilari 60 500 kisiydi.
\par 28 Boylarina göre Yusuf'un ogullari: Manasse ve Efrayim.
\par 29 Manasse soyundan gelenler: Makir soyundan Makir boyu -Makir Gilat'in babasiydi- Gilat soyundan Gilat boyu.
\par 30 Gilat soyundan gelenler sunlardi: Iezer soyundan Iezer boyu, Helek soyundan Helek boyu,
\par 31 Asriel soyundan Asriel boyu, Sekem soyundan Sekem boyu,
\par 32 Semida soyundan Semida boyu, Hefer soyundan Hefer boyu.
\par 33 Hefer oglu Selofhat'in ogullari olmadi; yalniz Mahla, Noa, Hogla, Milka ve Tirsa adinda kizlari vardi.
\par 34 Manasse boylari bunlardi, sayilari 52 700 kisiydi.
\par 35 Boylarina göre Efrayim soyundan gelenler sunlardi: Sutelah soyundan Sutelah boyu, Beker soyundan Beker boyu, Tahan soyundan Tahan boyu.
\par 36 Sutelah soyundan gelenler sunlardi: Eran soyundan Eran boyu.
\par 37 Efrayim boylari bunlardi, sayilari 32 500 kisiydi. Boylarina göre Yusuf'un soyundan gelenler bunlardi.
\par 38 Boylarina göre Benyamin soyundan gelenler: Bala soyundan Bala boyu, Asbel soyundan Asbel boyu, Ahiram soyundan Ahiram boyu,
\par 39 Sufam soyundan Sufam boyu, Hufam soyundan Hufam boyu.
\par 40 Bala'nin ogullari Ard'la Naaman'di. Ard soyundan Ard boyu, Naaman soyundan Naaman boyu.
\par 41 Boylarina göre Benyamin soyundan gelenler bunlardi, sayilari 45 600 kisiydi.
\par 42 Boylarina göre Dan soyundan gelenler sunlardi: Suham soyundan Suham boyu. Dan boyu buydu.
\par 43 Hepsi Suham boyundandi, sayilari 64 400 kisiydi.
\par 44 Boylarina göre Aser soyundan gelenler: Yimna soyundan Yimna boyu, Yisvi soyundan Yisvi boyu, Beria soyundan Beria boyu.
\par 45 Beria soyundan gelenler: Hever soyundan Hever boyu, Malkiel soyundan Malkiel boyu.
\par 46 Aser'in Serah adinda bir kizi vardi.
\par 47 Aser boylari bunlardi, sayilari 53 400 kisiydi.
\par 48 Boylarina göre Naftali soyundan gelenler: Yahseel soyundan Yahseel boyu, Guni soyundan Guni boyu,
\par 49 Yeser soyundan Yeser boyu, Sillem soyundan Sillem boyu.
\par 50 Boylarina göre Naftali boylari bunlardi, sayilari 45 400 kisiydi.
\par 51 Sayilan Israilliler'in toplami 601 730 erkekti.
\par 52 RAB Musa'ya söyle dedi:
\par 53 "Adlarinin sayisina göre ülke bunlara pay olarak bölüstürülecek.
\par 54 Sayica çok olana büyük, sayica az olana küçük pay vereceksin. Her oymaga kisi sayisina göre pay verilecek.
\par 55 Ancak ülke kura ile dagitilacak. Herkesin payi, ata oymaginin adina göre olacak.
\par 56 Küçük, büyük her oymagin payi kurayla dagitilacak."
\par 57 Boylarina göre sayilan Levililer sunlardi: Gerson soyundan Gerson boyu, Kehat soyundan Kehat boyu, Merari soyundan Merari boyu.
\par 58 Sunlar da Levili boylardi: Livni boyu, Hevron boyu, Mahli boyu, Musi boyu, Korah boyu. Kehat Amram'in babasiydi.
\par 59 Amram'in karisi Misir'da dogan, Levi soyundan gelme Yokevet'ti. Amram'a Harun'u, Musa'yi ve kizkardesleri Miryam'i dogurdu.
\par 60 Harun Nadav, Avihu, Elazar ve Itamar'in babasiydi.
\par 61 Nadav'la Avihu RAB'bin önünde kurallara aykiri bir ates sunarken öldüler.
\par 62 Levililer'den sayilan bir aylik ve daha yukari yastaki bütün erkekler 23 000 kisiydi. Bunlar öbür Israilliler'le birlikte sayilmadilar. Çünkü öbür Israilliler arasinda onlara pay verilmemisti.
\par 63 Seria Irmagi yaninda, Eriha karsisindaki Moav ovalarinda Musa'yla Kâhin Elazar'in saydiklari Israilliler bunlardi.
\par 64 Ancak, bu sayilanlarin arasinda Musa'yla Kâhin Harun'un Sina Çölü'nde saymis oldugu Israilliler'den kimse yoktu.
\par 65 Çünkü RAB o dönemde sayimi yapilan Israilliler'in kesinlikle çölde ölecegini söylemisti. Onlardan Yefunne oglu Kalev'le Nun oglu Yesu'dan baska kimse sag kalmamisti.

\chapter{27}

\par 1 Yusuf oglu Manasse'nin boylarindan Manasse oglu Makir oglu Gilat oglu Hefer oglu Selofhat'in Mahla, Noa, Hogla, Milka, Tirsa adindaki kizlari, Bulusma Çadiri'nin girisinde Musa'nin, Kâhin Elazar'in, önderlerin ve bütün toplulugun önüne gelip söyle dediler:
\par 3 "Babamiz çölde öldü. RAB'be baskaldiran Korah'in yandaslari arasinda degildi. Islemis oldugu günahtan ötürü öldü. Ogullari olmadi.
\par 4 Erkek çocugu olmadi diye babamizin adi kendi boyu arasindan neden yok olsun? Babamizin kardesleri arasinda bize de mülk verin."
\par 5 Musa onlarin davasini RAB'be götürdü.
\par 6 RAB Musa'ya söyle dedi:
\par 7 "Selofhat'in kizlari dogru söylüyor. Onlara amcalariyla birlikte miras olarak mülk verecek, babalarinin mirasini onlara aktaracaksin.
\par 8 "Israilliler'e de ki, 'Bir adam erkek çocugu olmadan ölürse, mirasini kizina vereceksiniz.
\par 9 Kizi yoksa mirasini kardeslerine,
\par 10 kardesleri yoksa amcalarina vereceksiniz.
\par 11 Amcalari da yoksa, mirasini bagli oldugu boyda kendisine en yakin akrabasina vereceksiniz. Yakini mirasi mülk edinsin. Musa'ya verdigim buyruk uyarinca, Israilliler için kesin bir kural olacak bu."
\par 12 Bundan sonra RAB Musa'ya, "Haavarim daglik bölgesine çik, Israilliler'e verecegim ülkeye bak" dedi,
\par 13 "Ülkeyi gördükten sonra, agabeyin Harun gibi sen de ölüp atalarina kavusacaksin.
\par 14 Çünkü ikiniz de Zin Çölü'nde buyruguma karsi çiktiniz. Topluluk sularda bana baskaldirdiginda, onlarin önünde kutsalligimi önemsemediniz." -Bunlar Zin Çölü'ndeki Kades'te Meriva sularidir.-
\par 15 Musa, "Bütün insan ruhlarinin Tanrisi RAB bu topluluga bir önder atasin" diye karsilik verdi,
\par 17 "O kisi toplulugun önünde yürüsün ve toplulugu yönetsin. Öyle ki, RAB'bin toplulugu çobansiz koyunlar gibi kalmasin."
\par 18 Bunun üzerine RAB, "Kendisinde RAB'bin Ruhu bulunan Nun oglu Yesu'yu yanina al, üzerine elini koy" dedi,
\par 19 "Onu Kâhin Elazar'la bütün toplulugun önüne çikar ve hepsinin önünde görevlendir.
\par 20 Bütün Israil toplulugu sözünü dinlesin diye ona yetkinden ver.
\par 21 Kâhin Elazar'in önüne çikacak; kâhin, Yesu için Urim* araciligiyla RAB'be danisacak. Bu yöntemle Elazar Yesu'yu ve bütün halki yönlendirecek."
\par 22 Musa RAB'bin kendisine buyurdugu gibi yapti. Yesu'yu Kâhin Elazar'in ve bütün toplulugun önüne götürdü.
\par 23 RAB'bin buyrugu uyarinca ellerini üzerine koyarak onu görevlendirdi.

\chapter{28}

\par 1 RAB Musa'ya söyle dedi:
\par 2 "Israilliler'e buyur ve de ki, 'Bana sunacaginiz sunuyu -yakilan sunu ve beni hosnut eden koku olarak sunacaginiz yiyecegi- belirlenen zamanda sunmaya dikkat edeceksiniz.
\par 3 Onlara de ki, 'RAB'be sunacaginiz yakilan sunu sudur: Günlük yakmalik sunu* olarak her gün bir yasinda kusursuz iki erkek kuzu sunacaksiniz.
\par 4 Kuzunun birini sabah, öbürünü aksamüstü sunun.
\par 5 Kuzuyla birlikte tahil sunusu* olarak dörtte bir hin sikma zeytinyagiyla yogrulmus onda bir efa ince un sunacaksiniz.
\par 6 Günlük yakmalik sunu, Sina Dagi'nda baslatilan, RAB'bi hosnut eden koku olarak yakilan sunudur.
\par 7 Kuzuyla birlikte dökmelik sunu olarak dörtte bir hin*fh* içki sunacaksiniz. Dökmelik sunuyu RAB için kutsal yerde dökeceksiniz.
\par 8 Öbür kuzuyu aksamüstü, yakilan sunu ve RAB'bi hosnut eden koku olarak, sabahki gibi tahil sunusu ve dökmelik sunuyla birlikte bana sunacaksiniz."
\par 9 "'Sabat Günü* bir yasinda kusursuz iki erkek kuzuyla birlikte tahil sunusu* olarak zeytinyagiyla yogrulmus onda iki efa ince un ve onun dökmelik sunusunu sunacaksiniz.
\par 10 Günlük yakmalik sunuyla* dökmelik sunusu disinda, her Sabat Günü sunulan yakmalik sunu budur."
\par 11 "'Her ayin ilk günü, RAB'be yakmalik sunu* olarak iki boga, bir koç ve bir yasinda kusursuz yedi erkek kuzu sunacaksiniz.
\par 12 Her bogayla birlikte tahil sunusu* olarak zeytinyagiyla yogrulmus onda üç efa ince un; koçla birlikte tahil sunusu olarak zeytinyagiyla yogrulmus onda iki efa ince un;
\par 13 her kuzuyla da tahil sunusu olarak zeytinyagiyla yogrulmus onda bir efa ince un sunacaksiniz. Bu, RAB'bi hosnut eden koku, yakilan sunu ve yakmalik sunu olacaktir.
\par 14 Her bogayla dökmelik sunu olarak yarim hin, koçla üçte bir hin, her kuzuyla dörtte bir hin sarap sunacaksiniz. Yil boyunca her Yeni Ay'da yapilacak yakmalik sunu budur.
\par 15 Günlük yakmalik sunuyla dökmelik sunusu disinda, RAB'be günah sunusu olarak bir teke sunulacak."
\par 16 "'RAB'bin Fisih* kurbani birinci ayin* on dördüncü günü kesilmelidir.
\par 17 On besinci gün bayram olacaktir; yedi gün mayasiz ekmek yiyeceksiniz.
\par 18 Ilk gün kutsal toplanti düzenleyecek, gündelik islerinizi yapmayacaksiniz.
\par 19 RAB için yakilan sunu, yakmalik sunu* olarak iki boga, bir koç ve bir yasinda yedi erkek kuzu sunacaksiniz. Sunacaginiz hayvanlar kusursuz olmali.
\par 20 Her bogayla birlikte tahil sunusu* olarak zeytinyagiyla yogrulmus onda üç efa, koçla onda iki efa, her kuzuyla onda bir efa ince un sunacaksiniz;
\par 22 günahlarinizi bagislatmak için de günah sunusu* olarak bir teke sunacaksiniz.
\par 23 Her sabah sunacaginiz günlük yakmalik sunuya ek olarak bunlari da sunacaksiniz.
\par 24 Böylece her gün RAB'bi hosnut eden koku olarak yakilan yiyecek sunusunu yedi gün sunacaksiniz. Bunu günlük yakmalik sunuyla dökmelik sunusuna ek olarak sunacaksiniz.
\par 25 Yedinci gün kutsal toplanti düzenleyecek, gündelik islerinizi yapmayacaksiniz."
\par 26 "'Ilk ürünleri kutlama günü, Haftalar Bayrami'nda* RAB'be yeni tahil sunusu* sundugunuzda kutsal toplanti düzenleyecek, gündelik islerinizi yapmayacaksiniz.
\par 27 RAB'bi hosnut eden koku, yakmalik sunu* olarak iki boga, bir koç ve bir yasinda yedi erkek kuzu sunacaksiniz.
\par 28 Her bogayla birlikte tahil sunusu olarak zeytinyagiyla yogrulmus onda üç efa, koçla birlikte onda iki efa, her kuzuyla da onda bir efa ince un sunacaksiniz;
\par 30 günahlarinizi bagislatmak için de bir teke sunacaksiniz.
\par 31 Günlük yakmalik sunuyla tahil sunusuna ek olarak bunlari dökmelik sunuyla birlikte sunacaksiniz. Sunacaginiz hayvanlar kusursuz olmali."

\chapter{29}

\par 1 "'Yedinci ayin* birinci günü kutsal toplanti düzenleyecek, gündelik islerinizi yapmayacaksiniz. O gün sizin için boru çalma günü olacak.
\par 2 RAB'bi hosnut eden koku, yakmalik sunu* olarak kusursuz bir boga, bir koç ve bir yasinda yedi erkek kuzu sunacaksiniz.
\par 3 Bogayla birlikte tahil sunusu* olarak zeytinyagiyla yogrulmus onda üç efa, koçla birlikte onda iki efa, her kuzuyla da onda bir efa ince un sunacaksiniz.
\par 5 Günahlarinizi bagislatmak için de günah sunusu* olarak bir teke sunacaksiniz.
\par 6 Kurala göre sunacaginiz aylik ve günlük yakmalik sunuyla dökmelik sunulara, tahil sunularina ek olarak bunlari sunacaksiniz. Bunlar RAB'bi hosnut eden koku olarak yakilan sunulardir."
\par 7 "'Yedinci ayin* onuncu günü kutsal bir toplanti düzenleyeceksiniz. O gün isteklerinizi denetleyecek, hiç is yapmayacaksiniz.
\par 8 RAB'bi hosnut eden koku, yakmalik sunu* olarak bir boga, bir koç ve bir yasinda yedi erkek kuzu sunacaksiniz. Sunacaginiz hayvanlar kusursuz olmali.
\par 9 Bogayla birlikte tahil sunusu* olarak zeytinyagiyla yogrulmus onda üç efa, koçla birlikte onda iki efa,
\par 10 her kuzuyla da onda bir efa ince un sunacaksiniz.
\par 11 Günah sunusu* için bir teke sunacaksiniz. Günahlarinizi bagislatmak için sunulan günah sunusu, günlük yakmalik sunuyla dökmelik ve tahil sunularina ek olarak bunlari da sunacaksiniz."
\par 12 "'Yedinci ayin* on besinci günü kutsal bir toplanti düzenleyecek, gündelik islerinizi yapmayacaksiniz. Bu bayrami RAB'bin onuruna yedi gün kutlayacaksiniz.
\par 13 RAB'bi hosnut eden koku olarak yakilan sunu, yakmalik sunu* olarak on üç boga, iki koç ve bir yasinda on dört erkek kuzu sunacaksiniz. Bu hayvanlar kusursuz olmali.
\par 14 Her bogayla birlikte tahil sunusu* olarak zeytinyagiyla yogrulmus onda üç efa, her koçla birlikte onda iki efa, her kuzuyla da onda bir efa ince un sunacaksiniz.
\par 16 Günah sunusu* için bir teke sunacaksiniz. Bu sunular günlük yakmalik sunuyla tahil sunularina ve dökmelik sunulara ek olacak.
\par 17 "'Ikinci gün kusursuz on iki boga, iki koç ve bir yasinda on dört erkek kuzu sunacaksiniz.
\par 18 Boga, koç ve kuzularla sayisina göre istenilen tahil sunularini ve dökmelik sunulari sunacaksiniz.
\par 19 Günah sunusu için bir teke sunacaksiniz. Bu sunular günlük yakmalik sunuyla tahil sunularina ve dökmelik sunulara ek olacak.
\par 20 "'Üçüncü gün kusursuz on bir boga, iki koç ve bir yasinda on dört erkek kuzu sunacaksiniz.
\par 21 Boga, koç ve kuzularla sayisina göre istenilen tahil sunularini ve dökmelik sunulari sunacaksiniz.
\par 22 Günah sunusu için bir teke sunacaksiniz. Bu sunular günlük yakmalik sunuyla tahil sunularina ve dökmelik sunulara ek olacak.
\par 23 "'Dördüncü gün kusursuz on boga, iki koç ve bir yasinda on dört erkek kuzu sunacaksiniz.
\par 24 Boga, koç ve kuzularla sayisina göre istenilen tahil sunularini ve dökmelik sunulari sunacaksiniz.
\par 25 Günah sunusu için bir teke sunacaksiniz. Bu sunular günlük yakmalik sunuyla tahil sunularina ve dökmelik sunulara ek olacak.
\par 26 "'Besinci gün kusursuz dokuz boga, iki koç ve bir yasinda on dört erkek kuzu sunacaksiniz.
\par 27 Boga, koç ve kuzularla sayisina göre istenilen tahil sunularini ve dökmelik sunulari sunacaksiniz.
\par 28 Günah sunusu için bir teke sunacaksiniz. Bu sunular günlük yakmalik sunuyla tahil sunularina ve dökmelik sunulara ek olacak.
\par 29 "'Altinci gün kusursuz sekiz boga, iki koç ve bir yasinda on dört erkek kuzu sunacaksiniz.
\par 30 Boga, koç ve kuzularla sayisina göre istenilen tahil sunularini ve dökmelik sunulari sunacaksiniz.
\par 31 Günah sunusu için bir teke sunacaksiniz. Bu sunular günlük yakmalik sunuyla tahil sunularina ve dökmelik sunulara ek olacak.
\par 32 "'Yedinci gün kusursuz yedi boga, iki koç ve bir yasinda on dört erkek kuzu sunacaksiniz.
\par 33 Boga, koç ve kuzularla sayisina göre istenilen tahil sunularini ve dökmelik sunulari sunacaksiniz.
\par 34 Günah sunusu için bir teke sunacaksiniz. Bu sunular günlük yakmalik sunuyla tahil sunularina ve dökmelik sunulara ek olacak.
\par 35 "'Sekizinci gün bir toplanti düzenleyecek, gündelik islerinizi yapmayacaksiniz.
\par 36 Yakmalik sunu, yakilan sunu, RAB'bi hosnut eden koku olarak kusursuz bir boga, bir koç ve bir yasinda yedi erkek kuzu sunacaksiniz.
\par 37 Boga, koç ve kuzularla sayisina göre istenilen tahil sunularini ve dökmelik sunulari sunacaksiniz.
\par 38 Günah sunusu için bir teke sunacaksiniz. Bu sunular günlük yakmalik sunuyla tahil sunularina ve dökmelik sunulara ek olacak.
\par 39 "'Adadiginiz adaklar ve gönülden verdiginiz sunularin yanisira, bayramlarinizda RAB'be yakmalik, tahil, dökmelik ve esenlik sunulariniz* olarak bunlari sunun."
\par 40 Musa RAB'bin kendisine buyurdugu her seyi Israilliler'e anlatti.

\chapter{30}

\par 1 Musa Israil'in oymak önderlerine söyle dedi: "RAB söyle buyurdu:
\par 2 'Eger bir adam RAB'be adak adar ya da ant içerek kendini yükümlülük altina sokarsa, verdigi sözü bozmayacak, agzindan her çikani yerine getirecek.
\par 3 "'Genç bir kadin babasinin evindeyken RAB'be adak adar ya da kendini yükümlülük altina sokarsa,
\par 4 babasi da onun RAB'be adadigi adagi ve kendini yükümlülük altina soktugunu duyar, ona karsi çikmazsa, kadinin adadigi adaklar ve kendini altina soktugu yükümlülük geçerli sayilacak.
\par 5 Ama babasi bunlari duydugu gün engel olursa, kadinin adadigi adaklar ve kendini altina soktugu yükümlülük geçerli sayilmayacak; RAB onu bagislayacak, çünkü babasi ona engel olmustur.
\par 6 "'Eger kadin adak adadiktan ya da düsünmeden kendini yükümlülük altina soktuktan sonra evlenirse,
\par 7 kocasi da bunu duyar ve ayni gün ona karsi çikmazsa, adadigi adaklar ve kendini altina soktugu yükümlülük geçerli sayilacak.
\par 8 Ama kocasi bunu duydugu gün engel olur, kadinin adadigi adagi ya da düsünmeden kendini altina soktugu yükümlülügü geçerli saymazsa, RAB kadini bagislayacaktir.
\par 9 "'Dul ya da bosanmis bir kadinin adadigi adak, kendini yükümlülük altina soktugu her sey geçerli sayilacak.
\par 10 "'Eger bir kadin evliyken bir adak adar ya da ant içerek kendini yükümlülük altina sokarsa,
\par 11 kocasi da bunu duyar, karsi çikmaz, ona engel olmazsa, kadinin adadigi bütün adaklar ya da kendini altina soktugu her yükümlülük geçerli sayilacak.
\par 12 Ama kocasi bunlari duydugu gün engel olursa, kadinin adadigi bütün adaklar ve kendini altina soktugu yükümlülük geçerli sayilmayacak. Kocasi geçersiz kilmistir, RAB kadini bagislayacak.
\par 13 Kocasi, kadinin kendi isteklerini denetlemesi için adadigi adagi ya da ant içerek kendini altina soktugu yükümlülügü onaylayabilir ya da geçersiz kilabilir.
\par 14 Eger kocasi bir gün içinde bu konuda ona karsi çikmazsa, bütün adaklarini ya da yükümlülüklerini onaylamis olur. Onlari duydugu gün kadina karsi çikmamakla onaylamis sayilir.
\par 15 Eger onlari duyduktan bir süre sonra engel olursa, kadinin suçundan kocasi sorumlu olacaktir."
\par 16 Erkekle karisi, babayla evinde yasayan genç kizi arasindaki iliski konusunda RAB'bin Musa'ya buyurdugu kurallar bunlardir.

\chapter{31}

\par 1 RAB Musa'ya, "Midyanlilar'dan Israilliler'in öcünü al; sonra ölüp atalarina kavusacaksin" dedi.
\par 3 Bunun üzerine Musa halka, "Midyanlilar'a karsi savasmak ve onlardan RAB'bin öcünü almak üzere aranizdan adamlar silahlandirin" dedi,
\par 4 "Savasa Israil'in her oymagindan bin kisi gönderin."
\par 5 Böylece Israil'in her oymagindan biner kisi olmak üzere 12.000 kisi seçilip savasa hazirlandi.
\par 6 Musa onlari -her oymaktan biner kisiyi- ve Kâhin Elazar oglu Pinehas'i savasa gönderdi. Pinehas yanina kutsal yere ait bazi esyalari ve çagri borazanlarini aldi.
\par 7 RAB'bin Musa'ya verdigi buyruk uyarinca, Midyanlilar'a savas açip bütün erkekleri öldürdüler.
\par 8 Öldürdükleri arasinda bes Midyan krali -Evi, Rekem, Sur, Hur ve Reva- da vardi. Beor oglu Balam'i da kiliçla öldürdüler.
\par 9 Midyanli kadinlarla çocuklarini tutsak alip bütün hayvanlarini, sürülerini, mallarini yagmaladilar.
\par 10 Midyanlilar'in yasadigi bütün kentleri, obalari atese verdiler.
\par 11 Insanlari, hayvanlari, yagmalanmis bütün mallari yanlarina aldilar.
\par 12 Tutsaklarla yagmalanmis mallari Seria Irmagi'nin yaninda, Eriha karsisinda, Moav ovalarindaki ordugahta konaklayan Musa'yla Kâhin Elazar'a ve Israil topluluguna getirdiler.
\par 13 Musa, Kâhin Elazar ve toplulugun önderleri onlari karsilamak için ordugahin disina çiktilar.
\par 14 Musa savastan dönen ordu komutanlarina -binbasilara, yüzbasilara- öfkelendi.
\par 15 Onlara, "Bütün kadinlari sag mi biraktiniz?" diye çikisti,
\par 16 "Bu kadinlar Balam'in verdigi ögüde uyarak Peor olayinda Israilliler'in RAB'be ihanet etmesine neden oldular. Bu yüzden RAB'bin toplulugu arasinda ölümcül hastalik basgösterdi.
\par 17 Simdi bütün erkek çocuklari ve erkekle yatmis kadinlari öldürün.
\par 18 Yalniz erkekle yatmamis genç kizlari kendiniz için sag birakin.
\par 19 "Aranizda birini öldüren ya da öldürülen birine dokunan herkes yedi gün ordugahin disinda kalsin. Üçüncü ve yedinci gün kendinizi de tutsaklarinizi da günahtan arindiracaksiniz.
\par 20 Her giysiyi, deriden, keçi kilindan, tahtadan yapilmis her nesneyi arindiracaksiniz."
\par 21 Bundan sonra Kâhin Elazar, savastan dönen askerlere, "RAB'bin Musa'ya buyurdugu yasanin kurali sudur" dedi,
\par 22 "Altini, gümüsü, tuncu*, demiri, kalayi, kursunu -atese dayanikli her nesneyi- atesten geçireceksiniz; ancak bundan sonra temiz sayilacak. Ayrica temizlenme suyuyla da arindiracaksiniz. Atese dayanikli olmayan nesneleri sudan geçireceksiniz.
\par 24 Yedinci gün giysilerinizi yikayin. Böylece temiz sayilacaksiniz. Sonra ordugaha girebilirsiniz."
\par 25 RAB Musa'ya söyle dedi:
\par 26 "Sen, Kâhin Elazar ve toplulugun aile baslari ele geçirilen insanlarla hayvanlari sayacaksiniz.
\par 27 Ele geçirilenleri savasa katilan askerlerle toplulugun geri kalani arasinda paylastiracaksiniz.
\par 28 Savasa katilan askerlere düsen paydan -insan, sigir, esek, davardan- vergi olarak RAB'be bes yüzde bir pay ayiracaksin.
\par 29 Bu vergiyi askerlere düsen yari paydan alacak, RAB'be armagan olarak Kâhin Elazar'a vereceksin.
\par 30 Öbür Israilliler'e düsen yaridan, gerek insanlardan, gerek hayvanlardan -sigir, esek, davardan- ellide birini alip RAB'bin Konutu'nun hizmetinden sorumlu olan Levililer'e vereceksin."
\par 31 Musa'yla Kâhin Elazar RAB'bin Musa'ya buyurdugu gibi yaptilar.
\par 32 Savasa katilan askerlerin ele geçirdiklerinden kalanlar sunlardi: 675 000 davar,
\par 33 72 000 sigir,
\par 34 61 000 esek,
\par 35 erkekle yatmamis 32 000 kiz.
\par 36 Savasa katilan askerlere düsen yari pay da suydu: 337 500 davar,
\par 37 bunlardan RAB'be vergi olarak 675 davar verildi;
\par 38 36 000 sigir, bunlardan RAB'be vergi olarak 72 sigir verildi;
\par 39 30 500 esek, bunlardan RAB'be vergi olarak 61 esek verildi;
\par 40 16 000 kisi, bunlardan RAB'be vergi olarak 32 kisi verildi.
\par 41 Musa, RAB'bin kendisine buyurdugu gibi, RAB'be ayrilan vergiyi Kâhin Elazar'a verdi.
\par 42 Musa'nin savasa katilan askerlerden alip Israilliler'e ayirdigi yari pay suydu:
\par 43 Topluluga düsen yari pay 337 500 davar,
\par 44 36 000 sigir,
\par 45 30 500 esek,
\par 46 16 000 kisi.
\par 47 Musa, RAB'bin kendisine buyurdugu gibi, Israilliler'e düsen yari paydan her elli kisiden ve hayvandan birini alip RAB'bin Konutu'nun hizmetinden sorumlu olan Levililer'e verdi.
\par 48 Ordu komutanlari -binbasilar ve yüzbasilar- Musa'ya gidip,
\par 49 "Efendimiz, yönetimimiz altindaki askerleri saydik, eksik yok" dediler,
\par 50 "Iste, ele geçirdigimiz altin esyalari -pazibentleri, bilezikleri, yüzükleri, küpeleri, kolyeleri- getirdik. Günahlarimizi bagislatmak için bunlari RAB'be sunuyoruz."
\par 51 Musa'yla Kâhin Elazar altini, her tür islenmis altin esyayi onlardan aldilar.
\par 52 Binbasi ve yüzbasilardan alip RAB'be armagan olarak sunduklari altinin toplam agirligi 16 750 sekeldi.
\par 53 Savasa katilan her asker kendine yagmalanmis maldan almisti.
\par 54 Musa'yla Kâhin Elazar binbasi ve yüzbasilardan aldiklari altini Israilliler için RAB'bin önünde bir animsatma sunusu olarak Bulusma Çadiri'na getirdiler.

\chapter{32}

\par 1 Çok sayida hayvani olan Rubenliler'le Gadlilar Yazer ve Gilat topraklarinin hayvanlar için uygun bir yer oldugunu gördüler.
\par 2 Musa'yla Kâhin Elazar'a ve toplulugun önderlerine giderek, "RAB'bin yardimiyla Israil halkinin ele geçirdigi Atarot, Divon, Yazer, Nimra, Hesbon, Elale, Sevam, Nevo, Beon kentlerini içeren bölge hayvanlar için uygun bir yer" dediler, "Kullarinizin da hayvanlari var.
\par 5 Bizden hosnut kaldiysaniz, bu ülkeyi mülk olarak bize verin ki, Seria Irmagi'nin karsi yakasina geçmek zorunda kalmayalim."
\par 6 Musa, "Israilli kardesleriniz savasa giderken siz burada mi kalacaksiniz?" diye karsilik verdi,
\par 7 "RAB'bin kendilerine verecegi ülkeye giden Israilliler'in neden cesaretini kiriyorsunuz?
\par 8 Ülkeyi arastirsinlar diye Kades-Barnea'dan gönderdigim babalariniz da ayni seyi yaptilar.
\par 9 Eskol Vadisi'ne kadar gidip ülkeyi gördükten sonra, RAB'bin kendilerine verecegi ülkeye gitmemeleri için Israilliler'in gözünü korkuttular.
\par 10 O gün RAB öfkelenerek söyle ant içti:
\par 11 'Madem bütün yürekleriyle ardimca yürümediler, Misir'dan çikanlardan yirmi ve daha yukari yastakilerin hiçbiri Ibrahim'e, Ishak'a, Yakup'a ant içerek söz verdigim ülkeyi görmeyecek.
\par 12 Kenaz soyundan Yefunne oglu Kalev'le Nun oglu Yesu'dan baskasi orayi görmeyecek. Çünkü onlar bütün yürekleriyle ardimca yürüdüler.
\par 13 Israilliler'e öfkelenen RAB, gözünde kötülük yapan o kusak büsbütün yok oluncaya dek kirk yil onlari çölde dolastirdi.
\par 14 "Iste, ey günahkârlar soyu, babalarinizin yerine siz geçtiniz ve RAB'bin Israil'e daha çok öfkelenmesine neden oluyorsunuz.
\par 15 Eger O'nun ardinca yürümekten vazgeçerseniz, bütün bu halki yine çölde birakacak; siz de bu halkin yok olmasina neden olacaksiniz."
\par 16 Gadlilar'la Rubenliler Musa'ya yaklasip, "Burada hayvanlarimiz için agillar yapmamiza, çocuklarimiz için yeniden kentler kurmamiza izin ver" dediler,
\par 17 "Kendimiz de hemen silahlanip Israilliler'i kendilerinin olacak ülkeye götürünceye dek onlara öncülük edecegiz. Ülke halki yüzünden çocuklarimiz surlu kentlerde yasayacak.
\par 18 Her Israilli mülküne kavusmadan evlerimize dönmeyecegiz.
\par 19 Seria Irmagi'nin karsi yakasinda onlarla birlikte mülk almayacagiz, çünkü bizim payimiz Seria Irmagi'nin dogu yakasina düstü."
\par 20 Musa söyle yanitladi: "Bu söylediklerinizi yapar, RAB'bin önünde savasa gitmek üzere silahlanip
\par 21 RAB düsmanlarini kovuncaya dek hepiniz O'nun önünde Seria Irmagi'nin karsi yakasina silahli olarak geçerseniz,
\par 22 ülke ele geçirildiginde dönebilir, RAB'be ve Israil'e karsi yükümlülügünüzden özgür olursunuz. RAB'bin önünde bu topraklar sizin olacaktir.
\par 23 "Ama söylediklerinizi yapmazsaniz, RAB'be karsi günah islemis olursunuz; günahinizin cezasini çekeceginizi bilmelisiniz.
\par 24 Çocuklariniz için yeniden kentler kurun, davarlariniz için agillar yapin. Yeter ki, verdiginiz sözü yerine getirin."
\par 25 Gadlilar'la Rubenliler, "Efendimiz, bize buyurdugun gibi yapacagiz" diye yanitladilar,
\par 26 "Çoluk çocugumuz, sigirlarimizla öbür hayvanlarimiz burada, Gilat kentlerinde kalacak.
\par 27 Ama buyurdugun gibi, silahlanmis olan herkes RAB'bin önünde savasmak üzere karsi yakaya geçecek."
\par 28 Musa Gadlilar'la Rubenliler hakkinda Kâhin Elazar'a, Nun oglu Yesu'ya ve Israil oymaklarinin aile baslarina buyruk verdi.
\par 29 Söyle dedi: "Gadlilar'la Rubenliler'den silahlanmis olan herkes RAB'bin önünde sizinle birlikte Seria Irmagi'nin karsi yakasina geçerse, ülkeyi de ele geçirirseniz, Gilat bölgesini miras olarak onlara vereceksiniz.
\par 30 Ama silahlanmis olarak sizinle irmagin karsi yakasina geçmezlerse, Kenan ülkesinde sizinle miras alacaklar."
\par 31 Gadlilar'la Rubenliler, "RAB'bin bize buyurdugunu yapacagiz" dediler,
\par 32 "Silahlanmis olarak RAB'bin önünde Kenan ülkesine gidecegiz. Ama alacagimiz mülk Seria Irmagi'nin dogu yakasinda olacak."
\par 33 Böylece Musa Gadlilar'la Rubenliler'e ve Yusuf oglu Manasse oymaginin yarisina Amorlular'in Krali Sihon'un ülkesiyle Basan Krali Og'un ülkesini ve çevrelerindeki topraklarla kentleri verdi.
\par 34 Gadlilar surlu Divon, Atarot, Aroer, Atrot-Sofan, Yazer, Yogboha, Beytnimra ve Beytharan kentlerini yeniden kurdular, davarlari için agillar yaptilar.
\par 37 Rubenliler Hesbon, Elale, Kiryatayim, Nevo, Baal-Meon -bu son iki ad degistirildi- ve Sivma kentlerini yeniden kurdular. Kurduklari kentlere yeni adlar verdiler.
\par 39 Manasse oglu Makir'in soyundan gelenler gidip Gilat'i ele geçirerek, orada yasayan Amorlular'i kovdular.
\par 40 Böylece Musa Gilat'i Manasse oglu Makir'in soyundan gelenlere verdi; onlar da oraya yerlestiler.
\par 41 Manasse soyundan Yair gidip Amorlular'in yerlesim birimlerini ele geçirdi ve bunlara Havvot-Yair adini verdi.
\par 42 Novah da Kenat'la çevresindeki köyleri ele geçirerek oraya kendi adini verdi.

\chapter{33}

\par 1 Musa'yla Harun önderliginde birlikler halinde Misir'dan çikan Israilliler sirasiyla asagidaki yolculuklari yaptilar.
\par 2 Musa RAB'bin buyrugu uyarinca sirasiyla yapilan yolculuklari kayda geçirdi. Yapilan yolculuklar sunlardir:
\par 3 Israilliler Fisih* kurbaninin ertesi günü -birinci ayin* on besinci günü- Misirlilar'in gözü önünde zafer havasi içinde Ramses'ten yola çiktilar.
\par 4 O sirada Misirlilar RAB'bin yok ettigi ilk dogan çocuklarini gömüyorlardi; RAB onlarin ilahlarini yargilamisti.
\par 5 Israilliler Ramses'ten yola çikip Sukkot'ta konakladilar.
\par 6 Sukkot'tan ayrilip çöl kenarindaki Etam'da konakladilar.
\par 7 Etam'dan ayrilip Baal-Sefon'un dogusundaki Pi-Hahirot'a döndüler, Migdol yakinlarinda konakladilar.
\par 8 Pi-Hahirot'tan ayrilip denizden çöle geçtiler. Etam Çölü'nde üç gün yürüdükten sonra Mara'da konakladilar.
\par 9 Mara'dan ayrilip on iki su kaynagi ve yetmis hurma agaci olan Elim'e giderek orada konakladilar.
\par 10 Elim'den ayrilip Kizildeniz* kiyisinda konakladilar.
\par 11 Kizildeniz'den ayrilip Sin Çölü'nde konakladilar.
\par 12 Sin Çölü'nden ayrilip Dofka'da konakladilar.
\par 13 Dofka'dan ayrilip Alus'ta konakladilar.
\par 14 Alus'tan ayrilip Refidim'de konakladilar. Orada halk için içecek su yoktu.
\par 15 Refidim'den ayrilip Sina Çölü'nde konakladilar.
\par 16 Sina Çölü'nden ayrilip Kivrot-Hattaava'da konakladilar.
\par 17 Kivrot-Hattaava'dan ayrilip Haserot'ta konakladilar.
\par 18 Haserot'tan ayrilip Ritma'da konakladilar.
\par 19 Ritma'dan ayrilip Rimmon-Peres'te konakladilar.
\par 20 Rimmon-Peres'ten ayrilip Livna'da konakladilar.
\par 21 Livna'dan ayrilip Rissa'da konakladilar.
\par 22 Rissa'dan ayrilip Kehelata'da konakladilar.
\par 23 Kehelata'dan ayrilip Sefer Dagi'nda konakladilar.
\par 24 Sefer Dagi'ndan ayrilip Harada'da konakladilar.
\par 25 Harada'dan ayrilip Makhelot'ta konakladilar.
\par 26 Makhelot'tan ayrilip Tahat'ta konakladilar.
\par 27 Tahat'tan ayrilip Terah'ta konakladilar.
\par 28 Terah'tan ayrilip Mitka'da konakladilar.
\par 29 Mitka'dan ayrilip Hasmona'da konakladilar.
\par 30 Hasmona'dan ayrilip Moserot'ta konakladilar.
\par 31 Moserot'tan ayrilip Bene-Yaakan'da konakladilar.
\par 32 Bene-Yaakan'dan ayrilip Hor-Hagidgat'ta konakladilar.
\par 33 Hor-Hagidgat'tan ayrilip Yotvata'da konakladilar.
\par 34 Yotvata'dan ayrilip Avrona'da konakladilar.
\par 35 Avrona'dan ayrilip Esyon-Gever'de konakladilar.
\par 36 Esyon-Gever'den ayrilip Zin Çölü'nde -Kades'te- konakladilar.
\par 37 Kades'ten ayrilip Edom sinirindaki Hor Dagi'nda konakladilar.
\par 38 Kâhin Harun RAB'bin buyrugu uyarinca Hor Dagi'na çikti. Israilliler'in Misir'dan çikislarinin kirkinci yili, besinci ayin birinci günü orada öldü.
\par 39 Hor Dagi'nda öldügünde Harun 123 yasindaydi.
\par 40 Kenan ülkesinin Negev bölgesinde yasayan Kenanli Arat Krali Israilliler'in geldigini duydu.
\par 41 Israilliler Hor Dagi'ndan ayrilip Salmona'da konakladilar.
\par 42 Salmona'dan ayrilip Punon'da konakladilar.
\par 43 Punon'dan ayrilip Ovot'ta konakladilar.
\par 44 Ovot'tan ayrilip Moav sinirindaki Iye-Haavarim'de konakladilar.
\par 45 Iyim'den ayrilip Divon-Gad'da konakladilar.
\par 46 Divon-Gad'dan ayrilip Almon-Divlatayma'da konakladilar.
\par 47 Almon-Divlatayma'dan ayrilip Nevo yakinlarindaki Haavarim daglik bölgesinde konakladilar.
\par 48 Haavarim daglik bölgesinden ayrilip Seria Irmagi yaninda, Eriha karsisindaki Moav ovalarinda konakladilar.
\par 49 Seria Irmagi boyunca Beythayesimot'tan Avel-Hassittim'e kadar Moav ovalarinda konakladilar.
\par 50 Orada, Seria Irmagi yaninda Eriha karsisindaki Moav ovalarinda RAB Musa'ya söyle dedi:
\par 51 "Israilliler'e de ki, 'Seria Irmagi'ndan Kenan ülkesine geçince,
\par 52 ülkede yasayan bütün halki kovacaksiniz. Oyma ve dökme putlarini yok edecek, tapinma yerlerini yikacaksiniz.
\par 53 Ülkeyi yurt edinecek, oraya yerleseceksiniz; çünkü mülk edinesiniz diye orayi size verdim.
\par 54 Ülkeyi boylariniz arasinda kurayla paylasacaksiniz. Büyük boya büyük pay, küçük boya küçük pay vereceksiniz. Kurada kime ne çikarsa, orasi onun olacak. Dagitimi atalarinizin oymaklarina göre yapacaksiniz.
\par 55 "'Ama ülkede yasayanlari kovmazsaniz, orada biraktiginiz halk gözlerinizde kanca, bögürlerinizde diken olacak. Yasayacaginiz ülkede size sikinti verecekler.
\par 56 Ben de onlara yapmayi tasarladigimi size yapacagim."

\chapter{34}

\par 1 RAB Musa'ya söyle dedi:
\par 2 "Israilliler'e de ki, 'Mülk olarak size düsecek Kenan ülkesine girince, sinirlariniz söyle olacak:
\par 3 "'Güney siniriniz Zin Çölü'nden Edom siniri boyunca uzanacak. Doguda, güney siniriniz Lut Gölü'nün ucundan baslayacak,
\par 4 Akrep Geçidi'nin güneyinden Zin'e geçip Kades-Barnea'nin güneyine dek uzanacak. Oradan Hasar-Addar'a ve Asmon'a,
\par 5 oradan da Misir Vadisi'ne uzanarak Akdeniz'de son bulacak.
\par 6 "'Bati siniriniz Akdeniz ve kiyisi olacak. Batida siniriniz bu olacak.
\par 7 "'Kuzey siniriniz Akdeniz'den Hor Dagi'na dek uzanacak.
\par 8 Hor Dagi'ndan Levo-Hamat'a, oradan Sedat'a,
\par 9 Zifron'a dogru uzanarak Hasar-Enan'da son bulacak. Kuzeyde siniriniz bu olacak.
\par 10 "'Dogu siniriniz Hasar-Enan'dan Sefam'a dek uzanacak.
\par 11 Siniriniz Sefam'dan Ayin'in dogusundaki Rivla'ya dek inecek. Oradan Kinneret Gölü'nün*fs* dogu kiyisindaki yamaçlara dek uzanacak.
\par 12 Oradan Seria Irmagi boyunca uzanacak ve Lut Gölü'nde son bulacak. "'Her yandan ülkenizin sinirlari bu olacaktir."
\par 13 Musa Israilliler'e, "Miras olarak kurayla paylastiracaginiz ülke budur" dedi, "RAB'bin buyrugu uyarinca ülke dokuz oymakla bir yarim oymak arasinda paylastirilacak.
\par 14 Çünkü Ruben oymagina bagli ailelerle Gad oymagina bagli aileler ve Manasse oymaginin öbür yarisi mülklerini aldilar.
\par 15 Bu iki oymakla yarim oymak mülklerini Eriha'nin karsisindaki Seria Irmagi'nin dogusunda aldilar."
\par 16 RAB Musa'ya söyle dedi:
\par 17 "Ülkeyi mülk olarak aranizda paylastiracak adamlar sunlardir: Kâhin Elazar ve Nun oglu Yesu.
\par 18 Ülkeyi mülk olarak paylastirmalari için her oymaktan birer önder görevlendirin.
\par 19 Su adamlari görevlendireceksiniz: "Yahuda oymagindan Yefunne oglu Kalev,
\par 20 Simon oymagindan Ammihut oglu Semuel,
\par 21 Benyamin oymagindan Kislon oglu Elidat,
\par 22 Dan oymagindan Yogli oglu önder Bukki,
\par 23 Yusufogullari'ndan: Manasse oymagindan Efot oglu önder Hanniel,
\par 24 Efrayim oymagindan Siftan oglu önder Kemuel,
\par 25 Zevulun oymagindan Parnak oglu önder Elisafan,
\par 26 Issakar oymagindan Azzan oglu önder Paltiel,
\par 27 Aser oymagindan Selomi oglu önder Ahihut,
\par 28 Naftali oymagindan Ammihut oglu önder Pedahel."
\par 29 Kenan ülkesinde Israilliler'e mülkü paylastirmalari için RAB'bin görevlendirdigi adamlar bunlardi.

\chapter{35}

\par 1 RAB Seria Irmagi yaninda Eriha karsisindaki Moav ovalarinda Musa'ya söyle dedi:
\par 2 "Israilliler'e buyruk ver, alacaklari mülkten oturmalari için Levililer'e kentler versinler. Kentlerin çevresinde otlaklar da vereceksiniz.
\par 3 Böylece yasamak için Levililer'in kentleri olacak; sigirlari, sürüleri, öbür hayvanlari için otlaklari da olacak.
\par 4 "Levililer'e vereceginiz kentlerin çevresindeki otlaklar kent surundan bin arsin uzaklikta olacak.
\par 5 Kent ortada olmak üzere, kent disindan doguda iki bin arsin, güneyde iki bin arsin, batida iki bin arsin, kuzeyde iki bin arsin ölçeceksiniz. Bu bölge kentler için otlak olacak."
\par 6 "Levililer'e vereceginiz kentlerden altisi siginak kent olacak; öyle ki, adam öldüren biri oraya kaçabilsin. Ayrica Levililer'e kirk iki kent daha vereceksiniz.
\par 7 Levililer'e otlaklariyla birlikte toplam kirk sekiz kent vereceksiniz.
\par 8 Israilliler'in mülkünden Levililer'e vereceginiz kentler her oymaga düsen pay oraninda olsun. Çok kenti olan oymak çok, az kenti olan oymak az sayida kent verecek."
\par 9 RAB Musa'yla konusmasini söyle sürdürdü:
\par 10 "Israilliler'e de ki, 'Seria Irmagi'ndan geçip Kenan ülkesine girince,
\par 11 siginak kentler olarak bazi kentler seçin. Öyle ki, istemeyerek birini öldüren kisi oraya kaçabilsin;
\par 12 öç alacak kisiden kaçip siginacak bir yeriniz olsun. Böylece adam öldüren kisi toplulugun önünde yargilanmadan öldürülmesin.
\par 13 Vereceginiz bu alti kent sizin için siginak kentler olacak.
\par 14 Siginak kentlerin üçünü Seria Irmagi'nin dogusundan, üçünü de Kenan ülkesinden seçeceksiniz.
\par 15 Bu alti kent Israilliler ve aralarinda yasayan yabancilarla yerli olmayan konuklar için siginak kentler olacak. Öyle ki, istemeyerek birini öldüren kisi oraya kaçabilsin.
\par 16 "'Eger biri demir bir aletle baska birine vurur, o kisi de ölürse, adam katildir ve kesinlikle öldürülecektir.
\par 17 Birinin elinde adam öldürebilecek bir tas varsa, bu tasla baska birine vurursa, o kisi de ölürse, adam katildir ve kesinlikle öldürülecektir.
\par 18 Ya da elinde adam öldürebilecek tahtadan bir alet varsa, bununla birine vurursa, o kisi de ölürse, adam katildir ve kesinlikle öldürülecektir.
\par 19 Ölenin öcünü alacak kisi, katili öldürecektir; onunla karsilasinca onu öldürecektir.
\par 20 Eger biri baska birine besledigi kinden ötürü onu iter ya da bile bile ona bir nesne firlatirsa, o kisi de ölürse,
\par 21 ya da besledigi kinden ötürü onu yumruklar, o kisi de ölürse, vuran kisi kesinlikle öldürülecektir; katildir. Ölenin öcünü alacak kisi katille karsilasinca onu öldürecektir.
\par 22 "'Eger biri bir baskasina kin beslemedigi halde ansizin onu iter ya da istemeyerek ona bir nesne firlatirsa,
\par 23 ya da onu görmeden üzerine öldürebilecek bir tas düsürürse, o kisi de ölürse, öldüren ölene kin beslemediginden ve ona zarar vermek istemediginden,
\par 24 topluluk adam öldürenle kan öcünü alacak kisi arasinda su kurallar uyarinca karar verecek:
\par 25 Topluluk adam öldüreni kan öcü alacak kisinin elinden korumali ve kaçmis oldugu siginak kente geri göndermeli. Kisi kutsal yagla meshedilmis* baskâhinin ölümüne dek orada kalmalidir.
\par 26 "'Ama adam öldüren kaçmis oldugu siginak kentin sinirini geçer,
\par 27 kan öcü alacak kisi de onu siginak kentin siniri disinda görür, kan öcü alacak kisi öldüreni öldürürse suçlu sayilmayacaktir.
\par 28 Çünkü adam öldüren, baskâhinin ölümüne dek siginak kentte kalmali. Ancak onun ölümünden sonra kendi topragina dönebilir.
\par 29 "'Bunlar kusaklar boyunca yasadiginiz her yerde sizin için kesin kural olacaktir.
\par 30 "'Adam öldüren, taniklarin tanikligiyla öldürülecek, bir tek kisinin tanikligiyla öldürülmeyecektir.
\par 31 "'Ölümü hak etmis katilin cani için bedel almayacaksiniz; o kesinlikle öldürülecektir.
\par 32 "'Siginak kente kaçmis olan birinin baskâhinin ölümünden önce topragina dönüp yasamasi için bedel almayacaksiniz.
\par 33 "'Içinde yasadiginiz ülkeyi kirletmeyeceksiniz. Kan dökmek ülkeyi kirletir. Içinde kan dökülen ülke ancak kan dökenin kaniyla bagislanir.
\par 34 "'Içinde oturdugunuz, benim de içinde yasadigim ülkeyi kirletmeyeceksiniz; çünkü ben Israilliler'in arasinda yasayan RAB'bim."

\chapter{36}

\par 1 Yusufogullari boylarindan Manasse oglu Makir oglu Gilat'in boyunun aile baslari gelip Musa'ya ve Israil'in aile basi olan önderlerine söyle dediler:
\par 2 "RAB ülkeyi mülk olarak kurayla Israilliler arasinda paylastirmasi için efendimiz Musa'ya buyruk verdi. Kardesimiz Selofhat'in mirasinin kizlarina verilmesi için de buyruk verildi.
\par 3 Eger Selofhat'in kizlari baska bir Israil oymagina bagli erkeklerle evlenirlerse, miraslari bizim ailelerimizden alinip kocalarinin bagli olduklari oymagin mirasina eklenecek. Böylece kurayla bize düsen pay eksilecek.
\par 4 Israilliler'in özgürlük yili kutlandiginda, kizlarin mirasi kocalarinin bagli oldugu oymaga eklenecek. Böylece onlarin mirasi atalarimizin oymagina düsen mirastan alinacak."
\par 5 Musa, RAB'bin buyrugu uyarinca, Israilliler'e söyle buyurdu: "Yusuf soyundan gelenlerin söyledikleri dogrudur.
\par 6 RAB Selofhat'in kizlari için söyle diyor: Selofhat'in kizlari babalarinin bagli oldugu oymak ve boydan herhangi bir erkekle evlenmekte özgürdürler.
\par 7 Israil'de miras bir oymaktan öbür oymaga geçmeyecek. Her Israilli atalarinin bagli oldugu oymagin mirasina bagli kalacak.
\par 8 Herhangi bir Israil oymaginda miras alan kiz, babasinin bagli oldugu oymak ve boydan biriyle evlenmelidir. Öyle ki, her Israilli atalarinin mirasini sahiplenebilsin.
\par 9 Miras bir oymaktan öbür oymaga geçmeyecek. Her Israil oymagi aldigi mirasa bagli kalacak."
\par 10 Selofhat'in kizlari Mahla, Tirsa, Hogla, Milka, Noa, RAB'bin Musa'ya verdigi buyruk uyarinca davranarak amcalarinin ogullariyla evlendiler.
\par 12 Böylece Yusuf oglu Manasse boylarindan erkeklerle evlendiler, dolayisiyla miraslari da babalarinin bagli oldugu boy ve oymakta kaldi.
\par 13 RAB'bin Musa araciligiyla Seria Irmagi yaninda Eriha karsisindaki Moav ovalarinda Israilliler'e verdigi buyruklar ve ilkeler bunlardi.


\end{document}