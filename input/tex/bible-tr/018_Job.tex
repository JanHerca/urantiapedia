\begin{document}

\title{Eyüp}


\chapter{1}

\par 1 Ûs ülkesinde Eyüp adinda bir adam yasardi. Kusursuz, dogru bir adamdi. Tanri'dan korkar, kötülükten kaçinirdi.
\par 2 Yedi oglu, üç kizi vardi.
\par 3 Yedi bin koyuna, üç bin deveye, bes yüz çift öküze, bes yüz çift esege ve pek çok köleye sahipti. Dogudaki insanlarin en zengini oydu.
\par 4 Ogullari sirayla evlerinde sölen verir, birlikte yiyip içmek için üç kizkardeslerini de çagirirlardi.
\par 5 Bu sölen dönemi bitince Eyüp onlari çagirtip kutsardi. Sabah erkenden kalkar, "Çocuklarim günah islemis, içlerinden Tanri'ya sövmüs olabilirler" diyerek her biri için yakmalik sunu* sunardi. Eyüp hep böyle yapardi.
\par 6 Bir gün ilahi varliklar RAB'bin huzuruna çikmak için geldiklerinde, Seytan da onlarla geldi.
\par 7 RAB Seytan'a, "Nereden geliyorsun?" dedi. Seytan, "Dünyada gezip dolasmaktan" diye yanitladi.
\par 8 RAB, "Kulum Eyüp'e bakip da düsündün mü?" dedi, "Çünkü dünyada onun gibisi yoktur. Kusursuz, dogru bir adamdir. Tanri'dan korkar, kötülükten kaçinir."
\par 9 Seytan, "Eyüp Tanri'dan bosuna mi korkuyor?" diye yanitladi.
\par 10 "Onu, ev halkini, sahip oldugu her seyi sen çitle çevirip korumadin mi? Elleriyle yaptigi her seyi bereketli kildin. Sürüleri bütün ülkeye yayildi.
\par 11 Ama elini uzatir da sahip oldugu her seyi yok edersen, yüzüne karsi sövecektir."
\par 12 RAB Seytan'a, "Peki" dedi, "Sahip oldugu her seyi senin eline birakiyorum, yalniz kendisine dokunma." Böylece Seytan RAB'bin huzurundan ayrildi.
\par 13 Bir gün Eyüp'ün ogullariyla kizlari agabeylerinin evinde yemek yiyip sarap içerken
\par 14 bir ulak gelip Eyüp'e söyle dedi: "Öküzler çift sürüyor, esekler onlarin yaninda otluyordu.
\par 15 Sabalilar baskin yapti, hepsini alip götürdü. Usaklari kiliçtan geçirdiler. Yalniz ben kaçip kurtuldum sana durumu bildirmek için."
\par 16 O daha sözünü bitirmeden baska bir ulak gelip, "Tanri ates yagdirdi" dedi, "Koyunlarla usaklari yakip küle çevirdi. Yalnizca ben kaçip kurtuldum durumu sana bildirmek için."
\par 17 O daha sözünü bitirmeden baska bir ulak gelip, "Kildaniler* üç bölük halinde develere saldirdi" dedi, "Hepsini alip götürdüler, usaklari kiliçtan geçirdiler. Yalnizca ben kurtuldum durumu sana bildirmek için."
\par 18 O daha sözünü bitirmeden baska bir ulak gelip, "Ogullarinla kizlarin agabeylerinin evinde yemek yiyip sarap içerken
\par 19 ansizin çölden siddetli bir rüzgar esti" dedi, "Evin dört kösesine çarpti; ev gençlerin üzerine yikildi, hepsi öldü. Yalniz ben kurtuldum durumu sana bildirmek için."
\par 20 Bunun üzerine Eyüp kalkti, kaftanini yirtip saçini sakalini kesti, yere kapanip tapindi.
\par 21 Dedi ki, "Bu dünyaya çiplak geldim, çiplak gidecegim. RAB verdi, RAB aldi, RAB'bin adina övgüler olsun!"
\par 22 Bütün bu olaylara karsin Eyüp günah islemedi ve Tanri'yi suçlamadi.

\chapter{2}

\par 1 Baska bir gün ilahi varliklar RAB'bin huzuruna çikmak için geldiklerinde Seytan da RAB'bin huzuruna çikmak için onlarla gelmisti.
\par 2 RAB Seytan'a, "Nereden geliyorsun?" dedi. Seytan, "Dünyada gezip dolasmaktan" diye yanitladi.
\par 3 RAB, "Kulum Eyüp'e bakip da düsündün mü?" dedi, "Çünkü dünyada onun gibisi yoktur. Kusursuz, dogru bir adamdir. Tanri'dan korkar, kötülükten kaçinir. Onu bos yere yok etmek için beni kiskirttin, ama o dogrulugunu hâlâ sürdürüyor."
\par 4 "Cana can!" diye yanitladi Seytan, "Insan cani için her seyini verir.
\par 5 Elini uzat da, onun etine, kemigine dokun, yüzüne karsi sövecektir."
\par 6 RAB, "Peki" dedi, "Onu senin eline birakiyorum. Yalniz canina dokunma."
\par 7 Böylece Seytan RAB'bin huzurundan ayrildi. Eyüp'ün bedeninde tepeden tirnaga kadar kötü çibanlar çikardi.
\par 8 Eyüp çibanlarini kasimak için bir çömlek parçasi aldi. Kül içinde oturuyordu.
\par 9 Karisi, "Hâlâ dogrulugunu sürdürüyor musun?" dedi, "Tanri'ya söv de öl bari!"
\par 10 Eyüp, "Aptal kadinlar gibi konusuyorsun" diye karsilik verdi, "Nasil olur? Tanri'dan gelen iyiligi kabul edelim de kötülügü kabul etmeyelim mi?" Bütün bu olaylara karsin Eyüp'ün agzindan günah sayilabilecek bir söz çikmadi.
\par 11 Eyüp'ün üç dostu -Temanli Elifaz, Suahli Bildat, Naamali Sofar- Eyüp'ün basina gelen bunca kötülügü duyunca kalkip bir araya geldiler. Acisini paylasmak, onu avutmak için yanina gitmek üzere anlastilar.
\par 12 Uzaktan onu taniyamadilar; yüksek sesle aglayip kaftanlarini yirtarak baslarina toprak saçtilar.
\par 13 Yedi gün yedi gece onunla birlikte yere oturdular. Kimse agzini açmadi, çünkü ne denli aci çektigini görüyorlardi.

\chapter{3}

\par 1 Sonunda Eyüp agzini açti ve dogdugu güne lanet edip söyle dedi:
\par 3 "Dogdugum gün yok olsun, 'Bir ogul dogdu denen gece yok olsun!
\par 4 Karanliga bürünsün o gün, Yüce Tanri onunla ilgilenmesin, Üzerine isik dogmasin.
\par 5 Karanlik ve ölüm gölgesi sahip çiksin o güne, Bulut çöksün üzerine; Isigini karanlik söndürsün.
\par 6 Zifiri karanlik yutsun o geceyi, Yilin günleri arasinda sayilmasin, Aylardan hiçbirine girmesin.
\par 7 Kisir olsun o gece, Sevinç sesi duyulmasin içinde.
\par 8 Günleri lanetleyenler, Livyatan'i* uyandirmaya hazir olanlar, O günü lanetlesin.
\par 9 Aksaminin yildizlari kararsin, Bos yere aydinligi beklesin, Tan atisini görmesin.
\par 10 Çünkü sikinti yüzü görmemem için Anamin rahminin kapilarini üstüme kapamadi.
\par 11 "Neden dogarken ölmedim, Rahimden çikarken son solugumu vermedim?
\par 12 Neden beni dizler, Emeyim diye memeler karsiladi?
\par 13 Çünkü simdi huzur içinde yatmis, Uyuyup dinlenmis olurdum;
\par 14 Yaptirdiklari kentler simdi viran olan Dünya krallari ve danismanlariyla birlikte,
\par 15 Evlerini gümüsle dolduran Altin sahibi önderlerle birlikte.
\par 16 Neden düsük bir çocuk gibi, Gün yüzü görmemis yavrular gibi topraga gömülmedim?
\par 17 Orada kötüler kargasayi birakir, Yorgunlar rahat eder.
\par 18 Tutsaklar huzur içinde yasar, Angaryacinin sesini duymazlar.
\par 19 Küçük de büyük de oradadir, Köle efendisinden özgürdür.
\par 20 "Niçin sikinti çekenlere isik, Aci içindekilere yasam verilir?
\par 21 Oysa onlar gelmeyen ölümü özler, Onu define arar gibi ararlar;
\par 22 Mezara kavusunca Neseden cosar, sevinç bulurlar.
\par 23 Neden yasam verilir nereye gidecegini bilmeyen insana, Çevresini Tanri'nin çitle çevirdigi kisiye?
\par 24 Çünkü iniltim ekmekten önce geliyor, Su gibi dökülmekte feryadim.
\par 25 Korktugum, Çekindigim basima geldi.
\par 26 Huzur yok, sükûnet yok, rahat yok, Yalniz kargasa var."

\chapter{4}

\par 1 Temanli Elifaz söyle yanitladi:
\par 2 "Biri sana bir sey söylemeye çalissa gücenir misin? Kim konusmadan durabilir?
\par 3 Evet, pek çoklarina sen ders verdin, Zayif elleri güçlendirdin,
\par 4 Tökezleyeni senin sözlerin ayakta tuttu, Titreyen dizleri sen pekistirdin.
\par 5 Ama simdi senin basina gelince gücüne gidiyor, Sana dokununca yilginliga düsüyorsun.
\par 6 Senin güvendigin Tanri'dan korkun degil mi, Umudun kusursuz yasaminda degil mi?
\par 7 "Düsün biraz: Hangi suçsuz yok oldu, Nerede dogrular yikima ugradi?
\par 8 Benim gördügüm kadariyla, fesat sürenler, Kötülük tohumu ekenler ektiklerini biçiyor.
\par 9 Tanri'nin soluguyla yok oluyor, Öfkesinin rüzgariyla tükeniyorlar.
\par 10 Aslanin kükremesi, homurtusu kesildi, Disleri kirildi genç aslanlarin.
\par 11 Aslan av bulamadigi için yok oluyor, Disi aslanin yavrulari dagiliyor.
\par 12 "Bir söz gizlice eristi bana, Fisiltisi kulagima ulasti.
\par 13 Gece rüyalarin dogurdugu düsünceler içinde, Insanlari agir uyku bastigi zaman,
\par 14 Beni dehset ve titreme aldi, Bütün kemiklerimi sarsti.
\par 15 Önümden bir ruh geçti, Tüylerim ürperdi.
\par 16 Durdu, ama ne oldugunu seçemedim. Bir suret duruyordu gözümün önünde, Çit çikmazken bir ses duydum:
\par 17 'Tanri karsisinda insan dogru olabilir mi? Kendisini yaratanin karsisinda temiz çikabilir mi?
\par 18 Bakin, Tanri kullarina güvenmez, Meleklerinde hata bulur da,
\par 19 Çamur evlerde oturanlara, Mayasi toprak olanlara, Güveden kolay ezilenlere mi güvenir?
\par 20 Ömürleri sabahtan aksama varmaz, Kimse farkina varmadan sonsuza dek yok olurlar.
\par 21 Içlerindeki çadir ipleri çekilince, Bilgelikten yoksun olarak ölüp giderler.

\chapter{5}

\par 1 "Haydi çagir, seni yanitlayan çikacak mi? Meleklerin hangisine yöneleceksin?
\par 2 Aptali üzüntü öldürür, Budalayi kiskançlik bitirir.
\par 3 Ben aptalin kök saldigini görünce, Hemen yurduna lanet ettim.
\par 4 Çocuklari güvenlikten uzak, Mahkeme kapisinda ezilir, Savunan çikmaz.
\par 5 Ürününü açlar yer, Dikenler arasindakini bile toplarlar; Mallarini susamislar yutmak ister.
\par 6 Çünkü dert topraktan çikmaz, Sikinti yerden bitmez.
\par 7 Havaya uçusan kivilcimlar gibi Sikinti çekmek için dogar insan.
\par 8 "Oysa ben Tanri'ya yönelir, Davami O'na birakirdim.
\par 9 Anlayamadigimiz büyük isler, Sayisiz sasilasi isler yapan O'dur.
\par 10 Yeryüzüne yagmur yagdirir, Tarlalara sular gönderir.
\par 11 Düskünleri yükseltir, Yaslilari esenlige çikarir.
\par 12 Kurnazlarin oyununu bozar, Düzenlerini gerçeklestiremesinler diye.
\par 13 Bilgeleri kurnazliklarinda yakalar, Düzenbazlarin oyunu son bulur.
\par 14 Gündüz karanliga toslar, Öglen, geceymis gibi el yordamiyla ararlar.
\par 15 Yoksulu onlarin kiliç gibi agzindan Ve güçlünün elinden O kurtarir.
\par 16 Yoksul umutlanir, Haksizlik agzini kapar.
\par 17 "Iste, ne mutlu Tanri'nin egittigi insana! Bu yüzden Her Seye Gücü Yeten'in yola getirisini küçümseme.
\par 18 Çünkü O hem yaralar hem sarar, O incitir, ama elleri sagaltir.
\par 19 Alti kez sikintiya düssen seni kurtarir, Yedinci kez de sana zarar vermez.
\par 20 Kitlikta ölümden, Savasta kiliçtan seni O koruyacak.
\par 21 Kamçilayan dillerden uzak kalacak, Yikim gelince korkmayacaksin.
\par 22 Yikima, açliga gülüp geçecek, Yabanil hayvanlardan ürkmeyeceksin.
\par 23 Çünkü tarladaki taslarla anlasacaksin, Yabanil hayvanlar seninle barisacak.
\par 24 Çadirinin güvenlik içinde oldugunu bilecek, Yurdunu yoklayinca eksik bulmayacaksin.
\par 25 Çocuklarinin çogalacagini bileceksin, Soyun ot gibi bitecek.
\par 26 Zamaninda toplanan demetler gibi, Mezara dinç gireceksin.
\par 27 "Iste arastirdik, dogrudur, Onun için bunu dinle ve belle."

\chapter{6}

\par 1 Eyüp söyle yanitladi:
\par 2 "Keske üzüntüm tartilabilse, Acim teraziye konabilseydi!
\par 3 Denizlerin kumundan agir gelirdi, Bu yüzden abuk sabuk konustum.
\par 4 Çünkü Her Seye Gücü Yeten'in oklari içimde, Ruhum onlarin zehirini içiyor, Tanri'nin dehsetleri karsima dizildi.
\par 5 Otu olan yaban esegi anirir mi, Yemi olan öküz bögürür mü?
\par 6 Tatsiz bir sey tuzsuz yenir mi, Yumurta akinda tat bulunur mu?
\par 7 Böyle yiyeceklere dokunmak istemiyorum, Beni hasta ediyorlar.
\par 8 "Keske dilegim yerine gelse, Tanri özledigimi bana verse!
\par 9 Kerem edip beni ezse, Elini çabuk tutup yasam bagimi kesse!
\par 10 Yine avunur, Amansiz derdime karsin sevinirdim, Çünkü Kutsal Olan'in sözlerini yadsimadim.
\par 11 Gücüm nedir ki, bekleyeyim? Sonum nedir ki, sabredeyim?
\par 12 Tas kadar güçlü müyüm, Etim tunçtan* mi?
\par 13 Çaresiz kalinca Kendimi kurtaracak gücüm mü olur?
\par 14 "Kederli insana dost sevgisi gerekir, Her Seye Gücü Yeten'den korkmaktan vaz geçse bile.
\par 15 Kardeslerim kuru bir dere gibi beni aldatti; Hani gürül gürül akan dereler vardir,
\par 16 Eriyen buzlarla tasan, Kar sulariyla beslenen,
\par 17 Ama kurak mevsimde akmayan, Sicakta yataklarinda tükenen dereler... Iste öyle aldattilar beni.
\par 18 O dereler için kervanlar yolundan sapar, Çöle çikip yok olurlar. Tema'nin kervanlari su arar, Saba'dan gelen yolcular umutla bakar.
\par 20 Ama oraya varinca umut bagladiklari için utanir, Hayal kirikligina ugrarlar.
\par 21 Artik siz de bir hiç oldunuz, Dehsete kapilip korkuyorsunuz.
\par 22 'Benim için bir sey verin Ya da, 'Rüsvet verip Beni düsmanin elinden kurtarin, Acimasizlarin elinden alin dedim mi?
\par 24 "Bana ögretin, susayim, Yanlisimi gösterin.
\par 25 Dogru söz acidir! Ama tartismalariniz neyi kanitliyor?
\par 26 Sözlerimi düzeltmek mi istiyorsunuz? Çaresizin sözlerini bos laf mi sayiyorsunuz?
\par 27 Öksüzün üzerine kura çeker, Arkadasinizin üzerine pazarlik ederdiniz.
\par 28 "Simdi lütfedip bana bakin, Yüzünüze karsi yalan söyleyecek degilim ya.
\par 29 Birakin artik, haksizlik etmeyin, Bir daha düsünün, davamda hakliyim.
\par 30 Agzimdan haksiz bir söz çikiyor mu, Damagim kötü niyeti ayirt edemiyor mu?

\chapter{7}

\par 1 "Yeryüzünde insan yasami savasi andirmiyor mu, Günleri gündelikçinin günlerinden farkli mi?
\par 2 Gölgeyi özleyen köle, Ücretini bekleyen gündelikçi gibi,
\par 3 Miras olarak bana bos aylar verildi, Payima sikintili geceler düstü.
\par 4 Yatarken, 'Ne zaman kalkacagim diye düsünüyorum, Ama gece uzadikça uzuyor, Gün dogana dek dönüp duruyorum.
\par 5 Bedenimi kurt, kabuk kaplamis, Çatlayan derimden irin akiyor.
\par 6 "Günlerim dokumacinin mekiginden hizli, Umutsuz tükenmekte.
\par 7 Ey Tanri, yasamimin bir soluk oldugunu animsa, Gözüm bir daha mutluluk yüzü görmeyecek.
\par 8 Su anda bana bakan gözler bir daha beni görmeyecek, Senin gözlerin üzerimde olacak, Ama ben yok olacagim.
\par 9 Bir bulutun dagilip gitmesi gibi, Ölüler diyarina inen bir daha çikmaz.
\par 10 Bir daha evine dönmez, Bulundugu yer artik onu tanimaz.
\par 11 "Bu yüzden sessiz kalmayacak, Içimdeki sikintiyi dile getirecegim; Canimin acisiyla yakinacagim.
\par 12 Ben deniz ya da deniz canavari miyim ki, Basima bekçi koydun?
\par 13 Yatagim beni rahatlatir, Dösegim acilarimi dindirir diye düsündügümde,
\par 14 Beni düslerle korkutuyor, Görümlerle yildiriyorsun.
\par 15 Öyle ki, bogulmayi, Ölmeyi su yasama yegliyorum.
\par 16 Yasamimdan tiksiniyor, Sonsuza dek yasamak istemiyorum; Çek elini benden, çünkü günlerimin anlami kalmadi.
\par 17 "Insan ne ki, onu büyütesin, Üzerinde kafa yorasin,
\par 18 Her sabah onu yoklayasin, Her an onu sinayasin?
\par 19 Gözünü üzerimden hiç ayirmayacak misin, Tükürügümü yutacak kadar bile beni rahat birakmayacak misin?
\par 20 Günah isledimse, ne yaptim sana, Ey insan gözcüsü? Niçin beni kendine hedef seçtin? Sana yük mü oldum?
\par 21 Niçin isyanimi bagislamaz, Suçumu affetmezsin? Çünkü yakinda topraga girecegim, Beni çok arayacaksin, ama ben artik olmayacagim."

\chapter{8}

\par 1 Suahli Bildat söyle yanitladi:
\par 2 "Ne zamana dek böyle konusacaksin? Sözlerin sert rüzgar gibi.
\par 3 Tanri adaleti saptirir mi, Her Seye Gücü Yeten dogru olani çarpitir mi?
\par 4 Ogullarin ona karsi günah islediyse, Isyanlarinin cezasini vermistir.
\par 5 Ama sen gayretle Tanri'yi arar, Her Seye Gücü Yeten'e yalvarirsan,
\par 6 Temiz ve dogruysan, O simdi bile senin için kollari sivayip Seni hak ettigin yere geri getirecektir.
\par 7 Baslangicin küçük olsa da, Sonun büyük olacak.
\par 8 "Lütfen, önceki kusaklara sor, Atalarinin neler ögrendigini iyice arastir.
\par 9 Çünkü biz daha dün dogduk, bir sey bilmeyiz, Yeryüzündeki günlerimiz sadece bir gölge.
\par 10 Onlar sana anlatip ögretmeyecek, Içlerindeki sözleri dile getirmeyecek mi?
\par 11 "Bataklik olmayan yerde kamis biter mi? Susuz yerde saz büyür mü?
\par 12 Henüz yesilken, kesilmeden, Otlardan önce kururlar.
\par 13 Tanri'yi unutan herkesin sonu böyledir, Tanrisiz insanin umudu böyle yok olur.
\par 14 Onun güvendigi sey kirilir, Dayanagi ise bir örümcek agidir.
\par 15 Örümcek agina yaslanir, ama ag çöker, Ona tutunur, ama ag tasimaz.
\par 16 Tanrisizlar güneste iyi sulanmis bitkiyi andirir, Dallari bahçenin üzerinden asar;
\par 17 Kökleri tas yiginina sarilir, Çakillarin arasinda yer aranir.
\par 18 Ama yerinden sökülürse, Yeri, 'Seni hiç görmedim diyerek onu yadsir.
\par 19 Iste sevinci böyle son bulur, Yerinde baska bitkiler biter.
\par 20 "Tanri kusursuz insani reddetmez, Kötülük edenlerin elinden tutmaz.
\par 21 O senin agzini yine gülüsle, Dudaklarini sevinç haykirisiyla dolduracaktir.
\par 22 Düsmanlarini utanç kaplayacak, Kötülerin çadiri yok olacaktir.

\chapter{9}

\par 1 Eyüp söyle yanitladi:
\par 2 "Biliyorum, gerçekten öyledir, Ama Tanri'nin önünde insan nasil hakli çikabilir?
\par 3 Biri O'nunla tartismak istese, Binde bir bile O'na yanit veremez.
\par 4 O'nun bilgisi derin, gücü essizdir, Kim O'na direndi de ayakta kaldi?
\par 5 O daglari yerinden oynatir da, Daglar farkina varmaz, Öfkeyle altüst eder onlari.
\par 6 Dünyayi yerinden oynatir, Direklerini titretir.
\par 7 Günese buyruk verir, dogmaz günes, Yildizlari mühürler.
\par 8 O'dur tek basina gökleri geren, Denizin dalgalari üzerinde yürüyen.
\par 9 Büyük Ayi'yi, Oryon'u, Ülker'i, Güney takimyildizlarini yaratan O'dur.
\par 10 Anlayamadigimiz büyük isler, Sayisiz sasilasi isler yapan O'dur.
\par 11 Iste, yanimdan geçer, O'nu göremem, Geçip gider, farkina bile varmam.
\par 12 Evet, O avini kaparsa, kim O'nu durdurabilir? Kim O'na, 'Ne yapiyorsun diyebilir?
\par 13 Tanri öfkesini dizginlemez, O'nun ayagina kapanir. Rahav'in yardimcilari bile
\par 14 "Nerde kaldi ki, ben O'na yanit vereyim, O'nunla tartismak için söz bulayim?
\par 15 Hakli olsam da O'na yanit veremez, Merhamet etmesi için yargicima yalvarirdim ancak.
\par 16 O'nu çagirsam, O da bana yanit verseydi, Yine de inanmazdim sesime kulak verdigine.
\par 17 O beni kasirgayla eziyor, Nedensiz yaralarimi çogaltiyor.
\par 18 Soluk almama izin vermiyor, Ancak beni aciya doyuruyor.
\par 19 Sorun güç sorunuysa, O güçlüdür! Adalet sorunuysa, kim O'nu mahkemeye çagirabilir?
\par 20 Suçsuz olsam agzim beni suçlar, Kusursuz olsam beni suçlu çikarir.
\par 21 "Kusursuz olsam da kendime aldirdigim yok, Yasamimi hor görüyorum.
\par 22 Hepsi bir, bu yüzden diyorum ki, 'O suçluyu da suçsuzu da yok ediyor.
\par 23 Kirbaç ansizin ölüm saçinca, O suçsuzlarin sikintisiyla eglenir.
\par 24 Dünya kötülerin eline verilmis, Yargiçlarin gözünü kapayan O'dur. O degilse, kimdir?
\par 25 "Günlerim kosucudan çabuk, Iyilik görmeden geçmekte.
\par 26 Kamis sandal gibi kayip gidiyor, Avinin üstüne süzülen kartal gibi.
\par 27 'Acilarimi unutayim, Üzgün çehremi degistirip gülümseyeyim desem,
\par 28 Bütün dertlerimden yilarim, Çünkü beni suçsuz saymayacagini biliyorum.
\par 29 Madem suçlanacagim, Neden bos yere ugrasayim?
\par 30 Sabun otuyla yikansam, Ellerimi kül suyuyla temizlesem,
\par 31 Beni yine pislige batirirsin, Giysilerim bile benden tiksinir.
\par 32 O benim gibi bir insan degil ki, O'na yanit vereyim, Birlikte mahkemeye gideyim.
\par 33 Keske aramizda bir hakem olsa da, Elini ikimizin üstüne koysa!
\par 34 Tanri sopasini üzerimden kaldirsin, Dehseti beni yildirmasin.
\par 35 O zaman konusur, O'ndan korkmazdim, Ama bu durumda bir sey yapamam.

\chapter{10}

\par 1 "Yasamimdan usandim, Özgürce yakinacak, Içimdeki aciyla konusacagim.
\par 2 Tanri'ya: Beni suçlama diyecegim, Ama söyle, niçin benimle çekisiyorsun.
\par 3 Hosuna mi gidiyor gaddarlik etmek, Kendi ellerinin emegini reddedip Kötülerin tasarilarini onaylamak?
\par 4 Sende insan gözü mü var? Insanin gördügü gibi mi görüyorsun?
\par 5 Günlerin ölümlü birinin günleri gibi, Yillarin insanin yillari gibi mi ki,
\par 6 Suçumu ariyor, Günahimi arastiriyorsun?
\par 7 Kötü olmadigimi, Senin elinden beni kimsenin kurtaramayacagini biliyorsun.
\par 8 "Senin ellerin bana biçim verdi, beni yaratti, Simdi dönüp beni yok mu edeceksin?
\par 9 Lütfen animsa, balçik gibi bana sen biçim verdin, Beni yine topraga mi döndüreceksin?
\par 10 Beni süt gibi dökmedin mi, Peynir gibi katilastirmadin mi?
\par 11 Bana et ve deri giydirdin, Beni kemiklerle, sinirlerle ördün.
\par 12 Bana yasam verdin, sevgi gösterdin, Ilgin ruhumu korudu.
\par 13 "Ama bunlari yüreginde gizledin, Biliyorum aklindakini:
\par 14 Günah isleseydim, beni gözlerdin, Suçumu cezasiz birakmazdin.
\par 15 Suçluysam, vay basima! Suçsuzken bile basimi kaldiramiyorum, Çünkü utanç doluyum, çaresizim.
\par 16 Basimi kaldirsam, aslan gibi beni avlar, Sasilasi gücünü yine gösterirsin üstümde.
\par 17 Bana karsi yeni taniklar çikarir, Öfkeni artirirsin. Ordularin dalga dalga üzerime geliyor.
\par 18 "Niçin dogmama izin verdin? Keske ölseydim, hiçbir göz beni görmeden!
\par 19 Hiç var olmamis olurdum, Rahimden mezara tasinirdim.
\par 20 Birkaç günlük ömrüm kalmadi mi? Beni rahat birak da biraz yüzüm gülsün;
\par 21 Dönüsü olmayan yere gitmeden önce, Karanlik ve ölüm gölgesi diyarina,
\par 22 Zifiri karanlik diyarina, Ölüm gölgesi, kargasa diyarina, Aydinligin karanligi andirdigi yere."

\chapter{11}

\par 1 Naamali Sofar söyle yanitladi:
\par 2 "Bunca söz yanitsiz mi kalsin? Çok konusan hakli mi sayilsin?
\par 3 Saçmaliklarin karsisinda sussun mu insanlar? Sen alay edince kimse seni utandirmasin mi?
\par 4 Tanri'ya, 'Inancim aridir diyorsun, 'Senin gözünde temizim.
\par 5 Ama keske Tanri konussa, Sana karsi agzini açsa da,
\par 6 Bilgeligin sirlarini bildirse! Çünkü bilgelik çok yönlüdür. Bil ki, Tanri günahlarindan bazilarini unuttu bile.
\par 7 "Tanri'nin derin sirlarini anlayabilir misin? Her Seye Gücü Yeten'in sinirlarina ulasabilir misin?
\par 8 Onlar gökler kadar yüksektir, ne yapabilirsin? Ölüler diyarindan derindir, nasil anlayabilirsin?
\par 9 Ölçüleri yeryüzünden uzun, Denizden genistir.
\par 10 "Gelip seni hapsetse, mahkemeye çagirsa, Kim O'na engel olabilir?
\par 11 Çünkü O yalancilari tanir, Kötülügü görür de dikkate almaz mi?
\par 12 Ne zaman yaban esegi insan dogurursa, Aptal da o zaman sagduyulu olur.
\par 13 "O'na yüregini adar, Ellerini açarsan,
\par 14 Isledigin günahi kendinden uzaklastirir, Çadirinda haksizliga yer vermezsen,
\par 15 Utanmadan basini kaldirir, Saglam ve korkusuz olabilirsin.
\par 16 Sikintilarini unutur, Akip gitmis sular gibi anarsin onlari.
\par 17 Yasamin öglen günesinden daha parlak olur, Karanlik sabaha döner.
\par 18 Güven duyarsin, çünkü umudun olur, Çevrene bakip güvenlik içinde yatarsin.
\par 19 Uzanirsin, korkutan olmaz, Birçoklari senden lütuf diler.
\par 20 Ama kötülerin gözlerinin feri sönecek, Kaçacak yer bulamayacaklar, Tek umutlari son soluklarini vermek olacak."

\chapter{12}

\par 1 Eyüp söyle yanitladi:
\par 2 "Kendinizi birsey sandiginiz belli, Ama bilgelik de sizinle birlikte ölecek!
\par 3 Sizin kadar benim de aklim var, Sizden asagi kalmam. Kim bilmez bunlari?
\par 4 "Gülünç oldum dostlarima, Ben ki, Tanri'ya yakarirdim, yanitlardi beni. Dogru ve kusursuz adam gülünç oldu.
\par 5 Kaygisizlar felaketi küçümser, Ayagi kayani umursamaz.
\par 6 Soyguncularin çadirlarinda rahatlik var, Tanri'yi gazaba getirenler güvenlik içinde, Tanri'ya degil, kendi bileklerine güveniyorlar.
\par 7 "Ama simdi sor hayvanlara, sana ögretsinler, Gökte uçan kuslara sor, sana anlatsinlar,
\par 8 Topraga söyle, sana ögretsin, Denizdeki baliklara sor, sana bilgi versinler.
\par 9 Hangisi bilmez Bunu RAB'bin yaptigini?
\par 10 Her yaratigin cani, Bütün insanligin solugu O'nun elindedir.
\par 11 Damagin yemegi tattigi gibi Kulak da sözleri denemez mi?
\par 12 Bilgelik yaslilarda, Akil uzun yasamdadir.
\par 13 "Bilgelik ve güç Tanri'ya özgüdür, O'ndadir ögüt ve akil.
\par 14 O'nun yiktigi onarilamaz, O'nun hapsettigi kisi özgür olamaz.
\par 15 Sulari tutarsa, kuraklik olur, Saliverirse dünyayi sel götürür.
\par 16 Güç ve zafer O'na aittir, Aldanan da aldatan da O'nundur.
\par 17 Danismanlari çaresiz kilar, Yargiçlari çildirtir.
\par 18 Krallarin bagladigi bagi çözer, Bellerine kusak baglar.
\par 19 Kâhinleri* çaresiz kilar, Koltuklarinda yillananlari devirir.
\par 20 Güvenilir danismanlari susturur, Yaslilarin aklini alir.
\par 21 Rezalet saçar soylular üzerine, Güçlülerin kusagini gevsetir.
\par 22 Karanliklarin derin sirlarini açar, Ölüm gölgesini aydinliga çikarir.
\par 23 Uluslari büyütür, uluslari yok eder, Uluslari genisletir, uluslari sürgün eder.
\par 24 Dünya önderlerinin aklini basindan alir, Yolu izi belirsiz bir çölde dolastirir onlari.
\par 25 Karanlikta el yordamiyla yürür, isik yüzü görmezler; Sarhos gibi dolastirir onlari.

\chapter{13}

\par 1 "Iste, gözlerim her seyi gördü, Kulagim duydu, anladi.
\par 2 Sizin bildiginizi ben de biliyorum, Sizden asagi kalmam.
\par 3 Ama ben Her Seye Gücü Yeten'le konusmak, Davami Tanri'yla tartismak istiyorum.
\par 4 Sizlerse yalan düzüyorsunuz, Hepiniz degersiz hekimlersiniz.
\par 5 Keske büsbütün sussaniz! Sizin için bilgelik olurdu bu.
\par 6 Simdi davami dinleyin, Yakinmama kulak verin.
\par 7 Tanri adina haksizlik mi edeceksiniz? O'nun adina yalan mi söyleyeceksiniz?
\par 8 O'nun tarafini mi tutacaksiniz? Tanri'nin davasini mi savunacaksiniz?
\par 9 Sizi sorguya çekerse, iyi mi olur? Insanlari aldattiginiz gibi O'nu da mi aldatacaksiniz?
\par 10 Gizlice O'nun tarafini tutarsaniz, Kuskusuz sizi azarlar.
\par 11 O'nun görkemi sizi yildirmaz mi? Dehseti üzerinize düsmez mi?
\par 12 Anlattiklariniz kül kadar degersizdir, Savunduklarinizsa çamurdan farksiz.
\par 13 "Susun, birakin ben konusayim, Basima ne gelirse gelsin.
\par 14 Hayatim tehlikeye girecekse girsin, Canim zora düsecekse düssün.
\par 15 Beni öldürecek, umudum kalmadi, Hiç olmazsa yürüdügüm yolun dogrulugunu yüzüne karsi savunayim.
\par 16 Aslinda bu benim kurtulusum olacak, Çünkü tanrisiz bir adam O'nun karsisina çikamaz.
\par 17 Sözlerimi iyi dinleyin, Kulaklarinizdan çikmasin söyleyeceklerim.
\par 18 Iste davami hazirladim, Hakli çikacagimi biliyorum.
\par 19 Kim suçlayacak beni? Biri varsa susar, son solugumu veririm.
\par 20 "Yalniz su iki seyi lütfet, Tanrim, O zaman kendimi senden gizlemeyecegim:
\par 21 Elini üstümden çek Ve dehsetinle beni yildirma.
\par 22 Sonra beni çagir, yanitlayayim, Ya da birak ben konusayim, sen yanitla.
\par 23 Suçlarim, günahlarim ne kadar? Bana suçumu, günahimi göster.
\par 24 Niçin yüzünü gizliyorsun, Beni düsman gibi görüyorsun?
\par 25 Rüzgarin sürükledigi yapraga dönmüsüm, Beni mi korkutacaksin? Kuru samani mi kovalayacaksin?
\par 26 Çünkü hakkimda aci seyler yaziyor, Gençligimde isledigim günahlari bana miras veriyorsun.
\par 27 Ayaklarimi tomruga vuruyor, Yollarimi gözetliyor, Izimi sürüyorsun.
\par 28 "Oysa insan telef olmus, çürük bir sey, Güve yemis giysi gibidir.

\chapter{14}

\par 1 "Insani kadin dogurur, Günleri sayili ve sikinti doludur.
\par 2 Çiçek gibi açip solar, Gölge gibi gelip geçer.
\par 3 Gözlerini böyle birine mi dikiyorsun, Yargilamak için önüne çagiriyorsun?
\par 4 Kim temizi kirliden çikarabilir? Hiç kimse!
\par 5 Madem insanin günleri belirlenmis, Aylarinin sayisi saptanmis, Sinir koymussun, öteye geçemez;
\par 6 Gözünü ondan ayir da, Çalisma saatini dolduran gündelikçi gibi rahat etsin.
\par 7 "Oysa bir agaç için umut vardir, Kesilse, yeniden sürgün verir, Eksilmez filizleri.
\par 8 Kökü yerde kocasa, Kütügü toprakta ölse bile,
\par 9 Su kokusu alir almaz filizlenir, Bir fidan gibi dal budak salar.
\par 10 Insan ise ölüp yok olur, Son solugunu verir ve her sey biter.
\par 11 Suyu akip giden göl Ya da kuruyan irmak nasil çöle dönerse,
\par 12 Insan da öyle, yatar, bir daha kalkmaz, Gökler yok oluncaya dek uyanmaz, Uyandirilmaz.
\par 13 "Keske beni ölüler diyarina gizlesen, Öfken geçinceye dek saklasan, Bana bir süre versen de, beni sonra animsasan.
\par 14 Insan ölür de dirilir mi? Baska biri nöbetimi devralincaya dek Savas boyunca umutla beklerdim.
\par 15 Sen çagirirdin, ben yanitlardim, Ellerinle yaptigin yaratigi özlerdin.
\par 16 O zaman adimlarimi sayar, Günahimin hesabini tutmazdin.
\par 17 Isyanimi torbaya koyup mühürler, Suçumu örterdin.
\par 18 "Ama dagin yikilip çöktügü, Kayanin yerinden tasindigi,
\par 19 Suyun tasi asindirdigi, Selin topragi sürükleyip götürdügü gibi, Insanin umudunu yok ediyorsun.
\par 20 Onu hep yenersin, yok olup gider, Çehresini degistirir, uzaga gönderirsin.
\par 21 Ogullari saygi görür, onun haberi olmaz, Asagilanirlar, anlamaz.
\par 22 Ancak kendi caninin acisini duyar, Yalniz kendisi için yas tutar.

\chapter{15}

\par 1 Temanli Elifaz söyle yanitladi:
\par 2 "Bilge kisi bos sözlerle yanitlar mi, Karnini dogu rüzgariyla doldurur mu?
\par 3 Bos sözlerle tartisir, Yararsiz söylevler verir mi?
\par 4 Tanri korkusunu bile ortadan kaldiriyor, Tanri'nin huzurunda düsünmeyi engelliyorsun.
\par 5 Çünkü suçun agzini kiskirtiyor, Hilekârlarin diliyle konusuyorsun.
\par 6 Kendi agzin seni suçluyor, ben degil, Dudaklarin sana karsi taniklik ediyor.
\par 7 "Ilk dogan insan sen misin? Yoksa daglardan önce mi var oldun?
\par 8 Tanri'nin sirrini mi dinledin de, Yalniz kendini bilge görüyorsun?
\par 9 Senin bildigin ne ki, biz bilmeyelim? Senin anladigin ne ki, bizde olmasin?
\par 10 Bizde ak saçli da yasli da var, Babandan bile yasli.
\par 11 Az mi geliyor Tanri'nin avutmasi sana, Söyledigi yumusak sözler?
\par 12 Niçin yüregin seni sürüklüyor, Gözlerin parildiyor,
\par 13 Tanri'ya öfkeni gösteriyorsun, Agzindan böyle sözler dökülüyor?
\par 14 Insan gerçekten temiz olabilir mi? Kadindan dogan biri dogru olabilir mi?
\par 15 Tanri meleklerine güvenmiyorsa, Gökler bile O'nun gözünde temiz degilse,
\par 16 Haksizligi su gibi içen Igrenç, bozuk insana mi güvenecek?
\par 17 "Dinle beni, sana açiklayayim, Gördügümü anlatayim,
\par 18 Bilgelerin atalarindan ögrenip bildirdigi, Gizlemedigi gerçekleri;
\par 19 O atalar ki, ülke yalniz onlara verilmisti, Aralarina henüz yabanci girmemisti.
\par 20 Kötü insan yasami boyunca kivranir, Zorbaya ayrilan yillar sayilidir.
\par 21 Dehset sesleri kulagindan eksilmez, Esenlik içindeyken soyguncunun saldirisina ugrar.
\par 22 Karanliktan kurtulabilecegine inanmaz, Kiliç onu gözler.
\par 23 'Nerede? diyerek ekmek ardinca dolasir, Karanlik günün yanibasinda oldugunu bilir.
\par 24 Aci ve sikinti onu yildirir, Savasa hazir bir kral gibi onu yener.
\par 25 Çünkü Tanri'ya el kaldirmis, Her Seye Gücü Yeten'e meydan okumus,
\par 26 Kalin, yumrulu kalkaniyla O'na inatla saldirmisti.
\par 27 "Yüzü semirdigi, Göbegi yag bagladigi halde,
\par 28 Yikilmis kentlerde, Tas yiginina dönmüs oturulmaz evlerde oturacak,
\par 29 Zengin olmayacak, serveti tükenecek, Mallari ülkeye yayilmayacaktir.
\par 30 Karanliktan kaçamayacak, Filizlerini alev kurutacak, Tanri'nin agzindan çikan solukla yok olacaktir.
\par 31 Bos seye güvenerek kendini aldatmasin, Çünkü ödülü bosluk olacaktir.
\par 32 Gününden önce isi tamamlanacak, Dali yesermeyecektir.
\par 33 Asma gibi korugunu dökecek, Zeytin agaci gibi çiçegini dagitacaktir.
\par 34 Çünkü tanrisizlar sürüsü kisir olur, Rüsvetçilerin çadirlarini ates yakip yok eder.
\par 35 Fesada gebe kalip kötülük dogururlar, Içleri yalan doludur."

\chapter{16}

\par 1 Eyüp söyle yanitladi:
\par 2 "Buna benzer çok sey duydum, Oysa siz avutmuyor, sikinti veriyorsunuz.
\par 3 Bos sözleriniz hiç sona ermeyecek mi? Nedir derdiniz, boyuna karsilik veriyorsunuz?
\par 4 Yerimde siz olsaydiniz, Ben de sizin gibi konusabilirdim; Size karsi güzel sözler dizer, Basimi sallayabilirdim.
\par 5 Agzimdan çikan sözlerle yüreklendirir, Dudaklarimdan dökülen avutucu sözlerle yatistirirdim sizi.
\par 6 "Konussam bile acim dinmez, Sussam ne degisir?
\par 7 Ey Tanri, beni tükettin, Bütün ev halkimi dagittin.
\par 8 Beni sikip burusturdun, bana karsi tanik oldu bu; Zayifligim kalkmis taniklik ediyor bana karsi.
\par 9 Tanri öfkeyle saldirip parçaliyor beni, Dislerini gicirdatiyor bana, Düsmanim gözlerini üzerime dikiyor.
\par 10 Insanlar bana dudak büküyor, Asagilayarak tokat atiyor, Birlesiyorlar bana karsi.
\par 11 Tanri haksizlara teslim ediyor beni, Kötülerin kucagina atiyor.
\par 12 Ben rahat yasiyordum, ama Tanri paraladi beni, Boynumdan tutup yere çaldi. Beni hedef yapti kendine.
\par 13 Okçulari beni kusatiyor, Acimadan böbreklerimi desiyor, Ödümü yerlere döküyor.
\par 14 Bedenimde gedik üstüne gedik açiyor, Dev gibi üzerime saldiriyor.
\par 15 "Giymek için çul diktim, Gururumu ayak altina aldim.
\par 16 Aglamaktan yüzüm kizardi, Gözlerimin alti morardi.
\par 17 Yine de ellerim siddetten uzak, Duam içtendir.
\par 18 "Ey toprak, kanimi örtme, Feryadim asla dinmesin.
\par 19 Daha simdiden tanigim göklerde, Beni savunan yücelerdedir.
\par 20 Dostlarim benimle egleniyor, Gözlerim Tanri'ya yas döküyor;
\par 21 Tanri kendisiyle insan arasinda Insanogluyla komsusu arasinda hak arasin diye.
\par 22 "Çünkü birkaç yil sonra, Dönüsü olmayan yolculuga çikacagim.

\chapter{17}

\par 1 "Yasama gücüm tükendi, günlerim kisaldi, Mezar gözlüyor beni.
\par 2 Çevremi alaycilar kusatmis, Gözümü onlarin asagilamasiyla açip kapiyorum.
\par 3 "Ey Tanri, kefilim ol kendine karsi, Baska kim var bana güvence verecek?
\par 4 Çünkü onlarin aklini anlayisa kapadin, Bu yüzden onlari zafere kavusturmayacaksin.
\par 5 Para için dostlarini satan adamin Çocuklarinin gözünün feri söner.
\par 6 "Tanri beni insanlarin diline düsürdü, Yüzüme tükürmekteler.
\par 7 Kederden gözümün feri söndü, Kollarim bacaklarim çirpi gibi.
\par 8 Dürüst insanlar buna sasiyor, Suçsuzlar tanrisizlara saldiriyor.
\par 9 Dogrular kendi yolunu tutuyor, Elleri temiz olanlar gittikçe güçleniyor.
\par 10 "Ama siz, hepiniz gelin yine deneyin! Aranizda bir bilge bulamayacagim.
\par 11 Günlerim geçti, tasarilarim, Dileklerim suya düstü.
\par 12 Bu insanlar geceyi gündüze çeviriyorlar, Karanliga 'Isik yakindir diyorlar.
\par 13 Ölüler diyarini evim diye gözlüyorsam, Yatagimi karanliga seriyorsam,
\par 14 Çukura 'Babam, Kurda 'Annem, kizkardesim diyorsam,
\par 15 Umudum nerede? Kim benim için umut görebilir?
\par 16 Umut benimle ölüler diyarina mi inecek? Topraga birlikte mi girecegiz?"

\chapter{18}

\par 1 Suahli Bildat söyle yanitladi:
\par 2 "Ne zaman bitecek bu sözler? Biraz anlayisli olun da konusalim.
\par 3 Niçin hayvan yerine konuyoruz, Gözünüzde aptal sayiliyoruz?
\par 4 Sen kendini öfkenle paraliyorsun, Senin ugruna dünyadan vaz mi geçilecek? Kayalar yerini mi degistirecek?
\par 5 "Evet, kötünün isigi sönecek, Atesinin alevi parlamayacak.
\par 6 Çadirindaki isik karanliga dönecek, Yanindaki kandil sönecek.
\par 7 Adimlarinin gücü zayiflayacak, Kurdugu düzene kendi düsecek.
\par 8 Ayaklari onu aga götürecek, Kendi ayagiyla tuzaga basacak.
\par 9 Topugu kapana girecek, Tuzak onu kapacak.
\par 10 Topraga gizlenmis bir ilmek, Yoluna koyulmus bir kapan bekliyor onu.
\par 11 Dehset saracak onu her yandan, Her adiminda onu kovalayacak.
\par 12 Gücünü kitlik kemirecek, Tökezleyince, felaket yaninda bitiverecek.
\par 13 Derisini hastalik yiyecek, Kollariyla bacaklarini ölüm yutacak.
\par 14 Güvenli çadirindan atilacak, Dehset kralinin önüne sürüklenecek.
\par 15 Çadirinda ates oturacak, Yurdunun üzerine kükürt saçilacak.
\par 16 Kökleri dipten kuruyacak, Dallari üstten solacak.
\par 17 Ülkede anisi yok olacak, Adi dünyadan silinecek.
\par 18 Isiktan karanliga sürülecek, Dünyadan kovulacak.
\par 19 Ne çocugu ne torunu kalacak halki arasinda, Yasadigi yerde kimsesi kalmayacak.
\par 20 Batidakiler onun yikimina sasacak, Dogudakiler dehset içinde bakacak.
\par 21 Evet, kötülerin yasami iste böyle son bulur, Tanri'yi tanimayanlarin varacagi yer budur."

\chapter{19}

\par 1 Eyüp söyle yanitladi:
\par 2 "Ne zamana dek beni üzecek, Sözlerinizle ezeceksiniz?
\par 3 On kez oldu beni asagiliyor, Hiç utanmadan saldiriyorsunuz.
\par 4 Yanlis yola sapmissam, Bu benim suçum.
\par 5 Kendinizi gerçekten benden üstün görüyor, Utancimi bana karsi kullaniyorsaniz,
\par 6 Bilin ki, Tanri bana haksizlik yapti, Beni agiyla kusatti.
\par 7 "Iste, 'Zorbalik bu! diye haykiriyorum, ama yanit yok, Yardim için bagiriyorum, ama adalet yok.
\par 8 Yoluma set çekti, geçemiyorum, Yollarimi karanliga bogdu.
\par 9 Üzerimden onurumu soydu, Basimdaki taci kaldirdi.
\par 10 Her yandan yikti beni, tükendim, Umudumu bir agaç gibi kökünden söktü.
\par 11 Öfkesi bana karsi alev alev yaniyor, Beni hasim sayiyor.
\par 12 Ordulari üstüme üstüme geliyor, Bana karsi rampalar yapiyor, Çadirimin çevresinde ordugah kuruyorlar.
\par 13 "Kardeslerimi benden uzaklastirdi, Tanidiklarim bana büsbütün yabancilasti.
\par 14 Akrabalarim ugramaz oldu, Yakin dostlarim beni unuttu.
\par 15 Evimdeki konuklarla hizmetçiler Beni yabanci sayiyor, Garip oldum gözlerinde.
\par 16 Kölemi çagiriyorum, yanitlamiyor, Dil döksem bile.
\par 17 Solugum karimi tiksindiriyor, Kardeslerim benden igreniyor.
\par 18 Çocuklar bile beni küçümsüyor, Ayaga kalksam benimle egleniyorlar.
\par 19 Bütün yakin dostlarim benden igreniyor, Sevdiklerim yüz çeviriyor.
\par 20 Bir deri bir kemige döndüm, Ölümün esigine geldim.
\par 21 "Ey dostlarim, aciyin bana, siz aciyin, Çünkü Tanri'nin eli vurdu bana.
\par 22 Neden Tanri gibi siz de beni kovaliyor, Etime doymuyorsunuz?
\par 23 "Keske simdi sözlerim yazilsa, Kitaba geçseydi,
\par 24 Demir kalemle, kursunla Sonsuza dek kalsin diye kayaya kazilsaydi!
\par 25 Oysa ben kurtaricimin yasadigini, Sonunda yeryüzüne gelecegini biliyorum.
\par 26 Derim yok olduktan sonra, Yeni bedenimle Tanri'yi görecegim.
\par 27 O'nu kendim görecegim, Kendi gözlerimle, baskasi degil. Yüregim bayiliyor bagrimda!
\par 28 Eger, 'Sikintinin kökü onda oldugu için Onu kovalim diyorsaniz,
\par 29 Kiliçtan korkmalisiniz, Çünkü kiliç cezasi öfkeli olur, O zaman adaletin var oldugunu göreceksiniz."

\chapter{20}

\par 1 Naamali Sofar söyle yanitladi:
\par 2 "Sikintili düsüncelerim beni yanit vermeye zorluyor, Bu yüzden çok heyecanliyim.
\par 3 Beni utandiran bir azar isitiyorum, Anlayisim yanit vermemi gerektiriyor.
\par 4 "Bilmiyor musun eskiden beri, Insan dünyaya geldiginden beri,
\par 5 Kötünün zafer çigligi kisadir, Tanrisizin sevinciyse bir anliktir.
\par 6 Boyu göklere erisse, Basi bulutlara degse bile,
\par 7 Sonsuza dek yok olacak, kendi pisligi gibi; Onu görmüs olanlar, 'Nerede o? diyecekler.
\par 8 Düs gibi uçacak, bir daha bulunamayacak, Gece görümü gibi yok olacak.
\par 9 Kendisini görmüs olan gözler bir daha onu görmeyecek, Yasadigi yerde artik görünmeyecektir.
\par 10 Çocuklari yoksullarin lütfunu dileyecek, Malini kendi eliyle geri verecektir.
\par 11 Kemiklerini dolduran gençlik atesi Kendisiyle birlikte toprakta yatacak.
\par 12 "Kötülük agzinda tatli gözükse, Onu dilinin altina gizlese bile,
\par 13 Tutsa, birakmasa, Damaginin altina saklasa bile,
\par 14 Yedigi yiyecek midesinde eksiyecek, Içinde kobra zehirine dönüsecek.
\par 15 Yuttugu servetleri kusacak, Tanri onlari midesinden çikaracak.
\par 16 Kobra zehiri emecek, Engeregin zehir disi onu öldürecek.
\par 17 Akarsularin, bal ve ayran akan derelerin Sefasini süremeyecek.
\par 18 Zahmetle kazandigini Yemeden geri verecek, Elde ettigi kazancin tadini çikaramayacak.
\par 19 Çünkü yoksullari ezip yüzüstü birakti, Kendi yapmadigi evi zorla aldi.
\par 20 "Hirsi yüzünden rahat nedir bilmedi, Serveti onu kurtaramayacak.
\par 21 Yediginden artakalan olmadi, Bu yüzden bollugu uzun sürmeyecek.
\par 22 Varlik içinde yokluk çekecek, Sikinti tepesine binecek.
\par 23 Karnini tika basa doyurdugunda, Tanri kizgin öfkesini ondan çikaracak, Üzerine gazap yagdiracak.
\par 24 Demir silahtan kaçacak olsa, Tunç* ok onu delip geçecek.
\par 25 Çekilince ok sirtindan, Parildayan ucu ödünden çikacak, Dehset çökecek üzerine.
\par 26 Koyu karanlik onun hazinelerini gözlüyor. Körüklenmemis ates onu yiyip bitirecek, Çadirinda artakalani tüketecek.
\par 27 Suçunu gökler açiga çikaracak, Yeryüzü ona karsi ayaklanacak.
\par 28 Varligini seller, Azgin sular götürecek Tanri'nin öfkelendigi gün.
\par 29 Budur kötünün Tanri'dan aldigi pay, Budur Tanri'nin ona verdigi miras."

\chapter{21}

\par 1 Eyüp söyle yanitladi:
\par 2 "Sözümü dikkatle dinleyin, Bana verdiginiz avuntu bu olsun.
\par 3 Birakin ben de konusayim, Ben konustuktan sonra alay edin.
\par 4 "Yakinmam insana mi karsi? Niçin sabirsizlanmayayim?
\par 5 Bana bakin da sasin, Elinizi agziniza koyun.
\par 6 Bunu düsündükçe içimi korku sariyor, Bedenimi titreme aliyor.
\par 7 Kötüler niçin yasiyor, Yaslandikça güçleri artiyor?
\par 8 Çocuklari sapasaglam çevrelerinde, Soylari gözlerinin önünde.
\par 9 Evleri güvenlik içinde, korkudan uzak, Tanri'nin sopasi onlara dokunmuyor.
\par 10 Bogalarinin çiftlesmesi hiç bosa çikmaz, Inekleri hep dogurur, hiç düsük yapmaz.
\par 11 Çocuklarini sürü gibi saliverirler, Yavrulari oynasir.
\par 12 Tef ve lir esliginde sarki söyler, Ney sesiyle eglenirler.
\par 13 Ömürlerini bolluk içinde geçirir, Esenlik içinde ölüler diyarina inerler.
\par 14 Tanri'ya, 'Bizden uzak dur! derler, 'Yolunu ögrenmek istemiyoruz.
\par 15 Her Seye Gücü Yeten kim ki, O'na kulluk edelim? Ne kazancimiz olur O'na dua etsek?
\par 16 Ama zenginlikleri kendi ellerinde degil. Kötülerin ögüdü benden uzak olsun.
\par 17 "Kaç kez kötülerin kandili söndü, Baslarina felaket geldi, Tanri öfkelendiginde paylarina düsen kederi verdi?
\par 18 Kaç kez rüzgarin sürükledigi saman gibi, Kasirganin uçurdugu saman çöpü gibi oldular?
\par 19 'Tanri babalarin cezasini çocuklarina çektirir diyorsunuz, Kendilerine çektirsin de bilsinler nasil oldugunu.
\par 20 Yikimlarini kendi gözleriyle görsünler, Her Seye Gücü Yeten'in gazabini içsinler.
\par 21 Çünkü sayili aylari sona erince Geride biraktiklari aileleri için niye kaygi çeksinler?
\par 22 "En yüksektekileri bile yargilayan Tanri'ya Kim akil ögretebilir?
\par 23 Biri gücünün dorugunda ölür, Büsbütün rahat ve kaygisiz.
\par 24 Bedeni iyi beslenmis, Ilikleri dolu.
\par 25 Ötekiyse aci içinde ölür, Iyilik nedir hiç tatmamistir.
\par 26 Toprakta birlikte yatarlar, Üzerlerini kurt kaplar.
\par 27 "Bakin, düsüncelerinizi, Bana zarar vermek için kurdugunuz düzenleri biliyorum.
\par 28 'Büyük adamin evi nerede? diyorsunuz, 'Kötülerin çadirlari nerede?
\par 29 Yolculara hiç sormadiniz mi? Anlattiklarina kulak asmadiniz mi?
\par 30 Felaket günü kötü insan esirgenir, Gazap günü ona kurtulus yolu gösterilir.
\par 31 Kim davranisini onun yüzüne vurur? Kim yaptiginin karsiligini ona ödetir?
\par 32 Mezarliga tasinir, Kabri basinda nöbet tutulur.
\par 33 Vadi topragi tatli gelir ona, Herkes ardindan gider, Önüsira gidenlerse sayisizdir.
\par 34 "Bos laflarla beni nasil avutursunuz? Yanitlarinizdan çikan tek sonuç yalandir."

\chapter{22}

\par 1 Temanli Elifaz söyle yanitladi:
\par 2 "Insan Tanri'ya yararli olabilir mi? Bilge kisinin bile O'na yarari dokunabilir mi?
\par 3 Dogrulugun Her Seye Gücü Yeten'e ne zevk verebilir, Kusursuz yasamin O'na ne kazanç saglayabilir?
\par 4 Seni azarlamasi, dava etmesi O'ndan korktugun için mi?
\par 5 Kötülügün büyük, Günahlarin sonsuz degil mi?
\par 6 Çünkü kardeslerinden nedensiz rehin aliyor, Onlari soyuyordun.
\par 7 Yorguna su içirmedin, Açtan ekmegi esirgedin;
\par 8 Ülkeye bileginle sahip oldun, Saygin biri olarak orada yasadin.
\par 9 Dul kadinlari eli bos çevirdin, Öksüzlerin kolunu kanadini kirdin.
\par 10 Bu yüzden her yanin tuzaklarla çevrili, Ansizin gelen korkuyla yiliyorsun,
\par 11 Her sey karariyor, göremez oluyorsun, Seller altina aliyor seni.
\par 12 "Tanri göklerin yükseklerinde degil mi? Yildizlara bak, ne kadar yüksekteler!
\par 13 Sen ise, 'Tanri ne bilir? diyorsun, 'Zifiri karanligin içinden yargilayabilir mi?
\par 14 Koyu bulutlar O'na engeldir, göremez, Gökkubbenin üzerinde dolasir.
\par 15 Kötülerin yürüdügü Eski yolu mu tutacaksin?
\par 16 Onlar ki, vakitleri gelmeden çekilip alindilar, Temellerini sel basti.
\par 17 Tanri'ya, 'Bizden uzak dur! dediler, 'Her Seye Gücü Yeten bize ne yapabilir?
\par 18 Ama onlarin evlerini iyilikle dolduran O'ydu. Bunun için kötülerin ögüdü benden uzak olsun.
\par 19 "Dogrular onlarin yikimini görüp sevinir, Suçsuzlar söyle diyerek eglenir:
\par 20 'Düsmanlarimiz yok edildi, Mallari yanip kül oldu.
\par 21 "Tanri'yla dost ol, baris ki, Bolluga eresin.
\par 22 Agzindan çikan ögretiyi benimse, Sözlerini yüreginde tut.
\par 23 Her Seye Gücü Yeten'e dönersen, eski haline kavusursun. Kötülügü çadirindan uzak tutar,
\par 24 Altinini yere, Ofir altinini vadideki çakillarin arasina atarsan,
\par 25 Her Seye Gücü Yeten senin altinin, Degerli gümüsün olur.
\par 26 O zaman Her Seye Gücü Yeten'den zevk alir, Yüzünü Tanri'ya kaldirirsin.
\par 27 O'na dua edersin, dinler seni, Adaklarini yerine getirirsin.
\par 28 Neye karar verirsen yapilir, Yollarini isik aydinlatir.
\par 29 Insanlar seni alçaltinca, güvenini yitirme, Çünkü Tanri alçakgönüllüleri kurtarir.
\par 30 O suçsuz olmayani bile kurtarir, Senin ellerinin temizligi sayesinde kurtulur suçlu."

\chapter{23}

\par 1 Eyüp söyle yanitladi:
\par 2 "Bugün de aci aci yakinacagim, Iniltime karsin Tanri'nin üzerimdeki eli agirdir.
\par 3 Keske O'nu nerede bulacagimi bilseydim, Tahtina varabilseydim!
\par 4 Davami önünde dile getirir, Kanitlarimi art arda siralardim.
\par 5 Bana verecegi yaniti ögrenir, Ne diyecegini anlardim.
\par 6 Essiz gücüyle bana karsi mi çikardi? Hayir, yalnizca dinlerdi beni.
\par 7 Hakli kisi davasini oraya, O'nun önüne getirebilirdi, Ben de yargilanmaktan sonsuza dek kurtulurdum.
\par 8 "Doguya gitsem orada degil, Batiya gitsem O'nu bulamiyorum.
\par 9 Kuzeyde is görse O'nu seçemiyorum, Güneye dönse O'nu göremiyorum.
\par 10 Ama O tuttugum yolu biliyor, Beni sinadiginda altin gibi çikacagim.
\par 11 Adimlarini yakindan izledim, Sapmadan yolunu tuttum.
\par 12 Agzindan çikan buyruklardan ayrilmadim, Günlük ekmegimden çok agzindan çikan sözlere deger verdim.
\par 13 "O tek basinadir, kim O'nu caydirabilir? Cani ne isterse onu yapar.
\par 14 Benimle ilgili kararini yerine getirir, Daha nice tasarisi vardir.
\par 15 Bu yüzden dehsete düserim huzurunda, Düsündükçe korkarim O'ndan.
\par 16 Tanri cesaretimi kirdi, Her Seye Gücü Yeten beni yildirdi.
\par 17 Karanlik beni susturamadi, Yüzümü örten koyu karanlik.

\chapter{24}

\par 1 "Niçin Her Seye Gücü Yeten yargi için vakit saptamiyor? Neden O'nu taniyanlar bu günleri görmesin?
\par 2 Insanlar sinir taslarini kaldiriyor, Çaldiklari sürüleri otlatiyorlar.
\par 3 Öksüzlerin esegini kovuyor, Dul kadinin öküzünü rehin aliyorlar.
\par 4 Yoksullari yoldan saptiriyor, Ülkenin düskünlerini gizlenmeye zorluyorlar.
\par 5 Bakin, yoksullar çöldeki yaban esekleri gibi Yiyecek bulmak için erkenden ise çikiyorlar, Çocuklarina yiyecegi kirlar sagliyor.
\par 6 Yemlerini tarlalardan topluyor, Kötülerin bagindaki artiklari eseliyorlar.
\par 7 Geceyi giysisiz, çiplak geçiriyorlar, Örtünecek seyleri yok sogukta.
\par 8 Daglara yagan saganaktan islaniyor, Siginaklari olmadigi için kayalara sariliyorlar.
\par 9 Öksüz memeden uzaklastiriliyor, Düskünün bebegi rehin aliniyor.
\par 10 Giysisiz, çiplak dolasiyor, Aç karnina demet tasiyorlar.
\par 11 Teraslar arasinda zeytin eziyor, Susuzluktan kavrulurken Sarap için üzüm sikiyorlar.
\par 12 Kentlerden insan iniltileri yükseliyor, Yarali canlar feryat ediyor, Ama Tanri haksizligi önemsemiyor.
\par 13 "Bunlar isiga baskaldiranlardir; Onun yolunu tanimaz, Izinde yürümezler.
\par 14 Gün kararinca katil kalkar, Düskünü, yoksulu öldürür, Hirsiz gibi sivisir geceleyin.
\par 15 Zina edenin gözü alaca karanliktadir, 'Beni kimse görmez diye düsünür, Yüzünü örtüyle gizler.
\par 16 Hirsizlar karanlikta evleri deler, Gündüz gizlenir, isik nedir bilmezler.
\par 17 Çünkü zifiri karanlik, sabahidir onlarin, Karanligin dehsetiyle dostturlar.
\par 18 "Diyorsunuz ki, 'Suyun üstündeki köpüktür onlar, Lanetlidir ülkedeki paylari, Kimse baglara gitmez.
\par 19 Kuraklik ve sicagin eriyen kari alip götürdügü gibi Ölüler diyari da günahlilari alip götürür.
\par 20 Rahim onlari unutacak, Kurtlara yem olacak, Bir daha anilmayacaklar. Haksizlik bir agaç gibi kirilacak.
\par 21 Onlar çocugu olmayan kisir kadinlari yolar, Dul kadina iyilik etmezler.
\par 22 Tanri, gücüyle zorbalari yok eder, Harekete geçince zorbalarin yasama umudu kalmaz.
\par 23 Tanri onlara güven verir, O'na güvenirler, Ama gözü yürüdükleri yoldadir.
\par 24 Kisa süre yükselir, sonra yok olurlar, Düserler, tipki ötekiler gibi alinip götürülür, Basak basi gibi kesilirler.
\par 25 "Böyle degilse, kim beni yalanci çikarabilir, Söylediklerimin bos oldugunu gösterebilir?"

\chapter{25}

\par 1 Suahli Bildat söyle yanitladi:
\par 2 "Egemenlik ve heybet Tanri'ya özgüdür, Yüce göklerde düzen kuran O'dur.
\par 3 Ordulari sayilabilir mi? Isigi kimin üzerine dogmaz?
\par 4 Insan Tanri'nin önünde nasil dogru olabilir? Kadindan dogan biri nasil temiz olabilir?
\par 5 O'nun gözünde ay parlak, Yildizlar temiz degilse,
\par 6 Nerede kaldi bir kurtçuk olan insan, Bir böcek olan insanoglu!"

\chapter{26}

\par 1 Eyüp söyle yanitladi:
\par 2 "Çaresize nasil yardim ettin! Güçsüz paziyi nasil kurtardin!
\par 3 Bilge olmayana ne ögütler verdin! Saglam bilgiyi pek güzel ögrettin!
\par 4 Bu sözleri kime söyledin? Senin agzindan konusan ruh kimin?
\par 5 "Sularin ve sularda yasayanlarin altinda Ölüler titriyor.
\par 6 Tanri'nin önünde ölüler diyari çiplaktir, Yikim diyari örtüsüz.
\par 7 O boslugun üzerine kuzey göklerini yayar, Hiçligin üzerine dünyayi asar.
\par 8 Bulutlarin içine sulari sarar, Bulutlar yirtilmaz onlarin agirligi altinda.
\par 9 Dolunayin yüzünü örter, Üstüne bulutlarini serper.
\par 10 Sularin yüzeyine sinir çizer Isikla karanligin ayrildigi yerde.
\par 11 Göklerin direkleri sarsilir, Saskina dönerler O azarlayinca.
\par 12 Gücüyle denizi çalkalar, Ustaca Rahav'i vurur.
\par 13 Gökler O'nun soluguyla açilir, O'nun eli parçalar kaçan yilani.
\par 14 Bunlar yaptiklarinin küçücük parçalari, O'ndan duydugumuz hafif bir fisiltidir. Gürleyen gücünü kim anlayabilir?"

\chapter{27}

\par 1 Eyüp anlatmaya devam etti:
\par 2 "Hakkimi elimden alan Tanri'nin varligi hakki için, Bana aci çektiren Her Seye Gücü Yeten'in hakki için,
\par 3 Içimde yasam belirtisi oldugu sürece, Tanri'nin solugu burnumda oldugu sürece,
\par 4 Agzimdan kötü söz çikmayacak, Dilimden yalan dökülmeyecek.
\par 5 Size asla hak vermeyecek, Son solugumu verene dek suçsuz oldugumu söyleyecegim.
\par 6 Dogruluguma sarilacak, onu birakmayacagim, Yasadigim sürece vicdanim beni suçlamayacak.
\par 7 "Düsmanlarim kötüler gibi, Bana saldiranlar haksizlar gibi cezalandirilsin.
\par 8 Tanrisiz insanin umudu nedir Tanri onu yok ettiginde, canini aldiginda?
\par 9 Basina sikinti geldiginde, Tanri feryadini duyar mi?
\par 10 Her Seye Gücü Yeten'den zevk alir mi? Her zaman Tanri'ya yakarir mi?
\par 11 "Tanri'nin gücünü size ögretecegim, Her Seye Gücü Yeten'in tasarisini gizlemeyecegim.
\par 12 Aslinda siz, hepiniz gördünüz bunu, Öyleyse ne diye bos bos konusuyorsunuz?
\par 13 "Kötünün Tanri'dan alacagi pay, Zorbanin Her Seye Gücü Yeten'den alacagi miras sudur:
\par 14 Çocuklari ne kadar çok olursa olsun, kiliçla öldürülecek, Soyu yeterince ekmek bulamayacaktir.
\par 15 Sag kalanlar hastaliktan ölüp gömülecek, Dul karilari aglamayacaktir.
\par 16 Kötü insan kum gibi gümüs yigsa, Yiginla giysi biriktirse,
\par 17 Onun biriktirdigini dogru insan giyecek, Gümüsü suçsuz paylasacak.
\par 18 Evini güve kozasi gibi insa eder, Bekçinin kurdugu çardak gibi.
\par 19 Zengin olarak yatar, ama bu öyle sürmez, Gözlerini açtiginda hepsi yok olup gitmistir.
\par 20 Dehset onu sel gibi basar, Kasirga gece kapar götürür.
\par 21 Dogu rüzgari onu uçurup götürür, Yerinden silip süpürür.
\par 22 Acimasizca üzerine eser, Elinden kaçmaya çalisirken.
\par 23 Onunla alay ederek el çirpar, Yerinden islik çalar."

\chapter{28}

\par 1 Gümüs maden ocagindan elde edilir, Altini aritmak için de bir yer vardir.
\par 2 Demir topraktan çikarilir, Bakirsa tastan.
\par 3 Insan karanliga son verir, Koyu karanligin, ölüm gölgesinin taslarini Son sinirina kadar arastirir.
\par 4 Maden kuyusunu insanlarin oturdugu yerden uzakta açar, Insan ayaginin unuttugu yerlerde, Herkesten uzak iplere sarilip sallanir.
\par 5 Ekmek topraktan çikar, Topragin alti ise yanmis, altüst olmustur.
\par 6 Kayalarindan laciverttasi çikar, Yüzeyi altin tozunu andirir.
\par 7 Yirtici kus yolu bilmez, Doganin gözü onu görmemistir.
\par 8 Güçlü hayvanlar oraya ayak basmamis, Aslan oradan geçmemistir.
\par 9 Madenci elini çakmak tasina uzatir, Daglari kökünden altüst eder.
\par 10 Kayalarin içinden tüneller açar, Gözleri degerli ne varsa görür.
\par 11 Irmaklarin kaynagini tikar, Gizli olani isiga çikarir.
\par 12 Ama bilgelik nerede bulunur? Aklin yeri neresi?
\par 13 Insan onun degerini bilmez, Yasayanlar diyarinda ona rastlanmaz.
\par 14 Engin, "Bende degil" der, Deniz, "Yanimda degil."
\par 15 Onun bedeli saf altinla ödenmez, Degeri gümüsle ölçülmez.
\par 16 Ona Ofir altiniyla, degerli oniksle, Laciverttasiyla deger biçilmez.
\par 17 Ne altin ne cam onunla karsilastirilabilir, Saf altin kaplara degisilmez.
\par 18 Yaninda mercanla billurun sözü edilmez, Bilgeligin degeri mücevherden üstündür.
\par 19 Kûs* topazi onunla denk sayilmaz, Saf altinla ona deger biçilmez.
\par 20 Öyleyse bilgelik nereden geliyor? Aklin yeri neresi?
\par 21 O bütün canlilarin gözünden uzaktir, Gökte uçan kuslardan bile saklidir.
\par 22 Yikim'la Ölüm: "Kulaklarimiz ancak fisiltisini duydu" der.
\par 23 Onun yolunu Tanri anlar, Yerini bilen O'dur.
\par 24 Çünkü O yeryüzünün uçlarina kadar bakar, Göklerin altindaki her seyi görür.
\par 25 Rüzgara güç verdigi, Sulari ölçtügü,
\par 26 Yagmura kural koydugu, Yildirima yol açtigi zaman,
\par 27 Bilgeligi görüp degerini biçti, Onu onaylayip arastirdi.
\par 28 Insana, "Iste Rab korkusu, bilgelik budur" dedi, "Kötülükten kaçinmak akilliliktir."

\chapter{29}

\par 1 Eyüp yine anlatmaya basladi:
\par 2 "Keske geçen aylar geri gelseydi, Tanri'nin beni kolladigi,
\par 3 Kandilinin basimin üstünde parladigi, Isigiyla karanlikta yürüdügüm günler,
\par 4 Keske olgunluk günlerim geri gelseydi, Tanri'nin çadirimi dostça korudugu,
\par 5 Her Seye Gücü Yeten'in henüz benimle oldugu, Çocuklarimin çevremde bulundugu,
\par 6 Yollarimin sütle yikandigi, Yanimdaki kayanin zeytinyagi akittigi günler!
\par 7 "Kent kapisina gidip Kürsümü meydana koydugumda,
\par 8 Gençler beni görüp gizlenir, Yaslilar kalkip ayakta dururlardi;
\par 9 Önderler konusmaktan çekinir, Elleriyle agizlarini kaparlardi;
\par 10 Soylularin sesi kesilir, Dilleri damaklarina yapisirdi.
\par 11 Beni duyan kutlar, Beni gören överdi;
\par 12 Çünkü yardim isteyen yoksulu, Destegi olmayan öksüzü kurtarirdim.
\par 13 Ölmekte olanin hayir duasini alir, Dul kadinin yüregini sevinçten costururdum.
\par 14 Dogrulugu giysi gibi giyindim, Adalet kaftanim ve sarigimdi sanki.
\par 15 Körlere göz, Topallara ayaktim.
\par 16 Yoksullara babalik eder, Garibin davasini üstlenirdim.
\par 17 Haksizin çenesini kirar, Avini dislerinin arasindan kapardim.
\par 18 "'Son solugumu yuvamda verecegim diye düsünüyordum, 'Günlerim kum taneleri kadar çok.
\par 19 Köküm sulara erisecek, Çiy geceyi dallarimda geçirecek.
\par 20 Aldigim övgüler tazelenecek, Elimdeki yay yenilenecek.
\par 21 "Insanlar beni saygiyla dinler, Ögüdümü sessizce beklerlerdi.
\par 22 Ben konustuktan sonra onlar konusmazdi, Sözlerim üzerlerine damlardi.
\par 23 Yagmuru beklercesine beni bekler, Son yagmurlari içercesine sözlerimi içerlerdi.
\par 24 Kendilerine gülümsedigimde gözlerine inanmazlardi, Güler yüzlülügüm onlara cesaret verirdi.
\par 25 Onlarin yolunu ben seçer, baslarinda dururdum, Askerlerinin ortasinda kral gibi otururdum, Yaslilari avutan biri gibiydim.

\chapter{30}

\par 1 "Ama simdi, yasi benden küçük olanlar Benimle alay etmekte, Oysa babalarini sürümün köpeklerinin Yanina koymaya tenezzül etmezdim.
\par 2 Çünkü güçleri tükenmisti, Bileklerinin gücü ne isime yarardi?
\par 3 Yoksulluktan, açliktan bitkindiler, Aksam çölde, issiz çorak yerlerde kök kemiriyorlardi.
\par 4 Çaliliklarda karapazi topluyor, Retem kökü yiyorlardi.
\par 5 Toplumdan kovuluyorlardi, Insanlar hirsizmislar gibi onlara bagiriyordu.
\par 6 Korkunç vadilerde, yerdeki deliklerde, Kaya kovuklarinda yasiyorlardi.
\par 7 Çalilarin arasinda anirir, Çali altinda birbirine sokulurlardi.
\par 8 Aptallarin, adi sani belirsiz insanlarin çocuklariydilar, Ülkeden kovulmuslardi.
\par 9 "Simdiyse destan oldum dillerine, Agizlarina doladilar beni.
\par 10 Benden tiksiniyor, uzak duruyorlar, Yüzüme tükürmekten çekinmiyorlar.
\par 11 Tanri ipimi çözüp beni alçalttigi için Dizginsiz davranmaya basladilar bana.
\par 12 Sagimdaki ayak takimi üzerime yürüyor, Ayaklarimi kaydiriyor, Bana karsi rampalar kuruyorlar.
\par 13 Yolumu kesiyor, Kimseden yardim görmeden Beni yok etmeye çalisiyorlar.
\par 14 Koca bir gedikten girer gibi ilerliyor, Yikintilar arasindan üzerime yuvarlaniyorlar.
\par 15 Dehset çöktü üzerime, Onurum rüzgara kapilmis gibi uçtu, Mutlulugum bulut gibi geçip gitti.
\par 16 "Simdi tükeniyorum, Aci günler beni ele geçirdi.
\par 17 Geceleri kemiklerim sizliyor, Beni kemiren acilar hiç durmuyor.
\par 18 Tanri'nin siddeti Üzerimdeki giysiye dönüstü, Gömlegimin yakasi gibi beni sikiyor.
\par 19 Beni çamura firlatti, Toza, küle döndüm.
\par 20 "Sana yakariyorum, ama yanit vermiyorsun, Ayaga kalktigimda gözünü bana dikiyorsun.
\par 21 Bana acimasiz davraniyor, Bileginin gücüyle beni eziyorsun.
\par 22 Beni kaldirip rüzgara bindiriyorsun, Firtinanin içinde darma duman ediyorsun.
\par 23 Biliyorum, beni ölüme, Bütün canlilarin toplanacagi yere götüreceksin.
\par 24 "Kuskusuz düsenin dostu olmaz, Felakete ugrayip yardim istediginde.
\par 25 Sikintiya düsenler için aglamaz miydim? Yoksullar için üzülmez miydim?
\par 26 Ama ben iyilik beklerken kötülük geldi, Isik umarken karanlik geldi.
\par 27 Içim kayniyor, rahatim yok, Önümde aci günler var.
\par 28 Yasli yasli dolasiyorum, günes yok, Topluluk içinde kalkip feryat ediyorum.
\par 29 Çakallarla kardes, Baykuslarla arkadas oldum.
\par 30 Derim karardi, soyuluyor, Kemiklerim atesten yaniyor.
\par 31 Lirimin sesi yas feryadina, Neyimin sesi aglayanlarin sesine döndü.

\chapter{31}

\par 1 "Gözlerimle antlasma yaptim Sehvetle bir kiza bakmamak için.
\par 2 Çünkü insanin yukaridan, Tanri'dan payi nedir, Yücelerden, Her Seye Gücü Yeten'den mirasi ne?
\par 3 Kötüler için felaket, Haksizlik yapanlar için bela degil mi?
\par 4 Yürüdügüm yollari görmüyor mu, Attigim her adimi saymiyor mu?
\par 5 "Eger yalan yolunda yürüdümse, Ayagim hileye segirttiyse,
\par 6 -Tanri beni dogru teraziyle tartsin, Kusursuz oldugumu görsün-
\par 7 Adimim yoldan saptiysa, Yüregim gözümü izlediyse, Ellerim pislige bulastiysa,
\par 8 Ektigimi baskalari yesin, Ekinlerim kökünden sökülsün.
\par 9 "Eger gönlümü bir kadina kaptirdiysam, Komsumun kapisinda pusuya yattiysam,
\par 10 Karim baskasinin bugdayini ögütsün, Onunla baska erkekler yatsin.
\par 11 Çünkü bu utanç verici, Yargilanmasi gereken bir suç olurdu.
\par 12 Yikim diyarina dek yakan bir atestir o, Bütün ürünümü kökünden kavururdu.
\par 13 "Benimle ters düstüklerinde Kölemin ve hizmetçimin hakkini yemissem,
\par 14 Tanri yargiladiginda ne yaparim? Hesap sordugunda ne yanit veririm?
\par 15 Beni ana karninda yaratan onu da yaratmadi mi? Rahimde bize biçim veren O degil mi?
\par 16 "Eger yoksullarin dilegini geri çevirdimse, Dul kadinin umudunu kirdimsa,
\par 17 Ekmegimi yalniz yedim, Öksüzle paylasmadimsa,
\par 18 Gençligimden beri öksüzü baba gibi büyütmedimse, Dogdugumdan beri dul kadina yol göstermedimse,
\par 19 Giysisi olmadigi için can çekisen birini Ya da örtüsü olmayan bir yoksulu gördüm de,
\par 20 Koyunlarimin yünüyle isitmadiysam, O da içinden beni kutsamadiysa,
\par 21 Mahkemede sözümün geçtigini bilerek Öksüze el kaldirdimsa,
\par 22 Kolum omuzumdan düssün, Kol kemigim kirilsin.
\par 23 Çünkü Tanri'dan gelecek beladan korkarim, O'nun görkeminden ötürü böyle bir sey yapamam.
\par 24 "Eger umudumu altina bagladimsa, Saf altina, 'Güvencim sensin dedimse,
\par 25 Servetim çok, Varligimi bilegimle kazandim diye sevindimse,
\par 26 Isildayan günese, Parildayarak hareket eden aya bakip da,
\par 27 Içimden ayartildimsa, Elim onlara taptigimi gösteren bir öpücük yolladiysa,
\par 28 Bu da yargilanacak bir suç olurdu, Çünkü yücelerdeki Tanri'yi yadsimis olurdum.
\par 29 "Eger düsmanimin yikimina sevindim, Basina kötülük geldi diye keyiflendimse,
\par 30 -Kimsenin canina lanet ederek Agzimin günah islemesine izin vermedim-
\par 31 Evimdeki insanlar, 'Eyüp'ün verdigi etle Karnini doyurmayan var mi? diye sormadiysa,
\par 32 -Hiçbir yabanci geceyi sokakta geçirmezdi, Çünkü kapim her zaman yolculara açikti-
\par 33 Kalabaliktan çok korktugum, Boylarin asagilamasindan yildigim, Susup disari çikmadigim için Suçumu bagrimda gizleyip Adem gibi isyanimi örttümse,
\par 35 -"Keske beni dinleyen biri olsa! Iste savunmami imzaliyorum, Her Seye Gücü Yeten bana yanit versin! Hasmimin yazdigi tomar elimde olsa,
\par 36 Kuskusuz onu omuzumda tasir, Taç gibi basima koyardim.
\par 37 Attigim her adimi ona bildirir, Kendisine bir önder gibi yaklasirdim.-
\par 38 "Topragim bana feryat ediyorsa, Sabanin açtigi yariklar bir agizdan agliyorsa,
\par 39 Ürününü para ödemeden yedimse Ya da üzerinde oturanlarin kalbini kirdimsa,
\par 40 Orada bugday yerine diken, Arpa yerine delice bitsin." Eyüp'ün konusmasi sona erdi.

\chapter{32}

\par 1 Böylece bu üç kisi Eyüp'e yanit vermekten vaz geçti, çünkü Eyüp kendi dogrulugundan emindi.
\par 2 Ram ailesinden Bûzlu Barakel oglu Elihu Eyüp'e çok öfkelendi. Çünkü Eyüp kendini Tanri'dan hakli görüyordu.
\par 3 Elihu Eyüp'ün üç arkadasina da öfkelendi, çünkü Eyüp'ü suçlamalarina karsin saglam bir yanit bulamamislardi.
\par 4 Elihu Eyüp'le konusmak için sirasini beklemisti, çünkü ötekiler yasça kendisinden büyüktü.
\par 5 Bu üç kisinin baska bir sey söyleyemeyecegini görünce öfkesi alevlendi.
\par 6 Bûzlu Barakel oglu Elihu söyle konustu: "Ben yasça küçügüm, sizse yaslisiniz. Bu yüzden çekindim, bildigimi söylemekten korktum.
\par 7 'Çok gün görenler konussun dedim, 'Çok yil yasayanlar bilgeligi ögretsin.
\par 8 Oysa insana ruh, Her Seye Gücü Yeten'in solugu akil verir.
\par 9 Akil yasta degil bastadir. Adaleti anlamak yasa bakmaz.
\par 10 "Bu yüzden, 'Beni dinleyin diyorum, Ben de bildigimi söyleyeyim.
\par 11 Siz konusurken ben bekledim, Siz ne diyeceginizi arastirirken Düsüncelerinizi dinledim.
\par 12 Bütün dikkatimi size çevirdim. Ama hiçbiriniz Eyüp'ün haksizligini kanitlayamadi, Onun söylediklerine karsilik veremedi.
\par 13 'Biz bilgelige eristik, Birakin Tanri onu haksiz çikarsin, insan degil demeyin.
\par 14 Ama Eyüp'ün sözlerinin hedefi ben degildim, Bu yüzden onu sizin sözlerinizle yanitlamayacagim.
\par 15 "Onlar yildi, yanit veremiyorlar artik, Söyleyecek seyleri kalmadi.
\par 16 Onlar konusmuyor diye ben beklemeli miyim, Duruyor, yanit vermiyorlar diye?
\par 17 Benim de söyleyecek sözüm var, Ben de bildigimi söyleyecegim.
\par 18 Çünkü içim dolu, Içimdeki ruh beni zorluyor.
\par 19 Içim açilmamis sarap gibi, Yeni sarap tulumlari gibi patlamak üzere.
\par 20 Konusup rahatlamaliyim, Agzimi açip yanitlamaliyim.
\par 21 Kimseye ayricalik göstermeyecek, Kimseye yaltaklanmayacagim.
\par 22 Çünkü yaltaklanmayi bilsem, Yaraticim beni hemen yok ederdi.

\chapter{33}

\par 1 "Ama simdi lütfen sözümü dinle, Eyüp, Söyleyecegim her seye kulak ver.
\par 2 Agzimi açtim açacagim, Söyleyeceklerim dilimin ucunda.
\par 3 Sözlerim temiz bir yürekten çikiyor, Dudaklarim bildiklerini içtenlikle söylüyor.
\par 4 Beni Tanri'nin Ruhu yaratti, Her Seye Gücü Yeten'in solugu yasam veriyor bana.
\par 5 Elinden gelirse beni yanitla, Kendini hazirla, karsimda dur.
\par 6 Tanri'nin önünde ben de tipki senin gibiyim, Ben de balçiktan yaratildim.
\par 7 Onun için dehsetim seni yildirmasin, Baskim sana agir gelmesin.
\par 8 "Sesin hâlâ kulaklarimda, Söyle demistin:
\par 9 'Ben kusursuz ve günahsizim, Temiz ve suçsuzum.
\par 10 Yine de Tanri bana karsi bahane ariyor, Beni düsman görüyor.
\par 11 Ayaklarimi tomruga vuruyor, Yollarimi gözetliyor.
\par 12 "Ama sana sunu söyleyeyim, Bu konuda haksizsin. Çünkü Tanri insandan büyüktür.
\par 13 Insanin hiçbir sözünü yanitlamiyor diye Niçin O'nunla çekisiyorsun?
\par 14 Çünkü insan anlamasa da, Tanri su ya da bu yolla konusur.
\par 15 Rüyada, geceleyin görümde, Insanlari agir uyku basinca, Yatakta yatarlarken,
\par 16 Kulaklarina konusur, Uyarisiyla onlari korkutur;
\par 17 Onlari yaptiklari kötülükten döndürmek, Gururdan uzak tutmak,
\par 18 Canlarini çukurdan, Hayatlarini ölümden kurtarmak için.
\par 19 Insan yataginda acilarla, Kemiklerinde dinmez sizilarla yola getirilir.
\par 20 Öyle ki, içi yemek kaldirmaz, En lezzetli yiyecekten tiksinir.
\par 21 Eti erir, görünmez olur, Gözükmeyen kemikleri ortaya çikar.
\par 22 Cani çukura, Hayati ölüm meleklerine yaklasir.
\par 23 "Yine de insana dogruyu bildirmek için Yaninda bir melek, bin melekten biri Arabulucu olarak bulunursa,
\par 24 Ona lütfeder de, 'Onu ölüm çukuruna inmekten kurtar, Ben fidyeyi buldum derse,
\par 25 Eti çocuk eti gibi yenilenir, Gençlik günlerine döner.
\par 26 Dua ettiginde Tanri ondan hosnut kalir, O da Tanri'nin yüzünü görüp sevinir. Tanri onun durumunu düzeltir.
\par 27 Sonra insanlarin önünde türkü çagirir: 'Günah isleyip dogru yoldan saptim, Ama Tanri hak ettigim cezayi vermedi bana,
\par 28 Canimi çukura inmekten O kurtardi, Isigi görmek için yasayacagim.
\par 29 "Iste, insanin canini çukurdan çikarmak, Onu yasam isigiyla aydinlatmak için Tanri bütün bunlari iki kez, Hatta üç kez yapar.
\par 31 "Iyi dinle, Eyüp, kulak ver, Sen sus, ben konusacagim.
\par 32 Söyleyecegin bir sey varsa söyle, Çünkü seni hakli çikarmak isterim.
\par 33 Yoksa, beni dinle, Sus da sana bilgelik ögreteyim."

\chapter{34}

\par 1 Elihu konusmasina söyle devam etti:
\par 2 "Ey bilgeler, sözlerimi dinleyin, Kulak verin bana, ey bilgi sahipleri.
\par 3 Çünkü damak nasil yemegi tadarsa, Kulak da sözleri sinar.
\par 4 Gelin, dogruyu seçelim, Iyiyi birlikte ögrenelim.
\par 5 "Çünkü Eyüp, 'Ben suçsuzum diyor, 'Tanri hakkimi elimden aldi.
\par 6 Hakli oldugum halde yalanci sayiliyorum, Suçsuz oldugum halde okunla yaraladin beni.
\par 7 Eyüp gibisi var mi? Alayi su gibi içiyor!
\par 8 Kötülük yapanlarla dostluk edip geziyor, Kötülerle ayni yolda yürüyor.
\par 9 Çünkü, 'Tanri'yi hosnut etmeye çalismak Insana yarar getirmez diyor.
\par 10 "Bu yüzden, ey sagduyulu insanlar, beni dinleyin! Tanri kötülük yapar mi, Her Seye Gücü Yeten haksizlik eder mi? Asla!
\par 11 Çünkü O herkese yaptiginin karsiligini öder, Hak ettigini basina getirir.
\par 12 Tanri kesinlikle kötülük etmez, Her Seye Gücü Yeten adaleti saptirmaz.
\par 13 Kim yeryüzünü O'na emanet etti? Kim O'nu bütün dünyanin basina atadi?
\par 14 Eger niyet eder de Ruhunu ve solugunu geri çekerse,
\par 15 Bütün insanlik bir anda yok olur, Insan yine topraga döner.
\par 16 "Aklin varsa dinle, Kulak ver sözlerime.
\par 17 Adaletten nefret eden hiç hüküm sürebilir mi? Adil ve güçlü olani suçlayacak misin?
\par 18 Krallara, 'Degersizsiniz, Soylulara, 'Kötüsünüz diyen,
\par 19 Önderlere ayricalik tanimayan, Zengini yoksuldan çok önemsemeyen O degil mi? Çünkü hepsi O'nun ellerinin isidir.
\par 20 Gece yarisi bir anda ölürler, Herkes sarsilir, ölüp gider, Güçlüler de insan eli degmeden alinip götürülür.
\par 21 "Tanri'nin gözleri insanlarin yolundan ayrilmaz, Attiklari her adimi görür.
\par 22 Kötülük yapanlarin gizlenebilecegi Ne karanlik bir yer vardir, ne de ölüm gölgesi.
\par 23 Yargilanmak için önüne gelsinler diye, Tanri insanlari sorgulamaya pek gerek duymaz.
\par 24 Arastirmadan güçlü insanlari kirar, Onlarin yerine baskalarini diker.
\par 25 Çünkü ne yaptiklarini bilir, Gece onlari deviriverir, ezilirler.
\par 26 Herkesin gözü önünde Kötülükleri yüzünden onlari cezalandirir;
\par 27 Artik O'nun ardindan gitmedikleri, Yollarinin hiçbirini dikkate almadiklari için.
\par 28 Yoksulun feryadini O'na duyurdular; Düskünlerin feryadini isitti.
\par 29 Ama Tanri sessiz kalirsa kim O'nu suçlayabilir? Yüzünü gizlerse kim O'nu görebilir? Bir ulusa karsi da bir insana karsi da O hep aynidir,
\par 30 Tanrisiz insan krallik etmesin, Halka tuzak kurmasin diye.
\par 31 "Kimse Tanri'ya, 'Suçluyum, artik kötülük yapmayacagim dedi mi,
\par 32 'Göremedigimi sen bana ögret, Haksizlik ettimse, bir daha etmem?
\par 33 O'nu reddettigin halde, Senin keyfince mi seni ödüllendirmeli? Çünkü karar verecek olan sensin, ben degil, Öyleyse anlat bana bildigini.
\par 34 "Sagduyulu insanlar, Beni dinleyen bilgeler diyecekler ki,
\par 35 'Eyüp bilgisizce konusuyor, Sözlerinin degeri yok.
\par 36 Kötü biri gibi yanitladigi için Keske Eyüp'ün sinanmasi sonsuza dek sürse!
\par 37 Çünkü günahina isyan da ekliyor, Önümüzde alay edercesine el çirpiyor, Tanri'ya karsi konustukça konusuyor."

\chapter{35}

\par 1 Elihu konusmasina söyle devam etti:
\par 2 "'Tanri'nin önünde hakliyim diyorsun. Dogru buluyor musun bunu?
\par 3 Ama hâlâ, 'Günah islemezsem Yararim ne, kazancim ne? diye soruyorsun.
\par 4 "Ben yanitlayayim seni Ve arkadaslarini.
\par 5 Göklere bak da gör, Üzerinde yükselen bulutlara göz gezdir.
\par 6 Günah islersen, Tanri'ya ne zarari olur? Isyanlarin çoksa ne olur O'na?
\par 7 Dogruysan, O'na verdigin nedir, Ya da ne alir O senin elinden?
\par 8 Kötülügün ancak senin gibi birine zarar verir, Dogrulugun ise yalniz insanoglu içindir.
\par 9 "Insanlar agir baski altinda feryat ediyor, Güçlülere karsi yardim istiyor.
\par 10 Ama kimse, 'Nerede Yaraticim Tanri? demiyor; O Tanri ki, gece bize ezgiler verir,
\par 11 Yeryüzündeki hayvanlardan çok bize ögretir Ve bizi gökteki kuslardan daha bilge kilar.
\par 12 Kötülerin gururu yüzünden insanlar feryat ediyor, Ama yanitlayan yok.
\par 13 Gerçek su ki, Tanri bos feryadi dinlemez, Her Seye Gücü Yeten bunu önemsemez.
\par 14 O'nu görmedigini söyledigin zaman bile Davan O'nun önündedir, bekle;
\par 15 Madem bu öfkeyle simdi cezalandirmadi, Isyana da pek aldirmaz diyorsun.
\par 16 Bu yüzden Eyüp agzini bos yere açiyor, Bilgisizce konustukça konusuyor."

\chapter{36}

\par 1 Elihu konusmasina söyle devam etti:
\par 2 "Biraz bekle, sana açiklayayim, Çünkü Tanri için söylenecek daha çok söz var.
\par 3 Bilgimi genis kaynaklardan toplayacagim, Yaraticima hak verecegim.
\par 4 Kuskusuz söyledigim hiçbir sey yalan degil, Karsinda bilgide yetkin biri var.
\par 5 "Tanri güçlüdür, ama kimseyi hor görmez, Güçlü ve amacinda kararli.
\par 6 Kötüleri yasatmaz, Ezilenin hakkini verir.
\par 7 Gözlerini dogru kisiden ayirmaz, Onu krallarla birlikte tahta oturtur, Sonsuza dek yükseltir.
\par 8 Ama insanlar zincire vurulur, Baski altinda tutulurlarsa,
\par 9 Onlara yaptiklarini, Gurura kapilip isyan ettiklerini bildirir.
\par 10 Ögüdünü dinletir, Kötülükten dönmelerini buyurur.
\par 11 Eger dinler ve O'na kulluk ederlerse, Kalan günlerini bolluk, Yillarini rahatlik içinde geçirirler.
\par 12 Ama dinlemezlerse ölür, Ders almadan yok olurlar.
\par 13 "Tanrisizlar öfkelerini içlerinde gizler, Kendilerini bagladiginda Tanri'dan yardim istemezler.
\par 14 Genç yasta ölüp giderler, Yasamlari putperest tapinaklarinda fuhsu is edinmis erkekler arasinda sona erer.
\par 15 Ama Tanri aci çekenleri aci çektikleri için kurtarir, Düskünlere kendini dinletir.
\par 16 "Evet, seni sikintidan çeker çikarirdi; Darligin olmadigi genis bir yere, Zengin yiyeceklerle bezenmis bir sofraya.
\par 17 Oysa simdi kötülerin hak ettigi cezayi çekiyorsun, Yargi ve adalet yakalamis seni.
\par 18 Dikkat et, para seni bastan çikarmasin, Büyük bir rüsvet seni saptirmasin.
\par 19 Zenginligin ya da bütün gücün yeter mi Sikinti çekmeni önlemeye?
\par 20 Halklarin yeryüzünden Yok edildigi geceyi özleme.
\par 21 Dikkat et, kötülüge dönme, Çünkü sen onu düskünlüge yegledin.
\par 22 "Iste Tanri gücüyle yükselir, O'nun gibi ögretmen var mi?
\par 23 Kim O'na ne yapmasi gerektigini söyleyebilir? Kim O'na, 'Haksizlik ettin diyebilir?
\par 24 O'nun islerini yüceltmelisin, animsa bunu, Insanlarin ezgilerle övdügü islerini.
\par 25 Bütün insanlar bunlari görmüstür, Herkes onlari uzaktan izler.
\par 26 Evet, Tanri öyle büyüktür ki, O'nu anlayamayiz, Varliginin süresi hesaplanamaz.
\par 27 "Su damlalarini yukari çeker, Buharindan yagmur damlatir.
\par 28 Bulutlar nemini döker, Insanlarin üzerine bol yagmur yagdirir.
\par 29 Bulutlari nasil yaydigini, Göksel konutundan nasil gürledigini kim anlayabilir?
\par 30 Simsekleri çevresine nasil yaydigina, Denizin dibine dek nasil ulastirdigina bakin.
\par 31 Tanri halklari böyle yönetir, Bol yiyecek saglar.
\par 32 Simsegi elleriyle tutar, Hedefine vurmasini buyurur.
\par 33 O'nun gürleyisi firtinayi haber verir, Sigirlar bile firtina kopacagini bildirir.

\chapter{37}

\par 1 "Yüregim titrer buna, Yerinden oynar.
\par 2 Dinleyin, gürleyen sesini dinleyin, Agzindan çikan sesi!
\par 3 Simsegini gögün altindaki her yere, Yeryüzünün dört bucagina salar.
\par 4 Ardindan bir ses gümbürder, Görkemli sesiyle gürler. Sesi duyulunca simsekleri alikoymaz.
\par 5 Tanri'nin sesi sasilacak biçimde gürler, O, anlayisimizin ötesinde büyük isler yapar.
\par 6 Çünkü kara, 'Yere düs der, Saganaga, 'Bütün siddetinle bosal.
\par 7 Yarattigi bütün insanlar ne yaptigini bilsin diye, Herkese isini biraktirir.
\par 8 Hayvanlar kovuklarina girer, Inlerinde otururlar.
\par 9 Kasirga yuvasindan kopar, Soguk saçilan rüzgarlardan.
\par 10 Tanri'nin solugu sulari dondurur, Genis sular buz tutar.
\par 11 Bulutlara nem yükler, Simsegini her yana yayar.
\par 12 Yeryüzünde ne buyurursa yapmak üzere Bulutlar O'nun istedigi yönde döner durur.
\par 13 Ya insanlari cezalandirmak Ya da yeryüzünü sulayip sevgisini göstermek için Yagmur gönderir.
\par 14 "Dinle, Eyüp, Dur da düsün Tanri'nin sasilasi islerini.
\par 15 Tanri'nin bulutlari nasil düzenledigini, Simsegini nasil çaktirdigini biliyor musun?
\par 16 Bulutlarin dengesini, Bilgisi kusursuz olanin sasilasi islerini biliyor musun?
\par 17 Dünyanin solugu kesildiginde Güneyin kavurucu rüzgari altinda Giysilerin seni terletmez mi?
\par 18 Dökme tunç* bir ayna kadar sert olan gökkubbeyi O'nunla birlikte yayabilir misin?
\par 19 "O'na ne söyleyecegimizi ögret bize, Çünkü karanlik yüzünden sözümüze düzen veremiyoruz.
\par 20 Konusmak istedigim O'na söylenebilir mi? Kimse yutulmak ister mi?
\par 21 Rüzgar geçip gögü temizlediginde Gökte parildayan isiga kimse bakamaz.
\par 22 Altin pariltisi geliyor kuzeyden, Tanri korkunç görkeme bürünmüs.
\par 23 Her Seye Gücü Yeten'e biz ulasamayiz. Gücü yücedir, Adaleti ve essiz dogruluguyla kimseyi ezmez.
\par 24 Bu yüzden insanlar O'na saygi duyar, Çünkü O, bilgeleri dikkate almaz."

\chapter{38}

\par 1 RAB kasirganin içinden Eyüp'ü söyle yanitladi:
\par 2 "Bilgisizce sözlerle Tasarimi karartan bu adam kim?
\par 3 Simdi erkek gibi kusagini beline vur da, Ben sorayim, sen anlat.
\par 4 "Ben dünyanin temelini atarken sen neredeydin? Anliyorsan söyle.
\par 5 Kim saptadi onun ölçülerini? Kuskusuz biliyorsun! Kim çekti ipi üzerine?
\par 6 Neyin üstüne yapildi temelleri? Kim koydu köse tasini,
\par 7 Sabah yildizlari birlikte sarki söylerken, Ilahi varliklar sevinçle çigrisirken?
\par 8 "Denizin ardindan kapilari kim kapadi, Ana rahminden fiskirdigi zaman;
\par 9 Ona bulutlari giysi, Koyu karanligi kundak yaptigim,
\par 10 Sinirini koydugum, Kapilariyla sürgülerini yerlestirdigim,
\par 11 'Buraya kadar gelip öteye geçmeyeceksin, Gururlu dalgalarin surada duracak dedigim zaman?
\par 12 "Sen ömründe sabaha buyruk verdin mi, Safaga yerini gösterdin mi;
\par 13 Yeryüzünün uçlarini tutsun, Oradaki kötüler silkilip atilsin diye?
\par 14 Mühür basilan balçik gibi biçim degistirir yeryüzü, Giysi kivrimlari gibi göze çarpar.
\par 15 Kötülerin isiklari alinir, Kalkan kollari kirilir.
\par 16 "Denizin kaynaklarina vardin mi, Gezdin mi enginin diplerinde?
\par 17 Ölüm kapilari sana gösterildi mi? Gördün mü ölüm gölgesinin kapilarini?
\par 18 Dünyanin genisligini kavradin mi? Anlat bana, bütün bunlari biliyorsan.
\par 19 "Isigin bulundugu yerin yolu nerede? Ya karanlik, onun yeri neresi?
\par 20 Onlari yerlerine götürebilir misin? Evlerinin yolunu biliyor musun?
\par 21 Bilmedigin sey yok zaten, Çünkü onlarla ayni zamanda dogmustun! O kadar yaslisin!
\par 22 "Karin ambarlarina girdin mi, Dolunun ambarlarini gördün mü?
\par 23 Ben onlari sikintili günler için, Kavga ve savas günleri için sakliyorum.
\par 24 Nerede isigin dagitildigi, Dogu rüzgarinin yeryüzüne saçildigi yere giden yol?
\par 25 Kim sellere kanal, Yildirimlara yol açti;
\par 26 Kimsenin yasamadigi topraklari, Insanin bulunmadigi çölü sulasin diye;
\par 27 Kurak ve issiz yeri doyursun, Ot bitirsin diye?
\par 28 Yagmurun babasi var mi? Çiy damlalarini kim yaratti?
\par 29 Buz kimin rahminden çikti? Göklerden düsen kiragiyi kim dogurdu,
\par 30 Sular tas gibi katilasip Enginin yüzü donunca?
\par 31 "Ülker yildizlarini baglayabilir misin? Oryon'un baglarini çözebilir misin?
\par 32 Mevsimlerinde çikartabilir misin takimyildizlari? Büyük ve Küçük Ayi'ya yol gösterebilir misin?
\par 33 Biliyor musun göklerin yasalarini? Tanri'nin yönetimini yeryüzünde kurabilir misin?
\par 34 "Basina bol yagmur yagsin diye Bulutlara sesini duyurabilir misin?
\par 35 Varip da, 'Buradayiz desinler diye, Simsekleri gönderebilir misin?
\par 36 Kim misirturnasina bilgelik, Horoza anlayis verdi?
\par 37 Kimin bulutlari sayacak bilgisi var? Kim göklerin tulumlarini bosaltabilir,
\par 38 Toprak sertlesip Parçalari birbirine yapisinca?
\par 39 "Disi aslanlar için sen avlanabilir misin, Genç aslanlarin karnini doyurabilir misin,
\par 40 Inlerine sindikleri, Çalilikta pusuya yattiklari zaman?
\par 41 Kuzguna yiyecegini kim sagliyor, Yavrulari Tanri'ya feryat edip Açliktan kivrandigi zaman?

\chapter{39}

\par 1 "Dag keçilerinin ne zaman dogurdugunu biliyor musun? Geyiklerin yavruladigi zamani sen mi gözlüyorsun?
\par 2 Sen mi sayiyorsun doguruncaya dek geçirdikleri aylari? Dogurduklari zamani biliyor musun?
\par 3 Çöküp yavrularini dogurur, Kurtulurlar sancilarindan.
\par 4 Güçlenir, kirda büyür yavrular, Gider, bir daha dönmezler.
\par 5 "Kim yaban esegini basi bos gönderdi, Kim baglarini çözdü?
\par 6 Yurt olarak ona bozkiri, Barinak olarak tuzlayi verdim.
\par 7 Kentteki kargasaya güler o, Sürücünün bagirdigini duymaz.
\par 8 Otlamak için tepeleri dolasir, Yesillik arar.
\par 9 "Yaban öküzü sana kulluk etmek ister mi? Geceyi senin yemliginin yaninda geçirir mi?
\par 10 Sabanla yarik açsin diye ona bag vurabilir misin? Arkanda, ovalarda tirmik çeker mi?
\par 11 Çok güçlü diye ona bel baglayabilir misin? Agir isini ona birakabilir misin?
\par 12 Ekinini getirecegine, Bugdayini harman yerinde toplayacagina güvenir misin?
\par 13 "Devekusunun kanatlari sevinçle dalgalanir, Ama leylegin kanatlari ve tüyleriyle kiyaslanamaz.
\par 14 Devekusu yumurtalarini yere birakir, Onlari kumda isitir,
\par 15 Ayak altinda ezilebileceklerini, Yabanil hayvanlarca çignenebileceklerini düsünmez.
\par 16 Yavrularina sert davranir, kendinin degilmis gibi, Çektigi zahmetin bosa gidecegine üzülmez.
\par 17 Çünkü Tanri ona bilgelik bagislamamis, Anlayistan pay vermemistir.
\par 18 Yine de kosmak için kabarinca Ata ve binicisine güler.
\par 19 "Sen mi ata güç verdin, Dalgalanan yeleyi boynuna giydirdin?
\par 20 Sen misin onu çekirge gibi siçratan, Gururlu kisnemesiyle korku saçtiran?
\par 21 Ayaklari topragi siddetle eser, Gücünden ötürü sevinçle cosar, Savasçinin üstüne yürür.
\par 22 Korkuya güler, hiçbir seyden yilmaz, Kiliç önünde geri adim atmaz.
\par 23 Ok kilifi, parildayan mizrak ve pala Üzerinde takirdar atin.
\par 24 Cosku ve heyecanla uzakliklari yutar, Boru çalinca duramaz yerinde.
\par 25 Boru çaldikça, 'Hi! diye kisner, Savas kokusunu, komutanlarin gürleyen sesini, Savas çigliklarini uzaklardan duyar.
\par 26 "Atmaca senin bilgeliginle mi süzülüyor, Kanatlarini güneye dogru açiyor?
\par 27 Kartal senin buyrugunla mi yükseliyor, Yuvasini yükseklere kuruyor?
\par 28 Uçurum kenarlarinda konakliyor, Sivri kayalar onun kalesi.
\par 29 Oradan gözetliyor yiyecegini, Gözleri avini uzaktan seçiyor.
\par 30 Onun yavrulari kanla beslenir, Lesler neredeyse, o da oradadir."

\chapter{40}

\par 1 RAB Eyüp'e söyle dedi:
\par 2 "Her Seye Gücü Yeten'le çatisan O'nu yola getirebilir mi? Tanri'yi suçlayan yanitlasin."
\par 3 O zaman Eyüp RAB'bi söyle yanitladi:
\par 4 "Bak, ben degersiz biriyim, Sana nasil yanit verebilirim? Agzimi elimle kapiyorum.
\par 5 Bir kez konustum, yanit almadim, Ikinci kez konusamam artik."
\par 6 RAB kasirganin içinden Eyüp'ü söyle yanitladi:
\par 7 "Simdi erkek gibi kusagini beline vur da, Ben sorayim, sen anlat.
\par 8 "Adaletimi bosa mi çikaracaksin? Kendini hakli çikarmak için beni mi suçlayacaksin?
\par 9 Sende Tanri'nin bilegi gibi bilek var mi? Sesin O'nunki gibi gürleyebilir mi?
\par 10 Öyleyse san ve serefe bürün, Görkem ve yücelik kusan.
\par 11 Gazabinin atesini saç, Gururluya bakip onu alçalt.
\par 12 Gururluya bakip onu çökert, Kötüleri bulunduklari yerde ez.
\par 13 Hepsini birlikte topraga göm, Mezarda yüzlerini kefenle sar.
\par 14 O zaman sag kolunun seni kurtarabilecegini Ben de kabul ederim.
\par 15 "Seninle birlikte yarattigim Behemot'a bak, Sigir gibi ot yiyor.
\par 16 Bak, ne güç var belinde, Karninin kaslari ne güçlü!
\par 17 Kuyrugunu sedir agaci gibi salliyor, Simsikidir uyluk lifleri.
\par 18 Kemikleri tunç* borular, Kaburgalari demir çubuklar gibidir.
\par 19 Tanri'nin yapitlari arasinda ilk sirayi alir, Yalniz Yaraticisi ona kiliçla yaklasir.
\par 20 Tepeler ürünlerini ona getirir, Bütün yabanil hayvanlar yaninda oynasir.
\par 21 Hünnap çalilari altinda, Kamislarla örtülü bir bataklikta yatar.
\par 22 Hünnaplar onu gölgelerinde saklar, Vadideki kavaklar kusatir.
\par 23 Irmak cossa bile o ürkmez, Güvenlik içindedir, Seria Irmagi bogazina dayansa bile.
\par 24 Gözleri açikken kim onu tutabilir, Kim kancayla burnunu delebilir?

\chapter{41}

\par 1 "Livyatan'i çengelle çekebilir misin, Dilini halatla baglayabilir misin?
\par 2 Burnuna sazdan ip takabilir misin, Kancayla çenesini delebilir misin?
\par 3 Yalvarip yakarir mi sana, Tatli tatli konusur mu?
\par 4 Seninle antlasma yapar mi, Onu ömür boyu köle edesin diye?
\par 5 Kusla oynar gibi onunla oynayabilir misin, Hizmetçilerin eglensin diye ona tasma takabilir misin?
\par 6 Balikçilar onun üzerine pazarlik eder mi? Tüccarlar aralarinda onu böler mi?
\par 7 Derisini zipkinlarla, Basini mizraklarla doldurabilir misin?
\par 8 Elini üzerine koy da, çikacak çingari gör, Bir daha yapmayacaksin bunu.
\par 9 Onu yakalamak için umutlanma, Görünüsü bile insanin ödünü patlatir.
\par 10 Onu uyandiracak kadar yürekli adam yoktur. Öyleyse benim karsimda kim durabilir?
\par 11 Kim benden hesap vermemi isteyebilir? Göklerin altinda ne varsa bana aittir.
\par 12 "Onun kollari, bacaklari, Zorlu gücü, güzel yapisi hakkinda Konusmadan edemeyecegim.
\par 13 Onun giysisinin önünü kim açabilir? Kim onun iki katli zirhini delebilir?
\par 14 Agzinin kapilarini açmaya kim yeltenebilir, Dehset verici disleri karsisinda?
\par 15 Simsiki kenetlenmistir Sirtindaki sira sira pullar,
\par 16 Öyle yakindir ki birbirine Aralarindan hava bile geçmez.
\par 17 Birbirlerine geçmisler, Yapismis, ayrilmazlar.
\par 18 Aksirmasi isik saçar, Gözleri safak gibi parildar.
\par 19 Agzindan alevler fiskirir, Kivilcimlar saçilir.
\par 20 Kaynayan kazandan, Yanan sazdan çikan duman gibi Burnundan duman tüter.
\par 21 Solugu kömürleri tutusturur, Alev çikar agzindan.
\par 22 Boynu güçlüdür, Dehset önü sira gider.
\par 23 Etinin katmerleri birbirine yapismis, Sertlesmis üzerinde, kimildamazlar.
\par 24 Gögsü tas gibi serttir, Degirmenin alt tasi gibi sert.
\par 25 Ayaga kalkti mi güçlüler dehsete düser, Çikardigi gürültüden ödleri patlar.
\par 26 Üzerine gidildi mi ne kiliç isler, Ne mizrak, ne cirit, ne de kargi.
\par 27 Demir saman gibi gelir ona, Tunç* çürük odun gibi.
\par 28 Oklar onu kaçirmaz, Aniz gibi gelir ona sapan taslari.
\par 29 Aniz sayilir onun için topuzlar, Vinlayan palaya güler.
\par 30 Keskin çömlek parçalari gibidir karninin alti, Döven gibi uzanir çamura.
\par 31 Derin sulari kaynayan kazan gibi fokurdatir, Denizi merhem çömlegi gibi karistirir.
\par 32 Ardinda parlak bir iz birakir, Insan enginin saçlari agarmis sanir.
\par 33 Yeryüzünde bir esi daha yoktur, Korkusuz bir yaratiktir.
\par 34 Kendini büyük gören her varligi asagilar, Gururlu her varligin krali odur."

\chapter{42}

\par 1 O zaman Eyüp RAB'bi söyle yanitladi:
\par 2 "Senin her seyi yapabilecegini biliyorum, Hiçbir amacina engel olunmaz.
\par 3 'Tasarimi bilgisizce karartan bu adam kim? Diye sordun. Kuskusuz anlamadigim seyleri konustum, Beni asan, bilmedigim sasilasi isleri.
\par 4 "'Dinle de konusayim dedin, 'Ben sorayim, sen anlat.
\par 5 Kulaktan duymaydi bildiklerim senin hakkinda, Simdiyse gözlerimle gördüm seni.
\par 6 Bu yüzden kendimi hor görüyor, Toz ve kül içinde tövbe ediyorum."
\par 7 RAB Eyüp'le konustuktan sonra, Temanli Elifaz'a: "Sana ve iki dostuna karsi öfkem alevlendi" dedi, "Çünkü kulum Eyüp gibi hakkimda dogruyu konusmadiniz.
\par 8 Simdi yedi boga, yedi koç alip kulum Eyüp'ün yanina gidin, kendiniz için yakmalik sunu* sunun. Kulum Eyüp sizin için dua etsin. Çünkü onun duasini kabul eder, aptalliginizin karsiligini vermem. Kulum Eyüp gibi hakkimda dogruyu konusmadiniz."
\par 9 Temanli Elifaz, Suahli Bildat, Naamali Sofar gidip RAB'bin söyledigini yaptilar. RAB de Eyüp'ün duasini kabul etti.
\par 10 Eyüp dostlari için dua ettikten sonra, RAB onu eski gönencine kavusturup ona önceki varliginin iki katini verdi.
\par 11 Bütün erkek ve kiz kardesleri, eski tanidiklarinin hepsi Eyüp'ün yanina gelip evinde onunla birlikte yemek yediler. Acisini paylasip RAB'bin basina getirmis oldugu felaketlerden ötürü onu avuttular. Her biri ona bir parça gümüs, bir de altin halka verdi.
\par 12 RAB Eyüp'ün sonunu basindan bereketli kildi. On dört bin koyuna, alti bin deveye, bin çift öküze, bin esege sahip oldu.
\par 13 Yedi oglu, üç kizi oldu.
\par 14 Ilk kizinin adini Yemima, ikincisinin Kesia, üçüncüsünün Keren-Happuk koydu.
\par 15 Ülkenin hiçbir yerinde Eyüp'ün kizlari kadar güzel kizlar yoktu. Babalari, kardeslerinin yanisira onlara da miras verdi.
\par 16 Bundan sonra Eyüp yüz kirk yil daha yasadi, ogullarini, dört göbek torunlarini gördü.
\par 17 Kocayip yasama doyarak öldü.


\end{document}