\begin{document}

\title{Ağıtlar}


\chapter{1}

\par 1 O kent ki, insan doluydu, Nasil da tek basina kaldi simdi! Büyüktü uluslar arasinda, Dul kadina döndü! Soyluydu iller arasinda, Angarya altina düstü!
\par 2 Geceleyin aci aci agliyor, Yanaklarinda gözyasi; Avutan tek kisi bile yok Bunca oynasi arasinda. Dostlari ona hainlik etti, Düsman oldu.
\par 3 Yahuda aci çekip agir kölelik ettikten sonra Sürgün edildi, Uluslarin arasinda oturuyor, Ama rahat bulamiyor. O sikintidayken ardina düsenler ona yetisti.
\par 4 Siyon'a giden yollar yas tutuyor, Çünkü bayramlara gelen yok. Bütün kapilari issiz, kâhinleri* inliyor, Erden kizlari sikintida, kendisi de aci çekiyor.
\par 5 Hasimlari basa geçti, düsmanlari rahat içinde. Çok isyan ettigi için RAB ona aci çektiriyor, Yavrulari hasimlarinin gözü önünde sürgüne gitti.
\par 6 Siyon kizinin* bütün güzelligi uçtu, Önderleri otlak bulamayan geyiklere döndü, Dermanlari kesildi Kendilerini kovalayanlarin önünde.
\par 7 Yerusalim sikinti içinde basibos dolasirken Eski günlerdeki varligini animsiyor. Halki hasminin eline düsüp de Yardimina kosan çikmayinca, Hasimlari haline bakip Yikilisina güldüler.
\par 8 Yerusalim büyük günah isledi, Bu yüzden kirlendi. Ona saygi duyanlarin hepsi Simdi onu hor görüyor, Çünkü onu çiplak gördüler. O da inleyip öbür yana dönüyor.
\par 9 Kirliligi eteklerindeydi, Sonunu düsünmedi; Bu yüzden düsüsü korkunç oldu, Avutani yok. "Ya RAB, düskün halimi gör, Çünkü düsmanim kazandi!"
\par 10 Degerli her seyine düsman el uzatti. Tapinagina baska uluslarin girdigini gördü, Topluluguna girmesini yasakladigin uluslar.
\par 11 Halki inleyip ekmek ariyor, Yeniden güçlerine kavusmak için Degerli neleri varsa ekmekle degistiler; "Bak da gör, ya RAB, ne kadar sefil oldum."
\par 12 "Ey sizler, yoldan geçenler, Sizin için önemi yok mu bunun? Bakin da görün, basima gelen dert gibisi var mi? Öyle bir dert ki, RAB öfkesinin alevlendigi gün Basima yagdirdi onu.
\par 13 Ates saldi yukaridan, Kemiklerimin içine isledi ates; Ag serdi ayaklarima, Geri çevirdi beni; Mahvetti, baygin kaldim bütün gün.
\par 14 Isyanlarim boyunduruga döndü, RAB'bin eliyle birbirine tutturulup Boynuma geçirildi, gücüm tükendi. Rab karsi duramadigim Insanlarin eline verdi beni.
\par 15 Hiçe saydi beni savunan yigitleri, Gençlerimi kirip geçirmek için çagri yapti ordulara, Rab erden Yahuda kizini Üzüm sikma çukurunda çignedi adeta.
\par 16 "Agliyorum bunlara, Gözlerimden yaslar bosaniyor; Çünkü beni avutan, Canimi tazeleyen benden uzak. Çocuklarim saskina döndü, Çünkü düsmanim üstün çikti."
\par 17 Siyon ellerini açmis, Ama onu avutan yok. RAB Yakup soyuna karsi buyruk verdi, Komsulari ona hasim olsun, dedi. Yerusalim aralarinda paçavraya döndü.
\par 18 "RAB haklidir, çünkü buyruguna karsi geldim. Simdi dinleyin, ey halklar, çektigim aciyi görün; Erden kizlarim, gençlerim sürgüne gitti.
\par 19 Oynaslarimi çagirdim, Ama aldattilar beni. Yeniden güçlerine kavusmak için yiyecek ararken Kâhinlerimle önderlerim kentte can verdi.
\par 20 Gör, ya RAB, ne sikintilar çektigimi, Içim kaniyor, yüregim buruk, Çünkü çok asilik ettim; Disarida kiliç beni çocuklarimdan ayirmakta, Içerdeyse ölüm kol gezmekte.
\par 21 Inledigimi duydular, Beni avutan olmadi. Bütün düsmanlarim basima gelen felaketi duydu, Sen yaptin diye sevinçten costular. Ilan ettigin günü getir, Onlar da benim gibi olsunlar.
\par 22 Yaptiklari her kötülügü animsa, Isyanlarimdan ötürü bana ne yaptinsa onlara da yap; Çünkü sürekli inliyor, bayginlik geçiriyorum."

\chapter{2}

\par 1 Rab öfkelenince Siyon kizini* nasil bulutla kapladi! Israil'in görkemini gökten yere firlatti, Öfkelendigi gün ayaginin taburesini animsamadi.
\par 2 Yakup soyunun yasadigi her yeri acimadan yuttu, Yahuda kizinin surlu kentlerini gazabiyla yikti, Yerle bir etti onlari, Kralligini ve önderlerini alçaltti.
\par 3 Kizgin öfkesiyle Israil'in gücünü kökünden kesti, Düsmanin önünde sag elini onlarin üstünden çekti, Çevresini yiyip bitiren alevli ates gibi Yakup soyunu yakti.
\par 4 Düsman gibi yayini gerdi, Hasim gibi sag elini kaldirdi, Göz zevkini oksayan herkesi öldürdü, Gazabini Siyon kizinin çadiri üstüne ates gibi döktü.
\par 5 Rab adeta bir düsman olup Israil'i yuttu, Bütün saraylarini yutup surlu kentlerini yikti, Yahuda kizinin feryadini, figanini arsa çikardi.
\par 6 Bahçe çardagini söker gibi kendi çardagini söküp atti, Bulusma yerini yok etti, RAB Siyon'da bayram ve Sabat* günlerini unutturdu, Siddetli öfkesi yüzünden krali da kâhini de reddetti.
\par 7 Rab sunagini atti, Tapinagini terk etti; Siyon saraylarini çeviren surlari düsman eline birakti. Bayram gününde oldugu gibi, Düsman RAB'bin Tapinagi'nda sevinç çigliklari atti.
\par 8 RAB Siyon kizinin surlarini yikmaya karar verdi, Ipi gerdi ve yikmaktan el çekmedi, Iç ve dis surlara yas tutturdu, Ikisinin de gücü tükendi.
\par 9 Siyon'un kapilari yere batti, RAB kapi sürgülerini kirip yok etti, Kraliyla önderleri baska uluslarin arasinda kaldi, Kutsal Yasa uygulanmaz oldu, Peygamberlerine RAB'den görüm gelmiyor artik.
\par 10 Siyon kizinin ileri gelenleri suskun, yere oturmus, Baslarina toprak saçip çul kusanmislar, Yerusalim'in erden kizlari yere egmis baslarini.
\par 11 Gözlerim tükenmekte aglamaktan, Içim kaniyor; Halkimin yikimindan Yüregim sizliyor, Çünkü kent meydanlarinda çocuklarla bebekler bayilmakta.
\par 12 Kent meydanlarinda yaralilar gibi bayilip Can çekisirken annelerinin bagrinda, "Ekmekle sarap nerede?" diye soruyorlar annelerine.
\par 13 Senin için ne diyeyim? Ey Yerusalim kizi*, seni neye benzeteyim? Ey Siyon'un erden kizi, sana neyi örnek göstereyim de Seni avutayim? Sendeki gedik deniz kadar büyük, Kim sana sifa verebilir?
\par 14 Peygamberlerin senin için bos ve anlamsiz görümler gördüler. Suçunu ortaya çikarsalardi, eski gönencine kavusabilirdin; Oysa seni ayartacak bos görümler gördüler.
\par 15 Yoldan geçen herkes el çirparak seninle alay ediyor, Yerusalim kizina bas sallayip islik çalarak, "Bütün dünyanin sevinci, güzellik simgesi dedikleri kent bu mu?" diyorlar.
\par 16 Düsmanlarinin hepsi seninle alay etti, Islik çalip dis gicirdatarak, "Onu yuttuk" diyorlar, "Iste bekledigimiz gün, sonunda gördük onu."
\par 17 RAB düsündügünü yapti, Geçmiste söyledigi sözü yerine getirdi, Yikti, acimadi, Düsmani senin haline sevindirdi, Hasimlarini güçlü kildi.
\par 18 Halk Rab'be yürekten feryat ediyor. Ey Siyon kizinin surlari, Gece gündüz gözyasin sel gibi aksin! Dinlenme, gözüne uyku girmesin!
\par 19 Kalk, gece her nöbet basinda haykir, Rab'bin huzurunda yüregini su gibi dök! Her sokak basinda açliktan bayilan çocuklarinin basi için O'na ellerini aç.
\par 20 "Bak, ya RAB, gör! Kime böyle yaptin? Kadinlar çocuklarini, sevgili yavrularini mi yesin? Kâhinle peygamber Rab'bin Tapinagi'nda mi öldürülsün?
\par 21 Gençler, yaslilar sokaklarda, yerlerde yatiyor, Kiliçtan geçirildi erden kizlarimla gençlerim, Öfkelendigin gün öldürdün onlari, acimadan bogazladin.
\par 22 Bir bayram günü davet eder gibi Beni dehsete düsürenleri davet ettin her yandan. RAB'bin öfkelendigi gün kaçip kurtulan, Sag kalan olmadi. Sevgiyle büyüttügüm çocuklarimi Düsmanim yok etti."

\chapter{3}

\par 1 RAB'bin gazap degnegi altinda aci çeken adam benim.
\par 2 Beni güttü, Isikta degil karanlikta yürüttü.
\par 3 Evet, dönüp dönüp bütün gün bana elini kaldiriyor.
\par 4 Etimi, derimi yipratti, kemiklerimi kirdi.
\par 5 Beni kusatti, Aci ve zahmetle sardi çevremi.
\par 6 Çoktan ölmüs ölüler gibi Beni karanlikta yasatti.
\par 7 Çevreme duvar çekti, disari çikamiyorum, Zincirimi agirlastirdi.
\par 8 Feryat edip yardim isteyince de Duama set çekiyor.
\par 9 Yontma taslarla yollarimi kesti, Dolastirdi yollarimi.
\par 10 Benim için O pusuya yatmis bir ayi, Gizlenmis bir aslandir.
\par 11 Yollarimi saptirdi, paraladi, Mahvetti beni.
\par 12 Yayini gerdi, okunu savurmak için Beni nisangah olarak dikti.
\par 13 Oklarini böbreklerime sapladi.
\par 14 Halkimin önünde gülünç düstüm, Gün boyu alay konusu oldum türkülerine.
\par 15 Beni aciya doyurdu, Bana doyasiya pelinsuyu içirdi.
\par 16 Dislerimi çakil taslariyla kirdi, Kül içinde diz çöktürdü bana.
\par 17 Esenlik yüzü görmedi canim, Mutlulugu unuttum.
\par 18 Bu yüzden diyorum ki, "Dermanim tükendi, RAB'den umudum kesildi."
\par 19 Acimi, basiboslugumu, Pelinotuyla ödü animsa!
\par 20 Hâlâ onlari düsünmekte Ve sikilmaktayim.
\par 21 Ama sunu animsadikça umutlaniyorum:
\par 22 RAB'bin sevgisi hiç tükenmez, Merhameti asla son bulmaz;
\par 23 Her sabah tazelenir onlar, Sadakatin büyüktür.
\par 24 "Benim payima düsen RAB'dir" diyor canim, "Bu yüzden O'na umut bagliyorum."
\par 25 RAB kendisini bekleyenler, O'nu arayan canlar için iyidir.
\par 26 RAB'bin kurtarisini sessizce beklemek iyidir.
\par 27 Insan için boyundurugu gençken tasimak iyidir.
\par 28 RAB insana boyunduruk takinca, Insan tek basina oturup susmali;
\par 29 Umudunu kesmeden yere kapanmali,
\par 30 Kendisine vurana yanagini dönüp Utanca doymali;
\par 31 Çünkü Rab kimseyi sonsuza dek geri çevirmez.
\par 32 Dert verse de, Büyük sevgisinden ötürü yine merhamet eder;
\par 33 Çünkü isteyerek aci çektirmez, Insanlari üzmez.
\par 34 Ülkedeki bütün tutsaklari ayak altinda ezmeyi,
\par 35 Yüceler Yücesi'nin huzurunda insan hakkini saptirmayi,
\par 36 Davasinda insana haksizlik etmeyi Rab dogru görmez.
\par 37 Rab buyurmadikça kim bir sey söyler de yerine gelir?
\par 38 Iyilikler gibi felaketler de Yüceler Yücesi'nin agzindan çikmiyor mu?
\par 39 Insan, yasayan insan Niçin günahlarinin cezasindan yakinir?
\par 40 Davranislarimizi sinayip gözden geçirelim, Yine RAB'be dönelim.
\par 41 Ellerimizin yanisira yüreklerimizi de göklerdeki Tanri'ya açalim:
\par 42 "Biz karsi çikip baskaldirdik, Sen bagislamadin.
\par 43 Öfkeyle örtünüp bizi kovaladin, Acimadan öldürdün.
\par 44 Dualar sana erismesin diye Bulutlari örtündün.
\par 45 Uluslar arasinda bizi pislige, süprüntüye çevirdin.
\par 46 Düsmanlarimizin hepsi bizimle alay etti.
\par 47 Dehset ve çukur, kirgin ve yikim çikti önümüze."
\par 48 Kirilan halkim yüzünden Gözlerimden sel gibi yaslar akiyor.
\par 49 Durup dinmeden yas bosaniyor gözümden,
\par 50 RAB göklerden bakip görünceye dek.
\par 51 Kentimdeki kizlarin halini gördükçe Yüregim sizliyor.
\par 52 Bos yere bana düsman olanlar bir kus gibi avladilar beni.
\par 53 Beni sarnica atip öldürmek istediler, Üzerime tas attilar.
\par 54 Sular basimdan asti, "Tükendim" dedim.
\par 55 Sarnicin dibinden sana yakardim, ya RAB;
\par 56 Sesimi, "Ahima, çagrima kulagini kapama!" dedigimi duydun.
\par 57 Seni çagirinca yaklasip, "Korkma!" dedin.
\par 58 Davami sen savundun, ya Rab, Canimi kurtardin.
\par 59 Bana yapilan haksizligi gördün, ya RAB, Davami sen gör.
\par 60 Benden nasil öç aldiklarini, Bana nasil dolap çevirdiklerini gördün.
\par 61 Asagilamalarini, ya RAB, Çevirdikleri bütün dolaplari, Bana saldiranlarin dediklerini, Gün boyu söylendiklerini duydun.
\par 63 Oturup kalkislarina bak, Alay konusu oldum türkülerine.
\par 64 Yaptiklarinin karsiligini ver, ya RAB.
\par 65 Inat etmelerini sagla, Lanetin üzerlerinden eksilmesin.
\par 66 Göklerinin altindan öfkeyle kovala, yok et onlari, ya RAB.

\chapter{4}

\par 1 Altin nasil donuklasti, Saf altin nasil degisti! Kutsal taslar sokak baslarina dagilmis.
\par 2 Degerleri saf altinla ölçülen Siyon çocuklari Nasil çömlekçi isi, toprak testi yerine sayilir oldu!
\par 3 Çakallar bile meme verip yavrularini emzirir, Ama halkim çöldeki devekuslari kadar acimasiz oldu.
\par 4 Susuzluktan emzikteki bebeklerin dili damagina yapisiyor, Çocuklar ekmek istiyor, veren yok.
\par 5 Onlar ki, yemegin en iyisini yerlerdi, Sokaklarda perisan oldular; Onlar ki, al giysiler içinde büyüdüler, Çöp yiginlarini kapisir oldular.
\par 6 Halkimin suçu el degmeden, bir anda yikilan Sodom'un günahindan daha büyüktür.
\par 7 Beyleri kardan temiz, sütten aktilar, Bedence mercandan kizil, laciverttasi kadar biçimliydiler.
\par 8 Simdiyse görünüsleri kömürden kara, Sokaklarda taninmaz oldular. Bir deri bir kemige döndüler, odun gibi kurudular.
\par 9 Kiliçla öldürülenler kitliktan ölenlerden mutludur, Çünkü kitliktan ölenler tarla ürününün yoklugundan yipranarak erimekteler.
\par 10 Merhametli kadinlar çocuklarini elleriyle pisirdiler, Halkim kirilirken yiyecek oldu bu kendilerine.
\par 11 RAB öfkesini bosaltti, kizgin öfkesini döktü, Temellerini yiyip bitiren atesi Siyon'un içinde tutusturdu.
\par 12 Dünyadaki krallarin ve insanlarin hiçbiri Yerusalim kapilarindan hasimlarin, düsmanlarin girecegine inanmazdi.
\par 13 Peygamberlerinin günahi, kâhinlerinin suçu yüzündendi bu, Çünkü onlar kentin ortasinda dogrularin kanini döktüler.
\par 14 Sokaklarda körler gibi dolasiyorlar, Kanla kirlendikleri için kimse giysilerine dokunamiyor.
\par 15 "Çekilin! Kirliler!" diye bagirdilar onlara, "Çekilin! Çekilin! Dokunmayin!" Kaçip basibos dolastiklarinda, Öteki uluslar, "Artik burada kalmasinlar" dediler.
\par 16 RAB kendisi dagitti onlari, Artik yüzlerine bakmayacak. Kâhinleri saymadilar, yaslilara acimadilar.
\par 17 Bos yere yardim beklemekten gözlerimizin feri sönüyor, Gözetleme kulesinde bizi kurtaramayacak bir ulusu bekledikçe bekledik.
\par 18 Izlerimizi sürüyorlar, Sokaklarimizda gezemez olduk. Sonumuz yaklasti, günlerimiz tükendi, Çünkü sonumuz geldi.
\par 19 Bizi kovalayanlar gökteki kartallardan çevikti, Daglarin üstünde kovaladilar bizi, Çölde bize pusu kurdular.
\par 20 Yasam solugumuz, RAB'bin meshettigi* kral onlarin çukurunda yakalandi; Hani onun için, "Uluslarin arasinda onun gölgesinde yasayacagiz" dedigimiz.
\par 21 Ûs ülkesinde yasayan Edom kizi, sevin, cos, Ancak kâse* sana da gelecek, sarhos olup soyunacaksin.
\par 22 Ey Siyon kizi*, suçunun cezasi sona erdi, RAB bir daha seni sürgüne göndermeyecek. Ama, ey Edom kizi, suçun yüzünden seni cezalandirip günahlarini ortaya çikaracak.

\chapter{5}

\par 1 Animsa, ya RAB, basimiza geleni, Bak da utancimizi gör.
\par 2 Mülkümüz yabancilara geçti, Evlerimiz ellere.
\par 3 Öksüz kaldik, babasiz, Annelerimiz dul kadinlara döndü.
\par 4 Suyumuzu parayla içtik, Odunumuzu parayla almak zorunda kaldik.
\par 5 Bizi kovalayanlar ensemizde, Yorgun düstük, rahatimiz yok.
\par 6 Ekmek için Misir'a, Asur'a el açtik.
\par 7 Atalarimiz günah isledi, Ama artik onlar yok; Suçlarinin cezasini biz yüklendik.
\par 8 Köleler üstümüzde saltanat sürüyor, Bizi ellerinden kurtaracak kimse yok.
\par 9 Çöldeki kiliçli haydutlar yüzünden Ekmegimizi canimiz pahasina kazaniyoruz.
\par 10 Kitligin yakici sicagindan Derimiz firin gibi kizardi.
\par 11 Siyon'da kadinlarin, Yahuda kentlerinde erden kizlarin irzina geçtiler.
\par 12 Önderler ellerinden asildi, Yaslilar saygi görmedi.
\par 13 Degirmen tasini gençler çevirdi, Çocuklar odun yükü altinda tökezledi.
\par 14 Yaslilar kent kapisinda oturmaz oldu, Gençler saz çalmaz oldu.
\par 15 Yüregimizin sevinci durdu, Oyunumuz yasa döndü.
\par 16 Taç düstü basimizdan, Vay basimiza! Çünkü günah isledik.
\par 17 Bu yüzden yüregimiz baygin, Bunlardan ötürü gözlerimiz karardi.
\par 18 Viran olan Siyon Dagi'nin üstünde Çakallar geziyor!
\par 19 Ama sen, sonsuza dek tahtinda oturursun, ya RAB, Egemenligin kusaklar boyu sürer.
\par 20 Niçin bizi hep unutuyorsun, Neden bizi uzun süre terk ediyorsun?
\par 21 Bizi kendine döndür, ya RAB, döneriz, Eski günlerimizi geri ver.
\par 22 Bizi büsbütün attiysan, Bize çok öfkelenmis olmalisin.


\end{document}