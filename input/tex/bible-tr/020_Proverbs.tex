\begin{document}

\title{Özdeyişler}


\chapter{1}

\par 1 Davut oglu Israil Krali Süleyman'in özdeyisleri:
\par 2 Bu özdeyisler, bilgelige ve terbiyeye ulasmak, Akillica sözleri anlamak,
\par 3 Basariya götüren terbiyeyi edinip Dogru, hakli ve adil olani yapmak,
\par 4 Saf kisiyi ihtiyatli, Genç adami bilgili ve saggörülü kilmak içindir.
\par 5 Özdeyislerle benzetmeleri, Bilgelerin sözleriyle bilmecelerini anlamak için Bilge kisi dinlesin ve kavrayisini artirsin, Akilli kisi yasam hüneri kazansin.
\par 7 RAB korkusudur bilginin temeli. Ahmaklarsa bilgeligi ve terbiyeyi küçümser.
\par 8 Oglum, babanin uyarilarina kulak ver, Annenin ögrettiklerinden ayrilma.
\par 9 Çünkü bunlar basin için sevimli bir çelenk, Boynun için gerdanlik olacaktir.
\par 10 Oglum, seni ayartmaya çalisan günahkârlara teslim olma.
\par 11 Söyle diyebilirler: "Bizimle gel, Adam öldürmek için pusuya yatalim, Zevk ugruna masum kisileri tuzaga düsürelim.
\par 12 Onlari ölüler diyari gibi diri diri, Ölüm çukuruna inenler gibi Bütünüyle yutalim.
\par 13 Bir sürü degerli mal ele geçirir, Evlerimizi ganimetle doldururuz.
\par 14 Gel, sen de bize katil, Tek bir kesemiz olacak."
\par 15 Oglum, böyleleriyle gitme, Onlarin tuttugu yoldan uzak dur.
\par 16 Çünkü ayaklari kötülüge kosar, Çekinmeden kan dökerler.
\par 17 Kuslarin gözü önünde ag sermek bosunadir.
\par 18 Baskasina pusu kuran kendi kurdugu pusuya düser. Yalniz kendi canidir tuzaga düsürdügü.
\par 19 Haksiz kazanca düskün olanlarin sonu böyledir. Bu düskünlük onlari canlarindan eder.
\par 20 Bilgelik disarida yüksek sesle haykiriyor, Meydanlarda sesleniyor.
\par 21 Kalabalik sokak baslarinda bagiriyor, Kentin giris kapilarinda sözlerini duyuruyor:
\par 22 "Ey budalalar, budalaligi ne zamana dek seveceksiniz? Alaycilar ne zamana dek alay etmekten zevk alacak? Akilsizlar ne zamana dek bilgiden nefret edecek?
\par 23 Uyardigimda yola gelin, o zaman size yüregimi açar, Sözlerimi anlamaniza yardim ederim.
\par 24 Ama sizi çagirdigim zaman beni reddettiniz. Elimi uzattim, umursayan olmadi.
\par 25 Duymazliktan geldiniz bütün ögütlerimi, Uyarilarimi duymak istemediniz.
\par 26 Bu yüzden ben de felaketinize sevinecegim. Belaya ugradiginizda, Bela üzerinize bir firtina gibi geldiginde, Bir kasirga gibi geldiginde felaketiniz, Sikintiya, kaygiya düstügünüzde, Sizinle alay edecegim.
\par 28 O zaman beni çagiracaksiniz, Ama yanitlamayacagim. Var gücünüzle arayacaksiniz beni, Ama bulamayacaksiniz.
\par 29 Çünkü bilgiden nefret ettiniz. RAB'den korkmayi reddettiniz.
\par 30 Ögütlerimi istemediniz, Uyarilarimin tümünü küçümsediniz.
\par 31 Bu nedenle tuttugunuz yolun meyvesini yiyeceksiniz, Kendi düzenbazliginiza doyacaksiniz.
\par 32 Bön adamlar dönekliklerinin kurbani olacak. Akilsizlar kaygisizliklarinin içinde yok olup gidecek.
\par 33 Ama beni dinleyen güvenlik içinde yasayacak, Kötülükten korkmayacak, huzur bulacak."

\chapter{2}

\par 1 Oglum, bilgelige kulak verip Yürekten akla yönelerek Sözlerimi kabul eder, Buyruklarimi aklinda tutarsan,
\par 3 Evet, akli çagirir, Ona gönülden seslenirsen,
\par 4 Gümüs ararcasina onu ararsan, Onu ararsan define arar gibi,
\par 5 RAB korkusunu anlar Ve Tanri'yi yakindan tanirsin.
\par 6 Çünkü bilgeligin kaynagi RAB'dir. O'nun agzindan bilgi ve anlayis çikar.
\par 7 Dogru kisileri basariya ulastirir, Kalkanidir dürüst yasayanlarin.
\par 8 Adil olanlarin adimlarini korur, Sadik kullarinin yolunu gözetir.
\par 9 O zaman anlarsin her iyi yolu, Neyin dogru, hakli ve adil oldugunu.
\par 10 Çünkü yüregin bilgelikle dolacak, Zevk alacaksin bilgiden.
\par 11 Saggörü sana bekçilik edecek Ve akil seni koruyacak.
\par 12 Bunlar seni kötü yoldan, Ahlaksizin sözlerinden kurtaracak.
\par 13 Onlar ki karanlik yollarda yürümek için Dogru yoldan ayrilirlar.
\par 14 Kötülük yapmaktan hoslanir, Zevk alirlar kötülügün asirisindan.
\par 15 Yollari dolambaçli, Yasayislari çarpiktir.
\par 16 Bilgelik, gençken evlendigi esini terk eden, Tanri'nin önünde içtigi andi unutan ahlaksiz kadindan, Sözleriyle yaltaklanan Vefasiz kadindan seni kurtaracak.
\par 18 O kadinin evi insani ölüme, Yollari ölülere götürür.
\par 19 Ona gidenlerden hiçbiri geri dönmez, Yasam yollarina erismez.
\par 20 Bu nedenle sen iyilerin yolunda yürü, Dogrularin izinden git.
\par 21 Çünkü ülkede yasayacak olan dogrulardir, Dürüst kisilerdir orada kalacak olan.
\par 22 Kötüler ülkeden sürülecek, Hainler sökülüp atilacak.

\chapter{3}

\par 1 Oglum, unutma ögrettiklerimi, Aklinda tut buyruklarimi.
\par 2 Çünkü bunlar ömrünü uzatacak, Yasam yillarini, esenligini artiracaktir.
\par 3 Sevgiyi, sadakati hiç yanindan ayirma, Bagla onlari boynuna, Yaz yüreginin levhasina.
\par 4 Böylece Tanri'nin ve insanlarin gözünde Begeni ve sayginlik kazanacaksin.
\par 5 RAB'be güven bütün yüreginle, Kendi aklina bel baglama.
\par 6 Yaptigin her iste RAB'bi an, O senin yolunu düze çikarir.
\par 7 Kendini bilge biri olarak görme, RAB'den kork, kötülükten uzak dur.
\par 8 Böylece bedenin saglik Ve ferahlik bulur.
\par 9 Servetinle ve ürününün turfandasiyla RAB'bi onurlandir.
\par 10 O zaman ambarlarin tika basa dolar, Teknelerin yeni sarapla dolup tasar.
\par 11 Oglum, RAB'bin terbiye edisini hafife alma, O'nun azarlamasindan usanma.
\par 12 Çünkü RAB, oglundan hosnut bir baba gibi, Sevdigini azarlar.
\par 13 Bilgelige erisene, Akli bulana ne mutlu!
\par 14 Gümüs kazanmaktansa onu kazanmak daha iyidir. Onun yarari altindan daha çoktur.
\par 15 Daha degerlidir mücevherden, Dileyecegin hiçbir sey onunla kiyaslanamaz.
\par 16 Sag elinde uzun ömür, Sol elinde zenginlik ve onur vardir.
\par 17 Yollari sevinç yollaridir, Evet, bütün yollari esenlige çikarir.
\par 18 Bilgelik yasam agacidir ona sarilanlara, Ne mutlu ona simsiki tutunanlara!
\par 19 RAB dünyanin temelini bilgelikle atti, Gökleri akillica yerlestirdi.
\par 20 Bilgisiyle enginler yarildi, Bulutlar suyunu verdi.
\par 21 Oglum, saglam ögüde, saggörüye tutun. Sakin gözünü ayirma onlardan.
\par 22 Onlar sana yasam verecek Ve boynuna güzel bir süs olacak.
\par 23 O zaman güvenlik içinde yol alirsin, Sendelemeden.
\par 24 Korkusuzca yatar, Tatli tatli uyursun.
\par 25 Beklenmedik felaketten, Ya da kötülerin ugradigi yikimdan korkma.
\par 26 Çünkü senin güvencen RAB'dir, Tuzaga düsmekten seni O koruyacaktir.
\par 27 Elinden geldikçe, Iyilige hakki olanlardan iyiligi esirgeme.
\par 28 Elinde varken komsuna, "Bugün git, yarin gel, o zaman veririm" deme.
\par 29 Sana güvenerek yaninda yasayan komsuna Kötülük tasarlama.
\par 30 Sana kötülük etmemis biriyle Yok yere çekisme.
\par 31 Zorba kisiye imrenme, Onun yollarindan hiçbirini seçme.
\par 32 Çünkü RAB sapkinlardan tiksinir, Ama dogrularin candan dostudur.
\par 33 RAB kötülerin evini lanetler, Dogrularin oturdugu yeriyse kutsar.
\par 34 RAB alaycilarla alay eder, Ama alçakgönüllülere lütfeder.
\par 35 Bilge kisiler onuru miras alacak, Akilsizlara yalniz utanç kalacak.

\chapter{4}

\par 1 Çocuklarim, babanizin uyarilarina kulak verin. Dikkat edin ki anlayisli olasiniz.
\par 2 Çünkü size iyi ders veriyorum, Ayrilmayin ögrettigimden.
\par 3 Ben bir çocukken babamin evinde, Annemin körpecik tek yavrusuyken,
\par 4 Babam bana sunu ögretti: "Söylediklerime yürekten saril, Buyruklarimi yerine getir ki yasayasin.
\par 5 Bilgeligi ve akli sahiplen, Söylediklerimi unutma, onlardan sapma.
\par 6 Bilgelikten ayrilma, o seni korur. Sev onu, seni gözetir.
\par 7 Bilgelige ilk adim onu sahiplenmektir. Bütün servetine mal olsa da akla sahip çik.
\par 8 Onu el üstünde tut, o da seni yüceltecek, Ona sarilirsan seni onurlandiracak.
\par 9 Basina zarif bir çelenk, Görkemli bir taç giydirecektir."
\par 10 Dinle oglum, sözlerimi benimse ki, Uzasin ömrün.
\par 11 Seni bilgelik yolunda egitir, Dogru yollara yöneltirim.
\par 12 Ayaklarin takilmadan yürür, Sürçmeden kosarsin.
\par 13 Aldigin terbiyeye saril, birakma, Onu uygula, çünkü odur yasamin.
\par 14 Kötülerin yoluna ayak basma, Yürüme alçaklarin yolunda,
\par 15 O yoldan sakin, yakinindan bile geçme, Yönünü degistirip geç.
\par 16 Çünkü kötülük etmedikçe uyuyamaz onlar, Uykulari kaçar saptirmadikça birilerini.
\par 17 Yedikleri ekmek kötülük, Içtikleri sarap zorbalik ürünüdür.
\par 18 Oysa dogrularin yolu safak isigi gibidir, Giderek ögle günesinin parlakligina erisir.
\par 19 Kötülerin yoluysa zifiri karanlik gibidir, Neden tökezlediklerini bilmezler.
\par 20 Oglum, sözlerime dikkat et, Dediklerime kulak ver.
\par 21 Aklindan çikmasin bunlar, Onlari yüreginde sakla.
\par 22 Çünkü onlari bulan için yasam, Bedeni için sifadir bunlar.
\par 23 Her seyden önce de yüregini koru, Çünkü yasam ondan kaynaklanir.
\par 24 Yalan çikmasin agzindan, Uzak tut dudaklarini sapik sözlerden.
\par 25 Gözlerin hep ileriye baksin, Dosdogru önüne!
\par 26 Gidecegin yolu düzle, O zaman bütün islerin saglam olur.
\par 27 Sapma saga sola, Ayagini kötülükten uzak tut.

\chapter{5}

\par 1 Oglum, bilgeligime dikkat et, Akillica sözlerime kulak ver.
\par 2 Böylelikle her zaman saggörülü olur, Dudaklarinla bilgiyi korursun.
\par 3 Zina eden kadinin bal damlar dudaklarindan, Agzi daha yumusaktir zeytinyagindan.
\par 4 Ama sonu pelinotu kadar aci, Iki agizli kiliç kadar keskindir.
\par 5 Ayaklari ölüme gider, Adimlari ölüler diyarina ulasir.
\par 6 Yasama giden yolu hiç düsünmez, Yollari dolasiktir, ama farkinda degil.
\par 7 Oglum, simdi beni dinle, Agzimdan çikan sözlerden ayrilma.
\par 8 Öyle kadinlardan uzak dur, Yaklasma evinin kapisina.
\par 9 Yoksa onurunu baskalarina, Yillarini bir gaddara kaptirirsin.
\par 10 Varini yogunu yer bitirir yabancilar, Emegin baska birinin evini bayindir kilar.
\par 11 Ah çekip inlersin ömrünün son günlerinde, Etinle bedenin tükendiginde.
\par 12 "Egitilmekten neden bu kadar nefret ettim, Yüregim uyarilari neden önemsemedi?" dersin.
\par 13 "Ögretmenlerimin sözünü dinlemedim, Beni egitenlere kulak vermedim.
\par 14 Halkin ve toplulugun arasinda Tam bir yikimin esigine gelmisim."
\par 15 Suyu kendi sarnicindan, Kendi kuyunun kaynagindan iç.
\par 16 Pinarlarin sokaklari, Akarsularin meydanlari mi sulamali?
\par 17 Yalniz senin olsun onlar, Paylasma yabancilarla.
\par 18 Çesmen bereketli olsun Ve gençken evlendigin karinla mutlu ol.
\par 19 Sevimli bir geyik, zarif bir ceylan gibi, Hep seni doyursun memeleri. Askiyla sürekli cos.
\par 20 Oglum, neden ahlaksiz bir kadinla cosasin, Neden baska birinin karisini koynuna alasin?
\par 21 RAB insanin tuttugu yolu gözler, Attigi her adimi denetler.
\par 22 Kötü kisiyi kendi suçlari ele verecek, Günahinin kemendi kiskivrak baglayacak onu.
\par 23 Asiri ahmakligi onu yoldan çikaracak, Terbiyeyi umursamadigi için ölecek.

\chapter{6}

\par 1 Oglum, eger birine kefil oldunsa, Onun borcunu yüklendinse,
\par 2 Düstünse tuzaga kendi sözlerinle, Agzinin sözleriyle yakalandinsa,
\par 3 O kisinin eline düstün demektir. Oglum, sunu yap ve kendini kurtar: Git, yere kapan onun önünde, Ona yalvar yakar.
\par 4 Gözlerine uyku girmesin, Agirlasmasin göz kapaklarin.
\par 5 Avcinin elinden ceylan gibi, Kusbazin elinden kus gibi kurtar kendini.
\par 6 Ey tembel kisi, git, karincalara bak, Onlarin yasamindan bilgelik ögren.
\par 7 Baskanlari, önderleri ya da yöneticileri olmadigi halde,
\par 8 Yazin erzaklarini biriktirirler, Yiyeceklerini toplarlar biçim mevsiminde.
\par 9 Ne zamana dek yatacaksin, ey tembel kisi? Ne zaman kalkacaksin uykundan?
\par 10 "Biraz kestireyim, biraz uyuklayayim, Ellerimi kavusturup söyle bir uyuyayim" demeye kalmadan,
\par 11 Yokluk bir haydut gibi, Yoksulluk bir akinci gibi gelir üzerine.
\par 12 Agzinda yalanla dolasan kisi, Soysuz ve fesatçidir.
\par 13 Göz kirpar, bir sürü ayak oyunu, El kol hareketleri yapar,
\par 14 Ahlaksiz yüreginde kötülük tasarlar, Çekismeler yaratir durmadan.
\par 15 Bu yüzden ansizin yikima ugrayacak, Birdenbire çaresizce yok olacak.
\par 16 RAB'bin nefret ettigi alti sey, Igrendigi yedi sey vardir:
\par 17 Gururlu gözler, Yalanci dil, Suçsuz kani döken eller,
\par 18 Düzenbaz yürek, Kötülüge segirten ayaklar,
\par 19 Yalan soluyan yalanci tanik Ve kardesler arasinda çekisme yaratan kisi.
\par 20 Oglum, babanin buyruklarina uy, Annenin ögrettiklerinden ayrilma.
\par 21 Bunlar sürekli yüreginin bagi olsun, Tak onlari boynuna.
\par 22 Yolunda sana rehber olacak, Seni koruyacaklar yattigin zaman; Söylesecekler seninle uyandiginda.
\par 23 Bu buyruklar sana çira, Ögretilenler isiktir. Egitici uyarilar yasam yolunu gösterir.
\par 24 Seni kötü kadindan, Baska birinin karisinin yaltaklanan dilinden Koruyacak olan bunlardir.
\par 25 Böyle kadinlarin güzelligi seni ayartmasin, Bakislari seni tutsak etmesin.
\par 26 Çünkü fahise yüzünden insan bir lokma ekmege muhtaç kalir, Baskasinin karisiyla yatmak da kisinin canina mal olur.
\par 27 Insan koynuna ates alir da, Giysisi yanmaz mi?
\par 28 Korlar üzerinde yürür de, Ayaklari kavrulmaz mi?
\par 29 Baskasinin karisiyla yatan adamin durumu budur. Böyle bir iliskiye giren cezasiz kalmaz.
\par 30 Aç hirsiz karnini doyurmak için çaliyorsa, Kimse onu hor görmez.
\par 31 Ama yakalanirsa, çaldiginin yedi katini ödemek zorunda; Varini yogunu vermek anlamina gelse bile.
\par 32 Zina eden adam sagduyudan yoksundur. Yaptiklariyla kendini yok eder.
\par 33 Payina düsen dayak ve onursuzluktur, Asla kurtulamaz utançtan.
\par 34 Çünkü kiskançlik kocanin öfkesini azdirir, Öç alirken acimasiz olur.
\par 35 Hiçbir fidye kabul etmez, Gönlünü alamazsin armaganlarin çokluguyla.

\chapter{7}

\par 1 Oglum, sözlerimi yerine getir, Aklinda tut buyruklarimi.
\par 2 Buyruklarimi yerine getir ki, yasayasin. Ögrettiklerimi gözünün bebegi gibi koru.
\par 3 Onlari yüzük gibi parmaklarina geçir, Yüreginin levhasina yaz.
\par 4 Bilgelige, "Sen kizkardesimsin", Akla, "Akrabamsin" de.
\par 5 Zina eden kadindan, Yaltaklanan ahlaksiz kadindan seni koruyacak olan bunlardir.
\par 6 Evimin penceresinden, Kafesin ardindan disariyi seyrederken,
\par 7 Bir sürü toy gencin arasinda, Sagduyudan yoksun bir delikanli çarpti gözüme.
\par 8 Aksamüzeri, alaca karanlikta, Aksam karanligi çökerken, O kadinin oturdugu sokaga saptigini, Onun evine yöneldigini gördüm.
\par 10 Derken kadin onu karsiladi, Fahise kiligiyla sinsice.
\par 11 Yaygaraci, dik basli biriydi kadin. Bir an bile durmaz evde.
\par 12 Kâh sokakta, kâh meydanlardadir. Sokak baslarinda pusuya yatar.
\par 13 Delikanliyi tutup öptü, Yüzü kizarmadan ona söyle dedi:
\par 14 "Esenlik kurbanlarimi kesmek zorundaydim, Adak sözümü bugün yerine getirdim.
\par 15 Bunun için seni karsilamaya, seni aramaya çiktim, Iste buldum seni!
\par 16 Dösegime Misir ipliginden dokunmus Renkli örtüler serdim.
\par 17 Yatagima mür*, öd Ve tarçin serptim.
\par 18 Haydi gel, sabaha dek doya doya seviselim, Asktan zevk alalim.
\par 19 Kocam evde degil, Uzun bir yolculuga çikti.
\par 20 Yanina para torbasini aldi, Dolunaydan önce eve dönmeyecek."
\par 21 Onu bir sürü çekici sözlerle bastan çikardi, Tatli diliyle pesinden sürükledi.
\par 22 Kesimevine götürülen öküz gibi Hemen izledi onu delikanli; Tuzaga düsen geyik gibi,
\par 23 Cigerini bir ok delene kadar; Kapana kosan bir kus gibi, Bunun yasamina mal olacagini bilmeden.
\par 24 Çocuklarim, simdi dinleyin beni, Kulak verin söylediklerime,
\par 25 Sakin o kadina gönül vermeyin, Onun yolundan gitmeyin.
\par 26 Yere serdigi bir sürü kurbani var, Öldürdügü kisilerin sayisi pek çok.
\par 27 Ölüler diyarina giden yoldur onun evi, Ölüm odalarina götürür.

\chapter{8}

\par 1 Bilgelik çagiriyor, Akil sesini yükseltiyor.
\par 2 Yol kenarindaki tepelerin basinda, Yollarin birlestigi yerde duruyor o.
\par 3 Kentin girisinde, kapilarin yaninda, Sesini yükseltiyor:
\par 4 "Ey insanlar, size sesleniyorum, Çagrim insan soyunadir!
\par 5 Ey bön kisiler, ihtiyatli olmayi ögrenin; Sagduyulu olmayi ögrenin, ey akilsizlar!
\par 6 Söyledigim yetkin sözleri dinleyin, Agzimi dogrulari söylemek için açarim.
\par 7 Agzim gerçegi duyurur, Çünkü dudaklarim kötülükten igrenir.
\par 8 Agzimdan çikan her söz dogrudur, Yoktur egri ya da sapik olani.
\par 9 Apaçiktir hepsi anlayana, Bilgiye erisen, dogrulugunu bilir onlarin.
\par 10 Gümüs yerine terbiyeyi, Saf altin yerine bilgiyi edinin.
\par 11 Çünkü bilgelik mücevherden degerlidir, Diledigin hiçbir sey onunla kiyaslanamaz.
\par 12 Ben bilgelik olarak ihtiyati kendime konut edindim. Bilgi ve saggörü bendedir.
\par 13 RAB'den korkmak kötülükten nefret etmek demektir. Kibirden, küstahliktan, Kötü yoldan, sapik agizdan nefret ederim.
\par 14 Ögüt ve saglam karar bana özgüdür. Akil ve güç kaynagi benim.
\par 15 Krallar sayemde egemenlik sürer, Hükümdarlar adil kurallar koyar.
\par 16 Önderler, adaletle yöneten soylular Sayemde yönetirler.
\par 17 Beni sevenleri ben de severim, Gayretle arayan beni bulur.
\par 18 Zenginlik ve onur, Kalici degerler ve bolluk bendedir.
\par 19 Meyvem altindan, saf altindan, Ürünüm seçme gümüsten daha iyidir.
\par 20 Dogruluk yolunda, Adaletin izinden yürürüm.
\par 21 Böylelikle, beni sevenleri servet sahibi yapar, Hazinelerini doldururum.
\par 22 RAB yaratma isine basladiginda Ilk beni yaratti,
\par 23 Dünya var olmadan önce, Ta baslangiçta, öncesizlikte yerimi aldim.
\par 24 Enginler yokken, Sulari bol pinarlar yokken dogdum ben.
\par 25 Daglar daha olusmadan, Tepeler belirmeden, RAB dünyayi, kirlari Ve dünyadaki topragin zerresini yaratmadan dogdum.
\par 27 RAB gökleri yerine koydugunda oradaydim, Engin denizleri ufukla çevirdiginde,
\par 28 Bulutlari olusturdugunda, Denizin kaynaklarini güçlendirdiginde,
\par 29 Sular buyrugundan öte geçmesinler diye Denize sinir çizdiginde, Dünyanin temellerini pekistirdiginde,
\par 30 Bas mimar olarak O'nun yanindaydim. Gün be gün sevinçle dolup tastim, Huzurunda hep costum.
\par 31 O'nun dünyasi mutlulugum, Insanlari sevincimdi.
\par 32 Çocuklarim, simdi beni dinleyin: Yolumu izleyenlere ne mutlu!
\par 33 Uyarilarimi dinleyin ve bilge kisiler olun, Görmezlikten gelmeyin onlari.
\par 34 Beni dinleyen, Her gün kapimi gözleyen, Kapimin esiginden ayrilmayan kisiye ne mutlu!
\par 35 Çünkü beni bulan yasam bulur Ve RAB'bin begenisini kazanir.
\par 36 Beni gözardi edense kendine zarar verir, Benden nefret eden, ölümü seviyor demektir."

\chapter{9}

\par 1 Bilgelik kendi evini yapti, Yedi diregini yonttu.
\par 2 Hayvanlarini kesti, Sarabini hazirlayip sofrasini kurdu.
\par 3 Kentin en yüksek noktalarina gönderdigi Hizmetçileri araciligiyla herkesi çagiriyor:
\par 4 "Kim safsa buraya gelsin" diyor. Sagduyudan yoksun olanlara da, "Gelin, yiyeceklerimi yiyin, Hazirladigim saraptan için" diyor.
\par 6 "Safligi birakin da yasayin, Aklin yolunu izleyin.
\par 7 "Alayciyi paylayan asagilanmayi hak eder, Kötü kisiyi azarlayan hakarete ugrar.
\par 8 Alayciyi azarlama, yoksa senden nefret eder. Bilge kisiyi azarlarsan, seni sever.
\par 9 Bilge kisiyi egitirsen Daha bilge olur, Dogru kisiye ögretirsen bilgisini artirir.
\par 10 RAB korkusudur bilgeligin temeli. Akil Kutsal Olan'i tanimaktir.
\par 11 Benim sayemde günlerin çogalacak, Ömrüne yillar katilacak.
\par 12 Bilgeysen, bilgeliginin yarari sanadir, Alayci olursan acisini yalniz sen çekersin."
\par 13 Akilsiz kadin yaygaraci Ve saftir, hiçbir sey bilmez.
\par 14 Evinin kapisinda, Kentin en yüksek yerinde bir iskemleye oturur; Yoldan geçenleri, Kendi yollarindan gidenleri çagirmak için,
\par 16 "Kim safsa buraya gelsin" der. Sagduyudan yoksun olanlara da,
\par 17 "Çalinti su tatli, Gizlice yenen yemek lezzetlidir" der.
\par 18 Ne var ki, evine girenler ölüme gittiklerini, Ona konuk olanlar Ölüler diyarinin dibine indiklerini bilmezler.

\chapter{10}

\par 1 Süleyman'in özdeyisleri: Bilge çocuk babasini sevindirir, Akilsiz çocuk annesini üzer.
\par 2 Haksizca kazanilan servetin yarari yoktur, Ama dogruluk ölümden kurtarir.
\par 3 RAB dogru kisiyi aç komaz, Ama kötülerin istegini bosa çikarir.
\par 4 Tembel eller insani yoksullastirir, Çaliskan el zengin eder.
\par 5 Akli basinda evlat ürünü yazin toplar, Hasatta uyuyansa ailesinin yüzkarasidir.
\par 6 Bereket dogru kisinin basina yagar, Kötülerse zorbaliklarini sözle gizler.
\par 7 Dogrular övgüyle, Kötüler nefretle anilir.
\par 8 Bilge kisi buyruklari kabul eder, Çenesi düsük ahmaksa yikima ugrar.
\par 9 Dürüst kisi güvenlik içinde yasar, Ama hileli yoldan giden açiga vurulacaktir.
\par 10 Sinsice göz kirpan, acilara neden olur. Çenesi düsük ahmak da yikima ugrar.
\par 11 Dogru kisinin agzi yasam pinaridir, Kötülerse zorbaliklarini sözle gizlerler.
\par 12 Nefret çekismeyi azdirir, Sevgi her suçu bagislar.
\par 13 Akilli kisinin dudaklarindan bilgelik akar, Ama sagduyudan yoksun olan sirtina kötek yer.
\par 14 Bilge kisi bilgi biriktirir, Ahmagin agziysa onu yikima yaklastirir.
\par 15 Zenginin serveti onun kalesidir, Fakirin yoksullugu ise onu yikima götürür.
\par 16 Dogru kisinin ücreti yasamdir, Kötünün geliriyse kendisine cezadir.
\par 17 Terbiyeye kulak veren yasam yolunu bulur. Uyarilari reddedense baskalarini yoldan saptirir.
\par 18 Nefretini gizleyen kisinin dudaklari yalancidir. Iftira yayan akilsizdir.
\par 19 Çok konusanin günahi eksik olmaz, Sagduyulu kisiyse dilini tutar.
\par 20 Dogru kisinin dili saf gümüs gibidir, Kötünün niyetleriyse degersizdir.
\par 21 Dogru kisinin sözleri birçoklarini besler, Ahmaklarsa sagduyu yoksunlugundan ölür.
\par 22 RAB'bin bereketidir kisiyi zengin eden, RAB buna dert katmaz.
\par 23 Kötülük akilsizlar için eglence gibidir. Akli basinda olanlar içinse bilgelik ayni seydir.
\par 24 Kötü kisinin korktugu basina gelir, Dogru kisiyse dilegine erisir.
\par 25 Kasirga gelince kötü kisiyi silip götürür; Ama dogru kisi sonsuza dek ayakta kalir.
\par 26 Disler için sirke, Gözler için duman neyse, Tembel ulak da kendisini gönderen için öyledir.
\par 27 RAB korkusu ömrü uzatir, Kötülerin yillariysa kisadir.
\par 28 Dogrunun umudu onu sevindirir, Kötünün beklentileriyse bosa çikar.
\par 29 RAB'bin yolu dürüst için siginak, Fesatçi içinse yikimdir.
\par 30 Dogru kisi hiçbir zaman sarsilmaz, Ama kötüler ülkede kalamaz.
\par 31 Dogru kisinin agzi bilgelik üretir, Sapik dilse kesilir.
\par 32 Dogru kisinin dudaklari söylenecek sözü bilir, Kötünün agzindansa sapik sözler çikar.

\chapter{11}

\par 1 RAB hileli teraziden igrenir, Hilesiz tartidansa hosnut kalir.
\par 2 Küstahligin ardindan utanç gelir, Ama bilgelik alçakgönüllülerdedir.
\par 3 Erdemlinin dürüstlügü ona yol gösterir, Hainin yalanciligiysa yikima götürür.
\par 4 Gazap günü servet ise yaramaz, Oysa dogruluk ölümden kurtarir.
\par 5 Dürüst insanin dogrulugu onun yolunu düzler, Kötü kisiyse kötülügü yüzünden yikilip düser.
\par 6 Erdemlinin dogrulugu onu kurtarir, Ama haini kendi hirsi ele verir.
\par 7 Kötü kisi öldügünde umutlari yok olur, Güvendigi güç de biter.
\par 8 Dogru kisi sikintidan kurtulur, Onun yerine sikintiyi kötü kisi çeker.
\par 9 Tanrisiz kisi baskalarini agziyla yikima götürür, Oysa dogrular bilgi sayesinde kurtulur.
\par 10 Dogrularin basarisina kent bayram eder, Kötülerin ölümüne sevinç çigliklari atilir.
\par 11 Dürüstlerin kutsamasiyla kent gelisir, Ama kötülerin agzi kenti yerle bir eder.
\par 12 Baskasini küçük gören sagduyudan yoksundur, Akilli kisiyse dilini tutar.
\par 13 Dedikoducu sir saklayamaz, Oysa güvenilir insan sirdas olur.
\par 14 Yol göstereni olmayan ulus düser, Danismani bol olan zafere gider.
\par 15 Yabanciya kefil olan mutlaka zarar görür, Kefaletten kaçinan güvenlik içinde yasar.
\par 16 Sevecen kadin onur, Zorbalarsa yalnizca servet kazanir.
\par 17 Iyilikseverin yarari kendinedir, Gaddarsa kendi basina bela getirir.
\par 18 Kötü kisinin kazanci aldaticidir, Dogruluk ekenin ödülüyse güvenlidir.
\par 19 Yürekten dogru olan yasama kavusur, Kötülügün ardindan giden ölümünü hazirlar.
\par 20 RAB sapik yürekliden igrenir, Dürüst yasayandan hosnut kalir.
\par 21 Bilin ki, kötü kisi cezasiz kalmaz, Dogrularin soyuysa kurtulur.
\par 22 Sagduyudan yoksun kadinin güzelligi, Domuzun burnundaki altin halkaya benzer.
\par 23 Dogrularin istegi hep iyilikle sonuçlanir, Kötülerin umutlariysa gazapla.
\par 24 Eliaçik olan daha çok kazanir, Hak yiyenin sonuysa yoksulluktur.
\par 25 Cömert olan bolluga erecek, Baskasina su verene su verilecek.
\par 26 Halk bugday istifleyeni lanetler, Ama bugday satani kutsar.
\par 27 Iyiligi amaç edinen begeni kazanir, Kötülügü amaç edinense kötülüge ugrar.
\par 28 Zenginligine güvenen tepetaklak gidecek, Oysa dogrular dalindaki yaprak gibi gelisecek.
\par 29 Ailesine sikinti çektirenin mirasi yeldir, Ahmaklar da bilgelerin kulu olur.
\par 30 Dogru kisinin isleri yasam agacinin meyvesine benzer, Bilge kisi insanlari kazanir.
\par 31 Bu dünyada dogru kisi bile cezalandirilirsa, Kötülerle günahlilarin cezalandirilacagi kesindir.

\chapter{12}

\par 1 Terbiye edilmeyi seven bilgiyi de sever, Azarlanmaktan nefret eden budaladir.
\par 2 Iyi kisi RAB'bin lütfuna erer, Ama düzenbazi RAB mahkûm eder.
\par 3 Kötülük kisiyi güvenlige kavusturmaz, Ama dogrularin kökü kazilamaz.
\par 4 Erdemli kadin kocasinin tacidir, Edepsiz kadinsa kocasini yer bitirir.
\par 5 Dogrularin tasarilari adil, Kötülerin ögütleri aldaticidir.
\par 6 Kötülerin sözleri ölüm tuzagidir, Dogrularin konusmasiysa onlari kurtarir.
\par 7 Kötüler yikilip yok olur, Dogru kisinin evi ayakta kalir.
\par 8 Kisi sagduyusu oraninda övülür, Çarpik düsünceliyse küçümsenir.
\par 9 Köle sahibi olup asagilanan Büyüklük taslayip ekmege muhtaç olandan yegdir.
\par 10 Dogru kisi hayvaniyla ilgilenir, Ama kötünün sevecenligi bile zalimcedir.
\par 11 Topragini isleyenin ekmegi bol olur, Hayal pesinde kosansa sagduyudan yoksundur.
\par 12 Kötü kisi kötülerin ganimetini ister, Ama dogru kisilerin kökü ürün verir.
\par 13 Kötü kisinin günahli sözleri kendisi için tuzaktir, Ama dogru kisi sikintiyi atlatir.
\par 14 Insan agzinin ürünüyle iyilige doyar, Elinin emegine göre de karsiligini alir.
\par 15 Ahmagin yolu kendi gözünde dogrudur, Bilge kisiyse ögüde kulak verir.
\par 16 Ahmak sinirlendigini hemen belli eder, Ama ihtiyatli olan asagilanmaya aldirmaz.
\par 17 Dürüst tanik dogruyu söyler, Yalanci taniksa hile solur.
\par 18 Düsünmeden söylenen sözler kiliç gibi keser, Bilgelerin diliyse sifa verir.
\par 19 Gerçek sözler sonsuza dek kalicidir, Oysa yalanin ömrü bir anliktir.
\par 20 Kötülük tasarlayanin yüregi hileci, Barisi ögütleyenin yüregiyse sevinçlidir.
\par 21 Dogru kisiye hiç zarar gelmez, Kötünün basiysa beladan kurtulmaz.
\par 22 RAB yalanci dudaklardan igrenir, Ama gerçege uyanlardan hosnut kalir.
\par 23 Ihtiyatli kisi bilgisini kendine saklar, Oysa akilsizin yüregi ahmakligini ilan eder.
\par 24 Çaliskanlarin eli egemenlik sürer, Tembellikse kölelige götürür.
\par 25 Kaygili yürek insani çökertir, Ama güzel söz sevindirir.
\par 26 Dogru kisi arkadasina da yol gösterir, Kötünün tuttugu yolsa kendini saptirir.
\par 27 Tembel kisi isini bitirmez, Oysa çaliskan degerli bir servet kazanir.
\par 28 Dogru yol yasam kaynagidir, Bu yol ölümsüzlüge götürür.

\chapter{13}

\par 1 Bilge kisi terbiye edilmeyi sever, Alayci kisi azarlansa da aldirmaz.
\par 2 Iyi insan agzindan çikan sözler için ödüllendirilir, Ama hainlerin soludugu zorbaliktir.
\par 3 Dilini tutan canini korur, Ama bosbogazin sonu yikimdir.
\par 4 Tembel caninin çektigini elde edemez, Çaliskanin istekleriyse tümüyle yerine gelir.
\par 5 Dogru kisi yalandan nefret eder, Kötünün sözleriyse igrençtir, yüzkarasidir.
\par 6 Dogruluk dürüst yasayani korur, Kötülük günahkâri yikar.
\par 7 Kimi hiçbir seyi yokken kendini zengin gösterir, Kimi serveti çokken kendini yoksul gösterir.
\par 8 Kisinin serveti gün gelir canina fidye olur, Oysa yoksul kisi tehdide aldirmaz.
\par 9 Dogrularin isigi parlak yanar, Kötülerin çirasi söner.
\par 10 Kibirden ancak kavga çikar, Ögüt dinleyense bilgedir.
\par 11 Havadan kazanilan para yok olur, Azar azar biriktirenin serveti çok olur.
\par 12 Ertelenen umut hayal kirikligina ugratir, Yerine gelen dilekse yasam verir.
\par 13 Uyarilara kulak asmayan bedelini öder, Buyruklara saygili olansa ödülünü alir.
\par 14 Bilgelerin ögrettikleri yasam kaynagidir, Insani ölüm tuzaklarindan uzaklastirir.
\par 15 Sagduyulu davranis sayginlik kazandirir, Hainlerin yoluysa yikima götürür.
\par 16 Ihtiyatli kisi isini bilerek yapar, Akilsiz kisiyse ahmakligini sergiler.
\par 17 Kötü ulak belaya düser, Güvenilir elçiyse sifa getirir.
\par 18 Terbiye edilmeye yanasmayani Yokluk ve utanç bekliyor, Ama azara kulak veren onurlandirilir.
\par 19 Yerine getirilen dilek mutluluk verir. Akilsiz kötülükten uzak kalamaz.
\par 20 Bilgelerle oturup kalkan bilge olur, Akilsizlarla dost olansa zarar görür.
\par 21 Günahkârin pesini felaket birakmaz, Dogrularin ödülüyse gönençtir.
\par 22 Iyi kisi torunlarina miras birakir, Günahkârin servetiyse dogru kisiye kalir.
\par 23 Yoksulun tarlasi bol ürün verebilir, Ama haksizlik bunu alip götürür.
\par 24 Oglundan degnegi esirgeyen, onu sevmiyor demektir. Seven baba özenle terbiye eder.
\par 25 Dogru kisinin yeterince yiyecegi vardir, Kötünün karniysa aç kalir.

\chapter{14}

\par 1 Bilge kadin evini yapar, Ahmak kadin evini kendi eliyle yikar.
\par 2 Dogru yolda yürüyen, RAB'den korkar, Yoldan sapan, RAB'bi hor görür.
\par 3 Ahmagin sözleri sirtina kötektir, Ama bilgenin dudaklari kendisini korur.
\par 4 Öküz yoksa yemlik bos kalir, Çünkü bol ürünü saglayan öküzün gücüdür.
\par 5 Güvenilir tanik yalan söylemez, Yalanci taniksa yalan solur.
\par 6 Alayci bilgeligi arasa da bulamaz, Akilli içinse bilgi edinmek kolaydir.
\par 7 Akilsiz kisiden uzak dur, Çünkü sana ögretecek bir seyi yok.
\par 8 Ihtiyatli kisinin bilgeligi, ne yapacagini bilmektir, Akilsizlarin ahmakligiysa aldanmaktir.
\par 9 Ahmaklar suç sunusuyla alay eder, Dürüstler ise iyi niyetlidir.
\par 10 Yürek kendi acisini bilir, Sevinciniyse kimse paylasmaz.
\par 11 Kötü kisinin evi yerle bir edilecek, Dogru kisinin konutuysa bayindir olacak.
\par 12 Öyle yol var ki, insana düz gibi görünür, Ama sonu ölümdür.
\par 13 Gülerken bile yürek sizlayabilir, Sevinç bitince aci yine görünebilir.
\par 14 Yüregi dönek olan tuttugu yolun, Iyi kisi de yaptiklarinin ödülünü alacaktir.
\par 15 Saf kisi her söze inanir, Ihtiyatli olansa attigi her adimi hesaplar.
\par 16 Bilge kisi korktugu için kötülükten uzaklasir, Akilsizsa büyüklük taslayip kendine güvenir.
\par 17 Çabuk öfkelenen ahmakça davranir, Düzenbazdan herkes nefret eder.
\par 18 Saf kisilerin mirasi akilsizliktir, Ihtiyatli kisilerin taci ise bilgidir.
\par 19 Alçaklar iyilerin önünde, Kötüler dogrularin kapisinda egilirler.
\par 20 Komsusu bile yoksulu sevmez, Oysa zenginin dostu çoktur.
\par 21 Komsuyu hor görmek günahtir, Ne mutlu mazluma lütfedene!
\par 22 Kötülük tasarlayan yolunu sasirmaz mi? Oysa iyilik tasarlayan sevgi ve sadakat kazanir.
\par 23 Her emek kazanç getirir, Ama bos lakirdi yoksulluga götürür.
\par 24 Bilgelerin taci servetleridir, Akilsizlarsa ahmakliklariyla taninir.
\par 25 Dürüst tanik can kurtarir, Yalanci tanik aldaticidir.
\par 26 RAB'den korkan tam güvenliktedir, RAB onun çocuklarina da siginak olacaktir.
\par 27 RAB korkusu yasam kaynagidir, Insani ölüm tuzaklarindan uzaklastirir.
\par 28 Kralin yüceligi halkinin çokluguna baglidir, Halk yok olursa hükümdar da mahvolur.
\par 29 Geç öfkelenen akillidir, Çabuk sinirlenen ahmakligini gösterir.
\par 30 Huzurlu yürek bedenin yasam kaynagidir, Hirs ise insani için için yer bitirir.
\par 31 Muhtaci ezen, Yaradani'ni hor görüyor demektir. Yoksula aciyansa Yaradan'i yüceltir.
\par 32 Kötü kisi ugradigi felaketle yikilir, Dogru insanin ölümde bile siginacak yeri var.
\par 33 Bilgelik akilli kisinin yüreginde barinir, Akilsizlar arasinda bile kendini belli eder.
\par 34 Dogruluk bir ulusu yüceltir, Oysa günah herhangi bir halk için utançtir.
\par 35 Kral sagduyulu kulunu begenir, Utanç getirene öfkelenir.

\chapter{15}

\par 1 Yumusak yanit gazabi yatistirir, Oysa yaralayici söz öfkeyi alevlendirir.
\par 2 Bilgenin dili bilgiyi iyi kullanir, Akilsizin agzindansa ahmaklik akar.
\par 3 RAB'bin gözü her yerde olani görür, Kötüleri de iyileri de gözler.
\par 4 Oksayici dil yasam verir, Çarpik dilse ruhu yaralar.
\par 5 Ahmak babasinin uyarilarini küçümser, Ihtiyatli kisi azara kulak verir.
\par 6 Dogru kisinin evi büyük hazine gibidir, Kötünün geliriyse sikinti kaynagidir.
\par 7 Bilgelerin dudaklari bilgi yayar, Ama akilsizlarin yüregi öyle degildir.
\par 8 RAB kötülerin kurbanindan igrenir, Ama dogrularin duasi O'nu hosnut eder.
\par 9 RAB kötü kisinin yolundan igrenir, Dogrulugun ardindan gideni sever.
\par 10 Yoldan sapan siddetle cezalandirilir Ve azarlanmaktan nefret eden ölüme gider.
\par 11 RAB, ölüm ve yikim diyarinda olup biteni bilir, Nerde kaldi ki insanin yüregi!
\par 12 Alayci kisi azarlanmaktan hoslanmaz, Bilgelere gidip danismaz.
\par 13 Mutlu yürek yüzü neselendirir, Acili yürek ruhu ezer.
\par 14 Akilli yürek bilgi arar, Akilsizin agziysa ahmaklikla beslenir.
\par 15 Mazlumun bütün günleri sikinti doludur, Mutlu bir yürekse sahibine sürekli ziyafettir.
\par 16 Yoksul olup RAB'den korkmak, Zengin olup kaygi içinde yasamaktan yegdir.
\par 17 Sevgi dolu bir ortamdaki sebze yemegi, Nefret dolu bir ortamdaki besili danadan yegdir.
\par 18 Huysuz kisi çekisme yaratir, Sabirli kisi kavgayi yatistirir.
\par 19 Tembelin yolu dikenli çit gibidir, Dogrunun yoluysa ana caddeye benzer.
\par 20 Bilge çocuk babasini sevindirir, Akilsiz çocuksa annesini küçümser.
\par 21 Sagduyudan yoksun kisi ahmakligiyla sevinir, Ama akilli insan dürüst bir yasam sürer.
\par 22 Karsilikli danisilmazsa tasarilar bosa çikar, Danismanlarin çokluguyla basariya ulasilir.
\par 23 Uygun yanit sahibini mutlu eder, Yerinde söylenen söz ne güzeldir!
\par 24 Sagduyulu kisi yukariya, yasama giden yoldadir, Bu da ölüler diyarina inmesini önler.
\par 25 RAB kibirlinin evini yikar, Dul kadinin sinirini korur.
\par 26 RAB kötünün tasarilarindan igrenir, Temiz düsüncelerden hosnut kalir.
\par 27 Kazanca düskün kisi kendi evine sikinti verir, Rüsvetten nefret edense rahat yasar.
\par 28 Dogru kisinin akli yanitini iyi tartar, Kötünün agzi kötülük saçar.
\par 29 RAB kötülerden uzak durur, Oysa dogrularin duasini duyar.
\par 30 Gülen gözler yüregi sevindirir, Iyi haber bedeni ferahlatir.
\par 31 Yasam veren uyarilari dinleyen, Bilgeler arasinda konaklar.
\par 32 Terbiyeden kaçan kendine zarar verir, Azara kulak verense sagduyu kazanir.
\par 33 RAB korkusu bilgelik ögretir, Alçakgönüllülük de onurun önkosuludur.

\chapter{16}

\par 1 Insan akliyla çok sey tasarlayabilir, Ama dilin verecegi yanit RAB'dendir.
\par 2 Insan her yaptigini temiz sanir, Ama niyetlerini tartan RAB'dir.
\par 3 Yapacagin isleri RAB'be emanet et, O zaman tasarilarin gerçeklesir.
\par 4 RAB her seyi amacina uygun yapar, Kötü kisinin yikim gününü de O hazirlar.
\par 5 RAB yüregi küstah olandan igrenir, Bilin ki, öyleleri cezasiz kalmaz.
\par 6 Sevgi ve baglilik suçlari bagislatir, RAB korkusu insani kötülükten uzaklastirir.
\par 7 RAB kisinin yasayisindan hosnutsa Düsmanlarini bile onunla baristirir.
\par 8 Dogrulukla kazanilan az sey Haksizlikla kazanilan büyük gelirden iyidir.
\par 9 Kisi yüreginde gidecegi yolu tasarlar, Ama adimlarini RAB yönlendirir.
\par 10 Tanri buyruklarini kralin agziyla açiklar, Bu nedenle kral adaleti çignememelidir.
\par 11 Dogru terazi ve baskül RAB'bindir, Bütün tarti agirliklarini O belirler.
\par 12 Krallar kötülükten igrenir, Çünkü tahtin güvencesi adalettir.
\par 13 Kral dogru söyleyenden hosnut kalir, Dürüst konusani sever.
\par 14 Kralin öfkesi ölüm habercisidir, Ama bilge kisi onu yatistirir.
\par 15 Kralin yüzü gülüyorsa, yasam demektir. Lütfu son yagmuru getiren bulut gibidir.
\par 16 Bilgelik kazanmak altindan daha degerlidir, Akla sahip olmak da gümüse yeglenir.
\par 17 Dürüstlerin tuttugu yol kötülükten uzaklastirir, Yoluna dikkat eden, canini korur.
\par 18 Gururun ardindan yikim, Kibirli ruhun ardindan da düsüs gelir.
\par 19 Mazlumlar arasinda alçakgönüllü biri olmak, Kibirlilerle çapul mali paylasmaktan iyidir.
\par 20 Ögüde kulak veren basariya ulasir, RAB'be güvenen mutlu olur.
\par 21 Bilge yüreklilere akilli denir, Tatli söz ikna gücünü artirir.
\par 22 Sagduyu, sahibine yasam kaynagi, Ahmakliksa ahmaklara cezadir.
\par 23 Bilgenin akli diline yön verir, Dudaklarinin ikna gücünü artirir.
\par 24 Hos sözler petek bali gibidir, Cana tatli ve bedene sifadir.
\par 25 Öyle yol var ki, insana düz gibi görünür, Ama sonu ölümdür.
\par 26 Emekçinin istahidir onu çalistiran, Çünkü açligi onu kamçilar.
\par 27 Alçaklar baskalarina kötülük tasarlar, Konusmalari kavurucu ates gibidir.
\par 28 Huysuz kisi çekismeyi körükler, Dedikoducu can dostlari ayirir.
\par 29 Zorba kisi baskalarini ayartir Ve onlari olumsuz yola yöneltir.
\par 30 Göz kirpmak düzenbazliga, Sinsi gülücükler kötülüge isarettir.
\par 31 Agarmis saçlar onur tacidir, Dogru yasayisla kazanilir.
\par 32 Sabirli kisi yigitten üstündür, Kendini denetleyen de kentler fethedenden üstündür.
\par 33 Insan kura atar, Ama her karari RAB verir.

\chapter{17}

\par 1 Huzur içinde kuru bir lokma, Kavga ve ziyafet dolu evden iyidir.
\par 2 Sagduyulu köle, Ailesini utanca sokan ogula egemen olur Ve kardeslerle birlikte mirastan pay alir.
\par 3 Altin ocakta, gümüs potada aritilir, Yüregi aritansa RAB'dir.
\par 4 Kötü kisi fesat yüklü dudaklari dinler, Yalanci da yikici dile kulak verir.
\par 5 Yoksulla alay eden, onu yaratani hor görür. Felakete sevinen cezasiz kalmaz.
\par 6 Torunlar yaslilarin tacidir, Çocuklarin övüncü anne babalaridir.
\par 7 Kurumlu sözler ahmaga nasil yakismazsa, Soyluya da yalanci dudaklar hiç yakismaz.
\par 8 Sahibinin gözünde rüsvet bir tilsimdir. Ne yapsa basarili olur.
\par 9 Sevgi isteyen kisi suçlari bagislar, Olayi diline dolayansa can dostlari ayirir.
\par 10 Akilli kisiyi azarlamak, Akilsiza yüz darbe vurmaktan etkilidir.
\par 11 Kötü kisi ancak baskaldirmaya egilimlidir, Ona gönderilecek ulak acimasiz olacaktir.
\par 12 Azginligi üstünde bir akilsizla karsilasmak, Yavrularindan edilmis disi ayiyla karsilasmaktan beterdir.
\par 13 Iyiligin karsiligini kötülükle ödeyenin Evinden kötülük eksik olmaz.
\par 14 Kavganin baslangici su sizintisina benzer, Bir patlamaya yol açmadan çekismeyi birak.
\par 15 Kötüyü aklayan da, dogruyu mahkûm eden de RAB'bi tiksindirir.
\par 16 Akilsiz biri bilgelik satin almak için niye para harcasin? Zaten sagduyudan yoksun!
\par 17 Dost her zaman sever, Kardes sikintili günde belli olur.
\par 18 Sagduyudan yoksun kisi el sikisip Baskasina kefil olur.
\par 19 Baskaldiriyi seven kavgayi sever, Kapisini yüksek yapan yikimina davetiye çikarir.
\par 20 Sapik yürekli kisi iyilik beklememeli. Diliyle aldatan da belaya düser.
\par 21 Akilsiz kendisini dogurana derttir, Ahmagin babasi sevinç nedir bilmez.
\par 22 Iç ferahligi saglik getirir, Ezik ruh ise bedeni yipratir.
\par 23 Kötü kisi adaleti saptirmak için Gizlice rüsvet alir.
\par 24 Akilli kisi gözünü bilgelikten ayirmaz, Akilsizin gözüyse hep sagda soldadir.
\par 25 Akilsiz çocuk babasina üzüntü, Annesine aci verir.
\par 26 Ne suçsuza ceza kesmek iyidir, Ne de görevliyi dürüst davrandigi için dövmek...
\par 27 Bilgili kisi az konusur, Akilli kisi sakin ruhludur.
\par 28 Çenesini tutup susan ahmak bile Bilge ve akilli sayilir.

\chapter{18}

\par 1 Geçimsiz kisi kendi çikari pesindedir, Iyi ögüde hep karsi çikar.
\par 2 Akilsiz kisi bir sey anlamaktan çok Kendi düsüncelerini açmaktan hoslanir.
\par 3 Kötülügü asagilanma, Ayibi utanç izler.
\par 4 Bilge kisinin agzindan çikan sözler derin sular gibidir, Bilgelik pinari da coskun bir akarsu.
\par 5 Kötüyü kayirmak da, Suçsuzdan adaleti esirgemek de iyi degildir.
\par 6 Akilsizin dudaklari çekismeye yol açar, Agzi da dayagi davet eder.
\par 7 Akilsizin agzi kendisini mahveder, Dudaklari da canina tuzaktir.
\par 8 Dedikodu tatli lokma gibidir, Insanin ta içine isler.
\par 9 Isini savsaklayan kisi Yikiciya kardestir.
\par 10 RAB'bin adi güçlü kuledir, Ona siginan dogru kisi için korunaktir.
\par 11 Zengin servetini bir kale, Asilmaz bir sur sanir.
\par 12 Yürekteki gururu düsüs, Alçakgönüllülügü ise onur izler.
\par 13 Dinlemeden yanit vermek Ahmaklik ve utançtir.
\par 14 Insanin ruhu hastalikta ona destektir. Ama ezik ruh nasil dayanabilir?
\par 15 Akilli kisi bilgiyi satin alir, Bilgenin kulagi da bilgi pesindedir.
\par 16 Armagan, verenin yolunu açar Ve kendisini büyüklerin önüne çikartir.
\par 17 Durusmada ilk konusan hakli görünür, Baskasi çikip onu sorgulayana dek.
\par 18 Kura çekismeleri sona erdirir, Güçlü rakipleri uzlastirir.
\par 19 Gücenmis kardes surlu kentten daha zor elde edilir. Çekisme sürgülü kale kapisi gibidir.
\par 20 Insanin karni agzinin meyvesiyle, Dudaklarinin ürünüyle doyar.
\par 21 Dil ölüme de götürebilir, yasama da; Konusmayi seven, dilin meyvesine katlanmak zorundadir.
\par 22 Iyi bir es bulan iyilik bulur Ve RAB'bin lütfuna erer.
\par 23 Yoksul acinma dilenir, Zenginin yanitiysa serttir.
\par 24 Yikima götüren dostlar vardir, Ama öyle dost var ki, kardesten yakindir insana.

\chapter{19}

\par 1 Dürüst yasayan bir yoksul olmak, Yalanci bir akilsiz olmaktan yegdir.
\par 2 Bilgisiz heves ise yaramaz, Acelecilik insani yanilgiya düsürür.
\par 3 Insanin ahmakligi yasamini yikar, Yine de içinden RAB'be öfkelenir.
\par 4 Zenginlik dost üstüne dost kazandirir. Oysa yoksulun dostu onu yüzüstü birakir.
\par 5 Yalanci tanik cezasiz kalmaz, Yalan soluyan kurtulamaz.
\par 6 Birçoklari önemli kisinin gözüne girmek Ve eli açik olanin dostu olmak ister.
\par 7 Yoksulun akrabalari bile onu sevmezse, Dostlarinin ondan uzak duracagi daha da kesindir. Ne kadar yalvarsa ona yaklasmazlar.
\par 8 Sagduyulu olan canini sever, Akli izleyen bolluga kavusur.
\par 9 Yalanci tanik cezasiz kalmaz, Yalan soluyan yok olur.
\par 10 Akilsizin gösterisli bir yasam sürmesi uygun degilse, Kölelerin önderlere egemen olmasi Hiç uygun degildir.
\par 11 Sagduyulu kisi sabirlidir, Kusurlari hos görmesi ona onur kazandirir.
\par 12 Kralin öfkesi genç aslanin kükreyisine benzer, Lütfuysa otlarin üzerine düsen çiy gibidir.
\par 13 Akilsiz çocuk babasinin basina beladir, Dirdir eden kadin sürekli damlayan su gibidir.
\par 14 Ev ve servet babadan mirastir, Ama sagduyulu kadin RAB'bin armaganidir.
\par 15 Tembellik insani uyusukluga iter, Haylaz kisi de aç kalir.
\par 16 Tanri buyruguna uyan canini korur, Gitmesi gereken yollari umursamayan ölür.
\par 17 Yoksula aciyan kisi RAB'be ödünç vermis olur, Yaptigi iyilik için RAB onu ödüllendirir.
\par 18 Henüz umut varken çocugunu egit, Onun yikimina neden olma.
\par 19 Huysuz insan cezasini çekmelidir. Onu bir kere kurtarsan da, hep ayni seyi yapman gerekir.
\par 20 Ögüde kulak ver, terbiyeyi kabul et ki, Ömrünün kalan kismi boyunca bilge olasin.
\par 21 Insan yüreginde çok sey tasarlar, Ama gerçeklesen, RAB'bin amacidir.
\par 22 Insandan istenen vefadir, Yoksul olmak yalanci olmaktan yegdir.
\par 23 RAB korkusu Doygun ve dertsiz bir yasama kavusturur.
\par 24 Tembel sahana daldirdigi elini Agzina geri götürmek bile istemez.
\par 25 Alayciyi döversen bön kisi ibret alir, Akilli kisiyi azarlarsan bilgisine bilgi katar.
\par 26 Babasina saldiran, annesini kovan çocuk, Ailesinin utanci ve yüzkarasidir.
\par 27 Oglum, uyarilara kulagini tikarsan, Bilgi kaynagi sözlerden saparsin.
\par 28 Niyeti bozuk tanik adaletle eglenir, Kötülerin agzi fesatla beslenir.
\par 29 Alaycilar için ceza, Akilsizlarin sirti için kötek hazirdir.

\chapter{20}

\par 1 Sarap insani alayci, içki gürültücü yapar, Onun etkisiyle yoldan sapan bilge degildir.
\par 2 Kralin öfkesi genç aslanin kükreyisine benzer, Onu kizdiran canindan olur.
\par 3 Kavgadan kaçinmak insan için onurdur, Oysa her ahmak tartismaya hazirdir.
\par 4 Sonbaharda çift sürmeyen tembel, Hasatta aradigini bulamaz.
\par 5 Insanin niyetleri derin bir kuyunun sulari gibidir, Akilli kisi onlari açiga çikarir.
\par 6 Insanlarin çogu, "Vefaliyim" der. Ama sadik birini kim bulabilir?
\par 7 Dogru ve dürüst bir babaya Sahip olan çocuklara ne mutlu!
\par 8 Yargi kürsüsünde oturan kral, Kötülügü gözleriyle ayiklar.
\par 9 Kim, "Yüregimi pak kildim, Günahimdan arindim" diyebilir?
\par 10 RAB hileli tartidan da, hileli ölçüden de tiksinir.
\par 11 Çocuk bile eylemleriyle kendini belli eder, Yaptiklari pak ve dogru mu, degil mi, anlasilir.
\par 12 Isiten kulagi da gören gözü de RAB yaratmistir.
\par 13 Uykuyu seversen yoksullasirsin, Uyanik durursan ekmegin bol olur.
\par 14 Alici, "Ise yaramaz, ise yaramaz" der, Ama alip gittikten sonra aldigiyla övünür.
\par 15 Bol bol altinin, mücevherin olabilir, Ama bilgi akitan dudaklar daha degerlidir.
\par 16 Tanimadigi birine kefil olanin giysisini al; Bir yabanci için yapiyorsa bunu, Giysisini rehin tut.
\par 17 Hileyle kazanilan yiyecek insana tatli gelir, Ama sonra agza dolan çakil gibidir.
\par 18 Tasarilarini danisarak yap, Yöntemlere uyarak savas.
\par 19 Dedikoducu sir saklayamaz, Bu nedenle agzi gevsek olanla arkadaslik etme.
\par 20 Annesine ya da babasina sövenin Isigi zifiri karanlikta sönecek.
\par 21 Tez elde edilen mirasin Sonu bereketli olmaz.
\par 22 "Bu kötülügü sana ödetecegim" deme; RAB'bi bekle, O seni kurtarir.
\par 23 RAB hileli tartidan tiksinir, Hileli teraziden hoslanmaz.
\par 24 Insanin adimlarini RAB yönlendirir; Öyleyse insan tuttugu yolu nasil anlayabilir?
\par 25 Düsünmeden adakta bulunmak Sakincalidir.
\par 26 Bilge kral kötüleri ayiklar, Harman döver gibi cezalandirir.
\par 27 Insanin ruhu RAB'bin isigidir, Iç varligin derinliklerine isler.
\par 28 Sevgi ve sadakat kralin güvencesidir. Onun tahtini saglamlastiran sevgidir.
\par 29 Gençlerin görkemi güçleri, Yaslilarin onuru agarmis saçlardir.
\par 30 Yaralayan darbeler kötülügü temizler, Kötek iç varligin derinliklerini paklar.

\chapter{21}

\par 1 Kralin yüregi RAB'bin elindedir, Kanaldaki su gibi onu istedigi yöne çevirir.
\par 2 Insan izledigi her yolun dogru oldugunu sanir, Ama niyetlerini tartan RAB'dir.
\par 3 RAB kendisine kurban sunulmasindan çok, Dogrulugun ve adaletin yerine getirilmesini ister.
\par 4 Küstah bakislar ve kibirli yürek Kötülerin çirasi ve günahidir.
\par 5 Çaliskanin tasarilari hep bollukla, Her türlü acelecilik hep yoklukla sonuçlanir.
\par 6 Yalan dolanla yapilan servet, Sis gibi geçicidir ve ölüm tuzagidir.
\par 7 Kötülerin zorbaligi kendilerini süpürüp götürür, Çünkü dogru olani yapmaya yanasmazlar.
\par 8 Suçlunun yolu dolambaçli, Pak kisinin yaptiklariysa dosdogrudur.
\par 9 Kavgaci kadinla ayni evde oturmaktansa, Damin kösesinde oturmak yegdir.
\par 10 Kötünün can attigi kötülüktür, Hiç kimseye acimaz.
\par 11 Alayci cezalandirilinca bön kisi akillanir, Bilge olan ögretilenden bilgi kazanir.
\par 12 Adil Olan, kötünün evini dikkatle gözler Ve kötüleri yikima ugratir.
\par 13 Yoksulun feryadina kulagini tikayanin Feryadina yanit verilmeyecektir.
\par 14 Gizlice verilen armagan öfkeyi, Koyna sokusturulan rüsvet de kizgin gazabi yatistirir.
\par 15 Hak yerine gelince dogru kisi sevinir, Fesatçi dehsete düser.
\par 16 Sagduyudan uzaklasan, Kendini ölüler arasinda bulur.
\par 17 Zevkine düskün olan yoksullasir, Saraba ve zeytinyagina düskün kisi de zengin olmaz.
\par 18 Kötü kisi dogru kisinin fidyesidir, Hain de dürüstün.
\par 19 Çölde yasamak, Can sikici ve kavgaci kadinla yasamaktan yegdir.
\par 20 Bilgenin evi degerli esya ve zeytinyagiyla doludur, Akilsizsa malini har vurup harman savurur.
\par 21 Dogrulugun ve sevginin ardindan kosan, Yasam, gönenç ve onur bulur.
\par 22 Bilge kisi güçlülerin kentine saldirip Güvendikleri kaleyi yikar.
\par 23 Agzini ve dilini tutan Basini beladan korur.
\par 24 Gururlu, küstah ve alayci: Bunlar kas kas kasilan insanin adlaridir.
\par 25 Tembelin istegi onu ölüme götürür, Çünkü elleri çalismaktan kaçinir;
\par 26 Bütün gün isteklerini siralar durur, Oysa dogru kisi esirgemeden verir.
\par 27 Kötülerin sundugu kurban igrençtir, Hele bunu kötü niyetle sunarlarsa.
\par 28 Yalanci tanik yok olur, Dinlemeyi bilenin tanikligiysa inandiricidir.
\par 29 Kötü kisi kendine güçlü bir görünüm verir, Erdemli insansa tuttugu yoldan emindir.
\par 30 RAB'be karsi basarili olabilecek Bilgelik, akil ve tasari yoktur.
\par 31 At savas günü için hazir tutulur, Ama zafer saglayan RAB'dir.

\chapter{22}

\par 1 Iyi ad büyük servetten, Sayginlik gümüs ve altindan yegdir.
\par 2 Zenginle yoksulun ortak yönü su: Her ikisini de RAB yaratti.
\par 3 Ihtiyatli kisi tehlikeyi görünce saklanir, Bönse öne atilir ve zarar görür.
\par 4 Alçakgönüllülügün ve RAB korkusunun ödülü, Zenginlik, onur ve yasamdir.
\par 5 Kötünün yolu diken ve tuzakla doludur. Canini korumak isteyen bunlardan uzak durur.
\par 6 Çocugu tutmasi gereken yola göre yetistir, Yaslandiginda o yoldan ayrilmaz.
\par 7 Zengin yoksullara egemen olur, Borç alan borç verenin kulu olur.
\par 8 Fesat eken dert biçer, Gazabinin degnegi yok olur.
\par 9 Cömert olan kutsanir, Çünkü yemegini yoksullarla paylasir.
\par 10 Alayciyi kov, kavga biter; Çekisme ve asagilamalar da sona erer.
\par 11 Yürek temizligini ve güzel sözleri seven, Kralin dostlugunu kazanir.
\par 12 RAB bilgiyi gözetip korur, Hainin sözlerini ise altüst eder.
\par 13 Tembel der ki, "Disarda aslan var, Sokaga çiksam beni parçalar."
\par 14 Sokak kadininin agzi dipsiz çukur gibidir, RAB'bin gazabina ugrayan oraya düser.
\par 15 Akilsizlik çocugun öz yapisindadir, Degnekle terbiye edilirse akilsizliktan uzaklasir.
\par 16 Servetini büyütmek için yoksulu ezenle Zengine armagan verenin sonu yoksulluktur.
\par 17 Kulak ver, bilgelerin sözlerini dinle, Ögrettigimi zihnine isle.
\par 18 Sözlerimi yüreginde saklarsan mutlu olursun, Onlar hep hazir olsun dudaklarinda.
\par 19 RAB'be güvenmen için Bugün bunlari sana, evet sana da bildiriyorum.
\par 20 Senin için otuz söz yazdim, Bilgi ve ögüt sözleri...
\par 21 Öyle ki, güvenilir, dogru sözleri bilesin, Böylece seni gönderene güvenilir yanit verebilesin.
\par 22 Yoksulu, yoksul oldugu için soymaya kalkma, Düskünü mahkemede ezme.
\par 23 Çünkü onlarin davasini RAB yüklenecek Ve onlari soyanlarin canini alacak.
\par 24 Huysuz kisiyle arkadaslik etme; Tez öfkelenenle yola çikma.
\par 25 Yoksa onun yollarina alisir, Kendini tuzaga düsmüs bulursun.
\par 26 El sikisip Baskasinin borcuna kefil olmaktan kaçin.
\par 27 Ödeyecek paran olmazsa, Altindaki dösege bile el koyarlar.
\par 28 Atalarinin belirledigi Eski sinir taslarinin yerini degistirme.
\par 29 Isinde usta birini görüyor musun? Öylesi siradan kisilere degil, Krallara bile hizmet eder.

\chapter{23}

\par 1 Bir önderle yemege oturdugunda Önüne konulana dikkat et.
\par 2 Istahina yenilecek olursan, Daya biçagi kendi bogazina.
\par 3 Onun lezzetli yemeklerini çekmesin canin, Böyle yemegin ardinda hile olabilir.
\par 4 Zengin olmak için didinip durma, Çikar bunu aklindan.
\par 5 Servet göz açip kapayana dek yok olur, Kanatlanip kartal gibi göklere uçar.
\par 6 Cimrinin verdigi yemegi yeme, Lezzetli yemeklerini çekmesin canin.
\par 7 Çünkü yedigin her seyin hesabini tutar, "Ye, iç" der sana, Ama yüregi senden yana degildir.
\par 8 Yedigin azicik yemegi kusarsin, Söyledigin güzel sözler de bosa gider.
\par 9 Akilsiza ögüt vermeye kalkma, Çünkü senin sözlerindeki sagduyuyu küçümser.
\par 10 Eski sinir taslarinin yerini degistirme, Öksüzlerin topragina el sürme.
\par 11 Çünkü onlarin Velisi güçlüdür Ve onlarin davasini sana karsi O yürütür.
\par 12 Uyarilari zihnine isle, Bilgi dolu sözlere kulak ver.
\par 13 Çocugunu terbiye etmekten geri kalma, Onu degnekle dövsen de ölmez.
\par 14 Onu degnekle döversen, Canini ölüler diyarindan kurtarirsin.
\par 15 Oglum, bilge yürekli olursan, Benim yüregim de sevinir.
\par 16 Dudaklarin dogru konustugunda Gönlüm de cosar.
\par 17 Günahkârlara imrenmektense, Sürekli RAB korkusunda yasa.
\par 18 Böylece bir gelecegin olur Ve umudun bosa çikmaz.
\par 19 Oglum, dinle ve bilge ol, Yüregini dogru yolda tut.
\par 20 Asiri sarap içenlerle, Ete düskün oburlarla arkadaslik etme.
\par 21 Çünkü ayyas ve obur kisi yoksullasir, Uyusukluk da insana paçavra giydirir.
\par 22 Sana yasam veren babanin sözlerine kulak ver, Yaslandigi zaman anneni hor görme.
\par 23 Gerçegi satin al ve satma; Bilgeligi, terbiyeyi, akli da.
\par 24 Dogru kisinin babasi costukça cosar, Bilgece davranan ogulun babasi sevinir.
\par 25 Annenle baban seninle cossun, Seni doguran sevinsin.
\par 26 Oglum, beni yürekten dinle, Gözünü gittigim yoldan ayirma.
\par 27 Çünkü fahise derin bir çukur, Ahlaksiz kadin dar bir kuyudur.
\par 28 Evet, soyguncu gibi pusuda bekler Ve birçok erkegi yoldan çikarir.
\par 29 Ah çeken kim? Vah çeken kim? Kimdir çekisip duran? Yakinan kim? Bos yere yaralanan kim? Gözleri kanli olan kim?
\par 30 Içmeye oturup kalkamayanlar, Karisik saraplari denemeye gidenlerdir.
\par 31 Sarabin kizil rengine, Kadehte isimasina, Bogazdan asagi süzülüvermesine bakma.
\par 32 Sonunda yilan gibi isirir, Engerek gibi sokar.
\par 33 Gözlerin garip seyler görür, Aklindan ahlaksizliklar geçer.
\par 34 Kendini kâh denizin ortasinda, Kâh gemi direginin tepesinde yatiyor sanirsin.
\par 35 "Dövdüler beni ama incinmedim, Vurdular ama farketmedim" dersin, "Yeniden içmek için ne zaman ayilacagim?"

\chapter{24}

\par 1 Kötülere imrenme, Onlarla birlikte olmayi isteme.
\par 2 Çünkü yürekleri zorbalik tasarlar, Dudaklari belalardan söz eder.
\par 3 Ev bilgelikle yapilir, Akilla pekistirilir.
\par 4 Bilgi sayesinde odalari Her türlü degerli, güzel esyayla dolar.
\par 5 Bilgelik güçten, Bilgi kaba kuvvetten üstündür.
\par 6 Savasmak için yöntem, Zafer kazanmak için birçok danisman gerekli.
\par 7 Ahmak için bilgelik ulasilamayacak kadar yüksektir, Kent kurulunda agzini açamaz.
\par 8 Kötülük tasarlayan kisi Düzenbaz olarak bilinecektir.
\par 9 Ahmakça tasarilar günahtir, Alayci kisiden herkes igrenir.
\par 10 Sikintili günde cesaretini yitirirsen, Gücün kit demektir.
\par 11 Ölüm tehlikesi içinde olanlari kurtar, Ölmek üzere olanlari esirge.
\par 12 "Iste bunu bilmiyordum" desen de, Insanin yüregindekini bilen sezmez mi? Senin canini koruyan anlamaz mi? Ödetmez mi herkese yaptigini?
\par 13 Oglum, bal ye, çünkü iyidir, Süzme bal damaga tatli gelir.
\par 14 Bilgelik de canin için öyledir, bilmis ol. Bilgeligi bulursan bir gelecegin olur Ve umudun bosa çikmaz.
\par 15 Ey kötü adam, dogru kisinin evine karsi pusuya yatma, Konutunu yikmaya kalkma.
\par 16 Çünkü dogru kisi yedi kez düsse yine kalkar, Ama kötüler felakette yikilir.
\par 17 Düsmanin düsüsüne keyiflenme, Sendelemesine sevinme.
\par 18 Yoksa RAB görür ve hosnut kalmaz Ve düsmanina duydugu öfke yatisir.
\par 19 Kötülük edenlere kizip üzülme, Onlara özenme.
\par 20 Çünkü kötülerin gelecegi yok, Çirasi sönecek onlarin.
\par 21 Oglum, RAB'be ve krala saygi göster, Onlara baskaldiranlarla arkadaslik etme.
\par 22 Çünkü onlar ansizin felakete ugrar, Insanin basina ne belalar getireceklerini kim bilir? Bilgelerin Öbür Özdeyisleri
\par 23 Sunlar da bilgelerin sözleridir: Yargilarken yan tutmak iyi degildir.
\par 24 Kötüye, "Suçsuzsun" diyen yargici Halklar lanetler, uluslar kinar.
\par 25 Ne mutlu suçluyu mahkûm edene! Herkes onu candan kutlar.
\par 26 Dürüst yanit Gerçek dostlugun isaretidir.
\par 27 Ilkin disardaki isini bitirip tarlani hazirla, Ondan sonra evini yap.
\par 28 Baskalarina karsi nedensiz taniklik etme Ve dudaklarinla aldatma.
\par 29 "Bana yaptigini ben de ona yapacagim, Ödetecegim bana yaptigini" deme.
\par 30 Tembelin tarlasindan, Sagduyudan yoksun kisinin bagindan geçtigimde
\par 31 Her yani dikenlerin, otlarin Kapladigini gördüm; Tas duvar da yikilmisti.
\par 32 Gördüklerimi derin derin düsündüm, Seyrettiklerimden ibret aldim.
\par 33 "Biraz kestireyim, biraz uyuklayayim, Ellerimi kavusturup söyle bir uyuyayim" demeye kalmadan,
\par 34 Yokluk bir haydut gibi, Yoksulluk bir akinci gibi gelir üzerine.

\chapter{25}

\par 1 Bundan sonrakiler de Süleyman'in özdeyisleridir. Bunlari Yahuda Krali Hizkiya'nin adamlari derledi.
\par 2 Tanri'yi gizli tuttugu seyler için, Krallariysa açiga çikardiklari için yüceltiriz.
\par 3 Gögün yüksekligi, yerin derinligi gibi, Krallarin aklindan geçen de kestirilemez.
\par 4 Cürufu gümüsten ayirinca, Kuyumcunun isleyecegi madde kalir.
\par 5 Kötüleri kralin huzurundan uzaklastirirsan Kralin tahti adaletle pekisir.
\par 6 Kralin önünde kendini yüceltme, Önemli kisiler arasinda yer edinmeye çalisma.
\par 7 Çünkü kralin seni bir soylunun önünde alasagi etmesindense, Sana, "Yukariya gel" demesi yegdir.
\par 8 Gördüklerinle hemencecik mahkemeye basvurma; Çünkü baskasi seni utandirabilir, Sonra ne yapacagini bilemezsin.
\par 9 Davani dogrudan komsunla gör; Baskasinin sirrini açiklama.
\par 10 Yoksa isiten seni utandirabilir Ve bu kötü ün yakani birakmaz.
\par 11 Yerinde söylenen söz, Gümüs oymalardaki altin elma gibidir.
\par 12 Altin küpe ya da altin bir süs neyse, Dinleyen kulak için bilgenin azarlamasi da öyledir.
\par 13 Hasatta kar serinligi nasilsa, Güvenilir ulak da kendisini gönderenler için öyledir. Böyle biri efendilerinin canina can katar.
\par 14 Yagmursuz bulut ve yel nasilsa, Vermedigi armaganla övünen kisi de öyledir.
\par 15 Sabirla bir hükümdar bile ikna edilir, Tatli dil en güçlü direnci kirar.
\par 16 Bal buldun mu yeteri kadar ye, Fazla doyarsan kusarsin.
\par 17 Baskalarinin evine seyrek git, Yoksa onlari bezdirir, nefretini kazanirsin.
\par 18 Baskasina karsi yalanci taniklik eden Topuz, kiliç ya da sivri ok gibidir.
\par 19 Sikintili günde haine güvenmek, Çürük dise ya da sakat ayaga güvenmek gibidir.
\par 20 Dertli kisiye ezgi söylemek, Soguk günde giysilerini üzerinden almaya, Ya da sodaya sirke katmaya benzer.
\par 21 Düsmanin acikmissa doyur, Susamissa su ver.
\par 22 Bunu yapmakla onu utanca bogarsin Ve RAB seni ödüllendirir.
\par 23 Kuzeyden esen rüzgar nasil yagmur getirirse, Iftiraci dil de öfkeli bakislara yol açar.
\par 24 Kavgaci kadinla ayni evde oturmaktansa, Damin kösesinde oturmak yegdir.
\par 25 Susamis kisi için soguk su neyse, Uzak ülkeden gelen iyi haber de öyledir.
\par 26 Kötünün önünde pes eden dogru kisi, Suyu bulanmis pinar, kirlenmis kuyu gibidir.
\par 27 Fazla bal yemek iyi degildir; Hep yüceltilmeyi beklemek de...
\par 28 Kendini denetleyemeyen kisi Yikilmis sursuz kent gibidir.

\chapter{26}

\par 1 Yaz ortasinda kar, hasatta yagmur uygun olmadigi gibi, Akilsiza da onur yakismaz.
\par 2 Öteye beriye uçusan serçe Ve kirlangiç gibi, Hak edilmemis lanet de tutmaz.
\par 3 Ata kirbaç, esege gem, Akilsizin sirtina da degnek gerek.
\par 4 Akilsiza ahmakligina göre karsilik verme, Yoksa sen de onun düzeyine inersin.
\par 5 Akilsiza ahmakligina uygun karsilik ver, Yoksa kendini bilge sanir.
\par 6 Akilsizin eliyle haber gönderen, Kendi ayaklarini kesen biri gibi, Kendine zarar verir.
\par 7 Akilsizin agzinda özdeyis, Kötürümün sarkan bacaklari gibidir.
\par 8 Akilsizi onurlandirmak, Tasi sapana baglamak gibidir.
\par 9 Sarhosun elindeki dikenli dal ne ise, Akilsizin agzinda özdeyis de odur.
\par 10 Oklarini gelisigüzel firlatan okçu neyse, Yoldan geçen akilsizi ya da sarhosu ücretle tutan da öyledir.
\par 11 Ahmakligini tekrarlayan akilsiz, Kusmuguna dönen köpek gibidir.
\par 12 Kendini bilge gören birini taniyor musun? Akilsiz bile ondan daha umut vericidir.
\par 13 Tembel, "Yolda aslan var, Sokaklarda aslan dolasiyor" der.
\par 14 Menteseleri üzerinde dönen kapi gibi, Tembel de yataginda döner durur.
\par 15 Tembel elini sahana daldirir, Yeniden agzina götürmeye üsenir.
\par 16 Tembel kendini, Akillica yanit veren yedi kisiden daha bilge sanir.
\par 17 Kendini ilgilendirmeyen bir kavgaya bulasan kisi, Yoldan geçen köpegi kulaklarindan tutana benzer.
\par 18 Atesli ve öldürücü oklar savuran bir deli neyse, Komsusunu aldatip, "Saka yapiyordum" Diyen de öyledir.
\par 20 Odun bitince ates söner, Dedikoducu yok olunca kavga diner.
\par 21 Kor için kömür, ates için odun neyse, Çekismeyi alevlendirmek için kavgaci da öyledir.
\par 22 Dedikodu tatli lokma gibidir, Insanin ta içine isler.
\par 23 Oksayici dudaklarla kötü yürek, Sirlanmis toprak kaba benzer.
\par 24 Yüregi nefret dolu kisi sözleriyle niyetini gizlemeye çalisir, Ama içi hile doludur.
\par 25 Güzel sözlerine kanma, Çünkü yüreginde yedi igrenç sey vardir.
\par 26 Nefretini hileyle örtse bile, Kötülügü toplumun önünde ortaya çikar.
\par 27 Baskasinin kuyusunu kazan içine kendi düser, Tasi yuvarlayan altinda kalir.
\par 28 Yalanci dil incittigi kisilerden nefret eder, Yaltaklanan agizdan yikim gelir.

\chapter{27}

\par 1 Yarinla övünme, Çünkü ne getirecegini bilemezsin.
\par 2 Seni kendi agzin degil, baskalari övsün, Kendi dudaklarin degil, yabanci övsün.
\par 3 Tas agirdir, kum bir yüktür, Ama ahmagin kiskirtmasi ikisinden de agirdir.
\par 4 Öfke zalim, hiddet azgindir, Ama kiskançliga kim dayanabilir?
\par 5 Açik bir azar, Gizli tutulan sevgiden iyidir.
\par 6 Düsmanin öpücükleri aldaticidir, Ama dostun seni iyiligin için yaralar.
\par 7 Tok insanin cani bali bile çekmez, Aç kisiye en aci sey tatli gelir.
\par 8 Yuvasindan uzak kalan kus nasilsa, Yurdundan uzak kalan insan da öyledir.
\par 9 Güzel koku ve buhur cani ferahlatir, Dostun verdigi ögüt insana tatli gelir.
\par 10 Kendi dostunu da babanin dostunu da birakma Ve felakete ugradigin gün kardesinin evine gitme; Yakin komsun uzaktaki kardesten yegdir.
\par 11 Oglum, bilgece davran ki yüregim sevinsin, Beni ayiplayana yanit vereyim.
\par 12 Ihtiyatli kisi tehlikeyi görünce saklanir, Bönse öne atilir ve zarar görür.
\par 13 Tanimadigi birine kefil olanin giysisini al; Bir yabanci için yapiyorsa bunu, Giysisini rehin tut.
\par 14 Sabah sabah komsuya verilen gürültülü bir selam Küfür sayilir.
\par 15 Kavgaci kadinin dirdiri Yagmurlu günde damlalarin dinmeyen sesi gibidir.
\par 16 Böyle bir kadini dizginlemeye kalkmak, Rüzgari ya da yagi avuçta tutmaya çalismak gibidir.
\par 17 Demir demiri biler, Insan da insani...
\par 18 Incir agacini budayan meyvesini yer, Efendisine hizmet eden onurlandirilir.
\par 19 Su görüntümüzü nasil yansitiyorsa, Yürek de insanin içini yansitir.
\par 20 Ölüm ve yikim diyari insana doymaz, Insanin gözü de hiç doymaz.
\par 21 Altin ocakta, gümüs potada sinanir, Insansa aldigi övgüyle sinanir.
\par 22 Ahmagi bugdayla birlikte dibekte tokmakla dövsen bile, Ahmakligindan kurtulmaz.
\par 23 Davarina iyi bak, Sigirlarina dikkat et.
\par 24 Çünkü zenginlik kalici degildir Ve taç kusaktan kusaga geçmez.
\par 25 Çayir biçilince, yeni çimen çikinca, Daglardaki otlar toplaninca,
\par 26 Kuzular seni giydirir, Tekeler tarlanin bedeli olur.
\par 27 Keçilerin sütü yalniz seni degil, Ev halkini, hizmetçilerini de doyurmaya yeter.

\chapter{28}

\par 1 Kötü kisi kendisini kovalayan olmasa bile kaçar, Dogrularsa genç aslan gibi yüreklidir.
\par 2 Ayaklanan ülke çok basli olur, Ama akilli, bilgili kisi düzeni saglar.
\par 3 Yoksulu ezen yoksul, Ürünü harap eden saganak yagmur gibidir.
\par 4 Yasayi terk eden kötüyü över, Yerine getirense kötüye karsi çikar.
\par 5 Kötüler adaletten anlamaz, RAB'be yönelenlerse her yönüyle anlar.
\par 6 Dürüst bir yoksul olmak, Yolsuzlukla zengin olmaktan yegdir.
\par 7 Kutsal Yasa'yi yerine getiren çocuk akillidir, Oburlarla arkadaslik edense babasini utandirir.
\par 8 Faiz ve tefecilikle malina mal katan kisi, Bunu yoksullara aciyan için biriktirir.
\par 9 Yasaya kulagini tikayanin Duasi da igrençtir.
\par 10 Dürüst kisileri kötü yola saptiran Kendi kazdigi çukura düser. Iyiligi, özü sözü bir olanlar miras alacak.
\par 11 Zengin kendini bilge sanir, Ama akilli yoksul onun içini okur.
\par 12 Dogrularin zaferi coskuyla kutlanir, Ama kötüler egemen olunca insan kaçacak yer arar.
\par 13 Günahlarini gizleyen basarili olmaz, Itiraf edip birakansa merhamet bulur.
\par 14 Günahtan çekinen ne mutludur! Inatçilik edense belaya düser.
\par 15 Yoksul halki yöneten kötü kisi Kükreyen aslan, saldirgan ayi gibidir.
\par 16 Gaddar önderin akli kittir; Haksiz kazançtan nefret edense uzun ömürlü olur.
\par 17 Adam öldürmekten vicdan azabi çeken, mezara dek kaçacaktir; Kimse ona yardim etmesin.
\par 18 Alni ak yasayan kurtulur, Yolsuzluk yapan ansizin yikima ugrar.
\par 19 Topragini isleyenin ekmegi bol olur, Hayal pesinde kosansa yoksulluga doyar.
\par 20 Güvenilir kisi bolluga erer, Zengin olmaya can atansa beladan kurtulamaz.
\par 21 Hatir gözetmek iyi degildir, Çünkü insan bir lokma ekmek için bile suç isler.
\par 22 Cimri servet pesinde kosar, Yoksulluga ugrayacagini düsünmez.
\par 23 Baskasini azarlayan sonunda Pohpohlayandan daha çok begeni kazanir.
\par 24 Annesini ya da babasini soymayi günah saymayan, Haydutla birdir.
\par 25 Açgözlü kavga çikarir, RAB'be güvenense bolluk içinde yasar.
\par 26 Kendine güvenen akilsizdir, Bilgece davranan güvenlikte olur.
\par 27 Yoksula verenin eksigi olmaz, Yoksulu görmezden gelense bir sürü lanete ugrar.
\par 28 Kötüler egemen olunca insan kaçacak yer arar, Ama kötüler yok olunca dogrular çogalir.

\chapter{29}

\par 1 Defalarca azarlandigi halde dikbaslilik eden, Ansizin yikima ugrayacak, çare yok.
\par 2 Dogru kisiler çogalinca halk sevinir, Kötü kisi hükümdar olunca halk inler.
\par 3 Bilgeligi seven babasini sevindirir, Fahiselerle dostluk eden malini yitirir.
\par 4 Adaletle yöneten kral ülkesini ayakta tutar, Agir vergiler koyansa çökertir.
\par 5 Baskasini pohpohlayan kisi, Ona tuzak kurar.
\par 6 Kötünün baskaldirisi kendine tuzak olur, Dogru kisiyse ezgi söyler ve sevinir.
\par 7 Dogru kisi yoksullarin hakkini verir, Kötü kisi hak hukuk nedir bilmez.
\par 8 Alayci kisiler kentleri bile karistirir, Bilgelerse öfkeyi yatistirir.
\par 9 Bilge kisiyle davasi olan ahmak Kizar, alay eder ve rahat vermez.
\par 10 Kana susamislar dürüst kisiden nefret eder, Dogrularsa onun canini korur.
\par 11 Akilsiz hep patlamaya hazirdir, Bilgeyse öfkesini dizginler.
\par 12 Hükümdar yalana kulak verirse, Bütün görevlileri de kötü olur.
\par 13 Zorbayla yoksulun ortak bir noktasi var: Ikisinin de gözünü açan RAB'dir.
\par 14 Yoksullari adaletle yöneten kralin Tahti hep güvenlikte olur.
\par 15 Degnekle terbiye bilgelik kazandirir, Kendi haline birakilan çocuksa annesini utandirir.
\par 16 Kötüler çogalinca baskaldiri da çogalir, Ama dogrular onlarin düsüsünü görecektir.
\par 17 Oglunu terbiye et, o da sana huzur verecek Ve gönlünü hosnut edecektir.
\par 18 Tanrisal esinden yoksun olan halk Sinir tanimaz olur. Ne mutlu Kutsal Yasa'yi yerine getirene!
\par 19 Köle salt sözle terbiye edilemez, Çünkü anlasa da kulak asmaz.
\par 20 Sözünü tartmadan konusan birini taniyor musun? Akilsizin durumu bile onunkinden daha umut vericidir.
\par 21 Çocuklugundan beri kölesini simartan, Sonunda cezasini çeker.
\par 22 Öfkeli kisi çekisme yaratir, Huysuz kisinin baskaldirisi eksik olmaz.
\par 23 Kibir insani küçük düsürür, Alçakgönüllülükse onur kazandirir.
\par 24 Hirsizla ortak olanin düsmani kendisidir, Mahkemede yemin etse de bildigini söylemez.
\par 25 Insandan korkmak tuzaktir, Ama RAB'be güvenen güvenlikte olur.
\par 26 Hükümdarin gözüne girmek isteyen çoktur, Ama RAB'dir insana adalet saglayan.
\par 27 Dogrular haksizlardan igrenir, Kötüler de dürüst yasayanlardan.

\chapter{30}

\par 1 Massali Yake oglu Agur'un sözleri: Bu adam söyle diyor: "Yoruldum, ey Tanrim, yoruldum ve tükendim.
\par 2 Gerçekten ben insanlarin en cahiliyim, Bende insan akli yok.
\par 3 Bilgeligi ögrenmedim, Kutsal Olan'a iliskin bilgiden de yoksunum.
\par 4 Kim göklere çikip indi? Kim yeli avuçlarinda topladi? Sulari giysisiyle sarip sarmalayan kim? Kim belirledi dünyanin sinirlarini? Adi nedir, oglunun adi nedir, biliyorsan söyle!
\par 5 Tanri'nin her sözü güvenilirdir, O kendisine siginan herkese kalkandir.
\par 6 O'nun sözüne bir sey katma, Yoksa seni azarlar, yalanci çikarsin.
\par 7 Ey Tanri, iki sey diledim senden: Ben ölmeden bunlari esirgeme benden.
\par 8 Sahtekârligi, yalani benden uzak tut, Bana ne yoksulluk ne de zenginlik ver; Payima düsen ekmegi ver, yeter.
\par 9 Yoksa bolluktan, `Kimmis RAB?' diye seni yadsir, Ya da yoksulluktan çalar Ve Tanrim'in adini lekelemis olurum.
\par 10 "Köleyi efendisine çekistirme, Yoksa sana lanet eder, sen de suçlu çikarsin.
\par 11 Öyleleri var ki, babalarina lanet eder, Annelerine deger vermezler.
\par 12 Öyleleri var ki, kendilerini tertemiz sanirlar, Oysa kötülüklerinden arinmis degiller.
\par 13 Öyleleri var ki, kendilerinden üstün kimse yok sanir, Herkese tepeden bakarlar.
\par 14 Öyleleri var ki, disleri kiliç, çeneleri biçaktir, Mazlumlarla yoksullari yutup yeryüzünden yok ederler.
\par 15 Sülügün iki kizi vardir, adlari `Ver, ver'dir. Hiç doymayan üç sey, `Yeter' demeyen dört sey vardir:
\par 16 Ölüler diyari, kisir rahim, Suya doymayan toprak ve `Yeter' demeyen ates.
\par 17 Babasiyla alay edenin, annesinin sözünü hor görenin Gözünü vadideki kargalar oyacak; O akbabalara yem olacak.
\par 18 Aklimin ermedigi üç sey, Anlamadigim dört sey var:
\par 19 Kartalin gökyüzünde, Yilanin kayada, Geminin denizde izledigi yol Ve erkegin genç kizla tuttugu yol.
\par 20 Zina eden kadinin yolu da söyledir: Yer, agzini siler, Sonra da, `Suç islemedim' der.
\par 21 Yeryüzü üç seyin altinda sarsilir; Katlanamadigi dört sey vardir:
\par 22 Kölenin kral olmasi, Budalanin doymasi,
\par 23 Nefret edilen kadinin evlenmesi Ve hizmetçinin haniminin yerine geçmesi.
\par 24 "Dünyada dört küçük yaratik var ki, Çok bilgece davranirlar:
\par 25 Karincalar güçlü olmayan bir topluluktur, Ama yiyeceklerini yazdan biriktirirler.
\par 26 Kaya tavsanlari* da güçsüz bir topluluktur, Ama yuvalarini kaya kovuklarinda yaparlar.
\par 27 Çekirgelerin krali yoktur, Ama bölük bölük ilerlerler.
\par 28 Kertenkele elle bile yakalanir, Ama kral saraylarinda bulunur.
\par 29 "Yürüyüsü gösterisli üç yaratik, Davranisi gösterisli dört yaratik var:
\par 30 Hayvanlarin en güçlüsü olan Ve hiçbir seyin önünde pes etmeyen aslan,
\par 31 Tazi, teke Ve ordusunun basindaki kral.
\par 32 "Eger budala gibi kendini yücelttinse Ya da kötülük tasarladinsa, Dur ve düsün!
\par 33 Çünkü nasil sütü dövünce tereyagi, Burnu sikinca kan çikarsa, Öfkeyi kurcalayinca da kavga çikar."

\chapter{31}

\par 1 Massa Krali Lemuel'in sözleri, Annesinin ona ögrettikleri:
\par 2 "Oglum, rahmimin ürünü, ne diyeyim? Adaklarimin yaniti oglum, ne diyeyim?
\par 3 Gücünü kadinlara, Gençligini krallari mahvedenlere kaptirma!
\par 4 "Sarap içmek krallara yakismaz, ey Lemuel, Krallara yakismaz! Içkiyi özlemek hükümdarlara yarasmaz.
\par 5 Çünkü içince kurallari unutur, Mazlumun hakkini yerler.
\par 6 Içkiyi çaresize, Sarabi kaygi çekene verin.
\par 7 Içsin ki yoksullugunu unutsun, Artik sefaletini anmasin.
\par 8 Agzini hakkini savunamayan için, Kimsesizin davasini gütmek için aç.
\par 9 Agzini aç ve adaletle yargila, Mazlumun, yoksulun hakkini savun." Erdemli Kadin*fu*
\par 10 Erdemli kadini kim bulabilir? Onun degeri mücevherden çok üstündür.
\par 11 Kocasi ona yürekten güvenir Ve kazanci eksilmez.
\par 12 Kadin ona kötülükle degil, Yasami boyunca iyilikle karsilik verir.
\par 13 Yün, keten bulur, Zevkle elleriyle isler.
\par 14 Ticaret gemileri gibidir, Yiyecegini uzaktan getirir.
\par 15 Gün agarmadan kalkar, Ev halkina yiyecek, hizmetçilerine paylarini verir.
\par 16 Bir tarlayi gözüne kestirip satin alir, El emegiyle kazandigi parayla bag diker.
\par 17 Giyinip kollarini sivar, Canla basla çalisir.
\par 18 Ticaretinin kârli oldugunu bilir, Çirasi gece boyunca yanar.
\par 19 Eliyle örekeyi tutar, Avucunda igi tutar.
\par 20 Mazluma kollarini açar, Yoksula elini uzatir.
\par 21 Kar yaginca ev halki için kaygilanmaz, Çünkü hepsinin iki kat giysisi vardir.
\par 22 Yatak örtüleri dokur, Kendi giysileri ince mor ketendendir.
\par 23 Kocasi ülkenin ileri gelenleriyle oturup kalkar, Kent kurulunda iyi taninir.
\par 24 Kadin diktigi keten giysilerle Ördügü kusaklari tüccara satar.
\par 25 Güç ve onurla kusanmistir, Gelecege güvenle bakar.
\par 26 Agzindan bilgelik akar, Dili iyilik ögütler.
\par 27 Ev halkinin islerini yönetir, Tembellik nedir bilmez.
\par 28 Çocuklari önünde ayaga kalkip onu kutlar, Kocasi onu över.
\par 29 "Soylu isler yapan çok kadin var, Ama sen hepsinden üstünsün" der.
\par 30 Çekicilik aldatici, güzellik bostur; Ama RAB'be saygili kadin övülmeye layiktir.
\par 31 Ellerinin hak ettigini verin kendisine, Yaptiklari için kent kurulunda övülsün.


\end{document}