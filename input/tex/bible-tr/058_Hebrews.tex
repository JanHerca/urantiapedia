\begin{document}

\title{İbraniler}


\chapter{1}

\par 1 Tanri eski zamanlarda peygamberler araciligiyla birçok kez çesitli yollardan atalarimiza seslendi.
\par 2 Bu son çagda da her seye mirasçi kildigi ve araciligiyla evreni yarattigi kendi Oglu'yla bize seslenmistir.
\par 3 Ogul, Tanri yüceliginin pariltisi, O'nun varliginin öz görünümüdür. Güçlü sözüyle her seyi devam ettirir. Günahlardan arinmayi sagladiktan sonra, yücelerde ulu Tanri'nin saginda oturdu.
\par 4 Meleklerden ne denli üstün bir adi miras aldiysa, onlardan o denli üstün oldu.
\par 5 Çünkü Tanri meleklerin herhangi birine, "Sen benim Oglum'sun, Bugün ben sana Baba oldum" Ya da, "Ben O'na Baba olacagim, O da bana Ogul olacak"demis midir?
\par 6 Yine Tanri ilk dogani dünyaya gönderirken diyor ki, "Tanri'nin bütün melekleri O'na tapinsin."
\par 7 Melekler için, "Kendi meleklerini rüzgar, Hizmetkârlarini ates alevi yapar" diyor.
\par 8 Ama Ogul için söyle diyor: "Ey Tanri, tahtin sonsuzluklar boyunca kalicidir, Egemenliginin asasi adalet asasidir.
\par 9 Dogrulugu sevdin, kötülükten nefret ettin. Bunun için Tanri, senin Tanrin, Seni sevinç yagiyla Arkadaslarindan daha çok meshetti*."
\par 10 Yine diyor ki, "Ya Rab, baslangiçta Dünyanin temellerini sen attin. Gökler de senin ellerinin yapitidir.
\par 11 Onlar yok olacak, ama sen kalicisin. Hepsi bir giysi gibi eskiyecek.
\par 12 Bir kaftan gibi düreceksin onlari, Bir giysi gibi degistirilecekler. Ama sen hep aynisin, Yillarin tükenmeyecek."
\par 13 Tanri meleklerin herhangi birine, "Ben düsmanlarini Ayaklarinin altina serinceye dek, Sagimda otur" demis midir?
\par 14 Bütün melekler kurtulusu miras alacaklara hizmet etmek için gönderilen görevli ruhlar degil midir?

\chapter{2}

\par 1 Bu nedenle, akintiya kapilip sürüklenmemek için isittiklerimizi daha çok önemsemeliyiz.
\par 2 Çünkü melekler araciligiyla bildirilen söz geçerli olduysa, her suç ve her sözdinlemezlik hak ettigi karsiligi aldiysa, bu denli büyük kurtulusu görmezlikten gelirsek nasil kurtulabiliriz? Baslangiçta Rab tarafindan bildirilen bu kurtulus, Rab'bi dinlemis olanlarca bize dogrulandi.
\par 4 Tanri da belirtiler, harikalar, çesitli mucizeler ve kendi istegi uyarinca dagittigi Kutsal Ruh armaganlariyla buna taniklik etti.Izleyicileriyle Özdeslesen Isa
\par 5 Tanri, sözünü ettigimiz gelecek dünyayi meleklere bagli kilmadi.
\par 6 Ama biri bir yerde söyle taniklik etmistir: "Ya Rab, insan ne ki, onu anasin, Ya da insanoglu ne ki, ona ilgi gösteresin?
\par 7 Onu meleklerden biraz asagi kildin, Basina yücelik ve onur tacini koydun, Ellerinin yapitlari üzerine onu görevlendirdin.
\par 8 Her seyi ayaklari altina sererek Ona bagimli kildin." Tanri her seyi insana bagimli kilmakla insana bagimli olmayan hiçbir sey birakmadi. Ne var ki, her seyin insana bagimli kilindigini henüz görmüyoruz.
\par 9 Ama meleklerden biraz asagi kilinmis olan Isa'yi, Tanri'nin lütfuyla herkes için ölümü tatsin diye çektigi ölüm acisi sonucunda yücelik ve onur taci giydirilmis olarak görüyoruz.
\par 10 Birçok ogulu yücelige eristirirken onlarin kurtulus öncüsünü acilarla yetkinlige erdirmesi, her seyi kendisi için ve kendi araciligiyla var eden Tanri'ya uygun düsüyordu.
\par 11 Çünkü hepsi -kutsal kilan da kutsal kilinanlar da- ayni Baba'dandir. Bunun içindir ki, Isa onlara "kardeslerim" demekten utanmiyor.
\par 12 "Adini kardeslerime duyuracagim, Toplulugun ortasinda Seni ilahilerle övecegim" diyor.
\par 13 Yine, "Ben O'na güvenecegim" ve yine, "Iste ben ve Tanri'nin bana verdigi çocuklar" diyor.
\par 14 Bu çocuklar etten ve kandan olduklari için Isa, ölüm gücüne sahip olani, yani Iblis'i, ölüm araciligiyla etkisiz kilmak üzere onlarla ayni insan yapisini aldi.
\par 15 Bunu, ölüm korkusu yüzünden yasamlari boyunca köle olanlarin hepsini özgür kilmak için yapti.
\par 16 Kuskusuz O, meleklere degil, Ibrahim'in soyundan olanlara yardim ediyor.
\par 17 Bunun için her yönden kardeslerine benzemesi gerekiyordu. Öyle ki, Tanri'ya hizmetinde merhametli ve sadik bir baskâhin olup halkin günahlarini bagislatabilsin.
\par 18 Çünkü kendisi denenip aci çektigi için denenenlere yardim edebilir.

\chapter{3}

\par 1 Bunun için, göksel çagriya ortak olan kutsal kardeslerim, dikkatinizi açikça benimsedigimiz inancin elçisi ve baskâhini Isa'ya çevirin.
\par 2 Musa Tanri'nin bütün evinde Tanri'ya nasil sadik kaldiysa, Isa da kendisini görevlendirene sadiktir.
\par 3 Evi yapan nasil evden daha çok saygi görürse, Isa da Musa'dan daha büyük yücelige layik sayildi.
\par 4 Her evin bir yapicisi vardir, her seyin yapicisi ise Tanri'dir.
\par 5 Musa, gelecekte söylenecek sözlere taniklik etmek için Tanri'nin bütün evinde bir hizmetkâr olarak sadik kaldi.
\par 6 Oysa Mesih, O'nun evi üzerinde yetkili ogul olarak sadiktir. Eger cesaretimizi ve övündügümüz umudu gevsemeden sonuna dek sürdürürsek, O'nun evi biziz.
\par 7 Bu nedenle, Kutsal Ruh'un dedigi gibi, "Bugün O'nun sesini duyarsaniz, Atalarinizin baskaldirdigi, Çölde O'nu sinadigi günkü gibi Yüreklerinizi nasirlastirmayin.
\par 9 Atalariniz beni orada sinayip denediler Ve kirk yil boyunca yaptiklarimi gördüler.
\par 10 Bu nedenle o kusaga darildim Ve dedim ki, 'Yürekleri hep kötüye sapar, Yollarimi ögrenmediler.
\par 11 Öfkelendigimde ant içtigim gibi, Onlar huzur diyarima asla girmeyecekler.'"
\par 12 Ey kardesler, hiçbirinizde diri Tanri'yi terk eden kötü, imansiz bir yüregin bulunmamasina dikkat edin.
\par 13 "Gün bugündür" denildikçe birbirinizi her gün yüreklendirin. Öyle ki, hiçbirinizin yüregi günahin aldaticiligiyla nasirlasmasin.
\par 14 Çünkü Mesih'e ortak olduk. Yalniz baslangiçtaki güvenimizi gevsemeden sonuna dek sürdürmeliyiz.
\par 15 Yukarida belirtildigi gibi, "Bugün O'nun sesini duyarsaniz, Atalarinizin baskaldirdigi günkü gibi Yüreklerinizi nasirlastirmayin."
\par 16 O'nun sesini isitip baskaldiran kimlerdi? Musa önderliginde Misir'dan çikanlarin hepsi degil mi?
\par 17 Tanri kimlere kirk yil dargin kaldi? Günah isleyip cesetleri çöle serilenlere degil mi?
\par 18 Sözünü dinlemeyenler disinda kendi huzur diyarina kimlerin girmeyecegine ant içti?
\par 19 Görüyoruz ki, imansizliklarindan ötürü oraya giremediler.

\chapter{4}

\par 1 Bu nedenle Tanri'nin huzur diyarina girme vaadi hâlâ geçerliyken, herhangi birinizin buna erismemis sayilmasindan korkalim.
\par 2 Çünkü onlar gibi biz de iyi haberi aldik. Ama onlar duyduklari sözü imanla birlestirmedikleri için bunun kendilerine bir yarari olmadi.
\par 3 Biz inanmis olanlar huzur diyarina gireriz. Nitekim Tanri söyle demistir: "Öfkelendigimde ant içtigim gibi, Onlar huzur diyarima asla girmeyecekler." Oysa Tanri dünyanin kurulusundan beri islerini tamamlamistir.
\par 4 Çünkü bir yerde yedinci günle ilgili sunu demistir: "Tanri bütün islerinden yedinci gün dinlendi."
\par 5 Bu konuda yine diyor ki, "Onlar huzur diyarima asla girmeyecekler."
\par 6 Demek ki, bazilarinin huzur diyarina girecegi kesindir. Daha önce iyi haberi almis olanlar söz dinlemedikleri için o diyara giremediler.
\par 7 Bu yüzden Tanri, uzun zaman sonra Davut'un araciligiyla, "bugün" diyerek yine bir gün belirliyor. Daha önce denildigi gibi, "Bugün O'nun sesini duyarsaniz, Yüreklerinizi nasirlastirmayin."
\par 8 Eger Yesu onlari huzura kavustursaydi, Tanri daha sonra bir baska günden söz etmezdi.
\par 9 Böylece Tanri halki için bir Sabat Günü* rahati kaliyor.
\par 10 Tanri islerinden nasil dinlendiyse, O'nun huzur diyarina giren de kendi islerinden öylece dinlenir.
\par 11 Bu nedenle o huzur diyarina girmeye gayret edelim; öyle ki, hiçbirimiz ayni tür sözdinlemezlikten ötürü düsmesin.
\par 12 Tanri'nin sözü diri ve etkilidir, iki agizli kiliçtan daha keskindir. Canla ruhu, ilikle eklemleri birbirinden ayiracak kadar derinlere isler; yüregin düsüncelerini, amaçlarini yargilar.
\par 13 Tanri'nin görmedigi hiçbir yaratik yoktur. Kendisine hesap verecegimiz Tanri'nin gözü önünde her sey çiplak ve açiktir.
\par 14 Tanri Oglu Isa gökleri asan büyük baskâhinimiz oldugu için açikça benimsedigimiz inanca simsiki sarilalim.
\par 15 Çünkü baskâhinimiz zayifliklarimizda bize yakinlik duyamayan biri degildir; tersine, her alanda bizim gibi denenmis, ama günah islememistir.
\par 16 Onun için Tanri'nin lütuf tahtina cesaretle yaklasalim; öyle ki, yardim gereksindigimizde merhamet görelim ve lütuf bulalim.

\chapter{5}

\par 1 Insanlar arasindan seçilen her baskâhin, günahlara karsilik sunular ve kurbanlar sunmak üzere Tanri'yla ilgili konularda insanlari temsil etmek için atanir.
\par 2 Bilgisizlere, yoldan sapanlara yumusak davranabilir. Çünkü kendisi de zayifliklarla kusatilmistir.
\par 3 Bundan ötürü, halk için oldugu gibi, kendisi için de günah sunusu* sunmak zorundadir.
\par 4 Kimse baskâhin olma onurunu kendi kendine alamaz; ancak Harun gibi, Tanri tarafindan çagrilirsa alir.
\par 5 Nitekim Mesih de baskâhin olmak için kendi kendini yüceltmedi. O'na, "Sen benim Oglum'sun, Bugün ben sana Baba oldum" diyen Tanri O'nu yüceltti.
\par 6 Baska bir yerde de diyor ki, "Melkisedek düzeni uyarinca Sen sonsuza dek kâhinsin."
\par 7 Mesih, yeryüzünde oldugu günlerde kendisini ölümden kurtaracak güçte olan Tanri'ya büyük feryat ve gözyaslariyla dua etti, yakardi ve Tanri korkusu nedeniyle isitildi.
\par 8 Ogul oldugu halde, çektigi acilarla söz dinlemeyi ögrendi.
\par 9 Yetkin kilininca, sözünü dinleyen herkes için sonsuz kurtulus kaynagi oldu.
\par 10 Çünkü Tanri tarafindan Melkisedek düzeni uyarinca baskâhin atanmisti.
\par 11 Bu konuda söyleyecek çok sözümüz var, ama kulaklariniz uyustugu için anlatmak zor.
\par 12 Simdiye dek ögretmen olmaniz gerekirken, Tanri sözlerinin temel ilkelerini size yeni bastan ögretecek birine ihtiyaciniz var. Size yine süt gerekli, kati yiyecek degil!
\par 13 Sütle beslenen herkes bebektir ve dogruluk sözünde deneyimsizdir.
\par 14 Kati yiyecek, yetiskinler içindir; onlar duyularini iyi ile kötüyü ayirt etmek üzere alistirmayla egitmis kisilerdir.

\chapter{6}

\par 1 Bunun için, ölü islerden tövbe etmenin ve Tanri'ya inanmanin temelini, vaftizler*, elle kutsama, ölülerin dirilisi ve sonsuz yargiyla ilgili ögretinin temelini yeni bastan atmadan Mesih'le ilgili ilk ögretileri asarak yetkinlige dogru ilerleyelim.
\par 3 Tanri izin verirse, bunu yapacagiz.
\par 4 Bir kez aydinlatilmis, göksel armagani tatmis ve Kutsal Ruh'a ortak edilmis, Tanri sözünün iyiligini ve gelecek çagin güçlerini tatmis olduklari halde yoldan sapanlari yeniden tövbe edecek duruma getirmeye olanak yoktur. Çünkü onlar Tanri'nin Oglu'nu adeta yeniden çarmiha geriyor, herkesin önünde asagiliyorlar.
\par 7 Üzerine sik sik yagan yagmuru emen ve kimler için isleniyorsa onlara yararli bitkiler üreten topragi Tanri bereketli kilar.
\par 8 Ama dikenli bitki, devedikeni üreten toprak yararsizdir; lanetlenmeye yakindir, sonu yanmaktir.
\par 9 Size gelince, sevgili kardesler, böyle konustugumuz halde, durumunuzun daha iyi olduguna, kurtulusa uygun düstügüne eminiz.
\par 10 Tanri adaletsiz degildir; emeginizi ve kutsallara hizmet etmis olarak ve etmeye devam ederek O'nun adina gösterdiginiz sevgiyi unutmaz.
\par 11 Umudunuzdan dogan tam güvenceye kavusmaniz için her birinizin sona dek ayni gayreti göstermesini diliyoruz.
\par 12 Tembel olmamanizi, vaat edilenleri iman ve sabir araciligiyla miras alanlarin örnegine uymanizi istiyoruz.
\par 13 Tanri Ibrahim'e vaatte bulundugu zaman, üzerine ant içecek daha üstün biri olmadigi için kendi üzerine ant içerek söyle dedi:
\par 14 "Seni kutsadikça kutsayacagim, Soyunu çogalttikça çogaltacagim."
\par 15 Böylece Ibrahim sabirla dayanarak vaade eristi.
\par 16 Insanlar kendilerinden üstün biri üzerine ant içerler. Onlar için ant, söyleneni dogrular ve her tartismayi sona erdirir.
\par 17 Tanri da amacinin degismezligini vaadin mirasçilarina daha açikça belirtmek istedigi için vaadini antla pekistirdi.
\par 18 Öyle ki, önümüze konan umuda tutunmak için Tanri'ya siginan bizler, Tanri'nin yalan söylemesi olanaksiz olan bu iki degismez sey araciligiyla büyük cesaret bulalim.
\par 19 Canlarimiz için gemi demiri gibi saglam ve güvenilir olan bu umut, perdenin* arkasindaki iç bölmeye geçer.
\par 20 Melkisedek düzeni uyarinca sonsuza dek baskâhin olan Isa oraya ugrumuza öncü olarak girdi.

\chapter{7}

\par 1 Bu Melkisedek, Salem Krali ve yüce Tanri'nin kâhiniydi*. Krallari bozguna ugratmaktan dönen Ibrahim'i karsilamis ve onu kutsamisti.
\par 2 Ibrahim de ona her seyin ondaligini verdi. Melkisedek, adinin anlamina göre, önce "Dogruluk Krali"dir; sonra da "Salem Krali", yani "Esenlik Krali"dir.
\par 3 Babasiz, annesizdir; soyagaci yoktur. Ne günlerinin baslangici, ne yasaminin sonu vardir. Tanri'nin Oglu gibi sonsuza dek kâhin kalacaktir.
\par 4 Bakin, büyük ata Ibrahim'in ganimetten ondalik verdigi bu adam ne kadar büyüktür!
\par 5 Leviogullari'ndan olup kâhinlik görevini üstlenenlere Kutsal Yasa* uyarinca halktan, yani Ibrahim'in soyundan olduklari halde, kardeslerinden ondalik almalari buyrulmustur.
\par 6 Melkisedek ise Levili* kâhinlerin soyundan olmadigi halde, vaatleri alan Ibrahim'den ondalik kabul etmis ve onu kutsamistir.
\par 7 Hiç kuskusuz, kutsayan kutsanandan üstündür.
\par 8 Birinde ölümlü insanlar ondalik aliyor, ötekinde yasadigina taniklik edilen biri aliyor.
\par 9 Ondalik alan Levi bile Ibrahim araciligiyla ondalik vermistir denebilir.
\par 10 Çünkü Melkisedek Ibrahim'i karsiladigi zaman, Levi hâlâ atasinin bedenindeydi.
\par 11 Eger Levililer'in kâhinligi* araciligiyla yetkinlige erisilebilseydi -nitekim Kutsal Yasa bu kâhinligi öngörerek halka verildi- Harun düzenine göre degil de, Melkisedek düzenine göre baska bir kâhinin gelmesinden söz etmeye ne gerek kalirdi?
\par 12 Çünkü kâhinlik degisince, Yasa da zorunlu olarak degisir.
\par 13 Kendisinden böyle söz edilen kisi baska bir oymaktandir. Bu oymaktan hiç kimse sunakta hizmet etmemistir.
\par 14 Rabbimiz'in Yahuda oymagindan geldigi açiktir. Musa bu oymaktan söz ederken kâhinlere iliskin bir sey söylemedi.
\par 15 Melkisedek benzeri baska bir kâhin ortaya çiktigindan, bu söyledigimiz artik daha da açiktir.
\par 16 O, Yasa'nin soyla ilgili önkosuluna göre degil, yok edilemez bir yasamin gücüne göre kâhin olmustur.
\par 17 Çünkü, "Melkisedek düzeni uyarinca Sen sonsuza dek kâhinsin" diye taniklik ediliyor.
\par 18 Önceki buyruk, zayifligi ve yararsizligi nedeniyle geçersiz kilindi.
\par 19 Çünkü Yasa hiçbir seyi yetkinlestiremedi. Bunun yerine, araciligiyla Tanri'ya yaklastigimiz daha saglam bir umut verildi.
\par 20 Bu da antsiz olmadi. Öbürleri ant içilmeden kâhin olmuslardi.
\par 21 Ama O kendisine, "Rab ant içti, kararindan dönmez, Sen sonsuza dek kâhinsin" diyen Tanri'nin andiyla kâhin oldu.
\par 22 Böylece Isa daha iyi bir antlasmanin kefili olmustur.
\par 23 Önceki düzende çok sayida kâhin görev aldi. Çünkü ölüm, görevlerini sürdürmelerini engelliyordu.
\par 24 Ama Isa sonsuza dek yasadigi için kâhinligi süreklidir.
\par 25 Bu nedenle O'nun araciligiyla Tanri'ya yaklasanlari tümüyle kurtaracak güçtedir. Çünkü onlara aracilik etmek için hep yasamaktadir.
\par 26 Böyle bir baskâhinimiz -kutsal, suçsuz, lekesiz, günahkârlardan ayrilmis, göklerden daha yücelere çikarilmis bir baskâhinimiz- olmasi uygundur.
\par 27 O, öbür baskâhinler gibi her gün önce kendi günahlari, sonra da halkin günahlari için kurbanlar sunmak zorunda degildir. Çünkü kendini sunmakla bunu ilk ve son kez yapti.
\par 28 Kutsal Yasa, zayifliklari olan insanlari baskâhin atamaktadir. Ama Yasa'dan sonra gelen ant sözü, sonsuza dek yetkin kilinmis olan Ogul'u baskâhin atamistir.

\chapter{8}

\par 1 Söylediklerimizin özü sudur: Göklerde, Yüce Olan'in tahtinin saginda oturan, kutsal yerde, insanin degil, Rab'bin kurdugu asil tapinma çadirinda görev yapan böyle bir baskâhinimiz vardir.
\par 3 Her baskâhin sunular, kurbanlar sunmak için atanir. Bu nedenle bizim baskâhinimizin de sunacak bir seyi olmasi gerekir.
\par 4 Eger kendisi yeryüzünde olsaydi, kâhin* olamazdi. Çünkü Kutsal Yasa uyarinca sunulari sunanlar var.
\par 5 Bunlar göktekinin örnegi ve gölgesi olan tapinakta hizmet ediyorlar. Nitekim Musa tapinma çadirini kurmak üzereyken Tanri tarafindan söyle uyarildi: "Her seyi sana dagda gösterilen örnege göre yapmaya dikkat et."
\par 6 Simdiyse, Isa daha iyi vaatler üzerine kurulmus daha iyi bir antlasmanin aracisi oldugu kadar, daha üstün bir göreve de sahip olmustur.
\par 7 Eger o ilk antlasma kusursuz olsaydi, ikincisine gerek duyulmazdi.
\par 8 Oysa halkini kusurlu bulan Tanri söyle diyor: "'Israil halkiyla ve Yahuda halkiyla Yeni bir antlasma yapacagim günler geliyor' Diyor Rab.
\par 9 'Atalarini Misir'dan çikarmak için Ellerinden tuttugum gün Onlarla yaptigim antlasmaya benzemeyecek. Çünkü onlar antlasmama bagli kalmadilar, Ben de onlardan yüz çevirdim' Diyor Rab.
\par 10 'O günlerden sonra Israil halkiyla Yapacagim antlasma sudur' diyor Rab, 'Yasalarimi zihinlerine isleyecegim, Yüreklerine yazacagim. Ben onlarin Tanrisi olacagim, Onlar da benim halkim olacak.
\par 11 Hiç kimse yurttasini, kardesini, Rab'bi tani diye egitmeyecek. Çünkü küçük büyük hepsi taniyacak beni.
\par 12 Çünkü suçlarini bagislayacagim, Günahlarini artik anmayacagim.'"
\par 13 Tanri, "Yeni bir antlasma" demekle ilkini eskimis saymistir. Eskiyip köhnelesense çok geçmeden yok olur.

\chapter{9}

\par 1 Ilk antlasmanin tapinma kurallari ve dünyasal tapinagi vardi.
\par 2 Bir çadir kurulmustu. Kutsal Yer* denen birinci bölmede kandillik, masa ve adak ekmekleri* bulunurdu.
\par 3 Ikinci perdenin arkasinda En Kutsal Yer* denen bir bölme vardi.
\par 4 Altin buhur sunagiyla her yani altinla kaplanmis Antlasma Sandigi* buradaydi. Sandigin içinde altindan yapilmis man* testisi, Harun'un filizlenmis degnegi ve antlasma levhalari vardi.
\par 5 Sandigin üstünde Bagislanma Kapagi'ni gölgeleyen yüce Keruvlar dururdu. Ama simdi bunlarin ayrintilarina giremeyiz.
\par 6 Her sey böyle düzenlendikten sonra kâhinler* her zaman çadirin ilk bölmesine girer, tapinma görevlerini yerine getirirler.
\par 7 Ama iç bölmeye yilda bir kez yalniz baskâhin girebilir. Üstelik kendisi için ve halkin bilmeden isledigi suçlar için sunacagi kurban kani olmaksizin giremez.
\par 8 Kutsal Ruh bununla çadirin ilk bölmesi durdukça, kutsal yere giden yolun henüz açikça gösterilmedigini belirtiyor.
\par 9 Bu, simdiki çag için bir örnektir; sunulan kurbanlarla sunularin tapinan kisinin vicdanini yetkinlestiremedigini gösteriyor.
\par 10 Bunlar yalniz yiyecek, içecek, çesitli dinsel yikanmalarla ilgilidir; yeni düzenin baslangicina kadar geçerli olan bedensel kurallardir.
\par 11 Ama Mesih, gelecek iyi seylerin baskâhini olarak ortaya çikti. Insan eliyle yapilmamis, yani bu yaratilistan olmayan daha büyük, daha yetkin çadirdan geçti.
\par 12 Tekelerle danalarin kaniyla degil, sonsuz kurtulusu saglayarak kendi kaniyla kutsal yere ilk ve son kez girdi.
\par 13 Tekelerle bogalarin kani ve serpilen düve külü murdar* olanlari kutsal kiliyor, bedensel açidan temizliyor.
\par 14 Öyleyse sonsuz Ruh araciligiyla kendini lekesiz olarak Tanri'ya sunmus olan Mesih'in kaninin, diri Tanri'ya kulluk edebilmemiz için vicdanimizi ölü islerden temizleyecegi ne kadar daha kesindir!
\par 15 Bu nedenle, çagrilmis olanlarin vaat edilen sonsuz mirasi almalari için Mesih yeni antlasmanin aracisi oldu. Kendisi onlari ilk antlasma*fx1* zamaninda isledikleri suçlardan kurtarmak için fidye olarak öldü.
\par 16 Ortada bir vasiyet*fx1* varsa, vasiyet edenin ölümünün kanitlanmasi gerekir.
\par 17 Çünkü vasiyet ancak ölümden sonra geçerli olur. Vasiyet eden yasadikça, vasiyetin hiçbir etkinligi yoktur.
\par 18 Bu nedenle ilk antlasma bile kan akitilmadan yürürlüge girmedi.
\par 19 Musa, Kutsal Yasa'nin her buyrugunu bütün halka bildirdikten sonra su, al yapagi, mercanköskotu ile danalarin ve tekelerin kanini alip hem kitabin hem de bütün halkin üzerine serpti.
\par 20 "Tanri'nin uymanizi buyurdugu antlasmanin kani budur" dedi.
\par 21 Ayni biçimde çadirin ve tapinmada kullanilan bütün esyalarin üzerine kan serpti.
\par 22 Nitekim Kutsal Yasa uyarinca hemen her sey kanla temiz kilinir, kan dökülmeden bagislama olmaz.
\par 23 Böylelikle asli göklerde olan örneklerin bu kurbanlarla, ama gökteki asillarinin bunlardan daha iyi kurbanlarla temiz kilinmasi gerekti.
\par 24 Çünkü Mesih, asil kutsal yerin örnegi olup insan eliyle yapilan kutsal yere degil, ama simdi bizim için Tanri'nin önünde görünmek üzere asil göge girdi.
\par 25 Baskâhin her yil kendisinin olmayan kanla En Kutsal Yer'e* girer; oysa Mesih kendisini tekrar tekrar sunmak için göge girmedi.
\par 26 Öyle olsaydi, dünyanin kurulusundan beri Mesih'in tekrar tekrar aci çekmesi gerekirdi. Oysa Mesih, kendisini bir kez kurban ederek günahi ortadan kaldirmak için çaglarin sonunda ortaya çikmistir.
\par 27 Bir kez ölmek, sonra da yargilanmak nasil insanlarin kaderiyse, Mesih de birçoklarinin günahlarini yüklenmek için bir kez kurban edildi. Ikinci kez, günah yüklenmek için degil, kurtulus getirmek için kendisini bekleyenlere görünecektir.

\chapter{10}

\par 1 Kutsal Yasa'da gelecek iyi seylerin asli yoktur, sadece gölgesi vardir. Bu nedenle Yasa, her yil sürekli ayni kurbanlari sunarak Tanri'ya yaklasanlari asla yetkinlige erdiremez.
\par 2 Erdirebilseydi, kurban sunmaya son verilmez miydi? Çünkü tapinanlar bir kez günahlarindan arindiktan sonra artik günahlilik duygusu kalmazdi.
\par 3 Ancak o kurbanlar insanlara yildan yila günahlarini animsatiyor.
\par 4 Çünkü bogalarla tekelerin kani günahlari ortadan kaldiramaz.
\par 5 Bunun için Mesih dünyaya gelirken söyle diyor: "Kurban ve sunu istemedin, Ama bana bir beden hazirladin.
\par 6 Yakmalik sunudan* ve günah sunusundan* Hosnut olmadin.
\par 7 O zaman söyle dedim: 'Kutsal Yazi tomarinda Benim için yazildigi gibi, Senin istegini yapmak üzere, Ey Tanri, iste geldim.'"
\par 8 Mesih ilkin, "Kurban, sunu, yakmalik sunu, günah sunusu istemedin ve bunlardan hosnut olmadin" dedi. Oysa bunlar Yasa'nin bir geregi olarak sunulur.
\par 9 Sonra, "Senin istegini yapmak üzere iste geldim" dedi. Yani ikinciyi geçerli kilmak için birinciyi ortadan kaldiriyor.
\par 10 Tanri'nin bu istegi uyarinca, Isa Mesih'in bedeninin ilk ve son kez sunulmasiyla kutsal kilindik.
\par 11 Her kâhin* her gün ayakta durup görevini yapar ve günahlari asla ortadan kaldiramayan ayni kurbanlari tekrar tekrar sunar.
\par 12 Oysa Mesih günahlar için sonsuza dek geçerli tek bir kurban sunduktan sonra Tanri'nin saginda oturdu.
\par 13 O zamandan beri düsmanlarinin, kendi ayaklarinin altina serilmesini bekliyor.
\par 14 Çünkü kutsal kilinanlari tek bir sunuyla sonsuza dek yetkinlige erdirmistir.
\par 15 Kutsal Ruh da bu konuda bize taniklik ediyor. Önce diyor ki,
\par 16 "Rab, 'O günlerden sonra Onlarla yapacagim antlasma sudur: Yasalarimi yüreklerine koyacagim, Zihinlerine yazacagim' diyor."
\par 17 Sonra sunu ekliyor: "Onlarin günahlarini ve suçlarini artik anmayacagim."
\par 18 Bunlarin bagislanmasi durumunda artik günah için sunuya gerek yoktur.
\par 19 Bu nedenle, ey kardesler, Isa'nin kani sayesinde perdede*, yani kendi bedeninde bize açtigi yeni ve diri yoldan kutsal yere girmeye cesaretimiz vardir.
\par 21 Tanri'nin evinden sorumlu büyük bir kâhinimiz* bulunmaktadir.
\par 22 .22 Öyleyse yüreklerimiz serpmeyle kötü vicdandan arinmis, bedenlerimiz temiz suyla yikanmis olarak, imanin verdigi tam güvenceyle, yürekten bir içtenlikle Tanri'ya yaklasalim.
\par 23 Açikça benimsedigimiz umuda simsiki tutunalim. Çünkü vaat eden Tanri güvenilirdir.
\par 24 Birbirimizi sevgi ve iyi isler için nasil gayrete getirebilecegimizi düsünelim.
\par 25 Bazilarinin alistigi gibi, bir araya gelmekten vazgeçmeyelim; o günün yaklastigini gördükçe birbirimizi daha da çok yüreklendirelim.
\par 26 Gerçegi ögrenip benimsedikten sonra, bile bile günah islemeye devam edersek, günahlar için artik kurban kalmaz; geriye sadece yarginin dehsetli beklenisi ve düsmanlari yiyip bitirecek kizgin ates kalir.
\par 28 Musa'nin Yasasi'ni hiçe sayan, iki ya da üç tanigin sözüyle acimasizca öldürülür.
\par 29 Eger bir kimse Tanri Oglu'nu ayaklar altina alir, kendisini kutsal kilan antlasma kanini bayagi sayar ve lütufkâr Ruh'a hakaret ederse, bundan ne kadar daha agir bir cezaya layik görülecek sanirsiniz?
\par 30 Çünkü, "Öç benimdir, karsiligini ben verecegim" ve yine, "Rab halkini yargilayacak" diyeni taniyoruz.
\par 31 Diri Tanri'nin eline düsmek korkunç bir seydir.
\par 32 Sizlerse aydinlandiktan sonra acilarla dolu büyük bir mücadeleye dayandiginiz o ilk günleri animsayin.
\par 33 Bazen sitemlere, sikintilara ugrayip seyirlik oldunuz, bazen de ayni durumda olanlarla dayanisma içine girdiniz.
\par 34 Hem hapistekilerin dertlerine ortak oldunuz, hem de daha iyi ve kalici bir maliniz oldugunu bilerek mallarinizin yagma edilmesini sevinçle karsiladiniz.
\par 35 Onun için cesaretinizi yitirmeyin; bu cesaretin ödülü büyüktür.
\par 36 Çünkü Tanri'nin istegini yerine getirmek ve vaat edilene kavusmak için dayanma gücüne ihtiyaciniz vardir.
\par 37 Artik, "Gelecek olan pek yakinda gelecek Ve gecikmeyecek.
\par 38 Dogru adamim, imanla yasayacaktir. Ama geri çekilirse, ondan hosnut olmayacagim."
\par 39 Bizler geri çekilip mahvolanlardan degiliz; iman edip canlarinin kurtulusuna kavusanlardaniz.

\chapter{11}

\par 1 Iman, umut edilenlere güvenmek, görünmeyen seylerin varligindan emin olmaktir.
\par 2 Atalarimiz bununla Tanri'nin begenisini kazandilar.
\par 3 Evrenin Tanri'nin buyruguyla yaratildigini, böylece görülenlerin görünmeyenlerden olustugunu iman sayesinde anliyoruz.
\par 4 Habil'in Tanri'ya Kayin'den daha iyi bir kurban sunmasi iman sayesinde oldu.Imani sayesinde dogru biri olarak Tanri'nin begenisini kazandi. Çünkü Tanri onun sundugu adaklari kabul etti. Nitekim Habil ölmüs oldugu halde, iman sayesinde hâlâ konusmaktadir.
\par 5 Iman sayesinde Hanok ölümü tatmamak üzere yukari alindi. Kimse onu bulamadi, çünkü Tanri onu yukari almisti. Yukari alinmadan önce Tanri'yi hosnut eden biri olduguna taniklik edildi.
\par 6 Iman olmadan Tanri'yi hosnut etmek olanaksizdir. Tanri'ya yaklasan, O'nun var olduguna ve kendisini arayanlari ödüllendirecegine iman etmelidir.
\par 7 Iman sayesinde Nuh, henüz olmamis olaylarla ilgili olarak Tanri tarafindan uyarilinca, Tanri korkusuyla ev halkinin kurtulusu için bir gemi yapti. Bununla dünyayi yargiladi ve imana dayanan dogrulugun mirasçisi oldu.
\par 8 Iman sayesinde Ibrahim miras alacagi yere gitmesi için çagrilinca, Tanri'nin sözünü dinledi ve nereye gidecegini bilmeden yola çikti.
\par 9 Iman sayesinde bir yabanci olarak vaat edilen ülkeye yerlesti. Ayni vaadin ortak mirasçilari olan Ishak ve Yakup'la birlikte çadirlarda yasadi.
\par 10 Çünkü mimari ve kurucusu Tanri olan temelli kenti bekliyordu.
\par 11 Iman sayesinde Sara'nin kendisi de kisir ve yasi geçmis oldugu halde vaat edeni güvenilir saydigindan çocuk sahibi olmak için güç buldu.
\par 12 Böylece tek bir adamdan, üstelik ölüden farksiz birinden gökteki yildizlar, deniz kiyisindaki kum kadar sayisiz torun meydana geldi.
\par 13 Bu kisilerin hepsi imanli olarak öldüler. Vaat edilenlere kavusamadilarsa da bunlari uzaktan görüp selamladilar, yeryüzünde yabanci ve konuk olduklarini açikça kabul ettiler.
\par 14 Böyle konusanlar bir vatan aradiklarini gösteriyorlar.
\par 15 Ayrildiklari ülkeyi düsünselerdi, geri dönmeye firsatlari olurdu.
\par 16 Ama onlar daha iyisini, yani göksel olani arzu ediyorlardi. Bunun içindir ki, Tanri onlarin Tanrisi olarak anilmaktan utanmiyor. Çünkü onlara bir kent hazirladi.
\par 17 Ibrahim sinandigi zaman imanla Ishak'i kurban olarak sundu. Vaatleri almis olan Ibrahim biricik oglunu kurban etmek üzereydi.
\par 18 Oysa Tanri ona, "Senin soyun Ishak'la sürecek" demisti.
\par 19 Ibrahim Tanri'nin ölüleri bile diriltebilecegini düsündü; nitekim Ishak'i simgesel sekilde ölümden geri aldi.
\par 20 Iman sayesinde Ishak gelecek olaylarla ilgili olarak Yakup'la Esav'i kutsadi.
\par 21 Yakup ölürken iman sayesinde Yusuf'un iki oglunu da kutsadi, degneginin ucuna yaslanarak Tanri'ya tapindi.
\par 22 Yusuf ölürken iman sayesinde Israilogullari'nin Misir'dan çikacagini animsatti ve kemiklerine iliskin buyruk verdi.
\par 23 Musa dogdugunda annesiyle babasi onu imanla üç ay gizlediler. Çünkü çocugun güzel oldugunu gördüler ve kralin fermanindan korkmadilar.
\par 24 Musa büyüyünce iman sayesinde firavunun kizinin oglu olarak taninmayi reddetti.
\par 25 Bir süre için günahin sefasini sürmektense, Tanri'nin halkiyla birlikte baski görmeyi yegledi.
\par 26 Mesih ugruna asagilanmayi Misir hazinelerinden daha büyük zenginlik saydi. Çünkü alacagi ödülü düsünüyordu.
\par 27 Kralin öfkesinden korkmadan imanla Misir'dan ayrildi. Görünmez Olan'i görür gibi dayandi.
\par 28 Ilk doganlari öldüren melek Israilliler'e dokunmasin diye Musa imanla, Fisih* kurbaninin kesilmesini ve kaninin kapilara sürülmesini sagladi.
\par 29 Iman sayesinde Israilliler karadan geçer gibi Kizildeniz'den* geçtiler. Misirlilar bunu deneyince boguldular.
\par 30 Israilliler yedi gün boyunca Eriha surlari çevresinde dolandilar; sonunda imanlari sayesinde surlar yikildi.
\par 31 Fahise Rahav casuslari dostça karsiladigi için imani sayesinde söz dinlemeyenlerle birlikte öldürülmedi.
\par 32 Daha ne diyeyim? Gidyon, Barak, Simson, Yiftah, Davut, Samuel ve peygamberlerle ilgili olanlari anlatsam, zaman yetmeyecek.
\par 33 Bunlar iman sayesinde ülkeler ele geçirdiler, adaleti sagladilar, vaat edilenlere kavustular, aslanlarin agzini kapadilar.
\par 34 Kizgin atesi söndürdüler, kilicin agzindan kaçip kurtuldular. Güçsüzlükte kuvvet buldular, savasta güçlendiler, yabanci ordulari bozguna ugrattilar.
\par 35 Kadinlar dirilen ölülerini geri aldilar. Baskalariysa saliverilmeyi reddederek dirilip daha iyi bir yasama kavusma umuduyla iskencelere katlandilar.
\par 36 Daha baskalari alaya alinip kamçilandi, hatta zincire vurulup hapsedildi.
\par 37 Taslandilar, testereyle biçildiler, kiliçtan geçirilip öldürüldüler. Koyun postu, keçi derisi içinde dolastilar, yoksulluk çektiler, sikintilara ugradilar, baski gördüler.
\par 38 Dünya onlara layik degildi. Çöllerde, daglarda, magaralarda, yeralti oyuklarinda dolanip durdular.
\par 39 Imanlari sayesinde bunlarin hepsi Tanri'nin begenisini kazandiklari halde, hiçbiri vaat edilene kavusmadi.
\par 40 Bizden ayri olarak yetkinlige ermesinler diye, Tanri bizim için daha iyi bir sey hazirlamisti.

\chapter{12}

\par 1 Iste çevremizi bu denli büyük bir taniklar bulutu sardigina göre, biz de her yükü ve bizi kolayca kusatan günahi üzerimizden siyirip atalim ve önümüze konan yarisi sabirla kosalim.
\par 2 Gözümüzü imanimizin öncüsü ve tamamlayicisi Isa'ya dikelim. O kendisini bekleyen sevinç ugruna utanci hiçe sayip çarmihta ölüme katlandi ve Tanri'nin tahtinin saginda oturdu.
\par 3 Yorulup cesaretinizi yitirmemek için, günahkârlarin bunca karsi koymasina katlanmis Olan'i düsünün.
\par 4 Günaha karsi verdiginiz mücadelede henüz kaninizi akitacak kadar direnmis degilsiniz.
\par 5 Size ogullar diye seslenen su ögüdü de unuttunuz: "Oglum, Rab'bin terbiye edisini hafife alma, Rab seni azarlayinca cesaretini yitirme.
\par 6 Çünkü Rab sevdigini terbiye eder, Ogulluga kabul ettigi herkesi cezalandirir."
\par 7 Terbiye edilmek ugruna acilara katlanmalisiniz. Tanri size ogullarina davranir gibi davraniyor. Hangi ogul babasi tarafindan terbiye edilmez?
\par 8 Herkesin gördügü terbiyeden yoksunsaniz, ogullar degil, yasadisi evlatlarsiniz.
\par 9 Kaldi ki, bizi terbiye eden dünyasal babalarimiz vardi ve onlara saygi duyardik. Öyleyse Ruhlar Babasi'na bagimli olup yasamamiz çok daha önemli degil mi?
\par 10 Babalarimiz bizi kisa bir süre için, uygun gördükleri gibi terbiye ettiler. Ama Tanri, kutsalligina ortak olalim diye bizi kendi yararimiza terbiye ediyor.
\par 11 Terbiye edilmek baslangiçta hiç tatli gelmez, aci gelir. Ne var ki, böyle egitilenler için bu sonradan esenlik veren dogrulugu üretir.
\par 12 Bunun için sarkik ellerinizi kaldirin, bükük dizlerinizi dogrultun, ayaklariniz için düz yollar yapin. Öyle ki, kötürüm olan parça eklemden çikmasin, tersine sifa bulsun.
\par 14 Herkesle baris içinde yasamaya, kutsal olmaya gayret edin. Kutsalliga sahip olmadan kimse Rab'bi göremeyecek.
\par 15 Dikkat edin, kimse Tanri'nin lütfundan yoksun kalmasin. Içinizde sizi rahatsiz edecek ve birçoklarini zehirleyecek aci bir kök filizlenmesin.
\par 16 Kimse fuhus yapmasin ya da ilk ogulluk hakkini bir yemege karsilik satan Esav gibi kutsal degerlere saygisizlik etmesin.
\par 17 Biliyorsunuz, Esav daha sonra kutsanma hakkini miras almak istediyse de geri çevrildi. Kutsanmak için gözyasi döküp yalvarmasina karsin, vermis oldugu kararin sonucunu degistiremedi.
\par 18 Sizler dokunulabilen, alev alev yanan daga, karanliga, koyu karanlik ve kasirgaya, gürleyen çagri borusuna, tanrisal sözleri ileten sese yaklasmis degilsiniz. O sesi isitenler, kendilerine bir sözcük daha söylenmesin diye yalvardilar.
\par 20 "Daga bir hayvan bile dokunsa taslanacak" buyruguna dayanamadilar.
\par 21 Görünüm öyle korkunçtu ki, Musa, "Çok korkuyorum, titriyorum" dedi.
\par 22 Oysa sizler Siyon* Dagi'na, yasayan Tanri'nin kenti olan göksel Yerusalim'e, bir bayram senligi içindeki onbinlerce melege, adlari göklerde yazilmis ilk doganlarin topluluguna yaklastiniz. Herkesin yargici olan Tanri'ya, yetkinlige erdirilmis dogru kisilerin ruhlarina, yeni antlasmanin aracisi olan Isa'ya ve Habil'in kanindan daha üstün bir anlam tasiyan serpmelik kana yaklastiniz.
\par 25 Bunlari söyleyeni reddetmemeye dikkat edin. Çünkü yeryüzünde kendilerini uyarani reddedenler kurtulamadilarsa, göklerden bizi uyarandan yüz çevirirsek, bizim de kurtulamayacagimiz çok daha kesindir.
\par 26 O zaman O'nun sesi yeri sarsmisti. Ama simdi, "Bir kez daha yalniz yeri degil, gögü de sarsacagim" diye söz vermistir.
\par 27 "Bir kez daha" sözü, sarsilanlarin, yani yaratilmis olan seylerin ortadan kaldirilacagini, böylelikle sarsilmayanlarin kalacagini anlatiyor.
\par 28 Böylece sarsilmaz bir egemenlige kavustugumuz için minnettar olalim. Öyle ki, Tanri'yi hosnut edecek biçimde saygi ve korkuyla tapinalim.
\par 29 Çünkü Tanrimiz yakip yok eden bir atestir.

\chapter{13}

\par 1 Kardes sevgisi sürekli olsun.
\par 2 Konuksever olmaktan geri kalmayin. Çünkü bu sayede bazilari bilmeden melekleri konuk ettiler.
\par 3 Hapiste olanlari, onlarla birlikte hapsedilmis gibi animsayin. Sizin de bir bedeniniz oldugunu düsünerek baski görenleri hatirlayin.
\par 4 Herkes evlilige saygi göstersin. Evlilik yatagi günahla lekelenmesin. Çünkü Tanri fuhus yapanlari, zina edenleri yargilayacak.
\par 5 Yasayisiniz para sevgisinden uzak olsun. Sahip olduklarinizla yetinin. Çünkü Tanri söyle dedi: "Seni asla terk etmeyecegim, Seni asla yüzüstü birakmayacagim."
\par 6 Böylece cesaretle diyoruz ki, "Rab benim yardimcimdir, korkmam; Insan bana ne yapabilir?"
\par 7 Tanri'nin sözünü size iletmis olan önderlerinizi animsayin. Yasayislarinin sonucuna bakarak onlarin imanini örnek alin.
\par 8 Isa Mesih dün, bugün ve sonsuza dek aynidir.
\par 9 Çesitli garip ögretilerin etkisine kapilip sürüklenmeyin. Yüregin yiyeceklerle degil, Tanri lütfuyla güçlenmesi iyidir. Yiyeceklere güvenenler hiçbir yarar görmediler.
\par 10 Bir sunagimiz var ki, tapinma çadirinda hizmet edenlerin ondan yemeye haklari yoktur.
\par 11 Baskâhin günah sunusu* olarak hayvanlarin kanini kutsal yere tasir, ama bu hayvanlarin cesetleri ordugahin disinda yakilir.
\par 12 Bunun gibi, Isa da kendi kaniyla halki kutsal kilmak için kent kapisinin disinda aci çekti.
\par 13 Öyleyse biz de O'nun ugradigi asagilanmaya katlanarak ordugahtan disariya çikip yanina gidelim.
\par 14 Çünkü burada kalici bir kentimiz yoktur, biz gelecekteki kenti özlüyoruz.
\par 15 Bu nedenle, Isa araciligiyla Tanri'ya sürekli övgü kurbanlari, yani O'nun adini açikça anan dudaklarin meyvesini sunalim.
\par 16 Iyilik yapmayi, sizde olani baskalariyla paylasmayi unutmayin. Çünkü Tanri bu tür kurbanlardan hosnut olur.
\par 17 Önderlerinizin sözünü dinleyin, onlara bagli kalin. Çünkü onlar canlariniz için hesap verecek kisiler olarak sizi kollarlar. Onlarin sözünü dinleyin ki, görevlerini inleyerek degil -bunun size yarari olmaz- sevinçle yapsinlar.
\par 18 Bizim için dua edin. Vicdanimizi temiz tuttugumuza, her bakimdan olumlu bir yasam sürmek istedigimize eminiz.
\par 19 Yaniniza tez zamanda dönebilmem için dua etmenizi özellikle rica ediyorum.
\par 20 Esenlik veren Tanri, koyunlarin büyük Çobani'ni, Rabbimiz Isa'yi sonsuza dek sürecek antlasmanin kaniyla ölümden diriltti.
\par 21 Tanri, istegini yerine getirebilmeniz için sizi her iyilikle donatsin; kendisini hosnut eden seyi Isa Mesih araciligiyla bizlerde gerçeklestirsin. Mesih'e sonsuzlara dek yücelik olsun! Amin.
\par 22 Kardesler, size rica ediyorum, ögütlerimi hos görün. Zaten size kisaca yazdim.
\par 23 Kardesimiz Timoteos'un saliverildigini bilmenizi istiyorum. Yakinda yanima gelirse, onunla birlikte sizi görmeye gelecegim.
\par 24 Önderlerinizin hepsine ve bütün kutsallara selam söyleyin. Italya'dan olanlar size selam ederler.
\par 25 Tanri'nin lütfu hepinizle birlikte olsun! Amin.


\end{document}