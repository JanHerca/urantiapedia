\begin{document}

\title{Koloseliler}


\chapter{1}

\par 1 Tanri'nin istegiyle Mesih Isa'nin elçisi atanan ben Pavlus ve kardesimiz Timoteos'tan, Kolose'de bulunan, Mesih'e ait kutsal ve sadik kardeslere selam! Babamiz Tanri'dan sizlere lütuf ve esenlik olsun.
\par 3 Sizler için dua ederken Tanri'ya, Rabbimiz Isa Mesih'in Babasi'na her zaman sükrediyoruz.
\par 4 Çünkü Mesih Isa'ya iman ettiginizi ve bütün kutsallari sevdiginizi duyduk.
\par 5 Imaniniz ve sevginiz göklerde sizin için sakli bulunan umuttan kaynaklaniyor. Bu umudun haberini gerçegin bildirisinden, size daha önce ulasan Müjde'den aldiniz. Müjde, onu isittiginiz ve Tanri'nin lütfunu gerçekten anladiginiz günden beri aranizda oldugu gibi, bütün dünyada da meyve vermekte, yayilmaktadir.
\par 7 Müjde'yi bizim adimiza Mesih'in güvenilir hizmetkâri olan sevgili emektasimiz Epafras'tan ögrendiniz.
\par 8 Ruh'tan kaynaklanan sevginizi de bize o bildirdi.
\par 9 Bunu isittigimiz günden beri biz de sizler için dua etmekten, tam bir bilgelik ve ruhsal anlayisla Tanri'nin istegini bütünüyle bilmenizi dilemekten geri kalmadik.
\par 10 Rab'be yarasir biçimde yasamaniz, O'nu her yönden hosnut etmeniz, her iyi iste meyve vererek Tanri'yi tanimakta ilerlemeniz için dua ediyoruz.
\par 11 Her seye sevinçle katlanip sabredebilmeniz için O'nun yüce gücüne dayanarak bütün kudretle güçlenmenizi diliyoruz.
\par 12 Bizi kutsallarin isiktaki mirasina ortak olmaya yeterli kilan Baba'ya sükretmeniz için dua ediyoruz.
\par 13 O bizi karanligin hükümranligindan kurtarip sevgili Oglu'nun egemenligine aktardi.
\par 14 O'nda kurtulusa, günahlarimizin bagisina sahibiz.
\par 15 Görünmez Tanri'nin görünümü, bütün yaratilisin ilk dogani O'dur.
\par 16 Nitekim yerde ve gökte, görünen ve görünmeyen her sey -tahtlar, egemenlikler, yönetimler, hükümranliklar- O'nda yaratildi. Her sey O'nun araciligiyla ve O'nun için yaratildi.
\par 17 Her seyden önce var olan O'dur ve her sey varligini O'nda sürdürmektedir.
\par 18 Bedenin, yani kilisenin* basi O'dur. Her seyde ilk yeri alsin diye baslangiç olan ve ölüler arasindan ilk dogan O'dur.
\par 19 Çünkü Tanri bütün dolulugunun O'nda bulunmasini uygun gördü.
\par 20 Mesih'in çarmihta akitilan kani araciligiyla esenligi saglamis olarak yerdeki ve gökteki her seyi O'nun araciligiyla kendisiyle baristirmaya razi oldu.
\par 21 Yaptiginiz kötülükler yüzünden bir zamanlar düsüncelerinizde Tanri'ya yabanci ve düsmandiniz.
\par 22 Simdiyse Mesih sizi Tanri'nin önüne kutsal, lekesiz ve kusursuz olarak çikarmak için öz bedeninin ölümü sayesinde sizi Tanri'yla baristirdi.
\par 23 Yeter ki, duydugunuz Müjde'nin verdigi umuttan kopmadan, imanda temellenip yerlesmis olarak kalin. Ben Pavlus, gögün altindaki bütün yaratilisa duyurulan bu Müjde'nin hizmetkâri oldum.
\par 24 Sizin için aci çektigime simdi seviniyorum. Mesih'in, kendi bedeni, yani kilise* ugruna çektigi sikintilardan eksik kalanlarini kendi bedenimde tamamliyorum.
\par 25 Tanri'nin sizin yarariniza bana verdigi görevle kilisenin hizmetkâri oldum. Görevim, Tanri'nin sözünü, yani geçmis çaglardan ve kusaklardan gizlenmis, ama simdi O'nun kutsallarina açiklanmis olan sirri eksiksiz duyurmaktir.
\par 27 Tanri kutsallarina bu sirrin uluslar arasinda ne denli yüce ve zengin oldugunu bildirmek istedi. Bu sirrin özü sudur: Mesih içinizdedir. Bu da size yücelige kavusma umudunu veriyor.
\par 28 Her insani Mesih'te yetkinlesmis olarak Tanri'ya sunmak için herkesi uyararak ve herkesi tam bir bilgelikle egiterek Mesih'i tanitiyoruz.
\par 29 O'nun kudretle bende etkin olan gücüne dayanarak ugrasip emek vermemin amaci da budur.

\chapter{2}

\par 1 Gerek sizler, gerek Laodikya'dakiler, gerekse sizler gibi yüzümü hiç görmemis olanlar için ne denli büyük bir ugras verdigimi bilmenizi isterim.
\par 2 Yüreklerinin cesaret bulmasini, sevgide birlesmelerini dilerim. Öyle ki, anlayisin verdigi tam güvenligin bütün zenginligine kavussunlar ve Tanri'nin sirrini, yani bilginin ve bilgeligin bütün hazinelerinin sakli oldugu Mesih'i tanisinlar.
\par 4 Kimse sizi kulagi oksayan sözlerle aldatmasin diye söylüyorum bunu.
\par 5 Çünkü her ne kadar bedence aranizda degilsem de, ruhça sizinle birlikteyim. Düzenliliginizi, Mesih'e imaninizin saglamligini görüp seviniyorum.
\par 6 Bu nedenle Rab Mesih Isa'yi nasil kabul ettinizse, O'nda öylece yasayin.
\par 7 Sükranla dolup tasarak O'nda köklenin ve gelisin, size ögretildigi gibi imanda güçlenin.
\par 8 Dikkatli olun! Mesih'e degil de, insanlarin gelenegine, dünyanin temel ilkelerine dayanan felsefeyle, bos ve aldatici sözlerle kimse sizi tutsak etmesin.
\par 9 Çünkü Tanriligin bütün dolulugu bedence Mesih'te bulunuyor.
\par 10 Siz de her yönetim ve hükümranligin basi olan Mesih'te doluluga kavustunuz.
\par 11 Ayrica Mesih'in gerçeklestirdigi sünnet sayesinde bedenin benliginden soyunarak elle yapilmayan sünnetle O'nda sünnet edildiniz.
\par 12 Vaftizde* O'nunla birlikte gömüldünüz, O'nu ölümden dirilten Tanri'nin gücüne iman ederek O'nunla birlikte dirildiniz.
\par 13 Sizler suçlariniz ve benliginizin sünnetsizligi yüzünden ölüyken, Tanri sizi Mesih'le birlikte yasama kavusturdu. Bütün suçlarimizi O bagisladi.
\par 14 Kurallariyla bize karsi ve aleyhimizde olan yazili antlasmayi sildi, onu çarmiha çakarak ortadan kaldirdi.
\par 15 Yönetimlerin ve hükümranliklarin elindeki silahlari alip onlari çarmihta yenerek açikça gözler önüne serdi.
\par 16 Bu nedenle kimse yiyecek içecek, bayram, yeni ay ya da Sabat Günü* konusunda sizi yargilamasin.
\par 17 Bunlar gelecek seylerin gölgesidir, asli ise Mesih'tedir.
\par 18 Sözde alçakgönüllülükte ve meleklere tapinmakta direnen, gördügü düslerin üzerinde durarak benligin düsünceleriyle bos yere böbürlenen, Bas'a tutunmayan hiç kimse sizi ödülünüzden yoksun birakmasin. Bütün beden eklemler ve baglar yardimiyla bu Bas'tan beslenip bütünlenmekte, Tanri'nin sagladigi büyümeyle gelismektedir.
\par 20 Mesih'le birlikte ölüp dünyanin temel ilkelerinden kurtuldugunuza göre, niçin dünyada yasayanlar gibi, "Sunu elleme", "Bunu tatma", "Suna dokunma" gibi kurallara uyuyorsunuz?
\par 22 Bu kurallarin hepsi, kullanildikça yok olacak nesnelerle ilgilidir; insanlarin buyruklarina, ögretilerine dayanir.
\par 23 Kuskusuz bu kurallarin gönüllü tapinma, sözde alçakgönüllülük, bedene eziyet açisindan bilgece bir görünüsü vardir; ama benligin tutkularini denetlemekte hiçbir yararlari yoktur.

\chapter{3}

\par 1 Mesih'le birlikte dirildiginize göre, gökteki degerlerin ardindan gidin. Mesih orada, Tanri'nin saginda oturuyor.
\par 2 Yeryüzündeki degil, gökteki degerleri düsünün.
\par 3 Çünkü siz öldünüz, yasaminiz Mesih'le birlikte Tanri'da saklidir.
\par 4 Yasaminiz olan Mesih göründügü zaman, siz de O'nunla birlikte yücelmis olarak görüneceksiniz.
\par 5 Bu nedenle bedenin dünyasal egilimlerini fuhsu, pisligi, sehveti, kötü arzulari ve putperestlikle es olan açgözlülügü öldürün.
\par 6 Bunlar yüzünden Tanri'nin gazabi söz dinlemeyenlerin üzerine geliyor.
\par 7 Geçmiste bunlarla iç içe yasadiginiz zaman siz de bu yollarda yürüdünüz.
\par 8 Ama simdi öfke, kizginlik, kötü niyet dahil, hepsini üzerinizden siyirip atin. Agzinizdan hiçbir iftira ya da edepsiz söz çikmasin.
\par 9 Birbirinize yalan söylemeyin. Çünkü eski yaradilisi kötü aliskanliklariyla birlikte üzerinizden çikarip attiniz;
\par 10 eksiksiz bilgiye erismek için Yaraticisi'na benzer olmak üzere yenilenen yeni yaradilisi giyindiniz.
\par 11 Bu yenilikte Grek* ve Yahudi, sünnetli ve sünnetsiz, barbar, Iskit, köle ve özgür ayrimi yoktur. Mesih her seydir ve her seydedir.
\par 12 Öyleyse, Tanri'nin kutsal ve sevgili seçilmisleri olarak yürekten sevecenligi, iyiligi, alçakgönüllülügü, sabri, yumusakligi giyinin.
\par 13 Birbirinize hosgörülü davranin. Birinizin ötekinden bir sikâyeti varsa, Rab'bin sizi bagisladigi gibi, siz de birbirinizi bagislayin.
\par 14 Bunlarin hepsinin üzerine yetkin birligin bagi olan sevgiyi giyinin.
\par 15 Mesih'in esenligi yüreklerinizde hakem olsun. Tek bir bedenin üyeleri olarak bu esenlige çagrildiniz. Sükredici olun!
\par 16 Mesih'in sözü bütün zenginligiyle içinizde yasasin. Tam bir bilgelikle birbirinize ögretin, ögüt verin, mezmurlar, ilahiler, ruhsal ezgiler söyleyerek yüreklerinizde sükranla Tanri'ya nagmeler yükseltin.
\par 17 Söylediginiz, yaptiginiz her seyi Rab Isa'nin adiyla, O'nun araciligiyla Baba Tanri'ya sükrederek yapin.
\par 18 Ey kadinlar, Rab'be ait olanlara yarasir biçimde kocalariniza bagimli olun.
\par 19 Ey kocalar, karilarinizi sevin. Onlara sert davranmayin.
\par 20 Ey çocuklar, her konuda anne babalarinizin sözünü dinleyin. Çünkü bu Rab'bi hosnut eder.
\par 21 Ey babalar, çocuklarinizi incitmeyin, yoksa cesaretleri kirilir.
\par 22 Ey köleler, dünyadaki efendilerinizin her sözünü dinleyin. Bunu, yalniz insanlari hosnut etmek isteyenler gibi göze hos görünen hizmetle degil, saf yürekle, Rab korkusuyla yapin.
\par 23 Rab'den miras ödülünü alacaginizi bilerek, her ne yaparsaniz, insanlar için degil, Rab için yapar gibi candan yapin. Rab Mesih'e kulluk ediyorsunuz.
\par 25 Haksizlik eden ettigi haksizligin karsiligini alacak, hiçbir ayrim yapilmayacaktir.

\chapter{4}

\par 1 Ey efendiler, gökte sizin de bir Efendiniz oldugunu bilerek kölelerinize adalet ve esitlikle davranin.
\par 2 Kendinizi duaya verin. Duada uyanik kalin, sükredin.
\par 3 Ayni zamanda bizim için de dua edin ki Tanri, sözünü yaymamiz ve ugruna hapsedildigim Mesih sirrini açiklamamiz için bize bir kapi açsin.
\par 4 Bu sirri gerektigi gibi açiklikla bildirebilmem için dua edin.
\par 5 Sizden olmayanlara karsi bilgece davranin. Firsati degerlendirin.
\par 6 Sözünüz tuzla terbiye edilmis gibi her zaman lütufla dolu olsun. Böylece herkese nasil karsilik vermek gerektigini bileceksiniz.
\par 7 Rab yolunda emektasim ve güvenilir bir hizmetkâr olan sevgili kardesimiz Tihikos, benimle ilgili her seyi size bildirecektir.
\par 8 Iste bu amaçla, durumumuzu iletmesi ve yüreklerinize cesaret vermesi için kendisini size gönderiyorum.
\par 9 Onunla birlikte, sizden biri olan, güvenilir ve sevgili kardes Onisimos'u da gönderiyorum. Burada olup biten her seyi size bildirecekler.
\par 10 Hapishane arkadasim Aristarhus ve Barnaba'nin yegeni Markos size selam ederler. Markos'la ilgili buyruklar aldiniz; yaniniza gelirse kendisini kabul edin.
\par 11 Yustus diye taninan Yesu da size selam eder. Tanri'nin Egemenligi* için çalisan emektasim Yahudiler yalniz bunlardir. Bunlar bana teselli oldular.
\par 12 Sizden biri ve Mesih Isa'nin kulu olan Epafras size selam eder. Tanri'nin her isteginden emin, yetkin kisiler olarak ayakta kalasiniz diye sizin için her zaman duayla mücadele ediyor.
\par 13 Gerek sizin gerekse Laodikya ve Hierapolis'tekiler için çok emek verdigine taniklik ederim.
\par 14 Sevgili hekim Luka'yla Dimas da size selam ederler.
\par 15 Laodikya'daki kardeslere, Nimfa'ya ve evindeki topluluga* selam edin.
\par 16 Bu mektup aranizda okunduktan sonra Laodikya kilisesine* de okutun. Siz de Laodikya'dan gelecek mektubu okuyun.
\par 17 Arhippus'a, "Rab yolunda üstlendigin görevi tamamlamaya dikkat et!" deyin.
\par 18 Ben Pavlus bu selami elimle yaziyorum. Zincire vuruldugumu unutmayin. Tanri'nin lütfu sizinle birlikte olsun.


\end{document}