\begin{document}

\title{Yeşeya}


\chapter{1}

\par 1 Yahuda krallari Uzziya, Yotam, Ahaz ve Hizkiya zamaninda Amots oglu Yesaya'nin Yahuda ve Yerusalim'le ilgili görümü:
\par 2 Ey gökler dinleyin, ey yeryüzü kulak ver! Çünkü RAB konusuyor: "Çocuklar yetistirip büyüttüm, Ama bana baskaldirdilar.
\par 3 Öküz sahibini, esek efendisinin yemligini bilir, Ama Israil halki bu kadarini bile bilmiyor, Halkim anlamiyor."
\par 4 Günahli ulusun, suç yüklü halkin, Kötülük yapan soyun, Bastan çikmis çocuklarin vay haline! RAB'bi terk ettiler, Israil'in Kutsali'ni hor gördüler, O'na sirt çevirdiler.
\par 5 Neden bir daha dövülesiniz? Neden vefasizligi sürdürüyorsunuz? Bas büsbütün hasta, yürek büsbütün yarali.
\par 6 Bedeniniz tepeden tirnaga sagliksiz, Taze darbe izleriyle, yara bereyle dolu, Temizlenmemis, yagla yumusatilmamis, sarilmamis.
\par 7 Ülkeniz issiz, kentleriniz atese verilmis. Yabancilar topraklarinizi Gözünüzün önünde yiyip bitiriyor! Sanki ülkenin kökünü kazimislar.
\par 8 Siyon kizi* bagdaki çardak, Salatalik bostanindaki kulübe gibi, Kusatilmis bir kent gibi kalakalmis.
\par 9 Her Seye Egemen RAB bazilarimizi Sag birakmamis olsaydi, Sodom gibi olur, Gomora'ya benzerdik.
\par 10 Ey Sodom yöneticileri, RAB'bin söylediklerini dinleyin; Ey Gomora halki, Tanrimiz'in yasasina kulak verin.
\par 11 "Kurbanlarinizin sayisi çokmus, Bana ne?" diyor RAB, "Yakmalik koç sunularina*, Besili hayvanlarin yagina doydum. Boga, kuzu, teke kani degil istedigim.
\par 12 Huzuruma geldiginizde Avlularimi çignemenizi mi istedim sizden?
\par 13 Anlamsiz sunular getirmeyin artik. Buhurdan igreniyorum. Kötülük dolu törenlere, Yeni Ay, Sabat Günü* kutlamalarina Ve düzenlediginiz toplantilara dayanamiyorum.
\par 14 Yeni Ay törenlerinizden, bayramlarinizdan nefret ediyorum. Bunlar bana yük oldu, Onlari tasimaktan yoruldum.
\par 15 "Ellerinizi açip bana yakardiginizda Gözlerimi sizden kaçiracagim. Ne kadar çok dua ederseniz edin dinlemeyecegim. Elleriniz kan dolu.
\par 16 Yikanip temizlenin, Kötülük yaptiginizi gözüm görmesin, Kötülük etmekten vazgeçin.
\par 17 Iyilik etmeyi ögrenin, Adaleti gözetin, zorbayi yola getirin, Öksüzün hakkini verin, Dul kadini savunun."
\par 18 RAB diyor ki, "Gelin, simdi davamizi görelim. Günahlariniz sizi kana boyamis bile olsa Kar gibi ak pak olacaksiniz. Elleriniz kirmiz böcegi gibi kizil olsa da Yapagi gibi bembeyaz olacak.
\par 19 Istekli olur, söz dinlerseniz, Ülkenin en iyi ürünlerini yiyeceksiniz.
\par 20 Ama direnip baskaldirirsaniz, Kiliç sizi yiyip bitirecek." Bunu söyleyen RAB'dir.
\par 21 Sadik kent nasil da fahise oldu! Adaletle doluydu, dogrulugun barinagiydi, Simdiyse katillerle doldu.
\par 22 Gümüsü cüruf oldu, Sarabina su katildi.
\par 23 Yöneticileri asilerle hirsizlarin isbirlikçisi; Hepsi rüsveti seviyor, Armagan pesine düsmüs. Öksüzün hakkini vermiyor, Dul kadinin davasini görmüyorlar.
\par 24 Bu yüzden Rab, Her Seye Egemen RAB, Israil'in Güçlüsü söyle diyor: "Hasimlarimi cezalandirip rahata kavusacagim, Düsmanlarimdan öç alacagim.
\par 25 Sana karsi duracak, Kül suyuyla aritir gibi seni cüruftan aritip temizleyecegim.
\par 26 Eskiden, baslangiçta oldugu gibi, Sana yöneticiler, danismanlar yetistirecegim. Ondan sonra 'Dogruluk Kenti, 'Sadik Kent diye adlandirilacaksin."
\par 27 Siyon adalet sayesinde, Tövbe edenleri de dogruluk sayesinde kurtulacak.
\par 28 Ama baskaldiranlarla günahlilar Birlikte yikima ugrayacaklar. RAB'bi terk edenler yok olacak.
\par 29 "Sevip altinda tapindiginiz yabanil fistik agaçlarindan utanacaksiniz, Putperest törenleriniz için seçtiginiz bahçelerden ötürü yüzünüz kizaracak.
\par 30 Yapraklari solmus yabanil fistik agacina, Susuz bahçeye döneceksiniz.
\par 31 Güçlü adamlariniz kitik gibi, Yaptiklari isler kivilcim gibi olacak; Ikisi birlikte yanacak ve söndüren olmayacak."

\chapter{2}

\par 1 Amots oglu Yesaya'nin Yahuda ve Yerusalim'le ilgili görümü:
\par 2 RAB'bin Tapinagi'nin kuruldugu dag, Son günlerde daglarin en yücesi, Tepelerin en yüksegi olacak. Oraya akin edecek uluslarin hepsi.
\par 3 Birçok halk gelecek, "Haydi, RAB'bin Dagi'na, Yakup'un Tanrisi'nin Tapinagi'na çikalim" diyecekler, "O bize kendi yolunu ögretsin, Biz de O'nun yolundan gidelim." Çünkü yasa Siyon'dan, RAB'bin sözü Yerusalim'den çikacak.
\par 4 RAB uluslar arasinda yargiçlik edecek, Birçok halkin arasindaki anlasmazliklari çözecek. Insanlar kiliçlarini çekiçle dövüp saban demiri, Mizraklarini bagci biçagi yapacaklar. Ulus ulusa kiliç kaldirmayacak, Savas egitimi yapmayacaklar artik.
\par 5 Ey Yakup soyu, gelin RAB'bin isiginda yürüyelim.
\par 6 Ya RAB, halkini, Yakup soyunu terk ettin, Çünkü yürekleri dogu kökenli inançlarla dolu. Filistliler gibi falcilikla ugrasiyor, Yabancilarla el sikisiyorlar.
\par 7 Ülkeleri altinla, gümüsle dolu, Hazinelerinin sonu yok, Sayisiz atlari, savas arabalari var.
\par 8 Ülkeleri putlarla dolu; Elleriyle yaptiklari, Parmaklariyla biçim verdikleri Putlarin önünde egiliyorlar.
\par 9 Bu yüzden herkes alçaltilip dize getirilecek. Onlari bagislama, ya RAB!
\par 10 RAB'bin dehsetinden, Yüce görkeminden kaçmak için kayaliklara gidin, Yerin altina saklanin.
\par 11 Insanin küstah bakislari alçaltilacak, Gururu kirilacak. O gün yalniz RAB yüceltilecek.
\par 12 Çünkü Her Seye Egemen RAB o gün Kibirlileri, gururlulari, kendini begenmisleri alçaltacak;
\par 13 Lübnan'in bütün ulu, yüksek sedir agaçlarini, Basan'in bütün meselerini yok edecek;
\par 14 Bütün ulu daglari, yüksek tepeleri,
\par 15 Bütün yüksek kuleleri, güçlü surlari Yerle bir edecek;
\par 16 Ticaret gemilerinin, güzel teknelerinin hepsini yok edecek.
\par 17 Insanlarin gururu, kibiri kirilacak, O gün yalniz RAB yüceltilecek,
\par 18 Putlar tümüyle ortadan kalkacak.
\par 19 RAB kalkip yeryüzünü sarsmaya baslayinca, Insanlar O'nun dehsetinden Ve yüce görkeminden kaçmak için Kayalik magaralara, yeralti kovuklarina saklanacaklar.
\par 20 O gün insanlar Yeryüzünü sarsmak üzere harekete geçen RAB'bin dehsetinden Ve yüce görkeminden kaçmak için Tapmak amaciyla yaptiklari altin, gümüs putlari Köstebeklere, yarasalara atip Kaya kovuklarina, uçurumlardaki yariklara saklanacaklar.
\par 22 Ölümlü insana güvenmekten vazgeçin. Onun ne degeri var ki?

\chapter{3}

\par 1 Bakin, Rab, Her Seye Egemen RAB, Her türlü yardim ve destegi, Yani ekmek ve suyu, Yigitlerle savasçilari, Yöneticilerle peygamberleri, Falcilarla ileri gelenleri, Takim komutanlariyla soylulari, danismanlari, Hünerli büyücülerle bilge muskacilari Yerusalim'den ve Yahuda'dan çekip alacak.
\par 4 Çocuklari onlara yönetici atayacak, Küçük çocuklar onlara egemen olacak.
\par 5 Insan insana, komsu komsuya haksizlik edecek. Genç yasliya, Siradan adam onurlu kisiye Hayasizca davranacak.
\par 6 Ailede bir kardes öbürüne sarilip, "Hiç olmazsa senin bir üstlügün var, Önderimiz ol! Bu yikintilari sen yönet" diyecek.
\par 7 O zaman adam söyle yanitlayacak: "Ben yaranizi saramam. Evimde ne yiyecek ne giyecek var. Beni halkin önderi yapmayin."
\par 8 Yerusalim sendeledi, Yahuda düstü. Çünkü söyledikleri de yaptiklari da RAB'be karsi; O'nun yüce varligini asagiliyor.
\par 9 Yüzlerindeki ifade onlara karsi taniklik ediyor. Sodom gibi günahlarini açikça söylüyor, gizlemiyorlar. Vay onlarin haline! Çünkü bu felaketi baslarina kendileri getirdiler.
\par 10 Dogru kisiye iyilik görecegini söyleyin. Çünkü iyiliklerinin meyvesini yiyecek.
\par 11 Vay kötülerin haline! Kötülük görecek, yaptiklarinin karsiligini alacaklar.
\par 12 Çocuklar halkimi eziyor, Kadinlar onu yönetiyor. Ey halkim, sana yol gösterenler Seni saptiriyor, yolunu sasirtiyorlar.
\par 13 RAB davasini görmek için yerini aldi, Halklari yargilamak için ayaga kalkiyor.
\par 14 RAB halkinin ileri gelenleri ve önderleriyle davasini görecek. Rab, Her Seye Egemen RAB onlara diyor ki, "Baglari yiyip bitiren sizsiniz, Evleriniz yoksullardan zorla aldiginiz malla dolu. Ne hakla halkimi eziyor, Yoksulu sömürüyorsunuz?"
\par 16 RAB söyle diyor: "Siyon kizlari kibirlidir, burunlari bir karis havada, göz kirparak geziyor, ayaklarindaki halhallari singirdatarak kiritiyorlar.
\par 17 Bu yüzden onlarin baslarinda yaralar çikaracagim, mahrem yerlerini açacagim."
\par 18 O gün Rab güzel halhallari, alin çatkilarini, hilalleri, küpeleri, bilezikleri, peçeleri, basliklari, ayak zincirlerini, kusaklari, koku siselerini, muskalari, yüzükleri, burun halkalarini, bayramlik giysileri, pelerinleri, sallari, keseleri, el aynalarini, keten giysileri, bas sargilarini, tülbentleri ortadan kaldiracak.
\par 24 O zaman güzel kokunun yerini pis koku, Kusagin yerini ip, Lüleli saçin yerini kel kafa, Süslü giysinin yerini çul, Güzelligin yerini daglama izi alacak.
\par 25 Erkekleri kiliçtan geçirilecek, Yigitleri savasta yok olacak.
\par 26 Siyon'un kapilari ah çekip yas tutacak; Kent, yerde oturan, Terk edilmis bir kadin gibi olacak.

\chapter{4}

\par 1 O gün yedi kadin bir erkegi tutup, "Kendi yemegimizi de giysimizi de saglariz; yeter ki senin adini alalim. Utancimizi gider!" diyecekler.
\par 2 O gün RAB'bin dali, Israil halkindan sag kalanlarin güzelligi ve görkemi olacak; ülkenin meyvesi de onlarin kivanci ve övüncü olacak.
\par 3 Siyon'da, yani Yerusalim'de sag kalanlara, "Yerusalim'de yasiyor" diye kaydedilenlere, "Kutsal" denilecek.
\par 4 Rab Siyon kizlarini pisliklerinden arindiracak. Yerusalim'de dökülen kani adil ve atesten bir ruhla temizleyecek.
\par 5 Sonra RAB Siyon Dagi'nin her yanini, orada toplananlarin üzerini gündüz bulutla, gece dumanla ve parlak alevle örtecek. Yüceligi onlarin üzerinde bir örtü olacak.
\par 6 Bu, bir çardak, gündüzün sicagina karsi gölge, yagmura, firtinaya karsi siginak ve korunak olacak.

\chapter{5}

\par 1 Sevgilimin bagi için yaktigi ezgiyi sevgilim için okuyayim: Topragi verimli bir tepede Sevgilimin bir bagi vardi.
\par 2 Topragi belleyip taslari ayikladi, Seçme asmalar dikip orta yere bir gözcü kulesi yapti. Üzüm sikmak için bir çukur kazdi Ve baginin üzüm vermesini bekledi. Ama bag yabanil üzüm verdi.
\par 3 Sevgilim diyor ki, "Ey Yerusalim'de yasayanlar ve Yahuda halki, lütfen benimle bagim arasinda hakem olun!
\par 4 Bagim için yapmadigim ne kaldi? Ben üzüm vermesini beklerken niçin yabanil üzüm verdi?
\par 5 Simdi bagima ne yapacagimi size söyleyeyim: Çitini söküp atacagim, varsin yiyip bitirsinler; duvarini yikacagim, varsin çignesinler.
\par 6 Viraneye çevirecegim onu; budanmayacak, çapalanmayacak; dikenli çalilar bitecek her yaninda. Üzerine yagmur yagdirmasinlar diye bulutlara buyruk verecegim."
\par 7 Her Seye Egemen RAB'bin bagi Israil halki, Hoslandigi fidan da Yahuda halkidir. RAB adalet bekledi, Zorbalik gördü; Dogruluk bekledi, Feryatlar duydu.
\par 8 Evlerine ev, tarlalarina tarla katanlarin vay haline! Oturacak yer kalmadi, Ülkede bir tek siz oturuyorsunuz.
\par 9 Her Seye Egemen RAB'bin söyle ant içtigini duydum: "Büyük ve gösterisli çok sayida ev issiz kalacak, Içinde oturan olmayacak.
\par 10 Çünkü on dönümlük bag ancak bir bat sarap, Bir homer tohum ancak bir efa tahil üretecek."
\par 11 Sabah erkenden kalkip içki pesinden kosanlarin, gece geç vakte kadar sarap içip kizisanlarin vay haline!
\par 12 Onlarin sölenlerinde lir, çenk, tef ve kaval çalinir, sarap içilir. Ama RAB'bin yaptiklarina dikkat etmez, ellerinin yapitina aldirmazlar.
\par 13 Halkim bilgisizligi yüzünden sürgün edilecek; saygin kisileri kitliktan ölecek, kalabaliklar susuzluktan kirilacak.
\par 14 Bu yüzden doymak bilmeyen ölüler diyari agzini ardina kadar açti; Yerusalim'in soylulari, siradan insanlari ve gürültülü bir sekilde eglenenleri oraya inecek.
\par 15 Hepsi alçaltilacak; dize getirilecek, küstah bakislari alçaltilacak.
\par 16 Ama Her Seye Egemen RAB adaletinden ötürü yüceltilecek. Kutsal Tanri dogruluguyla kutsal oldugunu gösterecek.
\par 17 Kuzular kendi otlaklarindaymis gibi otlayacak, zenginlerin issiz kalan konutlarini yabancilar ele geçirecek.
\par 18 Suçu yalanla örülmüs iplerle, günahi araba urganiyla çekenlerin vay haline!
\par 19 Diyorlar ki, "Tanri elini çabuk tutup isini hizlandirsin da görelim. Israil'in Kutsali tasarladigini yapsin da görelim."
\par 20 Kötüye iyi, iyiye kötü diyenlerin, karanligi isik, isigi karanlik yerine koyanlarin, aciya tatli, tatliya aci diyenlerin vay haline!
\par 21 Kendilerini bilge görenlerin, akilli sananlarin vay haline!
\par 22 Sarap içmekte sinir tanimayanlarin, içkileri karistirip içmekten çekinmeyenlerin, rüsvet ugruna kötüyü hakli çikaranlarin, haklilarin hakkini elinden alanlarin vay haline!
\par 24 Alev alev yanan ates, samani nasil yiyip bitirirse, kuru ot alevin içinde nasil birden tutusup yok olursa, onlar da kökten çürüyüp gidecek, çiçekleri toz gibi havaya savrulacak. Çünkü Her Seye Egemen RAB'bin yasasini reddettiler, Israil'in Kutsali'nin sözlerini küçümsediler.
\par 25 Bu yüzden RAB'bin halkina karsi öfkesi alevlendi, elini kaldirip onlari vurdu. Daglar titriyor, cesetler çöp gibi sokaklara serildi. Bütün bunlara karsin RAB'bin öfkesi dinmedi, eli hâlâ kalkmis durumda.
\par 26 RAB uzaktaki uluslari bir sancak isaretiyle, dünyanin en uzagindakileri islik sesiyle çagiracak; hizla, hemen gelecekler.
\par 27 Aralarinda yorulan, sendeleyen olmayacak; uyuklamayacak, uyumayacaklar. Gevsek kemer, kopuk çarik bagi olmayacak.
\par 28 Oklari sivri, yaylari kuruludur. Atlarinin toynaklari çakmaktasi, arabalarinin tekerlekleri kasirga gibidir.
\par 29 Askerleri disi aslan gibi, genç aslanlar gibi kükrüyor, homurdanarak avlarini kapip götürüyorlar. Kimse avlarini pençelerinden kurtaramiyor.
\par 30 O gün Israil'e karsi denizin gürleyisi gibi gürleyecekler. Karaya bakan biri karanlik ve sikinti görecek. Isik karanlik bulutlarla kaplanacak.

\chapter{6}

\par 1 Kral Uzziya'nin öldügü yil yüce ve görkemli Rab'bi gördüm; tahtta oturuyordu, giysisinin etekleri tapinagi dolduruyordu.
\par 2 Üzerinde Seraflar duruyordu; her birinin alti kanadi vardi; ikisiyle yüzlerini, ikisiyle ayaklarini örtüyor, öbür ikisiyle de uçuyorlardi.
\par 3 Birbirlerine söyle sesleniyorlardi: "Her Seye Egemen RAB Kutsal, kutsal, kutsaldir. Yüceligi bütün dünyayi dolduruyor."
\par 4 Seraflar'in sesinden kapi söveleriyle esikler sarsildi, tapinak dumanla doldu.
\par 5 "Vay basima! Mahvoldum" dedim, "Çünkü dudaklari kirli bir adamim, dudaklari kirli bir halkin arasinda yasiyorum. Buna karsin Kral'i, Her Seye Egemen RAB'bi gözlerimle gördüm."
\par 6 Seraflar'dan biri bana dogru uçtu, elinde sunaktan masayla aldigi bir kor vardi;
\par 7 onunla agzima dokunarak, "Iste bu kor dudaklarina degdi, suçun silindi, günahin bagislandi" dedi.
\par 8 Sonra Rab'bin sesini isittim: "Kimi göndereyim? Bizim için kim gidecek?" diyordu. "Ben! Beni gönder" dedim.
\par 9 "Git, bu halka sunu duyur" dedi, "'Duyacak duyacak, ama anlamayacaksiniz, Bakacak bakacak, ama görmeyeceksiniz!
\par 10 Bu halkin yüregini duygusuzlastir, Kulaklarini agirlastir, Gözlerini kapat. Öyle ki, gözleri görmesin, Kulaklari duymasin, yürekleri anlamasin Ve bana dönüp sifa bulmasinlar."
\par 11 "Ne vakte kadar, ya Rab?" diye sordum. Rab yanitladi: "Kentler viraneye dönüp kimsesiz kalincaya, Evler ipissiz oluncaya, Toprak büsbütün kiraçlasincaya kadar.
\par 12 Insanlari çok uzaklara sürecegim, Ülke bombos kalacak,
\par 13 Halkin onda biri kalsa da ülke mahvolacak. Ama devrildigi zaman kütügü kalan Yabanil fistik ve mese agaci gibi, Kutsal soy kütügünden çikacak."

\chapter{7}

\par 1 Uzziya oglu Yotam oglu Ahaz Yahuda Krali'yken, Aram Krali Resin'le Remalya oglu Israil Krali Pekah Yerusalim'e saldirdilar, ama ele geçiremediler.
\par 2 Davut'un torunlari Aram'in Efrayimliler'le güçbirligi ettigini duydular. Ahaz'la halkinin yürekleri rüzgarda sallanan orman agaçlari gibi titremeye basladi.
\par 3 Bu arada RAB Yesaya'ya söyle seslendi: "Ahaz'i karsilamak için oglun Sear-Yasuv'la birlikte Yukari Havuz'un su yolunun sonuna, Çirpici Tarlasi'na giden yola çik.
\par 4 Ona de ki, 'Dikkatli ve sakin ol, korkma! Su tüten iki yanik odun parçasinin -Aram Krali Resin'le Remalya'nin oglunun- öfkesinden korkma.
\par 5 Aram, Efrayim ve Remalya'nin oglu sizin için kötü seyler tasarliyor. Diyorlar ki,
\par 6 Haydi, Yahuda'ya saldiralim, halki korkutup ülkeyi ele geçirelim, Taveal'in oglunu kral ilan edelim.
\par 7 "'Buna karsilik Egemen RAB diyor ki, bu tasari asla gerçeklesmeyecek.
\par 8 Çünkü Sam sadece Aram'in baskenti, Resin de sadece Sam'in basidir. Efrayim'e gelince, altmis bes yil içinde paramparça edilip halk olmaktan çikacak.
\par 9 Samiriye sadece Efrayim'in baskenti, Remalya'nin oglu da sadece Samiriye'nin basidir. Bana güvenmezseniz, güvenlikte olamazsiniz."
\par 10 RAB Ahaz'a yine seslendi:
\par 11 "Tanrin RAB'den bir isaret iste; ölüler diyari kadar derin, gökler kadar yüksek olsun."
\par 12 Ama Ahaz, "Hayir, istemem, RAB'bi sinamam" dedi.
\par 13 Bunun üzerine Yesaya, "Dinleyin, ey Davut'un torunlari!" dedi, "Insanlarin sabrini tasirmaniz yetmezmis gibi simdi de Tanrim'in sabrini mi tasiriyorsunuz?
\par 14 Bundan ötürü Rab'bin kendisi size bir belirti verecek: Iste, kiz gebe kalip bir ogul doguracak; adini Immanuel koyacak.
\par 15 Çocuk kötüyü reddedip iyiyi seçecek yasa gelince tereyagi ve bal yiyecek.
\par 16 Ama çocuk kötüyü reddedip iyiyi seçecek yasa gelmeden, seni dehsete düsüren o iki kralin topraklari issiz kalacak.
\par 17 "RAB seni, halkini ve babanin soyunu Efrayim'in Yahuda'dan ayrildigi günden bu yana görülmemis bir felakete ugratacak; üzerinize Asur Krali'ni saldirtacak.
\par 18 "O gün RAB Misir irmaklarinin ta uçlarindan sinekleri, Asur topraklarindan arilari islikla çagiracak.
\par 19 Akin akin gelip derin vadilerde, kaya kovuklarinda, dikenli çaliliklarda, otlaklarda konaklayacaklar.
\par 20 "O gün Rab Firat'in ötesinden kiraladigi usturayla -Asur Krali'yla- sakalinizi, saçlarinizi, beden killarinizi tiras edecek.
\par 21 O günlerde bir inekle bir çift koyun besleyen
\par 22 aldigi bol süt sayesinde tereyagi yiyecek. Ülkede kalan herkes bal ve tereyagiyla beslenecek.
\par 23 "O gün bin gümüs degerinde bin asmaya sahip olan her bag dikenli çalilarla dolacak.
\par 24 Insanlar oralara okla, yayla gidecek. Çünkü ülkenin her yani dikenli çalilarla kaplanacak.
\par 25 Bir zamanlar çapalanip ekin ekilen tepeler korkudan kimsenin giremeyecegi dikenliklere dönecek, sigirin gezindigi, davarin çignedigi yerler olacak."

\chapter{8}

\par 1 RAB bana söyle dedi: "Büyük bir levha alip okunakli harflerle üzerine, 'Maher-Salal-Has-Baz yaz.
\par 2 Kâhin* Uriya ile Yeverekya oglu Zekeriya'yi kendime güvenilir tanik seçiyorum."
\par 3 Peygamber olan karim bundan bir süre sonra gebe kaldi ve bir erkek çocuk dogurdu. RAB bana, "Adini 'Maher-Salal-Has-Baz koy" dedi,
\par 4 "Çocuk daha 'Anne, baba demesini ögrenmeden, Sam'in serveti ve Samiriye'nin ganimeti Asur Krali'na götürülecek."
\par 5 RAB bana yine seslenip dedi ki,
\par 6 "Bu halk usul usul akan Siloah sularini reddettigi, Resin'le Remalya'nin ogluyla mutlu oldugu için,
\par 7 ben Rab, Firat'in kabaran güçlü sularini -bütün dehsetiyle Asur Krali'ni- üzerlerine salacagim. Yatagindan tasan irmak, kiyilarini su altinda birakacak.
\par 8 Yahuda'yi kaplayan sular her seyi silip süpürerek adam boyu yükselecek, ülkeni boydan boya dolduracak, ey Immanuel!"
\par 9 Ey halklar, yikima, bozguna ugrayacaksiniz. Yeryüzünün en uç köseleri, kulak verin. Savasmaya, bozguna ugramaya hazirlanin. Evet, savasa ve bozguna hazir olun.
\par 10 Istediginizi tasarlayin, hepsi bosa gidecek. Istediginiz kadar konusun, hiçbiri gerçeklesmeyecek. Çünkü Tanri bizimledir.
\par 11 RAB beni halkin tuttugu yoldan gitmeme konusunda siddetle uyararak söyle dedi:
\par 12 "Onlarin entrika dedigi her seye Siz entrika demeyin; Onlarin korktugundan korkmayin, yilmayin.
\par 13 "Her Seye Egemen RAB'bi kutsal sayin. Korkunuz, yilginiz O'ndan olsun.
\par 14 Tapinak O olacak. Israil'in iki kralligi içinse Sürçme tasi ve tökezleme kayasi, Yerusalim'de yasayanlar için Kapan ve tuzak olacak.
\par 15 Birçoklari sendeleyip düsecek, parçalanacak, Tuzaga düsüp ele geçecek."
\par 16 Ya RAB, ögrencilerim arasinda bildirimi koru, Ögretimi mühürle!
\par 17 Kendini Yakup'un soyundan gizleyen RAB'bi özlemle bekliyorum, umudum O'nda.
\par 18 Ben ve RAB'bin bana verdigi çocuklar, Siyon Dagi'nda oturan Her Seye Egemen RAB'bin Israil'deki belirtileri ve isaretleriyiz.
\par 19 Birileri size, "Fisildasip mirildanan medyumlarla ruh çagiranlara danisin" dediginde, "Halk kendi Tanrisi'na danismaz mi; yasayanlar için ölülere mi danisilir?" deyin.
\par 20 Tanri'nin ögretisine ve bildirisine dönmek gerek! Böyle düsünmezlerse, onlar için hiç safak sökmeyecek.
\par 21 Aç ve çaresiz, ülkede dolanip duracaklar. Aç kalinca öfkelenip krallarina, Tanrilari'na lanet edecekler. Yukariya da
\par 22 dünyaya da baksalar sikintidan, karanliktan, korkunç karanliktan baska bir sey görmeyecekler. Kovulacaklari yer koyu karanliktir.

\chapter{9}

\par 1 Bununla birlikte sikinti çekmis olan ülke karanlikta kalmayacak. Geçmiste Zevulun ve Naftali bölgelerini alçaltan Tanri, gelecekte Seria Irmagi'nin ötesinde, Deniz Yolu'nda, uluslarin yasadigi Celile'yi onurlandiracak.
\par 2 Karanlikta yürüyen halk Büyük bir isik görecek; Ölümün gölgeledigi diyarda Yasayanlarin üzerine isik parlayacak.
\par 3 Ya RAB, ulusu çogaltacak, sevincini artiracaksin. Ekin biçenlerin neselendigi, Ganimet paylasanlarin costugu gibi, Onlar da sevinecek senin önünde.
\par 4 Çünkü onlara yük olan boyundurugu, Omuzlarini döven degnegi, Onlara eziyet edenlerin sopasini paramparça edeceksin; Tipki Midyanlilar'i yenilgiye ugrattigin günkü gibi.
\par 5 Savasta giyilen çizmeleri Ve kana bulanmis giysileri Yakilacak, atese yem olacak.
\par 6 Çünkü bize bir çocuk dogacak, Bize bir ogul verilecek. Yönetim onun omuzlarinda olacak. Onun adi Harika Ögütçü, Güçlü Tanri, Ebedi Baba, Esenlik Önderi olacak.
\par 7 Davut'un tahti ve ülkesi üzerinde egemenlik sürecek. Egemenliginin ve esenliginin büyümesi son bulmayacak. Egemenligini adaletle, dogrulukla kuracak Ve sonsuza dek sürdürecek. Her Seye Egemen RAB'bin gayreti bunu saglayacak.
\par 8 Rab Israil için, Yakup soyu için yargisini bildirdi. Bu yargi yerine gelecek.
\par 9 Bütün halk, Efrayim ve Samiriye'de yasayanlar, Rab'bin bu yargisini duyacak. Gururlu ve küstah olan bu halk diyor ki,
\par 10 "Kerpiç evler yikildi, Ama yerlerine yontma tastan evler yapacagiz. Yabanil incir agaçlari kesildi, Ama yerlerine sedir agaçlari dikecegiz."
\par 11 Bundan dolayi RAB, Resin'in hasimlarini Halka karsi güçlendirecek; Düsmanlarini, dogudan Aramlilar'i, Batidan Filistliler'i ayaklandiracak. Bunlar agizlarini ardina kadar açip Israil'i yutacaklar. Bütün bunlara karsin RAB'bin öfkesi dinmedi, Eli hâlâ kalkmis durumda.
\par 13 Halk kendisini cezalandiran RAB'be dönmeyecek, Her Seye Egemen RAB'bi aramayacak.
\par 14 Bunun için RAB Israil'den basi da kuyrugu da Hurma dalini da sazi da Bir günde kesip atacak.
\par 15 Bas ileri gelen saygin kisi, Kuyruksa ögretisi sahte olan peygamberdir.
\par 16 Çünkü bu halki saptiranlar ona yol gösterenlerdir. Onlari izleyenler de yem oluyor.
\par 17 Bu yüzden Rab onlarin gençleri için sevinç duymayacak, Öksüzlerine, dul kadinlarina acimayacak. Çünkü hepsi tanrisizdir, kötülük yaparlar. Her agiz saçmaliyor. Bütün bunlara karsin RAB'bin öfkesi dinmedi, Eli hâlâ kalkmis durumda.
\par 18 Kötülük dikenli çalilari yiyip bitiren ates gibidir. Ormandaki çaliligi tutusturur, Duman sütunlari yükseltir.
\par 19 Her Seye Egemen RAB'bin öfkesi Ülkeyi ates gibi sardi. Halk atese yem olacak, Kardes kardesini esirgemeyecek.
\par 20 Insanlar surada burada bulduklarini yiyecekler, Ama aç kalacak, doymayacaklar. Herkes çocugunun etini yiyecek:
\par 21 Manasse Efrayim'i, Efrayim Manasse'yi yiyecek, Sonra birlikte Yahuda'nin üzerine yürüyecekler. Bütün bunlara karsin RAB'bin öfkesi dinmedi, Eli hâlâ kalkmis durumda.

\chapter{10}

\par 1 Yoksullardan adaleti esirgemek, Halkimin düskünlerinin hakkini elinden almak, Dullari avlamak, Öksüzlerin malini yagmalamak için Haksiz kararlar alanlarin, Adil olmayan yasalar çikaranlarin vay haline!
\par 3 Yargi günü Uzaklardan basiniza felaket geldiginde ne yapacaksiniz? Yardim için kime kosacaksiniz, Servetinizi nereye saklayacaksiniz?
\par 4 Tutsaklar arasinda bir köseye sinmek Ya da savasta ölmekten baska çareniz kalmayacak. Bütün bunlara karsin RAB'bin öfkesi dinmedi, Eli hâlâ kalkmis durumda.
\par 5 "Vay haline Asur, öfkemin degnegi! Elindeki sopa benim gazabimdir.
\par 6 Asur'u tanrisiz ulusa karsi salacagim; Soyup yagma etmesi, Sokaktaki çamur gibi onlari çignemesi, Öfkelendigim halkin üzerine yürümesi için Buyruk verecegim."
\par 7 Ama Asur Krali bundan da kötüsünü düsünüyor. Birçok ulusun kökünü kaziyip yok etmeyi tasarliyor.
\par 8 "Komutanlarimin hepsi birer kral degil mi?" diyor,
\par 9 "Kalno'yu, Karkamis gibi ele geçirmedim mi? Hama'nin sonu Arpat'inki, Samiriye'nin sonu Sam'inki gibi olmadi mi?
\par 10 Putlari Yerusalim ve Samiriye'ninkinden daha çok olan putperest ülkeleri nasil ele geçirdimse,
\par 11 Samiriye'ye ve putlarina ne yaptimsa, Yerusalim'e ve putlarina da yapamaz miyim?"
\par 12 Rab Siyon Dagi'na ve Yerusalim'e karsi tasarladiklarini yapip bitirdikten sonra söyle diyecek: "Asur Krali'ni kibirli yüregi, Övüngen bakislari yüzünden cezalandiracagim.
\par 13 Çünkü, 'Her seyi bilegimin gücüyle, Bilgeligimle yaptim diyor, 'Akilliyim, uluslari ayiran sinirlari yok ettim, Hazinelerini yagmaladim, Güçlü krallari tahtlarindan indirdim.
\par 14 Elimi yuvaya sokup kus yumurtalarini toplar gibi Uluslarin varini yogunu topladim. Terk edilmis yumurtalari nasil toplarlarsa, Ben de bütün ülkeleri öyle topladim. Kanat çirpan, agzini açan, Sesini çikaran olmadi."
\par 15 Balta kendisini kullanana karsi övünür mü? Testere kendisini kullanana karsi büyüklenir mi? Sanki degnek kendisini kaldirani sallayabilir, Sopa sahibini kaldirabilirmis gibi...
\par 16 Rab, Her Seye Egemen RAB, Asur'un güçlü adamlarini Yipratici hastalikla cezalandiracak. Ordulari alev alev yanacak.
\par 17 Israil'in Isigi ates, Israil'in Kutsali alev olacak; Asur'un dikenli çalilarini Bir gün içinde yakip bitirecek.
\par 18 Görkemli ormaniyla verimli tarlalari, Ölümcül bir hastaliga yakalanmis insan gibi Tümüyle harap olacak.
\par 19 Ormanda artakalan agaçlar Bir çocugun bile sayabilecegi kadar az olacak.
\par 20 O gün Israil'in sag kalanlari, Yakup'un kaçip kurtulan torunlari, Kendilerini yok etmek isteyene degil, Artik içtenlikle RAB'be, Israil'in Kutsali'na dayanacaklar.
\par 21 Geriye kalanlar, Yakup soyundan sag kalanlar, Güçlü Tanri'ya dönecekler.
\par 22 Ey Israil, halkin denizin kumu kadar çok olsa da, Ancak pek azi dönecek*fi*. Tümüyle adil bir yikim kararlastirildi.
\par 23 Rab, Her Seye Egemen RAB, Kararlastirilan yikimi bütün yeryüzünde gerçeklestirecek.
\par 24 Bu nedenle Rab, Her Seye Egemen RAB söyle diyor: "Ey sen, Siyon'da yasayan halkim, Asurlular, Misirlilar'in yaptigi gibi Sana degnekle vurduklarinda, Sopalarini sana karsi kaldirdiklarinda korkma.
\par 25 Çünkü çok yakinda gazabim sona erecek, Öfkem Asurlular'in yikimini saglayacak.
\par 26 Ben, Her Seye Egemen RAB, Midyanlilar'i Orev Kayasi'nda alt ettigim gibi, Onlari da kirbaçla alt edecegim. Degnegimi Misir'a karsi nasil denizin üzerine uzattimsa, Simdi yine öyle yapacagim.
\par 27 O gün Asur'un yükü sirtinizdan, Boyundurugu boynunuzdan kalkacak; Semirdiginiz için boyunduruk kirilacak."
\par 28 Ayat Kenti'ne saldirdilar, Migron'dan geçip agirliklarini Mikmas'ta biraktilar.
\par 29 Geçidi asarak Geva'da konakladilar. Rama Kenti korkudan titredi, Saul'un kenti Giva'da yasayan halk kaçisti.
\par 30 Ey Gallim halki, sesini yükselt! Ey Laysa halki, dinle! Zavalli Anatot halki!
\par 31 Madmena halki kaçiyor, Hagevim'de yasayanlar siginacak yer ariyor.
\par 32 Düsman bugün Nov'da duracak; Siyon Kenti'nin kuruldugu daga, Yerusalim Tepesi'ne yumruk sallayacak.
\par 33 Rab, Her Seye Egemen RAB düsmani Dal gibi kesip korkunç güçle yere çalacak. Uzun boylulari devirecek, Gururlulari alçaltacak.
\par 34 Ormandaki çaliliklari baltayla keser gibi Kesip devirecek onlari. Lübnan, Güçlü Olan'in önünde diz çökecek.

\chapter{11}

\par 1 Isay'in kütügünden yeni bir filiz çikacak, Kökünden bir fidan meyve verecek.
\par 2 RAB'bin Ruhu, bilgelik ve anlayis ruhu, Ögüt ve güç ruhu, bilgi ve RAB korkusu ruhu Onun üzerinde olacak.
\par 3 RAB korkusu hosuna gidecek. Gözüyle gördügüne göre yargilamayacak, Kulagiyla isittigine göre karar vermeyecek.
\par 4 Yoksullari adaletle yargilayacak, Yeryüzünde ezilenler için dürüstçe karar verecek. Dünyayi agzinin degnegiyle cezalandiracak, Kötüleri soluguyla öldürecek.
\par 5 Davranisinin temeli adalet ve sadakat olacak.
\par 6 Onun döneminde kurtla kuzu bir arada yasayacak, Parsla oglak birlikte yatacak, Buzagi, genç aslan ve besili sigir yanyana duracak, Onlari küçük bir çocuk güdecek.
\par 7 Inekle ayi birlikte otlayacak, Yavrulari bir arada yatacak. Aslan sigir gibi saman yiyecek.
\par 8 Emzikteki bebek kobra deligi üzerinde oynayacak, Sütten kesilmis çocuk elini engerek kovuguna sokacak.
\par 9 Kutsal dagimin hiçbir yerinde Kimse zarar vermeyecek, yok etmeyecek. Çünkü sular denizi nasil dolduruyorsa, Dünya da RAB'bin bilgisiyle dolacak. Sürgünler Geri Dönecek
\par 10 O gün Isay'in kökü ortaya çikacak, Halklara sancak olacak, Uluslar ona yönelecek. Kaldigi yer görkemli olacak.
\par 11 O gün Rab, Asur'dan, Misir, Patros, Kûs*, Elam, Sinar, Hama ve deniz kiyilarindan Halkinin sag kalanlarini kurtarmak için Ikinci kez elini uzatacak.
\par 12 Uluslar için sancak kaldiracak, Sürgün Israilliler'i toplayacak, Dagilmis Yahudalilar'i Dünyanin dört bucagindan bir araya getirecek.
\par 13 Efrayim halkinin kiskançligi yok olacak, Yahudalilar'i sikistiranlar ortadan kalkacak. Efrayim Yahuda'yi kiskanmayacak, Yahuda Efrayim'i sikistirmayacak.
\par 14 Batidaki Filistliler'e saldirip Hep birlikte dogudakilerin her seyini yagmalayacaklar. Edom ve Moav halklarinin topraklarina el koyacak, Ammonlular'a boyun egdirecekler.
\par 15 RAB Misir'in Süveys Körfezi'ni tümüyle kurutacak. Elinin bir sallayisiyla estirecegi kavurucu rüzgarla Firat'i süpürüp yedi dereye bölecek. Öyle ki, insanlar irmak yatagindan çarikla geçebilsin.
\par 16 RAB'bin Asur'da sag kalan halki için Bir çikis yolu olacak; Tipki Misir'dan çiktiklari gün Israilliler'in de bir çikis yolu oldugu gibi.

\chapter{12}

\par 1 Israil halki o gün, "Ya RAB, sana sükrederiz" diyecek, "Bize öfkelenmistin ama öfken dindi, Bizi avuttun.
\par 2 Tanri kurtulusumuzdur. O'na güvenecek, yilmayacagiz. Çünkü RAB gücümüz ve ezgimizdir. O kurtardi bizi."
\par 3 Kurtulus pinarlarindan sevinçle su alacaksiniz.
\par 4 O gün diyeceksiniz ki, "RAB'be sükredin, O'na yakarin, Halklara duyurun yaptiklarini, Adinin yüce oldugunu duyurun!
\par 5 RAB'be ezgiler söyleyin, Çünkü görkemli isler yapti. Bütün dünya bilsin bunu.
\par 6 Ey Siyon halki, sesini yükselt, sevinçle haykir! Çünkü aranizda bulunan Israil'in Kutsali büyüktür."

\chapter{13}

\par 1 Amots oglu Yesaya'nin Babil'le ilgili bildirisi:
\par 2 Çiplak dagin tepesine sancak dikin! Savasçilari yüksek sesle çagirip El sallayin ki Soylulara ayrilan kapilardan içeri girsinler.
\par 3 RAB seçtiklerine buyruk verdi, O'nun yüceligiyle övünen yigitleri Öfkesinin geregini yapmaya çagirdi.
\par 4 Daglardaki kalabaligin gürültüsünü dinleyin! Büyük bir halkin sesini andiriyor. Bir araya gelmis uluslarin Ve kralliklarin gümbürtüsünü dinleyin! Her Seye Egemen RAB bir orduyu savasa hazirliyor.
\par 5 Öfkesinin araçlariyla uzak bir ülkeden, Dünyanin öbür ucundan Bütün ülkeyi yerle bir etmek üzere geliyor.
\par 6 Feryat edin! Çünkü RAB'bin günü yakindir. Her Seye Gücü Yeten'in gönderecegi yikim gibi geliyor o gün.
\par 7 Bu yüzden ellerde derman kalmayacak, Her yürek eriyecek.
\par 8 Herkesi dehset saracak, Hepsi aci ve istirap içinde bogulacak, Doguran kadin gibi kivranacak, Saskin saskin birbirlerine bakacaklar; Yüzleri kizaracak.
\par 9 Iste RAB'bin acimasiz günü geliyor. Ülkeyi viraneye çevirip Içindeki günahkârlari ortadan kaldiracagi Gazap ve kizgin öfke dolu gün geliyor.
\par 10 Gökteki yildizlarla takimyildizlar isimayacak, Dogan günes kararacak, ay isigini vermez olacak.
\par 11 RAB diyor ki, "Kötülügünden ötürü dünyayi, Suçlarindan ötürü kötüleri cezalandiracagim. Kibirlilerin küstahligini sona erdirecek, Zalimlerin gururunu kiracagim.
\par 12 Insani saf altindan, Ofir altinindan daha ender kilacagim.
\par 13 Ben, Her Seye Egemen RAB, Gazaba geldigim, öfkemin alevlendigi gün Gökleri titretecegim, yer yerinden oynayacak.
\par 14 "Herkes kovalanan ceylan gibi, Çobansiz koyunlar gibi halkina dönecek, Ülkesine kaçacak.
\par 15 Yakalananin bedeni delik desik edilecek, Ele geçen kiliçtan geçirilecek.
\par 16 Yavrulari gözleri önünde parçalanacak, Evleri yagmalanacak, Kadinlarinin irzina geçilecek.
\par 17 "Gümüse deger vermeyen, Altini sevmeyen Medler'i Onlara karsi harekete geçirecegim.
\par 18 Oklariyla gençleri parçalayacak, Bebeklere acimayacak, Çocuklari esirgemeyecekler.
\par 19 Ben Tanri, Sodom ve Gomora'yi nasil yerle bir ettimse, Kildaniler'in* yüce gururu, Kralliklarin en güzeli olan Babil'i de yerle bir edecegim.
\par 20 Orada bir daha kimse yasamayacak, Kusaklar boyu kimse oturmayacak, Bedeviler çadir kurmayacak, Çobanlar sürülerini dinlendirmeyecek.
\par 21 Orasi yabanil hayvanlara barinak olacak, Evler çakallarla dolacak, Baykuslar yuva yapacak, tekeler oynasacak orada.
\par 22 Kalelerinde sirtlanlar, Görkemli saraylarinda çakallar uluyacak. Babil'in sonu yaklasti, günleri uzatilmayacak."

\chapter{14}

\par 1 Çünkü RAB Yakup soyuna aciyacak, Israil halkini yine seçip Topraklarina yerlestirecek. Yabancilar da Yakup soyuna katilip onlara baglanacak.
\par 2 Uluslar Israil halkini Kendi topraklarina götürecekler. Israil halki RAB'bin verdigi topraklarda onlari Erkek ve kadin köle olarak sahiplenecek. Kendisini tutsak edenleri tutsak edecek, Kendisini ezenlere egemen olacak.
\par 3 RAB Israil halkini acidan, sikintidan Ve yaptigi agir islerden kurtardigi gün
\par 4 Babil Krali'ni alaya alarak, "Halki ezenin nasil da sonu geldi!" diyecekler, "Zorbaligi nasil da sona erdi!"
\par 5 RAB kötülerin degnegini, Egemenlerin asasini kirdi.
\par 6 O asa ki, halklara gazapla vurdukça vurdu, Uluslari öfkeyle, dinmeyen zulümle yönetti.
\par 7 Bütün dünya esenlik ve baris içinde Sevinçle haykiriyor.
\par 8 Lübnan'in çam ve sedir agaçlari bile Kralin yok olusuna seviniyor. "Onun ölümünden beri kimse bizi kesmeye gelmiyor" diyorlar.
\par 9 Topragin altindaki ölüler diyari Babil Krali'ni karsilamak için sabirsizlaniyor. Onun gelisi ölüleri, Dünyanin eski önderlerini heyecanlandiriyor; Uluslari yönetmis krallari Tahtlarindan ayaga kaldiriyor.
\par 10 Hepsi ona seslenip diyecekler ki, "Sen de bizim gibi gücünü yitirdin, Bize benzedin."
\par 11 Görkemin de çenklerinin sesi de Ölüler diyarina indirildi. Altinda kurtlar kaynasacak, Üstünü kurtçuklar kaplayacak.
\par 12 Ey parlak yildiz, seherin oglu, Göklerden nasil da düstün! Ey uluslari ezip geçen, Nasil da yere yikildin!
\par 13 Içinden, "Göklere çikacagim" dedin, "Tahtimi Tanri'nin yildizlarindan daha yüksege koyacagim; Ilahlarin toplandigi dagda, Safon'un dorugunda oturacagim.
\par 14 Bulutlarin üstüne çikacak, Kendimi Yüceler Yücesi'yle esit kilacagim."
\par 15 Ancak ölüler diyarina, Ölüm çukurunun dibine Indirilmis bulunuyorsun.
\par 16 Seni görenler bakip bakip söyle düsünecekler: "Dünyayi sarsan, ülkeleri titreten, Yeryüzünü çöle çeviren, Kentleri yerle bir eden, Tutsaklari evlerine salivermeyen adam bu mu?"
\par 18 Uluslarin bütün krallari tek tek, Görkemli mezarlarda yatiyor.
\par 19 Ama sen reddedilen bir dal gibi Mezarindan disari atildin; Bedenleri kiliçla delinip Ölüm çukurunun dibine atilmis ölülerle örtülüsün; Ayak altinda çignenen les gibisin.
\par 20 Ülkeni harap edip halkini katlettigin için Baskalari gibi gömülmeyeceksin. Kötülük yapan soy bir daha anilmayacak.
\par 21 Atalarinin suçundan ötürü Babil Krali'nin ogullarini bogazlamak için yer hazirlayin. Kalkip dünyayi sahiplenmesinler, Yeryüzünü kentlerle doldurmasinlar.
\par 22 "Babil halkina karsi harekete geçecegim" Diyor Her Seye Egemen RAB, "Babil'in adini, sag kalanlarini, Ogullarini, torunlarini dünyadan silecegim." Böyle diyor RAB.
\par 23 "Babil'i baykus yuvasina, batakliga çevirecek, Yikim süpürgesiyle süpürecegim" Diyor Her Seye Egemen RAB.
\par 24 Her Seye Egemen RAB ant içerek söyle dedi: "Düsündügüm gibi olacak, Tasarladigim gibi gerçeklesecek.
\par 25 Asurlular'i kendi ülkemde ezecek, Daglarimda çigneyecegim. Halkim Asur'un boyundurugundan, Omuzlarindaki yükten kurtulacak.
\par 26 Iste bütün dünya için belirlenen tasari budur. Bütün uluslara karsi elim kalkmis durumda.
\par 27 Her Seye Egemen RAB'bin tasarisini kim bosa çikarabilir? Kalkmis durumdaki elini kim indirebilir?"
\par 28 Kral Ahaz'in öldügü yil gelen bildiri:
\par 29 Ey Filistliler, sizi döven degnek kirildi diye sevinmeyin. Çünkü yilanin kökünden engerek türeyecek, Onun ürünü uçan yilan olacak.
\par 30 Yoksullarin en yoksulu doyacak, Düskünler güvenlikte yatacak. Ama sizin kökünüzü kitlikla kurutacagim, Sag kalanlariniz da ölecek.
\par 31 Ulumaya basla ey kapi! Ey kent, feryat et! Ey Filistliler, eridiniz bastan basa. Kuzeyden toz duman yükseliyor, Düsman askerleri sira sira geliyor.
\par 32 O ulusun elçilerine ne yanit verilecek? "RAB Siyon'un temelini atti, Halkinin düskünleri oraya siginacak" denecek.

\chapter{15}

\par 1 Moav'la ilgili bildiri: Moav'in Ar Kenti bir gecede viraneye döndü, yok oldu, Moav'in Kîr Kenti bir gecede viraneye döndü, yok oldu.
\par 2 Bu yüzden Divon halki aglamak için tapinaga, Tapinma yerlerine çikti. Moav halki, Nevo ve Medeva için feryat ediyor. Insanlar saçlarini sakallarini kesiyor.
\par 3 Çul giyiyorlar sokaklarda, Damlarda, meydanlarda herkes feryat ediyor, Gözyaslari sel gibi.
\par 4 Hesbon ve Elale'nin haykirislari Yahas'a ulasiyor. Moav askerleri bu yüzden feryat ediyor, Yürekleri korku içinde.
\par 5 Yüregim sizliyor Moav için. Kaçanlar Soar'a, Eglat-Selisiya'ya ulasti, Aglaya aglaya çikiyorlar Luhit Yokusu'ndan, Horonayim yolunda yikimlarina agit yakiyorlar.
\par 6 Nimrim sulari kurudugu için otlar sararip soldu. Taze ot kalmadi, Ysaillik yok artik.
\par 7 Bu yüzden halk kazanip biriktirdigi ne varsa, Kavak Vadisi üzerinden tasiyacak.
\par 8 Haykirislari Moav topraklarinda yankilaniyor, Feryatlari Eglayim'e, Beer-Elim'e dek ulasti.
\par 9 Çünkü Dimon sulari kan dolu, Ama basina daha beterini getirecegim. Moav'dan kaçip kurtulanlarin, Ülkede sag kalanlarin üzerine aslan salacagim.

\chapter{16}

\par 1 Sela'dan çöl yoluyla Siyon Kenti'nin kuruldugu daga, Ülkenin hükümdarina kuzular gönderin.
\par 2 Moavli kizlar yuvalarindan atilmis, Öteye beriye uçusan kuslar gibi Arnon Irmagi'nin geçitlerinde dolasiyor.
\par 3 "Bize ögüt ver, bir karar al, Ögle sicaginda gece gibi gölge sal üstümüze. Kovulanlari sakla, kaçaklari ele verme" diyorlar.
\par 4 "Kovulanlarim seninle birlikte yasasin. Kirip geçirenlere karsi Biz Moavlilar'a siginak ol." Baski ve yikim son buldugunda, Ülkeyi çigneyenler yok oldugunda, Sevgiye dayanan bir yönetim kurulacak,
\par 5 Davut soyundan biri sadakatle krallik yapacak. Yargilarken adaleti arayacak, Dogru olani yapmakta tez davranacak.
\par 6 Moav'in ne denli gururlanip büyüklendigini, Kendini ne denli begendigini, Kibirlenip küstahlastigini duyduk. Övünmesi bosunadir.
\par 7 Bu yüzden Moavlilar Moav için feryat edecek, Hepsi feryat edecek. Kîr-Hereset'in üzüm pestillerini Animsayip üzülecek, yas tutacaklar.
\par 8 Çünkü Hesbon'un tarlalari, Sivma'nin asmalari kurudu. Uluslarin beyleri onlarin seçkin dallarini kirdilar. O dallar ki, Yazer'e erisir, çöle uzanirdi, Filizleri yayilir, gölü*fp* asardi.
\par 9 Bu yüzden Yazer için, Sivma'nin asmalari için aci aci agliyorum. Sizleri gözyaslarimla sulayacagim, Ey Hesbon ve Elale! Çünkü savas çigliklari yaz meyvelerinizin, Biçtiginiz ekinin üzerine düstü.
\par 10 Meyve bahçelerindeki sevinç ve nese yok oldu. Baglarda ne sarki söyleyen olacak, Ne sevinç çigligi atan. Üzüm sikma çukurlarinda çalisan kalmayacak, Sevinç çigliklarini susturdum.
\par 11 Yüregim bir lir gibi inliyor Moav için, Kîr-Hereset için içim sizliyor.
\par 12 Moav halki tapinma yerine çikarak kendini yoruyor, Dua etmek için tapinaga gidiyor, ama hepsi bosuna!
\par 13 RAB'bin Moav için geçmiste söyledigi budur.
\par 14 RAB simdi diyor ki, "Moav'in övündükleri de kalabalik halki da tam üç yil sonra rezil olacak. Sag kalan çok az sayida kisiyse güçsüz olacak."

\chapter{17}

\par 1 Sam'la ilgili bildiri: Iste Sam kent olmaktan çikacak, Enkaz yiginina dönecek.
\par 2 Aroer kentleri terk edilecek, Hayvan sürüleri orada yatacak, Onlari ürküten olmayacak.
\par 3 Efrayim'de surlu kent kalmayacak, Sam'in egemenligi yok olacak. Sag kalan Aramlilar'in onuru Israil'in onuru gibi kirilacak. Her Seye Egemen RAB böyle diyor.
\par 4 O gün Yakup soyunun görkemi sönecek, Hepsi bir deri bir kemik kalacak.
\par 5 Israil, ekinin elle biçilip Basaklarin devsirildigi bir tarla, Refaim Vadisi'nde hasattan sonra Basaklarin toplandigi bir tarla gibi olacak.
\par 6 Çok az kisi kurtulacak. Artakalanlarin sayisi, dövüldükten sonra tepesinde iki üç, Dal uçlarinda dört bes zeytin tanesi kalan Zeytin agaci gibi olacak. Israil'in Tanrisi RAB böyle diyor.
\par 7 O gün insanlar kendilerini yaratana bakacaklar, gözleri Israil'in Kutsali'ni görecek.
\par 8 Elleriyle yaptiklari sunaklara, parmaklariyla biçim verdikleri Asera* putlarina, buhur sunaklarina bakmayacaklar.
\par 9 O gün Israil'in güçlü kentleri Israilliler'den kaçan Amorlular'la Hivliler'in Terk ettigi kentler gibi issiz olacak.
\par 10 Çünkü, ey Israil, seni kurtaran Tanri'yi unuttun, Sigindigin Kaya'yi anmaz oldun. Bunun yerine, güzel fidanlar, ithal asmalar dikiyorsun.
\par 11 Onlar diktigin gün filizlenip Ertesi sabah tomurcuklanabilir. Ama hastalik ve dinmez aci gününde meyve vermeyecekler.
\par 12 Eyvah, çok sayida ulus kükrüyor, Azgin deniz gibi gürlüyorlar. Halklar güçlü sular gibi çagliyor.
\par 13 Halklar kabaran sular gibi çaglayabilir, Ama Tanri onlari azarlayinca uzaklara kaçacaklar. Rüzgarin önünde dagdaki saman ufagi gibi, Kasirganin önünde diken yumagi gibi savrulacaklar.
\par 14 Aksam dehset saçiyorlardi, Sabah olmadan yok olup gittiler. Bizi yagmalayanlarin, bizi soyanlarin sonu budur.

\chapter{18}

\par 1 Kûs* irmaklarinin ötesinde, Kanat viziltilarinin duyuldugu ülkenin vay haline!
\par 2 O ülke ki, elçilerini Sazdan kayiklarla Nil sularindan gönderir. Ey ayagina tez ulaklar, Irmaklarin böldügü ülkeye, Her yana korku saçan halka, Güçlü ve ezici ulusa, O uzun boylu, pürüzsüz tenli ulusa gidin!
\par 3 Ey sizler, dünyada yasayan herkes, Yeryüzünün ahalisi! Sancak daglarin tepesine dikilince dikkat edin. Boru çalininca dinleyin.
\par 4 Çünkü RAB bana söyle dedi: "Gün isiginda duru sicaklik gibi, Hasat döneminin sicakligindaki Çiy bulutu gibi durgun olacak Ve bulundugum yerden seyredecegim."
\par 5 Bagbozumundan önce çiçekler düsüp Üzümler olgunlasmaya yüz tutunca, Asmanin dallari biçakla kesilecek, Çubuklari koparilip atilacak.
\par 6 Hepsi dagin yirtici kuslarina, Yerin yabanil hayvanlarina terk edilecek. Yazin yirtici kuslara, Kisin yabanil hayvanlara yem olacaklar.
\par 7 O zaman irmaklarin böldügü ülke, Her yana korku saçan güçlü ve ezici halk, O uzun boylu, pürüzsüz tenli ulus, Her Seye Egemen RAB'be armaganlar getirecek. Her Seye Egemen RAB'bin adini koydugu Siyon Dagi'na getirecekler armaganlarini.

\chapter{19}

\par 1 Misir'la ilgili bildiri: Iste RAB hizla yol alan buluta binmis Misir'a geliyor! Misir putlari O'nun önünde titriyor, Misirlilar'in yüregi hopluyor.
\par 2 RAB diyor ki, "Misirlilar'i Misirlilar'a karsi ayaklandiracagim; Kardes kardese, komsu komsuya, kent kente, Ülke ülkeye karsi savasacak.
\par 3 Misirlilar'in cesareti tükenecek, Tasarilarini bosa çikaracagim. Yardim için putlara, ölülerin ruhlarina, Medyumlarla ruh çagiranlara danisacaklar.
\par 4 Misirlilar'i acimasiz bir efendiye teslim edecegim, Kati yürekli bir kral onlara egemen olacak." Rab, Her Seye Egemen RAB böyle diyor.
\par 5 Nil'in sulari çekilecek, Kuruyup çatlayacak yatagi.
\par 6 Su kanallari kokacak, Kuruyacak irmagin kollari, Kamislarla sazlar solacak.
\par 7 Nil kiyisinda, irmagin agzindaki sazlar, Nil boyunca ekili tarlalar kuruyacak, Savrulup yok olacak.
\par 8 Balikçilar yas tutacak, Nil'e olta atanlarin hepsi aglayacak, Suyun yüzüne ag atanlar perisan olacak.
\par 9 Taranmis keten isleyenler, Beyaz bez dokuyanlar umutsuzluga kapilacak.
\par 10 Dokumacilar bunalacak, Ücretliler sikintiya düsecek.
\par 11 Soan Kenti'nin önderleri ne kadar akilsiz! Firavunun bilge danismanlari Saçma sapan ögütler veriyorlar. Nasil olur da firavuna, "Biz bilgelerin ogullari, Eski zaman krallarinin torunlariyiz" diyorlar?
\par 12 Ey firavun, hani nerede senin bilgelerin? Her Seye Egemen RAB Misir'a karsi neler tasarladi, Bildirsinler bakalim sana eger biliyorlarsa.
\par 13 Soan Kenti'nin önderleri aptal olup çiktilar, Nof önderleri aldandilar, Misir oymaklarinin ileri gelenleri Misir'i saptirdilar.
\par 14 RAB onlarin aklini karistirdi; Kendi kusmugu içinde yalpalayan sarhos nasilsa, Misir'i da her alanda saptirdilar.
\par 15 Misir'da kimsenin yapabilecegi bir sey kalmadi; Ne basin ne kuyrugun, ne hurma dalinin ne de sazin.
\par 16 O gün Misirlilar kadin gibi olacaklar; Her Seye Egemen RAB'bin kendilerine karsi kalkan elinin önünde titreyip dehsete kapilacaklar.
\par 17 Yahuda Misir'i dehsete düsürecek. Yahuda dendi mi, Her Seye Egemen RAB'bin Misir'a karsi tasarladiklarini animsayan herkes dehsete kapilacak.
\par 18 O gün Misir'da Kenan dilini konusan bes kent olacak. Bu kentler Her Seye Egemen RAB'be baglilik andi içecekler; içlerinden biri 'Yikim Kenti diye adlandirilacak.
\par 19 O gün Misir'in ortasinda RAB için bir sunak, sinirinda da bir sütun dikilecek.
\par 20 Her Seye Egemen RAB için Misir'da bir belirti ve tanik olacak bu. Halk kendine baski yapanlardan ötürü RAB'be yakarinca, RAB onlari savunacak bir kurtarici gönderip özgür kilacak.
\par 21 RAB kendini Misirlilar'a tanitacak, onlar da o gün RAB'bi taniyacak, kurbanlarla, sunularla O'na tapinacaklar. RAB'be adak adayacak ve adaklarini yerine getirecekler.
\par 22 RAB Misirlilar'i hastalikla alabildigine cezalandiracak, sonra iyilestirecek. RAB'be yönelip yakaracaklar. RAB de onlari iyilestirecek.
\par 23 O gün Misir'la Asur arasinda bir yol olacak. Asurlu Misir'a, Misirli Asur'a gidip gelecek. Misirlilar'la Asurlular birlikte tapinacaklar.
\par 24 O gün Misir ve Asur'un yanisira Israil üçüncü ülke olacak. Dünya bu üçü sayesinde kutsanacak.
\par 25 Her Seye Egemen RAB, "Halkim Misir, ellerimin isi Asur ve mirasim Israil kutsansin" diyerek dünyayi kutsayacak.

\chapter{20}

\par 1 Asur ordusunun baskomutani, Asur Krali Sargon'un buyruguyla gelerek Asdot'a saldirip kenti ele geçirdigi yil RAB Amots oglu Yesaya araciligiyla söyle dedi: "Git, belindeki çulu çöz, ayagindaki çarigi çikar." Yesaya denileni yapti, çiplak ve yalinayak dolasmaya basladi.
\par 3 RAB dedi ki, "Misir'a ve Kûs'a* belirti ve ibret olsun diye kulum Yesaya nasil üç yil çiplak ve yalinayak dolastiysa,
\par 4 Asur Krali da Misir'a utanç olsun diye Misirli tutsaklarla Kûslu sürgünleri genç yasli demeden, çiplak ve yalinayak, mahrem yerleri açik yürütecek.
\par 5 Kûs'a bel baglayan, Misir'la övünen halk hüsrana ugrayacak, utanç içinde kalacak.
\par 6 Bu kiyi bölgesinde yasayanlar o gün, "Asur Krali'nin elinden kurtulmak için yardimina sigindigimiz, bel bagladigimiz uluslarin basina gelene bakin!" diyecekler, "Biz nasil kurtulacagiz?"

\chapter{21}

\par 1 Deniz kiyisindaki çölle ilgili bildiri: Negev'den firtinalar nasil üst üste gelirse, Çölden, korkunç ülkeden bir istilaci öyle geliyor.
\par 2 Korkunç bir görüm gördüm: Hain hainlik etmede, Harap eden harap etmede. Ey Elam, saldir! Ey Meday, onu kusat! Onun neden oldugu iniltileri sona erdirecegim.
\par 3 Gördüklerimden ötürü belime agri saplandi, Doguran kadinin agrilari gibi agrilar tuttu beni. Duyduklarimdan sarsildim, Gördüklerimden dehsete düstüm.
\par 4 Saskinim, titremeler sardi beni. Özledigim alaca karanlik bana korku veriyor artik.
\par 5 Gördügüm görümde sofrayi hazirliyor, Halilari seriyor, yiyip içiyorlar. Kalkin, ey önderler, kalkanlari yaglayin!
\par 6 Rab bana dedi ki, "Git, bir gözcü dik, gördügünü bildirsin.
\par 7 Savas arabalarinin, Atlara, eseklere, develere binmis insanlarin Çifter çifter geldigini görünce dikkat kesilsin."
\par 8 Gözcü, "Ey efendim, Her gün araliksiz gözcü kulesinde duruyor, Her gece yerimde nöbet tutuyorum" diye bagirdi,
\par 9 "Bak, savas arabalariyla atlilar Çifter çifter geliyor!" Sonra, "Yikildi, Babil yikildi!" diye haber verdi, "Taptiklari bütün putlar yere çalinip parçalandi!"
\par 10 Ey halkim, harman yerinde Bugday gibi dövülmüs olan halkim! Her Seye Egemen RAB'den, Israil'in Tanrisi'ndan duyduklarimi Size bildirdim.
\par 11 Duma ile ilgili bildiri: Biri Seir'den bana sesleniyor: "Ey gözcü, geceden geriye ne kaldi? Geceden geriye ne kaldi?"
\par 12 Yanitim söyle: "Sabah olmak üzere, Ama yine gece olacak. Soracaksaniz sorun, yine gelin."
\par 13 Arabistan'la ilgili bildiri: Arabistan çaliliklarinda geceleyeceksiniz, Ey Dedan kervanlari!
\par 14 Ey Tema'da oturanlar, Su getirin, susamislari karsilayin, Kaçip kurtulana ekmek verin.
\par 15 Çünkü onlar kiliçtan, yalin kiliçtan, Gerilmis yaydan, çetin çarpismalardan kaçtilar.
\par 16 Rab bana söyle dedi: "Kedar'in bütün övüncü tam bir yil sonra sona erecek.
\par 17 Okçulardan, Kedar savasçilarindan pek az sag kalan olacak." Bunu söyleyen, Israil'in Tanrisi RAB'dir.

\chapter{22}

\par 1 Görüm Vadisi'yle ilgili bildiri: Gürültü patirti içinde eglenen kent halki, Ne oldu size, neden hepiniz damlara çiktiniz? Ölenleriniz ne kiliçtan geçirildi, Ne de savasta öldü.
\par 3 Önderleriniz hep birlikte kaçtilar, Yaylarini kullanmadan tutsak alindilar. Uzaga kaçtiginiz halde ele geçenlerin hepsi tutsak edildi.
\par 4 Bunun için dedim ki, "Beni yalniz birakin, aci aci aglayayim. Halkimin ugradigi yikimdan ötürü Beni avutmaya kalkmayin."
\par 5 Çünkü Rab'bin, Her Seye Egemen RAB'bin Görüm Vadisi'nde kargasa, bozgun Ve dehset saçacagi gün, Duvarlarin yikilacagi, Daglara feryat edilecegi gün geliyor.
\par 6 Elamlilar ok kiliflarini sirtlanip savas arabalariyla, Atlilariyla geldiler. Kîr halki kalkanlarini açti.
\par 7 Verimli vadileriniz savas arabalariyla doldu, Atlilar kent kapilarinin karsisina dizildi.
\par 8 RAB'bin Yahuda'yi savunmasiz biraktigi gün Orman Sarayi'ndaki silahlara güvendiniz.
\par 9 Davut Kenti'nin duvarlarinda Çok sayida gedik oldugunu gördünüz, Asagi Havuz'da su depoladiniz,
\par 10 Yerusalim'deki evleri saydiniz, Surlari onarmak için evleri yiktiniz.
\par 11 Eski Havuz'un sulari için Iki surun arasinda bir depo yaptiniz. Ama bunu çok önceden tasarlayip Gerçeklestirmis olan Tanri'ya güvenmediniz, O'nu umursamadiniz.
\par 12 Rab, Her Seye Egemen RAB O gün sizi aglayip yas tutmaya, Saçlarinizi kesip çul kusanmaya çagirdi.
\par 13 Oysa siz keyif çatip eglendiniz, "Yiyelim içelim, nasil olsa yarin ölecegiz" diyerek Sigir, koyun kestiniz, Et yiyip sarap içtiniz.
\par 14 Her Seye Egemen RAB bana, "Siz ölene dek bu suçunuz bagislanmayacak" diye seslendi. Rab, Her Seye Egemen RAB böyle diyor.
\par 15 Rab, Her Seye Egemen RAB diyor ki, "Haydi, o kâhyaya, Sarayin sorumlusu Sevna'ya git ve de ki,
\par 16 'Burada ne isin var? Kimin var ki, kendine burada mezar kazdin, Yüksekte kendine mezar, kayada konut oydun?
\par 17 Ey güçlü kisi, RAB seni tuttugu gibi siddetle savuracak.
\par 18 Top gibi evirip çevirip Genis bir ülkeye firlatacak. Orada öleceksin, Gurur duydugun arabalarin orada kalacak. Efendinin evi için utanç nedenisin!
\par 19 Seni görevden alacak, Makamindan alasagi edecegim.
\par 20 "'O gün Hilkiya oglu kulum Elyakim'i çagirip
\par 21 Senin cüppeni ona giydirecegim. Senin kusaginla onu güçlendirip Yetkini ona verecegim. Yerusalim'de yasayanlara Ve Yahuda halkina o babalik yapacak.
\par 22 Davut'un evinin anahtarini ona teslim edecegim. Açtigini kimse kapayamayacak, Kapadigini kimse açamayacak.
\par 23 Onu saglam yere çakilmis çadir kazigi yapacagim, Ailesi için onur kürsüsü olacak.
\par 24 Ailenin agirligi -soyundan türeyen herkes- Taslardan kâselere kadar her küçük kap ona asilacak."
\par 25 Her Seye Egemen RAB diyor ki, "O gün saglam yere çakilmis kazik yerinden çikacak, kirilip düsecek, ona asilan yük de yok olacak." Çünkü RAB böyle diyor.

\chapter{23}

\par 1 Sur Kenti'yle ilgili bildiri: Ey ticaret gemileri, feryat edin! Çünkü Sur Kenti evleriyle, Limanlariyla birlikte yok oldu. Kittim'den size haber geldi.
\par 2 Ey kiyi halki ve denizcilerin zenginlestirdigi Sayda tüccarlari, susun!
\par 3 Sihor'un tahili, Nil'in ürünü Denizleri asar, Sur'a gelir saglardi. Uluslarin kâri ona akardi.
\par 4 Utan, ey Sayda, ey deniz kiyisindaki kale! Çünkü deniz sana sesleniyor: "Ne dogum agrisi çektim, ne de dogurdum. Ne delikanlilar büyüttüm, ne de kizlar."
\par 5 Sur'un haberi Misir'a ulasinca, Yüregi burkulacak insanlarin.
\par 6 Tarsis'e geçin, ey kiyida oturanlar, Feryat edin.
\par 7 Uzak ülkeleri yurt edinmis, Eglenceye düskün halkiniz, Eski, tarihsel kentiniz bu mu?
\par 8 Taçlar giydiren Sur'a karsi bu isi kim tasarladi? O kent ki, tüccarlari prenslerdi, Is adamlari dünyanin saygin kisileriydi.
\par 9 Görkeminin sonucu olan gururunu kirmak, Dünyaca ünlü bütün saygin kisilerini alçaltmak için Her Seye Egemen RAB tasarladi bunu.
\par 10 Kendi topraklarini Nil gibi basip geç, Ey Tarsis kizi, artik engel yok.
\par 11 RAB denizin üzerine elini uzatip ülkeleri titretti, Kenan kalelerinin yikilmasini buyurdu.
\par 12 "Eglencen sona erdi, ey Sayda, erden kiz!" dedi, "Kirletildin. Kalk, Kittim'e*fy* geç, Orada bile rahat yüzü görmeyeceksin."
\par 13 Kildan* ülkesine bak! O halk yok artik. Asurlular onlarin ülkesini yabanil hayvanlara verdi, Kusatma kuleleri diktiler, saraylarini soydular, Ülkeyi viraneye çevirdiler.
\par 14 Feryat edin, ey ticaret gemileri! Çünkü siginaginiz harap oldu.
\par 15 Bundan sonra Sur Kenti yetmis yil, bir kralin ömrü süresince unutulacak. Yetmis yil bitince, fahise için bestelenen türküdeki gibi Sur'a söyle denecek:
\par 16 "Bir lir al, dolas kenti, Ey sen, unutulmus fahise! Güzel çal seni animsamalari için, Türkü üstüne türkü söyle."
\par 17 Yetmis yil geçince RAB Sur'la ilgilenecek, ama Sur para için yine fahiselige dönecek. Dünyanin bütün kralliklariyla fahiselik edecek.
\par 18 Kentin ticaretten ve fuhustan kazandiklari RAB'be adanacak. Bunlar biriktirilmeyecek, hazineye konmayacak. Ticaretten kazandiklarini doyuncaya dek yesinler, güzel güzel giyinsinler diye RAB'bin önünde yasayanlara verilecek.

\chapter{24}

\par 1 Iste RAB yeryüzünü harap edip viraneye çevirecek, Yeryüzünü altüst edecek, Üzerinde yasayanlari darmadagin edecek.
\par 2 Ayrim yapilmayacak; Ne halkla kâhin arasinda, Ne köleyle efendi arasinda, Ne hizmetçiyle hanim arasinda, Ne aliciyla satici arasinda, Ne ödünç alanla ödünç veren arasinda, Ne faizciyle borç alan arasinda.
\par 3 Dünya tümüyle yagmalanip viraneye çevrilecek. RAB böyle söyledi.
\par 4 Dünya kuruyup büzülüyor, Yeryüzü solup büzülüyor, Dünyadaki soylular güçlerini yitiriyor.
\par 5 Dünyada yasayanlar onu kirletti. Çünkü Tanri'nin yasalarini çignediler, Kurallarini ayaklar altina aldilar, Ebedi antlasmayi bozdular.
\par 6 Bu yüzden lanet dünyayi yiyip bitirdi, Orada yasayanlar suçlarinin cezasini çekiyorlar. Yasayanlar bu nedenle yaniyor, pek azi kurtulacak.
\par 7 Yeni sarabin sonu geldi, Asmalar soldu, Bir zamanlar sevinçli olanlarin hepsi inliyor.
\par 8 Tefin coskun sesi kesildi, Eglenenlerin gürültüsü durdu, Lirin coskun sesi kesildi.
\par 9 Ezgi esliginde sarap içilmiyor artik, Içkinin tadi içene aci geliyor.
\par 10 Yikilan kent perisan durumda, Kimse girmesin diye her evin girisi kapandi.
\par 11 Insanlar sarap özlemiyle sokaklarda bagrisiyor, Sevinçten eser kalmadi, Dünyanin coskusu yok oldu.
\par 12 Kent viraneye döndü, Kapilari paramparça oldu.
\par 13 Çünkü zeytinler dökülsün diye dövülen agaç nasilsa, Bagbozumundan artakalan üzümler nasilsa, Dünyadaki bütün uluslar da öyle olacak.
\par 14 Sag kalanlar seslerini yükseltip Sevinç çigliklari atacak, Batida yasayanlar RAB'bin büyüklügü karsisinda Hayranlikla bagiracak.
\par 15 Onun için, doguda yasayanlar RAB'bi yüceltin, Deniz kiyisindakiler, Israil'in Tanrisi RAB'bin adini yüceltin.
\par 16 Dünyanin en uzak köselerinden ezgiler isitiyoruz: "Dogru Olan'a övgüler olsun!" Ama ben, "Bittim, bittim! Vay halime!" dedim, "Hainler hainliklerini sürdürüyor. Evet, hainler sürekli hainlik ediyorlar."
\par 17 Ey dünyada yasayanlar, Önünüzde dehset, çukur ve tuzak var.
\par 18 Dehset haberinden kaçan çukura düsecek, Çukurdan çikan tuzaga yakalanacak. Göklerin kapaklari açilacak, Dünyanin temelleri sarsilacak.
\par 19 Yeryüzü büsbütün çatlayip yarilacak, Sarsildikça sarsilacak.
\par 20 Dünya sarhos gibi yalpalayacak, Bir kulübe gibi sallanacak, Isyanlarinin agirligi altinda çökecek Ve bir daha kalkamayacak.
\par 21 O gün RAB yukarida, gökteki güçleri Ve asagida, yeryüzündeki krallari cezalandiracak.
\par 22 Zindana tikilan tutsaklar gibi Cezaevine kapatilacak Ve uzun süre sonra cezalandirilacaklar.
\par 23 Ayin yüzü kizaracak, günes utanacak. Çünkü Her Seye Egemen RAB Siyon Dagi'nda, Yerusalim'de krallik edecek. Halkin ileri gelenleri O'nun yüceligini görecek.

\chapter{25}

\par 1 Ya RAB, sensin benim Tanrim, Seni yüceltir, adini överim. Çünkü sen eskiden beri tasarladigin harikalari Tam bir sadakatle gerçeklestirdin.
\par 2 Kenti bir tas yiginina, Surlu kenti viraneye çevirdin. Yabancilarin kalesiydi, kent olmaktan çikti. Bir daha da onarilmayacak.
\par 3 Bundan ötürü güçlü uluslar seni onurlandiracak, Acimasiz uluslarin kentleri senden korkacak.
\par 4 Çünkü onlarin öfkesi Duvara çarpan saganak gibi yükselince, Sen yoksulun, sikinti içindeki düskünün kalesi, Saganaga karsi siginak, Sicaga karsi gölgelik oldun.
\par 5 Yabancilarin gürültüsünü çöl sicagi gibi bastirirsin. Bulutun gölgesi sicagi nasil kirarsa, Bu acimasiz adamlarin türküsü de öyle diniyor.
\par 6 Her Seye Egemen RAB bu dagda Bütün uluslara yagli yemeklerin Ve dinlendirilmis seçkin saraplarin sunuldugu Zengin bir sölen verecek.
\par 7 Bütün halklarin üzerindeki örtüyü, Bütün uluslarin üzerine örülmüs olan örtüyü Bu dagda kaldiracak.
\par 8 Ölümü sonsuza dek yutacak. Egemen RAB bütün yüzlerden gözyaslarini silecek. Halkinin utancini bütün yeryüzünden kaldiracak. Çünkü RAB böyle diyor.
\par 9 O gün diyecekler ki, "Iste Tanrimiz budur; O'na umut baglamistik, bizi kurtardi, RAB O'dur, O'na umut baglamistik, O'nun kurtarisiyla sevinip cosalim."
\par 10 RAB'bin eli bu dagin üzerinde kalacak, Ama Moav gübre çukurundaki saman gibi Kendi yerinde çignenecek.
\par 11 Oracikta yüzmek isteyen biri gibi ellerini uzatacak, Ama kurnazligina karsin RAB onun gururunu kiracak.
\par 12 RAB surlarindaki yüksek burçlari Devirip yikacak, yerle bir edecek.

\chapter{26}

\par 1 O gün Yahuda'da su ilahi söylenecek: Güçlü bir kentimiz var. Çünkü Tanri'nin kurtarisi Kente sur ve duvar gibidir.
\par 2 Açin kentin kapilarini, Sadik kalan dogru ulus içeri girsin.
\par 3 Sana güvendigi için Düsüncelerinde sarsilmaz olani Tam bir esenlik içinde korursun.
\par 4 RAB'be sonsuza dek güvenin, Çünkü RAB, evet RAB sonsuza dek kalici kayadir.
\par 5 Yüksekte oturani alçaltir, Yüce kenti yikar, Yerle bir eder.
\par 6 O kent ayak altinda, Mazlumlarin ayaklari, Yoksullarin adimlari altinda çignenecek.
\par 7 Dogru adamin yolu düzdür, Ey Dürüst Olan, dogru adamin yolunu sen düzlersin.
\par 8 Evet, ya RAB, ilkelerinin çizdigi yolda sana umut bagladik, Adin ve ünündür yüregimizin dilegi,
\par 9 Geceleri canim sana susar, Evet, içimde ruhum seni özler; Çünkü senin ilkelerin yeryüzünde oldukça, Orada oturanlar dogrulugu ögrenir.
\par 10 Kötüler lütfedilse bile dogrulugu ögrenmez. Dürüstlügün egemen oldugu diyarda haksizlik eder, RAB'bin büyüklügünü görmezler.
\par 11 Ya RAB, elin yükseldi, ama görmüyorlar, Halkin için gösterdigin gayreti görüp utansinlar. Evet, düsmanlarin için yaktigin ates onlari yiyip bitirecek.
\par 12 Ya RAB, bizi esenlige çikaracak sensin, Çünkü ne yaptiysak hepsi senin basarindir.
\par 13 Ey Tanrimiz RAB, senden baska efendiler bizi yönetti, Ama yalniz sana, senin adina yakaracagiz.
\par 14 O efendiler öldü, artik yasamiyorlar, Dirilmeyecek onlar. Çünkü onlari cezalandirip yok ettin, Anilmalarina son verdin.
\par 15 Ulusu çogalttin, ya RAB, Evet, ulusu çogalttin ve yüceltildin. Her yönde ülkenin sinirlarini genislettin.
\par 16 Ya RAB, sikintidayken seni aradilar. Onlari terbiye ettiginde sessizce yakararak içlerini döktüler.
\par 17 Dogum vakti yaklasan gebe kadin Çektigi sancidan ötürü nasil kivranir, feryat ederse, Senin önünde biz de öyle olduk, ya RAB.
\par 18 Gebe kaldik, kivrandik, Rüzgardan baska bir sey dogurmadik sanki. Ne dünyaya kurtulus saglayabildik, Ne de dünyada yasayanlari yasama kavusturabildik.
\par 19 Ama senin ölülerin yasayacak, Bedenleri dirilecek. Ey sizler, toprak altinda yatanlar, Uyanin, ezgiler söyleyin. Çünkü senin çiyin sabah çiyine benzer, Toprak ölülerini yasama kavusturacak.
\par 20 Haydi halkim, iç odalariniza girip Ardinizdan kapilarinizi kapatin, RAB'bin öfkesi geçene dek kisa süre gizlenin.
\par 21 Çünkü dünyada yasayanlari Suçlarindan ötürü cezalandirmak için RAB bulundugu yerden geliyor. Dünya üzerine dökülen kani açiga vuracak, Öldürülenleri artik saklamayacak.

\chapter{27}

\par 1 O gün RAB Livyatan'i*, o kaçan yilani, Evet Livyatan'i, o kivrila kivrila giden yilani Acimasiz, kocaman, güçlü kiliciyla cezalandiracak, Denizdeki canavari öldürecek.
\par 2 O gün RAB, "Sevdigim bag için ezgiler söyleyin" diyecek,
\par 3 "Ben RAB, bagin koruyucusuyum, Onu sürekli sularim. Kimse zarar vermesin diye Gece gündüz beklerim.
\par 4 Kizgin degilim. Keske karsima dikenli çalilar çiksa! Onlarin üzerine yürür, Tümünü atese verirdim.
\par 5 Ya da koruyuculuguma sarilsinlar, Barissinlar benimle, Evet, benimle barissinlar."
\par 6 Yakup soyu gelecekte kök salacak, Israil filizlenip çiçeklenecek, Yeryüzünü meyvesiyle dolduracak.
\par 7 RAB Israilliler'i, kendilerini cezalandiranlari cezalandirdigi gibi cezalandirdi mi? Ya da Israilliler'i baskalarini öldürdügü gibi öldürdü mü?
\par 8 RAB onlari yargiladi, Kovup sürgüne gönderdi. Dogu rüzgarinin estigi gün Onlari siddetli soluguyla savurdu.
\par 9 Böylece Yakup soyunun suçu bagislanacak. Günahlarinin kaldirilmasinin sonucu söyle olacak: Sunagin taslarini tebesir tasi gibi un ufak ettiklerinde Ne Asera* putu ne de buhur sunagi kalacak.
\par 10 Surlu kent terk edildi, Çöl kadar issiz, sahipsiz bir yurt oldu. Dana orada otlayip uzanacak, Filizlerini yiyip bitirecek.
\par 11 Kuruyan dallari koparilacak, Kadinlar gelip bunlari yakacaklar. Çünkü bu halk akilli bir halk degil. Bu yüzden onlari yaratan kendilerine acimayacak, Onlara biçim veren onlari kayirmayacak.
\par 12 Ey Israilogullari, o gün RAB gürül gürül akan Firat ile Misir Vadisi arasinda harman döver gibi, sizi birer birer toplayacak.
\par 13 Evet, o gün büyük bir boru çalinacak; Asur'da yitenlerle Misir'a sürgün edilenler gelip kutsal dagda, Yerusalim'de RAB'be tapinacaklar.

\chapter{28}

\par 1 Vay haline verimli vadinin basindaki kentin, Efrayimli sarhoslarin gurur tacinin! Saraba yenilmislerin yüce ve görkemli taci, solmakta olan çiçegi andiriyor.
\par 2 Rab'bin güçlü kudretli bir adami var. Dolu firtinasi gibi, harap eden kasirga gibi, silip süpüren güçlü sel gibi o kenti siddetle yere çalacak.
\par 3 Efrayimli sarhoslarin gurur taci ayaklar altinda çignenecek.
\par 4 Verimli vadinin basindaki kent, yüce ve görkemli taç, artik solmakta olan çiçegi andiran kent, mevsiminden önce olgunlasmis incir gibi görülür görülmez koparilip yutulacak.
\par 5 O gün Her Seye Egemen RAB, halkindan sag kalanlar için yücelik taci, güzellik çelengi olacak.
\par 6 Yargi kürsüsünde oturanlar için adalet ruhu, kent kapilarinda saldirilari geri püskürtenler için cesaret kaynagi olacak.
\par 7 Kâhinlerle peygamberler bile sarabin ve içkinin etkisiyle yalpalayip sendeliyor; içkinin etkisiyle yalpalayip sendeliyorlar, saraba yenik düsmüsler. Yanlis görümler görüyorlar, kararlarinda tutarsizlar.
\par 8 Sofralar kusmuk dolu, pislige bulasmamis yer yok!
\par 9 "Kimi egitmeye çalisiyor?" diyorlar, "Kime iletiyor bildirisini? Sütten yeni kesilmis, memeden yeni ayrilmis çocuklara mi?
\par 10 Çünkü bütün söyledigi buyruk üstüne buyruk, buyruk üstüne buyruk, kural üstüne kural, kural üstüne kural, biraz surdan, biraz burdan..."
\par 11 Öyle olsun, o zaman RAB bu halka yabanci dudaklarla, anlasilmaz bir dille seslenecek.
\par 12 Onlara, "Rahatlik budur, yorgunlarin rahat etmelerini saglayin, huzur budur" dedi, ama dinlemek istemediler.
\par 13 Bu yüzden RAB'bin sözü onlar için "Buyruk üstüne buyruk, buyruk üstüne buyruk, kural üstüne kural, kural üstüne kural, biraz surdan, biraz burdan"dir. Madem öyle, varsin sirtüstü düsüp yaralansinlar, kapana kisilip tutsak olsunlar.
\par 14 Bundan ötürü, ey alaycilar, Yerusalim'deki bu halki yöneten sizler, RAB'bin sözüne kulak verin.
\par 15 Söyle diyorsunuz: "Ölümle antlasma yaptik, ölüler diyariyla uyustuk; öyle ki, büyük bela ülkeden geçerken bize zarar vermeyecek. Çünkü yalanlari kendimize siginak yaptik, hilenin ardina gizlendik."
\par 16 Bu yüzden Egemen RAB diyor ki, "Iste Siyon'a saglam temel olarak bir tas, denenmis bir tas, degerli bir köse tasi yerlestiriyorum. Ona güvenen yenilmeyecek.
\par 17 Adaleti ölçü ipi, dogrulugu çekül yapacagim. Yalanlara dayanan siginagi dolu süpürüp götürecek, gizlendiginiz yerleri sel basacak.
\par 18 Ölümle yaptiginiz antlasma yürürlükten kaldirilacak, ölüler diyariyla uyusmaniz geçerli sayilmayacak. Büyük bela ülkeden geçerken sizi çigneyecek.
\par 19 Bu bela her geldiginde sizi süpürüp götürecek. Her gün, gece gündüz gelecek. Bu bildiriyi anlayan dehsete kapilacak.
\par 20 Yatak uzanamayacaginiz kadar kisa, örtü sarinamayacaginiz kadar dar olacak.
\par 21 Çünkü RAB, Perasim Dagi'nda oldugu gibi kalkacak, Givon Vadisi'nde oldugu gibi öfkelenecek. Ne kadar garip olsa da isini tamamlayacak, ne kadar tuhaf olsa da yapacagini yapacak.
\par 22 Alay etmeyin artik, yoksa zincirleriniz daha da kalinlasir. Çünkü bütün ülkenin kesin bir yikima ugrayacagini Rab'den, Her Seye Egemen RAB'den duydum.
\par 23 Kulak verin, sesimi isitin, dikkat edin, ne söyledigimi dinleyin.
\par 24 Çiftçi ekin ekmek için durmadan topragi sürer mi, boyuna eseleyip tirmiklar mi?
\par 25 Topragi düzledikten sonra çörekotunu, kimyonu serpmez mi? Bugdayi sira sira, arpayi ayirdigi yere, kizil bugdayi da onun yanina ekmez mi?
\par 26 Tanrisi ona uygun olani gösterir, onu egitir.
\par 27 Çünkü çörekotu harmanda keskin aletle dövülmez, kimyonun üzerinden tekerlekle geçilmez. Çörekotu degnekle, kimyon çubukla dövülür.
\par 28 Bugday ekmek yapmak için ögütülür, ama boyuna dövülmez. Harmanin üzerinden tekerlek ve atlar geçse de bugdayi ezmez.
\par 29 Bu isteki bilgelik de Her Seye Egemen RAB'den gelir. O'nun tasarilari harikadir, bilgelikte üstündür.

\chapter{29}

\par 1 Ariel, Ariel, Davut'un ordugah kurdugu kent, vay haline! Sen yila yil kat, bayramlarin süredursun.
\par 2 Ama seni sikintiya sokacagim. Feryat, figan edeceksin, Benim için sunak ocagi gibi olacaksin.
\par 3 Sana karsi çepeçevre ordugah kuracak, Çevreni rampalarla, kulelerle kusatacagim.
\par 4 Alçaltilacaksin, yerin altindan konusacak, Toz toprak içinden boguk boguk sesleneceksin. Sesin ölü sesi gibi yerden, Sözlerin fisilti gibi topragin içinden çikacak.
\par 5 Ama sayisiz düsmanlarin ince toz, Acimasiz ordulari savrulmus saman ufagi gibi olacak. Bir anda, ansizin,
\par 6 Her Seye Egemen RAB gök gürlemesiyle, Depremle, büyük gümbürtü, kasirga ve firtinayla, Her seyi yiyip bitiren ates aleviyle seni cezalandiracak.
\par 7 Sonra Ariel'e karsi savasan çok sayida ulus, Ona ve kalesine saldiranlarin hepsi, Onu sikintiya sokanlar bir rüya gibi, Gece görülen görüm gibi yok olup gidecekler.
\par 8 Rüyada yemek yedigini gören aç kisi, Uyandiginda hâlâ açtir; Rüyada su içtigini gören susuz kisi, Uyandiginda susuzluktan hâlâ baygindir. Iste Siyon Dagi'na karsi savasan Kalabalik uluslar da böyle olacak.
\par 9 Sasirin, saskina dönün, Kendinizi kör edin, görmez olun. Sarap içmeden sarhos olun, Içki içmeden sendeleyin.
\par 10 Çünkü RAB size uyusukluk ruhu verdi; Gözlerinizi mühürledi, ey peygamberler, Baslarinizi örttü, ey biliciler*.
\par 11 Sizin için bütün görüm Mühürlenmis bir kitabin sözleri gibi oldu. Insanlar böyle bir kitabi Okuma bilen birine verip, "Rica etsek sunu okur musun?" diye sorduklarinda, "Okuyamam, çünkü mühürlenmis" yanitini alirlar.
\par 12 Kitabi okuma bilmeyen birine verip, "Rica etsek sunu okur musun?" diye sorduklarinda ise, "Okuma bilmem" yanitini alirlar.
\par 13 Rab diyor ki, "Bu halk bana yaklasip Agizlariyla, dudaklariyla beni sayar, Ama yürekleri benden uzak. Benden korkmalari da Insanlardan ögrendikleri buyruklarin sonucudur.
\par 14 Onun için ben de bu halkin arasinda yine bir harika, Evet, sasilacak bir sey yapacagim. Bilgelerin bilgeligi yok olacak, Akillinin akli duracak."
\par 15 Tasarilarini RAB'den gizlemeye ugrasanlarin vay haline! Karanlikta is gören bu adamlar, "Bizi kim görecek, kim taniyacak?" diye düsünürler.
\par 16 Ne kadar ters düsünceler! Çömlekçi balçikla bir tutulur mu? Yapi, kendini yapan için, "Beni o yapmadi" diyebilir mi? Çömlek kendine biçim veren için, "O bir seyden anlamaz" diyebilir mi?
\par 17 Lübnan pek yakinda meyve bahçesine, Meyve bahçesi ormana dönmeyecek mi?
\par 18 O gün sagirlar kitabin sözlerini isitecek, Körler koyu karanlikta görecek.
\par 19 Düskünlerin RAB'de bulduklari sevinç artacak, Yoksullar Israil'in Kutsali sayesinde cosacak.
\par 20 Çünkü acimasizlar yok olacak, alaycilar silinecek, Kötülüge firsat kollayanlarin hepsi kesilip atilacak.
\par 21 Onlar ki, insani tek sözle davasinda suçlu çikarir, Kent kapisinda haksizi azarlayana tuzak kurar, Yok yere haklinin hakkini çignerler.
\par 22 Bundan dolayi, Ibrahim'i kurtarmis olan RAB Yakup soyuna diyor ki, "Yakup soyu artik utanmayacak, Yüzleri korkudan sararmayacak.
\par 23 Elimin yapiti olan çocuklarini Aralarinda gördüklerinde Adimi kutsal sayacaklar; Evet, Yakup'un Kutsali'ni kutsal sayacak, Israil'in Tanrisi'ndan korkacaklar.
\par 24 Yoldan sapmis olanlar kavrayisa, Yakinip duranlar bilgiye kavusacak."

\chapter{30}

\par 1 RAB, "Vay haline bu dikbasli soyun!" diyor, "Benim degil, kendi tasarilarini yerine getirip Ruhuma aykiri anlasmalar yaparak Günah üstüne günah isliyorlar.
\par 2 Bana danismadan firavunun korumasi altina girmek, Misir'in gölgesine siginmak için oraya gidiyorlar.
\par 3 Ne var ki, firavunun korumasi onlar için utanç, Misir'in gölgesine siginmalari onlar için rezillik olacak.
\par 4 Önderleri Soan'da oldugu, Elçileri Hanes'e ulastigi halde,
\par 5 Kendilerine yarari olmayan bir halk yüzünden hepsi utanacak. O halkin onlara ne yardimi ne de yarari olacak, Ancak onlari utandirip rezil edecek."
\par 6 Negev'deki hayvanlara iliskin bildiri: "Elçiler erkek ve disi aslanlarin, Engereklerin, uçan yilanlarin yasadigi Çetin ve sikintili bir bölgeden geçerler. Servetlerini eseklerin sirtina, Hazinelerini develerin hörgücüne yükleyip Kendilerine hiç yarari olmayan halka tasirlar.
\par 7 Misir'in yardimi bos ve yararsizdir, Bu yüzden Misir'a 'Haylaz Rahav adini verdim.
\par 8 "Simdi git, söyledigimi onlarin önünde Bir levhaya yazip kitaba geçir ki, Gelecekte kalici bir tanik olsun.
\par 9 Çünkü o asi bir halk, yalanci bir soy, RAB'bin yasasini duymak istemeyen bir soydur.
\par 10 Bilicilere*, 'Artik görüm görmeyin, Görenlere, 'Bizim için dogru seyler görmeyin, Bize güzel seyler söyleyin, asilsiz seyler açiklayin diyorlar,
\par 11 'Yoldan çekilin, yolu açin, Bizi Israil'in Kutsali'yla yüzlestirmekten vazgeçin."
\par 12 Bu nedenle Israil'in Kutsali diyor ki, "Madem bu bildiriyi reddettiniz, Baskiya ve hileye güvenip dayandiniz;
\par 13 Bu suçunuz yüksek bir surda Sirt veren çatlaga benziyor. Böyle bir sur birdenbire yikiliverir.
\par 14 O, toprak çömlek gibi parçalanacak. Parçalanmasi öyle siddetli olacak ki, Ocaktan ates almaya ya da sarniçtan su çikarmaya Yetecek büyüklükte bir parça kalmayacak."
\par 15 Egemen RAB, Israil'in Kutsali söyle diyor: "Bana dönün, huzur bulun, kurtulursunuz. Kaygilanmayin, bana güvenin, güçlü olursunuz. Ama bunu yapmak istemiyorsunuz.
\par 16 'Hayir, atlara binip kaçariz diyorsunuz, Bu yüzden kaçmak zorunda kalacaksiniz. 'Hizli atlara bineriz diyorsunuz, Bu yüzden sizi kovalayanlar da hizli olacak.
\par 17 Bir kisinin tehdidiyle bin kisi kaçacak, Bes kisinin tehdidiyle hepiniz kaçacaksiniz; Dag basinda bir gönder, Tepede bir sancak gibi kalana dek kaçacaksiniz.
\par 18 "Yine de RAB size lütfetmeyi özlemle bekliyor, Size merhamet göstermek için harekete geçiyor. Çünkü RAB adil Tanri'dir. Ne mutlu O'nu özlemle bekleyenlere!"
\par 19 "Ey Yerusalim'de oturan Siyon halki, Artik aglamayacaksin! Feryat ettiginde Rab sana nasil da lütfedecek! Feryadini duyar duymaz seni yanitlayacak.
\par 20 Rab ekmegi sikintiyla, Suyu cefayla verse de, Ögretmeniniz artik gizlenmeyecek, Gözünüzle göreceksiniz onu.
\par 21 Saga ya da sola sapacaginiz zaman, Arkanizdan, 'Yol budur, bu yoldan gidin Diyen sesini duyacaksiniz.
\par 22 Gümüs kapli oyma putlarinizi, Altin kaplama dökme putlarinizi 'Kirli ilan edecek, Kirli bir âdet bezi gibi atip 'Defol diyeceksiniz.
\par 23 Rab topraga ektiginiz tohum için yagmur verecek, Topragin ürünü olan yiyecek bol ve zengin olacak. O gün sigirlariniz genis otlaklarda otlanacak.
\par 24 Topragi isleyen öküzlerle esekler Kürekle, yabayla savrulmus, Tuzlanmis yem yiyecekler.
\par 25 Kalelerin düstügü o büyük kiyim günü Her yüksek dagda, her yüce tepede Akarsular olacak.
\par 26 RAB halkinin kiriklarini sardigi, Vurusuyla açtigi yaralari iyilestirdigi gün, Ay günes gibi parlayacak, Günes yedi kat, yedi günün toplam parlakligi kadar parlak olacak."
\par 27 Bakin, RAB'bin kendisi uzaktan geliyor, Kizgin öfkeyle kara bulut içinde. Dudaklari gazap dolu, Dili her seyi yiyip bitiren ates sanki.
\par 28 Solugu adam boynuna dek yükselmis taskin irmak gibi. Uluslari elekten geçirecek, degersizleri ayiracak, Halklarin agzina yoldan saptiran bir gem takacak.
\par 29 Ama sizler bayram gecesini kutlar gibi Ezgiler söyleyeceksiniz. RAB'bin dagina, Israil'in Kayasi'na Kaval esliginde çiktiginiz gibi Içten sevineceksiniz.
\par 30 RAB heybetli sesini isittirecek; Kizgin öfkeyle, her seyi yiyip bitiren ates aleviyle, Saganak yagmurla, firtina ve doluyla Bileginin gücünü gösterecek.
\par 31 Asur RAB'bin sesiyle dehsete düsecek, O'nun degnegiyle vurulacak.
\par 32 RAB'bin terbiye degnegiyle onlara indirdigi her darbeye Tef ve lir eslik edecek. RAB silahlarini savura savura onlarla savasacak.
\par 33 Tofet* çoktan hazirlandi, Evet, kral için hazirlandi. Genis ve yüksektir odun yigini, Atesi, odunu boldur. RAB kizgin kükürt selini andiran Soluguyla tutusturacak onu.

\chapter{31}

\par 1 Vay haline yardim bulmak için Misir'a inenlerin! Atlara, çok sayidaki savas arabalarina, Kalabalik atlilara güveniyorlar, Ama Israil'in Kutsali'na güvenmiyor, RAB'be yönelmiyorlar.
\par 2 Oysa bilge olan RAB'dir. Felaket getirebilir ve sözünü geri almaz. Kötülük yapan soya, Suç isleyenlerin yardimina karsi çikar.
\par 3 Misirlilar Tanri degil, insandir, Atlari da ruh degil, et ve kemiktir. RAB'bin eli kalkinca yardim eden tökezler, Yardim gören düser, hep birlikte yok olurlar.
\par 4 Çünkü RAB bana dedi ki, "Avinin basinda homurdanan aslan Bir araya çagrilan çobanlar toplulugunun Bagirip çagirmasindan yilmadigi, gürültüsüne aldirmadigi gibi, Her Seye Egemen RAB de Siyon Dagi'nin doruguna inip savasacak.
\par 5 Her Seye Egemen RAB Kanat açmis kuslar gibi koruyacak Yerusalim'i. Koruyup özgür kilacak, esirgeyip kurtaracak onu."
\par 6 Ey Israilogullari, Bunca vefasizlik ettiginiz RAB'be dönün.
\par 7 Çünkü hepiniz günahkâr ellerinizle yaptiginiz Altin ve gümüs putlari o gün reddedip atacaksiniz.
\par 8 Asur kilica yenik düsecek, Ama insan kilicina degil. Halki kiliçtan geçirilecek, Ama bu insan kilici olmayacak. Kimileri kaçip kurtulacak, Gençleri de angaryaya kosulacak.
\par 9 Asur Krali dehset içinde kaçacak, Önderleri sancagi görünce dehsete kapilacak. Siyon'da atesi, Yerusalim'de ocagi bulunan RAB söylüyor bunlari.

\chapter{32}

\par 1 Iste kral dogrulukla krallik yapacak, Önderler adaletle yönetecek.
\par 2 Her biri rüzgara karsi bir siginak, Firtinaya karsi bir barinak, çölde akarsu, Çorak yerde gölge salan Büyük bir kaya gibi olacak.
\par 3 Artik görenlerin gözleri kapanmayacak, Dinleyenler kulak kesilecek.
\par 4 Düsüncesizin akli bilgiye erecek, Kekeme açik seçik, akici konusacak.
\par 5 Artik budalaya soylu, Alçaga saygin denmeyecek.
\par 6 Çünkü budala saçmaliyor, Akli fikri hep kötülükte. Isi gücü fesat islemek, RAB'be iliskin yanlis sözler söylemek, Açlari aç birakmak, Susamislardan suyu esirgemek.
\par 7 Alçagin yöntemleri kötüdür; Yoksul davasinda hakli olsa da Onu yalanlarla yok etmek için Kötü düzenler tasarlar.
\par 8 Soylu kisiyse soylu seyler tasarlar, Yaptigi soylu islerle ayakta kalir.
\par 9 Ey tasasizca yasayan kadinlar, Kalkin, sesimi isitin; Ey kaygisiz kizlar, sözüme kulak verin!
\par 10 Bir yil kadar sonra sarsilacaksiniz, Ey kaygisiz kadinlar. Çünkü bagbozumu olmayacak, Devsirecek meyve bulunmayacak.
\par 11 Titreyin, ey tasasizca yasayan kadinlar, Sarsilin, ey kaygisizlar. Giysilerinizi çikarin, soyunup belinize çul kusanin.
\par 12 Güzel tarlalar, verimli asmalar, Halkimin diken ve çali bitmis topraklari için, Neseli kentteki mutluluk dolu evler için gögsünüzü dövün.
\par 14 Çünkü saray issiz, Kalabalik kent bombos kalacak. Ofel Mahallesi'yle gözcü kulesi Bir çayirliga dönecek; Yaban eseklerinin keyifle gezindigi, Sürülerin otladigi bir yer olacak.
\par 15 Ta ki yukaridan üzerimize ruh dökülene dek; O zaman çöl meyve bahçesine, Meyve bahçesi ormana dönecek.
\par 16 O zaman adalet çöle dek yayilacak, Dogruluk meyve bahçesinde yurt bulacak.
\par 17 Dogrulugun ürünü esenlik, Sonucu, sürekli huzur ve güven olacaktir.
\par 18 Halkim esenlik dolu evlerde, Güvenli ve rahat yerlerde yasayacak.
\par 19 Dolu ormanlari harap etse, Kent yerle bir olsa da,
\par 20 Sulak yerde tohum eken, Sigirini, esegini özgürce çayira salan sizlere ne mutlu!

\chapter{33}

\par 1 Vay sana, yikip yok eden Ama kendisi yikilmamis olan! Vay sana, ihanete ugramamis hain! Yikima son verir vermez sen de yikilacaksin, Ihanetin sona erer ermez sen de ihanete ugrayacaksin.
\par 2 Ya RAB, lütfet bize, Çünkü sana umut bagladik, Gün be gün gücümüz ol! Sikintiya düstügümüzde bizi kurtar.
\par 3 Kükreyisinden halklar kaçisir, Sen ayaga kalkinca uluslar darmadagin olur.
\par 4 Çekirgeler tarlayi nasil yagmalarsa, Ganimetiniz de öyle yagmalanacak, ey uluslar. Malinizin üzerine çekirge sürüsü gibi saldiracaklar.
\par 5 Yükseklerde oturan RAB yücedir, Siyon'u adalet ve dogrulukla doldurur.
\par 6 Yasadiginiz sürenin güvencesi O'dur. Bol bol kurtulus, bilgi ve bilgelik saglayacak. Halkin hazinesi RAB korkusudur.
\par 7 Iste, en yigitleri sokaklarda feryat ediyor, Baris elçileri aci aci agliyor.
\par 8 Anayollar bombos, Yolculuk eden kimse kalmadi. Düsman antlasmayi bozdu, kentleri hor gördü, Insanlari hiçe saydi.
\par 9 Ülke yas tutuyor, zayifliyor. Lübnan utancindan soldu, Saron Ovasi çöle döndü, Basan ve Karmel'de agaçlar yaprak döküyor.
\par 10 RAB diyor ki, "Simdi harekete geçecegim, Ne denli yüce ve üstün oldugumu gösterecegim.
\par 11 Samana gebe kalip aniz doguracaksiniz, Solugunuz sizi yiyip bitiren bir ates olacak.
\par 12 Halklar yanip kül olacak, Kesilip yakilan dikenli çali gibi olacak.
\par 13 "Ey uzaktakiler, ne yaptigimi isitin, Ey yakindakiler, gücümü anlayin."
\par 14 Siyon'daki günahkârlar dehset içinde, Tanrisizlari titreme aldi. "Her seyi yiyip bitiren atesin yaninda Hangimiz oturabilir? Sonsuza dek sönmeyecek alevin yaninda Hangimiz yasayabilir?" diye soruyorlar.
\par 15 Ama dogru yolda yürüyüp dogru dürüst konusan, Zorbalikla edinilen kazanci reddeden, Elini rüsvetten uzak tutan, Kan dökenlerin telkinlerine kulak vermeyen, Kötülük görmeye dayanamayan,
\par 16 Yükseklerde oturacak; Uçurumun basindaki kaleler onun korunagi olacak, Ekmegi saglanacak, hiç susuz kalmayacak.
\par 17 Krali bütün güzelligiyle görecek, Uçsuz bucaksiz ülkeyi seyredeceksin.
\par 18 "Haraci tartip kaydeden nerede, Kulelerden sorumlu olan nerede?" diyerek Geçmisteki dehsetli günleri düsüneceksin.
\par 19 Garip, anlasilmaz bir yabanci dil konusan O küstah halki artik görmeyeceksin.
\par 20 Bayramlarimizin kenti olan Siyon'a bak! Yerusalim'i bir esenlik yurdu, Kaziklari asla yerinden sökülmeyen, Gergi ipleri hiç kopmayan, Sarsilmaz bir çadir olarak görecek gözlerin.
\par 21 Heybetli RAB orada bizden yana olacak. Orasi genis irmaklarin, çaylarin yeri olacak. Bunlarin üzerinden ne kürekli tekneler, Ne de büyük gemiler geçecek.
\par 22 Çünkü yargicimiz RAB'dir; Yasamizi koyan RAB'dir, Kralimiz RAB'dir, bizi O kurtaracak.
\par 23 Senin gemilerinin halatlari gevsedi, Direklerinin dibini pekistirmediler, Yelkenleri açmadilar. O zaman büyük ganimet paylasilacak, Topallar bile yagmaya katilacak.
\par 24 Siyon'da oturan hiç kimse "Hastayim" demeyecek, Orada yasayan halkin suçu bagislanacak.

\chapter{34}

\par 1 Ey uluslar, isitmek için yaklasin! Ey halklar, kulak verin! Dünya ve üzerindeki herkes, Yeryüzü ve ondan türeyenlerin hepsi isitsin!
\par 2 RAB bütün uluslara öfkelendi, Onlarin ordularina karsi gazaba geldi. Onlari tümüyle mahvolmaya, Bogazlanmaya teslim edecek.
\par 3 Ölüleri disari atilacak, Pis kokacak cesetleri; Daglar kanlariyla sulanacak.
\par 4 Bütün gök cisimleri küçülecek, Gökler bir tomar gibi dürülecek; Gök cisimleri, asma yapragi, Incir yapragi gibi dökülecek.
\par 5 "Kilicim göklerde kanincaya kadar içti. Simdi de Edom'un, Tümüyle yikmaya karar verdigim halkin Üzerine inecek" diyor RAB.
\par 6 RAB'bin kilici kana, Kuzu ve teke kanina doydu; Yagla, koç böbreklerinin yagiyla kaplandi. Çünkü RAB'bin Bosra'da bir kurbani, Edom'da büyük bir kiyimi var.
\par 7 Onlarla birlikte yaban öküzleri, Körpe bogalarla güçlü bogalar da yere serilecek. Topraklari kana doyacak, yagla sulanacak.
\par 8 Çünkü RAB'bin bir öç günü, Siyon'un davasini güdecegi bir karsilik yili olacak.
\par 9 Edom dereleri zifte, topragi kükürde dönecek; ülkenin her yani yanan zift olacak.
\par 10 Zift gece gündüz sönmeyecek, dumani hep tütecek. Ülke kusaklar boyu issiz kalacak, sonsuza dek oradan kimse geçmeyecek.
\par 11 Baykuslarin mülkü olacak orasi, büyük baykuslarla kargalar yasayacak orada. RAB, Edom'un üzerine kargasa ipini, bosluk çekülünü gerecek.
\par 12 Kral ilan edebilecekleri soylular kalmayacak, bütün önderlerinin sonu gelecek.
\par 13 Saraylarinda dikenler, kalelerinde isirganlarla bögürtlenler bitecek. Orasi çakallarin barinagi, baykuslarin yurdu olacak.
\par 14 Yabanil hayvanlarla sirtlanlar orada bulusacak, tekeler karsilikli bögürecek. Lilit oraya yerlesip rahata kavusacak.
\par 15 Baykuslar orada yuva kurup yumurtlayacak, kuluçkaya yatip yavrularini kendi gölgelerinde toplayacak. Çaylaklar eslesip orada toplanacaklar.
\par 16 RAB'bin kitabini okuyup arastirin: Bunlardan hiçbiri eksik kalmayacak, esten yoksun hiçbir hayvan olmayacak. Çünkü bu buyruk RAB'bin agzindan çikti, Ruhu da onlari toplayacak.
\par 17 RAB onlar için kura çekti, ülkeyi ölçüp aralarinda pay etti. Orayi sonsuza dek sahiplenip kusaklar boyu orada yasayacaklar.

\chapter{35}

\par 1 Çöl ve kurak toprak sevinecek, Bozkir cosup çigdem gibi çiçeklenecek.
\par 2 Her yani çiçeklenip sevinçle cosacak, Sevincini haykiracak. Lübnan'in yüceligi, Karmel ve Saron'un görkemi ona verilecek. Insanlar RAB'bin yüceligini, Tanrimiz'in görkemini görecek.
\par 3 Gevsek elleri güçlendirin, Pekistirin çözülen dizleri.
\par 4 Yüregi kaygili olanlara, "Güçlü olun, korkmayin" deyin, "Iste Tanriniz geliyor! Öç almaya, karsilik vermeye geliyor. Sizi O kurtaracak."
\par 5 O zaman körlerin gözleri, Sagirlarin kulaklari açilacak;
\par 6 Topallar geyik gibi siçrayacak, Sevinçle haykiracak dilsizlerin dili. Çünkü çölde sular fiskiracak, Irmaklar akacak bozkirda.
\par 7 Kizgin kum havuza, Susuz toprak pinara dönüsecek. Çakallarin yattigi yerlerde Kamis, saz ve ot bitecek.
\par 8 Orada bir yol, bir anayol olacak, "Kutsal yol" diye anilacak, Murdar* kisiler geçemeyecek oradan. O yol kurtulmus olanlarin yoludur. O yolda yürüyenler, bön kisiler de olsa yoldan sapmayacak.
\par 9 Aslan olmayacak orada, Yirtici hayvan o yola çikmayacak; Orada bulunmayacaklar. Ancak kurtulmus olanlar yürüyecek o yolda.
\par 10 RAB'bin kurtardiklari dönecek, Sevinçle haykirarak Siyon'a varacaklar. Yüzlerinde sonsuz sevinç olacak. Onlarin olacak cosku ve sevinç, Üzüntü ve inilti kaçacak.

\chapter{36}

\par 1 Hizkiya'nin kralliginin on dördüncü yilinda Asur Krali Sanherib, Yahuda'nin surlu kentlerine saldirip hepsini ele geçirdi.
\par 2 Komutanini büyük bir orduyla Lakis'ten Yerusalim'e, Kral Hizkiya'ya gönderdi. Komutan Çirpici Tarlasi yolunda, Yukari Havuz'un su yolunun yaninda durdu.
\par 3 Saray sorumlusu Hilkiya oglu Elyakim, Yazman Sevna ve devlet tarihçisi Asaf oglu Yoah onu karsilamaya çikti.
\par 4 Komutan onlara söyle dedi: "Hizkiya'ya söyleyin. 'Büyük kral, Asur Krali diyor ki: Güvendigin sey ne, neye güveniyorsun?
\par 5 Savas tasarilarin ve gücün bos laftan baska birsey degil diyorum. Kime güveniyorsun da bana karsi ayaklaniyorsun?
\par 6 Iste sen su kirik kamis degnege, Misir'a güveniyorsun. Bu degnek kendisine yaslanan herkesin eline batar, deler. Firavun da kendisine güvenenler için böyledir.
\par 7 Yoksa bana, Tanrimiz RAB'be güveniyoruz mu diyeceksiniz? Hizkiya'nin Yahuda ve Yerusalim halkina, yalniz bu sunagin önünde tapinacaksiniz diyerek tapinma yerlerini, sunaklarini ortadan kaldirdigi Tanri degil mi bu?
\par 8 "Haydi, efendim Asur Krali'yla bahse giris. Binicileri saglayabilirsen sana iki bin at veririm.
\par 9 Misir'in savas arabalariyla atlilari saglayacagina güvensen bile, efendimin en küçük rütbeli komutanlarindan birini yenemezsin!
\par 10 Dahasi var: RAB'bin buyrugu olmadan mi saldirip ülkeyi yikmak için yola çiktigimi saniyorsun? RAB, 'Git, o ülkeyi yik dedi."
\par 11 Elyakim, Sevna ve Yoah, "Lütfen biz kullarinla Aramice* konus" diye karsilik verdiler, "Çünkü biz bu dili anlariz. Yahudice konusma. Surlarin üzerindeki halk bizi dinliyor."
\par 12 Komutan, "Efendim bu sözleri yalniz size ve efendinize söyleyeyim diye mi gönderdi beni?" dedi, "Surlarin üzerinde oturan bu halka, sizin gibi diskisini yemek, idrarini içmek zorunda kalacak olan herkese gönderdi."
\par 13 Sonra ayaga kalkip Yahudi dilinde bagirdi: "Büyük kralin, Asur Krali'nin sözlerini dinleyin!
\par 14 Kral diyor ki, 'Hizkiya sizi aldatmasin, o sizi kurtaramaz.
\par 15 RAB bizi mutlaka kurtaracak, bu kent Asur Krali'nin eline geçmeyecek diyen Hizkiya'ya kanmayin, RAB'be güvenmeyin.
\par 16 Hizkiya'yi dinlemeyin. Çünkü Asur Krali diyor ki, 'Teslim olun, bana gelin. Böylece ben gelip sizi kendi ülkeniz gibi bir ülkeye -tahil ve yeni sarap, ekmek ve üzüm dolu bir ülkeye- götürene kadar herkes kendi asmasindan, kendi incir agacindan yiyecek, kendi sarnicindan içecek.
\par 18 "'Hizkiya, RAB bizi kurtaracak diyerek sizi aldatmasin. Uluslarin ilahlari ülkelerini Asur Krali'nin elinden kurtarabildi mi?
\par 19 Hani nerede Hama'nin, Arpat'in ilahlari? Sefarvayim'in*fi* ilahlari nerede? Samiriye'yi elimden kurtarabildiler mi?
\par 20 Bütün bu ülkelerin ilahlarindan hangisi ülkesini elimden kurtardi ki, RAB Yerusalim'i elimden kurtarabilsin?"
\par 21 Herkes sustu, komutana tek sözle bile karsilik veren olmadi. Çünkü Kral Hizkiya, "Karsilik vermeyin" diye buyurmustu.
\par 22 Sonra saray sorumlusu Hilkiya oglu Elyakim, Yazman Sevna ve devlet tarihçisi Asaf oglu Yoah giysilerini yirttilar ve gidip komutanin söylediklerini Hizkiya'ya bildirdiler.

\chapter{37}

\par 1 Kral Hizkiya olanlari duyunca giysilerini yirtti, çul kusanip RAB'bin Tapinagi'na girdi.
\par 2 Saray sorumlusu Elyakim'i, Yazman Sevna'yi ve ileri gelen kâhinleri Amots oglu Peygamber Yesaya'ya gönderdi. Hepsi çul kusanmisti.
\par 3 Yesaya'ya söyle dediler: "Hizkiya diyor ki, 'Bugün sikinti, azar ve utanç günü. Çünkü çocuklarin dogum vakti geldi, ama doguracak güç yok.
\par 4 Yasayan Tanri'yi asagilamak için efendisi Asur Krali'nin gönderdigi komutanin söylediklerini belki Tanrin RAB duyar da duydugu sözlerden ötürü onlari cezalandirir. Bu nedenle sag kalanlarimiz için dua et."
\par 5 Yesaya, Kral Hizkiya'dan gelen görevlilere söyle dedi: "Efendinize sunlari söyleyin: 'RAB diyor ki, Asur Krali'nin adamlarindan benimle ilgili duydugunuz küfürlerden korkma.
\par 7 Onun içine öyle bir ruh koyacagim ki, bir haber üzerine kendi ülkesine dönecek. Orada onu kiliçla öldürtecegim."
\par 8 Komutan, Asur Krali'nin Lakis'ten ayrilip Livna'ya karsi savastigini duydu. Krala danismak için oraya gitti.
\par 9 Kûs* Krali Tirhaka'nin kendisiyle savasmak üzere yola çiktigini haber alan Asur Krali, Hizkiya'ya ulaklar göndererek söyle dedi:
\par 10 "Yahuda Krali Hizkiya'ya deyin ki, 'Güvendigin Tanrin, Yerusalim Asur Krali'nin eline teslim edilmeyecek diyerek seni aldatmasin.
\par 11 Asur krallarinin bütün ülkelere neler yaptigini, onlari nasil yerle bir ettigini duymussundur. Sen kurtulacagini mi saniyorsun?
\par 12 Atalarimin yok ettigi uluslari -Gozanlilar'i, Harranlilar'i, Resefliler'i, Telassar'da yasayan Edenliler'i- ilahlari kurtarabildi mi?
\par 13 Hani nerede Hama ve Arpat krallari? Lair, Sefarvayim, Hena, Ivva krallari nerede?"
\par 14 Hizkiya mektubu ulaklarin elinden alip okuduktan sonra RAB'bin Tapinagi'na çikti. RAB'bin önünde mektubu yere yayarak
\par 15 söyle dua etti:
\par 16 "Ey Keruvlar* arasinda taht kuran Israil'in Tanrisi, Her Seye Egemen RAB, bütün dünya kralliklarinin tek Tanrisi sensin. Yeri, gögü sen yarattin.
\par 17 Ya RAB, kulak ver de isit, gözlerini aç da gör, ya RAB; Sanherib'in söylediklerini, yasayan Tanri'yi nasil asagiladigini duy.
\par 18 Ya RAB, gerçek su ki, Asur krallari bütün uluslari ve ülkelerini viraneye çevirdiler.
\par 19 Ilahlarini yakip yok ettiler. Çünkü onlar tanri degil, insan eliyle biçimlendirilmis tahta ve taslardi.
\par 20 Ya RAB Tanrimiz, simdi bizi Sanherib'in elinden kurtar ki, bütün dünya kralliklari senin tek RAB oldugunu anlasin."
\par 21 Bunun üzerine Amots oglu Yesaya, Hizkiya'ya su haberi gönderdi: "Israil'in Tanrisi RAB söyle diyor: 'Asur Krali Sanherib'le ilgili olarak bana yalvardigin için diyorum ki, "'Erden kiz Siyon seni hor görüyor, Alay ediyor seninle. Yerusalim kizi* ardindan alayla bas salliyor.
\par 23 Sen kimi asagiladin, kime küfrettin? Kime sesini yükselttin? Israil'in Kutsali'na tepeden baktin!
\par 24 Usaklarin araciligiyla Rab'bi asagiladin. Bir sürü savas arabamla daglarin tepesine, Lübnan'in doruklarina çiktim, dedin. Yüksek sedir agaçlarini, seçme çamlarini kestim, Lübnan'in en uzak tepelerine, Gür ormanlarina ulastim.
\par 25 Kuyular kazdim, sular içtim, Misir'in bütün irmaklarini ayagimin tabaniyla kuruttum, dedin.
\par 26 "'Bütün bunlari çoktan yaptigimi, Çok önceden tasarladigimi duymadin mi? Surlu kentleri enkaz yiginlarina çevirmeni Simdi ben gerçeklestirdim.
\par 27 O kentlerde yasayanlarin kolu kanadi kirildi. Yilginlik ve utanç içindeydiler; Kir otuna, körpe filizlere, Damlarda büyümeden kavrulup giden ota döndüler.
\par 28 Senin oturusunu, kalkisini, Ne zaman gidip geldigini, Bana nasil öfkelendigini biliyorum.
\par 29 Bana duydugun öfkeden, Kulagima erisen küstahligindan ötürü Halkami burnuna, gemimi agzina takacak, Seni geldigin yoldan geri çevirecegim.
\par 30 "'Senin için belirti su olacak, ey Hizkiya: Bu yil kendiliginden yetiseni yiyeceksiniz, Ikinci yil ise ardindan biteni. Üçüncü yil ekip biçin, Baglar dikip ürününü yiyin.
\par 31 Yahudalilar'in kurtulup sag kalanlari Yine asagiya dogru kök salacak, Yukariya dogru meyve verecek.
\par 32 Çünkü sag kalanlar Yerusalim'den, Kurtulanlar Siyon Dagi'ndan çikacak. Her Seye Egemen RAB'bin gayretiyle olacak bu.
\par 33 "Bundan dolayi RAB Asur Krali'na iliskin söyle diyor: 'Bu kente girmeyecek, ok atmayacak. Kente kalkanla yaklasmayacak, Karsisinda rampa kurmayacak.
\par 34 Geldigi yoldan dönecek ve kente girmeyecek diyor RAB,
\par 35 'Kendim için ve kulum Davut'un hatiri için Bu kenti savunup kurtaracagim diyor."
\par 36 RAB'bin melegi gidip Asur ordugahinda yüz seksen bes bin kisiyi öldürdü. Ertesi sabah uyananlar salt cesetlerle karsilastilar.
\par 37 Bunun üzerine Asur Krali Sanherib ordugahini birakip çekildi. Ninova'ya döndü ve orada kaldi.
\par 38 Bir gün ilahi Nisrok'un tapinaginda tapinirken, ogullarindan Adrammelek'le Sareser, onu kiliçla öldürüp Ararat ülkesine kaçtilar. Yerine oglu Esarhaddon kral oldu.

\chapter{38}

\par 1 O günlerde Hizkiya ölümcül bir hastaliga yakalandi. Amots oglu Peygamber Yesaya ona gidip söyle dedi: "RAB diyor ki, 'Ev islerini düzene sok. Çünkü iyilesmeyecek, öleceksin."
\par 2 Hizkiya yüzünü duvara dönüp RAB'be yalvardi:
\par 3 "Ya RAB, yürekten bir sadakatle önünde nasil yasadigimi, gözünde iyi olani yaptigimi animsa lütfen." Sonra aci aci aglamaya basladi.
\par 4 Bunun üzerine RAB Yesaya'ya seslendi:
\par 5 "Git, Hizkiya'ya sunu söyle: 'Atan Davut'un Tanrisi RAB diyor ki: Duani isittim, gözyaslarini gördüm. Bak, ömrünü on bes yil daha uzatacagim.
\par 6 Bu kenti savunacak, seni de kenti de Asur Krali'nin elinden kurtaracagim.
\par 7 Sözümü gerçeklestirecegime iliskin sana verecegim belirti su olacak:
\par 8 RAB, batmakta olan günesin Ahaz'in insa ettigi basamaklarin üzerine düsen gölgesini on basamak kisaltacak." Böylece batmakta olan günesin gölgesi on basamak kisaldi.
\par 9 Yahuda Krali Hizkiya hastalanip iyilestikten sonra sunlari yazdi:
\par 10 "Hayatimin baharinda ölüler diyarinin kapilarindan geçip Ömrümün geri kalan yillarindan yoksun mu kalmaliyim?" demistim,
\par 11 "Yasayanlar diyarinda RAB'bi, evet, RAB'bi bir daha görmeyecegim, Bu dünyada yasayanlar gibi insan yüzü görmeyecegim bir daha.
\par 12 Evim bir çoban çadiri gibi bozuldu, alindi elimden. Dokumaci gibi dürdüm yasamimi, RAB tezgahtan beni kesti, Bir gün içinde sonumu getiriverdi.
\par 13 Sabirla bekledim sabaha kadar, RAB bir aslan gibi kirdi bütün kemiklerimi, Bir gün içinde sonumu getiriverdi.
\par 14 Kirlangiç gibi, turna gibi aci aci öttüm, Güvercin gibi inledim, gözlerim yoruldu yukari bakmaktan. Ya Rab, eziyet çekiyorum, Yardim et bana.
\par 15 "Ne diyeyim? Bana seslenen de bunu yapan da Rab'dir. Tattigim bu acilardan sonra daha dikkatli yasayacagim.
\par 16 Ya Rab, insanlar bunlarla yasarlar. Canim da bunlarin sayesinde yasiyor. Iyilestirdin, yasattin beni!
\par 17 Çektigim bunca aci esenlik bulmam içindi. Beni sevdigin için yikim çukuruna düsmekten alikoydun, Günahlarimi arkana attin.
\par 18 Çünkü ölüler diyari sana sükredemez, Ölüm övgüler sunmaz sana. Ölüm çukuruna inenler senin sadakatine umut baglayamaz.
\par 19 Diriler, yalniz diriler Bugün benim yaptigim gibi sana sükreder; Babalar senin sadakatini çocuklarina anlatir.
\par 20 Beni kurtaracak olan RAB'dir. Ömrümüz boyunca O'nun tapinaginda Telli çalgilarimizi çalacagiz."
\par 21 Yesaya, "Incir pestili getirin, Hizkiya'nin çibanina koyun, iyilesir" demisti.
\par 22 Hizkiya da, "RAB'bin Tapinagi'na çikacagima iliskin belirti nedir?" diye sormustu.

\chapter{39}

\par 1 O sirada Hizkiya'nin hastalanip iyilestigini duyan Baladan oglu Babil Krali Merodak-Baladan, ona mektuplarla birlikte bir armagan gönderdi.
\par 2 Bunlar Hizkiya'yi sevindirdi. O da deposundaki bütün degerli esyalari -altini, gümüsü, baharati, degerli yagi, silah deposunu ve hazine odalarindaki her seyi- elçilere gösterdi. Sarayinda da kralliginda da onlara göstermedigi hiçbir sey kalmadi.
\par 3 Peygamber Yesaya Kral Hizkiya'ya gidip, "Bu adamlar sana ne dediler, nereden gelmisler?" diye sordu. Hizkiya, "Uzak bir ülkeden, Babil'den gelmisler" diye karsilik verdi.
\par 4 Yesaya, "Sarayinda ne gördüler?" diye sordu. Hizkiya, "Sarayimdaki her seyi gördüler, hazinelerimde onlara göstermedigim hiçbir sey kalmadi" diye yanitladi.
\par 5 Bunun üzerine Yesaya söyle dedi: "Her Seye Egemen RAB'bin sözüne kulak ver.
\par 6 RAB diyor ki, 'Gün gelecek, sarayindaki her sey, atalarinin bugüne kadar bütün biriktirdikleri Babil'e tasinacak. Hiçbir sey kalmayacak.
\par 7 Soyundan gelen bazi çocuklar alinip götürülecek, Babil Krali'nin sarayinda hadim edilecek."
\par 8 Hizkiya, "RAB'den ilettigin bu söz iyi" dedi. Çünkü, "Nasil olsa yasadigim sürece baris ve güvenlik olacak" diye düsünüyordu.

\chapter{40}

\par 1 "Avutun halkimi" diyor Tanriniz, "Avutun!
\par 2 Yerusalim halkina dokunakli sözler söyleyin. Angaryanin bittigini, Suçlarinin cezasini ödediklerini, Günahlarinin cezasini RAB'bin elinden Iki katiyla aldiklarini ilan edin."
\par 3 Söyle haykiriyor bir ses: "Çölde RAB'bin yolunu hazirlayin, Bozkirda Tanrimiz için düz bir yol açin.
\par 4 Her vadi yükseltilecek, Her dag, her tepe alçaltilacak. Böylelikle engebeler düzlestirilecek, Sarp yerler ovaya dönüstürülecek.
\par 5 O zaman RAB'bin yüceligi görünecek, Bütün insanlar hep birlikte onu görecek. Bunu söyleyen RAB'dir."
\par 6 Ses, "Duyur" diyor. "Neyi duyurayim?" diye soruyorum. "Insan soyu ota benzer, Bütün vefasi kir çiçegi gibidir.
\par 7 RAB'bin solugu esince üzerlerine, Ot kurur, çiçek solar. Gerçekten de halk ottan farksizdir.
\par 8 Ot kurur, çiçek solar, Ama Tanrimiz'in sözü sonsuza dek durur."
\par 9 Ey Siyon'a müjde getiren, Yüksek daga çik! Ey Yerusalim'e müjde getiren, Yükselt sesini, bagir, Sesini yükselt, korkma. Yahuda kentlerine, "Iste, Tanriniz!" de.
\par 10 Iste Egemen RAB gücüyle geliyor, Kudretiyle egemenlik sürecek. Ücreti kendisiyle birlikte, Ödülü önündedir.
\par 11 Sürüsünü çoban gibi güdecek, Kollarina alacak kuzulari, Bagrinda tasiyacak; Usul usul yol gösterecek emziklilere.
\par 12 Kim denizleri avucuyla, Gökleri karisiyla ölçebildi? Yerin topragini ölçege sigdiran, Daglari kantarla, Tepeleri teraziyle tartabilen var mi?
\par 13 RAB'bin düsüncesine kim akil erdirebildi? O'na ögüt verip ögretebilen var mi?
\par 14 Akil almak, adalet yolunu ögrenmek için RAB kime danisti ki? O'na bilgi veren, anlayis yolunu bildiren var mi?
\par 15 RAB için uluslar kovada bir damla su, Terazideki toz zerrecigi gibidir. Adalari ince toz gibi tartar.
\par 16 Adaklari yakmaya yetmez Lübnan ormani, Yakmalik sunu* için az gelir hayvanlari.
\par 17 RAB'bin önünde bütün uluslar bir hiç gibidir, Hiçten bile asagi, degersiz sayilir.
\par 18 Öyleyse Tanri'yi kime benzeteceksiniz? Neyle karsilastiracaksiniz O'nu?
\par 19 Putu döküm isçisi yapar, Kuyumcu altinla kaplar, Gümüs zincirler döker.
\par 20 Böyle bir sunuya gücü yetmeyen yoksul kisi Çürümez bir agaç parçasi seçer. Yerinden kimildamaz bir put yapsin diye Usta bir isçi arar.
\par 21 Bilmiyor musunuz, duymadiniz mi? Baslangiçtan beri size bildirilmedi mi? Dünyanin temelleri atilali beri anlamadiniz mi?
\par 22 Gökkubbenin üstünde oturan RAB'dir, Yeryüzünde yasayanlarsa çekirge gibidir. Gökleri perde gibi geren, Oturmak için çadir gibi kuran O'dur.
\par 23 O'dur önderleri bir hiç eden, Dünyanin egemenlerini sifira indirgeyen.
\par 24 O önderler ki, yeni dikilmis, yeni ekilmis agaç gibi, Gövdeleri yere yeni kök salmisken RAB'bin solugu onlari kurutuverir, Kasirga saman gibi savurur.
\par 25 "Beni kime benzeteceksiniz ki, Esitim olsun?" diyor Kutsal Olan.
\par 26 Basinizi kaldirip göklere bakin. Kim yaratti bütün bunlari? Yildizlari sirayla görünür kiliyor, Her birini adiyla çagiriyor. Büyük kudreti, üstün gücü sayesinde hepsi yerli yerinde duruyor.
\par 27 Ey Yakup soyu, ey Israil! Neden, "RAB basima gelenleri görmüyor, Tanri hakkimi gözetmiyor?" diye yakiniyorsun?
\par 28 Bilmiyor musun, duymadin mi? Ebedi Tanri, RAB, bütün dünyayi yaratan, Ne yorulur ne de zayiflar, O'nun bilgisi kavranamaz.
\par 29 Yorulani güçlendirir, Takati olmayanin kudretini artirir.
\par 30 Gençler bile yorulup zayif düser, Yigitler tökezleyip düserler.
\par 31 RAB'be umut baglayanlarsa taze güce kavusur, Kanat açip yükselirler kartallar gibi. Kosar ama zayif düsmez, Yürür ama yorulmazlar.

\chapter{41}

\par 1 RAB diyor ki, "Susun karsimda, ey kiyi halklari! Halklar güçlerini tazelesin, Öne çikip konussunlar. Yargi için bir araya gelelim.
\par 2 "Dogudan adaleti harekete geçiren, Hizmete kosan kim? Uluslari önüne katiyor, krallara bas egdiriyor. Kiliciyla toz ediyor onlari, Yayiyla savrulan samana çeviriyor.
\par 3 Kovaliyor onlari, Ayak basmadigi bir yoldan esenlikle geçiyor.
\par 4 Bunlari yapip gerçeklestiren, Kusaklari baslangiçtan beri çagiran kim? Ben RAB, ilkim; sonuncularla da yine Ben olacagim."
\par 5 Kiyi halklari bunu görüp korktu. Dünyanin dört bucagi titriyor. Yaklasiyor, geliyorlar.
\par 6 Herkes komsusuna yardim ediyor, Kardesine, "Güçlü ol" diyor.
\par 7 Zanaatçi kuyumcuyu yüreklendiriyor, Madeni çekiçle düzleyen, "Lehim iyi oldu" diyerek örse vurani yüreklendiriyor. Kimildamasin diye putu yerine çiviliyor.
\par 8 "Ama sen, ey kulum Israil, Seçtigim Yakup soyu, Dostum Ibrahim'in torunlari!
\par 9 Sizleri dünyanin dört bucagindan topladim, En uzak yerlerden çagirdim. Dedim ki, 'Sen kulumsun, seni seçtim, Seni reddetmedim.
\par 10 Korkma, çünkü ben seninleyim, Yilma, çünkü Tanrin benim. Seni güçlendirecegim, evet, sana yardim edecegim; Zafer kazanan sag elimle sana destek olacagim.
\par 11 "Sana öfkelenenlerin hepsi utanacak, rezil olacak. Sana karsi çikanlar hiçe sayilip yok olacak.
\par 12 Seninle çekisenleri arasan da bulamayacaksin. Seninle savasanlar hiçten beter olacak.
\par 13 Çünkü sag elinden tutan, 'Korkma, sana yardim edecegim diyen Tanrin RAB benim.
\par 14 "Ey Yakup soyu, toprak kurdu, Ey Israil halki, korkma! Sana yardim edecegim" diyor RAB, Seni kurtaran Israil'in Kutsali.
\par 15 "Iste, seni disli, keskin, yepyeni bir harman döveni yaptim. Harman edip ufalayacaksin daglari, Tepeleri samana çevireceksin.
\par 16 Onlari savurdugunda rüzgar alip götürecek, Darmadagin edecek hepsini kasirga. Sense RAB'de sevinç bulacak, Israil'in Kutsali'yla övüneceksin.
\par 17 "Düskünlerle yoksullar su ariyor, ama yok. Dilleri kurumus susuzluktan. Ben RAB, onlari yanitlayacagim, Ben, Israil'in Tanrisi, onlari birakmayacagim.
\par 18 Çiplak tepeler üzerinde irmaklar, Vadilerde su kaynaklari yapacagim. Çölü havuza, Kurak topragi pinara çevirecegim.
\par 19 Çölü sedir, akasya, Mersin ve igde agaçlarina kavusturacagim. Bozkira çam, köknar Ve selviyi bir arada dikecegim.
\par 20 Öyle ki, insanlar görüp bilsinler, Hep birlikte düsünüp anlasinlar ki, Bütün bunlari RAB'bin eli yapmis, Israil'in Kutsali yaratmistir."
\par 21 "Davanizi sunun" diyor RAB, "Kanitlarinizi ortaya koyun" diyor Yakup'un Krali,
\par 22 "Putlarinizi getirin de olacaklari bildirsinler bize. Olup bitenleri bildirsinler ki düsünelim, Sonuçlarini bilelim. Ya da gelecekte olacaklari duyursunlar bize.
\par 23 Ey putlar, bundan sonra olacaklari bize bildirin de, Ilah oldugunuzu bilelim! Haydi bir iyilik ya da kötülük edin de, Hepimiz korkup dehsete düselim.
\par 24 Siz de yaptiklariniz da hiçten betersiniz, Sizi yegleyen igrençtir.
\par 25 "Kuzeyden birini harekete geçirdim, geliyor, Gün dogusundan bana yakaran biri. Çömlekçinin balçigi çignedigi gibi, Önderleri çamur gibi çigneyecek ayagiyla.
\par 26 Hanginiz bunu baslangiçtan bildirdi ki, bilelim, Kim önceden bildirdi ki, 'Haklisin diyelim? Konustugunuzu bildiren de Duyuran da Duyan da olmadi hiç.
\par 27 Siyon'a ilk, 'Iste, geldiler diyen benim. Yerusalim'e müjdeci gönderdim.
\par 28 Bakiyorum, aralarinda ögüt verebilecek kimse yok ki, Onlara danisayim, onlar da yanitlasinlar.
\par 29 Hepsi bombos, yaptiklari da bir hiç. Halkin putlari yalnizca yeldir, sifirdir."

\chapter{42}

\par 1 "Iste kendisine destek oldugum, Gönlümün hosnut oldugu seçtigim kulum! Ruhum'u onun üzerine koydum. Adaleti uluslara ulastiracak.
\par 2 Bagirip çagirmayacak, Sokakta sesini yükseltmeyecek.
\par 3 Ezilmis kamisi kirmayacak, Tüten fitili söndürmeyecek. Adaleti sadakatle ulastiracak.
\par 4 Yeryüzünde adaleti saglayana dek Umudunu, cesaretini yitirmeyecek. Kiyi halklari onun yasasina umut baglayacak."
\par 5 Gökleri yaratip geren, Yeryüzünü ve ürününü seren, Dünyadaki insanlara soluk, Orada yasayanlara ruh veren RAB Tanri diyor ki,
\par 6 "Ben, RAB, seni dogrulukla çagirdim, Elinden tutacak, Seni koruyacagim. Seni halka antlasma, Uluslara isik yapacagim.
\par 7 Öyle ki, kör gözleri açasin, Zindandaki tutsaklari, Cezaevi karanliginda yasayanlari özgür kilasin.
\par 8 "Ben RAB'bim, adim budur. Onurumu bir baskasina, Övgülerimi putlara birakmam.
\par 9 Bakin, önceden bildirdiklerim gerçeklesti. Simdi de yenilerini bildiriyorum; Bunlar ortaya çikmadan önce size duyuruyorum."
\par 10 Ey denizlere açilanlar ve denizlerdeki her sey, Kiyilar ve kiyi halklari, RAB'be yeni bir ilahi söyleyin, Dünyanin dört bucagindan O'nu ezgilerle övün.
\par 11 Bozkir ve bozkirdaki kentler, Kedar köylerinde yasayan halk Sesini yükseltsin. Sela'da oturanlar sevinçle haykirsin, Bagirsin daglarin doruklarindan.
\par 12 Hepsi RAB'bi onurlandirsin, Kiyi halklari O'nu övsün.
\par 13 Yigit gibi çikagelecek RAB, Savasçi gibi gayrete gelecek. Bagirip savas çigligi atacak, Düsmanlarina üstünlügünü gösterecek.
\par 14 "Uzun zamandir ses çikarmadim, Sustum, kendimi tuttum. Ama simdi feryat edecegim doguran kadin gibi, Nefesim tutulacak, kesik kesik soluyacagim.
\par 15 Harap edecegim daglari, tepeleri, Bütün Ysailliklerini kurutacagim. Irmaklari adalara çevirip havuzlari kurutacagim.
\par 16 Körlere bilmedikleri yolda rehberlik edecegim, Onlara kilavuz olacagim bilmedikleri yollarda, Karanligi önlerinde isiga, Engebeleri düzlüge çevirecegim. Yerine getirecegim sözler bunlardir. Onlardan geri dönmem.
\par 17 Oyma putlara güvenenler, Dökme putlara, 'Ilahlarimiz sizsiniz diyenlerse Geri döndürülüp büsbütün utandirilacaklar."
\par 18 "Ey sagirlar, isitin, Ey körler, bakin da görün!
\par 19 Kulum kadar kör olan var mi? Gönderdigim ulak kadar sagir olan var mi? Benimle barisik olan kadar, RAB'bin kulu kadar kör olan kim var?
\par 20 Pek çok sey gördünüz, ama aldirmiyorsunuz, Kulaklariniz açik, ama isitmiyorsunuz."
\par 21 Kendi dogrulugu ugruna Kutsal Yasa'yi Büyük ve yüce kilmak RAB'bi hosnut etti.
\par 22 Ama bu yagmalanmis, soyulmus bir halktir. Hepsi deliklere, cezaevlerine kapatilmislardir. Yagmalanmak için varlar, kurtaran yok. Soyulmak içinler, "Geri verin" diyen yok.
\par 23 Hanginiz kulak verecek? Gelecekte kim can kulagiyla dinleyecek?
\par 24 Yakup soyunun soyulmasina, Israil'in yagmalanmasina kim olur verdi? Kendisine karsi günah isledigimiz RAB degil mi? Çünkü O'nun yolunda yürümek istemediler, Yasasina kulak asmadilar.
\par 25 Bu yüzden kizgin öfkesini, Savasin siddetini üzerlerine yagdirdi. Ama ates çemberi içinde olduklarini farketmediler, Aldirmadilar kendilerini yakip bitiren atese.

\chapter{43}

\par 1 Ey Yakup soyu, seni yaratan, Ey Israil, sana biçim veren RAB simdi söyle diyor: "Korkma, çünkü seni kurtardim, Seni adinla çagirdim, sen benimsin.
\par 2 Sularin içinden geçerken seninle olacagim, Irmaklarin içinden geçerken su boyunu asmayacak. Atesin içinde yürürken yanmayacaksin, Alevler seni yakmayacak.
\par 3 Çünkü senin Tanrin, Israil'in Kutsali, Seni kurtaran RAB benim. Fidyen olarak Misir'i, Sana karsilik Kûs* ve Seva diyarlarini verdim.
\par 4 Gözümde degerli ve saygin oldugun, Seni sevdigim için, senin yerine insanlar, Canin karsiliginda halklar verecegim.
\par 5 Korkma, çünkü seninleyim, Soyundan olanlari dogudan getirecegim, Sizleri de batidan toplayacagim.
\par 6 "Kuzeye, 'Ver, güneye, 'Alikoyma; Ogullarimi uzaktan, Kizlarimi dünyanin dört bucagindan getir diyecegim.
\par 7 'Yüceligim için yaratip biçim verdigim, Adimla çagrilan herkesi, Evet, olusturdugum herkesi getirin diyecegim."
\par 8 Gözleri oldugu halde kör, Kulaklari oldugu halde sagir olan halki öne getir.
\par 9 Bütün uluslar bir araya gelsin, halklar toplansin. Içlerinden hangisi bunlari bildirebilir, Olup bitenleri bize duyurabilir? Taniklarini çagirip hakli olduklarini kanitlasinlar, Ötekiler de duyup, "Dogrudur" desinler.
\par 10 "Taniklarim sizlersiniz" diyor RAB, "Seçtigim kullar sizsiniz. Öyle ki beni taniyip bana güvenesiniz, Benim O oldugumu anlayasiniz. Benden önce bir tanri olmadi, Benden sonra da olmayacak.
\par 11 "Ben, yalniz ben RAB'bim, Benden baska kurtarici yoktur.
\par 12 Ben bildirdim, ben kurtardim, ben duyurdum, Aranizdaki yabanci ilahlar degil. Taniklarim sizsiniz" diyor RAB, "Tanri benim,
\par 13 Gün gün olali ben O'yum. Elimden kimse kurtaramaz. Ben yaparim, kim engel olabilir?"
\par 14 Kurtariciniz RAB, Israil'in Kutsali diyor ki, "Ugrunuza Babil üzerine bir ordu gönderecegim. Övündükleri gemilerle kaçan bütün Kildaniler'e* Boyun egdirecegim.
\par 15 Kutsaliniz, Israil'in Yaraticisi, Kraliniz RAB benim."
\par 16 Denizde geçit, azgin sularda yol açan, Atlarla savas arabalarini, Yigit savasçilari ve orduyu Yola çikaran RAB söyle diyor: "Onlar yatti, kalkamaz oldu, Fitil gibi bastirilip söndürüldüler.
\par 18 "Olup bitenlerin üzerinde durmayin, Düsünmeyin eski olaylari.
\par 19 Bakin, yeni bir sey yapiyorum! Olmaya basladi bile, farketmiyor musunuz? Çölde yol, kurak topraklarda irmaklar yapacagim.
\par 20 Kir hayvanlari, çakallarla baykuslar beni yüceltecek. Çünkü seçtigim halkin içmesi için çölde su, Kurak yerlerde irmaklar sagladim.
\par 21 Kendim için biçim verdigim bu halk Bana ait olan övgüleri ilan edecek."
\par 22 "Ne var ki, ey Yakup soyu, Yakardigin ben degildim, Benden usandin, ey Israil.
\par 23 Yakmalik sunu* için bana davar getirmediniz, Kurbanlarinizla beni onurlandirmadiniz. Sizi sunularla ugrastirmadim, Günnük isteyerek sizi usandirmadim.
\par 24 Benim için güzel kokulu kamis satin almadiniz, Doyurmadiniz beni kurbanlarinizin yagiyla. Tersine, beni günahlarinizla ugrastirdiniz, Suçlarinizla usandirdiniz.
\par 25 Kendi ugruna suçlarinizi silen benim, evet benim, Günahlarinizi anmaz oldum.
\par 26 "Geçmisi bana animsatin, hesaplasalim, Hakli çikmak için davanizi anlatin.
\par 27 Ilk ataniz günah isledi, Sözcüleriniz bana baskaldirdi.
\par 28 Bu yüzden tapinak görevlilerini bayagilastirdim; Yakup soyunu bütünüyle yikima, Israil'i rezillige mahkûm ettim."

\chapter{44}

\par 1 "Simdi, ey kulum Yakup soyu, Seçtigim Israil halki, dinle!
\par 2 Seni yaratan, rahimde sana biçim veren, Sana yardim edecek olan RAB söyle diyor: 'Korkma, ey kulum Yakup soyu, Ey seçtigim Ysaurun!
\par 3 "'Susamis topragi sulayacak, Kurumus toprakta dereler akitacagim. Çocuklarinin üzerine Ruhum'u dökecek, Soyunu kutsayacagim.
\par 4 Akarsu kiyisinda otlar arasinda yükselen Kavaklar gibi boy atacaklar.
\par 5 "Kimi, 'Ben RAB'be aitim diyecek, Kimi Yakup adini alacak, Kimi de eline 'RAB'be ait yazip Israil adini benimseyecek."
\par 6 RAB, Israil'in Krali ve Kurtaricisi, Her Seye Egemen RAB diyor ki, "Ilk ve son benim, Benden baska Tanri yoktur.
\par 7 Benim gibi olan var mi? Haber versin. Ezeli halkimi var ettigimden beri olup bitenleri, Bundan sonra olacaklari söyleyip siralasin, Evet, gelecek olaylari bildirsin!
\par 8 Yilmayin, korkmayin! Size çok önceden beri söyleyip açiklamadim mi? Taniklarim sizsiniz. Benden baska Tanri var mi? Hayir, baska Kaya yok; Ben bir baskasini bilmiyorum."
\par 9 Putlara biçim verenlerin hepsi bos insanlardir. Deger verdikleri nesneler hiçbir ise yaramaz. Putlarin taniklari onlardir; Ne bir sey görür ne de bir sey bilirler. Bunun sonucunda utanç içinde kalacaklar.
\par 10 Kim yararsiz ilaha biçim vermek, Dökme put yapmak ister?
\par 11 Bakin, bu putlarla ugrasanlarin hepsi utanacak. Onlari yapanlar salt insan. Hepsi toplanip yargilanmaya gelsin. Dehsete düsecek, utanacaklar birlikte.
\par 12 Demirci aletini alir, Kömür atesinde çalisir, Çekiçle demire biçim verir. Güçlü koluyla onu isler. Acikir, güçsüz kalir, su içmeyince tükenir.
\par 13 Marangoz iple ölçü alir, Tahtayi tebesirle çizer. Raspayla tahtayi biçimlendirir, Pergelle isaretler, insan biçimi verir. Insan güzelliginde, Evde duracak bir put yapar.
\par 14 Insan kendisi için sedir agaçlari keser, Palamut, mese agaçlari alir. Ormanda kendine bir agaç seçer. Bir çam diker, ama agaci büyüten yagmurdur.
\par 15 Sonra agaç odun olarak kullanilir. Insan aldigi odunla hem isinir, Hem tutusturup ekmek pisirir, Hem de bir ilah yapip tapinir. Yaptigi putun önünde yere kapanir.
\par 16 Odunun bir kismini yakar, Atesinde et kizartip karnini doyurur. Isininca bir oh çeker, "Isindim, atesin sicakligini duyuyorum" der.
\par 17 Artakalan odundan kendine bir ilah, Oyma put yapar; Önünde yere kapanip ona tapinir, "Beni kurtar, çünkü ilahim sensin" diye yakarir.
\par 18 Böyleleri anlamaz, bilmez. Çünkü gözleri de zihinleri de öylesine kapali ki, Görmez, anlamazlar.
\par 19 Durup düsünmez, bilmez, Anlamazlar ki söyle desinler: "Odunun bir kismini yakip Atesinde ekmek pisirdim, et kizartip yedim. Artakalanindan igrenç bir sey mi yapayim? Bir odun parçasinin önünde yere mi kapanayim?"
\par 20 Külle besleniyorlar. Aldanan yürekleri onlari saptiriyor. Canlarini kurtaramaz, "Sag elimdeki su nesne aldatici degil mi?" diyemezler.
\par 21 "Ey Yakup soyu, ey Israil, Söylediklerimi animsayin, çünkü kulumsunuz. Size ben biçim verdim, kulumsunuz; Seni unutmam, ey Israil.
\par 22 Isyanlarinizi bulut gibi, Günahlarinizi sis gibi sildim. Bana dönün, çünkü sizi kurtardim."
\par 23 Sevinçle haykirin, ey gökler, Çünkü bunu RAB yapti. Haykirin, ey yerin derinlikleri. Ey daglar, ey orman, ormandaki her agaç, Sevinç çigliklarina katilin. Çünkü RAB Yakup soyunu kurtararak Israil'de görkemini gösterdi.
\par 24 Sizi kurtaran, Size rahimde biçim veren RAB diyor ki, "Her seyi yaratan, Gökleri yalniz basina geren, Yeryüzünü tek basina seren, Sahte peygamberlerin belirtilerini bosa çikaran, Falcilarla alay eden, Bilgeleri geri çeviren, Bilgilerini saçmaliga dönüstüren, Kulunun sözlerini yerine getiren, Ulaklarinin peygamberlik sözlerini gerçeklestiren, Yerusalim için, 'Içinde oturulacak, Yahuda kentleri için, 'Yeniden kurulacak, Yikintilarini onaracagim diyen; Engine, 'Kuru! Sularini kurutacagim diyen, Kores için, 'O çobanimdir, Her istedigimi yerine getirecek, Yerusalim için, 'Yeniden kurulacak, Tapinak için, 'Temeli atilacak diyen RAB benim."

\chapter{45}

\par 1 RAB meshettigi* kisiye, Sag elinden tuttugu Kores'e sesleniyor. Uluslara onun önünde bas egdirecek, Krallari silahsizlandiracak, Bir daha kapanmayacak kapilar açacak. Ona söyle diyor:
\par 2 "Senin önünsira gidip Daglari düzleyecek, Tunç* kapilari kirip Demir sürgülerini parçalayacagim.
\par 3 Seni adinla çagiranin Ben RAB, Israil'in Tanrisi oldugumu anlayasin diye Karanlikta kalmis hazineleri, Gizli yerlerde sakli zenginlikleri sana verecegim.
\par 4 Sen beni tanimadigin halde Kulum Yakup soyu ve seçtigim Israil ugruna Seni adinla çagirip onurlu bir unvan verecegim.
\par 5 RAB benim, baskasi yok, Benden baska Tanri yok. Beni tanimadigin halde seni güçlü kilacagim.
\par 6 Öyle ki, dogudan batiya dek Benden baskasi olmadigini herkes bilsin. RAB benim, baskasi yok.
\par 7 Isigi biçimlendiren, karanligi yapan, Esenligi ve felaketi yaratan, Bütün bunlari yapan RAB benim.
\par 8 Ey gökler, yukaridan dogruluk damlatin, Ey bulutlar, dogruluk yagdirin. Toprak yarilsin, kurtulus meyvesi versin, Onunla birlikte dogruluk yetistirsin. Bunlari yaratan RAB benim."
\par 9 Kendine biçim verenle çekisenin vay haline! Kil, topraktan yapilmis çömlek parçasi, Kendisine biçim verene, "Ne yapiyorsun? Yarattigin nesnenin tutacagi yok" diyebilir mi?
\par 10 Babasina, "Dünyaya ne getirdin?" Ya da annesine, "Ne biçim sey dogurdun?" Diyenin vay haline!
\par 11 Israil'in Kutsali, Ona biçim veren RAB diyor ki, "Çocuklarimin gelecegi hakkinda beni sorgulayabilir, Ellerimin yapitlari hakkinda bana buyruk verebilir misiniz?
\par 12 Dünyayi ben yaptim, Üzerindeki insani ben yarattim. Benim ellerim gerdi gökleri, Bütün gök cisimleri benim buyrugumda.
\par 13 Kores'i dogrulukla harekete geçirecek, Yollarini düzleyecegim. Kentimi o onaracak, Sürgünlerimi ücret ya da ödül almadan o özgür kilacak." Böyle diyor Her Seye Egemen RAB.
\par 14 RAB diyor ki, "Misir'in ürettikleri, Kûs'un* ticaret gelirleri Ve uzun boylu Sevalilar size gelecek, sizin olacak. Zincire vurulmus olarak ardinizsira yürüyecekler. Önünüzde yere kapanip yalvaracaklar: 'Tanri yalniz sizinledir, Baskasi, baska Tanri yok."
\par 15 Gerçekten sen kendini gizleyen bir Tanri'sin, Ey Israil'in Tanrisi, ey Kurtarici!
\par 16 Put yapanlarin hepsi utandirilacak, rezil olacak. Utanç içinde uzaklasacaklar.
\par 17 Ama Israil RAB tarafindan kurtarilacak, Sonsuza dek sürecek kurtulusu. Çaglar boyunca utandirilmayacak, Asla rezil olmayacak.
\par 18 Çünkü gökleri yaratan RAB, Dünyayi yaratip biçimlendiren, pekistiren, Üzerinde yasanmasin diye degil, yasansin diye Biçimlendiren RAB -Tanri O'dur- söyle diyor: "RAB benim, baskasi yok.
\par 19 Ben gizlide, Karanliklar ülkesinin bir kösesinde konusmadim. Yakup soyuna, 'Beni olmayacak yerlerde arayin demedim. Dogru olani söyleyen, adil olani bildiren RAB benim."
\par 20 "Ey sizler, uluslardan kaçip kurtulanlar, Toplanip gelin, birlikte yaklasin! Tahtadan oyma putlar tasiyan, Kurtaramayan ilahlara yakaranlar bilgisizdir.
\par 21 Konusun, davanizi sunun, Birbirinize danisin. Bunlari çok önceden duyurup bildiren kim? Ben RAB, bildirmedim mi? Benden baska Tanri yok, adil Tanri ve Kurtarici benim. Yok benden baskasi.
\par 22 "Ey dünyanin dört bucagindakiler, Bana dönün, kurtulursunuz. Çünkü Tanri benim, baskasi yok.
\par 23 Kendi üzerime ant içtim, Agzimdan çikan söz dogrudur, bosa çikmaz: Her diz önümde çökecek, Her dil bana ant içecek.
\par 24 "Benim için söyle diyecekler: 'Dogruluk ve güç yalniz RAB'dedir, Insanlar O'na gelecek. RAB'be öfkelenenlerin hepsi utandirilacak.
\par 25 Ama bütün Israil soyu RAB tarafindan aklanacak, O'nunla övünecek.

\chapter{46}

\par 1 "Ilah Bel diz çökmüs, ilah Nebo sinmis, Putlari hayvanlara, öküzlere yüklenmis gidiyor. Tasinan bu nesneleriniz agirlik, Yorgun hayvana yük oldu.
\par 2 Birlikte sinmis, diz çökmüsler, Putlarini yük olmaktan kurtaramiyorlar. Sürgüne gidecek onlar.
\par 3 "Ey Yakup soyu, Israil'in sag kalanlari, Dogdunuz dogali yüklendigim, Rahimden çiktiniz çikali tasidigim sizler, Dinleyin beni:
\par 4 Siz yaslanincaya dek ben O'yum; Saçlariniz agarincaya dek Ben yüklenecegim sizi. Sizi ben yarattim, ben tasiyacagim, Evet, sizi ben yüklenecek, ben kurtaracagim.
\par 5 "Beni kime benzetecek, Kime denk tutacaksiniz? Kiminle karsilastiracaksiniz ki, benzer olalim?
\par 6 Kimisi bol keseden harcadigi altindan, Terazide tarttigi gümüsten Ücret karsiliginda kuyumcuya ilah yaptirir, Önünde yere kapanip tapinir.
\par 7 Onu omuzlayip tasir, yerine koyar. Öylece durur put, yerinden kimildamaz. Kendisine yakarana yanit veremez, Onu sikintisindan kurtaramaz.
\par 8 "Bunu animsayin, ey baskaldiranlar, Adam olun, aklinizdan çikarmayin!
\par 9 Çok önceden beri olup bitenleri animsayin. Çünkü Tanri benim, baskasi yok. Tanri benim, benzerim yok.
\par 10 Sonu ta baslangiçtan, Henüz olmamis olaylari çok önceden bildiren, 'Tasarim gerçeklesecek, Istedigim her seyi yapacagim diyen benim.
\par 11 Dogudan yirtici kusu, Uzak bir ülkeden Tasarimi gerçeklestirecek adami çagiran benim. Evet, bunlari söyledim, Kesinlikle yerine getirecek, Tasarladigimi yapacagim mutlaka.
\par 12 "Ey dikbaslilar, dogruluktan uzak olanlar, Dinleyin beni!
\par 13 Zaferim yaklasti, uzak degil; Kurtarisim gecikmeyecek. Güzelligim olan Israil için Siyon'u kurtaracagim."

\chapter{47}

\par 1 "Ey Babil, erden kiz, In asagi, topraga otur. Ey Kildani* kizi, Tahtin yok artik, yere otur. Bundan böyle, 'Nazik, narin demeyecekler sana.
\par 2 Bir çift degirmen tasi al da un ögüt, Çikar peçeni, kaldir etegini. Baldirini aç, irmaklardan geç.
\par 3 Çiplakligin sergilenecek, mahrem yerlerin görünecek. Öç alacagim, kimseyi esirgemeyecegim."
\par 4 Bizim kurtaricimiz Israil'in Kutsali'dir. O'nun adi "Her Seye Egemen RAB'dir!"
\par 5 RAB diyor ki, "Ey Kildani kizi, Karanliga çekilip sessizce otur. Çünkü bundan böyle 'Ülkeler kraliçesi demeyecekler sana.
\par 6 Halkima öfkelenmis, Mirasim oldugu halde onu bayagilastirip Eline teslim etmistim. Ama sen onlara acimadin, Yaslilara bile çok agir bir boyunduruk yükledin.
\par 7 'Sonsuza dek kraliçe olacagim diye düsünüyordun, Bunlari aklina getirmedin, sonuçlarini düsünmedin.
\par 8 "Ey simdi güvenlikte yasayan zevk düskünü, Içinden, 'Kraliçe benim, baskasi yok; Hiç dul kalmayacak, Evlat acisi görmeyecegim diyorsun. Dinle simdi:
\par 9 Bir gün içinde ikisi birden basina gelecek: Çok sayida büyüye, etkili muskalarina karsin Hem dul kalacak, Hem evlat acisini alabildigine yasayacaksin.
\par 10 "Kötülügüne güvendin, 'Beni gören yok diye düsündün. Bilgin ve bilgeligin seni saptirdi. Içinden, 'Kraliçe benim, baskasi yok diyordun.
\par 11 Ne var ki, felakete ugrayacaksin. Onu durduracak büyü yok elinde, Basina gelecek belayi önleyemeyeceksin. Üzerine ansizin hiç beklemedigin bir yikim gelecek.
\par 12 Gençliginden beri emek verdigin Muskalarina, çok sayida büyüye devam et; Belki yararini görür, Kimilerini titretirsin.
\par 13 Aldigin ögütlerin çoklugu Seni tüketti. Yildiz falcilarin, yildizbilimcilerin, Ay baslarinda ne olacagini bildirenlerin, Simdi kalksinlar da Basina geleceklerden seni kurtarsinlar.
\par 14 "Bak, hepsi anizdan farksiz, Ates yakacak onlari. Canlarini alevden kurtaramayacaklar. Ne isinmak için kor, Ne de karsisinda oturulacak ates olacak.
\par 15 Emek verdigin adamlar böyle olacak. Gençliginden beri alisveris ettigin herkes Kendi yoluna gidecek, Seni kurtaran olmayacak."

\chapter{48}

\par 1 "Dinle, ey Yakup soyu! Israil adiyla anilan, Yahuda soyundan gelen, RAB'bin adiyla ant içen sizler, Israil'in Tanrisi'na yakarir, Ama bunu dogrulukla, içtenlikle yapmazsiniz.
\par 2 Kutsal kentli* oldugunuzu, Israil'in Tanrisi'na dayandiginizi ileri sürersiniz. O'nun adi Her Seye Egemen RAB'dir.
\par 3 Olup bitenleri çok önceden bildirdim, Agzimi açip duyurdum. Ansizin yaptim ve gerçeklestiler.
\par 4 Inatçi oldugunuzu, Tunç* alinli, demir boyunlu oldugunuzu bildigim için
\par 5 Bunlari size çok önceden bildirdim, Olmadan önce duyurdum. Yoksa, 'Bunlari yapan putlarimizdir, Olmalarini buyuran Oyma ve dökme putlarimizdir derdiniz.
\par 6 Bunlari duydunuz, hepsini inceleyin. Peki, kabul etmeyecek misiniz? Simdiden size yeni seyler, Bilmediginiz gizli seyler açiklayacagim.
\par 7 Bunlar simdi yaratiliyor, Geçmiste degil; Bugüne kadar duymadiniz, Yoksa, 'Bunlari biliyorduk derdiniz.
\par 8 Ne duydunuz, ne de anladiniz, Öteden beri kulaklariniz tikali. Ne denli hain oldugunuzu biliyorum, Dogustan isyankâr oldugunuz biliniyor.
\par 9 Adim ugruna öfkemi geciktiriyorum. Ünümden ötürü kendimi tutuyorum, Yoksa sizi yok ederdim.
\par 10 Bakin, gümüsü aritir gibi olmasa da sizleri arittim, Sikinti ocaginda denedim.
\par 11 Bunu kendim için, evet, kendim için yapiyorum. Adimi bayagilastirmanizi nasil hos görebilirim? Bana ait olan onuru baskasina vermem."
\par 12 "Ey Yakup soyu, çagirdigim Israil, beni dinle: Ben O'yum; ilk Ben'im, son da Ben'im.
\par 13 Yeryüzünün temelini elimle attim, Gökleri sag elim gerdi. Onlari çagirdigimda Birlikte önümde dikilirler.
\par 14 "Toplanip dinleyin hepiniz: Putlardan hangisi bunlari önceden bildirebildi? RAB'bin sevdigi kisi O'nun Babil'e karsi tasarladigini yerine getirecek. Gücünü Kildaniler'e* karsi kullanacak.
\par 15 Ben, evet, ben söyledim, onu ben çagirdim, Onu getirdim, görevini basaracak.
\par 16 "Yaklasin bana, dinleyin söyleyeceklerimi: Baslangiçtan beri açikça konustum, O zamandan bu yana oradayim." Egemen RAB simdi beni ve Ruhu'nu gönderiyor.
\par 17 Sizleri kurtaran Israil'in Kutsali RAB diyor ki, "Yararli olani size ögreten, Gitmeniz gereken yolda sizi yürüten Tanriniz RAB benim.
\par 18 "Keske buyruklarima dikkat etseydiniz! O zaman esenliginiz irmak gibi, Dogrulugunuz denizin dalgalari gibi olurdu.
\par 19 Soyunuz kum gibi, Torunlariniz kum taneleri gibi olurdu. Adlari ne unutulur, Ne de huzurumdan yok olurdu."
\par 20 Babil'den çikin, Kildaniler'den kaçin, Sevinç çigliklariyla ilan edin bunu, Haberini duyurun, dünyanin dört bucagina yayin. "RAB, kulu Yakup'un soyunu kurtardi" deyin.
\par 21 Onlari çöllerden geçirirken susuzluk çekmediler, Onlar için sular akitti kayadan, Kayayi yardi, sular fiskirdi.
\par 22 "Kötülere esenlik yoktur" diyor RAB.

\chapter{49}

\par 1 Ey kiyi halklari, isitin beni, Uzaktaki halklar, iyi dinleyin. RAB beni ana rahmindeyken çagirdi, Annemin karnindayken adimi koydu.
\par 2 Agzimi keskin kiliç yapti, Elinin gölgesinde gizledi beni. Beni keskin bir ok yapti, Kendi ok kilifina sakladi.
\par 3 Bana, "Kulumsun, ey Israil, Görkemimi senin araciliginla gösterecegim" dedi.
\par 4 Ama ben, "Bosuna emek verdim" dedim, "Gücümü bos yere, bir hiç için tükettim. RAB yine de hakkimi savunur, Tanrim yaptiklarimin karsiligini verir."
\par 5 Kulu olmam için, Yakup soyunu kendisine geri getirmem, Israil'i önünde toplamam için Rahimde beni biçimlendiren RAB simdi söyle diyor: -O'nun gözünde onurluyum, Tanrim bana güç kaynagi oldu.-
\par 6 "Yakup'un oymaklarini canlandirmak, Sag kalan Israilliler'i geri getirmek için Kulum olman yeterli degil. Seni uluslara isik yapacagim. Öyle ki, kurtarisim yeryüzünün dört bucagina ulassin."
\par 7 Insanlarin hor gördügüne, Uluslarin igrendigine, Egemenlerin kulu olana Israil'in Kurtaricisi ve Kutsali RAB diyor ki, "Seni seçmis olan Israil'in Kutsali sadik RAB'den ötürü Krallar seni görünce ayaga kalkacak, Önderler yere kapanacak."
\par 8 RAB söyle diyor: "Lütuf zamaninda seni yanitlayacagim, Kurtulus günü sana yardim edecek, Seni koruyacagim. Seni halka antlasma olarak verecegim. Öyle ki, yikik ülkeyi yeniden kurasin, Mülk olarak yeni sahiplerine veresin.
\par 9 Tutsaklara, 'Çikin, Karanliktakilere, 'Disari çikin diyeceksin. Yol boyunca beslenecek, Her çiplak tepede otlak bulacaklar.
\par 10 Acikmayacak, susamayacaklar, Kavurucu sicak ve günes çarpmayacak onlari. Çünkü onlara merhamet eden kendilerine yol gösterecek Ve onlari pinarlara götürecek.
\par 11 Bütün daglarimi yola dönüstürecegim, Anayollarim yükseltilecek.
\par 12 Iste halkim ta uzaklardan, Kimi kuzeyden, kimi batidan, kimi de Sinim'den gelecek."
\par 13 Ey gökler, sevinçle haykirin, Neseyle cos, ey yeryüzü! Ey daglar, sevinç çigliklarina katilin, Çünkü RAB halkini avutacak, Ezilene merhamet gösterecek.
\par 14 Oysa Siyon, "RAB beni terk etti, Rab beni unuttu" diyordu.
\par 15 Ama RAB, "Kadin emzikteki çocugunu unutabilir mi?" diyor, "Rahminden çikan çocuktan sevecenligi esirger mi? Kadin unutabilir, Ama ben seni asla unutmam.
\par 16 Bak, adini avuçlarima kazidim, Duvarlarini gözlüyorum sürekli.
\par 17 Ogullarin kosar adim geliyor, Seni yikip viran edenlerse çikip gidecek.
\par 18 Basini kaldir da çevrene bir bak: Hepsi toplanmis sana geliyor. Ben RAB, varligim hakki için diyorum ki, Onlarin hepsi senin süsün olacak, Bir gelin gibi takinacaksin onlari.
\par 19 "Çünkü yikilmis, viraneye dönmüstün, Ülken yerle bir olmustu. Ama simdi halkina dar geleceksin, Seni harap etmis olanlar senden uzak duracaklar.
\par 20 Yitirdigini sandigin çocuklarinin sesini yine duyacaksin: 'Burasi bize dar geliyor, Yasayacak bir yer ver bize diyecekler.
\par 21 O zaman içinden, 'Kim dogurdu bunlari bana? diyeceksin, 'Çocuklarimi yitirmistim, doguramiyordum. Sürgüne gönderilmis, dislanmistim. Öyleyse bunlari kim büyüttü? Yapayalniz kalmistim, Nereden çikip geldi bunlar?"
\par 22 Egemen RAB diyor ki, "Bakin, uluslara elimle isaret verdigimde, Sancagimi yükselttigimde halklara, Senin ogullarini kucaklarinda getirecek, Kizlarini omuzlarinda tasiyacaklar.
\par 23 Krallar size babalik, Prensesler sütannelik yapacak, Yüzüstü yere kapanip Ayaklarinin tozunu yalayacaklar. O zaman benim RAB oldugumu anlayacaksin. Bana umut baglayan utandirilmayacak."
\par 24 Güçlünün ganimeti elinden alinabilir mi? Zorbanin elindeki tutsak kurtulabilir mi?
\par 25 Ama RAB diyor ki, "Evet, güçlünün elindeki tutsaklar alinacak, Zorbanin aldigi ganimet de kurtarilacak. Seninle çekisenle ben çekisecegim, Senin çocuklarini ben kurtaracagim.
\par 26 Sana zulmedenlere kendi etlerini yedirecegim, Tatli sarap içmis gibi kendi kanlariyla sarhos olacaklar. Böylece bütün insanlar bilecek ki Seni kurtaran RAB benim; Kurtaricin, Yakup'un Güçlüsü benim."

\chapter{50}

\par 1 RAB söyle diyor: "Bosadigim annenizin bosanma belgesi nerede? Hangi alacaklima sattim sizi? Suçlariniz yüzünden satildiniz, Anneniz isyanlariniz yüzünden dislandi.
\par 2 Geldigimde neden kimse yoktu, Çagirdigimda niçin yanit veren olmadi? Sizi kurtaramayacak kadar kisa mi elim, Ya da gücüm yok mu sizi özgür kilmaya? Azarlayarak denizi kurutur, Irmaklari çöle çeviririm. Su kalmayinca baliklar ölür ve kokar.
\par 3 Göklere karalar giydirir, Çul ederim onlarin örtüsünü."
\par 4 Yorgunlara sözle destek olmayi bileyim diye Egemen RAB bana egitilmislerin dilini verdi. Egitilenler gibi dinleyeyim diye kulagimi uyandirir her sabah.
\par 5 Egemen RAB kulagimi açti, Karsi koymadim, geri çekilmedim.
\par 6 Bana vuranlara sirtimi açtim, Yanaklarimi uzattim sakalimi yolanlara. Asagilamalardan, tükürükten yüzümü gizlemedim.
\par 7 Egemen RAB bana yardim ettigi için Utanç duymam. Kararimdan dönmem, Utandirilmayacagimi bilirim.
\par 8 Beni hakli çikaran yakinimda. Benden davaci olan kim, yüzleselim, Kimdir hasmim, karsima çiksin.
\par 9 Bana yardim eden Egemen RAB'dir, Kim suçlu çikaracak beni? Onlarin hepsi giysi gibi eskiyecek, Tümünü güve yiyip bitirecek.
\par 10 Aranizda RAB'den korkan, Kulunun sözünü dinleyen kim var? Karanlikta yürüyen, isigi olmayan, RAB'bin adina güvensin, Tanrisi'na dayansin.
\par 11 Ama ates yakan, Alevli oklar kusanan sizler, hepiniz, Atesinizin aydinliginda, Tutusturdugunuz alevli oklarin arasinda yürüyün. Benden alacaginiz sudur: Azap içinde yatacaksiniz.

\chapter{51}

\par 1 "Dogrulugun ardindan giden, RAB'be yönelen sizler, beni dinleyin: Yontuldugunuz kayaya, Çikarildiginiz tas ocagina bakin.
\par 2 Ataniz Ibrahim'e, sizi doguran Sara'ya bakin. Çagirdigimda tek kisiydi Ibrahim, Ama ben onu kutsayip çogalttim."
\par 3 RAB Siyon'u ve bütün yikintilarini avutacak. Siyon çölünü Aden'e, bozkiri RAB'bin bahçesine döndürecek. Orada cosku, sevinç, Sükran ve ezgi olacak.
\par 4 "Beni dinle, ey halkim, Bana kulak ver, ey ulusum! Yasa benden çikacak, Halklara isik olarak adaletimi yerlestirecegim.
\par 5 Zaferim yaklasti, Kurtarisim ortaya çikti. Halklari gücümle yönetecegim. Kiyi halklari bana umut bagladi, Umutla gücümü bekliyorlar.
\par 6 Basinizi kaldirip göklere bakin, Asagiya, yeryüzüne bakin. Çünkü bu gökler duman gibi dagilacak, Giysi gibi eskiyecek yeryüzü; Üzerinde yasayanlar sinek gibi ölecek. Ama benim kurtarisim sonsuz olacak, Ardi kesilmeyecek zaferimin.
\par 7 "Ey sizler, dogru olani bilenler, Yasami yüreginde tasiyan halk, dinleyin beni! Insanlarin asagilamalarindan korkmayin, Yilmayin sövgülerinden.
\par 8 Güvenin yedigi giysi gibi, Kurtçugun yedigi yapagi gibi yitecekler. Oysa zaferim sonsuza dek kalacak, Kurtarisim kusaklar boyu sürecek."
\par 9 Uyan, ey RAB'bin gücü, uyan, kudreti kusan! Eski günlerde, önceki kusaklar döneminde oldugu gibi uyan! Rahav'i parçalayan, Deniz canavarinin bedenini desen sen degil miydin?
\par 10 Denizi, engin sularin derinliklerini kurutan, Kurtulanlarin geçmesi için Denizin derinliklerini yola çeviren sen degil miydin?
\par 11 RAB'bin kurtardiklari dönecek, Sevinçle haykirarak Siyon'a varacaklar. Yüzlerinde sonsuz sevinç olacak. Onlarin olacak cosku ve sevinç, Üzüntü ve inilti kaçacak.
\par 12 RAB diyor ki, "Sizi avutan benim, evet benim. Siz kimsiniz ki, ölümlü insandan, Ottan farksiz insanoglundan korkarsiniz?
\par 13 Sizi yaratan, gökleri geren, Dünyanin temellerini atan RAB'bi Nasil olur da unutursunuz? Sizi yok etmeye hazirlanan zalimin öfkesinden Neden gün boyu yilip duruyorsunuz? Hani nerede zalimin gazabi?
\par 14 Zincire vurulmus tutsaklar Çok yakinda özgürlüge kavusacak. Ölüm çukuruna inmeyecek, Aç kalmayacaklar.
\par 15 Tanriniz RAB benim. Dalgalar gürlesin diye denizi çalkalayan benim." O'nun adi Her Seye Egemen RAB'dir!
\par 16 "Sözlerimi agzina koydum, Seni elimin gölgesiyle örttüm; Gökleri yerlestirmen, Yeryüzünün temellerini atman Ve Siyon'a, 'Halkim sensin demen için..."
\par 17 Uyan, ey Yerusalim, uyan, kalk ayaga! Sen ki, RAB'bin gazap kâsesini* O'nun elinden içtin. Tamamini içtin sersemleten kâsenin.
\par 18 Dogurdugun bunca oguldan sana yol gösteren yok, Elinden tutan da yok büyüttügün bunca oguldan.
\par 19 Basina çifte felaket geldi, kim bassagligi dileyecek? Yikim ve kirim, kitlik ve kiliç. Nasil avutayim seni?
\par 20 Ogullarin baygin, aga düsmüs ahular gibi Her sokak basinda yatiyor. RAB'bin öfkesine de Tanrin'in azarlayisina da doymuslar.
\par 21 Bu nedenle, ey ezilmis Yerusalim, Sarapsiz sarhos olmus halk, sunu dinle!
\par 22 Egemenin RAB, kendi halkini savunan Tanrin diyor ki, "Seni sersemleten kâseyi, gazabimin kâsesini Elinden aldim. Bir daha asla içmeyeceksin ondan.
\par 23 Onu sana eziyet edenlerin eline verecegim; Onlar ki sana, 'Yere yat da Üzerinden geçelim dediklerinde, Sirtini toprak, yol ettin."

\chapter{52}

\par 1 Uyan, ey Siyon, uyan, kudretini kusan. Ey Yerusalim, kutsal kent, güzel giysilerini giy. Çünkü sünnetsizlerle* murdarlar* Kapilarindan asla içeri girmeyecek artik.
\par 2 Üzerindeki tozu silk! Kalk, ey Yerusalim, tahtina otur, Boynundaki zinciri çöz, Ey Siyon, tutsak kiz.
\par 3 RAB diyor ki, "Karsiliksiz satilmistiniz, Parasiz kurtulacaksiniz."
\par 4 Egemen RAB diyor ki, "Halkim gurbette yasamak için önce Misir'a inmisti. Simdi de Asurlular onlari ezdi.
\par 5 Halkim bos yere alinip götürüldü, Benim burayla ne ilgim kaldi?" diyor RAB, "Yöneticileri feryat ediyor, Adima günboyu sövülüyor" diyor RAB.
\par 6 "Bundan ötürü halkim adimi bilecek, O gün, 'Iste ben diyenin ben oldugumu anlayacak."
\par 7 Daglari asip gelen müjdecinin ayaklari ne güzeldir! O müjdeci ki, esenlik duyuruyor. Iyilik müjdesi getiriyor, kurtulus haberi veriyor. Siyon halkina, "Tanriniz egemenlik sürüyor!" diye ilan ediyor.
\par 8 Dinleyin! Bekçileriniz seslerini yükseltiyor, Hep birlikte sevinçle haykiriyorlar. Çünkü RAB'bin Siyon'a dönüsünü gözleriyle görmekteler!
\par 9 Ey Yerusalim yikintilari, Hep birlikte sevinçle haykirip bagirin! Çünkü RAB halkini avuttu, Yerusalim'i kurtardi.
\par 10 Bütün uluslarin gözü önünde Kutsal kolunu sivadi, Dünyanin dört bucagi Tanrimiz'in kurtarisini görecek.
\par 11 Çekilin, çekilin, oradan çikin, Murdara dokunmayin. Oradan çikip temizlenin, Ey RAB'be tapinma araçlarini tasiyan sizler!
\par 12 Aceleyle çikmayacak, Kaçip gitmeyeceksiniz; Çünkü RAB önünüzden gidecek, Israil'in Tanrisi artçiniz olacak.
\par 13 Bakin, kulum basarili olacak; Üstün olacak, el üstünde tutulup alabildigine yüceltilecek.
\par 14 Birçoklari onun karsisinda dehsete düsüyor; Biçimi, görünüsü öyle bozuldu ki, Insana benzer yani kalmadi;
\par 15 Pek çok ulus ona sasacak, Onun önünde krallarin agizlari kapanacak. Çünkü kendilerine anlatilmamis olani görecek, Duymadiklarini anlayacaklar.

\chapter{53}

\par 1 Verdigimiz habere kim inandi? RAB'bin gücü kime açiklandi?
\par 2 O RAB'bin önünde bir fidan gibi, Kurak yerdeki kök gibi büyüdü. Bakilacak biçimden, güzellikten yoksundu. Gönlümüzü çeken bir görünüsü de yoktu.
\par 3 Insanlarca hor görüldü, Yapayalniz birakildi. Acilar adamiydi, hastaligi yakindan tanidi. Insanlarin yüz çevirdigi biri gibi hor görüldü, Ona deger vermedik.
\par 4 Aslinda hastaliklarimizi o üstlendi, Acilarimizi o yüklendi. Bizse Tanri tarafindan cezalandirildigini, Vurulup ezildigini sandik.
\par 5 Oysa, bizim isyanlarimiz yüzünden onun bedeni desildi, Bizim suçlarimiz yüzünden o eziyet çekti. Esenligimiz için gerekli olan ceza Ona verildi. Bizler onun yaralariyla sifa bulduk.
\par 6 Hepimiz koyun gibi yoldan sapmistik, Her birimiz kendi yoluna döndü. Yine de RAB hepimizin cezasini ona yükledi.
\par 7 O baski görüp eziyet çektiyse de Agzini açmadi. Kesime götürülen kuzu gibi, Kirkicilarin önünde sessizce duran koyun gibi Açmadi agzini.
\par 8 Acimasizca yargilanip ölüme götürüldü. Halkimin isyani ve hak ettigi ceza yüzünden Yasayanlar diyarindan atildi. Onun kusagindan bunu düsünen oldu mu?
\par 9 Siddete basvurmadigi, Agzindan hileli söz çikmadigi halde, Ona kötülerin yaninda bir mezar verildi, Ama öldügünde zenginin yanindaydi.
\par 10 Ne var ki, RAB onun ezilmesini uygun gördü, Aci çekmesini istedi. Canini suç sunusu* olarak sunarsa Soyundan gelenleri görecek ve günleri uzayacak. RAB'bin istemi onun araciligiyla gerçeklesecek.
\par 11 Canini feda ettigi için Gördükleriyle hosnut olacak. RAB'bin dogru kulu, kendisini kabul eden birçoklarini aklayacak. Çünkü onlarin suçlarini o üstlendi.
\par 12 Bundan dolayi ona ünlüler arasinda bir pay verecegim, Ganimeti güçlülerle paylasacak. Çünkü canini feda etti, baskaldiranlarla bir sayildi. Pek çoklarinin günahini o üzerine aldi, Baskaldiranlar için de yalvardi.

\chapter{54}

\par 1 "Çocuk dogurmayan ey kisir kadin, Sevinç çigliklari at; Ey dogum agrisi nedir bilmeyen sen, Sevinçle haykir, bagir. Çünkü terk edilmis kadinin, Evli kadindan daha çok çocugu olacaktir" diyor RAB.
\par 2 "Çadirinin alanini genislet, Perdelerini uzat, çekinme. Gergi iplerini de uzat, kaziklarini saglamlastir.
\par 3 Çünkü saga sola yayilacaksin, Soyundan gelenler uluslari mülk edinecek, Issiz kentlere yerlesecek.
\par 4 "Korkma, ayiplanmayacaksin, Utanma, asagilanmayacaksin. Unutacaksin gençliginde yasadigin utanci, Dulluk ayibini artik anmayacaksin.
\par 5 Çünkü kocan, seni yaratandir. O'nun adi Her Seye Egemen RAB'dir, Israil'in Kutsali'dir seni kurtaran. O'na bütün dünyanin Tanrisi denir."
\par 6 Tanrin diyor ki, "RAB seni terk edilmis, Ruhu kederli bir kadin, Genç yasta evlenip sonra dislanmis Bir kadin olarak çagiriyor:
\par 7 'Bir an için seni terk ettim, Ama büyük sevecenlikle geri getirecegim.
\par 8 Bir anlik taskin öfkeyle senden yüz çevirmistim, Ama sonsuz sadakatle sana sevecenlik gösterecegim." Seni kurtaran RAB böyle diyor.
\par 9 "Bu benim için Nuh tufani gibidir. Nuh tufaninin bir daha yeryüzünü Kaplamayacagina nasil ant içtimse, Sana öfkelenmeyecegime, Seni azarlamayacagima da ant içiyorum.
\par 10 Daglar yerinden kalksa, tepeler sarsilsa da Sadakatim senin üzerinden kalkmaz, Esenlik antlasmam sarsilmaz" Diyor sana merhamet eden RAB.
\par 11 "Ey kasirgaya tutulmus, Avuntu bulmamis ezik kent! Taslarini koyu harçla yerine koyacak, Temellerini laciverttasiyla atacagim.
\par 12 Kale burçlarini yakuttan, Kapilarini mücevherden, Surlarini degerli taslardan yapacagim.
\par 13 Bütün çocuklarini ben RAB egitecegim, Esenlikleri tam olacak.
\par 14 Dogrulukla güçlenecek, Baskidan uzak olacak, korkmayacaksin. Dehset senden uzak kalacak, sana yaklasmayacak.
\par 15 Sana saldiran olursa, benden olmadigini bil. Sana saldiran herkes önünde yenilgiye ugrayacak.
\par 16 "Iste, kor halindeki atesi üfleyen, Amaca uygun silah yapan demirciyi ben yarattim. Yok etsin diye yikiciyi da ben yarattim.
\par 17 Ama sana karsi yapilan hiçbir silah ise yaramayacak, Mahkemede seni suçlayan her dili Suçlu çikaracaksin. RAB'be kulluk edenlerin mirasi sudur: Onlarin gönenci bendendir" diyor RAB.

\chapter{55}

\par 1 "Ey susamis olanlar, sulara gelin, Parasi olmayanlar, gelin, satin alin, yiyin. Gelin, sarabi ve sütü parasiz, bedelsiz alin.
\par 2 Paranizi neden ekmek olmayana, Emeginizi doyurmayana harciyorsunuz? Beni iyi dinleyin ki, iyi olani yiyesiniz, Bollugun tadini çikarasiniz!
\par 3 "Kulak verin, bana gelin. Dinleyin ki yasayasiniz. Ben de sizinle sonsuz bir antlasma, Davut'a söz verdigim kalici iyilikleri içeren bir antlasma yapayim.
\par 4 Bakin, onu halklara tanik, Önder ve komutan yaptim.
\par 5 Tanimadiginiz uluslari çagiracaksiniz, Sizi tanimayan uluslar kosa kosa size gelecek. Tanriniz RAB'den, Israil'in Kutsali'ndan ötürü gelecekler. Çünkü RAB sizleri yüceltecek."
\par 6 Bulma firsati varken RAB'bi arayin, Yakindayken O'na yakarin.
\par 7 Kötü kisi yolunu, Fesatçi düsüncelerini biraksin; RAB'be dönsün, merhamet bulur, Tanrimiz'a dönsün, bol bol bagislanir.
\par 8 "Çünkü benim düsüncelerim Sizin düsünceleriniz degil, Sizin yollariniz benim yollarim degil" diyor RAB.
\par 9 "Çünkü gökler nasil yeryüzünden yüksekse, Yollarim da sizin yollarinizdan, Düsüncelerim düsüncelerinizden yüksektir.
\par 10 Gökten inen yagmur ve kar, Topragi sulamadan, yeri Ysaertmeden, Ekinciye tohum, yiyene ekmek vermeden Nasil göge dönmezse,
\par 11 Agzimdan çikan söz de öyle olacaktir. Bana bos dönmeyecek, Istemimi yerine getirecek, Yapmasi için onu gönderdigim isi basaracaktir.
\par 12 Sevinçle çikacak, Esenlikle geri götürüleceksiniz. Daglar, tepeler önünüzde sevinçle çigiracak, Kirdaki bütün agaçlar alkis tutacak.
\par 13 Dikenli çali yerine çam, Isirgan yerine mersin agaci bitecek. Bunlar bana ün getirecek, Yok olmayan sonsuz bir belirti olacak."

\chapter{56}

\par 1 RAB söyle diyor: "Adil ve dogru olani koruyup yerine getirin. Çünkü dogrulugum gelmek, Adaletim görünmek üzeredir.
\par 2 Bunu yapan insana, Buna simsiki sarilan insanogluna ne mutlu! Sabat Günü'nü* tutar, bayagilastirmaz, Her türlü kötülükten sakinir."
\par 3 RAB'be baglanan hiçbir yabanci, "Kuskusuz RAB beni halkindan ayiracak", Hiçbir hadim da, "Ben kuru bir agacim" demesin.
\par 4 Çünkü RAB diyor ki, "Sabat günlerimi tutan, Beni hosnut edeni seçen, Antlasmama simsiki bagli kalan hadima
\par 5 Evimde, evimin dört duvari arasinda Ogullardan da kizlardan da daha iyi bir anit ve ad verecegim; Yok edilemez, ebedi bir ad olacak bu.
\par 6 "RAB'be hizmet etmek, O'nun adini sevmek, Kulu olmak için O'na baglanan yabancilari, Sabat Günü'nü tutan, bayagilastirmayan, Antlasmama simsiki bagli kalan herkesi,
\par 7 Kutsal dagima getirip Dua evimde sevindirecegim. Yakmalik sunulariyla* kurbanlari Sunagimda kabul edilecek, Çünkü evime 'Bütün uluslarin dua evi denecek."
\par 8 Israil'in sürgünlerini toplayan Egemen RAB diyor ki, "Toplanmis olanlara katmak üzere Daha baskalarini da toplayacagim."
\par 9 Ey bütün kir hayvanlari, Ormanda yasayan bütün hayvanlar, Yiyip bitirmek için gelin!
\par 10 Israil'in bekçileri kördür, hepsi bilgisizdir. Havlayamayan dilsiz köpekler gibidirler. Uzanip düs görürler, Uykuyu pek severler!
\par 11 Doymak bilmeyen azgin köpeklere benzerler, Akli kit çobanlar bunlar! Kendi yollarina döndüler, Her biri yalniz kendi çikarini düsünüyor.
\par 12 Birbirlerine, "Haydi, sarap getirelim, Bol bol içki içelim! Yarin da bugün gibi geçecek, Hatta çok daha iyi olacak" diyorlar.

\chapter{57}

\par 1 Dogru kisi ölüp gidiyor, Kimsenin umurunda degil. Sadik adamlar da göçüp gidiyor; Kimse dogru kisinin göçüp gitmekle Kötülükten kurtuldugunun farkinda degil.
\par 2 Dogru kisi esenlige kavusur, Dogru yolda yürümüs olan mezarinda rahat uyur.
\par 3 Ama siz, ey falci kadinin çocuklari, Fahiselik ve zina edenlerin soyu, buraya gelin!
\par 4 Siz kiminle alay ediyorsunuz? Kime dudak büküyor, dil çikariyorsunuz? Agaçlar arasinda, bol yaprakli her agacin altinda Sehvetle yanip tutusan, Vadilerde, kaya kovuklarinda çocuklarini kurban eden, Isyan torunlari, yalan soyu degil misiniz siz?
\par 6 Sizin payiniz Vadinin düzgün taslarindan yapilan putlardir, Evet, sizin nasibiniz onlardir! Onlara dökmelik sunular döktünüz, Tahil sunulari* sundunuz. Bütün bunlardan sonra sizi cezalandirmaktan çekinecegimi mi saniyorsunuz?
\par 7 Yataginizi ulu, yüksek daga serdiniz, Oraya bile kurban kesmeye gidiyorsunuz.
\par 8 Kapilarinizin, sövelerinizin arkasina Igrenç simgeler koydunuz. Beni biraktiniz, Yataklarinizi ardina kadar açip içine girdiniz, Oynaslarinizla anlasip birlikte yatmaya can atiyorsunuz. Onlarin çiplakligini seyrettiniz.
\par 9 Çesit çesit hos kokular sürünüp ilah Molek'e yag götürdünüz. Elçilerinizi ta uzaklara gönderdiniz, Ölüler diyarina dek alçalttiniz kendinizi.
\par 10 Uzun yolculuklar sizi yordugu halde, "Pes ettim" demediniz. Gücünüzü tazeleyip durdunuz, Bu nedenle de tükenmediniz.
\par 11 "Sizi kaygilandiran, korkutan kim ki, Bana ihanet ediyor, beni anmiyor, Yüreginizde bana yer vermiyorsunuz? Benden korkmamanizin nedeni Uzun zamandir suskun kalisim degil mi?
\par 12 Sözde dogrulugunuzu da yaptiklarinizi da ilan edecegim, Bunlarin size yarari olmayacak.
\par 13 Feryat ettiginizde Topladiginiz putlar sizi kurtarsin bakalim! Rüzgar hepsini silip süpürecek, Bir soluk onlari alip götürecek. Bana siginansa ülkeyi mülk edinecek, Kutsal dagimi miras alacak."
\par 14 RAB diyor ki, "Toprak yigip yol yapin, Halkimin yolundaki engelleri kaldirin."
\par 15 Yüce ve görkemli Olan, Sonsuzlukta yasayan, adi Kutsal Olan diyor ki, "Yüksek ve kutsal yerde yasadigim halde, Alçakgönüllülerle, ezilenlerle birlikteyim. Yüreklerini sevindirmek için ezilenlerin yanindayim.
\par 16 Çünkü sonsuza dek davaci ve öfkeli olacak degilim, Öyle olsa, yarattigim canlarla ruhlar karsimda dayanamazdi.
\par 17 Haksiz kazanç suçuna öfkelenip halki cezalandirdim, Öfkeyle yüzümü çevirdim onlardan. Ne var ki, inatla kendi yollarindan gittiler.
\par 18 "Yaptiklarini gördüm, Ama onlari iyilestirip yol gösterecegim. Karsilik olarak hem onlari Hem de aralarinda yas tutanlari avutacagim.
\par 19 Dudaklardan övgü sözleri döktürecegim. Uzaktakine de yakindakine de Tam esenlik olsun" diyor RAB, "Hepsini iyilestirecegim."
\par 20 Ama kötüler çalkalanan deniz gibidir, O deniz ki, rahat duramaz, sulari çamur ve pislik savurur.
\par 21 "Kötülere esenlik yoktur" diyor Tanrim.

\chapter{58}

\par 1 "Avaz avaz bagirin, çekinmeyin, Sesinizi boru sesi gibi yükseltin; Halkima isyanlarini, Yakup soyuna günahlarini bildirin.
\par 2 Bana her gün danisiyor, Yollarimi ögrenmekten zevk duyuyorlarmis! Dogru davranan, Tanrisi'nin buyrugundan ayrilmayan bir ulusmus gibi... Benden adil yargilar diliyor, Bana yaklasmaktan zevk aliyorlarmis.
\par 3 Diyorlar ki, 'Oruç* tuttugumuzu neden görmüyor, Isteklerimizi denetledigimizi neden farketmiyorsun? "Bakin, oruç tuttugunuz gün keyfinize bakiyor, Isçilerinizi eziyorsunuz.
\par 4 Orucunuz kavgayla, çekismeyle, Siddetli yumruklasmayla bitiyor. Bugünkü gibi oruç tutmakla Sesinizi yükseklere duyuramazsiniz.
\par 5 Istedigim oruç bu mu saniyorsunuz? Insanin isteklerini denetlemesi gereken gün böyle mi olmali? Kamis gibi bas egip çul ve kül üzerine mi oturmali? Siz buna mi oruç, RAB'bi hosnut eden gün diyorsunuz?
\par 6 Benim istedigim oruç, Haksiz yere zincire, boyunduruga vurulanlari salivermek, Ezilenleri özgürlüge kavusturmak, Her türlü boyundurugu kirmak degil mi?
\par 7 Yiyeceginizi açla paylasmak degil mi? Barinaksiz yoksullari evinize alir, Çiplak gördügünüzü giydirir, Yakinlarinizdan yardiminizi esirgemezseniz,
\par 8 Isiginiz tan gibi agaracak, Çabucak sifa bulacaksiniz. Dogrulugunuz önünüzden gidecek, RAB'bin yüceligi artçiniz olacak.
\par 9 O zaman yardim çagrilarinizi RAB yanitlayacak, Feryat ettiginizde, 'Iste buradayim diyecek. "Eger boyunduruga, baskalarini suçlamaya, Kötücül konusmalara son verirseniz,
\par 10 Açlar ugruna kendinizi feda eder, Yoksullarin gereksinimini karsilarsaniz, Isiginiz karanlikta parlayacak, Karanliginiz öglen gibi isiyacak.
\par 11 RAB her zaman size yol gösterecek, Kurak topraklarda sizi doyurup güçlendirecek. Iyi sulanmis bahçe gibi, Tükenmez su kaynagi gibi olacaksiniz.
\par 12 Halkiniz eski yikintilari onaracak, Geçmis kusaklarin temelleri üzerine Yeni yapilar dikeceksiniz. 'Duvardaki gedikleri onaran, Sokaklari oturulacak hale getiren denecek sizlere.
\par 13 "Kutsal günümde dilediginizi yapmaz, Sabat Günü'nü* çignemezseniz, Sabat Günü'ne 'Zevkli, RAB'bin kutsal gününe 'Onurlu derseniz, Kendi yolunuzdan gitmez, Keyfinize bakmayip bos konulara dalmaz, O günü yüceltirseniz,
\par 14 RAB'den zevk alirsiniz. O zaman sizi yeryüzünün yüksek yerlerine çikarir, Ataniz Yakup'un mirasiyla doyururum." Bunu söyleyen RAB'dir.

\chapter{59}

\par 1 Bakin, RAB'bin eli kurtaramayacak kadar kisa, Kulagi duyamayacak kadar sagir degildir.
\par 2 Ama suçlariniz sizi Tanriniz'dan ayirdi. Günahlarinizdan ötürü O'nun yüzünü göremez, Sesinizi isittiremez oldunuz.
\par 3 Çünkü elleriniz kanla, Parmaklariniz suçla kirlendi. Dudaklariniz yalan söyledi, Diliniz kötülük mirildaniyor.
\par 4 Adaletle dava açan, Davasini dürüstçe savunan yok. Bos laflara güveniyor, yalan söylüyorlar. Fesada gebe kalip kötülük doguruyorlar.
\par 5 Engerek yumurtalari üzerinde kuluçkaya yatiyor, Örümcek agi dokuyorlar. Onlarin yumurtalarindan yiyen ölür, Kirilan yumurtadan engerek yavrusu çikar.
\par 6 Dokuduklari agdan giysi olmaz, Elleriyle yaptiklariyla örtünemezler. Eylemleri kötü eylemlerdir, Elleri zorbaligin araçlaridir.
\par 7 Ayaklari kötülüge kosar, Çekinmeden suçsuz kani dökerler. Akillari fikirleri hep kötülükte, Siddet ve yikim var yollarinda.
\par 8 Esenlik yolunu bilmezler, Izledikleri yolda adalet yoktur. Kendilerine çarpik yollar yaptilar, O yoldan gidenlerin hiçbiri esenlik nedir bilmez.
\par 9 Diyorlar ki, "Bu yüzden adalet bizden uzak, Dogruluk bize erisemiyor. Isik bekliyoruz, yalniz karanlik var; Parilti bekliyor, koyu karanlikta yürüyoruz.
\par 10 Kör gibi duvari el yordamiyla ariyor, Yolumuzu bulmaya çalisiyoruz. Ögle vakti alaca karanliktaymis gibi tökezliyoruz, Güçlüler arasinda ölüler gibiyiz.
\par 11 Hepimiz ayi gibi homurdaniyor, Güvercin gibi inim inim inliyoruz. Adalet bekliyoruz, ortada yok; Kurtulus bekliyoruz, bizden uzak.
\par 12 Çünkü sana çok kez baskaldirdik, Günahlarimiz bize karsi taniklik ediyor, Isyanlarimiz hep yanibasimizda. Suçlarimizi kabul ediyoruz.
\par 13 Baskaldirip RAB'bi yadsidik, Tanrimiz'i izlemez olduk. Zorbalik, isyan dolu sözler söyledik, Yüregimizde tasarladigimiz yalanlari mirildandik.
\par 14 Adalet püskürtüldü, dogruluk bizden uzak duruyor. Çünkü gerçek, kent meydaninda sendeleyip düstü, Dürüstlük aramiza giremez oldu.
\par 15 Hiçbir yerde gerçek yok, Kötülükten çekinen soyuluyor!" RAB olanlari gördü ve adaletin yokluguna üzüldü.
\par 16 Kimsenin olmadigini gördü, Aracilik edecek birinin olmadigina sasti. Kendi gücüyle kurtulus sagladi, Dogrulugu O'na destek oldu.
\par 17 Dogrulugu gögüslük gibi kusandi, Kurtulus migferini basina takti, Öç giysisini giydi, Gayreti kaftan gibi sarindi.
\par 18 Herkese yaptiklarinin karsiligini verecek. Düsmanlarina öfkeyle, Hasimlarina ve kiyi halklarina cezayla karsilik verecek.
\par 19 Böylece batidan doguya kadar insanlar RAB'bin adindan ve yüceliginden korkacak. Çünkü düsman azgin bir irmak gibi geldiginde, RAB'bin Ruhu onu kaçirtacak.
\par 20 RAB diyor ki, "Kurtarici Siyon'a, Yakup soyundan olup baskaldirmaktan vazgeçenlere gelecek.
\par 21 Bana gelince, onlarla yapacagim antlasma sudur: Üzerindeki Ruhum, agzina koydugum sözler Simdiden sonsuza dek senin, çocuklarinin, Torunlarinin agzindan düsmeyecek."

\chapter{60}

\par 1 "Kalk, parla; Çünkü Isigin geliyor, RAB'bin yüceligi üzerine doguyor.
\par 2 Dünyayi karanlik, halklari koyu karanlik örtüyor; Oysa RAB senin üzerine dogacak, Yüceligi üzerinde görünecek.
\par 3 Uluslar senin Isigina, Krallar üzerine dogan aydinliga gelecek.
\par 4 "Basini kaldir da çevrene bir bak, Hepsi toplanmis sana geliyor. Ogullarin uzaktan geliyor, Kizlarin kucakta tasiniyor.
\par 5 Bunu görünce yüzün parlayacak, Yüregin heyecandan hizli hizli çarpacak; Çünkü denizin zenginlikleri senin olacak, Uluslarin serveti sana akacak.
\par 6 "Deve sürüleri, Midyan'in ve Efa'nin deve yavrulari Senin topraklarini dolduracak. Bütün Saba halki geliyor, Altin ve günnük getiriyor, RAB'bin erdemlerini ilan ediyorlar.
\par 7 Kedar'in bütün sürüleri sana gelecek, Nevayot'un koçlari senin buyrugunda olacak, Sunagimin üzerinde kabul edilen sunular olarak sunulacak. Böylece görkemli tapinagimi daha görkemli kilacagim.
\par 8 "Nedir bunlar, bulut gibi, Yuvalarina yaklasan güvercinler gibi süzülüp gelenler?
\par 9 Bana umut baglayan kiyi halklarinin, Ticaret gemileri öncülügünde Senin çocuklarini altinlariyla, gümüsleriyle birlikte Tanrin RAB'bin onuruna Israil'in Kutsali'na Uzaktan getiren gemileridir bunlar. RAB seni görkemli kildi.
\par 10 "Yabancilar senin surlarini onaracak, Krallari sana hizmet edecek. Öfkelendigimde seni cezalandirdiysam da, Kabul ettigimde sana merhamet gösterecegim.
\par 11 Kapilarin hep açik duracak, Uluslarin serveti ve zafer alaylari ardinda yürütülen yenik Krallar Gece gündüz açik kalan bu kapilardan girsin diye.
\par 12 Çünkü sana kulluk etmeyen ulus ya da krallik yok olacak, Evet, o uluslar tam bir yikima ugrayacak.
\par 13 "Lübnan'in görkemi olan çam, köknar ve selvi agaçlari, Tapinagimi süslemek için hep birlikte sana tasinacak. Ayak bastigim yeri görkemli kilacagim.
\par 14 Seni ezenlerin çocuklari Gelip önünde egilecekler; Seni hor görenlerin hepsi, 'RAB'bin kenti, Israil'in Kutsali'nin Siyon'u Diyerek ayaklarina kapanacaklar.
\par 15 "Kimsenin ugramadigi, terk edilmis, Nefret edilen bir yer oldugun halde Seni sonsuz bir övünç kaynagi, Bütün kusaklarin sevinci kilacagim.
\par 16 Uluslar ve kralliklar Bir anne gibi seni emzirecekler. O zaman bileceksin ki, seni kurtaran RAB, Seni fidyeyle kurtaran, Yakup'un Güçlüsü benim.
\par 17 Sana tunç* yerine altin, Demir yerine gümüs, agaç yerine tunç, Tas yerine demir getirecegim. Barisi yöneticin, dogrulugu önderin yapacagim.
\par 18 Ülkenden siddet, sinir boylarindan Soygun ve yikim haberleri duyulmayacak artik. Surlarina Kurtulus, kapilarina Övgü adini vereceksin.
\par 19 "Gündüz isigin günes olmayacak artik, Ay da aydinlatmayacak seni; Çünkü RAB sonsuz isigin, Tanrin görkemin olacak.
\par 20 Artik günesin batmayacak, ayin çekilmeyecek, Çünkü RAB sonsuz isigin olacak, Sona erecek yas günlerin.
\par 21 Halkinin hepsi dogru kisiler olacak; El emegim, görkemimi göstermek için diktigim fidan, Ülkeyi sonsuza dek mülk edinecek.
\par 22 En küçük ailen bini bulacak, Sayica en az olani koca bir ulus olacak. Ben RAB, zamani gelince bunu hizlandiracagim."

\chapter{61}

\par 1 Egemen RAB'bin Ruhu üzerimdedir. Çünkü O beni yoksullara müjde iletmek için meshetti*. Yüregi ezik olanlarin yaralarini sarmak için, Tutsaklara serbest birakilacaklarini, Zindanlarda bulunanlara kurtulacaklarini, RAB'bin lütuf yilini, Tanrimiz'in öç alacagi günü ilan etmek, Yas tutanlarin hepsini avutmak, Siyon'da yas tutanlara yardim saglamak -Kül yerine çelenk, Yas yerine sevinç yagi, Çaresizlik ruhu yerine Onlara övgü giysisini vermek- için RAB beni gönderdi. Öyle ki, RAB'bin görkemini yansitmak için, Onlara "RAB'bin diktigi dogruluk agaçlari" densin.
\par 4 O zaman eski yikintilari yeniden insa edecek, Çoktan viraneye dönmüs yerleri yeniden kuracak, Kusaklar boyu yikik kalmis kentleri onaracaklar.
\par 5 Yabancilar sürülerinizi güdecek, Irgatiniz, bagciniz olacaklar.
\par 6 Sizlerse RAB'bin kâhinleri, Tanrimiz'in görevlileri diye çagrilacaksiniz. Uluslarin servetiyle beslenecek, Zenginlikleriyle övüneceksiniz.
\par 7 Utanç yerine iki kat onur bulacaksiniz, Asagilanma yerine payinizla sevineceksiniz, Böylece ülkenizde iki kat mülk edineceksiniz; Sevinciniz sonsuz olacak.
\par 8 "Çünkü ben RAB adaleti severim, Nefret ederim soygun ve haksizliktan. Sözümde durup hak ettiklerini verecek, Onlarla ebedi bir antlasma yapacagim.
\par 9 Soylarindan gelenler uluslar arasinda, Torunlari halklar arasinda taninacak. Onlari gören herkes RAB'bin kutsadigi soy olduklarini anlayacak."
\par 10 RAB'de büyük sevinç bulacagim, Tanrim'la yüregim cosacak. Çünkü çelenkle süslenmis güvey gibi, Takilarini kusanmis gelin gibi, Bana kurtulus giysisini giydirdi, Beni dogruluk kaftaniyla örttü.
\par 11 Toprak filizlerini nasil çikartir, Bahçe ekilen tohumlari nasil yetistirirse, Egemen RAB de dogruluk ve övgüyü Bütün uluslarin önünde öyle yetistirecek.

\chapter{62}

\par 1 Zaferi isik gibi parlayincaya, Kurtulusu mesale gibi yanincaya dek Siyon ugruna susmayacak, Yerusalim ugruna sessiz kalmayacagim.
\par 2 Uluslar senin zaferini, Bütün krallar görkemini görecek. RAB'bin kendi agziyla belirledigi yeni bir adla anilacaksin.
\par 3 RAB'bin elinde güzellik taci, Tanrin'in elinde krallik sarigi olacaksin.
\par 4 Artik sana "Terk edilmis", Ülkene "Virane" denmeyecek; Bunun yerine sana "Sevdigim", Ülkene "Evli" denecek. Çünkü RAB seni seviyor, Ülken de evli sayilacak.
\par 5 Bir delikanli bir kizla nasil evlenirse, Ogullarin da seninle öyle evlenecek. Güvey gelinle nasil sevinirse, Tanrin da seninle öyle sevinecek.
\par 6 Ey Yerusalim, surlarina bekçiler diktim, Gece gündüz hiç susmayacaklar. Ey RAB'be sözünü animsatanlar, Yerusalim'i pekistirene, Onu yeryüzünün övüncü kilana dek Durup dinlenmeden RAB'be yakarin, O'na rahat vermeyin.
\par 8 RAB sag elini, güçlü kolunu kaldirip ant içti: "Tahilini bir daha düsmanlarina yedirmeyecegim, Emek verdigin yeni sarabi yabancilar içmeyecek.
\par 9 Tahili devsiren yiyecek Ve RAB'be övgüler sunacak. Üzümü toplayan, Sarabini kutsal avlularimda içecek."
\par 10 Geçin, geçin kent kapilarindan! Halkin yolunu açin! Toprak yigip yol yapin, Taslari ayiklayin, uluslar için sancak dikin!
\par 11 RAB çagrisini dünyanin dört bucagina duyurdu: "Siyon kizina*, 'Iste kurtulusun geliyor deyin, 'Ücreti kendisiyle birlikte, ödülü önündedir."
\par 12 Siyon halkina, "RAB'bin fidyeyle kurtardigi kutsal halk" diyecekler. Ve sen Yerusalim, "Aranan, terk edilmemis kent" diye anilacaksin.

\chapter{63}

\par 1 Edom'dan, Bosra'dan Al giysiler içinde bu gelen kim? Göz kamastirici giysiler içinde, Büyük güçle yürüyen kim? "O benim! Adaleti duyuran, Kurtarmaya gücü olan."
\par 2 Giysilerin neden kirmizi? Üstün basin neden çukurda üzüm çigneyen biri gibi kizila bulanmis?
\par 3 "Çukurda üzümü tek basima çignedim, Yanimda halklardan kimse yoktu. Öfkeyle çignedim onlari, Gazapla ayaklarimin altina aldim. Kanlari giysilerime siçradi, bütün elbisemi kirletti.
\par 4 Çünkü öç alma günü yüregimdeydi, Halkimi kurtaracagim yil gelmisti.
\par 5 Baktim, yardim edecek kimse yoktu, Destek verecek kimsenin olmayisina sastim; Gücüm kurtulus sagladi, Gazabim bana destek oldu.
\par 6 Öfkeyle halklari çignedim, Onlari gazapla sarhos ettim, Yere akittim kanlarini."
\par 7 Sefkati ve iyiligi uyarinca Bizim için yaptiklarindan, evet, Israil halki için yaptigi bütün iyiliklerinden ötürü RAB'bin iyiliklerini ve övülesi islerini anacagim.
\par 8 RAB dedi ki, "Onlar kuskusuz benim halkim, Beni aldatmayacak çocuklardir." Böylece onlarin Kurtaricisi oldu.
\par 9 Sikinti çektiklerinde O da sikinti çekti. Huzurundan çikan melek onlari kurtardi. Sevgisi ve merhametinden ötürü onlari kurtardi, Geçmiste onlari sürekli yüklenip tasidi.
\par 10 Ama baskaldirip O'nun Kutsal Ruhu'nu incittiler. O da düsmanlari olup onlara karsi savasti.
\par 11 Sonra halki eski günleri, Musa'nin dönemini animsadi. "Çobanlariyla birlikte onlari denizden geçiren, Kutsal Ruhu'nu aralarina yerlestiren, Görkemli gücüyle Musa'nin saginda yol alan, Sonsuz onur kazanmak için önlerinde sulari yaran, Bir at nasil tökezlemeden kirdan geçerse Onlari deniz yatagindan öyle geçiren RAB nerede?" Diye sordular.
\par 14 Ovaya götürülen sürü gibi RAB'bin Ruhu onlari rahata kavusturdu. Iste adini onurlandirmak için Halkina böyle yol gösterdi.
\par 15 Ya RAB, gökten bak, Kutsal, görkemli ve yüce yerinden bizi gör! Gayretin, gücün nerede? Gönlündeki özlem ve merhameti Bizden esirgedin.
\par 16 Babamiz sensin. Ibrahim bizi tanimasa da, Israil bizi kabul etmese de, Babamiz'sin, ya RAB, Ezelden beri adin "Kurtaricimiz"dir.
\par 17 Ya RAB, neden bizi yolundan saptiriyor, Inatçi kiliyor, Senden korkmamizi engelliyorsun? Kullarin ugruna, Mirasin olan oymaklarin ugruna geri dön.
\par 18 Kutsal halkin kisa süre tapinagina sahip oldu, Ama düsmanlarimiz onu çignedi.
\par 19 Öteden beri yönetmedigin, Sana ait olmayan bir halk gibi olduk.

\chapter{64}

\par 1 Ya RAB, adini düsmanlarina duyurmak için Keske gökleri yarip insen! Daglar önünde sarsilsa! Gelisin, atesin çalilari tutusturmasina, Suyu kaynatmasina benzese! Uluslar senin önünde titrese!
\par 3 Beklemedigimiz olaganüstü isler yaparak Yeryüzüne indin, daglar önünde sarsildi.
\par 4 Çünkü kendisine umut baglayanlar için Etkin olan tek Tanri sensin; Senden baskasini hiçbir zaman hiç kimse isitmedi, Hiçbir kulak duymadi, hiçbir göz görmedi.
\par 5 Dogru olani sevinçle yapanlarin, Senin yollarindan yürüyüp seni unutmayanlarin yardimina kosarsin. Ama onlara karsi uzun süre günah isledigimizde öfkelendin. Nasil kurtuluruz?
\par 6 Hepimiz murdar* olanlara benzedik, Bütün dogru islerimiz kirli âdet bezi gibi. Yaprak gibi soluyoruz, Suçlarimiz rüzgar gibi sürükleyip götürüyor bizi.
\par 7 Adinla sana yakaran, sana tutunmak için çaba gösteren yok; Çünkü bizden yüz çevirdin, Suçlarimiz yüzünden bizi tükettin.
\par 8 Yine de Babamiz sensin, ya RAB, Biz kiliz, sen çömlekçisin. Hepimiz senin ellerinin eseriyiz.
\par 9 Ya RAB, fazla öfkelenme, Suçlarimizi sonsuza dek anma. Lütfen bak bize, hepimiz senin halkiniz.
\par 10 Kutsal kentlerin çöllesti, Siyon çöl oldu, Yerusalim viraneye döndü.
\par 11 Atalarimizin sana övgü sundugu Kutsal ve görkemli tapinagimiz yandi, Deger verdigimiz her yer yikintiya döndü.
\par 12 Bunlara karsin, ya RAB, Hâlâ kendini tutacak misin, Suskun kalip bize alabildigine eziyet çektirecek misin?

\chapter{65}

\par 1 "Beni sormayanlara göründüm, Aramayanlar beni buldu. Adimla anilmayan bir ulusa, 'Buradayim, buradayim dedim.
\par 2 Kötü yolda yürüyen, Kendi tasarilarinin ardinca giden Asi bir halka Bütün gün ellerimi uzatip durdum.
\par 3 O halk ki, bahçelerde kurban keserek, Tuglalar üzerinde buhur yakarak Gözümün içine baka baka boyuna öfkelendirir beni.
\par 4 Mezarlikta oturur, Gizli yerlerde geceler, Domuz eti yerler; Kaplarinda haram et var.
\par 5 Birbirlerine, 'Uzak dur, yaklasma derler, 'Çünkü ben senden daha kutsalim. Böyleleri burnumda duman, Bütün gün yanan atestir.
\par 6 "Bakin, yanit önümde yazili duruyor. Susmayacak, suçlarinin karsiligini verecegim. Onlarin da atalarinin da suçlarinin cezasini Baslarina getirecegim" diyor RAB. "Çünkü daglarin üzerinde buhur yaktilar, Tepelerin üzerinde beni asagiladilar. Bu nedenle eskiden yaptiklarinin karsiligini Baslarina getirecegim."
\par 8 RAB diyor ki, "Taneleri sulu salkimi görünce, Halk, 'Salkimi yok etmeyin, bereket onda diyor. Kullarimin hatiri için ben de öyle yapacagim, Onlarin hepsini yok etmeyecegim.
\par 9 Yakup soyunu sürdürecek, Daglarimi miras alacak olanlari Yahuda soyuna birakacagim. Seçtiklerim oralari miras alacak, Kullarim orada yasayacak.
\par 10 Saron, bana yönelen halkimin sürülerine agil, Akor Vadisi sigirlarina barinak olacak.
\par 11 "Ama sizler, RAB'bi terk edenler, Kutsal dagimi unutanlar, Talih ilahina sofra kuranlar, Kismet ilahina karisik sarap sunanlar,
\par 12 Ben de sizi kilica kismet edecegim, Bogazlanmak üzere egileceksiniz hepiniz. Çünkü çagirdigimda yanit vermediniz, Konustugumda dinlemediniz; Gözümde kötü olani yaptiniz, Hoslanmadigimi seçtiniz."
\par 13 Bu yüzden Egemen RAB diyor ki, "Bakin, kullarim yemek yiyecek, Ama siz aç kalacaksiniz. Kullarim içecek, Ama siz susuz kalacaksiniz. Kullarim sevinecek, Ama sizin yüzünüz kizaracak.
\par 14 Kullarim mutluluk içinde ezgiler söyleyecek, Ama siz yürek acisindan feryat edecek, Ezik bir ruhla haykiracaksiniz.
\par 15 Adiniz seçtiklerimin agzinda ancak lanet olarak kalacak. Egemen RAB sizi öldürecek, Ama kullarina baska bir ad verecek.
\par 16 Öyle ki, ülkede kim bereket istese Sadik Tanri'dan isteyecek; Ülkede kim ant içse, Sadik Tanri üzerine ant içecek. Çünkü geçmis sikintilar unutulup Gözümden saklanacak."
\par 17 "Çünkü bakin, yeni bir yeryüzü, Yeni bir gök yaratmak üzereyim; Geçmistekiler anilmayacak, akla bile gelmeyecek.
\par 18 Yaratacaklarimla sonsuza dek sevinip cosun; Çünkü Yerusalim'i cosku, Halkini sevinç kaynagi olarak yaratacagim.
\par 19 Yerusalim için sevinecek, Halkim için cosacagim. Orada aglayis ve feryat duyulmayacak artik.
\par 20 Orada birkaç gün yasayip ölen bebekler olmayacak, Yasini basini almadan kimse ölümü tatmayacak. Yüz yasinda ölen genç, Yüz yasina basmayan kisi lanetli sayilacak.
\par 21 Evler yapip içlerinde yasayacak, Baglar dikip meyvesini yiyecekler.
\par 22 Yaptiklari evlerde baskasi oturmayacak, Diktikleri bagin meyvesini baskasi yemeyecek. Çünkü halkim agaçlar gibi uzun yasayacak, Seçtiklerim, elleriyle ürettiklerinin tadini çikaracaklar.
\par 23 Emek vermeyecekler bos yere, Felakete ugrayan çocuklar dogurmayacaklar. Çünkü kendileri de çocuklari da RAB'bin kutsadigi soy olacak.
\par 24 Onlar bana yakarmadan yanit verecek, Daha konusurlarken isitecegim onlari.
\par 25 Kurtla kuzu birlikte otlayacak, Aslan sigir gibi saman yiyecek. Yilanin yiyecegiyse toprak olacak. Kutsal dagimin hiçbir yerinde Kimse zarar vermeyecek, yok etmeyecek." Böyle diyor RAB.

\chapter{66}

\par 1 RAB diyor ki, "Gökler tahtim, Yeryüzü ayaklarimin taburesidir. Nerede benim için yapacaginiz ev, Neresi dinlenecegim yer?
\par 2 Çünkü bütün bunlari ellerim yapti, Hepsi böylece var oldu" diyor RAB. "Ancak ben alçakgönüllüye, ruhu ezik olana, Sözümden titreyen kisiye deger veririm.
\par 3 Sigir bogazlayan, adam öldüren gibidir, Davar kurban eden, köpek boynu kiran, Tahil sunusu* getiren, domuz kani sunan, Anma sunusu olarak günnük yakan, putperest gibidir. Evet, bunlar kendi yollarini seçtiler, Yaptiklari igrençliklerden hoslaniyorlar.
\par 4 Ben de onlar için yikimi seçecek, Korktuklarini baslarina getirecegim. Çünkü çagirdigimda yanit veren olmadi, Konustugumda dinlemediler, Gözümde kötü olani yaptilar, Hoslanmadigimi seçtiler."
\par 5 RAB'bin sözünden titreyenler, Kulak verin O'nun söylediklerine: "Sizden nefret eden, Adimdan ötürü sizi dislayan kardesleriniz, 'RAB yüceltilsin de sevincinizi görelim! diyorlar. Utandirilacak olan onlardir.
\par 6 Kentten gürültülü sesler, Tapinaktan bir ses yükseliyor! Düsmanlarina hak ettikleri karsiligi veren RAB'bin sesidir bu.
\par 7 "Dogum sancisi çekmeden dogurdu, Sancisi tutmadan bir erkek çocuk dogurdu.
\par 8 Kim böyle bir sey duydu? Kim böyle seyler gördü? Bir ülke bir günde dogar mi, Bir anda dogar mi bir ulus? Ama Siyon, agrisi tutar tutmaz çocuklarini dogurdu.
\par 9 Dogum anina dek getiririm de Doguracak gücü vermez miyim?" diyor RAB. "Doguracak güç veren ben, rahmi kapatir miyim?" diyor Tanrin.
\par 10 "Yerusalim'le birlikte sevinin, Onu sevenler, hepiniz onun için cosun, Yerusalim için yas tutanlar, onunla sevinçle cosun.
\par 11 Öyle ki, onun avutucu memelerini emip doyasiniz, Kana kana içip Onun yüce bollugundan zevk alasiniz."
\par 12 Çünkü RAB diyor ki, "Bakin, esenligi bir irmak gibi, Uluslarin servetini taskin bir irmak gibi ona akitacagim. Ondan beslenecek, kucakta tasinacak, Dizleri üzerinde sallanacaksiniz.
\par 13 Çocugunu avutan bir anne gibi avutacagim sizi, Yerusalim'de avuntu bulacaksiniz.
\par 14 Bunlari gördügünüzde yüreginiz sevinecek, Bedenleriniz körpe ot gibi tazelenecek. Herkes bilecek ki, RAB'bin koruyucu eli kullarinin, Gazabi ise düsmanlarinin üzerindedir."
\par 15 Bakin, RAB atesle geliyor, Savas arabalari kasirga gibi. Siddetli öfkesini, Azarini alev alev dökmek üzere.
\par 16 Çünkü O bütün insanligi ates ve kiliçla yargilayacak, Pek çok kisiyi öldürecek.
\par 17 "Bahçelere girmek için kendilerini aritip kutsayanlar, domuz, fare ve öteki igrenç hayvanlarin etini yiyenlerin ortasinda durani izleyenler hep birlikte yok olacaklar*ff*" diyor RAB,
\par 18 "Çünkü ben onlarin eylemlerini de düsüncelerini de bilirim. Bütün uluslari ve dilleri bir araya toplayacagim an geliyor; gelip yüceligimi görecekler.
\par 19 "Aralarina bir belirti koyacagim. Onlardan kaçip kurtulanlari uluslara, Tarsis'e, Pûl'a, Lud'a -yay gerenlere- Tuval'a, Yâvan'a, ünümü duymamis, yüceligimi görmemis uzak kiyi halklarina gönderecegim. Uluslar arasinda yüceligimi ilan edecekler.
\par 20 Israilogullari tahil sunularini* pak kaplar içinde RAB'bin Tapinagi'na nasil getiriyorsa, onlar da bütün kardeslerinizi uluslardan atlarla, savas arabalariyla, at arabalariyla, katirlarla, develerle kutsal dagima, Yerusalim'e, RAB'be sunu olarak getirecekler." Böyle diyor RAB.
\par 21 "Onlarin arasindan kimilerini kâhin ve Levili olarak seçecegim" diyor RAB.
\par 22 "Çünkü yaratacagim yeni yer ve gök önümde nasil duracaksa, soyunuz ve adiniz da öyle duracak" diyor RAB.
\par 23 "Yeni Ay'dan Yeni Ay'a, Sabat Günü'nden* Sabat Günü'ne bütün insanlar önüme gelip bana tapinacaklar" diyor RAB.
\par 24 "Disari çiktiklarinda bana baskaldirmis olanlarin cesetlerini görecekler. Öylelerini kemiren kurt ölmez, yakan ates sönmez. Bütün insanlar onlardan igrenecek."


\end{document}