\begin{document}

\title{Hagay}


\chapter{1}

\par 1 Kral Darius'un kralliginin ikinci yilinda, altinci ayin* birinci günü RAB Peygamber Hagay araciligiyla Sealtiel'in torunu Yahuda Valisi Zerubbabil ve Yehosadak oglu Baskâhin Yesu'ya seslendi:
\par 2 "Her Seye Egemen RAB diyor ki, 'Bu halk, RAB'bin Tapinagi'ni yeniden kurmak için vakit daha gelmedi diyor.'"
\par 3 Sonra RAB, Peygamber Hagay araciligiyla söyle seslendi:
\par 4 "Bu tapinak yikik durumdayken, sizin agaç kaplamali evlerinizde oturmanizin sirasi mi?"
\par 5 Her Seye Egemen RAB diyor ki, "Simdi tuttugunuz yollari iyi düsünün!
\par 6 Çok ektiniz ama az biçtiniz; yiyorsunuz ama doyamiyorsunuz, içiyorsunuz ama neselenemiyorsunuz; giyiniyorsunuz ama isinamiyorsunuz; ücretinizi aliyorsunuz ama paranizi sanki delik keseye koyuyorsunuz."
\par 7 Her Seye Egemen RAB, "Tuttugunuz yollari iyi düsünün!" diyor, "Daglara çikip kütük getirin, tapinagi yeniden kurun. Öyle ki, ondan hosnut olayim, yüceltileyim.
\par 9 Bol ürün umdunuz ama az topladiniz. Eve ne getirdiyseniz üfleyip dagittim. Acaba neden?" Böyle soruyor Her Seye Egemen RAB. "Yikik duran tapinagimdan ötürü! Oysa hepiniz kendi evinizle ugrasiyorsunuz.
\par 10 Iste bunun içindir ki, gök çiyini, toprak ürününü sizden esirgiyor.
\par 11 Ülkeyi -daglarini, tahilini, yeni sarabini, zeytinyagini, topragin verdigi ürünleri, insanlarini, hayvanlarini, ellerinizin bütün emegini- kuraklikla cezalandirdim."
\par 12 Sealtiel'in torunu Zerubbabil, Yehosadak oglu Baskâhin Yesu ve sürgünden dönen halkin tümü Tanrilari RAB'bin sözüne, O'nun tarafindan gönderilen Peygamber Hagay'in sözlerine kulak verdiler. Halk RAB'den korktu.
\par 13 Sonra RAB'bin ulagi Hagay, RAB'bin su sözlerini halka bildirdi: "RAB, 'Ben sizinle birlikteyim' diyor."
\par 14 Böylece RAB Sealtiel'in torunu Yahuda Valisi Zerubbabil'i, Yehosadak oglu Baskâhin Yesu'yu ve sürgünden dönen halkin tümünü bu konuda harekete geçirdi. Darius'un kralliginin ikinci yilinda, altinci ayin* yirmi dördüncü günü gelip Tanrilari Her Seye Egemen RAB'bin Tapinagi'nda ise basladilar.

\chapter{2}

\par 1 Yedinci ayin* yirmi birinci günü RAB Peygamber Hagay araciligiyla söyle seslendi:
\par 2 "Sealtiel'in torunu Yahuda Valisi Zerubbabil'e, Yehosadak oglu Baskâhin Yesu'ya ve sürgünden dönen halka de ki,
\par 3 'Aranizda bu tapinagi önceki görkemiyle gören kaldi mi? Simdi size nasil görünüyor? Bir hiç olarak görünmüyor mu?
\par 4 Simdi sen, ey Zerubbabil, yüreklen!' RAB böyle diyor. 'Ey Yehosadak oglu Baskâhin Yesu, yüreklen! Ey ülke halki, yüreklen!' RAB böyle diyor. 'Isi sürdürün. Çünkü ben sizinle birlikteyim.' Böyle diyor Her Seye Egemen RAB.
\par 5 'Misir'dan çiktiginizda, size bu konuda söz verdim. Ruhum aranizdadir. Korkmayin!'
\par 6 "Her Seye Egemen RAB diyor ki, 'Kisa zamanda bir kez daha yeri, gögü, denizi, karayi sarsacagim.
\par 7 Bütün uluslari sarsacagim, degerli esyalarini buraya getirecekler. Ben de bu tapinagi görkemle dolduracagim.' Böyle diyor Her Seye Egemen RAB.
\par 8 'Gümüs de, altin da benim' diyor Her Seye Egemen RAB.
\par 9 'Yeni tapinagin görkemi, öncekinden daha büyük olacak. Buraya esenlik verecegim.' Böyle diyor Her Seye Egemen RAB."
\par 10 Darius'un kralliginin ikinci yilinda, dokuzuncu ayin* yirmi dördüncü günü RAB, Peygamber Hagay'a seslendi:
\par 11 "Her Seye Egemen RAB diyor ki, 'Kâhinlere* yasayla ilgili su soruyu sor:
\par 12 Eger biri giysisinin kivrimlari arasinda kutsanmis et tasir ve o kivrim ekmege, yemege, saraba, zeytinyagina ya da baska bir yiyecege degerse, o yiyecek kutsal olur mu?'" Kâhinler, "Hayir" diye yanitladilar.
\par 13 Hagay konusmasini söyle sürdürdü: "Ölüye dokundugu için kirli sayilan biri, bu yiyeceklerden birine dokunursa, o yiyecek kirlenmis olur mu?" Kâhinler, "Evet, kirlenmis olur" diye karsilik verdiler.
\par 14 Bunun üzerine Hagay söyle dedi: "RAB, 'Bu halk, bu ulus gözümde böyledir' diyor, 'Her yaptiklari, sunakta her sunduklari da kirlidir.'"
\par 15 "'Bugüne dek olanlari iyi düsünün; RAB'bin Tapinagi'nda tas üstüne tas konulmadan önce, yirmi ölçeklik bir tahil yiginina gelen biri, yalnizca on ölçek bulurdu; sarap teknesinden elli ölçek çikarmaya varan biri, yalnizca yirmi ölçek bulurdu.
\par 17 Ellerinizin bütün emegini samyeliyle, küfle, doluyla cezalandirdim. Yine de bana dönmediniz.' RAB böyle diyor.
\par 18 'Bugünden, dokuzuncu ayin* yirmi dördüncü gününden, RAB'bin Tapinagi'nin temelinin atildigi günden baslayarak olacaklari iyi düsünün.
\par 19 Ambarda hiç tohum kaldi mi? Asma, incir, nar, zeytin agaçlari bugüne dek ürün verdi mi? "'Bugünden baslayarak üzerinize bereket yagdiracagim.'"
\par 20 Ayin yirmi dördüncü günü RAB Hagay'a ikinci kez seslendi:
\par 21 "Yahuda Valisi Zerubbabil'e de ki, ben yeri, gögü sarsmak üzereyim.
\par 22 Krallarin tahtlarini devirecegim, yabanci uluslarin gücünü yok edecegim. Savas arabalariyla sürücülerini de devirecegim; atlarla binicileri düsecek, hepsi kardesinin kiliciyla öldürülecek.
\par 23 "Her Seye Egemen RAB 'O gün seni alacagim, ey Sealtiel'in torunu kulum Zerubbabil' diyor, 'Ve seni mühür yüzügü gibi yapacagim. Çünkü ben seni seçtim.' Böyle diyor Her Seye Egemen RAB."


\end{document}