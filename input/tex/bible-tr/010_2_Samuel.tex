\begin{document}

\title{2 Samuel}


\chapter{1}

\par 1 Saul'un ölümünden sonra Amalekliler'e karsi kazandigi zaferden dönen Davut Ziklak'ta iki gün kaldi.
\par 2 Üçüncü gün, Saul'un ordugahindan giysileri yirtilmis, basi toz toprak içinde bir adam geldi. Adam Davut'a yaklasinca önünde yere kapandi.
\par 3 Davut, "Nereden geliyorsun?" diye sordu. Adam, "Israil ordugahindan kaçip kurtuldum" dedi.
\par 4 Davut, "Ne oldu? Bana anlat" dedi. Adam askerlerin savas alanindan kaçtigini, birçogunun düsüp öldügünü, Saul'la oglu Yonatan'in da ölüler arasinda oldugunu anlatti.
\par 5 Davut, kendisine haberi veren genç adama, "Saul'la oglu Yonatan'in öldügünü nereden biliyorsun?" diye sordu.
\par 6 Genç adam söyle yanitladi: "Bir rastlanti sonucu Gilboa Dagi'ndaydim. Saul mizragina dayanmisti. Atlilarla savas arabalari ona dogru yaklasiyordu.
\par 7 Saul arkasina dönüp beni görünce seslendi. Ben de, 'Buyrun, buradayim dedim.
\par 8 "Saul, 'Sen kimsin? diye sordu. "'Ben bir Amalekli'yim diye yanitladim.
\par 9 "Saul, 'Ne olur üstüme var ve beni öldür! dedi, 'Çünkü çektigim acilardan kurtulmak istiyorum.
\par 10 Bu yüzden varip onu öldürdüm. Çünkü yere düstükten sonra yasayamayacagini biliyordum. Basindaki taçla kolundaki bilezigi aldim ve onlari buraya, efendime getirdim."
\par 11 Bunun üzerine Davut'la yanindakiler giysilerini yirttilar.
\par 12 Kiliçtan geçirilen Saul, oglu Yonatan ve RAB'bin halki olan Israilliler için aksama dek yas tutup agladilar, oruç* tuttular.
\par 13 Davut, kendisine haber getiren genç adama, "Nerelisin?" diye sordu. Adam, "Ben yabanciyim, bir Amalekli'nin ogluyum" dedi.
\par 14 Davut, "RAB'bin meshettigi* kisiye el kaldirip onu yok etmekten korkmadin mi?" diye sordu.
\par 15 Sonra adamlarindan birini çagirip, "Git, öldür onu!" diye buyurdu. Böylece adam Amalekli'yi vurup öldürdü.
\par 16 Davut Amalekli'ye, "Kanindan sen kendin sorumlusun" demisti, "Çünkü 'RAB'bin meshettigi kisiyi ben öldürdüm demekle kendine karsi agzinla taniklikta bulundun."
\par 17 Davut Saul'la oglu Yonatan için agit yakti.
\par 18 Sonra Yasar Kitabi'nda yazilan Yay adindaki agitin Yahuda halkina ögretilmesini buyurdu:
\par 19 "Ey Israil, senin yüceligin yüksek tepelerinde yok oldu! Güçlüler nasil da yere serildi!
\par 20 Haberi ne Gat'a duyurun, Ne de Askelon sokaklarinda yayin. Öyle ki, ne Filistliler'in kizlari sevinsin, Ne de sünnetsizlerin* kizlari cossun.
\par 21 Ey Gilboa daglari, Üzerinize ne çiy ne de yagmur düssün. Ürün veren tarlalariniz olmasin. Çünkü güçlünün kalkani, Bir daha yag sürülmeyecek olan Saul'un kalkani Orada bir yana atildi!
\par 22 Yonatan'in yayi yere serilmislerin kanindan, Yigitlerin bedenlerinden hiç geri çekilmedi. Saul'un kilici hiç bosa savrulmadi.
\par 23 Saul'la Yonatan tatli ve sevimliydiler, Yasamda da ölümde de ayrilmadilar. Kartallardan daha çevik, Aslanlardan daha güçlüydüler.
\par 24 Ey Israil kizlari! Sizi al renkli, süslü giysilerle donatan, Giysinizi altin süslerle bezeyen Saul için aglayin!
\par 25 Güçlüler nasil da yere serildi savasta! Yonatan senin yüksek tepelerinde ölü yatiyor.
\par 26 Senin için üzgünüm, kardesim Yonatan. Benim için çok degerliydin. Sevgin kadin sevgisinden daha üstündü.
\par 27 Güçlüler nasil da yere serildi! Savas silahlari yok oldu!"

\chapter{2}

\par 1 Bundan sonra Davut RAB'be, "Yahuda kentlerinden birine gideyim mi?" diye sordu. RAB, "Git" dedi. Davut, "Nereye gideyim?" diye sorunca, RAB, "Hevron'a" diye karsilik verdi.
\par 2 Bunun üzerine Davut, iki esiyle -Yizreelli Ahinoam ve Karmelli Naval'in dulu Avigayil'le- birlikte oraya gitti.
\par 3 Aileleriyle birlikte adamlarini da götürdü. Hevron'a bagli kentlere yerlestiler.
\par 4 Yahudalilar Hevron'a giderek orada Davut'u Yahuda Krali olarak meshettiler*. Saul'u gömenlerin Yaves-Gilatlilar oldugu Davut'a bildirildi.
\par 5 Davut onlara ulaklar göndererek söyle dedi: "Efendiniz Saul'u gömmekle ona yaptiginiz iyilikten dolayi RAB sizi kutsasin.
\par 6 RAB simdi size baglilikla, iyilikle davransin. Bunu yaptiginiz için ben de size ayni sekilde iyilik yapacagim.
\par 7 Simdi güçlü ve yürekli olun, çünkü efendiniz Saul öldü. Yahuda halki beni krallari olarak meshetti."
\par 8 Saul'un ordu komutani Ner oglu Avner, Saul oglu Is-Boset'i yanina alip Mahanayim'e götürmüstü.
\par 9 Avner onu orada Gilat, Asurlular, Yizreel, Efrayim, Benyamin ve bütün Israil'in krali yapti.
\par 10 Saul oglu Is-Boset kirk yasinda kral oldu ve Israil'de iki yil krallik yapti. Ancak Yahuda halki Davut'u destekledi.
\par 11 Davut Hevron'da Yahuda halkina yedi yil alti ay krallik yapti.
\par 12 Ner oglu Avner, Saul oglu Is-Boset'in adamlariyla birlikte Mahanayim'den Givon'a gitti.
\par 13 Seruya oglu Yoav'la Davut'un adamlari varip Givon Havuzu'nun yaninda onlari karsiladilar. Taraflardan biri havuzun bir yanina, öteki öbür yanina oturdu.
\par 14 Avner Yoav'a, "Ne olur gençler kalkip önümüzde dövüssünler" dedi. Yoav, "Olur, kalkip dövüssünler" diye karsilik verdi.
\par 15 Böylece Benyamin oymagindan Saul oglu Is-Boset'ten yana olanlardan on iki kisiyle Davut'un adamlarindan on iki kisi kalkip ileri atildi.
\par 16 Her biri karsitinin basindan tuttugu gibi kilicini bögrüne sapladi; birlikte yere serildiler. Bu yüzden Givon'daki o yere Helkat-Hassurim adi verildi.
\par 17 O gün savas çok çetin oldu. Davut'un adamlari Avner'le Israilliler'i yenilgiye ugrattilar.
\par 18 Seruya'nin üç oglu -Yoav, Avisay ve Asahel- de oradaydilar. Bir kir ceylani kadar hizli kosan Asahel
\par 19 saga sola sapmadan Avner'i kovaladi.
\par 20 Avner arkasina bakinca, "Asahel sen misin?" diye sordu. Asahel, "Evet, benim" diye karsilik verdi.
\par 21 Avner, "Saga ya da sola dön. Gençlerden birini yakala ve kendin için silahlarini al" dedi. Ama Asahel Avner'i kovalamaktan vazgeçmek istemedi.
\par 22 Avner Asahel'i bir daha uyardi: "Beni kovalamaktan vazgeç! Neden seni yere sereyim? Sonra kardesin Yoav'in yüzüne nasil bakarim?"
\par 23 Asahel pesini birakmayi reddedince Avner mizraginin arka ucuyla onu karnindan vurdu. Mizrak Asahel'in sirtindan çikti. Asahel orada düsüp öldü. Asahel'in düsüp öldügü yere varanlarin tümü orada durup beklediler.
\par 24 Ama Yoav'la Avisay Avner'i kovalamayi sürdürdüler. Günes batarken Givon kirsal bölgesine giden yolun üzerindeki Giah'a bakan Amma Tepesi'ne vardilar.
\par 25 Benyaminliler Avner'in çevresinde toplanarak bir birlik olusturdular. Bir tepenin basinda durup beklediler.
\par 26 Avner Yoav'a, "Kiliç sonsuza dek mi insanlari yok etsin?" diye seslendi, "Bu olayin aciyla sona erecegini anlamiyor musun? Kardeslerini kovalamaktan vazgeçmeleri için askerlere ne zaman buyruk vereceksin?"
\par 27 Yoav söyle karsilik verdi: "Yasayan Tanri'nin adiyla derim ki, seslenmeseydin askerler sabaha dek kardeslerini kovalamaktan vazgeçmeyecekti."
\par 28 Sonra Yoav boru çaldi. Herkes durdu. Bundan böyle Israil halkini ne kovaladilar, ne de onlarla savastilar.
\par 29 Avner'le adamlari bütün gece Arava Vadisi'nde yürüdüler. Seria Irmagi'ni geçerek Bitron yolundan Mahanayim'e vardilar.
\par 30 Yoav Avner'i kovalamaktan döndükten sonra orduyu topladi. Asahel'den baska, Davut'un adamlarindan on dokuz kisi eksikti.
\par 31 Oysa Davut'un adamlari Avner'i destekleyen Benyaminliler'i bozguna ugratip üç yüz altmis kisiyi öldürmüslerdi.
\par 32 Yoav'la adamlari Asahel'i götürüp Beytlehem'de babasinin mezarina gömdüler. Sonra bütün gece yürüyerek gün dogumunda Hevron'a vardilar.

\chapter{3}

\par 1 Saul'un soyuyla Davut'un soyu arasindaki savas uzun sürdü. Davut giderek güçlenirken, Saul'un soyu gitgide zayif düsüyordu.
\par 2 Davut'un Hevron'da dogan ogullari sunlardi: Ilk oglu Yizreelli Ahinoam'dan Amnon,
\par 3 ikincisi Karmelli Naval'in dulu Avigayil'den Kilav, üçüncüsü
\par 4 dördüncüsü Hagit'ten Adoniya, besincisi Avital'in oglu Sefatya,
\par 5 altincisi Davut'un esi Egla'dan Yitream. Davut'un bu ogullarinin hepsi Hevron'da dogdular.
\par 6 Saul'un soyuyla Davut'un soyu arasindaki savas sürerken, Avner Saul'un soyu arasinda güçleniyordu.
\par 7 Saul'un Aya kizi Rispa adinda bir cariyesi vardi. Bir gün Is-Boset Avner'e, "Neden babamin cariyesiyle yattin?" diye sordu.
\par 8 Is-Boset'in sorusuna çok öfkelenen Avner su karsiligi verdi: "Ben Yahuda tarafina geçen bir köpek basi miyim? Bugün bile baban Saul'un ailesine, kardeslerine, dostlarina bagliyim. Seni Davut'un eline teslim etmedim. Ama bugün bu kadin yüzünden beni suçluyorsun.
\par 9 RAB kralligi Saul'un soyundan alip Dan'dan Beer-Seva'ya kadar uzanan Israil ve Yahuda'da Davut'un kralligini kuracagina ant içti. Ben de bunu Davut için yapmazsam Tanri bana aynisini, hatta daha kötüsünü yapsin!"
\par 11 Is-Boset Avner'den korktugu için ona baska bir sey söyleyemedi.
\par 12 Avner kendi adina Davut'a ulaklar gönderip söyle dedi: "Ülke kimin ülkesi? Benimle bir antlasma yap; o zaman Israil'in tümünün sana baglanmasi için ben de senden yana olurum."
\par 13 Davut, "Iyi" diye yanitladi, "Seninle bir antlasma yaparim. Yalniz senden sunu istiyorum: Beni görmeye geldiginde Saul'un kizi Mikal'i da getir. Yoksa beni görmeyeceksin."
\par 14 Öte yandan Davut Saul oglu Is-Boset'e de ulaklar araciligiyla su haberi gönderdi: "Yüz Filistli'nin sünnet derisi karsiliginda nisanlandigim karim Mikal'i bana ver."
\par 15 Bunun üzerine Is-Boset, kadinin kocasi Layis oglu Paltiel'den alinip getirilmesi için adamlar gönderdi.
\par 16 Kocasi kadini aglaya aglaya Bahurim'e kadar izledi; sonra Avner ona, "Geri dön" deyince döndü.
\par 17 Avner Israil'in ileri gelenleriyle görüsüp onlara söyle demisti: "Siz bir süredir Davut'un kraliniz olmasini istiyorsunuz.
\par 18 Simdi bunu gerçeklestirin! Çünkü RAB, Davut hakkinda, 'Halkim Israil'i kulum Davut araciligiyla Filistliler'in ve bütün düsmanlarinin elinden kurtaracagim demisti."
\par 19 Avner Benyaminliler'le de görüstü, Israil'in ve bütün Benyamin halkinin uygun gördügü her seyi Davut'a bildirmek üzere Hevron'a gitti.
\par 20 Avner yirmi kisiyle birlikte Hevron'a, Davut'un yanina vardi. Davut Avner'le yanindakilere bir sölen verdi.
\par 21 Avner Davut'a, "Hemen gidip bütün Israil halkini efendim kralin yanina toplayayim" dedi, "Öyle ki, seninle bir antlasma yapsinlar. Sen de diledigin her yeri yönetebilesin." Bunun üzerine Davut Avner'i yoluna gönderdi. O da esenlikle gitti.
\par 22 Tam o sirada Davut'un adamlariyla Yoav, bir baskindan dönmüs, yanlarinda birçok yagmalanmis mal getirmislerdi. Ama Avner Hevron'da Davut'un yaninda degildi. Çünkü Davut onu göndermis, o da esenlikle gitmisti.
\par 23 Yoav'la yanindaki bütün askerler Hevron'a vardiginda, Ner oglu Avner'in krala geldigini, kralin onu gönderdigini, onun da esenlikle gittigini Yoav'a bildirdiler.
\par 24 Yoav krala gidip, "Ne yaptin?" dedi, "Baksana Avner ayagina kadar gelmis! Neden onu saliverdin? Çoktan gitmis!
\par 25 Ner oglu Avner'i tanirsin; seni kandirmak, nereye gidip geldigini, neler yaptigini ögrenmek için gelmistir."
\par 26 Davut'un yanindan çikan Yoav, Avner'in arkasindan ulaklar gönderdi. Ulaklar Avner'i Sira Sarnici'ndan geri getirdiler. Davut ise bundan habersizdi.
\par 27 Avner Hevron'a dönünce, Yoav onunla özel bir görüsme yapmak bahanesiyle, onu kent kapisina çekti. Kardesi Asahel'in kanini döktügü için, Avner'i orada karnindan vurup öldürdü.
\par 28 Davut bu haberi isitince söyle dedi: "RAB'bin önünde ben de, kralligim da Ner oglu Avner'in kanindan sonsuza dek suçsuzuz.
\par 29 Bu suçun sorumlusu Yoav'la babasinin bütün soyu olsun. Yoav'in soyundan irinli, deri hastaligina yakalanmis, koltuk degnegine dayanan, kiliçla öldürülen, açlik çeken kisiler hiç eksik olmasin!"
\par 30 Böylece Yoav'la kardesi Avisay, Givon'daki savasta kardesleri Asahel'i öldüren Avner'i öldürdüler.
\par 31 Sonra Davut Yoav'la yanindakilere su buyrugu verdi: "Giysilerinizi yirtip çula sarinin ve Avner'in ölüsü önünde yas tutun!" Kral Davut da cenazenin ardisira yürüdü.
\par 32 Avner'i Hevron'da gömdüler. Kral, Avner'in mezari basinda hiçkira hiçkira agladi. Oradaki herkes de agladi.
\par 33 Sonra kral, Avner için su agiti yakti: "Avner, bir budala gibi mi ölmeliydi?
\par 34 Ellerin bagli degildi, ayaklarina zincir vurulmamisti. Ama sen kötülerin önünde düsen biri gibi düstün!" Herkes Avner için yine agladi.
\par 35 Halk Davut'un yanina varip aksam olmadan bir seyler yemesi için üstelediyse de, Davut ant içerek söyle dedi: "Günes batmadan ekmek ya da baska herhangi bir sey tatmayacagim. Yoksa Tanri bana aynisini, hatta daha kötüsünü yapsin!"
\par 36 Herkes bunu benimsedi ve kralin yaptigi her seyden hosnut olduklari gibi, bundan da hosnut oldular.
\par 37 Ner oglu Avner'in öldürülmesinde kralin parmagi olmadigini o gün bütün Israil halki anladi.
\par 38 Kral adamlarina, "Bugün Israil'de bir önderin, büyük bir adamin öldügünü bilmiyor musunuz?" dedi,
\par 39 "Meshedilmis* bir kral oldugum halde bugün güçsüzüm. Seruya'nin ogullari benden daha zorlu. RAB kötülük edene yaptigi kötülüge göre karsilik versin!"

\chapter{4}

\par 1 Avner'in Hevron'da öldürüldügünü duyan Saul oglu Is-Boset korkuya kapildi. Bütün Israil halki da dehset içindeydi.
\par 2 Saul oglu Is-Boset'in iki akinci önderi vardi; bunlar Benyamin oymagindan Beerotlu Rimmon'un ogullariydi. Birinin adi Baana, öbürününki Rekav'di. -Beerot Benyamin oymagindan sayilirdi.
\par 3 Beerotlular Gittayim'e kaçmislardi. Yabanci olan bu halk bugün de orada gurbette yasiyor.-
\par 4 Saul oglu Yonatan'in Mefiboset adinda bir oglu vardi; iki ayagi da topaldi. Saul'la Yonatan'in ölüm haberi Yizreel'den ulastiginda, Mefiboset bes yasindaydi. Dadisi onu alip kaçmisti. Ne var ki, aceleyle kaçmaya çalisirken çocuk düsüp sakatlanmisti.
\par 5 Beerotlu Rimmon'un ogullari Rekav'la Baana yola koyuldular. Ögle sicaginda Is-Boset'in evine vardiklarinda Is-Boset uzanmis dinlenmekteydi.
\par 6 Bugday alacakmis gibi yaparak eve girdiler. O sirada Is-Boset yatak odasinda yataginda uzaniyordu. Adamlar Is-Boset'in karnini desip öldürdüler. Basini gövdesinden ayirip yanlarina aldilar. Rekav'la kardesi Baana kaçip bütün gece Arava yolundan ilerlediler.
\par 8 Is-Boset'in basini Hevron'da Kral Davut'a getirip, "Iste seni öldürmek isteyen düsmanin Saul'un oglu Is-Boset'in basi!" dediler, "RAB bugün Saul'dan ve onun soyundan efendimiz kralin öcünü aldi."
\par 9 Ama Davut Beerotlu Rimmon'un ogullari Rekav'la kardesi Baana'ya söyle karsilik verdi: "Beni her türlü sikintidan kurtaran yasayan RAB adiyla derim ki, iyi haber getirdigini sanarak bana Saul'un öldügünü bildiren adami yakalayip Ziklak'ta yasamina son verdim. Getirdigi iyi haber için verdigim ödül buydu!
\par 11 Dogru birini evinde, yataginda öldüren kötü kisilerin ölümü çok daha kesindir! Simdi sizi yeryüzünden yok ederek onun öcünü sizden almayacak miyim?"
\par 12 Sonra adamlarina buyruk verdi. Iki kardesi öldürüp ellerini, ayaklarini kestiler ve Hevron'daki havuzun yanina astilar. Is-Boset'in basini ise götürüp Hevron'da Avner'in mezarina gömdüler.

\chapter{5}

\par 1 Israil'in bütün oymaklari Hevron'da bulunan Davut'a gelip söyle dediler: "Biz senin etin, kemiginiz.
\par 2 Geçmiste Saul kralimizken, savasta Israil'e komuta eden sendin. RAB sana, 'Halkim Israil'i sen güdecek, onlara sen önder olacaksin diye söz verdi."
\par 3 Israil'in bütün ileri gelenleri Hevron'a, Kral Davut'un yanina gelince, kral RAB'bin önünde orada onlarla bir antlasma yapti. Onlar da Davut'u Israil Krali olarak meshettiler*.
\par 4 Davut otuz yasinda kral oldu ve kirk yil krallik yapti.
\par 5 Hevron'da yedi yil alti ay Yahuda'ya, Yerusalim'de otuz üç yil bütün Israil'e ve Yahuda'ya krallik yapti.
\par 6 Kral Davut'la adamlari Yerusalim'de yasayan Yevuslular'a saldirmak için yola çiktilar. Yevuslular Davut'a, "Sen buraya giremezsin, körlerle topallar bile seni geri püskürtebilir" dediler. "Davut buraya giremez" diye düsünüyorlardi.
\par 7 Ne var ki, Davut Siyon Kalesi'ni ele geçirdi. Daha sonra bu kaleye "Davut Kenti" adi verildi.
\par 8 Davut o gün adamlarina söyle demisti: "Yevuslular'i kim yenilgiye ugratirsa Davut'un nefret ettigi su 'Topallarla körlere su kanalindan ulasmali!" "Körlerle topallar saraya giremeyecek" denmesinin nedeni iste budur.
\par 9 Bundan sonra Davut "Davut Kenti" adini verdigi kalede oturmaya basladi. Çevredeki bölgeyi, Millo'dan* içeriye dogru uzanan bölümü insa etti.
\par 10 Davut giderek güçleniyordu. Çünkü Her Seye Egemen Tanri RAB onunlaydi.
\par 11 Sur Krali Hiram Davut'a ulaklar, sedir kütükleri, marangozlar ve tasçilar gönderdi. Bu adamlar Davut için bir saray yaptilar.
\par 12 Böylece Davut RAB'bin kendisini Israil Krali atadigini ve halki Israil'in hatiri için kralligini yücelttigini anladi.
\par 13 Davut Hevron'dan ayrildiktan sonra Yerusalim'de kendine daha birçok cariye ve kari aldi. Davut'un erkek ve kiz çocuklari oldu.
\par 14 Davut'un Yerusalim'de dogan çocuklarinin adlari sunlardi: Sammua, Sovav, Natan, Süleyman,
\par 15 Yivhar, Elisua, Nefek, Yafia,
\par 16 Elisama, Elyada ve Elifelet.
\par 17 Filistliler Davut'un Israil Krali olarak meshedildigini* duyunca, bütün Filist ordusu onu aramak için yola çikti. Bunu duyan Davut kaleye sigindi.
\par 18 Filistliler gelip Refaim Vadisi'ne yayilmislardi.
\par 19 Davut RAB'be danisti: "Filistliler'e saldirayim mi? Onlari elime teslim edecek misin?" RAB Davut'a, "Saldir" dedi, "Onlari kesinlikle eline teslim decegim."
\par 20 Bunun üzerine Davut Baal-Perasim'e gidip orada Filistliler'i bozguna ugratti. Sonra, "Her seyi yarip geçen sular gibi, RAB düsmanlarimi önümden yarip geçti" dedi. Bundan ötürü oraya Baal-Perasim adi verildi.
\par 21 Davut'la adamlari, Filistliler'in orada biraktigi putlari alip götürdüler.
\par 22 Filistliler bir kez daha gelip Refaim Vadisi'ne yayildilar.
\par 23 Davut RAB'be danisti. RAB söyle karsilik verdi: "Buradan saldirma! Onlari arkadan çevirip pelesenk agaçlarinin önünden saldir.
\par 24 Pelesenk agaçlarinin tepesinden yürüyüs sesi duyar duymaz, acele et. Çünkü ben Filist ordusunu bozguna ugratmak için önünsira gitmisim demektir."
\par 25 Davut RAB'bin kendisine buyurdugu gibi yapti ve Filistliler'i Geva'dan Gezer'e kadar bozguna ugratti.

\chapter{6}

\par 1 Davut Israil'deki bütün seçme adamlari topladi. Sayilari otuz bin kisiydi.
\par 2 Böylece Davut'la ordusu, sandigin üzerindeki Keruvlar* arasinda taht kuran Her Seye Egemen RAB'bin adiyla anilan Tanri'nin Sandigi'ni getirmek için Baale-Yahuda'ya gittiler.
\par 3 Tanri'nin Sandigi'ni Avinadav'in tepedeki evinden alip yeni bir arabaya koydular. Tanri'nin Sandigi'ni tasiyan yeni arabayi Avinadav'in ogullari Uzza'yla Ahyo sürüyordu. Ahyo sandigin önünden yürüyordu.
\par 5 Bu arada Davut'la bütün Israil halki da RAB'bin önünde lir, çenk, tef, çingirak ve ziller esliginde ezgiler okuyarak var güçleriyle bu olayi kutluyorlardi.
\par 6 Nakon'un harman yerine vardiklarinda öküzler tökezledi. Bu nedenle Uzza elini uzatip Tanri'nin Sandigi'ni tuttu.
\par 7 RAB Tanri saygisizca davranan Uzza'ya öfkelenerek onu orada yere çaldi. Uzza Tanri'nin Sandigi'nin yaninda öldü.
\par 8 Davut, RAB'bin Uzza'yi cezalandirmasina öfkelendi. O günden bu yana oraya Peres-Uzza denilir.
\par 9 Davut o gün RAB'den korkarak, "RAB'bin Sandigi nasil olur da bana gelir?" diye düsündü.
\par 10 RAB'bin Sandigi'ni Davut Kenti'ne götürmek istemedi. Bunun yerine sandigi Gatli Ovet-Edom'un evine götürdü.
\par 11 RAB'bin Sandigi Gatli Ovet-Edom'un evinde üç ay kaldi. RAB Ovet-Edom'u ve bütün ailesini kutsadi.
\par 12 "Tanri'nin Sandigi'ndan ötürü RAB Ovet-Edom'un ailesini ve ona ait her seyi kutsadi" diye Kral Davut'a bildirildi. Böylece Davut gidip Tanri'nin Sandigi'ni Ovet-Edom'un evinden Davut Kenti'ne sevinçle getirdi.
\par 13 RAB'bin Sandigi'ni tasiyanlar alti adim atinca, Davut bir bogayla besili bir dana kurban etti.
\par 14 Keten efod* kusanmis Davut, RAB'bin önünde var gücüyle oynuyordu.
\par 15 Davut'la bütün Israil halki, sevinç naralari ve boru sesi esliginde RAB'bin Sandigi'ni getiriyorlardi.
\par 16 RAB'bin Sandigi Davut Kenti'ne varinca, Saul'un kizi Mikal pencereden bakti. RAB'bin önünde oynayip ziplayan Kral Davut'u görünce, onu küçümsedi.
\par 17 RAB'bin Sandigi'ni getirip Davut'un bu amaçla kurdugu çadirin içindeki yerine koydular. Davut RAB'be yakmalik sunular* ve esenlik sunulari* sundu.
\par 18 Yakmalik sunulari ve esenlik sunularini sunmayi bitirince, Her Seye Egemen RAB'bin adiyla halki kutsadi.
\par 19 Ardindan kadin erkek herkese, bütün Israil topluluguna birer somun ekmekle birer hurma ve üzüm pestili dagitti. Sonra herkes evine döndü.
\par 20 Davut ailesini kutsamak için eve döndügünde, Saul'un kizi Mikal onu karsilamaya çikti. Davut'a söyle dedi: "Israil Krali bugün ne güzel bir ün kazandirdi kendine! Degersiz biri gibi, kullarinin cariyeleri önünde soyundun."
\par 21 Davut, "Baban ve bütün soyu yerine beni seçen ve halki Israil'e önder atayan RAB'bin önünde oynadim!" diye karsilik verdi, "Evet, RAB'bin önünde oynayacagim.
\par 22 Üstelik kendimi bundan daha da küçük düsürecegim, hiçe sayacagim. Ama sözünü ettigin o cariyeler beni onurlandiracaklar."
\par 23 Saul'un kizi Mikal'in ölene dek çocugu olmadi.

\chapter{7}

\par 1 Kral sarayina yerlesmisti. RAB de onu çevresindeki bütün düsmanlarindan koruyarak rahata kavusturdu.
\par 2 O sirada kral, Peygamber Natan'a, "Bak, ben sedir agacindan yapilmis bir sarayda oturuyorum. Oysa Tanri'nin Sandigi bir çadirda duruyor!" dedi.
\par 3 Natan, "Git, tasarladigin her seyi yap, çünkü RAB seninledir" diye karsilik verdi.
\par 4 O gece RAB Natan'a söyle seslendi:
\par 5 "Git, kulum Davut'a söyle de: 'RAB diyor ki, oturmam için bana sen mi tapinak yapacaksin?
\par 6 Israil halkini Misir'dan çikardigim günden bu yana konutta*fe* oturmadim. Bir çadirda orada burada konaklayarak dolasiyordum.
\par 7 Israilliler'le birlikte dolastigim yerlerin herhangi birinde, halkim Israil'i gütmesini buyurdugum Israil önderlerinden birine, neden bana sedir agacindan bir konut yapmadiniz diye hiç sordum mu?
\par 8 "Simdi kulum Davut'a söyle diyeceksin: 'Her Seye Egemen RAB diyor ki, halkim Israil'e önder olasin diye seni otlaklardan ve koyun gütmekten aldim.
\par 9 Her nereye gittiysen seninleydim. Önünden bütün düsmanlarini yok ettim. Adini dünyadaki büyük adamlarin adi gibi büyük kilacagim.
\par 10 Halkim Israil için bir yurt saglayip onlari oraya yerlestirecegim. Bundan böyle kendi yurtlarinda otursunlar, bir daha rahatsiz edilmesinler. Kötü kisiler de halkim Israil'e hakimler atadigim günden bu yana yaptiklari gibi, bir daha onlara baski yapmasinlar. Seni bütün düsmanlarindan kurtarip rahata kavusturacagim. "'RAB senin için bir soy*fe* yetistirecegini belirtiyor:
\par 12 Sen ölüp atalarina kavusunca, senden sonra soyundan birini ortaya çikarip kralligini pekistirecegim.
\par 13 Adima bir tapinak kuracak olan odur. Ben de onun kralliginin tahtini sonsuza dek sürdürecegim.
\par 14 Ben ona baba olacagim, o da bana ogul olacak. Kötülük yapinca, onu insanlarin degnegiyle, insanlarin vuruslariyla yola getirecegim.
\par 15 Ama senin önünden kaldirdigim Saul'dan esirgedigim sevgiyi hiçbir zaman esirgemeyecegim.
\par 16 Soyun ve kralligin sonsuza dek önümde duracak; tahtin sonsuza dek sürecektir."
\par 17 Böylece Natan bütün bu sözleri ve görümleri Davut'a aktardi.
\par 18 Bunun üzerine Kral Davut gelip RAB'bin önünde oturdu ve söyle dedi: "Ey Egemen RAB, ben kimim, ailem nedir ki, beni bu duruma getirdin?
\par 19 Ey Egemen RAB, sanki bu yetmezmis gibi, kulunun soyunun gelecegi hakkinda da söz verdin. Ey Egemen RAB, insanlarla hep böyle mi ilgilenirsin?
\par 20 Ben sana baska ne diyebilirim ki! Çünkü, ey Egemen RAB, kulunu taniyorsun.
\par 21 Sözünün hatiri için ve istegin uyarinca bu büyüklügü gösterdin ve kuluna bildirdin.
\par 22 "Yücesin, ey Egemen RAB! Bir benzerin yok, senden baska Tanri da yok! Bunu kendi kulaklarimizla duyduk.
\par 23 Halkin Israil'e benzer tek bir ulus yok dünyada. Kendi halkin olsun diye onlari kurtarmaya gittin. Çünkü onlar için de, ülken için de büyük ve görkemli isler yapmakla ün saldin. Misir'dan kendin için kurtardigin halkin önünden uluslari ve tanrilarini kovdun.
\par 24 Halkin Israil'i sonsuza dek kendi halkin olarak benimsedin ve sen de, ya RAB, onlarin Tanrisi oldun.
\par 25 "Simdi, ya RAB Tanri, kuluna ve onun soyuna iliskin verdigin sözü sonsuza dek tut, sözünü yerine getir.
\par 26 Öyle ki, insanlar, 'Her Seye Egemen RAB Israil'in Tanrisi'dir! diyerek adini sonsuza dek yüceltsinler ve kulun Davut'un soyu da önünde sürsün.
\par 27 "Ey Her Seye Egemen RAB, Israil'in Tanrisi! 'Senin için bir soy çikaracagim diye kuluna açikladin. Bundan dolayi kulun sana bu duayi etme yürekliligini buldu.
\par 28 Ey Egemen RAB, sen Tanri'sin! Sözlerin gerçektir ve kuluna bu iyi sözü verdin.
\par 29 Simdi önünde sonsuza dek sürmesi için kulunun soyunu kutsamani diliyorum. Çünkü, ey Egemen RAB, sen böyle söz verdin ve kulunun soyu kutsamanla sonsuza dek kutlu kilinacak."

\chapter{8}

\par 1 Bir süre sonra Davut Filistliler'i yenip boyundurugu altina aldi ve Meteg-Amma'yi Filistliler'in yönetiminden çikardi.
\par 2 Moavlilar'i da bozguna ugratti. Onlari yere yatirip iple ölçtü. Ölçtügü iki sirayi öldürdü, bir bütün sirayi sag birakti. Moavlilar Davut'un haraç ödeyen köleleri oldular.
\par 3 Davut Firat'a kadar kralligini yeniden kurmaya giden Sova Krali Rehov oglu Hadadezer'i de yendi.
\par 4 Bin yedi yüz atlisiyla yirmi bin yaya askerini ele geçirdi. Yüz savas arabasi için gereken atlarin disindaki bütün atlari da sakatladi.
\par 5 Sova Krali Hadadezer'e yardima gelen Sam Aramlilari'ndan yirmi iki bin kisiyi öldürdü.
\par 6 Sonra Sam Aramlilari'nin ülkesine askeri birlikler yerlestirdi. Onlar da Davut'un haraç ödeyen köleleri oldular. RAB Davut'u gittigi her yerde zafere ulastirdi.
\par 7 Davut Hadadezer'in komutanlarinin tasidigi altin kalkanlari alip Yerusalim'e götürdü.
\par 8 Ayrica Hadadezer'in yönetimindeki Betah ve Berotay kentlerinden bol miktarda tunç* aldi.
\par 9 Hama Krali Toi, Davut'un Hadadezer'in bütün ordusunu bozguna ugrattigini duydu.
\par 10 Toi Kral Davut'u selamlamak ve Hadadezer'le savasip yendigi için onu kutlamak üzere oglu Yoram'i ona gönderdi. Çünkü Toi Hadadezer'le sürekli savasmisti. Yoram Davut'a altin, gümüs, tunç armaganlar getirdi.
\par 11 Kral Davut bu armaganlari yendigi bütün uluslardan -Aram, Moav, Ammonlular, Filistliler ve Amalekliler'den- ele geçirdigi altin ve gümüsle birlikte RAB'be adadi. Bunun yanisira Sova Krali Rehov oglu Hadadezer'den yagmalanan altinla gümüsü de RAB'be adadi.
\par 13 Davut Tuz Vadisi'nde on sekiz bin Edomlu öldürüp dönünce üne kavustu.
\par 14 Edom'un her yanina askeri birlikler yerlestirdi. Edomlular'in tümü Davut'un köleleri oldular. RAB Davut'u gittigi her yerde zafere ulastirdi.
\par 15 Bütün Israil'de krallik yapan Davut halkina dogruluk ve adalet sagladi.
\par 16 Seruya oglu Yoav ordu komutani, Ahilut oglu Yehosafat devlet tarihçisiydi.
\par 17 Ahituv oglu Sadok'la Aviyatar oglu Ahimelek kâhin*, Seraya yazmandi.
\par 18 Yehoyada oglu Benaya Keretliler'le Peletliler'in* komutaniydi. Davut'un ogullariysa kâhindi.

\chapter{9}

\par 1 Davut, "Saul'un ailesinden daha sag kalan, Yonatan'in hatiri için iyilik edebilecegim kimse var mi?" diye sordu.
\par 2 Saul'un ailesinin Siva adinda bir hizmetkâri vardi. Onu Davut'un yanina çagirdilar. Kral, "Siva sen misin?" diye sordu. Siva, "Evet, ben kulunum" diye yanitladi.
\par 3 Kral, "Saul'un ailesinden sag kalan kimse yok mu?" diye sordu, "Tanri'nin iyiligini ona göstereyim." Siva, "Yonatan'in iki ayagi sakat bir oglu var" diye yanitladi.
\par 4 Kral, "Nerede o?" diye sordu. Siva, "Ammiel oglu Makir'in Lo-Devar'daki evinde" diye karsilik verdi.
\par 5 Böylece Kral Davut, Lo-Devar'dan Ammiel oglu Makir'in evinden onu yanina getirtti.
\par 6 Saul oglu Yonatan oglu Mefiboset, Davut'un yanina gelince, onun önünde yere kapandi. Davut, "Mefiboset!" diye seslendi. Mefiboset, "Evet, ben kulunum" diye yanitladi.
\par 7 Davut ona, "Korkma!" dedi, "Çünkü baban Yonatan'in hatiri için, sana kesinlikle iyilik edecegim. Atan Saul'un bütün topragini sana geri verecegim. Ve sen her zaman soframda yemek yiyeceksin."
\par 8 Mefiboset yere kapanip söyle dedi: "Kulun ne ki, benim gibi ölmüs bir köpekle ilgileniyorsun?"
\par 9 Kral Davut, Saul'un hizmetkâri Siva'yi çagirtip, "Önceden efendin Saul ile ailesine ait her seyi torunu Mefiboset'e verdim" dedi,
\par 10 "Sen, ogullarin ve kölelerin onun için topragi isleyip ürünü getireceksiniz. Öyle ki, efendinizin torununun yiyecek gereksinimi saglansin. Efendinin torunu Mefiboset her zaman benim soframda yemek yiyecektir." Siva'nin on bes oglu ve yirmi kölesi vardi.
\par 11 Siva, "Efendim kralin buyurdugu her seyi yapacagim" dedi. Mefiboset kralin çocuklarindan biri gibi onun sofrasinda yemek yedi.
\par 12 Mefiboset'in Mika adinda küçük bir oglu vardi. Siva'ya bagli herkes Mefiboset'e hizmet ediyordu.
\par 13 Iki ayagi sakat Mefiboset hep kralin sofrasinda yemek yediginden Yerusalim'de oturuyordu.

\chapter{10}

\par 1 Bir süre sonra Ammon Krali öldü, yerine oglu Hanun kral oldu.
\par 2 Davut, "Babasi bana iyilik ettigi gibi ben de Nahas oglu Hanun'a iyilik edecegim" diye düsünerek, babasinin ölümünden dolayi bas sagligi dilemek için Hanun'a görevliler gönderdi. Davut'un görevlileri Ammonlular'in ülkesine varinca,
\par 3 Ammon önderleri, efendileri Hanun'a söyle dediler: "Davut sana bu adamlari gönderdi diye babana saygi duydugunu mu saniyorsun? Davut, kenti arastirmak, casusluk etmek, yikmak için adamlarini sana gönderdi."
\par 4 Bunun üzerine Hanun Davut'un görevlilerini yakalatti. Sakallarinin yarisini tiras edip giysilerinin kalçayi kapatan kesimini ortadan kesti ve onlari öylece gönderdi.
\par 5 Davut bunu duyunca, onlari karsilamak üzere adamlar gönderdi. Çünkü görevliler çok utaniyorlardi. Kral, "Sakaliniz uzayincaya dek Eriha'da kalin, sonra dönün" diye buyruk verdi.
\par 6 Ammonlular, Davut'un nefretini kazandiklarini anlayinca, haber gönderip Beytrehov ve Sova'dan yirmi bin Aramli yaya asker, Maaka Krali'yla bin adamini ve Tov halkindan on iki bin adami kiraladilar.
\par 7 Davut bunu duyunca, Yoav'i ve güçlü adamlardan olusan bütün ordusunu onlara karsi gönderdi.
\par 8 Ammonlular çikip kent kapisinda savas düzeni aldilar. Aramli Sova'yla Rehov, Tov halki ve Maaka'nin adamlari da kirda savas düzenine girdiler.
\par 9 Önde, arkada düsman birliklerini gören Yoav, Israil'in iyi askerlerinden bazilarini seçerek Aramlilar'a karsi yerlestirdi.
\par 10 Geri kalan birlikleri de kardesi Avisay'in komutasina vererek Ammonlular'a karsi yerlestirdi.
\par 11 Yoav, "Aramlilar benden güçlü çikarsa, yardimima gelirsin" dedi, "Ama Ammonlular senden güçlü çikarsa, ben sana yardima gelirim.
\par 12 Güçlü ol! Halkimizin ve Tanrimiz'in kentleri ugruna yürekli olalim! RAB gözünde iyi olani yapsin."
\par 13 Yoav'la yanindakiler Aramlilar'a karsi savasmak için ileri atilinca, Aramlilar onlardan kaçti.
\par 14 Onlarin kaçistigini gören Ammonlular da Avisay'dan kaçarak kente girdiler. Bunun üzerine Yoav Ammonlular'la savasmaktan vazgeçerek Yerusalim'e gitti.
\par 15 Israilliler'in önünde bozguna ugradiklarini gören Aramlilar bir araya geldiler.
\par 16 Hadadezer, haber gönderip Firat Irmagi'nin karsi yakasindaki Aramlilar'i çagirtti. Aramlilar Hadadezer'in ordu komutani Sovak'in komutasinda Helam'a gittiler.
\par 17 Davut bunu duyunca, bütün Israil ordusunu topladi. Seria Irmagi'ni geçerek Helam'a vardilar. Aramlilar Davut'a karsi düzen alarak onunla savastilar.
\par 18 Ne var ki, Aramlilar Israilliler'in önünden kaçtilar. Davut onlardan yedi yüz savas arabasi sürücüsü ile kirk bin atli asker öldürdü. Hadadezer'in ordu komutani Sovak'i da vurdu. Sovak savas alaninda öldü.
\par 19 Hadadezer'in buyrugundaki krallarin hepsi bozguna ugradiklarini görünce, Israilliler'le baris yaparak onlara boyun egdiler. Aramlilar bundan böyle Ammonlular'a yardim etmekten kaçindilar.

\chapter{11}

\par 1 Ilkbaharda, krallarin savasa gittigi dönemde, Davut kendi subaylariyla birlikte Yoav'i ve bütün Israil ordusunu savasa gönderdi. Onlar Ammonlular'i yenilgiye ugratip Rabba Kenti'ni kusatirken, Davut Yerusalim'de kaliyordu.
\par 2 Bir aksamüstü Davut yatagindan kalkti, sarayin damina çikip gezinmeye basladi. Damdan yikanan bir kadin gördü. Kadin çok güzeldi.
\par 3 Davut onun kim oldugunu ögrenmek için birini gönderdi. Adam, "Kadin Eliam'in kizi Hititli* Uriya'nin karisi Bat-Seva'dir" dedi.
\par 4 Davut kadini getirmeleri için ulaklar gönderdi. Kadin Davut'un yanina geldi. Davut aybasi kirliliginden yeni arinmis olan kadinla yatti. Sonra kadin evine döndü.
\par 5 Gebe kalan kadin Davut'a, "Gebe kaldim" diye haber gönderdi.
\par 6 Bunun üzerine Davut Hititli Uriya'yi kendisine göndermesi için Yoav'a haber yolladi. Yoav da Uriya'yi Davut'a gönderdi.
\par 7 Uriya yanina varinca, Davut Yoav'in, ordunun ve savasin durumunu sordu.
\par 8 Sonra Uriya'ya, "Evine git, rahatina bak" dedi. Uriya saraydan çikinca, kral ardindan bir armagan gönderdi.
\par 9 Ne var ki, Uriya evine gitmedi, efendisinin bütün adamlariyla birlikte sarayin kapisinda uyudu.
\par 10 Davut Uriya'nin evine gitmedigini ögrenince, ona, "Yolculuktan geldin. Neden evine gitmedin?" diye sordu.
\par 11 Uriya, "Sandik da, Israilliler'le Yahudalilar da çardaklarda kaliyor" diye karsilik verdi, "Komutanim Yoav'la efendimin adamlari kirlarda konakliyor. Bu durumda nasil olur da ben yiyip içmek, karimla yatmak için evime giderim? Yasamin hakki için, böyle bir seyi kesinlikle yapmayacagim."
\par 12 Bunun üzerine Davut, "Bugün de burada kal, yarin seni gönderecegim" dedi. Uriya o gün de, ertesi gün de Yerusalim'de kaldi.
\par 13 Davut Uriya'yi çagirdi. Onu sarhos edene dek yedirip içirdi. Aksam olunca Uriya efendisinin adamlariyla birlikte uyumak üzere yattigi yere gitti. Yine evine gitmedi.
\par 14 Sabahleyin Davut Yoav'a bir mektup yazip Uriya araciligiyla gönderdi.
\par 15 Mektupta söyle yazdi: "Uriya'yi savasin en siddetli oldugu cepheye yerlestir ve yanindan çekil ki, vurulup ölsün."
\par 16 Böylece Yoav kenti kusatirken Uriya'yi yigit adamlarin bulundugunu bildigi yere yerlestirdi.
\par 17 Kent halki çikip Yoav'in askerleriyle savasti. Davut'un askerlerinden ölenler oldu. Hititli Uriya da ölenler arasindaydi.
\par 18 Yoav savasla ilgili ayrintili haberleri Davut'a iletmek üzere bir ulak gönderdi.
\par 19 Ulagi söyle uyardi: "Sen savasla ilgili ayrintili haberleri krala iletmeyi bitirdikten sonra,
\par 20 kral öfkelenip sana sunu sorabilir: 'Onlarla savasmak için kente neden o kadar çok yaklastiniz? Surdan ok atacaklarini bilmiyor muydunuz?
\par 21 Yerubbeset*ff* oglu Avimelek'i kim öldürdü? Teves'te surun üstünden bir kadin üzerine bir degirmen üst tasini atip onu öldürmedi mi? Öyleyse niçin sura o kadar çok yaklastiniz? O zaman, 'Kulun Hititli Uriya da öldü dersin."
\par 22 Ulak yola koyuldu. Davut'un yanina varinca, Yoav'in kendisine söylediklerinin tümünü ona iletti.
\par 23 "Adamlar bizden üstün çiktilar" dedi, "Kentten çikip bizimle kirda savastilar. Ama onlari kent kapisina kadar geri püskürttük.
\par 24 Bunun üzerine okçular adamlarina surdan ok attilar. Kralin adamlarindan bazilari öldü; kulun Hititli Uriya da öldü."
\par 25 Davut ulaga söyle dedi: "Yoav'a de ki, 'Bu olay seni üzmesin! Savasta kimin ölecegi belli olmaz. Kente karsi saldirinizi güçlendirin ve kenti yerle bir edin! Bu sözlerle onu yüreklendir."
\par 26 Uriya'nin karisi, kocasinin öldügünü duyunca, onun için yas tuttu.
\par 27 Yas süresi geçince, Davut onu sarayina getirtti. Kadin Davut'un karisi oldu ve ona bir ogul dogurdu. Ancak, Davut'un bu yaptigi RAB'bin hosuna gitmedi.

\chapter{12}

\par 1 RAB Natan'i Davut'a gönderdi. Natan Davut'un yanina gelince ona, "Bir kentte biri zengin, öbürü yoksul iki adam vardi" dedi,
\par 2 "Zengin adamin birçok koyunu, sigiri vardi.
\par 3 Ama yoksul adamin satin alip besledigi küçük bir disi kuzudan baska bir hayvani yoktu. Kuzu adamin yaninda, çocuklariyla birlikte büyüdü. Adamin yemeginden yer, tasindan içer, koynunda uyurdu. Yoksulun kizi gibiydi.
\par 4 Derken, zengin adama bir yolcu ugradi. Adam gelen konuga yemek hazirlamak için kendi koyunlarindan, sigirlarindan birini almaya kiyamadigindan yoksulun kuzusunu alip yolcuya yemek hazirladi."
\par 5 Zengin adama çok öfkelenen Davut Natan'a, "Yasayan RAB'bin adiyla derim ki, bunu yapan ölümü hak etmistir!" dedi,
\par 6 "Bunu yaptigi ve acimadigi için kuzuya karsilik dört katini ödemeli."
\par 7 Bunun üzerine Natan Davut'a, "O adam sensin!" dedi, "Israil'in Tanrisi RAB diyor ki, 'Ben seni Israil'e kral olarak meshettim* ve Saul'un elinden kurtardim.
\par 8 Sana efendinin evini verdim, karilarini da koynuna verdim. Israil ve Yahuda halkini da sana verdim. Bu az gelseydi, sana daha neler neler verirdim!
\par 9 Öyleyse neden RAB'bin gözünde kötü olani yaparak, onun sözünü küçümsedin? Hititli* Uriya'yi kiliçla öldürdün, Ammonlular'in kiliciyla canina kiydin. Karisini da kendine es olarak aldin.
\par 10 Bundan böyle, kiliç senin soyundan sonsuza dek eksik olmayacak. Çünkü beni küçümsedin ve Hititli Uriya'nin karisini kendine es olarak aldin.
\par 11 "RAB söyle diyor: 'Sana kendi soyundan kötülük getirecegim. Senin gözünün önünde karilarini alip bir yakinina verecegim; güpegündüz karilarinin koynuna girecek.
\par 12 Evet, sen o isi gizlice yaptin, ama ben bunu bütün Israil halkinin gözü önünde güpegündüz yapacagim!"
\par 13 Davut, "RAB'be karsi günah isledim" dedi. Natan, "RAB günahini bagisladi, ölmeyeceksin" diye karsilik verdi,
\par 14 "Ama sen bunu yapmakla, RAB'bin düsmanlarinin O'nu küçümsemesine neden oldun. Bu yüzden dogan çocugun kesinlikle ölecek."
\par 15 Bundan sonra Natan evine döndü. RAB Uriya'nin karisinin Davut'tan dogan çocugunun hastalanmasina neden oldu.
\par 16 Davut çocuk için Tanri'ya yalvarip oruç* tuttu; evine gidip gecelerini yerde yatarak geçirdi.
\par 17 Sarayin ileri gelenleri onu yerden kaldirmaya geldiler. Ama Davut kalkmak istemedi, onlarla yemek de yemedi.
\par 18 Yedinci gün çocuk öldü. Davut'un görevlileri çocugun öldügünü Davut'a bildirmekten çekindiler. Çünkü, "Çocuk daha yasarken onunla konustuk ama bizi dinlemedi" diyorlardi, "Simdi çocugun öldügünü ona nasil söyleriz? Kendisine zarar verebilir!"
\par 19 Davut görevlilerinin fisildastigini görünce, çocugun öldügünü anladi. Onlara, "Çocuk öldü mü?" diye sordu. "Evet, öldü" dediler.
\par 20 Bunun üzerine Davut yerden kalkti. Yikandi, güzel kokular sürünüp giysilerini degistirdi. RAB'bin Tapinagi'na gidip tapindi. Sonra evine döndü ve yemek istedi. Önüne konan yemegi yedi.
\par 21 Hizmetkârlari, "Neden böyle davraniyorsun?" diye sordular, "Çocuk yasarken oruç tuttun, agladin; ama ölünce kalkip yemek yemeye basladin."
\par 22 Davut söyle yanitladi: "Çocuk yasarken oruç tutup agladim. Çünkü, 'Kim bilir, RAB bana lütfeder de çocuk yasar diye düsünüyordum.
\par 23 Ama çocuk öldü. Artik neden oruç tutayim? Onu geri getirebilir miyim ki? Ben onun yanina gidecegim, ama o bana geri dönmeyecek."
\par 24 Davut karisi Bat-Seva'yi avuttu. Yanina girip onunla yatti. Bat-Seva bir ogul dogurdu. Çocugun adini Süleyman koydu. Çocugu seven RAB Peygamber Natan araciligiyla haber gönderdi ve hatiri için çocugun adini Yedidyah koydu.
\par 26 Bu sirada Ammonlular'in Rabba Kenti'ne karsi savasi sürdüren Yoav, saray semtini ele geçirdi.
\par 27 Sonra Davut'a ulaklar göndererek, "Rabba Kenti'ne karsi savasip su kaynaklarini ele geçirdim" dedi,
\par 28 "Simdi sen ordunun geri kalanlarini topla, kenti kusatip ele geçir; öyle ki, kenti ben ele geçirmeyeyim ve kent adimla anilmasin."
\par 29 Davut bütün askerlerini toplayip Rabba Kenti'ne gitti, kente karsi savasip ele geçirdi.
\par 30 Ammon Krali'nin basindaki taci aldi. Degerli taslarla süslü, agirligi bir talant altini bulan taci Davut'un basina koydular. Davut kentten çok miktarda mal yagmalayip götürdü.
\par 31 Orada yasayan halki disari çikarip testereyle, demir kazma ve baltayla yapilan islerde, tugla yapiminda çalistirdi. Davut bunu bütün Ammon kentlerinde uyguladi. Sonra bütün ordusuyla birlikte Yerusalim'e döndü.

\chapter{13}

\par 1 Davut'un oglu Avsalom'un Tamar adinda güzel bir kizkardesi vardi. Davut'un baska bir oglu, Amnon Tamar'a gönül verdi.
\par 2 Amnon üvey kizkardesi Tamar yüzünden yataga düsecek kadar üzüntüye kapildi. Çünkü Tamar erden bir kizdi ve Amnon ona bir sey yapmayi olanaksiz görüyordu.
\par 3 Amnon'un Davut'un kardesi Sima'nin oglu Yonadav adinda çok akilli bir arkadasi vardi.
\par 4 Yonadav Amnon'a, "Ey kral oglu, neden böyle her sabah üzgün görünüyorsun?" diye sordu, "Bana anlatamaz misin?" Amnon, "Üvey kardesim Avsalom'un kizkardesi Tamar'a gönül verdim" diye yanitladi.
\par 5 Yonadav, "Yataga yat ve hastaymis gibi yap" dedi, "Baban seni görmeye gelince ona söyle dersin: 'Lütfen kizkardesim Tamar gelip bana yiyecek versin. Yemegi önümde hazirlasin ki, ona bakayim, elinden yiyeyim."
\par 6 Böylece Amnon yataga yatip hastaymis gibi yapti. Kral onu görmeye gelince, Amnon, "Lütfen kizkardesim Tamar gelip önümde iki gözleme hazirlasin da elinden yiyeyim" dedi.
\par 7 Davut, sarayda yasayan Tamar'a, "Haydi kardesin Amnon'un evine gidip ona yiyecek hazirla" diye haber gönderdi.
\par 8 Tamar yatmakta olan kardesi Amnon'un evine gitti. Hamur alip yogurdu, önünde gözleme yapip pisirdi.
\par 9 Tavayi alip gözlemeyi önüne koyduysa da Amnon yemek istemedi. "Yanimdan herkesi çikarin" diye buyruk verdi. Herkes çikti.
\par 10 Sonra Amnon Tamar'a, "Yemegi yatak odama getir de, elinden yiyeyim" dedi. Tamar hazirladigi gözlemeleri kardesi Amnon'un yatak odasina götürdü.
\par 11 Yesin diye yemegi ona yaklastirinca, Amnon Tamar'i yakalayarak, "Gel, benimle yat, kizkardesim" dedi.
\par 12 Ama Tamar, "Hayir, kardesim, beni zorlama!" dedi, "Israil'de böyle sey yapilmamalidir! Bu igrençligi yapma!
\par 13 Sonra ben utancimi nasil üstümden atarim? Sense Israil'de alçak biri durumuna düsersin. Ne olur krala söyle; o beni senden esirgemez."
\par 14 Ne var ki, Amnon Tamar'i dinlemek istemedi. Daha güçlü oldugu için onunla zorla yatti.
\par 15 Bundan sonra Amnon Tamar'dan öylesine nefret etti ki, ona duydugu nefret, beslemis oldugu sevgiden daha güçlüydü. Amnon Tamar'a, "Kalk, git!" dedi.
\par 16 Tamar, "Hayir" dedi, "Çünkü beni kovman, bana yaptigin öbür kötülükten daha büyük bir kötülüktür." Ama Amnon onu dinlemek istemedi.
\par 17 Hizmetindeki usagi çagirip, "Bu kadini yanimdan disari çikar, ardindan da kapiyi sürgüle" dedi.
\par 18 Usak Tamar'i disari çikarip ardindan kapiyi sürgüledi. Tamar uzun kollu bir giysi giymisti. Kralin erden kizlari böyle giyinirlerdi.
\par 19 Tamar basina kül saçip sirtindaki uzun kollu giysiyi yirtti. Elini basina koyup aglaya aglaya gitti.
\par 20 Kardesi Avsalom ona, "Seninle birlikte olan kardesin Amnon muydu?" diye sordu, "Haydi, kizkardesim, sesini çikarma. O senin üvey kardesindir. Bu olayin üzerinde durma." Böylece Tamar, kardesi Avsalom'un evinde yalniz ve üzgün yasadi.
\par 21 Kral Davut olup bitenleri duyunca çok öfkelendi.
\par 22 Avsalom ise Amnon'a iyi kötü hiçbir sey söylemedi. Kizkardesi Tamar'a tecavüz ettigi için Amnon'dan nefret ediyordu.
\par 23 Tam iki yil sonra, Avsalom kralin bütün ogullarini kendi koyun kirkicilarinin bulundugu Efrayim Kenti yakinindaki Baal-Hasor'a çagirdi.
\par 24 Avsalom krala gelip, "Koyunlarimi kirktiriyorum" dedi, "Lütfen kral ve görevlileri de kuluna katilsin."
\par 25 Kral Davut, "Hayir, oglum, hepimiz gelmeyelim, sana yük oluruz" diye yanitladi. Avsalom üstelediyse de kral gitmek istemedi, ama onu kutsadi.
\par 26 Bunun üzerine Avsalom, "Öyleyse izin ver de kardesim Amnon bizimle gelsin" dedi. Kral, "Amnon neden seninle gelsin?" diye sordu.
\par 27 Ancak Avsalom üsteleyince, kral Amnon'u ve bütün öbür ogullarini onunla gönderdi.
\par 28 Avsalom hizmetkârlarina söyle buyurdu: "Dinleyin! Amnon'un saraptan iyice keyiflendigi ani bekleyin. Size 'Amnon'u vurun dedigim an onu öldürün. Korkmayin! Size buyrugu ben veriyorum. Güçlü ve yürekli olun!"
\par 29 Hizmetkârlar Avsalom'un buyruguna uyarak Amnon'u öldürdüler. Kralin öbür ogullari katirlarina atlayip kaçtilar.
\par 30 Onlar yoldayken, Avsalom'un kralin bütün ogullarini öldürdügü, hiçbirinin sag kalmadigi söylentisi Davut'a ulasti.
\par 31 Kral kalkip giysilerini yirtti, yere kapandi. Bütün görevlileri de, giysileri yirtilmis, yanibasindaydilar.
\par 32 Davut'un kardesi Sima'nin oglu Yonadav söyle dedi: "Efendim kral bütün ogullarinin öldürüldügünü sanmasin; yalniz Amnon öldü. Çünkü o üvey kizkardesi Tamar'a tecavüz ettigi günden bu yana, Avsalom buna kararliydi.
\par 33 Onun için, ey efendim kral, bütün ogullarinin öldügü haberini dikkate alma; çünkü yalniz Amnon öldü."
\par 34 Bu arada Avsalom kaçti. Nöbetçi tepenin yamacindaki bati yolundan büyük bir kalabaligin geldigini gördü.
\par 35 Yonadav krala, "Iste ogullarin geliyor! Kulunun dedigi gibi oldu" dedi.
\par 36 O konusmasini bitirir bitirmez, kralin ogullari oraya varip hiçkira hiçkira aglamaya basladilar. Kral ve görevlileri de aci aci agladilar.
\par 37 Avsalom Gesur Krali Ammihut oglu Talmay'in yanina kaçti. Davut ise oglu Amnon için sürekli yas tutuyordu.
\par 38 Gesur'a kaçan Avsalom orada üç yil kaldi.
\par 39 Kral Davut Avsalom'un yanina gitmeyi çok istiyordu. Çünkü Amnon'un ölümü konusunda avuntu bulmustu.

\chapter{14}

\par 1 Kral Davut'un Avsalom'u özledigini anlayan Seruya oglu Yoav, birini gönderip Tekoa'da yasayan bilge bir kadini getirtti. Yoav kadina, "Lütfen yasa bürün" dedi, "Yas giysilerini giy. Yag sürme ve ölü için günlerdir yas tutan bir kadin gibi davran.
\par 3 Krala git ve ona söyleyeceklerimi ilet." Sonra kadina neler söyleyecegini bildirdi.
\par 4 Tekoali kadin krala gitti. Önünde yüzüstü yere kapanarak, "Ey kral, yardim et!" dedi.
\par 5 Kral, "Neyin var?" diye sordu. Kadin, "Ben zavalli dul bir kadinim" diye yanitladi, "Kocam öldü.
\par 6 Ben kölenin iki oglu vardi. Ikisi tarlada kavgaya tutustular. Orada onlari ayiracak kimse yoktu. Biri öbürünü vurup öldürdü.
\par 7 Simdi bütün boy halki cariyene karsi çikip, 'Kardesini öldüreni bize teslim et diyor, 'Öldürdügü kardesinin canina karsilik onu öldürelim. Böylece mirasçiyi da ortadan kaldirmis oluruz. Iste geri kalan közümü de söndürecekler; yeryüzünde kocamin adini sürdürecek soy kalmayacak."
\par 8 Kral, "Evine dön, ben davanla ilgili buyruk verecegim" dedi.
\par 9 Tekoali kadin, "Efendim kral, bu olayin suçlusu ben ve babamin ev halki olsun" dedi, "Kral ve tahti suçsuz olsun."
\par 10 Kral, "Kim sana bir sey derse, onu bana getir" dedi, "Bir daha canini sikmaz."
\par 11 Kadin, "Öyleyse kral Tanrisi RAB'bin adina ant içsin de kanin öcünü alacak kisi yikimi büyütmesin" diye karsilik verdi, "Yoksa oglumu yok edecekler." Kral, "Yasayan RAB'bin adiyla derim ki, oglunun saçinin bir teline bile zarar gelmeyecektir" dedi.
\par 12 Kadin, "Izin ver de, efendim krala bir söz daha söyleyeyim" dedi. Kral, "Söyle" dedi.
\par 13 Kadin konusmasini söyle sürdürdü: "Neden Tanri'nin halkina karsi böyle bir sey tasarladin? Kral böyle konusmakla sanki kendini suçlu çikariyor. Çünkü sürgüne gönderdigi kisiyi geri getirmedi.
\par 14 Hepimizin ölecegi kesin, topraga dökülüp yeniden toplanamayan su gibiyiz. Ama Tanri can almaz; sürgüne gönderilen kisi kendisinden uzak kalmasin diye çözüm yollari düsünür.
\par 15 "Halk beni korkuttugu için efendim krala bunlari söylemeye geldim. 'Kralla konusayim, belki kölesinin dilegini yerine getirir diye düsündüm,
\par 16 'Belki kral oglumla beni öldürüp Tanri'nin halkindan yoksun birakmak isteyenin elinden kurtarmayi kabul eder.
\par 17 Efendim kralin sözü beni rahatlatsin dedim. Çünkü efendim kral iyiyi, kötüyü ayirt etmekte Tanri'nin melegi gibidir. Tanrin RAB seninle olsun!"
\par 18 Kral, "Sana bir soru soracagim, benden gerçegi saklama" dedi. Kadin, "Efendim kral, buyur" diye karsilik verdi.
\par 19 Kral, "Bütün bunlari seninle birlikte tasarlayan Yoav mi?" diye sordu. Kadin söyle yanitladi: "Yasamin hakki için derim ki, ey efendim kral, hiçbir sorunu yanitlamaktan kaçamam. Evet, bana buyruk veren ve kölene bütün bunlari söyleten kulun Yoav'dir.
\par 20 Kulun Yoav duruma bir çözüm getirmek için yapti bunu. Efendim, Tanri'nin bir melegi gibi bilgedir. Ülkede olup biten her seyi bilir."
\par 21 Bunun üzerine kral Yoav'a, "Istedigini yapacagim" dedi, "Git, genç Avsalom'u geri getir."
\par 22 Yoav yüzüstü yere kapanarak onu kutsadi ve, "Ey efendim kral, bugün benden hosnut oldugunu biliyorum, çünkü kulunun istegini yaptin" dedi.
\par 23 Yoav hemen Gesur'a gidip Avsalom'u Yerusalim'e getirdi.
\par 24 Ne var ki, kral, "Avsalom evine gitsin, yanima gelmesin" diye buyruk verdi. Bu yüzden Avsalom evine gitti; krali görmedi.
\par 25 Bütün Israil'de Avsalom kadar yakisikliligi için övülen kimse yoktu; tepeden tirnaga kusursuz biriydi.
\par 26 Avsalom saçini kestirdigi zaman tartardi. Saçi ona agirlik verdigi için her yil kestirirdi. Saçinin agirligi krallik ölçüsüne göre iki yüz sekel çekerdi.
\par 27 Avsalom'un üç oglu ve Tamar adinda çok güzel bir kizi vardi.
\par 28 Avsalom krali görmeden Yerusalim'de iki yil yasadi.
\par 29 Sonra Yoav'i krala göndermek için ona haber saldi. Ama Yoav gelmek istemedi. Avsalom ikinci kez haber gönderdi, Yoav yine gelmek istemedi.
\par 30 Avsalom kullarina, "Bakin, Yoav'in arpa tarlasi benimkine bitisiktir" dedi, "Gidin, tarlayi atese verin." Bunun üzerine gidip tarlayi atese verdiler.
\par 31 Yoav kalkip Avsalom'un evine gitti. "Kullarin neden tarlami atese verdi?" diye sordu.
\par 32 Avsalom söyle yanitladi: "Bak, sana, 'Buraya gel, seni krala göndereyim diye haber yolladim. Ona sunlari söylemeni isteyecektim: 'Neden Gesur'dan geldim? Orada kalsaydim benim için daha iyi olurdu. Artik krali görmek istiyorum. Bir suçum varsa, beni öldürsün."
\par 33 Bunun üzerine Yoav gidip Avsalom'un söylediklerini krala iletti. Kral Avsalom'u çagirtti. Avsalom kralin yanina gelip önünde yüzüstü yere kapandi. Kral da onu öptü.

\chapter{15}

\par 1 Bundan sonra Avsalom kendisine bir savas arabasi, atlar ve önünde kosacak elli kisi hazirladi.
\par 2 Sabah erkenden kalkip kent kapisina giden yolun kenarinda dururdu. Davasina baktirmak için krala gelen herkese seslenip, "Nerelisin?" diye sorardi. Adam hangi Israil oymagindan geldigini söylerdi.
\par 3 Avsalom ona söyle derdi: "Bak, ileri sürdügün savlar dogru ve hakli. Ne var ki, kral adina seni dinleyecek kimse yok."
\par 4 Sonra konusmasini söyle sürdürürdü: "Keske kral beni ülkeye yargiç atasa! Davasi ya da sorunu olan herkes bana gelse, ben de ona hakkini versem!"
\par 5 Biri önünde yüzüstü yere kapanmak üzere yaklasti mi, Avsalom elini uzatip adami tutar, öperdi.
\par 6 Davasina baktirmak için krala gelen Israilliler'in hepsine böyle davrandi. Böylelikle Israilliler'in gönlünü çeldi.
\par 7 Dört yil sonra Avsalom krala, "Izin ver de Hevron'a gidip RAB'be adagimi yerine getireyim" dedi,
\par 8 "Çünkü ben kulun Aram'in Gesur Kenti'nde yasarken, 'RAB beni Yerusalim'e geri getirirse, O'na Hevron'da tapinacagim diye adak adamistim."
\par 9 Kral, "Esenlikle git" dedi. Ne var ki, Hevron'a giden Avsalom bütün Israil oymaklarina gizlice ulaklar göndererek söyle dedi: "Boru sesini duyar duymaz, 'Avsalom Hevron'da kral oldu diyeceksiniz."
\par 11 Yerusalim'den çagrilan iki yüz kisi olup bitenden haberleri olmaksizin, iyi niyetle Avsalom'la birlikte gittiler.
\par 12 Avsalom kurbanlari keserken, Davut'un danismani Gilolu Ahitofel'i de Gilo Kenti'nden getirtti. Böylece ayaklanma güç kazandi. Çünkü Avsalom'u izleyen halkin sayisi giderek çogaliyordu.
\par 13 Bir ulak gelip Davut'a, "Israilliler Avsalom'a yürekten baglandi" dedi.
\par 14 Bunun üzerine Davut Yerusalim'de kendisiyle birlikte olan bütün görevlilerine söyle dedi: "Haydi kaçalim! Yoksa Avsalom'dan kaçip kurtulamayacagiz. Hemen gidelim! Yoksa Avsalom ardimizdan çabucak yetisip bizi yikima ugratir. Kenti de kiliçtan geçirir."
\par 15 Kralin görevlileri, "Efendimiz kral ne karar verirse yapmaya haziriz" diye yanitladilar.
\par 16 Böylece kral ardisira gelen bütün ev halkiyla birlikte yola koyuldu. Ancak saraya baksinlar diye on cariyesini orada birakti.
\par 17 Kralla yanindakiler kentin en son evinde durdular.
\par 18 Bütün kullari, Keretliler'le Peletliler* kralin yanindan geçtiler. Gat'tan ardisira gelmis olan alti yüz Gatli asker de kralin önünden geçti.
\par 19 Kral Gatli Ittay'a, "Neden sen de bizimle geliyorsun?" dedi, "Geri dön ve yeni kralla kal. Çünkü sen yurdundan sürülmüs bir yabancisin.
\par 20 Daha dün geldin. Bugün nereye gidecegimi kendim bilmezken, seni de bizimle birlikte mi dolastirayim? Kardeslerinle birlikte geri dön. Tanri'nin sevgisi ve sadakati üzerinde olsun!"
\par 21 Ama Ittay söyle yanitladi: "Efendim kral, yasayan RAB'bin adiyla ve yasamin hakki için derim ki, ister yasam, ister ölüm için olsun, sen neredeysen kulun ben de orada olacagim."
\par 22 Davut Ittay'a, "Yürü, geç!" dedi. Böylece Gatli Ittay yanindaki bütün adamlari ve çocuklariyla birlikte geçti.
\par 23 Halk geçerken, bütün yöre halki hiçkira hiçkira agliyordu. Kral Kidron Vadisi'ni geçti. Halk da kirlara dogru ilerledi.
\par 24 Kâhin Sadok'la Tanri'nin Antlasma Sandigi'ni* tasiyan Levililer de oradaydi. Tanri'nin Sandigi'ni yere koydular. Bütün halk kentten çikana dek Aviyatar sunular sundu.
\par 25 Sonra kral, Sadok'a, "Tanri'nin Sandigi'ni kente geri götür" dedi, "RAB benden hosnut kalirsa, beni geri getirir, sandigi ve kondugu yeri bana gösterir.
\par 26 Ama, 'Senden hosnut degilim derse, iste buradayim, bana uygun gördügünü yapsin."
\par 27 Kral Kâhin Sadok'la konusmasini söyle sürdürdü: "Sen bilici* degil misin? Oglun Ahimaas'i ve Aviyatar oglu Yonatan'i yanina al; Aviyatar'la birlikte esenlikle kente dönün.
\par 28 Sizden aydinlatici bir haber alana dek ben kirda, irmagin sig yerinde bekleyecegim."
\par 29 Böylece Sadok'la Aviyatar Tanri'nin Sandigi'ni Yerusalim'e geri götürüp orada kaldilar.
\par 30 Davut aglaya aglaya Zeytin Dagi'na çikiyordu. Basi örtülüydü, yalinayak yürüyordu. Yanindaki herkesin basi örtülüydü ve aglayarak daga çikiyorlardi.
\par 31 O sirada biri Davut'a, "Ahitofel Avsalom'dan yana olan suikastçilarin arasinda" diye bildirdi. Bunun üzerine Davut, "Ya RAB, Ahitofel'in ögüdünü bosa çikar" diye dua etti.
\par 32 Davut Tanri'ya tapilan tepenin doruguna varinca, Arkli Husay giysisi yirtilmis, basi toz toprak içinde onu karsiladi.
\par 33 Davut ona, "Benimle birlikte gelirsen, bana yük olursun" dedi,
\par 34 "Ama kente döner ve Avsalom'a, 'Ey kral, senin kulun olacagim; geçmiste babana nasil kulluk ettiysem, simdi de sana öyle kulluk edecegim dersen, Ahitofel'in ögüdünü benim için bosa çikarirsin.
\par 35 Kâhin Sadok ile Kâhin Aviyatar orada seninle birlikte olacaklar. Kralin sarayinda duydugun her seyi onlara bildir.
\par 36 Sadok oglu Ahimaas ile Aviyatar oglu Yonatan da oradalar. Bütün duyduklarinizi onlarin araciligiyla bana iletebilirsiniz."
\par 37 Böylece Davut'un dostu Husay Yerusalim'e gitti. Tam o sirada Avsalom da kente giriyordu.

\chapter{16}

\par 1 Davut tepenin dorugunu biraz geçince, Mefiboset'in hizmetkâri Siva palan vurulmus ve üzerlerine iki yüz ekmek, yüz salkim kuru üzüm, yüz tane taze meyve ve bir tulum sarap yüklü iki esekle onu karsiladi.
\par 2 Kral, Siva'ya, "Bunlari niçin getirdin?" diye sordu. Siva, "Esekler kral ailesinin binmesi, ekmekle taze meyve hizmetkârlarin yemesi, sarapsa kirda yorgun düsenlerin içmesi için" diye yanitladi.
\par 3 Kral, "Efendin Saul'un torunu nerede?" diye sordu. Siva, "Yerusalim'de kaliyor" diye yanitladi, "Çünkü 'Israil halki bugün atamin kralligini bana geri verecek diye düsünüyor."
\par 4 Kral, "Mefiboset'in her seyi senindir" dedi. Siva, "Önünde egilirim, efendim kral! Dilerim her zaman benden hosnut kalirsin" dedi.
\par 5 Kral Davut Bahurim'e vardiginda, Saul ailesinin geldigi boydan Gera oglu Simi adinda biri lanetler okuyarak ortaya çikti.
\par 6 Bütün askerler ve koruyucular Kral Davut'un saginda, solunda olmasina karsin, Simi Davut'la askerlerini tasliyordu.
\par 7 Simi lanetler okuyarak, "Çekil git, ey eli kanli, alçak adam!" diyordu,
\par 8 "RAB, yerine kral oldugun Saul ailesinin dökülen kanlarinin karsiligini sana verdi. RAB kralligi oglun Avsalom'a verdi. Sen eli kanli bir adam oldugun için bu yikima ugradin!"
\par 9 Seruya oglu Avisay krala, "Bu ölü köpek neden efendim krala lanet okusun?" dedi, "Izin ver de gidip basini uçurayim."
\par 10 Ama kral, "Bu sizin isiniz degil, ey Seruya ogullari!" dedi, "RAB ona, 'Davut'a lanet oku dedigi için lanet okuyorsa, kim, 'Bunu neden yapiyorsun diye sorabilir?"
\par 11 Sonra Davut Avisay'la askerlerine, "Öz oglum beni öldürmeye çalisirken, su Benyaminli'nin yaptigina sasmamali" dedi, "Birakin onu, lanet okusun, çünkü ona böyle yapmasini RAB buyurmustur.
\par 12 Belki RAB sikintimi görür de, bugün okunan lanetlerin karsiligini iyilikle verir."
\par 13 Davut'la adamlari yollarina devam ettiler. Davut'un karsisinda, dagin yamacinda yürüyen Simi, giderken ona lanet okuyor, tas, toprak atiyordu.
\par 14 Gidecekleri yere yorgun argin varan kralla yanindaki halk orada dinlendiler.
\par 15 Avsalom'la Israil halki Yerusalim'e girmislerdi. Ahitofel de Avsalom'la birlikteydi.
\par 16 Davut'un dostu Arkli Husay, Avsalom'un yanina varinca, "Yasasin kral! Yasasin kral!" diye bagirdi.
\par 17 Avsalom Husay'a, "Dostuna bagliligin bu mu? Neden dostunla gitmedin?" diye sordu.
\par 18 Husay, "Hayir" diye yanitladi, "Ben RAB'bin, bu halkin ve bütün Israilliler'in seçtigi kisiden yana olacagim, onun yaninda kalacagim.
\par 19 Üstelik Davut oglu Avsalom'dan baska kime hizmet edecegim? Babana nasil hizmet ettiysem, sana da öyle hizmet edecegim."
\par 20 Avsalom Ahitofel'e, "Ne yapmaliyiz, bize ögüt ver" dedi.
\par 21 Ahitofel, "Babanin saraya bakmak için biraktigi cariyelerle yat" diye karsilik verdi, "Böylece bütün Israil babanin nefretini kazandigini duyacak ve seni destekleyenlerin tümü kendilerini daha da güçlenmis bulacaklar."
\par 22 Sarayin daminda Avsalom için bir çadir kurdular. Avsalom bütün Israilliler'in gözü önünde babasinin cariyelerinin yanina girdi.
\par 23 O günlerde Ahitofel'in verdigi ögüt, Tanri sözünü ileten bir adaminki gibiydi. Davut da, Avsalom da onun ögüdünü öyle kabul ederlerdi.

\chapter{17}

\par 1 Ahitofel Avsalom'a söyle dedi: "Izin ver de on iki bin kisi seçeyim, bu gece kalkip Davut'un pesine düseyim.
\par 2 Davut yorgun ve güçsüzken ona saldirip gözünü korkutayim. Yanindakilerin hepsi kaçacaktir. Ben de yalniz Kral Davut'u öldürürüm.
\par 3 Sonra bütün halki sana geri getiririm. Halkin dönmesi, öldürmek istedigin adamin ölümüne baglidir. Böylece halk da esenlikte olur."
\par 4 Bu ögüt Avsalom'u ve Israil ileri gelenlerini hosnut etti.
\par 5 Avsalom, "Arkli Husay'i da çagirin, neler söyleyecegini duyalim" dedi.
\par 6 Husay gelince Avsalom, "Ahitofel bu ögüdü verdi" dedi, "Onun ögüdüne uyalim mi? Yoksa, sen ögüt ver."
\par 7 Husay Avsalom'a, "Bu kez Ahitofel'in verdigi ögüt iyi degil" dedi,
\par 8 "Baban Davut'la adamlarinin güçlü savasçilar olduklarini biliyorsun. Kirda yavrularindan yoksun birakilmis bir ayi gibi öfkeliler. Baban deneyimli bir savasçidir, geceyi askerlerle geçirmez.
\par 9 Su anda ya bir magarada ya da baska bir yerde gizlenmistir. Davut askerlerine karsi ilk saldiriyi yapinca, bunu her duyan, 'Avsalom'u destekleyenler arasinda kirim var diyecek.
\par 10 O zaman aslan yürekli yigitler bile korkuya kapilacak. Çünkü bütün Israilliler babanin güçlü, yanindakilerin de yigit oldugunu bilir.
\par 11 "Onun için sana ögüdüm su: Dan'dan Beer-Seva'ya kadar, kiyilarin kumu kadar olan Israilliler çevrene toplansin, sen de savasa katil.
\par 12 O zaman gizlendigi yerlerden birinde Davut'un üstüne yürürüz; yeryüzüne düsen çiy gibi üzerine gideriz. Onu da, yanindakilerin hiçbirini de yasatmayiz.
\par 13 Eger bir kente çekilirse, Israilliler o kente halatlar getirir, tek bir tas kalmayincaya dek kenti vadiye indiririz."
\par 14 Avsalom'la Israilliler, "Arkli Husay'in ögüdü Ahitofel'in ögüdünden daha iyi" dediler. Çünkü RAB, Avsalom'u yikima ugratmak için, Ahitofel'in iyi ögüdünü bosa çikarmayi tasarlamisti.
\par 15 Husay Kâhin Sadok'la Kâhin Aviyatar'a söyle dedi: "Ahitofel Avsalom'a ve Israil'in ileri gelenlerine böyle ögüt verdi, bense söyle ögüt verdim.
\par 16 Simdi siz Davut'a hemen su haberi gönderin: 'Geceyi kirdaki irmagin sig yerinde geçirme, duraksamadan karsi yakaya geç; yoksa kral da yanindakilerin tümü de yok olabilir."
\par 17 Bu sirada Yonatan'la Ahimaas Eyn-Rogel'de kaliyorlardi. Bir hizmetçi kiz gidip onlara olup bitenleri haber veriyor, onlar da gidip duyduklarini Kral Davut'a bildiriyorlardi. Çünkü kendileri kente girerken görünmeyi göze alamiyorlardi.
\par 18 Ama bir genç onlari görüp Avsalom'a bildirdi. Bunun üzerine Yonatan'la Ahimaas hemen oradan ayrilip Bahurim'de bir adamin evine gittiler. Evin avlusunda bir kuyu vardi. Yonatan'la Ahimaas kuyuya indiler.
\par 19 Adamin karisi bir örtü alip kuyunun agzina serdi. Bir sey belli olmasin diye örtünün üstüne basak yaydi.
\par 20 Avsalom'un görevlileri eve, kadinin yanina varinca, "Ahimaas'la Yonatan nerede?" diye sordular. Kadin, "Irmagin karsi yakasina geçtiler" diye yanitladi. Avsalom'un görevlileri onlari aramaya gittiler; bulamayinca Yerusalim'e döndüler.
\par 21 Adamlar gittikten sonra, Ahimaas'la Yonatan kuyudan çiktilar ve olup bitenleri bildirmek üzere Kral Davut'a gittiler. Ona, "Haydi, hemen irmagi geçin" dediler, "Çünkü Ahitofel size karsi böyle ögüt verdi."
\par 22 Bunun üzerine Davut'la yanindaki bütün halk Seria Irmagi'ni çabucak geçti. Safak söktügünde Seria Irmagi'ni geçmeyen bir kisi bile kalmamisti.
\par 23 Ahitofel, verdigi ögüde uyulmadigini görünce, esegine palan vurdu; yola koyulup kentine, evine döndü. Islerini düzene koyduktan sonra kendini asti. Ölüsünü babasinin mezarina gömdüler.
\par 24 Davut Mahanayim'e vardigi sirada Avsalom'la yanindaki Israil askerleri Seria Irmagi'ni geçtiler.
\par 25 Avsalom Yoav'in yerine Amasa'yi ordu komutani atamisti. Amasa Yitra adinda bir Ismaili'nin ogluydu. Annesi Nahas'in kizi Avigayil'di; Yoav'in annesi Seruya'nin kizkardesiydi.
\par 26 Avsalom'la Israilliler Gilat bölgesinde ordugah kurdular.
\par 27 Davut Mahanayim'e vardiginda, Ammonlular'in Rabba Kenti'nden Nahas oglu Sovi, Lo-Devarli Ammiel oglu Makir ve Rogelim'den Gilatli Barzillay ona yataklar, taslar, toprak kaplar getirdiler. Ayrica Davut'la yanindakilerin yemesi için bugday, arpa, un, kavrulmus bugday, bakla, mercimek, bal, tereyagi, inek peyniri ve koyun da getirdiler. "Halk kirda yorulmustur, aç ve susuzdur" diye düsünmüslerdi.

\chapter{18}

\par 1 Davut kendini destekleyen askerleri bir araya topladi. Onlara binbasilar ve yüzbasilar atadi.
\par 2 Sonra orduyu Seruya oglu Yoav'in, kardesi Avisay'in ve Gatli Ittay'in denetiminde üç kol halinde gönderdi. Kral askerlere, "Ben de sizinle birlikte gidecegim" dedi.
\par 3 Ancak askerler, "Bizimle gelmemelisin" diye karsilik verdiler, "Çünkü kaçmak zorunda kalirsak düsmanlarimiz bizi umursamaz; yarimiz ölse bile umursamazlar. Sen bizim gibi on bin adama degersin. Sen kentten bize yardim et, daha iyi."
\par 4 Kral, "Gözünüzde iyi olani yapacagim" dedi. Adamlari yüzer ve biner kisilik birlikler halinde kentten çikarken kral kapinin yaninda duruyordu.
\par 5 Kral, Yoav'a, Avisay'a ve Ittay'a, "Benim hatirim için genç Avsalom'a sert davranmayin" diye buyurdu. Bütün askerler kralin komutanlara Avsalom'a iliskin buyruk verdigini duydular.
\par 6 Davut'un ordusu Israilliler'le savasmak üzere tarlalara çikti. Savas Efrayim Ormani'nda basladi.
\par 7 Israil ordusu Davut'un adamlari önünde yenilgiye ugradi. Büyük bir kirim oldu. O gün yirmi bin kisi öldü.
\par 8 Savas her yana yayildi. O gün ormanda yok olanlarin sayisi kiliçtan geçirilenlerin sayisindan daha çoktu.
\par 9 Avsalom ansizin Davut'un adamlariyla karsilasti. Avsalom katira binmisti. Katir büyük bir yabanil fistik agacinin sik dallari altindan geçerken, Avsalom'un basi dallara takildi. Katir yoluna devam edince, Avsalom havada asili kaldi.
\par 10 Adamlardan biri bunu gördü. Yoav'a, "Avsalom'u bir yabanil fistik agacina asili gördüm" diye bildirdi.
\par 11 Yoav, haberi verene, "Onu gördün mü? Neden onu orada öldürmedin? Sana on parça gümüsle bir kemer verirdim" dedi.
\par 12 Ama adam, "Elime bin parça gümüs saysan bile, kralin ogluna elimi kaldirmam" diye yanitladi, "Çünkü kralin sana, Avisay'a ve Ittay'a, 'Benim hatirim için genç Avsalom'u koruyun diye buyruk verdigini duyduk.
\par 13 Oysa Avsalom'u öldürseydim -hiçbir sey kraldan gizli kalmaz- o zaman sen de beni savunmazdin."
\par 14 Yoav, "Seninle böyle vakit kaybedemem" dedi. Üç kargi aldi, yabanil fistik agacinda asili duran ve hâlâ sag olan Avsalom'un yüregine sapladi.
\par 15 Bunun üzerine Yoav'in silahlarini tasiyan on genç Avsalom'un çevresini sarip onu öldürdüler.
\par 16 Yoav boru çaldirinca, askerler Israilliler'i kovalamayi birakip geri döndüler. Yoav onlarin savasi sürdürmelerine engel oldu.
\par 17 Yoav'in askerleri Avsalom'u alip ormanda derin bir çukura attilar; üzerine büyük bir tas yigini yaptilar. Bütün Israilliler evlerine kaçtilar.
\par 18 Avsalom daha sagken bir direk alip kendisi için Kral Vadisi'ne dikmisti. Çünkü, "Adimi animsatacak bir oglum yok" diye düsünmüstü. Direge kendi adini vermisti. Bu direk bugün de Avsalom Aniti diye bilinir.
\par 19 Sadok oglu Ahimaas Yoav'a, "Izin ver de kosup krala RAB'bin onu düsmanlarinin elinden kurtardigini haber vereyim" dedi.
\par 20 Yoav, "Olmaz, bugün haberi götüren sen olmayacaksin" dedi, "Baska bir zaman haber götürürsün, ama bugün degil. Çünkü kralin oglu öldü."
\par 21 Sonra bir Kûslu'ya*, "Sen git, gördüklerini krala bildir" dedi. Kûslu Yoav'in önünde yere kapandi, sonra kosmaya basladi.
\par 22 Ama Sadok oglu Ahimaas yine, "Ne olursa olsun, izin ver, ben de Kûslu'nun ardisira kosayim" dedi. Yoav, "Oglum, neden kosmak istiyorsun?" dedi, "Sana ödül kazandiracak bir haberin yok ki!"
\par 23 Ahimaas, "Ne olursa olsun kosacagim" diye karsilik verdi. Yoav, "Kos öyleyse" dedi. Böylece Ahimaas Seria Ovasi yolundan kosarak Kûslu'yu geçti.
\par 24 Davut kentin iç ve dis kapilari arasinda oturuyordu. Nöbetçi surun yanindaki kapinin tepesine çikti. Çevreye göz gezdirince, tek basina kosan birini gördü.
\par 25 Krala seslenerek gördügünü bildirdi. Kral, "Tek basina geliyorsa, iyi haber getiriyor demektir" dedi. Adam gitgide yaklasiyordu.
\par 26 Nöbetçi kosan baska birini görünce, kapiciya, "Iste tek basina kosan bir adam daha!" diye seslendi. Kral, "O da iyi haber getiriyor" dedi.
\par 27 Nöbetçi, "Sanirim birinci adamin kosusu Sadok oglu Ahimaas'in kosusuna benziyor" dedi. Kral, "Ahimaas iyi adamdir" diye karsilik verdi, "Iyi haberle gelir."
\par 28 Ahimaas krala, "Her sey yolunda!" diye seslendi. Kralin önünde yüzüstü yere kapanarak, "Efendimiz krala el kaldiranlari teslim eden Tanrin RAB'be övgüler olsun!" dedi.
\par 29 Kral, "Genç Avsalom güvenlikte mi?" diye sordu. Ahimaas söyle yanitladi: "Yoav kralin hizmetkâri Kûslu'yla beni gönderdigi sirada büyük bir karisiklik gördüm, ama ne oldugunu anlamadim."
\par 30 Kral, "Bir yana çekilip burada bekle" dedi. Ahimaas da çekilip beklemeye basladi.
\par 31 Tam o sirada Kûslu geldi. "Efendimiz krala müjde!" dedi, "Bugün RAB sana karsi bütün ayaklananlarin elinden seni kurtardi."
\par 32 Kral Kûslu'ya, "Genç Avsalom güvenlikte mi?" diye sordu. Kûslu, "Efendimiz kral!" diye yanitladi, "Düsmanlarinin ve kötü amaçla sana karsi ayaklananlarin hepsinin sonu bu gencin sonu gibi olsun."
\par 33 Kral sarsildi. Giris kapisinin üstündeki odaya çikip agladi. Giderken, "Ah oglum Avsalom! Ah oglum, oglum Avsalom!" diye inliyordu, "Keske senin yerine ben ölseydim, oglum! Ah oglum Avsalom!"

\chapter{19}

\par 1 Yoav'a, "Kral Davut Avsalom için aglayip yas tutuyor" diye bildirdiler.
\par 2 O gün zafer ordu için yasa dönüstü. Çünkü kralin oglu için aci çektigini duymuslardi.
\par 3 Bu yüzden askerler, savas kaçaklari gibi, o gün kente utanarak girdiler.
\par 4 Kral ise yüzünü örtmüs, yüksek sesle, "Ah oglum Avsalom! Avsalom, oglum, oglum!" diye bagiriyordu.
\par 5 Yoav kralin bulundugu odaya giderek ona söyle dedi: "Bugün senin canini, ogullarinin, kizlarinin, eslerinin, cariyelerinin canlarini kurtaran adamlarinin hepsini utandirdin.
\par 6 Çünkü senden nefret edenleri seviyor, seni sevenlerden nefret ediyorsun: Bugün komutanlarinin ve adamlarinin senin gözünde degersiz oldugunu gösterdin. Evet, bugün anladim ki, Avsalom sag kalip hepimiz ölseydik senin için daha iyi olurdu!
\par 7 "Haydi kalk, gidip adamlarini yüreklendir! RAB'bin adiyla ant içerim ki, gitmezsen bu gece bir kisi bile seninle kalmayacak. Bu da, gençliginden simdiye dek basina gelen yikimlarin tümünden daha kötü olacak."
\par 8 Bunun üzerine kral gidip kentin kapisinda oturdu. Bütün askerlere, "Iste kral kentin kapisinda oturuyor" diye haber salindi. Onlar da kralin yanina geldiler. Bu arada Israilliler evlerine kaçmislardi.
\par 9 Israil oymaklarindan olan herkes birbiriyle tartisiyor ve, "Kral bizi düsmanlarimizin elinden kurtardi" diyordu, "Bizi Filistliler'in elinden kurtaran da odur. Simdiyse Avsalom yüzünden ülkeyi birakip kaçti.
\par 10 Bizi yönetmesi için meshettigimiz* Avsalom'sa savasta öldü. Öyleyse neden krali geri getirme konusunda susup duruyorsunuz?"
\par 11 Kral Davut Kâhin Sadok'la Kâhin Aviyatar'a su haberi gönderdi: "Yahuda'nin ileri gelenlerine deyin ki, 'Bütün Israil'de konusulanlar kralin konutuna dek ulastigina göre, krali sarayina geri getirmekte siz neden sonuncu oluyorsunuz?
\par 12 Siz kardeslerimsiniz; etim, kemigimsiniz! Krali geri getirmekte neden en son siz davraniyorsunuz?
\par 13 "Amasa'ya da söyle deyin: 'Sen etim, kemigimsin! Seni Yoav'in yerine ordunun sürekli komutani atamazsam, Tanri bana aynisini, hatta daha kötüsünü yapsin!"
\par 14 Böylece Davut bütün Yahudalilar'i derinden etkiledi. Yahudalilar krala, "Bütün adamlarinla birlikte dön!" diye haber gönderdiler.
\par 15 Kral dönüp Seria Irmagi'na vardi. Yahudalilar krali karsilamak ve Seria Irmagi'ndan geçirmek için Gilgal'a geldiler.
\par 16 Bahurim'den Benyaminli Gera oglu Simi de Kral Davut'u karsilamak için Yahudalilar'la birlikte çikageldi.
\par 17 Yaninda Benyamin oymagindan bin kisi vardi. Saul evinin hizmetkâri Siva da on bes oglu ve yirmi kölesiyle birlikte hemen Seria Irmagi'na, kralin yanina geldi.
\par 18 Kralin ev halkini karsiya geçirmek ve onun istedigini yapmak için irmagin sig yerinden karsi yakaya geçtiler. Kral Seria Irmagi'ni geçmek üzereydi ki, Gera oglu Simi kendini onun önüne atti.
\par 19 Krala, "Efendim, beni suçlu sayma" dedi, "Ey efendim kral, Yerusalim'den çiktigin gün isledigim suçu animsama, göz önünde tutma.
\par 20 Çünkü kulun günah isledigini biliyor. Efendim krali karsilamak için bugün bütün Yusuf soyundan ilk gelen benim."
\par 21 Seruya oglu Avisay, "Simi öldürülmeli, çünkü RAB'bin meshettigi kisiye lanet okudu" dedi.
\par 22 Davut, "Ey Seruya ogullari, bu sizin isiniz degil!" dedi, "Bugün bana düsman oldunuz. Israil'de bugün bir tek kisi öldürülmeyecek! Israil'in Krali oldugumu bilmiyor muyum?"
\par 23 Sonra ant içerek Simi'ye, "Ölmeyeceksin!" dedi.
\par 24 Saul'un torunu Mefiboset de krali karsilamaya gitti. Kralin gittigi günden esenlikle geri döndügü güne dek ayaklarini da, giysilerini de yikamamis, biyigini kesmemisti.
\par 25 Krali karsilamak için Yerusalim'den geldiginde, kral, "Mefiboset, neden benimle gelmedin?" diye sordu.
\par 26 Mefiboset söyle yanitladi: "Ey efendim kral! Kulun topal oldugundan, kulum Siva'ya, 'Esege palan vur da binip kralla birlikte gideyim dedim. Ama o beni kandirdi.
\par 27 Ayrica efendim kralin önünde kuluna kara çaldi. Ama sen, ey efendim kral, Tanri'nin bir melegi gibisin; gözünde dogru olani yap.
\par 28 Çünkü atamin ailesinin bütün bireyleri ölümü hak etmisken, kuluna sofrandakilerle birlikte yemek yeme ayricaligini tanidin. Artik senden daha baska bir sey dilemeye ne hakkim var, ey kral?"
\par 29 Kral, "Islerin hakkinda daha fazla konusmana gerek yok" dedi, "Sen ve Siva topraklari paylasin diye buyruk veriyorum."
\par 30 Mefiboset, "Madem efendim kral sarayina esenlikle döndü, bütün topraklari Siva alsin" diye karsilik verdi.
\par 31 Gilatli Barzillay da Seria Irmagi'ni geçiste krala eslik edip onu ugurlamak üzere Rogelim'den gelmisti.
\par 32 Barzillay çok yasliydi, seksen yasindaydi. Kral Mahanayim'de kaldigi sürece, geçimini o saglamisti. Çünkü Barzillay çok varlikliydi.
\par 33 Kral Barzillay'a, "Benimle karsiya geç, Yerusalim'de ben senin geçimini saglayacagim" dedi.
\par 34 Ama Barzillay, "Kaç yil ömrüm kaldi ki, seninle birlikte Yerusalim'e gideyim?" diye karsilik verdi,
\par 35 "Su anda seksen yasindayim. Iyi ile kötüyü ayirt edebilir miyim? Yedigimin, içtigimin tadini alabilir miyim? Kadin erkek sarkicilarin sesini duyabilir miyim? Öyleyse neden efendim krala daha fazla yük olayim?
\par 36 Kulun Seria Irmagi'ni kralla birlikte geçerek sana birazcik eslik edecek. Kral beni neden böyle ödüllendirsin?
\par 37 Izin ver de döneyim, kentimde, annemin babamin mezari yaninda öleyim. Ama kulun Kimham burada; o seninle karsiya geçsin. Uygun gördügünü ona yaparsin."
\par 38 Kral, "Kimham benimle karsiya geçecek ve ona senin uygun gördügünü yapacagim" dedi, "Benden ne dilersen yapacagim."
\par 39 Bundan sonra kralla bütün halk Seria Irmagi'ni geçti. Kral Barzillay'i öpüp kutsadi. Sonra Barzillay evine döndü.
\par 40 Kral Gilgal'a geçti. Kimham da onunla birlikte gitti. Bütün Yahudalilar'la Israilliler'in yarisi krala eslik ettiler.
\par 41 Sonra Israilliler krala varip söyle dediler: "Neden kardeslerimiz Yahudalilar seni çaldi? Neden seni, aile bireylerini ve bütün adamlarini Seria Irmagi'nin karsi yakasina geçirdiler?"
\par 42 Bunun üzerine Yahudalilar Israilliler'e, "Çünkü kral bizden biri!" dediler, "Buna neden kizdiniz? Kralin yiyeceklerinden bir sey yedik mi? Kendimize bir sey aldik mi?"
\par 43 Israilliler, "Kralda on payimiz var" diye yanitladilar, "Davut'ta sizden daha çok hakkimiz var. Öyleyse neden bizi küçümsüyorsunuz? Kralimizi geri getirmekten ilk söz eden biz degil miydik?" Ne var ki, Yahudalilar'in tepkisi Israilliler'inkinden daha sert oldu.

\chapter{20}

\par 1 O sirada Benyamin oymagindan Bikri oglu Seva adinda kötü bir adam bir rastlanti sonucu Gilgal'daydi. Seva boru çalip, "Isay oglu Davut'la ne ilgimiz Ne de payimiz var" dedi, "Ey Israilliler, herkes kendi evine dönsün!"
\par 2 Bunun üzerine bütün Israilliler Davut'u birakip Bikri oglu Seva'nin ardindan gitti. Yahudalilar ise krallarina bagli kalip Seria Irmagi'ndan Yerusalim'e dek ona eslik ettiler.
\par 3 Kral Davut Yerusalim'deki sarayina varinca, saraya bakmak için biraktigi on cariyeyi gözetim altina aldi, onlarin geçimini sagladi. Ancak yataklarina girmedi. Onlar da ölünceye dek göz altinda dul kadinlar gibi yasadilar.
\par 4 Davut Amasa'ya, "Üç gün içinde Yahudalilar'i yanima çagir. Sen de burada ol" dedi.
\par 5 Amasa Yahudalilar'i çagirmaya gitti. Ama belirlenen zamanda dönmedi.
\par 6 Bunun üzerine Davut Avisay'a, "Simdi Bikri oglu Seva bize Avsalom'dan daha büyük kötülük yapacak" dedi, "Efendinin adamlarini al ve onu kovala. Yoksa kendine surlu kentler bulup bizden kaçar."
\par 7 Böylece Yoav'in adamlari, Keretliler'le Peletliler* ve bütün koruyucular Bikri oglu Seva'yi kovalamak için Avisay'in komutasinda Yerusalim'den çiktilar.
\par 8 Givon'daki büyük kayanin yanina varinca, Amasa onlari karsilamaya geldi. Yoav savas giysisini giymisti. Giysinin üzerine bir kemer kusanmis, kemere kininda duran bir kiliç baglamisti. Yoav ilerlerken kiliç kinindan çikti.
\par 9 Yoav Amasa'ya, "Iyi misin, kardesim?" diye sordu. Onu öpmek için sag eliyle Amasa'nin sakalindan tuttu.
\par 10 Amasa Yoav'in elindeki kilici farketmedi. Yoav kilici karnina saplayinca, Amasa'nin bagirsaklari yere döküldü. Ikinci vurusa gerek kalmadan Amasa öldü. Bundan sonra Yoav'la kardesi Avisay, Bikri oglu Seva'yi kovalamayi sürdürdüler.
\par 11 Yoav'in adamlarindan biri, Amasa'nin ölüsü yaninda durup, "Yoav'i tutan ve Davut'tan yana olan herkes Yoav'in ardindan gitsin" dedi.
\par 12 Amasa'nin ölüsü yolun ortasinda kanlar içinde duruyordu. Yoav'in adami, ölüye yaklasan herkesin orada durdugunu görünce, Amasa'yi yoldan sürükleyip tarlaya götürdü ve üzerine bir örtü atti.
\par 13 Ölü yoldan kaldirildiktan sonra herkes Bikri oglu Seva'yi kovalamak için Yoav'in ardindan gitti.
\par 14 Seva bütün Israil oymaklarindan ve Berliler'in bölgesinden geçip Avel-Beytmaaka'ya geldi. Berliler de toplanip onu izleyerek kente girdiler.
\par 15 Yoav'la bütün adamlari varip Avel-Beytmaaka Kenti'nde Seva'yi kusattilar. Topraktan kentin suruna bitisik bir yigin yaptilar ve suru devirmek için yikmaya basladilar.
\par 16 O sirada bilge bir kadin kentin içinden seslendi: "Dinleyin! Dinleyin! Yoav'a buraya gelmesini söyleyin, onunla konusacagim."
\par 17 Yoav kadina yaklasti. Kadin, "Yoav sen misin?" diye sordu. Yoav, "Benim" diye yanitladi. Kadin, "Kölenin sözlerini dinle" dedi. Yoav, "Dinliyorum" dedi.
\par 18 Kadin konusmasini söyle sürdürdü: "Eskiden, 'Avel Kenti'ne danisin derlerdi ve sorunlari böyle çözerlerdi.
\par 19 Biz Israil'in esenligini isteyen güvenilir kisileriz. Sense Israil'e ana gibi kucak açan kentlerden birini yikmaya çalisiyorsun. Neden RAB'bin halkini yok etmek istiyorsun?"
\par 20 Yoav, "Asla!" diye yanitladi, "Ne yikmak, ne de yok etmek istiyorum.
\par 21 Durum öyle degil. Efrayim daglik bölgesinden Bikri oglu Seva adindaki adam Kral Davut'a baskaldirdi. Yalniz onu verin, ben de kentten geri çekileyim." Kadin, "Onun basi surun üzerinden sana atilacak" dedi.
\par 22 Sonra kadin bilgece ögüdüyle bütün halka gitti. Halk Bikri oglu Seva'nin basini kesip Yoav'a atti. Bunun üzerine Yoav boru çaldi. Adamlari kenti birakip evlerine gittiler. Yoav da Yerusalim'e, kralin yanina döndü.
\par 23 Yoav Israil ordusunun komutaniydi. Yehoyada oglu Benaya ise Keretliler'le Peletliler'in* komutaniydi.
\par 24 Adoram angaryasina çalisanlardan sorumluydu. Ahilut oglu Yehosafat devlet tarihçisiydi.
\par 25 Seva yazman, Sadok'la Aviyatar kâhindi.
\par 26 Yairli Ira ise Davut'un kâhiniydi.

\chapter{21}

\par 1 Davut'un döneminde, üç yil art arda kitlik oldu. Davut RAB'be danisti. RAB söyle yanitladi: "Buna kan döken Saul ile ailesi neden oldu. Çünkü Saul Givonlular'i öldürdü."
\par 2 Kral Givonlular'i çagirtip onlarla konustu. -Givonlular Israil soyundan degildi. Amorlular'dan sag kalan bir halkti. Israilliler onlari sag birakacaklarina ant içmislerdi. Ne var ki, Israil ve Yahuda halki için büyük gayret gösteren Saul onlari yok etmeye çalismisti.-
\par 3 Davut Givonlular'a, "Sizin için ne yapabilirim? RAB'bin halkini kutsamaniz için bu suçu nasil bagislatabilirim?" diye sordu.
\par 4 Givonlular ona söyle karsilik verdi: "Saul'la ailesinden ne altin ne de gümüs isteriz; Israil'de herhangi birini öldürmek de istemeyiz." Davut, "Ne isterseniz yaparim" dedi.
\par 5 Söyle karsilik verdiler: "Bizi yok etmeye çalisan ve Israil ülkesinin hiçbir yerinde yasamamamiz için bizi ortadan kaldirmayi tasarlayan adamin ogullarindan yedisi bize verilsin. RAB'bin seçilmisi Saul'un Giva Kenti'nde RAB'bin önünde onlari asalim." Kral, "Onlari verecegim" dedi.
\par 7 Kral, Saul oglu Yonatan'la RAB'bin önünde içtigi anttan ötürü, Yonatan oglu Mefiboset'i esirgedi.
\par 8 Onun yerine, Aya kizi Rispa'nin Saul'dan dogurdugu Armoni ve Mefiboset adindaki iki oglunu ve Saul kizi Merav'in Meholali Barzillay oglu Adriel'den dogurdugu bes oglunu aldi.
\par 9 Davut onlari Givonlular'in eline teslim etti. Givonlular onlari dagda, RAB'bin önünde astilar. Yedisi de ayni anda öldüler. Biçme zamaninin ilk günlerinde, arpa biçme zamaninin baslangicinda öldürüldüler.
\par 10 Aya kizi Rispa bir çul alip kendisi için bir kayanin üzerine serdi. Biçme zamaninin ilk günlerinden cesetlerin üzerine gökten yagmur yagana dek Rispa orada kaldi; cesetleri gündüzün yirtici kuslardan, geceleyin yabanil hayvanlardan korudu.
\par 11 Saul'un cariyesi Aya kizi Rispa'nin yaptiklari Davut'a bildirildi.
\par 12 Davut gidip Saul'un ve oglu Yonatan'in kemiklerini Yaves-Gilatlilar'dan aldi. Filistliler Gilboa Dagi'nda Saul'u öldürdükleri gün, onun ve oglunun cesetlerini Beytsean alaninda asmislardi. Yaves-Gilat halki da cesetleri gizlice oradan almisti.
\par 13 Davut Saul'un ve oglu Yonatan'in kemiklerini oradan getirdi. Asilmis yedi kisinin kemikleri de toplandi.
\par 14 Saul'la oglu Yonatan'in kemiklerini Benyamin bölgesindeki Sela'da Saul'un babasi Kis'in mezarina gömdüler. Kralin bütün buyruklarini yerine getirdiler. Bundan sonra Tanri ülkeyle ilgili yakarislari yanitladi.
\par 15 Filistliler'le Israilliler arasinda yeniden savas çikti. Davut'la adamlari gidip Filistliler'e karsi savastilar. O siralarda Davut bitkin düstü.
\par 16 Ucu üç yüz sekel agirliginda bir tunç* mizrak tasiyan ve yeni kiliç kusanan Rafaogullari'ndan Filistli Yisbi-Benov Davut'u öldürmeyi amaçliyordu. Ama Seruya oglu Avisay Davut'un yardimina kostu; saldirip onu öldürdü. Bundan sonra Davut'un adamlari ant içerek, Davut'a, "Israil'in isigini söndürmemek için bir daha bizimle birlikte savasa gelmeyeceksin" dediler.
\par 18 Bir süre sonra Filistliler'le Gov'da yine savas çikti. Bu savas sirasinda Husali Sibbekay Rafa soyundan Saf adindaki adami öldürdü.
\par 19 Israilliler'le Filistliler arasinda Gov'da bir savas daha çikti. Beytlehemli Yareoregim'in oglu Elhanan, Gatli Golyat'i öldürdü. Golyat'in mizraginin sapi dokumaci tezgahinin sirigi gibiydi.
\par 20 Gat'ta bir kez daha savas çikti. Orada dev gibi bir adam vardi. Elleri, ayaklari altisar parmakliydi. Toplam yirmi dört parmagi vardi. O da Rafa soyundandi.
\par 21 Adam Israilliler'e meydan okuyunca, Davut'un kardesi Sima'nin oglu Yonatan onu öldürdü.
\par 22 Bunlarin dördü de Gat'taki Rafa soyundandi. Davut'la adamlari tarafindan öldürüldüler.

\chapter{22}

\par 1 RAB, Davut'u bütün düsmanlarinin ve Saul'un elinden kurtardigi gün Davut RAB'be su ezgiyi okudu.
\par 2 Söyle dedi: "RAB benim kayam, siginagim, kurtaricimdir,
\par 3 Tanrim, kayamdir, O'na siginirim, Kalkanim, güçlü kurtaricim, Korunagim, siginacak yerimdir. Kurtaricim, zorbaliktan beni sen kurtarirsin!
\par 4 Övgüye deger RAB'be seslenir, Kurtulurum düsmanlarimdan.
\par 5 Çünkü ölüm dalgalari beni kusatti, Yikim selleri basti,
\par 6 Ölüler diyarinin baglari sardi, Ölüm tuzaklari çikti karsima.
\par 7 Sikinti içinde RAB'be yakardim, Tanrim'a seslendim. Tapinagindan sesimi duydu, Haykirisim kulaklarina ulasti.
\par 8 O zaman yeryüzü sarsilip sallandi, Titreyip sarsildi göklerin temelleri, Çünkü RAB öfkelenmisti.
\par 9 Burnundan duman yükseldi, Agzindan kavurucu ates Ve korlar fiskirdi.
\par 10 Kara buluta basarak Gökleri yarip indi.
\par 11 Bir Keruv'a* binip uçtu, Rüzgarin kanatlari üstünde belirdi.
\par 12 Karanligi örtündü, Kara bulutlari kendine çardak yapti.
\par 13 Varliginin pariltisindan Korlar savruluyordu.
\par 14 RAB göklerden gürledi, Duyurdu sesini Yüceler Yücesi.
\par 15 Savurup oklarini düsmanlarini dagitti, Simsek çaktirarak onlari saskina çevirdi.
\par 16 RAB'bin azarlamasindan, Burnundan çikan güçlü soluktan, Denizin dibi göründü, Yeryüzünün temelleri açiga çikti.
\par 17 RAB yukaridan elini uzatip tuttu, Çikardi beni derin sulardan.
\par 18 Beni zorlu düsmanimdan, Benden nefret edenlerden kurtardi, Çünkü onlar benden daha güçlüydü.
\par 19 Felaket günümde karsima dikildiler, Ama RAB bana destek oldu.
\par 20 Beni huzura kavusturdu, Kurtardi, çünkü benden hosnut kaldi.
\par 21 RAB dogrulugumun karsiligini verdi, Beni temiz ellerime göre ödüllendirdi.
\par 22 Çünkü RAB'bin yolunda yürüdüm, Tanrim'dan uzaklasarak kötülük yapmadim.
\par 23 O'nun bütün ilkelerini göz önünde tuttum, Kurallarindan ayrilmadim.
\par 24 O'nun önünde kusursuzdum, Suç islemekten sakindim.
\par 25 Bu yüzden RAB beni dogruluguma Ve gözünde pak yasayisima göre ödüllendirdi.
\par 26 Sadik kuluna sadakat gösterir, Kusursuz olana kusursuz davranirsin.
\par 27 Pak olanla pak olur, Egriye egri davranirsin.
\par 28 Alçakgönüllüleri kurtarir, Gururlulari gözler, gururunu kirarsin.
\par 29 Ya RAB, isigim sensin! Karanligimi aydinlatirsin.
\par 30 Desteginle akincilara saldirir, Seninle surlari asarim, Tanrim.
\par 31 Tanri'nin yolu kusursuzdur, RAB'bin sözü aridir. O kendisine siginan herkesin kalkanidir.
\par 32 Var mi RAB'den baska tanri? Tanrimiz'dan baska kaya var mi?
\par 33 Siginagim Tanri'dir, Yolumu dogru kilan O'dur.
\par 34 Ayaklar verdi bana, geyiklerinki gibi, Doruklarda tutar beni.
\par 35 Bana savasmayi ögretti, Kollarimla tunç* bir yayi gereyim diye.
\par 36 Bana zafer kalkanini bagislarsin, Alçakgönüllülügün beni yüceltir.
\par 37 Bastigim yerleri genisletirsin, Burkulmaz bileklerim.
\par 38 Düsmanlarimi kovalayip yok ettim, Hepsi yok olmadan geri dönmedim.
\par 39 Onlari ezip yok ettim, kalkamaz oldular, Ayaklarimin altina serildiler.
\par 40 Savas için beni güçle donattin, Bana baskaldiranlari önümde yere serdin.
\par 41 Düsmanlarimi kaçmak zorunda biraktin, Benden nefret edenleri yok ettim.
\par 42 Feryat ettiler, ama kurtaran çikmadi; RAB'bi çagirdilar, ama O yanit vermedi.
\par 43 Yerin tozu gibi onlari ezdim, Sokak çamuru gibi ayagimin altinda çignedim.
\par 44 Halkimin çekismelerinden beni kurtardin, Uluslara önder olarak beni korudun, Tanimadigim halklar bana kulluk ediyor.
\par 45 Yabancilar bana boyun egiyor, Duyar duymaz sözümü dinliyorlar.
\par 46 Yabancilarin betleri benizleri atti, Titreyerek çikiyorlar kalelerinden.
\par 47 RAB yasiyor! Kayam'a övgüler olsun! Yücelsin kurtaricim, Kayam Tanrim!
\par 48 O'dur öcümü alan, Halklari bana bagimli kilan.
\par 49 Düsmanlarimdan kurtarir, Baskaldiranlardan üstün kilar beni, Zorbalarin elinden alir.
\par 50 Bunun için uluslar arasinda sana sükredecegim, ya RAB, Adini ilahilerle övecegim.
\par 51 RAB kralini büyük zaferlere ulastirir, Meshettigi* krala, Davut'a ve soyuna Sonsuza dek sevgi gösterir."

\chapter{23}

\par 1 Davut'un son sözleri sunlardir: "Isay oglu Davut, Tanri'nin yükselttigi adam, Yakup'un Tanrisi'nin meshettigi*, Israil'in sevilen ezgi okuyucusu söyle diyor:
\par 2 RAB'bin Ruhu benim araciligimla konusuyor, Sözü dilimin ucundadir.
\par 3 Israil'in Tanrisi konustu, Israil'in kayasi bana dedi ki, 'Insanlari dogrulukla Ve Tanri korkusuyla yöneten kisi,
\par 4 Bulutsuz bir sabah, Safakta görünen gün isigi gibidir, Parlakligi yagmurdan sonra topraktan ot bitirir.
\par 5 Soyum da Tanri'yla böyle degil mi? O benimle sonsuza dek kalici, Her yönüyle düzenli ve güvenilir bir antlasma yapti. Kesin kurtulusa ve her dilegime kavusmami O saglamayacak mi?
\par 6 Kötülere gelince, Elle tutulamayan dikenler gibi Tümü bir yana atilacak.
\par 7 Dikenlere dokunan kisi, Demir bir araçla Ya da mizragin sapiyla dokunur. Dikenler olduklari yerde bütünüyle yakilacak."
\par 8 Davut'un yigit askerlerinin adlari sunlardir: Esnili Adino diye de bilinen üç kisinin önderi Tahkemonlu Yoseb-Bassevet bir saldirida sekiz yüz kisiyi öldürdü.
\par 9 Ikincisi, Ahohlu Dodo oglu Elazar. Pas-Dammim'de savasmak için toplanan Filistliler'e meydan okuyan Davut'un yanindaki üç yigitten biriydi. Israilliler o sirada geri çekilmislerdi.
\par 10 Ama Elazar yerinde durdu; eli yorulup kilica yapisincaya dek Filistliler'i öldürdü. O gün RAB büyük bir zafer sagladi. Israilliler yalniz yere serilenleri yagmalamak üzere Elazar'a döndüler.
\par 11 Üçüncüsü, Hararli Age oglu Samma'ydi. Filistliler Lahay'daki bir mercimek tarlasinin yaninda toplandiklarinda, Israilli askerler onlarin önünden kaçmisti.
\par 12 Ama Samma tarlanin ortasinda durup orayi savunmus, Filistliler'i öldürmüstü. RAB büyük bir zafer saglamisti.
\par 13 Biçme zamani Otuzlar'dan üçü Davut'un yanina, Adullam Magarasi'na gittiler. Bir Filist birligi Refaim Vadisi'nde ordugah kurmustu.
\par 14 Bu sirada Davut hisarda, ikinci Filist birligiyse Beytlehem'deydi.
\par 15 Davut özlemle, "Keske biri Beytlehem'de kapinin yanindaki kuyudan bana su getirse!" dedi.
\par 16 Bu Üçler Filist ordugahinin ortasindan geçerek Beytlehem'de kapinin yanindaki kuyudan su çekip Davut'a getirdiler. Ama Davut içmek istemedi; suyu yere dökerek RAB'be sundu.
\par 17 "Ya RAB, bunu yapmak benden uzak olsun!" dedi, "Canlarini tehlikeye atip giden bu üç kisinin kanini mi içeyim?" Bu yüzden suyu içmek istemedi. Bu üç kisinin yigitligi iste böyleydi.
\par 18 Yoav'in kardesi, Seruya oglu Avisay Üçler'in önderiydi. Mizragini kaldirip üç yüz kisiyi öldürdü. Bu yüzden Üçler kadar ünlendi.
\par 19 Üçler'in en saygin kisisiydi ve onlarin önderi oldu. Ama Üçler'den sayilmadi.
\par 20 Yehoyada oglu Kavseelli Benaya yürekli bir savasçiydi. Büyük isler basardi. Aslan yürekli iki Moavli'yi öldürdü. Ayrica karli bir gün çukura inip bir aslan öldürdü.
\par 21 Iri yari bir Misirli'yi da öldürdü. Misirli'nin elinde mizrak vardi. Benaya sopayla onun üzerine yürüdü. Mizragi elinden kaptigi gibi onu kendi mizragiyla öldürdü.
\par 22 Yehoyada oglu Benaya'nin yaptiklari bunlardir. Bu sayede o da üç yigitler kadar ünlendi.
\par 23 Benaya Otuzlar arasinda saygin bir yer edindiyse de, Üçler'den sayilmadi. Davut onu muhafiz birligi komutanligina atadi.
\par 24 Otuzlar'in arasinda sayilan ötekiler sunlardir: Yoav'in kardesi Asahel, Beytlehemli Dodo oglu Elhanan,
\par 25 Ikisi de Harotlu olan Samma ve Elika,
\par 26 Paletli Heles, Tekoali Ikkes oglu Ira,
\par 27 Anatotlu Aviezer, Husali Mevunnay,
\par 28 Ahohlu Salmon, Netofali Mahray,
\par 29 Netofali Baana oglu Helev, Benyaminogullari'ndan Givali Rivay oglu Ittay,
\par 30 Piratonlu Benaya, Gaas vadilerinden Hidday,
\par 31 Arvali Avialvon, Barhumlu Azmavet,
\par 32 Saalbonlu Elyahba, Yasan'in ogullari ve Yonatan,
\par 33 Hararli Samma, Hararli Sarar oglu Ahiam,
\par 34 Maakali Ahasbay oglu Elifelet, Gilolu Ahitofel oglu Eliam,
\par 35 Karmelli Hesray, Aravli Paaray,
\par 36 Sovali Natan oglu Yigal, Gatli Bani,
\par 37 Ammonlu Selek, Seruya oglu Yoav'in silah tasiyicisi Beerotlu Nahray,
\par 38 Yattirli Ira ve Garev,
\par 39 Hititli* Uriya. Tümü otuz yedi kisiydi.

\chapter{24}

\par 1 RAB Israil halkina yine öfkelendi. Davut'u onlara karsi kiskirtarak, "Git, Israil ve Yahuda halkini say" dedi.
\par 2 Kral, yaninda bulunan ordu komutani Yoav'a su buyrugu verdi: "Dan'dan Beer-Seva'ya dek Israil'in bütün oymaklarina gidip halki sayin ki, halkin sayisini bileyim."
\par 3 Ama Yoav, "RAB Tanrin halkini yüz kat daha çogaltsin, efendim kralim da bunu görsün!" diye karsilik verdi, "Ancak, efendim kralim neden bunu istiyor?"
\par 4 Gelgelelim kralin sözü Yoav'la birlik komutanlarinin sözünden baskin çikti. Böylece kralin yanindan ayrilip Israil'de sayim yapmaya gittiler.
\par 5 Seria Irmagi'ndan geçerek Aroer yakininda, vadinin ortasindaki kentin güneyinde konakladilar. Oradan Gat'i, Yazer'i, Gilat'i, Tahtim-Hodsi topraklarini, Dan-Yaan'i geçip Sayda'ya vardilar.
\par 7 Sonra Sur Kalesi'ne, Hivliler'le Kenanlilar'in bütün kentlerine ugradilar. Sonunda Yahuda ülkesinin Negev bölgesindeki Beer-Seva'ya ulastilar.
\par 8 Dokuz ay yirmi gün ülkeyi bastan basa dolastiktan sonra Yerusalim'e döndüler.
\par 9 Yoav sayimin sonucunu krala bildirdi: Israil'de kiliç kusanabilen sekiz yüz bin, Yahuda'daysa bes yüz bin kisi vardi.
\par 10 Davut sayim yaptiktan sonra kendisini suçlu buldu ve RAB'be, "Bunu yapmakla büyük günah isledim!" dedi, "Ya RAB, lütfen kulunun suçunu bagisla. Çünkü çok akilsizca davrandim."
\par 11 Ertesi sabah Davut uyandiginda, RAB Davut'un bilicisi* Peygamber Gad'a söyle dedi: "Gidip Davut'a de ki, 'RAB söyle diyor: Önüne üç seçenek koyuyorum. Bunlardan birini seç de sana onu yapayim."
\par 13 Gad Davut'a gidip durumu anlatti ve söyle dedi: "Ülkende yedi yil kitlik mi olsun? Yoksa seni kovalayan düsmanlarinin önünden üç ay kaçmak mi istersin? Ya da ülkende üç gün salgin hastalik mi olsun? Beni gönderene ne yanit vereyim, simdi iyice düsün."
\par 14 Davut, "Sikintim büyük" diye yanitladi, "Insan eline düsmektense, RAB'bin eline düselim. Çünkü O'nun acimasi büyüktür."
\par 15 Bunun üzerine RAB o sabahtan belirlenen zamana dek Israil ülkesine salgin hastalik gönderdi. Dan'dan Beer-Seva'ya dek halktan yetmis bin kisi öldü.
\par 16 Melek Yerusalim'i yok etmek için elini uzatinca, RAB gönderecegi yikimdan vazgeçti. Halki yok eden melege, "Yeter artik! Elini çek" dedi. RAB'bin melegi Yevuslu Aravna'nin harman yerinde duruyordu.
\par 17 Davut, halki öldüren melegi görünce, RAB'be, "Günah isleyen benim, ben suç isledim" dedi, "Bu koyunlar ne yapti ki? Ne olur beni ve babamin soyunu cezalandir."
\par 18 O gün Gad Davut'a gitti. Ona, "Gidip Yevuslu Aravna'nin harman yerinde RAB'be bir sunak kur" dedi.
\par 19 Davut Gad'in sözü uyarinca RAB'bin buyurdugu gibi gitti.
\par 20 Aravna bakinca kralla görevlilerinin kendisine dogru yaklastiklarini gördü. Varip kralin önünde yüzüstü yere kapandi.
\par 21 Sonra, "Efendim kral niçin kulunun yanina geldi?" diye sordu. Davut, "RAB'be bir sunak kurmak üzere harman yerini senden satin almak için" diye yanitladi, "Öyle ki, salgin hastalik halkin üzerinden kalksin."
\par 22 Aravna, "Efendim kral uygun gördügünü alip RAB'be sunsun" dedi, "Iste yakmalik sunu* için öküzler ve odun için dövenlerle öküzlerin takimlari!
\par 23 Ey kral, Aravna bütün bunlari sana veriyor." Sonra ekledi: "RAB Tanrin senden hosnut olsun!"
\par 24 Ne var ki kral, "Olmaz!" dedi, "Senden malini kesinlikle bir ücret karsiliginda satin alacagim. Çünkü Tanrim RAB'be karsiligini ödemeden yakmalik sunular sunmam." Böylece Davut harman yerini ve öküzleri elli sekel gümüs karsiliginda satin aldi.
\par 25 Davut orada RAB'be bir sunak kurup yakmalik sunulari* ve esenlik sunularini* sundu. RAB de ülkeyle ilgili yakariyi yanitladi ve salgin hastalik Israil'den kaldirildi.


\end{document}