\begin{document}

\title{Ezra}


\chapter{1}

\par 1 Pers Krali Kores'in kralliginin birinci yilinda RAB, Yeremya araciligiyla bildirdigi sözünü yerine getirmek amaciyla, Pers Krali Kores'i harekete geçirdi. Kores yönetimi altindaki bütün halklara su yazili bildiriyi duyurdu:
\par 2 "Pers Krali Kores söyle diyor: `Göklerin Tanrisi RAB yeryüzünün bütün kralliklarini bana verdi. Beni Yahuda'daki Yerusalim Kenti'nde kendisi için bir tapinak yapmakla görevlendirdi.
\par 3 Aranizda O'nun halkindan kim varsa Tanrisi onunla olsun. Yahuda'daki Yerusalim Kenti'ne gidip Israil'in Tanrisi RAB'bin, Yerusalim'deki Tanri'nin Tapinagi'ni yeniden yapsinlar.
\par 4 Kralligimda yasayan yerliler, sürgün olduklari yerlerde sag kalmis olanlara altin, gümüs, mal ve hayvanlar saglamakla birlikte Yerusalim'deki Tanri'nin Tapinagi'na gönülden sunular sunsun.'"
\par 5 Böylece Yahuda ve Benyamin oymaklarinin boy baslari, kâhinler*, Levililer ve ruhlari Tanri tarafindan harekete geçirilen herkes, RAB'bin Yerusalim'deki Tapinagi'ni yeniden yapmak için gidis hazirliklarina giristi.
\par 6 Komsulari gönülden verdikleri armaganlarin yanisira, altin, gümüs kaplar, mal, hayvan ve degerli armaganlarla onlari desteklediler.
\par 7 Pers Krali Kores de Nebukadnessar'in Yerusalim'deki RAB'bin Tapinagi'ndan alip kendi ilahinin tapinagina koymus oldugu kaplari çikardi. Bunlari hazine görevlisi Mitredat'a getirterek sayimini yaptirdi ve Yahuda önderi Sesbassar'a verdi.
\par 9 Sayim sonucu suydu: 30 altin legen, 1 000 gümüs legen, 29 tas,
\par 10 30 altin tas ve birbirinin benzeri 410 gümüs tas, 1 000 parça degisik kap.
\par 11 Altin ve gümüs esyalarin toplami 5 400 parçaydi. Sürgünler Babil'den Yerusalim'e dönerken Sesbassar bunlarin hepsini birlikte getirdi.

\chapter{2}

\par 1 Babil Krali Nebukadnessar'in Babil'e sürgün ettigi insanlar yasadiklari ilden Yerusalim ve Yahuda'daki kendi kentlerine döndü.
\par 2 Bunlar Zerubbabil, Yesu, Nehemya, Seraya, Reelaya, Mordekay, Bilsan, Mispar, Bigvay, Rehum ve Baana'nin önderliginde geldiler.
\par 3 Parosogullari: 2172
\par 4 Sefatyaogullari: 372
\par 5 Arahogullari: 775
\par 6 Yesu ve Yoav soyundan Pahat-Moavogullari: 2812
\par 7 Elamogullari: 1254
\par 8 Zattuogullari: 945
\par 9 Zakkayogullari: 760
\par 10 Baniogullari: 642
\par 11 Bevayogullari: 623
\par 12 Azgatogullari: 1222
\par 13 Adonikamogullari: 666
\par 14 Bigvayogullari: 2056
\par 15 Adinogullari: 454
\par 16 Hizkiya soyundan Aterogullari: 98
\par 17 Besayogullari: 323
\par 18 Yoraogullari: 112
\par 19 Hasumogullari: 223
\par 20 Gibbarogullari: 95
\par 21 Beytlehemliler: 123
\par 22 Netofalilar: 56
\par 23 Anatotlular: 128
\par 24 Azmavetliler: 42
\par 25 Kiryat-Yearimliler, Kefiralilar ve Beerotlular: 743
\par 26 Ramalilar ve Gevalilar: 621
\par 27 Mikmaslilar: 122
\par 28 Beytel ve Ay kentlerinden olanlar: 223
\par 29 Nevolular: 52
\par 30 Magbisliler: 156
\par 31 Öbür Elam Kenti'nden olanlar: 1254
\par 32 Harimliler: 320
\par 33 Lod, Hadit ve Ono kentlerinden olanlar: 725
\par 34 Erihalilar: 345
\par 35 Senaalilar: 3630.
\par 36 Kâhinler: Yesu soyundan Yedayaogullari: 973
\par 37 Immerogullari: 1052
\par 38 Pashurogullari: 1247
\par 39 Harimogullari: 1017.
\par 40 Levililer: Hodavya soyundan Yesu ve Kadmielogullari: 74.
\par 41 Ezgiciler: Asafogullari: 128.
\par 42 Tapinak kapi nöbetçileri: Sallumogullari, Aterogullari, Talmonogullari, Akkuvogullari, Hatitaogullari, Sovayogullari, toplam 139.
\par 43 Tapinak görevlileri: Sihaogullari, Hasufaogullari, Tabbaotogullari,
\par 44 Kerosogullari, Siahaogullari, Padonogullari,
\par 45 Levanaogullari, Hagavaogullari, Akkuvogullari,
\par 46 Hagavogullari, Salmayogullari, Hananogullari,
\par 47 Giddelogullari, Gaharogullari, Reayaogullari,
\par 48 Resinogullari, Nekodaogullari, Gazzamogullari,
\par 49 Uzzaogullari, Paseahogullari, Besayogullari,
\par 50 Asnaogullari, Meunimogullari, Nefusimogullari,
\par 51 Bakbukogullari, Hakufaogullari, Harhurogullari,
\par 52 Baslutogullari, Mehidaogullari, Harsaogullari,
\par 53 Barkosogullari, Siseraogullari, Temahogullari,
\par 54 Nesiahogullari, Hatifaogullari.
\par 55 Süleyman'in kullarinin soyu: Sotayogullari, Hassoferetogullari, Perudaogullari,
\par 56 Yalaogullari, Darkonogullari, Giddelogullari,
\par 57 Sefatyaogullari, Hattilogullari, Pokeret-Hassevayimogullari, Amiogullari.
\par 58 Tapinak görevlileriyle Süleyman'in kullarinin soyundan olanlar: 392.
\par 59 Tel-Melah, Tel-Harsa, Keruv, Addan ve Immer'den dönen, ancak hangi aileden olduklarini ve soylarinin Israil'den geldigini kanitlayamayanlar sunlardir:
\par 60 Delayaogullari, Toviyaogullari, Nekodaogullari: 652.
\par 61 Kâhinlerin soyundan: Hovayaogullari, Hakkosogullari ve Gilatli Barzillay'in kizlarindan biriyle evlenip kayinbabasinin adini alan Barzillay'in ogullari.
\par 62 Bunlar soy kütüklerini aradilar. Ama yazili bir kayit bulamayinca, kâhinlik görevi ellerinden alindi.
\par 63 Vali, Urim ile Tummim'i* kullanan bir kâhin çikincaya dek en kutsal yiyeceklerden yememelerini buyurdu.
\par 64 Bütün halk toplam 42 360 kisiydi.
\par 65 Ayrica 7 337 erkek ve kadin köle, kadinli erkekli 200 ezgici, 736 at, 245 katir, 435 deve, 6 720 esek vardi.
\par 68 Bazi aile baslari Yerusalim'deki RAB Tanri'nin Tapinagi'na varinca, tapinagin bulundugu yerde yeniden kurulmasi için gönülden armaganlar verdiler.
\par 69 Her biri gücü oraninda hazineye bu is için toplam 61 000 darik altin, 5 000 mina gümüs, 100 kâhin mintani bagisladi.
\par 70 Kâhinler, Levililer, halktan bazi kisiler -ezgiciler, tapinak görevlileri ve kapi nöbetçileri- kendi kentlerine yerlestiler. Böylece bütün Israilliler kentlerinde yasamaya basladilar.

\chapter{3}

\par 1 Israilliler kendi kentlerine yerlestikten sonra, yedinci ay* Yerusalim'de tek vücut halinde toplandilar.
\par 2 Yosadak oglu Yesu ve kâhin olan kardesleri, Sealtiel oglu Zerubbabil'le kardesleri Israil'in Tanrisi'nin sunagini yeniden kurdular. Amaçlari, Tanri adami Musa'nin yasasi uyarinca, sunagin üzerinde yakmalik sunular* sunmakti.
\par 3 Çevrelerinde yasayan halklardan korkmalarina karsin, sunagi eski temeli üzerine yeniden kurdular. Üzerinde RAB'be sabah, aksam öngörülen yakmalik sunulari sundular.
\par 4 Sonra yazilanlara uygun biçimde Çardak Bayrami'ni* kutladilar. Kural uyarinca, her gün için belirlenen sayiya göre, yakmalik sunulari sundular.
\par 5 Bundan sonra da günlük yakmalik sunuyu, Yeni Ay sunularini, RAB'bin belirledigi bütün kutsal bayramlarin sunularini ve RAB'be gönülden verilen sunulari sundular.
\par 6 RAB'bin Tapinagi'nin temeli henüz atilmadigi halde, yedinci ayin birinci günü RAB'be yakmalik sunular sunmaya basladilar.
\par 7 Israilliler tasçilarla marangozlara para ödediler. Ayrica sedir tomruklarini Lübnan'dan denize indirerek Yafa'ya getirmeleri için Saydalilar'a ve Surlular'a yiyecek, içecek, zeytinyagi sagladilar. Bütün bunlara Pers Krali Kores izin vermisti.
\par 8 Tanri'nin Yerusalim'deki Tapinagi'na vardiktan sonra, ikinci yilin ikinci ayinda* Sealtiel oglu Zerubbabil, Yosadak oglu Yesu ve bütün öteki kardesleri, kâhinler, Levililer ve sürgünden Yerusalim'e dönenlerin tümü ise basladilar. RAB'bin Tapinagi'nin yapimini denetlemek için yirmi ve daha yukari yastaki Levililer'i görevlendirdiler.
\par 9 Levililer'den Yesu, ogullari ve kardesleri, Yahuda soyundan Kadmiel ile ogullari, Henadat'in ogullariyla torunlari Tanri'nin Tapinagi'nin yapiminda çalisanlari denetleme isini hep birlikte yüklendiler.
\par 10 Yapicilar RAB'bin Tapinagi'nin temelini atinca, Israil Krali Davut'un kurali uyarinca kâhinler RAB'bi övmek için tören giysilerini giymis olarak ellerinde borazanlarla, Levili Asafogullari da zillerle yerlerini aldilar.
\par 11 RAB'be övgüler, sükranlar sunarak ezgi okudular: "RAB iyidir; Israil'e sevgisi sonsuzdur." RAB'bin Tapinagi'nin temeli atildigi için herkes yüksek sesle RAB'bi övmeye basladi.
\par 12 Eski tapinagi görmüs birçok yasli kâhin, Levili ve boy basi tapinagin temelinin atildigini görünce hiçkira hiçkira agladilar. Birçoklari da sevinç çigliklari atti.
\par 13 Sevinç çigliklari aglama sesinden ayirt edilemiyordu. Çünkü halk avaz avaz bagiriyordu. Ses uzak yerlerden bile duyuluyordu.

\chapter{4}

\par 1 Yahudalilar'la Benyaminliler'in düsmanlari, sürgünden dönenlerin Israil'in Tanrisi RAB için bir tapinak yaptigini duyunca,
\par 2 Zerubbabil'in ve boy baslarinin yanina vardilar. "Tapinagi sizinle birlikte kuralim" dediler, "Çünkü biz de, sizin gibi Tanriniz'a tapiyoruz; bizi buraya getiren Asur Krali Esarhaddon'un döneminden bu yana sizin Tanriniz'a kurban sunuyoruz."
\par 3 Ne var ki Zerubbabil, Yesu ve Israil'in öteki boy baslari, "Tanrimiz'a bir tapinak kurmak size düsmez" diye karsilik verdiler, "Pers Krali Kores'in buyrugu uyarinca, Israil'in Tanrisi RAB için tapinagi yalniz biz kuracagiz."
\par 4 Bunun üzerine çevre halki Yahudalilar'i tapinagin yapimindan caydirmak için korkutmaya, cesaretlerini kirmaya giristi.
\par 5 Tasarilarina engel olmak için Pers Krali Kores'in döneminden Pers Krali Darius'un kralligina dek rüsvetle danismanlar tuttular.
\par 6 Ahasveros'un kralliginin baslangicinda, Yahudalilar'in düsmanlari Yahuda ve Yerusalim'de yasayanlari suçlayan bir belge düzenlediler.
\par 7 Pers Krali Artahsasta'nin kralligi döneminde, Bislam, Mitredat, Taveel ve öbür çalisma arkadaslari Artahsasta'ya bir mektup yazdilar. Mektup Aramice yazilip çevrildi.
\par 8 Vali Rehum ile Yazman Simsay Kral Artahsasta'ya Yerusalim'i suçlayan bir mektup yazdilar. Mektup söyleydi:
\par 9 "Vali Rehum, Yazman Simsay ve öbür çalisma arkadaslari, yargiçlar, yöneticiler, görevliler, Persler, Erekliler, Babilliler, Elam topraklarindan gelen Sus halki,
\par 10 büyük ve onurlu Asurbanipal'in sürüp Samiriye Kenti'yle Firat'in bati yakasindaki bölgeye yerlestirdigi öbür halklarindan."
\par 11 Iste Kral Artahsasta'ya gönderilen mektubun örnegi: "Kral Artahsasta'ya, "Firat'in bati yakasindaki bölgede yasayan kullarindan:
\par 12 "Yönetimindeki öbür bölgelerden çikip bize gelen Yahudiler Yerusalim'e yerleserek o asi ve kötü kenti yeniden kurmaya basladilar. Bunu bilgine sunuyoruz. Temelini pekistiriyor, surlarini tamamliyorlar.
\par 13 Ey kral, bilmelisin ki, bu kent yeniden kurulur, surlari tamamlanirsa, Yahudiler yine vergi ödemeyecek; kralliginin geliri de azalacak.
\par 14 Biz sarayinin ekmegini yedik. Sana zarar gelmesine izin veremeyiz. Bunun için, haberin olsun diye bu mektubu gönderiyoruz.
\par 15 Atalarinin belgeleri arastirilsin. Kayitlarda bu kentin asi, krallara, valilere zarar veren bir kent oldugunu göreceksin. Bu kent öteden beri baskaldiran bir kenttir. Yerle bir edilmesinin nedeni de budur.
\par 16 Bu yüzden, ey kral, sana bildiriyoruz: Bu kent yeniden kurulur, surlari tamamlanirsa, Firat'in bati yakasindaki bölgede hiçbir payin kalmayacak."
\par 17 Kral su yaniti gönderdi: "Samiriye'de ve Firat'in bati yakasindaki öbür yerlerde yasayan Vali Rehum'a, Yazman Simsay'a ve öbür çalisma arkadaslarina selamlar.
\par 18 "Bize gönderdiginiz mektup çevrilip bana okundu.
\par 19 Buyrugum üzerine arastirma yapildi. Bu kentin öteden beri krallara baskaldirdigi, isyan ettigi, ayaklandigi saptandi.
\par 20 Yerusalim'i güçlü krallar yönetti. Firat'in bati yakasindaki bütün bölgede egemenlik sürdüler. Oradaki halktan vergi topladilar.
\par 21 Simdi isi durdurmalari için bu adamlara bir buyruk çikarin. Öyle ki, ben buyruk vermedikçe kent yeniden kurulmasin.
\par 22 Bu konuya özen gösterin; kralligima daha fazla zarar gelmesin."
\par 23 Kral Artahsasta'nin mektubunun örnegi kendilerine okunur okunmaz, Rehum, Yazman Simsay ve öbür çalisma arkadaslari hemen Yerusalim'e Yahudiler'in yanina gittiler ve zorla onlari durdurdular.
\par 24 Böylece Tanri'nin Yerusalim'deki Tapinagi'nin yapimi, Pers Krali Darius'un kralliginin ikinci yilina dek askida kaldi.

\chapter{5}

\par 1 O sirada Peygamber Hagay ile Iddo oglu Peygamber Zekeriya, Yahuda ve Yerusalim'deki Yahudiler'e Israil Tanrisi'nin adiyla peygamberlikte bulundular.
\par 2 Bunun üzerine Sealtiel oglu Zerubbabil ile Yosadak oglu Yesu Tanri'nin Yerusalim'deki Tapinagi'ni yeniden kurmaya giristiler. Tanri'nin peygamberleri de onlarla birlikteydi ve onlara yardim ediyordu.
\par 3 Firat'in bati yakasindaki bölgenin valisi Tattenay, Setar-Bozenay ve çalisma arkadaslari onlara gidip, "Bu tapinagi yeniden kurmak ve yapimini tamamlamak için size kim yetki verdi?" diye sordular.
\par 4 Yapiyi kuranlarin adlarini da sordular.
\par 5 Ama Tanrilari Yahudi ileri gelenlerini koruyordu. Bu yüzden, Darius'a bir haber gönderilip ondan yazili bir yanit alinincaya dek Yahudiler durdurulmadi.
\par 6 Firat'in bati yakasindaki bölgenin valisi Tattenay, Setar-Bozenay, çalisma arkadaslari ve oradaki görevlilerin Kral Darius'a gönderdikleri mektubun örnegi asagidadir.
\par 7 Gönderdikleri belge söyleydi: "Kral Darius'a candan selamlar.
\par 8 Yahuda Ili'ne, yüce Tanri'nin Tapinagi'na gittigimizi krala bildiririz. Tapinagi büyük taslarla kuruyor, duvarlarina kirisler yerlestiriyorlar. Tapinagin yapiminda canla basla çalisiliyor ve yapim isi basariyla ilerliyor.
\par 9 "Halkin ileri gelenlerine, `Bu tapinagi yeniden kurmak ve yapimini tamamlamak için size kim yetki verdi?' diye sorduk.
\par 10 Bilgine sunmak üzere önderlerinin kim olduklarini sana yazabilmek için adlarini da sorduk.
\par 11 "Iste bize verdikleri yanit: "`Biz yerin ve gögün Tanrisi'nin kullariyiz. Uzun yillar önce Israil'in büyük bir kralinin kurup yapimini tamamladigi tapinagi yeniden kuruyoruz.
\par 12 Ama atalarimiz Göklerin Tanrisi'ni öfkelendirdi. Bu yüzden Tanri onlari Babil Krali Kildani* Nebukadnessar'in eline teslim etti. Nebukadnessar tapinagi yikti, halki da Babil'e sürdü.
\par 13 Ama Babil Krali Kores, kralliginin birinci yilinda Tanri'nin Tapinagi'nin yeniden yapilmasi için buyruk verdi.
\par 14 Ayrica Nebukadnessar'in Tanri'nin Yerusalim'deki Tapinagi'ndan çikarip Babil'deki tapinaga götürdügü altin ve gümüs kaplari da oradan çikararak vali atadigi Sesbassar adli kisiye verdi.
\par 15 Kores ona, Bu kaplari al, gidip Yerusalim'deki tapinaga yerlestir dedi, Tanri'nin Tapinagi eski yerinde yeniden kur.
\par 16 Böylece Sesbassar gelip Tanri'nin Yerusalim'deki Tapinagi'nin temelini atti. O günden bu yana yapim isleri sürmekte, ama daha bitmedi.'
\par 17 "Kral uygun görüyorsa, Babil Sarayi'ndaki arsivde bir arastirma yapilsin. Tanri'nin Yerusalim'deki Tapinagi'nin yeniden kurulmasi için Kral Kores'in bir buyruk verip vermedigi saptansin. Sonra kral bu konuya iliskin kararini bize bildirsin."

\chapter{6}

\par 1 Kral Darius'un buyrugu uyarinca, Babil'de kayitlarin saklandigi odada arastirma yapildi.
\par 2 Medya Ili'ndeki Ahmeta Kalesi'nde bir tomar bulundu. Tomarin içinde sunlar yaziliydi:
\par 3 "Kral Kores, kralliginin birinci yilinda, Tanri'nin Yerusalim'deki Tapinagi'na iliskin söyle buyruk verdi: `Kurban kesmek üzere bu tapinagin yeniden kurulmasi için temel atilsin. Üç sira büyük tas, bir sira kiris dösensin. Yüksekligi ve genisligi altmisar arsin olsun. Giderler saraydan karsilansin.
\par 5 Nebukadnessar'in Yerusalim'deki Tanri'nin Tapinagi'ndan çikarip Babil'e getirdigi altin ve gümüs kaplar da geri verilsin. Yerusalim'deki tapinakta özel yerlerine götürülsün. Hepsi Tanri'nin Tapinagi'na konsun.'"
\par 6 "Onun için siz, Firat'in bati yakasindaki bölge valisi Tattenay, Setar-Bozenay, çalisma arkadaslari ve oranin görevlileri, tapinaktan uzak durun!
\par 7 Tanri'nin Tapinagi'nin yapim islerine karismayin. Birakin, Yahudiler'in valisiyle ileri gelenleri Tanri'nin Tapinagi'ni eski yerinde yeniden kursunlar.
\par 8 Ayrica Tanri'nin Tapinagi'nin yeniden kurulmasi için Yahudi ileri gelenlerine neler yapmaniz gerektigini de size buyuruyorum: Bu adamlarin bütün giderleri kralin hazinesinden, Firat'in bati yakasindaki bölgeden toplanan vergilerden ödensin. Öyle ki, yapim isleri aksamasin.
\par 9 Yerusalim'deki kâhinlerin bütün gereksinimlerini hiç aksatmadan her gün vereceksiniz: Göklerin Tanrisi'na sunulacak yakmalik sunular* için genç bogalar, koçlar, kuzular ve bugday, tuz, sarap, zeytinyagi.
\par 10 Öyle ki, Göklerin Tanrisi'ni hosnut eden sunular sunsunlar, ben ve ogullarim sag kalalim diye dua etsinler.
\par 11 Bundan baska buyuruyorum ki, bu karari degistirenin evinden bir kiris çikarilacak ve o kisi kirisin üzerine asilacak. Evi de bu suçtan ötürü çöplüge çevrilecek.
\par 12 Yerusalim'i adina konut seçen Tanri, bu karari degistirmeye, oradaki Tanri Tapinagi'ni yikmaya kalkisan her krali ve ulusu yok etsin. Ben Darius böyle buyurdum. Buyrugum özenle yerine getirilsin."
\par 13 Firat'in bati yakasindaki bölge valisi Tattenay, Setar-Bozenay ve çalisma arkadaslari Kral Darius'un buyrugunu özenle yerine getirdiler.
\par 14 Peygamber Hagay ile Iddo oglu Zekeriya'nin yaptiklari peygamberlik sayesinde Yahudi ileri gelenleri yapim islerini basariyla ilerlettiler. Israil Tanrisi'nin buyrugu ve Pers krallari Kores'in, Darius'un, Artahsasta'nin buyruklari uyarinca tapinagin yapimini bitirdiler.
\par 15 Tapinak Kral Darius'un kralliginin altinci yili, Adar ayinin* üçüncü günü tamamlandi.
\par 16 Israil halki -kâhinler, Levililer ve sürgünden dönenlerin tümü- Tanri'nin Tapinagi'nin adanmasini sevinçle kutladilar.
\par 17 Tanri'nin Tapinagi'nin adanmasi için yüz boga, iki yüz koç, dört yüz kuzu kurban ettiler. Oymaklarin sayisina göre, bütün Israilliler için günah sunusu* olarak on iki teke sundular.
\par 18 Musa'nin Kitabi'ndaki kural uyarinca Tanri'nin Yerusalim'deki hizmeti için kâhinlerle Levililer'i bagli olduklari bölüklere göre görevlere atadilar.
\par 19 Sürgünden dönenler birinci ayin* on dördüncü günü Fisih Bayrami'ni* kutladilar.
\par 20 Kâhinlerle Levililer hep birlikte kendilerini arindirdilar; dinsel açidan hepsi arindi. Levililer sürgünden dönenlerin tümü, kardesleri olan kâhinler ve kendileri için Fisih* kurbanini kestiler.
\par 21 Sürgünden dönen Israilliler'le Israil'in Tanrisi RAB'be yönelmek amaciyla kendilerini çevredeki halklarin kötü aliskanliklarindan ayirip onlara katilanlarin hepsi kurban etinden yedi.
\par 22 Mayasiz Ekmek Bayrami'ni* yedi gün sevinçle kutladilar. Çünkü RAB onlari sevindirdi ve Asur Krali'nin onlara iyi davranmasini, Israil'in Tanrisi Tanri'nin Tapinagi'nin yapim islerinde onlara yardim etmesini sagladi.

\chapter{7}

\par 1 Bu olaylardan sonra, Pers Krali Artahsasta'nin kralligi döneminde, Baskâhin Harun oglu Elazar oglu Pinehas oglu Avisua oglu Bukki oglu Uzzi oglu Zerahya oglu Merayot oglu Azarya oglu Amarya oglu Ahituv oglu Sadok oglu Sallum oglu Hilkiya oglu Azarya oglu Seraya oglu Ezra adinda biri Babil'den geldi. Ezra Israil'in Tanrisi RAB'bin Musa'ya verdigi yasayi iyi bilen bir bilgindi. Tanrisi RAB'bin yardimiyla kral ona her istedigini verdi.
\par 7 Kral Artahsasta'nin kralliginin yedinci yilinda Israil halkindan, kâhinlerden, Levililer'den, ezgicilerden, tapinak görevlilerinden ve kapi nöbetçilerinden bazilari Yerusalim'e gitti.
\par 8 Ezra, Artahsasta'nin kralliginin yedinci yilinin besinci ayinda* Yerusalim'e vardi.
\par 9 Birinci ayin birinci günü Babil'den ayrilmisti. Tanrisi'nin koruyucu eli sayesinde besinci ayin birinci günü Yerusalim'e vardi.
\par 10 Ezra kendini RAB'bin Yasasi'ni inceleyip uygulamaya ve Israil'de kurallari, ilkeleri ögretmeye adamisti.
\par 11 Kral Artahsasta'nin RAB'bin buyruklarini, Israil için koydugu kurallari iyi bilen Kâhin ve Bilgin Ezra'ya verdigi mektubun bir örnegi sudur:
\par 12 "Krallarin Krali Artahsasta'dan Gökler Tanrisi'nin Yasasi'nin bilgini Kâhin Ezra'ya selamlar!
\par 13 "Kralligimda yasayan Israil halkindan, kâhinlerden ve Levililer'den Yerusalim'e gitmek isteyen herkesin seninle gidebilmesi için buyruk veriyorum.
\par 14 Elindeki Tanrin'in Yasasi'nin uygulanip uygulanmadigi konusunda Yahuda ve Yerusalim'de arastirma yapman için, ben ve yedi danismanim seni görevlendirdik.
\par 15 Benim ve danismanlarimin Yerusalim'de konut kuran Israil'in Tanrisi'na gönülden verdigimiz altini, gümüsü birlikte götürmelisin.
\par 16 Babil Ili'nden elde edecegin altinin, gümüsün tümünü, halkin ve kâhinlerin Tanrilari'nin Yerusalim'deki Tapinagi'na gönülden verdikleri armaganlari da alip götürmelisin.
\par 17 "Bu parayla hemen bogalar, koçlar, kuzular, tahil sunulari* ve dökmelik sunular satin alacak ve hepsini Tanrin'in Yerusalim'deki Tapinagi'nin sunagi üzerinde sunacaksin.
\par 18 Sen ve kardeslerin artan altini, gümüsü Tanriniz'in istegi uyarinca dilediginiz gibi kullanin.
\par 19 Tanrin'in Tapinagi'nin hizmetinde kullanilmak üzere sana verilen kaplarin hepsini Yerusalim'in Tanrisi'na sun.
\par 20 Tanrin'in Tapinagi için ödemen gereken baska bir sey varsa, giderleri kralin hazinesinden karsilarsin.
\par 21 "Ben, Kral Artahsasta, Firat'in bati yakasindaki bölgenin bütün hazine görevlilerine buyruk veriyorum: Gökler Tanrisi'nin Yasasi'nin bilgini Kâhin Ezra'nin sizden her istedigini özenle yerine getirin.
\par 22 Kendisine gerektiginde yüz talanta*fj* kadar gümüs, yüz kor bugday, yüz bat sarap, yüz bat zeytinyagi ve istedigi kadar tuz saglayin.
\par 23 Göklerin Tanrisi kendi tapinagi için ne buyurursa, özenle yerine getirin. Öyle ki, bana ve ogullarima öfkelenmesin.
\par 24 Sunu da bilesiniz ki, kâhinlerden, Levililer'den, ezgicilerden, tapinak görevlilerinden ve kapi nöbetçilerinden, Tanri'nin Tapinagi'nin öbür hizmetkârlarindan vergi almaya yetkiniz yoktur.
\par 25 "Sana gelince, Ezra, sendeki Tanri bilgeligi uyarinca yargiçlar ata. Bunlar Firat'in bati yakasindaki bölgede yasayan bütün halka, Tanrin'in yasalarini bilenlerin hepsine adalet saglasinlar. Yasayi bilmeyenlere de siz ögreteceksiniz.
\par 26 Tanrin'in Yasasi'na ve kralin buyruklarina uymayanlar ya ölümle, ya sürgünle, ya mallarina el konarak, ya da hapsedilerek cezalandirilsin."
\par 27 Atalarimizin Tanrisi RAB'be övgüler olsun! Kralin yüregine Yerusalim'deki tapinagini onurlandirma istegini koydu.
\par 28 Kralin, danismanlarinin, güçlü komutanlarinin bana iyi davranmalarini sagladi. Tanrim RAB'bin eli üzerimde oldugundan yüreklendim. Israil ileri gelenlerinin bazilarini toplayip benimle Yerusalim'e dönmelerini sagladim.

\chapter{8}

\par 1 Artahsasta'nin kralligi döneminde ben Ezra'yla birlikte Babil'den dönen boy baslarinin ve onlarla birlikte kayitli olanlarin listesi:
\par 2 Pinehasogullari'ndan Gersom, Itamarogullari'ndan Daniel, Davutogullari'ndan Hattus.
\par 3 Sekanyaogullari'ndan, Parosogullari'ndan Zekeriya ve onunla birlikte bu boydan kaydedilen 150 erkek.
\par 4 Pahat-Moavogullari'ndan Zerahya oglu Elyehoenay ve onunla birlikte 200 erkek.
\par 5 Yahaziel oglu Sekanya'nin ogullarindan 300 erkek.
\par 6 Adinogullari'ndan Yonatan oglu Ebet ve onunla birlikte 50 erkek.
\par 7 Elamogullari'ndan Atalya oglu Yesaya ve onunla birlikte 70 erkek.
\par 8 Sefatyaogullari'ndan Mikael oglu Zevadya ve onunla birlikte 80 erkek.
\par 9 Yoavogullari'ndan Yehiel oglu Ovadya ve onunla birlikte 218 erkek.
\par 10 Yosifya oglu Selomit'in ogullarindan 160 erkek.
\par 11 Bevayogullari'ndan Bevay oglu Zekeriya ve onunla birlikte 28 erkek.
\par 12 Azgatogullari'ndan Hakkatan oglu Yohanan ve onunla birlikte 110 erkek.
\par 13 Adonikam'in küçük ogullarindan adlari Elifelet, Yeiel, Semaya olanlar ve onlarla birlikte 60 erkek.
\par 14 Bigvayogullari'ndan Utay, Zakkur ve onlarla birlikte 70 erkek.
\par 15 Onlari Ahava Kenti'ne dogru uzanan kanalin yanina topladim. Orada üç gün konakladik. Halkin ve kâhinlerin arasinda yoklama yaptigimda orada Levililer'den kimse olmadigini gördüm.
\par 16 Bunun üzerine Eliezer, Ariel, Semaya, Elnatan, Yariv, Elnatan, Natan, Zekeriya, Mesullam adindaki önderleri, Ögretmen Yoyariv'i ve Elnatan'i çagirttim.
\par 17 Sonra onlari Kasifya'da bulunan Önder Iddo'ya gönderdim. Iddo'ya ve tapinak görevlisi olan kardeslerine neler söylemeleri gerektigini bildirdim. Öyle ki, bize Tanrimiz'in Tapinagi'nda görev yapacak adamlar göndersinler.
\par 18 Tanrimiz'in iyiligi sayesinde Israil oglu Levi oglu Mahliogullari'ndan Serevya adinda bilge bir kisiyi bize gönderdiler. Kendisiyle birlikte ogullari ve kardesleri toplam on sekiz kisi geldi.
\par 19 Hasavya'yi, Merariogullari'ndan Yesaya'yi ve kardesleriyle ogullarini, toplam yirmi kisiyi de gönderdiler.
\par 20 Ayrica Levililer'e yardim etmek üzere Davut'la görevlilerinin atadigi tapinak görevlilerinden iki yüz yirmi kisi gönderdiler. Hepsinin adi listeye yazilmisti.
\par 21 Tanrimiz'in önünde alçakgönüllü davranmak, O'ndan kendimiz, çocuklarimiz, mallarimiz için güvenli bir yolculuk dilemek üzere orada, Ahava Kanali yaninda oruç* ilan ettim.
\par 22 Yolculugumuz sirasinda herhangi bir düsmandan bizi korumalari için, kraldan asker ve atli istemeye utaniyordum. Çünkü krala, "Tanrimiz kendisine yönelenlerin hepsine iyilik eder, ama kizgin öfkesi kendisini birakanlarin üzerindedir" demistik.
\par 23 Oruç tuttuk ve bu konuda Tanrimiz'a yakardik. O da yakarisimizi yanitladi.
\par 24 Serevya, Hasavya ve kardeslerinden on kisiyle birlikte on iki önde gelen kâhin seçtim.
\par 25 Kralin, danismanlarinin, komutanlarinin ve orada bulunan Israilliler'in Tanrimiz'in Tapinagi'na bagisladigi altini, gümüsü, kaplari tartip onlara verdim.
\par 26 Tartip verdiklerim sunlardir: 650 talant gümüs, 100 talant gümüs kap, 100 talant altin,
\par 27 bin dariklik yirmi altin tas ve altin kadar degerli, kaliteli, parlak tunçtan* iki kap.
\par 28 Onlara, "Siz RAB için kutsalsiniz, bu kaplar da öyle" dedim, "Altin ve gümüs, atalarinizin Tanrisi RAB'be gönülden sunulan sunudur.
\par 29 RAB'bin Yerusalim'deki Tapinagi'nin odalarinda, önde gelen kâhinlerin, Levililer'in, Israil'in boy baslarinin önünde tartincaya dek bunlari iyi koruyun."
\par 30 Böylece kâhinlerle Levililer Yerusalim'e, Tanrimiz'in Tapinagi'na götürülmek için tartilan altini, gümüsü, kaplari aldilar.
\par 31 Birinci ayin* on ikinci günü Yerusalim'e gitmek üzere Ahava Kanali'ndan ayrildik. Tanrimiz'in eli üzerimizdeydi; yol boyunca düsmandan, pusuya yatanlarin saldirisindan bizi korudu.
\par 32 Sonunda Yerusalim'e vardik. Orada üç gün kaldik.
\par 33 Dördüncü gün, Tanrimiz'in Tapinagi'na gidip altini, gümüsü, kaplari tarttik ve Uriya oglu Kâhin Meremot'a verdik. Pinehas oglu Elazar, Levili Yesu oglu Yozavat ve Binnuy oglu Noadya da onunla birlikteydi.
\par 34 Her sey sayildi, tartildi; tartilanlarin tümü aninda kayda geçirildi.
\par 35 Sürgünden dönenler Israil'in Tanrisi'na yakmalik sunular* sundular: Bütün Israil için on iki boga, doksan alti koç, yetmis yedi kuzu ve günah sunusu* olarak on iki teke. Bütün bunlar RAB'be yakmalik sunu olarak sunuldu.
\par 36 Ayrica kralin buyruklarini içeren belgeyi kralin satraplarina* ve Firat'in bati yakasindaki valilere verdiler. Bunlar Israil halkina ve Tanri'nin Tapinagi'na yardim etmislerdi.

\chapter{9}

\par 1 Bütün bunlardan sonra, önderler yanima gelerek söyle dediler: "Israil halki, kâhinlerle Levililer dahil, çevredeki halklarin -Kenanlilar'in, Hititler'in*, Perizliler'in, Yevuslular'in, Ammonlular'in, Moavlilar'in, Misirlilar'in, Amorlular'in- igrenç aliskanliklarindan kendilerini ayri tutmadi.
\par 2 Kendilerine ve ogullarina bu halklardan kiz aldilar. Böylece kutsal soy çevredeki halklarla karisti. Önderlerle görevliler bu hainlikte öncülük etti."
\par 3 Bunu duyunca giysimi ve cüppemi yirttim, saçimi sakalimi yoldum, dehset içinde oturakaldim.
\par 4 Sürgünden dönenlerin bu hainliginden ötürü Israil'in Tanrisi'nin sözlerinden titreyenlerin hepsi çevremde toplandi. Bense aksam sunusu sunulana dek dehset içinde kaldim.
\par 5 Aksam sunusu saati gelince üzüntümü bir yana birakip kalktim. Giysimle cüppem hâlâ yirtikti. Diz çöküp ellerimi Tanrim RAB'be açtim.
\par 6 Söyle dua ettim: "Ey Tanrim, yüzümü sana çevirmeye utaniyorum, sikiliyorum. Ey Tanrim, günahlarimiz basimizdan askin. Suçlarimiz göklere ulasti.
\par 7 Atalarimizin günlerinden bugüne dek suçlarimiz içinde bogulduk. Günahlarimiz yüzünden biz de, krallarimizla kâhinlerimiz de yabanci krallarin eline teslim edildik. Kiliçtan geçirildik, sürgüne gönderildik. Yagmalandik. Bugün de oldugu gibi asagilandik.
\par 8 "Simdiyse Tanrimiz RAB bir an için bize acidi. Sürgünden kurtulan bir azinlik birakti bize. Kutsal yerinde bize sarsilmaz bir destek verdi. Gözlerimizi aydinlatti. Köleligimizde bize yenilenme firsati sagladi.
\par 9 Köle oldugumuz halde Tanrimiz bizi köle birakmadi. Pers krallarinin bize iyi davranmalarini sagladi: Tanrimiz'in Tapinagi'ni yeniden kurmak, yikik yerleri onarmak için bize yenilenme firsati verdi. Yerusalim'de ve Yahuda'da bize bir korunma duvari verdi.
\par 10 "Ey Tanrimiz, bundan baska ne diyebiliriz? Kullarin peygamberler araciligiyla verdigin buyruklara uymadik. Söyle demistin: `Mülk edinmek için gitmekte oldugunuz ülke, orada yasayan halklarin igrençlikleriyle kirlenmistir. Igrençlikleri yüzünden ülke bastan basa murdarliklarla* doldu.
\par 12 Bunun için kizlarinizi onlarin ogullarina vermeyin. Onlarin kizlarini da ogullariniza almayin. Hiçbir zaman onlarin esenligi ve iyiligi için çalismayin. Öyle ki, güç bulasiniz, ülkenin iyi ürünlerini yiyesiniz ve ülkeyi sonsuza dek ogullariniza miras birakasiniz.'
\par 13 "Basimiza gelenlere yaptigimiz kötülükler ve büyük suçumuz neden oldu. Sen, ey Tanrimiz, bizi hak ettigimizden daha az cezalandirdin ve bize sürgünden kurtulan böyle bir azinlik biraktin.
\par 14 "Yine buyruklarina karsi gelecek miyiz? Bu igrençlikleri yapan halklarla evlilik bagiyla karisacak miyiz? Bunu yaparsak, tek kisi sag kalmadan yok edinceye dek bize öfkelenmeyecek misin?
\par 15 Ey Israil'in Tanrisi RAB, sen adilsin! Bugün sürgünden kurtulan bir azinlik olarak birakildik. Senin önünde durmaya hakkimiz olmadigi halde, suçlarimizin içinde önünde duruyoruz."

\chapter{10}

\par 1 Aglayarak kendini Tanri'nin Tapinagi'nin önünde yere atan Ezra dua edip günahlarini açikladi. Bu arada erkek, kadin, çocuk, Israilliler'den çok büyük bir topluluk Ezra'nin çevresine toplandi. Onlar da hiçkira hiçkira agliyordu.
\par 2 Elamogullari'ndan Yehiel oglu Sekanya, Ezra'ya söyle dedi: "Çevremizdeki halklardan yabanci karilar aldigimiz için Tanrimiz'a ihanet ettik. Buna karsin Israil için hâlâ umut var.
\par 3 Senin ve Tanrimiz'in buyruklari karsisinda titreyenlerin ögütleri uyarinca, bütün yabanci kadinlari ve çocuklarini uzaklastirmak için Tanrimiz'la simdi bir antlasma yapalim. Bu antlasma yasaya uygun olsun.
\par 4 Haydi kalk! Sorumluluk senin üzerinde. Biz seni destekleyecegiz. Güçlü ol ve gerekeni yap!"
\par 5 Bunun üzerine yerden kalkan Ezra önde gelen Levili kâhinlere ve öbür Israilliler'e söyleneni yapmalari için ant içirdi. Hepsi ant içti.
\par 6 Sonra Ezra Tanri'nin Tapinagi'nin önünden ayrilip Elyasiv oglu Yehohanan'in odasina gitti. Orada gecelerken ne yemek yedi, ne su içti. Sürgünden dönenler Tanri'ya bagli kalmadigi için yas tutuyordu.
\par 7 Sürgünden dönenlerin hepsinin Yerusalim'de toplanmasi için Yahuda ve Yerusalim'de bir duyuru yapildi:
\par 8 Halkin önderlerinin ve ileri gelenlerinin karari uyarinca, üç gün içinde gelmeyenin bütün malina el konulacak, kendisi de sürgünden dönenler toplulugundan atilacakti.
\par 9 Bütün Yahudali ve Benyaminli erkekler üç gün içinde Yerusalim'de toplandilar. Dokuzuncu ayin* yirminci günü hepsi Tanri'nin Tapinagi'nin önündeki alandaydi. Hem durumun öneminden, hem de yagmurdan ötürü herkes titriyordu.
\par 10 Kâhin Ezra kalkip, "Siz Tanri'ya ihanet ettiniz" dedi, "Yabanci kadinlarla evlendiniz. Israil'in suçuna suç kattiniz.
\par 11 Simdi atalarinizin Tanrisi RAB'be suçunuzu açiklayin. O'nun istedigini yapin. Çevredeki halklardan ve yabanci karilardan ayrilin."
\par 12 Topluluk yüksek sesle söyle karsilik verdi: "Bütün söylediklerini yapacagiz.
\par 13 Yalniz kalabalik çok, üstelik hava da yagmurlu. Disarda duracak gücümüz kalmadi. Hem bu bir iki günde çözülecek is degil. Çünkü bu konuda çok günah isledik.
\par 14 Bütün topluluk adina önderlerimiz bu konuyla ilgilensin. Sonra kentlerimizde yabanci kadinla evli olan herkes saptanan bir zamanda kentin ileri gelenleri ve yargiçlariyla birlikte gelsin. Yeter ki, Tanrimiz'in bu konudaki kizgin öfkesi üzerimizden kalksin."
\par 15 Ancak, Asahel oglu Yonatan, Tikva oglu Yahzeya ve onlari destekleyen Mesullam ile Levili Sabbetay buna karsi çiktilar.
\par 16 Sürgünden dönenler bu öneriye göre davrandilar. Kâhin Ezra adlarini belirterek her boydan boy baslarini seçti. Onuncu ayin birinci günü oturup konuyu incelemeye basladilar.
\par 17 Birinci ayin birinci günü yabanci kadinlarla evlenen bütün erkeklerin durumunu incelemeyi bitirdiler.
\par 18 Kâhinlerin soyundan gelip yabanci kadinlarla evlenenler sunlardi: Yosadak oglu Yesu'nun ogullarindan ve kardeslerinin soyundan Maaseya, Eliezer, Yariv, Gedalya.
\par 19 Bunlar karilarini kovacaklarina söz verdiler. Isledikleri suç için suç sunusu* olarak sürüden bir koç sundular.
\par 20 Immerogullari'ndan: Hanani, Zevadya.
\par 21 Harimogullari'ndan: Maaseya, Eliya, Semaya, Yehiel, Uzziya.
\par 22 Pashurogullari'ndan: Elyoenay, Maaseya, Ismail, Netanel, Yozavat, Elasa.
\par 23 Levililer'den: Yozavat, Simi, Kelaya -Kelita- Petahya, Yahuda, Eliezer.
\par 24 Ezgicilerden: Elyasiv. Tapinak kapi nöbetçilerinden: Sallum, Telem, Uri.
\par 25 Öbür Israilliler'den: Parosogullari'ndan: Ramya, Yizziya, Malkiya, Miyamin, Elazar, Malkiya, Benaya.
\par 26 Elamogullari'ndan: Mattanya, Zekeriya, Yehiel, Avdi, Yeremot, Eliya.
\par 27 Zattuogullari'ndan: Elyoenay, Elyasiv, Mattanya, Yeremot, Zavat, Aziza.
\par 28 Bevayogullari'ndan: Yehohanan, Hananya, Zabbay, Atlay.
\par 29 Baniogullari'ndan: Mesullam, Malluk, Adaya, Yasuv, Seal, Yeremot.
\par 30 Pahat-Moavogullari'ndan: Adna, Kelal, Benaya, Maaseya, Mattanya, Besalel, Binnuy, Manasse.
\par 31 Harimogullari'ndan: Eliezer, Yissiya, Malkiya, Semaya, Simon,
\par 32 Benyamin, Malluk, Semarya.
\par 33 Hasumogullari'ndan: Mattenay, Mattatta, Zavat, Elifelet, Yeremay, Manasse, Simi.
\par 34 Baniogullari'ndan: Maaday, Amram, Uel,
\par 35 Benaya, Bedeya, Keluhu,
\par 36 Vanya, Meremot, Elyasiv,
\par 37 Mattanya, Mattenay, Yaasay,
\par 38 Bani, Binnuy, Simi,
\par 39 Selemya, Natan, Adaya,
\par 40 Maknadvay, Sasay, Saray,
\par 41 Azarel, Selemya, Semarya,
\par 42 Sallum, Amarya, Yusuf.
\par 43 Nevoogullari'ndan: Yeiel, Mattitya, Zavat, Zevina, Yadday, Yoel, Benaya.
\par 44 Bunlarin hepsi yabanci kadinlarla evlenmisti. Bazilarinin bu kadinlardan çocuklari da vardi.


\end{document}