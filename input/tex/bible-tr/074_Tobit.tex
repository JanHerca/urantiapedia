\begin{document}

\title{Tobit}

\chapter{1}

\par 1 Gabael oğlu,Aduel oğlu, Ananyel oğlu, Tobyel oğlu Tobit'in öyküsü. Kendisi Asyel'in soyundan Naftali oymağındandı.
\par 2 Asur Kralı Şalmaneser'in zamanında Tisbe'den sürgüne gönderildi. Tisbe, yukarı Galile'de, Kadeş Naftali'nin güneyinde, Hazor'un yukarısında, Sefet'in kuzeybatı yönündedir.
\par 3 Ben Tobit, yaşadığım sürece doğru yoldan ayrılmadım, kendimi iyi işlere adadım. Asur ülkesinde, Ninovada, benim gibi sürgün olan kardeşlerime ve vatandaşlarıma pek çok sadaka verdim.
\par 4 Gençliğimde, İsrail ülkesinde evimdeyken, atam Naftali'nin bütün oymağı Davut'un soyundan ve Yeruşalim'den ayrıldı. Oysa bütün İsrail oymaklarının kurbanlarını sunmaları için bu kent seçilmişti; bu kentte tapınak -Tanrı'nın evi- yapılmıştı ve bütün gelecek kuşaklar için kutsanmıştı.
\par 5 Ama bütün kardeşlerim ve Naftali'nin soyu, İsrail Kralı Yeroboam'ın Galile dağlarında, Dan'da yaptırdığı danaya kurban kesiyordu.
\par 6 Kutsal bir yer olan Yeruşalim'i ziyarete giderken çoğu kez yapayalnızdım. Bütün İsrail'i sonsuza dek bağlayan yasayı yerine getiriyordum. Meyvelerden alınan ilk ürünle ve hayvanlarla, sığırların onda biriyle ve ilk kez kırkılan koyunlarla çarçabuk Yeruşalim'e gidiyordum.
\par 7 Bunları sunmak için Harun'un oğulları olan kâhinlere veriyordum. Yeruşalim'de hizmet eden Levililer'e şarapla mısırın, zeytinlerin, narlarla diğer meyvelerin onda birini veriyordum. Altı yıl süre ile bu ikinci onda bir gelirimin parasını alarak Yeruşalim'e gidip ödedim.
\par 8 Gelirimin üçüncü onda birini öksüzlere, dullara ve İsrailliler'le birlikte yaşayan yabancılara verdim. Her üç yılda bir bunu onlara bir armağan olarak verdim. Yemek yerken Musa'nın yasasına ve babamız Ananyel'in annesi Debora'nın öğütlerine uyardık. Çünkü babam ölmüştü ve ben öksüz kalmıştım.
\par 9 Büyüyüp genç bir erkek olunca, akrabalarımızdan Anna adında bir kadınla evlendim. Bir oğlumuz oldu, ona Tobyas adını verdim.
\par 10 Asur ülkesine sürgün edilince, beni alıp Ninova'ya götürdüler. Bütün kardeşlerim ve benim soyumdan olan kişiler putperestlerin yemeğini yedi. Ama ben putperestlerin yemeğini yemedim.
\par 11 Bütün yüreğimle 
\par 12 Tanrı'ma inanmayı sürdürdüğüm için,
\par 13 yüce Tanrı Şalmaneser'in benimle ilgilenmesini sağladı ve kralın besinini sağlamakla görevlendirildim.
\par 14 O ölünceye dek Medya'ya yolculuk ettim ve orada onun adına iş yaptım. Medya'da, Rages'te, Gabri'nin kardeşi Gabael'e on talant gümüş verdim.
\par 15 Şalmaneser ölünce, oğlu Sanherib onun yerine geçti. Medya'ya giden yollar kapanmıştı ve artık oraya gidemiyordum.
\par 16 Şalmaneser'in zamanında sadakamı çoğu kez benim soyumdan olan kardeşlerime verirdim.
\par 17 Ekmeğimi aç kalanlara, giysilerimi çıplak olanlara verirdim. Ninova'nın duvarlarının dışına atılan vatandaşlarımın cesetlerini gördüğüm zaman onları gömerdim.
\par 18 Sanherib'in öldürdüklerini de gömdüm. Tanrı onun sövgülerini cezalandırdı ve Yahudiye'den düzensiz biçimde geri çekildi. Öfkelenen Sanherib çok sayıda İsrailli'yi öldürdü. Cesetleri ben gizlice alıp gömdüm. Sanherib cesetleri aradı, ama bulamadı.
\par 19 Ninovalı bir kişi krala gidip gizlice gömdüğümü bildirdi. Krala söylenenleri öğrenince ve beni öldürmek için aradıklarını anlayınca, korkup kaçtım.
\par 20 Bütün varlığıma el kondu, tümünü hazine aldı. Elimde kalan salt eşim Anna ve oğlum Tobyas'tı.
\par 21 Kırk gün geçmeden, iki oğlu kralı öldürüp Ağrı Dağı'na kaçtı. Oğlu Esarhaddon kralın yerine geçti. Kardeşim Anael'in oğlu Akikar maliye bakanlığına atandı ve kendisine işleri yönetme yetkisi verildi.
\par 22 Akikar benim adıma aracılık etti ve Ninova'ya dönmeme izin verildi. Çünkü Asur Kralı Sanherib zamanında Akikar baş saki, mührü saklayan kişi, yönetici ve hazine bakanı görevlerini bir arada yürütüyordu. Esarhaddon, ondan görevlerini sürdürmesini istemişti. Akikar benim akrabamdı, yeğenimdi.

\chapter{2}

\par 1 Böylece, Esarhaddon döneminde eve döndüm ve eşim Anna'yla oğlum Tobyas'a kavuştum. Pentikost Bayramı'nda (Haftalar Bayramı'nda) güzel bir yemek vardı. Yemek için oturdum.
\par 2 Masayı ve pişirilen yemek türlerini önüme koydular. O zaman oğlum Tobyas'a şöyle dedim: "Oğlum, git de, Ninova'ya sürgün edilen kardeşlerimiz arasında dostluğu ve bağlılığı içten olan yoksul birini bul ve buraya getir ki, yemeğimi onunla paylaşayım. Geri dönünceye dek seni beklerim çocuğum. "
\par 3 Tobyas kardeşlerimiz arasında yoksul birini bulmak için dışarı çıktı. Geri dönüp, "Baba!" dedi. "Ne oluyor oğlum?" diye sordum. Konuşmasını sürdürdü: "Baba, ulusumuzdan birini şimdi öldürdüler; pazaryerinde onu boğup yere attılar. Hâlâ orada. "
\par 4 Yiyeceklere dokunmadan çabucak yerimden fırladım, adamın cesedini pazaryerinden alıp evimdeki odalardan birine koydum ve onu gömmek için güneş batıncaya dek bekledim.
\par 5 Yeniden odama girdim, yıkandım ve acıyla yemeğimi yedim.
\par 6 Bu arada Peygamber Amos'un Beytel'le ilgili sözlerini anımsadım: "Bayramlarınız yasa, Şarkılarınız çığlığa dönüşecektir. "
\par 7 Ağladım. Güneş batınca gidip bir mezar kazdım ve cesedi gömdüm.
\par 8 Komşularım gülüp şöyle dediler: "Bakın! Hâlâ kaygılanmıyor. Daha önce bu yüzden başına ödül konmuş, kaçmak zorunda kalmıştı. Şimdi de geri döndü ve yeniden ölüleri gömmeye başladı. "
\par 9 O gece banyo yaptım. Sonra avluya çıktım ve avludaki duvarın dibine uzanıp yattım. Hava sıcak olduğu için yüzümü örtmeden yattım.
\par 10 Duvarda, başımın üstünde serçeler olduğunu bilmiyordum. Onlardan damlayan sıcak birikintiler gözlerimin içine düştü. Ardından, gözlerimde beyaz benekler oluştu ve tedavi için hekimlere gitmek zorunda kaldım. Ama gözlerim için gereğinden çok merhem denemelerine karşın, benekler beni daha çok körleştiriyordu ve sonunda tümüyle kör oldum. Dört yıl süreyle görmüyordum. Tüm erkek kardeşlerim kaygılanmıştı. Akikar, Elimayis'e gidinceye dek, iki yıl süreyle bakımım için gerekli olan parayı karşıladı.
\par 11 Bundan sonra eşim Anna kadınların yaptığı işlerde çalışmaya başladı. Yün örüp kumaş dokudu.
\par 12 Kendisine ısmarlanan şeyleri götürüp yerine veriyor, ardından parasını alıyordu. Mart'ın yedisinde elindeki işi bitirip alıcıya götürdü. Parasını ödediler ve kendisine yemek için bir keçi yavrusu armağan ettiler.
\par 13 Evime gelince, keçi yavrusu melemeye başladı. Eşimi çağırıp ona şöyle dedim: "Bu yaratık nereden geliyor? Ya çalınmışsa! Çabuk sahibine geri götür, çalınmış yiyecekleri yemeye hakkımız yok. "
\par 14 Eşim şöyle dedi: "Hayır, bana ödenen ücretin dışında, bu bana verilen bir armağandır. " Ona inanmadım ve keçi yavrusunu sahibine götürmesini istedim. Bunu söylerken karşısında yüzüm kızardı. O bana şu yanıtı verdi: "Verdiğin sadakalardan ne haber? Yapmış olduğun iyi şeylerden ne haber? Bunların karşılığında başına gelenleri herkes biliyor."

\chapter{3}

\par 1 O zaman yüreğimde acı bir iç çekip ağladım ve inleyerek aşağıdaki duayı okudum:
\par 2 "Rabbim, sen herkesin hakkını gözetirsin, Tüm yapıtların doğrudur. Ortamın iyilik ve gerçektir, Dünyanın yargıcısın.
\par 3 Onun için Rabbim, Beni anımsa, beni gözet. Günahlarım için, Düşüncesizliğimin neden olduğu Yersiz davranışlarım için Ya da atalarımın günahları için beni cezalandırma.
\par 4 Çünkü sana karşı günah işledik, Buyruklarına uymadık, Sen de bizi terk ettin. Bu nedenle bizi soydular, Tutsak edip öldürdüler. Başka uluslara karıştık Ve o uluslar bizden söz etti, Bizi gülünç duruma soktu, Bizi hor gördü.
\par 5 Oysa bütün buyrukların doğrudur, Senin tutumun yersiz davranışlarımın sonucudur. Atalarımın yersiz davranışları da buna neden olmuştur, Çünkü senin buyruklarına uymadık, Senin katında doğru yolda yürümedik.
\par 6 Şimdi, bana istediğin gibi davran. Yaşamımı sonuçlandırmaktan kıvanç duy; Bundan böyle bu topraklarda yaşama isteğimi yitirdim; Yeniden toprak olmak istiyorum. Çünkü ben ölümü yaşama yeğ tutuyorum. Nedenini bilmeden yerildim ve acım sonsuzdur. Rabbim, beni bu acıdan kurtarmak için vereceğin yargıyı bekleyeceğim. Bırak da, sonsuza dek benim olacak eve gideyim. Rabbim, bana sırt çevirme, Çünkü acımasız belaya çatınca, Ölüm yaşama yeğ tutulur. Bana iftira atılmasından usandım. "
\par 7 Ekbatana'da Medya'da yaşayan Raguel'in kızı Sara'yı bir rastlantı sonucu aynı gün, babasının hizmetçilerinden biri aşağıladı.
\par 8 Sara'nın yedi kez evlendiğini bilmeniz gerek. Cinlerin en kötüsü olan Azmodius, damatları arka arkaya öldürmüştü. Bu işi evlenen çift birleşmeden önce yapmıştı. Hizmetçi kız şöyle dedi: "Evet, damatları sen öldürdün. Seni yedi kez evlendirdiler ve şansın hiç açılmadı.
\par 9 Damatların ölmesi, bizleri cezalandırman için bir neden değil. Sen de onların yanına git. Böylece, senin çocuğunu görmek olasılığından kurtulmuş oluruz!"
\par 10 O gün Sara çok üzüldü, hüngür hüngür ağladı ve kendini asmayı tasarlayarak babasının odasına çıktı. Ardından düşünceye daldı: "Babamı suçladıklarını varsay! Diyecekler ki, sevdiğin tek bir kızın vardı ve şimdi acıdan kendini astı. Yaşlı babama böyle bir acı verirsem o bu acıya dayanamaz ve ölüler ülkesine göçer. Kendimi asmamam daha iyi, ancak ölmek için ve daha çok yaşayıp aşağılanmamak için Tanrı'ya yalvarmalıyım. "
\par 11 Ardından pencerenin yanına gitti ve kollarını uzatarak şu duayı okudu: "Bağışlayan Tanrı, sen kutsalsın! Adını sonsuza dek kutsayalım, Yarattığın bütün yapıtlar Sonsuza dek seni kutsasın.
\par 12 Şimdi yüzümü kaldırıp Gözlerimi sana çeviriyorum.
\par 13 Sözün beni yaşadığım topraklardan kurtarsın, Bana iftira atılmasına artık dayanamıyorum.
\par 14 Rabbim, Tertemiz olduğumu biliyorsun, Bir erkek eli bana değmedi;
\par 15 Sürgün edildiğimiz bu ülkede Ne senin adına leke sürdüm Ne de babamın adına. Babamın tek kızıyım, Mirasçı olacak başka çocuğu yok. Yanında bir erkek kardeşi yok, Hiç akrabası kalmadı, Yaşamımı sürdürmem için bir neden yok. Doğrusu yedi koca kaybettim, Bir süre daha yaşayıp ne olacak? Yaşamımı noktalamak istemiyorsan, O takdirde bana acıyarak bak; Bana iftira atılmasına artık dayanamıyorum. "
\par 16 O süre içinde görkemli Tanrı'nın katında her ikisinin de duası kabul olundu.
\par 17 Tanrı, her ikisinin de derdine çare bulması için Rafael adındaki meleği onlara gönderdi. Rafael, Tanrı'nın yarattığı ışığı kendi gözleriyle görebilmesi için Tobit'in gözlerindeki beyaz benekleri yok edecekti. Raguel'in kızı Sara'yı en kötü cinlerden biri olan Azmodius'dan kurtaracak ve onu Tobit'in oğlu Tobyas'la evlendirecekti. Çünkü Sara'ya talip olan bütün erkeklerden önce, onunla evlenmek Tobyas'ın hakkıydı. Raguel'in kızı Sara yukarıdaki odadan aşağı inerken aynı anda Tobit avludan eve dönüyordu.

\chapter{4}

\par 1 Aynı gün Tobit Medya'da, Rages'te Gabael'e bıraktığı gümüşleri anımsadı.
\par 2 Şöyle düşündü: "Ölmek için dua edecek durumdayım. Ölmeden önce oğlum Tobyas'ı çağırıp ona para konusunda bilgi vermem iyi olur. "
\par 3 Oğlu Tobyas'ı çağırıp ona şöyle dedi: "Öldüğüm zaman onurlu bir şekilde gömülmemi sağla. Annene saygı göster ve yaşadığın sürece onu asla bırakma. Tüm isteklerini yerine getir ve onu asla üzme.
\par 4 Sen onun rahmindeyken senin için göze aldığı tehlikeleri anımsa, oğlum. Annen öldüğü zaman onu benim yanıma, aynı mezara göm.
\par 5 "Oğlum, yaşadığın sürece Tanrıya güven. Günah işlemek ya da Tanrı'nın yasalarına uymamak isteğini asla duyma. Yaşadığın sürece iyi işler yap, doğru olmayan biçimde asla davranma.
\par 6 Çünkü dürüst davranırsan bütün yaptıklarında başarılı olursun. Doğru davranan bütün insanlar aynı durumdadır.
\par 7 "Mallarından bir bölümünü sadaka vermek için ayır. Yoksul kişiye asla sırt çevirme. Tanrı da sana sırt çevirmez.
\par 8 Sadaka verirken varlığını ölçü olarak kullan. Varlıklıysan daha çok ver, varlıklı değilsen az ver, ama sadaka verirken eli sıkı olma.
\par 9 Böyle davranırsan, yoksulluk günleri için kendine büyük bir hazine hazırlamış olursun.
\par 10 Çünkü sadaka insanı ölümden kurtarır ve karanlığa gömülmesini önler.
\par 11 Sadakayı yüce Tanrı'nın katında vermek çok etkili bir sunudur.
\par 12 "Oğlum, düzensiz davranışlardan sakın. Seçeceğin eş babanın soyundan olsun. Babanın oymağından olmayan yabancı bir eş alma, çünkü bizler, peygamberlerin çocuklarıyız. Baştan beri atalarımız olan Nuh'u, İbrahim'i, İshak'ı ve Yakup'u anımsa. Tümü kendi soyundan eş aldı, çocuklarıyla kutsandılar ve onların soyu yeryüzünün mirasçısı olacaktır.
\par 13 Oğlum sen de kendi kardeşlerini yeğ tutmalısın. Kardeşlerini asla küçük görme, onlar ulusun oğulları ve kızlarıdırlar. Eşini onların arasından seç. Çünkü gurur, insanın kaygı duymasına ve yıkımına neden olur. Aylaklık yoksunluğa ve yoksulluğa neden olur, çünkü aylaklık açlığın anasıdır.
\par 14 "Senin için çalışanların ücretini çabucak öde, ücretlerini ödemeyi ertesi güne bırakma. Tanrı'ya hizmet edersen ödülünü alırsın. Oğlum, yaptıklarında dikkatli ol, bütün davranışlarında tutumun eğitilmiş kişininki gibi olsun.
\par 15 Sana nasıl davranılmasını istiyorsan başkalarına da öyle: davran. Sarhoş olacak kadar şarap içme, yolculuk ederken aşırı davranışlardan sakın.
\par 16 "Aç olanlara ekmeğini ve çıplak olanlara giysilerini ver. Varlıklıyken elinde bulunanların bir bölümünü sadakaya ayır; sadaka verdiğin zaman isteksiz olma.
\par 17 Erdemli kişinin mezarında ekmek sunmakta eli açık davran, ama günahlılara bir şey verme.
\par 18 "Bilge kişinin öğüdünü dinle, yararlı bir öğüdü asla küçümseme.
\par 19 Bir işe girişirken Tanrı'ya yalvar, sana yol göstermesi ve amacına ulaşabilmen için O'na yakar. Çünkü bilgelik bütün uluslara verilmemiştir, bütün iyilikleri veren Tanrı'dır. İsterse insanı yüceltir, isterse insanın yıkımına neden olur. Ölüler ülkesinin derinliklerine gömer. Bundan ötürü oğlum, öğütlerimi anımsa ve sözlerimin izleri asla yüreğinden silinmesin.
\par 20 "Şimdi, oğlum, Medya'da, Rages'te, Gabri'nin oğlu Gabael'de on talant gümüş bıraktığımı sana söylemem gerek.
\par 21 Yoksul olduksa kaygılanma, oğlum. Tanrı'dan korkuyorsan, her türlü günahtan uzak duruyorsan ve yaptıklarından Tanrı hoşnutsa, senin büyük bir servetin var demektir. "

\chapter{5}

\par 1 Bunun üzerine, Tobyas babası Tobit'e şöyle yanıt verdi: "Baba, bütün isteklerini yerine getireceğim.
\par 2 Ama ondan parayı nasıl alabilirim? Ne beni tanıyor ne de ben onu. Bana inanması ve gümüşü vermesi için kim olduğumu nasıl belirteceğim? Ayrıca, Medya'ya yolculuk yapmak için hangi yoldan gidileceğini bilmiyorum. "
\par 3 Tobit, oğlu Tobyas'a şöyle yanıt verdi: "Bir pusulayı ikimiz de imzaladık, pusulayı ikiye bölüp kestim, yarısı bende yarısı da onda kalsın diye. Pusulanın yarısını aldım, öbür yarısını da gümüşün yanına koydum. Düşün ki bu olaylar yirmi yıl önce oldu. Gümüşü de koruması için ona bıraktım! Bundan ötürü oğlum, seninle yolculuk yapacak güvenilir bir adam bul. Seninle yolculuk ederek kaybedeceği zaman için ona para ödeyeceğiz. Onunla Gabael'e gidip parayı al. "
\par 4 Tobyas, yolu bilen ve onunla birlikte Medya'ya gidecek bir adam bulmak üzere dışarı çıktı. Dışarıda Rafael adındaki melek onu bekliyordu. Ama Tobyas Tanrı'nın meleklerinden biriyle karşılaştığını anlayamadı.
\par 5 Tobyas sordu: "Nereden geliyorsun, dostum?" Melek, "İsrailli kardeşlerinden biriyim, iş bulmak için buralara geldim" diye yanıtladı. Tobyas yine sordu: "Medya'ya giden yolu biliyor musun?"
\par 6 "Kuşkusuz biliyorum" dedi, "Oraya çoğu kez gittim, bütün yolları ezbere biliyorum. Medya'ya sık sık gider, Gabael'de kalırdım. Medya'dan Rages'e gitmek iki gün sürer, Rages dağlık bölgededir, Ekbatana da ovanın ortasındadır. "
\par 7 Tobyas ona şöyle dedi: "Dostum beni bekle, gidip babamla görüşeyim. Sana gereksinmem var. Benimle gelirsen kaybedeceğin zaman karşılığında sana para öderim. "
\par 8 Melek, "Olur, beklerim, ama çabuk dön" dedi.
\par 9 Tobyas içeri girdi ve babasına İsrailli kardeşlerinden birini bulduğunu anlattı. Babası şöyle dedi: "Onu içeri getir, ailesi ve oymağı konusunda bilgi almak istiyorum. Senin için güvenilir bir arkadaş olup olmadığını anlamam gerek, oğlum. "
\par 10 Bunun üzerine Tobyas dışarı çıkıp onu çağırdı: "Dostum, babam seni istiyor. " Melek eve girdi. Tobit onu selamladı, o da karşılık verdi ve ona mutluluk diledi. Tobit, "Bir daha mutlu olabilir miyim?" dedi, "Ben körüm, gökten gelen ışığı artık göremiyorum. Işığı görmeyen ölüler gibi karanlığa gömülmüş durumdayım. Ben, canlıyken mezara girmiş bir adamım. Söylenenleri duyuyorum ama insanları göremiyorum. " Melek ona şöyle dedi: "Avunmaya çalış. Tanrı yakında seni iyileştirecektir. Avunmaya çalış. " Tobit de şöyle dedi: "Oğlum Tobyas Medya'ya gitmek istiyor. Ona yol gösterir misin? Kardeşim, sana para veririm. " Melek, "Onunla gitmek istiyorum" diye yanıtladı, "Bütün yolları biliyorum. Medya'ya çoğu kez gittim, bütün ovalarından ve dağlarından geçtim, bütün yolları bilirim. "
\par 11 Tobit şöyle dedi: "Kardeşim, sen hangi aileden ve hangi oymaktansın? Bana söyler misin?"
\par 12 Melek, "Benim oymağımdan sana ne?" dedi. Tobit, "Kimin oğlu olduğunu iyice bilmek ve adını öğrenmek istiyorum" dedi.
\par 13 Melek şöyle dedi: "Ben büyük Hananya'nın oğlu Azarya'yım, akrabalarından biriyim. "
\par 14 "Hoş geldin ve selamlar kardeşim!" dedi Tobit, "Ailenin adını bilmek istediğim için sakın alınma. Akrabam olduğunu, iyi ve onurlu bir soydan geldiğini öğreniyorum. Büyük Semeya'nın iki oğlunu, Hananya ile Natan'ı tanırım. Benimle Yeruşalim'e gelirlerdi. Orada birlikte dua ettik ve doğru yoldan asla ayrılmadılar. Kardeşlerin değerli kişilerdir, iyi bir soydan geliyorsun, hoş geldin. "
\par 15 Sözlerine şöyle devam etti: "Sana günde bir gümüş para ödeyeceğim, ayrıca, oğlum gibi senin de masraflarını karşılayacağım. Yolculuğu oğlumla tamamlarsan,
\par 16 sana anlaştığımızdan daha çok para öderim. " Melek yanıtladı: "Yolculuğu onunla tamamlayacağım. Kaygılanma, giderken de dönerken de her şey yolunda gidecek. Yol tehlikesizdir. "
\par 17 Tobit şöyle dedi: "Seni kutsarım, kardeşim!" Sonra oğluna dönüp şöyle dedi: "Oğlum, yolculuk için gereken hazırlıkları yap ve kardeşinle yola çık. Göklerdeki Tanrı dış ülkede sizi korusun ve her ikinizi de esenlikle bana geri göndersin! Tanrı'nın iyilik meleği sizinle olsun ve sizi korusun, oğlum!" Tobyas yola çıkmak üzere evden çıktı, annesiyle babasını öptü. Tobit, "İyi yolculuklar!" dedi.
\par 18 Tobyas'ın annesi ağlamaya başladı ve Tobit'e şöyle dedi: "Oğlumu niçin uzağa gönderiyorsun? O, dayandığımız asa değil mi, her işimize koşmuyor mu?
\par 19 Kuşkusuz önemli olan salt para değildir. Kuşkusuz para oğlumuz kadar değerli değildir.
\par 20 Tanrı 'nın bize verdiği yaşam biçimi bize yeterdi. "
\par 21 Tobit şöyle dedi: "Böyle şeyler düşünme. Her şey iyi olacak, oğlumuz gidip gelecek. Eve döndüğü gün iyi olduğunu sen de göreceksin. Böyle şeyler düşünme. Onlar için kaygılanma, bacım.
\par 22 Tanrı'nın iyilik meleklerinden biri onunla beraber gidecek, yolculuğu iyi geçecek. Bize esenlik içinde mutlu dönecek. "

\chapter{6}

\par 1 O da gözünün yaşını sildi.
\par 2 Delikanlı melekle yola çıktı. Köpek onları izliyordu. Sürekli yol aldılar ve ilk gün akşamüstü Dicle Irmağı'nın yanında kamp kurdular.
\par 3 Delikanlı ayaklarını yıkamak için ırmağa inmişti. Ama sudan sıçrayan büyük bir balık az daha ayağını yutacaktı. Delikanlı bir çığlık attı.
\par 4 Melek şöyle dedi: "Balığı yakala, onu kaçırma. " Delikanlı balığı yakalayıp ırmağın kenarına çekti.
\par 5 Melek devam etti: "Balığı yar, safrakesesini, yüreğini ve karaciğerini çıkar. Bunları bir yana ayır, bağırsaklarını at. Çünkü safrakesesinin, yüreğinin ve karaciğerinin şifa veren özellikleri vardır. "
\par 6 Delikanlı balığı yardı, safrakesesini, yüreğini ve karaciğerini ayırdı. Balığın bir kısmını yemek için kızarttı, geri kalanını da tuzda saklamak üzere bir yana koydu. Sonra yola çıktılar ve Medya'ya iyice yaklaştılar.
\par 7 O zaman delikanlı meleğe sordu: "Azarya kardeş, balığın yüreği, karaciğeri ve safrakesesi nasıl şifa verir?"
\par 8 Melek şöyle yanıtladı: "Balığın yüreğini ve karaciğerini yakarsan, oluşan duman, bir şeytanın veya kötü bir cinin musallat olduğu erkeğe veya kadına şifa verir. Bu tür bela, hiç bir iz bırakmadan, tümüyle ortadan kalkar.
\par 9 Safrakesesine gelince, gözlerinde beyaz benekler oluşan kişi için göz merhemi olarak kullanılır. Safrakesesini kullandıktan sonra, gözlerin şifa bulması için beneklerin üstüne üflemen yeter. "
\par 10 Medya'ya girdiler ve Ekbatana'ya çok yaklaştılar.
\par 11 Rafael delikanlıya, "Tobyas kardeş" dedi. Ötekisi yanıtladı: "Buyurun!" Melek konuşmasını sürdürdü: "Bu gece Raguel'de kalacağız. O senin akraban olur. Sara adında bir kızı var.
\par 12 Ama Sara'dan başka hiç bir oğlu veya kızı yok. En yakın akrabası sensin. O başkalarının değil, senindir ve babasının mirasında hak talep edebilirsin. Sara düşünceli, yürekli ve çok güzel bir kızdır. Babası onu çok sever.
\par 13 Onunla evlenmek hakkındır. Dinle, kardeşim: Bu akşam kız konusunda babasıyla konuşacağım, kızın seninle nişanlanmasını sağlayacağım. Rages'ten dönüşümüzde evlenirsiniz. İnan ki, Raguel'in bu önerimi geri çevirmeye ya da kızını başkasıyla nişanlamaya hakkı yoktur. Çünkü Musa'nın yasası böyle buyuruyor. Raguel kabul etmezse ölümünü istiyor demektir. Akrabası olman nedeniyle kızıyla evlenmek herkesten önce senin hakkın. Raguel bunu öğrenecektir. Bu nedenle dinle kardeşim, bu akşam kız konusunda konuşacağız ve seninle evlenmesini isteyeceğiz. Rages'ten dönüşümüzde onu da alır, senin evine götürürüz. "
\par 14 Tobyas Rafael'e şöyle yanıt verdi: "Azarya kardeş, duyduğuma göre, o kızı yedi kez evlendirmişler, ama yedi damat da zifaf odasında ölmüş. O odaya girdikleri gece ölmüşler, damatları bir cinin öldürdüğünü duydum.
\par 15 Bu nedenle biraz kaygılanıyorum. Kuşkusuz cin kıza bir kötülük yapmıyor, çünkü onu seviyor, ama bir erkek kıza yaklaşmak isterse, cin onu öldürüyor. Ben babamın tek oğluyum ve ölmek istemiyorum. Annemle babama böyle bir acı vermekten çekiniyorum. Çünkü bu acı onların mezara girmesine neden olur. Onları görecek başka oğulları yok. "
\par 16 Melek şöyle dedi: "Babanın öğüdünü unutacak mısın? O sana babanın ailesinden bir eş seçmeni salık verdi. Öyleyse dinle, kardeşim. Sen cin için kaygılanma, o kızla evlen. Sana söz veriyorum, bu gece onu sana eş olarak vermeyi kabul edecekler.
\par 17 Zifaf odasına girdiğin zaman, balığın yüreğini ve karaciğerini al ve bunların bir kısmını yakılan buhurun içine koy. Buhur yayılacak,
\par 18 cin de bu kokuyu alınca kaçacak. Bir daha kızın yanına asla dönmeyecek, öyle bir tehlike yoktur. Sonra birleşmeden önce, ikiniz de ayağa kalkın ve dua edin. Göklerdeki Tanrı'dan sizi bağışlamasını ve korumasını dileyin. Kaygılanma, o kız başından beri senindi ve onu sen kurtaracaksın. O kız seni izleyecektir ve sana söz veriyorum, sana vereceği çocuklar senin için birer erkek kardeş olacaktır. Duraksama. " Tobyas Rafael'in bu sözlerini duyunca ve Sara'nın babasının akrabalarından olduğunu, bacısı olduğunu anlayınca, Sara'ya o denli vuruldu ki, artık gönlünün sahibini bulmuş oldu.

\chapter{7}

\par 1 Ekbatana'ya girerken Tobyas şöyle dedi: "Azarya kardeş, beni çabucak kardeşimiz Raguel'e götür. " O da Raguel'in evine giden yolu Tobyas'a gösterdi. Raguel avlu kapısının yanında oturuyordu. İlk önce selam verdiler. O da yanıt verdi: "Hoş geldiniz, selamlar, kardeşlerim. " Onları eve götürdü.
\par 2 Eşi Edna'ya şöyle dedi: "Bu genç adam, kardeşim Tobit'e ne denli benziyor!"
\par 3 Edna nereden geldiklerini sordu. "Biz Ninova'ya sürgün edilen Naftali'nin oğullarıyız" dediler.
\par 4 Onlara, "Kardeşimiz Tobit'i tanıyor musunuz?" diye sordu. Onlar, "Evet, tanıyoruz" dediler. Edna yine sordu: "Kendisi nasıl?"
\par 5 Onlar, "Hâlâ yaşıyor ve iyidir" diye yanıt verdiler. Tobyas sözlerini şöyle sürdürdü: "O benim babamdır. "
\par 6 Raguel ayağa fırladı ve onu öpüp ağladı.
\par 7 Konuşabildiği zaman şöyle dedi: "Seni kutsarım, oğlum! Soylu bir babanın oğlusun. O denli erdemli ve o denli iyi işler yapan bir insanın kör olması ne acıklı!" Akrabası olan Tobyas'ın boynuna sarıldı ve gözyaşı döktü.
\par 8 Eşi Edna onun için ağladı, kızı Sara da gözyaşı döktü.
\par 9 Raguel sürüden bir koyun kesti ve konukları içtenlikle karşıladılar. Yıkandılar, banyo yaptılar ve sofraya oturdular. Sonra Tobyas Rafael'e şöyle dedi: "Azarya kardeş, bacın Sara'yı benim için Raguel'den ister misin?"
\par 10 Raguel bu sözleri duydu. Genç adama şöyle dedi: "Ye, iç ve elinden geldiği kadar bu geceden faydalan. Kızım Sara ile başka kimsenin evlenmeye hakkı yoktur. Buna ancak senin hakkın vardır, kardeşim. Kızımın en yakın akrabası olduğuna göre, ben onu senden başkasına veremem. Ancak, oğlum, seninle açık konuşmalıyım.
\par 11 Akrabalarımız arasından yedi kez ona bir koca bulmaya çalıştım, ancak hepsi de kızımın odasına girince, ilk gece öldü. Oğlum şimdi ye, iç. Tanrı seni bağışlayacak ve gönül rahatlığı verecektir. " Tobyas şöyle konuştu: "Sen benim hakkımda bir karara varıncaya dek yiyip içmekten anlamam. " Raguel yanıt verdi: "Dediğin gibi olsun. Çünkü Musa'nın yasası uyarınca o sana verilmiştir. Tanrı'nın buyruğu da senin olması yolundadır. Onun için kız kardeşini sana emanet ediyorum. Bundan böyle sen onun erkek kardeşisin, o da senin kız kardeşindir. Bugünden başlayarak, sonsuza dek, o sana verilmiştir. Göklerdeki Tanrı bu gece yardımcın olsun, oğlum. O seni bağışlasın ve sana gönül rahatlığı ver­sin."
\par 12 Raguel kızı Sara'yı çağırdı. Eli­ni tuttu ve kızını Tobyas'a vererek şöy­le dedi: "Onu sana emanet ediyorum. Musa'nın yasasına ve yargısına göre o senin karındır. Onu götür. Vicdanın rahat etsin ve onu babanın evine gö­tür. Göklerdeki Tanrı'nın yardımıyla yolun açık olsun, gönül rahatlığıyla git."
\par 13 Sonra kızının annesine döndü ve yazı yazmak için kâğıt istedi. Evli­lik sözleşmesini hazırladı. Musa'nın yasası uyarınca kızın Tobyas'la nasıl evlendirildiğini yazdı.
\par 14 Birlikte yiyip içtiler.
\par 15 
\par 16 Raguel eşi Edna'yı çağırıp şöyle dedi: "Ba­cım, ikinci odayı hazırla ve kızını ora­ya götür." Edna kocasının buyurduğu gibi, yatağı o odada hazırladı ve kızı­nı odaya götürdü. Kızı için ağladı, sonra gözünün yaşını sildi ve şöyle dedi: "Yürekli ol, kızım! Göklerdeki Tanrı acını sevince çevirsin! Yürekli ol, kızım!" Bunu dedikten sonra dışa­rı çıktı.

\chapter{8}

\par 1 Yiyip içtikten sonra yatma zama­nı geldi. Genç adamı yemek oda­sından yatak odasına götürdüler.
\par 2 Tob­yas, Rafael' in öğüdünü hatırladı. Çan­tasını açtı, balığın yüreğini ve karaci­ğerini çıkardı ve bunların bir kısmını yakılan buhurun içine koydu.
\par 3 Balığın buharı cine acı verdi ve cin hava yo­luyla Mısır'a kaçtı. Rafael onu izledi ve çabucak bağlayıp prangaya vurdu.
\par 4 Bu arada anne ve baba dışarıya çıkmış ve kapıyı kapatmıştı. Tobyas yataktan kalktı ve Sara'ya şöyle dedi: "Kalk, bacım! Sen ve ben Tanrı'ya dua etmeliyiz. Tanrı'dan bizi bağışla­masını ve korumasını dilemeliyiz."
\par 5 Sara ayağa kalktı ve Tanrı'nın onları koruması için birlikte dua ettiler. Tob­yas şöyle dua etti: "Ey atalarımızın Tanrısı, sen kutsalsın, Adın da kutsaldır sonsuza dek. Gökler seni kutsasın, Bütün yaptıkların sonsuza dek seni kutsasın.
\par 6 Adem'i sen yarattın, Ona yardımcı olması ve onu desteklemesi için Eşi Havva'yı sen yarattın, İkisinden insan soyu doğdu. Sen şöyle dedin: 'Erkeğin yalnız kalması iyi değildir, Ona benzeyen bir eş yaratalım.'
\par 7 Ben kız kardeşimi cinsel istek nedeniyle almıyorum, Onu içtenlikle bekâr olduğum için alıyorum. Bize karşı sevecen davran, ona ve bana acı, Birlikte yaşlanmamızı sağla."
\par 8 Birlikte "Âmin! Âmin!" diyerek gece yattılar.
\par 9 Ama Raguel kalkıp hizmetçileri­ni çağırdı. Hizmetçiler gelip bir me­zar kazmak için ona yardım ettiler.
\par 10 Raguel şöyle düşünüyordu: "Tanrı­nın izniyle ölmese keşke? Yoksa bizi bekleyen üstü kapalı alay ve utanç­tır."
\par 11 Mezar hazır olunca Raguel eve dönüp karısını çağırdı,
\par 12 ona şöyle dedi: "Odaya bir hizmetçi gönder de Tobyas'ın hâlâ yaşayıp yaşamadığını anlayalım. Çünkü ölmüşse, kimsenin haberi olmadan belki onu gömebili­riz."
\par 13 Hizmetçiyi çağırdılar, lambayı yaktılar ve kapıyı açtılar. Hizmetçi içeri girdi. Her ikisinin de derin bir uykuda olduğunu gördü.
\par 14 Hizmetçi dışarı çıkıp fısıldayarak şöyle dedi: "Delikanlı ölmedi, işler yolunda."
\par 15 Bunun üzerine Raguel Göklerdeki Tanrı'ya şükrederek şöyle dedi: "Sana hamdederiz, şükrederiz Tanrım, Saf gönüllerle hamdederiz, Sonsuza dek sana hamdedilsin, şükredilsin!
\par 16 Beni sevindirdiğin için sana hamdederim. Korktuğum başıma gelmedi, Onun yerine, bize karşı davranışında olağanüstü bir acıma vardı.
\par 17 Sana hamdederim, Tek oğul olan bu delikanlıya, Tek evlat olan bu kızcağıza acıdığın için. Efendim, onları bağışla ve koru, İzin ver de onların yaşamında Mutluluk ve bağışlanma olsun."
\par 18 Raguel gün doğmadan hizmetçi­lerin mezarı toprakla doldurmalarını sağladı.
\par 19 Eşine bol bol ekmek yap­masını söyledi. İki öküzle dört koyu­nu sürüden alıp getirdi ve onları pişirmeleri için hizmetçilerine buyruk verdi. Hazırlıklar başladı.
\par 20 Tobyas'ı çağırıp ona şöyle dedi: "On beş gün bir yere gidemezsin. Burada kalacak­sın, benimle yiyip içeceksin. Kızım çok acı çekti, onu yeniden mutlu ede­ceksin.
\par 21 Bundan başka varlığımın ya­rısı senin olsun ve hiç bir engelle karşılaşmadan babana dönersin. Eşim ve ben ölünce, varlığımın öbür yarısı da senin olacak. Yürekli ol, oğlum! Ben senin babanım ve Edna da senin annendir. Kız kardeşinin ailesi oldu­ğumuz gibi, bundan böyle senin de aileniz. Yürekli ol, oğlum!"

\chapter{9}

\par 1 Bundan sonra Tobyas Rafael'e dönüp şöyle dedi:
\par 2 "Azarya kar­deş, yanına dört hizmetçiyle iki deve al ve Rages'e git.
\par 3 Gabael'in evine git, ona pusulayı ver ve para işiyle ilgilen. Onu düğüne davet et.
\par 4 Bili­yorsun ki babam günleri sayıyor ve her geçen gün ona sıkıntı veriyor.
\par 5 Raguel'in yapacaklarına ilişkin içtiği andı gördün. Onun andı elimi kolumu bağlıyor." Rafael Medya'ya, Rages'e gitmek üzere dört hizmetçi ve iki deveyle yola çıktı. Vardıklarında Gabael'in evinde kaldılar ve Rafael ona pu­sulayı verdi. Ona Tobit'in oğlu Tob­yas'ın evlendiğini söyledi ve onu düğüne çağırdı. Gabael torbaları Rafa­el'in önünde saymaya başladı, mühür olduğu gibi duruyordu, bütün torba­ları develere yüklediler.
\par 6 Sabahleyin erkenden şölene yetişmek için birlikte yola çıktılar ve Raguel'in evine vardı­lar. Tobyas yemek yiyordu. Ayağa kalkıp onu karşıladı. Gabael hüngür hüngür ağlamaya başladı ve şu sözler­le onu kutsadı: "Davranışlarında adil ve cömert olan kusursuz bir babanın erdemli oğlu! Tanrı seni, eşini, eşinin anne ve babasını kutsasın! Kuzenim Tobit'in bu canlı suretini bana göster­diği için Tanrı'ya hamdederim!"

\chapter{10}

\par 1 Bu arada Tobit, yolculuk için ge­reken günleri sayıyordu. Gün sa­yısı dolmuştu, ama oğlu hâlâ dönme­mişti.
\par 2 "Orada onu alıkoymadıklarını umut ederim! Gabael'in ölmediğini umut ederim! Belki de ona para verecek kimse yoktu" diye düşündü.
\par 3 Tobit ta­salanmaya başladı.
\par 4 Eşi Anna sürekli olarak şöyle diyordu: "Oğlum öldü! O artık yaşamıyor!" Ağlayıp oğlu için yas tuttu. Sürekli söyleniyordu:
\par 5 "Eyvah! Oğlum, gözümüzün nuru, izin verdim de gittin." Tobit de ona,
\par 6 "Sus, ba­cım! Böyle şeyler düşünme" diyordu, "Oğlumuz iyidir. Orada onu alıkoyan bir şey var. Onun arkadaşı akrabamızdır ve ona güvenebiliriz. Yürekli ol, ba­cım. O yakında dönecektir."
\par 7 Eşi şöy­le diyordu: "Beni rahat bırak ve aldat­maya kalkma. Çocuğum öldü." Her gün, aniden evden dışarı çıkıp oğlunun gi­derken izlediği yola bakıyordu. Ancak kendi gözlerine inanıyordu. Güneş batınca yeniden eve dönüyor, geceleri ağlayıp inliyor ve uyuyamıyordu. Raguel, kızının düğün şöleninin on dört gün süreceği konusunda ant içmişti. On dört gün geçtikten sonra Tobyas onun yanına gidip şöyle dedi: "Artık bırak da gideyim, annemle babam beni yeniden görme umudunu yitirmişlerdir. Baba, senden şunu dili­yorum, bırak da babamın yanına döneyim. Babamdan ayrıldığım zaman onun ne kötü durumda olduğunu sana anlattım."
\par 8 Raguel Tobyas'a şöyle dedi: "Kal oğlum, benimle kal. Baban Tobit'e ulaklar gönderirim, senin haberlerini ona iletirler."
\par 9 Ama Tobyas direndi: "Olmaz, babamın evine dönmek için senden izin istiyorum."
\par 10 Raguel ses çıkarmadı ve kızı Sara'yı koruması için Tobyas'a verdi. Tobyas'a ayrıca varlığının yarısını verdi. Ona erkek ve kadın hizmetçiler, öküzler ve koyun­lar, eşekler ve develer, giysiler, para ve ev eşyası verdi.
\par 11 Onların sevincle yola koyulmalarına izin verdi. Ayrılırken Tobyas'a şöyle dedi: "Esenlikler dilerim, oğlum, yolun açık olsun Göklerdeki Tanrı sana ve eşin Sara'ya sevecenlikle davransın! Ölmeden önce çocuklarını görmek benim için bir umut."
\par 12 Kızı Sara'ya da şöyle dedi: "Şimdi kaynatanın evine git, çünkü bundan böyle onlar, seni dünyaya getiren ailen kadar sana yakındır. Gönül rahatlığıyla git kızım, yaşadığım süre­ce seninle ilgili salt iyi şeyler duyma­yı umut ediyorum." Onlara güle güle dedi ve yola koyuldular. Edna da Tobyas'a şöyle dedi: "Sevgili oğlum ve kardeşim, Tanrı seni yeniden geri getirsin! Ölmeden önce se­nin ve kızım Sara'nın çocuklarını gö­receğimi, bunun için gerektiği kadar yaşayacağımı umut ediyorum. Tanrı­nın katında, kızımı koruman için sana veriyorum. Yaşadığın sürece onu hiç bir zaman mutsuz etme. Gönül rahat­lığıyla git, oğlum. Bundan böyle ben senin annenim ve Sara da kız kardeşindir. Hep birlikte mutlu günler ge­çirmenizi dilerim!" Her ikisini de öptü. Onlar da sevinçle yola koyuldu­lar.
\par 13 Tobyas Raguel'in evinden ayrı­lırken içi rahat etmişti. Kıvanç duya­rak gökyüzünün ve yeryüzünün Tan­rısı'na, bütün evrenin egemenine hamdetti. Yolculuğunun iyi bir şekilde so­nuçlanmasını diledi. Raguel'i ve eşi Edna'yı kutsadı. "Yaşadığım sürece sizi onurlandırmak benim mutluluğum olacaktır!" dedi.

\chapter{11}

\par 1 Ninova'nın karşısındaki Kaserin'e yaklaşırken,
\par 2 Rafael şöy­le dedi: "Babanı ne denli kötü durum da bıraktığımızı biliyorsun. 3Biz eşin­den önce gidelim ve evi hazırlayalım, o da öbürleriyle birlikte arkadan ge­lir."
\par 4 Birlikte ilerlediler. Rafael, safrakesesini alması için Tobyas'ı uyar­mıştı. Köpek de onları izledi.
\par 5 Anna oturmuş oğlunun gelirken izleyeceği yola bakıyordu.
\par 6 Kuşkusuz oğlunun geldiğine inandı ve Tobit'e şöyle dedi: "Oğlun arkadaşı ile bera­ber geliyor."
\par 7 Tobyas babasının yanına gitme­den önce Rafael ona şöyle dedi: "Sa­na söz veriyorum, babanın gözleri şi­fa bulacaktır.
\par 8 Balığın safrakesesini gözlerine koymalısın. İlaç acı verecek ve gözlerinden ince beyaz bir zar çı­kacak. Baban görebilecek ve ışığa ba­kabilecek."
\par 9 Tobyas'ın annesi ona doğru koş­tu, oğlunun boynuna sarıldı ve şöyle dedi: "Artık ölsem de gam yemem, çünkü seni yine gördüm." Ve gözyaşı döktü.
\par 10 Tobit ayağa kalktı, avluyu sen­deleyerek geçip kapıya geldi. Tobyas ona doğru ilerledi.
\par 11 Balığın safrake­sesini elinde tutuyordu. Babasının göz­lerine üfledi. Kımıldamamasını söyle­yerek şöyle dedi: "Yürekli ol, baba!" Sonra ilacı sürdü ve bir süre orada bı­raktı.
\par 12 Ardından iki eliyle, babasının gözlerinin kenarlarından başlayarak zar biçimindeki deriyi soydu.
\par 13 Baba­sı boynuna sarıldı,
\par 14 gözyaşı döktü. Birden, "Görebiliyorum, oğlum, gö­zümün nuru!" diye bağırdı ve ekledi: "Tanrı'ya övgüler olsun! Yüce adına hamdolsun! Yol gösteren kutsal meleklerin için sana hamdolsun! Yüce adına sonsuza dek övgüler olsun!
\par 15 Çünkü beni şiddetle cezalandırmıştı, Şimdi bana acıdı Ve oğlum Tobyas'ı görüyorum." Tobyas eve girdi, yüksek sesle ve sevinçle Tanrı'yı övdü. Ardından ba­basına her şeyi anlattı: Yolculuğunun başarıyla geçtiğini, gümüşü geri getir­diğini, Raguel'in kızı Sara ile evlen­diğini, eşinin kendisini izlediğini, ya­kında olduğunu ve Ninova'nın kapısı­na yaklaştığını söyledi.
\par 16 Tobit gelinini karşılamak üzere Ninova'nın kapısına doğru ilerledi, bir yandan da kıvançla Tanrı'yı övüyordu. Ninova halkı, Tobit'in yanında yol gösteren biri olmadan yürüdüğü­nü ve eskisi gibi canlı biçimde ilerle­diğini görünce şaşakaldı.
\par 17 Tobit, Tanrı'nın kendisine nasıl acıdığını ve gözlerine şifa verdiğini herkese anlat­tı. Oğlu Tobyas'ın eşi Sara'yı karşıla­dı ve onu şu sözlerle kutsadı: "Hoş geldin, kızım! Seni bize gönderdiği için Tanrı'ya hamdederiz. Babanı kut­sarım, oğlum Tobyas'ı kutsarım ve seni de kutsarım, kızım. Sevinçle ve kutsama ile kendi evine hoş geldin. İçeri gir, kızım." Tobit, Ninova'da otu­ran bütün Yahudiler için o gün bir şö­len verdi.
\par 18 Kuzenleri Akikar ve Nadab, Tobit'in mutluluğunu paylaşmak için geldiler.

\chapter{12}

\par 1 Şölen bitince Tobit oğlu Tob­yas'ı çağırıp şöyle dedi: "Oğ­lum, seninle yolculuk eden arkadaşına ödeyeceğin parayı düşünmen gerek. Ona kararlaştırdığımızdan fazlasını ver!"
\par 2 Tobyas, "Baba, yardımları için ona ne ödemem gerek?" diye sordu, "Geri getirdiğimiz malların yarısını ona versem bile, yine de kazançlı sa­yılırım.
\par 3 Esenlikle sana dönmemi sağ­ladı, eşimi iyileştirdi, parayı da getirdi ve şimdi de seni iyileştirdi. Tüm yap­tıkları için ona ne vereyim?"
\par 4 Tobit şöyle dedi: "Geri getirdiği malların yarısını fazlasıyla kazandı."
\par 5 Bunun üzerine Tobyas arkadaşını çağırıp şöy­le dedi: "Tüm yaptıklarının karşılığın­da geri getirdiğin malların yarısını üc­ret olarak alabilirsin ve gönül rahatlı­ğıyla gidebilirsin."
\par 6 O zaman Rafael her ikisini yanı­na çağırıp şöyle dedi: "Tanrı'ya hamdedin ve bütün canlıların önünde size verdiği iyilikler için O'nu övün. O'na hamdedin ve adını yüceltin. Tanrı'nın yaptıklarını hak ettiği gibi, bütün insanların önünde açığa vurun ve O'na şükretmekten asla usanmayın.
\par 7 Bir kralın gizini saklamak doğrudur, aynı biçimde, Tanrı'nın yaptıklarını bildir­mek ve açığa vurmak doğrudur. Ya­raştığı gibi O'na şükredin. Doğru ola­nı yapın ve size hiç bir kötülük gel­mez.
\par 8 "Dua edip oruç tutmak ve sadaka verip doğru olanı yapmak varlıklı olup günah işlemekten daha iyidir. Altın biriktirmektense sadaka vermek daha iyidir.
\par 9 Sadaka insanı ölümden kurta­rır ve her türlü günahtan temizler. Sa­daka verenlerin günleri boş geçmemiştir.
\par 10 Günah işleyip kötülük yapanlarsa kendilerine zarar verirler.
\par 11 "Size bütün gerçeği anlataca­ğım, sizden bir şey saklamayacağım. Bir kralın gizini saklamanın doğru ol­duğunu ve aynı şekilde, Tanrı'nın yaptıklarını yaraşır biçimde açıklamanın doğru olduğunu size daha önce söyle­miştim.
\par 12 Şunu bil ki, sen Sara ile birlikte dua ederken yakarışlarınızı gör kemli Tanrı'ya sunan ve onları oku­yan bendim. Ölüyü gömdüğünüz za­man da durum böyleydi.
\par 13 Bir ölüyü gömmek için duraksamadan sofradan kalktığın zaman, kutsal inancını dene­mek için ben gönderilmiştim.
\par 14 Aynı biçimde Tanrı, seni ve gelinin Sara'yı iyileştirmem için beni gönderdi.
\par 15 Ben Rafael'im. Tanrı'nın yüce katına çık­mak için her an hazır bekleyen yedi melekten* biriyim." *Eski ve Yeni Antlaşma'da bu yedi melekten üçünün sözü geçer: Cebrail, Mikael ve Rafael.
\par 16 Her ikisinin gönlü, karşı konul­maz korku ve saygı duygularıyla dol­du, ürküntü ile yüzüstü yere kapandı­lar.
\par 17 Ama melek şöyle dedi: "Kaygı­lanmayın, gönlünüz rahat olsun. Son­suza dek Tanrı'ya hamdedin.
\par 18 Bana ge­lince, sizlerle birlikte olmak konusun­daki yargıyı ben vermiş değilim, Tan­rı'nın buyruğu böyle belirdi. Yaşadı­ğınız sürece O'na hamdetmeli, O'nu övmelisiniz.
\par 19 Beni yemek yerken gör­düğünüzü sandınız. Ancak bu, oluşan bir görüntü idi, başka bir şey değildi.
\par 20 Şimdi yeryüzünde Rabbinize hamde­din ve Tanrı'ya şükredin. O yukarıdan beni buraya gönderdi, ben de O'na dön­mek üzereyim. Olup bitenleri yazın." Rafael yukarıya yükseldi.
\par 21 Yeniden ayağa kalktıkları zaman, Rafael artık görünmüyordu.
\par 22 Tanrı'yı ilahilerle övdüler, böyle mucizeler yaptığı için O'na şükrettiler. Çünkü Tanrı'nın bir meleği onlara görünmüştü!

\chapter{13}

\par 1 Ve Tobit şöyle dedi: "Sonsuza dek yaşayan Tanrı'ya hamdolsun, Çünkü O'nun saltanatı tüm çağlarda sürer!
\par 2 O, hem cezalandırır hem bağışlar, İnsanları ölüler ülkesinin sonsuzluğuna gönderir. Sonra onları oradan çekip alır, Sonunda yok eder, O'nun elinden kurtuluş yoktur.
\par 3 O'nu övdüğünüzü bütün uluslara duyurun, Sizler ki İsrailoğulları'sınız! Çünkü O, onların arasında yaşamanızı sağlayarak,
\par 4 Bu konuda da size yüceliğini göstermiştir. Tüm yaşayan varlıkların önünde O'nu övün! O Efendimizdir, Tanrımızdır, Babamızdır, sonsuza dek Tanrı'dır.
\par 5 Günahlarınız için sizi cezalandırmasına karşın, Hepinize acıyacaktır. Çeşitli uluslara yayılmış olan sizleri Bir araya toplayacaktır.
\par 6 Bütün yüreğinizle ve bütün canınızla O'na dönerseniz, O'na karşı dürüst davranırsanız, O da size dönecektir. Artık yüzünü sizden gizlemeyecektir. Size ne denli iyi davrandığını düşünün, O'na yüksek sesle şükredin, Herkesin hakkını gözeten Rabbinize hamdedin. O her çağda saltanat sürüyor, O'nu yüceltin. Ben, bu ülkede sürgündeyken O'nu övüyorum, Günah işlemiş bir ulusa O'nun gücünü ve yüceliğini bildiriyorum. Günah işleyenler, O'na dönün, O'na karşı tutumunuz dürüst olsun. Olabilir ki, size karşı sevecen davranıp size acıyacaktır.
\par 7 Bana gelince, Tanrı'yı övüyorum Ve ruhum göklerin egemeni için kıvanç duyuyor.
\par 8 İzin ver de herkes O'nun yüceliğini anlatsın, Yeruşalim'de övülsün.
\par 9 Yeruşalim, kutsal kent, Tanrı yaptığın işler için seni şiddetle cezalandırdı, Ama O, dürüst kişilerin oğullarına acıyacaktır.
\par 10 Yaraştığı gibi Tanrı'ya hamdet, O tüm çağlarda sürekli saltanat sürmüştür. Tapınağın sevinçle yeniden yapılsın, Senin içinde sürgünde olanları avutsun, Senin içinde bütün acı çekenleri sevsin, Gelecek bütün kuşaklar için, sürekli olarak.
\par 11 Dünyanın her bölgesi üzerinde Görkemli bir ışık parlayacak, Değişik uluslar uzaktan gelecekler, Dünyanın öbür ucundan gelecekler, Tanrı'nın kutsal adına yakın olmak için, Ellerinde armağanlarla Göklerin egemenine gelecekler. Senin içinde her yeni kuşak Sürekli olarak sevincini duyuracak, Seçilmiş olan bu kentin adı Gelecek kuşaklarda sonsuza dek yaşayacak.
\par 12 Seni aşağılayanlara lanet olsun, Seni yıkana, Duvarlarını parçalayana, Kalelerini yıkıp yerle bir edene, Evlerini yakana lanet olsun! Bu kenti kuran bütün kişiler sonsuza dek kutsansın!
\par 13 O zaman sen coşacaksın, Dürüst kişilerin oğulları için sevineceksin, Çünkü tümü toplanacak, Her çağın Rabbi'ne hamdedecektir.
\par 14 Seni sevenler mutludur, Sende barış olmasından sevinenler mutludur! Verdiğin cezalar karşısında tasalananlar mutludur! Çünkü yakında senin içinde kıvanç duyacaklar, Gelecek günlerde ne denli kutsandığına tanık olacaklar.
\par 15 Ruhum o yüce egemene, Tanrı'ya hamdediyor.
\par 16 Çünkü Yeruşalim ve onun evi Sonsuza dek yeniden yapılacaktır. Senin görkemini görüp göklerin egemenini övmek için Benim soyumdan bir kişi yaşarsa ne mutlu! Yeruşalim'in kapıları Safirden ve zümrütten yapılacak, Bütün duvarların değerli taşlarla kaplanacak. Yeruşalim'in kaleleri altından yapılacak, Kale burcundaki mazgallı siperler Som altından yapılacak.
\par 17 Yeruşalim'in sokakları Yakutla ve Ofir'den gelen taşlarla döşenecek, Yeruşalim'in kapıları Coşku dolu ezgilerle yankılanacak. Yeruşalim'deki bütün evlerin halkı şöyle diyecek: 'Haleluya! İsrail'in Tanrısı'na hamdolsun.' Yeruşalim'de sonsuza dek Tanrı'nın kutsal adını hamdederek anacaklar."
\par l8 Tobit'in övgüleri burada son buluyor.

\chapter{14}

\par 1 Tobit yüz on iki yaşındayken gönül rahatlığıyla öldü ve Ni­nova'da onurlu biçimde gömüldü.
\par 2 Kör olduğu zaman altmış iki yaşındaydı. Şifa bulduktan sonra rahat yaşadı. Sa­daka veriyor, sürekli olarak Tanrı'ya hamdediyor ve O'nun yüceliğini övü­yordu.
\par 3 Ölmek üzereyken oğlu Tobyas'ı çağırdı ve şöyle buyurdu:
\par 4 "Oğlum, çocuklarını al ve buradan uzaklaşıp Medya'ya git, çünkü Nahum'un bildirdiği gibi Ninova ile ilgili Tanrı'nın sözüne inanıyorum. Her şey böyle olacaktır, İsrail peygamberlerinin, Tanrı­nın elçilerinin, Asur ve Ninova hakkında önceden haber verdikleri şeyler gerçekleşecektir. Onların söylediği bir tek söz boş değildir. Her şey za­manında olacaktır. Medya'yı güvenli­ğin açısından Asur'a ve Babil'e yeğ tutarım. Çünkü bildiğim kadarıyla Tan­rı'nın bütün söyledikleri gerçekleşe­cektir. Her şey böyle olacaktır, önce­den haber verilen olayların tümü gerçekleşecektir. İsrail topraklarında yaşayan bütün kardeşlerimiz için nüfus sayımı yapılacaktır ve kendi güzel ülkelerinden çok uzaklara sürüleceklerdir. Bütün İsrail ülkesi, Samiriye ile Yeruşalim çöle dönüşecektir ve Tanrı'nın evi bir süre için yıkık duru­ma gelecek ve yakılacaktır.
\par 5 Bundan sonra Tanrı onlara yeniden acıyacak, İsrail ülkesine geri getirecektir. O'nun evini yeniden yapacaklar, ancak zamanı gelinceye dek bu ev ilki gibi gü­zel olmayacaktır. Ardından, hepsi tut­saklıktan kurtulup geri dönecek ve Yeruşalim'i bütün görkemiyle yeniden kuracaklar. İsrail peygamberleri­nin önceden bildirdiği gibi Tanrı'nın evi Yeruşalim'de yeniden yapılacak­tır.
\par 6 "Bütün insanlar değişip eksiksiz bir içtenlikle Tanrı'dan korkacaklar­dır. Onları yanlış yola götüren düzmece tanrıları reddedecekler.
\par 7 Dürüst davranarak bütün çağların Tanrısı'na hamdedecekler. O günlerde kurtulacak olan bütün İsrailliler, Tanrı'yı iç­tenlikle yürekten anımsayacaktır. On­lar Yeruşalim'de toplanacaklar ve ar­dından İbrahim'in ülkesinde güvenlik içerisinde yaşayacaklardır. Bu ülke on­ların olacaktır ve Tanrı'yı içtenlikle sevenler sevinecektir. Günah işleyip kötülük yapanlar yeryüzünden kaybo­lacaktır.
\par 8 "Şimdi, çocuklarım, size bir gö­rev veriyorum: Tanrı için içtenlikle çalışın ve O'nu sevindirecek biçimde davranın. Doğru olmayı, sadaka ver­meyi, Tanrı'yı anmayı ve sürekli ola­rak, içtenlikle ve bütün güçleriyle Tan­rı'ya hamdetmeyi çocuklarınıza zo­runlu kılın.
\par 9 Şimdi oğlum, Ninovadan git, burada kalma.
\par 10 "Annenizi benim yanıma gö­münce, hangi gün, hangi saatte olursa olsun, çabucak gidin ve bu ülkede da­ha fazla kalmayın. Ben bu ülkede yü­rek doğruluğundan yoksunluğun ve kötülüğün utanmayan zaferini görü­yorum. Nadav'ın, babalığı Akikar'a yaptıklarını düşün. Henüz yaşarken, yeraltına girmek zorunda kalmamış mıydı? Ama Tanrı, suçlunun kötü davranışının cezasını, kurbanının göz­leri önünde verdi. Çünkü Akikar gün ışığına geri döndü. Oysa Nadav, Akikar'ın yaşamına karşı kötü amaçlarla plan yapmanın cezasını çekerek son­suza dek karanlığa gömüldü. Yaptığı iyi işlerden dolayı Akikar, Nadav'ın ona hazırladığı öldürücü tuzaktan kurtuldu. Oysa Nadav'ın bu tuzağın içine düşmesi yıkımına neden oldu.
\par 11 O halde, çocuklarım, sadaka verme­nin sonucunu görüyorsunuz, kötülüğün ölüme neden olduğunu anlıyorsu­nuz. Ama soluğum kesildi." Onu yatağına yatırdılar. Öldü ve onurlu bir biçimde gömüldü.
\par 12 Anne­si ölünce Tobyas onu babasının yanı­na gömdü. Sonra eşi ve çocuklarıyla beraber Medya'ya gitti. Kaynatası Raguel'le Ekbatana'da yaşadı.
\par 13 Eşinin yaşlanan annesi ve babasına her türlü ilgiyi ve saygıyı gösterdi. Onları Medya'da, Ekbatana'da gömdü. Babası Tobit'in mirasından başka Raguel'in mi­rası da Tobyas'a kaldı.
\par 14 Kendisi çok onurlandırıldı ve yüz on yedi yaşına dek yaşamını sürdürdü.
\par 15 Ölmeden önce Ninova'nın yıkılışına tanık oldu. Medya Kralı Çaksares'in Ninova hal­kını tutsak ettiğini ve sınır dışı edip Medya'ya getirdiğini gördü. Ninova halkına ve Asurlular'a verdiği acılar için Tanrı'ya hamdetti. Ölmeden önce Ninova'nın sonu nedeniyle sevinmek olanağını buldu ve Tanrı'ya sonsuza dek hamdetti. Âmin.




\end{document}