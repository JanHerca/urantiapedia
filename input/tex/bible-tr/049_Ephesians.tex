\begin{document}

\title{Efesliler}


\chapter{1}

\par 1 Tanri'nin istegiyle Mesih Isa'nin elçisi atanan ben Pavlus'tan Efes'te bulunan kutsallara, Mesih Isa'ya ait olan sadiklara selam!
\par 2 Babamiz Tanri'dan ve Rab Isa Mesih'ten sizlere lütuf ve esenlik olsun. Mesih'te Sahip Oldugumuz Ayricaliklar
\par 3 Bizi Mesih'te her ruhsal kutsamayla göksel yerlerde kutsamis olan Rabbimiz Isa Mesih'in Babasi Tanri'ya övgüler olsun.
\par 4 O kendi önünde sevgide kutsal ve kusursuz olmamiz için dünyanin kurulusundan önce bizi Mesih'te seçti.
\par 5 Kendi istegi ve iyi amaci uyarinca Isa Mesih araciligiyla kendisine ogullar olalim diye bizi önceden belirledi.
\par 6 Öyle ki, sevgili Oglu'nda bize bagisladigi yüce lütfu övülsün.
\par 7 Tam bir bilgelik ve anlayisla üzerimize yagdirdigi lütfunun zenginligi sayesinde Mesih'in kani araciligiyla Mesih'te kurtulusa, suçlarimizin bagislanmasina kavustuk.
\par 9 Tanri sir olan istegini, Mesih'te edindigi iyi amaç uyarinca bize açikladi.
\par 10 Zaman dolunca gerçeklestirecegi bu tasariya göre, yerdeki ve gökteki her seyi Mesih'te birlestirecek.
\par 11 Her seyi kendi istegi dogrultusunda düzenleyen Tanri'nin amaci uyarinca önceden belirlenip Mesih'te seçildik.
\par 12 Öyle ki, Mesih'e ilk umut baglayan bizler, O'nun yüceliginin övülmesi için yasayalim.
\par 13 Gerçegin bildirisini, kurtulusunuzun Müjdesi'ni duyup O'na iman ettiginizde, siz de vaat edilen Kutsal Ruh'la O'nda mühürlendiniz.
\par 14 Ruh, Tanri'nin yüceliginin övülmesi için Tanri'ya ait olanlarin kurtulusuna dek mirasimizin güvencesidir.
\par 15 Bunun için, Rab Isa'ya iman ettiginizi ve bütün kutsallari sevdiginizi duydugumdan beri ben de sizin için sürekli sükrediyor, sizi dualarimda hep aniyorum.
\par 17 Rabbimiz Isa Mesih'in Tanrisi, yüce Baba, kendisini tanimaniz için size bilgelik ve vahiy ruhunu versin diye dua ediyorum.
\par 18 O'nun çagrisindan dogan umudu, kutsallara verdigi mirasin yüce zenginligini ve iman eden bizler için etkin olan kudretinin askin büyüklügünü anlamaniz için, yüreklerinizin gözleri aydinlansin diye dua ediyorum. Bu kudret, Tanri'nin, Mesih'i ölümden diriltirken ve göksel yerlerde saginda oturturken O'nda sergiledigi üstün güçle ayni etkinliktedir.
\par 21 Tanri O'nu bütün yönetimlerin, hükümranliklarin, güç ve egemenliklerin, yalniz bu çagda degil, gelecek çagda da anilacak bütün adlarin çok üstüne çikardi.
\par 22 Her seyi ayaklari altina sererek O'na bagimli kildi. O'nu her seyin üzerinde bas olmak üzere kiliseye* verdi.
\par 23 Kilise O'nun bedenidir, her yönden her seyi dolduranin dolulugudur.

\chapter{2}

\par 1 Sizler bir zamanlar içinde yasadiginiz suçlardan ve günahlardan ötürü ölüydünüz. Bu dünyanin gidisine ve havadaki hükümranligin egemenine, yani söz dinlemeyen insanlarda simdi etkin olan ruha uymaktaydiniz.
\par 3 Bir zamanlar hepimiz böyle insanlarin arasinda, benligin ve aklin isteklerini yerine getirerek benligimizin tutkularina göre yasiyorduk. Dogal olarak ötekiler gibi biz de gazap çocuklariydik.
\par 4 Ama merhameti bol olan Tanri bizi çok sevdigi için, suçlarimizdan ötürü ölü oldugumuz halde, bizi Mesih'le birlikte yasama kavusturdu. O'nun lütfuyla kurtuldunuz.
\par 6 Tanri bizi Mesih Isa'da, Mesih'le birlikte diriltip göksel yerlerde oturttu.
\par 7 Bunu, Mesih Isa'da bize gösterdigi iyilikle, lütfunun sonsuz zenginligini gelecek çaglarda sergilemek için yapti.
\par 8 Iman yoluyla, lütufla kurtuldunuz. Bu sizin basariniz degil, Tanri'nin armaganidir.
\par 9 Kimsenin övünmemesi için iyi islerin ödülü degildir.
\par 10 Çünkü biz Tanri'nin yapitiyiz, O'nun önceden hazirladigi iyi isleri yapmak üzere Mesih Isa'da yaratildik.
\par 11 Bunun için, öteki uluslardan* dogan sizler bir zamanlar ne oldugunuzu animsayin: Bedende elle yapilmis sünnete sahip olup "sünnetli" diye anilanlarin "sünnetsiz" dedikleri sizler,
\par 12 o zaman Mesihsiz, Israil'de vatandasliktan yoksun, vaade dayanan antlasmalara yabanci, dünyada umutsuz ve tanrisizdiniz.
\par 13 Ama bir zamanlar uzak olan sizler, simdi Mesih Isa'da Mesih'in kani sayesinde yakin kilindiniz.
\par 14 Çünkü Mesih'in kendisi barisimizdir. Kutsal Yasa'yi*, buyruklari ve kurallariyla birlikte etkisiz kilarak iki toplulugu birlestirdi, aradaki engel duvarini, yani düsmanligi kendi bedeninde yikti. Amaci bu iki topluluktan kendisinde yeni bir insan yaratarak esenligi saglamak, düsmanligi çarmihta öldürmek ve çarmih araciligiyla bir bedende iki toplulugu Tanri'yla baristirmakti.
\par 17 O gelip hem uzakta olan sizlere hem de yakindakilere esenligi müjdeledi.
\par 18 O'nun araciligiyla hepimiz tek Ruh'ta Baba'nin huzuruna çikabiliriz.
\par 19 Böylece artik yabanci ve garip degil, kutsallarla birlikte yurttas ve Tanri'nin ev halkisiniz.
\par 20 Elçilerle peygamberlerden olusan temel üzerine insa edildiniz. Köse tasi Mesih Isa'nin kendisidir.
\par 21 Bütün yapi Rab'be ait kutsal bir tapinak olmak üzere O'nda kenetlenip yükseliyor.
\par 22 Siz de Ruh araciligiyla Tanri'nin konutu olmak üzere hep birlikte Mesih'te insa ediliyorsunuz.

\chapter{3}

\par 1 Bu nedenledir ki, ben Pavlus siz uluslar ugruna Mesih Isa'nin tutuklusu oldum.
\par 2 Tanri'nin bana bagisladigi lütfu size ulastirmakla görevlendirildigimi duymussunuzdur.
\par 3 Yukarida kisaca degindigim gibi Tanri, sir olan tasarisini bana vahiy yoluyla bildirdi.
\par 4 Bu mektubu okudugunuzda Mesih sirrini nasil kavradigimi anlayabilirsiniz.
\par 5 Bu sir önceki kusaklara açikça bildirilmemisti. Simdiyse Mesih'in kutsal elçilerine ve peygamberlerine Ruh araciligiyla açiklanmis bulunuyor.
\par 6 Söyle ki, öteki uluslar* da mirasa ortaktir, ayni bedenin üyeleridir ve Müjde araciligiyla Mesih Isa'da vaade ortaktir.
\par 7 Tanri'nin etkin gücüyle bana verilen lütuf armagani uyarinca bu Müjde'yi yaymakla görevlendirildim.
\par 8 Bütün kutsallarin en degersiziydim. Yine de Mesih'in akil ermez zenginligini uluslara müjdeleme ve her seyi yaratan Tanri'da öncesizlikten beri gizli tutulan sirrin nasil düzenlendigini bütün insanlara açiklama ayricaligi bana verildi.
\par 10 Öyle ki, Tanri'nin çok yönlü bilgeligi, kilise* araciligiyla göksel yerlerdeki yönetimlere ve hükümranliklara simdiki dönemde bildirilsin.
\par 11 Bu, Tanri'nin baslangiçtan beri tasarladigi ve Rabbimiz Mesih Isa'da yerine getirdigi amaca uygundu.
\par 12 Mesih'te ve Mesih'e olan imanimizla Tanri'ya cesaret ve güvenle yaklasabiliriz.
\par 13 Bu nedenle, ugrunuza çektigim sikintilar karsisinda yilmamanizi rica ediyorum. Bunlar size yücelik kazandirir.
\par 14 Bunun için, yerde ve gökte her ailenin adini kendisinden aldigi Baba'nin önünde diz çökerim.
\par 16 Baba'nin kendi yüceliginin zenginligi uyarinca Ruhu'yla sizi iç varliginizda kudretle güçlendirmesini ve Mesih'in iman yoluyla yüreklerinizde yasamasini dilerim. Öyle ki, Tanri'nin bütün doluluguyla dolmaniz için, sevgide köklenmis ve temellenmis olarak bütün kutsallarla birlikte Mesih'in sevgisinin ne denli genis ve uzun, yüksek ve derin oldugunu anlamaya, bilgiyi çok asan bu sevgiyi kavramaya gücünüz yetsin.
\par 20 Tanri, bizde etkin olan kudretiyle, diledigimiz ya da düsündügümüz her seyden çok daha fazlasini yapabilecek güçtedir.
\par 21 Kilisede* ve Mesih Isa'da bütün kusaklar boyunca sonsuzlara dek O'na yücelik olsun! Amin.

\chapter{4}

\par 1 Bu nedenle, Rab'bin ugruna tutuklu olan ben, aldiginiz çagriya yarasir biçimde yasamanizi rica ederim.
\par 2 Her bakimdan alçakgönüllü, yumusak huylu, sabirli olun. Birbirinize sevgiyle, hosgörüyle davranin.
\par 3 Ruh'un birligini esenlik bagiyla korumaya gayret edin.
\par 4 Çagrinizdan dogan tek bir umuda çagrildiginiz gibi, beden bir, Ruh bir, Rab bir, iman bir, vaftiz* bir, her seyden üstün, her seyle ve her seyde olan herkesin Tanrisi ve Babasi birdir.
\par 7 Ama lütuf her birimize Mesih'in armagani ölçüsünde bagislandi.
\par 8 Bunun için Kutsal Yazi söyle der: "Yüksege çikti ve tutsaklari pesine takti, Insanlara armaganlar verdi."
\par 9 Simdi bu "çikti" sözcügü, Mesih önce asagilara, yeryüzüne indi demek degil de nedir?
\par 10 Inen de O'dur, her seyi doldurmak üzere bütün göklerin çok üstüne çikan da O'dur.
\par 11 Kendisi kimini elçi, kimini peygamber, kimini müjdeci, kimini önder ve ögretmen atadi.
\par 12 Öyle ki, kutsallar hizmet görevini yapmak ve Mesih'in bedenini gelistirmek üzere donatilsin.
\par 13 Sonunda hepimiz imanda ve Tanri Oglu'nu tanimada birlige, yetkinlige, Mesih dolulugundaki olgunluk düzeyine erisecegiz.
\par 14 Böylece artik insanlarin kurnazligiyla, aldatici düzenler kurmaktaki becerileriyle, her ögretinin rüzgariyla çalkalanip öteye beriye sürüklenen çocuklar olmayacagiz.
\par 15 Tersine, sevgiyle gerçege uyarak bedenin basi olan Mesih'e dogru her yönden büyüyecegiz.
\par 16 O'nun önderliginde bütün beden, her eklemin yardimiyla kenetlenip kaynasmis olarak her üyesinin düzenli isleyisiyle büyüyüp sevgide gelisiyor.
\par 17 Bunun için sunu söylüyor ve Rab adina sizi uyariyorum: Artik öteki uluslar* gibi bos düsüncelerle yasamayin.
\par 18 Onlarin zihinleri karardi. Bilgisizlikleri ve yüreklerinin duygusuzlugu yüzünden Tanri'nin yasamina yabancilastilar.
\par 19 Bütün duyarliliklarini yitirip açgözlülükle her türlü pisligi yapmak üzere kendilerini sefahate verdiler.
\par 20 Ama siz Mesih'i böyle ögrenmediniz.
\par 21 Kuskusuz Isa'nin sesini duydunuz, O'ndaki gerçege uygun olarak O'nun yolunda egitildiniz.
\par 22 Önceki yasayisiniza ait olup aldatici tutkularla yozlasan eski yaradilisi üzerinizden siyirip atmayi, düsüncede ve ruhta yenilenmeyi,
\par 24 gerçek dogruluk ve kutsallikta Tanri'ya benzer yaratilan yeni yaradilisi giyinmeyi ögrendiniz.
\par 25 Bunun için yalani üzerinizden siyirip atarak her biriniz komsusuna gerçegi söylesin. Çünkü hepimiz ayni bedenin üyeleriyiz.
\par 26 Öfkelenin, ama günah islemeyin. Öfkenizin üzerine günes batmasin.
\par 27 Iblis'e de firsat vermeyin.
\par 28 Hirsizlik eden artik hirsizlik etmesin. Tersine, kendi elleriyle iyi olani yaparak emek versin; böylece ihtiyaci olanla paylasacak bir seyi olsun.
\par 29 Agzinizdan hiç kötü söz çikmasin. Isitenler yararlansin diye, ihtiyaca göre, baskalarinin gelismesine yarayacak olani söyleyin.
\par 30 Tanri'nin Kutsal Ruhu'nu kederlendirmeyin. Kurtulus günü için o Ruh'la mühürlendiniz.
\par 31 Her kötü niyetle birlikte her türlü kin, öfke, kizginlik, bagrisma ve iftira sizden uzak olsun.
\par 32 Birbirinize karsi iyi yürekli, sefkatli olun. Tanri sizi Mesih'te bagisladigi gibi, siz de birbirinizi bagislayin.

\chapter{5}

\par 1 Bunun için, sevgili çocuklari olarak Tanri'yi örnek alin.
\par 2 Mesih bizi nasil sevdiyse ve bizim için kendisini güzel kokulu bir sunu ve kurban olarak nasil Tanri'ya sunduysa, siz de öylece sevgi yolunda yürüyün.
\par 3 Aranizda fuhus, pislik ya da açgözlülük anilmasin bile. Kutsallara yarasmaz bu.
\par 4 Aranizda açik saçiklik, budalaca konusmalar, bayagi sakalar da olmasin. Bunlar size yakismaz. Bunun yerine sükredin.
\par 5 Sunu kesinlikle bilin ki, fuhus yapanin, pislige düskün olanin ya da putperest demek olan açgözlü kisinin, Mesih'in ve Tanri'nin Egemenligi'nde mirasi yoktur.
\par 6 Hiç kimse sizi bos sözlerle aldatmasin. Bu seylerden ötürü Tanri'nin gazabi söz dinlemeyenlerin üzerine gelir.
\par 7 Onun için böyleleriyle oturup kalkmayin.
\par 8 Bir zamanlar karanliktiniz, ama simdi Rab'de isiksiniz. Isik çocuklari olarak yasayin.
\par 9 Çünkü isigin meyvesi her iyilikte, dogrulukta ve gerçekte görülür.
\par 10 Rab'bi neyin hosnut ettigini ayirt edin.
\par 11 Karanligin meyvesiz islerine katilmayin. Tersine, onlari açiga çikarin.
\par 12 Karanliktakilerin gizlice yaptiklarindan söz etmek bile ayiptir.
\par 13 Isigin açiga vurdugu her sey görünür.
\par 14 Çünkü görünen her sey isiktir. Bunun için söyle deniyor: "Uyan, ey uyuyan! Ölümden diril! Mesih sana isik saçacak."
\par 15 Öyleyse nasil yasadiginiza çok dikkat edin. Bilgelikten yoksun olanlar gibi degil, bilgeler gibi yasayin.
\par 16 Firsati degerlendirin. Çünkü yasadigimiz günler kötüdür.
\par 17 Bunun için akilsiz olmayin, Rab'bin isteginin ne oldugunu anlayin.
\par 18 Sarapla sarhos olmayin, bu sizi sefahate götürür. Bunun yerine Ruh'la dolun:
\par 19 Birbirinize mezmurlar, ilahiler, ruhsal ezgiler söyleyin; yürekten Rab'be ezgiler, mezmurlar okuyun;
\par 20 durmadan, her sey için Rabbimiz Isa Mesih'in adiyla Baba Tanri'ya sükredin;
\par 21 Mesih'e duydugunuz saygidan ötürü birbirinize bagimli olun.
\par 22 Ey kadinlar, Rab'be bagimli oldugunuz gibi, kocalariniza bagimli olun.
\par 23 Çünkü Mesih bedenin kurtaricisi olarak kilisenin* basi oldugu gibi, erkek de kadinin basidir.
\par 24 Kilise Mesih'e bagimli oldugu gibi, kadinlar da her durumda kocalarina bagimli olsunlar.
\par 25 Ey kocalar, Mesih kiliseyi nasil sevip onun ugruna kendini feda ettiyse, siz de karilarinizi öyle sevin.
\par 26 Mesih kiliseyi suyla yikayip tanrisal sözle temizleyerek kutsal kilmak için kendini feda etti.
\par 27 Öyle ki, kiliseyi üzerinde leke, burusukluk ya da buna benzer bir sey olmadan, görkemli biçimde kendine sunabilsin. Amaci kilisenin kutsal ve kusursuz olmasidir.
\par 28 Ayni biçimde kocalar da karilarini kendi bedenleri gibi sevmelidir. Karisini seven kendini sever.
\par 29 Hiç kimse hiçbir zaman kendi bedeninden nefret etmemistir. Tersine, onu besler ve kayirir; tipki Mesih'in kiliseyi besleyip kayirdigi gibi.
\par 30 Çünkü bizler O'nun bedeninin üyeleriyiz.
\par 31 "Bunun için adam annesini babasini birakip karisina baglanacak, ikisi tek beden olacak."
\par 32 Bu sir büyüktür; ben bunu Mesih ve kiliseyle ilgili olarak söylüyorum.
\par 33 Size gelince, her biriniz karisini kendisi gibi sevsin. Kadin da kocasina saygi göstersin.

\chapter{6}

\par 1 Ey çocuklar, Rab yolunda anne babanizin sözünü dinleyin. Çünkü dogrusu budur.
\par 2 "Iyilik bulmak, yeryüzünde uzun ömürlü olmak için annene babana saygi göstereceksin." Vaat içeren ilk buyruk budur.
\par 4 Ey babalar, siz de çocuklarinizin öfkesini uyandirmayin. Onlari Rab'bin terbiye ve ögüdüyle büyütün.
\par 5 Ey köleler, dünyadaki efendilerinizin sözünü Mesih'in sözünü dinler gibi saygi ve korkuyla, saf yürekle dinleyin.
\par 6 Bunu, yalniz insanlari hosnut etmek isteyenler gibi göze hos görünmek için yapmayin. Mesih'in kullari olarak Tanri'nin istegini candan yerine getirin.
\par 7 Insanlara degil, Rab'be hizmet eder gibi gönülden hizmet edin.
\par 8 Çünkü ister köle ister özgür olsun, herkesin yaptigi her iyiligin karsiligini Rab'den alacagini biliyorsunuz.
\par 9 Ey efendiler, siz de kölelerinize ayni biçimde davranin. Artik onlari tehdit etmeyin. Onlarin da sizin de Efendiniz'in göklerde oldugunu ve insanlar arasinda ayrim yapmadigini biliyorsunuz.
\par 10 Son olarak Rab'de, O'nun üstün gücüyle güçlenin.
\par 11 Iblis'in hilelerine karsi durabilmek için Tanri'nin sagladigi bütün silahlari kusanin.
\par 12 Çünkü savasimiz insanlara karsi degil, yönetimlere, hükümranliklara, bu karanlik dünyanin güçlerine, kötülügün göksel yerlerdeki ruhsal ordularina karsidir.
\par 13 Bu nedenle, kötü günde dayanabilmek, gerekli her seyi yaptiktan sonra yerinizde durabilmek için Tanri'nin bütün silahlarini kusanin.
\par 14 Böylece, belinizi gerçekle kusatmis, gögsünüze dogruluk zirhini takmis ve ayaklariniza esenlik Müjdesi'ni yayma hazirligini giymis olarak yerinizde durun.
\par 16 Bunlarin hepsine ek olarak, Seytan'in bütün atesli oklarini söndürebileceginiz iman kalkanini alin.
\par 17 Kurtulus migferini ve Ruh'un kilicini, yani Tanri sözünü alin.
\par 18 Her türlü dua ve yalvarisla, her zaman Ruh'un yönetiminde dua edin. Bu amaçla, bütün kutsallar için yalvarista bulunarak tam bir adanmislikla uyanik durun.
\par 19 Agzimi her açtigimda bana gerekli söz verilsin diye benim için de dua edin; öyle ki, Müjde'nin sirrini cesaretle bildirebileyim.
\par 20 Ugruna zincire vurulmus durumda elçilik ettigim Müjde'yi gerektigi gibi cesaretle duyurabilmem için dua edin.
\par 21 Nasil oldugumu, ne yaptigimi sizin de bilmeniz için sevgili kardesimiz, Rab'bin güvenilir hizmetkâri Tihikos size her seyi bildirecektir.
\par 22 22 Kendisini bu amaçla, durumumuzu iletmesi ve yüreklerinize cesaret vermesi için size gönderiyorum.
\par 23 Baba Tanri'dan ve Rab Isa Mesih'ten kardeslere imanla birlikte esenlik ve sevgi diliyorum.
\par 24 Tanri'nin lütfu Rabbimiz Isa Mesih'i ölümsüz sevgiyle sevenlerin hepsiyle birlikte olsun.


\end{document}