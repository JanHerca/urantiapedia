\begin{document}

\title{Galatyalılar}


\chapter{1}

\par 1 Insanlarca ya da insan araciligiyla degil, Isa Mesih ve O'nu ölümden dirilten Baba Tanri araciligiyla elçi atanan ben Pavlus'tan ve benimle birlikte olan bütün kardeslerden Galatya'daki kiliselere* selam!
\par 3 Babamiz Tanri'dan ve Rab Isa Mesih'ten sizlere lütuf ve esenlik olsun.
\par 4 Mesih, Babamiz Tanri'nin istegine uyarak bizi simdiki kötü çagdan kurtarmak için günahlarimiza karsilik kendini feda etti.
\par 5 Tanri'ya sonsuzlara dek yücelik olsun! Amin.
\par 6 Sizi Mesih'in lütfuyla çagirani birakip degisik bir müjdeye böylesine çarçabuk dönmenize sasiyorum.
\par 7 Gerçekte baska bir müjde yoktur. Ancak aklinizi karistirip Mesih'in Müjdesi'ni çarpitmak isteyenler vardir.
\par 8 Ister biz, ister gökten bir melek size bildirdigimize ters düsen bir müjde bildirirse, lanet olsun ona!
\par 9 Daha önce söyledigimizi simdi yine söylüyorum: Bir kimse size kabul ettiginize ters düsen bir müjde bildirirse, ona lanet olsun!
\par 10 Simdi ben insanlarin onayini mi, Tanri'nin onayini mi ariyorum? Yoksa insanlari mi hosnut etmeye çalisiyorum? Eger hâlâ insanlari hosnut etmek isteseydim, Mesih'in kulu olmazdim.
\par 11 Kardeslerim, yaydigim Müjde'nin insandan kaynaklanmadigini bilmenizi istiyorum.
\par 12 Çünkü ben onu insandan almadim, kimseden de ögrenmedim. Bunu bana Isa Mesih vahiy yoluyla açikladi.
\par 13 Yahudi dinine bagli oldugum zaman nasil bir yasam sürdügümü duydunuz. Tanri'nin kilisesine* alabildigine zulmediyor, onu kirip geçiriyordum.
\par 14 Yahudi dininde yasitim olan soydaslarimin birçogundan daha ilerideydim, atalarimin geleneklerini savunmakta çok daha gayretliydim.
\par 15 Ama beni daha annemin rahmindeyken seçip lütfuyla çagiran Tanri, uluslara müjdelemem için Oglu'nu bana göstermeye razi olunca hemen insanlara danismadim;
\par 17 Yerusalim'e*, benden önce elçi olanlarin yanina da gitmedim; Arabistan'a gittim, sonra yine Sam'a döndüm.
\par 18 Bundan üç yil sonra Kefas'la* tanismak üzere Yerusalim'e gittim, on bes gün onun yaninda kaldim.
\par 19 Öbür elçilerden hiçbirini görmedim, yalniz Rab Isa'nin kardesi Yakup'u gördüm.
\par 20 Bakin, size yazdiklarimin yalan olmadigini Tanri'nin önünde belirtiyorum.
\par 21 Sonra Suriye ve Kilikya bölgelerine gittim.
\par 22 Yahudiye'nin Mesih'e ait kiliseleri beni sahsen tanimiyorlardi.
\par 23 Yalniz, "Bir zamanlar bize zulmeden adam, önceleri yikmaya çalistigi imani simdi yayiyor" dendigini duymuslardi.
\par 24 Böylece benden ötürü Tanri'yi yüceltiyorlardi.

\chapter{2}

\par 1 On dört yil aradan sonra Titus'u da yanima alip Barnaba'yla birlikte yine Yerusalim'e gittim.
\par 2 Vahiy uyarinca gittim. Bos yere kosmayayim ya da kosmus olmayayim diye, öteki uluslar* arasinda yaydigim Müjde'yi özel olarak ileri gelenlere sundum.
\par 3 Benimle birlikte olan Titus bile Grek* olmasina karsin sünnet edilmeye zorlanmadi.
\par 4 Ne var ki, Isa Mesih'te sahip oldugumuz özgürlügü el altindan ögrenmek ve böylece bizi kölelestirmek için gizlice aramiza sizan sahte kardesler vardi.
\par 5 Müjde gerçegi sürekli sizinle kalsin diye bir an bile onlara boyun egip teslim olmadik.
\par 6 Ama ileri gelenler -ne olduklari bence önemli degil, Tanri insanlar arasinda ayrim yapmaz- evet, bu ileri gelenler söylediklerime bir sey katmadilar.
\par 7 Tam tersine, Müjde'yi sünnetlilere* bildirme isi nasil Petrus'a verildiyse, sünnetsizlere* bildirme isinin de bana verildigini gördüler.
\par 8 Çünkü sünnetlilere elçilik etmesi için Petrus'ta etkin olan Tanri, öteki uluslara elçilik etmem için bende de etkin oldu.
\par 9 Toplulugun direkleri sayilan Yakup, Kefas* ve Yuhanna bana bagislanan lütfu sezince paydasligimizin isareti olarak bana ve Barnaba'ya sag ellerini uzattilar. Öteki uluslara bizlerin, Yahudiler'e kendilerinin gitmesini uygun gördüler.
\par 10 Ancak yoksullari animsamamizi istediler. Zaten ben de bunu yapmaya gayret ediyordum.
\par 11 Ne var ki, Kefas* Antakya'ya geldigi zaman, suçlu oldugu için ona açikça karsi geldim.
\par 12 Çünkü Yakup'un yanindan bazi adamlar gelmeden önce Kefas öteki uluslardan* olanlarla birlikte yemek yerdi. Ama o adamlar gelince sünnet yanlilarindan korkarak sünnetsizlerden* uzaklasti, onlarla yemek yemez oldu.
\par 13 Öbür Yahudiler de onun gibi ikiyüzlülük ettiler. Sonunda Barnaba bile onlarin ikiyüzlülügüne kapildi.
\par 14 Müjde gerçegine uygun davranmadiklarini görünce hepsinin önünde Kefas'a söyle dedim: "Yahudi oldugun halde Yahudi gibi degil, öteki uluslardan biri gibi yasiyorsun, nasil olur da uluslari Yahudi gibi yasamaya zorlarsin?
\par 15 Dogustan Yahudi olan bizler öteki uluslardan olan `günahlilar> degiliz.
\par 16 Yine de insanin Kutsal Yasa'nin* gereklerini yaparak degil, Isa Mesih'e iman ederek aklandigini biliyoruz. Bunun için biz de Yasa'nin gereklerini yaparak degil, Mesih'e iman ederek aklanalim diye Mesih Isa'ya iman ettik. Çünkü hiç kimse Yasa'nin gereklerini yaparak aklanmaz.
\par 17 Mesih'te aklanmak isterken kendimiz günahli çikarsak, Mesih günahin yardakçisi mi olur? Kesinlikle hayir!
\par 18 Yiktigimi yeniden kurarsam, yasayi çignedigimi kanitlamis olurum.
\par 19 Çünkü ben Tanri için yasamak üzere Yasa araciligiyla Yasa karsisinda öldüm.
\par 20 Mesih'le birlikte çarmiha gerildim. Artik ben yasamiyorum, Mesih bende yasiyor. Simdi bedende sürdürdügüm yasami, beni seven ve benim için kendini feda eden Tanri Oglu'na imanla sürdürüyorum.
\par 21 Tanri'nin lütfunu geçersiz saymis degilim. Çünkü aklanma Yasa araciligiyla saglanabilseydi, o zaman Mesih bos yere ölmüs olurdu."

\chapter{3}

\par 1 Ey akilsiz Galatyalilar! Sizi kim büyüledi? Isa Mesih çarmiha gerilmis olarak gözlerinizin önünde tasvir edilmedi mi?
\par 2 Sizden yalniz sunu ögrenmek istiyorum: Kutsal Ruh'u, Yasa'nin gereklerini yaparak mi, yoksa duyduklariniza iman ederek mi aldiniz?
\par 3 Bu kadar akilsiz misiniz? Ruh'la basladiktan sonra simdi insan çabasiyla mi bitirmeye çalisiyorsunuz?
\par 4 Bos yere mi bu kadar aci çektiniz? Gerçekten bosuna miydi?
\par 5 Size Kutsal Ruh'u veren ve aranizda mucizeler yaratan Tanri, bunu Yasa'nin gereklerini yaptiginiz için mi, yoksa duyduklariniza iman ettiginiz için mi yapiyor?
\par 6 Örnegin, "Ibrahim Tanri'ya iman etti, böylece aklanmis sayildi."
\par 7 Öyleyse sunu bilin ki, Ibrahim'in gerçek ogullari iman edenlerdir.
\par 8 Kutsal Yazi, Tanri'nin öteki uluslari* imanlarina göre aklayacagini önceden görerek Ibrahim'e, "Bütün uluslar senin araciliginla kutsanacak" müjdesini önceden verdi.
\par 9 Böylece iman edenler, iman etmis olan Ibrahim'le birlikte kutsanirlar.
\par 10 Yasa'nin gereklerini yapmis olmaya güvenenlerin hepsi lanet altindadir. Çünkü söyle yazilmistir: "Yasa Kitabi'nda yazili olan her seyi sürekli yerine getirmeyen herkes lanetlidir."
\par 11 Tanri katinda hiç kimsenin Yasa'yla aklanmadigi açiktir. Çünkü "Imanla aklanan yasayacaktir."
\par 12 Yasa imana dayali degildir. Tersine, "Yasa'nin gereklerini yapan, onlar sayesinde yasayacaktir."
\par 13 Ibrahim'e saglanan kutsama Mesih Isa araciligiyla uluslara saglansin ve bizler vaat edilen Ruh'u imanla alalim diye, Mesih bizim için lanetlenerek bizi Yasa'nin lanetinden kurtardi. Çünkü, "Agaç üzerine asilan herkes lanetlidir" diye yazilmistir.
\par 15 Kardesler, insan yasamindan bir örnek vereyim. Insanlar arasinda yapilmis bile olsa, onaylanmis bir antlasmayi kimse geçersiz saymaz, ona bir sey eklemez.
\par 16 Vaatler Ibrahim'e ve soyundan olana verildi. Tanri birçok kisiden söz ediyormus gibi, "Ve soyundan olanlara" demiyor; "Soyundan olana" demekle tek bir kisiden, yani Mesih'ten söz ediyor.
\par 17 Sunu demek istiyorum: Dört yüz otuz yil sonra gelen Yasa, Tanri'nin önceden onayladigi antlasmayi geçersiz kilmaz, vaadi ortadan kaldirmaz.
\par 18 Çünkü miras Yasa'ya bagliysa, artik vaade bagli degildir. Ama Tanri mirasi Ibrahim'e vaatle bagislamistir.
\par 19 Öyleyse Yasa'nin amaci neydi? Yasa suçlari ortaya çikarmak için antlasmaya eklendi. Vaadi alan ve Ibrahim'in soyundan olan Kisi gelene dek yürürlükte kalacakti. Melekler yoluyla, bir araci eliyle düzenlendi.
\par 20 Araci tek bir tarafa ait degildir; Tanri ise birdir.
\par 21 Öyleyse Kutsal Yasa Tanri'nin vaatlerine aykiri midir? Kesinlikle hayir! Çünkü yasam saglayabilen bir yasa verilseydi, elbette insanlar yasayla aklanirdi.
\par 22 Oysa Isa Mesih'e olan imana dayanan vaat iman edenlere verilsin diye, Kutsal Yazi bütün dünyayi günahin tutsagi ilan ediyor.
\par 23 Bu iman gelmeden önce Yasa altinda hapsedilmistik, gelecek iman açiklanincaya dek Yasa'nin tutuklusuyduk.
\par 24 Yani imanla aklanalim diye Mesih'in gelisine dek Yasa egitmenimiz oldu.
\par 25 Ama iman gelmis oldugundan, artik Yasa'nin denetiminde degiliz.
\par 26 Çünkü Mesih Isa'ya iman ettiginiz için hepiniz Tanri'nin ogullarisiniz.
\par 27 Vaftizde* Mesih'le birlesenlerinizin hepsi Mesih'i giyindi.
\par 28 Artik ne Yahudi ne Grek*, ne köle ne özgür, ne erkek ne disi ayrimi var. Hepiniz Mesih Isa'da birsiniz.
\par 29 Eger Mesih'e aitseniz, Ibrahim'in soyundansiniz, vaade göre de mirasçisiniz.

\chapter{4}

\par 1 Sunu demek istiyorum: Mirasçi her seyin sahibiyse de, çocuk oldugu sürece köleden farksizdir.
\par 2 Babasinin belirledigi zamana dek vasilerin, vekillerin gözetimi altindadir.
\par 3 Bunun gibi, biz de ruhsal yönden çocukken, dünyanin temel ilkelerine bagli yasayan kölelerdik.
\par 4 Ama zaman dolunca Tanri, Yasa altinda olanlari özgürlüge kavusturmak için kadindan dogan, Yasa altinda dogan öz Oglu'nu gönderdi. Öyle ki, bizler ogulluk hakkini alalim.
\par 6 Ogullar oldugunuz için Tanri öz Oglu'nun "Abba! Baba!" diye seslenen Ruhu'nu yüreklerinize gönderdi.
\par 7 Bu nedenle artik köle degil, ogullarsiniz. Ogullar oldugunuz için de Tanri sizi ayni zamanda mirasçi yapti.
\par 8 Ne var ki, eskiden Tanri'yi tanimadiginiz zamanlarda, gerçek olmayan tanrilara kölelik ettiniz.
\par 9 Simdiyse Tanri'yi tanidiniz, daha dogrusu Tanri tarafindan tanindiniz. Öyleyse nasil oluyor da bu degersiz, etkisiz ilkelere dönüyorsunuz? Yeniden onlarin kölesi mi olmak istiyorsunuz?
\par 10 Özel günler, aylar, mevsimler, yillar kutluyorsunuz!
\par 11 Sizin için korkuyorum. Yoksa ugrunuza bos yere mi emek verdim?
\par 12 Kardesler, size yalvariyorum, benim gibi olun. Çünkü ben de sizin gibi oldum. Bana hiç haksizlik etmediniz.
\par 13 Bildiginiz gibi, Müjde'yi size ilk kez bedensel hastaligim nedeniyle bildirmistim.
\par 14 Bedensel durumum sizin için çetin bir deneme oldugu halde beni ne hor gördünüz ne de reddettiniz. Tanri'nin bir melegini, hatta Mesih Isa'yi kabul eder gibi kabul ettiniz beni.
\par 15 Simdi o sevincinize ne oldu? Sizin için taniklik ederim ki, elinizden gelse gözlerinizi oyar bana verirdiniz.
\par 16 Peki, size gerçegi söyledigim için düsmaniniz mi oldum?
\par 17 Baskalari sizi kazanmaya gayret ediyor, ama niyetleri iyi degil. Kendileri için gayret edesiniz diye sizi bizden ayirmak istiyorlar.
\par 18 Niyet iyiyse, yalniz aranizda oldugum zaman degil, her zaman gayretli olmak iyidir.
\par 19 Çocuklarim! Mesih sizde biçimleninceye dek sizin için yine dogum agrisi çekiyorum.
\par 20 Simdi yaninizda bulunmayi ve sesimin tonunu degistirmeyi isterdim. Bu halinize sasiyorum!
\par 21 Kutsal Yasa altinda yasamak isteyen sizler, söyleyin bana, Yasa'nin ne dedigini bilmiyor musunuz?
\par 22 Ibrahim'in biri köle, biri de özgür kadindan iki oglu oldugu yazilidir.
\par 23 Köle kadindan olan olagan yoldan, özgür kadindan olansa vaat sonucu dogdu.
\par 24 Burada bir benzetme vardir. Bu kadinlar iki antlasmayi simgelemektedir. Biri Sina Dagi'ndandir, köle olacak çocuklar dogurur. Bu Hacer'dir.
\par 25 Hacer, Arabistan'daki Sina Dagi'ni simgeler. Simdiki Yerusalim'in karsiligidir. Çünkü çocuklariyla birlikte kölelik etmektedir.
\par 26 Oysa göksel Yerusalim özgürdür, annemiz odur.
\par 27 Nitekim söyle yazilmistir: "Sevin, çocuk dogurmayan ey kisir kadin! Dogum agrisi nedir bilmeyen sen, Yükselt sesini, haykir! Çünkü terk edilmis kadinin, Kocasi olandan daha çok çocugu var."
\par 28 Kardesler, Ishak gibi sizler de vaat çocuklarisiniz.
\par 29 Olagan yoldan dogan, Kutsal Ruh'a göre dogana o zaman nasil zulmettiyse, simdi de öyle oluyor.
\par 30 Ama Kutsal Yazi ne diyor? "Köle kadinla oglunu kov. Çünkü köle kadinin oglu Özgür kadinin ogluyla birlikte Asla mirasa ortak olmayacaktir."
\par 31 Iste böyle, kardesler, bizler köle kadinin degil, özgür kadinin çocuklariyiz.

\chapter{5}

\par 1 Mesih bizi özgür olalim diye özgür kildi. Bunun için dayanin. Bir daha kölelik boyunduruguna girmeyin.
\par 2 Bakin, ben Pavlus size diyorum ki, sünnet olursaniz Mesih'in size hiç yarari olmaz.
\par 3 Sünnet edilen her adami bir daha uyariyorum: Kutsal Yasa'nin tümünü yerine getirmek zorundadir.
\par 4 Yasa araciligiyla aklanmaya çalisan sizler Mesih'ten ayrildiniz, Tanri'nin lütfundan uzak düstünüz.
\par 5 Ama biz aklanmanin verdigi umudun gerçeklesmesini Ruh'a dayanarak, imanla bekliyoruz.
\par 6 Mesih Isa'da ne sünnetliligin ne de sünnetsizligin yarari vardir; yararli olan, sevgiyle etkisini gösteren imandir.
\par 7 Iyi kosuyordunuz. Sizi gerçege uymaktan kim alikoydu?
\par 8 Buna kanmaniz sizi çagiranin istegi degildir.
\par 9 "Azicik maya bütün hamuru kabartir."
\par 10 Baska türlü düsünmeyeceginize iliskin Rab'de size güvenim var. Ama aklinizi karistiran kim olursa olsun, cezasini çekecektir.
\par 11 Bana gelince, kardesler, eger hâlâ sünneti savunuyor olsaydim, bugüne dek baski görür müydüm? Öyle olsaydi, çarmih engeli ortadan kalkardi.
\par 12 Aklinizi çelenler keske kendilerini hadim etseler!
\par 13 Kardesler, siz özgür olmaya çagrildiniz. Ancak özgürlük benlik için firsat olmasin. Birbirinize sevgiyle hizmet edin.
\par 14 Bütün Kutsal Yasa tek bir sözde özetlenmistir: "Komsunu kendin gibi seveceksin."
\par 15 Ama birbirinizi isirip yiyorsaniz, dikkat edin, birbirinizi yok etmeyesiniz!
\par 16 Sunu demek istiyorum: Kutsal Ruh'un yönetiminde yasayin. O zaman benligin tutkularini asla yerine getirmezsiniz.
\par 17 Çünkü benlik Ruh'a, Ruh da benlige aykiri olani arzular. Bunlar birbirine karsittir; sonuç olarak, istediginizi yapamiyorsunuz.
\par 18 Ruh'un yönetimindeyseniz, Yasa'ya bagimli degilsiniz.
\par 19 Benligin isleri bellidir. Bunlar fuhus, pislik, sefahat, putperestlik, büyücülük, düsmanlik, çekisme, kiskançlik, öfke, bencil tutkular, ayriliklar, bölünmeler, çekememezlik, sarhosluk, çilgin eglenceler ve benzeri seylerdir. Sizi daha önce uyardigim gibi yine uyariyorum, böyle davrananlar Tanri Egemenligi'ni miras alamayacaklar.
\par 22 Ruh'un ürünüyse sevgi, sevinç, esenlik, sabir, sefkat, iyilik, baglilik, yumusak huyluluk ve özdenetimdir. Bu tür nitelikleri yasaklayan yasa yoktur.
\par 24 Mesih Isa'ya ait olanlar, benligi, tutku ve arzulariyla birlikte çarmiha germislerdir.
\par 25 Ruh sayesinde yasiyorsak, Ruh'un izinde yürüyelim.
\par 26 Bos yere övünen, birbirine meydan okuyan, birbirini kiskanan kisiler olmayalim.

\chapter{6}

\par 1 Kardesler, eger biri suç islerken yakalanirsa, ruhsal olan sizler, böyle birini yumusak ruhla yola getirin. Siz de ayartilmamak için kendinizi kollayin.
\par 2 Birbirinizin yükünü tasiyin, böylece Mesih'in Yasasi'ni yerine getirirsiniz.
\par 3 Kisi bir hiçken kendini bir sey saniyorsa, kendini aldatmis olur.
\par 4 Herkes kendi yaptiklarini denetlesin. O zaman baskasinin yaptiklariyla degil, yalniz kendi yaptiklariyla övünebilir.
\par 5 Herkes kendine düsen yükü tasimali.
\par 6 Tanri sözünde egitilen, kendisini egitenle bütün nimetleri paylassin.
\par 7 Aldanmayin, Tanri alaya alinmaz. Insan ne ekerse onu biçer.
\par 8 Kendi benligine eken, benlikten ölüm biçecektir. Ruh'a eken, Ruh'tan sonsuz yasam biçecektir.
\par 9 Iyilik yapmaktan usanmayalim. Gevsemezsek mevsiminde biçeriz.
\par 10 Bunun için firsatimiz varken herkese, özellikle iman ailesinin üyelerine iyilik yapalim.
\par 11 Bakin, size kendi elimle ne denli büyük harflerle yaziyorum!
\par 12 Bedende gösterise önem verenler, yalniz Mesih'in çarmihi ugruna zulüm görmemek için sizi sünnet olmaya zorluyorlar.
\par 13 Oysa sünnetlilerin kendileri bile Kutsal Yasa'yi yerine getirmiyor, sizin bedenlerinizle övünebilmek için sünnet olmanizi istiyorlar.
\par 14 Bana gelince, Rabbimiz Isa Mesih'in çarmihindan baska bir seyle asla övünmem. O'nun çarmihi araciligiyla dünya benim için ölüdür, ben de dünya için.
\par 15 Sünnetli olup olmamanin önemi yoktur, önemli olan yeni yaratilistir.
\par 16 Bu kurala uyan herkese ve Tanri'nin Israili'ne esenlik ve merhamet olsun.
\par 17 Bundan böyle kimse bana sorun çikarmasin. Çünkü ben Isa'nin yara izlerini bedenimde tasiyorum.
\par 18 Kardesler, Rabbimiz Isa Mesih'in lütfu ruhunuzla birlikte olsun! Amin.


\end{document}