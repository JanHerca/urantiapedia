\begin{document}

\title{1 Korintliler}


\chapter{1}

\par 1 Tanri'nin istegiyle Mesih Isa'nin elçisi olmaya çagrilan ben Pavlus ve kardesimiz Sostenis'ten Tanri'nin Korint'teki kilisesine* selam! Mesih Isa'da kutsal kilinmis, kutsal olmaya çagrilmis olan sizlere ve hepimizin Rabbi Isa Mesih'in adini her yerde anan herkese Babamiz Tanri'dan ve Rab Isa Mesih'ten lütuf ve esenlik olsun.
\par 4 Tanri'nin Mesih Isa'da size bagisladigi lütuftan ötürü sizin için her zaman Tanrim'a sükrediyorum.
\par 5 Mesih'le ilgili tanikligimiz sizde pekistigi gibi Mesih'te her bakimdan -her tür söz ve bilgi bakimindan- zenginlestiniz.
\par 7 Söyle ki, Rabbimiz Isa Mesih'in görünmesini beklerken hiçbir ruhsal armagandan yoksun degilsiniz.
\par 8 Rabbimiz Isa Mesih kendi gününde kusursuz olmaniz için sizi sonuna dek pekistirecektir.
\par 9 Sizleri Oglu Rabbimiz Isa Mesih'le paydasliga çagiran Tanri güvenilirdir.
\par 10 Kardesler, Rabbimiz Isa Mesih'in adiyla yalvariyorum: Hepiniz uyum içinde olun, aranizda bölünmeler olmadan ayni düsünce ve görüste birlesin.
\par 11 Kardeslerim, Kloi'nin ev halkindan aranizda çekismeler oldugunu ögrendim.
\par 12 Sunu demek istiyorum: Her biriniz, "Ben Pavlus yanlisiyim", "Ben Apollos yanlisiyim", "Ben Kefas* yanlisiyim" ya da "Ben Mesih yanlisiyim" diyormus.
\par 13 Mesih bölündü mü? Sizin için çarmiha gerilen Pavlus muydu? Pavlus'un adiyla mi vaftiz* edildiniz?
\par 14 Hiç kimse benim adimla vaftiz edildiginizi söylemesin diye Krispus'la Gayus'tan baska hiçbirinizi vaftiz etmedigim için Tanri'ya sükrediyorum.
\par 16 Evet, bir de Istefanas'in ev halkini vaftiz ettim; bunun disinda kimseyi vaftiz ettigimi animsamiyorum.
\par 17 Çünkü Mesih beni vaftiz etmeye degil, Mesih'in çarmihtaki ölümü bosa gitmesin diye, bilgece sözlere dayanmaksizin Müjde'yi yaymaya gönderdi.
\par 18 Çarmihla ilgili bildiri mahva gidenler için saçmalik, biz kurtulmakta olanlar içinse Tanri gücüdür.
\par 19 Nitekim söyle yazilmistir: "Bilgelerin bilgeligini yok edecegim, Akillilarin aklini bosa çikaracagim."
\par 20 Hani nerede bilge kisi? Din bilgini* nerede? Nerede bu çagin hünerli tartismacisi? Tanri dünya bilgeliginin saçma oldugunu göstermedi mi?
\par 21 Mademki dünya Tanri'nin bilgeligi uyarinca Tanri'yi kendi bilgeligiyle tanimadi, Tanri iman edenleri saçma sayilan bildiriyle kurtarmaya razi oldu.
\par 22 Yahudiler dogaüstü belirtiler ister, Grekler'se* bilgelik arar.
\par 23 Ama biz çarmiha gerilmis Mesih'i duyuruyoruz. Yahudiler bunu yüzkarasi, öteki uluslar da saçmalik sayarlar.
\par 24 Oysa Mesih, çagrilmis olanlar için -ister Yahudi ister Grek olsun- Tanri'nin gücü ve Tanri'nin bilgeligidir.
\par 25 Çünkü Tanri'nin "saçmaligi" insan bilgeliginden daha üstün, Tanri'nin "zayifligi" insan gücünden daha güçlüdür.
\par 26 Kardeslerim, aldiginiz çagriyi düsünün. Birçogunuz insan ölçülerine göre bilge, güçlü ya da soylu kisiler degildiniz.
\par 27 Ne var ki, Tanri bilgeleri utandirmak için dünyanin saçma saydiklarini, güçlüleri utandirmak için de dünyanin zayif saydiklarini seçti.
\par 28 Dünyanin önemli gördüklerini hiçe indirmek için dünyanin önemsiz, soysuz, degersiz gördüklerini seçti.
\par 29 Öyle ki, Tanri'nin önünde hiç kimse övünemesin.
\par 30 Ama siz Tanri sayesinde Mesih Isa'dasiniz. O bizim için tanrisal bilgelik, dogruluk, kutsallik ve kurtulus oldu.
\par 31 Bunun için yazilmis oldugu gibi, "Övünen, Rab'le övünsün."

\chapter{2}

\par 1 Kardesler, Tanri'yla ilgili bildiriyi duyurmak için size geldigimde, söz ustaligiyla ya da üstün bilgelikle gelmedim.
\par 2 Aranizdayken, Isa Mesih'ten ve O'nun çarmiha gerilisinden baska hiçbir sey bilmemeye kararliydim.
\par 3 Size zayiflik ve korku içinde geldim, tir tir titriyordum!
\par 4 Sözüm ve bildirim, insan bilgeliginin ikna edici sözlerine degil, Ruh'un kanitlayici gücüne dayaniyordu.
\par 5 Öyle ki, imaniniz insan bilgeligine degil, Tanri gücüne dayansin.
\par 6 Gerçi olgun kisiler arasinda bilgece sözler söylüyoruz; ama bu bilgelik ne simdiki çagin, ne de bu çagin gelip geçici önderlerinin bilgeligidir.
\par 7 Tanri'nin sakli bilgeliginden gizemli biçimde söz ediyoruz. Zamanin baslangicindan önce Tanri'nin bizim yüceligimiz için belirledigi bu bilgeligi bu çagin önderlerinden hiçbiri anlamadi. Anlasalardi yüce Rab'bi çarmiha germezlerdi.
\par 9 Yazilmis oldugu gibi, "Tanri'nin kendisini sevenler için hazirladiklarini Hiçbir göz görmedi, Hiçbir kulak duymadi, Hiçbir insan yüregi kavramadi."
\par 10 Oysa Tanri Ruh araciligiyla bunlari bize açikladi. Çünkü Ruh her seyi, Tanri'nin derin düsüncelerini bile arastirir.
\par 11 Insanin düsüncelerini, insanin içindeki ruhundan baska kim bilebilir? Bunun gibi, Tanri'nin düsüncelerini de Tanri'nin Ruhu'ndan baskasi bilemez.
\par 12 Tanri'nin bize lütfettiklerini bilelim diye, bu dünyanin ruhunu degil, Tanri'dan gelen Ruh'u aldik.
\par 13 Ruhsal kisilere ruhsal gerçekleri açiklarken, Tanri'nin lütfettiklerini insan bilgeliginin ögrettigi sözlerle degil, Ruh'un ögrettigi sözlerle bildiririz.
\par 14 Dogal kisi, Tanri'nin Ruhu'yla ilgili gerçekleri kabul etmez. Çünkü bunlar ona saçma gelir, ruhça degerlendirildikleri için bunlari anlayamaz.
\par 15 Ruhsal kisi her konuda yargi yürütebilir, ama kimse onun hakkinda yargi yürütemez.
\par 16 "Rab'bin düsüncesini kim bildi ki, O'na ögüt verebilsin?" Oysa biz Mesih'in düsüncesine sahibiz.

\chapter{3}

\par 1 Kardesler, ben sizinle ruhsal kisilerle konusur gibi konusamadim. Benlige uyanlarla, Mesih'te henüz bebeklik çaginda olanlarla konusur gibi konustum.
\par 2 Size süt verdim, kati yiyecek degil. Çünkü kati yiyecegi henüz yiyemiyordunuz. Simdi bile yiyemezsiniz.
\par 3 Çünkü hâlâ benlige uyuyorsunuz. Aranizda kiskançlik ve çekisme olmasi, benlige uydugunuzu, öbür insanlar gibi yasadiginizi göstermiyor mu?
\par 4 Biriniz, "Ben Pavlus yanlisiyim", ötekiniz, "Ben Apollos yanlisiyim" diyorsa, öbür insanlardan ne farkiniz kalir?
\par 5 Apollos kim, Pavlus kim? Iman etmenize araci olmus hizmetkârlardir. Rab her birimize bir görev vermistir.
\par 6 Tohumu ben ektim, Apollos suladi. Ama Tanri büyüttü.
\par 7 Önemli olan, eken ya da sulayan degil, ekileni büyüten Tanri'dir.
\par 8 Ekenle sulayanin degeri birdir. Her biri kendi emeginin karsiligini alacaktir.
\par 9 Biz Tanri'nin emektaslariyiz. Sizler de Tanri'nin tarlasi, Tanri'nin binasisiniz.
\par 10 Tanri'nin bana lütfettigi görev uyarinca bilge bir mimar gibi temel attim, baskalari da bu temel üzerine insa ediyor. Herkes nasil insa ettigine dikkat etsin.
\par 11 Çünkü hiç kimse atilan temelden, yani Isa Mesih'ten baska bir temel atamaz.
\par 12 Bu temel üzerine kimi altin, gümüs ya da degerli taslarla, kimi de tahta, ot ya da kamisla insa edecek.
\par 13 Herkesin yaptigi is belli olacak, yargi günü ortaya çikacak. Herkesin isi atesle açiga vurulacak. Ates her isin niteligini sinayacak.
\par 14 Bir kimsenin insa ettikleri atese dayanirsa, o kimse ödülünü alacak.
\par 15 Yaptiklari yanarsa, zarar edecek. Kendisi kurtulacak, ama atesten geçmis gibi olacaktir.
\par 16 Tanri'nin tapinagi oldugunuzu, Tanri'nin Ruhu'nun sizde yasadigini bilmiyor musunuz?
\par 17 Kim Tanri'nin tapinagini yikarsa, Tanri da onu yikacak. Çünkü Tanri'nin tapinagi kutsaldir ve o tapinak sizsiniz.
\par 18 Kimse kendini aldatmasin. Aranizdan biri bu çagin ölçülerine göre kendini bilge saniyorsa, bilge olmak için "akilsiz" olsun!
\par 19 Çünkü bu dünyanin bilgeligi Tanri'nin gözünde akilsizliktir. Yazilmis oldugu gibi, "O, bilgeleri kurnazliklarinda yakalar."
\par 20 Yine, "Rab bilgelerin düsüncelerinin bos oldugunu bilir" diye yazilmistir.
\par 21 Bu nedenle hiç kimse insanlarla övünmesin. Çünkü her sey sizindir.
\par 22 Pavlus, Apollos, Kefas*, dünya, yasam ve ölüm, simdiki ve gelecek zaman, her sey sizindir.
\par 23 Siz Mesih'insiniz, Mesih de Tanri'nindir.

\chapter{4}

\par 1 Böylece insanlar bizi Mesih'in hizmetkârlari ve Tanri'nin sirlarinin kâhyalari saysin.
\par 2 Kâhyada aranan baslica nitelik güvenilir olmasidir.
\par 3 Sizin tarafinizdan ya da olagan bir mahkeme tarafindan yargilanirsam hiç aldirmam. Kendi kendimi de yargilamiyorum.
\par 4 Kendimde bir kusur görmüyorum. Ama bu beni aklamaz. Beni yargilayan Rab'dir.
\par 5 Bu nedenle, belirlenen zamandan önce hiçbir seyi yargilamayin. Rab'bin gelisini bekleyin. O, karanligin gizlediklerini aydinliga çikaracak, yüreklerdeki amaçlari açiga vuracaktir. O zaman herkes Tanri'dan payina düsen övgüyü alacaktir.
\par 6 Kardesler, bizden örnek alarak, "Yazilmis olanin disina çikmayin" sözünün anlamini ögrenmeniz için bu ilkeleri sizin yarariniza kendime ve Apollos'a uyguladim. Öyle ki, hiç kimse biriyle övünüp bir baskasini hor görmesin.
\par 7 Seni baskasindan üstün kilan kim? Tanri'dan almadigin neyin var ki? Madem aldin, niçin almamis gibi övünüyorsun?
\par 8 Zaten tok ve zenginsiniz! Biz olmadan krallar olmussunuz! Keske gerçekten krallar olsaydiniz da, biz de sizinle birlikte krallik etseydik!
\par 9 Kanimca Tanri biz elçileri, en geriden gelen ölüm hükümlüleri gibi gözler önüne serdi. Hem melekler hem insanlar için, bütün evren için seyirlik oyun olduk.
\par 10 Biz Mesih ugruna akilsiziz, ama siz Mesih'te akillisiniz! Biz zayifiz, siz güçlüsünüz! Siz saygideger kisilersiniz, bizse degersiziz!
\par 11 Su ana dek aç, susuz, çiplagiz. Dövülüyoruz, barinacak yerimiz yok.
\par 12 Kendi ellerimizle çalisip emek veriyoruz. Bize sövenlere iyilik diliyoruz, zulmedilince sabrediyoruz.
\par 13 Iftiraya ugrayinca tatlilikla karsilik veriyoruz. Su ana dek adeta dünyanin süprüntüsü, her seyin döküntüsü olduk.
\par 14 Bunlari sizi utandirmak için degil, siz sevgili çocuklarimi uyarmak için yaziyorum.
\par 15 Çünkü Mesih'in yolunda sayisiz egiticiniz olsa da çok sayida babaniz yoktur. Size Müjde'yi ulastirmakla Mesih Isa'da manevi babaniz oldum.
\par 16 Bu nedenle beni örnek almaya çagiriyorum sizi.
\par 17 Rab'be sadik olan sevgili çocugum Timoteos'u bu amaçla size gönderiyorum. Her yerde, her kilisede* ögrettigim ve Mesih'te izledigim yollari o size animsatacaktir.
\par 18 Bazilariniz yaniniza gelmeyecegimi sanarak küstahlasiyor.
\par 19 Ama Rab dilerse yakinda yaniniza gelecegim. O zaman bu küstahlarin söylediklerini degil, güçlerinin ne oldugunu ögrenecegim.
\par 20 Çünkü Tanri'nin Egemenligi lafta degil, güçtedir.
\par 21 Ne istiyorsunuz? Size sopayla mi geleyim, yoksa sevgi ve yumusak bir ruhla mi?

\chapter{5}

\par 1 Aranizda fuhus oldugu söyleniyor, üstelik putperestler arasinda bile rastlanmayan türden bir fuhus! Biri babasinin karisini almis.
\par 2 Siz hâlâ böbürleniyorsunuz! Oysa yas tutup bu isi yapani aranizdan atmaniz gerekmez miydi?
\par 3 Bedence olmasa da ruhça aranizdayim. Bu suçu isleyeni, aranizdaymisim gibi Rabbimiz Isa'nin adiyla zaten yargilamis bulunuyorum. Ben ruhça aranizdayken Rabbimiz Isa'nin gücüyle toplandiginiz zaman, bedeninin yok olmasi için bu adami Seytan'a teslim edin ki, Rab Isa'nin gününde ruhu kurtulabilsin.
\par 6 Övünmeniz yersizdir. Azicik mayanin bütün hamuru kabarttigini bilmiyor musunuz?
\par 7 Yeni bir hamur olabilmek için eski mayadan arinip temizlenin. Zaten mayasizsiniz. Çünkü Fisih* kuzumuz Mesih kurban edildi.
\par 8 Bunun için eski mayayla -kin ve kötülük mayasiyla- degil, içtenligin ve dürüstlügün mayasiz ekmegiyle bayram edelim.
\par 9 Mektubumda size fuhus yapanlarla arkadaslik etmemenizi yazdim.
\par 10 Kuskusuz dünyadaki ahlaksizlari, açgözlüleri, soygunculari ya da putperestleri demek istemedim. Öyle olsaydi, dünyadan ayrilmak zorunda kalirdiniz!
\par 11 Ama simdi size sunu yaziyorum: Kardes diye bilinirken fuhus yapan, açgözlü, putperest, sövücü, ayyas ya da soyguncu olanla arkadaslik etmeyin, böyle biriyle yemek bile yemeyin.
\par 12 Inanlilar toplulugunun disindakileri yargilamaya benim ne hakkim var? Sizin de yargilamaniz gereken kisiler toplulugun içindekiler degil mi?
\par 13 Toplulugun disinda kalanlari Tanri yargilar. "Kötü adami aranizdan kovun!"

\chapter{6}

\par 1 Sizden birinin öbürüne karsi bir davasi varsa kutsallar önünde degil de, imansizlar önünde yargilanmaya cesaret eder mi?
\par 2 Kutsallarin dünyayi yargilayacagini bilmiyor musunuz? Madem dünyayi yargilayacaksiniz, böyle önemsiz davalari görmeye yeterli degil misiniz?
\par 3 Bu yasamla ilgili davalar bir yana, melekleri bile yargilayacagimizi bilmiyor musunuz?
\par 4 Bu yasamla ilgili davalariniz oldugunda, inanlilar toplulugunda* en önemsiz sayilanlari mi yargiç atiyorsunuz?
\par 5 Sizi utandirmak için söylüyorum bunu. Kardesler arasindaki davalarda yargiçlik edecek kadar bilge biri yok mu aranizda?
\par 6 Kardes kardese karsi dava açiyor, üstelik imansizlar önünde!
\par 7 Aslinda birbirinizden davaci olmaniz bile sizin için düpedüz yenilgidir. Haksizliga ugrasaniz daha iyi olmaz mi? Dolandirilsaniz daha iyi olmaz mi?
\par 8 Bunun yerine, siz kendiniz haksizlik edip baskasini dolandiriyorsunuz. Üstelik bunu kardeslerinize yapiyorsunuz.
\par 9 Günahkârlarin, Tanri Egemenligi'ni miras almayacagini bilmiyor musunuz? Aldanmayin! Ne fuhus yapanlar Tanri'nin Egemenligi'ni miras alacaktir, ne puta tapanlar, ne zina edenler, ne oglanlar, ne oglancilar, ne hirsizlar, ne açgözlüler, ne ayyaslar, ne sövücüler, ne de soyguncular.
\par 11 Bazilariniz böyleydiniz; ama yikandiniz, kutsal kilindiniz, Rab Isa Mesih adiyla ve Tanrimiz'in Ruhu araciligiyla aklandiniz.
\par 12 "Bana her sey serbest" diyorsunuz, ama her sey yararli degildir. "Bana her sey serbest" diyorsunuz, ama hiçbir seyin tutsagi olmayacagim.
\par 13 "Yemek mide için, mide de yemek içindir" diyorsunuz, ama Tanri hem mideyi hem de yemegi ortadan kaldiracaktir. Beden fuhus için degil, Rab içindir. Rab de beden içindir.
\par 14 Rab'bi dirilten Tanri, kudretiyle bizi de diriltecek.
\par 15 Bedenlerinizin Mesih'in üyeleri oldugunu bilmiyor musunuz? Mesih'in üyelerini alip bir fahisenin üyeleri mi yapayim? Asla!
\par 16 Yoksa fahiseyle birlesenin, onunla tek beden oldugunu bilmiyor musunuz? Çünkü "Ikisi tek beden olacak" deniyor.
\par 17 Rab'le birlesen kisiyse O'nunla tek ruh olur.
\par 18 Fuhustan kaçinin. Insanin isledigi bütün öbür günahlar bedenin disindadir; ama fuhus yapan, kendi bedenine karsi günah isler.
\par 19 Bedeninizin, Tanri'dan aldiginiz ve içinizdeki Kutsal Ruh'un tapinagi oldugunu bilmiyor musunuz? Kendinize ait degilsiniz.
\par 20 Bir bedel karsiligi satin alindiniz; onun için Tanri'yi bedeninizde yüceltin.

\chapter{7}

\par 1 Simdi bana yazdiginiz konulara gelelim: "Erkegin kadina dokunmamasi iyidir" diyorsunuz.
\par 2 Ama fuhustan ötürü her erkek karisiyla, her kadin da kocasiyla yasasin.
\par 3 Erkek karisina, kadin da kocasina hakkini versin.
\par 4 Kadinin bedeni kendisine degil, kocasina aittir. Bunun gibi, erkegin bedeni de kendisine degil, karisina aittir.
\par 5 Geçici bir süre için anlasip kendinizi duaya vermekten baska bir nedenle birbirinizi mahrum etmeyin. Sonra yine birlesin ki, kendinizi denetleyemediginiz için Seytan sizi ayartmasin.
\par 6 Bunu bir buyruk olarak degil, bir uzlasma yolu olarak söylüyorum.
\par 7 Herkesin benim gibi olmasini dilerdim. Ama herkesin Tanri'dan aldigi ruhsal bir armagani vardir; kiminin söyle, kiminin böyle.
\par 8 Yine de evli olmayanlarla dul kadinlara sunu söyleyeyim: Benim gibi kalsalar kendileri için iyi olur.
\par 9 Ama kendilerini denetleyemiyorlarsa, evlensinler. Çünkü için için yanmaktansa evlenmek daha iyidir.
\par 10 Evlilereyse sunu buyuruyorum, daha dogrusu Rab buyuruyor: Kadin kocasindan ayrilmasin.
\par 11 Ayrilirsa evlenmesin, ya da kocasiyla barissin. Erkek de karisini bosamasin.
\par 12 Geri kalanlara Rab degil, ben söylüyorum: Eger bir kardesin karisi iman etmemisse ama kendisiyle yasamaya raziysa, onu bosamasin.
\par 13 Bir kadinin kocasi iman etmemisse ama kendisiyle yasamaya raziysa, kadin onu bosamasin.
\par 14 Çünkü iman etmemis koca karisi araciligiyla, iman etmemis kadin da imanli kocasi araciligiyla kutsanir. Yoksa çocuklariniz murdar olurdu. Ama simdi kutsaldirlar.
\par 15 Iman etmeyen ayrilirsa ayrilsin. Kardes ya da kizkardes böyle durumlarda özgürdür. Tanri sizi baris içinde yasamaya çagirdi.
\par 16 Ey kadin, kocani kurtarip kurtaramayacagini nereden biliyorsun? Ey erkek, karini kurtarip kurtaramayacagini nereden biliyorsun?
\par 17 Ancak herkes Rab'bin kendisi için belirledigi duruma uygun biçimde, Tanri'dan aldigi çagriya göre yasasin. Bunu bütün kiliselere* buyuruyorum.
\par 18 Biri sünnetliyken mi çagrildi, sünnetsiz olmasin. Bir baskasi sünnetsizken mi çagrildi, sünnet olmasin.
\par 19 Sünnetli olup olmamak önemli degildir. Önemli olan, Tanri'nin buyruklarini yerine getirmektir.
\par 20 Herkes ne durumda çagrildiysa, o durumda kalsin.
\par 21 Köleyken mi çagrildin, üzülme. Ama özgür olabilirsen, firsati kaçirma!
\par 22 Çünkü Rab'bin çagrisini aldigi zaman köle olan kimse, simdi Rab'bin özgürüdür. Özgürken çagrilan kisi de Mesih'in kölesidir.
\par 23 Bir bedel karsiligi satin alindiniz, insanlara köle olmayin.
\par 24 Kardesler, herkes ne durumda çagrildiysa, Tanri önünde o durumda kalsin.
\par 25 Kizlara gelince, Rab'den onlarla ilgili bir buyruk almis degilim. Ama Rab'bin merhameti sayesinde güvenilir biri olarak düsündüklerimi söylüyorum.
\par 26 Öyle saniyorum ki, simdiki sikintilar nedeniyle insanin oldugu gibi kalmasi iyidir.
\par 27 Karin varsa, bosanmayi isteme. Karin yoksa, kendine es arama.
\par 28 Ama evlenirsen günah islemis olmazsin. Bir kiz da evlenirse günah islemis olmaz. Ne var ki, evlenenler bu yasamda sikintilarla karsilasacak. Ben sizi bu sikintilardan esirgemek istiyorum.
\par 29 Kardesler, sunu demek istiyorum: Zaman daralmistir. Bundan böyle, karisi olanlar karilari yokmus gibi, yas tutanlar yas tutmuyormus gibi, sevinenler sevinmiyormus gibi, mal alanlar mallari yokmus gibi, dünyadan yararlananlar alabildigine yararlanmiyormus gibi olsun. Çünkü dünyanin simdiki hali geçicidir.
\par 32 Kaygisiz olmanizi istiyorum. Evli olmayan erkek, Rab'bi nasil hosnut edecegini düsünerek Rab'bin isleri için kaygilanir.
\par 33 Evli erkekse karisini nasil hosnut edecegini düsünerek dünya isleri için kaygilanir.
\par 34 Böylece ilgisi bölünür. Evli olmayan kadin ya da kiz hem bedence hem ruhça kutsal olmak amaciyla Rab'bin isleri için kaygilanir. Evli kadinsa kocasini nasil hosnut edecegini düsünerek dünya isleri için kaygilanir.
\par 35 Bunu sizin iyiliginiz için söylüyorum, özgürlügünüzü kisitlamak için degil. Ilginizi dagitmadan, Rab'be adanmis olarak, O'na yarasir biçimde yasamanizi istiyorum.
\par 36 Bir kimse nisanli oldugu kiza yakisiksiz davrandigini düsünüyorsa, asiri tutkulari varsa ve evlenmesi gerekiyorsa, istedigini yapsin, günah islemis olmaz; evlensinler.
\par 37 Ama zorunluluk altinda bulunmayan, yüregi kararli, istedigini yapabilecek durumdaki kisi, nisanlisiyla evlenmemeye yüreginde karar vermisse, iyi eder.
\par 38 Kisacasi nisanlisiyla evlenen iyi eder, evlenmeyense daha iyi eder.
\par 39 Kadin, kocasi yasadikça kocasina baglidir. Kocasi ölürse diledigi kimseyle evlenmekte özgürdür; yeter ki, o kisi Rab'be ait biri olsun.
\par 40 Ama dul kadin, oldugu gibi kalirsa daha mutlu olur. Ben böyle düsünüyorum ve sanirim bende de Tanri'nin Ruhu vardir.

\chapter{8}

\par 1 Simdi putlara sunulan kurbanlarin etine gelelim. "Hepimizin bilgisi var" diyorsunuz, bunu biliyoruz. Bilgi insani böbürlendirir, sevgiyse gelistirir.
\par 2 Bir sey bildigini sanan, henüz bilmesi gerektigi gibi bilmiyordur.
\par 3 Ama Tanri'yi seveni Tanri bilir.
\par 4 Putlara sunulan kurban etinin yenmesine gelince, biliyoruz ki, "Dünyada put bir hiçtir" ve "Birden fazla Tanri yoktur".
\par 5 Yerde ya da gökte ilah diye adlandirilanlar varsa da -nitekim pekçok "ilah", pekçok "rab" vardir bizim için tek bir Tanri Baba vardir. O her seyin kaynagidir, bizler O'nun için yasiyoruz. Tek bir Rab var, O da Isa Mesih'tir. Her sey O'nun araciligiyla yaratildi, biz de O'nun araciligiyla yasiyoruz.
\par 7 Ne var ki, herkes bu bilgiye sahip degildir. Hâlâ putperest aliskanliklarinin etkisinde kalan bazilari, yedikleri etin puta sunuldugunu düsünüyorlar. Vicdanlari zayif oldugu için lekeleniyor.
\par 8 Yiyecek bizi Tanri'ya yaklastirmaz. Yemezsek bir kaybimiz olmaz, yersek de bir kazancimiz olmaz.
\par 9 Yalniz dikkat edin, bu özgürlügünüz vicdani zayif olanlarin sürçmesine neden olmasin.
\par 10 Eger zayif vicdanli biri, bilgili olan seni bir put tapinaginda sofraya oturmus görürse, puta sunulan kurbanin etini yemek için cesaret almaz mi?
\par 11 Sonuçta bu zayif vicdanli kisi, Mesih'in ugruna öldügü bu kardes, senin bilgin yüzünden mahvolur!
\par 12 Bu sekilde kardeslere karsi günah isleyip onlarin zayif vicdanlarini yaralayarak Mesih'e karsi günah islemis olursunuz.
\par 13 Bu nedenle, yedigim sey kardesimin sendeleyip düsmesine yol açacaksa, kardesimin düsmemesi için bir daha et yemeyecegim.

\chapter{9}

\par 1 Özgür degil miyim? Elçi degil miyim? Rabbimiz Isa'yi görmedim mi? Sizler Rab yolunda verdigim emegin ürünü degil misiniz?
\par 2 Baskalari için elçi degilsem bile, sizler için elçiyim ya! Rab yolunda elçiligimin kaniti sizsiniz.
\par 3 Beni sorguya çekenlere karsi kendimi böyle savunurum.
\par 4 Yiyip içmeye hakkimiz yok mu bizim?
\par 5 Öbür elçiler gibi, Rab'bin kardesleri ve Kefas* gibi, yanimizda imanli bir es gezdirmeye hakkimiz yok mu?
\par 6 Geçimi için çalismasi gereken yalniz Barnaba'yla ben miyim?
\par 7 Kim kendi parasiyla askerlik yapar? Kim bag diker de ürününü yemez? Kim sürüyü güder de sütünden içmez?
\par 8 Insansal açidan mi söylüyorum bunlari? Kutsal Yasa* da ayni seyleri söylemiyor mu?
\par 9 Musa'nin Yasasi'nda, "Harman döven öküzün agzini baglamayacaksin" diye yazilmistir. Tanri'nin kaygisi öküzler mi, yoksa bunu özellikle bizim için mi söylüyor? Kuskusuz, bizim için yazilmistir bu. Çünkü çift sürenin umutla sürmesi, harman dövenin de harmana ortak olma umuduyla dövmesi gerekir.
\par 11 Araniza ruhsal tohumlar ektiysek, sizden maddesel bir harman biçmemiz çok mu?
\par 12 Baskalarinin sizden yardim almaya haklari varsa, bizim daha çok hakkimiz yok mu? Ama biz bu hakkimizi kullanmadik. Mesih Müjdesi'nin yayilmasina engel olmayalim diye her seye katlaniyoruz.
\par 13 Tapinakta çalisanlarin tapinaktan beslendiklerini, sunakta görevli olanlarin da sunakta adanan adaklardan pay aldiklarini bilmiyor musunuz?
\par 14 Bunun gibi, Rab Müjde'yi yayanlarin da geçimlerini Müjde'den saglamasini buyurdu.
\par 15 Ama ben bu haklardan hiçbirini kullanmis degilim. Bunlar bana saglansin diye de yazmiyorum. Bunu yapmaktansa ölmeyi yeglerim. Kimse beni bu övünçten yoksun birakmayacaktir!
\par 16 Müjde'yi yayiyorum diye övünmeye hakkim yok. Çünkü bunu yapmakla yükümlüyüm. Müjde'yi yaymazsam vay halime!
\par 17 Eger Müjde'yi gönülden yayarsam, ödülüm olur; gönülsüzce yayarsam, yalnizca bana emanet edilen görevi yapmis olurum.
\par 18 Peki, ödülüm nedir? Müjde'yi karsiliksiz yaymak ve böylece Müjde'yi yaymaktan dogan hakkimi kullanmamaktir.
\par 19 Ben özgürüm, kimsenin kölesi degilim. Ama daha çok kisi kazanayim diye herkesin kölesi oldum.
\par 20 Yahudiler'i kazanmak için Yahudiler'e Yahudi gibi davrandim. Kendim Kutsal Yasa'nin denetimi altinda olmadigim halde, Yasa altinda olanlari kazanmak için onlara Yasa altindaymisim gibi davrandim.
\par 21 Tanri'nin Yasasi'na sahip olmayan biri degilim, Mesih'in Yasasi altindayim. Buna karsin, Yasa'ya sahip olmayanlari kazanmak için Yasa'ya sahip degilmisim gibi davrandim.
\par 22 Güçsüzleri kazanmak için onlarla güçsüz oldum. Ne yapip yapip bazilarini kurtarmak için herkesle her sey oldum.
\par 23 Bunlarin hepsini Müjde'de payim olsun diye, Müjde ugruna yapiyorum.
\par 24 Kosu alaninda yarisanlarin hepsi kostugu halde ödülü bir kisinin kazandigini bilmiyor musunuz? Öyle kosun ki ödülü kazanasiniz.
\par 25 Yarisa katilan herkes kendini her yönden denetler. Böyleleri bunu çürüyüp gidecek bir defne taci kazanmak için yaparlar. Bizse hiç çürümeyecek bir taç için yapiyoruz.
\par 26 Bunun içindir ki, amaçsizca kosan biri gibi kosmuyorum. Yumrugumu havayi döver gibi bosa atmiyorum.
\par 27 Müjde'yi baskalarina duyurduktan sonra kendim reddedilmemek için bedenime eziyet çektirip onu köle ediyorum.

\chapter{10}

\par 1 Kardesler, atalarimizin hepsinin bulut altinda korundugunu ve hepsinin denizden geçtigini bilmenizi istiyorum.
\par 2 Musa'ya baglanmak üzere hepsi bulutta ve denizde vaftiz* edildi.
\par 3 Hepsi ayni ruhsal yiyecegi yedi;
\par 4 hepsi ayni ruhsal içecegi içti. Artlarindan gelen ruhsal kayadan içtiler; o kaya Mesih'ti.
\par 5 Ne var ki, Tanri onlarin çogundan hosnut degildi; nitekim cesetleri çöle serildi.
\par 6 Bu olaylar, onlar gibi kötü seylere özlem duymamamiz için bize ders olsun diye oldu.
\par 7 Onlardan bazilari gibi puta tapanlar olmayin. Nitekim söyle yazilmistir: "Halk yiyip içmeye oturdu, sonra kalkip çilginca eglendi."
\par 8 Onlardan bazilari gibi fuhus yapmayalim. Fuhus yapanlarin yirmi üç bini bir günde yok oldu.
\par 9 Yine bazilari gibi Rab'bi denemeyelim. Böyle yapanlari yilanlar öldürdü.
\par 10 Kimileri gibi de söylenip durmayin. Söylenenleri ölüm melegi öldürdü.
\par 11 Bu olaylar baskalarina ders olsun diye onlarin basina geldi; çaglarin sonuna ulasmis olan bizleri uyarmak için yaziya geçirildi.
\par 12 Onun için, ayakta saglam durdugunu sanan dikkat etsin, düsmesin!
\par 13 Herkesin karsilastigi denemelerden baska denemelerle karsilasmadiniz. Tanrim güvenilirdir, gücünüzü asan biçimde denenmenize izin vermez. Dayanabilmeniz için denemeyle birlikte çikis yolunu da saglayacaktir.
\par 14 Bu nedenle, sevgili kardeslerim, putperestlikten kaçinin.
\par 15 Akli basinda insanlarla konusur gibi konusuyorum. Söylediklerimi kendiniz tartin.
\par 16 Tanri'ya sükrettigimiz sükran kâsesiyle Mesih'in kanina paydas olmuyor muyuz? Bölüp yedigimiz ekmekle Mesih'in bedenine paydas olmuyor muyuz?
\par 17 Ekmek bir oldugu gibi, biz de çok oldugumuz halde bir bedeniz. Çünkü hepimiz bir ekmegi paylasiyoruz.
\par 18 Israil halkina bakin; kurban etini yiyenler sunaga paydas degil midir?
\par 19 Öyleyse ne demek istiyorum? Puta sunulan kurban etinin bir özelligi mi var? Ya da putun bir önemi mi var?
\par 20 Hayir, yok! Dedigim su: Putperestler kurbanlarini Tanri'ya degil, cinlere sunuyorlar. Cinlerle paydas olmanizi istemem.
\par 21 Hem Rab'bin, hem cinlerin kâsesinden içemezsiniz; hem Rab'bin, hem cinlerin sofrasina ortak olamazsiniz.
\par 22 Yoksa Rab'bi kiskandirmaya mi çalisiyoruz? Biz O'ndan daha mi güçlüyüz?
\par 23 "Her sey serbest" diyorsunuz, ama her sey yararli degildir. "Her sey serbest" diyorsunuz, ama her sey yapici degildir.
\par 24 Herkes kendi yararini degil, baskalarinin yararini gözetsin.
\par 25 Kasaplar çarsisinda satilan her eti vicdan sorunu yapmadan, sorgusuz sualsiz yiyin.
\par 26 Çünkü "Yeryüzü ve içindeki her sey Rab'bindir."
\par 27 Iman etmemis biri sizi yemege çagirir, siz de gitmek isterseniz, önünüze konulan her seyi vicdan sorunu yapmadan, sorgusuz sualsiz yiyin.
\par 28 Ama biri size, "Bu kurban etidir" derse, hem bunu söyleyen için, hem de vicdan huzuru için yemeyin.
\par 29 Senin degil, öbür adamin vicdan huzuru için demek istiyorum. Benim özgürlügümü neden baskasinin vicdani yargilasin?
\par 30 Sükrederek yemege katilirsam, sükrettigim yiyecekten ötürü neden kinanayim?
\par 31 Sonuç olarak, ne yer ne içerseniz, ne yaparsaniz, her seyi Tanri'nin yüceligi için yapin.
\par 32 Yahudiler'in, Grekler'in* ya da Tanri toplulugunun* tökezleyip düsmesine neden olmayin.
\par 33 Ben de kendi yararimi degil, kurtulsunlar diye birçoklarinin yararini gözeterek herkesi her yönden hosnut etmeye çalisiyorum.

\chapter{11}

\par 1 Mesih'i örnek aldigim gibi, siz de beni örnek alin.
\par 2 Her durumda beni animsadiginiz ve size ilettigim ögretileri oldugu gibi korudugunuz için sizi övüyorum.
\par 3 Ama sunu da bilmenizi isterim: Her erkegin basi Mesih, kadinin basi erkek, Mesih'in basi da Tanri'dir.
\par 4 Basina bir sey takip dua ya da peygamberlik eden her erkek, basini küçükdüsürür.
\par 5 Ama basi açik dua ya da peygamberlik eden her kadin, basini küçük düsürür. Böylesinin, basi tiras edilmis bir kadindan farki yoktur.
\par 6 Kadin basini açarsa, saçini kestirsin. Ama kadinin saçini kestirmesi ya da tiras etmesi ayipsa, basini örtsün.
\par 7 Erkek basini örtmemeli; o, Tanri'nin benzeri ve yüceligidir. Kadin da erkegin yüceligidir.
\par 8 Çünkü erkek kadindan degil, kadin erkekten yaratildi.
\par 9 Erkek kadin için degil, kadin erkek için yaratildi.
\par 10 Bu nedenle ve melekler ugruna kadinin basi üzerinde yetkisi olmalidir.
\par 11 Ne var ki, Rab'de ne kadin erkekten ne de erkek kadindan bagimsizdir.
\par 12 Çünkü kadin erkekten yaratildigi gibi, erkek de kadindan dogar. Ama her sey Tanri'dandir.
\par 13 Siz kendiniz karar verin: Kadinin açik basla Tanri'ya dua etmesi uygun mu?
\par 14 Doganin kendisi bile size erkegin uzun saçli olmasinin kendisini küçük düsürdügünü, kadinin uzun saçli olmasinin ise kendisini yücelttigini ögretmiyor mu? Çünkü saç kadina örtü olarak verilmistir.
\par 16 Bu konuda çekismek isteyen varsa, sunu bilsin ki, bizim ya da Tanri'nin kiliselerinin* böyle bir aliskanligi yoktur.
\par 17 Toplantilariniz yarardan çok zarar getirdigi için asagidaki uyarilari yaparken sizi övemem.
\par 18 Birincisi, toplulukça* bir araya geldiginizde aranizda ayriliklar oldugunu duyuyorum. Buna biraz da inaniyorum.
\par 19 Çünkü Tanri'nin begenisini kazananlarin belli olmasi için aranizda bölünmeler olmasi gerekiyor!
\par 20 Toplandiginizda Rab'bin Sofrasi'na katilmak için toplanmiyorsunuz.
\par 21 Her biriniz ötekini beklemeden kendi yemegini yiyor. Kimi aç kaliyor, kimi sarhos oluyor.
\par 22 Yiyip içmek için evleriniz yok mu? Tanri'nin toplulugunu hor mu görüyorsunuz, yiyecegi olmayanlari utandirmak mi istiyorsunuz? Size ne diyeyim? Sizi öveyim mi? Bu konuda övemem!
\par 23 Size ilettigimi ben Rab'den ögrendim. Ele verildigi gece Rab Isa eline ekmek aldi, sükredip ekmegi böldü ve söyle dedi: "Bu sizin ugrunuza feda edilen bedenimdir. Beni anmak için böyle yapin."
\par 25 Ayni biçimde yemekten sonra kâseyi alip söyle dedi: "Bu kâse kanimla gerçeklesen yeni antlasmadir. Her içtiginizde beni anmak için böyle yapin."
\par 26 Bu ekmegi her yediginizde ve bu kâseden her içtiginizde, Rab'bin gelisine dek Rab'bin ölümünü ilan etmis olursunuz.
\par 27 Bu nedenle kim uygun olmayan biçimde ekmegi yer ya da Rab'bin kâsesinden içerse, Rab'bin bedenine ve kanina karsi suç islemis olur.
\par 28 Kisi önce kendini sinasin, sonra ekmekten yiyip kâseden içsin.
\par 29 Çünkü bedeni farketmeden yiyip içen, böyle yiyip içmekle kendi kendini mahkûm eder.
\par 30 Iste bu yüzden birçogunuz zayif ve hastadir, bazilariniz da ölmüstür.
\par 31 Kendimizi dogrulukla yargilasaydik, yargilanmazdik.
\par 32 Dünyayla birlikte mahkûm olmayalim diye Rab bizi yargilayip terbiye ediyor.
\par 33 Öyleyse kardeslerim, yemek için bir araya geldiginizde birbirinizi bekleyin.
\par 34 Aç olan karnini evde doyursun. Öyle ki, toplanmaniz yargilanmaniza yol açmasin. Öbür sorunlari ise geldigimde çözerim.

\chapter{12}

\par 1 Ruhsal armaganlara gelince, kardeslerim, bu konuda bilgisiz kalmanizi istemem.
\par 2 Biliyorsunuz, putperestken söyle ya da böyle saptirilip dilsiz putlara tapmaya yöneltilmistiniz.
\par 3 Bunun için bilmenizi isterim ki: Tanri'nin Ruhu araciligiyla konusan hiç kimse, "Isa'ya lanet olsun!" demez. Kutsal Ruh'un araciligi olmaksizin da kimse, "Isa Rab'dir" diyemez.
\par 4 Çesitli ruhsal armaganlar vardir, ama Ruh birdir.
\par 5 Çesitli görevler vardir, ama Rab birdir.
\par 6 Çesitli etkinlikler vardir, ama herkeste hepsini etkin kilan ayni Tanri'dir.
\par 7 Herkesin ortak yarari için herkese Ruh'u belli eden bir yetenek veriliyor.
\par 8 Ruh araciligiyla birine bilgece konusma yetenegi, ötekine ayni Ruh'tan bilgi iletme yetenegi, birine ayni Ruh araciligiyla iman, ötekine ayni Ruh araciligiyla hastalari iyilestirme armaganlari, birine mucize yapma olanaklari, birine peygamberlikte bulunma, birine ruhlari ayirt etme, birine çesitli dillerle konusma, bir baskasina da bu dilleri çevirme armagani veriliyor.
\par 11 Bunlarin tümünü etkin kilan tek ve ayni Ruh'tur. Ruh bunlari herkese diledigi gibi, ayri ayri dagitir.
\par 12 Beden bir olmakla birlikte birçok üyeden olusur ve çok sayidaki bu üyelerin epsi tek bir beden olusturur. Mesih de böyledir.
\par 13 Ister Yahudi ister Grek*, ister köle ister özgür olalim, hepimiz bir beden olmak üzere ayni Ruh'ta vaftiz edildik ve hepimizin ayni Ruh'tan içmesi saglandi.
\par 14 Iste beden tek üyeden degil, birçok üyeden olusur.
\par 15 Ayak, "El olmadigim için bedene ait degilim" derse, bu onu bedenden ayirmaz.
\par 16 Kulak, "Göz olmadigim için bedene ait degilim" derse, bu onu bedenden ayirmaz.
\par 17 Bütün beden göz olsaydi, nasil duyardik? Bütün beden kulak olsaydi, nasil koklardik?
\par 18 Gerçek su ki, Tanri bedenin her üyesini diledigi biçimde bedene yerlestirmistir.
\par 19 Eger hepsi bir tek üye olsaydi, beden olur muydu?
\par 20 Gerçek su ki, çok sayida üye, ama tek beden vardir.
\par 21 Göz ele, "Sana ihtiyacim yok!" ya da bas ayaklara, "Size ihtiyacim yok!" diyemez.
\par 22 Tam tersine, bedenin daha zayif görünen üyeleri vazgeçilmezdir.
\par 23 Bedenin daha az degerli saydigimiz üyelerine daha çok deger veririz. Böylece gösterissiz üyelerimiz daha gösterisli olur.
\par 24 Gösterisli üyelerimizin özene ihtiyaci yoktur. Ama Tanri, degeri az olana daha çok deger vererek bedende birligi sagladi.
\par 25 Öyle ki, bedende ayrilik olmasin, üyeler birbirini esit biçimde gözetsin.
\par 26 Bir üye aci çekerse, bütün üyeler birlikte aci çeker; bir üye yüceltilirse, bütün üyeler birlikte sevinir.
\par 27 Sizler Mesih'in bedenisiniz, bu bedenin ayri ayri üyelerisiniz.
\par 28 Tanri kilisede* ilkin elçileri, ikinci olarak peygamberleri, üçüncü olarak ögretmenleri, sonra mucize yapanlari, hastalari iyilestirme armaganlarina sahip olanlari, baskalarina yardim edenleri, yönetme yetenegi olanlari ve çesitli dillerle konusanlari atadi.
\par 29 Hepsi elçi mi? Hepsi peygamber mi? Hepsi ögretmen mi? Hepsi mucize yapar mi?
\par 30 Hepsinin hastalari iyilestirme armaganlari var mi? Hepsi bilmedigi dilleri konusabilir mi? Hepsi bu dilleri çevirebilir mi?
\par 31 Ama siz daha üstün armaganlari gayretle isteyin. Simdi size en iyi yolu göstereyim.

\chapter{13}

\par 1 Insanlarin ve meleklerin diliyle konussam, ama sevgim olmasa, ses çikaran çikaran bakirdan ya da zilden farkim kalmaz.
\par 2 Peygamberlikte bulunabilsem, bütün sirlari bilsem, her bilgiye sahip olsam,daglari yerinden oynatacak kadar büyük imanim olsa, ama sevgim olmasa, bir hiçim.
\par 3 Varimi yogumu sadaka olarak dagitsam, bedenimi yakilmak üzere teslim etsem, ama sevgim olmasa, bunun bana hiçbir yarari olmaz.
\par 4 Sevgi sabirlidir, sevgi sefkatlidir. Sevgi kiskanmaz, övünmez, böbürlenmez.
\par 5 Sevgi kaba davranmaz, kendi çikarini aramaz, kolay kolay öfkelenmez, kötülügün hesabini tutmaz.
\par 6 Sevgi haksizliga sevinmez, gerçek olanla sevinir.
\par 7 Sevgi her seye katlanir, her seye inanir, her seyi umut eder, her seye dayanir.
\par 8 Sevgi asla son bulmaz. Ama peygamberlikler ortadan kalkacak, diller sona erecek, bilgi ortadan kalkacaktir.
\par 9 Çünkü bilgimiz de peygamberligimiz de sinirlidir.
\par 10 Ne var ki, yetkin olan geldiginde sinirli olan ortadan kalkacaktir.
\par 11 Çocukken çocuk gibi konusur, çocuk gibi anlar, çocuk gibi düsünürdüm. Yetiskin biri olunca çocukça davranislari biraktim.
\par 12 Simdi her seyi aynadaki silik görüntü gibi görüyoruz, ama o zaman yüz yüze görüsecegiz. Simdi bilgim sinirlidir, ama o zaman bilindigim gibi tam bilecegim.
\par 13 Iste kalici olan üç sey vardir: Iman, umut, sevgi. Bunlarin en üstünü de sevgidir

\chapter{14}

\par 1 Sevginin ardinca kosun ve ruhsal armaganlari, özellikle peygamberlik yetenegini gayretle isteyin.
\par 2 Bilmedigi dilde konusan, insanlarla degil, Tanri'yla konusur. Kimse onu anlamaz. O, ruhuyla sirlar söyler.
\par 3 Peygamberlikte bulunansa insanlarin ruhça gelismesi, cesaret ve teselli bulmasi için insanlara seslenir.
\par 4 Bilmedigi dilde konusan kendi kendini gelistirir; ama peygamberlikte bulunan, inanlilar toplulugunu* gelistirir.
\par 5 Hepinizin dillerle konusmasini isterim, ama peygamberlikte bulunmanizi yeglerim. Diller inanlilar toplulugunun gelismesi için çevrilmedikçe peygamberlikte bulunan, dillerle konusandan üstündür.
\par 6 Simdi kardeslerim, yaniniza gelip dillerle konussam, ama size bir vahiy, bir bilgi, bir peygamberlik sözü ya da bir ögreti getirmesem, size ne yararim olur?
\par 7 Kaval ya da lir gibi ses veren cansiz nesneler bile degisik sesler çikarmasa, kaval mi, lir mi çalindigini kim anlar?
\par 8 Borazan belirgin bir ses çikarmasa, kim savasa hazirlanir?
\par 9 Bunun gibi, siz de anlasilir bir dil konusmazsaniz, söyledikleriniz nasil anlasilir? Havaya konusmus olursunuz!
\par 10 Kuskusuz dünyada çesit çesit diller vardir, hiçbiri de anlamsiz degildir.
\par 11 Ne var ki, konusulan dili anlamazsam, ben konusana yabanci olurum, konusan da bana yabanci olur.
\par 12 Bu nedenle, siz de ruhsal armaganlara heveslendiginize göre, inanlilar toplulugunu gelistiren ruhsal armaganlar bakimindan zenginlesmeye bakin.
\par 13 Bunun için, bilmedigi dili konusan, kendi söylediklerini çevirebilmek için dua etsin.
\par 14 Bilmedigim dille dua edersem ruhum dua eder, ama zihnimin buna katkisi olmaz.
\par 15 Öyleyse ne yapmaliyim? Ruhumla da zihnimle de dua edecegim. Ruhumla da zihnimle de ilahi söyleyecegim.
\par 16 Tanri'yi yalniz ruhunla översen, yeni katilanlar senin ne söyledigini bilmediginden, ettigin sükran duasina nasil "Amin!" desin?
\par 17 Uygun biçimde sükrediyor olabilirsin, ama bu baskasini gelistirmez.
\par 18 Dillerle hepinizden çok konustugum için Tanri'ya sükrediyorum.
\par 19 Ama inanlilar toplulugunda dillerle on bin söz söylemektense, baskalarini egitmek için zihnimden bes söz söylemeyi yeglerim.
\par 20 Kardesler, çocuk gibi düsünmeyin. Kötülük konusunda çocuklar gibi, ama düsünmekte yetiskinler gibi olun.
\par 21 Kutsal Yasa'da* söyle yazilmistir: "Rab, `Yabanci diller konusanlarin araciligiyla, Yabancilarin dudaklariyla bu halka seslenecegim; Yine de beni dinlemeyecekler!' diyor."
\par 22 Görülüyor ki, bilinmeyen diller imanlilar için degil, imansizlar için bir belirtidir. Peygamberlikse imansizlar için degil, imanlilar için bir belirtidir.
\par 23 Simdi bütün inanlilar toplulugu bir araya gelip hep birlikte bilmedikleri dillerle konusurlarken yeni katilanlar ya da iman etmeyenler içeri girerse, "Siz çildirmissiniz!" demezler mi?
\par 24 Ama herkes peygamberlikte bulunurken iman etmeyen ya da yeni katilan biri içeri girerse, söylenen her sözle günahli olduguna ikna edilecek, her sözle yargilanacak.
\par 25 Yüregindeki gizli düsünceler açiga çikacak ve, "Tanri gerçekten aranizdadir!" diyerek yüzüstü yere kapanip Tanri'ya tapinacaktir.
\par 26 Öyleyse ne diyelim, kardesler? Toplandiginizda her birinizin bir ilahisi, ögretecek bir konusu, bir vahyi, bilmedigi dilde söyleyecek bir sözü ya da bir çevirisi vardir. Her sey toplulugun gelismesi için olsun.
\par 27 Eger bilinmeyen dillerle konusulacaksa, iki ya da en çok üç kisi sirayla konussun, biri de söylenenleri çevirsin.
\par 28 Çeviri yapacak biri yoksa, bilmedigi dilde konusan, toplulukta sessiz kalsin, içinden Tanri'yla konussun.
\par 29 Iki ya da üç peygamber konussun, öbürleri söylenenleri iyice düsünüp tartsin.
\par 30 Toplantida oturanlardan birine vahiy gelirse, konusmakta olan sussun.
\par 31 Herkesin ögrenmesi ve cesaret bulmasi için hepiniz teker teker peygamberlikte bulunabilirsiniz.
\par 32 Peygamberlerin ruhlari peygamberlerin denetimi altindadir.
\par 33 Çünkü Tanri karisiklik degil, esenlik Tanrisi'dir. Kutsallarin bütün topluluklarinda böyledir.
\par 34 Kadinlar toplantilarinizda sessiz kalsin. Konusmalarina izin yoktur. Kutsal Yasa'nin da belirttigi gibi, uysal olsunlar.
\par 35 Ögrenmek istedikleri bir sey varsa, evde kocalarina sorsunlar. Çünkü kadinin toplanti sirasinda konusmasi ayiptir.
\par 36 Tanri'nin sözü sizden mi kaynaklandi, ya da yalniz size mi ulasti?
\par 37 Kendini peygamber ya da ruhça olgun sayan varsa, bilsin ki, size yazdiklarim Rab'bin buyrugudur.
\par 38 Bunlari önemsemeyenin kendisi de önemsenmesin.
\par 39 Özet olarak, kardeslerim, peygamberlikte bulunmayi gayretle isteyin, bilinmeyen dillerle konusulmasina engel olmayin. Ancak her sey uygun ve düzenli biçimde yapilsin.

\chapter{15}

\par 1 Simdi, kardesler, size bildirdigim, sizin de kabul edip bagli kaldiginiz Müjde'yi animsatmak istiyorum.
\par 2 Size müjdeledigim söze simsiki sarilirsaniz, onun araciligiyla kurtulursunuz. Yoksa bosuna iman etmis olursunuz.
\par 3 Aldigim bilgiyi size öncelikle ilettim: Kutsal Yazilar uyarinca Mesih günahlarimiza karsilik öldü, gömüldü ve Kutsal Yazilar uyarinca üçüncü gün ölümden dirildi.
\par 5 Kefas'a*, sonra Onikiler'e* göründü.
\par 6 Daha sonra da bes yüzden çok kardese ayni anda göründü. Bunlarin çogu hâlâ yasiyor, bazilariysa öldüler.
\par 7 Bundan sonra Yakup'a, sonra bütün elçilere, son olarak zamansiz dogmus bir çocuga benzeyen bana da göründü.
\par 9 Ben elçilerin en önemsiziyim. Tanri'nin kilisesine* zulmettigim için elçi olarak anilmaya bile layik degilim.
\par 10 Ama simdi neysem, Tanri'nin lütfuyla öyleyim. O'nun bana olan lütfu bosa gitmedi. Elçilerin hepsinden çok emek verdim. Aslinda ben degil, Tanri'nin bende olan lütfu emek verdi.
\par 11 Iste, gerek benim yaydigim, gerek öbür elçilerin yaydigi ve sizin de iman ettiginiz bildiri budur.
\par 12 Eger Mesih'in ölümden dirildigi duyuruluyorsa, nasil oluyor da aranizda bazilari ölüler dirilmez diyor?
\par 13 Ölüler dirilmezse, Mesih de dirilmemistir.
\par 14 Mesih dirilmemisse, bildirimiz de imaniniz da bostur.
\par 15 Bu durumda Tanri'yla ilgili tanikligimiz da yalan demektir. Çünkü Tanri'nin, Mesih'i dirilttigine taniklik ettik. Ama ölüler gerçekten dirilmezse, Tanri Mesih'i de diriltmemistir.
\par 16 Ölüler dirilmezse, Mesih de dirilmemistir.
\par 17 Mesih dirilmemisse imaniniz yararsizdir, siz de hâlâ günahlarinizin içindesiniz.
\par 18 Buna göre Mesih'e ait olarak ölmüs olanlar da mahvolmuslardir.
\par 19 Eger yalniz bu yasam için Mesih'e umut baglamissak, herkesten çok acinacak durumdayiz.
\par 20 Oysa Mesih, ölmüs olanlarin ilk örnegi olarak ölümden dirilmistir.
\par 21 Ölüm bir insan araciligiyla geldigine göre, ölümden dirilis de bir insan araciligiyla gelir.
\par 22 Herkes nasil Adem'de ölüyorsa, herkes Mesih'te yasama kavusacak.
\par 23 Her biri sirasi gelince dirilecek: Ilk örnek olarak Mesih, sonra Mesih'in gelisinde Mesih'e ait olanlar.
\par 24 Bundan sonra Mesih her yönetimi, her hükümranligi, her gücü ortadan kaldirip egemenligi Baba Tanri'ya teslim ettigi zaman son gelmis olacak.
\par 25 Çünkü Tanri bütün düsmanlarini ayaklari altina serinceye dek O'nun egemenlik sürmesi gerekir.
\par 26 Ortadan kaldirilacak son düsman ölümdür.
\par 27 Çünkü, "Tanri her seyi Mesih'in ayaklari altina sererek O'na bagimli kildi." "Her sey O'na bagimli kilindi" sözünün, her seyi Mesih'e bagimli kilan Tanri'yi içermedigi açiktir.
\par 28 Her sey Ogul'a bagimli kilininca, Ogul da her seyi kendisine bagimli kilan Tanri'ya bagimli olacaktir. Öyle ki, Tanri her seyde her sey olsun.
\par 29 Dirilis yoksa, ölüler için vaftiz* edilenler ne olacak? Ölüler gerçekten dirilmeyecekse, insanlar neden ölüler için vaftiz ediliyorlar?
\par 30 Biz de neden her saat kendimizi tehlikeye atiyoruz?
\par 31 Kardesler, sizinle ilgili olarak Rabbimiz Mesih Isa'da sahip oldugum övüncün hakki için her gün ölüyorum.
\par 32 Eger insansal nedenlerle Efes'te canavarlarla dövüstümse, bunun bana yarari ne? Eger ölüler dirilmeyecekse, "Yiyelim içelim, nasil olsa yarin ölecegiz."
\par 33 Aldanmayin, "Kötü arkadasliklar iyi huyu bozar."
\par 34 Uslanip kendinize gelin, artik günah islemeyin. Bazilariniz Tanri'yi hiç tanimiyor. Utanasiniz diye söylüyorum bunlari.
\par 35 Ama biri çikip, "Ölüler nasil dirilecek? Nasil bir bedenle gelecekler?" diye sorabilir.
\par 36 Ne akilsizca bir soru! Ektigin tohum ölmedikçe yasama kavusmaz ki!
\par 37 Ekerken, olusacak bitkinin kendisini degil, yalnizca tohumunu -bugday ya da baska bir bitkinin tohumunu ekersin.
\par 38 Tanri tohuma diledigi bedeni -her birine kendine özgü bedeni- verir.
\par 39 Her canlinin eti ayni degildir. Insan eti baska, hayvan eti baska, kus eti, balik eti baska baskadir.
\par 40 Göksel bedenler vardir, dünyasal bedenler vardir. Göksel olanlarin görkemi baska, dünyasal olanlarinki baskadir.
\par 41 Günesin görkemi baska, ayin görkemi baska, yildizlarin görkemi baskadir. Görkem bakimindan yildiz yildizdan farklidir.
\par 42 Ölülerin dirilisi de böyledir. Beden çürümeye mahkûm olarak gömülür, çürümez olarak diriltilir.
\par 43 Düskün olarak gömülür, görkemli olarak diriltilir. Zayif olarak gömülür, güçlü olarak diriltilir.
\par 44 Dogal beden olarak gömülür, ruhsal beden olarak diriltilir. Dogal beden oldugu gibi, ruhsal beden de vardir.
\par 45 Nitekim söyle yazilmistir: "Ilk insan Adem yasayan can oldu." Son Adem'se yasam veren ruh oldu.
\par 46 Önce ruhsal olan degil, dogal olan geldi. Ruhsal olan sonra geldi.
\par 47 Ilk insan yerden, yani topraktandir. Ikinci insan göktendir.
\par 48 Topraktan olan insan nasilsa, topraktan olanlar da öyledir. Göksel insan nasilsa, göksel olanlar da öyledir.
\par 49 Bizler topraktan olana nasil benzediysek, göksel olana da benzeyecegiz.
\par 50 Kardesler, sunu demek istiyorum, et ve kan Tanri'nin Egemenligi'ni miras alamaz. Çürüyen de çürümezligi miras alamaz.
\par 51 Iste size bir sir açikliyorum. Hepimiz ölmeyecegiz; son borazan çalininca hepimiz bir anda, göz açip kapayana dek degistirilecegiz. Evet, borazan çalinacak, ölüler çürümez olarak dirilecek, ve biz de degistirilecegiz.
\par 53 Çünkü bu çürüyen beden çürümezligi, bu ölümlü beden ölümsüzlügü giyinmelidir.
\par 54 Çürüyen ve ölümlü beden çürümezligi ve ölümsüzlügü giyinince, "Ölüm yok edildi, zafer kazanildi!" diye yazilmis olan söz yerine gelecektir.
\par 55 "Ey ölüm, zaferin nerede? Ey ölüm, dikenin nerede?"
\par 56 Ölümün dikeni günahtir. Günah ise gücünü Kutsal Yasa'dan alir.
\par 57 Rabbimiz Isa Mesih araciligiyla bizi zafere ulastiran Tanri'ya sükürler olsun!
\par 58 Bu nedenle, sevgili kardeslerim, Rab yolunda verdiginiz emegin bosa gitmeyecegini bilerek dayanin, sarsilmayin, Rab'bin isinde her zaman gayretli olun.

\chapter{16}

\par 1 Kutsallara yapilacak para yardimina gelince: Galatya kiliselerine* ne buyurduysam, siz de öyle yapin.
\par 2 Haftanin ilk günü* herkes kazancina göre bir miktar para ayirip biriktirsin. Öyle ki, yaniniza geldigimde para toplamaya gerek kalmasin.
\par 3 Oraya vardigimda, bagislarinizi götürmek üzere uygun gördügünüz kisileri tanitici mektuplarla Yerusalim'e* gönderecegim.
\par 4 Benim de gitmeme degerse, onlari yanima alip gidecegim.
\par 5 Makedonya'dan geçtikten sonra yaniniza gelecegim. Çünkü Makedonya'dan geçmek niyetindeyim.
\par 6 Belki bir süre yaninizda kalirim, hatta kisi da sizinle geçirebilirim. Öyleki, sonra nereye gidecek olsam, bana yardim edebilesiniz.
\par 7 Sizi öyle kisaca görüp geçmek istemiyorum. Rab'bin izniyle uzunca bir süre aninizda kalmayi umut ediyorum.
\par 8 Ama Pentikost Günü'ne dek Efes'te kalacagim.
\par 9 Çünkü büyük ve etkili isler yapmam için burada bana bir kapi açildi. Ne var ki, bana karsi çikanlar çoktur.
\par 10 Timoteos yaniniza gelirse, bir seyden korkmamasina dikkat edin. Çünkü o da benim gibi Rab'bin isini yapiyor.
\par 11 Kimse onu hor görmesin. Yanima gelmesi için onu esenlikle ugurlayin. Kardeslerle birlikte onun da gelmesini bekliyorum.
\par 12 Kardesimiz Apollos'a gelince, kardeslerle birlikte size gelmesi için ona çok ricada bulundum, ama simdilik gelmeye hiç de istekli degil. Firsat bulunca gelecek.
\par 13 Uyanik kalin, imanda dimdik durun, mert ve güçlü olun.
\par 14 Her seyi sevgiyle yapin.
\par 15 Ahaya'da ilk iman eden ve kendilerini kutsallarin hizmetine adayan Istefanas'in ev halkini bilirsiniz. Kardesler, size yalvaririm, bu gibilere ve onlarla birlikte çalisip emek verenlerin hepsine bagimli olun.
\par 17 Istefanas, Fortunatus ve Ahaykos'un gelisine sevindim. Yoklugunuzu bana unutturdular.
\par 18 Sizin ruhunuzu da benim ruhumu da ferahlattilar. Böylelerinin degerini bilin.
\par 19 Asya Ili'ndeki* kiliseler* size selam eder. Akvila ve Priska, evlerinde bulusan toplulukla* birlikte Rab'de size çok selam ederler.
\par 20 Buradaki bütün kardeslerin size selami var. Birbirinizi kutsal öpüsle selamlayin.
\par 21 Ben Pavlus, bu selami kendi elimle yaziyorum.
\par 22 Rab'bi sevmeyene lanet olsun. Maranata!
\par 23 Rab Isa'nin lütfu sizinle birlikte olsun.
\par 24 Hepinize Mesih Isa'da sevgiler! Amin.


\end{document}