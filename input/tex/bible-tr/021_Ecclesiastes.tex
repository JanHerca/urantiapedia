\begin{document}

\title{Vaiz}


\chapter{1}

\par 1 Bunlar Yerusalim'de krallik yapan Davut oglu Vaiz'in sözleridir:
\par 2 "Her sey bos, bombos, bombos!" diyor Vaiz.
\par 3 Ne kazanci var insanin Günesin altinda harcadigi onca emekten?
\par 4 Kusaklar gelir, kusaklar geçer, Ama dünya sonsuza dek kalir.
\par 5 Günes dogar, günes batar, Hep dogdugu yere kosar.
\par 6 Rüzgar güneye gider, kuzeye döner, Döne döne eserek Hep ayni yolu izler.
\par 7 Bütün irmaklar denize akar, Yine de deniz dolmaz. Irmaklar hep çiktiklari yere döner.
\par 8 Her sey yorucu, Sözcüklerle anlatilamayacak kadar. Göz görmekle doymuyor, Kulak isitmekle dolmuyor.
\par 9 Önce ne olduysa, yine olacak. Önce ne yapildiysa, yine yapilacak. Günesin altinda yeni bir sey yok.
\par 10 Var mi kimsenin, "Bak bu yeni!" diyebilecegi bir sey? Her sey çoktan, bizden yillar önce de vardi.
\par 11 Geçmis kusaklar animsanmiyor, Gelecek kusaklar da kendilerinden sonra gelenlerce animsanmayacak.
\par 12 Ben Vaiz, Yerusalim'de Israil kraliyken
\par 13 kendimi göklerin altinda yapilan her seyi bilgece arastirip incelemeye adadim. Tanri'nin ugrassinlar diye insanlara verdigi çetin bir zahmettir bu.
\par 14 Günesin altinda yapilan bütün isleri gördüm; hepsi bostur, rüzgari kovalamaya kalkismaktir!
\par 15 Egri olan dogrultulamaz, eksik olan sayilamaz.
\par 16 Kendi kendime, "Iste, bilgeligimi benden önce Yerusalim'de krallik yapan herkesten çok artirdim" dedim, "Alabildigine bilgi ve bilgelik edindim."
\par 17 Kendimi bilgi ve bilgeligi, deliligi ve akilsizligi anlamaya adadim. Gördüm ki, bu da yalnizca rüzgari kovalamaya kalkismakmis.
\par 18 Çünkü çok bilgelik çok keder dogurur, bilgi arttikça aci da artar.

\chapter{2}

\par 1 Kendi kendime, "Gel, zevki tat. Iyi mi, degil mi, gör" dedim. Ama gördüm ki, o da bos.
\par 2 Gülmeye, "Delilik", zevke, "Ne ise yarar?" dedim.
\par 3 Insanlarin göklerin altinda geçirdigi birkaç günlük ömürleri boyunca, yapacaklari iyi bir sey olup olmadigini görünceye dek, bilgeligimin önderliginde, bedenimi sarapla nasil canlandirayim, akilsizligi nasil ele alayim diye düsündüm durdum.
\par 4 Büyük islere girdim. Kendime evler insa ettim, baglar diktim.
\par 5 Bahçeler, parklar yaptim, oralara türlü türlü meyve agaçlari diktim.
\par 6 Dal budak salan orman agaçlarini sulamak için havuzlar yaptim.
\par 7 Kadin, erkek köleler satin aldim; evimde dogan kölelerim de vardi. Ayrica benden önce Yerusalim'de yasayan herkesten çok sigira, davara sahip oldum.
\par 8 Altin, gümüs biriktirdim; krallarin, illerin hazinelerini topladim. Kadin, erkek sarkicilar ve erkeklerin özlemi olan bir harem edindim.
\par 9 Böylece büyük üne kavustum, benden önce Yerusalim'de yasayanlarin hepsini astim. Bilgeligimden de bir sey yitirmedim.
\par 10 Gözümün diledigi hiçbir seyi kendimden esirgemedim. Gönlümü hiçbir zevkten alikoymadim. Yaptigim her isten zevk aldi gönlüm. Bütün emegimin ödülü bu oldu.
\par 11 Yaptigim bütün islere, Çektigim bütün emeklere bakinca, Gördüm ki, hepsi bos ve rüzgari kovalamaya kalkismakmis. Günesin altinda hiçbir kazanç yokmus.
\par 12 Sonra bilgelik, delilik, akilsizlik nedir diye baktim; Çünkü kralin yerine geçecek kisi Zaten yapilanin ötesinde ne yapabilir ki?
\par 13 Isigin karanliktan üstün oldugu gibi Bilgeligin de akilsizliktan üstün oldugunu gördüm.
\par 14 Bilge nereye gittigini görür, Ama akilsiz karanlikta yürür. Ikisinin de ayni sonu paylastigini gördüm.
\par 15 "Akilsizin basina gelen, benim de basima gelecek" Dedim kendi kendime, "Öyleyse kazancim ne bilgelikten?" "Bu da bos" dedim içimden.
\par 16 Çünkü akilsiz gibi, bilge de uzun süre anilmaz, Gelecekte ikisi de unutulur. Nitekim bilge de akilsiz gibi ölür!
\par 17 Böylece hayattan nefret ettim. Çünkü günesin altinda yapilan is çetindi bence. Her sey bos ve rüzgari kovalamaya kalkismakmis.
\par 18 Günesin altinda harcadigim bütün emekten nefret ettim. Çünkü her seyi benden sonra gelecek olana birakmak zorundayim.
\par 19 Kim bilir, bilge mi olacak, akilsiz mi? Günesin altinda bilgeligimi kullanarak harcadigim bütün emek üzerinde saltanat sürecek. Bu da bos.
\par 20 Bu yüzden günesin altinda harcadigim onca emege üzülmeye basladim.
\par 21 Çünkü biri bilgelik, bilgi ve beceriyle çalisir, sonunda her seyini hiç emek vermemis baska birine birakmak zorunda kalir. Bu da bos ve büyük bir hüsrandir.
\par 22 Çünkü ne kazanci var adamin, günesin altinda harcadigi bunca emekten, bunca kafa yormaktan?
\par 23 Günler boyunca çektigi zahmet aci ve dert dogurur. Gece bile içi rahat etmez. Bu da bos.
\par 24 Insan için yemekten, içmekten ve yaptigi isten zevk almaktan daha iyi bir sey yoktur. Gördüm ki, bu da Tanri'dandir.
\par 25 O'nsuz kim yiyebilir, kim zevk alabilir?
\par 26 Çünkü Tanri bilgiyi, bilgeligi, sevinci hosnut kaldigi insana verir. Günahkâra ise, yigma, biriktirme zahmeti verir; biriktirdiklerini Tanri'nin hosnut kaldigi insanlara biraksin diye. Bu da bos ve rüzgari kovalamaya kalkismakmis.

\chapter{3}

\par 1 Her seyin mevsimi, göklerin altindaki her olayin zamani vardir.
\par 2 Dogmanin zamani var, ölmenin zamani var. Dikmenin zamani var, sökmenin zamani var.
\par 3 Öldürmenin zamani var, sifa vermenin zamani var. Yikmanin zamani var, yapmanin zamani var.
\par 4 Aglamanin zamani var, gülmenin zamani var. Yas tutmanin zamani var, oynamanin zamani var.
\par 5 Tas atmanin zamani var, tas toplamanin zamani var. Kucaklasmanin zamani var, kucaklasmamanin zamani var.
\par 6 Aramanin zamani var, vazgeçmenin zamani var. Saklamanin zamani var, atmanin zamani var.
\par 7 Yirtmanin zamani var, dikmenin zamani var. Susmanin zamani var, konusmanin zamani var.
\par 8 Sevmenin zamani var, nefret etmenin zamani var. Savasin zamani var, barisin zamani var.
\par 9 Çalisanin harcadigi emekten ne kazanci var?
\par 10 Tanri'nin ugrassinlar diye insanlara verdigi zahmeti gördüm.
\par 11 O her seyi zamaninda güzel yapti. Insanlarin yüregine sonsuzluk kavramini koydu. Yine de insan Tanri'nin yaptigi isi basindan sonuna dek anlayamaz.
\par 12 Insan için yasami boyunca mutlu olmaktan, iyi yasamaktan daha iyi bir sey olmadigini biliyorum.
\par 13 Her insanin yiyip içmesi, yaptigi her isle doyuma ulasmasi bir Tanri armaganidir.
\par 14 Tanri'nin yaptigi her seyin sonsuza dek sürecegini biliyorum. Ona ne bir sey eklenebilir ne de ondan bir sey çikarilabilir. Tanri insanlarin kendisine saygi duymalari için bunu yapiyor.
\par 15 Simdi ne oluyorsa, geçmiste de oldu, Ne olacaksa, daha önce de olmustur. Tanri geçmis olaylarin hesabini soruyor.
\par 16 Günesin altinda bir sey daha gördüm: Adaletin ve dogrulugun yerini kötülük almis.
\par 17 Içimden "Tanri dogruyu da, kötüyü de yargilayacaktir" dedim, "Çünkü her olayin, her eylemin zamanini belirledi."
\par 18 Insanlara gelince, "Tanri hayvan olduklarini görsünler diye insanlari siniyor" diye düsündüm.
\par 19 Çünkü insanlarin basina gelen hayvanlarin da basina geliyor. Ayni sonu paylasiyorlar. Biri nasil ölüyorsa, öbürü de öyle ölüyor. Hepsi ayni solugu tasiyor. Insanin hayvandan üstünlügü yoktur. Çünkü her sey bos.
\par 20 Ikisi de ayni yere gidiyor; topraktan gelmis, topraga dönüyor.
\par 21 Kim biliyor insan ruhunun yukariya çiktigini, hayvan ruhunun asagiya, yeraltina indigini?
\par 22 Sonuçta insanin yaptigi isten zevk almasindan daha iyi bir sey olmadigini gördüm. Çünkü onun payina düsen budur. Kendisinden sonra olacaklari görmesi için kim onu geri getirebilir?

\chapter{4}

\par 1 Günesin altinda yapilan baskilara bir daha baktim, Ezilenlerin gözyaslarini gördüm; Avutanlari yok, Güç ezenlerden yana, Avutanlari yok.
\par 2 Çoktan ölmüs ölüleri, Hâlâ sag olan yasayanlardan daha mutlu gördüm.
\par 3 Ama henüz dogmamis, Günesin altinda yapilan kötülükleri görmemis olan Ikisinden de mutludur.
\par 4 Harcanan her emegin, yapilan her ustaca isin ardinda kiskançlik oldugunu gördüm. Bu da bos ve rüzgari kovalamaya kalkismakmis.
\par 5 Akilsiz ellerini kavusturup kendi kendini yer.
\par 6 Rahat kazanilan bir avuç dolusu Zahmtle, rüzgari kovalamaya kalkisarak kazanilan Iki avuç dolusundan daha iyidir.
\par 7 Günesin altinda bir bos sey daha gördüm:
\par 8 Yalniz bir adam vardi, Oglu da kardesi de yoktu. Çabalari dinmek nedir bilmezdi, Gözü zenginlige doymazdi. "Kimin için çalisiyorum, Neden kendimi zevkten yoksun birakiyorum?" diye sormazdi. Bu da bos ve çetin bir zahmettir.
\par 9 Iki kisi bir kisiden iyidir, Çünkü emeklerine iyi karsilik alirlar.
\par 10 Biri düserse, öteki kaldirir. Ama yalniz olup da düsenin vay haline! Onu kaldiran olmaz.
\par 11 Ayrica iki kisi birlikte yatarsa, birbirini isitir. Ama tek basina yatan nasil isinabilir?
\par 12 Yalniz biri yenik düser, Ama iki kisi direnebilir. Üç kat iplik kolay kolay kopmaz.
\par 13 Yoksul ama bilge bir genç artik ögüt almayi bilmeyen kocamis akilsiz kraldan iyidir.
\par 14 Çünkü genç, ülkesinde yoksulluk içinde dogsa bile cezaevinden kralliga yükselebilir.
\par 15 Günesin altinda yasayan herkesin kralin yerine geçen genci izledigini gördüm.
\par 16 Yeni kralin yönettigi halk sayisiz olabilir. Yine de sonrakiler ondan hosnut olmayabilir. Gerçekten bu da bos ve rüzgari kovalamaya kalkismaktir.

\chapter{5}

\par 1 Tanri'nin evine gittiginde davranisina dikkat et. Yaptiklari kötülügün farkinda olmayan akilsizlar gibi kurban sunmak için degil, dinlemek için yaklas.
\par 2 Agzini çabuk açma, Tanri'nin önünde hemen konuya girme, Çünkü Tanri gökte, sen yerdesin, Bu yüzden, az konus.
\par 3 Çok tasa kötü düs, Çok söz akilsizlik dogurur.
\par 4 Tanri'ya adak adayinca, yerine getirmekte gecikme. Çünkü O akilsizlardan hoslanmaz. Adagini yerine getir!
\par 5 Adamamak, adayip da yerine getirmemekten iyidir.
\par 6 Agzinin seni günaha sürüklemesine izin verme. Ulagin önünde: "Adagim yanlisti" deme. Tanri niçin senin sözlerine öfkelensin, yaptigin isi yok etsin?
\par 7 Çünkü çok düs kurmak hayalcilige ve laf kalabaligina yol açar; Tanri'ya saygi göster.
\par 8 Bir yerde yoksullara baski yapildigini, adaletin ve dogrulugun çignendigini görürsen sasma; çünkü üstü gözeten daha üst biri var, onlarin da üstleri var.
\par 9 Tarlalarin sürülmesini isteyen bir kral ülke için her bakimdan yararlidir.
\par 10 Parayi seven paraya doymaz, Zenginligi seven kazanciyla yetinmez. Bu da bostur.
\par 11 Mal çogaldikça yiyeni de çogalir. Sahibine ne yarari var, seyretmekten baska?
\par 12 Az yesin, çok yesin isçi rahat uyur, Ama zenginin mali zengini uyutmaz.
\par 13 Günesin altinda aci bir kötülük gördüm: Sahibinin zararina biriktirilen Ve bir talihsizlikle yok olup giden servet. Böyle bir servet sahibi baba olsa bile, Ogluna bir sey birakamaz.
\par 15 Annesinin rahminden çiplak çikar insan. Dünyaya nasil geldiyse öyle gider, Emeginden hiçbir sey götürmez elinde.
\par 16 Dünyaya nasil geldiyse öyle gider insan. Bu da aci bir kötülüktür. Ne kazanci var yel için zahmet çekmekten?
\par 17 Ömrü boyunca büyük üzüntü, hastalik, öfke içinde Karanlikta yiyor.
\par 18 Gördüm ki, iyi ve güzel olan su: Tanri'nin insana verdigi birkaç günlük ömür boyunca yemek, içmek, günesin altinda harcadigi emekten zevk almak. Çünkü insanin payina düsen budur.
\par 19 Üstelik Tanri bir insana mal mülk veriyor, onu yemesi, ödülünü almasi, yaptigi isten mutluluk duymasi için ona güç veriyorsa, bu bir Tanri armaganidir.
\par 20 Bu yüzden insan, geçen ömrünü pek düsünmez. Çünkü Tanri onun yüregini mutlulukla mesgul eder.

\chapter{6}

\par 1 Günesin altinda insana agir gelen bir kötülük gördüm:
\par 2 Adam vardir, Tanri kendisine mal, mülk, sayginlik verir, yerine gelmeyecek istegi yoktur. Ama Tanri yemesine izin vermez; bir yabanci yer. Bu da bos ve aci bir derttir.
\par 3 Bir adam yüz çocuk babasi olup uzun yillar yasamis, ama uzun ömrüne karsilik, zenginligin tadini çikaramamis, bir mezara bile gömülmemisse, düsük çocuk ondan iyidir derim.
\par 4 Çünkü düsük çocuk bos yere doguyor, karanlik içinde geçip gidiyor, adi karanliga gömülüyor.
\par 5 Ne günes yüzü görüyor, ne de bir sey taniyor. Öbür adam iki kez biner yil yasasa bile mutluluk duymaz, düsük çocuk ondan rahattir. Hepsi ayni yere gitmiyor mu?
\par 7 Insan hep bogazi için çalisir, Yine de doymaz.
\par 8 Bilgenin akilsizdan ne üstünlügü var? Yoksul baskasina nasil davranacagini bilmekle ne yarar saglar?
\par 9 Gözün gördügü gönlün çektiginden iyidir. Bu da bos ve rüzgari kovalamaya kalkismaktir.
\par 10 Ne varsa, adi çoktan konmustur, Insanin da ne oldugu biliniyor. Kimse kendinden güçlü olanla çekisemez.
\par 11 Söz çogaldikça anlam azalir, Bunun kime yarari olur?
\par 12 Çünkü gölge gibi gelip geçen kisa ve bos ömründe insana neyin yararli oldugunu kim bilebilir? Bir adama kendisinden sonra günesin altinda neler olacagini kim söyleyebilir?

\chapter{7}

\par 1 Iyi ad hos kokulu yagdan, Ölüm günü dogum gününden iyidir.
\par 2 Yas evine gitmek, sölen evine gitmekten iyidir. Çünkü her insanin sonu ölümdür, Yasayan herkes bunu aklinda tutmali.
\par 3 Üzüntü gülmekten iyidir, Çünkü yüz mahzun olunca yürek sevinir.
\par 4 Bilge kisinin akli yas evindedir, Akilsizin akliysa senlik evinde.
\par 5 Bilgenin azarini isitmek, Akilsizin türküsünü isitmekten iyidir.
\par 6 Çünkü akilsizin gülmesi, Kazanin altindaki çalilarin çatirtisi gibidir. Bu da bostur.
\par 7 Haksiz kazanç bilgeyi delirtir, Rüsvet karakteri bozar.
\par 8 Bir olayin sonu baslangicindan iyidir. Sabirli kibirliden iyidir.
\par 9 Çabuk öfkelenme, Çünkü öfke akilsizlarin bagrinda barinir.
\par 10 "Neden geçmis günler bugünlerden iyiydi?" diye sorma, Çünkü bu bilgece bir soru degil.
\par 11 Bilgelik miras kadar iyidir, Günesi gören herkes için yararlidir.
\par 12 Bilgelik siperdir, para da siper, Bilginin yarari ise sudur: Bilgelik ona sahip olan kisinin yasamini korur.
\par 13 Tanri'nin yaptigini düsün: O'nun egrilttigini kim dogrultabilir?
\par 14 Iyi günde mutlu ol, Ama kötü günde dikkatle düsün; Tanri birini öbürü gibi yapti ki, Insan kendisinden sonra neler olacagini bilmesin.
\par 15 Bos ömrümde sunlari gördüm: Dogru insan dogruluguna karsin ölüyor, Kötü insanin ise, kötülügüne karsin ömrü uzuyor.
\par 16 Ne çok dogru ol ne de çok bilge. Niçin kendini yok edesin?
\par 17 Ne çok kötü ol ne de akilsiz. Niçin vaktinden önce ölesin?
\par 18 Birini tutman iyidir, Öbüründen de elini çekme. Çünkü Tanri'ya saygi duyan ikisini de basarir.
\par 19 Bilgelik, bilge kisiyi kentteki on yöneticiden daha güçlü kilar.
\par 20 Çünkü yeryüzünde hep iyilik yapan, Hiç günah islemeyen dogru insan yoktur.
\par 21 Insanlarin söyledigi her söze aldirma, Yoksa usaginin bile sana sövdügünü duyabilirsin.
\par 22 Çünkü sen de birçok kez Baskalarina sövdügünü pekâlâ biliyorsun.
\par 23 Bütün bunlari bilgelikle denedim: "Bilge olacagim" dedim. Ama bu beni asiyordu.
\par 24 Bilgelik denen sey Uzak ve çok derindir, onu kim bulabilir?
\par 25 Böylece, bilgelik ve çözüm aramaya, incelemeye, kavramaya, Kötülügün akilsizlik, akilsizligin delilik oldugunu anlamaya kafa yordum.
\par 26 Kimi kadini ölümden aci buldum. O kadin ki, kendisi tuzak, yüregi kapan, elleri zincirdir. Tanri'nin hosnut kaldigi insan ondan kaçar, Günah isleyense ona tutsak olur.
\par 27 Vaiz diyor ki, "Sunu gördüm: Bir çözüm bulmak için Bir seyi öbürüne eklerken
\par 28 -Arastirip hâlâ bulamazken- Binde bir adam buldum, Ama aralarinda bir kadin bulamadim.
\par 29 Buldugum tek sey: Tanri insanlari dogru yaratti, Oysa onlar hâlâ karmasik çözümler ariyorlar."

\chapter{8}

\par 1 Bilge insan gibisi var mi? Kim olup bitenlerin anlamini bilebilir? Bilgelik insanin yüzünü aydinlatir, Sert görünüsünü degistirir.
\par 2 Kralin buyruguna uy, diyorum. Çünkü Tanri'nin önünde ant içtin.
\par 3 Kralin huzurundan ayrilmak için acele etme. Kötülüge bulasma. Çünkü o diledigi her seyi yapar.
\par 4 Kralin sözünde güç vardir. Kim ona, "Ne yapiyorsun?" diyebilir?
\par 5 Onun buyruguna uyan zarar görmez. Bilge kisi bunun zamanini ve yolunu bilir.
\par 6 Çünkü her isin bir zamani ve yolu vardir. Insanin derdi kendine yeter.
\par 7 Kimse gelecegi bilmez, Kim kime gelecegi bildirebilir?
\par 8 Rüzgari tutup ona egemen olmaya kimsenin gücü yetmedigi gibi, Ölüm gününe egemen olmaya da kimsenin gücü yetmez. Savastan kaçis olmadigi gibi, kötülük de sahibini kurtaramaz.
\par 9 Bütün bunlari gördüm ve günesin altinda yapilan her is üzerinde kafa yordum. Gün gelir, insanin insana egemenligi kendine zarar verir.
\par 10 Bir de kötülerin gömüldügünü gördüm. Kutsal yere girip çikar, kötülük yaptiklari kentte övülürlerdi. Bu da bos.
\par 11 Suçlu çabuk yargilanmazsa, insanlar kötülük etmek için cesaret bulur.
\par 12 Günahli yüz kez kötülük edip uzun yasasa bile, Tanri'dan korkanlarin, O'nun önünde saygiyla duranlarin iyilik görecegini biliyorum.
\par 13 Oysa kötü, Tanri'dan korkmadigi için iyilik görmeyecek, gölge gibi olan ömrü uzamayacaktir.
\par 14 Yeryüzünde bos bir sey daha var: Kötülerin hak ettigi dogrularin, dogrularin hak ettigiyse kötülerin basina geliyor. Bu da bos, diyorum.
\par 15 Mutlulugu övgüye deger buldum. Çünkü günesin altinda insan için yiyip içmekten, mutlu olmaktan daha iyi bir sey yoktur. Çünkü Tanri'nin günesin altinda kendisine verdigi ömür boyunca çektigi zahmetten insana kalacak olan budur.
\par 16 Bilgeligi ve dünyada çekilen zahmeti anlamak için kafami yorunca öyleleri var ki, gece gündüz gözüne uyku girmez
\par 17 Tanri'nin yaptigi her seyi gördüm. Insan günesin altinda olup bitenleri kesfedemez. Arayip bulmak için ne kadar çaba harcarsa harcasin, yine de anlamini bulamaz. Bilge kisi anladigini söylese bile gerçekten kavrayamaz.

\chapter{9}

\par 1 Böylece bütün bunlari düsünüp tasindim ve su sonuca vardim: Dogrular, bilgeler ve yaptiklari her sey Tanri'nin elindedir. Onlari sevginin mi, nefretin mi bekledigini kimse bilmez.
\par 2 Herkesin basina ayni sey geliyor. Dogrunun, iyinin, kötünün, temizin, kirlinin, kurban sunanla sunmayanin basina gelen sey ayni. Iyi insana ne oluyorsa, günahliya da oluyor; Ant içene ne oluyorsa, ant içmekten korkana da aynisi oluyor.
\par 3 Günesin altinda yapilan islerin tümünün kötü yani su ki, herkesin basina ayni sey geliyor. Üstelik insanlarin içi kötülük doludur, yasadiklari sürece içlerinde delilik vardir. Ardindan ölüp gidiyorlar.
\par 4 Yasayanlar arasindaki herkes için umut vardir. Evet, sag köpek ölü aslandan iyidir!
\par 5 Çünkü yasayanlar ölecegini biliyor, Ama ölüler hiçbir sey bilmiyor. Onlar için artik ödül yoktur, Anilari bile unutulmustur.
\par 6 Sevgileri, nefretleri, Kiskançliklari çoktan bitmistir. Günesin altinda yapilanlardan Bir daha paylari olmayacaktir.
\par 7 Git, sevinçle ekmegini ye, neseyle sarabini iç. Çünkü yaptiklarin bastan beri Tanri'nin hosuna gitti.
\par 8 Giysilerin hep ak olsun. Basindan zeytinyagi eksilmesin.
\par 9 Günesin altinda Tanri'nin sana verdigi bos ömrün bütün günlerini, bütün anlamsiz günlerini sevdigin karinla güzel güzel yasayarak geçir. Çünkü hayattan ve günesin altinda harcadigin emekten payina düsecek olan budur.
\par 10 Çalismak için eline ne geçerse, var gücünle çalis. Çünkü gitmekte oldugun ölüler diyarinda is, tasari, bilgi ve bilgelik yoktur.
\par 11 Günesin altinda bir sey daha gördüm: Yarisi hizli kosanlar, Savasi yigitler, Ekmegi bilgeler, Serveti akillilar, Begeniyi bilgililer kazanmaz. Ama zaman ve sans hepsinin önüne çikar.
\par 12 Dahasi insan kendi vaktini bilmez: Baligin acimasiz aga, kusun kapana düstügü gibi, Insanlar da üzerlerine ansizin çöken kötü zamana yakalanirlar.
\par 13 Günesin altinda bilgelik olarak sunu da gördüm, beni çok etkiledi:
\par 14 Çok az insanin yasadigi küçük bir kent vardi. Güçlü bir kral saldirip onu kusatti. Karsisina büyük rampalar kurdu.
\par 15 Kentte yoksul ama bilge bir adam vardi. Bilgeligiyle kenti kurtardi. Ne var ki, kimse bu yoksul adami anmadi.
\par 16 Bunun üzerine, "Bilgelik güçten iyidir" dedim, "Ne yazik ki, yoksul insanin bilgeligi küçümseniyor, söyledikleri dikkate alinmiyor."
\par 17 Bilgenin sessizce söyledigi sözler, Akilsizlar arasindaki önderin bagirisindan iyidir.
\par 18 Bilgelik silahtan iyidir, Ama bir deli çikar, her seyi berbat eder.

\chapter{10}

\par 1 Ölü sinekler attarin itirini kokutur. Biraz aptallik da bilgeligi ve sayginligi bastirir.
\par 2 Bilgenin yüregi hep dogruya egilimlidir, Akilsizin ise, hep yanlisa.
\par 3 Yolda yürürken bile akilsizin akli kittir, Akilsiz oldugunu herkese gösterir.
\par 4 Yöneticinin öfkesi sana karsi alevlenirse, Yerinden ayrilma; Çünkü serinkanlilik büyük yanlislari bastirir.
\par 5 Günesin altinda gördügüm bir haksizlik var, Yöneticiden kaynaklanan bir yanlisi andiriyor:
\par 6 Zenginler düsük makamlarda otururken, Aptallar yüksek makamlara ataniyor.
\par 7 Köleleri at sirtinda, Önderleri yerde köleler gibi yürürken gördüm.
\par 8 Çukur kazan içine kendi düser, Duvarda gedik açani yilan sokar.
\par 9 Tas çikaran tastan incinir, Odun yaran tehlikeye girer.
\par 10 Balta körse, agzi bilenmemisse, Daha çok güç gerektirir; Ama bilgelik basari dogurur.
\par 11 Yilan büyü yapilmadan önce sokarsa, Büyücünün yarari olmaz.
\par 12 Bilgenin agzindan çikan sözler benimsenir, Oysa akilsiz kendi agziyla yikimina yol açar.
\par 13 Sözünün basi aptallik, Sonu zirdeliliktir.
\par 14 Akilsiz konustukça konusur. Kimse ne olacagini bilmez. Kim ona kendisinden sonra ne olacagini bildirebilir?
\par 15 Akilsizin emegi kendini öylesine yipratir ki, Kente bile nasil gidecegini bilemez.
\par 16 Kralin bir çocuksa, Önderlerin sabah sölen veriyorsa, vay sana, ey ülke!
\par 17 Kralin soyluysa, Önderlerin sarhosluk için degil Güçlenmek için vaktinde yemek yiyorsa, ne mutlu sana, ey ülke!
\par 18 Tembellikten dam çöker, Miskinlikten çati akar.
\par 19 Sölen eglenmek için yapilir, Sarap yasama sevinç katar, Paraysa her ihtiyaci karsilar.
\par 20 Içinden bile krala sövme, Yatak odanda zengine lanet etme, Çünkü gökte uçan kuslar haber tasir, Kanatli varliklar söyledigini aktarir.

\chapter{11}

\par 1 Ekmegini suya at, Çünkü günler sonra onu bulursun.
\par 2 Yedi, hatta sekiz kisiye pay ver, Çünkü ülkenin basina ne felaket gelecegini bilemezsin.
\par 3 Bulutlar su yüklüyse, Yeryüzüne döker yagmurlarini. Agaç ister güneye ister kuzeye devrilsin, Devrildigi yerde kalir.
\par 4 Rüzgari gözeten ekmez, Bulutlara bakan biçmez.
\par 5 Ana rahmindeki çocugun nasil ruh ve beden aldigini bilmedigin gibi, Her seyi yaratan Tanri'nin yaptiklarini da bilemezsin.
\par 6 Tohumunu sabah ek, Aksam da elin bos durmasin. Çünkü bu mu iyi, su mu, Yoksa ikisi de ayni sonucu mu verecek, bilemezsin.
\par 7 Isik tatlidir, Günesi görmek güzeldir.
\par 8 Evet, insan uzun yillar yasarsa, Sevinçle yasasin. Ama karanlik günleri unutmasin, Çünkü onlar da az degil. Gelecek her sey bostur.
\par 9 Ey delikanli, gençliginle sevin, Birak gençlik günlerinde yüregin sevinç duysun. Gönlünün isteklerini, gözünün gördüklerini izle, Ama bil ki, bütün bunlar için Tanri seni yargilayacaktir.
\par 10 Öyleyse at tasayi yüreginden, Uzaklastir derdi bedeninden. Çünkü gençlik de dinçlik de bostur.

\chapter{12}

\par 1 Bu yüzden zor günler gelmeden, "Zevk almiyorum" diyecegin yillar yaklasmadan, Günes, isik, ay ve yildizlar kararmadan Ve yagmurdan sonra bulutlar geri dönmeden, Gençlik günlerinde seni yaratani animsa.
\par 3 O gün, evi bekleyenler titreyecek, Güçlüler egilecek, Ögütücüler azaldigi için duracak, Pencereden bakanlar kararacak.
\par 4 Degirmen sesi yavaslayinca, Sokaga açilan çift kapi kapanacak, Insanlar kus sesiyle uyanacak, Ama sarkilarin sesini duyamayacaklar.
\par 5 Dahasi yüksek yerden, Sokaktaki tehlikelerden korkacaklar; Badem agaci çiçek açacak, Çekirge agirlasacak, Tutku zayiflayacak. Çünkü insan sonsuzluk evine gidecek, Yas tutanlar sokakta dolasacak.
\par 6 Gümüs tel kopmadan, Altin tas kirilmadan, Testi çesmede parçalanmadan, Kuyu makarasi kirilmadan,
\par 7 Toprak geldigi yere dönmeden, Ruh onu veren Tanri'ya dönmeden, Seni yaratani animsa.
\par 8 "Her sey bos" diyor Vaiz, "Bombos!"
\par 9 Vaiz yalniz bilge degildi, bildiklerini halka da ögretiyordu. Hesap etti, arastirdi ve birçok özdeyisi düzene soktu.
\par 10 Güzel sözler bulmaya çalisti. Yazdiklari gerçek ve dogrudur.
\par 11 Bilgelerin sözleri üvendire gibidir, derledikleri özdeyislerse, iyi çakilan çivi gibi; bir tek Çoban*fg* tarafindan verilmisler.
\par 12 Bunlarin disindakilerden sakin, evladim. Çok kitap yazmanin sonu yoktur, fazla arastirma da bedeni yipratir.
\par 13 Her sey duyuldu, sonuç su: Tanri'ya saygi göster, buyruklarini yerine getir, Çünkü her insanin görevi budur.
\par 14 Tanri her isi, her gizli seyi yargilayacaktir, Ister iyi ister kötü olsun.


\end{document}