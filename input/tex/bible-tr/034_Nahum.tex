\begin{document}

\title{Nahum}


\chapter{1}

\par 1 Ninova ile ilgili bildiri, Elkoslu Nahum'un görümünü anlatan kitaptir.
\par 2 RAB kiskanç, öç alici bir Tanri'dir. Öç alir ve gazapla doludur. Hasimlarindan öç alir, Düsmanlarina karsi öfkesi süreklidir.
\par 3 RAB tez öfkelenmez ve çok güçlüdür. Suçlunun suçunu asla yanina koymaz. Geçtigi yerde kasirgalar, firtinalar kopar. O'nun ayaklarinin tozudur bulutlar.
\par 4 Bir buyrukla kurutur denizi, Kurutur bütün irmaklari. Solar Basan'in, Karmel Dagi'nin yesillikleri Ve Lübnan'in çiçekleri.
\par 5 Daglar RAB'bin önünde titrer, Erir tepeler. Yer sarsilir önünde. Dünya ve üzerinde yasayanlarin tümü titrer.
\par 6 O'nun gazabina kim karsi durabilir, Kim dayanabilir kizgin öfkesine? Ates gibi dökülür öfkesi, Kayalari paramparça eder.
\par 7 RAB iyidir, Siginaktir sikinti aninda. Korur kendisine siginanlari.
\par 8 Ama Ninova'yi azgin sellerle yok edecek, Düsmanlarini karanliga sürecek.
\par 9 RAB'be karsi neler tasarlarsaniz, Hepsini yok edecek. Ikinci kez kimse karsi koyamayacak.
\par 10 Birbirine dolasmis dikenler gibi, Kuru aniz gibi, Yanip biteceksiniz, ey ayyaslar.
\par 11 Ey Ninova, RAB'be karsi kötülük tasarlayan, Ser ögütleyen kisi senden çikti.
\par 12 RAB diyor ki, "Asurlular güçlü ve çok olsalar bile, yok olup gidecekler. Ey halkim, seni sikintiya soktuysam da, bir daha sokmayacagim.
\par 13 Simdi boyundurugunu parçalayip üzerindeki baglari koparacagim."
\par 14 RAB, "Artik soyunu sürdürecek torunlarin olmasin" Diye buyurdu, ey Ninova. "Tanrilarinin tapinagindaki oyma ve dökme putlari yok edecegim" diyor. "Mezarini hazirlayacagim. Çünkü sen asagiliksin."
\par 15 Iste, müjde getirenin ayaklari daglari asip geliyor, Size esenlik haberini getiriyor. Ey Yahudalilar, bayramlarinizi kutlayin, Adak sözünüzü yerine getirin. O kötü ulusun istilasina ugramayacaksiniz bir daha. Çünkü o büsbütün yok edildi.

\chapter{2}

\par 1 Saldiri altindasin, ey Ninova, surlarini koru, Yolu gözle, belini dogrult, topla bütün gücünü.
\par 2 Çünkü RAB Yakup'un soyunu Israil'in eski görkemine kavusturacak; Düsmanlari onlari perisan edip asmalarini harap etmis olsa bile.
\par 3 Askerlerinin kalkanlari kipkizil, Yigitler allar kusanmis. Savas arabalarinin demirleri hazirlik günü nasil da parildiyor! Çam mizraklar sallaniyor havada.
\par 4 Sokaklardan firtina gibi geçiyor savas arabalari, Meydanlardan kosusuyorlar her yöne, Simsek gibi segirtiyorlar. Görünüsleri mesalelerden farksiz.
\par 5 Ninova Krali topluyor seçkin askerlerini, Ama sendeliyorlar yolda. Saldiranlar kent surlarina dogru segirtiyor, Siperler kuruluyor.
\par 6 Irmaklarin kapilari açildi Ve yerle bir oldu saray.
\par 7 Tanri'nin dedigi oldu, soyup götürdüler kenti. Güvercinler gibi inliyor kadin köleler, Gögüslerini döverek.
\par 8 Kaçip gidiyor Ninova halki, Bosalan bir havuzun suyu gibi, "Durun, durun!" diye bagiriyorlar, Ama geri dönüp bakan yok.
\par 9 Yagmalayin altinini, gümüsünü, Yok servetinin sonu. Her tür degerli esyayla dolup tasiyor.
\par 10 Yikildi, yerle bir oldu, viraneye döndü Ninova. Eriyor yürekler, Bükülüyor dizler, titriyor bedenler, Herkesin beti benzi soluyor.
\par 11 Aslanlarin inine, Yavru aslanlarin beslendigi yere ne oldu? Aslanla disisinin ve yavrularinin korkusuzca gezindigi yere ne oldu?
\par 12 Aslan, yavrularina yetecek kadarini avladi, Disileri için avini bogazladi. Magarasini avladiklariyla, Inini kurbanlariyla doldurdu.
\par 13 Her Seye Egemen RAB, "Sana karsiyim" diyor, "Yakacagim savas arabalarini, Dumanlari tütecek. Genç aslanlarini kiliç yiyip tüketecek. Yeryüzünde av birakmayacagim sana Ulaklarinin sesi isitilmeyecek artik."

\chapter{3}

\par 1 Elleri kanli kentin vay haline! Yalanla, talanla dolu. Yagmalamaktan geri kalmiyor.
\par 2 Kamçi saklamalari, tekerlek gürültüleri, Kosan atlar, sarsilan savas arabalari,
\par 3 Saldiran atlilar, çakan kiliçlar, Parildayan mizraklar, yigin yigin ölüler... Sayisiz ceset. Yürürken ayaklar takiliyor ölülere.
\par 4 Her sey o alimli, büyücü fahisenin sinirsiz ahlaksizligindan oldu. Fahiseligiyle uluslari, büyüleriyle halklari kendine tutsak etti.
\par 5 Her Seye Egemen RAB diyor ki, "Sana karsiyim, ey Ninova! Savuracagim eteklerini yüzüne. Uluslara çiplakligini, Halklara ayip yerlerini gösterecegim.
\par 6 Seni pislikle sivayip rezil edecegim. Dehsetle seyredecek herkes seni.
\par 7 Seni kim görse kaçacak. 'Harabeye döndü Ninova' diyecekler, 'Kim dövünecek onun için? Nereden bulalim onu avutacak birilerini?'"
\par 8 Sen No-Amon'dan daha mi üstünsün? O kent ki, kanallar arasindaydi, Suyla çevrelenmisti, Kalesi Nil Irmagi, surlariysa sulardi.
\par 9 Kûs* ve Misir onun sinirsiz gücünün kaynagiydi. Pût ve Luv da yandaslariydi.
\par 10 Öyleyken tutsak düstü, halki sürüldü. Yavrulari köse baslarinda paramparça edildi. Soylulari için kura çekildi, Zincire vuruldu ileri gelenleri.
\par 11 Aciyla kendinden geçeceksin, ey Ninova, Düsmanlarindan korunacak yer arayacaksin.
\par 12 Senin kalelerin incir agacinin ilk olgunlasan meyvesi gibidir. Bir silkeleyiste yiyenin agzina düsecekler.
\par 13 Askerlerine bak! Kadin gibi hepsi. Kapilarin ardina kadar düsmana açik. Ates yiyip bitirmis kapi sürgülerini.
\par 14 Kusatma vakti için su biriktir kendine, Savunmani güçlendir. Tugla yapmak için kili çigne, Kaliplari hazirla.
\par 15 Orada ates seni yiyip bitirecek, Kiliç seni kesip biçecek. Genç çekirgelerin yiyip bitirdigi ekin gibi yok olacaksin. Çekirgeler gibi, genç çekirgeler gibi çogalmalisin.
\par 16 Tüccarlarinin sayisi gökteki yildizlardan çok. Ama düsmanlarin genç çekirgeler gibi ülkeyi talan edip gidecekler.
\par 17 Koruyucularinla görevlilerin serin günlerde duvarlara konan çekirgeler gibidir, Günes dogunca uçup kayiplara karisan çekirge sürüsü gibi.
\par 18 Ey Asur Krali, yöneticilerin öldü, Uyudu sonsuza dek soylularin. Halkin daglara dagildi. Onlari toplayacak kimse yok.
\par 19 Ugradigin felaketten kurtulus yok, yaralarin ölümcül. Basina gelenleri duyanlar sevinçle el ovusturuyorlar. Çünkü dinmeyen vahsetinden kim kaçabildi ki?


\end{document}