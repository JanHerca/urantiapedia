\begin{document}

\title{Mısırdan Çıkış}


\chapter{1}

\par 1 Yakup'la birlikte aileleriyle Misir'a giden Israilogullari'nin adlari sunlardir:
\par 2 Ruben, ªimon, Levi, Yahuda,
\par 3 Issakar, Zevulun, Benyamin,
\par 4 Dan, Naftali, Gad, Aser.
\par 5 Yakup'un soyundan gelenler toplam yetmis kisiydi. Yusuf zaten Misir'daydi.
\par 6 Zamanla Yusuf, kardesleri ve o kusagin hepsi öldü.
\par 7 Ama soylari artti; üreyip çogaldilar, gittikçe büyüdüler, ülke onlarla dolup tasti.
\par 8 Sonra Yusuf hakkinda bilgisi olmayan yeni bir kral Misir'da tahta çikti.
\par 9 Halkina, "Bakin, Israilliler sayica bizden daha çok" dedi,
\par 10 "Gelin, onlara karsi aklimizi kullanalim, yoksa daha da çogalirlar; bir savas çikarsa, düsmanlarimiza katilip bize karsi savasir, ülkeyi terk ederler."
\par 11 Böylece Misirlilar Israilliler'in basina onlari agir islere kosacak angaryacilar atadilar. Israilliler firavun için Pitom ve Ramses adinda ambarli kentler yaptilar.
\par 12 Ama Misirlilar baski yaptikça Israilliler daha da çogalarak bölgeye yayildilar. Misirlilar korkuya kapilarak
\par 13 Israilliler'i amansizca çalistirdilar.
\par 14 Her türlü tarla isi, harç ve kerpiç yapimi gibi agir islerle yasami onlara zehir ettiler. Bütün islerinde onlari amansizca kullandilar.
\par 15 Misir Krali, ªifra ve Pua adindaki Ibrani ebelere söyle dedi:
\par 16 "Ibrani kadinlarini dogum sandalyesinde dogurturken iyi bakin; çocuk erkekse öldürün, kizsa dokunmayin."
\par 17 Ama ebeler Tanri'dan korkan kimselerdi, Misir Krali'nin buyruguna uymayarak erkek çocuklari sag biraktilar.
\par 18 Bunun üzerine Misir Krali ebeleri çagirtip, "Niçin yaptiniz bunu?" diye sordu, "Neden erkek çocuklari sag biraktiniz?"
\par 19 Ebeler, "Ibrani kadinlar Misirli kadinlara benzemiyor" diye yanitladilar, "Çok güçlüler. Daha ebe gelmeden doguruyorlar."
\par 20 Tanri ebelere iyilik etti. Halk çogaldikça çogaldi.
\par 21 Ebeler kendisinden korktuklari için Tanri onlari ev bark sahibi yapti.
\par 22 Bunun üzerine firavun bütün halkina buyruk verdi: "Dogan her Ibrani erkek çocuk Nil'e atilacak, kizlar sag birakilacak."

\chapter{2}

\par 1 Levili bir adam kendi oymagindan bir kizla evlendi.
\par 2 Kadin gebe kaldi ve bir erkek çocuk dogurdu. Güzel bir çocuk oldugunu görünce, onu üç ay gizledi.
\par 3 Daha fazla gizleyemeyecegini anlayinca, hasir bir sepet alip katran ve ziftle sivadi. Içine çocugu yerlestirip Nil kiyisindaki sazliga birakti.
\par 4 Çocugun ablasi kardesine ne olacagini görmek için uzaktan gözlüyordu.
\par 5 O sirada firavunun kizi yikanmak için irmaga indi. Hizmetçileri irmak kiyisinda yürüyorlardi. Sazlarin arasindaki sepeti görünce, firavunun kizi onu getirmesi için hizmetçisini gönderdi.
\par 6 Sepeti açinca aglayan çocugu gördü. Ona aciyarak, "Bu bir Ibrani çocugu" dedi.
\par 7 Çocugun ablasi firavunun kizina, "Gidip bir Ibrani sütnine çagirayim mi?" diye sordu, "Senin için bebegi emzirsin."
\par 8 Firavunun kizi, "Olur" diye yanitladi. Kiz gidip bebegin annesini çagirdi.
\par 9 Firavunun kizi kadina, "Bu bebegi al, benim için emzir, ücretin neyse veririm" dedi. Kadin bebegi alip emzirdi.
\par 10 Çocuk büyüyünce, onu geri getirdi. Firavunun kizi çocugu evlat edindi. "Onu sudan çikardim" diyerek adini Musa koydu.
\par 11 Musa büyüdükten sonra bir gün soydaslarinin yanina gitti. Yaptiklari agir isleri seyrederken bir Misirli'nin bir Ibrani'yi dövdügünü gördü.
\par 12 Çevresine göz gezdirdi; kimse olmadigini anlayinca, Misirli'yi öldürüp kuma gizledi.
\par 13 Ertesi gün gittiginde, iki Ibrani'nin kavga ettigini gördü. Haksiz olana, "Niçin kardesini dövüyorsun?" diye sordu.
\par 14 Adam, "Kim seni basimiza yönetici ve yargiç atadi?" diye yanitladi, "Misirli'yi öldürdügün gibi beni de mi öldürmek istiyorsun?" O zaman Musa korkarak, "Bu is ortaya çikmis!" diye düsündü.
\par 15 Firavun olayi duyunca Musa'yi öldürtmek istedi. Ancak Musa ondan kaçip Midyan yöresine gitti. Bir kuyunun basinda otururken
\par 16 Midyanli bir kâhinin* yedi kizi su çekmeye geldi. Babalarinin sürüsünü suvarmak için yalaklari dolduruyorlardi.
\par 17 Ama bazi çobanlar gelip onlari kovmak istedi. Musa kizlarin yardimina kosup hayvanlarini suvardi.
\par 18 Sonra kizlar babalari Reuel'in yanina döndüler. Reuel, "Nasil oldu da bugün böyle tez geldiniz?" diye sordu.
\par 19 Kizlar, "Misirli bir adam bizi çobanlarin elinden kurtardi" diye yanitladilar, "Üstelik bizim için su çekip hayvanlara verdi."
\par 20 Babalari, "Nerede o?" diye sordu, "Niçin adami disarida biraktiniz? Gidin onu yemege çagirin."
\par 21 Musa Reuel'in yaninda kalmayi kabul etti. Reuel de kizi Sippora'yi onunla evlendirdi.
\par 22 Sippora bir erkek çocuk dogurdu. Musa, "Garibim bu yabanci ülkede" diyerek çocuga Gersom adini verdi.
\par 23 Aradan yillar geçti, bu arada Misir Krali öldü. Israilliler hâlâ kölelik altinda inliyor, feryat ediyorlardi. Sonunda yakarislari Tanri'ya eristi.
\par 24 Tanri iniltilerini duydu. Ibrahim, Ishak ve Yakup'la yaptigi antlasmayi animsadi.
\par 25 Israilliler'e bakti ve onlara ilgi gösterdi.

\chapter{3}

\par 1 Musa kayinbabasi Midyanli Kâhin Yitro'nun sürüsünü güdüyordu. Sürüyü çölün batisina sürdü ve Tanri Dagi'na, Horev'e vardi.
\par 2 RAB'bin melegi bir çalidan yükselen alevlerin içinde ona göründü. Musa bakti, çali yaniyor, ama tükenmiyor.
\par 3 "Çok garip" diye düsündü, "Gidip bir bakayim, çali neden tükenmiyor!"
\par 4 RAB Tanri Musa'nin yaklastigini görünce, çalinin içinden, "Musa, Musa!" diye seslendi. Musa, "Buyur!" diye yanitladi.
\par 5 Tanri, "Fazla yaklasma" dedi, "Çariklarini çikar. Çünkü bastigin yer kutsal topraktir.
\par 6 Ben babanin Tanrisi, Ibrahim'in Tanrisi, Ishak'in Tanrisi ve Yakup'un Tanrisi'yim." Musa yüzünü kapadi, çünkü Tanri'ya bakmaya korkuyordu.
\par 7 RAB, "Halkimin Misir'da çektigi sikintiyi yakindan gördüm" dedi, "Angaryacilar yüzünden ettikleri feryadi duydum. Acilarini biliyorum.
\par 8 Bu yüzden onlari Misirlilar'in elinden kurtarmak için geldim. O ülkeden çikarip genis ve verimli topraklara, süt ve bal akan ülkeye, Kenan, Hitit*, Amor, Periz, Hiv ve Yevus topraklarina götürecegim.
\par 9 Israilliler'in feryadi bana eristi. Misirlilar'in onlara yapmakta oldugu baskiyi görüyorum.
\par 10 ªimdi gel, halkim Israil'i Misir'dan çikarmak için seni firavuna göndereyim."
\par 11 Musa, "Ben kimim ki firavuna gidip Israilliler'i Misir'dan çikarayim?" diye karsilik verdi.
\par 12 Tanri, "Kuskun olmasin, ben seninle olacagim" dedi, "Seni benim gönderdigimin kaniti su olacak: Halki Misir'dan çikardigin zaman bu dagda bana tapinacaksiniz."
\par 13 Musa söyle karsilik verdi: "Israilliler'e gidip, 'Beni size atalarinizin Tanrisi gönderdi' dersem, 'Adi nedir?' diye sorabilirler. O zaman ne diyeyim?"
\par 14 Tanri, "Ben Ben'im" dedi, "Israilliler'e de ki, 'Beni size Ben Ben'im diyen gönderdi.'
\par 15 "Israilliler'e de ki, 'Beni size atalarinizin Tanrisi, Ibrahim'in Tanrisi, Ishak'in Tanrisi ve Yakup'un Tanrisi RAB gönderdi.' Sonsuza dek adim bu olacak. Kusaklar boyunca böyle anilacagim.
\par 16 Git, Israil ileri gelenlerini topla, onlara söyle de: 'Atalariniz Ibrahim'in, Ishak'in, Yakup'un Tanrisi RAB bana görünerek sunlari söyledi: Sizinle ve Misir'da size yapilanlarla yakindan ilgileniyorum.
\par 17 Söz verdim, sizi Misir'da çektiginiz sikintidan kurtaracagim; Kenan, Hitit, Amor, Periz, Hiv ve Yevus topraklarina, süt ve bal akan ülkeye götürecegim.'
\par 18 "Israil ileri gelenleri seni dinleyecekler. Sonra birlikte Misir Krali'na gidip, 'Ibraniler'in Tanrisi RAB bizimle görüstü' diyeceksiniz, 'ªimdi izin ver, Tanrimiz RAB'be kurban kesmek için çölde üç gün yol alalim.'
\par 19 Ama biliyorum, güçlü bir el zorlamadikça Misir Krali gitmenize izin vermeyecek.
\par 20 Elimi uzatacak ve aralarinda sasilasi isler yaparak Misir'i cezalandiracagim. O zaman sizi saliverecek.
\par 21 "Halkimin Misirlilar'in gözünde lütuf bulmasini saglayacagim. Gittiginizde eli bos gitmeyeceksiniz.
\par 22 Her kadin Misirli komsusundan ya da konugundan altin ve gümüs takilar, giysiler isteyecek. Ogullarinizi, kizlarinizi bunlarla süsleyeceksiniz. Misirlilar'i soyacaksiniz."

\chapter{4}

\par 1 Musa, "Ya bana inanmazlarsa?" dedi, "Sözümü dinlemez, 'RAB sana görünmedi' derlerse, ne olacak?"
\par 2 RAB, "Elinde ne var?" diye sordu. Musa, "Degnek" diye yanitladi.
\par 3 RAB, "Onu yere at" dedi. Musa degnegini yere atinca, degnek yilan oldu. Musa yilandan kaçti.
\par 4 RAB, "Elini uzat, kuyrugundan tut" dedi. Musa elini uzatip kuyrugunu tutunca yilan yine degnek oldu.
\par 5 RAB, "Bunu yap ki, atalari Ibrahim'in, Ishak'in, Yakup'un Tanrisi RAB'bin sana göründügüne inansinlar" dedi.
\par 6 Sonra, "Elini koynuna koy" dedi. Musa elini koynuna koydu. Çikardigi zaman eli bir deri hastaligina yakalanmis, kar gibi bembeyaz olmustu.
\par 7 RAB, "Elini yine koynuna koy" dedi. Musa elini yine koynuna koydu. Çikardigi zaman eli eski haline dönmüstü.
\par 8 RAB, "Eger sana inanmaz, ilk belirtiyi önemsemezlerse, ikinci belirtiye inanabilirler" dedi,
\par 9 "Bu iki belirtiye de inanmaz, sözünü dinlemezlerse, Nil'den biraz su alip kuru topraga dök. Irmaktan aldigin su toprakta kana dönecek."
\par 10 Musa RAB'be, "Aman, ya Rab!" dedi, "Ben kulun ne geçmiste, ne de benimle konusmaya basladigindan bu yana iyi bir konusmaci oldum. Çünkü dili agir, tutuk biriyim."
\par 11 RAB, "Kim agiz verdi insana?" dedi, "Insani sagir, dilsiz, görür ya da görmez yapan kim? Ben degil miyim?
\par 12 ªimdi git! Ben konusmana yardimci olacagim. Ne söylemen gerektigini sana ögretecegim."
\par 13 Musa, "Aman, ya Rab!" dedi, "Ne olur, benim yerime baskasini gönder."
\par 14 RAB Musa'ya öfkelendi ve, "Agabeyin Levili Harun var ya!" dedi, "Bilirim, o iyi konusur. Hem su anda seni karsilamaya geliyor. Seni görünce sevinecek.
\par 15 Onunla konus, ne söylemesi gerektigini anlat. Ikinizin konusmasina da yardimci olacak, ne yapacaginizi size ögretecegim.
\par 16 O sana sözcülük edecek, senin yerine halkla konusacak. Sen de onun için Tanri gibi olacaksin.
\par 17 Bu degnegi eline al, çünkü belirtileri onunla gerçeklestireceksin."
\par 18 Musa kayinbabasi Yitro'nun yanina döndü. Ona, "Izin ver, Misir'daki soydaslarimin yanina döneyim" dedi, "Bakayim, hâlâ yasiyorlar mi?" Yitro, "Esenlikle git" diye karsilik verdi.
\par 19 RAB Midyan'da Musa'ya, "Misir'a dön, çünkü canini almak isteyenlerin hepsi öldü" demisti.
\par 20 Böylece Musa karisini, ogullarini esege bindirdi; Tanri'nin buyurdugu degnegi de eline alip Misir'a dogru yola çikti.
\par 21 RAB Musa'ya, "Misir'a döndügünde, sana verdigim güçle bütün sasilasi isleri firavunun önünde yapmaya bak" dedi, "Ama ben onu inatçi yapacagim. Halki salivermeyecek.
\par 22 Sonra firavuna de ki, 'RAB söyle diyor: Israil benim ilk oglumdur.
\par 23 Sana, birak oglum gitsin, bana tapsin, dedim. Ama sen onu salivermeyi reddettin. Bu yüzden senin ilk oglunu öldürecegim.'"
\par 24 RAB yolda, bir konaklama yerinde Musa'yla karsilasti, onu öldürmek istedi.
\par 25 O anda Sippora keskin bir tas alip oglunu sünnet etti, derisini Musa'nin ayaklarina dokundurdu. "Gerçekten sen bana kanli güveysin" dedi.
\par 26 Böylece RAB Musa'yi esirgedi. Sippora Musa'ya sünnetten ötürü "Kanli güveysin" demisti.
\par 27 RAB Harun'a, "Çöle, Musa'yi karsilamaya git" dedi. Harun gitti, onu Tanri Dagi'nda karsilayip öptü.
\par 28 Musa duyurmasi için RAB'bin kendisine söyledigi bütün sözleri ve gerçeklestirmesini buyurdugu bütün belirtileri Harun'a anlatti.
\par 29 Musa'yla Harun varip Israil'in bütün ileri gelenlerini topladilar.
\par 30 Harun RAB'bin Musa'ya söylemis oldugu her seyi onlara anlatti. Musa da halkin önünde belirtileri gerçeklestirdi.
\par 31 Halk inandi; RAB'bin kendileriyle ilgilendigini, çektikleri sikintiyi görmüs oldugunu duyunca, egilip tapindilar.

\chapter{5}

\par 1 Sonra Musa'yla Harun firavuna gidip söyle dediler: "Israil'in Tanrisi RAB diyor ki, 'Halkimi birak gitsin, çölde bana bayram yapsin.'"
\par 2 Firavun, "RAB kim oluyor ki, O'nun sözünü dinleyip Israil halkini salivereyim?" dedi. "RAB'bi tanimiyorum. Israilliler'in gitmesine izin vermeyecegim."
\par 3 Musa'yla Harun, "Ibraniler'in Tanrisi bizimle görüstü" diye yanitladilar, "Izin ver, Tanrimiz RAB'be kurban kesmek için çölde üç gün yol alalim. Yoksa bizi salgin hastalik ya da kiliçla cezalandirabilir."
\par 4 Misir Firavunu, "Ey Musa ve Harun, niçin halki isinden alikoyuyorsunuz? Siz de isinizin basina dönün" dedi,
\par 5 "Bakin, halkiniz Misirlilar'dan daha kalabalik, oysa siz onlarin isini engellemeye çalisiyorsunuz."
\par 6 Firavun o gün angaryacilara ve halkin basindaki görevlilere buyruk verdi:
\par 7 "Kerpiç yapmak için artik halka saman vermeyeceksiniz. Gitsinler, kendi samanlarini kendileri toplasinlar.
\par 8 Önceki gibi ayni sayida kerpiç yapmalarini isteyin, kerpiç sayisini azaltmayin. Çünkü tembel insanlardir; bu yüzden, 'Gidelim, Tanrimiz'a kurban keselim' diye bagrisiyorlar.
\par 9 Islerini agirlastirin ki, mesgul olsunlar, yalan sözlere kulak asmasinlar."
\par 10 Angaryacilarla görevliler gidip Israilliler'e söyle dediler: "Firavun diyor ki, 'Artik size saman vermeyecegim.
\par 11 Gidin, nerede bulursaniz oradan kendinize saman alin. Ancak isiniz hiç hafifletilmeyecek.'"
\par 12 Böylece halk saman yerine aniz toplamak üzere bütün Misir'a dagildi.
\par 13 Angaryacilar, "Saman verildigi günlerdeki gibi gündelik görevlerinizi eksiksiz yerine getirin" diyerek onlara baski yapiyordu.
\par 14 Firavunun angaryacilarinin atadigi Israilli görevliler, "Niçin dün ve bugün daha önceki gibi gereken sayida kerpiç yaptirmadiniz?" diyerek dövüldüler.
\par 15 Bunun üzerine Israilli görevliler firavunun yanina varip yakindilar: "Neden kullarina böyle davraniyorsun?
\par 16 Neden bize saman verilmedigi halde, 'Kerpiç yapin!' deniyor? Iste kullarin dövülüyor, oysa suçlu senin kendi halkindir."
\par 17 Firavun, "Tembelsiniz siz, tembel!" diye karsilik verdi, "Bu yüzden 'Gidip RAB'be kurban keselim' diyorsunuz.
\par 18 Haydi, isinizin basina dönün. Size saman verilmeyecek; yine de ayni sayida kerpiç üreteceksiniz."
\par 19 Kendilerine, "Her gün üretmeniz gereken kerpiç sayisini azaltmayacaksiniz" dendiginde Israilli görevliler zor durumda olduklarini anladilar.
\par 20 Firavunun yanindan ayrilinca, kendilerini bekleyen Musa'yla Harun'a çikistilar.
\par 21 "RAB yaptiginizi görsün, cezanizi versin!" dediler, "Bizi firavunla görevlilerinin gözünde rezil ettiniz. Bizi öldürmeleri için ellerine bir kiliç verdiniz."
\par 22 Musa RAB'be döndü ve, "Ya Rab, niçin bu halka kötü davrandin?" dedi, "Beni bunun için mi gönderdin?
\par 23 Senin adina firavunla konusmaya gittim gideli firavun bu halka kötü davraniyor. Sen de kendi halkini kurtarmak için hiçbir sey yapmadin."

\chapter{6}

\par 1 RAB Musa'ya, "Firavuna ne yapacagimi simdi göreceksin" dedi, "Güçlü elimden ötürü Israil halkini saliverecek, güçlü elimden ötürü onlari ülkesinden kovacak."
\par 2 Tanri ayrica Musa'ya, "Ben RAB'bim" dedi,
\par 3 "Ibrahim'e, Ishak'a ve Yakup'a Her ªeye Gücü Yeten Tanri olarak göründüm, ama onlara kendimi RAB adiyla tanitmadim.
\par 4 Yabanci olarak yasadiklari Kenan ülkesini kendilerine vermek üzere onlarla antlasma yaptim.
\par 5 Misirlilar'in kölelestirdigi Israilliler'in iniltilerini duydum ve antlasmami hep andim.
\par 6 "Onun için Israilliler'e de ki, 'Ben RAB'bim. Sizi Misirlilar'in boyundurugundan çikaracak, onlarin kölesi olmaktan kurtaracagim. Onlari agir biçimde yargilayacak ve kudretli elimle sizi özgür kilacagim.
\par 7 Sizi kendi halkim yapacak ve Tanriniz olacagim. O zaman sizi Misirlilar'in boyundurugundan çikaran Tanriniz RAB'bin ben oldugumu bileceksiniz.
\par 8 Sizi Ibrahim'e, Ishak'a ve Yakup'a verecegime ant içtigim topraklara götürecegim. Orayi size mülk olarak verecegim. Ben RAB'bim.'"
\par 9 Musa bunlari Israilliler'e anlatti, ama umutlari kirildigi ve agir baski altinda olduklari için onu dinlemediler.
\par 10 RAB Musa'ya, "Git, Misir Firavunu'na Israilliler'i ülkesinden salivermesini söyle" dedi.
\par 11 (#6:10)
\par 12 Ama Musa, "Israilliler beni dinlemedikten sonra, firavun nasil dinler?" diye karsilik verdi, "Zaten iyi konusan biri degilim."
\par 13 RAB Musa ve Harun'la Israilliler ve Misir Firavunu hakkinda konustu. Israilliler'i Misir'dan çikarmalarini buyurdu.
\par 14 Israilliler'in aile önderleri sunlardir: Yakup'un ilk oglu Ruben'in ogullari: Hanok, Pallu, Hesron, Karmi. Ruben'in boylari bunlardir.
\par 15 ªimon'un ogullari: Yemuel, Yamin, Ohat, Yakin, Sohar ve Kenanli bir kadinin oglu ªaul. ªimon'un boylari bunlardir.
\par 16 Kayitlarina göre Leviogullari'nin adlari sunlardir: Gerson, Kehat, Merari. Levi 137 yil yasadi.
\par 17 Gerson'un ogullari boylarina göre sunlardir: Livni, ªimi.
\par 18 Kehat'in ogullari: Amram, Yishar, Hevron, Uzziel. Kehat 133 yil yasadi.
\par 19 Merari'nin ogullari: Mahli, Musi. Kayitlarina göre Levi boylari bunlardir.
\par 20 Amram halasi Yokevet'le evlendi. Yokevet ona Harun'la Musa'yi dogurdu. Amram 137 yil yasadi.
\par 21 Yishar'in ogullari: Korah, Nefek, Zikri.
\par 22 Uzziel'in ogullari: Misael, Elsafan, Sitri.
\par 23 Harun Nahson'un kizkardesi ve Amminadav'in kizi Eliseva'yla evlendi. Eliseva ona Nadav, Avihu, Elazar ve Itamar'i dogurdu.
\par 24 Korah'in ogullari: Assir, Elkana, Aviasaf. Korahlilar'in boylari bunlardir.
\par 25 Harun'un oglu Elazar Putiel'in kizlarindan biriyle evlendi. Karisi ona Pinehas'i dogurdu. Boylarina göre Levili aile önderleri bunlardir.
\par 26 RAB'bin, "Israilliler'i ordular halinde Misir'dan çikarin" dedigi Harun ve Musa bunlardir.
\par 27 Israilliler'i Misir'dan çikarmak için Misir Firavunu ile konusanlar da Musa'yla Harun'dur.
\par 28 RAB Misir'da Musa'yla konustugunda, ona, "Ben RAB'bim" dedi, "Sana söyledigim her seyi Misir Firavunu'na ilet."
\par 29 (#6:28)
\par 30 Musa RAB'bin huzurunda, "Ben iyi konusan biri degilim" diye karsilik verdi, "Firavun beni nasil dinler?"

\chapter{7}

\par 1 RAB, "Bak, seni firavuna karsi Tanri gibi yaptim" dedi, "Agabeyin Harun senin peygamberin olacak.
\par 2 Sana buyurdugum her seyi agabeyine anlat. O da firavuna Israilliler'i ülkesinden salivermesini söylesin.
\par 3 Ben firavunu inatçi yapacagim ki, belirtilerimi ve sasilasi islerimi Misir'da arttirabileyim.
\par 4 Ama firavun sizi dinlemeyecek. O zaman elimi Misir'in üzerine koyacagim ve onlari agir biçimde cezalandirarak halkim Israil'i ordular halinde Misir'dan çikaracagim.
\par 5 Misir'a karsi elimi kaldirdigim ve Israilliler'i aralarindan çikardigim zaman Misirlilar benim RAB oldugumu anlayacak."
\par 6 Musa'yla Harun RAB'bin buyurdugu gibi yaptilar.
\par 7 Firavunla konustuklarinda Musa seksen, Harun seksen üç yasindaydi.
\par 8 RAB Musa'yla Harun'a söyle dedi:
\par 9 "Firavun size, 'Bir mucize yapin' dediginde, söyle Harun'a, degnegini alip firavunun önüne atsin. Degnek yilan olacak."
\par 10 Böylece Musa'yla Harun firavunun yanina gittiler ve RAB'bin buyurdugu gibi yaptilar. Harun degnegini firavunla görevlilerinin önüne atti. Degnek yilan oluverdi.
\par 11 Bunun üzerine firavun kendi bilgelerini, büyücülerini çagirdi. Misirli büyücüler de büyüleriyle ayni seyi yaptilar.
\par 12 Her biri degnegini atti, degnekler yilan oldu. Ancak Harun'un degnegi onlarin degneklerini yuttu.
\par 13 Yine de, RAB'bin söyledigi gibi firavun inat etti ve Musa'yla Harun'u dinlemedi.
\par 14 RAB Musa'ya, "Firavun inat ediyor, halki salivermeyi reddediyor" dedi,
\par 15 "Sabah git, firavun Nil'e inerken onu karsilamak için irmak kiyisinda bekle. Yilana dönüsen degnegi eline al
\par 16 ve ona de ki, 'Halkimi saliver, çölde bana tapsinlar, demem için Ibraniler'in Tanrisi RAB beni sana gönderdi. Ama sen su ana kadar kulak asmadin.
\par 17 Benim RAB oldugumu sundan anla, diyor RAB. Iste, elimdeki degnegi irmagin sularina vuracagim, sular kana dönecek.
\par 18 Irmaktaki baliklar ölecek, irmak les gibi kokacak, Misirlilar artik irmagin suyunu içemeyecekler.'"
\par 19 Sonra RAB Musa'ya söyle buyurdu: "Harun'a de ki, 'Degnegini al ve elini Misir'in sulari üzerine -irmaklari, kanallari, havuzlari, bütün su birikintileri üzerine- uzat, hepsi kana dönsün. Bütün Misir'da tahta ve tas kaplardaki sular bile kana dönecek.'"
\par 20 Musa'yla Harun RAB'bin buyurdugu gibi yaptilar. Harun firavunla görevlilerinin gözü önünde degnegini kaldirip irmagin sularina vurdu. Bütün sular kana dönüstü.
\par 21 Irmaktaki baliklar öldü, irmak kokmaya basladi. Misirlilar irmagin suyunu içemez oldular. Misir'in her yerinde kan vardi.
\par 22 Misirli büyücüler de kendi büyüleriyle ayni seyi yaptilar. RAB'bin söyledigi gibi firavun inat etti ve Musa'yla Harun'u dinlemedi.
\par 23 Olanlara aldirmadan sarayina döndü.
\par 24 Misirlilar içecek su bulmak için irmak kiyisini kazmaya koyuldular. Çünkü irmagin suyunu içemiyorlardi.
\par 25 RAB'bin irmagi vurmasinin üzerinden yedi gün geçti.

\chapter{8}

\par 1 RAB Musa'ya söyle dedi: "Firavunun yanina git ve ona de ki, 'RAB söyle diyor: Halkimi saliver, bana tapsinlar.
\par 2 Eger halkimi salivermeyi reddedersen, bütün ülkeni kurbagalarla cezalandiracagim.
\par 3 Irmak kurbagalarla dolup tasacak. Kurbagalar çikip sarayina, yatak odana, yatagina, görevlilerinin ve halkinin evlerine, firinlarina, hamur teknelerine girecekler.
\par 4 Senin, halkinin, bütün görevlilerinin üstüne siçrayacaklar.'
\par 5 "Harun'a de ki, 'Elindeki degnegi irmaklarin, kanallarin, havuzlarin üzerine uzatip kurbagalari çikart; Misir'i kurbagalar kaplasin.'"
\par 6 Böylece Harun elini Misir'in sulari üzerine uzatti; kurbagalar çikip Misir'i kapladi.
\par 7 Ancak büyücüler de kendi büyüleriyle ayni seyi yaptilar ve ülkeye kurbagalari saldilar.
\par 8 Firavun Musa'yla Harun'u çagirtip, "RAB'be dua edin, benim ve halkimin üzerinden kurbagalari uzaklastirsin" dedi, "O zaman halkinizi RAB'be kurban kessinler diye saliverecegim."
\par 9 Musa, "Sen karar ver" diye karsilik verdi, "Bunu sana birakiyorum. Kurbagalar senden ve evlerinden uzak dursun, yalniz irmakta kalsinlar diye senin, görevlilerin ve halkin için ne zaman dua edeyim?"
\par 10 Firavun, "Yarin" dedi. Musa, "Peki, dedigin gibi olsun" diye karsilik verdi, "Böylece bileceksin ki, Tanrimiz RAB gibisi yoktur.
\par 11 Kurbagalar senden, evlerinden, görevlilerinden, halkindan uzaklasacak, yalniz irmakta kalacaklar."
\par 12 Musa'yla Harun firavunun yanindan ayrildilar. Musa RAB'bin firavunun basina getirdigi kurbaga belasi için RAB'be feryat etti.
\par 13 RAB Musa'nin istegini yerine getirdi. Kurbagalar evlerde, avlularda, tarlalarda öldüler.
\par 14 Kurbagalari yigin yigin topladilar. Ülke kokudan geçilmez oldu.
\par 15 Ancak firavun ülkenin rahatladigini görünce, RAB'bin söyledigi gibi inatçilik etti ve Musa'yla Harun'u dinlemedi.
\par 16 RAB Musa'ya söyle dedi: "Harun'a de ki, 'Degnegini uzatip yere vur, yerdeki toz sivrisinege dönüssün, bütün Misir'i kaplasin.'"
\par 17 Öyle yaptilar. Harun elindeki degnegi uzatip yere vurunca, insanlarla hayvanlarin üzerine sivrisinekler üsüstü. Misir'da yerin bütün tozu sivrisinege dönüstü.
\par 18 Büyücüler de kendi büyüleriyle tozu sivrisinege dönüstürmek istedilerse de basaramadilar. Insanlarin, hayvanlarin üzerini sivrisinek kapladi.
\par 19 Büyücüler firavuna, "Bu iste Tanri'nin parmagi var" dediler. Ne var ki, RAB'bin söyledigi gibi firavun inat etti, Musa'yla Harun'u dinlemedi.
\par 20 RAB Musa'ya söyle dedi: "Sabah erkenden kalk, firavun irmaga inerken onu karsila ve söyle de: 'RAB diyor ki, halkimi saliver, bana tapsinlar.
\par 21 Halkimi salivermezsen senin, görevlilerinin, halkinin, evlerinin üzerine atsinegi yagdiracagim. Misirlilar'in evleri ve üzerinde yasadiklari topraklar atsinekleriyle dolup tasacak.
\par 22 "'Ama o gün halkimin yasadigi Gosen bölgesinde farkli davranacagim. Orada atsinegi olmayacak. Böylece bileceksin ki, bu ülkede RAB benim.
\par 23 Kendi halkimla senin halkin arasina fark koyacagim. Yarin bu belirti gerçeklesecek.'"
\par 24 RAB dedigini yapti. Firavunun sarayina, görevlilerinin evlerine sürü sürü atsinegi gönderdi. Misir atsinegi yüzünden bastan sona harap oldu.
\par 25 Firavun Musa'yla Harun'u çagirtip, "Gidin, bu ülkede Tanriniz'a kurban kesin" dedi.
\par 26 Musa, "Bu dogru olmaz" diye karsilik verdi, "Çünkü Misirlilar Tanrimiz RAB'be kurban kesmeyi igrenç sayiyorlar. Igrenç saydiklari bu seyi gözlerinin önünde yaparsak bizi taslamazlar mi?
\par 27 Tanrimiz RAB'be kurban kesmek için, bize buyurdugu gibi üç gün çölde yol almaliyiz."
\par 28 Firavun, "Çölde Tanriniz RAB'be kurban kesmeniz için sizi saliveriyorum" dedi, "Yalniz çok uzaga gitmeyeceksiniz. ªimdi benim için dua edin."
\par 29 Musa, "Yarin atsineklerini firavunun, görevlilerinin, halkinin üzerinden uzaklastirsin diye, yanindan ayrilir ayrilmaz RAB'be dua edecegim" dedi, "Yalniz firavun RAB'be kurban kesmek için halkin gitmesini önleyerek bizi yine aldatmamali."
\par 30 Musa firavunun yanindan çikip RAB'be dua etti.
\par 31 RAB Musa'nin istegini yerine getirdi; firavunun, görevlilerinin, halkinin üzerinden atsineklerini uzaklastirdi. Tek sinek kalmadi.
\par 32 Öyleyken, firavun bir kez daha inatçilik etti ve halki salivermedi.

\chapter{9}

\par 1 RAB Musa'ya söyle dedi: "Firavunun yanina git ve ona de ki, 'Ibraniler'in Tanrisi RAB söyle diyor: Halkimi saliver, bana tapsinlar.
\par 2 Salivermeyi reddeder, onlari tutmakta diretirsen,
\par 3 RAB'bin eli kirlardaki hayvanlarinizi -atlari, esekleri, develeri, sigirlari, davarlari- büyük kirima ugratarak sizi cezalandiracak.
\par 4 RAB Israilliler'le Misirlilar'in hayvanlarina farkli davranacak. Israilliler'in hayvanlarindan hiçbiri ölmeyecek.'"
\par 5 RAB zamani da belirleyerek, "Yarin ülkede bunu yapacagim" dedi.
\par 6 Ertesi gün RAB dedigini yapti: Misirlilar'in hayvanlari büyük çapta öldü. Ama Israilliler'in hayvanlarindan hiçbiri ölmedi.
\par 7 Firavun adam gönderdi, Israilliler'in bir tek hayvaninin bile ölmedigini ögrendi. Öyleyken, inat etti ve halki salivermedi.
\par 8 RAB Musa'yla Harun'a, "Yaniniza iki avuç dolusu ocak kurumu alin" dedi, "Musa kurumu firavunun önünde göge dogru savursun.
\par 9 Kurum bütün Misir'in üzerinde ince bir toza dönüsecek; ülkenin her yanindaki insanlarin, hayvanlarin bedenlerinde irinli çibanlar çikacak."
\par 10 Böylece Musa'yla Harun ocak kurumu alip firavunun önünde durdular. Musa kurumu göge dogru savurdu. Insanlarda ve hayvanlarda irinli çibanlar çikti.
\par 11 Büyücüler çibandan ötürü Musa'nin karsisinda duramaz oldular. Çünkü bütün Misirlilar'da oldugu gibi onlarda da çibanlar çikmisti.
\par 12 RAB firavunu inatçi yapti, RAB'bin Musa'ya söyledigi gibi, firavun Musa'yla Harun'u dinlemedi.
\par 13 RAB Musa'ya söyle dedi: "Sabah erkenden kalkip firavunun huzuruna çik, de ki, 'Ibraniler'in Tanrisi RAB söyle diyor: Halkimi saliver, bana tapsinlar.
\par 14 Yoksa bu kez senin, görevlilerinin, halkinin üzerine bütün belalarimi yagdiracagim. Öyle ki, bu dünyada benim gibisi olmadigini ögrenesin.
\par 15 Çünkü elimi kaldirip seni ve halkini salgin hastalikla vurmus olsaydim, yeryüzünden silinmis olurdun.
\par 16 Gücümü sana göstermek, adimi bütün dünyaya tanitmak için seni ayakta tuttum.
\par 17 Hâlâ halkimi salivermiyor, onlara üstünlük tasliyorsun.
\par 18 Bu yüzden, yarin bu saatlerde Misir'a tarihinde görülmemis agir bir dolu yagdiracagim.
\par 19 ªimdi buyruk ver, hayvanlarin ve kirda neyin varsa hepsi siginaklara konsun. Dolu yaginca, eve getirilmeyen, kirda kalan bütün insanlarla hayvanlar ölecek.'"
\par 20 Firavunun görevlileri arasinda RAB'bin uyarisindan korkanlar köleleriyle hayvanlarini çabucak evlerine getirdiler.
\par 21 RAB'bin uyarisini önemsemeyenler ise köleleriyle hayvanlarini tarlada birakti.
\par 22 RAB Musa'ya, "Elini göge dogru uzat" dedi, "Misir'in her yerine, insanlarin, hayvanlarin, kirdaki bütün bitkilerin üzerine dolu yagsin."
\par 23 Musa degnegini göge dogru uzatinca RAB gök gürlemeleri ve dolu gönderdi. Yildirim düstü. RAB Misir'a dolu yagdirdi.
\par 24 ªiddetli dolu yagiyor, sürekli simsek çakiyordu. Misir Misir olali böylesi bir dolu görmemisti.
\par 25 Dolu Misir'da insandan hayvana dek kirdaki her seyi, bütün bitkileri mahvetti, bütün agaçlari kirdi.
\par 26 Yalniz Israilliler'in yasadigi Gosen bölgesine dolu düsmedi.
\par 27 Firavun Musa'yla Harun'u çagirtarak, "Bu kez günah isledim" dedi, "RAB hakli, ben ve halkim haksiziz.
\par 28 RAB'be dua edin, yeter bu gök gürlemeleri ve dolu. Sizi saliverecegim, artik burada kalmayacaksiniz."
\par 29 Musa, "Kentten çikinca, ellerimi RAB'be uzatacagim" dedi, "Gök gürlemeleri duracak, artik dolu yagmayacak. Böylece dünyanin RAB'be ait oldugunu bileceksin.
\par 30 Ama biliyorum, sen ve görevlilerin RAB Tanri'dan hâlâ korkmuyorsunuz."
\par 31 Keten ve arpa mahvolmustu; çünkü arpa basak vermis, keten çiçek açmisti.
\par 32 Ama bugday ve kizil bugday henüz bitmedigi için zarar görmemisti.
\par 33 Musa firavunun yanindan ayrilip kentten çikti. Ellerini RAB'be uzatti. Gök gürlemesi ve dolu durdu, yagmur dindi.
\par 34 Firavun yagmurun, dolunun, gök gürlemesinin kesildigini görünce, yine günah isledi. Hem kendisi, hem görevlileri inat ettiler.
\par 35 RAB'bin Musa araciligiyla söyledigi gibi, firavun inat ederek Israilliler'i salivermedi.

\chapter{10}

\par 1 RAB Musa'ya, "Firavunun yanina git" dedi, "Belirtilerimi aralarinda göstermek için firavunla görevlilerini inatçi yaptim.
\par 2 Misir'la nasil alay ettigimi, aralarinda gösterdigim belirtileri sen de çocuklarina, torunlarina anlat ki, benim RAB oldugumu bilesiniz."
\par 3 Musa'yla Harun firavunun yanina varip söyle dediler: "Ibraniler'in Tanrisi RAB diyor ki, 'Ne zamana dek alçakgönüllü olmayi reddedeceksin? Halkimi saliver, bana tapsinlar.
\par 4 Halkimi salivermeyi reddedersen, yarin ülkene çekirgeler gönderecegim.
\par 5 Yeryüzünü öylesine kaplayacaklar ki, toprak görünmez olacak. Doludan kurtulan ürünlerinizi, kirda biten bütün agaçlarinizi yiyecekler.
\par 6 Evlerine, bütün görevlilerinin, bütün Misirlilar'in evlerine çekirge dolacak. Ne babalarin, ne atalarin ömürlerince böylesini görmediler.'" Sonra Musa dönüp firavunun yanindan ayrildi.
\par 7 Görevlileri firavuna, "Ne zamana dek bu adam bize tuzak kuracak?" dediler, "Birak gitsinler, Tanrilari RAB'be tapsinlar. Misir harap oldu, hâlâ anlamiyor musun?"
\par 8 Böylece, Musa'yla Harun'u firavunun yanina geri getirdiler. Firavun, "Gidin, Tanriniz RAB'be tapin" dedi, "Ama kimler gidecek?"
\par 9 Musa, "Genç, yasli hep birlikte gidecegiz" dedi, "Ogullarimizi, kizlarimizi, davarlarimizi, sigirlarimizi yanimiza alacagiz. Çünkü RAB'be bayram yapmaliyiz."
\par 10 Firavun, "Alin çoluk çocugunuzu, gidin gidebilirseniz, RAB yardimciniz olsun!" dedi, "Bakin, kötü niyetiniz ne kadar açik.
\par 11 Olmaz. Yalniz erkekler gidip RAB'be tapsin. Zaten istediginiz de bu." Sonra Musa'yla Harun firavunun yanindan kovuldular.
\par 12 RAB Musa'ya, "Elini Misir'in üzerine uzat" dedi, "Çekirge yagsin; ülkenin bütün bitkilerini, doludan kurtulan her seyi yesinler."
\par 13 Musa degnegini Misir'in üzerine uzatti. Bütün o gün ve gece RAB ülkede dogu rüzgari estirdi. Sabah olunca da dogu rüzgari çekirgeleri getirdi.
\par 14 Misir'in üzerinde uçusan çekirgeler ülkeyi boydan boya kapladi. Öyle çoktular ki, böylesi hiçbir zaman görülmedi, kusaklar boyu da görülmeyecek.
\par 15 Topragin üzerini öyle kapladilar ki, ülke kapkara kesildi. Bütün bitkileri, dolunun zarar vermedigi agaçlarda kalan meyvelerin hepsini yediler. Misir'in hiçbir yerinde, ne agaçlarda, ne de kirdaki bitkilerde yesillik kalmadi.
\par 16 Firavun acele Musa'yla Harun'u çagirtti. "Tanriniz RAB'be ve size karsi günah isledim" dedi,
\par 17 "Lütfen bir kez daha günahimi bagislayin ve Tanriniz RAB'be dua edin; bu ölümcül belayi üzerimden uzaklastirsin."
\par 18 Musa firavunun yanindan çikip RAB'be dua etti.
\par 19 RAB rüzgari çok siddetli bati rüzgarina döndürdü. Rüzgar çekirgeleri sürükleyip Kizildeniz'e* döktü. Misir'da tek çekirge kalmadi.
\par 20 Ama RAB firavunu inatçi yapti. Firavun Israilliler'i salivermedi.
\par 21 RAB Musa'ya, "Elini göge dogru uzat" dedi, "Misir'i hissedilebilir bir karanlik kaplasin."
\par 22 Musa elini göge dogru uzatti, Misir üç gün koyu karanliga gömüldü.
\par 23 Üç gün boyunca kimse kimseyi göremez, yerinden kimildayamaz oldu. Yalniz Israilliler'in yasadigi yerler aydinlikti.
\par 24 Firavun Musa'yi çagirtti. "Gidin, RAB'be tapin" dedi, "Yalniz davarlarinizla sigirlariniz alikonacak. Çoluk çocugunuz sizinle birlikte gidebilir."
\par 25 Musa, "Ama Tanrimiz RAB'be kurban kesmemiz için bize kurbanlik ve yakmalik sunular* da vermelisin" diye karsilik verdi,
\par 26 "Hayvanlarimizi da yanimiza almaliyiz. Bir tirnak bile kalmamali burada. Çünkü Tanrimiz RAB'be tapmak için bazi hayvanlari kullanacagiz. Oraya varmadikça hangi hayvanlari RAB'be sunacagimizi bilemeyiz."
\par 27 Ancak RAB firavunu inatçi yapti, firavun Israilliler'i salivermeye yanasmadi.
\par 28 Musa'ya, "Git basimdan" dedi, "Sakin bir daha karsima çikma. Yüzümü gördügün gün ölürsün."
\par 29 Musa, "Dedigin gibi olsun" diye karsilik verdi, "Bir daha yüzünü görmeyecegim."

\chapter{11}

\par 1 RAB Musa'ya, "Firavunun ve Misir'in basina bir bela daha getirecegim" dedi, "O zaman gitmenize izin verecek, sizi buradan adeta kovacak.
\par 2 Halkina söyle, kadin erkek herkes komsusundan altin, gümüs esya istesin."
\par 3 RAB Israil halkinin Misirlilar'in gözünde lütuf bulmasini sagladi. Musa da Misir'da, firavunun görevlilerinin ve halkin gözünde çok büyüdü.
\par 4 Musa firavuna söyle dedi: "RAB diyor ki, 'Gece yarisi Misir'i boydan boya geçecegim.
\par 5 Tahtinda oturan firavunun ilk çocugundan, degirmendeki kadin kölenin ilk çocuguna kadar, hayvanlar dahil Misir'daki bütün ilk doganlar ölecek.
\par 6 Bütün Misir'da benzeri ne görülmüs, ne de görülecek büyük bir feryat kopacak.
\par 7 Israilliler'e ya da hayvanlarina bir köpek bile havlamayacak.' O zaman RAB'bin Israilliler'le Misirlilar'a nasil farkli davrandigini anlayacaksiniz.
\par 8 Bu görevlilerinin hepsi gelip önümde egilecek, 'Sen ve seni izleyenler, gidin!' diyecekler. Ondan sonra gidecegim." Musa firavunun yanindan büyük bir öfkeyle ayrildi.
\par 9 RAB Musa'ya, "Misir'da sasilasi islerim çogalsin diye firavun sizi dinlemeyecek" demisti.
\par 10 Musa'yla Harun firavunun önünde bütün bu sasilasi isleri yaptilar. Ama RAB firavunu inatçi yapti. Firavun Israilliler'i ülkesinden salivermedi.

\chapter{12}

\par 1 RAB Misir'da Musa'yla Harun'a, "Bu ay sizin için ilk ay*, yilin ilk ayi olacak" dedi,
\par 2 (#12:1)
\par 3 "Bütün Israil topluluguna bildirin: Bu ayin onunda herkes ailesine göre kendi ev halkina birer kuzu alacak.
\par 4 Eger bir kuzu*ff* bir aileye çok geliyorsa, aile bireylerinin sayisi ve herkesin yiyecegi miktar hesaplanacak ve aile kuzuyu*ff* en yakin komsusuyla paylasabilecek.
\par 5 Koyun ya da keçilerden seçeceginiz hayvan kusursuz, erkek ve bir yasinda olmali.
\par 6 Ayin on dördüne kadar ona bakacaksiniz. O aksamüstü bütün Israil toplulugu hayvanlari bogazlayacak.
\par 7 Hayvanin kanini alip, etin yenecegi evin yan ve üst kapi sövelerine sürecekler.
\par 8 O gece ateste kizartilmis et mayasiz ekmek ve aci otlarla yenmelidir.
\par 9 Eti çig veya haslanmis olarak degil, basi, bacaklari, bagirsaklari ve iskembesiyle birlikte kizartarak yiyeceksiniz.
\par 10 Sabaha kadar bitirmelisiniz. Artakalan olursa, sabah ateste yakacaksiniz.
\par 11 Eti söyle yemelisiniz: Beliniz kusanmis, çariklariniz ayaginizda, degneginiz elinizde olmali. Eti çabuk yemelisiniz. Bu RAB'bin Fisih* kurbanidir.
\par 12 "O gece Misir'dan geçecegim. Hem insanlarin hem de hayvanlarin bütün ilk doganlarini öldürecegim. Misir'in bütün ilahlarini yargilayacagim. Ben RAB'bim.
\par 13 Bulundugunuz evlerin üzerindeki kan sizin için belirti olacak. Kani görünce üzerinizden geçecegim. Misir'i cezalandirirken ölüm saçan size hiçbir zarar vermeyecek.
\par 14 Bu gün sizin için anma günü olacak. Bu günü RAB'bin bayrami olarak kutlayacaksiniz. Gelecek kusaklariniz boyunca sürekli bir kural olarak bu günü kutlayacaksiniz."
\par 15 "Yedi gün mayasiz ekmek yiyeceksiniz. Ilk gün evlerinizden mayayi kaldiracaksiniz. Kim bu yedi gün içinde mayali bir sey yerse, Israil'den atilacaktir.
\par 16 Birinci ve yedinci günler kutsal toplanti yapacaksiniz. O günler hiçbir is yapilmayacak. Herkes yalniz kendi yiyecegini hazirlayacak.
\par 17 Mayasiz Ekmek Bayrami'ni* kutlayacaksiniz, çünkü sizi ordular halinde o gün Misir'dan çikardim. Bu günü kalici bir kural olarak kusaklariniz boyunca kutlayacaksiniz.
\par 18 Birinci ayin* on dördüncü gününün aksamindan yirmi birinci gününün aksamina kadar mayasiz ekmek yiyeceksiniz.
\par 19 Evlerinizde yedi gün maya bulunmayacak. Mayali bir sey yiyen yerli yabanci herkes Israil toplulugundan atilacaktir.
\par 20 Mayali bir sey yemeyeceksiniz. Yasadiginiz her yerde mayasiz ekmek yiyeceksiniz."
\par 21 Musa Israil'in bütün ileri gelenlerini çagirtarak onlara söyle dedi: "Hemen gidin, aileleriniz için kendinize davarlar seçip Fisih* kurbani olarak bogazlayin.
\par 22 Bir demet mercanköskotu alin, legendeki kana batirip kani kapilarinizin yan ve üst sövelerine sürün. Sabaha kadar kimse evinden çikmasin.
\par 23 RAB Misirlilar'i öldürmek için gelecek, kapilarinizin yan ve üst sövelerindeki kani görünce üzerinden geçecek, ölüm saçanin evlerinize girip sizi öldürmesine izin vermeyecek.
\par 24 "Sen ve çocuklarin kalici bir kural olarak bu olayi kutlayacaksiniz.
\par 25 RAB'bin size söz verdigi topraklara girdiginiz zaman bu töreye uyacaksiniz.
\par 26 Çocuklariniz size, 'Bu törenin anlami nedir?' diye sorduklarinda,
\par 27 'Bu RAB'bin Fisih kurbanidir' diyeceksiniz, 'Çünkü RAB Misirlilar'i öldürürken evlerimizin üzerinden geçerek bizi bagisladi.'" Israilliler egilip tapindilar.
\par 28 Sonra gidip RAB'bin Musa'yla Harun'a verdigi buyrugu eksiksiz uyguladilar.
\par 29 Gece yarisi RAB tahtinda oturan firavunun ilk çocugundan zindandaki tutsagin ilk çocuguna kadar Misir'daki bütün insanlarin ve hayvanlarin ilk doganlarini öldürdü.
\par 30 O gece firavunla görevlileri ve bütün Misirlilar uyandi. Büyük feryat koptu. Çünkü ölüsü olmayan ev yoktu.
\par 31 Ayni gece firavun Musa'yla Harun'u çagirtti ve, "Kalkin!" dedi, "Siz ve Israilliler halkimin arasindan çikip gidin, istediginiz gibi RAB'be tapin.
\par 32 Dediginiz gibi davarlarinizi, sigirlarinizi da alin götürün. Beni de kutsayin!"
\par 33 Israilliler'in ülkeyi hemen terk etmesi için Misirlilar diretti. "Yoksa hepimiz ölecegiz!" diyorlardi.
\par 34 Böylece halk mayasi henüz katilmamis hamurunu aldi, giysilere sarili hamur teknelerini omuzlarinda tasidi.
\par 35 Israilliler Musa'nin dedigini yapmis, Misirlilar'dan altin, gümüs esya ve giysi istemislerdi.
\par 36 RAB Israilliler'in Misirlilar'in gözünde lütuf bulmasini sagladi. Misirlilar onlara istediklerini verdiler. Böylece Israilliler onlari soydular.
\par 37 Israilliler kadin ve çocuklarin disinda alti yüz bin kadar erkekle yaya olarak Ramses'ten Sukkot'a dogru yola çiktilar.
\par 38 Daha pek çok kisi de onlarla birlikte gitti. Yanlarinda çok sayida davar ve sigir vardi.
\par 39 Misir'dan getirdikleri hamurla mayasiz pide pisirdiler. Maya yoktu. Çünkü Misir'dan kovulmuslar, kendilerine azik hazirlayacak zaman bulamamislardi.
\par 40 Israilliler Misir'da dört yüz otuz yil yasadi.
\par 41 Dört yüz otuz yilin sonuncu günü RAB'bin halki ordular halinde Misir'i terk etti.
\par 42 O gece RAB Israilliler'i Misir'dan çikarmak için sürekli bekledi. Israilliler de kusaklar boyunca ayni gece RAB'bi yüceltmek için uyanik olmalidir.
\par 43 RAB Musa'yla Harun'a söyle dedi: "Fisih Bayrami'nin* kurallari sunlardir: Hiçbir yabanci Fisih* etini yemeyecek.
\par 44 Ama satin aldiginiz köleler sünnet edildikten sonra ondan yiyebilir.
\par 45 Konuklar ve ücretli isçiler ondan yemeyecek.
\par 46 Fisih eti evde yenmeli, evin disina çikarilmamali. Kemikleri kirmayacaksiniz.
\par 47 Bütün Israil toplulugu Fisih Bayrami'ni kutlayacak.
\par 48 Yaninizdaki yabanci bir konuk RAB'bin Fisih Bayrami'ni kutlamak isterse, önce evindeki bütün erkekler sünnet edilmeli; sonra yerel halktan biri gibi Israil halkina katilip bayrami kutlayabilir. Ama sünnetsiz* biri Fisih etini yemeyecektir.
\par 49 Ülkede dogan için de, aranizda yasayan yabanci için de ayni kural geçerlidir."
\par 50 Israilliler RAB'bin Musa'yla Harun'a verdigi buyrugu eksiksiz yerine getirdiler.
\par 51 O gün RAB Israilliler'i ordular halinde Misir'dan çikardi.

\chapter{13}

\par 1 RAB Musa'ya, "Bütün ilk doganlari bana adayin" dedi, "Israilliler arasinda insan olsun, hayvan olsun her rahmin ilk ürünü bana aittir."
\par 2 (#13:1)
\par 3 Musa halka, "Misir'dan, köle oldugunuz ülkeden çiktiginiz bugünü animsayin" dedi, "Çünkü RAB güçlü eliyle sizi oradan çikardi. Mayali hiçbir sey yenmeyecek.
\par 4 Bugün Aviv ayinda* buradan ayriliyorsunuz.
\par 5 RAB sizi Kenan, Hitit*, Amor, Hiv ve Yevus topraklarina, atalariniza verecegine ant içtigi süt ve bal akan ülkeye götürdügü zaman bu ay su törelere uyacaksiniz:
\par 6 Yedi gün mayasiz ekmek yiyecek, yedinci gün RAB'be bayram yapacaksiniz.
\par 7 O yedi gün içinde yalniz mayasiz ekmek yiyeceksiniz. Aranizda ve ülkenizin hiçbir yerinde mayali bir sey görülmeyecek.
\par 8 O gün ogullariniza, 'Misir'dan çiktigimizda RAB'bin bizim için yaptiklarindan dolayi bunlari yapiyoruz' diye anlatacaksiniz.
\par 9 Bu elinizde bir belirti ve alninizda bir anma isareti olacak; öyle ki, RAB'bin yasasi hep agzinizda olsun. Çünkü RAB güçlü eliyle sizi Misir'dan çikardi.
\par 10 Siz de her yil belirlenen tarihte bu kurali uygulamalisiniz.
\par 11 "RAB size ve atalariniza ant içerek söz verdigi gibi sizi Kenan topraklarina getirecektir. Orayi size verdigi zaman,
\par 12 ilk dogan erkek çocuklarinizin ve hayvanlarinizin hepsini RAB'be adayacaksiniz. Çünkü bunlar RAB'be aittir.
\par 13 Ilk dogan her sipanin bedelini bir kuzuyla ödeyin. Bedelini ödemezseniz, boynunu kirin. Bütün ilk dogan erkek çocuklarinizin bedelini ödemelisiniz.
\par 14 "Ilerde ogullariniz size, 'Bunun anlami ne?' diye sorduklarinda, 'RAB bizi güçlü eliyle Misir'dan, köle oldugumuz ülkeden çikardi' diye yanitlarsiniz,
\par 15 'Firavun bizi salivermemekte diretince, RAB Misir'da insanlarin ve hayvanlarin bütün ilk doganlarini öldürdü. Iste bunun için hayvanlarin ilk dogan erkek yavrularini RAB'be kurban ediyoruz. Ilk dogan erkek çocuklarimizin bedelini ise bir hayvanla ödüyoruz.'
\par 16 Bu uygulama elinizde bir belirti ve alninizda bir anma isareti olacak; RAB'bin bizi Misir'dan güçlü eliyle çikardigini animsatacak."
\par 17 Firavun Israilliler'i saliverdiginde, Filist yöresi yakin olmasina karsin, Tanri onlari oradan götürmedi. Çünkü, "Halk savasla karsilasinca, düsüncelerini degistirip Misir'a geri dönebilir" diye düsündü.
\par 18 Halki çöl yolundan Kizildeniz'e* dogru dolastirdi. Israilliler Misir'dan silahli çikmislardi.
\par 19 Musa Yusuf'un kemiklerini yanina almisti. Çünkü Yusuf Israil'in ogullarina, "Tanri kesinlikle size yardim edecek, kemiklerimi buradan götüreceksiniz" diye siki siki ant içirmisti.
\par 20 Sukkot'tan ayrilip çöl kenarinda, Etam'da konakladilar.
\par 21 Gece gündüz ilerlemeleri için, RAB gündüzün bir bulut sütunu içinde yol göstererek, geceleyin bir ates sütunu içinde isik vererek onlara öncülük ediyordu.
\par 22 Gündüz bulut sütunu, gece ates sütunu halkin önünden eksik olmadi.

\chapter{14}

\par 1 RAB Musa'ya, "Israilliler'e söyle, dönsünler" dedi, "Pi- Hahirot yakinlarinda, Migdol ile deniz arasinda, Baal-Sefon'un karsisinda deniz kiyisinda konaklasinlar.
\par 2 (#14:1)
\par 3 Firavun söyle düsünecek: 'Israilliler ülkede saskin saskin dolasiyorlardir, çöl onlari kusatmistir.'
\par 4 Firavunu inatçi yapacagim. Onlarin pesine düsecek. Böylece firavunla ordusunu yenerek yücelik kazanacagim. Misirlilar bilecek ki, ben RAB'bim." Israilliler söyleneni yaptilar.
\par 5 Halkin kaçtigi Misir Firavunu'na bildirilince, firavunla görevlileri onlara iliskin düsüncelerini degistirdiler: "Biz ne yaptik?" dediler, "Israilliler'i salivermekle kölelerimizi kaybetmis olduk!"
\par 6 Firavun savas arabasini hazirlatti, ordusunu yanina aldi.
\par 7 Seçme alti yüz savas arabasinin yanisira, Misir'in bütün savas arabalarini sorumlu sürücüleriyle birlikte yanina aldi.
\par 8 RAB Misir Firavunu'nu inatçi yapti. Firavun zafer havasi içinde ilerleyen Israilliler'in pesine düstü.
\par 9 Misirlilar firavunun bütün atlari, savas arabalari, atlilari, askerleriyle onlarin ardina düstüler ve deniz kiyisinda, Pi-Hahirot yakinlarinda, Baal-Sefon'un karsisinda konaklarken onlara yetistiler.
\par 10 Firavun yaklasirken, Israilliler Misirlilar'in arkalarindan geldigini görünce dehsete kapilarak RAB'be feryat ettiler.
\par 11 Musa'ya, "Misir'da mezar mi yoktu da bizi çöle ölmeye getirdin?" dediler, "Bak, Misir'dan çikarmakla bize ne yaptin!
\par 12 Misir'dayken sana, 'Birak bizi, Misirlilar'a kulluk edelim' demedik mi? Çölde ölmektense Misirlilar'a kulluk etsek bizim için daha iyi olurdu."
\par 13 Musa, "Korkmayin!" dedi, "Yerinizde durup bekleyin, RAB bugün sizi nasil kurtaracak görün. Bugün gördügünüz Misirlilar'i bir daha hiç görmeyeceksiniz.
\par 14 RAB sizin için savasacak, siz sakin olun yeter."
\par 15 RAB Musa'ya, "Niçin bana feryat ediyorsun?" dedi, "Israilliler'e söyle, ilerlesinler.
\par 16 Sen degnegini kaldir, elini denizin üzerine uzat. Sular yarilacak ve Israilliler kuru toprak üzerinde yürüyerek denizi geçecekler.
\par 17 Ben Misirlilar'i inatçi yapacagim ki, artlarina düssünler. Firavunu, bütün ordusunu, savas arabalarini, atlilarini yenerek yücelik kazanacagim.
\par 18 Firavun, savas arabalari ve atlilarindan ötürü yücelik kazandigim zaman, Misirlilar bilecek ki, ben RAB'bim."
\par 19 Israil ordusunun önünde yürüyen Tanri'nin melegi yerini degistirip arkaya geçti. Önlerindeki bulut sütunu da yerini degistirip arkalarina, Misir ve Israil ordularinin arasina geldi. Gece boyunca bulut bir yani karartiyor, öbür yani aydinlatiyordu. Bu yüzden, bütün gece iki taraf birbirine yaklasamadi.
\par 20 (#14:1)
\par 21 Musa elini denizin üzerine uzatti. RAB bütün gece güçlü dogu rüzgariyla sulari geri itti, denizi karaya çevirdi. Sular ikiye bölündü,
\par 22 Israilliler kuru toprak üzerinde yürüyerek denizi geçtiler. Sular saglarinda, sollarinda onlara duvar olusturdu.
\par 23 Misirlilar artlarindan geliyordu. Firavunun bütün atlari, savas arabalari, atlilari denizde onlari izliyordu.
\par 24 Sabah nöbetinde RAB ates ve bulut sütunundan Misir ordusuna bakti ve onlari saskina çevirdi.
\par 25 Arabalarinin tekerleklerini çikardi; öyle ki, arabalarini zorlukla sürdüler. Misirlilar, "Israilliler'den kaçalim!" dediler, "Çünkü RAB onlar için bizimle savasiyor."
\par 26 RAB Musa'ya, "Elini denizin üzerine uzat" dedi, "Sular Misirlilar'in, savas arabalarinin, atlilarinin üzerine dönsün."
\par 27 Musa elini denizin üzerine uzatti. Sabaha karsi deniz olagan haline döndü. Misirlilar sulardan kaçarken RAB onlari denizin ortasinda silkip atti.
\par 28 Geri dönen sular savas arabalarini, atlilari, Israilliler'in pesinden denize dalan firavunun bütün ordusunu yuttu. Onlardan bir kisi bile sag kalmadi.
\par 29 Ama Israilliler denizi kuru toprakta yürüyerek geçmislerdi. Sular saglarinda, sollarinda onlara duvar olusturmustu.
\par 30 RAB o gün Israilliler'i Misirlilar'in elinden kurtardi. Israilliler deniz kiyisinda Misirlilar'in ölülerini gördüler.
\par 31 RAB'bin Misirlilar'a gösterdigi büyük gücü gören Israil halki RAB'den korkup O'na ve kulu Musa'ya güvendi.

\chapter{15}

\par 1 Musa'yla Israilliler RAB'be su ezgiyi söylediler: "Ezgiler sunacagim RAB'be, Çünkü yüceldikçe yüceldi; Atlari da, atlilari da denize döktü.
\par 2 Rab gücüm ve ezgimdir, O kurtardi beni. O'dur Tanrim, Övgüler sunacagim O'na. O'dur babamin Tanrisi, Yüceltecegim O'nu.
\par 3 Savas eridir RAB, Adi RAB'dir.
\par 4 "Denize atti firavunun ordusunu, Savas arabalarini. Kizildeniz'de* boguldu seçme subaylari.
\par 5 Derin sulara gömüldüler, Tas gibi dibe indiler.
\par 6 "Senin sag elin, ya RAB, Senin sag elin korkunç güce sahiptir. Altinda düsmanlar kirilir.
\par 7 Devrilir sana baskaldiranlar büyük görkemin karsisinda, Gönderir gazabini aniz gibi tüketirsin onlari.
\par 8 Burnunun solugu karsisinda, Sular yigildi bir araya. Kabaran sular duvarlara dönüstü, Denizin göbegindeki derin sular dondu.
\par 9 Düsman böbürlendi: 'Peslerine düsüp yakalayacagim onlari' dedi, 'Bölüsecegim çapulu, Dilegimce yagmalayacagim, Kilicimi çekip yok edecegim onlari.'
\par 10 Üfledin solugunu, denize gömüldüler, Kursun gibi engin sulara battilar.
\par 11 "Var mi senin gibisi ilahlar arasinda, ya RAB? Senin gibi kutsallikta görkemli, heybetiyle övgüye deger, Harikalar yaratan var mi?
\par 12 Sag elini uzattin, Yer yuttu onlari.
\par 13 Öncülük edeceksin sevginle kurtardigin halka, Kutsal konutunun yolunu göstereceksin gücünle onlara.
\par 14 Uluslar duyup titreyecekler, Filist halkini dehset saracak.
\par 15 Edom beyleri korkuya kapilacak, Moav önderlerini titreme alacak, Kenan'da yasayanlarin tümü korkudan eriyecek.
\par 16 Korku ve dehset düsecek üzerlerine, Senin halkin geçinceye dek, ya RAB, Sahip oldugun bu halk geçinceye dek, Bileginin gücü karsisinda tas kesilecekler.
\par 17 Ya RAB, halkini içeri alacaksin. Kendi dagina, yasamak için seçtigin yere, Ellerinle kurdugun kutsal yere dikeceksin, ya Rab!
\par 18 RAB sonsuza dek egemen olacak."
\par 19 Firavunun atlari, savas arabalari, atlilari denize dalinca, RAB sulari onlarin üzerine çevirdi. Ama Israilliler denizi kuru toprakta yürüyerek geçtiler.
\par 20 Harun'un kizkardesi Peygamber Miryam tefini eline aldi, bütün kadinlar teflerle, oynayarak onu izlediler.
\par 21 Miryam onlara su ezgiyi söyledi: "Ezgiler sunun RAB'be, Çünkü yüceldikçe yüceldi, Atlari, atlilari denize döktü."
\par 22 Musa Israilliler'i Kizildeniz'in* ötesine çikardi. Sur Çölü'ne girdiler. Çölde üç gün yol aldilarsa da su bulamadilar.
\par 23 Mara'ya vardilar. Ama Mara'nin suyunu içemediler, çünkü su aciydi. Bu yüzden oraya Mara adi verildi.
\par 24 Halk, "Ne içecegiz?" diye Musa'ya yakinmaya basladi.
\par 25 Musa RAB'be yakardi. RAB ona bir agaç parçasi gösterdi. Musa onu suya atinca sular tatli oldu. Orada RAB onlar için bir kural ve ilke koydu, hepsini sinadi.
\par 26 "Ben, Tanriniz RAB'bin sözünü dikkatle dinler, gözümde dogru olani yapar, buyruklarima kulak verir, bütün kurallarima uyarsaniz, Misirlilar'a verdigim hastaliklarin hiçbirini size vermeyecegim" dedi, "Çünkü size sifa veren RAB benim."
\par 27 Sonra Elim'e gittiler. Orada on iki su kaynagi, yetmis hurma agaci vardi. Su kiyisinda konakladilar.

\chapter{16}

\par 1 Bütün Israil toplulugu Elim'den ayrildi. Misir'dan çiktiktan sonra ikinci ayin* on besinci günü Elim ile Sina arasindaki Sin Çölü'ne vardilar.
\par 2 Çölde hepsi Musa'yla Harun'a yakinmaya basladi.
\par 3 "Keske RAB bizi Misir'dayken öldürseydi" dediler, "Hiç degilse orada et kazanlarinin basina oturur, doyasiya yerdik. Ama siz bütün toplulugu açliktan öldürmek için bizi bu çöle getirdiniz."
\par 4 RAB Musa'ya, "Size gökten ekmek yagdiracagim" dedi, "Halk her gün gidip günlük ekmegini toplayacak. Böylece onlari sinayacagim: Benim yasama göre yasiyorlar mi, yasamiyorlar mi, görecegim.
\par 5 Altinci gün her gün topladiklarinin iki katini toplayip hazirlayacaklar."
\par 6 Musa'yla Harun Israilliler'e, "Bu aksam sizi Misir'dan RAB'bin çikardigini bileceksiniz" dediler,
\par 7 "Sabah da RAB'bin görkemini göreceksiniz. Çünkü RAB kendisine söylendiginizi duydu. Biz kimiz ki, bize söyleniyorsunuz?"
\par 8 Sonra Musa, "Aksam size yemek için et, sabah da dilediginiz kadar ekmek verilince, RAB'bin görkemini göreceksiniz" dedi, "Çünkü RAB kendisine söylendiginizi duydu. Biz kimiz ki? Siz bize degil, RAB'be söyleniyorsunuz."
\par 9 Musa Harun'a, "Bütün Israil topluluguna söyle, RAB'bin huzuruna gelsinler" dedi, "Çünkü RAB söylendiklerini duydu."
\par 10 Harun Israil topluluguna bunlari anlatirken, çöle dogru baktilar. RAB'bin görkemi bulutta görünüyordu.
\par 11 RAB Musa'ya söyle dedi:
\par 12 "Israilliler'in yakinmalarini duydum. Onlara de ki, 'Aksamüstü et yiyeceksiniz, sabah da ekmekle karninizi doyuracaksiniz. O zaman bileceksiniz ki, Tanriniz RAB benim.'"
\par 13 Aksam bildircinlar geldi, ordugahi sardi. Sabah ordugahin çevresini çiy kaplamisti.
\par 14 Çiy eriyince, toprakta, çölün yüzeyinde kiragiya benzer ince pulcuklar göründü.
\par 15 Bunu görünce Israilliler birbirlerine, "Bu da ne?" diye sordular. Çünkü ne oldugunu anlayamamislardi. Musa, "RAB'bin size yemek için verdigi ekmektir bu" dedi,
\par 16 "RAB'bin buyrugu sudur: 'Herkes yiyecegi kadar toplasin. Çadirinizdaki her kisi için birer omer alin.'"
\par 17 Israilliler söyleneni yaptilar. Kimi çok, kimi az topladi.
\par 18 Omerle ölçtüklerinde, çok toplayanin fazlasi, az toplayanin da eksigi yoktu. Herkes yiyecegi kadar toplamisti.
\par 19 Musa onlara, "Kimse sabaha bir parça bile birakmasin" dedi.
\par 20 Ama bazilari ona aldirmayip sabaha biraktilar. Biraktiklari kurtlanip kokmaya baslayinca Musa onlara öfkelendi.
\par 21 Her sabah herkes yiyecegi kadar topluyordu. Günes ortaligi isitinca, yerde kalanlar eriyordu.
\par 22 Altinci gün kisi basina iki omer, yani iki kat topladilar. Toplulugun önderleri gelip durumu Musa'ya bildirdiler.
\par 23 Musa, "RAB'bin buyrugu sudur" dedi, "'Yarin dinlenme günü, RAB için kutsal Sabat Günü'dür*. Pisireceginizi pisirin, haslayacaginizi haslayin. Artakalani bir kenara koyun, sabaha kalsin.'"
\par 24 Musa'nin buyurdugu gibi artakalani sabaha biraktilar. Ne koktu, ne kurtlandi.
\par 25 Musa, "Artakalani bugün yiyin" dedi, "Çünkü bugün RAB için Sabat Günü'dür. Bugün disarda ekmek bulamayacaksiniz.
\par 26 Alti gün ekmek toplayacaksiniz, ama yedinci gün olan Sabat Günü ekmek bulunmayacak."
\par 27 Yedinci gün bazilari ekmek toplamak için disari çikti, ama hiçbir sey bulamadilar.
\par 28 RAB Musa'ya, "Ne zamana dek buyruklarima ve yasalarima uymayi reddedeceksiniz?" dedi,
\par 29 "Size Sabat Günü'nü verdim. Bunun için altinci gün size iki günlük ekmek veriyorum. Yedinci gün herkes neredeyse orada kalsin, disari çikmasin."
\par 30 Böylece halk yedinci gün dinlendi.
\par 31 Israilliler o ekmege man adini verdiler. Kisnis tohumu gibi beyazimsi, tadi balli yufka gibiydi.
\par 32 Musa, "RAB'bin buyrugu sudur" dedi, "'Misir'dan sizi çikardigimda, gelecek kusaklarin çölde size yedirdigim ekmegi görmesi için, bir omer saklansin.'"
\par 33 Musa Harun'a, "Bir testi al, içine bir omer man doldur" dedi, "Gelecek kusaklar için saklanmak üzere onu RAB'bin huzuruna koy."
\par 34 RAB'bin Musa'ya buyurdugu gibi Harun mani saklanmak üzere Antlasma Levhalari'nin önüne koydu.
\par 35 Israilliler yerlestikleri Kenan topraklarina varincaya dek kirk yil man yediler.
\par 36 -Bir omer efanin onda biridir.-

\chapter{17}

\par 1 RAB'bin buyrugu uyarinca, bütün Israil toplulugu Sin Çölü'nden ayrildi, bir yerden öbürüne göçerek Refidim'de konakladi. Ancak orada içecek su yoktu.
\par 2 Musa'ya, "Bize içecek su ver" diye çikistilar. Musa, "Niçin bana çikisiyorsunuz?" dedi, "Neden RAB'bi deniyorsunuz?"
\par 3 Ama halk susamisti. "Niçin bizi Misir'dan çikardin?" diye Musa'ya söylendiler, "Bizi, çocuklarimizi, hayvanlarimizi susuzluktan öldürmek için mi?"
\par 4 Musa, "Bu halka ne yapayim?" diye RAB'be feryat etti, "Neredeyse beni taslayacaklar."
\par 5 RAB Musa'ya, "Halkin önüne geç" dedi, "Birkaç Israil ileri gelenini ve Nil'e vurdugun degnegi de yanina alip yürü.
\par 6 Ben Horev Dagi'nda bir kayanin üzerinde, senin önünde duracagim. Kayaya vuracaksin, halk içsin diye su fiskiracak." Musa Israil ileri gelenlerinin önünde denileni yapti.
\par 7 Oraya Massa ve Meriva adi verildi. Çünkü Israilliler orada Musa'ya çikismis ve, "Acaba RAB aramizda mi, degil mi?" diye RAB'bi denemislerdi.
\par 8 Amalekliler gelip Refidim'de Israilliler'e savas açtilar.
\par 9 Musa Yesu'ya, "Adam seç, git Amalekliler'le savas" dedi, "Yarin ben elimde Tanri'nin degnegiyle tepenin üzerinde duracagim."
\par 10 Yesu Musa'nin buyurdugu gibi Amalekliler'le savasti. Bu arada Musa, Harun ve Hur tepenin üzerine çiktilar.
\par 11 Musa elini kaldirdikça Israilliler, indirdikçe Amalekliler kazaniyordu.
\par 12 Ne var ki, Musa'nin elleri yoruldu. Bir tas getirip altina koydular. Musa üzerine oturdu. Bir yanda Harun, öbür yanda Hur Musa'nin ellerini yukarida tuttular. Günes batincaya dek Musa'nin elleri yukarida kaldi.
\par 13 Böylece Yesu Amalek ordusunu yenip kiliçtan geçirdi.
\par 14 RAB Musa'ya, "Bunu ani olarak kayda geç" dedi, "Yesu'ya da söyle, Amalekliler'in adini yeryüzünden büsbütün silecegim."
\par 15 Musa bir sunak yapti, adini "RAB sancagimdir" koydu.
\par 16 "Eller Rab'bin tahtina dogru kaldirildi" dedi, "RAB kusaklar boyunca Amalekliler'e karsi savasacak!"

\chapter{18}

\par 1 Musa'nin kayinbabasi Midyanli Kâhin Yitro, Tanri'nin Musa ve halki Israil için yaptigi her seyi, RAB'bin Israilliler'i Misir'dan nasil çikardigini duydu.
\par 2 Musa'nin kendisine göndermis oldugu karisi Sippora'yi ve iki oglunu yanina aldi. Musa, "Garibim bu yabanci diyarda" diyerek ogullarindan birine Gersom adini vermisti.
\par 3 (#18:2)
\par 4 Sonra, "Babamin Tanrisi bana yardim etti, beni firavunun kilicindan esirgedi" diyerek öbürüne de Eliezer adini koymustu.
\par 5 Yitro Musa'nin karisi ve ogullariyla birlikte Tanri Dagi'na, Musa'nin konakladigi çöle geldi.
\par 6 Musa'ya su haberi gönderdi: "Ben, kayinbaban Yitro, karin ve iki oglunla birlikte sana geliyoruz."
\par 7 Musa kayinbabasini karsilamaya çikti, önünde egilip onu öptü. Birbirinin hatirini sorup çadira girdiler.
\par 8 Musa Israilliler ugruna RAB'bin firavunla Misirlilar'a bütün yaptiklarini, yolda çektikleri sikintilari, RAB'bin kendilerini nasil kurtardigini kayinbabasina bir bir anlatti.
\par 9 Yitro RAB'bin Israilliler'e yaptigi iyiliklere, onlari Misirlilar'in elinden kurtardigina sevindi.
\par 10 "Sizi Misirlilar'in ve firavunun elinden kurtaran RAB'be övgüler olsun" dedi, "Halki Misir'in boyundurugundan O kurtardi.
\par 11 Artik biliyorum ki, RAB bütün ilahlardan büyüktür. Çünkü onlarin gurur duydugu seylerin üstesinden geldi."
\par 12 Sonra Tanri'ya yakmalik sunu* ve kurbanlar getirdi. Harun'la bütün Israil ileri gelenleri, Musa'nin kayinbabasiyla Tanri'nin huzurunda yemek yemeye geldiler.
\par 13 Ertesi gün Musa halkin davalarina bakmak için yargi kürsüsüne çikti. Halk sabahtan aksama kadar çevresinde ayakta durdu.
\par 14 Kayinbabasi Musa'nin halk için yaptiklarini görünce, "Nedir bu, halka yaptigin?" dedi, "Neden sen tek basina yargiç olarak oturuyorsun da herkes sabahtan aksama kadar çevrende bekliyor?"
\par 15 Musa, "Çünkü halk Tanri'nin istemini bilmek için bana geliyor" diye yanitladi,
\par 16 "Ne zaman bir sorunlari olsa, bana gelirler. Ben de taraflar arasinda karar veririm; Tanri'nin kurallarini, yasalarini onlara bildiririm."
\par 17 Kayinbabasi, "Yaptigin is iyi degil" dedi,
\par 18 "Hem sen, hem de yanindaki halk tükeneceksiniz. Bu isi tek basina kaldiramazsin. Sana agir gelir.
\par 19 Beni dinle, sana ögüt vereyim. Tanri seninle olsun. Tanri'nin önünde halki sen temsil etmeli, sorunlarini Tanri'ya sen iletmelisin.
\par 20 Kurallari, yasalari halka ögret, izlemeleri gereken yolu, yapacaklari isi göster.
\par 21 Bunun yanisira halkin arasindan Tanri'dan korkan, yetenekli, haksiz kazançtan nefret eden dürüst adamlar seç; onlari biner, yüzer, elliser, onar kisilik topluluklarin basina önder ata.
\par 22 Halka sürekli onlar yargiçlik etsin. Büyük davalari sana getirsinler, küçük davalari kendileri çözsünler. Böylece isini paylasmis olurlar. Yükün hafifler.
\par 23 Eger böyle yaparsan, Tanri da buyurursa, dayanabilirsin. Herkes esenlik içinde evine döner."
\par 24 Musa kayinbabasinin sözünü dinledi. Söyledigi her seyi yerine getirdi.
\par 25 Israilliler arasindan yetenekli adamlar seçti. Onlari biner, yüzer, elliser, onar kisilik topluluklarin basina önder atadi.
\par 26 Halka sürekli yargiçlik eden bu kisiler zor davalari Musa'ya getirdiler, küçük davalari ise kendileri çözdüler.
\par 27 Sonra Musa kayinbabasini ugurladi. Yitro da ülkesine döndü.

\chapter{19}

\par 1 Israilliler Misir'dan çiktiktan tam üç ay sonra Sina Çölü'ne vardilar.
\par 2 Refidim'den yola çikip Sina Çölü'ne girdiler. Orada, Sina Dagi'nin karsisinda konakladilar.
\par 3 Musa Tanri'nin huzuruna çikti. RAB dagdan kendisine seslendi: "Yakup soyuna, Israil halkina söyle diyeceksin:
\par 4 Misirlilar'a ne yaptigimi, sizi nasil kartal kanatlari üzerinde tasiyarak yanima getirdigimi gördünüz.
\par 5 Simdi sözümü dikkatle dinler, antlasmama uyarsaniz, bütün uluslar içinde öz halkim olursunuz. Çünkü yeryüzünün tümü benimdir.
\par 6 Siz benim için kâhinler kralligi, kutsal ulus olacaksiniz. Israilliler'e böyle söyleyeceksin."
\par 7 Musa gidip halkin ileri gelenlerini çagirdi ve RAB'bin kendisine buyurdugu her seyi onlara anlatti.
\par 8 Bütün halk bir agizdan, "RAB'bin söyledigi her seyi yapacagiz" diye yanitladilar. Musa halkin yanitini RAB'be iletti.
\par 9 RAB Musa'ya, "Sana koyu bir bulut içinde gelecegim" dedi, "Öyle ki, seninle konusurken halk isitsin ve her zaman sana güvensin." Musa halkin söylediklerini RAB'be iletti.
\par 10 RAB Musa'ya, "Git, bugün ve yarin halki arindir" dedi, "Giysilerini yikasinlar.
\par 11 Üçüncü güne hazir olsunlar. Çünkü üçüncü gün bütün halkin gözü önünde ben, RAB Sina Dagi'na inecegim.
\par 12 Dagin çevresine sinir çiz ve halka de ki, 'Sakin daga çikmayin, dagin etegine de yaklasmayin! Kim daga dokunursa, kesinlikle öldürülecektir.
\par 13 Ya taslanacak, ya da okla vurulacak; ona insan eli degmeyecek. Ister hayvan olsun ister insan, yasamasina izin verilmeyecek.' Ancak boru uzun uzun çalininca daga çikabilirler."
\par 14 Sonra Musa dagdan halkin yanina inip onlari arindirdi. Herkes giysilerini yikadi.
\par 15 Musa halka, "Üçüncü güne hazir olun" dedi, "Bu süre içinde cinsel iliskide bulunmayin."
\par 16 Üçüncü günün sabahi gök gürledi, simsekler çakti. Dagin üzerinde koyu bir bulut vardi. Derken, çok güçlü bir boru sesi duyuldu. Ordugahta herkes titremeye basladi.
\par 17 Musa halkin Tanri'yla görüsmek üzere ordugahtan çikmasina öncülük etti. Dagin eteginde durdular.
\par 18 Sina Dagi'nin her yanindan duman tütüyordu. Çünkü RAB dagin üstüne ates içinde inmisti. Dagdan ocak dumani gibi duman çikiyor, bütün dag siddetle sarsiliyordu.
\par 19 Boru sesi gitgide yükselince, Musa konustu ve Tanri gök gürlemeleriyle onu yanitladi.
\par 20 RAB Sina Dagi'nin üzerine indi, Musa'yi dagin tepesine çagirdi. Musa tepeye çikti.
\par 21 RAB, "Asagi inip halki uyar" dedi, "Sakin beni görmek için siniri geçmesinler, yoksa birçogu ölür.
\par 22 Bana yaklasan kâhinler de kendilerini kutsasinlar, yoksa onlari siddetle cezalandiririm."
\par 23 Musa, "Halk Sina Dagi'na çikamaz" diye karsilik verdi, "Çünkü sen, 'Dagin çevresine sinir çiz, onu kutsal kil' diyerek bizi uyardin."
\par 24 RAB, "Asagi inip Harun'u getir" dedi, "Ama kâhinlerle halk huzuruma gelmek için siniri geçmesinler. Yoksa onlari siddetle cezalandiririm."
\par 25 Bunun üzerine Musa asagi inip durumu halka anlatti.

\chapter{20}

\par 1 Tanri söyle konustu:
\par 2 "Seni Misir'dan, köle oldugun ülkeden çikaran Tanrin RAB benim.
\par 3 "Benden baska tanrin olmayacak.
\par 4 "Kendine yukarida gökyüzünde, asagida yeryüzünde ya da yer altindaki sularda yasayan herhangi bir canliya benzer put yapmayacaksin.
\par 5 Putlarin önünde egilmeyecek, onlara tapmayacaksin. Çünkü ben, Tanrin RAB, kiskanç bir Tanri'yim. Benden nefret edenin babasinin isledigi suçun hesabini çocuklarindan, üçüncü, dördüncü kusaklardan sorarim.
\par 6 Ama beni seven, buyruklarima uyan binlerce kusaga sevgi gösteririm.
\par 7 "Tanrin RAB'bin adini bos yere agzina almayacaksin. Çünkü RAB, adini bos yere agzina alanlari cezasiz birakmayacaktir.
\par 8 "Sabat Günü'nü* kutsal sayarak animsa.
\par 9 Alti gün çalisacak, bütün islerini yapacaksin.
\par 10 Ama yedinci gün bana, Tanrin RAB'be Sabat Günü olarak adanmistir. O gün sen, oglun, kizin, erkek ve kadin kölen, hayvanlarin, aranizdaki yabancilar dahil, hiçbir is yapmayacaksiniz.
\par 11 Çünkü ben, RAB yeri gögü, denizi ve bütün canlilari alti günde yarattim, yedinci gün dinlendim. Bu yüzden Sabat Günü'nü kutsadim ve kutsal bir gün olarak belirledim.
\par 12 "Annene babana saygi göster. Öyle ki, Tanrin RAB'bin sana verecegi ülkede ömrün uzun olsun.
\par 13 "Adam öldürmeyeceksin.
\par 14 "Zina etmeyeceksin.
\par 15 "Çalmayacaksin.
\par 16 "Komsuna karsi yalan yere taniklik etmeyeceksin.
\par 17 "Komsunun evine, karisina, erkek ve kadin kölesine, öküzüne, esegine, hiçbir seyine göz dikmeyeceksin."
\par 18 Halk gök gürlemelerini, boru sesini duyup simsekleri ve dagin basindaki dumani görünce korkudan titremeye basladi. Uzakta durarak
\par 19 Musa'ya, "Bizimle sen konus, dinleyelim" dediler, "Ama Tanri konusmasin, yoksa ölürüz."
\par 20 Musa, "Korkmayin!" diye karsilik verdi, "Tanri sizi denemek için geldi; Tanri korkusu üzerinizde olsun, günah islemeyesiniz diye."
\par 21 Musa Tanri'nin içinde bulundugu koyu karanliga yaklasirken halk uzakta durdu.
\par 22 RAB Musa'ya söyle dedi: "Israilliler'e de ki, 'Göklerden sizinle konustugumu gördünüz.
\par 23 Benim yanimsira baska ilahlar yapmayacaksiniz, altin ya da gümüs ilahlar dökmeyeceksiniz.
\par 24 Benim için toprak bir sunak yapacaksiniz. Yakmalik ve esenlik sunularinizi*, davarlarinizi, sigirlarinizi onun üzerinde sunacaksiniz. Adimi animsattigim her yere gelip sizi kutsayacagim.
\par 25 Eger bana tas sunak yaparsaniz, yontma tas kullanmayin. Çünkü kullanacaginiz alet sunagin kutsalligini bozar.
\par 26 Sunagimin üzerine basamakla çikmayacaksiniz. Çünkü çiplak yeriniz görünebilir.'"

\chapter{21}

\par 1 "Israilliler'e su ilkeleri bildir:
\par 2 "Ibrani bir köle satin alirsan, alti yil kölelik edecek, ama yedinci yil karsilik ödemeden özgür olacak.
\par 3 Bekâr geldiyse, yalniz kendisi özgür olacak; evli geldiyse, karisi da özgür olacak.
\par 4 Efendisi kendisine bir kadin verir ve o kadindan çocuklari olursa, kadin ve çocuklar efendisinde kalacak, yalniz kendisi gidecek.
\par 5 "Ama köle açikça, 'Ben efendimi, karimla çocuklarimi seviyorum, özgür olmak istemiyorum' derse,
\par 6 efendisi onu yargiç huzuruna çikaracak. Kapiya ya da kapi sövesine yaklastirip kulagini bizle delecek. Böylece köle yasam boyu efendisine hizmet edecek.
\par 7 "Eger bir adam kizini cariye olarak satarsa, kiz erkek köleler gibi özgür birakilmayacak.
\par 8 Efendisi kizla nisanlanir, sonra kizdan hoslanmazsa, kizin geri alinmasina izin vermelidir. Kizi aldattigi için onu yabancilara satamaz.
\par 9 Eger cariyeyi ogluna nisanlarsa, ona kendi kizi gibi davranmalidir.
\par 10 Eger ikinci bir kadinla evlenirse, ilk karisini nafakadan, giysiden, karilik haklarindan yoksun birakmamalidir.
\par 11 Eger bu üç hakki ona vermezse, kadin karsiliksiz özgür olacaktir."
\par 12 "Kim birini vurup öldürürse, kendisi de kesinlikle öldürülecektir.
\par 13 Ama olayda kasit yoksa, ona ben izin vermissem, size adamin kaçacagi yeri bildirecegim.
\par 14 Eger bir adam komsusuna düzen kurar, kasitli olarak saldirip onu öldürürse, sunagima bile kaçmis olsa, onu çikarip öldüreceksiniz.
\par 15 "Kim annesini ya da babasini döverse, kesinlikle öldürülecektir.
\par 16 "Kim adam kaçirirsa, onu ister satmis olsun, ister elinde tutsun, kesinlikle öldürülecektir.
\par 17 "Annesine ya da babasina lanet eden kesinlikle öldürülecektir.
\par 18 "Kavga çikar, bir adam komsusuna tasla ya da yumrukla vurur, vurulan adam ölmeyip yataga düser,
\par 19 sonra kalkip degnekle disarida gezebilirse, vuran adam suçsuz sayilacaktir. Yalniz yaralinin kaybettigi zamanin karsiligini ödeyecek ve tümüyle iyilesmesini saglayacaktir.
\par 20 "Bir adam erkek ya da kadin kölesini degnekle döverken öldürürse, kesinlikle cezalandirilacaktir.
\par 21 Ama köle hemen ölmez, bir iki gün sonra ölürse, köle sahibi ceza görmeyecektir. Çünkü köle onun mali sayilir.
\par 22 "Iki kisi kavga ederken gebe bir kadina çarpar, kadin erken dogum yapar ama baska bir zarar görmezse, saldirgan, kadinin kocasinin istedigi ve yargiçlarin onayladigi miktarda para cezasina çarptirilacaktir.
\par 23 Ama baska bir zarar varsa, cana karsilik can, göze karsilik göz, dise karsilik dis, ele karsilik el, ayaga karsilik ayak, yaniga karsilik yanik, yaraya karsilik yara, bereye karsilik bere ödenecektir.
\par 24 (#21:23)
\par 25 (#21:23)
\par 26 "Bir adam erkek ya da kadin kölesini gözüne vurarak kör ederse, gözüne karsilik onu özgür birakacaktir.
\par 27 Eger erkek ya da kadin kölesinin disini kirarsa, disine karsilik onu özgür birakacaktir."
\par 28 "Eger bir boga bir erkegi ya da kadini boynuzuyla vurup öldürürse, kesinlikle taslanacak ve eti yenmeyecektir. Boganin sahibi ise suçsuz sayilacaktir.
\par 29 Ama saldirganligi bilinen bir boganin sahibi uyarilmasina karsin bogasina sahip çikmazsa ve bogasi bir erkegi ya da kadini öldürürse, hem boga taslanacak, hem de sahibi öldürülecektir.
\par 30 Ancak, boganin sahibinden para cezasi istenirse, istenen miktari ödeyerek canini kurtarabilir.
\par 31 Boga ister erkek, ister kiz çocugunu öldürsün, ayni kural uygulanacaktir.
\par 32 Eger boga bir erkek ya da kadin köleyi öldürürse, kölenin efendisine otuz sekel gümüs verilecek ve boga taslanacaktir.
\par 33 "Bir adam bir çukur açar ya da kazdigi çukurun üzerini örtmezse ve çukura bir boga ya da bir esek düserse,
\par 34 çukuru kazan hayvanin bedelini ödeyecektir. Parayi hayvanin sahibine verecek, ölü hayvan kendisinin olacaktir.
\par 35 "Bir adamin bogasi komsusunun bogasini yaralar, yarali boga ölürse, sag bogayi satip parasini paylasacak, ölü hayvani da bölüseceklerdir.
\par 36 Eger boganin saldirgan oldugu ve sahibinin ona sahip çikmadigi biliniyorsa, bogaya karsilik boga verecek ve ölü hayvan kendisine kalacaktir."

\chapter{22}

\par 1 "Bir adam öküz ya da davar çalip bogazlar ya da satarsa, bir öküze karsilik bes öküz, bir koyuna karsilik dört koyun ödeyecektir.
\par 2 "Bir hirsiz bir eve girerken yakalanip öldürülürse, öldüren kisi suçlu sayilmaz.
\par 3 Ancak olay günes dogduktan sonra olmussa, kan dökmekten sorumlu sayilir. "Hirsiz çaldiginin karsiligini kesinlikle ödemelidir. Hiçbir seyi yoksa, hirsizlik yaptigi için köle olarak satilacaktir.
\par 4 Çaldigi mal -öküz, esek ya da koyun- sag olarak elinde yakalanirsa, iki katini ödeyecektir.
\par 5 "Tarlada ya da bagda hayvanlarini otlatan bir adam, hayvanlarinin baskasinin tarlasinda otlamasina izin verirse, zarari kendi tarlasinin ya da baginin en iyi ürünleriyle ödeyecektir.
\par 6 "Birinin yaktigi ates dikenlere siçrar, ekin demetleri, tarladaki ekin ya da tarla yanarsa, yangin çikaran kisi zarari ödeyecektir.
\par 7 "Biri komsusuna saklasin diye parasini ya da esyasini emanet eder ve bunlar komsusunun evinden çalinirsa, hirsiz yakalandiginda iki katini ödemelidir.
\par 8 Ama hirsiz yakalanmazsa, komsusunun esyasina el uzatip uzatmadiginin anlasilmasi için ev sahibi yargiç*fu* huzuruna çikmalidir.
\par 9 Emanete ihanet edilen konularda, öküz, esek, koyun, giysi, herhangi bir kayip esya için 'Bu benimdir' diyen her iki taraf sorunu yargicin huzuruna getirmelidir. Yargicin suçlu buldugu kisi komsusuna iki kat ödeyecektir.
\par 10 "Bir adam komsusuna korusun diye esek, öküz, koyun ya da herhangi bir hayvan emanet ettiginde, hayvan ölür, sakatlanir ya da kimse görmeden çalinirsa,
\par 11 komsusu adamin malina el uzatmadigina iliskin RAB'bin huzurunda ant içmelidir. Mal sahibi bunu kabul edecek ve komsusu bir sey ödemeyecektir.
\par 12 Ama mal gerçekten ondan çalinmissa, karsiligi sahibine ödenmelidir.
\par 13 Emanet hayvan parçalanmissa, adam parçalarini kanit olarak göstermelidir. Parçalanan hayvan için bir sey ödemeyecektir.
\par 14 "Biri komsusundan bir hayvan ödünç alir, sahibi yokken hayvan sakatlanir ya da ölürse, karsiligini ödemelidir.
\par 15 Ama sahibi hayvanla birlikteyse, ödünç alan karsiligini ödemeyecektir. Hayvan kiralanmissa, kayip ödenen kiraya sayilmalidir."
\par 16 "Eger biri nisanli olmayan bir kizi aldatip onunla yatarsa, baslik parasini ödemeli ve onunla evlenmelidir.
\par 17 Babasi kizini ona vermeyi reddederse, adam normal baslik parasi neyse onu ödemelidir.
\par 18 "Büyücü kadini yasatmayacaksiniz.
\par 19 "Hayvanlarla cinsel iliski kuran herkes öldürülecektir.
\par 20 "RAB'den baska bir ilaha kurban kesen ölüm cezasina çarptirilacaktir.
\par 21 "Yabanciya haksizlik ve baski yapmayacaksiniz. Çünkü siz de Misir'da yabanciydiniz.
\par 22 "Dul ve öksüz hakki yemeyeceksiniz.
\par 23 Yerseniz, bana feryat ettiklerinde onlari kesinlikle isitirim.
\par 24 Öfkem alevlenir, sizi kiliçtan geçirtirim. Kadinlariniz dul, çocuklariniz öksüz kalir.
\par 25 "Halkima, aranizda yasayan bir yoksula ödünç para verirseniz, ona tefeci gibi davranmayacaksiniz. Üzerine faiz eklemeyeceksiniz.
\par 26 Komsunuzun abasini rehin alirsaniz, gün batmadan geri vereceksiniz.
\par 27 Çünkü tek örtüsü abasidir, ancak onunla örtünebilir. Onsuz nasil yatar? Bana feryat ederse isitecegim, çünkü ben iyilikseverim.
\par 28 "Tanri'ya sövmeyeceksiniz. Halkinizin önderine lanet etmeyeceksiniz.
\par 29 "Ürününüzü ve siranizi sunmakta gecikmeyeceksiniz. Ilk dogan ogullarinizi bana vereceksiniz.
\par 30 Öküzlerinize, davarlariniza da ayni seyi yapacaksiniz. Yedi gün analariyla kalacaklar, sekizinci gün onlari bana vereceksiniz.
\par 31 "Benim kutsal halkim olacaksiniz. Bunun içindir ki, kirda parçalanmis hayvanlarin etini yemeyecek, köpeklerin önüne atacaksiniz."

\chapter{23}

\par 1 "Yalan haber tasimayacaksiniz. Haksiz yere taniklik ederek kötü kisiye yan çikmayacaksiniz.
\par 2 "Kötülük yapan kalabaligi izlemeyeceksiniz. Bir davada çogunluktan yana konusarak adaleti saptirmayacaksiniz.
\par 3 Durusmada yoksulu kayirmayacaksiniz.
\par 4 "Düsmaninizin yolunu sasirmis öküzüne ya da esegine rastlarsaniz, onu kendisine geri götüreceksiniz.
\par 5 Sizden nefret eden kisinin esegini yük altinda çökmüs görürseniz, kendi haline birakip gitmeyecek, ona yardimci olacaksiniz.
\par 6 "Durusmada yoksula karsi adaleti saptirmayacaksiniz.
\par 7 Yalandan uzak duracak, suçsuz ve dogru kisiyi öldürmeyeceksiniz. Çünkü ben kötü kisiyi aklamam.
\par 8 "Rüsvet almayacaksiniz. Çünkü rüsvet göreni kör eder, hakliyi haksiz çikarir.
\par 9 "Yabanciya baski yapmayacaksiniz. Yabanciligin ne oldugunu bilirsiniz. Çünkü siz de Misir'da yabanciydiniz.
\par 10 "Topraginizi alti yil ekecek, ürününü toplayacaksiniz.
\par 11 Ama yedinci yil nadasa birakacaksiniz; öyle ki, halkinizin arasindaki yoksullar yiyecek bulabilsin, onlardan artakalani da yabanil hayvanlar yesin. Baginiza ve zeytinliginize de ayni seyi yapin.
\par 12 "Alti gün çalisacak, yedinci gün dinleneceksiniz. Böylece hem öküzünüz, eseginiz dinlenir, hem de kadin kölenizin ogullari ve yabancilar rahat eder.
\par 13 "Söyledigim her seyi yerine getirin. Baska ilahlarin adini anmayin, agziniza almayin."
\par 14 "Yilda üç kez bana bayram yapacaksiniz.
\par 15 Size buyurdugum gibi, Aviv ayinin* belirli günlerinde yedi gün mayasiz ekmek yiyerek Mayasiz Ekmek Bayrami'ni* kutlayacaksiniz. Çünkü Misir'dan o ay çiktiniz. "Kimse huzuruma eli bos çikmasin.
\par 16 "Tarlaya ektiginiz ürünleri biçtiginizde ilk ürünlerle Hasat Bayrami'ni* kutlayacaksiniz. "Yil sonunda tarladan ürünlerinizi topladiginizda Ürün Devsirme Bayrami'ni* kutlayacaksiniz.
\par 17 "Bütün erkekleriniz yilda üç kez ben Egemen RAB'bin huzuruna çikacaklar.
\par 18 "Evinizde maya bulundugu sürece bana kurban kesmeyeceksiniz. "Bayramda bana kurban edilen hayvanin yagi sabaha birakilmamali.
\par 19 "Topraginizin seçme ilk ürünlerini Tanriniz RAB'bin Tapinagi'na getireceksiniz. "Oglagi anasinin sütünde haslamayacaksiniz."
\par 20 "Yolda sizi korumasi, hazirladigim yere götürmesi için önünüzden bir melek gönderiyorum.
\par 21 Ona dikkat edin, sözünü dinleyin, baskaldirmayin. Çünkü beni temsil ettigi için baskaldirinizi bagislamaz.
\par 22 Ama onun sözünü dikkatle dinler, bütün söylediklerimi yerine getirirseniz, düsmanlariniza düsman, hasimlariniza hasim olacagim.
\par 23 Melegim önünüzden gidecek, sizi Amor, Hitit*, Periz, Kenan, Hiv ve Yevus topraklarina götürecek. Onlari yok edecegim.
\par 24 Onlarin ilahlari önünde egilmeyecek, tapinmayacaksiniz; törelerini izlemeyeceksiniz. Tersine, ilahlarini yok edecek, dikili taslarini büsbütün parçalayacaksiniz.
\par 25 Tanriniz RAB'be tapacaksiniz. Ekmeginizi, suyunuzu bereketli kilacak, aranizdaki hastaliklari yok edecegim.
\par 26 Ülkenizde kisir ve çocuk düsüren kadin olmayacak. Size uzun ömür verecegim.
\par 27 "Dehsetimi önünüzden gönderecek, karsilasacaginiz bütün halklari saskina çevirecegim. Düsmanlariniz önünüzden kaçacak.
\par 28 Hivliler'i, Kenanlilar'i, Hititler'i önünüzden kovmalari için önünüzsira esekarisi gönderecegim.
\par 29 Ama onlari bir yil içinde kovmayacagim. Yoksa ülke viran olur, yabanil hayvanlar çogaldikça çogalir, sayilari sizi asar.
\par 30 Siz çogalincaya, topragi yurt edininceye dek onlari azar azar kovacagim.
\par 31 "Sinirlarinizi Kizildeniz'den* Filist Denizi'ne, çölden Firat Irmagi'na kadar genisletecegim. Ülke halkini elinize teslim edecegim. Onlari önünüzden kovacaksiniz.
\par 32 Onlarla ya da ilahlariyla antlasma yapmayacaksiniz.
\par 33 Onlari ülkenizde barindirmayacaksiniz. Yoksa bana karsi günah islemenize neden olurlar. Ilahlarina taparsaniz, size tuzak olur."

\chapter{24}

\par 1 RAB Musa'ya, "Sen, Harun, Nadav, Avihu ve Israil ileri gelenlerinden yetmis kisi bana gelin" dedi, "Bana uzaktan tapin.
\par 2 Yalniz sen bana yaklasacaksin. Ötekiler yaklasmamali. Halk seninle daga çikmamali."
\par 3 Musa gidip RAB'bin bütün buyruklarini, ilkelerini halka anlatti. Herkes bir agizdan, "RAB'bin her söyledigini yapacagiz" diye karsilik verdi.
\par 4 Musa RAB'bin bütün buyruklarini yazdi. Sabah erkenden kalkip dagin eteginde bir sunak kurdu, Israil'in on iki oymagini simgeleyen on iki tas sütun dikti.
\par 5 Sonra Israilli gençleri gönderdi. Onlar da RAB'be yakmalik sunular* sundular, esenlik kurbanlari olarak bogalar kestiler.
\par 6 Musa kanin yarisini legenlere doldurdu, öbür yarisini sunagin üzerine döktü.
\par 7 Sonra antlasma kitabini alip halka okudu. Halk, "RAB'bin her söyledigini yapacagiz, O'nu dinleyecegiz" dedi.
\par 8 Musa legenlerdeki kani halkin üzerine serpti ve, "Bütün bu sözler uyarinca, RAB'bin sizinle yaptigi antlasmanin kani budur" dedi.
\par 9 Sonra Musa, Harun, Nadav, Avihu ve Israil ileri gelenlerinden yetmis kisi daga çikarak
\par 10 Israil'in Tanrisi'ni gördüler. Tanri'nin ayaklari altinda laciverttasini andiran bir döseme vardi. Gök gibi duruydu.
\par 11 Tanri Israil soylularina zarar vermedi. Tanri'yi gördüler, sonra yiyip içtiler.
\par 12 RAB Musa'ya, "Daga, yanima gel" dedi, "Burada bekle, halkin ögrenmesi için üzerine yasalarla buyruklari yazdigim tas levhalari sana verecegim."
\par 13 Musa'yla yardimcisi Yesu hazirlandilar. Musa Tanri Dagi'na çikarken,
\par 14 Israil ileri gelenlerine, "Geri dönünceye kadar bizi burada bekleyin" dedi, "Harun'la Hur aranizda; kimin sorunu olursa onlara basvursun."
\par 15 Musa daga çikinca, bulut dagi kapladi.
\par 16 RAB'bin görkemi Sina Dagi'nin üzerine indi. Bulut dagi alti gün örttü. Yedinci gün RAB bulutun içinden Musa'ya seslendi.
\par 17 RAB'bin görkemi Israilliler'e dagin dorugunda yakici bir ates gibi görünüyordu.
\par 18 Musa bulutun içinden daga çikti. Kirk gün kirk gece dagda kaldi.

\chapter{25}

\par 1 RAB Musa'ya söyle dedi:
\par 2 "Israilliler'e söyle, bana armagan getirsinler. Gönülden veren herkesin armaganini alin.
\par 3 Onlardan alacaginiz armaganlar sunlardir: Altin, gümüs, tunç*;
\par 4 lacivert, mor, kirmizi iplik; ince keten, keçi kili,
\par 5 deri, kirmizi boyali koç derisi, akasya agaci,
\par 6 kandil için zeytinyagi, mesh yagiyla güzel kokulu buhur için baharat,
\par 7 baskâhinin efoduyla* gögüslügü* için oniks ve öbür kakma taslar.
\par 8 "Aralarinda yasamam için bana kutsal bir yer yapsinlar.
\par 9 Konutu ve esyalarini sana gösterecegim örnege tipatip uygun yapin."
\par 10 "Akasya agacindan bir sandik yapsinlar. Boyu iki buçuk, eni ve yüksekligi birer buçuk arsin olsun.
\par 11 Içini de disini da saf altinla kapla. Çevresine altin pervaz yap.
\par 12 Dört altin halka döküp dört ayagina tak. Ikisi bir yanda, ikisi öbür yanda olacak.
\par 13 Akasya agacindan siriklar yapip altinla kapla.
\par 14 Sandigin tasinmasi için siriklari yanlardaki halkalara geçir.
\par 15 Siriklar sandigin halkalarinda kalacak, çikarilmayacak.
\par 16 Antlasmanin tas levhalarini sana verecegim. Onlari sandigin içine koy.
\par 17 "Saf altindan bir Bagislanma Kapagi yap. Boyu iki buçuk, eni bir buçuk arsin olacak.
\par 18 Kapagin iki kenarina dövme altindan birer Keruv* yap.
\par 19 Keruvlar'dan birini bir kenara, öbürünü öteki kenara, kapakla tek parça halinde yap.
\par 20 Keruvlar yukari dogru açik kanatlariyla kapagi örtecek. Yüzleri birbirine dönük olacak ve kapaga bakacak.
\par 21 Kapagi sandigin üzerine, sana verecegim tas levhalari ise sandigin içine koy.
\par 22 Seninle orada, Levha Sandigi'nin üstündeki Keruvlar arasinda, Bagislanma Kapagi'nin üzerinde görüsecegim ve Israilliler için sana buyruklar verecegim."
\par 23 "Akasya agacindan bir masa yap. Boyu iki, eni bir, yüksekligi bir buçuk arsin olacak.
\par 24 Masayi saf altinla kapla. Çevresine altin pervaz yap.
\par 25 Pervazin çevresine dört parmak eninde bir kenarlik yaparak altin pervazla çevir.
\par 26 Masa için dört altin halka yap, dört ayak üzerindeki dört köseye yerlestir.
\par 27 Masanin tasinmasi için siriklarin içinden geçecegi halkalar kenarliga yakin olmali.
\par 28 Siriklari akasya agacindan yap, altinla kapla. Masa onlarla tasinacak.
\par 29 Masa için saf altindan tabaklar, sahanlar, dökmelik sunu testileri, taslari yap.
\par 30 Ekmekleri sürekli olarak huzuruma, masanin üzerine koyacaksin."
\par 31 "Saf altindan bir kandillik yap. Ayagi, gövdesi dövme altin olsun. Çanak, tomurcuk ve çiçek motifleri kendinden olsun.
\par 32 Kandillik üç kolu bir yanda, üç kolu öteki yanda olmak üzere alti kollu olacak.
\par 33 Her kolda badem çiçegini andiran üç çanak, tomurcuk ve çiçek motifi bulunacak. Alti kol da ayni olacak.
\par 34 Kandilligin gövdesinde badem çiçegini andiran dört çanak, tomurcuk ve çiçek motifi olacak.
\par 35 Kandillikten yükselen ilk iki kolun, ikinci iki kolun, üçüncü iki kolun altinda kendinden birer tomurcuk bulunacak. Toplam alti kol olacak.
\par 36 Tomurcuklari, kollari tek parça olan kandillik saf dövme altindan olacak.
\par 37 "Kandillik için yedi kandil yap; kandiller karsisini aydinlatacak biçimde yerlestirilsin.
\par 38 Fitil masalari, tablalari saf altindan olacak.
\par 39 Bütün takimlari dahil kandillige bir talant saf altin harcanacak.
\par 40 Her seyi sana dagda gösterilen örnege göre yapmaya dikkat et."

\chapter{26}

\par 1 "Tanri'nin Konutu'nu on perdeden yap. Perdeler lacivert, mor, kirmizi iplikle özenle dokunmus ince ketenden olsun, üzeri Keruvlar'la* ustaca süslensin.
\par 2 Her perdenin boyu yirmi sekiz, eni dört arsin olmali. Bütün perdeler ayni ölçüde olacak.
\par 3 Perdeler beser beser birbirine eklenerek iki takim perde yapilacak.
\par 4 Birinci takimin kenarina lacivert ilmekler aç. Öbür takimin kenarina da ayni seyi yap.
\par 5 Birinci takimin ilk perdesiyle ikinci takimin son perdesine elliser ilmek aç. Ilmekler birbirine karsi olmali.
\par 6 Elli altin kopça yap, perdeleri kopçalayarak çadiri birlestir. Böylece konut tek parça haline gelecek.
\par 7 "Konutun üstünü kaplayacak çadir için keçi kilindan on bir perde yap.
\par 8 Her perdenin boyu otuz, eni dört arsin olacak. On bir perde de ayni ölçüde olmali.
\par 9 Bes perde birbirine, alti perde birbirine birlestirilecek. Altinci perdeyi çadirin önünde katla.
\par 10 Her iki perde takiminin kenarlarina elliser ilmek aç.
\par 11 Elli tunç* kopça yap, kopçalari ilmeklere geçir ki, çadir tek parça haline gelsin.
\par 12 Çadirin perdelerinden artan yarim perde konutun arkasindan sarkacak.
\par 13 Perdelerin uzun kenarlarindan artan kumas çadirin yanlarindan birer arsin sarkarak konutu örtecek.
\par 14 Çadir için kirmizi boyali koç derilerinden bir örtü, onun üstüne de deriden baska bir örtü yap.
\par 15 "Konut için akasya agacindan dikine çerçeveler yap.
\par 16 Her çerçevenin boyu on*fk*, eni bir buçuk arsin olacak.
\par 17 Çerçevelerin birbirine uyan iki paralel çikintisi olacak. Konutun bütün çerçevelerini ayni biçimde yapacaksin.
\par 18 Konutun güneyi için yirmi çerçeve yap.
\par 19 Her çerçevenin altinda iki çikinti için birer taban olmak üzere, yirmi çerçevenin altinda kirk gümüs taban yap.
\par 20 Konutun öbür yani, yani kuzeyi için de yirmi çerçeve ve her çerçevenin altinda iki taban olmak üzere kirk gümüs taban yap.
\par 21 (#26:20)
\par 22 Konutun batiya bakacak arka tarafi için alti çerçeve yap.
\par 23 Arkada konutun köseleri için iki çerçeve yap.
\par 24 Bu köse çerçevelerinin alt tarafi ayri kalacak, üst tarafi ise birinci halkayla birlestirilecek. Iki köseyi olusturan iki çerçeve ayni biçimde olacak.
\par 25 Böylece sekiz çerçeve ve her çerçevenin altinda iki taban olmak üzere on alti gümüs taban olacak.
\par 26 "Konutun bir yanindaki çerçeveler için bes, öbür yanindaki çerçeveler için bes, batiya bakan arka tarafindaki çerçeveler için de bes olmak üzere akasya agacindan kirisler yap.
\par 27 (#26:26)
\par 28 Çerçevelerin ortasindaki kiris çadirin bir ucundan öbür ucuna geçecek.
\par 29 Çerçeveleri ve kirisleri altinla kapla, kirislerin geçecegi halkalari da altindan yap.
\par 30 "Konutu dagda sana gösterilen plana göre yap.
\par 31 "Lacivert, mor, kirmizi iplikle özenle dokunmus ince ketenden bir perde yap; üzerini Keruvlar'la ustaca süsle.
\par 32 Dört gümüs taban üstünde duran akasya agacindan altin kapli dört direk üzerine as. Çengelleri altin olacak.
\par 33 Perdeyi kopçalarin altina asip Levha Sandigi'ni perdenin arkasina koy. Perde Kutsal Yer'le* En Kutsal Yer'i* birbirinden ayiracak.
\par 34 Bagislanma Kapagi'ni En Kutsal Yer'de bulunan Levha Sandigi'nin üzerine koy.
\par 35 Masayi perdenin öbür tarafina, konutun kuzeye bakan yanina yerlestir; kandilligi masanin karsisina, konutun güney tarafina koy.
\par 36 "Çadirin giris bölümüne lacivert, mor, kirmizi iplikle özenle dokunmus ince ketenden nakisli bir perde yap.
\par 37 37 Perdeyi asmak için akasya agacindan bes direk yap, altinla kapla. Çengelleri de altin olacak. Direkler için tunçtan bes taban dök."

\chapter{27}

\par 1 "Sunagi akasya agacindan kare biçiminde yap. Eni ve boyu beser arsin, yüksekligi üç arsin olacak.
\par 2 Dört üst kösesine kendinden boynuzlar yaparak hepsini tunçla* kapla.
\par 3 Sunak için yag ve kül kovalari, kürekler, çanaklar, büyük çatallar, ates kaplari yap. Tümü tunç olacak.
\par 4 Ag biçiminde tunç bir izgara da yap, dört kösesine birer tunç halka tak.
\par 5 Izgarayi sunagin kenarinin altina koy. Öyle ki, asagi dogru sunagin yarisina yetissin.
\par 6 Sunak için akasya agacindan siriklar yap, tunçla kapla.
\par 7 Siriklar halkalara geçirilecek ve sunak tasinirken iki yaninda olacak.
\par 8 Sunagi tahtadan, içi bos yapacaksin. Tipki dagda sana gösterildigi gibi olacak."
\par 9 "Konuta bir avlu yap. Avlunun güney tarafi için yüz arsin boyunda, özenle dokunmus ince keten perdeler yapacaksin.
\par 10 Perdeler için yirmi direk yapilacak; direklerin tabanlari tunç*, çengelleri ve çengel çemberleri gümüs olacak.
\par 11 Kuzey tarafi için yüz arsin boyunda perdeler, yirmi direk, direkler için yirmi tunç taban yapilacak. Direklerin çengelleriyle çemberleri gümüsten olacak.
\par 12 "Avlunun bati tarafi için elli arsin boyunda perde, on direk, on taban yapilacak.
\par 13 Doguya bakan tarafta avlunun eni elli arsin olacak.
\par 14 Girisin bir tarafinda on bes arsin*fp* boyunda perde, üç direk ve üç taban olacak.
\par 15 Girisin öbür tarafinda da on bes arsin boyunda perde, üç direk ve üç taban olacak.
\par 16 "Avlunun girisinde lacivert, mor, kirmizi iplikle, özenle dokunmus ince ketenden yirmi arsin boyunda nakisli bir perde olacak. Dört diregi ve dört tabani bulunacak.
\par 17 Avlunun çevresindeki bütün direkler gümüs çemberlerle donatilacak. Çengelleri gümüs, tabanlari tunç olacak.
\par 18 Avlunun boyu yüz, eni elli, çevresindeki perdelerin yüksekligi bes arsin olacak. Perdeleri özenle dokunmus ince ketenden, tabanlari tunçtan olacak.
\par 19 Konutta her türlü hizmet için kullanilacak bütün aletler, konutun ve avlunun bütün kaziklari da tunçtan olacak."
\par 20 "Israil halkina buyruk ver, kandilin sürekli yanip isik vermesi için saf sikma zeytinyagi getirsinler.
\par 21 Harun'la ogullari kandilleri benim huzurumda, Bulusma Çadiri'nda, Levha Sandigi'nin önündeki perdenin disinda, aksamdan sabaha kadar yanar tutacaklar. Israilliler için kusaklar boyunca sürekli bir kural olacak bu."

\chapter{28}

\par 1 "Bana kâhinlik etmeleri için Israilliler arasindan agabeyin Harun'u, ogullari Nadav, Avihu, Elazar ve Itamar'i yanina al.
\par 2 Agabeyin Harun'a görkem ve sayginlik kazandirmak için kutsal giysiler yap.
\par 3 Bilgelik verdigim becerikli adamlara söyle, Harun'a giysi yapsinlar. Öyle ki, bana kâhinlik etmek için kutsal kilinmis olsun.
\par 4 Yapacaklari giysiler sunlardir: Gögüslük*, efod*, kaftan, nakisli mintan, sarik, kusak. Bana kâhinlik etmeleri için agabeyin Harun'a ve ogullarina bu kutsal giysileri yapacaklar.
\par 5 Altin sirma, lacivert, mor, kirmizi iplik, ince keten kullanacaklar."
\par 6 "Efodu* altin sirmayla, lacivert, mor, kirmizi iplikle, özenle dokunmus ince ketenden ustaca yapacaklar.
\par 7 Baglanabilmesi için iki kösesine takilmis ikiser omuzlugu olacak.
\par 8 Efodun üzerinde efod gibi ustaca dokunmus bir serit olacak. Efodun bir parçasi gibi lacivert, mor, kirmizi iplikle, altin sirmayla, özenle dokunmus ince ketenden olacak.
\par 9 Iki oniks tasi alacak, Israilogullari'nin adlarini, dogus sirasina göre altisini birinin, altisini ötekinin üzerine oyacaksin.
\par 10 (#28:9)
\par 11 Israilogullari'nin adlarini bu iki tasin üzerine usta oymacilarin mühür oydugu gibi oyacaksin. Taslari altin yuvalar içine koyduktan sonra Israilliler'in anilmasi için efodun omuzluklarina tak. Harun, anilmalari için onlarin adlarini RAB'bin önünde iki omuzunda tasiyacak.
\par 12 (#28:11)
\par 13 Altin yuvalar ve saf altindan iki zincir yap. Zincirleri örme kordon gibi yapip yuvalara yerlestir."
\par 14 (#28:13)
\par 15 "Usta isi bir karar gögüslügü* yap. Onu da efod* gibi, altin sirmayla, lacivert, mor, kirmizi iplikle, özenle dokunmus ince ketenden yap.
\par 16 Dört köse, eni ve boyu birer karis olacak; ikiye katlanacak.
\par 17 Üzerine dört sira tas yuvasi kak. Birinci sirada yakut, topaz, zümrüt;
\par 18 ikinci sirada firuze, laciverttasi, aytasi;
\par 19 üçüncü sirada gökyakut, agat, ametist;
\par 20 dördüncü sirada sari yakut, oniks ve yesim olacak. Taslar altin yuvalara kakilacak.
\par 21 On iki tas olacak. Üzerlerine mühür oyar gibi Israilogullari'nin adlari bir bir oyulacak. Bu taslar Israil'in on iki oymagini simgeleyecek.
\par 22 "Gögüslük için saf altindan örme zincirler yap.
\par 23 Iki altin halka yap, gögüslügün üst iki kösesine birer halka koy.
\par 24 Iki örme altin zinciri gögüslügün köselerindeki halkalara tak.
\par 25 Zincirlerin öteki iki ucunu iki yuvanin üzerinden geçirerek efodun ön tarafina, omuzluklarin üzerine bagla.
\par 26 Iki altin halka daha yap; her birini gögüslügün alt iki kösesine, efoda bitisik iç kenarina tak.
\par 27 Iki altin halka daha yap; efodun önündeki omuzluklara alttan, dikise yakin, ustaca dokunmus seridin yukarisina tak.
\par 28 Gögüslügün halkalariyla efodun halkalari lacivert kordonla birbirine baglanacak. Öyle ki, gögüslük efodun ustaca dokunmus seridinin yukarisinda kalsin ve efoddan ayrilmasin.
\par 29 "Harun Kutsal Yer'e* girerken, Israilogullari'nin adlarinin yazili oldugu karar gögüslügünü yüreginin üzerinde tasiyacak. Öyle ki, ben, RAB halkimi sürekli animsayayim.
\par 30 Urim'le Tummim'i* karar gögüslügünün içine koy; öyle ki, Harun ne zaman huzuruma çiksa yüreginin üzerinde olsunlar. Böylece Harun Israilogullari'nin karar vermek için kullandiklari Urim'le Tummim'i RAB'bin huzurunda sürekli yüreginin üzerinde tasiyacak."
\par 31 "Efodun* altina giyilen kaftani salt lacivert iplikten yap.
\par 32 Ortasinda bas geçecek kadar bir bosluk birak. Yirtilmamasi için boslugun kenarlarini yaka gibi dokuyarak çevir.
\par 33 Kaftanin kenarini çepeçevre lacivert, mor, kirmizi iplikten nar motifleriyle beze, aralarina altin çingiraklar tak.
\par 34 Etegin ucu bir altin çingirak, bir nar, bir altin çingirak, bir nar olmak üzere çepeçevre kaplanacak.
\par 35 Harun hizmet ederken bu kaftani giyecek. En Kutsal Yer'e*, huzuruma girip çikarken duyulan çingirak sesi onun ölmedigini gösterecek.
\par 36 "Saf altindan bir levha yap ve üzerine mühür oyar gibi 'RAB'be adanmistir' sözünü oy;
\par 37 lacivert bir kordonla sarigin ön tarafina bagla.
\par 38 Harun onu alninda tasiyacak. Israilliler kutsal bagislarini getirirken suç islemislerse, suçlarini Harun tasiyacak; onlar önümde kabul görsün diye levha sürekli Harun'un alninda bulunacak.
\par 39 "Ince ketenden islemeli bir mintan doku, ince ketenden bir sarik, bir de nakisli kusak yap.
\par 40 "Harun'un ogullarina mintanlar, kusaklar, görkem ve sayginlik kazandiracak basliklar yap.
\par 41 Bu giysileri agabeyin Harun'a ve ogullarina giydir; sonra bana kâhinlik etmeleri için onlari meshedip* ata ve kutsal kil.
\par 42 "Edep yerlerini örtmek için onlara keten donlar yap. Boyu belden uyluga kadar olacak.
\par 43 Harun'la ogullari Bulusma Çadiri'na girdiklerinde ya da Kutsal Yer'de* hizmet etmek üzere sunaga yaklastiklarinda, suç isleyip ölmemek için bu donlari giyecekler. Harun ve soyundan gelenler için sürekli bir kural olacak bu."

\chapter{29}

\par 1 "Bana kâhinlik edebilmeleri için, Harun'la ogullarini kutsal kilmak üzere sunlari yap: Bir boga ile iki kusursuz koç al.
\par 2 Ince bugday unundan mayasiz ekmek, zeytinyagiyla yogrulmus mayasiz pideler, üzerine yag sürülmüs mayasiz yufkalar yap.
\par 3 Bunlari bir sepete koyup boga ve iki koçla birlikte bana getir.
\par 4 Harun'la ogullarini Bulusma Çadiri'nin giris bölümüne getirip yika.
\par 5 Giysileri al; mintani, efodun* altina giyilen kaftani, efodu ve gögüslügü* Harun'a giydir. Efodun ustaca dokunmus seridini bagla.
\par 6 Basina sarigi sar, üzerine de kutsal taci koy.
\par 7 Sonra mesh yagini al, basina dökerek onu meshet*.
\par 8 Harun'un ogullarini öne çikarip onlara mintan giydir.
\par 9 Bellerine kusak bagla, baslarina baslik koy. Kalici bir kural olarak kâhinlik onlarin isi olacak. Böylece Harun'la ogullarini atamis olacaksin.
\par 10 "Bogayi Bulusma Çadiri'nin önüne getir, Harun'la ogullari ellerini boganin basina koysunlar.
\par 11 Bogayi huzurumda, Bulusma Çadiri'nin giris bölümünde keseceksin.
\par 12 Kanini parmaginla sunagin boynuzlarina sür, artan kani sunagin dibine dök.
\par 13 Hayvanin bagirsak ve iskembe yaglarini, karaciger perdesini, böbreklerini ve böbrek yaglarini sunagin üzerinde yakacaksin.
\par 14 Etini, derisini, gübresini de ordugahin disinda yak. Bu günah sunusudur*.
\par 15 "Bir koç getir, Harun'la ogullari ellerini koçun basina koysunlar.
\par 16 Koçu sen kes. Kanini sunagin her yanina dök.
\par 17 Koçu parçalara ayirip bagirsaklarini, iskembesini, ayaklarini yika, basla öteki parçalarin yanina koy.
\par 18 Sonra koçun tümünü sunagin üzerinde yak. Bu RAB'be sunulan yakmalik sunu*, RAB'bi hosnut eden koku, O'nun için yakilan sunudur.
\par 19 "Öteki koçu getir, Harun'la ogullari ellerini koçun basina koysunlar.
\par 20 Koçu sen kes. Kanini Harun'la ogullarinin sag kulak memelerine, sag el ve ayaklarinin bas parmaklarina sür. Artan kani sunagin her yanina dök.
\par 21 Sunagin üzerindeki kani ve mesh yagini Harun'la ogullarinin ve giysilerinin üzerine serp. Böylece Harun'la ogullari ve giysileri kutsal kilinmis olacak.
\par 22 "Koçun yagini, kuyruk yagini, bagirsak ve iskembe yaglarini, karaciger perdesini, böbreklerini, böbrek yaglarini ve sag budunu al. -Çünkü bu, biri göreve atanirken kesilen koçtur.-
\par 23 Huzurumdaki mayasiz ekmek sepetinden bir somun, yagli pide ve yufka al,
\par 24 hepsini Harun'la ogullarinin eline ver. Bunlari benim huzurumda sallamalik sunu olarak salla,
\par 25 sonra ellerinden alip sunakta yakmalik sunuyla birlikte beni hosnut eden koku olarak yak. Bu, RAB için yakilan sunudur.
\par 26 "Harun'un atanmasi için sunulacak koçun dösünü huzurumda sallamalik sunu olarak salla. O dös senin payin olacak.
\par 27 Harun'la ogullarinin atanmasi için kesilen koçun sallanmis olan dösüyle bagis olarak sunulan budunu bana ayir.
\par 28 Israilliler bunlari sürekli Harun'la ogullarinin payina ayiracak. Bu, Israilliler'in RAB'be sundugu esenlik kurbanlarindan biridir.
\par 29 "Harun'un kutsal giysileri, kendinden sonra ogullarina kalacak. Meshedilip atanirlarken bu giysileri giyecekler.
\par 30 Harun'un yerine kâhin olan oglu, Kutsal Yer'de* hizmet etmek üzere Bulusma Çadiri'na girdiginde yedi gün bu giysileri giyecek.
\par 31 "Harun'la ogullari göreve atanirken kesilen koçun etini kutsal bir yerde haslayacaksin. Haslanan eti ve sepetteki ekmegi Bulusma Çadiri'nin giris bölümünde yiyecekler.
\par 32 (#29:31)
\par 33 Atanip kutsal kilinmalari için günahlari bagislatan bu sunulari yalniz onlar yiyebilir. Yabanci biri yiyemez, çünkü bu sunular kutsaldir.
\par 34 Atanmalari için kesilen kurbanin etinden ya da ekmekten sabaha artan olursa, yakacaksin. Bunlar yenmeyecek, çünkü kutsaldir.
\par 35 "Harun'la ogullari için sana buyurduklarimin hepsini yap. Atanmalari yedi gün sürecek.
\par 36 Günah bagislatmak için günah sunusu olarak her gün bir boga sunacaksin. Sunagi arindirmak için günah sunusu sun, kutsal kilmak için de meshet.
\par 37 Yedi gün sunagi arindirarak kutsal kilacaksin. Böylece sunak çok kutsal olacak. Ona dokunan her sey de kutsal sayilacaktir."
\par 38 "Düzenli olarak her gün sunagin üzerinde bir yasinda iki erkek kuzu sunacaksiniz.
\par 39 Kuzunun birini sabah, öbürünü aksamüstü sunun.
\par 40 Kuzuyla birlikte dörtte bir hin sikma zeytinyagiyla yogrulmus onda bir efa ince un ve dökmelik sunu olarak dörtte bir hin sarap sunacaksiniz.
\par 41 Öbür kuzuyu aksamüstü, beni hosnut eden koku, yakilan sunu olarak, sabahki gibi tahil sunusu* ve dökmelik sunuyla birlikte bana sunacaksiniz.
\par 42 "Bu yakmalik sunu Bulusma Çadiri'nin giris bölümünde, RAB'bin huzurunda, kusaklar boyu sürekli sunulacaktir. Musa'yla konusmak için Israil halkiyla orada bulusacagim.
\par 43 Israilliler'le bulusurken çadir görkemimle kutsal kilinacak.
\par 44 "Bulusma Çadiri'ni ve sunagi kutsal kilacak, Harun'la ogullarini bana kâhinlik etmeleri için görevlendirecegim.
\par 45 Israilliler arasinda yasayacak, onlarin Tanrisi olacagim.
\par 46 Anlayacaklar ki, aralarinda yasamak için onlari Misir'dan çikaran Tanrilari RAB benim. Tanrilari RAB benim."

\chapter{30}

\par 1 "Üzerinde buhur yakmak için akasya agacindan bir sunak yap.
\par 2 Kare biçiminde, boyu ve eni birer arsin, yüksekligi iki arsin, boynuzlari kendinden olacak.
\par 3 Üstünü, yanlarini, boynuzlarini saf altinla kapla. Çevresine altin pervaz yap.
\par 4 Iki yandaki pervazin altina iki altin halka yap. Bunlar sunagin tasinmasi için siriklarin geçmesine yarayacak.
\par 5 Siriklari akasya agacindan yap ve altinla kapla.
\par 6 Sunagi Levha Sandigi'nin karsisindaki perdenin, sandigin üzerindeki Bagislanma Kapagi'nin önüne, seninle bulusacagim yere koy.
\par 7 "Harun her sabah kandillerin bakimini yaparken sunagin üzerinde güzel kokulu buhur yakacak.
\par 8 Aksamüstü kandilleri yakarken yine buhur yakacak. Böylece huzurumda kusaklar boyunca sürekli buhur yanacak.
\par 9 Sunagin üzerinde baska buhur, yakmalik sunu* ya da tahil sunusu* sunmayacaksiniz; üzerine dökmelik sunu dökmeyeceksiniz.
\par 10 Harun yilda bir kez sunagin boynuzlarini arindiracak. Kusaklariniz boyunca yilda bir kez günahlari bagislatmak için sunulan sununun kaniyla sunagi arindiracak. Sunak ben RAB için çok kutsaldir."
\par 11 RAB Musa'ya söyle dedi:
\par 12 "Israilliler'in sayimini yaptigin zaman, herkes canina karsilik bana bedel ödeyecektir. Öyle ki, sayim yapilirken baslarina bela gelmesin.
\par 13 Sayilan herkes armagan olarak bana yarim kutsal yerin sekeli verecektir. -Bir sekel yirmi geradir.-
\par 14 Sayilan yirmi yasindaki ve daha yukari yastaki herkes bana armagan verecektir.
\par 15 Canlarinizin bedeli olarak bu armagani verdiginizde, zengin yarim sekelden fazla, yoksul yarim sekelden eksik vermeyecek.
\par 16 Israilliler'den bedel olarak verilen paralari toplayacak, Bulusma Çadiri'nin hizmetinde kullanacaksin. Bu paralar canlarinizin bedeli olarak ben, RAB'be Israilliler'i hep animsatacak."
\par 17 RAB Musa'ya söyle dedi:
\par 18 "Yikanmak için tunç* bir kazan yap. Ayakligi da tunçtan olacak. Bulusma Çadiri ile sunagin arasina koyup içine su doldur.
\par 19 Harun'la ogullari ellerini, ayaklarini orada yikayacaklar.
\par 20 Bulusma Çadiri'na girmeden ya da RAB için yakilan sunuyu sunarak hizmet etmek üzere sunaga yaklasmadan önce, ölmemek için ellerini, ayaklarini yikamalilar. Harun'la soyunun bütün kusaklari boyunca sürekli bir kural olacak bu."
\par 21 (#30:20)
\par 22 RAB Musa'ya söyle dedi:
\par 23 "Su nadide baharati al: 500 sekel sivi mür*, yarisi kadar, yani 250'ser sekel güzel kokulu tarçin ve kamis,
\par 24 "500 kutsal yerin sekeli hiyarsembe, bir hin de zeytinyagi.
\par 25 Bunlardan itriyatçi ustaligiyla güzel kokulu kutsal bir mesh yagi yap. Ona kutsal mesh yagi denecek.
\par 26 Bulusma Çadiri'ni, Levha Sandigi'ni, masayla takimlarini, kandillikle takimlarini, buhur sunagini, yakmalik sunu* sunagiyla bütün akimlarini, kazani ve kazan ayakligini hep bu yagla meshet*.
\par 27 (#30:26)
\par 28 (#30:26)
\par 29 Onlari kutsal kil ki, çok kutsal olsunlar. Onlara degen her sey kutsal sayilacaktir.
\par 30 "Bana kâhin olmalari için Harun'la ogullarini meshedip kutsal kil.
\par 31 Israilliler'e de ki, 'Kusaklariniz boyunca bu kutsal mesh yagi yalniz benim için kullanilacak.
\par 32 Insan bedenine dökülmeyecek. Ayni reçeteyle benzeri yapilmayacak. O kutsaldir ve sizin için kutsal olacaktir.
\par 33 Onun benzerini yapan ya da kâhin olmayan birinin üzerine döken herkes halkinin arasindan atilacaktir.'"
\par 34 Sonra RAB Musa'ya söyle dedi: "Güzel kokulu baharat -kara günnük, onika, kasni ve saf günnük- al. Hepsi ayni ölçüde olsun.
\par 35 Bir itriyatçi ustaligiyla bunlardan güzel kokulu bir buhur yap. Tuzlanmis, saf ve kutsal olacak.
\par 36 Birazini çok ince döv, Bulusma Çadiri'nda seninle bulusacagim yere, Levha Sandigi'nin önüne koy. Sizin için çok kutsal olacaktir.
\par 37 Ayni reçeteyle kendinize buhur yapmayacaksiniz. Onu RAB için kutsal sayacaksiniz.
\par 38 Kim koklamak için aynisini yaparsa halkinin arasindan atilacaktir."

\chapter{31}

\par 1 RAB Musa'ya söyle dedi:
\par 2 "Bak, Yahuda oymagindan özellikle Hur oglu Uri oglu Besalel'i seçtim.
\par 3 Beceri, anlayis, bilgi ve her türlü ustalik vermek için onu ruhumla doldurdum.
\par 4 Öyle ki, altin, gümüs, tunç* isleyerek ustaca yapitlar üretsin;
\par 5 tas kesme ve kakmada, agaç oymaciliginda, her türlü sanat dalinda çalissin.
\par 6 Ayrica Dan oymagindan Ahisamak oglu Oholiav'i onunla çalismasi için görevlendirdim. Sana buyurdugum islerin hepsini yapabilsinler diye öteki becerikli adamlara üstün yetenek verdim.
\par 7 Bulusma Çadiri'ni, Levha Sandigi'ni, sandigin üzerindeki Bagislanma Kapagi'ni, çadirin bütün takimlarini,
\par 8 masayla takimlarini, saf altin kandillikle takimlarini, buhur sunagini,
\par 9 yakmalik sunu* sunagiyla takimlarini, kazanla kazan ayakligini,
\par 10 dokunmus giysileri -Kâhin Harun'un kutsal giysileriyle ogullarinin kâhin giysilerini-
\par 11 mesh yagini, kutsal yer için güzel kokulu buhuru tam sana buyurdugum gibi yapsinlar."
\par 12 RAB Musa'ya söyle buyurdu:
\par 13 "Israilliler'e de ki, 'Sabat* günlerimi kesinlikle tutmalisiniz. Çünkü o sizinle benim aramda kusaklar boyu sürecek bir belirtidir. Böylece anlayacaksiniz ki, sizi kutsal kilan RAB benim.
\par 14 "'Sabat Günü'nü tutmalisiniz, çünkü sizin için kutsaldir. Kim onun kutsalligini bozarsa, kesinlikle öldürülmeli. O gün çalisan herkes halkinin arasindan atilmali.
\par 15 Alti gün çalisilacak; ama yedinci gün RAB'be adanmis Sabat'tir, dinlenme günüdür. Sabat Günü çalisan herkes kesinlikle öldürülmelidir.
\par 16 Israilliler, sonsuza dek sürecek bir antlasma geregi olarak, Sabat Günü'nü kusaklar boyu kutlamaya özen gösterecekler.
\par 17 Bu, Israilliler'le benim aramda sürekli bir belirti olacaktir. Çünkü ben, RAB yeri gögü alti günde yarattim, yedinci gün ise son verip dinlendim.'"
\par 18 Tanri Sina Dagi'nda Musa'yla konusmasini bitirince, üzerine eliyle antlasma kosullarini yazdigi iki tas levhayi ona verdi.

\chapter{32}

\par 1 Halk Musa'nin dagdan inmedigini, geciktigini görünce, Harun'un çevresine toplandi. Ona, "Kalk, bize öncülük edecek bir ilah yap" dediler, "Bizi Misir'dan çikaran adama, Musa'ya ne oldu bilmiyoruz!"
\par 2 Harun, "Karilarinizin, ogullarinizin, kizlarinizin kulagindaki altin küpeleri çikarip bana getirin" dedi.
\par 3 Herkes kulagindaki küpeyi çikarip Harun'a getirdi.
\par 4 Harun altinlari topladi, oymaci aletiyle buzagi biçiminde dökme bir put yapti. Halk, "Ey Israilliler, sizi Misir'dan çikaran Tanriniz budur!" dedi.
\par 5 Harun bunu görünce, buzaginin önünde bir sunak yapti ve, "Yarin RAB'bin onuruna bayram olacak" diye ilan etti.
\par 6 Ertesi gün halk erkenden kalkip yakmalik sunular* sundu, esenlik sunulari* getirdi. Yiyip içmeye oturdu, sonra kalkip çilginca eglendi.
\par 7 RAB Musa'ya, "Asagi in" dedi, "Misir'dan çikardigin halkin bastan çikti.
\par 8 Buyurdugum yoldan hemen saptilar. Kendilerine dökme bir buzagi yaparak önünde tapindilar, kurban kestiler. 'Ey Israilliler, sizi Misir'dan çikaran ilahiniz budur!' dediler."
\par 9 RAB Musa'ya, "Bu halkin ne inatçi oldugunu biliyorum" dedi,
\par 10 "Simdi bana engel olma, birak öfkem alevlensin, onlari yok edeyim. Sonra seni büyük bir ulus yapacagim."
\par 11 Musa Tanrisi RAB'be yalvardi: "Ya RAB, niçin kendi halkina karsi öfken alevlensin? Onlari Misir'dan büyük kudretinle, güçlü elinle çikardin.
\par 12 Neden Misirlilar, 'Tanri kötü amaçla, daglarda öldürmek, yeryüzünden silmek için onlari Misir'dan çikardi' desinler? Öfkelenme, vazgeç halkina yapacagin kötülükten.
\par 13 Kullarin Ibrahim'i, Ishak'i, Israil'i animsa. Onlara kendi üzerine ant içtin, 'Soyunuzu gökteki yildizlar kadar çogaltacagim. Söz verdigim bu ülkenin tümünü soyunuza verecegim. Sonsuza dek onlara miras olacak' dedin."
\par 14 Böylece RAB halkina yapacagini söyledigi kötülükten vazgeçti.
\par 15 Musa döndü, elinde antlasma kosullari yazili iki tas levhayla dagdan indi. Levhalarin ön ve arka iki yüzü de yaziliydi.
\par 16 Onlari Tanri yapmisti, üzerlerindeki oyma yazilar O'nun yazisiydi.
\par 17 Yesu, bagrisan halkin sesini duyunca, Musa'ya, "Ordugahtan savas sesi geliyor!" dedi.
\par 18 Musa söyle yanitladi: "Ne yenenlerin, Ne de yenilenlerin sesidir bu; Ezgiler duyuyorum ben."
\par 19 Musa ordugaha yaklasinca, buzagiyi ve oynayan insanlari gördü; çok öfkelendi. Elindeki tas levhalari firlatip dagin eteginde parçaladi.
\par 20 Yaptiklari buzagiyi alip yakti, toz haline gelinceye dek ezdi, sonra suya serperek Israilliler'e içirdi.
\par 21 Harun'a, "Bu halk sana ne yapti ki, onlari bu korkunç günaha sürükledin?" dedi.
\par 22 Harun, "Öfkelenme, efendim!" diye karsilik verdi, "Bilirsin, halk kötülüge egilimlidir.
\par 23 Bana, 'Bize öncülük edecek bir ilah yap. Bizi Misir'dan çikaran adama, Musa'ya ne oldu bilmiyoruz' dediler.
\par 24 Ben de, 'Kimde altin varsa çikarsin' dedim. Altinlarini bana verdiler. Atese atinca, bu buzagi ortaya çikti!"
\par 25 Musa halkin basibos hale geldigini gördü. Çünkü Harun onlari dizginlememis, düsmanlarina alay konusu olmalarina neden olmustu.
\par 26 Musa ordugahin girisinde durdu, "RAB'den yana olanlar yanima gelsin!" dedi. Bütün Levililer çevresine toplandi.
\par 27 Musa söyle dedi: "Israil'in Tanrisi RAB diyor ki, 'Herkes kilicini kusansin. Ordugahta kapi kapi dolasarak kardesini, komsusunu, yakinini öldürsün.'"
\par 28 Levililer Musa'nin buyrugunu yerine getirdiler. O gün halktan üç bine yakin adam öldürüldü.
\par 29 Musa, "Bugün kendinizi RAB'be adamis oldunuz" dedi, "Herkes öz ogluna, öz kardesine düsman kesildigi için bugün RAB sizi kutsadi."
\par 30 Ertesi gün halka, "Korkunç bir günah islediniz" dedi, "Simdi RAB'bin huzuruna çikacagim. Belki günahinizi bagislatabilirim."
\par 31 Sonra RAB'be dönerek, "Çok yazik, bu halk korkunç bir günah isledi" dedi, "Kendilerine altin put yaptilar.
\par 32 Lütfen günahlarini bagisla, yoksa yazdigin kitaptan adimi sil."
\par 33 RAB, "Kim bana karsi günah islediyse onun adini silecegim" diye karsilik verdi,
\par 34 "Simdi git, halki sana söyledigim yere götür. Melegim sana öncülük edecek. Ama zamani gelince günahlarindan ötürü onlari cezalandiracagim."
\par 35 RAB halki cezalandirdi. Çünkü Harun'a buzagi yaptirmislardi.

\chapter{33}

\par 1 RAB Musa'ya, "Buradan git" dedi, "Sen ve Misir'dan çikardigin halk Ibrahim'e, Ishak'a, Yakup'a, 'Orayi senin soyuna verecegim' diye ant içtigim topraklara gidin.
\par 2 Süt ve bal akan ülkeye senden önce bir melek gönderecek, Kenan, Amor, Hitit*, Periz, Hiv ve Yevus halklarini oradan kovacagim. Ben sizinle gelmeyecegim, çünkü inatçi insanlarsiniz. Belki sizi yolda yok ederim."
\par 3 (#33:2)
\par 4 Halk bu kötü haberi duyunca yasa büründü. Kimse taki takmadi.
\par 5 Çünkü RAB Musa'ya söyle demisti: "Israilliler'e de ki, 'Siz inatçi insanlarsiniz. Bir an aranizda kalsam, sizi yok ederim. Simdi üzerinizdeki takilari çikarin, size ne yapacagima karar vereyim.'"
\par 6 Böylece Horev Dagi'ndan sonra Israilliler takilarini çikardi.
\par 7 Musa bir çadir alir, ordugahin disina, biraz öteye kurardi. Ona 'Bulusma Çadiri' derdi. Kim RAB'be danismak istese, ordugahin disindaki Bulusma Çadiri'na giderdi.
\par 8 Musa ne zaman çadira gitse, bütün halk kalkar, herkes çadirinin girisinde durarak Musa içeri girinceye kadar arkasindan bakardi.
\par 9 Musa çadira girince, bulut sütunu asagi iner, RAB Musa'yla konustugu sürece girisi kapardi.
\par 10 Bulut sütununun çadirin girisinde durdugunu gören herkes kalkar, kendi çadirinin girisinde tapinirdi.
\par 11 RAB Musa'yla iki arkadas gibi yüz yüze konusurdu. Sonra Musa ordugaha dönerdi. Ama genç yardimcisi Nun oglu Yesu çadirdan çikmazdi.
\par 12 Musa RAB'be söyle dedi: "Bana, 'Bu halka öncülük et' diyorsun, ama kimi benimle gönderecegini söylemedin. Bana, 'Seni adinla taniyorum, senden hosnudum' demistin.
\par 13 Eger benden hosnutsan, lütfen simdi bana yollarini göster ki, seni daha iyi taniyip hosnut etmeye devam edeyim. Unutma, bu ulus senin halkindir."
\par 14 RAB, "Varligim sana eslik edecek" diye yanitladi, "Seni rahata kavusturacagim."
\par 15 Musa, "Eger varligin bize eslik etmeyecekse, bizi buradan çikarma" dedi,
\par 16 "Yoksa benden ve halkindan hosnut kaldigin nereden bilinecek? Bize eslik etmenden, degil mi? Ancak o zaman benimle halkin yeryüzünün öteki halklarindan ayirt edilebiliriz."
\par 17 RAB, "Söyledigin gibi yapacagim" dedi, "Çünkü senden hosnut kaldim, adinla taniyorum seni."
\par 18 Musa, "Lütfen görkemini bana göster" dedi.
\par 19 RAB, "Bütün iyiligimi önünden geçirecegim" diye karsilik verdi, "Adimi, RAB adini senin önünde duyuracagim. Merhamet ettigime merhamet edecegim, acidigima aciyacagim.
\par 20 Ancak, yüzümü görmene izin veremem. Çünkü yüzümü gören yasayamaz."
\par 21 Sonra, "Yakinimda bir yer var" dedi, "Orada, kayanin üzerinde dur.
\par 22 Görkemim oradan geçerken seni kayanin kovuguna sokup geçinceye kadar elimle örtecegim.
\par 23 Elimi kaldirdigimda, sirtimi göreceksin. Ama yüzüm görülmeyecek."

\chapter{34}

\par 1 RAB Musa'ya, "Öncekiler gibi iki tas levha kes" dedi, "Kirdigin levhalarin üzerindeki sözleri onlara yazacagim.
\par 2 Sabaha kadar hazirlan, sabah olunca Sina Dagi'na çik; dagin tepesinde, huzurumda dur.
\par 3 Senden baska kimse daga çikmasin, dagin hiçbir yerinde kimse görülmesin. Dagin eteginde davar ya da sigir da otlamasin."
\par 4 Musa öncekiler gibi iki tas levha kesti. RAB'bin buyurdugu gibi sabah erkenden kalkti, tas levhalari yanina alarak Sina Dagi'na çikti.
\par 5 RAB bulutun içinde oraya inip onunla birlikte durdu ve adini RAB olarak açikladi.
\par 6 Musa'nin önünden geçerek, "Ben RAB'bim" dedi, "RAB, aciyan, lütfeden, tez öfkelenmeyen, sevgisi engin ve sadik Tanri.
\par 7 Binlercesine sevgi gösterir, suçlarini, isyanlarini, günahlarini bagislarim. Hiçbir suçu cezasiz birakmam. Babalarin isledigi suçun hesabini ogullarindan, torunlarindan, üçüncü, dördüncü kusaklardan sorarim."
\par 8 Musa hemen yere kapanip tapindi.
\par 9 "Ya Rab, eger benden hosnutsan, lütfen bizimle gel" dedi, "Bunlar inatçi insanlardir. Sen suçlarimizi, günahlarimizi bagisla. Bizi kendi mirasin olarak benimse."
\par 10 RAB, "Senin halkinla bir antlasma yapiyorum" dedi, "Onlarin önünde dünyada ve öteki uluslar arasinda görülmemis harikalar yapacagim. Arasinda yasadigin halk neler yapabilecegimi görecek. Senin için korkunç seyler yapacagim.
\par 11 Bugün sana verdigim buyrugu tut. Amor, Kenan, Hitit*, Periz, Hiv ve Yevus halklarini senin önünden kovacagim.
\par 12 Gidecegin ülkedeki insanlarla antlasma yapmaktan kaçin. Çünkü bu senin için bir tuzak olur.
\par 13 Onlarin sunaklarini yikacak, dikili taslarini parçalayacak, Asera* putlarini keseceksiniz.
\par 14 Baska ilahlara tapmayacaksiniz. Çünkü ben kiskanç bir RAB, kiskanç bir Tanri'yim.
\par 15 Ülke halkiyla herhangi bir antlasma yapmayin. Yoksa onlar baska ilahlara gönül verir, kurban keserken sizi de çagirirlar; siz de gider yersiniz.
\par 16 Kizlarini ogullariniza alirsiniz. Kizlar baska ilahlara gönül verirken ogullarinizi da artlarindan sürükler.
\par 17 "Dökme putlar yapmayacaksiniz.
\par 18 "Size buyurdugum gibi, Aviv ayinin* belirli günlerinde yedi gün mayasiz ekmek yiyerek Mayasiz Ekmek Bayrami'ni* kutlayacaksiniz. Çünkü Misir'dan Aviv ayinda çiktiniz.
\par 19 "Bütün ilk doganlar benimdir; ister sigir, ister davar olsun, ilk dogan erkek hayvanlarinizin tümü bana aittir.
\par 20 Ilk dogan sipanin bedelini bir kuzuyla ödeyin. Bedelini ödemeyecekseniz, sipanin boynunu kiracaksiniz. Bütün ilk dogan ogullarinizin bedelini ödemelisiniz. "Kimse huzuruma eli bos çikmasin.
\par 21 "Alti gün çalisacak, yedinci gün dinleneceksiniz. Ekim, biçim vakti bile olsa dinleneceksiniz.
\par 22 "Ilk bugday biçiminde Haftalar Bayrami*, yil sonunda da Ürün Devsirme Bayrami* yapacaksiniz.
\par 23 Bütün erkekleriniz yilda üç kez Israil'in Tanrisi ben Egemen RAB'bin huzuruna çikacaklar.
\par 24 Öteki uluslari önünüzden kovacak, sinirlarinizi genisletecegim. Yilda üç kez Tanriniz RAB'bin önüne çiktiginiz zaman, kimse ülkenize göz dikemeyecek.
\par 25 "Evinizde maya bulundugu sürece bana kurban kesmeyeceksiniz. Fisih* kurbani sabaha birakilmayacak.
\par 26 "Topraginizin seçme ilk ürünlerini Tanriniz RAB'bin Tapinagi'na getireceksiniz. "Oglagi anasinin sütünde haslamayacaksiniz."
\par 27 RAB Musa'ya, "Bunlari yaz" dedi, "Çünkü seninle ve Israilliler'le bu sözlere dayanarak antlasma yaptim."
\par 28 Musa orada kirk gün kirk gece RAB'le birlikte kaldi. Agzina ne ekmek koydu, ne de su. Antlasma sözlerini, on buyrugu tas levhalarin üzerine yazdi.
\par 29 Musa elinde iki antlasma levhasiyla Sina Dagi'ndan indi. RAB'le konustugu için yüzü isildiyordu, ama kendisi bunun farkinda degildi.
\par 30 Harun'la Israilliler Musa'nin isildayan yüzünü görünce, ona yaklasmaya korktular.
\par 31 Musa onlari yanina çagirdi. Harun'la Israil toplulugunun bütün önderleri çevresine toplandilar. Musa onlarla konustu.
\par 32 Sonra herkes ona yaklasti. Musa RAB'bin Sina Dagi'nda kendisine bildirdigi bütün buyruklari onlara verdi.
\par 33 Konusmasini bitirdikten sonra, yüzüne bir peçe takti.
\par 34 Ama ne zaman konusmak için RAB'bin huzuruna çiksa, ayrilincaya kadar peçeyi kaldirirdi. Dönünce de kendisine verilen buyruklari Israilliler'e bildirir,
\par 35 Israilliler de onun isildayan yüzünü görürlerdi. Sonra Musa içeri girip RAB'le görüsünceye kadar yine peçeyi takardi.

\chapter{35}

\par 1 Musa bütün Israil toplulugunu çagirarak, "RAB'bin yapmanizi buyurdugu isler sunlardir" dedi,
\par 2 "Alti gün çalisacaksiniz. Ama yedinci gün sizin için kutsal Sabat*, RAB'be adanmis dinlenme günü olacaktir. O gün çalisan herkes öldürülecektir.
\par 3 Sabat Günü konutlarinizda ates yakmayacaksiniz."
\par 4 Musa bütün Israil topluluguna seslenerek söyle dedi: "RAB'bin buyrugu sudur:
\par 5 Aranizda armaganlar toplayip RAB'be sunacaksiniz. Istekli olan herkes RAB'be altin, gümüs, tunç*; lacivert, mor, kirmizi iplik; ince keten, keçi kili, deri, kirmizi boyali koç derisi, akasya agaci armagan etsin.
\par 6 (#35:5)
\par 7 (#35:5)
\par 8 Kandil için zeytinyagi; mesh yagi ve güzel kokulu buhur için baharat;
\par 9 baskâhinin efoduyla* gögüslügü* için oniks ve öbür kakma taslari getirsin.
\par 10 "Aranizdaki bütün becerikli kisiler gelip RAB'bin buyurdugu her seyi yapsin.
\par 11 Konutu, çadirin iç ve dis örtüsünü, kopçalarini, çerçevelerini, kirislerini, direklerini, tabanlarini;
\par 12 sandigi ve siriklarini, Bagislanma Kapagi'ni, bölme perdesini,
\par 13 masayla siriklarini, bütün masa takimlarini, huzura konan ekmekleri*;
\par 14 isik için kandilligi ve takimlarini, kandilleri, kandiller için zeytinyagini;
\par 15 buhur sunagini ve siriklarini, mesh yagini, güzel kokulu buhuru; konutun giris bölümündeki perdeyi;
\par 16 yakmalik sunu* sunagini ve tunç izgarasini, siriklarini, bütün takimlarini, kazani ve kazan ayakligini;
\par 17 avlunun çevresindeki perdeleri, direkleri, direk tabanlarini, avlu kapisindaki perdeyi,
\par 18 konutun ve avlunun kaziklariyla iplerini;
\par 19 kutsal yerde hizmet etmek için dokunmus giysileri -Kâhin Harun'un giysileriyle ogullarinin kâhin giysilerini- yapsinlar."
\par 20 Israil toplulugu Musa'nin yanindan ayrildi.
\par 21 Her istekli, hevesli kisi Bulusma Çadiri'nin yapimi, hizmeti ve kutsal giysiler için RAB'be armagan getirdi.
\par 22 Kadin erkek herkes istekle geldi, RAB'be her çesit altin taki, bros, küpe, yüzük, kolye getirdi. RAB'be armagan ettikleri bütün takilar altindi.
\par 23 Ayrica kimde lacivert, mor, kirmizi iplik; ince keten, keçi kili, deri, kirmizi boyali koç derisi varsa getirdi.
\par 24 Gümüs ve tunç armaganlar sunan herkes onlari RAB'be adadi. Herhangi bir iste kullanilmak üzere kimde akasya agaci varsa getirdi.
\par 25 Bütün becerikli kadinlar elleriyle egirdikleri lacivert, mor, kirmizi ipligi, ince keteni getirdiler.
\par 26 Istekli, becerikli kadinlar da keçi kili egirdiler.
\par 27 Önderler efod ve gögüslük için oniks, kakma taslar,
\par 28 kandil, mesh yagi ve güzel kokulu buhur için baharat ve zeytinyagi getirdiler.
\par 29 Kadin erkek bütün istekli Israilliler RAB'bin Musa araciligiyla yapmalarini buyurdugu isler için RAB'be gönülden verilen sunu sundular.
\par 30 Musa Israilliler'e, "Bakin!" dedi, "RAB Yahuda oymagindan özellikle Hur oglu Uri oglu Besalel'i seçti.
\par 31 Beceri, anlayis, bilgi ve her türlü ustalik vermek için onu kendi Ruhu'yla doldurdu.
\par 32 Öyle ki, altin, gümüs, tunç* isleyerek ustaca yapitlar üretsin;
\par 33 tas kesme ve kakmada, agaç oymaciliginda, her türlü sanat dalinda çalissin.
\par 34 RAB ona ve Dan oymagindan Ahisamak oglu Oholiav'a ögretme yetenegi de verdi.
\par 35 Onlara üstün beceri verdi. Öyle ki, ustalik isteyen her türlü iste, oymacilikta, lacivert, mor, kirmizi iplik ve ince keten yapmada, dokuma ve nakis islerinde, her sanat dalinda yaratici olsunlar.

\chapter{36}

\par 1 "Besalel, Oholiav ve kutsal yerin yapiminda gereken isleri nasil yapacaklarina iliskin RAB'bin kendilerine bilgelik ve anlayis verdigi bütün becerikli kisiler her isi tam RAB'bin buyurdugu gibi yapacaklar."
\par 2 Musa Besalel'i, Oholiav'i, RAB'bin kendilerine bilgelik verdigi becerikli adamlari ve çalismaya istekli herkesi is basina çagirdi.
\par 3 Gelenler kutsal yerin yapiminda gereken isleri yapmak üzere Israilliler'in getirmis oldugu bütün armaganlari Musa'dan aldilar. Israilliler gönülden verdikleri sunulari her sabah Musa'ya getirmeye devam ettiler.
\par 4 Öyle ki, kutsal yerdeki isleri yapmakta olan ustalar islerini birakip bir bir Musa'nin yanina gelerek,
\par 5 "Halk RAB'bin yapilmasini buyurdugu is için gereginden fazla getiriyor" dediler.
\par 6 Bunun üzerine Musa buyruk verdi: "Ne erkek, ne kadin hiç kimse kutsal yere armagan olarak artik bir sey vermesin." Buyruk ordugahta ilan edildi. Böylece halkin daha çok armagan getirmesine engel olundu.
\par 7 Çünkü o ana kadar getirilenler isi bitirmek için yeter de artardi bile.
\par 8 Çalisanlar arasindaki becerikli adamlar konutu on perdeden yaptilar. Besalel onlari lacivert, mor, kirmizi iplikle, özenle dokunmus ince ketenden yapti, üzerini Keruvlar'la* ustaca süsledi.
\par 9 Her perdenin boyu yirmi sekiz, eni dört arsindi. Bütün perdeler ayni ölçüdeydi.
\par 10 Perdeleri beser beser birbirine ekleyerek iki takim perde yapti.
\par 11 Birinci takimin kenarina lacivert ilmekler açti. Öbür takimin kenarina da ayni seyi yapti.
\par 12 Birinci takimin ilk perdesiyle ikinci takimin son perdesine elliser ilmek açti; ilmekler birbirine karsiydi.
\par 13 Elli altin kopça yapti, perdeleri kopçalayarak çadiri birlestirdi. Böylece konut tek parça haline geldi.
\par 14 Konutun üstünü kaplayacak çadir için keçi kilindan on bir perde yapti.
\par 15 Her perdenin boyu otuz, eni dört arsindi. On bir perde de ayni ölçüdeydi.
\par 16 Bes perdeyi birbirine, alti perdeyi birbirine birlestirdi.
\par 17 Her iki perde takiminin kenarlarina elliser ilmek açti.
\par 18 Çadiri birlestirip tek parça haline getirmek için elli tunç* kopça yapti.
\par 19 Çadir için kirmizi boyali koç derisinden bir örtü, onun üstüne de deriden baska bir örtü yapti.
\par 20 Konut için akasya agacindan dikine çerçeveler yapti.
\par 21 Her çerçevenin boyu on, eni bir buçuk arsindi.
\par 22 Çerçevelerin birbirine uyan iki paralel çikintisi vardi. Konutun bütün çerçevelerini ayni biçimde yapti.
\par 23 Konutun güneyi için yirmi çerçeve yapti.
\par 24 Her çerçevenin altinda iki çikinti için birer taban olmak üzere, yirmi çerçevenin altinda kirk gümüs taban yapti.
\par 25 Konutun öbür yani, yani kuzeyi için de yirmi çerçeve ve her çerçevenin altinda iki taban olmak üzere kirk gümüs taban yapti.
\par 26 (#36:25)
\par 27 Konutun batiya bakacak arka tarafi için alti çerçeve yapti.
\par 28 Arkada konutun köseleri için iki çerçeve yapti.
\par 29 Bu köse çerçevelerinin alt tarafi ayri kaldi, üst tarafi ise birinci halkayla birlestirildi. Iki köseyi olusturan iki çerçeveyi ayni biçimde yapti.
\par 30 Böylece sekiz çerçeve ve her çerçevenin altinda iki taban olmak üzere on alti gümüs taban yapti.
\par 31 Konutun bir yanindaki çerçeveler için bes, öbür yanindaki çerçeveler için bes, batiya bakan arka tarafindaki çerçeveler için de bes olmak üzere akasya agacindan kirisler yapti.
\par 32 (#36:31)
\par 33 Çerçevelerin ortasindaki kirisi konutun bir ucundan öbür ucuna geçirdi.
\par 34 Çerçevelerle kirisleri altinla kapladi, kirislerin geçecegi halkalari da altindan yapti.
\par 35 Lacivert, mor, kirmizi iplikle, özenle dokunmus ince ketenden bir perde yapti, üzerini Keruvlar'la ustaca süsledi.
\par 36 Perde için akasya agacindan dört direk yaparak altinla kapladi. Çengelleri de altindi. Direkler için dört gümüs taban döktü.
\par 37 Çadirin giris bölümüne lacivert, mor, kirmizi iplikle, özenle dokunmus ince ketenden nakisli bir perde yapti.
\par 38 Perdeyi asmak için çengelli bes direk yaparak basliklarini, çemberlerini altinla kapladi. Direklere bes tunç taban yapti.

\chapter{37}

\par 1 Besalel Antlasma Sandigi'ni* akasya agacindan yapti. Boyu iki buçuk, eni ve yüksekligi birer buçuk arsindi.
\par 2 Içini de disini da saf altinla kapladi. Çevresine altin pervaz yapti.
\par 3 Ikisi bir yanda, ikisi öbür yanda olmak üzere sandigin dört kösesindeki ayaklara takmak için birer altin halka döktü.
\par 4 Akasya agacindan siriklar yapip altinla kapladi.
\par 5 Sandigin tasinmasi için siriklari yanlardaki halkalara geçirdi.
\par 6 Bagislanma Kapagi'ni saf altindan yapti. Boyu iki buçuk*fj*, eni bir buçuk arsindi*fk*.
\par 7 Kapagin iki kenarina dövme altindan birer Keruv* yapti.
\par 8 Keruvlar'dan birini bir kenara, öbürünü öteki kenara koyarak kapagi tek parça halinde yapti.
\par 9 Keruvlar yukari dogru açik kanatlariyla kapagi örtüyor, yüzleri birbirine dönük kapaga bakiyorlardi.
\par 10 Besalel akasya agacindan bir masa yapti. Boyu iki, eni bir, yüksekligi bir buçuk arsindi.
\par 11 Masayi saf altinla kapladi. Çevresine altin pervaz yapti.
\par 12 Pervazin çevresine dört parmak eninde bir kenarlik yaparak altin pervazla çevirdi.
\par 13 Masa için dört altin halka dökerek dört ayak üzerindeki dört köseye yerlestirdi.
\par 14 Masanin tasinmasi için siriklarin içinden geçecegi halkalar kenarliga yakindi.
\par 15 Siriklari akasya agacindan yapti, altinla kapladi.
\par 16 Masa için saf altindan tabaklar, sahanlar, dökmelik sunu testileri, taslari yapti.
\par 17 Saf altindan bir kandillik yapti. Ayagi, gövdesi dövme altindi. Çanak, tomurcuk ve çiçek motifleri kendindendi.
\par 18 Üç kolu bir yanda, üç kolu öteki yanda olmak üzere alti kolluydu.
\par 19 Her kolda badem çiçegini andiran üç çanak, tomurcuk ve çiçek motifi vardi. Alti kol da ayniydi.
\par 20 Kandilligin gövdesinde badem çiçegini andiran dört çanak, tomurcuk ve çiçek motifi bulunuyordu.
\par 21 Kandillikten yükselen ilk iki kolun, ikinci iki kolun, üçüncü iki kolun altinda kendinden birer tomurcuk vardi. Toplam alti koldu.
\par 22 Tomurcuklari, kollari tek parça olan kandillik saf dövme altindi.
\par 23 Kandillik için saf altindan yedi kandil, fitil masalari, tablalar yapti.
\par 24 Bütün takimlari dahil kandillige bir talant saf altin harcandi.
\par 25 Akasya agacindan bir buhur sunagi yapti. Kare biçiminde, boyu ve eni birer arsin, yüksekligi iki arsindi. Boynuzlari kendindendi.
\par 26 Üstünü, yanlarini, boynuzlarini saf altinla kapladi. Çevresine altin pervaz yapti.
\par 27 Iki yandaki pervazin altina iki altin halka yapti. Bunlar sunagin tasinmasi için siriklarin geçmesine yariyordu.
\par 28 Siriklari akasya agacindan yaparak altinla kapladi.
\par 29 Itriyatçi ustaligiyla kutsal mesh yagi ve güzel kokulu saf buhur yapti.

\chapter{38}

\par 1 Besalel yakmalik sunu* sunagini akasya agacindan kare biçiminde yapti. Eni ve boyu beser arsin, yüksekligi üç arsindi.
\par 2 Dört üst kösesine kendinden boynuzlar yaparak hepsini tunçla* kapladi.
\par 3 Sunagin bütün takimlarini -kovalari, kürekleri, çanaklari, büyük çatallari, ates kaplarini- tunçtan yapti.
\par 4 Kenarin altinda asagi dogru sunagin yarisina kadar ag biçiminde tunç bir izgara yapti.
\par 5 Tunç izgaranin dört kösesine tasima siriklarini geçirmek için birer halka döktü.
\par 6 Siriklari akasya agacindan yaparak tunçla kapladi.
\par 7 Sunagin tasinmasi için yan tarafindaki halkalara geçirdi. Sunagi tahtadan, içi bos yapti.
\par 8 Bulusma Çadiri'nin giris bölümünde hizmet eden kadinlarin aynalarindan tunç* ayaklikli tunç bir kazan yapti.
\par 9 Konuta bir avlu yapti. Avlunun güney tarafi için yüz arsin boyunda özenle dokunmus ince keten perdeler yapti.
\par 10 Perdeler için tabanlari tunç*, çengelleri ve çengel çemberleri gümüs yirmi direk yapti.
\par 11 Kuzey tarafi için yüz arsin boyunda perdeler, yirmi direk, direkler için yirmi tunç taban yapildi. Direklerin çengelleriyle çemberleri gümüstü.
\par 12 Avlunun bati tarafi için elli arsin boyunda perde, on direk, on taban yapildi. Direklerin çengelleriyle çemberleri gümüstü.
\par 13 Doguya bakan tarafta avlunun eni elli arsindi*ft*.
\par 14 Girisin bir tarafinda on bes arsin boyunda perde, üç direk ve üç taban;
\par 15 öbür tarafinda da on bes arsin*fu* boyunda perde, üç direk ve üç taban vardi.
\par 16 Avlunun çevresindeki bütün perdeler özenle dokunmus ince ketendi.
\par 17 Direklerin tabanlari tunç, çengelleriyle çemberleri gümüstü. Basliklari da gümüs kaplamaydi. Avlunun bütün direkleri gümüs çemberlerle donatilmisti.
\par 18 Avlunun girisindeki perde lacivert, mor, kirmizi iplikle, özenle dokunmus nakisli ince ketenden yapilmisti. Boyu yirmi, yüksekligi avlunun perdeleri gibi bes arsindi.
\par 19 Tunçtan dört diregi ve dört tabani vardi. Direklerin çengelleri, basliklarinin kaplamasi ve çemberleri gümüstü.
\par 20 Konutun ve konutu çevreleyen avlunun bütün kaziklari tunçtandi.
\par 21 Antlasma Levhalari'nin bulundugu konut için kullanilan malzeme miktarinin tümü Musa'nin buyrugu uyarinca, Kâhin Harun oglu Itamar'in yönetimindeki Levililer tarafindan kaydedildi.
\par 22 RAB'bin Musa'ya buyurdugu bütün isleri Yahuda oymagindan
\par 23 Dan oymagindan oymaci, yaratici, lacivert, mor, kirmizi iplik ve ince keten islemede usta nakisçi Ahisamak oglu Oholiav da ona yardim etti.
\par 24 Kutsal yerdeki bütün isler için kullanilan adanmis altin miktari kutsal yerin sekeliyle 29 talant 730 sekeldi.
\par 25 Toplulugun sayimindan elde edilen gümüs, kutsal yerin sekeliyle 100 talant 1 775 sekeldi.
\par 26 Sayimi yapilan yirmi ve daha yukari yastaki 603 550 kisiden adam basina bir beka, yani yarim kutsal yerin sekeli düsüyordu.
\par 27 Kutsal yer ve perde tabanlarinin dökümü için 100 talant gümüs kullanildi. Her tabana bir talant olmak üzere, 100 tabana 100 talant gümüs harcandi.
\par 28 Direklerin çengelleri, basliklarin kaplanmasi ve çemberleri için 1 775 sekel harcandi.
\par 29 Adanan tunç* 70 talant 2 400 sekeldi.
\par 30 Bununla Bulusma Çadiri'nin giris bölümündeki tabanlar, sunakla izgarasi ve bütün takimlari, avlu çevresindeki ve girisindeki tabanlar, bütün konut kaziklariyla avlu çevresindeki kaziklar yapildi.

\chapter{39}

\par 1 Kutsal yerde hizmet için lacivert, mor, kirmizi iplikten özenle dokunmus giysiler yaptilar. Ayrica RAB'bin Musa'ya buyurdugu gibi Harun'a kutsal giysiler yapildi.
\par 2 Efodu* altin sirmayla lacivert, mor, kirmizi iplikle, özenle dokunmus ince ketenden yaptilar.
\par 3 Altini ince tabakalar halinde dövüp lacivert, mor, kirmizi iplik ve ince keten arasina ustaca islemek için tel tel kestiler.
\par 4 Efodun iki kösesine tutturulmus omuzluklar yaparak birlestirdiler.
\par 5 Efodun üzerindeki ustaca dokunmus serit efodun bir parçasi gibi altin sirmayla lacivert, mor, kirmizi iplikle, özenle dokunmus ince ketendendi; tipki RAB'bin Musa'ya buyurdugu gibiydi.
\par 6 Altin yuvalar içine kakilmis, üzerine Israilogullari'nin adlari mühür gibi oyulmus oniksi isleyip
\par 7 Israilliler'in anilmasi için efodun omuzluklarina taktilar. Tipki RAB'bin Musa'ya buyurdugu gibi yaptilar.
\par 8 Efod* gibi altin sirmayla lacivert, mor, kirmizi iplikle, özenle dokunmus ince ketenden usta isi bir gögüslük* yaptilar.
\par 9 Dört köse, eni ve boyu birer karisti, ikiye katlanmisti.
\par 10 Üzerine dört sira tas yuvasi kaktilar. Birinci sirada yakut, topaz, zümrüt;
\par 11 ikinci sirada firuze, laciverttasi, aytasi;
\par 12 üçüncü sirada gökyakut, agat, ametist;
\par 13 dördüncü sirada sari yakut, oniks, yesim vardi. Taslar altin yuvalara kakilmisti.
\par 14 On iki tas vardi. Üzerlerine mühür oyar gibi Israilogullari'nin adlari bir bir oyulmustu. Bu taslar Israil'in on iki oymagini simgeliyordu.
\par 15 Gögüslük için saf altindan örme zincirler yaptilar.
\par 16 Ikiser tane altin yuva ve halka yaptilar. Gögüslügün üst iki kösesine birer halka koydular.
\par 17 Iki örme altin zinciri gögüslügün köselerindeki halkalara taktilar.
\par 18 Zincirlerin öteki iki ucunu iki yuvanin üzerinden geçirerek efodun ön tarafina, omuzluklarin üzerine bagladilar.
\par 19 Iki altin halka yaparak gögüslügün alt iki kösesine, efoda bitisik iç kenarina taktilar.
\par 20 Iki altin halka daha yaparak efodun önündeki omuzluklara alttan, dikise yakin, ustaca dokunmus seridin yukarisina taktilar.
\par 21 Gögüslügün halkalariyla efodun halkalarini lacivert kordonla birbirine bagladilar. Öyle ki, gögüslük efodun ustaca dokunmus seridinin yukarisinda kalsin ve efoddan ayrilmasin. Tipki RAB'bin Musa'ya buyurdugu gibi yaptilar.
\par 22 Efodun* altina giyilen kaftani ustaca dokunmus salt lacivert iplikten yaptilar.
\par 23 Ortasinda bas geçecek kadar bir bosluk biraktilar. Yirtilmamasi için boslugun kenarlarini yaka gibi dokuyarak çevirdiler.
\par 24 Kaftanin kenarini lacivert, mor, kirmizi iplikle, özenle dokunmus ince ketenden nar motifleriyle bezediler.
\par 25 Saf altindan çingiraklar yaptilar ve hizmet için kullanilan kaftanin eteginin ucundaki narlarin arasina, bir çingirak bir nar, bir çingirak bir nar olmak üzere çepeçevre koydular. Tipki RAB'bin Musa'ya buyurdugu gibi yaptilar.
\par 26 (#39:25)
\par 27 Harun'la ogullari için ince ketenden ustaca dokunmus mintanlar, sariklar, süslü basliklar, ince keten donlar, lacivert, mor, kirmizi iplikle, özenle dokunmus ince ketenden nakisli kusak yaptilar; tipki RAB'bin Musa'ya buyurdugu gibi.
\par 28 (#39:27)
\par 29 (#39:27)
\par 30 Kutsal tacin levhasini saf altindan yaparak üzerine mühür oyar gibi 'RAB'be adanmistir' sözünü yazdilar.
\par 31 Üstüne baglanmak üzere sariga lacivert bir kordon taktilar; tipki RAB'bin Musa'ya buyurdugu gibi.
\par 32 Böylece konutun, yani Bulusma Çadiri'nin bütün isleri tamamlandi. Israilliler her seyi tipki RAB'bin Musa'ya buyurdugu gibi yaptilar.
\par 33 Konutu, çadirla bütün takimlarini, kopçalarini, çerçevelerini, kirislerini, direklerini, tabanlarini; kirmizi boyali koç derisinden örtüyü, deri örtüyü, bölme perdesini; Levha Sandigi'yla siriklarini, Bagislanma Kapagi'ni; masayla takimlarini, Tanri'nin huzuruna konan ekmekleri*; saf altin kandilligi, üstüne dizilecek kandillerle takimlarini, kandil için zeytinyagini; altin sunagi, mesh yagini, güzel kokulu buhuru, çadirin giris bölümünün perdesini; tunç* sunakla izgarasini, siriklarini, bütün takimlarini, kazani, kazan ayakligini; avlunun perdelerini, direklerini, direk tabanlarini, avlu girisinin perdesini, iplerini, kaziklarini, konutta, yani Bulusma Çadiri'ndaki hizmet için gerekli bütün aletleri; kutsal yerdeki hizmet için dokunmus giysileri, Kâhin Harun'un kutsal giysilerini, ogullarinin kâhin giysilerini Musa'ya gösterdiler.
\par 34 (#39:33)
\par 35 (#39:33)
\par 36 (#39:33)
\par 37 (#39:33)
\par 38 (#39:33)
\par 39 (#39:33)
\par 40 (#39:33)
\par 41 (#39:33)
\par 42 Her seyi tipki RAB'bin Musa'ya buyurdugu gibi yaptilar.
\par 43 Musa bakti, bütün islerin RAB'bin buyurdugu gibi yapilmis oldugunu görünce onlari kutsadi.

\chapter{40}

\par 1 RAB Musa'ya söyle dedi:
\par 2 "Konutu, yani Bulusma Çadiri'ni birinci ayin* ilk günü kur.
\par 3 Levha Sandigi'ni oraya getirip perdeyle gizle.
\par 4 Masayi içeri getir, gereken her seyi üzerine diz. Kandilligi getirip kandillerini yak.
\par 5 Altin buhur sunagini Levha Sandigi'nin önüne koy, konutun giris bölümüne perdesini tak.
\par 6 Yakmalik sunu* sunagini konutun -Bulusma Çadiri'nin- giris bölümüne koy.
\par 7 Kazani çadirla sunak arasina koyup içine su doldur.
\par 8 Çadirin çevresini avluyla kapat, avlunun girisine perdesini as.
\par 9 "Sonra mesh yagiyla konutu ve içindeki bütün esyalari meshederek* kutsal kil. Böylece konutla takimlari kutsal olacak.
\par 10 Yakmalik sunu sunagiyla takimlarini meshet, sunagi kutsal kil. Sunak çok kutsal olacak.
\par 11 Kazan ve kazan ayakligini meshederek kutsal kil.
\par 12 "Harun'la ogullarini Bulusma Çadiri'nin giris bölümüne getirip yika.
\par 13 Harun'a kutsal giysileri giydir, bana kâhinlik etmesi için onu meshederek kutsal kil.
\par 14 Ogullarini getirip mintanlari giydir.
\par 15 Bana kâhinlik etmeleri için babalari gibi onlari da meshet. Bu mesh onlarin kusaklar boyu sürekli kâhin olmalarini saglayacak."
\par 16 Musa her seyi RAB'bin kendisine buyurdugu gibi yapti.
\par 17 Böylece ikinci yilin birinci ayinin birinci günü konut kuruldu.
\par 18 Musa konutu kurdu, tabanlarini koydu, çerçevelerini yerlestirdi, kirislerini takti, direklerini dikti.
\par 19 Çadiri tipki RAB'bin kendisine buyurdugu gibi konutun üzerine gerdi, çadir örtüsünü üzerine örttü.
\par 20 Antlasma Levhalari'ni sandiga koydu, sandik siriklarini takti, Bagislanma Kapagi'ni sandigin üzerine yerlestirdi.
\par 21 RAB'bin kendisine buyurdugu gibi Levha Sandigi'ni konuta getirdi, bölme perdesini asarak sandigi gizledi.
\par 22 Masayi Bulusma Çadiri'na, konutun kuzeyine, perdenin disina koydu.
\par 23 RAB'bin huzurunda, RAB'bin kendisine buyurdugu gibi üzerine ekmekleri dizdi.
\par 24 Kandilligi Bulusma Çadiri'na, masanin karsisina, konutun güneyine koydu.
\par 25 RAB'bin kendisine buyurdugu gibi, RAB'bin huzurunda kandilleri yakti.
\par 26 Altin sunagi Bulusma Çadiri'na, perdenin önüne koydu.
\par 27 RAB'bin kendisine buyurdugu gibi üzerinde güzel kokulu buhur yakti.
\par 28 Konutun giris bölümünün perdesini takti.
\par 29 RAB'bin kendisine buyurdugu gibi yakmalik sunu sunagini Bulusma Çadiri'nin giris bölümüne koydu, üzerinde yakmalik sunu ve tahil sunusu* sundu.
\par 30 Kazani Bulusma Çadiri ile sunak arasina koydu, yikanmak için içine su doldurdu.
\par 31 Musa, Harun ve Harun'un ogullari ellerini, ayaklarini orada yikadilar.
\par 32 Ne zaman Bulusma Çadiri'na girip sunaga yaklassalar RAB'bin Musa'ya buyurdugu gibi orada yikandilar.
\par 33 Musa konutla sunagi avluyla çevirdi. Avlunun girisine perdeyi asarak isi tamamladi.
\par 34 O zaman bulut Bulusma Çadiri'ni kapladi ve RAB'bin görkemi konutu doldurdu.
\par 35 Musa Bulusma Çadiri'na giremedi; çünkü bulut her yeri kaplamis, RAB'bin görkemi konutu doldurmustu.
\par 36 Israilliler ancak bulut konutun üzerinden kalkinca göçerlerdi.
\par 37 Bulut durdukça yerlerinden ayrilmaz, kalkacagi günü beklerlerdi.
\par 38 Böylece bütün yolculuklarinda konutun üzerinde gündüzün RAB'bin bulutu, gece de ates Israilliler'e yol gösterdi.


\end{document}