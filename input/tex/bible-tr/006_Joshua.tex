\begin{document}

\title{Yeşu}


\chapter{1}

\par 1 RAB, kulu Musa'nin ölümünden sonra onun yardimcisi Nun oglu Yesu'ya söyle seslendi:
\par 2 "Kulum Musa öldü. Simdi kalk, bütün halkla birlikte Seria Irmagi'ni geç. Size, Israil halkina verecegim ülkeye girin.
\par 3 Musa'ya söyledigim gibi, ayak basacaginiz her yeri size veriyorum.
\par 4 Sinirlariniz çölden Lübnan'a, büyük Firat Irmagi'ndan -bütün Hitit* ülkesi dahil- batidaki Akdeniz'e kadar uzanacak.
\par 5 Yasamin boyunca hiç kimse sana karsi koyamayacak; nasil Musa ile birlikte oldumsa, seninle de birlikte olacagim. Seni terk etmeyecegim, seni yüzüstü birakmayacagim.
\par 6 "Güçlü ve yürekli ol. Çünkü halki, atalarina verecegime ant içtigim ülkeyi miras almaya sen götüreceksin.
\par 7 Yeter ki, güçlü ve yürekli ol. Kulum Musa'nin sana buyurdugu Kutsal Yasa'nin tümünü yerine getirmeye dikkat et. Gittigin her yerde basarili olmak için bu yasadan ayrilma, saga sola sapma.
\par 8 Yasa Kitabi'nda yazilanlari dilinden düsürme. Tümünü özenle yerine getirmek için gece gündüz onu düsün. O zaman basarili olacak ve amacina ulasacaksin.
\par 9 Sana güçlü ve yürekli ol demedim mi? Korkma, yilma. Çünkü Tanrin RAB gidecegin her yerde seninle birlikte olacak."
\par 10 Bunun üzerine Yesu, halkin görevlilerine söyle buyurdu:
\par 11 "Ordugahin ortasindan geçip halka su buyrugu verin: 'Kendinize kumanya hazirlayin. Çünkü Tanriniz RAB'bin size verecegi ülkeye girip orayi mülk edinmek için üç gün sonra Seria Irmagi'ni geçeceksiniz."
\par 12 Yesu, Ruben ve Gad oymaklarina ve Manasse oymaginin yarisina da söyle dedi:
\par 13 "RAB'bin kulu Musa'nin, 'Tanriniz RAB bu ülkeyi size verip sizi rahata erdirecek dedigini animsayin.
\par 14 Kadinlariniz, çocuklariniz ve hayvanlariniz Seria Irmagi'nin dogusunda, Musa'nin size verdigi topraklarda kalsin. Ama sizler, bütün yigit savasçilar, silahli olarak kardeslerinizden önce irmagi geçip onlara yardim edin.
\par 15 RAB sizi rahata erdirdigi gibi, onlari da rahata erdirecek. Onlar Tanriniz RAB'bin verecegi ülkeyi mülk edindikten sonra siz de mülk edindiginiz topraklara, RAB'bin kulu Musa'nin Seria Irmagi'nin dogusunda size verdigi topraklara dönüp oraya yerlesin."
\par 16 Önderler Yesu'ya, "Bize ne buyurduysan yapacagiz" diye karsilik verdiler, "Bizi nereye gönderirsen gidecegiz.
\par 17 Her durumda Musa'nin sözünü dinledigimiz gibi, senin sözünü de dinleyecegiz. Yeter ki, Musa'yla birlikte olmus olan Tanrin RAB seninle de birlikte olsun.
\par 18 Sözünü dinlemeyen, buyruklarina karsi gelip baskaldiran ölümle cezalandirilacaktir. Yeter ki, sen güçlü ve yürekli ol."

\chapter{2}

\par 1 Nun oglu Yesu Sittim'den gizlice iki casus gönderdi. "Gidip ülkeyi, özellikle de Eriha'yi arastirin" dedi. Böylece yola çikan casuslar, Rahav adinda bir fahisenin evine gidip geceyi orada geçirdiler.
\par 2 Bu arada Eriha Krali'na, "Ülkemizi arastirmak üzere bu gece Israil halkindan buraya adamlar geldi" diye haber verildi.
\par 3 Bunun üzerine Eriha Krali, Rahav'a, "Sana gelip evinde kalan o adamlari disari çikar" diye haber gönderdi, "Çünkü onlar ülkemizi arastirmak için geldiler."
\par 4 Iki adami saklamis olan Rahav, "Adamlarin bana geldikleri dogru" dedi, "Ama ben nereli olduklarini bilmiyordum.
\par 5 Karanlik basar basmaz, kentin kapisi kapanmak üzereyken çiktilar. Nereye gittiklerini bilmiyorum. Hemen peslerinden giderseniz yetisirsiniz."
\par 6 Aslinda kadin onlari dama çikarmis, oraya sermis oldugu keten saplarinin altina gizlemisti.
\par 7 Kralin adamlariysa casuslari Seria Irmagi'nin geçitlerine giden yol boyunca kovaladilar. Onlar kentten çikar çikmaz kapi sürgülenmisti.
\par 8 Damdaki adamlar yatmadan önce kadin yanlarina çikti.
\par 9 "RAB'bin bu ülkeyi size verdigini biliyorum" dedi, "Sizden ötürü dehsete kapildik; ülkede yasayan herkesin korkudan dizlerinin bagi çözüldü.
\par 10 Çünkü Misir'dan çiktiginizda RAB'bin Kizildeniz'i* önünüzde nasil kuruttugunu, Seria Irmagi'nin ötesindeki Amorlu iki krala -Sihon ve Og'a- neler yaptiginizi, onlari nasil yok ettiginizi duyduk.
\par 11 Bunlari duydugumuzda korkudan dizlerimizin bagi çözüldü. Sizin korkunuzdan kimsede derman kalmadi. Çünkü Tanriniz RAB hem yukarida göklerde, hem de asagida yeryüzünde Tanri'dir.
\par 12 Size iyilik ettigim gibi, siz de aileme iyilik edeceginize lütfen RAB adina ant için. Annemi, babami, erkek ve kiz kardeslerimle ailelerini ölümden kurtarip hepimizi sag birakacaginiza iliskin bana güvenilir bir isaret verin."
\par 14 Adamlar, "Eger bu yaptiklarimizi açiga vurmazsaniz, yerinize ölmeye haziriz" dediler, "RAB bu ülkeyi bize verdiginde sana iyilik edip sözümüzü tutacagiz."
\par 15 Kent surlarinda bir evde oturan Rahav, adamlari iple pencereden asagi indirdi.
\par 16 Onlara, "Daga çikin, yoksa sizi kovalayanlarla karsilasabilirsiniz" dedi, "Onlar dönene kadar üç gün orada saklanin. Sonra yolunuza devam edersiniz."
\par 17 Adamlar Rahav'a, "Bize içirdigin andi tutmasina tutariz" dediler,
\par 18 "Ama ülkeye girdigimizde su kirmizi ipi bizi indirdigin pencereye bagla. Anneni, babani, kardeslerinle babanin bütün ev halkini yanina, kendi evine topla.
\par 19 Evinin kapisindan disariya çikan, kendi kanindan sorumlu olacak; böyle biri için sorumluluk kabul etmeyiz. Ama seninle birlikte evinde olan herhangi birine gelecek zarardan biz sorumluyuz.
\par 20 Ancak bu yaptiklarimizi açiga vurursan, içirdigin ant bizi baglamaz."
\par 21 Kadin, "Dediginiz gibi olsun" diye karsilik verdi. Onlari yola çikarip ugurladiktan sonra kirmizi ipi pencereye bagladi.
\par 22 Adamlar ayrilip daga çiktilar; kendilerini kovalayanlar dönünceye dek üç gün orada kaldilar. Kovalayanlar yol boyu onlari aradilarsa da bulamadilar.
\par 23 Iki adam geri dönmek üzere dagdan indi. Irmagi geçip Nun oglu Yesu'nun yanina vardilar ve baslarindan geçen her seyi ona anlattilar.
\par 24 Yesu'ya, "RAB gerçekten bütün ülkeyi elimize teslim etti" dediler, "Orada yasayan herkesin korkudan dizlerinin bagi çözüldü."

\chapter{3}

\par 1 Sabah erkenden kalkan Yesu, bütün Israil halkiyla birlikte Sittim'den yola çikip Seria Irmagi'na kadar geldi. Irmagi geçmeden orada konakladilar.
\par 2 Üçüncü günün sonunda ordugahi bastan basa geçen görevliler
\par 3 halka, "Levili kâhinlerin* Tanriniz RAB'bin Antlasma Sandigi'ni* yüklendiklerini gördügünüzde siz de yerinizden kalkip sandigi izleyin" diye buyurdular,
\par 4 "Böylece hangi yöne gideceginizi bileceksiniz. Çünkü daha önce bu yoldan hiç geçmediniz. Ama Antlasma Sandigi'na yaklasmayin; sandikla aranizda iki bin arsin kadar bir aralik kalsin."
\par 5 Yesu halka, "Kendinizi kutsayin" dedi, "Çünkü RAB yarin aranizda mucizeler yaratacak."
\par 6 Yesu kâhinlere, "Antlasma Sandigi'ni yüklenip halkin önüne geçin" dedi. Böylece kâhinler sandigi yüklenip halkin önünde yürümeye basladilar.
\par 7 Bu arada RAB Yesu'ya söyle dedi: "Musa'yla birlikte oldugum gibi, seninle de birlikte oldugumu anlamalari için bugün seni bütün Israil halkinin gözünde yüceltmeye baslayacagim.
\par 8 Antlasma Sandigi'ni tasiyan kâhinlere, 'Seria Irmagi'nin kiyisina varinca suda biraz ilerleyip durun diye buyruk ver."
\par 9 Yesu Israil halkina, "Yaklasin, Tanriniz RAB'bin söylediklerini dinleyin" dedikten sonra ekledi:
\par 10 "Yasayan Tanri'nin aranizda oldugunu, Kenan, Hitit*, Hiv, Periz, Girgas, Amor ve Yevus halklarini kesinlikle önünüzden sürecegini sundan anlayacaksiniz:
\par 11 Bütün yeryüzünün Egemeni'ne ait olan Antlasma Sandigi, sizden önce Seria Irmagi'ni geçecek.
\par 12 Simdi her oymaktan birer kisi olmak üzere Israil oymaklarindan kendinize on iki adam seçin.
\par 13 Bütün yeryüzünün Egemeni RAB'bin Antlasma Sandigi'ni tasiyan kâhinlerin ayaklari Seria Irmagi'nin sularina deger degmez, yukaridan asagiya akan sular kesilip bir yigin halinde birikecek."
\par 14 Halk Seria Irmagi'ni geçmek üzere konakladigi yerden yola çikti. Antlasma Sandigi'ni tasiyan kâhinler önden gidiyorlardi.
\par 15 Sandigi tasiyan kâhinler irmagin kiyisina varip suya ayak bastiklarinda -Seria Irmagi, ekin biçme zamaninda kabarir, kiyilarini basar-
\par 16 ta yukaridan gelen sular durdu, çok uzaklarda, Saretan yakininda bulunan Adam Kenti'nde bir yigin halinde yükselmeye basladi. Öyle ki, Arava -Lut- Gölü'ne akan sular tümüyle kesildi. Halk Eriha'nin karsisindan irmagi geçti.
\par 17 RAB'bin Antlasma Sandigi'ni tasiyan kâhinler, halkin tamami irmagi geçinceye dek kurumus irmak yataginin ortasinda kipirdamadan durdular. Böylece bütün Israil halki kurumus irmak yatagindan geçti.

\chapter{4}

\par 1 Halkin tümü Seria Irmagi'ni geçtikten sonra RAB Yesu'ya söyle seslendi:
\par 2 "Her oymaktan birer kisi olmak üzere halktan on iki adam seçin.
\par 3 Onlara sunu buyurun: 'Buradan, Seria Irmagi'nin ortasindan, kâhinlerin ayaklarini saglam biçimde bastiklari yerden birer tas alin. Bu taslari yaninizda götürüp geceyi geçireceginiz yere koyun."
\par 4 Böylece Yesu Israil'in her oymagindan birer kisi olmak üzere seçtigi on iki adami çagirdi.
\par 5 Onlara, "Irmagin ortasina, Tanriniz RAB'bin Antlasma Sandigi'na* kadar gidin" diye buyurdu, "Israil halkinin oymak sayisina göre her biriniz omuzuna birer tas alsin.
\par 6 Bunlar sizin için bir ani olacak. Çocuklariniz ilerde, 'Bu taslarin sizin için anlami ne? diye sorduklarinda,
\par 7 onlara diyeceksiniz ki, 'Seria Irmagi'nin sulari RAB'bin Antlasma Sandigi'nin önünde kesildi. Antlasma Sandigi irmaktan geçerken akan sular durdu. Bu taslar sonsuza dek Israil halki için bu olayin anisi olacak."
\par 8 Israilliler Yesu'nun buyrugunu yerine getirdiler. RAB'bin Yesu'ya söyledigi gibi, Israil oymaklarinin sayisina göre Seria Irmagi'nin ortasindan aldiklari on iki tasi konaklayacaklari yere götürüp bir araya yigdilar.
\par 9 Yesu ayrica Seria Irmagi'nin ortasina, Antlasma Sandigi'ni tasiyan kâhinlerin durdugu yere on iki tas diktirdi. Bu taslar bugün de oradadir.
\par 10 Böylece RAB'bin Yesu'ya, halka iletilmek üzere buyurdugu her sey yerine getirilinceye dek, sandigi tasiyan kâhinler Seria Irmagi'nin ortasinda durdular. Her sey Musa'nin Yesu'ya buyurdugu gibi yapildi. Halk da çabucak irmagi geçti.
\par 11 Halkin tümü geçtikten sonra kâhinler RAB'bin Antlasma Sandigi'yla birlikte halkin önüne geçtiler.
\par 12 Ruben ve Gad oymaklariyla Manasse oymaginin yarisi, Musa'nin kendilerine buyurdugu gibi, silahli olarak Israil halkinin önüne geçtiler.
\par 13 Böylece kirk bin kadar silahli adam savasmak üzere RAB'bin önünde Eriha ovalarina girdi.
\par 14 RAB o gün Yesu'yu bütün Israil halkinin gözünde yüceltti. Musa'ya yasami boyunca nasil saygi gösterdilerse, Yesu'ya da öyle saygi göstermeye basladilar.
\par 15 RAB Yesu'ya, "Levha Sandigi'ni tasiyan kâhinlerin Seria Irmagi'ndan çikmalarini buyur" dedi.
\par 17 Yesu da kâhinlere, "Seria Irmagi'ndan çikin" diye buyurdu.
\par 18 RAB'bin Antlasma Sandigi'ni tasiyan kâhinler Seria Irmagi'nin ortasindan ayrilip karaya ayak basar basmaz irmagin sulari eskisi gibi akmaya ve kiyilari basmaya basladi.
\par 19 Halk Seria Irmagi'ni birinci ayin* onuncu günü geçip Gilgal'da, Eriha'nin dogu sinirinda konakladi.
\par 20 Yesu irmaktan alinan on iki tasi Gilgal'a dikti.
\par 21 Sonra Israil halkina söyle dedi: "Çocuklariniz bir gün size, 'Bu taslarin anlami nedir? diye soracak olurlarsa,
\par 22 onlara, 'Israil halki Seria Irmagi'nin kurumus yatagindan geçti diyeceksiniz.
\par 23 'Tanriniz RAB Kizildeniz'i* geçisimiz boyunca önümüzde nasil kuruttuysa, Seria Irmagi'ni da geçisiniz boyunca önünüzde kuruttu.
\par 24 Öyle ki, yeryüzünün bütün halklari RAB'bin ne denli güçlü oldugunu anlasin; siz de Tanriniz RAB'den her zaman korkasiniz!"

\chapter{5}

\par 1 RAB'bin Seria Irmagi'nin sularini Israilliler'in önünde, halkin geçisi boyunca nasil kuruttugunu duyan bati yakasindaki Amorlu krallarla Akdeniz kiyisindaki Kenanli krallar, Israilliler'den ötürü can derdine düstüler; korkudan dizlerinin bagi çözüldü.
\par 2 Bu arada RAB, Yesu'ya söyle seslendi: "Kendine tastan biçaklar yap ve Israilliler'i eskisi gibi sünnet et."
\par 3 Böylece Yesu tastan yaptigi biçaklarla Israilliler'i Givat-Haaralot'ta sünnet etti.
\par 4 Bunu yapmasinin nedeni suydu: Israilliler Misir'dan çiktiklarinda savasabilecek yastaki bütün erkekler, Misir'dan çiktiktan sonra çölden geçerken ölmüslerdi.
\par 5 Misir'dan çikan erkeklerin hepsi sünnetliydi. Ama Misir'dan çiktiktan sonra yolda, çölde dogan erkeklerin hiçbiri sünnet olmamisti.
\par 6 Israilliler Misir'dan çiktiklarinda savasacak yasta olanlarin tümü ölünceye dek çölde kirk yil dolastilar. Çünkü RAB'bin sözünü dinlememislerdi. RAB bize verilmek üzere atalarimiza söz verdigi süt ve bal akan ülkeyi onlara göstermeyecegine ant içmisti.
\par 7 RAB onlarin yerine çocuklarini yasatti. Sünnetsiz* olan bu çocuklari Yesu sünnet etti. Çünkü yolda sünnet olmamislardi.
\par 8 Bütün erkekler sünnet edildikten sonra yaralari iyilesinceye dek ordugahta kaldilar.
\par 9 RAB Yesu'ya, "Misir'da ugradiginiz utanci bugün üzerinizden kaldirdim" dedi. Bugün de oraya Gilgal denmesinin nedeni budur.
\par 10 Gilgal'da, Eriha ovalarinda konaklamis olan Israil halki, ayin on dördüncü gününün aksami Fisih Bayrami'ni* kutladi.
\par 11 Bayramin ertesi günü, tam o gün, ülkenin ürününden mayasiz ekmek yaptilar ve kavrulmus basak yediler.
\par 12 Ülkenin ürününden yemeleri üzerine ertesi gün man* kesildi. Man kesilince Israilliler o yil Kenan topraklarinin ürünüyle beslendiler.
\par 13 Yesu Eriha'nin yakinindaydi. Basini kaldirinca önünde, kilicini çekmis bir adam gördü. Ona yaklasarak, "Sen bizden misin, karsi taraftan mi?" diye sordu.
\par 14 Adam, "Hiçbiri" dedi, "Ben RAB'bin ordusunun komutaniyim. Simdi geldim." O zaman Yesu yüzüstü yere kapanip ona tapindi. "Efendimin kuluna buyrugu nedir?" diye sordu.
\par 15 RAB'bin ordusunun komutani, "Çarigini çikar" dedi, "Çünkü bastigin yer kutsaldir." Yesu söyleneni yapti.

\chapter{6}

\par 1 Eriha Kenti'nin kapilari Israilliler yüzünden simsiki kapatilmisti. Ne giren vardi, ne de çikan.
\par 2 RAB Yesu'ya, "Iste Eriha'yi, kralini ve yigit savasçilarini senin eline teslim ediyorum" dedi,
\par 3 "Siz savasçilar, kentin çevresini günde bir kez olmak üzere alti gün dolanacaksiniz.
\par 4 Koç boynuzundan yapilmis birer boru tasiyan yedi kâhin sandigin önünden gitsin. Yedinci gün kentin çevresini yedi kez dolanin; bu arada kâhinler borularini çalsinlar.
\par 5 Kâhinlerin koç boynuzu borularini uzun uzun çaldiklarini isittiginizde, bütün halk yüksek sesle bagirsin. O zaman kentin surlari çökecek ve herkes bulundugu yerden dosdogru kente girecek."
\par 6 Nun oglu Yesu kâhinleri çagirip, "RAB'bin Antlasma Sandigi'ni* alin" dedi, "Yedi kâhin, ellerinde koç boynuzu borularla sandigin önünde yürüsün."
\par 7 Sonra halka, "Kalkin, kentin çevresini dolanmaya baslayin" dedi, "Silahli öncüler RAB'bin Sandigi'nin önünden gitsin."
\par 8 Yesu'nun bunlari halka söylemesinden sonra, koç boynuzu birer boru tasiyan yedi kâhin borularini çalarak RAB'bin önünde ilerlemeye basladilar. Onlari RAB'bin Antlasma Sandigi izliyordu.
\par 9 Silahli öncüler boru çalan kâhinlerin önünden, artçilar da sandigin arkasindan ilerliyor, bu arada borular çaliniyordu.
\par 10 Yesu halka su buyrugu verdi: "Savas naralari atmayin, sesinizi yükseltmeyin. 'Bagirin diyecegim güne dek agzinizdan tek bir söz çikmasin. Buyrugumu duyunca bagirin."
\par 11 Halk RAB'bin Sandigi'yla birlikte kentin çevresini bir kez dolandi, sonra ordugaha dönüp geceyi orada geçirdi.
\par 12 Ertesi sabah Yesu erkenden kalkti. Kâhinler de RAB'bin Sandigi'ni yüklendiler.
\par 13 Koç boynuzu borular tasiyan yedi kâhin RAB'bin Sandigi'nin önünde ilerliyor, bir yandan da borularini çaliyorlardi. Silahli öncüler onlarin önünden gidiyor, artçilar da RAB'bin Sandigi'ni izliyordu. Bu arada borular sürekli çaliniyordu.
\par 14 Böylece ikinci gün de kentin çevresini bir kez dolanip ordugaha döndüler. Ayni seyi alti gün yinelediler.
\par 15 Yedinci gün erkenden, safak sökerken kalkip kentin çevresini ayni sekilde yedi kez dolandilar. Kentin çevresini yalniz o gün yedi kez dolandilar.
\par 16 Kâhinler yedinci turda borularini çalinca, Yesu halka, "Bagirin! RAB kenti size verdi" dedi,
\par 17 "Kent, içindeki her seyle birlikte, RAB'be kosulsuz adanmistir. Yalniz gönderdigimiz ulaklari saklamis olan fahise Rahav'la evindekiler sag birakilacak.
\par 18 Sakin RAB'be adanan herhangi bir seye el sürmeyin. Adadiginiz seyleri alirsaniz Israil'in ordugahini felakete ve yikima sürüklersiniz.
\par 19 Bütün altinla gümüs, tunç* ve demir esya RAB'be ayrilmistir. Bunlar RAB'bin hazinesine girecek."
\par 20 Halk bagirmaya basladi, kâhinler de borularini çaldilar. Boru sesini isiten halk daha yüksek sesle bagirdi. Kentin surlari çöktü. Herkes bulundugu yerden dosdogru kente girdi. Böylece kenti ele geçirdiler.
\par 21 Kadin erkek, genç yasli, küçük ve büyük bas hayvanlardan eseklere dek, kentte ne kadar canli varsa, hepsini kiliçtan geçirip yok ettiler.
\par 22 Yesu ülkeye casus olarak gönderdigi iki adama, "O fahisenin evine gidin, ant içtiginiz gibi, kadini ve bütün yakinlarini disari çikarin" dedi.
\par 23 Eve giren genç casuslar Rahav'i, annesini, babasini, erkek kardesleriyle bütün akrabalarini ve kendisine ait olan her seyi alip Israil ordugahinin yakinina getirdiler.
\par 24 Sonra kenti içindekilerle birlikte atese verdiler. Ancak altini ve gümüsü, tunç ve demir esyayi RAB'bin Tapinagi'nin hazinesine koydular.
\par 25 Yesu fahise Rahav'a, babasinin ev halkiyla yakinlarina dokunmadi. Yesu'nun Eriha'yi arastirmak için gönderdigi ulaklari saklayan Rahav, bugün de Israilliler'in arasinda yasiyor.
\par 26 Bundan sonra Yesu söyle ant içti: "Bu kenti, Eriha'yi yeniden kurmaya kalkisan, RAB'bin lanetine ugrasin. Buna kalkisan kisi büyük oglunu kaybetme pahasina temel atacak, en küçük oglunu kaybetme pahasina da kentin kapilarini yerine takacak."
\par 27 RAB Yesu'yla birlikteydi. Yesu'nun ünü ülkenin her yanina yayildi.

\chapter{7}

\par 1 Ne var ki, Israilliler adanan esyalar konusunda RAB'be ihanet ettiler. Yahuda oymagindan Zerah oglu, Zavdi oglu, Karmi oglu Akan adanmis esyalarin bazilarini alinca, RAB Israilliler'e öfkelendi.
\par 2 Yesu, Eriha'dan Beytel'in dogusunda, Beytaven yakinindaki Ay Kenti'ne adamlar göndererek, "Gidip ülkeyi arastirin" dedi. Adamlar da gidip Ay Kenti'ni arastirdilar.
\par 3 Sonra Yesu'nun yanina dönerek ona, "Bütün halkin oraya gidip yorulmasina gerek yok" dediler, "Sayisi az olan Ay halkini yenmeye iki üç bin kisi yeter."
\par 4 Kentin üzerine yürüyen üç bin kadar Israilli, Ay halkinin önünde kaçmaya basladi.
\par 5 Ay halki onlardan otuz alti kadarini öldürdü, sag kalanlari da kentin kapisindan Sevarim'e dek kovaladi. Bayirdan asagi kaçanlari öldürdü. Korkudan Israilliler'in dizlerinin bagi çözüldü.
\par 6 Bunun üzerine Yesu giysilerini yirtarak Israil'in ileri gelenleriyle birlikte basindan asagi toprak döküp RAB'bin Sandigi'nin önünde yüzüstü yere kapandi ve aksama dek bu durumda kaldi.
\par 7 Ardindan söyle dedi: "Ey Egemen RAB, bizi Amorlular'in eline teslim edip yok etmek için mi Seria Irmagi'ndan geçirdin? Keske halimize razi olup irmagin ötesinde kalsaydik.
\par 8 Ya Rab, Israil halki dönüp düsmanlarinin önünden kaçtiktan sonra ben ne diyebilirim!
\par 9 Kenanlilar ve ülkede yasayan öbür halklar bunu duyunca çevremizi kusatacak, adimizi yeryüzünden silecekler. Ya sen, ya Rab, kendi yüce adin için ne yapacaksin?"
\par 10 RAB Yesu'ya söyle karsilik verdi: "Ayaga kalk! Neden böyle yüzüstü yere kapaniyorsun?
\par 11 Israilliler günah islediler. Onlarla yaptigim ve yerine getirmelerini buyurdugum antlasmayi bozdular. Kosulsuz adanmis esyalarin bir kismini çalip kendi esyalari arasina gizlediler ve yalan söylediler.
\par 12 Iste bu yüzden Israilliler düsmana karsi tutunamiyor, arkalarini dönüp düsmanlarinin önünden kaçiyor. Çünkü lanete ugradilar. Sizde bulunan adanmis esyalari yok etmezseniz, artik sizinle birlikte olmayacagim.
\par 13 Kalk, halki kutsa ve onlara de ki, 'Kendinizi yarin için kutsayin. Çünkü Israil'in Tanrisi RAB söyle diyor: Ey Israil, adanmis esyalarin bir kismini aldiniz. Bunlari yok etmedikçe düsmanlarinizin karsisinda dayanamazsiniz.
\par 14 Sabah olunca oymak oymak dizilip sirayla öne çikacaksiniz. RAB'bin belirleyecegi oymak, boy boy öne çikacak. RAB'bin belirleyecegi boy, aile aile öne çikacak. Yine RAB'bin belirleyecegi ailenin erkekleri teker teker öne çikacak.
\par 15 Adanmis esyalari aldigi belirlenen kisi, kendisine ait her seyle birlikte atese atilacak. Çünkü RAB'bin Antlasmasi'ni bozup Israil'de igrenç bir günah isledi."
\par 16 Sabah erkenden kalkan Yesu, Israil halkini oymak oymak öne çikardi. Bunlardan Yahuda oymagi belirlendi.
\par 17 Yahuda boylarini teker teker öne çikardiginda, Zerah boyu belirlendi. Zerahlilar aile aile öne çikarildiginda Zavdi ailesi belirlendi.
\par 18 Zavdi ailesinin erkekleri teker teker öne çikarildiginda Yahuda oymagindan Zerah oglu, Zavdi oglu, Karmi oglu Akan belirlendi.
\par 19 O zaman Yesu Akan'a, "Oglum" dedi, "Israil'in Tanrisi RAB'bin hakki için dogruyu söyle, ne yaptin, söyle bana, benden gizleme."
\par 20 Akan, "Dogru" diye karsilik verdi, "Israil'in Tanrisi RAB'be karsi günah isledim. Yaptigim su:
\par 21 Ganimetin içinde Sinar isi güzel bir kaftan, iki yüz sekel gümüs, elli sekel agirliginda bir külçe altin görünce dayanamayip aldim. En altta gümüs olmak üzere, tümünü çadirimin ortasinda topraga gömdüm."
\par 22 Yesu'nun görevlendirdigi adamlar hemen çadira kostular. Gömülmüs esyalari orada buldular. Gümüs en alttaydi.
\par 23 Tümünü çadirdan çikardilar, Yesu'ya ve Israil halkina getirip RAB'bin önünde yere serdiler.
\par 24 Yesu ile Israil halki, Zerah oglu Akan'i, gümüsü, altin külçeyi, kaftani, Akan'in ogullariyla kizlarini, sigir ve davarlariyla esegini, çadiriyla bütün esyalarini alip Akor Vadisi'ne götürdüler.
\par 25 Yesu Akan'a, "Bizi neden bu felakete sürükledin?" dedi, "RAB de bugün seni felakete sürükleyecek." Ardindan bütün Israil halki Akan'i tasa tuttu; kendisine ait ne varsa taslayip yakti.
\par 26 Akan'in üzerine taslardan büyük bir yigin yaptilar. Bu yigin bugün de duruyor. Bunun üzerine RAB'bin öfkesi dindi. Oranin bugün de Akor Vadisi diye anilmasinin nedeni budur.

\chapter{8}

\par 1 RAB Yesu'ya, "Korkma, yilma" dedi, "Bütün savasçilarini yanina alip Ay Kenti'nin üzerine yürü. Ay Krali'ni, halkini ve kenti bütün topraklariyla birlikte sana teslim ediyorum.
\par 2 Eriha'ya ve kralina ne yaptiysan, Ay Kenti'ne ve kralina da aynisini yap. Ama mal ve hayvanlardan olusan ganimeti kendinize ayirin. Kentin gerisinde pusu kur."
\par 3 Böylece Yesu bütün savasçilariyla birlikte Ay Kenti'nin üzerine yürümeye hazirlandi. Seçtigi otuz bin yigit savasçiyi geceleyin yola çikarirken
\par 4 onlara söyle buyurdu: "Gidip kentin gerisinde pusuya yatin. Kentin çok uzaginda durmayin. Hepiniz her an hazir olun.
\par 5 Ben yanimdaki halkla birlikte kente yaklasacagim. Bir önceki gibi, düsman kentten çikip üzerimize gelince, önlerinde kaçar gibi yapip
\par 6 onlari kentten uzaklastirincaya dek ardimizdan sürükleyecegiz. Önceki gibi onlardan kaçtigimizi sanacaklar. Biz kaçar gibi yaparken,
\par 7 siz de pusu kurdugunuz yerden çikip kenti ele geçirirsiniz. Tanrimiz RAB orayi elinize teslim edecek.
\par 8 Kenti ele geçirince atese verin. RAB'bin buyruguna göre hareket edin. Iste buyrugum budur."
\par 9 Ardindan Yesu onlari yolcu etti. Adamlar gidip Beytel ile Ay Kenti arasinda, Ay Kenti'nin batisinda pusuya yattilar. Yesu ise geceyi halkla birlikte geçirdi.
\par 10 Yesu sabah erkenden kalkarak halki topladi. Sonra kendisi ve Israil'in ileri gelenleri önde olmak üzere Ay Kenti'ne dogru yola çiktilar.
\par 11 Yesu, yanindaki bütün savasçilarla kentin üzerine yürüdü. Yaklasip kentin kuzeyinde ordugah kurdular. Kentle aralarinda bir vadi vardi.
\par 12 Yesu bes bin kisi kadar bir güce Beytel ile Ay Kenti arasinda, kentin batisinda pusu kurdurdu.
\par 13 Ardindan hem kuzeyde ordugah kuranlar, hem batida pusuya yatanlar savas düzenine girdiler. Yesu o gece vadide ilerledi.
\par 14 Bunu gören Ay Krali, kent halkiyla birlikte sabah erkenden kalkti. Zaman yitirmeden, Israilliler'e karsi savasmak üzere Arava bölgesinin karsisinda belirlenen yere çikti. Ne var ki, kentin gerisinde kendisine karsi kurulan pusudan habersizdi.
\par 15 Yesu ile yanindaki Israilliler, kent halki önünde bozguna ugramis gibi, çöle dogru kaçmaya basladilar.
\par 16 Kentteki bütün halk Israilliler'i kovalamaya çagrildi. Ama Yesu'yu kovalarken kentten uzaklastilar.
\par 17 Ay Kenti'yle Beytel'den Israilliler'i kovalamaya çikmayan tek kisi kalmamisti. Israilliler'i kovalamaya çikarlarken kent kapilarini açik biraktilar.
\par 18 RAB Yesu'ya, "Elindeki palayi Ay Kenti'ne dogru uzat; orayi senin eline teslim ediyorum" dedi. Yesu elindeki palayi kente dogru uzatti.
\par 19 Elini uzatir uzatmaz, pusudakiler yerlerinden firlayip kente girdiler; kenti ele geçirip hemen atese verdiler.
\par 20 Kentliler arkalarina dönüp bakinca, yanan kentten göklere yükselen dumani gördüler. Çöle dogru kaçan Israilliler de geri dönüp onlara saldirinca artik kaçacak hiçbir yerleri kalmadi.
\par 21 Pusuya yatmis olanlarin kenti ele geçirdigini, kentten dumanlar yükseldigini gören Yesu ile yanindaki Israilliler, geri dönüp Ay halkina saldirdilar.
\par 22 Kenti ele geçirenler de çikip saldiriya katilinca, kent halki iki yönden gelen Israilliler'in ortasinda kaldi. Israilliler tek canli birakmadan hepsini öldürdüler.
\par 23 Sag olarak tutsak aldiklari Ay Krali'ni Yesu'nun önüne çikardilar.
\par 24 Israilliler Ay Kenti'nden çikip kendilerini kirsal alanlarda ve çölde kovalayanlarin hepsini kiliçtan geçirdikten sonra kente dönüp geri kalanlari da kiliçtan geçirdiler.
\par 25 O gün Ay halkinin tümü öldürüldü. Öldürülenlerin toplami, kadin erkek, on iki bin kisiydi.
\par 26 Yesu kentte yasayanlarin tümü yok edilinceye dek pala tutan elini indirmedi.
\par 27 Israilliler, RAB'bin Yesu'ya verdigi buyruk uyarinca, kentin yalniz hayvanlariyla mallarini yagmaladilar.
\par 28 Ardindan Yesu Ay Kenti'ni atese verdi, yakip yikip viraneye çevirdi. Yikintilari bugün de duruyor.
\par 29 Ay Krali'ni agaca asip aksama dek orada birakan Yesu, günes batarken cesedi agaçtan indirerek kent kapisinin disina attirdi. Cesedin üzerine taslardan büyük bir yigin yaptilar. Bu yigin bugün de duruyor. Kutsal Yasa Halka Okunuyor
\par 30 Bundan sonra Yesu Eval Dagi'nda Israil'in Tanrisi RAB'be bir sunak yapti.
\par 31 Sunak, RAB'bin kulu Musa'nin Israil halkina verdigi buyruk uyarinca, Musa'nin Yasa Kitabi'nda yazildigi gibi yontulmamis, demir alet degmemis taslardan yapildi. RAB'be orada yakmalik sunular* sundular, esenlik kurbanlari kestiler.
\par 32 Yesu Musa'nin Israil halkinin önünde yazmis oldugu Kutsal Yasa'nin kopyasini orada tas levhalara yazdi.
\par 33 Bütün Israilliler, ileri gelenleriyle, görevlileriyle ve hakimleriyle birlikte -yabancilar dahil- RAB'bin Antlasma Sandigi'nin* iki yaninda, yüzleri, sandigi tasiyan Levili kâhinlere dönük olarak dizildiler. Halkin yarisi sirtini Gerizim Dagi'na, öbür yarisi da Eval Dagi'na verdi. Çünkü RAB'bin kulu Musa kutsanmalari için bu sekilde durmalarini daha önce buyurmustu.
\par 34 Ardindan Yesu yasanin tümünü, kutsama ve lanetle ilgili bölümleri Yasa Kitabi'nda yazili oldugu gibi okudu.
\par 35 Böylece Yesu'nun, yabancilarin da aralarinda bulundugu kadinli, çocuklu bütün Israil topluluguna, Musa'nin buyruklarindan okumadigi tek bir söz kalmadi.

\chapter{9}

\par 1 Seria Irmagi'nin ötesinde, daglik bölgede, Sefela'da ve Lübnan'a kadar uzanan Akdeniz kiyisindaki bütün krallar -Hitit*, Amor, Kenan, Periz, Hiv ve Yevus krallari- olup bitenleri duyunca,
\par 2 Yesu'ya ve Israil halkina karsi hep birlikte savasmak için bir araya geldiler.
\par 3 Givon halki ise Yesu'nun Eriha ve Ay kentlerine yaptiklarini duyunca
\par 4 hileye basvurdu. Kendilerine elçi süsü vererek eseklerinin sirtina yipranmis heybeler, eski, yirtik ve yamali sarap tulumlari yüklediler.
\par 5 Ayaklarinda yipranmis, yamali çariklar, sirtlarinda da eski püskü giysiler vardi. Azik torbalarindaki bütün ekmekler kurumus, küflenmisti.
\par 6 Adamlar Gilgal'daki ordugaha, Yesu'nun yanina gittiler. Ona ve Israil halkina, "Uzak bir ülkeden geldik" dediler, "Bizimle bir baris antlasmasi yapmanizi istiyoruz."
\par 7 Ama Israilliler Hivliler'e, "Sizinle neden antlasma yapalim?" diye karsilik verdiler, "Belki de yakinimizda yasiyorsunuz."
\par 8 Givonlular Yesu'ya, "Biz senin kullariniz" dediler. Yesu, "Kimsiniz, nereden geliyorsunuz?" diye sordu.
\par 9 Onlar da, "Çok uzak bir ülkeden kalkip geldik" dediler. "Çünkü Tanrin RAB'bin ününü duyduk. Tanrin'la ilgili haberleri, Misir'da yaptigi her seyi,
\par 10 Seria Irmagi'nin ötesindeki Amorlu iki krala, Hesbon Krali Sihon'a ve Astarot'ta egemenlik süren Basan Krali Og'a neler yaptigini da duyduk.
\par 11 Bunun üzerine önderlerimiz ve ülkemizin bütün halki bize söyle dediler: 'Onlari karsilamak için yaniniza yiyecek alip yola çikin ve onlara, biz sizin kullariniziz; bunun için bizimle bir baris antlasmasi yapmanizi istiyoruz deyin.
\par 12 Size gelmek için yola çiktigimiz gün azik olarak evden aldigimiz su ekmekler sicacikti. Bakin simdi, kurumus, küflenmisler.
\par 13 Sarap doldurdugumuz su tulumlar yeniydi, bakin nasil siyrilip yirtilmis. Bunca yol geldigimiz için giysilerimiz ve çariklarimiz yiprandi."
\par 14 Israilliler, RAB'be danismadan Givonlular'in sundugu yiyecekleri aldilar.
\par 15 Yesu da onlari sag birakacagina söz verip onlarla bir baris antlasmasi yapti. Toplulugun önderleri de antlasmaya bagli kalacaklarina ant içtiler.
\par 16 Ne var ki, antlasmadan üç gün sonra Givonlular'in yakinda, komsu topraklarda yasadiklarini ögrendiler.
\par 17 Bunun üzerine yola çikip üç gün sonra onlarin kentlerine vardilar. Bu kentler Givon, Kefira, Beerot ve Kiryat-Yearim'di.
\par 18 Ancak Israilliler bunlara dokunmadilar. Çünkü toplulugun önderleri, Israil'in Tanrisi RAB adina ant içmislerdi. Bu yüzden topluluk önderlere karsi söylenmeye basladi.
\par 19 Önderler ise, "Biz Israil'in Tanrisi RAB adina ant içtik; bu yüzden onlara el süremeyiz" diye karsilik verdiler,
\par 20 "Ant içtigimiz için onlari sag birakacagiz; yoksa Tanri'nin gazabina ugrariz."
\par 21 Sonra halka, "Onlari sag birakalim" dediler, "Ama bütün topluluk için odun kesip su çekmekle görevlendirilsinler." Böylece önderler vermis olduklari sözü tuttular.
\par 22 Ardindan Yesu Givonlular'i çagirip, "Yakinimizda yasadiginiz halde neden çok uzaktan geldiginizi söyleyip bizi aldattiniz?" dedi,
\par 23 "Bunun için artik lanetlisiniz. Hep köle kalacaksiniz. Tanrim'in Tapinagi için odun kesip su çekeceksiniz."
\par 24 Givonlular, "Efendimiz, Tanrin RAB'bin kulu Musa'ya verdigi buyrugu duyduk" diye karsilik verdiler, "Musa'ya bütün ülkeyi size vermesini, ülkede yasayanlarin hepsini yok etmenizi buyurdugunu duyduk. Sizden çok korktuk, can korkusuyla böyle davrandik.
\par 25 Simdi senin elindeyiz. Sana göre adil ve dogru olani yap."
\par 26 Bunun üzerine Yesu onlari Israilliler'in elinden kurtardi, öldürülmelerine izin vermedi.
\par 27 O gün onlari topluluk için ve gelecekte RAB'bin seçecegi yerde yapilacak RAB'bin sunagi için odun kesip su çekmekle görevlendirdi. Bugün de bu isi yapiyorlar.

\chapter{10}

\par 1 Yerusalim* Krali Adoni-Sedek, Yesu'nun Eriha'yi ele geçirip kralini ortadan kaldirdigi gibi, Ay Kenti'ni de ele geçirip tümüyle yiktigini, kralini öldürdügünü, Givon halkinin da Israilliler'le bir baris antlasmasi yapip onlarla birlikte yasadigini duyunca,
\par 2 büyük korkuya kapildi. Çünkü Givon, krallarin yasadigi kentler gibi büyük bir kentti; Ay Kenti'nden de büyüktü ve yigit bir halki vardi.
\par 3 Bu yüzden Yerusalim Krali Adoni-Sedek, Hevron Krali Hoham, Yarmut Krali Piram, Lakis Krali Yafia ve Eglon Krali Devir'e su haberi gönderdi:
\par 4 "Gelin bana yardim edin, Givon'a saldiralim. Çünkü Givon halki Yesu ve Israil halkiyla bir baris antlasmasi yapti."
\par 5 Böylece Amorlu bes kral -Yerusalim, Hevron, Yarmut, Lakis ve Eglon krallari- ordularini topladilar, hep birlikte gidip Givon'un karsisinda ordugah kurdular; sonra saldiriya geçtiler.
\par 6 Givonlular Gilgal'da ordugahta bulunan Yesu'ya su haberi gönderdiler: "Biz kullarini yalniz birakma. Elini çabuk tutup yardimimiza gel, bizi kurtar. Çünkü daglik bölgedeki bütün Amorlu krallar bize karsi birlesti."
\par 7 Bunun üzerine Yesu bütün savasçilari ve yigit adamlariyla birlikte Gilgal'dan yola çikti.
\par 8 Bu arada RAB Yesu'ya, "Onlardan korkma" dedi, "Onlari eline teslim ediyorum. Hiçbiri sana karsi koyamayacak."
\par 9 Gilgal'dan çikip bütün gece yol alan Yesu, Amorlular'a ansizin saldirdi.
\par 10 RAB Amorlular'i Israilliler'in önünde saskina çevirdi. Israilliler de onlari Givon'da büyük bir bozguna ugrattilar; Beythoron'a çikan yol boyunca, Azeka ve Makkeda'ya dek kovalayip öldürdüler.
\par 11 RAB Israilliler'den kaçan Amorlular'in üzerine Beythoron'dan Azeka'ya inen yol boyunca gökten iri iri dolu yagdirdi. Yagan dolunun altinda can verenler, Israilliler'in kiliçla öldürdüklerinden daha çoktu.
\par 12 RAB'bin Amorlular'i Israilliler'in karsisinda bozguna ugrattigi gün Yesu halkin önünde RAB'be söyle seslendi: "Dur, ey günes, Givon üzerinde Ve ay, sen de Ayalon Vadisi'nde."
\par 13 Halk, düsmanlarindan öcünü alincaya dek günes durdu, ay da yerinde kaldi. Bu olay Yasar Kitabi'nda da yazilidir. Günes, yaklasik bir gün boyunca gögün ortasinda durdu, batmakta gecikti.
\par 14 Ne bundan önce, ne de sonra RAB'bin bir insanin dilegini isittigi o günkü gibi bir gün olmamistir. Çünkü RAB Israil'den yana savasti.
\par 15 Yesu bundan sonra Israil halkiyla birlikte Gilgal'daki ordugaha döndü.
\par 16 Amorlu bes kral kaçip Makkeda'daki bir magarada gizlenmislerdi.
\par 17 Yesu'ya, "Bes kral Makkeda'daki bir magarada gizlenirken bulundu" diye haber verildi.
\par 18 Yesu, "Magaranin agzina büyük taslar yuvarlayin, orayi korumak için adamlar görevlendirin" dedi,
\par 19 "Ama siz durmayin, düsmani kovalayin; arkadan saldirip kentlere ulasmalarina engel olun. Tanriniz RAB onlari elinize teslim etmistir."
\par 20 Yesu ve Israilliler düsmani çok agir bir yenilgiye ugratip tamamini yok ettiler. Kurtulabilenler surlu kentlere sigindi.
\par 21 Sonra bütün halk güvenlik içinde Makkeda'daki ordugaha, Yesu'nun yanina döndü. Hiç kimse agzini açip Israilliler'e karsi bir sey söyleyemedi.
\par 22 Sonra Yesu adamlarina, "Magaranin agzini açin, bes krali çikarip bana getirin" dedi.
\par 23 Onlar da bes krali -Yerusalim, Hevron, Yarmut, Lakis ve Eglon krallarini- magaradan çikarip Yesu'ya getirdiler.
\par 24 Krallar getirilince, Yesu bütün Israil halkini topladi. Savasta kendisine eslik etmis olan komutanlara, "Yaklasin, ayaklarinizi bu krallarin boyunlari üzerine koyun" dedi. Komutanlar yaklasip ayaklarini krallarin boyunlari üzerine koydular.
\par 25 Yesu onlara, "Korkmayin, yilmayin; güçlü ve yürekli olun" dedi, "RAB savasacaginiz düsmanlarin hepsini bu duruma getirecek."
\par 26 Ardindan bes krali vurup öldürdü ve her birini bir agaca asti. Aksama dek öylece agaçlara asili kaldilar.
\par 27 Yesu'nun buyrugu üzerine gün batiminda krallarin cesetlerini agaçlardan indirdiler, gizlendikleri magaraya atip magaranin agzini büyük taslarla kapadilar. Bu taslar bugün de orada duruyor.
\par 28 Yesu ayni gün Makkeda'yi aldi, kralini ve halkini kiliçtan geçirdi. Kentte tek canli birakmadi, hepsini öldürdü. Makkeda Krali'na da Eriha Krali'na yaptiginin aynisini yapti. Güneydeki Kentlerin Ele Geçirilmesi
\par 29 Yesu Israil halkiyla birlikte Makkeda'dan Livna'nin üzerine yürüyüp kente saldirdi.
\par 30 RAB kenti ve kralini Israilliler'in eline teslim etti. Yesu kentin bütün halkini kiliçtan geçirdi. Tek canli birakmadi. Kentin kralina da Eriha Krali'na yaptiginin aynisini yapti.
\par 31 Bundan sonra Yesu Israil halkiyla birlikte Livna'dan Lakis üzerine yürüdü. Kentin karsisinda ordugah kurup saldiriya geçti.
\par 32 RAB Lakis'i Israilliler'in eline teslim etti. Yesu ertesi gün kenti aldi. Livna'da yaptigi gibi, halki ve kentteki bütün canlilari kiliçtan geçirdi.
\par 33 Bu arada Gezer Krali Horam Lakis'e yardima geldi. Yesu onu ve ordusunu yenilgiye ugratti; kimseyi sag birakmaksizin hepsini öldürdü.
\par 34 Israil halkiyla birlikte Lakis'ten Eglon üzerine yürüyen Yesu, kentin karsisinda ordugah kurup saldiriya geçti.
\par 35 Kenti ayni gün ele geçirdiler. Lakis'te yaptigi gibi, halki ve kentteki bütün canlilari o gün kiliçtan geçirip yok ettiler.
\par 36 Ardindan Yesu Israil halkiyla birlikte Eglon'dan Hevron üzerine yürüyüp saldiriya geçti.
\par 37 Kenti aldilar, kralini, halkini ve köylerindeki bütün canlilari kiliçtan geçirdiler. Eglon'da yaptiklari gibi, herkesi öldürdüler; kimseyi sag birakmadilar.
\par 38 Bundan sonra Yesu Israil halkiyla birlikte geri dönüp Devir'e saldirdi.
\par 39 Kraliyla birlikte Devir'i ve köylerini alip bütün halki kiliçtan geçirdi; tek canli birakmadi, hepsini öldürdü. Hevron'a, Livna'ya ve kralina ne yaptiysa, Devir'e ve kralina da aynisini yapti.
\par 40 Böylece Yesu daglik bölge, Negev, Sefela ve dag yamaçlari dahil, bütün ülkeyi ele geçirip buralardaki krallarin tümünü yenilgiye ugratti. Hiç kimseyi esirgemedi. Israil'in Tanrisi RAB'bin buyrugu uyarinca kimseyi sag birakmadi, hepsini öldürdü.
\par 41 Kades-Barnea'dan Gazze'ye kadar, Givon'a kadar uzanan bütün Gosen bölgesini egemenligi altina aldi.
\par 42 Bütün bu krallari ve topraklarini tek bir savasta ele geçirdi. Çünkü Israil'in Tanrisi RAB Israil'den yana savasmisti.
\par 43 Ardindan Yesu Israil halkiyla birlikte Gilgal'daki ordugaha döndü.

\chapter{11}

\par 1 Olup bitenleri duyan Hasor Krali Yavin, Madon Krali Yovav'a, Simron ve Aksaf krallarina,
\par 2 daglik kuzey bölgesinde, Kinneret Gölü'nün güneyindeki Arava'da, Sefela'da ve batida Dor Kenti sirtlarindaki krallara,
\par 3 dogu ve bati bölgelerindeki Kenan, Amor, Hitit*, Periz halklarina ve daglik bölgedeki Yevuslular'la Hermon Dagi'nin etegindeki Mispa bölgesinde yasayan Hivliler'e haber gönderdi.
\par 4 Bu krallar bütün ordulariyla, kiyilarin kumu kadar sayisiz askerleriyle, çok sayidaki at ve savas arabalariyla yola çiktilar.
\par 5 Bütün bu krallar Israilliler'e karsi savasmak üzere birleserek Merom sulari kiyisina gelip hep birlikte ordugah kurdular.
\par 6 Bu arada RAB Yesu'ya, "Onlardan korkma" diye seslendi, "Onlarin hepsini yarin bu saatlerde Israil'in önünde yere serecegim. Atlarini sakatlayip savas arabalarini atese ver."
\par 7 Böylece Yesu bütün ordusuyla birlikte Merom sulari kiyisindaki krallarin üzerine beklenmedik bir anda yürüdü ve onlara saldirdi.
\par 8 RAB onlari Israilliler'in eline teslim etti. Onlari bozguna ugratan Israilliler, kaçanlari Büyük Sayda'ya, Misrefot-Mayim'e ve doguda Mispe Vadisi'ne kadar kovalayip öldürdüler; kimseyi sag birakmadilar.
\par 9 Yesu, RAB'bin kendisine buyurdugu gibi yapti, atlarini sakatladi, savas arabalarini atese verdi.
\par 10 Yesu bundan sonra geri dönüp Hasor'u ele geçirdi, Hasor Krali'ni kiliçla öldürdü. Çünkü Hasor eskiden bütün bu kralliklarin basiydi.
\par 11 Israilliler kentteki bütün canlilari kiliçtan geçirip yok ettiler. Soluk alan bir tek kisiyi esirgemediler. Ardindan Yesu Hasor'u atese verdi.
\par 12 Böylece bütün bu kentlerle krallarini ele geçirdi. RAB'bin kulu Musa'nin buyrugu uyarinca hepsini kiliçtan geçirip yok etti.
\par 13 Ancak, Israilliler, Yesu'nun atese verdigi Hasor disinda, tepe üzerinde kurulu kentlerden hiçbirini atese vermediler.
\par 14 Bu kentlerdeki bütün mal ve hayvanlari ganimet olarak aldilar, insanlarin tümünü ise kiliçtan geçirip öldürdüler; soluk alan bir tek kisiyi esirgemediler.
\par 15 RAB'bin kulu Musa RAB'den aldigi buyruklari Yesu'ya aktarmisti. Yesu bunlara uydu ve RAB'bin Musa'ya buyurduklarini eksiksiz yerine getirdi.
\par 16 Böylece Yesu, daglik bölge, bütün Negev ve Gosen bölgesi, Sefela, Arava ve Israil daglariyla bu daglarin etekleri, Seir yönünde yükselen Halak Dagi'ndan Hermon Dagi'nin altindaki Lübnan Vadisi'nde bulunan Baal-Gat'a varincaya dek bütün topraklari ele geçirdi. Buralarin krallarini yakalayip öldürdü.
\par 18 Yesu bu krallarla uzun süre savasti.
\par 19 Givon'da yasayan Hivliler disinda, Israilliler'le baris antlasmasi yapan bir kent olmadi. Israilliler öbür kentlerin hepsini savasarak aldilar.
\par 20 Çünkü onlari Israil'e karsi savasmaya kararli yapan RAB'bin kendisiydi. Böylece RAB'bin Musa'ya buyurdugu gibi, Israilliler onlara acimadi, hepsini öldürüp yok ettiler.
\par 21 Yesu bundan sonra Anaklilar'in üzerine yürüdü. Onlari daglik bölgeden, Hevron, Devir ve Anav'dan, Yahuda ve Israil'in bütün daglik bölgelerinden söküp atti. Kentleriyle birlikte onlari tümüyle yok etti.
\par 22 Israilliler'in elindeki topraklarda hiç Anakli kalmadi. Yalniz Gazze, Gat ve Asdot'ta sag kalanlar oldu.
\par 23 RAB'bin Musa'ya söyledigi gibi, Yesu bütün ülkeyi ele geçirdi ve Israil oymaklari arasinda mülk olarak bölüstürdü. Böylece savas sona erdi, ülke barisa kavustu.

\chapter{12}

\par 1 Israilliler'in bozguna ugrattigi, Seria Irmagi'nin dogusunda, Arava'nin bütün dogusu ile Arnon Vadisi'nden Hermon Dagi'na kadar topraklarini ele geçirdigi krallar sunlardir:
\par 2 Hesbon'da oturan Amorlular'in Krali Sihon: Kralligi Arnon Vadisi kiyisindaki Aroer'den -vadinin ortasindan- basliyor, Ammonlular'in siniri olan Yabbuk Irmagi'na dek uzaniyor, Gilat'in yarisini içine aliyordu. Arava bölgesinin dogusu da ona aitti. Burasi Kinneret Gölü'nden Arava -Lut- Gölü'ne uzaniyor, doguda Beytyesimot'a, güneyde de Pisga Dagi'nin yamaçlarina variyordu.
\par 4 Sag kalan Refalilar'dan, Astarot ve Edrei'de oturan Basan Krali Og:
\par 5 Kral Og, Hermon Dagi, Salka, Gesurlular'la Maakalilar'in sinirina kadar bütün Basan'i ve Hesbon Krali Sihon'un sinirina kadar uzanan Gilat'in yarisini yönetiyordu.
\par 6 RAB'bin kulu Musa'nin ve Israilliler'in yenilgiye ugrattigi krallar bunlardi. RAB'bin kulu Musa bunlarin topraklarini Ruben ve Gad oymaklariyla Manasse oymaginin yarisina mülk olarak verdi.
\par 7 Lübnan Vadisi'ndeki Baal-Gat'tan, Seir yönünde yükselen Halak Dagi'na kadar Seria Irmagi'nin batisinda bulunan topraklarin krallari -Yesu ve Israilliler'in yenilgiye ugrattigi krallari- sunlardir: -Yesu, Hitit*, Amor, Kenan, Periz, Hiv ve Yevus halklarina ait daglik bölgeyi, Sefela'yi, Arava bölgesini, dag yamaçlarini, çölü ve Negev'i Israil oymaklari arasinda mülk olarak bölüstürdü.-
\par 9 Eriha Krali, Beytel yakinindaki Ay Kenti'nin Krali,
\par 10 Yerusalim Krali, Hevron Krali,
\par 11 Yarmut Krali, Lakis Krali,
\par 12 Eglon Krali, Gezer Krali,
\par 13 Devir Krali, Geder Krali,
\par 14 Horma Krali, Arat Krali,
\par 15 Livna Krali, Adullam Krali,
\par 16 Makkeda Krali, Beytel Krali,
\par 17 Tappuah Krali, Hefer Krali,
\par 18 Afek Krali, Saron Krali,
\par 19 Madon Krali, Hasor Krali,
\par 20 Simron-Meron Krali, Aksaf Krali,
\par 21 Taanak Krali, Megiddo Krali,
\par 22 Kedes Krali, Karmel'deki Yokneam Krali,
\par 23 Dor sirtlarindaki Dor Krali, Gilgal'daki Goyim Krali
\par 24 ve Tirsa Krali. Toplam otuz bir kral.

\chapter{13}

\par 1 Yesu kocamis, yasi hayli ilerlemisti. RAB ona, "Artik yaslandin, yasin hayli ilerledi" dedi, "Ama mülk olarak alinacak daha çok toprak var.
\par 2 "Alinacak topraklar sunlardir: Bütün Filist ve Gesur bölgeleri;
\par 3 -Misir'in dogusundaki Sihor Irmagi'ndan, kuzeyde Ekron sinirlarina kadar uzanan bölge Kenanlilar'a ait sayilirdi.- Gazze, Asdot, Askelon, Gat ve Ekron adli bes Filist beyligi ve Avlilar'in topraklari;
\par 4 güneyde bütün Kenan topraklari; Afek'e, yani Amor sinirina kadar, Saydalilar'a ait olan Meara;
\par 5 Gevalilar'in topraklari; Hermon Dagi etegindeki Baal-Gat'tan Levo-Hamat'a kadar dogu yönündeki bütün Lübnan topraklari.
\par 6 Lübnan'dan Misrefot-Mayim'e dek uzanan daglik bölgede yasayanlari, bütün Saydalilar'i Israilliler'in önünden söküp atacagim. Sana buyurdugum gibi, buralari kura ile Israilliler arasinda mülk olarak bölüstür.
\par 7 "Bu topraklari simdiden dokuz oymakla Manasse oymaginin yarisi arasinda mülk olarak bölüstür." Seria Irmagi'nin Dogusundaki Topraklar
\par 8 Manasse oymaginin öbür yarisi ile Ruben ve Gad oymaklari, RAB'bin kulu Musa'nin Seria Irmagi'nin dogusundaki topraklari kendilerine vermesiyle mülkten paylarini almislardi.
\par 9 Bu topraklar sunlardir: Arnon Vadisi kiyisinda Aroer'den vadinin ortasindaki kentle Divon'a kadar uzanan Medeva Yaylasi;
\par 10 Hesbon'da egemenlik sürmüs olan Amor Krali Sihon'un Ammon sinirina kadar uzanan bütün kentleri;
\par 11 Gilat, Gesur ve Maaka topraklari, Hermon Dagi'yla Salka'ya kadar bütün Basan;
\par 12 sag kalan Refalilar'dan biri olup Astarot ve Edrei'de egemenlik sürmüs olan Kral Og'un Basan'da kalan topraklarinin tümü. Musa'nin, krallarini yenilgiye ugratip ele geçirdigi topraklar bunlardi.
\par 13 Israilliler Gesurlular'i ve Maakalilar'i topraklarindan sürmediler; bunlar bugün de Israilliler arasinda yasiyorlar.
\par 14 Musa, yalniz Levi oymagina topraktan pay vermedi. RAB'den aldigi buyruga göre, Levililer'in payi Israil'in Tanrisi RAB için yakilan sunulardi.
\par 15 Musa'nin boy sayisina göre Ruben oymagina verdigi topraklar sunlardir:
\par 16 Arnon Vadisi kiyisinda Aroer'den vadinin ortasindaki kente kadar uzanan bölgeyle Medeva'nin çevresindeki yaylanin tümü;
\par 17 Hesbon ve buna bagli yayladaki bütün kentler; Divon, Bamot-Baal, Beytbaal-Meon,
\par 18 Yahsa, Kedemot, Mefaat,
\par 19 Kiryatayim ve Sivma, vadideki tepede kurulu Seret-Sahar,
\par 20 Beytpeor, Pisga yamaçlari, Beytyesimot,
\par 21 yayladaki kentlerle Hesbon'da egemenlik sürmüs olan Amor Krali Sihon'un bütün ülkesi. Musa Sihon'u ve Sihon'un egemenligi altindaki topraklarda yasayan Midyan beylerini -Evi, Rekem, Sur, Hur ve Reva'yi- yenilgiye ugratmisti.
\par 22 Öldürülenler arasinda Israilliler'in kiliçtan geçirdigi Beor oglu falci Balam da vardi.
\par 23 Rubenogullari'nin siniri Seria Irmagi'na dayaniyordu. Rubenogullari'na, boy sayisina göre köyleriyle birlikte mülk olarak verilen kentler bunlardi.
\par 24 Musa Gad oymagina da boy sayisina göre miras verdi.
\par 25 Verdigi topraklar sunlardi: Yazer bölgesi, bütün Gilat kentleri, Rabba yakinindaki Aroer'e kadar uzanan Ammonlular'a ait topraklarin yarisi;
\par 26 Hesbon'dan Ramat-Mispe'ye ve Betonim'e, Mahanayim'den Devir sinirina kadarki bölge;
\par 27 Seria Ovasi'ndaki Beytharam, Beytnimra, Sukkot, Safon, Hesbon Krali Sihon'un topraklarindan geri kalan bölüm, Kinneret Gölü'nün güney ucuna kadar uzanan Seria Irmagi'nin dogu yakasi.
\par 28 Gadogullari'na, boy sayisina göre köyleriyle birlikte mülk olarak verilen kentler bunlardi.
\par 29 Musa, Manasse oymaginin yarisina boy sayisina göre topraktan miras vermisti.
\par 30 Bu topraklar Mahanayim'den basliyor, Basan Krali Og'un ülkesini -bütün Basan'i- ve Yair'in Basan'daki yerlesim birimlerinin tümünü, yani toplam altmis kenti,
\par 31 Gilat'in yarisini, Basan Krali Og'un egemenligindeki Astarot ve Edrei kentlerini içine aliyordu. Buralar, Manasse oglu Makir'in soyuna, boy sayisina göre Makirogullari'nin yarisina ayrilmisti.
\par 32 Musa'nin, Eriha'nin dogusunda, Seria Irmagi'nin ötesinde kalan Moav ovalarindayken bölüstürdügü topraklar bunlardir.
\par 33 Ama Levi oymagina topraktan pay vermedi. Söz verdigi gibi, onlarin mirasi Israil'in Tanrisi RAB'bin kendisidir.

\chapter{14}

\par 1 Israilliler'in Kenan'da mülk edindigi topraklara gelince, bu topraklar Kâhin Elazar, Nun oglu Yesu ve Israil oymaklarinin boy baslari tarafindan miras olarak Israilliler arasinda bölüstürülmüstür.
\par 2 RAB'bin Musa araciligiyla buyurdugu gibi, paylar dokuz oymakla bir oymagin yarisi arasinda kura ile bölüstürüldü.
\par 3 Çünkü Musa iki oymakla yarim oymagin payini Seria Irmagi'nin dogusunda vermisti. Ama onlarla birlikte Levililer'e mülkten pay vermemisti.
\par 4 Yusuf'un soyundan gelenler, Manasse ve Efrayim diye iki oymak olusturuyordu. Levililer'e de yerlesecekleri kentler ve bu kentlerin çevresinde büyük ve küçük bas hayvanlarina ayrilan otlaklar disinda topraktan pay verilmedi.
\par 5 Israilliler, RAB'bin Musa'ya verdigi buyruga göre hareket edip ülkeyi paylastilar.
\par 6 Bu arada Yahudaogullari Gilgal'da bulunan Yesu'nun yanina geldiler. Kenizli Yefunne oglu Kalev Yesu'ya söyle dedi: "RAB'bin Kades-Barnea'da Tanri adami Musa'ya senin ve benim hakkimda neler söyledigini biliyorsun.
\par 7 RAB'bin kulu Musa ülkeyi arastirmak üzere beni Kades-Barnea'dan gönderdiginde kirk yasindaydim. Gördüklerimi ona açik yüreklilikle ilettim.
\par 8 Ne var ki, benimle gelmis olan soydaslarim halki korkuya düsürdüler. Ama ben tümüyle Tanrim RAB'bin yolundan gittim.
\par 9 Bu nedenle Musa o gün, 'Tümüyle Tanrim RAB'bin yolundan gittigin için ayak bastigin topraklar sonsuza dek sana ve ogullarina mülk olacak diye ant içti.
\par 10 RAB sözünü tuttu, beni yasatti. Israilliler çölden geçerken RAB'bin Musa'ya bu sözleri söyledigi günden bu yana kirk bes yil geçti. Simdi seksen bes yasindayim.
\par 11 Bugün de Musa'nin beni gönderdigi günkü kadar güçlüyüm. O günkü gibi hâlâ savasa gidip gelecek güçteyim.
\par 12 RAB'bin o gün söz verdigi gibi, bu daglik bölgeyi simdi bana ver. Orada Anaklilar'in yasadigini ve surlarla çevrili büyük kentleri oldugunu o gün sen de duymustun. Belki RAB bana yardim eder de, O'nun dedigi gibi, onlari oradan sürerim."
\par 13 Yesu Yefunne oglu Kalev'i kutsadi ve Hevron'u ona mülk olarak verdi.
\par 14 Böylece Hevron bugün de Kenizli Yefunne oglu Kalev'in mülküdür. Çünkü o, tümüyle Israil'in Tanrisi RAB'bin yolundan gitti.
\par 15 Hevron'un eski adi Kiryat-Arba'ydi. Arba, Anaklilar'in en güçlü adaminin adiydi. Böylece savas sona erdi ve ülke barisa kavustu.

\chapter{15}

\par 1 Boy sayisina göre Yahuda oymagina verilen bölge, güneyde Edom sinirina, en güneyde de Zin Çölü'ne kadar uzaniyordu.
\par 2 Güney sinirlari, Lut Gölü'nün güney ucundaki körfezden baslayip
\par 3 Akrep Geçidi'nin güneyine, oradan da Zin Çölü'ne geçiyor, Kades-Barnea'nin güneyinden Hesron'a ve Addar'a çikiyor, oradan da Karka'ya kivriliyor,
\par 4 Asmon'u asip Misir Vadisi'ne uzaniyor ve Akdeniz'de son buluyordu. Güney sinirlari buydu.
\par 5 Dogu siniri, Lut Gölü kiyisi boyunca Seria Irmagi'nin agzina kadar uzaniyordu. Kuzey siniri, Seria Irmagi'nin göl agzindaki körfezden basliyor,
\par 6 Beythogla'ya ulasip Beytarava'nin kuzeyinden geçiyor, Ruben oglu Bohan'in tasina variyordu.
\par 7 Sinir, Akor Vadisi'nden Devir'e çikiyor, vadinin güneyinde Adummim Yokusu karsisindaki Gilgal'a dogru kuzeye yöneliyor, buradan Eyn-Semes sularina uzanarak Eyn-Rogel'e dayaniyordu.
\par 8 Sonra Ben-Hinnom Vadisi'nden geçerek Yevus Kenti'nin -Yerusalim'in- güney sirtlarina çikiyor, buradan Refaim Vadisi'nin kuzey ucunda bulunan Hinnom Vadisi'nin batisindaki dagin doruguna yükseliyor,
\par 9 oradan da Neftoah sularinin kaynagina kivriliyor, Efron Dagi'ndaki kentlere uzanarak Baala'ya -Kiryat-Yearim'e- dönüyordu.
\par 10 Baala'dan batiya, Seir Dagi'na yönelen sinir, Yearim -Kesalon- Dagi'nin kuzey sirtlari boyunca uzanarak Beytsemes'e iniyor, Timna'ya variyordu.
\par 11 Sonra Ekron'un kuzey sirtlarina uzaniyor, Sikeron'a dogru kivrilarak Baala Dagi'na ulastiktan sonra Yavneel'e çikiyor, Akdeniz'de son buluyordu.
\par 12 Bati siniri Akdeniz'in kiyilariydi. Yahudaogullari'ndan gelen boylarin çepeçevre sinirlari buydu.
\par 13 Yesu, RAB'den aldigi buyruk uyarinca, Yahuda bölgesindeki Kiryat-Arba'yi -Hevron'u- Yefunne oglu Kalev'e miras olarak verdi. Arba, Anaklilar'in atasiydi.
\par 14 Kalev, Anak'in üç torununu, onun soyundan gelen Sesay, Ahiman ve Talmay'i oradan sürdü.
\par 15 Oradan eski adi Kiryat-Sefer olan Devir Kenti halkinin üzerine yürüdü.
\par 16 Kalev, "Kiryat-Sefer halkini yenip orayi ele geçirene kizim Aksa'yi es olarak verecegim" dedi.
\par 17 Kenti Kalev'in kardesi Kenaz'in oglu Otniel ele geçirdi. Bunun üzerine Kalev kizi Aksa'yi ona es olarak verdi.
\par 18 Kiz Otniel'in yanina varinca, onu babasindan bir tarla istemeye zorladi. Kalev, eseginden inen kizina, "Bir istegin mi var?" diye sordu.
\par 19 Kiz, "Bana bir armagan ver" dedi, "Madem Negev'deki topraklari bana verdin, su kaynaklarini da ver." Böylece Kalev yukari ve asagi su kaynaklarini ona verdi.
\par 20 Boy sayisina göre Yahudaogullari oymaginin payi buydu.
\par 21 Yahudaogullari oymaginin Edom sinirlarina dogru en güneyde kalan kentleri sunlardi: Kavseel, Eder, Yagur,
\par 22 Kina, Dimona, Adada,
\par 23 Kedes, Hasor, Yitnan,
\par 24 Zif, Telem, Bealot,
\par 25 Hasor-Hadatta, Keriyot-Hesron -Hasor-
\par 26 Amam, Sema, Molada,
\par 27 Hasar-Gadda, Hesmon, Beytpelet,
\par 28 Hasar-Sual, Beer-Seva, Bizyotya,
\par 29 Baala, Iyim, Esem,
\par 30 Eltolat, Kesil, Horma,
\par 31 Ziklak, Madmanna, Sansanna,
\par 32 Levaot, Silhim, Ayin ve Rimmon; köyleriyle birlikte yirmi dokuz kent.
\par 33 Sefela'dakiler, Estaol, Sora, Asna,
\par 34 Zanoah, Eyn-Gannim, Tappuah, Enam,
\par 35 Yarmut, Adullam, Soko, Azeka,
\par 36 Saarayim, Aditayim, Gedera ve Gederotayim; köyleriyle birlikte on dört kent.
\par 37 Senan, Hadasa, Migdal-Gad,
\par 38 Dilan, Mispe, Yokteel,
\par 39 Lakis, Boskat, Eglon,
\par 40 Kabbon, Lahmas, Kitlis,
\par 41 Gederot, Beytdagon, Naama ve Makkeda; köyleriyle birlikte on alti kent.
\par 42 Livna, Eter, Asan,
\par 43 Yiftah, Asna, Nesiv,
\par 44 Keila, Akziv ve Maresa; köyleriyle birlikte dokuz kent.
\par 45 Kasaba ve köyleriyle birlikte Ekron;
\par 46 Ekron'un batisi, Asdot'un çevresindeki bütün köyler;
\par 47 kasaba ve köyleriyle birlikte Asdot; Misir Vadisi'ne ve Akdeniz'in kiyisina kadar kasaba ve köyleriyle birlikte Gazze.
\par 48 Daglik bölgede Samir, Yattir, Soko,
\par 49 Danna, Kiryat-Sanna -Devir-
\par 50 Anav, Estemo, Anim,
\par 51 Gosen, Holon ve Gilo; köyleriyle birlikte on bir kent.
\par 52 Arav, Duma, Esan,
\par 53 Yanum, Beyttappuah, Afeka,
\par 54 Humta, Kiryat-Arba -Hevron- ve Sior; köyleriyle birlikte dokuz kent.
\par 55 Maon, Karmel, Zif, Yutta,
\par 56 Yizreel, Yokdeam, Zanoah,
\par 57 Kayin, Giva ve Timna; köyleriyle birlikte on kent.
\par 58 Halhul, Beytsur, Gedor,
\par 59 Maarat, Beytanot ve Eltekon; köyleriyle birlikte alti kent.
\par 60 Kiryat-Baal -Kiryat-Yearim- ve Rabba; köyleriyle birlikte iki kent.
\par 61 Çölde Beytarava, Middin, Sekaka,
\par 62 Nivsan, Tuz Kenti ve Eyn-Gedi; köyleriyle birlikte alti kent.
\par 63 Yahudaogullari Yerusalim'de yasayan Yevuslular'i oradan çikartamadilar. Yevuslular bugün de Yerusalim'de Yahudaogullari'yla birlikte yasiyorlar.

\chapter{16}

\par 1 Kurada Yusufogullari'na düsen topraklarin sinirlari, doguda Eriha sularinin dogusundan, Eriha'daki Seria Irmagi'ndan baslayarak çöle geçiyor, Eriha'dan Beytel'in daglik bölgesine çikiyor,
\par 2 Beytel'den Luz'a geçerek Arklilar'in sinirina, Atarot'a uzaniyordu.
\par 3 Sinir batida Yafletliler'in topraklarina, Asagi Beythoron bölgesine, oradan da Gezer'e iniyor ve Akdeniz'de son buluyordu.
\par 4 Böylece Yusuf'un soyundan gelen Manasse ve Efrayim paylarini almis oldular.
\par 5 Boy sayisina göre Efrayimogullari'na pay olarak verilen topraklarin sinirlari, doguda Atrot-Addar'dan yukari Beythoron'a kadar uzanarak
\par 6 Akdeniz'e variyordu. Sinir kuzeyde Mikmetat'ta doguya, Taanat-Silo'ya dönüyor, kentin dogusundan geçip Yanoah'a uzaniyor,
\par 7 buradan Atarot ve Naara'ya iniyor, Eriha'yi asarak Seria Irmagi'na ulasiyordu.
\par 8 Sinir Tappuah'tan batiya, Kana Vadisi'ne uzanip Akdeniz'de son buluyordu. Boy sayisina göre Efrayimogullari oymaginin payi buydu.
\par 9 Ayrica Manasseogullari'na düsen payda da Efrayimogullari'na ayrilan kentler ve bunlara bagli köyler vardi.
\par 10 Ne var ki, Efrayimogullari Gezer'de yasayan Kenanlilar'i buradan sürmediler. Kenanlilar bugüne kadar Efrayimogullari arasinda yasayip onlara ücretsiz hizmet etmek zorunda kaldilar.

\chapter{17}

\par 1 Yusuf'un büyük oglu Manasse'nin oymagi için kura çekildi. -Gilat ve Basan, Manasse'nin ilk oglu Makir'e verilmisti. Çünkü Gilatlilar'in atasi olan Makir büyük bir savasçiydi.-
\par 2 Manasse soyundan gelen öbürleri -Aviezer, Helek, Asriel, Sekem, Hefer ve Semidaogullari- bu kuranin içindeydi. Bunlar boylarina göre Yusuf oglu Manasse'nin erkek çocuklariydi.
\par 3 Bunlardan Manasse oglu, Makir oglu, Gilat oglu, Hefer oglu Selofhat'in erkek çocugu olmadi; yalniz Mahla, Noa, Hogla, Milka ve Tirsa adinda kizlari vardi.
\par 4 Bunlar, Kâhin Elazar'a, Nun oglu Yesu'ya ve önderlere gidip söyle dediler: "RAB, Musa'ya erkek akrabalarimizla birlikte bize de mirastan pay verilmesini buyurdu." RAB'bin bu buyrugu üzerine Yesu, amcalariyla birlikte onlara da mirastan pay verdi.
\par 5 Böylece Manasse oymagina Seria Irmagi'nin dogusundaki Gilat ve Basan bölgelerinden baska on pay verildi.
\par 6 Çünkü Manasse'nin kiz torunlari da erkek torunlarin yanisira mirastan pay almislardi. Gilat bölgesi ise Manasse'nin öbür ogullarina verilmisti.
\par 7 Manasse siniri Aser sinirindan Sekem yakinindaki Mikmetat'a uzaniyor, buradan güneye kivrilarak Eyn-Tappuah halkinin topraklarina variyordu.
\par 8 Tappuah Kenti'ni çevreleyen topraklar Manasse'nindi. Ama Manasse sinirindaki Tappuah Kenti Efrayimogullari'na aitti.
\par 9 Sonra sinir Kana Vadisi'ne iniyordu. Vadinin güneyinde Manasse kentleri arasinda Efrayim'e ait kentler de vardi. Manasse sinirlari vadinin kuzeyi boyunca uzanarak Akdeniz'de son buluyordu.
\par 10 Güneydeki topraklar Efrayim'in, kuzeydeki topraklarsa Manasse'nindi. Böylece Manasse bölgesi Akdeniz'le, kuzeyde Aser'le ve doguda Issakar'la sinirlanmisti.
\par 11 Issakar ve Aser'e ait topraklardaki Beytsean ve köyleri, Yivleam'la köyleri, Dor, yani Dor sirtlari halkiyla köyleri, Eyn- Dor halkiyla köyleri, Taanak halkiyla köyleri, Megiddo halkiyla köyleri Manasse'ye aitti.
\par 12 Ne var ki, Manasseogullari bu kentleri tümüyle ele geçiremediler. Çünkü Kenanlilar buralarda yasamaya kararliydi.
\par 13 Israilliler güçlenince, Kenanlilar'i sürecek yerde, onlari angaryasina çalistirmaya basladilar.
\par 14 Yusufogullari Yesu'ya gelip, "Mülk olarak bize neden tek kurayla tek pay verdin?" dediler, "Çok kalabaligiz. Çünkü RAB bizi bugüne dek alabildigine çogaltti."
\par 15 Yesu, "O kadar kalabaliksaniz ve Efrayim'in daglik bölgesi size dar geliyorsa, Perizliler'in ve Refalilar'in topraklarindaki ormanlara çikip kendinize yer açin" diye karsilik verdi.
\par 16 Yusufogullari, "Daglik bölge bize yetmiyor" dediler, "Ancak hem Beytsean ve köylerinde, hem de Yizreel Vadisi'nde oturanlarin, ovada yasayan bütün Kenanlilar'in demirden savas arabalari var."
\par 17 Yesu Yusufogullari'na, Efrayim ve Manasse oymaklarina söyle dedi: "Kalabaliksiniz ve çok güçlüsünüz. Tek kuraya kalmayacaksiniz.
\par 18 Daglik bölge de sizin olacak. Orasi ormanliktir, ama agaçlari kesip açacaginiz bütün topraklar sizin olur. Kenanlilar güçlüdür, demirden savas arabalarina sahiptirler ama, yine de onlari sürersiniz."

\chapter{18}

\par 1 Ülkenin denetimini eline geçiren Israil toplulugu Silo'da bir araya geldi. Orada Bulusma Çadiri'ni kurdular.
\par 2 Ne var ki, mülkten henüz paylarini almamis yedi Israil oymagi vardi.
\par 3 Yesu Israilliler'e, "Bu uyusuklugu üzerinizden ne zaman atacaksiniz, atalarinizin Tanrisi RAB'bin size verdigi topraklari ele geçirmek için daha ne kadar bekleyeceksiniz?" dedi.
\par 4 "Her oymaktan üçer adam seçin. Onlari, ülkeyi incelemeye gönderecegim. Mülk edinecekleri yerlerin sinirlarini belirleyip kayda geçirerek yanima dönsünler.
\par 5 Bu topraklari yedi bölgeye ayirsinlar. Yahuda güney bölgesinde, Yusufogullari kuzey bölgesinde kalsin.
\par 6 Yedi bölgeyi belirleyip kayda geçirdikten sonra, sonucu bana getirin. Burada, Tanrimiz RAB'bin önünde aranizda kura çekecegim.
\par 7 Levililer'e gelince, onlarin aranizda payi yoktur; miraslari RAB için kâhinlik yapmaktir. Gad ve Ruben oymaklariyla Manasse oymaginin yarisi ise RAB'bin kulu Musa'nin Seria Irmagi'nin dogusunda kendilerine verdigi mülkü almis bulunuyorlar."
\par 8 Yesu, topraklari kayda geçirmek için yola çikmak üzere olan adamlara, "Gidip topraklari inceleyin, kayda geçirip yanima dönün" diye buyurdu, "Sonra burada, Silo'da, RAB'bin önünde sizin için kura çekecegim."
\par 9 Adamlar yola çikip ülkeyi dolastilar; kent kent, yedi bölge halinde kayda geçirdikten sonra Silo'da, ordugahta bulunan Yesu'nun yanina döndüler.
\par 10 Yesu Silo'da RAB'bin önünde onlar için kura çekti ve topraklari Israil oymaklari arasinda bölüstürdü.
\par 11 Boy sayisina göre Benyaminogullari oymagi için kura çekildi. Paylarina düsen bölge Yahudaogullari'yla Yusufogullari'nin topraklari arasinda kaliyordu.
\par 12 Topraklarinin siniri kuzeyde Seria Irmagi'ndan basliyor, Eriha'nin kuzey sirtlarina dogru yükselerek batida daglik bölgeye uzaniyor, Beytaven kirlarinda son buluyordu.
\par 13 Sinir oradan Luz'a -Beytel'e- Luz'un güney sirtlarina geçiyor, Asagi Beythoron'un güneyindeki dagin üzerinde kurulu Atrot-Addar'a iniyor,
\par 14 bölgenin batisinda Beythoron'un güneyindeki dagdan güneye dönüyor ve Yahudaogullari'na ait Kiryat-Baal -Kiryat-Yearim- Kenti'nde son buluyordu. Bu bati tarafiydi.
\par 15 Güney tarafi Kiryat-Yearim'in bati varoslarindan basliyor, Neftoah sularinin kaynagina uzaniyordu.
\par 16 Sinir buradan Refaim Vadisi'nin kuzeyindeki Ben-Hinnom Vadisi'ne bakan dagin yamaçlarina variyor, Hinnom Vadisi'ni geçip Yevus'un güney sirtlarina, oradan da Eyn-Rogel'e iniyordu.
\par 17 Kuzeye kivrilan sinir Eyn-Semes ve Adummim Yokusu'nun karsisindaki Gelilot'a çikiyor, Ruben oglu Bohan'in tasina iniyor,
\par 18 sonra Arava Vadisi'nin kuzey sirtlarindan geçip Arava'ya sarkiyor,
\par 19 buradan Beythogla'nin kuzey yamaçlarina geçiyor, Lut Gölü'nün kuzey körfezinde, Seria Irmagi'nin güney agzinda bitiyordu. Güney siniri buydu.
\par 20 Seria Irmagi dogu sinirini olusturuyordu. Boy sayisina göre Benyaminogullari'nin payina düsen mülkün sinirlari çepeçevre buydu.
\par 21 Boy sayisina göre Benyaminogullari oymaginin payina düsen kentler sunlardi: Eriha, Beythogla, Emek-Kesis,
\par 22 Beytarava, Semarayim, Beytel,
\par 23 Avvim, Para, Ofra,
\par 24 Kefar-Ammoni, Ofni, Geva; köyleriyle birlikte on iki kent.
\par 25 Givon, Rama, Beerot,
\par 26 Mispe, Kefira, Mosa,
\par 27 Rekem, Yirpeel, Tarala,
\par 28 Sela, Haelef, Yevus -Yerusalim- Givat ve Kiryat; köyleriyle birlikte on dört kent. Boy sayisina göre Benyaminogullari'nin payi buydu.

\chapter{19}

\par 1 Ikinci kura Simon'a, boy sayisina göre Simonogullari oymagina düstü. Onlarin payi Yahudaogullari'na düsen payin sinirlari içinde kaliyordu.
\par 2 Bu pay Beer-Seva ya da Seva, Molada,
\par 3 Hasar-Sual, Bala, Esem,
\par 4 Eltolat, Betul, Horma,
\par 5 Ziklak, Beytmarkavot, Hasar-Susa,
\par 6 Beytlevaot ve Saruhen'i içeriyordu. Köyleriyle birlikte toplam on üç kent.
\par 7 Ayin, Rimmon, Eter ve Asan; köyleriyle birlikte dört kent.
\par 8 Baalat-Beer, yani Negev'deki Rama'ya kadar uzanan bu kentlerin çevresindeki bütün köyler de Simonogullari'na aitti. Boy sayisina göre Simonogullari oymaginin payi buydu.
\par 9 Simonogullari'na verilen pay Yahudaogullari'nin payindan alinmisti. Çünkü Yahudaogullari'nin payi ihtiyaçlarindan fazlaydi. Böylece Simonogullari'nin payi Yahuda oymaginin sinirlari içinde kaliyordu.
\par 10 Üçüncü kura boy sayisina göre Zevulunogullari'na düstü. Topraklarinin siniri Sarit'e kadar uzaniyordu.
\par 11 Sinir batida Marala'ya dogru çikiyor, Dabbeset'e erisip Yokneam karsisindaki vadiye uzaniyor,
\par 12 Sarit'ten doguya, gün dogusuna, Kislot-Tavor sinirina dönüyor, oradan Daverat'a dayaniyor ve Yafia'ya çikiyordu.
\par 13 Buradan yine doguya, Gat-Hefer ve Et-Kasin'e geçiyor, Rimmon'a uzaniyor, Nea'ya kivriliyordu.
\par 14 Kuzey siniri buradan Hannaton'a dönüyor ve Yiftahel Vadisi'nde son buluyordu.
\par 15 Kattat, Nahalal, Simron, Yidala ve Beytlehem; köyleriyle birlikte on iki kentti.
\par 16 Boy sayisina göre Zevulunogullari'nin payi köyleriyle birlikte bu kentlerdi.
\par 17 Dördüncü kura Issakar'a, boy sayisina göre Issakarogullari'na düstü.
\par 18 Yizreel, Kesullot, Sunem,
\par 19 Hafarayim, Sion, Anaharat,
\par 20 Rabbit, Kisyon, Eves,
\par 21 Remet, Eyn-Gannim, Eyn-Hadda ve Beytpasses bu sinirlarin içinde kaliyordu.
\par 22 Sinir Tavor, Sahasima ve Beytsemes boyunca uzanarak Seria Irmagi'nda son buluyordu. Köyleriyle birlikte on alti kentti.
\par 23 Boy sayisina göre Issakarogullari oymaginin payi köyleriyle birlikte bu kentlerdi.
\par 24 Besinci kura boy sayisina göre Aserogullari oymagina düstü.
\par 25 Sinirlari içindeki kentler Helkat, Hali, Beten, Aksaf,
\par 26 Allammelek, Amat ve Misal'di. Sinir batida Karmel ve Sihor-Livnat'a erisiyordu.
\par 27 Buradan doguya, Beytdagon'a dönüyor, Zevulun siniri ve Yiftahel Vadisi boyunca uzanarak kuzeyde Beytemek ve Neiel'e ulasiyordu. Kavul'un kuzeyinden,
\par 28 Evron, Rehov, Hammon ve Kana'ya geçerek Büyük Sayda'ya kadar çikiyordu.
\par 29 Buradan Rama'ya dönüyor, sonra surlarla çevrili Sur Kenti'ne uzaniyor, Hosa'ya dönerek Akziv yöresinde, Akdeniz'de son buluyordu.
\par 30 Umma, Afek ve Rehov; köyleriyle birlikte yirmi iki kent,
\par 31 boy sayisina göre Aserogullari oymagina verilen payin içinde kaliyordu.
\par 32 Altinci kura Naftali'ye, boy sayisina göre Naftaliogullari'na düstü.
\par 33 Sinirlari Helef ve Saanannim'deki büyük mese agacindan baslayarak Adami-Nekev ve Yavneel üzerinden Lakkum'a uzaniyor, Seria Irmagi'nda son buluyordu.
\par 34 Sinir buradan batiya yöneliyor, Aznot-Tavor'dan geçerek Hukok'a erisiyordu. Güneyde Zevulun topraklari, batida Aser topraklari, doguda ise Seria Irmagi vardi.
\par 35 Surlu kentler sunlardi: Siddim, Ser, Hammat, Rakkat, Kinneret,
\par 36 Adama, Rama, Hasor,
\par 37 Kedes, Edrei, Eyn-Hasor,
\par 38 Yiron, Migdal-El, Horem, Beytanat, Beytsemes; köyleriyle birlikte toplam on dokuz kent.
\par 39 Boy sayisina göre Naftaliogullari oymaginin payi köyleriyle birlikte bu kentlerdi.
\par 40 Yedinci kura boy sayisina göre Danogullari oymagina düstü.
\par 41 Mülklerinin siniri içinde kalan kentler sunlardi: Sora, Estaol, Ir-Semes,
\par 42 Saalabbin, Ayalon, Yitla,
\par 43 Elon, Timna, Ekron,
\par 44 Elteke, Gibbeton, Baalat,
\par 45 Yehut, Bene-Berak, Gat-Rimmon,
\par 46 Me-Yarkon ve Yafa'nin karsisindaki topraklarla birlikte Rakkon.
\par 47 Topraklarini yitiren Danogullari gidip Lesem'e saldirdilar. Kenti alip halkini kiliçtan geçirdikten sonra tümüyle isgal ederek oraya yerlestiler. Atalari Dan'in anisina buraya Dan adini verdiler.
\par 48 Boy sayisina göre Danogullari oymaginin payi köyleriyle birlikte bu kentlerdi.
\par 49 Israilliler bölgelere göre topraklari bölüstürme isini bitirdikten sonra, kendi topraklarindan Nun oglu Yesu'ya pay verdiler.
\par 50 RAB'bin buyrugu uyarinca, ona istedigi kenti, Efrayim'in daglik bölgesindeki Timnat-Serah'i verdiler. Yesu kenti onarip oraya yerlesti.
\par 51 Kâhin Elazar, Nun oglu Yesu ve Israil oymaklarinin boy baslari tarafindan Silo'da RAB'bin önünde, Bulusma Çadiri'nin kapisinda kura ile pay olarak bölüstürülen topraklar bunlardi. Böylece ülkeyi bölüstürme isini tamamladilar.

\chapter{20}

\par 1 Bundan sonra RAB Yesu'ya, "Musa araciligiyla size buyurdugum gibi, Israilliler'e kendileri için siginak olacak kentler seçmelerini söyle" dedi.
\par 3 "Öyle ki, istemeyerek, kazayla birini öldüren oraya kaçsin. Sizin de öç alacak kisiden kaçip siginacak bir yeriniz olsun.
\par 4 "Bu kentlerden birine kaçan kisi, kentin kapisina gidip durumunu kent ileri gelenlerine anlatsin. Onlar da onu kente, yanlarina kabul edip kendileriyle birlikte oturacagi bir yer versinler.
\par 5 Öç almak isteyen kisi adam öldürenin pesine düserse, kent ileri gelenleri onu teslim etmesinler. Çünkü adam öldüren öldürdügü kisiye önceden kin beslemiyordu, onu istemeyerek öldürdü.
\par 6 Bu kisi toplulugun önüne çikip yargilanincaya ve o dönemde görevli baskâhin ölünceye dek o kentte kalmalidir. Ondan sonra kaçip geldigi kente, kendi evine dönebilir."
\par 7 Böylece Naftali'nin daglik bölgesinde bulunan Celile'deki Kedes'i, Efrayim'in daglik bölgesindeki Sekem'i ve Yahuda'nin daglik bölgesindeki Kiryat-Arba'yi -Hevron'u- seçtiler.
\par 8 Ayrica Seria Irmagi'nin kiyisindaki Eriha'nin dogusunda, Ruben oymaginin sinirlari içindeki kirsal bölgede bulunan Beser Kenti'ni, Gad oymaginin sinirlari içinde Gilat'taki Ramot'u, Manasse oymagi sinirlari içinde de Basan'daki Golan'i belirlediler.
\par 9 Birini kazayla öldürüp kaçan bir Israilli'nin ya da Israilliler arasinda yasayan bir yabancinin, toplulugun önünde yargilanmadan öç almak isteyenlerce öldürülmesini önlemek için belirlenen kentler bunlardi.

\chapter{21}

\par 1 Levili boy baslari, Kenan topraklarinda, Silo'da, Kâhin Elazar, Nun oglu Yesu ve Israil oymaklarinin boy baslarina giderek, "RAB, Musa araciligiyla bize oturmak için kentler, hayvanlarimiz için de otlaklar verilmesini buyurmustu" dediler.
\par 3 Bunun üzerine Israilliler RAB'bin buyrugu uyarinca kendi paylarindan Levililer'e otlaklariyla birlikte su kentleri verdiler:
\par 4 Ilk kura Kehat boylarina düstü. Levililer'den olan Kâhin Harun'un ogullarina kurayla Yahuda, Simon ve Benyamin oymaklarindan on üç kent verildi.
\par 5 Geri kalan Kehatogullari'na Efrayim, Dan ve Manasse oymaginin yarisina ait boylardan alinan on kent kurayla verildi.
\par 6 Gersonogullari'na kurayla Issakar, Aser, Naftali oymaklarina ait boylardan ve Basan'da Manasse oymaginin yarisindan alinan on üç kent verildi.
\par 7 Merariogullari'na boy sayilarina göre Ruben, Gad ve Zevulun oymaklarindan alinan on iki kent verildi.
\par 8 Böylece RAB'bin Musa araciligiyla buyurdugu gibi, Israilliler otlaklariyla birlikte bu kentleri kurayla Levililer'e verdiler.
\par 9 Yahuda, Simon oymaklarindan alinan ve asagida adlari verilen kentler,
\par 10 ilk kurayi çeken Levili Kehat boylarindan Harunogullari'na ayrildi.
\par 11 Yahuda'nin daglik bölgesinde, Anaklilar'in atasi Arba'nin adiyla anilan Kiryat-Arba'yla -Hevron'la- çevresindeki otlaklar onlara verildi.
\par 12 Kentin tarlalariyla köyleri ise Yefunne oglu Kalev'e mülk olarak verilmisti.
\par 13 Kâhin Harun'un ogullarina kazayla adam öldürenler için siginak kent seçilen Hevron, Livna,
\par 14 Yattir, Estemoa,
\par 15 Holon, Devir,
\par 16 Ayin, Yutta ve Beytsemes kentleriyle bunlarin otlaklari -iki oymaktan toplam dokuz kent- verildi.
\par 17 Benyamin oymagindan da Givon, Geva,
\par 18 Anatot, Almon ve bunlarin otlaklari, toplam dört kent verildi.
\par 19 Böylece Harun'un soyundan gelen kâhinlere otlaklariyla birlikte verilen kentlerin toplam sayisi on üçü buldu.
\par 20 Kehatogullari'ndan geri kalan Levili ailelere gelince, kurada onlara düsen kentler Efrayim oymagindan alinmisti.
\par 21 Bunlar, Efrayim daglik bölgesinde bulunan ve kazayla adam öldürenler için siginak kent seçilen Sekem, Gezer,
\par 22 Kivsayim ve Beythoron olmak üzere otlaklariyla birlikte dört kentti.
\par 23 Dan oymagindan Elteke, Gibbeton, Ayalon ve Gat-Rimmon olmak üzere otlaklariyla birlikte dört kent;
\par 25 Manasse oymaginin yarisindan da Taanak, Gat-Rimmon ve otlaklari olmak üzere iki kent alindi.
\par 26 Böylece Kehatogullari'ndan geri kalan boylara otlaklariyla birlikte verilen kentlerin toplam sayisi onu buldu.
\par 27 Levili boylardan Gersonogullari'na, Manasse oymaginin yarisina ait Basan'da kazayla adam öldürenler için siginak kent seçilen Golan ve Beestera, otlaklariyla birlikte iki kent;
\par 28 Issakar oymagindan alinan Kisyon, Daverat, Yarmut ve Eyn-Gannim olmak üzere otlaklariyla birlikte dört kent;
\par 30 Aser oymagindan alinan Misal, Avdon, Helkat ve Rehov olmak üzere otlaklariyla birlikte dört kent;
\par 32 Naftali oymagindan alinan ve kazayla adam öldürenler için siginak kent seçilen Celile'deki Kedes, Hammot-Dor, Kartan ve otlaklari olmak üzere toplam üç kent.
\par 33 Boy sayisina göre Gersonogullari'na otlaklariyla birlikte verilen kentlerin toplam sayisi on üçü buldu.
\par 34 Merariogullari boylarina, geri kalan Levililer'e, Zevulun oymagindan alinan Yokneam, Karta,
\par 35 Dimna ve Nahalal olmak üzere otlaklariyla birlikte toplam dört kent;
\par 36 Ruben oymagindan alinan Beser, Yahsa, Kedemot ve Mefaat olmak üzere otlaklariyla birlikte dört kent;
\par 38 Gad oymagindan alinan ve kazayla adam öldürenler için siginak kent seçilen Gilat'taki Ramot, Mahanayim, Hesbon ve Yazer olmak üzere otlaklariyla birlikte toplam dört kent verildi.
\par 40 Boy sayisina göre Merariogullari'na, yani Levili boylarin geri kalanlarina kurayla verilen kentlerin sayisi on ikiydi.
\par 41 Israilliler'in topraklari içinde olup otlaklariyla birlikte Levililer'e verilen kentlerin toplami kirk sekizi buluyordu.
\par 42 Bu kentlerin hepsinin çevresinde otlaklari vardi. Otlaksiz kent yoktu.
\par 43 Böylece RAB atalarina vermeye ant içtigi bütün ülkeyi Israilliler'e vermis oldu. Israilliler de ülkeyi mülk edinip buraya yerlestiler.
\par 44 RAB atalarina ant içtigi gibi, onlari her yönden rahata erdirdi. Düsmanlarindan hiçbiri onlarin önünde tutunamadi. RAB hepsini onlarin eline teslim etti.
\par 45 RAB'bin Israil halkina verdigi sözlerden hiçbiri bos çikmadi; hepsi yerine geldi.

\chapter{22}

\par 1 Bundan sonra Yesu, Ruben ve Gad oymaklariyla Manasse oymaginin yarisini topladi.
\par 2 Onlara, "RAB'bin kulu Musa'nin size buyurdugu her seyi yaptiniz" dedi, "Benim bütün buyruklarimi da yerine getirdiniz.
\par 3 Bugüne dek, bunca zaman kardeslerinizi yalniz birakmadiniz; Tanriniz RAB'bin sizi yükümlü saydigi buyrugu yerine getirdiniz.
\par 4 Görüyorsunuz, Tanriniz RAB, kardeslerinize söyledigi gibi, onlari rahata kavusturdu. Simdi kalkin, RAB'bin kulu Musa'nin, Seria Irmagi'nin ötesinde size mülk olarak verdigi topraklardaki evlerinize dönün.
\par 5 RAB'bin kulu Musa'nin size verdigi buyruklari ve Kutsal Yasa'yi yerine getirmeye çok dikkat edin. Tanriniz RAB'bi sevin, tümüyle gösterdigi yolda yürüyün, buyruklarini yerine getirin, O'na bagli kalin, O'na candan ve yürekten hizmet edin."
\par 6 Sonra onlari kutsayip yolcu etti. Onlar da evlerine döndüler.
\par 7 Musa Manasse oymaginin yarisina Basan'da toprak vermisti. Yesu da oymagin öbür yarisina Seria Irmagi'nin batisinda, öbür kardesleri arasinda toprak vermisti. Bu oymaklari kutsayip evlerine gönderirken,
\par 8 "Evlerinize büyük servetle, çok sayida hayvanla, altin, gümüs, tunç*, demir ve çok miktarda giysiyle dönün" dedi, "Düsmanlarinizdan elde ettiginiz ganimeti kardeslerinizle paylasin."
\par 9 Böylece Rubenliler'le Gadlilar ve Manasse oymaginin yarisi, Kenan topraklarindaki Silo'dan, Israilliler'in yanindan ayrildilar; RAB'bin buyrugu uyarinca, Musa araciligiyla yurt edindikleri Gilat topraklarina -kendi mülkleri olan topraklara- dönmek üzere yola çiktilar.
\par 10 Rubenlilerle Gadlilar ve Manasse oymaginin yarisi, Seria Irmagi'nin Kenan topraklarinda kalan kesimine varinca, irmak kiyisinda büyük ve gösterisli bir sunak yaptilar.
\par 11 Rubenliler'le Gadlilar ve Manasse oymaginin yarisinin Kenan sinirinda, Seria Irmagi kiyisinda, Israilliler'e ait topraklarda bir sunak yaptiklarini
\par 12 duyan Israil toplulugu, onlara karsi savasmak üzere Silo'da toplandi.
\par 13 Ardindan Israilliler Kâhin Elazar'in oglu Pinehas'i Gilat bölgesine, Rubenliler'le Gadlilar'a ve Manasse oymaginin yarisina gönderdiler.
\par 14 Israil'in her oymagindan birer temsilci olmak üzere on oymak önderini de onunla birlikte gönderdiler. Bunlarin her biri bir Israil boyunun basiydi.
\par 15 Gilat topraklarina, Rubenliler'le Gadlilar'a ve Manasse oymaginin yarisina gelen temsilciler sunlari bildirdiler:
\par 16 "RAB'bin toplulugu, 'Bugün kendinize bir sunak yaparak RAB'be baskaldirdiniz, O'nu izlemekten vazgeçtiniz diyor, 'Israil'in Tanrisi'na karsi bu hainligi nasil yaparsiniz?
\par 17 Peor'un günahi bize yetmedi mi? RAB'bin toplulugu onun yüzünden felakete ugradi. Bugüne dek kendimizi bu günahtan temizleyebilmis degiliz.
\par 18 Bugün RAB'bi izlemekten vaz mi geçiyorsunuz? Eger bugün RAB'be isyan ederseniz, O da yarin bütün Israil topluluguna öfkelenir.
\par 19 Eger size ait olan topraklar murdarsa*, RAB'bin Tapinagi'nin bulundugu RAB'be ait topraklara gelip aramizda mülk edinin. Kendinize, Tanrimiz RAB'bin sunagindan baska bir sunak yaparak RAB'be ve bize karsi isyan etmeyin.
\par 20 Zerah oglu Akan RAB'be adanan ganimete ihanet ettiginde, bütün Israil toplulugu RAB'bin öfkesine ugramadi mi? Akan'in günahi yalniz kendisini ölüme götürmekle kalmadi!"
\par 21 Rubenliler'le Gadlilar ve Manasse oymaginin yarisi, Israil boy baslarina söyle karsilik verdiler:
\par 22 "Tanrilarin Tanrisi RAB, tanrilarin Tanrisi RAB her seyi biliyor; Israil de bilecek. Eger yaptigimizi, RAB'be isyan etmek ya da O'na ihanet etmek için yaptiysak, ya RAB, bugün bizi esirgeme!
\par 23 Eger sunagi, RAB'bi izlemekten vazgeçip yakmalik sunular* ve tahil ya da esenlik sunulari* sunmak için yaptiysak, RAB bizden hesap sorsun.
\par 24 Bunu yaparken kaygimiz suydu: Ogullariniz ilerde bizim ogullarimiza, 'Israil'in Tanrisi RAB ile ne ilginiz var?
\par 25 Ey Rubenliler ve Gadlilar, RAB Seria Irmagi'ni sizinle bizim aramizda sinir yapti. Sizin RAB'de hiçbir payiniz yoktur diyebilir, ogullarimizi RAB'be tapmaktan alikoyabilirler.
\par 26 Bu nedenle, kendimize bir sunak yapalim dedik. Yakmalik sunu* ya da kurban sunmak için degil,
\par 27 yalniz sizinle bizim aramizda ve bizden sonra gelecek kusaklar arasinda bir tanik olmasi için yaptik. Böylece RAB'bin Tapinagi'nda yakmalik sunularla, kurbanlarla ve esenlik sunulariyla RAB'be tapinacagiz. Ogullariniz da ilerde bizim ogullarimiza, 'RAB'de hiçbir payiniz yok diyemeyecekler.
\par 28 Söyle düsündük: Ilerde bize ya da gelecek kusaklarimiza böyle bir sey diyecek olurlarsa, biz de, 'Atalarimizin RAB için yaptigi sunagin örnegine bakin deriz. 'Yakmalik sunu ya da kurban sunmak için degildir bu. Sizinle bizim aramizdaki birligin tanigidir.
\par 29 RAB'be isyan etmek, bugün RAB'bi izlemekten vazgeçmek, yakmalik sunu, tahil sunusu ya da kurban sunmak için Tanrimiz RAB'bin sunagindan, tapinaginin önündeki sunaktan baska bir sunak yapmak bizden uzak olsun."
\par 30 Kâhin Pinehas ve onunla birlikte olan topluluk önderleri, yani Israil'in boy baslari, Rubenliler'le Gadlilar'in ve Manasseliler'in söylediklerini duyunca hosnut kaldilar.
\par 31 Bunun üzerine Kâhin Elazar'in oglu Pinehas, Rubenliler'le Gadlilar'a ve Manasseliler'e, "Simdi RAB'bin aramizda oldugunu biliyoruz" dedi, "Çünkü O'na ihanet etmediniz. Böylece Israilliler'i O'nun elinden kurtardiniz."
\par 32 Kâhin Elazar'in oglu Pinehas ve önderler, Rubenliler'le Gadlilar'in bulundugu Gilat topraklarindan Kenan topraklarina, Israilliler'in yanina dönüp olan biteni anlattilar.
\par 33 Anlatilanlardan hosnut kalan Israilliler Tanri'ya övgüler sundular. Rubenliler'le Gadlilar'in yasadiklari topraklarin üzerine yürüyüp savasmaktan ve orayi yakip yikmaktan bir daha söz etmediler.
\par 34 Rubenliler'le Gadlilar, "Bu sunak RAB'bin Tanri olduguna sizinle bizim aramizda taniktir" diyerek sunaga "Tanik" adini verdiler.

\chapter{23}

\par 1 RAB Israil'i çevresindeki bütün düsmanlarindan kurtarip esenlige kavusturdu. Aradan uzun zaman geçmisti. Yesu kocamis, yasi hayli ilerlemisti.
\par 2 Bu nedenle ileri gelenleri, boy baslarini, hakimleri, görevlileri, bütün Israil halkini topladi. Onlara, "Kocadim, yasim hayli ilerledi" dedi,
\par 3 "Tanriniz RAB'bin sizin yarariniza bütün bu uluslara neler yaptigini gördünüz. Çünkü sizin için savasan Tanriniz RAB'di.
\par 4 Iste Seria Irmagi'ndan gün batisindaki Akdeniz'e dek yok ettigim bütün bu uluslarla birlikte, geri kalan uluslarin topraklarini da kurayla oymaklariniza mülk olarak böldüm.
\par 5 Tanriniz RAB bu uluslari önünüzden püskürtüp sürecektir. Tanriniz RAB'bin size söz verdigi gibi, onlarin topraklarini mülk edineceksiniz.
\par 6 Musa'nin Yasa Kitabi'nda yazili olan her seyi korumak ve yerine getirmek için çok güçlü olun. Yazilanlardan saga sola sapmayin.
\par 7 Aranizda kalan uluslarla hiçbir iliskiniz olmasin; ilahlarinin adini anmayin; kimseye onlarin adiyla ant içirmeyin; onlara kulluk edip tapmayin.
\par 8 Bugüne dek yaptiginiz gibi, Tanriniz RAB'be simsiki bagli kalin.
\par 9 Çünkü RAB büyük ve güçlü uluslari önünüzden sürdü. Bugüne dek hiçbiri önünüzde tutunamadi.
\par 10 Biriniz bin kisiyi kovalayacak. Çünkü Tanriniz RAB, size söyledigi gibi, yerinize savasacak.
\par 11 Bunun için Tanriniz RAB'bi sevmeye çok dikkat edin.
\par 12 Çünkü O'na sirt çevirir, sag kalip aranizda yasayan bu uluslarla birlik olur, onlara kiz verip onlardan kiz alir, onlarla oturup kalkarsaniz,
\par 13 iyi bilin ki, Tanriniz RAB bu uluslari artik önünüzden sürmeyecek. Ve sizler Tanriniz RAB'bin size verdigi bu güzel topraklardan yok oluncaya dek bu uluslar sizin için tuzak, kapan, sirtinizda kirbaç, gözlerinizde diken olacaklar.
\par 14 "Iste her insan gibi ben de bu dünyadan göçüp gitmek üzereyim. Bütün varliginizla ve yüreginizle biliyorsunuz ki, Tanriniz RAB'bin size verdigi sözlerden hiçbiri bos çikmadi; hepsi gerçeklesti, bos çikan olmadi.
\par 15 Tanriniz RAB'bin size verdigi sözlerin tümü nasil gerçeklestiyse, Tanriniz RAB verdigi bu güzel topraklardan sizi yok edene dek sözünü ettigi bütün kötülükleri de öylece basiniza getirecektir.
\par 16 Tanriniz RAB'bin size buyurdugu antlasmayi bozarsaniz, gidip baska ilahlara kulluk eder, taparsaniz, RAB'bin öfkesi size karsi alevlenecek; RAB'bin size verdigi bu güzel ülkeden çabucak yok olup gideceksiniz."

\chapter{24}

\par 1 Yesu Israil oymaklarinin tümünü Sekem'de topladiktan sonra, Israil'in ileri gelenlerini, boy baslarini, hakimlerini, görevlilerini yanina çagirdi. Hepsi gelip Tanri'nin önünde durdular.
\par 2 Yesu bütün halka, "Israil'in Tanrisi RAB söyle diyor" diye söze basladi, "'Ibrahim'in ve Nahor'un babasi Terah ve öbür atalariniz eski çaglarda Firat Irmagi'nin ötesinde yasar, baska ilahlara kulluk ederlerdi.
\par 3 Ama ben ataniz Ibrahim'i irmagin öte yakasindan alip bütün Kenan topraklarinda dolastirdim; soyunu çogalttim, ona Ishak'i verdim.
\par 4 Ishak'a da Yakup ve Esav'i verdim. Esav'a mülk edinmesi için Seir daglik bölgesini bagisladim. Yakup'la ogullari ise Misir'a gittiler.
\par 5 Ardindan Musa ile Harun'u Misir'a gönderdim. Orada yaptiklarimla Misirlilar'i felakete ugrattim; sonra sizi Misir'dan çikardim.
\par 6 Evet, atalarinizi Misir'dan çikardim; gelip denize dayandilar. Misirlilar savas arabalariyla, atlilariyla atalarinizi Kizildeniz'e* dek kovaladilar.
\par 7 Atalariniz bana yakarinca, onlarla Misirlilar'in arasina karanlik çöktürdüm. Misirlilar'i deniz sulariyla örttüm. Misir'da yaptiklarimi gözlerinizle gördünüz. "'Uzun zaman çölde yasadiniz.
\par 8 Sonra sizi Seria Irmagi'nin ötesinde yasayan Amorlular'in topraklarina götürdüm. Size karsi savastiklarinda onlari elinize teslim ettim. Topraklarini yurt edindiniz. Onlari önünüzden yok ettim.
\par 9 Moav Krali Sippor oglu Balak, Israil'e karsi savasmaya hazirlandiginda, haber gönderip Beor oglu Balam'i size lanet etmeye çagirdi.
\par 10 Ama ben Balam'i dinlemeyi reddettim. O da sizi tekrar tekrar kutsadi; böylece sizi onun elinden kurtardim.
\par 11 Sonra Seria Irmagi'ni geçip Eriha'ya geldiniz. Size karsi savasan Erihalilar'i, Amor, Periz, Kenan, Hitit*, Girgas, Hiv ve Yevus halklarini elinize teslim ettim.
\par 12 Önden gönderdigim esekarisi Amorlu iki krali önünüzden kovdu. Bu isi kiliciniz ya da yayiniz yapmadi.
\par 13 Böylece, emek vermediginiz topraklari, kurmadiginiz kentleri size verdim. Buralarda yasiyor, dikmediginiz baglardan, zeytinliklerden yiyorsunuz."
\par 14 Yesu, "Bunun için RAB'den korkun, içtenlik ve baglilikla O'na kulluk edin" diye devam etti, "Atalarinizin Firat Irmagi'nin ötesinde ve Misir'da kulluk ettikleri ilahlari atin, RAB'be kulluk edin.
\par 15 Içinizden RAB'be kulluk etmek gelmiyorsa, atalarinizin Firat Irmagi'nin ötesinde kulluk ettikleri ilahlara mi, yoksa topraklarinda yasadiginiz Amorlular'in ilahlarina mi kulluk edeceksiniz, bugün karar verin. Ben ve ev halkim RAB'be kulluk edecegiz."
\par 16 Halk, "RAB'bi birakip baska ilahlara kulluk etmek bizden uzak olsun!" diye karsilik verdi,
\par 17 "Çünkü bizi ve atalarimizi Misir'da kölelikten kurtarip oradan çikaran, gözümüzün önünde o büyük mucizeleri yaratan, bütün yolculugumuz ve uluslar arasindan geçisimiz boyunca bizi koruyan Tanrimiz RAB'dir.
\par 18 RAB bu ülkede yasayan bütün uluslari, yani Amorlular'i önümüzden kovdu. Biz de O'na kulluk edecegiz. Çünkü Tanrimiz O'dur."
\par 19 Yesu, "Ama sizler RAB'be kulluk edemeyeceksiniz" dedi, "Çünkü O kutsal bir Tanri'dir, kiskanç bir Tanri'dir. Günahlarinizi, suçlarinizi bagislamayacak.
\par 20 RAB'bi birakip yabanci ilahlara kulluk ederseniz, RAB daha önce size iyilik etmisken, bu kez size karsi döner, sizi felakete ugratip yok eder."
\par 21 Halk, "Hayir! RAB'be kulluk edecegiz" diye karsilik verdi.
\par 22 O zaman Yesu halka, "Kulluk etmek üzere RAB'bi seçtiginize siz kendiniz taniksiniz" dedi. "Evet, biz tanigiz" dediler.
\par 23 Yesu, "Öyleyse simdi aranizdaki yabanci ilahlari atin. Yüreginizi Israil'in Tanrisi RAB'be verin" dedi.
\par 24 Halk, "Tanrimiz RAB'be kulluk edip O'nun sözünü dinleyecegiz" diye karsilik verdi.
\par 25 Yesu o gün Sekem'de halk adina bir antlasma yapti. Onlar için kurallar ve ilkeler belirledi.
\par 26 Bunlari Tanri'nin Yasa Kitabi'na da geçirdi. Sonra büyük bir tas alip oraya, RAB'bin Tapinagi'nin yanindaki yabanil fistik agacinin altina dikti.
\par 27 Ardindan bütün halka, "Iste tas bize tanik olsun" dedi, "Çünkü RAB'bin bize söyledigi bütün sözleri isitti. Tanriniz'i inkâr ederseniz bu tas size karsi taniklik edecek."
\par 28 Bundan sonra Yesu halki mülk aldiklari topraklara gönderdi.
\par 29 RAB'bin kulu Nun oglu Yesu bir süre sonra yüz on yasinda öldü.
\par 30 Onu Efrayim'in daglik bölgesindeki Gaas Dagi'nin kuzeyine, kendi mülkünün sinirlari içinde kalan Timnat-Serah'a gömdüler.
\par 31 Yesu yasadikça ve Yesu'dan sonra yasayan ve RAB'bin Israil için yaptigi her seyi bilen ileri gelenler durdukça Israil halki RAB'be kulluk etti.
\par 32 Israilliler Misir'dan çikarken Yusuf'un kemiklerini de yanlarinda getirmislerdi. Bunlari Yakup'un Sekem'deki tarlasina gömdüler. Yakup bu tarlayi Sekem'in babasi Hamor'un torunlarindan yüz parça gümüse satin almisti. Burasi Yusuf soyundan gelenlerin mülkü oldu.
\par 33 Harun'un oglu Elazar ölünce, onu Efrayim'in daglik bölgesinde oglu Pinehas'a verilen tepeye gömdüler.


\end{document}