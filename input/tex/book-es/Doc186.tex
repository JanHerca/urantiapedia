\chapter{Documento 186. Poco antes de la crucifixión}
\par
%\textsuperscript{(1997.1)}
\textsuperscript{186:0.1} CUANDO Jesús y sus acusadores salieron para ver a Herodes, el Maestro se volvió hacia el apóstol Juan y le dijo: «Juan, ya no puedes hacer nada más por mí. Ve a buscar a mi madre y tráela para que me vea antes de morir». Cuando Juan escuchó la petición de su Maestro, aunque no quería dejarlo solo entre sus enemigos, se apresuró a ir a Betania, donde toda la familia de Jesús estaba reunida y esperando en la casa de Marta y María, las hermanas de Lázaro, a quien Jesús había resucitado de entre los muertos.

\par
%\textsuperscript{(1997.2)}
\textsuperscript{186:0.2} Varias veces durante la mañana, los mensajeros habían llevado a Marta y María las noticias sobre el desarrollo del juicio de Jesús. Pero la familia de Jesús no llegó a Betania hasta unos minutos antes de que llegara Juan trayendo la petición de Jesús de ver a su madre antes de ser ejecutado. Después de que Juan Zebedeo les hubiera contado todo lo que había sucedido desde el arresto de Jesús a medianoche, su madre María partió inmediatamente en compañía de Juan para ver a su hijo mayor. Cuando María y Juan llegaron a la ciudad, Jesús, acompañado por los soldados romanos que iban a crucificarlo, ya había llegado al Gólgota.

\par
%\textsuperscript{(1997.3)}
\textsuperscript{186:0.3} Cuando María, la madre de Jesús, se marchó con Juan para ver a su hijo, su hermana Rut se negó a quedarse atrás con el resto de la familia. Puesto que estaba decidida a acompañar a su madre, su hermano Judá fue con ella. El resto de la familia del Maestro permaneció en Betania bajo la dirección de Santiago, y prácticamente cada hora, los mensajeros de David Zebedeo les traían noticias sobre el desarrollo del terrible acontecimiento de la ejecución de su hermano mayor, Jesús de Nazaret.

\section*{1. El final de Judas Iscariote}
\par
%\textsuperscript{(1997.4)}
\textsuperscript{186:1.1} El juicio de Jesús ante Pilatos terminó aproximadamente a las ocho y media de este viernes por la mañana, y el Maestro fue puesto en manos de los soldados romanos que iban a crucificarlo\footnote{\textit{Jesús entregado para ser cricificado}: Mt 27:26b,31b; Mc 15:20b; Lc 23:25b; Jn 19:16.}. En cuanto los romanos tomaron posesión de Jesús, el capitán de los guardias judíos regresó con sus hombres a su cuartel general en el templo. El sumo sacerdote y sus asociados sanedristas siguieron de cerca a los guardias, y fueron directamente a su lugar habitual de reunión en la sala de piedras labradas del templo. Allí encontraron a otros muchos miembros del sanedrín que esperaban para saber lo que se había hecho con Jesús. Mientras Caifás presentaba su informe al sanedrín sobre el juicio y la condena de Jesús, Judas apareció ante ellos para reclamar su recompensa por el papel que había jugado en el arresto y la sentencia de muerte de su Maestro.

\par
%\textsuperscript{(1997.5)}
\textsuperscript{186:1.2} Todos estos judíos detestaban a Judas; sólo miraban al traidor con unos sentimientos de total desprecio. A lo largo de todo el juicio de Jesús ante Caifás y durante su aparición ante Pilatos, a Judas le había remordido la conciencia\footnote{\textit{Los remordimientos de Judas}: Mt 27:3a.} por su comportamiento traidor. Y también empezaba a perder un poco sus ilusiones sobre la recompensa que iba a recibir como pago por su traición a Jesús. No le gustaba la frialdad y la indiferencia de las autoridades judías; sin embargo, contaba con ser recompensado ampliamente por su cobarde conducta. Preveía que sería llamado ante el pleno del sanedrín y que allí escucharía sus elogios mientras le conferían los honores adecuados como prueba del gran servicio que se vanagloriaba de haber prestado a su nación. Imaginad pues la gran sorpresa de este traidor egoísta cuando un criado del sumo sacerdote le tocó en el hombro, lo llamó para que saliera de la sala y le dijo: «Judas, me han encargado que te pague por haber traicionado a Jesús. Aquí está tu recompensa». Mientras le decía esto, el criado de Caifás le entregó a Judas una bolsa que contenía treinta monedas de plata\footnote{\textit{Treinta monedas de plata}: Mt 26:15.} ---el precio corriente de un buen esclavo sano.

\par
%\textsuperscript{(1998.1)}
\textsuperscript{186:1.3} Judas se quedó atónito, confundido. Se abalanzó para entrar en la sala, pero el portero se lo impidió. Quería apelar al sanedrín, pero no quisieron recibirlo. Judas no podía creer que estos dirigentes de los judíos le habían permitido traicionar a sus amigos y a su Maestro para ofrecerle después como recompensa treinta monedas de plata. Se sentía humillado, desilusionado y totalmente abrumado. Se alejó del templo, por así decirlo, como enajenado. Como un autómata, dejó caer la bolsa de dinero en su profundo bolsillo, el mismo bolsillo en el que había transportado durante tanto tiempo la bolsa que contenía los fondos apostólicos. Y estuvo vagando por la ciudad detrás de la muchedumbre que se dirigía a presenciar las crucifixiones.

\par
%\textsuperscript{(1998.2)}
\textsuperscript{186:1.4} Judas vio a lo lejos que levantaban el travesaño donde estaba clavado Jesús; al ver esto, volvió precipitadamente al templo, apartó a la fuerza al portero y se encontró en presencia del sanedrín\footnote{\textit{Judas regresa al sanedrín}: Mt 27:3b-4a.}, que estaba reunido todavía. El traidor estaba casi sin aliento y sumamente desconcertado, pero se las arregló para balbucear estas palabras: «He pecado porque he traicionado una sangre inocente. Me habéis insultado. Me habéis ofrecido dinero como recompensa por mi servicio ---el precio de un esclavo. Me arrepiento de haber hecho esto; aquí está vuestro dinero. Quiero escapar de la culpabilidad de este acto».

\par
%\textsuperscript{(1998.3)}
\textsuperscript{186:1.5} Cuando los dirigentes de los judíos escucharon a Judas, se mofaron de él. Uno de ellos, que estaba sentado cerca de donde se encontraba Judas, le hizo señas para que se fuera de la sala, y le dijo: «Tu Maestro ya ha sido ejecutado por los romanos, y en cuanto a tu culpabilidad, ¿qué nos importa a nosotros? Ocúpate tú de ella ---y ¡fuera de aquí!»\footnote{\textit{Judas rechazado}: Mt 27:4b.}

\par
%\textsuperscript{(1998.4)}
\textsuperscript{186:1.6} Cuando dejó la sala del sanedrín, Judas sacó de la bolsa las treinta monedas de plata y las lanzó al voleo sobre el suelo del templo\footnote{\textit{Judas devuelve la plata}: Mt 27:5a.}. Cuando el traidor salió del templo, estaba casi fuera de sí. Judas estaba pasando ahora por la experiencia de comprender la verdadera naturaleza del pecado. Todo el encanto, la fascinación y la embriaguez de las malas acciones se habían desvanecido. Ahora el malhechor se encontraba solo, frente a frente con el veredicto del juicio de su alma desilusionada y decepcionada. El pecado era fascinante y aventurero mientras se cometía, pero ahora había que hacer frente a la cosecha de los hechos desnudos y poco románticos.

\par
%\textsuperscript{(1998.5)}
\textsuperscript{186:1.7} Este antiguo embajador del reino de los cielos en la Tierra caminaba ahora solo y abandonado por las calles de Jerusalén. Su desesperación era terrible y casi absoluta. Continuó caminando por la ciudad y por fuera de sus muros, hasta descender a la terrible soledad del valle de Hinom, donde subió por las rocas escarpadas; cogió el cinturón de su vestido, ató un extremo a un pequeño árbol, anudó el otro alrededor de su cuello, y se arrojó al precipicio. Antes de morir, el nudo que había atado con sus manos nerviosas se soltó, y el cuerpo del traidor se hizo trizas al caer sobre las rocas puntiagudas de abajo\footnote{\textit{El suicidio de Judas}: Mt 27:5b; Hch 1:16-18.}.

\section*{2. La actitud del Maestro}
\par
%\textsuperscript{(1999.1)}
\textsuperscript{186:2.1} Cuando Jesús fue arrestado, sabía que su trabajo en la Tierra, en la similitud de la carne mortal, había terminado. Comprendía plenamente la clase de muerte que le esperaba, y le preocupaban poco los detalles de sus supuestos juicios.

\par
%\textsuperscript{(1999.2)}
\textsuperscript{186:2.2} Delante del tribunal de los sanedristas, Jesús rehusó responder al testimonio de los testigos perjuros\footnote{\textit{Jesús rehusó defenderse}: Mt 26:60-62a; Mc 14:56-61a.}. Sólo había una pregunta que siempre atraía una respuesta, ya fuera hecha por amigos o enemigos, y era la que se refería a la naturaleza y a la divinidad de su misión en la Tierra. Cuando le preguntaban si era el Hijo de Dios, daba infaliblemente una respuesta\footnote{\textit{Jesús respondió sobre su divinidad}: Mt 26:63b-64a; 27:11; Mc 14:61b-62a; 15:2; Lc 22:67-70; 23:3; Jn 18:33-37.}. Se negó firmemente a hablar cuanto estuvo en presencia del curioso y malvado Herodes\footnote{\textit{No respondió ante Herodes}: Lc 23:9.}. Delante de Pilatos sólo habló cuando pensó que podría ayudar a Pilatos, o a alguna otra persona, a conocer mejor la verdad mediante lo que él decía. Jesús había enseñado a sus apóstoles que era inútil que echaran sus perlas a los cerdos\footnote{\textit{No había que echar perlas a los cerdos}: Mt 7:6.}, y ahora se atrevía a practicar lo que había enseñado. Su conducta en ese momento ejemplificó la paciente sumisión de la naturaleza humana unida al silencio majestuoso y a la solemne dignidad de la naturaleza divina. Estaba enteramente dispuesto a discutir con Pilatos cualquier cuestión relacionada con las acusaciones políticas presentadas contra él ---cualquier cuestión que Jesús reconocía que pertenecía a la jurisdicción del gobernador.

\par
%\textsuperscript{(1999.3)}
\textsuperscript{186:2.3} Jesús estaba convencido de que la voluntad del Padre era que se sometiera al curso natural y normal de los acontecimientos humanos, tal como las demás criaturas mortales deben hacerlo, y por eso se negó incluso a emplear sus poderes puramente humanos de elocuencia persuasiva para influir sobre el resultado de las maquinaciones de sus semejantes mortales, socialmente miopes y espiritualmente ciegos. Aunque Jesús vivió y murió en Urantia, toda su carrera humana, desde el principio hasta el fin, fue un espectáculo destinado a influir e instruir a todo el universo que había creado y sostenido sin cesar.

\par
%\textsuperscript{(1999.4)}
\textsuperscript{186:2.4} Estos judíos miopes gritaban de manera indecente pidiendo la muerte del Maestro, mientras éste permanecía allí en un silencio impresionante, contemplando la escena de la muerte de una nación ---el propio pueblo de su padre terrenal.

\par
%\textsuperscript{(1999.5)}
\textsuperscript{186:2.5} Jesús había adquirido ese tipo de carácter humano que puede conservar la serenidad y afirmar su dignidad en medio de los insultos continuos e injustificados. No se le podía intimidar. Cuando fue atacado en primer lugar por el criado de Anás, sólo había sugerido la conveniencia de llamar a unos testigos que pudieran testificar debidamente contra él.

\par
%\textsuperscript{(1999.6)}
\textsuperscript{186:2.6} Desde el principio hasta el fin de su supuesto juicio ante Pilatos, las huestes celestiales que presenciaban los acontecimientos no pudieron abstenerse de transmitir al universo la descripción de la escena de «Pilatos procesado ante Jesús».

\par
%\textsuperscript{(1999.7)}
\textsuperscript{186:2.7} Cuando compareció ante Caifás y todos los testimonios perjuros se habían derrumbado, Jesús no dudó en responder a la pregunta del sumo sacerdote, proporcionando así con su propio testimonio la base que deseaban para condenarlo por blasfemia.

\par
%\textsuperscript{(1999.8)}
\textsuperscript{186:2.8} El Maestro nunca manifestó el menor interés por los esfuerzos bien intencionados, pero poco entusiastas, de Pilatos por conseguir su liberación. Compadecía realmente a Pilatos y se esforzó sinceramente por iluminar su mente ensombrecida. Permaneció enteramente pasivo ante todos los llamamientos del gobernador romano para que los judíos retiraran sus acusaciones criminales contra él. Durante toda esta penosa prueba, se comportó con una dignidad sencilla y una majestad sin ostentación. Ni siquiera quiso criticar la insinceridad de sus supuestos asesinos cuando éstos le preguntaron si era «el rey de los judíos». Aceptó esta designación con un mínimo de explicación modificativa, sabiendo que aunque habían escogido rechazarlo, sería el último en representar para ellos un verdadero jefe nacional, incluso en el sentido espiritual.

\par
%\textsuperscript{(2000.1)}
\textsuperscript{186:2.9} Jesús habló poco durante estos juicios, pero dijo lo suficiente como para mostrar a todos los mortales el tipo de carácter humano que un hombre puede perfeccionar en asociación con Dios, y para revelar a todo el universo la manera en que Dios se puede manifestar en la vida de la criatura cuando ésta escoge verdaderamente hacer la voluntad del Padre, volviéndose así un hijo activo del Dios vivo.

\par
%\textsuperscript{(2000.2)}
\textsuperscript{186:2.10} Su amor por los mortales ignorantes se revela plenamente mediante su paciencia y su gran serenidad frente a las burlas, los golpes y las bofetadas de los toscos soldados y de los criados irreflexivos. Ni siquiera se irritó\footnote{\textit{Jesús nunca se irritaba}: Mt 26:67-68; Mc 14:65; Lc 22:63-65; Jn 19:2-3.} cuando le vendaron los ojos y le golpearon burlonamente en la cara, exclamando: «Profetiza y dinos quién te ha golpeado».

\par
%\textsuperscript{(2000.3)}
\textsuperscript{186:2.11} Pilatos estaba más cerca de la verdad de lo que podía suponer cuando, después de haber hecho azotar a Jesús, lo presentó ante la multitud exclamando: «¡He aquí al hombre!»\footnote{\textit{Hé aquí al hombre}: Jn 19:5.} En verdad, el temeroso gobernador romano poco podía imaginar que en aquel mismo momento el universo permanecía atento, contemplando esta escena única en la que su amado Soberano se sometía así a la humillación de las burlas y los golpes de sus súbditos mortales ignorantes y envilecidos. Y cuando Pilatos habló, la frase «¡He aquí a Dios y al Hombre!» resonó por todo Nebadon. Desde ese día, incalculables millones de criaturas han continuado contemplando a este hombre en todo un universo, mientras que el Dios de Havona, el gobernante supremo del universo de universos, acepta al hombre de Nazaret como que satisface el ideal de las criaturas mortales de este universo local del tiempo y del espacio. En su vida incomparable, Jesús no dejó nunca de revelar Dios al hombre. Ahora, durante estos episodios finales de su carrera mortal y de su muerte posterior, efectuó una nueva y conmovedora revelación del hombre a Dios.

\section*{3. El fiable David Zebedeo}
\par
%\textsuperscript{(2000.4)}
\textsuperscript{186:3.1} Poco después de que Jesús fuera entregado a los soldados romanos al final de la audiencia ante Pilatos, un destacamento de guardias del templo se dirigió apresuradamente a Getsemaní para dispersar o arrestar a los seguidores del Maestro. Pero mucho antes de que llegaran, estos seguidores se habían dispersado. Los apóstoles se habían retirado a unos escondites designados; los griegos se habían separado y dirigido a diversas casas de Jerusalén; los demás discípulos habían desaparecido igualmente. David Zebedeo creía que los enemigos de Jesús regresarían, de manera que trasladó enseguida unas cinco o seis tiendas hacia la parte alta de la hondonada, cerca del lugar donde el Maestro se retiraba tan a menudo para orar y adorar. Tenía la intención de ocultarse aquí y de mantener al mismo tiempo un centro, o estación coordinadora, para su servicio de mensajeros. David apenas había abandonado el campamento cuando llegaron los guardias del templo. Como no encontraron a nadie allí, se contentaron con incendiar el campamento y luego regresaron apresuradamente al templo. Al escuchar el informe de los guardias, el sanedrín se convenció de que los seguidores de Jesús estaban tan asustados y sumisos, que ya no habría ningún peligro de motín o de cualquier intento por rescatar a Jesús de las manos de sus verdugos. Por fin podían respirar tranquilos; así pues levantaron la sesión y cada cual se fue por su lado para prepararse para la Pascua.

\par
%\textsuperscript{(2000.5)}
\textsuperscript{186:3.2} Tan pronto como Pilatos entregó a Jesús a los soldados romanos para que lo crucificaran, un mensajero salió precipitadamente hacia Getsemaní para informar a David, y en menos de cinco minutos ya habían partido los corredores hacia Betsaida, Pella, Filadelfia, Sidón, Siquem, Hebrón, Damasco y Alejandría. Estos mensajeros llevaban la noticia de que Jesús estaba a punto de ser crucificado por los romanos a instancias insistentes de los dirigentes de los judíos.

\par
%\textsuperscript{(2001.1)}
\textsuperscript{186:3.3} A lo largo de todo este trágico día, y hasta que salió el último mensaje indicando que el Maestro había sido colocado en el sepulcro, David envió a los mensajeros aproximadamente cada media hora con informes para los apóstoles, los griegos y la familia terrenal de Jesús, que estaba reunida en la casa de Lázaro en Betania. Cuando los mensajeros partieron con la noticia de que Jesús había sido sepultado, David despidió a su cuerpo de corredores locales para que celebraran la Pascua y descansaran el sábado que se avecinaba, dándoles instrucciones para que comparecieran discretamente ante él el domingo por la mañana en la casa de Nicodemo, donde tenía la intención de esconderse algunos días con Andrés y Simón Pedro.

\par
%\textsuperscript{(2001.2)}
\textsuperscript{186:3.4} Este David Zebedeo, con su manera de pensar tan peculiar, era el único de los principales discípulos de Jesús que se sentía inclinado a tomar al pie de la letra y como un hecho verdadero la afirmación del Maestro de que moriría y «resucitaría al tercer día»\footnote{\textit{Resucitar al tercer día}: Mt 16:21; 17:23a; 20:19b; 27:63; Mc 8:31; 9:31; 10:34b; Lc 9:22; 18:33; 24:7,46; Jn 20:9.}. David le había escuchado una vez hacer esta predicción, y como tenía la inclinación de tomarse las cosas en sentido literal, ahora se proponía reunir a sus mensajeros el domingo por la mañana temprano en la casa de Nicodemo a fin de tenerlos cerca para difundir la noticia, en el caso de que Jesús resucitara de entre los muertos. David descubrió enseguida que ninguno de los seguidores de Jesús esperaba que regresara tan pronto de la tumba; por eso habló poco sobre su convicción, y no dijo que había movilizado a todo su ejército de mensajeros para el domingo por la mañana temprano, excepto a los corredores que habían sido enviados el viernes por la mañana a las ciudades lejanas y a los centros de creyentes.

\par
%\textsuperscript{(2001.3)}
\textsuperscript{186:3.5} Y así, estos seguidores de Jesús, dispersos por todo Jerusalén y sus alrededores, compartieron la Pascua aquella noche, y al día siguiente permanecieron recluídos.

\section*{4. Los preparativos para la crucifixión}
\par
%\textsuperscript{(2001.4)}
\textsuperscript{186:4.1} Después de que Pilatos se hubiera lavado las manos\footnote{\textit{Pilatos se lava las manos}: Mt 27:24.} delante de la multitud, tratando así de escapar a la culpabilidad de haber entregado a un hombre inocente a la crucifixión simplemente porque temía resistirse al clamor de los dirigentes de los judíos, ordenó que el Maestro fuera entregado a los soldados romanos y dio instrucciones a su capitán para que fuera crucificado inmediatamente\footnote{\textit{Pilatos: crucificadle inmediatamente}: Mt 27:26c; Mc 15:15c; Jn 19:16.}. Al hacerse cargo de Jesús, los soldados lo llevaron de nuevo al patio del pretorio, y después de quitarle el manto que Herodes le había puesto, lo vistieron con su propia ropa. Estos soldados se burlaron y se mofaron de él, pero no le infligieron nuevos castigos físicos\footnote{\textit{Se burlan de Jesús pero no lo castigan más}: Mt 27:31ab; Mc 15:20ab.}. Jesús estaba ahora solo con estos soldados romanos. Sus amigos estaban escondidos, sus enemigos se habían ido por su camino, e incluso Juan Zebedeo ya no estaba a su lado.

\par
%\textsuperscript{(2001.5)}
\textsuperscript{186:4.2} Pilatos entregó a Jesús a los soldados poco después de las ocho de la mañana, y poco antes de las nueve partieron para el lugar de la crucifixión\footnote{\textit{La hora de partir}: Mt 27:31c; Mc 15:20c.}. Durante este intervalo de más de media hora, Jesús no dijo ni una sola palabra. La actividad ejecutiva de un gran universo estaba prácticamente detenida. Gabriel y los principales dirigentes de Nebadon o bien se encontraban reunidos aquí en Urantia, o prestaban mucha atención a los informes espaciales de los arcángeles, en un esfuerzo por mantenerse informados de lo que le estaba sucediendo al Hijo del Hombre en Urantia.

\par
%\textsuperscript{(2001.6)}
\textsuperscript{186:4.3} Cuando los soldados estuvieron preparados para salir con Jesús hacia el Gólgota, habían empezado a sentirse impresionados por su insólita serenidad y su dignidad extraordinaria, por su silencio sin queja.

\par
%\textsuperscript{(2001.7)}
\textsuperscript{186:4.4} Una gran parte del retraso en partir con Jesús para el lugar de la crucifixión se debió a que el capitán decidió, a última hora, llevarse consigo a dos ladrones\footnote{\textit{Dos ladrones}: Mt 27:38; Mc 15:27; Lc 23:32; Jn 19:18.} que habían sido condenados a muerte; puesto que Jesús iba a ser crucificado aquella mañana, el capitán romano pensó que estos dos también podían morir con él, en lugar de esperar hasta el fin de las festividades de la Pascua.

\par
%\textsuperscript{(2002.1)}
\textsuperscript{186:4.5} En cuanto prepararon a los ladrones, fueron conducidos al patio, donde uno de ellos contempló a Jesús por primera vez, pero el otro le había oído hablar a menudo tanto en el templo como muchos meses antes en el campamento de Pella.

\section*{5. Relación entre la muerte de Jesús y la Pascua}
\par
%\textsuperscript{(2002.2)}
\textsuperscript{186:5.1} No existe una relación directa entre la muerte de Jesús y la Pascua judía. Es verdad que el Maestro entregó su vida carnal en este día, el día de la preparación de la Pascua judía, y alrededor de la hora en que se sacrificaban los corderos pascuales en el templo. Pero la coincidencia de estos acontecimientos no indica de ninguna manera que la muerte del Hijo del Hombre en la Tierra tenga alguna relación con el sistema de los sacrificios judío. Jesús era judío pero, como Hijo del Hombre, era un mortal de los reinos. Los acontecimientos ya narrados, que condujeron a este momento en que el Maestro iba a ser crucificado de manera inminente, son suficientes para indicar que su muerte, que se produjo aproximadamente a esta hora, fue un asunto puramente natural y manejado por los hombres.

\par
%\textsuperscript{(2002.3)}
\textsuperscript{186:5.2} Fue el hombre, y no Dios, el que planeó y ejecutó la muerte de Jesús en la cruz. Es verdad que el Padre se negó a entremeterse en la marcha de los acontecimientos humanos en Urantia, pero el Padre Paradisiaco no decretó, pidió ni exigió la muerte de su Hijo tal como se llevó a cabo en la Tierra. Es un hecho que, tarde o temprano y de alguna manera, Jesús habría tenido que despojarse de su cuerpo mortal, poniendo fin a su encarnación, pero podría haberlo hecho de muchas formas, sin tener que morir en una cruz entre dos ladrones. Todo esto fue obra del hombre, y no de Dios.

\par
%\textsuperscript{(2002.4)}
\textsuperscript{186:5.3} En la época de su bautismo, el Maestro ya había completado la parte técnica de la experiencia requerida en la Tierra y en la carne, necesaria para finalizar su séptima y última donación en el universo. En aquel mismo momento, Jesús había realizado su deber en la Tierra. Toda la vida que vivió de allí en adelante, e incluso la manera en que murió, fueron un ministerio puramente personal por su parte por el bienestar y la elevación de sus criaturas mortales en este mundo y en otros mundos.

\par
%\textsuperscript{(2002.5)}
\textsuperscript{186:5.4} El evangelio de la buena nueva de que el hombre mortal puede, por la fe, volverse espiritualmente consciente de que es hijo de Dios, no depende de la muerte de Jesús. Es verdad, en efecto, que todo este evangelio del reino\footnote{\textit{El evangelio del reino}: Mt 4:23; 9:35; 24:14; Mc 1:14-15.} ha sido enormemente iluminado por la muerte del Maestro, pero lo fue aun más por su vida.

\par
%\textsuperscript{(2002.6)}
\textsuperscript{186:5.5} Todo lo que el Hijo del Hombre dijo o hizo en la Tierra embelleció enormemente las doctrinas de la filiación con Dios y de la fraternidad entre los hombres, pero estas relaciones esenciales entre Dios y los hombres son inherentes a los hechos universales del amor de Dios por sus criaturas y de la misericordia innata de los Hijos divinos. Estas relaciones conmovedoras y divinamente hermosas entre el hombre y su Hacedor, en este mundo y en todos los demás, en todo el universo de universos, han existido desde la eternidad; y no dependen en ningún sentido de las actuaciones donadoras periódicas de los Hijos Creadores de Dios, que asumen así la naturaleza y la semejanza de las inteligencias creadas por ellos, como una parte del precio que han de pagar para adquirir finalmente la soberanía ilimitada sobre sus universos locales respectivos.

\par
%\textsuperscript{(2002.7)}
\textsuperscript{186:5.6} Antes de la vida y la muerte de Jesús en Urantia, el Padre que está en los cielos amaba al hombre mortal de la Tierra tanto como lo ama después de esta manifestación trascendente de la asociación entre el hombre y Dios. Esta poderosa operación de la encarnación del Dios de Nebadon como hombre en Urantia no podía aumentar los atributos del Padre eterno, infinito y universal, pero sí enriqueció e iluminó a todos los demás administradores y criaturas del universo de Nebadon. Aunque el Padre que está en los cielos no nos ama más debido a esta donación de Miguel, todas las demás inteligencias celestiales sí lo hacen. Y esto es así porque Jesús no solamente hizo una revelación de Dios al hombre, sino que también hizo una nueva revelación del hombre a los Dioses y a las inteligencias celestiales del universo de universos.

\par
%\textsuperscript{(2003.1)}
\textsuperscript{186:5.7} Jesús no está a punto de morir como sacrificio por el pecado\footnote{\textit{Contra la idea de la expiación del pecado}: Mt 20:28; Mc 10:45; Ro 3:25-26; 5:6; 1 Co 15:3; 2 Co 5:21; Gl 3:13; 1 Ti 2:6; Heb 2:17-18; 1 Jn 2:2; 4:10.}. No va a expiar la culpabilidad moral innata de la raza humana. La humanidad no tiene esta culpabilidad racial ante Dios. La culpabilidad es estrictamente una cuestión de pecado personal y de rebelión consciente y deliberada contra la voluntad del Padre y la administración de sus Hijos.

\par
%\textsuperscript{(2003.2)}
\textsuperscript{186:5.8} El pecado y la rebelión no tienen nada que ver con el plan fundamental de donación de los Hijos Paradisiacos de Dios, aunque nos parezca que el plan de salvación es una característica provisional del plan de donación.

\par
%\textsuperscript{(2003.3)}
\textsuperscript{186:5.9} La salvación de Dios para los mortales de Urantia habría sido exactamente igual de eficaz e infaliblemente segura si Jesús no hubiera sido ejecutado por las manos crueles de unos mortales ignorantes. Si el Maestro hubiera sido recibido favorablemente por los mortales de la Tierra y si hubiera partido de Urantia abandonando voluntariamente su vida en la carne, el hecho del amor de Dios y de la misericordia del Hijo ---el hecho de la filiación con Dios--- no hubiera sido afectado de ninguna manera. Vosotros los mortales sois los hijos de Dios, y para que esta verdad se convierta en un hecho en vuestra experiencia personal, sólo se os pide una cosa: vuestra fe nacida del espíritu.