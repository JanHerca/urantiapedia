\chapter{Documento 147. El paréntesis de la visita a Jerusalén}
\par
%\textsuperscript{(1647.1)}
\textsuperscript{147:0.1} JESÚS y los apóstoles llegaron a Cafarnaúm el miércoles 17 de marzo y pasaron dos semanas en su cuartel general de Betsaida antes de partir para Jerusalén. Durante estas dos semanas, los apóstoles enseñaron a la gente en la orilla del mar, mientras que Jesús pasó mucho tiempo a solas en las colinas, ocupado en los asuntos de su Padre. En el transcurso de este período, Jesús, acompañado de Santiago y Juan Zebedeo, hizo dos viajes secretos a Tiberiades, donde se encontraron con los creyentes y los instruyeron en el evangelio del reino.

\par
%\textsuperscript{(1647.2)}
\textsuperscript{147:0.2} Muchos miembros de la casa de Herodes creían en Jesús y asistieron a estas reuniones. La influencia de estos creyentes dentro de la familia oficial de Herodes fue la que había contribuido a que disminuyera la enemistad de este gobernador hacia Jesús. Estos creyentes de Tiberiades habían explicado plenamente a Herodes que el «reino» que Jesús proclamaba era de naturaleza espiritual, y no una aventura política. Herodes daba bastante crédito a estos miembros de su propia casa y por eso no llegó a alarmarse indebidamente por la divulgación de las noticias sobre las enseñanzas y las curaciones de Jesús. No tenía objeciones al trabajo de Jesús como sanador o instructor religioso. A pesar de la actitud favorable de muchos consejeros de Herodes, e incluso del mismo Herodes, había un grupo de subordinados suyos que estaban tan influídos por los jefes religiosos de Jerusalén, que continuaron siendo enemigos encarnizados y amenazadores de Jesús y de los apóstoles; más tarde, este grupo contribuyó mucho a impedir sus actividades públicas. El peligro más grande para Jesús residía en los dirigentes religiosos de Jerusalén, y no en Herodes. Precisamente por esta razón, Jesús y los apóstoles pasaron tanto tiempo en Galilea e hicieron allí la mayor parte de su predicación pública, en lugar de hacerlo en Jerusalén y en Judea.

\section*{1. El servidor del centurión}
\par
%\textsuperscript{(1647.3)}
\textsuperscript{147:1.1} El día antes de prepararse para ir a Jerusalén a la fiesta de la Pascua, Mangus, un centurión o capitán de la guardia romana estacionada en Cafarnaúm, fue a ver a los jefes de la sinagoga, diciendo: «Mi fiel ordenanza está enfermo y a punto de morir. ¿Podríais ir a ver a Jesús en mi nombre para suplicarle que cure a mi servidor?»\footnote{\textit{El sirviente del centurión}: Mt 8:5-6; Lc 7:1-5.} El capitán romano actuó así porque pensaba que los dirigentes judíos tendrían más influencia sobre Jesús. Así pues, los ancianos fueron a ver a Jesús y su portavoz le dijo: «Maestro, te rogamos encarecidamente que vayas a Cafarnaúm para salvar al servidor favorito del centurión romano; este capitán es digno de tu atención porque ama a nuestra nación e incluso nos ha construido la sinagoga donde has hablado tantas veces».

\par
%\textsuperscript{(1647.4)}
\textsuperscript{147:1.2} Después de haberlos escuchado, Jesús les dijo: «Iré con vosotros». Cuando llegó con ellos a la casa del centurión, y antes de que hubieran entrado en su patio, el soldado romano envió a sus amigos para que saludaran a Jesús, con la instrucción de decirle: «Señor, no te molestes en entrar en mi casa, porque no soy digno de que vengas bajo mi techo. Tampoco me he considerado digno de ir a verte; por eso te he enviado a los ancianos de tu propio pueblo. Pero sé que puedes pronunciar la palabra allí mismo donde estás y que mi servidor se curará. Porque yo mismo estoy bajo las órdenes de otros, y tengo soldados a mis órdenes, y le digo a éste que vaya, y va; le digo a otro que venga, y viene, y a mis criados que hagan esto o aquello, y lo hacen»\footnote{\textit{La fe del centurión}: Mt 8:7-9; Lc 7:6-8.}.

\par
%\textsuperscript{(1648.1)}
\textsuperscript{147:1.3} Cuando Jesús oyó estas palabras, se volvió y dijo a sus apóstoles y a los que estaban con ellos: «Me maravilla la creencia de este gentil. En verdad, en verdad os digo que no he encontrado una fe tan grande, no, ni siquiera en Israel»\footnote{\textit{Jesús maravillado de la fe}: Lc 7:9-10.}. Jesús le dio la espalda a la casa, y dijo: «Vámonos de aquí». Los amigos del centurión entraron en la casa y le contaron a Mangus lo que Jesús había dicho. A partir de aquel momento, el servidor empezó a mejorar y finalmente recuperó su salud y utilidad normales\footnote{\textit{La curación del sirviente}: Mt 8:13.}.

\par
%\textsuperscript{(1648.2)}
\textsuperscript{147:1.4} Nunca hemos sabido exactamente qué es lo que sucedió en esta ocasión. Éste es simplemente el relato del suceso; en cuanto a si los seres invisibles contribuyeron o no a la curación del servidor del centurión, eso es algo que no se reveló a los que acompañaban a Jesús. Sólo conocemos el hecho de que el servidor se recuperó por completo.

\section*{2. El viaje a Jerusalén}
\par
%\textsuperscript{(1648.3)}
\textsuperscript{147:2.1} El martes 30 de marzo, por la mañana temprano, Jesús y el grupo apostólico iniciaron su viaje a Jerusalén para la Pascua, tomando el camino del valle del Jordán. Llegaron el viernes 2 de abril por la tarde, y como de costumbre, establecieron su cuartel general en Betania. Al pasar por Jericó, se detuvieron para descansar mientras que Judas depositaba una parte de los fondos comunes en el banco de un amigo de su familia. Era la primera vez que Judas transportaba un excedente de dinero, y este depósito permaneció intacto hasta que pasaron de nuevo por Jericó durante el último viaje memorable a Jerusalén, poco antes del juicio y la muerte de Jesús.

\par
%\textsuperscript{(1648.4)}
\textsuperscript{147:2.2} El grupo tuvo un viaje tranquilo hasta Jerusalén, pero apenas se habían instalado en Betania cuando empezaron a congregarse, de cerca y de lejos, personas que buscaban la curación para su cuerpo, el consuelo para su mente confusa y la salvación para su alma; eran tan numerosas que Jesús tuvo poco tiempo para descansar. Por esta razón, montaron las tiendas en Getsemaní, y el Maestro iba y venía de Betania a Getsemaní para evitar la multitud que lo asediaba constantemente. El grupo apostólico pasó casi tres semanas en Jerusalén, pero Jesús les ordenó que no predicaran en público, que se limitaran a la enseñanza en privado y al trabajo personal.

\par
%\textsuperscript{(1648.5)}
\textsuperscript{147:2.3} Celebraron la Pascua tranquilamente en Betania. Era la primera vez que Jesús y la totalidad de los doce compartían la fiesta pascual sin derramamiento de sangre. Los apóstoles de Juan no comieron la Pascua con Jesús y sus apóstoles; celebraron la fiesta con Abner y muchos de los primeros creyentes en las predicaciones de Juan. Ésta era la segunda Pascua que Jesús celebraba con sus apóstoles en Jerusalén.

\par
%\textsuperscript{(1648.6)}
\textsuperscript{147:2.4} Cuando Jesús y los doce partieron para Cafarnaúm, los apóstoles de Juan no regresaron con ellos. Se quedaron en Jerusalén y sus alrededores bajo la dirección de Abner, trabajando discretamente por la expansión del reino, mientras que Jesús y los doce regresaban para efectuar su labor en Galilea. Los veinticuatro nunca más volvieron a estar todos juntos hasta poco antes de que los setenta evangelistas recibieran su misión y su orden de partir. Pero los dos grupos cooperaban entre sí y prevalecían los mejores sentimientos, a pesar de sus diferencias de opinión.

\section*{3. En el estanque de Betesda}
\par
%\textsuperscript{(1649.1)}
\textsuperscript{147:3.1} Durante la tarde del segundo sábado en Jerusalén, mientras el Maestro y los apóstoles estaban a punto de participar en los servicios del templo, Juan le dijo a Jesús: «Ven conmigo, quisiera mostrarte algo». Juan llevó a Jesús por una de las puertas de Jerusalén hasta un estanque de agua llamado Betesda\footnote{\textit{El estanque de Betesda}: Jn 5:1-4.}. Alrededor de este estanque había una estructura de cinco pórticos, bajo los cuales permanecía un gran número de enfermos en busca de curación. Se trataba de un manantial caliente cuyas aguas rojizas burbujeaban a intervalos irregulares a causa de las acumulaciones de gases en las cavernas rocosas que se encontraban debajo del estanque. Muchos creían que esta perturbación periódica de las aguas calientes se debía a influencias sobrenaturales, y era creencia popular de que la primera persona que entrara en el agua después de una de estas perturbaciones se curaría de cualquier enfermedad que tuviera.

\par
%\textsuperscript{(1649.2)}
\textsuperscript{147:3.2} Los apóstoles estaban un poco inquietos por las restricciones impuestas por Jesús, y Juan, el más joven de los doce, se sentía particularmente impaciente por esta prohibición. Había llevado a Jesús al estanque pensando que el espectáculo de los enfermos allí reunidos conmovería tanto la compasión del Maestro que lo incitaría a efectuar un milagro de curación, y así todo Jerusalén se quedaría asombrado y pronto se pondría a creer en el evangelio del reino. Juan le dijo a Jesús: «Maestro, mira toda esta gente que sufre; ¿no hay nada que podamos hacer por ellos?» Y Jesús replicó: «Juan, ¿por qué me tientas para que me desvíe del camino que he escogido? ¿Por qué continúas deseando sustituir la proclamación del evangelio de la verdad eterna por la realización de prodigios y la curación de los enfermos? Hijo mío, no me está permitido hacer lo que deseas, pero reúne a esos enfermos y afligidos para que pueda dirigirles unas palabras de aliento y de consuelo eterno».

\par
%\textsuperscript{(1649.3)}
\textsuperscript{147:3.3} Al dirigirse a los allí reunidos, Jesús les dijo: «Muchos de vosotros estáis aquí, enfermos y afligidos, porque habéis vivido muchos años en el camino equivocado. Algunos sufren por los accidentes del tiempo, otros a consecuencia de los errores de sus antepasados, mientras que algunos de vosotros lucháis contra los obstáculos de las condiciones imperfectas de vuestra existencia temporal. Pero mi Padre trabaja, y yo quisiera trabajar, para mejorar vuestra condición en la Tierra, y más especialmente para asegurar vuestro estado eterno. Ninguno de nosotros puede hacer gran cosa por cambiar las dificultades de la vida, a menos que descubramos que el Padre que está en los cielos así lo quiere. Después de todo, todos estamos obligados a hacer la voluntad del Eterno. Si todos os pudierais curar de vuestras aflicciones físicas, indudablemente os admiraríais, pero es aun más importante que seáis purificados de toda enfermedad espiritual y que os encontréis curados de todas las dolencias morales. Todos sois hijos de Dios; sois los hijos del Padre celestial. Las trabas del tiempo pueden parecer afligiros, pero el Dios de la eternidad os ama. Cuando llegue la hora del juicio, no temáis, pues todos encontraréis no solamente justicia, sino una abundante misericordia. En verdad, en verdad os lo digo: Aquel que escucha el evangelio del reino y cree en esta enseñanza de la filiación con Dios, posee la vida eterna; esos creyentes pasan ya del juicio y de la muerte a la luz y a la vida. Y se acerca la hora en que incluso aquellos que están en la tumba escucharán la voz de la resurrección»\footnote{\textit{Mi Padre trabaja y yo trabajo}: Jn 5:17. \textit{Quienes escuchan y creen tienen vida eterna}: Jn 5:24-28; 6:40.}.

\par
%\textsuperscript{(1649.4)}
\textsuperscript{147:3.4} Muchos de los que lo escucharon creyeron en el evangelio del reino. Algunos de los afligidos se sintieron tan inspirados y revivificados espiritualmente, que anduvieron proclamando de acá para allá que también habían sido curados de sus dolencias físicas.

\par
%\textsuperscript{(1649.5)}
\textsuperscript{147:3.5} Un hombre que había estado muchos años deprimido y gravemente afligido con las dolencias de su mente perturbada, se regocijó con las palabras de Jesús, recogió su lecho y salió hacia su casa, aunque era el día del sábado. Este hombre angustiado había esperado todos estos años que \textit{alguien} le ayudara; era tan víctima del sentimiento de su propia impotencia que ni una sola vez había concebido la idea de ayudarse a sí mismo, aunque ésta era la única cosa que tenía que hacer para recuperarse ---recoger su lecho y salir caminando\footnote{\textit{Recoger su lecho y caminar}: Jn 5:5-9.}.

\par
%\textsuperscript{(1650.1)}
\textsuperscript{147:3.6} Jesús le dijo entonces a Juan: «Vámonos de aquí antes de que los principales sacerdotes y los escribas se encuentren con nosotros y se ofendan porque hemos dirigido unas palabras de vida a estos afligidos»\footnote{\textit{Temores de los sacerdotes y escribas}: Jn 5:10-15.}. Volvieron al templo para reunirse con sus compañeros, y todos partieron enseguida para pasar la noche en Betania. Juan nunca contó a los otros apóstoles la visita que había hecho con Jesús, este sábado por la tarde, al estanque de Betesda.

\section*{4. La regla de vida}
\par
%\textsuperscript{(1650.2)}
\textsuperscript{147:4.1} Al anochecer de este mismo sábado, en Betania, mientras que Jesús, los doce y un grupo de creyentes estaban reunidos alrededor del fuego en el jardín de Lázaro, Natanael le hizo a Jesús la pregunta siguiente: «Maestro, aunque nos has enseñado la versión positiva de la antigua regla de vida, indicándonos que deberíamos hacer a los demás lo que deseamos que nos hagan a nosotros, no discierno plenamente cómo podremos obrar siempre de acuerdo con este mandato. Permíteme ilustrar mi opinión citando el ejemplo de un hombre lascivo que mira con inmoralidad a su futura compañera de pecado. ¿Cómo podemos enseñar que este hombre malintencionado debería hacer a los demás lo que quisiera que le hicieran a él?»\footnote{\textit{La regla de oro}: Mt 7:12; Lc 6:31. \textit{La regla de oro (negativa)}: Tb 4:15.}

\par
%\textsuperscript{(1650.3)}
\textsuperscript{147:4.2} Cuando Jesús escuchó la pregunta de Natanael, se puso inmediatamente de pie, señaló al apóstol con el dedo, y dijo: «¡Natanael, Natanael! ¿Qué tipo de pensamientos mantienes en tu corazón? ¿No recibes mis enseñanzas como alguien que ha nacido del espíritu? ¿No escucháis la verdad como hombres con sabiduría y comprensión espiritual? Cuando os recomendé que hicierais por los demás lo que quisierais que hicieran por vosotros, me dirigía a unos hombres con ideales elevados, y no a unos que sentirían la tentación de tergiversar mi enseñanza, convirtiéndola en una licencia para estimular las malas acciones».

\par
%\textsuperscript{(1650.4)}
\textsuperscript{147:4.3} Cuando el Maestro hubo hablado, Natanael se levantó y dijo: «Pero Maestro, no deberías pensar que apruebo semejante interpretación de tu enseñanza. He hecho esta pregunta porque he supuesto que muchos hombres de este tipo podrían juzgar mal tus recomendaciones, y esperaba que nos darías una enseñanza adicional sobre estas cuestiones». Una vez que Natanael se hubo sentado, Jesús continuó hablando: «Sé bien, Natanael, que tu mente no aprueba ninguna idea de maldad de este tipo, pero me desilusiona que todos vosotros olvidéis con tanta frecuencia darle una interpretación auténticamente espiritual a mis enseñanzas corrientes, a unas instrucciones que debo daros en lenguaje humano y a la manera en que hablan los hombres. Permitidme ahora que os enseñe sobre los diversos niveles de significado ligados a la interpretación de esta regla de vida, a esta recomendación de `hacer por los demás lo que deseáis que ellos hagan por vosotros':»

\par
%\textsuperscript{(1650.5)}
\textsuperscript{147:4.4} «1. \textit{El nivel de la carne}. Esta interpretación puramente egoísta y lasciva tendría un buen ejemplo en la hipótesis de tu pregunta».

\par
%\textsuperscript{(1650.6)}
\textsuperscript{147:4.5} «2. \textit{El nivel de los sentimientos}. Este plano se encuentra un nivel por encima del de la carne, e implica que la compasión y la piedad realzan nuestra interpretación de esta regla de vida».

\par
%\textsuperscript{(1650.7)}
\textsuperscript{147:4.6} «3. \textit{El nivel de la mente}. Ahora entran en acción la razón de la mente y la inteligencia de la experiencia. El buen juicio dicta que esta regla de vida debería ser interpretada en consonancia con el idealismo más elevado, incorporado en la nobleza de un profundo respeto de sí mismo».

\par
%\textsuperscript{(1651.1)}
\textsuperscript{147:4.7} «4. \textit{El nivel del amor fraternal}. Aún más arriba se descubre el nivel de la consagración desinteresada al bienestar de nuestros semejantes. En este plano más elevado del servicio social entusiasta, que nace de la conciencia de la paternidad de Dios y del reconocimiento consiguiente de la fraternidad de los hombres, se descubre una interpretación nueva y mucho más hermosa de esta regla de vida fundamental».

\par
%\textsuperscript{(1651.2)}
\textsuperscript{147:4.8} «5. \textit{El nivel moral}. Luego, cuando alcancéis unos verdaderos niveles filosóficos de interpretación, cuando tengáis una verdadera comprensión de la \textit{rectitud} y de la \textit{maldad} en los acontecimientos, cuando percibáis la idoneidad eterna de las relaciones humanas, empezaréis a considerar este problema de interpretación como imaginaríais que una tercera persona de pensamientos elevados, idealista, sabia e imparcial consideraría e interpretaría este mandato, pero aplicado a vuestros problemas personales de adaptación a los incidentes de vuestra vida».

\par
%\textsuperscript{(1651.3)}
\textsuperscript{147:4.9} «6. \textit{El nivel espiritual}. En último lugar alcanzamos el nivel de la perspicacia del espíritu y de la interpretación espiritual, el nivel más elevado de todos, que nos impulsa a reconocer en esta regla de vida el mandamiento divino de tratar a todos los hombres como concebimos que Dios los trataría. Éste es el ideal universal de las relaciones humanas, y ésta es vuestra actitud ante todos estos problemas cuando vuestro deseo supremo es hacer siempre la voluntad del Padre. Quisiera pues que hicierais por todos los hombres lo que sabéis que yo haría por ellos en circunstancias semejantes».

\par
%\textsuperscript{(1651.4)}
\textsuperscript{147:4.10} Nada de lo que Jesús había dicho a los apóstoles hasta ese momento les había impresionado tanto. Continuaron discutiendo las palabras del Maestro hasta mucho después de que éste se hubiera retirado. Aunque Natanael tardó en recobrarse de la hipótesis de que Jesús no había interpretado bien el espíritu de su pregunta, los demás estaban más que agradecidos a su colega filosófico por haber tenido el valor de hacer una pregunta que incitaba tanto a la reflexión.

\section*{5. La visita a Simón el fariseo}
\par
%\textsuperscript{(1651.5)}
\textsuperscript{147:5.1} Aunque Simón no era un miembro del sanedrín judío, era un fariseo influyente de Jerusalén\footnote{\textit{Simón el fariseo}: Lc 7:36a.}. Era un creyente poco entusiasta, y a pesar de que podría ser criticado severamente por ello, se atrevió a invitar a Jesús y a sus asociados personales Pedro, Santiago y Juan a un banquete en su casa. Simón había observado al Maestro desde hacía mucho tiempo; estaba muy impresionado por sus enseñanzas y aun más por su personalidad.

\par
%\textsuperscript{(1651.6)}
\textsuperscript{147:5.2} Los fariseos ricos eran adictos a dar limosnas, y no evitaban la publicidad relacionada con su filantropía. A veces incluso hubieran tocado las trompetas cuando se disponían a dispensar la caridad a un mendigo. Cuando estos fariseos ofrecían un banquete a unos invitados distinguidos, tenían la costumbre de dejar abiertas las puertas de la casa para que incluso los mendigos de la calle pudieran entrar; éstos permanecían de pie junto a las paredes de la sala, detrás de los lechos de los convidados, para estar en condiciones de recibir los trozos de comida que los comensales pudieran lanzarles.

\par
%\textsuperscript{(1651.7)}
\textsuperscript{147:5.3} En esta ocasión particular, en la casa de Simón, entre la gente que entraba de la calle había una mujer de mala reputación que recientemente se había vuelto creyente en la buena nueva del evangelio del reino. Esta mujer era bien conocida en todo Jerusalén como la antigua dueña de un burdel considerado de alta categoría, situado muy cerca del patio de los gentiles del templo. Al aceptar las enseñanzas de Jesús, había cerrado su abominable negocio, y había persuadido a la mayoría de las mujeres que estaban asociadas con ella a que aceptaran el evangelio y cambiaran su forma de vida. A pesar de esto, los fariseos continuaban despreciándola mucho y estaba obligada a llevar el cabello suelto ---el distintivo de la prostitución. Esta mujer anónima había traído consigo un gran frasco de loción perfumada para ungir; permanecía de pie detrás de Jesús, que estaba recostado para comer, y empezó a ungirle los pies, al mismo tiempo que se los mojaba con sus lágrimas de gratitud, secándoselos con sus cabellos\footnote{\textit{Jesús ungido por una antigua prostituta}: Lc 7:36b-38.}. Cuando hubo terminado la unción, continuó llorando y besándole los pies.

\par
%\textsuperscript{(1652.1)}
\textsuperscript{147:5.4} Cuando Simón vio todo esto, se dijo para sus adentros: «Si este hombre fuera un profeta, hubiera percibido quién lo está tocando así y de qué tipo de mujer se trata; de una pecadora de mala fama». Sabiendo lo que pasaba por la mente de Simón, Jesús tomó la palabra y dijo: «Simón, hay algo que me gustaría decirte». Simón respondió: «Maestro, dilo». Entonces Jesús dijo: «Un rico prestamista tenía dos deudores. Uno le debía quinientos denarios y el otro cincuenta. Entonces, como ninguno de ellos tenía con qué pagarle, les perdonó la deuda a los dos. Según tú, Simón, ¿cuál de ellos lo amará más?» Simón contestó: «Supongo que aquel a quien más le perdonó»\footnote{\textit{Parábola de los dos deudores}: Lc 7:39-48.}. Y Jesús le dijo: «Has juzgado bien», y señalando a la mujer, continuó: «Simón, mira bien a esta mujer. He entrado en tu casa como invitado, y sin embargo no me has dado agua para mis pies. Esta mujer agradecida me ha lavado los pies con sus lágrimas y los ha secado con sus cabellos. No me has dado un beso amistoso de bienvenida, pero esta mujer, desde que entró, no ha dejado de besarme los pies. Has olvidado ungirme la cabeza con aceite, pero ella ha ungido mis pies con lociones costosas. ¿Cuál es el significado de todo esto? Simplemente que sus numerosos pecados le han sido perdonados, y esto la ha llevado a amar mucho. Pero los que sólo han recibido un poco de perdón a veces sólo aman un poco». Jesús se volvió hacia la mujer, la cogió de la mano, la levantó y le dijo: «En verdad te has arrepentido de tus pecados, y están perdonados. No te desanimes por la actitud irreflexiva y severa de tus semejantes; continúa tu camino en la alegría y la libertad del reino de los cielos».

\par
%\textsuperscript{(1652.2)}
\textsuperscript{147:5.5} Cuando Simón y sus amigos que estaban sentados comiendo con él escucharon estas palabras, se quedaron más que sorprendidos y empezaron a cuchichear entre ellos: «¿Quién es este hombre que se atreve incluso a perdonar los pecados?» Cuando Jesús los escuchó murmurar así, se volvió para despedir a la mujer, diciendo: «Mujer, vete en paz; tu fe te ha salvado»\footnote{\textit{Fe y perdón de los pecados}: Lc 7:49-50.}.

\par
%\textsuperscript{(1652.3)}
\textsuperscript{147:5.6} Cuando Jesús se levantó con sus amigos para irse, se volvió hacia Simón y le dijo: «Conozco tu corazón, Simón. Sé cómo estás desgarrado entre la fe y la duda, cómo estás desconcertado por el miedo y confundido por el orgullo; pero ruego por ti, para que te abandones a la luz y puedas experimentar en tu situación en la vida esas poderosas transformaciones de mente y de espíritu comparables a los cambios enormes que el evangelio del reino ya ha producido en el corazón de tu visitante no invitada ni bienvenida. Os declaro a todos que el Padre ha abierto las puertas del reino celestial a todos los que tienen la fe necesaria para entrar, y ningún hombre o asociación de hombres podrán cerrar esas puertas ni siquiera al alma más humilde o al pecador supuestamente más flagrante de la Tierra, si sinceramente aspiran a entrar»\footnote{\textit{La puerta abierta para entrar por la fe}: Ap 3:8.}. Y Jesús, con Pedro, Santiago y Juan, se despidieron de su anfitrión y fueron a reunirse con el resto de los apóstoles en el campamento del jardín de Getsemaní.

\par
%\textsuperscript{(1653.1)}
\textsuperscript{147:5.7} Aquella misma noche, Jesús dio a los apóstoles el inolvidable discurso sobre el valor relativo del estatus ante Dios y del progreso en la ascensión eterna hacia el Paraíso. Jesús dijo: «Hijos míos, si existe una verdadera conexión viviente entre el hijo y el Padre, el hijo está seguro de progresar continuamente hacia los ideales del Padre. Es verdad que al principio el hijo puede progresar lentamente, pero su progreso no es por ello menos seguro. Lo importante no es la rapidez de vuestro progreso, sino su certidumbre. Vuestros logros actuales no son tan importantes como el hecho de que la \textit{dirección} de vuestro progreso es hacia Dios. Aquello en lo que os estáis convirtiendo, día tras día, tiene infinitamente más importancia que lo que sois hoy».

\par
%\textsuperscript{(1653.2)}
\textsuperscript{147:5.8} «Esta mujer transformada, que algunos de vosotros habéis visto hoy en la casa de Simón, vive en este momento en un nivel muy inferior al de Simón y sus asociados bien intencionados. Pero estos fariseos están ocupados en el falso progreso de la ilusión de atravesar los círculos engañosos de los servicios ceremoniales sin sentido, mientras que esta mujer ha empezado, con una seriedad total, la larga y extraordinaria búsqueda de Dios; y su camino hacia el cielo no está bloqueado por el orgullo espiritual ni por la satisfacción moral de sí misma. Humanamente hablando, esta mujer está mucho más lejos de Dios que Simón, pero su alma sigue un movimiento progresivo; esta mujer está en camino hacia una meta eterna. En esta mujer están presentes unas enormes posibilidades espirituales para el futuro. Algunos de vosotros pueden no encontrarse en unos niveles realmente elevados de alma y de espíritu, pero estáis efectuando progresos diarios hacia Dios en el camino viviente que vuestra fe ha abierto. En cada uno de vosotros existen unas enormes posibilidades para el futuro. Es mucho mejor tener una fe limitada, pero viva y creciente, que poseer un gran intelecto con sus depósitos muertos de sabiduría mundana y de incredulidad espiritual»\footnote{\textit{Jesús, el camino viviente}: Jn 14:6; Heb 10:20.}.

\par
%\textsuperscript{(1653.3)}
\textsuperscript{147:5.9} Jesús previno seriamente a sus apóstoles contra la necedad del hijo de Dios que abusa del amor del Padre. Declaró que el Padre celestial no es un padre descuidado, negligente o tontamente indulgente, que siempre está dispuesto a indultar el pecado y a perdonar la imprudencia. Advirtió a sus oyentes que no aplicaran erróneamente sus ejemplos del padre y el hijo de manera que pudiera parecer que Dios es como uno de esos padres demasiado indulgentes y nada sabios, que conspiran con la necedad de la Tierra para provocar la ruina moral de sus hijos irreflexivos, contribuyendo así de manera cierta y directa a la delincuencia y a la pronta corrupción de sus propios descendientes. Jesús dijo: «Mi Padre no aprueba con indulgencia los actos y las prácticas de sus hijos que conducen a la destrucción y a la ruina de todo crecimiento moral y de todo progreso espiritual. Esas prácticas pecaminosas son una abominación a los ojos de Dios».

\par
%\textsuperscript{(1653.4)}
\textsuperscript{147:5.10} Jesús asistió a otras muchas reuniones y banquetes semiprivados con los grandes y los humildes, los ricos y los pobres de Jerusalén, antes de partir finalmente con sus apóstoles hacia Cafarnaúm. Muchos, en verdad, se hicieron creyentes en el evangelio del reino y fueron bautizados posteriormente por Abner y sus asociados, que se quedaron atrás para fomentar los intereses del reino en Jerusalén y sus alrededores.

\section*{6. El regreso a Cafarnaúm}
\par
%\textsuperscript{(1653.5)}
\textsuperscript{147:6.1} La última semana de abril, Jesús y los doce salieron de su cuartel general de Betania cerca de Jerusalén, y emprendieron su viaje de regreso a Cafarnaúm por el camino de Jericó y el Jordán.

\par
%\textsuperscript{(1654.1)}
\textsuperscript{147:6.2} Los sacerdotes principales y los jefes religiosos de los judíos tuvieron muchas reuniones secretas con el fin de decidir qué iban a hacer con Jesús. Todos estaban de acuerdo en que había que hacer algo para poner fin a su enseñanza, pero no se ponían de acuerdo en el método a emplear. Habían tenido la esperanza de que las autoridades civiles dispondrían de él como Herodes había puesto fin a la carrera de Juan, pero descubrieron que Jesús llevaba su actividad de tal manera que los funcionarios romanos no estaban muy alarmados por sus predicaciones. En consecuencia, en una reunión celebrada el día antes de la partida de Jesús para Cafarnaúm, decidieron que tenía que ser capturado bajo la acusación de un delito religioso, y ser juzgado por el sanedrín. Por esta razón, nombraron una comisión de seis espías secretos para que siguieran a Jesús y observaran sus palabras y sus actos; cuando hubieran acumulado suficientes pruebas de infracciones a la ley y de blasfemias, tenían que regresar con su informe a Jerusalén. Estos seis judíos alcanzaron en Jericó al grupo apostólico, que constaba de unos treinta miembros, y con el pretexto de que deseaban convertirse en discípulos, se unieron a la familia de seguidores de Jesús, permaneciendo con el grupo hasta el comienzo de la segunda gira de predicación en Galilea. En ese momento, tres de ellos volvieron a Jerusalén para presentar su informe a los principales sacerdotes y al sanedrín.

\par
%\textsuperscript{(1654.2)}
\textsuperscript{147:6.3} Pedro predicó a la multitud reunida en el vado del Jordán, y a la mañana siguiente se dirigieron río arriba hacia Amatus. Querían continuar directamente hasta Cafarnaúm, pero se había congregado tanta gente que se quedaron tres días, predicando, enseñando y bautizando. No se marcharon para casa hasta el sábado por la mañana temprano, primer día de mayo. Los espías de Jerusalén estaban seguros de que ahora podrían obtener la primera acusación contra Jesús ---la de violar el sábado--- puesto que se había atrevido a emprender su viaje el día del sábado. Pero iban a sufrir una desilusión porque, justo antes de partir, Jesús llamó a Andrés y le dio instrucciones, delante de todos ellos, para que sólo avanzaran unos mil metros, la distancia legal que los judíos podían recorrer el día del sábado.

\par
%\textsuperscript{(1654.3)}
\textsuperscript{147:6.4} Pero los espías no tuvieron que esperar mucho para tener la oportunidad de acusar a Jesús y a sus compañeros de violar el sábado. Al pasar el grupo por un camino estrecho, a ambos lados y al alcance de la mano se encontraba el trigo ondulante, que en esa época estaba madurando; como algunos de los apóstoles tenían hambre, arrancaron el grano maduro y se lo comieron. Entre los viajeros existía la costumbre de servirse grano mientras pasaban por la carretera, y por esta razón no se atribuía ninguna idea de maldad a esta conducta. Pero los espías cogieron esto como pretexto para atacar a Jesús. Cuando vieron a Andrés restregando el grano en su mano, se acercaron y le dijeron: «¿No sabes que es ilegal arrancar y restregar el grano el día del sábado?»\footnote{\textit{Los apóstoles trituran grano en sábado}: Mt 12:1-2; Mc 2:23-24; Lc 6:1-2.} Andrés respondió: «Pero tenemos hambre y sólo restregamos la cantidad suficiente para nuestras necesidades; ¿desde cuándo es un pecado comer grano el día del sábado?» Pero los fariseos replicaron: «No haces mal en comerlo, pero violas la ley al arrancar y restregar el grano entre tus manos; tu Maestro seguramente no aprobaría esa conducta». Entonces, Andrés dijo: «Si no es malo comerse el grano, seguramente restregarlo entre nuestras manos no es mucho más trabajo que masticarlo, cosa que permitís; ¿por qué hacéis un problema por estas nimiedades?» Cuando Andrés insinuó que eran unos sofistas, se indignaron y se precipitaron hacia Jesús, que caminaba detrás charlando con Mateo, y protestaron diciendo: «Mira, Maestro, tus apóstoles hacen lo que es ilegal el día del sábado; arrancan, restriegan y se comen el grano. Estamos seguros de que les vas a ordenar que dejen de hacerlo». Jesús dijo entonces a los acusadores: «En verdad sois celosos de la ley, y hacéis bien en recordar el sábado para santificarlo. Pero ¿no habéis leído nunca en las Escrituras que un día que David tenía hambre entró en la casa de Dios con sus compañeros, y se comieron el pan de la proposición, que nadie estaba autorizado a comer excepto los sacerdotes? Y David también dio de este pan a los que estaban con él. ¿Y no habéis leído en nuestra ley que es legal hacer muchas cosas necesarias el sábado? ¿Y no voy a veros comer, antes de que termine el día, lo que habéis traído para vuestras necesidades de hoy? Amigos míos, hacéis bien en defender el sábado, pero haríais mejor en proteger la salud y el bienestar de vuestros semejantes. Afirmo que el sábado ha sido hecho para el hombre, y no el hombre para el sábado. Y si estáis aquí con nosotros para vigilar mis palabras, entonces proclamaré abiertamente que el Hijo del Hombre es dueño incluso del sábado»\footnote{\textit{David comió pan de la proposición}: 1 Sam 21:3-6. \textit{Jesús señor del sábado}: Mt 12:3-8; Mc 2:25-28; Lc 6:3-5.}.

\par
%\textsuperscript{(1655.1)}
\textsuperscript{147:6.5} Los fariseos se quedaron asombrados y confundidos ante sus palabras de discernimiento y de sabiduría. Durante el resto del día se mantuvieron apartados y no se atrevieron a hacer más preguntas.

\par
%\textsuperscript{(1655.2)}
\textsuperscript{147:6.6} El antagonismo de Jesús hacia las tradiciones judías y los ceremoniales serviles era siempre \textit{positivo}. Consistía en lo que él hacía y afirmaba. El Maestro pasaba poco tiempo haciendo denuncias negativas. Enseñaba que los que conocen a Dios pueden gozar de la libertad de vivir sin engañarse a sí mismos con los desenfrenos del pecado. Jesús dijo a sus apóstoles: «Amigos, si estáis iluminados por la verdad y si sabéis realmente lo que hacéis, sois bienaventurados; pero si no conocéis el camino divino, sois desgraciados y ya quebrantáis la ley».

\section*{7. De regreso en Cafarnaúm}
\par
%\textsuperscript{(1655.3)}
\textsuperscript{147:7.1} El lunes 3 de mayo, alrededor del mediodía, Jesús y los doce llegaron en barco a Betsaida, procedentes de Tariquea. Viajaron en barco para eludir a los que los acompañaban. Pero al día siguiente, todos ellos, incluyendo a los espías oficiales de Jerusalén, habían encontrado de nuevo a Jesús.

\par
%\textsuperscript{(1655.4)}
\textsuperscript{147:7.2} El martes por la tarde, Jesús estaba dirigiendo una de sus clases habituales de preguntas y respuestas, cuando el jefe de los seis espías le dijo: «Hoy estaba hablando con uno de los discípulos de Juan, que está aquí asistiendo a tu enseñanza, y no acertábamos a comprender por qué nunca ordenas a tus discípulos que ayunen y recen, como nosotros los fariseos ayunamos, y como Juan lo mandó a sus discípulos». Refiriéndose a una declaración de Juan, Jesús respondió a este interrogador: «¿Acaso ayunan los pajes de honor cuando el novio está con ellos? Mientras el novio permanece con ellos, difícilmente pueden ayunar. Pero se acerca la hora en que el novio será apartado de allí, y entonces los pajes de honor ayunarán y orarán indudablemente. La oración es algo natural para los hijos de la luz\footnote{\textit{Hijos de la luz}: Lc 16:8; Jn 12:36; Ef 5:8; 1 Ts 5:5.}, pero el ayuno no forma parte del evangelio del reino de los cielos. Recordad que un sastre sabio no cose un trozo de tela nueva y sin encoger en un vestido viejo, por temor a que cuando se moje, encoja, y produzca un desgarrón aún mayor. Los hombres tampoco ponen el vino nuevo en odres viejos, para que el vino nuevo no reviente los odres y se pierdan tanto el vino como los odres. El hombre sabio pone el vino nuevo en odres nuevos. Por eso mis discípulos muestran sabiduría al no incorporar demasiadas cosas del viejo orden en la nueva enseñanza del evangelio del reino. Vosotros, que habéis perdido a vuestro instructor, podéis estar justificados si ayunáis durante un tiempo. El ayuno\footnote{\textit{Discurso sobre el ayuno}: Mt 9:14-17; Mc 2:18-22; Lc 5:33-38.} puede ser una parte apropiada de la ley de Moisés, pero en el reino venidero, los hijos de Dios estarán liberados del miedo y experimentarán la alegría en el espíritu divino». Cuando escucharon estas palabras, los discípulos de Juan se sintieron confortados mientras que los fariseos, por su parte, se quedaron aún más confundidos.

\par
%\textsuperscript{(1656.1)}
\textsuperscript{147:7.3} El Maestro procedió entonces a prevenir a sus oyentes contra el mantenimiento de la idea de que todas las antiguas enseñanzas tenían que ser totalmente reemplazadas por las nuevas doctrinas. Jesús dijo: «Lo que es antiguo, pero también \textit{verdadero}, debe permanecer. De la misma manera, lo que es nuevo, pero falso, debe ser rechazado. Tened la fe y el valor de aceptar lo que es nuevo y también verdadero. Recordad que está escrito: `No abandonéis a un viejo amigo, porque el nuevo no es comparable con él. Un amigo nuevo es como el vino nuevo; si se vuelve viejo, lo beberéis con alegría'.»\footnote{\textit{El viejo y el nuevo amigo}: Pr 27:10. \textit{El vino viejo y el nuevo}: Lc 5:39.}

\section*{8. La fiesta de la bondad espiritual}
\par
%\textsuperscript{(1656.2)}
\textsuperscript{147:8.1} Aquella noche, mucho después de que los oyentes habituales se hubieran retirado, Jesús continuó enseñando a sus apóstoles. Empezó esta lección especial citando al profeta Isaías:

\par
%\textsuperscript{(1656.3)}
\textsuperscript{147:8.2} «`¿Por qué habéis ayunado? ¿Por qué razón afligís vuestras almas mientras que continuáis encontrando placer en la opresión y deleitándoos con la injusticia? He aquí que ayunáis por amor a la contienda y a la disputa, y para golpear con el puño de la maldad. Pero ayunando de esta manera no haréis oír vuestras voces en el cielo»\footnote{\textit{¿Por qué habéis ayunado?}: Is 58:3-4.}.

\par
%\textsuperscript{(1656.4)}
\textsuperscript{147:8.3} «`¿Es éste el ayuno que he elegido ---un día para que el hombre aflija su alma? ¿Es para que incline la cabeza como un junco, para que se arrastre vestido de penitente? ¿Os atreveréis a decir que esto es un ayuno y un día aceptable a los ojos del Señor? ¿No es éste el ayuno que yo escogería: desatar las cadenas de la maldad, deshacer los nudos de las cargas pesadas, dejar libres a los oprimidos y romper todos los yugos? ¿No es compartir mi pan con el hambriento y traer a mi casa a los pobres sin hogar? Y cuando vea a los que están desnudos, los vestiré»\footnote{\textit{¿Por qué no hacer el bien?}: Is 58:5-7.}.

\par
%\textsuperscript{(1656.5)}
\textsuperscript{147:8.4} «`Entonces vuestra luz brotará como la mañana y vuestra salud crecerá con rapidez. Vuestra rectitud os precederá, mientras que la gloria del Señor será vuestra retaguardia. Entonces invocaréis al Señor y él os responderá; gritaréis con fuerza y él dirá: Aquí estoy. Hará todo esto si dejáis de oprimir, de condenar y de mostrar vanidad. El Padre desea más bien que extendáis vuestro corazón a los hambrientos y que ayudéis a las almas afligidas; entonces vuestra luz brillará en las tinieblas, e incluso vuestra obscuridad será como el mediodía. Entonces el Señor os guiará contínuamente, satisfaciendo vuestra alma y renovando vuestra fortaleza. Os volveréis como un jardín regado, como un manantial cuyas aguas no se agotan. Los que hacen estas cosas restablecerán las glorias perdidas; levantarán los cimientos de muchas generaciones; serán llamados los reconstructores de los muros rotos, los restauradores de los caminos seguros por los que se puede transitar'»\footnote{\textit{Vuestra luz brotará}: Is 58:8-12.}.

\par
%\textsuperscript{(1656.6)}
\textsuperscript{147:8.5} Luego, hasta muy entrada la noche, Jesús expuso a sus apóstoles la verdad de que era su fe la que les daba seguridad en el reino del presente y del futuro, y no la aflicción de su alma ni el ayuno del cuerpo. Exhortó a los apóstoles a que vivieran al menos a la altura de las ideas del profeta de antaño, y expresó la esperanza de que progresarían mucho, incluso más allá de los ideales de Isaías y de los antiguos profetas. Las últimas palabras que pronunció aquella noche fueron: «Creced en la gracia por medio de esa fe viviente que capta el hecho de que sois hijos de Dios, y al mismo tiempo reconoce a cada hombre como un hermano»\footnote{\textit{Creced en la gracia}: 2 P 3:18.}.

\par
%\textsuperscript{(1656.7)}
\textsuperscript{147:8.6} Eran más de las dos de la madrugada cuando Jesús dejó de hablar, y cada cual se retiró a descansar.