\begin{document}

\title{James}

Jas 1:1  神と主イエス・キリストとの僕ヤコブから、離散している十二部族の人々へ、あいさつをおくる。
Jas 1:2  わたしの兄弟たちよ。あなたがたが、いろいろな試錬に会った場合、それをむしろ非常に喜ばしいことと思いなさい。
Jas 1:3  あなたがたの知っているとおり、信仰がためされることによって、忍耐が生み出されるからである。
Jas 1:4  だから、なんら欠点のない、完全な、でき上がった人となるように、その忍耐力を十分に働かせるがよい。
Jas 1:5  あなたがたのうち、知恵に不足している者があれば、その人は、とがめもせずに惜しみなくすべての人に与える神に、願い求めるがよい。そうすれば、与えられるであろう。
Jas 1:6  ただ、疑わないで、信仰をもって願い求めなさい。疑う人は、風の吹くままに揺れ動く海の波に似ている。
Jas 1:7  そういう人は、主から何かをいただけるもののように思うべきではない。
Jas 1:8  そんな人間は、二心の者であって、そのすべての行動に安定がない。
Jas 1:9  低い身分の兄弟は、自分が高くされたことを喜びなさい。
Jas 1:10  また、富んでいる者は、自分が低くされたことを喜ぶがよい。富んでいる者は、草花のように過ぎ去るからである。
Jas 1:11  たとえば、太陽が上って熱風をおくると、草を枯らす。そしてその花は落ち、その美しい姿は消えうせてしまう。それと同じように、富んでいる者も、その一生の旅なかばで没落するであろう。
Jas 1:12  試錬を耐え忍ぶ人は、さいわいである。それを忍びとおしたなら、神を愛する者たちに約束されたいのちの冠を受けるであろう。
Jas 1:13  だれでも誘惑に会う場合、「この誘惑は、神からきたものだ」と言ってはならない。神は悪の誘惑に陥るようなかたではなく、また自ら進んで人を誘惑することもなさらない。
Jas 1:14  人が誘惑に陥るのは、それぞれ、欲に引かれ、さそわれるからである。
Jas 1:15  欲がはらんで罪を生み、罪が熟して死を生み出す。
Jas 1:16  愛する兄弟たちよ。思い違いをしてはいけない。
Jas 1:17  あらゆる良い贈り物、あらゆる完全な賜物は、上から、光の父から下って来る。父には、変化とか回転の影とかいうものはない。
Jas 1:18  父は、わたしたちを、いわば被造物の初穂とするために、真理の言葉によって御旨のままに、生み出して下さったのである。
Jas 1:19  愛する兄弟たちよ。このことを知っておきなさい。人はすべて、聞くに早く、語るにおそく、怒るにおそくあるべきである。
Jas 1:20  人の怒りは、神の義を全うするものではないからである。
Jas 1:21  だから、すべての汚れや、はなはだしい悪を捨て去って、心に植えつけられている御言を、すなおに受け入れなさい。御言には、あなたがたのたましいを救う力がある。
Jas 1:22  そして、御言を行う人になりなさい。おのれを欺いて、ただ聞くだけの者となってはいけない。
Jas 1:23  おおよそ御言を聞くだけで行わない人は、ちょうど、自分の生れつきの顔を鏡に映して見る人のようである。
Jas 1:24  彼は自分を映して見てそこから立ち去ると、そのとたんに、自分の姿がどんなであったかを忘れてしまう。
Jas 1:25  これに反して、完全な自由の律法を一心に見つめてたゆまない人は、聞いて忘れてしまう人ではなくて、実際に行う人である。こういう人は、その行いによって祝福される。
Jas 1:26  もし人が信心深い者だと自任しながら、舌を制することをせず、自分の心を欺いているならば、その人の信心はむなしいものである。
Jas 1:27  父なる神のみまえに清く汚れのない信心とは、困っている孤児や、やもめを見舞い、自らは世の汚れに染まずに、身を清く保つことにほかならない。
Jas 2:1  わたしの兄弟たちよ。わたしたちの栄光の主イエス・キリストへの信仰を守るのに、分け隔てをしてはならない。
Jas 2:2  たとえば、あなたがたの会堂に、金の指輪をはめ、りっぱな着物を着た人がはいって来ると同時に、みすぼらしい着物を着た貧しい人がはいってきたとする。
Jas 2:3  その際、りっぱな着物を着た人に対しては、うやうやしく「どうぞ、こちらの良い席にお掛け下さい」と言い、貧しい人には、「あなたは、そこに立っていなさい。それとも、わたしの足もとにすわっているがよい」と言ったとしたら、
Jas 2:4  あなたがたは、自分たちの間で差別立てをし、よからぬ考えで人をさばく者になったわけではないか。
Jas 2:5  愛する兄弟たちよ。よく聞きなさい。神は、この世の貧しい人たちを選んで信仰に富ませ、神を愛する者たちに約束された御国の相続者とされたではないか。
Jas 2:6  しかるに、あなたがたは貧しい人をはずかしめたのである。あなたがたをしいたげ、裁判所に引きずり込むのは、富んでいる者たちではないか。
Jas 2:7  あなたがたに対して唱えられた尊い御名を汚すのは、実に彼らではないか。
Jas 2:8  しかし、もしあなたがたが、「自分を愛するように、あなたの隣り人を愛せよ」という聖書の言葉に従って、このきわめて尊い律法を守るならば、それは良いことである。
Jas 2:9  しかし、もし分け隔てをするならば、あなたがたは罪を犯すことになり、律法によって違反者として宣告される。
Jas 2:10  なぜなら、律法をことごとく守ったとしても、その一つの点にでも落ち度があれば、全体を犯したことになるからである。
Jas 2:11  たとえば、「姦淫するな」と言われたかたは、また「殺すな」とも仰せになった。そこで、たとい姦淫はしなくても、人殺しをすれば、律法の違反者になったことになる。
Jas 2:12  だから、自由の律法によってさばかるべき者らしく語り、かつ行いなさい。
Jas 2:13  あわれみを行わなかった者に対しては、仮借のないさばきが下される。あわれみは、さばきにうち勝つ。
Jas 2:14  わたしの兄弟たちよ。ある人が自分には信仰があると称していても、もし行いがなかったら、なんの役に立つか。その信仰は彼を救うことができるか。
Jas 2:15  ある兄弟または姉妹が裸でいて、その日の食物にもこと欠いている場合、
Jas 2:16  あなたがたのうち、だれかが、「安らかに行きなさい。暖まって、食べ飽きなさい」と言うだけで、そのからだに必要なものを何ひとつ与えなかったとしたら、なんの役に立つか。
Jas 2:17  信仰も、それと同様に、行いを伴わなければ、それだけでは死んだものである。
Jas 2:18  しかし、「ある人には信仰があり、またほかの人には行いがある」と言う者があろう。それなら、行いのないあなたの信仰なるものを見せてほしい。そうしたら、わたしの行いによって信仰を見せてあげよう。
Jas 2:19  あなたは、神はただひとりであると信じているのか。それは結構である。悪霊どもでさえ、信じておののいている。
Jas 2:20  ああ、愚かな人よ。行いを伴わない信仰のむなしいことを知りたいのか。
Jas 2:21  わたしたちの父祖アブラハムは、その子イサクを祭壇にささげた時、行いによって義とされたのではなかったか。
Jas 2:22  あなたが知っているとおり、彼においては、信仰が行いと共に働き、その行いによって信仰が全うされ、
Jas 2:23  こうして、「アブラハムは神を信じた。それによって、彼は義と認められた」という聖書の言葉が成就し、そして、彼は「神の友」と唱えられたのである。
Jas 2:24  これでわかるように、人が義とされるのは、行いによるのであって、信仰だけによるのではない。
Jas 2:25  同じように、かの遊女ラハブでさえも、使者たちをもてなし、彼らを別な道から送り出した時、行いによって義とされたではないか。
Jas 2:26  霊魂のないからだが死んだものであると同様に、行いのない信仰も死んだものなのである。
Jas 3:1  わたしの兄弟たちよ。あなたがたのうち多くの者は、教師にならないがよい。わたしたち教師が、他の人たちよりも、もっときびしいさばきを受けることが、よくわかっているからである。
Jas 3:2  わたしたちは皆、多くのあやまちを犯すものである。もし、言葉の上であやまちのない人があれば、そういう人は、全身をも制御することのできる完全な人である。
Jas 3:3  馬を御するために、その口にくつわをはめるなら、その全身を引きまわすことができる。
Jas 3:4  また船を見るがよい。船体が非常に大きく、また激しい風に吹きまくられても、ごく小さなかじ一つで、操縦者の思いのままに運転される。
Jas 3:5  それと同じく、舌は小さな器官ではあるが、よく大言壮語する。見よ、ごく小さな火でも、非常に大きな森を燃やすではないか。
Jas 3:6  舌は火である。不義の世界である。舌は、わたしたちの器官の一つとしてそなえられたものであるが、全身を汚し、生存の車輪を燃やし、自らは地獄の火で焼かれる。
Jas 3:7  あらゆる種類の獣、鳥、這うもの、海の生物は、すべて人類に制せられるし、また制せられてきた。
Jas 3:8  ところが、舌を制しうる人は、ひとりもいない。それは、制しにくい悪であって、死の毒に満ちている。
Jas 3:9  わたしたちは、この舌で父なる主をさんびし、また、その同じ舌で、神にかたどって造られた人間をのろっている。
Jas 3:10  同じ口から、さんびとのろいとが出て来る。わたしの兄弟たちよ。このような事は、あるべきでない。
Jas 3:11  泉が、甘い水と苦い水とを、同じ穴からふき出すことがあろうか。
Jas 3:12  わたしの兄弟たちよ。いちじくの木がオリブの実を結び、ぶどうの木がいちじくの実を結ぶことができようか。塩水も、甘い水を出すことはできない。
Jas 3:13  あなたがたのうちで、知恵があり物わかりのよい人は、だれであるか。その人は、知恵にかなう柔和な行いをしていることを、よい生活によって示すがよい。
Jas 3:14  しかし、もしあなたがたの心の中に、苦々しいねたみや党派心をいだいているのなら、誇り高ぶってはならない。また、真理にそむいて偽ってはならない。
Jas 3:15  そのような知恵は、上から下ってきたものではなくて、地につくもの、肉に属するもの、悪魔的なものである。
Jas 3:16  ねたみと党派心とのあるところには、混乱とあらゆる忌むべき行為とがある。
Jas 3:17  しかし上からの知恵は、第一に清く、次に平和、寛容、温順であり、あわれみと良い実とに満ち、かたより見ず、偽りがない。
Jas 3:18  義の実は、平和を造り出す人たちによって、平和のうちにまかれるものである。
Jas 4:1  あなたがたの中の戦いや争いは、いったい、どこから起るのか。それはほかではない。あなたがたの肢体の中で相戦う欲情からではないか。
Jas 4:2  あなたがたは、むさぼるが得られない。そこで人殺しをする。熱望するが手に入れることができない。そこで争い戦う。あなたがたは、求めないから得られないのだ。
Jas 4:3  求めても与えられないのは、快楽のために使おうとして、悪い求め方をするからだ。
Jas 4:4  不貞のやからよ。世を友とするのは、神への敵対であることを、知らないか。おおよそ世の友となろうと思う者は、自らを神の敵とするのである。
Jas 4:5  それとも、「神は、わたしたちの内に住まわせた霊を、ねたむほどに愛しておられる」と聖書に書いてあるのは、むなしい言葉だと思うのか。
Jas 4:6  しかし神は、いや増しに恵みを賜う。であるから、「神は高ぶる者をしりぞけ、へりくだる者に恵みを賜う」とある。
Jas 4:7  そういうわけだから、神に従いなさい。そして、悪魔に立ちむかいなさい。そうすれば、彼はあなたがたから逃げ去るであろう。
Jas 4:8  神に近づきなさい。そうすれば、神はあなたがたに近づいて下さるであろう。罪人どもよ、手をきよめよ。二心の者どもよ、心を清くせよ。
Jas 4:9  苦しめ、悲しめ、泣け。あなたがたの笑いを悲しみに、喜びを憂いに変えよ。
Jas 4:10  主のみまえにへりくだれ。そうすれば、主は、あなたがたを高くして下さるであろう。
Jas 4:11  兄弟たちよ。互に悪口を言い合ってはならない。兄弟の悪口を言ったり、自分の兄弟をさばいたりする者は、律法をそしり、律法をさばくやからである。もしあなたが律法をさばくなら、律法の実行者ではなくて、その審判者なのである。
Jas 4:12  しかし、立法者であり審判者であるかたは、ただひとりであって、救うことも滅ぼすこともできるのである。しかるに、隣り人をさばくあなたは、いったい、何者であるか。
Jas 4:13  よく聞きなさい。「きょうか、あす、これこれの町へ行き、そこに一か年滞在し、商売をして一もうけしよう」と言う者たちよ。
Jas 4:14  あなたがたは、あすのこともわからぬ身なのだ。あなたがたのいのちは、どんなものであるか。あなたがたは、しばしの間あらわれて、たちまち消え行く霧にすぎない。
Jas 4:15  むしろ、あなたがたは「主のみこころであれば、わたしは生きながらえもし、あの事この事もしよう」と言うべきである。
Jas 4:16  ところが、あなたがたは誇り高ぶっている。このような高慢は、すべて悪である。
Jas 4:17  人が、なすべき善を知りながら行わなければ、それは彼にとって罪である。
Jas 5:1  富んでいる人たちよ。よく聞きなさい。あなたがたは、自分の身に降りかかろうとしているわざわいを思って、泣き叫ぶがよい。
Jas 5:2  あなたがたの富は朽ち果て、着物はむしばまれ、
Jas 5:3  金銀はさびている。そして、そのさびの毒は、あなたがたの罪を責め、あなたがたの肉を火のように食いつくすであろう。あなたがたは、終りの時にいるのに、なお宝をたくわえている。
Jas 5:4  見よ、あなたがたが労働者たちに畑の刈入れをさせながら、支払わずにいる賃銀が、叫んでいる。そして、刈入れをした人たちの叫び声が、すでに万軍の主の耳に達している。
Jas 5:5  あなたがたは、地上でおごり暮し、快楽にふけり、「ほふらるる日」のために、おのが心を肥やしている。
Jas 5:6  そして、義人を罪に定め、これを殺した。しかも彼は、あなたがたに抵抗しない。
Jas 5:7  だから、兄弟たちよ。主の来臨の時まで耐え忍びなさい。見よ、農夫は、地の尊い実りを、前の雨と後の雨とがあるまで、耐え忍んで待っている。
Jas 5:8  あなたがたも、主の来臨が近づいているから、耐え忍びなさい。心を強くしていなさい。
Jas 5:9  兄弟たちよ。互に不平を言い合ってはならない。さばきを受けるかも知れないから。見よ、さばき主が、すでに戸口に立っておられる。
Jas 5:10  兄弟たちよ。苦しみを耐え忍ぶことについては、主の御名によって語った預言者たちを模範にするがよい。
Jas 5:11  忍び抜いた人たちはさいわいであると、わたしたちは思う。あなたがたは、ヨブの忍耐のことを聞いている。また、主が彼になさったことの結末を見て、主がいかに慈愛とあわれみとに富んだかたであるかが、わかるはずである。
Jas 5:12  さて、わたしの兄弟たちよ。何はともあれ、誓いをしてはならない。天をさしても、地をさしても、あるいは、そのほかのどんな誓いによっても、いっさい誓ってはならない。むしろ、「しかり」を「しかり」とし、「否」を「否」としなさい。そうしないと、あなたがたは、さばきを受けることになる。
Jas 5:13  あなたがたの中に、苦しんでいる者があるか。その人は、祈るがよい。喜んでいる者があるか。その人は、さんびするがよい。
Jas 5:14  あなたがたの中に、病んでいる者があるか。その人は、教会の長老たちを招き、主の御名によって、オリブ油を注いで祈ってもらうがよい。
Jas 5:15  信仰による祈は、病んでいる人を救い、そして、主はその人を立ちあがらせて下さる。かつ、その人が罪を犯していたなら、それもゆるされる。
Jas 5:16  だから、互に罪を告白し合い、また、いやされるようにお互のために祈りなさい。義人の祈は、大いに力があり、効果のあるものである。
Jas 5:17  エリヤは、わたしたちと同じ人間であったが、雨が降らないようにと祈をささげたところ、三年六か月のあいだ、地上に雨が降らなかった。
Jas 5:18  それから、ふたたび祈ったところ、天は雨を降らせ、地はその実をみのらせた。
Jas 5:19  わたしの兄弟たちよ。あなたがたのうち、真理の道から踏み迷う者があり、だれかが彼を引きもどすなら、
Jas 5:20  かように罪人を迷いの道から引きもどす人は、そのたましいを死から救い出し、かつ、多くの罪をおおうものであることを、知るべきである。


\end{document}