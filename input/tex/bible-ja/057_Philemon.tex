\begin{document}

\title{フィレモンへの手紙}


\chapter{1}

\par 1 キリスト・イエスの囚人パウロと兄弟テモテから、わたしたちの愛する同労者ピレモン、
\par 2 姉妹アピヤ、わたしたちの戦友アルキポ、ならびに、あなたの家にある教会へ。
\par 3 わたしたちの父なる神と主イエス・キリストから、恵みと平安とが、あなたがたにあるように。
\par 4 わたしは、祈の時にあなたをおぼえて、いつもわたしの神に感謝している。
\par 5 それは、主イエスに対し、また、すべての聖徒に対するあなたの愛と信仰とについて、聞いているからである。
\par 6 どうか、あなたの信仰の交わりが強められて、わたしたちの間でキリストのためになされているすべての良いことが、知られて来るようになってほしい。
\par 7 兄弟よ。わたしは、あなたの愛によって多くの喜びと慰めとを与えられた。聖徒たちの心が、あなたによって力づけられたからである。
\par 8 こういうわけで、わたしは、キリストにあってあなたのなすべき事を、きわめて率直に指示してもよいと思うが、
\par 9 むしろ、愛のゆえにお願いする。すでに老年になり、今またキリスト・イエスの囚人となっているこのパウロが、
\par 10 捕われの身で産んだわたしの子供オネシモについて、あなたにお願いする。
\par 11 彼は以前は、あなたにとって無益な者であったが、今は、あなたにも、わたしにも、有益な者になった。
\par 12 彼をあなたのもとに送りかえす。彼はわたしの心である。
\par 13 わたしは彼を身近に引きとめておいて、わたしが福音のために捕われている間、あなたに代って仕えてもらいたかったのである。
\par 14 しかし、わたしは、あなたの承諾なしには何もしたくない。あなたが強制されて良い行いをするのではなく、自発的にすることを願っている。
\par 15 彼がしばらくの間あなたから離れていたのは、あなたが彼をいつまでも留めておくためであったかも知れない。
\par 16 しかも、もはや奴隷としてではなく、奴隷以上のもの、愛する兄弟としてである。とりわけ、わたしにとってそうであるが、ましてあなたにとっては、肉においても、主にあっても、それ以上であろう。
\par 17 そこで、もしわたしをあなたの信仰の友と思ってくれるなら、わたし同様に彼を受けいれてほしい。
\par 18 もし、彼があなたに何か不都合なことをしたか、あるいは、何か負債があれば、それをわたしの借りにしておいてほしい。
\par 19 このパウロが手ずからしるす、わたしがそれを返済する。この際、あなたが、あなた自身をわたしに負うていることについては、何も言うまい。
\par 20 兄弟よ。わたしはあなたから、主にあって何か益を得たいものである。わたしの心を、主にあって力づけてもらいたい。
\par 21 わたしはあなたの従順を堅く信じて、この手紙を書く。あなたは、確かにわたしが言う以上のことをしてくれるだろう。
\par 22 ついでにお願いするが、わたしのために宿を用意しておいてほしい。あなたがたの祈によって、あなたがたの所に行かせてもらえるように望んでいるのだから。
\par 23 キリスト・イエスにあって、わたしと共に捕われの身になっているエパフラスから、あなたによろしく。
\par 24 わたしの同労者たち、マルコ、アリスタルコ、デマス、ルカからも、よろしく。
\par 25 主イエス・キリストの恵みが、あなたがたの霊と共にあるように。


\end{document}