\begin{document}

\title{ヨハネの手紙二}


\chapter{1}

\par 1 長老のわたしから、真実に愛している選ばれた婦人とその子たちへ。あなたがたを愛しているのは、わたしだけではなく、真理を知っている者はみなそうである。
\par 2 それは、わたしたちのうちにあり、また永遠に共にあるべき真理によるのである。
\par 3 父なる神および父の御子イエス・キリストから、恵みとあわれみと平安とが、真理と愛のうちにあって、わたしたちと共にあるように。
\par 4 あなたの子供たちのうちで、わたしたちが父から受けた戒めどおりに、真理のうちを歩いている者があるのを見て、わたしは非常に喜んでいる。
\par 5 婦人よ。ここにお願いしたいことがある。それは、新しい戒めを書くわけではなく、初めから持っていた戒めなのであるが、わたしたちは、みんな互に愛し合おうではないか。
\par 6 父の戒めどおりに歩くことが、すなわち、愛であり、あなたがたが初めから聞いてきたとおりに愛のうちを歩くことが、すなわち、戒めなのである。
\par 7 なぜなら、イエス・キリストが肉体をとってこられたことを告白しないで人を惑わす者が、多く世にはいってきたからである。そういう者は、惑わす者であり、反キリストである。
\par 8 よく注意して、わたしたちの働いて得た成果を失うことがなく、豊かな報いを受けられるようにしなさい。
\par 9 すべてキリストの教をとおり過ごして、それにとどまらない者は、神を持っていないのである。その教にとどまっている者は、父を持ち、また御子をも持つ。
\par 10 この教を持たずにあなたがたのところに来る者があれば、その人を家に入れることも、あいさつすることもしてはいけない。
\par 11 そのような人にあいさつする者は、その悪い行いにあずかることになるからである。
\par 12 あなたがたに書きおくることはたくさんあるが、紙と墨とで書くことはすまい。むしろ、あなたがたのところに行き、直接はなし合って、共に喜びに満ちあふれたいものである。
\par 13 選ばれたあなたの姉妹の子供たちが、あなたによろしく。


\end{document}