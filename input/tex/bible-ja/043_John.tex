\begin{document}

\title{ヨハネによる福音書}


\chapter{1}

\par 1 初めに言があった。言は神と共にあった。言は神であった。
\par 2 この言は初めに神と共にあった。
\par 3 すべてのものは、これによってできた。できたもののうち、一つとしてこれによらないものはなかった。
\par 4 この言に命があった。そしてこの命は人の光であった。
\par 5 光はやみの中に輝いている。そして、やみはこれに勝たなかった。
\par 6 ここにひとりの人があって、神からつかわされていた。その名をヨハネと言った。
\par 7 この人はあかしのためにきた。光についてあかしをし、彼によってすべての人が信じるためである。
\par 8 彼は光ではなく、ただ、光についてあかしをするためにきたのである。
\par 9 すべての人を照すまことの光があって、世にきた。
\par 10 彼は世にいた。そして、世は彼によってできたのであるが、世は彼を知らずにいた。
\par 11 彼は自分のところにきたのに、自分の民は彼を受けいれなかった。
\par 12 しかし、彼を受けいれた者、すなわち、その名を信じた人々には、彼は神の子となる力を与えたのである。
\par 13 それらの人は、血すじによらず、肉の欲によらず、また、人の欲にもよらず、ただ神によって生れたのである。
\par 14 そして言は肉体となり、わたしたちのうちに宿った。わたしたちはその栄光を見た。それは父のひとり子としての栄光であって、めぐみとまこととに満ちていた。
\par 15 ヨハネは彼についてあかしをし、叫んで言った、「『わたしのあとに来るかたは、わたしよりもすぐれたかたである。わたしよりも先におられたからである』とわたしが言ったのは、この人のことである」。
\par 16 わたしたちすべての者は、その満ち満ちているものの中から受けて、めぐみにめぐみを加えられた。
\par 17 律法はモーセをとおして与えられ、めぐみとまこととは、イエス・キリストをとおしてきたのである。
\par 18 神を見た者はまだひとりもいない。ただ父のふところにいるひとり子なる神だけが、神をあらわしたのである。
\par 19 さて、ユダヤ人たちが、エルサレムから祭司たちやレビ人たちをヨハネのもとにつかわして、「あなたはどなたですか」と問わせたが、その時ヨハネが立てたあかしは、こうであった。
\par 20 すなわち、彼は告白して否まず、「わたしはキリストではない」と告白した。
\par 21 そこで、彼らは問うた、「それでは、どなたなのですか、あなたはエリヤですか」。彼は「いや、そうではない」と言った。「では、あの預言者ですか」。彼は「いいえ」と答えた。
\par 22 そこで、彼らは言った、「あなたはどなたですか。わたしたちをつかわした人々に、答を持って行けるようにしていただきたい。あなた自身をだれだと考えるのですか」。
\par 23 彼は言った、「わたしは、預言者イザヤが言ったように、『主の道をまっすぐにせよと荒野で呼ばわる者の声』である」。
\par 24 つかわされた人たちは、パリサイ人であった。
\par 25 彼らはヨハネに問うて言った、「では、あなたがキリストでもエリヤでもまたあの預言者でもないのなら、なぜバプテスマを授けるのですか」。
\par 26 ヨハネは彼らに答えて言った、「わたしは水でバプテスマを授けるが、あなたがたの知らないかたが、あなたがたの中に立っておられる。
\par 27 それがわたしのあとにあとにおいでになる方であって、わたしはその人のくつのひもを解く値うちもない」。
\par 28 これらのことは、ヨハネがバプテスマを授けていたヨルダンの向こうのベタニヤであったのである。
\par 29 その翌日、ヨハネはイエスが自分の方にこられるのを見て言った、「見よ、世の罪を取り除く神の小羊。
\par 30 『わたしのあとに来るかたは、わたしよりもすぐれたかたである。わたしよりも先におられたからである』とわたしが言ったのは、この人のことである。
\par 31 わたしはこのかたを知らなかった。しかし、このかたがイスラエルに現れてくださるそのことのために、わたしはきて、水でバプテスマを授けているのである」。
\par 32 ヨハネはまたあかしをして言った、「わたしは、御霊がはとのように天から下って、彼の上にとどまるのを見た。
\par 33 わたしはこの人を知らなかった。しかし、水でバプテスマを授けるようにと、わたしをおつかわしになったそのかたが、わたしに言われた、『ある人の上に、御霊が下ってとどまるのを見たら、その人こそは、御霊によってバプテスマを授けるかたである』。
\par 34 わたしはそれを見たので、このかたこそ神の子であると、あかしをしたのである」。
\par 35 その翌日、ヨハネはまたふたりの弟子たちと一緒に立っていたが、
\par 36 イエスが歩いておられるのに目をとめて言った、「見よ、神の小羊」。
\par 37 そのふたりの弟子は、ヨハネがそう言うのを聞いて、イエスについて行った。
\par 38 イエスはふり向き、彼らがついてくるのを見て言われた、「何か願いがあるのか」。彼らは言った、「ラビ(訳して言えば、先生)どこにおとまりなのですか」。
\par 39 イエスは彼らに言われた、「きてごらんなさい。そうしたらわかるだろう」。そこで彼らはついて行って、イエスの泊まっておられる所を見た。そして、その日はイエスのところに泊まった。時は午後四時ごろであった。
\par 40 ヨハネから聞いて、イエスについて行ったふたりのうちのひとりは、シモン・ペテロの兄弟アンデレであった。
\par 41 彼はまず自分の兄弟シモンに出会って言った、「わたしたちはメシヤ(訳せば、キリスト)にいま出会った」。
\par 42 そしてシモンをイエスのもとにつれてきた。イエスは彼に目をとめて言われた、「あなたはヨハネの子シモンである。あなたをケパ(訳せば、ペテロ)と呼ぶことにする」。
\par 43 その翌日、イエスはガリラヤに行こうとされたが、ピリポに出会って言われた、「わたしに従ってきなさい」。
\par 44 ピリポは、アンデレとペテロとの町ベツサイダの人であった。
\par 45 このピリポがナタナエルに出会って言った、「わたしたちは、モーセが律法の中にしるしており、預言者たちがしるしていた人、ヨセフの子、ナザレのイエスにいま出会った」。
\par 46 ナタナエルは彼に言った、「ナザレから、なんのよいものが出ようか」。ピリポは彼に言った、「きて見なさい」。
\par 47 イエスはナタナエルが自分の方に来るのを見て、彼について言われた、「見よ、あの人こそ、ほんとうのイスラエル人である。その心には偽りがない」。
\par 48 ナタナエルは言った、「どうしてわたしをご存じなのですか」。イエスは答えて言われた、「ピリポがあなたを呼ぶ前に、わたしはあなたが、いちじくの木の下にいるのを見た」。
\par 49 ナタナエルは答えた、「先生、あなたは神の子です。あなたはイスラエルの王です」。
\par 50 イエスは答えて言われた、「あなたが、いちじくの木の下にいるのを見たと、わたしが言ったので信じるのか。これよりも、もっと大きなことを、あなたは見るであろう」。
\par 51 また言われた、「よくよくあなたがたに言っておく。天が開けて、神の御使たちが人の子の上に上り下りするのを、あなたがたは見るであろう」。

\chapter{2}

\par 1 三日目にガリラヤのカナに婚礼があって、イエスの母がそこにいた。
\par 2 イエスも弟子たちも、その婚礼に招かれた。
\par 3 ぶどう酒がなくなったので、母はイエスに言った、「ぶどう酒がなくなってしまいました」。
\par 4 イエスは母に言われた、「婦人よ、あなたは、わたしと、なんの係わりがありますか。わたしの時は、まだきていません」。
\par 5 母は僕たちに言った、「このかたが、あなたがたに言いつけることは、なんでもして下さい」。
\par 6 そこには、ユダヤ人のきよめのならわしに従って、それぞれ四、五斗もはいる石の水がめが、六つ置いてあった。
\par 7 イエスは彼らに「かめに水をいっぱい入れなさい」と言われたので、彼らは口のところまでいっぱいに入れた。
\par 8 そこで彼らに言われた、「さあ、くんで、料理がしらのところに持って行きなさい」。すると、彼らは持って行った。
\par 9 料理がしらは、ぶどう酒になった水をなめてみたが、それがどこからきたのか知らなかったので、(水をくんだ僕たちは知っていた)花婿を呼んで
\par 10 言った、「どんな人でも、初めによいぶどう酒を出して、酔いがまわったころにわるいのを出すものだ。それだのに、あなたはよいぶどう酒を今までとっておかれました」。
\par 11 イエスは、この最初のしるしをガリラヤのカナで行い、その栄光を現された。そして弟子たちはイエスを信じた。
\par 12 そののち、イエスは、その母、兄弟たち、弟子たちと一緒に、カペナウムに下って、幾日かそこにとどまられた。
\par 13 さて、ユダヤ人の過越の祭が近づいたので、イエスはエルサレムに上られた。
\par 14 そして牛、羊、はとを売る者や両替する者などが宮の庭にすわり込んでいるのをごらんになって、
\par 15 なわでむちを造り、羊も牛もみな宮から追いだし、両替人の金を散らし、その台をひっくりかえし、
\par 16 はとを売る人々には「これらのものを持って、ここから出て行け。わたしの父の家を商売の家とするな」と言われた。
\par 17 弟子たちは、「あなたの家を思う熱心が、わたしを食いつくすであろう」と書いてあることを思い出した。
\par 18 そこで、ユダヤ人はイエスに言った、「こんなことをするからには、どんなしるしをわたしたちに見せてくれますか」。
\par 19 イエスは彼らに答えて言われた、「この神殿をこわしたら、わたしは三日のうちに、それを起すであろう」。
\par 20 そこで、ユダヤ人たちは言った、「この神殿を建てるのには、四十六年もかかっています。それだのに、あなたは三日のうちに、それを建てるのですか」。
\par 21 イエスは自分のからだである神殿のことを言われたのである。
\par 22 それで、イエスが死人の中からよみがえったとき、弟子たちはイエスがこう言われたことを思い出して、聖書とイエスのこの言葉とを信じた。
\par 23 過越の祭の間、イエスがエルサレムに滞在しておられたとき、多くの人々は、その行われたしるしを見て、イエスの名を信じた。
\par 24 しかしイエスご自身は、彼らに自分をお任せにならなかった。それは、すべての人を知っておられ、
\par 25 また人についてあかしする者を、必要とされなかったからである。それは、ご自身人の心の中にあることを知っておられたからである。

\chapter{3}

\par 1 パリサイ人のひとりで、その名をニコデモというユダヤ人の指導者があった。
\par 2 この人が夜イエスのもとにきて言った、「先生、わたしたちはあなたが神からこられた教師であることを知っています。神がご一緒でないなら、あなたがなさっておられるようなしるしは、だれにもできはしません」。
\par 3 イエスは答えて言われた、「よくよくあなたに言っておく。だれでも新しく生れなければ、神の国を見ることはできない」。
\par 4 ニコデモは言った、「人は年をとってから生れることが、どうしてできますか。もう一度、母の胎にはいって生れることができましょうか」。
\par 5 イエスは答えられた、「よくよくあなたに言っておく。だれでも、水と霊とから生れなければ、神の国にはいることはできない。
\par 6 肉から生れる者は肉であり、霊から生れる者は霊である。
\par 7 あなたがたは新しく生れなければならないと、わたしが言ったからとて、不思議に思うには及ばない。
\par 8 風は思いのままに吹く。あなたはその音を聞くが、それがどこからきて、どこへ行くかは知らない。霊から生れる者もみな、それと同じである」。
\par 9 ニコデモはイエスに答えて言った、「どうして、そんなことがあり得ましょうか」。
\par 10 イエスは彼に答えて言われた、「あなたはイスラエルの教師でありながら、これぐらいのことがわからないのか。
\par 11 よくよく言っておく。わたしたちは自分の知っていることを語り、また自分の見たことをあかししているのに、あなたがたはわたしたちのあかしを受けいれない。
\par 12 わたしが地上のことを語っているのに、あなたがたが信じないならば、天上のことを語った場合、どうしてそれを信じるだろうか。
\par 13 天から下ってきた者、すなわち人の子のほかには、だれも天に上った者はない。
\par 14 そして、ちょうどモーセが荒野でへびを上げたように、人の子もまた上げられなければならない。
\par 15 それは彼を信じる者が、すべて永遠の命を得るためである」。
\par 16 神はそのひとり子を賜わったほどに、この世を愛して下さった。それは御子を信じる者がひとりも滅びないで、永遠の命を得るためである。
\par 17 神が御子を世につかわされたのは、世をさばくためではなく、御子によって、この世が救われるためである。
\par 18 彼を信じる者は、さばかれない。信じない者は、すでにさばかれている。神のひとり子の名を信じることをしないからである。
\par 19 そのさばきというのは、光がこの世にきたのに、人々はそのおこないが悪いために、光よりもやみの方を愛したことである。
\par 20 悪を行っている者はみな光を憎む。そして、そのおこないが明るみに出されるのを恐れて、光にこようとはしない。
\par 21 しかし、真理を行っている者は光に来る。その人のおこないの、神にあってなされたということが、明らかにされるためである。
\par 22 こののち、イエスは弟子たちとユダヤの地に行き、彼らと一緒にそこに滞在して、バプテスマを授けておられた。
\par 23 ヨハネもサリムに近いアイノンで、バプテスマを授けていた。そこには水がたくさんあったからである。人々がぞくぞくとやってきてバプテスマを受けていた。
\par 24 そのとき、ヨハネはまだ獄に入れられてはいなかった。
\par 25 ところが、ヨハネの弟子たちとひとりのユダヤ人との間に、きよめのことで争論が起った。
\par 26 そこで彼らはヨハネのところにきて言った、「先生、ごらん下さい。ヨルダンの向こうであなたと一緒にいたことがあり、そして、あなたがあかしをしておられたあのかたが、バプテスマを授けており、皆の者が、そのかたのところへ出かけています」。
\par 27 ヨハネは答えて言った、「人は天から与えられなければ、何ものも受けることはできない。
\par 28 『わたしはキリストではなく、そのかたよりも先につかわされた者である』と言ったことをあかししてくれるのは、あなたがた自身である。
\par 29 花嫁をもつ者は花婿である。花婿の友人は立って彼の声を聞き、その声を聞いて大いに喜ぶ。こうして、この喜びはわたしに満ち足りている。
\par 30 彼は必ず栄え、わたしは衰える。
\par 31 上から来る者は、すべてのものの上にある。地から出る者は、地に属する者であって、地のことを語る。天から来る者は、すべてのものの上にある。
\par 32 彼はその見たところ、聞いたところをあかししているが、だれもそのあかしを受けいれない。
\par 33 しかし、そのあかしを受けいれる者は、神がまことであることを、たしかに認めたのである。
\par 34 神がおつかわしになったかたは、神の言葉を語る。神は聖霊を限りなく賜うからである。
\par 35 父は御子を愛して、万物をその手にお与えになった。
\par 36 御子を信じる者は永遠の命をもつ。御子に従わない者は、命にあずかることがないばかりか、神の怒りがその上にとどまるのである」。

\chapter{4}

\par 1 イエスが、ヨハネよりも多く弟子をつくり、またバプテスマを授けておられるということを、パリサイ人たちが聞き、それを主が知られたとき、
\par 2 (しかし、イエスみずからが、バプテスマをお授けになったのではなく、その弟子たちであった)
\par 3 ユダヤを去って、またガリラヤへ行かれた。
\par 4 しかし、イエスはサマリヤを通過しなければならなかった。
\par 5 そこで、イエスはサマリヤのスカルという町においでになった。この町は、ヤコブがその子ヨセフに与えた土地の近くにあったが、
\par 6 そこにヤコブの井戸があった。イエスは旅の疲れを覚えて、そのまま、この井戸のそばにすわっておられた。時は昼の十二時ごろであった。
\par 7 ひとりのサマリヤの女が水をくみにきたので、イエスはこの女に、「水を飲ませて下さい」と言われた。
\par 8 弟子たちは食物を買いに町に行っていたのである。
\par 9 すると、サマリヤの女はイエスに言った、「あなたはユダヤ人でありながら、どうしてサマリヤの女のわたしに、飲ませてくれとおっしゃるのですか」。これは、ユダヤ人はサマリヤ人と交際していなかったからである。
\par 10 イエスは答えて言われた、「もしあなたが神の賜物のことを知り、また、『水を飲ませてくれ』と言った者が、だれであるか知っていたならば、あなたの方から願い出て、その人から生ける水をもらったことであろう」。
\par 11 女はイエスに言った、「主よ、あなたは、くむ物をお持ちにならず、その上、井戸は深いのです。その生ける水を、どこから手に入れるのですか。
\par 12 あなたは、この井戸を下さったわたしたちの父ヤコブよりも、偉いかたなのですか。ヤコブ自身も飲み、その子らも、その家畜も、この井戸から飲んだのですが」。
\par 13 イエスは女に答えて言われた、「この水を飲む者はだれでも、またかわくであろう。
\par 14 しかし、わたしが与える水を飲む者は、いつまでも、かわくことがないばかりか、わたしが与える水は、その人のうちで泉となり、永遠の命に至る水が、わきあがるであろう」。
\par 15 女はイエスに言った、「主よ、わたしがかわくことがなく、また、ここにくみにこなくてもよいように、その水をわたしに下さい」。
\par 16 イエスは女に言われた、「あなたの夫を呼びに行って、ここに連れてきなさい」。
\par 17 女は答えて言った、「わたしには夫はありません」。イエスは女に言われた、「夫がないと言ったのは、もっともだ。
\par 18 あなたには五人の夫があったが、今のはあなたの夫ではない。あなたの言葉のとおりである」。
\par 19 女はイエスに言った、「主よ、わたしはあなたを預言者と見ます。
\par 20 わたしたちの先祖は、この山で礼拝をしたのですが、あなたがたは礼拝すべき場所は、エルサレムにあると言っています」。
\par 21 イエスは女に言われた、「女よ、わたしの言うことを信じなさい。あなたがたが、この山でも、またエルサレムでもない所で、父を礼拝する時が来る。
\par 22 あなたがたは自分の知らないものを拝んでいるが、わたしたちは知っているかたを礼拝している。救はユダヤ人から来るからである。
\par 23 しかし、まことの礼拝をする者たちが、霊とまこととをもって父を礼拝する時が来る。そうだ、今きている。父は、このような礼拝をする者たちを求めておられるからである。
\par 24 神は霊であるから、礼拝をする者も、霊とまこととをもって礼拝すべきである」。
\par 25 女はイエスに言った、「わたしは、キリストと呼ばれるメシヤがこられることを知っています。そのかたがこられたならば、わたしたちに、いっさいのことを知らせて下さるでしょう」。
\par 26 イエスは女に言われた、「あなたと話をしているこのわたしが、それである」。
\par 27 そのとき、弟子たちが帰って来て、イエスがひとりの女と話しておられるのを見て不思議に思ったが、しかし、「何を求めておられますか」とも、「何を彼女と話しておられるのですか」とも、尋ねる者はひとりもなかった。
\par 28 この女は水がめをそのままそこに置いて町に行き、人々に言った、
\par 29 「わたしのしたことを何もかも、言いあてた人がいます。さあ、見にきてごらんなさい。もしかしたら、この人がキリストかも知れません」。
\par 30 人々は町を出て、ぞくぞくとイエスのところへ行った。
\par 31 その間に弟子たちはイエスに、「先生、召しあがってください」とすすめた。
\par 32 ところが、イエスは言われた、「わたしには、あなたがたの知らない食物がある」。
\par 33 そこで、弟子たちが互に言った、「だれかが、何か食べるものを持ってきてさしあげたのであろうか」。
\par 34 イエスは彼らに言われた、「わたしの食物というのは、わたしをつかわされたかたのみこころを行い、そのみわざをなし遂げることである。
\par 35 あなたがたは、刈入れ時が来るまでには、まだ四か月あると、言っているではないか。しかし、わたしはあなたがたに言う。目をあげて畑を見なさい。はや色づいて刈入れを待っている。
\par 36 刈る者は報酬を受けて、永遠の命に至る実を集めている。まく者も刈る者も、共々に喜ぶためである。
\par 37 そこで、『ひとりがまき、ひとりが刈る』ということわざが、ほんとうのこととなる。
\par 38 わたしは、あなたがたをつかわして、あなたがたがそのために労苦しなかったものを刈りとらせた。ほかの人々が労苦し、あなたがたは、彼らの労苦の実にあずかっているのである」。
\par 39 さて、この町からきた多くのサマリヤ人は、「この人は、わたしのしたことを何もかも言いあてた」とあかしした女の言葉によって、イエスを信じた。
\par 40 そこで、サマリヤ人たちはイエスのもとにきて、自分たちのところに滞在していただきたいと願ったので、イエスはそこにふつか滞在された。
\par 41 そしてなお多くの人々が、イエスの言葉を聞いて信じた。
\par 42 彼らは女に言った、「わたしたちが信じるのは、もうあなたが話してくれたからではない。自分自身で親しく聞いて、この人こそまことに世の救主であることが、わかったからである」。
\par 43 ふつかの後に、イエスはここを去ってガリラヤへ行かれた。
\par 44 イエスはみずからはっきり、「預言者は自分の故郷では敬われないものだ」と言われたのである。
\par 45 ガリラヤに着かれると、ガリラヤの人たちはイエスを歓迎した。それは、彼らも祭に行っていたので、その祭の時、イエスがエルサレムでなされたことをことごとく見ていたからである。
\par 46 イエスは、またガリラヤのカナに行かれた。そこは、かつて水をぶどう酒にかえられた所である。ところが、病気をしているむすこを持つある役人がカペナウムにいた。
\par 47 この人が、ユダヤからガリラヤにイエスのきておられることを聞き、みもとにきて、カペナウムに下って、彼の子をなおしていただきたいと、願った。その子が死にかかっていたからである。
\par 48 そこで、イエスは彼に言われた、「あなたがたは、しるしと奇跡とを見ない限り、決して信じないだろう」。
\par 49 この役人はイエスに言った、「主よ、どうぞ、子供が死なないうちにきて下さい」。
\par 50 イエスは彼に言われた、「お帰りなさい。あなたのむすこは助かるのだ」。彼は自分に言われたイエスの言葉を信じて帰って行った。
\par 51 その下って行く途中、僕たちが彼に出会い、その子が助かったことを告げた。
\par 52 そこで、彼は僕たちに、そのなおりはじめた時刻を尋ねてみたら、「きのうの午後一時に熱が引きました」と答えた。
\par 53 それは、イエスが「あなたのむすこは助かるのだ」と言われたのと同じ時刻であったことを、この父は知って、彼自身もその家族一同も信じた。
\par 54 これは、イエスがユダヤからガリラヤにきてなされた第二のしるしである。

\chapter{5}

\par 1 こののち、ユダヤ人の祭があったので、イエスはエルサレムに上られた。
\par 2 エルサレムにある羊の門のそばに、ヘブル語でベテスダと呼ばれる池があった。そこには五つの廊があった。
\par 3 その廊の中には、病人、盲人、足なえ、やせ衰えた者などが、大ぜいからだを横たえていた。〔彼らは水の動くのを待っていたのである。
\par 4 それは、時々、主の御使がこの池に降りてきて水を動かすことがあるが、水が動いた時まっ先にはいる者は、どんな病気にかかっていても、いやされたからである。〕
\par 5 さて、そこに三十八年のあいだ、病気に悩んでいる人があった。
\par 6 イエスはその人が横になっているのを見、また長い間わずらっていたのを知って、その人に「なおりたいのか」と言われた。
\par 7 この病人はイエスに答えた、「主よ、水が動く時に、わたしを池の中に入れてくれる人がいません。わたしがはいりかけると、ほかの人が先に降りて行くのです」。
\par 8 イエスは彼に言われた、「起きて、あなたの床を取りあげ、そして歩きなさい」。
\par 9 すると、この人はすぐにいやされ、床をとりあげて歩いて行った。その日は安息日であった。
\par 10 そこでユダヤ人たちは、そのいやされた人に言った、「きょうは安息日だ。床を取りあげるのは、よろしくない」。
\par 11 彼は答えた、「わたしをなおして下さったかたが、床を取りあげて歩けと、わたしに言われました」。
\par 12 彼らは尋ねた、「取りあげて歩けと言った人は、だれか」。
\par 13 しかし、このいやされた人は、それがだれであるか知らなかった。群衆がその場にいたので、イエスはそっと出て行かれたからである。
\par 14 そののち、イエスは宮でその人に出会ったので、彼に言われた、「ごらん、あなたはよくなった。もう罪を犯してはいけない。何かもっと悪いことが、あなたの身に起るかも知れないから」。
\par 15 彼は出て行って、自分をいやしたのはイエスであったと、ユダヤ人たちに告げた。
\par 16 そのためユダヤ人たちは、安息日にこのようなことをしたと言って、イエスを責めた。
\par 17 そこで、イエスは彼らに答えられた、「わたしの父は今に至るまで働いておられる。わたしも働くのである」。
\par 18 このためにユダヤ人たちは、ますますイエスを殺そうと計るようになった。それは、イエスが安息日を破られたばかりではなく、神を自分の父と呼んで、自分を神と等しいものとされたからである。
\par 19 さて、イエスは彼らに答えて言われた、「よくよくあなたがたに言っておく。子は父のなさることを見てする以外に、自分からは何事もすることができない。父のなさることであればすべて、子もそのとおりにするのである。
\par 20 なぜなら、父は子を愛して、みずからなさることは、すべて子にお示しになるからである。そして、それよりもなお大きなわざを、お示しになるであろう。あなたがたが、それによって不思議に思うためである。
\par 21 すなわち、父が死人を起して命をお与えになるように、子もまた、そのこころにかなう人々に命を与えるであろう。
\par 22 父はだれをもさばかない。さばきのことはすべて、子にゆだねられたからである。
\par 23 それは、すべての人が父を敬うと同様に、子を敬うためである。子を敬わない者は、子をつかわされた父をも敬わない。
\par 24 よくよくあなたがたに言っておく。わたしの言葉を聞いて、わたしをつかわされたかたを信じる者は、永遠の命を受け、またさばかれることがなく、死から命に移っているのである。
\par 25 よくよくあなたがたに言っておく。死んだ人たちが、神の子の声を聞く時が来る。今すでにきている。そして聞く人は生きるであろう。
\par 26 それは、父がご自分のうちに生命をお持ちになっていると同様に、子にもまた、自分のうちに生命を持つことをお許しになったからである。
\par 27 そして子は人の子であるから、子にさばきを行う権威をお与えになった。
\par 28 このことを驚くには及ばない。墓の中にいる者たちがみな神の子の声を聞き、
\par 29 善をおこなった人々は、生命を受けるためによみがえり、悪をおこなった人々は、さばきを受けるためによみがえって、それぞれ出てくる時が来るであろう。
\par 30 わたしは、自分からは何事もすることができない。ただ聞くままにさばくのである。そして、わたしのこのさばきは正しい。それは、わたし自身の考えでするのではなく、わたしをつかわされたかたの、み旨を求めているからである。
\par 31 もし、わたしが自分自身についてあかしをするならば、わたしのあかしはほんとうではない。
\par 32 わたしについてあかしをするかたはほかにあり、そして、その人がするあかしがほんとうであることを、わたしは知っている。
\par 33 あなたがたはヨハネのもとへ人をつかわしたが、そのとき彼は真理についてあかしをした。
\par 34 わたしは人からあかしを受けないが、このことを言うのは、あなたがたが救われるためである。
\par 35 ヨハネは燃えて輝くあかりであった。あなたがたは、しばらくの間その光を喜び楽しもうとした。
\par 36 しかし、わたしには、ヨハネのあかしよりも、もっと力あるあかしがある。父がわたしに成就させようとしてお与えになったわざ、すなわち、今わたしがしているこのわざが、父のわたしをつかわされたことをあかししている。
\par 37 また、わたしをつかわされた父も、ご自分でわたしについてあかしをされた。あなたがたは、まだそのみ声を聞いたこともなく、そのみ姿を見たこともない。
\par 38 また、神がつかわされた者を信じないから、神の御言はあなたがたのうちにとどまっていない。
\par 39 あなたがたは、聖書の中に永遠の命があると思って調べているが、この聖書は、わたしについてあかしをするものである。
\par 40 しかも、あなたがたは、命を得るためにわたしのもとにこようともしない。
\par 41 わたしは人からの誉を受けることはしない。
\par 42 しかし、あなたがたのうちには神を愛する愛がないことを知っている。
\par 43 わたしは父の名によってきたのに、あなたがたはわたしを受けいれない。もし、ほかの人が彼自身の名によって来るならば、その人を受けいれるのであろう。
\par 44 互に誉を受けながら、ただひとりの神からの誉を求めようとしないあなたがたは、どうして信じることができようか。
\par 45 わたしがあなたがたのことを父に訴えると、考えてはいけない。あなたがたを訴える者は、あなたがたが頼みとしているモーセその人である。
\par 46 もし、あなたがたがモーセを信じたならば、わたしをも信じたであろう。モーセは、わたしについて書いたのである。
\par 47 しかし、モーセの書いたものを信じないならば、どうしてわたしの言葉を信じるだろうか」。

\chapter{6}

\par 1 そののち、イエスはガリラヤの海、すなわち、テベリヤ湖の向こう岸へ渡られた。
\par 2 すると、大ぜいの群衆がイエスについてきた。病人たちになさっていたしるしを見たからである。
\par 3 イエスは山に登って、弟子たちと一緒にそこで座につかれた。
\par 4 時に、ユダヤ人の祭である過越が間近になっていた。
\par 5 イエスは目をあげ、大ぜいの群衆が自分の方に集まって来るのを見て、ピリポに言われた、「どこからパンを買ってきて、この人々に食べさせようか」。
\par 6 これはピリポをためそうとして言われたのであって、ご自分ではしようとすることを、よくご承知であった。
\par 7 すると、ピリポはイエスに答えた、「二百デナリのパンがあっても、めいめいが少しずついただくにも足りますまい」。
\par 8 弟子のひとり、シモン・ペテロの兄弟アンデレがイエスに言った、
\par 9 「ここに、大麦のパン五つと、さかな二ひきとを持っている子供がいます。しかし、こんなに大ぜいの人では、それが何になりましょう」。
\par 10 イエスは「人々をすわらせなさい」と言われた。その場所には草が多かった。そこにすわった男の数は五千人ほどであった。
\par 11 そこで、イエスはパンを取り、感謝してから、すわっている人々に分け与え、また、さかなをも同様にして、彼らの望むだけ分け与えられた。
\par 12 人々がじゅうぶんに食べたのち、イエスは弟子たちに言われた、「少しでもむだにならないように、パンくずのあまりを集めなさい」。
\par 13 そこで彼らが集めると、五つの大麦のパンを食べて残ったパンくずは、十二のかごにいっぱいになった。
\par 14 人々はイエスのなさったこのしるしを見て、「ほんとうに、この人こそ世にきたるべき預言者である」と言った。
\par 15 イエスは人々がきて、自分をとらえて王にしようとしていると知って、ただひとり、また山に退かれた。
\par 16 夕方になったとき、弟子たちは海べに下り、
\par 17 舟に乗って海を渡り、向こう岸のカペナウムに行きかけた。すでに暗くなっていたのに、イエスはまだ彼らのところにおいでにならなかった。
\par 18 その上、強い風が吹いてきて、海は荒れ出した。
\par 19 四、五十丁こぎ出したとき、イエスが海の上を歩いて舟に近づいてこられるのを見て、彼らは恐れた。
\par 20 すると、イエスは彼らに言われた、「わたしだ、恐れることはない」。
\par 21 そこで、彼らは喜んでイエスを舟に迎えようとした。すると舟は、すぐ、彼らが行こうとしていた地に着いた。
\par 22 その翌日、海の向こう岸に立っていた群衆は、そこに小舟が一そうしかなく、またイエスは弟子たちと一緒に小舟にお乗りにならず、ただ弟子たちだけが船出したのを見た。
\par 23 しかし、数そうの小舟がテベリヤからきて、主が感謝されたのちパンを人々に食べさせた場所に近づいた。
\par 24 群衆は、イエスも弟子たちもそこにいないと知って、それらの小舟に乗り、イエスをたずねてカペナウムに行った。
\par 25 そして、海の向こう岸でイエスに出会ったので言った、「先生、いつ、ここにおいでになったのですか」。
\par 26 イエスは答えて言われた、「よくよくあなたがたに言っておく。あなたがたがわたしを尋ねてきているのは、しるしを見たためではなく、パンを食べて満腹したからである。
\par 27 朽ちる食物のためではなく、永遠の命に至る朽ちない食物のために働くがよい。これは人の子があなたがたに与えるものである。父なる神は、人の子にそれをゆだねられたのである」。
\par 28 そこで、彼らはイエスに言った、「神のわざを行うために、わたしたちは何をしたらよいでしょうか」。
\par 29 イエスは彼らに答えて言われた、「神がつかわされた者を信じることが、神のわざである」。
\par 30 彼らはイエスに言った、「わたしたちが見てあなたを信じるために、どんなしるしを行って下さいますか。どんなことをして下さいますか。
\par 31 わたしたちの先祖は荒野でマナを食べました。それは『天よりのパンを彼らに与えて食べさせた』と書いてあるとおりです」。
\par 32 そこでイエスは彼らに言われた、「よくよく言っておく。天からのパンをあなたがたに与えたのは、モーセではない。天からのまことのパンをあなたがたに与えるのは、わたしの父なのである。
\par 33 神のパンは、天から下ってきて、この世に命を与えるものである」。
\par 34 彼らはイエスに言った、「主よ、そのパンをいつもわたしたちに下さい」。
\par 35 イエスは彼らに言われた、「わたしが命のパンである。わたしに来る者は決して飢えることがなく、わたしを信じる者は決してかわくことがない。
\par 36 しかし、あなたがたに言ったが、あなたがたはわたしを見たのに信じようとはしない。
\par 37 父がわたしに与えて下さる者は皆、わたしに来るであろう。そして、わたしに来る者を決して拒みはしない。
\par 38 わたしが天から下ってきたのは、自分のこころのままを行うためではなく、わたしをつかわされたかたのみこころを行うためである。
\par 39 わたしをつかわされたかたのみこころは、わたしに与えて下さった者を、わたしがひとりも失わずに、終りの日によみがえらせることである。
\par 40 わたしの父のみこころは、子を見て信じる者が、ことごとく永遠の命を得ることなのである。そして、わたしはその人々を終りの日によみがえらせるであろう」。
\par 41 ユダヤ人らは、イエスが「わたしは天から下ってきたパンである」と言われたので、イエスについてつぶやき始めた。
\par 42 そして言った、「これはヨセフの子イエスではないか。わたしたちはその父母を知っているではないか。わたしは天から下ってきたと、どうして今いうのか」。
\par 43 イエスは彼らに答えて言われた、「互につぶやいてはいけない。
\par 44 わたしをつかわされた父が引きよせて下さらなければ、だれもわたしに来ることはできない。わたしは、その人々を終りの日によみがえらせるであろう。
\par 45 預言者の書に、『彼らはみな神に教えられるであろう』と書いてある。父から聞いて学んだ者は、みなわたしに来るのである。
\par 46 神から出た者のほかに、だれかが父を見たのではない。その者だけが父を見たのである。
\par 47 よくよくあなたがたに言っておく。信じる者には永遠の命がある。
\par 48 わたしは命のパンである。
\par 49 あなたがたの先祖は荒野でマナを食べたが、死んでしまった。
\par 50 しかし、天から下ってきたパンを食べる人は、決して死ぬことはない。
\par 51 わたしは天から下ってきた生きたパンである。それを食べる者は、いつまでも生きるであろう。わたしが与えるパンは、世の命のために与えるわたしの肉である」。
\par 52 そこで、ユダヤ人らが互に論じて言った、「この人はどうして、自分の肉をわたしたちに与えて食べさせることができようか」。
\par 53 イエスは彼らに言われた、「よくよく言っておく。人の子の肉を食べず、また、その血を飲まなければ、あなたがたの内に命はない。
\par 54 わたしの肉を食べ、わたしの血を飲む者には、永遠の命があり、わたしはその人を終りの日によみがえらせるであろう。
\par 55 わたしの肉はまことの食物、わたしの血はまことの飲み物である。
\par 56 わたしの肉を食べ、わたしの血を飲む者はわたしにおり、わたしもまたその人におる。
\par 57 生ける父がわたしをつかわされ、また、わたしが父によって生きているように、わたしを食べる者もわたしによって生きるであろう。
\par 58 天から下ってきたパンは、先祖たちが食べたが死んでしまったようなものではない。このパンを食べる者は、いつまでも生きるであろう」。
\par 59 これらのことは、イエスがカペナウムの会堂で教えておられたときに言われたものである。
\par 60 弟子たちのうちの多くの者は、これを聞いて言った、「これは、ひどい言葉だ。だれがそんなことを聞いておられようか」。
\par 61 しかしイエスは、弟子たちがそのことでつぶやいているのを見破って、彼らに言われた、「このことがあなたがたのつまずきになるのか。
\par 62 それでは、もし人の子が前にいた所に上るのを見たら、どうなるのか。
\par 63 人を生かすものは霊であって、肉はなんの役にも立たない。わたしがあなたがたに話した言葉は霊であり、また命である。
\par 64 しかし、あなたがたの中には信じない者がいる」。イエスは、初めから、だれが信じないか、また、だれが彼を裏切るかを知っておられたのである。
\par 65 そしてイエスは言われた、「それだから、父が与えて下さった者でなければ、わたしに来ることはできないと、言ったのである」。
\par 66 それ以来、多くの弟子たちは去っていって、もはやイエスと行動を共にしなかった。
\par 67 そこでイエスは十二弟子に言われた、「あなたがたも去ろうとするのか」。
\par 68 シモン・ペテロが答えた、「主よ、わたしたちは、だれのところに行きましょう。永遠の命の言をもっているのはあなたです。
\par 69 わたしたちは、あなたが神の聖者であることを信じ、また知っています」。
\par 70 イエスは彼らに答えられた、「あなたがた十二人を選んだのは、わたしではなかったか。それだのに、あなたがたのうちのひとりは悪魔である」。
\par 71 これは、イスカリオテのシモンの子ユダをさして言われたのである。このユダは、十二弟子のひとりでありながら、イエスを裏切ろうとしていた。

\chapter{7}

\par 1 そののち、イエスはガリラヤを巡回しておられた。ユダヤ人たちが自分を殺そうとしていたので、ユダヤを巡回しようとはされなかった。
\par 2 時に、ユダヤ人の仮庵の祭が近づいていた。
\par 3 そこで、イエスの兄弟たちがイエスに言った、「あなたがしておられるわざを弟子たちにも見せるために、ここを去りユダヤに行ってはいかがです。
\par 4 自分を公けにあらわそうと思っている人で、隠れて仕事をするものはありません。あなたがこれらのことをするからには、自分をはっきりと世にあらわしなさい」。
\par 5 こう言ったのは、兄弟たちもイエスを信じていなかったからである。
\par 6 そこでイエスは彼らに言われた、「わたしの時はまだきていない。しかし、あなたがたの時はいつも備わっている。
\par 7 世はあなたがたを憎み得ないが、わたしを憎んでいる。わたしが世のおこないの悪いことを、あかししているからである。
\par 8 あなたがたこそ祭に行きなさい。わたしはこの祭には行かない。わたしの時はまだ満ちていないから」。
\par 9 彼らにこう言って、イエスはガリラヤにとどまっておられた。
\par 10 しかし、兄弟たちが祭に行ったあとで、イエスも人目にたたぬように、ひそかに行かれた。
\par 11 ユダヤ人らは祭の時に、「あの人はどこにいるのか」と言って、イエスを捜していた。
\par 12 群衆の中に、イエスについていろいろとうわさが立った。ある人々は、「あれはよい人だ」と言い、他の人々は、「いや、あれは群衆を惑わしている」と言った。
\par 13 しかし、ユダヤ人らを恐れて、イエスのことを公然と口にする者はいなかった。
\par 14 祭も半ばになってから、イエスは宮に上って教え始められた。
\par 15 すると、ユダヤ人たちは驚いて言った、「この人は学問をしたこともないのに、どうして律法の知識をもっているのだろう」。
\par 16 そこでイエスは彼らに答えて言われた、「わたしの教はわたし自身の教ではなく、わたしをつかわされたかたの教である。
\par 17 神のみこころを行おうと思う者であれば、だれでも、わたしの語っているこの教が神からのものか、それとも、わたし自身から出たものか、わかるであろう。
\par 18 自分から出たことを語る者は、自分の栄光を求めるが、自分をつかわされたかたの栄光を求める者は真実であって、その人の内には偽りがない。
\par 19 モーセはあなたがたに律法を与えたではないか。それだのに、あなたがたのうちには、その律法を行う者がひとりもない。あなたがたは、なぜわたしを殺そうと思っているのか」。
\par 20 群衆は答えた、「あなたは悪霊に取りつかれている。だれがあなたを殺そうと思っているものか」。
\par 21 イエスは彼らに答えて言われた、「わたしが一つのわざをしたところ、あなたがたは皆それを見て驚いている。
\par 22 モーセはあなたがたに割礼を命じたので、(これは、実は、モーセから始まったのではなく、先祖たちから始まったものである)あなたがたは安息日にも人に割礼を施している。
\par 23 もし、モーセの律法が破られないように、安息日であっても割礼を受けるのなら、安息日に人の全身を丈夫にしてやったからといって、どうして、そんなにおこるのか。
\par 24 うわべで人をさばかないで、正しいさばきをするがよい」。
\par 25 さて、エルサレムのある人たちが言った、「この人は人々が殺そうと思っている者ではないか。
\par 26 見よ、彼は公然と語っているのに、人々はこれに対して何も言わない。役人たちは、この人がキリストであることを、ほんとうに知っているのではなかろうか。
\par 27 わたしたちはこの人がどこからきたのか知っている。しかし、キリストが現れる時には、どこから来るのか知っている者は、ひとりもいない」。
\par 28 イエスは宮の内で教えながら、叫んで言われた、「あなたがたは、わたしを知っており、また、わたしがどこからきたかも知っている。しかし、わたしは自分からきたのではない。わたしをつかわされたかたは真実であるが、あなたがたは、そのかたを知らない。
\par 29 わたしは、そのかたを知っている。わたしはそのかたのもとからきた者で、そのかたがわたしをつかわされたのである」。
\par 30 そこで人々はイエスを捕えようと計ったが、だれひとり手をかける者はなかった。イエスの時が、まだきていなかったからである。
\par 31 しかし、群衆の中の多くの者が、イエスを信じて言った、「キリストがきても、この人が行ったよりも多くのしるしを行うだろうか」。
\par 32 群衆がイエスについてこのようなうわさをしているのを、パリサイ人たちは耳にした。そこで、祭司長たちやパリサイ人たちは、イエスを捕えようとして、下役どもをつかわした。
\par 33 イエスは言われた、「今しばらくの間、わたしはあなたがたと一緒にいて、それから、わたしをおつかわしになったかたのみもとに行く。
\par 34 あなたがたはわたしを捜すであろうが、見つけることはできない。そしてわたしのいる所に、あなたがたは来ることができない」。
\par 35 そこでユダヤ人たちは互に言った、「わたしたちが見つけることができないというのは、どこへ行こうとしているのだろう。ギリシヤ人の中に離散している人たちのところにでも行って、ギリシヤ人を教えようというのだろうか。
\par 36 また、『わたしを捜すが、見つけることはできない。そしてわたしのいる所には来ることができないだろう』と言ったその言葉は、どういう意味だろう」。
\par 37 祭の終りの大事な日に、イエスは立って、叫んで言われた、「だれでもかわく者は、わたしのところにきて飲むがよい。
\par 38 わたしを信じる者は、聖書に書いてあるとおり、その腹から生ける水が川となって流れ出るであろう」。
\par 39 これは、イエスを信じる人々が受けようとしている御霊をさして言われたのである。すなわち、イエスはまだ栄光を受けておられなかったので、御霊がまだ下っていなかったのである。
\par 40 群衆のある者がこれらの言葉を聞いて、「このかたは、ほんとうに、あの預言者である」と言い、
\par 41 ほかの人たちは「このかたはキリストである」と言い、また、ある人々は、「キリストはまさか、ガリラヤからは出てこないだろう。
\par 42 キリストは、ダビデの子孫から、またダビデのいたベツレヘムの村から出ると、聖書に書いてあるではないか」と言った。
\par 43 こうして、群衆の間にイエスのことで分争が生じた。
\par 44 彼らのうちのある人々は、イエスを捕えようと思ったが、だれひとり手をかける者はなかった。
\par 45 さて、下役どもが祭司長たちやパリサイ人たちのところに帰ってきたので、彼らはその下役どもに言った、「なぜ、あの人を連れてこなかったのか」。
\par 46 下役どもは答えた、「この人の語るように語った者は、これまでにありませんでした」。
\par 47 パリサイ人たちが彼らに答えた、「あなたがたまでが、だまされているのではないか。
\par 48 役人たちやパリサイ人たちの中で、ひとりでも彼を信じた者があっただろうか。
\par 49 律法をわきまえないこの群衆は、のろわれている」。
\par 50 彼らの中のひとりで、以前にイエスに会いにきたことのあるニコデモが、彼らに言った、
\par 51 「わたしたちの律法によれば、まずその人の言い分を聞き、その人のしたことを知った上でなければ、さばくことをしないのではないか」。
\par 52 彼らは答えて言った、「あなたもガリラヤ出なのか。よく調べてみなさい、ガリラヤからは預言者が出るものではないことが、わかるだろう」。〔
\par 53 そして、人々はおのおの家に帰って行った。

\chapter{8}

\par 1 イエスはオリブ山に行かれた。
\par 2 朝早くまた宮にはいられると、人々が皆みもとに集まってきたので、イエスはすわって彼らを教えておられた。
\par 3 すると、律法学者たちやパリサイ人たちが、姦淫をしている時につかまえられた女をひっぱってきて、中に立たせた上、イエスに言った、
\par 4 「先生、この女は姦淫の場でつかまえられました。
\par 5 モーセは律法の中で、こういう女を石で打ち殺せと命じましたが、あなたはどう思いますか」。
\par 6 彼らがそう言ったのは、イエスをためして、訴える口実を得るためであった。しかし、イエスは身をかがめて、指で地面に何か書いておられた。
\par 7 彼らが問い続けるので、イエスは身を起して彼らに言われた、「あなたがたの中で罪のない者が、まずこの女に石を投げつけるがよい」。
\par 8 そしてまた身をかがめて、地面に物を書きつづけられた。
\par 9 これを聞くと、彼らは年寄から始めて、ひとりびとり出て行き、ついに、イエスだけになり、女は中にいたまま残された。
\par 10 そこでイエスは身を起して女に言われた、「女よ、みんなはどこにいるか。あなたを罰する者はなかったのか」。
\par 11 女は言った、「主よ、だれもございません」。イエスは言われた、「わたしもあなたを罰しない。お帰りなさい。今後はもう罪を犯さないように」。〕
\par 12 イエスは、また人々に語ってこう言われた、「わたしは世の光である。わたしに従って来る者は、やみのうちを歩くことがなく、命の光をもつであろう」。
\par 13 するとパリサイ人たちがイエスに言った、「あなたは、自分のことをあかししている。あなたのあかしは真実ではない」。
\par 14 イエスは彼らに答えて言われた、「たとい、わたしが自分のことをあかししても、わたしのあかしは真実である。それは、わたしがどこからきたのか、また、どこへ行くのかを知っているからである。しかし、あなたがたは、わたしがどこからきて、どこへ行くのかを知らない。
\par 15 あなたがたは肉によって人をさばくが、わたしはだれもさばかない。
\par 16 しかし、もしわたしがさばくとすれば、わたしのさばきは正しい。なぜなら、わたしはひとりではなく、わたしをつかわされたかたが、わたしと一緒だからである。
\par 17 あなたがたの律法には、ふたりによる証言は真実だと、書いてある。
\par 18 わたし自身のことをあかしするのは、わたしであるし、わたしをつかわされた父も、わたしのことをあかしして下さるのである」。
\par 19 すると、彼らはイエスに言った、「あなたの父はどこにいるのか」。イエスは答えられた、「あなたがたは、わたしをもわたしの父をも知っていない。もし、あなたがたがわたしを知っていたなら、わたしの父をも知っていたであろう」。
\par 20 イエスが宮の内で教えていた時、これらの言葉をさいせん箱のそばで語られたのであるが、イエスの時がまだきていなかったので、だれも捕える者がなかった。
\par 21 さて、また彼らに言われた、「わたしは去って行く。あなたがたはわたしを捜し求めるであろう。そして自分の罪のうちに死ぬであろう。わたしの行く所には、あなたがたは来ることができない」。
\par 22 そこでユダヤ人たちは言った、「わたしの行く所に、あなたがたは来ることができないと、言ったのは、あるいは自殺でもしようとするつもりか」。
\par 23 イエスは彼らに言われた、「あなたがたは下から出た者だが、わたしは上からきた者である。あなたがたはこの世の者であるが、わたしはこの世の者ではない。
\par 24 だからわたしは、あなたがたは自分の罪のうちに死ぬであろうと、言ったのである。もしわたしがそういう者であることをあなたがたが信じなければ、罪のうちに死ぬことになるからである」。
\par 25 そこで彼らはイエスに言った、「あなたは、いったい、どういうかたですか」。イエスは彼らに言われた、「わたしがどういう者であるかは、初めからあなたがたに言っているではないか。
\par 26 あなたがたについて、わたしの言うべきこと、さばくべきことが、たくさんある。しかし、わたしをつかわされたかたは真実なかたである。わたしは、そのかたから聞いたままを世にむかって語るのである」。
\par 27 彼らは、イエスが父について話しておられたことを悟らなかった。
\par 28 そこでイエスは言われた、「あなたがたが人の子を上げてしまった後はじめて、わたしがそういう者であること、また、わたしは自分からは何もせず、ただ父が教えて下さったままを話していたことが、わかってくるであろう。
\par 29 わたしをつかわされたかたは、わたしと一緒におられる。わたしは、いつも神のみこころにかなうことをしているから、わたしをひとり置きざりになさることはない」。
\par 30 これらのことを語られたところ、多くの人々がイエスを信じた。
\par 31 イエスは自分を信じたユダヤ人たちに言われた、「もしわたしの言葉のうちにとどまっておるなら、あなたがたは、ほんとうにわたしの弟子なのである。
\par 32 また真理を知るであろう。そして真理は、あなたがたに自由を得させるであろう」。
\par 33 そこで、彼らはイエスに言った、「わたしたちはアブラハムの子孫であって、人の奴隷になったことなどは、一度もない。どうして、あなたがたに自由を得させるであろうと、言われるのか」。
\par 34 イエスは彼らに答えられた、「よくよくあなたがたに言っておく。すべて罪を犯す者は罪の奴隷である。
\par 35 そして、奴隷はいつまでも家にいる者ではない。しかし、子はいつまでもいる。
\par 36 だから、もし子があなたがたに自由を得させるならば、あなたがたは、ほんとうに自由な者となるのである。
\par 37 わたしは、あなたがたがアブラハムの子孫であることを知っている。それだのに、あなたがたはわたしを殺そうとしている。わたしの言葉が、あなたがたのうちに根をおろしていないからである。
\par 38 わたしはわたしの父のもとで見たことを語っているが、あなたがたは自分の父から聞いたことを行っている」。
\par 39 彼らはイエスに答えて言った、「わたしたちの父はアブラハムである」。イエスは彼らに言われた、「もしアブラハムの子であるなら、アブラハムのわざをするがよい。
\par 40 ところが今、神から聞いた真理をあなたがたに語ってきたこのわたしを、殺そうとしている。そんなことをアブラハムはしなかった。
\par 41 あなたがたは、あなたがたの父のわざを行っているのである」。彼らは言った、「わたしたちは、不品行の結果うまれた者ではない。わたしたちにはひとりの父がある。それは神である」。
\par 42 イエスは彼らに言われた、「神があなたがたの父であるならば、あなたがたはわたしを愛するはずである。わたしは神から出た者、また神からきている者であるからだ。わたしは自分からきたのではなく、神からつかわされたのである。
\par 43 どうしてあなたがたは、わたしの話すことがわからないのか。あなたがたが、わたしの言葉を悟ることができないからである。
\par 44 あなたがたは自分の父、すなわち、悪魔から出てきた者であって、その父の欲望どおりを行おうと思っている。彼は初めから、人殺しであって、真理に立つ者ではない。彼のうちには真理がないからである。彼が偽りを言うとき、いつも自分の本音をはいているのである。彼は偽り者であり、偽りの父であるからだ。
\par 45 しかし、わたしが真理を語っているので、あなたがたはわたしを信じようとしない。
\par 46 あなたがたのうち、だれがわたしに罪があると責めうるのか。わたしは真理を語っているのに、なぜあなたがたは、わたしを信じないのか。
\par 47 神からきた者は神の言葉に聞き従うが、あなたがたが聞き従わないのは、神からきた者でないからである」。
\par 48 ユダヤ人たちはイエスに答えて言った、「あなたはサマリヤ人で、悪霊に取りつかれていると、わたしたちが言うのは、当然ではないか」。
\par 49 イエスは答えられた、「わたしは、悪霊に取りつかれているのではなくて、わたしの父を重んじているのだが、あなたがたはわたしを軽んじている。
\par 50 わたしは自分の栄光を求めてはいない。それを求めるかたが別にある。そのかたは、またさばくかたである。
\par 51 よくよく言っておく。もし人がわたしの言葉を守るならば、その人はいつまでも死を見ることがないであろう」。
\par 52 ユダヤ人たちが言った、「あなたが悪霊に取りつかれていることが、今わかった。アブラハムは死に、預言者たちも死んでいる。それだのに、あなたは、わたしの言葉を守る者はいつまでも死を味わうことがないであろうと、言われる。
\par 53 あなたは、わたしたちの父アブラハムより偉いのだろうか。彼も死に、預言者たちも死んだではないか。あなたは、いったい、自分をだれと思っているのか」。
\par 54 イエスは答えられた、「わたしがもし自分に栄光を帰するなら、わたしの栄光は、むなしいものである。わたしに栄光を与えるかたは、わたしの父であって、あなたがたが自分の神だと言っているのは、そのかたのことである。
\par 55 あなたがたはその神を知っていないが、わたしは知っている。もしわたしが神を知らないと言うならば、あなたがたと同じような偽り者であろう。しかし、わたしはそのかたを知り、その御言を守っている。
\par 56 あなたがたの父アブラハムは、わたしのこの日を見ようとして楽しんでいた。そしてそれを見て喜んだ」。
\par 57 そこでユダヤ人たちはイエスに言った、「あなたはまだ五十にもならないのに、アブラハムを見たのか」。
\par 58 イエスは彼らに言われた、「よくよくあなたがたに言っておく。アブラハムの生れる前からわたしは、いるのである」。
\par 59 そこで彼らは石をとって、イエスに投げつけようとした。しかし、イエスは身を隠して、宮から出て行かれた。

\chapter{9}

\par 1 イエスが道をとおっておられるとき、生れつきの盲人を見られた。
\par 2 弟子たちはイエスに尋ねて言った、「先生、この人が生れつき盲人なのは、だれが罪を犯したためですか。本人ですか、それともその両親ですか」。
\par 3 イエスは答えられた、「本人が罪を犯したのでもなく、また、その両親が犯したのでもない。ただ神のみわざが、彼の上に現れるためである。
\par 4 わたしたちは、わたしをつかわされたかたのわざを、昼の間にしなければならない。夜が来る。すると、だれも働けなくなる。
\par 5 わたしは、この世にいる間は、世の光である」。
\par 6 イエスはそう言って、地につばきをし、そのつばきで、どろをつくり、そのどろを盲人の目に塗って言われた、
\par 7 「シロアム(つかわされた者、の意)の池に行って洗いなさい」。そこで彼は行って洗った。そして見えるようになって、帰って行った。
\par 8 近所の人々や、彼がもと、こじきであったのを見知っていた人々が言った、「この人は、すわってこじきをしていた者ではないか」。
\par 9 ある人々は「その人だ」と言い、他の人々は「いや、ただあの人に似ているだけだ」と言った。しかし、本人は「わたしがそれだ」と言った。
\par 10 そこで人々は彼に言った、「では、おまえの目はどうしてあいたのか」。
\par 11 彼は答えた、「イエスというかたが、どろをつくって、わたしの目に塗り、『シロアムに行って洗え』と言われました。それで、行って洗うと、見えるようになりました」。
\par 12 人々は彼に言った、「その人はどこにいるのか」。彼は「知りません」と答えた。
\par 13 人々は、もと盲人であったこの人を、パリサイ人たちのところにつれて行った。
\par 14 イエスがどろをつくって彼の目をあけたのは、安息日であった。
\par 15 パリサイ人たちもまた、「どうして見えるようになったのか」、と彼に尋ねた。彼は答えた、「あのかたがわたしの目にどろを塗り、わたしがそれを洗い、そして見えるようになりました」。
\par 16 そこで、あるパリサイ人たちが言った、「その人は神からきた人ではない。安息日を守っていないのだから」。しかし、ほかの人々は言った、「罪のある人が、どうしてそのようなしるしを行うことができようか」。そして彼らの間に分争が生じた。
\par 17 そこで彼らは、もう一度この盲人に聞いた、「おまえの目をあけてくれたその人を、どう思うか」。「預言者だと思います」と彼は言った。
\par 18 ユダヤ人たちは、彼がもと盲人であったが見えるようになったことを、まだ信じなかった。ついに彼らは、目が見えるようになったこの人の両親を呼んで、
\par 19 尋ねて言った、「これが、生れつき盲人であったと、おまえたちの言っているむすこか。それではどうして、いま目が見えるのか」。
\par 20 両親は答えて言った、「これがわたしどものむすこであること、また生れつき盲人であったことは存じています。
\par 21 しかし、どうしていま見えるようになったのか、それは知りません。また、だれがその目をあけて下さったのかも知りません。あれに聞いて下さい。あれはもうおとなですから、自分のことは自分で話せるでしょう」。
\par 22 両親はユダヤ人たちを恐れていたので、こう答えたのである。それは、もしイエスをキリストと告白する者があれば、会堂から追い出すことに、ユダヤ人たちが既に決めていたからである。
\par 23 彼の両親が「おとなですから、あれに聞いて下さい」と言ったのは、そのためであった。
\par 24 そこで彼らは、盲人であった人をもう一度呼んで言った、「神に栄光を帰するがよい。あの人が罪人であることは、わたしたちにはわかっている」。
\par 25 すると彼は言った、「あのかたが罪人であるかどうか、わたしは知りません。ただ一つのことだけ知っています。わたしは盲であったが、今は見えるということです」。
\par 26 そこで彼らは言った、「その人はおまえに何をしたのか。どんなにしておまえの目をあけたのか」。
\par 27 彼は答えた、「そのことはもう話してあげたのに、聞いてくれませんでした。なぜまた聞こうとするのですか。あなたがたも、あの人の弟子になりたいのですか」。
\par 28 そこで彼らは彼をののしって言った、「おまえはあれの弟子だが、わたしたちはモーセの弟子だ。
\par 29 モーセに神が語られたということは知っている。だが、あの人がどこからきた者か、わたしたちは知らぬ」。
\par 30 そこで彼が答えて言った、「わたしの目をあけて下さったのに、そのかたがどこからきたか、ご存じないとは、不思議千万です。
\par 31 わたしたちはこのことを知っています。神は罪人の言うことはお聞きいれになりませんが、神を敬い、そのみこころを行う人の言うことは、聞きいれて下さいます。
\par 32 生れつき盲であった者の目をあけた人があるということは、世界が始まって以来、聞いたことがありません。
\par 33 もしあのかたが神からきた人でなかったら、何一つできなかったはずです」。
\par 34 これを聞いて彼らは言った、「おまえは全く罪の中に生れていながら、わたしたちを教えようとするのか」。そして彼を外へ追い出した。
\par 35 イエスは、その人が外へ追い出されたことを聞かれた。そして彼に会って言われた、「あなたは人の子を信じるか」。
\par 36 彼は答えて言った、「主よ、それはどなたですか。そのかたを信じたいのですが」。
\par 37 イエスは彼に言われた、「あなたは、もうその人に会っている。今あなたと話しているのが、その人である」。
\par 38 すると彼は、「主よ、信じます」と言って、イエスを拝した。
\par 39 そこでイエスは言われた、「わたしがこの世にきたのは、さばくためである。すなわち、見えない人たちが見えるようになり、見える人たちが見えないようになるためである」。
\par 40 そこにイエスと一緒にいたあるパリサイ人たちが、それを聞いてイエスに言った、「それでは、わたしたちも盲なのでしょうか」。
\par 41 イエスは彼らに言われた、「もしあなたがたが盲人であったなら、罪はなかったであろう。しかし、今あなたがたが『見える』と言い張るところに、あなたがたの罪がある。

\chapter{10}

\par 1 よくよくあなたがたに言っておく。羊の囲いにはいるのに、門からでなく、ほかの所からのりこえて来る者は、盗人であり、強盗である。
\par 2 門からはいる者は、羊の羊飼である。
\par 3 門番は彼のために門を開き、羊は彼の声を聞く。そして彼は自分の羊の名をよんで連れ出す。
\par 4 自分の羊をみな出してしまうと、彼は羊の先頭に立って行く。羊はその声を知っているので、彼について行くのである。
\par 5 ほかの人には、ついて行かないで逃げ去る。その人の声を知らないからである」。
\par 6 イエスは彼らにこの比喩を話されたが、彼らは自分たちにお話しになっているのが何のことだか、わからなかった。
\par 7 そこで、イエスはまた言われた、「よくよくあなたがたに言っておく。わたしは羊の門である。
\par 8 わたしよりも前にきた人は、みな盗人であり、強盗である。羊は彼らに聞き従わなかった。
\par 9 わたしは門である。わたしをとおってはいる者は救われ、また出入りし、牧草にありつくであろう。
\par 10 盗人が来るのは、盗んだり、殺したり、滅ぼしたりするためにほかならない。わたしがきたのは、羊に命を得させ、豊かに得させるためである。
\par 11 わたしはよい羊飼である。よい羊飼は、羊のために命を捨てる。
\par 12 羊飼ではなく、羊が自分のものでもない雇人は、おおかみが来るのを見ると、羊をすてて逃げ去る。そして、おおかみは羊を奪い、また追い散らす。
\par 13 彼は雇人であって、羊のことを心にかけていないからである。
\par 14 わたしはよい羊飼であって、わたしの羊を知り、わたしの羊はまた、わたしを知っている。
\par 15 それはちょうど、父がわたしを知っておられ、わたしが父を知っているのと同じである。そして、わたしは羊のために命を捨てるのである。
\par 16 わたしにはまた、この囲いにいない他の羊がある。わたしは彼らをも導かねばならない。彼らも、わたしの声に聞き従うであろう。そして、ついに一つの群れ、ひとりの羊飼となるであろう。
\par 17 父は、わたしが自分の命を捨てるから、わたしを愛して下さるのである。命を捨てるのは、それを再び得るためである。
\par 18 だれかが、わたしからそれを取り去るのではない。わたしが、自分からそれを捨てるのである。わたしには、それを捨てる力があり、またそれを受ける力もある。これはわたしの父から授かった定めである」。
\par 19 これらの言葉を語られたため、ユダヤ人の間にまたも分争が生じた。
\par 20 そのうちの多くの者が言った、「彼は悪霊に取りつかれて、気が狂っている。どうして、あなたがたはその言うことを聞くのか」。
\par 21 他の人々は言った、「それは悪霊に取りつかれた者の言葉ではない。悪霊は盲人の目をあけることができようか」。
\par 22 そのころ、エルサレムで宮きよめの祭が行われた。時は冬であった。
\par 23 イエスは、宮の中にあるソロモンの廊を歩いておられた。
\par 24 するとユダヤ人たちが、イエスを取り囲んで言った、「いつまでわたしたちを不安のままにしておくのか。あなたがキリストであるなら、そうとはっきり言っていただきたい」。
\par 25 イエスは彼らに答えられた、「わたしは話したのだが、あなたがたは信じようとしない。わたしの父の名によってしているすべてのわざが、わたしのことをあかししている。
\par 26 あなたがたが信じないのは、わたしの羊でないからである。
\par 27 わたしの羊はわたしの声に聞き従う。わたしは彼らを知っており、彼らはわたしについて来る。
\par 28 わたしは、彼らに永遠の命を与える。だから、彼らはいつまでも滅びることがなく、また、彼らをわたしの手から奪い去る者はない。
\par 29 わたしの父がわたしに下さったものは、すべてにまさるものである。そしてだれも父のみ手から、それを奪い取ることはできない。
\par 30 わたしと父とは一つである」。
\par 31 そこでユダヤ人たちは、イエスを打ち殺そうとして、また石を取りあげた。
\par 32 するとイエスは彼らに答えられた、「わたしは、父による多くのよいわざを、あなたがたに示した。その中のどのわざのために、わたしを石で打ち殺そうとするのか」。
\par 33 ユダヤ人たちは答えた、「あなたを石で殺そうとするのは、よいわざをしたからではなく、神を汚したからである。また、あなたは人間であるのに、自分を神としているからである」。
\par 34 イエスは彼らに答えられた、「あなたがたの律法に、『わたしは言う、あなたがたは神々である』と書いてあるではないか。
\par 35 神の言を託された人々が、神々といわれておるとすれば、(そして聖書の言は、すたることがあり得ない)
\par 36 父が聖別して、世につかわされた者が、『わたしは神の子である』と言ったからとて、どうして『あなたは神を汚す者だ』と言うのか。
\par 37 もしわたしが父のわざを行わないとすれば、わたしを信じなくてもよい。
\par 38 しかし、もし行っているなら、たといわたしを信じなくても、わたしのわざを信じるがよい。そうすれば、父がわたしにおり、また、わたしが父におることを知って悟るであろう」。
\par 39 そこで、彼らはまたイエスを捕えようとしたが、イエスは彼らの手をのがれて、去って行かれた。
\par 40 さて、イエスはまたヨルダンの向こう岸、すなわち、ヨハネが初めにバプテスマを授けていた所に行き、そこに滞在しておられた。
\par 41 多くの人々がイエスのところにきて、互に言った、「ヨハネはなんのしるしも行わなかったが、ヨハネがこのかたについて言ったことは、皆ほんとうであった」。
\par 42 そして、そこで多くの者がイエスを信じた。

\chapter{11}

\par 1 さて、ひとりの病人がいた。ラザロといい、マリヤとその姉妹マルタの村ベタニヤの人であった。
\par 2 このマリヤは主に香油をぬり、自分の髪の毛で、主の足をふいた女であって、病気であったのは、彼女の兄弟ラザロであった。
\par 3 姉妹たちは人をイエスのもとにつかわして、「主よ、ただ今、あなたが愛しておられる者が病気をしています」と言わせた。
\par 4 イエスはそれを聞いて言われた、「この病気は死ぬほどのものではない。それは神の栄光のため、また、神の子がそれによって栄光を受けるためのものである」。
\par 5 イエスは、マルタとその姉妹とラザロとを愛しておられた。
\par 6 ラザロが病気であることを聞いてから、なおふつか、そのおられた所に滞在された。
\par 7 それから弟子たちに、「もう一度ユダヤに行こう」と言われた。
\par 8 弟子たちは言った、「先生、ユダヤ人らが、さきほどもあなたを石で殺そうとしていましたのに、またそこに行かれるのですか」。
\par 9 イエスは答えられた、「一日には十二時間あるではないか。昼間あるけば、人はつまずくことはない。この世の光を見ているからである。
\par 10 しかし、夜あるけば、つまずく。その人のうちに、光がないからである」。
\par 11 そう言われたが、それからまた、彼らに言われた、「わたしたちの友ラザロが眠っている。わたしは彼を起しに行く」。
\par 12 すると弟子たちは言った、「主よ、眠っているのでしたら、助かるでしょう」。
\par 13 イエスはラザロが死んだことを言われたのであるが、弟子たちは、眠って休んでいることをさして言われたのだと思った。
\par 14 するとイエスは、あからさまに彼らに言われた、「ラザロは死んだのだ。
\par 15 そして、わたしがそこにいあわせなかったことを、あなたがたのために喜ぶ。それは、あなたがたが信じるようになるためである。では、彼のところに行こう」。
\par 16 するとデドモと呼ばれているトマスが、仲間の弟子たちに言った、「わたしたちも行って、先生と一緒に死のうではないか」。
\par 17 さて、イエスが行ってごらんになると、ラザロはすでに四日間も墓の中に置かれていた。
\par 18 ベタニヤはエルサレムに近く、二十五丁ばかり離れたところにあった。
\par 19 大ぜいのユダヤ人が、その兄弟のことで、マルタとマリヤとを慰めようとしてきていた。
\par 20 マルタはイエスがこられたと聞いて、出迎えに行ったが、マリヤは家ですわっていた。
\par 21 マルタはイエスに言った、「主よ、もしあなたがここにいて下さったなら、わたしの兄弟は死ななかったでしょう。
\par 22 しかし、あなたがどんなことをお願いになっても、神はかなえて下さることを、わたしは今でも存じています」。
\par 23 イエスはマルタに言われた、「あなたの兄弟はよみがえるであろう」。
\par 24 マルタは言った、「終りの日のよみがえりの時よみがえることは、存じています」。
\par 25 イエスは彼女に言われた、「わたしはよみがえりであり、命である。わたしを信じる者は、たとい死んでも生きる。
\par 26 また、生きていて、わたしを信じる者は、いつまでも死なない。あなたはこれを信じるか」。
\par 27 マルタはイエスに言った、「主よ、信じます。あなたがこの世にきたるべきキリスト、神の御子であると信じております」。
\par 28 マルタはこう言ってから、帰って姉妹のマリヤを呼び、「先生がおいでになって、あなたを呼んでおられます」と小声で言った。
\par 29 これを聞いたマリヤはすぐ立ち上がって、イエスのもとに行った。
\par 30 イエスはまだ村に、はいってこられず、マルタがお迎えしたその場所におられた。
\par 31 マリヤと一緒に家にいて彼女を慰めていたユダヤ人たちは、マリヤが急いで立ち上がって出て行くのを見て、彼女は墓に泣きに行くのであろうと思い、そのあとからついて行った。
\par 32 マリヤは、イエスのおられる所に行ってお目にかかり、その足もとにひれ伏して言った、「主よ、もしあなたがここにいて下さったなら、わたしの兄弟は死ななかったでしょう」。
\par 33 イエスは、彼女が泣き、また、彼女と一緒にきたユダヤ人たちも泣いているのをごらんになり、激しく感動し、また心を騒がせ、そして言われた、
\par 34 「彼をどこに置いたのか」。彼らはイエスに言った、「主よ、きて、ごらん下さい」。
\par 35 イエスは涙を流された。
\par 36 するとユダヤ人たちは言った、「ああ、なんと彼を愛しておられたことか」。
\par 37 しかし、彼らのある人たちは言った、「あの盲人の目をあけたこの人でも、ラザロを死なせないようには、できなかったのか」。
\par 38 イエスはまた激しく感動して、墓にはいられた。それは洞穴であって、そこに石がはめてあった。
\par 39 イエスは言われた、「石を取りのけなさい」。死んだラザロの姉妹マルタが言った、「主よ、もう臭くなっております。四日もたっていますから」。
\par 40 イエスは彼女に言われた、「もし信じるなら神の栄光を見るであろうと、あなたに言ったではないか」。
\par 41 人々は石を取りのけた。すると、イエスは目を天にむけて言われた、「父よ、わたしの願いをお聞き下さったことを感謝します。
\par 42 あなたがいつでもわたしの願いを聞きいれて下さることを、よく知っています。しかし、こう申しますのは、そばに立っている人々に、あなたがわたしをつかわされたことを、信じさせるためであります」。
\par 43 こう言いながら、大声で「ラザロよ、出てきなさい」と呼ばわれた。
\par 44 すると、死人は手足を布でまかれ、顔も顔おおいで包まれたまま、出てきた。イエスは人々に言われた、「彼をほどいてやって、帰らせなさい」。
\par 45 マリヤのところにきて、イエスのなさったことを見た多くのユダヤ人たちは、イエスを信じた。
\par 46 しかし、そのうちの数人がパリサイ人たちのところに行って、イエスのされたことを告げた。
\par 47 そこで、祭司長たちとパリサイ人たちとは、議会を召集して言った、「この人が多くのしるしを行っているのに、お互は何をしているのだ。
\par 48 もしこのままにしておけば、みんなが彼を信じるようになるだろう。そのうえ、ローマ人がやってきて、わたしたちの土地も人民も奪ってしまうであろう」。
\par 49 彼らのうちのひとりで、その年の大祭司であったカヤパが、彼らに言った、「あなたがたは、何もわかっていないし、
\par 50 ひとりの人が人民に代って死んで、全国民が滅びないようになるのがわたしたちにとって得だということを、考えてもいない」。
\par 51 このことは彼が自分から言ったのではない。彼はこの年の大祭司であったので、預言をして、イエスが国民のために、
\par 52 ただ国民のためだけではなく、また散在している神の子らを一つに集めるために、死ぬことになっていると、言ったのである。
\par 53 彼らはこの日からイエスを殺そうと相談した。
\par 54 そのためイエスは、もはや公然とユダヤ人の間を歩かないで、そこを出て、荒野に近い地方のエフライムという町に行かれ、そこに弟子たちと一緒に滞在しておられた。
\par 55 さて、ユダヤ人の過越の祭が近づいたので、多くの人々は身をきよめるために、祭の前に、地方からエルサレムへ上った。
\par 56 人々はイエスを捜し求め、宮の庭に立って互に言った、「あなたがたはどう思うか。イエスはこの祭にこないのだろうか」。
\par 57 祭司長たちとパリサイ人たちとは、イエスを捕えようとして、そのいどころを知っている者があれば申し出よ、という指令を出していた。

\chapter{12}

\par 1 過越の祭の六日まえに、イエスはベタニヤに行かれた。そこは、イエスが死人の中からよみがえらせたラザロのいた所である。
\par 2 イエスのためにそこで夕食の用意がされ、マルタは給仕をしていた。イエスと一緒に食卓についていた者のうちに、ラザロも加わっていた。
\par 3 その時、マリヤは高価で純粋なナルドの香油一斤を持ってきて、イエスの足にぬり、自分の髪の毛でそれをふいた。すると、香油のかおりが家にいっぱいになった。
\par 4 弟子のひとりで、イエスを裏切ろうとしていたイスカリオテのユダが言った、
\par 5 「なぜこの香油を三百デナリに売って、貧しい人たちに、施さなかったのか」。
\par 6 彼がこう言ったのは、貧しい人たちに対する思いやりがあったからではなく、自分が盗人であり、財布を預かっていて、その中身をごまかしていたからであった。
\par 7 イエスは言われた、「この女のするままにさせておきなさい。わたしの葬りの日のために、それをとっておいたのだから。
\par 8 貧しい人たちはいつもあなたがたと共にいるが、わたしはいつも共にいるわけではない」。
\par 9 大ぜいのユダヤ人たちが、そこにイエスのおられるのを知って、押しよせてきた。それはイエスに会うためだけではなく、イエスが死人のなかから、よみがえらせたラザロを見るためでもあった。
\par 10 そこで祭司長たちは、ラザロも殺そうと相談した。
\par 11 それは、ラザロのことで、多くのユダヤ人が彼らを離れ去って、イエスを信じるに至ったからである。
\par 12 その翌日、祭にきていた大ぜいの群衆は、イエスがエルサレムにこられると聞いて、
\par 13 しゅろの枝を手にとり、迎えに出て行った。そして叫んだ、「ホサナ、主の御名によってきたる者に祝福あれ、イスラエルの王に」。
\par 14 イエスは、ろばの子を見つけて、その上に乗られた。それは
\par 15 「シオンの娘よ、恐れるな。見よ、あなたの王がろばの子に乗っておいでになる」と書いてあるとおりであった。
\par 16 弟子たちは初めにはこのことを悟らなかったが、イエスが栄光を受けられた時に、このことがイエスについて書かれてあり、またそのとおりに、人々がイエスに対してしたのだということを、思い起した。
\par 17 また、イエスがラザロを墓から呼び出して、死人の中からよみがえらせたとき、イエスと一緒にいた群衆が、そのあかしをした。
\par 18 群衆がイエスを迎えに出たのは、イエスがこのようなしるしを行われたことを、聞いていたからである。
\par 19 そこで、パリサイ人たちは互に言った、「何をしてもむだだった。世をあげて彼のあとを追って行ったではないか」。
\par 20 祭で礼拝するために上ってきた人々のうちに、数人のギリシヤ人がいた。
\par 21 彼らはガリラヤのベツサイダ出であるピリポのところにきて、「君よ、イエスにお目にかかりたいのですが」と言って頼んだ。
\par 22 ピリポはアンデレのところに行ってそのことを話し、アンデレとピリポは、イエスのもとに行って伝えた。
\par 23 すると、イエスは答えて言われた、「人の子が栄光を受ける時がきた。
\par 24 よくよくあなたがたに言っておく。一粒の麦が地に落ちて死ななければ、それはただ一粒のままである。しかし、もし死んだなら、豊かに実を結ぶようになる。
\par 25 自分の命を愛する者はそれを失い、この世で自分の命を憎む者は、それを保って永遠の命に至るであろう。
\par 26 もしわたしに仕えようとする人があれば、その人はわたしに従って来るがよい。そうすれば、わたしのおる所に、わたしに仕える者もまた、おるであろう。もしわたしに仕えようとする人があれば、その人を父は重んじて下さるであろう。
\par 27 今わたしは心が騒いでいる。わたしはなんと言おうか。父よ、この時からわたしをお救い下さい。しかし、わたしはこのために、この時に至ったのです。
\par 28 父よ、み名があがめられますように」。すると天から声があった、「わたしはすでに栄光をあらわした。そして、更にそれをあらわすであろう」。
\par 29 すると、そこに立っていた群衆がこれを聞いて、「雷がなったのだ」と言い、ほかの人たちは、「御使が彼に話しかけたのだ」と言った。
\par 30 イエスは答えて言われた、「この声があったのは、わたしのためではなく、あなたがたのためである。
\par 31 今はこの世がさばかれる時である。今こそこの世の君は追い出されるであろう。
\par 32 そして、わたしがこの地から上げられる時には、すべての人をわたしのところに引きよせるであろう」。
\par 33 イエスはこう言って、自分がどんな死に方で死のうとしていたかを、お示しになったのである。
\par 34 すると群衆はイエスにむかって言った、「わたしたちは律法によって、キリストはいつまでも生きておいでになるのだ、と聞いていました。それだのに、どうして人の子は上げられねばならないと、言われるのですか。その人の子とは、だれのことですか」。
\par 35 そこでイエスは彼らに言われた、「もうしばらくの間、光はあなたがたと一緒にここにある。光がある間に歩いて、やみに追いつかれないようにしなさい。やみの中を歩く者は、自分がどこへ行くのかわかっていない。
\par 36 光のある間に、光の子となるために、光を信じなさい」。イエスはこれらのことを話してから、そこを立ち去って、彼らから身をお隠しになった。
\par 37 このように多くのしるしを彼らの前でなさったが、彼らはイエスを信じなかった。
\par 38 それは、預言者イザヤの次の言葉が成就するためである、「主よ、わたしたちの説くところを、だれが信じたでしょうか。また、主のみ腕はだれに示されたでしょうか」。
\par 39 こういうわけで、彼らは信じることができなかった。イザヤはまた、こうも言った、
\par 40 「神は彼らの目をくらまし、心をかたくなになさった。それは、彼らが目で見ず、心で悟らず、悔い改めていやされることがないためである」。
\par 41 イザヤがこう言ったのは、イエスの栄光を見たからであって、イエスのことを語ったのである。
\par 42 しかし、役人たちの中にも、イエスを信じた者が多かったが、パリサイ人をはばかって、告白はしなかった。会堂から追い出されるのを恐れていたのである。
\par 43 彼らは神のほまれよりも、人のほまれを好んだからである。
\par 44 イエスは大声で言われた、「わたしを信じる者は、わたしを信じるのではなく、わたしをつかわされたかたを信じるのであり、
\par 45 また、わたしを見る者は、わたしをつかわされたかたを見るのである。
\par 46 わたしは光としてこの世にきた。それは、わたしを信じる者が、やみのうちにとどまらないようになるためである。
\par 47 たとい、わたしの言うことを聞いてそれを守らない人があっても、わたしはその人をさばかない。わたしがきたのは、この世をさばくためではなく、この世を救うためである。
\par 48 わたしを捨てて、わたしの言葉を受けいれない人には、その人をさばくものがある。わたしの語ったその言葉が、終りの日にその人をさばくであろう。
\par 49 わたしは自分から語ったのではなく、わたしをつかわされた父ご自身が、わたしの言うべきこと、語るべきことをお命じになったのである。
\par 50 わたしは、この命令が永遠の命であることを知っている。それゆえに、わたしが語っていることは、わたしの父がわたしに仰せになったことを、そのまま語っているのである」。

\chapter{13}

\par 1 過越の祭の前に、イエスは、この世を去って父のみもとに行くべき自分の時がきたことを知り、世にいる自分の者たちを愛して、彼らを最後まで愛し通された。
\par 2 夕食のとき、悪魔はすでにシモンの子イスカリオテのユダの心に、イエスを裏切ろうとする思いを入れていたが、
\par 3 イエスは、父がすべてのものを自分の手にお与えになったこと、また、自分は神から出てきて、神にかえろうとしていることを思い、
\par 4 夕食の席から立ち上がって、上着を脱ぎ、手ぬぐいをとって腰に巻き、
\par 5 それから水をたらいに入れて、弟子たちの足を洗い、腰に巻いた手ぬぐいでふき始められた。
\par 6 こうして、シモン・ペテロの番になった。すると彼はイエスに、「主よ、あなたがわたしの足をお洗いになるのですか」と言った。
\par 7 イエスは彼に答えて言われた、「わたしのしていることは今あなたにはわからないが、あとでわかるようになるだろう」。
\par 8 ペテロはイエスに言った、「わたしの足を決して洗わないで下さい」。イエスは彼に答えられた、「もしわたしがあなたの足を洗わないなら、あなたはわたしとなんの係わりもなくなる」。
\par 9 シモン・ペテロはイエスに言った、「主よ、では、足だけではなく、どうぞ、手も頭も」。
\par 10 イエスは彼に言われた、「すでにからだを洗った者は、足のほかは洗う必要がない。全身がきれいなのだから。あなたがたはきれいなのだ。しかし、みんながそうなのではない」。
\par 11 イエスは自分を裏切る者を知っておられた。それで、「みんながきれいなのではない」と言われたのである。
\par 12 こうして彼らの足を洗ってから、上着をつけ、ふたたび席にもどって、彼らに言われた、「わたしがあなたがたにしたことがわかるか。
\par 13 あなたがたはわたしを教師、また主と呼んでいる。そう言うのは正しい。わたしはそのとおりである。
\par 14 しかし、主であり、また教師であるわたしが、あなたがたの足を洗ったからには、あなたがたもまた、互に足を洗い合うべきである。
\par 15 わたしがあなたがたにしたとおりに、あなたがたもするように、わたしは手本を示したのだ。
\par 16 よくよくあなたがたに言っておく。僕はその主人にまさるものではなく、つかわされた者はつかわした者にまさるものではない。
\par 17 もしこれらのことがわかっていて、それを行うなら、あなたがたはさいわいである。
\par 18 あなたがた全部の者について、こう言っているのではない。わたしは自分が選んだ人たちを知っている。しかし、『わたしのパンを食べている者が、わたしにむかってそのかかとをあげた』とある聖書は成就されなければならない。
\par 19 そのことがまだ起らない今のうちに、あなたがたに言っておく。いよいよ事が起ったとき、わたしがそれであることを、あなたがたが信じるためである。
\par 20 よくよくあなたがたに言っておく。わたしがつかわす者を受けいれる者は、わたしを受けいれるのである。わたしを受けいれる者は、わたしをつかわされたかたを、受けいれるのである」。
\par 21 イエスがこれらのことを言われた後、その心が騒ぎ、おごそかに言われた、「よくよくあなたがたに言っておく。あなたがたのうちのひとりが、わたしを裏切ろうとしている」。
\par 22 弟子たちはだれのことを言われたのか察しかねて、互に顔を見合わせた。
\par 23 弟子たちのひとりで、イエスの愛しておられた者が、み胸に近く席についていた。
\par 24 そこで、シモン・ペテロは彼に合図をして言った、「だれのことをおっしゃったのか、知らせてくれ」。
\par 25 その弟子はそのままイエスの胸によりかかって、「主よ、だれのことですか」と尋ねると、
\par 26 イエスは答えられた、「わたしが一きれの食物をひたして与える者が、それである」。そして、一きれの食物をひたしてとり上げ、シモンの子イスカリオテのユダにお与えになった。
\par 27 この一きれの食物を受けるやいなや、サタンがユダにはいった。そこでイエスは彼に言われた、「しようとしていることを、今すぐするがよい」。
\par 28 席を共にしていた者のうち、なぜユダにこう言われたのか、わかっていた者はひとりもなかった。
\par 29 ある人々は、ユダが金入れをあずかっていたので、イエスが彼に、「祭のために必要なものを買え」と言われたか、あるいは、貧しい者に何か施させようとされたのだと思っていた。
\par 30 ユダは一きれの食物を受けると、すぐに出て行った。時は夜であった。
\par 31 さて、彼が出て行くと、イエスは言われた、「今や人の子は栄光を受けた。神もまた彼によって栄光をお受けになった。
\par 32 彼によって栄光をお受けになったのなら、神ご自身も彼に栄光をお授けになるであろう。すぐにもお授けになるであろう。
\par 33 子たちよ、わたしはまだしばらく、あなたがたと一緒にいる。あなたがたはわたしを捜すだろうが、すでにユダヤ人たちに言ったとおり、今あなたがたにも言う、『あなたがたはわたしの行く所に来ることはできない』。
\par 34 わたしは、新しいいましめをあなたがたに与える、互に愛し合いなさい。わたしがあなたがたを愛したように、あなたがたも互に愛し合いなさい。
\par 35 互に愛し合うならば、それによって、あなたがたがわたしの弟子であることを、すべての者が認めるであろう」。
\par 36 シモン・ペテロがイエスに言った、「主よ、どこへおいでになるのですか」。イエスは答えられた、「あなたはわたしの行くところに、今はついて来ることはできない。しかし、あとになってから、ついて来ることになろう」。
\par 37 ペテロはイエスに言った、「主よ、なぜ、今あなたについて行くことができないのですか。あなたのためには、命も捨てます」。
\par 38 イエスは答えられた、「わたしのために命を捨てると言うのか。よくよくあなたに言っておく。鶏が鳴く前に、あなたはわたしを三度知らないと言うであろう」。

\chapter{14}

\par 1 「あなたがたは、心を騒がせないがよい。神を信じ、またわたしを信じなさい。
\par 2 わたしの父の家には、すまいがたくさんある。もしなかったならば、わたしはそう言っておいたであろう。あなたがたのために、場所を用意しに行くのだから。
\par 3 そして、行って、場所の用意ができたならば、またきて、あなたがたをわたしのところに迎えよう。わたしのおる所にあなたがたもおらせるためである。
\par 4 わたしがどこへ行くのか、その道はあなたがたにわかっている」。
\par 5 トマスはイエスに言った、「主よ、どこへおいでになるのか、わたしたちにはわかりません。どうしてその道がわかるでしょう」。
\par 6 イエスは彼に言われた、「わたしは道であり、真理であり、命である。だれでもわたしによらないでは、父のみもとに行くことはできない。
\par 7 もしあなたがたがわたしを知っていたならば、わたしの父をも知ったであろう。しかし、今は父を知っており、またすでに父を見たのである」。
\par 8 ピリポはイエスに言った、「主よ、わたしたちに父を示して下さい。そうして下されば、わたしたちは満足します」。
\par 9 イエスは彼に言われた、「ピリポよ、こんなに長くあなたがたと一緒にいるのに、わたしがわかっていないのか。わたしを見た者は、父を見たのである。どうして、わたしたちに父を示してほしいと、言うのか。
\par 10 わたしが父におり、父がわたしにおられることをあなたは信じないのか。わたしがあなたがたに話している言葉は、自分から話しているのではない。父がわたしのうちにおられて、みわざをなさっているのである。
\par 11 わたしが父におり、父がわたしにおられることを信じなさい。もしそれが信じられないならば、わざそのものによって信じなさい。
\par 12 よくよくあなたがたに言っておく。わたしを信じる者は、またわたしのしているわざをするであろう。そればかりか、もっと大きいわざをするであろう。わたしが父のみもとに行くからである。
\par 13 わたしの名によって願うことは、なんでもかなえてあげよう。父が子によって栄光をお受けになるためである。
\par 14 何事でもわたしの名によって願うならば、わたしはそれをかなえてあげよう。
\par 15 もしあなたがたがわたしを愛するならば、わたしのいましめを守るべきである。
\par 16 わたしは父にお願いしよう。そうすれば、父は別に助け主を送って、いつまでもあなたがたと共におらせて下さるであろう。
\par 17 それは真理の御霊である。この世はそれを見ようともせず、知ろうともしないので、それを受けることができない。あなたがたはそれを知っている。なぜなら、それはあなたがたと共におり、またあなたがたのうちにいるからである。
\par 18 わたしはあなたがたを捨てて孤児とはしない。あなたがたのところに帰って来る。
\par 19 もうしばらくしたら、世はもはやわたしを見なくなるだろう。しかし、あなたがたはわたしを見る。わたしが生きるので、あなたがたも生きるからである。
\par 20 その日には、わたしはわたしの父におり、あなたがたはわたしにおり、また、わたしがあなたがたにおることが、わかるであろう。
\par 21 わたしのいましめを心にいだいてこれを守る者は、わたしを愛する者である。わたしを愛する者は、わたしの父に愛されるであろう。わたしもその人を愛し、その人にわたし自身をあらわすであろう」。
\par 22 イスカリオテでない方のユダがイエスに言った、「主よ、あなたご自身をわたしたちにあらわそうとして、世にはあらわそうとされないのはなぜですか」。
\par 23 イエスは彼に答えて言われた、「もしだれでもわたしを愛するならば、わたしの言葉を守るであろう。そして、わたしの父はその人を愛し、また、わたしたちはその人のところに行って、その人と一緒に住むであろう。
\par 24 わたしを愛さない者はわたしの言葉を守らない。あなたがたが聞いている言葉は、わたしの言葉ではなく、わたしをつかわされた父の言葉である。
\par 25 これらのことは、あなたがたと一緒にいた時、すでに語ったことである。
\par 26 しかし、助け主、すなわち、父がわたしの名によってつかわされる聖霊は、あなたがたにすべてのことを教え、またわたしが話しておいたことを、ことごとく思い起させるであろう。
\par 27 わたしは平安をあなたがたに残して行く。わたしの平安をあなたがたに与える。わたしが与えるのは、世が与えるようなものとは異なる。あなたがたは心を騒がせるな、またおじけるな。
\par 28 『わたしは去って行くが、またあなたがたのところに帰って来る』と、わたしが言ったのを、あなたがたは聞いている。もしわたしを愛しているなら、わたしが父のもとに行くのを喜んでくれるであろう。父がわたしより大きいかたであるからである。
\par 29 今わたしは、そのことが起らない先にあなたがたに語った。それは、事が起った時にあなたがたが信じるためである。
\par 30 わたしはもはや、あなたがたに、多くを語るまい。この世の君が来るからである。だが、彼はわたしに対して、なんの力もない。
\par 31 しかし、わたしが父を愛していることを世が知るように、わたしは父がお命じになったとおりのことを行うのである。立て。さあ、ここから出かけて行こう。

\chapter{15}

\par 1 わたしはまことのぶどうの木、わたしの父は農夫である。
\par 2 わたしにつながっている枝で実を結ばないものは、父がすべてこれをとりのぞき、実を結ぶものは、もっと豊かに実らせるために、手入れしてこれをきれいになさるのである。
\par 3 あなたがたは、わたしが語った言葉によって既にきよくされている。
\par 4 わたしにつながっていなさい。そうすれば、わたしはあなたがたとつながっていよう。枝がぶどうの木につながっていなければ、自分だけでは実を結ぶことができないように、あなたがたもわたしにつながっていなければ実を結ぶことができない。
\par 5 わたしはぶどうの木、あなたがたはその枝である。もし人がわたしにつながっており、またわたしがその人とつながっておれば、その人は実を豊かに結ぶようになる。わたしから離れては、あなたがたは何一つできないからである。
\par 6 人がわたしにつながっていないならば、枝のように外に投げすてられて枯れる。人々はそれをかき集め、火に投げ入れて、焼いてしまうのである。
\par 7 あなたがたがわたしにつながっており、わたしの言葉があなたがたにとどまっているならば、なんでも望むものを求めるがよい。そうすれば、与えられるであろう。
\par 8 あなたがたが実を豊かに結び、そしてわたしの弟子となるならば、それによって、わたしの父は栄光をお受けになるであろう。
\par 9 父がわたしを愛されたように、わたしもあなたがたを愛したのである。わたしの愛のうちにいなさい。
\par 10 もしわたしのいましめを守るならば、あなたがたはわたしの愛のうちにおるのである。それはわたしがわたしの父のいましめを守ったので、その愛のうちにおるのと同じである。
\par 11 わたしがこれらのことを話したのは、わたしの喜びがあなたがたのうちにも宿るため、また、あなたがたの喜びが満ちあふれるためである。
\par 12 わたしのいましめは、これである。わたしがあなたがたを愛したように、あなたがたも互に愛し合いなさい。
\par 13 人がその友のために自分の命を捨てること、これよりも大きな愛はない。
\par 14 あなたがたにわたしが命じることを行うならば、あなたがたはわたしの友である。
\par 15 わたしはもう、あなたがたを僕とは呼ばない。僕は主人のしていることを知らないからである。わたしはあなたがたを友と呼んだ。わたしの父から聞いたことを皆、あなたがたに知らせたからである。
\par 16 あなたがたがわたしを選んだのではない。わたしがあなたがたを選んだのである。そして、あなたがたを立てた。それは、あなたがたが行って実をむすび、その実がいつまでも残るためであり、また、あなたがたがわたしの名によって父に求めるものはなんでも、父が与えて下さるためである。
\par 17 これらのことを命じるのは、あなたがたが互に愛し合うためである。
\par 18 もしこの世があなたがたを憎むならば、あなたがたよりも先にわたしを憎んだことを、知っておくがよい。
\par 19 もしあなたがたがこの世から出たものであったなら、この世は、あなたがたを自分のものとして愛したであろう。しかし、あなたがたはこの世のものではない。かえって、わたしがあなたがたをこの世から選び出したのである。だから、この世はあなたがたを憎むのである。
\par 20 わたしがあなたがたに『僕はその主人にまさるものではない』と言ったことを、おぼえていなさい。もし人々がわたしを迫害したなら、あなたがたをも迫害するであろう。また、もし彼らがわたしの言葉を守っていたなら、あなたがたの言葉をも守るであろう。
\par 21 彼らはわたしの名のゆえに、あなたがたに対してすべてそれらのことをするであろう。それは、わたしをつかわされたかたを彼らが知らないからである。
\par 22 もしわたしがきて彼らに語らなかったならば、彼らは罪を犯さないですんだであろう。しかし今となっては、彼らには、その罪について言いのがれる道がない。
\par 23 わたしを憎む者は、わたしの父をも憎む。
\par 24 もし、ほかのだれもがしなかったようなわざを、わたしが彼らの間でしなかったならば、彼らは罪を犯さないですんだであろう。しかし事実、彼らはわたしとわたしの父とを見て、憎んだのである。
\par 25 それは、『彼らは理由なしにわたしを憎んだ』と書いてある彼らの律法の言葉が成就するためである。
\par 26 わたしが父のみもとからあなたがたにつかわそうとしている助け主、すなわち、父のみもとから来る真理の御霊が下る時、それはわたしについてあかしをするであろう。
\par 27 あなたがたも、初めからわたしと一緒にいたのであるから、あかしをするのである。

\chapter{16}

\par 1 わたしがこれらのことを語ったのは、あなたがたがつまずくことのないためである。
\par 2 人々はあなたがたを会堂から追い出すであろう。更にあなたがたを殺す者がみな、それによって自分たちは神に仕えているのだと思う時が来るであろう。
\par 3 彼らがそのようなことをするのは、父をもわたしをも知らないからである。
\par 4 わたしがあなたがたにこれらのことを言ったのは、彼らの時がきた場合、わたしが彼らについて言ったことを、思い起させるためである。これらのことを初めから言わなかったのは、わたしがあなたがたと一緒にいたからである。
\par 5 けれども今わたしは、わたしをつかわされたかたのところに行こうとしている。しかし、あなたがたのうち、だれも『どこへ行くのか』と尋ねる者はない。
\par 6 かえって、わたしがこれらのことを言ったために、あなたがたの心は憂いで満たされている。
\par 7 しかし、わたしはほんとうのことをあなたがたに言うが、わたしが去って行くことは、あなたがたの益になるのだ。わたしが去って行かなければ、あなたがたのところに助け主はこないであろう。もし行けば、それをあなたがたにつかわそう。
\par 8 それがきたら、罪と義とさばきとについて、世の人の目を開くであろう。
\par 9 罪についてと言ったのは、彼らがわたしを信じないからである。
\par 10 義についてと言ったのは、わたしが父のみもとに行き、あなたがたは、もはやわたしを見なくなるからである。
\par 11 さばきについてと言ったのは、この世の君がさばかれるからである。
\par 12 わたしには、あなたがたに言うべきことがまだ多くあるが、あなたがたは今はそれに堪えられない。
\par 13 けれども真理の御霊が来る時には、あなたがたをあらゆる真理に導いてくれるであろう。それは自分から語るのではなく、その聞くところを語り、きたるべき事をあなたがたに知らせるであろう。
\par 14 御霊はわたしに栄光を得させるであろう。わたしのものを受けて、それをあなたがたに知らせるからである。
\par 15 父がお持ちになっているものはみな、わたしのものである。御霊はわたしのものを受けて、それをあなたがたに知らせるのだと、わたしが言ったのは、そのためである。
\par 16 しばらくすれば、あなたがたはもうわたしを見なくなる。しかし、またしばらくすれば、わたしに会えるであろう」。
\par 17 そこで、弟子たちのうちのある者は互に言い合った、「『しばらくすれば、わたしを見なくなる。またしばらくすれば、わたしに会えるであろう』と言われ、『わたしの父のところに行く』と言われたのは、いったい、どういうことなのであろう」。
\par 18 彼らはまた言った、「『しばらくすれば』と言われるのは、どういうことか。わたしたちには、その言葉の意味がわからない」。
\par 19 イエスは、彼らが尋ねたがっていることに気がついて、彼らに言われた、「しばらくすればわたしを見なくなる、またしばらくすればわたしに会えるであろうと、わたしが言ったことで、互に論じ合っているのか。
\par 20 よくよくあなたがたに言っておく。あなたがたは泣き悲しむが、この世は喜ぶであろう。あなたがたは憂えているが、その憂いは喜びに変るであろう。
\par 21 女が子を産む場合には、その時がきたというので、不安を感じる。しかし、子を産んでしまえば、もはやその苦しみをおぼえてはいない。ひとりの人がこの世に生れた、という喜びがあるためである。
\par 22 このように、あなたがたにも今は不安がある。しかし、わたしは再びあなたがたと会うであろう。そして、あなたがたの心は喜びに満たされるであろう。その喜びをあなたがたから取り去る者はいない。
\par 23 その日には、あなたがたがわたしに問うことは、何もないであろう。よくよくあなたがたに言っておく。あなたがたが父に求めるものはなんでも、わたしの名によって下さるであろう。
\par 24 今までは、あなたがたはわたしの名によって求めたことはなかった。求めなさい、そうすれば、与えられるであろう。そして、あなたがたの喜びが満ちあふれるであろう。
\par 25 わたしはこれらのことを比喩で話したが、もはや比喩では話さないで、あからさまに、父のことをあなたがたに話してきかせる時が来るであろう。
\par 26 その日には、あなたがたは、わたしの名によって求めるであろう。わたしは、あなたがたのために父に願ってあげようとは言うまい。
\par 27 父ご自身があなたがたを愛しておいでになるからである。それは、あなたがたがわたしを愛したため、また、わたしが神のみもとからきたことを信じたためである。
\par 28 わたしは父から出てこの世にきたが、またこの世を去って、父のみもとに行くのである」。
\par 29 弟子たちは言った、「今はあからさまにお話しになって、少しも比喩ではお話しになりません。
\par 30 あなたはすべてのことをご存じであり、だれもあなたにお尋ねする必要のないことが、今わかりました。このことによって、わたしたちはあなたが神からこられたかたであると信じます」。
\par 31 イエスは答えられた、「あなたがたは今信じているのか。
\par 32 見よ、あなたがたは散らされて、それぞれ自分の家に帰り、わたしをひとりだけ残す時が来るであろう。いや、すでにきている。しかし、わたしはひとりでいるのではない。父がわたしと一緒におられるのである。
\par 33 これらのことをあなたがたに話したのは、わたしにあって平安を得るためである。あなたがたは、この世ではなやみがある。しかし、勇気を出しなさい。わたしはすでに世に勝っている」。

\chapter{17}

\par 1 これらのことを語り終えると、イエスは天を見あげて言われた、「父よ、時がきました。あなたの子があなたの栄光をあらわすように、子の栄光をあらわして下さい。
\par 2 あなたは、子に賜わったすべての者に、永遠の命を授けさせるため、万民を支配する権威を子にお与えになったのですから。
\par 3 永遠の命とは、唯一の、まことの神でいますあなたと、また、あなたがつかわされたイエス・キリストとを知ることであります。
\par 4 わたしは、わたしにさせるためにお授けになったわざをなし遂げて、地上であなたの栄光をあらわしました。
\par 5 父よ、世が造られる前に、わたしがみそばで持っていた栄光で、今み前にわたしを輝かせて下さい。
\par 6 わたしは、あなたが世から選んでわたしに賜わった人々に、み名をあらわしました。彼らはあなたのものでありましたが、わたしに下さいました。そして、彼らはあなたの言葉を守りました。
\par 7 いま彼らは、わたしに賜わったものはすべて、あなたから出たものであることを知りました。
\par 8 なぜなら、わたしはあなたからいただいた言葉を彼らに与え、そして彼らはそれを受け、わたしがあなたから出たものであることをほんとうに知り、また、あなたがわたしをつかわされたことを信じるに至ったからです。
\par 9 わたしは彼らのためにお願いします。わたしがお願いするのは、この世のためにではなく、あなたがわたしに賜わった者たちのためです。彼らはあなたのものなのです。
\par 10 わたしのものは皆あなたのもの、あなたのものはわたしのものです。そして、わたしは彼らによって栄光を受けました。
\par 11 わたしはもうこの世にはいなくなりますが、彼らはこの世に残っており、わたしはみもとに参ります。聖なる父よ、わたしに賜わった御名によって彼らを守って下さい。それはわたしたちが一つであるように、彼らも一つになるためであります。
\par 12 わたしが彼らと一緒にいた間は、あなたからいただいた御名によって彼らを守り、また保護してまいりました。彼らのうち、だれも滅びず、ただ滅びの子だけが滅びました。それは聖書が成就するためでした。
\par 13 今わたしはみもとに参ります。そして世にいる間にこれらのことを語るのは、わたしの喜びが彼らのうちに満ちあふれるためであります。
\par 14 わたしは彼らに御言を与えましたが、世は彼らを憎みました。わたしが世のものでないように、彼らも世のものではないからです。
\par 15 わたしがお願いするのは、彼らを世から取り去ることではなく、彼らを悪しき者から守って下さることであります。
\par 16 わたしが世のものでないように、彼らも世のものではありません。
\par 17 真理によって彼らを聖別して下さい。あなたの御言は真理であります。
\par 18 あなたがわたしを世につかわされたように、わたしも彼らを世につかわしました。
\par 19 また彼らが真理によって聖別されるように、彼らのためわたし自身を聖別いたします。
\par 20 わたしは彼らのためばかりではなく、彼らの言葉を聞いてわたしを信じている人々のためにも、お願いいたします。
\par 21 父よ、それは、あなたがわたしのうちにおられ、わたしがあなたのうちにいるように、みんなの者が一つとなるためであります。すなわち、彼らをもわたしたちのうちにおらせるためであり、それによって、あなたがわたしをおつかわしになったことを、世が信じるようになるためであります。
\par 22 わたしは、あなたからいただいた栄光を彼らにも与えました。それは、わたしたちが一つであるように、彼らも一つになるためであります。
\par 23 わたしが彼らにおり、あなたがわたしにいますのは、彼らが完全に一つとなるためであり、また、あなたがわたしをつかわし、わたしを愛されたように、彼らをお愛しになったことを、世が知るためであります。
\par 24 父よ、あなたがわたしに賜わった人々が、わたしのいる所に一緒にいるようにして下さい。天地が造られる前からわたしを愛して下さって、わたしに賜わった栄光を、彼らに見させて下さい。
\par 25 正しい父よ、この世はあなたを知っていません。しかし、わたしはあなたを知り、また彼らも、あなたがわたしをおつかわしになったことを知っています。
\par 26 そしてわたしは彼らに御名を知らせました。またこれからも知らせましょう。それは、あなたがわたしを愛して下さったその愛が彼らのうちにあり、またわたしも彼らのうちにおるためであります」。

\chapter{18}

\par 1 イエスはこれらのことを語り終えて、弟子たちと一緒にケデロンの谷の向こうへ行かれた。そこには園があって、イエスは弟子たちと一緒にその中にはいられた。
\par 2 イエスを裏切ったユダは、その所をよく知っていた。イエスと弟子たちとがたびたびそこで集まったことがあるからである。
\par 3 さてユダは、一隊の兵卒と祭司長やパリサイ人たちの送った下役どもを引き連れ、たいまつやあかりや武器を持って、そこへやってきた。
\par 4 しかしイエスは、自分の身に起ろうとすることをことごとく承知しておられ、進み出て彼らに言われた、「だれを捜しているのか」。
\par 5 彼らは「ナザレのイエスを」と答えた。イエスは彼らに言われた、「わたしが、それである」。イエスを裏切ったユダも、彼らと一緒に立っていた。
\par 6 イエスが彼らに「わたしが、それである」と言われたとき、彼らはうしろに引きさがって地に倒れた。
\par 7 そこでまた彼らに、「だれを捜しているのか」とお尋ねになると、彼らは「ナザレのイエスを」と言った。
\par 8 イエスは答えられた、「わたしがそれであると、言ったではないか。わたしを捜しているのなら、この人たちを去らせてもらいたい」。
\par 9 それは、「あなたが与えて下さった人たちの中のひとりも、わたしは失わなかった」とイエスの言われた言葉が、成就するためである。
\par 10 シモン・ペテロは剣を持っていたが、それを抜いて、大祭司の僕に切りかかり、その右の耳を切り落した。その僕の名はマルコスであった。
\par 11 すると、イエスはペテロに言われた、「剣をさやに納めなさい。父がわたしに下さった杯は、飲むべきではないか」。
\par 12 それから一隊の兵卒やその千卒長やユダヤ人の下役どもが、イエスを捕え、縛りあげて、
\par 13 まずアンナスのところに引き連れて行った。彼はその年の大祭司カヤパのしゅうとであった。
\par 14 カヤパは前に、ひとりの人が民のために死ぬのはよいことだと、ユダヤ人に助言した者であった。
\par 15 シモン・ペテロともうひとりの弟子とが、イエスについて行った。この弟子は大祭司の知り合いであったので、イエスと一緒に大祭司の中庭にはいった。
\par 16 しかし、ペテロは外で戸口に立っていた。すると大祭司の知り合いであるその弟子が、外に出て行って門番の女に話し、ペテロを内に入れてやった。
\par 17 すると、この門番の女がペテロに言った、「あなたも、あの人の弟子のひとりではありませんか」。ペテロは「いや、そうではない」と答えた。
\par 18 僕や下役どもは、寒い時であったので、炭火をおこし、そこに立ってあたっていた。ペテロもまた彼らに交じり、立ってあたっていた。
\par 19 大祭司はイエスに、弟子たちのことやイエスの教のことを尋ねた。
\par 20 イエスは答えられた、「わたしはこの世に対して公然と語ってきた。すべてのユダヤ人が集まる会堂や宮で、いつも教えていた。何事も隠れて語ったことはない。
\par 21 なぜ、わたしに尋ねるのか。わたしが彼らに語ったことは、それを聞いた人々に尋ねるがよい。わたしの言ったことは、彼らが知っているのだから」。
\par 22 イエスがこう言われると、そこに立っていた下役のひとりが、「大祭司にむかって、そのような答をするのか」と言って、平手でイエスを打った。
\par 23 イエスは答えられた、「もしわたしが何か悪いことを言ったのなら、その悪い理由を言いなさい。しかし、正しいことを言ったのなら、なぜわたしを打つのか」。
\par 24 それからアンナスは、イエスを縛ったまま大祭司カヤパのところへ送った。
\par 25 シモン・ペテロは、立って火にあたっていた。すると人々が彼に言った、「あなたも、あの人の弟子のひとりではないか」。彼はそれをうち消して、「いや、そうではない」と言った。
\par 26 大祭司の僕のひとりで、ペテロに耳を切りおとされた人の親族の者が言った、「あなたが園であの人と一緒にいるのを、わたしは見たではないか」。
\par 27 ペテロはまたそれを打ち消した。するとすぐに、鶏が鳴いた。
\par 28 それから人々は、イエスをカヤパのところから官邸につれて行った。時は夜明けであった。彼らは、けがれを受けないで過越の食事ができるように、官邸にはいらなかった。
\par 29 そこで、ピラトは彼らのところに出てきて言った、「あなたがたは、この人に対してどんな訴えを起すのか」。
\par 30 彼らはピラトに答えて言った、「もしこの人が悪事をはたらかなかったなら、あなたに引き渡すようなことはしなかったでしょう」。
\par 31 そこでピラトは彼らに言った、「あなたがたは彼を引き取って、自分たちの律法でさばくがよい」。ユダヤ人らは彼に言った、「わたしたちには、人を死刑にする権限がありません」。
\par 32 これは、ご自身がどんな死にかたをしようとしているかを示すために言われたイエスの言葉が、成就するためである。
\par 33 さて、ピラトはまた官邸にはいり、イエスを呼び出して言った、「あなたは、ユダヤ人の王であるか」。
\par 34 イエスは答えられた、「あなたがそう言うのは、自分の考えからか。それともほかの人々が、わたしのことをあなたにそう言ったのか」。
\par 35 ピラトは答えた、「わたしはユダヤ人なのか。あなたの同族や祭司長たちが、あなたをわたしに引き渡したのだ。あなたは、いったい、何をしたのか」。
\par 36 イエスは答えられた、「わたしの国はこの世のものではない。もしわたしの国がこの世のものであれば、わたしに従っている者たちは、わたしをユダヤ人に渡さないように戦ったであろう。しかし事実、わたしの国はこの世のものではない」。
\par 37 そこでピラトはイエスに言った、「それでは、あなたは王なのだな」。イエスは答えられた、「あなたの言うとおり、わたしは王である。わたしは真理についてあかしをするために生れ、また、そのためにこの世にきたのである。だれでも真理につく者は、わたしの声に耳を傾ける」。
\par 38 ピラトはイエスに言った、「真理とは何か」。こう言って、彼はまたユダヤ人の所に出て行き、彼らに言った、「わたしには、この人になんの罪も見いだせない。
\par 39 過越の時には、わたしがあなたがたのために、ひとりの人を許してやるのが、あなたがたのしきたりになっている。ついては、あなたがたは、このユダヤ人の王を許してもらいたいのか」。
\par 40 すると彼らは、また叫んで「その人ではなく、バラバを」と言った。このバラバは強盗であった。

\chapter{19}

\par 1 そこでピラトは、イエスを捕え、むちで打たせた。
\par 2 兵卒たちは、いばらで冠をあんで、イエスの頭にかぶらせ、紫の上着を着せ、
\par 3 それから、その前に進み出て、「ユダヤ人の王、ばんざい」と言った。そして平手でイエスを打ちつづけた。
\par 4 するとピラトは、また出て行ってユダヤ人たちに言った、「見よ、わたしはこの人をあなたがたの前に引き出すが、それはこの人になんの罪も見いだせないことを、あなたがたに知ってもらうためである」。
\par 5 イエスはいばらの冠をかぶり、紫の上着を着たままで外へ出られると、ピラトは彼らに言った、「見よ、この人だ」。
\par 6 祭司長たちや下役どもはイエスを見ると、叫んで「十字架につけよ、十字架につけよ」と言った。ピラトは彼らに言った、「あなたがたが、この人を引き取って十字架につけるがよい。わたしは、彼にはなんの罪も見いだせない」。
\par 7 ユダヤ人たちは彼に答えた、「わたしたちには律法があります。その律法によれば、彼は自分を神の子としたのだから、死罪に当る者です」。
\par 8 ピラトがこの言葉を聞いたとき、ますますおそれ、
\par 9 もう一度官邸にはいってイエスに言った、「あなたは、もともと、どこからきたのか」。しかし、イエスはなんの答もなさらなかった。
\par 10 そこでピラトは言った、「何も答えないのか。わたしには、あなたを許す権威があり、また十字架につける権威があることを、知らないのか」。
\par 11 イエスは答えられた、「あなたは、上から賜わるのでなければ、わたしに対してなんの権威もない。だから、わたしをあなたに引き渡した者の罪は、もっと大きい」。
\par 12 これを聞いて、ピラトはイエスを許そうと努めた。しかしユダヤ人たちが叫んで言った、「もしこの人を許したなら、あなたはカイザルの味方ではありません。自分を王とするものはすべて、カイザルにそむく者です」。
\par 13 ピラトはこれらの言葉を聞いて、イエスを外へ引き出して行き、敷石(ヘブル語ではガバタ)という場所で裁判の席についた。
\par 14 その日は過越の準備の日であって、時は昼の十二時ころであった。ピラトはユダヤ人らに言った、「見よ、これがあなたがたの王だ」。
\par 15 すると彼らは叫んだ、「殺せ、殺せ、彼を十字架につけよ」。ピラトは彼らに言った、「あなたがたの王を、わたしが十字架につけるのか」。祭司長たちは答えた、「わたしたちには、カイザル以外に王はありません」。
\par 16 そこでピラトは、十字架につけさせるために、イエスを彼らに引き渡した。彼らはイエスを引き取った。
\par 17 イエスはみずから十字架を背負って、されこうべ(ヘブル語ではゴルゴダ)という場所に出て行かれた。
\par 18 彼らはそこで、イエスを十字架につけた。イエスをまん中にして、ほかのふたりの者を両側に、イエスと一緒に十字架につけた。
\par 19 ピラトは罪状書きを書いて、十字架の上にかけさせた。それには「ユダヤ人の王、ナザレのイエス」と書いてあった。
\par 20 イエスが十字架につけられた場所は都に近かったので、多くのユダヤ人がこの罪状書きを読んだ。それはヘブル、ローマ、ギリシヤの国語で書いてあった。
\par 21 ユダヤ人の祭司長たちがピラトに言った、「『ユダヤ人の王』と書かずに、『この人はユダヤ人の王と自称していた』と書いてほしい」。
\par 22 ピラトは答えた、「わたしが書いたことは、書いたままにしておけ」。
\par 23 さて、兵卒たちはイエスを十字架につけてから、その上着をとって四つに分け、おのおの、その一つを取った。また下着を手に取ってみたが、それには縫い目がなく、上の方から全部一つに織ったものであった。
\par 24 そこで彼らは互に言った、「それを裂かないで、だれのものになるか、くじを引こう」。これは、「彼らは互にわたしの上着を分け合い、わたしの衣をくじ引にした」という聖書が成就するためで、兵卒たちはそのようにしたのである。
\par 25 さて、イエスの十字架のそばには、イエスの母と、母の姉妹と、クロパの妻マリヤと、マグダラのマリヤとが、たたずんでいた。
\par 26 イエスは、その母と愛弟子とがそばに立っているのをごらんになって、母にいわれた、「婦人よ、ごらんなさい。これはあなたの子です」。
\par 27 それからこの弟子に言われた、「ごらんなさい。これはあなたの母です」。そのとき以来、この弟子はイエスの母を自分の家に引きとった。
\par 28 そののち、イエスは今や万事が終ったことを知って、「わたしは、かわく」と言われた。それは、聖書が全うされるためであった。
\par 29 そこに、酢いぶどう酒がいっぱい入れてある器がおいてあったので、人々は、このぶどう酒を含ませた海綿をヒソプの茎に結びつけて、イエスの口もとにさし出した。
\par 30 すると、イエスはそのぶどう酒を受けて、「すべてが終った」と言われ、首をたれて息をひきとられた。
\par 31 さてユダヤ人たちは、その日が準備の日であったので、安息日に死体を十字架の上に残しておくまいと、(特にその安息日は大事な日であったから)、ピラトに願って、足を折った上で、死体を取りおろすことにした。
\par 32 そこで兵卒らがきて、イエスと一緒に十字架につけられた初めの者と、もうひとりの者との足を折った。
\par 33 しかし、彼らがイエスのところにきた時、イエスはもう死んでおられたのを見て、その足を折ることはしなかった。
\par 34 しかし、ひとりの兵卒がやりでそのわきを突きさすと、すぐ血と水とが流れ出た。
\par 35 それを見た者があかしをした。そして、そのあかしは真実である。その人は、自分が真実を語っていることを知っている。それは、あなたがたも信ずるようになるためである。
\par 36 これらのことが起ったのは、「その骨はくだかれないであろう」との聖書の言葉が、成就するためである。
\par 37 また聖書のほかのところに、「彼らは自分が刺し通した者を見るであろう」とある。
\par 38 そののち、ユダヤ人をはばかって、ひそかにイエスの弟子となったアリマタヤのヨセフという人が、イエスの死体を取りおろしたいと、ピラトに願い出た。ピラトはそれを許したので、彼はイエスの死体を取りおろしに行った。
\par 39 また、前に、夜、イエスのみもとに行ったニコデモも、没薬と沈香とをまぜたものを百斤ほど持ってきた。
\par 40 彼らは、イエスの死体を取りおろし、ユダヤ人の埋葬の習慣にしたがって、香料を入れて亜麻布で巻いた。
\par 41 イエスが十字架にかけられた所には、一つの園があり、そこにはまだだれも葬られたことのない新しい墓があった。
\par 42 その日はユダヤ人の準備の日であったので、その墓が近くにあったため、イエスをそこに納めた。

\chapter{20}

\par 1 さて、一週の初めの日に、朝早くまだ暗いうちに、マグダラのマリヤが墓に行くと、墓から石がとりのけてあるのを見た。
\par 2 そこで走って、シモン・ペテロとイエスが愛しておられた、もうひとりの弟子のところへ行って、彼らに言った、「だれかが、主を墓から取り去りました。どこへ置いたのか、わかりません」。
\par 3 そこでペテロともうひとりの弟子は出かけて、墓へむかって行った。
\par 4 ふたりは一緒に走り出したが、そのもうひとりの弟子の方が、ペテロよりも早く走って先に墓に着き、
\par 5 そして身をかがめてみると、亜麻布がそこに置いてあるのを見たが、中へははいらなかった。
\par 6 シモン・ペテロも続いてきて、墓の中にはいった。彼は亜麻布がそこに置いてあるのを見たが、
\par 7 イエスの頭に巻いてあった布は亜麻布のそばにはなくて、はなれた別の場所にくるめてあった。
\par 8 すると、先に墓に着いたもうひとりの弟子もはいってきて、これを見て信じた。
\par 9 しかし、彼らは死人のうちからイエスがよみがえるべきことをしるした聖句を、まだ悟っていなかった。
\par 10 それから、ふたりの弟子たちは自分の家に帰って行った。
\par 11 しかし、マリヤは墓の外に立って泣いていた。そして泣きながら、身をかがめて墓の中をのぞくと、
\par 12 白い衣を着たふたりの御使が、イエスの死体のおかれていた場所に、ひとりは頭の方に、ひとりは足の方に、すわっているのを見た。
\par 13 すると、彼らはマリヤに、「女よ、なぜ泣いているのか」と言った。マリヤは彼らに言った、「だれかが、わたしの主を取り去りました。そして、どこに置いたのか、わからないのです」。
\par 14 そう言って、うしろをふり向くと、そこにイエスが立っておられるのを見た。しかし、それがイエスであることに気がつかなかった。
\par 15 イエスは女に言われた、「女よ、なぜ泣いているのか。だれを捜しているのか」。マリヤは、その人が園の番人だと思って言った、「もしあなたが、あのかたを移したのでしたら、どこへ置いたのか、どうぞ、おっしゃって下さい。わたしがそのかたを引き取ります」。
\par 16 イエスは彼女に「マリヤよ」と言われた。マリヤはふり返って、イエスにむかってヘブル語で「ラボニ」と言った。それは、先生という意味である。
\par 17 イエスは彼女に言われた、「わたしにさわってはいけない。わたしは、まだ父のみもとに上っていないのだから。ただ、わたしの兄弟たちの所に行って、『わたしは、わたしの父またあなたがたの父であって、わたしの神またあなたがたの神であられるかたのみもとへ上って行く』と、彼らに伝えなさい」。
\par 18 マグダラのマリヤは弟子たちのところに行って、自分が主に会ったこと、またイエスがこれこれのことを自分に仰せになったことを、報告した。
\par 19 その日、すなわち、一週の初めの日の夕方、弟子たちはユダヤ人をおそれて、自分たちのおる所の戸をみなしめていると、イエスがはいってきて、彼らの中に立ち、「安かれ」と言われた。
\par 20 そう言って、手とわきとを、彼らにお見せになった。弟子たちは主を見て喜んだ。
\par 21 イエスはまた彼らに言われた、「安かれ。父がわたしをおつかわしになったように、わたしもまたあなたがたをつかわす」。
\par 22 そう言って、彼らに息を吹きかけて仰せになった、「聖霊を受けよ。
\par 23 あなたがたがゆるす罪は、だれの罪でもゆるされ、あなたがたがゆるさずにおく罪は、そのまま残るであろう」。
\par 24 十二弟子のひとりで、デドモと呼ばれているトマスは、イエスがこられたとき、彼らと一緒にいなかった。
\par 25 ほかの弟子たちが、彼に「わたしたちは主にお目にかかった」と言うと、トマスは彼らに言った、「わたしは、その手に釘あとを見、わたしの指をその釘あとにさし入れ、また、わたしの手をそのわきにさし入れてみなければ、決して信じない」。
\par 26 八日ののち、イエスの弟子たちはまた家の内におり、トマスも一緒にいた。戸はみな閉ざされていたが、イエスがはいってこられ、中に立って「安かれ」と言われた。
\par 27 それからトマスに言われた、「あなたの指をここにつけて、わたしの手を見なさい。手をのばしてわたしのわきにさし入れてみなさい。信じない者にならないで、信じる者になりなさい」。
\par 28 トマスはイエスに答えて言った、「わが主よ、わが神よ」。
\par 29 イエスは彼に言われた、「あなたはわたしを見たので信じたのか。見ないで信ずる者は、さいわいである」。
\par 30 イエスは、この書に書かれていないしるしを、ほかにも多く、弟子たちの前で行われた。
\par 31 しかし、これらのことを書いたのは、あなたがたがイエスは神の子キリストであると信じるためであり、また、そう信じて、イエスの名によって命を得るためである。

\chapter{21}

\par 1 そののち、イエスはテベリヤの海べで、ご自身をまた弟子たちにあらわされた。そのあらわされた次第は、こうである。
\par 2 シモン・ペテロが、デドモと呼ばれているトマス、ガリラヤのカナのナタナエル、ゼベダイの子らや、ほかのふたりの弟子たちと一緒にいた時のことである。
\par 3 シモン・ペテロは彼らに「わたしは漁に行くのだ」と言うと、彼らは「わたしたちも一緒に行こう」と言った。彼らは出て行って舟に乗った。しかし、その夜はなんの獲物もなかった。
\par 4 夜が明けたころ、イエスが岸に立っておられた。しかし弟子たちはそれがイエスだとは知らなかった。
\par 5 イエスは彼らに言われた、「子たちよ、何か食べるものがあるか」。彼らは「ありません」と答えた。
\par 6 すると、イエスは彼らに言われた、「舟の右の方に網をおろして見なさい。そうすれば、何かとれるだろう」。彼らは網をおろすと、魚が多くとれたので、それを引き上げることができなかった。
\par 7 イエスの愛しておられた弟子が、ペテロに「あれは主だ」と言った。シモン・ペテロは主であると聞いて、裸になっていたため、上着をまとって海にとびこんだ。
\par 8 しかし、ほかの弟子たちは舟に乗ったまま、魚のはいっている網を引きながら帰って行った。陸からはあまり遠くない五十間ほどの所にいたからである。
\par 9 彼らが陸に上って見ると、炭火がおこしてあって、その上に魚がのせてあり、またそこにパンがあった。
\par 10 イエスは彼らに言われた、「今とった魚を少し持ってきなさい」。
\par 11 シモン・ペテロが行って、網を陸へ引き上げると、百五十三びきの大きな魚でいっぱいになっていた。そんなに多かったが、網はさけないでいた。
\par 12 イエスは彼らに言われた、「さあ、朝の食事をしなさい」。弟子たちは、主であることがわかっていたので、だれも「あなたはどなたですか」と進んで尋ねる者がなかった。
\par 13 イエスはそこにきて、パンをとり彼らに与え、また魚も同じようにされた。
\par 14 イエスが死人の中からよみがえったのち、弟子たちにあらわれたのは、これで既に三度目である。
\par 15 彼らが食事をすませると、イエスはシモン・ペテロに言われた、「ヨハネの子シモンよ、あなたはこの人たちが愛する以上に、わたしを愛するか」。ペテロは言った、「主よ、そうです。わたしがあなたを愛することは、あなたがご存じです」。イエスは彼に「わたしの小羊を養いなさい」と言われた。
\par 16 またもう一度彼に言われた、「ヨハネの子シモンよ、わたしを愛するか」。彼はイエスに言った、「主よ、そうです。わたしがあなたを愛することは、あなたがご存じです」。イエスは彼に言われた、「わたしの羊を飼いなさい」。
\par 17 イエスは三度目に言われた、「ヨハネの子シモンよ、わたしを愛するか」。ペテロは「わたしを愛するか」とイエスが三度も言われたので、心をいためてイエスに言った、「主よ、あなたはすべてをご存じです。わたしがあなたを愛していることは、おわかりになっています」。イエスは彼に言われた、「わたしの羊を養いなさい。
\par 18 よくよくあなたに言っておく。あなたが若かった時には、自分で帯をしめて、思いのままに歩きまわっていた。しかし年をとってからは、自分の手をのばすことになろう。そして、ほかの人があなたに帯を結びつけ、行きたくない所へ連れて行くであろう」。
\par 19 これは、ペテロがどんな死に方で、神の栄光をあらわすかを示すために、お話しになったのである。こう話してから、「わたしに従ってきなさい」と言われた。
\par 20 ペテロはふり返ると、イエスの愛しておられた弟子がついて来るのを見た。この弟子は、あの夕食のときイエスの胸近くに寄りかかって、「主よ、あなたを裏切る者は、だれなのですか」と尋ねた人である。
\par 21 ペテロはこの弟子を見て、イエスに言った、「主よ、この人はどうなのですか」。
\par 22 イエスは彼に言われた、「たとい、わたしの来る時まで彼が生き残っていることを、わたしが望んだとしても、あなたにはなんの係わりがあるか。あなたは、わたしに従ってきなさい」。
\par 23 こういうわけで、この弟子は死ぬことがないといううわさが、兄弟たちの間にひろまった。しかし、イエスは彼が死ぬことはないと言われたのではなく、ただ「たとい、わたしの来る時まで彼が生き残っていることを、わたしが望んだとしても、あなたにはなんの係わりがあるか」と言われただけである。
\par 24 これらの事についてあかしをし、またこれらの事を書いたのは、この弟子である。そして彼のあかしが真実であることを、わたしたちは知っている。
\par 25 イエスのなさったことは、このほかにまだ数多くある。もしいちいち書きつけるならば、世界もその書かれた文書を収めきれないであろうと思う。


\end{document}