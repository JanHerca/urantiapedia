\begin{document}

\title{ヨハネの手紙三}


\chapter{1}

\par 1 長老のわたしから、真実に愛している親愛なるガイオへ。
\par 2 愛する者よ。あなたのたましいがいつも恵まれていると同じく、あなたがすべてのことに恵まれ、またすこやかであるようにと、わたしは祈っている。
\par 3 兄弟たちがきて、あなたが真理に生きていることを、あかししてくれたので、ひじょうに喜んでいる。事実、あなたは真理のうちを歩いているのである。
\par 4 わたしの子供たちが真理のうちを歩いていることを聞く以上に、大きい喜びはない。
\par 5 愛する者よ。あなたが、兄弟たち、しかも旅先にある者につくしていることは、みな真実なわざである。
\par 6 彼らは、諸教会で、あなたの愛についてあかしをした。それらの人々を、神のみこころにかなうように送り出してくれたら、それは願わしいことである。
\par 7 彼らは、御名のために旅立った者であって、異邦人からは何も受けていない。
\par 8 それだから、わたしたちは、真理のための同労者となるように、こういう人々を助けねばならない。
\par 9 わたしは少しばかり教会に書きおくっておいたが、みんなのかしらになりたがっているデオテレペスが、わたしたちを受けいれてくれない。
\par 10 だから、わたしがそちらへ行った時、彼のしわざを指摘しようと思う。彼は口ぎたなくわたしたちをののしり、そればかりか、兄弟たちを受けいれようともせず、受けいれようとする人たちを妨げて、教会から追い出している。
\par 11 愛する者よ。悪にならわないで、善にならいなさい。善を行う者は神から出た者であり、悪を行う者は神を見たことのない者である。
\par 12 デメテリオについては、あらゆる人も、また真理そのものも、証明している。わたしたちも証明している。そして、あなたが知っているとおり、わたしたちの証明は真実である。
\par 13 あなたに書きおくりたいことはたくさんあるが、墨と筆とで書くことはすまい。
\par 14 すぐにでもあなたに会って、直接はなし合いたいものである。 (1:15) 平安が、あなたにあるように。友人たちから、あなたによろしく。友人たちひとりびとりに、よろしく。


\end{document}