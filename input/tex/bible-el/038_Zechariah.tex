\begin{document}

\title{Ζαχαρίας}


\chapter{1}

\par 1 Εν τω ογδόω μηνί, τω δευτέρω έτει του Δαρείου, έγεινε λόγος Κυρίου προς Ζαχαρίαν, τον υιόν του Βαραχίου υιού του Ιδδώ, τον προφήτην, λέγων,
\par 2 Ο Κύριος ωργίσθη μεγάλως επί τους πατέρας σας.
\par 3 Διά τούτο ειπέ προς αυτούς, Ούτω λέγει ο Κύριος των δυνάμεων· Επιστρέψατε προς εμέ, λέγει ο Κύριος των δυνάμεων, και θέλω επιστρέψει προς εσάς, λέγει ο Κύριος των δυνάμεων.
\par 4 Μη γίνεσθε ως οι πατέρες σας, προς τους οποίους οι πρότεροι προφήται έκραξαν λέγοντες, Ούτω λέγει ο Κύριος των δυνάμεων. Επιστρέψατε τώρα από των οδών υμών των πονηρών και των πράξεων υμών των πονηρών· και δεν υπήκουσαν και δεν έδωκαν προσοχήν εις εμέ, λέγει Κύριος.
\par 5 Οι πατέρες σας που είναι, και οι προφήται μήπως θέλουσι ζήσει εις τον αιώνα;
\par 6 Αλλ' οι λόγοι μου και τα διατάγματά μου, τα οποία προσέταξα εις τους δούλους μου τους προφήτας, δεν έφθασαν εις τους πατέρας σας; και αυτοί εστράφησαν και είπον, Καθώς ο Κύριος των δυνάμεων εβουλεύθη να κάμη εις ημάς, κατά τας οδούς ημών και κατά τας πράξεις ημών, ούτως έκαμεν εις ημάς.
\par 7 Εν τη εικοστή τετάρτη ημέρα του ενδεκάτου μηνός, όστις είναι ο μην Σαβάτ, εν τω δευτέρω έτει του Δαρείου, έγεινε λόγος Κυρίου προς Ζαχαρίαν τον υιόν του Βαραχίου υιού του Ιδδώ τον προφήτην, λέγων,
\par 8 Είδον την νύκτα και ιδού, άνθρωπος ιππεύων εφ' ίππου κοκκίνου και ούτος ίστατο μεταξύ των μυρσινών, αίτινες ήσαν εν κοιλώματι· και όπισθεν αυτού, ίπποι κόκκινοι, ποικίλοι και λευκοί.
\par 9 Και είπα, Κύριέ μου, τι είναι ούτοι; Και είπε προς εμέ ο άγγελος ο λαλών μετ' εμού, Εγώ θέλω σοι δείξει τι είναι ταύτα.
\par 10 Και ο άνθρωπος ο ιστάμενος μεταξύ των μυρσινών απεκρίθη και είπεν, Ούτοι είναι εκείνοι, τους οποίους ο Κύριος εξαπέστειλε να περιοδεύσωσι την γην.
\par 11 Και απεκρίθησαν προς τον άγγελον του Κυρίου τον ιστάμενον μεταξύ των μυρσινών και είπον, Ημείς περιωδεύσαμεν την γην και ιδού, πάσα η γη κάθηται και ησυχάζει.
\par 12 Και ο άγγελος του Κυρίου απεκρίθη και είπε, Κύριε των δυνάμεων, έως πότε δεν θέλεις σπλαγχνισθή συ την Ιερουσαλήμ και τας πόλεις του Ιούδα κατά των οποίων ηγανάκτησας τα εβδομήκοντα ταύτα έτη;
\par 13 Και ο Κύριος απεκρίθη προς τον άγγελον τον λαλούντα μετ' εμού λόγους καλούς λόγους παρηγορητικούς.
\par 14 Και είπε προς εμέ ο άγγελος ο λαλών μετ' εμού, Φώνησον, λέγων, Ούτω λέγει ο Κύριος των δυνάμεων· Είμαι ζηλότυπος διά την Ιερουσαλήμ και διά την Σιών εν ζηλοτυπία μεγάλη·
\par 15 και είμαι σφόδρα ωργισμένος κατά των εθνών των αμεριμνούντων· διότι ενώ εγώ ωργίσθην ολίγον, αυτά επεβοήθησαν το κακόν.
\par 16 Διά τούτο ούτω λέγει Κύριος· Εγώ επέστρεψα εις την Ιερουσαλήμ εν οικτιρμοίς· ο οίκός μου θέλει ανοικοδομηθή εν αυτή, λέγει ο Κύριος των δυνάμεων, και σχοινίον θέλει εκτανθή επί την Ιερουσαλήμ.
\par 17 Φώνησον έτι λέγων, Ούτω λέγει ο Κύριος των δυνάμεων· Αι πόλεις μου ότι θέλουσι πλημμυρήσει από αγαθών και ο Κύριος θέλει παρηγορήσει έτι την Σιών και θέλει εκλέξει πάλιν την Ιερουσαλήμ.
\par 18 Και ύψωσα τους οφθαλμούς μου και είδον και ιδού, τέσσαρα κέρατα·
\par 19 και είπα προς τον άγγελον τον λαλούντα μετ' εμού, Τι είναι ταύτα; Ο δε απεκρίθη προς εμέ, Ταύτα είναι τα κέρατα, τα οποία διεσκόρπισαν τον Ιούδαν, τον Ισραήλ και την Ιερουσαλήμ.
\par 20 Και ο Κύριος έδειξεν εις εμέ τέσσαρας τέκτονας·
\par 21 και είπα, Τι έρχονται ούτοι να κάμωσι; Και απεκρίθη λέγων, ταύτα είναι τα κέρατα τα οποία διεσκόρπισαν τον Ιούδαν, ώστε ουδείς εσήκωσε την κεφαλήν αυτού· και ούτοι ήλθον διά να φοβίσωσιν αυτά, διά να εκτινάξωσι τα κέρατα των εθνών, τα οποία εσήκωσαν το κέρας εναντίον της γης του Ιούδα διά να διασκορπίσωσιν αυτήν.

\chapter{2}

\par 1 Και ύψωσα τους οφθαλμούς μου και είδον και ιδού, ανήρ και σχοινίον μετρικόν εν τη χειρί αυτού·
\par 2 και είπα, Που υπάγεις συ; Ο δε είπε προς εμέ, να μετρήσω την Ιερουσαλήμ, διά να ίδω ποίον το πλάτος αυτής και ποίον το μήκος αυτής.
\par 3 Και ιδού, ο άγγελος ο λαλών μετ' εμού εξήλθε, και έτερος άγγελος εξήλθεν εις συνάντησιν αυτού
\par 4 και είπε προς αυτόν, Δράμε, λάλησον προς τον νεανίαν τούτον, λέγων, Η Ιερουσαλήμ θέλει κατοικηθή ατειχίστως εξ αιτίας του πλήθους των εν αυτή ανθρώπων και κτηνών·
\par 5 διότι εγώ, λέγει Κύριος, θέλω είσθαι εις αυτήν τείχος πυρός κύκλω και θέλω είσθαι προς δόξαν εν μέσω αυτής.
\par 6 Ω, ώ· φεύγετε από της γης του βορρά, λέγει Κύριος· διότι σας διεσκόρπισα προς τους τέσσαρας ανέμους του ουρανού, λέγει Κύριος.
\par 7 Ω, διασώθητι, Σιών, η κατοικούσα μετά της θυγατρός της Βαβυλώνος.
\par 8 Διότι ούτω λέγει ο Κύριος των δυνάμεων· Μετά την δόξαν με απέστειλε προς τα έθνη, τα οποία σας ελεηλάτησαν· διότι όστις εγγίζει εσάς, εγγίζει την κόρην του οφθαλμού αυτού.
\par 9 Διότι, ιδού, εγώ θέλω σείσει την χείρα μου επ' αυτά και θέλουσιν είσθαι λάφυρον εις τους δουλεύοντας αυτά· και θέλετε γνωρίσει ότι ο Κύριος των δυνάμεων με απέστειλε.
\par 10 Τέρπου και ευφραίνου, θύγατερ Σιών· διότι ιδού, εγώ έρχομαι και θέλω κατοικήσει εν μέσω σου, λέγει Κύριος.
\par 11 Και έθνη πολλά θέλουσιν ενωθή μετά του Κυρίου εν τη ημέρα εκείνη και θέλουσιν είσθαι λαός μου, και θέλω κατοικήσει εν μέσω σου, και θέλει, γνωρίσει ότι ο Κύριος των δυνάμεων με εξαπέστειλε προς σε.
\par 12 Και ο Κύριος θέλει κατακληρονομήσει τον Ιούδαν διά μερίδα αυτού εν τη γη τη αγία και θέλει εκλέξει πάλιν την Ιερουσαλήμ.
\par 13 Σιώπα, πάσα σαρξ, ενώπιον του Κυρίου· διότι εξηγέρθη από της κατοικίας της αγιότητος αυτού.

\chapter{3}

\par 1 Και μοι έδειξε τον Ιησούν, τον ιερέα τον μέγαν, ιστάμενον έμπροσθεν του αγγέλου του Κυρίου, και ο διάβολος ίστατο εκ δεξιών αυτού διά να αντισταθή εις αυτόν.
\par 2 Και είπε Κύριος προς τον διάβολον, Θέλει σε επιτιμήσει ο Κύριος, διάβολε· ναι, θέλει σε επιτιμήσει ο Κύριος, όστις εξέλεξε την Ιερουσαλήμ· δεν είναι ούτος δαυλός απεσπασμένος από πυρός;
\par 3 Ο δε Ιησούς ήτο ενδεδυμένος ιμάτια ρυπαρά και ίστατο ενώπιον του αγγέλου.
\par 4 Και απεκρίθη και είπε προς τους ισταμένους ενώπιον αυτού, λέγων, Αφαιρέσατε τα ιμάτια τα ρυπαρά απ' αυτού· και προς αυτόν είπεν, Ιδού, αφήρεσα από σου την ανομίαν σου και θέλω σε ενδύσει ιμάτια λαμπρά·
\par 5 και είπα, Ας επιθέσωσι μίτραν καθαράν επί την κεφαλήν αυτού. Και επέθεσαν την μίτραν την καθαράν επί την κεφαλήν αυτού και ενέδυσαν αυτόν ιμάτια· ο δε άγγελος του Κυρίου παρίστατο.
\par 6 Και διεμαρτυρήθη ο άγγελος του Κυρίου προς τον Ιησούν, λέγων,
\par 7 Ούτω λέγει ο Κύριος των δυνάμεων· Εάν περιπατήσης εν ταις οδοίς μου και εάν φυλάξης τας εντολάς μου, τότε συ θέλεις κρίνει έτι τον οίκόν μου και θέλεις φυλάττει έτι τας αυλάς μου και θέλω σοι δώσει να περιπατής μεταξύ των ενταύθα ισταμένων.
\par 8 Άκουε τώρα, Ιησού ο ιερεύς ο μέγας, συ και οι εταίροί σου οι καθήμενοι ενώπιόν σου, επειδή αυτοί είναι άνθρωποι θαυμάσιοι· διότι ιδού εγώ θέλω φέρει έξω τον δούλον μου τον Βλαστόν.
\par 9 Διότι ιδού, ο λίθος, τον οποίον έθεσα έμπροσθεν του Ιησού, επί τον ένα τούτον λίθον είναι επτά οφθαλμοί· ιδού, εγώ θέλω εγχαράξει το εγχάραγμα αυτού, λέγει ο Κύριος των δυνάμεων, και θέλω εξαλείψει την ανομίαν της γης εκείνης εν μιά ημέρα.
\par 10 Εν τη ημέρα εκείνη, λέγει ο Κύριος των δυνάμεων, θέλετε προσκαλέσει έκαστος τον πλησίον αυτού υπό την άμπελον αυτού και υπό την συκήν αυτού.

\chapter{4}

\par 1 Και επέστρεψεν ο άγγελος ο λαλών μετ' εμού και με εξήγειρεν ως άνθρωπον εξεγειρόμενον από του ύπνου αυτού,
\par 2 και είπε προς εμέ, Τι βλέπεις συ; Και είπα, Εθεώρησα και ιδού, λυχνία όλη χρυσή και δοχείον επί της κορυφής αυτής, και οι επτά λύχνοι αυτής επ' αυτής και επτά σωλήνες εις τους λύχνους τους επί της κορυφής αυτής,
\par 3 και δύο ελαίαι επάνωθεν αυτής, μία εκ δεξιών του δοχείου και μία εξ αριστερών αυτής.
\par 4 Και απεκρίθην και είπα προς τον άγγελον τον λαλούντα μετ' εμού, λέγων, Τι είναι ταύτα, κύριέ μου;
\par 5 Και απεκρίθη ο άγγελος ο λαλών μετ' εμού και είπε προς εμέ, Δεν γνωρίζεις τι είναι ταύτα; Και είπα, Ουχί, κύριέ μου.
\par 6 Και απεκρίθη και είπε προς εμέ, λέγων, Ούτος είναι ο λόγος του Κυρίου προς τον Ζοροβάβελ, λέγων, Ουχί διά δυνάμεως ουδέ διά ισχύος αλλά διά του Πνεύματός μου, λέγει ο Κύριος των δυνάμεων.
\par 7 Τις είσαι συ, το όρος το μέγα, έμπροσθεν του Ζοροβάβελ; πεδιάς· και θέλει εκφέρει τον ακρογωνιαίον λίθον εν αλαλαγμώ, Χάρις, χάρις εις αυτόν
\par 8 Και έγεινε λόγος Κυρίου προς εμέ, λέγων,
\par 9 Αι χείρες του Ζοροβάβελ έθεσαν το θεμέλιον του οίκου τούτου και αι χείρες αυτού θέλουσι τελειώσει αυτόν· και θέλεις γνωρίσει ότι ο Κύριος των δυνάμεων με απέστειλε προς εσάς.
\par 10 Διότι τις κατεφρόνησε την ημέραν των μικρών πραγμάτων; θέλουσι βεβαίως χαρή και θέλουσιν ιδεί τον κασσιτέρινον λίθον εν τη χειρί του Ζοροβάβελ οι επτά εκείνοι οφθαλμοί του Κυρίου, οι περιτρέχοντες διά πάσης της γης.
\par 11 Τότε απεκρίθην και είπα προς αυτόν, Τι είναι αι δύο αύται ελαίαι επί τα δεξιά της λυχνίας και επί τα αριστερά αυτής;
\par 12 Και απεκρίθην εκ δευτέρου και είπα προς αυτόν, Τι είναι οι δύο κλάδοι των ελαιών, οίτινες διά των δύο χρυσών σωλήνων εκκενόνουσιν εξ εαυτών το έλαιον εις την χρυσήν λυχνίαν;
\par 13 Και είπε προς εμέ λέγων, Δεν γνωρίζεις τι είναι ταύτα; Και είπα, Ουχί, κύριέ μου.
\par 14 Τότε είπεν, Ούτοι είναι οι δύο κεχρισμένοι, οι παριστάμενοι πλησίον του Κυρίου πάσης της γης.

\chapter{5}

\par 1 Και πάλιν ύψωσα τους οφθαλμούς μου και είδον και ιδού, τόμος πετώμενος.
\par 2 Και είπε προς εμέ, Τι βλέπεις συ; Και απεκρίθην, Βλέπω τόμον πετώμενον, το μήκος αυτού είκοσι πηχών και το πλάτος αυτού δέκα πηχών.
\par 3 Και είπε προς εμέ, Αύτη είναι η κατάρα η εξερχομένη επί το πρόσωπον πάσης της γής· διότι πας όστις κλέπτει θέλει εξολοθρευθή, ως γράφεται εν αυτώ εντεύθεν· και πας όστις ομνύει θέλει εξολοθρευθή, ως γράφεται εν αυτώ εκείθεν.
\par 4 Θέλω εκφέρει αυτήν, λέγει ο Κύριος των δυνάμεων, και θέλει εισέλθει εις τον οίκον του κλέπτου και εις τον οίκον του ομνύοντος εις το όνομά μου ψευδώς· και θέλει διαμείνει εν μέσω του οίκου αυτού, και θέλει εξολοθρεύσει αυτόν και τα ξύλα αυτού και τους λίθους αυτού.
\par 5 Και ο άγγελος ο λαλών μετ' εμού εξήλθε και είπε προς εμέ, Ύψωσον τώρα τους οφθαλμούς σου και ιδέ τι είναι τούτο το εξερχόμενον.
\par 6 Και είπα, Τι είναι τούτο; Ο δε είπε, τούτο το οποίον εξέρχεται είναι εφά· και είπε, Τούτο είναι η παράστασις αυτών καθ' όλην την γην.
\par 7 Και ιδού, εσηκόνετο τάλαντον μολύβδου· και ιδού, μία γυνή εκάθητο εν τω μέσω του εφά.
\par 8 Και είπεν, Αύτη είναι η ασέβεια. Και έρριψεν αυτήν εις το μέσον του εφά, και έρριψε το μολύβδινον ζύγιον εις το στόμα αυτού.
\par 9 Τότε ύψωσα τους οφθαλμούς μου και είδον και ιδού, εξήρχοντο δύο γυναίκες και άνεμος ήτο εν ταις πτέρυξιν αυτών, διότι αύται είχον πτέρυγας ως πτέρυγας πελαργού· και εσήκωσαν το εφά αναμέσον της γης και του ουρανού.
\par 10 Και είπα προς τον άγγελον τον λαλούντα μετ' εμού, Που φέρουσιν αύται το εφά;
\par 11 Και είπε προς εμέ, Διά να οικοδομήσωσι δι' αυτό οίκον εν τη γη Σενναάρ· και θέλει στηριχθή και θέλει τεθή εκεί επί την βάσιν αυτού.

\chapter{6}

\par 1 Και πάλιν ύψωσα τους οφθαλμούς μου και είδον και ιδού, τέσσαρες άμαξαι εξήρχοντο εκ του μέσου δύο ορέων και τα όρη ήσαν όρη χάλκινα.
\par 2 Εν τη αμάξη τη πρώτη ήσαν ίπποι κόκκινοι, και εν τη αμάξη τη δευτέρα ίπποι μέλανες,
\par 3 και εν τη αμάξη τη τρίτη ίπποι λευκοί, και εν τη αμάξη τη τετάρτη ίπποι ποικίλοι ψαροί.
\par 4 Και απεκρίθην και είπα προς τον άγγελον τον λαλούντα μετ' εμού, Τι είναι ταύτα, κύριέ μου;
\par 5 Και απεκρίθη ο άγγελος και είπε προς εμέ, Ταύτα είναι οι τέσσαρες άνεμοι του ουρανού, οίτινες εξέρχονται εκ της στάσεως αυτών ενώπιον του Κυρίου πάσης της γής·
\par 6 οι ίπποι οι μέλανες οι εν τη μιά εξέρχονται προς την γην του βορρά, και οι λευκοί εξέρχονται κατόπιν αυτών, και οι ποικίλοι εξέρχονται προς την γην του νότου.
\par 7 Και οι ψαροί εξήλθον και εζήτησαν να υπάγωσι διά να περιέλθωσι την γην. Και είπεν, Υπάγετε, περιέλθετε την γην. Και περιήλθον την γην.
\par 8 Και έκραξε προς εμέ και ελάλησε προς εμέ, λέγων, Ιδέ, οι εξερχόμενοι προς την γην του βορρά ανέπαυσαν το πνεύμά μου εν τη γη του βορρά.
\par 9 Και έγεινε λόγος Κυρίου προς εμέ, λέγων,
\par 10 Λάβε εκ των ανδρών της αιχμαλωσίας, εκ του Χελδαΐ, εκ του Τωβία και εκ του Ιεδαΐα, των ελθόντων εκ Βαβυλώνος, και ελθέ την αυτήν εκείνην ημέραν και είσελθε εις τον οίκον του Ιωσίου, υιού του Σοφονίου·
\par 11 και λάβε αργύριον και χρυσίον και κάμε στεφάνους και επίθες επί την κεφαλήν του Ιησού, υιού του Ιωσεδέκ, του ιερέως του μεγάλου,
\par 12 και λάλησον προς αυτόν, λέγων, Ούτω λέγει ο Κύριος των δυνάμεων, λέγων, Ιδού, ο ανήρ, του οποίου το όνομα είναι ο Βλαστός· και θέλει βλαστήσει εκ του τόπου αυτού και θέλει οικοδομήσει τον ναόν του Κυρίου.
\par 13 Ναι, αυτός θέλει οικοδομήσει τον ναόν του Κυρίου, και αυτός θέλει λάβει την δόξαν και θέλει καθήσει και διοικήσει επί του θρόνου αυτού και θέλει είσθαι ιερεύς επί του θρόνου αυτού, και βουλή ειρήνης θέλει είσθαι μεταξύ των δύο τούτων.
\par 14 Και στέφανοι θέλουσιν είσθαι διά τον Ελέμ και διά τον Τωβίαν και διά τον Ιεδαΐαν και διά τον Ειν τον υιόν του Σοφονίου προς μνημόσυνον εν τω ναώ του Κυρίου.
\par 15 Και οι μακράν θέλουσιν ελθεί και οικοδομήσει εν τω ναώ του Κυρίου· και θέλετε γνωρίσει ότι ο Κύριος των δυνάμεων με απέστειλε προς εσάς· και τούτο θέλει γείνει, εάν υπακούσητε ακριβώς εις την φωνήν Κυρίου του Θεού σας.

\chapter{7}

\par 1 Και εν τω τετάρτω έτει του βασιλέως Δαρείου έγεινε λόγος Κυρίου προς τον Ζαχαρίαν τη τετάρτη του εννάτου μηνός, του Χισλεύ·
\par 2 και εξαπέστειλαν εις τον οίκον του Θεού τον Σαρεσέρ και τον Ρεγέμ-μέλεχ και τους ανθρώπους αυτών, διά να εξιλεώσωσι το πρόσωπον του Κυρίου,
\par 3 να λαλήσωσι προς τους ιερείς τους εν τω οίκω του Κυρίου των δυνάμεων και προς τους προφήτας, λέγοντες, Να κλαύσω εν τω μηνί τω πέμπτω αποχωρισθείς, καθώς ήδη έκαμον τοσαύτα έτη;
\par 4 Και έγεινε λόγος του Κυρίου των δυνάμεων προς εμέ, λέγων,
\par 5 Λάλησον προς πάντα τον λαόν της γης και προς τους ιερείς, λέγων, και επενθείτε εν τω πέμπτω και εν τω εβδόμω μηνί τα εβδομήκοντα εκείνα έτη, ενηστεύετε τωόντι δι' εμέ; δι' εμέ;
\par 6 Και ότε ετρώγετε και ότε επίνετε, δεν ετρώγετε και επίνετε δι' εαυτούς;
\par 7 δεν είναι ούτοι οι λόγοι, τους οποίους ο Κύριος ελάλησε διά των προτέρων προφητών, ότε η Ιερουσαλήμ ήτο κατωκημένη και εν ευημερία και αι πόλεις αυτής κύκλω εν αυτή, ότε κατωκείτο το μεσημβρινόν και η πεδινή;
\par 8 Και έγεινε λόγος Κυρίου προς τον Ζαχαρίαν, λέγων,
\par 9 Ούτω λέγει ο Κύριος των δυνάμεων, λέγων, Κρίνετε κρίσιν αληθείας και κάμνετε έλεος και οικτιρμόν, έκαστος προς τον αδελφόν αυτού,
\par 10 και μη καταδυναστεύετε την χήραν και τον ορφανόν, τον ξένον και τον πένητα, και μηδείς από σας ας μη βουλεύηται κακόν κατά του αδελφού αυτού εν τη καρδία αυτού.
\par 11 Αλλ' ηρνήθησαν να προσέξωσι και έστρεψαν νώτα απειθή και εβάρυναν τα ώτα αυτών διά να μη ακούσωσι.
\par 12 Ναι, αυτοί έκαμον τας καρδίας αυτών αδάμαντα, ώστε να μη ακούσωσι τον νόμον και τους λόγους, τους οποίους ο Κύριος των δυνάμεων εξαπέστειλεν εν τω πνεύματι αυτού διά των προτέρων προφητών· διά τούτο ήλθεν οργή μεγάλη παρά του Κυρίου των δυνάμεων.
\par 13 Όθεν καθώς αυτός έκραξε και αυτοί δεν εισήκουον, ούτως αυτοί έκραξαν και εγώ δεν εισήκουον, λέγει ο Κύριος των δυνάμεων·
\par 14 αλλά διεσκόρπισα αυτούς ως δι' ανεμοστροβίλου εις πάντα τα έθνη, τα οποία δεν εγνώριζον. Και ο τόπος ηρημώθη κατόπιν αυτών, ώστε δεν υπήρχεν ο διαβαίνων ουδέ ο επιστρέφων· και έθεσαν την γην την επιθυμητήν εις ερήμωσιν.

\chapter{8}

\par 1 Και έγεινε λόγος του Κυρίου των δυνάμεων, λέγων,
\par 2 Ούτω λέγει ο Κύριος των δυνάμεων· είμαι ζηλότυπος διά την Σιών εν ζηλοτυπία μεγάλη και είμαι ζηλότυπος δι' αυτήν εν οργή μεγάλη.
\par 3 Ούτω λέγει Κύριος· Επέστρεψα εις την Σιών και θέλω κατοικήσει εν μέσω της Ιερουσαλήμ· και η Ιερουσαλήμ θέλει ονομασθή πόλις αληθείας, και το όρος του Κυρίου των δυνάμεων όρος άγιον.
\par 4 Ούτω λέγει ο Κύριος των δυνάμεων· Έτι θέλουσι καθήσει πρεσβύτεροι και πρεσβύτεραι εν ταις πλατείαις της Ιερουσαλήμ, και έκαστος με την ράβδον αυτού εν τη χειρί αυτού από του πλήθους των ημερών.
\par 5 Και αι πλατείαι της πόλεως θέλουσιν είσθαι πλήρεις παιδίων και κορασίων παιζόντων εν ταις πλατείαις αυτής.
\par 6 Ούτω λέγει ο Κύριος των δυνάμεων· Εάν φανή θαυμαστόν εις τους οφθαλμούς του υπολοίπου του λαού τούτου εν ταις ημέραις εκείναις, μήπως θέλει φανή θαυμαστόν και εις τους οφθαλμούς μου; λέγει ο Κύριος των δυνάμεων.
\par 7 Ούτω λέγει ο Κύριος των δυνάμεων· Ιδού, εγώ θέλω σώσει τον λαόν μου από της γης της ανατολής και από της γης της δύσεως του ηλίου,
\par 8 και θέλω φέρει αυτούς και θέλουσι κατοικήσει εν μέσω της Ιερουσαλήμ· και θέλουσιν είσθαι λαός μου και εγώ θέλω είσθαι Θεός αυτών εν αληθεία και δικαιοσύνη.
\par 9 Ούτω λέγει ο Κύριος των δυνάμεων· Ενισχύσατε τας χείρας σας, σεις οι ακούοντες εν ταις ημέραις ταύταις τους λόγους τούτους διά στόματος των προφητών, οίτινες ήσαν εν τη ημέρα καθ' ην εθεμελιώθη ο οίκος του Κυρίου των δυνάμεων, διά να οικοδομηθή ο ναός.
\par 10 Διότι προ των ημερών εκείνων δεν ήτο μισθός διά τον άνθρωπον ουδέ μισθός διά το κτήνος ουδέ ειρήνη εις τον εξερχόμενον ή εισερχόμενον εξ αιτίας της θλίψεως, διότι εξαπέστειλα πάντας τους ανθρώπους έκαστον κατά του πλησίον αυτού.
\par 11 Αλλά τώρα, εγώ δεν θέλω φέρεσθαι προς το υπόλοιπον του λαού τούτου καθώς εν ταις αρχαίαις ημέραις, λέγει ο Κύριος των δυνάμεων.
\par 12 Διότι ο σπόρος θέλει είσθαι της ειρήνης· η άμπελος θέλει δώσει τον καρπόν αυτής και η γη θέλει δώσει τα γεννήματα αυτής και οι ουρανοί θέλουσι δώσει την δρόσον αυτών, και θέλω κληροδοτήσει εις το υπόλοιπον του λαού τούτου πάντα ταύτα.
\par 13 Και καθώς ήσθε κατάρα μεταξύ των εθνών, οίκος Ιούδα και οίκος Ισραήλ, ούτω θέλω σας διασώσει και θέλετε είσθαι ευλογία· μη φοβείσθε· ας ενισχύωνται αι χείρές σας.
\par 14 Διότι ούτω λέγει ο Κύριος των δυνάμεων· Καθ' ον τρόπον εστοχάσθην να σας τιμωρήσω, ότε οι πατέρες σας με παρώργισαν, λέγει ο Κύριος των δυνάμεων, και δεν μετενόησα,
\par 15 ούτω πάλιν εβουλεύθην εν ταις ημέραις ταύταις να αγαθοποιήσω την Ιερουσαλήμ και τον οίκον του Ιούδα· μη φοβείσθε.
\par 16 Ούτοι είναι οι λόγοι, τους οποίους θέλετε κάμει· λαλείτε έκαστος την αλήθειαν προς τον πλησίον αυτού· αλήθειαν και κρίσιν ειρήνης κρίνετε εν ταις πύλαις σας.
\par 17 Και μη βουλεύεσθε κακόν εν ταις καρδίαις σας έκαστος κατά του πλησίον αυτού και όρκον ψευδή μη αγαπάτε· διότι πάντα ταύτα είναι εκείνα, τα οποία μισώ, λέγει Κύριος.
\par 18 Και έγεινε λόγος του Κυρίου των δυνάμεων προς εμέ, λέγων,
\par 19 Ούτω λέγει ο Κύριος των δυνάμεων· Η νηστεία του τετάρτου μηνός και η νηστεία του πέμπτου και η νηστεία του εβδόμου και η νηστεία του δεκάτου θέλουσιν είσθαι εις τον οίκον Ιούδα εν χαρά και εν ευφροσύνη και εν ευθύμοις εορταίς· όθεν αγαπάτε την αλήθειαν και την ειρήνην.
\par 20 Ούτω λέγει ο Κύριος των δυνάμεων· Έτι θέλουσιν ελθεί λαοί και οι κατοικούντες πόλεις πολλάς·
\par 21 και οι κάτοικοι της μιας θέλουσιν υπάγει εις την άλλην, λέγοντες, Ας υπάγωμεν σπεύδοντες διά να εξιλεώσωμεν το πρόσωπον του Κυρίου και να εκζητήσωμεν τον Κύριον των δυνάμεων· θέλω υπάγει και εγώ.
\par 22 Και λαοί πολλοί και έθνη ισχυρά θέλουσιν ελθεί διά να εκζητήσωσι τον Κύριον των δυνάμεων εν Ιερουσαλήμ και να εξιλεώσωσι το πρόσωπον του Κυρίου.
\par 23 Ούτω λέγει ο Κύριος των δυνάμεων· Εν ταις ημέραις εκείναις δέκα άνδρες εκ πασών των γλωσσών των εθνών θέλουσι πιάσει σφιγκτά, ναι, θέλουσι πιάσει σφιγκτά το κράσπεδον ενός Ιουδαίου, λέγοντες· θέλομεν υπάγει με σάς· διότι ηκούσαμεν ότι ο Θεός είναι με σας.

\chapter{9}

\par 1 Το φορτίον του λόγου του Κυρίου κατά της γης Αδράχ και της Δαμασκού, της αναπαύσεως αυτού· διότι του Κυρίου είναι το να εφορά τους ανθρώπους και πάσας τας φυλάς του Ισραήλ·
\par 2 ότι και κατά της Αιμάθ, ήτις είναι όμορος αυτής, κατά της Τύρου και Σιδώνος, αν και ήναι σοφαί σφόδρα.
\par 3 Και η Τύρος ωκοδόμησεν εις εαυτήν οχύρωμα και επεσώρευσεν αργύριον ως χώμα και χρυσίον ως πηλόν των οδών.
\par 4 Ιδού, ο Κύριος θέλει σκυλεύσει αυτήν και θέλει πατάξει εν τη θαλάσση την δύναμιν αυτής, και αυτή θέλει καταναλωθή εν πυρί.
\par 5 Θέλει ιδεί η Ασκάλων και φοβηθή, και η Γάζα και θέλει λυπηθή σφόδρα, και η Ακκαρών, διότι η προσδοκία αυτής θέλει ματαιωθή· και θέλει απολεσθή ο βασιλεύς εκ της Γάζης και η Ασκάλων δεν θέλει κατοικείσθαι.
\par 6 Και αλλογενής θέλει καθήσει εν τη Αζώτω, και θέλω καθαιρέσει την υπερηφανίαν των Φιλισταίων.
\par 7 Και θέλω αφαιρέσει το αίμα αυτών από του στόματος αυτών και τα βδελύγματα αυτών από του μέσου των οδόντων αυτών· και ο εναπολειφθείς θέλει είσθαι και αυτός διά τον Θεόν ημών, και θέλει είσθαι ως χιλίαρχος εις τον Ιούδαν· και Ακκαρών θέλει είσθαι ως ο Ιεβουσαίος.
\par 8 Και θέλω στρατοπεδεύσει κύκλω του οίκου μου εναντίον στρατεύματος, εναντίον διαβαίνοντος και εναντίον επιστρέφοντος· και καταδυναστεύων δεν θέλει περάσει πλέον επ' αυτούς· διότι τώρα είδον με τους οφθαλμούς μου.
\par 9 Χαίρε σφόδρα, θύγατερ Σιών· αλάλαζε, θύγατερ Ιερουσαλήμ· ιδού, ο βασιλεύς σου έρχεται προς σέ· αυτός είναι δίκαιος και σώζων· πραΰς και καθήμενος επί όνου και επί πώλου υιού υποζυγίου.
\par 10 Και θέλω εξολοθρεύσει την άμαξαν από του Εφραΐμ και τον ίππον από της Ιερουσαλήμ και θέλει εξολοθρευθή το τόξον το πολεμικόν, και αυτός θέλει λαλήσει ειρήνην προς τα έθνη, και η εξουσία αυτού θέλει είσθαι από θαλάσσης έως θαλάσσης και από ποταμού έως των περάτων της γης.
\par 11 Και περί σου, διά το αίμα της διαθήκης σου εγώ εξήγαγον τους δεσμίους σου εκ λάκκου ανύδρου.
\par 12 Επιστρέψατε εις το οχύρωμα, δέσμιοι της ελπίδος· έτι και σήμερον κηρύττω ότι θέλω ανταποδώσει διπλά εις σε.
\par 13 Διότι ενέτεινα τον Ιούδαν δι' εμαυτόν ως τόξον· ισχυρώς ενέτεινα τον Εφραΐμ και εξήγειρα τα τέκνα σου, Σιών, κατά των τέκνων σου, Ελλάς· και σε έκαμον ως ρομφαίαν μαχητού.
\par 14 Και ο Κύριος θέλει φανή επ' αυτούς και το βέλος αυτού θέλει εξέλθει ως αστραπή· και Κύριος ο Θεός θέλει σαλπίσει εν σάλπιγγι και θέλει κινηθή με ανεμοστροβίλους του νότου.
\par 15 Ο Κύριος των δυνάμεων θέλει υπερασπίζεσθαι αυτούς και θέλουσι καταναλώσει τους εναντίους και καταβάλει με λίθους σφενδόνης και θέλουσι πίει και θορυβήσει ως από οίνου· και θέλουσιν εμπλησθή ως φιάλη και ως αι γωνίαι του θυσιαστηρίου.
\par 16 Και Κύριος ο Θεός αυτών θέλει σώσει αυτούς εν τη ημέρα εκείνη, ως το ποίμνιον του λαού αυτού, επειδή ως λίθοι διαδήματος θέλουσιν υψωθή επί την γην αυτού.
\par 17 Διότι πόση είναι η αγαθότης αυτού και πόση η ώραιότης αυτού· ο σίτος θέλει κάμει ευθύμους τους νεανίσκους και το γλεύκος τας παρθένους.

\chapter{10}

\par 1 Ζητείτε παρά του Κυρίου υετόν εν τω καιρώ της οψίμου βροχής· και ο Κύριος θέλει κάμει αστραπάς και θέλει δώσει εις αυτούς βροχάς όμβρου, εις έκαστον βοτάνην εν τω αγρώ.
\par 2 Διότι τα είδωλα ελάλησαν ματαιότητα και οι μάντεις είδον οράσεις ψευδείς και ελάλησαν ενύπνια μάταια· παρηγόρουν ματαίως· διά τούτο μετετοπίσθησαν ως ποίμνιον· εταράχθησαν, διότι δεν υπήρχε ποιμήν.
\par 3 Ο θυμός μου εξήφθη κατά των ποιμένων και θέλω τιμωρήσει τους τράγους· διότι ο Κύριος των δυνάμεων επεσκέφθη το ποίμνιον αυτού, τον οίκον Ιούδα, και έκαμεν αυτούς ως ίππον αυτού ένδοξον εν μάχη.
\par 4 Απ' αυτού εξήλθεν η γωνία, απ' αυτού ο πάσσαλος, απ' αυτού το πολεμικόν τόξον, απ' αυτού πας ηγεμών ομού.
\par 5 Και θέλουσιν είσθαι ως ισχυροί, καταπατούντες τους πολεμίους εν τω πηλώ των οδών, εν τη μάχη· και θέλουσι πολεμήσει, διότι ο Κύριος είναι μετ' αυτών, και οι αναβάται των ίππων θέλουσι καταισχυνθή.
\par 6 Και θέλω ενισχύσει τον οίκον Ιούδα και τον οίκον Ιωσήφ θέλω σώσει, και θέλω επαναφέρει αυτούς, διότι ηλέησα αυτούς· και θέλουσιν είσθαι ως εάν δεν είχον αποβάλει αυτούς· διότι εγώ είμαι Κύριος ο Θεός αυτών και θέλω εισακούσει αυτών.
\par 7 Και οι Εφραϊμίται θέλουσιν είσθαι ως ισχυρός και η καρδία αυτών θέλει χαρή ως από οίνου· και τα τέκνα αυτών θέλουσιν ιδεί και χαρή· η καρδία αυτών θέλει ευφρανθή εις τον Κύριον.
\par 8 Θέλω συρίξει εις αυτούς και θέλω συνάξει αυτούς· διότι εγώ ελύτρωσα αυτούς· και θέλουσι πληθυνθή καθώς ποτέ επληθύνθησαν.
\par 9 Και θέλω σπείρει αυτούς μεταξύ των λαών· και θέλουσι με ενθυμηθή εν απομεμακρυσμένοις τόποις· και θέλουσι ζήσει μετά των τέκνων αυτών και θέλουσιν επιστρέψει.
\par 10 Και θέλω επαναφέρει αυτούς εκ γης Αιγύπτου και συνάξει αυτούς εκ της Ασσυρίας· και θέλω φέρει αυτούς εις την γην Γαλαάδ και εις τον Λίβανον, και δεν θέλει εξαρκέσει εις αυτούς.
\par 11 Και θέλει περάσει διά της θαλάσσης εν θλίψει και θέλει πατάξει τα κύματα εν τη θαλάσση και πάντα τα βάθη του ποταμού θέλουσι ξηρανθή, και η υπερηφανία της Ασσυρίας θέλει καταβληθή και το σκήπτρον της Αιγύπτου θέλει αφαιρεθή.
\par 12 Και θέλω ενισχύσει αυτούς εις τον Κύριον, και θέλουσι περιπατεί εν τω ονόματι αυτού, λέγει Κύριος.

\chapter{11}

\par 1 Άνοιξον, Λίβανε, τας θύρας σου και ας καταφάγη πυρ τας κέδρους σου.
\par 2 Ολόλυξον, ελάτη, διότι έπεσεν κέδρος· διότι οι μεγιστάνες ηφανίσθησαν· ολολύξατε, δρυς της Βασάν, διότι το δάσος το απρόσιτον κατεκόπη.
\par 3 Φωνή ακούεται ποιμένων θρηνούντων, διότι η δόξα αυτών ηφανίσθη· φωνή βρυχωμένων σκύμνων, διότι το φρύαγμα του Ιορδάνου εταπεινώθη.
\par 4 Ούτω λέγει Κύριος ο Θεός μου· Ποίμαινε το ποίμνιον της σφαγής,
\par 5 το οποίον οι αγοράσαντες αυτό σφάζουσιν ατιμωρήτως· οι δε πωλούντες αυτό λέγουσιν, Ευλογητός ο Κύριος, διότι επλούτησα, και αυτοί οι ποιμένες αυτού δεν φείδονται αυτού.
\par 6 Διά τούτο δεν θέλω φεισθή πλέον των κατοίκων του τόπου, λέγει Κύριος, αλλ' ιδού, εγώ θέλω παραδώσει τους ανθρώπους έκαστον εις την χείρα του πλησίον αυτού και εις την χείρα του βασιλέως αυτού, και θέλουσι κατακόψει την γην και δεν θέλω ελευθερώσει αυτούς εκ της χειρός αυτών.
\par 7 Και εποίμανα το ποίμνιον της σφαγής, το όντως τεταλαιπωρημένον ποίμνιον. Και έλαβον εις εμαυτόν δύο ράβδους, την μίαν εκάλεσα Κάλλος και την άλλην εκάλεσα Δεσμούς, και εποίμανα το ποίμνιον.
\par 8 Και εξωλόθρευσα τρεις ποιμένας εν ενί μηνί· και η ψυχή μου εβαρύνθη αυτούς και η ψυχή δε αυτών απεστράφη εμέ.
\par 9 Τότε είπα, Δεν θέλω σας ποιμαίνει· το αποθνήσκον ας αποθνήσκη και το απολωλός ας απόλλυται και τα εναπολειπόμενα ας τρώγωσιν έκαστον την σάρκα του πλησίον αυτού.
\par 10 Και έλαβον την ράβδον μου, το Κάλλος, και κατέκοψα αυτήν, διά να ακυρώσω την διαθήκην μου, την οποίαν έκαμον προς πάντας τους λαούς τούτους,
\par 11 και ηκυρώθη εν τη ημέρα εκείνη· και ούτω το ποίμνιον το τεταλαιπωρημένον, το οποίον απέβλεπεν εις εμέ, εγνώρισεν ότι ούτος ήτο ο λόγος του Κυρίου.
\par 12 Και είπα προς αυτούς, Εάν σας φαίνηται καλόν, δότε μοι τον μισθόν μου· ει δε μη, αρνήθητε αυτόν. Και έστησαν τον μισθόν μου τριάκοντα αργύρια.
\par 13 Και είπε Κύριος προς εμέ, Ρίψον αυτά εις τον κεραμέα, την έντιμον τιμήν, με την οποίαν ετιμήθην υπ' αυτών. Και έλαβον τα τριάκοντα αργύρια και έρριψα αυτά εν τω οίκω του Κυρίου εις τον κεραμέα.
\par 14 Και κατέκοψα την άλλην μου ράβδον, τους Δεσμούς, διά να ακυρώσω την αδελφότητα μεταξύ Ιούδα και Ισραήλ.
\par 15 Και είπε Κύριος προς εμέ, Λάβε εις σεαυτόν έτι τα εργαλεία ποιμένος ασυνέτου.
\par 16 Διότι ιδού, εγώ θέλω αναστήσει ποιμένα επί την γην, όστις δεν θέλει επισκέπτεσθαι τα απολωλότα, δεν θέλει ζητεί το διεσκορπισμένον και δεν θέλει ιατρεύει το συντετριμμένον ουδέ θέλει ποιμαίνει το υγιές· αλλά θέλει τρώγει την σάρκα του παχέος και κατακόπτει τους όνυχας αυτών.
\par 17 Ουαί εις τον μάταιον ποιμένα, τον εγκαταλείποντα το ποίμνιον· ρομφαία θέλει ελθεί επί τον βραχίονα αυτού και επί τον δεξιόν οφθαλμόν αυτού· ο βραχίων αυτού θέλει ολοτελώς ξηρανθή και ο δεξιός οφθαλμός αυτού ολοκλήρως αμαυρωθή.

\chapter{12}

\par 1 Το φορτίον του λόγου του Κυρίου περί του Ισραήλ. Ούτω λέγει Κύριος, ο εκτείνων τους ουρανούς και θεμελιών την γην και μορφόνων το πνεύμα του ανθρώπου εντός αυτού·
\par 2 Ιδού, εγώ καθιστώ την Ιερουσαλήμ ποτήριον ζάλης εις πάντας τους λαούς κύκλω, και επί τον Ιούδαν έτι θέλει είσθαι τούτο εν τη πολιορκία τη κατά της Ιερουσαλήμ.
\par 3 Και εν τη ημέρα εκείνη θέλω καταστήσει την Ιερουσαλήμ προς πάντας τους λαούς λίθον καταβαρύνοντα· πάντες όσοι επιφορτισθώσιν αυτόν θέλουσι κατασυντριφθή, όταν πάντα τα έθνη της γης συναχθώσιν εναντίον αυτής.
\par 4 Εν τη ημέρα εκείνη, λέγει Κύριος, θέλω πατάξει πάντα ίππον εν εκστάσει και τον αναβάτην αυτού εν παραφροσύνη, και θέλω ανοίξει τους οφθαλμούς μου επί τον οίκον Ιούδα και θέλω πατάξει εν αποτυφλώσει πάντα ίππον των λαών.
\par 5 Και οι άρχοντες του Ιούδα θέλουσιν ειπεί εν τη καρδία αυτών, Στήριγμα είναι εις εμέ οι κάτοικοι της Ιερουσαλήμ διά του Κυρίου των δυνάμεων του Θεού αυτών.
\par 6 Εν τη ημέρα εκείνη θέλω καταστήσει τους άρχοντας του Ιούδα ως εστίαν πυρός εις ξύλα και ως λαμπάδα πυρός εις χειρόβολον, και θέλουσι καταφάγει πάντας τους λαούς κύκλω, εκ δεξιών και εξ αριστερών· και η Ιερουσαλήμ θέλει κατοικηθή πάλιν εν τω τόπω αυτής, εν Ιερουσαλήμ.
\par 7 Και ο Κύριος θέλει σώσει πρώτον τας σκηνάς του Ιούδα, διά να μη μεγαλύνηται η δόξα του οίκου του Δαβίδ και η δόξα των κατοίκων της Ιερουσαλήμ κατά του Ιούδα.
\par 8 Εν τη ημέρα εκείνη ο Κύριος θέλει υπερασπισθή τους κατοίκους της Ιερουσαλήμ· και ο αδύνατος μεταξύ αυτών εν τη ημέρα εκείνη θέλει είσθαι ως ο Δαβίδ και ο οίκος του Δαβίδ ως Θεός, ως άγγελος Κυρίου, ενώπιον αυτών.
\par 9 Και εν τη ημέρα εκείνη θέλω ζητήσει να εξολοθρεύσω πάντα τα έθνη τα ερχόμενα κατά της Ιερουσαλήμ.
\par 10 Και θέλω εκχέει επί τον οίκον Δαβίδ και επί τους κατοίκους της Ιερουσαλήμ πνεύμα χάριτος και ικεσιών· και θέλουσιν επιβλέψει προς εμέ, τον οποίον εξεκέντησαν, και θέλουσι πενθήσει δι' αυτόν ως πενθεί τις διά τον μονογενή αυτού, και θέλουσι λυπηθή δι' αυτόν, ως ο λυπούμενος διά τον πρωτότοκον αυτού.
\par 11 Εν τη ημέρα εκείνη θέλει είσθαι πένθος μέγα εν Ιερουσαλήμ ως το πένθος της Αδαδριμμών εν τη πεδιάδι Μεγιδδών.
\par 12 Και θέλει πενθήσει η γη, πάσα οικογένεια καθ' εαυτήν· η οικογένεια του οίκου Δαβίδ καθ' εαυτήν και αι γυναίκες αυτών καθ' εαυτάς, η οικογένεια του οίκου Νάθαν καθ' εαυτήν και αι γυναίκες αυτών καθ' εαυτάς,
\par 13 η οικογένεια του οίκου Λευΐ καθ' εαυτήν και αι γυναίκες αυτών καθ' εαυτάς, η οικογένεια Σιμεΐ καθ' εαυτήν και αι γυναίκες αυτών καθ' εαυτάς,
\par 14 πάσαι αι εναπολειφθείσαι οικογένειαι, εκάστη οικογένεια καθ' εαυτήν και αι γυναίκες αυτών καθ' εαυτάς.

\chapter{13}

\par 1 Εν τη ημέρα εκείνη θέλει είσθαι πηγή ανεωγμένη εις τον οίκον Δαβίδ και εις τους κατοίκους της Ιερουσαλήμ διά την αμαρτίαν και διά την ακαθαρσίαν.
\par 2 Και εν τη ημέρα εκείνη, λέγει ο Κύριος των δυνάμεων, θέλω εξολοθρεύσει τα ονόματα των ειδώλων από της γης και δεν θέλει πλέον είσθαι ενθύμησις αυτών, και έτι θέλω αφαιρέσει τους προφήτας και το πνεύμα το ακάθαρτον από της γης.
\par 3 Και εάν τις προφητεύη έτι, τότε ο πατήρ αυτού και η μήτηρ αυτού οι γεννήσαντες αυτόν θέλουσιν ειπεί προς αυτόν, δεν θέλεις ζήσει· διότι ψεύδη λαλείς εν τω ονόματι του Κυρίου. Και ο πατήρ αυτού και η μήτηρ αυτού οι γεννήσαντες αυτόν θέλουσι διατραυματίσει αυτόν, όταν προφητεύη.
\par 4 Και εν τη ημέρα εκείνη οι προφήται θέλουσι καταισχυνθή, έκαστος εκ της οράσεως αυτού, όταν προφητεύη, και δεν θέλουσιν ενδύεσθαι ένδυμα τρίχινον διά να απατώσι.
\par 5 Και θέλει ειπεί, Εγώ δεν είμαι προφήτης· είμαι άνθρωπος γεωργός· διότι άνθρωπος με εμίσθωσεν εκ νεότητός μου.
\par 6 Και εάν τις είπη προς αυτόν, Τι είναι αι πληγαί αύται εν μέσω των χειρών σου; θέλει αποκριθή, Εκείναι, τας οποίας επληγώθην εν τω οίκω των φίλων μου.
\par 7 Ρομφαία, εξύπνησον κατά του ποιμένος μου και κατά του ανδρός του συνεταίρου μου, λέγει ο Κύριος των δυνάμεων· πάταξον τον ποιμένα και τα πρόβατα θέλουσι διασκορπισθή· θέλω όμως στρέψει την χείρα μου επί τους μικρούς.
\par 8 Και εν πάση τη γη, λέγει Κύριος, δύο μέρη θέλουσιν εξολοθρευθή εν αυτή και εκλείψει, το δε τρίτον θέλει εναπολειφθή εν αυτή.
\par 9 Και θέλω περάσει το τρίτον διά πυρός· και θέλω καθαρίσει αυτούς ως καθαρίζεται το αργύριον, και θέλω δοκιμάσει αυτούς ως δοκιμάζεται το χρυσίον· αυτοί θέλουσιν επικαλεσθή το όνομά μου κα εγώ θέλω εισακούσει αυτούς· θέλω ειπεί, ούτος είναι λαός μου· και αυτοί θέλουσιν ειπεί, Ο Κύριος είναι ο Θεός μου.

\chapter{14}

\par 1 Ιδού, η ημέρα του Κυρίου έρχεται και το λάφυρόν σου θέλει διαμερισθή εν τω μέσω σου.
\par 2 Και θέλω συνάξει πάντα τα έθνη κατά της Ιερουσαλήμ εις μάχην· και θέλει αλωθή η πόλις και αι οικίαι θέλουσι λεηλατηθή και αι γυναίκες θέλουσι βιασθή, και το ήμισυ της πόλεως θέλει εξέλθει εις αιχμαλωσίαν, το δε υπόλοιπον του λαού δεν θέλει εξολοθρευθή εκ της πόλεως.
\par 3 Και ο Κύριος θέλει εξέλθει και θέλει πολεμήσει κατά των εθνών εκείνων, ως ότε επολέμησεν εν τη ημέρα της μάχης.
\par 4 Και οι πόδες αυτού θέλουσι σταθή κατά την ημέραν εκείνην επί του όρους των ελαιών, του απέναντι της Ιερουσαλήμ εξ ανατολών· και το όρος των ελαιών θέλει σχισθή κατά το μέσον αυτού προς ανατολάς και προς δυσμάς και θέλει γείνει κοιλάς μεγάλη σφόδρα· και το ήμισυ του όρους θέλει συρθή προς βορράν και το ήμισυ αυτού προς νότον.
\par 5 Και θέλετε καταφύγει εις την κοιλάδα των ορέων μου· διότι η κοιλάς των ορέων θέλει φθάνει έως εις Ασάλ· και θέλετε φύγει, καθώς εφύγετε απ' έμπροσθεν του σεισμού εν ταις ημέραις Οζίου του βασιλέως του Ιούδα· και Κύριος ο Θεός μου θέλει ελθεί και μετά σου πάντες οι άγιοι.
\par 6 Και εν εκείνη τη ημέρα το φως δεν θέλει είσθαι λαμπρόν ουδέ συνεσκοτασμένον·
\par 7 αλλά θέλει είσθαι μία ημέρα, ήτις είναι γνωστή εις τον Κύριον, ούτε ημέρα ούτε νύξ· και προς την εσπέραν θέλει είσθαι φως.
\par 8 Και εν τη ημέρα εκείνη ύδατα ζώντα θέλουσιν εξέλθει εξ Ιερουσαλήμ, το ήμισυ αυτών προς την θάλασσαν την ανατολικήν και το ήμισυ αυτών προς την θάλασσαν την δυτικήν· εν θέρει και εν χειμώνι θέλει είσθαι ούτω.
\par 9 Και ο Κύριος θέλει είσθαι βασιλεύς εφ' όλην την γήν· εν τη ημέρα εκείνη θέλει είσθαι Κύριος εις και το όνομα αυτού εν.
\par 10 Πας ο τόπος θέλει μεταβληθή εις πεδιάδα, από Γαβαά έως Ριμμών, κατά νότον της Ιερουσαλήμ· και αύτη θέλει υψωθή και κατοικηθή εν τω τόπω αυτής, από της πύλης του Βενιαμίν έως του τόπου της πρώτης πύλης, έως της πύλης των γωνιών και του πύργου Ανανεήλ, μέχρι των ληνών του βασιλέως.
\par 11 Και θέλουσι κατοικήσει εν αυτή, και δεν θέλει είσθαι πλέον αφανισμός· και η Ιερουσαλήμ θέλει κάθησθαι εν ασφαλεία.
\par 12 Και αύτη θέλει είσθαι η πληγή, με την οποίαν ο Κύριος θέλει πληγώσει πάντας τους λαούς τους στρατεύσαντας κατά της Ιερουσαλήμ· η σαρξ αυτών θέλει τήκεσθαι ενώ ίστανται επί τους πόδας αυτών, και οι οφθαλμοί αυτών θέλουσι διαλυθή εν ταις οπαίς αυτών, και η γλώσσα αυτών θέλει διαλυθή εν τω στόματι αυτών.
\par 13 Και εν τη ημέρα εκείνη ταραχή του Κυρίου μεγάλη θέλει είσθαι μεταξύ αυτών, και θέλουσι πιάνει έκαστος την χείρα του πλησίον αυτού και η χειρ αυτού θέλει εγείρεσθαι κατά της χειρός του πλησίον αυτού.
\par 14 Και ο Ιούδας έτι θέλει πολεμήσει εν Ιερουσαλήμ· και ο πλούτος πάντων των εθνών κύκλω θέλει συναχθή, χρυσίον και αργύριον και ιμάτια, εις πλήθος μέγα.
\par 15 Και η πληγή του ίππου, του ημιόνου, της καμήλου και του όνου και πάντων των κτηνών, τα οποία θέλουσιν είσθαι εν τοις στρατοπέδοις εκείνοις, τοιαύτη θέλει είσθαι ως η πληγή αύτη.
\par 16 Και πας όστις υπολειφθή εκ πάντων των εθνών, των ελθόντων κατά της Ιερουσαλήμ, θέλει αναβαίνει κατ' έτος διά να προσκυνή τον Βασιλέα· τον Κύριον των δυνάμεων, και να εορτάζη την εορτήν της σκηνοπηγίας.
\par 17 Και όσοι εκ των οικογενειών της γης δεν αναβώσιν εις Ιερουσαλήμ, διά να προσκυνήσωσι τον Βασιλέα, τον Κύριον των δυνάμεων, εις αυτούς δεν θέλει είσθαι βροχή.
\par 18 Και εάν η οικογένεια της Αιγύπτου δεν αναβή και δεν έλθη, επί τους οποίους δεν είναι βροχή, εις αυτούς θέλει είσθαι η πληγή, ην ο Κύριος θέλει πληγώσει τα έθνη τα μη αναβαίνοντα διά να εορτάσωσι την εορτήν της σκηνοπηγίας.
\par 19 Αύτη θέλει είσθαι η ποινή της Αιγύπτου και η ποινή πάντων των εθνών των μη θελόντων να αναβώσι διά να εορτάσωσι την εορτήν της σκηνοπηγίας.
\par 20 Εν τη ημέρα εκείνη θέλει είσθαι επί τους κώδωνας των ίππων, ΑΓΙΑΣΜΟΣ ΕΙΣ ΤΟΝ ΚΥΡΙΟΝ· και οι λέβητες εν τω οίκω του Κυρίου θέλουσιν είσθαι ως αι φιάλαι έμπροσθεν του θυσιαστηρίου.
\par 21 Και πας λέβης εν Ιερουσαλήμ και εν Ιούδα θέλει είσθαι αγιασμός εις τον Κύριον των δυνάμεων· και πάντες οι θυσιάζοντες θέλουσιν ελθεί και λάβει εξ αυτών και εψήσει εν αυτοίς· και εν τη ημέρα εκείνη δεν θέλει είσθαι πλέον Χαναναίος εν τω οίκω του Κυρίου των δυνάμεων.


\end{document}