\begin{document}

\title{Isaiah}


\chapter{1}

\par 1 Όρασις Ησαΐου υιού Αμώς, την οποίαν είδε περί του Ιούδα και της Ιερουσαλήμ, εν ταις ημέραις Οζίου Ιωάθαμ, Άχαζ και Εζεκίου, βασιλέων Ιούδα.
\par 2 Ακούσατε, ουρανοί, και ακροάσθητι, γή· διότι ο Κύριος ελάλησεν· Υιούς έθρεψα και ύψωσα, αλλ' αυτοί απεστάτησαν απ' εμού.
\par 3 Ο βους γνωρίζει τον κτήτορα αυτού και ο όνος την φάτνην του κυρίου αυτού ο Ισραήλ δεν γνωρίζει, ο λαός μου δεν εννοεί.
\par 4 Ουαί, έθνος αμαρτωλόν, λαέ πεφορτωμένε ανομίαν, σπέρμα κακοποιών υιοί διεφθαρμένοι εγκατέλιπον τον Κύριον, κατεφρόνησαν τον Άγιον του Ισραήλ, εστράφησαν εις τα οπίσω.
\par 5 Διά τι παιδευόμενοι θέλετε επιπροσθέτει στασιασμόν; όλη η κεφαλή είναι άρρωστος και όλη η καρδία κεχαυνωμένη·
\par 6 από ίχνους ποδός μέχρι κεφαλής δεν υπάρχει εν αυτώ ακεραιότης αλλά τραύματα και μελανίσματα και έλκη σεσηπότα δεν εξεπιέσθησαν ουδέ εδέθησαν ουδέ εμαλακώθησαν δι' αλοιφής
\par 7 η γη σας είναι έρημος, αι πόλεις σας πυρίκαυστοι την γην σας ξένοι κατατρώγουσιν έμπροσθέν σας· και είναι έρημος, ως πεπορθημένη υπό αλλοφύλων
\par 8 και η θυγάτηρ Σιών εγκαταλελειμμένη ως καλύβη εν αμπελώνι, ως οπωροφυλάκιον εν κήπω αγγουρίων ως πόλις πολιορκουμένη.
\par 9 Αν ο Κύριος των δυνάμεων δεν ήθελεν αφήσει εις ημάς μικρόν υπόλοιπον, ως τα Σόδομα ηθέλομεν γείνει, με τα Γόμορρα ηθέλομεν εξομοιωθή.
\par 10 Ακούσατε τον λόγον του Κυρίου, άρχοντες Σοδόμων ακροάσθητι τον νόμον του Θεού ημών, λαέ Γομόρρων.
\par 11 Τίνα χρείαν έχω του πλήθους των θυσιών σας; λέγει Κύριος· κεχορτασμένος είμαι από ολοκαυτωμάτων κριών και από πάχους των σιτευτών και δεν ευαρεστούμαι εις αίμα ταύρων ή αρνίων ή τράγων.
\par 12 Όταν έρχησθε να εμφανισθήτε ενώπιόν μου, τις εζήτησεν εκ των χειρών σας τούτο, να πατήτε τας αυλάς μου;
\par 13 Μη φέρετε πλέον, ματαίας προσφοράς το θυμίαμα είναι βδέλυγμα εις εμέ τας νεομηνίας και τα σάββατα, την συγκάλεσιν των συνάξεων, δεν δύναμαι να υποφέρω, ανομίαν και πανηγυρικήν σύναξιν.
\par 14 Τας νεομηνίας σας και τας διατεταγμένας εορτάς σας μισεί η ψυχή μου είναι φορτίον εις εμέ εβαρύνθην να υποφέρω.
\par 15 Και όταν εκτείνητε τας χείρας σας, θέλω κρύπτει τους οφθαλμούς μου από σας ναι, όταν πληθύνητε δεήσεις, δεν θέλω εισακούει αι χείρές σας είναι πλήρεις αιμάτων.
\par 16 Λούσθητε, καθαρίσθητε· αποβάλετε την κακίαν των πράξεών σας απ' έμπροσθεν των οφθαλμών μου παύσατε πράττοντες το κακόν,
\par 17 μάθετε να πράττητε το καλόν· εκζητήσατε κρίσιν, κάμετε ευθύτητα εις τον δεδυναστευμένον, κρίνατε τον ορφανόν, προστατεύσατε την δίκην της χήρας
\par 18 Έλθετε τώρα, και ας διαδικασθώμεν, λέγει Κύριος εάν αι αμαρτίαι σας ήναι ως το πορφυρούν, θέλουσι γείνει λευκαί ως χιών εάν ήναι ερυθραί ως κόκκινον, θέλουσι γείνει ως λευκόν μαλλίον.
\par 19 Εάν θέλητε και υπακούσητε, θέλετε φάγει τα αγαθά της γής·
\par 20 εάν όμως δεν θέλητε και αποστατήσητε, θέλετε καταφαγωθή υπό μαχαίρας διότι το στόμα του Κυρίου ελάλησε.
\par 21 Πως η πιστή πόλις κατεστάθη πόρνη, ήτο πλήρης κρίσεων η δικαιοσύνη κατώκει εν αυτή αλλά τώρα, φονείς.
\par 22 Ο άργυρός σου κατεστάθη σκωρία, ο οίνός σου συνεκεράσθη μεθ' ύδατος.
\par 23 Οι άρχοντές σου είναι απειθείς και σύντροφοι κλεπτών· πάντες αγαπώσι δώρα και κυνηγούσιν αντιπληρωμάς δεν κρίνουσι τον ορφανόν ουδέ έρχεται η δίκη της χήρας προς αυτούς.
\par 24 Διά τούτο λέγει ο Κύριος, ο Κύριος των δυνάμεων, ο Κραταιός του Ισραήλ, Ω, θέλω χορτασθή επί τους εναντίους μου και θέλω εκδικηθή κατά των εχθρών μου
\par 25 και θέλω στρέψει την χείρα μου επί σε και αποκαθαρίσει την σκωρίαν σου και αφαιρέσει όλον σου τον κασσίτερον.
\par 26 Και θέλω αποκαταστήσει τους κριτάς σου ως το πρότερον και τους συμβούλους σου ως το απ' αρχής μετά ταύτα θέλεις ονομασθή η πόλις της δικαιοσύνης, η πιστή πόλις.
\par 27 Η Σιών θέλει εξαγορασθή διά κρίσεως, και οι επιστρέψαντες αυτής διά δικαιοσύνης.
\par 28 Και οι παράνομοι και οι αμαρτωλοί ομού θέλουσι καταστραφή, και οι εγκαταλιπόντες τον Κύριον θέλουσι καταναλωθή.
\par 29 Διότι θέλετε καταισχυνθή διά τα άλση, τα οποία επεθυμήσατε, και θέλετε εντραπή διά τους κήπους, τους οποίους εξελέξατε.
\par 30 Επειδή θέλετε γείνει ως δρυς, της οποίας τα φύλλα μαραίνονται, και ως κήπος, όστις δεν έχει ύδωρ.
\par 31 Και ο ισχυρός θέλει είσθαι ως καλάμιον στυπίου, και το έργον αυτού ως σπινθήρ, και θέλουσι καυθή και τα δύο ομού, και δεν θέλει είσθαι ο σβύνων.

\chapter{2}

\par 1 Ο λόγος, ο γενόμενος δι' οράματος εις τον Ησαΐαν τον υιόν του Αμώς, περί του Ιούδα και της Ιερουσαλήμ.
\par 2 Εν ταις εσχάταις ημέραις το όρος του οίκου του Κυρίου θέλει στηριχθή επί της κορυφής των ορέων και υψωθή υπεράνω των βουνών και πάντα τα έθνη θέλουσι συρρέει εις αυτό,
\par 3 και πολλοί λαοί θέλουσιν υπάγει και ειπεί, Έλθετε και ας αναβώμεν εις το όρος του Κυρίου, εις τον οίκον του Θεού του Ιακώβ και θέλει διδάξει ημάς τας οδούς αυτού, και θέλομεν περιπατήσει εν ταις τρίβοις αυτού. Διότι εκ Σιών θέλει εξέλθει νόμος και λόγος Κυρίου εξ Ιερουσαλήμ.
\par 4 Και θέλει κρίνει αναμέσον των εθνών και θέλει ελέγξει πολλούς λαούς και θέλουσι σφυρηλατήσει τας μαχαίρας αυτών διά υνία και τας λόγχας αυτών διά δρέπανα δεν θέλει σηκώσει μάχαιραν έθνος εναντίον έθνους, ουδέ θέλουσι μάθει πλέον τον πόλεμον.
\par 5 Οίκος Ιακώβ, έλθετε και ας περιπατήσωμεν εν τω φωτί του Κυρίου.
\par 6 Βεβαίως συ εγκατέλιπες τον λαόν σου, τον οίκον Ιακώβ, διότι ενεπλήσθησαν της ανατολής και έγειναν μάντεις ως οι Φιλισταίοι, και συνηνώθησαν μετά των τέκνων των αλλοφύλων.
\par 7 Και η γη αυτών ενεπλήσθη αργυρίου και χρυσίου, και δεν είναι τέλος των θησαυρών αυτών ενεπλήσθη η γη αυτών και ίππων, και δεν είναι τέλος των αμαξών αυτών.
\par 8 Και η γη αυτών ενεπλήσθη από ειδώλων ελάτρευσαν το ποίημα των χειρών αυτών, εκείνο το οποίον οι δάκτυλοι αυτών έκαμον
\par 9 και ο κοινός άνθρωπος υπέκυψε και ο μεγάλος εταπεινώθη και δεν θέλεις συγχωρήσει αυτούς.
\par 10 Είσελθε εις τον βράχον και κρύφθητι εις το χώμα, διά τον φόβον του Κυρίου και διά την δόξαν της μεγαλειότητος αυτού.
\par 11 Οι υπερήφανοι οφθαλμοί του ανθρώπου θέλουσι ταπεινωθή, και η έπαρσις των ανθρώπων θέλει υποκύψει· μόνος δε ο Κύριος θέλει υψωθή εν εκείνη τη ημέρα.
\par 12 Διότι ημέρα Κυρίου των δυνάμεων θέλει επέλθει επί πάντα αλαζόνα και υπερήφανον και επί πάντα υψωμένον και θέλει ταπεινωθή
\par 13 και επί πάσας τας κέδρους του Λιβάνου τας υψηλάς και επηρμένας και επί πάσας τας δρυς της Βασάν,
\par 14 και επί πάντα τα υψηλά όρη και επί πάντα τα υψωμένα βουνά,
\par 15 και επί πάντα πύργον υψηλόν και επί παν τείχος περιπεφραγμένον,
\par 16 και επί πάντα τα πλοία της Θαρσείς και επί πάντα τα ηδονικά θεάματα.
\par 17 Και το ύψος του ανθρώπου θέλει υποκύψει, και η έπαρσις των ανθρώπων θέλει ταπεινωθή· μόνος δε ο Κύριος θέλει υψωθή εν εκείνη τη ημέρα.
\par 18 Και τα είδωλα θέλουσιν ολοκλήρως καταστραφή.
\par 19 Και αυτοί θέλουσιν εισέλθει εις τα σπήλαια των βράχων και εις τας τρύπας της γης, διά τον φόβον του Κυρίου και διά την δόξαν της μεγαλειότητος αυτού, όταν εγερθή διά να κλονίση την γην.
\par 20 Εν εκείνη τη ημέρα θέλει ρίψει ο άνθρωπος εις τους ασπάλακας και εις τας νυκτερίδας τα αργυρά αυτού είδωλα και τα χρυσά αυτού είδωλα, τα οποία έκαμεν εις εαυτόν διά να προσκυνή·
\par 21 διά να εισέλθωσιν εις τας σχισμάς των βράχων και εις τα σπήλαια των πετρών, διά τον φόβον του Κυρίου και διά την δόξαν της μεγαλειότητος αυτού, όταν εγερθή διά να κλονίση την γην.
\par 22 Παραιτήθητε από ανθρώπου, του οποίου η πνοή είναι εις τους μυκτήρας αυτού διότι εις τι είναι άξιος λόγου;

\chapter{3}

\par 1 Διότι ιδού, ο Κύριος, ο Κύριος των δυνάμεων, θέλει αφαιρέσει από της Ιερουσαλήμ και από του Ιούδα υποστήριγμα και βοήθειαν, άπαν το υποστήριγμα του άρτου και άπαν το υποστήριγμα του ύδατος,
\par 2 ισχυρόν και πολεμιστήν, κριτήν και προφήτην και συνετόν και πρεσβύτερον,
\par 3 πεντηκόνταρχον και έντιμον και σύμβουλον και σοφόν τεχνίτην και συνετόν γοητευτήν.
\par 4 Και θέλω δώσει παιδάρια άρχοντας αυτών, και νήπια θέλουσιν εξουσιάζει επ' αυτών.
\par 5 Και ο λαός θέλει καταδυναστεύεσθαι, άνθρωπος υπό ανθρώπου, και έκαστος υπό του πλησίον αυτού· το παιδίον θέλει αλαζονεύεσθαι προς τον γέροντα, και ο ποταπός προς τον έντιμον.
\par 6 Εάν τις πιάση τον αδελφόν αυτού εκ του οίκου του πατρός αυτού, λέγων, Ιμάτιον έχεις, γενού αρχηγός ημών, και ο αφανισμός ούτος ας ήναι υπό την χείρα σου.
\par 7 Εν εκείνη τη ημέρα θέλει ομόσει, λέγων, δεν θέλω γείνει θεραπευτής διότι εν τη οικία μου δεν είναι ούτε άρτος ούτε ιμάτιον· μη με κάμητε αρχηγόν του λαού
\par 8 διότι ηφανίσθη η Ιερουσαλήμ και έπεσεν ο Ιούδας, επειδή η γλώσσα αυτών και τα έργα αυτών ήναι εναντία εις τον Κύριον, παροξύνωσι τους οφθαλμούς της δόξης αυτού.
\par 9 Η ύψωσις του προσώπου αυτών μαρτυρεί εναντίον αυτών· και κηρύττουσι την αμαρτίαν αυτών ως τα Σόδομα· δεν κρύπτουσιν αυτήν. Ουαί εις την ψυχήν αυτών διότι ανταπέδωκαν εις εαυτούς κακά.
\par 10 Είπατε προς τον δίκαιον ότι καλόν θέλει είσθαι εις αυτόν· διότι θέλει φάγει τον καρπόν των έργων αυτού.
\par 11 Ουαί εις τον άνομον κακόν θέλει είσθαι εις αυτόν διότι η ανταπόδοσις των χειρών αυτού θέλει γείνει εις αυτόν.
\par 12 Τον λαόν μου, παιδάρια καταδυναστεύουσιν αυτόν, και γυναίκες εξουσιάζουσιν επ' αυτού. Λαέ μου, οι οδηγοί σου σε κάμνουσι να πλανάσαι και καταστρέφουσι την οδόν των βημάτων σου.
\par 13 Ο Κύριος εξεγείρεται διά να δικάση και ίσταται διά να κρίνη τους λαούς.
\par 14 Ο Κύριος θέλει εισέλθει εις κρίσιν μετά των πρεσβυτέρων του λαού αυτού και μετά των αρχόντων αυτού· διότι σεις κατεφάγετε τον αμπελώνα· τα αρπάγματα του πτωχού είναι εν ταις οικίαις υμών.
\par 15 Διά τι καταδυναστεύετε τον λαόν μου και καταθλίβετε τα πρόσωπα των πτωχών; λέγει Κύριος ο Θεός των δυνάμεων.
\par 16 Και λέγει Κύριος, Επειδή αι θυγατέρες της Σιών υπερηφανεύθησαν και περιπατούσι με υψωμένον τράχηλον και με όμματα άσεμνα, περιπατούσαι τρυφηλά και τρίζουσαι με τους πόδας αυτών,
\par 17 διά τούτο ο Κύριος θέλει φαλακρώσει την κορυφήν της κεφαλής των θυγατέρων της Σιών, και ο Κύριος θέλει εκκαλύψει την αισχύνην αυτών.
\par 18 Εν εκείνη τη ημέρα ο Κύριος θέλει αφαιρέσει την δόξαν των τριζόντων στολισμών και τα εμπλόκια και τους μηνίσκους,
\par 19 τα περιδέραια και τα βραχιόλια και τας καλύπτρας,
\par 20 τους κεκρυφάλους και τας περισκελίδας και τα κεφαλόδεσμα και τας μυροθήκας και τα ενώτια,
\par 21 τα δακτυλίδια και τα έρρινα,
\par 22 τας ποικίλας στολάς και τα επενδύματα και τα περικαλύμματα και τα θυλάκια,
\par 23 τα κάτοπτρα και τα λεπτά λινά και τας μίτρας και τα θέριστρα.
\par 24 Και αντί της γλυκείας οσμής θέλει είσθαι δυσωδία και αντί ζώνης σχοινίον και αντί καλλικομίας φαλάκρωμα και αντί επιστομαχίου περίζωμα σάκκινον ηλιόκαυμα αντί ώραιότητος.
\par 25 Οι άνδρες σου θέλουσι πέσει εν μαχαίρα και η δύναμίς σου εν πολέμω.
\par 26 Και αι πύλαι αυτής θέλουσι στενάξει και πενθήσει και αυτή θέλει κοίτεσθαι επί του εδάφους ηρημωμένη.

\chapter{4}

\par 1 Και εν εκείνη τη ημέρα επτά γυναίκες θέλουσι πιάσει ένα άνδρα λέγουσαι, θέλομεν τρώγει τον άρτον ημών και θέλομεν ενδύεσθαι τα ιμάτια ημών μόνον ας κράζεται το όνομά σου εφ ημάς, διά να αφαιρέσης το όνειδος ημών.
\par 2 Εν εκείνη τη ημέρα ο κλάδος του Κυρίου θέλει είσθαι ώραίος και ένδοξος και ο καρπός της γης εξαίρετος και ευφρόσυνος εις τους διασωθέντας εκ του Ισραήλ
\par 3 και ο υπόλοιπος εν Σιών και ο εναπολειφθείς εν Ιερουσαλήμ θέλει ονομασθή άγιος, πάντες οι γεγραμμένοι μεταξύ των ζώντων εν Ιερουσαλήμ,
\par 4 όταν εκπλύνη ο Κύριος την ακαθαρσίαν των θυγατέρων της Σιών και καθαρίση το αίμα της Ιερουσαλήμ εκ μέσου αυτής διά πνεύματος κρίσεως και διά πνεύματος καύσεως.
\par 5 Και ο Κύριος θέλει δημιουργήσει επί πάντα τόπον του όρους Σιών και επί τας συναθροίσεις αυτής νεφέλην και καπνόν την ημέραν, εν δε τη νυκτί λαμπρότητα φλογερού πυρός· διότι επί πάσαν την δόξαν θέλει είσθαι υπεράσπισις,
\par 6 και θέλει είσθαι σκηνή, διά να επισκιάζη από της καύσεως εν ημέρα, και διά να ήναι καταφύγιον και σκέπη από ανεμοζάλης και από βροχής.

\chapter{5}

\par 1 Τώρα θέλω ψάλλει εις τον ηγαπημένον μου άσμα του αγαπητού μου περί του αμπελώνος αυτού. Ο ηγαπημένος μου είχεν αμπελώνα επί λόφου παχυτάτου.
\par 2 Και περιέφραξεν αυτόν, και συνήθροισεν εξ αυτού τους λίθους και εφύτευσεν αυτόν με τα πλέον εκλεκτά κλήματα και έκτισε πύργον εν τω μέσω αυτού και κατεσκεύασεν έτι ληνόν εν αυτώ και περιέμενε να κάμη σταφύλια, αλλ' έκαμεν αγριοστάφυλα.
\par 3 Και τώρα, κάτοικοι Ιερουσαλήμ και άνδρες Ιούδα, κρίνατε, παρακαλώ, αναμέσον εμού και του αμπελώνός μου.
\par 4 Τι ήτο δυνατόν να κάμω έτι εις τον αμπελώνά μου και δεν έκαμον εις αυτόν; διά τι λοιπόν, ενώ περιέμενον να κάμη σταφύλια, έκαμεν αγριοστάφυλα;
\par 5 Τώρα λοιπόν θέλω σας αναγγείλει τι θέλω κάμει εγώ εις τον αμπελώνά μου· θέλω αφαιρέσει τον φραγμόν αυτού και θέλει καταφαγωθή θέλω χαλάσει τον τοίχον αυτού και θέλει καταπατηθή·
\par 6 και θέλω καταστήσει αυτόν έρημον δεν θέλει κλαδευθή ουδέ σκαφθή, αλλά θέλουσι βλαστήσει εκεί τρίβολοι και άκανθαι θέλω προστάξει έτι τα νέφη να μη βρέξωσι βροχήν επ' αυτόν.
\par 7 Αλλ' ο αμπελών του Κυρίου των δυνάμεων είναι ο οίκος του Ισραήλ και οι άνδρες Ιούδα το αγαπητόν αυτού φυτόν και περιέμενε κρίσιν, πλην ιδού, καταδυνάστευσις δικαιοσύνην, πλην ιδού, κραυγή.
\par 8 Ουαί εις εκείνους, οίτινες ενόνουσιν οικίαν με οικίαν και συνάπτουσιν αγρόν με αγρόν, εωσού μη μείνη τόπος, διά να κατοικώσι μόνοι εν τω μέσω της γης.
\par 9 Εις τα ώτα μου είπεν ο Κύριος των δυνάμεων, Βεβαίως πολλαί οικίαι θέλουσι μείνει ηρημωμέναι, μεγάλαι και καλαί, χωρίς κατοίκων
\par 10 ναι, δέκα στρέμματα αμπελώνος θέλουσι δώσει εν βαθ, και ο σπόρος ενός χομόρ θέλει δώσει εν εφά.
\par 11 Ουαί εις εκείνους, οίτινες εξεγειρόμενοι το πρωΐ ζητούσι σίκερα· οίτινες εξακολουθούσι μέχρι της εσπέρας, εωσού εξάψη ο οίνος αυτούς.
\par 12 Και η κιθάρα και η λύρα, το τύμπανον και ο αυλός και ο οίνος είναι εν τοις συμποσίοις αυτών αλλά δεν παρατηρούσι το έργον του Κυρίου και δεν θεωρούσι την ενέργειαν των χειρών αυτού.
\par 13 Διά τούτο ο λαός μου εφέρθη εις αιχμαλωσίαν, διότι δεν έχει επίγνωσιν και οι έντιμοι αυτών λιμοκτονούσι, και το πλήθος αυτών κατεξηράνθη υπό δίψης.
\par 14 Διά ταύτα επλάτυνεν ο άδης εαυτόν και διήνοιξεν υπέρμετρα το στόμα αυτού· και η δόξα αυτών και το πλήθος αυτών και ο θόρυβος αυτών και οι εντρυφώντες θέλουσι καταβή εις αυτόν.
\par 15 Και ο κοινός άνθρωπος θέλει υποκύψει, και ο δυνατός θέλει ταπεινωθή, και οι οφθαλμοί των υψηλών θέλουσι χαμηλωθή.
\par 16 Ο δε Κύριος των δυνάμεων θέλει υψωθή εις κρίσιν, και ο Θεός ο Άγιος θέλει αγιασθή εις δικαιοσύνην.
\par 17 Τότε τα αρνία θέλουσι βοσκηθή κατά την συνήθειαν αυτών, και ξένοι θέλουσι φάγει τους ερήμους τόπους των παχέων.
\par 18 Ουαί εις εκείνους, οίτινες επισύρουσι την ανομίαν διά σχοινίων ματαιότητος και την αμαρτίαν ως διά λωρίων αμάξης
\par 19 οίτινες λέγουσιν, Ας σπεύση, ας επιταχύνη το έργον αυτού διά να ίδωμεν και η βουλή του Αγίου του Ισραήλ ας πλησιάση και ας έλθη, διά να μάθωμεν.
\par 20 Ουαί εις εκείνους, οίτινες λέγουσι το κακόν καλόν και το καλόν κακόν οίτινες θέτουσι το σκότος διά φως και το φως διά σκότος οίτινες θέτουσι το πικρόν διά γλυκύ και το γλυκύ διά πικρόν.
\par 21 Ουαί εις τους όσοι είναι σοφοί εις τους οφθαλμούς αυτών και φρόνιμοι ενώπιον εαυτών.
\par 22 Ουαί εις τους όσοι είναι δυνατοί εις το να πίνωσιν οίνον και ισχυροί εις το να σμίγωσι σίκερα
\par 23 οίτινες δικαιόνουσι τον παράνομον διά δώρα, και το δίκαιον του δικαίου αφαιρούσιν απ' αυτού.
\par 24 Διά τούτο, ως η γλώσσα του πυρός κατατρώγει την καλάμην και το άχυρον αφανίζεται εν τη φλογί, ούτως η ρίζα αυτών θέλει κατασταθή ως σηπεδών, και το άνθος αυτών θέλει αναβή ως κονιορτός διότι απέρριψαν τον νόμον του Κυρίου των δυνάμεων και κατεφρόνησαν τον λόγον του Αγίου του Ισραήλ.
\par 25 Διά τούτο ο θυμός του Κυρίου εξήφθη εναντίον του λαού αυτού, και εκτείνας την χείρα αυτού εναντίον αυτών επάταξεν αυτούς· τα δε όρη έτρεμον, και τα πτώματα αυτών έγειναν ως κοπρία εν μέσω των οδών. Εν πάσι τούτοις ο θυμός αυτού δεν απεστράφη, αλλ' η χειρ αυτού είναι έτι εξηπλωμένη.
\par 26 Και θέλει υψώσει εις τα έθνη σημείον από μακράν, και θέλει συρίξει εις αυτά απ' άκρου της γης και ιδού, ταχέως θέλουσιν ελθεί μετά σπουδής·
\par 27 ουδείς θέλει αποκάμει ουδέ προσκρούσει μεταξύ αυτών ουδείς θέλει νυστάξει ουδέ κοιμηθή ουδέ η ζώνη της οσφύος αυτών θέλει λυθή, ουδέ το λωρίον των υποδημάτων αυτών θέλει κοπή
\par 28 των οποίων τα βέλη είναι οξέα και πάντα τα τόξα αυτών εντεταμένα οι όνυχες των ίππων αυτών θέλουσι νομισθή ως πυροβόλος πέτρα, και οι τροχοί των αμαξών αυτών ως ανεμοστρόβιλος
\par 29 τα βρυχήματα αυτών θέλουσιν είσθαι ως λέοντος θέλουσι βρυχάσθαι ως σκύμνοι λέοντος· ναι, θέλουσι βρυχάσθαι και θέλουσι συναρπάσει το θήραμα και φύγει και ουδείς ο ελευθερών.
\par 30 Και όταν κατ' εκείνην την ημέραν βοήσωσιν εναντίον αυτών ως βοή της θαλάσσης, θέλουσιν εμβλέψει εις την γην και ιδού, σκότος, λύπη, και το φως εσκοτίσθη εν τω ουρανώ αυτής.

\chapter{6}

\par 1 Κατά το έτος εν ω απέθανεν Οζίας ο βασιλεύς, είδον τον Κύριον καθήμενον επί θρόνου υψηλού και επηρμένου, και το κράσπεδον αυτού εγέμισε τον ναόν.
\par 2 Άνωθεν αυτού ίσταντο Σεραφείμ ανά εξ πτέρυγας έχοντα έκαστον με τας δύο εκάλυπτε το πρόσωπον αυτού και με τας δύο εκάλυπτε τους πόδας αυτού και με τας δύο επέτα.
\par 3 Και έκραζε το εν προς το άλλο και έλεγεν, Άγιος, άγιος, άγιος ο Κύριος των δυνάμεων πάσα η γη είναι πλήρης της δόξης αυτού.
\par 4 Και οι παραστάται της θύρας εσείσθησαν εκ της φωνής του κράζοντος, και ο οίκος επλήσθη καπνού.
\par 5 Τότε είπα, Ω τάλας εγώ διότι εχάθην επειδή είμαι άνθρωπος ακαθάρτων χειλέων και κατοικώ εν μέσω λαού ακαθάρτων χειλέων επειδή οι οφθαλμοί μου είδον τον Βασιλέα, τον Κύριον των δυνάμεων.
\par 6 Τότε επέτασε προς εμέ εν εκ των Σεραφείμ έχον εν τη χειρί αυτού άνθρακα πυρός, τον οποίον έλαβε διά της λαβίδος από του θυσιαστηρίου.
\par 7 Και ήγγισεν αυτόν εις το στόμα μου και είπεν, Ιδού, τούτο ήγγισε τα χείλη σου και η ανομία σου εξηλείφθη και η αμαρτία σου εκαθαρίσθη.
\par 8 Και ήκουσα την φωνήν του Κυρίου, λέγοντος, Τίνα θέλω αποστείλει, και τις θέλει υπάγει διά ημάς; Τότε είπα, Ιδού, εγώ, απόστειλόν με.
\par 9 Και είπεν, Ύπαγε και ειπέ προς τούτον τον λαόν, με την ακοήν θέλετε ακούσει και δεν θέλετε εννοήσει και βλέποντες θέλετε ιδεί και δεν θέλετε καταλάβει
\par 10 επαχύνθη η καρδία του λαού τούτου, και έγειναν βαρέα τα ώτα αυτών, και έκλεισαν τους οφθαλμούς αυτών, διά να μη βλέπωσι με τους οφθαλμούς αυτών και ακούωσι με τα ώτα αυτών και νοήσωσι με την καρδίαν αυτών και επιστρέψωσι και θεραπευθώσι.
\par 11 Τότε είπα, Κύριε, έως πότε; Και απεκρίθη, Εωσού ερημωθώσιν αι πόλεις, ώστε να μη υπάρχη κάτοικος, και αι οικίαι, ώστε να μη υπάρχη άνθρωπος, και η γη να ερημωθή παντάπασιν
\par 12 και απομακρύνη ο Κύριος τους ανθρώπους, και γείνη μεγάλη εγκατάλειψις εν τω μέσω της γης.
\par 13 Έτι όμως θέλει μείνει εν αυτή εν δέκατον, και αυτό πάλιν θέλει καταφαγωθή καθώς η τερέβινθος και η δρυς, των οποίων ο κορμός μένει εν αυταίς όταν κόπτωνται, ούτω το άγιον σπέρμα θέλει είσθαι ο κορμός αυτής.

\chapter{7}

\par 1 Και εν ταις ημέραις του Άχαζ, υιού του Ιωάθαμ, υιού του Οζίου, βασιλέως του Ιούδα, Ρεσίν ο βασιλεύς της Συρίας, και Φεκά ο υιός του Ρεμαλία, βασιλεύς του Ισραήλ, ανέβησαν επί την Ιερουσαλήμ διά να πολεμήσωσιν αυτήν αλλά δεν ηδυνήθησαν να εκπολιορκήσωσιν αυτήν.
\par 2 Και ανήγγειλαν προς τον οίκον Δαβίδ λέγοντες, Η Συρία συνεφώνησε μετά του Εφραΐμ. Και η καρδία του Άχαζ και η καρδία του λαού αυτού εκλονίσθη, ως τα δένδρα του δάσους κλονίζονται υπό του ανέμου.
\par 3 Τότε είπεν ο Κύριος προς τον Ησαΐαν, Έξελθε τώρα εις συνάντησιν του Άχαζ, συ και Σεάρ-ιασούβ ο υιός σου, εις το άκρον του υδραγωγού της άνω κολυμβήθρας κατά την μεγάλην οδόν του αγρού του γναφέως
\par 4 και ειπέ προς αυτόν, Πρόσεχε να μένης ήσυχος μη φοβηθής μηδέ μικροψυχήσης, διά τας δύο ουράς των καπνιζόντων τούτων δαυλών, διά τον άγριον θυμόν του Ρεσίν και της Συρίας, και του υιού του Ρεμαλία.
\par 5 Επειδή η Συρία, ο Εφραΐμ και ο υιός του Ρεμαλία εβουλεύθησαν κακήν βουλήν εναντίον σου, λέγοντες,
\par 6 Ας αναβώμεν εναντίον του Ιούδα και ας στενοχωρήσωμεν αυτόν, και ας διαμερισθώμεν αυτόν εις εαυτούς, και ας βάλωμεν βασιλέα εν μέσω αυτού, τον υιόν του Ταβεήλ·
\par 7 ούτω λέγει Κύριος ο Θεός· Τούτο δεν θέλει σταθή ουδέ θέλει γείνει.
\par 8 Διότι η κεφαλή της Συρίας είναι η Δαμασκός, και η κεφαλή της Δαμασκού ο Ρεσίν και εις εξήκοντα πέντε έτη ο Εφραΐμ θέλει συντριφθή, ώστε να μη ήναι λαός.
\par 9 Και η κεφαλή του Εφραΐμ είναι η Σαμάρεια, και η κεφαλή της Σαμαρείας ο υιός του Ρεμαλία. Εάν δεν πιστεύητε, δεν θέλετε βεβαίως στερεωθή.
\par 10 Και ελάλησεν έτι ο Κύριος προς τον Άχαζ, λέγων,
\par 11 Ζήτησον σημείον παρά Κυρίου του Θεού σου ζήτησον αυτό ή εις το βάθος ή εις το ύψος άνω.
\par 12 Αλλ' ο Άχαζ είπε, δεν θέλω ζητήσει ουδέ θέλω πειράσει τον Κύριον.
\par 13 Και είπεν ο Ησαΐας, Ακούσατε τώρα, οίκος Δαβίδ μικρόν πράγμα είναι διά σας να βαρύνητε ανθρώπους, και θέλετε βαρύνει έτι και τον Θεόν μου;
\par 14 Διά τούτο ο Κύριος αυτός θέλει σας δώσει σημείον ιδού, η παρθένος θέλει συλλάβει και γεννήσει υιόν, και θέλει καλεσθή το όνομα αυτού Εμμανουήλ.
\par 15 Βούτυρον και μέλι θέλει φάγει, εωσού μάθη να απορρίπτη το κακόν και να εκλέγη το αγαθόν.
\par 16 Διότι πριν μάθη το παιδίον να απορρίπτη το κακόν και να εκλέγη το αγαθόν, η γη, την οποίαν αποστρέφεσαι, θέλει εγκαταλειφθή υπό των δύο βασιλέων αυτής.
\par 17 Ο Κύριος θέλει φέρει επί σε, και επί τον λαόν σου, και επί τον οίκον του πατρός σου, ημέρας, αίτινες δεν ήλθον αφ' ης ημέρας εχωρίσθη από του Ιούδα ο Εφραΐμ, διά του βασιλέως της Ασσυρίας.
\par 18 Και εν εκείνη τη ημέρα θέλει συρίξει ο Κύριος εις τας μυίας τας εν τοις εσχάτοις των ποταμών της Αιγύπτου, και εις τας μελίσσας τας εν τη γη της Ασσυρίας
\par 19 και θέλουσιν ελθεί και αναπαυθή πάσαι επί τας ηρημωμένας κοιλάδας και εν ταις τρύπαις των βράχων και επί πάσαν βάτον και επί παν ώραίον δένδρον.
\par 20 Εν τη αυτή ημέρα ο Κύριος θέλει ξυρίσει, με το ξυράφιον το μεμισθωμένον από του πέραν του ποταμού, μετά του βασιλέως της Ασσυρίας, την κεφαλήν και τας τρίχας των ποδών και τον πώγωνα έτι θέλει αφαιρέσει.
\par 21 Και εν εκείνη τη ημέρα άνθρωπος τρέφων μίαν δάμαλιν και δύο πρόβατα,
\par 22 από της αφθονίας του γάλακτος, το οποίον θέλουσι δίδει, βούτυρον θέλει τρώγει διότι βούτυρον και μέλι θέλει τρώγει έκαστος, όστις υπελείφθη εν τω μέσω της γης.
\par 23 Και εν εκείνη τη ημέρα πας τόπος, εν ω ήσαν χίλιαι άμπελοι χιλίων αργυρίων, θέλει είσθαι διά τριβόλους και ακάνθας.
\par 24 Με βέλη και με τόξα θέλουσιν ελθεί εκεί· διότι πάσα η γη θέλει κατασταθή τρίβολοι και άκανθαι.
\par 25 Και παν όρος γεγεωργημένον με δίκελλαν, όπου δεν ήλθε φόβος τριβόλων και ακανθών, θέλει είσθαι διά να εξαποστέλλωνται εκεί βόες και διά να καταπατήται υπό προβάτων.

\chapter{8}

\par 1 Και είπε Κύριος προς εμέ, Λάβε εις σεαυτόν τόμον μέγαν, και γράψον εν αυτώ διά γραφίδος ανθρώπου περί του Μαχέρ-σαλάλ-χας-βάζ·
\par 2 Και παρέλαβον εις εμαυτόν πιστούς μάρτυρας, Ουρίαν τον ιερέα και Ζαχαρίαν τον υιόν του Ιεβερεχίου.
\par 3 Και προσήλθον προς την προφήτισσαν, ήτις συνέλαβε και εγέννησεν υιόν. Και είπε Κύριος προς εμέ, Κάλεσον το όνομα αυτού Μαχέρ-σαλάλ-χας-βαζ
\par 4 διότι πριν μάθη το παιδίον να προφέρη, Πάτερ μου και μήτέρ μου, τα πλούτη της Δαμασκού και τα λάφυρα της Σαμαρείας θέλουσι διαρπαχθή έμπροσθεν του βασιλέως της Ασσυρίας.
\par 5 Και ελάλησεν έτι Κύριος προς εμέ, λέγων,
\par 6 Επειδή ο λαός ούτος απέβαλε τα ύδατα του Σιλωάμ τα ρέοντα ησύχως, και χαίρει εις τον Ρεσίν και εις τον υιόν του Ρεμαλία,
\par 7 διά τούτο, ιδού, ο Κύριος αναβιβάζει επ' αυτούς τα ύδατα του ποταμού, τα δυνατά και τα πολλά, τον βασιλέα της Ασσυρίας και πάσαν την δόξαν αυτού· και θέλει υπερβή πάντα τα αυλάκια αυτού και πλημμυρήσει πάσας τας όχθας αυτού
\par 8 και θέλει περάσει διά του Ιούδα, θέλει πλημμυρήσει και υπεραναβή, θέλει φθάσει μέχρι λαιμού και το εξάπλωμα των πτερύγων αυτού θέλει γεμίσει το πλάτος της γης σου, Εμμανουήλ.
\par 9 Ενώθητε, λαοί, και θέλετε κατακοπή· και ακροάσθητε, πάντες οι εν τοις εσχάτοις της γής· ζώσθητε, και θέλετε κατακοπή· ζώσθητε, και θέλετε κατακοπή.
\par 10 Βουλεύθητε βουλήν, και θέλει ματαιωθή· λαλήσατε λόγον, και δεν θέλει σταθή διότι μεθ' ημών ο Θεός.
\par 11 Διότι ούτως ελάλησε Κύριος προς εμέ εν χειρί κραταιά, και με εδίδαξε να μη περιπατώ εν τη οδώ του λαού τούτου, λέγων,
\par 12 Μη είπητε, Συνωμοσία, περί παντός εκείνου, περί του οποίου ο λαός ούτος θέλει ειπεί, Συνωμοσία και τον φόβον αυτού μη φοβηθήτε μηδέ τρομάξητε.
\par 13 Τον Κύριον των δυνάμεων, αυτόν αγιάσατε και αυτός ας ήναι ο φόβος σας και αυτός ας ήναι ο τρόμος σας.
\par 14 Και θέλει είσθαι διά αγιαστήριον θέλει είσθαι όμως διά πέτραν προσκόμματος και διά βράχον πτώσεως εις τους δύο οίκους του Ισραήλ διά παγίδα και διά βρόχον εις τους κατοίκους της Ιερουσαλήμ.
\par 15 Και πολλοί θέλουσι προσκόψει επ' αυτά και πέσει και συντριφθή και παγιδευθή και πιασθή.
\par 16 Δέσον την μαρτυρίαν, σφράγισον τον νόμον μεταξύ των μαθητών μου.
\par 17 Εγώ δε θέλω περιμείνει τον Κύριον, όστις κρύπτει το πρόσωπον αυτού από του οίκου Ιακώβ, και επ' αυτόν θέλω είσθαι πεποιθώς.
\par 18 Ιδού, εγώ και τα παιδία, τα οποία μοι έδωκεν ο Κύριος, διά σημεία και τεράστια εις τον Ισραήλ παρά του Κυρίου των δυνάμεων, του κατοικούντος εν τω όρει Σιών.
\par 19 Και όταν σας είπωσιν, Ερωτήσατε τους έχοντας πνεύμα μαντείας και τους νεκρομάντεις, τους μορμυρίζοντας και ψιθυρίζοντας, αποκρίθητε, Ο λαός δεν θέλει ερωτήσει τον Θεόν αυτού; θέλει προστρέξει εις τους νεκρούς περί των ζώντων;
\par 20 Εις τον νόμον και εις την μαρτυρίαν εάν δεν λαλώσι κατά τον λόγον τούτον, βεβαίως δεν είναι φως εν αυτοίς.
\par 21 Και θέλουσι περάσει δι' αυτής της γης σκληρώς βεβαρημένοι και λιμώττοντες και όταν πεινάσωσι θέλουσιν αγανακτεί, και θέλουσι κακολογεί τον βασιλέα αυτών και τον Θεόν αυτών, και θέλουσιν αναβλέψει εις τα άνω.
\par 22 Έπειτα θέλουσιν εμβλέψει εις την γην και ιδού, ταραχή και σκότος, θάμβωμα αγωνίας και θέλουσιν εξωσθή εις το σκότος.

\chapter{9}

\par 1 Δεν θέλει είσθαι όμως τοιούτον θάμβωμα εις την γην την τεθλιμμένην· εν τοις προτέροις καιροίς εξουθένωσε την γην Ζαβουλών και την γην Νεφθαλείμ· εν δε τοις υστέροις έκαμεν ένδοξα τα μέρη τα προς την οδόν της θαλάσσης, πέραν του Ιορδάνου, την Γαλιλαίαν των εθνών.
\par 2 Ο λαός ο περιπατών εν τω σκότει είδε φως μέγα· εις τους καθημένους εν γη σκιάς θανάτου, φως έλαμψεν επ' αυτούς.
\par 3 Επολλαπλασίασας το έθνος, ηύξησας εις αυτήν την χαράν· χαίρουσιν έμπροσθέν σου κατά την χαράν του θερισμού, καθώς αγάλλονται οι διαμεριζόμενοι τα λάφυρα.
\par 4 Διότι συ συνέτριψας τον ζυγόν του φορτίου αυτού και την ράβδον του ώμου αυτού και την μάστιγα του καταδυναστεύοντος αυτόν, καθώς εν τη ημέρα του Μαδιάμ.
\par 5 Διότι πάσα περικνημίς πολεμιστού μαχομένου μετά θορύβου και πάσα στολή κεκυλισμένη εις αίματα θέλει είσθαι διά καύσιν και ύλην πυρός.
\par 6 Διότι παιδίον εγεννήθη εις ημάς, υιός εδόθη εις ημάς· και η εξουσία θέλει είσθαι επί τον ώμον αυτού· και το όνομα αυτού θέλει καλεσθή Θαυμαστός, Σύμβουλος, Θεός ισχυρός, Πατήρ του μέλλοντος αιώνος, Άρχων ειρήνης.
\par 7 Εις την αύξησιν της εξουσίας αυτού και της ειρήνης δεν θέλει είσθαι τέλος, επί τον θρόνον του Δαβίδ και επί την βασιλείαν αυτού, διά να διατάξη αυτήν και να στερεώση αυτήν εν κρίσει και δικαιοσύνη από του νυν και έως αιώνος. Ο ζήλος του Κυρίου των δυνάμεων θέλει εκτελέσει τούτο.
\par 8 Ο Κύριος απέστειλε λόγον κατά του Ιακώβ και έπεσεν επί τον Ισραήλ.
\par 9 Και πας ο λαός θέλει γνωρίσει τούτο, ο Εφραΐμ και ο κάτοικος της Σαμαρείας, οίτινες λέγουσιν υπερηφάνως και με έπαρσιν καρδίας,
\par 10 οι πλίνθοι έπεσον, πλην ημείς θέλομεν κτίσει με πελεκητάς πέτρας· αι συκομορέαι εκόπησαν, πλην ημείς θέλομεν αλλάξει αυτάς εις κέδρους.
\par 11 Διά τούτο ο Κύριος θέλει εξεγείρει τους εχθρούς του Ρεσίν εναντίον αυτού και συνενώσει τους πολεμίους αυτού·
\par 12 τους Συρίους έμπροσθεν και τους Φιλισταίους όπισθεν· και θέλουσι καταφάγει τον Ισραήλ με ανοικτόν στόμα. Εν πάσι τούτοις ο θυμός αυτού δεν απεστράφη, αλλ' η χειρ αυτού είναι έτι εξηπλωμένη.
\par 13 Πλην ο λαός δεν επιστρέφει προς τον πατάξαντα αυτόν, ουδέ εκζητούσι τον Κύριον των δυνάμεων.
\par 14 Διά τούτο ο Κύριος θέλει εκκόψει από του Ισραήλ κεφαλήν και ουράν, κλάδον και σπάρτον, εν μιά ημέρα.
\par 15 Ο πρεσβύτης και ο έντιμος, αυτός είναι η κεφαλή· και ο προφήτης όστις διδάσκει ψεύδη, αυτός είναι η ουρά.
\par 16 Διότι οι μακαρίζοντες τον λαόν τούτον πλανώσιν αυτόν· και οι μακαριζόμενοι υπ' αυτών αφανίζονται.
\par 17 Διά τούτο ο Κύριος δεν θέλει ευφρανθή εις τους νεανίσκους αυτών, ουδέ θέλει ελεήσει τους ορφανούς και τας χήρας αυτών· επειδή πάντες είναι υποκριταί και κακοποιοί, και παν στόμα λαλεί ασεβώς. Εν πάσι τούτοις ο θυμός αυτού δεν απεστράφη, αλλ' η χειρ αυτού είναι έτι εξηπλωμένη.
\par 18 Διότι η ανομία αφανίζει ως το πυρ, το κατατρώγον τους τριβόλους και τας ακάνθας και το φλέγον εν τοις πυκνοτάτοις του δάσους· και αυτά θέλουσιν αναβή εις στήλην περιτυλισσομένου καπνού.
\par 19 Από του θυμού του Κυρίου των δυνάμεων η γη εσκοτίσθη, και ο λαός θέλει είσθαι ως ύλη πυρός· άνθρωπος δεν θέλει ελεήσει τον αδελφόν αυτού.
\par 20 Και θέλει αρπάσει εις τα δεξιά, πλην θέλει πεινάσει· και θέλει φάγει εις τα αριστερά, πλην δεν θέλει χορτασθή· θέλουσι φάγει πας άνθρωπος την σάρκα του βραχίονος αυτού·
\par 21 ο Μανασσής τον Εφραΐμ και ο Εφραΐμ τον Μανασσήν· και αυτοί ομού θέλουσιν είσθαι εναντίον του Ιούδα. Εν πάσι τούτοις ο θυμός αυτού δεν απεστράφη, αλλ' η χειρ αυτού είναι έτι εξηπλωμένη.

\chapter{10}

\par 1 Ουαί εις τους ψηφίζοντας ψηφίσματα άδικα και εις τους γραμματείς τους γράφοντας καταδυνάστευσιν·
\par 2 διά να στερήσωσι τον ενδεή από της κρίσεως, και διά να αρπάσωσι το δίκαιον των πτωχών του λαού μου, διά να γείνωσι λάφυρον αυτών αι χήραι, και να γυμνώσωσι τους ορφανούς.
\par 3 Και τι θέλετε κάμει εν τη ημέρα της επισκέψεως και εν τω ολέθρω όστις θέλει ελθεί μακρόθεν; προς τίνα θέλετε προστρέξει διά βοήθειαν; και που θέλετε αφήσει την δόξαν σας,
\par 4 ειμή ότι θέλουσιν υποκύψει εις τα δεσμά, και θέλουσι πέσει υποκάτω των πεφονευμένων; Εν πάσι τούτοις ο θυμός αυτού δεν απεστράφη, αλλ' η χειρ αυτού είναι έτι εξηπλωμένη.
\par 5 Ουαί εις τον Ασσύριον, την ράβδον του θυμού μου, αν και η εν τη χειρί αυτού μάστιξ ήναι η οργή μου.
\par 6 Θέλω αποστείλει αυτόν επί έθνος υποκριτικόν, και θέλω δώσει εις αυτόν προσταγήν κατά του λαού του θυμού μου, διά να λαφυραγωγήση λάφυρα και να λεηλατήση λεηλασίαν και να καταπατήση αυτούς ως τον πηλόν των οδών.
\par 7 Πλην αυτός δεν εννοεί ούτως και η καρδία αυτού δεν διαλογίζεται ούτως· αλλά τούτο φρονεί εν τη καρδία αυτού, να καταστρέψη και να εξολοθρεύση έθνη ουκ ολίγα.
\par 8 Διότι λέγει, οι άρχοντές μου δεν είναι πάντες βασιλείς;
\par 9 δεν είναι η Χαλάνη ως η Χαρχεμίς; δεν είναι η Αιμάθ ως η Αρφάδ; δεν είναι η Σαμάρεια ως η Δαμασκός;
\par 10 Καθώς η χειρ μου κατεκράτησε τα βασίλεια των ειδώλων, των οποίων τα γλυπτά ίσχυον μάλλον παρά τα της Ιερουσαλήμ και της Σαμαρείας,
\par 11 δεν θέλω κάμει, ως έκαμον εις την Σαμάρειαν και εις τα είδωλα αυτής, ούτω και εις την Ιερουσαλήμ και εις τα είδωλα αυτής;
\par 12 Διά τούτο, αφού ο Κύριος εκτελέση άπαν το έργον αυτού επί το όρος Σιών και επί την Ιερουσαλήμ, θέλω παιδεύσει, λέγει, τον καρπόν της επηρμένης καρδίας του βασιλέως της Ασσυρίας και την αλαζονείαν των υψηλών οφθαλμών αυτού.
\par 13 Διότι λέγει, Εν τη δυνάμει της χειρός μου έκαμον τούτο και διά της σοφίας μου, επειδή είμαι συνετός· και μετέστησα τα όρια των λαών και διήρπασα τους θησαυρούς αυτών και καθήρεσα, ως ισχυρός, τους εν ύψει καθημένους·
\par 14 και η χειρ μου εύρηκεν ως φωλεάν τα πλούτη των λαών· και καθώς συνάγει τις ωά αφειμένα, ούτω συνήγαγον πάσαν την γην εγώ· και ουδείς εκίνησε πτέρυγα ή ήνοιξε στόμα ή εψιθύρισεν.
\par 15 Ήθελε καυχηθή η αξίνη κατά του κόπτοντος δι' αυτής; ήθελε μεγαλαυχήσει το πριόνιον κατά του κινούντος αυτό; ως εάν ήθελε κινηθή η ράβδος κατά των υψούντων αυτήν, ως εάν ήθελεν υψώσει εαυτήν η βακτηρία ως μη ούσα ξύλον.
\par 16 Διά τούτο ο Κύριος, ο Κύριος των δυνάμεων, θέλει αποστείλει εις τους παχείς αυτού ισχνότητα· και υπό την δόξαν αυτού θέλει εξαφθή καύσις, ως καύσις πυρός.
\par 17 Και το φως του Ισραήλ θέλει γείνει πυρ και ο Άγιος αυτού φλόξ· και θέλει καύσει και καταφάγει τας ακάνθας αυτού και τους τριβόλους αυτού εν μιά ημέρα·
\par 18 και θέλει αφανίσει την δόξαν του δάσους αυτού και του καρποφόρου αγρού αυτού από ψυχής έως σαρκός· και θέλουσιν είσθαι ως όταν σημαιοφόρος λιποψυχή.
\par 19 Το δε υπόλοιπον των δένδρων του δάσους αυτού θέλει είσθαι ευάριθμον, ώστε παιδίον να καταγράψη αυτά.
\par 20 Και εν εκείνη τη ημέρα, το υπόλοιπον του Ισραήλ και οι διασεσωσμένοι του οίκου Ιακώβ δεν θέλουσι πλέον επιστηρίζεσθαι επί τον πατάξαντα αυτούς, αλλά θέλουσιν επιστηρίζεσθαι επί Κύριον, τον Άγιον του Ισραήλ, κατά αλήθειαν.
\par 21 Το υπόλοιπον θέλει επιστρέψει, το υπόλοιπον του Ιακώβ, προς τον ισχυρόν Θεόν.
\par 22 Διότι αν και ο λαός σου, Ισραήλ, ήναι ως η άμμος της θαλάσσης, υπόλοιπον εξ αυτών θέλει επιστρέψει· η αποφασισθείσα κατανάλωσις θέλει συντελεσθή εν δικαιοσύνη.
\par 23 Διότι Κύριος ο Θεός των δυνάμεων θέλει κάμει κατανάλωσιν, βεβαίως προσδιωρισμένην, εν μέσω πάσης της γής·
\par 24 διά τούτο, ούτω λέγει Κύριος ο Θεός των δυνάμεων. Λαέ μου, όστις κατοικείς εν Σιών, μη φοβηθής από του Ασσυρίου· θέλει σε πατάξει εν ράβδω και θέλει σηκώσει την βακτηρίαν αυτού εναντίον σου κατά τον τρόπον της Αιγύπτου·
\par 25 διότι έτι ολίγον και η οργή θέλει παύσει· και ο θυμός μου θέλει είσθαι εις όλεθρον εκείνων.
\par 26 Και ο Κύριος των δυνάμεων θέλει σηκώσει επ' αυτόν μάστιγα, κατά την πληγήν της Μαδιάμ εν τω βράχω Ωρήβ· και καθώς η ράβδος αυτού υψώθη επί την θάλασσαν, ούτω θέλει υψώσει αυτήν κατά τον τρόπον της Αιγύπτου.
\par 27 Και εν εκείνη τη ημέρα το φορτίον αυτού θέλει αφαιρεθή από του ώμου σου και ο ζυγός αυτού από του τραχήλου σου, και ο ζυγός θέλει συντριφθή εξ αιτίας του χρίσματος.
\par 28 Αυτός ήλθεν εις Αϊάθ, επέρασεν εις Μιγρών. Εν Μιχμάς θέλει αποθέσει τα σκεύη αυτού·
\par 29 διέβησαν το πέρασμα· κατέλυσαν εν Γεβά· η Ραμά ετρόμαξεν· η Γαβαά του Σαούλ έφυγεν.
\par 30 Ύψωσον την φωνήν σου, θυγάτηρ της Γαλλείμ· κάμε αυτήν, πτωχή Αναθώθ, να ακουσθή εν Λαισά.
\par 31 Η Μαδμηνά μετετοπίσθη· οι κάτοικοι της Γεβείμ έφυγον ομού.
\par 32 Και εν τη ημέρα εκείνη θέλει μείνει εν Νωβ, θέλει σείσει την χείρα αυτού κατά του όρους της θυγατρός της Σιών, κατά του λόφου της Ιερουσαλήμ.
\par 33 Ιδού, ο Κύριος, ο Κύριος των δυνάμεων, θέλει κόψει τους κλάδους μετά κρότου τρομερού· και οι υψωμένοι θέλουσι συντριφθή και οι επηρμένοι θέλουσι ταπεινωθή.
\par 34 Και τα πυκνά τού δάσους θέλει κόψει εν σιδήρω, και ο Λίβανος θέλει πέσει δι' ισχυρού.

\chapter{11}

\par 1 Και θέλει εξέλθει ράβδος εκ του κορμού του Ιεσσαί, και κλάδος θέλει αναβή εκ των ριζών αυτού·
\par 2 και το πνεύμα του Κυρίου θέλει αναπαυθή επ' αυτόν, πνεύμα σοφίας και συνέσεως, πνεύμα βουλής και δυνάμεως, πνεύμα γνώσεως και φόβου του Κυρίου·
\par 3 και θέλει κάμει αυτόν οξύνουν εις τον φόβον του Κυρίου, ώστε δεν θέλει κρίνει κατά την θεωρίαν των οφθαλμών αυτού ουδέ θέλει ελέγχει κατά την ακρόασιν των ωτίων αυτού·
\par 4 αλλ' εν δικαιοσύνη θέλει κρίνει τους πτωχούς, και εν ευθύτητι θέλει υπερασπίζεσθαι τους ταπεινούς της γής· και θέλει πατάξει την γην εν τη ράβδω του στόματος αυτού, και διά της πνοής των χειλέων αυτού θέλει θανατόνει τον ασεβή.
\par 5 Και δικαιοσύνη θέλει είσθαι η ζώνη της οσφύος αυτού και πίστις η ζώνη των πλευρών αυτού.
\par 6 Και ο λύκος θέλει συγκατοικεί μετά του αρνίου, και λεοπάρδαλις θέλει αναπαύεσθαι μετά του εριφίου· και ο μόσχος και ο σκύμνος και τα σιτευτά ομού, και μικρόν παιδίον θέλει οδηγεί αυτά.
\par 7 Και η δάμαλις και η άρκτος θέλουσι συμβόσκεσθαι, τα τέκνα αυτών θέλουσιν αναπαύεσθαι ομού, και ο λέων θέλει τρώγει άχυρον καθώς ο βους.
\par 8 Και το θηλάζον παιδίον θέλει παίζει εις την τρύπαν της ασπίδος, και το απογεγαλακτισμένον παιδίον θέλει βάλλει την χείρα αυτού εις την φωλεάν του βασιλίσκου.
\par 9 Δεν θέλουσι κακοποιεί ουδέ φθείρει εν όλω τω αγίω μου όρει· διότι η γη θέλει είσθαι πλήρης της γνώσεως του Κυρίου, καθώς τα ύδατα σκεπάζουσι την θάλασσαν.
\par 10 Και εν εκείνη τη ημέρα προς την ρίζαν του Ιεσσαί, ήτις θέλει ίστασθαι σημαία των λαών, προς αυτόν θέλουσι προστρέξει τα έθνη, και η ανάπαυσις αυτού θέλει είσθαι δόξα.
\par 11 Και εν εκείνη τη ημέρα ο Κύριος θέλει βάλει την χείρα αυτού πάλιν δευτέραν φοράν διά να αναλάβη το υπόλοιπον του λαού αυτού, το οποίον θέλει μείνει, από της Ασσυρίας και από της Αιγύπτου και από του Παθρώς και από της Αιθιοπίας και από του Ελάμ και από του Σενναάρ και από του Αιμάθ και από των νήσων της θαλάσσης.
\par 12 Και θέλει υψώσει σημαίαν εις τα έθνη, και θέλει συνάξει τους απερριμμένους του Ισραήλ και συναθροίσει τους διεσκορπισμένους του Ιούδα από των τεσσάρων γωνιών της γης.
\par 13 Και ο φθόνος του Εφραΐμ θέλει αφαιρεθή, και οι εχθρευόμενοι του Ιούδα θέλουσιν αποκοπή· ο Εφραΐμ δεν θέλει φθονεί τον Ιούδαν και ο Ιούδας δεν θέλει θλίβει τον Εφραΐμ.
\par 14 Αλλά θέλουσιν ορμήσει επί τα όρια των Φιλισταίων προς την δύσιν· θέλουσι λεηλατήσει και τους υιούς της ανατολής πάντας ομού· θέλουσι βάλει την χείρα αυτών επί τον Εδώμ και Μωάβ· και οι υιοί Αμμών θέλουσιν υποταχθή εις αυτούς.
\par 15 Και ο Κύριος θέλει καταξηράνει την γλώσσαν της Αιγυπτιακής θαλάσσης· και διά του βιαίου αυτού ανέμου θέλει σείσει την χείρα αυτού επί τον ποταμόν, και θέλει πατάξει αυτόν εις επτά ρεύματα, και θέλει κάμει να διαβαίνωσι με υποδήματα.
\par 16 Και θέλει είσθαι οδός πλατεία εις το υπόλοιπον του λαού αυτού, το οποίον θέλει μείνει, από της Ασσυρίας· ως ήτο εις τον Ισραήλ, καθ' ην ημέραν ανέβη εκ γης Αιγύπτου.

\chapter{12}

\par 1 Και εν εκείνη τη ημέρα θέλεις ειπεί, Κύριε, θέλω σε δοξολογήσει· διότι αν και ωργίσθης εναντίον μου, εστράφη ο θυμός σου και με παρηγόρησας.
\par 2 Ιδού, ο Θεός είναι η σωτηρία μου· θέλω θαρρεί και δεν θέλω φοβείσθαι· διότι Κύριος ο Θεός είναι η δύναμίς μου και το άσμα· και εστάθη η σωτηρία μου.
\par 3 Και εν ευφροσύνη θέλετε αντλήσει ύδωρ εκ των πηγών της σωτηρίας.
\par 4 Και εν εκείνη τη ημέρα θέλετε ειπεί, Δοξολογείτε τον Κύριον, επικαλείσθε το όνομα αυτού, κάμετε γνωστά εις τα έθνη τα έργα αυτού, μνημονεύετε ότι υψώθη το όνομα αυτού.
\par 5 Ψάλλετε εις τον Κύριον· διότι έκαμεν υψηλά· γνωστόν είναι εις πάσαν την γην.
\par 6 Αγάλλου και ευφραίνου, κάτοικε της Σιών· διότι ο Άγιος του Ισραήλ είναι μέγας εν τω μέσω σου.

\chapter{13}

\par 1 Η κατά Βαβυλώνος όρασις, την οποίαν είδεν Ησαΐας ο υιός του Αμώς.
\par 2 Σηκώσατε σημαίαν επί το όρος το υψηλόν, υψώσατε την φωνήν προς αυτούς, σείσατε την χείρα διά να εισέλθωσιν εις τας πύλας των αρχόντων.
\par 3 Εγώ προσέταξα τους διωρισμένους μου, μάλιστα έκραξα τους δυνατούς μου, διά να εκτελέσωσι τον θυμόν μου, τους χαίροντας εις την δόξαν μου.
\par 4 Φωνή πλήθους επί τα όρη ως μεγάλου λαού· θορυβώδης φωνή των βασιλείων των εθνών συνηγμένων· ο Κύριος των δυνάμεων επισκέπτεται το στράτευμα της μάχης.
\par 5 Έρχονται από γης μακράς, εκ των περάτων του ουρανού, ο Κύριος και τα όπλα της αγανακτήσεως αυτού, διά να αφανίσωσι πάσαν την γην.
\par 6 Ολολύζετε, διότι η ημέρα του Κυρίου επλησίασε· θέλει ελθεί ως όλεθρος από του Παντοδυνάμου.
\par 7 Διά τούτο πάσαι αι χείρες θέλουσιν εκλυθή, και πάσα καρδία ανθρώπου θέλει διαλυθή.
\par 8 Και θέλουσι τρομάξει· πόνοι και θλίψεις θέλουσι κατακυριεύσει αυτούς· θέλουσιν είσθαι εν πόνω, ως τίκτουσα· θέλουσι μείνει εκστατικοί ο εις προς τον άλλον· τα πρόσωπα αυτών θέλουσιν είσθαι πεφλογισμένα.
\par 9 Ιδού, η ημέρα του Κυρίου έρχεται, σκληρά και πλήρης θυμού και οργής φλογεράς, διά να καταστήση την γην έρημον· και θέλει εξαλείψει απ' αυτής τους αμαρτωλούς αυτής.
\par 10 Διότι τα άστρα του ουρανού και οι αστερισμοί αυτού δεν θέλουσι δώσει το φως αυτών· ο ήλιος θέλει σκοτισθή εν τη ανατολή αυτού, και η σελήνη δεν θέλει εκπέμψει το φως αυτής.
\par 11 Και θέλω παιδεύσει τον κόσμον διά την κακίαν αυτού και τους ασεβείς διά την ανομίαν αυτών και θέλω παύσει την μεγαλαυχίαν των υπερηφάνων και ταπεινώσει την υψηλοφροσύνην των φοβερών.
\par 12 Θέλω καταστήσει άνθρωπον πολυτιμότερον υπέρ χρυσίον καθαρόν· μάλιστα άνθρωπον υπέρ το χρυσίον του Οφείρ.
\par 13 Διά τούτο θέλω ταράξει τους ουρανούς, και η γη θέλει σεισθή από του τόπου αυτής, εν τω θυμώ του Κυρίου των δυνάμεων και εν τη ημέρα της φλογεράς οργής αυτού.
\par 14 Και θέλουσιν είσθαι ως δορκάδιον κυνηγούμενον και ως πρόβατον εγκαταλελειμμένον· θέλουσι στρέφεσθαι έκαστος προς τον λαόν αυτού και θέλουσι φεύγει έκαστος εις τον τόπον αυτού.
\par 15 Πας ο ευρεθείς θέλει διαπερασθή· και πάντες οι συνηθροισμένοι θέλουσι πέσει διά μαχαίρας.
\par 16 Και τα τέκνα αυτών θέλουσι συντριφθή έμπροσθεν αυτών· αι οικίαι αυτών θέλουσι λεηλατηθή, και αι γυναίκες αυτών θέλουσι βιασθή.
\par 17 Ιδού, θέλω επεγείρει τους Μήδους εναντίον αυτών, οίτινες δεν θέλουσι συλλογισθή αργύριον· και εις το χρυσίον, δεν θέλουσιν ηδυνθή εις αυτό·
\par 18 αλλά τα τόξα αυτών θέλουσι συντρίψει τους νεανίσκους· και δεν θέλουσιν ελεήσει τον καρπόν της κοιλίας· ο οφθαλμός αυτών δεν θέλει φεισθή παιδία.
\par 19 Και η Βαβυλών, η δόξα των βασιλείων, το ένδοξον καύχημα των Χαλδαίων, θέλει είσθαι ως ότε κατέστρεψεν ο Θεός τα Σόδομα και τα Γόμορρα·
\par 20 ουδέποτε θέλει κατοικηθή ουδέ θέλει κατασκηνωθή έως γενεάς και γενεάς· ούτε Άραβες θέλουσι στήσει τας σκηνάς αυτών εκεί, ούτε ποιμένες θέλουσιν αναπαύεσθαι εκεί·
\par 21 αλλά θηρία θέλουσιν αναπαύεσθαι εκεί· και αι οικίαι αυτών θέλουσιν είσθαι πλήρεις ολολυζόντων ζώων· και στρουθοκάμηλοι θέλουσι κατοικεί εκεί και σάτυροι θέλουσι χορεύει εκεί·
\par 22 και οι αίλουροι θέλουσι φωνάζει εν ταις ηρημωμέναις οικίαις αυτών και θώες εν τοις παλατίοις της τρυφής· και ο καιρός αυτής πλησιάζει να έλθη, και αι ημέραι αυτής δεν θέλουσιν επιμακρυνθή.

\chapter{14}

\par 1 Διότι ο Κύριος θέλει ελεήσει τον Ιακώβ, και θέλει έτι εκλέξει τον Ισραήλ και καταστήσει αυτούς εν τη γη αυτών· και οι ξένοι θέλουσιν ενωθή μετ' αυτών και θέλουσι προσκολληθή εις τον οίκον του Ιακώβ.
\par 2 Και οι λαοί θέλουσι λάβει αυτούς και φέρει αυτούς εις τον τόπον αυτών· και ο οίκος του Ισραήλ θέλει κληρονομήσει αυτούς εν τη γη του Κυρίου διά δούλους και δούλας· και θέλουσιν είσθαι αιχμάλωτοι αυτών οι αιχμαλωτίσαντες αυτούς, και θέλουσι γείνει κύριοι των καταθλιβόντων αυτούς.
\par 3 Και καθ' ην ημέραν ο Κύριος θέλει σε αναπαύσει από της θλίψεώς σου και από του φόβου σου και από της σκληράς δουλείας, εις την οποίαν ήσο καταδεδουλωμένος,
\par 4 θέλεις μεταχειρισθή την παροιμίαν ταύτην κατά του βασιλέως της Βαβυλώνος, λέγων, Πως επαύθη ο καταδυνάστης· πως επαύθη η φορολόγος του χρυσίου.
\par 5 Ο Κύριος συνέτριψε την ράβδον των ασεβών, το σκήπτρον των δυναστών.
\par 6 Ο πατάσσων εν θυμώ τον λαόν με ακατάπαυστον κτύπημα, ο δεσπόζων εν οργή επί τα έθνη, καταδιώκεται, και ουδείς ο κωλύων.
\par 7 Πάσα η γη αναπαύεται, ησυχάζει· εκφωνούσιν άσματα αγαλλιάσεως.
\par 8 Χαίρουσιν επί σε και αι έλατοι, αι κέδροι του Λιβάνου, λέγουσαι, Αφού συ εκοιμήθης, δενδροτόμος δεν ανέβη εφ' ημάς.
\par 9 Ο άδης κάτωθεν εκινήθη διά σε, διά να απαντήση την έλευσίν σου· διά σε εξήγειρε τους νεκρούς, πάντας τους ηγεμόνας της γής· εσήκωσεν εκ των θρόνων αυτών πάντας τους βασιλείς των εθνών.
\par 10 Πάντες ούτοι θέλουσιν αποκριθή και ειπεί προς σε, Και συ έγεινες αδύνατος, καθώς ημείς; κατεστάθης όμοιος ημών;
\par 11 Η μεγαλαυχία σου κατηνέχθη εις τον τάφον και ο θόρυβος των μουσικών σου οργάνων· ο σκώληξ είναι εστρωμένος υποκάτω σου και οι σκώληκες σε σκεπάζουσι·
\par 12 πως έπεσες εκ του ουρανού, Εωσφόρε, υιέ της αυγής· συνετρίφθης κατά γης, συ ο καταπατών τα έθνη.
\par 13 Συ δε έλεγες εν τη καρδία σου, Θέλω αναβή εις τον ουρανόν, θέλω υψώσει τον θρόνον μου υπεράνω των άστρων του Θεού· και θέλω καθήσει επί το όρος της συνάξεως, προς τα μέρη του βορρά·
\par 14 θέλω αναβή επί τα ύψη των νεφελών· θέλω είσθαι όμοιος του Υψίστου.
\par 15 Εις τον άδην όμως θέλεις καταβή, εις τα βάθη του λάκκου.
\par 16 Οι βλέποντές σε θέλουσιν ενατενίσει προς σε, θέλουσι σε παρατηρεί, λέγοντες, ούτος είναι ο άνθρωπος ο ποιών την γην να τρέμη, ο σείων τα βασίλεια;
\par 17 Ο ερημόνων την οικουμένην και καταστρέφων τας πόλεις αυτής; ο μη απολύων εις τας οικίας αυτών τους δεσμίους αυτού;
\par 18 Πάντες οι βασιλείς των εθνών, πάντες αναπαύονται εν δόξη, έκαστος εν τη οικία αυτού·
\par 19 συ δε απερρίφθης του τάφου σου ως κλάδος βδελυκτός, ιμάτιον κεκεντημένων, πεφονευμένων εν μαχαίρα, καταβαινόντων εις τας πέτρας του λάκκου· ως πτώμα καταπατούμενον.
\par 20 Δεν θέλεις ενωθή μετ' αυτών εις ενταφιασμόν, διότι ηφάνισας την γην σου, εφόνευσας τον λαόν σου· το σπέρμα των κακοποιών ουδέποτε θέλει ονομασθή.
\par 21 Ετοιμάσατε σφαγήν εις τα τέκνα αυτού διά την ανομίαν των πατέρων αυτών, διά να μη σηκωθώσι και κληρονομήσωσι την γην, και γεμίσωσι το πρόσωπον της οικουμένης από πόλεων.
\par 22 Διότι θέλω σηκωθή εναντίον αυτών, λέγει ο Κύριος των δυνάμεων· και θέλω εξαλείψει από της Βαβυλώνος το όνομα και το υπόλοιπον και υιόν και έκγονον, λέγει Κύριος.
\par 23 Και θέλω καταστήσει αυτήν κληρονομίαν εχίνων και λίμνας υδάτων· και θέλω σαρώσει αυτήν με το σάρωθρον της απωλείας, λέγει ο Κύριος των δυνάμεων.
\par 24 Ώμοσεν ο Κύριος των δυνάμεων, λέγων, Εξάπαντος καθώς εβουλεύθην, ούτω θέλει γείνει· και καθώς απεφάσισα, ούτω θέλει μείνει,
\par 25 να συντρίψω τον Ασσύριον εν τη γη μου και να καταπατήσω αυτόν επί των ορέων μου· τότε ο ζυγός αυτού θέλει σηκωθή απ' αυτών και το φορτίον αυτού θέλει αφαιρεθή από των ώμων αυτών.
\par 26 Αύτη είναι η βουλή η βεβουλευμένη καθ' όλης της γής· και αύτη η χειρ η εξηπλωμένη επί πάντα τα έθνη.
\par 27 Διότι ο Κύριος των δυνάμεων απεφάσισε και τις θέλει αναιρέσει; και η χειρ αυτού εξηπλώθη και τις θέλει αποστρέψει αυτήν;
\par 28 Εν τω έτει, καθ' ο απέθανεν ο βασιλεύς Άχαζ, έγεινεν αύτη η όρασις.
\par 29 Μη χαίρε, Παλαιστίνη πάσα, διότι συνετρίφθη η ράβδος του πατάξαντός σε· επειδή εκ της ρίζης του όφεως θέλει εξέλθει βασιλίσκος, και ο καρπός αυτού θέλει είσθαι φλογερός πετώμενος όφις.
\par 30 Και οι πρωτότοκοι του πτωχού θέλουσι τραφή και οι ενδεείς θέλουσιν αναπαύεσθαι εν ασφαλεία· και θέλω θανατώσει την ρίζαν σου με πείναν, και θέλω φονεύσει το υπόλοιπόν σου.
\par 31 Ολόλυζε, πύλη· βόα, πόλις· εχάθης, Παλαιστίνη πάσα· διότι έρχεται καπνός από βορρά, και ουδέ εις θέλει λείψει από της εκστρατεύσεως αυτού εν τοις ωρισμένοις καιροίς.
\par 32 Και τις απόκρισις θέλει δοθή εις τους πρέσβεις των εθνών; Ότι ο Κύριος εθεμελίωσε την Σιών, και επ' αυτήν θέλουσιν ελπίζει οι πτωχοί του λαού αυτού.

\chapter{15}

\par 1 Η κατά Μωάβ όρασις. Επειδή η Αρ Μωάβ επορθήθη την νύκτα και ηφανίσθη· επειδή η Κιρ Μωάβ επορθήθη την νύχτα και ηφανίσθη·
\par 2 ανέβη εις τον οίκον και εις Δαιβών, τους υψηλούς τόπους, διά να κλαύση· ο Μωάβ θέλει ολολύξει διά την Νεβώ, και διά την Μεδεβά· πάσαι αι κεφαλαί θέλουσι φαλακρωθή, παν γένειον θέλει ξυρισθή.
\par 3 Εν ταις οδοίς αυτών θέλουσιν είσθαι περιεζωσμένοι σάκκους· επί των δωμάτων αυτών και εν ταις πλατείαις αυτών πάντες θέλουσιν ολολύξει μετά κλαυθμού μεγάλου.
\par 4 Και η Εσεβών θέλει βοά, και η Ελεαλή· η βοή αυτών θέλει ακουσθή έως Ιασσά· διά τούτο οι οπλοφόροι άνδρες του Μωάβ θέλουσιν ολολύξει· η ψυχή αυτών θέλει ολολύξει δι' αυτούς.
\par 5 Η καρδία μου θέλει αναβοήσει διά τον Μωάβ· οι φυγάδες αυτού θέλουσι τρέξει έως Σηγώρ ως τριετής δάμαλις· διότι κλαίοντες θέλουσιν αναβή διά της αναβάσεως της Λουείθ· διότι εν τη οδώ της Οροναΐμ θέλουσιν υψώσει φωνήν εξολοθρευμού·
\par 6 διότι τα ύδατα της Νιμρείμ θέλουσιν εκλείψει· διότι ο χόρτος εξηράνθη, η χλόη εξέλιπε, δεν υπάρχει ουδέν χλωρόν·
\par 7 διά τούτο η αφθονία, την οποίαν συνήξαν, και εκείνο το οποίον απεταμίευσαν, θέλει φερθή εις την κοιλάδα των ιτεών.
\par 8 Διότι η φωνή έφθασε κύκλω εις τα όρια του Μωάβ· ο ολολυγμός αυτής έως Εγλαΐμ, και ο ολολυγμός αυτής εις Βηρ-αιλείμ.
\par 9 Διότι τα ύδατα της Δειμών θέλουσι γεμισθή αίματος· διότι θέλω επιφέρει έτι δεινά επί Δειμών, λέοντας επί τον διασωθέντα εκ του Μωάβ και επί τα υπόλοιπα του τόπου.

\chapter{16}

\par 1 Αποστείλατε το αρνίον προς τον άρχοντα της γης, από Σελά εν τη ερήμω προς το όρος της θυγατρός της Σιών.
\par 2 Διότι ως πτηνόν πλανώμενον, από της φωλεάς αυτού δεδιωγμένον, ούτως αι θυγατέρες του Μωάβ θέλουσιν είσθαι κατά τας διαβάσεις του Αρνών.
\par 3 Βουλεύου, εκτέλεσον το δίκαιον· εν τω μέσω της ημέρας κάμε την σκιάν σου ως νύκτα· κρύψον τους διωκομένους· μη φανερώσης τον περιπλανώμενον.
\par 4 Οι δεδιωγμένοι μου ας παροικήσωσι παρά σοι, Μωάβ· γενού εις αυτούς σκέπη από προσώπου του πορθητού· διότι ο αρπακτήρ ετελείωσεν, ο πορθητής έπαυσεν, οι καταδυνάσται εξωλοθρεύθησαν από της γης.
\par 5 Και μετά ελέους θέλει συσταθή ο θρόνος, και επ' αυτόν θέλει καθήσει εν αληθεία, εν τη σκηνή του Δαβίδ, ο κρίνων και εκζητών κρίσιν και σπεύδων δικαιοσύνην.
\par 6 Ηκούσαμεν την υπερηφανίαν του Μωάβ, είναι λίαν υπερήφανος· την υψηλοφροσύνην αυτού και την αλαζονείαν αυτού και την μανίαν αυτού· τα ψεύδη αυτού θέλουσι ματαιωθή.
\par 7 Διά τούτο ο Μωάβ θέλει ολολύξει· πάντες θέλουσιν ολολύξει διά τον Μωάβ· θέλετε θρηνολογήσει διά τα θεμέλια της Κιρ-αρεσέθ· εκτυπήθησαν βεβαίως.
\par 8 Διότι αι πεδιάδες της Εσεβών είναι ητονημέναι και η άμπελος της Σιβμά· οι κύριοι των εθνών κατεσύντριψαν τα καλήτερα αυτής φυτά, τα οποία ήρχοντο έως της Ιαζήρ, και περιεπλανώντο διά της ερήμου· οι κλάδοι αυτής ήσαν εξηπλωμένοι, διέβαινον την θάλασσαν.
\par 9 Διά τούτο μετά κλαυθμού της Ιαζήρ θέλω κλαύσει την άμπελον της Σιβμά θέλω σε βρέξει με τα δάκρυά μου, Εσεβών και Ελεαλή· διότι επί τους θερινούς καρπούς σου και επί τον θερισμόν σου επέπεσεν αλαλαγμός.
\par 10 Και αφηρέθη η ευφροσύνη και η αγαλλίασις από της καρποφόρου πεδιάδος· και εις τους αμπελώνάς σου δεν θέλουσιν είσθαι πλέον άσματα ουδέ φωναί αγαλλιάσεως· οι ληνοπάται δεν θέλουσι πατεί οίνον εν τοις ληνοίς· εγώ κατέπαυσα τον αλαλαγμόν του τρυγητού.
\par 11 Όθεν τα εντόσθιά μου θέλουσιν ηχήσει ως κιθάρα διά τον Μωάβ, και τα εσωτερικά μου διά την Κιρ-άρες.
\par 12 Και ο Μωάβ, όταν φανή ότι απέκαμεν επί τους βωμούς αυτού, θέλει εισέλθει εις το αγιαστήριον αυτού διά να προσευχηθή· πλην δεν θέλει επιτύχει.
\par 13 Ούτος είναι ο λόγος, τον οποίον έκτοτε ελάλησε Κύριος περί του Μωάβ.
\par 14 Τώρα δε ο Κύριος ελάλησε λέγων, Εις τρία έτη, ως είναι τα έτη του μισθωτού, η δόξα του Μωάβ θέλει καταφρονηθή μεθ' όλου του μεγάλου πλήθους αυτού· και το υπόλοιπον θέλει είσθαι πολύ ολίγον και αδύνατον.

\chapter{17}

\par 1 Η κατά Δαμασκού όρασις. Ιδού, η Δαμασκός πεπαυμένη του να ήναι πόλις, και θέλει είσθαι σωρός ερειπίων.
\par 2 Αι πόλεις της Αροήρ εγκατελείφθησαν· θέλουσιν είσθαι διά τα ποίμνια, τα οποία θέλουσιν αναπαύεσθαι εκεί, και δεν θέλει είσθαι ο φοβίζων.
\par 3 Και θέλει εκλείψει από του Εφραΐμ η βοήθεια και το βασίλειον από της Δαμασκού, και το υπόλοιπον της Συρίας θέλει γείνει ως η δόξα των υιών του Ισραήλ, λέγει ο Κύριος των δυνάμεων.
\par 4 Και εν τη ημέρα εκείνη η δόξα του Ιακώβ θέλει σμικρυνθή και το πάχος της σαρκός αυτού θέλει ισχνωθή.
\par 5 Και θέλει είσθαι, ως όταν ο θεριστής συνάγη τον σίτον και θερίζη τα αστάχυα διά του βραχίονος αυτού· και θέλει είσθαι ως ο συνάγων αστάχυα εν τη κοιλάδι Ραφαείμ.
\par 6 Θέλουσιν όμως μείνει εν αυτή ρώγες, ως εν τω τιναγμώ της ελαίας, δύο τρεις ελαίαι επί της κορυφής των υψηλοτέρων κλάδων, τέσσαρες πέντε επί των μακροτέρων αυτής καρποφόρων κλάδων, λέγει Κύριος ο Θεός του Ισραήλ.
\par 7 Εν εκείνη τη ημέρα ο άνθρωπος θέλει αναβλέψει προς τον Ποιητήν αυτού και οι οφθαλμοί αυτού θέλουσιν ενατενίσει προς τον Άγιον του Ισραήλ.
\par 8 Και δεν θέλει αναβλέψει προς τους βωμούς, το έργον των χειρών αυτού, ουδέ θέλει σεβασθή εκείνο το οποίον έκαμον οι δάκτυλοι αυτού, ούτε τα άλση ούτε τα είδωλα.
\par 9 Εν εκείνη τη ημέρα αι οχυραί πόλεις αυτού θέλουσιν είσθαι ως εγκαταλελειμμένος κλάδος και ακρότατον κλωνάριον, το οποίον αφήκαν εξ αιτίας των υιών του Ισραήλ· και θέλει είσθαι ερήμωσις.
\par 10 Επειδή ελησμόνησας τον Θεόν της σωτηρίας σου και δεν ενεθυμήθης τον βράχον της δυνάμεώς σου, διά τούτο θέλεις φυτεύσει ευάρεστα φυτά, και θέλεις κάμει την εμφύτευσιν με ξένα βλαστήματα·
\par 11 την ημέραν θέλεις κάμει το φυτόν σου να αυξηθή, και το πρωΐ θέλεις κάμει τον σπόρον σου να ανθήση πλην το θέρος θέλει διαρπαχθή, εν τη ημέρα του πόνου και της απηλπισμένης θλίψεως.
\par 12 Ουαί εις το πλήθος πολλών λαών, οίτινες κάμνουσι ταραχήν ως την ταραχήν των θαλασσών· και εις τον θόρυβον των εθνών, τα οποία θορυβούσιν ως θόρυβον υδάτων πολλών.
\par 13 Τα έθνη θέλουσι θορυβήσει ως θόρυβον υδάτων πολλών· αλλ' ο Θεός θέλει ελέγξει αυτά, και θέλουσι φύγει μακράν και θέλουσιν εκδιωχθή, ως το άχυρον των βουνών έμπροσθεν του ανέμου και ως κονιορτός έμπροσθεν του ανεμοστροβίλου.
\par 14 Προς το εσπέρας, ιδού, ταραχή· και πριν της αυγής δεν υπάρχει. Αύτη είναι η μερίς των λεηλατούντων ημάς και ο κλήρος των διαρπαζόντων ημάς.

\chapter{18}

\par 1 Ουαί, γη σκιάζουσα διά των πτερύγων, η πέραν των ποταμών της Αιθιοπίας,
\par 2 η εξαποστέλλουσα πρέσβεις διά θαλάσσης και με πλοία σπάρτινα επί των υδάτων. Υπάγετε, ταχύδρομοι αγγελιαφόροι, προς έθνος διηρπαγμένον και κατεσπαραγμένον, προς λαόν τρομερόν από της αρχής αυτού έως της σήμερον, έθνος μεμετρημένον και καταπεπατημένον, του οποίου την γην διήρπασαν οι ποταμοί.
\par 3 Πάντες οι κάτοικοι του κόσμου και οι ενοικούντες επί της γης, βλέπετε, όταν υψωθή σημαία επί τα όρη· και ακούσατε, όταν εκπεμφθή φωνή σάλπιγγος.
\par 4 Διότι ούτως είπε Κύριος προς εμέ· Θέλω ησυχάσει και θέλω επιβλέψει εις το κατοικητήριόν μου, ως καύσων λαμπρότερος του φωτός, ως νεφέλη δρόσου εν τω καύσωνι του θέρους.
\par 5 Διότι πριν του θέρους, όταν το βλάστημα γείνη τέλειον και η αγουρίδα ωριμάση εκ του άνθους, θέλει κόψει τους βλαστούς διά κλαδευτηρίων και τας κληματίδας αποκόψας θέλει αφαιρέσει.
\par 6 Θέλουσιν εγκαταλειφθή ομού διά τα όρνεα των βουνών και διά τα θηρία της γής· και τα όρνεα θέλουσι περάσει το θέρος επ' αυτούς, και πάντα τα θηρία της γης θέλουσι διαχειμάσει επ' αυτούς.
\par 7 Εν εκείνω τω καιρώ θέλει φερθή δώρον προς τον Κύριον των δυνάμεων εκ λαού διηρπαγμένου και κατεσπαραγμένου και εκ λαού τρομερού από της αρχής αυτού έως της σήμερον, έθνους μεμετρημένου και καταπεπατημένου, του οποίου την γην διήρπασαν οι ποταμοί, εις τον τόπον του ονόματος του Κυρίου των δυνάμεων, το όρος Σιών.

\chapter{19}

\par 1 Η κατά της Αιγύπτου όρασις. Ιδού, ο Κύριος επιβαίνει επί νεφέλης κούφης και θέλει επέλθει επί την Αίγυπτον· και τα είδωλα της Αιγύπτου θέλουσι σεισθή από προσώπου αυτού, και η καρδία της Αιγύπτου θέλει διαλυθή εν μέσω αυτής.
\par 2 Και θέλει σηκώσει Αιγυπτίους κατά Αιγυπτίων, και θέλουσι πολεμήσει έκαστος κατά του αδελφού αυτού και έκαστος κατά του πλησίον αυτού· πόλις κατά πόλεως, βασιλεία κατά βασιλείας.
\par 3 Και θέλει εκλείψει το πνεύμα της Αιγύπτου εν μέσω αυτής· και θέλω ανατρέψει την βουλήν αυτής· και θέλουσιν ερωτήσει τα είδωλα και τους μάγους και τους εγγαστριμύθους και τους μάντεις.
\par 4 Και θέλω παραδώσει τους Αιγυπτίους εις χείρα σκληρών κυρίων· και βασιλεύς άγριος θέλει εξουσιάζει αυτούς, λέγει ο Κύριος, ο Κύριος των δυνάμεων.
\par 5 Και τα ύδατα θέλουσιν εκλείψει εκ των θαλασσών και ο ποταμός θέλει αφανισθή και καταξηρανθή.
\par 6 Και οι ποταμοί θέλουσι στειρεύσει· οι ρύακες οι περιπεφραγμένοι θέλουσι κενωθή και καταξηρανθή· ο κάλαμος και ο σπάρτος θέλουσι μαρανθή·
\par 7 τα λιβάδια πλησίον των ρυάκων, επί των στομίων των ρυάκων, και παν το εσπαρμένον παρά τους ρύακας θέλει ξηρανθή, απορριφθή και αφανισθή.
\par 8 Και οι αλιείς θέλουσι στενάξει και πάντες οι ρίπτοντες άγκιστρον εις τους ρύακας θέλουσι θρηνήσει και οι βάλλοντες δίκτυα επί τα ύδατα θέλουσι νεκρωθή.
\par 9 Και οι εργαζόμενοι εις λεπτόν λινάριον και οι πλέκοντες δίκτυα θέλουσι ταραχθή.
\par 10 Και οι στύλοι αυτής θέλουσι συντριφθή και πάντες οι κερδαίνοντες από ιχθυοτροφείων.
\par 11 Βεβαίως οι άρχοντες της Τάνεως είναι μωροί, η βουλή των σοφών συμβούλων του Φαραώ κατεστάθη άλογος· πως λέγετε έκαστος προς τον Φαραώ, Εγώ είμαι υιός σοφών, υιός αρχαίων βασιλέων;
\par 12 Που, που, οι σοφοί σου; και ας είπωσι τώρα προς σε, και ας καταλάβωσι τι εβουλεύθη ο Κύριος των δυνάμεων κατά της Αιγύπτου.
\par 13 Οι άρχοντες της Τάνεως εμωράνθησαν, οι άρχοντες της Μέμφεως επλανήθησαν· και επλάνησαν την Αίγυπτον οι άρχοντες των φυλών αυτής.
\par 14 Ο Κύριος εκέρασεν εν τω μέσω αυτής πνεύμα παραφροσύνης· και επλάνησαν την Αίγυπτον εις πάντα τα έργα αυτής, ως ο μεθύων πλανάται εν τω εμετώ αυτού.
\par 15 Και δεν θέλει είσθαι έργον διά την Αίγυπτον, το οποίον η κεφαλή ή η ουρά, ο κλάδος ή ο σπάρτος, να δύναται να κάμη.
\par 16 Εν εκείνη τη ημέρα οι Αιγύπτιοι θέλουσιν είσθαι ως γυναίκες, και θέλουσι τρομάξει και φοβηθή από της χειρός του Κυρίου των δυνάμεων σειομένης, την οποίαν σείει επ' αυτούς.
\par 17 Και η γη του Ιούδα θέλει είσθαι φρίκη εις τους Αιγυπτίους· πας όστις ενθυμείται αυτήν θέλει φρίττει, διά την βουλήν του Κυρίου των δυνάμεων, την οποίαν απεφάσισεν εναντίον αυτών.
\par 18 Εν εκείνη τη ημέρα πέντε πόλεις θέλουσιν είσθαι εν τη γη της Αιγύπτου λαλούσαι την γλώσσαν της Χαναάν και ομνύουσαι εις τον Κύριον των δυνάμεων· η μία θέλει ονομάζεσθαι η πόλις Αχέρες.
\par 19 Εν εκείνη τη ημέρα θέλει είσθαι εν τω μέσω της γης Αιγύπτου θυσιαστήριον εις τον Κύριον και στήλη κατά το όριον αυτής εις τον Κύριον.
\par 20 Και θέλει είσθαι εν τη γη της Αιγύπτου διά σημείον και μαρτυρίαν εις τον Κύριον των δυνάμεων· διότι θέλουσι βοά προς τον Κύριον εξ αιτίας των καταθλιβόντων, και θέλει εξαποστείλει προς αυτούς σωτήρα και μέγαν και θέλει σώσει αυτούς.
\par 21 Και θέλει γνωρισθή ο Κύριος εις τους Αιγυπτίους· και οι Αιγύπτιοι θέλουσι γνωρίσει τον Κύριον εν εκείνη τη ημέρα και θέλουσι προσφέρει θυσίαν και προσφοράν· και θέλουσιν ευχηθή ευχήν εις τον Κύριον και εκπληρώσει αυτήν.
\par 22 Και θέλει κτυπήσει ο Κύριος την Αίγυπτον· θέλει κτυπήσει και θεραπεύσει αυτήν· και θέλουσιν επιστραφή εις τον Κύριον· και θέλει παρακαλεσθή υπ' αυτών και θέλει ιατρεύσει αυτούς.
\par 23 Εν εκείνη τη ημέρα θέλει είσθαι οδός μεγάλη από της Αιγύπτου προς την Ασσυρίαν· και οι Ασσύριοι θέλουσιν ελθεί εις την Αίγυπτον, και οι Αιγύπτιοι εις την Ασσυρίαν, και οι Αιγύπτιοι μετά των Ασσυρίων θέλουσι δουλεύσει εις τον Κύριον.
\par 24 Εν εκείνη τη ημέρα ο Ισραήλ θέλει είσθαι ο τρίτος μετά του Αιγυπτίου και μετά του Ασσυρίου· ευλογία εν μέσω της γης θέλει είσθαι·
\par 25 διότι ο Κύριος των δυνάμεων θέλει ευλογήσει αυτούς λέγων, Ευλογημένη η Αίγυπτος ο λαός μου και η Ασσυρία το έργον των χειρών μου και ο Ισραήλ η κληρονομία μου.

\chapter{20}

\par 1 Εν τω έτει καθ' ο ο Ταρτάν ήλθεν εις την Άζωτον, ότε απέστειλεν αυτόν ο Σαργών βασιλεύς της Ασσυρίας και επολέμησε κατά της Αζώτου και εκυρίευσεν αυτήν,
\par 2 κατά τον αυτόν καιρόν ελάλησεν ο Κύριος προς Ησαΐαν τον υιόν του Αμώς, λέγων, Ύπαγε και λύσον τον σάκκον από της οσφύος σου και έκβαλε τα σανδάλια σου από των ποδών σου. Και έκαμεν ούτω, περιπατών γυμνός και ανυπόδητος.
\par 3 Και είπε Κύριος, Καθώς ο δούλός μου Ησαΐας περιεπάτει γυμνός και ανυπόδητος τρία έτη, διά σημείον και τεράστιον κατά της Αιγύπτου και κατά της Αιθιοπίας,
\par 4 ούτως ο βασιλεύς της Ασσυρίας θέλει απαγάγει τους Αιγυπτίους δεσμίους και τους Αιθίοπας αιχμαλώτους, νέους και γέροντας, γυμνούς και ανυποδήτους, με γυμνά μάλιστα τα οπίσθια αυτών, προς καταισχύνην της Αιγύπτου.
\par 5 Και θέλουσι τρομάξει και εντραπή διά την Αιθιοπίαν, το θάρρος αυτών· και διά την Αίγυπτον, το καύχημα αυτών.
\par 6 Και οι κάτοικοι του τόπου τούτου θέλουσι λέγει εν εκείνη τη ημέρα, Ιδού, τοιούτον είναι το καταφύγιον ημών, εις το οποίον καταφεύγομεν προς βοήθειαν, διά να ελευθερωθώμεν από του βασιλέως της Ασσυρίας· και πως ημείς θέλομεν σωθή;

\chapter{21}

\par 1 Η κατά της ερήμου της θαλάσσης όρασις. Καθώς οι διαβαίνοντες ανεμοστρόβιλοι της μεσημβρίας, ούτως έρχεται από της ερήμου, από γης τρομεράς.
\par 2 Σκληρόν όραμα εφανερώθη εις εμέ· ο καταδυναστεύων καταδυναστεύει και ο πορθών πορθεί. Ανάβηθι, Ελάμ· πολιόρκησον, Μηδία· έπαυσα πάσας τας καταδυναστείας αυτής.
\par 3 Διά τούτο η οσφύς μου είναι πλήρης οδύνης· πόνοι με εκυρίευσαν, ως οι πόνοι της τικτούσης· εκυρτώθην εις την ακρόασιν αυτού· συνεταράχθην εις την θέαν αυτού.
\par 4 Η καρδία μου κλονίζεται· τρόμος με εξέπληξεν· η νυξ της ευφροσύνης μου εις φρίκην μετεβλήθη εν εμοί.
\par 5 Ετοιμάζεται η τράπεζα· φυλάττουσι σκοπιάν, τρώγουσι, πίνουσι· σηκώθητε, στρατάρχαι, ετοιμάσατε ασπίδας.
\par 6 Διότι ο Κύριος είπεν ούτω προς εμέ· Ύπαγε, στήσον σκοπευτήν, διά να αναγγέλλη ό,τι βλέπει.
\par 7 Και είδεν αναβάτας δύο ιππείς, αναβάτην όνου και αναβάτην καμήλου· και επρόσεξεν επιμελώς μετά πολλής προσοχής.
\par 8 Και εφώναξεν ως λέων, Ακαταπαύστως, κύριέ μου, ίσταμαι εν τη σκοπιά την ημέραν και φυλάττω πάσας τας νύκτας·
\par 9 και ιδού, έρχονται εδώ αναβάται άνδρες δύο ιππείς. Και απεκρίθη και είπεν, Έπεσεν, έπεσεν η Βαβυλών, και πάσαι αι γλυπταί εικόνες των θεών αυτής συνετρίφθησαν κατά γης.
\par 10 Αλώνισμά μου και σίτε του αλωνίου μου, εφανέρωσα εις εσάς εκείνο, το οποίον ήκουσα παρά του Κυρίου των δυνάμεων, του Θεού του Ισραήλ.
\par 11 Η κατά Δουμά όρασις. Προς εμέ φωνάζει από Σηείρ, Φρουρέ, τι περί της νυκτός; φρουρέ, τι περί της νυκτός;
\par 12 Ο φρουρός είπε, Το πρωΐ ήλθεν, έτι και η νύξ· αν θέλητε να ερωτήσητε, ερωτάτε· επιστρέψατε, έλθετε.
\par 13 Η κατά Αραβίας όρασις. Εν τω δάσει της Αραβίας θέλετε διανυκτερεύσει, συνοδίαι των Δαιδανιτών.
\par 14 Φέρετε ύδωρ εις συνάντησιν του διψώντος, κάτοικοι της γης Θαιμάν· προϋπαντάτε με άρτους τον φεύγοντα.
\par 15 Διότι φεύγουσιν από προσώπου των ξιφών, από προσώπου του γεγυμνωμένου ξίφους και από προσώπου του εντεταμένου τόξου και από προσώπου της ορμής του πολέμου.
\par 16 Διότι ο Κύριος είπεν ούτω προς εμέ· Εντός ενός έτους, ως είναι τα έτη του μισθωτού, θέλει εκλείψει βεβαίως πάσα η δόξα της Κηδάρ·
\par 17 και το υπόλοιπον του αριθμού των ισχυρών τοξοτών εκ των υιών του Κηδάρ θέλουσιν ελαττωθή· διότι Κύριος ο Θεός του Ισραήλ ελάλησε.

\chapter{22}

\par 1 Όρασις κατά της κοιλάδος του οράματος. Τι σοι έγεινε τώρα, ότι ανέβης συ πάσα εις τα δώματα;
\par 2 Συ, η πλήρης βοής, πόλις θορύβου, πόλις ευθυμίας· οι πεφονευμένοι σου δεν εφονεύθησαν διά μαχαίρας ουδέ απέθανον εν μάχη.
\par 3 Πάντες οι άρχοντές σου έφυγον ομού· φεύγοντες από του τόξου, εδεσμεύθησαν πάντες οι ευρισκόμενοι εν σοί· οι μακρόθεν καταφυγόντες εδεσμεύθησαν ομού.
\par 4 Διά τούτο είπα, Σύρθητε απ' εμού· θέλω κλαύσει πικρώς· μη αγωνίζεσθε να με παρηγορήσητε διά την διαρπαγήν της θυγατρός του λαού μου.
\par 5 Διότι είναι ημέρα ταραχής και καταπατήσεως και αμηχανίας εν τη κοιλάδι του οράματος παρά Κυρίου του Θεού των δυνάμεων, ημέρα καταστροφής των τειχών· και η κραυγή θέλει φθάσει εις τα όρη.
\par 6 Και ο Ελάμ έλαβε την φαρέτραν με αμάξας ανδρών και ιππείς, και ο Κιρ εξεσκέπασε την ασπίδα.
\par 7 Και αι εκλεκταί κοιλάδες σου εγεμίσθησαν αμαξών, και οι ιππείς παρετάχθησαν εν τη πύλη.
\par 8 Και εσηκώθη το κάλυμμα του Ιούδα· και εν τη ημέρα εκείνη ενέβλεψας εις την οπλοθήκην της οικίας του δάσους.
\par 9 Και είδετε ότι αι χαλάστραι της πόλεως του Δαβίδ είναι πολλαί, και συνηθροίσατε τα ύδατα του κάτω υδροστασίου.
\par 10 Και απηριθμήσατε τας οικίας της Ιερουσαλήμ, και διά να οχυρώσητε το τείχος εχαλάσατε τας οικίας.
\par 11 Εκάμετε προς τούτοις μεταξύ των δύο τειχών λάκκον διά το ύδωρ του παλαιού υδροστασίου· αλλά δεν ανεβλέψατε προς τον Ποιητήν τούτων ουδέ εθεωρήσατε προς τον παλαιόθεν κτίσαντα αυτά.
\par 12 Και εν εκείνη τη ημέρα Κύριος ο Θεός των δυνάμεων σας εκάλεσεν εις κλαυθμόν και εις πένθος και εις ξύρισμα και εις ζώσιμον σάκκου·
\par 13 αλλ' ιδού, χαρά και ευθυμία· σφάζουσι βόας και θύουσι πρόβατα, τρώγουσι κρέατα και πίνουσιν οίνον, λέγοντες, Ας φάγωμεν και ας πίωμεν· διότι αύριον θέλομεν αποθάνει.
\par 14 Και ανεκαλύφθη εις τα ώτα μου παρά του Κυρίου των δυνάμεων, Βεβαίως αύτη η ανομία σας δεν θέλει καθαρισθή εωσού αποθάνητε, λέγει Κύριος ο Θεός των δυνάμεων.
\par 15 Ούτω λέγει Κύριος ο Θεός των δυνάμεων· Ύπαγε, είσελθε προς τον θησαυροφύλακα τούτον, προς τον Σομνάν, τον επιστάτην του οίκου, και ειπέ,
\par 16 Τι έχεις εδώ; και εδώ τίνα έχεις, ώστε να κατασκευάσης ενταύθα μνημείον εις σεαυτόν; κατασκευάζει το μνήμα αυτού υψηλά και κόπτει εν πέτρα κατοικίαν εις εαυτόν.
\par 17 Ιδού, ο Κύριος θέλει σε εκβάλει εκβολήν βιαίαν και θέλει σε περικαλύψει αισχύνην.
\par 18 Θέλει βεβαίως σε στροφογυρίσει και τινάξει βιαίως ως σφαίραν εις τόπον ευρύχωρον· εκεί θέλεις αποθάνει και εκεί θέλουσιν είσθαι αι άμαξαι της δόξης σου, ω αίσχος του οίκου του κυρίου σου.
\par 19 Και θέλω σε εξώσει από της στάσεώς σου και θέλει σε κρημνίσει από του αξιώματός σου.
\par 20 Και εν εκείνη τη ημέρα θέλω καλέσει τον δούλον μου Ελιακείμ, τον υιόν του Χελκίου.
\par 21 Και θέλω ενδύσει αυτόν την στολήν σου και θέλω περιζώσει αυτόν την ζώνην σου, και την εξουσίαν σου θέλω δώσει εις την χείρα αυτού και θέλει είσθαι πατήρ εις τους κατοίκους της Ιερουσαλήμ και εις τον οίκον του Ιούδα.
\par 22 Και θέλω βάλει επί τον ώμον αυτού το κλειδίον του οίκου του Δαβίδ· και θέλει ανοίγει και ουδείς θέλει κλείει· και θέλει κλείει και ουδείς θέλει ανοίγει.
\par 23 Και θέλω στηρίξει αυτόν ως πάσσαλον εν τόπω στερεώ και θέλει είσθαι ως θρόνος δόξης του οίκου του πατρός αυτού.
\par 24 Και απ' αυτού θέλουσι κρεμάσει πάσαν την δόξαν του οίκου του πατρός αυτού, τους εκγόνους και απογόνους, πάντα τα σκεύη τα μικρά, από των σκευών των ποτηρίων έως πάντων των σκευών των φιαλών.
\par 25 Εν εκείνη τη ημέρα, λέγει ο Κύριος των δυνάμεων, το εστηριγμένον καρφίον εν τω στερεώ τόπω θέλει κινηθή και θέλει εκβληθή και πέσει, και το φορτίον το επ' αυτού θέλει κρημνισθή· διότι ο Κύριος ελάλησε.

\chapter{23}

\par 1 Η κατά της Τύρου όρασις. Ολολύζετε, πλοία της Θαρσείς· διότι εξωλοθρεύθη, ώστε να μη υπάρχη οικία μηδέ είσοδος· εκ της γης των Κητιαίων ανηγγέλθη τούτο προς αυτούς.
\par 2 Σιωπήσατε, κάτοικοι της νήσου· συ, την οποίαν εγέμισαν οι έμποροι της Σιδώνος, οι διαβαίνοντες επί της θαλάσσης.
\par 3 Και το εισόδημα αυτής είναι ο σπόρος του Σιώρ, το θέρος του ποταμού, φερόμενα διά πολλών υδάτων· και αύτη έγεινε το εμπόριον των εθνών.
\par 4 Αισχύνθητι, Σιδών· διότι η θάλασσα ελάλησε, το οχύρωμα της θαλάσσης, λέγουσα, Δεν κοιλοπονώ ουδέ γεννώ ουδέ ανατρέφω νέους ουδέ μεγαλώνω παρθένους.
\par 5 Όταν ακουσθή εν Αιγύπτω, θέλουσι λυπηθή ακούοντες περί της Τύρου.
\par 6 Διέλθετε εις Θαρσείς· ολολύξατε, κάτοικοι της νήσου.
\par 7 Αύτη είναι η εύθυμος πόλις σας, της οποίας η αρχαιότης είναι εκ παλαιών ημερών; οι πόδες αυτής θέλουσι φέρει αυτήν μακράν διά να παροικήση.
\par 8 Τις εβουλεύθη τούτο κατά της Τύρου, ήτις διανέμει στέμματα, της οποίας οι έμποροι είναι ηγεμόνες, της οποίας οι πραγματευταί είναι οι ένδοξοι της γης;
\par 9 Ο Κύριος των δυνάμεων εβουλεύθη τούτο, διά να καταισχύνη την υπερηφανίαν πάσης δόξης, να εξευτελίση πάντα ένδοξον της γης.
\par 10 Διαπέρασον την γην σου ως ποταμός, θυγάτηρ της Θαρσείς· δύναμις πλέον δεν υπάρχει.
\par 11 Εξέτεινε την χείρα αυτού επί την θάλασσαν, έσεισε βασίλεια· ο Κύριος έδωκε προσταγήν κατά της Χαναάν, διά να καταστρέψωσι τα οχυρώματα αυτής.
\par 12 Και είπε, δεν θέλεις αγάλλεσθαι πλέον, παρθένε κατατεθλιμμένη, θυγάτηρ της Σιδώνος· σηκώθητι, πέρασον προς τους Κητιαίους· ουδέ εκεί θέλεις έχει ανάπαυσιν.
\par 13 Ιδού, η γη των Χαλδαίων· ούτος ο λαός δεν υπήρχεν· ο Ασσύριος εθεμελίωσεν αυτόν διά τους κατοικούντας την έρημον· ήγειραν τους πύργους αυτής, ύψωσαν τα παλάτια αυτής· και κατέστησεν αυτήν ερείπια.
\par 14 Ολολύζετε, πλοία της Θαρσείς· διότι ηρημώθη το οχύρωμά σας.
\par 15 Και εν εκείνη τη ημέρα η Τύρος θέλει λησμονηθή εβδομήκοντα έτη, κατά τας ημέρας ενός βασιλέως· μετά δε τα εβδομήκοντα έτη θέλει είσθαι εν τη Τύρω ως άσμα της πόρνης.
\par 16 Λάβε κιθάραν, περίελθε την πόλιν, πόρνη λησμονημένη, παίζε γλυκά, άδε πολλά άσματα, διά να σε ενθυμηθώσι.
\par 17 Και μετά τα εβδομήκοντα έτη, ο Κύριος θέλει επισκεφθή την Τύρον· και αυτή θέλει επιστρέψει εις το μίσθωμα αυτής, και θέλει πορνεύεσθαι μετά πάντων των βασιλείων του κόσμου επί προσώπου της γης.
\par 18 Και το εμπόριον αυτής και το μίσθωμα αυτής θέλει αφιερωθή εις τον Κύριον· δεν θέλει θησαυρισθή ουδέ ταμιευθή· διότι το εμπόριον αυτής θέλει είσθαι διά τους κατοικούντας ενώπιον του Κυρίου· διά να τρώγωσιν εις χορτασμόν και να έχωσιν ενδύματα πολυχρόνια.

\chapter{24}

\par 1 Ιδού, ο Κύριος κενόνει την γην και ερημόνει αυτήν και ανατρέπει αυτήν και διασκορπίζει τους κατοίκους αυτής.
\par 2 Και θέλει είσθαι, ως ο λαός, ούτως ο ιερεύς· ως ο θεράπων, ούτως ο κύριος αυτού· ως η θεράπαινα, ούτως η κυρία αυτής· ως ο αγοραστής, ούτως ο πωλητής· ως ο δανειστής, ούτως ο δανειζόμενος· ως ο λαμβάνων τόκον, ούτως ο πληρόνων τόκον εις αυτόν.
\par 3 Ολοκλήρως θέλει κενωθή η γη και ολοκλήρως θέλει γυμνωθή· διότι ο Κύριος ελάλησε τον λόγον τούτον.
\par 4 Η γη πενθεί, μαραίνεται, ο κόσμος ατονεί, μαραίνεται, οι υψηλοί εκ των λαών της γης είναι ητονημένοι.
\par 5 Και η γη εμολύνθη υποκάτω των κατοίκων αυτής· διότι παρέβησαν τους νόμους, ήλλαξαν το διάταγμα, ηθέτησαν διαθήκην αιώνιον.
\par 6 Διά τούτο η αρά κατέφαγε την γην και οι κατοικούντες εν αυτή ηρημώθησαν· διά τούτο οι κάτοικοι της γης κατεκαύθησαν και ολίγοι άνθρωποι έμειναν.
\par 7 Ο νέος οίνος πενθεί, η άμπελος είναι εν ατονία, πάντες οι ευφραινόμενοι την καρδίαν στενάζουσιν.
\par 8 Η ευφροσύνη των τυμπάνων παύει· ο θόρυβος των ευθυμούντων τελειόνει· παύει της κιθάρας η ευφροσύνη.
\par 9 δεν θέλουσι πίνει οίνον μετά ασμάτων· το σίκερα θέλει είσθαι πικρόν εις τους πίνοντας αυτό.
\par 10 Η πόλις της ερημώσεως ηφανίσθη· πάσα οικία εκλείσθη, ώστε να μη εισέλθη μηδείς.
\par 11 Κραυγή είναι εν ταις οδοίς διά τον οίνον· πάσα ευθυμία παρήλθεν· η χαρά του τόπου έφυγεν.
\par 12 Ερημία έμεινεν εν τη πόλει, και η πύλη εκτυπήθη υπό αφανισμού·
\par 13 όταν γείνη ούτως εν μέσω της γης μεταξύ των λαών, θέλει είσθαι ως τιναγμός ελαίας, ως το σταφυλολόγημα αφού παύση ο τρυγητός.
\par 14 Ούτοι θέλουσιν υψώσει την φωνήν αυτών, θέλουσι ψάλλει διά την μεγαλειότητα του Κυρίου, θέλουσι μεγαλοφωνεί από της θαλάσσης.
\par 15 Διά τούτο δοξάσατε τον Κύριον εν ταις κοιλάσι, το όνομα Κυρίου του Θεού του Ισραήλ εν ταις νήσοις της θαλάσσης.
\par 16 Απ' άκρου της γης ηκούσαμεν άσματα, Δόξα εις τον δίκαιον. Αλλ' εγώ είπα, Ταλαιπωρία μου, ταλαιπωρία μου· ουαί εις εμέ· οι άπιστοι απίστως έπραξαν· ναι, οι άπιστοι πολλά απίστως έπραξαν.
\par 17 Φόβος και λάκκος και παγίς είναι επί σε, κάτοικε της γης.
\par 18 Και ο φεύγων από του ήχου του φόβου θέλει πέσει εις τον λάκκον· και ο αναβαίνων εκ μέσου του λάκκου θέλει πιασθή εις την παγίδα· διότι αι θυρίδες άνωθεν είναι ανοικταί, και τα θεμέλια της γης σείονται.
\par 19 Η γη κατεσυντρίφθη, η γη ολοκλήρως διελύθη, η γη εκινήθη εις υπερβολήν.
\par 20 Η γη θέλει κλονισθή εδώ και εκεί ως ο μεθύων και θέλει μετακινηθή ως καλύβη· και η ανομία αυτής θέλει βαρύνει επ' αυτήν· και θέλει πέσει και πλέον δεν θέλει σηκωθή.
\par 21 Και εν εκείνη τη ημέρα ο Κύριος θέλει παιδεύσει το στράτευμα των υψηλών εν τω ύψει και τους βασιλείς της γης επί της γης.
\par 22 Και θέλουσι συναχθή, καθώς συνάγονται οι αιχμάλωτοι εις τον λάκκον, και θέλουσι κλεισθή εν τη φυλακή, και μετά πολλάς ημέρας θέλει γείνει επίσκεψις εις αυτούς.
\par 23 Τότε η σελήνη θέλει εντραπή και ο ήλιος θέλει αισχυνθή, όταν ο Κύριος των δυνάμεων βασιλεύση εν τω όρει Σιών και εν Ιερουσαλήμ και δοξασθή ενώπιον των πρεσβυτέρων αυτού.

\chapter{25}

\par 1 Κύριε, συ είσαι ο Θεός μου· θέλω σε υψόνει, θέλω υμνεί το όνομά σου· διότι έκαμες θαυμάσια· αι απ' αρχής βουλαί σου είναι πίστις και αλήθεια.
\par 2 Διότι συ κατέστησας πόλιν σωρόν· πόλιν ωχυρωμένην, ερείπιον· τα οχυρώματα των αλλογενών, ώστε να μη ήναι πόλις· ουδέποτε θέλουσιν ανοικοδομηθή.
\par 3 Διά τούτο ο ισχυρός λαός θέλει σε δοξάσει, η πόλις των τρομερών εθνών θέλει σε φοβηθή.
\par 4 Διότι εστάθης δύναμις εις τον πτωχόν, δύναμις του ενδεούς εν τη στενοχωρία αυτού, καταφύγιον εναντίον της ανεμοζάλης, σκιά εναντίον του καύσωνος, όταν το φύσημα των τρομερών προσβάλη ως ανεμοζάλη κατά τοίχου.
\par 5 Θέλεις καταπαύσει τον θόρυβον των αλλογενών, ως τον καύσωνα εν ξηρώ τόπω, τον καύσωνα διά της σκιάς του νέφους· ο θρίαμβος των τρομερών θέλει ταπεινωθή.
\par 6 Και επί του όρους τούτου ο Κύριος των δυνάμεων θέλει κάμει εις πάντας τους λαούς ευωχίαν από παχέων, ευωχίαν από οίνων εν τη τρυγία αυτών, από παχέων μεστών μυελού, από οίνων κεκαθαρισμένων επί της τρυγίας.
\par 7 Και εν τω όρει τούτω θέλει αφανίσει το πρόσωπον του περικαλύμματος του περικαλύπτοντος πάντας τους λαούς και το κάλυμμα το καλύπτον επί πάντα τα έθνη.
\par 8 Θέλει καταπίει τον θάνατον εν νίκη· και Κύριος ο Θεός θέλει σπογγίσει τα δάκρυα από πάντων των προσώπων· και θέλει εξαλείψει το όνειδος του λαού αυτού από πάσης της γής· διότι ο Κύριος ελάλησε.
\par 9 Και εν εκείνη τη ημέρα θέλουσιν ειπεί, Ιδού, ούτος είναι ο Θεός ημών· περιεμείναμεν αυτόν και θέλει σώσει ημάς· ούτος είναι ο Κύριος· περιεμείναμεν αυτόν· θέλομεν χαρή και ευφρανθή εν τη σωτηρία αυτού.
\par 10 Διότι εν τω όρει τούτω η χειρ του Κυρίου θέλει αναπαυθή, και ο Μωάβ θέλει καταπατηθή υποκάτω αυτού, καθώς καταπατείται το άχυρον διά τον κοπρώνα.
\par 11 Και θέλει εξαπλώσει τας χείρας αυτού εν τω μέσω αυτών, καθώς ο κολυμβών εξαπλόνει τας χείρας αυτού διά να κολυμβήση· και θέλει ταπεινώσει την υπερηφανίαν αυτών μετά των πανουργευμάτων των χειρών αυτών.
\par 12 Και τα υψηλά οχυρώματα των τειχών σου θέλουσι ταπεινωθή, κρημνισθή, κατεδαφισθή έως εδάφους.

\chapter{26}

\par 1 Εν εκείνη τη ημέρα το άσμα τούτο θέλει ψαλή εν γη Ιούδα Έχομεν πόλιν οχυράν· σωτηρίαν θέλει βάλει ο Θεός αντί τειχών και προτειχισμάτων.
\par 2 Ανοίξατε τας πύλας και θέλει εισέλθει το δίκαιον έθνος το φυλάττον την αλήθειαν.
\par 3 Θέλεις φυλάξει εν τελεία ειρήνη το πνεύμα το επί σε επιστηριζόμενον, διότι επί σε θαρρεί.
\par 4 Θαρρείτε επί τον Κύριον πάντοτε· διότι εν Κυρίω τω Θεώ είναι αιώνιος δύναμις.
\par 5 Διότι ταπεινόνει τους κατοικούντας εν υψηλοίς· κρημνίζει την υψηλήν πόλιν· κρημνίζει αυτήν έως εδάφους· καταβάλλει αυτήν έως χώματος.
\par 6 Ο πους θέλει καταπατήσει αυτήν, οι πόδες του πτωχού, τα βήματα του ενδεούς.
\par 7 Η οδός του δικαίου είναι η ευθύτης· συ, ευθύτατε, σταθμίζεις την οδόν του δικαίου.
\par 8 Ναι, εν τη οδώ, των κρίσεών σου, Κύριε, σε περιεμείναμεν· ο πόθος της ψυχής ημών είναι εις το όνομά σου και εις την ενθύμησίν σου.
\par 9 Με την ψυχήν μου σε επόθησα την νύκτα· ναι, με το πνεύμά μου εντός μου σε εξεζήτησα το πρωΐ· διότι όταν αι κρίσεις σου ήναι εν τη γη, οι κάτοικοι του κόσμου θέλουσι μάθει δικαιοσύνην.
\par 10 Και αν ελεηθή ο ασεβής, δεν θέλει μάθει δικαιοσύνην· εν τη γη της ευθύτητος θέλει πράξει αδίκως και δεν θέλει εμβλέψει εις την μεγαλειότητα του Κυρίου.
\par 11 Η χειρ σου, Κύριε, υψούται, αλλ' αυτοί δεν θέλουσιν ιδεί· θέλουσιν όμως ιδεί και καταισχυνθή· ο ζήλος ο υπέρ του λαού σου, μάλιστα το πυρ το κατά των εχθρών σου θέλει καταφάγει αυτούς.
\par 12 Κύριε, ειρήνην θέλεις δώσει εις ημάς· διότι συ έκαμες και πάντα ημών τα έργα διά ημάς.
\par 13 Κύριε ο Θεός ημών, άλλοι κύριοι, πλην σου, εξουσίασαν εφ' ημάς· αλλά τώρα διά σου μόνον θέλομεν αναφέρει το όνομά σου.
\par 14 Απέθανον, δεν θέλουσιν αναζήσει· ετελεύτησαν, δεν θέλουσιν αναστηθή· διά τούτο επεσκέφθης και εξωλόθρευσας αυτούς και εξήλειψας παν το μνημόσυνον αυτών.
\par 15 Επλήθυνας το έθνος, Κύριε, επλήθυνας το έθνος· εδοξάσθης· εμάκρυνας αυτό εις πάντα τα έσχατα της γης.
\par 16 Κύριε, εν τη θλίψει προσέτρεξαν προς σέ· εξέχεαν στεναγμόν, ότε η παιδεία σου ήτο επ' αυτούς.
\par 17 Ως έγκυος γυνή, όταν πλησιάση εις την γένναν, κοιλοπονεί, φωνάζουσα εν τοις πόνοις αυτής, ούτως εγείναμεν ενώπιόν σου, Κύριε.
\par 18 Συνελάβομεν, εκοιλοπονήσαμεν, πλην ως να εγεννήσαμεν άνεμον· ουδεμίαν ελευθέρωσιν κατωρθώσαμεν εν τη γή· ουδέ έπεσαν οι κάτοικοι του κόσμου.
\par 19 Οι νεκροί σου θέλουσι ζήσει, μετά του νεκρού σώματός μου θέλουσιν αναστηθή· εξεγέρθητε και ψάλλετε, σεις οι κατοικούντες εν τω χώματι· διότι η δρόσος σου είναι ως η δρόσος των χόρτων, και η γη θέλει εκρίψει τους νεκρούς.
\par 20 Ελθέ, λαέ μου, είσελθε εις τα ταμείά σου και κλείσον τας θύρας σου οπίσω σου· κρύφθητι διά ολίγον καιρόν, εωσού παρέλθη η οργή.
\par 21 Διότι, ιδού, ο Κύριος εξέρχεται από του τόπου αυτού διά να παιδεύση τους κατοίκους της γης ένεκεν της ανομίας αυτών· η δε γη θέλει ανακαλύψει τα αίματα αυτής και δεν θέλει σκεπάσει πλέον τους πεφονευμένους αυτής.

\chapter{27}

\par 1 Εν εκείνη τη ημέρα θέλει παιδεύσει ο Κύριος, διά της μαχαίρας αυτού της σκληράς και μεγάλης και δυνατής, τον Λευϊάθαν, τον λοξοβάτην όφιν, ναι, τον Λευϊάθαν, τον σκολιόν όφιν· και θέλει αποκτείνει τον δράκοντα τον εν τη θαλάσση.
\par 2 Εν εκείνη τη ημέρα ψάλλετε προς αυτήν, Άμπελος αγαπητή·
\par 3 εγώ ο Κύριος θέλω φυλάττει αυτήν· κατά πάσαν στιγμήν θέλω ποτίζει αυτήν· διά να μη βλάψη αυτήν μηδείς, νύκτα και ημέραν θέλω φυλάττει αυτήν·
\par 4 οργή δεν είναι εν εμοί· τις ήθελεν αντιτάξει εναντίον μου τριβόλους και ακάνθας εν τη μάχη; ήθελον περάσει διά μέσου αυτών, ήθελον κατακαύσει ταύτα ομού·
\par 5 ή ας πιασθή από της δυνάμεώς μου, διά να κάμη ειρήνην μετ' εμού· και θέλει κάμει μετ' εμού ειρήνην.
\par 6 Εις το ερχόμενον θέλει ριζώσει τον Ιακώβ· ο Ισραήλ θέλει ανθήσει και βλαστήσει και γεμίσει το πρόσωπον της οικουμένης από καρπών.
\par 7 Μήπως επάταξεν αυτόν, καθώς επάταξε τους πατάξαντας αυτόν; ή εθανατώθη κατά τον θάνατον των θανατωθέντων υπ' αυτού;
\par 8 Με μέτρον θέλεις διαμαχήσει μετ' αυτής, όταν αποβάλης αυτήν· συμμετρεί τον σφοδρόν αυτού άνεμον εν τη ημέρα του ανατολικού ανέμου.
\par 9 Όθεν με τούτο θέλει καθαρισθή η ανομία του Ιακώβ· και τούτο θέλει είσθαι άπας ο καρπός, να εξαλειφθή η αμαρτία αυτού, όταν κατασυντρίψη πάντας τους λίθους των βωμών ως λεπτόν κονιορτόν ασβέστου, και τα άλση και τα είδωλα δεν μένωσι πλέον όρθια.
\par 10 Διότι η ωχυρωμένη πόλις θέλει ερημωθή, η κατοικία θέλει παραιτηθή και εγκαταλειφθή ως έρημος· εκεί θέλει βοσκηθή το μοσχάριον και εκεί θέλει αναπαυθή και καταφάγει τους κλάδους αυτής.
\par 11 Όταν οι κλάδοι αυτής ξηρανθώσι, θέλουσιν αποκοπή· αι γυναίκες θέλουσιν ελθεί και κατακαύσει αυτούς· διότι είναι λαός ασύνετος· όθεν ο ποιήσας αυτόν δεν θέλει οικτείρει αυτόν και ο πλάσας αυτόν δεν θέλει ελεήσει αυτόν.
\par 12 Και εν εκείνη τη ημέρα ο Κύριος θέλει εκτινάξει από της διώρυγος του ποταμού έως του ρεύματος της Αιγύπτου, και σεις θέλετε συναχθή καθ' ένα έκαστος, σεις υιοί Ισραήλ.
\par 13 Και εν εκείνη τη ημέρα θέλει σαλπιγχθή μεγάλη σάλπιγξ, και θέλουσιν ελθεί οι καταφθειρόμενοι εν τη γη της Ασσυρίας και οι αποδεδιωγμένοι εν τη γη της Αιγύπτου, και θέλουσι λατρεύσει τον Κύριον επί του όρους του αγίου εν Ιερουσαλήμ.

\chapter{28}

\par 1 Ουαί εις τον στέφανον της υπερηφανίας των μεθύσων του Εφραΐμ, των οποίων η ένδοξος ώραιότης είναι άνθος μαραινόμενον· οίτινες επί της κορυφής των παχειών κοιλάδων κατακυριεύονται υπό του οίνου.
\par 2 Ιδού, ο Κύριος έχει ισχυρόν και δυνατόν όστις ως θόρυβος χαλάζης, ως καταστρεπτικός ανεμοστρόβιλος· ως κατακλυσμός ισχυρών υδάτων πλημμυρούντων, θέλει καταρρίψει εις την γην τα πάντα διά της χειρός αυτού.
\par 3 Ο στέφανος της υπερηφανίας των μεθύσων του Εφραΐμ θέλει καταπατηθή υπό τους πόδας.
\par 4 Και το άνθος της ενδόξου ώραιότητος αυτών, το επί της κορυφής της παχείας κοιλάδος, μαραινόμενον θέλει γείνει ως ο πρώϊμος καρπός προ του θέρους· τον οποίον ο ιδών αυτόν, καθώς λάβη εν τη χειρί αυτού, καταπίνει αυτόν.
\par 5 Εν εκείνη τη ημέρα ο Κύριος των δυνάμεων θέλει είσθαι στέφανος δόξης και διάδημα ώραιότητος εις το υπόλοιπον του λαού αυτού,
\par 6 και πνεύμα κρίσεως εις τον καθήμενον διά κρίσιν, και δύναμις εις τους απωθούντας τον πόλεμον έως των πυλών.
\par 7 Πλην και αυτοί επλανήθησαν υπό οίνου και παρεδρόμησαν υπό σίκερα· ο ιερεύς και ο προφήτης επλανήθησαν υπό σίκερα, κατεπόθησαν υπό οίνου, παρεδρόμησαν υπό σίκερα· πλανώνται εν τη δράσει, προσκόπτουσιν εν τη κρίσει.
\par 8 Διότι πάσαι αι τράπεζαι είναι πλήρεις εμετού και ακαθαρσίας, ουδείς τόπος μένει καθαρός.
\par 9 Τίνα θέλει διδάξει την σοφίαν; και τίνα θέλει κάμει να καταλάβη την διδασκαλίαν; αυτοί είναι ως βρέφη απογεγαλακτισμένα, απεσπασμένα από των μαστών.
\par 10 Διότι με διδασκαλίαν επί διδασκαλίαν, με διδασκαλίαν επί διδασκαλίαν, με στίχον επί στίχον, στίχον επί στίχον, ολίγον εδώ, ολίγον εκεί,
\par 11 διότι με χείλη ψελλίζοντα και με άλλην γλώσσαν θέλει ομιλεί προς τούτον τον λαόν·
\par 12 προς τον οποίον είπεν, Αύτη είναι η ανάπαυσις, με την οποίαν δύνασθε να αναπαύσητε τον κεκοπιασμένον, και αύτη είναι η άνεσις· αλλ' αυτοί δεν ηθέλησαν να ακούσωσι.
\par 13 Και ο λόγος του Κυρίου θέλει είσθαι προς αυτούς διδασκαλία επί διδασκαλίαν, διδασκαλία επί διδασκαλίαν, στίχος επί στίχον, στίχος επί στίχον, ολίγον εδώ, ολίγον εκεί· διά να περιπατήσωσι και να προσκόπτωσιν εις τα οπίσω και να συντριφθώσι και να παγιδευθώσι και να πιασθώσι.
\par 14 Διά τούτο ακούσατε τον λόγον του Κυρίου, άνθρωποι χλευασταί, οι οδηγούντες τούτον τον λαόν τον εν Ιερουσαλήμ.
\par 15 Επειδή είπετε, Ημείς εκάμομεν συνθήκην μετά του θανάτου και συνεφωνήσαμεν μετά του άδου· όταν η μάστιξ πλημμυρούσα διαβαίνη, δεν θέλει ελθεί εις ημάς· διότι εκάμομεν καταφύγιον ημών το ψεύδος και υπό την ψευδοσύνην θέλομεν κρυφθή·
\par 16 διά τούτο ούτω λέγει Κύριος ο Θεός· Ιδού, θέτω εν τη Σιών θεμέλιον, λίθον, λίθον εκλεκτόν, έντιμον ακρογωνιαίον, θεμέλιον ασφαλές· ο πιστεύων επ' αυτόν δεν θέλει καταισχυνθή.
\par 17 Και θέλω βάλει την κρίσιν εις τον κανόνα και την δικαιοσύνην εις την στάθμην· και η χάλαζα θέλει εξαφανίσει το καταφύγιον του ψεύδους, και τα ύδατα θέλουσι πλημμυρίσει τον κρυψώνα.
\par 18 Και η μετά του θανάτου συνθήκη σας θέλει ακυρωθή, και μετά του άδου συμφωνία σας δεν θέλει σταθή· όταν η πλημμυρούσα μάστιξ διαβαίνη, τότε θέλετε καταπατηθή υπ' αυτής.
\par 19 Ευθύς όταν διαβή, θέλει σας πιάσει· διότι καθ' εκάστην πρωΐαν θέλει διαβαίνει ημέραν και νύκτα· και μόνον το να ακούση τις την βοήν, θέλει είσθαι φρίκη.
\par 20 Διότι η κλίνη είναι μικροτέρα παρά ώστε να δύναταί τις να εξαπλωθή· και το σκέπασμα στενώτερον παρά ώστε να δύναται να περιτυλιχθή.
\par 21 Διότι ο Κύριος θέλει σηκωθή ως εν τω όρει Φερασείμ, θέλει θυμωθή ως εν τη κοιλάδι του Γαβαών, διά να ενεργήση το έργον αυτού, το παράδοξον έργον αυτού, και να εκτελέση την πράξιν αυτού, την εξαίσιον πράξιν αυτού.
\par 22 Τώρα λοιπόν μη ήσθε χλευασταί, διά να μη γείνωσι δυνατώτερα τα δεσμά σας· διότι εγώ ήκουσα παρά Κυρίου του Θεού των δυνάμεων συντέλειαν και απόφασιν επί πάσαν την γην.
\par 23 Ακροάσθητε και ακούσατε την φωνήν μου· προσέξατε και ακούσατε τον λόγον μου.
\par 24 Ο αροτριών μήπως όλην την ημέραν αροτριά διά να σπείρη, διανοίγων και βωλοκοπών τον αγρόν αυτού;
\par 25 Αφού εξομαλύνη το πρόσωπον αυτού, δεν διασκορπίζει τον άρακον και διασπείρει το κύμινον και βάλλει τον σίτον εις το καλήτερον μέρος και την κριθήν εις τον διωρισμένον αυτής τόπον και την βρίζαν εις το μέρος αυτού το ανήκον;
\par 26 Διότι ο Θεός αυτού μανθάνει αυτόν να διακρίνη, και διδάσκει αυτόν.
\par 27 Διότι δεν αλωνίζεται ο άρακος διά αλωνιστικού οργάνου, ουδέ αμάξης τροχός περιστρέφεται επί το κύμινον· αλλά διά ράβδου κτυπάται ο άρακος και διά βακτηρίας το κύμινον.
\par 28 Ο δε σίτος του άρτου κατασυντρίβεται· αλλά δεν θέλει διά πάντα αλωνίζει αυτόν, ουδέ θέλει συντρίψει αυτόν διά του τροχού της αμάξης αυτού, ουδέ θέλει λεπτύνει αυτόν διά των ονύχων των ίππων αυτού.
\par 29 Και τούτο εξήλθε παρά του Κυρίου των δυνάμεων, του θαυμαστού εν βουλή, του μεγάλου εν συνέσει.

\chapter{29}

\par 1 Ουαί εις την Αριήλ, την Αριήλ, την πόλιν όπου κατώκησεν ο Δαβίδ· προσθέσατε ενιαυτόν επί ενιαυτόν· ας σφάζωσιν εορταστικάς θυσίας.
\par 2 Αλλ' εγώ θέλω στενοχωρήσει την Αριήλ, και εκεί θέλει είσθαι βάρος και θλίψις· και εις εμέ θέλει είσθαι ως Αριήλ.
\par 3 Και θέλω στρατοπεδεύσει εναντίον σου κύκλω, και θέλω στήσει πολιορκίαν κατά σου με χαράκωμα, και θέλω ανεγείρει φρούρια εναντίον σου.
\par 4 Και θέλεις ριφθή κάτω, θέλεις λαλεί από του εδάφους και η λαλιά σου θέλει είσθαι ταπεινή εκ του χώματος, και η φωνή σου εκ του εδάφους θέλει είσθαι ως του εγγαστριμύθου και η λαλιά σου θέλει ψιθυρίζει εκ του χώματος.
\par 5 Το δε πλήθος των εχθρών σου θέλει είσθαι ως κονιορτός και το πλήθος των φοβερών ως άχυρον φερόμενον υπό ανέμου· ναι, τούτο θέλει γείνει αιφνιδίως εν μιά στιγμή.
\par 6 Θέλει γείνει εις σε επίσκεψις παρά του Κυρίου των δυνάμεων, μετά βροντής και μετά σεισμού και φωνής μεγάλης, μετά ανεμοζάλης και ανεμοστροβίλου και φλογός πυρός κατατρώγοντος.
\par 7 Και το πλήθος πάντων των εθνών των πολεμούντων εναντίον της Αριήλ, πάντες βεβαίως οι μαχόμενοι εναντίον αυτής και των οχυρωμάτων αυτής και οι στενοχωρούντες αυτήν θέλουσιν είσθαι ως όνειρον νυκτερινού οράματος.
\par 8 Καθώς μάλιστα ο πεινών ονειρεύεται ότι ιδού, τρώγει· πλην εξεγείρεται και η ψυχή αυτού είναι κενή· ή καθώς ο διψών ονειρεύεται ότι ιδού, πίνει· πλην εξεγείρεται και ιδού, είναι ητονημένος και η ψυχή αυτού διψά· ούτω θέλουσιν είσθαι τα πλήθη πάντων των εθνών των πολεμούντων εναντίον του όρους Σιών.
\par 9 Στήτε και θαυμάσατε· αναβοήσατε και ανακράξατε· ούτοι μεθύουσιν αλλ' ουχί υπό οίνου· παραφέρονται αλλ' ουχί υπό σίκερα.
\par 10 Διότι ο Κύριος εξέχεεν εφ' υμάς πνεύμα βαθέος ύπνου και έκλεισε τους οφθαλμούς υμών· περιεκάλυψε τους προφήτας και τους άρχοντας υμών, τους βλέποντας οράσεις.
\par 11 Και πάσα όρασις θέλει είσθαι εις εσάς ως λόγια εσφραγισμένου βιβλίου, το οποίον ήθελον δώσει εις τινά εξεύροντα να αναγινώσκη, λέγοντες, Ανάγνωθι τούτο, παρακαλώ· και εκείνος λέγει, Δεν δύναμαι, διότι είναι εσφραγισμένον·
\par 12 και δίδουσι το βιβλίον εις μη εξεύροντα να αναγινώσκη και λέγουσιν, Ανάγνωθι τούτο, παρακαλώ· και εκείνος λέγει, δεν εξεύρω να αναγινώσκω.
\par 13 Διά τούτο ο Κύριος λέγει, Επειδή ο λαός ούτος με πλησιάζει διά του στόματος αυτού και με τιμά διά των χειλέων αυτού, αλλ' η καρδία αυτού απέχει μακράν απ' εμού, και με σέβονται, διδάσκοντες διδασκαλίας, εντάλματα ανθρώπων·
\par 14 διά τούτο, ιδού, θέλω προσθέσει να κάμω θαυμαστόν έργον μεταξύ τούτου του λαού, θαυμαστόν έργον και εξαίσιον· διότι η σοφία των σοφών αυτού θέλει χαθή και η σύνεσις των συνετών αυτού θέλει κρυφθή.
\par 15 Ουαί εις τους σκάπτοντας βαθέως διά να κρύψωσι την βουλήν αυτών από του Κυρίου, και των οποίων τα έργα είναι εν τω σκότει, και λέγουσι, Τις βλέπει ημάς; και τις εξεύρει ημάς;
\par 16 Ω διεστραμμένοι, ο κεραμεύς θέλει νομισθή ως πηλός; το πλάσμα θέλει ειπεί περί του πλάσαντος αυτό, ούτος δεν με έπλασεν; ή το ποίημα θέλει ειπεί περί του ποιήσαντος αυτό, Ούτος δεν είχε νόησιν;
\par 17 Δεν θέλει είσθαι έτι πολύ ολίγος καιρός και ο Λίβανος θέλει μεταβληθή εις καρποφόρον πεδιάδα, και η καρποφόρος πεδιάς θέλει λογισθή ως δάσος;
\par 18 Και εν εκείνη τη ημέρα οι κωφοί θέλουσιν ακούσει τους λόγους του βιβλίου, και οι οφθαλμοί των τυφλών θέλουσιν ιδεί, ελευθερωθέντες εκ του σκότους και εκ της ομίχλης.
\par 19 Και οι πραείς θέλουσιν επαυξήσει την χαράν αυτών εν Κυρίω, και οι πτωχοί των ανθρώπων θέλουσιν ευφρανθή διά τον Άγιον του Ισραήλ.
\par 20 Διότι ο τρομερός εξέλιπε και ο χλευαστής εξωλοθρεύθη και πάντες οι παραφυλάττοντες την ανομίαν εξηλείφθησαν·
\par 21 οίτινες κάμνουσι τον άνθρωπον πταίστην διά ένα λόγον, και στήνουσι παγίδα εις τον ελέγχοντα εν τη πύλη, και με ψεύδος διαστρέφουσι το δίκαιον.
\par 22 Όθεν ο Κύριος ο λυτρώσας τον Αβραάμ ούτω λέγει περί του οίκου Ιακώβ· ο Ιακώβ δεν θέλει πλέον αισχυνθή, και το πρόσωπον αυτού δεν θέλει πλέον ωχριάσει.
\par 23 Αλλ' όταν ίδη τα τέκνα αυτού, το έργον των χειρών μου, εν μέσω αυτού, θέλουσιν αγιάσει το όνομά μου και θέλουσιν αγιάσει τον Άγιον του Ιακώβ και θέλουσι φοβείσθαι τον Θεόν του Ισραήλ.
\par 24 Οι δε πλανώμενοι κατά το πνεύμα θέλουσιν ελθεί εις σύνεσιν, και οι γογγύζοντες θέλουσι μάθει διδασκαλίαν.

\chapter{30}

\par 1 Ουαί εις τα αποστατήσαντα τέκνα, λέγει Κύριος, τα οποία λαμβάνουσι βουλήν, πλην ουχί παρ' εμού· και τα οποία κάμνουσι συνθήκας, πλην ουχί διά του πνεύματός μου, διά να προσθέσωσιν αμαρτίαν εις αμαρτίαν·
\par 2 τα οποία υπάγουσι διά να καταβώσιν εις Αίγυπτον, και δεν ερωτώσι το στόμα μου, διά να ενδυναμωθώσι με την δύναμιν του Φαραώ και να εμπιστευθώσιν εις την σκιάν της Αιγύπτου.
\par 3 Η δε δύναμις του Φαραώ θέλει είσθαι αισχύνη σας και η πεποίθησις επί την σκιάν της Αιγύπτου όνειδος.
\par 4 Διότι οι αρχηγοί αυτού εστάθησαν εν Τάνει και οι πρέσβεις αυτού ήλθον εις Χανές.
\par 5 Πάντες ησχύνθησαν διά λαόν, όστις δεν ηδυνήθη να ωφελήση αυτούς ουδέ να σταθή βοήθεια ή όφελος αλλά καταισχύνη και μάλιστα όνειδος.
\par 6 Η κατά των ζώων της Μεσημβρίας όρασις. Εν τη γη της θλίψεως και της στενοχωρίας, όπου ευρίσκονται ο δυνατός λέων και ο λέων ο γηραλέος, η έχιδνα και ο φλογερός πτερωτός όφις, εκεί θέλουσι φέρει τα πλούτη αυτών επί ώμων οναρίων και τους θησαυρούς αυτών επί του κυρτώματος των καμήλων, προς λαόν όστις δεν θέλει ωφελήσει αυτούς.
\par 7 Διότι οι Αιγύπτιοι εις μάτην και ανωφελώς θέλουσι βοηθήσει· όθεν εβόησα περί τούτου, Η δύναμις αυτών είναι να κάθηνται ήσυχοι.
\par 8 Τώρα ύπαγε, γράψον τούτο έμπροσθεν αυτών επί πινακιδίου, και σημείωσον αυτό εν βιβλίω, διά να σώζηται εις τον μέλλοντα καιρόν έως αιώνος·
\par 9 ότι ούτος είναι λαός απειθής, ψευδείς υιοί, υιοί μη θέλοντες να ακούσωσι τον νόμον του Κυρίου·
\par 10 οίτινες λέγουσι προς τους βλέποντας, Μη βλέπετε· και προς τους προφήτας, Μη προφητεύετε εις ημάς τα ορθά, λαλείτε προς ημάς κολακευτικά, προφητεύετε απατηλά·
\par 11 αποσύρθητε από της οδού, εκκλίνατε από της τρίβου, σηκώσατε απ' έμπροσθεν ημών τον Άγιον του Ισραήλ.
\par 12 Όθεν ούτω λέγει ο Άγιος του Ισραήλ· Επειδή καταφρονείτε τον λόγον τούτον και ελπίζετε επί την απάτην και πονηρίαν και επιστηρίζεσθε επί ταύτα·
\par 13 διά τούτο η ανομία αύτη θέλει είσθαι εις εσάς ως χάλασμα ετοιμόρροπον, ως κοιλία εις υψηλόν τοίχον, του οποίου ο συντριμμός έρχεται εξαίφνης εν μιά στιγμή.
\par 14 Και θέλει συντρίψει αυτό ως σύντριμμα αγγείου οστρακίνου, κατασυντριβομένου ανηλεώς, ώστε να μη ευρίσκηται εν τοις θρύμμασιν αυτού όστρακον, διά να λάβη τις πυρ από της εστίας ή να λάβη ύδωρ εκ του λάκκου.
\par 15 Διότι ούτω λέγει Κύριος ο Θεός, ο Άγιος του Ισραήλ· Εν τη επιστροφή και αναπαύσει θέλετε σωθή· εν τη ησυχία και πεποιθήσει θέλει είσθαι η δύναμίς σας· αλλά δεν ηθελήσατε·
\par 16 και είπετε, Ουχί· αλλά θέλομεν φεύγει έφιπποι· διά τούτο θέλετε φεύγει· και, Θέλομεν ιππεύσει επί ταχύποδας· διά τούτο οι διώκοντές σας θέλουσιν είσθαι ταχύποδες.
\par 17 Θέλετε φεύγει χίλιοι εν τη απειλή ενός, και πάντες εν τη απειλή πέντε, εωσού μείνητε ως στύλος επί κορυφής όρους και ως σημαία επί λόφου.
\par 18 Και ούτω θέλει προσμείνει ο Κύριος διά να σας ελεήση, και ούτω θέλει υψωθή διά να σας οικτειρήση· διότι ο Κύριος είναι Θεός κρίσεως· μακάριοι πάντες οι προσμένοντες αυτόν.
\par 19 Διότι ο λαός θέλει κατοικήσει εν Σιών εν Ιερουσαλήμ· δεν θέλεις κλαύσει πλέον· θέλει βεβαίως σε ελεήσει εν τη φωνή της κραυγής σου· όταν ακούση αυτήν, θέλει σοι αποκριθή.
\par 20 Και αν ο Κύριος σας δίδη άρτον θλίψεως και ύδωρ στενοχωρίας, οι διδάσκαλοί σου όμως δεν θέλουσιν αφαιρεθή πλέον, αλλ' οι οφθαλμοί σου θέλουσι βλέπει τους διδασκάλους σου·
\par 21 και τα ώτα σου θέλουσιν ακούει λόγον όπισθέν σου, λέγοντα, Αύτη είναι η οδός, περιπατείτε εν αυτή· όταν στρέφησθε επί τα δεξιά και όταν στρέφησθε επί τα αριστερά.
\par 22 Και θέλετε αποστραφή ως μεμιασμένα το επικάλυμμα των αργυρών γλυπτών σας και τον στολισμόν των χρυσών χωνευτών σας· θέλεις απορρίψει αυτά ως ράκος ακάθαρτον· θέλεις ειπεί προς αυτά, Φεύγετε από εδώ.
\par 23 Τότε θέλει δώσει βροχήν διά τον σπόρον σου, τον οποίον ήθελες σπείρει εν τω αγρώ· και άρτον του γεννήματος της γης, όστις θέλει είσθαι παχύς και άφθονος· εν εκείνη τη ημέρα τα κτήνη σου θέλουσι βόσκεσθαι εν ευρυχώροις νομαίς.
\par 24 Και οι βόες και αι νέαι όνοι, τα οποία εργάζονται την γην, θέλουσι τρώγει καθαρόν άχυρον λελικμημένον διά του πτυαρίου και ανεμιστηρίου.
\par 25 Και θέλουσιν είσθαι επί παντός υψηλού όρους και επί παντός υψηλού λόφου, ποταμοί και ρεύματα υδάτων, εν τη ημέρα της μεγάλης σφαγής, όταν οι πύργοι καταπίπτωσι.
\par 26 Το δε φως της σελήνης θέλει είσθαι ως το φως του ηλίου, και το φως του ηλίου θέλει είσθαι επταπλάσιον ως το φως επτά ημερών, εν τη ημέρα καθ' ην ο Κύριος επιδένει το σύντριμμα του λαού αυτού και θεραπεύει την πληγήν του τραυματισμού αυτών.
\par 27 Ιδού, το όνομα του Κυρίου έρχεται μακρόθεν· φλογερός είναι ο θυμός αυτού και το φορτίον βαρύ· τα χείλη αυτού είναι πλήρη αγανακτήσεως και η γλώσσα αυτού ως πυρ κατατρώγον·
\par 28 και η πνοή αυτού ως ρεύμα πλημμυρίζον, φθάνον έως μέσου του τραχήλου, διά να κοσκινίση τα έθνη εν τω κοσκίνω της ματαιώσεως· και θέλει είσθαι εις τας σιαγόνας των λαών χαλινός, όστις θέλει κάμει αυτούς να περιπλανώνται.
\par 29 Εις εσάς θέλει είσθαι άσμα, καθώς εν τη νυκτί πανηγυριζομένης εορτής· και ευφροσύνη καρδίας, καθώς ότε υπάγουσι μετά αυλών διά να έλθωσιν εις το όρος του Κυρίου, προς τον Ισχυρόν του Ισραήλ.
\par 30 Και θέλει κάμει ο Κύριος να ακουσθή η δόξα της φωνής αυτού, και θέλει δείξει την κατάβασιν του βραχίονος αυτού μετά της αγανακτήσεως του θυμού και της φλογός του κατατρώγοντος πυρός, των εκτιναγμών και της ανεμοζάλης και των λίθων της χαλάζης.
\par 31 Διότι ο Ασσύριος διά της φωνής του Κυρίου θέλει καταβληθή· εν ράβδω θέλει κτυπηθή.
\par 32 Και όθεν διαβή η διωρισμένη ράβδος, την οποίαν ο Κύριος θέλει καταφέρει επ' αυτόν, τύμπανα και κιθάραι θέλουσιν είσθαι· και διά πολέμων τρομερών θέλει πολεμήσει κατ' αυτών.
\par 33 Διότι ο Τοφέθ είναι προ καιρού παρεσκευασμένος· ναι, διά τον βασιλέα ητοιμασμένος· αυτός έκαμεν αυτόν βαθύν και πλατύν· η πυρά αυτού είναι πυρ και ξύλα πολλά· η πνοή του Κυρίου ως ρεύμα θείου θέλει εξάψει αυτήν.

\chapter{31}

\par 1 Ουαί εις τους καταβαίνοντας εις Αίγυπτον διά βοήθειαν και επιστηριζομένους επί ίππους και θαρρούντας επί αμάξας, διότι είναι πολυάριθμοι· και επί ιππείς, διότι είναι πολύ δυνατοί· και δεν αποβλέπουσιν εις τον Άγιον του Ισραήλ και τον Κύριον δεν εκζητούσι.
\par 2 Πλην αυτός είναι σοφός και θέλει επιφέρει κακά και δεν θέλει ανακαλέσει τους λόγους αυτού, αλλά θέλει σηκωθή επί τους οίκους των κακοποιών και επί την βοήθειαν των εργαζομένων την ανομίαν.
\par 3 Οι δε Αιγύπτιοι είναι άνθρωποι και ουχί Θεός· και οι ίπποι αυτών σάρκες και ουχί πνεύμα. Όταν ο Κύριος εκτείνη την χείρα αυτού, και ο βοηθών θέλει προσκόψει και ο βοηθούμενος θέλει πέσει και πάντες ομού θέλουσιν απολεσθή.
\par 4 Διότι ούτως ελάλησε Κύριος προς εμέ· Καθώς ο λέων και ο σκύμνος του λέοντος βρυχώμενος επί το θήραμα αυτού, αν και συνήχθη εναντίον αυτού πλήθος βοσκών, δεν πτοείται εις την φωνήν αυτών ουδέ συστέλλεται εις τον θόρυβον αυτών· ούτως ο Κύριος των δυνάμεων θέλει καταβή διά να πολεμήση υπέρ του όρους της Σιών και υπέρ των λόφων αυτής.
\par 5 Ως πτηνά διαπετώμενα επί τους νεοσσούς, ούτως ο Κύριος των δυνάμεων θέλει υπερασπισθή την Ιερουσαλήμ, υπερασπιζόμενος και ελευθερόνων αυτήν, διαβαίνων και σώζων αυτήν.
\par 6 Επιστράφητε προς εκείνον, από του οποίου οι υιοί του Ισραήλ όλως απεστάτησαν.
\par 7 Διότι εν εκείνη τη ημέρα πας άνθρωπος θέλει ρίψει τα αργυρά αυτού είδωλα και τα χρυσά αυτού είδωλα, τα οποία αι χείρές σας κατεσκεύασαν εις εσάς αμαρτίαν.
\par 8 Τότε ο Ασσύριος θέλει πέσει εν μαχαίρα ουχί ανδρός· και μάχαιρα ουχί ανθρώπου θέλει καταφάγει αυτόν· και θέλει φεύγει από προσώπου της μαχαίρας, και οι νέοι αυτού θέλουσιν είσθαι διά φόρον.
\par 9 Και από του φόβου θέλει παραδράμει το οχύρωμα αυτού, και οι αρχηγοί αυτού θέλουσι κατατρομάξει εις την σημαίαν, λέγει Κύριος, του οποίου το πυρ είναι εν Σιών και η κάμινος αυτού εν Ιερουσαλήμ.

\chapter{32}

\par 1 Ιδού, βασιλεύς θέλει βασιλεύσει εν δικαιοσύνη, και άρχοντες θέλουσιν άρχει εν κρίσει.
\par 2 Και ο άνθρωπος θέλει είσθαι ως σκέπη από του ανέμου και ως καταφύγιον από της τρικυμίας· ως ποταμοί ύδατος εν ξηρά γη, ως σκιά μεγάλου βράχου εν γη διψώση.
\par 3 Και οι οφθαλμοί των βλεπόντων δεν θέλουσιν είσθαι εσκοτισμένοι, και τα ώτα των ακουόντων θέλουσιν είσθαι προσεκτικά.
\par 4 Και η καρδία των θρασέων θέλει καταλάβει σοφίαν, και η γλώσσα των τραυλιζόντων θέλει επιταχύνει να λαλή καθαρά.
\par 5 Ο αχρείος δεν θέλει ονομάζεσθαι πλέον ελευθέριος, και ο φιλάργυρος δεν θέλει λέγεσθαι μεγαλοπρεπής.
\par 6 Διότι ο αχρείος θέλει λαλεί αχρεία, και η καρδία αυτού θέλει εργάζεσθαι ανομίαν, διά να εκτελή πονηρίαν και να προφέρη πλάνην εναντίον του Κυρίου, ώστε να στερή την ψυχήν του πεινώντος και να εμποδίζη την πόσιν εις τον διψώντα.
\par 7 Του δε φιλαργύρου τα όπλα είναι άδικα· αυτός βουλεύεται πονηρίας διά να αφανίση τον πτωχόν με λόγους ψευδείς, έτι και όταν ο ενδεής λαλή δίκαια.
\par 8 Αλλ' ο ελευθέριος βουλεύεται ελευθέρια και επί ελευθέρια θέλει στηρίζεσθαι αυτός.
\par 9 Σηκώθητε, γυναίκες εύποροι· ακούσατε την φωνήν μου, θυγατέρες αμέριμνοι· ακροάσθητε τους λόγους μου·
\par 10 ημέρας και έτη θέλετε είσθαι τεταραγμέναι, σεις αι αμέριμνοι· διότι ο τρυγητός θέλει χαθή, η συγκομιδή δεν θέλει ελθεί·
\par 11 τρέμετε, αι εύποροι· ταράχθητε, αι αμέριμνοι· ενδύθητε και γυμνώθητε και περιζώσατε τας οσφύας με σάκκον.
\par 12 Θέλουσι κτυπήσει τα στήθη διά τους ηδονικούς αγρούς, διά τους καρποφόρους αμπελώνας.
\par 13 Άκανθαι και τρίβολοι θέλουσι βλαστήσει επί την γην του λαού μου· έτι και επί πάσας τας οικίας της χαράς εν τη ευφραινομένη πόλει.
\par 14 Διότι τα παλάτια θέλουσιν εγκαταλειφθή· το πλήθος της πόλεως θέλει ερημωθή· τα φρούρια και οι πύργοι θέλουσι κατασταθή έως αιώνος σπήλαια, τρυφή αγρίων όνων, βοσκή ποιμνίων·
\par 15 εωσού το πνεύμα εξ ύψους εκχυθή εφ' ημάς και η έρημος γείνη πεδιάς καρποφόρος, η δε καρποφόρος πεδιάς λογισθή ως δάσος.
\par 16 Τότε κρίσις θέλει κατασκηνώσει εν τη ερήμω και δικαιοσύνη θέλει κατοικήσει εν τη καρποφόρω πεδιάδι.
\par 17 Το δε έργον της δικαιοσύνης θέλει είσθαι ειρήνη· και το αποτέλεσμα της δικαιοσύνης ησυχία και ασφάλεια εις τον αιώνα.
\par 18 Και ο λαός μου θέλει κατοικεί ειρηνικήν κατοικίαν και οικήματα ασφαλή και ησύχους τόπους ευπορίας,
\par 19 και θέλει πίπτει χάλαζα κατασυντρίβουσα το δάσος, και η πόλις με όλεθρον θέλει ανατραπή.
\par 20 Μακάριοι σεις οι σπείροντες πλησίον παντός ύδατος, οι εξαποστέλλοντες εκεί τους πόδας του βοός και της όνου.

\chapter{33}

\par 1 Ουαί εις σε, όστις πορθείς και δεν επορθήθης· και καταδυναστεύεις και δεν κατεδυναστεύθης· όταν παύσης πορθών, θέλεις πορθηθή· όταν τελειώσης καταδυναστεύων, θέλεις καταδυναστευθή.
\par 2 Κύριε, ελέησον ημάς· σε προσμένομεν· έσο βραχίων αυτών καθ' εκάστην πρωΐαν και σωτηρία ημών εν καιρώ θλίψεως.
\par 3 Από της φωνής του θορύβου οι λαοί έφυγον· από της ανυψώσεώς σου τα έθνη διεσκορπίσθησαν.
\par 4 Και τα λάφυρά σας θέλουσι συναχθή, καθώς συνάγουσιν οι βρούχοι· θέλουσι πηδήσει επ' αυτόν, καθώς η ακρίς πηδά εδώ και εκεί.
\par 5 Ο Κύριος υψώθη· διότι κατοικεί εν υψηλοίς· ενέπλησε την Σιών κρίσεως και δικαιοσύνης.
\par 6 Σοφία δε και επιστήμη θέλουσιν είσθαι η στερέωσις των καιρών σου και η σωτήριος δύναμις· ο φόβος του Κυρίου, αυτός είναι ο θησαυρός αυτού.
\par 7 Ιδού, οι ανδρείοι αυτών θέλουσι βοήσει έξωθεν, και οι πρέσβεις της ειρήνης θέλουσι κλαύσει πικρώς.
\par 8 Αι οδοί ηρημώθησαν, οι οδοιπόροι έπαυσαν· διέλυσε την συνθήκην, απέβαλε τας πόλεις, δεν λογίζεται άνθρωπον.
\par 9 Η γη πενθεί, μαραίνεται· ο Λίβανος αισχύνεται, κατακόπτεται· ο Σαρών ομοιάζει έρημον· και η Βασάν και ο Κάρμηλος κατετινάχθησαν.
\par 10 Τώρα θέλω σηκωθή, λέγει Κύριος· τώρα θέλω υψωθή, τώρα θέλω μεγαλυνθή.
\par 11 Χνούν θέλετε συλλάβει και άχυρον θέλετε γεννήσει· η πνοή σας ως πυρ θέλει σας καταφάγει.
\par 12 Και οι λαοί θέλουσιν είσθαι ως καύσεις ασβέστου· ως άκανθαι κεκομμέναι θέλουσι καυθή εν πυρί.
\par 13 Οι μακράν, ακούσατε τι έκαμον· και σεις οι πλησίον, γνωρίσατε την δύναμίν μου.
\par 14 Οι αμαρτωλοί εν Σιών θέλουσι τρομάξει· τρόμος θέλει καταλάβει τους υποκριτάς, ώστε θέλουσι λέγει, Τις μεταξύ ημών θέλει κατοικήσει μετά του κατατρώγοντος πυρός; τις μεταξύ ημών θέλει κατοικήσει μετά των αιωνίων καύσεων;
\par 15 Ο περιπατών εν δικαιοσύνη και ο λαλών εν ευθύτητι· ο καταφρονών το κέρδος των δυναστεύσεων, ο σείων τας χείρας αυτού από δωροληψίας, ο εμφράττων τα ώτα αυτού διά να μη ακούη περί αίματος, και ο κλείων τους οφθαλμούς αυτού διά να μη ίδη το κακόν·
\par 16 ούτος θέλει κατοικήσει εν τοις υψηλοίς· οι τόποι της υπερασπίσεως αυτού θέλουσιν είσθαι τα οχυρώματα των βράχων· άρτος θέλει δοθή εις αυτόν· το ύδωρ αυτού θέλει είσθαι βέβαιον·
\par 17 Οι οφθαλμοί σου θέλουσιν ιδεί τον βασιλέα εν τη ώραιότητι αυτού· θέλουσιν ιδεί την γην την μακράν.
\par 18 Η καρδία σου θέλει μελετά τον παρελθόντα τρόμον, φωνάζουσα, Που είναι ο γραμματεύς; που ο συζητητής; που ο λογιστής των πύργων;
\par 19 δεν θέλεις ιδεί λαόν άγριον, λαόν βαθείας φωνής, ώστε να μη διακρίνης· τραυλιζούσης γλώσσης, ώστε να μη εννοής.
\par 20 Ανάβλεψον εις την Σιών, την πόλιν των εορτών ημών· οι οφθαλμοί σου θέλουσιν ιδεί την Ιερουσαλήμ ήσυχον κατοικίαν, σκηνήν ήτις δεν θέλει καταβιβασθή· οι πάσσαλοι αυτής δεν θέλουσι μετακινηθή εις τον αιώνα και ουδέν εκ των σχοινίων αυτής θέλει κοπή.
\par 21 Αλλ' εκεί ο Κύριος της δόξης θέλει είσθαι εις ημάς τόπος πλατέων ποταμών και ρευμάτων· εκεί δεν θέλει εισέλθει πλοίον διά κωπίων ούτε ναυς μεγαλοπρεπής θέλει περάσει εκείθεν.
\par 22 Διότι ο Κύριος είναι ο κριτής ημών· ο Κύριος είναι ο νομοθέτης ημών· ο Κύριος είναι ο βασιλεύς ημών· αυτός θέλει σώσει ημάς.
\par 23 Τα σχοινία σου εχαυνώθησαν· δεν δύνανται να στερεώσωσι το κατάρτιον αυτών, δεν δύνανται να εξαπλώσωσι τα πανία· τότε λεία μεγάλων λαφύρων θέλει διαμερισθή· οι χωλοί θέλουσι διαρπάσει την λείαν.
\par 24 Και ο κάτοικος δεν θέλει λέγει, Ητόνησα· ο λαός ο κατοικών εν αυτή θέλει λάβει άφεσιν ανομίας.

\chapter{34}

\par 1 Πλησιάσατε, έθνη, διά να ακούσητε· και προσέξατε, λαοί· ας ακούση η γη και το πλήρωμα αυτής· η οικουμένη και πάντα όσα γεννώνται εν αυτή.
\par 2 Διότι ο θυμός του Κυρίου είναι επί πάντα τα έθνη, και η φλογερά οργή αυτού επί πάντα τα στρατεύματα αυτών· κατέστρεψεν αυτά ολοκλήρως· παρέδωκεν αυτά εις σφαγήν.
\par 3 Οι δε πεφονευμένοι αυτών θέλουσι ριφθή έξω, και η δυσωδία αυτών θέλει αναδοθή από των πτωμάτων αυτών· τα δε όρη θέλουσι διαλυθή από του αίματος αυτών.
\par 4 Και πάσα η στρατιά του ουρανού θέλει λυώσει, και οι ουρανοί θέλουσι περιτυλιχθή ως βιβλίον, και πάσα η στρατιά αυτών θέλει πέσει, καθώς πίπτει το φύλλον από της αμπέλου και καθώς πίπτουσι τα φύλλα από της συκής.
\par 5 Διότι η μάχαιρά μου εμεθύσθη εν τω ουρανώ· ιδού, επί την Ιδουμαίαν και επί τον λαόν της καταστροφής μου θέλει καταβή διά κρίσιν.
\par 6 Η μάχαιρα του Κυρίου είναι πλήρης αίματος· επαχύνθη με το πάχος, με το αίμα των αρνίων και τράγων, με το πάχος των νεφρών των κριών· διότι ο Κύριος έχει θυσίαν εν Βοσόρρα και σφαγήν μεγάλην εν τη γη της Ιδουμαίας.
\par 7 Και οι μονόκεροι θέλουσι καταβή μετ' αυτών και οι μόσχοι μετά των ταύρων· και η γη αυτών θέλει μεθυσθή από αίματος, και το χώμα αυτών θέλει παχυνθή από πάχους.
\par 8 Διότι είναι ημέρα εκδικήσεως του Κυρίου, ενιαυτός ανταποδόσεων διά την κρίσιν της Σιών.
\par 9 Και τα ρεύματα αυτής θέλουσι μεταβληθή εις πίσσαν και το χώμα αυτής εις θείον, και η γη αυτής θέλει κατασταθή πίσσα καιομένη·
\par 10 νύκτα και ημέραν δεν θέλει σβεσθή· ο καπνός αυτής θέλει αναβαίνει ακαταπαύστως· από γενεάς εις γενεάν θέλει μένει ηρημωμένη· και δεν θέλει υπάρχει ο διαβαίνων δι' αυτής εις αιώνα αιώνος.
\par 11 Αλλ' ο πελεκάν και ο ακανθόχοιρος θέλουσι κληρονομήσει αυτήν· και η γλαύξ και ο κόραξ θέλουσι κατοικεί εν αυτή· και ο Κύριος θέλει εξαπλώσει επ' αυτής, σχοινίον ερημώσεως και στάθμην κρημνισμού.
\par 12 Θέλουσι καλέσει εις την βασιλείαν τους μεγιστάνας αυτής αλλ' ουδείς θέλει είσθαι εκεί· και πάντες οι άρχοντες αυτής θέλουσιν ελθεί εις το μηδέν.
\par 13 Και άκανθαι θέλουσι βλαστήσει εν τοις παλατίοις αυτής, κνίδαι και βάτοι εν τοις οχυρώμασιν αυτής· και θέλει είσθαι κατοικία θώων, αυλή στρουθοκαμήλων.
\par 14 Και οι λύκοι θέλουσι συναπαντάσθαι εκεί με τους αιλούρους· και ο σάτυρος θέλει φωνάζει προς τον σύντροφον αυτού· ο κόκκυξ έτι θέλει αναπαύεσθαι εκεί, ευρίσκων εις εαυτόν τόπον αναπαύσεως.
\par 15 Εκεί θέλει εμφωλεύει ο νυκτοκόραξ και θέλει γεννά και επωάζει και συνάγει τους νεοσσούς υπό την σκιάν αυτού· εκεί θέλουσι συνάγεσθαι και οι γύπες, έκαστος μετά του συντρόφου αυτού.
\par 16 Ζητήσατε εν τω βιβλίω του Κυρίου και αναγνώσατε· ουδέν εκ τούτων θέλει λείψει, ουδέν θέλει είσθαι χωρίς του συντρόφου αυτού· διότι αυτό το στόμα του Κυρίου προσέταξε, και αυτό το πνεύμα αυτού συνήγαγε ταύτα.
\par 17 Και αυτός έρριψε τον κλήρον περί αυτών, και η χειρ αυτού διεμοίρασεν εις αυτά με στάθμην εκείνην την γήν· θέλουσι κληρονομήσει αυτήν εις τον αιώνα· από γενεάς εις γενεάν θέλουσι κατοικεί εν αυτή.

\chapter{35}

\par 1 Η έρημος και η άνυδρος θέλουσιν ευφρανθή δι' αυτά, και η ερημία θέλει αγαλλιασθή και ανθήσει ως ρόδον.
\par 2 Θέλει ανθήσει εν αφθονία και αγαλλιασθή μάλιστα χαίρουσα και αλαλάζουσα· η δόξα του Λιβάνου θέλει δοθή εις αυτήν, η τιμή του Καρμήλου και Σαρών· οι τόποι ούτοι θέλουσιν ιδεί την δόξαν του Κυρίου και την μεγαλωσύνην του Θεού ημών.
\par 3 Ενισχύσατε τας κεχαυνωμένας χείρας· και στερεώσατε τα παραλελυμένα γόνατα.
\par 4 Είπατε προς τους πεφοβισμένους την καρδίαν, Ισχύσατε, μη φοβείσθε· ιδού, ο Θεός σας θέλει ελθεί μετ' εκδικήσεως, ο Θεός μετά ανταποδόσεως· αυτός θέλει ελθεί και θέλει σας σώσει.
\par 5 Τότε οι οφθαλμοί των τυφλών θέλουσιν ανοιχθή και τα ώτα των κωφών θέλουσιν ακούσει.
\par 6 Τότε ο χωλός θέλει πηδά ως έλαφος και η γλώσσα του μογιλάλου θέλει ψάλλει· διότι εν τη ερήμω θέλουσιν αναβλύσει ύδατα και ρεύματα εν τη ερημία.
\par 7 Και η ξηρά γη θέλει κατασταθή λίμνη και η διψώσα γη πηγαί ύδατος· εν τη κατοικία των θώων, όπου εκοίτοντο, θέλει είσθαι χλόη μετά καλάμων και σπάρτων.
\par 8 Και εκεί θέλει είσθαι λεωφόρος και οδός και θέλει ονομασθή, Οδός αγία· ο ακάθαρτος δεν θέλει περάσει δι' αυτής αλλά θέλει είσθαι δι' αυτούς ο οδεύων και οι μωροί δεν θέλουσι πλανάσθαι.
\par 9 Λέων δεν θέλει είσθαι εκεί και θηρίον αρπακτικόν δεν θέλει αναβή εκεί· δεν θέλει ευρεθή εκεί· αλλά οι λελυτρωμένοι θέλουσι περιπατεί εκεί.
\par 10 Και οι λελυτρωμένοι του Κυρίου θέλουσιν επιστρέψει και ελθεί εν αλαλαγμώ εις την Σιών· και ευφροσύνη αιώνιος θέλει είσθαι επί της κεφαλής αυτών· αγαλλίασιν και ευφροσύνην θέλουσιν απολαύσει· η λύπη δε και ο στεναγμός θέλουσι φύγει.

\chapter{36}

\par 1 Εν τω δεκάτω τετάρτω έτει του βασιλέως Εζεκίου ανέβη Σενναχειρείμ ο βασιλεύς της Ασσυρίας επί πάσας τας οχυράς πόλεις του Ιούδα και εκυρίευσεν αυτάς.
\par 2 Και απέστειλεν ο βασιλεύς της Ασσυρίας τον Ραβ-σάκην από Λαχείς εις Ιερουσαλήμ, προς τον βασιλέα Εζεκίαν, μετά δυνάμεως μεγάλης. Και εστάθη εν τω υδραγωγώ, της άνω κολυμβήθρας εν τη μεγάλη οδώ του αγρού του γναφέως.
\par 3 Τότε εξήλθον προς αυτόν Ελιακείμ, ο υιός του Χελκίου, ο οικονόμος, και Σομνάς ο γραμματεύς και Ιωάχ, ο υιός του Ασάφ, ο υπομνηματογράφος.
\par 4 Και είπε προς αυτούς ο Ραβ-σάκης, Είπατε τώρα προς τον Εζεκίαν, Ούτω λέγει ο βασιλεύς ο μέγας, ο βασιλεύς της Ασσυρίας· Ποίον είναι το θάρρος, επί το οποίον θαρρείς;
\par 5 Λέγεις, πλην είναι λόγοι χειλέων, Έχω βουλήν και δύναμιν διά πόλεμον. Αλλ' επί τίνα θαρρείς ώστε απεστάτησας εναντίον μου;
\par 6 Ιδού, θαρρείς επί την ράβδον του συντετριμμένου εκείνου καλάμου, επί την Αίγυπτον· επί του οποίου εάν τις επιστηριχθή, θέλει εμπηχθή εις την χείρα αυτού και τρυπήσει αυτήν· τοιούτος είναι ο Φαραώ ο βασιλεύς της Αιγύπτου προς πάντας τους θαρρούντας επ' αυτόν.
\par 7 Αλλ' εάν είπης προς εμέ, Επί Κύριον τον Θεόν ημών θαρρούμεν, δεν είναι αυτός, του οποίου τους υψηλούς τόπους και τα θυσιαστήρια αφήρεσεν ο Εζεκίας και είπε προς τον Ιούδαν και προς την Ιερουσαλήμ, Έμπροσθεν τούτου του θυσιαστηρίου θέλετε προσκυνήσει;
\par 8 Τώρα λοιπόν δος ενέχυρα εις τον κύριόν μου τον βασιλέα της Ασσυρίας, και εγώ θέλω σοι δώσει δισχιλίους ίππους, αν δύνασαι από μέρους σου να δώσης επιβάτας επ' αυτούς.
\par 9 Πως λοιπόν θέλεις τρέψει οπίσω το πρόσωπον ενός τοπάρχου εκ των ελαχίστων δούλων του κυρίου μου, και ήλπισας επί την Αίγυπτον διά αμάξας και διά ιππείς;
\par 10 Και τώρα, άνευ του Κυρίου ανέβην εγώ επί τον τόπον τούτον, διά να καταστρέψω αυτόν; ο Κύριος είπε προς εμέ, Ανάβα επί την γην ταύτην και κατάστρεψον αυτήν.
\par 11 Τότε είπεν ο Ελιακείμ και ο Σομνάς και ο Ιωάχ προς τον Ραβ-σάκην, Λάλησον, παρακαλώ, προς τους δούλους σου εις την Συριακήν γλώσσαν· διότι καταλαμβάνομεν αυτήν· και μη λάλει προς ημάς Ιουδαϊστί εις επήκοον του λαού του επί του τείχους.
\par 12 Αλλ' ο Ραβ-σάκης είπε, Μήπως ο κύριός μου απέστειλεν εμέ προς τον κύριόν σου και προς σε, διά να λαλήσω τους λόγους τούτους; δεν με απέστειλε προς τους άνδρας τους καθημένους επί του τείχους διά να φάγωσι την κόπρον αυτών και να πίωσι το ούρον αυτών με σας;
\par 13 Τότε ο Ραβ-σάκης εστάθη και εφώνησεν Ιουδαϊστί μετά φωνής μεγάλης και είπεν, Ακούσατε τους λόγους του βασιλέως του μεγάλου, του βασιλέως της Ασσυρίας·
\par 14 ούτω λέγει ο βασιλεύς· Μη σας απατά ο Εζεκίας· διότι δεν θέλει δυνηθή να σας λυτρώση.
\par 15 Και μη σας κάμνη ο Εζεκίας να θαρρήτε επί τον Κύριον, λέγων, Ο Κύριος βεβαίως θέλει μας λυτρώσει· η πόλις αύτη δεν θέλει παραδοθή εις την χείρα του βασιλέως της Ασσυρίας.
\par 16 Μη ακούετε του Εζεκίου· διότι ούτω λέγει ο βασιλεύς της Ασσυρίας· Κάμετε συμβιβασμόν μετ' εμού και εξέλθετε προς εμέ· και φάγετε έκαστος από της αμπέλου αυτού και έκαστος από της συκής αυτού και πίετε έκαστος από των υδάτων της δεξαμενής αυτού·
\par 17 εωσού έλθω και σας λάβω εις γην ομοίαν με την γην σας, γην σίτου και οίνου, γην άρτου και αμπελώνων.
\par 18 Μη σας απατά ο Εζεκίας, λέγων, Ο Κύριος θέλει μας λυτρώσει. Ελύτρωσέ τις εκ των θεών των εθνών την γην αυτού εκ της χειρός του βασιλέως της Ασσυρίας;
\par 19 Που οι θεοί της Αιμάθ και Αρφάδ; που οι θεοί της Σεφαρουΐμ; μήπως ελύτρωσαν εκ της χειρός μου την Σαμάρειαν;
\par 20 Τίνες μεταξύ πάντων των θεών των τόπων τούτων ελύτρωσαν την γην αυτών εκ της χειρός μου, ώστε και ο Κύριος να λυτρώση την Ιερουσαλήμ εκ της χειρός μου;
\par 21 Εκείνοι δε εσιώπων και δεν απεκρίθησαν λόγον προς αυτόν· διότι ο βασιλεύς είχε προστάξει, λέγων, Μη αποκριθήτε προς αυτόν.
\par 22 Τότε Ελιακείμ ο υιός του Χελκίου, ο οικονόμος, και Σομνάς ο γραμματεύς, και Ιωάχ ο υιός του Ασάφ, ο υπομνηματογράφος, ήλθον προς τον Εζεκίαν με διεσχισμένα ιμάτια και απήγγειλαν προς αυτόν τους λόγους του Ραβ-σάκη.

\chapter{37}

\par 1 Και ότε ήκουσεν ο βασιλεύς Εζεκίας, διέσχισε τα ιμάτια αυτού και εσκεπάσθη με σάκκον και εισήλθεν εις τον οίκον του Κυρίου.
\par 2 Και απέστειλεν Ελιακείμ τον οικονόμον και Σομνάν τον γραμματέα και τους πρεσβυτέρους των ιερέων εσκεπασμένους με σάκκους, προς τον προφήτην Ησαΐαν, τον υιόν του Αμώς·
\par 3 και είπον προς αυτόν, Ούτω λέγει ο Εζεκίας· Ημέρα θλίψεως και ονειδισμού και βλασφημίας, η ημέρα αύτη· διότι τα τέκνα ήλθον εις την ακμήν της γέννας, πλην δύναμις δεν είναι εις την τίκτουσαν·
\par 4 είθε να ήκουσε Κύριος ο Θεός σου τους λόγους του Ραβ-σάκη, τον οποίον ο βασιλεύς της Ασσυρίας ο κύριος αυτού απέστειλε διά να ονειδίση τον ζώντα Θεόν, και να υβρίση διά των λόγων, τους οποίους ήκουσε Κύριος ο Θεός σου· διά τούτο ύψωσον δέησιν υπέρ του υπολοίπου του σωζομένου.
\par 5 Και ήλθον προς τον Ησαΐαν οι δούλοι του βασιλέως Εζεκίου.
\par 6 Και είπε προς αυτούς ο Ησαΐας, Ούτω θέλετε ειπεί προς τον κύριόν σας· Ούτω λέγει Κύριος· Μη φοβού από των λόγων, τους οποίους ήκουσας, διά των οποίων οι δούλοι του βασιλέως της Ασσυρίας με ωνείδισαν·
\par 7 ιδού, εγώ θέλω βάλει εις αυτόν τοιούτον πνεύμα, ώστε ακούσας θόρυβον θέλει επιστρέψει εις την γην αυτού· και θέλω κάμει αυτόν να πέση διά μαχαίρας εν τη γη αυτού.
\par 8 Ο Ραβ-σάκης λοιπόν επέστρεψε και εύρηκε τον βασιλέα της Ασσυρίας πολεμούντα εναντίον της Λιβνά· διότι ήκουσεν ότι έφυγεν από Λαχείς.
\par 9 Και ο βασιλεύς ήκουσε να λέγωσι περί Θιρακά του βασιλέως της Αιθιοπίας, Εξήλθε να σε πολεμήση. Και ότε ήκουσε τούτο, απέστειλε πρέσβεις προς τον Εζεκίαν, λέγων,
\par 10 Ούτω θέλετε ειπεί προς Εζεκίαν, τον βασιλέα του Ιούδα, λέγοντες, Ο Θεός σου, επί τον οποίον θαρρείς, ας μη σε απατά, λέγων, Η Ιερουσαλήμ δεν θέλει παραδοθή εις την χείρα του βασιλέως της Ασσυρίας.
\par 11 Ιδού, συ ήκουσας τι έκαμον οι βασιλείς της Ασσυρίας εις πάντας τους τόπους, καταστρέφοντες αυτούς· και συ θέλεις λυτρωθή;
\par 12 Μήπως οι θεοί των εθνών ελύτρωσαν εκείνους, τους οποίους οι πατέρες μου κατέστρεψαν, την Γωζάν και την Χαρράν και Ρεσέφ και τους υιούς του Εδέν, τους εν Τελασσάρ;
\par 13 Που ο βασιλεύς της Αιμάθ και ο βασιλεύς της Αρφάδ και ο βασιλεύς της πόλεως Σεφαρουΐμ, Ενά και Αυά;
\par 14 Και λαβών ο Εζεκίας την επιστολήν εκ της χειρός των πρέσβεων ανέγνωσεν αυτήν· και ανέβη ο Εζεκίας εις τον οίκον του Κυρίου και εξετύλιξεν αυτήν ενώπιον του Κυρίου.
\par 15 Και προσηυχήθη εις τον Κύριον ο Εζεκίας λέγων,
\par 16 Κύριε των δυνάμεων, Θεέ του Ισραήλ, ο καθήμενος επί των χερουβείμ, συ αυτός είσαι ο Θεός, ο μόνος, πάντων των βασιλείων της γής· συ έκαμες τον ουρανόν και την γην.
\par 17 Κλίνον, Κύριε, το ους σου και άκουσον· άνοιξον, Κύριε, τους οφθαλμούς σου και ιδέ· και άκουσον πάντας τους λόγους του Σενναχειρείμ, όστις απέστειλε τούτον διά να ονειδίση τον ζώντα Θεόν.
\par 18 Αληθώς, Κύριε, οι βασιλείς της Ασσυρίας ηρήμωσαν πάντα τα έθνη και τους τόπους αυτών,
\par 19 και έρριψαν εις το πυρ τους θεούς αυτών· διότι δεν ήσαν θεοί, αλλ' έργον χειρών ανθρώπων, ξύλα και λίθοι· διά τούτο κατέστρεψαν αυτούς.
\par 20 Τώρα λοιπόν, Κύριε Θεέ ημών, σώσον ημάς εκ της χειρός αυτού· διά να γνωρίσωσι πάντα τα βασίλεια της γης, ότι συ είσαι ο Κύριος, ο μόνος.
\par 21 Τότε απέστειλεν Ησαΐας ο υιός του Αμώς προς Εζεκίαν, λέγων, Ούτω λέγει Κύριος ο Θεός του Ισραήλ· Ήκουσα όσα προσηυχήθης εις εμέ κατά του Σενναχειρείμ, βασιλέως της Ασσυρίας.
\par 22 Ούτος είναι ο λόγος, τον οποίον ο Κύριος ελάλησε περί αυτού· Σε κατεφρόνησε, σε ενέπαιξεν η παρθένος, θυγάτηρ της Σιών· οπίσω σου έσεισε κεφαλήν η θυγάτηρ της Ιερουσαλήμ.
\par 23 Τίνα ωνείδισας και εβλασφήμησας; και κατά τίνος ύψωσας φωνήν και εσήκωσας υψηλά τους οφθαλμούς σου; κατά του Αγίου του Ισραήλ.
\par 24 Τον Κύριον ωνείδισας διά των δούλων σου και είπας, Με το πλήθος των αμαξών μου ανέβην εγώ εις το ύψος των ορέων, εις τα πλευρά του Λιβάνου· και θέλω κόψει τας υψηλάς κέδρους αυτού, τας εκλεκτάς ελάτους αυτού· και θέλω εισέλθει εις το ύψος των άκρων αυτού, εις το δάσος του Καρμήλου αυτού·
\par 25 εγώ ανέσκαψα και έπιον ύδατα· και με το ίχνος των ποδών μου εξήρανα πάντας τους ποταμούς των πολιορκουμένων.
\par 26 Μη δεν ήκουσας ότι εγώ έκαμον τούτο παλαιόθεν και από ημερών αρχαίων εβουλεύθην αυτό; τώρα δε εξετέλεσα τούτο, ώστε να ήσαι διά να καταστρέφης πόλεις ωχυρωμένας εις ερειπίων σωρούς·
\par 27 διά τούτο οι κάτοικοι αυτών ήσαν μικράς δυνάμεως, ετρόμαξαν και κατησχύνθησαν· ήσαν ως ο χόρτος του αγρού και ως η χλόη, ως ο χόρτος των δωμάτων και ως ο σίτος ο καιόμενος πριν καλαμώση.
\par 28 Πλην εγώ εξεύρω την κατοικίαν σου και την έξοδόν σου και την είσοδόν σου και την κατ' εμού λύσσαν σου.
\par 29 Επειδή η κατ' εμού λύσσα σου και η αλαζονεία σου ανέβησαν εις τα ώτα μου, διά τούτο θέλω βάλει τον κρίκον μου εις τους μυκτήράς σου και τον χαλινόν μου εις τα χείλη σου, και θέλω σε επιστρέψει διά της οδού δι' ης ήλθες.
\par 30 Και τούτο θέλει είσθαι εις σε το σημείον· το έτος τούτο θέλετε φάγει ό,τι είναι αυτοφυές· και το δεύτερον έτος, ό,τι εκφύεται από του αυτού· το δε τρίτον έτος, σπείρατε και θερίσατε και φυτεύσατε αμπελώνας και φάγετε τον καρπόν αυτών.
\par 31 Και το υπόλοιπον εκ του οίκου Ιούδα, το διασωθέν, θέλει ριζώσει πάλιν υποκάτωθεν και θέλει δώσει επάνω καρπούς.
\par 32 Διότι εξ Ιερουσαλήμ θέλει εξέλθει το υπόλοιπον και εκ του όρους Σιών το διασωθέν· ο ζήλος του Κυρίου των δυνάμεων θέλει εκτελέσει τούτο.
\par 33 Όθεν ούτω λέγει Κύριος περί του βασιλέως της Ασσυρίας· δεν θέλει εισέλθει εις την πόλιν ταύτην, ουδέ θέλει τοξεύσει εκεί βέλος, ουδέ θέλει προβάλει κατ' αυτής ασπίδας, ουδέ θέλει υψώσει εναντίον αυτής πρόχωμα·
\par 34 διά της οδού δι' ης ήλθε, δι' αυτής θέλει επιστρέψει και εις την πόλιν ταύτην δεν θέλει εισέλθει, λέγει ο Κύριος·
\par 35 διότι θέλω υπερασπισθή την πόλιν ταύτην, ώστε να σώσω αυτήν, ένεκεν εμού και ένεκεν του δούλου μου Δαβίδ.
\par 36 Τότε εξήλθεν ο άγγελος του Κυρίου και επάταξεν εν τω στρατοπέδω των Ασσυρίων εκατόν ογδοήκοντα πέντε χιλιάδας· και ότε εξηγέρθησαν το πρωΐ, ιδού, ήσαν πάντες σώματα νεκρά.
\par 37 Και εσηκώθη και έφυγε και επέστρεψε Σενναχειρείμ ο βασιλεύς της Ασσυρίας και κατώκησεν εν Νινευή.
\par 38 Και ενώ προσεκύνει εν τω οίκω Νισρώκ του θεού αυτού, Αδραμμέλεχ και Σαρασάρ οι υιοί αυτού επάταξαν αυτόν εν μαχαίρα, αυτοί δε έφυγον εις γην Αρμενίας· εβασίλευσε δε αντ' αυτού Εσαραδδών ο υιός αυτού.

\chapter{38}

\par 1 Κατ' εκείνας ημέρας ηρρώστησεν ο Εζεκίας εις θάνατον· και ήλθε προς αυτόν Ησαΐας ο προφήτης ο υιός του Αμώς και είπε προς αυτόν, Ούτω λέγει Κύριος· Διάταξον περί του οίκου σου· επειδή αποθνήσκεις και δεν θέλεις ζήσει.
\par 2 Τότε έστρεψεν ο Εζεκίας το πρόσωπον αυτού προς τον τοίχον και προσηυχήθη εις τον Κύριον,
\par 3 και είπε, Δέομαι, Κύριε, ενθυμήθητι τώρα πως περιεπάτησα ενώπιόν σου εν αληθεία και εν καρδία τελεία και έπραξα το αρεστόν ενώπιόν σου. Και έκλαυσεν ο Εζεκίας κλαυθμόν μέγαν.
\par 4 Τότε έγεινε λόγος Κυρίου προς τον Ησαΐαν λέγων,
\par 5 Ύπαγε και ειπέ προς τον Εζεκίαν, Ούτω λέγει Κύριος ο Θεός του Δαβίδ του πατρός σου· Ήκουσα την προσευχήν σου, είδον τα δάκρυά σου· ιδού, θέλω προσθέσει εις τας ημέρας σου δεκαπέντε έτη·
\par 6 και θέλω ελευθερώσει σε και την πόλιν ταύτην εκ της χειρός του βασιλέως της Ασσυρίας και θέλω υπερασπισθή την πόλιν ταύτην·
\par 7 και τούτο θέλει είσθαι εις σε το σημείον παρά Κυρίου ότι θέλει κάμει ο Κύριος το πράγμα τούτο, το οποίον ελάλησεν·
\par 8 ιδού, θέλω στρέψει οπίσω δέκα βαθμούς την σκιάν των βαθμών, τους οποίους κατέβη εις το ηλιακόν ώρολόγιον του Άχαζ. Και εστράφη ο ήλιος δέκα βαθμούς, διά των οποίων είχε καταβή.
\par 9 Ταύτα είναι τα γραφέντα υπό Εζεκίου βασιλέως του Ιούδα, ότε ηρρώστησε και ανέλαβεν εκ της αρρωστίας αυτού·
\par 10 Εγώ είπα, Εν τη μεσημβρία των ημερών μου θέλω υπάγει εις τας πύλας του τάφου· εστερήθην το υπόλοιπον των ετών μου.
\par 11 Είπα, δεν θέλω ιδεί πλέον τον Κύριον, τον Κύριον, εν γη ζώντων· δεν θέλω ιδεί πλέον άνθρωπον μετά των κατοίκων του κόσμου.
\par 12 Η ζωή μου έφυγε και μετετοπίσθη απ' εμού ως ποιμένος σκηνή· εκόπη η ζωή μου ως υπό υφαντού· από του στημονίου θέλει με κόψει· από πρωΐας έως εσπέρας θέλεις με τελειώσει.
\par 13 Εστοχαζόμην έως πρωΐας, ως λέων θέλει συντρίψει πάντα τα οστά μου· από πρωΐας έως εσπέρας θέλεις με τελειώσει.
\par 14 Ως γερανός, ως χελιδών, ούτω εψέλλιζον· ωδυρόμην ως τρυγών· οι οφθαλμοί μου απέκαμον ατενίζοντες εις τα άνω. Καταθλίβομαι, Κύριε· ανακούφισόν με.
\par 15 Τι να είπω; αυτός και είπε προς εμέ και εξετέλεσε· θέλω διάγει πάντα τα έτη μου εν τη πικρία της ψυχής μου.
\par 16 Εν τούτοις, Κύριε, ζώσιν οι άνθρωποι, και εν πάσι τούτοις υπάρχει ζωή του πνεύματός μου· συ βεβαίως με θεραπεύεις και με αναζωοποιείς.
\par 17 Ιδού, αντί ειρήνης επήλθεν επ' εμέ μεγάλη πικρία· αλλά συ, δι' αγάπην της ψυχής μου, ελύτρωσας αυτήν από του λάκκου της φθοράς· διότι έρριψας οπίσω των νώτων σου πάσας τας αμαρτίας μου.
\par 18 Διότι ο τάφος δεν θέλει σε υμνήσει· ο θάνατος δεν θέλει σε δοξολογήσει· οι καταβαίνοντες εις τον λάκκον δεν θέλουσιν ελπίζει επί την αλήθειάν σου.
\par 19 Ο ζων, ο ζων, αυτός θέλει σε υμνεί, καθώς εγώ ταύτην την ημέραν· ο πατήρ θέλει εις τα τέκνα γνωστοποιήσει την αλήθειάν σου.
\par 20 Ο Κύριος ήλθε να με σώση· διά τούτο θέλομεν ψάλλει το άσμά μου επί εντεταμένων οργάνων πάσας τας ημέρας της ζωής ημών εν τω οίκω του Κυρίου.
\par 21 Διότι ο Ησαΐας είχεν ειπεί, Ας λάβωσι παλάθην σύκων και ας βάλωσιν αυτήν ως έμπλαστρον επί το έλκος και θέλει ιατρευθή.
\par 22 Και ο Εζεκίας είχεν ειπεί, Τι είναι το σημείον ότι εγώ θέλω αναβή εις τον οίκον του Κυρίου;

\chapter{39}

\par 1 Κατ' εκείνον τον καιρόν Μερωδάχ-βαλαδάν, ο υιός του Βαλαδάν, βασιλεύς της Βαβυλώνος, έστειλεν επιστολάς και δώρα προς τον Εζεκίαν, ακούσας ότι ηρρώστησε και ανέλαβε.
\par 2 Και εχάρη δι' αυτά ο Εζεκίας και έδειξεν εις αυτούς τον οίκον των πολυτίμων πραγμάτων αυτού, τον άργυρον και τον χρυσόν και τα αρώματα και τα πολύτιμα μύρα και πάσαν την οπλοθήκην αυτού και παν ό,τι ευρίσκετο εν τοις θησαυροίς αυτού· δεν ήτο ουδέν εν τω οίκω αυτού ουδέ υπό πάσαν την εξουσίαν αυτού, το οποίον ο Εζεκίας δεν έδειξεν εις αυτούς.
\par 3 Τότε ήλθεν Ησαΐας ο προφήτης προς τον βασιλέα Εζεκίαν και είπε προς αυτόν, Τι λέγουσιν ούτοι οι άνθρωποι και πόθεν ήλθον προς σε; Και ο Εζεκίας είπεν, Από γης μακράς έρχονται προς εμέ, από Βαβυλώνος.
\par 4 Ο δε είπε, Τι είδον εν τω οίκω σου; Και απεκρίθη ο Εζεκίας, Είδον παν ό,τι είναι εν τω οίκω μου· δεν είναι ουδέν εν τοις θησαυροίς μου, το οποίον δεν έδειξα εις αυτούς.
\par 5 Τότε είπεν ο Ησαΐας προς τον Εζεκίαν, Άκουσον τον λόγον του Κυρίου των δυνάμεων.
\par 6 Ιδού, έρχονται ημέραι, καθ' ας παν ό,τι είναι εν τω οίκω σου και ό,τι οι πατέρες σου εναπεταμίευσαν μέχρι της ημέρας ταύτης, θέλει μετακομισθή εις την Βαβυλώνα· δεν θέλει μείνει ουδέν, λέγει Κύριος·
\par 7 και εκ των υιών σου, οίτινες θέλουσιν εξέλθει από σου, τους οποίους θέλεις γεννήσει, θέλουσι λάβει· και θέλουσι γείνει ευνούχοι εν τω παλατίω του βασιλέως της Βαβυλώνος.
\par 8 Τότε είπεν ο Εζεκίας προς τον Ησαΐαν, Καλός ο λόγος του Κυρίου, τον οποίον ελάλησας. Είπεν έτι, Διότι θέλει είσθαι ειρήνη και ασφάλεια εν ταις ημέραις μου.

\chapter{40}

\par 1 Παρηγορείτε, παρηγορείτε τον λαόν μου, λέγει ο Θεός σας.
\par 2 Λαλήσατε παρηγορητικά προς την Ιερουσαλήμ, και φωνήσατε προς αυτήν, ότι ο καιρός της ταπεινώσεως αυτής επληρώθη, ότι η ανομία αυτής συνεχωρήθη· διότι έλαβεν εκ της χειρός Κυρίου διπλάσιον διά πάσας τας αμαρτίας αυτής.
\par 3 Φωνή βοώντος εν τη ερήμω, Ετοιμάσατε την οδόν του Κυρίου. ευθείας κάμετε εν τη ερήμω τας τρίβους του Θεού ημών.
\par 4 Πάσα φάραγξ θέλει υψωθή και παν όρος και βουνός θέλει ταπεινωθή· και τα σκολιά θέλουσι γείνει ευθέα· και οι τραχείς τόποι ομαλοί·
\par 5 και η δόξα του Κυρίου θέλει φανερωθή και πάσα σαρξ ομού θέλει ιδεί· διότι το στόμα του Κυρίου ελάλησε.
\par 6 Φωνή λέγουσα, Φώνησον· και είπε, Τι να φωνήσω; πάσα σαρξ είναι χόρτος και πάσα η δόξα αυτής ως άνθος του αγρού.
\par 7 Ο χόρτος εξηράνθη, το άνθος εμαράνθη· διότι πνεύμα Κυρίου έπνευσεν επ' αυτό· χόρτος τη αληθεία είναι ο λαός.
\par 8 Ο χόρτος εξηράνθη, το άνθος εμαράνθη· ο λόγος όμως του Θεού ημών μένει εις τον αιώνα.
\par 9 Συ, ο φέρων εις την Σιών αγαθάς αγγελίας, ανάβα εις το όρος το υψηλόν· συ, ο φέρων αγαθάς αγγελίας εις την Ιερουσαλήμ, ύψωσον ισχυρώς την φωνήν σου· ύψωσον· μη φοβού· ειπέ προς τας πόλεις του Ιούδα, Ιδού, ο Θεός υμών.
\par 10 Ιδού, Κύριος ο Θεός θέλει ελθεί μετά δυνάμεως και ο βραχίων αυτού θέλει εξουσιάζει δι' αυτόν· ιδού, ο μισθός αυτού είναι μετ' αυτού και η αμοιβή αυτού ενώπιον αυτού.
\par 11 Θέλει βοσκήσει το ποίμνιον αυτού ως ποιμήν· θέλει συνάξει τα αρνία διά του βραχίονος αυτού και βαστάσει εν τω κόλπω αυτού· και θέλει οδηγεί τα θηλάζοντα.
\par 12 Τις εμέτρησε τα ύδατα εν τω κοιλώματι της χειρός αυτού και εστάθμισε τους ουρανούς με την σπιθαμήν και συμπεριέλαβεν εν μέτρω το χώμα της γης και εζύγισε τα όρη διά στατήρος και τους λόφους διά πλάστιγγος;
\par 13 Τις εστάθμισε το πνεύμα του Κυρίου ή έγεινε σύμβουλος αυτού και, εδίδαξεν αυτόν;
\par 14 Μετά τίνος συνεβουλεύθη, και τις εσυνέτισεν αυτόν και εδίδαξεν αυτόν την οδόν της κρίσεως και παρέδωκεν εις αυτόν επιστήμην και έδειξεν εις αυτόν την οδόν της συνέσεως;
\par 15 Ιδού, τα έθνη είναι ως σανίς από κάδου και λογίζονται ως η λεπτή σκόνη της πλάστιγγος· ιδού, η μετατοπίζει τας νήσους ως σκόνην.
\par 16 Και ο Λίβανος δεν είναι ικανός εις καύσιν ουδέ τα ζώα αυτού ικανά εις ολοκαύτωμα.
\par 17 Πάντα τα έθνη ενώπιον αυτού είναι ως μηδέν· λογίζονται παρ' αυτώ ολιγώτερον παρά το μηδέν και την ματαιότητα.
\par 18 Με τίνα λοιπόν θέλετε εξομοιώσει τον Θεόν; ή τι ομοίωμα θέλετε προσαρμόσει εις αυτόν;
\par 19 Ο τεχνίτης χωνεύει εικόνα γλυπτήν, και ο χρυσοχόος εκτείνει χρυσόν επ' αυτήν και χύνει αργυράς αλύσεις.
\par 20 Ο πτωχός κάμνων προσφοράν εκλέγει ξύλον άσηπτον· και ζητεί εις εαυτόν επιδέξιον τεχνίτην, διά να κατασκευάση εικόνα γλυπτήν μη σαλευομένην.
\par 21 Δεν εγνωρίσατε; δεν ηκούσατε; δεν ανηγγέλθη προς εσάς εξ αρχής; δεν ενοήσατε από καταβολής της γης;
\par 22 Αυτός είναι ο καθήμενος επί τον γύρον της γης και οι κάτοικοι αυτής είναι ως ακρίδες· ο εκτείνων τους ουρανούς ως παραπέτασμα και εξαπλόνων αυτούς ως σκηνήν προς κατοίκησιν·
\par 23 ο φέρων τους ηγεμόνας εις το μηδέν και καθιστών ως ματαιότητα τους κριτάς της γης.
\par 24 Αλλ' ουδέ θέλουσι φυτευθή· αλλ' ουδέ θέλουσι σπαρθή· αλλ' ουδέ θέλει ριζωθή εν τη γη το στέλεχος αυτών· μόνον να πνεύση επ' αυτούς, θέλουσι πάραυτα ξηρανθή και ο ανεμοστρόβιλος θέλει αναρπάσει αυτούς ως άχυρον.
\par 25 Με τίνα λοιπόν θέλετε με εξομοιώσει και θέλω εξισωθή; λέγει ο Άγιος.
\par 26 Σηκώσατε υψηλά τους οφθαλμούς σας και ιδέτε, τις εποίησε ταύτα; Ο εξάγων το στράτευμα αυτών κατά αριθμόν· ο ονομαστί καλών ταύτα πάντα εν τη μεγαλειότητι της δυνάμεως αυτού, διότι είναι ισχυρός εις εξουσίαν· δεν λείπει ουδέν.
\par 27 Διά τι λέγεις, Ιακώβ, και λαλείς, Ισραήλ, Η οδός μου είναι κεκρυμμένη από του Κυρίου και η κρίσις μου παραμελείται υπό του Θεού μου;
\par 28 Δεν εγνώρισας; δεν ήκουσας, ότι ο αιώνιος Θεός, ο Κύριος, ο Ποιητής των άκρων της γης, δεν ατονεί και δεν αποκάμνει; δεν εξιχνιάζεται η φρόνησις αυτού.
\par 29 Δίδει ισχύν εις τους ητονημένους και αυξάνει την δύναμιν εις τους αδυνάτους.
\par 30 Και οι νέοι θέλουσιν ατονήσει και αποκάμει, και οι εκλεκτοί νέοι θέλουσιν αδυνατήσει παντάπασιν·
\par 31 αλλ' οι προσμένοντες τον Κύριον θέλουσιν ανανεώσει την δύναμιν αυτών· θέλουσιν αναβή με πτέρυγας ως αετοί· θέλουσι τρέξει και δεν θέλουσιν αποκάμει· θέλουσι περιπατήσει και δεν θέλουσιν ατονήσει.

\chapter{41}

\par 1 Σιωπάτε ενώπιόν μου, νήσοι· οι λαοί ας ανανεώσωσι δύναμιν· και ας πλησιάσωσι και τότε ας λαλήσωσιν· ας προσέλθωμεν ομού εις κρίσιν.
\par 2 Τις ήγειρε τον δίκαιον από της ανατολής, προσεκάλεσεν αυτόν κατά πόδας αυτού, παρέδωκεν εις αυτόν τα έθνη και κατέστησεν αυτόν κύριον επί τους βασιλείς; τις παρέδωκεν αυτούς εις την μάχαιραν αυτού ως χώμα, και εις το τόξον αυτού ως άχυρον ωθούμενον από ανέμου;
\par 3 Κατεδίωξεν αυτούς και διήλθεν ασφαλώς διά της οδού, την οποίαν δεν είχε περιπατήσει με τους πόδας αυτού.
\par 4 Τις ενήργησε και έκαμε τούτο, καλών τας γενεάς απ' αρχής; Εγώ ο Κύριος, ο πρώτος και ο μετά των εσχάτων· εγώ αυτός.
\par 5 Αι νήσοι είδον και εφοβήθησαν· τα πέρατα της γης ετρόμαξαν, επλησίασαν και ήλθον.
\par 6 Εβοήθησαν έκαστος τον πλησίον αυτού· και είπε προς τον αδελφόν αυτού, Ίσχυε.
\par 7 Και ο ξυλουργός ενίσχυε τον χρυσοχόον και ο λεπτύνων με την σφύραν, τον σφυροκοπούντα επί τον άκμονα, λέγων, Καλόν είναι διά την συγκόλλησιν· και στερεόνει αυτό με καρφία, διά να μη κινήται.
\par 8 Αλλά συ, Ισραήλ, δούλέ μου, Ιακώβ, εκλεκτέ μου, το σπέρμα Αβραάμ του αγαπητού μου,
\par 9 συ, τον οποίον έλαβον εκ των άκρων της γης και σε εκάλεσα εκ των εσχάτων αυτής και σοι είπα, Συ είσαι ο δούλός μου· εγώ σε εξέλεξα και δεν θέλω σε απορρίψει·
\par 10 μη φοβού· διότι εγώ είμαι μετά σού· μη τρόμαζε· διότι εγώ είμαι ο Θεός σου· σε ενίσχυσα· μάλιστα σε εβοήθησα· μάλιστα σε υπερησπίσθην διά της δεξιάς της δικαιοσύνης μου.
\par 11 Ιδού, πάντες οι ωργισμένοι κατά σου θέλουσι καταισχυνθή και εντραπή· θέλουσιν είσθαι ως μηδέν· και οι αντίδικοί σου θέλουσιν αφανισθή.
\par 12 Θέλεις ζητήσει αυτούς και δεν θέλεις ευρεί αυτούς, τους εναντιουμένους εις σέ· οι πολεμούντες κατά σου θέλουσι γείνει μηδέν και ως εξουθένημα.
\par 13 Διότι εγώ Κύριος ο Θεός σου είμαι ο κρατών την δεξιάν σου, λέγων προς σε, Μη φοβού· εγώ θέλω σε βοηθήσει.
\par 14 Μη φοβού, σκώληξ Ιακώβ, θνητοί του Ισραήλ· εγώ θέλω σε βοηθεί, λέγει ο Κύριος· και λυτρωτής σου είναι ο Άγιος του Ισραήλ.
\par 15 Ιδού, εγώ θέλω σε κάμει νέον κοπτερόν αλωνιστήριον όργανον οδοντωτόν· θέλεις αλωνίσει τα όρη και λεπτύνει αυτά, και θέλεις κάμει τους λόφους ως λεπτόν άχυρον.
\par 16 Θέλεις ανεμίσει αυτά και ο άνεμος θέλει σηκώσει αυτά και ο ανεμοστρόβιλος θέλει διασκορπίσει αυτά· συ δε θέλεις ευφρανθή εις τον Κύριον και θέλεις δοξασθή εν τω Αγίω του Ισραήλ.
\par 17 Όταν οι πτωχοί και ενδεείς ζητήσωσιν ύδωρ και δεν υπάρχη, η γλώσσα δε αυτών ξηραίνηται υπό δίψης, εγώ ο Κύριος θέλω εισακούσει αυτούς, ο Θεός του Ισραήλ δεν θέλω εγκαταλείψει αυτούς.
\par 18 Θέλω ανοίξει ποταμούς εν υψηλοίς τόποις και πηγάς εν μέσω των κοιλάδων· θέλω κάμει την έρημον λίμνας υδάτων και την ξηράν γην πηγάς υδάτων.
\par 19 Εν τη ερήμω θέλω εμφυτεύσει την κέδρον, το δένδρον της σίττης και τον μύρτον και την ελαίαν· εν τη ακατοικήτω γη θέλω βάλει την έλατον, την πεύκην και τον πύξον ομού·
\par 20 διά να ίδωσι και να γνωρίσωσι και να στοχασθώσι και να εννοήσωσιν ομού, ότι η χειρ του Κυρίου έκαμε τούτο και ο Άγιος του Ισραήλ εδημιούργησεν αυτό.
\par 21 Παραστήσατε την δίκην σας, λέγει Κύριος· προφέρετε τα ισχυρά σας επιχειρήματα, λέγει ο βασιλεύς του Ιακώβ.
\par 22 Ας πλησιάσωσι και ας δείξωσιν εις ημάς τι θέλει συμβή· ας αναγγείλωσι τα πρότερα, τι ήσαν, διά να στοχασθώμεν αυτά και να γνωρίσωμεν τα έσχατα αυτών· ή ας αναγγείλωσι προς ημάς τα μέλλοντα.
\par 23 Αναγγείλατε τα συμβησόμενα εις το μετέπειτα, διά να γνωρίσωμεν ότι είσθε θεοί· κάμετε έτι καλόν ή κάμετε κακόν, διά να θαυμάσωμεν και να ίδωμεν ομού.
\par 24 Ιδού, σεις είσθε ολιγώτερον παρά το μηδέν, και το έργον σας χειρότερον παρά το μηδέν· όστις σας εκλέγει, είναι βδέλυγμα.
\par 25 Ήγειρα ένα εκ βορρά και θέλει έλθει· απ' ανατολών ηλίου θέλει επικαλείσθαι το όνομά μου· και θέλει πατήσει επί τους ηγεμόνας ως επί πηλόν και ως ο κεραμεύς καταπατεί τον άργιλον.
\par 26 Τις ανήγγειλε ταύτα απ' αρχής, διά να γνωρίσωμεν; και προ του καιρού, διά να είπωμεν, αυτός είναι ο δίκαιος; Αλλ' ουδείς ο αναγγέλλων· αλλ' ουδείς ο διακηρύττων· αλλ' ουδείς ο ακούων τους λόγους σας.
\par 27 Εγώ ο πρώτος θέλω ειπεί προς την Σιών, Ιδού, ιδού, ταύτα· και θέλω δώσει εις την Ιερουσαλήμ τον ευαγγελιζόμενον.
\par 28 Διότι εθεώρησα και δεν ήτο ουδείς, ναι, μεταξύ αυτών, αλλά δεν υπήρχε σύμβουλος δυνάμενος να αποκριθή λόγον, ότε ηρώτησα αυτούς.
\par 29 Ιδού, πάντες είναι ματαιότης, τα έργα αυτών μηδέν· τα χωνευτά αυτών άνεμος και ματαιότης.

\chapter{42}

\par 1 Ιδού, ο δούλός μου, τον οποίον υπεστήριξα· ο εκλεκτός μου, εις τον οποίον η ψυχή μου ευηρεστήθη· έθεσα το πνεύμά μου επ' αυτόν· θέλει εξαγγείλει κρίσιν εις τα έθνη.
\par 2 Δεν θέλει φωνάξει ουδέ θέλει ανακράξει ουδέ θέλει κάμει την φωνήν αυτού να ακουσθή εν ταις οδοίς.
\par 3 Κάλαμον συντεθλασμένον δεν θέλει συντρίψει και λινάριον καπνίζον δεν θέλει σβύσει· θέλει εκφέρει κρίσιν εν αληθεία.
\par 4 Δεν θέλει εκλίπει ουδέ θέλει μικροψυχήσει, εωσού βάλη κρίσιν εν τη γή· και αι νήσοι θέλουσι προσμένει τον νόμον αυτού.
\par 5 Ούτω λέγει ο Θεός ο Κύριος, ο ποιήσας τους ουρανούς και εκτείνας αυτούς· ο στερεώσας την γην και τα γεννώμενα εξ αυτής· ο διδούς πνοήν εις τον λαόν τον επ' αυτής και πνεύμα εις τους περιπατούντας επ' αυτής·
\par 6 Εγώ ο Κύριος σε εκάλεσα εν δικαιοσύνη, και θέλω κρατεί την χείρα σου και θέλω σε φυλάττει και θέλω σε καταστήσει διαθήκην του λαού, φως των εθνών·
\par 7 διά να ανοίξης τους οφθαλμούς των τυφλών, να εκβάλης τους δεσμίους εκ των δεσμών, τους καθημένους εν σκότει εκ του οίκου της φυλακής.
\par 8 Εγώ είμαι ο Κύριος· τούτο είναι το όνομά μου· και δεν θέλω δώσει την δόξαν μου εις άλλον ουδέ την αίνεσίν μου εις τα γλυπτά.
\par 9 Ιδού, ήλθον τα απ' αρχής· και εγώ αναγγέλλω νέα πράγματα· πριν εκφύωσι, λαλώ περί αυτών εις εσάς.
\par 10 Ψάλλετε εις τον Κύριον άσμα νέον, την δόξαν αυτού εκ των άκρων της γης, σεις οι καταβαίνοντες εις την θάλασσαν και πάντα τα εν αυτή· αι νήσοι και οι κατοικούντες αυτάς.
\par 11 Η έρημος και αι πόλεις αυτής ας υψώσωσι φωνήν, αι κώμαι τας οποίας κατοικεί ο Κηδάρ· ας ψάλλωσιν οι κάτοικοι της Σελά, ας αλαλάζωσιν εκ των κορυφών των ορέων.
\par 12 Ας δώσωσι δόξαν εις τον Κύριον και ας αναγγείλωσι την αίνεσιν αυτού εν ταις νήσοις.
\par 13 Ο Κύριος θέλει εξέλθει ως ισχυρός· θέλει διεγείρει ζήλον ως πολεμιστής· θέλει φωνάξει, μάλιστα θέλει βρυχήσει, θέλει υπερισχύσει κατά των πολεμίων αυτού.
\par 14 Από πολλού εσιώπησα· θέλω μείνει ήσυχος; θέλω κρατήσει εμαυτόν; τώρα θέλω φωνάξει ως η τίκτουσα· θέλω καταστρέψει και καταπίει ομού.
\par 15 Θέλω ερημώσει όρη και λόφους και καταξηράνει πάντα τον χόρτον αυτών· και θέλω καταστήσει τους ποταμούς νήσους και τας λίμνας θέλω ξηράνει.
\par 16 Και θέλω φέρει τους τυφλούς δι' οδού την οποίαν δεν ήξευρον, θέλω οδηγήσει αυτούς εις τρίβους τας οποίας δεν εγνώριζον· το σκότος θέλω κάμει φως έμπροσθεν αυτών και τα σκολιά ευθέα. Ταύτα τα πράγματα θέλω κάμει εις αυτούς και δεν θέλω εγκαταλείψει αυτούς.
\par 17 Εστράφησαν εις τα οπίσω, κατησχύνθησαν οι θαρρούντες επί τα γλυπτά, οι λέγοντες προς τα χωνευτά, σεις είσθε οι θεοί ημών.
\par 18 Ακούσατε, κωφοί· και ανοίξατε τους οφθαλμούς σας, τυφλοί, διά να ίδητε.
\par 19 Τις τυφλός, παρά ο δούλός μου; ή κωφός, παρά ο μηνυτής μου, τον οποίον απέστειλα; τις τυφλός, παρά ο τέλειος; και τις τυφλός, παρά ο δούλος του Κυρίου;
\par 20 Βλέπεις πολλά αλλά δεν παρατηρείς· ανοίγεις τα ώτα αλλά δεν ακούεις.
\par 21 Ο Κύριος ευνόησε προς αυτόν ένεκεν της δικαιοσύνης αυτού· θέλει μεγαλύνει τον νόμον αυτού και καταστήσει έντιμον.
\par 22 Πλην αυτός είναι λαός διηρπαγμένος και γεγυμνωμένος· είναι πάντες πεπαγιδευμένοι εν σπηλαίοις και κεκρυμμένοι εν ταις φυλακαίς· είναι λάφυρον και δεν υπάρχει ο λυτρόνων· διάρπαγμα, και ουδείς ο λέγων, Επίστρεψον αυτό.
\par 23 Τις από σας θέλει δώσει ακρόασιν εις τούτο; θέλει προσέξει και ακούσει εις το μετά ταύτα;
\par 24 Τις παρέδωκε τον Ιακώβ εις διαρπαγήν και τον Ισραήλ εις λεηλατιστάς; ουχί ο Κύριος, αυτός εις τον οποίον ημαρτήσαμεν; διότι δεν ηθέλησαν να περιπατήσωσιν εν ταις οδοίς αυτού ουδέ υπήκουσαν εις τον νόμον αυτού.
\par 25 Διά τούτο εξέχεεν επ' αυτόν την σφοδρότητα της οργής αυτού και την ορμήν του πολέμου· και συνέφλεξεν αυτόν πανταχόθεν αλλ' αυτός δεν ενόησε· και έκαυσεν αυτόν αλλ' αυτός δεν έβαλε τούτο εν τη καρδία αυτού.

\chapter{43}

\par 1 Και τώρα ούτω λέγει Κύριος, ο δημιουργός σου, Ιακώβ, και ο πλάστης σου, Ισραήλ· Μη φοβού· διότι εγώ σε ελύτρωσα, σε εκάλεσα με το όνομά σου· εμού είσαι.
\par 2 Όταν διαβαίνης διά των υδάτων, μετά σου θέλω είσθαι· και όταν διά των ποταμών, δεν θέλουσι πλημμυρήσει επί σέ· όταν περιπατής διά του πυρός, δεν θέλεις καή ουδέ θέλει εξαφθή η φλόξ επί σε.
\par 3 Διότι εγώ είμαι Κύριος ο Θεός σου, ο Άγιος του Ισραήλ, ο Σωτήρ σου· διά αντίλυτρόν σου έδωκα την Αίγυπτον· υπέρ σου την Αιθιοπίαν και Σεβά.
\par 4 Αφότου εστάθης πολύτιμος εις τους οφθαλμούς μου, εδοξάσθης και εγώ σε ηγάπησα· και θέλω δώσει ανθρώπους πολλούς υπέρ σου και λαούς υπέρ της κεφαλής σου.
\par 5 Μη φοβού· διότι εγώ είμαι μετά σού· από ανατολών θέλω φέρει το σπέρμα σου και από δυσμών θέλω σε συνάξει·
\par 6 Θέλω ειπεί προς τον βορράν, Δός· και προς τον νότον, Μη εμποδίσης· φέρε τους υιούς μου από μακράν και τας θυγατέρας μου από των άκρων της γης,
\par 7 πάντας όσοι καλούνται με το όνομά μου· διότι εδημιούργησα αυτούς διά την δόξαν μου, έπλασα αυτούς και έκαμα αυτούς.
\par 8 Εξάγαγε τον λαόν τον τυφλόν και έχοντα οφθαλμούς και τον κωφόν και έχοντα ώτα.
\par 9 Ας συναθροισθώσι πάντα τα έθνη και ας συναχθώσιν οι λαοί· τις μεταξύ αυτών ανήγγειλε τούτο και έδειξεν εις ημάς τα πρότερα; ας φέρωσι τους μάρτυρας αυτών και ας δικαιωθώσιν· και ας ακούσωσι και ας είπωσι, Τούτο είναι αληθινόν.
\par 10 Σεις είσθε μάρτυρές μου, λέγει Κύριος, και ο δούλός μου, τον οποίον εξέλεξα, διά να μάθητε και να πιστεύσητε εις εμέ και να εννοήσητε ότι εγώ αυτός είμαι· προ εμού άλλος Θεός δεν υπήρξεν ουδέ θέλει υπάρχει μετ' εμέ.
\par 11 Εγώ, εγώ είμαι ο Κύριος· και εκτός εμού σωτήρ δεν υπάρχει.
\par 12 Εγώ ανήγγειλα και έσωσα και έδειξα· και δεν εστάθη εις εσάς ξένος θεός· σεις δε είσθε μάρτυρές μου, λέγει Κύριος, και εγώ ο Θεός.
\par 13 Και πριν γείνη η ημέρα, εγώ αυτός ήμην· και δεν υπάρχει ο λυτρόνων εκ της χειρός μου· θέλω κάμει και τις δύναται να εμποδίση αυτό;
\par 14 Ούτω λέγει Κύριος, ο Λυτρωτής σας, ο Άγιος του Ισραήλ· διά σας εξαπέστειλα εις την Βαβυλώνα και κατέβαλον πάντας τους φυγάδας αυτής και τους Χαλδαίους τους εγκαυχωμένους εις τα πλοία.
\par 15 Εγώ είμαι ο Κύριος, ο Άγιός σας, ο Ποιητής του Ισραήλ, ο Βασιλεύς σας.
\par 16 Ούτω λέγει Κύριος, όστις έκαμεν οδόν εις την θάλασσαν και τρίβον εις τα ισχυρά ύδατα·
\par 17 όστις εξήγαγεν αμάξας και ίππους, στράτευμα και ρωμαλέους· πάντα ομού εξηπλώθησαν κάτω, δεν εσηκώθησαν· ηφανίσθησαν, εσβέσθησαν ως στυπίον.
\par 18 Μη ενθυμήσθε τα πρότερα και μη συλλογίζεσθε τα παλαιά.
\par 19 Ιδού, εγώ θέλω κάμει νέον πράγμα· τώρα θέλει ανατείλει· δεν θέλετε γνωρίσει αυτό; θέλω βεβαίως κάμει οδόν εν τη ερήμω, ποταμούς εν τη ανύδρω.
\par 20 Τα θηρία του αγρού θέλουσι με δοξάσει, οι θώες και οι στρουθοκάμηλοι· διότι δίδω ύδατα εις την έρημον, ποταμούς εις την άνυδρον, διά να ποτίσω τον λαόν μου, τον εκλεκτόν μου.
\par 21 Ο λαός, τον οποίον έπλασα εις εμαυτόν, θέλει διηγείσθαι την αίνεσίν μου.
\par 22 Αλλά συ, Ιακώβ, δεν με επεκαλέσθης· αλλά συ, Ισραήλ, εβαρύνθης απ' εμού.
\par 23 Δεν προσέφερες εις εμέ τα αρνία των ολοκαυτωμάτων σου ουδέ με ετίμησας με τας θυσίας σου. Εγώ δεν σε εδούλωσα με προσφοράς ουδέ σε εβάρυνα με θυμίαμα·
\par 24 δεν ηγόρασας με αργύριον κάλαμον αρωματικόν δι' εμέ, ουδέ με ενέπλησας από του πάχους των θυσιών σου· αλλά με εδούλωσας με τας αμαρτίας σου, με επεβάρυνας με τας ανομίας σου.
\par 25 Εγώ, εγώ είμαι, όστις εξαλείφω τας παραβάσεις σου ένεκεν εμού, και δεν θέλω ενθυμηθή τας αμαρτίας σου.
\par 26 Ενθύμισόν με· ας κριθώμεν ομού· λέγε συ, διά να δικαιωθής.
\par 27 Ο προπάτωρ σου ημάρτησε και οι διδάσκαλοί σου ηνόμησαν εις εμέ.
\par 28 Διά τούτο θέλω καταστήσει βεβήλους τους άρχοντας του αγιαστηρίου, και θέλω παραδώσει τον Ιακώβ εις κατάραν και τον Ισραήλ εις ονειδισμούς.

\chapter{44}

\par 1 Αλλά τώρα άκουσον, δούλέ μου Ιακώβ, και Ισραήλ τον οποίον εξέλεξα.
\par 2 Ούτω λέγει Κύριος, όστις σε έκαμε και σε έπλασεν εκ κοιλίας και θέλει σε βοηθήσει· Μη φοβού, δούλέ μου Ιακώβ, και συ, Ιεσουρούν, τον οποίον εξέλεξα.
\par 3 Διότι θέλω εκχέει ύδωρ επί τον διψώντα και ποταμούς επί την ξηράν· θέλω εκχέει το πνεύμά μου επί το σπέρμα σου και την ευλογίαν μου επί τους εκγόνους σου·
\par 4 και θέλουσι βλαστήσει ως μεταξύ χόρτου, ως ιτέαι παρά τους ρύακας των υδάτων.
\par 5 Ο μεν θέλει λέγει, Εγώ είμαι του Κυρίου· ο δε θέλει ονομάζεσθαι με το όνομα Ιακώβ· και άλλος θέλει υπογράφεσθαι με την χείρα αυτού εις τον Κύριον και επονομάζεσθαι με το όνομα Ισραήλ.
\par 6 Ούτω λέγει Κύριος ο Βασιλεύς του Ισραήλ και ο Λυτρωτής αυτού, ο Κύριος των δυνάμεων· Εγώ είμαι ο πρώτος και εγώ ο έσχατος· και εκτός εμού δεν υπάρχει Θεός.
\par 7 Και τις ως εγώ θέλει κράξει και αναγγείλει και διατάξει εις εμέ, αφού εσύστησα τον παλαιόν λαόν; και τα επερχόμενα και τα μέλλοντα ας αναγγείλωσι προς αυτούς.
\par 8 Μη φοβείσθε μηδέ τρομάζετε· έκτοτε δεν σε έκαμα να ακούσης και ανήγγειλα τούτο; σεις είσθε μάλιστα μάρτυρές μου· εκτός εμού υπάρχει Θεός; βεβαίως δεν υπάρχει βράχος· δεν γνωρίζω ουδένα.
\par 9 Όσοι κατασκευάζουσιν είδωλα, πάντες είναι ματαιότης· και τα πολυέραστα αυτών είδωλα δεν ωφελούσι· και αυτοί είναι μάρτυρες αυτών ότι δεν βλέπουσιν ουδέ νοούσι, διά να καταισχυνθώσι.
\par 10 Τις έπλασε θεόν ή έχυσεν είδωλον, το οποίον ουδέν ωφελεί;
\par 11 Ιδού, πάντες οι σύντροφοι αυτού θέλουσιν αισχυνθή· και οι τεχνίται, αυτοί είναι εξ ανθρώπων· ας συναχθώσι πάντες ομού· ας παρασταθώσι· θέλουσι φοβηθή, θέλουσιν εντραπή πάντες ομού.
\par 12 Ο χαλκεύς κόπτει σίδηρον και εργάζεται εις τους άνθρακας και με τα σφυρία μορφόνει αυτό και κατασκευάζει αυτό με την δύναμιν των βραχιόνων αυτού· μάλιστα πεινά και η δύναμις αυτού αποκάμνει· ύδωρ δεν πίνει και ατονεί.
\par 13 Ο ξυλουργός εξαπλόνει τον κανόνα, σημειόνει αυτό με στάθμην, ομαλύνει αυτό με ροκάνια και σημειόνει αυτό διά του διαβήτου και κάμνει αυτό κατά την ανθρωπίνην μορφήν, κατά ανθρωπίνην ώραιότητα, διά να κατοική εν τη οικία.
\par 14 Κόπτει εις εαυτόν κέδρους και λαμβάνει την κυπάρισσον και την δρυν, τα οποία εκλέγει εις εαυτόν μεταξύ των δένδρων του δάσους· φυτεύει πεύκην και η βροχή αυξάνει αυτήν.
\par 15 Και θέλει είσθαι χρήσιμον εις τον άνθρωπον διά καύσιμον· και εξ αυτού λαμβάνει και θερμαίνεται· προσέτι καίει αυτό και ψήνει άρτον· προσέτι κάμνει αυτό θεόν και προσκυνεί αυτό· κάμνει αυτό είδωλον και γονατίζει έμπροσθεν αυτού.
\par 16 Το ήμισυ αυτού καίει εν πυρί· με το άλλο ήμισυ τρώγει το κρέας· ψήνει το ψητόν και χορταίνει· και θερμαίνεται, λέγων, Ω εθερμάνθην, είδον το πύρ·
\par 17 και το εναπολειφθέν αυτού κάμνει θεόν, το γλυπτόν αυτού· γονατίζει έμπροσθεν αυτού και προσκυνεί αυτό και προσεύχεται εις αυτό και λέγει, Λύτρωσόν με, διότι είσαι ο θεός μου.
\par 18 Δεν καταλαμβάνουσιν ουδέ νοούσι· διότι έκλεισε τους οφθαλμούς αυτών διά να μη βλέπωσι, και τας καρδίας αυτών διά να μη νοώσι.
\par 19 Και ουδείς συλλογίζεται εν τη καρδία αυτού ουδέ είναι γνώσις εν αυτώ ουδέ νόησις, ώστε να είπη, Το ήμισυ αυτού έκαυσα εν πυρί· έτι έψησα άρτον επί των ανθράκων αυτού· έψησα κρέας και έφαγον· έπειτα θέλω κάμει το υπόλοιπον αυτού βδέλυγμα; θέλω προσκυνήσει δένδρου κορμόν;
\par 20 Βόσκεται από στάκτης· η ηπατημένη καρδία αυτού απεπλάνησεν αυτόν, διά να μη δύναται να ελευθερώση την ψυχήν αυτού μηδέ να είπη, Τούτο, τη εν τη δεξιά μου, δεν είναι ψεύδος;
\par 21 Ενθυμού ταύτα, Ιακώβ και Ισραήλ· διότι δούλός μου είσαι· εγώ σε έπλασα· δούλός μου είσαι· Ισραήλ, δεν θέλεις λησμονηθή υπ' εμού.
\par 22 Εξήλειψα ως πυκνήν ομίχλην τας παραβάσεις σου, και ως νέφος τας αμαρτίας σου· επίστρεψον προς εμέ· διότι εγώ σε ελύτρωσα.
\par 23 Ψάλλετε, ουρανοί· διότι ο Κύριος έκαμε τούτο· αλαλάξατε, τα κάτω της γής· εκβάλετε φωνήν αγαλλιάσεως, όρη, δάση και πάντα τα εν αυτοίς δένδρα· διότι ο Κύριος ελύτρωσε τον Ιακώβ και εδοξάσθη εν τω Ισραήλ.
\par 24 Ούτω λέγει ο Κύριος, όστις σε ελύτρωσε και σε έπλασεν εκ κοιλίας· Εγώ είμαι ο Κύριος ο ποιήσας τα πάντα· ο μόνος εκτείνας τους ουρανούς, ο στερεώσας την γην απ' εμαυτού·
\par 25 ο ματαιόνων τα σημεία των ψευδολόγων και καθιστών παράφρονας τους μάντεις· ο ανατρέπων τους σοφούς και μωραίνων την επιστήμην αυτών·
\par 26 ο στερεόνων τον λόγον του δούλου μου και εκπληρών την βουλήν των μηνυτών μου· ο λέγων προς την Ιερουσαλήμ, Θέλεις κατοικισθή· και προς τας πόλεις του Ιούδα, Θέλετε ανακτισθή και θέλω ανορθώσει τα ερείπια αυτού·
\par 27 ο λέγων προς την άβυσσον, Γενού ξηρά και θέλω ξηράνει τους ποταμούς σου·
\par 28 ο λέγων προς τον Κύρον, Ούτος είναι ο βοσκός μου και θέλει εκπληρώσει πάντα τα θελήματά μου· και ο λέγων προς την Ιερουσαλήμ, Θέλεις ανακτισθή· και προς τον ναόν, Θέλουσι τεθή τα θεμέλιά σου.

\chapter{45}

\par 1 Ούτω λέγει Κύριος προς τον κεχρισμένον αυτού, τον Κύρον, του οποίου την δεξιάν χείρα εκράτησα, διά να υποτάξω τα έθνη έμπροσθεν αυτού· και θέλω λύσει την οσφύν των βασιλέων, διά να ανοίξω τα δίθυρα έμπροθεν αυτού· και αι πύλαι δεν θέλουσι κλεισθή.
\par 2 Εγώ θέλω υπάγει έμπροθέν σου και εξομαλύνει τας σκολιάς οδούς· θέλω συντρίψει τας χαλκίνας θύρας και κόψει τους σιδηρούς μοχλούς.
\par 3 Και θέλω σοι δώσει θησαυρούς φυλαττομένους εν σκότει και πλούτη κερυμμένα εν αποκρύφοις· διά να γνωρίσης ότι εγώ είμαι ο Κύριος ο καλών σε κατ' όνομα, ο Θεός του Ισραήλ.
\par 4 Διά τον Ιακώβ τον δούλον μου και τον Ισραήλ τον εκλεκτόν μου σε εκάλεσα μάλιστα με το όνομά σου, σε επωνόμασα, αν και δεν με εγνώρισας.
\par 5 Εγώ είμαι ο Κύριος και δεν είναι άλλος· δεν υπάρχει εκτός εμού Θεός· εγώ σε περιέζωσα, αν και δεν με εγνώρισας,
\par 6 διά να γνωρίσωσιν από ανατολών ηλίου και από δυσμών, ότι εκτός εμού δεν υπάρχει ουδείς· εγώ είμαι ο Κύριος και δεν υπάρχει άλλος·
\par 7 ο κατασκευάσας το φως και ποιήσας το σκότος· ο ποιών ειρήνην και κτίζων κακόν· εγώ ο Κύριος ποιώ πάντα ταύτα.
\par 8 Σταλάξατε δρόσον άνωθεν, ουρανοί, και ας ράνωσιν αι νεφέλαι δικαιοσύνην· ας ανοίξη η γη και ας γεννήση σωτηρίαν και ας βλαστήση δικαιοσύνην ομού· εγώ ο Κύριος εποίησα τούτο.
\par 9 Ουαί εις τον αντιμαχόμενον προς τον Ποιητήν αυτού. Ας αντιμάχεται το όστρακον προς τα όστρακα της γής· ο πηλός θέλει ειπεί προς τον πλάττοντα αυτόν, Τι κάμνεις; ή το έργον σου, Ούτος δεν έχει χείρας;
\par 10 Ουαί εις τον λέγοντα προς τον πατέρα, τι γεννάς; προς την γυναίκα, τι κοιλοπονείς;
\par 11 Ούτω λέγει Κύριος, ο Άγιος του Ισραήλ και ο Πλάστης αυτού· Ερωτάτέ με διά τα μέλλοντα περί των υιών μου και περί του έργου των χειρών μου προστάξατέ με.
\par 12 Εγώ έκτισα την γην και εποίησα άνθρωπον επ' αυτής· εγώ διά των χειρών μου εξέτεινα τους ουρανούς και έδωκα διαταγάς εις πάσαν την στρατιάν αυτών.
\par 13 Εγώ εξήγειρα εκείνον εις δικαιοσύνην και θέλω διευθύνει πάσας τας οδούς αυτού· αυτός θέλει οικοδομήσει την πόλιν μου και θέλει επιστρέψει τους αιχμαλώτους μου, ουχί με λύτρον ουδέ με δώρα, λέγει ο Κύριος των δυνάμεων.
\par 14 Ούτω λέγει Κύριος· Ο κόπος της Αιγύπτου και το εμπόριον της Αιθιοπίας και των Σαβαίων, ανδρών μεγαλοσώμων, θέλουσι περάσει εις σε και σου θέλουσιν είσθαι· οπίσω σου θέλουσιν ακολουθεί· με αλύσεις θέλουσι περάσει και θέλουσι σε προσκυνήσει, θέλουσι σε ικετεύσει, λέγοντες, Βεβαίως ο Θεός είναι εν σοι, και δεν υπάρχει ουδείς άλλος Θεός.
\par 15 Τωόντι συ είσαι Θεός κρυπτόμενος, Θεέ του Ισραήλ, ο Σωτήρ.
\par 16 Πάντες ούτοι θέλουσιν αισχυνθή και εντραπή· οι εργάται των ειδώλων θέλουσι φύγει εν καταισχύνη πάντες ομού.
\par 17 Ο δε Ισραήλ θέλει σωθή διά του Κυρίου σωτηρίαν αιώνιον· δεν θέλετε αισχυνθή ουδέ εντραπή αιωνίως.
\par 18 Διότι ούτω λέγει Κύριος, ο ποιήσας τους ουρανούς· αυτός ο Θεός, ο πλάσας την γην και ποιήσας αυτήν· όστις αυτός εστερέωσεν αυτήν, έκτισεν αυτήν ουχί ματαίως αλλ' έπλασεν αυτήν διά να κατοικήται· Εγώ είμαι ο Κύριος και δεν υπάρχει άλλος.
\par 19 Δεν ελάλησα εν κρυπτώ ουδέ εν σκοτεινώ τόπω της γής· δεν είπα προς το σπέρμα του Ιακώβ, Ζητήσατέ με ματαίως· εγώ είμαι ο Κύριος, ο λαλών δικαιοσύνην, ο αναγγέλλων ευθύτητα.
\par 20 Συνάχθητε και έλθετε· πλησιάσατε ομού, οι σεσωσμένοι των εθνών· δεν έχουσι νόησιν, όσοι σηκόνουσι το γλυπτόν ξύλον αυτών και προσεύχονται εις θεόν μη δυνάμενον να σώση.
\par 21 Απαγγείλατε και φέρετε αυτούς πλησίον· μάλιστα, ας συμβουλευθώσιν ομού· τις ανήγγειλε τούτο απ' αρχής; τις εφανέρωσε τούτο εξ εκείνου του καιρού; ουχί εγώ ο Κύριος; και δεν υπάρχει εκτός εμού άλλος Θεός· Θεός δίκαιος και Σωτήρ· δεν υπάρχει εκτός εμού.
\par 22 Εις εμέ βλέψατε και σώθητε, πάντα τα πέρατα της γής· διότι εγώ είμαι ο Θεός και δεν υπάρχει άλλος.
\par 23 Ώμοσα εις εμαυτόν· ο λόγος εξήλθεν εκ του στόματός μου εν δικαιοσύνη και δεν θέλει επιστραφή, Ότι παν γόνυ θέλει κάμψει εις εμέ, πάσα γλώσσα θέλει ομνύει εις εμέ.
\par 24 Βεβαίως θέλουσιν ειπεί περί εμού, Εν τω Κυρίω είναι η δικαιοσύνη και η δύναμις· εις αυτόν θέλουσι προσέλθει και θέλουσι καταισχυνθή πάντες οι οργιζόμενοι εναντίον αυτού.
\par 25 Εν τω Κυρίω θέλει δικαιωθή άπαν το σπέρμα του Ισραήλ.

\chapter{46}

\par 1 Κατεκάμφθη ο Βηλ, έκυψεν ο Νεβώ· τα είδωλα αυτών επετέθησαν επί ζώων και κτηνών· αι άμαξαι υμών ήσαν πεφορτισμέναι φορτίον κοπιαστικόν.
\par 2 Κύπτουσι, κάμπτουσιν ομού· δεν δύνανται να σώσωσι το φορτίον αλλά και αυτά φέρονται εις αιχμαλωσίαν.
\par 3 Ακούσατέ μου, οίκος Ιακώβ και παν το υπόλοιπον του οίκου Ισραήλ, τους οποίους εσήκωσα από κοιλίας, τους οποίους εβάστασα από μήτρας·
\par 4 και έως του γήρατός σας εγώ αυτός είμαι· και έως των λευκών τριχών εγώ θέλω σας βαστάσει· εγώ σας έκαμα και εγώ θέλω σας σηκώσει· ναι, εγώ θέλω σας βαστάσει και σώσει.
\par 5 Με τίνα θέλετε με εξομοιώσει και θέλετε με εξισώσει και με συγκρίνει και θέλομεν είσθαι όμοιοι;
\par 6 Χύνουσι χρυσίον εκ του βαλαντίου και ζυγίζουσιν αργύριον διά του στατήρος και μισθόνουσι χρυσοχόον και κατασκευάζει αυτό θεόν· έπειτα προσπίπτουσι και προσκυνούσι·
\par 7 σηκόνουσιν αυτόν επ' ώμου· φέρουσιν αυτόν και θέτουσιν αυτόν εις τον τόπον αυτού και ίσταται· δεν θέλει μετασαλεύσει εκ του τόπου αυτού· προσέτι βοώσι προς αυτόν αλλά δεν δύναται να αποκριθή ουδέ να σώση αυτούς από της συμφοράς αυτών.
\par 8 Ενθυμήθητε τούτο και δείχθητε άνθρωποι· ανακαλέσατε αυτό εις τον νούν σας, αποστάται.
\par 9 Ενθυμήθητε τα πρότερα, τα απ' αρχής· διότι εγώ είμαι ο Θεός και δεν υπάρχει άλλος· εγώ είμαι ο Θεός και ουδείς όμοιός μου·
\par 10 όστις απ' αρχής αναγγέλλω το τέλος και από πρότερον τα μη γεγονότα, λέγων, Η βουλή μου θέλει σταθή και θέλω εκτελέσει άπαν το θέλημά μου·
\par 11 όστις κράζω το αρπακτικόν πτηνόν εξ ανατολών, τον άνδρα της βουλής μου από γης μακράν· ναι, ελάλησα και θέλω κάμει να γείνη· εβουλεύθην και θέλω εκτελέσει αυτό.
\par 12 Ακούσατέ μου, σκληροκάρδιοι, οι μακράν από της δικαιοσύνης.
\par 13 Επλησίασα την δικαιοσύνην μου· δεν θέλει είσθαι μακράν και η σωτηρία μου δεν θέλει βραδύνει· και θέλω δώσει εν Σιών σωτηρίαν εις τον Ισραήλ, την δόξαν μου.

\chapter{47}

\par 1 Κατάβα και κάθησον επί του χώματος, παρθένε θυγάτηρ της Βαβυλώνος· κάθησον κατά γής· θρόνος πλέον δεν είναι, θυγάτηρ των Χαλδαίων· διότι δεν θέλεις πλέον ονομασθή απαλή και τρυφερά.
\par 2 Πίασον τον χειρόμυλον και άλεθε άλευρον· εκκάλυψον τους πλοκάμους σου, γύμνωσον τους πόδας, εκκάλυψον τας κνήμας, πέρασον τους ποταμούς.
\par 3 Η γύμνωσίς σου θέλει εκκαλυφθή· ναι, η αισχύνη σου θέλει φανή· εκδίκησιν θέλω λάβει και δεν θέλω φεισθή άνθρωπον.
\par 4 Του Λυτρωτού ημών το όνομα είναι, Ο Κύριος των δυνάμεων, ο Άγιος του Ισραήλ.
\par 5 Κάθησον σιωπώσα και είσελθε εις το σκότος, θυγάτηρ των Χαλδαίων· διότι δεν θέλεις πλέον ονομάζεσθαι, Η κυρία των βασιλείων.
\par 6 Ωργίσθην κατά του λαού μου, εμίανα την κληρονομίαν μου και παρέδωκα αυτούς εις την χείρα σου· πλην συ δεν έδειξας εις αυτούς έλεος· σφόδρα εβάρυνας τον ζυγόν σου επί τον γέροντα.
\par 7 Και είπας, εις τον αιώνα θέλω είσθαι κυρία· ώστε δεν έβαλες ταύτα εν τη καρδία σου ουδέ ενεθυμήθης τα έσχατα αυτών.
\par 8 Τώρα λοιπόν άκουσον τούτο, η παραδεδομένη εις τας τρυφάς, η κατοικούσα αμερίμνως, η λέγουσα εν τη καρδία σου, Εγώ είμαι και εκτός εμού ουδεμία άλλη· δεν θέλω καθήσει χήρα και δεν θέλω γνωρίσει ατέκνωσιν.
\par 9 Τα δύο ταύτα θέλουσι βεβαίως ελθεί επί σε εξαίφνης εν μιά ημέρα, ατέκνωσις και χηρεία· θέλουσιν ελθεί επί σε καθ' ολοκληρίαν διά το πλήθος των μαγειών σου, διά την μεγάλην αφθονίαν των γοητευμάτων σου·
\par 10 διότι εθαρρεύθης επί την πονηρίαν σου και είπας, δεν με βλέπει ουδείς. Η σοφία σου και η επιστήμη σου σε απεπλάνησαν· και είπας εν τη καρδία σου, Εγώ είμαι και εκτός εμού ουδεμία άλλη.
\par 11 Διά τούτο θέλει ελθεί κακόν επί σε, χωρίς να εξεύρης πόθεν γεννάται· και συμφορά θέλει πέσει κατά σου, χωρίς να δύνασαι να αποστρέψης αυτήν· και όλεθρος θέλει ελθεί, αιφνιδίως επί σε, χωρίς να εξεύρης.
\par 12 Στήθι τώρα με τας γοητείας σου και με το πλήθος των μαγειών σου, εις τας οποίας ηγωνίσθης εκ νεότητός σου· αν δύνασαι να ωφεληθής, αν δύνασαι να υπερισχύσης.
\par 13 Απέκαμες εν τω πλήθει των βουλών σου. Ας σηκωθώσι τώρα οι ουρανοσκόποι, οι αστρολόγοι, οι μηνολόγοι προγνωστικοί, και ας σε σώσωσιν εκ των επερχομένων επί σε.
\par 14 Ιδού, θέλουσιν είσθαι ως άχυρον· πυρ θέλει κατακαύσει αυτούς· δεν θέλουσι δυνηθή να σώσωσιν εαυτούς από της δυνάμεως της φλογός· δεν θέλει μείνει άνθραξ διά να θερμανθή τις ουδέ πυρ διά να καθήση έμπροσθεν αυτού.
\par 15 Τοιούτοι θέλουσιν είσθαι εις σε εκείνοι, μετά των οποίων εκ νεότητός σου εκοπίασας, οι έμποροί σου· θέλουσι φύγει περιπλανώμενοι έκαστος εις το μέρος αυτού· ουδείς θέλει σε σώσει.

\chapter{48}

\par 1 Ακούσατε τούτο, οίκος Ιακώβ· οι κληθέντες με το όνομα του Ισραήλ και εξελθόντες εκ της πηγής του Ιούδα· οι ομνύοντες εις το όνομα του Κυρίου και αναφέροντες τον Θεόν του Ισραήλ, πλην ουχί εν αληθεία ουδέ εν δικαιοσύνη.
\par 2 Διότι λαμβάνουσι το όνομα αυτών εκ της πόλεως της αγίας και επιστηρίζονται επί τον Θεόν του Ισραήλ· το όνομα αυτού είναι, Ο Κύριος των δυνάμεων.
\par 3 Έκτοτε ανήγγειλα τα απ' αρχής· και εξήλθον εκ του στόματός μου και διεκήρυξα αυτά· έκαμα ταύτα αιφνιδίως και έγειναν.
\par 4 Επειδή γνωρίζω ότι είσαι σκληρός, και ο τράχηλός σου είναι νεύρον σιδηρούν και το μέτωπόν σου χάλκινον.
\par 5 Έκτοτε δε ανήγγειλα τούτο προς σέ· πριν γείνη διεκήρυξα τούτο εις σε, διά να μη είπης, Το είδωλόν μου έκαμε ταύτα· και το γλυπτόν μου και το χυτόν μου προσέταξε ταύτα.
\par 6 Ηκουσας· ιδέ πάντα ταύτα· και δεν θέλετε ομολογήσει; από τούδε διακηρύττω προς σε νέα, μάλιστα αποκεκρυμμένα, και τα οποία συ δεν ήξευρες.
\par 7 Τώρα έγειναν και ουχί παλαιόθεν, και ουδέ προ της ημέρας ταύτης ήκουσας περί αυτών, διά να μη είπης, Ιδού, εγώ ήξευρον ταύτα.
\par 8 Ούτε ήκουσας ούτε ήξευρες ούτε απ' αρχής ηνοίχθησαν τα ώτα σου· διότι ήξευρον έτι βεβαίως ήθελες φερθή απίστως και εκ κοιλίας ωνομάσθης παραβάτης.
\par 9 Ένεκεν του ονόματός μου θέλω μακρύνει τον θυμόν μου, και διά τον έπαινόν μου θέλω βασταχθή προς σε, ώστε να μη σε εξολοθρεύσω.
\par 10 Ιδού, σε εκαθάρισα, πλην ουχί ως άργυρον· σε κατέστησα εκλεκτόν εν τω χωνευτηρίω της θλίψεως.
\par 11 Ένεκεν εμού, ένεκεν εμού θέλω κάμει τούτο· διότι πως ήθελε μολυνθή το όνομά μου; ναι, δεν θέλω δώσει την δόξαν μου εις άλλον.
\par 12 Ακουσόν μου, Ιακώβ, και Ισραήλ τον οποίον εγώ εκάλεσα· εγώ αυτός είμαι· εγώ ο πρώτος, εγώ και ο έσχατος.
\par 13 Και η χειρ μου εθεμελίωσε την γην και η δεξιά μου εμέτρησε με σπιθαμήν τους ουρανούς· όταν καλώ αυτούς, παρίστανται ομού.
\par 14 Συνάχθητε, πάντες σεις, και ακούσατε· τις εκ τούτων ανήγγειλε ταύτα; Ο Κύριος ηγάπησεν αυτόν· όθεν θέλει εκπληρώσει το θέλημα αυτού επί την Βαβυλώνα και ο βραχίων αυτού θέλει είσθαι επί τους Χαλδαίους.
\par 15 Εγώ, εγώ ελάλησα· ναι, εκάλεσα αυτόν· έφερα αυτόν και εγώ θέλω ευοδώσει την οδόν αυτού.
\par 16 Πλησιάσατε προς εμέ, ακούσατε τούτο· απ' αρχής δεν ελάλησα εν κρυπτώ· αφότου έγεινε τούτο, εγώ ήμην εκεί και τώρα Κύριος ο Θεός απέστειλεν εμέ και το πνεύμα αυτού.
\par 17 Ούτω λέγει Κύριος, ο Λυτρωτής σου, ο Άγιος του Ισραήλ· Εγώ είμαι Κύριος ο Θεός σου, ο διδάσκων σε διά την ωφέλειάν σου, ο οδηγών σε διά της οδού δι' ης έπρεπε να υπάγης.
\par 18 Είθε να ήκουες τα προστάγματά μου τότε η ειρήνη σου ήθελεν είσθαι ως ποταμός και η δικαιοσύνη σου ως κύματα θαλάσσης·
\par 19 και το σπέρμα σου ήθελεν είσθαι ως η άμμος και τα έκγονα της κοιλίας σου ως τα λιθάρια αυτής· το όνομα αυτού δεν ήθελεν αποκοπή ουδέ εξαλειφθή απ' έμπροσθέν μου.
\par 20 Εξέλθετε εκ της Βαβυλώνος, φεύγετε από των Χαλδαίων, μετά φωνής αλαλαγμού αναγγείλατε, διακηρύξατε τούτο, εκφωνήσατε αυτό έως εσχάτου της γης, είπατε, Ο Κύριος ελύτρωσε τον δούλον αυτού Ιακώβ.
\par 21 Και δεν εδίψησαν, ότε ώδήγει αυτούς διά της ερήμου· έκαμε να ρεύσωσι δι' αυτούς ύδατα εκ πέτρας· και έσχισε την πέτραν και τα ύδατα έρρευσαν.
\par 22 Ειρήνη δεν είναι εις τους ασεβείς, λέγει Κύριος.

\chapter{49}

\par 1 Ακούσατέ μου, αι νήσοι· και προσέξατε, λαοί μακρυνοί· Ο Κύριος με εκάλεσεν εκ κοιλίας· εκ των σπλάγχνων της μητρός μου ανέφερε το όνομά μου.
\par 2 Και έκαμε το στόμα μου ως μάχαιραν οξείαν· υπό την σκιάν της χειρός αυτού με έκρυψε, και με έκαμεν ως βέλος εκλεκτόν, και εν τη φαρέτρα αυτού με έκρυψε,
\par 3 και είπε προς εμέ, Συ είσαι ο δούλός μου, Ισραήλ, εις τον οποίον θέλω δοξασθή.
\par 4 Και εγώ είπα, Ματαίως εκοπίασα· εις ουδέν και εις μάτην κατηνάλωσα την δύναμίν μου· πλην η κρίσις μου είναι μετά του Κυρίου και το έργον μου μετά του Θεού μου.
\par 5 Τώρα λοιπόν λέγει Κύριος, ο πλάσας με εκ κοιλίας δούλον αυτού, διά να επαναφέρω τον Ιακώβ προς αυτόν και διά να συναχθή προς αυτόν ο Ισραήλ, και θέλω δοξασθή εις τους οφθαλμούς του Κυρίου, και ο Θεός μου θέλει είσθαι η δύναμίς μου·
\par 6 και είπε, Μικρόν είναι το να ήσαι δούλός μου διά να ανορθώσης τας φυλάς του Ιακώβ και να επαναφέρης το υπόλοιπον του Ισραήλ· θέλω προσέτι σε δώσει φως εις τα έθνη, διά να ήσαι η σωτηρία μου έως εσχάτου της γης.
\par 7 Ούτω λέγει Κύριος, ο Λυτρωτής του Ισραήλ, ο Άγιος αυτού, προς εκείνον τον οποίον καταφρονεί άνθρωπος, προς εκείνον τον οποίον βδελύττεται έθνος, προς τον δούλον των εξουσιαστών· Βασιλείς θέλουσι σε ιδεί και σηκωθή, ηγεμόνες και θέλουσι σε προσκυνήσει, ένεκεν του Κυρίου, όστις είναι πιστός, του Αγίου του Ισραήλ, όστις σε εξέλεξεν
\par 8 Ούτω λέγει Κύριος· Εν καιρώ δεκτώ επήκουσά σου και εν ημέρα σωτηρίας σε εβοήθησα· και θέλω σε διαφυλάξει και θέλω σε δώσει εις διαθήκην των λαών, διά να ανορθώσης την γην, να κληροδοτήσης κληρονομίας ηρημωμένας·
\par 9 λέγων προς τους δεσμίους, Εξέλθετε· προς τους εν τω σκότει, Ανακαλύφθητε. Θέλουσι βοσκηθή πλησίον των οδών, και αι βοσκαί αυτών θέλουσιν είσθαι εν πάσι τοις υψηλοίς τόποις.
\par 10 δεν θέλουσι πεινάσει ουδέ διψήσει· δεν θέλει προσβάλλει αυτούς ούτε καύσων ούτε ήλιος· διότι ο ελεών αυτούς θέλει οδηγήσει αυτούς και διά πηγών υδάτων θέλει φέρει αυτούς.
\par 11 Και θέλω κάμει πάντα τα όρη μου οδούς, και αι τρίβοι μου θέλουσιν υψωθή.
\par 12 Ιδού, ούτοι θέλουσιν ελθεί μακρόθεν· και ιδού, ούτοι από βορρά και από νότου και ούτοι από της γης του Σινείμ.
\par 13 Ευφραίνεσθε, ουρανοί· και αγάλλου, η γή· αλαλάξατε, τα όρη· διότι ο Κύριος παρηγόρησε τον λαόν αυτού και τους τεθλιμμένους αυτού ελέησεν.
\par 14 Αλλ' η Σιών είπεν, Ο Κύριος με εγκατέλιπε και ο Κύριός μου με ελησμόνησε.
\par 15 Δύναται γυνή να λησμονήση το θηλάζον βρέφος αυτής, ώστε να μη ελεήση το τέκνον της κοιλίας αυτής; αλλά και αν αύται λησμονήσωσιν, εγώ όμως δεν θέλω σε λησμονήσει.
\par 16 Ιδού, επί των παλαμών μου σε εζωγράφισα· τα τείχη σου είναι πάντοτε ενώπιόν μου.
\par 17 Τα τέκνα σου θέλουσιν ελθεί μετά σπουδής· οι δε καταστρέφοντές σε και ερημόνοντές σε θέλουσιν εξέλθει από σου.
\par 18 Ύψωσον κύκλω τους οφθαλμούς σου και ιδέ· πάντες ούτοι συναθροίζονται ομού, έρχονται προς σε. Ζω εγώ, λέγει Κύριος, ότι συ θέλεις ενδυθή πάντας τούτους ως κόσμημα, και ως νύμφη θέλεις στολισθή αυτούς.
\par 19 Διότι οι ηφανισμένοι σου και οι ηρημωμένοι σου τόποι και η γη σου η κατεφθαρμένη θέλουσιν είσθαι τώρα παραπολύ μάλιστα στενοί διά τους κατοίκους σου· εκείνοι δε, οίτινες σε κατέτρωγον, θέλουσι μακρυνθή από σου.
\par 20 Τα τέκνα, τα οποία θέλεις αποκτήσει μετά την ατεκνίαν σου, θέλουσιν ειπεί προσέτι εις τα ώτα σου, Στενός είναι ο τόπος δι' εμέ· κάμε εις εμέ τόπον διά να κατοικήσω.
\par 21 Τότε θέλεις ειπεί εν τη καρδία σου, Τις εγέννησεν εις εμέ ταύτα, ενώ εγώ ήμην ητεκνωμένη και έρημος, αιχμάλωτος και μεταφερομένη; ταύτα δε τις εξέθρεψεν; ιδού, εγώ είχον εγκαταλειφθή μόνη· ταύτα που ήσαν;
\par 22 Ούτω λέγει Κύριος ο Θεός· Ιδού, θέλω υψώσει την χείρα μου προς τα έθνη και στήσει την σημαίαν μου προς τους λαούς, και θέλουσι φέρει τους υιούς σου εν ταις αγκάλαις και αι θυγατέρες σου θέλουσι φερθή επ' ώμων·
\par 23 και βασιλείς θέλουσιν είσθαι οι παιδοτρόφοι σου και αι βασίλισσαι αυτών αι τροφοί σου· θέλουσι σε προσκυνήσει με το πρόσωπον προς την γην και γλείφει το χώμα των ποδών σου· και θέλεις γνωρίσει, ότι εγώ είμαι ο Κύριος και ότι οι προσμένοντές με δεν θέλουσιν αισχυνθή.
\par 24 Δύναται το λάφυρον να αφαιρεθή από του ισχυρού ή να ελευθερωθώσιν οι δικαίως αιχμαλωτισθέντες;
\par 25 Αλλ' ο Κύριος ούτω λέγει· Και οι αιχμάλωτοι του ισχυρού θέλουσιν αφαιρεθή και το λάφυρον του τρομερού θέλει αποσπασθή· διότι εγώ θέλω δικολογήσει προς τους δικολογούντας κατά σου και εγώ θέλω σώσει τα τέκνα σου.
\par 26 Τους δε καταθλίβοντάς σε θέλω κάμει να φάγωσι τας ιδίας αυτών σάρκας· και θέλουσι μεθυσθή με το ίδιον αυτών αίμα, ως με νέον οίνον· και θέλει γνωρίσει πάσα σαρξ, ότι εγώ ο Κύριος είμαι ο Σωτήρ σου και ο Λυτρωτής σου, ο Ισχυρός του Ιακώβ.

\chapter{50}

\par 1 Ούτω λέγει Κύριος· Που είναι το έγγραφον του διαζυγίου της μητρός σας, δι' ου απέβαλον αυτήν; ή τις είναι εκ των δανειστών μου, εις τον οποίον σας επώλησα; Ιδού, διά τας ανομίας σας επωλήθητε, και διά τας παραβάσεις σας απεβλήθη η μήτηρ σας.
\par 2 Διά τι, ότε ήλθον, δεν υπήρχεν ουδείς; και ότε εκάλεσα, δεν υπήρχεν ο αποκρινόμενος; Εσμικρύνθη ποσώς η χειρ μου, ώστε να μη δύναται να λυτρώση; ή δεν έχω δύναμιν να ελευθερώσω; Ιδού, εγώ με την επιτίμησίν μου εξήρανα την θάλασσαν, έκαμα έρημον τους ποταμούς· οι ιχθύες αυτών εξηράνθησαν δι' έλλειψιν ύδατος και απέθανον υπό δίψης.
\par 3 Εγώ περιενδύω τους ουρανούς σκότος και θέτω σάκκον το περικάλυμμα αυτών.
\par 4 Κύριος ο Θεός έδωκεν εις εμέ γλώσσαν πεπαιδευμένων, διά να εξεύρω πως να λαλήσω λόγον εν καιρώ προς τον βεβαρυμένον· εγείρει από πρωΐ εις πρωΐ, εγείρει το ωτίον μου, διά να ακούω ως οι πεπαιδευμένοι.
\par 5 Κύριος ο Θεός ήνοιξεν ωτίον εν εμοί και εγώ δεν ηπείθησα ουδέ εστράφην οπίσω.
\par 6 Τον νώτόν μου έδωκα εις τους μαστιγούντας και τας σιαγόνας μου εις τους μαδίζοντας· δεν έκρυψα το πρόσωπόν μου από υβρισμών και εμπτυσμάτων.
\par 7 Διότι Κύριος ο Θεός θέλει με βοηθήσει· διά τούτο δεν ενετράπην· διά τούτο έθεσα το πρόσωπόν μου ως πέτραν σκληράν και εξεύρω ότι δεν θέλω καταισχυνθή.
\par 8 Πλησίον είναι ο δικαιόνων με· τις θέλει κριθή μετ' εμού; ας παρασταθώμεν ομού· τις είναι η αντίδικός μου; ας πλησιάση εις εμέ.
\par 9 Ιδού, Κύριος ο Θεός θέλει με βοηθήσει· τις θέλει με καταδικάσει ιδού, πάντες ούτοι θέλουσι παλαιωθή ως ιμάτιον· ο σκώληξ θέλει καταφάγει αυτούς.
\par 10 Τις είναι μεταξύ σας ο φοβούμενος τον Κύριον, ο υπακούων εις την φωνήν του δούλου αυτού; ούτος, και αν περιπατή εν σκότει και δεν έχη φως, ας θαρρή επί το όνομα του Κυρίου και ας επιστηρίζεται επί τον Θεόν αυτού.
\par 11 Ιδού, πάντες σεις, οι ανάπτοντες πυρ και περικυκλούμενοι με σπινθήρας, περιπατείτε εν τω φωτί του πυρός σας και διά των σπινθήρων τους οποίους εξήψατε. Τούτο σας έγεινεν υπό της χειρός μου, εν λύπη θέλετε κοίτεσθαι.

\chapter{51}

\par 1 Ακούσατέ μου, σεις οι ακολουθούντες την δικαιοσύνην, οι ζητούντες τον Κύριον· εμβλέψατε εις τον βράχον, εκ του οποίου ελατομήθητε, και εις το στόμιον του λάκκου, εκ του οποίου ανωρύχθητε.
\par 2 Εμβλέψατε εις τον Αβραάμ τον πατέρα σας και εις την Σάρραν, ήτις σας εγέννησε· διότι εκάλεσα αυτόν όντα ένα και ευλόγησα αυτόν και επλήθυνα αυτόν.
\par 3 Ο Κύριος λοιπόν θέλει παρηγορήσει την Σιών· αυτός θέλει παρηγορήσει πάντας τους ηρημωμένους τόπους αυτής· και θέλει κάμει την έρημον αυτής ως την Εδέμ και την ερημίαν αυτής ως παράδεισον του Κυρίου· ευφροσύνη και αγαλλίασις θέλει ευρίσκεσθαι εν αυτή, δοξολογία και φωνή αινέσεως.
\par 4 Ακουσόν μου, λαέ μου· και δος ακρόασιν εις εμέ, έθνος μου· διότι νόμος θέλει εξέλθει παρ' εμού και θέλω στήσει την κρίσιν μου διά φως των λαών.
\par 5 Η δικαιοσύνη μου πλησιάζει· η σωτηρία μου εξήλθε και οι βραχίονές μου θέλουσι κρίνει τους λαούς· αι νήσοι θέλουσι προσμένει εμέ και θέλουσιν ελπίζει επί τον βραχίονά μου.
\par 6 Υψώσατε τους οφθαλμούς σας εις τους ουρανούς και βλέψατε εις την γην κάτω· διότι οι ουρανοί θέλουσι διαλυθή ως καπνός και η γη θέλει παλαιωθή ως ιμάτιον και οι κατοικούντες εν αυτή θέλουσιν αποθάνει εξίσου· αλλ' η σωτηρία μου θέλει είσθαι εις τον αιώνα και η δικαιοσύνη μου δεν θέλει εκλείψει.
\par 7 Ακούσατέ μου, σεις οι γνωρίζοντες δικαιοσύνην· λαέ, εν τη καρδία του οποίου είναι ο νόμος μου· μη φοβείσθε τον ονειδισμόν των ανθρώπων μηδέ ταράττεσθε εις τας ύβρεις αυτών.
\par 8 Διότι ως ιμάτιον θέλει καταφάγει αυτούς ο σκώληξ και ως μαλλίον θέλει καταφάγει αυτούς ο σκώρος· αλλ' η δικαιοσύνη μου θέλει μένει εις τον αιώνα και η σωτηρία μου εις γενεάς γενεών.
\par 9 Εξεγέρθητι, εξεγέρθητι, ενδύθητι δύναμιν, βραχίων Κυρίου· εξεγέρθητι ως εν ταις αρχαίαις ημέραις, εν ταις παλαιαίς γενεαίς. Δεν είσαι συ, ο πατάξας την Ραάβ και τραυματίσας τον δράκοντα;
\par 10 Δεν είσαι συ, ο ξηράνας την θάλασσαν, τα ύδατα της μεγάλης αβύσσου; ο ποιήσας τα βάθη της θαλάσσης οδόν διαβάσεως των λελυτρωμένων;
\par 11 Και οι λελυτρωμένοι του Κυρίου θέλουσιν επιστρέψει και ελθεί εν αλαλαγμώ εις Σιών· και ευφροσύνη αιώνιος θέλει είσθαι επί της κεφαλής αυτών· αγαλλίασιν και ευφροσύνην θέλουσιν απολαύσει· η λύπη και ο στεναγμός θέλουσι φύγει.
\par 12 Εγώ, εγώ είμαι ο παρηγορών υμάς. Συ τις είσαι και φοβείσαι από ανθρώπου θνητού και από υιού ανθρώπου, όστις θέλει γείνει ως χόρτος·
\par 13 και ελησμόνησας Κύριον τον Ποιητήν σου, τον εκτείναντα τους ουρανούς και θεμελιώσαντα την γήν· και εφοβείσο πάντοτε καθ' ημέραν την οργήν του καταθλίβοντός σε, ως εάν ήτο έτοιμος να καταστρέψη; και που είναι τώρα η οργή του καταθλίβοντος;
\par 14 Ο ηχμαλωτισμένος σπεύδει να λυθή και να μη αποθάνη εν τω λάκκω μηδέ να στερηθή τον άρτον αυτού·
\par 15 διότι εγώ είμαι Κύριος ο Θεός σου, ο ταράττων την θάλασσαν και ηχούσι τα κύματα αυτής· Κύριος των δυνάμεων το όνομα αυτού.
\par 16 Και έθεσα τους λόγους μου εις το στόμα σου και σε εσκέπασα με την σκιάν της χειρός μου, διά να στερεώσω τους ουρανούς και να θεμελιώσω την γήν· και διά να είπω προς την Σιών, Λαός μου είσαι.
\par 17 Εξεγέρθητι, εξεγέρθητι, ανάστηθι, Ιερουσαλήμ, ήτις έπιες εκ της χειρός του Κυρίου το ποτήριον του θυμού αυτού· έπιες, εξεκένωσας και αυτήν την τρυγίαν του ποτηρίου της ζάλης.
\par 18 Εκ πάντων των υιών, τους οποίους εγέννησε, δεν υπάρχει ο οδηγών αυτήν· ουδέ είναι εκ πάντων των υιών, τους οποίους εξέθρεψεν, ο πιάνων αυτήν εκ της χειρός.
\par 19 Τα δύο ταύτα ήλθον επί σέ· τις θέλει σε συλλυπηθή; ερήμωσις και καταστροφή και πείνα και μάχαιρα· διά τίνος να σε παρηγορήσω;
\par 20 Οι υιοί σου απενεκρώθησαν· κοίτονται απ' άκρου πασών των οδών, ως άγριος ταύρος εν δικτύοις· είναι πλήρεις του θυμού του Κυρίου, της επιτιμήσεως του Θεού σου.
\par 21 Όθεν, άκουε τώρα τούτο, τεθλιμμένη και μεθύουσα, πλην ουχί εξ οίνου·
\par 22 ούτω λέγει ο Κύριός σου, ο Κύριος και ο Θεός σου, ο δικολογών υπέρ του λαού αυτού· Ιδού, έλαβον εκ των χειρών σου το ποτήριον της ζάλης, την τρυγίαν του ποτηρίου του θυμού μου· δεν θέλεις πλέον πίει αυτό του λοιπού·
\par 23 και θέλω βάλει αυτό εις την χείρα των καταθλιβόντων σε, οίτινες είπον προς την ψυχήν σου, Κύψον, διά να περάσωμεν· και συ έβαλες το σώμα σου ως γην και ως οδόν εις τους διαβαίνοντας.

\chapter{52}

\par 1 Εξεγέρθητι, εξεγέρθητι, ενδύθητι την δύναμίν σου, Σιών· ενδύθητι τα ιμάτια της μεγαλοπρεπείας σου, Ιερουσαλήμ, πόλις αγία· διότι του λοιπού δεν θέλει πλέον εισέλθει εις σε ο απερίτμητος και ακάθαρτος.
\par 2 Εκτινάχθητι από το χώμα· σηκώθητι, κάθησον, Ιερουσαλήμ· λύσον τα δεσμά από του τραχήλου σου, αιχμάλωτος θυγάτηρ της Σιών.
\par 3 Διότι ούτω λέγει Κύριος· Επωλήθητε διά μηδέν και θέλετε λυτρωθή άνευ αργυρίου.
\par 4 Διότι ούτω λέγει Κύριος ο Θεός· Ο λαός μου κατέβη το πρότερον εις την Αίγυπτον διά να παροικήση εκεί και οι Ασσύριοι αναιτίως κατέθλιψαν αυτούς.
\par 5 Τώρα λοιπόν, τι έχω να κάμω εδώ, λέγει Κύριος, επειδή ο λαός μου ελήφθη διά μηδέν; οι εξουσιάζοντες επ' αυτού κάμνουσιν εαυτόν να ολολύζη, λέγει Κύριος· και το όνομά μου βλασφημείται πάντοτε καθ' ημέραν.
\par 6 Διά τούτο ο λαός μου θέλει γνωρίσει το όνομά μου· διά τούτο θέλει γνωρίσει εν εκείνη τη ημέρα, ότι εγώ είμαι ο λαλών· ιδού, εγώ.
\par 7 Πόσον ωραίοι είναι επί των ορέων οι πόδες του ευαγγελιζομένου, του κηρύττοντος ειρήνην· του ευαγγελιζομένου αγαθά, του κηρύττοντος σωτηρίαν, του λέγοντος προς την Σιών· Ο Θεός σου βασιλεύει.
\par 8 Οι φύλακές σου θέλουσιν υψώσει φωνήν· εν φωναίς ομού θέλουσιν αλαλάζει· διότι θέλουσιν ιδεί οφθαλμός προς οφθαλμόν, όταν ο Κύριος ανορθώση την Σιών.
\par 9 Αλαλάξατε, ευφράνθητε ομού, ηρημωμένοι τόποι της Ιερουσαλήμ· διότι ο Κύριος παρηγόρησε τον λαόν αυτού, ελύτρωσε την Ιερουσαλήμ.
\par 10 Ο Κύριος εγύμνωσε τον άγιον βραχίονα αυτού ενώπιον πάντων των εθνών· και πάντα τα πέρατα της γης θέλουσιν ιδεί την σωτηρίαν του Θεού ημών.
\par 11 Σύρθητε, σύρθητε, εξέλθετε εκείθεν, μη εγγίσητε ακάθαρτον· εξέλθετε εκ μέσου αυτής· καθαρίσθητε σεις οι βαστάζοντες τα σκεύη του Κυρίου·
\par 12 διότι δεν θέλετε εξέλθει εν βία, ουδέ μετά φυγής θέλετε οδοιπορήσει· διότι ο Κύριος θέλει υπάγει έμπροσθέν σας και ο Θεός του Ισραήλ θέλει είσθαι η οπισθοφυλακή σας.
\par 13 Ιδού, ο δούλός μου θέλει ευοδωθή· θέλει υψωθή και δοξασθή και αναβή υψηλά σφόδρα.
\par 14 Καθώς πολλοί έμειναν εκστατικοί επί σε, τόσον ήτο το πρόσωπον αυτού άδοξον παρά παντός ανθρώπου και το είδος αυτού παρά των υιών των ανθρώπων.
\par 15 Ούτω θέλει ραντίσει πολλά έθνη· οι βασιλείς θέλουσι φράξει το στόμα αυτών επ' αυτόν· διότι θέλουσιν ιδεί εκείνο το οποίον δεν ελαλήθη προς αυτούς· και θέλουσι νοήσει εκείνο, το οποίον δεν ήκουσαν.

\chapter{53}

\par 1 Τις επίστευσεν εις το κήρυγμα ημών; και ο βραχίων του Κυρίου εις τίνα απεκαλύφθη;
\par 2 διότι ανέβη ενώπιον αυτού ως τρυφερόν φυτόν και ως ρίζα από ξηράς γής· δεν έχει είδος ουδέ κάλλος· και είδομεν αυτόν και δεν είχεν ώραιότητα ώστε να επιθυμώμεν αυτόν.
\par 3 Καταπεφρονημένος και απερριμμένος υπό των ανθρώπων· άνθρωπος θλίψεων και δόκιμος ασθενείας· και ως άνθρωπος από του οποίου αποστρέφει τις το πρόσωπον, κατεφρονήθη και ως ουδέν ελογίσθημεν αυτόν.
\par 4 Αυτός τωόντι τας ασθενείας ημών εβάστασε και τας θλίψεις ημών επεφορτίσθη· ημείς δε ενομίσαμεν αυτόν τετραυματισμένον, πεπληγωμένον υπό Θεού και τεταλαιπωρημένον.
\par 5 Αλλ' αυτός ετραυματίσθη διά τας παραβάσεις ημών, εταλαιπωρήθη διά τας ανομίας ημών· η τιμωρία, ήτις έφερε την ειρήνην ημών, ήτο επ' αυτόν· και διά των πληγών αυτού ημείς ιάθημεν.
\par 6 Πάντες ημείς επλανήθημεν ως πρόβατα· εστράφημεν έκαστος εις την οδόν αυτού· και ο Κύριος έθεσεν επ' αυτόν την ανομίαν πάντων ημών.
\par 7 Αυτός ήτο κατατεθλιμμένος και βεβασανισμένος αλλά δεν ήνοιξε το στόμα αυτού· εφέρθη ως αρνίον επί σφαγήν, και ως πρόβατον έμπροσθεν του κείροντος αυτό άφωνον, ούτω δεν ήνοιξε το στόμα αυτού.
\par 8 Από καταθλίψεως και κρίσεως ανηρπάχθη· την δε γενεάν αυτού τις θέλει διηγηθή; διότι εσηκώθη από της γης των ζώντων· διά τας παραβάσεις του λαού μου ετραυματίσθη.
\par 9 Και ο τάφος αυτού διωρίσθη μετά των κακούργων· πλην εις τον θάνατον αυτού εστάθη μετά του πλουσίου· διότι δεν έκαμεν ανομίαν ουδέ ευρέθη δόλος εν τω στόματι αυτού.
\par 10 Αλλ' ο Κύριος ηθέλησε να βασανίση αυτόν· εταλαιπώρησεν αυτόν. Αφού όμως δώσης την ψυχήν αυτού προσφοράν περί αμαρτίας, θέλει ιδεί έκγονα, θέλει μακρύνει τας ημέρας αυτού, και το θέλημα του Κυρίου θέλει ευοδωθή εν τη χειρί αυτού.
\par 11 Θέλει ιδεί τους καρπούς του πόνου της ψυχής αυτού και θέλει χορτασθή· ο δίκαιος δούλός μου θέλει δικαιώσει πολλούς διά της επιγνώσεως αυτού· διότι αυτός θέλει βαστάσει τας ανομίας αυτών.
\par 12 Διά τούτο θέλω δώσει εις αυτόν μερίδα μετά των μεγάλων και τους ισχυρούς θέλει μοιρασθή λάφυρον, διότι παρέδωκε την ψυχήν αυτού εις θάνατον και μετά ανόμων ελογίσθη και αυτός εβάστασε τας αμαρτίας πολλών και θέλει μεσιτεύσει υπέρ των ανόμων.

\chapter{54}

\par 1 Ευφράνθητι, στείρα, η μη τίκτουσα· αναβόησον εν αγαλλιάσει και τέρπου, η μη ωδίνουσα· διότι πλειότερα είναι τα τέκνα της ηρημωμένης παρά τα τέκνα της εχούσης τον άνδρα, λέγει Κύριος.
\par 2 Πλάτυνον τον τόπον της σκηνής σου και ας εκτείνωσι τα παραπετάσματα των κατοικιών σου· μη φεισθής· μάκρυνον τα σχοινία σου και στερέωσον τους πασσάλους σου.
\par 3 Διότι θέλεις εκταθή εις τα δεξιά και εις τα αριστερά· και το σπέρμα σου θέλει κληρονομήσει τα έθνη και θέλει κάμει τας ηρημωμένας πόλεις να κατοικισθώσι.
\par 4 Μη φοβού· διότι δεν θέλεις καταισχυνθή· μηδέ εντρέπου· διότι δεν θέλεις αισχυνθή· διότι θέλεις λησμονήσει την αισχύνην της νεότητός σου και δεν θέλεις ενθυμηθή πλέον το όνειδος της χηρείας σου.
\par 5 Διότι ο ανήρ σου είναι ο Ποιητής σου· το όνομα αυτού είναι, Ο Κύριος των δυνάμεων· και ο Λυτρωτής σου είναι ο Άγιος του Ισραήλ· αυτός θέλει ονομασθή, Ο Θεός πάσης της γης.
\par 6 Διότι ο Κύριος σε εκάλεσεν ως γυναίκα εγκαταλελειμμένην και τεθλιμμένην το πνεύμα και γυναίκα νεότητος αποβεβλημένην, λέγει ο Θεός σου.
\par 7 Σε εγκατέλιπον διά ολίγον καιρόν· πλην με έλεος μέγα θέλω σε περισυνάξει.
\par 8 Εν θυμώ μικρώ έκρυψα το πρόσωπόν μου από σου διά μίαν στιγμήν· με έλεος όμως αιώνιον θέλω σε ελεήσει, λέγει Κύριος ο Λυτρωτής σου.
\par 9 Διότι τούτο είναι εις εμέ ως τα ύδατα του Νώε· επειδή, καθώς ώμοσα ότι τα ύδατα του Νώε δεν θέλουσιν επέλθει πλέον επί την γην, ούτως ώμοσα ότι δεν θέλω θυμωθή πλέον κατά σου ουδέ σε ελέγξει.
\par 10 Διότι τα όρη θέλουσι μετατοπισθή και οι λόφοι μετακινηθή· πλην το έλεός μου δεν θέλει εκλείψει από σου ουδέ η διαθήκη της ειρήνης μου μετακινηθή, λέγει Κύριος ο ελεών σε.
\par 11 Ω τεθλιμμένη, τεταραγμένη, απαρηγόρητος, ιδού, εγώ θέλω στρώσει τους λίθους σου εκ μαρμάρων πορφυρών και θέλω βάλει τα θεμέλιά σου εκ σαπφείρων.
\par 12 Και θέλω κάμει τας επάλξεις σου εξ αχάτου και τας πύλας σου εξ ανθράκων και άπαντα τον περίβολόν σου εκ λίθων εκλεκτών.
\par 13 Πάντες δε οι υιοί σου θέλουσιν είσθαι διδακτοί του Κυρίου, και θέλει είσθαι μεγάλη η ειρήνη των υιών σου.
\par 14 Εν δικαιοσύνη θέλεις στερεωθή· μακράν από της καταδυναστείας θέλεις είσθαι, διότι δεν θέλεις φοβείσθαι· και από του τρόμου, διότι δεν θέλει σε πλησιάσει.
\par 15 Ιδού, βεβαίως θέλουσι συναχθή ομού εναντίον σου, πλην ουχί δι' εμού· Όσοι συναχθώσιν ομού εναντίον σου, ένεκα σου, θέλουσι πέσει.
\par 16 Ιδού, εγώ έκαμον τον χαλκέα, όστις φυσά τους άνθρακας εν τω πυρί και εξάγει το εργαλείον διά το έργον αυτού· και εγώ έκαμον τον πορθητήν διά να καταστρέφη.
\par 17 Ουδέν όπλον κατασκευασθέν εναντίον σου θέλει ευοδωθή· και πάσαν γλώσσαν, ήτις ήθελε κινηθή κατά σου, θέλεις νικήσει εν τη κρίσει. Αύτη είναι η κληρονομία των δούλων του Κυρίου· και η δικαιοσύνη αυτών είναι εξ εμού, λέγει ο Κύριος.

\chapter{55}

\par 1 Ω πάντες οι διψώντες, έλθετε εις τα ύδατα· και οι μη έχοντες αργύριον, έλθετε, αγοράσατε και φάγετε· ναι έλθετε, αγοράσατε οίνον και γάλα άνευ αργυρίου και άνευ αντιτίμου.
\par 2 Διά τι εξοδεύετε αργύρια ουχί εις άρτον; και τον κόπον σας ουχί εις χορτασμόν; ακούσατέ μου μετά προσοχής και θέλετε φάγει αγαθά και η ψυχή σας θέλει ευφρανθή εις το πάχος.
\par 3 Κλίνατε το ωτίον σας και έλθετε προς εμέ· ακούσατε και η ψυχή σας θέλει ζήσει· και θέλω κάμει προς εσάς αιώνιον διαθήκην, τα ελέη του Δαβίδ τα πιστά.
\par 4 Ιδού, έδωκα αυτόν μαρτύριον εις τους λαούς, άρχοντα και προστάττοντα εις τους λαούς.
\par 5 Ιδού, θέλεις καλέσει έθνος, το οποίον δεν εγνώριζες· και έθνη, τα οποία δεν σε εγνώριζον, θέλουσι τρέξει προς σε, διά Κύριον τον Θεόν σου και διά τον Άγιον του Ισραήλ· διότι σε εδόξασε.
\par 6 Ζητείτε τον Κύριον, ενόσω δύναται να ευρεθή· επικαλείσθε αυτόν, ενόσω είναι πλησίον.
\par 7 Ας εγκαταλίπη ο ασεβής την οδόν αυτού και ο άδικος τας βουλάς αυτού· και ας επιστρέψη προς τον Κύριον, και θέλει ελεήσει αυτόν· και προς τον Θεόν ημών, διότι αυτός θέλει συγχωρήσει αφθόνως.
\par 8 Διότι αι βουλαί μου δεν είναι βουλαί υμών ουδέ οδοί υμών αι οδοί μου, λέγει Κύριος.
\par 9 Αλλ' όσον είναι υψηλοί οι ουρανοί από της γης, ούτως αι οδοί μου είναι υψηλότεραι των οδών υμών και αι βουλαί μου των βουλών υμών.
\par 10 Διότι καθώς καταβαίνει η βροχή και η χιών εκ του ουρανού και δεν επιστρέφει εκεί, αλλά ποτίζει την γην και κάμνει αυτήν να εκφύη και να βλαστάνη, διά να δώση σπόρον εις τον σπείροντα και άρτον εις τον εσθίοντα,
\par 11 ούτω θέλει είσθαι ο λόγος μου ο εξερχόμενος εκ του στόματός μου· δεν θέλει επιστρέψει εις εμέ κενός, αλλά θέλει εκτελέσει το θέλημά μου και θέλει ευοδωθή εις ό,τι αυτόν αποστέλλω.
\par 12 Διότι θέλετε εξέλθει εν χαρά και οδηγηθή εν ειρήνη· τα όρη και οι λόφοι θέλουσιν αντηχήσει έμπροσθέν σας υπό αγαλλιάσεως και πάντα τα δένδρα του αγρού θέλουσιν επικροτήσει τας χείρας.
\par 13 Αντί της ακάνθης θέλει αναβή κυπάρισσος, αντί της κνίδης θέλει αναβή μυρσίνη· και τούτο θέλει είσθαι εις τον Κύριον διά όνομα, διά σημείον αιώνιον, το οποίον δεν θέλει εκλείψει.

\chapter{56}

\par 1 Ούτω λέγει Κύριος· Φυλάττετε κρίσιν και κάμνετε δικαιοσύνην· διότι η σωτηρία μου πλησιάζει να έλθη και η δικαιοσύνη μου να αποκαλυφθή.
\par 2 Μακάριος ο άνθρωπος όστις κάμνει τούτο, και ο υιός του ανθρώπου όστις κρατεί αυτό· όστις φυλάττει το σάββατον, ώστε να μη βεβηλώση αυτό, και κρατεί την χείρα αυτού, ώστε να μη πράξη μηδέν κακόν.
\par 3 Ο δε υιός του αλλογενούς, ο προστεθειμένος εις τον Κύριον, ας μη είπη, λέγων, Ο Κύριος διόλου θέλει με χωρίσει από του λαού αυτού· μηδέ ο ευνούχος ας λέγη·, Ιδού, εγώ είμαι δένδρον ξηρόν.
\par 4 Διότι ούτω λέγει Κύριος· εις τους ευνούχους, όσοι φυλάττουσι τα σάββατά μου και εκλέγουσι τα αρέσκοντα εις εμέ και κρατούσι την διαθήκην μου,
\par 5 εις αυτούς μάλιστα θέλω δώσει εν τω οίκω μου και εντός των τειχών μου τόπον και όνομα καλήτερον παρά των υιών και των θυγατέρων· εις αυτούς θέλω δώσει όνομα αιώνιον, το οποίον δεν θέλει εκλείψει.
\par 6 Περί δε των υιών του αλλογενούς, οίτινες ήθελον προστεθή εις τον Κύριον, διά να δουλεύωσιν εις αυτόν και να αγαπώσι το όνομα του Κυρίου, διά να ήναι δούλοι αυτού· όσοι φυλάττουσι το σάββατον, ώστε να μη βεβηλώσωσιν αυτό και κρατούσι την διαθήκην μου·
\par 7 και τούτους θέλω φέρει εις το άγιόν μου όρος και θέλω ευφράνει αυτούς εν τω οίκω της προσευχής μου· τα ολοκαυτώματα αυτών και αι θυσίαι αυτών θέλουσιν είσθαι δεκταί επί το θυσιαστήριόν μου· διότι ο οίκός μου θέλει ονομάζεσθαι, Οίκος προσευχής διά πάντας τους λαούς.
\par 8 Ούτω λέγει Κύριος ο Θεός ο συνάγων τους διεσκορπισμένους του Ισραήλ· Θέλω συνάξει έτι και άλλους εις αυτόν, εκτός των συνηγμένων αυτού.
\par 9 Έλθετε, φάγετε, πάντα τα ζώα του αγρού, πάντα τα θηρία του δάσους.
\par 10 Οι δε φύλακες αυτού είναι τυφλοί· πάντες χωρίς νοήσεως· πάντες κύνες άλαλοι, μη δυνάμενοι να υλακτήσωσι· κοιμώμενοι, κοιτόμενοι, αγαπώντες νυσταγμόν·
\par 11 ναι, κύνες αδηφάγοι, οίτινες δεν γνωρίζουσι χορτασμόν και ποιμένες, οίτινες δεν γνωρίζουσι σύνεσιν· πάντες είναι εστραμμένοι προς την οδόν αυτών, έκαστος εις το μέρος αυτού, διά το κέρδος αυτών.
\par 12 Έλθετε, λέγουσι, θέλω φέρει οίνον και θέλομεν μεθυσθή με σίκερα· και αύριον θέλει είσθαι ως η ημέρα αύτη, πολύ πλέον άφθονος.

\chapter{57}

\par 1 Ο δίκαιος αποθνήσκει και ουδείς βάλλει τούτο εν τη καρδία αυτού· και οι άνδρες ελέους συλλέγονται, χωρίς να εννοή τις, αν ο δίκαιος συλλέγεται απ' έμπροσθεν της κακίας.
\par 2 Θέλει εισέλθει εις ειρήνην· οι περιπατούντες εν τη ευθύτητι αυτών, θέλουσιν αναπαυθή εν ταις κλίναις αυτών.
\par 3 Σεις δε οι υιοί της μαγίσσης, σπέρμα μοιχού και πόρνης, πλησιάσατε εδώ.
\par 4 Κατά τίνος εντρυφάτε; κατά τίνος επλατύνατε το στόμα, εξετείνατε την γλώσσαν; δεν είσθε τέκνα ανομίας, σπέρμα ψεύδους,
\par 5 φλογιζόμενοι με τα είδωλα υπό παν δένδρον πράσινον, σφάζοντες τα τέκνα εν ταις φάραγξιν, υπό τους κρήμνους των βράχων;
\par 6 Η μερίς σου είναι μεταξύ των χαλίκων των χειμάρρων· ούτοι, ούτοι είναι η κληρονομία σου· και εις αυτούς εξέχεας σπονδάς, προσέφερες προσφοράν εξ αλφίτων· εις ταύτα θέλω ευαρεστηθή;
\par 7 Επί όρους υψηλού και μετεώρου έβαλες την κλίνην σου· και εκεί ανέβης διά να προσφέρης θυσίαν.
\par 8 Και οπίσω των θυρών και των παραστατών έστησας το μνημόσυνόν σου· διότι εξεσκέπασας σεαυτήν αποστατήσασα απ' εμού και ανέβης· επλάτυνας την κλίνην σου και συνεφώνησας μετ' εκείνων· ηγάπησας την κλίνην αυτών, εξέλεξας τους τόπους·
\par 9 υπήγες μάλιστα προς τον βασιλέα με χρίσματα και ηύξησας τα αρώματά σου και απέστειλας μακράν τους πρέσβεις σου και εταπείνωσας σεαυτήν μέχρις άδου.
\par 10 Εκοπίασας εις το μάκρος της οδού σου· και δεν είπας, εις μάτην κοπιάζω· εύρηκας το ζην διά της χειρός σου· διά τούτο δεν απέκαμες.
\par 11 Και τίνα επτοήθης ή εφοβήθης, ώστε να ψευσθής και να μη με ενθυμηθής μηδέ να θέσης τούτο εν τη καρδία σου; δεν είναι, διότι εγώ εσιώπησα, μάλιστα προ πολλού, διά τούτο συ δεν με εφοβήθης;
\par 12 Εγώ θέλω απαγγείλει την δικαιοσύνην σου και τα έργα σου· όμως δεν θέλουσι σε ωφελήσει.
\par 13 Όταν αναβοήσης, ας σε ελευθερώσωσιν οι συνηγμένοι σου· αλλ' ο άνεμος θέλει αφαρπάσει πάντας αυτούς· η ματαιότης θέλει λάβει αυτούς· ο ελπίζων όμως επ' εμέ θέλει κληρονομήσει την γην και αποκτήσει το άγιόν μου όρος.
\par 14 Και θέλω ειπεί, Υψώσατε, υψώσατε, ετοιμάσατε την οδόν, εκβάλετε το πρόσκομμα από της οδού του λαού μου.
\par 15 Διότι ούτω λέγει ο Ύψιστος και ο Υπέρτατος, ο κατοικών την αιωνιότητα, του οποίου το όνομα είναι Ο Αγιος· Εγώ κατοικώ εν υψηλοίς και εν αγίω τόπω· και μετά του συντετριμμένου την καρδίαν και του ταπεινού το πνεύμα, διά να ζωοποιώ το πνεύμα των ταπεινών και να ζωοποιώ την καρδίαν των συντετριμμένων.
\par 16 Διότι δεν θέλω δικολογεί αιωνίως ουδέ θέλω είσθαι πάντοτε ωργισμένος· επειδή τότε ήθελον εκλείψει απ' έμπροσθέν μου το πνεύμα και αι ψυχαί τας οποίας έκαμον.
\par 17 Διά την ανομίαν της αισχροκερδείας αυτού ωργίσθην και επάταξα αυτόν· έκρυψα το πρόσωπόν μου και ωργίσθην· αλλά αυτός ηκολούθησε πεισματωδώς την οδόν της καρδίας αυτού.
\par 18 Είδον τας οδούς αυτού και θέλω ιατρεύσει αυτόν· και θέλω οδηγήσει αυτόν και δώσει πάλιν παρηγορίας εις αυτόν και εις τους τεθλιμμένους αυτού.
\par 19 Εγώ δημιουργώ τον καρπόν των χειλέων· ειρήνην, ειρήνην, εις τον μακράν και εις τον πλησίον, λέγει Κύριος· και θέλω ιατρεύσει αυτόν.
\par 20 Οι δε ασεβείς είναι ως η τεταραγμένη θάλασσα, όταν δεν δύναται να ησυχάση· και τα κύματα αυτής εκρίπτουσι καταπάτημα και πηλόν.
\par 21 Ειρήνη δεν είναι εις τους ασεβείς, λέγει ο Θεός μου.

\chapter{58}

\par 1 Αναβόησον δυνατά, μη φεισθής· ύψωσον την φωνήν σου ως σάλπιγγα και ανάγγειλον προς τον λαόν μου τας ανομίας αυτών και προς τον οίκον Ιακώβ τας αμαρτίας αυτών.
\par 2 Με ζητούσιν όμως καθ' ημέραν και επιθυμούσι να μανθάνωσι τας οδούς μου, ως έθνος το οποίον έκαμε δικαιοσύνην και δεν εγκατέλιπε την κρίσιν του Θεού αυτού· ζητούσι παρ' εμού κρίσεις δικαιοσύνης· επιθυμούσι να πλησιάζωσιν εις τον Θεόν.
\par 3 Διά τι ενηστεύσαμεν, λέγουσι, και δεν είδες; εταλαιπωρήσαμεν την ψυχήν ημών και δεν εγνώρισας; Ιδού, εν τη ημέρα της νηστείας σας ευρίσκετε ηδονήν και καταθλίβετε πάντας τους μισθωτούς σας.
\par 4 Ιδού, νηστεύετε διά δίκας και έριδας και γρονθίζετε ασεβώς· μη νηστεύετε, καθώς την σήμερον, διά να ακουσθή άνωθεν η φωνή σας.
\par 5 Τοιαύτη είναι η νηστεία, την οποίαν εγώ εξέλεξα; να ταλαιπωρή ο άνθρωπος την ψυχήν αυτού μίαν ημέραν; να κλίνη την κεφαλήν αυτού ως σπάρτον και να υποστρόνη σάκκον και στάκτην εις εαυτόν; νηστείαν θέλεις ονομάσει τούτο και ημέραν δεκτήν εις τον Κύριον;
\par 6 Η νηστεία την οποίαν εγώ εξέλεξα, δεν είναι αύτη; το να λύης τους δεσμούς της κακίας, το να διαλύης τα βαρέα φορτία και το να αφίνης ελευθέρους τους καταδεδυναστευμένους και το να συντρίβης πάντα ζυγόν;
\par 7 Δεν είναι το να διαμοιράζης τον άρτον σου εις τον πεινώντα και να εισάγης εις την οικίαν σου τους αστέγους πτωχούς; όταν βλέπης τον γυμνόν, να ενδύης αυτόν, και να μη κρύπτης σεαυτόν από της σαρκός σου;
\par 8 Τότε το φως σου θέλει εκλάμψει ως η αυγή και η υγιεία σου ταχέως θέλει βλαστήσει· και η δικαιοσύνη σου θέλει προπορεύεσθαι έμπροσθέν σου· η δόξα του Κυρίου θέλει είσθαι η οπισθοφυλακή σου.
\par 9 Τότε θέλεις κράζει και ο Κύριος θέλει αποκρίνεσθαι· θέλεις φωνάζει και εκείνος θέλει λέγει, Ιδού, εγώ. Εάν εκβάλης εκ μέσου σου τον ζυγόν, την ανάτασιν του δακτύλου και τους ματαίους λόγους·
\par 10 και ανοίγης την ψυχήν σου προς τον πεινώντα και ευχαριστής την τεθλιμμένην ψυχήν· τότε το φως σου θέλει ανατέλλει εν τω σκότει και το σκότος σου θέλει είσθαι ως μεσημβρία.
\par 11 Και ο Κύριος θέλει σε οδηγεί πάντοτε και χορταίνει την ψυχήν σου εν ανομβρίαις και παχύνει τα οστά σου· και θέλεις είσθαι ως κήπος ποτιζόμενος και ως πηγή ύδατος, της οποίας τα ύδατα δεν εκλείπουσι.
\par 12 Και οι από σου θέλουσιν οικοδομήσει τας παλαιάς ερημώσεις· θέλεις ανεγείρει τα θεμέλια πολλών γενεών· και θέλεις ονομασθή, Ο επιδιορθωτής των χαλασμάτων, Ο ανορθωτής των οδών διά τον κατοικισμόν.
\par 13 Εάν αποστρέψης τον πόδα σου από του σαββάτου, από του να κάμνης τα θελήματά σου εν τη αγία μου ημέρα, και ονομάζης το σάββατον τρυφήν, αγίαν ημέραν του Κυρίου, έντιμον, και τιμάς αυτό, μη ακολουθών τας οδούς σου μηδέ ευρίσκων εν αυτώ το θέλημά σου μηδέ λαλών τους λόγους σου,
\par 14 τότε θέλεις εντρυφά εν Κυρίω· και εγώ θέλω σε ιππεύσει επί τους υψηλούς τόπους της γης και σε θρέψει με την κληρονομίαν του πατρός σου Ιακώβ· διότι το στόμα τον Κυρίου ελάλησε.

\chapter{59}

\par 1 Ιδού, η χειρ του Κυρίου δεν εσμικρύνθη, ώστε να μη δύναται να σώση, ουδέ το ωτίον αυτού εβάρυνεν, ώστε να μη δύναται να ακούση·
\par 2 αλλ' αι ανομίαι σας έβαλον χωρίσματα μεταξύ υμών και του Θεού υμών, και αι αμαρτίαι σας έκρυψαν το πρόσωπον αυτού από σας, διά να μη ακούη.
\par 3 Διότι αι χείρές σας είναι μεμολυσμέναι από αίματος και οι δάκτυλοί σας από ανομίας· τα χείλη σας ελάλησαν ψεύδη· η γλώσσα σας εμελέτησε κακίαν.
\par 4 Ουδείς εκζητεί την δικαιοσύνην ουδέ κρίνει εν αληθεία· θαρρούσιν επί την ματαιότητα και λαλούσι ψεύδη· συλλαμβάνουσι κακίαν και γεννώσιν ανομίαν.
\par 5 Βασιλίσκου ωά επωάζουσι και ιστόν αράχνης υφαίνουσιν· όστις φάγη εκ των ωών αυτών, αποθνήσκει· και αν σπάσης κανέν, εξέρχεται έχιδνα.
\par 6 Τα πανία αυτών δεν θέλουσι χρησιμεύσει εις ενδύματα, ουδέ θέλουσιν ενδυθή από των έργων αυτών· τα έργα αυτών είναι έργα ανομίας, και το έργον της βίας είναι εν ταις χερσίν αυτών.
\par 7 Οι πόδες αυτών τρέχουσι προς το κακόν και σπεύδουσι να χύσωσιν αίμα αθώον· οι διαλογισμοί αυτών είναι διαλογισμοί ανομίας· ερήμωσις και καταστροφή είναι εν ταις οδοίς αυτών.
\par 8 Την οδόν της ειρήνης δεν γνωρίζουσι· και δεν είναι κρίσις εις τα βήματα αυτών· αυτοί εις εαυτούς διέστρεψαν τας οδούς αυτών· πας ο περιπατών εν αυταίς δεν γνωρίζει ειρήνην.
\par 9 Διά τούτο η κρίσις είναι μακράν αφ' ημών και η δικαιοσύνη δεν μας φθάνει· προσμένομεν φως και ιδού, σκότος· λάμψιν, και περιπατούμεν εν ζόφω.
\par 10 Ψηλαφώμεν τον τοίχον ως οι τυφλοί, και ψηλαφώμεν ως οι μη έχοντες οφθαλμούς· εν μεσημβρία προσκόπτομεν ως εν νυκτί· είμεθα εν μέσω των αγαθών ως νεκροί.
\par 11 Πάντες βρυχώμεθα ως άρκτοι και στενάζομεν ως τρυγόνες· κρίσιν προσμένομεν αλλά δεν υπάρχει· σωτηρίαν αλλ' είναι μακράν αφ' ημών.
\par 12 Διότι αι παραβάσεις ημών επληθύνθησαν ενώπιόν σου, και αι αμαρτίαι ημών είναι μάρτυρες καθ' ημών· διότι μεθ' ημών είναι αι παραβάσεις ημών· και τας ανομίας ημών ημείς γνωρίζομεν αυτάς·
\par 13 παρέβημεν και εψεύσθημεν προς τον Κύριον και απεμακρύνθημεν από όπισθεν του Θεού ημών· ελαλήσαμεν άδικα και στασιαστικά· συνελάβομεν και επροφέραμεν εκ της καρδίας λόγους ψεύδους.
\par 14 Και η κρίσις εστράφη οπίσω και η δικαιοσύνη ίσταται μακράν· διότι η αλήθεια έπεσεν εν τη οδώ και η ευθύτης δεν δύναται να εισχωρήση.
\par 15 Ναι, εξέλιπεν η αλήθεια· και ο εκκλίνων από του κακού γίνεται θήραμα. Και είδε Κύριος και δυσηρεστήθη ότι δεν υπήρχε κρίσις·
\par 16 και είδεν ότι δεν υπήρχεν άνθρωπος, και εθαύμασεν ότι δεν υπήρχεν ο μεσιτεύων· όθεν ο βραχίων αυτού ενήργησεν εις αυτόν σωτηρίαν· και η δικαιοσύνη αυτού, αυτή εβάστασεν αυτόν.
\par 17 Και ενεδύθη δικαιοσύνην ως θώρακα και περιέθηκε την περικεφαλαίαν της σωτηρίας επί την κεφαλήν αυτού· και εφόρεσεν ως ιμάτιον τα ενδύματα της εκδικήσεως και ως επένδυμα περιενεδύθη τον ζήλον.
\par 18 Κατά τα έργα αυτών, ούτω θέλει ανταποδώσει, οργήν εις τους εναντίους αυτού, ανταπόδοσιν εις τους εχθρούς αυτού· θέλει κάμει ανταπόδοσιν και εις τας νήσους.
\par 19 Και θέλουσι φοβηθή το όνομα του Κυρίου από δυσμών και την δόξαν αυτού από ανατολών ηλίου· όταν ο εχθρός επέλθη ως ποταμός, το πνεύμα του Κυρίου θέλει υψώσει σημαίαν εναντίον αυτού.
\par 20 Και ο Λυτρωτής θέλει ελθεί εις Σιών και προς τους όσοι εκ του Ιακώβ επιστρέφουσιν από της παραβάσεως, λέγει Κύριος.
\par 21 Παρ' εμού δε αύτη είναι η προς αυτούς διαθήκη μου, λέγει Κύριος· το πνεύμά μου το επί σε και οι λόγοι μου, τους οποίους έθεσα εν τω στόματί σου, δεν θέλουσι λείψει από του στόματός σου ούτε από του στόματος του σπέρματός σου ούτε από του στόματος του σπέρματος του σπέρματός σου, από του νυν και έως αιώνος, λέγει Κύριος.

\chapter{60}

\par 1 Σηκώθητι, φωτίζου· διότι το φως σου ήλθε, και η δόξα του Κυρίου ανέτειλεν επί σε.
\par 2 Διότι ιδού, σκότος θέλει σκεπάσει την γην και ζόφος τα έθνη· επί σε όμως θέλει ανατείλει ο Κύριος και η δόξα αυτού θέλει φανερωθή επί σε.
\par 3 Και τα έθνη θέλουσιν ελθεί εις το φως σου και οι βασιλείς εις την λάμψιν της ανατολής σου.
\par 4 Ύψωσον κύκλω τους οφθαλμούς σου και ιδέ· πάντες ούτοι συναθροίζονται, έρχονται προς σέ· οι υιοί σου θέλουσιν ελθεί μακρόθεν και αι θυγατέρες σου θέλουσι τραφή εις τα πλευρά σου.
\par 5 Τότε θέλεις ιδεί και χαρή, και η καρδία σου θέλει εκπλαγή και πλατυνθή· διότι η αφθονία της θαλάσσης θέλει στραφή προς σέ· αι δυνάμεις των εθνών θέλουσιν ελθεί προς σε.
\par 6 Πλήθος καμήλων θέλει σε σκεπάσει, αι δρομάδες του Μαδιάμ και του Γεφά· πάντες οι από Σεβά βέλουσιν ελθεί· χρυσίον και λίβανον θέλουσι φέρει· και θέλουσιν ευαγγελίζεσθαι τους επαίνους του Κυρίου.
\par 7 Πάντα τα πρόβατα του Κηδάρ θέλουσι συναχθή προς σέ· οι κριοί του Νεβαϊώθ θέλουσιν είσθαι εις χρήσίν σου· θέλουσι προσφερθή επί το θυσιαστήριόν μου ευπρόσδεκτοι, και εγώ θέλω δοξάσει τον οίκον της δόξης μου.
\par 8 Τίνες είναι οι πετώμενοι ως νέφη και ως περιστεραί εις τας θυρίδας αυτών;
\par 9 Αι νήσοι βεβαίως θέλουσι προσμείνει εμέ και εν πρώτοις τα πλοία της Θαρσείς, διά να φέρωσι μακρόθεν τους υιούς σου, το αργύριον αυτών και το χρυσίον αυτών μετ' αυτών, διά το όνομα Κυρίου του Θεού σου και διά τον Άγιον του Ισραήλ, διότι σε εδόξασε.
\par 10 Και οι υιοί των αλλογενών θέλουσιν ανοικοδομήσει τα τείχη σου, και οι βασιλείς αυτών θέλουσι σε υπηρετήσει· διότι εν τη οργή μου σε επάταξα, πλην διά την εύνοιάν μου σε ηλέησα.
\par 11 Και αι πύλαι σου θέλουσιν είσθαι πάντοτε ανοικταί· δεν θέλουσι κλεισθή ημέραν και νύκτα, διά να εισάγωσιν εις σε τας δυνάμεις των εθνών και να εισφέρωνται οι βασιλείς αυτών.
\par 12 Διότι το έθνος και η βασιλεία, τα οποία δεν ήθελον σε δουλεύσει, θέλουσιν αφανισθή· ναι, τα έθνη εκείνα θέλουσιν ολοκλήρως ερημωθή.
\par 13 Η δόξα του Λιβάνου θέλει η ελθεί εις σε, η έλατος, η πεύκη και ο πύξος ομού, διά να στολίσωσι τον τόπον του αγιαστηρίου μου· και θέλω δοξάσει τον τόπον των ποδών μου.
\par 14 Και τα τέκνα των λυπησάντων σε θέλουσιν ελθεί υποκλίνοντα προς σέ· και πάντες οι καταφρονήσαντές σε θέλουσι προσκυνήσει τα ίχνη των ποδών σου· και θέλουσι σε ονομάζει, Η πόλις του Κυρίου, Η Σιών του Αγίου του Ισραήλ.
\par 15 Αντί του ότι εγκατελείφθης και εμισήθης, ώστε ουδείς διέβαινε διά μέσου σου, θέλω σε καταστήσει αιώνιον αγαλλίαμα, ευφροσύνην εις γενεάς γενεών.
\par 16 Και θέλεις θηλάσει το γάλα των εθνών και θέλεις θηλάσει τους μαστούς των βασιλέων· και θέλεις γνωρίσει ότι εγώ ο Κύριος είμαι ο Σωτήρ σου και ο Λυτρωτής σου, ο Ισχυρός του Ιακώβ.
\par 17 Αντί χαλκού θέλω φέρει χρυσίον και αντί σιδήρου θέλω φέρει αργύριον και αντί ξύλου χαλκόν και αντί λίθων σίδηρον· και θέλω καταστήσει τους αρχηγούς σου ειρήνην και τους επιστάτας σου δικαιοσύνην.
\par 18 Δεν θέλει πλέον ακούεσθαι βία εν τη γη σου, ερήμωσις και καταστροφή εν τοις ορίοις σου· αλλά θέλεις ονομάζει τα τείχη σου Σωτηρίαν και τας πύλας σου Αίνεσιν.
\par 19 Δεν θέλει είσθαι πλέον εν σοι ο ήλιος φως της ημέρας, ουδέ η σελήνη διά της λάμψεως αυτής θέλει σε φωτίζει· αλλ' ο Κύριος θέλει είσθαι εις σε φως αιώνιον και ο Θεός σου η δόξα σου.
\par 20 Ο ήλιός σου δεν θέλει δύει πλέον ουδέ θέλει λείψει η σελήνη σου· διότι ο Κύριος θέλει είσθαι το αιώνιόν σου φως, και αι ημέραι του πένθους σου θέλουσι τελειωθή.
\par 21 Και ο λαός σου θέλουσιν είσθαι πάντες δίκαιοι· θέλουσι κληρονομήσει την γην διαπαντός, ο κλάδος του φυτεύματός μου, το έργον των χειρών μου, διά να δοξάζωμαι.
\par 22 Το ελάχιστον θέλει γείνει χίλια· και το ολιγοστόν ισχυρόν έθνος· εγώ ο Κύριος θέλω επιταχύνει τούτο κατά τον καιρόν αυτού.

\chapter{61}

\par 1 Πνεύμα Κυρίου του Θεού είναι επ' εμέ· διότι ο Κύριος με έχρισε διά να ευαγγελίζωμαι εις τους πτωχούς· με απέστειλε διά να ιατρεύσω τους συντετριμμένους την καρδίαν, να κηρύξω ελευθερίαν εις τους αιχμαλώτους και άνοιξιν δεσμωτηρίου εις τους δεσμίους·
\par 2 διά να κηρύξω ενιαυτόν ευπρόσδεκτον του Κυρίου και ημέραν εκδικήσεως του Θεού ημών· διά να παρηγορήσω πάντας τους πενθούντας·
\par 3 διά να θέσω εις τους πενθούντας εν Σιών, να δώσω εις αυτούς ώραιότητα αντί της στάκτης, έλαιον ευφροσύνης αντί του πένθους, στολήν αινέσεως αντί του πνεύματος της ακηδίας· διά να ονομάζωνται δένδρα δικαιοσύνης, φύτευμα του Κυρίου, εις δόξαν αυτού.
\par 4 Και θέλουσιν ανοικοδομήσει τας παλαιάς ερημώσεις, θέλουσιν ανεγείρει τα αρχαία ερείπια, και θέλουσιν ανακαινίσει τας ερήμους πόλεις, τας ηρημωμένας από γενεάς γενεών.
\par 5 Και αλλογενείς θέλουσιν ίστασθαι και βόσκει τα ποίμνιά σας, και οι υιοί των αλλογενών θέλουσιν είσθαι οι γεωργοί σας και οι αμπελουργοί σας.
\par 6 Σεις δε ιερείς του Κυρίου θέλετε ονομάζεσθαι· λειτουργούς του Θεού ημών θέλουσι σας λέγει· θέλετε τρώγει τα αγαθά των εθνών και εις την δόξαν αυτών θέλετε καυχάσθαι.
\par 7 Αντί της αισχύνης σας θέλετε έχει διπλάσια· και αντί της εντροπής θέλουσιν έχει αγαλλίασιν εν τη κληρονομία αυτών· όθεν εν τη γη αυτών θέλουσι κληρονομήσει το διπλούν· αιώνιος ευφροσύνη θέλει είσθαι εις αυτούς.
\par 8 Διότι εγώ είμαι ο Κύριος, ο αγαπών δικαιοσύνην, ο μισών αρπαγήν και αδικίαν· και θέλω ανταποδώσει το έργον αυτών πιστά και θέλω κάμει προς αυτούς διαθήκην αιώνιον.
\par 9 Και το σπέρμα αυτών θέλει φημισθή μεταξύ των εθνών και οι έκγονοι αυτών μεταξύ των λαών· πας ο βλέπων αυτούς θέλει γνωρίζει αυτούς, ότι είναι το σπέρμα, το οποίον ο Κύριος ευλόγησε.
\par 10 Θέλω ευφρανθή τα μέγιστα επί τον Κύριον· η ψυχή μου θέλει αγαλλιασθή εις τον Θεόν μου· διότι με ενέδυσεν ιμάτιον σωτηρίας, με εφόρεσεν επένδυμα δικαιοσύνης, ως νυμφίον ευπρεπισμένον με μίτραν και ως νύμφην κεκοσμημένην με τα πολύτιμα αυτής καλλωπίσματα.
\par 11 Διότι καθώς η γη αναδίδει το βλάστημα αυτής και καθώς ο κήπος εκφύει τα σπειρόμενα εν αυτώ ούτω Κύριος ο Θεός θέλει κάμει την δικαιοσύνην και την αίνεσιν να βλαστήσωσιν ενώπιον πάντων των εθνών.

\chapter{62}

\par 1 Διά την Σιών δεν θέλω σιωπήσει και διά την Ιερουσαλήμ δεν θέλω ησυχάσει, εωσού η δικαιοσύνη αυτής εξέλθη ως λάμψις και η σωτηρία αυτής ως λαμπάς καιομένη.
\par 2 Και θέλουσιν ιδεί τα έθνη την δικαιοσύνην σου και πάντες οι βασιλείς την δόξαν σου· και θέλεις ονομασθή με νέον όνομα, το οποίον του Κυρίου το στόμα θέλει ονομάσει.
\par 3 Και θέλεις είσθαι στέφανος δόξης εν χειρί Κυρίου και διάδημα βασιλικόν εν τη παλάμη του Θεού σου.
\par 4 Δεν θέλεις πλέον ονομασθή, Εγκαταλελειμμένη· ουδέ η γη σου θέλει πλέον ονομασθή, Ηρημωμένη· αλλά θέλεις ονομασθή, Η ευδοκία μου εν αυτή· και η γη σου, Η νενυμφευμένη· διότι ο Κύριος ηυδόκησεν επί σε, και η γη σου θέλει είσθαι νενυμφευμένη.
\par 5 Διότι καθώς ο νέος νυμφεύεται με παρθένον, ούτως οι υιοί σου θέλουσι συνοικεί μετά σού· και καθώς ο νυμφίος ευφραίνεται εις την νύμφην, ούτως ο Θεός σου θέλει ευφρανθή εις σε.
\par 6 Επί των τειχών σου, Ιερουσαλήμ, κατέστησα φύλακας, οίτινες ποτέ δεν θέλουσι σιωπά ούτε ημέραν ούτε νύκτα· όσοι ανακαλείτε τον Κύριον, μη φυλάττετε σιωπήν.
\par 7 Και μη δίδετε εις αυτόν ανάπαυσιν, εωσού συστήση και εωσού κάμη την Ιερουσαλήμ αίνεσιν επί της γης.
\par 8 Ο Κύριος ώμοσεν επί την δεξιάν αυτού και επί τον βραχίονα της δυνάμεως αυτού, δεν θέλω δώσει πλέον τον σίτόν σου τροφήν εις τους εχθρούς σου· και οι υιοί του αλλογενούς δεν θέλουσι πίνει τον οίνόν σου, διά τον οποίον εμόχθησας·
\par 9 αλλ' οι θερίζοντες θέλουσι τρώγει αυτόν και αινεί τον Κύριον· και οι τρυγώντες θέλουσι πίνει αυτόν εν ταις αυλαίς της αγιότητός μου.
\par 10 Περάσατε, περάσατε διά των πυλών· ετοιμάσατε την οδόν του λαού· επισκευάσατε, επισκευάσατε την οδόν· εκρίψατε τους λίθους· υψώσατε σημαίαν προς τους λαούς.
\par 11 Ιδού, ο Κύριος διεκήρυξεν έως των άκρων της γης, Είπατε προς την θυγατέρα της Σιών, Ιδού, ο Σωτήρ σου έρχεται· ιδού, ο μισθός αυτού είναι μετ' αυτού και το έργον αυτού ενώπιον αυτού.
\par 12 Και θέλουσιν ονομάσει αυτούς, Ο Άγιος λαός, Ο λελυτρωμένος του Κυρίου· και συ θέλεις ονομασθή, Επιζητουμένη, πόλις ουκ εγκαταλελειμμένη.

\chapter{63}

\par 1 Τις ούτος, ο ερχόμενος εξ Εδώμ, με ιμάτια ερυθρά εκ Βοσόρρας; ούτος ο ένδοξος εις την στολήν αυτού, ο περιπατών εν τη μεγαλειότητι της δυνάμεως αυτού; Εγώ, ο λαλών εν δικαιοσύνη, ο ισχυρός εις το σώζειν.
\par 2 Διά τι είναι ερυθρά η στολή σου και τα ιμάτιά σου όμοια ανθρώπου πατούντος εν ληνώ;
\par 3 Επάτησα μόνος τον ληνόν, και ουδείς εκ των λαών ήτο μετ' εμού· και κατεπάτησα αυτούς εν τω θυμώ μου και κατελάκτισα αυτούς εν τη οργή μου· και το αίμα αυτών ερραντίσθη επί τα ιμάτιά μου και εμόλυνα όλην μου την στολήν.
\par 4 Διότι η ημέρα της εκδικήσεως ήτο εν τη καρδία μου, και έφθασεν ο ενιαυτός των λελυτρωμένων μου.
\par 5 Και περιέβλεψα και δεν υπήρχεν ο βοηθών· και εθαύμασα ότι δεν υπήρχεν ο υποστηρίζων· όθεν ο βραχίων μου ενήργησε σωτηρίαν εις εμέ· και ο θυμός μου, αυτός με υπεστήριξε.
\par 6 Και κατεπάτησα τους λαούς εν τη οργή μου και εμέθυσα αυτούς εκ του θυμού μου και κατεβίβασα εις την γην το αίμα αυτών.
\par 7 Θέλω αναφέρει τους οικτιρμούς του Κυρίου, τας αινέσεις του Κυρίου, κατά πάντα όσα ο Κύριος έκαμεν εις ημάς, και την μεγάλην αγαθότητα προς τον οίκον Ισραήλ, την οποίαν έδειξε προς αυτούς κατά τους οικτιρμούς αυτού και κατά το πλήθος του ελέους αυτού.
\par 8 Διότι είπε, Βεβαίως λαός μου είναι αυτοί, τέκνα τα οποία δεν θέλουσι ψευσθή· και υπήρξεν ο Σωτήρ αυτών.
\par 9 Κατά πάσας τας θλίψεις αυτών εθλίβετο, και ο άγγελος της παρουσίας αυτού έσωσεν αυτούς· εν τη αγάπη αυτού και εν τη ευσπλαγχνία αυτού αυτός ελύτρωσεν αυτούς· και εσήκωσεν αυτούς και εβάστασεν αυτούς πάσας τας ημέρας του αιώνος.
\par 10 Αυτοί όμως ηπείθησαν και ελύπησαν το άγιον πνεύμα αυτού· διά τούτο εστράφη ώστε να γείνη εχθρός αυτών, αυτός επολέμησεν αυτούς.
\par 11 Τότε ενεθυμήθη τας αρχαίας ημέρας, τον Μωϋσήν, τον λαόν αυτού, λέγων, Που είναι ο αναβιβάσας αυτούς από της θαλάσσης μετά του ποιμένος του ποιμνίου αυτού; που ο θέσας το άγιον αυτού πνεύμα εν τω μέσω αυτών;
\par 12 Ο οδηγήσας αυτούς διά της δεξιάς του Μωϋσέως με τον ένδοξον βραχίονα αυτού, ο διασχίσας τα ύδατα έμπροσθεν αυτών, διά να κάμη εις εαυτόν όνομα αιώνιον;
\par 13 Ο οδηγήσας αυτούς διά της αβύσσου, ως ίππον διά της ερήμου, χωρίς να προσκόψωσι;
\par 14 Το πνεύμα του Κυρίου ανέπαυσεν αυτούς ως κτήνος καταβαίνον εις την κοιλάδα· ούτως ώδήγησας τον λαόν σου, διά να κάμης εις σεαυτόν ένδοξον όνομα.
\par 15 Επίβλεψον εξ ουρανού και ιδέ εκ της κατοικίας της αγιότητός σου και της δόξης σου· που ο ζήλος σου και η δύναμίς σου, το πλήθος του ελέους σου και των οικτιρμών σου; απεκλείσθησαν εις εμέ;
\par 16 Συ βεβαίως είσαι ο Πατήρ ημών, αν και ο Αβραάμ δεν εξεύρη ημάς και ο Ισραήλ δεν γνωρίζη ημάς· συ, Κύριε, είσαι ο Πατήρ ημών· Λυτρωτής ημών είναι το όνομά σου απ' αιώνος.
\par 17 Διά τι Κύριε, αφήκας ημάς να αποπλανώμεθα από των οδών σου και να σκληρύνωμεν την καρδίαν ημών, ώστε να μη σε φοβώμεθα; επίστρεψον ένεκεν των δούλων σου, των φυλών της κληρονομίας σου.
\par 18 Ως πράγμα ελάχιστον κατεκυρίευσαν τον άγιόν σου λαόν· οι εναντίοι ημών κατεπάτησαν το αγιαστήριόν σου.
\par 19 Κατεστάθημεν ως εκείνοι, επί τους οποίους δεν εδέσποσας ποτέ ουδέ επεκλήθη το όνομά σου επ' αυτούς.

\chapter{64}

\par 1 Είθε να έσχιζες τους ουρανούς, να κατέβαινες, να διελύοντο τα όρη εν τη παρουσία σου,
\par 2 ως πυρ καίον θάμνους, ως πυρ κάμνον το ύδωρ να κοχλάζη, διά να γείνη το όνομά σου γνωστόν εις τους εναντίους σου, να λάβη τρόμος τα έθνη εν τη παρουσία σου.
\par 3 Ότε έκαμες τρομερά πράγματα, οποία δεν επροσμέναμεν, κατέβης, και τα όρη διελύθησαν εν τη παρουσία σου.
\par 4 Διότι εκ του αιώνος δεν έμαθον οι άνθρωποι, τα ώτα αυτών δεν ήκουσαν, οι οφθαλμοί αυτών δεν είδον Θεόν εκτός σου, όστις να έκαμε τοιαύτα εις τους επικαλουμένους αυτόν.
\par 5 Έρχεσαι εις συνάντησιν του ευφραινομένου και εργαζομένου δικαιοσύνην, των ενθυμουμένων σε εν ταις οδοίς σου· ιδού, συ ωργίσθης, διότι ημείς ημαρτήσαμεν· εάν διεμένομεν εν αυταίς, ηθέλομεν σωθή;
\par 6 Πάντες τωόντι εγείναμεν ως ακάθαρτον πράγμα, και πάσα η δικαιοσύνη ημών είναι ως ρυπαρόν ιμάτιον· διά τούτο επέσαμεν πάντες ως το φύλλον, και αι ανομίαι ημών αφήρπασαν ημάς ως ο άνεμος.
\par 7 Και δεν υπάρχει ο επικαλούμενος το όνομά σου, ο εγειρόμενος διά να πιασθή από σού· διότι έκρυψας το πρόσωπόν σου αφ' ημών και ηφάνισας ημάς διά της χειρός των ανομιών ημών.
\par 8 Αλλά τώρα, Κύριε, συ είσαι ο Πατήρ ημών· ημείς είμεθα ο πηλός και συ ο Πλάστης ημών· και πάντες είμεθα το έργον της χειρός σου.
\par 9 Μη οργίζου σφόδρα, Κύριε, μηδέ ενθυμού πάντοτε την ανομίαν· και τώρα επίβλεψον, δεόμεθα· λαός σου είμεθα πάντες.
\par 10 Αι άγιαι πόλεις σου έγειναν έρημοι, η Σιών έγεινεν έρημος, η Ιερουσαλήμ ηρημωμένη.
\par 11 Ο άγιος ημών και ο ώραίος ημών οίκος, εν ω οι πατέρες ημών σε εδοξολόγουν, κατεκάη εν πυρί· και πάντα τα εις ημάς αγαπητά ηφανίσθησαν.
\par 12 Θέλεις, Κύριε, κρατήσει σεαυτόν εν τούτοις; θέλεις σιωπήσει και θέλεις θλίψει ημάς έως σφόδρα;

\chapter{65}

\par 1 Εζητήθην παρά των μη ερωτώντων περί εμού· ευρέθην παρά των ζητούντων με· είπα, Ιδού, εγώ, ιδού, εγώ, προς έθνος μη καλούμενον με το όνομά μου.
\par 2 Εξήπλωσα τας χείρας μου όλην την ημέραν προς λαόν απειθή, περιπατούντα εν οδώ ουχί καλή, οπίσω των διαβουλίων αυτών,
\par 3 λαόν παροξύνοντά με πάντοτε κατά πρόσωπόν μου, θυσιάζοντα εν κήποις και θυμιάζοντα επί πλίνθων,
\par 4 μένοντα εν τοις μνήμασι και διανυκτερεύοντα εν αποκρύφοις, τρώγοντα χοίρειον κρέας και εν τοις αγγείοις αυτού έχοντα ζωμόν ακαθάρτων πραγμάτων,
\par 5 λέγοντα, Μακράν απ' εμού, μη με εγγίσης, διότι είμαι αγιώτερός σου. Ούτοι είναι καπνός εις τους μυκτήράς μου, πυρ καιόμενον όλην την ημέραν.
\par 6 Ιδού, γεγραμμένον είναι ενώπιόν μου, δεν θέλω σιωπήσει αλλά θέλω ανταποδώσει, ναι, θέλω ανταποδώσει εις τους κόλπους αυτών
\par 7 τας ανομίας σας και τας ανομίας των πατέρων σας ομού, λέγει Κύριος, οίτινες εθυμίασαν επί των ορέων και με εβλασφήμησαν επί των λόφων· διά τούτο θέλω αντιπληρώσει εις τους κόλπους αυτών τα απ' αρχής έργα αυτών.
\par 8 Ούτω λέγει Κύριος· Καθώς όταν ευρίσκηται γλεύκος εν τη σταφυλή, λέγουσι, Μη φθείρης αυτό, διότι είναι ευλογία εν αυτώ· ούτω θέλω κάμει ένεκεν των δούλων μου, διά να μη εξολοθρεύσω πάντας.
\par 9 Και θέλω εξάξει σπέρμα εξ Ιακώβ και κληρονόμον των ορέων μου εξ Ιούδα· και οι εκλεκτοί μου θέλουσι κληρονομήσει αυτά και οι δούλοί μου θέλουσι κατοικήσει εκεί.
\par 10 Και ο Σαρών θέλει είσθαι μάνδρα των ποιμνίων και η κοιλάς του Αχώρ τόπος εις ανάπαυσιν των βουκολίων, διά τον λαόν μου τον εκζητούντά με.
\par 11 Εσάς όμως, τους εγκαταλείποντας τον Κύριον, τους λησμονούντας το άγιόν μου όρος, τους ετοιμάζοντας τράπεζαν εις τον Γάδην και τους κάμνοντας σπονδήν εις τον Μένι,
\par 12 θέλω σας αριθμήσει διά την μάχαιραν και πάντες θέλετε κύψει εις την σφαγήν· διότι εκάλουν και δεν απεκρίνεσθε· ελάλουν και δεν ηκούετε· αλλ' επράττετε το κακόν ενώπιόν μου και εξελέγετε το μη αρεστόν εις εμέ.
\par 13 Όθεν ούτω λέγει Κύριος ο Θεός· Ιδού, οι δούλοί μου θέλουσι φάγει, σεις δε θέλετε πεινάσει· ιδού, οι δούλοί μου θέλουσι πίει, σεις δε θέλετε διψήσει· ιδού, οι δούλοί μου θέλουσιν ευφρανθή, σεις δε θέλετε αισχυνθή·
\par 14 ιδού, οι δούλοί μου θέλουσιν αλαλάζει εν ευθυμία, σεις δε θέλετε βοά εν πόνω καρδίας και ολολύζει υπό καταθλίψεως πνεύματος.
\par 15 Και θέλετε αφήσει το όνομά σας εις τους εκλεκτούς μου διά κατάραν· διότι Κύριος ο Θεός θέλει σε θανατώσει και με άλλο όνομα θέλει ονομάσει τους δούλους αυτού,
\par 16 διά να μακαρίζη εαυτόν εις τον Θεόν της αληθείας ο μακαρίζων εαυτόν επί της γής· και να ομνύη εις τον Θεόν της αληθείας ο ομνύων επί της γής· διότι αι πρότεραι θλίψεις ελησμονήθησαν και διότι εκρύφθησαν από των οφθαλμών μου.
\par 17 Επειδή ιδού, νέους ουρανούς κτίζω και νέαν γήν· και δεν θέλει είσθαι μνήμη των προτέρων ουδέ θέλουσιν ελθεί εις τον νούν.
\par 18 Αλλ' ευφραίνεσθε και χαίρετε πάντοτε εις εκείνο το οποίον κτίζω· διότι, ιδού, κτίζω την Ιερουσαλήμ αγαλλίαμα και τον λαόν αυτής ευφροσύνην.
\par 19 Και θέλω αγάλλεσθαι εις την Ιερουσαλήμ και ευφραίνεσθαι εις τον λαόν μου· και δεν θέλει ακουσθή πλέον εν αυτή φωνή κλαυθμού και φωνή κραυγής.
\par 20 Δεν θέλει είσθαι πλέον εκεί βρέφος ολιγοήμερον και γέρων όστις δεν επλήρωσε τας ημέρας αυτού· διότι το παιδίον θέλει αποθνήσκει εκατόν ετών, ο δε εκατόν ετών αμαρτωλός θέλει είσθαι επικατάρατος.
\par 21 Και θέλουσιν οικοδομήσει οικίας και κατοικήσει, και θέλουσι φυτεύσει αμπελώνας και φάγει τον καρπόν αυτών.
\par 22 δεν θέλουσι κτίσει αυτοί και άλλος να κατοικήση· δεν θέλουσι φυτεύσει αυτοί και άλλος να φάγη· διότι αι ημέραι του λαού μου είναι ως αι ημέραι του δένδρου και οι εκλεκτοί μου θέλουσι παλαιώσει το έργον των χειρών αυτών.
\par 23 Δεν θέλουσι κοπιάζει εις μάτην ουδέ θέλουσι τεκνοποιεί διά καταστροφήν· διότι είναι σπέρμα των ευλογημένων του Κυρίου και οι έκγονοι αυτών μετ' αυτών.
\par 24 Και πριν αυτοί κράξωσιν, εγώ θέλω αποκρίνεσθαι· και ενώ αυτοί λαλούσιν, εγώ θέλω ακούει.
\par 25 Ο λύκος και το αρνίον θέλουσι βόσκεσθαι ομού, και ο λέων θέλει τρώγει άχυρον ως ο βούς· άρτος δε του όφεως θέλει είσθαι το χώμα· εν όλω τω αγίω μου όρει δεν θέλουσι κάμνει ζημίαν ουδέ φθοράν, λέγει Κύριος.

\chapter{66}

\par 1 Ούτω λέγει Κύριος· Ο ουρανός είναι θρόνος μου και η γη υποπόδιον των ποδών μου· ποίος είναι ο οίκος, τον οποίον ηθέλετε οικοδομήσει δι' εμέ; και ποίος είναι ο τόπος της αναπαύσεώς μου;
\par 2 Διότι η χειρ μου έκαμε πάντα ταύτα και έγειναν πάντα ταύτα, λέγει Κύριος· εις τίνα λοιπόν θέλω επιβλέψει; εις τον πτωχόν και συντετριμμένον το πνεύμα και τρέμοντα τον λόγον μου.
\par 3 Όστις δε σφάζει βουν, είναι ως ο φονεύων άνθρωπον· όστις θυσιάζει αρνίον, ως ο κόπτων κυνός λαιμόν· όστις προσφέρει προσφοράν εξ αλφίτων, ως προσφέρων αίμα χοίρειον· όστις θυσιάζει, ως ο ευλογών είδωλον. Ναι, αυτοί εξέλεξαν τας οδούς αυτών, και η ψυχή αυτών ηδύνεται εις τα βδελύγματα αυτών.
\par 4 Και εγώ λοιπόν θέλω εκλέξει τα εις αυτούς ολέθρια και θέλω φέρει επ' αυτούς όσα φοβούνται· διότι εκάλουν και ουδείς απεκρίνετο· ελάλουν και δεν ήκουον· αλλ' έπραττον το κακόν ενώπιόν μου και εξέλεγον το μη αρεστόν εις εμέ.
\par 5 Ακούσατε τον λόγον του Κυρίου, σεις οι τρέμοντες τον λόγον αυτού· οι αδελφοί σας, οίτινες σας μισούσι και σας αποβάλλουσιν ένεκεν του ονόματός μου, είπαν, Ας δοξασθή ο Κύριος· πλην αυτός θέλει φανή εις χαράν σας, εκείνοι δε θέλουσι καταισχυνθή.
\par 6 Φωνή κραυγής έρχεται εκ της πόλεως, φωνή εκ του ναού, φωνή του Κυρίου, όστις κάμνει ανταπόδοσιν εις τους εχθρούς αυτού.
\par 7 Πριν κοιλοπονήση, εγέννησε· πριν έλθωσιν οι πόνοι αυτής, ηλευθερώθη και εγέννησεν αρσενικόν.
\par 8 Τις ήκουσε τοιούτον πράγμα; τις είδε τοιαύτα; ήθελε γεννήσει η γη εν μιά ημέρα; ή έθνος ήθελε γεννηθή ενταυτώ αλλ' η Σιών άμα εκοιλοπόνησεν, εγέννησε τα τέκνα αυτής.
\par 9 Εγώ, ο φέρων εις την γένναν, δεν ήθελον κάμει να γεννήση; λέγει Κύριος· εγώ, ο κάμνων να γεννώσιν, ήθελον κλείσει την μήτραν; λέγει ο Θεός σου.
\par 10 Ευφράνθητε μετά της Ιερουσαλήμ και αγάλλεσθε μετ' αυτής, πάντες οι αγαπώντες αυτήν· χαρήτε χαράν μετ' αυτής, πάντες οι πενθούντες δι' αυτήν·
\par 11 διά να θηλάσητε και να χορτασθήτε από των μαστών των παρηγοριών αυτής· διά να εκθηλάσητε και να εντρυφήσητε εις την αφθονίαν της δόξης αυτής.
\par 12 διότι ούτω λέγει Κύριος· Ιδού, εις αυτήν θέλω στρέψει την ειρήνην ως ποταμόν, και την δόξαν των εθνών ως χείμαρρον πλημμυρούντα· τότε θέλετε θηλάσει, θέλετε βασταχθή επί των πλευρών και κολακευθή επί των γονάτων αυτής.
\par 13 Ως παιδίον, το οποίον παρηγορεί η μήτηρ αυτού, ούτως εγώ θέλω σας παρηγορήσει· και θέλετε παρηγορηθή εν τη Ιερουσαλήμ.
\par 14 Και θέλετε ιδεί, και η καρδία σας θέλει ευφρανθή και τα οστά σας θέλουσιν ανθήσει ως χόρτος· και η χειρ του Κυρίου θέλει γνωρισθή προς τους δούλους αυτού, η δε οργή προς τους εχθρούς αυτού.
\par 15 Διότι, ιδού, ο Κύριος θέλει ελθεί εν πυρί, και αι άμαξαι αυτού θέλουσιν είσθαι ως ανεμοστρόβιλος, διά να αποδώση την οργήν αυτού με ορμήν και την επιτίμησιν αυτού με φλόγας πυρός.
\par 16 Διότι εν πυρί Κυρίου και εν τη μαχαίρα αυτού θέλει κριθή πάσα σαρξ, και οι πεφονευμένοι του Κυρίου θέλουσιν είσθαι πολλοί.
\par 17 Οι αγιαζόμενοι και καθαριζόμενοι εν τοις κήποις ο εις κατόπιν του άλλου αναφανδόν, τρώγοντες χοίρειον κρέας και τα βδελύγματα και τον ποντικόν, ούτοι θέλουσι καταναλωθή ομού, λέγει Κύριος.
\par 18 Διότι εγώ εξεύρω τα έργα αυτών και τους διαλογισμούς αυτών· και έρχομαι διά να συνάξω πάντα τα έθνη και τας γλώσσας· και θέλουσιν ελθεί και ιδεί την δόξαν μου.
\par 19 Και θέλω στήσει σημείον μεταξύ αυτών· και τους σεσωσμένους εξ αυτών θέλω εξαποστείλει εις τα έθνη, εις Θαρσείς, Φούλ και Λούδ, οίτινες σύρουσι τόξον, εις Θουβάλ και Ιαυάν, εις τας νήσους τας μακράν, οίτινες δεν ήκουσαν την φήμην μου ουδέ είδον την δόξαν μου· και θέλουσι κηρύξει την δόξάν μου μεταξύ των εθνών.
\par 20 Και θέλουσι φέρει πάντας τους αδελφούς σας εκ πάντων των εθνών προσφοράν εις τον Κύριον, επί ίππων και επί αμαξών και επί φορείων και επί ημιόνων και επί ταχυδρόμων ζώων, προς το άγιόν μου όρος, την Ιερουσαλήμ, λέγει Κύριος, καθώς τα τέκνα του Ισραήλ φέρουσι την εξ αλφίτων προσφοράν εν καθαρώ αγγείω προς τον οίκον του Κυρίου.
\par 21 Και προσέτι θέλω λάβει εξ αυτών ιερείς και Λευΐτας, λέγει Κύριος.
\par 22 Διότι ως οι νέοι ουρανοί και η νέα γη, τα οποία εγώ θέλω κάμει, θέλουσι διαμένει ενώπιόν μου, λέγει Κύριος, ούτω θέλει διαμένει το σπέρμα σας και το όνομά σας.
\par 23 Και από νέας σελήνης έως άλλης και από σαββάτου έως άλλου θέλει έρχεσθαι πάσα σαρξ διά να προσκυνή ενώπιόν μου, λέγει Κύριος.
\par 24 Και θέλουσιν εξέλθει και ιδεί τα κώλα των ανθρώπων, οίτινες εστάθησαν παραβάται εναντίον μου· διότι ο σκώληξ αυτών δεν θέλει τελευτήσει και το πυρ αυτών δεν θέλει σβεσθή· και θέλουσιν είσθαι βδέλυγμα εις πάσαν σάρκα.


\end{document}