\begin{document}

\title{Zephaniah}


\chapter{1}

\par Ο λόγος του Κυρίου, ο γενόμενος προς Σοφονίαν τον υιόν του Χουσεί, υιού του Γεδαλίου, υιού του Αμαρίου υιού του Ιζκίου, εν ταις ημέραις Ιωσίου, υιού του Αμών βασιλέως του Ιούδα.
\par 2 Θέλω αφανίσει παντελώς τα πάντα από προσώπου της γης, λέγει Κύριος.
\par 3 Θέλω αφανίσει άνθρωπον και κτήνος θέλω αφανίσει τα πετεινά του ουρανού και τους ιχθύας της θαλάσσης και τα προσκόμματα μετά των ασεβών και θέλω εξολοθρεύσει τον άνθρωπον από προσώπου της γης, λέγει Κύριος.
\par 4 Και θέλω εκτείνει την χείρα μου επί τον Ιούδαν και επί πάντας τους κατοίκους της Ιερουσαλήμ, και θέλω εξολοθρεύσει το υπόλοιπον του Βάαλ από του τόπου τούτου και το όνομα των ειδωλοθυτών μετά των ιερέων,
\par 5 και τους προσκυνούντας επί των δωμάτων την στρατιάν του ουρανού και τους προσκυνούντας και ομνύοντας εις τον Κύριον και τους ομνύοντας εις τον Μαλχόμ,
\par 6 και τους εκκλίνοντας από όπισθεν του Κυρίου και τους μη ζητούντας τον Κύριον μηδέ εξερευνώντας αυτόν.
\par 7 Σιώπα ενώπιον Κυρίου του Θεού, διότι εγγύς είναι η ημέρα του Κυρίου· διότι ο Κύριος ητοίμασε θυσίαν, διώρισε τους κεκλημένους αυτού.
\par 8 Και εν τη ημέρα της θυσίας του Κυρίου θέλω εκδικηθή τους άρχοντας και τα τέκνα του βασιλέως και πάντας τους ενδεδυμένους ενδύματα αλλότρια.
\par 9 Εν εκείνη τη ημέρα θέλω εκδικηθή και πάντας τους υπερπηδώντας τα κατώφλια, τους γεμίζοντας τους οίκους των κυρίων αυτών εξ αρπαγής και δόλου.
\par 10 Και εν τη ημέρα εκείνη, λέγει Κύριος, θέλει είσθαι θόρυβος κραυγής από της ιχθυϊκής πύλης και ολολυγμός από της πύλης της δευτέρας και συντριμμός μέγας από των λόφων.
\par 11 Ολολύξατε, οι κάτοικοι της Μακτές, διότι εξωλοθρεύθη πας ο λαός ο εμπορικός· κατεκόπησαν πάντες οι φέροντες αργύριον.
\par 12 Και εν τω καιρώ εκείνω θέλω εξερευνήσει την Ιερουσαλήμ με λύχνους και εκδικηθή προς άνδρας τους αναπαυομένους επί την τρυγίαν αυτών, τους λέγοντας εν τη καρδία αυτών, Ο Κύριος δεν θέλει αγαθοποιήσει ουδέ θέλει κακοποιήσει.
\par 13 Διά τούτο τα αγαθά αυτών θέλουσιν είσθαι εις διαρπαγήν και οι οίκοι αυτών εις αφανισμόν, και θέλουσιν οικοδομήσει οικίας και δεν θέλουσι κατοικήσει, και θέλουσι φυτεύσει αμπελώνας και δεν θέλουσι πίει τον οίνον αυτών.
\par 14 Εγγύς είναι η ημέρα του Κυρίου η μεγάλη, εγγύς, και σπεύδει σφόδρα· φωνή της ημέρας του Κυρίου· πικρώς θέλει φωνάξει εκεί ο ισχυρός.
\par 15 Ημέρα οργής η ημέρα εκείνη, ημέρα θλίψεως και στενοχωρίας, ημέρα ερημώσεως και αφανισμού, ημέρα σκότους και γνόφου, ημέρα νεφέλης και ομίχλης,
\par 16 ημέρα σάλπιγγος και αλαλαγμού κατά των οχυρών πόλεων και κατά των υψηλών πύργων.
\par 17 Και θέλω καταθλίψει τους ανθρώπους και θέλουσι περιπατεί ως τυφλοί, διότι ημάρτησαν εις τον Κύριον· και το αίμα αυτών θέλει διαχυθή ως κόνις και αι σάρκες αυτών ως κόπρος.
\par 18 Αλλ' ουδέ το αργύριον αυτών ουδέ το χρυσίον αυτών θέλει δυνηθή να λυτρώση αυτούς εν τη ημέρα της οργής του Κυρίου, και πάσα η γη θέλει καταναλωθή υπό του πυρός του ζήλου αυτού· διότι θέλει κάμει συντέλειαν, μάλιστα ταχείαν, επί πάντας τους κατοικούντας την γην.

\chapter{2}

\par Συνάχθητε και συναθροίσθητε, το έθνος το μη επιθυμητόν,
\par 2 πριν το ψήφισμα γεννήση το αποτέλεσμα αυτού και η ημέρα παρέλθη ως χνούς, πριν επέλθη εφ' υμάς η έξαψις του Κυρίου, πριν επέλθη εφ' υμάς η ημέρα του θυμού του Κυρίου.
\par 3 Ζητείτε τον Κύριον, πάντες οι πραείς της γης, οι εκτελέσαντες τας κρίσεις αυτού· ζητείτε δικαιοσύνην, ζητείτε πραότητα, ίσως σκεπασθήτε εν τη ημέρα της οργής του Κυρίου.
\par 4 Διότι η Γάζα θέλει εγκαταλειφθή και η Ασκάλων θέλει ερημωθή· θέλουσιν εκδιώξει την Άζωτον εν καιρώ μεσημβρίας και η Ακκαρών θέλει εκριζωθή.
\par 5 Ουαί εις τους κατοίκους των παραλίων της θαλάσσης, εις το έθνος των Χερεθαίων· ο λόγος του Κυρίου είναι εναντίον σας, Χαναάν, γη των Φιλισταίων, και θέλω σε αφανίσει, ώστε να μη υπάρχη ο κατοικών.
\par 6 Και το παράλιον της θαλάσσης θέλει είσθαι κατοικίαι και σπήλαια ποιμένων και μάνδραι ποιμνίων.
\par 7 Και το παράλιον τούτο θέλει είσθαι διά το υπόλοιπον του οίκου Ιούδα· εκεί θέλουσι βόσκει· εν τοις οίκοις της Ασκάλωνος θέλουσι καταλύει το εσπέρας· διότι Κύριος ο Θεός αυτών θέλει επισκεφθή αυτούς και αποστρέψει την αιχμαλωσίαν αυτών.
\par 8 Ήκουσα τους ονειδισμούς του Μωάβ και τας ύβρεις των υιών Αμμών, διά των οποίων ωνείδιζον τον λαόν μου και εμεγαλύνοντο κατά των ορίων αυτού.
\par 9 Διά τούτο, Ζω εγώ, λέγει ο Κύριος των δυνάμεων, ο Θεός του Ισραήλ, εξάπαντος ο Μωάβ θέλει είσθαι ως τα Σόδομα και οι υιοί Αμμών ως τα Γόμορρα, τόπος κνίδων και αλυκαί και παντοτεινή ερήμωσις· το υπόλοιπον του λαού μου θέλει λαφυραγωγήσει αυτούς και το υπόλοιπον του έθνους μου θάλει κατακληρονομήσει αυτούς.
\par 10 Τούτο θέλει γείνει εις αυτούς διά την υπερηφανίαν αυτών, διότι ωνείδισαν και εμεγαλύνθησαν κατά του λαού του Κυρίου των δυνάμεων.
\par 11 Ο Κύριος θέλει είσθαι τρομερός εναντίον αυτών, διότι θέλει εξολοθρεύσει πάντας τους θεούς της γής· και θέλουσι προσκυνήσει αυτόν, έκαστος εκ του τόπου αυτού, πάσαι αι νήσοι των εθνών.
\par 12 Και σεις, Αιθίοπες, θέλετε διαπερασθή διά της ρομφαίας μου.
\par 13 Και θέλει εκτείνει την χείρα αυτού κατά του βορρά και αφανίσει την Ασσυρίαν, και θέλει καταστήσει την Νινευή εις αφανισμόν, τόπον άνυδρον ως η έρημος.
\par 14 Και ποίμνια θέλουσι βόσκεσθαι εν μέσω αυτής, πάντα τα ζώα των εθνών· και ο πελεκάν και ο ακανθόχοιρος θέλουσι κατοικεί εν τοις ανωφλίοις αυτής· η φωνή αυτών θέλει ηχήσει εις τα παράθυρα· ερήμωσις θέλει είσθαι εν ταις πύλαις, διότι θέλει γυμνωθή από των κεδρίνων έργων.
\par 15 Αύτη είναι η ευφραινομένη πόλις, η κατοικούσα αμερίμνως, η λέγουσα εν τη καρδία αυτής, Εγώ είμαι και δεν είναι άλλη εκτός εμού. Πως κατεστάθη έρημος, κατάλυμα θηρίων· πας ο διαβαίνων δι' αυτής θέλει συρίξει και κινήσει την χείρα αυτού.

\chapter{3}

\par Ουαί η παραδεδειγματισμένη και μεμολυσμένη· η πόλις η καταθλίβουσα
\par 2 Δεν υπήκουσεν εις την φωνήν· δεν εδέχθη διόρθωσιν· δεν ήλπισεν επί τον Κύριον· δεν επλησίασεν εις τον Θεόν αυτής.
\par 3 Οι άρχοντες αυτής είναι εν αυτή λέοντες ωρυόμενοι· οι κριταί αυτής λύκοι της εσπέρας· δεν αφίνουσιν ουδέν διά το πρωΐ.
\par 4 Οι προφήται αυτής είναι προπετείς, άνθρωποι δόλιοι· οι ιερείς αυτής εβεβήλωσαν το αγιαστήριον, ηθέτησαν τον νόμον.
\par 5 Ο Κύριος είναι δίκαιος εν μέσω αυτής· δεν θέλει κάμει αδικίαν· κατά πάσαν πρωΐαν φέρει την κρίσιν αυτού εις φως, δεν απολείπει· αλλ' ο διεφθαρμένος δεν γνωρίζει αισχύνην.
\par 6 Εξωλόθρευσα έθνη· οι πύργοι αυτών είναι ηρημωμένοι· ηρήμωσα τας οδούς αυτών, ώστε να μη υπάρχη διαβαίνων· αι πόλεις αυτών ηφανίσθησαν, ώστε δεν υπάρχει ουδείς κατοικών.
\par 7 Είπα, Βεβαίως ήθελες με φοβηθή, ήθελες δεχθή παιδείαν, και η κατοικία αυτής δεν ήθελεν εξολοθρευθή, όσον και αν ετιμώρουν αυτήν· πλην αυτοί έσπευσαν να διαφθείρωσι πάσας τας πράξεις αυτών.
\par 8 Διά τούτο προσμένετέ με, λέγει Κύριος, μέχρι της ημέρας καθ' ην εγείρομαι προς λεηλασίαν· διότι η απόφασίς μου είναι να συνάξω τα έθνη, να συναθροίσω τα βασίλεια, να εκχέω επ' αυτά την αγανάκτησίν μου, όλην την έξαψιν της οργής μου· επειδή πάσα η γη θέλει καταναλωθή υπό του πυρός του ζήλου μου.
\par 9 Διότι τότε θέλω αποκαταστήσει εις τους λαούς γλώσσαν καθαράν, διά να επικαλώνται πάντες το όνομα του Κυρίου, να δουλεύωσιν αυτόν υπό ένα ζυγόν.
\par 10 Από του πέραν των ποταμών της Αιθιοπίας οι ικέται μου, η θυγάτηρ των διεσπαρμένων μου, θέλουσι φέρει την προσφοράν μου.
\par 11 Εν τη ημέρα εκείνη δεν θέλεις αισχύνεσθαι διά πάσας τας πράξεις σου, δι' ων ηνόμησας εναντίον μου· διότι τότε θέλω αφαιρέσει εκ μέσου σου τους καυχωμένους εις την μεγαλοπρέπειάν σου, και δεν θέλεις πλέον μεγαλαυχεί κατά του όρους του αγίου μου.
\par 12 Και θέλω αφήσει εν μέσω σου λαόν τεθλιμμένον και πτωχόν, και ούτοι θέλουσιν ελπίζει επί το όνομα του Κυρίου.
\par 13 Το υπόλοιπον του Ισραήλ δεν θέλει πράξει ανομίαν ουδέ λαλήσει ψεύδη, ουδέ θέλει ευρεθή εν τω στόματι αυτών γλώσσα δολία· διότι αυτοί θέλουσι βόσκει και πλαγιάζει, και δεν θέλει υπάρχει ο εκφοβών.
\par 14 Ψάλλε, θύγατερ Σιών· αλαλάξατε, Ισραήλ· τέρπου και ευφραίνου εξ όλης καρδίας, θύγατερ Ιερουσαλήμ.
\par 15 Αφήρεσεν ο Κύριος τας κρίσεις σου, απέστρεψε τον εχθρόν σου· βασιλεύς του Ισραήλ είναι ο Κύριος εν μέσω σου· δεν θέλεις πλέον ιδεί κακόν.
\par 16 Εν τη ημέρα εκείνη θέλει λεχθή προς την Ιερουσαλήμ, Μη φοβού· Σιών, ας μη εκλύωνται αι χείρές σου.
\par 17 Κύριος ο Θεός σου, ο εν μέσω σου, ο δυνατός, θέλει σε σώσει, θέλει ευφρανθή επί σε εν χαρά, θέλει αναπαύεσθαι εις την αγάπην αυτού, θέλει ευφραίνεσθαι εις σε εν άσμασι.
\par 18 Θέλω συνάξει τους λελυπημένους διά τας επισήμους εορτάς, τους όντας από σου, εις τους οποίους ήτο βάρος ο ονειδισμός.
\par 19 Ιδού, εν τω καιρώ εκείνω θέλω αφανίσει πάντας τους καταθλίβοντάς σε· και θέλω σώσει την χωλαίνουσαν και συνάξει την εξωσμένην· και θέλω καταστήσει αυτούς έπαινον και δόξαν εν παντί τόπω της αισχύνης αυτών.
\par 20 Εν τω καιρώ εκείνω θέλω σας φέρει και εν τω καιρώ εκείνω θέλω σας συνάξει· διότι θέλω σας κάμει ονομαστούς και επαινετούς μεταξύ πάντων των λαών της γης, όταν εγώ αποστρέψω την αιχμαλωσίαν σας έμπροσθεν των οφθαλμών σας, λέγει Κύριος.


\end{document}