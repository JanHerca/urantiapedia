\begin{document}

\title{Obadiah}


\chapter{1}

\par Όρασις Αβδιού. Ούτω λέγει Κύριος ο Θεός περί του Εδώμ· Ηκούσαμεν αγγελίαν παρά Κυρίου και μηνυτής απεστάλη προς τα έθνη, Εγέρθητε και ας εγερθώμεν εναντίον αυτού εις πόλεμον.
\par 2 Ιδού, σε κατέστησα μικρόν μεταξύ των εθνών· είσαι καταπεφρονημένος σφόδρα.
\par 3 Η υπερηφανία της καρδίας σου ηπάτησε σε τον κατοικούντα εν τοις κοιλώμασι των κρημνών, του οποίου η κατοικία είναι υψηλή, όστις λέγει εν τη καρδία αυτού, Τις θέλει με καταβιβάσει εις την γην;
\par 4 Εάν μετεωρισθής ως αετός και εάν θέσης την φωλεάν σου αναμέσον των άστρων, και εκείθεν θέλω σε καταβιβάσει, λέγει Κύριος.
\par 5 Εάν κλέπται ήρχοντο προς σε, εάν λησταί διά νυκτός -πως εξηλείφθης- δεν ήθελον αρπάσει το αρκούν εις αυτούς; εάν τρυγηταί ήρχοντο προς σε, δεν ήθελον αφήσει επιφυλλίδας;
\par 6 Πως εξηρευνήθη ο Ησαύ· απεκαλύφθησαν οι κρυψώνες αυτού.
\par 7 Πάντες οι άνδρες της συμμαχίας σου σε συνώδευσαν έως του ορίου σου· οι άνθρωποι, οίτινες ήσαν εν ειρήνη μετά σου, σε ηπάτησαν και υπερίσχυσαν εναντίον σου· οι τρώγοντες τον άρτον σου έβαλον ενέδραν υποκάτω σου· δεν υπάρχει σύνεσις εν αυτώ.
\par 8 Εν τη ημέρα εκείνη, λέγει Κύριος, δεν θέλω απολέσει και τους σοφούς από του Εδώμ και την σύνεσιν από του όρους του Ησαύ;
\par 9 Και οι μαχηταί σου, Θαιμάν, θέλουσι πτοηθή, διά να εκκοπή εν σφαγή πας άνθρωπος εκ του όρους του Ησαύ.
\par 10 Διά την αδικίαν την προς τον αδελφόν σου Ιακώβ θέλει σε καλύψει αισχύνη και θέλεις εκκοπή διαπαντός.
\par 11 Εν τη ημέρα καθ' ην ίστασο απέναντι, τη ημέρα καθ' ην οι αλλογενείς έφεραν εις αιχμαλωσίαν το στράτευμα αυτού και οι αλλότριοι εισήλθον εις τας πύλας αυτού και έβαλον κλήρους επί την Ιερουσαλήμ, έσο και συ ως εις αυτών.
\par 12 Δεν έπρεπεν όμως να επιβλέπης εις την ημέραν του αδελφού σου, εις την ημέραν της αποξενώσεως αυτού, ουδέ να επιχαίρης κατά των υιών του Ιούδα εν τη ημέρα του αφανισμού αυτών, ουδέ να μεγαλορρημονής εν τη ημέρα της θλίψεως αυτών.
\par 13 Δεν έπρεπε να εισέλθης εις την πύλην του λαού μου εν τη ημέρα της συμφοράς αυτών, ουδέ να θεωρής και συ την θλίψιν αυτών εν τη ημέρα της συμφοράς αυτών, ουδέ να επιβάλης χείρα επί την περιουσίαν αυτών εν τη ημέρα της συμφοράς αυτών,
\par 14 ουδέ έπρεπε να σταθής επί τας διεξόδους, διά να αποκλείης τους διασωζομένους αυτού ουδέ να παραδώσης τους υπολοίπους αυτού εν τη ημέρα της θλίψεως αυτών
\par 15 διότι εγγύς είναι η ημέρα του Κυρίου επί πάντα τα έθνη· καθώς έκαμες θέλει γείνει εις σέ· η ανταπόδοσίς σου θέλει στρέψει επί την κεφαλήν σου.
\par 16 Διότι καθώς σεις επίετε επί το όρος το άγιόν μου, ούτω θέλουσι πίνει διαπαντός πάντα τα έθνη· ναι, θέλουσι πίνει και θέλουσιν εκροφεί και θέλουσιν είσθαι ως οι μη υπάρχοντες
\par 17 Επί δε του όρους Σιών θέλει είσθαι σωτηρία και θέλει είσθαι άγιον· και ο οίκος Ιακώβ θέλει κατακληρονομήσει τας κληρονομίας αυτών·
\par 18 και ο οίκος Ιακώβ θέλει είσθαι πυρ και ο οίκος Ιωσήφ φλόξ, ο δε οίκος Ησαύ ως καλάμη· και θέλουσιν εξαφθή κατ' αυτών και καταφάγει αυτούς· και δεν θέλει είσθαι υπόλοιπον του οίκου Ησαύ· διότι Κύριος ελάλησε.
\par 19 Και οι της μεσημβρίας θέλουσι κατακληρονομήσει το όρος του Ησαύ και οι της πεδινής τους Φιλισταίους· και θέλουσι κατακληρονομήσει τους αγρούς του Εφραΐμ και τους αγρούς της Σαμαρείας, ο δε Βενιαμίν την Γαλαάδ,
\par 20 και το αιχμαλωτισθέν τούτο στράτευμα των υιών Ισραήλ την γην εκείνην των Χαναναίων έως Σαρεπτά, και οι αιχμαλωτισθέντες της Ιερουσαλήμ, οι εν Σεφαράδ, θέλουσι κατακληρονομήσει τας πόλεις του νότου·
\par 21 και θέλουσιν αναβή σωτήρες εις το όρος Σιών, διά να κρίνωσι το όρος του Ησαύ· και του Κυρίου θέλει είσθαι η βασιλεία.


\end{document}