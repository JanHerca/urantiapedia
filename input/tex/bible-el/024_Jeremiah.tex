\begin{document}

\title{Jeremiah}


\chapter{1}

\par Οι λόγοι του Ιερεμίου υιού του Χελκίου, εκ των ιερέων των εν Αναθώθ εν γη Βενιαμίν·
\par 2 προς τον οποίον έγεινε λόγος Κυρίου εν ταις ημέραις του Ιωσίου υιού του Αμών βασιλέως Ιούδα, κατά το δέκατον τρίτον έτος της βασιλείας αυτού.
\par 3 Έγεινε και εν ταις ημέραις του Ιωακείμ, υιού του Ιωσίου βασιλέως Ιούδα, μέχρι του τέλους του ενδεκάτου έτους του Σεδεκίου, υιού του Ιωσίου βασιλέως Ιούδα, μέχρι της αιχμαλωσίας της Ιερουσαλήμ, κατά τον πέμπτον μήνα.
\par 4 Και λόγος Κυρίου έγεινε προς εμέ λέγων,
\par 5 Πριν σε μορφώσω εν τη κοιλία, σε εγνώρισα· και πριν εξέλθης εκ της μήτρας, σε ηγίασα· προφήτην εις τα έθνη σε κατέστησα.
\par 6 Και εγώ είπα, Ω, Κύριε Θεέ, ιδού, δεν εξεύρω να λαλήσω διότι είμαι παιδίον.
\par 7 Ο δε Κύριος είπε προς εμέ, Μη λέγε, είμαι παιδίον· διότι θέλεις υπάγει προς πάντας, προς τους οποίους θέλω σε εξαποστείλει· και πάντα όσα σε προστάξω, θέλεις ειπεί.
\par 8 Μη φοβηθής από προσώπου αυτών· διότι εγώ είμαι μετά σου διά να σε ελευθερόνω, λέγει Κύριος.
\par 9 Και εξέτεινε Κύριος την χείρα αυτού και ήγγισε το στόμα μου· και είπε Κύριος προς εμέ, Ιδού, έθεσα τους λόγους μου εν τω στόματί σου.
\par 10 Ιδέ, σε κατέστησα σήμερον επί τα έθνη και επί τας βασιλείας, διά να εκριζόνης και να κατασκάπτης και να καταστρέφης και να κατεδαφίζης, να ανοικοδομής και να καταφυτεύης.
\par 11 Λόγος Κυρίου έγεινεν έτι προς εμέ λέγων, Τι βλέπεις συ, Ιερεμία; Και είπα, Βλέπω βακτηρίαν αμυγδαλίνην.
\par 12 Και είπε Κύριος προς εμέ, Καλώς είδες· διότι εγώ θέλω ταχύνει να εκπληρώσω τον λόγον μου.
\par 13 Και έγεινε λόγος Κυρίου προς εμέ εκ δευτέρου λέγων, Τι βλέπεις συ; Και είπα, Βλέπω λέβητα αναβράζοντα· και το πρόσωπον αυτού είναι προς βορράν.
\par 14 Και είπε Κύριος προς εμέ, Από βορρά θέλει εκχυθή το κακόν επί πάντας τους κατοίκους της γης.
\par 15 Διότι ιδού, εγώ θέλω καλέσει πάσας τας οικογενείας των βασιλείων του βορρά, λέγει Κύριος· και θέλουσιν ελθεί και θέλουσι θέσει έκαστος τον θρόνον αυτού εν τη εισόδω των πυλών της Ιερουσαλήμ και επί πάντα τα τείχη αυτής κύκλω και επί πάσας τας πόλεις του Ιούδα.
\par 16 Και θέλω προφέρει τας κρίσεις μου εναντίον αυτών περί πάσης της κακίας αυτών· διότι με εγκατέλιπον και εθυμίασαν εις θεούς αλλοτρίους και προσεκύνησαν τα έργα των χειρών αυτών.
\par 17 Συ λοιπόν περίζωσον την οσφύν σου και σηκώθητι και ειπέ προς αυτούς πάντα όσα εγώ σε προστάξω· μη φοβηθής από προσώπου αυτών, μήποτε τάχα σε αφήσω να πέσης εις αμηχανίαν έμπροσθεν αυτών.
\par 18 Διότι, ιδού, εγώ σε έθεσα σήμερον ως πόλιν οχυράν και ως στήλην σιδηράν και ως τείχη χάλκινα εναντίον πάσης της γης, εναντίον των βασιλέων του Ιούδα, εναντίον των αρχόντων αυτού, εναντίον των ιερέων αυτού και εναντίον του λαού της γής·
\par 19 και θέλουσι σε πολεμήσει αλλά δεν θέλουσιν υπερισχύσει εναντίον σου· διότι εγώ είμαι μετά σου διά να σε ελευθερόνω, λέγει Κύριος.

\chapter{2}

\par Και έγεινε λόγος Κυρίου προς εμέ λέγων,
\par 2 Ύπαγε και βόησον εις τα ώτα της Ιερουσαλήμ λέγων, Ούτω λέγει Κύριος· Ενθυμούμαι περί σου την προς σε ευμένειάν μου εν τη νεότητί σου, την αγάπην της νυμφεύσεώς σου, ότε με ηκολούθεις εν τη ερήμω, εν γη ασπάρτω·
\par 3 ο Ισραήλ ήτο άγιος εις τον Κύριον, απαρχή των γεννημάτων αυτού· πάντες οι κατατρώγοντες αυτόν ήσαν ένοχοι· κακόν ήλθεν επ' αυτούς, λέγει Κύριος.
\par 4 Ακούσατε τον λόγον του Κυρίου, οίκος Ιακώβ και πάσαι αι συγγένειαι του οίκου Ισραήλ·
\par 5 Ούτω λέγει Κύριος· Ποίαν αδικίαν εύρηκαν εν εμοί οι πατέρες σας, ώστε απεμακρύνθησαν απ' εμού και περιεπάτηααν οπίσω της ματαιότητος και εματαιώθησαν;
\par 6 και δεν είπον, Που είναι ο Κύριος, ο αναβιβάσας ημάς εκ γης Αιγύπτου, ο οδηγήσας ημάς διά της ερήμου, διά τόπου ερημίας και χασμάτων, διά τόπου ανυδρίας και σκιάς θανάτου, διά τόπου τον οποίον δεν επέρασεν άνθρωπος και όπου άνθρωπος δεν κατώκησε;
\par 7 Και σας εισήγαγον εις τόπον καρποφόρον, διά να τρώγητε τους καρπούς αυτού και τα αγαθά αυτού· αφού όμως εισήλθετε, εμιάνατε την γην μου και κατεστήσατε βδέλυγμα την κληρονομίαν μου.
\par 8 Οι ιερείς δεν είπον, Που είναι ο Κύριος; και οι κρατούντες τον νόμον δεν με εγνώρισαν· και οι ποιμένες εγίνοντο παραβάται εναντίον μου, και οι προφήται προεφήτευον διά του Βάαλ και περιεπάτουν οπίσω πραγμάτων ανωφελών.
\par 9 Διά τούτο έτι θέλω κριθή με εσάς, λέγει Κύριος, και με τους υιούς των υιών σας θέλω κριθή.
\par 10 Διότι διάβητε εις τας νήσους των Κητιαίων και ιδέτε· και πέμψατε εις Κηδάρ· και παρατηρήσατε επιμελώς, και ιδέτε αν εστάθη τοιούτον πράγμα.
\par 11 Ήλλαξεν έθνος θεούς, αν και ούτοι δεν ήναι θεοί; ο λαός μου όμως ήλλαξε την δόξαν αυτού με πράγμα ανωφελές.
\par 12 Εκπλάγητε, ουρανοί, διά τούτο, και φρίξατε, συνταράχθητε σφόδρα, λέγει Κύριος.
\par 13 Διότι δύο κακά έπραξεν ο λαός μου· εμέ εγκατέλιπον, την πηγήν των ζώντων υδάτων, και έσκαψαν εις εαυτούς λάκκους, λάκκους συντετριμμένους, οίτινες δεν δύνανται να κρατήσωσιν ύδωρ.
\par 14 Μήπως είναι δούλος ο Ισραήλ; ή δούλος οικογενής; διά τι κατεστάθη λάφυρον;
\par 15 Οι σκύμνοι εβρύχησαν επ' αυτόν, εξέδωκαν την φωνήν αυτών και κατέστησαν την γην αυτού έρημον· αι πόλεις αυτού κατεκάησαν και έμειναν ακατοίκητοι.
\par 16 Οι υιοί προσέτι της Νωφ και της Τάφνης συνέτριψαν την κορυφήν σου.
\par 17 Δεν έκαμες τούτο συ εις σεαυτόν, διότι εγκατέλιπες Κύριον τον Θεόν σου ότε σε ώδήγει εν τη οδώ;
\par 18 Και τώρα τι έχεις να κάμης εν τη οδώ της Αιγύπτου, διά να πίης τα ύδατα Σιώρ; ή τι έχεις να κάμης εν τη οδώ της Ασσυρίας, διά να πίης τα ύδατα του ποταμού;
\par 19 Η ασέβειά σου θέλει σε παιδεύσει και αι παραβάσεις σου θέλουσι σε ελέγξει· γνώρισον λοιπόν και ιδέ, ότι είναι κακόν και πικρόν, το ότι εγκατέλιπες Κύριον τον Θεόν σου, και δεν είναι ο φόβος μου εν σοι, λέγει Κύριος ο Θεός των δυνάμεων.
\par 20 Επειδή προ πολλού συνέτριψα τον ζυγόν σου, διέσπασα τα δεσμά σου, και συ είπας, δεν θέλω σταθή παραβάτης πλέον· ενώ επί πάντα υψηλόν λόφον και υποκάτω παντός δένδρου πρασίνου περιεπλανήθης εκπορνεύων.
\par 21 Εγώ δε σε εφύτευσα άμπελον εκλεκτήν, σπέρμα όλως αληθινόν· πως λοιπόν μετεβλήθης εις παρεφθαρμένον κλήμα αμπέλου ξένης εις εμέ;
\par 22 Διά τούτο και εάν πλυθής με νίτρον και πληθύνης εις σεαυτόν το σμήγμα, η ανομία σου μένει σεσημειωμένη ενώπιόν μου, λέγει Κύριος ο Θεός.
\par 23 Πως δύνασαι να είπης, δεν εμιάνθην, δεν υπήγα οπίσω των Βααλείμ; ιδέ την οδόν σου εν τη φάραγγι, γνώρισον τι έπραξας· είσαι ταχεία δρομάς διατρέχουσα εν ταις οδοίς αυτής·
\par 24 όνος αγρία συνειθισμένη εις την έρημον, αναπνέουσα τον αέρα κατά την επιθυμίαν της καρδίας αυτής· την ορμήν αυτής, τις δύναται να επιστρέψη αυτήν; πάντες οι ζητούντες αυτήν δεν θέλουσι κοπιάζει· εν τω μηνί αυτής θέλουσιν ευρεί αυτήν.
\par 25 Κράτησον τον πόδα σου από του να περιπατήσης ανυπόδητος, και τον λάρυγγά σου από δίψης· αλλά συ είπας, εις μάτην· ουχί· διότι ηγάπησα ξένους και κατόπιν αυτών θέλω υπάγει.
\par 26 Καθώς ο κλέπτης αισχύνεται όταν ευρεθή, ούτω θέλει αισχυνθή ο οίκος Ισραήλ, αυτοί, οι βασιλείς αυτών, οι άρχοντες αυτών και οι ιερείς αυτών και οι προφήται αυτών·
\par 27 οίτινες λέγουσι προς το ξύλον, Πατήρ μου είσαι· και προς τον λίθον, Συ με εγέννησας· διότι έστρεψαν νώτα προς εμέ και ουχί πρόσωπον· εν τω καιρώ όμως της συμφοράς αυτών θέλουσιν ειπεί, Ανάστηθι και σώσον ημάς.
\par 28 Και που είναι οι θεοί σου, τους οποίους έκαμες εις σεαυτόν; ας αναστηθώσιν, εάν δύνανται να σε σώσωσιν εν τω καιρώ της συμφοράς σου· διότι κατά τον αριθμόν των πόλεών σου ήσαν οι θεοί σου, Ιούδα.
\par 29 Διά τι ηθέλετε κριθή μετ' εμού; σεις πάντες είσθε παραβάται εις εμέ, λέγει Κύριος.
\par 30 Εις μάτην επάταξα τα τέκνα σας· δεν εδέχθησαν διόρθωσιν· η μάχαιρά σας κατέφαγε τους προφήτας σας ως λέων εξολοθρεύων.
\par 31 Ω γενεά, ιδέτε τον λόγον του Κυρίου· Εστάθην έρημος εις τον Ισραήλ, γη σκότους; διά τι λέγει ο λαός μου, Ημείς είμεθα κύριοι· δεν θέλομεν ελθεί πλέον προς σε;
\par 32 Δύναται η κόρη να λησμονήση τους στολισμούς αυτής, η νύμφη τον καλλωπισμόν αυτής; και όμως ο λαός μου με ελησμόνησεν ημέρας αναριθμήτους.
\par 33 Διά τι καλλωπίζεις την οδόν σου διά να ζητής εραστάς; εις τρόπον ώστε και εδίδαξας τας οδούς σου εις τας κακάς.
\par 34 Έτι εις τα κράσπεδά σου ευρέθησαν αίματα ψυχών πτωχών αθώων· δεν εύρηκα αυτά ανορύττων, αλλ' επί πάντα ταύτα.
\par 35 Και όμως λέγεις, Επειδή είμαι αθώος, βεβαίως ο θυμός αυτού θέλει αποστραφή απ' εμού. Ιδού, εγώ θέλω κριθή μετά σου, διότι λέγεις, Δεν ημάρτησα.
\par 36 Διά τι περιπλανάσαι τόσον διά να αλλάξης την οδόν σου; θέλεις καταισχυνθή και υπό της Αιγύπτου, καθώς κατησχύνθης υπό της Ασσυρίας.
\par 37 Ναι, θέλεις εξέλθει εντεύθεν με τας χείρας σου επί την κεφαλήν σου· διότι ο Κύριος απέβαλε τας ελπίδας σου και δεν θέλεις ευημερήσει εις αυτάς.

\chapter{3}

\par Λέγουσιν, Εάν τις αποβάλη την γυναίκα αυτού και αναχωρήση από αυτού και γείνη άλλου ανδρός, θέλει επιστρέψει πάλιν εκείνος προς αυτήν; η γη εκείνη δεν θέλει όλως μιανθή; συ επόρνευσας μεν μετά πολλών εραστών· επίστρεψον δε πάλιν προς εμέ, λέγει Κύριος.
\par 2 Σήκωσον τους οφθαλμούς σου προς τους υψηλούς τόπους, και ιδέ που δεν εσέλγησας. Εν ταις οδοίς εκάθησας δι' αυτούς, ως ο Άραψ εν τη ερήμω, και εμίανας την γην με τας πορνείας σου και με την κακίαν σου.
\par 3 Διά τούτο εκρατήθησαν αι βροχαί, και δεν έγεινε βροχή όψιμος· και συ είχες το μέτωπον της πόρνης, απέβαλες πάσαν εντροπήν.
\par 4 Δεν θέλεις κράζει από του νυν προς εμέ, Πάτερ μου, συ είσαι ο οδηγός της νεότητός μου;
\par 5 Θέλει διατηρεί την οργήν αυτού διαπαντός; θέλει φυλάττει αυτήν έως τέλους; ιδού, ελάλησας και έπραξας τα κακά, όσον ηδυνήθης.
\par 6 Ο Κύριος είπεν έτι προς εμέ εν ταις ημέραις Ιωσίου του βασιλέως, Είδες εκείνα, τα οποία Ισραήλ η αποστάτις έπραξεν; υπήγεν επί παν υψηλόν όρος και υποκάτω παντός πρασίνου δένδρου και επόρνευσεν εκεί.
\par 7 Και αφού έπραξε πάντα ταύτα, είπα, Επίστρεψον προς εμέ· και δεν επέστρεψε. Και είδε τούτο Ιούδας η άπιστος αυτής αδελφή.
\par 8 Και είδον ότι ενώ επειδή Ισραήλ η αποστάτις εμοίχευσεν εγώ απέπεμψα αυτήν και έδωκα εις αυτήν το γράμμα του διαζυγίου αυτής, Ιούδας η άπιστος αυτής αδελφή δεν εφοβήθη αλλ' υπήγε και επόρνευσε και αυτή.
\par 9 Και με την διαφήμισιν της πορνείας αυτής εμίανε τον τόπον και εμοίχευσε μετά των λίθων και μετά των ξύλων.
\par 10 Και εν πάσι τούτοις Ιούδας η άπιστος αυτής αδελφή δεν επέστρεψε προς εμέ εξ όλης της καρδίας αυτής αλλά ψευδώς, λέγει Κύριος.
\par 11 Και είπε Κύριος προς εμέ, Ισραήλ η αποστάτις εδικαίωσεν εαυτήν περισσότερον παρά Ιούδας η άπιστος.
\par 12 Ύπαγε και διακήρυξον τους λόγους τούτους προς τον βορράν και ειπέ, Επίστρεψον, Ισραήλ η αποστάτις, λέγει Κύριος, και δεν θέλω κάμει να πέση η οργή μου εφ' υμάς· διότι ελεήμων είμαι, λέγει Κύριος· δεν θέλω φυλάττει την οργήν διαπαντός.
\par 13 Μόνον γνώρισον την ανομίαν σου, ότι ημάρτησας εις Κύριον τον Θεόν σου, και διήρεσας τας οδούς σου εις τους ξένους υποκάτω παντός πρασίνου δένδρου, και δεν υπηκούσατε εις την φωνήν μου, λέγει Κύριος.
\par 14 Επιστρέψατε, υιοί αποστάται, λέγει Κύριος, αν και εγώ σας απεστράφην· και θέλω σας λάβει ένα εκ πόλεως και δύο εκ συγγενείας και θέλω σας εισάξει εις την Σιών·
\par 15 και θέλω σας δώσει ποιμένας κατά την καρδίαν μου και θέλουσι σας ποιμάνει εν γνώσει και συνέσει.
\par 16 Και όταν πληθυνθήτε και αυξηνθήτε επί της γης, εν εκείναις ταις ημέραις, λέγει Κύριος, δεν θέλουσι προφέρει πλέον, Η κιβωτός της διαθήκης του Κυρίου, ουδέ θέλει αναβή επί καρδίαν αυτών, ουδέ θέλουσιν ενθυμηθή αυτήν, ουδέ θέλουσιν επισκεφθή, ουδέ θέλει κατασκευασθή πλέον.
\par 17 Εν εκείνω τω καιρώ θέλουσιν ονομάσει την Ιερουσαλήμ· θρόνον του Κυρίου· και πάντα τα έθνη θέλουσι συναχθή προς αυτήν εν τω ονόματι του Κυρίου, προς την Ιερουσαλήμ· και δεν θέλουσι περιπατήσει πλέον οπίσω της ορέξεως της πονηράς αυτών καρδίας.
\par 18 Εν εκείναις ταις ημέραις ο οίκος Ιούδα θέλει περιπατήσει μετά του οίκου Ισραήλ, και θέλουσιν ελθεί ομού από της γης του βορρά, εις την γην την οποίαν εκληροδότησα εις τους πατέρας σας.
\par 19 Αλλ' εγώ είπα, Πως θέλω σε κατατάξει μεταξύ των τέκνων και δώσει εις σε γην επιθυμητήν, ένδοξον κληρονομίαν των δυνάμεων των εθνών; Και είπα, Συ θέλεις με κράξει, Πάτερ μου· και δεν θέλεις αποστρέψει από όπισθέν μου.
\par 20 Βεβαίως καθώς γυνή αθετεί εις τον άνδρα αυτής, ούτως ηθετήσατε εις εμέ, οίκος Ισραήλ, λέγει Κύριος.
\par 21 Φωνή ηκούσθη επί των υψηλών τόπων, κλαυθμός και δεήσεις των υιών Ισραήλ· διότι διέστρεψαν την οδόν αυτών, ελησμόνησαν Κύριον τον Θεόν αυτών.
\par 22 Επιστρέψατε, υιοί αποστάται, και θέλω ιατρεύσει τας αποστασίας σας. Ιδού, ημείς ερχόμεθα προς σέ· διότι συ είσαι Κύριος ο Θεός ημών.
\par 23 Τωόντι εις μάτην ελπίζεται σωτηρία εκ των λόφων και εκ του πλήθους των ορέων· μόνον εν Κυρίω τω Θεώ ημών είναι η σωτηρία του Ισραήλ.
\par 24 Διότι η αισχύνη κατέφαγε τον κόπον των πατέρων ημών εκ της νεότητος ημών· τα ποίμνια αυτών και τας αγέλας αυτών, τους υιούς αυτών και τας θυγατέρας αυτών.
\par 25 Εν τη αισχύνη ημών κατακείμεθα, και η ατιμία ημών καλύπτει ημάς· διότι ημαρτήσαμεν εις Κύριον τον Θεόν ημών, ημείς και οι πατέρες ημών, εκ της νεότητος ημών έως της ημέρας ταύτης, και δεν υπηκούσαμεν εις την φωνήν Κυρίου του Θεού ημών.

\chapter{4}

\par Εάν επιστρέψης, Ισραήλ, λέγει Κύριος, επίστρεψον προς εμέ· και εάν εκβάλης τα βδελύγματά σου απ' έμπροσθέν μου, τότε δεν θέλεις μετατοπισθή.
\par 2 Και θέλεις ομόσει, λέγων, Ζη Κύριος, εν αληθεία εν κρίσει και εν δικαιοσύνη· και τα έθνη θέλουσιν ευλογείσθαι εν αυτώ και εν αυτώ θέλουσι δοξασθή.
\par 3 Διότι ούτω λέγει Κύριος προς τους άνδρας Ιούδα και προς την Ιερουσαλήμ· Αροτριάσατε τους κεχερσωμένους αγρούς σας και μη σπείρετε μεταξύ ακανθών.
\par 4 Περιτμήθητε εις τον Κύριον και αφαιρέσατε τας ακροβυστίας της καρδίας σας, άνδρες Ιούδα και κάτοικοι της Ιερουσαλήμ, μήποτε εξέλθη ο θυμός μου ως πυρ και εξαφθή, και ουδείς θέλει είσθαι ο σβέσων, ένεκεν της κακίας των πράξεών σας.
\par 5 Αναγγείλατε προς τον Ιούδαν και κηρύξατε προς την Ιερουσαλήμ· και είπατε και ηχήσατε σάλπιγγα εις την γήν· βοήσατε, συναθροίσθητε και είπατε, Συνάχθητε και ας εισέλθωμεν εις τας ωχυρωμένας πόλεις.
\par 6 Υψώσατε σημαίαν προς την Σιών· σύρθητε, μη σταθήτε· διότι εγώ θέλω φέρει κακόν από βορρά και συντριμμόν μέγαν.
\par 7 Ο λέων ανέβη εκ του δάσους αυτού και ο εξολοθρευτής των εθνών εσηκώθη· εξήλθεν εκ του τόπου αυτού διά να ερημώση την γην σου· αι πόλεις σου θέλουσι καταστραφή, ώστε δεν θέλει είσθαι ουδείς ο κατοικών.
\par 8 Διά τούτο περιζώσθητε σάκκους, θρηνήσατε και ολολύξατε· διότι ο φλογερός θυμός του Κυρίου δεν εστράφη αφ' ημών.
\par 9 Και εν εκείνη τη ημέρα, λέγει Κύριος, η καρδία του βασιλέως θέλει χαθή και η καρδία των αρχόντων· και οι ιερείς θέλουσιν εκθαμβηθή και οι προφήται θέλουσιν εκπλαγή.
\par 10 Τότε είπα, Ω Κύριε Θεέ απατών λοιπόν ηπάτησας τον λαόν τούτον και την Ιερουσαλήμ, λέγων, Ειρήνην θέλετε έχει· ενώ η μάχαιρα έφθασεν έως της ψυχής.
\par 11 Εν εκείνω τω καιρώ θέλουσιν ειπεί προς τον λαόν τούτον και προς την Ιερουσαλήμ, Άνεμος καυστικός των υψηλών τόπων της ερήμου φυσά προς την θυγατέρα του λαού μου, ουχί διά να ανεμίση ουδέ διά να καθαρίση·
\par 12 άνεμος σφοδρότερος παρά τούτους θέλει ελθεί δι' εμέ· εγώ δε τώρα θέλω εκφέρει κρίσεις εις αυτούς.
\par 13 Ιδού, ως νεφέλη θέλει αναβή, και αι άμαξαι αυτού θέλουσιν είσθαι ως ανεμοστρόβιλος· οι ίπποι αυτού είναι ελαφρότεροι των αετών. Ουαί εις ημάς, διότι εχάθημεν.
\par 14 Ιερουσαλήμ, απόπλυνον την καρδίαν σου από κακίας, διά να σωθής· έως πότε θέλουσι κατοικεί εν σοι οι μάταιοι διαλογισμοί σου;
\par 15 Διότι φωνή αναγγέλλει εκ του Δαν και κηρύττει θλίψιν από του όρους Εφραΐμ.
\par 16 Ενθυμίσατε τούτο εις τα έθνη· ιδού, διακηρύξατε εναντίον της Ιερουσαλήμ, ότι πολιορκηταί έρχονται από γης μακράς και εκπέμπουσι την φωνήν αυτών εναντίον των πόλεων Ιούδα.
\par 17 Ως φύλακες αγρού παρετάχθησαν εναντίον αυτής κυκλόθεν· διότι απεστάτησεν εναντίον μου, λέγει Κύριος.
\par 18 Αι οδοί σου και τα επιτηδεύματά σου επροξένησαν εις σε ταύτα· αύτη η κακία σου εστάθη μάλιστα πικρά, ναι, έφθασεν έως της καρδίας σου.
\par 19 Τα εντόσθιά μου, τα εντόσθιά μου· πονώ εις τα βάθη της καρδίας μου· η καρδία μου θορυβείται εν εμοί· δεν δύναμαι να σιωπήσω, διότι ήκουσας, ψυχή μου, ήχον σάλπιγγος, αλαλαγμόν πολέμου.
\par 20 Συντριμμός επί συντριμμόν διακηρύττεται· διότι πάσα η γη ερημούται· εξαίφνης αι σκηναί μου ηρημώθησαν και τα παραπετάσματά μου εν μιά στιγμή.
\par 21 Έως πότε θέλω βλέπει την σημαίαν, θέλω ακούει τον ήχον της σάλπιγγος;
\par 22 Διότι ο λαός μου είναι άφρων· δεν με εγνώρισαν· είναι υιοί άφρονες και δεν έχουσι σύνεσιν· είναι σοφοί εις το να κακοποιώσι, να αγαθοποιώσιν όμως δεν εξεύρουσιν.
\par 23 Επέβλεψα επί την γην και ιδού, άμορφος και έρημος· και εις τους ουρανούς και δεν υπήρχε το φως αυτών.
\par 24 Είδον τα όρη και ιδού, έτρεμον και πάντες οι λόφοι κατεσείοντο.
\par 25 Είδον και ιδού, δεν υπήρχεν άνθρωπος και πάντα τα πετεινά του ουρανού είχον φύγει.
\par 26 Είδον και ιδού, ο Κάρμηλος έρημος και πάσαι αι πόλεις αυτού κατηδαφισμέναι από προσώπου Κυρίου, από του φλογερού θυμού αυτού.
\par 27 Διότι ούτω λέγει Κύριος· πάσα η γη θέλει είσθαι έρημος· συντέλειαν όμως δεν θέλω κάμει.
\par 28 Διά τούτο η γη θέλει πενθήσει και οι ουρανοί άνωθεν θέλουσι συσκοτάσει· διότι εγώ ελάλησα, απεφάσισα και δεν θέλω μετανοήσει ουδέ θέλω επιστρέψει από τούτου.
\par 29 Πάσα η πόλις θέλει φύγει υπό του θορύβου των ιππέων και των τοξοτών θέλουσιν ελθεί εις τα δάση και αναβή επί τους βράχους· πάσα πόλις θέλει εγκαταλειφθή και δεν θέλει είσθαι άνθρωπος κατοικών εν αυταίς.
\par 30 Και συ, ηφανισμένη, τι θέλεις κάμει; και αν ενδυθής κόκκινον, και αν στολισθής με στολισμούς χρυσούς, και αν διατείνης με στίμμι τους οφθαλμούς σου, εις μάτην θέλεις καλλωπισθή· οι ερασταί σου θέλουσι σε καταφρονήσει, θέλουσι ζητεί την ζωήν σου.
\par 31 Διότι ήκουσα φωνήν ως κοιλοπονούσης, στεναγμόν ως πρωτογεννώσης φωνήν της θυγατρός Σιών, ήτις θρηνολογεί εαυτήν, εκτείνει τας χείρας αυτής, λέγουσα, Ουαί εις εμέ τώρα, διότι η ψυχή μου εκλείπει ένεκεν των φονευτών.

\chapter{5}

\par Περιέλθετε εν ταις οδοίς της Ιερουσαλήμ και ιδέτε τώρα και μάθετε και ζητήσατε εν ταις πλατείαις αυτής, εάν δύνασθε να εύρητε άνθρωπον, εάν υπάρχη ο ποιών κρίσιν, ο ζητών αλήθειαν· και θέλω συγχωρήσει εις αυτήν.
\par 2 Και αν λέγωσι, Ζη ο Κύριος, ψευδώς τωόντι ομνύουσι.
\par 3 Κύριε, δεν επιβλέπουσιν οι οφθαλμοί σου επί την αλήθειαν; εμαστίγωσας αυτούς και δεν επόνεσαν· κατηνάλωσας αυτούς και δεν ηθέλησαν να δεχθώσι διόρθωσιν εσκλήρυναν τα πρόσωπα αυτών υπέρ τον βράχον· δεν ηθέλησαν να επιστρέψωσι.
\par 4 Τότε εγώ είπα, Ούτοι βεβαίως είναι πτωχοί· είναι άφρονες· διότι δεν γνωρίζουσι την οδόν του Κυρίου, την κρίσιν του Θεού αυτών·
\par 5 θέλω υπάγει προς τους μεγάλους και θέλω λαλήσει προς αυτούς· διότι αυτοί εγνώρισαν την οδόν του Κυρίου, την κρίσιν του Θεού αυτών· αλλά και ούτοι πάντες ομού συνέτριψαν τον ζυγόν, έκοψαν τους δεσμούς.
\par 6 Διά τούτο λέων εκ του δάσους θέλει φονεύσει αυτούς, λύκος της ερήμου θέλει εξολοθρεύσει αυτούς, πάρδαλις θέλει κατασκοπεύσει επί τας πόλεις αυτών· πας όστις εξέλθη εκείθεν, θέλει κατασπαραχθή· διότι επληθύνθησαν αι παραβάσεις αυτών, ηυξήνθησαν αι αποστασίαι αυτών.
\par 7 Πως θέλω συγχωρήσει εις σε διά τούτο; οι υιοί σου με εγκατέλιπον και ώμνυον εις τους μη θεούς· αφού εχόρτασα αυτούς, τότε εμοίχευον και συνεσωρεύοντο εις οίκον πόρνης.
\par 8 Ήσαν ως οι κεχορτασμένοι ίπποι το πρωΐ· έκαστος εχρεμέτιζε κατόπιν της γυναικός του πλησίον αυτού.
\par 9 Δεν θέλω κάμει διά ταύτα επίσκεψιν; λέγει Κύριος· και η ψυχή μου δεν θέλει εκδικηθή εναντίον έθνους τοιούτου;
\par 10 Ανάβητε επί τα τείχη αυτής και κρημνίζετε, πλην μη κάμητε συντέλειαν· αφαιρέσατε τας επάλξεις αυτής, διότι δεν είναι του Κυρίου·
\par 11 διότι ο οίκος Ισραήλ και ο οίκος Ιούδα εφέρθησαν πολλά απίστως προς εμέ, λέγει Κύριος.
\par 12 Ηρνήθησαν τον Κύριον και είπον, Δεν είναι αυτός, και δεν θέλει ελθεί κακόν εφ' ημάς, ουδέ θέλομεν ιδεί μάχαιραν ή πείναν·
\par 13 και οι προφήται είναι άνεμος και ο λόγος δεν υπάρχει εν αυτοίς· εις αυτούς θέλει γείνει ούτω.
\par 14 Διά τούτο ούτω λέγει Κύριος ο Θεός των δυνάμεων· Επειδή λαλείτε τον λόγον τούτον, ιδού, εγώ θέλω κάμει τους λόγους μου εν τω στόματί σου πυρ και τον λαόν τούτον ξύλα και θέλει καταφάγει αυτούς.
\par 15 Ιδού, εγώ θέλω φέρει εφ' υμάς έθνος μακρόθεν, οίκος Ισραήλ, λέγει Κύριος· είναι έθνος ισχυρόν, είναι έθνος αρχαίον, έθνος του οποίου δεν γνωρίζεις την γλώσσαν ουδέ καταλαμβάνεις τι λέγουσιν.
\par 16 Η φαρέτρα αυτών είναι ως τάφος ανεωγμένος· είναι πάντες ισχυροί.
\par 17 Και θέλουσι κατατρώγει τον θερισμόν σου και τον άρτον σου, τον οποίον οι υιοί σου και αι θυγατέρες σου ήθελον τρώγει· θέλουσι κατατρώγει τα ποίμνιά σου και τας αγέλας σου· θέλουσι κατατρώγει τους αμπελώνάς σου και τας συκέας σου· θέλουσιν εξολοθρεύσει διά της ρομφαίας τας οχυράς πόλεις σου, επί τας οποίας συ ήλπιζες.
\par 18 Και όμως, εν ταις ημέραις εκείναις, λέγει Κύριος, δεν θέλω κάμει συντέλειαν εις εσάς.
\par 19 Και όταν είπητε, Διά τι έκαμε Κύριος ο Θεός ημών πάντα ταύτα εις ημάς; τότε θέλεις ειπεί προς αυτούς, Καθώς με εγκατελίπετε και εδουλεύσατε θεούς ξένους εν τη γη υμών, ούτω θέλετε δουλεύσει ξένους εν γη ουχί υμών.
\par 20 Αναγγείλατε τούτο προς τον οίκον Ιακώβ και κηρύξατε αυτό εν Ιούδα, λέγοντες;
\par 21 Ακούσατε τώρα τούτο, λαέ μωρέ και ασύνετε· οίτινες οφθαλμούς έχετε και δεν βλέπετε· ώτα έχετε και δεν ακούετε·
\par 22 εμέ δεν φοβείσθε; λέγει Κύριος· δεν θέλετε τρέμει ενώπιόν μου, όστις έθεσα την άμμον όριον της θαλάσσης κατά πρόσταγμα αιώνιον, και δεν θέλει υπερβή αυτό· και τα κύματα αυτής συνταράσσονται, όμως δεν θέλουσιν υπερισχύσει· και ηχούσιν, όμως δεν θέλουσιν υπερβή αυτό;
\par 23 Αλλ' ούτος ο λαός έχει καρδίαν στασιαστικήν και απειθή· απεστάτησαν και απήλθον.
\par 24 Και δεν είπον εν τη καρδία αυτών, Ας φοβηθώμεν τώρα Κύριον τον Θεόν ημών, όστις δίδει βροχήν πρώϊμον και όψιμον εν τω καιρώ αυτής· φυλάττει δι' ημάς τας διωρισμένας εβδομάδας του θερισμού.
\par 25 Αι ανομίαι σας απέστρεψαν ταύτα και αι αμαρτίαι σας εμπόδισαν το αγαθόν από σας.
\par 26 Διότι ευρέθησαν εν τω λαώ μου ασεβείς· έστησαν ενέδραν, καθώς ο στήνων βρόχια· θέτουσι παγίδα, συλλαμβάνουσιν ανθρώπους.
\par 27 Καθώς το κλωβίον είναι πλήρες πτηνών, ούτως οι οίκοι αυτών είναι πλήρεις δόλου· διά τούτο εμεγαλύνθησαν και επλούτησαν.
\par 28 Επαχύνθησαν, αποστίλβουσιν· υπερέβησαν μάλιστα τας πράξεις των ασεβών· δεν κρίνουσι την κρίσιν, την κρίσιν του ορφανού, και ευημερούσι· και το δίκαιον των πενήτων δεν κρίνουσι.
\par 29 Δεν θέλω κάμει διά ταύτα επίσκεψιν; λέγει Κύριος· η ψυχή μου δεν θέλει εκδικηθή εναντίον έθνους, τοιούτου;
\par 30 Έκπληξις και φρίκη έγειναν εν τη γη.
\par 31 Οι προφήται προφητεύουσι ψευδώς και οι ιερείς δεσπόζουσι διά μέσου αυτών· και ο λαός μου αγαπά ούτω· και τι θέλετε κάμει εις το μετά ταύτα;

\chapter{6}

\par Υιοί Βενιαμίν, φύγετε μετά σπουδής εκ μέσου της Ιερουσαλήμ και ηχήσατε σάλπιγγα εν Θεκουέ και υψώσατε σημείον εκ πυρός εν Βαιθ-ακκερέμ· διότι κακόν προκύπτει από βορρά και συντριμμός μέγας.
\par 2 Παρωμοίασα την θυγατέρα της Σιών με χαρίεσσαν και τρυφεράν γυναίκα.
\par 3 Οι ποιμένες και τα ποίμνια αυτών θέλουσιν ελθεί εις αυτήν· θέλουσι στήσει σκηνάς κύκλω εναντίον αυτής· θέλουσι ποιμαίνει έκαστος εν τω τόπω αυτού.
\par 4 Ετοιμάσατε πόλεμον κατ' αυτής· σηκώθητε και ας αναβώμεν εν μεσημβρία. Ουαί εις ημάς, διότι κλίνει η ημέρα, διότι εκτείνονται αι σκιαί της εσπέρας.
\par 5 Σηκώθητε και ας αναβώμεν διά νυκτός και ας καταστρέψωμεν τα παλάτια αυτής.
\par 6 Διότι ούτω λέγει ο Κύριος των δυνάμεων· Κατακόψατε δένδρα και υψώσατε περιχαρακώματα εναντίον της Ιερουσαλήμ. Αύτη είναι η πόλις, εφ' ην πρέπει να γείνη επίσκεψις· είναι όλη καταδυναστεία εν μέσω αυτής.
\par 7 Καθώς η πηγή αναβρύει τα ύδατα αυτής, ούτως αυτή αναβρύει την κακίαν αυτής· βία και αρπαγή ακούονται εν αυτή· ενώπιόν μου ακαταπαύστως είναι πόνος και πληγαί.
\par 8 Σωφρονίσθητι, Ιερουσαλήμ, μήποτε αποσυρθή η ψυχή μου από σού· μήποτε σε καταστήσω έρημον, γην ακατοίκητον.
\par 9 Ούτω λέγει ο Κύριος των δυνάμεων· θέλουσι σταφυλολογήσει ολοτελώς ως άμπελον τα υπόλοιπα του Ισραήλ· επίστρεψον την χείρα σου ως ο τρυγητής εις τα καλάθια.
\par 10 Προς τίνα θέλω λαλήσει και διαμαρτυρηθή, διά να ακούσωσιν; ιδού, το ωτίον αυτών είναι απερίτμητον και δεν δύνανται να ακούσωσιν· ιδού, ο λόγος του Κυρίου είναι προς αυτούς όνειδος· δεν ηδύνονται εις αυτόν.
\par 11 Διά τούτο είμαι πλήρης από θυμού του Κυρίου· απέκαμον κρατών εμαυτόν· θέλω εκχέει αυτόν επί τα νήπια έξωθεν και επί την σύναξιν των νέων ομού· διότι και ο ανήρ θέλει πιασθή μετά της γυναικός και ο ηλικιωμένος μετά του πλήρους ημερών.
\par 12 Και αι οικίαι αυτών θέλουσι περάσει εις άλλους, οι αγροί και αι γυναίκες ομού, διότι θέλω εκτείνει την χείρα μου επί τους κατοίκους της γης, λέγει Κύριος.
\par 13 Διότι από μικρού αυτών έως μεγάλου αυτών πας τις εδόθη εις την πλεονεξίαν· και από προφήτου έως ιερέως πας τις πράττει ψεύδος.
\par 14 Και ιάτρευσαν το σύντριμμα της θυγατρός του λαού μου επιπολαίως, λέγοντες, Ειρήνη, ειρήνη· και δεν υπάρχει ειρήνη.
\par 15 Μήπως ησχύνθησαν, ότε έπραξαν βδέλυγμα; μάλιστα παντελώς δεν ησχύνθησαν ουδέ ηρυθρίασαν· διά τούτο θέλουσι πέσει μεταξύ των πιπτόντων· όταν επισκεφθώ αυτούς, θέλουσιν απολεσθή, είπε Κύριος.
\par 16 Ούτω λέγει Κύριος· Στήτε επί τας οδούς και ιδέτε και ερωτήσατε περί των αιωνίων τρίβων, που είναι η αγαθή οδός, και περιπατείτε εν αυτή, και θέλετε ευρεί ανάπαυσιν εις τας ψυχάς σας. Αλλ' αυτοί είπον, δεν θέλομεν περιπατήσει εν αυτή.
\par 17 Και κατέστησα σκοπούς εφ' υμάς, λέγων, Ακούσατε τον ήχον της σάλπιγγος. Αλλ' είπον, δεν θέλομεν ακούσει.
\par 18 Διά τούτο ακούσατε, έθνη, και συ, συναγωγή, γνώρισον τι είναι μεταξύ αυτών.
\par 19 Άκουε, γή· ιδού, εγώ θέλω φέρει κακόν επί τον λαόν τούτον, τον καρπόν των διαλογισμών αυτών, διότι δεν επρόσεξαν εις τους λόγους μου και εις τον νόμον μου, αλλ' απέρριψαν αυτόν.
\par 20 Τι προς εμέ ο φερόμενος λίβανος από Σεβά και το από γης μακράς ευώδες κιννάμωμον; τα ολοκαυτώματά σας δεν είναι δεκτά ουδέ αι θυσίαι σας ευάρεστοι εις εμέ.
\par 21 Διά τούτο ούτω λέγει Κύριος· Ιδού, εγώ θέλω βάλει προσκόμματα έμπροσθεν του λαού τούτου και οι πατέρες και οι υιοί ομού θέλουσι προσκόψει επ' αυτά, ο γείτων και ο φίλος αυτού θέλουσιν απολεσθή.
\par 22 Ούτω λέγει ο Κύριος· Ιδού, λαός έρχεται από της γης του βορρά, και έθνος μέγα θέλει εγερθή από των άκρων της γης.
\par 23 Τόξον και λόγχην θέλουσι κρατεί· είναι σκληροί και ανίλεοι· φωνή αυτών εκεί ως θάλασσα, και επιβαίνουσιν επί ίππους, παρατεταγμένοι ως άνδρες εις πόλεμον εναντίον σου, θυγάτηρ της Σιών.
\par 24 Ηκούσαμεν την φήμην αυτών· αι χείρες ημών παρελύθησαν· στενοχωρία κατέλαβεν ημάς, ωδίνες ως τικτούσης.
\par 25 Μη εξέλθητε εις τον αγρόν και εν οδώ μη περιπατείτε· διότι η ρομφαία του εχθρού είναι τρόμος πανταχόθεν.
\par 26 Θυγάτηρ του λαού μου, περιζώσθητι σάκκον και κυλίσθητι εις στάκτην· πένθος μονογενούς κάμε εις σεαυτήν· θρήνησον πικρώς· διότι ο εξολοθρευτής θέλει ελθεί εξαίφνης εφ' ημάς.
\par 27 Σε έθεσα σκοπιάν, φρούριον μεταξύ του λαού μου, διά να γνωρίσης και να εξερευνήσης την οδόν αυτών.
\par 28 Πάντες είναι όλως απειθείς, περιπατούσι κακολογούντες· είναι χαλκός και σίδηρος· πάντες είναι διεφθαρμένοι.
\par 29 Το φυσητήριον εκαύθη· ο μόλυβδος κατηναλώθη υπό του πυρός· ο χωνευτής διαλύει εις μάτην· διότι οι κακοί δεν εχωρίσθησαν.
\par 30 Αργύριον αποδεδοκιμασμένον θέλουσιν ονομάσει αυτούς, διότι ο Κύριος απεδοκίμασεν αυτούς.

\chapter{7}

\par Ο λόγος ο γενόμενος προς τον Ιερεμίαν παρά Κυρίου, λέγων,
\par 2 Στήθι εν τη πύλη του οίκου του Κυρίου και κήρυξον εκεί τον λόγον τούτον και ειπέ, Ακούσατε τον λόγον του Κυρίου, πάντες οι Ιούδα, οι διά των πυλών τούτων εισερχόμενοι διά να προσκυνήτε τον Κύριον.
\par 3 Ούτω λέγει ο Κύριος των δυνάμεων, ο Θεός του Ισραήλ· Διορθώσατε τας οδούς σας και τας πράξεις σας, και θέλω σας στερεώσει εν τω τόπω τούτω.
\par 4 Μη πεποίθατε εις λόγους ψευδείς, λέγοντες, Ο ναός του Κυρίου, ο ναός του Κυρίου, ο ναός του Κυρίου είναι ούτος.
\par 5 Διότι εάν αληθώς διορθώσητε τας οδούς σας και τας πράξεις σας· εάν εντελώς εκτελέσητε κρίσιν αναμέσον ανθρώπου και του πλησίον αυτού·
\par 6 εάν δεν καταδυναστεύητε τον ξένον, τον ορφανόν και την χήραν, και δεν χύνητε αίμα αθώον εν τω τόπω τούτω μηδέ περιπατήτε οπίσω ξένων θεών εις φθοράν σας·
\par 7 τότε θέλω σας κάμει να κατοικήτε εν τω τόπω τούτω, εν τη γη την οποίαν έδωκα εις τους πατέρας σας εις αιώνα αιώνος.
\par 8 Ιδού, σεις πεποίθατε εις λόγους ψευδείς, εκ των οποίων δεν θέλετε ωφεληθή.
\par 9 Κλέπτετε, φονεύετε και μοιχεύετε και ομνύετε ψευδώς και θυμιάζετε εις τον Βάαλ και περιπατείτε οπίσω άλλων θεών, τους οποίους δεν γνωρίζετε·
\par 10 έπειτα έρχεσθε και ίστασθε ενώπιόν μου εν τω οίκω τούτω, εφ' ον εκλήθη το όνομά μου, και λέγετε, Ηλευθερώθημεν, διά να κάμνητε πάντα ταύτα τα βδελύγματα;
\par 11 Σπήλαιον ληστών έγεινεν ενώπιόν σας ο οίκος ούτος, εφ' ον εκλήθη το όνομά μου; ιδού, αυτός εγώ είδον ταύτα, λέγει Κύριος.
\par 12 Αλλ' υπάγετε τώρα εις τον τόπον μου τον εν Σηλώ, όπου έθεσα το όνομά μου κατ' αρχάς, και ιδέτε τι έκαμον εις αυτόν διά την κακίαν του λαού μου Ισραήλ.
\par 13 Και τώρα, επειδή επράξατε πάντα ταύτα τα έργα, λέγει Κύριος, και ελάλησα προς εσάς, εγειρόμενος πρωΐ και λαλών, και δεν ηκούσατε· και σας έκραξα και δεν απεκρίθητε·
\par 14 διά τούτο θέλω κάμει εις τον οίκον, εφ' ον εκλήθη το όνομά μου, εις τον οποίον σεις θαρρείτε, και εις τον τόπον τον οποίον έδωκα εις εσάς και εις τους πατέρας σας, καθώς έκαμα εις την Σηλώ·
\par 15 και θέλω σας απορρίψει από του προσώπου μου, καθώς απέρριψα πάντας τους αδελφούς σας, άπαν το σπέρμα του Εφραΐμ.
\par 16 Διά τούτο συ μη προσεύχου υπέρ του λαού τούτου και μη ύψονε φωνήν ή δέησιν υπέρ αυτών μηδέ μεσίτευε προς εμέ· διότι δεν θέλω σου εισακούσει.
\par 17 Δεν βλέπεις τι κάμνουσιν αυτοί εν ταις πόλεσι του Ιούδα και εν ταις οδοίς της Ιερουσαλήμ;
\par 18 Οι υιοί συλλέγουσι ξύλα και οι πατέρες ανάπτουσι το πυρ και αι γυναίκες ζυμόνουσι την ζύμην, διά να κάμωσι πέμματα εις την βασίλισσαν του ουρανού και να κάμωσι, σπονδάς εις άλλους θεούς, διά να με παροξύνωσι.
\par 19 Μήπως εμέ παροξύνουσι; λέγει Κύριος· ουχί εαυτους προς καταισχύνην των προσώπων αυτών;
\par 20 Διά τούτο ούτω λέγει Κύριος ο Θεός· Ιδού, η οργή μου και ο θυμός μου εκχέονται επί τον τόπον τούτον, επί άνθρωπον και επί κτήνος και επί τα δένδρα του αγρού και επί τον καρπόν της γής· και θέλει εξαφθή και δεν θέλει σβεσθή.
\par 21 Ούτω λέγει ο Κύριος των δυνάμεων, ο Θεός του Ισραήλ· προσθέσατε τα ολοκαυτώματά σας εις τας θυσίας σας και φάγετε κρέας.
\par 22 Διότι δεν ελάλησα προς τους πατέρας σας ουδέ έδωκα εις αυτούς εντολάς, καθ' ην ημέραν εξήγαγον αυτούς εκ γης Αιγύπτου, περί ολοκαυτωμάτων και θυσιών·
\par 23 αλλά τον λόγον τούτον προσέταξα εις αυτούς, λέγων, Ακούσατε την φωνήν μου και θέλω είσθαι Θεός σας, και σεις θέλετε είσθαι λαός μου· και περιπατείτε εν πάσαις ταις οδοίς, τας οποίας διώρισα εις εσάς, διά να ευημερήτε·
\par 24 πλην δεν ήκουσαν ουδέ έκλιναν το ωτίον αυτών, αλλά περιεπάτησαν εν ταις βουλαίς, εν ταις ορέξεσι της πονηράς αυτών καρδίας, και υπήγον εις τα οπίσω και ουχί εις τα εμπρός.
\par 25 Αφ' ης ημέρας εξήλθον οι πατέρες σας εκ γης Αιγύπτου έως της ημέρας ταύτης, εξαπέστειλα προς εσάς πάντας τους δούλους μου τους προφήτας, καθ' ημέραν εγειρόμενος πρωΐ και αποστέλλων·
\par 26 πλην δεν μου υπήκουσαν ουδέ έκλιναν το ωτίον αυτών, αλλ' εσκλήρυναν τον τράχηλον αυτών, έπραξαν χειρότερα των πατέρων αυτών.
\par 27 Διά τούτο θέλεις λαλήσει προς αυτούς πάντας τούτους τους λόγους και δεν θέλουσι σε ακούσει· και θέλεις φωνάξει προς αυτούς και δεν θέλουσι σοι αποκριθή.
\par 28 Θέλεις όμως ειπεί προς αυτούς, Τούτο είναι το έθνος, το οποίον δεν ακούει την φωνήν Κυρίου του Θεού αυτού ουδέ δέχεται παιδείαν· η αλήθεια εξέλιπε και εχάθη από του στόματος αυτών.
\par 29 Κούρευσον την κεφαλήν σου, Ιερουσαλήμ, και απόρριψον τας τρίχας, και ανάλαβε θρήνον επί τους υψηλούς τόπους· διότι ο Κύριος απέρριψε και εγκατέλιπε την γενεάν, κατά της οποίας ωργίσθη.
\par 30 Διότι οι υιοί Ιούδα έπραξαν πονηρά ενώπιόν μου, λέγει Κύριος· έθεσαν τα βδελύγματα αυτών εν τω οίκω εφ' ον εκλήθη το όνομά μου, διά να μιάνωσιν αυτόν.
\par 31 Και ωκοδόμησαν τους υψηλούς τόπους του Τοφέθ, όστις είναι εν τη φάραγγι του υιού Εννόμ, διά να καίωσι τους υιούς αυτών και τας θυγατέρας αυτών εν πυρί· το οποίον δεν προσέταξα ουδέ ανέβη επί την καρδίαν μου.
\par 32 Διά τούτο, ιδού, έρχονται ημέραι, λέγει Κύριος, καθ' ας δεν θέλει ονομάζεσθαι πλέον Τοφέθ ουδέ Φάραγξ του υιού Εννόμ, αλλ' η φάραγξ της σφαγής· διότι θέλουσι θάπτει εν Τοφέθ, εωσού να μη υπάρχη τόπος.
\par 33 Και τα πτώματα του λαού τούτου θέλουσιν είσθαι τροφή εις τα πετεινά του ουρανού και εις τα θηρία της γής· και δεν θέλει είσθαι ο εκφοβίζων.
\par 34 Και θέλω παύσει από των πόλεων του Ιούδα και από των οδών της Ιερουσαλήμ την φωνήν της χαράς και την φωνήν της ευφροσύνης, την φωνήν του νυμφίου και την φωνήν της νύμφης· διότι η γη θέλει κατασταθή έρημος.

\chapter{8}

\par Εν τω καιρώ εκείνω, λέγει Κύριος, θέλουσιν εκρίψει τα οστά των βασιλέων του Ιούδα και τα οστά των αρχόντων αυτού και τα οστά των ιερέων, και τα οστά των προφητών και τα οστά των κατοίκων της Ιερουσαλήμ, από των τάφων αυτών·
\par 2 και θέλουσιν απλώσει αυτά κατέναντι του ηλίου και της σελήνης και κατέναντι πάσης της στρατιάς του ουρανού, τα οποία ηγάπησαν και τα οποία ελάτρευσαν και οπίσω των οποίων περιεπάτησαν και τα οποία εξεζήτησαν και τα οποία προσεκύνησαν· δεν θέλουσι συναχθή ουδέ ταφή· θέλουσιν είσθαι διά κοπρίαν επί του προσώπου της γης.
\par 3 Και ο θάνατος θέλει είσθαι προτιμότερος παρά την ζωήν εις άπαν το υπόλοιπον των εναπολειφθέντων από εκείνης της πονηράς γενεάς, όσοι ήθελον μείνει εν πάσι τοις τόποις, όπου ήθελον εξώσει αυτούς, λέγει ο Κύριος των δυνάμεων.
\par 4 Και θέλεις ειπεί προς αυτούς, Ούτω λέγει Κύριος· Εάν τις πέση, δεν σηκόνεται; εάν τις εκκλίνη, δεν θέλει επιστρέψει;
\par 5 Διά τι ο λαός ούτος της Ιερουσαλήμ εστράφη παντοτεινήν στροφήν; προσηλόνονται εις την απάτην, αρνούνται να επιστρέψωσιν.
\par 6 Ηκροάσθην και ήκουσα, αλλά δεν ελάλησαν εν ευθύτητι· δεν υπάρχει ουδείς μετανοών από της κακίας αυτού, λέγων, Τι έπραξα; πας τις εστράφη εις την οδόν αυτού, ως ίππος εφορμών εις την μάχην.
\par 7 Και αυτός ο πελαργός εν τω ουρανώ γνωρίζει τους διωρισμένους καιρούς αυτού· και η τρυγών και ο γερανός και η χελιδών φυλάττουσι τον καιρόν της ελεύσεως αυτών· ο δε λαός μου δεν γνωρίζει την κρίσιν του Κυρίου.
\par 8 Πως λέγετε, Είμεθα σοφοί, και ο νόμος του Κυρίου είναι μεθ' ημών; ιδού, βεβαίως εις μάτην έγεινε τούτο· ο κάλαμος των γραμματέων είναι ψευδής.
\par 9 Οι σοφοί κατησχύνθησαν, επτοήθησαν και συνελήφθησαν, διότι απέρριψαν τον λόγον του Κυρίου· και ποία σοφία είναι εν αυτοίς;
\par 10 Διά τούτο θέλω δώσει τας γυναίκας αυτών εις άλλους, τους αγρούς αυτών εις εκείνους οίτινες θέλουσι κληρονομήσει αυτούς· διότι πας τις από μικρού έως μεγάλου εδόθη εις πλεονεξίαν· από προφήτου έως ιερέως, πας τις πράττει ψεύδος.
\par 11 Διότι ιάτρευσαν το σύντριμμα της θυγατρός του λαού μου επιπολαίως, λέγοντες, Ειρήνη, ειρήνη· και δεν υπάρχει ειρήνη.
\par 12 Μήπως ησχύνθησαν ότι έπραξαν βδέλυγμα; μάλιστα ουδόλως δεν ησχύνθησαν ουδέ ηρυθρίασαν· διά τούτο θέλουσι πέσει μεταξύ των πιπτόντων· εν τω καιρώ της επισκέψεως αυτών θέλουσιν απολεσθή, είπε Κύριος.
\par 13 Εξάπαντος θέλω αναλώσει αυτούς, λέγει Κύριος· δεν θέλουσιν είσθαι σταφυλαί εν τη αμπέλω ουδέ σύκα εν τη συκέα και το φύλλον θέλει μαρανθή· και τα αγαθά, τα οποία έδωκα εις αυτούς, θέλουσι φύγει απ' αυτών.
\par 14 Διά τι καθήμεθα; συνάχθητε και ας εισέλθωμεν εις τας οχυράς πόλεις και ας κατασιωπήσωμεν εκεί, διότι Κύριος ο Θεός ημών κατεσιώπησεν ημάς και επότισεν ημάς ύδωρ χολής, επειδή ημαρτήσαμεν εις τον Κύριον.
\par 15 Επροσμείναμεν ειρήνην, αλλ' ουδέν αγαθόν· καιρόν θεραπείας, αλλ' ιδού, ταραχή.
\par 16 Το φρύαγμα των ίππων αυτού ηκούσθη από Δάν· πάσα η γη εσείσθη από του ήχου του χρεμετισμού των ρωμαλέων ίππων αυτού· διότι ήλθον και κατέφαγον την γην και το πλήρωμα αυτής· την πόλιν και τους κατοικούντας εν αυτή·
\par 17 διότι, ιδού, εγώ εξαποστέλλω προς εσάς όφεις, βασιλίσκους, οίτινες δεν θέλουσι γοητεύεσθαι αλλά θέλουσι σας δαγκάνει, λέγει Κύριος.
\par 18 Ηθέλησα να παρηγορηθώ από της λύπης, αλλ' η καρδία μου είναι εκλελυμένη εντός μου.
\par 19 Ιδού, φωνή κραυγής της θυγατρός του λαού μου από γης μακράς. Δεν είναι ο Κύριος εν Σιών; ο βασιλεύς αυτής δεν είναι εν αυτή; Διά τι με παρώργισαν με τα γλυπτά αυτών, με ματαιότητας ξένας;
\par 20 Παρήλθεν ο θερισμός, ετελείωσε το θέρος, και ημείς δεν εσώθημεν.
\par 21 Διά το σύντριμμα της θυγατρός του λαού μου επληγώθην, είμαι εις πένθος, έκπληξις με κατέλαβε.
\par 22 Δεν είναι βάλσαμον εν Γαλαάδ; δεν είναι εκεί ιατρός; διά τι λοιπόν η θυγάτηρ του λαού μου δεν ανέλαβε την υγείαν αυτής;

\chapter{9}

\par Είθε να ήτο η κεφαλή μου ύδατα και οι οφθαλμοί μου πηγή δακρύων, διά να κλαίω ημέραν και νύκτα τους πεφονευμένους της θυγατρός του λαού μου.
\par 2 Είθε να είχον εν τη ερήμω κατάλυμα οδοιπόρων, διά να εγκαταλείψω τον λαόν μου και να απέλθω απ' αυτών· διότι πάντες είναι μοιχοί, άθροισμα απίστων.
\par 3 Ενέτειναν και την γλώσσαν αυτών ως τόξον ψεύδους· και ίσχυσαν επί της γης, ουχί υπέρ της αληθείας· διότι προχωρούσιν από κακίας εις κακίαν και εμέ δεν γνωρίζουσι, λέγει Κύριος.
\par 4 Φυλάττεσθε έκαστος από του πλησίον αυτού και επ' ουδένα αδελφόν μη πεποίθατε· διότι πας αδελφός θέλει πάντοτε υποσκελίζει και πας πλησίον θέλει περιπατεί εν δολιότητι.
\par 5 Και θέλουσιν απατά έκαστος τον πλησίον αυτού και δεν θέλουσι λαλεί την αλήθειαν· εδίδαξαν την γλώσσαν αυτών να λαλή ψεύδη, αποκάμνουσι πράττοντες ανομίαν.
\par 6 Η κατοικία σου είναι εν μέσω δολιότητος· εν τη δολιότητι αρνούνται να με γνωρίσωσι, λέγει Κύριος.
\par 7 Διά τούτο ούτω λέγει ο Κύριος των δυνάμεων· Ιδού, θέλω βάλει αυτούς εν χωνευτηρίω και θέλω δοκιμάσει αυτούς· διότι πως θέλω κάμει ένεκεν της θυγατρός του λαού μου;
\par 8 Η γλώσσα αυτών είναι βέλος εξακοντιζόμενον· λαλεί δόλια· έκαστος λαλεί ειρηνικά διά του στόματος αυτού προς τον πλησίον αυτού, πλην εν τη καρδία αυτού στήνει ενέδραν κατ' αυτού.
\par 9 Δεν θέλω επισκεφθή αυτούς διά ταύτα; λέγει Κύριος· η ψυχή μου δεν θέλει εκδικηθή εναντίον έθνους, τοιούτου;
\par 10 Διά τα όρη θέλω αναλάβει κλαυθμόν και θρήνον και διά τας βοσκάς της ερήμου οδυρμόν, διότι ηφανίσθησαν, ώστε δεν υπάρχει άνθρωπος διαβαίνων, ουδέ ακούεται φωνή ποιμνίου· από του πτηνού του ουρανού έως του κτήνους, έφυγον, απήλθον.
\par 11 Και θέλω καταστήσει την Ιερουσαλήμ εις σωρούς, κατοικίαν θώων· και τας πόλεις του Ιούδα θέλω κάμει ερήμωσιν, ώστε να μη υπάρχη ο κατοικών.
\par 12 Τις είναι ο άνθρωπος ο σοφός, όστις δύναται να εννοήση τούτο; και προς τον οποίον ελάλησε το στόμα του Κυρίου, διά να αναγγείλη αυτό, διά τι η γη εχάθη, ηφανίσθη ως έρημος, ώστε να μη υπάρχη ο διαβαίνων;
\par 13 Και είπε Κύριος, διότι εγκατέλιπον τον νόμον μου, τον οποίον έθεσα έμπροσθεν αυτών και δεν υπήκουσαν εις την φωνήν μου και δεν περιεπάτησαν εν αυτώ·
\par 14 αλλά περιεπάτησαν οπίσω της ορέξεως της καρδίας αυτών και οπίσω των Βααλείμ, τα οποία οι πατέρες αυτών εδίδαξαν αυτούς·
\par 15 διά τούτο, ούτω λέγει ο Κύριος των δυνάμεων, ο Θεός του Ισραήλ· Ιδού, εγώ θέλω θρέψει αυτούς, τον λαόν τούτον· με αψίνθιον και ύδωρ χολής θέλω ποτίσει αυτούς·
\par 16 και θέλω διασκορπίσει αυτούς εν τοις έθνεσι, τα οποία αυτοί και οι πατέρες αυτών δεν εγνώρισαν· και θέλω αποστείλει την μάχαιραν οπίσω αυτών, εωσού αναλώσω αυτούς.
\par 17 Ούτω λέγει ο Κύριος των δυνάμεων. Συλλογίσθητε και καλέσατε τας θρηνούσας να έλθωσι· και αποστείλατε διά τας σοφάς να έλθωσι·
\par 18 και ας σπεύσωσι και ας αναλάβωσιν οδυρμόν περί ημών και ας καταβιβάσωσιν οι οφθαλμοί ημών δάκρυα και τα βλέφαρα ημών ας ρεύσωσιν ύδατα.
\par 19 Διότι φωνή θρήνου ηκούσθη από Σιών, Πως απωλέσθημεν· κατησχύνθημεν σφόδρα, διότι εγκατελίπομεν την γην, διότι αι κατοικίαι ημών εξέρριψαν ημάς.
\par 20 Ακούσατε λοιπόν, γυναίκες, τον λόγον του Κυρίου, και ας δεχθή το ωτίον σας τον λόγον του στόματος αυτού, και διδάξατε τας θυγατέρας σας οδυρμόν και εκάστη την πλησίον αυτής θρήνον.
\par 21 Διότι θάνατος ανέβη διά των θυρίδων ημών, εισήλθεν εις τα παλάτια ημών, διά να εκκόψη τα νήπια από των οδών τους νέους από των πλατειών.
\par 22 Ειπέ, Ούτω λέγει Κύριος· Και τα πτώματα των ανθρώπων θέλουσι ριφθή ως κοπρία επί πρόσωπον αγρού και ως δράγμα οπίσω θεριστού, και δεν θέλει υπάρχει ο συνάγων.
\par 23 Ούτω λέγει Κύριος· Ας μη καυχάται ο σοφός εις την σοφίαν αυτού, και ας μη καυχάται ο δυνατός εις την δύναμιν αυτού, ας μη καυχάται ο πλούσιος εις τον πλούτον αυτού·
\par 24 αλλ' ο καυχώμενος ας καυχάται εις τούτο, ότι εννοεί και γνωρίζει εμέ, ότι εγώ είμαι ο Κύριος, ο ποιών έλεος, κρίσιν και δικαιοσύνην επί της γής· επειδή εις ταύτα ευαρεστούμαι, λέγει Κύριος.
\par 25 Ιδού, έρχονται ημέραι, λέγει Κύριος, και θέλω κάμει επίσκεψιν επί πάντας τους περιτετμημένους μετά των απεριτμήτων·
\par 26 επί την Αίγυπτον και επί τον Ιούδαν και επί τον Εδώμ και επί τους υιούς Αμμών και επί τον Μωάβ και επί πάντας τους περικείροντας την κόμην, τους κατοικούντας εν τη ερήμω· διότι πάντα τα έθνη είναι απερίτμητα και πας ο οίκος Ισραήλ απερίτμητος την καρδίαν.

\chapter{10}

\par Ακούσατε τον λόγον, τον οποίον ο Κύριος λαλεί προς εσάς, οίκος Ισραήλ.
\par 2 Ούτω λέγει Κύριος· Μη μανθάνετε την οδόν των εθνών και εις τα σημεία του ουρανού μη πτοείσθε, διότι τα έθνη πτοούνται εις αυτά.
\par 3 Διότι τα νόμιμα των λαών είναι μάταια, διότι κόπτουσι ξύλον εκ του δάσους, έργον χειρών τέκτονος με τον πέλεκυν.
\par 4 Καλλωπίζουσιν αυτό με άργυρον και χρυσόν· στερεόνουσιν αυτό με καρφία και με σφύρας, διά να μη κινήται.
\par 5 Είναι όρθια ως φοίνιξ, αλλά δεν λαλούσιν· έχουσι χρείαν να βαστάζωνται, διότι δεν δύνανται να περιπατήσωσι. Μη φοβείσθε αυτά· διότι δεν δύνανται να κακοποιήσωσιν, ουδέ είναι δυνατόν εις αυτά να αγαθοποιήσωσι.
\par 6 Δεν υπάρχει όμοιός σου, Κύριε· είσαι μέγας και μέγα το όνομά σου εν δυνάμει.
\par 7 Τις δεν ήθελε σε φοβείσθαι, Βασιλεύ των εθνών; διότι εις σε ανήκει τούτο, διότι μεταξύ πάντων των σοφών των εθνών και εν πάσι τοις βασιλείοις αυτών δεν υπάρχει όμοιός σου.
\par 8 Αλλ' είναι παντάπασι κτηνώδεις και άφρονες· διδασκαλία ματαιοτήτων είναι το ξύλον.
\par 9 Αργύριον κεχυμένον εις πλάκας εφέρθη από Θαρσείς και χρυσίον από Ουφάζ, έργον τεχνίτου και χειρών χρυσοχόου· κυανούν και πορφυρούν είναι το ένδυμα αυτών· έργον σοφών πάντα ταύτα.
\par 10 Αλλ' ο Κύριος είναι Θεός αληθινός, είναι Θεός ζων και βασιλεύς αιώνιος· εν τη οργή αυτού η γη θέλει σεισθή και τα έθνη δεν θέλουσιν ανθέξει εις την αγανάκτησιν αυτού.
\par 11 Ούτω θέλετε ειπεί προς αυτούς· οι θεοί, οίτινες δεν έκαμον τον ουρανόν και την γην, θέλουσιν αφανισθή από της γης και υποκάτωθεν του ουρανού τούτου.
\par 12 Αυτός εποίησε την γην διά της δυνάμεως αυτού, εστερέωσε την οικουμένην εν τη σοφία αυτού, και εξέτεινε τους ουρανούς εν τη συνέσει αυτού.
\par 13 Όταν εκπέμπη την φωνήν αυτού, συνίσταται πλήθος υδάτων εν ουρανοίς, και ανάγει νεφέλας από των άκρων της γής· κάμνει αστραπάς διά βροχήν και εξάγει άνεμον από των θησαυρών αυτού.
\par 14 Πας άνθρωπος εμωράνθη υπό της γνώσεως αυτού, πας χωνευτής κατησχύνθη υπό των γλυπτών· διότι ψεύδος είναι το χωνευτόν αυτού και πνοή δεν υπάρχει εν αυτώ.
\par 15 Ματαιότης ταύτα, έργον πλάνης· εν τω καιρώ της επισκέψεως αυτών θέλουσιν απολεσθή.
\par 16 Η μερίς του Ιακώβ δεν είναι ως αυτά· διότι αυτός είναι ο πλάσας τα πάντα, και ο Ισραήλ είναι η ράβδος της κληρονομίας αυτού· Κύριος των δυνάμεων το όνομα αυτού.
\par 17 Σύναξον εκ της γης την περιουσίαν σου, συ, η κατοικούσα εν οχυρώματι.
\par 18 Διότι ούτω λέγει Κύριος· Ιδού, εγώ θέλω εκσφενδονίσει τους κατοίκους της γης ταύτην την φοράν και θέλω στενοχωρήσει αυτούς, ώστε να εύρωσιν αυτό.
\par 19 Ουαί εις εμέ διά την θραύσίν μου· η πληγή μου είναι οδυνηρά. αλλ' εγώ είπα, Τωόντι τούτο είναι πόνος μου, και πρέπει να υποφέρω αυτόν.
\par 20 Η σκηνή μου ηρημώθη και πάντα τα σχοινία μου κατεκόπησαν· οι υιοί μου εχωρίσθησαν απ' εμού και δεν υπάρχουσι· δεν υπάρχει πλέον ο εκτείνων την σκηνήν μου και σηκόνων τα παραπετάσματά μου.
\par 21 Επειδή οι ποιμένες εμωράνθησαν και τον Κύριον δεν εξεζήτησαν, διά τούτο δεν θέλουσιν ευοδωθή και πάντα τα ποίμνια αυτών θέλουσι διασκορπισθή.
\par 22 Ιδού, ήχος θορύβου έρχεται και συγκίνησις μεγάλη εκ της γης του βορρά, διά να καταστήση τας πόλεις του Ιούδα ερήμωσιν, κατοικίαν θώων.
\par 23 Κύριε, γνωρίζω ότι η οδός του ανθρώπου δεν εξαρτάται απ' αυτού· του περιπατούντος ανθρώπου δεν είναι το να κατευθύνη τα διαβήματα αυτού.
\par 24 Κύριε, παίδευσόν με, πλην εν κρίσει· μη εν τω θυμώ σου, διά να μη με συντελέσης.
\par 25 Έκχεε τον θυμόν σου επί τα έθνη τα μη γνωρίζοντά σε, και επί γενεάς, αίτινες δεν επικαλούνται το όνομά σου· διότι κατέφαγον τον Ιακώβ και κατηνάλωσαν αυτόν και κατέφθειραν αυτόν και ηρήμωσαν την κατοικίαν αυτού.

\chapter{11}

\par Ο λόγος, ο γενόμενος προς Ιερεμίαν παρά Κυρίου, λέγων,
\par 2 Ακούσατε τους λόγους της διαθήκης ταύτης και λαλήσατε προς τους άνδρας Ιούδα και προς τους κατοίκους της Ιερουσαλήμ·
\par 3 και ειπέ προς αυτούς, Ούτω λέγει Κύριος ο Θεός του Ισραήλ. Επικατάρατος ο άνθρωπος, όστις δεν υπακούει εις τους λόγους της διαθήκης ταύτης,
\par 4 την οποίαν προσέταξα εις τους πατέρας υμών, καθ' ην ημέραν εξήγαγον αυτούς εκ γης Αιγύπτου, εκ της καμίνου της σιδηράς, λέγων, Ακούσατε της φωνής μου και πράττετε αυτά, κατά πάντα όσα προσέταξα εις εσάς· και θέλετε είσθαι λαός μου, και εγώ θέλω είσθαι Θεός υμών·
\par 5 διά να εκπληρώσω τον όρκον, τον οποίον ώμοσα προς τους πατέρας υμών, να δώσω εις αυτούς γην ρέουσαν γάλα και μέλι, ως εν τη ημέρα ταύτη. Τότε απεκρίθην και είπα, Αμήν, Κύριε.
\par 6 Και ο Κύριος είπε προς εμέ, Διακήρυξον πάντας τους λόγους τούτους εν ταις πόλεσι του Ιούδα και εν ταις οδοίς της Ιερουσαλήμ, λέγων, Ακούσατε τους λόγους της διαθήκης ταύτης και πράττετε αυτούς.
\par 7 Διότι ρητώς διεμαρτυρήθην προς τους πατέρας υμών, καθ' ην ημέραν ανεβίβασα αυτούς εκ γης Αιγύπτου μέχρι της σήμερον, εγειρόμενος πρωΐ και διαμαρτυρόμενος, λέγων, Ακούσατε της φωνής μου.
\par 8 Αλλά δεν ήκουσαν και δεν έκλιναν το ωτίον αυτών, αλλά περιεπάτησαν έκαστος εν ταις ορέξεσι της πονηράς αυτών καρδίας· διά τούτο θέλω φέρει επ' αυτούς πάντας τους λόγους της διαθήκης ταύτης, την οποίαν προσέταξα να πράττωσι, αλλά δεν έπραξαν.
\par 9 Και είπε Κύριος προς εμέ, Συνωμοσία ευρέθη μεταξύ των ανδρών Ιούδα και μεταξύ των κατοίκων της Ιερουσαλήμ.
\par 10 Επέστρεψαν εις τας αδικίας των προπατόρων αυτών, οίτινες δεν ηθέλησαν να ακούσωσι τους λόγους μου· και αυτοί υπήγαν οπίσω άλλων θεών, διά να λατρεύωσιν αυτούς· ο οίκος Ισραήλ και ο οίκος Ιούδα ηθέτησαν την διαθήκην μου, την οποίαν έκαμα προς τους πατέρας αυτών.
\par 11 Διά τούτο ούτω λέγει Κύριος· Ιδού, θέλω φέρει επ' αυτούς κακόν, εκ του οποίου δεν θέλουσι δυνηθή να εξέλθωσι· και θέλουσι βοήσει προς εμέ και δεν θέλω εισακούσει αυτούς.
\par 12 Τότε αι πόλεις του Ιούδα και οι κάτοικοι της Ιερουσαλήμ θέλουσιν υπάγει και θέλουσι βοήσει προς τους θεούς, εις τους οποίους θυμιάζουσι· πλην δεν θέλουσι σώσει εαυτούς παντελώς εν καιρώ της ταλαιπωρίας αυτών.
\par 13 Διότι κατά τον αριθμόν των πόλεών σου ήσαν οι θεοί σου, Ιούδα· και κατά τον αριθμόν των οδών της Ιερουσαλήμ ανηγείρατε βωμούς εις τα αισχρά, βωμούς διά να θυμιάζητε εις τον Βάαλ.
\par 14 Διά τούτο συ μη προσεύχου υπέρ του λαού τούτου και μη ύψωνε φωνήν ή δέησιν υπέρ αυτών· διότι εγώ δεν θέλω εισακούσει, όταν κράζωσι προς εμέ εν καιρώ της ταλαιπωρίας αυτών.
\par 15 Τι έχει να κάμη η ηγαπημένη μου εν τω οίκω μου, αφού έπραξεν ασέλγειαν με πολλούς, και το κρέας το άγιον αφηρέθη από σου; όταν πράττης το κακόν, τότε ευφραίνεσαι.
\par 16 Ο Κύριος εκάλεσε το όνομά σου, Ελαίαν αειθαλή, ώραίαν, καλλίκαρπον· μετ' ήχου θορύβου μεγάλου εξήφθη πυρ επ' αυτήν και οι κλάδοι αυτής συνεθλάσθησαν.
\par 17 Διότι ο Κύριος των δυνάμεων, όστις σε εφύτευσεν, επρόφερε κακόν εναντίον σου, διά την κακίαν του οίκου Ισραήλ και του οίκου Ιούδα, την οποίαν έπραξαν καθ' εαυτών, ώστε να με παροργίσωσι θυμιάζοντες εις τον Βάαλ.
\par 18 Και ο Κύριος έδωκεν εις εμέ γνώσιν και εγνώρισα· τότε έδειξας εις εμέ τας πράξεις αυτών.
\par 19 Αλλ' εγώ ήμην ως αρνίον άκακον φερόμενον εις σφαγήν· και δεν ενόησα ότι συνεβουλεύθησαν βουλάς εναντίον μου, λέγοντες, Ας καταστρέψωμεν το δένδρον μετά του καρπού αυτού και ας εκκόψωμεν αυτόν από της γης των ζώντων, ώστε το όνομα αυτού να μη μνημονευθή πλέον.
\par 20 Αλλ' ω Κύριε των δυνάμεων, ο κρίνων δικαίως, ο δοκιμάζων τους νεφρούς και την καρδίαν, ας ίδω την εκδίκησίν σου επ' αυτούς, διότι προς σε εφανέρωσα την δίκην μου.
\par 21 Διά τούτο ούτω λέγει Κύριος περί των ανδρών της Αναθώθ, οίτινες ζητούσι την ζωήν σου, λέγοντες, Μη προφητεύσης εν τω ονόματι του Κυρίου, διά να μη αποθάνης υπό τας χείρας ημών·
\par 22 διά τούτο ούτω λέγει Κύριος των δυνάμεων· Ιδού, θέλω επισκεφθή αυτούς· οι νέοι θέλουσιν αποθάνει εν μαχαίρα· οι υιοί αυτών αι θυγατέρες αυτών θέλουσι τελευτήσει υπό πείνης·
\par 23 και δεν θέλει μείνει υπόλοιπον εξ αυτών· διότι θέλω φέρει κακόν επί τους άνδρας της Αναθώθ, εν τω ενιαυτώ της επισκέψεως αυτών.

\chapter{12}

\par Δίκαιος είσαι Κύριε, όταν δικολογώμαι μετά σού· πλην ας διαλεχθώ μετά σου περί των κρίσεών σου· διατί η οδός των ασεβών ευοδούται; διά τι ευημερούσι πάντες οι φερόμενοι απίστως;
\par 2 Εφύτευσας αυτούς, μάλιστα ερριζώθησαν· αυξάνουσι, μάλιστα καρποφορούσι· συ είσαι πλησίον του στόματος αυτών και μακράν από των νεφρών αυτών.
\par 3 Αλλά συ, Κύριε, με γνωρίζεις· με είδες και εδοκίμασας την καρδίαν μου ενώπιόν σου· σύρε αυτούς ως πρόβατα διά σφαγήν και ετοίμασον αυτούς διά την ημέραν της σφαγής.
\par 4 Έως πότε θέλει πενθεί η γη, και ο χόρτος παντός αγρού θέλει ξηραίνεσθαι διά την κακίαν των κατοιούντων εν αυτή; Ηφανίσθησαν τα κτήνη και τα πτηνά, διότι είπον, δεν θέλει ιδεί τα έσχατα ημών.
\par 5 Εάν τρέξης μετά των πεζών και σε κάμωσι να ατονήσης, τότε πως θέλεις αντιπαραταχθή προς τους ίππους; και εάν απέκαμες εν τη γη της ειρήνης, εφ' ην ήλπιζες, τότε πως θέλεις κάμει εις το φρύαγμα του Ιορδάνου;
\par 6 Διότι και οι αδελφοί σου και ο οίκος του πατρός σου και αυτοί εφέρθησαν απίστως προς σέ· ναι, αυτοί εβόησαν όπισθέν σου μεγαλοφώνως· μη πιστεύσης αυτούς, και αν λαλήσωσι καλά προς σε.
\par 7 Εγκατέλιπον τον οίκόν μου, αφήκα την κληρονομίαν μου, έδωκα την ηγαπημένην της ψυχής μου εις τας χείρας των εχθρών αυτής.
\par 8 Η κληρονομία μου έγεινεν εις εμέ ως λέων εν δρυμώ· εξέπεμψε την φωνήν αυτής εναντίον μου· διά τούτο εμίσησα αυτήν.
\par 9 Η κληρονομία μου είναι εις εμέ όρνεον αρπακτικόν, τα όρνεα κύκλω είναι εναντίον αυτής· έλθετε, συνάχθητε, πάντα τα θηρία του αγρού, έλθετε να καταφάγητε αυτήν.
\par 10 Ποιμένες πολλοί διέφθειραν τον αμπελώνά μου, κατεπάτησαν την μερίδα μου, κατέστησαν την μερίδα την επιθυμητήν μου έρημον άβατον.
\par 11 Παρέδωκαν αυτήν εις ερήμωσιν· ερημωθείσα πενθεί ενώπιόν μου· πάσα η γη ηρημώθη, διότι δεν υπάρχει ο φροντίζων.
\par 12 Επί πάσας τας υψηλάς θέσεις της ερήμου ήλθον οι λεηλάται· διότι η μάχαιρα του Κυρίου θέλει καταφάγει απ' άκρου της γης έως άκρου της γής· εις ουδεμίαν σάρκα δεν θέλει είσθαι ειρήνη.
\par 13 Έσπειραν σίτον αλλά θέλουσι θερίσει ακάνθας· εκοπίασαν αλλά δεν θέλουσιν ωφεληθή· και θέλετε αισχυνθή διά τα προϊόντα σας από του φλογερού θυμού του Κυρίου.
\par 14 Ούτω λέγει ο Κύριος κατά πάντων των κακών γειτόνων μου, οίτινες εγγίζουσι την κληρονομίαν την οποίαν κληροδότησα εις τον λαόν μου τον Ισραήλ· Ιδού, θέλω αποσπάσει αυτούς από της γης αυτών, και θέλω αποσπάσει τον οίκον Ιούδα εκ μέσου αυτών.
\par 15 Και αφού αποσπάσω αυτούς, θέλω επιστρέψει και ελεήσει αυτούς, και θέλω επαναφέρει έκαστον εις την κληρονομίαν αυτού και έκαστον εις την γην αυτού.
\par 16 Και εάν μάθωσι καλώς τας οδούς του λαού μου, να ομνύωσιν εις το όνομά μου, Ζη Κύριος, καθώς εδίδαξαν τον λαόν μου να ομνύη εις τον Βάαλ, τότε θέλουσιν οικοδομηθή εν τω μέσω του λαού μου.
\par 17 Αλλ' εάν δεν υπακούσωσι, θέλω αποσπάσει ολοτελώς και εξολοθρεύσει το έθνος εκείνο, λέγει Κύριος.

\chapter{13}

\par Ούτως είπε Κύριος προς εμέ· Ύπαγε και απόκτησον εις σεαυτόν ζώνην λινήν και περίβαλε αυτήν επί την οσφύν σου και εις ύδωρ μη βάλης αυτήν.
\par 2 Απέκτησα λοιπόν την ζώνην κατά τον λόγον του Κυρίου και περιέβαλον επί την οσφύν μου.
\par 3 Και έγεινε λόγος Κυρίου προς εμέ εκ δευτέρου, λέγων,
\par 4 Λάβε την ζώνην την οποίαν απέκτησας, την επί την οσφύν σου, και σηκωθείς ύπαγε εις τον Ευφράτην και κρύψον αυτήν εκεί εν τη οπή του βράχου.
\par 5 Υπήγα λοιπόν και έκρυψα αυτήν πλησίον του Ευφράτου, καθώς προσέταξεν εις εμέ ο Κύριος.
\par 6 Και μετά πολλάς ημέρας είπε Κύριος προς εμέ, Σηκωθείς ύπαγε εις τον Ευφράτην και λάβε εκείθεν την ζώνην, την οποίαν προσέταξα εις σε να κρύψης εκεί.
\par 7 Και υπήγα εις τον Ευφράτην και έσκαψα και έλαβον την ζώνην εκ του τόπου όπου έκρυψα αυτήν· και ιδού, η ζώνη ήτο εφθαρμένη, δεν ήτο χρήσιμος εις ουδέν.
\par 8 Τότε έγεινε λόγος Κυρίου προς εμέ, λέγων,
\par 9 Ούτω λέγει Κύριος· κατά τούτον τον τρόπον θέλω φθείρει την υπερηφανίαν του Ιούδα και την μεγάλην υπερηφανίαν της Ιερουσαλήμ.
\par 10 Ο κακός ούτος λαός, οίτινες αρνούνται το να υπακούωσιν εις τους λόγους μου, και περιπατούσιν εν ταις ορέξεσι της καρδίας αυτών και υπάγουσιν οπίσω άλλων θεών, διά να λατρεύωσιν αυτούς και να προσκυνώσιν αυτούς, θέλει είσθαι εξάπαντος ως η ζώνη αύτη, ήτις δεν είναι χρήσιμος εις ουδέν.
\par 11 Διότι καθώς η ζώνη κολλάται εις την οσφύν του ανθρώπου, ούτως εκόλλησα εις εμαυτόν πάντα τον οίκον Ισραήλ και πάντα τον οίκον Ιούδα, λέγει Κύριος· διά να ήναι εις εμέ λαός και όνομα και καύχημα και δόξα· αλλά δεν υπήκουσαν.
\par 12 Διά τούτο θέλεις λαλήσει προς αυτούς τον λόγον τούτον· Ούτω λέγει Κύριος ο Θεός του Ισραήλ· πας ασκός θέλει γεμισθή οίνου· και αυτοί θέλουσιν ειπεί προς σε, Μήπως τωόντι δεν γνωρίζομεν ότι πας ασκός θέλει γεμισθή οίνου;
\par 13 Τότε θέλεις ειπεί προς αυτούς, Ούτω λέγει Κύριος· Ιδού, θέλω γεμίσει πάντας τους κατοίκους της γης ταύτης και τους βασιλείς τους καθημένους επί τον θρόνον του Δαβίδ και τους ιερείς και τους προφήτας και πάντας τους κατοίκους της Ιερουσαλήμ, από μεθυσμού.
\par 14 Και θέλω συντρίψει αυτούς μετ' αλλήλων, και τους πατέρας και τους υιούς ομού, λέγει Κύριος· δεν θέλω σπλαγχνισθή ουδέ φεισθή ουδέ ελεήσει, αλλά θέλω εξολοθρεύσει αυτούς.
\par 15 Ακούσατε και ακροάσθητε· μη επαίρεσθε· διότι ο Κύριος ελάλησε.
\par 16 Δότε δόξαν εις Κύριον τον Θεόν υμών, πριν φέρη σκότος και πριν οι πόδες σας προσκόψωσιν επί τα σκοτεινά όρη, και ενώ προσμένετε φως, μετατρέψη αυτό εις σκιάν θανάτου και καταστήση αυτό πυκνόν σκότος.
\par 17 Αλλ' εάν δεν ακούσητε τούτο, η ψυχή μου θέλει κλαύσει κρυφίως διά την υπερηφανίαν υμών, και ο οφθαλμός μου θέλει κλαύσει πικρά και καταρρεύσει δάκρυα, διότι το ποίμνιον του Κυρίου φέρεται εις αιχμαλωσίαν.
\par 18 Είπατε προς τον βασιλέα και προς την βασίλισσαν, Ταπεινώθητε, καθήσατε· διότι θέλει καταβιβασθή από των κεφαλών υμών ο στέφανος της δόξης υμών.
\par 19 Αι πόλεις του νότου θέλουσι κλεισθή και δεν θέλει είσθαι ο ανοίγων· ο Ιούδας άπας θέλει φερθή εις αιχμαλωσίαν, ολοκλήρως θέλει φερθή αιχμάλωτος.
\par 20 Υψώσατε τους οφθαλμούς υμών και θεωρήσατε τους ερχομένους από βορρά· που είναι το ποίμνιον το δοθέν εις σε, τα ώραίά σου πρόβατα;
\par 21 Τι θέλεις ειπεί, όταν σε επισκεφθή; διότι συ εδίδαξας αυτούς να άρχωσιν επί σου ως ηγεμόνες· δεν θέλουσι σε συλλάβει πόνοι, ως γυναίκα τίκτουσαν;
\par 22 Και εάν είπης εν τη καρδία σου, Διά τι συνέβησαν εις εμέ ταύτα; διά το πλήθος της ανομίας σου εσηκώθησαν τα κράσπεδά σου και εγυμνώθησαν αι πτέρναι σου.
\par 23 Δύναται ο Αιθίοψ να αλλάξη το δέρμα αυτού ή η πάρδαλις τα ποικίλματα αυτής; τότε δύνασθε και σεις να κάμητε καλόν, οι μαθόντες το κακόν.
\par 24 Διά τούτο θέλω σκορπίσει αυτούς ως άχυρον φερόμενον υπό ανέμου της ερήμου.
\par 25 Ούτος είναι παρ' εμού ο κλήρός σου, το μερίδιον το μεμετρημένον εις σε, λέγει Κύριος· διότι με ελησμόνησας και ήλπισας επί το ψεύδος.
\par 26 Διά τούτο και εγώ θέλω σηκώσει τα κράσπεδά σου επί το πρόσωπόν σου, και θέλει φανή η αισχύνη σου.
\par 27 Είδον τας μοιχείας σου και τους χρεμετισμούς σου, την αισχρότητα της πορνείας σου, τα βδελύγματά σου επί τους λόφους, επί τας πεδιάδας. Ουαί εις σε, Ιερουσαλήμ δεν θέλεις καθαρισθή; μετά, πότε έτι;

\chapter{14}

\par Ο λόγος του Κυρίου ο γενόμενος προς Ιερεμίαν περί της ανομβρίας.
\par 2 Ο Ιούδας πενθεί και αι πύλαι αυτού είναι περίλυποι· κοίτονται κατά γης μελανειμονούσαι· και ανέβη η κραυγή της Ιερουσαλήμ.
\par 3 Και οι μεγιστάνες αυτής απέστειλαν τους νέους αυτών διά ύδωρ· ήλθον εις τα φρέατα, δεν εύρηκαν ύδωρ· επέστρεψαν με τα αγγεία αυτών κενά· ησχύνθησαν και ενετράπησαν και εσκέπασαν τας κεφαλάς αυτών.
\par 4 Επειδή η γη εσχίσθη, διότι δεν ήτο βροχή επί της γης, οι γεωργοί ησχύνθησαν, εσκέπασαν τας κεφαλάς αυτών.
\par 5 Και η έλαφος έτι, γεννήσασα εν τη πεδιάδι, εγκατέλιπε το τέκνον αυτής, επειδή χόρτος δεν ήτο.
\par 6 Και οι άγριοι όνοι εστάθησαν επί τους υψηλούς τόπους, ερρόφουν τον αέρα ως θώες· οι οφθαλμοί αυτών εμαράνθησαν, επειδή χόρτος δεν ήτο.
\par 7 Κύριε, αν και αι ανομίαι ημών καταμαρτυρώσιν εναντίον ημών, κάμε όμως διά το όνομά σου· διότι αι αποστασίαι ημών επληθύνθησαν· εις σε ημαρτήσαμεν.
\par 8 Ελπίς του Ισραήλ, σωτήρ αυτού εν καιρώ θλίψεως, διά τι ήθελες είσθαι ως πάροικος εν τη γη και ως οδοιπόρος εκκλίνων εις κατάλυμα;
\par 9 Διά τι ήθελες είσθαι ως άνθρωπος εκστατικός, ως ισχυρός μη δυνάμενος να σώση; Αλλά συ, Κύριε, εν μέσω ημών είσαι, και το όνομά σου εκλήθη εφ' ημάς· μη εγκαταλίπης ημάς.
\par 10 Ούτω λέγει Κύριος προς τον λαόν τούτον· Επειδή ηγάπησαν να πλανώνται και δεν εκράτησαν τους πόδας αυτών, διά τούτο ο Κύριος δεν ηυδόκησεν εις αυτούς· τώρα θέλει ενθυμηθή την ανομίαν αυτών και επισκεφθή τας αμαρτίας αυτών.
\par 11 Και είπε Κύριος προς εμέ, Μη προσεύχου υπέρ του λαού τούτου διά καλόν.
\par 12 Και εάν νηστεύσωσι, δεν θέλω εισακούσει της κραυγής αυτών· και εάν προσφέρωσιν ολοκαυτώματα και προσφοράν, δεν θέλω ευδοκήσει εις αυτά· αλλά θέλω καταναλώσει αυτούς εν μαχαίρα και εν πείνη και εν λοιμώ.
\par 13 Και είπα, Ω, Κύριε Θεέ, ιδού, οι προφήται λέγουσι προς αυτούς, δεν θέλετε ιδεί μάχαιραν ουδέ θέλει είσθαι πείνα εις εσάς, αλλά θέλω σας δώσει ειρήνην ασφαλή εν τω τόπω τούτω.
\par 14 Και είπε Κύριος προς εμέ, Ψευδή προφητεύουσιν οι προφήται εν τω ονόματί μου· εγώ δεν απέστειλα αυτούς ουδέ προσέταξα εις αυτούς ουδέ ελάλησα προς αυτούς· αυτοί προφητεύουσιν εις εσάς όρασιν ψευδή και μαντείαν και ματαιότητα και την δολιότητα της καρδίας αυτών.
\par 15 Διά τούτο ούτω λέγει Κύριος περί των προφητών των προφητευόντων εν τω ονόματί μου, ενώ εγώ δεν απέστειλα, αυτούς αλλ' αυτοί λέγουσι, Μάχαιρα και πείνα δεν θέλει είσθαι εν τω τόπω τούτω· εν μαχαίρα και εν πείνη θέλουσι συντελεσθή οι προφήται εκείνοι.
\par 16 Ο δε λαός, εις τους οποίους αυτοί προφητεύουσι, θέλουσιν είσθαι ερριμμένοι εν ταις οδοίς της Ιερουσαλήμ υπό πείνης και μαχαίρας· και δεν θέλει είσθαι ο θάπτων αυτούς, τας γυναίκας αυτών και τους υιούς αυτών και τας θυγατέρας αυτών· και θέλω εκχέει επ' αυτούς την κακίαν αυτών.
\par 17 Διά τούτο θέλεις ειπεί προς αυτούς τον λόγον τούτον· Ας χύσωσιν οι οφθαλμοί μου δάκρυα, νύκτα και ημέραν, και ας μη παύσωσι· διότι η παρθένος, η θυγάτηρ του λαού μου, συνετρίφθη σύντριμμα μέγα, πληγήν οδυνηράν σφόδρα.
\par 18 Εάν εξέλθω εις την πεδιάδα, τότε ιδού, οι πεφονευμένοι εν μαχαίρα· και εάν εισέλθω εις την πόλιν, τότε ιδού, οι νενεκρωμένοι υπό της πείνης, ο δε προφήτης έτι και ο ιερεύς εμπορεύονται επί της γης και δεν αισθάνονται.
\par 19 Απέρριψας παντάπασι τον Ιούδαν; απεστράφη την Σιών η ψυχή σου; Διά τι επάταξας ημάς, και δεν υπάρχει θεραπεία εις ημάς; επροσμένομεν ειρήνην, αλλ' ουδέν αγαθόν· και τον καιρόν της θεραπείας, και ιδού, ταραχή.
\par 20 Γνωρίζομεν, Κύριε, την ασέβειαν ημών, την ανομίαν των πατέρων ημών, ότι ημαρτήσαμεν εις σε.
\par 21 Μη αποστραφής ημάς, διά το όνομά σου· μη ατιμάσης τον θρόνον της δόξης σου· ενθυμήθητι, μη διασκεδάσης την διαθήκην σου την προς ημάς.
\par 22 Υπάρχει μεταξύ των ματαιοτήτων των εθνών διδούς βροχήν; ή οι ουρανοί δίδουσιν υετούς; δεν είσαι συ αυτός ο δοτήρ, Κύριε Θεέ ημών; διά τούτο θέλομεν σε προσμένει· διότι συ έκαμες πάντα ταύτα.

\chapter{15}

\par Και είπε Κύριος προς εμέ, Και αν ο Μωϋσής και ο Σαμουήλ ίσταντο ενώπιόν μου, η ψυχή μου δεν ήθελεν είσθαι υπέρ του λαού τούτου· αποδίωξον αυτούς απ' έμπροσθέν μου και ας εξέλθωσι.
\par 2 Και εάν είπωσι προς σε, Που θέλομεν εξέλθει; τότε θέλεις ειπεί προς αυτούς, Ούτω λέγει Κύριος· Όσοι είναι διά τον θάνατον, εις θάνατον· και όσοι διά την μάχαιραν, εις μάχαιραν· και όσοι διά την πείναν, εις πείναν· και όσοι διά την αιχμαλωσίαν, εις αιχμαλωσίαν.
\par 3 Και θέλω επιφέρει επ' αυτούς τέσσαρα είδη, λέγει Κύριος· την μάχαιραν διά σφαγήν, και τους κύνας διά σπαραγμόν, και τα πετεινά του ουρανού, και τα θηρία της γης, διά να καταφάγωσι και να αφανίσωσι.
\par 4 Και θέλω παραδώσει αυτούς εις διασποράν εν πάσι τοις βασιλείοις της γής· εξ αιτίας του Μανασσή, υιού Εζεκίου βασιλέως του Ιούδα, δι' όσα έπραξεν εν Ιερουσαλήμ.
\par 5 Διότι τις θέλει σε οικτείρει, Ιερουσαλήμ; ή τις θέλει σε συλλυπηθή; ή τις θέλει στραφή διά να ερωτήση, Πως έχεις;
\par 6 Συ με εγκατέλιπες, λέγει Κύριος, υπήγες εις τα οπίσω· διά τούτο θέλω εκτείνει την χείρα μου επί σε και θέλω σε αφανίσει· απέκαμον ελεών.
\par 7 Και θέλω εκλικμήσει αυτούς με το λικμητήριον εν ταις πύλαις της γής· θέλω ατεκνώσει αυτούς, θέλω αφανίσει τον λαόν μου, διότι δεν επιστρέφουσιν από των οδών αυτών.
\par 8 Αι χήραι αυτών επληθύνθησαν ενώπιόν μου υπέρ την άμμον της θαλάσσης· έφερα επ' αυτούς, επί τας μητέρας των νέων, λεηλάτην εν μεσημβρία· επέφερα επ' αυτάς εξαίφνης ταραχάς και τρόμους.
\par 9 Εκείνη, ήτις εγέννησεν επτά, απέκαμε, παρέδωκε το πνεύμα· ο ήλιος αυτής έδυσεν, ενώ, ήτο έτι ημέρα· κατησχύνθη και εταράχθη· το δε υπόλοιπον αυτών θέλω παραδώσει εις την μάχαιραν έμπροσθεν των εχθρών αυτών, λέγει Κύριος.
\par 10 Ουαί εις εμέ, μήτέρ μου, διότι εγέννησας εμέ άνδρα έριδος και άνδρα φιλονεικίας μεθ' όλης της γης. Ούτε ετόκισα ούτε με ετόκισαν· και όμως πας τις εξ αυτών με καταράται.
\par 11 Ο Κύριος λέγει, Βεβαίως το υπόλοιπόν σου θέλει είσθαι καλόν· βεβαίως θέλω μεσιτεύσει υπέρ σου προς τον εχθρόν εν καιρώ συμφοράς και εν καιρώ θλίψεως.
\par 12 Ο σίδηρος θέλει συντρίψει τον σίδηρον του βορρά και τον χαλκόν;
\par 13 Τα υπάρχοντά σου και τους θησαυρούς σου θέλω παραδώσει εις λεηλασίαν άνευ ανταλλάγματος, και τούτο διά πάσας τας αμαρτίας σου και κατά πάντα τα όριά σου.
\par 14 Και θέλω σε περάσει μετά των εχθρών σου εις τόπον τον οποίον δεν γνωρίζεις· διότι πυρ εξήφθη εν τω θυμώ μου, το οποίον θέλει εκκαυθή καθ' υμών.
\par 15 Συ, Κύριε, γνωρίζεις· ενθυμήθητί με και επίσκεψαί με και εκδίκησόν με από των καταδιωκόντων με· μη με αρπάσης εν τη μακροθυμία σου· γνώρισον ότι διά σε υπέφερα ονειδισμόν.
\par 16 Καθώς ευρέθησαν οι λόγοι σου, κατέφαγον αυτούς· και ο λόγος σου ήτο εν εμοί χαρά και αγαλλίασις της καρδίας μου· διότι το όνομά σου εκλήθη επ' εμέ, Κύριε Θεέ των δυνάμεων.
\par 17 Δεν εκάθησα εν συνεδρίω χλευαστών και συνευφράνθην· εκάθησα μόνος εξ αιτίας της χειρός σου· διότι συ με ενέπλησας αδημονίας.
\par 18 Διά τι ο πόνος μου είναι παντοτεινός και η πληγή μου ανίατος, μη θέλουσα να ιατρευθή; θέλεις είσθαι διόλου εις εμέ ως ψεύστης και ως ύδατα απατηλά;
\par 19 Διά τούτο ούτω λέγει Κύριος· Εάν επιστρέψης, τότε θέλω σε αποκαταστήσει πάλιν, και θέλεις ίστασθαι ενώπιόν μου· και εάν αποχωρίσης το τίμιον από του αχρείου, θέλεις είσθαι ως το στόμα μου· αυτοί ας επιστρέψωσι προς σε, αλλά συ με επιστρέψης προς αυτούς.
\par 20 Και θέλω σε κάμει προς τούτον τον λαόν οχυρόν χαλκούν τείχος· και θέλουσι σε πολεμήσει, αλλά δεν θέλουσιν υπερισχύσει εναντίον σου, διότι εγώ είμαι μετά σου διά να σε σώζω και να σε ελευθερόνω, λέγει Κύριος.
\par 21 Και θέλω σε ελευθερώσει εκ της χειρός των πονηρών και θέλω σε λυτρώσει εκ της χειρός των καταδυναστευόντων.

\chapter{16}

\par Και έγεινε λόγος Κυρίου προς εμέ, λέγων,
\par 2 Μη λάβης εις σεαυτόν γυναίκα μηδέ να γείνωσιν εις σε υιοί μηδέ θυγατέρες εν τω τόπω τούτω.
\par 3 Διότι ούτω λέγει Κύριος περί των υιών και περί των θυγατέρων των γεννωμένων εν τω τόπω τούτω και περί των μητέρων αυτών, αίτινες εγέννησαν αυτούς, και περί των πατέρων αυτών, οίτινες ετεκνοποίησαν αυτούς εν τη γη ταύτη·
\par 4 θέλουσιν αποθάνει με οδυνηρόν θάνατον· δεν θέλουσι κλαυθή ουδέ θέλουσι ταφή· θέλουσιν είσθαι διά κοπρίαν επί το πρόσωπον της γης, και θέλουσιν αφανισθή υπό μαχαίρας και υπό πείνης, και τα πτώματα αυτών θέλουσιν είσθαι τροφή εις τα πετεινά του ουρανού και εις τα θηρία της γης.
\par 5 Διότι ούτω λέγει Κύριος· Μη εισέλθης εις οίκον πένθους και μη υπάγης να πενθήσης μηδέ να συγκλαύσης αυτούς· διότι αφήρεσα την ειρήνην μου από του λαού τούτου, λέγει Κύριος, το έλεος και τους οικτιρμούς.
\par 6 Και θέλουσιν αποθάνει μεγάλοι και μικροί εν τη γη ταύτη· δεν θέλουσι ταφή ουδέ θέλουσι κλαύσει αυτούς ουδέ θέλουσι κάμει εντομάς εις τα σώματα αυτών ουδέ θέλουσι ξυρισθή δι' αυτούς·
\par 7 ουδέ θέλουσι διαμοιράσει άρτον εις το πένθος προς παρηγορίαν αυτών διά τον τεθνεώτα, ουδέ θέλουσι ποτίσει αυτούς το ποτήριον της παρηγορίας διά τον πατέρα αυτών ή διά την μητέρα αυτών.
\par 8 Και δεν θέλεις εισέλθει εις οίκον συμποσίου, διά να συγκαθήσης μετ' αυτών διά να φάγης και να πίης.
\par 9 Διότι ούτω λέγει ο Κύριος των δυνάμεων, ο Θεός του Ισραήλ Ιδού, εγώ θέλω παύσει από του τόπου τούτου, ενώπιον των οφθαλμών υμών και εν ταις ημέραις υμών, την φωνήν της χαράς και την φωνήν της ευφροσύνης, την φωνήν του νυμφίου και την φωνήν της νύμφης.
\par 10 Και όταν αναγγείλης προς τον λαόν τούτον πάντας τούτους τους λόγους, και είπωσι προς σε, Διά τι ο Κύριος επρόφερεν άπαν τούτο το μέγα κακόν εναντίον ημών; και τις η ανομία ημών; και τις η αμαρτία ημών, την οποίαν ημαρτήσαμεν εις Κύριον τον Θεόν ημών;
\par 11 Τότε θέλεις ειπεί προς αυτούς, Επειδή με εγκατέλιπον οι πατέρες υμών, λέγει Κύριος, και υπήγαν οπίσω άλλων θεών και ελάτρευσαν αυτούς και προσεκύνησαν αυτούς και εγκατέλιπον εμέ και τον νόμον μου δεν εφύλαξαν·
\par 12 και επειδή σεις επράξατε χειρότερα παρά τους πατέρας υμών, και ιδού, περιπατείτε έκαστος οπίσω της ορέξεως της πονηράς αυτού καρδίας, ώστε να μη υπακούητε εις εμέ·
\par 13 διά τούτο θέλω απορρίψει υμάς από της γης ταύτης, εις την γην, την οποίαν δεν εγνωρίσατε, υμείς και οι πατέρες υμών· και εκεί θέλετε λατρεύσει άλλους θεούς ημέραν και νύκτα· διότι δεν θέλω κάμει έλεος προς υμάς.
\par 14 Διά τούτο ιδού, έρχονται ημέραι, λέγει Κύριος, και δεν θέλουσιν ειπεί πλέον, Ζη Κύριος, όστις ανήγαγε τους υιούς Ισραήλ εκ γης Αιγύπτου,
\par 15 αλλά, Ζη Κύριος, όστις ανήγαγε τους υιούς Ισραήλ εκ της γης του βορρά και εκ πάντων των τόπων όπου είχε διώξει αυτούς· και θέλω επαναφέρει αυτούς πάλιν εις την γην αυτών, την οποίαν έδωκα εις τους πατέρας αυτών.
\par 16 Ιδού, θέλω αποστείλει πολλούς αλιείς, λέγει Κύριος, και θέλουσιν αλιεύσει αυτούς· και μετά ταύτα θέλω αποστείλει πολλούς κυνηγούς και θέλουσι θηρεύσει αυτούς από παντός όρους και από παντός λόφου και από των σχισμών των βράχων.
\par 17 Διότι οι οφθαλμοί μου είναι επί πάσας τας οδούς αυτών· δεν είναι κεκρυμμέναι από του προσώπου μου ουδέ η ανομία αυτών είναι κεκρυμμένη απ' έμπροσθεν των οφθαλμών μου.
\par 18 Και πρώτον θέλω ανταποδώσει διπλά την ανομίαν αυτών και την αμαρτίαν αυτών· διότι εμίαναν την γην μου με τα πτώματα των βδελυγμάτων αυτών και ενέπλησαν την κληρονομίαν μου από των μιασμάτων αυτών.
\par 19 Κύριε, δύναμίς μου και φρούριόν μου και καταφυγή μου εν ημέρα θλίψεως, τα έθνη θέλουσιν ελθεί προς σε από των περάτων της γης και θέλουσιν ειπεί, Βεβαίως οι πατέρες ημών εκληρονόμησαν ψεύδος, ματαιότητα και τα ανωφελή.
\par 20 Θέλει κάμει άνθρωπος εις εαυτόν θεούς τους μη όντας θεούς;
\par 21 Διά τούτο, ιδού, θέλω κάμει αυτούς ταύτην την φοράν να γνωρίσωσι, θέλω κάμει αυτούς να γνωρίσωσι την χείρα μου και την δύναμίν μου, και θέλουσι γνωρίσει ότι το όνομά μου είναι ο Κύριος.

\chapter{17}

\par Η αμαρτία του Ιούδα είναι γεγραμμένη με γραφίδα σιδηράν, με όνυχα αδαμάντινον, ενεχαράχθη επί της πλακός της καρδίας αυτών και επί των κεράτων των θυσιαστηρίων υμών·
\par 2 ώστε οι υιοί αυτών ενθυμούνται τα θυσιαστήρια αυτών και τα άλση αυτών, μετά των πρασίνων δένδρων επί τους υψηλούς λόφους.
\par 3 Ω όρος μου εν τη πεδιάδι, θέλω δώσει την περιουσίαν σου και πάντας τους θησαυρούς σου εις διαρπαγήν και τους υψηλούς σου τόπους κατά πάντα τα όριά σου, διά την αμαρτίαν.
\par 4 Και συ, μάλιστα αυτή συ, θέλεις εκβληθή από της κληρονομίας σου, την οποίαν έδωκα εις σε, και θέλω σε καταδουλώσει εις τους εχθρούς σου, εν γη την οποίαν δεν εγνώρισας· διότι πυρ εξήψατε εν τω θυμώ μου, το οποίον θέλει καίεσθαι εις τον αιώνα.
\par 5 Ούτω λέγει Κύριος· Επικατάρατος ο άνθρωπος, όστις ελπίζει επί άνθρωπον και κάμνει σάρκα βραχίονα αυτού και του οποίου η καρδία απομακρύνεται από του Κυρίου.
\par 6 Διότι θέλει είσθαι ως η αγριομυρίκη εν ερήμω, και δεν θέλει ιδεί όταν έλθη το αγαθόν· αλλά θέλει κατοικεί τόπους ξηρούς εν ερήμω, γην αλμυράν και ακατοίκητον.
\par 7 Ευλογημένος ο άνθρωπος ο ελπίζων επί Κύριον και του οποίου ο Κύριος είναι η ελπίς.
\par 8 Διότι θέλει είσθαι ως δένδρον πεφυτευμένον πλησίον των υδάτων, το οποίον εξαπλόνει τας ρίζας αυτού πλησίον του ποταμού, και δεν θέλει ιδεί όταν έρχηται το καύμα αλλά το φύλλον αυτού θέλει θάλλει· και δεν θέλει μεριμνήσει εν τω έτει της ανομβρίας ουδέ θέλει παύσει από του να κάμνη καρπόν.
\par 9 Η καρδία είναι απατηλή υπέρ πάντα και σφόδρα διεφθαρμένη· τις δύναται να γνωρίση αυτήν;
\par 10 Εγώ ο Κύριος εξετάζω την καρδίαν, δοκιμάζω τους νεφρούς, διά να δώσω εις έκαστον κατά τας οδούς αυτού, κατά τον καρπόν των έργων αυτού.
\par 11 Καθώς η πέρδιξ η επωάζουσα και μη νεοσσεύουσα, ούτως ο αποκτών πλούτη αδίκως θέλει αφήσει αυτά εις το ήμισυ των ημερών αυτού και εις τα έσχατα αυτού θέλει είσθαι άφρων.
\par 12 Θρόνος δόξης υψωμένος εξ αρχής είναι ο τόπος του αγιαστηρίου ημών.
\par 13 Κύριε, η ελπίς του Ισραήλ, πάντες οι εγκαταλείποντές σε θέλουσι καταισχυνθή και οι αποστάται εμού θέλουσι γραφθή εν τη γή· διότι εγκατέλιπον τον Κύριον, την πηγήν των ζώντων υδάτων.
\par 14 Ιασαί με, Κύριε, και θέλω ιαθή· σώσον με και θέλω σωθή· διότι συ είσαι το καύχημά μου·
\par 15 Ιδού, ούτοι λέγουσι προς εμέ, Που ο λόγος του Κυρίου; ας έλθη τώρα.
\par 16 Αλλ' εγώ δεν απεσύρθην από του να σε ακολουθώ ως ποιμήν, ουδέ επεθύμησα την ημέραν της θλίψεως· συ εξεύρεις τούτο· τα εξελθόντα εκ των χειλέων μου ήσαν ενώπιόν σου.
\par 17 Μη γείνης εις εμέ τρόμος· συ είσαι η ελπίς μου εν ημέρα συμφοράς·
\par 18 Ας καταισχυνθώσιν οι καταδιώκοντές με, εγώ δε ας μη καταισχυνθώ· ας τρομάξωσιν εκείνοι αλλ' ας μη τρομάξω εγώ· φέρε επ' αυτούς ημέραν συμφοράς και σύντριψον αυτούς διπλούν σύντριμμα.
\par 19 Ούτως είπε Κύριος προς εμέ· Ύπαγε και στήθι εν τη πύλη των υιών του λαού σου, δι' ης εισέρχονται οι βασιλείς Ιούδα και δι' ης εξέρχονται, και εν πάσαις ταις πύλαις της Ιερουσαλήμ·
\par 20 και ειπέ προς αυτούς, Ακούσατε τον λόγον του Κυρίου, βασιλείς Ιούδα και, πας ο Ιούδας και πάντες οι κάτοικοι της Ιερουσαλήμ, οι εισερχόμενοι διά των πυλών τούτων·
\par 21 ούτω λέγει Κύριος· Προσέχετε εις εαυτούς, και μη βαστάζετε φορτίον την ημέραν του σαββάτου μηδέ εμβιβάζετε διά των πυλών της Ιερουσαλήμ·
\par 22 μηδέ εκφέρετε φορτίον εκ των οικιών σας την ημέραν του σαββάτου και μη κάμνετε μηδεμίαν εργασίαν· αλλά αγιάζετε την ημέραν του σαββάτου, καθώς προσέταξα εις τους πατέρας υμών·
\par 23 δεν υπήκουσαν όμως ουδέ έκλιναν το ωτίον αυτών, αλλ' εσκλήρυναν τον τράχηλον αυτών διά να μη ακούσωσι και διά να μη δεχθώσι νουθεσίαν.
\par 24 Αλλ' εάν υπακούσητε εις εμέ, λέγει Κύριος, ώστε να μη εμβιβάζητε φορτίον διά των πυλών της πόλεως ταύτης την ημέραν του σαββάτου, αλλά να αγιάζητε την ημέραν του σαββάτου μη κάμνοντες εν αυτή μηδεμίαν εργασίαν·
\par 25 τότε θέλουσιν εισέλθει διά των πυλών της πόλεως ταύτης βασιλείς και άρχοντες καθήμενοι επί του θρόνου του Δαβίδ, εποχούμενοι επί αμάξας και ίππους, αυτοί και οι άρχοντες αυτών, οι άνδρες Ιούδα και οι κάτοικοι της Ιερουσαλήμ· και η πόλις αύτη θέλει κατοικείσθαι εις τον αιώνα.
\par 26 Και θέλουσιν ελθεί εκ των πόλεων Ιούδα και εκ των πέριξ της Ιερουσαλήμ και εκ της γης Βενιαμίν και εκ της πεδινής και εκ των ορέων και εκ του νότου, φέροντες ολοκαυτώματα και θυσίας και προσφοράς εξ αλφίτων και λίβανον, φέροντες έτι και προσφοράς ευχαριστηρίους εις τον οίκον του Κυρίου.
\par 27 Αλλ' εάν δεν μου υπακούσητε, ώστε να αγιάζητε την ημέραν του σαββάτου και να μη βαστάζητε φορτίον και εμβιβάζητε εις τας πύλας της Ιερουσαλήμ την ημέραν του σαββάτου, τότε θέλω ανάψει πυρ εν ταις πύλαις αυτής και θέλει καταφάγει τα παλάτια της Ιερουσαλήμ και δεν θέλει σβεσθή.

\chapter{18}

\par Ο λόγος ο γενόμενος προς Ιερεμίαν παρά Κυρίου, λέγων,
\par 2 Σηκώθητι και κατάβηθι εις τον οίκον του κεραμέως, και εκεί θέλω σε κάμει να ακούσης τους λόγους μου.
\par 3 Τότε κατέβην εις τον οίκον του κεραμέως, και ιδού, ειργάζετο έργον επί τους τροχούς.
\par 4 Και εχαλάσθη το αγγείον, το οποίον έκαμνεν εκ πηλού, εν τη χειρί του κεραμέως, και πάλιν έκαμεν αυτό άλλο αγγείον, καθώς ήρεσεν εις τον κεραμέα να κάμη.
\par 5 Τότε έγεινε λόγος Κύριου προς εμέ, λέγων,
\par 6 Οίκος Ισραήλ, δεν δύναμαι να κάμω εις εσάς, καθώς ούτος ο κεραμεύς; λέγει Κύριος. Ιδού, ως ο πηλός εν τη χειρί του κεραμέως, ούτω σεις, οίκος Ισραήλ, είσθε εν τη χειρί μου.
\par 7 Εν τη στιγμή, καθ' ην ήθελον λαλήσει κατά έθνους ή κατά βασιλείας, διά να εκριζώσω και να κατασκάψω και να καταστρέψω,
\par 8 εάν το έθνος εκείνο, κατά του οποίου ελάλησα, επιστρέψη από της κακίας αυτού, θέλω μετανοήσει περί του κακού, το οποίον εβουλεύθην να κάμω εις αυτό.
\par 9 Και εν τη στιγμή, καθ' ην ήθελον λαλήσει περί έθνους ή περί βασιλείας, να οικοδομήσω και να φυτεύσω,
\par 10 εάν κάμη κακόν ενώπιόν μου, ώστε να μη υπακούη της φωνής μου, τότε θέλω μετανοήσει περί του καλού, με το οποίον είπα ότι θέλω αγαθοποιήσει αυτό.
\par 11 Και τώρα ειπέ προς τους άνδρας Ιούδα και προς τους κατοίκους της Ιερουσαλήμ, λέγων, Ούτω λέγει Κύριος· Ιδού, εγώ ετοιμάζω κακόν καθ' υμών και βουλεύομαι βουλήν καθ' υμών· επιστρέψατε λοιπόν έκαστος από της πονηράς οδού αυτού και διορθώσατε τας οδούς υμών και τας πράξεις υμών.
\par 12 Οι δε είπον, Εις μάτην· διότι οπίσω των διαβουλίων ημών θέλομεν περιπατεί και έκαστος τας ορέξεις της καρδίας αυτού της πονηράς θέλομεν πράττει.
\par 13 Διά τούτο ούτω λέγει Κύριος· Ερωτήσατε τώρα μεταξύ των εθνών, τις ήκουσε τοιαύτα; η παρθένος του Ισραήλ έκαμε φρικτά σφόδρα.
\par 14 Θέλει τις αφήσει τον χιονώδη Λίβανον διά τον βράχον της πεδιάδος; ή θέλουσιν εγκαταλίπει τα δροσερά εκρέοντα ύδατα διά τα μακρόθεν ερχόμενα;
\par 15 Αλλ' ο λαός μου ελησμόνησεν εμέ, εθυμίασεν εις την ματαιότητα και προσέκοψαν εν ταις οδοίς αυτών, ταις αιωνίοις τρίβοις, διά να περιπατώσιν εν τρίβοις οδού μη εξωμαλισμένης·
\par 16 διά να καταστήσωσι την γην αυτών ερήμωσιν και χλευασμόν αιώνιον· πας ο διαβαίνων δι' αυτής θέλει μένει έκθαμβος και σείει την κεφαλήν αυτού.
\par 17 Θέλω διασκορπίσει αυτούς έμπροσθεν του εχθρού ως καυστικός άνεμος· θέλω δείξει εις αυτούς νώτα και ουχί πρόσωπον εν τη ημέρα της συμφοράς αυτών.
\par 18 Τότε είπον, Έλθετε και ας συμβουλευθώμεν βουλάς κατά του Ιερεμίου· διότι νόμος δεν θέλει χαθή από ιερέως ουδέ βουλή από σοφού ουδέ λόγος από προφήτου· έλθετε και ας πατάξωμεν αυτόν με την γλώσσαν και ας μη προσέξωμεν εις μηδένα των λόγων αυτού.
\par 19 Πρόσεξον εις εμέ, Κύριε, και άκουσον την φωνήν των διαφιλονεικούντων με εμέ.
\par 20 Θέλει ανταποδοθή κακόν αντί καλού; διότι έσκαψαν λάκκον διά την ψυχήν μου. Ενθυμήθητι ότι εστάθην ενώπιόν σου διά να λαλήσω υπέρ αυτών αγαθά, διά να αποστρέψω τον θυμόν σου απ' αυτών.
\par 21 Διά τούτο παράδος τους υιούς αυτών εις την πείναν και δος αυτούς εις χείρας μαχαίρας, και ας γείνωσιν αι γυναίκες αυτών άτεκνοι και χήραι· και οι άνδρες αυτών ας θανατωθώσιν· οι νεανίσκοι αυτών ας πέσωσι διά μαχαίρας εν τη μάχη.
\par 22 Ας ακουσθή κραυγή εκ των οικιών αυτών, όταν φέρης εξαίφνης επ' αυτούς λεηλάτας· διότι έσκαψαν λάκκον διά να με πιάσωσι και έκρυψαν παγίδας διά τους πόδας μου.
\par 23 Συ δε, Κύριε, γνωρίζεις πάσαν την κατ' εμού βουλήν αυτών εις το να με θανατώσωσι· μη συγχωρήσης την ανομίαν αυτών, και την αμαρτίαν αυτών μη εξαλείψης απ' έμπροσθέν σου· αλλά ας καταστραφώσιν ενώπιόν σου· ενέργησον κατ' αυτών εν τω καιρώ του θυμού σου.

\chapter{19}

\par Ούτω λέγει Κύριος· Ύπαγε και απόκτησον λάγηνον πηλίνην κεραμέως, και φέρε τινάς εκ των πρεσβυτέρων του λαού και εκ των πρεσβυτέρων των ιερέων·
\par 2 και έξελθε εις την φάραγγα του υιού Εννόμ, την πλησίον της εισόδου της ανατολικής πύλης, και διακήρυξον εκεί τους λόγους, τους οποίους θέλω λαλήσει προς σε.
\par 3 Και ειπέ, Ακούσατε τον λόγον του Κυρίου, βασιλείς Ιούδα και κάτοικοι της Ιερουσαλήμ. Ούτω λέγει ο Κύριος των δυνάμεων, ο Θεός του Ισραήλ· Ιδού, θέλω φέρει επί τον τόπον τούτον κακά, τα οποία παντός ακούοντος θέλουσιν ηχήσει τα ώτα αυτού.
\par 4 Διότι εγκατέλιπον εμέ και εβεβήλωσαν τον τόπον τούτον και εθυμίασαν εν αυτώ εις άλλους θεούς, τους οποίους δεν εγνώρισαν, αυτοί και οι πατέρες αυτών και οι βασιλείς Ιούδα, και εγέμισαν τον τόπον τούτον από αίματος αθώων.
\par 5 Και ωκοδόμησαν τους υψηλούς τόπους του Βάαλ, διά να καίωσι τους υιούς αυτών εν πυρί, ολοκαυτώματα προς τον Βάαλ· το οποίον δεν προσέταξα ουδέ ελάλησα ουδέ ανέβη επί την καρδίαν μου.
\par 6 Διά τούτο, ιδού, έρχονται ημέραι, λέγει Κύριος, και ο τόπος ούτος δεν θέλει καλείσθαι πλέον Τοφέθ ουδέ Φάραγξ του υιού Εννόμ, αλλ' Η φάραγξ της σφαγής.
\par 7 Και θέλω ματαιώσει την βουλήν του Ιούδα και της Ιερουσαλήμ εν τω τόπω τούτω· και θέλω κάμει αυτούς να πέσωσι διά μαχαίρας έμπροσθεν των εχθρών αυτών και διά των χειρών των ζητούντων την ζωήν αυτών· τα δε πτώματα αυτών θέλω δώσει βρώσιν εις τα πετεινά του ουρανού και εις τα θηρία της γης.
\par 8 Και θέλω καταστήσει την πόλιν ταύτην ερήμωσιν και συριγμόν· πας ο διαβαίνων δι' αυτής θέλει μένει έκθαμβος και θέλει συρίξει διά πάσας τας πληγάς αυτής.
\par 9 και θέλω κάμει αυτούς να φάγωσι την σάρκα των υιών αυτών και την σάρκα των θυγατέρων αυτών, και θέλουσι φάγει έκαστος την σάρκα του φίλου αυτού εν τη πολιορκία και στενοχωρία, με την οποίαν οι εχθροί αυτών και οι ζητούντες την ζωήν αυτών θέλουσι στενοχωρήσει αυτούς.
\par 10 Τότε θέλεις συντρίψει την λάγηνον έμπροσθεν των ανδρών των εξελθόντων μετά σού·
\par 11 και θέλεις ειπεί προς αυτούς, Ούτω λέγει ο Κύριος των δυνάμεων. Ούτω θέλω συντρίψει τον λαόν τούτον και την πόλιν ταύτην, καθώς συντρίβει τις το αγγείον του κεραμέως, το οποίον δεν δύναται να διορθωθή πλέον· και θέλουσι θάπτει αυτούς εν Τοφέθ, εωσού να μη υπάρχη τόπος εις ταφήν.
\par 12 Ούτω θέλω κάμει εις τον τόπον τούτον, λέγει Κύριος, και εις τους κατοίκους αυτού, και θέλω κάμει την πόλιν ταύτην ως την Τοφέθ·
\par 13 και οι οίκοι της Ιερουσαλήμ και οι οίκοι των βασιλέων του Ιούδα θέλουσι μιανθή, καθώς ο τόπος της Τοφέθ· μετά πασών των οικιών, επί των δωμάτων των οποίων εθυμίασαν προς άπασαν την στρατιάν του ουρανού και έκαμαν σπονδάς εις άλλους θεούς.
\par 14 Τότε ήλθεν ο Ιερεμίας εκ της Τοφέθ, όπου ο Κύριος απέστειλεν αυτόν διά να προφητεύση· και σταθείς εν τη αυλή του οίκου του Κυρίου είπε προς πάντα τον λαόν,
\par 15 Ούτω λέγει ο Κύριος των δυνάμεων, ο Θεός του Ισραήλ· Ιδού, θέλω φέρει επί την πόλιν ταύτην και επί πάσας τας κώμας αυτής πάντα τα κακά όσα ελάλησα κατ' αυτής· διότι εσκλήρυναν τον τράχηλον αυτών, ώστε να μη ακούσωσι τους λόγους μου.

\chapter{20}

\par Ο δε Πασχώρ, ο υιός του Ιμμήρ ο ιερεύς, ο και προϊστάμενος εν τω οίκω του Κυρίου, ήκουσε τον Ιερεμίαν προφητεύοντα τους λόγους τούτους.
\par 2 Και επάταξεν ο Πασχώρ Ιερεμίαν τον προφήτην και έβαλεν αυτόν εις το δεσμωτήριον το εν τη άνω πύλη του Βενιαμίν, το εν τω οίκω του Κυρίου.
\par 3 Και την επαύριον εξήγαγεν ο Πασχώρ τον Ιερεμίαν εκ του δεσμωτηρίου. Και ο Ιερεμίας είπε προς αυτόν, Ο Κύριος δεν εκάλεσε το όνομά σου Πασχώρ, αλλά Μαγόρ-μισσαβίβ.
\par 4 Διότι ούτω λέγει Κύριος· Ιδού, θέλω σε κάμει τρόμον εις σεαυτόν και εις πάντας τους φίλους σου· και θέλουσι πέσει διά της μαχαίρας των εχθρών αυτών και οι οφθαλμοί σου θέλουσιν ιδεί τούτο· και θέλω δώσει πάντα τον Ιούδαν εις την χείρα του βασιλέως της Βαβυλώνος, και θέλει φέρει αυτούς αιχμαλώτους εις την Βαβυλώνα και θέλει πατάξει αυτούς εν μαχαίρα.
\par 5 Και θέλω δώσει πάσαν την δύναμιν της πόλεως ταύτης και πάντας τους κόπους αυτής και πάντα τα πολύτιμα αυτής και πάντας τους θησαυρούς των βασιλέων Ιούδα θέλω δώσει εις την χείρα των εχθρών αυτών, και θέλουσι λεηλατήσει αυτούς και λάβει αυτούς και φέρει αυτούς εις την Βαβυλώνα.
\par 6 Και συ, Πασχώρ, και πάντες οι κατοικούντες εν τω οίκω σου, θέλετε υπάγει εις αιχμαλωσίαν· και θέλεις ελθεί εις την Βαβυλώνα, και εκεί θέλεις αποθάνει και εκεί θέλεις ταφή, συ και πάντες οι φίλοι σου, εις τους οποίους προεφήτευσας ψευδώς.
\par 7 Κύριε, με εδελέασας και εδελεάσθην· υπερίσχυσας κατ' εμού και κατίσχυσας· έγεινα χλευασμός όλην την ημέραν· πάντες με εμπαίζουσι.
\par 8 Διότι αφού ήνοιξα στόμα, βοώ, φωνάζω βίαν και αρπαγήν· όθεν ο λόγος του Κυρίου έγεινεν εις εμέ προς ονειδισμόν και προς χλευασμόν όλην την ημέραν.
\par 9 Και είπα, Δεν θέλω αναφέρει περί αυτού ουδέ θέλω λαλήσει πλέον εν τω ονόματι αυτού· όμως ο λόγος αυτού ήτο εν τη καρδία μου ως καιόμενον πυρ περικεκλεισμένον εν τοις οστέοις μου, και απέκαμον χαλινόνων εμαυτόν και δεν ηδυνάμην πλέον.
\par 10 Διότι ήκουσα ύβριν παρά πολλών· τρόμος πανταχόθεν· Κατηγορήσατε, λέγουσι, και θέλομεν κατηγορήσει αυτόν. Πάντες οι ειρηνεύοντες μετ' εμού παρεφύλαττον το πρόσκομμά μου, λέγοντες, Ίσως δελεασθή, και θέλομεν υπερισχύσει εναντίον αυτού και εκδικηθή κατ' αυτού.
\par 11 Αλλ' ο Κύριος είναι μετ' εμού ως ισχυρός πολεμιστής· διά τούτο οι διώκταί μου θέλουσι προσκόψει και δεν θέλουσιν υπερισχύσει· θέλουσι καταισχυνθή σφόδρα· διότι δεν ενόησαν· η αιώνιος αισχύνη αυτών δεν θέλει λησμονηθή.
\par 12 Αλλά, Κύριε των δυνάμεων, ο δοκιμάζων τον δίκαιον, ο βλέπων τους νεφρούς και την καρδίαν, ας ίδω την εκδίκησίν σου επ' αυτούς· διότι εις σε εφανέρωσα την κρίσιν μου.
\par 13 Ψάλλετε εις τον Κύριον, αινείτε τον Κύριον· διότι ηλευθέρωσε την ψυχήν του πτωχού εκ χειρός πονηρευομένων.
\par 14 Επικατάρατος η ημέρα, καθ' ην εγεννήθην· η ημέρα καθ' ην η μήτηρ μου με εγέννησεν, ας μη ήναι ευλογημένη.
\par 15 Επικατάρατος ο άνθρωπος, όστις ευηγγελίσατο προς τον πατέρα μου, λέγων, Εγεννήθη εις σε παιδίον άρσεν, ευφραίνων αυτόν σφόδρα.
\par 16 Και ας ήναι ο άνθρωπος εκείνος ως αι πόλεις, τας οποίας ο Κύριος κατέστρεψε και δεν μετεμελήθη· και ας ακούση κραυγήν το πρωΐ και αλαλαγμόν εν μεσημβρία.
\par 17 Διά τι δεν εθανατώθην εκ μήτρας; ή η μήτηρ μου δεν έγεινε τάφος εις εμέ και η μήτρα αυτής δεν με εβάστασεν εις αιώνιον σύλληψιν;
\par 18 διά τι εξήλθον εκ της μήτρας, διά να βλέπω μόχθον και λύπην και να τελειώσωσιν αι ημέραι μου εν αισχύνη;

\chapter{21}

\par Ο λόγος ο γενόμενος προς Ιερεμίαν παρά Κυρίου, ότε απέστειλε προς αυτόν ο βασιλεύς Σεδεκίας τον Πασχώρ υιόν του Μελχίου και τον Σοφονίαν υιόν του Μαασίου τον ιερέα, λέγων,
\par 2 Ερώτησον, παρακαλώ, τον Κύριον περί ημών· διότι Ναβουχοδονόσορ ο βασιλεύς της Βαβυλώνος ήγειρε πόλεμον καθ' ημών· ίσως ο Κύριος ενεργήση εις ημάς κατά πάντα τα θαυμάσια αυτού, ώστε να απέλθη αφ' ημών.
\par 3 Τότε είπε προς αυτούς ο Ιερεμίας, Ούτω θέλετε ειπεί προς τον Σεδεκίαν.
\par 4 Ούτω λέγει Κύριος, ο Θεός του Ισραήλ· Ιδού, εγώ στρέφω εις τα οπίσω τα όπλα του πολέμου τα εν ταις χερσίν υμών, με τα οποία σεις πολεμείτε κατά του βασιλέως της Βαβυλώνος και των Χαλδαίων, οίτινες σας πολιορκούσιν έξωθεν των τειχών· και θέλω συνάξει αυτούς εις το μέσον της πόλεως ταύτης.
\par 5 Και εγώ θέλω πολεμήσει εναντίον σας με χείρα εξηπλωμένην και με βραχίονα κραταιόν και θυμόν και με αγανάκτησιν και με οργήν μεγάλην.
\par 6 Και θέλω πατάξει τους κατοίκους της πόλεως ταύτης και άνθρωπον και κτήνος· υπό λοιμού μεγάλου θέλουσιν αποθάνει.
\par 7 Και μετά ταύτα, λέγει Κύριος, θέλω παραδώσει Σεδεκίαν τον βασιλέα του Ιούδα και τους δούλους αυτού και τον λαόν και τους εναπολειφθέντας εν τη πόλει ταύτη από του λοιμού, από της μαχαίρας και από της πείνης, εις την χείρα του Ναβουχοδονόσορ, βασιλέως της Βαβυλώνος, και εις την χείρα των εχθρών αυτών και εις την χείρα των ζητούντων την ψυχήν αυτών· και αυτός θέλει πατάξει αυτούς εν στόματι μαχαίρας· δεν θέλει φεισθή αυτούς ουδέ θέλει οικτείρει ουδέ θέλει σπλαγχνισθή αυτούς.
\par 8 Και προς τον λαόν τούτον θέλεις ειπεί, Ούτω λέγει Κύριος· Ιδού, έθεσα ενώπιόν σας την οδόν της ζωής και την οδόν του θανάτου.
\par 9 Όστις κάθηται εν τη πόλει ταύτη, θέλει αποθάνει υπό μαχαίρας και υπό πείνης και υπό λοιμού· αλλ' όστις εξέλθη και προχωρήση προς τους Χαλδαίους οίτινες σας πολιορκούσι, θέλει ζήσει και η ζωή αυτού θέλει είσθαι ως λάφυρον εις αυτόν.
\par 10 Διότι έστησα το πρόσωπόν μου εναντίον της πόλεως ταύτης προς κακόν και ουχί προς καλόν, λέγει Κύριος· θέλει παραδοθή εις την χείρα του βασιλέως της Βαβυλώνος και θέλει κατακαύσει αυτήν εν πυρί.
\par 11 Περί δε του οίκου του βασιλέως του Ιούδα, ειπέ, Ακούσατε τον λόγον του Κυρίου·
\par 12 οίκος Δαβίδ, ούτω λέγει Κύριος· Κρίνετε κρίσιν το πρωΐ και ελευθερόνετε τον γεγυμνωμένον εκ της χειρός του δυνάστου, μήποτε η οργή μου εξέλθη ως πυρ και εκκαυθή, χωρίς να υπάρχη ο σβέσων, εξ αιτίας της κακίας των έργων σας.
\par 13 Ιδού, εγώ είμαι εναντίον εις σε, λέγει Κύριος, την καθημένην εν τη κοιλάδι και εν τω βράχω της πεδιάδος, εναντίον εις εσάς τους λέγοντας, Τις θέλει καταβή εναντίον ημών, ή τις θέλει εισέλθει εις τας κατοικίας ημών;
\par 14 Και θέλω σας τιμωρήσει κατά τον καρπόν των έργων σας, λέγει Κύριος· και θέλω ανάψει πυρ εν τω δάσει αυτής και θέλει καταφάγει πάντα τα πέριξ αυτής.

\chapter{22}

\par Ούτω λέγει Κύριος· Κατάβηθι εις τον οίκον του βασιλέως του Ιούδα και λάλησον εκεί τον λόγον τούτον,
\par 2 και ειπέ, Άκουσον τον λόγον του Κυρίου, βασιλεύ του Ιούδα, ο καθήμενος επί του θρόνου του Δαβίδ, συ και οι δούλοί σου και ο λαός σου, οι εισερχόμενοι διά των πυλών τούτων·
\par 3 Ούτω λέγει Κύριος· Κάμνετε κρίσιν και δικαιοσύνην και ελευθερόνετε τον γεγυμνωμένον εκ της χειρός του δυνάστου· και μη αδικείτε μηδέ καταδυναστεύετε τον ξένον, τον ορφανόν και την χήραν και αίμα αθώον μη χύνετε εν τω τόπω τούτω.
\par 4 Διότι εάν τωόντι κάμνητε τον λόγον τούτον, τότε θέλουσιν εισέλθει διά των πυλών του οίκου τούτου βασιλείς καθήμενοι επί του θρόνου του Δαβίδ, εποχούμενοι επί αμαξών και ίππων, αυτοί και οι δούλοι αυτών και ο λαός αυτών.
\par 5 Αλλ' εάν δεν ακούσητε τους λόγους τούτους, ομνύω εις εμαυτόν, λέγει Κύριος, ότι ο οίκος ούτος θέλει κατασταθή έρημος.
\par 6 Διότι ούτω λέγει Κύριος προς τον οίκον του βασιλέως του Ιούδα. Συ είσαι Γαλαάδ εις εμέ και κορυφή του Λιβάνου· αλλά θέλω σε καταστήσει ερημίαν, πόλεις ακατοικήτους.
\par 7 Και θέλω ετοιμάσει εναντίον σου εξολοθρευτάς, έκαστον μετά των όπλων αυτού· και θέλουσι κατακόψει τας εκλεκτάς κέδρους σου και ρίψει εις το πυρ.
\par 8 Και πολλά έθνη θέλουσι διαβή διά της πόλεως ταύτης και θέλουσιν ειπεί έκαστος προς τον πλησίον αυτού, Διά τι ο Κύριος έκαμεν ούτως εις ταύτην την μεγάλην πόλιν;
\par 9 Και θέλουσιν αποκριθή, Διότι εγκατέλιπον την διαθήκην Κυρίου του Θεού αυτών και προσεκύνησαν άλλους θεούς και ελάτρευσαν αυτούς.
\par 10 Μη κλαίετε τον αποθανόντα και μη θρηνείτε αυτόν· κλαύσατε πικρώς τον εξερχόμενον, διότι δεν θέλει επιστρέψει πλέον και ιδεί την γην της γεννήσεως αυτού.
\par 11 Διότι ούτω λέγει Κύριος περί του Σαλλούμ, υιού του Ιωσίου, βασιλέως του Ιούδα, του βασιλεύοντος αντί Ιωσίου του πατρός αυτού, όστις εξήλθεν εκ του τόπου τούτου· Δεν θέλει επιστρέψει εκεί πλέον,
\par 12 αλλά θέλει αποθάνει εν τω τόπω όπου έφεραν αυτόν αιχμάλωτον, και δεν θέλει ιδεί πλέον την γην ταύτην.
\par 13 Ουαί εις τον οικοδομούντα τον οίκον αυτού ουχί εν δικαιοσύνη και τα υπερώα αυτού ουχί εν ευθύτητι, τον μεταχειριζόμενον την εργασίαν του πλησίον αυτού αμισθί και μη αποδίδοντα εις αυτόν τον μισθόν του κόπου αυτού,
\par 14 τον λέγοντα, Θέλω οικοδομήσει εις εμαυτόν οίκον μέγαν και υπερώα ευρύχωρα, και ανοίγοντα εις εαυτόν παράθυρα και στεγάζοντα με κέδρον και χρωματίζοντα με μίλτον.
\par 15 Θέλεις βασιλεύει, διότι εγκλείεις σεαυτόν εις κέδρον; ο πατήρ σου δεν έτρωγε και έπινε, και επειδή έκαμνε κρίσιν και δικαιοσύνην, ευημέρει;
\par 16 Έκρινε την κρίσιν του πτωχού και του πένητος και τότε ευημέρει· δεν ήτο τούτο να με γνωρίζη; λέγει Κύριος.
\par 17 Αλλ' οι οφθαλμοί σου και η καρδία σου δεν είναι παρά εις την πλεονεξίαν σου και εις το να εκχέης αίμα αθώον και εις την δυναστείαν και εις την βίαν, διά να κάμνης ταύτα.
\par 18 Διά τούτο ούτω λέγει Κύριος περί του Ιωακείμ, υιού του Ιωσίου, βασιλέως του Ιούδα· Δεν θέλουσι κλαύσει αυτόν, λέγοντες, Ουαί αδελφέ μου Ουαί αδελφή δεν θέλουσι κλαύσει αυτόν, λέγοντες, Ουαί κύριε ή, Ουαί δόξα
\par 19 Θέλει ταφή ταφήν όνου, συρόμενος και ριπτόμενος πέραν των πυλών της Ιερουσαλήμ.
\par 20 Ανάβηθι εις τον Λίβανον και βόησον και ύψωσον την φωνήν σου προς την Βασάν και βόησον από Αβαρίμ· διότι ηφανίσθησαν πάντες οι ερασταί σου.
\par 21 Ελάλησα προς σε εν τη ευημερία σου, αλλ' είπας, Δεν θέλω ακούσει· ούτος εστάθη ο τρόπος σου εκ νεότητός σου, ότι δεν υπήκουσας εις την φωνήν μου.
\par 22 Ο άνεμος θέλει καταβοσκήσει πάντας τους ποιμένας σου και οι ερασταί σου θέλουσιν υπάγει εις αιχμαλωσίαν· τότε, ναι, θέλεις αισχυνθή και εντραπή διά πάσας τας ασεβείας σου.
\par 23 Συ ήτις κατοικείς εν τω Λιβάνω, ήτις κάμνεις την φωλεάν σου εν ταις κέδροις, πόσον αξιοθρήνητος θέλεις είσθαι, όταν έλθωσι λύπαι επί σε, ωδίνες ως τικτούσης.
\par 24 Ζω εγώ, λέγει Κύριος, και εάν ο Χονίας, ο υιός του Ιωακείμ, βασιλεύς του Ιούδα, ήθελε γείνει σφραγίς επί την δεξιάν μου χείρα, και εκείθεν ήθελον σε αποσπάσει·
\par 25 και θέλω σε παραδώσει εις την χείρα των ζητούντων την ψυχήν σου και εις την χείρα εκείνων, των οποίων το πρόσωπον φοβείσαι, ναι, εις την χείρα του Ναβουχοδονόσορ βασιλέως της Βαβυλώνος και εις την χείρα των Χαλδαίων.
\par 26 Και θέλω απορρίψει σε και την μητέρα σου, ήτις σε εγέννησεν, εις γην ξένην όπου δεν εγεννήθητε, και εκεί θέλετε αποθάνει.
\par 27 Εις δε την γην, εις την οποίαν η ψυχή αυτών επιθυμεί να επιστρέψωσιν, εκεί δεν θέλουσιν επιστρέψει.
\par 28 Ο άνθρωπος ούτος ο Χονίας κατεστάθη είδωλον καταπεφρονημένον και συντετριμμένον; σκεύος, εν ω δεν υπάρχει χάρις; διά τι απεβλήθησαν, αυτός και το σπέρμα αυτού, και ερρίφθησαν εις τόπον, τον οποίον δεν γνωρίζουσιν;
\par 29 Ω γη, γη, γη, άκουε τον λόγον του Κυρίου.
\par 30 Ούτω λέγει Κύριος· Γράψατε τον άνθρωπον τούτον άτεκνον, άνθρωπον όστις δεν θέλει ευοδοθή εν ταις ημέραις αυτού· διότι δεν θέλει ευοδοθή εκ του σπέρματος αυτού άνθρωπος καθήμενος επί τον θρόνον του Δαβίδ και εξουσιάζων πλέον επί του Ιούδα.

\chapter{23}

\par Ουαί εις τους ποιμένας τους φθείροντας και διασκορπίζοντας τα πρόβατα της βοσκής μου, λέγει Κύριος.
\par 2 Διά τούτο ούτω λέγει Κύριος, ο Θεός του Ισραήλ, κατά των ποιμένων, οίτινες ποιμαίνουσι τον λαόν μου. Σεις διεσκορπίσατε τα πρόβατά μου και απεδιώξατε αυτά και δεν επεσκέφθητε αυτά· ιδού, εγώ θέλω επισκεφθή εφ' υμάς την κακίαν των έργων υμών, λέγει Κύριος.
\par 3 Και εγώ θέλω συνάξει το υπόλοιπον των προβάτων μου εκ πάντων των τόπων, όπου εδίωξα αυτά, και θέλω επιστρέψει αυτά πάλιν εις τας βοσκάς αυτών, και θέλουσι καρποφορήσει και πληθυνθή·
\par 4 και θέλω καταστήσει ποιμένας επ' αυτά και θέλουσι ποιμαίνει αυτά· και δεν θέλουσι φοβηθή πλέον ουδέ τρομάξει ουδέ εκλείψει, λέγει Κύριος.
\par 5 Ιδού, έρχονται ημέραι, λέγει Κύριος, και θέλω ανεγείρει εις τον Δαβίδ βλαστόν δίκαιον, και βασιλεύς θέλει βασιλεύσει και ευημερήσει και εκτελέσει κρίσιν και δικαιοσύνην επί της γης.
\par 6 Εν ταις ημέραις αυτού ο Ιούδας θέλει σωθή και ο Ισραήλ θέλει κατοικήσει εν ασφαλεία· και τούτο είναι το όνομα αυτού, με το οποίον θέλει ονομασθή, Ο Κύριος η δικαιοσύνη ημών.
\par 7 Διά τούτο, ιδού, έρχονται ημέραι, λέγει Κύριος, και δεν θέλουσιν ειπεί πλέον, Ζη ο Κύριος, όστις ανήγαγε τους υιούς Ισραήλ εκ γης Αιγύπτου·
\par 8 αλλά, Ζη ο Κύριος, όστις ανήγαγε και όστις έφερε το σπέρμα του οίκου Ισραήλ εκ της γης του βορρά και εκ πάντων των τόπων, όπου είχα διώξει αυτούς· και θέλουσι κατοικήσει εν τη γη αυτών.
\par 9 Ένεκεν των προφητών η καρδία μου συντρίβεται εντός μου· σαλεύονται πάντα τα οστά μου· είμαι ως άνθρωπος μεθύων και ως άνθρωπος συνεχόμενος υπό οίνου, εξ αιτίας του Κυρίου και εξ αιτίας των λόγων της αγιότητος αυτού.
\par 10 Διότι η γη είναι πλήρης μοιχών· διότι εξ αιτίας του όρκου πενθεί η γή· εξηράνθησαν αι βοσκαί της ερήμου και η οδός αυτών έγεινε πονηρά και η δύναμις αυτών άδικος.
\par 11 Διότι και ο προφήτης και ο ιερεύς εμολύνθησαν· ναι, εν τω οίκω μου εύρηκα τας ασεβείας αυτών, λέγει Κύριος.
\par 12 Διά τούτο η οδός αυτών θέλει είσθαι εις αυτούς ως ολίσθημα εν τω σκότει· θέλουσιν ωθηθή και πέσει εν αυτή· διότι θέλω φέρει κακόν επ' αυτούς εν τω ενιαυτώ της επισκέψεως αυτών, λέγει Κύριος.
\par 13 Είδον μεν αφροσύνην εν τοις προφήταις της Σαμαρείας. προεφήτευσαν διά του Βάαλ και επλάνων τον λαόν μου τον Ισραήλ.
\par 14 αλλ' εν τοις προφήταις της Ιερουσαλήμ είδον φρίκην· μοιχεύουσι και περιπατούσιν εν ψεύδει και ενισχύουσι τας χείρας των κακούργων, ώστε ουδείς επιστρέφει από της κακίας αυτού· πάντες ούτοι είναι εις εμέ ως Σόδομα και οι κάτοικοι αυτής ως Γόμορρα.
\par 15 Διά τούτο ούτω λέγει ο Κύριος των δυνάμεων κατά των προφητών· Ιδού, εγώ θέλω ψωμίσει αυτούς αψίνθιον και θέλω ποτίσει αυτούς ύδωρ χολής· διότι εκ των προφητών της Ιερουσαλήμ εξήλθεν ο μολυσμός εις άπαντα τον τόπον.
\par 16 Ούτω λέγει ο Κύριος των δυνάμεων· Μη ακούετε τους λόγους των προφητών των προφητευόντων εις εσάς· ούτοι σας κάμνουσι ματαίους· λαλούσιν οράσεις από της καρδίας αυτών, ουχί από στόματος Κυρίου.
\par 17 Λέγουσι πάντοτε προς τους καταφρονούντάς με, Ο Κύριος είπεν, Ειρήνη θέλει είσθαι εις εσάς· και λέγουσι προς πάντα περιπατούντα κατά τας ορέξεις της καρδίας αυτού, Δεν θέλει ελθεί κακόν εφ' υμάς·
\par 18 διότι τις παρεστάθη εν τη βουλή του Κυρίου και είδε και ήκουσε τον λόγον αυτού; τις επρόσεξεν εις τον λόγον αυτού και ήκουσεν;
\par 19 Ιδού, ανεμοστρόβιλος παρά Κυρίου εξήλθε με ορμήν, και ανεμοστρόβιλος ορμητικός θέλει εξορμήσει επί την κεφαλήν των ασεβών.
\par 20 Ο θυμός του Κυρίου δεν θέλει αποστρέψει εωσού εκτελέση και εωσού κάμη τους στοχασμούς της καρδίας αυτού· εν ταις εσχάταις ημέραις θέλετε νοήσει τούτο εντελώς.
\par 21 Δεν απέστειλα τους προφήτας τούτους και αυτοί έτρεξαν· δεν ελάλησα προς αυτούς και αυτοί προεφήτευσαν·
\par 22 αλλ' εάν ήθελον παρασταθή εν τη βουλή μου, τότε ήθελον κάμει τον λαόν μου να ακούση τους λόγους μου, και ήθελον αποστρέψει αυτούς από της πονηράς οδού αυτών και από της κακίας των έργων αυτών.
\par 23 Θεός εγγύθεν είμαι εγώ, λέγει Κύριος, και ουχί Θεός μακρόθεν;
\par 24 Δύναταί τις να κρυφθή εν κρυφίοις τόποις και εγώ να μη ίδω αυτόν; λέγει Κύριος. Δεν πληρώ εγώ τον ουρανόν και την γην; λέγει Κύριος.
\par 25 Ήκουσα τι λέγουσιν οι προφήται, οι προφητεύοντες εν τω ονόματί μου ψεύδος, λέγοντες, Είδον ενύπνιον, είδον ενύπνιον.
\par 26 Έως πότε θέλει είσθαι τούτο εν τη καρδία των προφητών των προφητευόντων ψεύδος; ναι, προφητεύουσι τας απάτας της καρδίας αυτών·
\par 27 οίτινες στοχάζονται να κάμωσι τον λαόν μου να λησμονήση το όνομά μου, διά των ενυπνίων αυτών τα οποία διηγούνται έκαστος προς τον πλησίον αυτού, καθώς ελησμόνησαν οι πατέρες αυτών το όνομά μου διά τον Βάαλ.
\par 28 Ο προφήτης εις τον οποίον είναι ενύπνιον, ας διηγηθή το ενύπνιον· και εκείνος, εις τον οποίον είναι ο λόγος μου, ας λαλήση τον λόγον μου εν αληθεία. Τι είναι το άχυρον προς τον σίτον; λέγει Κύριος.
\par 29 Δεν είναι ο λόγος μου ως πυρ; λέγει ο Κύριος· και ως σφύρα κατασυντρίβουσα τον βράχον;
\par 30 Διά τούτο, ιδού, εγώ είμαι εναντίον των προφητών, λέγει Κύριος, οίτινες κλέπτουσι τους λόγους μου, έκαστος από του πλησίον αυτού.
\par 31 Ιδού, εγώ είμαι εναντίον των προφητών, λέγει Κύριος, οίτινες κινούσι τας γλώσσας αυτών και λέγουσιν, Αυτός λέγει.
\par 32 Ιδού, εγώ είμαι εναντίον των προφητευόντων ενύπνια ψευδή, λέγει Κύριος, οίτινες διηγούνται αυτά και πλανώσι τον λαόν μου με τα ψεύδη αυτών και με την αφροσύνην αυτών· ενώ εγώ δεν απέστειλα αυτούς ουδέ προσέταξα αυτούς· διά τούτο ουδόλως θέλουσιν ωφελήσει τον λαόν τούτον, λέγει Κύριος.
\par 33 Και εάν ο λαός ούτος ή ο προφήτης ή ο ιερεύς σε ερωτήσωσι, λέγοντες, Τι είναι το φορτίον του Κυρίου; τότε θέλεις ειπεί προς αυτούς, Τι το φορτίον; θέλω βεβαίως σας εγκαταλείψει, λέγει Κύριος.
\par 34 Τον δε προφήτην και τον ιερέα και τον λαόν, όστις είπη, Το φορτίον του Κυρίου, εγώ βεβαίως θέλω παιδεύσει τον άνθρωπον εκείνον και τον οίκον αυτού.
\par 35 Ούτω θέλετε ειπεί, έκαστος προς τον πλησίον αυτού και έκαστος προς τον αδελφόν αυτού· Τι απεκρίθη ο Κύριος; και, Τι ελάλησεν ο Κύριος;
\par 36 Και φορτίον Κυρίου δεν θέλετε αναφέρει πλέον· επειδή το φορτίον θέλει είσθαι εις έκαστον ο λόγος αυτού· διότι διεστρέψατε τους λόγους του Θεού του ζώντος, του Κυρίου των δυνάμεων, του Θεού ημών.
\par 37 Ούτω θέλεις ειπεί προς τον προφήτην· Τι σοι απεκρίθη ο Κύριος; και, Τι ελάλησεν ο Κύριος;
\par 38 Αλλ' επειδή λέγετε, Το φορτίον του Κυρίου, διά τούτο ούτω λέγει Κύριος· Επειδή λέγετε τον λόγον τούτον, Το φορτίον του Κυρίου, εγώ δε απέστειλα προς εσάς, λέγων, δεν θέλετε λέγει, Το φορτίον του Κυρίου·
\par 39 διά τούτο, Ιδού, εγώ θέλω σας λησμονήσει παντελώς και θέλω απορρίψει υμάς και την πόλιν την οποίαν έδωκα εις υμάς και εις τους πατέρας υμών, από προσώπου μου·
\par 40 και θέλω φέρει εφ' υμάς όνειδος αιώνιον, και καταισχύνην αιώνιον, ήτις δεν θέλει λησμονηθή.

\chapter{24}

\par Ο Κύριος έδειξεν εις εμέ και ιδού, δύο κάλαθοι σύκων κείμενοι έμπροσθεν του ναού του Κυρίου, αφού ηχμαλώτισε Ναβουχοδονόσορ ο βασιλεύς της Βαβυλώνος Ιεχονίαν τον υιόν του Ιωακείμ, βασιλέα του Ιούδα, και τους άρχοντας του Ιούδα και τους ξυλουργούς και τους χαλκείς εξ Ιερουσαλήμ και έφερεν αυτούς εις την Βαβυλώνα.
\par 2 Ο κάλαθος ο εις είχε σύκα κάλλιστα, ως τα σύκα τα πρώϊμα· ο δε κάλαθος άλλος σύκα κάκιστα, τα οποία διά την αχρειότητα δεν ετρώγοντο.
\par 3 Και είπε Κύριος προς εμέ, Τι βλέπεις, Ιερεμία; Και είπα, Σύκα· τά σύκα τα καλά είναι κάλλιστα, τα δε κακά κάκιστα, ώστε διά την αχρειότητα δεν τρώγονται.
\par 4 Πάλιν έγεινε λόγος Κυρίου προς εμέ λέγων,
\par 5 Ούτω λέγει Κύριος ο Θεός του Ισραήλ· Καθώς τα καλά ταύτα σύκα, ούτω θέλω επιμεληθή τους αιχμαλωτισθέντας εκ του Ιούδα, τους οποίους εξαπέστειλα εκ του τόπου τούτου εις την γην των Χαλδαίων διά καλόν.
\par 6 Διότι θέλω επιστηρίξει τους οφθαλμούς μου επ' αυτούς διά καλόν, και θέλω αποκαταστήσει αυτούς εν τη γη ταύτη και οικοδομήσει αυτούς και δεν θέλω κατακρηνίσει, και θέλω φυτεύσει αυτούς και εν θέλω εκριζώσει.
\par 7 Και θέλω δώσει εις αυτούς καρδίαν διά να με γνωρίσωσιν, ότι εγώ είμαι ο Κύριος· και θέλουσιν είσθαι λαός μου και εγώ θέλω είσθαι Θεός αυτών· διότι θέλουσιν επιτρέψει εις εμέ εξ όλης καρδίας αυτών.
\par 8 Και καθώς τα σύκα τα κακά, τα οποία διά την αχρειότητα δεν τρώγονται, ούτω βεβαίως λέγει Κύριος, Ούτω θέλω παραδώσει Σεδεκίαν τον βασιλέα του Ιούδα και τους μεγιστάνας αυτού και το υπόλοιπον της Ιερουσαλήμ, το εναπολειφθέν εν τη γη ταύτη, και τους κατοικούντας εν τη γη της Αιγύπτου·
\par 9 και θέλω παραδώσει αυτούς εις διασποράν εις πάντα τα βασίλεια της γης προς κακόν, εις όνειδος και εις παροιμίαν, εις λοιδορίαν και εις κατάραν, εν πάσι τοις τόποις όπου θέλω διώξει αυτούς.
\par 10 Και θέλω αποστείλει προς αυτούς την μάχαιραν, την πείναν και τον λοιμόν, εωσού αφανισθώσιν επάνωθεν από της γης, την οποίαν έδωκα εις αυτούς και εις τους πατέρας αυτών.

\chapter{25}

\par Ο λόγος ο γενόμενος προς τον Ιερεμίαν περί παντός του λαού του Ιούδα εν τω τετάρτω έτει του Ιωακείμ υιού του Ιωσίου, βασιλέως τον Ιούδα, το οποίον ήτο το πρώτον έτος του Ναβουχοδονόσορ, βασιλέως της Βαβυλώνος·
\par 2 τον οποίον Ιερεμίας ο προφήτης ελάλησε προς πάντα τον λαόν του Ιούδα και προς πάντας τους κατοίκους της Ιερουσαλήμ, λέγων,
\par 3 Από του δεκάτου τρίτου έτους του Ιωσίου· υιού του Αμών, βασιλέως του Ιούδα, έως της ημέρας ταύτης, ήτις είναι το εικοστόν τρίτον έτος, ο λόγος του Κυρίου έγεινε προς εμέ και ελάλησα προς εσάς, εγειρόμενος πρωΐ και λαλών· και δεν ηκούσατε.
\par 4 Και απέστειλε Κύριος προς εσάς πάντας τους δούλους αυτού τους προφήτας, εγειρόμενος πρωΐ και αποστέλλων· και δεν ηκούσατε ουδέ εκλίνατε το ωτίον σας διά να ακροασθήτε.
\par 5 Οίτινες είπον, Στράφητε τώρα έκαστος από της οδού αυτού της πονηράς και από της κακίας των έργων σας, και κατοικήσατε επί της γης, την οποίαν ο Κύριος έδωκεν εις εσάς και εις τους πατέρας σας εις τον αιώνα του αιώνος·
\par 6 και μη πορεύεσθε οπίσω άλλων θεών, διά να λατρεύητε και να προσκυνήτε αυτούς, και μη με παροργίζετε με τα έργα των χειρών σας και δεν θέλω σας κάμει κακόν.
\par 7 Αλλά δεν μου ηκούσατε, λέγει Κύριος· διά να με παροργίσητε με τα έργα των χειρών σας προς κακόν σας.
\par 8 Διά τούτο ούτω λέγει ο Κύριος των δυνάμεων· Επειδή δεν ηκούσατε τους λόγους μου,
\par 9 ιδού, εγώ θέλω αποστείλει και λάβει πάσας τας οικογενείας του βορρά, λέγει Κύριος, και τον Ναβουχοδονόσορ βασιλέα της Βαβυλώνος· τον δούλον μου, και θέλω φέρει αυτούς επί την γην ταύτην και επί τους κατοίκους αυτής και επί πάντα ταύτα τα έθνη κύκλω, και θέλω εξολοθρεύσει αυτούς και καταστήσει αυτούς έκπληξιν και ερημώσεις αιωνίους.
\par 10 Και θέλω αφαιρέσει απ' αυτών την φωνήν της χαράς και την φωνήν της ευφροσύνης, την φωνήν του νυμφίου και την φωνήν της νύμφης, τον ήχον των μυλοπετρών και το φως του λύχνου.
\par 11 Και πάσα αύτη η γη θέλει είσθαι εις ερήμωσιν και θάμβος, και τα έθνη ταύτα θέλουσι δουλεύσει τον βασιλέα της Βαβυλώνος εβδομήκοντα έτη.
\par 12 Και όταν συμπληρωθώσι τα εβδομήκοντα έτη, θέλω ανταποδώσει επί τον βασιλέα της Βαβυλώνος και επί το έθνος εκείνο, λέγει Κύριος, την ανομίαν αυτών, και επί την γην των Χαλδαίων, και θέλω καταστήσει αυτήν ερήμωσιν αιώνιον.
\par 13 Και θέλω φέρει επί την γην εκείνην πάντας τους λόγους μου, τους οποίους ελάλησα κατ' αυτής, άπαν το γεγραμμένον εν τω βιβλίω τούτω, το οποίον ο Ιερεμίας προεφήτευσε κατά πάντων των εθνών.
\par 14 Διότι έθνη πολλά και βασιλείς μεγάλοι θέλουσι καταδουλώσει και αυτούς· και θέλω ανταποδώσει εις αυτούς κατά τας πράξεις αυτών και κατά τα έργα των χειρών αυτών.
\par 15 Διότι ούτω λέγει προς εμέ Κύριος ο Θεός του Ισραήλ, Λάβε το ποτήριον τούτο του οίνου του θυμού μου εκ της χειρός μου και πότισον εξ αυτού πάντα τα έθνη, προς τα οποία εγώ σε αποστέλλω·
\par 16 και θέλουσι πίει και θέλουσι ταραχθή και παραφρονήσει εξ αιτίας της μαχαίρας, την οποίαν εγώ θέλω αποστείλει εν μέσω αυτών.
\par 17 Τότε έλαβον το ποτήριον εκ της χειρός του Κυρίου και επότισα πάντα τα έθνη, προς τα οποία ο Κύριος με απέστειλε·
\par 18 την Ιερουσαλήμ και τας πόλεις του Ιούδα και τους βασιλείς αυτού και τους μεγιστάνας αυτού, διά να καταστήσω αυτούς ερήμωσιν, θάμβος, συριγμόν και κατάραν, καθώς την ημέραν ταύτην·
\par 19 τον Φαραώ βασιλέα της Αιγύπτου και τους δούλους αυτού και τους μεγιστάνας αυτού και άπαντα τον λαόν αυτού,
\par 20 και πάντα τον σύμμικτον λαόν και πάντας τους βασιλείς της γης Ουζ και πάντας τους βασιλείς της γης των Φιλισταίων και την Ασκάλωνα και την Γάζαν και την Ακκαρών και το υπόλοιπον της Αζώτου,
\par 21 τον Εδώμ και τον Μωάβ και τους υιούς Αμμών,
\par 22 και πάντας τους βασιλείς της Τύρου και πάντας τους βασιλείς της Σιδώνος και τους βασιλείς των νήσων των πέραν της θαλάσσης,
\par 23 την Δαιδάν και την Θαιμά και την Βουζ και πάντας τους περικείροντας την κόμην·
\par 24 και πάντας τους βασιλείς της Αραβίας και πάντας τους βασιλείς των συμμίκτων λαών των κατοικούντων εν τη ερήμω,
\par 25 και πάντας τους βασιλείς της Ζιμβρί και πάντας τους βασιλείς της Ελάμ και πάντας τους βασιλείς των Μήδων,
\par 26 και πάντας τους βασιλείς του βορρά τους μακράν και τους εγγύς, ένα μετά του άλλου, και πάντα τα βασίλεια της οικουμένης, τα επί προσώπου της γής· και ο βασιλεύς της Σησάχ θέλει πίει μετ' αυτούς.
\par 27 Διά τούτο θέλεις ειπεί προς αυτούς, Ούτω λέγει ο Κύριος των δυνάμεων, ο Θεός του Ισραήλ· Πίετε και μεθύσατε και εμέσατε και πέσετε και μη σηκωθήτε, εξ αιτίας της μαχαίρας, την οποίαν εγώ θέλω αποστείλει εν μέσω υμών.
\par 28 Και εάν δεν θελήσωσι να λάβωσι το ποτήριον εκ της χειρός σου διά να πίωσι, τότε θέλεις ειπεί προς αυτούς, Ούτω λέγει ο Κύριος των δυνάμεων· Εξάπαντος θέλετε πίει.
\par 29 Διότι ιδού, ενώ εγώ επί την πόλιν, επί της οποίας εκλήθη το όνομά μου, αρχίζω να φέρω κακόν, σεις θέλετε μείνει λοιπόν ατιμώρητοι; δεν θέλετε μείνει ατιμώρητοι, διότι εγώ θέλω καλέσει μάχαιραν επί πάντας τους κατοίκους της γης, λέγει ο Κύριος των δυνάμεων.
\par 30 Διά τούτο, συ προφήτευσον κατ' αυτών πάντας τους λόγους τούτους και ειπέ προς αυτούς, Ο Κύριος θέλει βρυχήσει εξ ύψους και εκπέμψει την φωνήν αυτού από της κατοικίας της αγιότητος αυτού· θέλει βρυχήσει δυνατά επί της κατοικίας αυτού· θέλει βοήσει ως οι ληνοπατούντες κατά πάντων των κατοίκων της γης.
\par 31 Θόρυβος θέλει φθάσει έως των περάτων της γής· διότι ο Κύριος έχει κρίσιν μετά των εθνών· αυτός διαδικάζεται μετά πάσης σαρκός· θέλει παραδώσει τους ασεβείς εις μάχαιραν, λέγει Κύριος.
\par 32 Ούτω λέγει ο Κύριος των δυνάμεων· Ιδού, κακόν θέλει εξέλθει από έθνους εις έθνος και ανεμοστρόβιλος μέγας θέλει εγερθή εκ των άκρων της γης.
\par 33 Και εν εκείνη τη ημέρα θέλουσι κοίτεσθαι τεθανατωμένοι παρά Κυρίου απ' άκρου της γης έως άκρου της γής· δεν θέλουσι θρηνολογηθή ουδέ συναχθή ουδέ ταφή· θέλουσιν είσθαι διά κοπρίαν επί της επιφανείας της γης.
\par 34 Ολολύξατε, ποιμένες, και αναβοήσατε· και κυλίσθητε εις το χώμα, οι έγκριτοι του ποιμνίου· διότι επληρώθησαν αι ημέραι σας διά την σφαγήν και διά τον σκορπισμόν σας, και θέλετε πέσει ως σκεύος εκλεκτόν.
\par 35 Και θέλει λείψει η φυγή από των ποιμένων και η σωτηρία από των εγκρίτων του ποιμνίου.
\par 36 Φωνή κραυγής των ποιμένων και ολολυγμός των εγκρίτων του ποιμνίου· διότι ο Κύριος ηφάνισε την βοσκήν αυτών.
\par 37 Και αι ειρηνικαί κατοικίαι κατηδαφίσθησαν εξ αιτίας της φλογεράς οργής του Κυρίου.
\par 38 Κατέλιπε το κατοικητήριον αυτού ως ο λέων, διότι η γη αυτών κατεστάθη έρημος εξ αιτίας της αγριότητος του καταδυναστεύοντος και εξ αιτίας του θυμού της οργής αυτού.

\chapter{26}

\par Εν τη αρχή της βασιλείας του Ιωακείμ υιού του Ιωσίου, βασιλέως του Ιούδα, έγεινεν ο λόγος ούτος παρά Κυρίου, λέγων,
\par 2 Ούτω λέγει Κύριος· Στήθι εν τη αυλή του οίκου του Κυρίου και λάλησον προς πάσας τας πόλεις του Ιούδα τας ερχομένας διά να προσκυνήσωσιν εν τω οίκω του Κυρίου, πάντας τους λόγους, τους οποίους προσέταξα εις σε να λαλήσης προς αυτούς· μη αφαιρέσης λόγον.
\par 3 Ίσως θέλουσιν ακούσει και επιστρέψει έκαστος από της οδού αυτού της πονηράς και μετανοήσω περί του κακού, το οποίον βουλεύομαι να κάμω εις αυτούς διά την κακίαν των έργων αυτών.
\par 4 Και θέλεις ειπεί προς αυτούς, Ούτω λέγει Κύριος· Εάν δεν μου ακούσητε, ώστε να περιπατήτε εν τω νόμω μου, τον οποίον έθεσα έμπροσθέν σας,
\par 5 να υπακούητε εις τους λόγους των δούλων μου των προφητών, τους οποίους απέστειλα προς εσάς εγειρόμενος πρωΐ και αποστέλλων, πλην σεις δεν ηκούσατε,
\par 6 τότε θέλω καταστήσει τον οίκον τούτον ως την Σηλώ, και την πόλιν ταύτην θέλω καταστήσει κατάραν εις πάντα τα έθνη της γης.
\par 7 Και ήκουσαν οι ιερείς και οι προφήται και πας ο λαός τον Ιερεμίαν, λαλούντα τους λόγους τούτους εν τω οίκω του Κυρίου.
\par 8 Και αφού ο Ιερεμίας έπαυσε λαλών πάντα όσα προσέταξεν εις αυτόν ο Κύριος να λαλήση προς πάντα τον λαόν, οι ιερείς και οι προφήται και πας ο λαός συνέλαβον αυτόν λέγοντες, Θέλεις εξάπαντος θανατωθή·
\par 9 διά τι προεφήτευσας εν ονόματι Κυρίου λέγων, Ο οίκος ούτος θέλει είσθαι ως η Σηλώ και η πόλις αύτη θέλει ερημωθή· ώστε να μη ήναι ο κατοικών; Και πας ο λαός συνήχθη κατά του Ιερεμίου εν τω οίκω του Κυρίου.
\par 10 Και ακούσαντες οι άρχοντες του Ιούδα τα πράγματα ταύτα, ανέβησαν εκ του οίκου του βασιλέως εις τον οίκον του Κυρίου και εκάθησαν εν τη εισόδω της νέας πύλης του Κυρίου.
\par 11 Τότε οι ιερείς και οι προφήται ελάλησαν προς τους άρχοντας και προς πάντα τον λαόν λέγοντες, Κρίσις θανάτου πρέπει εις τον άνθρωπον τούτον, διότι προεφήτευσε κατά της πόλεως ταύτης, ως ηκούσατε με τα ώτα σας.
\par 12 Και ελάλησεν ο Ιερεμίας προς πάντας τους άρχοντας και προς πάντα τον λαόν λέγων, Ο Κύριος με απέστειλε διά να προφητεύσω κατά του οίκου τούτου και κατά της πόλεως ταύτης πάντας τους λόγους τους οποίους ηκούσατε.
\par 13 Διά τούτο τώρα διορθώσατε τας οδούς υμών και τας πράξεις υμών και υπακούσατε εις την φωνήν Κυρίου του Θεού υμών· και ο Κύριος θέλει μετανοήσει περί του κακού, το οποίον ελάλησε καθ' υμών.
\par 14 Εγώ δε, ιδού, είμαι εν ταις χερσίν υμών· κάμετε εις εμέ, όπως είναι καλόν και όπως αρεστόν εις τους οφθαλμούς υμών.
\par 15 Πλην εξεύρετε μετά βεβαιότητος, ότι εάν με θανατώσητε, αίμα αθώον θέλετε βεβαίως φέρει εφ' υμάς και επί την πόλιν ταύτην και επί τους κατοίκους αυτής· διότι τη αληθεία ο Κύριος με απέστειλε προς υμάς, διά να λαλήσω εις τα ώτα υμών πάντας τους λόγους τούτους.
\par 16 Τότε οι άρχοντες και άπας ο λαός είπον προς τους ιερείς και προς τους προφήτας, δεν υπάρχει κρίσις θανάτου εις τον άνθρωπον τούτον· διότι εν τω ονόματι Κυρίου του Θεού ημών ελάλησε προς ημάς.
\par 17 Τότε εσηκώθησαν τινές εκ των πρεσβυτέρων του τόπου και ελάλησαν προς άπασαν την συναγωγήν του λαού, λέγοντες,
\par 18 Ο Μιχαίας ο Μωρασθίτης προεφήτευεν εν ταις ημέραις Εζεκίου βασιλέως του Ιούδα και ελάλησε προς πάντα τον λαόν του Ιούδα λέγων, Ούτω λέγει ο Κύριος των δυνάμεων· Η Σιών θέλει αροτριασθή ως αγρός, και η Ιερουσαλήμ θέλει γείνει σωροί λίθων και το όρος του οίκου ως υψηλοί τόποι δρυμού.
\par 19 Μήπως ο Εζεκίας ο βασιλεύς του Ιούδα και πας ο Ιούδας εθανάτωσαν αυτόν; δεν εφοβήθη τον Κύριον και παρεκάλεσε το πρόσωπον του Κυρίου, και ο Κύριος μετενόησε περί του κακού, το οποίον ελάλησε κατ' αυτών; Ημείς λοιπόν ηθέλομεν προξενήσει μέγα κακόν κατά των ψυχών ημών.
\par 20 Και προσέτι υπήρξεν άνθρωπος προφητεύων εν ονόματι Κυρίου, Ουρίας ο υιός του Σεμαΐου από Κιριάθ-ιαρείμ, και προεφήτευσε κατά της πόλεως ταύτης και κατά της γης ταύτης κατά πάντας τους λόγους του Ιερεμίου.
\par 21 Και ότε ήκουσεν ο βασιλεύς Ιωακείμ και πάντες οι δυνατοί αυτού και πάντες οι άρχοντες τους λόγους αυτού, ο βασιλεύς εζήτει να θανατώση αυτόν· ακούσας δε ο Ουρίας εφοβήθη και έφυγε και υπήγεν εις την Αίγυπτον.
\par 22 Και απέστειλεν Ιωακείμ ο βασιλεύς άνδρας εις την Αίγυπτον, τον Ελναθάν υιόν του Αχβώρ και άνδρας μετ' αυτού εις την Αίγυπτον·
\par 23 και εξήγαγον τον Ουρίαν εκ της Αιγύπτου και έφεραν αυτόν προς τον βασιλέα Ιωακείμ, και επάταξεν αυτόν εν μαχαίρα και έρριψε το πτώμα αυτού εις τους τάφους του όχλου.
\par 24 Πλην η χειρ του Αχικάμ υιού του Σαφάν ήτο μετά του Ιερεμία, διά να μη παραδώσωσιν αυτόν εις την χείρα του λαού ώστε να θανατώσωσιν αυτόν.

\chapter{27}

\par Εν τη αρχή της βασιλείας του Ιωακείμ υιού του Ιωσίου, βασιλέως του Ιούδα, έγεινεν ο λόγος ούτος προς τον Ιερεμίαν παρά Κυρίου, λέγων,
\par 2 Ούτω λέγει Κύριος προς εμέ· Κάμε εις σεαυτόν δεσμά και ζυγούς και επίθες αυτά επί τον τράχηλόν σου·
\par 3 και πέμψον αυτά προς τον βασιλέα του Εδώμ και προς τον βασιλέα του Μωάβ και προς τον βασιλέα των υιών Αμμών και προς τον βασιλέα της Τύρου και προς τον βασιλέα της Σιδώνος, διά χειρός των μηνυτών των ερχομένων εις την Ιερουσαλήμ προς τον Σεδεκίαν βασιλέα του Ιούδα·
\par 4 και πρόσταξον αυτούς να είπωσι προς τους κυρίους αυτών, Ούτω λέγει ο Κύριος των δυνάμεων, ο Θεός του Ισραήλ· Ούτω θέλετε ειπεί προς τους κυρίους υμών·
\par 5 Εγώ έκαμον την γην, τον άνθρωπον και τα ζώα τα επί προσώπου της γης, διά της δυνάμεώς μου της μεγάλης και διά του βραχίονός μου του εξηπλωμένου· και έδωκα αυτήν εις όντινα ηυδόκησα.
\par 6 Και τώρα εγώ έδωκα πάντας τούτους τους τόπους εις την χείρα του Ναβουχοδονόσορ, βασιλέως της Βαβυλώνος, του δούλου μου· και αυτά τα θηρία του αγρού έδωκα εις αυτόν διά να υπηρετώσιν αυτόν.
\par 7 Και πάντα τα έθνη θέλουσι δουλεύσει εις αυτόν και εις τον υιόν αυτού και εις τον υιόν του υιού αυτού, εωσού έλθη ο καιρός της γης και αυτού, και έθνη πολλά και βασιλείς μεγάλοι θέλουσι καταδουλώσει αυτόν.
\par 8 Και το έθνος και το βασίλειον, το οποίον δεν θέλει δουλεύσει εις αυτόν τον Ναβουχοδονόσορ, τον βασιλέα της Βαβυλώνος, και το οποίον δεν θέλει βάλει τον τράχηλον αυτού υπό τον ζυγόν του βασιλέως της Βαβυλώνος, το έθνος εκείνο θέλω τιμωρήσει, λέγει Κύριος, εν μαχαίρα και εν πείνη και εν λοιμώ, εωσού εξολοθρεύσω αυτό διά χειρός εκείνου.
\par 9 Και σεις, μη ακούετε τους προφήτας σας μήτε τους μάντεις σας μήτε τους ενυπνιαστάς σας μήτε τους οιωνοσκόπους σας μήτε τους μάγους σας, οίτινες λαλούσι προς εσάς, λέγοντες, δεν θέλετε δουλεύσει εις τον βασιλέα της Βαβυλώνος·
\par 10 διότι αυτοί προφητεύουσι ψεύδος προς εσάς, διά να σας απομακρύνωσιν από της γης σας, και διά να σας διώξω και να απολεσθήτε.
\par 11 το δε έθνος, το οποίον υποβάλη τον τράχηλον αυτού υπό τον ζυγόν του βασιλέως της Βαβυλώνος και δουλεύση εις αυτόν, εκείνο θέλω αφήσει να μένη ακόμη εν τη γη αυτού, λέγει Κύριος· και θέλει εργάζεσθαι αυτήν και κατοικεί εν αυτή.
\par 12 Ελάλησα και προς τον Σεδεκίαν βασιλέα του Ιούδα κατά πάντας τους λόγους τούτους, λέγων, Φέρετε τους τραχήλους σας υπό τον ζυγόν του βασιλέως της Βαβυλώνος και δουλεύσατε εις αυτόν και εις τον λαόν αυτού, και θέλετε ζήσει.
\par 13 Διά τι θέλετε να αποθάνητε, συ και ο λαός σου, εν μαχαίρα, εν πείνη και εν λοιμώ, καθώς ο Κύριος ελάλησε κατά του έθνους, το οποίον δεν δουλεύση εις τον βασιλέα της Βαβυλώνος;
\par 14 Διά τούτο, μη ακούετε τους λόγους των προφητών, οίτινες λαλούσι προς εσάς, λέγοντες· Δεν θέλετε δουλεύσει εις τον βασιλέα της Βαβυλώνος· διότι αυτοί προφητεύουσιν εις εσάς ψεύδος.
\par 15 Διότι εγώ δεν απέστειλα αυτούς, λέγει Κύριος, και αυτοί προφητεύουσι ψευδώς εν τω ονόματί μου· διά να σας διώξω και να απολεσθήτε, σεις και οι προφήται οι προφητεύοντες προς εσάς.
\par 16 Ελάλησα και προς τους ιερείς και προς πάντα τούτον τον λαόν λέγων, Ούτω λέγει Κύριος· Μη ακούετε τους λόγους των προφητών σας, οίτινες προφητεύουσι προς εσάς, λέγοντες, Ιδού, τα σκεύη του οίκου του Κυρίου θέλουσιν επανακομισθή εντός ολίγου από της Βαβυλώνος· διότι αυτοί προφητεύουσι ψεύδος προς εσάς.
\par 17 Μη ακούετε αυτούς· δουλεύσατε εις τον βασιλέα της Βαβυλώνος και θέλετε ζήσει· διά τι η πόλις αύτη να ερημωθή;
\par 18 Εάν δε αυτοί ήναι προφήται και εάν ο λόγος του Κυρίου ήναι μετ' αυτών, ας ικετεύσωσι τώρα τον Κύριον των δυνάμεων, ώστε τα σκεύη τα εναπολειφθέντα εν τω οίκω του Κυρίου και τω οίκω του βασιλέως του Ιούδα και εν Ιερουσαλήμ να μη υπάγωσιν εις την Βαβυλώνα.
\par 19 Διότι ούτω λέγει ο Κύριος των δυνάμεων περί των στύλων και περί της θαλάσσης και περί των βάσεων και περί του υπολοίπου των σκευών των εναπολειφθέντων εν τη πόλει ταύτη,
\par 20 τα οποία Ναβουχοδονόσορ ο βασιλεύς της Βαβυλώνος δεν έλαβεν, ότε έφερεν αιχμάλωτον τον Ιεχονίαν, υιόν του Ιωακείμ βασιλέως του Ιούδα, από Ιερουσαλήμ εις Βαβυλώνα και πάντας τους άρχοντας του Ιούδα και της Ιερουσαλήμ·
\par 21 μάλιστα ούτω λέγει ο Κύριος των δυνάμεων, ο Θεός του Ισραήλ, περί των σκευών των εναπολειφθέντων εν τω οίκω του Κυρίου και τω οίκω του βασιλέως του Ιούδα και εν Ιερουσαλήμ·
\par 22 αυτά θέλουσι μετακομισθή εις την Βαβυλώνα και θέλουσιν είσθαι εκεί έως της ημέρας καθ' ην θέλω επισκεφθή αυτά; λέγει Κύριος· τότε θέλω επαναφέρει αυτά και αποκαταστήσει αυτά εις τον τόπον τούτον.

\chapter{28}

\par Και εν τω αυτώ έτει, εν τη αρχή της βασιλείας του Σεδεκίου βασιλέως του Ιούδα, εν τω τετάρτω έτει, εν τω πέμπτω μηνί, Ανανίας ο υιός του Αζώρ ο προφήτης, ο από Γαβαών, ελάλησε προς εμέ εν τω οίκω του Κυρίου ενώπιον των ιερέων και παντός του λαού, λέγων,
\par 2 Ούτως είπεν ο Κύριος των δυνάμεων, ο Θεός του Ισραήλ, λέγων, Συνέτριψα τον ζυγόν του βασιλέως της Βαβυλώνος.
\par 3 Εν τω διαστήματι δύο ολοκλήρων ετών θέλω επαναφέρει εις τον τόπον τούτον πάντα τα σκεύη του οίκου του Κυρίου, τα οποία Ναβουχοδονόσορ ο βασιλεύς της Βαβυλώνος έλαβεν εκ του τόπου τούτου και έφερεν αυτά εις την Βαβυλώνα·
\par 4 και εις τον τόπον τούτον θέλω επαναφέρει, λέγει Κύριος, Ιεχονίαν τον υιόν του Ιωακείμ τον βασιλέα του Ιούδα και πάντας τους αιχμαλώτους του Ιούδα, οίτινες εφέρθησαν εις την Βαβυλώνα· διότι θέλω συντρίψει τον ζυγόν του βασιλέως της Βαβυλώνος.
\par 5 Και ελάλησεν Ιερεμίας ο προφήτης προς τον προφήτην Ανανίαν ενώπιον των ιερέων και ενώπιον παντός του λαού του παρεστώτος εν τω οίκω του Κυρίου·
\par 6 και είπεν Ιερεμίας ο προφήτης, Αμήν· ο Κύριος να κάμη ούτω ο Κύριος να εκπληρώση τους λόγους σου, τους οποίους συ προεφήτευσας, να επαναφέρη από της Βαβυλώνος εις τον τόπον τούτον τα σκεύη του οίκου του Κυρίου και παν ό,τι ηχμαλωτίσθη
\par 7 Πλην άκουσον τώρα τον λόγον τούτον, τον οποίον εγώ λαλώ εις τα ώτα σου και εις τα ώτα παντός του λαού·
\par 8 Οι προφήται, οίτινες εστάθησαν προ εμού και προ σου έκπαλαι, προεφήτευσαν και κατά πολλών τόπων και κατά μεγάλων βασιλείων, περί πολέμου και περί κακών και περί λοιμού·
\par 9 ο προφήτης, όστις προφητεύει περί ειρήνης, όταν ο λόγος του προφήτου εκπληρωθή, τότε θέλει γνωρισθή ο προφήτης, ότι αληθώς απέστειλεν αυτόν ο Κύριος.
\par 10 Τότε ο Ανανίας ο προφήτης έλαβε τον ζυγόν από του τραχήλου του προφήτου Ιερεμίου και συνέτριψεν αυτόν.
\par 11 Και ελάλησεν ο Ανανίας ενώπιον παντός του λαού, λέγων, Ούτω λέγει Κύριος· κατά τούτον τον τρόπον θέλω συντρίψει τον ζυγόν του Ναβουχοδονόσορ, βασιλέως της Βαβυλώνος, από του τραχήλου πάντων των εθνών εν τω διαστήματι δύο ολοκλήρων ετών. Και ο προφήτης Ιερεμίας υπήγε την οδόν αυτού.
\par 12 Και έγεινε λόγος Κυρίου προς Ιερεμίαν, αφού Ανανίας ο προφήτης συνέτριψε τον ζυγόν από του τραχήλου του προφήτου Ιερεμίου, λέγων,
\par 13 Ύπαγε και ειπέ προς τον Ανανίαν, λέγων, Ούτω λέγει Κύριος· Συ συνέτριψας τους ζυγούς τους ξυλίνους· αλλ' αντί τούτων θέλεις κάμει ζυγούς σιδηρούς.
\par 14 Διότι ούτω λέγει ο Κύριος των δυνάμεων, ο Θεός του Ισραήλ· Ζυγόν σιδηρούν έθεσα επί τον τράχηλον πάντων τούτων των εθνών διά να δουλεύωσιν εις τον Ναβουχοδονόσορ τον βασιλέα της Βαβυλώνος· και θέλουσι δουλεύσει εις αυτόν· και αυτά τα θηρία του αγρού έδωκα εις αυτόν.
\par 15 Τότε είπεν Ιερεμίας ο προφήτης προς τον Ανανίαν τον προφήτην, Άκουσον τώρα, Ανανία· δεν σε απέστειλεν ο Κύριος· αλλά συ κάμνεις τον λαόν τούτον να ελπίζη εις ψεύδος.
\par 16 Διά τούτο ούτω λέγει Κύριος· Ιδού, εγώ θέλω σε απορρίψει από προσώπου της γής· εν τούτω τω έτει θέλεις αποθάνει, διότι ελάλησας στασιασμόν κατά του Κυρίου.
\par 17 Και απέθανεν Ανανίας ο προφήτης εν εκείνω τω έτει, τον έβδομον μήνα.

\chapter{29}

\par Και ούτοι είναι οι λόγοι της επιστολής, την οποίαν Ιερεμίας ο προφήτης έστειλεν από Ιερουσαλήμ προς τους υπολοίπους των πρεσβυτέρων της αιχμαλωσίας και προς τους ιερείς και προς τους προφήτας και προς πάντα τον λαόν, τον οποίον ο Ναβουχοδονόσορ έφερεν αιχμάλωτον από Ιερουσαλήμ εις την Βαβυλώνα,
\par 2 αφού Ιεχονίας ο βασιλεύς και η βασίλισσα και οι ευνούχοι, οι άρχοντες του Ιούδα και της Ιερουσαλήμ και οι ξυλουργοί και οι χαλκείς εξήλθον από Ιερουσαλήμ,
\par 3 διά χειρός Ελασά υιού του Σαφάν και του Γεμαρίου υιού του Χελκίου, τους οποίους Σεδεκίας ο βασιλεύς του Ιούδα απέστειλεν εις την Βαβυλώνα προς Ναβουχοδονόσορ τον βασιλέα της Βαβυλώνος· λέγων,
\par 4 Ούτω λέγει ο Κύριος των δυνάμεων, ο Θεός του Ισραήλ, προς πάντας εκείνους, οίτινες εφέρθησαν αιχμάλωτοι, τους οποίους εγώ έκαμον να φερθώσιν αιχμάλωτοι από Ιερουσαλήμ εις την Βαβυλώνα·
\par 5 οικοδομήσατε οίκους και κατοικήσατε· και φυτεύσατε κήπους και φάγετε τον καρπόν αυτών·
\par 6 λάβετε γυναίκας και γεννήσατε υιούς και θυγατέρας· και λάβετε γυναίκας διά τους υιούς σας και δότε τας θυγατέρας σας εις άνδρας και ας γεννήσωσιν υιούς και θυγατέρας και πληθύνθητε εκεί και μη σμικρυνθήτε·
\par 7 και ζητήσατε την ειρήνην της πόλεως, όπου εγώ σας έκαμον να φερθήτε αιχμάλωτοι, και προσεύχεσθε υπέρ αυτής προς τον Κύριον· διότι εν τη ειρήνη αυτής θέλετε έχει ειρήνην.
\par 8 Διότι ούτω λέγει ο Κύριος των δυνάμεων, ο Θεός του Ισραήλ· Ας μη απατώσιν υμάς οι προφήται υμών οι εν μέσω υμών και οι μάντεις υμών, και μη ακούετε τα ενύπνια υμών τα οποία υμείς ονειρεύεσθε·
\par 9 διότι προφητεύουσι ψευδώς προς υμάς επί τω ονόματί μου· εγώ δεν απέστειλα αυτούς, λέγει Κύριος.
\par 10 Διότι ούτω λέγει Κύριος· Ότι αφού πληρωθώσιν εβδομήκοντα έτη εν Βαβυλώνι, θέλω επισκεφθή υμάς και θέλω εκτελέσει προς υμάς τον λόγον μου τον αγαθόν, να επαναφέρω υμάς εις τον τόπον τούτον.
\par 11 Διότι εγώ γνωρίζω τας βουλάς τας οποίας βουλεύομαι περί υμών, λέγει Κύριος, βουλάς ειρήνης και ουχί κακού, διά να δώσω εις υμάς το προσδοκώμενον τέλος.
\par 12 Τότε θέλετε κράξει προς εμέ και θέλετε υπάγει και προσευχηθή εις εμέ και θέλω σας εισακούσει.
\par 13 Και θέλετε με ζητήσει και ευρεί, όταν με εκζητήσητε εξ όλης της καρδίας υμών.
\par 14 Και θέλω ευρεθή από σας, λέγει Κύριος· και θέλω αποστρέψει την αιχμαλωσίαν σας και θέλω σας συνάξει εκ πάντων των εθνών και εκ πάντων των τόπων όπου σας εδίωξα, λέγει Κύριος· και θέλω σας επαναφέρει εις τον τόπον, όθεν σας έκαμον να φερθήτε αιχμάλωτοι.
\par 15 Επειδή είπετε, Ο Κύριος εσήκωσεν εις ημάς προφήτας εν Βαβυλώνι,
\par 16 γνωρίσατε, ότι ούτω λέγει Κύριος περί του βασιλέως του καθημένου επί του θρόνου Δαβίδ και περί παντός του λαού του κατοικούντος εν τη πόλει ταύτη και περί των αδελφών σας, των μη εξελθόντων μεθ' υμών εις αιχμαλωσίαν·
\par 17 ούτω λέγει ο Κύριος των δυνάμεων· Ιδού, θέλω αποστείλει επ' αυτούς την μάχαιραν, την πείναν και τον λοιμόν, και θέλω καταστήσει αυτούς ως τα σύκα τα αχρεία, τα οποία διά την αχρειότητα δεν τρώγονται.
\par 18 Και θέλω καταδιώξει αυτούς εν μαχαίρα, εν πείνη και εν λοιμώ· και θέλω παραδώσει αυτούς εις διασποράν εν πάσι τοις βασιλείοις της γης, ώστε να ήναι κατάρα και θάμβος και συριγμός και όνειδος εν πάσι τοις έθνεσιν όπου εδίωξα αυτούς·
\par 19 διότι δεν ήκουσαν τους λόγους μου, λέγει Κύριος, τους οποίους έστειλα προς αυτούς διά των δούλων μου των προφητών, εγειρόμενος πρωΐ και αποστέλλων· και δεν υπηκούσατε, λέγει Κύριος.
\par 20 Ακούσατε λοιπόν τον λόγον του Κυρίου, πάντες σεις οι αιχμαλωτισθέντες, τους οποίους εξαπέστειλα από Ιερουσαλήμ εις Βαβυλώνα.
\par 21 Ούτω λέγει ο Κύριος των δυνάμεων, ο Θεός του Ισραήλ, περί του Αχαάβ υιού του Κωλαίου και περί του Σεδεκίου υιού του Μαασίου, οίτινες προφητεύουσι ψεύδη προς εσάς εν τω ονόματί μου· Ιδού, θέλω παραδώσει αυτούς εις την χείρα του Ναβουχοδονόσορ, βασιλέως της Βαβυλώνος, και θέλει πατάξει αυτούς ενώπιόν σας.
\par 22 Και εξ αυτών θέλουσι λάβει κατάραν εν πάσι τοις αιχμαλώτοις του Ιούδα τοις εν Βαβυλώνι, λέγοντες, Ο Κύριος να σε κάμη ως τον Σεδεκίαν και ως τον Αχαάβ, τους οποίους ο βασιλεύς της Βαβυλώνος έψησεν εν πυρί·
\par 23 διότι έπραξαν αφροσύνην εν Ισραήλ και εμοίχευον τας γυναίκας των πλησίον αυτών και ελάλουν λόγους ψευδείς εν τω ονόματί μου, τους οποίους δεν προσέταξα εις αυτούς· και εγώ εξεύρω και είμαι μάρτυς, λέγει Κύριος.
\par 24 Και προς Σεμαΐαν τον Νεαιλαμίτην θέλεις λαλήσει, λέγων,
\par 25 Ούτω λέγει ο Κύριος των δυνάμεων, ο Θεός του Ισραήλ, λέγων, Επειδή συ απέστειλας επιστολάς εν τω ονόματί σου προς πάντα τον λαόν τον εν Ιερουσαλήμ και προς τον Σοφονίαν τον υιόν του Μαασίου τον ιερέα και προς πάντας τους ιερείς, λέγων,
\par 26 Ο Κύριος σε κατέστησεν ιερέα αντί Ιωδαέ του ιερέως, διά να ήσθε επιστάται εις τον οίκον του Κυρίου επί πάντα άνθρωπον μαινόμενον και προφητεύοντα, διά να βάλλης αυτόν εις φυλακήν και εις δεσμά·
\par 27 τώρα λοιπόν διά τι δεν ήλεγξας Ιερεμίαν τον εξ Αναθώθ, όστις προφητεύει εις εσάς;
\par 28 επειδή αυτός διά τούτο επέστειλε προς ημάς εις την Βαβυλώνα, λέγων, Η αιχμαλωσία αύτη είναι μακρά· οικοδομήσατε οικίας και κατοικήσατε· και φυτεύσατε κήπους και φάγετε τον καρπόν αυτών.
\par 29 Και Σοφονίας ο ιερεύς ανέγνωσε την επιστολήν ταύτην εις επήκοον του Ιερεμίου του προφήτου.
\par 30 Και έγεινε λόγος Κυρίου προς τον Ιερεμίαν, λέγων,
\par 31 Απόστειλον προς πάντας τους αιχμαλώτους, λέγων, Ούτω λέγει Κύριος περί Σεμαΐα του Νεαιλαμίτου. Επειδή ο Σεμαΐας προεφήτευσε προς εσάς και εγώ δεν απέστειλα αυτόν και σας έκαμε να ελπίζητε εις ψεύδος,
\par 32 διά τούτο ούτω λέγει Κύριος· Ιδού, θέλω επισκεφθή Σεμαΐαν τον Νεαιλαμίτην και το σπέρμα αυτού· αυτός δεν θέλει έχει άνθρωπον κατοικούντα μεταξύ του λαού τούτου, ουδέ θέλει ιδεί το καλόν, το οποίον εγώ θέλω κάμει εις τον λαόν μου, λέγει Κύριος· διότι ελάλησε στασιασμόν κατά του Κυρίου.

\chapter{30}

\par Ο λόγος ο γενόμενος προς τον Ιερεμίαν παρά Κυρίου, λέγων,
\par 2 Ούτως είπε Κύριος ο Θεός του Ισραήλ, λέγων, Γράψον εις σεαυτόν εν βιβλίω πάντας τους λόγους, τους οποίους ελάλησα προς σέ·
\par 3 διότι, ιδού, έρχονται ημέραι, λέγει Κύριος, και θέλω επιστρέψει την αιχμαλωσίαν του λαού μου Ισραήλ και Ιούδα, λέγει Κύριος· και θέλω επιστρέψει αυτούς εις την γην, την οποίαν έδωκα εις τους πατέρας αυτών, και θέλουσι κυριεύσει αυτήν.
\par 4 Και ούτοι είναι οι λόγοι, τους οποίους ελάλησε Κύριος περί του Ισραήλ και περί του Ιούδα.
\par 5 Διότι ούτω λέγει Κύριος· Ηκούσαμεν φωνήν τρομεράν, φόβον και ουχί ειρήνην.
\par 6 Ερωτήσατε τώρα και ιδέτε, εάν άρσεν τίκτη· διά τι βλέπω έκαστον άνδρα με τας χείρας αυτού επί την οσφύν αυτού, ως τίκτουσαν, και πάντα τα πρόσωπα εστράφησαν εις ωχρίασιν;
\par 7 Ουαί· διότι μεγάλη είναι η ημέρα εκείνη· ομοία αυτής δεν υπήρξε και είναι καιρός της στενοχωρίας του Ιακώβ· πλην θέλει σωθή εξ αυτής.
\par 8 Και εν τη ημέρα εκείνη, λέγει ο Κύριος των δυνάμεων, θέλω συντρίψει τον ζυγόν αυτού από του τραχήλου σου και θέλω διασπάσει τα δεσμά σου και ξένοι δεν θέλουσι πλέον καταδουλώσει αυτόν·
\par 9 αλλά θέλουσι δουλεύει Κύριον τον Θεόν αυτών και Δαβίδ τον βασιλέα αυτών, τον οποίον θέλω αναστήσει εις αυτούς.
\par 10 Συ δε μη φοβού, δούλέ μου Ιακώβ, λέγει Κύριος· μηδέ δειλιάσης, Ισραήλ· διότι, ιδού, θέλω σε σώσει από του μακρυνού τόπου και το σπέρμα σου από της γης της αιχμαλωσίας αυτών· και ο Ιακώβ θέλει επιστρέψει και θέλει ησυχάσει και αναπαυθή και δεν θέλει υπάρχει ο εκφοβών.
\par 11 Διότι εγώ είμαι μετά σου, λέγει Κύριος, διά να σε σώσω· και αν κάμω συντέλειαν πάντων των εθνών όπου σε διεσκόρπισα, εις σε όμως δεν θέλω κάμει συντέλειαν, αλλά θέλω σε παιδεύσει εν κρίσει και δεν θέλω όλως σε αθωώσει.
\par 12 Διότι ούτω λέγει Κύριος· Το σύντριμμά σου είναι ανίατον, η πληγή σου αλγεινή.
\par 13 δεν υπάρχει ο κρίνων την κρίσιν σου, ώστε να ανορθωθής· δεν υπάρχουσι διά σε φάρμακα θεραπευτικά.
\par 14 Πάντες οι αγαπητοί σου σε ελησμόνησαν· δεν σε ζητούσι· διότι σε επλήγωσα εν πληγή εχθρού, εν τιμωρία σκληρά, εξ αιτίας του πλήθους των ανομιών σου· αι αμαρτίαι σου επληθύνθησαν.
\par 15 Τι βοάς διά το σύντριμμά σου; ο πόνος σου είναι ανίατος εξ αιτίας του πλήθους των ανομιών σου· αι αμαρτίαι σου επληθύνθησαν· διά τούτο έκαμον ταύτα εις σε.
\par 16 Διά τούτο πάντες οι κατατρώγοντές σε θέλουσι καταφαγωθή· και πάντες οι εναντίοι σου, πάντες ομού θέλουσιν υπάγει εις αιχμαλωσίαν· και οι λαφυραγωγούντές σε θέλουσι γείνει λάφυρον και πάντας τους διαρπάζοντάς σε θέλω δώσει εις διαρπαγήν.
\par 17 Διότι θέλω αποκαταστήσει την υγίειαν εις σε και θέλω σε ιατρεύσει από των πληγών σου, λέγει Κύριος· διότι αυτοί σε ωνόμασαν Απερριμμένην, λέγοντες, Αύτη είναι η Σιών· δεν υπάρχει ο ζητών αυτήν.
\par 18 Ούτω λέγει Κύριος. Ιδού, εγώ θέλω επιστρέψει από της αιχμαλωσίας τας σκηνάς του Ιακώβ και θέλω οικτείρει τας κατοικίας αυτού· και η πόλις θέλει οικοδομηθή επί των ερειπίων αυτής, και ο ναός θέλει αποκατασταθή κατά την διάταξιν αυτού.
\par 19 Και εξ αυτών θέλει εξέρχεσθαι ευχαριστία και φωνή αγαλλομένων· και θέλω πολλαπλασιάσει αυτούς και δεν θέλουσιν ολιγοστεύσει· και θέλω δοξάσει αυτούς και δεν θέλουσι σμικρυνθή.
\par 20 Και τα τέκνα αυτών θέλουσιν είσθαι ως το πρότερον, και η συναγωγή αυτών θέλει στερεωθή ενώπιόν μου, και θέλω τιμωρήσει πάντας τους καταθλίβοντας αυτούς.
\par 21 Και ο άρχων αυτών θέλει είσθαι εξ αυτών και ο εξουσιαστής αυτών θέλει εξέρχεσθαι εκ μέσου αυτών· και θέλω κάμει αυτόν να πλησιάζη και θέλει πλησιάζει εις εμέ· διότι τις είναι ούτος, όστις εγγυάται την καρδίαν αυτού διά να πλησιάζη προς εμέ; λέγει Κύριος.
\par 22 Και θέλετε είσθαι λαός μου και εγώ θέλω είσθαι Θεός υμών.
\par 23 Ιδού, ανεμοστρόβιλος παρά Κυρίου εξήλθε με ορμήν, ανεμοστρόβιλος αφανίζων· θέλει εξορμήσει επί την κεφαλήν των ασεβών.
\par 24 Ο φλογερός θυμός του Κυρίου δεν θέλει επιστρέψει, εωσού εκτελέση και εωσού εκπληρώση τας βουλάς της καρδίας αυτού· εν ταις εσχάταις ημέραις θέλετε νοήσει τούτο.

\chapter{31}

\par Εν τω αυτώ καιρώ, λέγει Κύριος, θέλω είσθαι ο Θεός πασών των οικογενειών του Ισραήλ και αυτοί θέλουσιν είσθαι λαός μου.
\par 2 Ούτω λέγει Κύριος· Ο λαός ο εναπολειφθείς από της μαχαίρας εύρηκε χάριν εν τη ερήμω· ο Ισραήλ υπήγε να εύρη ανάπαυσιν.
\par 3 Ο Κύριος εφάνη παλαιόθεν εις εμέ, λέγων, Ναι, σε ηγάπησα αγάπησιν η αιώνιον· διά τούτο σε είλκυσα με έλεος.
\par 4 Πάλιν θέλω σε οικοδομήσει και θέλεις οικοδομηθή, παρθένε του Ισραήλ· θέλεις ευπρεπισθή πάλιν με τα τύμπανά σου και θέλεις εξέρχεσθαι εις τους χορούς των αγαλλομένων.
\par 5 Θέλεις φυτεύσει πάλιν αμπελώνας επί των ορέων της Σαμαρείας· οι φυτευταί θέλουσι φυτεύσει και θέλουσι τρώγει τον καρπόν.
\par 6 Διότι θέλει είσθαι ημέρα, καθ' ην οι φύλακες επί του όρους Εφραΐμ θέλουσι φωνάζει, Σηκώθητε και ας αναβώμεν εις την Σιών προς Κύριον τον Θεόν ημών.
\par 7 Διότι ούτω λέγει Κύριος· Ψάλλετε εν αγαλλιάσει διά τον Ιακώβ και αλαλάξατε διά την κεφαλήν των εθνών· κηρύξατε, αινέσατε και είπατε, Σώσον, Κύριε, τον λαόν σου το υπόλοιπον του Ισραήλ.
\par 8 Ιδού εγώ θέλω φέρει αυτούς εκ της γης του βορρά, και θέλω συνάξει αυτούς από των εσχάτων της γης, και μετ' αυτών τον τυφλόν και τον χωλόν, την έγκυον και την γεννώσαν ομού· συνάθροισμα μέγα θέλει επιστρέψει ενταύθα.
\par 9 Μετά κλαυθμού θέλουσιν ελθεί και μετά δεήσεων θέλω επαναφέρει αυτούς· θέλω οδηγήσει αυτούς παρά ποταμούς υδάτων δι' ευθείας οδού, καθ' ην δεν θέλουσι προσκόψει· διότι είμαι πατήρ εις τον Ισραήλ και ο Εφραΐμ είναι ο πρωτότοκός μου.
\par 10 Ακούσατε, έθνη, τον λόγον του Κυρίου, και αναγγείλατε εις τας νήσους τας μακράν και είπατε, Ο διασκορπίσας τον Ισραήλ θέλει συνάξει αυτόν και θέλει φυλάξει αυτόν ως ο βοσκός το ποίμνιον αυτού.
\par 11 Διότι ο Κύριος εξηγόρασε τον Ιακώβ και ελύτρωσεν αυτόν εκ χειρός του δυνατωτέρου αυτού.
\par 12 Και θέλουσιν ελθεί και ψάλλει επί του ύψους της Σιών, και θέλουσι συρρεύσει εις τα αγαθά του Κυρίου, εις σίτον και εις οίνον και εις έλαιον και εις τα γεννήματα των προβάτων και των βοών, και η ψυχή αυτών θέλει είσθαι ως παράδεισος περιποτιζόμενος· και παντελώς δεν θέλουσι λυπηθή πλέον.
\par 13 Τότε θέλει χαρή η παρθένος εν τω χορώ, και οι νέοι και οι γέροντες ομού· και θέλω στρέψει το πένθος αυτών εις χαράν και θέλω παρηγορήσει αυτούς και ευφράνει αυτούς μετά την θλίψιν αυτών.
\par 14 Και θέλω χορτάσει την ψυχήν των ιερέων από παχύτητος, και ο λαός μου θέλει χορτασθή από των αγαθών μου, λέγει Κύριος.
\par 15 Ούτω λέγει Κύριος· Φωνή ηκούσθη εν Ραμά, θρήνος, κλαυθμός, οδυρμός· η Ραχήλ, κλαίουσα τα τέκνα αυτής, δεν ήθελε να παρηγορηθή διά τα τέκνα αυτής, διότι δεν υπάρχουσιν.
\par 16 Ούτω λέγει Κύριος· Παύσον την φωνήν σου από κλαυθμού και τους οφθαλμούς σου από δακρύων· διότι το έργον σου θέλει ανταμειφθή, λέγει Κύριος· και θέλουσιν επιστρέψει εκ της γης του εχθρού.
\par 17 Και είναι ελπίς εις τα έσχατά σου, λέγει Κύριος, και τα τέκνα σου θέλουσιν επιστρέψει εις τα όρια αυτών.
\par 18 Ήκουσα τωόντι τον Εφραΐμ λέγοντα εν οδυρμοίς, Με επαίδευσας, και επαιδεύθην ως μόσχος αδάμαστος· επίστρεψόν με και θέλω επιστρέψει· διότι συ είσαι Κύριος ο Θεός μου·
\par 19 βεβαίως αφού επέστρεψα, μετενόησα, και αφού εδιδάχθην, εκτύπησα επί τον μηρόν μου· ησχύνθην και μάλιστα ηρυθρίασα, διότι εβάστασα το όνειδος της νεότητός μου.
\par 20 Ο Εφραΐμ είναι υιός αγαπητός εις εμέ; παιδίον φίλτατον; διότι αφού ελάλησα εναντίον αυτού, πάντοτε ενθυμούμαι αυτόν, διά τούτο τα σπλάγχνα μου ηχούσι δι' αυτόν· θέλω βεβαίως σπλαγχνισθή αυτόν, λέγει Κύριος.
\par 21 Στήσον σημεία της οδού, κάμε εις σεαυτόν σωρούς υψηλούς· προσήλωσον την καρδίαν σου εις την λεωφόρον, εις την οδόν δι' ης υπήγες· επίστρεψον, παρθένε του Ισραήλ, επίστρεψον εις αυτάς τας πόλεις σου.
\par 22 Έως πότε θέλεις περιφέρεσθαι, θυγάτηρ αποστάτρια; διότι ο Κύριος εποίησε νέον πράγμα εν τη γή· Γυνή θέλει περικυκλώσει άνδρα.
\par 23 Ούτω λέγει ο Κύριος των δυνάμεων, ο Θεός του Ισραήλ· Έτι θέλουσι λέγει τον λόγον τούτον εν τη γη του Ιούδα και εν ταις πόλεσιν αυτού, όταν επίστρέψω την αιχμαλωσίαν αυτών, Ο Κύριος να σε ευλογήση, κατοικία δικαιοσύνης, όρος αγιότητος.
\par 24 Και θέλουσι κατοικήσει εν αυτή ο Ιούδας και πάσαι αι πόλεις αυτού ομού, οι γεωργοί και οι εξερχόμενοι μετά των ποιμνίων·
\par 25 διότι εχόρτασα την εκλελυμένην ψυχήν και ενέπλησα πάσαν τεθλιμμένην ψυχήν.
\par 26 Διά τούτο εξύπνησα και εθεώρησα, και ο ύπνος μου εστάθη γλυκύς εις εμέ.
\par 27 Ιδού, έρχονται ημέραι, λέγει Κύριος, και θέλω σπείρει τον οίκον Ισραήλ και τον οίκον Ιούδα με σπέρμα ανθρώπου και με σπέρμα κτήνους.
\par 28 Και καθώς εγρηγόρουν επ' αυτούς διά να εκριζόνω και να κατασκάπτω και να κατεδαφίζω και να καταστρέφω και να καταθλίβω, ούτω θέλω γρηγορήσει επ' αυτούς διά να ανοικοδομώ και να φυτεύω, λέγει Κύριος.
\par 29 Εν ταις ημέραις εκείναις δεν θέλουσι λέγει πλέον, Οι πατέρες έφαγον όμφακα και οι οδόντες των τέκνων ημωδίασαν·
\par 30 αλλ' έκαστος θέλει αποθνήσκει διά την ανομίαν αυτού· πας άνθρωπος, όστις φάγη τον όμφακα, τούτου οι οδόντες θέλουσιν αιμωδιάσει.
\par 31 Ιδού, έρχονται ημέραι, λέγει Κύριος, και θέλω κάμει προς τον οίκον Ισραήλ και προς τον οίκον Ιούδα διαθήκην νέαν·
\par 32 ουχί κατά την διαθήκην, την οποίαν έκαμον προς τους πατέρας αυτών, καθ' ην ημέραν επίασα αυτούς από της χειρός διά να εξαγάγω αυτούς εκ γης Αιγύπτου· διότι αυτοί παρέβησαν την διαθήκην μου και εγώ απεστράφην αυτούς, λέγει Κύριος·
\par 33 αλλ' αύτη θέλει είσθαι η διαθήκη, την οποίαν θέλω κάμει προς τον οίκον Ισραήλ· μετά τας ημέρας εκείνας, λέγει Κύριος, θέλω θέσει τον νόμον μου εις τα ενδόμυχα αυτών και θέλω γράψει αυτόν εν ταις καρδίαις αυτών· και θέλω είσθαι Θεός αυτών και αυτοί θέλουσιν είσθαι λαός μου.
\par 34 Και δεν θέλουσι διδάσκει πλέον έκαστος τον πλησίον αυτού και έκαστος τον αδελφόν αυτού, λέγων, Γνωρίσατε τον Κύριον· διότι πάντες ούτοι θέλουσι με γνωρίζει από μικρού αυτών έως μεγάλου αυτών, λέγει Κύριος· διότι θέλω συγχωρήσει την ανομίαν αυτών και την αμαρτίαν αυτών δεν θέλω ενθυμείσθαι πλέον.
\par 35 Ούτω λέγει Κύριος, ο διδούς τον ήλιον εις φως της ημέρας, τας διατάξεις της σελήνης και των αστέρων εις φως της νυκτός, ο ταράττων την θάλασσαν, και τα κύματα αυτής βομβούσι· Κύριος των δυνάμεων το όνομα αυτού·
\par 36 Εάν αι διατάξεις αύται εκλείψωσιν απ' έμπροσθέν μου, λέγει Κύριος, τότε και το σπέρμα του Ισραήλ θέλει παύσει από του να ήναι έθνος ενώπιόν μου πάσας τας ημέρας.
\par 37 Ούτω λέγει Κύριος· Εάν ο ουρανός άνω δύναται να μετρηθή και τα θεμέλια της γης κάτω να εξιχνιασθώσι, τότε και εγώ θέλω απορρίψει παν το σπέρμα του Ισραήλ διά πάντα όσα έπραξαν, λέγει Κύριος.
\par 38 Ιδού, έρχονται ημέραι, λέγει Κύριος, και η πόλις θέλει οικοδομηθή εις τον Κύριον από του πύργου Ανανεήλ έως της πύλης της γωνίας.
\par 39 Και σχοινίον διαμετρήσεως θέλει εξέλθει έτι απέναντι αυτής επί τον λόφον Γαρήβ και θέλει περιέλθει έως Γοάθ.
\par 40 Και πάσα η κοιλάς των πτωμάτων και της στάκτης και πάντες οι αγροί έως του χειμάρρου Κέδρων, έως της γωνίας της πύλης των ίππων προς ανατολάς, θέλουσιν είσθαι άγιοι εις τον Κύριον· δεν θέλει πλέον εκριζωθή ουδέ καταστραφή εις τον αιώνα.

\chapter{32}

\par Ο λόγος ο γενόμενος προς τον Ιερεμίαν παρά Κυρίου εν τω δεκάτω έτει του Σεδεκίου βασιλέως του Ιούδα, το οποίον ήτο το δέκατον όγδοον έτος του Ναβουχοδονόσορ.
\par 2 Και τότε το στράτευμα του βασιλέως της Βαβυλώνος επολιόρκει την Ιερουσαλήμ· και ο Ιερεμίας ο προφήτης ήτο κεκλεισμένος εν τη αυλή της φυλακής, της εν τω οίκω του βασιλέως του Ιούδα.
\par 3 Διότι Σεδεκίας ο βασιλεύς του Ιούδα είχε κλείσει αυτόν, λέγων, Διά τι συ προφητεύεις λέγων, Ούτω λέγει Κύριος, Ιδού, εγώ θέλω παραδώσει την πόλιν ταύτην εις την χείρα του βασιλέως της Βαβυλώνος και θέλει κυριεύσει αυτήν·
\par 4 και Σεδεκίας ο βασιλεύς του Ιούδα δεν θέλει εκφύγει εκ της χειρός των Χαλδαίων, αλλά θέλει βεβαίως παραδοθή εις την χείρα του βασιλέως της Βαβυλώνος και θέλει λαλήσει μετ' αυτού στόμα προς στόμα και οι οφθαλμοί αυτού θέλουσιν ιδεί τους οφθαλμούς αυτού·
\par 5 και θέλει φέρει τον Σεδεκίαν εις την Βαβυλώνα και εκεί θέλει είσθαι, εωσού επισκεφθώ αυτόν, λέγει Κύριος· και εάν πολεμήσητε τους Χαλδαίους, δεν θέλετε ευδοκιμήσει.
\par 6 Και είπεν ο Ιερεμίας, Έγεινε λόγος Κυρίου προς εμέ, λέγων,
\par 7 Ιδού, Αναμεήλ, ο υιός του Σαλλούμ του θείου σου, θέλει ελθεί προς σε, λέγων, Αγόρασον εις σεαυτόν τον αγρόν μου τον εν Αναθώθ· διότι εις σε ανήκει το δικαίωμα εξαγοράς διά να αγοράσης αυτόν.
\par 8 Και ήλθε προς εμέ Αναμεήλ, ο υιός του θείου μου, εις την αυλήν της φυλακής, κατά τον λόγον του Κυρίου, και είπε προς εμέ, Αγόρασον, παρακαλώ, τον αγρόν μου τον εν Αναθώθ, τον εν τη γη Βενιαμίν· διότι εις σε ανήκει το δικαίωμα της κληρονομίας και εις σε η εξαγορά· αγόρασον αυτόν εις σεαυτόν. Τότε εγνώρισα ότι λόγος Κυρίου ήτο ούτος.
\par 9 Και ηγόρασα παρά του Αναμεήλ, υιού του θείου μου, τον αγρόν τον εν Αναθώθ και εζύγισα προς αυτόν τα χρήματα, δεκαεπτά σίκλους αργυρίου.
\par 10 Και έγραψα το συμφωνητικόν και εσφράγισα και έβαλον μάρτυρας και εζύγισα τα χρήματα εν τη πλάστιγγι.
\par 11 Και έλαβον το συμφωνητικόν της αγοράς, το εσφραγισμένον κατά τον νόμον και την συνήθειαν και το ανοικτόν·
\par 12 και έδωκα το συμφωνητικόν της αγοράς εις τον Βαρούχ τον υιόν του Νηρίου υιού του Μαασίου, έμπροσθεν του Αναμεήλ υιού του θείου μου και έμπροσθεν των μαρτύρων των υπογραψάντων το συμφωνητικόν της αγοράς, έμπροσθεν πάντων των Ιουδαίων των καθημένων εν τη αυλή της φυλακής.
\par 13 Και προσέταξα τον Βαρούχ έμπροσθεν αυτών, λέγων,
\par 14 Ούτω λέγει ο Κύριος των δυνάμεων, ο Θεός του Ισραήλ· Λάβε τα συμφωνητικά ταύτα, το συμφωνητικόν τούτο της αγοράς και το εσφραγισμένον και το συμφωνητικόν τούτο το ανοικτόν· και θες αυτά εις αγγείον πήλινον, διά να διαμένωσιν ημέρας πολλάς.
\par 15 Διότι ούτω λέγει ο Κύριος των δυνάμεων, ο Θεός του Ισραήλ· Οικίαι και αγροί και άμπελοι θέλουσιν αποκτηθή πάλιν εν ταύτη τη γη.
\par 16 Αφού δε έδωκα το συμφωνητικόν, της αγοράς εις τον Βαρούχ τον υιόν του Νηρίου προσηυχήθην εις τον Κύριον, λέγων,
\par 17 Ω Κύριε Θεέ· ιδού, συ έκαμες τον ουρανόν και την γην εν τη δυνάμει σου τη μεγάλη και εν τω βραχίονί σου τω εξηπλωμένω· δεν είναι ουδέν πράγμα δύσκολον εις σε.
\par 18 Κάμνεις έλεος εις χιλιάδας και ανταποδίδεις την ανομίαν των πατέρων εις τον κόλπον των τέκνων αυτών μετ' αυτούς· ο Θεός ο μέγας, ο ισχυρός, Κύριος των δυνάμεων το όνομα αυτού,
\par 19 μέγας εν βουλή και δυνατός εν έργοις· διότι οι οφθαλμοί σου είναι ανεωγμένοι επί πάσας τας οδούς των υιών των ανθρώπων, διά να δώσης εις έκαστον κατά τας οδούς αυτού και κατά τον καρπόν των έργων αυτού·
\par 20 όστις έκαμες σημεία και τέρατα εν τη γη της Αιγύπτου, γνωστά έως της ημέρας ταύτης, και εν Ισραήλ και εν ανθρώποις· και έκαμες εις σεαυτόν όνομα, ως την ημέραν ταύτην.
\par 21 και εξήγαγες τον λαόν σου τον Ισραήλ εκ γης Αιγύπτου με σημεία και με τέρατα και με κραταιάν χείρα και με βραχίονα εξηπλωμένον και με τρόμον μέγαν·
\par 22 και έδωκας εις αυτούς την γην ταύτην, την οποίαν ώμοσας προς τους πατέρας αυτών να δώσης εις αυτούς, γην ρέουσαν γάλα και μέλι·
\par 23 και εισήλθον και εκληρονόμησαν αυτήν· αλλά δεν υπήκουσαν εις την φωνήν σου ουδέ περιεπάτησαν εν τω νόμω σου· δεν έκαμον ουδέν εκ πάντων όσα προσέταξας εις αυτούς να κάμωσι· διά τούτο επέφερες επ' αυτούς άπαν τούτο το κακόν.
\par 24 Ιδού, τα χαρακώματα έφθασαν εις την πόλιν, διά να κυριεύσωσιν αυτήν· και η πόλις εδόθη εις την χείρα των Χαλδαίων των πολεμούντων κατ' αυτής, εξ αιτίας της μαχαίρας και της πείνης και του λοιμού· και ό,τι ελάλησας, έγεινε· και ιδού, βλέπεις·
\par 25 και συ είπας προς εμέ, Κύριε Θεέ, Αγόρασον εις σεαυτόν τον αγρόν δι' αργυρίου και παράστησον μάρτυρας· και η πόλις εδόθη εις την χείρα των Χαλδαίων.
\par 26 Και έγεινε λόγος Κυρίου προς τον Ιερεμίαν, λέγων,
\par 27 Ιδού, εγώ είμαι Κύριος ο Θεός πάσης σαρκός· είναι τι πράγμα δύσκολον εις εμέ;
\par 28 διά τούτο ούτω λέγει Κύριος· Ιδού, θέλω παραδώσει την πόλιν ταύτην εις την χείρα των Χαλδαίων και εις την χείρα του Ναβουχοδονόσορ βασιλέως της Βαβυλώνος, και θέλει κυριεύσει αυτήν·
\par 29 και οι Χαλδαίοι οι πολεμούντες κατά της πόλεως ταύτης θέλουσιν ελθεί και βάλει πυρ εις την πόλιν ταύτην και κατακαύσει αυτήν και τας οικίας, επί τα δώματα των οποίων εθυμίαζον εις τον Βάαλ και έκαμνον σπονδάς εις άλλους θεούς, διά να με παροργίσωσι.
\par 30 Διότι οι υιοί Ισραήλ και οι υιοί Ιούδα κακόν μόνον έκαμνον ενώπιόν μου εκ νεότητος αυτών· διότι οι υιοί Ισραήλ άλλο δεν έκαμνον, παρά να με παροργίζωσι διά των έργων των χειρών αυτών, λέγει Κύριος.
\par 31 Διότι η πόλις αύτη εστάθη εις εμέ ερεθισμός της οργής μου και του θυμού μου, αφ' ης ημέρας ωκοδόμησαν αυτήν έως της ημέρας ταύτης, διά να απορρίψω αυτήν απ' έμπροσθέν μου,
\par 32 ένεκεν πάσης της κακίας των υιών Ισραήλ και των υιών Ιούδα, την οποίαν έκαμον διά να με παροργίσωσιν, αυτοί, οι βασιλείς αυτών, οι άρχοντες αυτών, οι ιερείς αυτών και οι προφήται αυτών και οι άνδρες Ιούδα και οι κάτοικοι της Ιερουσαλήμ.
\par 33 Και έστρεψαν νώτα προς εμέ και ουχί πρόσωπον· και εδίδασκον αυτούς εγειρόμενος πρωΐ και διδάσκων, πλην δεν ήκουσαν, ώστε να λάβωσι παιδείαν·
\par 34 αλλ' έθεσαν τα βδελύγματα αυτών εν τω οίκω, εφ' ον εκλήθη το όνομά μου, διά να μιάνωσιν αυτόν.
\par 35 Και ωκοδόμησαν τους υψηλούς τόπους του Βάαλ τους εν τη φάραγγι του υιού Εννόμ, διά να διαπεράσωσι τους υιούς αυτών και τας θυγατέρας αυτών διά του πυρός εις τον Μολόχ· το οποίον δεν προσέταξα εις αυτούς ουδέ ανέβη επί την καρδίαν μου, να πράξωσι το βδέλυγμα τούτο, ώστε να κάμωσι τον Ιούδαν να αμαρτάνη.
\par 36 Και τώρα διά ταύτα ούτω λέγει Κύριος, ο Θεός του Ισραήλ, περί της πόλεως ταύτης, περί ης υμείς λέγετε, Θέλει παραδοθή εις την χείρα του βασιλέως της Βαβυλώνος, διά μαχαίρας και διά πείνης και διά λοιμού·
\par 37 ιδού, θέλω συνάξει αυτούς εκ πάντων των τόπων, όπου εδίωξα αυτούς εν τη οργή μου και εν τω θυμώ μου και εν τη μεγάλη αγανακτήσει μου· και θέλω επιστρέψει αυτούς εις τον τόπον τούτον και θέλω κατοικίσει αυτούς εν ασφαλεία·
\par 38 και θέλουσιν είσθαι λαός μου και εγώ θέλω είσθαι Θεός αυτών·
\par 39 και θέλω δώσει εις αυτούς καρδίαν μίαν και οδόν μίαν, διά να με φοβώνται πάσας τας ημέρας, διά το καλόν αυτών και των τέκνων αυτών μετ' αυτούς·
\par 40 και θέλω κάμει διαθήκην αιώνιον προς αυτούς, ότι δεν θέλω αποστρέψει απ' οπίσω αυτών, διά να αγαθοποιώ αυτούς· και θέλω δώσει τον φόβον μου εις τας καρδίας αυτών, διά να μη αποστατήσωσιν απ' εμού·
\par 41 και θέλω ευφραίνεσθαι επ' αυτούς εις το να αγαθοποιώ αυτούς, και θέλω φυτεύσει αυτούς εν τη γη ταύτη κατά αλήθειαν, εξ όλης μου της καρδίας και εξ όλης μου της ψυχής.
\par 42 Διότι ούτω λέγει Κύριος· Καθώς επέφερα επί τούτον τον λαόν πάντα ταύτα τα μεγάλα κακά, ούτω θέλω επιφέρει επ' αυτούς πάντα τα αγαθά, τα οποία εγώ ελάλησα περί αυτών.
\par 43 Και θέλουσιν αποκτηθή αγροί εν τη γη ταύτη, περί της οποίας σεις λέγετε, Είναι έρημος χωρίς ανθρώπου ή κτήνους· παρεδόθη εις την χείρα των Χαλδαίων.
\par 44 Θέλουσιν αγοράζει αγρούς δι' αργυρίου και υπογράφει συμφωνητικά και σφραγίζει και θέλουσι παριστάνει μάρτυρας, εν τη γη Βενιαμίν και εν τοις πέριξ Ιερουσαλήμ και εν ταις πόλεσι του Ιούδα και εν ταις πόλεσι της ορεινής και εν ταις πόλεσι της πεδινής και εν ταις πόλεσι του νότου· διότι θέλω επιστρέψει την αιχμαλωσίαν αυτών, λέγει Κύριος.

\chapter{33}

\par Και έγεινε λόγος Κυρίου προς τον Ιερεμίαν εκ δευτέρου, ενώ αυτός ήτο έτι κεκλεισμένος εν τη αυλή της φυλακής, λέγων,
\par 2 Ούτω λέγει Κύριος ο κτίσας αυτήν, Κύριος ο πλάσας αυτήν διά να στερεώση αυτήν· Κύριος το όνομα αυτού·
\par 3 Κράξον προς εμέ και θέλω σοι αποκριθή και σοι δείξει μεγάλα και απόκρυφα, τα οποία δεν γνωρίζεις.
\par 4 Διότι ούτω λέγει Κύριος ο Θεός του Ισραήλ περί των οικιών της πόλεως ταύτης και περί των οικιών των βασιλέων του Ιούδα, αίτινες θέλουσι καταστραφή από χαρακωμάτων και από μαχαίρας,
\par 5 των ερχομένων διά να πολεμήσωσι προς τους Χαλδαίους και διά να εμπλήσωσιν αυτάς με τα πτώματα των ανθρώπων, τους οποίους εγώ θέλω πατάξει εν τη οργή μου και εν τω θυμώ μου και διά πάσας τας κακίας των οποίων έκρυψα το πρόσωπόν μου από της πόλεως ταύτης·
\par 6 ιδού, εγώ θέλω φέρει εις αυτήν υγιείαν και ίασιν και θέλω ιατρεύσει αυτούς, και θέλω κάμει αυτούς να ίδωσιν αφθονίαν ειρήνης και αληθείας.
\par 7 Και θέλω επιστρέψει την αιχμαλωσίαν του Ιούδα και την αιχμαλωσίαν του Ισραήλ, και θέλω οικοδομήσει αυτούς ως το πρότερον·
\par 8 και θέλω καθαρίσει αυτούς από πάσης της ανομίας αυτών, με την οποίαν ημάρτησαν εις εμέ· και θέλω συγχωρήσει πάσας τας ανομίας αυτών, με τας οποίας ημάρτησαν εις εμέ και με τας οποίας απεστάτησαν απ' εμού.
\par 9 Και η πόλις αύτη θέλει είσθαι εις εμέ όνομα ευφροσύνης, αίνεσις και δόξα έμπροσθεν πάντων των εθνών της γης, τα οποία θέλουσιν ακούσει πάντα τα αγαθά, τα οποία εγώ κάμνω εις αυτούς· και θέλουσιν εκπλαγή και τρομάξει διά πάντα τα αγαθά και διά πάσαν την ειρήνην, την οποίαν θέλω κάμει εις αυτήν.
\par 10 Ούτω λέγει Κύριος· Πάλιν θέλει ακουσθή εν τω τόπω τούτω, περί του οποίου σεις λέγετε, Είναι έρημος, χωρίς ανθρώπου και χωρίς κτήνους εν ταις πόλεσι του Ιούδα και εν ταις πλατείαις της Ιερουσαλήμ, αίτινες είναι έρημοι, χωρίς ανθρώπου και χωρίς κατοίκου και χωρίς κτήνους,
\par 11 η φωνή της χαράς και η φωνή της ευφροσύνης, η φωνή του νυμφίου και η φωνή της νύμφης, φωνή των λεγόντων, Αινείτε τον Κύριον των δυνάμεων, διότι αγαθός ο Κύριος, διότι το έλεος αυτού μένει εις τον αιώνα· και των προσφερόντων ευχαριστηρίους προσφοράς εις τον οίκον του Κυρίου· διότι θέλω επιστρέψει την αιχμαλωσίαν της γης, ως το πρότερον, λέγει Κύριος.
\par 12 Ούτω λέγει ο Κύριος των δυνάμεων· Πάλιν εν τω τόπω τούτω όστις είναι έρημος, χωρίς ανθρώπου και χωρίς κτήνους, και εν πάσαις ταις πόλεσιν αυτού, θέλουσιν είσθαι μάνδραι ποιμένων διά να αναπαύωσι τα ποίμνια.
\par 13 Εν ταις πόλεσι της ορεινής, εν ταις πόλεσι της πεδινής και εν ταις πόλεσι του νότου και εν τη γη Βενιαμίν και εν τοις πέριξ της Ιερουσαλήμ και εν ταις πόλεσι του Ιούδα θέλουσι περάσει πάλιν τα ποίμνια υπό την χείρα του αριθμούντος, λέγει Κύριος.
\par 14 Ιδού, έρχονται ημέραι, λέγει Κύριος, και θέλω εκτελέσει τον αγαθόν εκείνον λόγον, τον οποίον ελάλησα περί του οίκου Ισραήλ και περί του οίκου Ιούδα.
\par 15 Εν ταις ημέραις εκείναις και εν τω καιρώ εκείνω θέλω αναβλαστήσει εις τον Δαβίδ βλαστόν δικαιοσύνης, και θέλει εκτελέσει κρίσιν και δικαιοσύνην εν τη γη.
\par 16 Εν εκείναις ταις ημέραις ο Ιούδας θέλει σωθή και η Ιερουσαλήμ θέλει κατοικήσει εν ασφαλεία· και τούτο είναι το όνομα, με το οποίον θέλει ονομασθή, Ο Κύριος η δικαιοσύνη ημών.
\par 17 Διότι ούτω λέγει Κύριος· Δεν θέλει λείψει από του Δαβίδ άνθρωπος καθήμενος επί τον θρόνον του οίκου Ισραήλ·
\par 18 ούτε από των ιερέων των Λευϊτών θέλει λείψει άνθρωπος ενώπιόν μου, διά να προσφέρη ολοκαυτώματα και να καίη προσφοράς εξ αλφίτων και να κάμνη θυσίας πάσας τας ημέρας.
\par 19 Και έγεινε λόγος Κυρίου προς τον Ιερεμίαν λέγων,
\par 20 Ούτω λέγει Κύριος· Εάν ήναι δυνατόν να καταλύσητε την διαθήκην μου της ημέρας και την διαθήκην μου της νυκτός, ώστε να μη ήναι πλέον ημέρα και νυξ εν τω καιρώ αυτών,
\par 21 τότε θέλει δυνηθή να καταλυθή και η διαθήκη μου η προς τον Δαβίδ τον δούλον μου, ώστε να μη έχη υιόν διά να βασιλεύη επί του θρόνου αυτού, και η προς τους Λευΐτας τους ιερείς, τους λειτουργούς μου.
\par 22 Καθώς η στρατιά του ουρανού δεν δύναται να αριθμηθή ουδέ η άμμος της θαλάσσης να μετρηθή, ούτω θέλω πληθύνει το σπέρμα Δαβίδ του δούλου μου και τους Λευΐτας τους λειτουργούντας εις εμέ.
\par 23 Και έγεινε λόγος Κυρίου προς τον Ιερεμίαν, λέγων,
\par 24 Δεν είδες τι ελάλησεν ο λαός ούτος, λέγων, Τας δύο οικογενείας, τας οποίας ο Κύριος εξέλεξεν, απέρριψεν αυτάς; ούτως αυτοί κατεφρόνησαν τον λαόν μου, ώστε δεν λογίζεται πλέον έθνος εις αυτούς.
\par 25 Ούτω λέγει Κύριος· Εάν δεν έκαμον την διαθήκην μου της ημέρας και της νυκτός, και εάν δεν διέταξα τους νόμους του ουρανού και της γης,
\par 26 τότε θέλω απορρίψει το σπέρμα του Ιακώβ και του Δαβίδ του δούλου μου, ώστε να μη λάβω εκ του σπέρματος αυτού κυβερνήτας επί το σπέρμα του Αβραάμ, του Ισαάκ και του Ιακώβ· διότι θέλω επιστρέψει την αιχμαλωσίαν αυτών και θέλω οικτείρει αυτούς.

\chapter{34}

\par Ο λόγος ο γενόμενος προς Ιερεμίαν παρά Κυρίου, ότε Ναβουχοδονόσορ ο βασιλεύς της Βαβυλώνος και πάσα η δύναμις αυτού και πάντα τα βασίλεια της γης τα υποκείμενα υπό την χείρα αυτού και πάντες οι λαοί επολέμουν κατά της Ιερουσαλήμ και κατά πασών των πόλεων αυτής, λέγων,
\par 2 Ούτω λέγει Κύριος ο Θεός του Ισραήλ· Ύπαγε και λάλησον προς τον Σεδεκίαν τον βασιλέα του Ιούδα και ειπέ προς αυτόν, Ούτω λέγει Κύριος· Ιδού, θέλω παραδώσει την πόλιν ταύτην εις την χείρα του βασιλέως της Βαβυλώνος, και θέλει κατακαύσει αυτήν εν πυρί·
\par 3 και συ δεν θέλεις εκφύγει εκ της χειρός αυτού, αλλ' εξάπαντος θέλεις πιασθή και παραδοθή εις την χείρα αυτού· και οι οφθαλμοί σου θέλουσιν ιδεί τους οφθαλμούς του βασιλέως της Βαβυλώνος, και το στόμα αυτού θέλει λαλήσει εις το στόμα σου, και θέλεις υπάγει εις την Βαβυλώνα.
\par 4 Άκουσον όμως τον λόγον του Κυρίου, Σεδεκία βασιλεύ του Ιούδα· ούτω λέγει Κύριος περί σού· Δεν θέλεις αποθάνει διά μαχαίρας·
\par 5 εν ειρήνη θέλεις αποθάνει, και κατά τας καύσεις τας εις τους πατέρας σου, τους προγενεστέρους βασιλείς, τους υπάρξαντας προ σου, ούτω θέλουσι κάμει καύσεις εις σέ· και θέλουσι κλαύσει, λέγοντες, Ουαί, Κύριε· διότι εγώ ελάλησα τον λόγον, λέγει Κύριος.
\par 6 Και ελάλησεν Ιερεμίας ο προφήτης προς Σεδεκίαν τον βασιλέα του Ιούδα πάντας τους λόγους τούτους εν Ιερουσαλήμ·
\par 7 το δε στράτευμα του βασιλέως της Βαβυλώνος επολέμει κατά της Ιερουσαλήμ και κατά πασών των πόλεων του Ιούδα των εναπολειφθεισών, κατά της Λαχείς και κατά της Αζηκά· διότι αύται εναπελείφθησαν εν ταις πόλεσιν Ιούδα, πόλεις οχυραί.
\par 8 Ο λόγος ο γενόμενος προς τον Ιερεμίαν παρά Κυρίου, αφού ο βασιλεύς Σεδεκίας έκαμε συνθήκην μετά παντός του λαού του εν Ιερουσαλήμ, διά να κηρύξη εις αυτούς άφεσιν·
\par 9 ώστε να αποπέμψη έκαστος τον δούλον αυτού και έκαστος την δούλην αυτού, Εβραίον ή Εβραίαν, ελευθέρους, διά να μη έχη μηδείς δούλον Ιουδαίον αδελφόν αυτού·
\par 10 και ήκουσαν πάντες οι άρχοντες και πας ο λαός, οι εισελθόντες εις την συνθήκην, του να αποπέμψωσιν έκαστος τον δούλον αυτού και έκαστος την δούλην αυτού ελευθέρους, ώστε να μη έχωσι πλέον δούλους αυτούς· υπήκουσαν λοιπόν και απέπεμψαν·
\par 11 μετά ταύτα όμως τους δούλους και τας δούλας, τους οποίους απέπεμψαν ελευθέρους, έκαμον να επιστρέψωσι, και καθυπέβαλον αυτούς να ήναι δούλοι και δούλαι·
\par 12 και έγεινε λόγος Κυρίου προς τον Ιερεμίαν παρά Κυρίου, λέγων,
\par 13 Ούτω λέγει Κύριος ο Θεός του Ισραήλ· Εγώ έκαμον διαθήκην προς τους πατέρας σας, καθ' ην ημέραν εξήγαγον αυτούς εκ γης Αιγύπτου, εξ οίκου δουλείας, λέγων,
\par 14 Εν τω τέλει επτά ετών αποπέμψατε έκαστος τον αδελφόν αυτού τον Εβραίον, όστις επωλήθη εις σε και σε υπηρέτησεν εξ έτη· τότε θέλεις αποπέμψει αυτόν ελεύθερον από σού· αλλ' οι πατέρες σας δεν μου ήκουσαν ουδέ έκλιναν το ωτίον αυτών.
\par 15 Και σεις τώρα είχετε επιστρέψει και κάμει το ευθές ενώπιόν μου, κηρύττοντες έκαστος άφεσιν εις τον πλησίον αυτού· και είχετε κάμει συνθήκην ενώπιόν μου εν τω οίκω, εφ' ον εκλήθη το όνομά μου·
\par 16 αλλ' επεστρέψατε και εμιάνατε το όνομά μου, και εκάμετε έκαστος τον δούλον αυτού και έκαστος την δούλην αυτού να επιστρέψωσι, τους οποίους είχετε αποπέμψει ελευθέρους κατά την θέλησιν αυτών, και καθυπεβάλετε αυτούς διά να ήναι εις εσάς δούλοι και δούλαι.
\par 17 Διά τούτο ούτω λέγει Κύριος· Σεις δεν μου ηκούσατε, να κηρύξητε άφεσιν έκαστος εις τον αδελφόν αυτού και έκαστος εις τον πλησίον αυτού· ιδού λοιπόν, λέγει Κύριος, εγώ κηρύττω άφεσιν εναντίον σας εις την μάχαιραν, εις τον λοιμόν και εις την πείναν· και θέλω σας παραδώσει εις διασποράν εν πάσι τοις βασιλείοις της γης.
\par 18 Και θέλω παραδώσει τους ανθρώπους τους αθετήσαντας την διαθήκην μου, οίτινες δεν εξετέλεσαν τους λόγους της διαθήκης, την οποίαν έκαμον ενώπιόν μου, ότε έσχισαν τον μόσχον εις δύο και επέρασαν μεταξύ των τμημάτων αυτού,
\par 19 τους άρχοντας του Ιούδα και τους άρχοντας της Ιερουσαλήμ, τους ευνούχους και τους ιερείς και πάντα τον λαόν του τόπου, οίτινες επέρασαν μεταξύ των τμημάτων του μόσχου·
\par 20 και θέλω παραδώσει αυτούς εις την χείρα των εχθρών αυτών και εις την χείρα των ζητούντων την ψυχήν αυτών· τα δε πτώματα αυτών θέλουσιν είσθαι διά τροφήν εις τα πετεινά του ουρανού και εις τα θηρία της γης.
\par 21 Και Σεδεκίαν τον βασιλέα του Ιούδα και τους άρχοντας αυτού θέλω παραδώσει εις την χείρα των εχθρών αυτών και εις την χείρα των ζητούντων την ψυχήν αυτών και εις την χείρα του στρατεύματος του βασιλέως της Βαβυλώνος, οίτινες ανεχώρησαν από εσάς.
\par 22 Ιδού, θέλω προστάξει, λέγει Κύριος, και θέλω επιστρέψει αυτούς εις την πόλιν ταύτην· και θέλουσι πολεμήσει κατ' αυτής και κυριεύσει αυτήν και κατακαύσει αυτήν εν πυρί· και θέλω κάμει ερήμωσιν τας πόλεις του Ιούδα, ώστε να μη υπάρχη ο κατοικών.

\chapter{35}

\par Ο λόγος ο γενόμενος προς τον Ιερεμίαν παρά Κυρίου εν ταις ημέραις του Ιωακείμ υιού του Ιωσίου, βασιλέως του Ιούδα, λέγων,
\par 2 Ύπαγε προς τον οίκον των Ρηχαβιτών και λάλησον προς αυτούς και φέρε αυτούς εις τον οίκον του Κυρίου, εις εν των δωματίων, και πότισον αυτούς οίνον.
\par 3 Τότε έλαβον Ιααζανίαν, τον υιόν του Ιερεμίου, υιού του Χαβασινία, και τους αδελφούς αυτού και πάντας τους υιούς αυτού και πάντα τον οίκον των Ρηχαβιτών,
\par 4 και έφερα αυτούς προς τον οίκον του Κυρίου, εις το δωμάτιον των υιών του Ανάν, υιού του Ιγδαλίου, ανθρώπου του Θεού, το οποίον ήτο πλησίον του δωματίου των αρχόντων του επί του δωματίου του Μαασίου υιού του Σαλλούμ, του φύλακος της αυλής·
\par 5 και έθεσα έμπροσθεν των υιών του οίκου των Ρηχαβιτών αγγεία πλήρη οίνου και ποτήρια, και είπα προς αυτούς, Πίετε οίνον.
\par 6 Και είπον, Δεν θέλομεν πίει οίνον· διότι Ιωναδάβ, ο υιός του Ρηχάβ, ο πατήρ ημών, προσέταξεν εις ημάς λέγων, Δεν θέλετε πίει οίνον, σεις και οι υιοί σας, εις τον αιώνα·
\par 7 ουδέ οικίαν θέλετε οικοδομήσει ουδέ σπέρμα θέλετε σπείρει ουδέ αμπελώνα θέλετε φυτεύσει ουδέ θέλετε έχει· αλλ' εν σκηναίς θέλετε κατοικεί πάσας τας ημέρας σας, διά να ζήσητε πολλάς ημέρας επί της γης, εν ή παροικείτε.
\par 8 Και υπηκούσαμεν εις την φωνήν του Ιωναδάβ, υιού του Ρηχάβ, του πατρός ημών, κατά πάντα όσα προσέταξεν εις ημάς, να μη πίωμεν οίνον πάσας τας ημέρας ημών, ημείς, αι γυναίκες ημών, οι υιοί ημών και αι θυγατέρες ημών·
\par 9 μηδέ να οικοδομήσωμεν οικίας διά να κατοικώμεν· και δεν είχομεν αμπελώνα ή αγρόν ή σπέρμα·
\par 10 αλλά κατωκήσαμεν εν σκηναίς και υπηκούσαμεν και επράξαμεν κατά πάντα όσα προσέταξεν εις ημάς Ιωναδάβ ο πατήρ ημών·
\par 11 ότε όμως Ναβουχοδονόσορ ο βασιλεύς της Βαβυλώνος ανέβη εις τον τόπον, είπομεν, Έλθετε και ας υπάγωμεν εις Ιερουσαλήμ, εξ αιτίας του στρατεύματος των Χαλδαίων και εξ αιτίας του στρατεύματος των Συρίων· και κατοικούμεν εν Ιερουσαλήμ.
\par 12 Και έγεινε λόγος Κυρίου προς τον Ιερεμίαν, λέγων,
\par 13 Ούτω λέγει ο Κύριος των δυνάμεων, ο Θεός του Ισραήλ· Ύπαγε και ειπέ προς τους ανθρώπους του Ιούδα και προς τους κατοίκους της Ιερουσαλήμ, Δεν θέλετε λάβει παιδείαν διά να ακούητε τους λόγους μου; λέγει Κύριος.
\par 14 Οι μεν λόγοι του Ιωναδάβ, υιού του Ρηχάβ, όστις προσέταξεν εις τους υιούς αυτού να μη πίνωσιν οίνον, εξετελέσθησαν· και έως της ημέρας ταύτης δεν πίνουσι, διότι υπακούουσιν εις την προσταγήν του πατρός αυτών· εγώ δε ελάλησα προς εσάς, εγειρόμενος πρωΐ και λαλών, πλην δεν μου ηκούσατε.
\par 15 Και απέστειλα προς εσάς πάντας τους δούλους μου τους προφήτας, εγειρόμενος πρωΐ και αποστέλλων, λέγων, Επιστρέψατε ήδη έκαστος από της οδού αυτού της πονηράς και διορθώσατε τας πράξεις υμών και μη υπάγετε οπίσω άλλων θεών διά να λατρεύητε αυτούς, και θέλετε κατοικήσει εν τη γη, την οποίαν έδωκα εις εσάς και εις τους πατέρας σας· αλλά δεν εκλίνατε το ωτίον σας και δεν μου εισηκούσατε.
\par 16 Επειδή οι υιοί του Ιωναδάβ υιού του Ρηχάβ εξετέλεσαν την προσταγήν του πατρός αυτών, την οποίαν προσέταξεν εις αυτούς, ο δε λαός ούτος δεν μου εισήκουσε,
\par 17 διά τούτο ούτω λέγει Κύριος ο Θεός των δυνάμεων, ο Θεός του Ισραήλ· Ιδού, θέλω φέρει επί τον Ιούδαν και επί πάντας τους κατοίκους της Ιερουσαλήμ πάντα τα κακά, τα οποία ελάλησα κατ' αυτών, διότι ελάλησα προς αυτούς και δεν ήκουσαν, και έκραξα προς αυτούς και δεν απεκρίθησαν.
\par 18 Και είπεν ο Ιερεμίας προς τον οίκον των Ρηχαβιτών, Ούτω λέγει ο Κύριος των δυνάμεων, ο Θεός του Ισραήλ· Επειδή υπηκούσατε εις την προσταγήν Ιωναδάβ του πατρός σας και εφυλάξατε πάσας τας εντολάς αυτού και εκάμετε κατά πάντα όσα προσέταξεν εις εσάς,
\par 19 διά τούτο ούτω λέγει ο Κύριος των δυνάμεων, ο Θεός του Ισραήλ· δεν θέλει λείψει άνθρωπος από του Ιωναδάβ υιού του Ρηχάβ παριστάμενος ενώπιόν μου εις τον αιώνα.

\chapter{36}

\par Και εν τω τετάρτω έτει του Ιωακείμ, υιού του Ιωσίου βασιλέως του Ιούδα, έγεινεν ο λόγος ούτος προς τον Ιερεμίαν παρά Κυρίου, λέγων,
\par 2 Λάβε εις σεαυτόν τόμον βιβλίου και γράψον εν αυτώ πάντας τους λόγους, τους οποίους ελάλησα προς σε κατά του Ισραήλ και κατά του Ιούδα και κατά πάντων των εθνών αφ' ης ημέρας ελάλησα προς σε, από των ημερών του Ιωσίου έως της ημέρας ταύτης·
\par 3 ίσως ακούση ο οίκος Ιούδα πάντα τα κακά, τα οποία εγώ βουλεύομαι να κάμω εις αυτούς, ώστε να επιστρέψωσιν έκαστος από της οδού αυτού της πονηράς και να συγχωρήσω την ανομίαν αυτών και την αμαρτίαν αυτών.
\par 4 Και εκάλεσεν ο Ιερεμίας τον Βαρούχ τον υιόν του Νηρίου, και ο Βαρούχ έγραψεν εκ στόματος του Ιερεμίου πάντας τους λόγους του Κυρίου, τους οποίους ελάλησε προς αυτόν, επί τόμου βιβλίου.
\par 5 Και προσέταξεν ο Ιερεμίας τον Βαρούχ, λέγων, Εγώ είμαι υπό φύλαξιν· δεν δύναμαι να εισέλθω εις τον οίκον του Κυρίου·
\par 6 διά τούτο είσελθε συ και ανάγνωσον εν τω τόμω, τον οποίον έγραψας εκ στόματός μου τους λόγους του Κυρίου, εις τα ώτα του λαού εν τω οίκω του Κυρίου εν ημέρα νηστείας· και θέλεις προσέτι αναγνώσει αυτούς εις τα ώτα παντός του Ιούδα, όσοι έρχονται εκ των πόλεων αυτών·
\par 7 ίσως η δέησις αυτών φθάση ενώπιον του Κυρίου και επιστρέψωσιν έκαστος από της οδού αυτού της πονηράς· διότι μέγας είναι ο θυμός και η οργή, την οποίαν ο Κύριος ελάλησε κατά του λαού τούτου.
\par 8 Και έκαμεν ο Βαρούχ ο υιός του Νηρίου κατά πάντα όσα προσέταξεν εις αυτόν Ιερεμίας ο προφήτης, αναγνώσας εν τω βιβλίω τους λόγους του Κυρίου εν τω οίκω του Κυρίου.
\par 9 Και εν τω πέμπτω έτει του Ιωακείμ, υιού του Ιωσίου βασιλέως του Ιούδα, εν τω εννάτω μηνί, εκήρυξαν νηστείαν ενώπιον του Κυρίου πας ο λαός εν Ιερουσαλήμ και πας ο λαός ο ερχόμενος εκ των πόλεων Ιούδα εις Ιερουσαλήμ.
\par 10 Και ανέγνωσεν ο Βαρούχ εν τω βιβλίω τους λόγους του Ιερεμίου εν τω οίκω του Κυρίου, εν τω δωματίω του Γεμαρίου, υιού του Σαφάν, του γραμματέως, εν τη αυλή τη άνω, εν τη εισόδω της νέας πύλης του οίκου του Κυρίου, εις τα ώτα παντός του λαού.
\par 11 Και ήκουσε Μιχαΐας ο υιός του Γεμαρίου, υιού του Σαφάν, εκ του βιβλίου πάντας τους λόγους του Κυρίου,
\par 12 και κατέβη προς τον οίκον του βασιλέως, εις το δωμάτιον του γραμματέως· και ιδού, πάντες οι άρχοντες εκάθηντο εκεί, Ελισαμά ο γραμματεύς και Δελαΐας ο υιός του Σεμαΐου και Ελναθάν ο υιός του Αχβώρ και Γεμαρίας ο υιός του Σαφάν και Σεδεκίας ο υιός του Ανανίου και πάντες οι άρχοντες.
\par 13 Και ανήγγειλε προς αυτούς ο Μιχαΐας πάντας τους λόγους τους οποίους ήκουσεν, ότε ο Βαρούχ ανεγίνωσκε το βιβλίον εις τα ώτα του λαού.
\par 14 Και απέστειλαν πάντες οι άρχοντες προς τον Βαρούχ Ιουδεί τον υιόν του Νεθανίου, υιού του Σελεμίου, υιού του Χουσεί, λέγοντες, Τον τόμον, τον οποίον ανέγνωσας εις τα ώτα του λαού, λάβε αυτόν εις την χείρα σου και ελθέ. Και έλαβεν ο Βαρούχ ο υιός του Νηρίου τον τόμον εις την χείρα αυτού και ήλθε προς αυτούς.
\par 15 Και είπον προς αυτόν, Κάθησον τώρα και ανάγνωσον τώρα εις τα ώτα ημών· και ανέγνωσεν ο Βαρούχ εις τα ώτα αυτών.
\par 16 Και ως ήκουσαν πάντας τους λόγους, εξεπλάγησαν προς αλλήλους και είπον προς τον Βαρούχ, Θέλομεν βεβαίως αναγγείλει προς τον βασιλέα πάντας τους λόγους τούτους.
\par 17 Και ηρώτησαν τον Βαρούχ, λέγοντες, Ειπέ προς ημάς τώρα, πως έγραψας πάντας τους λόγους τούτους εκ του στόματος αυτού;
\par 18 Και είπε προς αυτούς ο Βαρούχ, Από του στόματος αυτού προέφερε προς εμέ πάντας τους λόγους τούτους, και εγώ έγραφον με μελάνην εν τω βιβλίω.
\par 19 Και είπον οι άρχοντες προς τον Βαρούχ, Ύπαγε, κρύφθητι, συ και ο Ιερεμίας· και άνθρωπος ας μη εξεύρη που είσθε.
\par 20 Και εισήλθον προς τον βασιλέα εις την αυλήν· αφήκαν όμως τον τόμον εν τω δωματίω Ελισαμά του γραμματέως και ανήγγειλαν εις τα ώτα του βασιλέως πάντας τους λόγους.
\par 21 Και απέστειλεν ο βασιλεύς τον Ιουδεί να λάβη τον τόμον· και έλαβεν εκ του δωματίου Ελισαμά του γραμματέως. Και ανέγνωσεν αυτόν ο Ιουδεί εις τα ώτα του βασιλέως και εις τα ώτα πάντων των αρχόντων των παρεστώτων περί τον βασιλέα.
\par 22 Ο δε βασιλεύς εκάθητο εν τω οίκω τω χειμερινώ, εν τω εννάτω μηνί, και ήτο έμπροσθεν αυτού εστία καίουσα.
\par 23 Και καθώς ο Ιουδεί ανεγίνωσκε τρεις και τέσσαρας σελίδας, εκείνος έκοπτεν αυτό διά του μαχαιριδίου του γραμματέως και έρριπτεν εις το πυρ το επί της εστίας, εωσού κατηναλώθη άπας ο τόμος εν τω πυρί τω επί της εστίας.
\par 24 Και δεν ετρόμαξαν ουδέ διέσχισαν τα ιμάτια αυτών ο βασιλεύς και πάντες οι δούλοι αυτού οι ακούσαντες πάντας τους λόγους τούτους.
\par 25 Και ενώ, μάλιστα ο Ελναθάν και ο Δελαΐας και ο Γεμαρίας εμεσίτευον προς τον βασιλέα, να μη καύση τον τόμον, δεν ήκουσεν αυτούς.
\par 26 Και προσέταξεν ο βασιλεύς τον Ιεραμεήλ τον υιόν του Αμμέλεχ και τον Σεραΐαν τον υιόν του Αζριήλ και τον Σελεμίαν τον υιόν του Αβδιήλ, να πιάσωσι τον Βαρούχ τον γραμματέα και τον Ιερεμίαν τον προφήτην· πλην ο Κύριος έκρυψεν αυτούς.
\par 27 Και έγεινε λόγος Κυρίου προς τον Ιερεμίαν, αφού ο βασιλεύς κατέκαυσε τον τόμον και τους λόγους, τους οποίους έγραψεν ο Βαρούχ εκ στόματος του Ιερεμίου, λέγων,
\par 28 Λάβε πάλιν εις σεαυτόν άλλον τόμον και γράψον επ' αυτώ πάντας τους προτέρους λόγους, οίτινες ήσαν εν τω πρώτω τόμω, τον οποίον κατέκαυσεν Ιωακείμ ο βασιλεύς του Ιούδα·
\par 29 και προς τον Ιωακείμ, τον βασιλέα του Ιούδα, θέλεις ειπεί, Ούτω λέγει Κύριος· Συ κατέκαυσας τον τόμον τούτον, λέγων, Διά τι έγραψας εν αυτώ, λέγων, Ο βασιλεύς της Βαβυλώνος θέλει ελθεί εξάπαντος και θέλει εξολοθρεύσει την γην ταύτην και κάμει να εκλείψη απ' αυτής άνθρωπος και κτήνος;
\par 30 Διά τούτο ούτω λέγει Κύριος περί του Ιωακείμ του βασιλέως του Ιούδα· δεν θέλει έχει καθήμενον επί του θρόνου του Δαβίδ· και το πτώμα αυτού θέλει εκριφθή την ημέραν εις το καύμα και την νύκτα εις τον παγετόν·
\par 31 και θέλω παιδεύσει αυτόν και το σπέρμα αυτού και τους δούλους αυτού διά την ανομίαν αυτών· και θέλω φέρει επ' αυτούς και επί τους κατοίκους της Ιερουσαλήμ και επί τους ανθρώπους του Ιούδα πάντα τα κακά, τα οποία ελάλησα προς αυτούς και δεν ήκουσαν.
\par 32 Και έλαβεν ο Ιερεμίας άλλον τόμον και έδωκεν αυτόν εις τον Βαρούχ, τον υιόν του Νηρίου, τον γραμματέα, και έγραψεν εν αυτώ εκ στόματος του Ιερεμίου πάντας τους λόγους του βιβλίου, το οποίον κατέκαυσεν εν πυρί Ιωακείμ ο βασιλεύς του Ιούδα· και έτι προσετέθησαν εις αυτούς πολλοί λόγοι παρόμοιοι.

\chapter{37}

\par Και εβασίλευσε Σεδεκίας ο βασιλεύς, ο υιός του Ιωσίου, αντί Χονίου υιού του Ιωακείμ, τον οποίον Ναβουχοδονόσορ ο βασιλεύς της Βαβυλώνος κατέστησε βασιλέα εν τη γη Ιούδα.
\par 2 Και δεν ήκουσεν αυτός και οι δούλοι αυτού και ο λαός του τόπου τους λόγους του Κυρίου, τους οποίους ελάλησε διά Ιερεμίου του προφήτου.
\par 3 Και απέστειλεν ο βασιλεύς Σεδεκίας τον Ιεουχάλ υιόν του Σελεμίου και τον Σοφονίαν υιόν του Μαασίου, τον ιερέα, προς Ιερεμίαν τον προφήτην, λέγων, Δεήθητι, παρακαλώ, υπέρ ημών προς Κύριον τον Θεόν ημών.
\par 4 Ο δε Ιερεμίας εισήρχετο και εξήρχετο μεταξύ του λαού, και δεν είχον βάλει αυτόν εις φυλακήν.
\par 5 Και εξήλθε το στράτευμα του Φαραώ εκ της Αιγύπτου· και ότε οι Χαλδαίοι οι πολιορκούντες την Ιερουσαλήμ ήκουσαν την φήμην αυτών, ανεχώρησαν από Ιερουσαλήμ.
\par 6 Και έγεινε λόγος Κυρίου προς Ιερεμίαν τον προφήτην, λέγων,
\par 7 Ούτω λέγει Κύριος ο Θεός του Ισραήλ· Ούτω θέλετε ειπεί προς τον βασιλέα του Ιούδα, όστις απέστειλεν υμάς προς εμέ διά να με ερωτήσητε· Ιδού, το στράτευμα του Φαραώ το εξελθόν εις βοήθειαν υμών θέλει επιστρέψει εις την γην αυτού, την Αίγυπτον·
\par 8 και οι Χαλδαίοι θέλουσιν επαναστρέψει και πολεμήσει κατά της πόλεως ταύτης και θέλουσι κυριεύσει αυτήν και κατακαύσει αυτήν εν πυρί.
\par 9 Ούτω λέγει Κύριος· Μη πλανάσθε, λέγοντες, οι Χαλδαίοι εξάπαντος θέλουσιν απέλθει αφ' ημών· επειδή δεν θέλουσιν απέλθει.
\par 10 Διότι και αν πατάξητε άπαν το στράτευμα των Χαλδαίων, το οποίον σας πολεμεί, και εναπολειφθώσι πεπληγωμένοι τινές μεταξύ αυτών, ούτοι θέλουσι σηκωθή έκαστος εκ της σκηνής αυτού και κατακαύσει την πόλιν ταύτην εν πυρί.
\par 11 Και ότε το στράτευμα των Χαλδαίων απήλθεν από Ιερουσαλήμ διά τον φόβον του στρατεύματος του Φαραώ,
\par 12 τότε εξήλθεν ο Ιερεμίας εξ Ιερουσαλήμ, διά να υπάγη εις την γην Βενιαμίν, ώστε να υπεκφύγη εκείθεν μεταξύ του λαού.
\par 13 Και ότε αυτός ήλθεν εις την πύλην Βενιαμίν, ο αρχηγός της φρουράς ευρίσκετο εκεί, του οποίου το όνομα ήτο Ιρεΐας υιός του Σελεμίου, υιού του Ανανίου· και επίασε τον Ιερεμίαν τον προφήτην, λέγων, Συ προσφεύγεις προς τους Χαλδαίους.
\par 14 Και είπεν ο Ιερεμίας, Ψεύδος είναι· εγώ δεν προσφεύγω προς τους Χαλδαίους. Πλην δεν ήκουσεν αυτόν· και επίασεν ο Ιρεΐας τον Ιερεμίαν και έφερεν αυτόν προς τους άρχοντας.
\par 15 Και ωργίσθησαν οι άρχοντες κατά του Ιερεμίου και επάταξαν αυτόν και εφυλάκισαν αυτόν εν τη οικία Ιωνάθαν τον γραμματέως, διότι ταύτην είχον κάμει δεσμωτήριον.
\par 16 Ότε δε ο Ιερεμίας εισήλθεν εις τον λάκκον και εις τας κρύπτας και εκάθησεν ο Ιερεμίας εκεί πολλάς ημέρας,
\par 17 τότε απέστειλε Σεδεκίας ο βασιλεύς και έλαβεν αυτόν, και ηρώτησεν αυτόν ο βασιλεύς κρυφίως εν τη οικία αυτού και είπεν, Είναι λόγος παρά Κυρίου; Και ο Ιερεμίας είπεν, είναι· και είπεν, εις την χείρα του βασιλέως της Βαβυλώνος θέλεις παραδοθή.
\par 18 Και είπεν ο Ιερεμίας προς τον βασιλέα Σεδεκίαν, Τι ημάρτησα εις σε ή εις τους δούλους σου ή εις τον λαόν τούτον, και με εβάλετε εις το δεσμωτήριον;
\par 19 και που είναι οι προφήταί σας οι προφητεύσαντες εις εσάς, λέγοντες, Ο βασιλεύς της Βαβυλώνος δεν θέλει ελθεί εφ' υμάς και επί την γην ταύτην;
\par 20 διά τούτο άκουσον τώρα, παρακαλώ, κύριέ μου βασιλεύ· ας γείνη δεκτή, παρακαλώ, η δέησίς μου ενώπιόν σου· και μη με επαναστρέψης εις την οικίαν Ιωνάθαν του γραμματέως, διά να μη αποθάνω εκεί.
\par 21 Τότε προσέταξεν ο βασιλεύς Σεδεκίας και εφύλαττον τον Ιερεμίαν εν τη αυλή της φυλακής, και έδιδον εις αυτόν καθ' ημέραν ολίγον άρτον εκ των αρτοπωλείων, εωσού εξέλιπεν όλος ο άρτος της πόλεως. Και έμεινεν ο Ιερεμίας εν τη αυλή της φυλακής.

\chapter{38}

\par Και ήκουσαν Σεφατίας ο υιός του Ματθάν και Γεδαλίας ο υιός του Πασχώρ και Ιουχάλ ο υιός του Σελεμίου και Πασχώρ ο υιός του Μαλχίου τους λόγους, τους οποίους ο Ιερεμίας ελάλησε προς πάντα τον λαόν, λέγων,
\par 2 Ούτω λέγει Κύριος· Όστις κάθηται εν τη πόλει ταύτη, θέλει αποθάνει υπό μαχαίρας, υπό πείνης και υπό λοιμού· αλλ' όστις εξέλθη προς τους Χαλδαίους, θέλει ζήσει· και η ζωή αυτού θέλει είσθαι ως λάφυρον εις αυτόν, και θέλει ζήσει·
\par 3 ούτω λέγει Κύριος· Η πόλις αύτη θέλει εξάπαντος παραδοθή εις την χείρα του στρατεύματος του βασιλέως της Βαβυλώνος και θέλει κυριεύσει αυτήν.
\par 4 Και είπον οι άρχοντες προς τον βασιλέα, Ας θανατωθή, παρακαλούμεν, ο άνθρωπος ούτος· διότι εκλύει ούτω τας χείρας των ανδρών των πολεμιστών των εναπολειφθέντων εν τη πόλει ταύτη και τας χείρας παντός του λαού, λαλών προς αυτούς τοιούτους λόγους· διότι ο άνθρωπος ούτος δεν ζητεί το καλόν του λαού τούτου αλλά το κακόν.
\par 5 Και είπε Σεδεκίας ο βασιλεύς, Ιδού, εις την χείρα σας είναι· διότι ο βασιλεύς δεν δύναται ουδέν εναντίον σας.
\par 6 Τότε έλαβον τον Ιερεμίαν, και έρριψαν αυτόν εις τον λάκκον του Μαλχίου υιού του Αμμέλεχ τον εν τη αυλή της φυλακής, και κατεβίβασαν τον Ιερεμίαν διά σχοινίων· και εν τω λάκκω δεν ήτο ύδωρ αλλά βόρβορος, και εχώθη ο Ιερεμίας εις τον βόρβορον.
\par 7 Και ότε ήκουσεν Αβδέ-μέλεχ ο Αιθίοψ, εις των ευνούχων, ο εν τη οικία του βασιλέως, ότι έβαλον τον Ιερεμίαν εις τον λάκκον, ενώ ο βασιλεύς εκάθητο εν τη πύλη Βενιαμίν,
\par 8 εξήλθεν ο Αβδέ-μέλεχ εκ της οικίας του βασιλέως και ελάλησε προς τον βασιλέα, λέγων,
\par 9 Κύριέ μου βασιλεύ, οι άνθρωποι ούτοι έπραξαν κακά εις όσα έκαμον εις τον Ιερεμίαν τον προφήτην, τον οποίον έρριψαν εις τον λάκκον· και αυτός ήθελεν αποθάνει υπό πείνης εν τω τόπω όπου είναι, διότι δεν είναι πλέον άρτος εν τη πόλει.
\par 10 Και προσέταξεν ο βασιλεύς τον Αβδέ-μέλεχ τον Αιθίοπα, λέγων, Λάβε εντεύθεν τριάκοντα ανθρώπους μετά σου και αναβίβασον τον Ιερεμίαν τον προφήτην εκ του λάκκου, πριν αποθάνη.
\par 11 Και έλαβεν ο Αβδέ-μέλεχ τους ανθρώπους μεθ' εαυτού, και εισήλθεν εις την οικίαν του βασιλέως υπό το θησαυροφυλάκιον, και εκείθεν έλαβε παλαιά ράκη και παλαιά σεσηπότα αποφόρια και κατεβίβασεν αυτά διά σχοινίων εις τον λάκκον προς τον Ιερεμίαν.
\par 12 Και είπε προς τον Ιερεμίαν Αβδέ-μέλεχ ο Αιθίοψ, Βάλε τώρα τα παλαιά ράκη και τα σεσηπότα αποφόρια υπό τας μασχάλας σου, υποκάτω των σχοινίων. Και έκαμεν ο Ιερεμίας ούτω.
\par 13 Και έσυραν τον Ιερεμίαν διά των σχοινίων και ανεβίβασαν αυτόν εκ του λάκκου· και έμεινεν ο Ιερεμίας εν τη αυλή της φυλακής.
\par 14 Και απέστειλε Σεδεκίας ο βασιλεύς και έφερε τον Ιερεμίαν τον προφήτην προς εαυτόν, εις την τρίτην είσοδον την εν τω οίκω του Κυρίου· και είπεν ο βασιλεύς προς τον Ιερεμίαν, Θέλω να σε ερωτήσω εν πράγμα· μη κρύψης απ' εμού μηδέν.
\par 15 Και είπεν ο Ιερεμίας προς τον Σεδεκίαν, Εάν φανερώσω τούτο προς σε, δεν θέλεις τωόντι με θανατώσει; και εάν σε συμβουλεύσω, δεν θέλεις με ακούσει;
\par 16 Και ώμοσε κρυφίως Σεδεκίας ο βασιλεύς προς τον Ιερεμίαν, λέγων, Ζη Κύριος, όστις έκαμεν εις ημάς την ψυχήν ταύτην, δεν θέλω σε θανατώσει ουδέ θέλω σε δώσει εις την χείρα των ανθρώπων τούτων, οίτινες ζητούσι την ψυχήν σου.
\par 17 Και είπεν ο Ιερεμίας προς τον Σεδεκίαν, Ούτω λέγει Κύριος ο Θεός των δυνάμεων, ο Θεός του Ισραήλ· Εάν τωόντι εξέλθης προς τους άρχοντας του βασιλέως της Βαβυλώνος, τότε η ψυχή σου θέλει ζήσει και η πόλις αύτη δεν θέλει κατακαυθή εν πυρί, και θέλεις ζήσει συ και ο οίκός σου.
\par 18 αλλ' εάν δεν εξέλθης προς τους άρχοντας του βασιλέως της Βαβυλώνος, τότε η πόλις αύτη θέλει παραδοθή εις την χείρα των Χαλδαίων και θέλουσι κατακαύσει αυτήν εν πυρί και συ δεν θέλεις εκφύγει εκ της χειρός αυτών.
\par 19 Και είπε Σεδεκίας ο βασιλεύς προς τον Ιερεμίαν, Εγώ φοβούμαι τους Ιουδαίους, οίτινες κατέφυγον προς τους Χαλδαίους, μήποτε με παραδώσωσιν εις την χείρα αυτών και με εμπαίξωσι.
\par 20 Και είπεν ο Ιερεμίας, δεν θέλουσι σε παραδώσει. Υπάκουσον, παρακαλώ, εις την φωνήν του Κυρίου, την οποίαν εγώ λαλώ προς σέ· και θέλει είσθαι καλόν εις σε και η ψυχή σου θέλει ζήσει.
\par 21 Εάν όμως συ δεν εξέλθης, ούτος είναι ο λόγος, τον οποίον ο Κύριος έδειξεν εις εμέ.
\par 22 Και ιδού, πάσαι αι γυναίκες αι εναπολειφθείσαι εν τη οικία του βασιλέως του Ιούδα θέλουσιν αχθή προς τους άρχοντας του βασιλέως της Βαβυλώνος, και αύται θέλουσι λέγει, Οι άνδρες οι ειρηνικοί σου σε εδελέασαν και υπερίσχυσαν εναντίον σου· εβυθίσθησαν οι πόδες σου εις τον βόρβορον και αυτοί εσύρθησαν οπίσω·
\par 23 και πάσαι αι γυναίκές σου και τα τέκνα σου θέλουσιν αχθή προς τους Χαλδαίους· και συ δεν θέλεις εκφύγει εκ της χειρός αυτών, αλλά θέλεις πιασθή υπό της χειρός του βασιλέως της Βαβυλώνος· και θέλεις κάμει την πόλιν ταύτην να κατακαυθή εν πυρί.
\par 24 Και είπεν ο Σεδεκίας προς τον Ιερεμίαν, Ας μη μάθη μηδείς περί των λόγων τούτων και δεν θέλεις αποθάνει.
\par 25 Και εάν οι άρχοντες ακούσωσιν ότι ώμίλησα μετά σου και έλθωσι προς σε και σοι είπωσιν, Ανάγγειλον προς ημάς τώρα τι ελάλησας προς τον βασιλέα, μη κρύψης αυτό αφ' ημών και δεν θέλομεν σε θανατώσει· και τι ο βασιλεύς ελάλησε προς σέ·
\par 26 τότε θέλεις ειπεί προς αυτούς, Εγώ υπέβαλον την δέησίν μου ενώπιον του βασιλέως, διά να μη με επαναστρέψη εις την οικίαν του Ιωνάθαν, ώστε να αποθάνω εκεί.
\par 27 Ήλθον δε πάντες οι άρχοντες προς τον Ιερεμίαν και ηρώτησαν αυτόν· και ανήγγειλε προς αυτούς κατά πάντας τους λόγους εκείνους, τους οποίους προσέταξεν ο βασιλεύς. Και αυτοί έπαυσαν να ομιλώσι μετ' αυτού, διότι δεν ηκούσθη το πράγμα.
\par 28 Και έμεινεν ο Ιερεμίας εν τη αυλή της φυλακής, έως της ημέρας καθ' ην εκυριεύθη η Ιερουσαλήμ· και ήτο εκεί, ότε η Ιερουσαλήμ εκυριεύθη.

\chapter{39}

\par Εν τω εννάτω έτει του Σεδεκίου βασιλέως του Ιούδα, τον δέκατον μήνα, ήλθε Ναβουχοδονόσορ ο βασιλεύς της Βαβυλώνος και άπαν το στράτευμα αυτού κατά της Ιερουσαλήμ και επολιόρκουν αυτήν.
\par 2 Εν δε τω ενδεκάτω έτει του Σεδεκίου, τον τέταρτον μήνα, την εννάτην του μηνός, επορθήθη η πόλις.
\par 3 Και πάντες οι άρχοντες του βασιλέως της Βαβυλώνος εισήλθον και εκάθησαν εν τη μεσαία πύλη, Νεργάλ-σαρεσέρ, Σαμγάρ-νεβώ, Σαρσεχείμ, Ραβ-σαρείς, Νεργάλ-σαρεσέρ, Ραβ-μαγ και πάντες οι λοιποί άρχοντες του βασιλέως της Βαβυλώνος.
\par 4 Και ως είδεν αυτούς Σεδεκίας ο βασιλεύς του Ιούδα και πάντες οι άνδρες του πολέμου, έφυγον και εξήλθον την νύκτα εκ της πόλεως διά της οδού του κήπου του βασιλέως, διά της πύλης της μεταξύ των δύο τειχών· και εξήλθε διά της οδού της πεδιάδος.
\par 5 Το δε στράτευμα των Χαλδαίων κατεδίωξεν οπίσω αυτών, και έφθασαν τον Σεδεκίαν εις τας πεδιάδας της Ιεριχώ· και συνέλαβον αυτόν και ανήγαγον αυτόν προς τον Ναβουχοδονόσορ βασιλέα της Βαβυλώνος εις Ριβλά, εν γη Αιμάθ, και επρόφερε καταδίκην επ' αυτόν.
\par 6 Και έσφαξεν ο βασιλεύς της Βαβυλώνος τους υιούς του Σεδεκίου εν Ριβλά ενώπιον αυτού, και πάντας τους άρχοντας του Ιούδα έσφαξεν ο βασιλεύς της Βαβυλώνος.
\par 7 Και τους οφθαλμούς του Σεδεκίου εξετύφλωσε και έδεσεν αυτόν με δύο χαλκίνας αλύσεις, διά να φέρη αυτόν εις την Βαβυλώνα.
\par 8 Και την οικίαν του βασιλέως και τας οικίας του λαού κατέκαυσαν οι Χαλδαίοι εν πυρί, και τα τείχη της Ιερουσαλήμ κατηδάφισαν.
\par 9 Το δε υπόλοιπον του λαού το εναπολειφθέν εν τη πόλει και τους προσφυγόντας, οίτινες προσέφυγον εις αυτόν, και το υπόλοιπον του λαού το εναπολειφθέν έφερεν αιχμάλωτον εις Βαβυλώνα Νεβουζαραδάν ο αρχισωματοφύλαξ.
\par 10 Εκ δε του λαού τους πτωχούς τους μη έχοντας μηδέν αφήκεν ο Νεβουζαραδάν ο αρχισωματοφύλαξ εν τη γη του Ιούδα και έδωκεν εις αυτούς αμπελώνας και αγρούς εν τω καιρώ εκείνω.
\par 11 Και έδωκε διαταγήν Ναβουχοδονόσορ ο βασιλεύς της Βαβυλώνος περί του Ιερεμίου εις τον Νεβουζαραδάν τον αρχισωματοφύλακα, λέγων,
\par 12 Λάβε αυτόν και επιμελήθητι αυτού και μη κάμης εις αυτόν κακόν· αλλ' όπως λαλήση προς σε, ούτω κάμε εις αυτόν.
\par 13 Και απέστειλεν ο Νεβουζαραδάν ο αρχισωματοφύλαξ και ο Νεβουσαζβάν, ο Ραβ-σαρείς και ο Νεργάλ-σαρεσέρ, ο Ραβ-μαγ και πάντες οι άρχοντες του βασιλέως της Βαβυλώνος,
\par 14 απέστειλαν και έλαβον τον Ιερεμίαν εκ της αυλής της φυλακής και παρέδωκαν αυτόν εις τον Γεδαλίαν, υιόν του Αχικάμ υιού του Σαφάν, διά να φέρη αυτόν εις τον οίκον αυτού· και κατώκησε μεταξύ του λαού.
\par 15 Και έγεινε λόγος Κυρίου προς τον Ιερεμίαν, ενώ ήτο κεκλεισμένος εν τη αυλή της φυλακής, λέγων,
\par 16 Ύπαγε και λάλησον προς Αβδέ-μέλεχ τον Αιθίοπα, λέγων, Ούτω λέγει ο Κύριος των δυνάμεων, ο Θεός του Ισραήλ· Ιδού, εγώ θέλω φέρει τους λόγους μου επί την πόλιν ταύτην διά κακόν και ουχί διά καλόν· και θέλουσιν εκτελεσθή ενώπιόν σου την ημέραν εκείνην.
\par 17 Θέλω όμως σε σώσει εν τη ημέρα εκείνη, λέγει Κύριος, και δεν θέλεις παραδοθή εις την χείρα των ανθρώπων, των οποίων συ φοβείσαι το πρόσωπον,
\par 18 διότι εξάπαντος θέλω σε σώσει και δεν θέλεις πέσει διά μαχαίρας, αλλ' η ζωή σου θέλει είσθαι ως λάφυρον εις σε, επειδή πέποιθας επ' εμέ, λέγει Κύριος.

\chapter{40}

\par Ο λόγος ο γενόμενος προς Ιερεμίαν παρά Κυρίου, αφού Νεβουζαραδάν ο αρχισωματοφύλαξ εξαπέστειλεν αυτόν από Ραμά, ότε είχε λάβει αυτόν δεδεμένον με χειρόδεσμα μεταξύ πάντων των μετοικισθέντων από Ιερουσαλήμ και Ιούδα, οίτινες εφέροντο αιχμάλωτοι εις την Βαβυλώνα.
\par 2 Και επίασεν ο αρχισωματοφύλαξ τον Ιερεμίαν και είπε προς αυτόν, Κύριος ο Θεός σου ελάλησε τα κακά ταύτα επί τον τόπον τούτον.
\par 3 Και επέφερεν αυτά ο Κύριος και έκαμε καθώς είπεν· επειδή ημαρτήσατε εις τον Κύριον και δεν υπηκούσατε εις την φωνήν αυτού, διά τούτο έγεινεν εις εσάς το πράγμα τούτο.
\par 4 Και τώρα ιδού, σε έλυσα σήμερον εκ των χειροδέσμων των επί των χειρών σου· εάν φαίνηται εις σε καλόν να έλθης μετ' εμού εις την Βαβυλώνα, ελθέ, και εγώ θέλω σε περιποιηθή· αλλ' εάν φαίνηται εις σε κακόν να έλθης μετ' εμού εις την Βαβυλώνα, μείνον· ιδού, πας ο τόπος είναι έμπροσθέν σου· όπου σοι φαίνεται καλόν και αρεστόν να υπάγης, εκεί ύπαγε.
\par 5 Και επειδή δεν εστρέφετο, Επίστρεψον λοιπόν, είπε, προς τον Γεδαλίαν, υιόν του Αχικάμ υιού του Σαφάν, τον οποίον ο βασιλεύς της Βαβυλώνος κατέστησεν επί τας πόλεις του Ιούδα, και κατοίκησον μετ' αυτού μεταξύ του λαού· ή ύπαγε όπου σοι φαίνεται αρεστόν να υπάγης. Και έδωκεν εις αυτόν ο αρχισωματοφύλαξ ζωοτροφίας και δώρα και εξαπέστειλεν αυτόν.
\par 6 Και υπήγεν ο Ιερεμίας προς Γεδαλίαν τον υιόν του Αχικάμ εις Μισπά και κατώκησε μετ' αυτού μεταξύ του λαού του εναπολειφθέντος εν τη γη.
\par 7 Ακούσαντες δε πάντες οι αρχηγοί των στρατευμάτων των εν τω αγρώ, αυτοί και οι άνδρες αυτών, ότι ο βασιλεύς της Βαβυλώνος κατέστησε Γεδαλίαν τον υιόν του Αχικάμ επί την γην και έτι ενεπιστεύθη εις αυτόν άνδρας και γυναίκας και παιδία και εκ των πτωχών της γης, εκ των μη μετοικισθέντων εις την Βαβυλώνα,
\par 8 ήλθον προς τον Γεδαλίαν εις Μισπά και Ισμαήλ ο υιός του Νεθανίου και Ιωανάν και Ιωνάθαν οι υιοί του Καρηά και Σεραΐας ο υιός του Τανουμέθ και οι υιοί του Ιωφή του Νετωφαθίτου και Ιεζανίας υιός Μααχαθίτου τινός, αυτοί και οι άνδρες αυτών.
\par 9 Και ώμοσε προς αυτούς Γεδαλίας ο υιός του Αχικάμ υιού του Σαφάν και προς τους άνδρας αυτών, λέγων, Μη φοβείσθε να ήσθε δούλοι των Χαλδαίων· κατοικήσατε εν τη γη και δουλεύετε εις τον βασιλέα της Βαβυλώνος και θέλει είσθαι καλόν εις εσάς.
\par 10 Εγώ δε, ιδού, θέλω κατοικήσει εν Μισπά, διά να παρίσταμαι ενώπιον των Χαλδαίων, οίτινες θέλουσιν ελθεί προς ημάς· και σεις συνάξατε οίνον και οπωρικά και έλαιον και βάλετε αυτά εις τα αγγείά σας και κατοικήσατε εν ταις πόλεσιν υμών, τας οποίας κρατείτε.
\par 11 Παρομοίως πάντες οι Ιουδαίοι οι εν Μωάβ και οι μεταξύ των υιών Αμμών και οι εν Εδώμ και οι εν πάσι τοις τόποις, ότε ήκουσαν ότι ο βασιλεύς της Βαβυλώνος αφήκεν υπόλοιπον εις τον Ιούδαν και ότι κατέστησεν επ' αυτούς Γεδαλίαν τον υιόν του Αχικάμ υιού του Σαφάν,
\par 12 τότε επέστρεψαν πάντες οι Ιουδαίοι εκ πάντων των τόπων όπου ήσαν διεσπαρμένοι και ήλθον εις την γην του Ιούδα, προς τον Γεδαλίαν εις Μισπά, και εσύναξαν οίνον και οπωρικά πολλά σφόδρα.
\par 13 Ο δε Ιωανάν ο υιός του Καρηά και πάντες οι αρχηγοί των στρατευμάτων των εν τω αγρώ ήλθον προς τον Γεδαλίαν εις Μισπά,
\par 14 και είπον προς αυτόν, Εξεύρεις τωόντι ότι ο Βααλείς ο βασιλεύς των υιών Αμμών απέστειλε τον Ισμαήλ τον υιόν του Νεθανίου διά να σε φονεύση; Αλλ' ο Γεδαλίας ο υιός του Αχικάμ δεν επίστευσεν εις αυτούς.
\par 15 Τότε Ιωανάν ο υιός του Καρηά ελάλησε κρυφίως προς τον Γεδαλίαν εν Μισπά, λέγων, Ας υπάγω τώρα και ας πατάξω τον Ισμαήλ τον υιόν του Νεθανίου και ουδείς θέλει μάθει αυτό· διά τι να σε φονεύση και ούτω πάντες οι Ιουδαίοι, οι συνηγμένοι περί σε, να διασκορπισθώσι και το υπόλοιπον του Ιούδα να απολεσθή;
\par 16 Αλλ' ο Γεδαλίας ο υιός του Αχικάμ είπε προς Ιωανάν τον υιόν του Καρηά, Μη κάμης το πράγμα τούτο, διότι ψευδή λέγεις περί του Ισμαήλ.

\chapter{41}

\par Και εν τω εβδόμω μηνί ο Ισμαήλ ο υιός του Νεθανίου, υιού του Ελισαμά, εκ του βασιλικού σπέρματος και εκ των αρχόντων του βασιλέως, και δέκα άνδρες μετ' αυτού, ήλθον προς τον Γεδαλίαν τον υιόν του Αχικάμ εις Μισπά· και συνέφαγον εκεί άρτον εν Μισπά.
\par 2 Και εσηκώθη Ισμαήλ ο υιός του Νεθανίου και οι δέκα άνδρες οι όντες μετ' αυτού και επάταξαν τον Γεδαλίαν τον υιόν του Αχικάμ υιού του Σαφάν διά ρομφαίας και εθανάτωσαν αυτόν, τον οποίον ο βασιλεύς της Βαβυλώνος είχε καταστήσει επί την γην.
\par 3 Και πάντας τους Ιουδαίους τους όντας μετ' αυτού, μετά του Γεδαλίου εν Μισπά, και τους Χαλδαίους τους ευρεθέντας εκεί, άνδρας πολεμιστάς, επάταξεν ο Ισμαήλ.
\par 4 Και την δευτέραν ημέραν, αφού εθανάτωσε τον Γεδαλίαν, και ουδείς είχε μάθει αυτό,
\par 5 τότε τινές από Συχέμ, από Σηλώ και από Σαμαρείας, ογδοήκοντα άνδρες, εξυρισμένοι τους πώγωνας και διεσχισμένοι τα ιμάτια και με εντομάς εις το σώμα, ήρχοντο μετά προσφοράς και λιβανίου εν τη χειρί αυτών, διά να φέρωσιν εις τον οίκον του Κυρίου.
\par 6 Και εξήλθεν Ισμαήλ ο υιός του Νεθανίου εις απάντησιν αυτών εκ Μισπά, κλαίων ενώ επορεύετο· και ότε απήντησεν αυτούς, είπε προς αυτούς, Εισέλθετε προς Γεδαλίαν τον υιόν του Αχικάμ.
\par 7 Και ότε ήλθον εις το μέσον της πόλεως, Ισμαήλ ο υιός του Νεθανίου έσφαξεν αυτούς και έρριψεν εις το μέσον του λάκκου, αυτός και οι άνδρες οι μετ' αυτού.
\par 8 Δέκα όμως άνδρες ευρέθησαν μεταξύ αυτών και είπον προς τον Ισμαήλ, Μη μας θανατώσης, διότι έχομεν θησαυρούς εν τω αγρώ, σίτον και κριθήν και έλαιον και μέλι. Ούτως εκρατήθη και δεν εθανάτωσεν αυτούς μεταξύ των αδελφών αυτών.
\par 9 Ο δε λάκκος, εις τον οποίον ο Ισμαήλ έρριψε πάντα τα πτώματα των ανδρών, τους οποίους επάταξεν εξ αιτίας του Γεδαλίου, ήτο εκείνος, τον οποίον ο βασιλεύς Ασά είχε κάμει υπό φόβου του Βαασά βασιλέως του Ισραήλ· τούτον ο Ισμαήλ ο υιός του Νεθανίου εγέμισεν από των θανατωθέντων.
\par 10 Και ηχμαλώτισεν ο Ισμαήλ άπαν το υπόλοιπον του λαού το εν Μισπά, τας θυγατέρας του βασιλέως και πάντα τον λαόν τον εναπολειφθέντα εν Μισπά, τον οποίον Νεβουζαραδάν ο αρχισωματοφύλαξ είχεν εμπιστευθή εις τον Γεδαλίαν τον υιόν του Αχικάμ· και ηχμαλώτισεν αυτά ο Ισμαήλ ο υιός του Νεθανίου και ανεχώρησε, διά να περάση προς τους υιούς Αμμών.
\par 11 Και ότε ήκουσεν Ιωανάν ο υιός του Καρηά και πάντες οι αρχηγοί των στρατευμάτων οι μετ' αυτού πάντα τα κακά, τα οποία έκαμεν Ισμαήλ ο υιός του Νεθανίου,
\par 12 έλαβον πάντας τους άνδρας και υπήγαν να πολεμήσωσι με τον Ισμαήλ τον υιόν του Νεθανίου· και εύρηκαν αυτόν πλησίον των πολλών υδάτων των εν Γαβαών.
\par 13 Και ως είδε πας ο λαός ο μετά του Ισμαήλ Ιωανάν τον υιόν του Καρηά και πάντας τους αρχηγούς των στρατευμάτων των μετ' αυτού, εχάρησαν.
\par 14 Και εστράφησαν πας ο λαός, τον οποίον ο Ισμαήλ ηχμαλώτισεν από Μισπά, και επέστρεψαν και υπήγαν μετά του Ιωανάν υιού του Καρηά.
\par 15 Αλλ' ο Ισμαήλ ο υιός του Νεθανίου εξέφυγεν από του Ιωανάν μετά οκτώ ανδρών και υπήγε προς τους υιούς Αμμών.
\par 16 Και έλαβεν ο Ιωανάν ο υιός του Καρηά και πάντες οι αρχηγοί των στρατευμάτων οι μετ' αυτού, άπαν το υπόλοιπον του λαού, το οποίον ηλευθέρωσεν από του Ισμαήλ υιού του Νεθανίου, από Μισπά, αφού επάταξε Γεδαλίαν τον υιόν του Αχικάμ, τους δυνατούς άνδρας του πολέμου και τας γυναίκας και τα παιδία και τους ευνούχους, τους οποίους ηχμαλώτισεν από Γαβαών,
\par 17 και υπήγαν και κατώκησαν εν τη κατοικία του Χιμάμ, τη πλησίον της Βηθλεέμ, διά να υπάγωσι να εισέλθωσιν εις την Αίγυπτον,
\par 18 εξ αιτίας των Χαλδαίων· διότι εφοβήθησαν από αυτούς, επειδή ο Ισμαήλ ο υιός του Νεθανίου επάταξε τον Γεδαλίαν τον υιόν του Αχικάμ, τον οποίον ο βασιλεύς της Βαβυλώνος κατέστησεν επί την γην.

\chapter{42}

\par Και προσήλθον πάντες οι αρχηγοί των στρατευμάτων και Ιωανάν ο υιός του Καρηά και Ιεζανίας ο υιός του Ωσαΐου και πας ο λαός από μικρού έως μεγάλου,
\par 2 και είπον προς Ιερεμίαν τον προφήτην, Ας γείνη δεκτή, παρακαλούμεν, η δέησις ημών ενώπιόν σου, και δεήθητι υπέρ ημών προς Κύριον τον Θεόν σου περί πάντων τούτων των εναπολειφθέντων· διότι εμείναμεν ολίγοι εκ πολλών, καθώς οι οφθαλμοί σου βλέπουσιν ημάς
\par 3 διά να φανερώση εις ημάς Κύριος ο Θεός σου την οδόν εις την οποίαν πρέπει να περιπατήσωμεν και το πράγμα το οποίον πρέπει να κάμωμεν.
\par 4 Και είπε προς αυτούς Ιερεμίας ο προφήτης, Ηκουσα· ιδού, θέλω δεηθή προς Κύριον τον Θεόν υμών κατά τους λόγους υμών, και οποιονδήποτε λόγον ο Κύριος αποκριθή περί υμών, θέλω αναγγείλει προς υμάς· δεν θέλω κρύψει ουδέν αφ' υμών.
\par 5 Και αυτοί είπον προς τον Ιερεμίαν, Ο Κύριος ας ήναι αληθής και πιστός μάρτυς μεταξύ ημών, ότι βεβαίως θέλομεν κάμει κατά πάντας τους λόγους, καθ' ους Κύριος ο Θεός σου σε αποστείλη προς ημάς·
\par 6 είτε καλόν και είτε κακόν, θέλομεν υπακούσει εις την φωνήν Κυρίου του Θεού ημών, προς τον οποίον ημείς σε αποστέλλομεν, διά να γείνη καλόν εις ημάς, όταν υπακούσωμεν εις την φωνήν Κυρίου του Θεού ημών.
\par 7 Και μετά δέκα ημέρας έγεινε λόγος Κυρίου προς τον Ιερεμίαν.
\par 8 Και εκάλεσε τον Ιωανάν τον υιόν του Καρηά και πάντας τους αρχηγούς των στρατευμάτων τους μετ' αυτού και πάντα τον λαόν, από μικρού έως μεγάλου,
\par 9 και είπε προς αυτούς, Ούτω λέγει Κύριος ο Θεός του Ισραήλ, προς τον οποίον με απεστείλατε διά να υποβάλω την δέησιν υμών ενώπιον αυτού·
\par 10 Εάν εξακολουθήτε να κατοικήτε εν τη γη ταύτη, τότε θέλω σας οικοδομήσει και δεν θέλω σας κατακρημνίσει, και θέλω σας φυτεύσει και δεν θέλω σας εκριζώσει, διότι μετενόησα διά το κακόν το οποίον έκαμα εις εσάς.
\par 11 Μη φοβηθήτε από του βασιλέως της Βαβυλώνος, από του οποίου τώρα φοβείσθε· μη φοβηθήτε απ' αυτού, λέγει Κύριος, διότι εγώ είμαι μεθ' υμών, διά να σώσω υμάς και να ελευθερώσω υμάς εκ της χειρός αυτού.
\par 12 Και θέλω δώσει οικτιρμούς εις υμάς, διά να οικτείρη υμάς και να επιστρέψη υμάς εις την γην υμών.
\par 13 Αλλ' εάν σεις λέγητε, δεν θέλομεν κατοικήσει εν τη γη ταύτη, μη υπακούοντες εις την φωνήν Κυρίου του Θεού υμών,
\par 14 λέγοντες, Ουχί· αλλά θέλομεν εισέλθει εις την γην της Αιγύπτου, όπου δεν θέλομεν βλέπει πόλεμον, και ήχον σάλπιγγος δεν θέλομεν ακούει, και από άρτον δεν θέλομεν πεινάσει, και εκεί θέλομεν κατοικήσει·
\par 15 διά τούτο, ακούσατε τώρα τον λόγον του Κυρίου, σεις οι υπόλοιποι του Ιούδα· ούτω λέγει ο Κύριος των δυνάμεων, ο Θεός του Ισραήλ· Εάν σεις προσηλώσητε το πρόσωπόν σας εις το να εισέλθητε εις την Αίγυπτον και υπάγητε να παροικήσητε εκεί,
\par 16 τότε η μάχαιρα, την οποίαν σεις φοβείσθε, θέλει σας φθάσει εκεί εν τη γη της Αιγύπτου· και η πείνα, από της οποίας σεις τρομάζετε, θέλει είσθαι προσκεκολλημένη οπίσω σας εκεί εν τη Αιγύπτω, και εκεί θέλετε αποθάνει·
\par 17 και πάντες οι άνδρες οι προσηλώσαντες το πρόσωπον αυτών εις το να υπάγωσιν εις την Αίγυπτον διά να παροικήσωσιν εκεί, θέλουσιν αποθάνει εν μαχαίρα, εν πείνη και εν λοιμώ· και ουδείς εξ αυτών θέλει μείνει ή εκφύγει από του κακού, το οποίον εγώ θέλω φέρει επ' αυτούς.
\par 18 Διότι ούτω λέγει ο Κύριος των δυνάμεων, ο Θεός του Ισραήλ· Καθώς ο θυμός μου και η οργή μου εξεχύθησαν επί τους κατοίκους της Ιερουσαλήμ, ούτως η οργή μου θέλει εκχυθή εφ' υμάς, όταν εισέλθητε εις την Αίγυπτον· και θέλετε είσθαι εις βδέλυγμα και εις θάμβος και εις κατάραν και εις όνειδος· και δεν θέλετε ιδεί πλέον τον τόπον τούτον.
\par 19 Ο Κύριος είπε περί υμών, ω υπόλοιποι του Ιούδα, Μη υπάγητε εις την Αίγυπτον· γνωρίσατε καλώς ότι σήμερον διεμαρτυρήθην εναντίον σας.
\par 20 Διότι σεις εδολιεύθητε εν ταις ψυχαίς υμών, ότε με απεστείλατε προς Κύριον τον Θεόν υμών, λέγοντες, Δεήθητι υπέρ ημών προς Κύριον τον Θεόν ημών· και κατά πάντα όσα λαλήση Κύριος ο Θεός ημών, ούτως απάγγειλον προς ημάς και θέλομεν κάμει.
\par 21 Και απήγγειλα σήμερον προς εσάς· και δεν υπηκούσατε εις την φωνήν Κυρίου του Θεού υμών ουδέ εις πάντα, διά τα οποία με απέστειλε προς εσάς.
\par 22 Τώρα λοιπόν εξεύρετε βεβαίως, ότι θέλετε αποθάνει εν μαχαίρα, εν πείνη και εν λοιμώ, εν τω τόπω όπου επιθυμείτε να υπάγητε διά να παροικήσητε εκεί.

\chapter{43}

\par Και ότε έπαυσεν ο Ιερεμίας λαλών προς πάντα τον λαόν πάντας τους λόγους Κυρίου του Θεού αυτών, διά τους οποίους Κύριος ο Θεός αυτών απέστειλεν αυτόν προς αυτούς, πάντας τους λόγους τούτους,
\par 2 τότε ελάλησεν Αζαρίας ο υιός του Ωσαΐου και Ιωανάν ο υιός του Καρηά και πάντες οι υπερήφανοι άνδρες, λέγοντες προς τον Ιερεμίαν· Ψεύδεσαι· Κύριος ο Θεός ημών δεν σε απέστειλε να είπης, Μη υπάγητε εις την Αίγυπτον διά να παροικήσητε εκεί·
\par 3 αλλ' ο Βαρούχ ο υιός του Νηρίου σε διεγείρει εναντίον ημών, διά να μας παραδώσης εις την χείρα των Χαλδαίων, να μας θανατώσωσι και να μας φέρωσιν αιχμαλώτους εις την Βαβυλώνα.
\par 4 Και δεν υπήκουσεν Ιωανάν ο υιός του Καρηά και πάντες οι αρχηγοί των στρατευμάτων και πας ο λαός εις την φωνήν του Κυρίου, να κατοικήσωσιν εν γη Ιούδα·
\par 5 αλλ' ο Ιωανάν ο υιός του Καρηά και πάντες οι αρχηγοί των στρατευμάτων έλαβον πάντας τους υπολοίπους του Ιούδα, τους επιστρέψαντας εκ πάντων των εθνών, όπου είχον διασπαρή, διά να κατοικήσωσιν εν γη Ιούδα,
\par 6 τους άνδρας και τας γυναίκας και τα παιδία και τας θυγατέρας του βασιλέως και πάντα άνθρωπον, τον οποίον Νεβουζαραδάν ο αρχισωματοφύλαξ είχεν αφήσει μετά του Γεδαλίου υιού του Αχικάμ υιού του Σαφάν, και τον Ιερεμίαν τον προφήτην και τον Βαρούχ τον υιόν του Νηρίου·
\par 7 και εισήλθον εις την γην της Αιγύπτου· διότι δεν υπήκουσαν εις την φωνήν του Κυρίου· και ήλθον έως Τάφνης.
\par 8 Και έγεινε λόγος Κυρίου προς τον Ιερεμίαν εν Τάφνης, λέγων,
\par 9 Λάβε εις την χείρα σου λίθους μεγάλους και κρύψον αυτούς έμπροσθεν των οφθαλμών των ανδρών των Ιουδαίων εν αργίλλω εν τη κεραμεική καμίνω τη προς την είσοδον της οικίας του Φαραώ, εν Τάφνης·
\par 10 και ειπέ προς αυτούς, Ούτω λέγει ο Κύριος των δυνάμεων, ο Θεός του Ισραήλ· Ιδού, θέλω εξαποστείλει και λάβει τον Ναβουχοδονόσορ τον βασιλέα της Βαβυλώνος, τον δούλον μου, και θέλω θέσει τον θρόνον αυτού επάνω των λίθων τούτων, τους οποίους έκρυψα· και θέλει απλώσει την βασιλικήν αυτού σκηνήν επάνω αυτών.
\par 11 Και θέλει ελθεί και πατάξει την γην της Αιγύπτου και παραδώσει τους μεν διά θάνατον, εις θάνατον, τους δε διά αιχμαλωσίαν, εις αιχμαλωσίαν, τους δε διά ρομφαίαν, εις ρομφαίαν.
\par 12 Και θέλω ανάψει πυρ εν ταις οικίαις των θεών της Αιγύπτου, και τας μεν θέλει κατακαύσει τους δε θέλει φέρει εις αιχμαλωσίαν· και θέλει ενδυθή την γην της Αιγύπτου, καθώς ο ποιμήν ενδύεται το ιμάτιον αυτού, και θέλει εξέλθει εκείθεν εν ειρήνη.
\par 13 Και θέλει συντρίψει τα είδωλα του οίκου του ηλίου, του εν τη γη της Αιγύπτου· και τους οίκους των θεών των Αιγυπτίων θέλει κατακαύσει εν πυρί.

\chapter{44}

\par Ο λόγος ο γενόμενος προς τον Ιερεμίαν περί πάντων των Ιουδαίων των κατοικούντων εν τη γη της Αιγύπτου, των κατοικούντων εν Μιγδώλ και εν Τάφνης και εν Νωφ και εν τη γη Παθρώς, λέγων,
\par 2 Ούτω λέγει ο Κύριος των δυνάμεων, ο Θεός του Ισραήλ· σεις είδετε πάντα τα κακά τα οποία επέφερα επί την Ιερουσαλήμ και επί πάσας τας πόλεις του Ιούδα, και ιδού, αύται έρημοι την σήμερον και δεν υπάρχει ο κατοικών εν αυταίς,
\par 3 εξ αιτίας της κακίας αυτών, την οποίαν έπραξαν διά να με παροργίσωσιν, υπάγοντες να θυμιάζωσι και να λατρεύωσιν άλλους θεούς, τους οποίους δεν εγνώρισαν αυτοί, σεις, ουδέ οι πατέρες σας.
\par 4 Και απέστειλα προς εσάς πάντας τους δούλους μου τους προφήτας, εγειρόμενος πρωΐ και αποστέλλων, λέγων, Μη πράττετε το βδελυρόν τούτο πράγμα, το οποίον μισώ.
\par 5 Αλλά δεν ήκουσαν ουδέ έκλιναν το ωτίον αυτών διά να επιστρέψωσιν από της κακίας αυτών, ώστε να μη θυμιάζωσιν εις άλλους θεούς.
\par 6 Διά τούτο εξεχύθη η οργή μου και ο θυμός μου και εξεκαύθη εν ταις πόλεσι του Ιούδα και εν ταις πλατείαις της Ιερουσαλήμ· και έγειναν έρημοι, άβατοι, ως την ημέραν ταύτην.
\par 7 Και τώρα ούτω λέγει Κύριος ο Θεός των δυνάμεων, ο Θεός του Ισραήλ· Διά τι σεις πράττετε το μέγα τούτο κακόν εναντίον των ψυχών σας, ώστε να αφανίσητε αφ' υμών άνδρα και γυναίκα, νήπιον και θηλάζον, εκ μέσου του Ιούδα, διά να μη μείνη εις εσάς υπόλοιπον·
\par 8 παροργίζοντές με διά των έργων των χειρών σας, θυμιάζοντες εις άλλους θεούς εν τη γη της Αιγύπτου, όπου ήλθετε να παροικήσητε εκεί, ώστε να αφανίσητε εαυτούς και να γείνητε κατάρα και όνειδος μεταξύ πάντων των εθνών της γης.
\par 9 Μήπως ελησμονήσατε τας κακίας των πατέρων σας και τας κακίας των βασιλέων του Ιούδα και τας κακίας των γυναικών αυτών και τας κακίας σας και τας κακίας των γυναικών σας, τας οποίας έπραξαν εν τη γη του Ιούδα και εν ταις πλατείαις της Ιερουσαλήμ;
\par 10 Δεν εταπεινώθησαν έως της ημέρας ταύτης ουδέ εφοβήθησαν ουδέ περιεπάτησαν εν τω νόμω μου και εν τοις διατάγμασί μου, τα οποία έθεσα ενώπιόν σας και ενώπιον των πατέρων σας.
\par 11 Διά τούτο ούτω λέγει ο Κύριος των δυνάμεων, ο Θεός του Ισραήλ· Ιδού, εγώ θέλω στήσει το πρόσωπόν μου εναντίον υμών εις κακόν, και διά να εξολοθρεύσω πάντα τον Ιούδαν.
\par 12 Και θέλω λάβει τους υπολοίπους του Ιούδα, οίτινες έστησαν το πρόσωπον αυτών εις το να υπάγωσιν εις την γην της Αιγύπτου, διά να παροικήσωσιν εκεί, και θέλουσι καταναλωθή πάντες εν τη γη της Αιγύπτου· θέλουσι πέσει εν μαχαίρα, θέλουσι καταναλωθή εν πείνη· από μικρού έως μεγάλου εν μαχαίρα και εν πείνη θέλουσιν αποθάνει· και θέλουσιν είσθαι εις βδέλυγμα, εις θάμβος και εις κατάραν και εις όνειδος.
\par 13 Διότι θέλω επισκεφθή τους κατοικούντας εν τη γη της Αιγύπτου, ως επεσκέφθην την Ιερουσαλήμ, εν μαχαίρα, εν πείνη και εν λοιμώ.
\par 14 Και ουδείς εκ των υπολοίπων του Ιούδα, των απελθόντων εις την γην της Αιγύπτου διά να παροικήσωσιν εκεί, θέλει εκφύγει ή διασωθή, διά να επιστρέψη εις την γην του Ιούδα, εις την οποίαν αυτοί έχουσι προσηλωμένην την ψυχήν αυτών, διά να επιστρέψωσι να κατοικήσωσιν εκεί· διότι δεν θέλουσιν επιστρέψει, ειμή οι διασεσωσμένοι.
\par 15 Και πάντες οι άνδρες οι γνωρίζοντες ότι αι γυναίκες αυτών εθυμίαζον εις άλλους θεούς, και πάσαι αι γυναίκες αι παρεστώσαι, σύναξις μεγάλη, και πας ο λαός οι κατοικούντες εν τη γη της Αιγύπτου, εν Παθρώς, απεκρίθησαν προς τον Ιερεμίαν, λέγοντες,
\par 16 Περί του λόγου, τον οποίον ελάλησας προς ημάς εν ονόματι Κυρίου, δεν θέλομεν σου ακούσει·
\par 17 αλλά θέλομεν εξάπαντος κάμνει παν πράγμα εξερχόμενον εκ του στόματος ημών, διά να θυμιάζωμεν εις την βασίλισσαν του ουρανού και να κάμνωμεν σπονδάς εις αυτήν, καθώς εκάμνομεν, ημείς και οι πατέρες ημών, οι βασιλείς ημών και οι άρχοντες ημών, εν ταις πόλεσι του Ιούδα και εν ταις πλατείαις της Ιερουσαλήμ· και εχορταίνομεν άρτον και διεκείμεθα καλώς και κακόν δεν εβλέπομεν.
\par 18 Αλλ' αφ' ότου επαύσαμεν θυμιάζοντες εις την βασίλισσαν του ουρανού και κάμνοντες σπονδάς εις αυτήν, ώστε εστερήθημεν πάντων και κατηναλώθημεν εν μαχαίρα και εν πείνη.
\par 19 Και ότε ημείς εθυμιάζομεν εις την βασίλισσαν του ουρανού και εκάμνομεν σπονδάς εις αυτήν, μήπως άνευ των ανδρών ημών εκάμνομεν εις αυτήν πέμματα διά να προσκυνώμεν αυτήν και εκάμνομεν εις αυτήν σπονδάς;
\par 20 Και είπεν ο Ιερεμίας προς πάντα τον λαόν, προς άνδρας τε και γυναίκας και προς πάντα τον λαόν, τους αποκριθέντας προς αυτόν ούτω, λέγων,
\par 21 Μήπως το θυμίαμα, το οποίον εθυμιάζετε εν ταις πόλεσι του Ιούδα και εν ταις πλατείαις της Ιερουσαλήμ, σεις και οι πατέρες σας, οι βασιλείς σας και οι άρχοντές σας και ο λαός του τόπου, δεν ενεθυμήθη αυτό ο Κύριος και δεν ανέβη εις την καρδίαν αυτού;
\par 22 Ώστε ο Κύριος δεν ηδυνήθη πλέον να υποφέρη, εξ αιτίας της κακίας των έργων σας, εξ αιτίας των βδελυγμάτων, τα οποία επράττετε· όθεν η γη σας κατεστάθη ερήμωσις και θάμβος και κατάρα, άνευ κατοίκου, ως την ημέραν ταύτην.
\par 23 Επειδή εθυμιάζετε και επειδή ημαρτάνετε εις τον Κύριον και δεν υπηκούσατε εις την φωνήν του Κυρίου ουδέ περιεπατήσατε εν τω νόμω αυτού και εν τοις διατάγμασιν αυτού και εν τοις μαρτυρίοις αυτού, διά τούτο συνέβη εις εσάς το κακόν τούτο, ως την ημέραν ταύτην.
\par 24 Και είπεν ο Ιερεμίας προς πάντα τον λαόν και προς πάσας τας γυναίκας, Ακούσατε τον λόγον του Κυρίου, πας ο Ιούδας, ο εν τη γη της Αιγύπτου·
\par 25 ούτως ελάλησεν ο Κύριος των δυνάμεων, ο Θεός του Ισραήλ, λέγων, Σεις και αι γυναίκές σας και ελαλήσατε διά του στόματός σας και εξετελέσατε διά της χειρός σας, λέγοντες, Θέλομεν εξάπαντος εκπληρώσει τας ευχάς ημών, τας οποίας ηυχήθημεν, να θυμιάζωμεν εις την βασίλισσαν του ουρανού και να κάμνωμεν σπονδάς εις αυτήν· εξάπαντος λοιπόν θέλετε εκπληρώσει τας ευχάς σας και εξάπαντος θέλετε εκτελέσει τας ευχάς σας.
\par 26 Διά τούτο ακούσατε τον λόγον του Κυρίου, πας ο Ιούδας, οι κατοικούντες εν τη γη της Αιγύπτου· Ιδού, ώμοσα εις το όνομά μου το μέγα, λέγει Κύριος, ότι το όνομά μου δεν θέλει ονομασθή πλέον εν τω στόματι ουδενός ανδρός του Ιούδα, καθ' όλην την γην της Αιγύπτου, ώστε να λέγη, Ζη Κύριος ο Θεός.
\par 27 Ιδού, εγώ θέλω επαγρυπνεί επ' αυτούς εις κακόν και ουχί εις καλόν· και πάντες οι άνδρες του Ιούδα οι εν τη γη της Αιγύπτου θέλουσι καταναλωθή εν μαχαίρα και εν πείνη, εωσού εκλείψωσιν.
\par 28 Οι δε διασεσωσμένοι από της μαχαίρας, ολίγοι τον αριθμόν, θέλουσιν επιστρέψει εκ γης Αιγύπτου εις γην Ιούδα· και πάντες οι υπόλοιποι του Ιούδα, οι απελθόντες εις την γην της Αιγύπτου διά να παροικήσωσιν εκεί, θέλουσι γνωρίσει τίνος λόγος θέλει πληρωθή, ο εμός, ή αυτών.
\par 29 Και τούτο θέλει είσθαι σημείον εις σας, λέγει Κύριος, ότι εγώ θέλω σας τιμωρήσει εν τω τόπω τούτω, διά να γνωρίσητε ότι οι λόγοι μου θέλουσιν εξάπαντος πληρωθή εναντίον σας εις κακόν·
\par 30 ούτω λέγει Κύριος· Ιδού, εγώ θέλω παραδώσει τον Φαραώ-ουαφρή, βασιλέα της Αιγύπτου, εις την χείρα των εχθρών αυτού και εις την χείρα των ζητούντων την ψυχήν αυτού, καθώς παρέδωκα τον Σεδεκίαν βασιλέα του Ιούδα εις την χείρα του Ναβουχοδονόσορ βασιλέως της Βαβυλώνος, του εχθρού αυτού και ζητούντος την ψυχήν αυτού.

\chapter{45}

\par Ο λόγος, τον οποίον Ιερεμίας ο προφήτης ελάλησε προς τον Βαρούχ, τον υιόν του Νηρίου, ότε έγραψε τους λόγους τούτους εν βιβλίω εκ στόματος του Ιερεμίου, εν τω τετάρτω έτει του Ιωακείμ υιού του Ιωσίου, βασιλέως του Ιούδα, λέγων,
\par 2 Ούτω λέγει Κύριος ο Θεός του Ισραήλ περί σου, Βαρούχ·
\par 3 Είπας, Ουαί εις εμέ τώρα διότι ο Κύριος επρόσθεσε πόνον επί την θλίψιν μου· απέκαμον εν τω στεναγμώ μου και ανάπαυσιν δεν ευρίσκω.
\par 4 Ούτω θέλεις ειπεί προς αυτόν· Ούτω λέγει ο Κύριος· Ιδού, εκείνο το οποίον ωκοδόμησα, εγώ θέλω κατεδαφίσει· και εκείνο το οποίον εφύτευσα, εγώ θέλω εκριζώσει, και σύμπασαν την γην αυτήν.
\par 5 Και συ ζητείς εις σεαυτόν μεγάλα; μη ζητής· διότι, ιδού, εγώ θέλω φέρει κακά επί πάσαν σάρκα, λέγει Κύριος, αλλά την ζωήν σου θέλω δώσει εις σε ως λάφυρον, εν πάσι τοις τόποις όπου υπάγης.

\chapter{46}

\par Ο λόγος του Κυρίου ο γενόμενος προς τον Ιερεμίαν τον προφήτην κατά των εθνών.
\par 2 κατά της Αιγύπτου, κατά της δυνάμεως του Φαραώ-νεχαώ βασιλέως της Αιγύπτου, ήτις ήτο παρά τον ποταμόν Ευφράτην εν Χαρκεμίς, την οποίαν επάταξε Ναβουχοδονόσορ ο βασιλεύς της Βαβυλώνος εν τω τετάρτω έτει του Ιωακείμ, υιού του Ιωσίου, βασιλέως του Ιούδα.
\par 3 Αναλάβετε ασπίδα και θυρεόν και προσέλθετε εις πόλεμον.
\par 4 Ζεύξατε τους ίππους και ανάβητε, ιππείς, και παραστάθητε με περικεφαλαίας· στιλβώσατε τας λόγχας, ενδύθητε τους θώρακας.
\par 5 Διά τι είδον αυτούς επτοημένους, τρεπομένους εις τα οπίσω; οι δε ισχυροί αυτών συνετρίβησαν και έφυγον μετά σπουδής, χωρίς να βλέπωσιν εις τα οπίσω· τρόμος πανταχόθεν, λέγει Κύριος.
\par 6 Ο ταχύς ας μη εκφύγη, και ο ισχυρός ας μη διασωθή· θέλουσι προσκόψει και θέλουσι πέσει προς βορράν, παρά τον ποταμόν Ευφράτην.
\par 7 Τις ούτος, ο αναβαίνων ως πλημμύρα, του οποίου τα ύδατα κυλινδούνται ως ποταμοί;
\par 8 Η Αίγυπτος αναβαίνει ως πλημμύρα και τα ύδατα αυτής κυλινδούνται ως ποταμοί· και λέγει, Θέλω αναβή· θέλω σκεπάσει την γήν· θέλω αφανίσει την πόλιν και τους κατοικούντας εν αυτή.
\par 9 Αναβαίνετε, ίπποι, και μαίνεσθε, άμαξαι· και ας εξέλθωσιν οι ισχυροί, οι Αιθίοπες και οι Λίβυες οι κρατούντες την ασπίδα και οι Λύδιοι οι κρατούντες και εντείνοντες τόξον.
\par 10 Διότι αύτη η ημέρα είναι εις Κύριον τον Θεόν των δυνάμεων, ημέρα εκδικήσεως, διά να εκδικηθή τους εχθρούς αυτού· και η μάχαιρα θέλει καταφάγει αυτούς και θέλει χορτασθή και μεθυσθή από του αίματος αυτών· διότι Κύριος ο Θεός των δυνάμεων έχει θυσίαν εν τη γη του βορρά, παρά τον ποταμόν Ευφράτην.
\par 11 Ανάβα εις Γαλαάδ και λάβε βάλσαμον, παρθένε, θυγάτηρ της Αιγύπτου· ματαίως θέλεις πληθύνει τα ιατρικά· θεραπεία δεν υπάρχει διά σε.
\par 12 Τα έθνη ήκουσαν την αισχύνην σου, και η κραυγή σου ενέπλησε την γήν· διότι ισχυρός προσέκρουσεν επ' ισχυρόν, επί το αυτό έπεσον αμφότεροι.
\par 13 Ο λόγος, τον οποίον ελάλησεν ο Κύριος προς Ιερεμίαν τον προφήτην, περί της ελεύσεως του Ναβουχοδονόσορ βασιλέως της Βαβυλώνος, διά να πατάξη την γην της Αιγύπτου·
\par 14 Αναγγείλατε εν Αιγύπτω και κηρύξατε εν Μιγδώλ και κηρύξατε εν Νωφ και εν Τάφνης· είπατε, Παραστάθητι και ετοιμάσθητι· διότι η μάχαιρα κατέφαγε τους περί σε.
\par 15 Διά τι εστρώθησαν κατά γης οι ανδρείοί σου; δεν στέκουσι, διότι ο Κύριος απέσπρωξεν αυτούς.
\par 16 Επλήθυνε τους προσκρούοντας, μάλιστα έπιπτεν ο εις επί τον άλλον· και έλεγον, Σηκώθητι και ας επαναστρέψωμεν εις τον λαόν ημών και εις την γην της γεννήσεως ημών από προσώπου της εξολοθρευτικής μαχαίρας.
\par 17 Εβόησαν εκεί, Φαραώ, ο βασιλεύς της Αιγύπτου, απωλέσθη, επέρασε τον διωρισμένον καιρόν.
\par 18 Ζω εγώ, λέγει ο Βασιλεύς, του οποίου το όνομα είναι ο Κύριος των δυνάμεων, Εξάπαντος καθώς το Θαβώρ είναι μεταξύ των ορέων και καθώς ο Κάρμηλος πλησίον της θαλάσσης, ούτω θέλει ελθεί εκείνος.
\par 19 Θυγάτηρ, η κατοικούσα εν Αιγύπτω, παρασκευάσθητι εις αιχμαλωσίαν· διότι η Νωφ θέλει αφανισθή και ερημωθή, ώστε να μη υπάρχη ο κατοικών.
\par 20 Η Αίγυπτος είναι ως δάμαλις ώραιοτάτη, πλην ο όλεθρος έρχεται· έρχεται από βορρά.
\par 21 Και αυτοί οι μισθωτοί αυτής είναι εν μέσω αυτής ως μόσχοι παχείς· διότι και αυτοί εστράφησαν, έφυγον ομού· δεν εστάθησαν, επειδή η ημέρα της συμφοράς αυτών ήλθεν επ' αυτούς, ο καιρός της επισκέψεως αυτών.
\par 22 Η φωνή αυτής θέλει εξέλθει ως όφεως· διότι θέλουσι κινηθή εν δυνάμει και θέλουσιν επέλθει επ' αυτήν με πελέκεις, ως ξυλοκόποι.
\par 23 Θέλουσι κατακόψει το δάσος αυτής, λέγει Κύριος, αν και ήναι αμέτρητον· διότι είναι κατά το πλήθος υπέρ την ακρίδα και αναρίθμητοι.
\par 24 Θέλει καταισχυνθή η θυγάτηρ της Αιγύπτου· θέλει παραδοθή εις την χείρα του λαού του βορρά.
\par 25 Ο Κύριος των δυνάμεων, ο Θεός του Ισραήλ, λέγει, Ιδού, θέλω τιμωρήσει το πλήθος της Νω και τον Φαραώ και την Αίγυπτον και τους θεούς αυτής και τους βασιλείς αυτής, τον Φαραώ αυτόν και τους επ' αυτόν θαρρούντας·
\par 26 και θέλω παραδώσει αυτούς εις την χείρα των ζητούντων την ψυχήν αυτών και εις την χείρα του Ναβουχοδονόσορ βασιλέως της Βαβυλώνος και εις την χείρα των δούλων αυτού· και μετά ταύτα θέλει κατοικηθή, καθώς εις τας πρότερον ημέρας, λέγει Κύριος.
\par 27 Συ δε μη φοβηθής, δούλέ μου Ιακώβ, μηδέ δειλιάσης, Ισραήλ· διότι ιδού, θέλω σε σώσει από του μακρυνού τόπου και το σπέρμα σου από της γης της αιχμαλωσίας αυτών· και ο Ιακώβ θέλει επιστρέψει και θέλει ησυχάσει και αναπαυθή και δεν θέλει υπάρχει ο εκφοβών.
\par 28 Μη φοβηθής συ, δούλέ μου Ιακώβ, λέγει Κύριος· διότι εγώ είμαι μετά σού· διότι και αν κάμω συντέλειαν πάντων των εθνών όπου σε έξωσα, εις σε όμως δεν θέλω κάμει συντέλειαν, αλλά θέλω σε παιδεύσει εν κρίσει και δεν θέλω όλως σε αθωώσει.

\chapter{47}

\par Ο λόγος του Κυρίου, ο γενόμενος προς Ιερεμίαν τον προφήτην, κατά των Φιλισταίων, πριν πατάξη την Γάζαν ο Φαραώ.
\par 2 Ούτω λέγει Κύριος· Ιδού, ύδατα αναβαίνουσιν από βορρά, και θέλουσιν είσθαι χείμαρρος πλημμυρών, και θέλουσι πλημμυρήσει την γην και το πλήρωμα αυτής, την πόλιν και τους κατοικούντας εν αυτή· τότε οι άνθρωποι θέλουσιν αναβοήσει και πάντες οι κάτοικοι της γης θέλουσιν ολολύξει.
\par 3 Υπό του κρότου των πατημάτων των όπλων των ρωμαλέων αυτού ίππων, υπό του σεισμού των αμαξών αυτού, υπό του ήχου των τροχών αυτού, οι πατέρες δεν θέλουσι στραφή προς τα τέκνα διά την ατονίαν των χειρών,
\par 4 εξ αιτίας της ημέρας της επερχομένης διά να απολέση πάντας τους Φιλισταίους, να εκκόψη από της Τύρου και της Σιδώνος πάντα εναπολειφθέντα βοηθόν· διότι ο Κύριος θέλει αφανίσει τους Φιλισταίους, το υπόλοιπον της νήσου Καφθόρ.
\par 5 Φαλάκρωμα ήλθεν επί την Γάζαν· η Ασκάλων απωλέσθη μετά του υπολοίπου της κοιλάδος αυτών. Έως πότε θέλεις κάμνει εντομάς εις σεαυτήν;
\par 6 Ω μάχαιρα του Κυρίου, έως πότε δεν θέλεις ησυχάσει; είσελθε εις την θήκην σου, αναπαύθητι και ησύχασον.
\par 7 Πως να ησυχάσης; διότι ο Κύριος έδωκεν εις αυτήν παραγγελίαν κατά της Ασκάλωνος και κατά της παραθαλασσίου· εκεί διώρισεν αυτήν.

\chapter{48}

\par Κατά του Μωάβ. Ούτω λέγει ο Κύριος των δυνάμεων, ο Θεός του Ισραήλ· Ουαί εις την Νεβώ· διότι απωλέσθη· η Κιριαθαΐμ κατησχύνθη, εκυριεύθη· η Μισγάβ κατησχύνθη και ετρόμαξε.
\par 2 δεν θέλει είσθαι πλέον καύχημα εις τον Μωάβ· εν Εσεβών κακόν εβουλεύθησαν εναντίον αυτής· Έλθετε και ας εξαλείψωμεν αυτήν από του να ήναι έθνος· και συ, Μαδμέν, θέλεις κατεδαφισθή· μάχαιρα θέλει σε καταδιώξει.
\par 3 Φωνή κραυγής από Οροναΐμ, λεηλασία και σύντριμμα μέγα.
\par 4 Ο Μωάβ συνετρίβη· τα παιδία αυτού εξέπεμψαν κραυγήν.
\par 5 Διότι εις την ανάβασιν της Λουείθ θέλει υψωθή κλαυθμός επί κλαυθμόν, επειδή εις την κατάβασιν του Οροναΐμ ήκουσαν οι εχθροί κραυγήν συντρίμματος.
\par 6 Φύγετε, σώσατε την ζωήν σας, και γένεσθε ως αγριομυρίκη εν τη ερήμω.
\par 7 Διότι, επειδή ήλπισας επί τα οχυρώματά σου και επί τους θησαυρούς σου, και συ αυτός θέλεις πιασθή· και ο Χεμώς θέλει εξέλθει εις αιχμαλωσίαν, οι ιερείς αυτού και οι άρχοντες αυτού ομού.
\par 8 Και θέλει ελθεί επί πάσαν πόλιν ο εξολοθρευτής, και πόλις δεν θέλει εκφύγει· η κοιλάς ότι θέλει απολεσθή και η πεδινή θέλει αφανισθή, καθώς είπε Κύριος.
\par 9 Δότε πτέρυγας εις τον Μωάβ, διά να πετάξη και να εκφύγη· διότι αι πόλεις αυτού θέλουσιν ερημωθή, χωρίς να υπάρχη ο κατοικών εν αυταίς.
\par 10 Επικατάρατος ο ποιών το έργον του Κυρίου αμελώς· και επικατάρατος ο αποσύρων την μάχαιραν αυτού από αίματος.
\par 11 Ο Μωάβ εστάθη ατάραχος εκ νεότητος αυτού και ανεπαύετο επί την τρυγίαν αυτού και δεν εξεκενώθη από αγγείον εις αγγείον ουδέ υπήγεν εις αιχμαλωσίαν· διά τούτο η γεύσις αυτού έμεινεν εις αυτόν, και η οσμή αυτού δεν μετεβλήθη.
\par 12 Διά τούτο, ιδού, έρχονται ημέραι, λέγει Κύριος, και θέλω αποστείλει επ' αυτόν μετατοπιστάς και θέλουσι μετατοπίσει αυτόν· και θέλουσιν εκκενώσει τα αγγεία αυτού και συντρίψει τους πίθους αυτού.
\par 13 Και ο Μωάβ θέλει αισχυνθή διά τον Χεμώς, καθώς ησχύνθη ο οίκος Ισραήλ διά την Βαιθήλ την ελπίδα αυτών.
\par 14 Πως λέγετε, Ημείς είμεθα ισχυροί και άνδρες δυνατοί εις πόλεμον;
\par 15 Ο Μωάβ ελεηλατήθη, και επυρπολήθησαν αι πόλεις αυτού, και οι εκλεκτοί νέοι αυτού κατέβησαν εις σφαγήν, λέγει ο Βασιλεύς, του οποίου το όνομα είναι ο Κύριος των δυνάμεων.
\par 16 Η συμφορά του Μωάβ πλησιάζει να έλθη, και η θλίψις αυτού σπεύδει σφόδρα.
\par 17 Πάντες οι κύκλω αυτού, θρηνήσατε αυτόν· και πάντες οι γνωρίζοντες το όνομα αυτού, είπατε, Πως συνετρίβη η δυνατή ράβδος, η ένδοξος βακτηρία.
\par 18 Θυγάτηρ, η κατοικούσα εν Δαιβών, κατάβα από της δόξης και κάθησον εν ανύδρω· διότι ο λεηλάτης του Μωάβ αναβαίνει επί σε και θέλει αφανίσει τα οχυρώματά σου.
\par 19 Η κατοικούσα εν Αροήρ, στήθι πλησίον της οδού και παρατήρησον· ερώτησον τον φεύγοντα και την διασωζομένην και ειπέ, Τι έγεινεν;
\par 20 Ο Μωάβ κατησχύνθη· διότι συνετρίβη· ολόλυξον και βόησον. αναγγείλατε εις Αρνών ότι ο Μωάβ ελεηλατήθη,
\par 21 και κρίσις ήλθεν επί την γην της πεδινής, επί Ωλών και επί Ιαασά και επί Μηφαάθ,
\par 22 και επί Δαιβών και επί Νεβώ και επί Βαιθ-δεβλαθαΐμ,
\par 23 και επί Κιριαθαΐμ και επί Βαιθ-γαμούλ και επί Βαιθ-μεών,
\par 24 και επί Κεριώθ και επί Βοσόρρα και επί πάσας τας πόλεις της γης Μωάβ, τας μακράν και τας εγγύς.
\par 25 Το κέρας του Μωάβ συνεθλάσθη και ο βραχίων αυτού συνετρίβη, λέγει Κύριος.
\par 26 Μεθύσατε αυτόν· διότι εμεγαλύνθη κατά του Κυρίου· και ο Μωάβ θέλει κυλισθή εις τον εμετόν αυτού και θέλει είσθαι εις γέλωτα και αυτός.
\par 27 Διότι μήπως ο Ισραήλ δεν εστάθη γέλως εις σε; μήπως ευρέθη μεταξύ κλεπτών; διότι οσάκις ομιλείς περί αυτού, σκιρτάς υπό χαράς.
\par 28 Κάτοικοι του Μωάβ, καταλίπετε τας πόλεις και κατοικήσατε εν πέτρα και γένεσθε ως περιστερά φωλεύουσα εις τα πλάγια του στόματος του σπηλαίου.
\par 29 Ηκούσαμεν την υπερηφανίαν του Μωάβ, του καθ' υπερβολήν υπερηφάνου· την υψηλοφροσύνην αυτού και την αλαζονείαν αυτού και την υπερηφανίαν αυτού και την έπαρσιν της καρδίας αυτού.
\par 30 Εγώ γνωρίζω την μανίαν αυτού, λέγει Κύριος, πλην ουχί ούτω· τα ψεύδη αυτού δεν θέλουσι τελεσφορήσει.
\par 31 Διά τούτο θέλω ολολύξει διά τον Μωάβ και θέλω αναβοήσει διά όλον τον Μωάβ· θέλουσι θρηνολογήσει διά τους άνδρας της Κιρ-έρες.
\par 32 Άμπελε της Σιβμά, θέλω κλαύσει διά σε υπέρ τον κλαυθμόν της Ιαζήρ· τα κλήματά σου διεπέραααν την θάλασσαν, έφθασαν έως της θαλάσσης της Ιαζήρ· ο λεηλάτης επέπεσεν επί το θέρος σου και επί τον τρυγητόν σου.
\par 33 Και χαρά και αγαλλίασις εξηλείφθη από της καρποφόρου πεδιάδος και από της γης Μωάβ· και αφήρεσα τον οίνον από των ληνών· ουδείς θέλει ληνοπατήσει αλαλάζων· αλαλαγμός δεν θέλει ακουσθή.
\par 34 Διά την κραυγήν της Εσεβών, ήτις έφθασεν έως Ελεαλή και έως Ιαάς, αυτοί έδωκαν την φωνήν αυτών από Σηγώρ έως Οροναΐμ ως δάμαλις τριετής· διότι και τα ύδατα του Νιμρείμ θέλουσιν εκλείψει.
\par 35 Και θέλω παύσει εν τω Μωάβ, λέγει Κύριος, τον προσφέροντα ολοκαύτωμα εις τους υψηλούς τόπους και τον θυμιάζοντα εις τους θεούς αυτού.
\par 36 Διά τούτο η καρδία μου θέλει βομβήσει διά τον Μωάβ ως αυλός και η καρδία μου θέλει βομβήσει ως αυλός διά τους άνδρας της Κιρ-έρες· διότι τα αποκτηθέντα εις αυτήν αγαθά απωλέσθησαν.
\par 37 Διότι πάσα κεφαλή θέλει είσθαι φαλακρά και πας πώγων εξυρισμένος· επί πάσας τας χείρας θέλουσιν είσθαι εντομαί και επί την οσφύν σάκκος.
\par 38 Επί πάντα τα δώματα του Μωάβ και επί πάσας τας πλατείας αυτού θρήνος θέλει είσθαι· διότι συνέτριψα τον Μωάβ ως σκεύος εν ω δεν υπάρχει χάρις, λέγει Κύριος.
\par 39 Ολολύξατε, λέγοντες, Πως συνετρίβη· πως ο Μωάβ έστρεψε τα νώτα εν καταισχύνη· ούτως ο Μωάβ θέλει είσθαι γέλως και φρίκη εις πάντας τους περί αυτόν.
\par 40 Διότι ούτω λέγει Κύριος· Ιδού, θέλει πετάξει ως αετός, και θέλει απλώσει τας πτέρυγας αυτού επί τον Μωάβ.
\par 41 Η Κεριώθ εκυριεύθη και τα οχυρώματα επιάσθησαν, και αι καρδίαι των ισχυρών του Μωάβ θέλουσιν είσθαι κατ' εκείνην την ημέραν ως καρδία γυναικός κοιλοπονούσης.
\par 42 Και ο Μωάβ θέλει εξαλειφθή από του να ήναι λαός, διότι εμεγαλύνθη κατά του Κυρίου.
\par 43 Φόβος και λάκκος και παγίς θέλουσιν είσθαι επί σε, κάτοικε του Μωάβ, λέγει Κύριος.
\par 44 Ο εκφυγών από του φόβου θέλει πέσει εις τον λάκκον, και ο αναβάς εκ του λάκκου θέλει πιασθή εν τη παγίδι· διότι θέλω φέρει επ' αυτόν, επί τον Μωάβ, το έτος της επισκέψεως αυτών, λέγει Κύριος.
\par 45 Οι φυγόντες εστάθησαν υπό την σκιάν της Εσεβών ητονημένοι· πυρ όμως θέλει εξέλθει εξ Εσεβών και φλόξ εκ μέσου της Σηών, και θέλει καταφάγει το όριον του Μωάβ και την ακρόπολιν των θορυβούντων πολεμιστών.
\par 46 Ουαί εις σε, Μωάβ· ο λαός του Χεμώς απωλέσθη· διότι οι υιοί σου επιάσθησαν αιχμάλωτοι και αι θυγατέρες σου αιχμάλωτοι.
\par 47 Αλλ' εγώ θέλω επιστρέψει την αιχμαλωσίαν του Μωάβ εν ταις εσχάταις ημέραις, λέγει Κύριος. Μέχρι τούτου η κρίσις του Μωάβ.

\chapter{49}

\par Περί των υιών Αμμών. Ούτω λέγει Κύριος· Μήπως δεν έχει υιούς ο Ισραήλ; δεν έχει κληρονόμον; διά τι ο Μαλχόμ εκληρονόμησε την Γαδ και ο λαός αυτού κατοικεί εν ταις πόλεσιν εκείνου;
\par 2 Διά τούτο, ιδού, έρχονται ημέραι, λέγει Κύριος, και θέλω κάμει να ακουσθή εν Ραββά των υιών Αμμών θόρυβος πολέμου· και θέλει είσθαι σωρός ερειπίων και αι κώμαι αυτής θέλουσι κατακαυθή εν πυρί· τότε ο Ισραήλ θέλει κληρονομήσει τους κληρονομήσαντας αυτόν, λέγει Κύριος.
\par 3 Ολόλυξον, Εσεβών, διότι η Γαί ελεηλατήθη· βοήσατε, αι κώμαι της Ραββά, περιζώσθητε σάκκους· θρηνήσατε και περιδράμετε διά των φραγμών· διότι ο Μαλχόμ θέλει υπάγει εις αιχμαλωσίαν, οι ιερείς αυτού και οι άρχοντες αυτού ομού.
\par 4 Διά τι καυχάσαι εις τας κοιλάδας; η κοιλάς σου διέρρευσε, θυγάτηρ αποστάτρια, ήτις ήλπιζες επί τους θησαυρούς σου, λέγουσα, Τις θέλει ελθεί εναντίον μου;
\par 5 Ιδού, εγώ φέρω φόβον επί σε, λέγει Κύριος ο Θεός των δυνάμεων, από πάντων των περιοίκων σου· και θέλετε διασκορπισθή έκαστος κατά πρόσωπον αυτού· και δεν θέλει υπάρχει ο συνάξων τον πλανώμενον.
\par 6 Και μετά ταύτα θέλω επιστρέψει την αιχμαλωσίαν των υιών Αμμών, λέγει Κύριος.
\par 7 Περί του Εδώμ. Ούτω λέγει ο Κύριος των δυνάμεων· δεν είναι πλέον σοφία εν Θαιμάν; εχάθη η βουλή από των συνετών; έφυγεν η σοφία αυτών;
\par 8 Φύγετε, στραφήτε, κάμετε τόπους βαθείς διά κατοικίαν, κάτοικοι της Δαιδάν· διότι θέλω φέρει επ' αυτόν τον όλεθρον του Ησαύ, τον καιρόν της επισκέψεως αυτού.
\par 9 Εάν ήρχοντο προς σε τρυγηταί, δεν ήθελον αφήσει επιφυλλίδας; εάν κλέπται διά νυκτός, ήθελον αρπάσει το αρκούν εις αυτούς.
\par 10 Αλλ' εγώ εγύμνωσα τον Ησαύ, ανεκάλυψα τους κρυψώνας αυτού, και δεν θέλει δυνηθή να κρυφθή· ελεηλατήθη το σπέρμα αυτού και οι αδελφοί αυτού και οι γείτονες αυτού, και αυτός δεν υπάρχει.
\par 11 Άφες τα ορφανά σου· εγώ θέλω ζωογονήσει αυτά· και αι χήραί σου ας ελπίζωσιν επ' εμέ.
\par 12 Διότι ούτω λέγει Κύριος· Ιδού, εκείνοι εις τους οποίους δεν προσήκε να πίωσιν από του ποτηρίου, τωόντι έπιον· και συ θέλεις μείνει όλως ατιμώρητος; δεν θέλεις μείνει ατιμώρητος αλλ' εξάπαντος θέλεις πίει.
\par 13 Διότι ώμοσα εις εμαυτόν, λέγει Κύριος, ότι η Βοσόρρα θέλει είσθαι εις θάμβος, εις όνειδος, εις ερήμωσιν και εις κατάραν· και πάσαι αι πόλεις αυτής θέλουσιν είσθαι έρημοι εις τον αιώνα.
\par 14 Ήκουσα αγγελίαν παρά Κυρίου και μηνυτής απεστάλη προς τα έθνη, λέγων, Συνάχθητε και έλθετε εναντίον αυτής και σηκώθητε εις πόλεμον.
\par 15 Διότι ιδού, θέλω σε κάμει μικρόν μεταξύ των εθνών, ευκαταφρόνητον μεταξύ των ανθρώπων.
\par 16 Η τρομερότης σου σε ηπάτησε και η υπερηφανία της καρδίας σου, συ ο κατοικών εν τοις κοιλώμασι των κρημνών, ο κατέχων το ύψος των βουνών· και αν υψώσης την φωλεάν σου ως ο αετός, και εκείθεν θέλω σε καταβιβάσει, λέγει Κύριος.
\par 17 Και ο Εδώμ θέλει είσθαι εις θάμβος· πας ο διαβαίνων δι' αυτού θέλει εκθαμβηθή και θέλει συρίξει επί πάσαις ταις πληγαίς αυτού.
\par 18 Καθώς κατεστράφησαν τα Σόδομα και τα Γόμορρα και τα πλησιόχωρα αυτών, λέγει Κύριος, ούτως άνθρωπος δεν θέλει κατοικήσει εκεί ουδέ υιός ανθρώπου θέλει παροικήσει εκεί.
\par 19 Ιδού, θέλει αναβή ως λέων από του φρυάγματος του Ιορδάνου εναντίον της κατοικίας του δυνατού· αλλ' εγώ ταχέως θέλω εκδιώξει τούτον απ' αυτής· και όστις είναι ο εκλεκτός μου, τούτον θέλω καταστήσει επ' αυτήν· διότι τις όμοιός μου; και τις θέλει αντισταθή εις εμέ; και τις ο ποιμήν εκείνος, όστις θέλει σταθή κατά πρόσωπόν μου;
\par 20 Διά τούτο ακούσατε την βουλήν του Κυρίου, την οποίαν εβουλεύθη κατά του Εδώμ, και τους λογισμούς αυτού, τους οποίους ελογίσθη κατά των κατοίκων της Θαιμάν· Εξάπαντος και τα ελάχιστα του ποιμνίου θέλουσι κατασύρει αυτούς· εξάπαντος η κατοικία αυτών θέλει ερημωθή μετ' αυτών.
\par 21 Από του ήχου της αλώσεως αυτών εσείσθη η γή· ο ήχος της φωνής αυτής ηκούσθη εν τη Ερυθρά θαλάσση.
\par 22 Ιδού, θέλει αναβή και πετάξει ως αετός, και θέλει απλώσει τας πτέρυγας αυτού επί Βοσόρραν· και εν τη ημέρα εκείνη η καρδία των ισχυρών του Εδώμ θέλει είσθαι ως καρδία γυναικός κοιλοπονούσης.
\par 23 Περί της Δαμασκού. Κατησχύνθη η Αιμάθ και η Αρφάδ· διότι ήκουσαν κακήν αγγελίαν· ανελύθησαν· ταραχή είναι εν τη θαλάσση· δεν δύναται να ησυχάση.
\par 24 Η Δαμασκός παρελύθη, εστράφη εις φυγήν, και τρόμος κατέλαβεν αυτήν· αγωνία και πόνοι εκυρίευσαν αυτήν ως τικτούσης.
\par 25 Πως δεν εναπελείφθη η πόλις η ευκλεής, η πόλις της ευφροσύνης μου.
\par 26 Διά τούτο οι νέοι αυτής θέλουσι πέσει εν ταις πλατείαις αυτής, και πάντες οι άνδρες οι πολεμισταί θέλουσιν απολεσθή κατ' εκείνην την ημέραν, λέγει ο Κύριος των δυνάμεων.
\par 27 Και θέλω ανάψει πυρ εν τω τείχει της Δαμασκού και θέλει καταφάγει τα παλάτια του Βεν-αδάδ.
\par 28 Περί της Κηδάρ, και περί των βασιλείων της Ασώρ, τα οποία επάταξε Ναβουχοδονόσορ ο βασιλεύς της Βαβυλώνος. Ούτω λέγει Κύριος· Σηκώθητε, ανάβητε προς την Κηδάρ και λεηλατήσατε τους υιούς της ανατολής.
\par 29 Θέλουσι κυριεύσει τας σκηνάς αυτών και τα ποίμνια αυτών· θέλουσι λάβει εις εαυτούς τα παραπετάσματα αυτών και πάσαν την αποσκευήν αυτών και τας καμήλους αυτών· και θέλουσι βοήσει προς αυτούς, Τρόμος πανταχόθεν.
\par 30 Φύγετε, υπάγετε μακράν, κάμετε τόπους βαθείς διά κατοικίαν, κάτοικοι της Ασώρ, λέγει Κύριος· διότι Ναβουχοδονόσορ ο βασιλεύς της Βαβυλώνος εβουλεύθη βουλήν εναντίον σας και ελογίσθη λογισμούς εναντίον σας.
\par 31 Σηκώθητε, ανάβητε εις το ήσυχον έθνος το κατοικούν εν ασφαλεία, λέγει Κύριος· οίτινες δεν έχουσι πύλας ουδέ μοχλούς αλλά κατοικούσι μόνοι·
\par 32 και αι κάμηλοι αυτών θέλουσιν είσθαι λεηλασία και το πλήθος των κτηνών αυτών λάφυρον· και θέλω διασκορπίσει αυτούς εις πάντας τους ανέμους, προς τους κατοικούντας εν τοις απωτάτοις μέρεσι· και θέλω επιφέρει τον όλεθρον αυτών εκ πάντων των περάτων αυτών, λέγει Κύριος.
\par 33 Και η Ασώρ θέλει είσθαι κατοικία θώων, έρημος εις τον αιώνα· δεν θέλει κατοικεί εκεί άνθρωπος και δεν θέλει παροικεί εν αυτή υιός ανθρώπου.
\par 34 Ο λόγος του Κυρίου, ο γενόμενος προς Ιερεμίαν τον προφήτην, κατά της Ελάμ εν τη αρχή της βασιλείας του Σεδεκίου βασιλέως του Ιούδα, λέγων,
\par 35 Ούτω λέγει ο Κύριος των δυνάμεων· Ιδού, θέλω συντρίψει το τόξον της Ελάμ, την αρχήν της δυνάμεως αυτών.
\par 36 Και θέλω φέρει επί την Ελάμ τους τέσσαρας ανέμους εκ των τεσσάρων άκρων του ουρανού, και θέλω διασκορπίσει αυτούς εις πάντας τούτους τους ανέμους· και δεν θέλει είσθαι έθνος, όπου οι δεδιωγμένοι της Ελάμ δεν θέλουσιν ελθεί.
\par 37 Διότι θέλω κατατρομάξει την Ελάμ έμπροσθεν των εχθρών αυτών και έμπροσθεν των ζητούντων την ψυχήν αυτών· και θέλω επιφέρει κακόν επ' αυτούς, τον θυμόν της οργής μου, λέγει Κύριος· και θέλω αποστείλει την μάχαιραν οπίσω αυτών, εωσού αναλώσω αυτούς.
\par 38 Και θέλω στήσει τον θρόνον μου εν Ελάμ, και θέλω εξολοθρεύσει εκείθεν βασιλέα και μεγιστάνας, λέγει Κύριος.
\par 39 Πλην εν ταις εσχάταις ημέραις θέλω επιστρέψει την αιχμαλωσίαν της Ελάμ, λέγει Κύριος.

\chapter{50}

\par Ο λόγος, τον οποίον ελάλησε Κύριος κατά της Βαβυλώνος, κατά της γης των Χαλδαίων, διά Ιερεμίου του προφήτου.
\par 2 Αναγγείλατε εν τοις έθνεσι και κηρύξατε και υψώσατε σημαίαν· κηρύξατε, μη κρύψητε· είπατε, Εκυριεύθη η Βαβυλών, κατησχύνθη ο Βηλ, συνετρίβη ο Μερωδάχ· κατησχύνθησαν τα είδωλα αυτής, συνετρίβησαν τα βδελύγματα αυτής.
\par 3 Διότι από βορρά αναβαίνει έθνος εναντίον αυτής, το οποίον θέλει καταστήσει την γην αυτής έρημον, και δεν θέλει υπάρχει ο κατοικών εν αυτή· από ανθρώπου έως κτήνους θέλουσι μετατοπισθή, θέλουσι φύγει.
\par 4 Εν ταις ημέραις εκείναις και εν τω καιρώ εκείνω, λέγει Κύριος, θέλουσιν ελθεί οι υιοί Ισραήλ, αυτοί και οι υιοί Ιούδα ομού, βαδίζοντες και κλαίοντες· θέλουσιν υπάγει και ζητήσει Κύριον τον Θεόν αυτών.
\par 5 Θέλουσιν ερωτήσει περί της οδού της Σιών με τα πρόσωπα αυτών προς εκεί, λέγοντες, Έλθετε και ας ενωθώμεν μετά του Κυρίου εν διαθήκη αιωνίω, ήτις δεν θέλει λησμονηθή.
\par 6 Ο λαός μου έγεινε πρόβατα απολωλότα· οι ποιμένες αυτών παρέτρεψαν αυτούς, περιεπλάνησαν αυτούς εις τα όρη· υπήγαν από όρους εις βουνόν, ελησμόνησαν την μάνδραν αυτών.
\par 7 Πάντες οι ευρίσκοντες αυτούς κατέτρωγον αυτούς, και οι εχθροί αυτών είπον, Δεν πταίομεν, διότι ημάρτησαν εις Κύριον, την κατοικίαν της δικαιοσύνης· ναι, εις Κύριον, την ελπίδα των πατέρων αυτών.
\par 8 Φύγετε εκ μέσου της Βαβυλώνος και εξέλθετε εκ της γης των Χαλδαίων και γείνετε ως κριοί έμπροσθεν ποιμνίων.
\par 9 Διότι ιδού, εγώ θέλω εγείρει και αναβιβάσει επί Βαβυλώνα συναγωγήν εθνών μεγάλων εκ γης βορρά, και θέλουσι παραταχθή εναντίον αυτής· εκείθεν θέλει αλωθή· τα βέλη αυτών θέλουσιν είσθαι ως εμπείρου ισχυρού· δεν θέλουσιν επιστρέψει κενά.
\par 10 Και η Χαλδαία θέλει είσθαι λάφυρον· πάντες οι λεηλατούντες αυτήν θέλουσι χορτασθή, λέγει Κύριος.
\par 11 Επειδή ηυφραίνεσθε και εκαυχάθε, φθορείς της κληρονομίας μου, επειδή εσκιρτάτε ως δάμαλις επί χλόης και εχρεμετίζετε ως ρωμαλέοι ίπποι,
\par 12 η μήτηρ σας κατησχύνθη σφόδρα· η γεννήτριά σας ενετράπη· ιδού, αυτή θέλει είσθαι η εσχάτη των εθνών, έρημος, γη ξηρά και άβατος.
\par 13 Εξ αιτίας της οργής του Κυρίου δεν θέλει κατοικηθή, αλλά θέλει ερημωθή άπασα· πας ο διαβαίνων διά της Βαβυλώνος θέλει εκθαμβηθή και συρίξει επί πάσαις ταις πληγαίς αυτής.
\par 14 Παρατάχθητε εναντίον της Βαβυλώνος κύκλω· πάντες οι εντείνοντες τόξον, τοξεύσατε κατ' αυτής, μη φείδεσθε βελών· διότι ημάρτησεν εις Κύριον.
\par 15 Αλαλάξατε επ' αυτή κύκλω· παρέδωκεν εαυτήν· έπεσαν τα θεμέλια αυτής, κατηδαφίσθησαν τα τείχη αυτής· διότι τούτο είναι η εκδίκησις του Κυρίου· εκδικήθητε αυτήν· καθώς αυτή έκαμε, κάμετε εις αυτήν.
\par 16 Εκκόψατε από Βαβυλώνος τον σπείροντα και τον κρατούντα δρέπανον εν καιρώ θερισμού· από προσώπου της εξολοθρευτικής μαχαίρας θέλουσιν επιστρέψει έκαστος εις τον λαόν αυτού, και θέλουσι φύγει έκαστος εις την γην αυτού.
\par 17 Ο Ισραήλ είναι πρόβατον πλανώμενον· λέοντες εκυνήγησαν αυτό· πρώτος ο βασιλεύς της Ασσυρίας κατέφαγεν αυτόν· και ύστερον ούτος ο Ναβουχοδονόσορ, ο βασιλεύς της Βαβυλώνος, κατεσύντριψε τα οστά αυτού.
\par 18 Διά τούτο ούτω λέγει ο Κύριος των δυνάμεων, ο Θεός του Ισραήλ· Ιδού, εγώ θέλω τιμωρήσει τον βασιλέα της Βαβυλώνος και την γην αυτού, καθώς ετιμώρησα τον βασιλέα της Ασσυρίας.
\par 19 Και θέλω αποκαταστήσει τον Ισραήλ εν τη κατοικία αυτού, και θέλει βόσκεσθαι τον Κάρμηλον και την Βασάν, και η ψυχή αυτού θέλει χορτασθή επί το όρος Εφραΐμ και Γαλαάδ.
\par 20 Εν ταις ημέραις εκείναις και εν τω καιρώ εκείνω, λέγει Κύριος, η ανομία του Ισραήλ θέλει ζητηθή και δεν θέλει υπάρχει, και αι αμαρτίαι του Ιούδα και δεν θέλουσιν ευρεθή· διότι θέλω συγχωρήσει όσους αφήσω υπόλοιπον.
\par 21 Ανάβα επί την γην των καταδυναστών, επ' αυτήν και επί τους κατοίκους της Φεκώδ· αφάνισον και εξολόθρευσον κατόπιν αυτών, λέγει Κύριος, και κάμε κατά πάντα όσα προσέταξα εις σε.
\par 22 Φωνή πολέμου εν τη γη και σύντριμμα μέγα.
\par 23 Πως συνεθλάσθη και συνετρίβη η σφύρα πάσης της γής· πως έγεινεν η Βαβυλών εις θάμβος μεταξύ των εθνών.
\par 24 Έστησα παγίδα εις σε, μάλιστα και επιάσθης, Βαβυλών, και συ δεν εγνώρισας· ευρέθης μάλιστα και συνελήφθης, διότι εις τον Κύριον αντεστάθης.
\par 25 Ο Κύριος ήνοιξε την οπλοθήκην αυτού και εξήγαγε το όπλα της οργής αυτού· διότι το έργον τούτο έχει Κύριος ο Θεός των δυνάμεων εν τη γη των Χαλδαίων.
\par 26 Έλθετε επ' αυτήν από των περάτων· ανοίξατε τας αποθήκας αυτής· καταστήσατε αυτήν ως σωρούς και εξολοθρεύσατε αυτήν· ας μη μείνη εξ αυτής υπόλοιπον.
\par 27 Σφάξατε πάντας τους μόσχους αυτής· ας καταβώσιν εις σφαγήν· ουαί εις αυτούς· διότι ήλθεν η ημέρα αυτών, ο καιρός της επισκέψεως αυτών.
\par 28 Φωνή φευγόντων και διασωζομένων από της γης Βαβυλώνος, διά να αναγγείλη εν Σιών την εκδίκησιν Κυρίου του Θεού ημών, την εκδίκησιν του ναού αυτού.
\par 29 Συγκαλέσατε τους τοξότας επί Βαβυλώνα· πάντες οι εντείνοντες τόξον, στρατοπεδεύσατε κατ' αυτής κύκλω· μηδείς εξ αυτής ας μη διασωθή· ανταπόδοτε εις αυτήν κατά το έργον αυτής· κατά πάντα όσα έκαμε, κάμετε εις αυτήν· διότι υπερηφανεύθη κατά του Κυρίου, κατά του Αγίου του Ισραήλ.
\par 30 Διά τούτο οι νέοι αυτής θέλουσι πέσει εν ταις πλατείαις αυτής, και πάντες οι άνδρες αυτής οι πολεμισταί θέλουσιν απολεσθή κατ' εκείνην την ημέραν, λέγει Κύριος.
\par 31 Ιδού, εγώ είμαι εναντίον σου, ω επηρμένη, λέγει Κύριος ο Θεός των δυνάμεων· διότι ήλθεν η ημέρα σου, ο καιρός της επισκέψεώς σου.
\par 32 Και ο επηρμένος θέλει προσκόψει και πέσει, και δεν θέλει υπάρχει ο αναστήσων αυτόν· και θέλω ανάψει πυρ εν ταις πόλεσιν αυτού και ο θέλει καταφάγει πάντα τα πέριξ αυτού.
\par 33 Ούτω λέγει ο Κύριος των δυνάμεων· οι υιοί Ισραήλ και οι υιοί Ιούδα κατεδυναστεύθησαν ομού· και πάντες οι αιχμαλωτίσαντες αυτούς κατεκράτησαν αυτούς· ηρνήθησαν να απολύσωσιν αυτούς.
\par 34 Πλην ο Λυτρωτής αυτών είναι Ισχυρός· Κύριος των δυνάμεων το όνομα αυτού· θέλει εξάπαντος διαδικάσει την δίκην αυτών, διά να αναπαύση την γην και να ταράξη τους κατοίκους της Βαβυλώνος.
\par 35 Μάχαιρα επί τους Χαλδαίους, λέγει Κύριος, και επί τους κατοίκους της Βαβυλώνος και επί τους μεγιστάνας αυτής και επί τους σοφούς αυτής.
\par 36 Μάχαιρα επί τους ψευδοπροφήτας και θέλουσι παραφρονήσει· μάχαιρα επί τους ισχυρούς αυτής και θέλουσι τρομάξει.
\par 37 Μάχαιρα επί τους ίππους αυτών και επί τας αμάξας αυτών και επί πάντα τον σύμμικτον λαόν τον εν μέσω αυτής, και θέλουσιν είσθαι ως γυναίκες· μάχαιρα επί τους θησαυρούς αυτής και θέλουσι διαρπαχθή.
\par 38 Ξηρασία επί τα ύδατα αυτής, και θέλουσι ξηρανθή· διότι είναι γη των γλυπτών και εμωράνθησαν εν τοις ειδώλοις αυτών.
\par 39 Διά τούτο θηρία και αίλουροι θέλουσι κατοικήσει εκεί και στρουθοκάμηλοι θέλουσι κατοικήσει εν αυτή και δεν θέλει κατοικηθή πλέον εις τον αιώνα· και ουδείς θέλει κατασκηνώσει εν αυτή εις γενεάν και γενεάν.
\par 40 Καθώς κατέστρεψεν ο Θεός τα Σόδομα και τα Γόμορρα και τα πλησιόχωρα αυτών, λέγει Κύριος, ούτως άνθρωπος δεν θέλει κατοικήσει εκεί ουδέ υιός ανθρώπου θέλει παροικήσει εν αυτή.
\par 41 Ιδού, λαός θέλει ελθεί από βορρά και έθνος μέγα, και βασιλείς πολλοί θέλουσιν εγερθή από των εσχάτων της γης.
\par 42 Τόξον και λόγχην θέλουσι κρατεί· είναι σκληροί και ανίλεοι· η φωνή αυτών ηχεί ως θάλασσα, και επιβαίνουσιν επί ίππους, παρατεταγμένοι ως άνδρες εις πόλεμον, εναντίον σου, θυγάτηρ Βαβυλώνος.
\par 43 Ήκουσεν ο βαιλεύς της Βαβυλώνος την φήμην αυτών και αι χείρες αυτού παρελύθησαν· στενοχωρία συνέλαβεν αυτόν, ωδίνες ως τικτούσης.
\par 44 Ιδού, θέλει αναβή ως λέων από του φρυάγματος του Ιορδάνου εναντίον της κατοικίας του δυνατού· αλλ' εγώ ταχέως θέλω εκδιώξει αυτούς απ' αυτής· και όστις είναι ο εκλεκτός σου, τούτον θέλω καταστήσει επ' αυτήν· διότι τις όμοιός μου; και τις θέλει αντισταθή εις εμέ; και τις είναι ο ποιμήν εκείνος, όστις θέλει σταθή κατά πρόσωπόν μου;
\par 45 Διά τούτο ακούσατε την βουλήν του Κυρίου, την οποίαν εβουλεύθη κατά της Βαβυλώνος, και τους λογισμούς αυτού, τους οποίους ελογίσθη κατά της γης των Χαλδαίων· εξάπαντος και τα ελάχιστα του ποιμνίου θέλουσι κατασύρει αυτούς· εξάπαντος η κατοικία αυτών θέλει ερημωθή μετ' αυτών.
\par 46 Από του ήχου της αλώσεως της Βαβυλώνος εσείσθη η γη, και η κραυγή ηκούσθη εν τοις έθνεσι.

\chapter{51}

\par Ούτω λέγει Κύριος· Ιδού, εγώ εγείρω άνεμον φθοροποιόν επί την Βαβυλώνα και επί τους κατοίκους αυτής τους υψώσαντας την καρδίαν αυτών κατ' εμού.
\par 2 Και θέλω εξαποστείλει επί την Βαβυλώνα λικμητάς και θέλουσιν εκλικμήσει αυτήν και εκκενώσει την γην αυτής· διότι εν τη ημέρα της συμφοράς κυκλόθεν θέλουσιν είσθαι εναντίον αυτής.
\par 3 Τοξότης επί τοξότην ας εντείνη το τόξον αυτού και επί τον πεποιθότα επί τον θώρακα αυτού· και μη φείδεσθε τους νέους αυτής· εξολοθρεύσατε άπαν το στράτευμα αυτής.
\par 4 Και οι τραυματίαι θέλουσι πέσει εν τη γη των Χαλδαίων και οι κατακεκεντημένοι εν ταις οδοίς αυτής.
\par 5 Διότι ο Ισραήλ δεν εγκατελείφθη ουδέ ο Ιούδας παρά του Θεού αυτού, παρά του Κυρίου των δυνάμεων, αν και η γη αυτών ενεπλήσθη ανομίας εναντίον του Αγίου του Ισραήλ.
\par 6 Φύγετε εκ μέσου της Βαβυλώνος και διασώσατε έκαστος την ψυχήν αυτού· μη απολεσθήτε εν τη ανομία αυτής· διότι είναι καιρός εκδικήσεως του Κυρίου· ανταπόδομα αυτός ανταποδίδει εις αυτήν.
\par 7 Η Βαβυλών εστάθη ποτήριον χρυσούν εν τη χειρί του Κυρίου, μεθύον πάσαν την γήν· από του οίνου αυτής έπιον τα έθνη· διά τούτο τα έθνη παρεφρόνησαν.
\par 8 Έπεσεν εξαίφνης η Βαβυλών και συνετρίβη· ολολύζετε δι' αυτήν· λάβετε βάλσαμον διά τον πόνον αυτής, ίσως ιατρευθή.
\par 9 Μετεχειρίσθημεν ιατρικά διά την Βαβυλώνα αλλά δεν ιατρεύθη· εγκαταλείψατε αυτήν, και ας απέλθωμεν έκαστος εις την γην αυτού· διότι η κρίσις αυτής έφθασεν εις τον ουρανόν και υψώθη έως του στερεώματος.
\par 10 Ο Κύριος εφανέρωσε την δικαιοσύνην ημών· έλθετε και ας διηγηθώμεν εν Σιών το έργον Κυρίου του Θεού ημών.
\par 11 Στιλβώσατε τα βέλη· πυκνώσατε τας ασπίδας· ο Κύριος ήγειρε το πνεύμα των βασιλέων των Μήδων· διότι ο σκοπός αυτού είναι εναντίον της Βαβυλώνος, διά να εξολοθρεύση αυτήν· επειδή η εκδίκησις του Κυρίου είναι εκδίκησις του ναού αυτού.
\par 12 Υψώσατε σημαίαν επί τα τείχη της Βαβυλώνος, ενδυναμώσατε την φρουράν, στήσατε φυλακάς, ετοιμάσατε ενέδρας· διότι ο Κύριος και εβουλεύθη και θέλει εκτελέσει εκείνο, το οποίον ελάλησεν εναντίον των κατοίκων της Βαβυλώνος.
\par 13 Ω η κατοικούσα επί υδάτων πολλών, η πλήρης θησαυρών, ήλθε το τέλος σου, το πέρας της πλεονεξίας σου.
\par 14 Ο Κύριος των δυνάμεων ώμοσεν εις εαυτόν, λέγων, Εξάπαντος θέλω σε γεμίσει από ανθρώπων ως από ακρίδων, και θέλουσιν εκπέμψει αλαλαγμόν εναντίον σου.
\par 15 Αυτός εποίησε την γην εν τη δυνάμει αυτού, εστερέωσε την οικουμένην εν τη σοφία αυτού και εξέτεινε τους ουρανούς εν τη συνέσει αυτού.
\par 16 Όταν εκπέμπη την φωνήν αυτού, συνίσταται πλήθος υδάτων εν ουρανοίς, και ανάγει νεφέλας από των άκρων της γής· κάμνει αστραπάς διά βροχήν και εξάγει άνεμον εκ των θησαυρών αυτού.
\par 17 Πας άνθρωπος εμωράνθη υπό της γνώσεως αυτού· πας χωνευτής κατησχύνθη από των γλυπτών· διότι ψεύδος είναι το χωνευτόν αυτού και πνοή εν αυτώ δεν υπάρχει.
\par 18 Ματαιότης ταύτα, έργον πλάνης· εν καιρώ επισκέψεως αυτών θέλουσιν απολεσθή.
\par 19 Η μερίς του Ιακώβ δεν είναι ως αυτά, διότι αυτός είναι ο πλάσας τα πάντα· και ο Ισραήλ είναι η ράβδος της κληρονομίας αυτού· Κύριος των δυνάμεων το όνομα αυτού.
\par 20 Συ ήσο πέλεκύς μου, όπλα πολέμου, και διά σου συνέτριψα έθνη και διά σου εξωλόθρευσα βασίλεια·
\par 21 και διά σου συνέτριψα ίππον και αναβάτην αυτού, και διά σου συνέτριψα άμαξαν και αναβάτην αυτής·
\par 22 και διά σου συνέτριψα άνδρα και γυναίκα, και διά σου συνέτριψα γέροντα και νέον, και διά σου συνέτριψα νεανίσκον και παρθένον·
\par 23 και διά σου συνέτριψα ποιμένα και ποίμνιον αυτού, και διά σου συνέτριψα γεωργόν και ζεύγος αυτού· και διά σου συνέτριψα στρατηγούς και άρχοντας.
\par 24 Και θέλω ανταποδώσει επί την Βαβυλώνα και επί πάντας τους κατοίκους της Χαλδαίας πάσαν την κακίαν αυτών, την οποίαν έπραξαν εν Σιών ενώπιόν σας, λέγει Κύριος.
\par 25 Ιδού, εγώ είμαι εναντίον σου, όρος φθοροποιόν, λέγει Κύριος, το οποίον φθείρεις πάσαν την γήν· και θέλω εκτείνει την χείρα μου επί σε και θέλω σε κατακυλίσει από των βράχων και σε καταστήσει όρος πυρίκαυστον.
\par 26 Και δεν θέλουσι λάβει από σου λίθον διά γωνίαν ουδέ λίθον διά θεμέλια, αλλά θέλεις είσθαι ερήμωσις αιωνία, λέγει Κύριος.
\par 27 Υψώσατε σημαίαν επί την γην, σαλπίσατε σάλπιγγα εν τοις έθνεσιν, ετοιμάσατε έθνη κατ' αυτής, παραγγείλατε κατ' αυτής, εις τα βασίλεια του Αραράτ, του Μιννί και του Ασχενάζ· καταστήσατε επ' αυτήν αρχηγούς· αναβιβάσατε ίππους ως ακρίδας ορθότριχας.
\par 28 Ετοιμάσατε κατ' αυτής έθνη, τους βασιλείς των Μήδων, τους στρατηγούς αυτής και πάντας τους άρχοντας αυτής και πάσαν την γην της επικρατείας αυτής.
\par 29 Και η γη θέλει σεισθή και στενάξει· διότι η βουλή του Κυρίου θέλει εκτελεσθή κατά της Βαβυλώνος, διά να καταστήση την γην της Βαβυλώνος έρημον, άνευ κατοίκου.
\par 30 Οι ισχυροί της Βαβυλώνος εξέλιπον από του να πολεμώσιν, έμειναν εν τοις οχυρώμασιν· η δύναμις αυτών εξέλιπεν· έγειναν ως γυναίκες· έκαυσαν τας κατοικίας αυτής· οι μοχλοί αυτής συνετρίβησαν.
\par 31 Ταχυδρόμος θέλει δράμει εις απάντησιν ταχυδρόμου και μηνυτής εις απάντησιν μηνυτού, διά να αναγγείλωσι προς τον βασιλέα της Βαβυλώνος ότι η πόλις αυτού ηλώθη από των άκρων αυτής,
\par 32 και ότι αι διαβάσεις επιάσθησαν, και κατέκαυσαν εν πυρί τους καλαμώνας, και οι άνδρες του πολέμου κατετρόμαξαν.
\par 33 Διότι ούτω λέγει ο Κύριος των δυνάμεων, ο Θεός του Ισραήλ. Η θυγάτηρ της Βαβυλώνος είναι ως αλώνιον, καιρός είναι να καταπατηθή· έτι ολίγον και θέλει ελθεί ο καιρός του θερισμού αυτής.
\par 34 Ναβουχοδονόσορ ο βασιλεύς της Βαβυλώνος με κατέφαγε, με συνέτριψε, με κατέστησεν αγγείον άχρηστον, με κατέπιεν ως δράκων, ενέπλησε την κοιλίαν αυτού από των τρυφερών μου, με έξωσεν.
\par 35 Η εις εμέ και εις την σάρκα μου αδικία ας έλθη επί την Βαβυλώνα, θέλει ειπεί η κατοικούσα την Σιών· και το αίμα μου επί τους κατοίκους της Χαλδαίας, θέλει ειπεί η Ιερουσαλήμ.
\par 36 Διά τούτο ούτω λέγει Κύριος· Ιδού, εγώ θέλω διαδικάσει την δίκην σου και θέλω εκδικήσει την εκδίκησίν σου, και θέλω καταστήσει την θάλασσαν αυτής ξηράν και θέλω ξηράνει την πηγήν αυτής.
\par 37 Και η Βαβυλών θέλει είσθαι εις σωρούς, κατοικητήριον θώων, θάμβος και συριγμός, άνευ κατοίκου.
\par 38 Θέλουσι βρυχάσθαι ομού ως λέοντες, θέλουσιν ωρύεσθαι ως σκύμνοι λεόντων.
\par 39 Θέλω κάμει αυτούς να θερμανθώσιν εν τοις συμποσίοις αυτών, και θέλω μεθύσει αυτούς, διά να ευθυμήσωσι και να υπνώσωσιν ύπνον αιώνιον και να μη εξυπνήσωσι, λέγει Κύριος.
\par 40 Θέλω καταβιβάσει αυτούς ως αρνία εις σφαγήν, ως κριούς μετά τράγων.
\par 41 Πως ηλώθη η Σησάχ και εθηρεύθη το καύχημα πάσης της γής· πως έγεινεν Βαβυλών θάμβος εν τοις έθνεσιν.
\par 42 Η θάλασσα ανέβη επί την Βαβυλώνα· κατεκαλύφθη υπό του πλήθους των κυμάτων αυτής.
\par 43 Αι πόλεις αυτής κατεστάθησαν θάμβος, γη άνυδρος και άβατος γη, εν ή δεν κατοικεί ουδείς άνθρωπος ουδέ υιός ανθρώπου διέρχεται δι' αυτής.
\par 44 Και θέλω τιμωρήσει τον Βηλ εν Βαβυλώνι και εκβάλει εκ του στόματος αυτού όσα κατέπιε· και τα έθνη δεν θέλουσι συναχθή πλέον προς αυτόν, και αυτό το τείχος της Βαβυλώνος θέλει πέσει.
\par 45 Λαέ μου, εξέλθετε εκ μέσου αυτής και σώσατε έκαστος την ψυχήν αυτού από της οργής του θυμού του Κυρίου·
\par 46 μήποτε η καρδία σας χαυνωθή και φοβηθήτε υπό της αγγελίας ήτις θέλει ακουσθή εν τη γή· θέλει δε ελθεί αγγελία το εν έτος, και μετά τούτο η αγγελία το άλλο έτος, και καταδυναστεία εν τη γη, εξουσιαστής εναντίον εξουσιαστού.
\par 47 Διά τούτο, ιδού, έρχονται ημέραι και θέλω κάμει εκδίκησιν επί τα γλυπτά της Βαβυλώνος· και πάσα η γη αυτής θέλει καταισχυνθή και πάντες οι τετραυματισμένοι αυτής θέλουσι πέσει εν τω μέσω αυτής.
\par 48 Τότε οι ουρανοί και η γη και πάντα τα εν αυτοίς θέλουσιν αλαλάξει επί Βαβυλώνα· διότι οι εξολοθρευταί θέλουσιν ελθεί επ' αυτήν από βορρά, λέγει Κύριος.
\par 49 Καθώς η Βαβυλών έκαμε τους τετραυματισμένους του Ισραήλ να πέσωσιν, ούτω και εν Βαβυλώνι θέλουσι πέσει οι τετραυματισμένοι πάσης της γης.
\par 50 Οι εκφυγόντες την μάχαιραν, υπάγετε, μη στέκεσθε· ενθυμήθητε τον Κύριον μακρόθεν, και Ιερουσαλήμ ας αναβή επί την καρδίαν σας.
\par 51 Κατησχύνθημεν, διότι ηκούσαμεν ονειδισμόν· αισχύνη κατεκάλυψε το πρόσωπον ημών· διότι εισήλθον ξένοι εις το αγιαστήριον του οίκου του Κυρίου.
\par 52 Διά τούτο, ιδού, έρχονται ημέραι, λέγει Κύριος, και θέλω κάμει εκδίκησιν επί τα γλυπτά αυτής· και καθ' όλην την γην αυτής οι τετραυματισμένοι θέλουσιν οιμώζει.
\par 53 Και αν η Βαβυλών αναβή έως του ουρανού και αν οχυρώση το ύψος της δυνάμεως αυτής, θέλουσιν ελθεί παρ' εμού εξολοθρευταί επ' αυτήν, λέγει Κύριος.
\par 54 Φωνή κραυγής έρχεται από Βαβυλώνος και συντριμμός μέγας εκ γης Χαλδαίων,
\par 55 διότι ο Κύριος εξωλόθρευσε την Βαβυλώνα και ηφάνισεν εξ αυτής την μεγάλην φωνήν· τα δε κύματα εκείνων ηχούσιν· ο θόρυβος της φωνής αυτών εξακούεται ως υδάτων πολλών·
\par 56 διότι ο εξολοθρευτής ήλθεν επ' αυτήν, επί την Βαβυλώνα, και οι δυνατοί αυτής συνελήφθησαν, τα τόξα αυτών συνετρίβησαν· επειδή Κύριος ο Θεός των ανταποδόσεων θέλει εξάπαντος κάμει ανταπόδοσιν.
\par 57 Και θέλω μεθύσει τους ηγεμόνας αυτής και τους σοφούς αυτής, τους στρατηγούς αυτής και τους άρχοντας αυτής και τους δυνατούς αυτής· και θέλουσι κοιμηθή ύπνον αιώνιον και δεν θέλουσιν εξυπνήσει, λέγει ο Βασιλεύς, του οποίου το όνομα είναι ο Κύριος των δυνάμεων.
\par 58 Ούτω λέγει ο Κύριος των δυνάμεων· τα πλατέα τείχη της Βαβυλώνος θέλουσιν ολοτελώς κατασκαφή, και αι πύλαι αυτής αι υψηλαί θέλουσι κατακαή εν πυρί, και όσα οι λαοί εκοπίασαν θέλουσιν είσθαι εις μάτην, και όσα τα έθνη εμόχθησαν θέλουσιν είσθαι διά το πυρ.
\par 59 Ο λόγος, τον οποίον Ιερεμίας ο προφήτης προσέταξεν εις τον Σεραΐαν τον υιόν του Νηρίου υιού του Μαασίου, ότε επορεύετο μετά του Σεδεκίου βασιλέως του Ιούδα εις Βαβυλώνα, κατά το τέταρτον έτος της βασιλείας αυτού· ήτο δε ο Σεραΐας κοιτωνάρχης.
\par 60 Έγραψε δε ο Ιερεμίας εν ενί βιβλίω πάντα τα κακά, τα οποία έμελλον να έλθωσιν επί την Βαβυλώνα, πάντας τους λόγους τούτους τους γεγραμμένους κατά της Βαβυλώνος.
\par 61 Και είπεν ο Ιερεμίας προς τον Σεραΐαν, Όταν έλθης εις την Βαβυλώνα και ίδης και αναγνώσης πάντας τους λόγους τούτους,
\par 62 τότε θέλεις ειπεί, Κύριε, συ ελάλησας κατά του τόπου τούτου, διά να εξολοθρεύσης αυτόν, ώστε να μη υπάρχη ο κατοικών εν αυτώ, από ανθρώπου έως κτήνους, αλλά να ήναι ερήμωσις αιωνία.
\par 63 Και αφού τελειώσης αναγινώσκων το βιβλίον τούτο, θέλεις δέσει επ' αυτό λίθον και ρίψει αυτό εις το μέσον του Ευφράτου·
\par 64 και θέλεις ειπεί, Ούτω θέλει βυθισθή η Βαβυλών, και δεν θέλει σηκωθή εκ των κακών, τα οποία εγώ θέλω φέρει επ' αυτήν· και οι Βαβυλώνιοι θέλουσιν εξασθενήσει. Μέχρι τούτου είναι οι λόγοι του Ιερεμίου.

\chapter{52}

\par Ενός και είκοσι ετών ηλικίας ήτο ο Σεδεκίας ότε εβασίλευσε, και εβασίλευσεν ένδεκα έτη εν Ιερουσαλήμ· το δε όνομα της μητρός αυτού ήτο Αμουτάλ, θυγάτηρ του Ιερεμίου από Λιβνά.
\par 2 Και έπραξε πονηρά ενώπιον του Κυρίου, κατά πάντα όσα έπραξεν ο Ιωαχείμ.
\par 3 Διότι από του θυμού του Κυρίου του γενομένου κατά Ιερουσαλήμ και Ιούδα, εωσού απέρριψεν αυτούς από προσώπου αυτού, ο Σεδεκίας απεστάτησε κατά του βασιλέως της Βαβυλώνος.
\par 4 Και εν τω ενάτω έτει της βασιλείας αυτού, τον δέκατον μήνα, την δεκάτην του μηνός, ήλθε Ναβουχοδονόσορ ο βασιλεύς της Βαβυλώνος, αυτός και άπαν το στράτευμα αυτού, επί την Ιερουσαλήμ, και εστρατοπέδευσαν εναντίον αυτής και ωκοδόμησαν περιτείχισμα κατ' αυτής κύκλω.
\par 5 Και η πόλις επολιορκείτο μέχρι του ενδεκάτου έτους του βασιλέως Σεδεκίου.
\par 6 Εν τω τετάρτω μηνί, την ενάτην του μηνός, η πείνα εκραταιώθη εν τη πόλει και δεν υπήρχεν άρτος διά τον λαόν του τόπου.
\par 7 Και εξεπορθήθη η πόλις και πάντες οι άνδρες του πολέμου έφυγον και εξήλθον εκ της πόλεως την νύκτα, διά της οδού της πύλης της μεταξύ των δύο τειχών, της πλησίον του βασιλικού κήπου· οι δε Χαλδαίοι ήσαν πλησίον της πόλεως κύκλω, και υπήγον κατά την οδόν της πεδιάδος.
\par 8 Το δε στράτευμα των Χαλδαίων κατεδίωξεν οπίσω του βασιλέως και έφθασαν τον Σεδεκίαν εις τας πεδιάδας της Ιεριχώ· και άπαν το στράτευμα αυτού διεσκορπίσθη από πλησίον αυτού.
\par 9 Και αυνέλαβον τον βασιλέα και ανήγαγον αυτόν προς τον βασιλέα της Βαβυλώνος εις Ριβλά εν τη γη Αιμάθ και επρόφερε καταδίκην επ' αυτόν.
\par 10 Και έσφαξεν ο βασιλεύς της Βαβυλώνος τους υιούς του Σεδεκίου έμπροσθεν των οφθαλμών αυτού· έσφαξεν ότι και πάντας τους άρχοντας Ιούδα εν Ριβλά.
\par 11 Και τους οφθαλμούς του Σεδεκίου εξετύφλωσε, και έδεσεν αυτόν με δύο χαλκίνας αλύσεις· και έφερεν αυτόν ο βασιλεύς της Βαβυλώνος εις Βαβυλώνα και έβαλεν αυτόν εις οίκον φυλακής έως της ημέρας του θανάτου αυτού.
\par 12 Εν δε τω πέμπτω μηνί, τη δεκάτη του μηνός, του δεκάτου εννάτου έτους του Ναβουχοδονόσορ βασιλέως της Βαβυλώνος ήλθεν επί Ιερουσαλήμ Νεβουζαραδάν ο αρχισωματοφύλαξ, ο παριστάμενος ενώπιον του βασιλέως της Βαβυλώνος,
\par 13 και κατέκαυσε τον οίκον του Κυρίου και τον οίκον του βασιλέως, και πάντας τους οίκους της Ιερουσαλήμ και πάντα μέγαν οίκον κατέκαυσεν εν πυρί.
\par 14 Και άπαν το στράτευμα των Χαλδαίων, το μετά του αρχισωματοφύλακος, κατεκρήμνισαν πάντα τα τείχη της Ιερουσαλήμ κύκλω.
\par 15 Και εκ των πτωχών του λαού και το υπόλοιπον του λαού το εναπολειφθέν εν τη πόλει και τους φυγόντας, οίτινες προσέφυγον προς τον βασιλέα της Βαβυλώνος, και το εναπολειφθέν του πλήθους, μετώκισε Νεβουζαραδάν ο αρχισωματοφύλαξ.
\par 16 Εκ των πτωχών όμως της γης αφήκε Νεβουζαραδάν ο αρχισωματοφύλαξ διά αμπελουργούς και διά γεωργούς.
\par 17 Και τους στύλους τους χαλκίνους τους εν τω οίκω του Κυρίου και τας βάσεις και την χαλκίνην θάλασσαν την εν τω οίκω του Κυρίου κατέκοψαν οι Χαλδαίοι, και μετεκόμισαν όλον τον χαλκόν αυτών εις την Βαβυλώνα.
\par 18 Έλαβον δε και τους λέβητας και τα πτυάρια και τα λυχνοψάλιδα και τας λεκάνας και τα θυμιατήρια και πάντα τα σκεύη τα χάλκινα, διά των οποίων έκαμνον την υπηρεσίαν.
\par 19 Έλαβε προσέτι ο αρχισωματοφύλαξ και τους κρατήρας και τα πυροδοχεία, και τας λεκάνας και τους λέβητας και τας λυχνίας και τα θυμιατήρια και τας φιάλας, όσα ήσαν χρυσά και όσα αργυρά·
\par 20 τους δύο στύλους, την μίαν θάλασσαν και τους δώδεκα χαλκίνους μόσχους τους αντί βάσεων, τα οποία έκαμεν ο βασιλεύς Σολομών διά τον οίκον του Κυρίου· ο χαλκός πάντων τούτων των σκευών ήτο αζύγιστος.
\par 21 Περί δε των στύλων, το ύψος του ενός στύλου ήτο δεκαοκτώ πηχών, και ζώνη πηχών δώδεκα περιεκύκλονεν αυτόν, και το πάχος αυτού δακτύλων τεσσάρων· ήτο κενός.
\par 22 Και το κιονόκρανον το επ' αυτού χάλκινον· το δε ύψος του ενός κιονοκράνου πέντε πηχών και το δικτυωτόν και τα ρόδια επί του κιονοκράνου κύκλω, τα πάντα χάλκινα· τα αυτά είχε και ο δεύτερος στύλος μετά των ροδίων.
\par 23 Και ήσαν ενενήκοντα εξ ρόδια κρεμάμενα· πάντα τα ρόδια τα επί του δικτυωτού ήσαν εκατόν κύκλω.
\par 24 Και έλαβεν ο αρχισωματοφύλαξ Σεραΐαν τον πρώτον ιερέα και Σοφονίαν τον δεύτερον ιερέα και τους τρεις θυρωρούς·
\par 25 και εκ της πόλεως έλαβεν ένα ευνούχον, όστις ήτο επιστάτης επί των ανδρών των πολεμιστών, και επτά άνδρας εκ των παρισταμένων έμπροσθεν του βασιλέως, τους ευρεθέντας εν τη πόλει, και τον γραμματέα τον άρχοντα των στρατευμάτων, όστις έκαμνε την στρατολογίαν του λαού της γης, και εξήκοντα άνδρας εκ του λαού της γης, τους ευρεθέντας εν μέσω της πόλεως.
\par 26 Και λαβών αυτούς Νεβουζαραδάν ο αρχισωματοφύλαξ έφερεν αυτούς προς τον βασιλέα της Βαβυλώνος εις Ριβλά.
\par 27 Και επάταξεν αυτούς ο βασιλεύς της Βαβυλώνος και εθανάτωσεν αυτούς εν Ριβλά, εν τη γη Αιμάθ. Ούτω μετωκίσθη ο Ιούδας από της γης αυτού.
\par 28 Ούτος είναι ο λαός, τον οποίον μετώκισεν ο Ναβουχοδονόσορ, εν τω εβδόμω έτει, τρεις χιλιάδας και εικοσιτρείς Ιουδαίους·
\par 29 εν τω δεκάτω ογδόω έτει του Ναβουχοδονόσορ μετώκισεν αυτός από Ιερουσαλήμ οκτακοσίας τριάκοντα δύο ψυχάς·
\par 30 εν τω εικοστώ τρίτω έτει του Ναβουχοδονόσορ μετώκισε Νεβουζαραδάν ο αρχισωματοφύλαξ εκ των Ιουδαίων επτακοσίας τεσσαράκοντα πέντε ψυχάς· πάσαι αι ψυχαί τέσσαρες χιλιάδες και εξακόσιαι.
\par 31 Εν δε τω τριακοστώ εβδόμω έτει της μετοικεσίας του Ιωακείμ βασιλέως του Ιούδα, τον δωδέκατον μήνα, την εικοστήν πέμπτην του μηνός, Ευείλ-μερωδάχ ο βασιλεύς της Βαβυλώνος, κατά το έτος καθ' ο εβασίλευσεν, ανύψωσε την κεφαλήν του Ιωακείμ βασιλέως του Ιούδα και εξήγαγεν αυτόν εκ του οίκου της φυλακής,
\par 32 και ελάλησεν ευμενώς μετ' αυτού και έθεσε τον θρόνον αυτού επάνωθεν του θρόνου των βασιλέων των μετ' αυτού εν Βαβυλώνι.
\par 33 Και ήλλαξε τα ιμάτια της φυλακής αυτού· και έτρωγεν άρτον πάντοτε μετ' αυτού πάσας τας ημέρας της ζωής αυτού.
\par 34 Και το σιτηρέσιον αυτού ήτο παντοτεινόν σιτηρέσιον διδόμενον εις αυτόν παρά του βασιλέως της Βαβυλώνος, ημερήσιος χορηγία μέχρι της ημέρας του θανάτου αυτού, πάσας τας ημέρας της ζωής αυτού.


\end{document}