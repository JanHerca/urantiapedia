\begin{document}

\title{Psalms}


\chapter{1}

\par Μακάριος ο άνθρωπος, όστις δεν περιεπάτησεν εν βουλή ασεβών, και εν οδώ αμαρτωλών δεν εστάθη, και επί καθέδρας χλευαστών δεν εκάθησεν·
\par 2 αλλ' εν τω νόμω του Κυρίου είναι το θέλημα αυτού, και εν τω νόμω αυτού μελετά ημέραν και νύκτα.
\par 3 Και θέλει είσθαι ως δένδρον πεφυτευμένον παρά τους ρύακας των υδάτων, το οποίον δίδει τον καρπόν αυτού εν τω καιρώ αυτού, και το φύλλον αυτού δεν μαραίνεται· και πάντα, όσα αν πράττη, θέλουσιν ευοδωθή.
\par 4 Δεν θέλουσιν είσθαι ούτως οι ασεβείς· αλλ' ως το λεπτόν άχυρον, το οποίον εκρίπτει ο άνεμος.
\par 5 Διά τούτο δεν θέλουσιν εγερθή οι ασεβείς εν τη κρίσει, ουδέ οι αμαρτωλοί εν τη συνάξει των δικαίων.
\par 6 Διότι γνωρίζει ο Κύριος την οδόν των δικαίων· η δε οδός των ασεβών θέλει απολεσθή.

\chapter{2}

\par Διά τι εφρύαξαν τα έθνη και οι λαοί εμελέτησαν μάταια;
\par 2 Παρεστάθησαν οι βασιλείς της γης, και οι άρχοντες συνήχθησαν ομού, κατά του Κυρίου, και κατά του χριστού αυτού, λέγοντες,
\par 3 Ας διασπάσωμεν τους δεσμούς αυτών, και ας απορρίψωμεν αφ' ημών τας αλύσεις αυτών.
\par 4 Ο καθήμενος εν ουρανοίς θέλει γελάσει· ο Κύριος θέλει εκμυκτηρίσει αυτούς.
\par 5 Τότε θέλει λαλήσει προς αυτούς εν τη οργή αυτού, και εν τω θυμώ αυτού θέλει συνταράξει αυτούς.
\par 6 Αλλ' εγώ, θέλει ειπεί, έχρισα τον Βασιλέα μου επί Σιών, το όρος το άγιόν μου.
\par 7 Εγώ θέλω αναγγείλει το πρόσταγμα· ο Κύριος είπε προς εμέ, Υιός μου είσαι σύ· εγώ σήμερον σε εγέννησα·
\par 8 Ζήτησον παρ' εμού, και θέλω σοι δώσει τα έθνη κληρονομίαν σου, και ιδιοκτησίαν σου τα πέρατα της γής·
\par 9 θέλεις ποιμάνει αυτούς εν ράβδω σιδηρά· ως σκεύος κεραμέως θέλεις συντρίψει αυτούς.
\par 10 Τώρα λοιπόν, βασιλείς, συνετίσθητε· διδάχθητε, κριταί της γης.
\par 11 Δουλεύετε τον Κύριον εν φόβω και αγάλλεσθε εν τρόμω.
\par 12 Φιλείτε τον Υιόν, μήποτε οργισθή, και απολεσθήτε εκ της οδού, όταν εξαφθή ταχέως ο θυμός αυτού. Μακάριοι πάντες οι πεποιθότες επ' αυτόν.

\chapter{3}

\par «Ψαλμός του Δαβίδ, ότε έφυγεν απ' έμπροσθεν του υιού αυτού Αβεσσαλώμ.» Κύριε, πόσον επληθύνθησαν οι εχθροί μου πολλοί επανίστανται επ' εμέ·
\par 2 πολλοί λέγουσι περί της ψυχής μου, δεν είναι δι' αυτόν σωτηρία εν τω Θεώ· Διάψαλμα.
\par 3 Αλλά συ, Κύριε, είσαι η ασπίς μου, η δόξα μου και ο υψόνων την κεφαλήν μου.
\par 4 Έκραξα με την φωνήν μου προς τον Κύριον, και εισήκουσέ μου εκ του όρους του αγίου αυτού. Διάψαλμα.
\par 5 Εγώ επλαγίασα και εκοιμήθην· εξηγέρθην· διότι ο Κύριος με υποστηρίζει.
\par 6 Δεν θέλω φοβηθή από μυριάδων λαού των αντιπαρατασσομένων κατ' εμού κύκλω.
\par 7 Ανάστηθι, Κύριε· σώσον με, Θεέ μου· διότι συ επάταξας πάντας τους εχθρούς μου κατά της σιαγόνος· συνέτριψας τους οδόντας των ασεβών.
\par 8 Του Κυρίου είναι η σωτηρία· επί τον λαόν σου είναι η ευλογία σου. Διάψαλμα.

\chapter{4}

\par «Εις τον πρώτον μουσικόν, επί Νεγινώθ. Ψαλμός του Δαβίδ.» Όταν επικαλώμαι, εισάκουέ μου Θεέ της δικαιοσύνης μου· εν στενοχωρία με επλάτυνας· ελέησόν με και εισάκουσον της προσευχής μου.
\par 2 Υιοί ανθρώπων, έως πότε μετατρέπετε την δόξαν μου εις καταισχύνην, αγαπάτε ματαιότητα και ζητείτε ψεύδος; Διάψαλμα.
\par 3 Αλλά μάθετε ότι εξέλεξεν ο Κύριος τον όσιον αυτού· ο Κύριος θέλει ακούσει, όταν κράζω προς αυτόν.
\par 4 Οργίζεσθε και μη αμαρτάνετε· λαλείτε εν ταις καρδίαις υμών επί της κλίνης υμών και ησυχάζετε. Διάψαλμα.
\par 5 Θυσιάσατε θυσίας δικαιοσύνης και ελπίσατε επί τον Κύριον.
\par 6 Πολλοί λέγουσι, Τις θέλει δείξει εις ημάς το αγαθόν; Ύψωσον εφ' ημάς το φως του προσώπου σου, Κύριε.
\par 7 Έδωκας μεγαλητέραν ευφροσύνην εις την καρδίαν μου, παρ' όσην απολαμβάνουσιν αυτοί, όταν πληθύνηται ο σίτος αυτών και ο οίνος αυτών.
\par 8 Εν ειρήνη θέλω και πλαγιάσει και κοιμηθή· διότι συ μόνος, Κύριε, με κατοικίζεις εν ασφαλεία.

\chapter{5}

\par «Εις τον πρώτον μουσικόν, επί Νεγιλώθ. Ψαλμός του Δαβίδ.» Ακροάσθητι, Κύριε, τους λόγους μου· νόησον τον στεναγμόν μου.
\par 2 Πρόσεξον εις την φωνήν της κραυγής μου, Βασιλεύ μου και Θεέ μου· Διότι εις σε θέλω προσευχηθή.
\par 3 Κύριε, το πρωΐ θέλεις ακούσει την φωνήν μου· το πρωΐ θέλω παρασταθή εις σε και θέλω προσδοκά.
\par 4 Διότι δεν είσαι συ Θεός θέλων την ασέβειαν· ο πονηρευόμενος δεν θέλει κατοικεί πλησίον σου.
\par 5 Ουδέ θέλουσι σταθή οι άφρονες έμπροσθεν των οφθαλμών σου· μισείς πάντας τους εργάτας της ανομίας.
\par 6 Θέλεις εξολοθρεύσει τους λαλούντας το ψεύδος· ο Κύριος βδελύττεται τον άνθρωπον τον αιμοβόρον και τον δόλιον.
\par 7 Αλλ' εγώ διά του πλήθους του ελέους σου θέλω εισέλθει εις τον οίκόν σου· θέλω προσκυνήσει προς τον ναόν της αγιότητός σου μετά φόβου σου.
\par 8 Κύριε, οδήγησόν με εν τη δικαιοσύνη σου, ένεκα των εχθρών μου· κατεύθυνον την οδόν σου έμπροσθέν μου.
\par 9 Διότι δεν είναι εν τω στόματι αυτών αλήθεια· η καρδία αυτών είναι πονηρία· τάφος ανεωγμένος ο λάρυγξ αυτών· διά της γλώσσης αυτών κολακεύουσι.
\par 10 Καταδίκασον αυτούς, Θεέ· ας αποτύχωσι των διαβουλίων αυτών· έξωσον αυτούς διά το πλήθος των παραβάσεων αυτών, διότι απεστάτησαν εναντίον σου.
\par 11 Ας ευφραίνωνται δε πάντες οι ελπίζοντες επί σέ· ας χαίρωσι διαπαντός, διότι συ περισκεπάζεις αυτούς· ας καυχώνται ομοίως επί σε οι αγαπώντες το όνομά σου.
\par 12 Διότι συ, Κύριε, θέλεις ευλογήσει τον δίκαιον· θέλεις περισκεπάσει αυτόν με ευμένειαν, ως με ασπίδα.

\chapter{6}

\par «Εις τον πρώτον μουσικόν, επί Νεγινώθ, επί Σεμινίθ. Ψαλμός του Δαβίδ.» Κύριε, μη με ελέγξης εν τω θυμώ σου, μηδέ εν τη οργή σου παιδεύσης με.
\par 2 Ελέησόν με, Κύριε, διότι είμαι αδύνατος· ιάτρευσόν με, Κύριε, διότι εταράχθησαν τα οστά μου.
\par 3 Και η ψυχή μου εταράχθη σφόδρα· αλλά συ, Κύριε, έως πότε;
\par 4 Επίστρεψον, Κύριε· λύτρωσον την ψυχήν μου· σώσον με διά το έλεός σου.
\par 5 Διότι εν τω θανάτω δεν υπάρχει ενθύμησις περί σού· εν τω άδη τις θέλει σε δοξολογήσει;
\par 6 Απέκαμον εν τω στεναγμώ μου· όλην την νύκτα λούω την κλίνην μου· με τα δάκρυά μου καταβρέχω την στρωμνήν μου.
\par 7 Ο οφθαλμός μου εμαράνθη εκ της θλίψεως· εγήρασεν εξ αιτίας πάντων των εχθρών μου.
\par 8 Απομακρύνθητε απ' εμού, πάντες οι εργάται της ανομίας, διότι ήκουσεν ο Κύριος την φωνήν του κλαυθμού μου.
\par 9 Ήκουσεν ο Κύριος την δέησίν μου· ο Κύριος εδέχθη την προσευχήν μου.
\par 10 Ας αισχυνθώσι και ας ταραχθώσι σφόδρα πάντες οι εχθροί μου· ας στραφώσιν εις τα οπίσω· ας καταισχυνθώσιν αιφνιδίως.

\chapter{7}

\par «Σιγαϊών του Δαβίδ, το οποίον έψαλλεν εις τον Κύριον, διά τους λόγους Χούς του Βενιαμίτου.» Κύριε ο Θεός μου, επί σε ελπίζω· σώσον με εκ πάντων των διωκόντων με και ελευθέρωσόν με·
\par 2 μήποτε ο εχθρός αρπάση ως λέων την ψυχήν μου και διασπαράξη, χωρίς να υπάρξη ελευθερωτής.
\par 3 Κύριε ο Θεός μου, εάν εγώ έπραξα τούτο, εάν εις τας χείρας μου ήναι ανομία·
\par 4 εάν ανταπέδωκα κακόν εις τον ειρηνεύοντα μετ' εμού, ή κατέθλιψα τον αναιτίως διώκοντά με·
\par 5 ας καταδιώξη ο εχθρός την ψυχήν μου και ας φθάση αυτήν· και ας καταπατήση εις γην την ζωήν μου, και ας καταβάλη την δόξαν μου εις το χώμα. Διάψαλμα.
\par 6 Ανάστηθι, Κύριε, εν τη οργή σου· υψώθητι ένεκα της λύσσης των εχθρών μου· και εγέρθητι δι' εμέ εις την κρίσιν την οποίαν προσέταξας.
\par 7 Και η σύναξις των λαών θέλει σε κυκλώσει· και συ επίστρεψον, να καθήσης υπεράνωθεν αυτής εις ύψος.
\par 8 Ο Κύριος θέλει κρίνει τους λαούς. Κρίνόν με, Κύριε, κατά την δικαιοσύνην μου, και κατά την ακεραιότητά μου, την εν εμοί.
\par 9 Ας τελειώση πλέον η κακία των ασεβών· και στερέωσον τον δίκαιον, συ ο Θεός ο δίκαιος, ο εξετάζων καρδίας και νεφρούς.
\par 10 Η ασπίς μου είναι εν τω Θεώ, όστις σώζει τους ευθείς την καρδίαν.
\par 11 Ο Θεός είναι κριτής δίκαιος και Θεός οργιζόμενος καθ' εκάστην ημέραν.
\par 12 Εάν ο ασεβής δεν επιστραφή, θέλει ακονίσει την ρομφαίαν αυτού· ενέτεινε το τόξον αυτού και ητοίμασεν αυτό·
\par 13 και δι' αυτόν ητοίμασεν όργανα θανάτου· προσήρμοσε τα βέλη αυτού εναντίον των διωκτών.
\par 14 Ιδού, ο ασεβής κοιλοπονεί ανομίαν· συνέλαβε δε πονηρίαν και εγέννησε ψεύδος·
\par 15 Έκσαψε λάκκον και εβάθυνεν αυτόν· πλην αυτός θέλει πέσει εις τον βόθρον, τον οποίον έκαμεν.
\par 16 Η πονηρία αυτού θέλει επιστρέψει κατά της κεφαλής αυτού, και η καταδυναστεία αυτού θέλει καταβή επί την κορυφήν αυτού.
\par 17 Εγώ θέλω επαινεί τον Κύριον κατά την δικαιοσύνην αυτού, και θέλω ψαλμωδεί εις το όνομα Κυρίου του Υψίστου.

\chapter{8}

\par «Εις τον πρώτον μουσικόν, επί Γιττίθ. Ψαλμός του Δαβίδ.» Κύριε ο Κύριος ημών, πόσον είναι θαυμαστόν το όνομά σου εν πάση τη γή· όστις έθεσας την δόξαν σου υπεράνω των ουρανών.
\par 2 Εκ στόματος νηπίων και θηλαζόντων ητοίμασας αίνεσιν ένεκα των εχθρών σου, διά να καταργήσης τον εχθρόν και τον εκδικητήν.
\par 3 Όταν θεωρώ τους ουρανούς σου, το έργον των δακτύλων σου, την σελήνην και τους αστέρας, τα οποία συ εθεμελίωσας,
\par 4 Τι είναι ο άνθρωπος, ώστε να ενθυμήσαι αυτόν; ή ο υιός του ανθρώπου, ώστε να επισκέπτησαι αυτόν;
\par 5 Συ δε έκαμες αυτόν ολίγον τι κατώτερον των αγγέλων, και με δόξαν και τιμήν εστεφάνωσας αυτόν.
\par 6 Κατέστησας αυτόν κύριον επί τα έργα των χειρών σου· πάντα υπέταξας υποκάτω των ποδών αυτού·
\par 7 πάντα τα πρόβατα και τους βόας, έτι δε και τα ζώα του αγρού·
\par 8 τα πετεινά του ουρανού, και τους ιχθύας της θαλάσσης, πάντα τα διαπορευόμενα τας οδούς των θαλασσών.
\par 9 Κύριε ο Κύριος ημών, πόσον είναι θαυμαστόν το όνομά σου εν πάση τη γη.

\chapter{9}

\par «Εις τον πρώτον μουσικόν, επί Μούθ-λαββέν. Ψαλμός του Δαβίδ.» Θέλω σε δοξολογήσει, Κύριε, εν όλη καρδία μου· θέλω διηγηθή πάντα τα θαυμάσιά σου.
\par 2 Θέλω ευφρανθή και χαρή εν σοί· θέλω ψαλμωδήσει εις το όνομά σου, Ύψιστε.
\par 3 Όταν στραφώσιν οι εχθροί μου εις τα οπίσω, πέσωσι και αφανισθώσιν απ' έμπροσθέν σου.
\par 4 Διότι συ έκαμες την κρίσιν μου και την δίκην μου· εκάθησας επί θρόνου κρίνων εν δικαιοσύνη·
\par 5 Επετίμησας τα έθνη· εξωλόθρευσας τον ασεβή· το όνομα αυτών εξήλειψας εις τον αιώνα του αιώνος·
\par 6 Εχθρέ, αι ερημώσεις εξέλιπον διαπαντός· και κατηδάφισας πόλεις· το μνημόσυνον αυτών εχάθη μετ' αυτών.
\par 7 Αλλ' ο Κύριος διαμένει εις τον αιώνα· ητοίμασε τον θρόνον αυτού διά κρίσιν.
\par 8 Και αυτός θέλει κρίνει την οικουμένην εν δικαιοσύνη· θέλει κρίνει τους λαούς εν ευθύτητι.
\par 9 Και ο Κύριος θέλει είσθαι καταφύγιον εις τον πένητα, καταφύγιον εν καιρώ θλίψεως.
\par 10 Και θέλουσιν ελπίσει επί σε οι γνωρίζοντες το όνομά σου· διότι δεν εγκατέλιπες τους εκζητούντάς σε, Κύριε.
\par 11 Ψαλμωδείτε εις τον Κύριον, τον κατοικούντα εν Σιών· αναγγείλατε μεταξύ των λαών τα κατορθώματα αυτού·
\par 12 διότι όταν κάμνη εκζήτησιν αιμάτων, ενθυμείται αυτούς· δεν λησμονεί την κραυγήν των ταλαιπωρουμένων.
\par 13 Ελέησόν με, Κύριε· ιδέ την θλίψιν μου την εκ των εχθρών μου, συ ο υψόνων με εκ των πυλών του θανάτου,
\par 14 διά να διηγηθώ πάσας τας αινέσεις σου εν ταις πύλαις της θυγατρός Σιών· εγώ θέλω αγαλλιάσθαι διά την σωτηρίαν σου.
\par 15 Τα έθνη κατεβυθίσθησαν εις τον λάκκον, τον οποίον έκαμον· εν τη παγίδι, την οποίαν έκρυψαν, επιάσθη ο πους αυτών.
\par 16 Ο Κύριος γνωρίζεται διά την κρίσιν, την οποίαν κάμνει· ο ασεβής παγιδεύεται εν τω έργω των χειρών αυτού· Ιγαϊών· Διάψαλμα.
\par 17 Οι ασεβείς θέλουσιν επιστραφή εις τον άδην· πάντα τα έθνη τα λησμονούντα τον Θεόν.
\par 18 Διότι δεν θέλει λησμονηθή διαπαντός ο πτωχός· η προσδοκία των πενήτων δεν θέλει απολεσθή διαπαντός.
\par 19 Ανάστηθι, Κύριε· ας μη υπερισχύη άνθρωπος· ας κριθώσι τα έθνη ενώπιόν σου.
\par 20 Κατάστησον, Κύριε, νομοθέτην επ' αυτούς· ας γνωρίσωσι τα έθνη, ότι είναι άνθρωποι. Διάψαλμα.

\chapter{10}

\par Διά τι, Κύριε, ίστασαι μακρόθεν; κρύπτεσαι εν καιρώ θλίψεως;
\par 2 Εν τη υπερηφανία του ασεβούς κατακαίεται ο πτωχός· ας πιασθώσιν εν ταις πανουργίαις, τας οποίας διαλογίζονται.
\par 3 Διότι ο ασεβής καυχάται εις τας επιθυμίας της ψυχής αυτού, και ο πλεονέκτης μακαρίζει εαυτόν· καταφρονεί τον Κύριον.
\par 4 Ο ασεβής διά την αλαζονείαν του προσώπου αυτού δεν θέλει εκζητήσει τον Κύριον· πάντες οι διαλογισμοί αυτού είναι ότι δεν υπάρχει Θεός.
\par 5 Αι οδοί αυτού μολύνονται εν παντί καιρώ· αι κρίσεις σου είναι πολύ υψηλά από προσώπου αυτού· φυσά εναντίον πάντων των εχθρών αυτού.
\par 6 Είπεν εν τη καρδία αυτού, δεν θέλω σαλευθή από γενεάς εις γενεάν· διότι δεν θέλω πέσει εις δυστυχίαν.
\par 7 Το στόμα αυτού γέμει κατάρας και απάτης και δόλου· υπό την γλώσσαν αυτού είναι κακία και ανομία.
\par 8 Κάθηται εν ενέδρα των προαυλίων, εν αποκρύφοις, διά να φονεύση τον αθώον· οι οφθαλμοί αυτού παραμονεύουσι τον πένητα.
\par 9 Ενεδρεύει εν αποκρύφω, ως λέων εν τω σπηλαίω αυτού· ενεδρεύει διά να αρπάση τον πτωχόν· αρπάζει τον πτωχόν, όταν σύρη αυτόν εν τη παγίδι αυτού.
\par 10 Κύπτει, χαμηλόνει, διά να πέσωσιν οι πτωχοί εις τους όνυχας αυτού.
\par 11 Είπεν εν τη καρδία αυτού, ελησμόνησεν ο Θεός· έκρυψε το πρόσωπόν αυτού· δεν θέλει ιδεί ποτέ.
\par 12 Ανάστηθι, Κύριε· Θεέ, ύψωσον την χείρα σου· μη λησμονήσης τους τεθλιμμένους.
\par 13 Διά τι παρώξυνεν ο ασεβής τον Θεόν; είπεν εν τη καρδία αυτού, Δεν θέλεις εξετάσει.
\par 14 Είδες· διότι συ παρατηρείς την αδικίαν και την ύβριν, διά να ανταποδώσης με την χείρα σου· εις σε αφιερόνεται ο πτωχός· εις τον ορφανόν συ είσαι ο βοηθός.
\par 15 Σύντριψον τον βραχίονα του ασεβούς και πονηρού· εξερεύνησον την ασέβειαν αυτού, εωσού μη εύρης αυτήν πλέον.
\par 16 Ο Κύριος είναι βασιλεύς εις τον αιώνα του αιώνος· τα έθνη θέλουσιν εξαλειφθή εκ της γης αυτού.
\par 17 Κύριε, εισήκουσας την επιθυμίαν των πενήτων· θέλεις στηρίξει την καρδίαν αυτών, θέλεις κάμει προσεκτικόν το ωτίον σου·
\par 18 διά να κρίνης τον ορφανόν και τον τεταπεινωμένον, ώστε ο άνθρωπος ο γήϊνος να μη καταδυναστεύη πλέον.

\chapter{11}

\par «Εις τον πρώτον μουσικόν. Ψαλμός του Δαβίδ.» Επί τον Κύριον πέποιθα· πως λέγετε εις την ψυχήν μου, Φεύγε εις το όρος σας, ως πτηνόν;
\par 2 Διότι, ιδού, οι ασεβείς ενέτειναν τόξον· ητοίμασαν τα βέλη αυτών επί την χορδήν, διά να τοξεύσωσιν εν σκότει τους ευθείς την καρδίαν.
\par 3 Εάν τα θεμέλια καταστραφώσιν, ο δίκαιος τι δύναται να κάμη;
\par 4 Ο Κύριος είναι εν τω ναώ τω αγίω αυτού· ο Κύριος εν τω ουρανώ έχει τον θρόνον αυτού· οι οφθαλμοί αυτού βλέπουσι, τα βλέφαρα αυτού εξετάζουσι, τους υιούς των ανθρώπων.
\par 5 Ο Κύριος εξετάζει τον δίκαιον· τον δε ασεβή και τον αγαπώντα την αδικίαν μισεί η ψυχή αυτού.
\par 6 Θέλει βρέξει επί τους ασεβείς παγίδας· πυρ και θείον και ανεμοζάλη είναι η μερίς του ποτηρίου αυτών.
\par 7 Διότι δίκαιος ων ο Κύριος, αγαπά δικαιοσύνην· το πρόσωπον αυτού βλέπει ευθύτητα.

\chapter{12}

\par «Εις τον πρώτον μουσικόν, επί Σεμινίθ. Ψαλμός του Δαβίδ.» Σώσον, Κύριε· διότι εξέλιπεν όσιος, διότι εχάθησαν οι φιλαλήθεις μεταξύ των υιών των ανθρώπων.
\par 2 Έκαστος λαλεί ματαιότητα προς τον πλησίον αυτού· με χείλη δόλια λαλούσιν από διπλής καρδίας.
\par 3 Ας εξολοθρεύση ο Κύριος πάντα τα χείλη τα δόλια, την γλώσσαν την μεγαλορρήμονα.
\par 4 Διότι είπον, Θέλομεν υπερισχύσει διά της γλώσσης ημών· τα χείλη ημών είναι ημέτερα· τις θέλει είσθαι κύριος εφ' ημάς;
\par 5 Διά την ταλαιπωρίαν των πτωχών, διά τον στεναγμόν των πενήτων, τώρα θέλω εγερθή, λέγει ο Κύριος· θέλω θέσει εν ασφαλεία εκείνον, κατά του οποίου φυσά ο ασεβής.
\par 6 Τα λόγια του Κυρίου είναι λόγια καθαρά· αργύριον δεδοκιμασμένον εν πηλίνω χωνευτηρίω, κεκαθαρισμένον επταπλασίως.
\par 7 Συ, Κύριε, θέλεις φυλάξει αυτούς· θέλεις διατηρήσει αυτούς από της γενεάς ταύτης εις τον αιώνα.
\par 8 Οι ασεβείς περιπατούσι κύκλω, όταν οι αχρείοι υψωθώσι μεταξύ των υιών των ανθρώπων.

\chapter{13}

\par «Εις τον πρώτον μουσικόν. Ψαλμός του Δαβίδ.» Έως πότε, Κύριε, θέλεις με λησμονεί διαπαντός; έως πότε θέλεις κρύπτει το πρόσωπόν σου απ' εμού;
\par 2 Έως πότε θέλω έχει βουλάς εν τη ψυχή μου, οδύνας καθ' ημέραν εν τη καρδία μου· έως πότε θέλει υψόνεσθαι ο εχθρός μου επ' εμέ;
\par 3 Επίβλεψον· εισάκουσόν μου, Κύριε ο Θεός μου· φώτισον τους οφθαλμούς μου, μήποτε υπνώσω τον ύπνον του θανάτου·
\par 4 Μήποτε είπη ο εχθρός μου, Υπερίσχυσα κατ' αυτού, και οι θλίβοντές με αγαλλιασθώσιν, εάν σαλευθώ.
\par 5 Αλλ' εγώ ήλπισα επί το έλεός σου· η καρδία μου θέλει αγάλλεσθαι εις την σωτηρίαν σου.
\par 6 Θέλω ψάλλει εις τον Κύριον, διότι με αντήμειψε.

\chapter{14}

\par «Εις τον πρώτον μουσικόν. Ψαλμός του Δαβίδ.» Είπεν ο άφρων εν τη καρδία αυτού, δεν υπάρχει Θεός. Διεφθάρησαν· έγειναν βδελυροί εις τα έργα· δεν υπάρχει πράττων αγαθόν.
\par 2 Ο Κύριος διέκυψεν εξ ουρανού επί τους υιούς των ανθρώπων διά να ίδη εάν ήναι τις έχων σύνεσιν, εκζητών τον Θεόν.
\par 3 Πάντες εξέκλιναν, ομού εξηχρειώθησαν· δεν υπάρχει πράττων αγαθόν· δεν υπάρχει ουδέ εις.
\par 4 Δεν έχουσι γνώσιν πάντες οι εργαζόμενοι την ανομίαν, οι κατατρώγοντες τον λαόν μου ως βρώσιν άρτου; τον Κύριον δεν επεκαλέσθησαν.
\par 5 Εκεί εφοβήθησαν φόβον· διότι ο Θεός είναι εν τη γενεά των δικαίων.
\par 6 Κατησχύνατε την βουλήν του πτωχού, διότι ο Κύριος είναι η καταφυγή αυτού.
\par 7 Τις θέλει δώσει εκ Σιών την σωτηρίαν του Ισραήλ; όταν ο Κύριος επιστρέψη τον λαόν αυτού από της αιχμαλωσίας, θέλει αγάλλεσθαι ο Ιακώβ, θέλει ευφραίνεσθαι ο Ισραήλ.

\chapter{15}

\par «Ψαλμός του Δαβίδ.» Κύριε, τις θέλει κατοικήσει εν τη σκηνή σου; τις θέλει κατοικήσει εν τω όρει τω αγίω σου;
\par 2 Ο περιπατών εν ακεραιότητι και εργαζόμενος δικαιοσύνην, και λαλών αλήθειαν εν τη καρδία αυτού·
\par 3 Ο μη καταλαλών διά της γλώσσης αυτού, μηδέ πράττων κακόν εις τον φίλον αυτού, μηδέ δεχόμενος ονειδισμόν κατά του πλησίον αυτού.
\par 4 Εις τους οφθαλμούς αυτού καταφρονείται ο αχρείος· τιμά δε τους φοβουμένους τον Κύριον· ομνύει εις τον πλησίον αυτού και δεν αθετεί·
\par 5 δεν δίδει το αργύριον αυτού επί τόκω, ουδέ λαμβάνει δώρα κατά του αθώου. Ο πράττων ταύτα δεν θέλει σαλευθή εις τον αιώνα.

\chapter{16}

\par «Μικτάμ του Δαβίδ.» Φύλαξόν με, Θεέ, διότι επί σε ήλπισα.
\par 2 Συ ψυχή μου, είπας προς τον Κύριον, Συ είσαι ο Κύριός μου· η αγαθότης μου δεν εκτείνεται εις σέ·
\par 3 Αλλ' εις τους αγίους τους όντας εν τη γη και εις τους εξαιρέτους, εις τους οποίους είναι όλη μου η ευχαρίστησις.
\par 4 Οι πόνοι των τρεχόντων κατόπιν άλλων θεών θέλουσι πολλαπλασιασθή· εγώ δεν θέλω προσφέρει τας εξ αίματος σπονδάς αυτών, ουδέ θέλω λάβει εις τα χείλη μου τα ονόματα αυτών.
\par 5 Ο Κύριος είναι η μερίς της κληρονομίας μου και του ποτηρίου μου· συ διαφυλάττεις τον κλήρόν μου.
\par 6 Αι μερίδες μου έπεσον εις τόπους τερπνούς· έλαβον ώραιοτάτην κληρονομίαν.
\par 7 Θέλω ευλογεί τον Κύριον τον νουθετήσαντά με· έτι και εν καιρώ νυκτός με διδάσκουσιν οι νεφροί μου.
\par 8 Ενώπιόν μου είχον τον Κύριον διαπαντός· διότι είναι εκ δεξιών μου, διά να μη σαλευθώ.
\par 9 Διά τούτο ευφράνθη η καρδία μου και ηγαλλίασεν η γλώσσα μου· έτι δε και η σαρξ μου θέλει αναπαυθή επ' ελπίδι.
\par 10 Διότι δεν θέλεις εγκαταλείψει την ψυχήν μου εν τω άδη, ουδέ θέλεις αφήσει τον Οσιόν σου να ίδη διαφθοράν.
\par 11 Εφανέρωσας εις εμέ την οδόν της ζωής· χορτασμός ευφροσύνης είναι το πρόσωπόν σου· τερπνότητες είναι διαπαντός εν τη δεξιά σου.

\chapter{17}

\par «Προσευχή του Δαβίδ.» Άκουσον, Κύριε, το δίκαιον· πρόσεξον εις την δέησίν μου· ακροάσθητι την προσευχήν μου, την γινομένην ουχί με χείλη δόλια.
\par 2 Ας εξέλθη η κρίσις μου παρά του προσώπου σου· οι οφθαλμοί σου ας ίδωσι την ευθύτητα.
\par 3 Ηρεύνησας την καρδίαν μου· επεσκέφθης αυτήν εν καιρώ νυκτός· εδοκίμασάς με και δεν ηύρες ουδέν εν εμοί· ο στοχασμός μου δεν είναι διάφορος των λόγων μου.
\par 4 Ως προς τα έργα των ανθρώπων, εγώ διά των λόγων των χειλέων σου εφυλάχθην από των οδών των παρανόμων.
\par 5 Στήριξον τα διαβήματά μου εν ταις οδοίς σου, διά να μη σαλευθώσιν οι πόδες μου.
\par 6 Εγώ σε επεκαλέσθην, Θεέ, διότι θέλεις μου εισακούσει· Κλίνον εις εμέ το ωτίον σου, άκουσον τους λόγους μου.
\par 7 Θαυμάστωσον τα ελέη σου, συ ο σώζων τους ελπίζοντας επί σε εκ των επανισταμένων κατά της δεξιάς σου.
\par 8 Φύλαξόν με ως κόρην οφθαλμού· κρύψον με υπό την σκιάν των πτερύγων σου
\par 9 απ' έμπροσθεν των ασεβών των ταλαιπωρούντων με· οι εχθροί της ψυχής μου με περιεκύκλωσαν.
\par 10 Υπερεπάχυναν· το στόμα αυτών λαλεί υπερήφανα.
\par 11 Τώρα περιεκύκλωσαν τα διαβήματα ημών· προσήλωσαν τους οφθαλμούς αυτών διά να κρημνίσωσιν ημάς κατά γής·
\par 12 Ως λέων επιθυμών να κατασπαράξη· και ως σκύμνος, καθήμενος εν αποκρύφοις.
\par 13 Ανάστηθι, Κύριε· πρόφθασον αυτόν, υποσκέλισον αυτόν· ελευθέρωσον την ψυχήν μου από του ασεβούς, όστις είναι η ρομφαία σου·
\par 14 Από ανθρώπων, Κύριε, της χειρός σου· από ανθρώπων του κόσμου, οίτινες λαμβάνουσι την μερίδα αυτών εν ταύτη τη ζωή, και των οποίων την κοιλίαν γεμίζεις από των θησαυρών σου· εχόρτασαν τους υιούς, και αφίνουσι τα υπόλοιπα αυτών εις τους εγγόνους αυτών.
\par 15 Εγώ δε εν δικαιοσύνη θέλω ιδεί το πρόσωπόν σου· θέλω χορτασθή από της θεωρίας σου, όταν εξεγερθώ.

\chapter{18}

\par «Εις τον πρώτον μουσικόν. Ψαλμός του Δαβίδ δούλου του Κυρίου, όστις ελάλησε προς τον Κύριον τους λόγους της ωδής ταύτης, καθ' ην ημέραν ηλευθέρωσεν αυτόν ο Κύριος εκ της χειρός πάντων των εχθρών αυτού και εκ της χειρός του Σαούλ· και είπε,» Θέλω σε αγαπά, Κύριε, η ισχύς μου.
\par 2 Ο Κύριος είναι πέτρα μου και φρούριόν μου και ελευθερωτής μου· Θεός μου, βράχος μου· επ' αυτόν θέλω ελπίζει· η ασπίς μου και το κέρας της σωτηρίας μου· υψηλός πύργος μου.
\par 3 Θέλω επικαλεσθή τον αξιΰμνητον Κύριον, και εκ των εχθρών μου θέλω σωθή.
\par 4 Πόνοι θανάτου με περιεκύκλωσαν, και χείμαρροι ανομίας με κατετρόμαξαν·
\par 5 Πόνοι του άδου με περιεκύκλωσαν, παγίδες θανάτου με έφθασαν.
\par 6 Εν τη στενοχωρία μου επεκαλέσθην τον Κύριον, και προς τον Θεόν μου εβόησα. Ήκουσεν εκ του ναού αυτού της φωνής μου, και η κραυγή μου ήλθεν ενώπιον αυτού εις τα ώτα αυτού.
\par 7 Τότε εσαλεύθη και έντρομος έγεινεν η γη, και τα θεμέλια των ορέων εταράχθησαν και εσαλεύθησαν, διότι ωργίσθη.
\par 8 Καπνός ανέβαινεν εκ των μυκτήρων αυτού, και πυρ κατατρώγον εκ του στόματος αυτού· άνθρακες ανήφθησαν απ' αυτού.
\par 9 Και έκλινε τους ουρανούς και κατέβη, και γνόφος υπό τους πόδας αυτού.
\par 10 Και επέβη επί χερουβείμ και επετάσθη· και επέταξεν επί πτερύγων ανέμων.
\par 11 Έθεσε το σκότος απόκρυφον τόπον αυτού· η σκηνή αυτού, πέριξ αυτού ήσαν ύδατα σκοτεινά, νέφη πυκνά των αέρων.
\par 12 Εκ της λάμψεως της έμπροσθεν αυτού διήλθον τα νέφη αυτού, χάλαζα και άνθρακες πυρός.
\par 13 Και εβρόντησεν εν ουρανοίς ο Κύριος, και ο Ύψιστος έδωκε την φωνήν αυτού· χάλαζα και άνθρακες πυρός.
\par 14 Και απέστειλε τα βέλη αυτού και εσκόρπισεν αυτούς· και αστραπάς επλήθυνε και συνετάραξεν αυτούς.
\par 15 Και εφάνησαν τα βάθη των υδάτων και ανεκαλύφθησαν τα θεμέλια της οικουμένης, από της επιτιμήσεώς σου, Κύριε, από του φυσήματος της πνοής των μυκτήρων σου.
\par 16 Εξαπέστειλεν εξ ύψους· έλαβέ με· είλκυσέ με εξ υδάτων πολλών.
\par 17 Ηλευθέρωσέ με εκ του δυνατού εχθρού μου, και εκ των μισούντων με, διότι ήσαν δυνατώτεροί μου.
\par 18 Προέφθασάν με εν τη ημέρα της θλίψεώς μου· αλλ' ο Κύριος εστάθη το αντιστήριγμά μου·
\par 19 και εξήγαγέ με εις ευρυχωρίαν· ηλευθέρωσέ με διότι ηυδόκησεν εις εμέ.
\par 20 Αντήμειψέ με ο Κύριος κατά την δικαιοσύνην μου· κατά την καθαρότητα των χειρών μου ανταπέδωκεν εις εμέ.
\par 21 Διότι εφύλαξα τας οδούς του Κυρίου, και δεν ησέβησα εκκλίνας από του Θεού μου.
\par 22 Διότι πάσαι αι κρίσεις αυτού ήσαν έμπροσθέν μου, και τα διατάγματα αυτού δεν απεμάκρυνα απ' εμού·
\par 23 και εστάθην άμεμπτος προς αυτόν, και εφυλάχθην από της ανομίας μου.
\par 24 Και ανταπέδωκεν εις εμέ ο Κύριος κατά την δικαιοσύνην μου, κατά την καθαρότητα των χειρών μου έμπροσθεν των οφθαλμών αυτού.
\par 25 Μετά οσίου όσιος θέλεις είσθαι· μετά ανδρός τελείου τέλειος θέλεις είσθαι·
\par 26 μετά καθαρού, καθαρός θέλεις είσθαι· και μετά διεστραμμένου διεστραμμένως θέλεις φερθή.
\par 27 Διότι συ θέλεις σώσει λαόν τεθλιμμένον· οφθαλμούς δε υπερηφάνων θέλεις ταπεινώσει.
\par 28 Διότι συ θέλεις φωτίσει τον λύχνον μου· Κύριος ο Θεός μου θέλει φωτίσει το σκότος μου.
\par 29 Διότι διά σου θέλω διασπάσει στράτευμα, και διά του Θεού μου θέλω υπερπηδήσει τείχος.
\par 30 Του Θεού, η οδός αυτού είναι άμωμος· ο λόγος του Κυρίου είναι δεδοκιμασμένος· είναι ασπίς πάντων των ελπιζόντων επ' αυτόν.
\par 31 Διότι τις Θεός πλην του Κυρίου; και τις φρούριον πλην του Θεού ημών;
\par 32 Ο Θεός είναι ο περιζωννύων με δύναμιν, και καθιστών άμωμον την οδόν μου.
\par 33 Κάμνει τους πόδας μου ως των ελάφων και με στήνει επί τους υψηλούς τόπους μου.
\par 34 Διδάσκει τας χείρας μου εις πόλεμον, και έκαμε τόξον χαλκούν τους βραχίονάς μου.
\par 35 Και έδωκας εις εμέ την ασπίδα της σωτηρίας σου· και η δεξιά σου με υπεστήριξε και η αγαθότης σου με εμεγάλυνεν.
\par 36 Επλάτυνας τα βήματά μου υποκάτω μου, και οι πόδες μου δεν εκλονίσθησαν.
\par 37 Κατεδίωξα τους εχθρούς μου και έφθασα αυτούς· και δεν επέστρεψα εωσού συνετέλεσα αυτούς.
\par 38 Συνέτριψα αυτούς και δεν ηδυνήθησαν να ανεγερθώσιν· έπεσον υπό τους πόδας μου.
\par 39 Και περιέζωσάς με δύναμιν εις πόλεμον· συνέκαμψας υποκάτω μου τους επανισταμένους επ' εμέ.
\par 40 Και έκαμες τους εχθρούς μου να τρέψωσιν εις εμέ τα νώτα, και εξωλόθρευσα τους μισούντάς με.
\par 41 Εβόησαν, και ουδείς ο σώζων· προς τον Κύριον, και δεν εισήκουσεν αυτών.
\par 42 Και κατελέπτυνα αυτούς ως κόνιν κατά πρόσωπον ανέμου· απετίναξα αυτούς ως τον πηλόν των οδών.
\par 43 Ηλευθέρωσάς με εκ των αντιλογιών του λαού· κατέστησάς με κεφαλήν εθνών· λαός, τον οποίον δεν εγνώρισα, εδούλευσεν εις εμέ.
\par 44 Μόλις ήκουσαν, και υπήκουσαν εις εμέ· ξένοι υπετάχθησαν εις εμέ.
\par 45 Ξένοι παρελύθησαν και κατετρόμαξαν εκ των αποκρύφων τόπων αυτών.
\par 46 Ζη Κύριος, και ευλογημένον το φρουριόν μου· και ας υψωθή ο Θεός της σωτηρίας μου·
\par 47 ο Θεός ο εκδικών με και υποτάσσων λαούς υποκάτω μου·
\par 48 όστις με ελευθερόνει εκ των εχθρών μου. Ναι, με υψόνεις υπεράνω των επανισταμένων επ' εμέ· ηλευθέρωσάς με από ανδρός αδίκου.
\par 49 Διά τούτο θέλω σε υμνεί, Κύριε, μεταξύ των εθνών, και εις το όνομά σου θέλω ψάλλει.
\par 50 Αυτός μεγαλύνει τας σωτηρίας του βασιλέως αυτού, και κάμνει έλεος εις τον κεχρισμένον αυτού, εις τον Δαβίδ και εις το σπέρμα αυτού έως αιώνος.

\chapter{19}

\par «Εις τον πρώτον μουσικόν. Ψαλμός του Δαβίδ.» Οι ουρανοί διηγούνται την δόξαν του Θεού, και το στερέωμα αναγγέλλει το έργον των χειρών αυτού.
\par 2 Η ημέρα προς την ημέραν λαλεί λόγον, και η νυξ προς την νύκτα αναγγέλλει γνώσιν.
\par 3 Δεν είναι λαλιά ουδέ λόγος, των οποίων η φωνή δεν ακούεται.
\par 4 Εις πάσαν την γην εξήλθεν ο φθόγγος αυτών και έως των περάτων της οικουμένης οι λόγοι αυτών. Εν αυτοίς έθεσε σκηνήν διά τον ήλιον·
\par 5 και ούτος εξέρχεται ως νυμφίος εκ του θαλάμου αυτού· αγάλλεται ως ο ανδρείος εις το να τρέξη το στάδιον·
\par 6 απ' άκρου του ουρανού είναι η έξοδος αυτού· και το κατάντημα αυτού έως άκρου αυτού· και δεν κρύπτεται ουδέν από της θερμότητος αυτού.
\par 7 Ο νόμος του Κυρίου είναι άμωμος, επιστρέφων ψυχήν· η μαρτυρία του Κυρίου πιστή, σοφίζουσα τον απλούν·
\par 8 τα διατάγματα του Κυρίου ευθέα, ευφραίνοντα καρδίαν· η εντολή του Κυρίου λαμπρά, φωτίζουσα οφθαλμούς·
\par 9 ο φόβος του Κυρίου καθαρός, διαμένων εις τον αιώνα· αι κρίσεις του Κυρίου αληθιναί, δίκαιαι εν ταυτώ·
\par 10 πλέον επιθυμηταί παρά το χρυσίον, μάλιστα παρά πλήθος καθαρού χρυσίου, και γλυκύτεραι υπέρ το μέλι και τους σταλαγμούς της κηρήθρας.
\par 11 Ο δούλός σου μάλιστα νουθετείται δι' αυτών· εις την τήρησιν αυτών η ανταμοιβή είναι μεγάλη.
\par 12 Τις συναισθάνεται τα εαυτού αμαρτήματα; καθάρισόν με από των κρυφίων αμαρτημάτων.
\par 13 Και έτι προφύλαξον τον δούλον σου από υπερηφανιών· ας μη με κυριεύσωσι· τότε θέλω είσθαι τέλειος, και θέλω καθαρισθή από μεγάλης παρανομίας.
\par 14 Ας ήναι ευάρεστα τα λόγια του στόματός μου και η μελέτη της καρδίας μου ενώπιόν σου, Κύριε, φρούριόν μου και λυτρωτά μου.

\chapter{20}

\par «Εις τον πρώτον μουσικόν. Ψαλμός του Δαβίδ.» Ο Κύριος να σου υπακούση εν ημέρα θλίψεως το όνομα του Θεού του Ιακώβ να σε υπερασπίση.
\par 2 Να σοι εξαποστείλη βοήθειαν εκ του αγιαστηρίου και εκ της Σιών να σε υποστηρίξη.
\par 3 Να ενθυμηθή πάσας τας προσφοράς σου και να προσδεχθή το ολοκαύτωμά σου. Διάψαλμα.
\par 4 Να σοι δώση κατά την καρδίαν σου και να εκπληρώση πάσαν βουλήν σου.
\par 5 Εις την σωτηρίαν σου θέλομεν χαρή· και εις το όνομα του Θεού ημών θέλομεν υψώσει τας σημαίας· ο Κύριος να εκπληρώση πάντα τα αιτήματά σου.
\par 6 Τώρα εγνώρισα ότι έσωσεν ο Κύριος τον χριστόν αυτού· θέλει εισακούσει αυτού εκ του ουρανού της αγιότητος αυτού· η σωτηρία της δεξιάς αυτού γίνεται εν δυνάμει.
\par 7 Οι μεν εις αμάξας, οι δε εις ίππους, αλλ' ημείς εις το όνομα Κυρίου του Θεού ημών θέλομεν καυχάσθαι·
\par 8 ούτοι συνεκάμφθησαν και έπεσον· ημείς δε ανέστημεν και ανωρθώθημεν.
\par 9 Κύριε, σώσον τον βασιλέα· και εισάκουσον ημών, καθ' ην ημέραν σε επικαλεσθώμεν.

\chapter{21}

\par «Εις τον πρώτον μουσικόν. Ψαλμός του Δαβίδ.» Κύριε, εν τη δυνάμει σου θέλει ευφραίνεσθαι ο βασιλεύς· και πόσον θέλει υπεραγάλλεσθαι εν τη σωτηρία σου.
\par 2 Την επιθυμίαν της καρδίας αυτού έδωκας εις αυτόν, και της αιτήσεως των χειλέων αυτού δεν εστέρησας αυτόν. Διάψαλμα.
\par 3 Διότι προέφθασας αυτόν εν ευλογίαις αγαθότητος· έθεσας επί την κεφαλήν αυτού στέφανον εκ καθαρού χρυσίου.
\par 4 Ζωήν σε εζήτησε, και έδωκας εις αυτόν μακρότητα ημερών εις αιώνα αιώνος.
\par 5 Μεγάλη η δόξα αυτού διά της σωτηρίας σου· τιμήν και μεγαλοπρέπειαν έθεσας επ' αυτόν.
\par 6 Διότι έθεσας αυτόν ευλογίαν εις τον αιώνα· υπερεύφρανας αυτόν διά του προσώπου σου.
\par 7 Διότι ο βασιλεύς ελπίζει επί τον Κύριον, και διά του ελέους του Υψίστου δεν θέλει σαλευθή.
\par 8 Η χειρ σου θέλει ευρεί πάντας τους εχθρούς σου· η δεξιά σου θέλει ευρεί τους μισούντάς σε.
\par 9 Θέλεις κάμει αυτούς ως κάμινον πυρός εν τω καιρώ της οργής σου· ο Κύριος θέλει καταπίει αυτούς εν τω θυμώ αυτού· και πυρ θέλει καταφάγει αυτούς.
\par 10 Θέλεις αφανίσει από της γης τον καρπόν αυτών, και το σπέρμα αυτών από των υιών των ανθρώπων.
\par 11 Διότι εμηχανεύθησαν κακά εναντίον σου· διελογίσθησαν βουλήν, αλλά δεν ίσχυσαν.
\par 12 Διά τούτο θέλεις κάμει αυτούς να τρέψωσι τα νώτα, όταν επί τας χορδάς σου ετοιμάσης τα βέλη σου κατά του προσώπου αυτών.
\par 13 Υψώθητι, Κύριε, εν τη δυνάμει σου· θέλομεν υμνεί και ψαλμωδεί την δύναμίν σου.

\chapter{22}

\par «Εις τον πρώτον μουσικόν, επί Αγέλεθ Σάχαρ. Ψαλμός του Δαβίδ.» Θεέ μου, Θεέ μου, διά τι με εγκατέλιπες; διά τι ίστασαι μακράν από της σωτηρίας μου και από των λόγων των στεναγμών μου;
\par 2 Θεέ μου, κράζω την ημέραν, και δεν αποκρίνεσαι· και την νύκτα, και δεν σιωπώ.
\par 3 Συ δε ο Άγιος κατοικείς μεταξύ των επαίνων του Ισραήλ.
\par 4 Επί σε ήλπισαν οι πατέρες ημών· ήλπισαν, και ηλευθέρωσας αυτούς.
\par 5 Προς σε έκραξαν και εσώθησαν· επί σε ήλπισαν και δεν κατησχύνθησαν.
\par 6 Εγώ δε είμαι σκώληξ και ουχί άνθρωπος· όνειδος ανθρώπων και εξουθένημα του λαού.
\par 7 Με εμυκτήρισαν πάντες οι βλέποντές με· ανοίγουσι τα χείλη, κινούσι την κεφαλήν, λέγοντες,
\par 8 Ήλπισεν επί τον Κύριον· ας ελευθερώση αυτόν· ας σώση αυτόν, επειδή θέλει αυτόν.
\par 9 Αλλά συ είσαι ο εκσπάσας με εκ της κοιλίας· η ελπίς μου από των μαστών της μητρός μου.
\par 10 Επί σε ερρίφθην εκ μήτρας· εκ κοιλίας της μητρός μου συ είσαι ο Θεός μου.
\par 11 Μη απομακρυνθής απ' εμού· διότι η θλίψις είναι πλησίον· διότι ουδείς ο βοηθών.
\par 12 Ταύροι πολλοί με περιεκύκλωσαν· ταύροι δυνατοί εκ Βασάν με περιεστοίχισαν.
\par 13 Ήνοιξαν επ' εμέ το στόμα αυτών, ως λέων αρπάζων και βρυχόμενος.
\par 14 Εξεχύθην ως ύδωρ, και εξηρθρώθησαν πάντα τα οστά μου· η καρδία μου έγεινεν ως κηρίον, κατατήκεται εν μέσω των εντοσθίων μου.
\par 15 Η δύναμίς μου εξηράνθη ως όστρακον, και η γλώσσα μου εκολλήθη εις τον λάρυγγά μου· και συ με κατεβίβασας εις το χώμα του θανάτου.
\par 16 Διότι κύνες με περιεκύκλωσαν· σύναξις πονηρευομένων με περιέκλεισεν· ετρύπησαν τας χείρας μου και τους πόδας μου·
\par 17 Δύναμαι να αριθμήσω πάντα τα οστά μου· ούτοι με ενατενίζουσι και με παρατηρούσι.
\par 18 Διεμερίσθησαν τα ιμάτιά μου εις εαυτούς· και επί τον ιματισμόν μου έβαλον κλήρον.
\par 19 Αλλά συ, Κύριε, μη απομακρυνθής· συ, η δύναμίς μου, σπεύσον εις βοήθειάν μου.
\par 20 Ελευθέρωσον από ρομφαίας την ψυχήν μου την μεμονωμένην μου από δυνάμεως κυνός.
\par 21 Σώσόν με εκ στόματος λέοντος και εισάκουσόν μου, ελευθερόνων με από κεράτων μονοκερώτων.
\par 22 Θέλω διηγείσθαι το όνομά σου προς τους αδελφούς μου· εν μέσω συνάξεως θέλω σε επαινεί.
\par 23 Οι φοβούμενοι τον Κύριον, αινείτε αυτόν· άπαν το σπέρμα Ιακώβ, δοξάσατε αυτόν· και φοβήθητε αυτόν, άπαν το σπέρμα Ισραήλ.
\par 24 Διότι δεν εξουθένωσε και δεν απεστράφη την θλίψιν του τεθλιμμένου, και δεν έκρυψε το πρόσωπον αυτού απ' αυτού· και ότε εβόησε προς αυτόν, εισήκουσεν.
\par 25 Από σου θέλει αρχίζει η αίνεσίς μου εν εκκλησία μεγάλη· θέλω αποδώσει τας ευχάς μου ενώπιον των φοβουμένων αυτόν.
\par 26 Οι τεθλιμμένοι θέλουσι φάγει και θέλουσι χορτασθή· θέλουσιν αινέσει τον Κύριον οι εκζητούντες αυτόν· η καρδία σας θέλει ζη εις τον αιώνα.
\par 27 Θέλουσιν ενθυμηθή και επιστραφή προς τον Κύριον πάντα τα πέρατα της γής· και θέλουσι προσκυνήσει ενώπιόν σου πάσαι αι φυλαί των εθνών.
\par 28 Διότι του Κυρίου είναι η βασιλεία, και αυτός εξουσιάζει τα έθνη.
\par 29 Θέλουσι φάγει και θέλουσι προσκυνήσει πάντες οι παχείς της γής· ενώπιον αυτού θέλουσι κλίνει πάντες οι καταβαίνοντες εις το χώμα· και ουδείς την ζωήν αυτού θέλει δυνηθή να φυλάξη.
\par 30 Οι μεταγενέστεροι θέλουσι δουλεύσει αυτόν· θέλουσιν αναγραφή εις τον Κύριον ως γενεά αυτού.
\par 31 Θέλουσιν ελθεί και αναγγείλει την δικαιοσύνην αυτού, προς λαόν, όστις μέλλει να γεννηθή· διότι αυτός έκαμε τούτο.

\chapter{23}

\par «Ψαλμός του Δαβίδ.» Ο Κύριος είναι ο ποιμήν μου· δεν θέλω στερηθή ουδενός.
\par 2 Εις βοσκάς χλοεράς με ανέπαυσεν· εις ύδατα αναπαύσεως με ωδήγησεν.
\par 3 Ηνώρθωσε την ψυχήν μου· με ώδήγησε διά τρίβων δικαιοσύνης ένεκεν του ονόματος αυτού.
\par 4 Και εν κοιλάδι σκιάς θανάτου εάν περιπατήσω, δεν θέλω φοβηθή κακόν· διότι συ είσαι μετ' εμού· η ράβδος σου και η βακτηρία σου, αύται με παρηγορούσιν.
\par 5 Ητοίμασας έμπροσθέν μου τράπεζαν απέναντι των εχθρών μου· ήλειψας εν ελαίω την κεφαλήν μου· το ποτήριόν μου υπερχειλίζει.
\par 6 Βεβαίως χάρις και έλεος θέλουσι με ακολουθεί πάσας τας ημέρας της ζωής μου· και θέλω κατοικεί εν τω οίκω του Κυρίου εις μακρότητα ημερών.

\chapter{24}

\par «Ψαλμός του Δαβίδ.» Του Κυρίου είναι η γη και το πλήρωμα αυτής· η οικουμένη και οι κατοικούντες εν αυτή.
\par 2 Διότι αυτός εθεμελίωσεν αυτήν επί των θαλασσών, και εστερέωσεν αυτήν επί των ποταμών.
\par 3 Τις θέλει αναβή εις το όρος του Κυρίου; και τις θέλει σταθή εν τω τόπω τω αγίω αυτού;
\par 4 Ο αθώος τας χείρας και ο καθαρός την καρδίαν· όστις δεν έδωκεν εις ματαιότητα την ψυχήν αυτού και δεν ώμοσε μετά δολιότητος.
\par 5 Ούτος θέλει λάβει ευλογίαν παρά Κυρίου και δικαιοσύνην παρά του Θεού της σωτηρίας αυτού.
\par 6 Αύτη είναι η γενεά των εκζητούντων αυτόν, των ζητούντων το πρόσωπόν σου, Θεέ του Ιακώβ. Διάψαλμα.
\par 7 Σηκώσατε, πύλαι, τας κεφαλάς σας, και υψώθητε, θύραι αιώνιοι, και θέλει εισέλθει ο Βασιλεύς της δόξης.
\par 8 Τις ούτος ο Βασιλεύς της δόξης; ο Κύριος ο κραταιός και δυνατός, ο Κύριος ο δυνατός εν πολέμω.
\par 9 Σηκώσατε, πύλαι, τας κεφαλάς σας, και υψώθητε, θύραι αιώνιοι, και θέλει εισέλθει ο Βασιλεύς της δόξης.
\par 10 Τις είναι ούτος ο Βασιλεύς της δόξης; ο Κύριος των δυνάμεων· αυτός είναι ο Βασιλεύς της δόξης. Διάψαλμα.

\chapter{25}

\par «Ψαλμός του Δαβίδ.» Προς σε, Κύριε, ύψωσα την ψυχήν μου.
\par 2 Θεέ μου, επί σε ήλπισα· ας μη καταισχυνθώ, ας μη χαρώσιν επ' εμέ οι εχθροί μου.
\par 3 Βεβαίως πάντες οι προσμένοντές σε δεν θέλουσι καταισχυνθή· ας καταισχυνθώσιν οι μωροί παραβάται.
\par 4 Δείξον μοι, Κύριε, τας οδούς σου· δίδαξόν με τα βήματά σου.
\par 5 Οδήγησόν με εν τη αληθεία σου και δίδαξόν με· διότι συ είσαι ο Θεός της σωτηρίας μου· σε προσμένω όλην την ημέραν.
\par 6 Μνήσθητι, Κύριε, τους οικτιρμούς σου και τα ελέη σου, διότι είναι απ' αιώνος.
\par 7 Τας αμαρτίας της νεότητός μου και τας παραβάσεις μου μη μνησθής· κατά το έλεός σου μνήσθητί μου συ, Κύριε, ένεκεν της αγαθότητός σου.
\par 8 Αγαθός και ευθύς ο Κύριος· διά τούτο θέλει διδάξει τους αμαρτωλούς την οδόν.
\par 9 Θέλει οδηγήσει τους πράους εν κρίσει και θέλει διδάξει τους πράους την οδόν αυτού.
\par 10 Πάσαι αι οδοί του Κυρίου είναι έλεος και αλήθεια εις τους φυλάττοντας την διαθήκην αυτού και τα μαρτύρια αυτού.
\par 11 Ένεκεν του ονόματός σου, Κύριε, συγχώρησον την ανομίαν μου, διότι είναι μεγάλη.
\par 12 Τις είναι ο άνθρωπος ο φοβούμενος τον Κύριον; αυτόν θέλει διδάξει την οδόν, την οποίαν πρέπει να εκλέξη·
\par 13 Η ψυχή αυτού θέλει κατοικεί εν αγαθοίς, και το σπέρμα αυτού θέλει κληρονομήσει την γην.
\par 14 Το απόρρητον του Κυρίου είναι μετά των φοβουμένων αυτόν και την διαθήκην αυτού θέλει φανερώσει εις αυτούς.
\par 15 Οι οφθαλμοί μου είναι διαπαντός προς τον Κύριον, διότι αυτός θέλει εξαγάγει εκ παγίδος τους πόδας μου.
\par 16 Επίβλεψον επ' εμέ και ελέησόν με, διότι μεμονωμένος και τεθλιμμένος είμαι.
\par 17 Αι θλίψεις της καρδίας μου ηύξησαν· εξάγαγέ με εκ των στενοχωριών μου.
\par 18 Ιδέ την θλίψιν μου και τον μόχθον μου, και άφες πάσας τας αμαρτίας μου.
\par 19 Ιδέ τους εχθρούς μου, διότι επληθύνθησαν και μίσος άδικον με εμίσησαν.
\par 20 Φύλαξον την ψυχήν μου και σώσον με· ας μη καταισχυνθώ, διότι επί σε ήλπισα.
\par 21 Ακακία και ευθύτης ας με περιφυλάττωσι, διότι σε προσέμεινα.
\par 22 Λύτρωσον, Θεέ, τον Ισραήλ εκ πασών των θλίψεων αυτού.

\chapter{26}

\par «Ψαλμός του Δαβίδ.» Κρίνον με, Κύριε· διότι εγώ περιεπάτησα εν ακακία μου· και επί τον Κύριον ήλπισα, και δεν θέλω σαλευθή.
\par 2 Εξέτασόν με, Κύριε, και δοκίμασόν με· δοκίμασον τους νεφρούς μου και την καρδίαν μου.
\par 3 Διότι το έλεος σου είναι έμπροσθεν των οφθαλμών μου· και περιεπάτησα εν τη αληθεία σου.
\par 4 Δεν εκάθησα μετά ανθρώπων ματαίων· και μετά υποκριτών δεν θέλω υπάγει.
\par 5 Εμίσησα την σύναξιν των πονηρευομένων, και μετά ασεβών δεν θέλω καθήσει.
\par 6 Θέλω νίψει εν αθωότητι τας χείρας μου και θέλω περικυκλώσει το θυσιαστήριόν σου, Κύριε·
\par 7 διά να κάμω να αντηχήση φωνή αινέσεως, και διά να διηγηθώ πάντα τα θαυμάσιά σου.
\par 8 Κύριε, ηγάπησα την κατοίκησιν του οίκου σου και τον τόπον της σκηνής της δόξης σου.
\par 9 Μη συμπεριλάβης μετά αμαρτωλών την ψυχήν μου και μετά ανδρών αιμάτων την ζωήν μου·
\par 10 εις των οποίων τας χείρας είναι ανομία, και η δεξιά αυτών πλήρης δώρων.
\par 11 Αλλ' εγώ θέλω περιπατεί εν ακακία μου· λύτρωσόν με και ελέησόν με.
\par 12 Ο πους μου ίσταται εν τη ευθύτητι· εν εκκλησίαις θέλω ευλογεί τον Κύριον.

\chapter{27}

\par «Ψαλμός του Δαβίδ.» Ο Κύριος είναι φως μου και σωτηρία μου· τίνα θέλω φοβηθή; ο Κύριος είναι δύναμις της ζωής μου· από τίνος θέλω δειλιάσει;
\par 2 Ότε επλησίασαν επ' εμέ οι πονηρευόμενοι, διά να καταφάγωσι την σάρκα μου, οι αντίδικοι και οι εχθροί μου, αυτοί προσέκρουσαν και έπεσον.
\par 3 Και αν παραταχθή εναντίον μου στράτευμα, η καρδία μου δεν θέλει φοβηθή· και αν πόλεμος σηκωθή επ' εμέ, και τότε θέλω ελπίζει.
\par 4 Εν εζήτησα παρά του Κυρίου, τούτο θέλω εκζητεί· το να κατοικώ εν τω οίκω του Κυρίου πάσας τας ημέρας της ζωής μου, να θεωρώ το κάλλος του Κυρίου και να επισκέπτωμαι τον ναόν αυτού.
\par 5 Διότι εν ημέρα συμφοράς θέλει με κρύψει εν τη σκηνή αυτού· Θέλει με κρύψει εν τω αποκρύφω της σκηνής αυτού· θέλει με υψώσει επί βράχον·
\par 6 και ήδη η κεφαλή μου θέλει υψωθή υπεράνω των εχθρών μου των περικυκλούντων με· και θέλω θυσιάσει εν τη σκηνή αυτού θυσίας αλαλαγμού· θέλω υμνεί και θέλω ψαλμωδεί εις τον Κύριον.
\par 7 Άκουσον, Κύριε, της φωνής μου, όταν κράζω· και ελέησόν με και εισάκουσόν μου.
\par 8 Ζητήσατε το πρόσωπόν μου, είπε περί σου η καρδία μου. Το πρόσωπόν σου, Κύριε, θέλω ζητήσει.
\par 9 Μη κρύψης απ' εμού το πρόσωπόν σου· μη απορρίψης εν οργή τον δούλον σου· συ εστάθης βοήθειά μου· μη με αφήσης και μη με εγκαταλείψης, Θεέ της σωτηρίας μου.
\par 10 Και αν ο πατήρ μου και η μήτηρ μου με εγκαταλείψωσιν, ο Κύριος όμως θέλει με προσδεχθή.
\par 11 Δίδαξόν με, Κύριε, την οδόν σου και οδήγησόν με εν οδώ ευθεία ένεκεν των εχθρών μου.
\par 12 Μη με παραδώσης εις την επιθυμίαν των εχθρών μου· διότι ηγέρθησαν κατ' εμού μάρτυρες ψευδείς και πνέοντες αδικίαν.
\par 13 Ουαί εάν δεν επίστευον να ίδω τα αγαθά του Κυρίου εν γη ζώντων.
\par 14 Πρόσμενε τον Κύριον· ανδρίζου, και ας κραταιωθή η καρδία σου· και πρόσμενε τον Κύριον.

\chapter{28}

\par «Ψαλμός του Δαβίδ.» Προς σε θέλω κράξει, Κύριε· φρούριόν μου, μη σιωπήσης προς εμέ· μήποτε σιωπήσης προς εμέ, και ομοιωθώ με τους καταβαίνοντας εις τον λάκκον.
\par 2 Άκουσον της φωνής των δεήσεών μου, όταν κράζω προς σε, όταν υψόνω τας χείρας μου προς τον ναόν τον άγιόν σου.
\par 3 Μη με σύρης μετά των ασεβών και μετά των εργαζομένων ανομίαν, οίτινες λαλούντες ειρήνην μετά των πλησίον αυτών, έχουσι κακίαν εν ταις καρδίαις αυτών.
\par 4 Δος εις αυτούς κατά τα έργα αυτών και κατά την πονηρίαν των επιχειρήσεων αυτών· κατά τα έργα των χειρών αυτών δος εις αυτούς· απόδος εις αυτούς την ανταμοιβήν αυτών.
\par 5 Επειδή δεν προσέχουσιν εις τας πράξεις του Κυρίου και εις τα έργα των χειρών αυτού, θέλει κατακρημνίσει αυτούς και δεν θέλει ανοικοδομήσει αυτούς.
\par 6 Ευλογητός ο Κύριος, διότι ήκουσε της φωνής των δεήσεών μου.
\par 7 Ο Κύριος είναι δύναμίς μου και ασπίς μου· επ' αυτόν ήλπισεν η καρδία μου, και εβοηθήθην· διά τούτο ηγαλλίασεν η καρδία μου, και με τας ωδάς μου θέλω υμνεί αυτόν.
\par 8 Ο Κύριος είναι δύναμις του λαού αυτού· αυτός είναι και υπεράσπισις της σωτηρίας του κεχρισμένου αυτού.
\par 9 Σώσον τον λαόν σου και ευλόγησον την κληρονομίαν σου· και ποίμαινε αυτούς και ύψωσον αυτούς έως αιώνος.

\chapter{29}

\par «Ψαλμός του Δαβίδ.» Απόδοτε εις τον Κύριον, υιοί των δυνατών, απόδοτε εις τον Κύριον δόξαν και κράτος.
\par 2 Απόδοτε εις τον Κύριον την δόξαν του ονόματος αυτού· προσκυνήσατε τον Κύριον εν τω μεγαλοπρεπεί αγιαστηρίω αυτού.
\par 3 Η φωνή του Κυρίου είναι επί των υδάτων· ο Θεός της δόξης βροντά· ο Κύριος είναι επί υδάτων πολλών.
\par 4 Η φωνή του Κυρίου είναι δυνατή· η φωνή του Κυρίου είναι μεγαλοπρεπής.
\par 5 Η φωνή του Κυρίου συντρίβει κέδρους· και συντρίβει Κύριος τας κέδρους του Λιβάνου·
\par 6 Και κάμνει αυτάς να σκιρτώσιν ως μόσχος τον Λίβανον και το Σιριών ως νέος μονόκερως.
\par 7 Η φωνή του Κυρίου καταδιαιρεί τας φλόγας του πυρός.
\par 8 Η φωνή του Κυρίου σείει την έρημον· ο Κύριος σείει την έρημον Κάδης.
\par 9 Η φωνή του Κυρίου κάμνει να κοιλοπονώσιν αι έλαφοι και γυμνόνει τα δάση· εν δε τω ναώ αυτού πας τις κηρύττει την δόξαν αυτού.
\par 10 Ο Κύριος κάθηται επί τον κατακλυσμόν· και κάθηται ο Κύριος Βασιλεύς εις τον αιώνα.
\par 11 Ο Κύριος θέλει δώσει δύναμιν εις τον λαόν αυτού· ο Κύριος θέλει ευλογήσει τον λαόν αυτού εν ειρήνη.

\chapter{30}

\par «Ψαλμός ωδής εις τον εγκαινιασμόν του οίκου του Δαβίδ.» Θέλω σε μεγαλύνει, Κύριε· διότι συ με ανύψωσας, και δεν εύφρανας τους εχθρούς μου επ' εμέ.
\par 2 Κύριε ο Θεός μου, εβόησα προς σε, και με εθεράπευσας.
\par 3 Κύριε, ανεβίβασας εξ άδου την ψυχήν μου· μ' εφύλαξας την ζωήν, διά να μη καταβώ εις τον λάκκον.
\par 4 Ψαλμωδήσατε εις τον Κύριον, οι όσιοι αυτού, και υμνείτε την μνήμην της αγιωσύνης αυτού.
\par 5 Διότι η οργή αυτού διαρκεί μίαν μόνην στιγμήν· ζωή όμως είναι εν τη ευμενεία αυτού· το εσπέρας δύναται να συγκατοικήση κλαυθμός, αλλά το πρωΐ έρχεται αγαλλίασις.
\par 6 Εγώ δε είπα εν τη ευτυχία μου, δεν θέλω σαλευθή εις τον αιώνα.
\par 7 Κύριε, διά της ευμενείας σου εστερέωσας το όρος μου. Απέκρυψας το πρόσωπόν σου, και εταράχθην.
\par 8 Προς σε, Κύριε, έκραξα· και προς τον Κύριον εδεήθην.
\par 9 Τις ωφέλεια εν τω αίματί μου, εάν καταβώ εις τον λάκκον; μήπως θέλει σε υμνεί ο κονιορτός; θέλει αναγγέλλει την αλήθειάν σου;
\par 10 Άκουσον, Κύριε, και ελέησόν με· Κύριε, γενού βοηθός μου.
\par 11 Μετέβαλες εις εμέ τον θρήνόν μου εις χαράν· έλυσας τον σάκκόν μου και με περιέζωσας ευφροσύνην·
\par 12 διά να ψαλμωδή εις σε η δόξα μου και να μη σιωπά. Κύριε ο Θεός μου, εις τον αιώνα θέλω σε υμνεί.

\chapter{31}

\par «Εις τον πρώτον μουσικόν. Ψαλμός του Δαβίδ.» Επί σε, Κύριε, ήλπισα ας μη καταισχυνθώ εις τον αιώνα· εν τη δικαιοσύνη σου σώσον με.
\par 2 Κλίνον εις εμέ το ωτίον σου· τάχυνον να με ελευθερώσης· γενού εις εμέ ισχυρός βράχος· οίκος καταφυγής, διά να με σώσης.
\par 3 Διότι πέτρα μου και φρούριόν μου είσαι· και ένεκεν του ονόματός σου οδήγησόν με και διάθρεψόν με.
\par 4 Εκβαλέ με εκ της παγίδος, την οποίαν έκρυψαν δι' εμέ· διότι είσαι η δύναμίς μου.
\par 5 Εις τας χείρας σου παραδίδω το πνεύμά μου· συ με ελύτρωσας, Κύριε ο Θεός της αληθείας.
\par 6 Εμίσησα τους προσέχοντας εις τας ματαιότητας του ψεύδους· εγώ δε επί τον Κύριον ελπίζω.
\par 7 Θέλω αγάλλεσθαι και ευφραίνεσθαι εις το έλεός σου· διότι είδες την θλίψιν μου, εγνώρισας την ψυχήν μου εν στενοχωρίαις,
\par 8 και δεν με συνέκλεισας εις την χείρα του εχθρού· έστησας εν ευρυχωρία τους πόδας μου.
\par 9 Ελέησόν με, Κύριε, διότι είμαι εν θλίψει· εμαράνθη από της λύπης ο οφθαλμός μου, η ψυχή μου και η κοιλία μου.
\par 10 Διότι εξέλιπεν εν οδύνη η ζωή μου και τα έτη μου εν στεναγμοίς· ησθένησεν από ταλαιπωρίας μου η δύναμίς μου, και τα οστά μου κατεφθάρησαν.
\par 11 Εις πάντας τους εχθρούς μου έγεινα όνειδος και εις τους γείτονάς μου σφόδρα, και φόβος εις τους γνωστούς μου· οι βλέποντές με έξω έφευγον απ' εμού.
\par 12 Ελησμονήθην από της καρδίας ως νεκρός· έγεινα ως σκεύος συντετριμμένον.
\par 13 Διότι ήκουσα τον ονειδισμόν πολλών· φόβος ήτο πανταχόθεν· ότε συνεβουλεύθησαν κατ' εμού· εμηχανεύθησαν να αφαιρέσωσι την ζωήν μου.
\par 14 Αλλ' εγώ επί σε, Κύριε, ήλπισα· είπα, συ είσαι ο Θεός μου.
\par 15 Εις τας χείρας σου είναι οι καιροί μου· λύτρωσόν με εκ χειρός των εχθρών μου και εκ των καταδιωκόντων με.
\par 16 Επίφανον το πρόσωπον σου επί τον δούλον σου· σώσον με εν τω ελέει σου.
\par 17 Κύριε, ας μη καταισχυνθώ, διότι σε επεκαλέσθην· ας καταισχυνθώσιν οι ασεβείς, ας σιωπήσωσιν εν τω άδη.
\par 18 Άλαλα ας γείνωσι τα χείλη τα δόλια, τα λαλούντα σκληρώς κατά του δικαίου εν υπερηφανία και καταφρονήσει.
\par 19 Πόσον μέγάλη είναι η αγαθότης σου, την οποίαν εφύλαξας εις τους φοβουμένους σε και ενήργησας εις τους ελπίζοντας επί σε έμπροσθεν των υιών των ανθρώπων.
\par 20 Θέλεις κρύψει αυτούς εν αποκρύφω του προσώπου σου από της αλαζονείας των ανθρώπων· θέλεις κρύψει αυτούς εν σκηνή από της αντιλογίας των γλωσσών.
\par 21 Ευλογητός ο Κύριος, διότι εθαυμάστωσε το έλεος αυτού προς εμέ εν πόλει οχυρά.
\par 22 Εγώ δε είπα εν τη εκπλήξει μου, Απερρίφθην απ' έμπροσθεν των οφθαλμών σου· πλην συ ήκουσας της φωνής των δεήσεών μου, ότε εβόησα προς σε.
\par 23 Αγαπήσατε τον Κύριον, πάντες οι όσιοι αυτού· ο Κύριος φυλάττει τους πιστούς, και ανταποδίδει περισσώς εις τους πράττοντας την υπερηφανίαν.
\par 24 Ανδρίζεσθε, και ας κραταιωθή η καρδία σας, πάντες οι ελπίζοντες επί Κύριον.

\chapter{32}

\par «Ψαλμός του Δαβίδ. Μασχίλ.» Μακάριος εκείνος, του οποίου συνεχωρήθη η παράβασις, του οποίου εσκεπάσθη η αμαρτία.
\par 2 Μακάριος ο άνθρωπος, εις τον οποίον ο Κύριος δεν λογαριάζει ανομίαν και εις του οποίου το πνεύμα δεν υπάρχει δόλος.
\par 3 Ότε απεσιώπησα, επαλαιώθησαν τα οστά μου εκ του ολολυγμού μου όλην την ημέραν.
\par 4 Επειδή ημέραν και νύκτα εβαρύνθη η χειρ σου επ' εμέ· η υγρότης μου μετεβλήθη εις θερινήν ξηρασίαν. Διάψαλμα.
\par 5 Την αμαρτίαν μου εφανέρωσα προς σε, και την ανομίαν μου δεν έκρυψα· είπα, Εις τον Κύριον θέλω εξομολογηθή τας παραβάσεις μου· και συ συνεχώρησας την ανομίαν της αμαρτίας μου. Διάψαλμα.
\par 6 Διά τούτο πας όσιος θέλει προσεύχεσθαι προς σε εν καιρώ προσήκοντι· βεβαίως εν κατακλυσμώ πολλών υδάτων ταύτα δεν θέλουσιν εγγίζει εις αυτόν.
\par 7 Συ είσαι η σκέπη μου· θέλεις με φυλάττει από θλίψεως· αγαλλίασιν λυτρώσεως θέλεις με περικυκλόνει. Διάψαλμα.
\par 8 Εγώ θέλω σε συνετίσει και θέλω σε διδάξει την οδόν, εις την οποίαν πρέπει να περιπατής· θέλω σε συμβουλεύει· επί σε θέλει είσθαι ο οφθαλμός μου.
\par 9 Μη γίνεσθε ως ίππος, ως ημίονος, εις τα οποία δεν υπάρχει σύνεσις· των οποίων το στόμα πρέπει να κρατήται εν κημώ και χαλινώ, άλλως δεν ήθελον πλησιάζει εις σε.
\par 10 Πολλαί αι μάστιγες του ασεβούς· τον δε ελπίζοντα επί Κύριον έλεος θέλει περικυκλώσει.
\par 11 Ευφραίνεσθε εις τον Κύριον και αγάλλεσθε, δίκαιοι· και αλαλάξατε, πάντες οι ευθείς την καρδίαν.

\chapter{33}

\par Αγάλλεσθε, δίκαιοι, εν Κυρίω· εις τους ευθείς αρμόζει η αίνεσις.
\par 2 Υμνείτε τον Κύριον εν κιθάρα· εν ψαλτηρίω δεκαχόρδω ψαλμωδήσατε εις αυτόν.
\par 3 Ψάλλετε εις αυτόν άσμα νέον· καλώς σημαίνετε τα όργανά σας εν αλαλαγμώ.
\par 4 Διότι ευθύς είναι ο λόγος του Κυρίου, και πάντα τα έργα αυτού μετά αληθείας.
\par 5 Αγαπά δικαιοσύνην και κρίσιν· από του ελέους του Κυρίου είναι πλήρης η γη.
\par 6 Με τον λόγον του Κυρίου έγειναν οι ουρανοί, και διά της πνοής του στόματος αυτού πάσα η στρατιά αυτών.
\par 7 Συνήγαγεν ως σωρόν τα ύδατα της θαλάσσης· έβαλεν εις αποθήκας τας αβύσσους.
\par 8 Ας φοβηθή τον Κύριον πάσα η γή· ας τρομάξωσιν απ' αυτού πάντες οι κάτοικοι της οικουμένης.
\par 9 Διότι αυτός είπε, και έγεινεν· αυτός προσέταξε, και εστερεώθη.
\par 10 Ο Κύριος ματαιόνει την βουλήν των εθνών, ανατρέπει τους διαλογισμούς των λαών.
\par 11 Η βουλή του Κυρίου μένει εις τον αιώνα· οι λογισμοί της καρδίας αυτού εις γενεάν και γενεάν.
\par 12 Μακάριον το έθνος, του οποίου ο Θεός είναι ο Κύριος. Ο λαός, τον οποίον εξέλεξε διά κληρονομίαν αυτού.
\par 13 Ο Κύριος διέκυψεν εξ ουρανού· είδε πάντας τους υιούς των ανθρώπων.
\par 14 Εκ του τόπου της κατοικήσεως αυτού θεωρεί πάντας τους κατοίκους της γης.
\par 15 Εξ ίσου έπλασε τας καρδίας αυτών· γνωρίζει πάντα τα έργα αυτών.
\par 16 Δεν σώζεται βασιλεύς διά πλήθους στρατεύματος· ο δυνατός δεν ελευθερούται διά της μεγάλης αυτού ανδρείας.
\par 17 Μάταιος ο ίππος προς σωτηρίαν· και διά της πολλής αυτού δυνάμεως δεν θέλει σώσει.
\par 18 Ιδού, ο οφθαλμός του Κυρίου είναι επί τους φοβουμένους αυτόν· επί τους ελπίζοντας επί το έλεος αυτού·
\par 19 διά να ελευθερώση εκ θανάτου την ψυχήν αυτών, και εν καιρώ πείνης να διαφυλάξη αυτούς εις ζωήν.
\par 20 Η ψυχή ημών προσμένει τον Κύριον· αυτός είναι βοηθός ημών και ασπίς ημών.
\par 21 Διότι εις αυτόν θέλει ευφρανθή η καρδία ημών, επειδή επί το όνομα αυτού το άγιον ηλπίσαμεν.
\par 22 Γένοιτο, Κύριε, το έλεός σου εφ' ημάς, καθώς ηλπίσαμεν επί σε.

\chapter{34}

\par «Ψαλμός του Δαβίδ, ότε μετέβαλε τον τρόπον αυτού έμπροσθεν του Αβιμέλεχ· ούτος δε απέλυσεν αυτόν, και απήλθε.» Θέλω ευλογεί τον Κύριον εν παντί καιρώ· η αίνεσις αυτού θέλει είσθαι διαπαντός εν τω στόματί μου.
\par 2 Εις τον Κύριον θέλει καυχάσθαι η ψυχή μου· οι ταπεινοί θέλουσιν ακούσει, και θέλουσι χαρή.
\par 3 Μεγαλύνατε τον Κύριον μετ' εμού, και ας υψώσωμεν ομού το όνομα αυτού.
\par 4 Εξεζήτησα τον Κύριον, και επήκουσέ μου, και εκ πάντων των φόβων μου με ηλευθέρωσεν.
\par 5 Απέβλεψαν προς αυτόν και εφωτίσθησαν, και τα πρόσωπα αυτών δεν κατησχύνθησαν.
\par 6 Ούτος ο πτωχός έκραξε, και ο Κύριος εισήκουσε, και εκ πασών των θλίψεων αυτού έσωσεν αυτόν.
\par 7 Άγγελος Κυρίου στρατοπεδεύει κύκλω των φοβουμένων αυτόν και ελευθερόνει αυτούς.
\par 8 Γεύθητε και ιδέτε ότι αγαθός ο Κύριος· μακάριος ο άνθρωπος ο ελπίζων επ' αυτόν.
\par 9 Φοβήθητε τον Κύριον, οι άγιοι αυτού· διότι δεν υπάρχει στέρησις εις τους φοβουμένους αυτόν.
\par 10 Οι πλούσιοι πτωχεύουσι και πεινώσιν· αλλ' οι εκζητούντες τον Κύριον δεν στερούνται ουδενός αγαθού.
\par 11 Έλθετε, τέκνα, ακούσατέ μου· τον φόβον του Κυρίου θέλω σας διδάξει.
\par 12 Τις είναι ο άνθρωπος όστις θέλει ζωήν, αγαπά ημέρας, διά να ίδη καλόν;
\par 13 Φύλαττε την γλώσσαν σου από κακού, και τα χείλη σου από του να λαλώσι δόλον·
\par 14 Έκκλινον από του κακού και πράττε το αγαθόν· ζήτει ειρήνην και κυνήγει αυτήν.
\par 15 Οι οφθαλμοί του Κυρίου είναι επί τους δικαίους, και τα ώτα αυτού εις την κραυγήν αυτών.
\par 16 Το πρόσωπον του Κυρίου είναι κατά των πραττόντων κακόν, διά να αφανίση από της γης το μνημόσυνον αυτών.
\par 17 Έκραξαν οι δίκαιοι, και ο Κύριος εισήκουσε, και εκ πασών των θλίψεων αυτών ελευθέρωσεν αυτούς.
\par 18 Ο Κύριος είναι πλησίον των συντετριμμένων την καρδίαν, και σώζει τους ταπεινούς το πνεύμα.
\par 19 Πολλαί αι θλίψεις του δικαίου, αλλ' εκ πασών τούτων θέλει ελευθερώσει αυτόν ο Κύριος.
\par 20 Αυτός φυλάττει πάντα τα οστά αυτού· ουδέν εκ τούτων θέλει συντριφθή.
\par 21 Η κακία θέλει θανατώσει τον αμαρτωλόν· και οι μισούντες τον δίκαιον θέλουσιν απολεσθή.
\par 22 Ο Κύριος λυτρόνει την ψυχήν των δούλων αυτού, και δεν θέλουσιν απολεσθή πάντες οι ελπίζοντες επ' αυτόν.

\chapter{35}

\par «Ψαλμός του Δαβίδ.» Δίκασον, Κύριε, τους δικαζομένους μετ' εμού· πολέμησον τους πολεμούντάς με.
\par 2 Ανάλαβε όπλον και ασπίδα και ανάστηθι εις βοήθειάν μου.
\par 3 Και δράξον το δόρυ και σύγκλεισον την οδόν των καταδιωκόντων με· ειπέ εις την ψυχήν μου, Εγώ είμαι η σωτηρία σου.
\par 4 Ας αισχυνθώσι και ας εντραπώσιν οι ζητούντες την ψυχήν μου· ας στραφώσιν εις τα οπίσω και ας αισχυνθώσιν οι βουλευόμενοι το κακόν μου.
\par 5 Ας ήναι ως λεπτόν άχυρον κατά πρόσωπον ανέμου, και άγγελος Κυρίου ας διώκη αυτούς.
\par 6 Ας ήναι η οδός αυτών σκότος και ολίσθημα, και άγγελος Κυρίου ας καταδιώκη αυτούς.
\par 7 Διότι αναιτίως έκρυψαν δι' εμέ την παγίδα αυτών εν λάκκω· αναιτίως έσκαψαν αυτόν διά την ψυχήν μου.
\par 8 Ας έλθη επ' αυτόν όλεθρος απροσδόκητος· και η παγίς αυτού, την οποίαν έκρυψεν, ας συλλάβη αυτόν· ας πέση εις αυτήν εν ολέθρω.
\par 9 Η δε ψυχή μου θέλει αγάλλεσθαι εις τον Κύριον, θέλει χαίρει εις την σωτηρίαν αυτού.
\par 10 Πάντα τα οστά μου θέλουσιν ειπεί, Κύριε, τις όμοιός σου, όστις ελευθερόνεις τον πτωχόν από του ισχυροτέρου αυτού, και τον πτωχόν και τον πένητα από του διαρπάζοντος αυτόν;
\par 11 Σηκωθέντες μάρτυρες άδικοι, με ηρώτων περί πραγμάτων, τα οποία εγώ δεν ήξευρον·
\par 12 Ανταπέδωκαν εις εμέ κακόν αντί καλού· στέρησιν εις την ψυχήν μου.
\par 13 Εγώ δε, ότε αυτοί ήσαν εν θλίψει, ενεδυόμην σάκκον· εταπείνωσα εν νηστεία την ψυχήν μου· και η προσευχή μου επέστρεφεν εις τον κόλπον μου.
\par 14 Εφερόμην ως προς φίλον, ως προς αδελφόν μου· έκυπτον σκυθρωπάζων, ως ο πενθών διά την μητέρα αυτού.
\par 15 Αλλ' αυτοί εχάρησαν διά την συμφοράν μου και συνήχθησαν· συνήχθησαν εναντίον μου οι χαμερπείς, και εγώ δεν ήξευρον· με εξέσχιζον και δεν έπαυον·
\par 16 Μετά υποκριτικών χλευαστών εν συμποσίοις έτριζον κατ' εμού τους οδόντας αυτών.
\par 17 Κύριε, πότε θέλεις ιδεί; ελευθέρωσον την ψυχήν μου από του ολέθρου αυτών, την μεμονωμένην μου εκ των λεόντων.
\par 18 Εγώ θέλω σε υμνεί εν μεγάλη συνάξει· μεταξύ πολυαρίθμου λαού θέλω σε υμνεί.
\par 19 Ας μη χαρώσιν επ' εμέ οι εχθρευόμενοί με αδίκως· οι μισούντές με αναιτίως ας μη νεύωσι με τους οφθαλμούς.
\par 20 Διότι δεν ελάλουν ειρήνην, αλλά εμελέτων δόλους κατά των ησυχαζόντων επί της γής·
\par 21 και επλάτυναν κατ' εμού το στόμα αυτών, Λέγοντες, Εύγε, εύγε· είδεν ο οφθαλμός ημών.
\par 22 Είδες, Κύριε· μη σιωπήσης· Κύριε, μη απομακρυνθής απ' εμού.
\par 23 Εγέρθητι και εξύπνησον διά την κρίσιν μου, Θεέ μου και Κύριέ μου, διά την δίκην μου.
\par 24 Κρίνόν με, Κύριε ο Θεός μου, κατά την δικαιοσύνην σου, και ας μη χαρώσιν επ' εμέ.
\par 25 Ας μη είπωσιν εν ταις καρδίαις αυτών, Εύγε, ψυχή ημών. μηδέ ας είπωσι, Κατεπίομεν αυτόν.
\par 26 Ας αισχυνθώσι και ας εντραπώσιν ομού οι επιχαίροντες εις το κακόν μου· ας ενδυθώσιν εντροπήν και όνειδος οι μεγαλαυχούντες κατ' εμού.
\par 27 Ας ευφρανθώσι και ας χαρώσιν οι θέλοντες την δικαιοσύνην μου· και διαπαντός ας λέγωσιν, Ας μεγαλυνθή ο Κύριος, όστις θέλει την ειρήνην του δούλου αυτού.
\par 28 Και η γλώσσα μου θέλει μελετά την δικαιοσύνην σου και τον έπαινόν σου όλην την ημέραν.

\chapter{36}

\par «Εις τον πρώτον μουσικόν. Ψαλμός του Δαβίδ, δούλου του Κυρίου.» Του ασεβούς η παρανομία λέγει εν τη καρδία μου, δεν είναι φόβος Θεού έμπροσθεν των οφθαλμών αυτού.
\par 2 Διότι απατά εαυτόν εις τους οφθαλμούς αυτού περί του ότι θέλει ευρεθή η ανομία αυτού διά να μισηθή.
\par 3 Τα λόγια του στόματος αυτού είναι ανομία και δόλος· δεν ηθέλησε να νοήση διά να πράττη το αγαθόν.
\par 4 Ανομίαν διαλογίζεται επί της κλίνης αυτού· ίσταται εν οδώ ουχί καλή· το κακόν δεν μισεί.
\par 5 Κύριε, έως του ουρανού φθάνει το έλεός σου, η αλήθειά σου έως των νεφελών.
\par 6 Η δικαιοσύνη σου είναι ως τα υψηλά όρη· αι κρίσεις σου άβυσσος μεγάλη· ανθρώπους και κτήνη σώζεις, Κύριε.
\par 7 Πόσον πολύτιμον είναι το έλεός σου, Θεέ. Διά τούτο οι υιοί των ανθρώπων ελπίζουσιν επί την σκιάν των πτερύγων σου.
\par 8 Θέλουσι χορτασθή από του πάχους του οίκου σου, και από του χειμάρρου της τρυφής σου θέλεις ποτίσει αυτούς.
\par 9 Διότι μετά σου είναι η πηγή της ζωής· εν τω φωτί σου θέλομεν ιδεί φως.
\par 10 Έκτεινον το έλεός σου προς τους γνωρίζοντάς σε, και την δικαιοσύνην σου προς τους ευθείς την καρδίαν.
\par 11 Ας μη έλθη επ' εμέ πους υπερηφανίας· και χειρ ασεβών ας μη με σαλεύση.
\par 12 Εκεί έπεσον οι εργάται της ανομίας· κατεσπρώχθησαν και δεν θέλουσι δυνηθή να ανεγερθώσι.

\chapter{37}

\par «Ψαλμός του Δαβίδ.» Μη αγανάκτει διά τους πονηρευομένους, μηδέ ζήλευε τους εργάτας της ανομίας.
\par 2 Διότι ως χόρτος ταχέως θέλουσι κοπή, και ως χλωρά βοτάνη θέλουσι καταμαρανθή.
\par 3 Έλπιζε επί Κύριον και πράττε το αγαθόν· κατοίκει την γην και νέμου την αλήθειαν·
\par 4 και ευφραίνου εν Κυρίω, και θέλει σοι δώσει τα ζητήματα της καρδίας σου.
\par 5 Ανάθες εις τον Κύριον την οδόν σου και έλπιζε επ' αυτόν, και αυτός θέλει ενεργήσει·
\par 6 και θέλει εξάξει ως φως την δικαιοσύνην σου και την κρίσιν σου ως μεσημβρίαν.
\par 7 Αναπαύου επί τον Κύριον και πρόσμενε αυτόν· μη αγανάκτει διά τον κατευοδούμενον εν τη οδώ αυτού, διά άνθρωπον πράττοντα παρανομίας.
\par 8 Παύσον από θυμού και άφες την οργήν· μηδόλως αγανάκτει ώστε να πράττης πονηρά.
\par 9 Διότι οι πονηρευόμενοι θέλουσιν εξολοθρευθή· οι δε προσμένοντες τον Κύριον, ούτοι θέλουσι κληρονομήσει την γην.
\par 10 Διότι έτι μικρόν και ο ασεβής δεν θέλει υπάρχει· και θέλεις ζητήσει τον τόπον αυτού, και δεν θέλει ευρεθή·
\par 11 οι πραείς όμως θέλουσι κληρονομήσει την γήν· και θέλουσι κατατρυφά εν πολλή ειρήνη.
\par 12 Ο ασεβής μηχανάται κατά του δικαίου, και τρίζει κατ' αυτού τους οδόντας αυτού.
\par 13 Ο Κύριος θέλει γελάσει επ' αυτώ, επειδή βλέπει ότι έρχεται η ημέρα αυτού.
\par 14 Οι ασεβείς εξέσπασαν ρομφαίαν και ενέτειναν το τόξον αυτών, διά να καταβάλωσι τον πτωχόν και τον πένητα, διά να σφάξωσι τους περιπατούντας εν ευθύτητι.
\par 15 Η ρομφαία αυτών θέλει εμβή εις την καρδίαν αυτών, και τα τόξα αυτών θέλουσι συντριφθή.
\par 16 Κάλλιον το ολίγον του δικαίου παρά ο πλούτος πολλών ασεβών.
\par 17 Διότι οι βραχίονες των ασεβών θέλουσι συντριφθή· τους δε δικαίους υποστηρίζει ο Κύριος.
\par 18 Γινώσκει ο Κύριος τας ημέρας των αμέμπτων· και η κληρονομία αυτών θέλει είσθαι εις τον αιώνα·
\par 19 δεν θέλουσι καταισχυνθή εν καιρώ πονηρώ· και εν ημέραις πείνης θέλουσι χορτασθή.
\par 20 Οι δε ασεβείς θέλουσιν εξολοθρευθή· και οι εχθροί του Κυρίου, ως το πάχος των αρνίων, θέλουσιν αναλωθή· εις καπνόν θέλουσι διαλυθή.
\par 21 Δανείζεται ο ασεβής και δεν αποδίδει, ο δε δίκαιος ελεεί και δίδει.
\par 22 Διότι οι ευλογημένοι αυτού θέλουσι κληρονομήσει την γήν· οι δε κατηραμένοι αυτού θέλουσιν εξολοθρευθή.
\par 23 Όταν υπό Κυρίου κατευθύνωνται τα διαβήματα του ανθρώπου, η οδός αυτού είναι αρεστή εις αυτόν.
\par 24 Εάν πέση, δεν θέλει συντριφθή· διότι ο Κύριος υποστηρίζει την χείρα αυτού.
\par 25 Νέος ήμην και ήδη εγήρασα, και δεν είδον δίκαιον εγκαταλελειμμένον ουδέ το σπέρμα αυτού ζητούν άρτον.
\par 26 Όλην την ημέραν ελεεί και δανείζει, και το σπέρμα αυτού είναι εις ευλογίαν.
\par 27 Έκκλινον από του κακού και πράττε το αγαθόν, και θέλεις διαμένει εις τον αιώνα.
\par 28 Διότι ο Κύριος αγαπά κρίσιν, και δεν εγκαταλείπει τους οσίους αυτού· εις τον αιώνα θέλουσι διαφυλαχθή· το δε σπέρμα των ασεβών θέλει εξολοθρευθή.
\par 29 Οι δίκαιοι θέλουσι κληρονομήσει την γην, και επ' αυτής θέλουσι κατοικεί εις τον αιώνα.
\par 30 Το στόμα του δικαίου μελετά σοφίαν, και η γλώσσα αυτού λαλεί κρίσιν.
\par 31 Ο νόμος του Θεού αυτού είναι εν τη καρδία αυτού· τα διαβήματα αυτού δεν θέλουσιν ολισθήσει.
\par 32 Κατασκοπεύει ο αμαρτωλός τον δίκαιον και ζητεί να θανατώση αυτόν.
\par 33 Ο Κύριος δεν θέλει αφήσει αυτόν εις τας χείρας αυτού, ουδέ θέλει καταδικάσει αυτόν όταν κρίνη αυτόν.
\par 34 Πρόσμενε τον Κύριον και φύλαττε την οδόν αυτού, και θέλει σε υψώσει διά να κληρονομήσης την γήν· όταν εξολοθρευθώσιν οι ασεβείς, θέλεις ιδεί.
\par 35 Είδον τον ασεβή υπερυψούμενον και εξηπλωμένον ως την χλωράν δάφνην·
\par 36 αλλ' ηφανίσθη· και ιδού, δεν υπήρχε· και εζήτησα αυτόν, και δεν ευρέθη.
\par 37 Παρατήρει τον άκακον και βλέπε τον ευθύν, ότι εις τον ειρηνικόν άνθρωπον θέλει είσθαι εγκατάλειμμα·
\par 38 οι δε παραβάται θέλουσιν όλως εξολοθρευθή· των ασεβών το εγκατάλειμμα θέλει αποκοπή.
\par 39 Των δικαίων όμως η σωτηρία είναι παρά Κυρίου· αυτός είναι η δύναμις αυτών εν καιρώ θλίψεως.
\par 40 Και θέλει βοηθήσει αυτούς ο Κύριος, και ελευθερώσει αυτούς· θέλει ελευθερώσει αυτούς από ασεβών και σώσει αυτούς· διότι ήλπισαν επ' αυτόν.

\chapter{38}

\par «Ψαλμός του Δαβίδ εις ανάμνησιν.» Κύριε, μη με ελέγξης εν τω θυμώ σου, μηδέ εν τη οργή σου παιδεύσης με.
\par 2 Διότι τα βέλη σου ενεπήχθησαν εις εμέ και η χειρ σου καταπιέζει με.
\par 3 Δεν υπάρχει υγεία εν τη σαρκί μου εξ αιτίας της οργής σου. δεν είναι ειρήνη εις τα οστά μου εξ αιτίας της αμαρτίας μου.
\par 4 Διότι αι ανομίαι μου υπερέβησαν την κεφαλήν μου· ως φορτίον βαρύ υπερεβάρυναν επ' εμέ.
\par 5 Εβρώμησαν και εσάπησαν αι πληγαί μου εξ αιτίας της ανοησίας μου.
\par 6 Εταλαιπωρήθην, εκυρτώθην εις άκρον· όλην την ημέραν περιπατώ σκυθρωπός.
\par 7 Διότι τα εντόσθιά μου γέμουσι φλογώσεως, και δεν υπάρχει υγεία εν τη σαρκί μου.
\par 8 Ησθένησα και καθ' υπερβολήν κατεκόπην· βρυχώμαι από της αδημονίας της καρδίας μου.
\par 9 Κύριε, ενώπιόν σου είναι πάσα η επιθυμία μου, και ο στεναγμός μου δεν κρύπτεται από σου.
\par 10 Η καρδία μου ταράττεται, η δύναμίς μου με εγκαταλείπει· και το φως των οφθαλμών μου, και αυτό δεν είναι μετ' εμού.
\par 11 Οι φίλοι μου και οι πλησίον μου στέκουσιν απέναντι της πληγής μου, και οι πλησιέστεροί μου στέκουσιν από μακρόθεν.
\par 12 Και οι ζητούντες την ψυχήν μου στήνουσιν εις εμέ παγίδας· και οι εκζητούντες το κακόν μου λαλούσι πονηρά, και μελετώσι δόλους όλην την ημέραν.
\par 13 Αλλ' εγώ ως κωφός δεν ήκουον και ήμην ως άφωνος, μη ανοίγων το στόμα αυτού.
\par 14 Και ήμην ως άνθρωπος μη ακούων και μη έχων αντιλογίαν εν τω στόματι αυτού.
\par 15 Διότι επί σε, Κύριε, ήλπισα· συ θέλεις μου εισακούσει, Κύριε ο Θεός μου.
\par 16 Επειδή είπα, Ας μη χαρώσιν επ' εμέ· όταν ολισθήση ο πους μου, αυτοί μεγαλαυχούσι κατ' εμού.
\par 17 Διότι είμαι έτοιμος να πέσω, και ο πόνος μου είναι πάντοτε έμπροσθέν μου.
\par 18 Επειδή εγώ θέλω αναγγέλλει την ανομίαν μου, θέλω λυπείσθαι διά την αμαρτίαν μου.
\par 19 Αλλ' οι εχθροί μου ζώσιν, υπερισχύουσι· και επληθύνθησαν οι μισούντές με αδίκως.
\par 20 Και οι ανταποδίδοντες κακόν αντί καλού είναι εναντίοι μου, επειδή κυνηγώ το καλόν.
\par 21 Μη με εγκαταλίπης, Κύριε· Θεέ μου, μη απομακρυνθής απ' εμού.
\par 22 Τάχυνον εις βοήθειάν μου, Κύριε, η σωτηρία μου.

\chapter{39}

\par «Εις τον πρώτον μουσικόν, τον Ιεδουθούν. Ψαλμός του Δαβίδ.» Είπα, Θέλω προσέχει εις τας οδούς μου, διά να μη αμαρτάνω διά της γλώσσης μου· θέλω φυλάττει το στόμα μου με χαλινόν, ενώ είναι ο ασεβής έμπροσθέν μου.
\par 2 Εστάθην άφωνος και σιωπηλός· εσιώπησα και από του να λέγω καλόν· και ο πόνος μου ανεταράχθη.
\par 3 Εθερμάνθη η καρδία μου εντός μου· ενώ εμελέτων, εξήφθη εν εμοί πύρ· ελάλησα διά της γλώσσης μου και είπα,
\par 4 Κάμε γνωστόν εις εμέ, Κύριε, το τέλος μου και τον αριθμόν των ημερών μου, τις είναι, διά να γνωρίσω πόσον έτι θέλω ζήσει.
\par 5 Ιδού, μέτρον σπιθαμής κατέστησας τας ημέρας μου, και ο καιρός της ζωής μου είναι ως ουδέν έμπροσθέν σου· επ' αληθείας πας άνθρωπος, καίτοι στερεός, είναι όλως ματαιότης. Διάψαλμα.
\par 6 Βεβαίως ο άνθρωπος περιπατεί εν φαντασία· βεβαίως εις μάτην ταράττεται· θησαυρίζει, και δεν εξεύρει τις θέλει συνάξει αυτά.
\par 7 Και τώρα, Κύριε, τι περιμένω; η ελπίς μου είναι επί σε.
\par 8 Από πασών των ανομιών μου λύτρωσόν με· μη με κάμης όνειδος του άφρονος.
\par 9 Έγεινα άφωνος· δεν ήνοιξα το στόμα μου, επειδή συ έκαμες τούτο.
\par 10 Απομάκρυνον απ' εμού την πληγήν σου· από της πάλης της χειρός σου εγώ απέκαμον.
\par 11 Όταν δι' ελέγχων παιδεύης άνθρωπον διά ανομίαν, Κατατρώγεις ως σκώληξ την ώραιότητα αυτού· τω όντι ματαιότης πας άνθρωπος. Διάψαλμα.
\par 12 Εισάκουσον, Κύριε, της προσευχής μου και δος ακρόασιν εις την κραυγήν μου· μη παρασιωπήσης εις τα δάκρυά μου. Διότι πάροικος είμαι παρά σοι και παρεπίδημος, καθώς πάντες οι πατέρες μου.
\par 13 Παύσαι απ' εμού, διά να αναλάβω δύναμιν, πριν αποδημήσω και δεν υπάρχω πλέον.

\chapter{40}

\par «Εις τον πρώτον μουσικόν. Ψαλμός του Δαβίδ.» Περιέμεινα εν υπομονή τον Κύριον, και έκλινε προς εμέ και ήκουσε της κραυγής μου·
\par 2 και με ανεβίβασεν εκ λάκκου ταλαιπωρίας, εκ βορβορώδους πηλού, και έστησεν επί πέτραν τους πόδας μου, εστερέωσε τα βήματά μου·
\par 3 και έβαλεν εν τω στόματί μου άσμα νέον, ύμνον εις τον Θεόν ημών· θέλουσιν ιδεί πολλοί και θέλουσι φοβηθή και θέλουσιν ελπίσει επί Κύριον.
\par 4 Μακάριος ο άνθρωπος, όστις έθεσε τον Κύριον ελπίδα αυτού και δεν αποβλέπει εις τους υπερηφάνους και εις τους κλίνοντας επί ψεύδη.
\par 5 Πολλά έκαμες συ, Κύριε ο Θεός μου, τα θαυμάσιά σου· και τους περί ημών διαλογισμούς σου δεν είναι δυνατόν να εκθέση τις εις σέ· εάν ήθελον να απαγγέλλω και να ομιλώ περί αυτών, υπερβαίνουσι πάντα αριθμόν.
\par 6 Θυσίαν και προσφοράν δεν ηθέλησας· διήνοιξας εν εμοί ώτα· ολοκαύτωμα και προσφοράν περί αμαρτίας δεν εζήτησας.
\par 7 Τότε είπα, Ιδού, έρχομαι· εν τω τόμω του βιβλίου είναι γεγραμμένον περί εμού·
\par 8 χαίρω, Θεέ μου, να εκτελώ το θέλημά σου· και ο νόμος σου είναι εν τω μέσω της καρδίας μου.
\par 9 Εκήρυξα δικαιοσύνην εν συνάξει μεγάλη· ιδού, δεν εμπόδισα τα χείλη μου, Κύριε, συ εξεύρεις.
\par 10 Την δικαιοσύνην σου δεν έκρυψα εντός της καρδίας μου· την αλήθειάν σου και την σωτηρίαν σου ανήγγειλα· δεν έκρυψα το έλεός σου και την αλήθειάν σου από συνάξεως μεγάλης.
\par 11 Συ, Κύριε, μη απομακρύνης τους οικτιρμούς σου απ' εμού· το έλεός σου και η αλήθειά σου ας με περιφρουρώσι διαπαντός.
\par 12 Διότι με περιεκύκλωσαν αναρίθμητα κακά· με κατέφθασαν αι ανομίαι μου, και δεν δύναμαι να θεωρώ αυτάς· επληθύνθησαν υπέρ τας τρίχας της κεφαλής μου· και η καρδία μου με εγκαταλείπει.
\par 13 Ευδόκησον, Κύριε, να με ελευθερώσης Κύριε, τάχυνον εις βοήθειάν μου.
\par 14 Ας αισχυνθώσι και ας εκτραπώσιν ομού οι ζητούντες την ψυχήν μου, διά να απολέσωσιν αυτήν· ας στραφώσιν εις τα οπίσω και ας εντραπώσιν οι θέλοντες το κακόν μου.
\par 15 Ας εξολοθρευθώσι διά μισθόν της αισχύνης αυτών οι λέγοντες προς εμέ, εύγε, εύγε.
\par 16 Ας αγάλλωνται και ας ευφραίνωνται εις σε πάντες οι ζητούντές σε· οι αγαπώντες την σωτηρίαν σου ας λέγωσι διαπαντός, Μεγαλυνθήτω ο Κύριος.
\par 17 Εγώ δε είμαι πτωχός και πένης· αλλ' ο Κύριος φροντίζει περί εμού· η βοήθειά μου και ο ελευθερωτής μου συ είσαι· Θεέ μου, μη βραδύνης.

\chapter{41}

\par «Εις τον πρώτον μουσικόν. Ψαλμός του Δαβίδ.» Μακάριος ο επιβλέπων εις τον πτωχόν· εν ημέρα θλίψεως θέλει ελευθερώσει αυτόν ο Κύριος.
\par 2 Ο Κύριος θέλει φυλάξει αυτόν και διατηρήσει την ζωήν αυτού· μακάριος θέλει είσθαι επί της γής· και δεν θέλεις παραδώσει αυτόν εις την επιθυμίαν των εχθρών αυτού.
\par 3 Ο Κύριος θέλει ενδυναμόνει αυτόν επί της κλίνης της ασθενείας· εν τη αρρωστία αυτού συ θέλεις στρόνει όλην την κλίνην αυτού.
\par 4 Εγώ είπα, Κύριε, ελέησόν με· ίασαι την ψυχήν μου, διότι ήμαρτον εις σε.
\par 5 Οι εχθροί μου λέγουσι κακά περί εμού, Πότε θέλει αποθάνει, και θέλει απολεσθή το όνομα αυτού;
\par 6 Και εάν τις έρχηται να με ίδη, ομιλεί ματαιότητα· η καρδία αυτού συνάγει εις εαυτήν ανομίαν· εξελθών έξω, λαλεί αυτήν.
\par 7 Κατ' εμού ψιθυρίζουσιν ομού πάντες οι μισούντές με· κατ' εμού διαλογίζονται κακά λέγοντες,
\par 8 Πράγμα κακόν εκολλήθη εις αυτόν· και κατάκοιτος ων δεν θέλει πλέον σηκωθή.
\par 9 Και αυτός ο άνθρωπος, μετά του οποίου έζων ειρηνικώς, επί τον οποίον ήλπισα, όστις έτρωγε τον άρτον μου, εσήκωσεν επ' εμέ πτέρναν.
\par 10 Αλλά συ, Κύριε, ελέησόν με και ανάστησόν με, και θέλω ανταποδώσει εις αυτούς.
\par 11 Εκ τούτου γνωρίζω ότι συ με ευνοείς, επειδή δεν θριαμβεύει κατ' εμού ο εχθρός μου.
\par 12 Εμέ δε, συ με εστήριξας εις την ακεραιότητά μου, και με εστερέωσας ενώπιόν σου εις τον αιώνα.
\par 13 Ευλογητός Κύριος ο Θεός του Ισραήλ, απ' αιώνος και έως αιώνος. Αμήν και αμήν.

\chapter{42}

\par «Εις τον πρώτον μουσικόν, Μασχίλ, διά τους υιούς Κορέ.» Καθώς επιποθεί η έλαφος τους ρύακας των υδάτων, ούτως η ψυχή μου σε επιποθεί, Θεέ.
\par 2 Διψά η ψυχή μου τον Θεόν, τον Θεόν τον ζώντα· πότε θέλω ελθεί και θέλω φανή ενώπιον του Θεού;
\par 3 Τα δάκρυά μου έγειναν τροφή μου ημέραν και νύκτα, όταν μοι λέγωσι καθ' ημέραν, Που είναι ο Θεός σου;
\par 4 Ταύτα ενεθυμήθην και εξέχεα την ψυχήν μου εντός μου, ότι διέβαινον μετά του πλήθους και περιεπάτουν μετ' αυτού έως του οίκου του Θεού, εν φωνή χαράς και αινέσεως, μετά πλήθους εορτάζοντος.
\par 5 Διά τι είσαι περίλυπος, ψυχή μου; και διά τι ταράττεσαι εντός μου; έλπισον επί τον Θεόν· επειδή έτι θέλω υμνεί αυτόν· το πρόσωπον αυτού είναι σωτηρία.
\par 6 Θεέ μου, η ψυχή μου είναι περίλυπος εντός μου· διά τούτο θέλω σε ενθυμείσθαι εκ γης Ιορδάνου και Ερμωνείμ εκ του όρους Μισάρ.
\par 7 Άβυσσος προσκαλεί άβυσσον εις τον ήχον των καταρρακτών σου· πάντα τα κύματά σου και αι τρικυμίαι σου διήλθον επ' εμέ.
\par 8 Εν τη ημέρα θέλει προστάξει ο Κύριος το έλεος αυτού· εν δε τη νυκτί θέλει είσθαι μετ' εμού η ωδή αυτού, η προσευχή μου προς τον Θεόν της ζωής μου.
\par 9 Θέλω ειπεί προς τον Θεόν, την πέτραν μου, διά τι με ελησμόνησας; διά τι περιπατώ σκυθρωπός εκ της καταθλίψεως του εχθρού;
\par 10 Οι εχθροί μου ονειδίζοντές με συντρίβουσι τα οστά μου, λέγοντές μοι καθ' ημέραν, Που είναι ο Θεός σου;
\par 11 Διά τι είσαι περίλυπος, ψυχή μου; και διά τι ταράττεσαι εντός μου; έλπισον επί τον Θεόν· επειδή έτι θέλω υμνεί αυτόν· αυτός είναι η σωτηρία του προσώπου μου και ο Θεός μου.

\chapter{43}

\par Κρίνόν με, Θεέ, και δίκασον την δίκην μου κατά έθνους ανοσίου· από ανθρώπου απάτης και ανομίας ελευθέρωσόν με·
\par 2 Διότι συ είσαι ο Θεός της δυνάμεώς μου· διά τι με απέβαλες; διά τι περιπατώ σκυθρωπός εκ της καταθλίψεως του εχθρού;
\par 3 Εξαπόστειλον το φως σου και την αλήθειάν σου· αυτά ας με οδηγώσιν· ας με φέρωσιν εις το όρος της αγιότητός σου και εις τα σκηνώματά σου.
\par 4 Τότε θέλω εισέλθει εις το θυσιαστήριον του Θεού, εις τον Θεόν, την ευφροσύνην της αγαλλιάσεώς μου· και θέλω σε δοξολογεί εν κιθάρα, ω Θεέ, ο Θεός μου.
\par 5 Διά τι είσαι περίλυπος, ψυχή μου; και διά τι ταράττεσαι εντός μου; έλπισον επί τον Θεόν· επειδή έτι θέλω υμνεί αυτόν· αυτός είναι η σωτηρία του προσώπου μου και ο Θεός μου.

\chapter{44}

\par «Εις τον πρώτον μουσικόν, διά τους υιούς Κορέ· Μασχίλ.» Θεέ, με τα ώτα ημών ηκούσαμεν, οι πατέρες ημών διηγήθησαν προς ημάς το έργον, το οποίον έπραξας εν ταις ημέραις αυτών, εν ημέραις αρχαίαις.
\par 2 Συ διά της χειρός σου εξεδίωξας έθνη και εφύτευσας αυτούς· κατέθλιψας λαούς και απεδίωξας αυτούς.
\par 3 Διότι δεν εκληρονόμησαν την γην διά της ρομφαίας αυτών, και ο βραχίων αυτών δεν έσωσεν αυτούς· αλλ' η δεξιά σου και ο βραχίων σου και το φως του προσώπου σου· διότι ευηρεστήθης εις αυτούς.
\par 4 Συ είσαι ο βασιλεύς μου, Θεέ, ο διορίζων τας σωτηρίας του Ιακώβ.
\par 5 Διά σου θέλομεν καταβάλει τους εχθρούς ημών· διά του ονόματός σου θέλομεν καταπατήσει τους επανισταμένους εφ' ημάς·
\par 6 Διότι δεν θέλω ελπίσει επί το τόξον ουδέ η ρομφαία μου θέλει με σώσει.
\par 7 Διότι συ έσωσας ημάς εκ των εχθρών ημών και κατήσχυνας τους μισούντας ημάς·
\par 8 εις τον Θεόν θέλομεν καυχάσθαι όλην την ημέραν, και το όνομά σου εις τον αιώνα θέλομεν υμνεί. Διάψαλμα.
\par 9 Όμως απέβαλες και κατήσχυνας ημάς, και δεν εξέρχεσαι πλέον μετά των στρατευμάτων ημών.
\par 10 Έκαμες ημάς να στρέψωμεν εις τα οπίσω έμπροσθεν του εχθρού· και οι μισούντες ημάς διαρπάζουσι τα ημέτερα εις εαυτούς.
\par 11 Παρέδωκας ημάς ως πρόβατα εις βρώσιν και εις τα έθνη διεσκόρπισας ημάς.
\par 12 Επώλησας τον λαόν σου άνευ τιμής, και δεν ηύξησας τον πλούτόν σου εκ της πωλήσεως αυτών.
\par 13 Κατέστησας ημάς όνειδος εις τους γείτονας ημών, κατάγελων και χλευασμόν εις τους πέριξ ημών.
\par 14 Κατέστησας ημάς παροιμίαν μεταξύ των εθνών, κίνησιν κεφαλής μεταξύ των λαών.
\par 15 Όλην την ημέραν η εντροπή μου είναι ενώπιόν μου, και η αισχύνη του προσώπου μου με εκάλυψε·
\par 16 διά την φωνήν του ονειδίζοντος και υβρίζοντος· διά τον εχθρόν και εκδικητήν.
\par 17 Πάντα ταύτα ήλθον εφ' ημάς, όμως δεν σε ελησμονήσαμεν και δεν ηθετήσαμεν την διαθήκην σου·
\par 18 Ουδέ εστράφη εις τα οπίσω η καρδία ημών, ουδέ εξέκλιναν τα βήματα ημών από της οδού σου.
\par 19 Αν και συνέτριψας ημάς εν τω τόπω των δρακόντων και περιεκάλυψας ημάς διά της σκιάς του θανάτου.
\par 20 Εάν ελησμονούμεν το όνομα του Θεού ημών και εξετείνομεν τας χείρας ημών εις Θεόν αλλότριον,
\par 21 ο Θεός δεν ήθελεν εξετάσει τούτο; διότι αυτός εξεύρει τα κρύφια της καρδίας.
\par 22 Ότι ένεκα σου θανατούμεθα όλην την ημέραν· ελογίσθημεν ως πρόβατα σφαγής.
\par 23 Εξεγέρθητι, διά τι καθεύδεις, Κύριε; εξεγέρθητι, μη αποβάλης ημάς διαπαντός.
\par 24 Διά τι κρύπτεις το πρόσωπόν σου; λησμονείς την ταλαιπωρίαν ημών και την καταδυνάστευσιν ημών;
\par 25 Διότι εταπεινώθη έως χώματος η ψυχή ημών· εκολλήθη εις την γην η κοιλία ημών.
\par 26 Ανάστηθι εις βοήθειαν ημών και λύτρωσον ημάς ένεκεν του ελέους σου.

\chapter{45}

\par «Εις τον πρώτον μουσικόν, επί Σοσανίμ, διά τους υιούς Κορέ· Μασχίλ· ωδή υπέρ του αγαπητού.» Η καρδία μου αναβρύει λόγον αγαθόν· εγώ λέγω τα έργα μου προς τον βασιλέα· η γλώσσα μου είναι κάλαμος γραμματέως ταχυγράφου.
\par 2 Συ είσαι ώραιότερος των υιών των ανθρώπων· εξεχύθη χάρις εις τα χείλη σου· διά τούτο σε ευλόγησεν ο Θεός εις τον αιώνα.
\par 3 Περίζωσον την ρομφαίαν σου επί τον μηρόν σου, δυνατέ, εν τη δόξη σου και εν τη μεγαλοπρεπεία σου.
\par 4 Και κατευοδού εν τη μεγαλειότητί σου και βασίλευε εν αληθεία και πραότητι και δικαιοσύνη· και η δεξιά σου θέλει σοι δείξει φοβερά πράγματα.
\par 5 Τα βέλη σου είναι οξέα· λαοί υποκάτω σου θέλουσι πέσει· και αυτά θέλουσιν εμπηχθή εις την καρδίαν των εχθρών του βασιλέως.
\par 6 Ο θρόνος σου, Θεέ, είναι εις τον αιώνα του αιώνος· σκήπτρον ευθύτητος είναι το σκήπτρον της βασιλείας σου.
\par 7 Ηγάπησας δικαιοσύνην και εμίσησας αδικίαν· διά τούτο έχρισέ σε ο Θεός, ο Θεός σου, έλαιον αγαλλιάσεως υπέρ τους μετόχους σου.
\par 8 Σμύρναν και αλόην και κασίαν ευοδιάζουσι πάντα τα ιμάτιά σου, όταν εξέρχησαι εκ των ελεφαντίνων παλατίων, διά των οποίων σε εύφραναν.
\par 9 Θυγατέρες βασιλέων παρίστανται εν ταις τιμαίς σου· η βασίλισσα εστάθη εκ δεξιών σου εστολισμένη με χρυσίον Οφείρ.
\par 10 Άκουσον, θύγατερ, και ιδέ, και κλίνον το ωτίον σου· και λησμόνησον τον λαόν σου και τον οίκον του πατρός σου·
\par 11 και θέλει επιθυμήσει ο βασιλεύς το κάλλος σου· διότι αυτός είναι ο κύριός σου· και προσκύνησον αυτόν.
\par 12 Και η θυγάτηρ της Τύρου θέλει παρασταθή με δώρα· το πρόσωπόν σου θέλουσιν ικετεύσει οι πλούσιοι του λαού.
\par 13 Όλη η δόξα της θυγατρός του βασιλέως είναι έσωθεν· το ένδυμα αυτής είναι χρυσοΰφαντον.
\par 14 Θέλει φερθή προς τον βασιλέα με ιμάτιον κεντητόν· παρθένοι σύντροφοι αυτής, κατόπιν αυτής, θέλουσι φερθή εις σε.
\par 15 Θέλουσι φερθή εν ευφροσύνη και αγαλλιάσει· θέλουσιν εισέλθει εις το παλάτιον του βασιλέως.
\par 16 Αντί των πατέρων σου θέλουσιν είσθαι οι υιοί σου· αυτούς θέλεις καταστήσει άρχοντας επί πάσαν την γην.
\par 17 Θέλω μνημονεύει το όνομά σου εις πάσας τας γενεάς· διά τούτο οι λαοί θέλουσι σε υμνεί εις αιώνα αιώνος.

\chapter{46}

\par «Εις τον πρώτον μουσικόν, διά τους υιούς Κορέ· ωδή επί Αλαμώθ.» Ο Θεός είναι καταφυγή ημών και δύναμις, βοήθεια ετοιμοτάτη εν ταις θλίψεσι.
\par 2 Διά τούτο δεν θέλομεν φοβηθή, και αν σαλευθή η γη και μετατοπισθώσι τα όρη εις το μέσον των θαλασσών·
\par 3 και αν ηχώσι και ταράττωνται τα ύδατα αυτών· και σείωνται τα όρη διά το έπαρμα αυτών. Διάψαλμα.
\par 4 Ποταμός, και οι ρύακες αυτού θέλουσιν ευφραίνει την πόλιν του Θεού, τον άγιον τόπον των σκηνωμάτων του Υψίστου.
\par 5 Ο Θεός είναι εν τω μέσω αυτής· δεν θέλει σαλευθή· θέλει βοηθήσει αυτήν ο Θεός από του χαράγματος της αυγής.
\par 6 Εφρύαξαν τα έθνη· εσαλεύθησαν αι βασιλείαι· έδωκε φωνήν αυτού· η γη ανελύθη.
\par 7 Ο Κύριος των δυνάμεων είναι μεθ' ημών· προπύργιον ημών είναι ο Θεός του Ιακώβ. Διάψαλμα.
\par 8 Έλθετε, ιδέτε τα έργα του Κυρίου· οποίας καταστροφάς έκαμεν εν τη γη.
\par 9 Καταπαύει τους πολέμους έως των περάτων της γής· συντρίβει τόξον και κατακόπτει λόγχην· καίει αμάξας εν πυρί.
\par 10 Ησυχάσατε και γνωρίσατε ότι εγώ είμαι ο Θεός· θέλω υψωθή μεταξύ των εθνών· θέλω υψωθή εν τη γη.
\par 11 Ο Κύριος των δυνάμεων είναι μεθ' ημών· προπύργιον ημών είναι ο Θεός του Ιακώβ. Διάψαλμα.

\chapter{47}

\par «Εις τον πρώτον μουσικόν. Ψαλμός διά τους υιούς Κορέ.» Πάντες οι λαοί, κροτήσατε χείρας· αλαλάξατε εις τον Θεόν εν φωνή αγαλλιάσεως.
\par 2 Διότι ο Κύριος είναι ύψιστος, φοβερός, Βασιλεύς μέγας επί πάσαν την γην.
\par 3 Υπέταξε λαούς εις ημάς και έθνη υπό τους πόδας ημών.
\par 4 Έκλεξε διά ημάς την κληρονομίαν την δόξαν του Ιακώβ, τον οποίον ηγάπησε. Διάψαλμα.
\par 5 Ανέβη ο Θεός εν αλαλαγμώ, ο Κύριος εν φωνή σάλπιγγος.
\par 6 Ψάλατε εις τον Θεόν, ψάλατε· ψάλατε εις τον Βασιλέα ημών, ψάλατε.
\par 7 Διότι Βασιλεύς πάσης της γης είναι ο Θεός· ψάλατε μετά συνέσεως.
\par 8 Ο Θεός βασιλεύει επί τα έθνη· ο Θεός κάθηται επί του θρόνου της αγιότητος αυτού.
\par 9 Οι άρχοντες των λαών συνήχθησαν μετά του λαού του Θεού του Αβραάμ· διότι του Θεού είναι αι ασπίδες της γής· υψώθη σφόδρα.

\chapter{48}

\par «Ωιδή Ψαλμού διά τους υιούς Κορέ.» Μέγας ο Κύριος και αινετός σφόδρα εν τη πόλει του Θεού ημών, τω όρει της αγιότητος αυτού.
\par 2 Ωραίον την θέσιν, χαρά πάσης της γης, είναι το όρος Σιών, προς τα πλάγια του βορρά· η πόλις του Βασιλέως του μεγάλου·
\par 3 ο Θεός εν τοις παλατίοις αυτής γνωρίζεται ως προπύργιον.
\par 4 Διότι, ιδού, οι βασιλείς συνήχθησαν· διήλθον ομού.
\par 5 Αυτοί, ως είδον, εθαύμασαν· εταράχθησαν και μετά σπουδής έφυγον.
\par 6 Τρόμος συνέλαβεν αυτούς εκεί· πόνοι ως τικτούσης.
\par 7 Δι' ανέμου ανατολικού συντρίβεις τα πλοία της Θαρσείς.
\par 8 Καθώς ηκούσαμεν, ούτω και είδομεν εν τη πόλει του Κυρίου των δυνάμεων, εν τη πόλει του Θεού ημών· ο Θεός θέλει θεμελιώσει αυτήν εις τον αιώνα. Διάψαλμα.
\par 9 Μελετώμεν, Θεέ, το έλεός σου εν μέσω του ναού σου.
\par 10 Κατά το όνομά σου, Θεέ, ούτω και η αίνεσίς σου είναι έως των περάτων της γής· η δεξιά σου είναι πλήρης δικαιοσύνης.
\par 11 Ας ευφραίνεται το όρος Σιών, ας αγάλλωνται αι θυγατέρες του Ιούδα διά τας κρίσεις σου.
\par 12 Κυκλώσατε την Σιών και περιέλθετε αυτήν· αριθμήσατε τους πύργους αυτής.
\par 13 Θέσατε την προσοχήν σας εις τα περιτειχίσματα αυτής· περιεργάσθητε τα παλάτια αυτής· διά να διηγήσθε εις γενεάν μεταγενεστέραν·
\par 14 Διότι ούτος ο Θεός είναι ο Θεός ημών εις τον αιώνα του αιώνος· αυτός θέλει οδηγεί ημάς μέχρι θανάτου.

\chapter{49}

\par «Εις τον πρώτον μουσικόν. Ψαλμός διά τους υιούς Κορέ.» Ακούσατε ταύτα, πάντες οι λαοί· ακροάσθητε, πάντες οι κάτοικοι της οικουμένης·
\par 2 μικροί τε και μεγάλοι, πλούσιοι ομού και πένητες.
\par 3 Το στόμα μου θέλει λαλήσει σοφίαν· και η μελέτη της καρδίας μου είναι σύνεσις.
\par 4 Θέλω κλίνει εις παραβολήν το ωτίον μου· θέλω εκθέσει εν κιθάρα το αίνιγμά μου.
\par 5 Διά τι να φοβώμαι εν ημέραις συμφοράς, όταν με περικυκλώση η ανομία των ενεδρευόντων με;
\par 6 Οίτινες ελπίζουσιν εις τα αγαθά αυτών και καυχώνται εις το πλήθος του πλούτου αυτών·
\par 7 ουδείς δύναται ποτέ να εξαγοράση αδελφόν, μηδέ να δώση εις τον Θεόν λύτρον δι' αυτόν·
\par 8 διότι πολύτιμος είναι η απολύτρωσις της ψυχής αυτών, και ανεύρητος διαπαντός,
\par 9 ώστε να ζη αιωνίως, να μη ίδη διαφθοράν.
\par 10 Διότι βλέπει τους σοφούς αποθνήσκοντας, καθώς και τον άφρονα και τον ανόητον απολλυμένους και καταλείποντας εις άλλους τα αγαθά αυτών.
\par 11 Ο εσωτερικός λογισμός αυτών είναι ότι οι οίκοι αυτών θέλουσιν υπάρχει εις τον αιώνα, αι κατοικίαι αυτών εις γενεάν και γενεάν· ονομάζουσι τα υποστατικά αυτών με τα ίδια αυτών ονόματα.
\par 12 Πλην ο άνθρωπος ο εν τιμή δεν διαμένει, ωμοιώθη με τα κτήνη τα φθειρόμενα.
\par 13 Αύτη η οδός αυτών είναι μωρία αυτών· και όμως οι απόγονοι αυτών ηδύνονται εις τα λόγια αυτών. Διάψαλμα.
\par 14 Ως πρόβατα εβλήθησαν εις τον άδην· θάνατος θέλει ποιμάνει αυτούς· και οι ευθείς θέλουσι κατακυριεύσει αυτούς το πρωΐ· η δε δύναμις αυτών θέλει παλαιωθή εν τω άδη, αφού έκαστος αφήση την κατοικίαν αυτού.
\par 15 Αλλ' ο Θεός θέλει λυτρώσει την ψυχήν μου εκ χειρός άδου· διότι θέλει με δεχθή. Διάψαλμα.
\par 16 Μη φοβού όταν πλουτήση άνθρωπος, όταν αυξήση η δόξα της οικίας αυτού·
\par 17 διότι εν τω θανάτω αυτού, δεν θέλει συμπαραλάβει ουδέν, ουδέ θέλει καταβή κατόπιν αυτού η δόξα αυτού.
\par 18 Αν και ηυλόγησε την ψυχήν αυτού εν τη ζωή αυτού, και οι άνθρωποι θέλωσι σε επαινεί αγαθοποιούντα σεαυτόν,
\par 19 Θέλει υπάγει εις την γενεάν των πατέρων αυτού· εις τον αιώνα δεν θέλουσιν ιδεί φως.
\par 20 Ο άνθρωπος ο εν τιμή και μη εννοών ωμοιώθη με τα κτήνη τα φθειρόμενα.

\chapter{50}

\par «Ψαλμός του Ασάφ.» Ο Θεός των θεών, ο Κύριος ελάλησε, και εκάλεσε την γην, από ανατολής ηλίου έως δύσεως αυτού.
\par 2 Εκ της Σιών, ήτις είναι η εντέλεια της ώραιότητος, έλαμψεν ο Θεός.
\par 3 Θέλει ελθεί ο Θεός ημών και δεν θέλει σιωπήσει· πυρ κατατρώγον θέλει είσθαι έμπροσθεν αυτού και πέριξ αυτού σφοδρά ανεμοζάλη,
\par 4 θέλει προσκαλέσει τους ουρανούς άνωθεν και την γην, διά να κρίνη τον λαόν αυτού.
\par 5 Συναθροίσατέ μοι τους οσίους μου, οίτινες έκαμον μετ' εμού συνθήκην επί θυσίας.
\par 6 Και οι ουρανοί θέλουσιν αναγγέλλει την δικαιοσύνην αυτού· διότι ο Θεός, αυτός είναι ο Κριτής. Διάψαλμα.
\par 7 Άκουσον, λαέ μου, και θέλω λαλήσει· Ισραήλ, και θέλω διαμαρτυρήσει κατά σού· Ο Θεός, ο Θεός σου είμαι εγώ.
\par 8 Δεν θέλω σε ελέγξει διά τας θυσίας σου, τα δε ολοκαυτώματά σου είναι διαπαντός ενώπιόν μου.
\par 9 Δεν θέλω δεχθή εκ του οίκου σου μόσχον, τράγους εκ των ποιμνίων σου.
\par 10 Διότι εμού είναι πάντα τα θηρία του δάσους, τα κτήνη τα επί χιλίων ορέων.
\par 11 Γνωρίζω πάντα τα πετεινά των ορέων, και τα θηρία του αγρού είναι μετ' εμού.
\par 12 Εάν πεινάσω, δεν θέλω ειπεί τούτο προς σέ· διότι εμού είναι η οικουμένη και το πλήρωμα αυτής.
\par 13 Μήπως εγώ θέλω φάγει κρέας ταύρων ή πίει αίμα τράγων;
\par 14 Θυσίασον εις τον Θεόν θυσίαν αινέσεως, και απόδος εις τον Ύψιστον τας ευχάς σου·
\par 15 και επικαλού εμέ εν ημέρα θλίψεως, θέλω σε ελευθερώσει, και θέλεις με δοξάσει.
\par 16 Προς δε τον ασεβή είπεν ο Θεός· Τι προς σε, να διηγήσαι τα διατάγματά μου και να αναλαμβάνης την διαθήκην μου εν τω στόματί σου;
\par 17 Συ δε μισείς παιδείαν και απορρίπτεις οπίσω σου τους λόγους μου.
\par 18 Εάν ίδης κλέπτην, τρέχεις μετ' αυτού· και μετά των μοιχών είναι η μερίς σου.
\par 19 Παραδίδεις το στόμα σου εις την κακίαν, και η γλώσσα σου περιπλέκει δολιότητα.
\par 20 Καθήμενος λαλείς κατά του αδελφού σου· βάλλεις σκάνδαλον κατά του υιού της μητρός σου.
\par 21 Ταύτα έπραξας, και εσιώπησα· υπέλαβες ότι είμαι τω όντι όμοιός σου· θέλω σε ελέγξει, και θέλω παραστήσει πάντα έμπροσθεν των οφθαλμών σου.
\par 22 Θέσατε λοιπόν τούτο εις τον νούν σας, οι λησμονούντες τον Θεόν, μήποτε σας αρπάσω, και ουδείς ο λυτρώσων.
\par 23 Ο προσφέρων θυσίαν αινέσεως, ούτος με δοξάζει· και εις τον ευθετούντα την οδόν αυτού θέλω δείξει την σωτηρίαν του Θεού.

\chapter{51}

\par «Εις τον πρώτον μουσικόν. Ψαλμός του Δαβίδ, ότε ήλθε Νάθαν ο προφήτης προς αυτόν, αφού εισήλθε προς την Βηθσαβεέ.» Ελέησόν με, ω Θεέ, κατά το έλεός σου· κατά το πλήθος των οικτιρμών σου εξάλειψον τα ανομήματά μου.
\par 2 Πλύνόν με μάλλον και μάλλον από της ανομίας μου και από της αμαρτίας μου καθάρισόν με.
\par 3 Διότι τα ανομήματά μου εγώ γνωρίζω, και η αμαρτία μου ενώπιόν μου είναι διαπαντός.
\par 4 Εις σε, εις σε μόνον ήμαρτον και το πονηρόν ενώπιόν σου έπραξα· διά να δικαιωθής εν τοις λόγοις σου και να ήσαι άμεμπτος εις τας κρίσεις σου.
\par 5 Ιδού, συνελήφθην εν ανομία, και εν αμαρτία με εγέννησεν μήτηρ μου.
\par 6 Ιδού, ηγάπησας αλήθειαν εν τη καρδία, και εις τα ενδόμυχα θέλεις με διδάξει σοφίαν.
\par 7 Ράντισόν με με ύσσωπον, και θέλω είσθαι καθαρός· πλύνόν με, και θέλω είσθαι λευκότερος χιόνος.
\par 8 Κάμε με να ακούσω αγαλλίασιν και ευφροσύνην, διά να ευφρανθώσι τα οστά, τα οποία συνέθλασας.
\par 9 Απόστρεψον το πρόσωπόν σου από των αμαρτιών μου και πάσας τας ανομίας μου εξάλειψον.
\par 10 Καρδίαν καθαράν κτίσον εν εμοί, Θεέ· και πνεύμα ευθές ανανέωσον εντός μου.
\par 11 Μη με απορρίψης από του προσώπου σου· και το πνεύμα το άγιόν σου μη αφαιρέσης απ' εμού.
\par 12 Απόδος μοι την αγαλλίασιν της σωτηρίας σου και με πνεύμα ηγεμονικόν στήριξόν με.
\par 13 Θέλω διδάξει εις τους παραβάτας τας οδούς σου· και αμαρτωλοί θέλουσιν επιστρέφει εις σε.
\par 14 Ελευθέρωσόν με από αιμάτων, Θεέ, Θεέ της σωτηρίας μου· η γλώσσα μου θέλει ψάλλει εν αγαλλιάσει την δικαιοσύνην σου.
\par 15 Κύριε, άνοιξον τα χείλη μου· και το στόμα μου θέλει αναγγέλλει την αίνεσίν σου.
\par 16 Διότι δεν θέλεις θυσίαν, άλλως ήθελον προσφέρει· εις ολοκαυτώματα δεν αρέσκεσαι.
\par 17 Θυσίαι του Θεού είναι πνεύμα συντετριμμένον· καρδίαν συντετριμμένην και τεταπεινωμένην, Θεέ, δεν θέλεις καταφρονήσει.
\par 18 Ευεργέτησον την Σιών διά της ευνοίας σου· οικοδόμησον τα τείχη της Ιερουσαλήμ.
\par 19 Τότε θέλεις ευαρεστηθή εις θυσίας δικαιοσύνης, εις προσφοράς και ολοκαυτώματα· τότε θέλουσι προσφέρει μόσχους επί το θυσιαστήριόν σου.

\chapter{52}

\par «Εις τον πρώτον μουσικόν, Μασχίλ του Δαβίδ, ότε ήλθε Δωήκ ο Ιδουμαίος και ανήγγειλε προς τον Σαούλ, και είπε προς αυτόν. Ήλθεν ο Δαβίδ εις την οικίαν του Αχιμέλεχ.» Τι καυχάσαι εις την κακίαν, δυνατέ; το έλεος του Θεού διαμένει εις τον αιώνα.
\par 2 Η γλώσσα σου μελετά κακίας· ως ξυράφιον ηκονημένον εργάζεται δόλον.
\par 3 Ηγάπησας το κακόν μάλλον παρά το αγαθόν, το ψεύδος παρά να λαλής δικαιοσύνην. Διάψαλμα.
\par 4 Ηγάπησας πάντας τους λόγους του αφανισμού, γλώσσαν δολίαν.
\par 5 Διά τούτο ο Θεός θέλει σε εξολοθρεύσει διαπαντός· θέλει σε αποσπάσει και σε μετατοπίσει εκ της σκηνής σου, και θέλει σε εκριζώσει εκ γης ζώντων. Διάψαλμα.
\par 6 Και οι δίκαιοι θέλουσιν ιδεί και φοβηθή· και θέλουσι γελάσει επ' αυτόν λέγοντες,
\par 7 Ιδού, ο άνθρωπος, όστις δεν έθεσε τον Θεόν δύναμιν αυτού. αλλ' ήλπισεν επί το πλήθος του πλούτου αυτού και επεστηρίζετο επί την πονηρίαν αυτού.
\par 8 Εγώ δε θέλω είσθαι ως ελαία ακμάζουσα εν τω οίκω του Θεού· ελπίζω επί το έλεος του Θεού εις τον αιώνα του αιώνος.
\par 9 Θέλω σε δοξολογεί πάντοτε, διότι έκαμες ούτω· και θέλω ελπίζει επί το όνομά σου, διότι είναι αγαθόν έμπροσθεν των οσίων σου.

\chapter{53}

\par «Εις τον πρώτον μουσικόν, επί Μαχαλάθ· Μασχίλ του Δαβίδ.» Είπεν ο άφρων εν τη καρδία αυτού, δεν υπάρχει Θεός. Διεφθάρησαν και έγειναν βδελυροί διά την ανομίαν· δεν υπάρχει πράττων αγαθόν.
\par 2 Ο Θεός εξ ουρανού διέκυψεν επί τους υιούς των ανθρώπων, διά να ίδη εάν ήναι τις έχων σύνεσιν, εκζητών τον Θεόν.
\par 3 Πάντες εξέκλιναν· ομού εξηχρειώθησαν· δεν υπάρχει πράττων αγαθόν, δεν υπάρχει ουδέ εις.
\par 4 Δεν έχουσι γνώσιν οι εργαζόμενοι την ανομίαν, οι κατατρώγοντες τον λαόν μου ως βρώσιν άρτου; τον Θεόν δεν επεκαλέσθησαν.
\par 5 Εκεί εφοβήθησαν φόβον, όπου δεν ήτο φόβος, διότι ο Θεός διεσκόρπισε τα οστά των στρατοπεδευόντων κατά σού· κατήσχυνας αυτούς, διότι ο Θεός κατεφρόνησεν αυτούς.
\par 6 Τις θέλει δώσει εκ Σιών την σωτηρίαν του Ισραήλ; όταν ο Θεός επιστρέψη τον λαόν αυτού από της αιχμαλωσίας, θέλει αγάλλεσθαι ο Ιακώβ, θέλει ευφραίνεσθαι ο Ισραήλ.

\chapter{54}

\par «Εις τον πρώτον μουσικόν, επί Νεγινώθ· Μασχίλ του Δαβίδ, ότε ήλθον οι Ζιφαίοι και είπον προς τον Σαούλ, Δεν είναι κεκρυμμένος ο Δαβίδ παρ' ημίν;» Θεέ, σώσον με εν τω ονόματί σου και εν τη δυνάμει σου κρίνον με.
\par 2 Θεέ, άκουσον της προσευχής μου· ακροάσθητι των λόγων του στόματός μου.
\par 3 Διότι ξένοι ηγέρθησαν κατ' εμού, και καταδυνάσται ζητούσι την ψυχήν μου· δεν έθεσαν τον Θεόν ενώπιον αυτών. Διάψαλμα.
\par 4 Ιδού, ο Θεός με βοηθεί· ο Κύριος είναι μετά των υποστηριζόντων την ψυχήν μου.
\par 5 Θέλει στρέψει το κακόν επί τους εχθρούς μου· εξολόθρευσον αυτούς εν τη αληθεία σου.
\par 6 Αυτοπροαιρέτως θέλω θυσιάσει εις σέ· θέλω δοξολογεί το όνομά σου, Κύριε, διότι είναι αγαθόν.
\par 7 Διότι εκ πάσης στενοχωρίας με ελύτρωσε, και ο οφθαλμός μου είδε την εκδίκησιν επί τους εχθρούς μου.

\chapter{55}

\par «Εις τον πρώτον μουσικόν, επί Νεγινώθ· Μασχίλ του Δαβίδ.» Δος ακρόασιν, Θεέ, εις την προσευχήν μου, και μη αποσυρθής από της δεήσεώς μου.
\par 2 Πρόσεξον εις εμέ και εισάκουσόν μου· λυπούμαι εν τη μελέτη μου και ταράττομαι,
\par 3 από της φωνής του εχθρού, από της καταθλίψεως του ασεβούς· διότι ρίπτουσιν επ' εμέ ανομίαν και μετ' οργής με μισούσιν.
\par 4 Η καρδία μου καταθλίβεται εντός μου, και φόβος θανάτου έπεσεν επ' εμέ.
\par 5 Φόβος και τρόμος ήλθεν επ' εμέ, και φρίκη με εκάλυψε.
\par 6 Και είπα, Τις να μοι έδιδε πτέρυγας ως περιστεράς· ήθελον πετάξει και αναπαυθή.
\par 7 Ιδού, ήθελον απομακρυνθή φεύγων, ήθελον διατρίβει εν τη ερήμω. Διάψαλμα.
\par 8 Ήθελον ταχύνει την φυγήν μου από της ορμής του ανέμου, από της θυέλλης.
\par 9 Καταπόντισον αυτούς, Κύριε· διαίρεσον τας γλώσσας αυτών· διότι είδον καταδυναστείαν και έριδα εν τη πόλει.
\par 10 Ημέραν και νύκτα περικυκλούσιν αυτήν περί τα τείχη αυτής· και ανομία και ύβρις είναι εν τω μέσω αυτής·
\par 11 πονηρία εν τω μέσω αυτής· και απάτη και δόλος δεν λείπουσιν από των πλατειών αυτής.
\par 12 Επειδή δεν με ωνείδισεν εχθρός, το οποίον ήθελον υποφέρει· δεν ηγέρθη επ' εμέ ο μισών με· τότε ήθελον κρυφθή απ' αυτού·
\par 13 Αλλά συ, άνθρωπε ομόψυχε, οδηγέ μου και γνωστέ μου·
\par 14 οίτινες συνωμιλούμεν μετά γλυκύτητος, συνεπορευόμεθα εις τον οίκον του Θεού.
\par 15 Ας έλθη θάνατος επ' αυτούς· ας καταβώσι ζώντες εις τον άδην· διότι μεταξύ αυτών, εν ταις κατοικίαις αυτών, είναι κακίαι.
\par 16 Εγώ προς τον Θεόν θέλω κράζει, και ο Κύριος θέλει με σώσει.
\par 17 Εσπέρας και πρωΐ και μεσημβρίαν θέλω παρακαλεί και φωνάζει· και θέλει ακούσει της φωνής μου.
\par 18 Θέλει λυτρώσει εν ειρήνη την ψυχήν μου από της μάχης της κατ' εμού· διότι πολλοί είναι εναντίον μου.
\par 19 Ο Θεός, ο υπάρχων προ των αιώνων, θέλει εισακούσει και θέλει ταπεινώσει αυτούς· Διάψαλμα· διότι δεν μεταβάλλουσι τρόπον ουδέ φοβούνται τον Θεόν.
\par 20 Έκαστος εκτείνει τας χείρας αυτού επί τους ειρηνεύοντας μετ' αυτού· αθετεί την συνθήκην αυτού.
\par 21 Το στόμα αυτού είναι απαλώτερον βουτύρου, αλλ' εν τη καρδία αυτού είναι πόλεμος· τα λόγια αυτού είναι μαλακώτερα ελαίου, πλην είναι ξίφη γυμνά.
\par 22 Επίρριψον επί τον Κύριον το φορτίον σου, και αυτός θέλει σε ανακουφίσει· δεν θέλει ποτέ συγχωρήσει να σαλευθή ο δίκαιος.
\par 23 Αλλά συ, Θεέ, θέλεις καταβιβάσει αυτούς εις φρέαρ απωλείας· άνδρες αιμάτων και δολιότητος δεν θέλουσι φθάσει εις το ήμισυ των ημερών αυτών· αλλ' εγώ θέλω ελπίζει επί σε.

\chapter{56}

\par «Εις τον πρώτον μουσικόν, επί Ιωνάθ-ελέμ-ρεχοκίμ, Μικτάμ του Δαβίδ, οπότε εκράτησαν αυτόν οι Φιλισταίοι εν Γαθ.» Ελέησόν με, ω Θεέ, διότι άνθρωπος χάσκει να με καταπίη· όλην την ημέραν πολεμών με καταθλίβει.
\par 2 Οι εχθροί μου χάσκουσιν όλην την ημέραν να με καταπίωσι· διότι πολλοί είναι, Ύψιστε, οι πολεμούντές με.
\par 3 Καθ' ην ημέραν φοβηθώ, επί σε θέλω ελπίζει·
\par 4 εν τω Θεώ θέλω αινέσει τον λόγον αυτού· επί τον Θεόν ήλπισα· δεν θέλω φοβηθή· τι να μοι κάμη σαρξ;
\par 5 Καθ' εκάστην μεταπλάττουσι τα λόγια μου· πάντες οι διαλογισμοί αυτών είναι κατ' εμού εις κακόν.
\par 6 Συνάγονται, κρύπτονται, παραφυλάττουσι τα βήματά μου, πως να πιάσωσι την ψυχήν μου.
\par 7 Θέλουσι λυτρωθή διά της ανομίας; Θεέ, εν τη οργή σου κατακρήμνισον τους λαούς.
\par 8 Συ αριθμείς τας αποπλανήσεις μου· θες τα δάκρυά μου εις την φιάλην σου· δεν είναι ταύτα εν τω βιβλίω σου;
\par 9 Τότε θέλουσιν επιστρέψει οι εχθροί μου εις τα οπίσω, καθ' ην ημέραν σε επικαλεσθώ· εξεύρω τούτο, διότι ο Θεός είναι υπέρ εμού.
\par 10 Εν τω Θεώ θέλω αινέσει τον λόγον αυτού· εν τω Κυρίω θέλω αινέσει τον λόγον αυτού.
\par 11 Επί τον Θεόν ελπίζω· δεν θέλω φοβηθή· τι να μοι κάμη άνθρωπος;
\par 12 Επάνω μου, Θεέ, είναι αι προς σε ευχαί μου· θέλω σοι αποδίδει δοξολογίας.
\par 13 Διότι ελύτρωσας την ψυχήν μου εκ θανάτου, ουχί και τους πόδας μου εξ ολισθήματος, διά να περιπατώ ενώπιον του Θεού εν τω φωτί των ζώντων;

\chapter{57}

\par «Εις τον πρώτον μουσικόν, επί Αλτασχέθ, Μικτάμ του Δαβίδ, ότε έφευγεν από προσώπου του Σαούλ εις το σπήλαιον.» Ελέησόν με, ω Θεέ, ελέησόν με· διότι επί σε πέποιθεν η ψυχή μου, και επί την σκιάν των πτερύγων σου θέλω ελπίζει, εωσού παρέλθωσιν αι συμφοραί.
\par 2 Θέλω κράζει προς τον Θεόν τον Ύψιστον, προς τον Θεόν τον ευοδούντα τα πάντα δι' εμέ.
\par 3 Θέλει εξαποστείλει εξ ουρανού και θέλει με σώσει· θέλει καταστήσει όνειδος τον χάσκοντα να με καταπίη· Διάψαλμα· ο Θεός θέλει εξαποστείλει το έλεος αυτού και την αλήθειαν αυτού.
\par 4 Η ψυχή μου είναι μεταξύ λεόντων· κοίτομαι μεταξύ φλογερών ανθρώπων, των οποίων οι οδόντες είναι λόγχαι και βέλη και η γλώσσα αυτών ξίφος οξύ.
\par 5 Υψώθητι, Θεέ, επί τους ουρανούς· η δόξα σου ας ήναι εφ' όλην την γην.
\par 6 Παγίδα ητοίμασαν εις τα βήματά μου· η ψυχή μου εκινδύνευε να πέση· έσκαψαν έμπροσθέν μου λάκκον, ενέπεσαν εις αυτόν. Διάψαλμα.
\par 7 Ετοίμη είναι η καρδία μου, Θεέ, ετοίμη είναι η καρδία μου· θέλω ψάλλει και ψαλμωδεί.
\par 8 Εξεγέρθητι, δόξα μου· εξεγέρθητι, ψαλτήριον και κιθάρα· θέλω εξεγερθή το πρωΐ.
\par 9 Θέλω σε επαινέσει, Κύριε, μεταξύ λαών· θέλω ψαλμωδεί εις σε μεταξύ εθνών.
\par 10 Διότι εμεγαλύνθη έως των ουρανών το έλεός σου, και έως των νεφελών η αλήθειά σου.
\par 11 Υψώθητι, Θεέ, επί τους ουρανούς· η δόξα σου ας ήναι εφ' όλην την γην.

\chapter{58}

\par «Εις τον πρώτον μουσικόν, επί Αλ-τασχέθ, Μικτάμ του Δαβίδ.» Αληθώς άρα λαλείτε δικαιοσύνην; κρίνετε μετ' ευθύτητος, υιοί των ανθρώπων;
\par 2 Μάλιστα εν τη καρδία εργάζεσθε αδικίας· διαμοιράζετε την αδικίαν των χειρών σας εν τη γη.
\par 3 Απεξενώθησαν οι ασεβείς εκ μήτρας· επλανήθησαν από κοιλίας οι λαλούντες ψεύδος.
\par 4 Έχουσι φαρμάκιον ως το φαρμάκιον του όφεως· είναι όμοιοι με την κωφήν ασπίδα, ήτις φράττει τα ώτα αυτής·
\par 5 ήτις δεν θέλει να ακούση την φωνήν των γοήτων, των γοητευόντων τόσον επιδεξίως.
\par 6 Θεέ, σύντριψον αυτών τους οδόντας εν τω στόματι αυτών· Κύριε, κατάθραυσον τους κυνόδοντας των λεόντων.
\par 7 Ας διαλυθώσιν ως ύδωρ και ας ρεύσωσι· θέλει εκπέμψει τα βέλη αυτού, εωσού εξολοθρευθώσιν.
\par 8 Ως κοχλίας διαλυόμενος ας παρέλθωσιν· ως εξάμβλωμα γυναικός ας μη ίδωσι τον ήλιον.
\par 9 Πριν αυξηθώσιν αι άκανθαί σας, ώστε να γείνωσι ράμνοι, ζώντας ως εν οργή, θέλει αρπάσει αυτούς εν ανεμοστροβίλω.
\par 10 Ο δίκαιος θέλει ευφρανθή, όταν ίδη την εκδίκησιν· τους πόδας αυτού θέλει νίψει εν τω αίματι του ασεβούς.
\par 11 Και έκαστος θέλει λέγει, Επ' αληθείας είναι καρπός διά τον δίκαιον· επ' αληθείας είναι Θεός, κρίνων επί της γης.

\chapter{59}

\par «Εις τον πρώτον μουσικόν, επί Αλ-τασχέθ, Μικτάμ του Δαβίδ, ότε έστειλε ο Σαούλ, και παρεφύλαττον την οικίαν αυτού διά να θανατώσωσιν αυτόν.» Ελευθέρωσόν με εκ των εχθρών μου, Θεέ μου· υπεράσπισόν με από των επανισταμένων επ' εμέ.
\par 2 Ελευθέρωσόν με από των εργαζομένων την ανομίαν και σώσον με από ανδρών αιμάτων.
\par 3 Διότι, ιδού, ενεδρεύουσι την ψυχήν μου· δυνατοί συνήχθησαν κατ' εμού· ουχί, Κύριε, διά ανομίαν μου ουδέ διά αμαρτίαν μου·
\par 4 χωρίς να υπάρχη εν εμοί ανομία, τρέχουσι και ετοιμάζονται. Εξεγέρθητι εις συνάντησίν μου και ιδέ.
\par 5 Συ λοιπόν, Κύριε ο Θεός των δυνάμεων, ο Θεός του Ισραήλ, εξύπνησον διά να επισκεφθής πάντα τα έθνη. Μη ελεήσης μηδένα εκ των δολίων παραβατών. Διάψαλμα.
\par 6 Επιστρέφουσι το εσπέρας· υλακτούσιν ως κύνες και κυκλούσι την πόλιν.
\par 7 Ιδού, αυτοί εκχέουσι λόγους διά του στόματος αυτών· ρομφαίαι είναι εις τα χείλη αυτών· επειδή λέγουσι, Τις ακούει;
\par 8 Αλλά συ, Κύριε, θέλεις γελάσει επ' αυτούς· θέλεις μυκτηρίσει πάντα τα έθνη.
\par 9 Εν τη δυνάμει αυτών επί σε θέλω ελπίζει· διότι συ, Θεέ, είσαι το προπύργιόν μου.
\par 10 Ο Θεός του ελέους μου θέλει με προφθάσει· ο Θεός θέλει με κάμει να ίδω την εκδίκησιν επί τους παραφυλάττοντάς με.
\par 11 Μη φονεύσης αυτούς, μήποτε λησμονήση αυτό ο λαός μου· διασκόρπισον αυτούς εν τη δυνάμει σου και ταπείνωσον αυτούς, Κύριε, η ασπίς ημών.
\par 12 Διά την αμαρτίαν του στόματος αυτών, διά τους λόγους των χειλέων αυτών, ας πιασθώσιν εν τη υπερηφανία αυτών· και διά την κατάραν και το ψεύδος, τα οποία λαλούσι.
\par 13 Κατάστρεψον αυτούς, εν οργή κατάστρεψον αυτούς, ώστε να μη υπάρχωσι· και ας γνωρίσωσιν ότι ο Θεός δεσπόζει εν Ιακώβ, έως των περάτων της γης. Διάψαλμα.
\par 14 Ας επιστρέφωσι λοιπόν το εσπέρας, ας υλακτώσιν ως κύνες και ας περικυκλώσι την πόλιν.
\par 15 Ας περιπλανώνται διά τροφήν· και αν δεν χορτασθώσιν, ας γογγύζωσιν.
\par 16 Εγώ δε θέλω ψάλλει την δύναμίν σου, και το πρωΐ θέλω υμνολογεί εν αγαλλιάσει το έλεός σου· διότι έγεινες προπύργιόν μου και καταφύγιον εν τη ημέρα της θλίψεώς μου.
\par 17 Ω δύναμίς μου, σε θέλω ψαλμωδεί· διότι συ, Θεέ, είσαι το προπύργιόν μου, ο Θεός του ελέους μου.

\chapter{60}

\par «Εις τον πρώτον μουσικόν, επί Σουσάν-εδούθ, Μικτάμ του Δαβίδ προς διδασκαλίαν, οπότε επολέμησε την Συρίαν της Μεσοποταμίας και την Συρίαν Σωβά, ο δε Ιωάβ επέστρεψε και επάταξε του Εδώμ εν τη κοιλάδι του άλατος δώδεκα χιλιάδας.» Θεέ, απέρριψας ημάς· διεσκόρπισας ημάς· ωργίσθης· επίστρεψον εις ημάς.
\par 2 Έσεισας την γήν· διέσχισας αυτήν· ίασαι τα συντρίμματα αυτής, διότι σαλεύεται.
\par 3 Έδειξας εις τον λαόν σου πράγματα σκληρά· επότισας ημάς οίνον παραφροσύνης.
\par 4 Έδωκας εις τους φοβουμένους σε σημαίαν, διά να υψόνηται υπέρ της αληθείας. Διάψαλμα.
\par 5 Διά να ελευθερόνωνται οι αγαπητοί σου, σώσον διά της δεξιάς σου και επάκουσόν μου.
\par 6 Ο Θεός ελάλησεν εν τω αγιαστηρίω αυτού· θέλω χαίρει· θέλω μοιράσει την Συχέμ και την κοιλάδα Σοκχώθ θέλω διαμετρήσει.
\par 7 Εμού είναι ο Γαλαάδ και εμού ο Μανασσής· ο μεν Εφραΐμ είναι η δύναμις της κεφαλής μου· ο δε Ιούδας ο νομοθέτης μου·
\par 8 Ο Μωάβ είναι η λεκάνη του νιψίματός μου· επί τον Εδώμ θέλω ρίψει το υπόδημά μου· αλάλαξον επ' εμοί, Παλαιστίνη.
\par 9 Τις θέλει με φέρει εις την περιτετειχισμένην πόλιν; τις θέλει με οδηγήσει έως Εδώμ;
\par 10 Ουχί συ, Θεέ, ο απορρίψας ημάς; και δεν θέλεις εξέλθει, Θεέ, μετά των στρατευμάτων ημών;
\par 11 Βοήθησον ημάς από της θλίψεως· διότι ματαία είναι η παρά των ανθρώπων σωτηρία.
\par 12 Διά του Θεού θέλομεν κάμει ανδραγαθίας, και αυτός θέλει καταπατήσει τους εχθρούς ημών.

\chapter{61}

\par «Εις τον πρώτον μουσικόν, επί Νεγινώθ. Ψαλμός του Δαβίδ.» Εισάκουσον, Θεέ, της κραυγής μου· πρόσεξον εις την προσευχήν μου.
\par 2 Από των περάτων της γης προς σε θέλω κράζει, όταν λιποθυμή η καρδία μου. Οδήγησόν με εις την πέτραν, ήτις είναι παραπολύ υψηλή δι' εμέ.
\par 3 Διότι συ έγεινες καταφυγή μου, πύργος ισχυρός έμπροσθεν του εχθρού.
\par 4 Εν τη σκηνή σου θέλω παροικεί διαπαντός· θέλω καταφύγει υπό την σκέπην των πτερύγων σου. Διάψαλμα.
\par 5 Διότι συ, Θεέ, εισήκουσας των ευχών μου· έδωκάς μοι την κληρονομίαν των φοβουμένων το όνομά σου.
\par 6 Θέλεις προσθέσει ημέρας εις τας ημέρας του βασιλέως· τα έτη αυτού ας ήναι εις γενεάν και γενεάν.
\par 7 Θέλει διαμένει εις τον αιώνα ενώπιον του Θεού· κάμε να διαφυλάττωσιν αυτόν το έλεος και η αλήθεια.
\par 8 Ούτω θέλω ψαλμωδεί διαπαντός το όνομά σου, διά να εκπληρώ τας ευχάς μου καθ' ημέραν.

\chapter{62}

\par «Εις τον πρώτον μουσικόν, διά Ιεδουθούν. Ψαλμός του Δαβίδ.» Επί τον Θεόν βεβαίως αναπαύεται η ψυχή μου· εξ αυτού πηγάζει η σωτηρία μου.
\par 2 Αυτός βεβαίως είναι πέτρα μου και σωτηρία μου· προπύργιον μου· δεν θέλω σαλευθή πολύ.
\par 3 Έως πότε θέλετε επιβουλεύεσθαι εναντίον ανθρώπου; σεις πάντες θέλετε φονευθή· είσθε ως τοίχος κεκλιμένος και φραγμός ετοιμόρροπος.
\par 4 Δεν συμβουλεύονται παρά να ρίψωσι αυτόν από του ύψους αυτού· αγαπώσι το ψεύδος· διά μεν του στόματος αυτών ευλογούσι, διά δε της καρδίας αυτών καταρώνται. Διάψαλμα.
\par 5 Αλλά συ, ω ψυχή μου, επί τον Θεόν αναπαύου, διότι εξ αυτού κρέμαται η ελπίς μου.
\par 6 Αυτός βεβαίως είναι πέτρα μου και σωτηρία μου· προπύργιόν μου· δεν θέλω σαλευθή.
\par 7 Εν τω Θεώ είναι η σωτηρία μου και η δόξα μου· η πέτρα της δυνάμεώς μου, το καταφύγιόν μου, είναι εν τω Θεώ.
\par 8 Ελπίζετε επ' αυτόν εν παντί καιρώ· ανοίγετε, λαοί, ενώπιον αυτού τας καρδίας σας· ο Θεός είναι καταφύγιον εις ημάς. Διάψαλμα.
\par 9 Οι κοινοί άνθρωποι βεβαίως είναι ματαιότης, οι άρχοντες ψεύδος· εν τη πλάστιγγι πάντες ομού είναι ελαφρότεροι αυτής της ματαιότητος.
\par 10 Μη ελπίζετε επί αδικίαν και επί αρπαγήν μη ματαιόνεσθε· πλούτος εάν ρέη, μη προσηλόνετε την καρδίαν σας.
\par 11 Άπαξ ελάλησεν ο Θεός, δις ήκουσα τούτο, ότι η δύναμις είναι του Θεού·
\par 12 και σου είναι, Κύριε, το έλεος· Διότι συ θέλεις αποδώσει εις έκαστον κατά τα έργα αυτού.

\chapter{63}

\par «Ψαλμός του Δαβίδ, ότε ευρίσκετο εν τη ερήμω Ιούδα.» Θεέ, συ είσαι ο Θεός μου· από πρωΐας σε ζητώ· σε διψά η ψυχή μου, σε ποθεί η σαρξ μου, εν γη ερήμω, ξηρά και ανύδρω·
\par 2 διά να βλέπω την δύναμίν σου και την δόξαν σου, καθώς σε είδον εν τω αγιαστηρίω.
\par 3 Διότι το έλεός σου είναι καλήτερον παρά την ζωήν· τα χείλη μου θέλουσι σε επαινεί.
\par 4 Ούτω θέλω σε ευλογεί εν τη ζωή μου· εν τω ονόματί σου θέλω υψόνει τας χείρας μου.
\par 5 Ως από πάχους και μυελού θέλει χορτασθή η ψυχή μου και διά χειλέων αγαλλιάσεως θέλει υμνεί το στόμα μου,
\par 6 Όταν σε ενθυμώμαι επί της στρωμνής μου, εις σε μελετώ εν ταις φυλακαίς της νυκτός.
\par 7 Επειδή εστάθης βοήθειά μου· διά τούτο υπό την σκιάν των πτερύγων σου θέλω χαίρει.
\par 8 Προσεκολλήθη η ψυχή μου κατόπιν σου· η δεξιά σου με υποστηρίζει.
\par 9 Οι δε ζητούντες την ψυχήν μου, διά να εξολοθρεύσωσιν αυτήν, θέλουσιν εμβή εις τα κατώτατα μέρη της γής·
\par 10 θέλουσι πέσει διά ρομφαίας· θέλουσιν είσθαι μερίς αλωπέκων.
\par 11 Ο δε βασιλεύς θέλει ευφρανθή επί τον Θεόν· θέλει δοξασθή πας ο ομνύων εις αυτόν· διότι θέλει φραχθή το στόμα των λαλούντων ψεύδος.

\chapter{64}

\par «Εις τον πρώτον μουσικόν. Ψαλμός του Δαβίδ.» Άκουσον, Θεέ, της φωνής μου εν τη δεήσει μου· από του φόβου του εχθρού φύλαξον την ζωήν μου.
\par 2 Σκέπασόν με από συμβουλίου πονηρών, από φρυάγματος εργαζομένων ανομίαν·
\par 3 οίτινες ακονώσιν ως ρομφαίαν την γλώσσαν αυτών· ετοιμάζουσιν ως βέλη λόγους πικρούς,
\par 4 διά να τοξεύωσι κρυφίως τον άμεμπτον· εξαίφνης τοξεύουσιν αυτόν και δεν φοβούνται.
\par 5 Στερεούνται επί πονηρού πράγματος· μελετώσι να κρύπτωσι παγίδας, λέγοντες, Τις θέλει ιδεί αυτούς;
\par 6 Ανιχνεύουσιν ανομίας· απέκαμον ανιχνεύοντες επιμελώς· εκάστου δε τα εντός και η καρδία είναι βυθός.
\par 7 Αλλ' ο Θεός θέλει τοξεύσει αυτούς· από αιφνιδίου βέλους θέλουσιν είσθαι αι πληγαί αυτών.
\par 8 Και οι λόγοι της γλώσσης αυτών θέλουσι πέσει επ' αυτού· θέλουσι φεύγει πάντες οι βλέποντες αυτούς.
\par 9 Και θέλει φοβηθή πας άνθρωπος, και θέλουσι διηγηθή το έργον του Θεού και εννοήσει τας εργασίας αυτού.
\par 10 Ο δίκαιος θέλει ευφρανθή εις τον Κύριον και θέλει ελπίζει επ' αυτόν· και θέλουσι καυχάσθαι πάντες οι ευθείς την καρδίαν.

\chapter{65}

\par «Εις τον πρώτον μουσικόν. Ψαλμός ωδής του Δαβίδ.» Σε προσμένει ύμνος, Θεέ, εν Σιών· και εις σε θέλει αποδοθή η ευχή.
\par 2 Ω ακούων προσευχήν, εις σε θέλει έρχεσθαι πάσα σαρξ.
\par 3 Λόγοι ανομίας υπερίσχυσαν κατ' εμού· συ θέλεις καθαρίσει τας παραβάσεις ημών.
\par 4 Μακάριος εκείνος, τον οποίον εξέλεξας και προσέλαβες, διά να κατοική εν ταις αυλαίς σου· θέλομεν χορτασθή από των αγαθών του οίκου σου, του αγίου ναού σου.
\par 5 Διά τρομερών πραγμάτων μετά δικαιοσύνης θέλεις αποκρίνεσθαι προς ημάς, Θεέ της σωτηρίας ημών, η ελπίς πάντων των περάτων της γης, και των μακράν εν θαλάσση·
\par 6 ο στερεόνων διά της δυνάμεώς σου τα όρη, ο περιεζωσμένος ισχύν·
\par 7 ο κατασιγάζων τον ήχον της θαλάσσης, τον ήχον των κυμάτων αυτής και τον θόρυβον των λαών.
\par 8 και αυτοί οι κατοικούντες τα πέρατα φοβούνται τα σημεία σου· χαροποιείς τας αρχάς της αυγής και της εσπέρας.
\par 9 Επισκέπτεσαι την γην και ποτίζεις αυτήν· υπερπλουτίζεις αυτήν· ο ποταμός του Θεού είναι πλήρης υδάτων· ετοιμάζεις τον σίτον αυτών, επειδή ούτω διέταξας.
\par 10 Τα αυλάκια αυτής ποτίζεις· εξομαλίζεις τους βώλους αυτής· απαλύνεις αυτήν διά σταλακτής βροχής· ευλογείς τα βλαστήματα αυτής.
\par 11 Στεφανόνεις το έτος με τα αγαθά σου· και τα ίχνη σου σταλάζουσι πάχος.
\par 12 Σταλάζουσιν αι βοσκαί της ερήμου και οι λόφοι περιζώνονται χαράν.
\par 13 Αι πεδιάδες είναι ενδεδυμέναι ποίμνια και αι κοιλάδες εσκεπασμέναι υπό σίτου· αλαλάζουσι και έτι υμνολογούσι.

\chapter{66}

\par «Εις τον πρώτον μουσικόν. Ωιδή ψαλμού.» Αλαλάξατε εις τον Θεόν, πάσα η γη.
\par 2 Ψάλατε την δόξαν του ονόματος αυτού· κάμετε ένδοξον τον ύμνον αυτού.
\par 3 Είπατε προς τον Θεόν, Πόσον είναι φοβερά τα έργα σου διά το μέγεθος της δυνάμεώς σου, υποκρίνονται υποταγήν εις σε οι εχθροί σου.
\par 4 Πάσα η γη θέλει σε προσκυνεί και ψαλμωδεί εις σέ· θέλουσι ψαλμωδεί το όνομά σου. Διάψαλμα.
\par 5 Έλθετε και ιδέτε τα έργα του Θεού· είναι φοβερός εις τας πράξεις προς τους υιούς των ανθρώπων.
\par 6 Μετέβαλε την θάλασσαν εις ξηράν· πεζοί διέβησαν διά του ποταμού· εκεί ευφράνθημεν εις αυτόν.
\par 7 Διά της δυνάμεως αυτού δεσπόζει εις τον αιώνα· οι οφθαλμοί αυτού επιβλέπουσιν επί τα έθνη· οι αποστάται ας μη υψόνωσιν εαυτούς. Διάψαλμα.
\par 8 Ευλογείτε, λαοί, τον Θεόν ημών, και κάμετε να ακουσθή η φωνή της αινέσεως αυτού·
\par 9 όστις διαφυλάττει εν ζωή την ψυχήν ημών και δεν αφίνει να κλονίζωνται οι πόδες ημών.
\par 10 Διότι συ ηρεύνησας ημάς, Θεέ· εδοκίμασας ημάς, ως δοκιμάζεται το αργύριον.
\par 11 Ενέβαλες ημάς εις το δίκτυον· έθεσας βαρύ φορτίον επί τα νώτα ημών.
\par 12 Επεβίβασας ανθρώπους επί τας κεφαλάς ημών· διήλθομεν διά πυρός και ύδατος· και εξήγαγες ημάς εις αναψυχήν.
\par 13 Θέλω εισέλθει εις τον οίκόν σου με ολοκαυτώματα· θέλω σοι αποδώσει τας ευχάς μου,
\par 14 τας οποίας επρόφεραν τα χείλη μου, και ελάλησε το στόμα μου, εν τη θλίψει μου.
\par 15 Παχέα ολοκαυτώματα κριών θέλω σοι προσφέρει μετά θυμιάματος· θέλω προσφέρει βόας μετά τράγων. Διάψαλμα.
\par 16 Έλθετε, ακούσατε, πάντες οι φοβούμενοι τον Θεόν· και θέλω διηγηθή όσα έκαμεν εις την ψυχήν μου.
\par 17 Προς αυτόν εβόησα διά του στόματός μου, και υψώθη διά της γλώσσης μου.
\par 18 Εάν εθεώρουν αδικίαν εν τη καρδία μου, ο Κύριος δεν ήθελεν ακούσει·
\par 19 αλλ' ο Θεός βεβαίως εισήκουσεν· επρόσεξεν εις την φωνήν της προσευχής μου.
\par 20 Ευλογητός ο Θεός, όστις δεν απεμάκρυνε την προσευχήν μου και το έλεος αυτού απ' εμού.

\chapter{67}

\par «Εις τον πρώτον μουσικόν, επί Νεγινώθ. Ψαλμός ωδής.» Ο Θεός να σπλαγχνισθή ημάς και να ευλογήση ημάς να επιφάνη το πρόσωπόν αυτού εφ' ημάς Διάψαλμα.
\par 2 Διά να γνωρισθή εν τη γη η οδός σου, εν πάσι τοις έθνεσιν η σωτηρία σου.
\par 3 Ας σε υμνώσιν οι λαοί, Θεέ· ας σε υμνώσι πάντες οι λαοί.
\par 4 Ας ευφρανθώσι και ας αλαλάξωσι τα έθνη· διότι θέλεις κρίνει τους λαούς εν ευθύτητι και τα έθνη εν τη γη θέλεις οδηγήσει. Διάψαλμα.
\par 5 Ας σε υμνώσιν οι λαοί, Θεέ, ας σε υμνώσι πάντες οι λαοί.
\par 6 Η γη θέλει δίδει τον καρπόν αυτής· θέλει μας ευλογήσει ο Θεός, ο Θεός ημών.
\par 7 Θέλει μας ευλογήσει ο Θεός, και θέλουσι φοβηθή αυτόν πάντα τα πέρατα της γης.

\chapter{68}

\par «Εις τον πρώτον μουσικόν. Ψαλμός ωδής του Δαβίδ.» Ας εγερθή ο Θεός, και ας διασκορπισθώσιν οι εχθροί αυτού· και ας φύγωσιν απ' έμπροσθεν αυτού οι μισούντες αυτόν.
\par 2 Καθώς αφανίζεται ο καπνός, ούτως αφάνισον αυτούς· καθώς διαλύεται ο κηρός έμπροσθεν του πυρός, ούτως ας απολεσθώσιν οι ασεβείς από προσώπου του Θεού.
\par 3 Οι δε δίκαιοι ας ευφραίνωνται· ας αγάλλωνται ενώπιον του Θεού· και ας τέρπωνται εν ευφροσύνη.
\par 4 Ψάλλετε εις τον Θεόν· ψαλμωδείτε εις το όνομα αυτού· ετοιμάσατε τας οδούς εις τον επιβαίνοντα επί των ερήμων· Κύριος είναι το όνομα αυτού· και αγάλλεσθε ενώπιον αυτού.
\par 5 Πατήρ των ορφανών και κριτής των χηρών, είναι ο Θεός εν τω αγίω αυτού τόπω.
\par 6 Ο Θεός κατοικίζει εις οικογένειαν τους μεμονωμένους· εξάγει τους δεσμίους εις αφθονίαν· οι δε αποστάται κατοικούσιν εν γη ανύδρω.
\par 7 Θεέ, ότε εξήλθες έμπροσθεν του λαού σου, ότε περιεπάτεις διά της ερήμου· Διάψαλμα·
\par 8 η γη εσείσθη, και αυτοί οι ουρανοί έσταξαν από προσώπου του Θεού· το Σινά αυτό εσείσθη από προσώπου του Θεού, του Θεού του Ισραήλ.
\par 9 Θεέ, έπεμψας βροχήν άφθονον εις την κληρονομίαν σου, και εν τη αδυναμία αυτής συ ανεζωοποίησας αυτήν.
\par 10 Η συναγωγή σου κατώκησεν εν αυτή· Θεέ, έκαμες ετοιμασίαν εις τον πτωχόν διά την αγαθότητά σου.
\par 11 Ο Κύριος έδωκε λόγον· οι ευαγγελιζόμενοι ήσαν στράτευμα μέγα.
\par 12 Βασιλείς στρατευμάτων φεύγοντες έφυγον, και αι διαμένουσαι εν τη οικία εμοίραζον τα λάφυρα.
\par 13 Και αν εκοίτεσθε εν μέσω εστίας, όμως θέλετε είσθαι ως πτέρυγες περιστεράς περιηργυρωμένης, και της οποίας τα πτερά είναι περικεχρυσωμένα από κιτρίνου χρυσίου.
\par 14 Ότε ο Παντοδύναμος διεσκόρπιζε βασιλείς εν αυτή, έγεινε λευκή ως η χιών εν Σαλμών.
\par 15 Το όρος του Θεού είναι ως το όρος της Βασάν· όρος υψηλόν ως το όρος της Βασάν.
\par 16 Διά τι ζηλοτυπείτε, όρη υψηλά; τούτο είναι το όρος, εν ω ευδόκησεν ο Θεός να κατοική· ο Κύριος, ναι, εν αυτώ θέλει κατοικεί εις τον αιώνα.
\par 17 Αι άμαξαι του Θεού είναι δισμύριαι χιλιάδες χιλιάδων· ο Κύριος είναι μεταξύ αυτών ως εν Σινά, εν τω αγίω τόπω.
\par 18 Ανέβης εις ύψος· ηχμαλώτισας αιχμαλωσίαν· έλαβες χαρίσματα διά τους ανθρώπους· έτι δε και διά τους απειθείς, διά να κατοικής μεταξύ αυτών, Κύριε Θεέ.
\par 19 Ευλογητός Κύριος, όστις καθ' ημέραν επιφορτίζεις ημάς αγαθά· ο Θεός της σωτηρίας ημών. Διάψαλμα.
\par 20 Ο Θεός ημών είναι Θεός σωτηρίας· και Κυρίου του Θεού είναι η λύτρωσις από του θανάτου.
\par 21 Ο Θεός εξάπαντος θέλει συντρίψει την κεφαλήν των εχθρών αυτού· και την τετριχωμένην κορυφήν του περιπατούντος εν ταις ανομίαις αυτού.
\par 22 Ο Κύριος είπε, Θέλω επαναφέρει εκ Βασάν, θέλω επαναφέρει τον λαόν μου εκ των βαθέων της θαλάσσης·
\par 23 διά να βαφή ο πους σου εν τω αίματι των εχθρών σου και η γλώσσα των κυνών σου εξ αυτού.
\par 24 Εθεωρήθησαν τα βήματά σου, Θεέ· τα βήματα του Θεού μου, του βασιλέως μου, εν τω αγιαστηρίω.
\par 25 Προεπορεύοντο οι ψάλται· κατόπιν οι παίζοντες όργανα, εν τω μέσω νεάνιδες τυμπανίστριαι.
\par 26 Εν εκκλησίαις ευλογείτε τον Θεόν· ευλογείτε τον Κύριον, οι εκ της πηγής του Ισραήλ.
\par 27 Εκεί ήτο ο μικρός Βενιαμίν, ο αρχηγός αυτών· οι άρχοντες Ιούδα και ο λαός αυτών· οι άρχοντες Ζαβουλών και οι άρχοντες Νεφθαλί.
\par 28 Διέταξεν ο Θεός σου την δύναμίν σου· στερέωσον, Θεέ, τούτο, το οποίον ενήργησας εις ημάς.
\par 29 Διά τον ναόν σου τον εν Ιερουσαλήμ, βασιλείς θέλουσι προσφέρει εις σε δώρα.
\par 30 Επιτίμησον τα θηρία του καλαμώνος, το πλήθος των ταύρων και τους μόσχους των λαών, εωσού έκαστος προσφέρη υποταγήν με πλάκας αργυρίου· διασκόρπισον τους λαούς τους αγαπώντας πολέμους.
\par 31 Θέλουσιν ελθεί μεγιστάνες εξ Αιγύπτου· η Αιθιοπία ταχέως θέλει εκτείνει τας χείρας αυτής προς τον Θεόν.
\par 32 Αι βασιλείαι της γης, ψάλλετε εις τον Θεόν, ψαλμωδείτε εις τον Κύριον· Διάψαλμα·
\par 33 εις τον επιβαίνοντα επί τους ουρανούς των έκπαλαι ουρανών· ιδού εκπέμπει την φωνήν αυτού, φωνήν κραταιάν.
\par 34 Απόδοτε το κράτος εις τον Θεόν· η μεγαλοπρέπεια αυτού είναι επί τον Ισραήλ και η δύναμις αυτού επί τους ουρανούς.
\par 35 Φοβερός είσαι, Θεέ, εκ των αγιαστηρίων σου· ο Θεός του Ισραήλ είναι ο διδούς κράτος και δύναμιν εις τον λαόν αυτού. Ευλογητός ο Θεός.

\chapter{69}

\par «Εις τον πρώτον μουσικόν, υπό Σοσανίμ. Ψαλμός του Δαβίδ.» Σώσόν με, Θεέ, διότι εισήλθον ύδατα έως ψυχής μου.
\par 2 Εβυθίσθην εις βαθύν πηλόν, όπου δεν είναι τόπος στερεός διά να σταθώ· έφθασα εις τα βάθη των υδάτων, και το ρεύμα με κατακλύζει.
\par 3 Ητόνησα κράζων· ο λάρυγξ μου εξηράνθη· απέκαμον οι οφθαλμοί μου από του να περιμένω τον Θεόν μου.
\par 4 Οι μισούντές με αναιτίως επληθύνθησαν υπέρ τας τρίχας της κεφαλής μου· εκραταιώθησαν οι εχθροί μου οι προσπαθούντες να με αφανίσωσιν αδίκως. Τότε εγώ επέστρεψα ό,τι δεν ήρπασα.
\par 5 Θεέ, συ γνωρίζεις την αφροσύνην μου· και τα πλημμελήματά μου δεν είναι κεκρυμμένα από σου.
\par 6 Ας μη αισχυνθώσιν εξ αιτίας μου, Κύριε Θεέ των δυνάμεων, οι προσμένοντές σε· ας μη εντραπώσι δι' εμέ οι εκζητούντές σε, Θεέ του Ισραήλ.
\par 7 Διότι ένεκα σου υπέφερα ονειδισμόν· αισχύνη εκάλυψε το πρόσωπόν μου.
\par 8 Ξένος έγεινα εις τους αδελφούς μου, και αλλογενής εις τους υιούς της μητρός μου·
\par 9 Διότι ο ζήλος του οίκου σου με κατέφαγε· και οι ονειδισμοί των ονειδιζόντων σε επέπεσον επ εμέ.
\par 10 Και έκλαυσα ταλαιπωρών εν νηστεία την ψυχήν μου, αλλά τούτο έγεινεν εις όνειδός μου.
\par 11 Και έκαμα τον σάκκον ένδυμά μου και έγεινα εις αυτούς παροιμία.
\par 12 Κατ' εμού λαλούσιν οι καθήμενοι εν ταις πύλαις, και έγεινα άσμα των μεθυόντων.
\par 13 Εγώ δε προς σε κατευθύνω την προσευχήν μου, Κύριε· καιρός ευμενείας είναι· Θεέ, κατά το πλήθος του ελέους σου, επάκουσόν μου, κατά την αλήθειαν της σωτηρίας σου.
\par 14 Ελευθέρωσόν με από του πηλού, διά να μη βυθισθώ· ας ελευθερωθώ εκ των μισούντων με και εκ των βαθέων των υδάτων.
\par 15 Ας μη με κατακλύση το ρεύμα των υδάτων, μηδέ ας με καταπίη ο βυθός· και το φρέαρ ας μη κλείση το στόμα αυτού επ' εμέ.
\par 16 Εισάκουσόν μου, Κύριε, διότι αγαθόν είναι το έλεός σου· κατά το πλήθος των οικτιρμών σου επίβλεψον επ' εμέ.
\par 17 Και μη κρύψης το πρόσωπόν σου από του δούλου σου· επειδή θλίβομαι, ταχέως επάκουσόν μου.
\par 18 Πλησίασον εις την ψυχήν μου· λύτρωσον αυτήν· ένεκα των εχθρών μου λύτρωσόν με.
\par 19 Συ γνωρίζεις τον ονειδισμόν μου και την αισχύνην μου και την εντροπήν μου· ενώπιόν σου είναι πάντες οι θλίβοντές με.
\par 20 Ονειδισμός συνέτριψε την καρδίαν μου· και είμαι περίλυπος· περιέμεινα δε συλλυπούμενον, αλλά δεν υπήρξε, και παρηγορητάς, αλλά δεν εύρηκα.
\par 21 Και έδωκαν εις εμέ χολήν διά φαγητόν μου, και εις την δίψαν μου με επότισαν όξος.
\par 22 Ας γείνη η τράπεζα αυτών έμπροσθεν αυτών εις παγίδα και εις ανταπόδοσιν και εις βρόχον.
\par 23 Ας σκοτισθώσιν οι οφθαλμοί αυτών διά να μη βλέπωσι· και την ράχιν αυτών διαπαντός κύρτωσον.
\par 24 Έκχεε επ' αυτούς την οργήν σου· και ο θυμός της αγανακτήσεώς σου ας συλλάβη αυτούς.
\par 25 Ας γείνωσιν έρημα τα παλάτια αυτών· εν ταις σκηναίς αυτών ας μη ήναι ο κατοικών.
\par 26 Διότι εκείνον, τον οποίον συ επάταξας, αυτοί κατεδίωξαν· και λαλούσι περί του πόνου εκείνων, τους οποίους επλήγωσας.
\par 27 Πρόσθες ανομίαν επί την ανομίαν αυτών, και ας μη εισέλθωσιν εις την δικαιοσύνην σου.
\par 28 Ας εξαλειφθώσιν εκ βίβλου ζώντων και μετά των δικαίων ας μη καταγραφθώσιν.
\par 29 Εμέ δε, τον πτωχόν και λελυπημένον, η σωτηρία σου, Θεέ, ας με υψώση.
\par 30 Θέλω αινέσει το όνομα του Θεού εν ωδή και θέλω μεγαλύνει αυτόν εν ύμνοις.
\par 31 Τούτο βεβαίως θέλει αρέσει εις τον Κύριον, υπέρ μόσχον νέον έχοντα κέρατα και οπλάς.
\par 32 Οι ταπεινοί θέλουσιν ιδεί· θέλουσι ευφρανθή· και η καρδία υμών των εκζητούντων τον Θεόν θέλει ζήσει.
\par 33 Διότι εισακούει των πενήτων ο Κύριος και τους δεσμίους αυτού δεν καταφρονεί.
\par 34 Ας αινέσωσιν αυτόν οι ουρανοί και η γη, αι θάλασσαι και πάντα τα κινούμενα εν αυταίς.
\par 35 Διότι ο Θεός θέλει σώσει την Σιών, και θέλει οικοδομήσει τας πόλεις του Ιούδα· και θέλουσι κατοικήσει εκεί και θέλουσι κληρονομήσει αυτήν.
\par 36 Και το σπέρμα των δούλων αυτού θέλει κληρονομήσει αυτήν, και οι αγαπώντες το όνομα αυτού θέλουσι κατοικεί εν αυτή.

\chapter{70}

\par «Εις τον πρώτον μουσικόν. Ψαλμός του Δαβίδ, εις ανάμνησιν.» Θεέ, τάχυνον να με ελευθερώσης· τάχυνον, Κύριε, εις βοήθειάν μου.
\par 2 Ας αισχυνθώσι και ας εντραπώσιν οι ζητούντες την ψυχήν μου· ας στραφώσιν εις τα οπίσω και ας εντραπώσιν οι θέλοντες το κακόν μου.
\par 3 Ας στραφώσιν οπίσω προς αμοιβήν της αισχύνης αυτών οι λέγοντες, εύγε, εύγε.
\par 4 Ας αγάλλωνται και ας ευφραίνωνται εις σε πάντες οι ζητούντές σε· και οι αγαπώντες την σωτηρίαν σου ας λέγωσι διαπαντός, Μεγαλυνθήτω ο Θεός.
\par 5 Εγώ δε είμαι πτωχός και πένης· Θεέ, τάχυνον προς εμέ· συ είσαι βοήθειά μου και ελευθερωτής μου· Κύριε, μη βραδύνης.

\chapter{71}

\par Επί σε, Κύριε, ήλπισα· ας μη καταισχυνθώ ποτέ.
\par 2 Διά την δικαιοσύνην σου λύτρωσόν με και ελευθέρωσόν με· Κλίνον προς εμέ το ωτίον σου και σώσον με.
\par 3 Γενού εις εμέ τόπος οχυρός, διά να καταφεύγω πάντοτε· συ διέταξας να με σώσης, διότι πέτρα μου και φρούριόν μου είσαι.
\par 4 Θεέ μου, λύτρωσόν με εκ δυνάμεως ασεβούς, εκ χειρός παρανόμου και αδίκου.
\par 5 Διότι συ είσαι η ελπίς μου, Κύριε Θεέ· το θάρρος μου εκ νεότητός μου.
\par 6 Επί σε επεστηρίχθην εκ της κοιλίας· συ είσαι σκέπη μου εκ των σπλάγχνων της μητρός μου· εις σε θέλει είσθαι πάντοτε ο ύμνος μου.
\par 7 Ως τέρας κατεστάθην εις τους πολλούς· αλλά συ είσαι το δυνατόν καταφύγιόν μου,
\par 8 Ας εμπλησθή το στόμα μου από του ύμνου σου, από της δόξης σου, όλην την ημέραν.
\par 9 Μη με απόρρίψης εν καιρώ γήρατος· όταν εκλείπη η δύναμίς μου, μη με εγκαταλίπης.
\par 10 Διότι οι εχθροί μου λαλούσι περί εμού· και οι παραφυλάττοντες την ψυχήν μου συμβουλεύονται ομού,
\par 11 Λέγοντες, Ο Θεός εγκατέλιπεν αυτόν· καταδιώξατε και πιάσατε αυτόν, διότι δεν υπάρχει ο σώζων.
\par 12 Θεέ, μη μακρυνθής απ' εμού· Θεέ μου, τάχυνον εις βοήθειάν μου.
\par 13 Ας αισχυνθώσιν, ας εξαλειφθώσιν οι εχθροί της ψυχής μου· ας σκεπασθώσι από ονείδους και εντροπής οι ζητούντες το κακόν μου.
\par 14 Εγώ δε πάντοτε θέλω ελπίζει, και θέλω προσθέτει επί πάντας τους επαίνους σου.
\par 15 Το στόμα μου θέλει κηρύττει την δικαιοσύνην σου και την σωτηρίαν σου όλην την ημέραν· διότι δεν δύναμαι να απαριθμήσω αυτάς.
\par 16 Θέλω περιπατεί εν τη δυνάμει Κυρίου του Θεού· θέλω μνημονεύει την δικαιοσύνην σου, σου μόνου.
\par 17 Θεέ, συ με εδίδαξας εκ νεότητός μου· και μέχρι του νυν εκήρυττον τα θαυμάσιά σου.
\par 18 Μη με εγκαταλίπης μηδέ μέχρι του γήρατος και πολιάς, Θεέ, εωσού κηρύξω τον βραχίονά σου εις ταύτην την γενεάν, την δύναμίν σου εις πάντας τους μεταγενεστέρους.
\par 19 Διότι η δικαιοσύνη σου, Θεέ, είναι υπερυψωμένη· διότι έκαμες μεγαλεία Θεέ, τις όμοιός σου,
\par 20 όστις έδειξας εις εμέ θλίψεις πολλάς και ταλαιπωρίας, και πάλιν με ανεζωοποίησας και εκ των αβύσσων της γης πάλιν ανήγαγές με;
\par 21 Ηύξησας το μεγαλείόν μου και επιστρέψας με παρηγόρησας.
\par 22 Και εγώ, Θεέ μου, θέλω δοξολογεί εν τω οργάνω του ψαλτηρίου σε και την αλήθειάν σου· εις σε θέλω ψαλμωδεί εν κιθάρα, Άγιε του Ισραήλ.
\par 23 Θέλουσιν αγάλλεσθαι τα χείλη μου, όταν εις σε ψαλμωδώ· και η ψυχή μου, την οποίαν ελύτρωσας.
\par 24 Έτι δε η γλώσσα μου όλην την ημέραν θέλει μελετά την δικαιοσύνην σου· διότι ενετράπησαν, διότι κατησχύνθησαν, οι ζητούντες το κακόν μου.

\chapter{72}

\par «Ψαλμός διά τον Σολομώντα.» Θεέ, δος την κρίσιν σου εις τον βασιλέα και την δικαιοσύνην σου εις τον υιόν του βασιλέως·
\par 2 Διά να κρίνη τον λαόν σου εν δικαιοσύνη και τους πτωχούς σου εν κρίσει.
\par 3 Τα όρη θέλουσι φέρει ειρήνην εις τον λαόν και οι λόφοι δικαιοσύνην.
\par 4 Θέλει κρίνει τους πτωχούς του λαού· θέλει σώσει τους υιούς των πενήτων και συντρίψει τον καταδυναστεύοντα.
\par 5 Θέλουσι σε φοβείσθαι ενόσω διαμένει ο ήλιος και η σελήνη, εις γενεάς γενεών.
\par 6 Θέλει καταβή ως βροχή επί θερισμένον λειβάδιον· ως ρανίδες σταλάζουσαι επί την γην.
\par 7 Εν ταις ημέραις αυτού θέλει ανθεί ο δίκαιος· και αφθονία ειρήνης θέλει είσθαι εωσού μη υπάρξη η σελήνη.
\par 8 Και θέλει κατακυριεύει από θαλάσσης έως θαλάσσης και από του ποταμού έως των περάτων της γης.
\par 9 Έμπροσθεν αυτού θέλουσι γονυκλιτήσει οι κατοικούντες εν ερήμοις, και οι εχθροί αυτού θέλουσι γλείψει το χώμα.
\par 10 Οι βασιλείς της Θαρσείς και των νήσων θέλουσι προσφέρει προσφοράς· οι βασιλείς της Αραβίας και της Σεβά θέλουσι προσφέρει δώρα.
\par 11 Και θέλουσι προσκυνήσει αυτόν πάντες οι βασιλείς· πάντα τα έθνη θέλουσι δουλεύσει αυτόν.
\par 12 Διότι θέλει ελευθερώσει τον πτωχόν κράζοντα και τον πένητα και τον αβοήθητον.
\par 13 Θέλει ελεήσει τον πτωχόν και τον πένητα· και τας ψυχάς των πενήτων θέλει σώσει.
\par 14 Εκ δόλου και εξ αδικίας θέλει λυτρόνει τας ψυχάς αυτών· και πολύτιμον θέλει είσθαι το αίμα αυτών εις τους οφθαλμούς αυτού.
\par 15 Και θέλει ζη, και θέλει δοθή εις αυτόν από του χρυσίου της Αραβίας, και θέλει γίνεσθαι πάντοτε προσευχή υπέρ αυτού· όλην την ημέραν θέλουσιν ευλογεί αυτόν.
\par 16 Δράξ σίτου εάν υπάρχη εν τη γη επί των κορυφών των ορέων, ο καρπός αυτού θέλει σείεσθαι ως ο Λίβανος· και οι κάτοικοι εν τη πόλει θέλουσιν εξανθήσει ως ο χόρτος της γης.
\par 17 Το όνομα αυτού θέλει διαμένει εις τον αιώνα· το όνομα αυτού θέλει διαρκεί ενόσω διαμένει ο ήλιος· και οι άνθρωποι θέλουσιν ευλογείσθαι εν αυτώ· πάντα τα έθνη θέλουσι μακαρίζει αυτόν.
\par 18 Ευλογητός Κύριος ο Θεός, ο Θεός του Ισραήλ, όστις μόνος κάμνει θαυμάσια·
\par 19 και ευλογημένον το ένδοξον όνομα αυτού εις τον αιώνα· και ας πληρωθή από της δόξης αυτού η πάσα γη. Αμήν, και αμήν.
\par 20 Ετελείωσαν αι προσευχαί του Δαβίδ υιού του Ιεσσαί.

\chapter{73}

\par «Ψαλμός του Ασάφ.» Αγαθός τωόντι είναι ο Θεός εις τον Ισραήλ, εις τους καθαρούς την καρδίαν.
\par 2 Εμού δε, οι πόδες μου σχεδόν εκλονίσθησαν· παρ' ολίγον ωλίσθησαν τα βήματά μου.
\par 3 Διότι εζήλευσα τους μωρούς, βλέπων την ευτυχίαν των ασεβών.
\par 4 Επειδή δεν είναι λύπαι εις τον θάνατον αυτών, αλλ' η δύναμις αυτών είναι στερεά.
\par 5 Δεν είναι εν κόποις, ως οι άλλοι άνθρωποι· ουδέ μαστιγόνονται μετά των λοιπών ανθρώπων.
\par 6 διά τούτο περικυκλόνει αυτούς η υπερηφανία ως περιδέραιον· η αδικία σκεπάζει αυτούς ως ιμάτιον.
\par 7 Οι οφθαλμοί αυτών εξέχουσιν εκ του πάχους· εξεπέρασαν τας επιθυμίας της καρδίας αυτών.
\par 8 Εμπαίζουσι και λαλούσιν εν πονηρία καταδυναστείαν· λαλούσιν υπερηφάνως.
\par 9 Θέτουσιν εις τον ουρανόν το στόμα αυτών, και η γλώσσα αυτών διατρέχει την γην.
\par 10 Διά τούτο θέλει στραφή ενταύθα ο λαός αυτού· και ύδατα ποτηρίου πλήρους εκθλίβονται δι' αυτούς.
\par 11 Και λέγουσι, Πως γνωρίζει ταύτα ο Θεός; και υπάρχει γνώσις εν τω Υψίστω;
\par 12 Ιδού, ούτοι είναι ασεβείς και ευτυχούσι διαπαντός· αυξάνουσι τα πλούτη αυτών.
\par 13 Άρα, ματαίως εκαθάρισα την καρδίαν μου και ένιψα εν αθωότητι τας χείρας μου.
\par 14 Διότι εμαστιγώθην όλην την ημέραν και ετιμωρήθην πάσαν αυγήν.
\par 15 Αν είπω, Θέλω ομιλεί ούτως· ιδού, εξυβρίζω εις την γενεάν των υιών σου.
\par 16 Και εστοχάσθην να εννοήσω τούτο, πλην μ' εφάνη δύσκολον·
\par 17 εωσού εισελθών εις το αγιαστήριον του Θεού, ενόησα τα τέλη αυτών.
\par 18 Συ βεβαίως έθεσας αυτούς εις τόπους ολισθηρούς· έρριψας αυτούς εις κρημνόν.
\par 19 Πως διά μιας κατήντησαν εις ερήμωσιν Ηφανίσθησαν, απωλέσθησαν υπό αιφνιδίου ολέθρου.
\par 20 Ως όνειρον εξεγειρομένου Κύριε, όταν εγερθής, θέλεις αφανίσει την εικόνα αυτών.
\par 21 Ούτως εκαίετο η καρδία μου, και τα νεφρά μου εβασανίζοντο·
\par 22 και εγώ ήμην ανόητος και δεν εγνώριζον· κτήνος ήμην ενώπιόν σου.
\par 23 Εγώ όμως είμαι πάντοτε μετά σού· συ με επίασας από της δεξιάς μου χειρός.
\par 24 Διά της συμβουλής σου θέλεις με οδηγήσει και μετά ταύτα θέλεις με προσλάβει εν δόξη.
\par 25 Τίνα άλλον έχω εν τω ουρανώ; και επί της γης δεν θέλω άλλον παρά σε.
\par 26 Ητόνησεν η σαρξ μου και η καρδία μου· αλλ' ο Θεός είναι η δύναμις της καρδίας μου και η μερίς μου εις τον αιώνα.
\par 27 Διότι, ιδού, όσοι απομακρύνονται από σου, θέλουσιν απολεσθή· συ εξωλόθρευσας πάντας τους εκκλίνοντας από σου.
\par 28 Αλλά δι' εμέ, το να προσκολλώμαι εις τον Θεόν είναι το αγαθόν μου· έθεσα την ελπίδα μου επί Κύριον τον Θεόν, διά να κηρύττω πάντα τα έργα σου.

\chapter{74}

\par «Μασχίλ του Ασάφ.» Διά τι, Θεέ, απέρριψας ημάς διαπαντός; διά τι καπνίζει η οργή σου εναντίον των προβάτων της βοσκής σου;
\par 2 Μνήσθητι της συναγωγής σου, την οποίαν απέκτησας απ' αρχής· την ράβδον της κληρονομίας σου, την οποίαν ελύτρωσας· τούτο το όρος Σιών, εν ω κατώκησας.
\par 3 Κίνησον τα βήματά σου προς τας παντοτεινάς ερημώσεις, προς παν κακόν, το οποίον έπραξεν ο εχθρός εν τω αγιαστηρίω.
\par 4 Οι εχθροί σου βρυχώνται εν τω μέσω των συναγωγών σου· έθεσαν σημαίας τας σημαίας αυτών.
\par 5 Γνωστόν έγεινεν· ως εάν τις σηκόνων πέλεκυν καταφέρη επί πυκνά δένδρα,
\par 6 ούτω τώρα αυτοί συνέτριψαν διά μιας με πελέκεις και σφυρία, τα πελεκητά έργα αυτού.
\par 7 Κατέκαυσαν εν πυρί το αγιαστήριόν σου έως εδάφους· εβεβήλωσαν το κατοικητήριον του ονόματός σου.
\par 8 Είπον εν τη καρδία αυτών, Ας εξολοθρεύσωμεν αυτούς ομού· κατέκαυσαν πάσας τας συναγωγάς του Θεού εν τη γη.
\par 9 Τα σημεία ημών δεν βλέπομεν· δεν υπάρχει πλέον προφήτης ουδέ γνωρίζων μεταξύ ημών το έως πότε.
\par 10 Έως πότε, Θεέ, θέλει ονειδίζει ο εναντίος; θέλει βλασφημεί ο εχθρός το όνομά σου διαπαντός;
\par 11 Διά τι αποστρέφεις την χείρα σου, και την δεξιάν σου; έκβαλε αυτήν εκ μέσου του κόλπου σου και αφάνισον αυτούς.
\par 12 Αλλ' ο Θεός είναι εξ αρχής Βασιλεύς μου, εργαζόμενος σωτηρίαν εν μέσω της γης.
\par 13 Συ διεχώρισας διά της δυνάμεώς σου την θάλασσαν· συ συνέτριψας τας κεφαλάς των δρακόντων εν τοις ύδασι.
\par 14 Συ συνέτριψας τας κεφαλάς τον Λευϊάθαν· έδωκας αυτόν βρώσιν εις τον λαόν, τον κατοικούντα εν ερήμοις.
\par 15 Συ ήνοιξας πηγάς και χειμάρρους· εξήρανας ποταμούς δυνατούς.
\par 16 Σού είναι η ημέρα και σου η νύξ· συ ητοίμασας το φως και τον ήλιον.
\par 17 Συ έθεσας πάντα τα όρια της γής· συ έκαμες το θέρος και τον χειμώνα.
\par 18 Μνήσθητι τούτου, ότι ο εχθρός ωνείδισε τον Κύριον· και λαός άφρων εβλασφήμησε το όνομά σου.
\par 19 Μη παραδώσης εις τα θηρία την ψυχήν της τρυγόνος σου· την σύναξιν των πενήτων σου μη λησμονήσης διαπαντός.
\par 20 Επίβλεψον επί την διαθήκην σου· διότι επλήσθησαν οι σκοτεινοί της γης τόποι από οίκων καταδυναστείας.
\par 21 Ας μη στραφή ο ταλαίπωρος εις τα οπίσω κατησχυμμένος· ο πτωχός και ο πένης ας επαινώσι το όνομά σου.
\par 22 Ανάστα, Θεέ· δίκασον την δίκην σου· μνήσθητι του ονειδισμού, τον οποίον εις σε κάμνει ο άφρων όλην την ημέραν.
\par 23 Μη λησμονήσης την φωνήν των εχθρών σου· ο θόρυβος των επανισταμένων κατά σου αυξάνει διαπαντός.

\chapter{75}

\par «Εις τον πρώτον μουσικόν, επί Αλ-τασχέθ. Ψαλμός ωδής του Ασάφ.» Δοξολογούμέν σε, Θεέ, δοξολογούμεν, διότι πλησίον ημών είναι το όνομά σου· κηρύττονται τα θαυμάσιά σου.
\par 2 Όταν λάβω τον ωρισμένον καιρόν, εγώ θέλω κρίνει εν ευθύτητι.
\par 3 Διελύθη η γη και πάντες οι κάτοικοι αυτής· εγώ εστερέωσα τους στύλους αυτής. Διάψαλμα.
\par 4 Είπα προς τους άφρονας, μη γίνεσθε άφρονες· και προς τους ασεβείς, μη υψώνετε κέρας.
\par 5 Μη υψόνετε εις ύψος το κέρας υμών· μη λαλείτε με τράχηλον σκληρόν.
\par 6 Διότι ούτε εξ ανατολών, ούτε εκ δυσμών, ούτε εκ της ερήμου, έρχεται ύψωσις.
\par 7 Αλλ' ο Θεός είναι ο Κριτής· τούτον ταπεινόνει και εκείνον υψόνει.
\par 8 Διότι εν τη χειρί του Κυρίου είναι ποτήριον πλήρες κεράσματος οίνου ακράτου, και εκ τούτου θέλει χύσει· πλην την τρυγίαν αυτού θέλουσι στραγγίσει πάντες οι ασεβείς της γης και θέλουσι πίει.
\par 9 Εγώ δε θέλω κηρύττει διαπαντός, θέλω ψαλμωδεί εις τον Θεόν του Ιακώβ.
\par 10 Και πάντα τα κέρατα των ασεβών θέλω συντρίψει· τα δε κέρατα των δικαίων θέλουσιν υψωθή.

\chapter{76}

\par Εις τον πρώτον μουσικόν, επί Νεγινώθ. Ψαλμός ωδής του Ασάφ. Γνωστός είναι εν τη Ιουδαία ο Θεός· εν τω Ισραήλ μέγα το όνομα αυτού.
\par 2 Η δε σκηνή αυτού είναι εν Σαλήμ, και το κατοικητήριον αυτού εν Σιών.
\par 3 Εκεί συνέτριψε τα βέλη του τόξου, την ασπίδα και την ρομφαίαν και τον πόλεμον. Διάψαλμα.
\par 4 Είσαι λαμπρότερος υπέρ τα όρη των αρπακτήρων.
\par 5 Οι θρασυκάρδιοι εγυμνώθησαν· εκοιμήθησαν τον ύπνον αυτών· και ουδείς των ρωμαλέων ανδρών εύρηκε τας χείρας αυτού.
\par 6 Από επιτιμήσεώς σου, Θεέ του Ιακώβ, έπεσον εις βαθύτατον ύπνον και η άμαξα και ο ίππος.
\par 7 Συ είσαι φοβερός· και τις δύναται να σταθή έμπροσθέν σου, όταν οργισθής;
\par 8 Εξ ουρανού έκαμες να ακουσθή κρίσις· η γη εφοβήθη και ησύχασεν,
\par 9 ότε εσηκώθη εις κρίσιν ο Θεός, διά να σώση πάντας τους πράους της γης. Διάψαλμα.
\par 10 Βεβαίως ο θυμός του ανθρώπου θέλει καταντήσει εις έπαινον σου· θέλεις χαλινώσει το υπόλοιπον του θυμού.
\par 11 Κάμετε ευχάς και απόδοτε εις Κύριον τον Θεόν σας· πάντες οι κύκλω αυτού ας φέρωσι δώρα εις τον φοβερόν·
\par 12 τον αφαιρούντα το πνεύμα των αρχόντων, τον φοβερόν εις τους βασιλείς της γης.

\chapter{77}

\par «Εις τον πρώτον μουσικόν, διά Ιεδουθούν. Ψαλμός του Ασάφ.» Η φωνή μου είναι προς τον Θεόν, και εβόησα· η φωνή μου είναι προς τον Θεόν, και έδωκεν εις εμέ ακρόασιν.
\par 2 Εν ημέρα θλίψεώς μου εξεζήτησα τον Κύριον· εξέτεινον την νύκτα τας χείρας μου και δεν έπαυον· η ψυχή μου δεν ήθελε να παρηγορηθή.
\par 3 Ενεθυμήθην τον Θεόν και εταράχθην· διελογίσθην, και ωλιγοψύχησε το πνεύμά μου. Διάψαλμα.
\par 4 Εκράτησας τους οφθαλμούς μου εν αγρυπνία· εταράχθην και δεν ηδυνάμην να λαλήσω.
\par 5 Διελογίσθην τας αρχαίας ημέρας, τα έτη των αιώνων.
\par 6 Ανακαλώ εις μνήμην την ωδήν μου· την νύκτα διαλογίζομαι μετά της καρδίας μου, και το πνεύμά μου διερευνά·
\par 7 μήποτε ο Κύριος με αποβάλη αιωνίως, και δεν θέλει είσθαι ευμενής πλέον;
\par 8 ή εξέλιπε διαπαντός το έλεος αυτού; έπαυσεν ο λόγος αυτού εις γενεάν και γενεάν;
\par 9 Μήποτε ελησμόνησε να ελεή ο Θεός; μήποτε εν τη οργή αυτού θέλει κλείσει τους οικτιρμούς αυτού; Διάψαλμα.
\par 10 Τότε είπα, Αδυναμία μου είναι τούτο· αλλοιούται η δεξιά του Υψίστου;
\par 11 Θέλω μνημονεύει τα έργα του Κυρίου· ναι, θέλω μνημονεύει τα απ' αρχής θαυμάσιά σου·
\par 12 και θέλω μελετά εις πάντα τα έργα σου, και περί των πράξεών σου θέλω διαλογίζεσθαι.
\par 13 Θεέ, εν τω αγιαστηρίω είναι η οδός σου· τις Θεός μέγας, ως ο Θεός;
\par 14 Συ είσαι ο Θεός ο ποιών θαυμάσια· εφανέρωσας μεταξύ των λαών την δύναμίν σου.
\par 15 Ελύτρωσας διά του βραχίονός σου τον λαόν σου, τους υιούς Ιακώβ και Ιωσήφ. Διάψαλμα.
\par 16 Τα ύδατα σε είδον, Θεέ, τα ύδατα σε είδον και εφοβήθησαν· εταράχθησαν και αι άβυσσοι.
\par 17 Πλημμύραν υδάτων έχυσαν αι νεφέλαι· φωνήν έδωκαν οι ουρανοί· και τα βέλη σου διεπέταξαν.
\par 18 Η φωνή της βροντής σου ήτο εν τω ουρανίω τροχώ· εφώτισαν αι αστραπαί την οικουμένην· εσαλεύθη και έντρομος έγεινεν η γη.
\par 19 Διά της θαλάσσης είναι η οδός σου και αι τρίβοι σου εν ύδασι πολλοίς, και τα ίχνη σου δεν γνωρίζονται.
\par 20 Ωδήγησας ως πρόβατα τον λαόν σου διά χειρός Μωϋσέως και Ααρών.

\chapter{78}

\par «Μασχίλ του Ασάφ.» Άκουσον, λαέ μου, τον νόμον μου· κλίνατε τα ώτα σας εις τα λόγια του στόματός μου.
\par 2 Θέλω ανοίξει εν παραβολή το στόμα μου· θέλω προφέρει πράγματα αξιομνημόνευτα, τα απ' αρχής·
\par 3 όσα ηκούσαμεν και εγνωρίσαμεν και οι πατέρες ημών διηγήθησαν εις ημάς.
\par 4 Δεν θέλομεν κρύψει αυτά από των τέκνων αυτών εις την επερχομένην γενεάν, διηγούμενοι τους επαίνους του Κυρίου και την δύναμιν αυτού και τα θαυμάσια αυτού, τα οποία έκαμε.
\par 5 Και έστησε μαρτύριον εν τω Ιακώβ και νόμον έθεσεν εν τω Ισραήλ, τα οποία προσέταξεν εις τους πατέρας ημών, να κάμνωσιν αυτά γνωστά εις τα τέκνα αυτών·
\par 6 διά να γνωρίζη αυτά η γενεά η επερχομένη, οι υιοί οι μέλλοντες να γεννηθώσι· και αυτοί, όταν αναστηθώσι, να διηγώνται εις τα τέκνα αυτών·
\par 7 διά να θέσωσιν επί τον Θεόν την ελπίδα αυτών, και να μη λησμονώσι τα έργα του Θεού, αλλά να φυλάττωσι τας εντολάς αυτού·
\par 8 και να μη γείνωσιν, ως οι πατέρες αυτών, γενεά διεστραμμένη και απειθής· γενεά, ήτις δεν εφύλαξεν ευθείαν την καρδίαν αυτής, και δεν εστάθη πιστόν μετά του Θεού το πνεύμα αυτής·
\par 9 ως οι υιοί του Εφραΐμ, οίτινες ώπλισμένοι, βαστάζοντες τόξα, εστράφησαν οπίσω την ημέραν της μάχης.
\par 10 Δεν εφύλαξαν την διαθήκην του Θεού, και εν τω νόμω αυτού δεν ηθέλησαν να περιπατώσι·
\par 11 και ελησμόνησαν τα έργα αυτού και τα θαυμάσια αυτού, τα οποία έδειξεν εις αυτούς.
\par 12 Έμπροσθεν των πατέρων αυτών έκαμε θαυμάσια, εν τη γη της Αιγύπτου, τη πεδιάδι Τάνεως.
\par 13 Διέσχισε την θάλασσαν και διεπέρασεν αυτούς και έστησε τα ύδατα ως σωρόν·
\par 14 και ώδήγησεν αυτούς την ημέραν εν νεφέλη και όλην την νύκτα εν φωτί πυρός.
\par 15 Διέσχισε πέτρας εν τη ερήμω και επότισεν αυτούς ως εκ μεγάλων αβύσσων·
\par 16 και εξήγαγε ρύακας εκ της πέτρας και κατεβίβασεν ύδατα ως ποταμούς.
\par 17 Αλλ' αυτοί εξηκολούθουν έτι αμαρτάνοντες εις αυτόν, παροξύνοντες τον Ύψιστον εν ανύδρω τόπω·
\par 18 και επείρασαν τον Θεόν εν τη καρδία αυτών, ζητούντες βρώσιν κατά την όρεξιν αυτών·
\par 19 και ελάλησαν κατά του Θεού, λέγοντες, Μήπως δύναται ο Θεός να ετοιμάση τράπεζαν εν τη ερήμω;
\par 20 Ιδού, επάταξε την πέτραν, και έρρευσαν ύδατα και χείμαρροι επλημμύρησαν· μήπως δύναται να δώση και άρτον; ή να ετοιμάση κρέας εις τον λαόν αυτού;
\par 21 Διά τούτο ήκουσεν ο Κύριος και ωργίσθη· και πυρ εξήφθη κατά του Ιακώβ, έτι δε και οργή ανέβη κατά του Ισραήλ·
\par 22 διότι δεν επίστευσαν εις τον Θεόν, ουδέ ήλπισαν επί την σωτηρίαν αυτού·
\par 23 ενώ προσέταξε τας νεφέλας από άνωθεν και τας θύρας του ουρανού ήνοιξε,
\par 24 και έβρεξεν εις αυτούς μάννα διά να φάγωσι και σίτον ουρανού έδωκεν εις αυτούς·
\par 25 άρτον αγγέλων έφαγεν ο άνθρωπος· τροφήν έστειλεν εις αυτούς μέχρι χορτασμού.
\par 26 Εσήκωσεν εν τω ουρανώ ανατολικόν άνεμον, και διά της δυνάμεως αυτού επέφερε τον νότον·
\par 27 και έβρεξεν επ' αυτούς κρέας ως το χώμα και πετεινά πτερωτά ως την άμμον της θαλάσσης·
\par 28 και έκαμε να πέσωσιν εις το μέσον του στρατοπέδου αυτών, κύκλω των σκηνών αυτών.
\par 29 Και έφαγον και εχορτάσθησαν σφόδρα· και έφερεν εις αυτούς την επιθυμίαν αυτών·
\par 30 δεν είχον χωρισθή από της επιθυμίας αυτών, έτι ήτο εν τω στόματι αυτών βρώσις αυτών,
\par 31 και οργή του Θεού ανέβη επ' αυτούς, και εφόνευσε τους μεγαλητέρους εξ αυτών και τους εκλεκτούς του Ισραήλ κατέβαλεν.
\par 32 Εν πάσι τούτοις ημάρτησαν έτι και δεν επίστευσαν εις τα θαυμάσια αυτού.
\par 33 Διά τούτο συνετέλεσεν εν ματαιότητι τας ημέρας αυτών και τα έτη αυτών εν ταραχή.
\par 34 Ότε εθανάτονεν αυτούς, τότε εξεζήτουν αυτόν, και επέστρεφον και από όρθρου προσέτρεχον εις τον Θεόν·
\par 35 και ενεθυμούντο, ότι ο Θεός ήτο φρούριον αυτών και ο Θεός ο Ύψιστος λυτρωτής αυτών.
\par 36 Αλλ' εκολάκευον αυτόν διά του στόματος αυτών και διά της γλώσσης αυτών εψεύδοντο προς αυτόν·
\par 37 Η δε καρδία αυτών δεν ήτο ευθεία μετ' αυτού, και δεν ήσαν πιστοί εις την διαθήκην αυτού.
\par 38 Αυτός όμως οικτίρμων συνεχώρησε την ανομίαν αυτών και δεν ηφάνισεν αυτούς· αλλά πολλάκις ανέστελλε τον θυμόν αυτού, και δεν διήγειρεν όλην την οργήν αυτού·
\par 39 και ενεθυμήθη ότι ήσαν σάρξ· άνεμος παρερχόμενος και μη επιστρέφων.
\par 40 Ποσάκις παρώξυναν αυτόν εν τη ερήμω, παρώργισαν αυτόν εν τη ανύδρω,
\par 41 και εστράφησαν και επείρασαν τον Θεόν, και τον Άγιον του Ισραήλ παρώξυναν.
\par 42 Δεν ενεθυμήθησαν την χείρα αυτού, την ημέραν καθ' ην ελύτρωσεν αυτούς από του εχθρού·
\par 43 πως έδειξεν εν Αιγύπτω τα σημεία αυτού και τα θαυμάσια αυτού εν τη πεδιάδι Τάνεως·
\par 44 και μετέβαλεν εις αίμα τους ποταμούς αυτών και τους ρύακας αυτών, διά να μη πίωσιν.
\par 45 Απέστειλεν επ' αυτούς κυνόμυιαν και κατέφαγεν αυτούς, και βατράχους και εφάνισαν αυτούς.
\par 46 Και παρέδωκε τους καρπούς αυτών εις τον βρούχον και τους κόπους αυτών εις την ακρίδα.
\par 47 Κατηφάνισε διά της χαλάζης τας αμπέλους αυτών και τας συκαμίνους αυτών με πέτρας χαλάζης·
\par 48 και παρέδωκεν εις την χάλαζαν τα κτήνη αυτών και τα ποίμνια αυτών εις τους κεραυνούς.
\par 49 Απέστειλεν επ' αυτούς την έξαψιν του θυμού αυτού, την αγανάκτησιν και την οργήν και την θλίψιν, αποστέλλων αυτά δι' αγγέλων κακοποιών.
\par 50 Ήνοιξεν οδόν εις την οργήν αυτού· δεν εφείσθη από του θανάτου την ψυχήν αυτών, και παρέδωκεν εις θανατικόν την ζωήν αυτών·
\par 51 και επάταξε παν πρωτότοκον εν Αιγύπτω, την απαρχήν της δυνάμεως αυτών εν ταις σκηναίς του Χάμ·
\par 52 και εσήκωσεν εκείθεν ως πρόβατα τον λαόν αυτού και ώδήγησεν αυτούς ως ποίμνιον εν τη ερήμω·
\par 53 και ώδήγησεν αυτούς εν ασφαλεία, και δεν εδειλίασαν· τους δε εχθρούς αυτών εσκέπασεν η θάλασσα.
\par 54 Και εισήγαγεν αυτούς εις το όριον της αγιότητος αυτού, το όρος τούτο, το οποίον απέκτησεν η δεξιά αυτού·
\par 55 και εξεδίωξεν απ' έμπροσθεν αυτών τα έθνη και διεμοίρασεν αυτά κληρονομίαν με σχοινίον, και εν ταις σκηναίς αυτών κατώκισε τας φυλάς του Ισραήλ.
\par 56 Και όμως επείρασαν και παρώξυναν τον Θεόν τον ύψιστον και δεν εφύλαξαν τα μαρτύρια αυτού·
\par 57 αλλ' εστράφησαν και εφέρθησαν απίστως, ως οι πατέρες αυτών· εστράφησαν ως τόξον στρεβλόν·
\par 58 και παρώργισαν αυτόν με τους υψηλούς αυτών τόπους, και με τα γλυπτά αυτών διήγειραν αυτόν εις ζηλοτυπίαν.
\par 59 Ήκουσεν ο Θεός και υπερωργίσθη και εβδελύχθη σφόδρα τον Ισραήλ·
\par 60 και εγκατέλιπε την σκηνήν του Σηλώ, την σκηνήν όπου κατώκησε μεταξύ των ανθρώπων·
\par 61 και παρέδωκεν εις αιχμαλωσίαν την δύναμιν αυτού και την δόξαν αυτού εις χείρα εχθρού·
\par 62 και παρέδωκεν εις ρομφαίαν τον λαόν αυτού και υπερωργίσθη κατά της κληρονομίας αυτού·
\par 63 τους νέους αυτών κατέφαγε πυρ, και αι παρθένοι αυτών δεν ενυμφεύθησαν·
\par 64 οι ιερείς αυτών έπεσον εν μαχαίρα, και αι χήραι αυτών δεν επένθησαν.
\par 65 Τότε εξηγέρθη ως εξ ύπνου ο Κύριος, ως άνθρωπος δυνατός, βοών από οίνου·
\par 66 και επάταξε τους εχθρούς αυτού εις τα οπίσω· όνειδος αιώνιον έθεσεν επ' αυτούς.
\par 67 Και απέρριψε την σκηνήν Ιωσήφ, και την φυλήν Εφραΐμ δεν εξέλεξεν.
\par 68 Αλλ' εξέλεξε την φυλήν Ιούδα, το όρος της Σιών, το οποίον ηγάπησε.
\par 69 Και ωκοδόμησεν ως υψηλά παλάτια το αγιαστήριον αυτού, ως την γην την οποίαν εθεμελίωσεν εις τον αιώνα.
\par 70 Και εξέλεξε Δαβίδ τον δούλον αυτού και ανέλαβεν αυτόν εκ των ποιμνίων των προβάτων·
\par 71 Εξόπισθεν των θηλαζόντων προβάτων έφερεν αυτόν, διά να ποιμαίνη Ιακώβ τον λαόν αυτού και Ισραήλ την κληρονομίαν αυτού·
\par 72 Και εποίμανεν αυτούς κατά την ακακίαν της καρδίας αυτού· και διά της συνέσεως των χειρών αυτού ώδήγησεν αυτούς.

\chapter{79}

\par «Ψαλμός του Ασάφ.» Θεέ, ήλθον έθνη εις την κληρονομίαν σου· εμίαναν τον ναόν τον άγιόν σου· κατέστησαν την Ιερουσαλήμ εις σωρούς ερειπίων·
\par 2 έδωκαν τα πτώματα των δούλων σου βρώσιν εις τα πετεινά του ουρανού, την σάρκα των οσίων σου εις τα θηρία της γης.
\par 3 Εξέχεαν το αίμα αυτών ως ύδωρ κύκλω της Ιερουσαλήμ, και δεν υπήρχεν ο θάπτων.
\par 4 Εγείναμεν όνειδος εις τους γείτονας ημών, κατάγελως και χλευασμός εις τους πέριξ ημών.
\par 5 Έως πότε, Κύριε; θέλεις οργίζεσθαι διαπαντός; θέλει καίει ως πυρ η ζηλοτυπία σου;
\par 6 Έκχεον την οργήν σου επί τα έθνη τα μη γνωρίζοντά σε και επί τα βασίλεια τα μη επικαλεσθέντα το όνομά σου·
\par 7 διότι κατέφαγον τον Ιακώβ, και το κατοικητήριον αυτού ηρήμωσαν.
\par 8 Μη ενθυμηθής καθ' ημών τας ανομίας των αρχαίων· ταχέως ας προφθάσωσιν ημάς οι οικτιρμοί σου, διότι εταπεινώθημεν σφόδρα.
\par 9 Βοήθησον ημάς, Θεέ της σωτηρίας ημών, ένεκεν της δόξης του ονόματός σου· και ελευθέρωσον ημάς και γενού ίλεως εις τας αμαρτίας ημών, ένεκεν του ονόματός σου.
\par 10 Διά τι να είπωσι τα έθνη, Που είναι ο Θεός αυτών; Ας γνωρισθή εις τα έθνη έμπροσθεν ημών, η εκδίκησις του εκχυθέντος αίματος των δούλων σου.
\par 11 Ας έλθη ενώπιόν σου ο στεναγμός των δεσμίων· κατά την μεγαλωσύνην του βραχίονός σου σώσον τους καταδεδικασμένους εις θάνατον·
\par 12 και απόδος εις τους γείτονας ημών επταπλάσια εις τον κόλπον αυτών τον ονειδισμόν αυτών, με τον οποίον σε ωνείδισαν, Κύριε.
\par 13 Ημείς δε ο λαός σου και τα πρόβατα της βοσκής σου Θέλομεν σε δοξολογεί εις τον αιώνα· από γενεάς εις γενεάν θέλομεν αναγγέλλει την αίνεσίν σου.

\chapter{80}

\par «Εις τον πρώτον μουσικόν, επί Σοσανίμ-εδούθ. Ψαλμός του Ασάφ.» Ακροάσθητι, ο ποιμαίνων τον Ισραήλ, συ ο οδηγών ως ποίμνιον τον Ιωσήφ· ο καθήμενος επί των χερουβείμ, εμφανίσθητι.
\par 2 Έμπροσθεν του Εφραΐμ και του Βενιαμίν και του Μανασσή διέγειρον την δύναμίν σου και ελθέ εις σωτηρίαν ημών.
\par 3 Επίστρεψον ημάς, Θεέ, και επίλαμψον το πρόσωπόν σου, και θέλομεν λυτρωθή.
\par 4 Κύριε Θεέ των δυνάμεων, έως πότε θέλεις οργίζεσθαι κατά της προσευχής του λαού σου;
\par 5 Τρέφεις αυτούς με άρτον δακρύων και ποτίζεις αυτούς αφθόνως με δάκρυα.
\par 6 Έκαμες ημάς έριδα εις τους γείτονας ημών· και οι εχθροί ημών γελώσι μεταξύ αλλήλων.
\par 7 Επίστρεψον ημάς, Θεέ των δυνάμεων, και επίλαμψον το πρόσωπον σου, και θέλομεν λυτρωθή.
\par 8 Άμπελον εξ Αιγύπτου μετεκόμισας· εξεδίωξας έθνη και εφύτευσας αυτήν.
\par 9 Ητοίμασας τόπον έμπροσθεν αυτής και βαθέως ερρίζωσας αυτήν· και εγέμισε την γην.
\par 10 Εσκεπάσθησαν τα όρη υπό της σκιάς αυτής, και αι αναδενδράδες αυτής ήσαν ως αι υψηλαί κέδροι.
\par 11 Εξέτεινε τα κλήματα αυτής έως θαλάσσης και τους βλαστούς αυτής έως του ποταμού.
\par 12 Διά τι εκρήμνισας τους φραγμούς αυτής, και τρυγώσιν αυτήν πάντες οι διαβαίνοντες την οδόν;
\par 13 Ερημόνει αυτήν ο αγριόχοιρος εκ του δάσους, και το θηρίον του αγρού νέμεται αυτήν.
\par 14 Επίστρεψον, δεόμεθα, Θεέ των δυνάμεων· επίβλεψον εξ ουρανού και ιδέ, και επίσκεψαι την άμπελον ταύτην,
\par 15 και το φυτόν, το οποίον εφύτευσεν η δεξιά σου και τον βλαστόν, τον οποίον εκραταίωσας εις σεαυτόν.
\par 16 Εκαύθη εν πυρί· εκόπη· εχάθησαν από επιτιμήσεως του προσώπου σου.
\par 17 Ας ήναι η χειρ σου επί τον άνδρα της δεξιάς σου· επί τον υιόν του ανθρώπου, τον οποίον έκαμες δυνατόν εις σεαυτόν.
\par 18 Και ημείς δεν θέλομεν εκκλίνει από σού· ζωοποίησον ημάς, και το όνομά σου θέλομεν επικαλείσθαι.
\par 19 Επίστρεψον ημάς, Κύριε Θεέ των δυνάμεων· επίλαμψον το πρόσωπόν σου, και θέλομεν λυτρωθή.

\chapter{81}

\par «Εις τον πρώτον μουσικόν, επί Γιττίθ. Ψαλμός του Ασάφ.» Ψάλατε εν ευφροσύνη εις τον Θεόν, την δύναμιν ημών· αλαλάξατε εις τον Θεόν του Ιακώβ.
\par 2 Υψώσατε ψαλμωδίαν και κρούετε τύμπανον, κιθάραν τερπνήν μετά ψαλτηρίου.
\par 3 Σαλπίσατε σάλπιγγα εν νεομηνία, εν καιρώ ωρισμένω, εν τη ημέρα της εορτής ημών.
\par 4 Διότι πρόσταγμα είναι τούτο εις τον Ισραήλ, νόμος του Θεού του Ιακώβ.
\par 5 Εις μαρτύριον διέταξε τούτο εις τον Ιωσήφ, ότε εξήλθε κατά της γης Αιγύπτου· όπου ήκουσα γλώσσαν, την οποίαν δεν εγνώριζον·
\par 6 απεμάκρυνα από του φορτίου τον ώμον αυτού· αι χείρες αυτού έπαυσαν από κοφίνου·
\par 7 εν καιρώ θλίψεως επεκαλέσθης, και σε ελύτρωσα· σοι απεκρίθην· από του αποκρύφου τόπου της βροντής· σε εδοκίμασα εν τοις ύδασι της αντιλογίας. Διάψαλμα.
\par 8 Άκουσον, λαέ μου, και θέλω διαμαρτυρηθή κατά σού· Ισραήλ, εάν μου ακούσης,
\par 9 Ας μη ήναι εις σε θεός ξένος, και μη προσκυνήσης θεόν αλλότριον.
\par 10 Εγώ είμαι Κύριος ο Θεός σου, όστις σε ανήγαγεν εκ γης Αιγύπτου· πλάτυνον το στόμα σου, και θέλω γεμίσει αυτό.
\par 11 Αλλ' ο λαός μου δεν ήκουσε της φωνής μου, και ο Ισραήλ δεν επρόσεξεν εις εμέ.
\par 12 Διά τούτο παρέδωκα αυτούς εις τας επιθυμίας της καρδίας αυτών· και περιεπάτησαν εν ταις βουλαίς αυτών.
\par 13 Είθε να μου ήκουεν ο λαός μου, και ο Ισραήλ να περιεπάτει εις τας οδούς μου·
\par 14 πάραυτα ήθελον καταβάλει τους εχθρούς αυτών, και κατά των θλιβόντων αυτούς ήθελον στρέψει την χείρα μου.
\par 15 Οι μισούντες τον Κύριον ήθελον αποτύχει εναντίον αυτού, ο δε καιρός εκείνων ήθελε διαμένει πάντοτε·
\par 16 και ήθελε θρέψει αυτούς με το πάχος του σίτου, και με μέλι εκ πέτρας ήθελον σε χορτάσει.

\chapter{82}

\par «Ψαλμός του Ασάφ.» Ο Θεός ίσταται εν τη συνάξει των δυνατών· αναμέσον των θεών θέλει κρίνει.
\par 2 Έως πότε θέλετε κρίνει αδίκως, και θέλετε προσωποληπτεί τους ασεβείς; Διάψαλμα.
\par 3 Κρίνατε τον πτωχόν και τον ορφανόν· κάμετε δικαιοσύνην εις τον τεθλιμμένον και πένητα.
\par 4 Ελευθερόνετε τον πτωχόν και τον πένητα· λυτρόνετε αυτόν εκ χειρός των ασεβών.
\par 5 Δεν γνωρίζουσιν, ουδέ νοούσι· περιπατούσιν εν σκότει· πάντα τα θεμέλια της γης σαλεύονται.
\par 6 Εγώ είπα, θεοί είσθε σεις και υιοί Υψίστου πάντες·
\par 7 Σεις όμως ως άνθρωποι αποθνήσκετε, και ως εις των αρχόντων πίπτετε.
\par 8 Ανάστα, Θεέ, κρίνον την γήν· διότι συ θέλεις κατακληρονομήσει πάντα τα έθνη.

\chapter{83}

\par «Ωδή Ψαλμού του Ασάφ.» Θεέ, μη σιωπήσης· μη σιγήσης και μη ησυχάσης, Θεέ.
\par 2 Διότι, ιδού, οι εχθροί σου θορυβούσι, και οι μισούντές σε ύψωσαν κεφαλήν.
\par 3 Κακήν βουλήν έλαβον κατά του λαού σου και συνεβουλεύθησαν κατά των εκλεκτών σου.
\par 4 Είπον, Έλθετε, και ας εξολοθρεύσωμεν αυτούς από του να ήναι έθνος· και το όνομα του Ισραήλ ας μη μνημονεύηται πλέον.
\par 5 Διότι εκ συμφώνου συνεβουλεύθησαν ομού· συνεμάχησαν κατά σού·
\par 6 αι σκηναί του Εδώμ και οι Ισμαηλίται· ο Μωάβ και οι Αγαρηνοί·
\par 7 Ο Γεβάλ και ο Αμμών και ο Αμαλήκ· οι Φιλισταίοι μετά των κατοικούντων την Τύρον.
\par 8 Και αυτός ο Ασσούρ ηνώθη μετ' αυτών· εβοήθησαν τους υιούς του Λωτ. Διάψαλμα.
\par 9 Κάμε εις αυτούς ως εις τους Μαδιανίτας, ως εις τον Σισάραν, ως εις τον Ιαβείν εν τω χειμάρρω Κεισών·
\par 10 οίτινες απωλέσθησαν εν Εν-δώρ· έγειναν κόπρος διά την γην.
\par 11 Κάμε τους άρχοντας αυτών ως τον Ωρήβ και ως τον Ζήβ· και ως τον Ζεβεέ και ως τον Σαλμανάν πάντας τους αρχηγούς αυτών·
\par 12 οίτινες είπον, Ας κληρονομήσωμεν εις εαυτούς τα κατοικητήρια του Θεού.
\par 13 Θεέ μου, κάμε αυτούς ως τροχόν, ως άχυρον κατά πρόσωπον ανέμου.
\par 14 Ως το πυρ καίει το δάσος, και ως η φλόξ κατακαίει τα όρη,
\par 15 ούτω καταδίωξον αυτούς με την ανεμοζάλην σου, και με τον ανεμοστρόβιλον σου κατατρόμαξον αυτούς.
\par 16 Γέμισον τα πρόσωπα αυτών από ατιμίας, και θέλουσι ζητήσει το όνομά σου, Κύριε.
\par 17 Ας καταισχυνθώσι και ας ταραχθώσι διαπαντός· και ας εντραπώσι και ας απολεσθώσι·
\par 18 και ας γνωρίσωσιν ότι συ, του οποίου το όνομα είναι Κύριος, είσαι ο μόνος Ύψιστος επί πάσαν την γην.

\chapter{84}

\par «Εις τον πρώτον μουσικόν, επί Γιττίθ. Ψαλμός διά τους υιούς Κορέ.» Πόσον αγαπηταί είναι αι σκηναί σου, Κύριε των δυνάμεων
\par 2 Επιποθεί και μάλιστα λιποθυμεί ψυχή μου διά τας αυλάς του Κυρίου· η καρδία μου και η σαρξ μου αγαλλιώνται διά τον Θεόν τον ζώντα.
\par 3 Ναι, το στρουθίον εύρηκε κατοικίαν, και η τρυγών φωλεάν εις εαυτήν, όπου θέτει τους νεοσσούς αυτής, τα θυσιαστήριά σου, Κύριε των δυνάμεων, Βασιλεύ μου και Θεέ μου.
\par 4 Μακάριοι οι κατοικούντες εν τω οίκω σου· πάντοτε θέλουσι σε αινεί. Διάψαλμα.
\par 5 Μακάριος ο άνθρωπος, του οποίου η δύναμις είναι εν σοί· εν τη καρδία των οποίων είναι αι οδοί σου·
\par 6 οίτινες διαβαίνοντες διά της κοιλάδος του κλαυθμώνος καθιστώσιν αυτήν πηγήν υδάτων· και η βροχή έτι γεμίζει τους λάκκους.
\par 7 Προβαίνουσιν από δυνάμεως εις δύναμιν· έκαστος αυτών φαίνεται ενώπιον του Θεού εν Σιών.
\par 8 Κύριε, Θεέ των δυνάμεων, εισάκουσον της προσευχής μου· Ακροάσθητι, Θεέ του Ιακώβ. Διάψαλμα.
\par 9 Ιδέ, Θεέ, η ασπίς ημών, και επίβλεψον εις το πρόσωπον του χριστού σου.
\par 10 Διότι καλητέρα είναι μία ημέρα εν ταις αυλαίς σου υπέρ χιλιάδας· ήθελον προτιμήσει να ήμαι θυρωρός εν τω οίκω του Θεού μου, παρά να κατοικώ εν ταις σκηναίς της πονηρίας.
\par 11 Διότι ήλιος και ασπίς είναι Κύριος ο Θεός· χάριν και δόξαν θέλει δώσει ο Κύριος· δεν θέλει στερήσει ουδενός αγαθού τους περιπατούντας εν ακακία.
\par 12 Κύριε των δυνάμεων, μακάριος ο άνθρωπος ο ελπίζων επί σε.

\chapter{85}

\par «Εις τον πρώτον μουσικόν. Ψαλμός διά τους υιούς Κορέ.» Ευηρεστήθης, Κύριε, εις την γην σου· έφερες από της αιχμαλωσίας τον Ιακώβ.
\par 2 Συνεχώρησας την ανομίαν του λαού σου· εσκέπασας πάσας τας αμαρτίας αυτών. Διάψαλμα.
\par 3 Κατέπαυσας πάσαν την οργήν σου· απέστρεψας από της οργής του θυμού σου.
\par 4 Επίστρεψον ημάς, Θεέ της σωτηρίας ημών, και κατάπαυσον τον καθ' ημών θυμόν σου.
\par 5 Θέλεις είσθαι διαπαντός ωργισμένος εις ημάς; θέλεις επεκτείνει την οργήν σου από γενεάς εις γενεάν;
\par 6 Δεν θέλεις πάλιν ζωοποιήσει ημάς, διά να ευφραίνηται ο λαός σου εις σε;
\par 7 Δείξον εις ημάς, Κύριε, το έλεός σου και δος εις ημάς την σωτηρίαν σου.
\par 8 Θέλω ακούσει τι θέλει λαλήσει Κύριος ο Θεός· διότι θέλει λαλήσει ειρήνην προς τον λαόν αυτού και προς τους οσίους αυτού· και ας μη επιστρέψωσιν εις αφροσύνην.
\par 9 Βεβαίως πλησίον των φοβουμένων αυτόν είναι η σωτηρία αυτού, διά να κατοική δόξα εν τη γη ημών.
\par 10 Έλεος και αλήθεια συναπηντήθησαν· δικαιοσύνη και ειρήνη εφιλήθησαν.
\par 11 Αλήθεια εκ της γης θέλει αναβλαστήσει· και δικαιοσύνη εξ ουρανού θέλει κύψει.
\par 12 Ο Κύριος βεβαίως θέλει δώσει το αγαθόν· και η γη ημών θέλει δώσει τον καρπόν αυτής.
\par 13 Δικαιοσύνη έμπροσθεν αυτού θέλει προπορεύεσθαι, και θέλει βάλει αυτήν εις την οδόν των διαβημάτων αυτού.

\chapter{86}

\par «Προσευχή του Δαβίδ.» Κλίνον, Κύριε, το ωτίον σου· επάκουσόν μου, διότι πτωχός και πένης είμαι εγώ.
\par 2 Φύλαξον την ψυχήν μου, διότι είμαι όσιος· συ, Θεέ μου, σώσον τον δούλον σου τον ελπίζοντα επί σε.
\par 3 Ελέησόν με, Κύριε, διότι προς σε κράζω όλην την ημέραν.
\par 4 Εύφρανον την ψυχήν του δούλου σου, διότι προς σε, Κύριε, υψόνω την ψυχήν μου.
\par 5 Διότι συ, Κύριε, είσαι αγαθός και εύσπλαγχνος και πολυέλεος εις πάντας τους επικαλουμένους σε.
\par 6 Ακροάσθητι, Κύριε, της προσευχής μου και πρόσεξον εις την φωνήν των δεήσεών μου.
\par 7 Εν ημέρα θλίψεώς μου θέλω σε επικαλείσθαι, διότι θέλεις μου εισακούει.
\par 8 Δεν είναι όμοιός σου μεταξύ των θεών, Κύριε· ουδέ έργα όμοια των έργων σου.
\par 9 Πάντα τα έθνη, τα οποία έκαμες, θέλουσιν ελθεί και προσκυνήσει ενώπιόν σου, Κύριε, και θέλουσι δοξάσει το όνομά σου·
\par 10 διότι μέγας είσαι και κάμνεις θαυμάσια· συ είσαι Θεός μόνος.
\par 11 Δίδαξόν με, Κύριε, την οδόν σου, και θέλω περιπατεί εν τη αληθεία σου· προσήλονε την καρδίαν μου εις τον φόβον του ονόματός σου.
\par 12 Θέλω σε αινεί, Κύριε ο Θεός μου, εν όλη τη καρδία μου και θέλω δοξάζει το όνομά σου εις τον αιώνα·
\par 13 διότι μέγα επ' εμέ το έλεός σου· και ηλευθέρωσας την ψυχήν μου εξ άδου κατωτάτου.
\par 14 Θεέ, οι υπερήφανοι εσηκώθησαν κατ' εμού, και αι συνάξεις των βιαστών εζήτησαν την ψυχήν μου· και δεν σε έθεσαν ενώπιόν αυτών.
\par 15 Αλλά συ, Κύριε, είσαι Θεός οικτίρμων και ελεήμων, μακρόθυμος και πολυέλεος και αληθινός.
\par 16 Επίβλεψον επ' εμέ και ελέησόν με· δος την δύναμίν σου εις τον δούλον σου και σώσον τον υιόν της δούλης σου.
\par 17 Κάμε εις εμέ σημείον εις αγαθόν, διά να ίδωσιν οι μισούντές με και να αισχυνθώσι· διότι συ, Κύριε, με εβοήθησας και με παρηγόρησας.

\chapter{87}

\par «Ψαλμός ωδής διά τους υιούς Κορέ.» Το θεμέλιον αυτού είναι εις τα όρη τα άγια.
\par 2 Αγαπά ο Κύριος τας πύλας της Σιών υπέρ πάντα τα σκηνώματα του Ιακώβ.
\par 3 Ένδοξα ελαλήθησαν περί σου, πόλις του Θεού. Διάψαλμα.
\par 4 Θέλω αναφέρει την Ραάβ και την Βαβυλώνα μεταξύ των γνωριζόντων με· ιδού, η Παλαιστίνη και η Τύρος μετά της Αιθιοπίας· ούτος εγεννήθη εκεί.
\par 5 Και περί της Σιών θέλουσιν ειπεί, ούτος και εκείνος εγεννήθη εν αυτή· και αυτός ο Ύψιστος θέλει στερεώσει αυτήν.
\par 6 Ο Κύριος θέλει αριθμήσει, όταν καταγράψη τους λαούς, ότι ούτος εγεννήθη εκεί. Διάψαλμα.
\par 7 Και οι ψάλται καθώς και οι λαληταί των οργάνων θέλουσι λέγει, Πάσαι αι πηγαί μου είναι εν σοι.

\chapter{88}

\par «Ωιδή ψαλμού διά τους υιούς Κορέ, εις τον πρώτον μουσικόν, επί Μαχαλάθ-λεανώθ, Μασχίλ του Αιμάν του Εζραΐτου.» Κύριε ο Θεός της σωτηρίας μου, ημέραν και νύκτα έκραξα ενώπιόν σου·
\par 2 Ας έλθη ενώπιόν σου η προσευχή μου· κλίνον το ωτίον σου εις την κραυγήν μου·
\par 3 Διότι ενεπλήσθη κακών η ψυχή μου, και η ζωή μου πλησιάζει εις τον άδην.
\par 4 Συγκατηριθμήθην μετά των καταβαινόντων εις τον λάκκον· έγεινα ως άνθρωπος μη έχων δύναμιν·
\par 5 εγκαταλελειμμένος μεταξύ των νεκρών, ως οι πεφονευμένοι, κοιτώμενοι εν τω τάφω, τους οποίους δεν ενθυμείσαι πλέον, και οίτινες απεκόπησαν από της χειρός σου.
\par 6 Μ' έβαλες εις τον κατώτατον λάκκον, εις το σκότος, εις τα βάθη.
\par 7 Επ' εμέ επεστηρίχθη ο θυμός σου, και πάντα τα κύματά σου επέφερες επ' εμέ. Διάψαλμα.
\par 8 Εμάκρυνας τους γνωστούς μου απ' εμού· με έκαμες βδέλυγμα προς αυτούς· απεκλείσθην και δεν δύναμαι να εξέλθω.
\par 9 Ο οφθαλμός μου ητόνησεν από της θλίψεως· σε επεκαλέσθην, Κύριε, όλην την ημέραν· ήπλωσα προς σε τας χείρας μου.
\par 10 Μήπως εις τους νεκρούς θέλεις κάμει θαυμάσια; ή οι τεθνεώτες θέλουσι σηκωθή και θέλουσι σε αινέσει; Διάψαλμα.
\par 11 Μήπως εν τω τάφω θέλουσι διηγείσθαι το έλεός σου ή την αλήθειάν σου εν τη φθορά;
\par 12 Μήπως θέλουσι γνωρισθή εν τω σκότει τα θαυμάσιά σου και η δικαιοσύνη σου εν τω τόπω της λήθης.
\par 13 Αλλ' εγώ προς σε, Κύριε, έκραξα· και το πρωΐ η προσευχή μου θέλει σε προφθάσει.
\par 14 Διά τι, Κύριε, απορρίπτεις την ψυχήν μου, αποκρύπτεις το πρόσωπόν σου απ' εμού;
\par 15 Είμαι τεθλιμμένος και ψυχομαχών εκ νεότητος· δοκιμάζω τους φόβους σου και ευρίσκομαι εν αμηχανία.
\par 16 Επ' εμέ διήλθον αι οργαί σου· οι τρόμοι σου με εφάνισαν.
\par 17 Ως ύδατα με περιετριγύρισαν όλην την ημέραν· ομού με περιεκύκλωσαν.
\par 18 Απεμάκρυνας απ' εμού τον αγαπητόν και τον φίλον· οι γνωστοί μου είναι αφανείς.

\chapter{89}

\par «Μασχίλ του Εθάν του Εζραΐτου.» Τα ελέη του Κυρίου εις τον αιώνα θέλω ψάλλει· διά του στόματός μου θέλω αναγγέλλει την αλήθειάν σου εις γενεάν και γενεάν.
\par 2 Διότι είπα, το έλεός σου θέλει θεμελιωθή εις τον αιώνα· εν τοις ουρανοίς θέλεις στερεώσει την αλήθειάν σου.
\par 3 Έκαμα διαθήκην μετά του εκλεκτού μου· ώμοσα προς Δαβίδ τον δούλον μου·
\par 4 Διαπαντός θέλω στερεώσει το σπέρμα σου, και θέλω οικοδομήσει τον θρόνον σου εις γενεάν και γενεάν. Διάψαλμα.
\par 5 Και οι ουρανοί θέλουσιν υμνεί τα θαυμάσιά σου, Κύριε· και η αλήθειά σου θέλει εξυμνείσθαι εν τη συνάξει των αγίων.
\par 6 Διότι τις εν τω ουρανώ δύναται να εξισωθή με τον Κύριον; Τις μεταξύ των υιών των δυνατών δύναται να ομοιωθή με τον Κύριον;
\par 7 Ο Θεός είναι φοβερός σφόδρα εν τη βουλή των αγίων και σεβαστός εν πάσι τοις κύκλω αυτού.
\par 8 Κύριε Θεέ των δυνάμεων, τις όμοιός σου; δυνατός είσαι, Κύριε, και η αλήθειά σου είναι κύκλω σου.
\par 9 Συ δεσπόζεις την έπαρσιν της θαλάσσης· όταν σηκόνωνται τα κύματα αυτής, συ ταπεινόνεις αυτά.
\par 10 Συ συνέτριψας την Ραάβ ως τραυματίαν· διά του βραχίονος της δυνάμεώς σου διεσκόρπισας τους εχθρούς σου.
\par 11 Σού είναι οι ουρανοί και σου η γη την οικουμένην και το πλήρωμα αυτής, συ εθεμελίωσας αυτά.
\par 12 Τον βορράν και τον νότον, συ έκτισας αυτούς· Θαβώρ και Αερμών εις το όνομά σου θέλουσιν αγάλλεσθαι.
\par 13 Έχεις ισχυρόν τον βραχίονα· κραταιά είναι η χειρ σου· υψηλή η δεξιά σου.
\par 14 Η δικαιοσύνη και η κρίσις είναι η βάσις του θρόνου σου· το έλεος και η αλήθεια θέλουσι προπορεύεσθαι έμπροσθεν του προσώπου σου.
\par 15 Μακάριος ο λαός ο γινώσκων αλαλαγμόν· θέλουσι περιπατεί, Κύριε, εν τω φωτί του προσώπου σου.
\par 16 Εις το όνομά σου θέλουσιν αγάλλεσθαι όλην την ημέραν· και εις την δικαιοσύνην σου θέλουσιν υψωθή.
\par 17 Διότι συ είσαι το καύχημα της δυνάμεως αυτών· και διά της ευμενείας σου θέλει υψωθή το κέρας ημών.
\par 18 Διότι ο Κύριος είναι η ασπίς ημών· και ο Άγιος του Ισραήλ ο βασιλεύς ημών.
\par 19 Ελάλησας τότε δι' οράματος προς τον όσιόν σου και είπας· έθεσα βοήθειαν επί τον δυνατόν· ύψωσα εκλεκτόν εκ του λαού·
\par 20 Εύρηκα Δαβίδ τον δούλον μου· με το έλαιον το άγιόν μου έχρισα αυτόν·
\par 21 η χειρ μου θέλει στερεόνει αυτόν· και ο βραχίων μου θέλει ενδυναμόνει αυτόν.
\par 22 δεν θέλει υπερισχύσει εχθρός κατ' αυτού· ουδέ υιός ανομίας θέλει ταλαιπωρήσει αυτόν.
\par 23 Και θέλω κατακόψει απ' έμπροσθεν αυτού τους εχθρούς αυτού· και τους μισούντας αυτόν θέλω κατατροπώσει.
\par 24 Η δε αλήθειά μου και το έλεός μου θέλουσιν είσθαι μετ' αυτού· και εν τω ονόματί μου θέλει υψωθή το κέρας αυτού.
\par 25 Και θέλω θέσει την χείρα αυτού επί την θάλασσαν, και επί τους ποταμούς την δεξιάν αυτού.
\par 26 Αυτός θέλει κράξει προς εμέ, Πατήρ μου είσαι, Θεός μου και πέτρα της σωτηρίας μου.
\par 27 Εγώ βεβαίως θέλω κάμει αυτόν πρωτότοκόν μου, Ύψιστον επί τους βασιλείς της γης.
\par 28 Διαπαντός θέλω φυλάττει εις αυτόν το έλεός μου, και η διαθήκη μου θέλει είσθαι στερεά μετ' αυτού.
\par 29 Και θέλω κάμει να διαμένη το σπέρμα αυτού εις τον αιώνα, και ο θρόνος αυτού ως αι ημέραι του ουρανού.
\par 30 Εάν εγκαταλίπωσιν οι υιοί αυτού τον νόμον μου και εις τας κρίσεις μου δεν περιπατήσωσιν·
\par 31 Εάν παραβώσι τα διατάγματά μου και δεν φυλάξωσι τας εντολάς μου·
\par 32 Τότε θέλω επισκεφθή με ράβδον τας παραβάσεις αυτών και με πληγάς τας παρανομίας αυτών.
\par 33 Το έλεός μου όμως δεν θέλω αφαιρέσει απ' αυτού, ουδέ θέλω ψευσθή κατά της αληθείας μου.
\par 34 Δεν θέλω παραβή την διαθήκην μου, ουδέ θέλω αθετήσει ό,τι εξήλθεν εκ των χειλέων μου.
\par 35 Άπαξ ώμοσα εις την αγιότητά μου, ότι δεν θέλω ψευσθή προς τον Δαβίδ.
\par 36 Το σπέρμα αυτού θέλει διαμένει εις τον αιώνα και ο θρόνος αυτού ως ο ήλιος, ενώπιόν μου·
\par 37 Ως η σελήνη θέλει στερεωθή εις τον αιώνα και μάρτυς πιστός εν τω ουρανώ. Διάψαλμα.
\par 38 Αλλά συ απέβαλες και εβδελύχθης, ωργίσθης κατά του χριστού σου·
\par 39 ηκύρωσας την διαθήκην του δούλου σου· εβεβήλωσας το διάδημα αυτού έως της γης.
\par 40 Κατέβαλες πάντας τους φραγμούς αυτού· ηφάνισας τα οχυρώματα αυτού·
\par 41 διαρπάζουσιν αυτόν πάντες οι διαβαίνοντες την οδόν· κατεστάθη όνειδος εις τους γείτονας αυτού.
\par 42 Ύψωσας την δεξιάν των εναντίων αυτού· εύφρανας πάντας τους εχθρούς αυτού·
\par 43 ήμβλυνας μάλιστα το κοπτερόν της ρομφαίας αυτού και δεν εστερέωσας αυτόν εν τη μάχη·
\par 44 Έπαυσας την δόξαν αυτού και τον θρόνον αυτού έρριψας κατά γης.
\par 45 Ωλιγόστευσας τας ημέρας της νεότητος αυτού· ενέδυσας αυτόν με αισχύνην. Διάψαλμα.
\par 46 Έως πότε, Κύριε; θέλεις κρύπτεσθαι διαπαντός; θέλει καίεσθαι ως πυρ η οργή σου;
\par 47 Μνήσθητι πόσον βραχύς είναι ο καιρός μου, εν τίνι ματαιότητι εποίησας πάντας τους υιούς των ανθρώπων.
\par 48 Τις άνθρωπος θέλει ζήσει και δεν θέλει ιδεί θάνατον; τις θέλει λυτρώσει την ψυχήν αυτού εκ της χειρός του άδου; Διάψαλμα.
\par 49 Που είναι τα ελέη σου τα αρχαία, Κύριε, τα οποία ώμοσας προς τον Δαβίδ εν τη αληθεία σου;
\par 50 Μνήσθητι, Κύριε, του ονειδισμού των δούλων σου, τον οποίον φέρω εν τω κόλπω μου υπό τοσούτων πολυαρίθμων λαών·
\par 51 με τον οποίον ωνείδισαν οι εχθροί σου, Κύριε· με τον οποίον ωνείδισαν τα ίχνη του χριστού σου.
\par 52 Ευλογητός Κύριος εις τον αιώνα. Αμήν, και αμήν.

\chapter{90}

\par «Προσευχή του Μωϋσέως, του ανθρώπου του Θεού.» Κύριε, συ έγεινες εις ημάς καταφυγή εις γενεάν και γενεάν.
\par 2 Πριν γεννηθώσι τα όρη, και πλάσης την γην και την οικουμένην, και από του αιώνος έως του αιώνος, συ είσαι ο Θεός.
\par 3 Επαναφέρεις τον άνθρωπον εις τον χούν· και λέγεις, Επιστρέψατε, υιοί των ανθρώπων.
\par 4 Διότι χίλια έτη ενώπιόν σου είναι ως ημέρα η χθές, ήτις παρήλθε, και ως φυλακή νυκτός.
\par 5 Κατακλύζεις αυτούς· είναι ως όνειρον της αυγής, ως χόρτος όστις παρέρχεται·
\par 6 το πρωΐ ανθεί και παρακμάζει· το εσπέρας κόπτεται και ξηραίνεται.
\par 7 Διότι εκλείπομεν εν τη οργή σου και εν τω θυμώ σου ταραττόμεθα.
\par 8 Έθεσας τας ανομίας ημών ενώπιόν σου, τα κρύφια ημών εις το φως του προσώπου σου.
\par 9 Επειδή πάσαι αι ημέραι ημών παρέρχονται εν τη οργή σου· διατρέχομεν τα έτη ημών ως διανόημα.
\par 10 Αι ημέραι της ζωής ημών είναι καθ' εαυτάς εβδομήκοντα έτη, και εάν εν ευρωστία, ογδοήκοντα έτη· πλην και το καλήτερον μέρος αυτών είναι κόπος και πόνος, διότι ταχέως παρέρχεται και εμείς πετώμεν.
\par 11 Τις γνωρίζει την δύναμιν της οργής σου και του θυμού σου αναλόγως του φόβου σου;
\par 12 Δίδαξον ημάς να μετρώμεν ούτω τας ημέρας ημών, ώστε να προσκολλώμεν τας καρδίας ημών εις την σοφίαν.
\par 13 Επίστρεψον, Κύριε· έως πότε; και γενού ίλεως εις τους δούλους σου.
\par 14 Χόρτασον ημάς του ελέους σου από πρωΐας, και θέλομεν αγάλλεσθαι και ευφραίνεσθαι κατά πάσας τας ημέρας ημών.
\par 15 Εύφρανον ημάς αντί των ημερών, καθ' ας έθλιψας ημάς, των ετών καθ' α είδομεν κακά.
\par 16 Ας γείνη το έργον σου φανερόν εις τους δούλους σου και η δόξα σου εις τους υιούς αυτών·
\par 17 και ας ήναι η λαμπρότης Κυρίου του Θεού ημών εφ' ημάς· και το έργον των χειρών ημών στερέονε εφ' ημάς· ναι, το έργον των χειρών ημών, στερέονε αυτό.

\chapter{91}

\par Ο κατοικών υπό την σκέπην του Υψίστου υπό την σκιάν του Παντοκράτορος θέλει διατρίβει.
\par 2 Θέλω λέγει προς τον Κύριον, Συ είσαι καταφυγή μου και φρούριόν μου· Θεός μου· επ' αυτόν θέλω ελπίζει.
\par 3 Διότι αυτός θέλει σε λυτρόνει εκ της παγίδος των κυνηγών και εκ θανατηφόρου λοιμού.
\par 4 Με τα πτερά αυτού θέλει σε σκεπάζει, και υπό τας πτέρυγας αυτού θέλεις είσθαι ασφαλής· η αλήθεια αυτού είναι πανοπλία και ασπίς.
\par 5 Δεν θέλεις φοβείσθαι από φόβου νυκτερινού, την ημέραν από βέλους πετωμένου.
\par 6 Από θανατικού, το οποίον περιπατεί εν σκότει· από ολέθρου, όστις ερημόνει εν μεσημβρία·
\par 7 Χιλιάς θέλει πίπτει εξ αριστερών σου και μυριάς εκ δεξιών σου· πλην εις σε δεν θέλει πλησιάζει.
\par 8 Μόνον με τους οφθαλμούς σου θέλεις θεωρεί και θέλεις βλέπει των ασεβών την ανταπόδοσιν.
\par 9 Επειδή συ τον Κύριον, την ελπίδα μου, τον Ύψιστον έκαμες καταφύγιόν σου,
\par 10 δεν θέλει συμβαίνει εις σε κακόν, και μάστιξ δεν θέλει πλησιάζει εις την σκηνήν σου.
\par 11 Διότι θέλει προστάξει εις τους αγγέλους αυτού περί σου, διά να σε διαφυλάττωσιν εν πάσαις ταις οδοίς σου.
\par 12 Θέλουσι σε σηκόνει επί των χειρών αυτών, διά να μη προσκόψης προς λίθον τον πόδα σου.
\par 13 Θέλεις πατήσει επί λέοντα και επί ασπίδα· θέλεις καταπατήσει σκύμνον και δράκοντα.
\par 14 Επειδή έθεσεν εις εμέ την αγάπην αυτού, διά τούτο θέλω λυτρώσει αυτόν· θέλω υψώσει αυτόν, διότι εγνώρισε το όνομά μου.
\par 15 Θέλει με επικαλείσθαι, και θέλω εισακούει αυτού· μετ' αυτού θέλω είσθαι εν θλίψει· θέλω λυτρόνει αυτόν και θέλω δοξάζει αυτόν.
\par 16 Θέλω χορτάσει αυτόν μακρότητα ημερών και θέλω δείξει εις αυτόν την σωτηρίαν μου.

\chapter{92}

\par «Ψαλμός ωδής διά την ημέραν του Σαββάτου.» Αγαθόν το να δοξολογή τις τον Κύριον και να ψαλμωδή εις το όνομά σου, Υψιστε·
\par 2 να αναγγέλλη το πρωΐ το έλεός σου και την αλήθειάν σου πάσαν νύκτα,
\par 3 με δεκάχορδον όργανον και με ψαλτήριον· με ωδήν και κιθάραν.
\par 4 Διότι με εύφρανας, Κύριε, εν τοις ποιήμασί σου· θέλω αγάλλεσθαι εν τοις έργοις των χειρών σου.
\par 5 Πόσον μεγάλα είναι τα έργα σου, Κύριε βαθείς είναι οι διαλογισμοί σου σφόδρα.
\par 6 Ο άνθρωπος ο ανόητος δεν γνωρίζει, και ο μωρός δεν εννοεί τούτο·
\par 7 ότι οι ασεβείς βλαστάνουσιν ως ο χόρτος, και ανθούσι πάντες οι εργάται της ανομίας, διά να αφανισθώσιν αιωνίως.
\par 8 Αλλά συ, Κύριε, είσαι ύψιστος εις τον αιώνα.
\par 9 Διότι, ιδού, οι εχθροί σου, Κύριε, διότι, ιδού, οι εχθροί σου θέλουσιν εξολοθρευθή· θέλουσι διασκορπισθή πάντες οι εργάται της ανομίας.
\par 10 Αλλά συ θέλεις υψώσει ως του μονοκέρωτος το κέρας μου· εγώ θέλω χρισθή με νέον έλαιον·
\par 11 και ο οφθαλμός μου θέλει ιδεί την εκδίκησιν των εχθρών μου· τα ώτα μου θέλουσιν ακούσει περί των κακοποιών των επανισταμένων κατ' εμού.
\par 12 Ο δίκαιος ως φοίνιξ θέλει ανθεί· ως κέδρος του Λιβάνου θέλει αυξάνει.
\par 13 Πεφυτευμένοι εν τω οίκω του Κυρίου, θέλουσιν ανθεί εν ταις αυλαίς του Θεού ημών·
\par 14 θέλουσι καρποφορεί και εν αυτώ τω βαθεί γήρατι, θέλουσιν είσθαι ακμάζοντες και ανθηροί·
\par 15 διά να αναγγέλλωσιν ότι δίκαιος είναι ο Κύριος, το φρούριόν μου· και δεν υπάρχει αδικία εν αυτώ.

\chapter{93}

\par Ο Κύριος βασιλεύει· μεγαλοπρέπειαν είναι ενδεδυμένος· ενδεδυμένος είναι ο Κύριος δύναμιν και περιεζωσμένος· και την οικουμένην εστερέωσεν, ώστε δεν θέλει σαλευθή.
\par 2 Απ' αρχής είναι εστερεωμένος ο θρόνος σου· από του αιώνος συ είσαι.
\par 3 Ύψωσαν οι ποταμοί, Κύριε, ύψωσαν οι ποταμοί την φωνήν αυτών· οι ποταμοί ύψωσαν τα κύματα αυτών.
\par 4 Ο Κύριος ο εν υψίστοις είναι δυνατώτερος υπέρ τον ήχον πολλών υδάτων, υπέρ τα δυνατά κύματα της θαλάσσης·
\par 5 τα μαρτύριά σου είναι πιστά σφόδρα· εις τον οίκόν σου ανήκει αγιότης, Κύριε, εις μακρότητα ημερών.

\chapter{94}

\par Θεέ των εκδικήσεων, Κύριε, Θεέ των εκδικήσεων, εμφάνηθι.
\par 2 Υψώθητι, Κριτά της γής· απόδος ανταπόδοσιν εις τους υπερηφάνους.
\par 3 Έως πότε οι ασεβείς, Κύριε, έως πότε οι ασεβείς θέλουσι θριαμβεύει;
\par 4 Έως πότε θέλουσι προφέρει και λαλεί σκληρά; θέλουσι καυχάσθαι πάντες οι εργάται της ανομίας;
\par 5 Τον λαόν σου, Κύριε, καταθλίβουσι και την κληρονομίαν σου κακοποιούσι.
\par 6 Την χήραν και τον ξένον φονεύουσι και θανατόνουσι τους ορφανούς.
\par 7 Και λέγουσι, δεν θέλει ιδεί ο Κύριος ουδέ θέλει νοήσει ο Θεός του Ιακώβ.
\par 8 Εννοήσατε, οι άφρονες μεταξύ του λαού· και οι μωροί, πότε θέλετε φρονιμεύσει;
\par 9 Ο φυτεύσας το ωτίον, δεν θέλει ακούσει; ο πλάσας τον οφθαλμόν, δεν θέλει ιδεί;
\par 10 Ο σωφρονίζων τα έθνη, δεν θέλει ελέγξει; ο διδάσκων τον άνθρωπον γνώσιν;
\par 11 Ο Κύριος γνωρίζει τους διαλογισμούς των ανθρώπων, ότι είναι μάταιοι.
\par 12 Μακάριος ο άνθρωπος, τον οποίον σωφρονίζεις, Κύριε, και διά του νόμου σου διδάσκεις αυτόν·
\par 13 διά να αναπαύης αυτόν από των ημερών της συμφοράς, εωσού σκαφθή λάκκος εις τον ασεβή.
\par 14 Διότι δεν θέλει απορρίψει ο Κύριος τον λαόν αυτού, και την κληρονομίαν αυτού δεν θέλει εγκαταλείψει.
\par 15 Επειδή η κρίσις θέλει επιστρέψει εις την δικαιοσύνην, και θέλουσιν ακολουθήσει αυτήν πάντες οι ευθείς την καρδίαν.
\par 16 Τις θέλει σηκωθή υπέρ εμού κατά των πονηρευομένων; τις θέλει παρασταθή υπέρ εμού κατά των εργατών της ανομίας;
\par 17 Εάν ο Κύριος δεν με εβοήθει, παρ' ολίγον ήθελε κατοικήσει ψυχή μου εν τη σιωπή.
\par 18 Ότε έλεγον, ωλίσθησεν ο πους μου, το έλεός σου, Κύριε, με εβοήθει.
\par 19 Εν τω πλήθει των αμηχανιών της καρδίας μου, αι παρηγορίαι σου εύφραναν την ψυχήν μου.
\par 20 Μήπως έχει μετά σου συγκοινωνίαν ο θρόνος της ανομίας, όστις μηχανάται αδικίαν αντί νόμου;
\par 21 Αυτοί εφορμώσι κατά της ψυχής του δικαίου και αίμα αθώον καταδικάζουσιν.
\par 22 Αλλ' ο Κύριος είναι εις εμέ καταφύγιον και ο Θεός μου το φρούριον της ελπίδος μου.
\par 23 Και θέλει επιστρέψει επ' αυτούς την ανομίαν αυτών και εν τη πονηρία αυτών θέλει αφανίσει αυτούς· Κύριος ο Θεός ημών θέλει αφανίσει αυτούς.

\chapter{95}

\par Δεύτε, ας αγαλλιασθώμεν εις τον Κύριον· ας αλαλάξωμεν εις το φρούριον της σωτηρίας ημών.
\par 2 Ας προφθάσωμεν ενώπιον αυτού μετά δοξολογίας· εν ψαλμοίς ας αλαλάξωμεν εις αυτόν.
\par 3 Διότι Θεός μέγας είναι ο Κύριος, και Βασιλεύς μέγας υπέρ πάντας τους θεούς.
\par 4 Διότι εις αυτού την χείρα είναι τα βάθη της γής· και τα ύψη των ορέων είναι αυτού.
\par 5 Διότι αυτού είναι η θάλασσα, και αυτός έκαμεν αυτήν· και την ξηράν αι χείρες αυτού έπλασαν.
\par 6 Δεύτε, ας προσκυνήσωμεν και ας προσπέσωμεν· ας γονατίσωμεν ενώπιον του Κυρίου, του Ποιητού ημών.
\par 7 Διότι αυτός είναι ο Θεός ημών· και ημείς λαός της βοσκής αυτού και πρόβατα της χειρός αυτού. Σήμερον εάν ακούσητε της φωνής αυτού,
\par 8 μη σκληρύνητε την καρδίαν σας, ως εν τω παροργισμώ, ως εν τη ημέρα του πειρασμού εν τη ερήμω·
\par 9 όπου οι πατέρες σας με επείρασαν, με εδοκίμασαν και είδον τα έργα μου.
\par 10 Τεσσαράκοντα έτη δυσηρεστήθην με την γενεάν εκείνην, και είπα, ούτος είναι λαός πεπλανημένος την καρδίαν, και αυτοί δεν εγνώρισαν τας οδούς μου.
\par 11 Διά τούτο ώμοσα εν τη οργή μου, ότι εις την ανάπαυσίν μου δεν θέλουσιν εισέλθει.

\chapter{96}

\par Ψάλατε εις τον Κύριον άσμα νέον· ψάλατε εις τον Κύριον, πάσα η γη.
\par 2 Ψάλατε εις τον Κύριον· ευλογείτε το όνομα αυτού· κηρύττετε από ημέρας εις ημέραν την σωτηρίαν αυτού.
\par 3 Αναγγείλατε εν τοις έθνεσι την δόξαν αυτού, εν πάσι τοις λαοίς τα θαυμάσια αυτού.
\par 4 Διότι μέγας ο Κύριος και αξιΰμνητος σφόδρα· είναι φοβερός υπέρ πάντας τους θεούς.
\par 5 Διότι πάντες οι θεοί των εθνών είναι είδωλα· ο δε Κύριος τους ουρανούς εποίησε.
\par 6 Δόξα και μεγαλοπρέπεια είναι ενώπιον αυτού· ισχύς και ώραιότης εν τω αγιαστηρίω αυτού.
\par 7 Απόδοτε εις τον Κύριον, πατριαί των λαών, απόδοτε εις τον Κύριον δόξαν και ισχύν.
\par 8 Απόδοτε εις τον Κύριον την δόξαν του ονόματος αυτού· λάβετε προσφοράς και εισέλθετε εις τας αυλάς αυτού.
\par 9 Προσκυνήσατε τον Κύριον εν τω μεγαλοπρεπεί αγιαστηρίω αυτού· φοβείσθε από προσώπου αυτού, πάσα η γη.
\par 10 Είπατε εν τοις έθνεσιν, Ο Κύριος βασιλεύει· η οικουμένη θέλει βεβαίως είσθαι εστερεωμένη· δεν θέλει σαλευθή· αυτός θέλει κρίνει τους λαούς εν ευθύτητι.
\par 11 Ας ευφραίνωνται οι ουρανοί, και ας αγάλλεται η γή· ας ηχή η θάλασσα και το πλήρωμα αυτής.
\par 12 Ας χαίρωσιν αι πεδιάδες και πάντα τα εν αυταίς· τότε θέλουσιν αγάλλεσθαι πάντα τα δένδρα του δάσους
\par 13 ενώπιον του Κυρίου· διότι έρχεται, διότι έρχεται διά να κρίνη την γήν· θέλει κρίνει την οικουμένην εν δικαιοσύνη και τους λαούς εν τη αληθεία αυτού.

\chapter{97}

\par Ο Κύριος βασιλεύει· ας αγάλλεται η γή· ας ευφραίνεται το πλήθος των νήσων.
\par 2 Νεφέλη και ομίχλη είναι κύκλω αυτού· δικαιοσύνη και κρίσις η βάσις του θρόνου αυτού.
\par 3 Πυρ προπορεύεται έμπροσθεν αυτού και καταφλέγει πανταχόθεν, τους εχθρούς αυτού.
\par 4 Αι αστραπαί αυτού φωτίζουσι την οικουμένην· είδεν η γη και εσαλεύθη.
\par 5 Τα όρη διαλύονται ως κηρός από της παρουσίας του Κυρίου, από της παρουσίας του Κυρίου πάσης της γής·
\par 6 Αναγγέλλουσιν οι ουρανοί την δικαιοσύνην αυτού, και βλέπουσι πάντες οι λαοί την δόξαν αυτού.
\par 7 Ας αισχυνθώσι πάντες οι λατρεύοντες τα γλυπτά, οι καυχώμενοι εις τα είδωλα· προσκυνείτε αυτόν, πάντες οι θεοί.
\par 8 Ήκουσεν η Σιών και ευφράνθη, και εχάρησαν αι θυγατέρες του Ιούδα διά τας κρίσεις σου, Κύριε.
\par 9 Διότι συ, Κύριε, είσαι ύψιστος εφ' όλην την γήν· σφόδρα υπερυψώθης υπέρ πάντας τους θεούς.
\par 10 Οι αγαπώντες τον Κύριον, μισείτε το κακόν· αυτός φυλάττει τας ψυχάς των οσίων αυτού· ελευθερόνει αυτούς εκ χειρός ασεβών.
\par 11 Φως σπείρεται διά τον δίκαιον και ευφροσύνη διά τους ευθείς την καρδίαν.
\par 12 Ευφραίνεσθε, δίκαιοι, εν Κυρίω, και υμνείτε την μνήμην της αγιωσύνης αυτού.

\chapter{98}

\par Ψάλατε εις τον Κύριον άσμα νέον· διότι έκαμε θαυμάσια· η δεξιά αυτού και ο βραχίων ο άγιος αυτού ενήργησαν εις αυτόν σωτηρίαν.
\par 2 Ο Κύριος έκαμε γνωστήν την σωτηρίαν αυτού· έμπροσθεν των εθνών απεκάλυψε την δικαιοσύνην αυτού.
\par 3 Ενεθυμήθη το έλεος αυτού και την αλήθειαν αυτού προς τον οίκον του Ισραήλ· πάντα τα πέρατα της γης είδον την σωτηρίαν του Θεού ημών.
\par 4 Αλαλάξατε εις τον Κύριον, πάσα η γή· ευφραίνεσθε και αγάλλεσθε και ψαλμωδείτε.
\par 5 Ψαλμωδείτε εις τον Κύριον εν κιθάρα· εν κιθάρα και φωνή ψαλμωδίας.
\par 6 Μετά σαλπίγγων και εν φωνή κερατίνης αλαλάξατε ενώπιον του Βασιλέως Κυρίου.
\par 7 Ας ηχή η θάλασσα και το πλήρωμα αυτής· η οικουμένη και οι κατοικούντες εν αυτή.
\par 8 Οι ποταμοί ας κροτώσι χείρας, τα όρη ας αγάλλωνται ομού,
\par 9 ενώπιον του Κυρίου· διότι έρχεται διά να κρίνη την γήν· θέλει κρίνει την οικουμένην εν δικαιοσύνη και τους λαούς εν ευθύτητι.

\chapter{99}

\par Ο Κύριος βασιλεύει, ας τρέμωσιν οι λαοί· ο καθήμενος επί των χερουβείμ· ας σεισθή η γη.
\par 2 Ο Κύριος είναι μέγας εν Σιών, και επί πάντας τους λαούς είναι υψηλός.
\par 3 Ας δοξολογώσι το όνομά σου το μέγα και φοβερόν, διότι είναι άγιον·
\par 4 και την δύναμιν του βασιλέως, όστις αγαπά την δικαιοσύνην· συ διώρισας την ευθύτητα, συ έκαμες κρίσιν και δικαιοσύνην εις τον Ιακώβ.
\par 5 Υψούτε Κύριον τον Θεόν ημών και προσκυνείτε εις το υποπόδιον των ποδών αυτού· διότι είναι άγιος.
\par 6 Μωϋσής και Ααρών μεταξύ των ιερέων αυτού, και Σαμουήλ μεταξύ των επικαλουμένων το όνομα αυτού, επεκαλούντο τον Κύριον, και αυτός εισήκουεν αυτών.
\par 7 Εν στύλω νεφέλης ελάλει προς αυτούς· εφύλαξε τα μαρτύρια αυτού και τα προστάγματα, τα οποία έδωκεν εις αυτούς·
\par 8 Κύριε Θεέ ημών, συ εισήκουες αυτών· έγεινας εις αυτούς Θεός συγχωρητικός, πλην και εκδικητής διά τας πράξεις αυτών.
\par 9 Υψούτε Κύριον τον Θεόν ημών και προσκυνείτε εις το όρος το άγιον αυτού· διότι άγιος Κύριος ο Θεός ημών.

\chapter{100}

\par «Ψαλμός δοξολογίας.» Αλαλάξατε εις τον Κύριον, πάσα η γη.
\par 2 Δουλεύσατε εις τον Κύριον εν ευφροσύνη· έλθετε ενώπιον αυτού εν αγαλλιάσει.
\par 3 Γνωρίσατε ότι ο Κύριος είναι ο Θεός· αυτός έκαμεν ημάς, και ουχί ημείς· ημείς είμεθα λαός αυτού και πρόβατα της βοσκής αυτού.
\par 4 Εισέλθετε εις τας πύλας αυτού εν δοξολογία, εις τας αυλάς αυτού εν ύμνω· δοξολογείτε αυτόν· ευλογείτε το όνομα αυτού.
\par 5 Διότι ο Κύριος είναι αγαθός· εις τον αιώνα μένει το έλεος αυτού, και έως γενεάς και γενεάς η αλήθεια αυτού.

\chapter{101}

\par Ψαλμός του Δαβίδ. Έλεος και κρίσιν θέλω ψάλλει· εις σε, Κύριε, θέλω ψαλμωδεί.
\par 2 Θέλω είσθαι συνετός εν οδώ αμώμω· πότε θέλεις ελθεί προς εμέ; θέλω περιπατεί εν ακεραιότητι της καρδίας μου εν μέσω του οίκου μου.
\par 3 Δεν θέλω βάλει προ οφθαλμών μου πράγμα πονηρόν· μισώ τους ποιούντας παράνομα· ουδέν τούτων θέλει κολληθή εις εμέ.
\par 4 Η διεστραμμένη καρδία θέλει αποβληθή απ' εμού· τον πονηρόν δεν θέλω γνωρίζει.
\par 5 Τον καταλαλούντα κρυφίως τον πλησίον αυτού, τούτον θέλω εξολοθρεύει· τον έχοντα υπερήφανον βλέμμα και επηρμένην καρδίαν, τούτον δεν θέλω υποφέρει.
\par 6 Οι οφθαλμοί μου θέλουσιν είσθαι επί τους πιστούς της γης, διά να συγκατοικώσι μετ' εμού· ο περιπατών εν οδώ αμώμω, ούτος θέλει με υπηρετεί.
\par 7 Δεν θέλει κατοικεί εν μέσω του οίκου μου ο εργαζόμενος απάτην· ο λαλών ψεύδος δεν θέλει στερεωθή έμπροσθεν των οφθαλμών μου.
\par 8 Κατά πάσαν πρωΐαν θέλω εξολοθρεύει πάντας τους ασεβείς της γης, διά να εκκόψω εκ της πόλεως του Κυρίου πάντας τους εργάτας της ανομίας.

\chapter{102}

\par «Προσευχή του τεθλιμμένου, όταν αδημονή, και εκχέη το παράπονον αυτού ενώπιον του Κυρίου.» Κύριε, εισάκουσον της προσευχής μου, και η κραυγή μου ας έλθη προς σε.
\par 2 Μη κρύψης το πρόσωπόν σου απ' εμού· καθ' ην ημέραν θλίβομαι, κλίνον προς εμέ το ωτίον σου· καθ' ην ημέραν σε επικαλούμαι, ταχέως επάκουέ μου.
\par 3 Διότι εξέλιπον ως καπνός αι ημέραι μου, και τα οστά μου ως φρύγανον κατεξηράνθησαν.
\par 4 Επληγώθη η καρδία μου και εξηράνθη ως χόρτος, ώστε ελησμόνησα να τρώγω τον άρτον μου.
\par 5 Από φωνής του στεναγμού μου εκολλήθησαν τα οστά μου εις το δέρμα μου.
\par 6 Κατεστάθην όμοιος του ερημικού πελεκάνος· έγεινα ως νυκτοκόραξ εν ταις ερήμοις.
\par 7 Αγρυπνώ και είμαι ως στρουθίον μονάζον επί δώματος.
\par 8 Όλην την ημέραν με ονειδίζουσιν οι εχθροί μου· οι μαινόμενοι ομνύουσι κατ' εμού.
\par 9 Διότι έφαγον στάκτην ως άρτον και συνεκέρασα με δάκρυα το ποτόν μου,
\par 10 Εξ αιτίας της οργής σου και της αγανακτήσεώς σου· διότι σηκώσας με έρριψας κάτω.
\par 11 Αι ημέραι μου παρέρχονται ως σκιά, και εγώ εξηράνθην ως χόρτος.
\par 12 Συ δε, Κύριε, εις τον αιώνα διαμένεις, και το μνημόσυνον σου εις γενεάν και γενεάν.
\par 13 Συ θέλεις σηκωθή, θέλεις σπλαγχνισθή την Σιών· διότι είναι καιρός να ελεήσης αυτήν, διότι ο διωρισμένος καιρός έφθασεν.
\par 14 Επειδή οι δούλοι σου αρέσκονται εις τους λίθους αυτής και σπλαγχνίζονται το χώμα αυτής.
\par 15 Τότε τα έθνη θέλουσι φοβηθή το όνομα του Κυρίου, και πάντες οι βασιλείς της γης την δόξαν σου.
\par 16 Όταν ο Κύριος οικοδομήση την Σιών θέλει φανή εν τη δόξα αυτού.
\par 17 Θέλει επιβλέψει επί την προσευχήν των εγκαταλελειμμένων και δεν θέλει καταφρονήσει την δέησιν αυτών.
\par 18 Τούτο θέλει γραφθή διά την γενεάν την επερχομένην· και ο λαός, όστις θέλει δημιουργηθή, θέλει αινεί τον Κύριον.
\par 19 Διότι έκυψεν εκ του ύψους του αγιαστηρίου αυτού, εξ ουρανού επέβλεψεν ο Κύριος επί την γην,
\par 20 διά να ακούση τον στεναγμόν των δεσμίων, διά να λύση τους καταδεδικασμένους εις θάνατον·
\par 21 διά να κηρύττωσιν εν Σιών το όνομα του Κυρίου και την αίνεσιν αυτού εν Ιερουσαλήμ,
\par 22 όταν συναχθώσιν ομού οι λαοί και αι βασιλείαι, διά να δουλεύσωσι τον Κύριον.
\par 23 Ηδυνάτισεν εν τη οδώ την ισχύν μου· συνέτεμε τας ημέρας μου.
\par 24 Εγώ είπα, μη με αρπάσης, Θεέ μου, εν τω ημίσει των ημερών μου· τα έτη σου είναι εις γενεάς γενεών.
\par 25 Κατ' αρχάς συ, Κύριε, την γην εθεμελίωσας, και έργα των χειρών σου είναι οι ουρανοί.
\par 26 Αυτοί θέλουσιν απολεσθή, συ δε διαμένεις· και πάντες ως ιμάτιον θέλουσι παλαιωθή· ως περιένδυμα θέλεις τυλίξει αυτούς, και θέλουσιν αλλαχθή·
\par 27 συ όμως είσαι ο αυτός, και τα έτη σου δεν θέλουσιν εκλείψει.
\par 28 Οι υιοί των δούλων σου θέλουσι κατοικεί, και το σπέρμα αυτών θέλει διαμένει ενώπιόν σου.

\chapter{103}

\par «Ψαλμός του Δαβίδ.» Ευλόγει, η ψυχή μου, τον Κύριον· και πάντα τα εντός μου το όνομα το άγιον αυτού.
\par 2 Ευλόγει, η ψυχή μου, τον Κύριον, και μη λησμονής πάσας τας ευεργεσίας αυτού·
\par 3 τον συγχωρούντα πάσας τας ανομίας σου· τον ιατρεύοντα πάσας τας αρρωστίας σου·
\par 4 τον λυτρόνοντα εκ της φθοράς την ζωήν σου· τον στεφανούντά σε με έλεος και οικτιρμούς·
\par 5 τον χορτάζοντα εν αγαθοίς το γήράς σου· η νεότης σου ανανεούται ως του αετού.
\par 6 Ο Κύριος κάμνει δικαιοσύνην και κρίσιν εις πάντας τους αδικουμένους.
\par 7 Εφανέρωσε τας οδούς αυτού εις τον Μωϋσήν, τα έργα αυτού εις τους υιούς Ισραήλ.
\par 8 Οικτίρμων και ελεήμων είναι ο Κύριος, μακρόθυμος και πολυέλεος.
\par 9 Δεν θέλει δικολογεί διαπαντός ουδέ θέλει φυλάττει την οργήν αυτού εις τον αιώνα.
\par 10 Δεν έκαμεν εις ημάς κατά τας αμαρτίας ημών, ουδέ ανταπέδωκεν εις ημάς κατά τας ανομίας ημών.
\par 11 Διότι όσον είναι το ύψος του ουρανού υπεράνω της γης, τόσον μέγα είναι το έλεος αυτού προς τους φοβουμένους αυτόν.
\par 12 Όσον απέχει η ανατολή από της δύσεως, τόσον εμάκρυνεν αφ' ημών τας ανομίας ημών.
\par 13 Καθώς σπλαγχνίζεται ο πατήρ τα τέκνα, ούτως ο Κύριος σπλαγχνίζεται τους φοβουμένους αυτόν.
\par 14 Διότι αυτός γνωρίζει την πλάσιν ημών, ενθυμείται ότι είμεθα χώμα.
\par 15 Του ανθρώπου αι ημέραι είναι ως χόρτος· ως το άνθος του αγρού, ούτως ανθεί.
\par 16 Διότι διέρχεται ο άνεμος επ' αυτού, και δεν υπάρχει πλέον· και ο τόπος αυτού δεν γνωρίζει αυτό πλέον.
\par 17 Το δε έλεος του Κυρίου είναι από του αιώνος και έως του αιώνος επί τους φοβουμένους αυτόν· και η δικαιοσύνη αυτού επί τους υιούς των υιών·
\par 18 επί τους φυλάττοντας την διαθήκην αυτού και ενθυμουμένους τας εντολάς αυτού διά να εκπληρώσιν αυτάς.
\par 19 Ο Κύριος ητοίμασε τον θρόνον αυτού εν τω ουρανώ, και η βασιλεία αυτού δεσπόζει τα πάντα.
\par 20 Ευλογείτε τον Κύριον, άγγελοι αυτού, δυνατοί εν ισχύϊ, οι εκτελούντες τον λόγον αυτού, οι ακούοντες της φωνής του λόγου αυτού.
\par 21 Ευλογείτε τον Κύριον, πάσαι αι δυνάμεις αυτού· λειτουργοί αυτού, οι εκτελούντες το θέλημα αυτού.
\par 22 Ευλογείτε τον Κύριον, πάντα τα έργα αυτού εν παντί τόπω της δεσποτείας αυτού. Ευλόγει, η ψυχή μου, τον Κύριον.

\chapter{104}

\par Ευλόγει, η ψυχή μου, τον Κύριον. Κύριε Θεέ μου, εμεγαλύνθης σφόδρα· τιμήν και μεγαλοπρέπειαν είσαι ενδεδυμένος·
\par 2 ο περιτυλιττόμενος το φως ως ιμάτιον, ο εκτείνων τον ουρανόν ως καταπέτασμα·
\par 3 ο στεγάζων με ύδατα τα υπερώα αυτού· ο ποιών τα νέφη άμαξαν αυτού· ο περιπατών επί πτερύγων ανέμων·
\par 4 ο ποιών τους αγγέλους αυτού πνεύματα, τους λειτουργούς αυτού πυρός φλόγα·
\par 5 ο θεμελιών την γην επί την βάσιν αυτής, διά να μη σαλευθή εις τον αιώνα του αιώνος.
\par 6 Με την άβυσσον, ως με ιμάτιον, εκάλυψας αυτήν· τα ύδατα εστάθησαν επί των ορέων·
\par 7 από επιτιμήσεώς σου έφυγον· από της φωνής της βροντής σου εσύρθησαν εν βία·
\par 8 ανέβησαν εις τα όρη, κατέβησαν εις τας κοιλάδας, εις τόπον, τον οποίον διώρισας δι' αυτά·
\par 9 έθεσας όριον, το οποίον δεν θέλουσιν υπερβή ουδέ θέλουσιν επιστρέψει διά να σκεπάσωσι την γην.
\par 10 Ο εξαποστέλλων πηγάς εις τας φάραγγας, διά να ρέωσιν αναμέσον των ορέων·
\par 11 ποτίζουσι πάντα τα θηρία του αγρού· οι άγριοι όνοι σβύνουσι την δίψαν αυτών·
\par 12 πλησίον αυτών τα πετεινά του ουρανού κατασκηνούσι, και αναμέσον των κλάδων κελαδούσιν.
\par 13 Ο ποτίζων τα όρη εκ των υπερώων αυτού· από του καρπού των έργων σου χορταίνει η γη.
\par 14 Ο αναδίδων χόρτον διά τα κτήνη και βοτάνην προς χρήσιν του ανθρώπου, διά να εξάγη τροφήν εκ της γης,
\par 15 και οίνον ευφραίνοντα την καρδίαν του ανθρώπου, έλαιον διά να λαμπρύνη το πρόσωπον αυτού, και άρτον στηρίζοντα την καρδίαν του ανθρώπου.
\par 16 Εχορτάσθησαν τα δένδρα του Κυρίου· αι κέδροι του Λιβάνου, τας οποίας εφύτευσεν·
\par 17 Όπου τα πετεινά κάμνουσι φωλεάς· αι πεύκαι είναι η κατοικία του πελαργού.
\par 18 Τα όρη τα υψηλά είναι διά τας δορκάδας· αι πέτραι καταφυγή εις τους δασύποδας.
\par 19 Έκαμε την σελήνην διά τους καιρούς· ο ήλιος γνωρίζει την δύσιν αυτού.
\par 20 Φέρεις σκότος, και γίνεται νύξ· εν αυτή περιφέρονται πάντα τα θηρία του δάσους·
\par 21 οι σκύμνοι βρυχώνται διά να αρπάσωσι, και να ζητήσωσι παρά του Θεού την τροφήν αυτών.
\par 22 Ο ήλιος ανατέλλει· συνάγονται και πλαγιάζουσιν εν τοις σπηλαίοις αυτών·
\par 23 εξέρχεται ο άνθρωπος εις το έργον αυτού και εις την εργασίαν αυτού έως εσπέρας.
\par 24 Πόσον μεγάλα είναι τα έργα σου, Κύριε· τα πάντα εν σοφία εποίησας· η γη είναι πλήρης των ποιημάτων σου·
\par 25 αύτη η θάλασσα η μεγάλη και ευρύχωρος. Εκεί είναι ερπετά αναρίθμητα, ζώα μικρά μετά μεγάλων·
\par 26 εκεί διατρέχουσι τα πλοία· εκεί ο Λευϊάθαν ούτος, τον οποίον έπλασας διά να παίζη εν αυτή.
\par 27 Πάντα ταύτα επί σε ελπίζουσι, διά να δώσης εν καιρώ την τροφήν αυτών.
\par 28 Δίδεις εις αυτά, συνάγουσιν· ανοίγεις την χείρα σου, χορταίνουσιν αγαθά.
\par 29 Αποστρέφεις το πρόσωπόν σου, ταράττονται· σηκόνεις την πνοήν αυτών, αποθνήσκουσι και εις το χώμα αυτών επιστρέφουσιν·
\par 30 εξαποστέλλεις το πνεύμά σου, κτίζονται, και ανανεόνεις το πρόσωπον της γης.
\par 31 Η δόξα του Κυρίου έστω εις τον αιώνα· ας ευφραίνεται ο Κύριος εις τα έργα αυτού·
\par 32 ο επιβλέπων επί την γην και κάμνων αυτήν να τρέμη· εγγίζει τα όρη, και καπνίζουσι.
\par 33 Θέλω ψάλλει εις τον Κύριον ενόσω ζώ· θέλω ψαλμωδεί εις τον Θεόν μου ενόσω υπάρχω.
\par 34 Η εις αυτόν μελέτη μου θέλει είσθαι γλυκεία· εγώ θέλω ευφραίνεσθαι εις τον Κύριον.
\par 35 Ας εκλείψωσιν οι αμαρτωλοί από της γης και οι ασεβείς ας μη υπάρχωσι πλέον. Ευλόγει, η ψυχή μου, τον Κύριον. Αλληλούϊα.

\chapter{105}

\par Δοξολογείτε τον Κύριον· επικαλείσθε το όνομα αυτού· κάμετε γνωστά εν τοις λαοίς τα έργα αυτού.
\par 2 Ψάλλετε εις αυτόν· ψαλμωδείτε εις αυτόν· λαλείτε περί πάντων των θαυμασίων αυτού.
\par 3 Καυχάσθε εις το άγιον αυτού όνομα· ας ευφραίνεται η καρδία των εκζητούντων τον Κύριον.
\par 4 Ζητείτε τον Κύριον και την δύναμιν αυτού· εκζητείτε το πρόσωπον αυτού διαπαντός.
\par 5 Μνημονεύετε των θαυμασίων αυτού τα οποία έκαμε· των τεραστίων αυτού και των κρίσεων του στόματος αυτού·
\par 6 Σπέρμα Αβραάμ του δούλου αυτού, υιοί Ιακώβ, οι εκλεκτοί αυτού.
\par 7 Αυτός είναι Κύριος ο Θεός ημών· εν πάση τη γη είναι αι κρίσεις αυτού.
\par 8 Μνημονεύετε πάντοτε της διαθήκης αυτού, του λόγου, τον οποίον προσέταξεν εις χιλίας γενεάς,
\par 9 της διαθήκης, την οποίαν έκαμε προς τον Αβραάμ, και του όρκου αυτού προς τον Ισαάκ·
\par 10 και εβεβαίωσεν αυτόν προς τον Ιακώβ διά νόμου, προς τον Ισραήλ διά διαθήκην αιώνιον,
\par 11 λέγων, Εις σε θέλω δώσει την γην Χαναάν, μερίδα της κληρονομίας σας.
\par 12 Ενώ ήσαν αυτοί ολιγοστοί τον αριθμόν, ολίγοι, και πάροικοι εν αυτή,
\par 13 και διήρχοντο από έθνους εις έθνος, από βασιλείου εις άλλον λαόν,
\par 14 δεν αφήκεν άνθρωπον να αδικήση αυτούς· μάλιστα υπέρ αυτών ήλεγξε βασιλείς,
\par 15 λέγων, μη εγγίσητε τους κεχρισμένους μου και μη κακοποιήσητε τους προφήτας μου.
\par 16 Και εκάλεσε πείναν επί την γήν· συνέτριψε παν στήριγμα άρτου.
\par 17 Απέστειλεν έμπροσθεν αυτών άνθρωπον, Ιωσήφ τον πωληθέντα ως δούλον·
\par 18 του οποίου τους πόδας έσφιγξαν εν δεσμοίς· έβαλον αυτόν εις τα σίδηρα·
\par 19 εωσού έλθη ο λόγος αυτού· ο λόγος του Κυρίου εδοκίμασεν αυτόν.
\par 20 Απέστειλεν ο βασιλεύς και έλυσεν αυτόν· ο άρχων των λαών, και ηλευθέρωσεν αυτόν.
\par 21 Κατέστησεν αυτόν κύριον του οίκου αυτού, και άρχοντα επί πάντων των κτημάτων αυτού·
\par 22 διά να παιδεύη τους άρχοντας αυτού κατά την αρέσκειαν αυτού, και να διδάξη σοφίαν τους πρεσβυτέρους αυτού.
\par 23 Τότε ήλθεν ο Ισραήλ εις την Αίγυπτον, και ο Ιακώβ παρώκησεν εν γη Χαμ.
\par 24 Και ο Κύριος ηύξησε σφόδρα τον λαόν αυτού, και εκραταίωσεν αυτόν υπέρ τους εχθρούς αυτού.
\par 25 Εστράφη η καρδία αυτών εις το να μισώσι τον λαόν αυτού, εις το να δολιεύωνται εναντίον των δούλων αυτού.
\par 26 Εξαπέστειλε Μωϋσήν τον δούλον αυτού, και Ααρών, τον οποίον εξέλεξεν.
\par 27 Εξετέλεσαν εν μέσω αυτών τους λόγους των σημείων αυτού και τα θαυμάσια αυτού εν γη Χαμ.
\par 28 Εξαπέστειλε σκότος, και εσκότασε· και δεν ηπείθησαν εις τους λόγους αυτού.
\par 29 Μετέβαλε τα ύδατα αυτών εις αίμα και εθανάτωσε τους ιχθύας αυτών.
\par 30 Η γη αυτών ανέβρυσε βατράχους, έως των ταμείων των βασιλέων αυτών.
\par 31 Είπε, και ήλθε κυνόμυια, και σκνίπες εις πάντα τα όρια αυτών.
\par 32 Έδωκεν εις αυτούς χάλαζαν αντί βροχής, και πυρ φλογερόν εις την γην αυτών·
\par 33 και επάταξε τας αμπέλους αυτών και τας συκέας αυτών, και συνέτριψε τα δένδρα των ορίων αυτών.
\par 34 Είπε, και ήλθεν ακρίς, και βρούχος αναρίθμητος·
\par 35 και κατέφαγε πάντα τον χόρτον εν τη γη αυτών, και κατέφαγε τον καρπόν της γης αυτών.
\par 36 Και επάταξε παν πρωτότοκον εν τη γη αυτών, την απαρχήν πάσης δυνάμεως αυτών.
\par 37 Και εξήγαγεν αυτούς μετά αργυρίου και χρυσίου, και δεν υπήρχεν ασθενής εν ταις φυλαίς αυτών.
\par 38 Ευφράνθη η Αίγυπτος εις την έξοδον αυτών· διότι ο φόβος αυτών είχεν επιπέσει επ' αυτούς.
\par 39 Εξήπλωσε νεφέλην διά να σκεπάζη αυτούς, και πυρ διά να φέγγη την νύκτα.
\par 40 Εζήτησαν, και έφερεν ορτύκια· και άρτον ουρανού εχόρτασεν αυτούς.
\par 41 Διήνοιξε την πέτραν, και ανέβλυσαν ύδατα, και διέρρευσαν ποταμοί εν τόποις ανύδροις.
\par 42 Διότι ενεθυμήθη τον λόγον τον άγιον αυτού, τον προς Αβραάμ τον δούλον αυτού.
\par 43 Και εξήγαγε τον λαόν αυτού εν αγαλλιάσει, τους εκλεκτούς αυτού εν χαρά·
\par 44 και έδωκεν εις αυτούς τας γαίας των εθνών, και εκληρονόμησαν τους κόπους των λαών·
\par 45 διά να φυλάττωσι τα διατάγματα αυτού, και να εκτελώσι τους νόμους αυτού. Αλληλούϊα.

\chapter{106}

\par Αλληλούϊα. Αινείτε τον Κύριον, διότι είναι αγαθός· διότι το έλεος αυτού μένει εις τον αιώνα.
\par 2 Τις δύναται να κηρύξη τα κραταιά έργα του Κυρίου, να κάμη ακουστάς πάσας τας αινέσεις αυτού;
\par 3 Μακάριοι οι φυλάττοντες κρίσιν, οι πράττοντες δικαιοσύνην εν παντί καιρώ.
\par 4 Μνήσθητί μου, Κύριε, εν τη ευμενεία τη προς τον λαόν σου· επισκέφθητί με εν τη σωτηρία σου·
\par 5 διά να βλέπω το καλόν των εκλεκτών σου, διά να ευφραίνωμαι εν τη ευφροσύνη του έθνους σου, διά να καυχώμαι μετά της κληρονομίας σου.
\par 6 Ημαρτήσαμεν μετά των πατέρων ημών· ηνομήσαμεν, ησεβήσαμεν.
\par 7 Οι πατέρες ημών εν Αιγύπτω δεν ενόησαν τα θαυμάσιά σου· δεν ενεθυμήθησαν το πλήθος του ελέους σου, και σε παρώργισαν εν τη θαλάσση, εν τη Ερυθρά θαλάσση.
\par 8 Και όμως έσωσεν αυτούς διά το όνομα αυτού, διά να κάμη γνωστά τα κραταιά έργα αυτού.
\par 9 Και επετίμησε την Ερυθράν θάλασσαν, και εξηράνθη· και διεβίβασεν αυτούς διά των αβύσσων ως δι' ερήμου·
\par 10 και έσωσεν αυτούς εκ της χειρός του μισούντος αυτούς, και ελύτρωσεν αυτούς εκ της χειρός του εχθρού.
\par 11 Και τα ύδατα κατεκάλυψαν τους εχθρούς αυτών· δεν απελείφθη ουδέ εις εξ αυτών.
\par 12 Τότε επίστευσαν εις τους λόγους αυτού· έψαλαν την αίνεσιν αυτού.
\par 13 Πλην ταχέως ελησμόνησαν τα έργα αυτού· δεν περιέμειναν την βουλήν αυτού·
\par 14 Αλλ' επεθύμησαν επιθυμίαν εν τη ερήμω, και επείρασαν τον Θεόν εν τη ανύδρω.
\par 15 Και έδωκεν εις αυτούς την αίτησιν αυτών· απέστειλεν όμως εις αυτούς νόσον θανατηφόρον.
\par 16 Εφθόνησαν έτι τον Μωϋσήν εν τω στρατοπέδω και τον Ααρών τον άγιον του Κυρίου.
\par 17 Η γη ήνοιξε και κατέπιε τον Δαθάν, και εσκέπασε την συναγωγήν του Αβειρών·
\par 18 και πυρ εξήφθη εν τη συναγωγή αυτών· η φλόξ κατέκαυσε τους ασεβείς.
\par 19 Κατεσκεύασαν μόσχον εν Χωρήβ, και προσεκύνησαν το χωνευτόν·
\par 20 και μετήλλαξαν την δόξαν αυτών εις ομοίωμα βοός τρώγοντος χόρτον.
\par 21 Ελησμόνησαν τον Θεόν τον σωτήρα αυτών τον ποιήσαντα μεγαλεία εν Αιγύπτω,
\par 22 θαυμάσια εν γη Χαμ, φοβερά εν τη Ερυθρά θαλάσση.
\par 23 Και είπε να εξολοθρεύση αυτούς, αν ο Μωϋσής ο εκλεκτός αυτού δεν ίστατο εν τη θραύσει ενώπιον αυτού, διά να αποστρέψη την οργήν αυτού, ώστε να μη αφανίση αυτούς.
\par 24 Κατεφρόνησαν έτι την γην την επιθυμητήν· δεν επίστευσαν εις τον λόγον αυτού·
\par 25 και εγόγγυσαν εν ταις σκηναίς αυτών· δεν εισήκουσαν της φωνής του Κυρίου.
\par 26 Διά τούτο εσήκωσε την χείρα αυτού κατ' αυτών, διά να καταστρέψη αυτούς εν τη ερήμω.
\par 27 και να στρέψη το σπέρμα αυτών μεταξύ των εθνών και να διασκορπίση αυτούς εις τους τόπους.
\par 28 Και προσεκολλήθησαν εις τον Βέελ-φεγώρ, και έφαγον θυσίας νεκρών·
\par 29 και παρώξυναν αυτόν εν τοις έργοις αυτών, ώστε εφώρμησεν επ' αυτούς η πληγή.
\par 30 Αλλά σταθείς ο Φινεές έκαμε κρίσιν· και η πληγή έπαυσε·
\par 31 και ελογίσθη εις αυτόν διά δικαιοσύνην, εις γενεάν και γενεάν έως αιώνος.
\par 32 Και παρώξυναν αυτόν εν τοις ύδασι της αντιλογίας, και έπαθε κακώς ο Μωϋσής δι' αυτούς·
\par 33 διότι παρώργισαν το πνεύμα αυτού, ώστε ελάλησεν αστοχάστως διά των χειλέων αυτού.
\par 34 Δεν εξωλόθρευσαν τα έθνη τα οποία ο Κύριος προσέταξεν εις αυτούς·
\par 35 αλλ' εσμίχθησαν μετά των εθνών και έμαθον τα έργα αυτών·
\par 36 και ελάτρευσαν τα γλυπτά αυτών, τα οποία έγειναν παγίς εις αυτούς·
\par 37 και εθυσίασαν τους υιούς αυτών και τας θυγατέρας αυτών εις τα δαιμόνια·
\par 38 Και έχυσαν αίμα αθώον, το αίμα των υιών αυτών και των θυγατέρων αυτών τους οποίους εθυσίασαν εις τα γλυπτά της Χαναάν· και εμιάνθη η γη εξ αιμάτων.
\par 39 Και εμολύνθησαν με τα έργα αυτών, και επόρνευσαν με τας πράξεις αυτών.
\par 40 Διά τούτο η οργή του Κυρίου εξήφθη κατά του λαού αυτού, και εβδελύχθη την κληρονομίαν αυτού,
\par 41 Και παρέδωκεν αυτούς εις τας χείρας των εθνών· και εκυρίευσαν αυτούς οι μισούντες αυτούς.
\par 42 Και έθλιψαν αυτούς οι εχθροί αυτών, και εταπεινώθησαν υπό τας χείρας αυτών.
\par 43 Πολλάκις ελύτρωσεν αυτούς, αλλ' αυτοί παρώργισαν αυτόν με τας βουλάς αυτών· διό εταπεινώθησαν διά την ανομίαν αυτών.
\par 44 Πλην επέβλεψεν επί την θλίψιν αυτών, ότε ήκουσε την κραυγήν αυτών·
\par 45 και ενεθυμήθη την προς αυτούς διαθήκην αυτού και μετεμελήθη κατά το πλήθος του ελέους αυτού.
\par 46 Και έκαμεν αυτούς να εύρωσιν έλεος ενώπιον πάντων των αιχμαλωτισάντων αυτούς.
\par 47 Σώσον ημάς, Κύριε ο Θεός ημών, και συνάγαγε ημάς από των εθνών, διά να δοξολογώμεν το όνομά σου το άγιον και να καυχώμεθα εις την αίνεσίν σου.
\par 48 Ευλογητός Κύριος ο Θεός του Ισραήλ, από του αιώνος και έως του αιώνος· και ας λέγη πας ο λαός, Αμήν. Αλληλούϊα.

\chapter{107}

\par Δοξολογείτε τον Κύριον, διότι είναι αγαθός, διότι το έλεος αυτού μένει εις τον αιώνα.
\par 2 Ας λέγωσιν ούτως οι λελυτρωμένοι του Κυρίου, τους οποίους ελύτρωσεν εκ χειρός του εχθρού·
\par 3 και συνήγαγεν αυτούς εκ των χωρών, από ανατολής και δύσεως από βορρά και από νότου.
\par 4 Περιεπλανώντο εν τη ερήμω, εν οδώ ανύδρω· ουδέ εύρισκον πόλιν διά κατοίκησιν.
\par 5 Ήσαν πεινώντες και διψώντες· η ψυχή αυτών απέκαμνεν εν αυτοίς.
\par 6 Τότε εβόησαν προς τον Κύριον εν τη θλίψει αυτών· και ηλευθέρωσεν αυτούς από των αναγκών αυτών.
\par 7 Και ωδήγησεν αυτούς δι' ευθείας οδού, διά να υπάγωσιν εις πόλιν κατοικίας.
\par 8 Ας υμνολογώσιν εις τον Κύριον τα ελέη αυτού και τα θαυμάσια αυτού τα προς τους υιούς των ανθρώπων·
\par 9 Διότι εχόρτασε ψυχήν διψώσαν, και ψυχήν πεινώσαν ενέπλησεν από αγαθών.
\par 10 Τους καθημένους εν σκότει και σκιά θανάτου, τους δεδεμένους εν θλίψει και εν σιδήρω·
\par 11 διότι ηπείθησαν εις τα λόγια του Θεού και την βουλήν του Υψίστου κατεφρόνησαν·
\par 12 διά τούτο εταπείνωσε την καρδίαν αυτών εν κόπω· έπεσον, και δεν υπήρχεν ο βοηθών.
\par 13 Τότε εβόησαν προς τον Κύριον εν τη θλίψει αυτών, και έσωσεν αυτούς από των αναγκών αυτών·
\par 14 εξήγαγεν αυτούς εκ του σκότους και εκ της σκιάς του θανάτου και τα δεσμά αυτών συνέτριψεν.
\par 15 Ας υμνολογώσιν εις τον Κύριον τα ελέη αυτού και τα θαυμάσια αυτού τα προς τους υιούς των ανθρώπων·
\par 16 διότι συνέτριψε πύλας χαλκίνας και μοχλούς σιδηρούς κατέκοψεν.
\par 17 Οι άφρονες βασανίζονται διά τας παραβάσεις αυτών και διά τας ανομίας αυτών.
\par 18 Παν φαγητόν βδελύττεται η ψυχή αυτών, και πλησιάζουσιν έως των πυλών του θανάτου.
\par 19 Τότε βοώσι προς τον Κύριον εν τη θλίψει αυτών, και σώζει αυτούς από των αναγκών αυτών·
\par 20 αποστέλλει τον λόγον αυτού και ιατρεύει αυτούς και ελευθερόνει από της φθοράς αυτών.
\par 21 Ας υμνολογώσιν εις τον Κύριον τα ελέη αυτού, και τα θαυμάσια αυτού τα προς τους υιούς των ανθρώπων·
\par 22 και ας θυσιάζωσι θυσίας αινέσεως και ας κηρύττωσι τα έργα αυτού εν αγαλλιάσει.
\par 23 Οι καταβαίνοντες εις την θάλασσαν με πλοία, κάμνοντες εργασίας εν ύδασι πολλοίς,
\par 24 αυτοί βλέπουσι τα έργα του Κυρίου και τα θαυμάσια αυτού τα γινόμενα εις τα βάθη·
\par 25 Διότι προστάζει, και εγείρεται άνεμος καταιγίδος, και υψόνει τα κύματα αυτής.
\par 26 Αναβαίνουσιν έως των ουρανών και καταβαίνουσιν έως των αβύσσων· η ψυχή αυτών τήκεται υπό της συμφοράς.
\par 27 Σείονται και κλονίζονται ως ο μεθύων, και πάσα η σοφία αυτών χάνεται.
\par 28 Τότε κράζουσι προς τον Κύριον εν τη θλίψει αυτών, και εξάγει αυτούς από των αναγκών αυτών.
\par 29 Κατασιγάζει την ανεμοζάλην, και σιωπώσι τα κύματα αυτής.
\par 30 Και ευφραίνονται, διότι ησύχασαν· και οδηγεί αυτούς εις τον επιθυμητόν λιμένα αυτών.
\par 31 Ας υμνολογώσιν εις τον Κύριον τα ελέη αυτού και τα θαυμάσια αυτού τα προς τους υιούς των ανθρώπων·
\par 32 και ας υψόνωσιν αυτόν εν τη συνάξει του λαού, και εν τω συνεδρίω των πρεσβυτέρων ας αινώσιν αυτόν.
\par 33 Μεταβάλλει ποταμούς εις έρημον και πηγάς υδάτων εις ξηρασίαν·
\par 34 την καρποφόρον γην εις αλμυράν, διά την κακίαν των κατοικούντων εν αυτή.
\par 35 Μεταβάλλει την έρημον εις λίμνας υδάτων και την ξηράν γην εις πηγάς υδάτων.
\par 36 Και εκεί κατοικίζει τους πεινώντας, και συγκροτούσι πόλεις εις κατοίκησιν·
\par 37 και σπείρουσιν αγρούς και φυτεύουσιν αμπελώνας, οίτινες κάμνουσι καρπούς γεννήματος.
\par 38 Και ευλογεί αυτούς, και πληθύνονται σφόδρα, και δεν ολιγοστεύει τα κτήνη αυτών.
\par 39 Ολιγοστεύουσιν όμως έπειτα και ταπεινόνονται, από της στενοχωρίας, της συμφοράς και του πόνου.
\par 40 Επιχέει καταφρόνησιν επί τους άρχοντας και κάμνει αυτούς να περιπλανώνται εν ερήμω αβάτω.
\par 41 Τον δε πένητα υψόνει από της πτωχείας και καθιστά ως ποίμνια τας οικογενείας.
\par 42 Οι ευθείς βλέπουσι και ευφραίνονται· πάσα δε ανομία θέλει εμφράξει το στόμα αυτής.
\par 43 Όστις είναι σοφός ας παρατηρή ταύτα· και θέλουσιν εννοήσει τα ελέη του Κυρίου.

\chapter{108}

\par «Ωιδή Ψαλμού του Δαβίδ.» Ετοίμη είναι η καρδία μου, Θεέ· θέλω ψάλλει και θέλω ψαλμωδεί εν τη δόξη μου.
\par 2 Εξεγέρθητι, ψαλτήριον, και κιθάρα· θέλω εξεγερθή το πρωΐ.
\par 3 Θέλω σε επαινέσει, Κύριε, μεταξύ λαών, και θέλω ψαλμωδεί εις σε μεταξύ εθνών·
\par 4 διότι εμεγαλύνθη έως των ουρανών το έλεός σου και έως των νεφελών η αλήθειά σου.
\par 5 Υψώθητι, Θεέ, επί τους ουρανούς· και η δόξα σου ας ήναι εφ' όλην την γήν·
\par 6 διά να ελευθερόνωνται οι αγαπητοί σου· σώσον διά της δεξιάς σου και επάκουσόν μου.
\par 7 Ο Θεός ελάλησεν εν τω αγιαστηρίω αυτού· θέλω χαίρει, θέλω μοιράσει την Συχέμ και την κοιλάδα Σοκχώθ θέλω διαμετρήσει·
\par 8 Εμού είναι ο Γαλαάδ, εμού ο Μανασσής· ο μεν Εφραΐμ είναι η δύναμις της κεφαλής μου· ο δε Ιούδας ο νομοθέτης μου·
\par 9 Ο Μωάβ είναι η λεκάνη του πλυσίματός μου· επί τον Εδώμ θέλω ρίψει το υπόδημά μου· θέλω αλαλάξει επί την Παλαιστίνην.
\par 10 Τις θέλει με φέρει εις την περιτετειχισμένην πόλιν; τις θέλει με οδηγήσει έως Εδώμ;
\par 11 ουχί συ, Θεέ, ο απορρίψας ημάς; και δεν θέλεις εξέλθει, Θεέ, μετά των στρατευμάτων ημών;
\par 12 Βοήθησον ημάς από της θλίψεως, διότι ματαία είναι η παρά των ανθρώπων σωτηρία.
\par 13 Διά του Θεού θέλομεν κάμει ανδραγαθίας· και αυτός θέλει καταπατήσει τους εχθρούς ημών.

\chapter{109}

\par «Εις τον πρώτον μουσικόν. Ψαλμός του Δαβίδ.» Θεέ της αινέσεώς μου, μη σιωπήσης·
\par 2 διότι στόμα ασεβούς και στόμα δολίου ηνοίχθησαν επ' εμέ· ελάλησαν κατ' εμού με γλώσσαν ψευδή·
\par 3 και με λόγους μίσους με περιεκύκλωσαν και με επολέμησαν αναιτίως.
\par 4 Αντί της αγάπης μου είναι αντίδικοι εις εμέ· εγώ δε προσεύχομαι.
\par 5 Και ανταπέδωκαν εις εμέ κακόν αντί καλού, και μίσος αντί της αγάπης μου.
\par 6 Κατάστησον ασεβή επ' αυτόν· και διάβολος ας στέκη εκ δεξιών αυτού.
\par 7 Όταν κρίνηται, ας εξέλθη καταδεδικασμένος· και η προσευχή αυτού ας γείνη εις αμαρτίαν.
\par 8 Ας γείνωσιν αι ημέραι αυτού ολίγαι· άλλος ας λάβη την επισκοπήν αυτού.
\par 9 Ας γείνωσιν οι υιοί αυτού ορφανοί και η γυνή αυτού χήρα.
\par 10 Και ας περιπλανώνται πάντοτε οι υιοί αυτού και ας γείνωσιν επαίται, και ας ζητώσιν εκ των ερειπίων αυτών.
\par 11 Ας παγιδεύση ο δανειστής πάντα τα υπάρχοντα αυτού· και ας διαρπάσωσιν οι ξένοι τους κόπους αυτού.
\par 12 Ας μη υπάρχη ο ελεών αυτόν, και ας μη ήναι ο οικτείρων τα ορφανά αυτού.
\par 13 Ας εξολοθρευθώσιν οι έκγονοι αυτού· εν τη επερχομένη γενεά ας εξαλειφθή το όνομα αυτών.
\par 14 Ας έλθη εις ενθύμησιν ενώπιον του Κυρίου η ανομία των πατέρων αυτού· και η αμαρτία της μητρός αυτού ας μη εξαλειφθή·
\par 15 Ας ήναι πάντοτε ενώπιον του Κυρίου, διά να εκκόψη από της γης το μνημόσυνον αυτών.
\par 16 Διότι δεν ενεθυμήθη να κάμη έλεος· αλλά κατέτρεξεν άνθρωπον πένητα και πτωχόν, διά να θανατώση τον συντετριμμένον την καρδίαν.
\par 17 Επειδή ηγάπησε κατάραν, ας έλθη επ' αυτόν· επειδή δεν ηθέλησεν ευλογίαν, ας απομακρυνθή απ' αυτού.
\par 18 Επειδή ενεδύθη κατάραν ως ιμάτιον αυτού, ας εισέλθη ως ύδωρ εις τα εντόσθια αυτού και ως έλαιον εις τα οστά αυτού·
\par 19 Ας γείνη εις αυτόν ως το ιμάτιον, το οποίον ενδύεται και ως η ζώνη, την οποίαν πάντοτε περιζώννυται.
\par 20 Αύτη ας ήναι των αντιδίκων μου η αμοιβή παρά του Κυρίου, και των λαλούντων κακά κατά της ψυχής μου.
\par 21 Αλλά συ, Κύριε Θεέ, ενέργησον μετ' εμού διά το όνομά σου· επειδή είναι αγαθόν το έλεός σου, λύτρωσόν με.
\par 22 Διότι πτωχός και πένης είμαι, και η καρδία μου είναι πεπληγωμένη εντός μου.
\par 23 Παρήλθον ως σκιά, όταν εκκλίνη· εκτινάζομαι ως η ακρίς.
\par 24 Τα γόνατά μου ητόνησαν από της νηστείας και η σαρξ μου εξέπεσεν από του πάχους αυτής.
\par 25 Και εγώ έγεινα όνειδος εις αυτούς· ότε με είδον, εκίνησαν τας κεφαλάς αυτών.
\par 26 Βοήθησόν μοι, Κύριε ο Θεός μου· σώσον με κατά το έλεός σου·
\par 27 και ας γνωρίσωσιν ότι η χειρ σου είναι τούτο· ότι συ, Κύριε, έκαμες αυτό.
\par 28 Αυτοί θέλουσι καταράσθαι, συ δε θέλεις ευλογεί· θέλουσι σηκωθή, πλην θέλουσι καταισχυνθή· ο δε δούλός σου θέλει ευφραίνεσθαι.
\par 29 Ας ενδυθώσιν εντροπήν οι αντίδικοί μου· και ας φορέσωσιν ως επένδυμα την αισχύνην αυτών.
\par 30 Θέλω δοξολογεί σφόδρα τον Κύριον διά του στόματός μου, και εν μέσω πολλών θέλω υμνολογεί αυτόν·
\par 31 Διότι ίσταται εν τη δεξιά του πτωχού, διά να λυτρόνη αυτόν εκ των καταδικαζόντων την ψυχήν αυτού.

\chapter{110}

\par «Ψαλμός του Δαβίδ.» Είπεν ο Κύριος προς τον Κύριόν μου, Κάθου εκ δεξιών μου, εωσού θέσω τους εχθρούς σου υποπόδιον των ποδών σου.
\par 2 Εκ της Σιών θέλει εξαποστείλει ο Κύριος την ράβδον της δυνάμεώς σου· κατακυρίευε εν μέσω των εχθρών σου.
\par 3 Ο λαός σου θέλει είσθαι πρόθυμος εν τη ημέρα της δυνάμεώς σου, εν τω μεγαλοπρεπεί αγιαστηρίω αυτού· οι νέοι σου θέλουσιν είσθαι εις σε ως δρόσος, η εξερχομένη εκ της μήτρας της αυγής.
\par 4 Ώμοσεν ο Κύριος και δεν θέλει μεταμεληθή, συ είσαι ιερεύς εις τον αιώνα κατά την τάξιν Μελχισεδέκ.
\par 5 Ο Κύριος ο εκ δεξιών σου θέλει συντρίψει βασιλείς εν τη ημέρα της οργής αυτού.
\par 6 Θέλει κρίνει εν τοις έθνεσι· θέλει γεμίσει την γην πτωμάτων· Θέλει συντρίψει κεφαλήν δεσπόζοντος επί πολλών τόπων.
\par 7 Θέλει πίει εκ του χειμάρρου εν τη οδώ αυτού· διά τούτο θέλει υψώσει κεφαλήν.

\chapter{111}

\par Αινείτε τον Κύριον. Θέλω εξυμνεί τον Κύριον εν όλη καρδία, εν βουλή ευθέων και εν συνάξει.
\par 2 Μεγάλα τα έργα του Κυρίου, εξηκριβωμένα υπό πάντων των ηδυνομένων εις αυτά.
\par 3 Ένδοξον και μεγαλοπρεπές το έργον αυτού, και η δικαιοσύνη αυτού μένει εις τον αιώνα.
\par 4 Αξιομνημόνευτα έκαμε τα θαυμάσια αυτού· ελεήμων και οικτίρμων είναι ο Κύριος.
\par 5 Έδωκε τροφήν εις τους φοβουμένους αυτόν· θέλει ενθυμείσθαι διαπαντός την διαθήκην αυτού.
\par 6 Ανήγγειλε προς τον λαόν αυτού την δύναμιν των έργων αυτού, διά να δώση εις αυτούς κληρονομίαν εθνών.
\par 7 Τα έργα των χειρών αυτού είναι αλήθεια και κρίσις· αληθιναί πάσαι αι εντολαί αυτού·
\par 8 εστερεωμέναι εις τον αιώνα του αιώνος, πεποιημέναι εν αληθεία και ευθύτητι.
\par 9 Απέστειλε λύτρωσιν προς τον λαόν αυτού· διώρισε την διαθήκην αυτού εις τον αιώνα· άγιον και φοβερόν το όνομα αυτού.
\par 10 Αρχή σοφίας φόβος Κυρίου· πάντες οι πράττοντες αυτάς έχουσι σύνεσιν καλήν· η αίνεσις αυτού μένει εις τον αιώνα.

\chapter{112}

\par Αινείτε τον Κύριον. Μακάριος ο άνθρωπος ο φοβούμενος τον Κύριον· εις τας εντολάς αυτού ηδύνεται σφόδρα.
\par 2 Το σπέρμα αυτού θέλει είσθαι δυνατόν εν τη γή· η γενεά των ευθέων θέλει ευλογηθή·
\par 3 Αγαθά και πλούτη θέλουσιν είσθαι εν τω οίκω αυτού, και η δικαιοσύνη αυτού μένει εις τον αιώνα.
\par 4 Φως ανατέλλει εν τω σκότει διά τους ευθείς· είναι ελεήμων και οικτίρμων και δίκαιος.
\par 5 Ο καλός άνθρωπος ελεεί και δανείζει· οικονομεί τα πράγματα αυτού εν κρίσει.
\par 6 Βεβαίως δεν θέλει ποτέ κλονισθή· εις μνημόσυνον αιώνιον θέλει είσθαι ο δίκαιος.
\par 7 Από κακής φήμης δεν θέλει φοβηθή· η καρδία αυτού είναι στερεά, ελπίζουσα επί τον Κύριον.
\par 8 Εστηριγμένη είναι η καρδία αυτού· δεν θέλει φοβηθή, εωσού ίδη την εκδίκησιν επί τους εχθρούς αυτού.
\par 9 Εσκόρπισεν, έδωκεν εις τους πένητας· η δικαιοσύνη αυτού μένει εις τον αιώνα· το κέρας αυτού θέλει υψωθή εν δόξη.
\par 10 Ο ασεβής θέλει ιδεί και θέλει οργισθή· θέλει τρίξει τους οδόντας αυτού και θέλει αναλυθή· η επιθυμία των ασεβών θέλει απολεσθή.

\chapter{113}

\par Αινείτε τον Κύριον. Αινείτε, δούλοι του Κυρίου, αινείτε το όνομα του Κυρίου.
\par 2 Είη το όνομα Κυρίου ευλογημένον από του νυν και έως του αιώνος.
\par 3 Από ανατολών ηλίου έως δυσμών αυτού, ας αινήται το όνομα του Κυρίου.
\par 4 Ο Κύριος είναι υψηλός επί πάντα τα έθνη· επί τους ουρανούς είναι η δόξα αυτού.
\par 5 Τις ως Κύριος ο Θεός ημών; ο κατοικών εν υψηλοίς·
\par 6 Ο συγκαταβαίνων διά να επιβλέπη τα εν τω ουρανώ και τα εν τη γή·
\par 7 ο εγείρων από του χώματος τον πτωχόν και από της κοπρίας ανυψών τον πένητα,
\par 8 διά να καθίση αυτόν μετά των αρχόντων, μετά των αρχόντων του λαού αυτού·
\par 9 ο κατοικίζων την στείραν εν οίκω, μητέρα ευφραινομένην εις τέκνα. Αλληλούϊα.

\chapter{114}

\par Ότε εξήλθεν ο Ισραήλ εξ Αιγύπτου, ο οίκος του Ιακώβ εκ λαού βαρβάρου,
\par 2 Ο Ιούδας έγεινεν άγιος αυτού, ο Ισραήλ δεσποτεία αυτού.
\par 3 Η θάλασσα είδε και έφυγεν· ο Ιορδάνης εστράφη εις τα οπίσω·
\par 4 τα όρη εσκίρτησαν ως κριοί, οι λόφοι ως αρνία.
\par 5 Τι σοι συνέβη, θάλασσα, ότι έφυγες; και συ, Ιορδάνη, ότι εστράφης εις τα οπίσω;
\par 6 τα όρη, ότι εσκιρτήσατε ως κριοί; και οι λόφοι, ως αρνία;
\par 7 Τρέμε, γη, από προσώπου του Κυρίου, από προσώπου του Θεού του Ιακώβ·
\par 8 όστις μετέβαλε την πέτραν εις λίμνας υδάτων, τον σκληρόν βράχον εις πηγάς υδάτων.

\chapter{115}

\par Μη εις ημάς, Κύριε, μη εις ημάς, αλλ' εις το όνομά σου δος δόξαν, διά το έλεός σου, διά την αλήθειάν σου.
\par 2 Διά τι να είπωσι τα έθνη, και που είναι ο Θεός αυτών;
\par 3 Αλλ' ο Θεός ημών είναι εν τω ουρανώ· πάντα όσα ηθέλησεν εποίησε.
\par 4 Τα είδωλα αυτών είναι αργύριον και χρυσίον, έργα χειρών ανθρώπων·
\par 5 Στόμα έχουσι και δεν λαλούσιν· οφθαλμούς έχουσι και δεν βλέπουσιν·
\par 6 ώτα έχουσι και δεν ακούουσι· μυκτήρας έχουσι και δεν οσφραίνονται·
\par 7 Χείρας έχουσι και δεν ψηλαφώσι· πόδας έχουσι και δεν περιπατούσιν· ουδέ ομιλούσι διά του λάρυγγος αυτών.
\par 8 Όμοιοι αυτών ας γείνωσιν οι ποιούντες αυτά, πας ο ελπίζων επ' αυτά.
\par 9 Ο Ισραήλ ήλπισεν επί Κύριον· αυτός είναι βοηθός και ασπίς αυτών.
\par 10 Ο οίκος του Ααρών ήλπισεν επί Κύριον· αυτός είναι βοηθός και ασπίς αυτών.
\par 11 Οι φοβούμενοι τον Κύριον ήλπισαν επί Κύριον· αυτός είναι βοηθός και ασπίς αυτών.
\par 12 Ο Κύριος μας ενεθυμήθη· θέλει ευλογεί, θέλει ευλογεί τον οίκον Ισραήλ· θέλει ευλογεί τον οίκον Ααρών.
\par 13 Θέλει ευλογεί τους φοβουμένους τον Κύριον, τους μικρούς μετά των μεγάλων.
\par 14 Ο Κύριος θέλει αυξήσει υμάς, υμάς και τα τέκνα υμών.
\par 15 Σεις είσθε οι ευλογημένοι του Κυρίου, του ποιήσαντος τον ουρανόν και την γην.
\par 16 Οι ουρανοί των ουρανών είναι του Κυρίου, την δε γην έδωκεν εις τους υιούς των ανθρώπων.
\par 17 Οι νεκροί δεν θέλουσιν αινέσει τον Κύριον, ουδέ πάντες οι καταβαίνοντες εις τον τόπον της σιωπής·
\par 18 αλλ' ημείς θέλομεν ευλογεί τον Κύριον, από του νυν και έως του αιώνος. Αλληλούϊα.

\chapter{116}

\par Χαίρω ότι ο Κύριος εισήκουσε της φωνής μου, των δεήσεών μου·
\par 2 ότι έκλινε το ωτίον αυτού προς εμέ· και ενόσω ζω, θέλω επικαλείσθαι αυτόν.
\par 3 Πόνοι θανάτου με περιεκύκλωσαν, και στενοχωρίαι του άδου με εύρηκαν· θλίψιν και πόνον απήντησα.
\par 4 Και επεκαλέσθην το όνομα του Κυρίου· ω Κύριε, λύτρωσον την ψυχήν μου.
\par 5 Ελεήμων ο Κύριος και δίκαιος· και εύσπλαγχνος ο Θεός ημών.
\par 6 Ο Κύριος φυλάττει τους απλούς· εταλαιπωρήθην, και με έσωσεν.
\par 7 Επίστρεψον, ψυχή μου, εις την ανάπαυσίν σου, διότι ο Κύριος σε ευηργέτησε.
\par 8 Διότι ελύτρωσας την ψυχήν μου εκ θανάτου, τους οφθαλμούς μου από δακρύων, τους πόδας μου από ολισθήματος.
\par 9 Θέλω περιπατεί ενώπιον του Κυρίου εν γη ζώντων.
\par 10 Επίστευσα, διά τούτο ελάλησα· εγώ ήμην σφόδρα τεθλιμμένος·
\par 11 εγώ είπα εν τη εκπλήξει μου, πας άνθρωπος είναι ψεύστης.
\par 12 Τι να ανταποδώσω εις τον Κύριον, διά πάσας τας ευεργεσίας αυτού τας προς εμέ;
\par 13 θέλω λάβει το ποτήριον της σωτηρίας και θέλω επικαλεσθή το όνομα του Κυρίου.
\par 14 Τας ευχάς μου θέλω αποδώσει εις τον Κύριον, τώρα ενώπιον παντός του λαού αυτού.
\par 15 Πολύτιμος ενώπιον του Κυρίου ο θάνατος των οσίων αυτού.
\par 16 Ναι, Κύριε διότι είμαι δούλός σου· είμαι δούλός σου, υιός της δούλης σου· συ έλυσας τα δεσμά μου.
\par 17 Εις σε θέλω θυσιάσει θυσίαν αινέσεως και το όνομα του Κυρίου θέλω επικαλεσθή.
\par 18 Τας ευχάς μου θέλω αποδώσει εις τον Κύριον, τώρα έμπροσθεν παντός του λαού αυτού·
\par 19 εν ταις αυλαίς του οίκου του Κυρίου, εν μέσω σου, Ιερουσαλήμ. Αλληλούϊα.

\chapter{117}

\par Αινείτε τον Κύριον, πάντα τα έθνη· δοξολογείτε αυτόν, πάντες οι λαοί·
\par 2 Διότι το έλεος αυτού είναι μέγα εφ' ημάς· και η αλήθεια του Κυρίου μένει εις τον αιώνα. Αλληλούϊα.

\chapter{118}

\par Δοξολογείτε τον Κύριον, διότι είναι αγαθός, διότι το έλεος αυτού μένει εις τον αιώνα.
\par 2 Ας είπη τώρα ο Ισραήλ, ότι το έλεος αυτού μένει εις τον αιώνα.
\par 3 Ας είπη τώρα ο οίκος Ααρών, ότι το έλεος αυτού μένει εις τον αιώνα.
\par 4 Ας είπωσι τώρα οι φοβούμενοι τον Κύριον, ότι το έλεος αυτού μένει εις τον αιώνα.
\par 5 Εν θλίψει επεκαλέσθην τον Κύριον· ο Κύριος μου υπήκουσε, δους ευρυχωρίαν.
\par 6 Ο Κύριος είναι υπέρ εμού· δεν θέλω φοβηθή· τι να μοι κάμη άνθρωπος;
\par 7 Ο Κύριος είναι υπέρ εμού μεταξύ των βοηθούντων με· και εγώ θέλω ιδεί την εκδίκησιν επί τους εχθρούς μου.
\par 8 Κάλλιον να ελπίζη τις επί Κύριον, παρά να θαρρή επ' άνθρωπον.
\par 9 Κάλλιον να ελπίζη τις επί Κύριον, παρά να θαρρή επ' άρχοντας.
\par 10 Πάντα τα έθνη με περιεκύκλωσαν· αλλ' εν τω ονόματι τον Κυρίου θέλω κατατροπώσει αυτούς.
\par 11 Με περιεκύκλωσαν, ναι, με περιεκύκλωσαν πανταχόθεν· αλλ' εν τω ονόματι του Κυρίου θέλω κατατροπώσει αυτούς.
\par 12 Με περιεκύκλωσαν ως μέλισσαι· εσβέσθησαν ως πυρ ακανθών· διότι εν τω ονόματι του Κυρίου θέλω κατατροπώσει αυτούς.
\par 13 Με ώθησας δυνατά διά να πέσω· αλλ' ο Κύριος με εβοήθησε.
\par 14 Δύναμίς μου και ύμνος είναι ο Κύριος, και έγεινεν εις εμέ σωτηρία.
\par 15 Φωνή αγαλλιάσεως και σωτηρίας είναι εν σκηναίς δικαίων· η δεξιά του Κυρίου κάμνει κατορθώματα.
\par 16 Η δεξιά του Κυρίου υψώθη· η δεξιά του Κυρίου κάμνει κατορθώματα.
\par 17 Δεν θέλω αποθάνει αλλά θέλω ζήσει και θέλω διηγείσθαι τα έργα του Κυρίου.
\par 18 Αυστηρώς με επαίδευσεν ο Κύριος, αλλά δεν με παρέδωκεν εις θάνατον.
\par 19 Ανοίξατε εις εμέ τας πύλας της δικαιοσύνης· θέλω εισέλθει εις αυτάς και θέλω δοξολογήσει τον Κύριον.
\par 20 Αύτη είναι η πύλη του Κυρίου· οι δίκαιοι θέλουσιν εισέλθει εις αυτήν.
\par 21 Θέλω σε δοξολογεί, διότι μου επήκουσας και έγεινες εις εμέ σωτηρία.
\par 22 Ο λίθος, τον οποίον απεδοκίμασαν οι οικοδομούντες, ούτος έγεινε κεφαλή γωνίας·
\par 23 παρά Κυρίου έγεινεν αύτη και είναι θαυμαστή εν οφθαλμοίς ημών.
\par 24 Αύτη είναι η ημέρα, την οποίαν έκαμεν ο Κύριος· ας αγαλλιασθώμεν και ας ευφρανθώμεν εν αυτή.
\par 25 Ω Κύριε, σώσον, δέομαι· ω Κύριε, ευόδωσον, δέομαι.
\par 26 Ευλογημένος ο ερχόμενος εν ονόματι Κυρίου· σας ευλογήσαμεν εκ του οίκου του Κυρίου.
\par 27 Ο Θεός είναι ο Κύριος και έδειξε φως εις ημάς· φέρετε την θυσίαν δεδεμένην με σχοινία έως των κεράτων του θυσιαστηρίου.
\par 28 Συ είσαι ο Θεός μου, και θέλω σε δοξολογεί· ο Θεός μου, θέλω σε υψόνει.
\par 29 Δοξολογείτε τον Κύριον, διότι είναι αγαθός, διότι το έλεος αυτού μένει εις τον αιώνα.

\chapter{119}

\par Άλεφ. Μακάριοι οι άμωμοι εν οδώ· οι περιπατούντες εν τω νόμω του Κυρίου·
\par 2 Μακάριοι οι φυλάττοντες τα μαρτύρια αυτού, οι εκζητούντες αυτόν εξ όλης καρδίας·
\par 3 αυτοί βεβαίως δεν πράττουσιν ανομίαν· εν ταις οδοίς αυτού περιπατούσι.
\par 4 συ προσέταξας να φυλάττωνται ακριβώς αι εντολαί σου.
\par 5 Είθε να κατευθύνωνται αι οδοί μου, διά να φυλάττω τα διατάγματά σου
\par 6 Τότε δεν θέλω αισχυνθή, όταν επιβλέπω εις πάντα τα προστάγματά σου.
\par 7 Θέλω σε δοξολογεί εν ευθύτητι καρδίας, όταν μάθω τας κρίσεις της δικαιοσύνης σου.
\par 8 Τα διατάγματά σου θέλω φυλάττει· μη με εγκαταλίπης ολοκλήρως.
\par 9 Βεθ. Τίνι τρόπω θέλει καθαρίζει ο νέος την οδόν αυτού; φυλάττων τους λόγους σου.
\par 10 Εξ όλης της καρδίας μου σε εξεζήτησα· με μη αφήσης να αποπλανηθώ από των προσταγμάτων σου.
\par 11 Εν τη καρδία μου εφύλαξα τα λόγιά σου, διά να μη αμαρτάνω εις σε.
\par 12 Ευλογητός είσαι, Κύριε· δίδαξόν με τα διατάγματά σου.
\par 13 Διά των χειλέων μου διηγήθην πάσας τας κρίσεις του στόματός σου.
\par 14 Εν τη οδώ των μαρτυρίων σου ευφράνθην, ως διά πάντα τα πλούτη.
\par 15 Εις τας εντολάς σου θέλω μελετά, και εις τας οδούς σου θέλω ενατενίζει.
\par 16 Εις τα διατάγματά σου θέλω εντρυφά· δεν θέλω λησμονήσει τους λόγους σου.
\par 17 Γίμελ. Αντάμειψον τον δούλον σου· ούτω θέλω ζήσει, και θέλω φυλάξει τον λόγον σου.
\par 18 Άνοιξον τους οφθαλμούς μου, και θέλω βλέπει τα θαυμάσια τα εκ του νόμου σου.
\par 19 Πάροικος είμαι εγώ εν τη γή· μη κρύψης απ' εμού τα προστάγματά σου.
\par 20 Η ψυχή μου λιποθυμεί εκ του πόθου τον οποίον έχω εις τας κρίσεις σου πάντοτε.
\par 21 Συ επετίμησας τους επικαταράτους υπερηφάνους, τους εκκλίνοντας από των προσταγμάτων σου.
\par 22 Σήκωσον απ' εμού το όνειδος και την καταφρόνησιν· διότι εφύλαξα τα μαρτύριά σου.
\par 23 Άρχοντες τωόντι εκάθισαν και ελάλουν εναντίον μου· αλλ' ο δούλός σου εμελέτα εις τα διατάγματά σου.
\par 24 Τα μαρτυριά σου βεβαίως είναι η τρυφή μου και οι σύμβουλοί μου.
\par 25 Δάλεθ. Η ψυχή μου εκολλήθη εις το χώμα· ζωοποίησόν με κατά τον λόγον σου.
\par 26 Εφανέρωσα τας οδούς μου, και μου εισήκουσας· δίδαξόν με τα διατάγματά σου.
\par 27 Κάμε με να εννοώ την οδόν των εντολών σου, και θέλω μελετά εις τα θαυμάσιά σου.
\par 28 Η ψυχή μου τήκεται υπό θλίψεως· στερέωσόν με κατά τον λόγον σου.
\par 29 Απομάκρυνον απ' εμού την οδόν του ψεύδους, και χάρισόν μοι τον νόμον σου.
\par 30 Την οδόν της αληθείας εξέλεξα· προ οφθαλμών μου έθεσα τας κρίσεις σου.
\par 31 Προσεκολλήθην εις τα μαρτύριά σου· Κύριε, μη με καταισχύνης.
\par 32 Την οδόν των προσταγμάτων σου θέλω τρέχει, όταν πλατύνης την καρδίαν μου.
\par 33 Ε. Δίδαξόν με, Κύριε, την οδόν των διαταγμάτων σου, και θέλω φυλάττει αυτήν μέχρι τέλους.
\par 34 Συνέτισόν με, και θέλω φυλάττει τον νόμον σου· ναι, θέλω φυλάττει αυτόν εν όλη καρδία.
\par 35 Οδήγησόν με εις την οδόν των προσταγμάτων σου· διότι ευφραίνομαι εις αυτήν.
\par 36 Κλίνον την καρδίαν μου εις τα μαρτύριά σου και μη εις πλεονεξίαν.
\par 37 Απόστρεψον τους οφθαλμούς μου από του να βλέπωσι ματαιότητα· ζωοποίησόν με εν τη οδώ σου.
\par 38 Εκτέλεσον τον λόγον σου προς τον δούλον σου, όστις είναι δεδομένος εις τον φόβον σου.
\par 39 Αφαίρεσον το όνειδός μου, το οποίον φοβούμαι· διότι αι κρίσεις σου είναι αγαθαί.
\par 40 Ιδού, επεθύμησα τας εντολάς σου· ζωοποίησόν με διά της δικαιοσύνης σου.
\par 41 Βάου. Και ας έλθη επ εμέ το έλεός σου, Κύριε, και η σωτηρία σου κατά τον λόγον σου.
\par 42 Τότε θέλω αποκριθή προς τον ονειδίζοντά με· διότι ελπίζω επί τον λόγον σου.
\par 43 Και μη αφαιρέσης ολοτελώς από του στόματός μου τον λόγον της αληθείας· διότι ήλπισα επί τας κρίσεις σου.
\par 44 Και θέλω φυλάττει τον νόμον σου διαπαντός, εις τον αιώνα του αιώνος.
\par 45 Και θέλω περιπατεί εν ευρυχωρία· διότι εξεζήτησα τας εντολάς σου.
\par 46 Και θέλω ομιλεί περί των μαρτυρίων σου έμπροσθεν βασιλέων, και δεν θέλω αισχυνθή.
\par 47 Και θέλω εντρυφά εις τα προστάγματά σου, τα οποία ηγάπησα.
\par 48 Και θέλω υψόνει τας χείρας μου προς τα προστάγματά σου, τα οποία ηγάπησα· και θέλω μελετά εις τα διατάγματά σου.
\par 49 Ζάϊν. Ενθυμήθητι τον λόγον τον προς τον δούλον σου, εις τον οποίον με επήλπισας.
\par 50 Αύτη είναι η παρηγορία μου εν τη θλίψει μου, ότι ο λόγος σου με εζωοποίησεν.
\par 51 Οι υπερήφανοι με εχλεύαζον σφόδρα· αλλ' εγώ από του νόμου σου δεν εξέκλινα.
\par 52 Ενεθυμήθην τας απ' αιώνος κρίσεις σου, Κύριε, και παρηγορήθην.
\par 53 Φρίκη με κατέλαβεν εξ αιτίας των ασεβών, των εγκαταλειπόντων τον νόμον σου.
\par 54 Τα διατάγματά σου υπήρξαν εις εμέ ψαλμωδίαι εν τω οίκω της παροικίας μου.
\par 55 Ενεθυμήθην εν νυκτί το όνομά σου, Κύριε· και εφύλαξα τον νόμον σου.
\par 56 Τούτο έγεινεν εις εμέ, διότι εφύλαξα τας εντολάς σου.
\par 57 Χεθ. Συ, Κύριε, μερίς μου είσαι· είπα να φυλάξω τους λόγους σου.
\par 58 Παρεκάλεσα το πρόσωπόν σου εν όλη καρδία· ελέησόν με κατά τον λόγον σου.
\par 59 Διελογίσθην τας οδούς μου και έστρεψα τους πόδας μου εις τα μαρτύριά σου.
\par 60 Έσπευσα και δεν εβράδυνα να φυλάξω τα προστάγματά σου.
\par 61 Στίφη ασεβών με περιεκύκλωσαν· αλλ' εγώ δεν ελησμόνησα τον νόμον σου.
\par 62 Το μεσονύκτιον εγείρομαι διά να σε δοξολογώ διά τας κρίσεις της δικαιοσύνης σου.
\par 63 Εγώ είμαι μέτοχος πάντων των φοβουμένων σε και φυλαττόντων τας εντολάς σου.
\par 64 Η γη, Κύριε, είναι πλήρης του ελέους σου· δίδαξόν με τα διατάγματά σου.
\par 65 Τεθ. Συ, Κύριε, ευηργέτησας τον δούλον σου κατά τον λόγον σου.
\par 66 Δίδαξόν με φρόνησιν και γνώσιν· διότι επίστευσα εις τα προστάγματά σου.
\par 67 Πριν ταλαιπωρηθώ, εγώ επλανώμην· αλλά τώρα εφύλαξα τον λόγον σου.
\par 68 Συ είσαι αγαθός και αγαθοποιός· δίδαξόν με τα διατάγματά σου.
\par 69 Οι υπερήφανοι έπλεξαν κατ' εμού ψεύδος· αλλ' εγώ εν όλη καρδία θέλω φυλάττει τας εντολάς σου.
\par 70 Η καρδία αυτών έπηξεν ως πάχος· αλλ' εγώ εντρυφώ εις τον νόμον σου.
\par 71 Καλόν έγεινεν εις εμέ ότι εταλαιπωρήθην, διά να μάθω τα διατάγματά σου.
\par 72 Ο νόμος του στόματός σου είναι καλήτερος εις εμέ, υπέρ χιλιάδας χρυσίου και αργυρίου.
\par 73 Ιώδ. Αι χείρές σου με έκαμαν και με έπλασαν· συνέτισόν με, και θέλω μάθει τα προστάγματά σου.
\par 74 Οι φοβούμενοί σε θέλουσι με ιδεί και ευφρανθή, διότι ήλπισα επί τον λόγον σου.
\par 75 Γνωρίζω, Κύριε, ότι αι κρίσεις σου είναι δικαιοσύνη, και ότι πιστώς με εταλαιπώρησας.
\par 76 Ας με παρηγορήση, δέομαι, το έλεός σου, κατά τον λόγον σου τον προς τον δούλον σου.
\par 77 Ας έλθωσιν επ' εμέ οι οικτιρμοί σου, διά να ζώ· διότι ο νόμος σου είναι η τρυφή μου.
\par 78 Ας αισχυνθώσιν οι υπερήφανοι, διότι ζητούσιν αδίκως να με ανατρέψωσιν· αλλ' εγώ θέλω μελετά εις τας εντολάς σου.
\par 79 Ας επιστρέψωσιν εις εμέ οι φοβούμενοί σε, και οι γνωρίζοντες τα μαρτύριά σου·
\par 80 Ας ήναι η καρδία μου άμωμος εις τα διατάγματά σου, διά να μη αισχυνθώ.
\par 81 Καφ. Λιποθυμεί η ψυχή μου διά την σωτηρίαν σου· επί τον λόγον σου ελπίζω.
\par 82 Οι οφθαλμοί μου απέκαμον διά τον λόγον σου, λέγοντες, Πότε θέλεις με παρηγορήσει;
\par 83 Διότι έγεινα ως ασκός εν τω καπνώ· αλλά τα διατάγματά σου δεν ελησμόνησα.
\par 84 Πόσαι είναι αι ημέραι του δούλου σου; πότε θέλεις κάμει κρίσιν εναντίον των καταδιωκόντων με;
\par 85 Οι υπερήφανοι, οι εναντίοι του νόμου σου, έσκαψαν εις εμέ λάκκους.
\par 86 Πάντα τα προστάγματά σου είναι αλήθεια· αδίκως με κατατρέχουσι· βοήθησόν μοι.
\par 87 Παρ' ολίγον με κατέστρεψαν εις την γήν· αλλ' εγώ δεν εγκατέλιπον τας εντολάς σου.
\par 88 Ζωοποίησόν με κατά το έλεός σου· και θέλω φυλάξει τα μαρτύρια του στόματός σου.
\par 89 Λάμεδ. Εις τον αιώνα, Κύριε, διαμένει ο λόγος σου εν τω ουρανώ·
\par 90 η αλήθειά σου εις γενεάν και γενεάν· εθεμελίωσας την γην, και διαμένει.
\par 91 Κατά τας διατάξεις σου διαμένουσιν έως της σήμερον, διότι τα σύμπαντα είναι δούλοι σου.
\par 92 Εάν ο νόμος σου δεν ήτο η τρυφή μου, τότε ήθελον χαθή εν τη θλίψει μου.
\par 93 Εις τον αιώνα δεν θέλω λησμονήσει τας εντολάς σου, διότι εν αυταίς με εζωοποίησας.
\par 94 Σος είμαι εγώ· σώσον με· διότι τας εντολάς σου εξεζήτησα.
\par 95 Οι ασεβείς με περιέμενον διά να με αφανίσωσιν· αλλ' εγώ θέλω προσέχει εις τα μαρτύριά σου.
\par 96 Εις πάσαν τελειότητα είδον όριον· αλλ' ο νόμος σου είναι πλατύς σφόδρα.
\par 97 Μεμ. Πόσον αγαπώ τον νόμον σου· όλην την ημέραν είναι μελέτη μου.
\par 98 Διά των προσταγμάτων σου με έκαμες σοφώτερον των εχθρών μου, διότι είναι πάντοτε μετ' εμού.
\par 99 Είμαι συνετώτερος πάντων των διδασκόντων με· διότι τα μαρτύριά σου είναι μελέτη μου.
\par 100 Είμαι συνετώτερος των γερόντων· διότι εφύλαξα τας εντολάς σου.
\par 101 Από πάσης οδού πονηράς εκώλυσα τους πόδας μου, διά να φυλάξω τον λόγον σου.
\par 102 Από των κρίσεών σου δεν εξέκλινα· διότι συ με εδίδαξας.
\par 103 Πόσον γλυκείς είναι οι λόγοι σου εις τον ουρανίσκον μου· είναι υπέρ μέλι εις το στόμα μου.
\par 104 Εκ των εντολών σου έγεινα συνετός· διά τούτο εμίσησα πάσαν οδόν ψεύδους.
\par 105 Νούν. Λύχνος εις τους πόδας μου είναι ο λόγος σου και φως εις τας τρίβους μου.
\par 106 Ώμοσα και θέλω εμμένει να φυλάττω τας κρίσεις της δικαιοσύνης σου.
\par 107 Εταλαιπωρήθην σφόδρα· Κύριε, ζωοποίησόν με κατά τον λόγον σου.
\par 108 Πρόσδεξαι, δέομαι, τας προαιρετικάς προσφοράς του στόματός μου, Κύριε· και δίδαξόν με τας κρίσεις σου.
\par 109 Η ψυχή μου είναι πάντοτε εν κινδύνω· τον νόμον σου όμως δεν ελησμόνησα.
\par 110 Οι ασεβείς έστησαν εις εμέ παγίδα· αλλ' εγώ από των εντολών σου δεν εξέκλινα.
\par 111 Τα μαρτύριά σου εκληρονόμησα εις τον αιώνα· διότι ταύτα είναι η αγαλλίασις της καρδίας μου.
\par 112 Έκλινα την καρδίαν μου εις το να κάμνω τα διατάγματά σου πάντοτε μέχρι τέλους.
\par 113 Σάμεχ. Εμίσησα τους διεστραμμένους στοχασμούς· τον δε νόμον σου ηγάπησα.
\par 114 Συ είσαι η σκέπη μου και η ασπίς μου· επί τον λόγον σου ελπίζω.
\par 115 Απομακρύνθητε απ' εμού οι πονηρευόμενοι· διότι θέλω φυλάττει τα προστάγματα του Θεού μου.
\par 116 Υποστήριζέ με κατά τον λόγον σου και θέλω ζή· και μη με καταισχύνης εις την ελπίδα μου.
\par 117 Υποστήριζέ με και θέλω σωθή· και θέλω προσέχει διαπαντός εις τα διατάγματά σου.
\par 118 Συ κατεπάτησας πάντας τους εκκλίνοντας από των διαταγμάτων σου· διότι ματαία είναι η δολιότης αυτών.
\par 119 Αποσκυβαλίζεις πάντας τους πονηρούς της γής· διά τούτο ηγάπησα τα μαρτύριά σου.
\par 120 Έφριξεν η σαρξ μου από του φόβου σου, και από των κρίσεών σου εφοβήθην.
\par 121 Νγάϊν. Έκαμα κρίσιν και δικαιοσύνην· μη με παραδώσης εις τους αδικούντάς με.
\par 122 Γενού εγγυητής του δούλου σου εις καλόν· ας μη με καταθλίψωσιν οι υπερήφανοι.
\par 123 Οι οφθαλμοί μου απέκαμον διά την σωτηρίαν σου και διά τον λόγον της δικαιοσύνης σου.
\par 124 Κάμε μετά του δούλου σου κατά το έλεός σου και δίδαξόν με τα διατάγματά σου.
\par 125 Δούλος σου είμαι εγώ· συνέτισόν με, και θέλω γνωρίσει τα μαρτύριά σου.
\par 126 Καιρός είναι διά να ενεργήση ο Κύριος· ηκύρωσαν τον νόμον σου.
\par 127 Διά τούτο ηγάπησα τα προστάγματά σου υπέρ χρυσίον, και υπέρ χρυσίον καθαρόν.
\par 128 Διά τούτο εγνώρισα ορθάς πάσας τας εντολάς σου περί παντός πράγματος· και εμίσησα πάσαν οδόν ψεύδους.
\par 129 Πε. Θαυμαστά είναι τα μαρτύριά σου· διά τούτο εφύλαξεν αυτά η ψυχή μου.
\par 130 Η φανέρωσις των λόγων σου φωτίζει· συνετίζει τους απλούς.
\par 131 Ήνοιξε το στόμα μου και ανεστέναξα· διότι επεθύμησα τα προστάγματά σου.
\par 132 Επίβλεψον επ' εμέ και ελέησόν με, καθώς συνειθίζεις προς τους αγαπώντας το όνομά σου.
\par 133 Στερέωσον τα βήματά μου εις τον λόγον σου· και ας μη με κατακυριεύση μηδεμία ανομία.
\par 134 Λύτρωσόν με από καταδυναστείας ανθρώπων, και θέλω φυλάττει τας εντολάς σου.
\par 135 Επίφανον το πρόσωπόν σου επί τον δούλον σου, και δίδαξόν με τα διατάγματά σου.
\par 136 Ρύακας υδάτων κατεβίβασαν οι οφθαλμοί μου, επειδή δεν φυλάττουσι τον νόμον σου.
\par 137 Τσάδε. Δίκαιος είσαι, Κύριε, και ευθείαι αι κρίσεις σου.
\par 138 Τα μαρτύριά σου, τα οποία διέταξας, είναι δικαιοσύνη και υπερτάτη αλήθεια.
\par 139 Ο ζήλος μου με κατέφαγε, διότι ελησμόνησαν τους λόγους σου οι εχθροί μου.
\par 140 Ο λόγος σου είναι κεκαθαρισμένος σφόδρα· διά τούτο ο δούλός σου αγαπά αυτόν.
\par 141 Μικρός είμαι και εξουδενωμένος· δεν ελησμόνησα όμως τας εντολάς σου.
\par 142 Η δικαιοσύνη σου είναι δικαιοσύνη εις τον αιώνα, και ο νόμος σου αλήθεια.
\par 143 Θλίψεις και στενοχωρίαι με εύρηκαν· τα προστάγματά σου όμως είναι η χαρά μου.
\par 144 Τα μαρτύριά σου είναι δικαιοσύνη εις τον αιώνα· Συνέτισόν με και θέλω ζήσει.
\par 145 Κοφ. Έκραξα εν όλη καρδία· άκουσόν μου, Κύριε, και θέλω φυλάξει τα διατάγματά σου.
\par 146 Έκραξα προς σέ· σώσον με, και θέλω φυλάξει τα μαρτύριά σου.
\par 147 Προέλαβον την αυγήν και έκραξα· ήλπισα επί τον λόγον σου.
\par 148 Οι οφθαλμοί μου προλαμβάνουσι τας νυκτοφυλακάς, διά να μελετώ εις τον λόγον σου.
\par 149 Άκουσον της φωνής μου κατά το έλεός σου· ζωοποίησόν με, Κύριε, κατά την κρίσιν σου.
\par 150 Επλησίασαν οι ακολουθούντες την πονηρίαν· εξέκλιναν από του νόμου σου.
\par 151 Συ, Κύριε, είσαι πλησίον, και πάντα τα προστάγματά σου είναι αλήθεια.
\par 152 Προ πολλού εγνώρισα εκ των μαρτυρίων σου, ότι εις τον αιώνα εθεμελίωσας αυτά.
\par 153 Ρες. Ιδέ την θλίψιν μου και ελευθέρωσόν με· διότι δεν ελησμόνησα τον νόμον σου.
\par 154 Δίκασον την δίκην μου και λύτρωσόν με· ζωοποίησόν με κατά τον λόγον σου.
\par 155 Μακράν από ασεβών η σωτηρία· διότι δεν εκζητούσι τα διατάγματά σου.
\par 156 Μεγάλοι οι οικτιρμοί σου, Κύριε· ζωοποίησόν με κατά τας κρίσεις σου.
\par 157 Πολλοί είναι οι καταδιώκοντές με και οι θλίβοντές με· αλλ' από των μαρτυρίων σου δεν εξέκλινα.
\par 158 Είδον τους παραβάτας και εταράχθην· διότι δεν εφύλαξαν τον λόγον σου.
\par 159 Ιδέ πόσον αγαπώ τας εντολάς σου· Κύριε, ζωοποίησόν με κατά το έλεός σου.
\par 160 Το κεφάλαιον του λόγου σου είναι η αλήθεια· και εις τον αιώνα μένουσι πάσαι αι κρίσεις της δικαιοσύνης σου.
\par 161 Σχίν. Άρχοντες με κατεδίωξαν αναιτίως· αλλ' η καρδία μου τρέμει από του λόγου σου.
\par 162 Αγάλλομαι εις τον λόγον σου, ως ο ευρίσκων λάφυρα πολλά.
\par 163 Μισώ και βδελύττομαι το ψεύδος· τον νόμον σου αγαπώ.
\par 164 Επτάκις της ημέρας σε αινώ, διά τας κρίσεις της δικαιοσύνης σου.
\par 165 Ειρήνην πολλήν έχουσιν οι αγαπώντες τον νόμον σου· και εις αυτούς δεν υπάρχει πρόσκομμα.
\par 166 Ήλπισα επί την σωτηρίαν σου, Κύριε· και έπραξα τα προστάγματά σου.
\par 167 Εφύλαξεν η ψυχή μου τα μαρτύριά σου· και ηγάπησα αυτά σφόδρα.
\par 168 Εφύλαξα τας εντολάς σου και τα μαρτύριά σου· διότι πάσαι αι οδοί μου είναι ενώπιόν σου.
\par 169 Ταυ. Ας πλησιάση η κραυγή μου ενώπιόν σου, Κύριε· συνέτισόν με κατά τον λόγον σου.
\par 170 Ας έλθη η δέησίς μου ενώπιόν σου· λύτρωσόν με κατά τον λόγον σου.
\par 171 Τα χείλη μου θέλουσι προφέρει ύμνον, όταν με διδάξης τα διατάγματά σου.
\par 172 Η γλώσσα μου θέλει λαλεί τον λόγον σου· διότι πάντα τα προστάγματά σου είναι δικαιοσύνη.
\par 173 Ας ήναι η χειρ σου εις βοήθειάν μου· διότι εξέλεξα τας εντολάς σου.
\par 174 Επεθύμησα την σωτηρίαν σου, Κύριε· και ο νόμος σου είναι τρυφή μου.
\par 175 Ας ζήση η ψυχή μου και θέλει σε αινεί· και αι κρίσεις σου ας με βοηθώσι.
\par 176 Περιεπλανήθην ως πρόβατον απολωλός· ζήτησον τον δούλον σου· διότι δεν ελησμόνησα τα προστάγματά σου.

\chapter{120}

\par «Ωδή των Αναβαθμών.» Εν τη θλίψει μου έκραξα προς τον Κύριον, και εισήκουσέ μου.
\par 2 Κύριε, λύτρωσον την ψυχήν μου από χειλέων ψευδών, από γλώσσης δολίας.
\par 3 Τι θέλει σοι δώσει ή τι θέλει σοι προσθέσει, η δολία γλώσσα;
\par 4 Τα ηκονημένα βέλη του δυνατού, μετά ανθράκων αρκεύθου.
\par 5 Ουαί εις εμέ, διότι παροικώ εν Μεσέχ, κατοικώ εν ταις σκηναίς του Κηδάρ·
\par 6 Πολύν καιρόν κατώκησεν η ψυχή μου μετά των μισούντων την ειρήνην.
\par 7 Εγώ αγαπώ την ειρήνην· αλλ' όταν ομιλώ, αυτοί ετοιμάζονται διά πόλεμον.

\chapter{121}

\par «Ωδή των Αναβαθμών.» Υψόνω τους οφθαλμούς μου προς τα όρη· πόθεν θέλει ελθεί η βοήθειά μου;
\par 2 Η βοήθειά μου έρχεται από του Κυρίου, του ποιήσαντος τον ουρανόν και την γην.
\par 3 Δεν θέλει αφήσει να κλονισθή ο πους σου· ουδέ θέλει νυστάξει ο φυλάττων σε.
\par 4 Ιδού, δεν θέλει νυστάξει ουδέ θέλει αποκοιμηθή, ο φυλάττων τον Ισραήλ.
\par 5 Ο Κύριος είναι ο φύλαξ σου· ο Κύριος είναι η σκέπη σου εκ δεξιών σου.
\par 6 Την ημέραν ο ήλιος δεν θέλει σε βλάψει, ουδέ η σελήνη την νύκτα.
\par 7 Ο Κύριος θέλει σε φυλάττει από παντός κακού· θέλει φυλάττει την ψυχήν σου.
\par 8 Ο Κύριος θέλει φυλάττει την έξοδόν σου και την είσοδόν σου, από του νυν και έως του αιώνος.

\chapter{122}

\par «Ωιδή των Αναβαθμών, του Δαβίδ.» Ευφράνθην ότε μοι είπον, Ας υπάγωμεν εις τον οίκον του Κυρίου·
\par 2 Οι πόδες ημών θέλουσιν ίστασθαι εν ταις πύλαις σου, Ιερουσαλήμ·
\par 3 Ιερουσαλήμ, η ωκοδομημένη ως πόλις συνηρμοσμένη ομού.
\par 4 Εκεί αναβαίνουσιν αι φυλαί, αι φυλαί του Κυρίου, κατά το διατεταγμένον εις τον Ισραήλ, διά να δοξολογήσωσι το όνομα του Κυρίου.
\par 5 Διότι εκεί ετέθησαν θρόνοι διά κρίσιν, οι θρόνοι του οίκου του Δαβίδ.
\par 6 Ζητείτε την ειρήνην της Ιερουσαλήμ· ας ευτυχώσιν οι αγαπώντές σε.
\par 7 Ας ήναι ειρήνη εις τα τείχη σου, αφθονία εις τα παλάτιά σου.
\par 8 Ένεκεν των αδελφών μου και των πλησίον μου, θέλω λέγει τώρα, Ειρήνη εις σέ·
\par 9 Ένεκεν του οίκου Κυρίου του Θεού ημών, θέλω ζητεί το καλόν σου.

\chapter{123}

\par «Ωιδή των Αναβαθμών.» Ύψωσα τους οφθαλμούς μου προς σε τον κατοικούντα εν ουρανοίς.
\par 2 Ιδού, καθώς οι οφθαλμοί των δούλων ατενίζουσιν εις την χείρα των κυρίων αυτών, καθώς οι οφθαλμοί της δούλης εις την χείρα της κυρίας αυτής, ούτως οι οφθαλμοί ημών προς Κύριον τον Θεόν ημών, εωσού ελεήση ημάς.
\par 3 Ελέησον ημάς, Κύριε, ελέησον ημάς διότι εχορτάσθημεν σφόδρα από εξουδενώσεως.
\par 4 Καθ' υπερβολήν εχορτάσθη η ψυχή ημών από της ύβρεως των τρυφώντων, από της εξουδενώσεως των υπερηφάνων.

\chapter{124}

\par «Ωιδή των Αναβαθμών, του Δαβίδ.» Αν δεν ήτο ο Κύριος μεθ' ημών, ας είπη τώρα ο Ισραήλ·
\par 2 αν δεν ήτο ο Κύριος μεθ' ημών, ότε εσηκώθησαν άνθρωποι εφ' ημάς,
\par 3 ζώντας ήθελον μας καταπίει τότε, ενώ ο θυμός αυτών εφλέγετο εναντίον ημών·
\par 4 Τότε τα ύδατα ήθελον μας καταποντίσει, ο χείμαρρος ήθελε περάσει επάνωθεν της ψυχής ημών·
\par 5 τότε τα ύδατα τα επηρμένα ήθελον περάσει επάνωθεν της ψυχής ημών.
\par 6 Ευλογητός Κύριος, όστις δεν παρέδωκεν ημάς θήραμα εις τους οδόντας αυτών.
\par 7 Η ψυχή ημών ελυτρώθη ως πτηνόν από της παγίδος των θηρευτών· η παγίς συνετρίβη, και ημείς ελυτρώθημεν.
\par 8 Η βοήθεια ημών είναι εν τω ονόματι του Κυρίου, του ποιήσαντος τον ουρανόν και την γην.

\chapter{125}

\par «Ωιδή των Αναβαθμών.» Οι πεποιθότες επί Κύριον είναι ως το όρος Σιών, το οποίον δεν θέλει σαλευθή· εις τον αιώνα διαμένει.
\par 2 Καθώς η Ιερουσαλήμ κυκλόνεται υπό των ορέων, ούτως ο Κύριος κυκλόνει τον λαόν αυτού από του νυν και έως του αιώνος.
\par 3 Διότι δεν θέλει διαμένει η ράβδος της ασεβείας επί τον κλήρον των δικαίων, διά να μη εκτείνωσιν οι δίκαιοι τας χείρας αυτών εις την ανομίαν.
\par 4 Αγαθοποίησον, Κύριε, τους αγαθούς και τους ευθείς την καρδίαν.
\par 5 Τους δε εκκλίνοντας εις τας σκολιάς οδούς των, ο Κύριος θέλει απαγάγει αυτούς μετά των εργαζομένων την ανομίαν. Ειρήνη επί τον Ισραήλ.

\chapter{126}

\par «Ωιδή των Αναβαθμών.» Ότε ο Κύριος επανέφερε τους αιχμαλώτους της Σιών, ήμεθα ως οι ονειρευόμενοι.
\par 2 Τότε ενεπλήσθη το στόμα ημών από γέλωτος και η γλώσσα ημών από αγαλλιάσεως· τότε έλεγον μεταξύ των εθνών, Μεγαλεία έκαμε δι' αυτούς ο Κύριος.
\par 3 Μεγαλεία έκαμεν ο Κύριος δι' ημάς· ενεπλήσθημεν χαράς.
\par 4 Επίστρεψον, Κύριε, τους αιχμαλώτους ημών, ως τους χειμάρρους εν τω νότω.
\par 5 Οι σπείροντες μετά δακρύων εν αγαλλιάσει θέλουσι θερίσει.
\par 6 Όστις εξέρχεται και κλαίει, βαστάζων σπόρον πολύτιμον, ούτος βεβαίως θέλει επιστρέψει εν αγαλλιάσει, βαστάζων τα χειρόβολα αυτού.

\chapter{127}

\par «Ωιδή των Αναβαθμών, του Σολομώντος.» Εάν ο Κύριος δεν οικοδομήση οίκον, εις μάτην κοπιάζουσιν οι οικοδομούντες αυτόν· εάν ο Κύριος δεν φυλάξη πόλιν, εις μάτην αγρυπνεί ο φυλάττων.
\par 2 Μάταιον είναι εις εσάς να σηκόνησθε πρωΐ, να πλαγιάζητε αργά, τρώγοντες τον άρτον του κόπου· ο Κύριος βεβαίως δίδει ύπνον εις τον αγαπητόν αυτού.
\par 3 Ιδού, κληρονομία παρά του Κυρίου είναι τα τέκνα· μισθός αυτού ο καρπός της κοιλίας.
\par 4 Καθώς είναι τα βέλη εν τη χειρί του δυνατού, ούτως οι υιοί της νεότητος.
\par 5 Μακάριος ο άνθρωπος, όστις εγέμισε την βελοθήκην αυτού εκ τούτων· οι τοιούτοι δεν θέλουσι καταισχυνθή, όταν λαλώσι μετά των εχθρών εν τη πύλη.

\chapter{128}

\par «Ωιδή των Αναβαθμών.» Μακάριος πας ο φοβούμενος τον Κύριον, ο περιπατών εν ταις οδοίς αυτού.
\par 2 Διότι θέλεις τρώγει από του κόπου των χειρών σου· μακάριος θέλεις είσθαι, και ευτυχία εις σε.
\par 3 Η γυνή σου θέλει είσθαι ως άμπελος εύκαρπος εις τα πλάγια της οικίας σου· οι υιοί σου ως νεόφυτα ελαιών κύκλω της τραπέζης σου.
\par 4 Ιδού, ούτω θέλει ευλογηθή ο άνθρωπος ο φοβούμενος τον Κύριον.
\par 5 Ο Κύριος θέλει σε ευλογήσει εκ της Σιών, και θέλεις ιδεί το καλόν της Ιερουσαλήμ πάσας τας ημέρας της ζωής σου·
\par 6 και θέλεις ιδεί υιούς των υιών σου· ειρήνη επί τον Ισραήλ.

\chapter{129}

\par «Ωιδή των Αναβαθμών.» Πολλάκις με επολέμησαν εκ νεότητός μου, ας είπη τώρα ο Ισραήλ·
\par 2 Πολλάκις με επολέμησαν εκ νεότητός μου· αλλά δεν υπερίσχυσαν εναντίον μου.
\par 3 Οι γεωργοί ηροτρίασαν επί των νώτων μου· έσυραν μακρά τα αυλάκια αυτών.
\par 4 Αλλά δίκαιος ο Κύριος· κατέκοψε τα σχοινία των ασεβών.
\par 5 Ας αισχυνθώσι και ας στραφώσιν εις τα οπίσω πάντες οι μισούντες την Σιών.
\par 6 Ας γείνωσιν ως ο χόρτος των δωμάτων, όστις πριν εκριζωθή ξηραίνεται·
\par 7 από του οποίου δεν γεμίζει ο θεριστής την χείρα αυτού, ουδέ ο δένων τα χειρόβολα τον κόλπον αυτού·
\par 8 ώστε οι διαβάται δεν θέλουσιν ειπεί, Ευλογία Κυρίου εφ' υμάς· σας ευλογούμεν εν ονόματι Κυρίου.

\chapter{130}

\par «Ωιδή των Αναβαθμών.» Εκ βαθέων έκραξα προς σε, Κύριε.
\par 2 Κύριε, εισάκουσον της φωνής μου· ας ήναι τα ώτα σου προσεκτικά εις την φωνήν των δεήσεών μου.
\par 3 Εάν, Κύριε, παρατηρήσης ανομίας, Κύριε, τις θέλει δυνηθή να σταθή;
\par 4 Παρά σοι όμως είναι συγχώρησις, διά να σε φοβώνται.
\par 5 Προσέμεινα τον Κύριον, προσέμεινεν η ψυχή μου, και ήλπισα επί τον λόγον αυτού.
\par 6 Η ψυχή μου προσμένει τον Κύριον, μάλλον παρά τους προσμένοντας την αυγήν, ναι, τους προσμένοντας την αυγήν.
\par 7 Ας ελπίζη ο Ισραήλ επί τον Κύριον· διότι παρά τω Κυρίω είναι έλεος, και λύτρωσις πολλή παρ' αυτώ·
\par 8 και αυτός θέλει λυτρώσει τον Ισραήλ από πασών των ανομιών αυτού.

\chapter{131}

\par «Ωιδή των Αναβαθμών, του Δαβίδ.» Κύριε, δεν υπερηφανεύθη η καρδία μου ουδέ υψώθησαν οι οφθαλμοί μου· ουδέ περιπατώ εις πράγματα μεγάλα και υψηλότερα υπέρ εμέ.
\par 2 Βεβαίως, υπέταξα και καθησύχασα την ψυχήν μου, ως το απογεγαλακτισμένον παιδίον πλησίον της μητρός αυτού· η ψυχή μου είναι εν εμοί ως απογεγαλακτισμένον παιδίον.
\par 3 Ας ελπίζη ο Ισραήλ επί τον Κύριον, από του νυν και έως του αιώνος.

\chapter{132}

\par «Ωιδή των Αναβαθμών.» Ενθυμήθητι, Κύριε, τον Δαβίδ, και πάντας τους αγώνας αυτού·
\par 2 πως ώμοσε προς τον Κύριον και έκαμεν ευχήν εις τον ισχυρόν Θεόν του Ιακώβ·
\par 3 Δεν θέλω εισέλθει υπό την στέγην του οίκου μου, δεν θέλω αναβή εις την κλίνην της στρωμνής μου,
\par 4 δεν θέλω δώσει ύπνον εις τους οφθαλμούς μου, νυσταγμόν εις τα βλέφαρά μου,
\par 5 εωσού εύρω τόπον διά τον Κύριον, κατοικίαν διά τον ισχυρόν Θεόν του Ιακώβ.
\par 6 Ιδού, ηκούσαμεν περί αυτής εν Εφραθά· ευρήκαμεν αυτήν εις τας πεδιάδας του Ιαάρ.
\par 7 Ας εισέλθωμεν εις τας σκηνάς αυτού· ας προσκυνήσωμεν εις το υποπόδιον των ποδών αυτού.
\par 8 Ανάστηθι, Κύριε, εις την ανάπαυσίν σου, συ και η κιβωτός της δυνάμεώς σου.
\par 9 Οι ιερείς σου ας ενδυθώσι δικαιοσύνην, και οι όσιοί σου ας αγάλλωνται.
\par 10 Ένεκεν Δαβίδ του δούλου σου μη αποστρέψης το πρόσωπον του κεχρισμένου σου.
\par 11 Ώμοσεν ο Κύριος αλήθειαν προς τον Δαβίδ, δεν θέλει αθετήσει αυτήν, Εκ του καρπού του σώματός σου θέλω θέσει επί τον θρόνον σου.
\par 12 Εάν φυλάξωσιν οι υιοί σου την διαθήκην μου, και τα μαρτύριά μου τα οποία θέλω διδάξει αυτούς, και οι υιοί αυτών θέλουσι καθίσει διαπαντός επί του θρόνου σου.
\par 13 Διότι εξέλεξεν ο Κύριος την Σιών· ευηρεστήθη να κατοική εν αυτή.
\par 14 Αύτη είναι η ανάπαυσίς μου εις τον αιώνα του αιώνος· ενταύθα θέλω κατοικεί, διότι ηγάπησα αυτήν.
\par 15 Θέλω ευλογήσει εν ευλογία τας τροφάς αυτής· τους πτωχούς αυτής θέλω χορτάσει άρτον·
\par 16 και τους ιερείς αυτής θέλω ενδύσει σωτηρίαν· και οι όσιοι αυτής θέλουσιν αγάλλεσθαι εν αγαλλιάσει.
\par 17 Εκεί θέλω κάμει να βλαστήση κέρας εις τον Δαβίδ· ητοίμασα λύχνον διά τον κεχρισμένον μου.
\par 18 Τους εχθρούς αυτού θέλω ενδύσει αισχύνην· επί δε αυτόν θέλει ανθεί το διάδημα αυτού.

\chapter{133}

\par «Ωιδή των Αναβαθμών, του Δαβίδ.» Ιδού, τι καλόν και τι τερπνόν, να συγκατοικώσιν εν ομονοία αδελφοί.
\par 2 Είναι ως το πολύτιμον μύρον επί την κεφαλήν, το καταβαίνον επί τον πώγωνα, τον πώγωνα του Ααρών· το καταβαίνον επί το στόμιον του ενδύματος αυτού·
\par 3 ως η δρόσος του Αερμών, η καταβαίνουσα επί τα όρη της Σιών· διότι εκεί διώρισεν ο Κύριος την ευλογίαν, ζωήν έως του αιώνος.

\chapter{134}

\par «Ωιδή των Αναβαθμών.» Ιδού, ευλογείτε τον Κύριον, πάντες οι δούλοι του Κυρίου, οι ιστάμενοι την νύκτα εν τω οίκω του Κυρίου.
\par 2 Υψώσατε τας χείρας σας εις τα άγια και ευλογείτε τον Κύριον.
\par 3 Να σε ευλογήση ο Κύριος εκ Σιών, ο ποιήσας τον ουρανόν και την γην.

\chapter{135}

\par Αινείτε τον Κύριον. Αινείτε το όνομα του Κυρίου· αινείτε, δούλοι του Κυρίου,
\par 2 Οι ιστάμενοι εν τω οίκω του Κυρίου, εν ταις αυλαίς του οίκου του Θεού ημών.
\par 3 Αινείτε τον Κύριον, διότι αγαθός ο Κύριος· ψαλμωδήσατε εις το όνομα αυτού, διότι είναι τερπνόν.
\par 4 Διότι τον Ιακώβ εξέλεξεν εις εαυτόν ο Κύριος, τον Ισραήλ εις θησαυρόν αυτού.
\par 5 Διότι εγώ εγνώρισα ότι μέγας ο Κύριος· και ο Κύριος ημών είναι υπέρ πάντας τους θεούς.
\par 6 Πάντα όσα ηθέλησεν ο Κύριος εποίησεν, εν τω ουρανώ και εν τη γη, εν ταις θαλάσσαις και εν πάσαις ταις αβύσσοις.
\par 7 Αναβιβάζει νεφέλας από των εσχάτων της γής· κάμνει αστραπάς διά βροχήν· εκβάλλει ανέμους εκ των θησαυρών αυτού.
\par 8 Όστις επάταξε τα πρωτότοκα της Αιγύπτου, από ανθρώπου έως κτήνους·
\par 9 εξαπέστειλε σημεία και τέρατα εις το μέσον σου, Αίγυπτε, επί τον Φαραώ και επί πάντας τους δούλους αυτού.
\par 10 Όστις επάταξεν έθνη μεγάλα και απέκτεινε βασιλείς κραταιούς·
\par 11 τον Σηών, βασιλέα των Αμορραίων, και τον Ωγ, βασιλέα της Βασάν, και πάσας τας βασιλείας Χαναάν·
\par 12 και έδωκε την γην αυτών κληρονομίαν, κληρονομίαν εις τον Ισραήλ τον λαόν αυτού.
\par 13 Το όνομά σου, Κύριε, μένει εις τον αιώνα· το μνημόσυνόν σου, Κύριε, εις γενεάν και γενεάν.
\par 14 Διότι θέλει κρίνει ο Κύριος τον λαόν αυτού· και τους δούλους αυτού θέλει ελεήσει.
\par 15 Τα είδωλα των εθνών είναι αργύριον και χρυσίον, έργον χειρών ανθρώπου.
\par 16 Στόμα έχουσι και δεν λαλούσιν· οφθαλμούς έχουσι και δεν βλέπουσιν.
\par 17 Ώτα έχουσι και δεν ακούουσιν· ουδέ είναι πνοή εν τω στόματι αυτών.
\par 18 Όμοιοι αυτών ας γείνωσιν οι ποιούντες αυτά· πας ο ελπίζων επ' αυτά
\par 19 Οίκος Ισραήλ, ευλογήσατε τον Κύριον· οίκος Ααρών, ευλογήσατε τον Κύριον·
\par 20 οίκος Λευΐ, ευλογήσατε τον Κύριον· οι φοβούμενοι τον Κύριον, ευλογήσατε τον Κύριον.
\par 21 Ευλογητός ο Κύριος εν Σιών, ο κατοικών εν Ιερουσαλήμ. Αλληλούϊα.

\chapter{136}

\par Δοξολογείτε τον Κύριον, διότι είναι αγαθός, διότι εις τον αιώνα το έλεος αυτού.
\par 2 Δοξολογείτε τον Θεόν των θεών· διότι εις τον αιώνα το έλεος αυτού.
\par 3 Δοξολογείτε τον Κύριον των κυρίων· διότι εις τον αιώνα το έλεος αυτού.
\par 4 Τον μόνον ποιούντα θαυμάσια μεγάλα· διότι εις τον αιώνα το έλεος αυτού.
\par 5 Τον ποιήσαντα τους ουρανούς εν συνέσει· διότι εις τον αιώνα το έλεος αυτού.
\par 6 Τον στερεώσαντα την γην επί των υδάτων· διότι εις τον αιώνα το έλεος αυτού.
\par 7 Τον ποιήσαντα τους φωστήρας τους μεγάλους· διότι εις τον αιώνα το έλεος αυτού·
\par 8 τον ήλιον, διά να εξουσιάζη επί της ημέρας· διότι εις τον αιώνα το έλεος αυτού·
\par 9 την σελήνην και τους αστέρας, διά να εξουσιάζωσιν επί της νυκτός· διότι εις τον αιώνα το έλεος αυτού.
\par 10 Τον πατάξαντα την Αίγυπτον εις τα πρωτότοκα αυτής· διότι εις τον αιώνα το έλεος αυτού·
\par 11 και εξαγαγόντα τον Ισραήλ εκ μέσου αυτής· διότι εις τον αιώνα το έλεος αυτού·
\par 12 Εν χειρί κραταιά και εν βραχίονι ηπλωμένω· διότι εις τον αιώνα το έλεος αυτού.
\par 13 Τον διαιρέσαντα την Ερυθράν θάλασσαν εις δύο μέρη· διότι εις τον αιώνα το έλεος αυτού·
\par 14 και διαβιβάσαντα τον Ισραήλ διά μέσου αυτής· διότι εις τον αιώνα το έλεος αυτού·
\par 15 και καταστρέψαντα τον Φαραώ και το στράτευμα αυτού εν τη Ερυθρά θαλάσση· διότι εις τον αιώνα το έλεος αυτού.
\par 16 Τον οδηγήσαντα τον λαόν αυτού εν τη ερήμω· διότι εις τον αιώνα το έλεος αυτού.
\par 17 Τον πατάξαντα βασιλείς μεγάλους· διότι εις τον αιώνα το έλεος αυτού·
\par 18 και αποκτείναντα βασιλείς κραταιούς· διότι εις τον αιώνα το έλεος αυτού·
\par 19 τον Σηών, βασιλέα των Αμορραίων· διότι εις τον αιώνα το έλεος αυτού·
\par 20 και τον Ωγ βασιλέα της Βασάν· διότι εις τον αιώνα το έλεος αυτού·
\par 21 και δόντα την γην αυτήν εις κληρονομίαν· διότι εις τον αιώνα το έλεος αυτού·
\par 22 κληρονομίαν εις τον Ισραήλ τον δούλον αυτού· διότι εις τον αιώνα το έλεος αυτού.
\par 23 Τον μνησθέντα ημών εν τη ταπεινώσει ημών· διότι εις τον αιώνα το έλεος αυτού·
\par 24 και λυτρώσαντα ημάς εκ των εχθρών ημών· διότι εις τον αιώνα το έλεος αυτού.
\par 25 Τον διδόντα τροφήν εις πάσαν σάρκα· διότι εις τον αιώνα το έλεος αυτού.
\par 26 Δοξολογείτε τον Θεόν του ουρανού· διότι εις τον αιώνα το έλεος αυτού.

\chapter{137}

\par Επί των ποταμών Βαβυλώνος, εκεί εκαθίσαμεν και εκλαύσαμεν, ότε ενεθυμήθημεν την Σιών.
\par 2 Επί τας ιτέας εν μέσω αυτής εκρεμάσαμεν τας κιθάρας ημών.
\par 3 Διότι οι αιχμαλωτίσαντες ημάς εκεί εζήτησαν παρ' ημών λόγους ασμάτων· και οι ερημώσαντες ημάς ύμνον, λέγοντες, Ψάλατε εις ημάς εκ των ωδών της Σιών.
\par 4 Πως να ψάλωμεν την ωδήν του Κυρίου επί ξένης γης;
\par 5 Εάν σε λησμονήσω, Ιερουσαλήμ, ας λησμονήση η δεξιά μου
\par 6 Ας κολληθή η γλώσσα μου εις τον ουρανίσκον μου, εάν δεν σε ενθυμώμαι· εάν δεν προτάξω την Ιερουσαλήμ εις την αρχήν της ευφροσύνης μου
\par 7 Μνήσθητι, Κύριε, των υιών Εδώμ, οίτινες την ημέραν της Ιερουσαλήμ έλεγον, Κατεδαφίσατε, κατεδαφίσατε αυτήν έως των θεμελίων αυτής.
\par 8 Θυγάτηρ Βαβυλώνος, η μέλλουσα να ερημωθής, μακάριος όστις σοι ανταποδώση την ανταμοιβήν των όσα έπραξας εις ημάς
\par 9 Μακάριος όστις πιάση και ρίψη τα νήπιά σου επί την πέτραν

\chapter{138}

\par «Ψαλμός του Δαβίδ.» Θέλω σε δοξολογήσει εν όλη καρδία μου· θέλω ψαλμωδήσει εις σε έμπροσθεν των θεών.
\par 2 Θέλω προσκυνήσει προς τον ναόν τον άγιόν σου· και θέλω δοξολογήσει το όνομά σου διά το έλεός σου και διά την αλήθειάν σου· διότι εμεγάλυνας τον λόγον σου υπέρ πάσαν την φήμην σου.
\par 3 Καθ' ην ημέραν έκραξα, μου εισήκουσας· με ενίσχυσας με δύναμιν εν τη ψυχή μου.
\par 4 Θέλουσι σε δοξολογήσει, Κύριε, πάντες οι βασιλείς της γης, όταν ακούσωσι τους λόγους του στόματός σου·
\par 5 και θέλουσι ψάλλει εν ταις οδοίς του Κυρίου, ότι μεγάλη η δόξα του Κυρίου·
\par 6 ότι ο Κύριος είναι υψηλός και επιβλέπει επί τον ταπεινόν· τον δε υψηλόφρονα γινώσκει μακρόθεν.
\par 7 Εάν περιπατήσω εν μέσω στενοχωρίας, θέλεις με ζωοποιήσει· θέλεις εκτείνει την χείρα σου κατά της οργής των εχθρών μου· και η δεξιά σου θέλει με σώσει.
\par 8 Ο Κύριος θέλει εκτελέσει τα περί εμού· Κύριε, το έλεός σου μένει εις τον αιώνα· τα έργα των χειρών σου μη παραβλέψης.

\chapter{139}

\par «Εις τον πρώτον μουσικόν. Ψαλμός του Δαβίδ.» Κύριε, εδοκίμασάς με και με εγνώρισας.
\par 2 Συ γνωρίζεις το κάθισμά μου και την έγερσίν μου· νοείς τους λογισμούς μου από μακρόθεν.
\par 3 Εξερευνάς το περιπάτημά μου και το πλαγίασμά μου και πάσας τας οδούς μου γνωρίζεις.
\par 4 Διότι και πριν έλθη ο λόγος εις την γλώσσαν μου, ιδού, Κύριε, γνωρίζεις το παν.
\par 5 Με περικυκλόνεις όπισθεν και έμπροσθεν, και έθεσας επ' εμέ την χείρα σου.
\par 6 Η γνώσις αύτη είναι υπερθαύμαστος εις εμέ· είναι υψηλή· δεν δύναμαι να φθάσω εις αυτήν.
\par 7 Που να υπάγω από του πνεύματός σου; και από του προσώπου σου που να φύγω;
\par 8 Εάν αναβώ εις τον ουρανόν, είσαι εκεί· εάν πλαγιάσω εις τον άδην, ιδού, συ.
\par 9 Εάν λάβω τας πτέρυγας της αυγής και κατοικήσω εις τα έσχατα της θαλάσσης,
\par 10 και εκεί θέλει με οδηγήσει η χειρ σου και η δεξιά σου θέλει με κρατεί.
\par 11 Εάν είπω, Αλλά το σκότος θέλει με σκεπάσει, και η νυξ θέλει είσθαι φως περί εμέ·
\par 12 και αυτό το σκότος δεν σκεπάζει ουδέν από σού· και η νυξ λάμπει ως η ημέρα· εις σε το σκότος είναι ως το φως.
\par 13 Διότι συ εμόρφωσας τους νεφρούς μου· με περιετύλιξας εν τη κοιλία της μητρός μου.
\par 14 Θέλω σε υμνεί, διότι φοβερώς και θαυμασίως επλάσθην· θαυμάσια είναι τα έργα σου· και η ψυχή μου κάλλιστα γνωρίζει τούτο.
\par 15 Δεν εκρύφθησαν τα οστά μου από σου, ενώ επλαττόμην εν τω κρυπτώ και διεμορφονόμην εν τοις κατωτάτοις της γης.
\par 16 Το αδιαμόρφωτον του σώματός μου είδον οι οφθαλμοί σου· και εν τω βιβλίω σου πάντα ταύτα ήσαν γεγραμμένα, ως και αι ημέραι καθ' ας εσχηματίζοντο, και ενώ ουδέν εκ τούτων υπήρχε·
\par 17 πόσον δε πολύτιμοι είναι εις εμέ αι βουλαί σου, Θεέ· πόσον εμεγαλύνθη ο αριθμός αυτών.
\par 18 Εάν ήθελον να απαριθμήσω αυτάς, υπερβαίνουσι την άμμον· εξυπνώ, και έτι είμαι μετά σου.
\par 19 Βεβαίως θέλεις θανατώσει τους ασεβείς, Θεέ· απομακρύνθητε λοιπόν απ' εμού, άνδρες αιμάτων.
\par 20 Διότι λαλούσι κατά σου ασεβώς· οι εχθροί σου λαμβάνουσι το όνομά σου επί ματαίω.
\par 21 Μη δεν μισώ, Κύριε, τους μισούντάς σε; και δεν αγανακτώ κατά των επανισταμένων επί σε;
\par 22 Με τέλειον μίσος μισώ αυτούς· διά εχθρούς έχω αυτούς.
\par 23 Δοκίμασόν με, Θεέ, και γνώρισον την καρδίαν μου· εξέτασόν με και μάθε τους στοχασμούς μου·
\par 24 και ιδέ, αν υπάρχη εν εμοί οδός ανομίας· και οδήγησόν με εις την οδόν την αιώνιον.

\chapter{140}

\par «Εις τον πρώτον μουσικόν. Ψαλμός του Δαβίδ.» Ελευθέρωσόν με, Κύριε, από ανθρώπου πονηρού· λύτρωσόν με από ανθρώπου αδίκου·
\par 2 Οίτινες διαλογίζονται πονηρά εν τη καρδία· όλην την ημέραν παρατάττονται εις πολέμους.
\par 3 Ηκόνησαν την γλώσσαν αυτών ως όφεως· φαρμάκιον ασπίδος είναι υπό τα χείλη αυτών. Διάψαλμα.
\par 4 Φύλαξόν με, Κύριε, από χειρών ασεβούς· λύτρωσόν με από ανθρώπου αδίκου· οίτινες εμηχανεύθησαν να υποσκελίσωσι τα διαβήματά μου.
\par 5 Οι υπερήφανοι έκρυψαν κατ' εμού παγίδα, και με σχοινία ήπλωσαν δίκτυα εις την διάβασίν μου· έστησαν δι' εμέ βρόχια. Διάψαλμα.
\par 6 Είπα προς τον Κύριον, Συ είσαι ο Θεός μου· ακροάσθητι, Κύριε, της φωνής των δεήσεών μου.
\par 7 Κύριε Θεέ, η δύναμις της σωτηρίας μου, συ περιεσκέπασας την κεφαλήν μου εν ημέρα πολέμου.
\par 8 Μη δώσης, Κύριε, εις τον ασεβή τας επιθυμίας αυτού· μη αφήσης να εκτελεσθή ο στοχασμός αυτού, μήποτε υψωθώσι. Διάψαλμα.
\par 9 Η πονηρία των χειλέων των περικυκλούντων με ας σκεπάση την κεφαλήν αυτών.
\par 10 Άνθρακες πεπυρακτωμένοι ας πέσωσιν επ' αυτούς· ας ριφθώσιν εις το πυρ, εις λάκκους βαθείς, διά να μη εγερθώσι πλέον.
\par 11 Άνθρωπος κακόγλωσσος ας μη στερεωθή επί της γής· η κακία θέλει καταδιώξει τον άδικον άνθρωπον, εωσού απολέση αυτόν.
\par 12 Εξεύρω ότι ο Κύριος θέλει κάμει την κρίσιν του τεθλιμμένου και την δίκην των πτωχών.
\par 13 Βεβαίως οι δίκαιοι θέλουσι δοξολογεί το όνομά σου· οι ευθείς θέλουσι κατοικεί έμπροσθεν του προσώπου σου.

\chapter{141}

\par «Ψαλμός του Δαβίδ.» Κύριε, προς σε έκραξα· σπεύσον προς εμέ· ακροάσθητι της φωνής μου, όταν κράζω προς σε.
\par 2 Ας κατευθυνθή ενώπιόν σου η προσευχή μου ως θυμίαμα· η ύψωσις των χειρών μου ας γείνη ως θυσία εσπερινή.
\par 3 Βάλε, Κύριε, φυλακήν εις το στόμα μου· φύλαττε την θύραν των χειλέων μου.
\par 4 Μη εκκλίνης την καρδίαν μου εις πράγμα πονηρόν, ώστε να εκτελώ πράξεις ασεβείς μετά ανθρώπων εργαζομένων ανομίαν· μηδέ να φάγω από των εκλεκτών αυτών φαγητών.
\par 5 Ας με κτυπά ο δίκαιος· τούτο θέλει είσθαι έλεος· και ας με ελέγχη· τούτο θέλει είσθαι μύρον εξαίρετον· δεν θέλει βλάψει την κεφαλήν μου· διότι μάλιστα και θέλω προσεύχεσθαι υπέρ αυτών εν ταις συμφοραίς αυτών.
\par 6 Ότε οι αρχηγοί αυτών περιήρχοντο εις τόπους πετρώδεις, ήκουσαν τα λόγιά μου, ότι ήσαν γλυκέα.
\par 7 Τα οστά ημών διασκορπίζονται εν τω στόματι του τάφου, ως όταν τις κόπτη και σχίζη ξύλα επί την γην.
\par 8 Διά τούτο οι οφθαλμοί μου, Κύριε Θεέ, ατενίζουσι προς σέ· επί σε ήλπισα· μη καταστρέψης την ψυχήν μου.
\par 9 Φύλαξόν με από της παγίδος, την οποίαν έστησαν δι' εμέ, και από των βρόχων των εργαζομένων ανομίαν.
\par 10 Ας πέσωσιν ομού οι ασεβείς εις τα δίκτυα αυτών, ενώ εγώ θέλω περάσει αβλαβής.

\chapter{142}

\par «Μασχίλ του Δαβίδ· προσευχή ότε ήτο εν τω σπηλαίω.» Με την φωνήν μου έκραξα προς τον Κύριον· με την φωνήν μου προς τον Κύριον εδεήθην.
\par 2 Θέλω εκχέει ενώπιον αυτού την δέησίν μου· την θλίψιν μου ενώπιον αυτού θέλω απαγγείλει.
\par 3 Ότε το πνεύμά μου ήτο κατατεθλιμμένον εν εμοί, τότε συ εγνώρισας την οδόν μου. Παγίδα έκρυψαν δι' εμέ εν τη οδώ την οποίαν περιεπάτουν.
\par 4 Έβλεπον εις τα δεξιά και παρετήρουν, και δεν υπήρχεν ο γνωρίζων με· καταφύγιον εχάθη απ' εμού, δεν υπήρχεν ο εκζητών την ψυχήν μου.
\par 5 Προς σε, Κύριε, έκραξα, και είπα, συ είσαι η καταφυγή μου, η μερίς μου εν γη ζώντων.
\par 6 Πρόσεξον εις την φωνήν μου, διότι ταλαιπωρούμαι σφόδρα· ελευθέρωσόν με εκ των καταδιωκόντων με, διότι είναι δυνατώτεροί μου.
\par 7 Εξάγαγε εκ φυλακής την ψυχήν μου, διά να δοξολογώ το όνομά σου. Οι δίκαιοι θέλουσι με περικυκλώσει, όταν με ανταμείψης.

\chapter{143}

\par «Ψαλμός του Δαβίδ.» Εισάκουσον, Κύριε, της προσευχής μου· ακροάσθητι των δεήσεών μου· αποκρίθητι προς εμέ κατά την αλήθειάν σου, κατά την δικαιοσύνην σου.
\par 2 Και μη εισέλθης εις κρίσιν μετά του δούλου σου· διότι δεν θέλει δικαιωθή ενώπιόν σου ουδείς άνθρωπος ζων.
\par 3 Διότι κατεδίωξεν ο εχθρός την ψυχήν μου· εταπείνωσεν έως εδάφους την ζωήν με εκάθισεν εις σκοτεινούς τόπους, ως τους αιωνίους νεκρούς.
\par 4 Διά τούτο το πνεύμά μου είναι κατατεθλιμμένον εν εμοί, και η καρδία μου τεταραγμένη εντός μου.
\par 5 Ενθυμούμαι τας αρχαίας ημέρας· διαλογίζομαι πάντα τα έργα σου· μελετώ εις τα ποιήματα των χειρών σου.
\par 6 Εκτείνω προς σε τας χείρας μου· η ψυχή μου σε διψά ως γη άνυδρος. Διάψαλμα.
\par 7 Ταχέως εισάκουσόν μου, Κύριε· το πνεύμά μου εκλείπει· μη κρύψης το πρόσωπόν σου απ εμού, και ομοιωθώ μετά των καταβαινόντων εις τον λάκκον.
\par 8 Κάμε με να ακούσω το πρωΐ το έλεός σου· διότι επί σε έθεσα το θάρρος μου· κάμε με να γνωρίσω την οδόν, εις την οποίαν πρέπει να περιπατώ· διότι προς σε ύψωσα την ψυχήν μου.
\par 9 Ελευθέρωσόν με εκ των εχθρών μου, Κύριε· προς σε κατέφυγον.
\par 10 Δίδαξόν με να κάμνω το θέλημά σου· διότι συ είσαι ο Θεός μου· το πνεύμά σου το αγαθόν ας με οδηγήση εις οδόν ευθείαν.
\par 11 Ένεκεν του ονόματός σου, Κύριε, ζωοποίησόν με· διά την δικαιοσύνην σου εξάγαγε την ψυχήν μου εκ της στενοχωρίας.
\par 12 Και διά το έλεός σου εξολόθρευσον τους εχθρούς μου, και αφάνισον πάντας τους θλίβοντας την ψυχήν μου· διότι εγώ είμαι δούλός σου.

\chapter{144}

\par «Ψαλμός του Δαβίδ.» Ευλογητός ο Κύριος, το φρούριόν μου, ο διδάσκων τας χείρας μου εις πόλεμον, τους δακτύλους μου εις μάχην·
\par 2 το έλεός μου και το οχύρωμά μου, το υψηλόν καταφύγιόν μου και ο ελευθερωτής μου· η ασπίς μου, επί τον οποίον ήλπισα, όστις υποτάσσει τον λαόν μου υπ' εμέ.
\par 3 Κύριε, τι είναι ο άνθρωπος, και γνωρίζεις αυτόν; ή ο υιός του ανθρώπου, και συλλογίζεσαι αυτόν;
\par 4 Ο άνθρωπος ομοιάζει την ματαιότητα· αι ημέραι αυτού είναι ως σκιά παρερχομένη.
\par 5 Κύριε, κλίνον τους ουρανούς σου και κατάβηθι· έγγισον τα όρη, και θέλουσι καπνίσει.
\par 6 Άστραψον αστραπήν, και θέλεις διασκορπίσει αυτούς· ρίψον τα βέλη σου, και θέλεις εξολοθρεύσει αυτούς.
\par 7 Εξαπόστειλον την χείρα σου εξ ύψους· λύτρωσόν με και ελευθέρωσόν με εξ υδάτων πολλών, εκ χειρός των υιών του αλλοτρίου,
\par 8 των οποίων το στόμα λαλεί ματαιότητα, και η δεξιά αυτών είναι δεξιά ψεύδους.
\par 9 Θεέ, ωδήν νέαν θέλω ψάλλει εις σέ· εν ψαλτηρίω δεκαχόρδω θέλω ψαλμωδεί εις σέ·
\par 10 τον διδόντα σωτηρίαν εις τους βασιλείς· τον λυτρόνοντα Δαβίδ τον δούλον αυτού από ρομφαίας πονηράς.
\par 11 Λύτρωσόν με και ελευθέρωσόν με από χειρός των υιών του αλλοτρίου, των οποίων το στόμα λαλεί ματαιότητα, και η δεξιά αυτών είναι δεξιά ψεύδους·
\par 12 διά να ήναι οι υιοί ημών ως νεόφυτα, αυξάνοντες εις την νεότητα αυτών· αι θυγατέρες ημών ως ακρογωνιαίοι λίθοι τετορνευμένοι προς στολισμόν παλατίου·
\par 13 Αι αποθήκαι ημών πλήρεις, ώστε να δίδωσι παν είδος τροφής· τα πρόβατα ημών πληθυνόμενα εις χιλιάδας και μυριάδας εν τοις αγροίς ημών·
\par 14 οι βόες ημών πολύτοκοι· να μη υπάρχη μήτε έφοδος εχθρών μήτε εξόρμησις, μηδέ κραυγή εν ταις πλατείαις ημών.
\par 15 Μακάριος ο λαός, όστις ευρίσκεται εν τοιαύτη καταστάσει μακάριος ο λαός, του οποίου ο Κύριος είναι ο Θεός αυτού.

\chapter{145}

\par «Αίνεσις του Δαβίδ.» Θέλω σε υψόνει, Θεέ μου, βασιλεύ· και θέλω ευλογεί το όνομά σου εις τον αιώνα και εις τον αιώνα.
\par 2 Καθ' εκάστην ημέραν θέλω σε ευλογεί· και θέλω αινεί το όνομά σου εις τον αιώνα και εις τον αιώνα.
\par 3 Μέγας ο Κύριος και αξιΰμνητος σφόδρα· και η μεγαλωσύνη αυτού ανεξιχνίαστος.
\par 4 Γενεά εις γενεάν θέλει επαινεί τα έργα σου, και τα μεγαλείά σου θέλουσι διηγείσθαι.
\par 5 Θέλω λαλεί περί της ενδόξου μεγαλοπρεπείας της μεγαλειότητός σου και περί των θαυμαστών έργων σου·
\par 6 και θέλουσι λέγει την δύναμιν των φοβερών σου κατορθωμάτων, και θέλω διηγείσθαι την μεγαλωσύνην σου·
\par 7 Θέλουσι διαδίδει την μνήμην του πλήθους της αγαθότητός σου, και θέλουσιν αλαλάξει την δικαιοσύνην σου.
\par 8 Ελεήμων και οικτίρμων ο Κύριος· μακρόθυμος και πολυέλεος.
\par 9 Αγαθός ο Κύριος προς πάντας· και οι οικτιρμοί αυτού επί πάντα τα ποιήματα αυτού.
\par 10 Πάντα τα ποιήματά σου, Κύριε, θέλουσι σε αινεί· και οι όσιοί σου θέλουσι σε ευλογεί.
\par 11 Την δόξαν της βασιλείας σου θέλουσι κηρύττει και θέλουσι διηγείσθαι το μεγαλείόν σου·
\par 12 διά να γνωστοποιήσωσιν εις τους υιούς των ανθρώπων τα μεγαλεία αυτού, και την δόξαν της μεγαλοπρεπείας της βασιλείας αυτού.
\par 13 Η βασιλεία σου βασιλεία πάντων των αιώνων, και η δεσποτεία σου εν πάση γενεά και γενεά.
\par 14 Ο Κύριος υποστηρίζει πάντας τους πίπτοντας και ανορθοί πάντας τους κεκυρτωμένους.
\par 15 Οι οφθαλμοί πάντων αποβλέπουσι προς σέ· και συ δίδεις εις αυτούς την τροφήν αυτών εν καιρώ.
\par 16 Ανοίγεις την χείρα σου και χορταίνεις την επιθυμίαν παντός ζώντος.
\par 17 Δίκαιος ο Κύριος εν πάσαις ταις οδοίς αυτού και αγαθός εν πάσι τοις έργοις αυτού.
\par 18 Ο Κύριος είναι πλησίον πάντων των επικαλουμένων αυτόν· πάντων των επικαλουμένων αυτόν εν αληθεία.
\par 19 Εκπληροί την επιθυμίαν των φοβουμένων αυτόν, και της κραυγής αυτών εισακούει και σώζει αυτούς.
\par 20 Ο Κύριος φυλάττει πάντας τους αγαπώντας αυτόν· θέλει δε εξολοθρεύσει πάντας τους ασεβείς.
\par 21 Το στόμα μου θέλει λαλεί την αίνεσιν του Κυρίου· και πάσα σαρξ ας ευλογή το όνομα το άγιον αυτού εις τον αιώνα και εις τον αιώνα.

\chapter{146}

\par Αινείτε τον Κύριον. Αίνει, η ψυχή μου, τον Κύριον.
\par 2 Θέλω αινεί τον Κύριον ενόσω ζώ· θέλω ψαλμωδεί εις τον Θεόν μου ενόσω υπάρχω.
\par 3 Μη πεποίθατε επ' άρχοντας, επί υιόν ανθρώπου, εκ του οποίου δεν είναι σωτηρία.
\par 4 Το πνεύμα αυτού εξέρχεται· αυτός επιστρέφει εις την γην αυτού· εν εκείνη τη ημέρα οι διαλογισμοί αυτού αφανίζονται.
\par 5 Μακάριος εκείνος, του οποίου βοηθός είναι ο Θεός του Ιακώβ· του οποίου η ελπίς είναι επί Κύριον τον Θεόν αυτού·
\par 6 τον ποιήσαντα τον ουρανόν και την γην, την θάλασσαν και πάντα τα εν αυτοίς· τον φυλάττοντα αλήθειαν εις τον αιώνα·
\par 7 τον ποιούντα κρίσιν εις τους αδικουμένους· τον διδόντα τροφήν εις τους πεινώντας. Ο Κύριος ελευθερόνει τους δεσμίους.
\par 8 Ο Κύριος ανοίγει τους οφθαλμούς των τυφλών· ο Κύριος ανορθοί τους κεκυρτωμένους· ο Κύριος αγαπά τους δικαίους·
\par 9 ο Κύριος διαφυλάττει τους ξένους· υπερασπίζεται τον ορφανόν και την χήραν, την δε οδόν των αμαρτωλών καταστρέφει.
\par 10 Ο Κύριος θέλει βασιλεύει εις τον αιώνα· ο Θεός σου, Σιών, εις γενεάν και γενεάν. Αλληλούϊα.

\chapter{147}

\par Αινείτε τον Κύριον· διότι είναι καλόν να ψάλλωμεν εις τον Θεόν ημών· διότι είναι τερπνόν, η αίνεσις πρέπουσα.
\par 2 Ο Κύριος οικοδομεί την Ιερουσαλήμ· θέλει συνάξει τους διεσπαρμένους του Ισραήλ.
\par 3 Ιατρεύει τους συντετριμμένους την καρδίαν και δένει τας πληγάς αυτών.
\par 4 Αριθμεί τα πλήθη των άστρων· Καλεί τα πάντα ονομαστί.
\par 5 Μέγας ο Κύριος ημών και μεγάλη η δύναμις αυτού· η σύνεσις αυτού αμέτρητος.
\par 6 Ο Κύριος υψόνει τους πράους, τους δε ασεβείς ταπεινόνει έως εδάφους.
\par 7 Ψάλατε εις τον Κύριον ευχαριστούντες· ψαλμωδείτε εις τον Θεόν ημών εν κιθάρα·
\par 8 τον σκεπάζοντα τον ουρανόν με νεφέλας· τον ετοιμάζοντα βροχήν διά την γήν· τον αναδιδόντα χόρτον επί των ορέων·
\par 9 τον διδόντα εις τα κτήνη την τροφήν αυτών και εις τους νεοσσούς των κοράκων, οίτινες κράζουσι προς αυτόν.
\par 10 Δεν χαίρει εις την δύναμιν του ίππου· δεν ηδύνεται εις τους πόδας του ανδρός.
\par 11 Ο Κύριος ηδύνεται εις τους φοβουμένους αυτόν, εις τους ελπίζοντας επί το έλεος αυτού.
\par 12 Επαίνει, Ιερουσαλήμ, τον Κύριον· αίνει τον Θεόν σου, Σιών.
\par 13 Διότι ενεδυνάμωσε τους μοχλούς των πυλών σου· ηυλόγησε τους υιούς σου εν μέσω σου.
\par 14 Βάλλει ειρήνην εις τα όριά σου· σε χορταίνει με το πάχος του σίτου.
\par 15 Αποστέλλει το πρόσταγμα αυτού εις την γην, ο λόγος αυτού τρέχει ταχύτατα.
\par 16 Δίδει χιόνα ως μαλλίον· διασπείρει την πάχνην ως στάκτην.
\par 17 Ρίπτει τον κρύσταλλον αυτού ως κομμάτια· έμπροσθεν του ψύχους αυτού τις δύναται να σταθή;
\par 18 Αποστέλλει τον λόγον αυτού και διαλύει αυτά· φυσά τον άνεμον αυτού, και τα ύδατα ρέουσιν.
\par 19 Αναγγέλλει τον λόγον αυτού προς τον Ιακώβ, τα διατάγματα αυτού και τας κρίσεις αυτού προς τον Ισραήλ.
\par 20 Δεν έκαμεν ούτως εις ουδέν έθνος· ουδέ εγνώρισαν τας κρίσεις αυτού. Αλληλούϊα.

\chapter{148}

\par Αινείτε τον Κύριον. Αινείτε τον Κύριον εκ των ουρανών· αινείτε αυτόν εν τοις υψίστοις.
\par 2 Αινείτε αυτόν, πάντες οι άγγελοι αυτού· αινείτε αυτόν, πάσαι αι δυνάμεις αυτού.
\par 3 Αινείτε αυτόν, ήλιε και σελήνη· αινείτε αυτόν, πάντα τα άστρα του φωτός.
\par 4 Αινείτε αυτόν, οι ουρανοί των ουρανών, και τα ύδατα τα υπεράνω των ουρανών.
\par 5 Ας αινώσι το όνομα του Κυρίου· διότι αυτός προσέταξε, και εκτίσθησαν·
\par 6 και εστερέωσεν αυτά εις τον αιώνα και εις τον αιώνα· έθεσε διάταγμα, το οποίον δεν θέλει παρέλθει.
\par 7 Αινείτε τον Κύριον εκ της γης, δράκοντες και πάσαι άβυσσοι·
\par 8 πυρ και χάλαζα, χιών και ατμίς, ανεμοστρόβιλος, ο εκτελών τον λόγον αυτού·
\par 9 τα όρη και πάντα τα βουνά· δένδρα καρποφόρα και πάσαι κέδροι·
\par 10 τα θηρία και πάντα τα κτήνη· ερπετά και πετεινά πτερωτά.
\par 11 Βασιλείς της γης και πάντες λαοί· άρχοντες και πάντες κριταί της γής·
\par 12 νέοι τε και παρθένοι, γέροντες μετά νεωτέρων
\par 13 ας αινώσι το όνομα του Κυρίου· διότι το όνομα αυτού μόνου είναι υψωμένον·
\par 14 Η δόξα αυτού είναι επί την γην και τον ουρανόν· και αυτός ύψωσε κέρας εις τον λαόν αυτού, ύμνον εις πάντας τους οσίους αυτού, εις τους υιούς Ισραήλ, λαόν όστις είναι πλησίον αυτού. Αλληλούϊα.

\chapter{149}

\par Αινείτε τον Κύριον. Ψάλατε εις τον Κύριον ωδήν νέαν, την αίνεσιν αυτού εν τη συνάξει των οσίων.
\par 2 Ας ευφραίνεται ο Ισραήλ εις τον Ποιητήν αυτού· οι υιοί της Σιών ας αγάλλωνται εις τον Βασιλέα αυτών.
\par 3 Ας αινώσι το όνομα αυτού χοροστατούντες· εν τυμπάνω και κιθάρα ας ψαλμωδώσιν εις αυτόν.
\par 4 Διότι ο Κύριος ευδοκεί εις τον λαόν αυτού· θέλει δοξάσει τους πράους εν σωτηρία.
\par 5 Οι όσιοι θέλουσιν αγάλλεσθαι εν δόξη· θέλουσιν αγάλλεσθαι επί τας κλίνας αυτών.
\par 6 Αι εξυμνήσεις του Θεού θέλουσιν είσθαι εν τω λάρυγγι αυτών, και ρομφαία δίστομος εν τη χειρί αυτών·
\par 7 διά να κάμνωσιν εκδίκησιν εις τα έθνη, παιδείαν εις τους λαούς·
\par 8 διά να δέσωσι τους βασιλείς αυτών με αλύσεις· και τους ενδόξους αυτών με δεσμά σιδηρά·
\par 9 διά να κάμωσιν επ' αυτούς την γεγραμμένην κρίσιν. Η δόξα αύτη θέλει είσθαι εις πάντας τους οσίους αυτού. Αλληλούϊα.

\chapter{150}

\par Αινείτε τον Κύριον. Αινείτε τον Θεόν εν τω αγιαστηρίω αυτού· αινείτε αυτόν εν τω στερεώματι της δυνάμεως αυτού.
\par 2 Αινείτε αυτόν διά τα μεγαλεία αυτού· αινείτε αυτόν κατά το πλήθος της μεγαλωσύνης αυτού.
\par 3 Αινείτε αυτόν εν ήχω σάλπιγγος· αινείτε αυτόν εν ψαλτηρίω και κιθάρα.
\par 4 Αινείτε αυτόν εν τυμπάνω και χοροστασία· αινείτε αυτόν εν χορδαίς και οργάνω.
\par 5 Αινείτε αυτόν εν κυμβάλοις ευήχοις· αινείτε αυτόν εν κυμβάλοις αλαλαγμού.
\par 6 Πάσα πνοή ας αινή τον Κύριον. Αλληλούϊα.


\end{document}