\begin{document}

\title{Βασιλειών Γ'}


\chapter{1}

\par 1 Και ο βασιλεύς Δαβίδ ήτο γέρων, προβεβηκώς την ηλικίαν· και εσκέπαζον αυτόν με ιμάτια, πλην δεν εθερμαίνετο.
\par 2 Και είπον οι δούλοι αυτού προς αυτόν, Ας ζητήσωσι διά τον κύριόν μου τον βασιλέα νεάνιδα παρθένον, διά να ίσταται έμπροσθεν του βασιλέως και να περιθάλπη αυτόν, και να κοιμάται εις τον κόλπον σου, διά να θερμαίνηται ο κύριός μου ο βασιλεύς.
\par 3 Και εζήτησαν εν πάσι τοις ορίοις του Ισραήλ νεάνιδα ώραίαν· και εύρηκαν την Αβισάγ την Σουναμίτιν, και έφεραν αυτήν προς τον βασιλέα.
\par 4 Ήτο δε η νεάνις ώραία σφόδρα, και περιέθαλπε τον βασιλέα, και υπηρέτει αυτόν· πλην ο βασιλεύς δεν εγνώρισεν αυτήν.
\par 5 Τότε Αδωνίας ο υιός της Αγγείθ επήρθη εις εαυτόν, λέγων, Εγώ θέλω βασιλεύσει και ητοίμασεν εις εαυτόν αμάξας και ιππείς και πεντήκοντα άνδρας προτρέχοντας έμπροσθεν αυτού.
\par 6 Ο δε πατήρ αυτού δεν επίκραινέ ποτέ αυτόν, λέγων, Διά τι συ πράττεις ούτω; ήτο δε και ώραίος την όψιν σφόδρα· και η μήτηρ αυτού εγέννησεν αυτόν μετά τον Αβεσσαλώμ.
\par 7 Και συνελάλησε μετά του Ιωάβ υιού της Σερουΐας, και μετά Αβιάθαρ του ιερέως· και ούτοι, ακολουθήσαντες τον Αδωνίαν, εβοήθουν αυτόν.
\par 8 Σαδώκ όμως ο ιερεύς και Βεναΐας ο υιός του Ιωδαέ και Νάθαν ο προφήτης και Σιμεΐ και Ρεΐ και οι δυνατοί του Δαβίδ δεν ήσαν μετά του Αδωνία.
\par 9 Και έσφαξεν ο Αδωνίας πρόβατα και βόας και σιτευτά πλησίον της πέτρας του Ζωελέθ, ήτις είναι πλησίον της Εν-ρωγήλ, και εκάλεσε πάντας τους αδελφούς αυτού τους υιούς του βασιλέως και πάντας τους άνδρας του Ιούδα τους δούλους του βασιλέως.
\par 10 Τον Νάθαν όμως τον προφήτην και τον Βεναΐαν και τους δυνατούς και Σολομώντα τον αδελφόν αυτού δεν εκάλεσε.
\par 11 Και είπεν ο Νάθαν προς την Βηθ-σαβεέ την μητέρα του Σολομώντος, λέγων, Δεν ήκουσας ότι εβασίλευσεν Αδωνίας ο υιός της Αγγείθ, και ο κύριος ημών Δαβίδ δεν εξεύρει τούτο;
\par 12 τώρα λοιπόν ελθέ να σοι δώσω, παρακαλώ, συμβουλήν, διά να σώσης την ζωήν σου και την ζωήν του υιού σου Σολομώντος·
\par 13 ύπαγε και είσελθε προς τον βασιλέα Δαβίδ και ειπέ προς αυτόν, Κύριέ μου βασιλεύ, συ δεν ώμοσας εις την δούλην σου, λέγων, Βεβαίως Σολομών ο υιός σου θέλει βασιλεύσει μετ' εμέ, και αυτός θέλει καθίσει επί του θρόνου μου; διά τι λοιπόν εβασίλευσεν ο Αδωνίας;
\par 14 ιδού, ενώ έτι συ λαλείς εκεί μετά του βασιλέως, θέλω ελθεί και εγώ κατόπιν σου και θέλω αναπληρώσει τους λόγους σου.
\par 15 Και εισήλθεν η Βηθ-σαβεέ προς τον βασιλέα εις τον κοιτώνα· ήτο δε ο βασιλεύς γέρων σφόδρα και Αβισάγ η Σουναμίτις υπηρέτει τον βασιλέα.
\par 16 Και κύψασα η Βηθ-σαβεέ, προσεκύνησε τον βασιλέα. Και ο βασιλεύς είπε, Τι έχεις;
\par 17 Η δε είπε προς αυτόν, Κύριέ μου, συ ώμοσας εις Κύριον τον Θεόν σου προς την δούλην σου, λέγων, Βεβαίως ο Σολομών, ο υιός σου, θέλει βασιλεύσει μετ' εμέ, και αυτός θέλει καθίσει επί του θρόνου μου·
\par 18 αλλά τώρα, ιδού, ο Αδωνίας εβασίλευσε· και συ τώρα, κύριέ μου βασιλεύ, δεν εξεύρεις τούτο·
\par 19 και έσφαξε βόας και σιτευτά και πρόβατα εν αφθονία, και εκάλεσε πάντας τους υιούς του βασιλέως και Αβιάθαρ τον ιερέα και Ιωάβ τον αρχιστράτηγον· τον δούλον σου όμως Σολομώντα δεν εκάλεσεν·
\par 20 αλλ' εις σε, κύριέ μου βασιλεύ, εις σε αποβλέπουσιν οι οφθαλμοί παντός του Ισραήλ, διά να απαγγείλης προς αυτούς τις θέλει καθίσει επί του θρόνου του κυρίου μου του βασιλέως μετ' αυτόν·
\par 21 ειδεμή, αφού ο κύριός μου ο βασιλεύς κοιμηθή μετά των πατέρων αυτού, εγώ και ο υιός μου ο Σολομών θέλομεν θεωρείσθαι πταίσται.
\par 22 Και ιδού, ενώ αυτή ελάλει έτι μετά του βασιλέως, ήλθε και Νάθαν ο προφήτης.
\par 23 Και ανήγγειλαν προς τον βασιλέα, λέγοντες, Ιδού, Νάθαν ο προφήτης. Και εισελθών ενώπιον του βασιλέως, προσεκύνησε τον βασιλέα κατά πρόσωπον αυτού έως εδάφους.
\par 24 Και είπεν ο Νάθαν, Κύριέ μου βασιλεύ, συ είπας, Ο Αδωνίας θέλει βασιλεύσει μετ' εμέ και αυτός θέλει καθίσει επί του θρόνου μου;
\par 25 διότι κατέβη σήμερον και έσφαξε βόας και σιτευτά και πρόβατα εν αφθονία, και εκάλεσε πάντας τους υιούς του βασιλέως και τους στρατηγούς και Αβιάθαρ τον ιερέα· και ιδού, τρώγουσι και πίνουσιν ενώπιον αυτού και λέγουσι, Ζήτω ο βασιλεύς Αδωνίας·
\par 26 εμέ δε, εμέ τον δούλον σου, και Σαδώκ τον ιερέα και Βεναΐαν τον υιόν του Ιωδαέ και Σολομώντα τον δούλον σου δεν εκάλεσε·
\par 27 παρά του κυρίου μου του βασιλέως έγεινε το πράγμα τούτο, και δεν εφανέρωσας εις τον δούλον σου τις θέλει καθίσει επί του θρόνου του κυρίου μου του βασιλέως μετ' αυτόν;
\par 28 Και απεκρίθη ο βασιλεύς Δαβίδ και είπε, Καλέσατέ μοι την Βηθ-σαβεέ. Και εισήλθεν ενώπιον του βασιλέως και εστάθη έμπροσθεν του βασιλέως.
\par 29 Και ώμοσεν ο βασιλεύς και είπε, Ζη Κύριος, όστις ελύτρωσε την ψυχήν μου εκ πάσης στενοχωρίας,
\par 30 βεβαίως, καθώς ώμοσα προς σε εις Κύριον τον Θεόν του Ισραήλ, λέγων, ότι Σολομών ο υιός σου θέλει βασιλεύσει μετ' εμέ, και αυτός θέλει καθίσει αντ' εμού επί του θρόνου μου, ούτω θέλω κάμει την ημέραν ταύτην.
\par 31 Τότε η Βηθ-σαβεέ, κύψασα κατά πρόσωπον έως εδάφους, προσεκύνησε τον βασιλέα και είπε, Ζήτω ο κύριός μου ο βασιλεύς Δαβίδ εις τον αιώνα.
\par 32 Και είπεν ο βασιλεύς Δαβίδ, Καλέσατέ μοι Σαδώκ τον ιερέα και Νάθαν τον προφήτην και Βεναΐαν τον υιόν του Ιωδαέ. Και ήλθον ενώπιον του βασιλέως.
\par 33 Και είπε προς αυτούς ο βασιλεύς, Λάβετε μεθ' εαυτών τους δούλους του κυρίου σας και καθίσατε Σολομώντα τον υιόν μου επί την ημίονόν μου και καταβιβάσατε αυτόν εις Γιών·
\par 34 και ας χρίσωσιν αυτόν εκεί Σαδώκ ο ιερεύς και Νάθαν ο προφήτης βασιλέα επί τον Ισραήλ· και σαλπίσατε διά της σάλπιγγος και είπατε, Ζήτω ο βασιλεύς Σολομών·
\par 35 τότε θέλετε αναβή κατόπιν αυτού, διά να έλθη και να καθίση επί τον θρόνον μου· και αυτός θέλει βασιλεύσει αντ' εμού· και αυτόν προσέταξα να ήναι ηγεμών επί τον Ισραήλ και επί τον Ιούδαν.
\par 36 Και απεκρίθη Βεναΐας ο υιός του Ιωδαέ προς τον βασιλέα, και είπεν, Αμήν· ούτως ας επικυρώση Κύριος ο Θεός του κυρίου μου του βασιλέως·
\par 37 καθώς εστάθη ο Κύριος μετά του κυρίου μου του βασιλέως, ούτω να ήναι και μετά του Σολομώντος, και να μεγαλύνη τον θρόνον αυτού υπέρ τον θρόνον του κυρίου μου του βασιλέως Δαβίδ.
\par 38 Τότε κατέβη Σαδώκ ο ιερεύς και Νάθαν ο προφήτης και Βεναΐας ο υιός του Ιωδαέ και οι Χερεθαίοι και οι Φελεθαίοι, και εκάθισαν τον Σολομώντα επί την ημίονον του βασιλέως Δαβίδ και έφεραν αυτόν εις Γιών.
\par 39 Και έλαβε Σαδώκ ο ιερεύς το κέρας του ελαίου εκ της σκηνής και έχρισε τον Σολομώντα. Και εσάλπισαν διά της σάλπιγγος· και είπε πας ο λαός, Ζήτω ο βασιλεύς Σολομών.
\par 40 Και ανέβη πας ο λαός κατόπιν αυτού· και έπαιζεν ο λαός αυλούς και ευφραίνετο ευφροσύνην μεγάλην, και η γη εσχίζετο εκ των φωνών αυτών.
\par 41 Και ήκουσεν Αδωνίας και πάντες οι κεκλημένοι αυτού, καθώς ετελείωσαν να τρώγωσι. Και ότε ήκουσεν ο Ιωάβ την φωνήν της σάλπιγγος, είπε, Τις η φωνή αύτη της πόλεως θορυβούσης;
\par 42 Ενώ έτι ελάλει, ιδού, Ιωνάθαν, ο υιός Αβιάθαρ του ιερέως, ήλθε· και είπεν ο Αδωνίας προς αυτόν, Είσελθε· διότι συ είσαι ανήρ γενναίος και φέρεις αγαθάς αγγελίας.
\par 43 Και αποκριθείς ο Ιωνάθαν είπε προς τον Αδωνίαν, Βεβαίως κύριος ημών ο βασιλεύς Δαβίδ έκαμε βασιλέα τον Σολομώντα·
\par 44 και απέστειλε μετ' αυτού ο βασιλεύς Σαδώκ τον ιερέα και Νάθαν τον προφήτη και Βεναΐαν τον υιόν του Ιωδαέ και τους Χερεθαίους και τους Φελεθαίους, και εκάθισαν αυτόν επί την ημίονον του βασιλέως·
\par 45 και έχρισαν αυτόν Σαδώκ ο ιερεύς και Νάθαν ο προφήτης βασιλέα εν Γιών· και ανέβησαν εκείθεν ευφραινόμενοι, και η πόλις αντήχησεν· αύτη είναι η φωνή, την οποίαν ηκούσατε·
\par 46 και μάλιστα εκάθησεν ο Σολομών επί του θρόνου της βασιλείας·
\par 47 και εισήλθον έτι οι δούλοι του βασιλέως να ευχηθώσι τον κύριον ημών τον βασιλέα Δαβίδ, λέγοντες, Ο Θεός να λαμπρύνη το όνομα του Σολομώντος υπέρ το όνομά σου, και να μεγαλύνη τον θρόνον σου και να μεγαλύνη τον θρόνον αυτού υπέρ τον θρόνον σου. και προσεκύνησεν ο βασιλεύς επί της κλίνης·
\par 48 και είπε προσέτι ο βασιλεύς ούτως· Ευλογητός Κύριος ο Θεός του Ισραήλ, όστις έδωκεν εις εμέ σήμερον διάδοχον καθήμενον επί του θρόνου μου, και οι οφθαλμοί μου βλέπουσι τούτο.
\par 49 Τότε πάντες οι κεκλημένοι, οι μετά του Αδωνία, εξεπλάγησαν και σηκωθέντες, υπήγαν έκαστος την οδόν αυτού.
\par 50 Ο δε Αδωνίας εφοβήθη από προσώπου του Σολομώντος και σηκωθείς υπήγε και επιάσθη από των κεράτων του θυσιαστηρίου.
\par 51 Και ανήγγειλαν προς τον Σολομώντα, λέγοντες, Ιδού, ο Αδωνίας φοβείται τον βασιλέα Σολομώντα· και ιδού, επιάσθη από των κεράτων του θυσιαστηρίου, λέγων, Ας ομόση προς εμέ σήμερον ο βασιλεύς Σολομών, ότι δεν θέλει θανατώσει τον δούλον αυτού διά ρομφαίας.
\par 52 Και είπεν ο Σολομών, Εάν σταθή ανήρ αγαθός, ουδέ μία εκ των τριχών αυτού θέλει πέσει επί την γήν· εάν όμως ευρεθή κακία εν αυτώ θέλει θανατωθή.
\par 53 Και απέστειλεν ο βασιλεύς Σολομών, και κατεβίβασαν αυτόν από του θυσιαστηρίου· και ήλθε και προσεκύνησε τον βασιλέα Σολομώντα· και είπε προς αυτόν ο Σολομών, Ύπαγε εις τον οίκόν σου.

\chapter{2}

\par 1 Επλησίασαν δε αι ημέραι του Δαβίδ να αποθάνη· και παρήγγειλε προς τον Σολομώντα τον υιόν αυτού, λέγων,
\par 2 Εγώ υπάγω την οδόν πάσης της γής· συ δε ίσχυε και έσο ανήρ·
\par 3 και φύλαττε τας εντολάς Κυρίου του Θεού σου, περιπατών εις τας οδούς αυτού, φυλάττων τα διατάγματα αυτού, τα προστάγματα αυτού και τας κρίσεις αυτού και τα μαρτύρια αυτού, ως είναι γεγραμμένον εν τω νόμω του Μωϋσέως, διά να ευημερής εις πάντα όσα πράττεις και πανταχού όπου αν στραφής·
\par 4 διά να στηρίξη ο Κύριος τον λόγον αυτού, τον οποίον ελάλησε περί εμού, λέγων, Εάν οι υιοί σου προσέχωσιν εις την οδόν αυτών ώστε να περιπατώσιν ενώπιόν μου εν αληθεία, εξ όλης της καρδίας αυτών και εξ όλης της ψυχής αυτών, βεβαίως δεν θέλει εκλείψει εις σε ανήρ επάνωθεν του θρόνου του Ισραήλ.
\par 5 Και έτι συ εξεύρεις όσα έκαμεν εις εμέ Ιωάβ ο υιός της Σερουΐας, τι έκαμεν εις τους δύο αρχηγούς των στρατευμάτων του Ισραήλ, εις τον Αβενήρ τον υιόν του Νηρ, και εις τον Αμασά τον υιόν του Ιεθέρ, τους οποίους εφόνευσε, και έχυσε το αίμα του πολέμου εν ειρήνη και έβαλε το αίμα του πολέμου εις την ζώνην αυτού, την περί την οσφύν αυτού, και εις τα υποδήματα αυτού τα εις τους πόδας αυτού.
\par 6 Κάμε λοιπόν κατά την σοφίαν σου, και η πολιά αυτού ας μη καταβή εις τον άδην εν ειρήνη.
\par 7 Προς τους υιούς όμως του Βαρζελλαΐ του Γαλααδίτου κάμε έλεος, και ας ήναι εκ των εσθιόντων επί της τραπέζης σου· διότι ούτως επλησίασαν προς εμέ, ότε έφευγον από προσώπου του Αβεσσαλώμ του αδελφού σου.
\par 8 Και ιδού, μετά σου Σιμεΐ ο υιός του Γηρά, ο Βενιαμίτης, από Βαουρείμ, όστις με κατηράσθη κατάραν οδυνηράν καθ' ην ημέραν επορευόμην εις Μαχαναΐμ· κατέβη όμως προς απάντησίν μου εις τον Ιορδάνην, και ώμοσα προς αυτόν εις τον Κύριον, λέγων, Δεν θέλω σε θανατώσει διά ρομφαίας.
\par 9 Τώρα λοιπόν μη αθωώσης αυτόν· διότι είσαι ανήρ σοφός και εξεύρεις τι πρέπει να κάμης εις αυτόν, και να καταβιβάσης την πολιάν αυτού με αίμα εις τον άδην.
\par 10 Και εκοιμήθη ο Δαβίδ μετά των πατέρων αυτού και ετάφη εν τη πόλει Δαβίδ.
\par 11 Αι ημέραι δε, τας οποίας εβασίλευσεν ο Δαβίδ επί τον Ισραήλ, έγειναν τεσσαράκοντα έτη· επτά έτη εβασίλευσεν εν Χεβρών και τριάκοντα τρία εβασίλευσεν εν Ιερουσαλήμ.
\par 12 Και εκάθησεν ο Σολομών επί του θρόνου Δαβίδ του πατρός αυτού· και εστερεώθη η βασιλεία αυτού σφόδρα.
\par 13 Αδωνίας δε ο υιός της Αγγείθ ήλθε προς την Βηθ-σαβεέ, την μητέρα του Σολομώντος. Η δε είπεν, Έρχεσαι εν ειρήνη; Και είπεν, Εν ειρήνη.
\par 14 Έπειτα είπεν, Έχω λόγον τινά να είπω προς σε. Η δε είπε, Λάλησον.
\par 15 Και είπε, Συ εξεύρεις ότι εις εμέ ανήκεν η βασιλεία και εις εμέ είχε στήσει πας ο Ισραήλ το πρόσωπον αυτού, διά να βασιλεύσω· η βασιλεία όμως εστράφη και έγεινε του αδελφού μου· διότι παρά Κυρίου έγεινεν εις αυτόν·
\par 16 τώρα λοιπόν ζητώ μίαν αίτησιν παρά σού· μη αρνηθής ταύτην εις εμέ. Η δε είπε προς αυτόν, Λάλει.
\par 17 Και είπεν, Ειπέ, παρακαλώ, προς τον Σολομώντα τον βασιλέα, διότι δεν θέλει σοι αρνηθή τούτο, να δώση εις εμέ την Αβισάγ την Σουναμίτιν διά γυναίκα.
\par 18 Και είπεν η Βηθ-σαβεέ, Καλώς· εγώ θέλω λαλήσει περί σου προς τον βασιλέα.
\par 19 Και εισήλθεν η Βηθ-σαβεέ προς τον βασιλέα Σολομώντα, διά να λαλήση προς αυτόν περί του Αδωνίου. Και εσηκώθη ο βασιλεύς εις απάντησιν αυτής και προσεκύνησεν αυτήν· έπειτα εκάθησεν επί τον θρόνον αυτού, και ετέθη θρόνος εις την μητέρα του βασιλέως· και εκάθησεν εις τα δεξιά αυτού.
\par 20 Και είπε, Μίαν μικράν αίτησιν ζητώ παρά σού· μη αρνηθής ταύτην εις εμέ. Και είπε προς αυτήν ο βασιλεύς, Ζήτησον, μήτηρ μου· διότι δεν θέλω σοι αρνηθή.
\par 21 Η δε είπεν, Ας δοθή η Αβισάγ η Σουναμίτις εις τον Αδωνίαν τον αδελφόν σου διά γυναίκα.
\par 22 Και αποκριθείς ο βασιλεύς Σολομών είπε προς την μητέρα αυτού, Και διά τι συ ζητείς την Αβισάγ την Σουναμίτιν διά τον Αδωνίαν; ζήτησον δι' αυτόν και την βασιλείαν, διότι είναι μεγαλήτερός μου αδελφός· και δι' αυτόν και διά τον Αβιάθαρ τον ιερέα και διά τον Ιωάβ τον υιόν της Σερουΐας.
\par 23 Και ώμοσεν ο βασιλεύς Σολομών προς τον Κύριον, λέγων, Ούτω να κάμη ο Θεός εις εμέ και ούτω να προσθέση, εάν ο Αδωνίας δεν ελάλησε τον λόγον τούτον κατά της ζωής αυτού·
\par 24 και τώρα, ζη Κύριος, όστις με εστερέωσε και με εκάθισεν επί του θρόνου Δαβίδ του πατρός μου, και όστις έκαμεν εις εμέ οίκον, καθώς υπεσχέθη, σήμερον θέλει θανατωθή ο Αδωνίας.
\par 25 Και εξαπέστειλεν ο βασιλεύς Σολομών διά χειρός του Βεναΐα, υιού του Ιωδαέ, και έπεσεν επ' αυτόν και απέθανε.
\par 26 Προς δε τον Αβιάθαρ τον ιερέα είπεν ο βασιλεύς, Εις Αναθώθ ύπαγε, εις τους αγρούς σου· διότι είσαι άξιος θανάτου· αλλά την ημέραν ταύτην δεν θέλω σε θανατώσει, επειδή εσήκωσας την κιβωτόν Κυρίου του Θεού έμπροσθεν Δαβίδ του πατρός μου και επειδή εκακοπάθησας εις πάντα όσα εκακοπάθησεν ο πατήρ μου.
\par 27 Και απέβαλεν ο Σολομών τον Αβιάθαρ από του να ήναι ιερεύς του Κυρίου· διά να πληρωθή ο λόγος του Κυρίου, τον οποίον ελάλησε περί του οίκου του Ηλεί εν Σηλώ.
\par 28 Και η φήμη ήλθε μέχρι του Ιωάβ· διότι ο Ιωάβ έκλινεν οπίσω του Αδωνίου, αν και δεν έκλινεν οπίσω του Αβεσσαλώμ. Και έφυγεν ο Ιωάβ εις την σκηνήν του Κυρίου και επιάσθη από των κεράτων του θυσιαστηρίου.
\par 29 Και απηγγέλθη προς τον βασιλέα Σολομώντα, Ότι ο Ιωάβ έφυγεν εις την σκηνήν του Κυρίου· και ιδού, είναι πλησίον του θυσιαστηρίου. Τότε απέστειλεν ο Σολομών Βεναΐαν τον υιόν του Ιωδαέ, λέγων, Ύπαγε, πέσον επ' αυτόν.
\par 30 Και ήλθεν ο Βεναΐας εις την σκηνήν του Κυρίου και είπε προς αυτόν, ούτω λέγει ο βασιλεύς· Έξελθε. Ο δε είπεν, Ουχί, αλλ' ενταύθα θέλω αποθάνει. Και ανέφερεν ο Βεναΐας απόκρισιν προς τον βασιλέα, λέγων, Ούτως είπεν ο Ιωάβ και ούτω μοι απεκρίθη.
\par 31 Ο δε βασιλεύς είπε προς αυτόν, Κάμε ως είπε, και πέσον επ' αυτόν και θάψον αυτόν· διά να εξαλείψης το αθώον αίμα, το οποίον έχυσεν ο Ιωάβ, απ' εμού και από του οίκου του πατρός μου·
\par 32 και ο Κύριος θέλει στρέψει το αίμα αυτού κατά της κεφαλής αυτού, όστις έπεσεν επί δύο άνδρας δικαιοτέρους και καλητέρους παρ' αυτόν, και εθανάτωσεν αυτούς διά ρομφαίας, μη ειδότος του πατρός μου Δαβίδ, τον Αβενήρ τον υιόν του Νηρ, τον αρχιστράτηγον του Ισραήλ, και τον Αμασά τον υιόν του Ιεθέρ, τον αρχιστράτηγον του Ιούδα·
\par 33 και θέλουσιν επιστρέψει τα αίματα αυτών κατά της κεφαλής του Ιωάβ και κατά της κεφαλής του σπέρματος αυτού, εις τον αιώνα· επί δε τον Δαβίδ και επί το σπέρμα αυτού και επί τον οίκον αυτού και επί τον θρόνον αυτού θέλει είσθαι ειρήνη παρά Κυρίου έως αιώνος.
\par 34 Τότε ανέβη Βεναΐας ο υιός του Ιωδαέ, και έπεσεν επ' αυτόν και εθανάτωσεν αυτόν· και ετάφη εν τω οίκω αυτού εν τη ερήμω.
\par 35 Και κατέστησεν ο βασιλεύς αντ' αυτού Βεναΐαν τον υιόν του Ιωδαέ επί του στρατεύματος· και Σαδώκ τον ιερέα κατέστησεν ο βασιλεύς αντί του Αβιάθαρ.
\par 36 Και αποστείλας ο βασιλεύς εκάλεσε τον Σιμεΐ και είπε προς αυτόν, Οικοδόμησον εις σεαυτόν οίκον εν Ιερουσαλήμ και κατοίκει εκεί, και μη εξέλθης εκείθεν εις ουδέν μέρος·
\par 37 διότι καθ' ην ημέραν εξέλθης και περάσης τον χείμαρρον Κέδρων, έξευρε βεβαίως ότι εξάπαντος θέλεις θανατωθή· το αίμα σου θέλει είσθαι επί την κεφαλήν σου.
\par 38 Και είπεν ο Σιμεΐ προς τον βασιλέα, Καλός ο λόγος· καθώς είπεν ο κύριός μου ο βασιλεύς, ούτω θέλει κάμει ο δούλός σου. Και εκάθησεν ο Σιμεΐ εν Ιερουσαλήμ ημέρας πολλάς.
\par 39 Και μετά τρία έτη, δύο εκ των δούλων του Σιμεΐ εδραπέτευσαν προς τον Αγχούς, υιόν του Μααχά, τον βασιλέα της Γάθ· και ανήγγειλαν προς τον Σιμεΐ, λέγοντες, Ιδού, οι δούλοί σου είναι εν Γαθ.
\par 40 Και ο Σιμεΐ εσηκώθη και έστρωσε την όνον αυτού και υπήγεν εις Γαθ προς τον Αγχούς, διά να ζητήση τους δούλους αυτού· και υπήγεν ο Σιμεΐ και έφερε τους δούλους αυτού από Γαθ.
\par 41 Και απηγγέλθη προς τον Σολομώντα, ότι ο Σιμεΐ υπήγεν από Ιερουσαλήμ εις Γαθ και επέστρεψε.
\par 42 Και αποστείλας ο βασιλεύς εκάλεσε τον Σιμεΐ και είπε προς αυτόν, Δεν σε ώρκισα εις τον Κύριον και διεμαρτυρήθην προς σε, λέγων, Έξευρε βεβαίως, ότι καθ' ην ημέραν εξέλθης και περιπατήσης έξω οπουδήποτε, εξάπαντος θέλεις αποθάνει; και συ μοι είπας, Καλός ο λόγος, τον οποίον ήκουσα·
\par 43 διά τι λοιπόν δεν εφύλαξας τον όρκον του Κυρίου και την προσταγήν, την οποίαν προσέταξα εις σε;
\par 44 Και είπεν ο βασιλεύς προς τον Σιμεΐ, Συ εξεύρεις όλην την κακίαν, την οποίαν γνωρίζει η καρδία σου, τι έπραξας εις τον Δαβίδ τον πατέρα μου· διά τούτο ο Κύριος έστρεψε την κακίαν σου κατά της κεφαλής σου·
\par 45 ο δε βασιλεύς Σολομών θέλει είσθαι ευλογημένος, και ο θρόνος του Δαβίδ εστερεωμένος ενώπιον του Κυρίου έως αιώνος.
\par 46 Τότε ο βασιλεύς προσέταξε Βεναΐαν τον υιόν του Ιωδαέ, όστις εξελθών έπεσεν επ' αυτόν, και απέθανε. Και η βασιλεία εστερεώθη εν τη χειρί του Σολομώντος.

\chapter{3}

\par 1 Έκαμε δε ο Σολομών επιγαμίαν μετά του Φαραώ, βασιλέως της Αιγύπτου, και έλαβε την θυγατέρα του Φαραώ· και έφερεν αυτήν εις την πόλιν Δαβίδ, εωσού ετελείωσε να οικοδομή τον οίκον αυτού και τον οίκον του Κυρίου και το τείχος της Ιερουσαλήμ κύκλω.
\par 2 Πλην ο λαός εθυσίαζεν επί τους υψηλούς τόπους, επειδή δεν ήτο ωκοδομημένος οίκος εις το όνομα του Κυρίου, έως των ημερών εκείνων.
\par 3 Και ηγάπησεν ο Σολομών τον Κύριον, περιπατών εις τα προστάγματα Δαβίδ του πατρός αυτού· μόνον εθυσίαζε και εθυμίαζεν επί τους υψηλούς τόπους.
\par 4 Και υπήγεν ο βασιλεύς εις Γαβαών, διά να θυσιάση εκεί· διότι εκείνος ήτο ο υψηλός τόπος ο μέγας· χίλια ολοκαυτώματα προσέφερεν ο Σολομών επί το θυσιαστήριον εκείνο.
\par 5 Εφάνη δε ο Κύριος εν Γαβαών εις τον Σολομώντα καθ' ύπνον διά νυκτός· και είπεν ο Θεός, Ζήτησον τι να σοι δώσω.
\par 6 Ο δε Σολομών είπε, Συ έκαμες μέγα έλεος προς τον δούλον σου Δαβίδ τον πατέρα μου, επειδή περιεπάτησεν ενώπιόν σου εν αληθεία και εν δικαιοσύνη και εν ευθύτητι καρδίας μετά σού· και εφύλαξας εις αυτόν το μέγα τούτο έλεος και έδωκας εις αυτόν υιόν καθήμενον επί του θρόνου αυτού, καθώς την ημέραν ταύτην·
\par 7 και τώρα, Κύριε Θεέ μου, συ έκαμες τον δούλον σου βασιλέα αντί Δαβίδ του πατρός μου· και εγώ είμαι παιδάριον μικρόν· δεν εξεύρω πως να εξέρχωμαι και να εισέρχωμαι·
\par 8 και ο δούλός σου είναι εν μέσω του λαού σου, τον οποίον εξέλεξας, λαού μεγάλου, όστις εκ του πλήθους δεν δύναται να αριθμηθή ουδέ να λογαριασθή·
\par 9 δος λοιπόν εις τον δούλον σου καρδίαν νοήμονα εις το να κρίνη τον λαόν σου, διά να διακρίνω μεταξύ καλού και κακού· διότι τις δύναται να κρίνη τον λαόν σου τούτον τον μέγαν;
\par 10 Και ήρεσεν ο λόγος εις τον Κύριον, ότι ο Σολομών εζήτησε το πράγμα τούτο.
\par 11 Και είπεν ο Θεός προς αυτόν, Επειδή εζήτησας το πράγμα τούτο, και δεν εζήτησας εις σεαυτόν πολυζωΐαν, και δεν εζήτησας εις σεαυτόν πλούτη, και δεν εζήτησας την ζωήν των εχθρών σου, αλλ' εζήτησας εις σεαυτόν σύνεσιν διά να εννοής κρίσιν,
\par 12 ιδού, έκαμα κατά τους λόγους σου· ιδού, έδωκα εις σε καρδίαν σοφήν και συνετήν, ώστε δεν εστάθη πρότερόν σου όμοιός σου, ουδέ μετά σε θέλει αναστηθή όμοιός σου·
\par 13 έτι δε έδωκα εις σε και ό,τι δεν εζήτησας, και πλούτον και δόξαν, ώστε μεταξύ των βασιλέων δεν θέλει είσθαι ουδείς όμοιός σου καθ' όλας τας ημέρας σου·
\par 14 και εάν περιπατής εις τας οδούς μου, φυλάττων τα διατάγματά μου και τας εντολάς μου, καθώς περιεπάτησε Δαβίδ ο πατήρ σου, τότε θέλω μακρύνει τας ημέρας σου.
\par 15 Και εξύπνησεν ο Σολομών· και ιδού, ήτο ενύπνιον. Και ήλθεν εις Ιερουσαλήμ και εστάθη ενώπιον της κιβωτού της διαθήκης του Κυρίου, και προσέφερεν ολοκαυτώματα και έκαμεν ειρηνικάς προσφοράς και έκαμε συμπόσιον εις πάντας τους δούλους αυτού.
\par 16 Τότε ήλθον δύο γυναίκες πόρναι προς τον βασιλέα και εστάθησαν έμπροσθεν αυτού.
\par 17 Και είπεν η μία γυνή, Ω, κύριέ μου εγώ και η γυνή αύτη κατοικούμεν εν τη αυτή οικία, και εγέννησα συγκατοικούσα μετ' αυτής·
\par 18 την δε τρίτην ημέραν αφού εγώ εγέννησα, εγέννησε και η γυνή αύτη· και ήμεθα ομού· δεν ήτο ξένος μεθ' ημών εν τη οικία· μόνον ημείς αι δύο ήμεθα εν τη οικία·
\par 19 και την νύκτα απέθανεν ο υιός της γυναικός ταύτης, επειδή εκοιμήθη επ' αυτόν·
\par 20 και αυτή σηκωθείσα το μεσονύκτιον, έλαβε τον υιόν μου εκ του πλαγίου μου, ενώ η δούλη σου εκοιμάτο, και έβαλεν αυτόν εις τον κόλπον αυτής· τον δε υιόν αυτής τον νεκρόν έβαλεν εις τον κόλπον μου·
\par 21 και ότε εσηκώθην το πρωΐ, διά να θηλάσω τον υιόν μου, ιδού, ήτο νεκρός· πλην αφού το πρωΐ παρετήρησα αυτό, ιδού, δεν ήτο ο υιός μου τον οποίον εγέννησα.
\par 22 Η δε άλλη γυνή είπεν, Ουχί, αλλ' ο ζων είναι ο υιός μου, ο δε νεκρός είναι ο υιός σου. Η δε είπεν, Ουχί, αλλ' ο νεκρός είναι ο υιός σου, ο δε ζων είναι ο υιός μου. Ούτως ελάλησαν ενώπιον του βασιλέως.
\par 23 Και είπεν ο βασιλεύς, Η μεν λέγει, Ούτος ο ζων είναι ο υιός μου, ο δε νεκρός είναι ο υιός σου· η δε λέγει, Ουχί, αλλ' ο νεκρός είναι ο υιός σου, ο δε ζων είναι ο υιός μου.
\par 24 Και είπεν ο βασιλεύς, φέρετέ μοι μάχαιραν. Και έφεραν την μάχαιραν έμπροσθεν του βασιλέως.
\par 25 Και είπεν ο βασιλεύς, Διαιρέσατε εις δύο το παιδίον το ζων, και δότε το ήμισυ εις την μίαν και το ήμισυ εις την άλλην.
\par 26 Τότε η γυνή, της οποίας ήτο ο υιός ο ζων, ελάλησε προς τον βασιλέα, διότι τα σπλάγχνα αυτής επόνεσαν διά τον υιόν αυτής, και είπεν, Ω, κύριέ μου, δος εις αυτήν το παιδίον το ζων, και κατ' ουδένα τρόπον μη θανατώσης αυτό. Η δε άλλη είπε, Μήτε ιδικόν μου ας ήναι, μήτε ιδικόν σου· διαιρέσατε αυτό.
\par 27 Τότε αποκριθείς ο βασιλεύς, είπε, Δότε εις αυτήν το παιδίον το ζων, και κατ' ουδένα τρόπον μη θανατώσητε αυτό· αύτη είναι μήτηρ αυτού.
\par 28 Και ήκουσε πας ο Ισραήλ περί της κρίσεως, την οποίαν ο βασιλεύς έκρινε, και εφοβήθησαν τον βασιλέα· διότι είδον ότι σοφία Θεού ήτο εν αυτώ διά να κάμνη κρίσιν·

\chapter{4}

\par 1 Ο δε βασιλεύς Σολομών εβασίλευεν επί πάντα τον Ισραήλ.
\par 2 Και ούτοι ήσαν οι άρχοντες, τους οποίους είχεν· Αζαρίας, ο υιός του Σαδώκ, αυλάρχης·
\par 3 Ελιορέφ και Αχιά, οι υιοί του Σεισά, γραμματείς· Ιωσαφάτ, ο υιός του Αχιούδ, υπομνηματογράφος·
\par 4 και Βεναΐας, ο υιός του Ιωδαέ, επί του στρατεύματος· και Σαδώκ και Αβιάθαρ, ιερείς·
\par 5 και Αζαρίας, ο υιός του Νάθαν, επί τους σιτάρχας· και Ζαβούδ, υιός του Νάθαν, πρώτος αξιωματικός, φίλος του βασιλέως·
\par 6 και Αχισάρ, οικονόμος· και Αδωνιράμ, ο υιός του Αβδά, επί των φόρων.
\par 7 είχε δε ο Σολομών δώδεκα σιτάρχας επί πάντα τον Ισραήλ, και προέβλεπον τας τροφάς εις τον βασιλέα και εις τον οίκον αυτού· ενός μηνός πρόβλεψιν έκαμνεν έκαστος τον χρόνον.
\par 8 Και ταύτα είναι τα ονόματα αυτών· ο υιός του Ουρ σιτάρχης εν τω όρει Εφραΐμ·
\par 9 ο υιός του Δεκέρ, εν Μακάς και εν Σααλβίμ και Βαιθ-σεμές και Αιλών της Βαιθ-ανάν·
\par 10 ο υιός του Έσεδ, εν Αρουβώθ· υπό τούτον ήτο Σωχώ και πάσα γη Εφέρ·
\par 11 ο υιός του Αβιναάβ, εν πάση τη Νάφαθ-δώρ· ούτος είχε γυναίκα Ταφάθ, την θυγατέρα του Σολομώντος·
\par 12 Βαανά, ο υιός του Αχιλούδ, εν Θαανάχ και Μεγιδδώ και πάση τη αιθ-σαν, ήτις είναι πλησίον της Σαρθανά υπό την Ιεζραέλ, από Βαιθ-σαν έως Αβέλ-μεολά, έως επέκεινα Ιοκμεάμ·
\par 13 ο υιός του Γεβέρ, εν Ραμώθ-γαλαάδ· ούτος είχε τας κώμας του Ιαείρ, υιού Μανασσή, τας εν Γαλαάδ· ούτος είχε και την επαρχίαν Αργόβ, την εν Βασάν, εξήκοντα πόλεις μεγάλας με τείχη και χαλκίνους μοχλούς·
\par 14 Αχιναδάβ, ο υιός του Ιδδώ, εν Μαχαναΐμ·
\par 15 Αχιμάας, εν Νεφθαλί· και ούτος έλαβε διά γυναίκα Βασεμάθ, την θυγατέρα του Σολομώντος·
\par 16 Βαανά, ο υιός του Χουσαΐ, εν Ασιήρ και εν Αλώθ·
\par 17 Ιωσαφάτ, ο υιός του Φαρούα, εν Ισσάχαρ·
\par 18 Σιμεΐ, ο υιός του Ηλά, εν Βενιαμίν·
\par 19 Γεβέρ, ο υιός του Ουρεί, εν γη Γαλαάδ, τη γη του Σηών βασιλέως των Αμορραίων και του Ωγ βασιλέως της Βασάν· και ήτο ο μόνος σιτάρχης εν ταύτη τη γη.
\par 20 Ο Ιούδας και ο Ισραήλ ήσαν πολυάριθμοι ως η άμμος η παρά την θάλασσαν κατά το πλήθος, τρώγοντες και πίνοντες και ευθυμούντες.
\par 21 Και εξουσίαζεν ο Σολομών επί πάντα τα βασίλεια, από του ποταμού έως της γης των Φιλισταίων, και έως των ορίων της Αιγύπτου· και έφερον δώρα και ήσαν δούλοι εις τον Σολομώντα καθ' όλας τας ημέρας της ζωής αυτού.
\par 22 Η δε τροφή του Σολομώντος διά μίαν ημέραν ήτο τριάκοντα κόροι σεμιδάλεως και εξήκοντα κόροι αλεύρου,
\par 23 δέκα βόες σιτευτοί και είκοσι βόες νομαδικοί και εκατόν πρόβατα, εκτός ελάφων και αγρίων αιγών και δορκάδων και πτηνών θρεπτών.
\par 24 Διότι εξουσίαζεν επί πάσαν την γην εντεύθεν του ποταμού, από Θαψά έως Γάζης, επί πάντας τους βασιλείς εντεύθεν του ποταμού· και είχεν ειρήνην πανταχόθεν κύκλω αυτού.
\par 25 Κατώκει δε ο Ιούδας και ο Ισραήλ εν ασφαλεία, έκαστος υπό την άμπελον αυτού και υπό την συκήν αυτού, από Δαν έως Βηρ-σαβεέ, πάσας τας ημέρας του Σολομώντος.
\par 26 Και είχεν ο Σολομών τεσσαράκοντα χιλιάδας σταύλους ίππων διά τας αμάξας αυτού και δώδεκα χιλιάδας ιππείς.
\par 27 Και οι σιτάρχαι εκείνοι προεμήθευον τροφάς διά τον βασιλέα Σολομώντα και διά πάντας τους προσερχομένους εις την τράπεζαν του βασιλέως Σολομώντος, έκαστος εις τον μήνα αυτού· δεν άφινον να γίνηται ουδεμία έλλειψις.
\par 28 Έφερον έτι κριθάς και άχυρον διά τους ίππους και τας ημιόνους, εις τον τόπον όπου ήσαν, έκαστος κατά το διωρισμένον εις αυτόν.
\par 29 Και έδωκεν ο Θεός εις τον Σολομώντα σοφίαν και φρόνησιν πολλήν σφόδρα και έκτασιν πνεύματος, ως η άμμος η παρά το χείλος της θαλάσσης.
\par 30 Και υπερέβη η σοφία του Σολομώντος την σοφίαν πάντων των κατοίκων της ανατολής και πάσαν την σοφίαν της Αιγύπτου·
\par 31 διότι ήτο σοφώτερος παρά πάντας τους ανθρώπους, παρά τον Εθάν τον Εζραΐτην και τον Αιμάν και τον Χαλκόλ και τον Δαρδά, τους υιούς του Μαώλ· και η φήμη αυτού ήτο εις πάντα τα έθνη κύκλω.
\par 32 Και ελάλησε τρισχιλίας παροιμίας· και αι ωδαί αυτού ήσαν χίλιαι και πέντε.
\par 33 Και ελάλησε περί δένδρων, από της κέδρου της εν τω Λιβάνω, μέχρι της υσσώπου της εκφυομένης επί του τοίχου· ελάλησεν έτι περί τετραπόδων και περί πτηνών και περί ερπετών και περί ιχθύων.
\par 34 Και ήρχοντο εκ πάντων των λαών διά να ακούσωσι την σοφίαν του Σολομώντος, παρά πάντων των βασιλέων της γης, όσοι ήκουον την σοφίαν αυτού.

\chapter{5}

\par 1 Και απέστειλεν ο Χειράμ βασιλεύς της Τύρου τους δούλους αυτού προς τον Σολομώντα, ακούσας ότι έχρισαν αυτόν βασιλέα αντί του πατρός αυτού· διότι ο Χειράμ ηγάπα πάντοτε τον Δαβίδ.
\par 2 Και απέστειλεν ο Σολομών προς τον Χειράμ, λέγων,
\par 3 Συ εξεύρεις ότι Δαβίδ ο πατήρ μου δεν ηδυνήθη να οικοδομήση οίκον εις το όνομα Κυρίου του Θεού αυτού, εξ αιτίας των πολέμων των περικυκλούντων αυτόν πανταχόθεν, εωσού ο Κύριος έβαλε τους εχθρούς αυτού υπό τα ίχνη των ποδών αυτού·
\par 4 αλλά τώρα Κύριος ο Θεός μου έδωκεν εις εμέ ανάπαυσιν πανταχόθεν· δεν υπάρχει ούτε επίβουλος ούτε απάντημα κακόν·
\par 5 και ιδού, εγώ λέγω να οικοδομήσω οίκον εις το όνομα Κυρίου του Θεού μου, καθώς ο Κύριος ελάλησε προς τον Δαβίδ τον πατέρα μου, λέγων, Ο υιός σου, τον οποίον θέλω βάλει αντί σου επί τον θρόνον σου, ούτος θέλει οικοδομήσει τον οίκον εις το όνομά μου·
\par 6 τώρα λοιπόν πρόσταξον να κόψωσιν εις εμέ κέδρους εκ του Λιβάνου· και οι δούλοι μου θέλουσιν είσθαι μετά των δούλων σου· και θέλω δώσει εις σε μισθόν διά τους δούλους σου, κατά πάντα όσα είπας· διότι συ εξεύρεις ότι μεταξύ ημών δεν είναι ουδείς ούτως έμπειρος να κόπτη ξύλα, ως οι Σιδώνιοι.
\par 7 Και ως ήκουσεν ο Χειράμ τους λόγους του Σολομώντος, εχάρη σφόδρα και είπεν, Ευλογητός Κύριος σήμερον, όστις έδωκεν εις τον Δαβίδ υιόν σοφόν επί τον λαόν τον πολύν τούτον.
\par 8 Και απέστειλεν ο Χειράμ προς τον Σολομώντα, λέγων, Ήκουσα περί όσων εμήνυσας προς εμέ· εγώ θέλω κάμει παν το θέλημά σου διά ξύλα κέδρινα και διά ξύλα πεύκινα·
\par 9 οι δούλοί μου θέλουσι καταβιβάζει αυτά εκ του Λιβάνου εις την θάλασσαν· και εγώ θέλω κάμει να φέρωσιν αυτά εις σχεδίας διά της θαλάσσης μέχρι του τόπου όντινα μηνύσης προς εμέ, και να λύσωσιν αυτά εκεί· συ δε θέλεις παραλάβει αυτά· θέλεις δε εκπληρώσει και συ το θέλημά μου, δίδων τροφάς διά τον οίκόν μου.
\par 10 Έδιδε λοιπόν ο Χειράμ εις τον Σολομώντα ξύλα κέδρινα και ξύλα πεύκινα, όσα ήθελεν.
\par 11 Ο δε Σολομών έδωκεν εις τον Χειράμ είκοσι χιλιάδας κόρων σίτου διά τροφήν του οίκου αυτού και είκοσι κόρους ελαίου κοπανισμένου· ούτως έδιδεν ο Σολομών εις τον Χειράμ κατ' έτος.
\par 12 Και έδωκεν ο Κύριος εις τον Σολομώντα σοφίαν, καθώς είπε προς αυτόν· και ήτο ειρήνη μεταξύ Χειράμ και Σολομώντος· και έκαμον συνθήκην αμφότεροι.
\par 13 Έκαμε δε ο βασιλεύς Σολομών ανδρολογίαν εκ παντός του Ισραήλ, και ήτο η ανδρολογία τριάκοντα χιλιάδες ανδρών.
\par 14 Και απέστελλεν αυτούς εις τον Λίβανον, δέκα χιλιάδας τον μήνα κατά αλλαγήν· ένα μήνα ήσαν εν τω Λιβάνω και δύο μήνας εν τοις οίκοις αυτών· επί δε της ανδρολογίας ήτο ο Αδωνιράμ.
\par 15 Και είχεν ο Σολομών εβδομήκοντα χιλιάδας αχθοφόρων και ογδοήκοντα χιλιάδας λιθοτόμων εν τω όρει·
\par 16 εκτός των επιστατών των διωρισμένων παρά του Σολομώντος, οίτινες ήσαν επί των έργων, τρεις χιλιάδες και τριακόσιοι, επιστατούντες επί τον λαόν τον δουλεύοντα εις τα έργα.
\par 17 Προσέταξε δε ο βασιλεύς, και μετέφεραν λίθους μεγάλους, λίθους εκλεκτούς, λίθους πελεκητούς, διά τα θεμέλια του οίκου.
\par 18 Και επελέκησαν οι οικοδόμοι του Σολομώντος και οι οικοδόμοι του Χειράμ και οι Γίβλιοι, και ητοίμασαν τα ξύλα και τους λίθους, διά να οικοδομήσωσι τον οίκον.

\chapter{6}

\par 1 Και εν τω τετρακοσιοστώ και ογδοηκοστώ έτει από της εξόδου των υιών Ισραήλ εκ γης Αιγύπτου, το τέταρτον έτος της βασιλείας του Σολομώντος επί τον Ισραήλ, κατά τον μήνα Ζιφ, όστις είναι ο δεύτερος μην, ήρχισε να οικοδομή τον οίκον του Κυρίου.
\par 2 Και του οίκου, τον οποίον ο βασιλεύς Σολομών ωκοδόμησεν εις τον Κύριον, το μήκος αυτού ήτο εξήκοντα πηχών, και το πλάτος αυτού είκοσι, και το ύψος αυτού τριάκοντα πηχών.
\par 3 Το δε πρόναον, το κατά πρόσωπον του ναού του οίκου, είχε μήκος είκοσι πηχών, κατά το πλάτος του οίκου· και πλάτος δέκα πηχών έμπροσθεν του οίκου.
\par 4 Και έκαμεν εις τον οίκον παράθυρα πλάγια αδιόρατα.
\par 5 Και ωκοδόμησε κολλητά με τον τοίχον του οίκου οικήματα κύκλω, κολλητά με τους τοίχους του οίκου κύκλω, και του ναού και του χρηστηρίου· ούτως έκαμεν οικήματα κύκλω.
\par 6 Του κατωτέρου οικήματος το πλάτος ήτο πέντε πηχών, και του μέσου εξ πηχών το πλάτος, και του τρίτου επτά πηχών το πλάτος· διότι έξωθεν του οίκου έκαμε στενά υποστηρίγματα κύκλω, διά να μη εισέρχωνται αι δοκοί εις τους τοίχους του οίκου.
\par 7 Και ο οίκος, ενώ ωκοδομείτο, ωκοδομήθη με λίθους προητοιμασμένους πριν μετακομισθώσιν εκεί· ώστε ούτε σφύρα ούτε πέλεκυς ουδέν σιδηρούν εργαλείον δεν ηκούσθη εν τω οίκω ενώ ωκοδομείτο.
\par 8 Η θύρα των μέσων οικημάτων ήτο εις την δεξιάν πλευράν του οίκου· και ανέβαινον εις τα οικήματα του μέσου διά κλίμακος ελικοειδούς, και εκ του μέσου εις τα τριώροφα.
\par 9 Ούτως ωκοδόμησε τον οίκον και ετελείωσεν αυτόν· και εστέγασε τον οίκον με οροφάς κοιλωτάς και κοσμήματα εκ κέδρου.
\par 10 Και ωκοδόμησε τα οικήματα κολλητά εφ' όλον τον οίκον, πέντε πηχών το ύψος· και συνείχοντο μετά του οίκου διά ξύλων κεδρίνων.
\par 11 Και ήλθεν ο λόγος του Κυρίου προς τον Σολομώντα, λέγων,
\par 12 Περί του οίκου τούτου, τον οποίον συ οικοδομείς, εάν περιπατής εις τα διατάγματά μου και εκτελής τας κρίσεις μου και φυλάττης πάσας τας εντολάς μου περιπατών εις αυτάς, τότε θέλω βεβαιώσει τον λόγον μου μετά σου, τον οποίον ελάλησα προς Δαβίδ τον πατέρα σού·
\par 13 και θέλω κατοικεί εκ μέσω των υιών Ισραήλ, και δεν θέλω εγκαταλίπει τον λαόν μου Ισραήλ.
\par 14 Ούτως ωκοδόμησεν ο Σολομών τον οίκον και συνετέλεσεν αυτόν.
\par 15 Και εσανίδωσε τους τοίχους του οίκου έσωθεν με σανίδας κεδρίνας, από του εδάφους του οίκου έως των τοίχων της στέγης· με ξύλον εσκέπασεν αυτά έσωθεν· και εσκέπασε το έδαφος του οίκου με σανίδας πευκίνας.
\par 16 Εσανίδωσεν έτι με σανίδας κεδρίνας είκοσι πήχας εις το ενδότερον του οίκου, από του εδάφους έως των τοίχων· και εσανίδωσεν αυτό έσωθεν διά να ήναι το χρηστήριον, το άγιον των αγίων.
\par 17 Ο δε οίκος, τουτέστι ο ναός ο κατέμπροσθεν, ήτο τεσσαράκοντα πηχών μήκους.
\par 18 Και τα κέδρινα ξύλα του οίκου έσωθεν ήσαν γεγλυμμένα με κάλυκας και ανοικτά άνθη· τα πάντα κέδρινα· δεν εφαίνετο λίθος.
\par 19 Και ητοίμασε το χρηστήριον εις το ενδότερον του οίκου, διά να θέση εκεί την κιβωτόν της διαθήκης του Κυρίου.
\par 20 Και το χρηστήριον είχε κατά πρόσωπον είκοσι πηχών μήκος και είκοσι πηχών πλάτος και είκοσι πηχών ύψος· και εσκέπασεν αυτό με καθαρόν χρυσίον· ούτως εσκέπασε και το θυσιαστήριον με κέδρον.
\par 21 Και εσκέπασεν ο Σολομών τον οίκον έσωθεν με καθαρόν χρυσίον· και έκαμε χώρισμα με αλύσεις χρυσάς έμπροσθεν του χρηστηρίου και εσκέπασεν αυτό με χρυσίον.
\par 22 Και όλον τον οίκον εσκέπασε με χρυσίον, εωσού συνετέλεσεν όλον τον οίκον· εσκέπασεν έτι με χρυσίον όλον το θυσιαστήριον, το πλησίον του χρηστηρίου.
\par 23 Έσωθεν δε του χρηστηρίου έκαμε δύο χερουβείμ εκ ξύλου ελαίας, δέκα πηχών το ύψος.
\par 24 Και ήτο πέντε πήχαι η μία πτέρυξ του χερούβ και πέντε πήχαι η άλλη πτέρυξ του χερούβ· από του άκρου της μιας πτέρυγος αυτού έως του άκρου της άλλης πτέρυγος αυτού, δέκα πήχαι.
\par 25 Και το άλλο χερούβ ήτο δέκα πηχών· του αυτού μέτρου και της αυτής κατασκευής ήσαν αμφότερα τα χερουβείμ.
\par 26 Το ύψος του ενός χερούβ δέκα πηχών, και ούτω του άλλου χερούβ.
\par 27 Και έθεσε τα χερουβείμ εν μέσω του ενδοτάτου οίκου· και είχον τα χερουβείμ τας πτέρυγας αυτών εξηπλωμένας, ώστε η πτέρυξ του ενός ήγγιζε τον ένα τοίχον· και η πτέρυξ του άλλου χερούβ ήγγιζε τον άλλον τοίχον· και αι πτέρυγες αυτών ήγγιζον, η μία την άλλην, εν τω μέσω του οίκου.
\par 28 Και εσκέπασε τα χερουβείμ με χρυσίον.
\par 29 Και πάντας τους τοίχους του οίκου κύκλω ενέγλυψε με γλυπτά σχήματα χερουβείμ και φοινίκων και ανοικτών ανθέων, έσωθεν και έξωθεν.
\par 30 Και το έδαφος του οίκου εσκέπασε με χρυσίον, έσωθεν και έξωθεν.
\par 31 Και διά την είσοδον του χρηστηρίου έκαμε θύρας εκ ξύλου ελαίας· το ανώφλιον και οι παραστάται ήσαν εν πεντάγωνον.
\par 32 Και αι δύο θύραι ήσαν εκ ξύλου ελαίας· και ενέγλυψεν επ' αυταίς γλυπτά χερουβείμ και φοίνικας και ανοικτά άνθη, και εσκέπασεν αυτά με χρυσίον, εφαπλώσας το χρυσίον επί τα χερουβείμ και επί τους φοίνικας.
\par 33 Ούτως έκαμε και εις την πύλην του ναού παραστάτας εκ ξύλου ελαίας, εν τετράγωνον.
\par 34 Και αι δύο θύραι ήσαν εκ ξύλου πευκίνου· τα δύο φύλλα της μιας θύρας εδιπλόνοντο, και τα δύο φύλλα της άλλης θύρας εδιπλόνοντο.
\par 35 Και ενέγλυψεν επ' αυτάς χερουβείμ και φοίνικας και ανοικτά άνθη· και εσκέπασεν αυτά με χρυσίον εφηρμοσμένον επί την ανάγλυφον εργασίαν.
\par 36 Και ωκοδόμησε την εσωτέραν αυλήν με τρεις σειράς πελεκητών λίθων και με μίαν σειράν δοκών κεδρίνων.
\par 37 Εν τω τετάρτω έτει, τον μήνα Ζιφ, ετέθησαν τα θεμέλια του οίκου του Κυρίου·
\par 38 και εν τω ενδεκάτω έτει, τον μήνα Βουλ, όστις είναι ο όγδοος μην, ετελειώθη ο οίκος κατά πάντα τα μέρη αυτού και κατά πάσαν την κατασκευήν αυτού· ούτως εις επτά έτη ωκοδόμησεν αυτόν.

\chapter{7}

\par 1 Και τον οίκον αυτού ωκοδόμησεν ο Σολομών εις δεκατρία έτη, και ετελείωσεν όλον τον οίκον αυτού.
\par 2 Και ωκοδόμησε τον οίκον του δάσους του Λιβάνου· το μήκος αυτού ήτο εκατόν πηχών, και το πλάτος αυτού πεντήκοντα πηχών, και το ύψος αυτού τριάκοντα πηχών, επί τεσσάρων σειρών στύλων κεδρίνων, με δοκούς κεδρίνους επί των στύλων.
\par 3 Και εστεγάσθη με κέδρον άνωθεν των δοκών, αίτινες επεστηρίζοντο επί τεσσαράκοντα πέντε στύλων, δεκαπέντε εις την σειράν.
\par 4 Και ήσαν παράθυρα εις τρεις σειράς, και ανταπεκρίνετο παράθυρον εις παράθυρον κατά τρεις σειράς.
\par 5 Και πάσαι αι θύραι και οι παραστάται ήσαν τετράγωνοι, με τα παράθυρα· και ανταπεκρίνετο παράθυρον εις παράθυρον κατά τρεις σειράς.
\par 6 Και έκαμε την στοάν εκ στύλων· το μήκος αυτής πεντήκοντα πηχών, και το πλάτος αυτής τριάκοντα πηχών· και ήτο η στοά κατέμπροσθεν των στύλων του οίκου, ώστε οι στύλοι και αι δοκοί ήσαν κατά πρόσωπον αυτών.
\par 7 Έκαμεν έτι στοάν διά τον θρόνον, όπου έμελλε να κρίνη, την στοάν της κρίσεως· και ήτο εστρωμένη με κέδρον εκ του ενός μέρους του εδάφους έως του άλλου.
\par 8 Και ο οίκος αυτού, εις τον οποίον εκάθητο, είχε μίαν άλλην αυλήν έσωθεν της στοάς, ούσαν της αυτής κατασκευής. Ο Σολομών έκαμεν έτι οίκον διά την θυγατέρα του Φαραώ, την οποίαν είχε λάβει, όμοιον με την στοάν ταύτην.
\par 9 Πάντα ταύτα ήσαν εκ λίθων πολυτελών, κατά τα μέτρα των πριονισμένων λίθων, πριονισμένων διά πριονίου, έσωθεν και έξωθεν, εκ θεμελίου μέχρι του γείσου, και έξωθεν έως της μεγάλης αυλής.
\par 10 Και το θεμέλιον ήτο εκ λίθων πολυτελών, λίθων μεγάλων, λίθων δέκα πηχών και λίθων οκτώ πηχών.
\par 11 Και επάνωθεν ήσαν λίθοι πολυτελείς, κατά το μέτρον των πριονισμένων λίθων, και κέδροι.
\par 12 Και η μεγάλη αυλή κυκλόθεν ήτο εκ τριών σειρών λίθων πριονισμένων και εκ μιας σειράς κεδρίνων δοκών, καθώς η εσωτέρα αυλή του οίκου του Κυρίου και καθώς η στοά του οίκου.
\par 13 Και έστειλεν ο βασιλεύς Σολομών και έλαβε τον Χειράμ εκ της Τύρου.
\par 14 Ούτος ήτο υιός γυναικός χήρας εκ φυλής Νεφθαλί, και ο πατήρ αυτού ανήρ Τύριος, χαλκουργός· και ήτο πλήρης τέχνης και συνέσεως και επιστήμης εις το να εργάζηται παν έργον εν χαλκώ. Και ήλθε προς τον βασιλέα Σολομώντα και έκαμε πάντα τα έργα αυτού.
\par 15 Διότι έχυσε τους δύο χαλκίνους στύλους, δεκαοκτώ πηχών ύψους έκαστον στύλον· γραμμή δε δώδεκα πηχών περιεκύκλονεν έκαστον αυτών.
\par 16 Και έκαμεν εκ χυτού χαλκού δύο επιθέματα, διά να θέση αυτά επί τας κεφαλάς των στύλων· το ύψος του ενός επιθέματος πέντε πηχών, και το ύψος του άλλου επιθέματος πέντε πηχών·
\par 17 και δίκτυα πλεκτά ειργασμένα αλυσιδωτά εκ συρμάτων, διά τα επιθέματα τα επί της κεφαλής των στύλων· επτά διά το εν επίθεμα, και επτά διά το άλλο επίθεμα.
\par 18 Και έκαμε τους στύλους, και δύο σειράς ροδίων κυκλόθεν επί το εν δίκτυον, διά να σκεπάση με ρόδια τα επιθέματα τα επί της κεφαλής των στύλων· και έκαμε το αυτό εις το άλλο επίθεμα.
\par 19 Και τα επιθέματα, τα επί της κεφαλής των στύλων εν τη στοά, ήσαν εργασίας κρίνων τεσσάρων πηχών.
\par 20 Και τα επιθέματα τα επί των δύο στύλων είχον ρόδια και επάνωθεν, πλησίον της κοιλίας, της παρά το δικτυωτόν· και τα ρόδια ήσαν διακόσια κατά σειράν κυκλόθεν εφ' εκάστου επιθέματος.
\par 21 Και έστησε τους στύλους εις την στοάν του ναού· και έστησε τον στύλον τον δεξιόν, και εκάλεσε το όνομα αυτού Ιαχείν· και έστησε τον στύλον τον αριστερόν, και εκάλεσε το όνομα αυτού Βοάς.
\par 22 Και επί την κεφαλήν των στύλων ήτο εργασία κρίνων· ούτως ετελειώθη η κατασκευή των στύλων.
\par 23 Έκαμεν έτι την χυτήν θάλασσαν, δέκα πηχών από χείλους εις χείλος, στρογγύλην κύκλω· και το ύψος αυτής πέντε πηχών· και γραμμή τριάκοντα πηχών περιεζώννυεν αυτήν κύκλω.
\par 24 Και υπό το χείλος αυτής κύκλω ήσαν ανάγλυφα εις σχήμα κολοκύνθης περικυκλούντα αυτήν, δέκα κατά πήχην, περικυκλούντα την θάλασσαν κύκλω· αι δύο σειραί των αναγλύφων ήσαν χυμέναι ομού με αυτήν.
\par 25 Ίστατο δε επί δώδεκα βοών· τρεις έβλεπον προς βορράν, και τρεις έβλεπον προς δυσμάς, και τρεις έβλεπον προς νότον, και τρεις έβλεπον προς ανατολάς· και η θάλασσα έκειτο επ' αυτών· και όλα τα οπίσθια αυτών ήσαν προς τα έσω.
\par 26 Και το πάχος αυτής ήτο μιας παλάμης, και το χείλος αυτής κατεσκευασμένον ως χείλος ποτηρίου, ως άνθος κρίνου· εχώρει δε δύο χιλιάδας βαθ.
\par 27 Έκαμεν έτι δέκα βάσεις χαλκίνας· τεσσάρων πηχών το μήκος της μιας βάσεως, και τεσσάρων πηχών το πλάτος αυτής, και τριών πηχών το ύψος αυτής.
\par 28 Η δε εργασία των βάσεων ήτο τοιαύτη· είχον συγκλείσματα, και τα συγκλείσματα ήσαν εντός των κιονίσκων.
\par 29 Και επί των συγκλεισμάτων των εντός των κιονίσκων ήσαν λέοντες, βόες και χερουβείμ· και επί των κιονίσκων ήτο άνωθεν το υποβάσταγμα· υποκάτωθεν δε των λεόντων και βοών ήσαν κροσσοί ανάγλυφοι κρεμάμενοι.
\par 30 Και εκάστη βάσις είχε τέσσαρας χαλκίνους τροχούς και άξονας χαλκίνους· και αι τέσσαρες γωνίαι αυτής είχον ώμους· υπό τον λουτήρα ήσαν οι ώμοι χυτοί, έκαστος απέναντι των κροσσών.
\par 31 Και το στόμα αυτής, έσωθεν της κεφαλίδος και άνωθεν, ήτο μία πήχη· ήτο δε το στόμα αυτής στρογγύλον, κατεσκευασμένον εις το υποβάσταγμα, μία πήχη και ημίσεια· και έτι επάνω τούτου του στόματος αυτής ήσαν εγχαράγματα μετά των συγκλεισμάτων αυτών, τετράγωνα όντα, ουχί στρογγύλα.
\par 32 Και υπό τα συγκλείσματα ήσαν τέσσαρες τροχοί· και οι άξονες των τροχών ηνόνοντο με την βάσιν· και το ύψος εκάστου τροχού ήτο μιας πήχης και ημισείας.
\par 33 Και η εργασία των τροχών ήτο ως η εργασία του τροχού της αμάξης· οι άξονες αυτών και αι πλήμναι αυτών και επίσωτρα αυτών και αι ακτίνες αυτών ήσαν όλα χυτά.
\par 34 Και ήσαν τέσσαρες ώμοι εις τας τέσσαρας γωνίας εκάστης βάσεως· και οι ώμοι ήσαν συνέχεια της βάσεως.
\par 35 Και εν τη κορυφή της βάσεως ήτο στρογγύλον περίζωμα ημισείας πήχης το ύψος· και εν τη κορυφή της βάσεως τα χείλη αυτής και τα συγκλείσματα αυτής ήσαν εκ της αυτής.
\par 36 Επί δε τας πλάκας των χειλέων αυτής και επί τα συγκλείσματα αυτής, ενεχάραξε χερουβείμ, λέοντας και φοίνικας, κατά αναλογίαν εκάστης, και κροσσούς κύκλω.
\par 37 Κατά τούτον τον τρόπον έκαμε τας δέκα βάσεις· πάσαι είχον το αυτό χύσιμον, το αυτό μέτρον, το αυτό εγχάραγμα.
\par 38 Έκαμεν έτι δέκα λουτήρας χαλκίνους· έκαστος λουτήρ εχώρει τεσσαράκοντα βάθ· έκαστος λουτήρ ήτο τεσσάρων πηχών· και εφ' εκάστην των δέκα βάσεων ήτο εις λουτήρ.
\par 39 Και έθεσε τας βάσεις, πέντε επί το δεξιόν πλάγιον του οίκου και πέντε επί το αριστερόν πλάγιον του οίκου· και έθεσε την θάλασσαν κατά το δεξιόν πλάγιον του οίκου προς ανατολάς απέναντι του νοτίου μέρους.
\par 40 Και έκαμεν ο Χειράμ τους λουτήρας και τα πτύαρια και τας λεκάνας. Ούτως ετελείωσεν ο Χειράμ κάμνων πάντα τα έργα, τα οποία έκαμεν εις τον βασιλέα Σολομώντα διά τον οίκον του Κυρίου·
\par 41 τους δύο στύλους και τας σφαίρας των επιθεμάτων, των επί της κεφαλής των δύο στύλων· και τα δύο δικτυωτά, διά να σκεπάζωσι τας δύο σφαίρας των επιθεμάτων των επί της κεφαλής των στύλων·
\par 42 και τετρακόσια ρόδια διά τα δύο δικτυωτά, δύο σειράς ροδίων δι' έκαστον δικτυωτόν, διά να σκεπάζωσι τας δύο σφαίρας των επιθεμάτων των επί των στύλων·
\par 43 και τας δέκα βάσεις και τους δέκα λουτήρας επί των βάσεων·
\par 44 και την μίαν θάλασσαν, και τους δώδεκα βόας υποκάτω της θαλάσσης·
\par 45 και τους λέβητας και τα πτυάρια και τας λεκάνας· πάντα ταύτα τα σκεύη, τα οποία ο Χειράμ έκαμεν εις τον βασιλέα Σολομώντα διά τον οίκον του Κυρίου, ήσαν εκ χαλκού λαμπρού.
\par 46 Εν τη πεδιάδι του Ιορδάνου έχυσεν αυτά ο βασιλεύς, εν γη αργιλλώδει, μεταξύ Σοκχώθ και Σαρθάν.
\par 47 Και ο Σολομών αφήκε πάντα τα σκεύη αζύγιστα, διότι ήσαν πολλά σφόδρα· το βάρος του χαλκού δεν ηδύνατο λογαριασθή.
\par 48 Και έκαμεν ο Σολομών πάντα τα σκεύη τα του οίκου του Κυρίου, το θυσιαστήριον το χρυσούν, και την τράπεζαν την χρυσήν, επί της οποίας ετίθεντο οι άρτοι της προθέσεως,
\par 49 και τας λυχνίας, πέντε εκ δεξιών και πέντε εξ αριστερών, έμπροσθεν του χρηστηρίου, εκ χρυσίου καθαρού, και τα άνθη και τους λύχνους και τας λαβίδας εκ χρυσού,
\par 50 και τας φιάλας και τα λυχνοψάλιδα και τας λεκάνας και τους κρατήρας και τα θυμιατήρια εκ χρυσίου καθαρού, και τους στρόφιγγας εκ χρυσίου, διά τας θύρας του οίκου του εσωτάτου, του αγίου των αγίων, και διά τας θύρας του οίκου του ναού.
\par 51 Και συνετελέσθη άπαν το έργον, το οποίον έκαμεν ο βασιλεύς Σολομών διά τον οίκον του Κυρίου. Και εισέφερεν ο Σολομών τα αφιερώματα Δαβίδ του πατρός αυτού· το αργύριον και το χρυσίον, και τα σκεύη έθεσεν εν τοις θησαυροίς του οίκου του Κυρίου.

\chapter{8}

\par 1 Τότε συνήθροισεν ο βασιλεύς Σολομών προς εαυτόν εν Ιερουσαλήμ τους πρεσβυτέρους του Ισραήλ και πάντας τους αρχηγούς των φυλών, τους οικογενάρχας των υιών Ισραήλ, διά να αναβιβάσωσι την κιβωτόν της διαθήκης του Κυρίου εκ της πόλεως Δαβίδ, ήτις είναι η Σιών.
\par 2 Και συνηθροίσθησαν πάντες οι άνδρες Ισραήλ προς τον βασιλέα Σολομώντα εν τη εορτή κατά τον μήνα Εθανείμ, όστις είναι ο έβδομος μην.
\par 3 Και ήλθον πάντες οι πρεσβύτεροι του Ισραήλ και εσήκωσαν οι ιερείς την κιβωτόν.
\par 4 Και ανεβίβασαν την κιβωτόν του Κυρίου και την σκηνήν του μαρτυρίου και πάντα τα σκεύη τα άγια τα εν τη σκηνή· οι ιερείς και οι Λευΐται ανεβίβασαν αυτά.
\par 5 Και ο βασιλεύς Σολομών και πάσα η συναγωγή του Ισραήλ, οι συναχθέντες προς αυτόν, ήσαν μετ' αυτού έμπροσθεν της κιβωτού, θυσιάζοντες πρόβατα και βόας, όσα δεν ήτο δυνατόν να λογαριασθώσι και να αριθμηθώσι διά το πλήθος.
\par 6 Και εισήγαγον οι ιερείς την κιβωτόν της διαθήκης του Κυρίου εις τον τόπον αυτής, εις το χρηστήριον του οίκου, εις τα άγια των αγίων, υποκάτω των πτερύγων των χερουβείμ.
\par 7 Διότι τα χερουβείμ είχον εξηπλωμένας τας πτέρυγας επί τον τόπον της κιβωτού, και τα χερουβείμ εκάλυπτον την κιβωτόν και τους μοχλούς αυτής άνωθεν.
\par 8 Και εξείχον οι μοχλοί, και εφαίνοντο τα άκρα των μοχλών εκ του αγίου τόπου έμπροσθεν του χρηστηρίου, έξωθεν όμως δεν εφαίνοντο· και είναι εκεί έως της σήμερον.
\par 9 Δεν ήσαν εν τη κιβωτώ ειμή αι δύο λίθιναι πλάκες, τας οποίας ο Μωϋσής έθεσεν εκεί εν Χωρήβ, όπου ο Κύριος έκαμε διαθήκην προς τους υιούς Ισραήλ, ότε εξήλθον εκ γης Αιγύπτου.
\par 10 Και ως εξήλθον οι ιερείς εκ του αγιαστηρίου, η νεφέλη ενέπλησε τον οίκον του Κυρίου·
\par 11 και δεν ηδύναντο οι ιερείς να σταθώσι διά να λειτουργήσωσιν, εξ αιτίας της νεφέλης· διότι η δόξα του Κυρίου ενέπλησε τον οίκον του Κυρίου.
\par 12 Τότε ελάλησεν ο Σολομών, Ο Κύριος είπεν ότι θέλει κατοικεί εν γνόφω·
\par 13 ωκοδόμησα εις σε οίκον κατοικήσεως, τόπον διά να κατοικής αιωνίως.
\par 14 Και στρέψας ο βασιλεύς το πρόσωπον αυτού, ευλόγησε πάσαν την συναγωγήν του Ισραήλ· πάσα δε η συναγωγή του Ισραήλ ίστατο.
\par 15 Και είπεν, Ευλογητός Κύριος ο Θεός του Ισραήλ, όστις εξετέλεσε διά της χειρός αυτού εκείνο το οποίον ελάλησε διά του στόματος αυτού προς Δαβίδ τον πατέρα μου, λέγων,
\par 16 Αφ' ης ημέρας εξήγαγον τον λαόν μου τον Ισραήλ εξ Αιγύπτου, δεν εξέλεξα από πασών των φυλών του Ισραήλ ουδεμίαν πόλιν διά να οικοδομηθή οίκος, ώστε να ήναι το όνομά μου εκεί· αλλ' εξέλεξα τον Δαβίδ, διά να ήναι επί τον λαόν μου Ισραήλ.
\par 17 Και ήλθεν εις την καρδίαν Δαβίδ του πατρός μου να οικοδομήση οίκον εις το όνομα Κυρίου του Θεού του Ισραήλ.
\par 18 Αλλ' ο Κύριος είπε προς Δαβίδ τον πατέρα μου, Επειδή ήλθεν εις την καρδίαν σου να οικοδομήσης οίκον εις το όνομά μου, καλώς μεν έκαμες ότι συνέλαβες τούτο εν τη καρδία σου·
\par 19 πλην συ δεν θέλεις οικοδομήσει τον οίκον· αλλ' ο υιός σου, όστις θέλει εξέλθει εκ της οσφύος σου, ούτος θέλει οικοδομήσει τον οίκον εις το όνομά μου.
\par 20 Ο Κύριος λοιπόν εξεπλήρωσε τον λόγον αυτού, τον οποίον ελάλησε· και εγώ ανέστην αντί Δαβίδ του πατρός μου, και εκάθησα επί του θρόνου του Ισραήλ, καθώς ελάλησεν ο Κύριος, και ωκοδόμησα τον οίκον εις το όνομα Κυρίου του Θεού του Ισραήλ.
\par 21 Και διώρισα εκεί τόπον διά την κιβωτόν, εν ή κείται η διαθήκη του Κυρίου, την οποίαν έκαμε προς τους πατέρας ημών, ότε εξήγαγεν αυτούς εκ γης Αιγύπτου.
\par 22 Και σταθείς ο Σολομών έμπροσθεν του θυσιαστηρίου του Κυρίου, ενώπιον πάσης της συναγωγής του Ισραήλ, εξέτεινε τας χείρας αυτού προς τον ουρανόν,
\par 23 και είπε, Κύριε Θεέ του Ισραήλ, δεν είναι Θεός όμοιός σου εκ τω ουρανώ άνω και επί της γης κάτω, όστις φυλάττεις την διαθήκην και το έλεος προς τους δούλους σου τους περιπατούντας ενώπιόν σου εν όλη τη καρδία αυτών·
\par 24 όστις εφύλαξας προς τον δούλον σου Δαβίδ τον πατέρα μου όσα ελάλησας προς αυτόν· και ελάλησας διά του στόματός σου και εξετέλεσας διά της χειρός σου, καθώς την ημέραν ταύτην.
\par 25 Και τώρα, Κύριε Θεέ του Ισραήλ, φύλαξον προς τον δούλον σου Δαβίδ τον πατέρα μου εκείνο το οποίον υπεσχέθης προς αυτόν, λέγων, Δεν θέλει εκλείψει εις σε ανήρ απ' έμπροσθέν μου καθήμενος επί του θρόνου του Ισραήλ, μόνον εάν προσέχωσιν οι υιοί σου εις την οδόν αυτών, διά να περιπατώσιν ενώπιόν μου, καθώς συ περιεπάτησας ενώπιόν μου.
\par 26 Τώρα λοιπόν, Θεέ του Ισραήλ, ας αληθεύση, δέομαι, ο λόγος σου, τον οποίον ελάλησας προς τον δούλον σου Δαβίδ τον πατέρα μου.
\par 27 Αλλά θέλει αληθώς κατοικήσει Θεός επί της γης; ιδού, ο ουρανός και ο ουρανός των ουρανών δεν είναι ικανοί να σε χωρέσωσι· πόσον ολιγώτερον ο οίκος ούτος, τον οποίον ωκοδόμησα.
\par 28 Πλην επίβλεψον επί την προσευχήν του δούλου σου και επί την δέησιν αυτού, Κύριε Θεέ μου, ώστε να εισακούσης της κραυγής και της δεήσεως, την οποίαν ο δούλός σου δέεται ενώπιόν σου την σήμερον.
\par 29 διά να ήναι οι οφθαλμοί σου ανεωγμένοι προς τον οίκον τούτον νύκτα και ημέραν, προς τον τόπον περί του οποίου είπας, Το όνομά μου θέλει είσθαι εκεί· διά να εισακούης της δεήσεως, την οποίαν ο δούλός σου θέλει δέεσθαι εν τω τόπω τούτω.
\par 30 Και επάκουε της δεήσεως του δούλου σου και του λαού σου Ισραήλ, όταν προσεύχωνται εν τω τόπω τούτω· και άκουε συ εκ του τόπου της κατοικήσεώς σου, εκ του ουρανού· και ακούων, γίνου ίλεως.
\par 31 Εάν αμαρτήση τις άνθρωπος εις τον πλησίον αυτού και ζητήση όρκον παρ' αυτού διά να κάμη αυτόν να ορκισθή, και ο όρκος έλθη έμπροσθεν του θυσιαστηρίου σου εν τω οίκω τούτω,
\par 32 τότε συ επάκουσον εκ του ουρανού και ενέργησον και κρίνον τους δούλους σου, καταδικάζων μεν τον άνομον, ώστε να στρέψης κατά της κεφαλής αυτού την πράξιν αυτού, δικαιόνων δε τον δίκαιον, ώστε να αποδώσης εις αυτόν κατά την δικαιοσύνην αυτού.
\par 33 Όταν κτυπηθή ο λαός σου Ισραήλ έμπροσθεν του εχθρού, διότι ημάρτησαν εις σε, και επιστρέψωσι προς σε και δοξάσωσι το όνομά σου και προσευχηθώσι και δεηθώσιν ενώπιόν σου εν τω οίκω τούτω,
\par 34 τότε συ επάκουσον εκ του ουρανού και συγχώρησον την αμαρτίαν του λαού σου Ισραήλ, και επανάγαγε αυτούς εις την γην, την οποίαν έδωκας εις τους πατέρας αυτών.
\par 35 Όταν ο ουρανός κλεισθή, και δεν γίνηται βροχή, διότι ημάρτησαν εις σε, εάν προσευχηθώσι προς τον τόπον τούτον και δοξάσωσι το όνομά σου και επιστρέψωσιν από των αμαρτιών αυτών, αφού ταπεινώσης αυτούς,
\par 36 τότε συ επάκουσον εκ του ουρανού και συγχώρησον την αμαρτίαν των δούλων σου και του λαού σου Ισραήλ, διδάξας αυτούς την οδόν την αγαθήν, εις την οποίαν πρέπει να περιπατώσι, και δος βροχήν επί την γην σου, την οποίαν έδωκας εις τον λαόν σου διά κληρονομίαν.
\par 37 Πείνα εάν γείνη εν τη γη, θανατικόν εάν γείνη, ανεμοφθορία, ερυσίβη, ακρίς, βρούχος εάν γείνη, ο εχθρός αυτών εάν πολιορκήση αυτούς εν τω τόπω της κατοικίας αυτών, οποιαδήποτε πληγή, οποιαδήποτε νόσος γείνη,
\par 38 πάσαν προσευχήν, πάσαν δέησιν γινομένην υπό παντός ανθρώπου, υπό παντός του λαού σου Ισραήλ, όταν γνωρίση έκαστος την πληγήν της καρδίας αυτού και εκτείνη τας χείρας αυτού προς τον οίκον τούτον,
\par 39 τότε συ επάκουσον εκ του ουρανού, του τόπου της κατοικήσεώς σου, και συγχώρησον και ενέργησον και δος εις έκαστον κατά πάσας τας οδούς αυτού, όπως γνωρίζεις την καρδίαν αυτού, διότι συ, μόνος συ, γνωρίζεις τας καρδίας πάντων των υιών ανθρώπων.
\par 40 διά να σε φοβώνται πάσας τας ημέρας όσας ζώσιν επί προσώπου της γης, την οποίαν έδωκας εις τους πατέρας ημών.
\par 41 Και τον ξένον έτι, όστις δεν είναι εκ του λαού σου Ισραήλ, αλλ' έρχεται από γης μακράς διά το όνομά σου,
\par 42 διότι θέλουσιν ακούσει το όνομά σου το μέγα και την χείρα σου την κραταιάν και τον βραχίονά σου τον εξηπλωμένον, όταν έλθη και προσευχηθή προς τον οίκον τούτον,
\par 43 συ επάκουσον εκ του ουρανού, του τόπου της κατοικήσεώς σου, και ενέργησον κατά πάντα περί όσων ο ξένος σε επικαλεσθή· διά να γνωρίσωσι πάντες οι λαοί της γης το όνομά σου, να σε φοβώνται, καθώς ο λαός σου Ισραήλ· και διά να γνωρίσωσιν ότι το όνομά σου εκλήθη επί τον οίκον τούτον, τον οποίον ωκοδόμησα.
\par 44 Όταν ο λαός σου εξέλθη εις πόλεμον εναντίον των εχθρών αυτών, όπου αποστείλης αυτούς, και προσευχηθώσιν εις τον Κύριον, προς την πόλιν, την οποίαν εξέλεξας, και τον οίκον, τον οποίον ωκοδόμησα εις το όνομά σου,
\par 45 τότε επάκουσον εκ του ουρανού της προσευχής αυτών και της δεήσεως αυτών, και κάμε το δίκαιον αυτών.
\par 46 Όταν αμαρτήσωσιν εις σε, διότι ουδείς άνθρωπος είναι αναμάρτητος, και οργισθής εις αυτούς και παραδώσης αυτούς εις τον εχθρόν, ώστε οι αιχμαλωτισταί να φέρωσιν αυτούς αιχμαλώτους εις την γην του εχθρού, μακράν ή πλησίον,
\par 47 και έλθωσιν εις εαυτούς εν τη γη, όπου εφέρθησαν αιχμάλωτοι, και επιστρέψωσι και δεηθώσι προς σε εν τη γη των αιχμαλωτισάντων αυτούς, λέγοντες, Ημάρτομεν, ηνομήσαμεν, ηδικήσαμεν,
\par 48 και επιστρέψωσι προς σε εξ όλης της καρδίας αυτών και εξ όλης της ψυχής αυτών, εν τη γη των εχθρών των αιχμαλωτισάντων αυτούς, και προσευχηθώσι προς σε προς την γην αυτών την οποίαν έδωκας εις τους πατέρας αυτών, την πόλιν την οποίαν εξέλεξας, και τον οίκον τον οποίον ωκοδόμησα εις το όνομά σου,
\par 49 τότε επάκουσον εκ του ουρανού, του τόπου της κατοικήσεώς σου, της προσευχής αυτών και της δεήσεως αυτών και κάμε το δίκαιον αυτών,
\par 50 και συγχώρησον εις τον λαόν σου, τον αμαρτήσαντα εις σε, και άφες πάσας τας παραβάσεις αυτών, διά των οποίων έγειναν παραβάται εναντίον σου, και κίνησον εις οικτιρμόν αυτών τους αιχμαλωτίσαντας αυτούς ώστε να οικτείρωσιν αυτούς·
\par 51 διότι λαός σου και κληρονομία σου είναι, τον οποίον εξήγαγες εξ Αιγύπτου, εκ μέσου του σιδηρού χωνευτηρίου.
\par 52 Ας ήναι λοιπόν οι οφθαλμοί σου ανεωγμένοι εις την δέησιν του δούλου σου και εις την δέησιν του λαού σου Ισραήλ, διά να εισακούης αυτούς περί όσων σε επικαλεσθώσι,
\par 53 διότι συ εξεχώρισας αυτούς από πάντων των λαών της γης, διά να ήναι κληρονομία σου, καθώς ελάλησας διά χειρός Μωϋσέως του δούλου σου, ότε εξήγαγες τους πατέρας ημών εξ Αιγύπτου, Δέσποτα Κύριε.
\par 54 Και αφού ετελείωσεν ο Σολομών να κάμνη όλην την προσευχήν και την δέησιν ταύτην προς τον Κύριον, εσηκώθη απ' έμπροσθεν του θυσιαστηρίου του Κυρίου, όπου ήτο γονυπετής με τας χείρας αυτού εξηπλωμένας προς τον ουρανόν.
\par 55 Και εστάθη και ευλόγησε πάσαν την σύναξιν του Ισραήλ μετά φωνής μεγάλης, λέγων,
\par 56 Ευλογητός Κύριος, όστις έδωκεν ανάπαυσιν εις τον λαόν αυτού Ισραήλ, κατά πάντα όσα υπεσχέθη· δεν έπεσεν ουδέ εις εκ πάντων των λόγων των αγαθών, τους οποίους ελάλησε διά χειρός Μωϋσέως του δούλου αυτού.
\par 57 Γένοιτο Κύριος ο Θεός ημών μεθ' ημών, καθώς ήτο μετά των πατέρων ημών να μη αφήση ημάς, μηδέ να εγκαταλείψη ημάς·
\par 58 διά να επικλίνη τας καρδίας ημών εις εαυτόν ώστε να περιπατώμεν εις πάσας τας οδούς αυτού και να φυλάττωμεν τας εντολάς αυτού και τα διατάγματα αυτού και τας κρίσεις αυτού, τα οποία προσέταξεν εις τους πατέρας ημών.
\par 59 Και ούτοι οι λόγοι μου, τους οποίους εδεήθην ενώπιον του Κυρίου, να ήναι ημέραν και νύκτα πλησίον Κυρίου του Θεού ημών, διά να κάμνη το δίκαιον του δούλου αυτού και το δίκαιον του λαού αυτού Ισραήλ, κατά την ανάγκην εκάστης ημέρας·
\par 60 διά να γνωρίσωσι πάντες οι λαοί της γης, ότι ο Κύριος, αυτός είναι ο Θεός, ουδείς άλλος.
\par 61 Ας ήναι λοιπόν η καρδία σας τελεία προς Κύριον τον Θεόν ημών, διά να περιπατήτε εις τα διατάγματα αυτού και να φυλάττητε τας εντολάς αυτού, καθώς εν τη ημέρα ταύτη.
\par 62 Και ο βασιλεύς και πας ο Ισραήλ μετ' αυτού, προσέφεραν θυσίαν ενώπιον του Κυρίου.
\par 63 Και εθυσίασεν Σολομών τας θυσίας τας ειρηνικάς, τας οποίας προσέφερεν εις τον Κύριον, εικοσιδύο χιλιάδας βοών και εκατόν είκοσι χιλιάδας προβάτων· ούτως εγκαινίασαν τον οίκον του Κυρίου ο βασιλεύς και πάντες οι υιοί Ισραήλ.
\par 64 Την αυτήν ημέραν καθιέρωσεν ο βασιλεύς το μέσον της αυλής της κατά πρόσωπον του οίκου του Κυρίου· διότι εκεί προσέφερε τα ολοκαυτώματα και την εξ αλφίτων προσφοράν και το στέαρ των ειρηνικών προσφορών· επειδή το θυσιαστήριον το χάλκινον, το κατ' έμπροσθεν του Κυρίου, ήτο μικρόν ώστε να χωρέση τα ολοκαυτώματα και την εξ αλφίτων προσφοράν και το στέαρ των ειρηνικών προσφορών.
\par 65 Και κατ' εκείνον τον καιρόν έκαμεν Σολομών την εορτήν, και πας ο Ισραήλ μετ' αυτού, σύναξις μεγάλη, από της εισόδου Αιμάθ μέχρι του ποταμού Αιγύπτου, ενώπιον Κυρίου του Θεού ημών, επτά ημέρας και επτά ημέρας, δεκατέσσαρας ημέρας.
\par 66 την ογδόην ημέραν απέλυσε τον λαόν· και ευλόγησαν τον βασιλέα και ανεχώρησαν εις τας σκηνάς αυτών, χαίροντες και ευφραινόμενοι εκ καρδίας, διά πάντα τα αγαθά όσα ο Κύριος έκαμε προς Δαβίδ τον δούλον αυτού και προς Ισραήλ τον λαόν αυτού.

\chapter{9}

\par 1 Και αφού ετελείωσεν ο Σολομών, να οικοδομή τον οίκον του Κυρίου και τον οίκον του βασιλέως και πάντα όσα επεθύμει ο Σολομών και ήθελε να κάμη,
\par 2 εφάνη ο Κύριος εις τον Σολομώντα δευτέραν φοράν, καθώς εφάνη εις αυτόν εν Γαβαών.
\par 3 Και είπεν ο Κύριος προς αυτόν, Ήκουσα της προσευχής σου και της δεήσεώς σου, την οποίαν εδεήθης ενώπιόν μου. Ηγίασα τον οίκον τούτον, τον οποίον ωκοδόμησας, διά να θέσω εκεί το όνομά μου εις τον αιώνα· και θέλουσιν είσθαι οι οφθαλμοί μου και η καρδία μου εκεί διά παντός.
\par 4 Και συ εάν περιπατήσης ενώπιόν μου, καθώς περιεπάτησε Δαβίδ ο πατήρ σου, εν ακεραιότητι καρδίας και εν ευθύτητι, ώστε να κάμνης κατά πάντα όσα προσέταξα εις σε, να φυλάττης τα διατάγματά μου και τας κρίσεις μου,
\par 5 τότε θέλω στερεώσει τον θρόνον της βασιλείας σου επί τον Ισραήλ εις τον αιώνα, καθώς υπεσχέθην προς Δαβίδ τον πατέρα σου, λέγων, Δεν θέλει εκλείψει εις σε ανήρ επάνωθεν του θρόνου του Ισραήλ.
\par 6 Εάν ποτέ στραφήτε απ' εμού, σεις ή τα τέκνα σας, και δεν φυλάξητε τας εντολάς μου, τα διατάγματά μου, τα οποία έθεσα έμπροσθέν σας, αλλά υπάγητε και λατρεύσητε άλλους θεούς και προσκυνήσητε αυτούς,
\par 7 τότε θέλω εκριζώσει τον Ισραήλ από προσώπου της γης, την οποίαν έδωκα εις αυτούς· και τον οίκον τούτον, τον οποίον ηγίασα διά το όνομά μου, θέλω απορρίψει από προσώπου μου· και ο Ισραήλ θέλει είσθαι εις παροιμίαν και εμπαιγμόν μεταξύ πάντων των λαών.
\par 8 Περί δε του οίκου τούτου, όστις έγεινε τόσον υψηλός, πας ο διαβαίνων πλησίον αυτού θέλει μένει έκθαμβος και θέλει κάμνει συριγμόν· και θέλουσι λέγει, Διά τι ο Κύριος έκαμεν ούτως εις την γην ταύτην και εις τον οίκον τούτον;
\par 9 Και θέλουσιν αποκρίνεσθαι, Επειδή εγκατέλιπον Κύριον τον Θεόν αυτών, όστις εξήγαγε τους πατέρας αυτών εκ γης Αιγύπτου, και προσεκολλήθησαν εις άλλους θεούς και προσεκύνησαν αυτούς και ελάτρευσαν αυτούς, διά τούτο ο Κύριος επέφερεν επ' αυτούς άπαν τούτο το κακόν.
\par 10 Εν δε τω τέλει των είκοσι ετών καθ' α ο Σολομών ωκοδόμησε τους δύο οίκους, τον οίκον του Κυρίου και τον οίκον του βασιλέως,
\par 11 ο δε Χειράμ ο βασιλεύς της Τύρου είχε βοηθήσει τον Σολομώντα με ξύλα κέδρου και με ξύλα πεύκης και με χρυσίον, καθ' όλην την επιθυμίαν αυτού, τότε ο βασιλεύς Σολομών έδωκεν εις τον Χειράμ είκοσι πόλεις εν τη γη της Γαλιλαίας.
\par 12 Και εξήλθεν ο Χειράμ από της Τύρου διά να ίδη τας πόλεις, τας οποίας έδωκεν ο Σολομών εις αυτόν· και δεν ήρεσαν εις αυτόν.
\par 13 Και είπε, Τι είναι αι πόλεις αύται, τας οποίας μοι έδωκας, αδελφέ μου; και εκάλεσεν αυτάς γην Καβούλ, έως της ημέρας ταύτης.
\par 14 Και απέστειλεν ο Χειράμ εις τον βασιλέα εκατόν είκοσι τάλαντα χρυσίου.
\par 15 Ούτος δε είναι ο τρόπος του φόρου, τον οποίον επέβαλεν ο βασιλεύς Σολομών, διά να οικοδομήση τον οίκον του Κυρίου και τον οίκον εαυτού και την Μιλλώ και το περιτείχισμα της Ιερουσαλήμ και την Ασώρ και την Μεγιδδώ και την Γεζέρ.
\par 16 Διότι Φαραώ ο βασιλεύς Αιγύπτου είχεν αναβή και κυριεύσει την Γεζέρ και κατακαύσει αυτήν εν πυρί, και τους Χαναναίους τους κατοικούντας εν τη πόλει είχε φονεύσει και είχε δώσει αυτήν δώρον εις την θυγατέρα αυτού, την γυναίκα του Σολομώντος.
\par 17 Και ωκοδόμησεν ο Σολομών την Γεζέρ και την Βαιθ-ωρών την κατωτέραν,
\par 18 και την Βααλάθ, και την Θαδμώρ εν τη ερήμω της γης,
\par 19 και πάσας τας πόλεις των αποθηκών, τας οποίας είχεν ο Σολομών, και τας πόλεις των αμαξών και τας πόλεις των ιππέων και ό,τι επεθύμησεν ο Σολομών να οικοδομήση εν Ιερουσαλήμ και εν τω Λιβάνω και εν πάση τη γη της επικρατείας αυτού.
\par 20 Πάντα δε τον λαόν τον εναπολειφθέντα εκ των Αμορραίων, των Χετταίων, των Φερεζαίων, των Ευαίων και των Ιεβουσαίων, οίτινες δεν ήσαν εκ των υιών Ισραήλ,
\par 21 αλλ' εκ των τέκνων εκείνων των εναπολειφθέντων εν τη γη, τους οποίους οι υιοί Ισραήλ δεν ηδυνήθησαν να εξολοθρεύσωσιν, επί τούτους ο Σολομών επέβαλε φόρον έως της ημέρας ταύτης.
\par 22 Εκ δε των υιών Ισραήλ ο Σολομών δεν έκαμεν ουδένα δούλον· διότι ήσαν άνδρες πολεμισταί και θεράποντες αυτού και μεγιστάνες αυτού και ταξίαρχοι αυτού και άρχοντες των αμαξών αυτού και των ιππέων αυτού.
\par 23 Οι δε αρχηγοί των επιστατούντων επί τα έργα του Σολομώντος, ήσαν πεντακόσιοι πεντήκοντα, εξουσιάζοντες επί τον λαόν τον δουλεύοντα εις τα έργα.
\par 24 Ανέβη δε η θυγάτηρ του Φαραώ εκ της πόλεως Δαβίδ εις τον οίκον αυτής, τον οποίον ο Σολομών ωκοδόμησε δι' αυτήν· τότε ωκοδόμησε την Μιλλώ.
\par 25 Και προσέφερεν ο Σολομών τρίς του ενιαυτού ολοκαυτώματα και ειρηνικάς προσφοράς επί του θυσιαστηρίου, το οποίον ωκοδόμησεν εις τον Κύριον, και εθυμίαζεν επί του όντος έμπροσθεν του Κυρίου· ούτως ετελείωσε τον οίκον.
\par 26 Έκαμε δε στόλον ο βασιλεύς Σολομών εν Εσιών-γάβερ, ήτις είναι πλησίον της Αιλώθ, επί το χείλος της Ερυθράς θαλάσσης, εν τη γη Εδώμ.
\par 27 Και απέστειλεν ο Χειράμ εις τον στόλον εκ των δούλων αυτού ναύτας εμπείρους της θαλάσσης, μετά των δούλων του Σολομώντος.
\par 28 Και ήλθον εις Οφείρ και έλαβον εκείθεν τετρακόσια και είκοσι τάλαντα χρυσίου και έφεραν προς τον βασιλέα Σολομώντα.

\chapter{10}

\par 1 Ακούσασα δε η βασίλισσα της Σεβά την περί του ονόματος του Κυρίου φήμην του Σολομώντος, ήλθε διά να δοκιμάση αυτόν δι' αινιγμάτων.
\par 2 Και ήλθεν εις Ιερουσαλήμ μετά συνοδίας μεγάλης σφόδρα, μετά καμήλων πεφορτωμένων αρώματα και χρυσόν πολύν σφόδρα και λίθους πολυτίμους· και ότε ήλθε προς τον Σολομώντα, ελάλησε μετ' αυτού περί πάντων όσα είχεν εν τη καρδία αυτής.
\par 3 Και εξήγησεν εις αυτήν ο Σολομών πάντα τα ερωτήματα αυτής· δεν εστάθη ουδέν κεκρυμμένον από του βασιλέως, το οποίον δεν εξήγησεν εις αυτήν.
\par 4 Και ιδούσα η βασίλισσα της Σεβά πάσαν την σοφίαν του Σολομώντος και τον οίκον τον οποίον ωκοδόμησε,
\par 5 και τα φαγητά της τραπέζης αυτού και την καθεδρίασιν των δούλων αυτού και την στάσιν των υπουργών αυτού και τον ιματισμόν αυτών και τους οινοχόους αυτού και την ανάβασιν αυτού, δι' ης ανέβαινεν εις τον οίκον του Κυρίου, έγεινεν έκθαμβος.
\par 6 Και είπε προς τον βασιλέα, Αληθής ήτο ο λόγος, τον οποίον ήκουσα εν τη γη μου, περί των έργων σου και περί της σοφίας σου.
\par 7 Αλλά δεν επίστευον εις τους λόγους, εωσού ήλθον, και είδον οι οφθαλμοί μου· και ιδού, το ήμισυ δεν απηγγέλθη εις εμέ· η σοφία σου και η ευημερία σου υπερβαίνουσι την φήμην την οποίαν ήκουσα·
\par 8 μακάριοι οι άνδρες σου, μακάριοι οι δούλοί σου ούτοι, οι ιστάμενοι πάντοτε ενώπιόν σου, οι ακούοντες την σοφίαν σου·
\par 9 έστω Κύριος ο Θεός σου ευλογημένος, όστις ευηρεστήθη εις σε, διά να σε θέση επί τον θρόνον του Ισραήλ· επειδή ο Κύριος ηγάπησεν εις τον αιώνα τον Ισραήλ, διά τούτο σε κατέστησε βασιλέα, διά να κάμνης κρίσιν και δικαιοσύνην.
\par 10 Και έδωκεν εις τον βασιλέα εκατόν είκοσι τάλαντα χρυσίου και αρώματα πολλά σφόδρα και λίθους πολυτίμους· δεν ήλθε πλέον τόση αφθονία αρωμάτων, ως εκείνα τα οποία η βασίλισσα της Σεβά έδωκεν εις τον βασιλέα Σολομώντα.
\par 11 Και ο στόλος έτι του Χειράμ, όστις έφερε το χρυσίον από Οφείρ, έφερεν από Οφείρ και μέγα πλήθος ξύλων αλμουγείμ και λίθους τιμίους.
\par 12 Και έκαμεν ο βασιλεύς εκ των ξύλων αλμουγείμ αναβάσεις εις τον οίκον του Κυρίου και εις τον οίκον του βασιλέως, και κιθάρας και ψαλτήρια διά τους μουσικούς· τοιαύτα ξύλα αλμουγείμ δεν είχον ελθεί ουδέ φανή έως της ημέρας ταύτης.
\par 13 Και έδωκεν ο βασιλεύς Σολομών εις την βασίλισσαν της Σεβά πάντα όσα ηθέλησεν, όσα εζήτησεν, εκτός των όσα έδωκεν εις αυτήν οίκοθεν ο βασιλεύς Σολομών. Και επέστρεψε και ήλθεν εις την γην αυτής, αυτή και οι δούλοι αυτής.
\par 14 Το βάρος δε του χρυσίου, το οποίον ήρχετο εις τον Σολομώντα κατ' έτος, ήτο εξακόσια εξήκοντα εξ τάλαντα χρυσίου,
\par 15 εκτός του συναγομένου εκ των τελωνών και εκ των πραγματειών των εμπόρων και εκ πάντων των βασιλέων της Αραβίας και εκ των σατραπών της γης.
\par 16 Και έκαμεν ο βασιλεύς Σολομών διακοσίους θυρεούς εκ χρυσίου σφυρηλάτου· εξακόσιοι σίκλοι χρυσίου εξωδεύοντο εις έκαστον θυρεόν·
\par 17 και τριακοσίας ασπίδας εκ χρυσίου σφυρηλάτου· τρεις μναι χρυσίου εξωδεύοντο εις εκάστην ασπίδα· και έθεσεν αυτάς ο βασιλεύς εν τω οίκω του δάσους του Λιβάνου.
\par 18 Έκαμεν έτι ο βασιλεύς θρόνον μέγαν ελεφάντινον και εσκέπασεν αυτόν με καθαρόν χρυσίον.
\par 19 είχε δε ο θρόνος εξ βαθμίδας, και η κορυφή του θρόνου ήτο στρογγύλη όπισθεν αυτού, και αγκώνας εντεύθεν και εντεύθεν της καθέδρας και δύο λέοντας ισταμένους εις τα πλάγια των αγκώνων.
\par 20 Επί δε των εξ βαθμίδων, εκεί ίσταντο δώδεκα λέοντες εκατέρωθεν· παρόμοιον δεν κατεσκευάσθη εις ουδέν βασίλειον.
\par 21 Και πάντα τα σκεύη του ποτού του βασιλέως Σολομώντος ήσαν εκ χρυσίου, και πάντα τα σκεύη του οίκου του δάσους του Λιβάνου εκ χρυσίου καθαρού· ουδέν εξ αργυρίου· το αργύριον ελογίζετο εις ουδέν εν ταις ημέραις του Σολομώντος.
\par 22 Διότι είχεν ο βασιλεύς εν τη θαλάσση στόλον της Θαρσείς μετά του στόλου του Χειράμ· άπαξ κατά τριετίαν ήρχετο ο στόλος από Θαρσείς, φέρων χρυσόν και άργυρον, οδόντας ελέφαντος και πιθήκους και παγώνια.
\par 23 Και εμεγαλύνθη ο βασιλεύς Σολομών υπέρ πάντας τους βασιλείς της γης εις πλούτον και εις σοφίαν.
\par 24 Και πάσα η γη εζήτει το πρόσωπον του Σολομώντος, διά να ακούσωσι την σοφίαν αυτού, την οποίαν ο Θεός έδωκεν εις την καρδίαν αυτού.
\par 25 Και έφερον έκαστος αυτών το δώρον αυτού, σκεύη αργυρά και σκεύη χρυσά και στολάς και πανοπλίας και αρώματα, ίππους και ημιόνους, κατ' έτος.
\par 26 Και συνήθροισεν ο Σολομών αμάξας και ιππείς· και είχε χιλίας τετρακοσίας αμάξας και δώδεκα χιλιάδας ιππέων, τους οποίους έθεσεν εις τας πόλεις των αμαξών και πλησίον του βασιλέως εν Ιερουσαλήμ.
\par 27 Και κατέστησεν ο βασιλεύς εν Ιερουσαλήμ τον άργυρον ως λίθους, και τας κέδρους κατέστησεν ως τας εν τη πεδιάδι συκαμίνους, διά την αφθονίαν.
\par 28 Εγίνετο δε εις τον Σολομώντα εξαγωγή ίππων και λινού νήματος εξ Αιγύπτου· το μεν νήμα ελάμβανον οι έμποροι του βασιλέως εις ωρισμένην τιμήν.
\par 29 Εκάστη δε άμαξα ανέβαινε και εξήρχετο εξ Αιγύπτου διά εξακοσίους σίκλους αργυρούς, και έκαστος ίππος διά εκατόν πεντήκοντα· και ούτω διά πάντας τους βασιλείς των Χετταίων και διά τους βασιλείς της Συρίας η εξαγωγή εγίνετο διά χειρός αυτών.

\chapter{11}

\par 1 Ηγάπησε δε ο βασιλεύς Σολομών πολλάς ξένας γυναίκας, εκτός της θυγατρός του Φαραώ, Μωαβίτιδας, Αμμωνίτιδας, Ιδουμαίας, Σιδωνίας, Χετταίας·
\par 2 εκ των εθνών περί των οποίων ο Κύριος είπε προς τους υιούς Ισραήλ, Δεν θέλετε εισέλθει προς αυτά, ουδέ αυτά θέλουσιν εισέλθει προς εσάς, μήποτε εκκλίνωσι τας καρδίας σας κατόπιν των θεών αυτών· εις αυτά ο Σολομών προσεκολλήθη με έρωτα.
\par 3 Και είχε γυναίκας βασιλίδας επτακοσίας και παλλακάς τριακοσίας· και αι γυναίκες αυτού εξέκλιναν την καρδίαν αυτού.
\par 4 Διότι ότε εγήρασεν ο Σολομών, αι γυναίκες αυτού εξέκλιναν την καρδίαν αυτού κατόπιν άλλων θεών· και η καρδία αυτού δεν ήτο τελεία μετά του Κυρίου του Θεού αυτού, ως η καρδία Δαβίδ του πατρός αυτού.
\par 5 Και επορεύθη ο Σολομών κατόπιν της Αστάρτης, της θεάς των Σιδωνίων, και κατόπιν του Μελχώμ, του βδελύγματος των Αμμωνιτών.
\par 6 Και έπραξεν ο Σολομών πονηρά ενώπιον του Κυρίου και δεν επορεύθη εντελώς κατόπιν του Κυρίου, ως Δαβίδ ο πατήρ αυτού.
\par 7 Τότε ωκοδόμησεν ο Σολομών υψηλόν τόπον εις τον Χεμώς, το βδέλυγμα του Μωάβ, εν τω όρει τω απέναντι της Ιερουσαλήμ, και εις τον Μολόχ, το βδέλυγμα των υιών Αμμών.
\par 8 Και ούτως έκαμε δι' όλας τας γυναίκας αυτού τας ξένας, αίτινες εθυμίαζον και εθυσίαζον εις τους θεούς αυτών.
\par 9 Και ωργίσθη ο Κύριος κατά του Σολομώντος επειδή η καρδία αυτού εξέκλινεν από του Κυρίου του Θεού του Ισραήλ, όστις εφανερώθη δις εις αυτόν,
\par 10 και προσέταξεν εις αυτόν περί του πράγματος τούτου, να μη υπάγη κατόπιν άλλων θεών· δεν εφύλαξεν όμως εκείνο, το οποίον ο Κύριος προσέταξε.
\par 11 Διά τούτο είπεν ο Κύριος εις τον Σολομώντα, Επειδή τούτο ευρέθη εν σοι, και δεν εφύλαξας την διαθήκην μου και τα διατάγματά μου, τα οποία προσέταξα εις σε, θέλω εξάπαντος διαρρήξει την βασιλείαν από σου και δώσει αυτήν εις τον δούλον σου·
\par 12 πλην εν ταις ημέραις σου δεν θέλω κάμει τούτο, χάριν Δαβίδ του πατρός σου· εκ της χειρός του υιού σου θέλω διαρρήξει αυτήν·
\par 13 δεν θέλω όμως διαρρήξει πάσαν την βασιλείαν· μίαν φυλήν θέλω δώσει εις τον υιόν σου, χάριν Δαβίδ του δούλου μου, και χάριν της Ιερουσαλήμ, την οποίαν εξέλεξα.
\par 14 Και εσήκωσεν ο Κύριος αντίπαλον εις τον Σολομώντα, τον Αδάδ τον Ιδουμαίον· ούτος ήτο εκ του σπέρματος των βασιλέων της Ιδουμαίας.
\par 15 Διότι, ότε ήτο ο Δαβίδ εν τη Ιδουμαία και Ιωάβ ο αρχιστράτηγος ανέβη να θάψη τους θανατωθέντας και επάταξε παν αρσενικόν εν τη Ιδουμαία,
\par 16 επειδή εξ μήνας εκάθησεν εκεί ο Ιωάβ μετά παντός του Ισραήλ, εωσού εξωλόθρευσε παν αρσενικόν εκ της Ιδουμαίας,
\par 17 τότε ο Αδάδ έφυγεν, αυτός και μετ' αυτού τινές Ιδουμαίοι εκ των δούλων του πατρός αυτού, διά να υπάγωσιν εις την Αίγυπτον· ήτο δε ο Αδάδ μικρόν παιδίον.
\par 18 Και εσηκώθησαν εκ της Μαδιάμ και ήλθον εις Φαράν· και έλαβον μεθ' εαυτών άνδρας εκ Φαράν και ήλθον εις Αίγυπτον προς τον Φαραώ βασιλέα της Αιγύπτου· όστις έδωκεν εις αυτόν οικίαν και διέταξεν εις αυτόν τροφάς και γην έδωκεν εις αυτόν.
\par 19 Και εύρηκεν ο Αδάδ μεγάλην χάριν ενώπιον του Φαραώ, ώστε έδωκεν εις αυτόν γυναίκα την αδελφήν της γυναικός αυτού, την αδελφήν της Ταχπενές της βασιλίσσης.
\par 20 Και εγέννησεν εις αυτόν η αδελφή της Ταχπενές τον Γενουβάθ τον υιόν αυτού, τον οποίον η Ταχπενές απεγαλάκτισεν εντός του οίκου του Φαραώ· και ήτο ο Γενουβάθ εν τω οίκω του Φαραώ, μεταξύ των υιών του Φαραώ.
\par 21 Και ότε ήκουσεν ο Αδάδ εν Αιγύπτω ότι εκοιμήθη ο Δαβίδ μετά των πατέρων αυτού και ότι απέθανεν Ιωάβ ο αρχιστράτηγος, είπεν ο Αδάδ προς τον Φαραώ, Εξαπόστειλόν με, διά να απέλθω εις την γην μου.
\par 22 Και είπε προς αυτόν ο Φαραώ, Αλλά τι σοι λείπει πλησίον μου, και ιδού, συ ζητείς να απέλθης εις την γην σου; Και απεκρίθη, Ουδέν· αλλ' εξαπόστειλόν με, παρακαλώ.
\par 23 Και εσήκωσεν ο Θεός εις αυτόν και άλλον αντίπαλον, τον Ρεζών, υιόν του Ελιαδά, όστις είχε φύγει από του κυρίου αυτού Αδαδέζερ, βασιλέως της Σωβά·
\par 24 και συναθροίσας εις εαυτόν άνδρας, έγεινεν αρχηγός συμμορίας, ότε επάταξεν ο Δαβίδ τους από Σωβά· και υπήγαν εις Δαμασκόν και κατώκησαν εκεί και εβασίλευσαν εν Δαμασκώ·
\par 25 και ήτο αντίπαλος του Ισραήλ πάσας τας ημέρας του Σολομώντος, εκτός των κακών, τα οποία έκαμεν ο Αδάδ· και επηρέαζε τον Ισραήλ, βασιλεύων επί της Συρίας.
\par 26 Και ο Ιεροβοάμ, υιός του Ναβάτ, Εφραθαίος από Σαρηδά, δούλος του Σολομώντος, του οποίου η μήτηρ ωνομάζετο Σερουά, γυνή χήρα, και ούτος εσήκωσε χείρα κατά του βασιλέως.
\par 27 Αύτη δε ήτο η αιτία, διά την οποίαν εσήκωσε χείρα κατά του βασιλέως· ο Σολομών ωκοδόμει την Μιλλώ και έκλεισε το χάλασμα της πόλεως Δαβίδ του πατρός αυτού·
\par 28 και ήτο ο άνθρωπος Ιεροβοάμ δυνατός εν ισχύϊ· και είδεν ο Σολομών τον νέον ότι ήτο φίλεργος και κατέστησεν αυτόν επιστάτην επί πάντα τα φορτία του οίκου Ιωσήφ.
\par 29 Και κατ' εκείνον τον καιρόν, ότε ο Ιεροβοάμ εξήλθεν εξ Ιερουσαλήμ, εύρηκεν αυτόν καθ' οδόν ο προφήτης Αχιά ο Σηλωνίτης, ενδεδυμένος ιμάτιον νέον· και ήσαν οι δύο μόνοι εν τη πεδιάδι.
\par 30 Και επίασεν ο Αχιά το νέον ιμάτιον, το οποίον εφόρει, και έσχισεν αυτό εις δώδεκα τμήματα·
\par 31 και είπε προς τον Ιεροβοάμ, Λάβε εις σεαυτόν δέκα τμήματα· διότι ούτω λέγει Κύριος ο Θεός του Ισραήλ· Ιδού, θέλω διαρρήξει την βασιλείαν εκ της χειρός του Σολομώντος και δώσει τας δέκα φυλάς εις σέ·
\par 32 θέλει μένει όμως εις αυτόν μία φυλή, χάριν του δούλου μου Δαβίδ και χάριν της Ιερουσαλήμ, της πόλεως, την οποίαν εξέλεξα εκ πασών των φυλών του Ισραήλ·
\par 33 διότι με εγκατέλιπον και ελάτρευσαν Αστάρτην την θεάν των Σιδωνίων, Χεμώς τον θεόν των Μωαβιτών και Μελχώμ τον θεόν των υιών Αμμών· και δεν περιεπάτησαν εις τας οδούς μου διά να πράττωσι το ευθές ενώπιόν μου, και να φυλάττωσι τα διατάγματά μου και τας κρίσεις μου, ως Δαβίδ ο πατήρ αυτού·
\par 34 δεν θέλω όμως λάβει πάσαν την βασιλείαν εκ της χειρός αυτού, αλλά θέλω διατηρήσει αυτόν ηγεμόνα πάσας τας ημέρας της ζωής αυτού, χάριν Δαβίδ του δούλου μου, τον οποίον εξέλεξα, διότι εφύλαττε τας εντολάς μου και τα διατάγματά μου·
\par 35 θέλω όμως λάβει την βασιλείαν εκ της χειρός του υιού αυτού και δώσει αυτήν εις σε, τας δέκα φυλάς·
\par 36 εις δε τον υιόν αυτού θέλω δώσει μίαν φυλήν, διά να έχη Δαβίδ ο δούλός μου λύχνον πάντοτε έμπροσθέν μου εν Ιερουσαλήμ, τη πόλει την οποίαν εξέλεξα εις εμαυτόν διά να θέσω το όνομά μου εκεί.
\par 37 και σε θέλω λάβει, και θέλεις βασιλεύσει κατά πάντα όσα η ψυχή σου επιθυμεί και θέλεις είσθαι βασιλεύς επί τον Ισραήλ·
\par 38 και εάν εισακούσης εις πάντα όσα σε προστάζω και περιπατής εις τας οδούς μου και πράττης το ευθές ενώπιόν μου, φυλάττων τα διατάγματά μου και τας εντολάς μου, καθώς έκαμνε Δαβίδ ο δούλός μου, τότε θέλω είσθαι μετά σου και θέλω οικοδομήσει εις σε οίκον ασφαλή, καθώς ωκοδόμησα εις τον Δαβίδ, και θέλω δώσει τον Ισραήλ εις σέ·
\par 39 και θέλω κακουχήσει το σπέρμα του Δαβίδ διά τούτο, πλην ουχί διά παντός.
\par 40 Όθεν εζήτησεν ο Σολομών να θανατώση τον Ιεροβοάμ. Και σηκωθείς ο Ιεροβοάμ, έφυγεν εις Αίγυπτον προς Σισάκ τον βασιλέα της Αιγύπτου, και ήτο εν Αιγύπτω εωσού απέθανεν ο Σολομών.
\par 41 Αι δε λοιπαί των πράξεων του Σολομώντος και πάντα όσα έκαμε, και η σοφία αυτού, δεν είναι γεγραμμένα εν τω βιβλίω των πράξεων του Σολομώντος;
\par 42 Αι δε ημέραι, όσας εβασίλευσεν ο Σολομών εν Ιερουσαλήμ επί πάντα τον Ισραήλ, ήσαν τεσσαράκοντα έτη.
\par 43 Και εκοιμήθη ο Σολομών μετά των πατέρων αυτού και ετάφη εν τη πόλει Δαβίδ του πατρός αυτού· και εβασίλευσεν αντ' αυτού Ροβοάμ ο υιός αυτού.

\chapter{12}

\par 1 Και υπήγεν ο Ροβοάμ εις Συχέμ· διότι εις Συχέμ ήρχετο πας ο Ισραήλ διά να κάμη αυτόν βασιλέα.
\par 2 Και ως ήκουσε τούτο Ιεροβοάμ ο υιός του Ναβάτ, όστις ήτο έτι εν Αιγύπτω, όπου είχε φύγει από προσώπου του βασιλέως Σολομώντος, έμεινεν έτι ο Ιεροβοάμ εν Αιγύπτω·
\par 3 απέστειλαν όμως και εκάλεσαν αυτόν. Τότε ήλθον ο Ιεροβοάμ και πάσα η συναγωγή του Ισραήλ και ελάλησαν προς τον Ροβοάμ, λέγοντες,
\par 4 Ο πατήρ σου εσκλήρυνε τον ζυγόν ημών· τώρα λοιπόν την δουλείαν την σκληράν του πατρός σου και τον ζυγόν αυτού τον βαρύν, τον οποίον επέβαλεν εφ' ημάς, ελάφρωσον συ, και θέλομεν σε δουλεύει.
\par 5 Ο δε είπε προς αυτούς, Αναχωρήσατε έως τρεις ημέρας· έπειτα επιστρέψατε προς εμέ. Και ανεχώρησεν ο λαός.
\par 6 Και συνεβουλεύθη ο βασιλεύς Ροβοάμ τους πρεσβυτέρους, οίτινες παρίσταντο ενώπιον Σολομώντος του πατρός αυτού έτι ζώντος, λέγων, Τι με συμβουλεύετε σεις να αποκριθώ προς τον λαόν τούτον;
\par 7 Και ελάλησαν προς αυτόν, λέγοντες, Εάν σήμερον γείνης δούλος εις τον λαόν τούτον και δουλεύσης αυτούς και αποκριθής προς αυτούς και λαλήσης αγαθούς λόγους προς αυτούς, τότε θέλουσιν είσθαι δούλοί σου διά παντός.
\par 8 Απέρριψεν όμως την συμβουλήν των πρεσβυτέρων, την οποίαν έδωκαν εις αυτόν, και συνεβουλεύθη τους νέους, τους συνανατραφέντας μετ' αυτού τους παρισταμένους ενώπιον αυτού.
\par 9 Και είπε προς αυτούς, Τι με συμβουλεύετε σεις να αποκριθώμεν προς τον λαόν τούτον, όστις ελάλησε προς εμέ, λέγων, Ελάφρωσον τον ζυγόν, τον οποίον ο πατήρ σου επέβαλεν εφ' ημάς;
\par 10 Και ελάλησαν προς αυτόν οι νέοι, οι συνανατραφέντες μετ' αυτού, λέγοντες, ούτω θέλεις λαλήσει προς τον λαόν τούτον, όστις ελάλησε προς σε, λέγων, Ο πατήρ σου εβάρυνε τον ζυγόν ημών, αλλά συ ελάφρωσον αυτόν εις ημάς· ούτω θέλεις λαλήσει προς αυτούς· Ο μικρός μου δάκτυλος θέλει είσθαι παχύτερος της οσφύος του πατρός μου·
\par 11 τώρα λοιπόν, ο μεν πατήρ μου επεφόρτισεν εις εσάς ζυγόν βαρύν, εγώ δε θέλω κάμει βαρύτερον τον ζυγόν σας· ο πατήρ μου σας επαίδευσε με μάστιγας, αλλ' εγώ θέλω σας παιδεύσει με σκορπίους.
\par 12 Και ήλθεν ο Ιεροβοάμ και πας ο λαός προς τον Ροβοάμ την τρίτην ημέραν, ως είχε λαλήσει ο βασιλεύς, λέγων, Επανέλθετε προς εμέ την τρίτην ημέραν.
\par 13 Και απεκρίθη ο βασιλεύς προς τον λαόν σκληρώς και εγκατέλιπε την συμβουλήν των πρεσβυτέρων, την οποίαν έδωκαν εις αυτόν·
\par 14 και ελάλησε προς αυτούς κατά την συμβουλήν των νέων, λέγων, Ο πατήρ μου εβάρυνε τον ζυγόν σας, αλλ' εγώ θέλω κάμει βαρύτερον τον ζυγόν σας· ο πατήρ μου σας επαίδευσε με μάστιγας, αλλ' εγώ θέλω σας παιδεύσει με σκορπίους.
\par 15 Και δεν εισήκουσεν ο βασιλεύς εις τον λαόν· διότι το πράγμα έγεινε παρά Κυρίου, διά να εκτελέση τον λόγον αυτού, τον οποίον ο Κύριος ελάλησε διά του Αχιά του Σηλωνίτου προς Ιεροβοάμ τον υιόν του Ναβάτ.
\par 16 Και ιδών πας ο Ισραήλ ότι ο βασιλεύς δεν εισήκουσεν εις αυτούς, απεκρίθη ο λαός προς τον βασιλέα, λέγων, Τι μέρος έχομεν ημείς εν τω Δαβίδ; ουδεμίαν κληρονομίαν έχομεν εν τω υιώ του Ιεσσαί· εις τας σκηνάς σου, Ισραήλ· πρόβλεψον τώρα, Δαβίδ, περί του οίκου σου. Και ανεχώρησεν ο Ισραήλ εις τας σκηνάς αυτού.
\par 17 Περί δε των υιών Ισραήλ των κατοικούντων εν ταις πόλεσιν Ιούδα, ο Ροβοάμ εβασίλευσεν επ' αυτούς.
\par 18 Και απέστειλεν ο βασιλεύς Ροβοάμ τον Αδωράμ, τον επί των φόρων· και ελιθοβόλησεν αυτόν πας ο Ισραήλ με λίθους, και απέθανεν. Όθεν έσπευσεν ο βασιλεύς Ροβοάμ να αναβή εις την άμαξαν, διά να φύγη εις Ιερουσαλήμ.
\par 19 Ούτως απεστάτησεν ο Ισραήλ από του οίκου του Δαβίδ έως της ημέρας ταύτης.
\par 20 Ότε δε ήκουσε πας ο Ισραήλ ότι ο Ιεροβοάμ επέστρεψεν, απέστειλαν και εκάλεσαν αυτόν εις την συναγωγήν και έκαμον αυτόν βασιλέα επί πάντα τον Ισραήλ· δεν ηκολούθησε τον οίκον του Δαβίδ, ειμή η φυλή του Ιούδα μόνη.
\par 21 Και ελθών ο Ροβοάμ εις Ιερουσαλήμ, συνήθροισε πάντα τον οίκον Ιούδα και την φυλήν Βενιαμίν, εκατόν ογδοήκοντα χιλιάδας εκλεκτών πολεμιστών, διά να πολεμήσωσι κατά του οίκου του Ισραήλ, όπως επαναφέρωσι την βασιλείαν εις τον Ροβοάμ τον υιόν του Σολομώντος.
\par 22 Έγεινεν όμως λόγος Θεού προς τον Σεμαΐαν, άνθρωπον του Θεού, λέγων,
\par 23 Λάλησον προς Ροβοάμ, τον υιόν του Σολομώντος, τον βασιλέα του Ιούδα, και προς πάντα τον οίκον Ιούδα και Βενιαμίν και προς το επίλοιπον του λαού, λέγων,
\par 24 ούτω λέγει Κύριος· Δεν θέλετε αναβή ουδέ πολεμήσει εναντίον των αδελφών σας των υιών Ισραήλ· επιστρέψατε έκαστος εις τον οίκον αυτού· διότι παρ' εμού έγεινε το πράγμα τούτο. Και υπήκουσαν εις τον λόγον του Κυρίου και επέστρεψαν να υπάγωσι, κατά τον λόγον του Κυρίου.
\par 25 Τότε ωκοδόμησεν ο Ιεροβοάμ την Συχέμ επί του όρους Εφραΐμ, και κατώκησεν εν αυτή· έπειτα εξήλθεν εκείθεν και ωκοδόμησε την Φανουήλ.
\par 26 Και είπεν ο Ιεροβοάμ εν τη καρδία αυτού. Τώρα θέλει επιστρέψει η βασιλεία εις τον οίκον του Δαβίδ·
\par 27 εάν ο λαός ούτος αναβή διά να προσφέρη θυσίας εν τω οίκω του Κυρίου εν Ιερουσαλήμ, τότε η καρδία του λαού τούτου θέλει επιστρέψει προς τον κύριον αυτού, τον Ροβοάμ βασιλέα του Ιούδα, και θέλουσι θανατώσει εμέ και επιστρέψει προς Ροβοάμ τον βασιλέα του Ιούδα.
\par 28 Έλαβε λοιπόν ο βασιλεύς βουλήν και έκαμε δύο μόσχους χρυσούς, και είπε προς αυτούς, Φθάνει εις εσάς να αναβαίνητε εις Ιερουσαλήμ· ιδού, οι θεοί σου, Ισραήλ, οίτινες σε ανήγαγον εκ γης Αιγύπτου.
\par 29 Και έθεσε τον ένα εν Βαιβήλ και τον άλλον έθεσεν εν Δαν.
\par 30 Και έγεινε το πράγμα τούτο αιτία αμαρτίας· διότι επορεύετο ο λαός έως εις Δαν, διά να προσκυνή ενώπιον του ενός.
\par 31 Και έκαμεν οίκους επί των υψηλών τόπων και έκαμεν ιερείς εκ των εσχάτων του λαού, οίτινες δεν ήσαν εκ των υιών Λευΐ.
\par 32 Και έκαμεν ο Ιεροβοάμ εορτήν εν τω μηνί τω ογδόω, εν τη δεκάτη πέμπτη ημέρα του μηνός, ως την εορτήν την εν Ιούδα, και ανέβη επί το θυσιαστήριον. Ούτως έκαμεν εν Βαιθήλ, θυσιάζων εις τους μόσχους τους οποίους έκαμε· και κατέστησεν εν Βαιθήλ τους ιερείς των υψηλών τόπων, τους οποίους έκαμε.
\par 33 Και ανέβη επί το θυσιαστήριον το οποίον έκαμεν εν Βαιθήλ, την δεκάτην πέμπτην ημέραν του ογδόου μηνός, εν τω μηνί τον οποίον εφεύρεν από της καρδίας αυτού· και έκαμεν εορτήν εις τους υιούς Ισραήλ, και ανέβη επί το θυσιαστήριον, διά να θυμιάση.

\chapter{13}

\par 1 Και ιδού, ήλθεν άνθρωπος του Θεού εξ Ιούδα εις Βαιθήλ με λόγον του Κυρίου· ο δε Ιεροβοάμ ίστατο επί του θυσιαστηρίου, διά να θυμιάση.
\par 2 Και εφώνησε προς το θυσιαστήριον με λόγον του Κυρίου, και είπε, Θυσιαστήριον, θυσιαστήριον, ούτω λέγει Κύριος· Ιδού, υιός θέλει γεννηθή εις τον οίκον του Δαβίδ, Ιωσίας το όνομα αυτού, και θέλει θυσιάσει επί σε τους ιερείς των υψηλών τόπων, τους θυμιάζοντας επί σε, και οστά ανθρώπων θέλουσι καυθή επί σε.
\par 3 Και έδωκε σημείον την αυτήν ημέραν, λέγων, Τούτο είναι το σημείον, το οποίον ελάλησεν ο Κύριος· Ιδού, το θυσιαστήριον θέλει διασχισθή, και η στάκτη η επ' αυτού θέλει εκχυθή.
\par 4 Και ότε ήκουσεν ο βασιλεύς Ιεροβοάμ τον λόγον του ανθρώπου του Θεού, τον οποίον εφώνησε προς το θυσιαστήριον εν Βαιθήλ, εξέτεινε την χείρα αυτού από του θυσιαστηρίου, λέγων, Συλλάβετε αυτόν. Και εξηράνθη η χειρ αυτού, την οποίαν εξέτεινεν επ' αυτόν, ώστε δεν ηδυνήθη να επιστρέψη αυτήν προς εαυτόν.
\par 5 Και διεσχίσθη το θυσιαστήριον και εξεχύθη η στάκτη από του θυσιαστηρίου, κατά το σημείον το οποίον έδωκεν ο άνθρωπος του Θεού διά του λόγου του Κυρίου.
\par 6 Και απεκρίθη ο βασιλεύς και είπε προς τον άνθρωπον του Θεού, Δεήθητι, παρακαλώ, Κυρίου του Θεού σου και προσευχήθητι υπέρ εμού, διά να επιστρέψη η χειρ μου προς εμέ. Και εδεήθη ο άνθρωπος του Θεού προς τον Κύριον, και επέστρεψεν η χειρ του βασιλέως προς αυτόν και αποκατεστάθη ως το πρότερον.
\par 7 Και είπεν ο βασιλεύς προς τον άνθρωπον του Θεού, Είσελθε μετ' εμού εις τον οίκον και λάβε τροφήν, και θέλω σοι δώσει δώρα.
\par 8 Αλλ' ο άνθρωπος του Θεού είπε προς τον βασιλέα, Το ήμισυ του οίκου σου αν μοι δώσης, δεν θέλω εισέλθει μετά σού· ουδέ θέλω φάγει άρτον ουδέ θέλω πίει ύδωρ εν τω τόπω τούτω·
\par 9 διότι ούτως είναι προστεταγμένον εις εμέ διά του λόγου του Κυρίου, λέγοντος, Μη φάγης άρτον και μη πίης ύδωρ και μη επιστρέψης διά της οδού, διά της οποίας ήλθες.
\par 10 Και ανεχώρησε δι' άλλης οδού και δεν επέστρεψε διά της οδού, διά της οποίας ήλθεν εις Βαιθήλ.
\par 11 Κατώκει δε εν Βαιθήλ γέρων τις προφήτης· και ήλθον οι υιοί αυτού και διηγήθησαν προς αυτόν πάντα τα έργα, τα οποία έκαμεν ο άνθρωπος του Θεού την ημέραν εκείνην εν Βαιθήλ· διηγήθησαν δε προς τον πατέρα αυτών και τους λόγους, τους οποίους ελάλησε προς τον βασιλέα.
\par 12 Και είπε προς αυτούς ο πατήρ αυτών, Διά τίνος οδού ανεχώρησεν; είχον δε ιδεί οι υιοί αυτού διά τίνος οδού ανεχώρησεν ο άνθρωπος του Θεού ο ελθών εξ Ιούδα.
\par 13 Και είπε προς τους υιούς αυτού, Ετοιμάσατέ μοι την όνον. Και ητοίμασαν εις αυτόν την όνον· και εκάθησεν επ' αυτήν,
\par 14 και υπήγε κατόπιν του ανθρώπου του Θεού και εύρηκεν αυτόν καθήμενον υπό δρύν· και είπε προς αυτόν, συ είσαι ο άνθρωπος του Θεού ο ελθών εξ Ιούδα; Ο δε είπεν, Εγώ.
\par 15 Και είπε προς αυτόν, Ελθέ μετ' εμού εις την οικίαν και φάγε άρτον.
\par 16 Ο δε είπε, Δεν δύναμαι να επιστρέψω μετά σου ουδέ να έλθω μετά σού· ουδέ θέλω φάγει άρτον ουδέ θέλω πίει ύδωρ μετά σου εν τω τόπω τούτω·
\par 17 διότι ελαλήθη προς εμέ διά του λόγου του Κυρίου, Μη φάγης άρτον μηδέ πίης ύδωρ εκεί, μηδέ επιστρέψης υπάγων διά της οδού διά της οποίας ήλθες.
\par 18 Είπε δε προς αυτόν, Και εγώ προφήτης είμαι, καθώς σύ· και άγγελος ελάλησε προς εμέ διά του λόγου του Κυρίου, λέγων, Επίστρεψον αυτόν μετά σου εις την οικίαν σου, διά να φάγη άρτον και να πίη ύδωρ. Εψεύσθη δε προς αυτόν.
\par 19 Και επέστρεψε μετ' αυτού και έφαγεν άρτον εν τω οίκω αυτού και έπιεν ύδωρ.
\par 20 Και ενώ εκάθηντο εις την τράπεζαν, ήλθεν ο λόγος του Κυρίου προς τον προφήτην τον επιστρέψαντα αυτόν·
\par 21 και εφώνησε προς τον άνθρωπον του Θεού τον ελθόντα εξ Ιούδα, λέγων, Ούτω λέγει Κύριος. Επειδή παρήκουσας της φωνής του Κυρίου και δεν εφύλαξας την εντολήν, την οποίαν Κύριος ο Θεός σου προσέταξεν εις σε,
\par 22 αλλ' επέστρεψας και έφαγες άρτον και έπιες ύδωρ εν τω τόπω, περί του οποίου είπε προς σε, Μη φάγης άρτον μηδέ πίης ύδωρ· το σώμα σου δεν θέλει εισέλθει εις τον τάφον των πατέρων σου.
\par 23 Και αφού έφαγεν άρτον και αφού έπιεν, ητοίμασεν εκείνος την όνον εις αυτόν, εις τον προφήτην τον οποίον επέστρεψε.
\par 24 Και ανεχώρησεν· εύρε δε αυτόν λέων καθ' οδόν και εθανάτωσεν αυτόν· και το σώμα αυτού ήτο ερριμμένον εν τη οδώ· η δε όνος ίστατο πλησίον αυτού και ο λέων ίστατο πλησίον του σώματος.
\par 25 Και ιδού, άνδρες διαβαίνοντες είδον το σώμα ερριμμένον εν τη οδώ και τον λέοντα ιστάμενον πλησίον του σώματος· και ελθόντες απήγγειλαν τούτο εν τη πόλει, όπου κατώκει ο προφήτης ο γέρων.
\par 26 Και ότε ήκουσεν ο προφήτης ο επιστρέψας αυτόν εκ της οδού, είπεν, Ούτος είναι ο άνθρωπος του Θεού, όστις παρήκουσε της φωνής του Κυρίου· διά τούτο παρέδωκεν αυτόν ο Κύριος εις τον λέοντα, και διεσπάραξεν αυτόν και εθανάτωσεν αυτόν, κατά τον λόγον του Κυρίου, τον οποίον ελάλησε προς αυτόν.
\par 27 Και ελάλησε προς τους υιούς αυτού, λέγων, Στρώσατε εις εμέ την όνον. Και έστρωσαν.
\par 28 Και υπήγε και εύρηκε το σώμα αυτού ερριμμένον εν τη οδώ, και την όνον και τον λέοντα ισταμένους πλησίον του σώματος· ο λέων δεν έφαγε το σώμα ουδέ διεσπάραξε την όνον.
\par 29 Και εσήκωσεν ο προφήτης το σώμα του ανθρώπου του Θεού, και επέθεσεν αυτό επί την όνον, και ανέφερεν αυτόν· και ήλθεν εις την πόλιν ο προφήτης ο γέρων, διά να πενθήση και να θάψη αυτόν.
\par 30 Και έθεσε το σώμα αυτού εν τω τάφω αυτού· και επένθησαν επ' αυτόν, λέγοντες, Φευ αδελφέ μου
\par 31 Και αφού έθαψεν αυτόν, ελάλησε προς τους υιούς αυτού, λέγων, Αφού αποθάνω, θάψατε και εμέ εν τω τάφω, όπου ετάφη ο άνθρωπος του Θεού· θέσατε τα οστά μου πλησίον των οστέων αυτού·
\par 32 διότι θέλει εξάπαντος εκτελεσθή το πράγμα, το οποίον εφώνησε διά του λόγου του Κυρίου κατά του θυσιαστηρίου εν Βαιθήλ και κατά πάντων των οίκων των υψηλών τόπων, οίτινες είναι εις τας πόλεις της Σαμαρείας.
\par 33 Μετά το πράγμα τούτο δεν επέστρεψεν ο Ιεροβοάμ εκ της οδού αυτού της κακής, αλλ' έκαμε πάλιν εκ των εσχάτων του λαού ιερείς των υψηλών τόπων· όστις ήθελε, καθιέρονεν αυτόν, και εγίνετο ιερεύς των υψηλών τόπων.
\par 34 Και έγεινε το πράγμα τούτο αιτία αμαρτίας εις τον οίκον του Ιεροβοάμ, ώστε να εξολοθρεύση και να αφανίση αυτόν από προσώπου της γης.

\chapter{14}

\par 1 Κατ' εκείνον τον καιρόν ηρρώστησεν Αβιά ο υιός του Ιεροβοάμ.
\par 2 Και είπεν ο Ιεροβοάμ προς την γυναίκα αυτού, Σηκώθητι, παρακαλώ, και μετασχηματίσθητι, ώστε να μη γνωρίσωσιν ότι είσαι γυνή του Ιεροβοάμ, και ύπαγε εις Σηλώ· ιδού, εκεί είναι Αχιά ο προφήτης, όστις είπε προς εμέ ότι θέλω βασιλεύσει επί τον λαόν τούτον·
\par 3 και λάβε εις την χείρα σου δέκα άρτους και κολλύρια και σταμνίον μέλιτος και ύπαγε προς αυτόν· αυτός θέλει σοι αναγγείλει τι θέλει γείνει εις το παιδίον.
\par 4 Και έκαμεν ούτως η γυνή του Ιεροβοάμ· και σηκωθείσα, υπήγεν εις Σηλώ και ήλθεν εις τον οίκον του Αχιά. Ο δε Αχιά δεν ηδύνατο να βλέπη· διότι οι οφθαλμοί αυτού ημβλυώπουν εκ του γήρατος αυτού.
\par 5 Είχε δε ειπεί ο Κύριος προς τον Αχιά, Ιδού, η γυνή του Ιεροβοάμ έρχεται να ζητήση παρά σου λόγον περί του υιού αυτής, διότι είναι άρρωστος· ούτω και ούτω θέλεις λαλήσει προς αυτήν· διότι, όταν εισέλθη, θέλει προσποιηθή ότι είναι άλλη.
\par 6 Και ως ήκουσεν ο Αχιά τον ήχον των ποδών αυτής, ενώ εισήρχετο εις την θύραν, είπεν, Είσελθε, γυνή του Ιεροβοάμ· διά τι προσποιείσαι ότι είσαι άλλη; αλλ' εγώ είμαι απόστολος προς σε σκληρών αγγελιών·
\par 7 ύπαγε, ειπέ προς τον Ιεροβοάμ, ούτω λέγει Κύριος ο Θεός του Ισραήλ· Επειδή εγώ σε ύψωσα εκ μέσου του λαού και σε κατέστησα ηγεμόνα επί τον λαόν μου Ισραήλ,
\par 8 και διαρρήξας την βασιλείαν από του οίκου του Δαβίδ, έδωκα αυτήν εις σε, και συ δεν εστάθης καθώς ο δούλός μου Δαβίδ, όστις εφύλαξε τας εντολάς μου και όστις με ηκολούθησεν εξ όλης αυτού της καρδίας, εις το να κάμνη μόνον το ευθές ενώπιόν μου,
\par 9 αλλ' υπερέβης εις το κακόν πάντας όσοι εστάθησαν πρότεροί σου, διότι υπήγες και έκαμες εις σεαυτόν άλλους θεούς και χωνευτά είδωλα, διά να με παροργίσης, και με απέρριψας οπίσω της ράχης σου.
\par 10 διά τούτο, ιδού, θέλω φέρει κακόν επί τον οίκον του Ιεροβοάμ, και θέλω εξολοθρεύσει του Ιεροβοάμ τον ουρούντα εις τον τοίχον, τον πεφυλαγμένον και τον αφειμένον εν τω Ισραήλ, και θέλω σαρώσει κατόπιν του οίκου του Ιεροβοάμ, καθώς σαρόνει τις την κόπρον εωσού εκλείψη·
\par 11 όστις εκ του Ιεροβοάμ αποθάνη εν τη πόλει, οι κύνες θέλουσι καταφάγει αυτόν· και όστις αποθάνη εν τω αγρώ, τα πετεινά του ουρανού θέλουσι καταφάγει αυτόν· διότι ο Κύριος ελάλησε.
\par 12 Συ λοιπόν σηκωθείσα ύπαγε εις την οικίαν σου· ενώ οι πόδες σου εμβαίνουσιν εις την πόλιν, το παιδίον θέλει αποθάνει
\par 13 και θέλει πενθήσει αυτό πας ο Ισραήλ, και θέλουσιν ενταφιάσει αυτό· διότι αυτό μόνον εκ του Ιεροβοάμ θέλει ελθεί εις τον τάφον, επειδή εν αυτώ ευρέθη τι καλόν ενώπιον Κυρίου, του Θεού του Ισραήλ, εν τω οίκω του Ιεροβοάμ.
\par 14 Και θέλει αναστήσει ο Κύριος εις εαυτόν βασιλέα επί τον Ισραήλ, όστις θέλει εξολοθρεύσει τον οίκον του Ιεροβοάμ την ημέραν εκείνην· αλλά τι; τώρα μάλιστα.
\par 15 Και θέλει πατάξει ο Κύριος τον Ισραήλ, ώστε να κινήται ως κάλαμος εν τω ύδατι, και θέλει εκριζώσει τον Ισραήλ εκ της γης ταύτης της αγαθής, την οποίαν έδωκεν εις τους πατέρας αυτών, και διασκορπίσει αυτούς πέραν του ποταμού· επειδή έκαμον τα άλση αυτών, διά να παροργίσωσι τον Κύριον·
\par 16 και θέλει παραδώσει τον Ισραήλ εξ αιτίας των αμαρτιών του Ιεροβοάμ, όστις ημάρτησε και όστις έκαμε τον Ισραήλ να αμαρτήση.
\par 17 Και εσηκώθη η γυνή του Ιεροβοάμ και ανεχώρησε και ήλθεν εις Θερσά· καθώς αυτή επάτησε το κατώφλιον της θύρας του οίκου, απέθανε το παιδίον·
\par 18 και έθαψαν αυτό· και επένθησεν αυτό πας ο Ισραήλ, κατά τον λόγον του Κυρίου, τον οποίον ελάλησε διά του δούλου αυτού Αχιά του προφήτου.
\par 19 Αι δε λοιπαί των πράξεων του Ιεροβοάμ, πως επολέμησε και τίνι τρόπω εβασίλευσεν, ιδού, είναι γεγραμμένα εν τω βιβλίω των χρονικών των βασιλέων του Ισραήλ.
\par 20 Και αι ημέραι, τας οποίας εβασίλευσεν ο Ιεροβοάμ, ήσαν εικοσιδύο έτη· και εκοιμήθη μετά των πατέρων αυτού, και εβασίλευσεν αντ' αυτού Ναδάβ ο υιός αυτού.
\par 21 Ο δε Ροβοάμ ο υιός του Σολομώντος εβασίλευσεν επί τον Ιούδαν. Τεσσαρακόντα και ενός έτους ήτο ο Ροβοάμ ότε έγεινε βασιλεύς, και εβασίλευσε δεκαεπτά έτη εν Ιερουσαλήμ, τη πόλει την οποίαν ο Κύριος εξέλεξεν εκ πασών των φυλών του Ισραήλ διά να θέση το όνομα αυτού εκεί. Και το όνομα της μητρός αυτού ήτο Νααμά η Αμμωνίτις.
\par 22 Έπραξε δε ο Ιούδας πονηρά ενώπιον του Κυρίου και παρώξυναν αυτόν εις ζηλοτυπίαν με τας αμαρτίας αυτών, τας οποίας ημάρτησαν υπέρ πάντα όσα έπραξαν οι πατέρες αυτών.
\par 23 Διότι και αυτοί ωκοδόμησαν εις εαυτούς τόπους υψηλούς, και έκαμον αγάλματα και άλση επί παντός υψηλού λόφου και υποκάτω παντός δένδρου πρασίνου.
\par 24 Ήσαν δε έτι εν τη γη και σοδομίται και έπραττον κατά πάντα τα βδελύγματα των εθνών, τα οποία ο Κύριος εξεδίωξεν απ' έμπροσθεν των υιών Ισραήλ.
\par 25 Και εν τω πέμπτω έτει της βασιλείας του Ροβοάμ, ανέβη Σισάκ ο βασιλεύς της Αιγύπτου εναντίον της Ιερουσαλήμ.
\par 26 Και έλαβε τους θησαυρούς του οίκου του Κυρίου και τους θησαυρούς του οίκου του βασιλέως· τα πάντα έλαβεν· έλαβεν έτι πάσας τας χρυσάς ασπίδας, τας οποίας έκαμεν ο Σολομών.
\par 27 Και αντί τούτων ο βασιλεύς Ροβοάμ έκαμε χαλκίνας ασπίδας και παρέδωκεν αυτάς εις τας χείρας των αρχόντων των δορυφόρων, οίτινες εφύλαττον την θύραν του οίκου του βασιλέως.
\par 28 Και ότε εισήρχετο ο βασιλεύς εις τον οίκον του Κυρίου, εβάσταζον αυτάς οι δορυφόροι έπειτα επανέφερον αυτάς εις το οίκημα των δορυφόρων.
\par 29 Αι δε λοιπαί των πράξεων του Ροβοάμ και πάντα όσα έκαμε, δεν είναι γεγραμμένα εν τω βιβλίω των χρονικών των βασιλέων του Ιούδα;
\par 30 Ήτο δε πόλεμος αναμέσον Ροβοάμ και Ιεροβοάμ πάσας τας ημέρας.
\par 31 Και εκοιμήθη ο Ροβοάμ μετά των πατέρων αυτού και ετάφη μετά των πατέρων αυτού εν τη πόλει Δαβίδ. Και το όνομα της μητρός αυτού ήτο Νααμά η Αμμωνίτις. Εβασίλευσε δε αντ' αυτού Αβιάμ ο υιός αυτού.

\chapter{15}

\par 1 Και εβασίλευσεν ο Αβιάμ επί τον Ιούδαν, κατά το δέκατον όγδοον έτος της βασιλείας του Ιεροβοάμ υιού του Ναβάτ.
\par 2 Τρία έτη εβασίλευσεν εν Ιερουσαλήμ. Και το όνομα της μητρός αυτού ήτο Μααχά, θυγάτηρ του Αβεσσαλώμ.
\par 3 Και περιεπάτησεν εις πάσας τας αμαρτίας του πατρός αυτού, τας οποίας έπραξε προ αυτού· και δεν ήτο η καρδία αυτού τελεία μετά Κυρίου του Θεού αυτού, καθώς η καρδία Δαβίδ του πατρός αυτού.
\par 4 Αλλ' όμως, χάριν του Δαβίδ, έδωκεν εις αυτόν Κύριος ο Θεός αυτού λύχνον εν Ιερουσαλήμ, αναστήσας τον υιόν αυτού μετ' αυτόν, και στερεώσας την Ιερουσαλήμ·
\par 5 διότι ο Δαβίδ έκαμνε το ευθές ενώπιον Κυρίου και δεν εξέκλινε πάσας τας ημέρας της ζωής αυτού από πάντων όσα προσέταξεν εις αυτόν, εκτός της υποθέσεως Ουρίου του Χετταίου.
\par 6 Ήτο δε πόλεμος αναμέσον Ροβοάμ και Ιεροβοάμ πάσας τας ημέρας της ζωής αυτού.
\par 7 Αι δε λοιπαί των πράξεων του Αβιάμ και πάντα όσα έπραξε, δεν είναι γεγραμμένα εν τω βιβλίω των χρονικών των βασιλέων του Ιούδα; Και ήτο πόλεμος αναμέσον Αβιάμ και Ιεροβοάμ.
\par 8 Και εκοιμήθη ο Αβιάμ μετά των πατέρων αυτού, και έθαψαν αυτόν εν τη πόλει Δαβίδ· εβασίλευσε δε αντ' αυτού Ασά ο υιός αυτού.
\par 9 Και εβασίλευσεν ο Ασά επί τον Ιούδαν, κατά το εικοστόν έτος του Ιεροβοάμ βασιλέως του Ισραήλ.
\par 10 Και εβασίλευσεν εν Ιερουσαλήμ έτη τεσσαράκοντα και εν. Το δε όνομα της μητρός αυτού ήτο Μααχά, θυγάτηρ του Αβεσσαλώμ.
\par 11 Και έκαμνεν ο Ασά το ευθές ενώπιον Κυρίου, καθώς Δαβίδ ο πατήρ αυτού.
\par 12 Και αφήρεσεν εκ της γης τους σοδομίτας και εσήκωσε πάντα τα είδωλα, τα οποία έκαμον οι πατέρες αυτού.
\par 13 Έτι δε και την μητέρα αυτού την Μααχά, και αυτήν απέβαλε του να ήναι βασίλισσα, επειδή έκαμεν είδωλον εις άλσος· και κατέκοψεν ο Ασά το είδωλον αυτής και έκαυσεν αυτό πλησίον του χειμάρρου Κέδρων.
\par 14 Οι υψηλοί όμως τόποι δεν αφηρέθησαν· πλην η καρδία του Ασά ήτο τελεία μετά του Κυρίου πάσας τας ημέρας αυτού.
\par 15 Και έφερεν εις τον οίκον του Κυρίου τα αφιερώματα του πατρός αυτού και τα εαυτού αφιερώματα, άργυρον και χρυσίον και σκεύη.
\par 16 Ήτο δε πόλεμος αναμέσον Ασά και Βαασά βασιλέως του Ισραήλ πάσας τας ημέρας αυτών.
\par 17 Και ανέβη Βαασά ο βασιλεύς του Ισραήλ εναντίον του Ιούδα και ωκοδόμησε την Ραμά, διά να μη αφίνη μηδένα να εξέρχηται μηδέ να εισέρχηται προς Ασά τον βασιλέα του Ιούδα.
\par 18 Τότε έλαβεν ο Ασά άπαν το αργύριον και το χρυσίον το εναπολειφθέν εν τοις θησαυροίς του οίκου του Κυρίου και εν τοις θησαυροίς του οίκου του βασιλέως, και παρέδωκεν αυτά εις τας χείρας των δούλων αυτού· και απέστειλεν αυτούς ο βασιλεύς Ασά προς τον Βεν-αδάδ, υιόν του Ταβριμών, υιού του Εσιών, βασιλέα της Συρίας, τον κατοικούντα εν Δαμασκώ, λέγων,
\par 19 Ας γείνη συνθήκη αναμέσον εμού και σου, ως ήτο αναμέσον του πατρός μου και του πατρός σου· ιδού, απέστειλα προς σε δώρον αργυρίου και χρυσίου· ύπαγε, διάλυσον την συνθήκην σου την προς τον Βαασά, βασιλέα του Ισραήλ, διά να αναχωρήση απ' εμού.
\par 20 Και εισήκουσεν ο Βεν-αδάδ εις τον βασιλέα Ασά, και απέστειλε τους αρχηγούς των δυνάμεων αυτού εναντίον των πόλεων του Ισραήλ, και επάταξε την Ιϊών και την Δαν και την Αβέλ-βαίθ-μααχά, και πάσαν την Χιννερώθ, μετά πάσης της γης Νεφθαλί.
\par 21 Και ως ήκουσεν ο Βαασά, έπαυσε να οικοδομή την Ραμά και εκάθησεν εν Θερσά.
\par 22 Τότε συνεκάλεσεν ο βασιλεύς Ασά πάντα τον Ιούδαν, χωρίς τινός εξαιρέσεως· και εσήκωσαν τους λίθους της Ραμά και τα ξύλα αυτής, με τα οποία ο Βαασά έκαμε την οικοδομήν· και ωκοδόμησεν ο βασιλεύς Ασά με ταύτα την Γεβά του Βενιαμίν και την Μισπά.
\par 23 Αι δε λοιπαί πασών των πράξεων του Ασά και πάντα τα κατορθώματα αυτού και πάντα όσα έπραξε, και αι πόλεις τας οποίας ωκοδόμησε, δεν είναι γεγραμμένα εν τω βιβλίω των χρονικών των βασιλέων του Ιούδα; Εν τω καιρώ δε του γήρατος αυτού ηρρώστησε τους πόδας αυτού.
\par 24 Και εκοιμήθη ο Ασά μετά των πατέρων αυτού και ετάφη μετά των πατέρων αυτού εν τη πόλει Δαβίδ του πατρός αυτού· εβασίλευσε δε αντ' αυτού Ιωσαφάτ ο υιός αυτού.
\par 25 Και εβασίλευσε Ναδάβ ο υιός του Ιεροβοάμ επί τον Ισραήλ, το δεύτερον έτος του Ασά βασιλέως του Ιούδα, και εβασίλευσεν επί τον Ισραήλ δύο έτη.
\par 26 Και έπραξε πονηρά ενώπιον του Κυρίου και περιεπάτησεν εις την οδόν του πατρός αυτού και εις την αμαρτίαν αυτού, διά της οποίας έκαμε τον Ισραήλ να αμαρτήση.
\par 27 Συνώμοσε δε κατ' αυτού Βαασά ο υιός του Αχιά, εκ του οίκου Ισσάχαρ· και επάταξεν αυτόν ο Βαασά εν Γιββεθών, ήτις ήτο των Φιλισταίων· διότι ο Ναδάβ και πας ο Ισραήλ επολιόρκουν την Γιββεθών.
\par 28 Ο Βαασά λοιπόν εθανάτωσεν αυτόν κατά το τρίτον έτος του Ασά βασιλέως του Ιούδα, και εβασίλευσεν αντ' αυτού.
\par 29 Και καθώς εβασίλευσεν, επάταξεν όλον τον οίκον του Ιεροβοάμ· δεν αφήκεν εις τον Ιεροβοάμ ουδέν ζων, εωσού εξωλόθρευσεν αυτόν, κατά τον λόγον του Κυρίου, τον οποίον ελάλησε διά του δούλου αυτού Αχιά του Σηλωνίτου,
\par 30 διά τας αμαρτίας του Ιεροβοάμ, τας οποίας ημάρτησε, και διά των οποίων έκαμε τον Ισραήλ να αμαρτήση, και διά τον παροργισμόν με τον οποίον παρώργισε Κύριον τον Θεόν του Ισραήλ.
\par 31 Αι δε λοιπαί των πράξεων του Ναδάβ και πάντα όσα έπραξε, δεν είναι γεγραμμένα εν τω βιβλίω των χρονικών των βασιλέων του Ισραήλ;
\par 32 Ήτο δε πόλεμος αναμέσον Ασά και Βαασά βασιλέως του Ισραήλ πάσας τας ημέρας αυτών.
\par 33 Κατά το τρίτον έτος του Ασά βασιλέως του Ιούδα, εβασίλευσε Βαασά ο υιός του Αχιά επί πάντα τον Ισραήλ εν Θερσά· και εβασίλευσεν εικοσιτέσσαρα έτη.
\par 34 Και έπραξε πονηρά ενώπιον του Κυρίου, και περιεπάτησεν εις την οδόν του Ιεροβοάμ και εις την αμαρτίαν αυτού, διά της οποίας έκαμε τον Ισραήλ να αμαρτήση.

\chapter{16}

\par 1 Και ήλθε λόγος Κυρίου προς τον Ιηού, τον υιόν του Ανανί, εναντίον του Βαασά, λέγων,
\par 2 Επειδή, ενώ σε ύψωσα εκ του χώματος, και σε κατέστησα ηγεμόνα επί τον λαόν μου Ισραήλ, συ περιεπάτησας εις την οδόν του Ιεροβοάμ, και έκαμες τον λαόν μου Ισραήλ να αμαρτήση, διά να με παροργίσης διά των αμαρτιών αυτών,
\par 3 ιδού, εγώ εξολοθρεύω κατά κράτος τον Βαασά και τον οίκον αυτού· και θέλω καταστήσει τον οίκόν σου ως τον οίκον του Ιεροβοάμ υιού του Ναβάτ·
\par 4 όστις εκ του Βαασά αποθάνη εν τη πόλει, οι κύνες θέλουσι φάγει αυτόν· και όστις εξ αυτού αποθάνη εν τοις αγροίς, τα πετεινά του ουρανού θέλουσι φάγει αυτόν.
\par 5 Αι δε λοιπαί των πράξεων του Βαασά και όσα έπραξε και τα κατορθώματα αυτού δεν είναι γεγραμμένα εν τω βιβλίω των χρονικών των βασιλέων του Ισραήλ;
\par 6 Και εκοιμήθη ο Βαασά μετά των πατέρων αυτού και ετάφη εν Περσά· εβασίλευσε δε αντ' αυτού Ηλά ο υιός αυτού.
\par 7 Και έτι διά Ιηού του προφήτου, υιού του Ανανί, ήλθεν ο λόγος του Κυρίου κατά του Βαασά και κατά του οίκου αυτού και κατά πασών των κακιών όσας έπραξεν ενώπιον του Κυρίου, παροργίσας αυτόν διά των έργων των χειρών αυτού, ώστε να γείνη καθώς ο οίκος του Ιεροβοάμ· και διότι εθανάτωσεν αυτόν.
\par 8 Κατά το εικοστόν έκτον έτος του Ασά βασιλέως του Ιούδα, εβασίλευσεν ο Ηλά υιός του Βαασά επί τον Ισραήλ εν Φερσά και εβασίλευσε δύο έτη.
\par 9 Συνώμοσε δε κατ' αυτού ο δούλος αυτού Ζιμβρί, ο αρχηγός του ημίσεως των πολεμικών αμαξών, ενώ ήτο εν Θερσά πίνων και μεθύων εν τω οίκω του Αρσά, οικονόμου του οίκου αυτού εν Θερσά.
\par 10 Και εισήλθεν ο Ζιμβρί και επάταξεν αυτόν και εθανάτωσεν αυτόν, εις το εικοστόν έβδομον έτος του Ασά βασιλέως του Ιούδα, και εβασίλευσεν αντ' αυτού.
\par 11 Και ως εβασίλευσεν, άμα εκάθησεν επί του θρόνου αυτού, επάταξε πάντα τον οίκον του Βαασά· δεν αφήκεν εις αυτόν ουρούντα προς τοίχον ουδέ συγγενείς αυτού ουδέ φίλους αυτού.
\par 12 Και εξωλόθρευσεν ο Ζιμβρί πάντα τον οίκον του Βαασά, κατά τον λόγον του Κυρίου, τον οποίον ελάλησεν εναντίον του Βαασά διά Ιηού του προφήτου,
\par 13 διά πάσας τας αμαρτίας του Βαασά και τας αμαρτίας Ηλά του υιού αυτού, τας οποίας ημάρτησαν, και διά των οποίων έκαμον τον Ισραήλ να αμαρτήση, παροργίσαντες, Κύριον τον Θεόν του Ισραήλ διά των ματαιοτήτων αυτών.
\par 14 Αι δε λοιπαί των πράξεων του Ηλά και πάντα όσα έπραξε, δεν είναι γεγραμμένα εν τω βιβλίω των χρονικών των βασιλέων του Ισραήλ;
\par 15 Κατά το εικοστόν έβδομον έτος του Ασά βασιλέως του Ιούδα, εβασίλευσεν ο Ζιμβρί επτά ημέρας εν Θερσά. Ο δε λαός ήτο εστρατοπεδευμένος κατά της Γιββεθών, ήτις ήτο των Φιλισταίων.
\par 16 Και ακούσας ο λαός ο εστρατοπεδευμένος ότι έλεγον, Ο Ζιμβρί συνώμοσε και μάλιστα επάταξε τον βασιλέα, άπας ο Ισραήλ έκαμε τον Αμρί, τον αρχηγόν του στρατεύματος, βασιλέα επί τον Ισραήλ την ημέραν εκείνην εν τω στρατοπέδω.
\par 17 Και ανέβη ο Αμρί και άπας ο Ισραήλ μετ' αυτού από Γιββεθών, και επολιόρκησαν την Θερσά.
\par 18 Και ως είδεν ο Ζιμβρί ότι εκυριεύθη η πόλις, εισήλθεν εις το παλάτιον του οίκου του βασιλέως και έκαυσεν εφ' εαυτόν τον οίκον του βασιλέως εν πυρί και απέθανε,
\par 19 διά τας αμαρτίας αυτού, τας οποίας ημάρτησε, πράξας πονηρά ενώπιον του Κυρίου, επειδή περιεπάτησεν εις την οδόν του Ιεροβοάμ και εις τας αμαρτίας αυτού, τας οποίας έπραξε, κάμνων τον Ισραήλ να αμαρτήση.
\par 20 Αι δε λοιπαί των πράξεων του Ζιμβρί και η συνωμοσία αυτού, την οποίαν έκαμε, δεν είναι γεγραμμένα εν τω βιβλίω των χρονικών των βασιλέων του Ισραήλ;
\par 21 Τότε διηρέθη ο λαός του Ισραήλ εις δύο μέρη· το ήμισυ του λαού ηκολούθησε τον Θιβνί υιόν του Γινάθ, διά να κάμη αυτόν βασιλέα· και το ήμισυ ηκολούθησε τον Αμρί.
\par 22 Ο λαός όμως ο ακολουθήσας τον Αμρί υπερίσχυσε κατά του λαού του ακολουθήσαντος τον Θιβνί υιόν του Γινάθ· και απέθανεν ο Θιβνί, και εβασίλευσεν ο Αμρί.
\par 23 Κατά το τριακοστόν πρώτον έτος του Ασά βασιλέως του Ιούδα, εβασίλευσεν Αμρί επί τον Ισραήλ, και εβασίλευσε δώδεκα έτη· εξ έτη εβασίλευσεν εν Θερσά.
\par 24 Και ηγόρασε το όρος της Σαμαρείας παρά του Σεμέρ διά δύο τάλαντα αργυρίου, και έκτισε πόλιν επί του όρους και εκάλεσε το όνομα της πόλεως, την οποίαν έκτισε, κατά το όνομα του Σεμέρ, κυρίου του όρους, Σαμάρειαν.
\par 25 Έπραξε δε ο Αμρί πονηρά ενώπιον του Κυρίου και έπραξε χειρότερα παρά πάντας τους προ αυτού·
\par 26 και περιεπάτησεν εις πάσας τας οδούς του Ιεροβοάμ, υιού του Ναβάτ, και εις τας αμαρτίας εκείνου, διά των οποίων έκαμε τον Ισραήλ να αμαρτήση, παροργίσας Κύριον τον Θεόν του Ισραήλ διά των ματαιοτήτων αυτών.
\par 27 Αι δε λοιπαί των πράξεων του Αμρί τας οποίας έπραξε και τα κατορθώματα αυτού όσα έκαμε, δεν είναι γεγραμμένα εν τω βιβλίω των χρονικών των βασιλέων του Ισραήλ;
\par 28 Και εκοιμήθη ο Αμρί μετά των πατέρων αυτού και ετάφη εν Σαμαρεία· εβασίλευσε δε αντ' αυτού Αχαάβ ο υιός αυτού.
\par 29 Ο δε Αχαάβ ο υιός του Αμρί εβασίλευσεν επί τον Ισραήλ κατά το τριακοστόν όγδοον έτος του Ασά βασιλέως του Ιούδα· και εβασίλευσεν Αχαάβ ο υιός του Αμρί επί τον Ισραήλ εν Σαμαρεία εικοσιδύο έτη.
\par 30 Και έπραξεν ο Αχαάβ ο υιός του Αμρί πονηρά ενώπιον του Κυρίου, υπέρ πάντας τους προ αυτού.
\par 31 Και ως αν ήτο μικρόν το να περιπατή εις τας αμαρτίας του Ιεροβοάμ, υιού του Ναβάτ, έλαβεν έτι διά γυναίκα Ιεζάβελ, την θυγατέρα του Εθβαάλ, βασιλέως των Σιδωνίων, και υπήγε και ελάτρευσε τον Βάαλ και προσεκύνησεν αυτόν.
\par 32 Και ανήγειρε βωμόν εις τον Βάαλ εντός του οίκου του Βάαλ, τον οποίον ωκοδόμησεν εν Σαμαρεία.
\par 33 Και έκαμεν ο Αχαάβ άλσος· και διά να παροργίση Κύριον τον Θεόν του Ισραήλ, έπραξεν ο Αχαάβ περισσότερον παρά πάντας τους βασιλείς του Ισραήλ, όσοι εστάθησαν προ αυτού.
\par 34 Εν ταις ημέραις αυτού ωκοδόμησε Χιήλ ο Βαιθηλίτης την Ιεριχώ· έβαλε τα θεμέλια αυτής επί Αβειρών του πρωτοτόκου αυτού, και έστησε τας πύλας αυτής επί Σεγούβ του νεωτέρου υιού αυτού, κατά τον λόγον του Κυρίου, τον οποίον ελάλησε διά Ιησού υιού του Ναυή.

\chapter{17}

\par 1 Και είπεν Ηλίας ο Θεσβίτης, ο εκ των κατοίκων της Γαλαάδ, προς τον Αχαάβ, Ζη Κύριος ο Θεός του Ισραήλ, έμπροσθεν του οποίου παρίσταμαι, δεν θέλει είσθαι τα έτη ταύτα δρόσος και βροχή, ειμή διά του λόγου του στόματός μου.
\par 2 Και ήλθεν ο λόγος του Κυρίου προς αυτόν, λέγων,
\par 3 Αναχώρησον εντεύθεν και στρέψον προς ανατολάς και κρύφθητι πλησίον του χειμάρρου Χερίθ, του απέναντι του Ιορδάνου·
\par 4 και θέλεις πίνει εκ του χειμάρρου· προσέταξα δε τους κόρακας να σε τρέφωσιν εκεί.
\par 5 Και υπήγε και έκαμε κατά τον λόγον του Κυρίου· διότι υπήγε και εκάθησε πλησίον του χειμάρρου Χερίθ, του απέναντι του Ιορδάνου.
\par 6 Και οι κόρακες έφερον προς αυτόν άρτον και κρέας το πρωΐ, και άρτον και κρέας το εσπέρας· και έπινεν εκ του χειμάρρου.
\par 7 Μετά δε τινάς ημέρας εξηράνθη ο χείμαρρος, επειδή δεν έγεινε βροχή επί της γης.
\par 8 Και ήλθεν ο λόγος του Κυρίου προς αυτόν, λέγων,
\par 9 Σηκωθείς ύπαγε εις Σαρεπτά της Σιδώνος και κάθισον εκεί· ιδού, προσέταξα εκεί γυναίκα χήραν να σε τρέφη.
\par 10 Και σηκωθείς υπήγεν εις Σαρεπτά. Και ως ήλθεν εις την πύλην της πόλεως, ιδού, εκεί γυνή χήρα συνάγουσα ξυλάρια· και εφώνησε προς αυτήν και είπε, Φέρε μοι, παρακαλώ, ολίγον ύδωρ εν αγγείω, διά να πίω.
\par 11 Και ενώ υπήγε να φέρη αυτό, εφώνησε προς αυτήν και είπε, Φέρε μοι παρακαλώ, κομμάτιον άρτου εν τη χειρί σου.
\par 12 Η δε είπε, Ζη Κύριος ο Θεός σου, δεν έχω ψωμίον, αλλά μόνον μίαν χεριάν αλεύρου εις το πιθάριον και ολίγον έλαιον εις το ρωγίον· και ιδού, συνάγω δύο ξυλάρια, διά να υπάγω και να κάμω αυτό δι' εμαυτήν και διά τον υιόν μου, και να φάγωμεν αυτό και να αποθάνωμεν.
\par 13 Ο δε Ηλίας είπε προς αυτήν, Μη φοβού· ύπαγε, κάμε ως είπας· πλην εξ αυτού κάμε εις εμέ πρώτον μίαν μικράν πήτταν και φέρε εις εμέ, και έπειτα κάμε διά σεαυτήν και διά τον υιόν σου·
\par 14 διότι ούτω λέγει Κύριος ο Θεός του Ισραήλ· το πιθάριον του αλεύρου δεν θέλει κενωθή, ουδέ το ρωγίον του ελαίου θέλει ελαττωθή, έως της ημέρας καθ' ην ο Κύριος θέλει δώσει βροχήν επί προσώπου της γης.
\par 15 Η δε υπήγε και έκαμε κατά τον λόγον του Ηλία· και έτρωγεν αυτή και αυτός και ο οίκος αυτής ημέρας πολλάς·
\par 16 το πιθάριον του αλεύρου δεν εκενώθη, ουδέ το ρωγίον του ελαίου ηλαττώθη, κατά τον λόγον του Κυρίου, τον οποίον ελάλησε διά του Ηλία.
\par 17 Μετά δε τα πράγματα ταύτα, ηρρώστησεν ο υιός της γυναικός, της κυρίας του οίκου· και η αρρωστία αυτού ήτο δυνατή σφόδρα, εωσού δεν έμεινε πνοή εν αυτώ.
\par 18 Και είπε προς τον Ηλίαν, Τι έχεις μετ' εμού, άνθρωπε του Θεού; ήλθες προς εμέ διά να φέρης εις ενθύμησιν τας ανομίας μου και να θανατώσης τον υιόν μου;
\par 19 Ο δε είπε προς αυτήν, Δος μοι τον υιόν σου. Και έλαβεν αυτόν εκ του κόλπου αυτής και ανεβίβασεν αυτόν εις το υπερώον, όπου αυτός εκάθητο, και επλαγίασεν αυτόν επί την κλίνην αυτού.
\par 20 Και ανεβόησε προς τον Κύριον και είπε, Κύριε Θεέ μου· επέφερες κακόν και εις την χήραν, παρά τη οποία εγώ παροικώ, ώστε να θανατώσης τον υιόν αυτής;
\par 21 Και εξηπλώθη τρίς επί το παιδάριον και ανεβόησε προς τον Κύριον και είπε, Κύριε Θεέ μου, ας επανέλθη, δέομαι, η ψυχή του παιδαρίου τούτου εντός αυτού.
\par 22 Και εισήκουσεν ο Κύριος της φωνής του Ηλία· και επανήλθεν η ψυχή του παιδαρίου εντός αυτού και ανέζησε.
\par 23 Και έλαβεν ο Ηλίας το παιδάριον, και κατεβίβασεν αυτό από του υπερώου εις τον οίκον και έδωκεν αυτό εις την μητέρα αυτού. Και είπεν ο Ηλίας, Βλέπε, ζη ο υιός σου.
\par 24 Και είπεν η γυνή προς τον Ηλίαν, Τώρα γνωρίζω εκ τούτου ότι είσαι άνθρωπος του Θεού, και ο λόγος του Κυρίου εν τω στόματί σου είναι αλήθεια.

\chapter{18}

\par 1 Και μετά πολλάς ημέρας ήλθεν ο λόγος του Κυρίου προς τον Ηλίαν κατά το τρίτον έτος, λέγων, Ύπαγε, φανερώθητι εις τον Αχαάβ· και θέλω δώσει βροχήν επί το πρόσωπον της γης.
\par 2 Και υπήγεν ο Ηλίας να φανερωθή εις τον Αχαάβ. Η δε πείνα επεβάρυνεν εις την Σαμάρειαν.
\par 3 Και εκάλεσεν ο Αχαάβ τον Οβαδία τον οικονόμον. Ο δε Οβαδία εφοβείτο τον Κύριον σφόδρα·
\par 4 διότι, ότε η Ιεζάβελ εξωλόθρευε τους προφήτας του Κυρίου, ο Οβαδία έλαβεν εκατόν προφήτας και έκρυψεν αυτούς ανά πεντήκοντα εις σπήλαιον, και διέτρεφεν αυτούς εν άρτω και ύδατι.
\par 5 Και είπεν ο Αχαάβ προς τον Οβαδία, Περίελθε εις την γην, εις πάσας τας πηγάς των υδάτων και εις πάντας τους χειμάρρους· ίσως εύρωμεν χόρτον, διά να σώσωμεν την ζωήν των ίππων και των ημιόνων και να μη στερηθώμεν τα κτήνη.
\par 6 Εμέρισαν λοιπόν την γην εις εαυτούς, διά να διέλθωσιν αυτήν· ο μεν Αχαάβ απήλθε διά μιας οδού κατά μόνας, ο δε Οβαδία απήλθε δι' άλλης οδού κατά μόνας.
\par 7 Και ενώ ήτο ο Οβαδία καθ' οδόν ιδού, ο Ηλίας συνήντησεν αυτόν· και εκείνος εγνώρισεν αυτόν και έπεσε κατά πρόσωπον αυτού και είπε, Συ είσαι, κύριέ μου Ηλία;
\par 8 Ο δε είπε προς αυτόν, Εγώ· ύπαγε, ειπέ προς τον κύριόν σου, Ιδού, ο Ηλίας.
\par 9 Και εκείνος είπε, Τι ημάρτησα, ώστε θέλεις να παραδώσης τον δούλον σου εις την χείρα του Αχαάβ, διά να με θανατώση;
\par 10 Ζη Κύριος ο Θεός σου, δεν είναι έθνος ή βασίλειον, όπου δεν έστειλεν ο κύριός μου να σε ζητώσι και ότε έλεγον, Δεν είναι, αυτός ώρκιζε το βασίλειον και το έθνος, ότι δεν σε εύρηκαν.
\par 11 Και τώρα συ λέγεις, Ύπαγε, ειπέ προς τον κύριόν σου, Ιδού, ο Ηλίας.
\par 12 Και καθώς εγώ αναχωρήσω από σου, το πνεύμα του Κυρίου θέλει σε φέρει όπου δεν εξεύρω· και όταν υπάγω και αναγγείλω τούτο προς τον Αχαάβ, και δεν σε εύρη, θέλει με θανατώσει. Αλλ' ο δούλός σου φοβούμαι τον Κύριον εκ νεότητός μου.
\par 13 Δεν απηγγέλθη προς τον κύριόν μου τι έκαμα, ότε η Ιεζάβελ εθανάτονε τους προφήτας του Κυρίου, τίνι τρόπω έκρυψα εκατόν άνδρας εκ των προφητών του Κυρίου ανά πεντήκοντα εις σπήλαιον, και διέθρεψα αυτούς εν άρτω και ύδατι;
\par 14 Και τώρα συ λέγεις, Ύπαγε, ειπέ προς τον κύριόν σου, Ιδού, ο Ηλίας· αλλ' αυτός θέλει με θανατώσει.
\par 15 Και είπεν Ηλίας, Ζη ο Κύριος των δυνάμεων, έμπροσθεν του οποίου παρίσταμαι, ότι σήμερον θέλω εμφανισθή εις αυτόν.
\par 16 Υπήγε λοιπόν ο Οβαδία εις συνάντησιν του Αχαάβ και απήγγειλε προς αυτόν. Και ο Αχαάβ υπήγεν εις συνάντησιν του Ηλία.
\par 17 Και ως είδεν ο Αχαάβ τον Ηλίαν, είπε προς αυτόν ο Αχαάβ, Συ είσαι ο διαταράττων τον Ισραήλ;
\par 18 Ο δε είπε, Δεν διαταράττω εγώ τον Ισραήλ, αλλά συ και ο οίκος του πατρός σου· διότι σεις εγκατελίπετε τας εντολάς του Κυρίου και υπήγες κατόπιν των Βααλείμ·
\par 19 τώρα λοιπόν απόστειλον, συνάθροισον προς εμέ πάντα τον Ισραήλ εις το όρος τον Κάρμηλον, και τους προφήτας του Βάαλ τους τετρακοσίους πεντήκοντα, και τους τετρακοσίους προφήτας των αλσών, οίτινες τρώγουσιν εις την τράπεζαν της Ιεζάβελ.
\par 20 Και απέστειλεν ο Αχαάβ προς πάντας τους υιούς Ισραήλ και συνήθροισε τους προφήτας εις το όρος τον Κάρμηλον.
\par 21 Και προσήλθεν ο Ηλίας προς πάντα τον λαόν και είπεν, Έως πότε χωλαίνετε μεταξύ δύο φρονημάτων; εάν ο Κύριος ήναι Θεός, ακολουθείτε αυτόν· αλλ' εάν ο Βάαλ, ακολουθείτε τούτον. Και ο λαός δεν απεκρίθη προς αυτόν λόγον.
\par 22 Τότε είπεν ο Ηλίας προς τον λαόν, Εγώ μόνος έμεινα προφήτης του Κυρίου· οι δε προφήται του Βάαλ είναι τετρακόσιοι πεντήκοντα άνδρες·
\par 23 ας δώσωσι λοιπόν εις ημάς δύο μόσχους· και ας εκλέξωσι τον ένα μόσχον δι' εαυτούς, και ας διαμελίσωσιν αυτόν και ας επιθέσωσιν αυτόν επί των ξύλων και πυρ ας μη βάλωσι και εγώ θέλω ετοιμάσει τον άλλον μόσχον και επιθέσει επί των ξύλων και πυρ δεν θέλω βάλει,
\par 24 και επικαλέσθητε το όνομα των θεών σας, και εγώ θέλω επικαλεσθή το όνομα του Κυρίου· και ο Θεός, όστις εισακούση διά πυρός, ούτος ας ήναι ο Θεός. Και αποκριθείς πας ο λαός, είπε, Καλός ο λόγος.
\par 25 Και είπεν ο Ηλίας προς τους προφήτας του Βάαλ, Εκλέξατε εις εαυτούς τον ένα μόσχον και ετοιμάσατε αυτόν πρώτοι διότι είσθε πολλοί· και επικαλέσθητε το όνομα των θεών σας, πυρ όμως μη βάλητε.
\par 26 Και έλαβον τον μόσχον τον δοθέντα εις αυτούς και ητοίμασαν αυτόν, και επεκαλούντο το όνομα του Βάαλ από πρωΐας μέχρι μεσημβρίας, λέγοντες, Επάκουσον ημών, Βάαλ· και ουκ ην φωνή και ουκ ην ακρόασις· και επήδων περί το θυσιαστήριον, το οποίον ωκοδόμησαν.
\par 27 Και περί την μεσημβρίαν ο Ηλίας μυκτηρίζων αυτούς έλεγεν, Επικαλείσθε μετά φωνής μεγάλης· διότι θεός είναι ή συνομιλεί ή ασχολείται ή είναι εις οδοιπορίαν ή ίσως κοιμάται και θέλει εξυπνήσει.
\par 28 Και επεκαλούντο μετά φωνής μεγάλης και κατετέμνοντο κατά την συνήθειαν αυτών με μαχαίρας και με λόγχας, εωσού αίμα εξεχύθη επ' αυτούς.
\par 29 Και αφού παρήλθεν η μεσημβρία, και αυτοί προεφήτευον μέχρι της ώρας της προσφοράς, και ουκ ην φωνή και ουκ ην ακρόασις και ουκ ην προσοχή,
\par 30 τότε είπεν ο Ηλίας προς πάντα τον λαόν, Πλησιάσατε προς εμέ. Και πας ο λαός επλησίασε προς αυτόν. Και επιδιώρθωσε το θυσιαστήριον του Κυρίου, το κεκρημνισμένον.
\par 31 Και έλαβεν ο Ηλίας δώδεκα λίθους, κατά τον αριθμόν των φυλών των υιών Ιακώβ, προς τον οποίον ήλθεν ο λόγος του Κυρίου, λέγων, Ισραήλ θέλει είσθαι το όνομά σου·
\par 32 και ωκοδόμησε τους λίθους θυσιαστήριον εις το όνομα του Κυρίου· και έκαμεν αύλακα περί το θυσιαστήριον, χωρούσαν δύο μέτρα σπόρου.
\par 33 Και εστοίβασε τα ξύλα και διεμέλισε τον μόσχον και επέθεσεν αυτόν επί των ξύλων.
\par 34 Και είπε, Γεμίσατε ύδατος τέσσαρας υδρίας και χύσατε επί το ολοκαύτωμα και επί τα ξύλα. Και είπε, Δευτερώσατε· και εδευτέρωσαν. Και είπε, Τριττώσατε· και ετρίττωσαν.
\par 35 Και περιέτρεχε το ύδωρ πέριξ του θυσιαστηρίου· και η αύλαξ έτι εγέμισεν ύδατος.
\par 36 Και την ώραν της προσφοράς επλησίασεν Ηλίας ο προφήτης και είπε, Κύριε, Θεέ του Αβραάμ, του Ισαάκ και του Ισραήλ, ας γείνη γνωστόν σήμερον, ότι συ είσαι Θεός εν τω Ισραήλ και εγώ δούλός σου, και κατά τον λόγον σου έκαμα πάντα ταύτα τα πράγματα·
\par 37 επάκουσόν μου, Κύριε, επάκουσόν μου, διά να γνωρίση ο λαός ούτος ότι συ Κύριος είσαι ο Θεός, και συ επέστρεψας την καρδίαν αυτών οπίσω.
\par 38 Τότε έπεσε πυρ παρά Κυρίου και κατέφαγε το ολοκαύτωμα και τα ξύλα και τους λίθους και το χώμα, και έγλειψε το ύδωρ το εν τη αύλακι.
\par 39 Και ότε είδε πας ο λαός, έπεσον κατά πρόσωπον αυτών και είπον, Ο Κύριος, αυτός είναι ο Θεός· ο Κύριος, αυτός είναι ο Θεός.
\par 40 Και είπε προς αυτούς ο Ηλίας, Πιάσατε τους προφήτας του Βάαλ· μηδείς εξ αυτών ας μη διασωθή. Και επίασαν αυτούς· και κατεβίβασεν αυτούς ο Ηλίας εις τον χείμαρρον Κεισών και έσφαξεν αυτούς εκεί.
\par 41 Και είπεν ο Ηλίας προς τον Αχαάβ, Ανάβα, φάγε και πίε. διότι είναι φωνή πλήθους βροχής.
\par 42 Και ανέβη ο Αχαάβ διά να φάγη και να πίη. Ο δε Ηλίας ανέβη εις την κορυφήν του Καρμήλου και έκυψεν εις την γην και έβαλε το πρόσωπον αυτού αναμέσον των γονάτων αυτού,
\par 43 και είπε προς τον υπηρέτην αυτού, Ανάβα τώρα, βλέψον προς την θάλασσαν. Και ανέβη και έβλεψε και είπε, Δεν είναι ουδέν. Ο δε είπεν, Ύπαγε πάλιν, έως επτάκις.
\par 44 Και την εβδόμην φοράν είπεν, Ιδού, νέφος μικρόν, ως παλάμη ανθρώπου, αναβαίνει εκ της θαλάσσης. Και είπεν, Ανάβα, ειπέ προς τον Αχαάβ, Ζεύξον την άμαξάν σου, και κατάβα, διά να μη σε εμποδίση η βροχή.
\par 45 Και εν τω μεταξύ ο ουρανός συνεσκότασεν εκ νεφών και ανέμου, και έγεινε βροχή μεγάλη. Και ανέβη ο Αχαάβ εις την άμαξαν αυτού και υπήγεν εις Ιεζραέλ.
\par 46 Και χειρ Κυρίου εστάθη επί τον Ηλίαν· και συνέσφιγξε την οσφύν αυτού και έτρεχεν έμπροσθεν του Αχαάβ έως της εισόδου της Ιεζραέλ.

\chapter{19}

\par 1 Και απήγγειλεν ο Αχαάβ προς την Ιεζάβελ πάντα όσα έκαμεν ο Ηλίας, και τίνι τρόπω εθανάτωσεν εν ρομφαία πάντας τους προφήτας.
\par 2 Και απέστειλε μηνυτήν η Ιεζάβελ προς τον Ηλίαν, λέγουσα, Ούτω να κάμωσιν οι θεοί και ούτω να προσθέσωσιν, εάν αύριον περί την ώραν ταύτην δεν καταστήσω την ζωήν σου ως την ζωήν ενός εξ εκείνων.
\par 3 Και φοβηθείς, εσηκώθη και ανεχώρησε διά την ζωήν αυτού, και ήλθεν εις Βηρ-σαβεέ την του Ιούδα και αφήκεν εκεί τον υπηρέτην αυτού.
\par 4 Αυτός δε υπήγεν εις την έρημον μιας ημέρας οδόν, και ήλθε και εκάθησεν υπό τινά άρκευθον· και επεθύμησε καθ' εαυτόν να αποθάνη και είπεν, Αρκεί· τώρα, Κύριε, λάβε την ψυχήν μου· διότι δεν είμαι εγώ καλήτερος των πατέρων μου.
\par 5 Και πλαγιάσας απεκοιμήθη υποκάτω μιας αρκεύθου, και ιδού, άγγελος ήγγισεν αυτόν και είπε προς αυτόν, Σηκώθητι, φάγε.
\par 6 Και ανέβλεψε, και ιδού, πλησίον της κεφαλής αυτού άρτος εγκρυφίας και αγγείου ύδατος. Και έφαγε και έπιε και πάλιν επλαγίασε.
\par 7 Και επέστρεψεν ο άγγελος του Κυρίου εκ δευτέρου και ήγγισεν αυτόν και είπε, Σηκώθητι, φάγε· διότι πολλή είναι η οδός από σου.
\par 8 Και σηκωθείς, έφαγε και έπιε, και με την δύναμιν της τροφής εκείνης ώδοιπόρησε τεσσαράκοντα ημέρας και τεσσαράκοντα νύκτας, έως Χωρήβ του όρους του Θεού.
\par 9 Και εισήλθεν εκεί εις σπήλαιον και έκαμεν εκεί κατάλυμα· και ιδού, ήλθε λόγος Κυρίου προς αυτόν και είπε προς αυτόν, Τι κάμνεις ενταύθα, Ηλία;
\par 10 Ο δε είπεν, Εστάθην εις άκρον ζηλωτής υπέρ Κυρίου του Θεού των δυνάμεων· διότι οι υιοί Ισραήλ εγκατέλιπον την διαθήκην σου, τα θυσιαστήριά σου κατέστρεψαν και τους προφήτας σου εθανάτωσαν εν ρομφαία· και εναπελείφθην εγώ μόνος· και ζητούσι την ζωήν μου, διά να αφαιρέσωσιν αυτήν.
\par 11 Και είπεν, Έξελθε και στάθητι επί το όρος ενώπιον Κυρίου. Και ιδού, ο Κύριος διέβαινε, και άνεμος μέγας και δυνατός έσχιζε τα όρη και συνέτριβε τους βράχους έμπροσθεν του Κυρίου· ο Κύριος δεν ήτο εν τω ανέμω· και μετά τον άνεμον σεισμός· ο Κύριος δεν ήτο εν τω σεισμώ·
\par 12 και μετά τον σεισμόν, πύρ· ο Κύριος δεν ήτο εν τω πυρί· και μετά το πυρ, ήχος λεπτού αέρος.
\par 13 Και ως ήκουσεν ο Ηλίας, εσκέπασε το πρόσωπον αυτού με την μηλωτήν αυτού και εξήλθε και εστάθη εις την είσοδον του σπηλαίου. Και ιδού, φωνή προς αυτόν, λέγουσα, Τι κάμνεις ενταύθα, Ηλία;
\par 14 Και είπεν, Εστάθην εις άκρον ζηλωτής υπέρ Κυρίου του Θεού των δυνάμεων· διότι οι υιοί του Ισραήλ εγκατέλιπον την διαθήκην σου, τα θυσιαστήριά σου κατέστρεψαν και τους προφήτας σου εθανάτωσαν εν ρομφαία· και εναπελείφθην εγώ μόνος· και ζητούσι την ζωήν μου, διά να αφαιρέσωσιν αυτήν.
\par 15 Και είπε Κύριος προς αυτόν, Ύπαγε, επίστρεψον εις την οδόν σου προς την έρημον της Δαμασκού· και όταν έλθης, χρίσον τον Αζαήλ βασιλέα επί την Συρίαν·
\par 16 τον δε Ιηού τον υιόν του Νιμσί θέλεις χρίσει βασιλέα επί τον Ισραήλ· και τον Ελισσαιέ τον υιόν του Σαφάτ, από Αβέλ-μεολά, θέλεις χρίσει προφήτην αντί σού·
\par 17 και θέλει συμβή, ώστε τον διασωθέντα εκ της ρομφαίας του Αζαήλ, θέλει θανατώσει ο Ιηού· και τον διασωθέντα εκ της ρομφαίας του Ιηού, θέλει θανατώσει ο Ελισσαιέ·
\par 18 αφήκα όμως εις τον Ισραήλ επτά χιλιάδας, πάντα τα γόνατα, όσα δεν έκλιναν εις τον Βάαλ, και παν στόμα το οποίον δεν ησπάσθη αυτόν.
\par 19 Και αναχωρήσας εκείθεν, εύρηκε τον Ελισσαιέ τον υιόν του Σαφάτ, ενώ ώργονε με δώδεκα ζεύγη βοών έμπροσθεν αυτού, αυτός ων εις το δωδέκατον· και επέρασεν ο Ηλίας από πλησίον αυτού και έρριψεν επ' αυτόν την μηλωτήν αυτού.
\par 20 Ο δε αφήκε τους βόας και έτρεξε κατόπιν του Ηλία και είπεν, Ας ασπασθώ, παρακαλώ, τον πατέρα μου και την μητέρα μου, και τότε θέλω σε ακολουθήσει. Και είπε προς αυτόν, Ύπαγε, επίστρεψον· διότι τι έκαμα εις σε;
\par 21 Και έστρεψεν εξόπισθεν αυτού και έλαβεν εν ζεύγος βοών και έσφαξεν αυτούς, και έψησε το κρέας αυτών με τα εργαλεία των βοών και έδωκεν εις τον λαόν, και έφαγον. Τότε σηκωθείς, υπήγε κατόπιν του Ηλία και υπηρέτει αυτόν.

\chapter{20}

\par 1 Ο δε Βεν-αδάδ βασιλεύς της Συρίας συνήθροισε πάσαν την δύναμιν αυτού· ήσαν δε μετ' αυτού τριάκοντα δύο βασιλείς και ίπποι και άμαξαι και ανέβη και επολιόρκησε την Σαμάρειαν και επολέμει αυτήν.
\par 2 Και απέστειλε μηνυτάς προς Αχαάβ τον βασιλέα του Ισραήλ εις την πόλιν και είπε προς αυτόν, Ούτω λέγει ο Βεν-αδάδ·
\par 3 το αργύριόν σου και το χρυσίον σου είναι εμού· και αι γυναίκες σου και τα τέκνα σου τα ώραία είναι εμού.
\par 4 Και απεκρίθη ο βασιλεύς του Ισραήλ και είπε, Κατά τον λόγον σου, κύριέ μου βασιλεύ, σου είμαι εγώ και πάντα όσα έχω.
\par 5 Και επανήλθον οι μηνυταί και είπον, Ούτως αποκρίνεται ο Βεν-αδάδ, λέγων· Επειδή απέστειλα προς σε, λέγων, Το αργύριόν σου και το χρυσίον σου και τας γυναίκάς σου και τα τέκνα σου θέλεις παραδώσει εις εμέ,
\par 6 αύριον βεβαίως περί την ώραν ταύτην θέλω αποστείλει τους δούλους μου προς σε, και θέλουσιν ερευνήσει τον οίκόν σου και τους οίκους των δούλων σου· και ό,τι είναι επιθυμητόν εις τους οφθαλμούς σου, θέλουσι βάλει εις τας χείρας αυτών και θέλουσι λάβει αυτό.
\par 7 Τότε εκάλεσεν ο βασιλεύς του Ισραήλ πάντας τους πρεσβυτέρους του τόπου και είπε, Στοχασθήτε, παρακαλώ, και ιδέτε ότι ούτος κακίαν ζητεί· διότι απέστειλε προς εμέ διά τας γυναίκάς μου και διά τα τέκνα μου και διά το αργύριόν μου και διά το χρυσίον μου, και δεν ηρνήθην ουδέν εις αυτόν.
\par 8 Και είπον προς αυτόν πάντες οι πρεσβύτεροι και πας ο λαός, Μη υπακούσης μηδέ συγκατανεύσης.
\par 9 Είπε λοιπόν προς τους μηνυτάς του Βεν-αδάδ, Είπατε προς τον κύριόν μου τον βασιλέα, Πάντα όσα εμήνυσας προς τον δούλον σου κατ' αρχάς, θέλω κάμει τούτο όμως το πράγμα δεν δύναμαι να κάμω. Και οι μηνυταί ανεχώρησαν και έφεραν προς αυτόν την απόκρισιν.
\par 10 Και αναπέστειλεν ο Βεν-αδάδ προς αυτόν, λέγων, Ούτω να κάμωσιν εις εμέ οι θεοί και ούτω να προσθέσωσιν, εάν το χώμα της Σαμαρείας αρκέση διά μίαν χεριάν εις πάντα τον λαόν, τον ακολουθούντά με.
\par 11 Και απεκρίθη ο βασιλεύς του Ισραήλ και είπεν, Είπατε προς αυτόν, Όστις περιζώννυται τα όπλα, ας μη μεγαλαυχή ως ο εκδυόμενος αυτά.
\par 12 Ότε δε ο Βεν-αδάδ ήκουσε τον λόγον τούτον, έτυχε πίνων, αυτός και οι βασιλείς οι μετ' αυτού εις τας σκηνάς, και είπε προς τους δούλους αυτού, Παρατάχθητε. Και παρετάχθησαν κατά της πόλεως.
\par 13 Και ιδού, προσήλθε προς τον Αχαάβ τον βασιλέα του Ισραήλ προφήτης τις, λέγων, Ούτω λέγει Κύριος· Βλέπεις άπαν το πλήθος τούτο το μέγα; ιδού, εγώ παραδίδω αυτό εις την χείρα σου σήμερον· και θέλεις γνωρίσει ότι εγώ είμαι ο Κύριος.
\par 14 Και είπεν ο Αχαάβ, Διά τίνος; Ο δε απεκρίθη, Ούτω λέγει Κύριος· Διά των θεραπόντων των αρχόντων των επαρχιών. Τότε είπε, Τις θέλει συγκροτήσει την μάχην; Και απεκρίθη, Συ.
\par 15 Τότε ηρίθμησε τους θεράποντας των αρχόντων των επαρχιών· και ήσαν διακόσιοι τριάκοντα δύο· και μετ' αυτούς, ηρίθμησεν άπαντα τον λαόν, πάντας τους υιούς Ισραήλ, επτά χιλιάδας.
\par 16 Και εξήλθον περί την μεσημβρίαν. Ο δε Βεν-αδάδ έπινε και εμέθυεν εις τας σκηνάς, αυτός και οι βασιλείς, οι τριάκοντα δύο βασιλείς οι σύμμαχοι αυτού.
\par 17 Και εξήλθον πρώτοι οι θεράποντες των αρχόντων των επαρχιών· και απέστειλεν ο Βεν-αδάδ να μάθη· και απήγγειλαν προς αυτόν, λέγοντες, Άνδρες εξήλθον εκ της Σαμαρείας.
\par 18 Ο δε είπεν, Εάν εξήλθον ειρηνικώς, συλλάβετε αυτούς ζώντας· και εάν εξήλθον διά πόλεμον, πάλιν ζώντας συλλάβετε αυτούς.
\par 19 Εξήλθον λοιπόν εκ της πόλεως ούτοι οι θεράποντες των αρχόντων των επαρχιών, και το στράτευμα το οποίον ηκολούθει αυτούς.
\par 20 Και επάταξεν έκαστος τον άνθρωπον αυτού· και οι Σύριοι έφυγον· και κατεδίωξεν αυτούς ο Ισραήλ· ο δε Βεν-αδάδ ο βασιλεύς της Συρίας διεσώθη έφιππος μετά των ιππέων.
\par 21 Και εξήλθεν ο βασιλεύς του Ισραήλ και επάταξε τους ιππείς και τας αμάξας, και έκαμεν εις τους Συρίους σφαγήν μεγάλην.
\par 22 Και προσήλθεν ο προφήτης προς τον βασιλέα του Ισραήλ και είπε προς αυτόν, Ύπαγε, ενδυναμώθητι και σκέφθητι, και ιδέ τι θέλεις κάμει διότι εν τη επιστροφή του έτους ο βασιλεύς της Συρίας θέλει αναβή εναντίον σου.
\par 23 Είπον δε προς αυτόν οι δούλοι του βασιλέως της Συρίας, Ο θεός αυτών είναι θεός των βουνών· διά τούτο υπερίσχυσαν καθ' ημών· εάν δε πολεμήσωμεν αυτούς εν τη πεδιάδι, βεβαίως θέλομεν υπερισχύσει κατ' αυτών.
\par 24 Κάμε λοιπόν το πράγμα τούτο· έκβαλε τους βασιλείς, έκαστον εκ του τόπου αυτού· και βάλε αντ' αυτών στρατηγούς·
\par 25 συ δε συνάθροισον εις σεαυτόν στράτευμα, όσον στράτευμα εκ των μετά σου έπεσε, και ίππον αντί ίππου και άμαξαν αντί αμάξης· και ας πολεμήσωμεν αυτούς εν τη πεδιάδι, και βεβαίως θέλομεν υπερισχύσει κατ' αυτών. Και εισήκουσε της φωνής αυτών και έκαμεν ούτω.
\par 26 Και εν τη επιστροφή του έτους ηρίθμησεν ο Βεν-αδάδ τους Συρίους και ανέβη εις Αφέκ, διά να πολεμήση κατά του Ισραήλ.
\par 27 Και οι υιοί Ισραήλ ηριθμήθησαν, και προπαρασκευασθέντες υπήγον εις συνάντησιν αυτών· και εστρατοπέδευσαν οι υιοί Ισραήλ απέναντι αυτών, ως δύο μικρά ποίμνια αιγών· οι δε Σύριοι εγέμισαν την γην.
\par 28 Και προσήλθεν ο άνθρωπος του Θεού και ελάλησε προς τον βασιλέα του Ισραήλ, και είπεν, Ούτω λέγει Κύριος· Επειδή οι Σύριοι είπον, Ο Κύριος είναι Θεός των βουνών, αλλ' ουχί Θεός των κοιλάδων, διά τούτο θέλω παραδώσει εις την χείρα σου άπαν το μέγα τούτο πλήθος, και θέλετε γνωρίσει ότι εγώ είμαι ο Κύριος.
\par 29 Και ήσαν εστρατοπεδευμένοι αντικρύ αλλήλων επτά ημέρας. Και την εβδόμην ημέραν συνεκροτήθη η μάχη· και επάταξαν οι υιοί Ισραήλ τους Συρίους εκατόν χιλιάδας πεζών εν ημέρα μιά.
\par 30 Οι δε εναπολειφθέντες έφυγον εις Αφέκ, προς την πόλιν· και έπεσε το τείχος επί εικοσιεπτά χιλιάδας εκ των ανδρών των εναπολειφθέντων. Και έφυγεν ο Βεν-αδάδ και εισήλθεν εις την πόλιν και εκρύφθη από κοιτώνος εις κοιτώνα.
\par 31 Και είπον προς αυτόν οι δούλοι αυτού, Ιδού τώρα, ηκούσαμεν ότι οι βασιλείς του οίκου Ισραήλ είναι βασιλείς ελεήμονες· ας βάλωμεν λοιπόν σάκκους επί τας οσφύας ημών και σχοινία επί τας κεφαλάς ημών, και ας εξέλθωμεν προς τον βασιλέα του Ισραήλ· ίσως θέλει σοι χαρίσει την ζωήν.
\par 32 Περιεζώσθησαν λοιπόν σάκκους εις τας οσφύας αυτών και σχοινία εις τας κεφαλάς αυτών, και ήλθον προς τον βασιλέα του Ισραήλ και είπον, Ο δούλός σου Βεν-αδάδ λέγει, Ας ζήση η ψυχή μου, παρακαλώ. Και είπε, Ζη ακόμη; αδελφός μου είναι.
\par 33 Και οι άνδρες έλαβον τούτο διά καλόν οιωνόν, και έσπευσαν να στερεώσωσι το εξελθόν εκ του στόματος αυτού· και είπον, Ο αδελφός σου Βεν-αδάδ. Και είπεν, Υπάγετε, φέρετε αυτόν. Ότε δε ήλθε προς αυτόν ο Βεν-αδάδ, εκείνος ανεβίβασεν αυτόν εις την άμαξαν αυτού.
\par 34 Και είπε προς αυτόν ο Βεν-αδάδ, τας πόλεις, τας οποίας έλαβεν ο πατήρ μου παρά του πατρός σου, θέλω αποδώσει και θέλεις στήσει εις σεαυτόν οχυρώματα εν Δαμασκώ, καθώς έστησεν ο πατήρ μου εν Σαμαρεία. Και εγώ, είπεν ο Αχαάβ, θέλω σε εξαποστείλει επί ταύτη τη συνθήκη. Ούτως έκαμε συνθήκην μετ' αυτού και εξαπέστειλεν αυτόν.
\par 35 Άνθρωπος δε τις εκ των υιών των προφητών είπε προς τον πλησίον αυτού εν λόγω Κυρίου, Κτύπησόν με, παρακαλώ. Αλλά δεν ηθέλησεν ο άνθρωπος να κτυπήση αυτόν.
\par 36 Και είπε προς αυτόν, Επειδή δεν υπήκουσας της φωνής του Κυρίου, ιδού, καθώς αναχωρήσης απ' εμού, λέων θέλει σε θανατώσει. Και ως ανεχώρησεν απ' αυτού, εύρηκεν αυτόν λέων και εθανάτωσεν αυτόν.
\par 37 Ευρών έπειτα άλλον άνθρωπον, είπε, Κτύπησόν με, παρακαλώ. Και ο άνθρωπος εκτύπησεν αυτόν, και κτυπήσας επλήγωσε.
\par 38 Τότε ανεχώρησεν ο προφήτης και εστάθη επί της οδού διά τον βασιλέα, μεταμεμορφωμένος με κάλυμμα επί τους οφθαλμούς αυτού.
\par 39 Και ως διέβαινεν ο βασιλεύς αυτός εβόησε προς τον βασιλέα, και είπεν, Ο δούλός σου εξήλθεν εις το μέσον της μάχης· και ιδού, άνθρωπος στραφείς κατά μέρος έφερε τινά προς εμέ, και είπε, Φύλαττε τον άνθρωπον τούτον· εάν ποτέ φύγη, τότε η ζωή σου θέλει είσθαι αντί της ζωής αυτού, ή θέλεις πληρώσει εν τάλαντον αργυρίου·
\par 40 και ενώ ο δούλός σου ησχολείτο εδώ και εκεί, αυτός έφυγε. Και είπε προς αυτόν ο βασιλεύς του Ισραήλ, αύτη είναι η κρίσις σου· αυτός συ απεφάσισας αυτήν.
\par 41 Τότε έσπευσε και αφήρεσε το κάλυμμα από των οφθαλμών αυτού· και εγνώρισεν αυτόν ο βασιλεύς του Ισραήλ ότι ήτο εκ των προφητών.
\par 42 Και είπε προς αυτόν, ούτω λέγει Κύριος· Επειδή συ εξαπέστειλας από της χειρός σου άνθρωπον, τον οποίον εγώ είχον αποφασίσει εις όλεθρον, διά τούτο η ζωή σου θέλει είσθαι αντί της ζωής αυτού, και ο λαός σου αντί του λαού αυτού.
\par 43 Και απήλθεν ο βασιλεύς του Ισραήλ εις τον οίκον αυτού σκυθρωπός και δυσηρεστημένος και ήλθεν εις την Σαμάρειαν.

\chapter{21}

\par 1 Μετά δε ταύτα τα πράγματα Ναβουθαί ο Ιεζραηλίτης είχεν αμπελώνα εν Ιεζραέλ, πλησίον του παλατίου του Αχαάβ βασιλέως της Σαμαρείας.
\par 2 Και ελάλησεν ο Αχαάβ προς τον Ναβουθαί, λέγων, Δος μοι τον αμπελώνά σου, διά να έχω αυτόν κήπον λαχάνων, επειδή είναι πλησίον του οίκου μου· και θέλω σοι δώσει αντ' αυτού αμπελώνα καλήτερον παρ' αυτού· ή, αν ήναι αρεστόν εις σε, θέλω σοι δώσει το αντίτιμον αυτού εις αργύριον.
\par 3 Ο δε Ναβουθαί είπε προς τον Αχαάβ, Μη γένοιτο εις εμέ παρά Θεού, να δώσω την κληρονομίαν των πατέρων μου εις σε.
\par 4 Και ήλθεν ο Αχαάβ εις τον οίκον αυτού σκυθρωπός και δυσηρεστημένος διά τον λόγον, τον οποίον ελάλησε προς αυτόν Ναβουβαί ο Ιεζραηλίτης, ειπών, Δεν θέλω σοι δώσει την κληρονομίαν των πατέρων μου. Και επλαγίασεν επί της κλίνης αυτού και απέστρεψε το πρόσωπον αυτού και δεν έφαγεν άρτον.
\par 5 Και ήλθε προς αυτόν Ιεζάβελ η γυνή αυτού και είπε προς αυτόν, Διά τι το πνεύμά σου είναι περίλυπον, ώστε δεν τρώγεις άρτον;
\par 6 Ο δε είπε προς αυτήν, Επειδή ελάλησα προς Ναβουθαί τον Ιεζραηλίτην, και είπα προς αυτόν, Δος μοι τον αμπελώνά σου δι' αργυρίου· ή, αν αγαπάς, θέλω σοι δώσει άλλον αμπελώνα αντ' αυτού. και εκείνος απεκρίθη, Δεν θέλω σοι δώσει τον αμπελώνά μου.
\par 7 Και είπε προς αυτόν Ιεζάβελ η γυνή αυτού, Συ τώρα βασιλεύεις επί τον Ισραήλ; σηκώθητι, φάγε άρτον, και ας ήναι εύθυμος η καρδία σου· εγώ θέλω σοι δώσει τον αμπελώνα Ναβουθαί του Ιεζραηλίτου.
\par 8 Τότε έγραψεν επιστολάς εν ονόματι του Αχαάβ και εσφράγισε διά της σφραγίδος αυτού, και απέστειλε τας επιστολάς προς τους πρεσβυτέρους και προς τους άρχοντας, τους όντας εν τη πόλει αυτού, τους κατοικούντας μετά του Ναβουβαί.
\par 9 Και έγραφεν εν ταις επιστολαίς, λέγουσα, Κηρύξατε νηστείαν και καθίσατε τον Ναβουθαί επί κεφαλής του λαού·
\par 10 και παρακαθίσατε δύο άνδρας κακούς αντικρύ αυτού, και ας μαρτυρήσωσι κατ' αυτού, λέγοντες, Συ εβλασφήμησας τον Θεόν και τον βασιλέα· και εκβάλετε αυτόν και λιθοβολήσατε αυτόν, και ας αποθάνη.
\par 11 Και έκαμον οι άνδρες της πόλεως αυτού, οι πρεσβύτεροι και οι άρχοντες οι κατοικούντες εν τη πόλει αυτού, καθώς εμήνυσε προς αυτούς η Ιεζάβελ, κατά το γεγραμμένον εν ταις επιστολαίς τας οποίας έστειλε προς αυτούς.
\par 12 Εκήρυξαν νηστείαν και εκάθησαν τον Ναβουθαί επί κεφαλής του λαού·
\par 13 και εισήλθον δύο άνδρες κακοί και εκάθισαν αντικρύ αυτού· και εμαρτύρησαν οι άνδρες οι κακοί κατ' αυτού, κατά του Ναβουθαί, ενώπιον του λαού, λέγοντες, Ο Ναβουθαί εβλασφήμησε τον Θεόν και τον βασιλέα. Τότε εξέβαλον αυτόν έξω της πόλεως και ελιθοβόλησαν αυτόν με λίθους, και απέθανε.
\par 14 Και απέστειλαν προς την Ιεζάβελ, λέγοντες, Ο Ναβουβαί ελιθοβολήθη και απέθανε.
\par 15 Και ως ήκουσεν η Ιεζάβελ ότι ο Ναβουβαί ελιθοβολήθη και απέθανεν, είπεν η Ιεζάβελ προς τον Αχαάβ, Σηκώθητι, κληρονόμησον τον αμπελώνα Ναβουθαί του Ιεζραηλίτου, τον οποίον δεν ήθελε να σοι δώση δι' αργυρίου· διότι ο Ναβουθαί δεν ζη αλλ' απέθανε.
\par 16 Και ως ήκουσεν ο Αχαάβ ότι ο Ναβουθαί απέθανεν, εσηκώθη ο Αχαάβ να καταβή εις τον αμπελώνα του Ναβουθαί του Ιεζραηλίτου, διά να κληρονομήση αυτόν.
\par 17 Και ήλθεν ο λόγος του Κυρίου προς Ηλίαν τον Θεσβίτην, λέγων,
\par 18 Σηκώθητι, κατάβα εις συνάντησιν του Αχαάβ, βασιλέως του Ισραήλ, όστις κατοικεί εν Σαμαρεία· ιδού, εν τω αμπελώνι του Ναβουθαί είναι, όπου κατέβη διά να κληρονομήση αυτόν·
\par 19 και θέλεις λαλήσει προς αυτόν, λέγων, Ούτω λέγει Κύριος· Εφόνευσας και έτι εκληρονόμησας; Και θέλεις λαλήσει προς αυτόν, λέγων, ούτω λέγει Κύριος· Εν τω τόπω, όπου οι κύνες έγλειψαν το αίμα του Ναβουθαί, θέλουσι γλείψει οι κύνες το αίμα σου, ναι, σου.
\par 20 Και είπεν ο Αχαάβ προς τον Ηλίαν, Με εύρηκας, εχθρέ μου; Και απεκρίθη, Σε εύρηκα· διότι επώλησας σεαυτόν εις το να πράττης το πονηρόν ενώπιον του Κυρίου.
\par 21 Ιδού, λέγει Κύριος, Εγώ θέλω φέρει κακόν επί σε, και θέλω σαρώσει κατόπιν σου και εξολοθρεύσει του Αχαάβ τον ουρούντα προς τον τοίχον και τον πεφυλαγμένον και τον αφειμένον μεταξύ του Ισραήλ·
\par 22 και θέλω καταστήσει τον οίκόν σου ως τον οίκον του Ιεροβοάμ υιού του Ναβάτ, και ως τον οίκον του Βαασά υιού του Αχιά, διά τον παροργισμόν τον οποίον με παρώργιαας, και έκαμες τον Ισραήλ να αμαρτήση.
\par 23 Και περί της Ιεζάβελ έτι ελάλησεν ο Κύριος, λέγων, Οι κύνες θέλουσι καταφάγει την Ιεζάβελ πλησίον του προτειχίσματος της Ιεζραέλ·
\par 24 όστις εκ του Αχαάβ αποθάνη εν τη πόλει, οι κύνες θέλουσι καταφάγει αυτόν· και όστις αποθάνη εν τω αγρώ, τα πετεινά του ουρανού θέλουσι καταφάγει αυτόν.
\par 25 Ουδείς τωόντι δεν εστάθη όμοιος του Αχαάβ, όστις επώλησεν εαυτόν εις το να πράττη πονηρά ενώπιον του Κυρίου, όπως εκίνει αυτόν Ιεζάβελ η γυνή αυτού.
\par 26 Και έπραξε βδελυρά σφόδρα ακολουθών τα είδωλα, κατά πάντα όσα έπραττον οι Αμορραίοι, τους οποίους ο Κύριος εξεδίωξεν απ' έμπροσθεν των υιών Ισραήλ.
\par 27 Ως δε ήκουσεν ο Αχαάβ τους λόγους τούτους, διέρρηξε τα ιμάτια αυτού και έβαλε σάκκον επί την σάρκα αυτού και ενήστευσε, και εκοίτετο περιτετυλιγμένος σάκκον και εβάδιζε κεκυφώς.
\par 28 Ήλθε δε ο λόγος του Κυρίου προς Ηλίαν τον Θεσβίτην, λέγων,
\par 29 Είδες πως εταπεινώθη ο Αχαάβ ενώπιόν μου; επειδή εταπεινώθη ενώπιόν μου, δεν θέλω φέρει το κακόν εν ταις ημέραις αυτού· εν ταις ημέραις του υιού αυτού θέλω φέρει το κακόν επί τον οίκον αυτού.

\chapter{22}

\par 1 Παρήλθον δε τρία έτη άνευ πολέμου αναμέσον της Συρίας και του Ισραήλ.
\par 2 Κατά δε το τρίτον έτος κατέβη Ιωσαφάτ ο βασιλεύς του Ιούδα προς τον βασιλέα του Ισραήλ.
\par 3 Και είπεν ο βασιλεύς του Ισραήλ προς τους δούλους αυτού, Εξεύρετε ότι η Ραμώθ-γαλαάδ είναι ημών, και ημείς σιωπώμεν εις το να λάβωμεν αυτήν εκ της χειρός του βασιλέως της Συρίας;
\par 4 Και είπε προς τον Ιωσαφάτ, Έρχεσαι μετ' εμού διά να πολεμήσωμεν την Ραμώθ-γαλαάδ; Και είπεν ο Ιωσαφάτ προς τον βασιλέα του Ισραήλ, Εγώ είμαι καθώς συ, ο λαός μου καθώς ο λαός σου, οι ίπποι μου καθώς οι ίπποι σου.
\par 5 Και είπεν ο Ιωσαφάτ προς τον βασιλέα του Ισραήλ, Ερώτησον, παρακαλώ, τον λόγον του Κυρίου σήμερον.
\par 6 Και συνήθροισεν ο βασιλεύς του Ισραήλ τους προφήτας, περίπου τετρακοσίους άνδρας, και είπε προς αυτούς, να υπάγω εναντίον της Ραμώθ-γαλαάδ να πολεμήσω, ή να απέχω; οι δε είπον, Ανάβα, και ο Κύριος θέλει παραδώσει αυτήν εις την χείρα του βασιλέως.
\par 7 Και είπεν ο Ιωσαφάτ, Δεν είναι ενταύθα έτι προφήτης του Κυρίου, διά να ερωτήσωμεν δι' αυτού;
\par 8 Και είπεν ο βασιλεύς του Ισραήλ προς τον Ιωσαφάτ, Είναι έτι άνθρωπός τις, Μιχαίας, ο υιός του Ιεμλά, διά του οποίου δυνάμεθα να ερωτήσωμεν τον Κύριον· πλην μισώ αυτόν· διότι δεν προφητεύει καλόν περί εμού, αλλά κακόν. Και είπεν ο Ιωσαφάτ, Ας μη λαλή ο βασιλεύς ούτως.
\par 9 Και εκάλεσεν ο βασιλεύς του Ισραήλ ένα ευνούχον και είπε, Σπεύσον να φέρης Μιχαίαν τον υιόν του Ιεμλά.
\par 10 Ο δε βασιλεύς του Ισραήλ και Ιωσαφάτ ο βασιλεύς του Ιούδα εκάθηντο, έκαστος επί του θρόνου αυτού, ενδεδυμένοι στολάς, εν τόπω ανοικτώ κατά την είσοδον της πύλης της Σαμαρείας· και πάντες οι προφήται προεφήτευον έμπροσθεν αυτών.
\par 11 Και Σεδεκίας ο υιός του Χαναανά είχε κάμει εις εαυτόν σιδηρά κέρατα· και είπεν, Ούτω λέγει Κύριος· Διά τούτων θέλεις κερατίσει τους Συρίους, εωσού συντελέσης αυτούς.
\par 12 Και πάντες οι προφήται προεφήτευον ούτω, λέγοντες, Ανάβα εις Ραμώθ-γαλαάδ και ευοδού· διότι ο Κύριος θέλει παραδώσει αυτήν εις την χείρα του βασιλέως.
\par 13 Και ο μηνυτής, όστις υπήγε να καλέση τον Μιχαίαν, είπε προς αυτόν, λέγων, Ιδού τώρα, οι λόγοι των προφητών φανερόνουσιν εξ ενός στόματος καλόν περί του βασιλέως· ο λόγος σου λοιπόν ας ήναι ως ο λόγος ενός εξ εκείνων, και λάλησον το καλόν.
\par 14 Ο δε Μιχαίας είπε, Ζη Κύριος, ό,τι μοι είπη ο Κύριος, τούτο θέλω λαλήσει.
\par 15 Ήλθε λοιπόν προς τον βασιλέα. Και είπεν ο βασιλεύς προς αυτόν, Μιχαία, να υπάγωμεν εις Ραμώθ-γαλαάδ διά να πολεμήσωμεν, ή να απέχωμεν; Ο δε απεκρίθη προς αυτόν, Ανάβα και ευοδού· διότι ο Κύριος θέλει παραδώσει αυτήν εις την χείρα του βασιλέως.
\par 16 Και είπε προς αυτόν ο βασιλεύς, Έως ποσάκις θέλω σε ορκίζει, να μη λέγης προς εμέ παρά την αλήθειαν εν ονόματι Κυρίου;
\par 17 Ο δε είπεν, είδον πάντα τον Ισραήλ διεσπαρμένον επί τα όρη, ως πρόβατα μη έχοντα ποιμένα. Και είπε Κύριος, Ούτοι δεν έχουσι κύριον· ας επιστρέψωσιν έκαστος εις τον οίκον αυτού εν ειρήνη.
\par 18 Και είπεν ο βασιλεύς του Ισραήλ προς τον Ιωσαφάτ, Δεν σοι είπα έτι δεν θέλει προφητεύσει καλόν περί εμού, αλλά κακόν;
\par 19 Και ο Μιχαίας είπεν, Άκουσον λοιπόν τον λόγον του Κυρίου. Είδον τον Κύριον καθήμενον επί του θρόνου αυτού, και πάσαν την στρατιάν του ουρανού παρισταμένην περί αυτόν, εκ δεξιών αυτού και εξ αριστερών αυτού.
\par 20 Και είπε Κύριος, Τις θέλει απατήσει τον Αχαάβ, ώστε να αναβή και να πέση εν Ραμώθ-γαλαάδ; Και ο μεν είπεν ούτως, ο δε είπεν ούτως.
\par 21 Και εξήλθε το πνεύμα και εστάθη ενώπιον Κυρίου και είπεν, Εγώ θέλω απατήσει αυτόν.
\par 22 Και είπε Κύριος προς αυτό, Τίνι τρόπω; Και είπε, Θέλω εξέλθει και θέλω είσθαι πνεύμα ψεύδους εν τω στόματι πάντων των προφητών αυτού. Και είπε Κύριος, Θέλεις απατήσει και έτι θέλεις κατορθώσει· έξελθε και κάμε ούτω.
\par 23 Τώρα λοιπόν, ιδού, ο Κύριος έβαλε πνεύμα ψεύδους εν τω στόματι πάντων τούτων των προφητών σου, και ο Κύριος ελάλησε κακόν επί σε.
\par 24 Τότε πλησιάσας Σεδεκίας ο υιός του Χαναανά, ερράπισε τον Μιχαίαν επί την σιαγόνα και είπε, Διά ποίας οδού επέρασε το Πνεύμα του Κυρίου απ' εμού, διά να λαλήση προς σε;
\par 25 Και είπεν ο Μιχαίας, Ιδού, θέλεις ιδεί, καθ' ην ημέραν θέλεις εισέρχεσθαι από ταμείου εις ταμείον διά να κρυφθής.
\par 26 Και είπεν ο βασιλεύς του Ισραήλ, Πιάσατε τον Μιχαίαν και επαναφέρετε αυτόν προς Αμών τον άρχοντα της πόλεως και προς Ιωάς τον υιόν του βασιλέως·
\par 27 και είπατε, Ούτω λέγει ο βασιλεύς· Βάλετε τούτον εις την φυλακήν και τρέφετε αυτόν με άρτον θλίψεως και με ύδωρ θλίψεως, εωσού επιστρέψω εν ειρήνη.
\par 28 Και είπεν ο Μιχαίας, Εάν τωόντι επιστρέψης εν ειρήνη, ο Θεός δεν ελάλησε δι' εμού. Και είπεν, Ακούσατε σεις, πάντες οι λαοί.
\par 29 Και ανέβη ο βασιλεύς του Ισραήλ και Ιωσαφάτ ο βασιλεύς του Ιούδα εις Ραμώθ-γαλαάδ.
\par 30 Και είπεν ο βασιλεύς του Ισραήλ προς τον Ιωσαφάτ, Εγώ θέλω μετασχηματισθή και εισέλθει εις την μάχην· συ δε ενδύθητι την στολήν σου. Και μετεσχηματίσθη ο βασιλεύς του Ισραήλ και εισήλθεν εις την μάχην.
\par 31 Ο δε βασιλεύς της Συρίας είχε προστάξει τους τριάκοντα δύο αμαξάρχας αυτού, λέγων, Μη πολεμείτε μήτε μικρόν μήτε μέγαν, αλλά μόνον τον βασιλέα του Ισραήλ.
\par 32 Και ως είδον οι αμαξάρχαι τον Ιωσαφάτ, τότε αυτοί είπον, Βεβαίως ούτος είναι ο βασιλεύς του Ισραήλ. Και περιεστράφησαν διά να πολεμήσωσιν αυτόν· αλλ' ο Ιωσαφάτ ανεβόησεν.
\par 33 Ιδόντες δε οι αμαξάρχαι ότι δεν ήτο ο βασιλεύς του Ισραήλ, επέστρεψαν από της καταδιώξεως αυτού.
\par 34 Άνθρωπος δε τις, τοξεύσας ασκόπως, εκτύπησε τον βασιλέα του Ισραήλ μεταξύ των αρθρώσεων του θώρακος· ο δε είπε προς τον ηνίοχον αυτού, Στρέψον την χείρα σου και έκβαλέ με εκ του στρατεύματος· διότι επληγώθην.
\par 35 Και η μάχη εμεγαλύνθη εν τη ημέρα εκείνη· ο δε βασιλεύς ίστατο επί της αμάξης αντικρύ των Συρίων, και προς το εσπέρας απέθανε· και το αίμα έρρεεν εκ της πληγής εις τον κόλπον της αμάξης.
\par 36 Και περί την δύσιν του ηλίου έγεινε διακήρυξις εν τω στρατοπέδω, λέγουσα, Έκαστος εις την πόλιν αυτού και έκαστος εις τον τόπον αυτού.
\par 37 Και απέθανεν ο βασιλεύς και εκομίσθη εις Σαμάρειαν· και ενεταφίασαν τον βασιλέα εν Σαμαρεία.
\par 38 Και έπλυναν την άμαξαν εις το υδροστάσιον της Σαμαρείας· έπλυναν έτι και τα όπλα αυτού· και έγλειψαν οι κύνες το αίμα αυτού, κατά τον λόγον του Κυρίου, τον οποίον ελάλησεν.
\par 39 Αι δε λοιπαί των πράξεων του Αχαάβ και πάντα όσα έκαμε, και ο ελεφάντινος οίκος τον οποίον ωκοδόμησε και πάσαι αι πόλεις, τας οποίας έκτισε, δεν είναι γεγραμμένα εν τω βιβλίω των χρονικών των βασιλέων του Ισραήλ;
\par 40 Και εκοιμήθη ο Αχαάβ μετά των πατέρων αυτού, και εβασίλευσεν αντ' αυτού Οχοζίας ο υιός αυτού.
\par 41 Ο δε Ιωσαφάτ ο υιός του Ασά εβασίλευσεν επί τον Ιούδα, το τέταρτον έτος του Αχαάβ βασιλέως του Ισραήλ.
\par 42 Ο Ιωσαφάτ ήτο τριάκοντα πέντε ετών ηλικίας ότε εβασίλευσε· και εβασίλευσεν εικοσιπέντε έτη εν Ιερουσαλήμ· το δε όνομα της μητρός αυτού ήτο Αζουβά, θυγάτηρ του Σιλεΐ.
\par 43 Και περιεπάτησεν εις πάσας τας οδούς Ασά του πατρός αυτού· δεν εξέκλινεν απ' αυτών, πράττων το ευθές ενώπιον του Κυρίου. Οι υψηλοί όμως τόποι δεν αφηρέθησαν· ο λαός εθυσίαζεν έτι και εθυμίαζεν εν τοις υψηλοίς τόποις.
\par 44 Και είχεν ειρήνην ο Ιωσαφάτ μετά του βασιλέως του Ισραήλ.
\par 45 Αι δε λοιπαί των πράξεων του Ιωσαφάτ, και τα κατορθώματα αυτού όσα έκαμε, και οι πόλεμοι αυτού, δεν είναι γεγραμμένα εν τω βιβλίω των χρονικών των βασιλέων του Ιούδα;
\par 46 Και το υπόλοιπον των σοδομιτών, το εναπολειφθέν εν ταις ημέραις Ασά του πατρός αυτού, αυτός εξήλειψεν από της γης.
\par 47 Τότε δεν υπήρχε βασιλεύς εν Εδώμ· διοικητής ήτο βασιλεύς.
\par 48 Ο Ιωσαφάτ έκαμε πλοία εν Θαρσείς, διά να πλεύσωσιν εις Οφείρ διά χρυσίον· πλην δεν υπήγον, διότι τα πλοία συνετρίφθησαν εν Εσιών-γάβερ.
\par 49 Τότε είπεν Οχοζίας ο υιός του Αχαάβ προς τον Ιωσαφάτ, Ας υπάγωσιν οι δούλοί μου μετά των δούλων σου εις τα πλοία· ο Ιωσαφάτ όμως δεν ηθέλησε.
\par 50 Και εκοιμήθη ο Ιωσαφάτ μετά των πατέρων αυτού και ετάφη μετά των πατέρων αυτού εν τη πόλει Δαβίδ του πατρός αυτού· εβασίλευσε δε αντ' αυτού Ιωράμ ο υιός αυτού.
\par 51 Οχοζίας ο υιός του Αχαάβ εβασίλευσεν επί τον Ισραήλ εν Σαμαρεία, το δέκατον έβδομον έτος του Ιωσαφάτ βασιλέως του Ιούδα· και εβασίλευσε δύο έτη επί τον Ισραήλ.
\par 52 Και έπραξε τα πονηρά ενώπιον του Κυρίου, και περιεπάτησεν εις την οδόν του πατρός αυτού και εις την οδόν της μητρός αυτού και εις την οδόν του Ιεροβοάμ υιού του Ναβάτ, όστις έκαμε τον Ισραήλ να αμαρτήση·
\par 53 διότι ελάτρευσε τον Βάαλ και προσεκύνησεν αυτόν, και παρώργισε Κύριον τον Θεόν του Ισραήλ, κατά πάντα όσα έπραξεν ο πατήρ αυτού.


\end{document}