\begin{document}

\title{Nahum}


\chapter{1}

\par 1 Η κατά της Νινευή προφητεία· βιβλίον της οράσεως Ναούμ του Ελκοσαίου.
\par 2 Ζηλότυπος είναι ο Θεός και εκδικείται ο Κύριος· ο Κύριος εκδικείται και οργίζεται· ο Κύριος θέλει εκδικηθή τους εναντίους αυτού και φυλάττει οργήν κατά των εχθρών αυτού.
\par 3 Ο Κύριος είναι μακρόθυμος και μέγας την ισχύν, και ουδόλως θέλει αθωώσει τον ασεβή· η οδός του Κυρίου είναι μετά ανεμοστροβίλου και θυέλλης, και νεφέλαι ο κονιορτός των ποδών αυτού.
\par 4 Επιτιμά την θάλασσαν και ξηραίνει αυτήν και καταξηραίνει πάντας τους ποταμούς· μαραίνεται η Βασάν και ο Κάρμηλος και το άνθος του Λιβάνου μαραίνεται.
\par 5 Τα όρη σείονται απ' αυτού και οι λόφοι διαλύονται, η δε γη τρέμει από της παρουσίας αυτού, ναι, η οικουμένη, και πάντες οι κατοικούντες εν αυτή.
\par 6 Τις δύναται να ανθέξη ενώπιον της αγανακτήσεως αυτού; και τις δύναται να σταθή εις την έξαψιν της οργής αυτού; ο θυμός αυτού εκχέεται ως πυρ και οι βράχοι συντρίβονται έμπροσθεν αυτού.
\par 7 Ο Κύριος είναι αγαθός, οχύρωμα εν ημέρα θλίψεως, και γνωρίζει τους ελπίζοντας επ' αυτόν.
\par 8 Πλην με πλημμύραν κατακλύζουσαν θέλει κάμει συντέλειαν του τόπου αυτής, και σκότος θέλει καταδιώξει τους εχθρούς αυτού.
\par 9 Τι βουλεύεσθε κατά του Κυρίου; αυτός θέλει κάμει συντέλειαν· θλίψις δεν θέλει επέλθει εκ δευτέρου.
\par 10 Διότι ενώ συμπεριπλέκονται ως άκανθαι και μεθύουσιν ως μεθυσταί, θέλουσι καταναλωθή ως κατάξηρον άχυρον.
\par 11 Από σου εξήλθε διαλογιζόμενος πονηρά κατά του Κυρίου, σύμβουλος πονηρός.
\par 12 Ούτω λέγει Κύριος· Αν και ήναι εν τη ακμή αυτών και έτι πολλοί, θέλουσιν όμως κουρευθή, όταν αυτός διαβή· αν και σε κατέθλιψα, δεν θέλω σε καταθλίψει πλέον.
\par 13 Διότι τώρα θέλω συντρίψει τον ζυγόν αυτού από σου και θέλω διαρρήξει τους δεσμούς σου.
\par 14 Και ο Κύριος έδωκε προσταγήν περί σου, ότι δεν θέλει σπαρθή πλέον εκ του ονόματός σου· από του οίκου των θεών σου θέλω εκκόψει τα γλυπτά και τα χωνευτά· θέλω κάμει αυτόν τάφον σου, διότι είσαι βδελυκτός.
\par 15 Ιδού, επί των ορέων οι πόδες του ευαγγελιζομένου, του κηρύττοντος ειρήνην. Εόρταζε, Ιούδα, τας επισήμους εορτάς σου, απόδος τας ευχάς σου, διότι ο εξολοθρευτής δεν θέλει διαβή πλέον διά σού· ολοτελώς απεκόπη.

\chapter{2}

\par 1 Ο κατασυντρίβων ανέβη έμπροσθεν του προσώπου σου· φύλαττε το οχύρωμα, σκόπευσον την οδόν, ενίσχυσον τας οσφύς, ενδυνάμωσον την ισχύν σου σφόδρα.
\par 2 Επειδή ο Κύριος απέστρεψε την δόξαν του Ιακώβ καθώς την δόξαν του Ισραήλ· διότι οι τινακταί εξετίναξαν αυτούς και διέφθειραν τα κλήματα αυτών.
\par 3 Η ασπίς των ισχυρών αυτού είναι κοκκινοβαφής, οι άνδρες δυνάμεως ενδεδυμένοι ερυθρά· αι άμαξαι θέλουσι κινείσθαι με σίδηρον λάμποντα εν τη ημέρα της ετοιμασίας αυτού, και τα ελάτινα δόρατα θέλουσι σεισθή τρομερά.
\par 4 Αι άμαξαι θέλουσι θορυβείσθαι εν ταις οδοίς, θέλουσι συγκρούεσθαι η μία μετά της άλλης εν ταις πλατείαις· η θέα αυτών θέλει είσθαι ως λαμπάδες, θέλουσι τρέχει ως αστραπαί.
\par 5 Θέλει ενθυμηθή τους ανδρείους αυτού· αλλά θέλουσι κατολισθήσει εν τη οδώ αυτών· θέλουσι σπεύσει εις τα τείχη αυτής και ο συνασπισμός θέλει ετοιμασθή.
\par 6 Αι πύλαι των ποταμών θέλουσιν ανοιχθή και τα παλάτια θέλουσι διαλυθή.
\par 7 Και η καθεστώσα θέλει γυμνωθή, θέλει μετοικισθή, και αι δούλαι αυτής θέλουσιν αναδίδει στεναγμούς ως η φωνή των περιστερών, τύπτουσαι τα στήθη αυτών.
\par 8 Και η Νινευή είναι παλαιόθεν ως λίμνη υδάτων· ταύτα όμως θέλουσι φύγει. Στήτε, στήτε, θέλουσι φωνάζει· και ουδείς ο βλέπων οπίσω.
\par 9 Λαφυραγωγείτε το αργύριον, λαφυραγωγείτε το χρυσίον· διότι δεν είναι τέλος των θησαυρών αυτής· είναι πλήθος παντός σκεύους επιθυμητού.
\par 10 Εξεκενώθη και εξετινάχθη και ηρημώθη και η καρδία διαλύεται και τα γόνατα κλονίζονται και ωδίνες είναι εις πάσας τας οσφύς, τα δε πρόσωπα πάντων είναι απησβολωμένα.
\par 11 Που είναι το κατοικητήριον των λεόντων και η βοσκή των σκύμνων, όπου ο λέων, ο γηραιός λέων, περιεπάτει και ο σκύμνος του λέοντος, και δεν υπήρχεν ο εκφοβών;
\par 12 Ο λέων διεσπάραττεν ικανά διά τους σκύμνους αυτού και απέπνιγε διά τας λεαίνας αυτού, και εγέμιζε τα σπήλαια αυτού από θηράματος και τα κατοικητήρια αυτού από αρπαγής.
\par 13 Ιδού, εγώ είμαι εναντίον σου, λέγει ο Κύριος των δυνάμεων, και θέλω καύσει τας αμάξας σου μέχρι καπνού και η ρομφαία θέλει καταφάγει τους σκύμνους σου, και θέλω εξολοθρεύσει το θήραμά σου εκ της γης, και δεν θέλει ακουσθή πλέον η φωνή των πρέσβεών σου.

\chapter{3}

\par 1 Ουαί εις την πόλιν των αιμάτων· όλη είναι πλήρης ψεύδους και αρπαγής· το θήραμα δεν απολείπει.
\par 2 Φωνή μαστίγων ακούεται και φωνή θορύβου τροχών και ίππων ορμώντων και αρμάτων αναπηδώντων,
\par 3 ιππέως αναβαίνοντος και ρομφαίας στιλβούσης και λόγχης εξαστραπτούσης, και πλήθος τραυματιζομένων και μέγας αριθμός πτωμάτων, και δεν είναι τέλος των πτωμάτων· προσκόπτουσιν εις τα πτώματα αυτών·
\par 4 από του πλήθους των πορνειών της θελκτικής πόρνης, της εμπείρου εις γοητείας, ήτις πωλεί έθνη διά των πορνειών αυτής και φυλάς διά των γοητειών αυτής.
\par 5 Ιδού, εγώ είμαι εναντίον σου, λέγει ο Κύριος των δυνάμεων· και θέλω ανασηκώσει τα κράσπεδά σου επί το πρόσωπόν σου, και θέλω δείξει εις τα έθνη την αισχύνην σου και εις τα βασίλεια την ατιμίαν σου.
\par 6 Και θέλω ρίψει βδελυράν ακαθαρσίαν επί σε και θέλω σε καταισχύνει και θέλω σε καταστήσει εις θέαμα.
\par 7 Και πάντες οι βλέποντές σε θέλουσι φεύγει από σου και θέλουσι λέγει, Η Νινευή ηρημώθη· τις θέλει συλλυπηθή αυτήν; πόθεν θέλω ζητήσει παρηγορητάς διά σε;
\par 8 είσαι καλητέρα της Νω Αμμών, της κειμένης μεταξύ των ποταμών, της περικυκλουμένης από υδάτων, της οποίας προμαχών ήτο η θάλασσα και τείχος αυτής το πέλαγος;
\par 9 Η Αιθιοπία ήτο η ισχύς αυτής και η Αίγυπτος και άλλοι απέραντοι· η Φούθ και οι Λίβυες ήσαν οι βοηθοί σου.
\par 10 Αλλά και αυτή μετωκίσθη, υπήγεν εις αιχμαλωσίαν, τα δε νήπια αυτής συνετρίφθησαν επί των άκρων πασών των οδών· και έρριψαν κλήρους επί τους ενδόξους αυτής άνδρας, και πάντες οι μεγιστάνες αυτής εδέθησαν με αλύσεις.
\par 11 Και συ θέλεις μεθυσθή, θέλεις μένει αφανής· και συ θέλεις ζητήσει δύναμιν εναντίον του εχθρού.
\par 12 Πάντα τα οχυρώματά σου θέλουσιν είσθαι ως συκαί με τα πρωτοφανή σύκα αυτών· εάν σεισθώσι, θέλουσι βεβαίως πέσει εις το στόμα του τρώγοντος.
\par 13 Ιδού, ο λαός σου είναι γυναίκες εν μέσω σου· αι πύλαι της γης σου θέλουσιν είσθαι όλως ανεωγμέναι εις τους εχθρούς σου· το πυρ θέλει καταφάγει τους μοχλούς σου.
\par 14 Ανάσυρον εις σεαυτόν ύδωρ διά την πολιορκίαν, ενδυνάμωσον τα οχυρώματά σου· είσελθε εις τον πηλόν και πάτησον την άργιλλον, επισκεύασον την κεραμικήν κάμινον·
\par 15 εκεί θέλει σε καταφάγει το πύρ· η ρομφαία θέλει σε εξολοθρεύσει, θέλει σε καταφάγει ως βρούχος· πληθύνου ως βρούχος, πληθύνου ως ακρίς.
\par 16 Επλήθυνας τους εμπόρους σου υπέρ τα άστρα του ουρανού· ο βρούχος εξηπλώθη και εξεπέταξεν.
\par 17 Οι μεγιστάνές σου είναι ως ακρίδες και οι σατράπαι σου ως μεγάλαι ακρίδες, αίτινες επικάθηνται επί τους φραγμούς εν ημέρα ψύχους· αλλ' όταν ο ήλιος ανατείλη, φεύγουσι και ο τόπος αυτών δεν γνωρίζεται που ήσαν.
\par 18 Οι ποιμένες σου ενύσταξαν, βασιλεύ της Ασσυρίας· οι δυνατοί σου απεκοιμήθησαν· ο λαός σου εσκορπίσθη επί τα όρη και δεν υπάρχει ο συνάγων.
\par 19 Δεν είναι ίασις εις το σύντριμμά σου· η πληγή σου είναι χαλεπή· πάντες οι ακούοντες την αγγελίαν σου θέλουσι κροτήσει χείρας επί σέ· διότι επί τίνα δεν επήλθε πάντοτε η κακία σου;


\end{document}