\begin{document}

\title{Εκκλησιαστής}


\chapter{1}

\par 1 Λόγοι του Εκκλησιαστού, υιού του Δαβίδ, βασιλέως εν Ιερουσαλήμ.
\par 2 Ματαιότης ματαιοτήτων, είπεν ο Εκκλησιαστής· ματαιότης ματαιοτήτων, τα πάντα ματαιότης.
\par 3 Τις ωφέλεια εις τον άνθρωπον εκ παντός του μόχθου αυτού, τον οποίον μοχθεί υπό τον ήλιον;
\par 4 Γενεά υπάγει, και γενεά έρχεται· η δε γη διαμένει εις τον αιώνα.
\par 5 Και ανατέλλει ο ήλιος, και δύει ο ήλιος· και σπεύδει προς τον τόπον αυτού, όθεν ανέτειλεν.
\par 6 Υπάγει προς τον νότον ο άνεμος, και επιστρέφει προς τον βορράν· ακαταπαύστως περιστρεφόμενος υπάγει, και επανέρχεται επί τους κύκλους αυτού, ο άνεμος.
\par 7 Πάντες οι ποταμοί υπάγουσιν εις την θάλασσαν, και η θάλασσα ποτέ δεν γεμίζει· εις τον τόπον όθεν ρέουσιν οι ποταμοί, εκεί πάλιν επιστρέφουσι, διά να υπάγωσι.
\par 8 Πάντα τα πράγματα είναι εν κόπω· δεν δύναται άνθρωπος να εκφράση τούτο· ο οφθαλμός δεν χορταίνει βλέπων, και το ωτίον δεν γεμίζει ακούον.
\par 9 ό,τι έγεινε, τούτο πάλιν θέλει γείνει· και ό,τι συνέβη, τούτο πάλιν θέλει συμβή· και δεν είναι ουδέν νέον υπό τον ήλιον.
\par 10 Υπάρχει πράγμα, περί του οποίου δύναταί τις να είπη, Ιδέ, τούτο είναι νέον; τούτο έγεινεν ήδη εις τους αιώνας οίτινες υπήρξαν προ ημών.
\par 11 Δεν είναι μνήμη των προγεγονότων, ουδέ θέλει είσθαι μνήμη των επιγενησομένων μετά ταύτα, εις τους μέλλοντας να υπάρξωσιν έπειτα.
\par 12 Εγώ ο Εκκλησιαστής εστάθην βασιλεύς επί τον Ισραήλ εν Ιερουσαλήμ·
\par 13 και έδωκα την καρδίαν μου εις το να εκζητήσω και να ερευνήσω διά της σοφίας περί πάντων των γινομένων υπό τον ουρανόν· τον οχληρόν τούτον περισπασμόν ο Θεός έδωκεν εις τους υιούς των ανθρώπων, διά να μοχθώσιν εν αυτώ.
\par 14 Είδον πάντα τα έργα τα γινόμενα υπό τον ήλιον, και ιδού, τα πάντα ματαιότης και θλίψις πνεύματος.
\par 15 Το στρεβλόν δεν δύναται να γείνη ευθές, και αι ελλείψεις δεν δύνανται να αριθμηθώσιν.
\par 16 Εγώ ελάλησα εν τη καρδία μου λέγων, Ιδού, εγώ εμεγαλύνθην και ηυξήνθην εις σοφίαν υπέρ πάντας τους υπάρξαντας προ εμού εν Ιερουσαλήμ, και η καρδία μου απήλαυσε πολλήν σοφίαν και γνώσιν.
\par 17 Και έδωκα την καρδίαν μου εις το να γνωρίση σοφίαν και εις το να γνωρίση ανοησίαν και αφροσύνην· πλην εγνώρισα ότι και τούτο είναι θλίψις πνεύματος.
\par 18 Διότι εν πολλή σοφία είναι πολλή λύπη· και όστις προσθέτει γνώσιν, προσθέτει πόνον.

\chapter{2}

\par 1 Εγώ είπα εν τη καρδία μου, Ελθέ τώρα, να σε δοκιμάσω δι' ευφροσύνης· και εντρύφα εις αγαθά· και ιδού, και τούτο ματαιότης.
\par 2 Είπα περί του γέλωτος, Είναι μωρία· και περί της χαράς, Τι ωφελεί αύτη;
\par 3 Εσκέφθην εν τη καρδία μου να ευφραίνω την σάρκα μου με οίνον, ενώ έτι η καρδία μου ησχολείτο εις την σοφίαν· και να κρατήσω την μωρίαν, εωσού ίδω τι είναι το αγαθόν εις τους υιούς των ανθρώπων, διά να κάμνωσιν αυτό υπό τον ουρανόν πάσας τας ημέρας της ζωής αυτών.
\par 4 Έκαμον πράγματα μεγάλα εις εμαυτόν· ωκοδόμησα εις εμαυτόν οικίας· εφύτευσα δι' εμαυτόν αμπελώνας.
\par 5 Έκαμον δι' εμαυτόν κήπους και παραδείσους και εφύτευσα εν αυτοίς δένδρα παντός καρπού.
\par 6 Έκαμον δι' εμαυτόν δεξαμενάς υδάτων, διά να ποτίζω εξ αυτών το άλσος το κατάφυτον εκ δένδρων.
\par 7 Απέκτησα δούλους και δούλας και είχον δούλους οικογενείς· απέκτησα έτι αγέλας και ποίμνια περισσότερα υπέρ πάντας τους υπάρξαντας προ εμού εν Ιερουσαλήμ.
\par 8 Συνήθροισα εις εμαυτόν και αργύριον και χρυσίον και εκλεκτά κειμήλια βασιλέων και τόπων· απέκτησα εις εμαυτόν άδοντας και αδούσας και τα εντρυφήματα των υιών των ανθρώπων, παν είδος παλλακίδων.
\par 9 Και εμεγαλύνθην και ηυξήνθην υπέρ πάντας τους υπάρξαντας προ εμού εν Ιερουσαλήμ· και η σοφία μου έμενεν εν εμοί.
\par 10 Και παν ό,τι εζήτησαν οι οφθαλμοί μου, δεν ηρνήθην εις αυτούς· δεν εμπόδισα την καρδίαν μου από πάσης ευφροσύνης, διότι η καρδία μου ευφραίνετο εις πάντας τους μόχθους μου· και τούτο ήτο η μερίς μου εκ παντός του μόχθου μου.
\par 11 Και παρετήρησα εγώ εν πάσι τοις έργοις μου τα οποία έκαμον αι χείρές μου, και εν παντί τω μόχθω τον οποίον εμόχθησα, και ιδού, τα πάντα ματαιότης και θλίψις πνεύματος, και ουδέν όφελος υπό τον ήλιον.
\par 12 Και εστράφην εγώ διά να παρατηρήσω την σοφίαν και την μωρίαν και την αφροσύνην· διότι τι θέλει κάμει άνθρωπος ελθών μετά τον βασιλέα; ό,τι έκαμον ήδη.
\par 13 Και εγώ είδον ότι η σοφία υπερέχει της αφροσύνης, καθώς το φως υπερέχει του σκότους.
\par 14 Του σοφού οι οφθαλμοί είναι εν τη κεφαλή αυτού, ο δε άφρων περιπατεί εν τω σκότει· πλην εγώ εγνώρισα έτι ότι εν συνάντημα θέλει συναντήσει εις πάντας τούτους.
\par 15 Διά τούτο είπα εγώ εν τη καρδία μου, Καθώς συμβαίνει εις τον άφρονα, ούτω θέλει συμβή και εις εμέ· διά τι λοιπόν εγώ να γείνω σοφώτερος; όθεν εσυμπέρανα πάλιν εν τη καρδία μου, ότι και τούτο είναι ματαιότης.
\par 16 Διότι δεν θέλει μένει διαπαντός η μνήμη του σοφού ουδέ του άφρονος· επειδή εν ταις επερχομέναις ημέραις τα πάντα θέλουσι πλέον λησμονηθή. Και πως θέλει αποθάνει ο σοφός μετά του άφρονος;
\par 17 Διά τούτο εμίσησα την ζωήν, διότι μοχθηρά εφάνησαν εις εμέ τα έργα τα γενόμενα υπό τον ήλιον· επειδή τα πάντα ματαιότης και θλίψις πνεύματος.
\par 18 Εμίσησα έτι εγώ πάντα τον μόχθον μου, τον οποίον είχον μοχθήσει υπό τον ήλιον· διότι αφίνω αυτόν εις τον άνθρωπον όστις θέλει σταθή μετ' εμέ.
\par 19 Και τις οίδεν αν θέλη είσθαι σοφός η άφρων; και όμως θέλει εξουσιάσει επί παντός του μόχθου μου, τον οποίον εμόχθησα και εις τον οποίον έδειξα την σοφίαν μου υπό τον ήλιον· ματαιότης και τούτο.
\par 20 Όθεν εγώ στραφείς απήλπισα την καρδίαν μου περί παντός του μόχθου, τον οποίον εμόχθησα υπό τον ήλιον.
\par 21 Διότι είναι άνθρωπος, του οποίου ο μόχθος εστάθη εν σοφία και γνώσει και εν ορθότητι· και όμως αφίνει αυτόν εις άλλον διά μερίδα αυτού, όστις δεν εκοπίασεν εις αυτόν· και τούτο ματαιότης και κακόν μέγα.
\par 22 Διότι τις ωφέλεια εις τον άνθρωπον από παντός του μόχθου αυτού και από της θλίψεως της καρδίας αυτού, εις τα οποία μοχθεί υπό τον ήλιον;
\par 23 Επειδή πάσαι αι ημέραι αυτού είναι πόνος, και οι μόχθοι αυτού λύπη· και την νύκτα έτι η καρδία αυτού δεν κοιμάται· είναι και τούτο ματαιότης.
\par 24 Δεν είναι αγαθόν εις τον άνθρωπον να τρώγη και να πίνη και να κάμνη την ψυχήν αυτού να απολαμβάνη καλόν εκ του μόχθου αυτού; και τούτο είδον εγώ, ότι είναι από της χειρός του Θεού.
\par 25 Διότι τις θέλει φάγει και τις θέλει εντρυφήσει υπέρ εμέ;
\par 26 Επειδή ο Θεός εις τον άνθρωπον τον αρεστόν ενώπιον αυτού δίδει σοφίαν και γνώσιν και χαράν· εις δε τον αμαρτωλόν δίδει περισπασμόν, εις το να προσθέτη και να επισωρεύη, διά να δώση αυτά εις τον αρεστόν ενώπιον αυτού· και τούτο ματαιότης και θλίψις πνεύματος.

\chapter{3}

\par 1 Χρόνος είναι εις πάντα, και καιρός παντί πράγματι υπό τον ουρανόν.
\par 2 Καιρός του γεννάσθαι και καιρός του αποθνήσκειν· καιρός του φυτεύειν και καιρός του εκριζόνειν το πεφυτευμένον·
\par 3 καιρός του αποκτείνειν και καιρός του ιατρεύειν· καιρός του καταστρέφειν και καιρός του οικοδομείν·
\par 4 καιρός του κλαίειν και καιρός του γελάν· καιρός του πενθείν και καιρός του χορεύειν·
\par 5 καιρός του διασκορπίζειν λίθους και καιρός του συνάγειν λίθους· καιρός του εναγκαλίζεσθαι και καιρός του απομακρύνεσθαι από του εναγκαλισμού·
\par 6 καιρός του αποκτήσαι και καιρός του απολέσαι· καιρός του φυλάττειν και καιρός του ρίπτειν·
\par 7 καιρός του σχίζειν και καιρός του ράπτειν· καιρός του σιγάν και καιρός του λαλείν·
\par 8 καιρός του αγαπήσαι και καιρός του μισήσαι· καιρός πολέμου και καιρός ειρήνης.
\par 9 Τις ωφέλεια εις τον εργαζόμενον από όσα αυτός μοχθεί;
\par 10 Είδον τον περισπασμόν, τον οποίον έδωκεν ο Θεός εις τους υιούς των ανθρώπων διά να μοχθώσιν εν αυτώ.
\par 11 Τα πάντα έκαμε καλά εν τω καιρώ εκάστου· και τον κόσμον υπέβαλεν εις την διάνοιαν αυτών, χωρίς ο άνθρωπος να δύναται να εξιχνιάση απ' αρχής μέχρι τέλους το έργον, το οποίον ο Θεός έκαμεν.
\par 12 Εγνώρισα ότι δεν είναι άλλο καλόν δι' αυτούς, ειμή να ευφραίνηταί τις και να κάμνη καλόν εν τη ζωή αυτού.
\par 13 Και έτι το να τρώγη πας άνθρωπος και να πίνη και να απολαμβάνη καλόν εκ παντός του μόχθου αυτού, είναι χάρισμα Θεού.
\par 14 Εγνώρισα ότι πάντα όσα έκαμεν ο Θεός, τα αυτά θέλουσιν είσθαι διαπαντός· δεν είναι δυνατόν να προσθέση τις εις αυτά ουδέ να αφαιρέση απ' αυτών· και ο Θεός έκαμε τούτο διά να φοβώνται ενώπιον αυτού.
\par 15 ό,τι έγεινεν, ήδη είναι· και ό,τι θέλει γείνει, ήδη έγεινε· και ο Θεός ανακαλεί τα παρελθόντα.
\par 16 Και είδον έτι υπό τον ήλιον τον τόπον της κρίσεως, και εκεί είναι η ανομία· και τον τόπον της δικαιοσύνης, και εκεί η ανομία.
\par 17 Είπα εγώ εν τη καρδία μου, Ο Θεός θέλει κρίνει τον δίκαιον και τον ασεβή· διότι δι' έκαστον πράγμα και επί παντός έργου είναι καιρός εκεί.
\par 18 Είπα εγώ εν τη καρδία μου περί της καταστάσεως των υιών των ανθρώπων, ότι θέλει δοκιμάσει αυτούς ο Θεός, και θέλουσιν ιδεί ότι αυτοί καθ' εαυτούς είναι κτήνη.
\par 19 Διότι το συνάντημα των υιών των ανθρώπων είναι και το συνάντημα του κτήνους· και εν συνάντημα είναι εις αυτούς· καθώς αποθνήσκει τούτο, ούτως αποθνήσκει και εκείνος· και η αυτή πνοή είναι εις πάντας· και ο άνθρωπος δεν υπερτερεί κατ' ουδέν το κτήνος· διότι τα πάντα είναι ματαιότης.
\par 20 Τα πάντα καταντώσιν εις τον αυτόν τόπον· τα πάντα έγειναν εκ του χώματος και τα πάντα επιστρέφουσιν εις το χώμα.
\par 21 Τις γνωρίζει το πνεύμα των υιών των ανθρώπων, αν αυτό αναβαίνη εις τα άνω, και το πνεύμα του κτήνους, αν αυτό καταβαίνη κάτω εις την γην;
\par 22 Είδον λοιπόν ότι δεν είναι καλήτερον, ειμή το να ευφραίνηται ο άνθρωπος εις τα έργα αυτού· διότι αυτή είναι η μερίς αυτού· επειδή τις θέλει φέρει αυτόν διά να ίδη το γενησόμενον μετ' αυτόν;

\chapter{4}

\par 1 Τότε εγώ εστράφην και είδον πάσας τας αδικίας τας γινομένας υπό τον ήλιον· και ιδού, δάκρυα των αδικουμένων, και δεν υπήρχεν εις αυτούς ο παρηγορών· η δε δύναμις ήτο εν τη χειρί των αδικούντων αυτούς· και δεν υπήρχεν εις αυτούς ο παρηγορών.
\par 2 Όθεν εγώ εμακάρισα τους τελευτήσαντας, τους ήδη αποθανόντας, μάλλον παρά τους ζώντας, όσοι ζώσιν έτι.
\par 3 Καλήτερος δε αμφοτέρων είναι, όστις δεν υπήρξεν έτι, όστις δεν είδε τα πονηρά έργα τα γινόμενα υπό τον ήλιον.
\par 4 Προσέτι εγώ εθεώρησα πάντα μόχθον και πάσαν επίτευξιν έργου, ότι διά τούτο ο άνθρωπος φθονείται υπό του πλησίον αυτού· και τούτο ματαιότης και θλίψις πνεύματος.
\par 5 Ο άφρων περιπλέκει τας χείρας αυτού και τρώγει την εαυτού σάρκα.
\par 6 Καλήτερον μία δραξ πλήρης αναπαύσεως παρά δύο πλήρεις μόχθου και θλίψεως πνεύματος.
\par 7 Πάλιν εστράφην εγώ και είδον ματαιότητα υπό τον ήλιον·
\par 8 υπάρχει τις και δεν έχει δεύτερον· ναι, δεν έχει ούτε υιόν ούτε αδελφόν· και όμως δεν παύει από παντός του μόχθου αυτού· μάλιστα ο οφθαλμός αυτού δεν χορταίνει πλούτου· και δεν λέγει, διά τίνα εγώ κοπιάζω και στερώ την ψυχήν μου από αγαθών; και τούτο είναι ματαιότης και περισπασμός λυπηρός.
\par 9 Καλήτεροι οι δύο υπέρ τον ένα· επειδή αυτοί έχουσι καλήν αντιμισθίαν εν τω κόπω αυτών.
\par 10 Διότι, εάν πέσωσιν, ο εις θέλει σηκώσει τον σύντροφον αυτού· αλλ' ουαί εις τον ένα, όστις πέση και δεν έχη δεύτερον να σηκώση αυτόν.
\par 11 Πάλιν, εάν δύο πλαγιάσωσιν ομού, τότε θερμαίνονται· ο εις όμως πως θέλει θερμανθή;
\par 12 Και εάν τις υπερισχύση κατά του ενός, οι δύο θέλουσιν αντισταθή εις αυτόν· και το τριπλούν σχοινίον δεν κόπτεται ταχέως.
\par 13 Καλήτερον πτωχόν και σοφόν παιδίον παρά βασιλεύς γέρων και άφρων, όστις δεν είναι πλέον επιδεκτικός νουθεσίας·
\par 14 διότι το μεν εξέρχεται εκ του οίκου των δεσμίων διά να βασιλεύση· ο δε και βασιλεύς γεννηθείς καθίσταται πένης.
\par 15 Είδον πάντας τους ζώντας τους περιπατούντας υπό τον ήλιον, μετά του υιού, του δευτέρου, όστις θέλει σταθή αντ' αυτού.
\par 16 Δεν υπάρχει τέλος εις πάντα τον λαόν, εις πάντας τους προϋπάρξαντας αυτών· αλλ' ουδέ οι μετά ταύτα θέλουσιν ευφρανθή εις αυτόν· λοιπόν και τούτο ματαιότης και θλίψις πνεύματος.

\chapter{5}

\par 1 Φύλαττε τον πόδα σου, όταν υπάγης εις τον οίκον του Θεού· και προθυμού μάλλον να ακούης, παρά να προσφέρης θυσίαν αφρόνων, οίτινες δεν αισθάνονται ότι πράττουσι κακώς.
\par 2 Μη σπεύδε διά του στόματός σου, και η καρδία σου ας μη επιταχύνη να προφέρη λόγον ενώπιον του Θεού· διότι ο Θεός είναι εν τω ουρανώ, συ δε επί της γής· όθεν οι λόγοι σου ας ήναι ολίγοι.
\par 3 Επειδή το μεν όνειρον έρχεται εν τω πλήθει των περισπασμών· η δε φωνή του άφρονος εν τω πλήθει των λόγων.
\par 4 Όταν ευχηθής ευχήν εις τον Θεόν, μη βραδύνης να αποδώσης αυτήν· διότι δεν ευαρεστείται εις τους άφρονας· απόδος ό,τι ηυχήθης.
\par 5 Κάλλιον να μη ευχηθής, παρά ευχηθείς να μη αποδώσης.
\par 6 Μη συγχωρήσης εις το στόμα σου να φέρη επί σε αμαρτίαν· μηδέ είπης ενώπιον του αγγέλου, ότι ήτο εξ αγνοίας· διά τι να οργισθή ο Θεός εις την φωνήν σου και να αφανίση τα έργα των χειρών σου;
\par 7 Διότι εν τω πλήθει των ονείρων και εν τω πλήθει των λόγων είναι ματαιότητες· συ δε φοβού τον Θεόν.
\par 8 Εάν ίδης κατάθλιψιν πένητος και παραβίασιν κρίσεως και δικαιοσύνης εν τη χώρα, μη θαυμάσης διά τούτο· διότι ο επί τον υψηλόν υψηλότερος επιτηρεί· και επί τούτους υψηλότεροι.
\par 9 Η γη ωφελεί υπέρ πάντα· και αυτός ο βασιλεύς υπό των αγρών υπηρετείται.
\par 10 Ο αγαπών το αργύριον δεν θέλει χορτασθή αργυρίου· ουδέ εισοδημάτων ο αγαπών την αφθονίαν· ματαιότης και τούτο.
\par 11 Πληθυνομένων των αγαθών πληθύνονται και οι τρώγοντες αυτά· και τις η ωφέλεια εις τους κυρίους αυτών, ειμή το να θεωρώσιν αυτά διά των οφθαλμών αυτών;
\par 12 Ο ύπνος του εργαζομένου είναι γλυκύς, είτε ολίγον φάγη, είτε πολύ· ο δε του πλουσίου χορτασμός δεν αφίνει αυτόν να κοιμάται.
\par 13 Υπάρχει κακόν θλιβερόν, το οποίον είδον υπό τον ήλιον· πλούτος φυλαττόμενος υπό του έχοντος αυτόν προς βλάβην αυτού.
\par 14 Και ο πλούτος εκείνος χάνεται υπό συμφοράς κακής· αυτός δε γεννά υιόν και δεν έχει ουδέν εν τη χειρί αυτού.
\par 15 Καθώς εξήλθεν εκ της κοιλίας της μητρός αυτού, γυμνός θέλει επιστρέψει, υπάγων καθώς ήλθε· και δεν θέλει βαστάζει ουδέν εκ του κόπου αυτού, διά να έχη εν τη χειρί αυτού.
\par 16 Και τούτο έτι κακόν θλιβερόν, καθώς ήλθεν, ούτω να υπάγη· και τις ωφέλεια εις αυτόν ότι εκοπίασε διά τον άνεμον;
\par 17 Θέλει προσέτι τρώγει κατά πάσας αυτού τας ημέρας εν σκότει και εν πολλή λύπη και αρρωστία και βασάνω.
\par 18 Ιδού, τι είδον εγώ αγαθόν· είναι καλόν να τρώγη τις και να πίνη και να απολαμβάνη τα αγαθά όλου του κόπου αυτού, τον οποίον κοπιάζει υπό τον ήλιον, κατά τον αριθμόν των ημερών της ζωής αυτού, όσας έδωκεν ο Θεός εις αυτόν· διότι τούτο είναι η μερίς αυτού.
\par 19 Και εις όντινα άνθρωπον ο Θεός δόσας πλούτη και υπάρχοντα, έδωκεν εις αυτόν και εξουσίαν να τρώγη απ' αυτών και να λαμβάνη το μερίδιον αυτού και να ευφραίνεται εις τον κόπον αυτού, τούτο είναι δώρον Θεού·
\par 20 διότι δεν θέλει ενθυμείσθαι πολύ τας ημέρας της ζωής αυτού· επειδή ο Θεός αποκρίνεται εις την καρδίαν αυτού δι' ευφροσύνης.

\chapter{6}

\par 1 Υπάρχει κακόν, το οποίον είδον υπό τον ήλιον, και τούτο συχνόν μεταξύ των ανθρώπων·
\par 2 Άνθρωπος, εις τον οποίον ο Θεός έδωκε πλούτον και υπάρχοντα και δόξαν, ώστε δεν στερείται η ψυχή αυτού από πάντων όσα ήθελεν επιθυμήσει· πλην ο Θεός δεν έδωκεν εις αυτόν εξουσίαν να τρώγη εξ αυτών, αλλά τρώγει αυτά ξένος· και τούτο ματαιότης και είναι νόσος κακή.
\par 3 Εάν άνθρωπος γεννήση εκατόν τέκνα και ζήση πολλά έτη, ώστε αι ημέραι των ετών αυτού να γείνωσι πολλαί, και η ψυχή αυτού δεν χορταίνη αγαθού και δεν λάβη και ταφήν, λέγω ότι το εξάμβλωμα είναι καλήτερον παρ' αυτόν.
\par 4 Διότι ήλθεν εν ματαιότητι και θέλει υπάγει εν σκότει, και το όνομα αυτού θέλει σκεπασθή υπό σκότους·
\par 5 δεν είδεν, ουδέ εγνώρισε τον ήλιον, έχει όμως περισσοτέραν ανάπαυσιν παρ' εκείνον,
\par 6 και δισχίλια έτη αν ζήση και καλόν δεν ίδη· δεν υπάγουσι πάντες εις τον αυτόν τόπον;
\par 7 Πας ο μόχθος του ανθρώπου είναι διά το στόμα αυτού; και όμως η ψυχή δεν χορταίνει.
\par 8 Διότι κατά τι υπερβαίνει ο σοφός τον άφρονα; κατά τι ο πτωχός, αν και εξεύρη να περιπατή έμπροσθεν των ζώντων;
\par 9 Κάλλιον είναι να βλέπη τις διά των οφθαλμών, παρά να περιπλανάται με την ψυχήν· και τούτο ματαιότης και θλίψις πνεύματος.
\par 10 ό,τι έγεινεν, έλαβεν ήδη το όνομα αυτού, και εγνωρίσθη ότι ούτος είναι άνθρωπος· και δεν δύναται να κριθή μετά του ισχυροτέρου αυτού·
\par 11 Επειδή είναι πολλά πράγματα πληθύνοντα την ματαιότητα, τις ωφέλεια εις τον άνθρωπον;
\par 12 Διότι τις γνωρίζει τι είναι καλόν διά τον άνθρωπον εν τη ζωή, κατά πάσας τας ημέρας της ζωής της ματαιότητος αυτού, τας οποίας διέρχεται ως σκιάν; διότι τις θέλει απαγγείλει προς τον άνθρωπον, τι θέλει είσθαι μετ' αυτόν υπό τον ήλιον;

\chapter{7}

\par 1 Κάλλιον όνομα καλόν παρά πολύτιμον μύρον· και η ημέρα του θανάτου παρά την ημέραν της γεννήσεως.
\par 2 Κάλλιον να υπάγη τις εις οίκον πένθους, παρά να υπάγη εις οίκον συμποσίου· διότι τούτο είναι το τέλος παντός ανθρώπου, και ο ζων θέλει βάλει αυτό εις την καρδίαν αυτού.
\par 3 Κάλλιον η λύπη παρά τον γέλωτα· διότι εκ της σκυθρωπότητος του προσώπου η καρδία γίνεται φαιδροτέρα.
\par 4 Η καρδία των σοφών είναι εν οίκω πένθους· αλλ' η καρδία των αφρόνων εν οίκω ευφροσύνης.
\par 5 Κάλλιον εις τον άνθρωπον να ακούη επίπληξιν σοφού, παρά να ακούη άσμα αφρόνων·
\par 6 διότι καθώς είναι ο ήχος των ακανθών υποκάτω του λέβητος, ούτως ο γέλως του άφρονος και τούτο ματαιότης.
\par 7 Βεβαίως η καταδυναστεία παραλογίζει τον σοφόν· και το δώρον διαφθείρει την καρδίαν.
\par 8 Κάλλιον το τέλος του πράγματος παρά την αρχήν αυτού· καλήτερος ο μακρόθυμος παρά τον υψηλόφρονα.
\par 9 Μη σπεύδε εν τω πνεύματί σου να θυμόνης· διότι ο θυμός αναπαύεται εν τω κόλπω των αφρόνων.
\par 10 Μη είπης, Τις η αιτία, διά την οποίαν αι παρελθούσαι ημέραι ήσαν καλήτεραι παρά ταύτας; διότι δεν ερωτάς φρονίμως περί τούτου.
\par 11 Η σοφία είναι καλή ως η κληρονομία, και ωφέλιμος εις τους βλέποντας τον ήλιον.
\par 12 Διότι η σοφία είναι σκέπη, ως είναι σκέπη το αργύριον· πλην η υπεροχή της γνώσεως είναι, ότι η σοφία ζωοποιεί τους έχοντας αυτήν.
\par 13 Θεώρει το έργον του Θεού· διότι τις δύναται να κάμη ευθές εκείνο, το οποίον αυτός έκαμε στρεβλόν;
\par 14 Εν ημέρα ευτυχίας ευφραίνου, εν δε ημέρα δυστυχίας σκέπτου· διότι ο Θεός έκαμε το εν αντίστιχον του άλλου, διά να μη ευρίσκη ο άνθρωπος μηδέν οπίσω αυτού.
\par 15 Τα πάντα είδον εν ταις ημέραις της ματαιότητός μου· υπάρχει δίκαιος, όστις αφανίζεται εν τη δικαιοσύνη αυτού· και υπάρχει ασεβής, όστις μακροημερεύει εν τη κακία αυτού.
\par 16 Μη γίνου δίκαιος παραπολύ, και μη φρόνει σεαυτόν υπέρμετρα σοφόν· διά τι να αφανισθής;
\par 17 Μη γίνου κακός παραπολύ, και μη έσο άφρων· διά τι να αποθάνης προ του καιρού σου;
\par 18 Είναι καλόν να κρατής τούτο, από δε εκείνου να μη αποσύρης την χείρα σου· διότι ο φοβούμενος τον Θεόν θέλει εκφύγει πάντα ταύτα.
\par 19 Η σοφία ενδυναμόνει τον σοφόν περισσότερον παρά δέκα εξουσιάζοντες, οίτινες είναι εν τη πόλει.
\par 20 Διότι δεν υπάρχει άνθρωπος δίκαιος επί της γης, όστις να πράττη το καλόν και να μη αμαρτάνη.
\par 21 Προσέτι, μη δώσης την προσοχήν σου εις πάντας τους λόγους όσοι λέγονται· μήποτε ακούσης τον δούλον σου καταρώμενόν σε·
\par 22 διότι πολλάκις και η καρδία σου γνωρίζει ότι και συ παρομοίως κατηράσθης άλλους.
\par 23 Πάντα ταύτα εδοκίμασα διά της σοφίας· είπα, Θέλω γείνει σοφός· αλλ' αύτη απεμακρύνθη απ' εμού.
\par 24 ό,τι είναι πολύ μακράν και εις άκρον βαρύ, τις δύναται να εύρη τούτο;
\par 25 Εγώ περιήλθον εν τη καρδία μου διά να μάθω και να ανιχνεύσω, και να εκζητήσω σοφίαν και τον λόγον των πραγμάτων, και να γνωρίσω την ασέβειαν της αφροσύνης και την ηλιθιότητα της ανοησίας.
\par 26 Και εύρον ότι πικροτέρα είναι παρά θάνατον η γυνή, της οποίας η καρδία είναι παγίδες και δίκτυα και αι χείρες αυτής δεσμά· ο αρεστός ενώπιον του Θεού θέλει εκφύγει απ' αυτής· ο δε αμαρτωλός θέλει συλληφθή εν αυτή.
\par 27 Ιδέ, τούτο εύρηκα, λέγει ο Εκκλησιαστής, εξετάζων εν προς εν, διά να εύρω τον λόγον·
\par 28 τον οποίον έτι η ψυχή μου εκζητεί αλλά δεν ευρίσκω· άνδρα ένα μεταξύ χιλίων εύρηκα· γυναίκα όμως μίαν μεταξύ πασών τούτων δεν εύρηκα.
\par 29 Ιδού, τούτο μόνον εύρηκα· ότι ο Θεός έκαμε τον άνθρωπον ευθύν, αλλ' αυτοί επεζήτησαν λογισμούς πολλούς.

\chapter{8}

\par 1 Τις είναι ως ο σοφός; και τις γνωρίζει την λύσιν των πραγμάτων; η σοφία του ανθρώπου φαιδρύνει το πρόσωπον αυτού, και η σκληρότης του προσώπου αυτού θέλει μεταβληθή.
\par 2 Εγώ σε συμβουλεύω να φυλάττης την προσταγήν του βασιλέως, και διά τον όρκον του Θεού.
\par 3 Μη σπεύδε να φύγης απ' έμπροσθεν αυτού· μη εμμένης εις πράγμα κακόν· διότι παν ό,τι θελήση, κάμνει.
\par 4 Εν τω λόγω του βασιλέως είναι εξουσία· και τις θέλει ειπεί προς αυτόν, Τι κάμνεις;
\par 5 Ο φυλάττων την προσταγήν δεν θέλει δοκιμάσει πράγμα κακόν· και η καρδία του σοφού γνωρίζει τον καιρόν και τον τρόπον.
\par 6 Παντί πράγματι είναι καιρός και τρόπος· όθεν η αθλιότης του ανθρώπου είναι πολλή επ' αυτόν·
\par 7 διότι δεν γνωρίζει τι θέλει συμβή· επειδή τις δύναται να απαγγείλη προς αυτόν πως θέλει ακολουθήσει;
\par 8 Δεν υπάρχει άνθρωπος έχων εξουσίαν επί του πνεύματος, ώστε να εμποδίζη το πνεύμα· ουδέ έχων εξουσίαν επί της ημέρας του θανάτου· και εν τω πολέμω δεν είναι αποφυγή· και η ασέβεια δεν θέλει ελευθερώσει τους έχοντας αυτήν.
\par 9 Πάντα ταύτα είδον, και προσήλωσα τον νούν μου εις παν έργον, το οποίον γίνεται υπό τον ήλιον· είναι καιρός καθ' ον ο άνθρωπος εξουσιάζει άνθρωπον προς βλάβην αυτού.
\par 10 Και ούτως είδον τους ασεβείς ενταφιασθέντας, οίτινες ήλθον και απήλθον από της γης της αγίας και ελησμονήθησαν εν τη πόλει, όπου είχον πράξει ούτω· και τούτο ματαιότης.
\par 11 Επειδή η κατά του πονηρού έργου απόφασις δεν εκτελείται ταχέως, διά τούτο η καρδία των υιών των ανθρώπων είναι όλη έκδοτος εις το να πράττη το κακόν.
\par 12 Αν και ο αμαρτωλός πράττη κακόν, εκατοντάκις και μακροημερεύη, εγώ όμως γνωρίζω βεβαίως ότι θέλει είσθαι καλόν εις τους φοβουμένους τον Θεόν, οίτινες φοβούνται από προσώπου αυτού·
\par 13 εις δε τον ασεβή δεν θέλει είσθαι καλόν, και δεν θέλουσι μακρυνθή αι ημέραι αυτού, αίτινες παρέρχονται ως σκιά· διότι δεν φοβείται από προσώπου του Θεού.
\par 14 Υπάρχει ματαιότης, ήτις γίνεται επί της γής· ότι είναι, δίκαιοι εις τους οποίους συμβαίνει κατά τα έργα των ασεβών, και είναι ασεβείς εις τους οποίους συμβαίνει κατά τα έργα των δικαίων· είπα ότι και τούτο ματαιότης.
\par 15 Διά τούτο εγώ επήνεσα την ευφροσύνην· διότι ο άνθρωπος δεν έχει καλήτερόν υπό τον ήλιον, ειμή να τρώγη και να πίνη και να ευφραίνηται· και τούτο θέλει μείνει εις αυτόν από του κόπου αυτού εν ταις ημέραις της ζωής αυτού, τας οποίας ο Θεός έδωκεν εις αυτόν υπό τον ήλιον.
\par 16 Αφού έδωκα την καρδίαν μου εις το να γνωρίσω την σοφίαν και να ίδω τον περισπασμόν τον γινόμενον επί της γης, διότι ούτε ημέραν ούτε νύκτα δεν βλέπουσιν ύπνον εις τους οφθαλμούς αυτών·
\par 17 τότε είδον παν το έργον του Θεού, ότι άνθρωπος δεν δύναται να εύρη το έργον το οποίον έγεινεν υπό τον ήλιον· επειδή όσον και αν κοπιάση ο άνθρωπος ζητών, βεβαίως δεν θέλει ευρεί· έτι δε και ο σοφός εάν είπη να γνωρίση αυτό, δεν θέλει δυνηθή να εύρη.

\chapter{9}

\par 1 Διότι άπαν τούτο εσκέφθην εν τη καρδία μου, διά να εξιχνιάσω άπαν τούτο, ότι οι δίκαιοι και οι σοφοί, και τα έργα αυτών, είναι εν χειρί Θεού· δεν υπάρχει άνθρωπος γνωρίζων, είτε αγάπη θέλει είσθαι είτε μίσος· τα πάντα είναι έμπροσθεν αυτών.
\par 2 Πάντα συμβαίνουσιν επίσης εις πάντας· εν συνάντημα είναι εις τον δίκαιον και εις τον ασεβή, εις τον αγαθόν και εις τον καθαρόν και εις τον ακάθαρτον, και εις τον θυσιάζοντα και εις τον μη θυσιάζοντα· ως ο αγαθός, ούτω και ο αμαρτωλός· ο ομνύων ως ο φοβούμενος τον όρκον.
\par 3 Τούτο είναι το κακόν μεταξύ πάντων των γινομένων υπό τον ήλιον, ότι εν συνάντημα είναι εις πάντας· και μάλιστα η καρδία των υιών των ανθρώπων είναι πλήρης κακίας, και αφροσύνη είναι εν τη καρδία αυτών ενόσω ζώσι, και μετά ταύτα υπάγουσι προς τους νεκρούς.
\par 4 Διότι εις τον έχοντα κοινωνίαν μεταξύ πάντων των ζώντων είναι ελπίς· επειδή κύων ζων είναι καλήτερος παρά λέοντα νεκρόν.
\par 5 Διότι οι ζώντες γνωρίζουσιν ότι θέλουσιν αποθάνει· αλλ' οι νεκροί δεν γνωρίζουσιν ουδέν ουδέ έχουσι πλέον απόλαυσιν· επειδή το μνημόσυνον αυτών ελησμονήθη.
\par 6 Έτι και η αγάπη αυτών και το μίσος αυτών και ο φθόνος αυτών ήδη εχάθη· και δεν θέλουσιν έχει πλέον εις τον αιώνα μερίδα εις πάντα όσα γίνονται υπό τον ήλιον.
\par 7 Ύπαγε, φάγε τον άρτον σου εν ευφροσύνη και πίε τον οίνον σου εν ευθύμω καρδία· διότι ήδη ο Θεός ευαρεστείται εις τα έργα σου.
\par 8 Εν παντί καιρώ ας ήναι λευκά τα ιμάτιά σου· και έλαιον ας μη εκλείψη από της κεφαλής σου.
\par 9 Χαίρου ζωήν μετά της γυναικός, την οποίαν ηγάπησας, πάσας τας ημέρας της ζωής της ματαιότητός σου, αίτινες σοι εδόθησαν υπό τον ήλιον, πάσας τας ημέρας της ματαιότητός σου· διότι τούτο είναι η μερίς σου εν τη ζωή και εν τω μόχθω σου, τον οποίον μοχθείς υπό τον ήλιον.
\par 10 Πάντα όσα εύρη η χειρ σου να κάμη, κάμε κατά την δύναμίν σου· διότι δεν είναι πράξις ούτε λογισμός ούτε γνώσις ούτε σοφία εν τω άδη όπου υπάγεις.
\par 11 Επέστρεψα και είδον υπό τον ήλιον, ότι ο δρόμος δεν είναι εις τους ταχύποδας, ουδέ ο πόλεμος εις τους δυνατούς, αλλ' ουδέ ο άρτος εις τους σοφούς, αλλ' ουδέ τα πλούτη εις τους νοήμονας, αλλ' ουδέ η χάρις εις τους αξίους· διότι καιρός και περίστασις συναντά εις πάντας αυτούς.
\par 12 Διότι ουδέ ο άνθρωπος γνωρίζει τον καιρόν αυτού· καθώς οι ιχθύες οίτινες πιάνονται εν κακώ δικτύω, και καθώς τα πτηνά, τα οποία πιάνονται εν παγίδι, ούτω παγιδεύονται οι υιοί των ανθρώπων εν καιρώ κακώ όταν εξαίφνης επέλθη επ' αυτούς.
\par 13 Και ταύτην την σοφίαν είδον υπό τον ήλιον, και εφάνη εις εμέ μεγάλη·
\par 14 Ήτο μικρά πόλις και άνδρες εν αυτή ολίγοι· και ήλθε κατ' αυτής βασιλεύς μέγας και επολιόρκησεν αυτήν και ωκοδόμησεν εναντίον αυτής προχώματα μεγάλα·
\par 15 αλλ' ευρέθη εν αυτή άνθρωπος πτωχός και σοφός, και αυτός διά της σοφίας αυτού ηλευθέρωσε την πόλιν· πλην ουδείς ενεθυμήθη τον πτωχόν εκείνον άνθρωπον.
\par 16 Τότε εγώ είπα, Η σοφία είναι καλητέρα παρά την δύναμιν, αν και η σοφία του πτωχού καταφρονήται και οι λόγοι αυτού δεν εισακούωνται.
\par 17 Οι λόγοι των σοφών εν ησυχία ακούονται μάλλον παρά την κραυγήν του εξουσιάζοντος μετά αφρόνων.
\par 18 Η σοφία είναι καλητέρα παρά τα όπλα του πολέμου· εις δε αμαρτωλός αφανίζει μεγάλα καλά.

\chapter{10}

\par 1 Μυίαι αποθνήσκουσαι κάμνουσι το μύρον του μυρεψού να βρωμά, να αναβράζη· και μικρά αφροσύνη ατιμάζει τον εν υπολήψει επί σοφία και τιμή.
\par 2 Η καρδία του σοφού είναι εν τη δεξιά αυτού· η δε καρδία του άφρονος εν τη αριστερά αυτού.
\par 3 Και έτι όταν ο άφρων περιπατή εν τη οδώ, λείπει σύνεσις αυτού και αναγγέλλει προς πάντας ότι είναι άφρων.
\par 4 Εάν το πνεύμα του ηγεμόνος εγερθή εναντίον σου, μη αφήσης τον τόπον σου· διότι η γλυκύτης καταπαύει αμαρτίας μεγάλας.
\par 5 Είναι κακόν το οποίον είδον υπό τον ήλιον, λάθος, λέγω, προερχόμενον από του εξουσιαστού·
\par 6 ότι ο άφρων βάλλεται εις μεγάλας αξίας, οι δε πλούσιοι κάθηνται εν ταπεινώ τόπω.
\par 7 Είδον δούλους εφ' ίππων και άρχοντας περιπατούντας ως δούλους επί της γης.
\par 8 Όστις σκάπτει λάκκον, θέλει πέσει εις αυτόν· και όστις χαλά φραγμόν, όφις θέλει δαγκάσει αυτόν.
\par 9 Ο μετακινών λίθους θέλει βλαφθή υπ' αυτών· ο σχίζων ξύλα θέλει κινδυνεύσει εν αυτοίς.
\par 10 Εάν ο σίδηρος αμβλυνθή και δεν ακονίση τις την κόψιν αυτού, πρέπει να προσθέση δύναμιν· η σοφία δε είναι ωφέλιμος προς διεύθυνσιν.
\par 11 Εάν ο όφις δαγκάνη χωρίς συριγμού, πλην και ο συκοφάντης καλήτερος δεν είναι.
\par 12 Οι λόγοι του στόματος του σοφού είναι χάρις· τα δε χείλη του άφρονος θέλουσι καταπίει αυτόν.
\par 13 Η αρχή των λόγων του στόματος αυτού είναι αφροσύνη· και το τέλος της ομιλίας αυτού κακή μωρία.
\par 14 Ο άφρων προσέτι πληθύνει λόγους, ενώ ο άνθρωπος δεν εξεύρει τι μέλλει γενέσθαι· και τις δύναται να απαγγείλη προς αυτόν τι θέλει είσθαι μετ' αυτόν;
\par 15 Ο μόχθος των αφρόνων απαυδίζει αυτούς, επειδή δεν εξεύρουσι να υπάγωσιν εις την πόλιν.
\par 16 Ουαί εις σε, γη, της οποίας ο βασιλεύς είναι νέος, και οι άρχοντές σου τρώγουσι το πρωΐ
\par 17 Μακαρία συ, γη, της οποίας ο βασιλεύς είναι υιός ευγενών, και οι άρχοντές σου τρώγουσιν εν καιρώ προς ενίσχυσιν και ουχί προς μέθην
\par 18 Διά την πολλήν οκνηρίαν πίπτει η στέγασις· και διά την αργίαν των χειρών σταλάζει η οικία.
\par 19 Δι' ευθυμίαν κάμνουσι συμπόσια, και ο οίνος ευφραίνει τους ζώντας· το δε αργύριον αποκρίνεται εις πάντα.
\par 20 Μη καταρασθής τον βασιλέα μηδέ εν τη διανοία σου· και μη καταρασθής τον πλούσιον εν τω ταμείω του κοιτώνός σου· διότι πτηνόν του ουρανού θέλει φέρει την φωνήν, και το έχον τας πτέρυγας θέλει αναγγείλει το πράγμα.

\chapter{11}

\par 1 Ρίψον τον άρτον σου επί πρόσωπον των υδάτων· διότι εν ταις πολλαίς ημέραις θέλεις ευρεί αυτόν.
\par 2 Δος μερίδιον εις επτά και έτι εις οκτώ· διότι δεν εξεύρεις τι κακόν θέλει γείνει επί της γης.
\par 3 Εάν τα νέφη ήναι πλήρη, θέλουσι διαχύσει βροχήν επί την γήν· και εάν δένδρον πέση προς τον νότον ή προς τον βορράν, εν τω τόπω όπου πέση το δένδρον, εκεί θέλει μείνει.
\par 4 Όστις παρατηρεί τον άνεμον, δεν θέλει σπείρει· και όστις θεωρεί τα νέφη, δεν θέλει θερίσει.
\par 5 Καθώς δεν γνωρίζεις τις η οδός του ανέμου ουδέ τίνι τρόπω μορφόνονται τα οστά εν τη κοιλία της κυοφορούσης, ούτω δεν γνωρίζεις τα έργα του Θεού, όστις κάμνει τα πάντα.
\par 6 Σπείρε τον σπόρον σου το πρωΐ, και την εσπέραν ας μη ησυχάση η χειρ σου· διότι δεν εξεύρεις τι θέλει ευδοκιμήσει, τούτο ή εκείνο, ή εάν και τα δύο ήναι επίσης αγαθά.
\par 7 Γλυκύ βέβαια είναι το φως, και ευάρεστον εις τους οφθαλμούς να βλέπωσι τον ήλιον·
\par 8 αλλά και εάν ο άνθρωπος ζήση έτη πολλά και ευφραίνηται εν πάσι τούτοις, ας ενθυμηθή όμως τας ημέρας του σκότους, ότι θέλουσιν είσθαι πολλαί. Πάντα τα συμβαίνοντα ματαιότης.
\par 9 Ευφραίνου, νεανίσκε, εν τη νεότητί σου· και η καρδία σου ας σε χαροποιή εν ταις ημέραις της νεότητός σου· και περιπάτει κατά τας επιθυμίας της καρδίας σου και κατά την όρασιν των οφθαλμών σου· πλην έξευρε, ότι διά πάντα ταύτα ο Θεός θέλει σε φέρει εις κρίσιν.
\par 10 Και αφαίρεσον τον θυμόν από της καρδίας σου, και απομάκρυνον την πονηρίαν από της σαρκός σου· διότι η νεότης και η παιδική ηλικία είναι ματαιότης.

\chapter{12}

\par 1 Και ενθυμού τον Πλάστην σου εν ταις ημέραις της νεότητός σου· πριν έλθωσιν αι κακαί ημέραι, και φθάσωσι τα έτη εις τα οποία θέλεις ειπεί, Δεν έχω ευχαρίστησιν εις αυτά·
\par 2 πριν σκοτισθή ο ήλιος και το φως και σελήνη και οι αστέρες, και επανέλθωσι τα νέφη μετά την βροχήν·
\par 3 ότε οι φύλακες της οικίας θέλουσι τρέμει, και οι άνδρες οι ισχυροί θέλουσι κλονίζεσθαι, και αι αλέθουσαι θέλουσι παύσει· διότι ωλιγοστεύθησαν, και αι βλέπουσαι διά των θυρίδων θέλουσιν αμαυρωθή·
\par 4 και αι θύραι θέλουσι κλεισθή εν τη οδώ, ότε θέλει ασθενήσει η φωνή της αλεθούσης, και ο άνθρωπος θέλει εξεγείρεσθαι εις την φωνήν του στρουθίου, και πάσαι αι θυγατέρες του άσματος ατονίσωσιν·
\par 5 ότε θέλουσι φοβείσθαι το ύψος και θέλουσι τρέμει εν τη οδώ· ότε η αμυγδαλέα θέλει ανθήσει και η ακρίς θέλει προξενεί βάρος και η όρεξις θέλει εκλείψει· διότι ο άνθρωπος υπάγει εις τον αιώνιον οίκον αυτού και οι πενθούντες περικυκλούσι τας οδούς·
\par 6 πριν λυθή η αργυρά άλυσος και σπάση ο λύχνος ο χρυσούς ή συντριφθή η υδρία εν τη πηγή ή χαλάση ο τροχός εν τω φρέατι,
\par 7 και επιστρέψη το χώμα εις την γην, καθώς ήτο, και το πνεύμα επιστρέψη εις τον Θεόν, όστις έδωκεν αυτό.
\par 8 Ματαιότης ματαιοτήτων, είπεν ο Εκκλησιαστής· τα πάντα ματαιότης.
\par 9 Και όσον περισσότερον ο Εκκλησιαστής εστάθη σοφός, τόσον περισσότερον εδίδαξε την γνώσιν εις τον λαόν· μάλιστα επρόσεξε και ηρεύνησε και έβαλεν εις τάξιν πολλάς παροιμίας.
\par 10 Ο Εκκλησιαστής εζήτησε να εύρη λόγους ευαρέστους· και το γεγραμμένον ήτο ευθύτης και λόγοι αληθείας.
\par 11 Οι λόγοι των σοφών είναι ως βούκεντρα και ως καρφία εμπεπηγμένα υπό των διδασκάλων των συναθροισάντων αυτούς· εδόθησαν δε παρά του αυτού ποιμένος.
\par 12 Περιπλέον δε τούτων, μάθε, υιέ μου, ότι το να κάμνη τις πολλά βιβλία δεν έχει τέλος, και η πολλή μελέτη είναι μόχθος εις την σάρκα.
\par 13 Ας ακούσωμεν το τέλος της όλης υποθέσεως· φοβού τον Θεόν και φύλαττε τας εντολάς αυτού, επειδή τούτο είναι το παν του ανθρώπου.
\par 14 Διότι ο Θεός θέλει φέρει εις κρίσιν παν έργον και παν κρυπτόν, είτε αγαθόν είτε πονηρόν.


\end{document}