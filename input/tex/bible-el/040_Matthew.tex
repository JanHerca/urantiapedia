\begin{document}

\title{Matthew}


\chapter{1}

\par Βίβλος της γενεαλογίας του Ιησού Χριστού, υιού του Δαβίδ, υιού του Αβραάμ.
\par 2 Ο Αβραάμ εγέννησε τον Ισαάκ, Ισαάκ δε εγέννησε τον Ιακώβ, Ιακώβ δε εγέννησε τον Ιούδαν και τους αδελφούς αυτού,
\par 3 Ιούδας δε εγέννησε τον Φαρές και τον Ζαρά εκ της Θάμαρ, Φαρές δε εγέννησε τον Εσρώμ, Εσρώμ δε εγέννησε τον Αράμ,
\par 4 Αράμ δε εγέννησε τον Αμιναδάβ, Αμιναδάβ δε εγέννησε τον Ναασσών, Ναασσών δε εγέννησε τον Σαλμών,
\par 5 Σαλμών δε εγέννησε τον Βοόζ εκ της Ραχάβ, Βοόζ δε εγέννησε τον Ωβήδ εκ της Ρούθ, Ωβήδ δε εγέννησε τον Ιεσσαί,
\par 6 Ιεσσαί δε εγέννησε τον Δαβίδ τον βασιλέα. Δαβίδ δε ο βασιλεύς εγέννησε τον Σολομώντα εκ της γυναικός του Ουρίου,
\par 7 Σολομών δε εγέννησε τον Ροβοάμ, Ροβοάμ δε εγέννησε τον Αβιά, Αβιά δε εγέννησε τον Ασά,
\par 8 Ασά δε εγέννησε τον Ιωσαφάτ, Ιωσαφάτ δε εγέννησε τον Ιωράμ, Ιωράμ δε εγέννησε τον Οζίαν,
\par 9 Οζίας δε εγέννησε τον Ιωάθαμ, Ιωάθαμ δε εγέννησε τον Άχαζ, Άχαζ δε εγέννησε τον Εζεκίαν,
\par 10 Εζεκίας δε εγέννησε τον Μανασσή, Μανασσής δε εγέννησε τον Αμών, Αμών δε εγέννησε τον Ιωσίαν,
\par 11 Ιωσίας δε εγέννησε τον Ιεχονίαν και τους αδελφούς αυτού επί της μετοικεσίας Βαβυλώνος.
\par 12 Μετά δε την μετοικεσίαν Βαβυλώνος Ιεχονίας εγέννησε τον Σαλαθιήλ, Σαλαθιήλ δε εγέννησε τον Ζοροβάβελ,
\par 13 Ζοροβάβελ δε εγέννησε τον Αβιούδ, Αβιούδ δε εγέννησε τον Ελιακείμ, Ελιακείμ δε εγέννησε τον Αζώρ,
\par 14 Αζώρ δε εγέννησε τον Σαδώκ, Σαδώκ δε εγέννησε τον Αχείμ, Αχείμ δε εγέννησε τον Ελιούδ,
\par 15 Ελιούδ δε εγέννησε τον Ελεάζαρ, Ελεάζαρ δε εγέννησε τον Ματθάν, Ματθάν δε εγέννησε τον Ιακώβ,
\par 16 Ιακώβ δε εγέννησε τον Ιωσήφ τον άνδρα της Μαρίας, εξ ης εγεννήθη Ιησούς ο λεγόμενος Χριστός.
\par 17 Πάσαι λοιπόν αι γενεαί από Αβραάμ έως Δαβίδ είναι γενεαί δεκατέσσαρες, και από Δαβίδ έως της μετοικεσίας Βαβυλώνος γενεαί δεκατέσσαρες, και από της μετοικεσίας Βαβυλώνος έως του Χριστού γενεαί δεκατέσσαρες.
\par 18 Του δε Ιησού Χριστού η γέννησις ούτως ήτο. Αφού ηρραβωνίσθη η μήτηρ αυτού Μαρία μετά του Ιωσήφ, πριν συνέλθωσιν, ευρέθη εν γαστρί έχουσα εκ Πνεύματος Αγίου.
\par 19 Ιωσήφ δε ο ανήρ αυτής, δίκαιος ων και μη θέλων να θεατρίση αυτήν, ηθέλησε να απολύση αυτήν κρυφίως.
\par 20 Ενώ δε αυτός διελογίσθη ταύτα, ιδού, άγγελος Κυρίου εφάνη κατ' όναρ εις αυτόν, λέγων· Ιωσήφ, υιέ του Δαβίδ, μη φοβηθής να παραλάβης Μαριάμ την γυναίκα σου· διότι το εν αυτή γεννηθέν είναι εκ Πνεύματος Αγίου.
\par 21 Θέλει δε γεννήσει υιόν και θέλεις καλέσει το όνομα αυτού Ιησούν· διότι αυτός θέλει σώσει τον λαόν αυτού από των αμαρτιών αυτών.
\par 22 Τούτο δε όλον έγεινε διά να πληρωθή το ρηθέν υπό του Κυρίου διά του προφήτου, λέγοντος·
\par 23 Ιδού, η παρθένος θέλει συλλάβει και θέλει γεννήσει υιόν, και θέλουσι καλέσει το όνομα αυτού Εμμανουήλ, το οποίον μεθερμηνευόμενον είναι, Μεθ' ημών ο Θεός.
\par 24 Εξεγερθείς δε ο Ιωσήφ από του ύπνου έκαμεν ως προσέταξεν αυτόν ο άγγελος Κυρίου και παρέλαβε την γυναίκα αυτού,
\par 25 και δεν εγνώριζεν αυτήν, εωσού εγέννησε τον υιόν αυτής τον πρωτότοκον και εκάλεσε το όνομα αυτού Ιησούν.

\chapter{2}

\par Αφού δε εγεννήθη ο Ιησούς εν Βηθλεέμ της Ιουδαίας επί των ημερών Ηρώδου του βασιλέως, ιδού, μάγοι από ανατολών ήλθον εις Ιεροσόλυμα, λέγοντες·
\par 2 Που είναι ο γεννηθείς βασιλεύς των Ιουδαίων; διότι είδομεν τον αστέρα αυτού εν τη ανατολή και ήλθομεν διά να προσκυνήσωμεν αυτόν.
\par 3 Ακούσας δε Ηρώδης ο βασιλεύς, εταράχθη και πάσα η Ιεροσόλυμα μετ' αυτού,
\par 4 και συνάξας πάντας τους αρχιερείς και γραμματείς του λαού, ηρώτα να μάθη παρ' αυτών που ο Χριστός γεννάται.
\par 5 Εκείνοι δε είπον προς αυτόν· Εν Βηθλεέμ της Ιουδαίας· διότι ούτως είναι γεγραμμένον διά του προφήτου·
\par 6 Και συ, Βηθλεέμ, γη Ιούδα, δεν είσαι ουδόλως ελαχίστη μεταξύ των ηγεμόνων του Ιούδα· διότι εκ σου θέλει εξέλθει ηγούμενος, όστις θέλει ποιμάνει τον λαόν μου τον Ισραήλ.
\par 7 Τότε ο Ηρώδης καλέσας κρυφίως τους μάγους εξηκρίβωσε παρ' αυτών τον καιρόν του φαινομένου αστέρος,
\par 8 και πέμψας αυτούς εις Βηθλεέμ, είπε· Πορευθέντες ακριβώς εξετάσατε περί του παιδίου, αφού δε εύρητε, απαγγείλατέ μοι, διά να έλθω και εγώ να προσκυνήσω αυτό.
\par 9 Εκείνοι δε ακούσαντες του βασιλέως ανεχώρησαν· και ιδού, ο αστήρ τον οποίον είδον εν τη ανατολή προεπορεύετο αυτών, εωσού ελθών εστάθη επάνω όπου ήτο το παιδίον.
\par 10 Ιδόντες δε τον αστέρα εχάρησαν χαράν μεγάλην σφόδρα,
\par 11 και ελθόντες εις την οικίαν εύρον το παιδίον μετά Μαρίας της μητρός αυτού, και πεσόντες προσεκύνησαν αυτό, και ανοίξαντες τους θησαυρούς αυτών προσέφεραν εις αυτό δώρα, χρυσόν και λίβανον και σμύρναν·
\par 12 και αποκαλυφθέντες θεόθεν κατ' όναρ να μη επιστρέψωσι προς τον Ηρώδην, δι' άλλης οδού ανεχώρησαν εις την χώραν αυτών.
\par 13 Αφού δε αυτοί ανεχώρησαν, ιδού, άγγελος Κυρίου φαίνεται κατ' όναρ εις τον Ιωσήφ, λέγων· Εγερθείς παράλαβε το παιδίον και την μητέρα αυτού και φεύγε εις Αίγυπτον, και έσο εκεί εωσού είπω σοι· διότι μέλλει ο Ηρώδης να ζητήση το παιδίον, διά να απολέση αυτό.
\par 14 Ο δε εγερθείς παρέλαβε το παιδίον και την μητέρα αυτού διά νυκτός και ανεχώρησεν εις Αίγυπτον,
\par 15 και ήτο εκεί έως της τελευτής του Ηρώδου, διά να πληρωθή το ρηθέν υπό του Κυρίου διά του προφήτου λέγοντος· Εξ Αιγύπτου εκάλεσα τον υιόν μου.
\par 16 Τότε ο Ηρώδης, ιδών ότι ενεπαίχθη υπό των μάγων, εθυμώθη σφόδρα και αποστείλας εφόνευσε πάντας τους παίδας τους εν Βηθλεέμ και εν πάσι τοις ορίοις αυτής από δύο ετών και κατωτέρω κατά τον καιρόν, τον οποίον εξηκρίβωσε παρά των μάγων.
\par 17 Τότε επληρώθη το ρηθέν υπό Ιερεμίου του προφήτου, λέγοντος·
\par 18 Φωνή ηκούσθη εν Ραμά, θρήνος και κλαυθμός και οδυρμός πολύς· η Ραχήλ έκλαιε τα τέκνα αυτής, και δεν ήθελε να παρηγορηθή, διότι δεν υπάρχουσι.
\par 19 Τελευτήσαντος δε του Ηρώδου ιδού, άγγελος Κυρίου φαίνεται κατ' όναρ εις τον Ιωσήφ εν Αιγύπτω,
\par 20 λέγων· Εγερθείς παράλαβε το παιδίον και την μητέρα αυτού και ύπαγε εις γην Ισραήλ· διότι απέθανον οι ζητούντες την ψυχήν του παιδίου.
\par 21 Ο δε εγερθείς παρέλαβε το παιδίον και την μητέρα αυτού και ήλθεν εις γην Ισραήλ.
\par 22 Ακούσας δε ότι ο Αρχέλαος βασιλεύει επί της Ιουδαίας αντί Ηρώδου του πατρός αυτού, εφοβήθη να υπάγη εκεί· αποκαλυφθείς δε θεόθεν κατ' όναρ ανεχώρησεν εις τα μέρη της Γαλιλαίας,
\par 23 και ελθών κατώκησεν εις πόλιν λεγομένην Ναζαρέτ, διά να πληρωθή το ρηθέν διά των προφητών· ότι Ναζωραίος θέλει ονομασθή.

\chapter{3}

\par Εν εκείναις δε ταις ημέραις έρχεται Ιωάννης ο βαπτιστής, κηρύττων εν τη ερήμω της Ιουδαίας
\par 2 και λέγων· Μετανοείτε· διότι επλησίασεν η βασιλεία των ουρανών.
\par 3 Διότι ούτος είναι ο ρηθείς υπό Ησαΐου του προφήτου, λέγοντος· Φωνή βοώντος εν τη ερήμω, ετοιμάσατε την οδόν του Κυρίου, ευθείας κάμετε τας τρίβους αυτού.
\par 4 Αυτός δε ο Ιωάννης είχε το ένδυμα αυτού από τριχών καμήλου και ζώνην δερματίνην περί την οσφύν αυτού, η δε τροφή αυτού ήτο ακρίδες και μέλι άγριον.
\par 5 Τότε εξήρχετο προς αυτόν η Ιεροσόλυμα και πάσα η Ιουδαία και πάντα τα περίχωρα του Ιορδάνου,
\par 6 και εβαπτίζοντο εν τω Ιορδάνη υπ' αυτού, εξομολογούμενοι τας αμαρτίας αυτών.
\par 7 Ιδών δε πολλούς εκ των Φαρισαίων και Σαδδουκαίων ερχομένους εις το βάπτισμα αυτού, είπε προς αυτούς· Γεννήματα εχιδνών, τις έδειξεν εις εσάς να φύγητε από της μελλούσης οργής;
\par 8 Κάμετε λοιπόν καρπούς αξίους της μετανοίας,
\par 9 και μη φαντασθήτε να λέγητε καθ' εαυτούς, Πατέρα έχομεν τον Αβραάμ· διότι σας λέγω ότι δύναται ο Θεός εκ των λίθων τούτων να αναστήση τέκνα εις τον Αβραάμ.
\par 10 Ήδη δε και η αξίνη κείται προς την ρίζαν των δένδρων· παν λοιπόν δένδρον μη κάμνον καρπόν καλόν εκκόπτεται και εις πυρ βάλλεται.
\par 11 Εγώ μεν σας βαπτίζω εν ύδατι εις μετάνοιαν· ο δε οπίσω μου ερχόμενος είναι ισχυρότερός μου, του οποίου δεν είμαι άξιος να βαστάσω τα υποδήματα· αυτός θέλει σας βαπτίσει εν Πνεύματι Αγίω και πυρί.
\par 12 Όστις κρατεί το πτυάριον εν τη χειρί αυτού και θέλει διακαθαρίσει το αλώνιον αυτού και θέλει συνάξει τον σίτον αυτού εις την αποθήκην, το δε άχυρον θέλει κατακαύσει εν πυρί ασβέστω.
\par 13 Τότε έρχεται ο Ιησούς από της Γαλιλαίας εις τον Ιορδάνην προς τον Ιωάννην διά να βαπτισθή υπ' αυτού.
\par 14 Ο δε Ιωάννης εκώλυεν αυτόν, λέγων, Εγώ χρείαν έχω να βαπτισθώ υπό σου, και συ έρχεσαι προς εμέ;
\par 15 Αποκριθείς δε ο Ιησούς είπε προς αυτόν· Άφες τώρα· διότι ούτως είναι πρέπον εις ημάς να εκπληρώσωμεν πάσαν δικαιοσύνην. Τότε αφίνει αυτόν.
\par 16 Και βαπτισθείς ο Ιησούς ανέβη ευθύς από του ύδατος· και ιδού, ηνοίχθησαν εις αυτόν οι ουρανοί, και είδε το Πνεύμα του Θεού καταβαίνον ως περιστεράν και ερχόμενον επ' αυτόν·
\par 17 και ιδού φωνή εκ των ουρανών, λέγουσα· Ούτος είναι ο Υιός μου ο αγαπητός, εις τον οποίον ευηρεστήθην.

\chapter{4}

\par Τότε ο Ιησούς εφέρθη υπό του Πνεύματος εις την έρημον διά να πειρασθή υπό του διαβόλου,
\par 2 και νηστεύσας ημέρας τεσσαράκοντα και νύκτας τεσσαράκοντα, ύστερον επείνασε.
\par 3 Και ελθών προς αυτόν ο πειράζων είπεν· Εάν ήσαι Υιός του Θεού, ειπέ να γείνωσιν άρτοι οι λίθοι ούτοι.
\par 4 Ο δε αποκριθείς είπεν· Είναι γεγραμμένον, Με άρτον μόνον δεν θέλει ζήσει ο άνθρωπος, αλλά με πάντα λόγον εξερχόμενον διά στόματος Θεού.
\par 5 Τότε παραλαμβάνει αυτόν ο διάβολος εις την αγίαν πόλιν και στήνει αυτόν επί το πτερύγιον του ιερού
\par 6 και λέγει προς αυτόν, Εάν ήσαι Υιός του Θεού, ρίψον σεαυτόν κάτω· διότι είναι γεγραμμένον, Ότι θέλει προστάξει εις τους αγγέλους αυτού περί σου, και θέλουσι σε σηκώνει επί των χειρών αυτών, διά να μη προσκόψης προς λίθον τον πόδα σου.
\par 7 Είπε προς αυτόν ο Ιησούς· Πάλιν είναι γεγραμμένον, δεν θέλεις πειράσει Κύριον τον Θεόν σου.
\par 8 Πάλιν παραλαμβάνει αυτόν ο διάβολος εις όρος πολύ υψηλόν, και δεικνύει εις αυτόν πάντα τα βασίλεια του κόσμου και την δόξαν αυτών,
\par 9 και λέγει προς αυτόν· Ταύτα πάντα θέλω σοι δώσει, εάν πεσών προσκυνήσης με.
\par 10 Τότε ο Ιησούς λέγει προς αυτόν· Ύπαγε, Σατανά· διότι είναι γεγραμμένον, Κύριον τον Θεόν σου θέλεις προσκυνήσει και αυτόν μόνον θέλεις λατρεύσει.
\par 11 Τότε αφίνει αυτόν ο διάβολος, και ιδού, άγγελοι προσήλθον και υπηρέτουν αυτόν.
\par 12 Ακούσας δε ο Ιησούς ότι ο Ιωάννης παρεδόθη, ανεχώρησεν εις την Γαλιλαίαν.
\par 13 Και αφήσας την Ναζαρέτ ήλθε και κατώκησεν εις Καπερναούμ την παραθαλασσίαν εν τοις ορίοις Ζαβουλών και Νεφθαλείμ.
\par 14 Διά να πληρωθή το ρηθέν διά Ησαΐου του προφήτου λέγοντος·
\par 15 Γη Ζαβουλών και γη Νεφθαλείμ, κατά την οδόν της θαλάσσης, πέραν του Ιορδάνου, Γαλιλαία των εθνών.
\par 16 Ο λαός ο καθήμενος εν σκότει είδε φως μέγα, και εις τους καθημένους εν τόπω και σκιά θανάτου φως ανέτειλεν εις αυτούς.
\par 17 Από τότε ήρχισεν ο Ιησούς να κηρύττη και να λέγη· Μετανοείτε διότι επλησίασεν η βασιλεία των ουρανών.
\par 18 Περιπατών δε ο Ιησούς παρά την θάλασσαν της Γαλιλαίας, είδε δύο αδελφούς, Σίμωνα τον λεγόμενον Πέτρον και Ανδρέαν τον αδελφόν αυτού, ρίπτοντας δίκτυον εις την θάλασσαν· διότι ήσαν αλιείς·
\par 19 και λέγει προς αυτούς· Έλθετε οπίσω μου και θέλω σας κάμει αλιείς ανθρώπων.
\par 20 Οι δε αφήσαντες ευθύς τα δίκτυα, ηκολούθησαν αυτόν.
\par 21 Και προχωρήσας εκείθεν είδεν άλλους δύο αδελφούς, Ιάκωβον τον του Ζεβεδαίου και Ιωάννην τον αδελφόν αυτού, εν τω πλοίω μετά Ζεβεδαίου του πατρός αυτών επισκευάζοντας τα δίκτυα αυτών, και εκάλεσεν αυτούς.
\par 22 Οι δε αφήσαντες ευθύς το πλοίον και τον πατέρα αυτών, ηκολούθησαν αυτόν.
\par 23 Και περιήρχετο ο Ιησούς όλην την Γαλιλαίαν, διδάσκων εν ταις συναγωγαίς αυτών και κηρύττων το ευαγγέλιον της βασιλείας και θεραπεύων πάσαν νόσον και πάσαν ασθένειαν μεταξύ του λαού.
\par 24 Και διήλθεν η φήμη αυτού εις όλην την Συρίαν, και έφερον προς αυτόν πάντας τους κακώς έχοντας υπό διαφόρων νοσημάτων και βασάνων συνεχομένους και δαιμονιζομένους και σεληνιαζομένους και παραλυτικούς, και εθεράπευσεν αυτούς·
\par 25 και ηκολούθησαν αυτόν όχλοι πολλοί από της Γαλιλαίας και Δεκαπόλεως και Ιεροσολύμων και Ιουδαίας και από πέραν του Ιορδάνου.

\chapter{5}

\par Ιδών δε τους όχλους, ανέβη εις το όρος και αφού εκάθησε, προσήλθον προς αυτόν οι μαθηταί αυτού,
\par 2 και ανοίξας το στόμα αυτού εδίδασκεν αυτούς, λέγων.
\par 3 Μακάριοι οι πτωχοί τω πνεύματι, διότι αυτών είναι η βασιλεία των ουρανών.
\par 4 Μακάριοι οι πενθούντες, διότι αυτοί θέλουσι παρηγορηθή.
\par 5 Μακάριοι οι πραείς, διότι αυτοί θέλουσι κληρονομήσει την γην.
\par 6 Μακάριοι οι πεινώντες και διψώντες την δικαιοσύνην, διότι αυτοί θέλουσι χορτασθή.
\par 7 Μακάριοι οι ελεήμονες, διότι αυτοί θέλουσιν ελεηθή.
\par 8 Μακάριοι οι καθαροί την καρδίαν, διότι αυτοί θέλουσιν ιδεί τον Θεόν.
\par 9 Μακάριοι οι ειρηνοποιοί, διότι αυτοί θέλουσιν ονομασθή υιοί Θεού.
\par 10 Μακάριοι οι δεδιωγμένοι ένεκεν δικαιοσύνης, διότι αυτών είναι η βασιλεία των ουρανών.
\par 11 Μακάριοι είσθε, όταν σας ονειδίσωσι και διώξωσι και είπωσιν εναντίον σας πάντα κακόν λόγον ψευδόμενοι ένεκεν εμού.
\par 12 Χαίρετε και αγαλλιάσθε, διότι ο μισθός σας είναι πολύς εν τοις ουρανοίς· επειδή ούτως εδίωξαν τους προφήτας τους προ υμών.
\par 13 Σεις είσθε το άλας της γής· εάν δε το άλας διαφθαρή, με τι θέλει αλατισθή; εις ουδέν πλέον χρησιμεύει ειμή να ριφθή έξω και να καταπατήται υπό των ανθρώπων.
\par 14 Σεις είσθε το φως του κόσμου· πόλις κειμένη επάνω όρους δεν δύναται να κρυφθή·
\par 15 ουδέ ανάπτουσι λύχνον και θέτουσιν αυτόν υπό τον μόδιον, αλλ' επί τον λυχνοστάτην, και φέγγει εις πάντας τους εν τη οικία.
\par 16 Ούτως ας λάμψη το φως σας έμπροσθεν των ανθρώπων, διά να ίδωσι τα καλά σας έργα και δοξάσωσι τον Πατέρα σας τον εν τοις ουρανοίς.
\par 17 Μη νομίσητε ότι ήλθον να καταλύσω τον νόμον ή τους προφήτας· δεν ήλθον να καταλύσω, αλλά να εκπληρώσω.
\par 18 Διότι αληθώς σας λέγω, έως αν παρέλθη ο ουρανός και η γη, ιώτα εν ή μία κεραία δεν θέλει παρέλθει από του νόμου, εωσού εκπληρωθώσι πάντα.
\par 19 Όστις λοιπόν αθετήση μίαν των εντολών τούτων των ελαχίστων και διδάξη ούτω τους ανθρώπους, ελάχιστος θέλει ονομασθή εν τη βασιλεία των ουρανών· όστις δε εκτελέση και διδάξη, ούτος μέγας θέλει ονομασθή εν τη βασιλεία των ουρανών.
\par 20 Επειδή σας λέγω ότι εάν μη περισσεύση η δικαιοσύνη σας πλειότερον της των γραμματέων και Φαρισαίων, δεν θέλετε εισέλθει εις την βασιλείαν των ουρανών.
\par 21 Ηκούσατε ότι ερρέθη εις τους αρχαίους, Μη φονεύσης· όστις δε φονεύση, θέλει είσθαι ένοχος εις την κρίσιν.
\par 22 Εγώ όμως σας λέγω ότι πας ο οργιζόμενος αναιτίως κατά του αδελφού αυτού θέλει είσθαι ένοχος εις την κρίσιν· και όστις είπη προς τον αδελφόν αυτού Ρακά, θέλει είσθαι ένοχος εις το συνέδριον· όστις δε είπη Μωρέ, θέλει είσθαι ένοχος εις την γέενναν του πυρός.
\par 23 Εάν λοιπόν προσφέρης το δώρον σου εις το θυσιαστήριον και εκεί ενθυμηθής ότι ο αδελφός σου έχει τι κατά σου,
\par 24 άφες εκεί το δώρον σου έμπροσθεν του θυσιαστηρίου, και ύπαγε πρώτον φιλιώθητι με τον αδελφόν σου, και τότε ελθών πρόσφερε το δώρον σου.
\par 25 Ειρήνευσον με τον αντίδικόν σου ταχέως, ενόσω είσαι καθ' οδόν μετ' αυτού, μήποτε σε παραδώση ο αντίδικος εις τον κριτήν και ο κριτής σε παραδώση εις τον υπηρέτην, και ριφθής εις φυλακήν·
\par 26 αληθώς σοι λέγω, δεν θέλεις εξέλθει εκείθεν, εωσού αποδώσης το έσχατον λεπτόν.
\par 27 Ηκούσατε ότι ερρέθη εις τους αρχαίους, μη μοιχεύσης.
\par 28 Εγώ όμως σας λέγω ότι πας ο βλέπων γυναίκα διά να επιθυμήση αυτήν ήδη εμοίχευσεν αυτήν εν τη καρδία αυτού.
\par 29 Εάν ο οφθαλμός σου ο δεξιός σε σκανδαλίζη, έκβαλε αυτόν και ρίψον από σού· διότι σε συμφέρει να χαθή εν των μελών σου και να μη ριφθή όλον το σώμα σου εις την γέενναν.
\par 30 Και εάν η δεξιά σου χειρ σε σκανδαλίζη, έκκοψον αυτήν και ρίψον από σού· διότι σε συμφέρει να χαθή εν των μελών σου, και να μη ριφθή όλον το σώμα σου εις την γέενναν.
\par 31 Ερρέθη προς τούτοις ότι όστις χωρισθή την γυναίκα αυτού, ας δώση εις αυτήν διαζύγιον.
\par 32 Εγώ όμως σας λέγω ότι όστις χωρισθή την γυναίκα αυτού παρεκτός λόγου πορνείας, κάμνει αυτήν να μοιχεύηται, και όστις λάβη γυναίκα κεχωρισμένην, γίνεται μοιχός.
\par 33 Πάλιν ηκούσατε ότι ερρέθη εις τους αρχαίους, Μη επιορκήσης, αλλά εκπλήρωσον εις τον Κύριον τους όρκους σου.
\par 34 Εγώ όμως σας λέγω να μη ομόσητε μηδόλως· μήτε εις τον ουρανόν, διότι είναι θρόνος του Θεού·
\par 35 μήτε εις την γην, διότι είναι υποπόδιον των ποδών αυτού· μήτε εις τα Ιεροσόλυμα, διότι είναι πόλις του μεγάλου βασιλέως·
\par 36 μήτε εις την κεφαλήν σου να ομόσης, διότι δεν δύνασαι μίαν τρίχα να κάμης λευκήν ή μέλαιναν.
\par 37 Αλλ' ας ήναι ο λόγος σας Ναι ναι, Ου, ού· το δε πλειότερον τούτων είναι εκ του πονηρού.
\par 38 Ηκούσατε ότι ερρέθη, Οφθαλμόν αντί οφθαλμού και οδόντα αντί οδόντος.
\par 39 Εγώ όμως σας λέγω να μη αντισταθήτε προς τον πονηρόν· αλλ' όστις σε ραπίση εις την δεξιάν σου σιαγόνα, στρέψον εις αυτόν και την άλλην·
\par 40 και εις τον θέλοντα να κριθή μετά σου και να λάβη τον χιτώνα σου, άφες εις αυτόν και το ιμάτιον·
\par 41 και αν σε αγγαρεύση τις μίλιον εν, ύπαγε μετ' αυτού δύο.
\par 42 Εις τον ζητούντα παρά σου δίδε και τον θέλοντα να δανεισθή από σου μη αποστραφής.
\par 43 Ηκούσατε ότι ερρέθη, θέλεις αγαπά τον πλησίον σου και μίσει τον εχθρόν σου.
\par 44 Εγώ όμως σας λέγω, Αγαπάτε τους εχθρούς σας, ευλογείτε εκείνους, οίτινες σας καταρώνται, ευεργετείτε εκείνους, οίτινες σας μισούσι, και προσεύχεσθε υπέρ εκείνων, οίτινες σας βλάπτουσι και σας κατατρέχουσι,
\par 45 διά να γείνητε υιοί του Πατρός σας του εν τοις ουρανοίς, διότι αυτός ανατέλλει τον ήλιον αυτού επί πονηρούς και αγαθούς και βρέχει επί δικαίους και αδίκους.
\par 46 Διότι εάν αγαπήσητε τους αγαπώντάς σας, ποίον μισθόν έχετε; και οι τελώναι δεν κάμνουσι το αυτό;
\par 47 και εάν ασπασθήτε τους αδελφούς σας μόνον, τι περισσότερον κάμνετε; και οι τελώναι δεν κάμνουσιν ούτως;
\par 48 εστέ λοιπόν σεις τέλειοι, καθώς ο Πατήρ σας ο εν τοις ουρανοίς είναι τέλειος.

\chapter{6}

\par Προσέχετε να μη κάμνητε την ελεημοσύνην σας έμπροσθεν των ανθρώπων διά να βλέπησθε υπ' αυτών· ει δε μη, δεν έχετε μισθόν πλησίον του Πατρός σας του εν τοις ουρανοίς.
\par 2 Όταν λοιπόν κάμνης ελεημοσύνην, μη σαλπίσης έμπροσθέν σου, καθώς κάμνουσιν οι υποκριταί εν ταις συναγωγαίς και εν ταις οδοίς, διά να δοξασθώσιν υπό των ανθρώπων· αληθώς σας λέγω, έχουσιν ήδη τον μισθόν αυτών.
\par 3 Όταν δε συ κάμνης ελεημοσύνην, ας μη γνωρίση η αριστερά σου τι κάμνει η δεξιά σου,
\par 4 διά να ήναι η ελεημοσύνη σου εν τω κρυπτώ, και ο Πατήρ σου ο βλέπων εν τω κρυπτώ αυτός θέλει σοι ανταποδώσει εν τω φανερώ.
\par 5 Και όταν προσεύχησαι, μη έσο ως οι υποκριταί, διότι αγαπώσι να προσεύχωνται ιστάμενοι εν ταις συναγωγαίς και εν ταις γωνίαις των πλατειών, διά να φανώσιν εις τους ανθρώπους· αληθώς σας λέγω ότι έχουσιν ήδη τον μισθόν αυτών.
\par 6 Συ όμως, όταν προσεύχησαι, είσελθε εις το ταμείον σου, και κλείσας την θύραν σου προσευχήθητι εις τον Πατέρα σου τον εν τω κρυπτώ, και ο Πατήρ σου ο βλέπων εν τω κρυπτώ θέλει σοι ανταποδώσει εν τω φανερώ.
\par 7 Όταν δε προσεύχησθε, μη βαττολογήσητε ως οι εθνικοί· διότι νομίζουσιν ότι με την πολυλογίαν αυτών θέλουσιν εισακουσθή.
\par 8 Μη ομοιωθήτε λοιπόν με αυτούς· διότι εξεύρει ο Πατήρ σας τίνων έχετε χρείαν, πριν σεις ζητήσητε παρ' αυτού.
\par 9 Ούτω λοιπόν προσεύχεσθε σείς· Πάτερ ημών ο εν τοις ουρανοίς· αγιασθήτω το όνομά σου·
\par 10 ελθέτω η βασιλεία σου· γενηθήτω το θέλημά σου, ως εν ουρανώ, και επί της γής·
\par 11 τον άρτον ημών τον επιούσιον δος εις ημάς σήμερον·
\par 12 και συγχώρησον εις ημάς τας αμαρτίας ημών, καθώς και ημείς συγχωρούμεν εις τους αμαρτάνοντας εις ημάς·
\par 13 και μη φέρης ημάς εις πειρασμόν, αλλά ελευθέρωσον ημάς από του πονηρού. Διότι σου είναι η βασιλεία και η δύναμις και η δόξα εις τους αιώνας· αμήν.
\par 14 Διότι εάν συγχωρήσητε εις τους ανθρώπους τα πταίσματα αυτών, θέλει συγχωρήσει και εις εσάς ο Πατήρ σας ο ουράνιος·
\par 15 εάν όμως δεν συγχωρήσητε εις τους ανθρώπους τα πταίσματα αυτών, ουδέ ο Πατήρ σας θέλει συγχωρήσει τα πταίσματά σας.
\par 16 Και όταν νηστεύητε, μη γίνεσθε ως οι υποκριταί σκυθρωποί· διότι αφανίζουσι τα πρόσωπα αυτών, διά να φανώσιν εις τους ανθρώπους ότι νηστεύουσιν· αληθώς σας λέγω, ότι έχουσιν ήδη τον μισθόν αυτών.
\par 17 Συ όμως όταν νηστεύης, άλειψον την κεφαλήν σου και νίψον το πρόσωπόν σου,
\par 18 διά να μη φανής εις τους ανθρώπους ότι νηστεύεις, αλλ' εις τον Πατέρα σου τον εν τω κρυπτώ, και ο Πατήρ σου ο βλέπων εν τω κρυπτώ θέλει σοι ανταποδώσει εν τω φανερώ.
\par 19 Μη θησαυρίζετε εις εαυτούς θησαυρούς επί της γης, όπου σκώληξ και σκωρία αφανίζει και όπου κλέπται διατρυπούσι και κλέπτουσιν.
\par 20 Αλλά θησαυρίζετε εις εαυτούς θησαυρούς εν ουρανώ, όπου ούτε σκώληξ ούτε σκωρία αφανίζει και όπου κλέπται δεν διατρυπούσιν ουδέ κλέπτουσιν·
\par 21 επειδή όπου είναι ο θησαυρός σας, εκεί θέλει είσθαι και η καρδία σας.
\par 22 Ο λύχνος του σώματος είναι ο οφθαλμός· εάν λοιπόν ο οφθαλμός σου ήναι καθαρός, όλον το σώμα σου θέλει είσθαι φωτεινόν·
\par 23 εάν όμως ο οφθαλμός σου ήναι πονηρός, όλον το σώμα σου θέλει είσθαι σκοτεινόν. Εάν λοιπόν το φως το εν σοι ήναι σκότος, το σκότος πόσον;
\par 24 Ουδείς δύναται δύο κυρίους να δουλεύη· διότι ή τον ένα θέλει μισήσει και τον άλλον θέλει αγαπήσει, ή εις τον ένα θέλει προσκολληθή και τον άλλον θέλει καταφρονήσει. Δεν δύνασθε να δουλεύητε Θεόν και μαμμωνά.
\par 25 Διά τούτο σας λέγω, μη μεριμνάτε περί της ζωής σας τι να φάγητε και τι να πίητε, μηδέ περί του σώματός σας τι να ενδυθήτε· δεν είναι η ζωή τιμιώτερον της τροφής και το σώμα του ενδύματος;
\par 26 Εμβλέψατε εις τα πετεινά του ουρανού, ότι δεν σπείρουσιν ουδέ θερίζουσιν ουδέ συνάγουσιν εις αποθήκας, και ο Πατήρ σας ο ουράνιος τρέφει αυτά· σεις δεν είσθε πολύ ανώτεροι αυτών;
\par 27 Αλλά τις από σας μεριμνών δύναται να προσθέση μίαν πήχην εις το ανάστημα αυτού;
\par 28 Και περί ενδύματος τι μεριμνάτε; Παρατηρήσατε τα κρίνα του αγρού πως αυξάνουσι· δεν κοπιάζουσιν ουδέ κλώθουσι.
\par 29 Σας λέγω όμως ότι ουδέ ο Σολομών εν πάση τη δόξη αυτού ενεδύθη ως εν τούτων.
\par 30 Αλλ' εάν τον χόρτον του αγρού, όστις σήμερον υπάρχει και αύριον ρίπτεται εις κλίβανον, ο Θεός ενδύη ούτω, δεν θέλει ενδύσει πολλώ μάλλον εσάς, ολιγόπιστοι;
\par 31 Μη μεριμνήσητε λοιπόν λέγοντες, Τι να φάγωμεν ή τι να πίωμεν ή τι να ενδυθώμεν;
\par 32 Διότι πάντα ταύτα ζητούσιν οι εθνικοί· επειδή εξεύρει ο Πατήρ σας ο ουράνιος ότι έχετε χρείαν πάντων τούτων.
\par 33 Αλλά ζητείτε πρώτον την βασιλείαν του Θεού και την δικαιοσύνην αυτού, και ταύτα πάντα θέλουσι σας προστεθή.
\par 34 Μη μεριμνήσητε λοιπόν περί της αύριον· διότι η αύριον θέλει μεριμνήσει τα εαυτής· αρκετόν είναι εις την ημέραν το κακόν αυτής.

\chapter{7}

\par Μη κρίνετε, διά να μη κριθήτε·
\par 2 διότι με οποίαν κρίσιν κρίνετε θέλετε κριθή, και με οποίον μέτρον μετρείτε θέλει αντιμετρηθή εις εσάς.
\par 3 Και διά τι βλέπεις το ξυλάριον το εν τω οφθαλμώ του αδελφού σου, την δε δοκόν την εν τω οφθαλμώ σου δεν παρατηρείς;
\par 4 Η πως θέλεις ειπεί προς τον αδελφόν σου, Άφες να εκβάλω το ξυλάριον από του οφθαλμού σου, ενώ η δοκός είναι εν τω οφθαλμώ σου;
\par 5 Υποκριτά, έκβαλε πρώτον την δοκόν εκ του οφθαλμού σου, και τότε θέλεις ιδεί καθαρώς διά να εκβάλης το ξυλάριον εκ του οφθαλμού του αδελφού σου.
\par 6 Μη δώσητε το άγιον εις τους κύνας μηδέ ρίψητε τους μαργαρίτας σας έμπροσθεν των χοίρων, μήποτε καταπατήσωσιν αυτούς με τους πόδας αυτών και στραφέντες σας διασχίσωσιν.
\par 7 Αιτείτε, και θέλει σας δοθή· ζητείτε, και θέλετε ευρεί, κρούετε, και θέλει σας ανοιχθή.
\par 8 Διότι πας ο αιτών λαμβάνει και ο ζητών ευρίσκει και εις τον κρούοντα θέλει ανοιχθή.
\par 9 Η τις άνθρωπος είναι από σας, όστις εάν ο υιός αυτού ζητήση άρτον, μήπως θέλει δώσει εις αυτόν λίθον;
\par 10 και εάν ζητήση οψάριον, μήπως θέλει δώσει εις αυτόν όφιν;
\par 11 εάν λοιπόν σεις, πονηροί όντες, εξεύρητε να δίδητε καλάς δόσεις εις τα τέκνα σας, πόσω μάλλον ο Πατήρ σας ο εν τοις ουρανοίς θέλει δώσει αγαθά εις τους ζητούντας παρ' αυτού;
\par 12 Λοιπόν πάντα όσα αν θέλητε να κάμνωσιν εις εσάς οι άνθρωποι, ούτω και σεις κάμνετε εις αυτούς· διότι ούτος είναι ο νόμος και οι προφήται.
\par 13 Εισέλθετε διά της στενής πύλης· διότι πλατεία είναι η πύλη και ευρύχωρος η οδός η φέρουσα εις την απώλειαν, και πολλοί είναι οι εισερχόμενοι δι' αυτής.
\par 14 Επειδή στενή είναι η πύλη και τεθλιμμένη η οδός η φέρουσα εις την ζωήν, και ολίγοι είναι οι ευρίσκοντες αυτήν.
\par 15 Προσέχετε δε από των ψευδοπροφητών, οίτινες έρχονται προς εσάς με ενδύματα προβάτων, έσωθεν όμως είναι λύκοι άρπαγες.
\par 16 Από των καρπών αυτών θέλετε γνωρίσει αυτούς. Μήποτε συνάγουσιν από ακανθών σταφύλια ή από τριβόλων σύκα;
\par 17 ούτω παν δένδρον καλόν κάμνει καλούς καρπούς, το δε σαπρόν δένδρον κάμνει κακούς καρπούς.
\par 18 Δεν δύναται δένδρον καλόν να κάμνη καρπούς κακούς, ουδέ δένδρον σαπρόν να κάμνη καρπούς καλούς.
\par 19 Παν δένδρον μη κάμνον καρπόν καλόν εκκόπτεται και εις πυρ βάλλεται.
\par 20 Άρα από των καρπών αυτών θέλετε γνωρίσει αυτούς.
\par 21 Δεν θέλει εισέλθει εις την βασιλείαν των ουρανών πας ο λέγων προς εμέ, Κύριε, Κύριε, αλλ' ο πράττων το θέλημα του Πατρός μου του εν τοις ουρανοίς.
\par 22 Πολλοί θέλουσιν ειπεί προς εμέ εν εκείνη τη ημέρα, Κύριε, Κύριε, δεν προεφητεύσαμεν εν τω ονόματί σου, και εν τω ονόματί σου εξεβάλομεν δαιμόνια, και εν τω ονόματί σου εκάμομεν θαύματα πολλά;
\par 23 Και τότε θέλω ομολογήσει προς αυτούς ότι ποτέ δεν σας εγνώρισα· φεύγετε απ' εμού οι εργαζόμενοι την ανομίαν.
\par 24 Πας λοιπόν όστις ακούει τους λόγους μου τούτους και κάμνει αυτούς, θέλω ομοιώσει αυτόν με άνδρα φρόνιμον, όστις ωκοδόμησε την οικίαν αυτού επί την πέτραν·
\par 25 και κατέβη η βροχή και ήλθον οι ποταμοί και έπνευσαν οι άνεμοι και προσέβαλον εις την οικίαν εκείνην, και δεν έπεσε· διότι ήτο τεθεμελιωμένη επί την πέτραν.
\par 26 Και πας ο ακούων τους λόγους μου τούτους και μη κάμνων αυτούς θέλει ομοιωθή με άνδρα μωρόν, όστις ωκοδόμησε την οικίαν αυτού επί την άμμον·
\par 27 και κατέβη η βροχή και ήλθον οι ποταμοί και έπνευσαν οι άνεμοι και προσέβαλον εις την οικίαν εκείνην, και έπεσε, και ήτο η πτώσις αυτής μεγάλη.
\par 28 Ότε δε ετελείωσεν ο Ιησούς τους λόγους τούτους, εξεπλήττοντο οι όχλοι διά την διδαχήν αυτού·
\par 29 διότι εδίδασκεν αυτούς ως έχων εξουσίαν, και ουχί ως οι γραμματείς.

\chapter{8}

\par Ότε δε κατέβη από του όρους, ηκολούθησαν αυτόν όχλοι πολλοί.
\par 2 Και ιδού, λεπρός ελθών προσεκύνει αυτόν, λέγων· Κύριε, εάν θέλης, δύνασαι να με καθαρίσης.
\par 3 Και εκτείνας την χείρα ο Ιησούς ήγγισεν αυτόν, λέγων· Θέλω, καθαρίσθητι. Και ευθύς εκαθαρίσθη η λέπρα αυτού.
\par 4 Και λέγει προς αυτόν ο Ιησούς· Πρόσεχε μη είπης τούτο εις μηδένα, αλλ' ύπαγε, δείξον σεαυτόν εις τον ιερέα και πρόσφερε το δώρον, το οποίον προσέταξεν ο Μωϋσής διά μαρτυρίαν εις αυτούς.
\par 5 Ότε δε εισήλθεν ο Ιησούς εις Καπερναούμ, προσήλθε προς αυτόν εκατόνταρχος παρακαλών αυτόν
\par 6 και λέγων· Κύριε, ο δούλός μου κείται εν τη οικία παραλυτικός, δεινώς βασανιζόμενος.
\par 7 Και λέγει προς αυτόν ο Ιησούς· Εγώ ελθών θέλω θεραπεύσει αυτόν.
\par 8 Και αποκριθείς ο εκατόνταρχος είπε· Κύριε, δεν είμαι άξιος να εισέλθης υπό την στέγην μου· αλλά μόνον ειπέ λόγον, και θέλει ιατρευθή ο δούλός μου.
\par 9 Διότι και εγώ είμαι άνθρωπος υπό εξουσίαν, έχων υπ' εμαυτόν στρατιώτας, και λέγω προς τούτον, Ύπαγε, και υπάγει, και προς άλλον, Έρχου, και έρχεται, και προς τον δούλον μου, Κάμε τούτο, και κάμνει.
\par 10 Ακούσας δε ο Ιησούς εθαύμασε και είπε προς τους ακολουθούντας· Αληθώς σας λέγω, ουδέ εν τω Ισραήλ εύρον τοσαύτην πίστιν.
\par 11 Σας λέγω δε ότι πολλοί θέλουσιν ελθεί από ανατολών και δυσμών και θέλουσι καθήσει μετά του Αβραάμ και Ισαάκ και Ιακώβ εν τη βασιλεία των ουρανών,
\par 12 οι δε υιοί της βασιλείας θέλουσιν εκβληθή εις το σκότος το εξώτερον· εκεί θέλει είσθαι ο κλαυθμός και ο τριγμός των οδόντων.
\par 13 Και είπεν ο Ιησούς προς τον εκατόνταρχον, Ύπαγε, και ως επίστευσας, ας γείνη εις σε. Και ιατρεύθη ο δούλος αυτού εν τη ώρα εκείνη.
\par 14 Και ελθών ο Ιησούς εις την οικίαν του Πέτρου, είδε την πενθεράν αυτού κατάκοιτον και πάσχουσαν πυρετόν·
\par 15 και επίασε την χείρα αυτής, και αφήκεν αυτήν ο πυρετός, και εσηκώθη και υπηρέτει αυτούς.
\par 16 Και ότε έγεινεν εσπέρα, έφεραν προς αυτόν δαιμονιζομένους πολλούς, και εξέβαλε τα πνεύματα με λόγον και πάντας τους κακώς έχοντας εθεράπευσε,
\par 17 διά να πληρωθή το ρηθέν διά Ησαΐου του προφήτου, λέγοντος· Αυτός τας ασθενείας ημών έλαβε και τας νόσους εβάστασεν.
\par 18 Ιδών δε ο Ιησούς πολλούς όχλους περί εαυτόν, προσέταξε να αναχωρήσωσιν εις το πέραν.
\par 19 Και πλησιάσας εις γραμματεύς είπε προς αυτόν, Διδάσκαλε, θέλω σοι ακολουθήσει όπου αν υπάγης.
\par 20 Και λέγει προς αυτόν ο Ιησούς· Αι αλώπεκες έχουσι φωλεάς και τα πετεινά του ουρανού κατοικίας, ο δε Υιός του ανθρώπου δεν έχει που να κλίνη την κεφαλήν.
\par 21 Άλλος δε εκ των μαθητών αυτού είπε προς αυτόν· Κύριε, συγχώρησόν μοι να υπάγω πρώτον και να θάψω τον πατέρα μου.
\par 22 Ο δε Ιησούς είπε προς αυτόν· Ακολούθει μοι και άφες τους νεκρούς να θάψωσι τους εαυτών νεκρούς.
\par 23 Και ότε εισήλθεν εις το πλοίον, ηκολούθησαν αυτόν οι μαθηταί αυτού.
\par 24 Και ιδού, τρικυμία μεγάλη έγεινεν εν τη θαλάσση, ώστε το πλοίον εσκεπάζετο υπό των κυμάτων· αυτός δε εκοιμάτο.
\par 25 Και προσελθόντες οι μαθηταί αυτού εξύπνισαν αυτόν, λέγοντες· Κύριε, σώσον ημάς, χανόμεθα.
\par 26 Και λέγει προς αυτούς· Διά τι είσθε δειλοί, ολιγόπιστοι; Τότε σηκωθείς επετίμησε τους ανέμους και την θάλασσαν, και έγεινε γαλήνη μεγάλη.
\par 27 Οι δε άνθρωποι εθαύμασαν, λέγοντες· Οποίος είναι ούτος, ότι και οι άνεμοι και η θάλασσα υπακούουσιν εις αυτόν;
\par 28 Και ότε ήλθεν εις το πέραν εις την χώραν των Γεργεσηνών, υπήντησαν αυτόν δύο δαιμονιζόμενοι εξερχόμενοι εκ των μνημείων, άγριοι καθ' υπερβολήν, ώστε ουδείς ηδύνατο να περάση διά της οδού εκείνης.
\par 29 Και ιδού, έκραξαν λέγοντες· Τι είναι μεταξύ ημών και σου, Ιησού, Υιέ του Θεού; ήλθες εδώ προ καιρού να μας βασανίσης;
\par 30 Ήτο δε μακράν απ' αυτών αγέλη χοίρων πολλών βοσκομένη.
\par 31 Και οι δαίμονες παρεκάλουν αυτόν, λέγοντες· Εάν μας εκβάλης, επίτρεψον εις ημάς να απέλθωμεν εις την αγέλην των χοίρων.
\par 32 Και είπε προς αυτούς· Υπάγετε. Και εκείνοι εξελθόντες υπήγαν εις την αγέλην των χοίρων· και ιδού, ώρμησε πάσα η αγέλη των χοίρων κατά του κρημνού εις την θάλασσαν και απέθανον εν τοις ύδασιν.
\par 33 Οι δε βόσκοντες έφυγον και ελθόντες εις την πόλιν, απήγγειλαν πάντα και τα των δαιμονιζομένων.
\par 34 Και ιδού, πάσα η πόλις εξήλθεν εις συνάντησιν του Ιησού, και ιδόντες αυτόν παρεκάλεσαν να μεταβή από των ορίων αυτών.

\chapter{9}

\par Και εμβάς εις το πλοίον, διεπέρασε και ήλθεν εις την εαυτού πόλιν.
\par 2 Και ιδού, έφερον προς αυτόν παραλυτικόν κείμενον επί κλίνης· και ιδών ο Ιησούς την πίστιν αυτών, είπε προς τον παραλυτικόν· Θάρρει, τέκνον· συγκεχωρημέναι είναι εις σε αι αμαρτίαι σου.
\par 3 Και ιδού, τινές εκ των γραμματέων είπον καθ' εαυτούς· Ούτος βλασφημεί.
\par 4 Και ιδών ο Ιησούς τους διαλογισμούς αυτών, είπε· Διά τι σεις διαλογίζεσθε πονηρά εν ταις καρδίαις σας;
\par 5 Διότι τι είναι ευκολώτερον, να είπω, Συγκεχωρημέναι είναι αι αμαρτίαι σου, ή να είπω, Εγέρθητι και περιπάτει;
\par 6 Αλλά διά να γνωρίσητε ότι εξουσίαν έχει ο Υιός του ανθρώπου επί της γης να συγχωρή αμαρτίας, τότε λέγει προς τον παραλυτικόν· Εγερθείς σήκωσον την κλίνην σου και ύπαγε εις τον οίκόν σου.
\par 7 Και εγερθείς ανεχώρησεν εις τον οίκον αυτού.
\par 8 Ιδόντες δε οι όχλοι, εθαύμασαν και εδόξασαν τον Θεόν, όστις έδωκε τοιαύτην εξουσίαν εις τους ανθρώπους.
\par 9 Και διαβαίνων ο Ιησούς εκείθεν είδεν άνθρωπον καθήμενον εις το τελώνιον, Ματθαίον λεγόμενον, και λέγει προς αυτόν· Ακολούθει μοι. Και σηκωθείς ηκολούθησεν αυτόν.
\par 10 Και ενώ εκάθητο εις την τράπεζαν εν τη οικία, ιδού, πολλοί τελώναι και αμαρτωλοί ελθόντες συνεκάθηντο μετά του Ιησού και των μαθητών αυτού.
\par 11 Και ιδόντες οι Φαρισαίοι είπον προς τους μαθητάς αυτού· Διά τι ο Διδάσκαλός σας τρώγει μετά των τελωνών και αμαρτωλών;
\par 12 Ο δε Ιησούς ακούσας είπε προς αυτούς· Δεν έχουσι χρείαν ιατρού οι υγιαίνοντες, αλλ' οι πάσχοντες.
\par 13 Υπάγετε δε και μάθετε τι είναι, Έλεον θέλω και ουχί θυσίαν. Διότι δεν ήλθον διά να καλέσω δικαίους αλλά αμαρτωλούς εις μετάνοιαν.
\par 14 Τότε έρχονται προς αυτόν οι μαθηταί του Ιωάννου, λέγοντες· Διά τι ημείς και οι Φαρισαίοι νηστεύομεν πολλά, οι δε μαθηταί σου δεν νηστεύουσι;
\par 15 Και είπε προς αυτούς ο Ιησούς· Μήπως δύνανται οι υιοί του νυμφώνος να πενθώσιν, ενόσω είναι μετ' αυτών ο νυμφίος; θέλουσιν όμως ελθεί ημέραι, όταν αφαιρεθή απ' αυτών ο νυμφίος, και τότε θέλουσι νηστεύσει.
\par 16 Και ουδείς βάλλει επίρραμμα αγνάφου πανίου επί ιμάτιον παλαιόν· διότι αφαιρεί το αναπλήρωμα αυτού από του ιματίου, και γίνεται σχίσμα χειρότερον.
\par 17 Ουδέ βάλλουσιν οίνον νέον εις ασκούς παλαιούς· ει δε μη, σχίζονται οι ασκοί, και ο οίνος εκχέεται και οι ασκοί φθείρονται· αλλά βάλλουσιν οίνον νέον εις ασκούς νέους, και αμφότερα διατηρούνται.
\par 18 Ενώ αυτός ελάλει ταύτα προς αυτούς, ιδού, άρχων τις ελθών προσεκύνει αυτόν, λέγων ότι η θυγάτηρ μου ετελεύτησε προ ολίγου· αλλά ελθέ και βάλε την χείρα σου επ' αυτήν και θέλει ζήσει.
\par 19 Και σηκωθείς ο Ιησούς ηκολούθησεν αυτόν και οι μαθηταί αυτού.
\par 20 Και ιδού, γυνή αιμορροούσα δώδεκα έτη, πλησιάσασα όπισθεν ήγγισε το άκρον του ιματίου αυτού·
\par 21 διότι έλεγε καθ' εαυτήν, Εάν μόνον εγγίσω το ιμάτιον αυτού, θέλω σωθή.
\par 22 Ο δε Ιησούς επιστραφείς και ιδών αυτήν είπε· Θάρρει, θύγατερ· η πίστις σου σε έσωσε. Και εσώθη η γυνή από της ώρας εκείνης.
\par 23 Και ελθών ο Ιησούς εις την οικίαν του άρχοντος και ιδών τους αυλητάς και τον όχλον θορυβούμενον,
\par 24 λέγει προς αυτούς· Αναχωρείτε· διότι δεν απέθανε το κοράσιον, αλλά κοιμάται. Και κατεγέλων αυτόν.
\par 25 Ότε δε εξεβλήθη ο όχλος, εισελθών επίασε την χείρα αυτής, και εσηκώθη το κοράσιον.
\par 26 Και διεδόθη η φήμη αύτη εις όλην την γην εκείνην.
\par 27 Και ενώ ανεχώρει εκείθεν ο Ιησούς, ηκολούθησαν αυτόν δύο τυφλοί, κράζοντες και λέγοντες· Ελέησον ημάς, υιέ του Δαβίδ.
\par 28 Και ότε εισήλθεν εις την οικίαν, επλησίασαν εις αυτόν οι τυφλοί, και λέγει προς αυτούς ο Ιησούς· Πιστεύετε ότι δύναμαι να κάμω τούτο; Λέγουσι προς αυτόν· Ναι, Κύριε.
\par 29 Τότε ήγγισε τους οφθαλμούς αυτών, λέγων· Κατά την πίστιν σας ας γείνη εις εσάς.
\par 30 Και ηνοίχθησαν αυτών οι οφθαλμοί· προσέταξε δε αυτούς εντόνως ο Ιησούς, λέγων· Προσέχετε, ας μη εξεύρη τούτο μηδείς.
\par 31 Αλλ' εκείνοι εξελθόντες διεφήμισαν αυτόν εν όλη τη γη εκείνη.
\par 32 Ενώ δε αυτοί εξήρχοντο, ιδού, έφεραν προς αυτόν άνθρωπον κωφόν δαιμονιζόμενον·
\par 33 και αφού εξεβλήθη το δαιμόνιον, ελάλησεν ο κωφός, και εθαύμασαν οι όχλοι, λέγοντες ότι ποτέ δεν εφάνη τοιούτον εν τω Ισραήλ.
\par 34 Οι δε Φαρισαίοι έλεγον· Διά του άρχοντος των δαιμονίων εκβάλλει τα δαιμόνια.
\par 35 Και περιήρχετο ο Ιησούς τας πόλεις πάσας και τας κώμας, διδάσκων εν ταις συναγωγαίς αυτών και κηρύττων το ευαγγέλιον της βασιλείας και θεραπεύων πάσαν νόσον και πάσαν ασθένειαν εν τω λαώ.
\par 36 Ιδών δε τους όχλους, εσπλαγχνίσθη δι' αυτούς, διότι ήσαν εκλελυμένοι και εσκορπισμένοι ως πρόβατα μη έχοντα ποιμένα.
\par 37 Τότε λέγει προς τους μαθητάς αυτού· Ο μεν θερισμός πολύς, οι δε εργάται ολίγοι·
\par 38 παρακαλέσατε λοιπόν τον κύριον του θερισμού, διά να αποστείλη εργάτας εις τον θερισμόν αυτού.

\chapter{10}

\par Και προσκαλέσας τους δώδεκα μαθητάς αυτού, έδωκεν εις αυτούς εξουσίαν κατά πνευμάτων ακαθάρτων, ώστε να εκβάλλωσιν αυτά και να θεραπεύωσι πάσαν νόσον και πάσαν ασθένειαν.
\par 2 Τα δε ονόματα των δώδεκα αποστόλων είναι ταύτα· πρώτος Σίμων ο λεγόμενος Πέτρος και Ανδρέας ο αδελφός αυτού, Ιάκωβος ο του Ζεβεδαίου και Ιωάννης ο αδελφός αυτού,
\par 3 Φίλιππος και Βαρθολομαίος, Θωμάς και Ματθαίος ο τελώνης, Ιάκωβος ο του Αλφαίου και Λεββαίος ο επονομασθείς Θαδδαίος,
\par 4 Σίμων ο Κανανίτης και ο Ιούδας ο Ισκαριώτης, όστις και παρέδωκεν αυτόν.
\par 5 Τούτους τους δώδεκα απέστειλεν ο Ιησούς και παρήγγειλεν εις αυτούς, λέγων· Εις οδόν εθνών μη υπάγητε, και εις πόλιν Σαμαρειτών μη εισέλθητε·
\par 6 υπάγετε δε μάλλον προς τα πρόβατα τα απολωλότα του οίκου Ισραήλ.
\par 7 Και υπάγοντες κηρύττετε, λέγοντες ότι επλησίασεν η βασιλεία των ουρανών.
\par 8 Ασθενούντας θεραπεύετε, λεπρούς καθαρίζετε, νεκρούς εγείρετε, δαιμόνια εκβάλλετε· δωρεάν ελάβετε, δωρεάν δότε.
\par 9 Μη έχετε χρυσόν μηδέ άργυρον μηδέ χαλκόν εις τας ζώνας σας,
\par 10 μη σακκίον διά την οδόν μηδέ δύο χιτώνας μηδέ υποδήματα μηδέ ράβδον· διότι ο εργάτης είναι άξιος της τροφής αυτού.
\par 11 Εις οποίαν δε πόλιν ή κώμην εισέλθητε, εξετάσατε τις είναι άξιος εν αυτή, και εκεί μείνατε εωσού εξέλθητε.
\par 12 Εισερχόμενοι δε εις την οικίαν χαιρετήσατε αυτήν.
\par 13 Και εάν μεν η οικία ήναι αξία, ας έλθη η ειρήνη σας επ' αυτήν· αλλ' εάν δεν ήναι αξία, η ειρήνη σας ας επιστρέψη εις εσάς.
\par 14 Και όστις δεν σας δεχθή μηδέ ακούση τους λόγους σας, εξερχόμενοι της οικίας ή της πόλεως εκείνης εκτινάξατε τον κονιορτόν των ποδών σας.
\par 15 Αληθώς σας λέγω, Ελαφροτέρα θέλει είσθαι η τιμωρία εν ημέρα κρίσεως εις την γην των Σοδόμων και Γομόρρων παρά εις την πόλιν εκείνην.
\par 16 Ιδού, εγώ σας αποστέλλω ως πρόβατα εν μέσω λύκων· γίνεσθε λοιπόν φρόνιμοι ως οι όφεις και απλοί ως αι περιστεραί.
\par 17 Προσέχετε δε από των ανθρώπων· διότι θέλουσι σας παραδώσει εις συνέδρια και εν ταις συναγωγαίς αυτών θέλουσι σας μαστιγώσει·
\par 18 και έτι ενώπιον ηγεμόνων και βασιλέων θέλετε φερθή ένεκεν εμού προς μαρτυρίαν εις αυτούς και εις τα έθνη.
\par 19 Όταν δε σας παραδίδωσι, μη μεριμνήσητε πως ή τι θέλετε λαλήσει· διότι θέλει σας δοθή εν εκείνη τη ώρα τι πρέπει να λαλήσητε.
\par 20 Επειδή σεις δεν είσθε οι λαλούντες, αλλά το Πνεύμα του Πατρός σας, το οποίον λαλεί εν υμίν.
\par 21 Θέλει δε παραδώσει αδελφός αδελφόν εις θάνατον και πατήρ τέκνον, και θέλουσιν επαναστή τέκνα κατά γονέων και θέλουσι θανατώσει αυτούς·
\par 22 και θέλετε είσθαι μισούμενοι υπό πάντων διά το όνομά μου· ο δε υπομείνας έως τέλους, ούτος θέλει σωθή.
\par 23 Όταν δε σας διώκωσιν εν τη πόλει ταύτη, φεύγετε εις την άλλην· διότι αληθώς σας λέγω, δεν θέλετε τελειώσει τας πόλεις του Ισραήλ, εωσού έλθη ο Υιός του ανθρώπου.
\par 24 Δεν είναι μαθητής ανώτερος του διδασκάλου ουδέ δούλος ανώτερος του κυρίου αυτού.
\par 25 Αρκετόν είναι εις τον μαθητήν να γείνη ως ο διδάσκαλος αυτού, και ο δούλος ως ο κύριος αυτού. Εάν τον οικοδεσπότην ωνόμασαν Βεελζεβούλ, πόσω μάλλον τους οικιακούς αυτού;
\par 26 Μη φοβηθήτε λοιπόν αυτούς· διότι δεν είναι ουδέν κεκαλυμμένον, το οποίον δεν θέλει ανακαλυφθή, και κρυπτόν, το οποίον δεν θέλει γνωσθή.
\par 27 ό,τι σας λέγω εν τω σκότει, είπατε εν τω φωτί, και ό,τι ακούετε εις το ωτίον, κηρύξατε επί των δωμάτων.
\par 28 Και μη φοβηθήτε από των αποκτεινόντων το σώμα, την δε ψυχήν μη δυναμένων να αποκτείνωσι· φοβήθητε δε μάλλον τον δυνάμενον και ψυχήν και σώμα να απολέση εν τη γεέννη.
\par 29 Δύο στρουθία δεν πωλούνται δι' εν ασσάριον; και εν εξ αυτών δεν θέλει πέσει επί την γην άνευ του θελήματος του Πατρός σας.
\par 30 Υμών δε και αι τρίχες της κεφαλής είναι πάσαι ηριθμημέναι.
\par 31 Μη φοβηθήτε λοιπόν· πολλών στρουθίων διαφέρετε σεις.
\par 32 Πας λοιπόν όστις με ομολογήση έμπροσθεν των ανθρώπων, θέλω ομολογήσει και εγώ αυτόν έμπροσθεν του Πατρός μου του εν ουρανοίς.
\par 33 Όστις δε με αρνηθή έμπροσθεν των ανθρώπων, θέλω αρνηθή αυτόν και εγώ έμπροσθεν του Πατρός μου του εν ουρανοίς.
\par 34 Μη νομίσητε ότι ήλθον να βάλω ειρήνην επί την γήν· δεν ήλθον να βάλω ειρήνην, αλλά μάχαιραν.
\par 35 Διότι ήλθον να διαχωρίσω άνθρωπον κατά του πατρός αυτού και θυγατέρα κατά της μητρός αυτής και νύμφην κατά της πενθεράς αυτής·
\par 36 και εχθροί του ανθρώπου θέλουσιν είσθαι οι οικιακοί αυτού.
\par 37 Όστις αγαπά πατέρα ή μητέρα υπέρ εμέ, δεν είναι άξιος εμού· και όστις αγαπά υιόν ή θυγατέρα υπέρ εμέ, δεν είναι άξιος εμού·
\par 38 και όστις δεν λαμβάνει τον σταυρόν αυτού και ακολουθεί οπίσω μου, δεν είναι άξιος εμού.
\par 39 Όστις εύρη την ζωήν αυτού θέλει απολέσει αυτήν, και όστις απολέση την ζωήν αυτού δι' εμέ θέλει ευρεί αυτήν.
\par 40 Όστις δέχεται εσάς εμέ δέχεται, και όστις δέχεται εμέ δέχεται τον αποστείλαντά με.
\par 41 Ο δεχόμενος προφήτην εις όνομα προφήτου μισθόν προφήτου θέλει λάβει, και ο δεχόμενος δίκαιον εις όνομα δικαίου, μισθόν δικαίου θέλει λάβει.
\par 42 Και όστις ποτίση ένα των μικρών τούτων ποτήριον μόνον ψυχρού ύδατος εις όνομα μαθητού, αληθώς σας λέγω, δεν θέλει χάσει τον μισθόν αυτού.

\chapter{11}

\par Και ότε ετελείωσεν ο Ιησούς διατάττων εις τους δώδεκα μαθητάς αυτού, μετέβη εκείθεν διά να διδάσκη και να κηρύττη εν ταις πόλεσιν αυτών.
\par 2 Ο δε Ιωάννης, ακούσας εν τω δεσμωτηρίω τα έργα του Χριστού, έπεμψε δύο των μαθητών αυτού,
\par 3 και είπε προς αυτόν· Συ είσαι ο ερχόμενος, ή άλλον προσδοκώμεν;
\par 4 Και αποκριθείς ο Ιησούς είπε προς αυτούς· Υπάγετε και απαγγείλατε προς τον Ιωάννην όσα ακούετε και βλέπετε·
\par 5 τυφλοί αναβλέπουσι και χωλοί περιπατούσι, λεπροί καθαρίζονται και κωφοί ακούουσι, νεκροί εγείρονται και πτωχοί ευαγγελίζονται·
\par 6 και μακάριος είναι όστις δεν σκανδαλισθή εν εμοί.
\par 7 Ενώ δε ούτοι ανεχώρουν, ήρχισεν ο Ιησούς να λέγη προς τους όχλους περί του Ιωάννου· Τι εξήλθετε εις την έρημον να ίδητε; κάλαμον υπό ανέμου σαλευόμενον;
\par 8 Αλλά τι εξήλθετε να ίδητε; άνθρωπον ενδεδυμένον μαλακά ιμάτια; ιδού, οι τα μαλακά φορούντες εν τοις οίκοις των βασιλέων ευρίσκονται.
\par 9 Αλλά τι εξήλθετε να ίδητε; προφήτην; ναι, σας λέγω, και περισσότερον προφήτου.
\par 10 Διότι ούτος είναι, περί του οποίου είναι γεγραμμένον· Ιδού, εγώ αποστέλλω τον άγγελόν μου προ προσώπου σου, όστις θέλει κατασκευάσει την οδόν σου έμπροσθέν σου.
\par 11 Αληθώς σας λέγω, μεταξύ των γεννηθέντων υπό γυναικών δεν ηγέρθη μεγαλήτερος Ιωάννου του βαπτιστού· πλην ο μικρότερος εν τη βασιλεία των ουρανών είναι μεγαλήτερος αυτού.
\par 12 Από δε των ημερών Ιωάννου του βαπτιστού έως του νυν η βασιλεία των ουρανών βιάζεται, και οι βιασταί αρπάζουσιν αυτήν.
\par 13 Διότι πάντες οι προφήται και ο νόμος έως Ιωάννου προεφήτευσαν.
\par 14 Και αν θέλητε να δεχθήτε τούτο, αυτός είναι ο Ηλίας, όστις έμελλε να έλθη.
\par 15 Ο έχων ώτα διά να ακούη ας ακούη.
\par 16 Αλλά με τι να ομοιώσω την γενεάν ταύτην; είναι ομοία με παιδάρια καθήμενα εν ταις αγοραίς και φωνάζοντα προς τους συντρόφους αυτών,
\par 17 και λέγοντα· Αυλόν σας επαίξαμεν, και δεν εχορεύσατε, σας εθρηνωδήσαμεν, και δεν εκλαύσατε.
\par 18 Διότι ήλθεν ο Ιωάννης μήτε τρώγων μήτε πίνων, και λέγουσι· Δαιμόνιον έχει.
\par 19 Ήλθεν ο Υιός του ανθρώπου τρώγων και πίνων, και λέγουσιν· Ιδού, άνθρωπος φάγος και οινοπότης, φίλος τελωνών και αμαρτωλών. Και εδικαιώθη η σοφία από των τέκνων αυτής.
\par 20 Τότε ήρχισε να ονειδίζη τας πόλεις εν αις έγειναν τα πλειότερα θαύματα αυτού, διότι δεν μετενόησαν·
\par 21 Ουαί εις σε, Χοραζίν, ουαί εις σε, Βηθσαϊδάν· διότι εάν τα θαύματα τα γενόμενα εν υμίν εγίνοντο εν τη Τύρω και Σιδώνι προ πολλού ήθελον μετανοήσει εν σάκκω και σποδώ.
\par 22 Πλην σας λέγω εις την Τύρον και Σιδώνα ελαφροτέρα θέλει είσθαι η τιμωρία εν ημέρα κρίσεως παρά εις εσάς.
\par 23 Και συ, Καπερναούμ, η υψωθείσα έως του ουρανού, θέλεις καταβιβασθή έως άδου· διότι εάν τα θαύματα τα γενόμενα εν σοι εγίνοντο εν Σοδόμοις, ήθελον μείνει μέχρι της σήμερον.
\par 24 Πλην σας λέγω, ότι εις την γην των Σοδόμων ελαφροτέρα θέλει είσθαι η τιμωρία εν ημέρα κρίσεως παρά εις σε.
\par 25 Εν εκείνω τω καιρώ αποκριθείς ο Ιησούς είπε· Δοξάζω σε, Πάτερ, κύριε του ουρανού και της γης, ότι απέκρυψας ταύτα από σοφών και συνετών και απεκάλυψας αυτά εις νήπια·
\par 26 ναι, ω Πάτερ, διότι ούτως έγεινεν αρεστόν έμπροσθέν σου.
\par 27 Πάντα παρεδόθησαν εις εμέ από του Πατρός μου· και ουδείς γινώσκει τον Υιόν ει μη ο Πατήρ· ουδέ τον Πατέρα γινώσκει τις ειμή ο Υιός και εις όντινα θέλει ο Υιός να αποκαλύψη αυτόν.
\par 28 Έλθετε προς με, πάντες οι κοπιώντες και πεφορτισμένοι, και εγώ θέλω σας αναπαύσει.
\par 29 Άρατε τον ζυγόν μου εφ' υμάς και μάθετε απ' εμού, διότι πράος είμαι και ταπεινός την καρδίαν, και θέλετε ευρεί ανάπαυσιν εν ταις ψυχαίς υμών·
\par 30 διότι ο ζυγός μου είναι καλός και το φορτίον μου ελαφρόν.

\chapter{12}

\par Εν εκείνω τω καιρώ επορεύετο ο Ιησούς διά των σπαρτών εν σαββάτω· οι δε μαθηταί αυτού επείνασαν και ήρχισαν να ανασπώσιν αστάχυα και να τρώγωσιν.
\par 2 Οι δε Φαρισαίοι ιδόντες είπον προς αυτόν· Ιδού, οι μαθηταί σου πράττουσιν ό,τι δεν συγχωρείται να πράττηται το σάββατον.
\par 3 Ο δε είπε προς αυτούς· Δεν ανεγνώσατε τι έπραξεν ο Δαβίδ ότε επείνασεν αυτός και οι μετ' αυτού;
\par 4 πως εισήλθεν εις τον οίκον του Θεού και έφαγε τους άρτους της προθέσεως, τους οποίους δεν ήτο συγκεχωρημένον εις αυτόν να φάγη, ούτε εις τους μετ' αυτού, ειμή εις τους ιερείς μόνους;
\par 5 Η δεν ανεγνώσατε εν τω νόμω ότι εν τοις σάββασιν οι ιερείς βεβηλόνουσι το σάββατον εν τω ιερώ και είναι αθώοι;
\par 6 Σας λέγω δε ότι εδώ είναι μεγαλήτερος του ιερού.
\par 7 Εάν όμως εγνωρίζετε τι είναι Έλεον θέλω και ουχί θυσίαν, δεν ηθέλετε καταδικάσει τους αθώους.
\par 8 Διότι ο Υιός του ανθρώπου είναι κύριος και του σαββάτου.
\par 9 Και μεταβάς εκείθεν ήλθεν εις την συναγωγήν αυτών.
\par 10 Και ιδού, ήτο άνθρωπος έχων την χείρα ξηράν· και ηρώτησαν αυτόν λέγοντες· Συγχωρείται τάχα να θεραπεύη τις εν τω σαββάτω; διά να κατηγορήσωσιν αυτόν.
\par 11 Ο δε είπε προς αυτούς· Τις άνθρωπος από σας θέλει είσθαι, όστις έχων πρόβατον εν, εάν τούτο πέση εν τω σαββάτω εις λάκκον, δεν θέλει πιάσει και σηκώσει αυτό;
\par 12 πόσον λοιπόν διαφέρει άνθρωπος προβάτου; ώστε συγχωρείται εν τω σαββάτω να αγαθοποιή τις.
\par 13 Τότε λέγει προς τον άνθρωπον· Έκτεινον την χείρα σου· και εξέτεινε, και αποκατεστάθη υγιής ως η άλλη.
\par 14 Οι δε Φαρισαίοι εξελθόντες συνεβουλεύθησαν κατ' αυτού, διά να απολέσωσιν αυτόν.
\par 15 Αλλ' ο Ιησούς νοήσας ανεχώρησεν εκείθεν· και ηκολούθησαν αυτόν όχλοι πολλοί, και εθεράπευσεν αυτούς πάντας.
\par 16 Και παρήγγειλεν εις αυτούς αυστηρώς διά να μη φανερώσωσιν αυτόν,
\par 17 διά να πληρωθή το ρηθέν διά Ησαΐου του προφήτου, λέγοντος·
\par 18 Ιδού, ο δούλός μου, τον οποίον εξέλεξα, ο αγαπητός μου, εις τον οποίον η ψυχή μου ευηρεστήθη· θέλω θέσει το Πνεύμα μου επ' αυτόν, και θέλει εξαγγείλει κρίσιν εις τα έθνη·
\par 19 δεν θέλει αντιλογήσει ουδέ κραυγάσει, ουδέ θέλει ακούσει τις την φωνήν αυτού εν ταις πλατείαις.
\par 20 Κάλαμον συντετριμμένον δεν θέλει θλάσει και λινάριον καπνίζον δεν θέλει σβέσει, εωσού εκφέρη εις νίκην την κρίσιν·
\par 21 Και εν τω ονόματι αυτού θέλουσιν ελπίσει τα έθνη.
\par 22 Τότε εφέρθη προς αυτόν δαιμονιζόμενος τυφλός και κωφός, και εθεράπευσεν αυτόν, ώστε ο τυφλός και κωφός και ελάλει και έβλεπε.
\par 23 Και εξεπλήττοντο πάντες οι όχλοι και έλεγον· Μήπως είναι ούτος ο υιός του Δαβίδ;
\par 24 Οι δε Φαρισαίοι ακούσαντες είπον· Ούτος δεν εκβάλλει τα δαιμόνια ειμή διά του Βεελζεβούλ, του άρχοντος των δαιμονίων.
\par 25 Νοήσας δε ο Ιησούς τους διαλογισμούς αυτών, είπε προς αυτούς· Πάσα βασιλεία διαιρεθείσα καθ' εαυτής ερημούται, και πάσα πόλις ή οικία διαιρεθείσα καθ' εαυτής δεν θέλει σταθή.
\par 26 Και αν ο Σατανάς τον Σατανάν εκβάλλη, διηρέθη καθ' εαυτού· πως λοιπόν θέλει σταθή η βασιλεία αυτού;
\par 27 Και αν εγώ διά του Βεελζεβούλ εκβάλλω τα δαιμόνια, οι υιοί σας διά τίνος εκβάλλουσι; διά τούτο αυτοί θέλουσιν είσθαι κριταί σας.
\par 28 Αλλ' εάν εγώ διά Πνεύματος Θεού εκβάλλω τα δαιμόνια, άρα έφθασεν εις εσάς η βασιλεία του Θεού.
\par 29 Η πως δύναταί τις να εισέλθη εις την οικίαν του δυνατού και να διαρπάση τα σκεύη αυτού, εάν πρώτον δεν δέση τον δυνατόν; και τότε θέλει διαρπάσει την οικίαν αυτού.
\par 30 Όστις δεν είναι μετ' εμού είναι κατ' εμού, και όστις δεν συνάγει μετ' εμού σκορπίζει.
\par 31 Διά τούτο σας λέγω, Πάσα αμαρτία και βλασφημία θέλει συγχωρηθή εις τους ανθρώπους, η κατά του Πνεύματος όμως βλασφημία δεν θέλει συγχωρηθή εις τους ανθρώπους·
\par 32 και όστις είπη λόγον κατά του Υιού του ανθρώπου, θέλει συγχωρηθή εις αυτόν· όστις όμως είπη κατά του Πνεύματος του Αγίου, δεν θέλει συγχωρηθή εις αυτόν ούτε εν τούτω τω αιώνι ούτε εν τω μέλλοντι.
\par 33 Η κάμετε το δένδρον καλόν, και τον καρπόν αυτού καλόν, ή κάμετε το δένδρον σαπρόν, και τον καρπόν αυτού σαπρόν· διότι εκ του καρπού γνωρίζεται το δένδρον.
\par 34 Γεννήματα εχιδνών, πως δύνασθε να λαλήτε καλά όντες πονηροί; διότι εκ του περισσεύματος της καρδίας λαλεί το στόμα.
\par 35 Ο καλός άνθρωπος εκ του καλού θησαυρού της καρδίας εκβάλλει τα καλά, και ο πονηρός άνθρωπος εκ του πονηρού θησαυρού εκβάλλει πονηρά.
\par 36 Σας λέγω δε ότι διά πάντα λόγον αργόν, τον οποίον ήθελον λαλήσει οι άνθρωποι, θέλουσιν αποδώσει λόγον δι' αυτόν εν ημέρα κρίσεως.
\par 37 Διότι εκ των λόγων σου θέλεις δικαιωθή, και εκ των λόγων σου θέλεις καταδικασθή.
\par 38 Τότε απεκρίθησάν τινές των γραμματέων και Φαρισαίων, λέγοντες· Διδάσκαλε, θέλομεν να ίδωμεν σημείον από σου.
\par 39 Εκείνος δε αποκριθείς είπε προς αυτούς· Γενεά πονηρά και μοιχαλίς σημείον ζητεί, και σημείον δεν θέλει δοθή εις αυτήν ειμή το σημείον Ιωνά του προφήτου.
\par 40 Διότι ως ο Ιωνάς ήτο εν τη κοιλία του κήτους τρεις ημέρας και τρεις νύκτας, ούτω θέλει είσθαι ο Υιός του ανθρώπου εν τη καρδία της γης τρεις ημέρας και τρεις νύκτας.
\par 41 Άνδρες Νινευίται θέλουσιν αναστηθή εν τη κρίσει μετά της γενεάς ταύτης και θέλουσι κατακρίνει αυτήν, διότι μετενόησαν εις το κήρυγμα του Ιωνά, και ιδού, πλειότερον του Ιωνά είναι εδώ.
\par 42 Η βασίλισσα του νότου θέλει σηκωθή εν τη κρίσει μετά της γενεάς ταύτης και θέλει κατακρίνει αυτήν, διότι ήλθεν εκ των περάτων της γης διά να ακούση την σοφίαν του Σολομώντος, και ιδού, πλειότερον του Σολομώντος είναι εδώ.
\par 43 Όταν δε το ακάθαρτον πνεύμα εξέλθη από του ανθρώπου, διέρχεται δι' ανύδρων τόπων και ζητεί ανάπαυσιν και δεν ευρίσκει.
\par 44 Τότε λέγει· Ας επιστρέψω εις τον οίκόν μου, όθεν εξήλθον· και ελθόν ευρίσκει αυτόν κενόν, σεσαρωμένον και εστολισμένον.
\par 45 Τότε υπάγει και παραλαμβάνει μεθ' εαυτού επτά άλλα πνεύματα πονηρότερα εαυτού, και εισελθόντα κατοικούσιν εκεί, και γίνονται τα έσχατα του ανθρώπου εκείνου χειρότερα των πρώτων. Ούτω θέλει είσθαι και εις την γενεάν ταύτην την πονηράν.
\par 46 Ενώ δε αυτός ελάλει έτι προς τους όχλους, ιδού, η μήτηρ και οι αδελφοί αυτού ίσταντο έξω, ζητούντες να λαλήσωσι προς αυτόν.
\par 47 Είπε δε τις προς αυτόν· Ιδού, η μήτηρ σου και οι αδελφοί σου ίστανται έξω, ζητούντες να λαλήσωσι προς σε.
\par 48 Ο δε αποκριθείς προς τον ειπόντα τούτο προς αυτόν είπε· Τις είναι η μήτηρ μου και τίνες είναι οι αδελφοί μου;
\par 49 Και εκτείνας την χείρα αυτού προς τους μαθητάς αυτού είπεν· Ιδού η μήτηρ μου και οι αδελφοί μου.
\par 50 Διότι όστις κάμη το θέλημα του Πατρός μου του εν ουρανοίς, αυτός μου είναι αδελφός και αδελφή και μήτηρ.

\chapter{13}

\par Εν εκείνη δε τη ημέρα εξελθών ο Ιησούς από της οικίας εκάθητο πλησίον της θαλάσσης·
\par 2 και συνήχθησαν προς αυτόν όχλοι πολλοί, ώστε εμβάς εις το πλοίον εκάθητο, και πας ο όχλος ίστατο επί τον αιγιαλόν.
\par 3 Και ελάλησε προς αυτούς πολλά διά παραβολών, λέγων· Ιδού, εξήλθεν ο σπείρων διά να σπείρη.
\par 4 Και ενώ έσπειρεν, άλλα μεν έπεσον παρά την οδόν, και ήλθον τα πετεινά και κατέφαγον αυτά·
\par 5 άλλα δε έπεσον επί τα πετρώδη, όπου δεν είχον γην πολλήν, και ευθύς ανεφύησαν, επειδή δεν είχον βάθος γης,
\par 6 και ότε ανέτειλεν ο ήλιος εκαυματίσθησαν και επειδή δεν είχον ρίζαν εξηράνθησαν·
\par 7 άλλα δε έπεσον επί τας ακάνθας, και ανέβησαν αι άκανθαι και απέπνιξαν αυτά·
\par 8 άλλα δε έπεσον επί την γην την καλήν και έδιδον καρπόν το μεν εκατόν, το δε εξήκοντα, το δε τριάκοντα.
\par 9 Ο έχων ώτα διά να ακούη, ας ακούη.
\par 10 Και προσελθόντες οι μαθηταί, είπον προς αυτόν· Διά τι λαλείς προς αυτούς διά παραβολών;
\par 11 Ο δε αποκριθείς είπε προς αυτούς· Διότι εις εσάς εδόθη να γνωρίσητε τα μυστήρια της βασιλείας των ουρανών, εις εκείνους όμως δεν εδόθη.
\par 12 Διότι όστις έχει, έτι θέλει δοθή εις αυτόν και θέλει περισσευθή· όστις όμως δεν έχει, και ό,τι έχει θέλει αφαιρεθή απ' αυτού.
\par 13 Διά τούτο λαλώ προς αυτούς διά παραβολών, διότι βλέποντες δεν βλέπουσι και ακούοντες δεν ακούουσιν ουδέ νοούσι.
\par 14 Και εκπληρούται επ' αυτών η προφητεία του Ησαΐου η λέγουσα· Με την ακοήν θέλετε ακούσει και δεν θέλετε εννοήσει, και βλέποντες θέλετε ιδεί και δεν θέλετε καταλάβει·
\par 15 διότι επαχύνθη η καρδία του λαού τούτου, και με τα ώτα βαρέως ήκουσαν και τους οφθαλμούς αυτών έκλεισαν μήποτε ίδωσι με τους οφθαλμούς και ακούσωσι με τα ώτα και νοήσωσι με την καρδίαν και επιστρέψωσι, και ιατρεύσω αυτούς.
\par 16 Υμών δε οι οφθαλμοί είναι μακάριοι, διότι βλέπουσι, και τα ώτα σας, διότι ακούουσιν.
\par 17 Επειδή αληθώς σας λέγω ότι πολλοί προφήται και δίκαιοι επεθύμησαν να ίδωσιν όσα βλέπετε, και δεν είδον, και να ακούσωσιν όσα ακούετε, και δεν ήκουσαν.
\par 18 Σεις λοιπόν ακούσατε την παραβολήν του σπείροντος.
\par 19 Παντός ακούοντος τον λόγον της βασιλείας και μη νοούντος, έρχεται ο πονηρός και αρπάζει το εσπαρμένον εν τη καρδία αυτού· ούτος είναι ο σπαρθείς παρά την οδόν.
\par 20 Ο δε επί τα πετρώδη σπαρθείς, ούτος είναι ο ακούων τον λόγον και ευθύς μετά χαράς δεχόμενος αυτόν·
\par 21 δεν έχει όμως ρίζαν εν εαυτώ, αλλ' είναι πρόσκαιρος, όταν δε γείνη θλίψις ή διωγμός διά τον λόγον, ευθύς σκανδαλίζεται.
\par 22 Ο δε εις τας ακάνθας σπαρθείς, ούτος είναι ο ακούων τον λόγον, έπειτα η μέριμνα του αιώνος τούτου και η απάτη του πλούτου συμπνίγει τον λόγον, και γίνεται άκαρπος.
\par 23 Ο δε σπαρθείς επί την γην την καλήν, ούτος είναι ο ακούων τον λόγον και νοών· όστις και καρποφορεί και κάμνει ο μεν εκατόν, ο δε εξήκοντα, ο δε τριάκοντα.
\par 24 Άλλην παραβολήν παρέθηκεν εις αυτούς, λέγων· Ωμοιώθη η βασιλεία των ουρανών με άνθρωπον, όστις έσπειρε καλόν σπόρον εν τω αγρώ αυτού·
\par 25 αλλ' ενώ εκοιμώντο οι άνθρωποι, ήλθεν ο εχθρός αυτού και έσπειρε ζιζάνια ανά μέσον του σίτου και ανεχώρησεν.
\par 26 Ότε δε εβλάστησεν ο χόρτος και έκαμε καρπόν, τότε εφάνησαν και τα ζιζάνια.
\par 27 Προσελθόντες δε οι δούλοι του οικοδεσπότου είπον προς αυτόν· Κύριε, καλόν σπόρον δεν έσπειρας εν τω αγρώ σου; πόθεν λοιπόν έχει τα ζιζάνια;
\par 28 Ο δε είπε προς αυτούς· Εχθρός άνθρωπος έκαμε τούτο· οι δε δούλοι είπον προς αυτόν· Θέλεις λοιπόν να υπάγωμεν και να συλλέξωμεν αυτά;
\par 29 Ο δε είπεν· Ουχί, μήποτε συλλέγοντες τα ζιζάνια εκριζώσητε μετ' αυτών τον σίτον·
\par 30 αφήσατε να συναυξάνωσιν αμφότερα μέχρι του θερισμού, και εν τω καιρώ του θερισμού θέλω ειπεί προς τους θεριστάς· Συλλέξατε πρώτον τα ζιζάνια και δέσατε αυτά εις δέσμας διά να κατακαύσητε αυτά, τον δε σίτον συνάξατε εις την αποθήκην μου.
\par 31 Άλλην παραβολήν παρέθηκεν εις αυτούς, λέγων· Ομοία είναι η βασιλεία των ουρανών με κόκκον σινάπεως, τον οποίον λαβών άνθρωπος έσπειρεν εν τω αγρώ αυτού·
\par 32 το οποίον είναι μεν μικρότερον πάντων των σπερμάτων, όταν όμως αυξηθή είναι μεγαλήτερον των λαχάνων και γίνεται δένδρον, ώστε έρχονται τα πετεινά του ουρανού και κατασκηνούσιν εν τοις κλάδοις αυτού.
\par 33 Άλλην παραβολήν είπε προς αυτούς· Ομοία είναι η βασιλεία των ουρανών με προζύμιον, το οποίον λαβούσα γυνή ενέκρυψεν εις τρία μέτρα αλεύρου, εωσού έγεινεν όλον ένζυμον.
\par 34 Ταύτα πάντα ελάλησεν ο Ιησούς διά παραβολών προς τους όχλους και χωρίς παραβολής δεν ελάλει προς αυτούς,
\par 35 διά να πληρωθή το ρηθέν διά του προφήτου, λέγοντος· Θέλω ανοίξει εν παραβολαίς το στόμα μου, θέλω απαγγείλει πράγματα κεκρυμμένα από καταβολής κόσμου.
\par 36 Τότε αφήσας τους όχλους ήλθεν εις την οικίαν ο Ιησούς. Και προσήλθον προς αυτόν οι μαθηταί αυτού, λέγοντες· Εξήγησον εις ημάς την παραβολήν των ζιζανίων του αγρού.
\par 37 Ο δε αποκριθείς είπε προς αυτούς· Ο σπείρων τον καλόν σπόρον είναι ο Υιός του ανθρώπου·
\par 38 ο δε αγρός είναι ο κόσμος· ο δε καλός σπόρος, ούτοι είναι οι υιοί της βασιλείας· τα δε ζιζάνια είναι οι υιοί του πονηρού·
\par 39 ο δε εχθρός, όστις έσπειρεν αυτά, είναι ο διάβολος· ο δε θερισμός είναι η συντέλεια του αιώνος· οι δε θερισταί είναι οι άγγελοι.
\par 40 Καθώς λοιπόν συλλέγονται τα ζιζάνια και κατακαίονται εν πυρί, ούτω θέλει είσθαι εν τη συντελεία του αιώνος τούτου.
\par 41 Θέλει αποστείλει ο Υιός του ανθρώπου τους αγγέλους αυτού, και θέλουσι συλλέξει εκ της βασιλείας αυτού πάντα τα σκάνδαλα και τους πράττοντας την ανομίαν,
\par 42 και θέλουσι ρίψει αυτούς εις την κάμινον του πυρός· εκεί θέλει είσθαι ο κλαυθμός και ο τριγμός των οδόντων.
\par 43 Τότε οι δίκαιοι θέλουσιν εκλάμψει ως ο ήλιος εν τη βασιλεία του Πατρός αυτών. Ο έχων ώτα διά να ακούη ας ακούη.
\par 44 Πάλιν ομοία είναι η βασιλεία των ουρανών με θησαυρόν κεκρυμμένον εν τω αγρώ, τον οποίον ευρών άνθρωπος έκρυψε, και από της χαράς αυτού υπάγει και πωλεί πάντα όσα έχει και αγοράζει τον αγρόν εκείνον.
\par 45 Πάλιν ομοία είναι η βασιλεία των ουρανών με άνθρωπον έμπορον ζητούντα καλούς μαργαρίτας·
\par 46 όστις ευρών ένα πολύτιμον μαργαρίτην, υπήγε και επώλησε πάντα όσα είχε και ηγόρασεν αυτόν.
\par 47 Πάλιν ομοία είναι η βασιλεία των ουρανών με δίκτυον, το οποίον ερρίφθη εις την θάλασσαν και συνήγαγεν από παντός είδους·
\par 48 το οποίον, αφού εγεμίσθη, ανεβίβασαν επί τον αιγιαλόν και καθήσαντες συνέλεξαν τα καλά εις αγγεία, τα δε αχρεία έρριψαν έξω.
\par 49 Ούτω θέλει είσθαι εν τη συντελεία του αιώνος. Θέλουσιν εξέλθει οι άγγελοι και θέλουσιν αποχωρίσει τους πονηρούς εκ μέσου των δικαίων,
\par 50 και θέλουσι ρίψει αυτούς εις την κάμινον του πυρός· εκεί θέλει είσθαι ο κλαυθμός και ο τριγμός των οδόντων.
\par 51 Λέγει προς αυτούς ο Ιησούς· Ενοήσατε ταύτα πάντα; Λέγουσι προς αυτόν· Ναι, Κύριε.
\par 52 Ο δε είπε προς αυτούς· Διά τούτο πας γραμματεύς, μαθητευθείς εις τα περί της βασιλείας των ουρανών, είναι όμοιος με άνθρωπον οικοδεσπότην, όστις εκβάλλει εκ του θησαυρού αυτού νέα και παλαιά.
\par 53 Και αφού ετελείωσεν ο Ιησούς τας παραβολάς ταύτας, ανεχώρησεν εκείθεν,
\par 54 και ελθών εις την πατρίδα αυτού, εδίδασκεν αυτούς εν τη συναγωγή αυτών, ώστε εξεπλήττοντο και έλεγον· Πόθεν εις τούτον η σοφία αύτη και αι δυνάμεις;
\par 55 δεν είναι ούτος ο υιός του τέκτονος; η μήτηρ αυτού δεν λέγεται Μαριάμ, και οι αδελφοί αυτού Ιάκωβος και Ιωσής και Σίμων και Ιούδας;
\par 56 και αι αδελφαί αυτού δεν είναι πάσαι παρ' ημίν; πόθεν λοιπόν εις τούτον ταύτα πάντα;
\par 57 Και εσκανδαλίζοντο εν αυτώ. Ο δε Ιησούς είπε προς αυτούς· Δεν είναι προφήτης άνευ τιμής ειμή εν τη πατρίδι αυτού και εν τη οικία αυτού.
\par 58 Και δεν έκαμεν εκεί πολλά θαύματα διά την απιστίαν αυτών.

\chapter{14}

\par Κατ' εκείνον τον καιρόν ήκουσεν Ηρώδης ο τετράρχης την φήμην του Ιησού
\par 2 και είπε προς τους δούλους αυτού· Ούτος είναι Ιωάννης ο Βαπτιστής· αυτός ηγέρθη από των νεκρών, και διά τούτο ενεργούσιν αι δυνάμεις εν αυτώ.
\par 3 Διότι ο Ηρώδης συλλαβών τον Ιωάννην έδεσεν αυτόν και έβαλεν εν φυλακή διά Ηρωδιάδα την γυναίκα Φιλίππου του αδελφού αυτού.
\par 4 Διότι έλεγε προς αυτόν ο Ιωάννης· Δεν σοι είναι συγκεχωρημένον να έχης αυτήν.
\par 5 Και θέλων να θανατώση αυτόν εφοβήθη τον όχλον, διότι είχον αυτόν ως προφήτην.
\par 6 Ότε δε ετελούντο τα γενέθλια του Ηρώδου, εχόρευσεν η θυγάτηρ της Ηρωδιάδος εν τω μέσω και ήρεσεν εις τον Ηρώδην·
\par 7 όθεν μεθ' όρκου ώμολόγησεν εις αυτήν να δώση ό,τι αν ζητήση.
\par 8 Η δε, παρακινηθείσα υπό της μητρός αυτής, Δος μοι, λέγει, εδώ επί πίνακι την κεφαλήν Ιωάννου του Βαπτιστού.
\par 9 Και ελυπήθη ο βασιλεύς, διά τους όρκους όμως και τους συγκαθημένους προσέταξε να δοθή,
\par 10 και πέμψας απεκεφάλισε τον Ιωάννην εν τη φυλακή.
\par 11 Και εφέρθη η κεφαλή αυτού επί πίνακι και εδόθη εις το κοράσιον, και έφερεν αυτήν προς την μητέρα αυτής.
\par 12 Και προσελθόντες οι μαθηταί αυτού εσήκωσαν το σώμα και έθαψαν αυτό, και ελθόντες απήγγειλαν τούτο εις τον Ιησούν.
\par 13 Και ακούσας ο Ιησούς ανεχώρησεν εκείθεν εν πλοίω εις έρημον τόπον κατ' ιδίαν· και ακούσαντες οι όχλοι ηκολούθησαν αυτόν πεζοί από των πόλεων.
\par 14 Και ότε ο Ιησούς, είδε πολύν όχλον και εσπλαγχνίσθη δι' αυτούς και εθεράπευσε τους αρρώστους αυτών.
\par 15 Ότε δε έγεινεν εσπέρα, προσήλθον προς αυτόν οι μαθηταί αυτού, λέγοντες· Έρημος είναι ο τόπος και η ώρα ήδη παρήλθεν· απόλυσον τους όχλους, διά να υπάγωσιν εις τας κώμας και αγοράσωσιν εις εαυτούς τροφάς.
\par 16 Ο δε Ιησούς είπε προς αυτούς· Δεν έχουσι χρείαν να υπάγωσι· δότε εις αυτούς σεις να φάγωσιν.
\par 17 Οι δε λέγουσι προς αυτόν· Δεν έχομεν εδώ ειμή πέντε άρτους και δύο οψάρια.
\par 18 Ο δε είπε· Φέρετέ μοι αυτά εδώ.
\par 19 Και προστάξας τους όχλους να καθήσωσιν επί τα χόρτα, και λαβών τους πέντε άρτους και τα δύο οψάρια, αναβλέψας εις τον ουρανόν ευλόγησε, και κόψας έδωκεν εις τους μαθητάς τους άρτους, οι δε μαθηταί εις τους όχλους.
\par 20 Και έφαγον πάντες και εχορτάσθησαν, και εσήκωσαν το περίσσευμα των κλασμάτων, δώδεκα κοφίνους πλήρεις.
\par 21 οι δε τρώγοντες ήσαν έως πεντακισχίλιοι άνδρες, εκτός γυναικών και παιδίων.
\par 22 Και ευθύς ηνάγκασεν ο Ιησούς τους μαθητάς αυτού να εμβώσιν εις το πλοίον και να υπάγωσι προ αυτού εις το πέραν, εωσού απολύση τους όχλους.
\par 23 Και αφού απέλυσε τους όχλους, ανέβη εις το όρος κατ' ιδίαν διά να προσευχηθή. Και ότε έγεινεν εσπέρα, ήτο μόνος εκεί.
\par 24 Το δε πλοίον ήτο ήδη εν τω μέσω της θαλάσσης, βασανιζόμενον υπό των κυμάτων· διότι ήτο εναντίος ο άνεμος.
\par 25 Εν δε τη τετάρτη φυλακή της νυκτός υπήγε προς αυτούς ο Ιησούς, περιπατών επί την θάλασσαν.
\par 26 Και ιδόντες αυτόν οι μαθηταί επί την θάλασσαν περιπατούντα, εταράχθησαν, λέγοντες ότι φάντασμα είναι, και από του φόβου έκραξαν.
\par 27 Ευθύς δε ελάλησε προς αυτούς ο Ιησούς λέγων· Θαρσείτε, εγώ είμαι· μη φοβείσθε.
\par 28 Αποκριθείς δε προς αυτόν ο Πέτρος είπε· Κύριε, εάν ήσαι συ, πρόσταξόν με να έλθω προς σε επί τα ύδατα.
\par 29 Ο δε είπεν, Ελθέ. Και καταβάς από του πλοίου ο Πέτρος περιεπάτησεν επί τα ύδατα, διά να έλθη προς τον Ιησούν.
\par 30 Βλέπων όμως τον άνεμον δυνατόν εφοβήθη, και αρχίσας να καταποντίζηται, έκραξε λέγων· Κύριε, σώσον με.
\par 31 Και ευθύς ο Ιησούς εκτείνας την χείρα επίασεν αυτόν και λέγει προς αυτόν· Ολιγόπιστε, εις τι εδίστασας;
\par 32 Και αφού εισήλθον εις το πλοίον, έπαυσεν ο άνεμος·
\par 33 οι δε εν τω πλοίω ελθόντες προσεκύνησαν αυτόν, λέγοντες· Αληθώς Θεού Υιός είσαι.
\par 34 Και διαπεράσαντες ήλθον εις την γην Γεννησαρέτ.
\par 35 Και γνωρίσαντες αυτόν οι άνθρωποι του τόπου εκείνου, απέστειλαν εις όλην την περίχωρον εκείνην και έφεραν προς αυτόν πάντας τους πάσχοντας,
\par 36 και παρεκάλουν αυτόν να εγγίσωσι μόνον το άκρον του ιματίου αυτού· και όσοι ήγγισαν ιατρεύθησαν.

\chapter{15}

\par Τότε προσέρχονται προς τον Ιησούν οι από Ιεροσολύμων γραμματείς και Φαρισαίοι, λέγοντες·
\par 2 Διά τι οι μαθηταί σου παραβαίνουσιν την παράδοσιν των πρεσβυτέρων; διότι δεν νίπτονται τας χείρας αυτών όταν τρώγωσιν άρτον.
\par 3 Ο δε αποκριθείς είπε προς αυτούς· Διά τι και σεις παραβαίνετε την εντολήν του Θεού διά την παράδοσίν σας;
\par 4 Διότι ο Θεός προσέταξε, λέγων· Τίμα τον πατέρα σου και την μητέρα· και, Ο κακολογών πατέρα ή μητέρα εξάπαντος να θανατόνηται·
\par 5 σεις όμως λέγετε· Όστις είπη προς τον πατέρα ή προς την μητέρα, Δώρον είναι ό,τι ήθελες ωφεληθή εξ εμού, αρκεί, και δύναται να μη τιμήση τον πατέρα αυτού ή την μητέρα αυτού·
\par 6 και ηκυρώσατε την εντολήν του Θεού διά την παράδοσίν σας.
\par 7 Υποκριταί, καλώς προεφήτευσε περί υμών ο Ησαΐας, λέγων·
\par 8 Ο λαός ούτος με πλησιάζει με το στόμα αυτών και με τα χείλη με τιμά, η δε καρδία αυτών μακράν απέχει απ' εμού.
\par 9 Εις μάτην δε με σέβονται, διδάσκοντες διδασκαλίας, εντάλματα ανθρώπων.
\par 10 Και προσκαλέσας τον όχλον, είπε προς αυτούς· Ακούετε και νοείτε.
\par 11 Δεν μολύνει τον άνθρωπον το εισερχόμενον εις το στόμα, αλλά το εξερχόμενον εκ του στόματος τούτο μολύνει τον άνθρωπον.
\par 12 Τότε προσελθόντες οι μαθηταί αυτού, είπον προς αυτόν· Εξεύρεις ότι οι Φαρισαίοι ακούσαντες τον λόγον τούτον εσκανδαλίσθησαν;
\par 13 Ο δε αποκριθείς είπε· Πάσα φυτεία, την οποίαν δεν εφύτευσεν ο Πατήρ μου ο ουράνιος, θέλει εκριζωθή.
\par 14 Αφήσατε αυτούς· είναι οδηγοί τυφλοί τυφλών· τυφλός δε τυφλόν εάν οδηγή, αμφότεροι εις βόθρον θέλουσι πέσει.
\par 15 Αποκριθείς δε ο Πέτρος είπε προς αυτόν· Εξήγησον εις ημάς την παραβολήν ταύτην.
\par 16 Και ο Ιησούς είπεν· Έτι και σεις ασύνετοι είσθε;
\par 17 Δεν εννοείτε έτι ότι παν το εισερχόμενον εις το στόμα καταβαίνει εις την κοιλίαν και εκβάλλεται εις αφεδρώνα;
\par 18 Τα δε εξερχόμενα εκ του στόματος εκ της καρδίας εξέρχονται, και εκείνα μολύνουσι τον άνθρωπον.
\par 19 Διότι εκ της καρδίας εξέρχονται διαλογισμοί πονηροί, φόνοι, μοιχείαι, πορνείαι, κλοπαί, ψευδομαρτυρίαι, βλασφημίαι.
\par 20 Ταύτα είναι τα μολύνοντα τον άνθρωπον· το δε να φάγη τις με ανίπτους χείρας δεν μολύνει τον άνθρωπον.
\par 21 Και εξελθών εκείθεν ο Ιησούς ανεχώρησεν εις τα μέρη Τύρου και Σιδώνος.
\par 22 Και ιδού, γυνή Χαναναία, εξελθούσα από των ορίων εκείνων, εκραύγασε προς αυτόν λέγουσα· Ελέησόν με, Κύριε, υιέ του Δαβίδ· η θυγάτηρ μου κακώς δαιμονίζεται.
\par 23 Ο δε δεν απεκρίθη προς αυτήν λόγον. Και προσελθόντες οι μαθηταί αυτού, παρεκάλουν αυτόν, λέγοντες· Απόλυσον αυτήν, διότι κράζει όπισθεν ημών.
\par 24 Ο δε αποκριθείς είπε· Δεν απεστάλην ειμή εις τα πρόβατα τα απολωλότα του οίκου Ισραήλ.
\par 25 Η δε ελθούσα προσεκύνει αυτόν, λέγουσα· Κύριε, βοήθει μοι.
\par 26 Ο δε αποκριθείς είπε· Δεν είναι καλόν να λάβη τις τον άρτον των τέκνων και να ρίψη εις τα κυνάρια.
\par 27 Η δε είπε· Ναι, Κύριε· αλλά και τα κυνάρια τρώγουσιν από των ψιχίων των πιπτόντων από της τραπέζης των κυρίων αυτών.
\par 28 Τότε αποκριθείς ο Ιησούς είπε προς αυτήν· Ω γύναι, μεγάλη σου η πίστις· ας γείνη εις σε ως θέλεις. Και ιατρεύθη η θυγάτηρ αυτής από της ώρας εκείνης.
\par 29 Και μεταβάς εκείθεν ο Ιησούς, ήλθε παρά την θάλασσαν της Γαλιλαίας, και αναβάς εις το όρος εκάθητο εκεί.
\par 30 Και ήλθον προς αυτόν όχλοι πολλοί έχοντες μεθ' εαυτών χωλούς, τυφλούς, κωφούς, κουλλούς και άλλους πολλούς· και έρριψαν αυτούς εις τους πόδας του Ιησού, και εθεράπευσεν αυτούς·
\par 31 ώστε οι όχλοι εθαύμασαν βλέποντες κωφούς λαλούντας, κουλλούς υγιείς, χωλούς περιπατούντας και τυφλούς βλέποντας· και εδόξασαν τον Θεόν του Ισραήλ.
\par 32 Ο δε Ιησούς, προσκαλέσας τους μαθητάς αυτού, είπε· Σπλαγχνίζομαι διά τον όχλον, διότι τρεις ήδη ημέρας μένουσι πλησίον μου και δεν έχουσι τι να φάγωσι· και να απολύσω αυτούς νήστεις δεν θέλω, μήποτε αποκάμωσι καθ' οδόν.
\par 33 Και λέγουσι προς αυτόν οι μαθηταί αυτού· Πόθεν εις ημάς εν τη ερημία άρτοι τόσοι, ώστε να χορτάσωμεν τόσον όχλον;
\par 34 Και λέγει προς αυτούς ο Ιησούς· Πόσους άρτους έχετε; οι δε είπον· Επτά, και ολίγα οψαράκια.
\par 35 Και προσέταξε τους όχλους να καθήσωσιν επί την γην.
\par 36 Και λαβών τους επτά άρτους και τα οψάρια, αφού ευχαρίστησεν, έκοψε και έδωκεν εις τους μαθητάς αυτού, οι δε μαθηταί εις τον όχλον.
\par 37 Και έφαγον πάντες και εχορτάσθησαν, και εσήκωσαν το περίσσευμα των κλασμάτων επτά σπυρίδας πλήρεις·
\par 38 οι δε τρώγοντες ήσαν τετρακισχίλιοι άνδρες εκτός γυναικών και παιδίων.
\par 39 Αφού δε απέλυσε τους όχλους, εισήλθεν εις το πλοίον και ήλθεν εις τα όρια Μαγδαλά.

\chapter{16}

\par Και προσελθόντες οι Φαρισαίοι και οι Σαδδουκαίοι, πειράζοντες αυτόν εζήτησαν να δείξη εις αυτούς σημείον εκ του ουρανού.
\par 2 Ο δε αποκριθείς είπε προς αυτούς· Όταν γείνη εσπέρα, λέγετε· Καλωσύνη· διότι κοκκινίζει ο ουρανός·
\par 3 και το πρωΐ· Σήμερον χειμών· διότι κοκκινίζει σκυθρωπάζων ο ουρανός. Υποκριταί, το μεν πρόσωπον του ουρανού εξεύρετε να διακρίνητε, τα δε σημεία των καιρών δεν δύνασθε;
\par 4 Γενεά πονηρά και μοιχαλίς σημείον ζητεί, και σημείον δεν θέλει δοθή εις αυτήν ειμή το σημείον Ιωνά του προφήτου. Και αφήσας αυτούς ανεχώρησε.
\par 5 Και ελθόντες οι μαθηταί αυτού εις το πέραν, ελησμόνησαν να λάβωσιν άρτους.
\par 6 Ο δε Ιησούς είπε προς αυτούς· Βλέπετε και προσέχετε από της ζύμης των Φαρισαίων και Σαδδουκαίων.
\par 7 Και εκείνοι διελογίζοντο εν εαυτοίς, λέγοντες ότι άρτους δεν ελάβομεν.
\par 8 Νοήσας δε ο Ιησούς, είπε προς αυτούς· Τι διαλογίζεσθε εν εαυτοίς, ολιγόπιστοι, ότι άρτους δεν ελάβετε;
\par 9 έτι δεν καταλαμβάνετε, ουδέ ενθυμείσθε τους πέντε άρτους των πεντακισχιλίων και πόσους κοφίνους ελάβετε;
\par 10 ουδέ τους επτά άρτους των τετρακισχιλίων και πόσας σπυρίδας ελάβετε;
\par 11 πως δεν καταλαμβάνετε ότι περί άρτου δεν σας είπον να προσέχητε από της ζύμης των Φαρισαίων και Σαδδουκαίων;
\par 12 Τότε ενόησαν, ότι δεν είπε να προσέχωσιν από της ζύμης του άρτου, αλλ' από της διδαχής των Φαρισαίων και Σαδδουκαίων.
\par 13 Ότε δε ήλθεν ο Ιησούς εις τα μέρη της Καισαρείας της Φιλίππου, ηρώτα τους μαθητάς αυτού, λέγων· Τίνα με λέγουσιν οι άνθρωποι ότι είμαι εγώ ο Υιός του ανθρώπου;
\par 14 Οι δε είπον· Άλλοι μεν Ιωάννην τον Βαπτιστήν, άλλοι δε Ηλίαν και άλλοι Ιερεμίαν ή ένα των προφητών.
\par 15 Λέγει προς αυτούς· Αλλά σεις τίνα με λέγετε ότι είμαι;
\par 16 Και αποκριθείς ο Σίμων Πέτρος είπε· Συ είσαι ο Χριστός ο Υιός του Θεού του ζώντος.
\par 17 Και αποκριθείς ο Ιησούς είπε προς αυτόν· Μακάριος είσαι, Σίμων, υιέ του Ιωνά, διότι σαρξ και αίμα δεν σοι απεκάλυψε τούτο, αλλ' ο Πατήρ μου ο εν τοις ουρανοίς.
\par 18 Και εγώ δε σοι λέγω ότι συ είσαι Πέτρος, και επί ταύτης της πέτρας θέλω οικοδομήσει την εκκλησίαν μου, και πύλαι άδου δεν θέλουσιν ισχύσει κατ' αυτής.
\par 19 Και θέλω σοι δώσει τα κλειδία της βασιλείας των ουρανών, και ό,τι εάν δέσης επί της γης, θέλει είσθαι δεδεμένον εν τοις ουρανοίς, και ό,τι εάν λύσης επί της γης, θέλει είσθαι λελυμένον εν τοις ουρανοίς.
\par 20 Τότε παρήγγειλεν εις τους μαθητάς αυτού να μη είπωσι προς μηδένα ότι αυτός είναι Ιησούς ο Χριστός.
\par 21 Από τότε ήρχισεν ο Ιησούς να δεικνύη εις τους μαθητάς αυτού ότι πρέπει να υπάγη εις Ιεροσόλυμα και να πάθη πολλά από των πρεσβυτέρων και αρχιερέων και γραμματέων, και να θανατωθή, και την τρίτην ημέραν να αναστηθή.
\par 22 Και παραλαβών αυτόν ο Πέτρος κατ' ιδίαν ήρχισε να επιτιμά αυτόν, λέγων· Γενού ίλεως εις σεαυτόν, Κύριε· δεν θέλει γείνει τούτο εις σε.
\par 23 Εκείνος δε στραφείς είπε προς τον Πέτρον· Ύπαγε οπίσω μου, Σατανά· σκάνδαλόν μου είσαι· διότι δεν φρονείς τα του Θεού, αλλά τα των ανθρώπων.
\par 24 Τότε ο Ιησούς είπε προς τους μαθητάς αυτού· Εάν τις θέλη να έλθη οπίσω μου, ας απαρνηθή εαυτόν και ας σηκώση τον σταυρόν αυτού και ας με ακολουθή.
\par 25 Διότι όστις θέλει να σώση την ζωήν αυτού, θέλει απολέσει αυτήν· και όστις απολέση την ζωήν αυτού ένεκεν εμού, θέλει ευρεί αυτήν.
\par 26 Επειδή τι ωφελείται άνθρωπος εάν τον κόσμον όλον κερδήση, την δε ψυχήν αυτού ζημιωθή; ή τι θέλει δώσει άνθρωπος εις ανταλλαγήν της ψυχής αυτού;
\par 27 Διότι μέλλει ο Υιός του ανθρώπου να έλθη εν τη δόξη του Πατρός αυτού μετά των αγγέλων αυτού, και τότε θέλει αποδώσει εις έκαστον κατά την πράξιν αυτού.
\par 28 Αληθώς σας λέγω, είναι τινές των εδώ ισταμένων, οίτινες δεν θέλουσι γευθή θάνατον, εωσού ίδωσι τον Υιόν του ανθρώπου ερχόμενον εν τη βασιλεία αυτού.

\chapter{17}

\par Και μεθ' ημέρας εξ παραλαμβάνει Ιησούς τον Πέτρον και Ιάκωβον και Ιωάννην τον αδελφόν αυτού και αναβιβάζει αυτούς εις όρος υψηλόν κατ' ιδίαν·
\par 2 και μετεμορφώθη έμπροσθεν αυτών, και έλαμψε το πρόσωπον αυτού ως ο ήλιος, τα δε ιμάτια αυτού έγειναν λευκά ως το φως.
\par 3 Και ιδού, εφάνησαν εις αυτούς Μωϋσής και Ηλίας συλλαλούντες μετ' αυτού.
\par 4 Αποκριθείς δε ο Πέτρος είπε προς τον Ιησούν· Κύριε, καλόν είναι να ήμεθα εδώ· εάν θέλης, ας κάμωμεν εδώ τρεις σκηνάς, διά σε μίαν και διά τον Μωϋσήν μίαν και μίαν διά τον Ηλίαν.
\par 5 Ενώ αυτός ελάλει έτι, ιδού, νεφέλη φωτεινή επεσκίασεν αυτούς, και ιδού, φωνή εκ της νεφέλης λέγουσα· Ούτος είναι ο Υιός μου ο αγαπητός, εις τον οποίον ευηρεστήθην· αυτού ακούετε.
\par 6 Και ακούσαντες οι μαθηταί έπεσον κατά πρόσωπον αυτών και εφοβήθησαν σφόδρα.
\par 7 Και προσελθών ο Ιησούς επίασεν αυτούς και είπεν· Εγέρθητε και μη φοβείσθε.
\par 8 Υψώσαντες δε τους οφθαλμούς αυτών, δεν είδον ουδένα ειμή τον Ιησούν μόνον.
\par 9 Και ενώ κατέβαινον από του όρους, παρήγγειλεν εις αυτούς ο Ιησούς, λέγων· Μη είπητε προς μηδένα το όραμα, εωσού ο Υιός του ανθρώπου αναστηθή εκ νεκρών.
\par 10 Και ηρώτησαν αυτόν οι μαθηταί αυτού, λέγοντες· Διά τι λοιπόν λέγουσιν οι γραμματείς ότι πρέπει να έλθη ο Ηλίας πρώτον;
\par 11 Ο δε Ιησούς αποκριθείς είπε προς αυτούς· Ο Ηλίας μεν έρχεται πρώτον και θέλει αποκαταστήσει πάντα·
\par 12 σας λέγω όμως ότι ήλθεν ήδη ο Ηλίας, και δεν εγνώρισαν αυτόν, αλλ' έπραξαν εις αυτόν όσα ηθέλησαν· ούτω και ο Υιός του ανθρώπου μέλλει να πάθη υπ' αυτών.
\par 13 Τότε ενόησαν οι μαθηταί, ότι περί Ιωάννου του Βαπτιστού είπε προς αυτούς.
\par 14 Και ότε ήλθον προς τον όχλον, επλησίασεν εις αυτόν άνθρωπός τις γονυπετών εις αυτόν και λέγων·
\par 15 Κύριε, ελέησόν μου τον υιόν, διότι σεληνιάζεται και κακώς πάσχει· διότι πολλάκις πίπτει εις το πυρ και πολλάκις εις το ύδωρ.
\par 16 Και έφερα αυτόν προς τους μαθητάς σου, αλλά δεν ηδυνήθησαν να θεραπεύσωσιν αυτόν.
\par 17 Αποκριθείς δε ο Ιησούς είπεν· Ω γενεά άπιστος και διεστραμμένη, έως πότε θέλω είσθαι μεθ' υμών; έως πότε θέλω υποφέρει υμάς; φέρετέ μοι αυτόν εδώ.
\par 18 Και επετίμησεν αυτόν ο Ιησούς, και εξήλθεν απ' αυτού το δαιμόνιον και εθεραπεύθη το παιδίον από της ώρας εκείνης.
\par 19 Τότε προσελθόντες οι μαθηταί προς τον Ιησούν κατ' ιδίαν, είπον· Διά τι ημείς δεν ηδυνήθημεν να εκβάλωμεν αυτό;
\par 20 Ο δε Ιησούς είπε προς αυτούς· Διά την απιστίαν σας. Διότι αληθώς σας λέγω, Εάν έχητε πίστιν ως κόκκον σινάπεως, θέλετε ειπεί προς το όρος τούτο, Μετάβηθι εντεύθεν εκεί, και θέλει μεταβή· και δεν θέλει είσθαι ουδέν αδύνατον εις εσάς.
\par 21 Τούτο δε το γένος δεν εξέρχεται, ειμή διά προσευχής και νηστείας.
\par 22 Και ενώ διέτριβον εν τη Γαλιλαία, είπε προς αυτούς ο Ιησούς· Μέλλει ο Υιός του ανθρώπου να παραδοθή εις χείρας ανθρώπων·
\par 23 και θέλουσι θανατώσει αυτόν, και την τρίτην ημέραν θέλει αναστηθή. Και ελυπήθησαν σφόδρα.
\par 24 Ότε δε ήλθον εις την Καπερναούμ, προσήλθον προς τον Πέτρον οι λαμβάνοντες τα δίδραχμα και είπον· Ο διδάσκαλός σας δεν πληρόνει τα δίδραχμα;
\par 25 Λέγει, Ναι. Και ότε εισήλθεν εις την οικίαν, προέλαβεν αυτόν ο Ιησούς λέγων· Τι σοι φαίνεται, Σίμων; οι βασιλείς της γης από τίνων λαμβάνουσι φόρους ή δασμόν; από των υιών αυτών ή από των ξένων;
\par 26 Λέγει προς αυτόν ο Πέτρος· Από των ξένων. Είπε προς αυτόν ο Ιησούς· Άρα ελεύθεροι είναι οι υιοί.
\par 27 Πλην διά να μη σκανδαλίσωμεν αυτούς, ύπαγε εις την θάλασσαν και ρίψον άγκιστρον και το πρώτον οψάριον, το οποίον αναβή, λάβε, και ανοίξας το στόμα αυτού θέλεις ευρεί στατήρα· εκείνον λαβών δος εις αυτούς δι' εμέ και σε.

\chapter{18}

\par Εν εκείνη τη ώρα ήλθον οι μαθηταί προς τον Ιησούν, λέγοντες· Τις άρα είναι μεγαλήτερος εν τη βασιλεία των ουρανών;
\par 2 Και προσκαλέσας ο Ιησούς παιδίον, έστησεν αυτό εν τω μέσω αυτών
\par 3 και είπεν· Αληθώς σας λέγω, εάν δεν επιστρέψητε και γείνητε ως τα παιδία, δεν θέλετε εισέλθει εις την βασιλείαν των ουρανών.
\par 4 Όστις λοιπόν ταπεινώση εαυτόν ως το παιδίον τούτο, ούτος είναι ο μεγαλήτερος εν τη βασιλεία των ουρανών.
\par 5 Και όστις δεχθή εν τοιούτον παιδίον εις το όνομά μου, εμέ δέχεται·
\par 6 όστις όμως σκανδαλίση ένα των μικρών τούτων των πιστευόντων εις εμέ, συμφέρει εις αυτόν να κρεμασθή μύλου πέτρα επί τον τράχηλον αυτού και να καταποντισθή εις το πέλαγος της θαλάσσης.
\par 7 Ουαί εις τον κόσμον διά τα σκάνδαλα· διότι είναι ανάγκη να έλθωσι τα σκάνδαλα· πλην ουαί εις τον άνθρωπον εκείνον, διά του οποίου το σκάνδαλον έρχεται.
\par 8 Και εάν η χειρ σου ή ο πους σου σε σκανδαλίζη, έκκοψον αυτό και ρίψον από σού· καλήτερον σοι είναι να εισέλθης εις την ζωήν χωλός ή κουλλός, παρά έχων δύο χείρας ή δύο πόδας να ριφθής εις το πυρ το αιώνιον.
\par 9 Και εάν ο οφθαλμός σου σε σκανδαλίζη, έκβαλε αυτόν και ρίψον από σού· καλήτερον σοι είναι μονόφθαλμος να εισέλθης εις την ζωήν, παρά έχων δύο οφθαλμούς να ριφθής εις την γέενναν του πυρός.
\par 10 Προσέχετε μη καταφρονήσητε ένα των μικρών τούτων· διότι σας λέγω ότι οι άγγελοι αυτών εν τοις ουρανοίς διαπαντός βλέπουσι το πρόσωπον του Πατρός μου του εν ουρανοίς.
\par 11 Επειδή ο Υιός του ανθρώπου ήλθε διά να σώση το απολωλός.
\par 12 Τι σας φαίνεται; εάν άνθρωπός τις έχη εκατόν πρόβατα και πλανηθή εν εξ αυτών, δεν αφίνει τα ενενήκοντα εννέα και υπάγων επί τα όρη, ζητεί το πλανώμενον;
\par 13 Και εάν συμβή να εύρη αυτό, αληθώς σας λέγω ότι χαίρει δι' αυτό μάλλον παρά διά τα ενενήκοντα εννέα τα μη πεπλανημένα.
\par 14 Ούτω δεν είναι θέλημα έμπροσθεν του Πατρός σας του εν ουρανοίς να απολεσθή εις των μικρών τούτων.
\par 15 Εάν δε αμαρτήση εις σε ο αδελφός σου, ύπαγε και έλεγξον αυτόν μεταξύ σου και αυτού μόνου· εάν σου ακούση, εκέρδησας τον αδελφόν σου·
\par 16 εάν όμως δεν ακούση, παράλαβε μετά σου έτι ένα ή δύο, διά να βεβαιωθή πας λόγος επί στόματος δύο μαρτύρων ή τριών.
\par 17 Και εάν παρακούση αυτών, ειπέ τούτο προς την εκκλησίαν· αλλ' εάν και της εκκλησίας παρακούση, ας είναι εις σε ως ο εθνικός και ο τελώνης.
\par 18 Αληθώς σας λέγω, Όσα εάν δέσητε επί της γης, θέλουσιν είσθαι δεδεμένα εν τω ουρανώ, και όσα εάν λύσητε επί της γης, θέλουσιν είσθαι λελυμένα εν τω ουρανώ.
\par 19 Πάλιν σας λέγω ότι εάν δύο από σας συμφωνήσωσιν επί της γης περί παντός πράγματος, περί του οποίου ήθελον κάμει αίτησιν, θέλει γείνει εις αυτούς παρά του Πατρός μου του εν ουρανοίς.
\par 20 Διότι όπου είναι δύο ή τρεις συνηγμένοι εις το όνομά μου, εκεί είμαι εγώ εν τω μέσω αυτών.
\par 21 Τότε προσελθών προς αυτόν ο Πέτρος, είπε· Κύριε, ποσάκις αν αμαρτήση εις εμέ ο αδελφός μου και θέλω συγχωρήσει αυτόν; έως επτάκις;
\par 22 Λέγει προς αυτόν ο Ιησούς· Δεν σοι λέγω έως επτάκις, αλλ' έως εβδομηκοντάκις επτά.
\par 23 Διά τούτο η βασιλεία των ουρανών ωμοιώθη με άνθρωπον βασιλέα, όστις ηθέλησε να θεωρήση λογαριασμόν μετά των δούλων αυτού.
\par 24 Και ότε ήρχισε να θεωρή, εφέρθη προς αυτόν εις οφειλέτης μυρίων ταλάντων.
\par 25 Και επειδή δεν είχε να αποδώση, προσέταξεν ο κύριος αυτού να πωληθή αυτός και η γυνή αυτού και τα τέκνα και πάντα όσα είχε, και να αποδοθή το οφειλόμενον.
\par 26 Πεσών λοιπόν ο δούλος προσεκύνει αυτόν, λέγων· Κύριε, μακροθύμησον εις εμέ, και πάντα θέλω σοι αποδώσει.
\par 27 Σπλαγχνισθείς δε ο κύριος του δούλου εκείνου, απέλυσεν αυτόν και το δάνειον αφήκεν εις αυτόν.
\par 28 Αφού όμως εξήλθεν ο δούλος εκείνος, εύρεν ένα των συνδούλων αυτού, όστις εχρεώστει εις αυτόν εκατόν δηνάρια, και πιάσας αυτόν έπνιγε, λέγων· Απόδος μοι ό,τι χρεωστείς.
\par 29 Πεσών λοιπόν ο σύνδουλος αυτού εις τους πόδας αυτού, παρεκάλει αυτόν λέγων· Μακροθύμησον εις εμέ, και πάντα θέλω σοι αποδώσει.
\par 30 Εκείνος όμως δεν ήθελεν, αλλ' απελθών έβαλεν αυτόν εις φυλακήν, εωσού αποδώση το οφειλόμενον.
\par 31 Ιδόντες δε οι σύνδουλοι αυτού τα γενόμενα, ελυπήθησαν σφόδρα και ελθόντες εφανέρωσαν προς τον κύριον αυτών πάντα τα γενόμενα.
\par 32 Τότε προσκαλέσας αυτόν ο κύριος αυτού, λέγει προς αυτόν· Δούλε πονηρέ, παν το χρέος εκείνο σοι αφήκα, επειδή με παρεκάλεσας·
\par 33 δεν έπρεπε και συ να ελεήσης τον σύνδουλόν σου, καθώς και εγώ σε ηλέησα;
\par 34 Και οργισθείς ο κύριος αυτού παρέδωκεν αυτόν εις τους βασανιστάς, εωσού αποδώση παν το οφειλόμενον εις αυτόν.
\par 35 Ούτω και ο Πατήρ μου ο επουράνιος θέλει κάμει εις εσάς, εάν δεν συγχωρήσητε εκ καρδίας σας έκαστος εις τον αδελφόν αυτού τα πταίσματα αυτών.

\chapter{19}

\par Και ότε ετελείωσεν ο Ιησούς τους λόγους τούτους, ανεχώρησεν από της Γαλιλαίας και ήλθεν εις τα όρια της Ιουδαίας πέραν του Ιορδάνου.
\par 2 Και ηκολούθησαν αυτόν όχλοι πολλοί, και εθεράπευσεν αυτούς εκεί.
\par 3 Και ήλθον προς αυτόν οι Φαρισαίοι, πειράζοντες αυτόν και λέγοντες προς αυτόν· Συγχωρείται εις τον άνθρωπον να χωρισθή την γυναίκα αυτού διά πάσαν αιτίαν;
\par 4 Ο δε αποκριθείς είπε προς αυτούς· Δεν ανεγνώσατε ότι ο πλάσας απ' αρχής άρσεν και θήλυ έπλασεν αυτούς
\par 5 και είπεν, Ένεκεν τούτου θέλει αφήσει άνθρωπος τον πατέρα και την μητέρα και θέλει προσκολληθή εις την γυναίκα αυτού, και θέλουσιν είσθαι οι δύο εις σάρκα μίαν;
\par 6 Ώστε δεν είναι πλέον δύο, αλλά μία σαρξ. Εκείνο λοιπόν το οποίον ο Θεός συνέζευξεν, άνθρωπος ας μη χωρίζη.
\par 7 Λέγουσι προς αυτόν· Διά τι λοιπόν ο Μωϋσής προσέταξε να δώση έγγραφον διαζυγίου και να χωρισθή αυτήν;
\par 8 Λέγει προς αυτούς· Διότι ο Μωϋσής διά την σκληροκαρδίαν σας συνεχώρησεν εις εσάς να χωρίζησθε τας γυναίκάς σας· απ' αρχής όμως δεν έγεινεν ούτω.
\par 9 Σας λέγω δε ότι όστις χωρισθή την γυναίκα αυτού εκτός διά πορνείαν και νυμφευθή άλλην, γίνεται μοιχός· και όστις νυμφευθή γυναίκα κεχωρισμένην, γίνεται μοιχός.
\par 10 Λέγουσι προς αυτόν οι μαθηταί αυτού· Εάν ούτως έχη η υποχρέωσις του ανδρός προς την γυναίκα, δεν συμφέρει να νυμφευθή.
\par 11 Ο δε είπε προς αυτούς· Δεν δύνανται πάντες να δεχθώσι τον λόγον τούτον, αλλ' εις όσους είναι δεδομένον.
\par 12 Διότι είναι ευνούχοι, οίτινες εκ κοιλίας μητρός εγεννήθησαν ούτω, και είναι ευνούχοι, οίτινες ευνουχίσθησαν υπό των ανθρώπων, και είναι ευνούχοι, οίτινες ευνούχισαν εαυτούς διά την βασιλείαν των ουρανών. Όστις δύναται να δεχθή τούτο, ας δεχθή.
\par 13 Τότε εφέρθησαν προς αυτόν παιδία, διά να επιθέση τας χείρας επ' αυτά και να ευχηθή· οι δε μαθηταί επέπληξαν αυτά.
\par 14 Πλην ο Ιησούς είπεν· Αφήσατε τα παιδία και μη εμποδίζετε αυτά να έλθωσι προς εμέ· διότι των τοιούτων είναι η βασιλεία των ουρανών.
\par 15 Και αφού επέθηκεν επ' αυτά τας χείρας, ανεχώρησεν εκείθεν.
\par 16 Και ιδού, προσελθών τις είπε προς αυτόν· Διδάσκαλε αγαθέ, τι καλόν να πράξω διά να έχω ζωήν αιώνιον;
\par 17 Ο δε είπε προς αυτόν· Τι με λέγεις αγαθόν; ουδείς αγαθός ειμή εις, ο Θεός. Αλλ' εάν θέλης να εισέλθης εις την ζωήν, φύλαξον τας εντολάς.
\par 18 Λέγει προς αυτόν· Ποίας; Και ο Ιησούς είπε· Το μη φονεύσης, μη μοιχεύσης, μη κλέψης, μη ψευδομαρτυρήσης,
\par 19 τίμα τον πατέρα σου και την μητέρα, και θέλεις αγαπά τον πλησίον σου ως σεαυτόν.
\par 20 Λέγει προς αυτόν ο νεανίσκος· Πάντα ταύτα εφύλαξα εκ νεότητός μου· τι μοι λείπει έτι;
\par 21 Είπε προς αυτόν ο Ιησούς· Εάν θέλης να ήσαι τέλειος, ύπαγε, πώλησον τα υπάρχοντά σου και δος εις πτωχούς, και θέλεις έχει θησαυρόν εν ουρανώ, και ελθέ, ακολούθει μοι.
\par 22 Ακούσας δε ο νεανίσκος τον λόγον, ανεχώρησε λυπούμενος· διότι είχε κτήματα πολλά.
\par 23 Και ο Ιησούς είπε προς τους μαθητάς αυτού· Αληθώς σας λέγω ότι δυσκόλως θέλει εισέλθει πλούσιος εις την βασιλείαν των ουρανών.
\par 24 Και πάλιν σας λέγω, Ευκολώτερον είναι να περάση κάμηλος διά τρυπήματος βελόνης παρά πλούσιος να εισέλθη εις την βασιλείαν του Θεού.
\par 25 Ακούσαντες δε οι μαθηταί αυτού εξεπλήττοντο σφόδρα, λέγοντες· Τις λοιπόν δύναται να σωθή;
\par 26 Εμβλέψας δε ο Ιησούς, είπε προς αυτούς· Παρά ανθρώποις τούτο αδύνατον είναι, παρά τω Θεώ όμως τα πάντα είναι δυνατά.
\par 27 Τότε αποκριθείς ο Πέτρος, είπε προς αυτόν· Ιδού, ημείς αφήκαμεν πάντα και σοι ηκολουθήσαμεν· τι λοιπόν θέλει είσθαι εις ημάς;
\par 28 Ο δε Ιησούς είπε προς αυτούς· Αληθώς σας λέγω ότι σεις οι ακολουθήσαντές μοι, εν τη παλιγγενεσία, όταν καθήση ο Υιός του ανθρώπου επί του θρόνου της δόξης αυτού, θέλετε καθήσει και σεις επί δώδεκα θρόνους κρίνοντες τας δώδεκα φυλάς του Ισραήλ.
\par 29 Και πας όστις αφήκεν οικίας ή αδελφούς ή αδελφάς ή πατέρα ή μητέρα ή γυναίκα ή τέκνα ή αγρούς ένεκεν του ονόματός μου, εκατονταπλάσια θέλει λάβει και ζωήν αιώνιον θέλει κληρονομήσει.
\par 30 Πολλοί όμως πρώτοι θέλουσιν είσθαι έσχατοι και έσχατοι πρώτοι.

\chapter{20}

\par Διότι η βασιλεία των ουρανών είναι ομοία με άνθρωπον οικοδεσπότην, όστις εξήλθεν άμα τω πρωΐ διά να μισθώση εργάτας διά τον αμπελώνα αυτού.
\par 2 Αφού δε συνεφώνησε μετά των εργατών προς εν δηνάριον την ημέραν, απέστειλεν αυτούς εις τον αμπελώνα αυτού.
\par 3 Και εξελθών περί την τρίτην ώραν, είδεν άλλους ισταμένους εν τη αγορά αργούς,
\par 4 και προς εκείνους είπεν· Υπάγετε και σεις εις τον αμπελώνα, και ό,τι είναι δίκαιον θέλω σας δώσει. Και εκείνοι υπήγον.
\par 5 Πάλιν εξελθών περί την έκτην και ενάτην ώραν, έκαμεν ωσαύτως.
\par 6 Περί δε την ενδεκάτην ώραν εξελθών εύρεν άλλους ισταμένους αργούς, και λέγει προς αυτούς· Διά τι ίστασθε εδώ όλην την ημέραν αργοί;
\par 7 Λέγουσι προς αυτόν· Διότι ουδείς εμίσθωσεν ημάς. Λέγει προς αυτούς· Υπάγετε και σεις εις τον αμπελώνα, και ό,τι είναι δίκαιον θέλετε λάβει.
\par 8 Αφού δε έγεινεν εσπέρα, λέγει ο κύριος του αμπελώνος προς τον επίτροπον αυτού· Κάλεσον τους εργάτας και απόδος εις αυτούς τον μισθόν, αρχίσας από των εσχάτων έως των πρώτων.
\par 9 Και ελθόντες οι περί την ενδεκάτην ώραν μισθωθέντες, έλαβον ανά εν δηνάριον.
\par 10 Ελθόντες δε οι πρώτοι, ενόμισαν ότι θέλουσι λάβει πλειότερα, έλαβον όμως και αυτοί ανά εν δηνάριον.
\par 11 Και λαβόντες εγόγγυζον κατά του οικοδεσπότου,
\par 12 λέγοντες ότι, Ούτοι οι έσχατοι μίαν ώραν έκαμον, και έκαμες αυτούς ίσους με ημάς, οίτινες εβαστάσαμεν το βάρος της ημέρας και τον καύσωνα.
\par 13 Ο δε αποκριθείς είπε προς ένα εξ αυτών· Φίλε, δεν σε αδικώ· δεν συνεφώνησας εν δηνάριον μετ' εμού;
\par 14 λάβε το σον και ύπαγε· θέλω δε να δώσω εις τούτον τον έσχατον ως και εις σε.
\par 15 Η δεν έχω την εξουσίαν να κάμω ό,τι θέλω εις τα εμά; ή ο οφθαλμός σου είναι πονηρός διότι εγώ είμαι αγαθός;
\par 16 Ούτω θέλουσιν είσθαι οι έσχατοι πρώτοι και οι πρώτοι έσχατοι· διότι πολλοί είναι οι κεκλημένοι, ολίγοι δε οι εκλεκτοί.
\par 17 Και αναβαίνων ο Ιησούς εις Ιεροσόλυμα, παρέλαβε τους δώδεκα μαθητάς κατ' ιδίαν εν τη οδώ και είπε προς αυτούς.
\par 18 Ιδού, αναβαίνομεν εις Ιεροσόλυμα, και ο Υιός του ανθρώπου θέλει παραδοθή εις τους αρχιερείς και γραμματείς και θέλουσι καταδικάσει αυτόν εις θάνατον,
\par 19 και θέλουσι παραδώσει αυτόν εις τα έθνη διά να εμπαίξωσι και μαστιγώσωσι και σταυρώσωσι, και τη τρίτη ημέρα θέλει αναστηθή.
\par 20 Τότε προσήλθε προς αυτόν η μήτηρ των υιών του Ζεβεδαίου μετά των υιών αυτής, προσκυνούσα και ζητούσα τι παρ' αυτού.
\par 21 Ο δε είπε προς αυτήν· Τι θέλεις; Λέγει προς αυτόν· Ειπέ να καθήσωσιν ούτοι οι δύο υιοί μου εις εκ δεξιών σου και εις εξ αριστερών εν τη βασιλεία σου.
\par 22 Αποκριθείς δε ο Ιησούς είπε· Δεν εξεύρετε τι ζητείτε. Δύνασθε να πίητε το ποτήριον, το οποίον εγώ μέλλω να πίω, και να βαπτισθήτε το βάπτισμα, το οποίον εγώ βαπτίζομαι; Λέγουσι προς αυτόν· Δυνάμεθα.
\par 23 Και λέγει προς αυτούς· το μεν ποτήριόν μου θέλετε πίει; και το βάπτισμα το οποίον εγώ βαπτίζομαι θέλετε βαπτισθή· το να καθήσητε όμως εκ δεξιών μου και εξ αριστερών μου δεν είναι εμού να δώσω, ειμή εις όσους είναι ητοιμασμένον υπό του Πατρός μου.
\par 24 Και ακούσαντες οι δέκα ηγανάκτησαν περί των δύο αδελφών.
\par 25 Ο δε Ιησούς προσκαλέσας αυτούς, είπεν· Εξεύρετε ότι οι άρχοντες των εθνών κατακυριεύουσιν αυτά και οι μεγάλοι κατεξουσιάζουσιν αυτά.
\par 26 Ούτως όμως δεν θέλει είσθαι εν υμίν, αλλ' όστις θέλει να γείνη μέγας εν υμίν, ας ήναι υπηρέτης υμών,
\par 27 και όστις θέλη να ήναι πρώτος εν υμίν, ας ήναι δούλος υμών·
\par 28 καθώς ο Υιός του ανθρώπου δεν ήλθε διά να υπηρετηθή, αλλά διά να υπηρετήση και να δώση την ζωήν αυτού λύτρον αντί πολλών.
\par 29 Και ενώ εξήρχοντο από της Ιεριχώ, ηκολούθησεν αυτόν όχλος πολύς.
\par 30 Και ιδού, δύο τυφλοί καθήμενοι παρά την οδόν, ακούσαντες ότι ο Ιησούς διαβαίνει, έκραξαν λέγοντες· Ελέησον ημάς, Κύριε, υιέ του Δαβίδ.
\par 31 Ο δε όχλος επέπληξεν αυτούς διά να σιωπήσωσιν· αλλ' εκείνοι έκραζον δυνατώτερα, λέγοντες· Ελέησον ημάς, Κύριε, υιέ του Δαβίδ.
\par 32 Και σταθείς ο Ιησούς, έκραξεν αυτούς και είπε· Τι θέλετε να σας κάμω;
\par 33 Λέγουσι προς αυτόν· Κύριε, να ανοιχθώσιν οι οφθαλμοί ημών.
\par 34 Και ο Ιησούς σπλαγχνισθείς ήγγισε τους οφθαλμούς αυτών· και ευθύς ανέβλεψαν αυτών οι οφθαλμοί, και ηκολούθησαν αυτόν.

\chapter{21}

\par Και ότε επλησίασαν εις Ιεροσόλυμα και ήλθον εις Βηθφαγή προς το όρος των ελαιών, τότε ο Ιησούς απέστειλε δύο μαθητάς,
\par 2 λέγων προς αυτούς· Υπάγετε εις την κώμην την απέναντι υμών, και ευθύς θέλετε ευρεί όνον δεδεμένην και πωλάριον μετ' αυτής· λύσατε και φέρετέ μοι.
\par 3 Και εάν τις σας είπη τι, θέλετε ειπεί ότι ο Κύριος έχει χρείαν αυτών· και ευθύς θέλει αποστείλει αυτά.
\par 4 Τούτο δε όλον έγεινε διά να πληρωθή το ρηθέν διά του προφήτου, λέγοντος·
\par 5 Είπατε προς την θυγατέρα Σιών, Ιδού, ο βασιλεύς σου έρχεται προς σε πραΰς και καθήμενος επί όνου και πώλου υιού υποζυγίου.
\par 6 Πορευθέντες δε οι μαθηταί και ποιήσαντες καθώς προσέταξεν αυτούς ο Ιησούς,
\par 7 έφεραν την όνον και το πωλάριον, και έβαλον επάνω αυτών τα ιμάτια αυτών και επεκάθισαν αυτόν επάνω αυτών.
\par 8 Ο δε περισσότερος όχλος έστρωσαν τα ιμάτια εαυτών εις την οδόν, άλλοι δε έκοπτον κλάδους από των δένδρων και έστρωνον εις την οδόν.
\par 9 Οι δε όχλοι οι προπορευόμενοι και οι ακολουθούντες έκραζον, λέγοντες· Ωσαννά τω υιώ Δαβίδ· ευλογημένος ο ερχόμενος εν ονόματι Κυρίου· Ωσαννά εν τοις υψίστοις.
\par 10 Και ότε εισήλθεν εις Ιεροσόλυμα, εσείσθη πάσα η πόλις, λέγουσα· Τις είναι ούτος;
\par 11 Οι δε όχλοι έλεγον· Ούτος είναι Ιησούς ο προφήτης ο από Ναζαρέτ της Γαλιλαίας.
\par 12 Και εισήλθεν ο Ιησούς εις το ιερόν του Θεού και εξέβαλε πάντας τους πωλούντας και αγοράζοντας εν τω ιερώ, και τας τραπέζας των αργυραμοιβών ανέτρεψε και τα καθίσματα των πωλούντων τας περιστεράς,
\par 13 και λέγει προς αυτούς· Είναι γεγραμμένον, Ο οίκός μου οίκος προσευχής θέλει ονομάζεσθαι, σεις δε εκάμετε αυτόν σπήλαιον ληστών.
\par 14 Και προσήλθον προς αυτόν τυφλοί και χωλοί εν τω ιερώ και εθεράπευσεν αυτούς.
\par 15 Ιδόντες δε οι αρχιερείς και οι γραμματείς τα θαυμάσια, τα οποία έκαμε, και τους παίδας κράζοντας εν τω ιερώ και λέγοντας, Ωσαννά τω υιώ Δαβίδ, ηγανάκτησαν
\par 16 και είπον προς αυτόν· Ακούεις τι λέγουσιν ούτοι; Ο δε Ιησούς λέγει προς αυτούς· Ναί· ποτέ δεν ανεγνώσατε ότι εκ στόματος νηπίων και θηλαζόντων ητοίμασας αίνεσιν;
\par 17 Και αφήσας αυτούς εξήλθεν έξω της πόλεως εις Βηθανίαν και διενυκτέρευσεν εκεί.
\par 18 Ότε δε το πρωΐ επέστρεφεν εις την πόλιν, επείνασε·
\par 19 και ιδών μίαν συκήν επί της οδού, ήλθε προς αυτήν και ουδέν ηύρεν επ' αυτήν ειμή φύλλα μόνον, και λέγει προς αυτήν· Να μη γείνη πλέον από σου καρπός εις τον αιώνα. Και παρευθύς εξηράνθη η συκή.
\par 20 Και ιδόντες οι μαθηταί, εθαύμασαν λέγοντες· Πως παρευθύς εξηράνθη συκή;
\par 21 Αποκριθείς δε ο Ιησούς είπε προς αυτούς· Αληθώς σας λέγω, εάν έχητε πίστιν και δεν διστάσητε, ουχί μόνον το της συκής θέλετε κάμει, αλλά και εις το όρος τούτο αν είπητε, Σηκώθητι και ρίφθητι εις την θάλασσαν, θέλει γείνει·
\par 22 και πάντα όσα αν ζητήσητε εν τη προσευχή έχοντες πίστιν θέλετε λάβει.
\par 23 Και ότε ήλθεν εις το ιερόν, προσήλθον προς αυτόν, ενώ εδίδασκεν οι αρχιερείς και οι πρεσβύτεροι του λαού, λέγοντες· Εν ποία εξουσία πράττεις ταύτα, και τις σοι έδωκε την εξουσίαν ταύτην;
\par 24 Αποκριθείς δε ο Ιησούς, είπε προς αυτούς· Θέλω σας ερωτήσει και εγώ ένα λόγον, τον οποίον εάν μοι είπητε, και εγώ θέλω σας ειπεί εν ποία εξουσία πράττω ταύτα·
\par 25 το βάπτισμα του Ιωάννου πόθεν ήτο, εξ ουρανού ή εξ ανθρώπων; Και εκείνοι διελογίζοντο καθ' εαυτούς λέγοντες· Εάν είπωμεν, Εξ ουρανού, θέλει ειπεί προς ημάς, Διά τι λοιπόν δεν επιστεύσατε εις αυτόν·
\par 26 εάν δε είπωμεν, Εξ ανθρώπων, φοβούμεθα τον όχλον· διότι πάντες έχουσι τον Ιωάννην ως προφήτην.
\par 27 Και αποκριθέντες προς τον Ιησούν, είπον· Δεν εξεύρομεν. Είπε προς αυτούς και αυτός· Ουδέ εγώ λέγω προς υμάς εν ποία εξουσία πράττω ταύτα.
\par 28 Αλλά τι σας φαίνεται; Άνθρωπος τις είχε δύο υιούς, και ελθών προς τον πρώτον είπε· Τέκνον, ύπαγε σήμερον εργάζου εν τω αμπελώνι μου.
\par 29 Ο δε αποκριθείς είπε· Δεν θέλω· ύστερον όμως μετανοήσας υπήγε.
\par 30 Και ελθών προς τον δεύτερον είπεν ωσαύτως. Και εκείνος αποκριθείς είπεν· Εγώ υπάγω, κύριε· και δεν υπήγε.
\par 31 Τις εκ των δύο έκαμε το θέλημα του πατρός; Λέγουσι προς αυτόν· Ο πρώτος. Λέγει προς αυτούς ο Ιησούς· Αληθώς σας λέγω ότι οι τελώναι και αι πόρναι υπάγουσι πρότερον υμών εις την βασιλείαν του Θεού.
\par 32 Διότι ήλθε προς υμάς ο Ιωάννης εν οδώ δικαιοσύνης, και δεν επιστεύσατε εις αυτόν· οι τελώναι όμως και αι πόρναι επίστευσαν εις αυτόν· σεις δε ιδόντες δεν μετεμελήθητε ύστερον, ώστε να πιστεύσητε εις αυτόν.
\par 33 Άλλην παραβολήν ακούσατε. Ήτο άνθρωπός τις οικοδεσπότης, όστις εφύτευσεν αμπελώνα και περιέβαλεν εις αυτόν φραγμόν και έσκαψεν εν αυτώ ληνόν και ωκοδόμησε πύργον, και εμίσθωσεν αυτόν εις γεωργούς και απεδήμησεν.
\par 34 Ότε δε επλησίασεν ο καιρός των καρπών, απέστειλε τους δούλους αυτού προς τους γεωργούς διά να λάβωσι τους καρπούς αυτού.
\par 35 Και πιάσαντες οι γεωργοί τους δούλους αυτού, άλλον μεν έδειραν, άλλον δε εφόνευσαν, άλλον δε ελιθοβόλησαν.
\par 36 Πάλιν απέστειλεν άλλους δούλους πλειοτέρους των πρώτων, και έκαμον εις αυτούς ωσαύτως.
\par 37 Ύστερον δε απέστειλε προς αυτούς τον υιόν αυτού λέγων· Θέλουσιν εντραπή τον υιόν μου.
\par 38 Αλλ' οι γεωργοί, ιδόντες τον υιόν, είπον προς αλλήλους· Ούτος είναι ο κληρονόμος· έλθετε, ας φονεύσωμεν αυτόν και ας κατακρατήσωμεν την κληρονομίαν αυτού.
\par 39 Και πιάσαντες αυτόν, εξέβαλον έξω του αμπελώνος και εφόνευσαν.
\par 40 Όταν λοιπόν έλθη ο κύριος του αμπελώνος, τι θέλει κάμει εις τους γεωργούς εκείνους;
\par 41 Λέγουσι προς αυτόν· Κακούς κακώς θέλει απολέσει αυτούς, και τον αμπελώνα θέλει μισθώσει εις άλλους γεωργούς, οίτινες θέλουσιν αποδώσει εις αυτόν τους καρπούς εν τοις καιροίς αυτών.
\par 42 Λέγει προς αυτούς ο Ιησούς· Ποτέ δεν ανεγνώσατε εν ταις γραφαίς, Ο λίθος, τον οποίον απεδοκίμασαν οι οικοδομούντες, ούτος έγεινε κεφαλή γωνίας· παρά Κυρίου έγεινεν αύτη και είναι θαυμαστή εν οφθαλμοίς υμών;
\par 43 Διά τούτο λέγω προς υμάς ότι θέλει αφαιρεθή αφ' υμών η βασιλεία του Θεού και θέλει δοθή εις έθνος κάμνον τους καρπούς αυτής·
\par 44 και όστις πέση επί τον λίθον τούτον θέλει συντριφθή· εις όντινα δε επιπέση, θέλει κατασυντρίψει αυτόν.
\par 45 Και ακούσαντες οι αρχιερείς και οι Φαρισαίοι τας παραβολάς αυτού, ενόησαν ότι περί αυτών λέγει·
\par 46 και ζητούντες να πιάσωσιν αυτόν, εφοβήθησαν τους όχλους, επειδή είχον αυτόν ως προφήτην.

\chapter{22}

\par Και αποκριθείς ο Ιησούς πάλιν είπε προς αυτούς διά παραβολών, λέγων·
\par 2 Ωμοιώθη η βασιλεία των ουρανών με άνθρωπον βασιλέα, όστις έκαμε γάμους εις τον υιόν αυτού·
\par 3 και απέστειλε τους δούλους αυτού να καλέσωσι τους προσκεκλημένους εις τους γάμους, και δεν ήθελον να έλθωσι.
\par 4 Πάλιν απέστειλεν άλλους δούλους, λέγων· Είπατε προς τους προσκεκλημένους· Ιδού, το γεύμα μου ητοίμασα, οι ταύροι μου και τα θρεπτά είναι εσφαγμένα και πάντα είναι έτοιμα· έλθετε εις τους γάμους.
\par 5 Εκείνοι όμως αμελήσαντες απήλθον, ο μεν εις τον αγρόν αυτού, ο δε εις το εμπόριον αυτού·
\par 6 οι δε λοιποί πιάσαντες τους δούλους αυτού ύβρισαν και εφόνευσαν.
\par 7 Ακούσας δε ο βασιλεύς ωργίσθη, και πέμψας τα στρατεύματα αυτού απώλεσε τους φονείς εκείνους και την πόλιν αυτών κατέκαυσε.
\par 8 Τότε λέγει προς τους δούλους αυτού· Ο μεν γάμος είναι έτοιμος, οι δε προσκεκλημένοι δεν ήσαν άξιοι·
\par 9 υπάγετε λοιπόν εις τας διεξόδους των οδών, και όσους αν εύρητε καλέσατε εις τους γάμους.
\par 10 Και εξελθόντες οι δούλοι εκείνοι εις τας οδούς, συνήγαγον πάντας όσους εύρον, κακούς τε και καλούς· και εγεμίσθη ο γάμος από ανακεκλιμένων.
\par 11 Εισελθών δε ο βασιλεύς διά να θεωρήση τους ανακεκλιμένους, είδεν εκεί άνθρωπον μη ενδεδυμένον ένδυμα γάμου,
\par 12 και λέγει προς αυτόν· Φίλε, πως εισήλθες ενταύθα μη έχων ένδυμα γάμου; Ο δε απεστομώθη.
\par 13 Τότε είπεν ο βασιλεύς προς τους υπηρέτας· Δέσαντες αυτού πόδας και χείρας, σηκώσατε αυτόν και ρίψατε εις το σκότος το εξώτερον· εκεί θέλει είσθαι ο κλαυθμός και ο τριγμός των οδόντων.
\par 14 Διότι πολλοί είναι οι κεκλημένοι, ολίγοι δε οι εκλεκτοί.
\par 15 Τότε υπήγον οι Φαρισαίοι και συνεβουλεύθησαν πως να παγιδεύσωσιν αυτόν εν λόγω.
\par 16 Και αποστέλλουσι προς αυτόν τους μαθητάς αυτών μετά των Ηρωδιανών, λέγοντες· Διδάσκαλε, εξεύρομεν ότι αληθής είσαι και την οδόν του Θεού εν αληθεία διδάσκεις και δεν σε μέλει περί ουδενός· διότι δεν βλέπεις εις πρόσωπον ανθρώπων·
\par 17 ειπέ λοιπόν προς ημάς, Τι σοι φαίνεται; είναι συγκεχωρημένον να δώσωμεν δασμόν εις τον Καίσαρα ή ουχί;
\par 18 Γνωρίσας δε ο Ιησούς την πονηρίαν αυτών, είπε· Τι με πειράζετε, υποκριταί;
\par 19 δείξατέ μοι το νόμισμα του δασμού· οι δε έφεραν προς αυτόν δηνάριον.
\par 20 Και λέγει προς αυτούς· Τίνος είναι η εικών αύτη και η επιγραφή;
\par 21 Λέγουσι προς αυτόν· Του Καίσαρος. Τότε λέγει προς αυτούς· Απόδοτε λοιπόν τα του Καίσαρος εις τον Καίσαρα και τα του Θεού εις τον Θεόν.
\par 22 Και ακούσαντες εθαύμασαν, και αφήσαντες αυτόν ανεχώρησαν.
\par 23 Εν εκείνη τη ημέρα προσήλθον προς αυτόν Σαδδουκαίοι, οι λέγοντες ότι δεν είναι ανάστασις, και ηρώτησαν αυτόν, λέγοντες·
\par 24 Διδάσκαλε, ο Μωϋσής είπεν, Εάν τις αποθάνη μη έχων τέκνα, θέλει νυμφευθή ο αδελφός αυτού την γυναίκα αυτού και θέλει αναστήσει σπέρμα εις τον αδελφόν αυτού.
\par 25 Ήσαν δε παρ' ημίν επτά αδελφοί· και ο πρώτος αφού ενυμφεύθη ετελεύτησε, και μη έχων τέκνον, αφήκε την γυναίκα αυτού εις τον αδελφόν αυτού·
\par 26 ομοίως και ο δεύτερος, και ο τρίτος, έως των επτά.
\par 27 Ύστερον δε πάντων απέθανε και η γυνή.
\par 28 Εν τη αναστάσει λοιπόν τίνος των επτά θέλει είσθαι γυνή; διότι πάντες έλαβον αυτήν.
\par 29 Αποκριθείς δε ο Ιησούς, είπε προς αυτούς· Πλανάσθε μη γνωρίζοντες τας γραφάς μηδέ την δύναμιν του Θεού.
\par 30 Διότι εν τη αναστάσει ούτε νυμφεύονται ούτε νυμφεύουσιν, αλλ' είναι ως άγγελοι του Θεού εν ουρανώ.
\par 31 Περί δε της αναστάσεως των νεκρών δεν ανεγνώσατε το ρηθέν προς εσάς υπό του Θεού, λέγοντος·
\par 32 Εγώ είμαι ο Θεός του Αβραάμ και ο Θεός του Ισαάκ και ο Θεός του Ιακώβ; δεν είναι ο Θεός Θεός νεκρών, αλλά ζώντων.
\par 33 Και ακούσαντες οι όχλοι, εξεπλήττοντο διά την διδαχήν αυτού.
\par 34 Οι δε Φαρισαίοι, ακούσαντες ότι απεστόμωσε τους Σαδδουκαίους, συνήχθησαν ομού.
\par 35 Και εις εξ αυτών, νομικός, ηρώτησε πειράζων αυτόν και λέγων·
\par 36 Διδάσκαλε, ποία εντολή είναι μεγάλη εν τω νόμω;
\par 37 Και ο Ιησούς είπε προς αυτόν· Θέλεις αγαπά Κύριον τον Θεόν σου εξ όλης της καρδίας σου και εξ όλης της ψυχής σου και εξ όλης της διανοίας σου.
\par 38 Αύτη είναι πρώτη και μεγάλη εντολή.
\par 39 Δευτέρα δε ομοία αυτής· Θέλεις αγαπά τον πλησίον σου ως σεαυτόν.
\par 40 Εν ταύταις ταις δύο εντολαίς όλος ο νόμος και οι προφήται κρέμανται.
\par 41 Και ενώ ήσαν συνηγμένοι οι Φαρισαίοι, ηρώτησεν αυτούς ο Ιησούς,
\par 42 λέγων· Τι σας φαίνεται περί του Χριστού; τίνος υιός είναι; Λέγουσι προς αυτόν· Του Δαβίδ.
\par 43 Λέγει προς αυτούς· Πως λοιπόν ο Δαβίδ διά Πνεύματος ονομάζει αυτόν Κύριον, λέγων,
\par 44 Είπεν ο Κύριος προς τον Κύριόν μου, Κάθου εκ δεξιών μου εωσού θέσω τους εχθρούς σου υποπόδιον των ποδών σου;
\par 45 Εάν λοιπόν ο Δαβίδ ονομάζη αυτόν Κύριον, πως είναι υιός αυτού;
\par 46 Και ουδείς ηδύνατο να αποκριθή προς αυτόν λόγον· ουδ' ετόλμησέ τις απ' εκείνης της ημέρας να ερωτήση πλέον αυτόν.

\chapter{23}

\par Τότε ο Ιησούς ελάλησε προς τους όχλους και προς τους μαθητάς αυτού,
\par 2 λέγων· Επί της καθέδρας του Μωϋσέως εκάθησαν οι γραμματείς και οι Φαρισαίοι.
\par 3 Πάντα λοιπόν όσα αν είπωσι προς εσάς να φυλάττητε, φυλάττετε και πράττετε, κατά δε τα έργα αυτών μη πράττετε· επειδή λέγουσι και δεν πράττουσι.
\par 4 Διότι δένουσι φορτία βαρέα και δυσβάστακτα και επιθέτουσιν επί τους ώμους των ανθρώπων, δεν θέλουσιν όμως ουδέ διά του δακτύλου αυτών να κινήσωσιν αυτά.
\par 5 Πράττουσι δε πάντα τα έργα αυτών διά να βλέπωνται υπό των ανθρώπων. Και πλατύνουσι τα φυλακτήρια αυτών και μεγαλύνουσι τα κράσπεδα των ιματίων αυτών,
\par 6 και αγαπώσι τον πρώτον τόπον εν τοις δείπνοις και τας πρωτοκαθεδρίας εν ταις συναγωγαίς
\par 7 και τους ασπασμούς εν ταις αγοραίς και να ονομάζωνται υπό των ανθρώπων Ραββί, Ραββί·
\par 8 σεις όμως μη ονομασθήτε Ραββί· διότι εις είναι ο καθηγητής σας, ο Χριστός· πάντες δε σεις αδελφοί είσθε.
\par 9 Και πατέρα σας μη ονομάσητε επί της γής· διότι εις είναι ο Πατήρ σας, ο εν τοις ουρανοίς.
\par 10 Μηδέ ονομασθήτε καθηγηταί· διότι εις είναι ο καθηγητής σας, ο Χριστός.
\par 11 Ο δε μεγαλήτερος από σας θέλει είσθαι υπηρέτης σας.
\par 12 Όστις δε υψώση εαυτόν θέλει ταπεινωθή, και όστις ταπεινώση εαυτόν θέλει υψωθή.
\par 13 Αλλ' ουαί εις εσάς, γραμματείς και Φαρισαίοι, υποκριταί, διότι κατατρώγετε τας οικίας των χηρών και τούτο επί προφάσει ότι κάμνετε μακράς προσευχάς· διά τούτο θέλετε λάβει μεγαλητέραν καταδίκην.
\par 14 Ουαί εις εσάς, γραμματείς και Φαρισαίοι, υποκριταί, διότι κλείετε την βασιλείαν των ουρανών έμπροσθεν των ανθρώπων· επειδή σεις δεν εισέρχεσθε ουδέ τους εισερχομένους αφίνετε να εισέλθωσιν.
\par 15 Ουαί εις εσάς, γραμματείς και Φαρισαίοι, υποκριταί, διότι περιέρχεσθε την θάλασσαν και την ξηράν διά να κάμητε ένα προσήλυτον, και όταν γείνη, κάμνετε αυτόν υιόν της γεέννης διπλότερον υμών.
\par 16 Ουαί εις εσάς, οδηγοί τυφλοί, οι λέγοντες· Όστις ομόση εν τω ναώ είναι ουδέν, όστις όμως ομόση εν τω χρυσώ του ναού, υποχρεούται.
\par 17 Μωροί και τυφλοί· διότι τις είναι μεγαλήτερος, ο χρυσός ή ο ναός ο αγιάζων τον χρυσόν;
\par 18 Καί· Όστις ομόση εν τω θυσιαστηρίω, είναι ουδέν, όστις όμως ομόση εν τω δώρω τω επάνω αυτού, υποχρεούται.
\par 19 Μωροί και τυφλοί· διότι τι είναι μεγαλήτερον, το δώρον ή το θυσιαστήριον το αγιάζον το δώρον;
\par 20 Ο ομόσας λοιπόν εν τω θυσιαστηρίω ομνύει εν αυτώ και εν πάσι τοις επάνω αυτού·
\par 21 και ο ομόσας εν τω ναώ ομνύει εν αυτώ και εν τω κατοικούντι αυτόν.
\par 22 Και ο ομόσας εν τω ουρανώ, ομνύει εν τω θρόνω του Θεού και εν τω καθημένω επάνω αυτού.
\par 23 Ουαί εις εσάς, γραμματείς και Φαρισαίοι, υποκριταί, διότι αποδεκατίζετε το ηδύοσμον και το άνηθον και το κύμινον, και αφήκατε τα βαρύτερα του νόμου, την κρίσιν και τον έλεον και την πίστιν· ταύτα έπρεπε να πράττητε και εκείνα να μη αφίνητε.
\par 24 Οδηγοί τυφλοί, οίτινες διυλίζετε τον κώνωπα, την δε κάμηλον καταπίνετε.
\par 25 Ουαί εις εσάς, γραμματείς και Φαρισαίοι, υποκριταί, διότι καθαρίζετε το έξωθεν του ποτηρίου και του πινακίου, έσωθεν όμως γέμουσιν εξ αρπαγής και ακρασίας.
\par 26 Φαρισαίε τυφλέ, καθάρισον πρώτον το εντός του ποτηρίου και του πινακίου, διά να γείνη και το εκτός αυτών καθαρόν.
\par 27 Ουαί εις εσάς, γραμματείς και Φαρισαίοι υποκριταί, διότι ομοιάζετε με τάφους ασβεστωμένους, οίτινες έξωθεν μεν φαίνονται ωραίοι, έσωθεν όμως γέμουσιν οστέων νεκρών και πάσης ακαθαρσίας.
\par 28 Ούτω και σεις έξωθεν μεν φαίνεσθε εις τους ανθρώπους δίκαιοι, έσωθεν όμως είσθε πλήρεις υποκρίσεως και ανομίας.
\par 29 Ουαί εις εσάς, γραμματείς και Φαρισαίοι υποκριταί· διότι οικοδομείτε τους τάφους των προφητών και στολίζετε τα μνημεία των δικαίων,
\par 30 και λέγετε· Εάν ήμεθα εν ταις ημέραις των πατέρων ημών, δεν ηθέλομεν είσθαι συγκοινωνοί αυτών εν τω αίματι των προφητών.
\par 31 Ώστε μαρτυρείτε εις εαυτούς ότι είσθε υιοί των φονευσάντων τους προφήτας.
\par 32 Αναπληρώσατε και σεις το μέτρον των πατέρων σας.
\par 33 Όφεις, γεννήματα εχιδνών· πως θέλετε φύγει από της καταδίκης της γεέννης;
\par 34 Διά τούτο ιδού, εγώ αποστέλλω προς εσάς προφήτας και σοφούς και γραμματείς, και εξ αυτών θέλετε θανατώσει και σταυρώσει, και εξ αυτών θέλετε μαστιγώσει εν ταις συναγωγαίς σας και διώξει από πόλεως εις πόλιν,
\par 35 διά να έλθη εφ' υμάς παν αίμα δίκαιον εκχυνόμενον επί της γης από του αίματος Άβελ του δικαίου έως του αίματος Ζαχαρίου υιού Βαραχίου, τον οποίον εφονεύσατε μεταξύ του ναού και του θυσιαστηρίου.
\par 36 Αληθώς σας λέγω, Πάντα ταύτα θέλουσιν ελθεί επί την γενεάν ταύτην.
\par 37 Ιερουσαλήμ, Ιερουσαλήμ, η φονεύουσα τους προφήτας και λιθοβολούσα τους απεσταλμένους προς σέ· ποσάκις ηθέλησα να συνάξω τα τέκνα σου καθ' ον τρόπον συνάγει η όρνις τα ορνίθια εαυτής υπό τας πτέρυγας, και δεν ηθελήσατε.
\par 38 Ιδού, αφίνεται εις εσάς ο οίκός σας έρημος.
\par 39 Διότι σας λέγω, δεν θέλετε με ιδεί εις το εξής, εωσού είπητε, Ευλογημένος ο ερχόμενος εν ονόματι Κυρίου.

\chapter{24}

\par Και εξελθών ο Ιησούς ανεχώρει από του ιερού· και προσήλθον οι μαθηταί αυτού διά να επιδείξωσιν εις αυτόν τας οικοδομάς του ιερού.
\par 2 Ο δε Ιησούς είπε προς αυτούς· Δεν βλέπετε πάντα ταύτα; αληθώς σας λέγω, δεν θέλει αφεθή εδώ λίθος επί λίθον, όστις δεν θέλει κατακρημνισθή.
\par 3 Και ενώ εκάθητο επί του όρους των Ελαιών, προσήλθον προς αυτόν οι μαθηταί κατ' ιδίαν, λέγοντες· Ειπέ προς ημάς πότε θέλουσι γείνει ταύτα, και τι το σημείον της παρουσίας σου και της συντελείας του αιώνος;
\par 4 Και αποκριθείς ο Ιησούς, είπε προς αυτούς· Βλέπετε μη σας πλανήση τις.
\par 5 Διότι πολλοί θέλουσιν ελθεί επί τω ονόματί μου, λέγοντες, Εγώ είμαι ο Χριστός, και πολλούς θέλουσι πλανήσει.
\par 6 Θέλετε δε ακούσει πολέμους και φήμας πολέμων· προσέχετε μη ταραχθήτε· επειδή πάντα ταύτα πρέπει να γείνωσιν, αλλά δεν είναι έτι το τέλος.
\par 7 Διότι θέλει εγερθή έθνος επί έθνος και βασιλεία επί βασιλείαν, και θέλουσι γείνει πείναι και λοιμοί και σεισμοί κατά τόπους·
\par 8 πάντα δε ταύτα είναι αρχή ωδίνων.
\par 9 Τότε θέλουσι σας παραδώσει εις θλίψιν και θέλουσι σας θανατώσει, και θέλετε είσθαι μισούμενοι υπό πάντων των εθνών διά το όνομά μου.
\par 10 Και τότε θέλουσι σκανδαλισθή πολλοί και θέλουσι παραδώσει αλλήλους και θέλουσι μισήσει αλλήλους.
\par 11 Και πολλοί ψευδοπροφήται θέλουσιν εγερθή και πλανήσει πολλούς,
\par 12 και επειδή θέλει πληθυνθή η ανομία, η αγάπη των πολλών θέλει ψυχρανθή.
\par 13 Ο δε υπομείνας έως τέλους, ούτος θέλει σωθή.
\par 14 Και θέλει κηρυχθή τούτο το ευαγγέλιον της βασιλείας εν όλη τη οικουμένη προς μαρτυρίαν εις πάντα τα έθνη, και τότε θέλει ελθεί το τέλος.
\par 15 Όταν λοιπόν ίδητε το βδέλυγμα της ερημώσεως, το λαληθέν διά του προφήτου Δανιήλ, ιστάμενον εν τω τόπω τω αγίω -ο αναγινώσκων ας εννοή-
\par 16 τότε οι εν τη Ιουδαία ας φεύγωσιν επί τα όρη·
\par 17 όστις ευρεθή επί του δώματος, ας μη καταβή διά να λάβη τι εκ της οικίας αυτού·
\par 18 και όστις ευρεθή εν τω αγρώ, ας μη επιστρέψη οπίσω διά να λάβη τα ιμάτια αυτού.
\par 19 Ουαί δε εις τας εγκυμονούσας και τας θηλαζούσας εν εκείναις ταις ημέραις.
\par 20 Προσεύχεσθε δε διά να μη γείνη η φυγή υμών εν χειμώνι μηδέ εν σαββάτω.
\par 21 Διότι τότε θέλει είσθαι θλίψις μεγάλη, οποία δεν έγεινεν απ' αρχής κόσμου έως του νυν, ουδέ θέλει γείνει.
\par 22 Και αν δεν συνετέμνοντο αι ημέραι εκείναι, δεν ήθελε σωθή ουδεμία σάρξ· διά τους εκλεκτούς όμως θέλουσι συντμηθή αι ημέραι εκείναι.
\par 23 Τότε εάν τις είπη προς υμάς· Ιδού εδώ είναι ο Χριστός ή εδώ, μη πιστεύσητε·
\par 24 διότι θέλουσιν εγερθή ψευδόχριστοι και ψευδοπροφήται και θέλουσι δείξει σημεία μεγάλα και τέρατα, ώστε να πλανήσωσιν, ει δυνατόν, και τους εκλεκτούς.
\par 25 Ιδού, προείπα προς υμάς.
\par 26 Εάν λοιπόν είπωσι προς υμάς, Ιδού, εν τη ερήμω είναι, μη εξέλθητε, Ιδού, εν τοις ταμείοις, μη πιστεύσητε·
\par 27 διότι καθώς η αστραπή εξέρχεται από ανατολών και φαίνεται έως δυσμών, ούτω θέλει είσθαι και η παρουσία του Υιού του ανθρώπου.
\par 28 Διότι όπου είναι το πτώμα, εκεί θέλουσι συναχθή οι αετοί.
\par 29 Ευθύς δε μετά την θλίψιν των ημερών εκείνων ο ήλιος θέλει σκοτισθή και η σελήνη δεν θέλει δώσει το φέγγος αυτής, και οι αστέρες θέλουσι πέσει από του ουρανού και αι δυνάμεις των ουρανών θέλουσι σαλευθή.
\par 30 Και τότε θέλει φανή το σημείον του Υιού του ανθρώπου εν τω ουρανώ, και τότε θέλουσι θρηνήσει πάσαι αι φυλαί της γης και θέλουσιν ιδεί τον Υιόν του ανθρώπου ερχόμενον επί των νεφελών του ουρανού μετά δυνάμεως και δόξης πολλής.
\par 31 Και θέλει αποστείλει τους αγγέλους αυτού μετά σάλπιγγος φωνής μεγάλης, και θέλουσι συνάξει τους εκλεκτούς αυτού εκ των τεσσάρων ανέμων απ' άκρων ουρανών έως άκρων αυτών.
\par 32 Από δε της συκής μάθετε την παραβολήν· Όταν ο κλάδος αυτής γείνη ήδη απαλός και εκβλαστάνη τα φύλλα, γνωρίζετε ότι πλησιάζει το θέρος·
\par 33 ούτω και σεις, όταν ίδητε πάντα ταύτα, εξεύρετε ότι πλησίον είναι επί τας θύρας.
\par 34 Αληθώς σας λέγω, δεν θέλει παρέλθει η γενεά αύτη, εωσού γείνωσι πάντα ταύτα.
\par 35 Ο ουρανός και η γη θέλουσι παρέλθει, οι δε λόγοι μου δεν θέλουσι παρέλθει.
\par 36 Περί δε της ημέρας εκείνης και της ώρας ουδείς γινώσκει, ουδέ οι άγγελοι των ουρανών, ειμή ο Πατήρ μου μόνος·
\par 37 και καθώς αι ημέραι του Νώε, ούτω θέλει είσθαι και η παρουσία του Υιού του ανθρώπου.
\par 38 Διότι καθώς εν ταις ημέραις ταις προ του κατακλυσμού ήσαν τρώγοντες και πίνοντες, νυμφευόμενοι και νυμφεύοντες, έως της ημέρας καθ' ην ο Νώε εισήλθεν εις την κιβωτόν,
\par 39 και δεν ενόησαν, εωσού ήλθεν ο κατακλυσμός και εσήκωσε πάντας, ούτω θέλει είσθαι και η παρουσία του Υιού του ανθρώπου.
\par 40 Τότε δύο θέλουσιν είσθαι εν τω αγρώ· ο εις παραλαμβάνεται και ο εις αφίνεται·
\par 41 δύο γυναίκες θέλουσιν αλέθει εν τω μύλω, μία παραλαμβάνεται και μία αφίνεται.
\par 42 Αγρυπνείτε λοιπόν, διότι δεν εξεύρετε ποία ώρα έρχεται ο Κύριος υμών.
\par 43 Τούτο δε γινώσκετε ότι εάν ήξευρεν ο οικοδεσπότης εν ποία φυλακή της νυκτός έρχεται ο κλέπτης, ήθελεν αγρυπνήσει και δεν ήθελεν αφήσει να διορυχθή η οικία αυτού.
\par 44 Διά τούτο και σεις γίνεσθε έτοιμοι, διότι καθ' ην ώραν δεν στοχάζεσθε, έρχεται ο Υιός του ανθρώπου.
\par 45 Τις λοιπόν είναι ο πιστός και φρόνιμος δούλος, τον οποίον ο κύριος αυτού κατέστησεν επί των υπηρετών αυτού, διά να δίδη εις αυτούς την τροφήν εν καιρώ;
\par 46 Μακάριος ο δούλος εκείνος, τον οποίον όταν έλθη ο κύριος αυτού θέλει ευρεί πράττοντα ούτως.
\par 47 Αληθώς σας λέγω ότι θέλει καταστήσει αυτόν επί πάντων των υπαρχόντων αυτού.
\par 48 Εάν δε είπη ο κακός εκείνος δούλος εν τη καρδία αυτού, Βραδύνει να έλθη ο κύριός μου,
\par 49 και αρχίση να δέρη τους συνδούλους, να τρώγη δε και να πίνη μετά των μεθυόντων,
\par 50 θέλει ελθεί ο κύριος του δούλου εκείνου καθ' ην ημέραν δεν προσμένει και καθ' ην ώραν δεν εξεύρει,
\par 51 και θέλει αποχωρίσει αυτόν, και το μέρος αυτού θέλει θέσει μετά των υποκριτών· εκεί θέλει είσθαι ο κλαυθμός και ο τριγμός των οδόντων.

\chapter{25}

\par Τότε θέλει ομοιωθή η βασιλεία των ουρανών με δέκα παρθένους, αίτινες λαβούσαι τας λαμπάδας αυτών εξήλθον εις απάντησιν του νυμφίου.
\par 2 Πέντε δε εξ αυτών ήσαν φρόνιμοι και πέντε μωραί.
\par 3 Αίτινες μωραί, λαβούσαι τας λαμπάδας αυτών, δεν έλαβον μεθ' εαυτών έλαιον·
\par 4 αι φρόνιμοι όμως έλαβον έλαιον εν τοις αγγείοις αυτών μετά των λαμπάδων αυτών.
\par 5 Και επειδή ο νυμφίος εβράδυνεν, ενύσταξαν πάσαι και εκοιμώντο.
\par 6 Εν τω μέσω δε της νυκτός έγεινε κραυγή· Ιδού, ο νυμφίος έρχεται, εξέλθετε εις απάντησιν αυτού.
\par 7 Τότε εσηκώθησαν πάσαι αι παρθένοι εκείναι και ητοίμασαν τας λαμπάδας αυτών.
\par 8 Και αι μωραί είπον προς τας φρονίμους· Δότε εις ημάς εκ του ελαίου σας, διότι αι λαμπάδες ημών σβύνονται.
\par 9 Απεκρίθησαν δε αι φρόνιμοι, λέγουσαι· Μήποτε δεν αρκέση εις ημάς και εις εσάς· όθεν υπάγετε κάλλιον προς τους πωλούντας και αγοράσατε εις εαυτάς.
\par 10 Ενώ δε απήρχοντο διά να αγοράσωσιν, ήλθεν ο νυμφίος και αι έτοιμοι εισήλθον μετ' αυτού εις τους γάμους, και εκλείσθη η θύρα.
\par 11 Ύστερον δε έρχονται και αι λοιπαί παρθένοι, λέγουσαι· Κύριε, Κύριε, άνοιξον εις ημάς.
\par 12 Ο δε αποκριθείς είπεν· Αληθώς σας λέγω, δεν σας γνωρίζω.
\par 13 Αγρυπνείτε λοιπόν, διότι δεν εξεύρετε την ημέραν ουδέ την ώραν, καθ' ην ο Υιός του ανθρώπου έρχεται.
\par 14 Διότι θέλει ελθεί ως άνθρωπος, όστις αποδημών εκάλεσε τους δούλους αυτού και παρέδωκεν εις αυτούς τα υπάρχοντα αυτού,
\par 15 και εις άλλον μεν έδωκε πέντε τάλαντα, εις άλλον δε δύο, εις άλλον δε εν, εις έκαστον κατά την ιδίαν αυτού ικανότητα, και απεδήμησεν ευθύς.
\par 16 Υπήγε δε ο λαβών τα πέντε τάλαντα και εργαζόμενος δι' αυτών έκαμεν άλλα πέντε τάλαντα.
\par 17 Ωσαύτως και ο τα δύο εκέρδησε και αυτός άλλα δύο.
\par 18 Ο δε λαβών το εν υπήγε και έσκαψεν εις την γην και έκρυψε το αργύριον του κυρίου αυτού.
\par 19 Μετά δε καιρόν πολύν έρχεται ο κύριος των δούλων εκείνων και θεωρεί λογαριασμόν μετ' αυτών.
\par 20 Και ελθών ο λαβών τα πέντε τάλαντα, προσέφερεν άλλα πέντε τάλαντα, λέγων· Κύριε, πέντε τάλαντα μοι παρέδωκας· ιδού, άλλα πέντε τάλαντα εκέρδησα επ' αυτοίς.
\par 21 Και είπε προς αυτόν ο κύριος αυτού· Εύγε, δούλε αγαθέ και πιστέ· εις τα ολίγα εστάθης πιστός, επί πολλών θέλω σε καταστήσει· είσελθε εις την χαράν του κυρίου σου.
\par 22 Προσελθών δε και ο λαβών τα δύο τάλαντα είπε· Κύριε, δύο τάλαντα μοι παρέδωκας· ιδού, άλλα δύο τάλαντα εκέρδησα επ' αυτοίς.
\par 23 Είπε προς αυτόν ο κύριος αυτού· Εύγε, δούλε αγαθέ και πιστέ· εις τα ολίγα εστάθης πιστός, επί πολλών θέλω σε καταστήσει· είσελθε εις την χαράν του κυρίου σου.
\par 24 Προσελθών δε και ο λαβών το εν τάλαντον, είπε· Κύριε, σε εγνώρισα ότι είσαι σκληρός άνθρωπος, θερίζων όπου δεν έσπειρας και συνάγων όθεν δεν διεσκόρπισας·
\par 25 και φοβηθείς υπήγα και έκρυψα το τάλαντόν σου εν τη γή· ιδού, έχεις το σον.
\par 26 Αποκριθείς δε ο κύριος αυτού, είπε προς αυτόν· Πονηρέ δούλε και οκνηρέ· ήξευρες ότι θερίζω όπου δεν έσπειρα και συνάγω όθεν δεν διεσκόρπισα·
\par 27 έπρεπε λοιπόν να βάλης το αργύριόν μου εις τους τραπεζίτας, και ελθών εγώ ήθελον λάβει το εμόν μετά τόκου.
\par 28 Λάβετε λοιπόν απ' αυτού το τάλαντον, και δότε εις τον έχοντα τα δέκα τάλαντα.
\par 29 Διότι εις πάντα τον έχοντα θέλει δοθή και περισσευθή, από δε του μη έχοντος και εκείνο το οποίον έχει θέλει αφαιρεθή απ' αυτού.
\par 30 Και τον αχρείον δούλον ρίψατε εις το σκότος το εξώτερον· εκεί θέλει είσθαι ο κλαυθμός και ο τριγμός των οδόντων.
\par 31 Όταν δε έλθη ο Υιός του ανθρώπου εν τη δόξη αυτού και πάντες οι άγιοι άγγελοι μετ' αυτού, τότε θέλει καθήσει επί του θρόνου της δόξης αυτού,
\par 32 και θέλουσι συναχθή έμπροσθεν αυτού πάντα τα έθνη, και θέλει χωρίσει αυτούς απ' αλλήλων, καθώς ο ποιμήν χωρίζει τα πρόβατα από των εριφίων,
\par 33 και θέλει στήσει τα μεν πρόβατα εκ δεξιών αυτού, τα δε ερίφια εξ αριστερών.
\par 34 Τότε ο Βασιλεύς θέλει ειπεί προς τους εκ δεξιών αυτού· Έλθετε οι ευλογημένοι του Πατρός μου, κληρονομήσατε την ητοιμασμένην εις εσάς βασιλείαν από καταβολής κόσμου.
\par 35 Διότι επείνασα, και μοι εδώκατε να φάγω, εδίψησα, και με εποτίσατε, ξένος ήμην, και με εφιλοξενήσατε,
\par 36 γυμνός, και με ενεδύσατε, ησθένησα, και με επεσκέφθητε, εν φυλακή ήμην, και ήλθετε προς εμέ.
\par 37 Τότε θέλουσιν αποκριθή προς αυτόν οι δίκαιοι, λέγοντες· Κύριε, πότε σε είδομεν πεινώντα και εθρέψαμεν, ή διψώντα και εποτίσαμεν;
\par 38 πότε δε σε είδομεν ξένον και εφιλοξενήσαμεν, ή γυμνόν και ενεδύσαμεν;
\par 39 πότε δε σε είδομεν ασθενή ή εν φυλακή και ήλθομεν προς σε;
\par 40 Και αποκριθείς ο Βασιλεύς θέλει ειπεί προς αυτούς· Αληθώς σας λέγω, καθ' όσον εκάμετε εις ένα τούτων των αδελφών μου των ελαχίστων, εις εμέ εκάμετε.
\par 41 Τότε θέλει ειπεί και προς τους εξ αριστερών· Υπάγετε απ' εμού οι κατηραμένοι εις το πυρ το αιώνιον, το ητοιμασμένον διά τον διάβολον και τους αγγέλους αυτού.
\par 42 Διότι επείνασα, και δεν μοι εδώκατε να φάγω, εδίψησα, και δεν με εποτίσατε,
\par 43 ξένος ήμην, και δεν με εφιλοξενήσατε, γυμνός, και δεν με ενεδύσατε, ασθενής και εν φυλακή, και δεν με επεσκέφθητε.
\par 44 Τότε θέλουσιν αποκριθή προς αυτόν και αυτοί, λέγοντες· Κύριε, πότε σε είδομεν πεινώντα ή διψώντα ή ξένον ή γυμνόν ή ασθενή ή εν φυλακή, και δεν σε υπηρετήσαμεν;
\par 45 Τότε θέλει αποκριθή προς αυτούς, λέγων· Αληθώς σας λέγω, καθ' όσον δεν εκάμετε εις ένα τούτων των ελαχίστων, ουδέ εις εμέ εκάμετε.
\par 46 Και θέλουσιν απέλθει ούτοι μεν εις κόλασιν αιώνιον, οι δε δίκαιοι εις ζωήν αιώνιον.

\chapter{26}

\par Και ότε ετελείωσεν ο Ιησούς πάντας τους λόγους τούτους, είπε προς τους μαθητάς αυτού·
\par 2 Εξεύρετε ότι μετά δύο ημέρας γίνεται το πάσχα, και ο Υιός του ανθρώπου παραδίδεται διά να σταυρωθή.
\par 3 Τότε συνήχθησαν οι αρχιερείς και οι γραμματείς και οι πρεσβύτεροι του λαού εις την αυλήν του αρχιερέως του λεγομένου Καϊάφα,
\par 4 και συνεβουλεύθησαν να συλλάβωσι τον Ιησούν με δόλον και να θανατώσωσιν.
\par 5 Έλεγον δέ· μη εν τη εορτή, διά να μη γείνη θόρυβος εν τω λαώ.
\par 6 Ότε δε ο Ιησούς ήτο εν Βηθανία εν τη οικία Σίμωνος του λεπρού,
\par 7 προσήλθε προς αυτόν γυνή έχουσα αλάβαστρον μύρου βαρυτίμου, και κατέχεεν αυτό επί την κεφαλήν αυτού, ενώ εκάθητο εις την τράπεζαν.
\par 8 Ιδόντες δε οι μαθηταί αυτού, ηγανάκτησαν λέγοντες· Εις τι η απώλεια αύτη;
\par 9 διότι ηδύνατο τούτο το μύρον να πωληθή με πολλήν τιμήν και να δοθή εις τους πτωχούς.
\par 10 Νοήσας δε ο Ιησούς, είπε προς αυτούς· Διά τι ενοχλείτε την γυναίκα; διότι έργον καλόν έπραξεν εις εμέ.
\par 11 Διότι τους πτωχούς πάντοτε έχετε μεθ' εαυτών, εμέ όμως πάντοτε δεν έχετε.
\par 12 Επειδή χύσασα αύτη το μύρον τούτο επί του σώματός μου, έκαμε τούτο διά τον ενταφιασμόν μου.
\par 13 Αληθώς σας λέγω, όπου εάν κηρυχθή το ευαγγέλιον τούτο εν όλω τω κόσμω, θέλει λαληθή και τούτο, το οποίον έπραξεν αύτη, εις μνημόσυνον αυτής.
\par 14 Τότε υπήγεν εις των δώδεκα, ο λεγόμενος Ιούδας Ισκαριώτης, προς τους αρχιερείς
\par 15 και είπε· Τι θέλετε να μοι δώσητε, και εγώ θέλω σας παραδώσει αυτόν; Και εκείνοι έδωκαν εις αυτόν τριάκοντα αργύρια.
\par 16 Και από τότε εζήτει ευκαιρίαν διά να παραδώση αυτόν.
\par 17 Την δε πρώτην των αζύμων προσήλθον οι μαθηταί προς τον Ιησούν, λέγοντες προς αυτόν· Που θέλεις να σοι ετοιμάσωμεν διά να φάγης το πάσχα;
\par 18 Και εκείνος είπεν· Υπάγετε εις την πόλιν προς τον δείνα και είπατε προς αυτόν· Ο Διδάσκαλος λέγει, Ο καιρός μου επλησίασεν· εν τη οικία σου θέλω κάμει το πάσχα μετά των μαθητών μου.
\par 19 Και έκαμον οι μαθηταί καθώς παρήγγειλεν εις αυτούς ο Ιησούς, και ητοίμασαν το πάσχα.
\par 20 Ότε δε έγεινεν εσπέρα, εκάθητο εις την τράπεζαν μετά των δώδεκα.
\par 21 Και ενώ έτρωγον, είπεν· Αληθώς σας λέγω ότι εις εξ υμών θέλει με παραδώσει.
\par 22 Και λυπούμενοι σφόδρα, ήρχισαν να λέγωσι προς αυτόν έκαστος αυτών· Μήπως εγώ είμαι, Κύριε;
\par 23 Ο δε αποκριθείς είπεν· Ο εμβάψας μετ' εμού εν τω πινακίω την χείρα, ούτος θέλει με παραδώσει.
\par 24 Ο μεν Υιός του ανθρώπου υπάγει, καθώς είναι γεγραμμένον περί αυτού· ουαί δε εις τον άνθρωπον εκείνον, διά του οποίου ο Υιός του ανθρώπου παραδίδεται· καλόν ήτο εις τον άνθρωπον εκείνον, αν δεν ήθελε γεννηθή.
\par 25 Αποκριθείς δε ο Ιούδας, όστις παρέδιδεν αυτόν, είπε· Μήπως εγώ είμαι, Ραββί; Λέγει προς αυτόν· Συ είπας.
\par 26 Και ενώ έτρωγον, λαβών ο Ιησούς τον άρτον και ευλογήσας έκοψε και έδιδεν εις τους μαθητάς και είπε· Λάβετε, φάγετε· τούτο είναι το σώμα μου·
\par 27 και λαβών το ποτήριον και ευχαριστήσας, έδωκεν εις αυτούς, λέγων· Πίετε εξ αυτού πάντες·
\par 28 διότι τούτο είναι το αίμα μου το της καινής διαθήκης, το υπέρ πολλών εκχυνόμενον εις άφεσιν αμαρτιών.
\par 29 Σας λέγω δε ότι δεν θέλω πίει εις το εξής εκ τούτου του γεννήματος της αμπέλου έως της ημέρας εκείνης, όταν πίνω αυτό νέον μεθ' υμών εν τη βασιλεία του Πατρός μου.
\par 30 Και αφού ύμνησαν, εξήλθον εις το όρος των ελαιών.
\par 31 Τότε λέγει προς αυτούς ο Ιησούς· Πάντες υμείς θέλετε σκανδαλισθή εν εμοί την νύκτα ταύτην· διότι είναι γεγραμμένον, Θέλω πατάξει τον ποιμένα, και θέλουσι διασκορπισθή τα πρόβατα της ποίμνης·
\par 32 αφού δε αναστηθώ, θέλω υπάγει πρότερον υμών εις την Γαλιλαίαν.
\par 33 Αποκριθείς δε ο Πέτρος, είπε προς αυτόν· Και αν πάντες σκανδαλισθώσιν εν σοι, εγώ ποτέ δεν θέλω σκανδαλισθή.
\par 34 Είπε προς αυτόν ο Ιησούς· Αληθώς σοι λέγω ότι ταύτην την νύκτα, πριν φωνάξη ο αλέκτωρ, τρίς θέλεις με απαρνηθή.
\par 35 Λέγει προς αυτόν ο Πέτρος· Και αν γείνη χρεία να αποθάνω μετά σου, δεν θέλω σε απαρνηθή. Ομοίως είπον και πάντες οι μαθηταί.
\par 36 Τότε έρχεται μετ' αυτών ο Ιησούς εις χωρίον λεγόμενον Γεθσημανή και λέγει προς τους μαθητάς· Καθήσατε αυτού, εωσού υπάγω και προσευχηθώ εκεί.
\par 37 Και παραλαβών τον Πέτρον και τους δύο υιούς του Ζεβεδαίου, ήρχισε να λυπήται και να αδημονή.
\par 38 Τότε λέγει προς αυτούς· Περίλυπος είναι η ψυχή μου έως θανάτου· μείνατε εδώ και αγρυπνείτε μετ' εμού.
\par 39 Και προχωρήσας ολίγον έπεσεν επί πρόσωπον αυτού, προσευχόμενος και λέγων· Πάτερ μου, εάν ήναι δυνατόν, ας παρέλθη απ' εμού το ποτήριον τούτο· πλην ουχί ως εγώ θέλω, αλλ' ως συ.
\par 40 Και έρχεται προς τους μαθητάς και ευρίσκει αυτούς κοιμωμένους, και λέγει προς τον Πέτρον· Ούτω δεν ηδυνήθητε μίαν ώραν να αγρυπνήσητε μετ' εμού;
\par 41 αγρυπνείτε και προσεύχεσθε, διά να μη εισέλθητε εις πειρασμόν. Το μεν πνεύμα πρόθυμον, η δε σαρξ ασθενής.
\par 42 Πάλιν εκ δευτέρου υπήγε και προσευχήθη, λέγων· Πάτερ μου, εάν δεν ήναι δυνατόν τούτο το ποτήριον να παρέλθη απ' εμού χωρίς να πίω αυτό, γενηθήτω το θέλημά σου.
\par 43 Και ελθών ευρίσκει αυτούς πάλιν κοιμωμένους· διότι οι οφθαλμοί αυτών ήσαν βεβαρημένοι.
\par 44 Και αφήσας αυτούς υπήγε πάλιν και προσευχήθη εκ τρίτου, ειπών τον αυτόν λόγον.
\par 45 Τότε έρχεται προς τους μαθητάς αυτού και λέγει προς αυτούς· Κοιμάσθε το λοιπόν και αναπαύεσθε· ιδού, επλησίασεν η ώρα και ο Υιός του ανθρώπου παραδίδεται εις χείρας αμαρτωλών.
\par 46 Εγέρθητε, ας υπάγωμεν· Ιδού, επλησίασεν ο παραδίδων με.
\par 47 Και ενώ αυτός ελάλει έτι, ιδού, ο Ιούδας εις των δώδεκα ήλθε, και μετ' αυτού όχλος πολύς μετά μαχαιρών και ξύλων παρά των αρχιερέων και πρεσβυτέρων του λαού.
\par 48 Ο δε παραδίδων αυτόν έδωκεν εις αυτούς σημείον, λέγων· Όντινα φιλήσω, αυτός είναι· πιάσατε αυτόν.
\par 49 Και ευθύς πλησιάσας προς τον Ιησούν, είπε· Χαίρε, Ραββί, και κατεφίλησεν αυτόν.
\par 50 Ο δε Ιησούς είπε προς αυτόν· Φίλε, διά τι ήλθες; Τότε προσελθόντες επέβαλον τας χείρας επί τον Ιησούν και επίασαν αυτόν.
\par 51 Και ιδού, εις των μετά του Ιησού εκτείνας την χείρα έσυρε την μάχαιραν αυτού, και κτυπήσας τον δούλον του αρχιερέως απέκοψε το ωτίον αυτού.
\par 52 Τότε λέγει προς αυτόν ο Ιησούς· Επίστρεψον την μάχαιράν σου εις τον τόπον αυτής· διότι πάντες όσοι πιάσωσι μάχαιραν διά μαχαίρας θέλουσιν απολεσθή.
\par 53 Η νομίζεις ότι δεν δύναμαι ήδη να παρακαλέσω τον Πατέρα μου, και θέλει στήσει πλησίον μου περισσοτέρους παρά δώδεκα λεγεώνας αγγέλων;
\par 54 πως λοιπόν θέλουσι πληρωθή αι γραφαί ότι ούτω πρέπει να γείνη;
\par 55 Εν εκείνη τη ώρα είπεν ο Ιησούς προς τους όχλους· Ως επί ληστήν εξήλθετε μετά μαχαιρών και ξύλων να με συλλάβητε; καθ' ημέραν εκαθήμην πλησίον υμών διδάσκων εν τω ιερώ, και δεν με επιάσατε.
\par 56 Τούτο δε όλον έγεινε διά να πληρωθώσιν αι γραφαί των προφητών. Τότε οι μαθηταί πάντες αφήσαντες αυτόν έφυγον.
\par 57 Οι δε πιάσαντες τον Ιησούν έφεραν προς Καϊάφαν τον αρχιερέα, όπου συνήχθησαν οι γραμματείς και οι πρεσβύτεροι.
\par 58 Ο δε Πέτρος ηκολούθει αυτόν από μακρόθεν έως της αυλής του αρχιερέως, και εισελθών έσω εκάθητο μετά των υπηρετών διά να ίδη το τέλος.
\par 59 Οι δε αρχιερείς και οι πρεσβύτεροι και το συνέδριον όλον εζήτουν ψευδομαρτυρίαν κατά του Ιησού, διά να θανατώσωσιν αυτόν,
\par 60 και δεν εύρον· και πολλών ψευδομαρτύρων προσελθόντων, δεν εύρον. Ύστερον δε προσελθόντες δύο ψευδομάρτυρες,
\par 61 είπον· Ούτος είπε, Δύναμαι να χαλάσω τον ναόν του Θεού και διά τριών ημερών να οικοδομήσω αυτόν.
\par 62 Και σηκωθείς ο αρχιερεύς είπε προς αυτόν· Δεν αποκρίνεσαι; τι μαρτυρούσιν ούτοι κατά σου;
\par 63 Ο δε Ιησούς εσιώπα. Και αποκριθείς ο αρχιερεύς είπε προς αυτόν· Σε ορκίζω εις τον Θεόν τον ζώντα να είπης προς ημάς αν συ ήσαι ο Χριστός ο Υιός του Θεού.
\par 64 Λέγει προς αυτόν ο Ιησούς· Συ είπας· πλην σας λέγω, Εις το εξής θέλετε ιδεί τον Υιόν του ανθρώπου καθήμενον εκ δεξιών της δυνάμεως και ερχόμενον επί των νεφελών του ουρανού.
\par 65 Τότε ο αρχιερεύς διέσχισε τα ιμάτια αυτού, λέγων ότι εβλασφήμησε· τι χρείαν έχομεν πλέον μαρτύρων; ιδού, τώρα ηκούσατε την βλασφημίαν αυτού·
\par 66 τι σας φαίνεται; Και εκείνοι αποκριθέντες είπον· Ένοχος θανάτου είναι.
\par 67 Τότε ενέπτυσαν εις το πρόσωπον αυτού και εγρόνθισαν αυτόν, άλλοι δε ερράπισαν,
\par 68 λέγοντες· Προφήτευσον εις ημάς, Χριστέ, τις είναι όστις σε εκτύπησεν;
\par 69 Ο δε Πέτρος εκάθητο έξω εν τη αυλή και προσήλθε προς αυτόν μία δούλη, λέγουσα· Και συ ήσο μετά Ιησού του Γαλιλαίου.
\par 70 Ο δε ηρνήθη έμπροσθεν πάντων, λέγων· Δεν εξεύρω τι λέγεις.
\par 71 Και ότε εξήλθεν εις τον πυλώνα, είδεν αυτόν άλλη και λέγει προς τους εκεί, Και ούτος ήτο μετά Ιησού του Ναζωραίου.
\par 72 Και πάλιν ηρνήθη μεθ' όρκου ότι δεν γνωρίζω τον άνθρωπον.
\par 73 Μετ' ολίγον δε προσελθόντες οι εστώτες, είπον προς τον Πέτρον· Αληθώς και συ εξ αυτών είσαι· διότι η λαλιά σου σε κάμνει φανερόν.
\par 74 Τότε ήρχισε να καταναθεματίζη και να ομνύη ότι δεν γνωρίζω τον άνθρωπον. Και ευθύς εφώναξεν ο αλέκτωρ.
\par 75 Και ενεθυμήθη ο Πέτρος τον λόγον του Ιησού, όστις είχεν ειπεί προς αυτόν ότι πριν φωνάξη ο αλέκτωρ, τρίς θέλεις με απαρνηθή· και εξελθών έξω έκλαυσε πικρώς.

\chapter{27}

\par Ότε δε έγεινε πρωΐ, συνεβουλεύθησαν πάντες οι αρχιερείς και οι πρεσβύτεροι του λαού κατά του Ιησού διά να θανατώσωσιν αυτόν·
\par 2 και δέσαντες αυτόν, έφεραν και παρέδωκαν αυτόν εις τον Πόντιον Πιλάτον τον ηγεμόνα.
\par 3 Τότε ιδών Ιούδας ο παραδόσας αυτόν ότι κατεδικάσθη, μεταμεληθείς επέστρεψε τα τριάκοντα αργύρια εις τους πρεσβυτέρους,
\par 4 λέγων· Ήμαρτον παραδόσας αίμα αθώον. Οι δε είπον· Τι προς ημάς; συ όψει.
\par 5 Και ρίψας τα αργύρια εν τω ναώ, ανεχώρησε και απελθών εκρεμάσθη.
\par 6 Οι δε αρχιερείς, λαβόντες τα αργύρια, είπον· Δεν είναι συγκεχωρημένον να βάλωμεν αυτά εις το θησαυροφυλάκιον, διότι είναι τιμή αίματος.
\par 7 Και συμβουλευθέντες ηγόρασαν με αυτά τον αγρόν του κεραμέως, διά να ενταφιάζωνται εκεί οι ξένοι.
\par 8 Διά τούτο ωνομάσθη ο αγρός εκείνος Αγρός αίματος έως της σήμερον.
\par 9 Τότε επληρώθη το ρηθέν διά Ιερεμίου του προφήτου, λέγοντος· Και έλαβον τα τριάκοντα αργύρια, την τιμήν του εκτιμηθέντος, τον οποίον εξετίμησαν από των υιών Ισραήλ,
\par 10 και έδωκαν αυτά εις τον αγρόν του κεραμέως, καθώς μοι παρήγγειλεν ο Κύριος.
\par 11 Ο δε Ιησούς εστάθη έμπροσθεν του ηγεμόνος· και ηρώτησεν αυτόν ο ηγεμών, λέγων· Συ είσαι ο βασιλεύς των Ιουδαίων; Ο δε Ιησούς είπε προς αυτόν· Συ λέγεις.
\par 12 Και ενώ εκατηγορείτο υπό των αρχιερέων και των πρεσβυτέρων, ουδέν απεκρίθη.
\par 13 Τότε λέγει προς αυτόν ο Πιλάτος· Δεν ακούεις πόσα σου καταμαρτυρούσι;
\par 14 Και δεν απεκρίθη προς αυτόν ουδέ προς ένα λόγον, ώστε ο ηγεμών εθαύμαζε πολύ.
\par 15 Κατά δε την εορτήν εσυνείθιζεν ο ηγεμών να απολύη εις τον όχλον ένα δέσμιον, όντινα ήθελον.
\par 16 Και είχον τότε δέσμιον περιβόητον λεγόμενον Βαραββάν.
\par 17 Ενώ λοιπόν ήσαν συνηγμένοι, είπε προς αυτούς ο Πιλάτος· Τίνα θέλετε να σας απολύσω; τον Βαραββάν ή τον Ιησούν τον λεγόμενον Χριστόν;
\par 18 Επειδή ήξευρεν ότι διά φθόνον παρέδωκαν αυτόν.
\par 19 Ενώ δε εκάθητο επί του βήματος, απέστειλε προς αυτόν η γυνή αυτού, λέγουσα· Άπεχε του δικαίου εκείνου· διότι πολλά έπαθον σήμερον κατ' όναρ δι' αυτόν.
\par 20 Οι δε αρχιερείς και οι πρεσβύτεροι έπεισαν τους όχλους να ζητήσωσι τον Βαραββάν, τον δε Ιησούν να απολέσωσι.
\par 21 Και αποκριθείς ο ηγεμών είπε προς αυτούς· Τίνα θέλετε από των δύο να σας απολύσω; οι δε είπον· Τον Βαραββάν.
\par 22 Λέγει προς αυτούς ο Πιλάτος· Τι λοιπόν να κάμω τον Ιησούν τον λεγόμενον Χριστόν; Λέγουσι προς αυτόν πάντες· Σταυρωθήτω.
\par 23 Ο δε ηγεμών είπε· Και τι κακόν έπραξεν; Οι δε περισσότερον έκραζον, λέγοντες· Σταυρωθήτω.
\par 24 Και ιδών ο Πιλάτος ότι ουδέν ωφελεί, αλλά μάλλον θόρυβος γίνεται, λαβών ύδωρ ένιψε τας χείρας αυτού έμπροσθεν του όχλου, λέγων· Αθώος είμαι από του αίματος του δικαίου τούτου· υμείς όψεσθε.
\par 25 Και αποκριθείς πας ο λαός είπε· το αίμα αυτού ας ήναι εφ' ημάς και επί τα τέκνα ημών.
\par 26 Τότε απέλυσεν εις αυτούς τον Βαραββάν, τον δε Ιησούν μαστιγώσας παρέδωκε διά να σταυρωθή.
\par 27 Τότε οι στρατιώται του ηγεμόνος, παραλαβόντες τον Ιησούν εις το πραιτώριον, συνήθροισαν επ' αυτόν όλον το τάγμα των στρατιωτών·
\par 28 και εκδύσαντες αυτόν ενέδυσαν αυτόν χλαμύδα κοκκίνην,
\par 29 και πλέξαντες στέφανον εξ ακανθών, έθεσαν επί την κεφαλήν αυτού και κάλαμον εις την δεξιάν αυτού, και γονυπετήσαντες έμπροσθεν αυτού, ενέπαιζον αυτόν, λέγοντες· Χαίρε, ο βασιλεύς των Ιουδαίων·
\par 30 και εμπτύσαντες εις αυτόν έλαβον τον κάλαμον και έτυπτον εις την κεφαλήν αυτού.
\par 31 Και αφού ενέπαιξαν αυτόν, εξέδυσαν αυτόν την χλαμύδα και ενέδυσαν αυτόν τα ιμάτια αυτού, και έφεραν αυτόν διά να σταυρώσωσιν.
\par 32 Ενώ δε εξήρχοντο, εύρον άνθρωπον Κυρηναίον, ονομαζόμενον Σίμωνα· τούτον ηγγάρευσαν διά να σηκώση τον σταυρόν αυτού.
\par 33 Και ότε ήλθον εις τόπον λεγόμενον Γολγοθά, όστις λέγεται Κρανίου τόπος,
\par 34 έδωκαν εις αυτόν να πίη όξος μεμιγμένον μετά χολής· και γευθείς δεν ήθελε να πίη.
\par 35 Αφού δε εσταύρωσαν αυτόν διεμερίσθησαν τα ιμάτια αυτού, βάλλοντες κλήρον, διά να πληρωθή το ρηθέν υπό του προφήτου, Διεμερίσθησαν τα ιμάτιά μου εις εαυτούς και επί τον ιματισμόν μου έβαλον κλήρον.
\par 36 Και καθήμενοι εφύλαττον αυτόν εκεί.
\par 37 Και έθεσαν επάνωθεν της κεφαλής αυτού την κατηγορίαν αυτού γεγραμμένην· Ούτος εστιν Ιησούς ο βασιλεύς των Ιουδαίων.
\par 38 Τότε εσταυρώθησαν μετ' αυτού δύο λησταί, εις εκ δεξιών και εις εξ αριστερών.
\par 39 οι δε διαβαίνοντες εβλασφήμουν αυτόν, κινούντες τας κεφαλάς αυτών
\par 40 και λέγοντες· Ο χαλών τον ναόν και διά τριών ημερών οικοδομών, σώσον σεαυτόν· αν ήσαι Υιός του Θεού, κατάβα από του σταυρού.
\par 41 Ομοίως δε και οι αρχιερείς εμπαίζοντες μετά των γραμματέων και πρεσβυτέρων, έλεγον.
\par 42 Άλλους έσωσεν, εαυτόν δεν δύναται να σώση· αν ήναι βασιλεύς του Ισραήλ, ας καταβή τώρα από του σταυρού και θέλομεν πιστεύσει εις αυτόν·
\par 43 πέποιθεν επί τον Θεόν, ας σώση τώρα αυτόν, εάν θέλη αυτόν· επειδή είπεν ότι Θεού Υιός είμαι.
\par 44 Το αυτό δε και οι λησταί οι συσταυρωθέντες μετ' αυτού ωνείδιζον εις αυτόν.
\par 45 Από δε έκτης ώρας σκότος έγεινεν εφ' όλην την γην έως ώρας εννάτης·
\par 46 περί δε την εννάτην ώραν ανεβόησεν ο Ιησούς μετά φωνής μεγάλης, λέγων· Ηλί, Ηλί, λαμά σαβαχθανί; τουτέστι, Θεέ μου, Θεέ μου, διά τι με εγκατέλιπες;
\par 47 Και τινές των εκεί εστώτων ακούσαντες, έλεγον ότι τον Ηλίαν φωνάζει ούτος.
\par 48 Και ευθύς έδραμεν εις εξ αυτών και λαβών σπόγγον και γεμίσας όξους και περιθέσας εις κάλαμον επότιζεν αυτόν.
\par 49 Οι δε λοιποί έλεγον· Άφες, ας ίδωμεν αν έρχηται ο Ηλίας να σώση αυτόν.
\par 50 Ο δε Ιησούς πάλιν κράξας μετά φωνής μεγάλης, αφήκε το πνεύμα.
\par 51 Και ιδού, το καταπέτασμα του ναού εσχίσθη εις δύο από άνωθεν έως κάτω, και η γη εσείσθη και αι πέτραι εσχίσθησαν,
\par 52 και τα μνημεία ηνοίχθησαν και πολλά σώματα των κεκοιμημένων αγίων ανέστησαν,
\par 53 και εξελθόντες εκ των μνημείων μετά την ανάστασιν αυτού εισήλθον εις την αγίαν πόλιν και ενεφανίσθησαν εις πολλούς.
\par 54 Ο δε εκατόνταρχος και οι μετ' αυτού φυλάττοντες τον Ιησούν, ιδόντες τον σεισμόν και τα γενόμενα, εφοβήθησαν σφόδρα, λέγοντες· Αληθώς Θεού Υιός ήτο ούτος.
\par 55 Ήσαν δε εκεί γυναίκες πολλαί από μακρόθεν θεωρούσαι, αίτινες ηκολούθησαν τον Ιησούν από της Γαλιλαίας υπηρετούσαι αυτόν·
\par 56 μεταξύ των οποίων ήτο Μαρία η Μαγδαληνή, και Μαρία η μήτηρ του Ιακώβου και Ιωσή, και η μήτηρ των υιών Ζεβεδαίου.
\par 57 Ότε δε έγεινεν εσπέρα, ήλθεν άνθρωπος πλούσιος από Αριμαθαίας, το όνομα Ιωσήφ, όστις και αυτός εμαθήτευσεν εις τον Ιησούν·
\par 58 ούτος ελθών προς τον Πιλάτον, εζήτησε το σώμα του Ιησού. Τότε ο Πιλάτος προσέταξε να αποδοθή το σώμα.
\par 59 Και λαβών το σώμα ο Ιωσήφ, ετύλιξεν αυτό με σινδόνα καθαράν,
\par 60 και έθεσεν αυτό εν τω νέω αυτού μνημείω, το οποίον ελατόμησεν εν τη πέτρα, και προσκυλίσας λίθον μέγαν εις την θύραν του μνημείου ανεχώρησεν.
\par 61 Ήτο δε εκεί Μαρία η Μαγδαληνή και η άλλη Μαρία, καθήμεναι απέναντι του τάφου.
\par 62 Και τη επαύριον, ήτις είναι μετά την παρασκευήν, συνήχθησαν οι αρχιερείς και οι Φαρισαίοι προς τον Πιλάτον
\par 63 λέγοντες· Κύριε, ενεθυμήθημεν ότι εκείνος ο πλάνος είπεν έτι ζων, Μετά τρεις ημέρας θέλω αναστηθή.
\par 64 Πρόσταξον λοιπόν να ασφαλισθή ο τάφος έως της τρίτης ημέρας, μήποτε οι μαθηταί αυτού ελθόντες διά νυκτός κλέψωσιν αυτόν και είπωσι προς τον λαόν, Ανέστη εκ των νεκρών· και θέλει είσθαι η εσχάτη πλάνη χειροτέρα της πρώτης.
\par 65 Είπε δε προς αυτούς ο Πιλάτος· Έχετε φύλακας· υπάγετε, ασφαλίσατε καθώς εξεύρετε.
\par 66 Οι δε υπήγον και ησφάλισαν τον τάφον, σφραγίσαντες τον λίθον και επιστήσαντες τους φύλακας.

\chapter{28}

\par Αφού δε επέρασε το σάββατον, περί τα χαράγματα της πρώτης ημέρας της εβδομάδος ήλθε Μαρία η Μαγδαληνή και η άλλη Μαρία, διά να θεωρήσωσι τον τάφον.
\par 2 Και ιδού έγεινε σεισμός μέγας· διότι άγγελος Κυρίου καταβάς εξ ουρανού ήλθε και απεκύλισε τον λίθον από της θύρας και εκάθητο επάνω αυτού.
\par 3 Ήτο δε η όψις αυτού ως αστραπή και το ένδυμα αυτού λευκόν ως χιών.
\par 4 Και από του φόβου αυτού εταράχθησαν οι φύλακες και έγειναν ως νεκροί.
\par 5 Αποκριθείς δε ο άγγελος είπε προς τας γυναίκας· μη φοβείσθε σείς· διότι εξεύρω ότι Ιησούν τον εσταυρωμένον ζητείτε·
\par 6 δεν είναι εδώ· διότι ανέστη, καθώς είπεν. Έλθετε, ίδετε τον τόπον όπου έκειτο ο Κύριος.
\par 7 Και υπάγετε ταχέως και είπατε προς τους μαθητάς αυτού ότι ανέστη εκ των νεκρών, και ιδού, υπάγει πρότερον υμών εις την Γαλιλαίαν· εκεί θέλετε ιδεί αυτόν· ιδού, σας είπον.
\par 8 Και εξελθούσαι ταχέως από του μνημείου μετά φόβου και χαράς μεγάλης έδραμον να απαγγείλωσι προς τους μαθητάς αυτού.
\par 9 Ενώ δε ήρχοντο να απαγγείλωσι προς τους μαθητάς αυτού, ιδού, ο Ιησούς απήντησεν αυτάς, λέγων· Χαίρετε. Και εκείναι προσελθούσαι επίασαν τους πόδας αυτού και προσεκύνησαν αυτόν.
\par 10 Τότε λέγει προς αυτάς ο Ιησούς· Μη φοβείσθε· υπάγετε, απαγγείλατε προς τους αδελφούς μου, διά να υπάγωσιν εις την Γαλιλαίαν· και εκεί θέλουσι με ιδεί.
\par 11 Ενώ δε αυταί απήρχοντο, ιδού, τινές των φυλάκων ελθόντες εις την πόλιν απήγγειλαν προς τους αρχιερείς πάντα τα γενόμενα,
\par 12 Και συναχθέντες μετά των πρεσβυτέρων και συμβουλευθέντες έδωκαν εις τους στρατιώτας αργύρια ικανά,
\par 13 λέγοντες· Είπατε ότι οι μαθηταί αυτού ελθόντες διά νυκτός έκλεψαν αυτόν, ενώ ημείς εκοιμώμεθα.
\par 14 Και εάν ακουσθή τούτο ενώπιον του ηγεμόνος, ημείς θέλομεν πείσει αυτόν και εσάς θέλομεν κάμει αμερίμνους.
\par 15 Εκείνοι δε λαβόντες τα αργύρια, έπραξαν ως εδιδάχθησαν. Και διεφημίσθη ο λόγος ούτος παρά τοις Ιουδαίοις μέχρι της σήμερον.
\par 16 Οι δε ένδεκα μαθηταί υπήγον εις την Γαλιλαίαν, εις το όρος όπου παρήγγειλεν εις αυτούς ο Ιησούς.
\par 17 Και ιδόντες αυτόν προσεκύνησαν αυτόν, τινές δε εδίστασαν.
\par 18 Και προσελθών ο Ιησούς, ελάλησε προς αυτούς, λέγων· Εδόθη εις εμέ πάσα εξουσία εν ουρανώ και επί γης.
\par 19 Πορευθέντες λοιπόν μαθητεύσατε πάντα τα έθνη, βαπτίζοντες αυτούς εις το όνομα του Πατρός και του Υιού και του Αγίου Πνεύματος,
\par 20 διδάσκοντες αυτούς να φυλάττωσι πάντα όσα παρήγγειλα εις εσάς· και ιδού, εγώ είμαι μεθ' υμών πάσας τας ημέρας έως της συντελείας του αιώνος. Αμήν.


\end{document}