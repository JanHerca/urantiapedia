\begin{document}

\title{2 Χρονικών}


\chapter{1}

\par 1 Και εκραταιώθη ο Σολομών ο υιός του Δαβίδ εις την βασιλείαν αυτού· και Κύριος ο Θεός αυτού ήτο μετ' αυτού, και εμεγάλυνεν αυτόν εις άκρον.
\par 2 Και ελάλησεν ο Σολομών προς πάντα τον Ισραήλ, προς τους χιλιάρχους και εκατοντάρχους και προς τους κριτάς και προς πάντας τους άρχοντας παντός του Ισραήλ, τους αρχηγούς των πατριών·
\par 3 και υπήγαν ο Σολομών και πάσα η σύναξις μετ' αυτού εις τον υψηλόν τόπον τον εν Γαβαών· διότι εκεί ήτο η σκηνή του μαρτυρίου του Θεού, την οποίαν Μωϋσής, ο δούλος του Κυρίου, έκαμεν εν τη ερήμω.
\par 4 Ο δε Δαβίδ είχεν αναβιβάσει την κιβωτόν του Θεού από Κιριάθ-ιαρείμ εις τον τόπον τον οποίον προητοίμασεν ο Δαβίδ δι' αυτήν· διότι είχε στήσει σκηνήν δι' αυτήν εν Ιερουσαλήμ.
\par 5 Και το χαλκούν θυσιαστήριον, το οποίον έκαμε Βεσελεήλ ο υιός του Ουρί, υιού του Ωρ, ήτο εκεί έμπροσθεν της σκηνής του Κυρίου· και εξεζήτησαν αυτό ο Σολομών και η σύναξις.
\par 6 Και ανέβη ο Σολομών εκεί επί το χαλκούν θυσιαστήριον ενώπιον του Κυρίου, το εν τη σκηνή του μαρτυρίου, και προσέφερεν επ' αυτό χίλια ολοκαυτώματα.
\par 7 Κατ' εκείνην την νύκτα εφάνη ο Θεός εις τον Σολομώντα και είπε προς αυτόν, Ζήτησον τι να σοι δώσω.
\par 8 Ο δε Σολομών είπε προς τον Θεόν, Συ έκαμες μέγα έλεος προς Δαβίδ τον πατέρα μου, και με κατέστησας βασιλέα αντ' αυτού·
\par 9 τώρα, Κύριε Θεέ, ας βεβαιωθή ο λόγος σου ο προς τον Δαβίδ τον πατέρα μου· διότι συ με έκαμες βασιλέα επί λαόν πολύν ως το χώμα της γής·
\par 10 δος τώρα εις εμέ σοφίαν και σύνεσιν, διά να εξέρχωμαι και να εισέρχωμαι έμπροσθεν του λαού τούτου· διότι τις δύναται να κρίνη τον λαόν σου τούτον τον μέγαν;
\par 11 Και είπεν ο Θεός προς τον Σολομώντα, Επειδή συνέλαβες τούτο εν τη καρδία σου, και δεν εζήτησας πλούτη, αγαθά και δόξαν ουδέ την ζωήν των μισούντων σε, ουδέ πολυζωΐαν εζήτησας, αλλ' εζήτησας εις σεαυτόν σοφίαν και σύνεσιν, διά να κρίνης τον λαόν μου, επί τον οποίον σε έκαμα βασιλέα·
\par 12 η σοφία και η σύνεσις δίδεται εις σέ· και πλούτον και αγαθά και δόξαν θέλω δώσει εις σε, ως δεν έγεινεν εις τους βασιλείς τους προ σου, ουδέ εις τους μετά σε θέλουσι γείνει τοιαύτα.
\par 13 Τότε επέστρεψεν ο Σολομών εις Ιερουσαλήμ, από του υψηλού τόπου του εν Γαβαών, απ' έμπροσθεν της σκηνής του μαρτυρίου, και εβασίλευεν επί τον Ισραήλ.
\par 14 Και συνήθροισεν ο Σολομών αμάξας και ιππείς· και είχε χιλίας τετρακοσίας αμάξας και δώδεκα χιλιάδας ιππέων, τους οποίους έθεσεν εις τας πόλεις των αμαξών και πλησίον του βασιλέως εν Ιερουσαλήμ.
\par 15 Και κατέστησεν εν Ιερουσαλήμ ο βασιλεύς τον άργυρον και τον χρυσόν ως λίθους, και τας κέδρους κατέστησεν ως τας εν τη πεδιάδι συκαμίνους, διά την αφθονίαν.
\par 16 Εγίνετο δε εις τον Σολομώντα εξαγωγή ίππων και λινού νήματος εξ Αιγύπτου· το μεν λινούν νήμα ελάμβανον οι έμποροι του βασιλέως εις ωρισμένην τιμήν.
\par 17 Ανεβίβαζον δε και έφερον εξ Αιγύπτου μίαν άμαξαν διά εξακοσίους σίκλους αργυρούς, και έκαστον ίππον διά εκατόν πεντήκοντα· και ούτω διά πάντας τους βασιλείς των Χετταίων και διά τους βασιλείς της Συρίας η εξαγωγή εγίνετο διά χειρός αυτών.

\chapter{2}

\par 1 Και απεφάσισεν ο Σολομών να οικοδομήση οίκον εις το όνομα του Κυρίου και οίκον βασιλικόν εις εαυτόν,
\par 2 Και ηρίθμησεν ο Σολομών εβδομήκοντα χιλιάδας ανδρών αχθοφόρων, και ογδοήκοντα χιλιάδας λιθοτόμων εν τω όρει, και τρεις χιλιάδας εξακοσίους επιστάτας επ' αυτών.
\par 3 Και απέστειλεν ο Σολομών προς Χουράμ τον βασιλέα της Τύρου, λέγων, Καθώς έκαμες εις τον Δαβίδ τον πατέρα μου, και έπεμψας προς αυτόν κέδρους διά να οικοδομήση εις εαυτόν οίκον να κατοικήση εν αυτώ, ούτω κάμε και εις εμέ.
\par 4 Ιδού, εγώ οικοδομώ οίκον εις το όνομα Κυρίου του Θεού μου, διά να καθιερώσω τούτον εις αυτόν, διά να προσφέρηται ενώπιον αυτού θυμίαμα ευωδίας και οι παντοτεινοί άρτοι της προθέσεως και τα ολοκαυτώματα τα πρωϊνά και εσπερινά, εν τοις σάββασι και εν ταις νεομηνίαις και εν ταις επισήμοις εορταίς Κυρίου του Θεού ημών· τούτο είναι χρέος του Ισραήλ εις τον αιώνα.
\par 5 Και ο οίκος τον οποίον οικοδομώ είναι μέγας· διότι μέγας ο Θεός ημών υπέρ πάντας τους θεούς.
\par 6 Αλλά τις δύναται να οικοδομήση εις αυτόν οίκον, ενώ ο ουρανός και ο ουρανός των ουρανών δεν είναι ικανοί να χωρέσωσιν αυτόν; Τις δε είμαι εγώ, ώστε να οικοδομήσω οίκον εις αυτόν; ειμή μόνον διά να θυσιάζω ενώπιον αυτού;
\par 7 Τώρα λοιπόν απόστειλον προς εμέ άνδρα σοφόν εις το να εργάζηται εις χρυσόν και εις άργυρον και εις χαλκόν και εις σίδηρον και εις πορφύραν και εις κόκκινον και εις κυανούν, και επιστήμονα εις το εγγλύφειν γλυφάς μετά των σοφών των μετ' εμού εν τη Ιουδαία και εν τη Ιερουσαλήμ, τους οποίους Δαβίδ ο πατήρ μου ητοίμασεν.
\par 8 Απόστειλόν μοι και ξύλα κέδρινα, πεύκινα και ξύλα αλγουμείμ εκ του Λιβάνου· διότι εγώ γνωρίζω ότι οι δούλοί σου εξεύρουσι να κόπτωσι ξύλα εν τω Λιβάνω· και ιδού, οι δούλοί μου θέλουσιν είσθαι μετά των δούλων σου,
\par 9 διά να ετοιμάσωσιν εις εμέ ξύλα εν αφθονία· διότι ο οίκος τον οποίον εγώ οικοδομώ θέλει είσθαι μέγας και θαυμαστός.
\par 10 Και ιδού, θέλω δώσει εις τους δούλους σου τους ξυλοτόμους είκοσι χιλιάδας κόρους σίτου κοπανισμένου, και είκοσι χιλιάδας κόρους κριθής, και είκοσι χιλιάδας βαθ οίνου, και είκοσι χιλιάδας βαθ ελαίου.
\par 11 Και απεκρίθη ο Χουράμ ο βασιλεύς της Τύρου δι' επιστολής, την οποίαν έστειλε προς τον Σολομώντα, Επειδή ο Κύριος ηγάπησε τον λαόν αυτού, σε κατέστησε βασιλέα επ' αυτούς·
\par 12 είπεν έτι ο Χουράμ, Ευλογητός Κύριος ο Θεός του Ισραήλ, ο ποιητής του ουρανού και της γης, όστις έδωκεν εις τον Δαβίδ τον βασιλέα υιόν σοφόν, έχοντα φρόνησιν και σύνεσιν, όστις θέλει οικοδομήσει οίκον εις τον Κύριον και οίκον βασιλικόν εις εαυτόν·
\par 13 αποστέλλω λοιπόν τώρα άνθρωπον σοφόν, έχοντα σύνεσιν, του Χουράμ του πατρός μου,
\par 14 υιόν γυναικός εκ των θυγατέρων Δαν και πατρός Τυρίου, επιστήμονα εις το να εργάζηται εις χρυσόν και εις άργυρον, εις χαλκόν, εις σίδηρον, εις λίθους και εις ξύλα, εις πορφύραν, εις κυανούν και εις βύσσον και εις κόκκινον· και εις το εγγλύφειν παν είδος γλυφής, και εφευρίσκειν πάσαν εφεύρεσιν εις ό,τι προβληθή εις αυτόν, μετά των σοφών σου και μετά των σοφών του κυρίου μου Δαβίδ του πατρός σου·
\par 15 τώρα λοιπόν τον σίτον και την κριθήν, το έλαιον και τον οίνον, τα οποία ο κύριός μου είπεν, ας στείλη προς τους δούλους αυτού·
\par 16 και ημείς θέλομεν κόψει ξύλα εκ του Λιβάνου, κατά πάσαν την χρείαν σου, και θέλομεν φέρει αυτά προς σε με σχεδίας διά θαλάσσης εις Ιόππην· και συ θέλεις αναβιβάσει αυτά εις Ιερουσαλήμ.
\par 17 Και ηρίθμησεν ο Σολομών πάντας τους άνδρας τους ξένους τους εν γη Ισραήλ, μετά τον αριθμόν καθ' ον Δαβίδ ο πατήρ αυτού ηρίθμησεν αυτούς· και ευρέθησαν εκατόν πεντήκοντα τρεις χιλιάδες και εξακόσιοι.
\par 18 Και εξ αυτών έκαμεν εβδομήκοντα χιλιάδας αχθοφόρων, και ογδοήκοντα χιλιάδας λιθοτόμων εν τω όρει, και τρεις χιλιάδας εξακοσίους εργοδιώκτας επί τον λαόν.

\chapter{3}

\par 1 Και ήρχισεν ο Σολομών να οικοδομή τον οίκον του Κυρίου εν Ιερουσαλήμ εν τω όρει Μοριά, όπου εφάνη ο Κύριος εις τον Δαβίδ τον πατέρα αυτού, εν τω τόπω τον οποίον ητοίμασεν ο Δαβίδ εν τω αλωνίω Ορνάν του Ιεβουσαίου.
\par 2 Και ήρχισε να οικοδομή τη δευτέρα του δευτέρου μηνός, εν τω τετάρτω έτει της βασιλείας αυτού.
\par 3 Τούτο δε ήτο το σχέδιον του Σολομώντος διά να οικοδομήση τον οίκον του Θεού· το μήκος εις πήχας, κατά το πρώτον μέτρον, ήτο εξήκοντα πηχών, και το πλάτος είκοσι πηχών,
\par 4 Και το πρόναον, το κατά πρόσωπον του οίκου, είχε μήκος κατά το πλάτος του οίκου είκοσι πηχών, και ύψος εκατόν είκοσι· και εσκέπασεν αυτό έσωθεν με χρυσίον καθαρόν.
\par 5 Και εστέγασε τον οίκον τον μέγαν με ξύλα πεύκινα, τα οποία και εσκέπασε με χρυσόν καθαρόν, και ενέγλυψεν επ' αυτόν φοίνικας και αλύσεις.
\par 6 Και εκόσμησε τον οίκον με λίθους τιμίους διά ώραιότητα· το δε χρυσίον ήτο χρυσίον Φαρουΐμ.
\par 7 Εσκέπασεν έτι με χρυσίον τον οίκον, τας δοκούς, τους παραστάτας και τους τοίχους αυτού και τας θύρας αυτού· και ενέγλυψε χερουβείμ επί των τοίχων.
\par 8 Και έκαμε τον οίκον του αγίου των αγίων, το μήκος αυτού κατά το πλάτος του οίκου, είκοσι πηχών, και το πλάτος αυτού είκοσι πηχών· και εσκέπασεν αυτόν με χρυσίον καθαρόν εξακοσίων ταλάντων.
\par 9 το βάρος δε των καρφίων ήτο πεντήκοντα σίκλοι χρυσίου. Και εσκέπασε τα υπερώα με χρυσίον.
\par 10 Και εν τω οίκω του αγίου των αγίων έκαμε δύο χερουβείμ εργασίας γλυπτής και εσκέπασεν αυτά με χρυσίον.
\par 11 Και αι πτέρυγες των χερουβείμ είχον μήκος είκοσι πηχών· η μία πτέρυξ πέντε πηχών, εγγίζουσα τον τοίχον του οίκου· και η άλλη πτέρυξ πέντε πηχών, εγγίζουσα την πτέρυγα του άλλου χερούβ.
\par 12 Και η μία πτέρυξ του άλλου χερούβ πέντε πηχών, εγγίζουσα τον τοίχον του οίκου· και η άλλη πτέρυξ πέντε πηχών, απτομένη της πτέρυγος του άλλου χερούβ.
\par 13 Αι πτέρυγες των χερουβείμ τούτων εξηπλούντο είκοσι πήχας· και αυτά ίσταντο επί τους πόδας αυτών, τα δε πρόσωπα αυτών έβλεπον προς τον οίκον.
\par 14 Και έκαμε το καταπέτασμα εκ κυανού και πορφύρας και κοκκίνου και βύσσου, και ύφανεν επ' αυτού χερουβείμ.
\par 15 Έκαμεν έτι έμπροσθεν του οίκου δύο στύλους τριάκοντα πέντε πηχών το μήκος, και το επίθεμα το επί της κεφαλής εκάστου, πέντε πηχών.
\par 16 Και έκαμεν αλύσεις εν τω χρηστηρίω, και έβαλεν αυτάς επί των κεφαλών των στύλων· και έκαμεν εκατόν ρόδια και έβαλεν αυτά επί των αλύσεων.
\par 17 Και έστησε τους στύλους κατά πρόσωπον του ναού, ένα εκ δεξιών και ένα εξ αριστερών· και εκάλεσε το όνομα του εκ δεξιών Ιαχείν και το όνομα του εξ αριστερών Βοάς.

\chapter{4}

\par 1 Και έκαμε θυσιαστήριον χαλκούν, είκοσι πηχών το μήκος αυτού, και είκοσι πηχών το πλάτος αυτού, και δέκα πηχών το ύψος αυτού.
\par 2 Έκαμεν έτι την χυτήν θάλασσαν δέκα πηχών από χείλους εις χείλος, στρογγύλην κύκλω, και το ύψος αυτής πέντε πηχών· και γραμμή τριάκοντα πηχών περιεζώννυεν αυτήν κύκλω.
\par 3 Και υπό το χείλος αυτής ήτο ομοίωμα βοών, περικυκλούντων αυτήν κύκλω, δέκα κατά πήχην, περικυκλούντες την θάλασσαν κύκλω. Αι δύο σειραί των βοών ήσαν χυμέναι ομού με αυτήν.
\par 4 Ίστατο δε επί δώδεκα βοών· τρεις έβλεπον προς βορράν, και τρεις έβλεπον προς δυσμάς, και τρεις έβλεπον προς νότον, και τρεις έβλεπον προς ανατολάς· και η θάλασσα έκειτο επ' αυτών· και όλα τα οπίσθια αυτών ήσαν προς τα έσω.
\par 5 Και το πάχος αυτής ήτο μιας παλάμης και το χείλος αυτής κατεσκευασμένον ως χείλος ποτηρίου, ως άνθος κρίνου· εχώρει δε πλήρης ούσα τρεις χιλιάδας βαθ.
\par 6 Έκαμεν έτι δέκα λουτήρας, και έθεσε πέντε εκ δεξιών και πέντε εξ αριστερών, διά να πλύνωσιν εν αυτοίς· εκεί έπλυνον όσα ήσαν διά ολοκαύτωσιν· η θάλασσα όμως ήτο διά να νίπτωνται εν αυτή οι ιερείς.
\par 7 Και έκαμε τας χρυσάς λυχνίας δέκα, κατά το διατεταγμένον περί αυτών, και έθεσεν αυτάς εν τω ναώ, πέντε εκ δεξιών και πέντε εξ αριστερών.
\par 8 Και έκαμε δέκα τραπέζας, και έθεσεν αυτάς εν τω ναώ, πέντε εκ δεξιών και πέντε εξ αριστερών. Και έκαμεν εκατόν χρυσάς λεκάνας.
\par 9 Και έκαμε την αυλήν των ιερέων και την μεγάλην αυλήν, και θύρας διά την αυλήν, και εσκέπασε τας θύρας αυτών με χαλκόν.
\par 10 Και έθεσε την θάλασσαν κατά το δεξιόν πλευρόν προς ανατολάς, απέναντι του μεσημβρινού μέρους.
\par 11 Και έκαμεν ο Χουράμ τους λέβητας και τα πτυάρια και τας λεκάνας· και ετελείωσεν ο Χουράμ κάμνων το έργον, το οποίον έκαμνεν εις τον βασιλέα Σολομώντα διά τον οίκον του Θεού·
\par 12 τους δύο στύλους, και τας σφαίρας και τα δύο επιθέματα τα επί της κεφαλής των στύλων και τα δύο δικτυωτά διά να σκεπάζωσι τας δύο σφαίρας των επιθεμάτων των επί της κεφαλής των στύλων·
\par 13 και τετρακόσια ρόδια διά τα δύο δικτυωτά, δύο σειράς ροδίων δι' έκαστον δικτυωτόν, διά να σκεπάζωσι τας δύο σφαίρας των επιθεμάτων των επί των στύλων.
\par 14 Έκαμεν έτι τας βάσεις και έκαμε τους λουτήρας επί των βάσεων·
\par 15 την μίαν θάλασσαν και τους δώδεκα βόας υποκάτω αυτής.
\par 16 Και τους λέβητας και τα πτυάρια και τας κρεάγρας και πάντα τα σκεύη αυτών έκαμε Χουράμ ο πατήρ αυτού εις τον βασιλέα Σολομώντα διά τον οίκον του Κυρίου, εκ λαμπρού χαλκού.
\par 17 Εν τη πεδιάδι του Ιορδάνου έχυσεν αυτά ο βασιλεύς, εν γη αργιλλώδει μεταξύ Σοκχώθ και Σαρηδαθά.
\par 18 Ούτως έκαμεν ο Σολομών πάντα ταύτα τα σκεύη εν αφθονία μεγάλη· διότι δεν ηδύνατο να λογαριασθή το βάρος του χαλκού.
\par 19 Και έκαμεν ο Σολομών πάντα τα σκεύη τα του οίκου του Θεού και το θυσιαστήριον το χρυσούν και τας τραπέζας, και επ' αυτών ετίθεντο οι άρτοι της προθέσεως·
\par 20 και τας λυχνίας και τους λύχνους αυτών, διά να καίωσι κατά το διατεταγμένον ενώπιον του χρηστηρίου, εκ χρυσίου καθαρού·
\par 21 και τα άνθη και τους λύχνους και τας λαβίδας εκ χρυσίου, και τούτο χρυσίον καθαρόν·
\par 22 και τα λυχνοψάλιδα και τας λεκάνας και τους κρατήρας και τα θυμιατήρια εκ χρυσίου καθαρού, και η είσοδος του οίκου, αι εσώτεραι θύραι αυτού διά το άγιον των αγίων και αι θύραι του οίκου του ναού ήσαν εκ χρυσίου.

\chapter{5}

\par 1 Και συνετελέσθη άπαν το έργον το οποίον έκαμεν ο Σολομών διά τον οίκον του Κυρίου· και εισέφερεν ο Σολομών τα αφιερώματα Δαβίδ του πατρός αυτού· και το αργύριον και το χρυσίον και πάντα τα σκεύη έθεσεν εν τοις θησαυροίς του οίκου του Θεού.
\par 2 Τότε συνήθροισεν ο Σολομών εις Ιερουσαλήμ τους πρεσβυτέρους του Ισραήλ και πάντας τους αρχηγούς των φυλών, τους οικογενάρχας των υιών Ισραήλ, διά να αναβιβάσωσι την κιβωτόν της διαθήκης. του Κυρίου εκ της πόλεως Δαβίδ, ήτις είναι η Σιών.
\par 3 Και συνηθροίσθησαν πάντες οι άνδρες Ισραήλ προς τον βασιλέα εν τη εορτή του εβδόμου μηνός.
\par 4 Και ήλθον πάντες οι πρεσβύτεροι του Ισραήλ· και εσήκωσαν οι Λευΐται την κιβωτόν.
\par 5 Και ανεβίβασαν την κιβωτόν και την σκηνήν του μαρτυρίου και πάντα τα σκεύη τα άγια τα εν τη σκηνή· οι ιερείς και οι Λευΐται ανεβίβασαν αυτά.
\par 6 Και ο βασιλεύς Σολομών και πάσα η συναγωγή του Ισραήλ, οι συναχθέντες προς αυτόν, ήσαν έμπροσθεν της κιβωτού, θυσιάζοντες πρόβατα και βόας, όσα δεν ήτο δυνατόν να λογαριασθώσιν ουδέ να αριθμηθώσι διά το πλήθος.
\par 7 Και εισήγαγον οι ιερείς την κιβωτόν της διαθήκης του Κυρίου εις τον τόπον αυτής, εις το χρηστήριον του οίκου, εις τα άγια των αγίων, υποκάτω των πτερύγων των χερουβείμ·
\par 8 διότι τα χερουβείμ είχον εξηπλωμένας τας πτέρυγας επί τον τόπον της κιβωτού, και τα χερουβείμ εκάλυπτον την κιβωτόν και τους μοχλούς αυτής άνωθεν·
\par 9 και εξείχον οι μοχλοί, και εφαίνοντο τα άκρα των μοχλών έξω της κιβωτού έμπροσθεν του χρηστηρίου· έξωθεν όμως δεν εφαίνοντο. Και είναι εκεί έως της σήμερον.
\par 10 Δεν ήτο εν τη κιβωτώ ειμή αι δύο πλάκες, τας οποίας έθεσεν ο Μωϋσής εκεί εν Χωρήβ, όπου ο Κύριος έκαμε διαθήκην προς τους υιούς Ισραήλ, ότε εξήλθον εξ Αιγύπτου.
\par 11 Και ως εξήλθον οι ιερείς εκ του αγιαστηρίου, διότι πάντες οι ιερείς οι ευρεθέντες ηγιάσθησαν, χωρίς να ήναι διατεταγμένοι κατά διαιρέσεις·
\par 12 και οι Λευΐται οι ψαλτωδοί, πάντες οι του Ασάφ, του Αιμάν, του Ιεδουθούν, και οι υιοί αυτών και οι αδελφοί αυτών, ενδεδυμένοι βύσσον, εν κυμβάλοις και ψαλτηρίοις και κιθάραις, ίσταντο κατά ανατολάς του θυσιαστηρίου, και μετ' αυτών εκατόν είκοσι ιερείς σαλπίζοντες διά σαλπίγγων·
\par 13 τότε, ως ήχησαν οι σαλπιγκταί και οι ψαλτωδοί ομού μιά φωνή, υμνούντες και δοξολογούντες τον Κύριον, και καθώς ύψωσαν την φωνήν διά σαλπίγγων και κυμβάλων και οργάνων μουσικών, και ύμνουν τον Κύριον, λέγοντες, Ότι είναι αγαθός, ότι εις τον αιώνα το έλεος αυτού, τότε ο οίκος ενεπλήσθη νεφέλης, ο οίκος του Κυρίου,
\par 14 και δεν ηδύναντο οι ιερείς να σταθώσι διά να λειτουργήσωσιν, εξ αιτίας της νεφέλης· διότι η δόξα του Κυρίου ενέπλησε τον οίκον του Θεού.

\chapter{6}

\par 1 Τότε ελάλησεν ο Σολομών, Ο Κύριος είπεν ότι θέλει κατοικεί εν γνόφω·
\par 2 αλλ' εγώ ωκοδόμησα εις σε οίκον κατοικήσεως και τόπον διά να κατοικής αιωνίως.
\par 3 Και στρέψας ο βασιλεύς το πρόσωπον αυτού, ευλόγησε πάσαν την συναγωγήν του Ισραήλ· πάσα δε η συναγωγή του Ισραήλ ίστατο.
\par 4 Και είπεν, Ευλογητός Κύριος ο Θεός του Ισραήλ, όστις εξετέλεσε διά των χειρών αυτού εκείνο το οποίον ελάλησε διά του στόματος αυτού προς Δαβίδ τον πατέρα μου, λέγων,
\par 5 Αφ' ης ημέρας εξήγαγον τον λαόν μου εκ γης Αιγύπτου, δεν εξέλεξα από πασών των φυλών του Ισραήλ ουδεμίαν πόλιν, διά να οικοδομηθή οίκος, ώστε να ήναι το όνομά μου εκεί· ουδέ εξέλεξα άνδρα, διά να ήναι κυβερνήτης επί τον λαόν μου Ισραήλ·
\par 6 αλλ' εξέλεξα την Ιερουσαλήμ, διά να ήναι το όνομά μου εκεί· και εξέλεξα τον Δαβίδ, διά να ήναι επί τον λαόν μου Ισραήλ.
\par 7 Και ήλθεν εις την καρδίαν Δαβίδ του πατρός μου να οικοδομήση οίκον εις το όνομα Κυρίου του Θεού του Ισραήλ.
\par 8 Αλλ' ο Κύριος είπε προς Δαβίδ τον πατέρα μου, Επειδή ήλθεν εις την καρδίαν σου να οικοδομήσης οίκον εις το όνομά μου, καλώς μεν έκαμες ότι συνέλαβες τούτο εν τη καρδία σου·
\par 9 πλην συ δεν θέλεις οικοδομήσει τον οίκον· αλλ' ο υιός σου, όστις θέλει εξέλθει εκ της οσφύος σου, ούτος θέλει οικοδομήσει τον οίκον εις το όνομά μου.
\par 10 Ο Κύριος λοιπόν επλήρωσε τον λόγον αυτού τον οποίον ελάλησε· και εγώ ανέστην αντί Δαβίδ του πατρός μου και εκάθησα επί του θρόνου του Ισραήλ, καθώς ελάλησε Κύριος, και ωκοδόμησα τον οίκον εις το όνομα Κυρίου του Θεού του Ισραήλ·
\par 11 και έθεσα εκεί την κιβωτόν, εν ή κείται η διαθήκη του Κυρίου, την οποίαν έκαμε προς τους υιούς Ισραήλ.
\par 12 Και σταθείς ο Σολομών έμπροσθεν του θυσιαστηρίου του Κυρίου, ενώπιον πάσης της συναγωγής του Ισραήλ, εξέτεινε τας χείρας αυτού·
\par 13 διότι ο Σολομών έκαμε βάσιν χαλκίνην, έχουσαν πέντε πηχών μήκος, και πέντε πηχών πλάτος, και τριών πηχών ύψος· και έθεσεν αυτήν εν τω μέσω της αυλής· και σταθείς επ' αυτής έπεσεν επί τα γόνατα αυτού ενώπιον πάσης της συναγωγής του Ισραήλ και εξέτεινε τας χείρας αυτού προς τον ουρανόν,
\par 14 και είπε, Κύριε Θεέ του Ισραήλ, δεν είναι Θεός όμοιός σου εν τω ουρανώ και επί της γής· όστις φυλάττεις την διαθήκην και το έλεος προς τους δούλους σου, τους περιπατούντας ενώπιόν σου εν όλη τη καρδία αυτών·
\par 15 όστις εφύλαξας προς τον δούλον σου Δαβίδ τον πατέρα μου όσα ελάλησας προς αυτόν, και ελάλησας διά του στόματός σου και εξετέλεσας διά της χειρός σου, καθώς την ημέραν ταύτην.
\par 16 Και τώρα, Κύριε Θεέ του Ισραήλ, φύλαξον προς τον δούλον σου Δαβίδ τον πατέρα μου εκείνο το οποίον υπεσχέθης προς αυτόν, λέγων, Δεν θέλει εκλείψει εις σε ανήρ απ' έμπροσθέν μου καθήμενος επί του θρόνου του Ισραήλ, μόνον εάν προσέχωσιν οι υιοί σου εις την οδόν αυτών, διά να περιπατώσιν εις τον νόμον μου, καθώς συ περιεπάτησας ενώπιόν μου.
\par 17 Τώρα λοιπόν, Κύριε Θεέ του Ισραήλ, ας αληθεύση ο λόγος σου, τον οποίον ελάλησας προς τον δούλον σου τον Δαβίδ.
\par 18 Αλλά θέλει αληθώς κατοικήσει Θεός μετά ανθρώπου επί της γης; Ιδού, ο ουρανός, και ο ουρανός των ουρανών, δεν είναι ικανοί να σε χωρέσωσι πόσον ολιγώτερον ο οίκος ούτος, τον οποίον ωκοδόμησα;
\par 19 Πλην επίβλεψον επί την προσευχήν του δούλου σου και επί την δέησιν αυτού, Κύριε Θεέ μου, ώστε να επακούσης της κραυγής και της δεήσεως, την οποίαν ο δούλός σου δέεται ενώπιόν σου·
\par 20 διά να ήναι οι οφθαλμοί σου ανεωγμένοι προς τον οίκον τούτον ημέραν και νύκτα, προς τον τόπον περί του οποίου είπας ότι θέλεις θέσει το όνομά σου εκεί, διά να επακούης της δεήσεως την οποίαν ο δούλός σου θέλει δέεσθαι εν τω τόπω τούτω.
\par 21 Και επάκουε των δεήσεων του δούλου σου και του λαού σου Ισραήλ, όταν προσεύχωνται εν τω τόπω τούτω· και άκουε συ εκ του τόπου της κατοικήσεώς σου, εκ του ουρανού· και ακούων, γίνου ίλεως.
\par 22 Εάν αμαρτήση άνθρωπος εις τον πλησίον αυτού και ζητήση όρκον παρ' αυτού διά να κάμη αυτόν να ορκισθή, και ο όρκος έλθη έμπροσθεν του θυσιαστηρίου σου εν τω οίκω τούτω,
\par 23 τότε συ επάκουσον εκ του ουρανού και ενέργησον και κρίνον τους δούλους σου, ανταποδίδων μεν εις τον άνομον, ώστε να στρέψης κατά της κεφαλής αυτού την πράξιν αυτού, δικαιόνων δε τον δίκαιον, ώστε να αποδώσης εις αυτόν κατά την δικαιοσύνην αυτού.
\par 24 Και εάν κτυπηθή ο λαός σου Ισραήλ έμπροσθεν του εχθρού, διότι ημάρτησαν εις σε, και επιστρέψωσι και δοξάσωσι το όνομά σου και προσευχηθώσι και δεηθώσι προς σε εν τω οίκω τούτω,
\par 25 τότε συ επάκουσον εκ του ουρανού και συγχώρησον την αμαρτίαν του λαού σου Ισραήλ, και επανάγαγε αυτούς εις την γην την οποίαν έδωκας εις αυτούς και εις τους πατέρας αυτών.
\par 26 Όταν ο ουρανός κλεισθή και δεν γίνηται βροχή, διότι ημάρτησαν εις σε, εάν προσευχηθώσι προς τον τόπον τούτον και δοξάσωσι το όνομά σου και επιστρέψωσιν από των αμαρτιών αυτών, αφού ταπεινώσης αυτούς,
\par 27 τότε συ επάκουσον εκ του ουρανού και συγχώρησον την αμαρτίαν των δούλων σου και του λαού σου Ισραήλ, διδάξας αυτούς την οδόν την αγαθήν εις την οποίαν πρέπει να περιπατώσι και δος βροχήν επί την γην σου, την οποίαν έδωκας εις τον λαόν σου διά κληρονομίαν.
\par 28 Πείνα εάν γείνη εκ τη γη, θανατικόν εάν γείνη, ανεμοφθορία και ερυσίβη, ακρίς και βρούχος εάν γείνη, οι εχθροί αυτών εάν πολιορκήσωσιν αυτούς εν τω τόπω της κατοικίας αυτών, οποιαδήποτε πληγή και οποιαδήποτε νόσος γείνη,
\par 29 πάσαν προσευχήν, πάσαν δέησιν γινομένην υπό παντός ανθρώπου και υπό παντός του λαού σου Ισραήλ, όταν γνωρίση έκαστος την πληγήν αυτού και τον πόνον αυτού και εκτείνη τας χείρας αυτού προς τον οίκον τούτον,
\par 30 τότε συ επάκουσον εκ του ουρανού, του τόπου της κατοικήσεώς σου, και συγχώρησον και δος εις έκαστον κατά πάσας τας οδούς αυτού, όπως γνωρίζεις την καρδίαν αυτού, διότι συ, μόνος συ, γνωρίζεις τας καρδίας των υιών των ανθρώπων·
\par 31 διά να σε φοβώνται, ώστε να περιπατώσιν εν ταις οδοίς σου πάσας τας ημέρας όσας ζώσιν επί προσώπου της γης, την οποίαν έδωκας εις τους πατέρας ημών.
\par 32 Και τον ξένον έτι, όστις δεν είναι εκ του λαού σου Ισραήλ, αλλ' έρχεται από γης μακράς διά το όνομά σου το μέγα, και διά την χείρα σου την κραταιάν, και διά τον βραχίονά σου τον εξηπλωμένον, εάν έλθωσι και προσευχηθώσι προς τον οίκον τούτον,
\par 33 τότε συ επάκουσον εκ του ουρανού, εκ του τόπου της κατοικήσεώς σου, και κάμε κατά πάντα περί όσων ο ξένος σε επικαλεσθή, διά να γνωρίσωσι πάντες οι λαοί της γης το όνομά σου και να σε φοβώνται, καθώς ο λαός σου ο Ισραήλ, και διά να γνωρίσωσιν ότι το όνομά σου εκλήθη επί τον οίκον τούτον, τον οποίον ωκοδόμησα.
\par 34 Όταν ο λαός σου εξέλθη εις πόλεμον εναντίον των εχθρών αυτών, διά της οδού δι' ης αποστείλης αυτούς, και προσευχηθώσιν εις σε προς την πόλιν ταύτην την οποίαν εξέλεξας, και τον οίκον τον οποίον ωκοδόμησα εις το όνομά σου,
\par 35 τότε επάκουσον εκ του ουρανού της προσευχής αυτών και της δεήσεως αυτών, και κάμε το δίκαιον αυτών.
\par 36 Όταν αμαρτήσωσιν εις σε, διότι ουδείς άνθρωπος είναι αναμάρτητος, και οργισθής εις αυτούς, και παραδώσης αυτούς έμπροσθεν του εχθρού, και οι αιχμαλωτισταί φέρωσιν αυτούς αιχμαλώτους εις γην μακράν ή πλησίον,
\par 37 και έλθωσιν εις εαυτούς εν τη γη όπου εφέρθησαν αιχμάλωτοι, και επιστρέψωσι και δεηθώσι προς σε εν τη γη της αιχμαλωσίας αυτών, λέγοντες, Ημάρτομεν, ηνομήσαμεν και ηδικήσαμεν·
\par 38 και επιστρέψωσι προς σε εξ όλης της καρδίας αυτών και εξ όλης της ψυχής αυτών, εν τη γη της αιχμαλωσίας αυτών όπου εφέρθησαν αιχμάλωτοι, και προσευχηθώσι προς την γην αυτών την οποίαν έδωκας εις τους πατέρας αυτών, και την πόλιν την οποίαν εξέλεξας, και προς τον οίκον τον οποίον ωκοδόμησα εις το όνομά σου,
\par 39 τότε επάκουσον εκ του ουρανού, εκ του τόπου της κατοικήσεώς σου, της προσευχής αυτών και των δεήσεων αυτών, και κάμε το δίκαιον αυτών και συγχώρησον εις τον λαόν σου τον αμαρτήσαντα εις σε.
\par 40 Τώρα, Θεέ μου, ας ήναι, δέομαι, ανεωγμένοι οι οφθαλμοί σου και προσεκτικά τα ώτα σου εις την προσευχήν την γινομένην εν τω τόπω τούτω.
\par 41 Και τώρα, ανάστηθι, Κύριε Θεέ, εις την ανάπαυσίν σου, συ και η κιβωτός της δυνάμεώς σου· οι ιερείς σου, Κύριε Θεέ, ας ενδυθώσι σωτηρίαν, και οι όσιοί σου ας ευφρανθώσιν εν αγαθοίς.
\par 42 Κύριε Θεέ, μη απορρίψης το πρόσωπον του κεχρισμένου σου. ενθυμήθητι τα ελέη Δαβίδ του δούλου σου.

\chapter{7}

\par 1 Και αφού ετελείωσεν ο Σολομών προσευχόμενος, κατέβη το πυρ εκ του ουρανού και κατέφαγε τα ολοκαυτώματα και τας θυσίας· και δόξα Κυρίου ενέπλησε τον οίκον.
\par 2 Και δεν ηδύναντο οι ιερείς να εισέλθωσιν εις τον οίκον του Κυρίου, διότι δόξα Κυρίου ενέπλησε τον οίκον του Κυρίου.
\par 3 Πάντες δε οι υιοί Ισραήλ, βλέποντες το πυρ καταβαίνον και την δόξαν του Κυρίου επί τον οίκον, έπεσον κατά πρόσωπον επί την γην, επί το λιθόστρωτον, και προσεκύνησαν και εδόξασαν τον Κύριον, λέγοντες, Ότι είναι αγαθός· ότι εις τον αιώνα το έλεος αυτού.
\par 4 Τότε ο βασιλεύς και πας ο λαός προσέφεραν θυσίας ενώπιον του Κυρίου·
\par 5 και εθυσίασεν ο βασιλεύς Σολομών την θυσίαν, εικοσιδύο χιλιάδας βοών και εκατόν είκοσι χιλιάδας προβάτων. Ούτως εγκαινίασαν ο βασιλεύς και πας ο λαός τον οίκον του Θεού.
\par 6 Και ίσταντο οι ιερείς εις τας υπηρεσίας αυτών, και οι Λευΐται μετά των μουσικών οργάνων του Κυρίου, τα οποία Δαβίδ ο βασιλεύς έκαμε διά να δοξάζωσι τον Κύριον, Ότι εις τον αιώνα το έλεος αυτού, έχοντες εν ταις χερσίν αυτών τους ύμνους του Δαβίδ· και εσάλπιζον οι ιερείς κατέναντι αυτών, και πας ο Ισραήλ ίστατο.
\par 7 Καθιέρωσεν έτι ο Σολομών το μέσον της αυλής, της κατά πρόσωπον του οίκου του Κυρίου· διότι εκεί προσέφερε τα ολοκαυτώματα και το στέαρ των ειρηνικών προσφορών· επειδή το θυσιαστήριον το χάλκινον, το οποίον ο Σολομών έκαμε, δεν ηδύνατο να χωρέση τα ολοκαυτώματα και την εξ αλφίτων προσφοράν και το στέαρ.
\par 8 Και κατ' εκείνον τον καιρόν έκαμεν ο Σολομών την εορτήν επτά ημέρας, και πας ο Ισραήλ μετ' αυτού, σύναξις μεγάλη σφόδρα, από της εισόδου Αιμάθ μέχρι του ποταμού της Αιγύπτου.
\par 9 Και εν τη ογδόη ημέρα έκαμον σύναξιν πάνδημον· διότι έκαμον τον εγκαινιασμόν του θυσιαστηρίου επτά ημέρας, και την εορτήν επτά ημέρας.
\par 10 Και εν τη εικοστή τρίτη ημέρα του εβδόμου μηνός απέλυσε τον λαόν εις τας σκηνάς αυτών, ευφραινομένους και αγαλλομένους την καρδίαν διά τα αγαθά όσα έκαμεν ο Κύριος προς τον Δαβίδ και προς τον Σολομώντα και προς τον Ισραήλ τον λαόν αυτού.
\par 11 Και ετελείωσεν ο Σολομών τον οίκον του Κυρίου και τον οίκον του βασιλέως· και παν ό,τι ήλθεν εις την καρδίαν του Σολομώντος να κάμη εν τω οίκω του Κυρίου και εν τω οίκω αυτού ευωδώθη.
\par 12 Και εφάνη ο Κύριος εις τον Σολομώντα διά νυκτός, και είπε προς αυτόν, Ήκουσα της προσευχής σου και εξέλεξα τον τόπον τούτον εις εμαυτόν διά οίκον θυσίας.
\par 13 Εάν κλείσω τον ουρανόν και δεν γίνηται βροχή, και εάν προστάξω την ακρίδα να καταφάγη την γην, και εάν αποστείλω θανατικόν μεταξύ του λαού μου,
\par 14 και ο λαός μου, επί τον οποίον εκλήθη το όνομά μου, ταπεινώσωσιν εαυτούς και προσευχηθώσι και εκζητήσωσι το πρόσωπόν μου και επιστρέψωσιν από των οδών αυτών των πονηρών, τότε εγώ θέλω επακούσει εκ του ουρανού και θέλω συγχωρήσει την αμαρτίαν αυτών και θεραπεύσει την γην αυτών.
\par 15 Τώρα οι οφθαλμοί μου θέλουσιν είσθαι ανεωγμένοι και τα ώτα μου προσεκτικά εις την προσευχήν την γινομένην εν τω τόπω τούτω.
\par 16 Διότι τώρα εξέλεξα και ηγίασα τον οίκον τούτον, διά να ήναι το όνομά μου εκεί έως αιώνος· και οι οφθαλμοί μου και η καρδία μου θέλουσιν είσθαι εκεί πάσας τας ημέρας.
\par 17 Και συ, εάν περιπατής ενώπιόν μου, καθώς περιεπάτησε Δαβίδ ο πατήρ σου, και κάμνης κατά πάντα όσα προσέταξα εις σε, και φυλάττης τα διατάγματά μου και τας κρίσεις μου,
\par 18 τότε θέλω στερεώσει τον θρόνον της βασιλείας σου, καθώς υπεσχέθην προς Δαβίδ τον πατέρα σου, λέγων, Δεν θέλει εκλείψει εις σε ανήρ ηγεμονεύων επί τον Ισραήλ.
\par 19 Αλλ' εάν σεις αποστρέψητε και εγκαταλείψητε τα διατάγματά μου και τας εντολάς μου, τας οποίας έθεσα έμπροσθέν σας, και υπάγητε και λατρεύσητε άλλους θεούς και προσκυνήσητε αυτούς,
\par 20 τότε θέλω εκριζώσει αυτούς από της γης μου, την οποίαν έδωκα εις αυτούς· και τον οίκον τούτον, τον οποίον ηγίασα διά το όνομά μου, θέλω απορρίψει από προσώπου μου και θέλω κάμει αυτόν παροιμίαν και εμπαιγμόν μεταξύ πάντων των λαών.
\par 21 Και ο οίκος ούτος, όστις έγεινε τόσον υψηλός, θέλει είσθαι έκστασις εις πάντας τους διαβαίνοντας παρ' αυτόν· και θέλουσι λέγει, Διά τι ο Κύριος έκαμεν ούτως εις την γην ταύτην και εις τον οίκον τούτον;
\par 22 Και θέλουσιν αποκρίνεσθαι, Επειδή εγκατέλιπον Κύριον τον Θεόν των πατέρων αυτών, όστις εξήγαγεν αυτούς εξ Αιγύπτου, και προσεκολλήθησαν εις άλλους θεούς και προσεκύνησαν αυτούς και ελάτρευσαν αυτούς· διά τούτο επέφερεν επ' αυτούς άπαν τούτο το κακόν.

\chapter{8}

\par 1 Εν δε τω τέλει των είκοσι ετών, καθ' α ο Σολομών ωκοδόμησε τον οίκον του Κυρίου και τον οίκον εαυτού,
\par 2 τας πόλεις τας οποίας ο Χουράμ είχε δώσει εις τον Σολομώντα, ο Σολομών ωκοδόμησεν αυτάς και κατώκισεν εκεί τους υιούς Ισραήλ.
\par 3 Και υπήγεν ο Σολομών εις Αιμάθ-σωβά και υπερίσχυσεν εναντίον αυτής.
\par 4 Και ωκοδόμησε την Θαδμώρ εν τη ερήμω και πάσας τας πόλεις των αποθηκών, τας οποίας ωκοδόμησεν εν Αιμάθ.
\par 5 Ωικοδόμησεν έτι την Βαιθ-ωρών την άνω και την Βαιθ-ωρών την κάτω, πόλεις ωχυρωμένας με τείχη, πύλας και μοχλούς·
\par 6 και την Βααλάθ και πάσας τας πόλεις των αποθηκών, τας οποίας είχεν ο Σολομών, και πάσας τας πόλεις των αμαξών και τας πόλεις των ιππέων και παν ό,τι επεθύμησεν ο Σολομών να οικοδομήση εν Ιερουσαλήμ και εν τω Λιβάνω και εν πάση τη γη της επικρατείας αυτού.
\par 7 Πάντα δε τον λαόν τον εναπολειφθέντα εκ των Χετταίων και των Αμορραίων και των Φερεζαίων και των Ευαίων και των Ιεβουσαίων, οίτινες δεν ήσαν εκ του Ισραήλ,
\par 8 αλλ' εκ των τέκνων εκείνων, των εναπολειφθέντων εν τη γη μετ' αυτούς, τους οποίους οι υιοί Ισραήλ δεν εξωλόθρευσαν, επί τούτους ο Σολομών επέβαλε φόρον έως της ημέρας ταύτης.
\par 9 Εκ δε των υιών Ισραήλ ο Σολομών δεν έκαμε δούλους διά το έργον αυτού, διότι ήσαν άνδρες πολεμισταί, και πρώταρχοι και άρχοντες των αμαξών αυτού και των ιππέων αυτού.
\par 10 Εκ τούτων ήσαν οι αρχηγοί των επιστατών, τους οποίους είχεν ο βασιλεύς Σολομών, διακόσιοι πεντήκοντα, εξουσιάζοντες επί τον λαόν.
\par 11 Και ανεβίβασεν ο Σολομών την θυγατέρα του Φαραώ εκ της πόλεως Δαβίδ, εις τον οίκον τον οποίον ωκοδόμησε δι' αυτήν· διότι είπεν, Η γυνή μου δεν θέλει κατοικεί εν τω οίκω Δαβίδ του βασιλέως του Ισραήλ, επειδή το μέρος, όπου η κιβωτός του Κυρίου εισήλθεν, είναι άγιον.
\par 12 Τότε προσέφερεν ο Σολομών ολοκαυτώματα εις τον Κύριον επί του θυσιαστηρίου του Κυρίου, το οποίον ωκοδόμησε κατ' έμπροσθεν του προνάου,
\par 13 κατά το απαιτούμενον εκάστης ημέρας του να προσφέρωσι κατά τας εντολάς του Μωϋσέως, εν τοις σάββασι και εν ταις νεομηνίαις και εν ταις επισήμοις εορταίς ταις γινομέναις τρίς του ενιαυτού, εν τη εορτή των αζύμων και εν τη εορτή των εβδομάδων και εν τη εορτή των σκηνών.
\par 14 Και κατέστησε, κατά την διάταξιν Δαβίδ του πατρός αυτού, τας διαιρέσεις των ιερέων εις την υπηρεσίαν αυτών, και τους Λευΐτας εις τας φυλακάς αυτών διά να υμνώσι και να λειτουργώσι κατέναντι των ιερέων, κατά το απαιτούμενον εκάστης ημέρας· και τους πυλωρούς κατά τας διαιρέσεις αυτών εις εκάστην πύλην· διότι τοιαύτη ήτο η εντολή Δαβίδ του ανθρώπου του Θεού.
\par 15 Και δεν παρεδρόμησαν από της εντολής του βασιλέως περί των ιερέων και Λευϊτών εις ουδέν πράγμα ουδέ εις τα περί των θησαυρών.
\par 16 Ήτο δε ετοιμασία δι' άπαν το έργον του Σολομώντος, αφ' ης ημέρας εθεμελιώθη ο οίκος του Κυρίου, εωσού εξετελέσθη. Ούτως ετελειώθη ο οίκος του Κυρίου.
\par 17 Τότε υπήγεν ο Σολομών εις Εσιών-γάβερ και εις Αιλώθ, επί το χείλος της θαλάσσης εν τη γη Εδώμ.
\par 18 Και απέστειλεν ο Χουράμ προς αυτόν, διά χειρός των δούλων αυτού πλοία και δούλους ειδήμονας της θαλάσσης· και υπήγαν μετά των δούλων του Σολομώντος εις Οφείρ, και έλαβον εκείθεν τετρακόσια πεντήκοντα τάλαντα χρυσίου και έφεραν αυτά προς τον βασιλέα Σολομώντα.

\chapter{9}

\par 1 Ακούσασα δε η βασίλισσα της Σεβά την φήμην του Σολομώντος, ήλθεν εις Ιερουσαλήμ, διά να δοκιμάση τον Σολομώντα δι' αινιγμάτων, έχουσα συνοδίαν μεγάλην σφόδρα και καμήλους φορτωμένας αρώματα και χρυσόν άφθονον και λίθους πολυτίμους· και, ότε ήλθε προς τον Σολομώντα, ελάλησε μετ' αυτού περί πάντων όσα είχεν εν τη καρδία αυτής.
\par 2 Και εξήγησεν εις αυτήν ο Σολομών πάντα τα ερωτήματα αυτής· και δεν εστάθη ουδέν κεκρυμμένον από του Σολομώντος, το οποίον δεν εξήγησεν εις αυτήν.
\par 3 Και ιδούσα η βασίλισσα της Σεβά την σοφίαν του Σολομώντος και τον οίκον, τον οποίον ωκοδόμησε,
\par 4 και τα φαγητά της τραπέζης αυτού, και την καθεδρίασιν των δούλων αυτού, και την στάσιν των υπουργών αυτού και τον ιματισμόν αυτών, και τους οινοχόους αυτού και τον ιματισμόν αυτών, και την ανάβασιν αυτού δι' ης ανέβαινεν εις τον οίκον του Κυρίου, έγεινεν έκθαμβος·
\par 5 και είπε προς τον βασιλέα, Αληθής ο λόγος, τον οποίον ήκουσα εν τη γη μου, περί των έργων σου και περί της σοφίας σου·
\par 6 αλλά δεν επίστευον εις τους λόγους αυτών, εωσού ήλθον και είδον οι οφθαλμοί μου· και ιδού, το ήμισυ του πλήθους της σοφίας σου δεν απηγγέλθη προς εμέ· συ υπερβαίνεις την φήμην την οποίαν ήκουσα·
\par 7 μακάριοι οι άνδρες σου και μακάριοι οι δούλοί σου ούτοι, οι ιστάμενοι πάντοτε ενώπιόν σου και ακούοντες την σοφίαν σου·
\par 8 έστω Κύριος ο Θεός σου ευλογημένος, όστις ευηρεστήθη εις σε, διά να σε θέση επί του θρόνου αυτού, να ήσαι βασιλεύς εις Κύριον τον Θεόν σου· επειδή ο Θεός σου ηγάπησε τον Ισραήλ, ώστε να στερεώση αυτούς εις τον αιώνα, διά τούτο σε κατέστησε βασιλέα επ' αυτούς, διά να κάμνης κρίσιν και δικαιοσύνην.
\par 9 Και έδωκεν εις τον βασιλέα εκατόν είκοσι τάλαντα χρυσίου και αρώματα πολλά σφόδρα και λίθους πολυτίμους· και δεν εστάθησαν ποτέ τοιαύτα αρώματα, οποία η βασίλισσα της Σεβά έδωκεν εις τον βασιλέα Σολομώντα.
\par 10 Και οι δούλοι δε του Χουράμ και οι δούλοι του Σολομώντος, οίτινες έφερον χρυσίον από Οφείρ, έφερον και ξύλον αλγουμείμ και λίθους πολυτίμους.
\par 11 Και έκαμεν ο βασιλεύς εκ των ξύλων αλγουμείμ αναβάσεις εις τον οίκον του Κυρίου και εις τον οίκον του βασιλέως, και κιθάρας και ψαλτήρια διά τους μουσικούς· και τοιαύτα δεν εφάνησαν πρότερον εν τη γη Ιούδα.
\par 12 Και έδωκεν ο βασιλεύς Σολομών εις την βασίλισσαν της Σεβά πάντα όσα ηθέλησεν, όσα εζήτησε, πλειότερα των όσα έφερε προς τον βασιλέα. Και επέστρεψε και ανεχώρησεν εις την γην αυτής, αυτή και οι δούλοι αυτής.
\par 13 Το βάρος δε του χρυσίου, το οποίον ήρχετο εις τον Σολομώντα κατ' έτος, ήτο εξακόσια εξήκοντα εξ τάλαντα χρυσίου,
\par 14 εκτός του συναγομένου εκ των τελωνών και των εμπόρων και πάντων των βασιλέων της Αραβίας και των σατραπών της γης, οίτινες έφερον χρυσίον και αργύριον προς τον Σολομώντα.
\par 15 Και έκαμεν ο βασιλεύς Σολομών διακοσίους θυρεούς εκ χρυσίου σφυρηλάτου· εξακόσιοι σίκλοι χρυσίου σφυρηλάτου εξωδεύθησαν εις έκαστον θυρεόν·
\par 16 και τριακοσίας ασπίδας εκ χρυσίου σφυρηλάτου· τριακόσιοι σίκλοι χρυσίου εξωδεύθησαν εις εκάστην ασπίδα. Και έθεσεν αυτάς ο βασιλεύς εν τω οίκω του δάσους του Λιβάνου.
\par 17 Έκαμεν έτι ο βασιλεύς, θρόνον μέγαν ελεφάντινον και εσκέπασεν αυτόν με καθαρόν χρυσίον.
\par 18 είχε δε ο θρόνος εξ βαθμίδας και υποπόδιον χρυσούν, συνδεδεμένα με τον θρόνον, και αγκώνας εντεύθεν και εντεύθεν της καθέδρας, και δύο λέοντας ισταμένους εις τα πλάγια των αγκώνων·
\par 19 και δώδεκα λέοντες ίσταντο εκεί, εκατέρωθεν επί των εξ βαθμίδων. Παρόμοιον δεν κατεσκευάσθη εις ουδέν βασίλειον.
\par 20 Και πάντα τα σκεύη του ποτού του βασιλέως Σολομώντος ήσαν εκ χρυσίου, και πάντα τα σκεύη του οίκου του δάσους του Λιβάνου εκ χρυσίου καθαρού· ουδέν εξ αργυρίου· το αργύριον ελογίζετο εις ουδέν εν ταις ημέραις του Σολομώντος.
\par 21 Διότι είχε πλοία ο βασιλεύς πορευόμενα εις Θαρσείς μετά των δούλων του Χουράμ· άπαξ κατά τριετίαν ήρχοντο τα πλοία από Θαρσείς, φέροντα χρυσόν και άργυρον, οδόντας ελέφαντος και πιθήκους και παγώνια.
\par 22 Και εμεγαλύνθη ο βασιλεύς Σολομών υπέρ πάντας τους βασιλείς της γης εις πλούτον και εις σοφίαν.
\par 23 Και πάντες οι βασιλείς της γης εζήτουν το πρόσωπον του Σολομώντος, διά να ακούσωσι την σοφίαν αυτού, την οποίαν ο Θεός έθεσεν εις την καρδίαν αυτού.
\par 24 Και έφερον έκαστος αυτών το δώρον αυτού, σκεύη αργυρά και σκεύη χρυσά και στολάς, πανοπλίας και αρώματα, ίππους και ημιόνους, κατ' έτος.
\par 25 Και είχεν ο Σολομών τέσσαρας χιλιάδας σταύλους ίππων και αμαξών και δώδεκα χιλιάδας ιππέων, τους οποίους έθεσεν ο βασιλεύς εις τας πόλεις των αμαξών και πλησίον εαυτού εν Ιερουσαλήμ.
\par 26 Και εβασίλευεν επί πάντας τους βασιλείς από του ποταμού έως της γης των Φιλισταίων και των ορίων της Αιγύπτου.
\par 27 Και κατέστησεν ο βασιλεύς τον άργυρον εν Ιερουσαλήμ ως λίθους, και τας κέδρους κατέστησεν ως τας εν τη πεδιάδι συκαμίνους, διά την αφθονίαν.
\par 28 Και έφερον προς τον Σολομώντα ίππους εξ Αιγύπτου και εκ πάντων των τόπων.
\par 29 αι δε λοιπαί πράξεις του Σολομώντος, αι πρώται και αι έσχαται, δεν είναι γεγραμμέναι εν τω βιβλίω Νάθαν του προφήτου και εν τη προφητεία Αχιά του Σηλωνίτου και εν τοις οράμασιν Ιδδώ του βλέποντος, τοις γενομένοις εναντίον Ιεροβοάμ υιού του Ναβάτ;
\par 30 Εβασίλευσε δε ο Σολομών εν Ιερουσαλήμ, επί πάντα τον Ισραήλ, τεσσαράκοντα έτη.
\par 31 Και εκοιμήθη ο Σολομών μετά των πατέρων αυτού· και έθαψαν αυτόν εν τη πόλει Δαβίδ του πατρός αυτού· εβασίλευσε δε αντ' αυτού Ροβοάμ ο υιός αυτού.

\chapter{10}

\par 1 Και υπήγεν ο Ροβοάμ εις Συχέμ· διότι ήρχετο πας ο Ισραήλ εις Συχέμ διά να κάμη αυτόν βασιλέα.
\par 2 Και ως ήκουσε τούτο Ιεροβοάμ ο υιός του Ναβάτ, όστις ήτο εν Αιγύπτω, όπου είχε φύγει από προσώπου Σολομώντος του βασιλέως, επέστρεψεν ο Ιεροβοάμ εξ Αιγύπτου,
\par 3 διότι απέστειλαν και εκάλεσαν αυτόν. Τότε ήλθον ο Ιεροβοάμ και πας ο Ισραήλ, και ελάλησαν προς τον Ροβοάμ, λέγοντες,
\par 4 Ο πατήρ σου εσκλήρυνε τον ζυγόν ημών· τώρα λοιπόν την δουλείαν την σκληράν του πατρός σου και τον ζυγόν αυτού τον βαρύν, τον οποίον επέβαλεν εφ' ημάς, ελάφρωσον συ, και θέλομεν σοι δουλεύει.
\par 5 Ο δε είπε προς αυτούς, Επανέλθετε προς εμέ μετά τρεις ημέρας. Και ανεχώρησεν ο λαός.
\par 6 Και συνεβουλεύθη ο βασιλεύς Ροβοάμ τους πρεσβυτέρους, οίτινες παρίσταντο ενώπιον Σολομώντος του πατρός αυτού έτι ζώντος, λέγων, Τι με συμβουλεύετε σεις να αποκριθώ προς τον λαόν τούτον;
\par 7 Και ελάλησαν προς αυτόν, λέγοντες, Εάν φερθής ευμενώς προς τον λαόν τούτον και ευαρεστήσης εις αυτούς, και λαλήσης προς αυτούς αγαθούς λόγους, τότε θέλουσιν είσθαι δούλοί σου διά παντός.
\par 8 Απέρριψεν όμως την συμβουλήν των πρεσβυτέρων, την οποίαν έδωκαν εις αυτόν, και συνεβουλεύθη τους νέους τους συνανατραφέντας μετ' αυτού, τους παρισταμένους ενώπιον αυτού.
\par 9 Και είπε προς αυτούς, Τι με συμβουλεύετε σεις να αποκριθώμεν προς τον λαόν τούτον, όστις ελάλησε προς εμέ, λέγων, Ελάφρωσον τον ζυγόν τον οποίον ο πατήρ σου επέβαλεν εφ' ημάς;
\par 10 Και ελάλησαν προς αυτόν οι νέοι οι συνανατραφέντες μετ' αυτού, λέγοντες, Ούτω θέλεις λαλήσει προς τον λαόν, όστις ελάλησε προς σε, λέγων, Ο πατήρ σου εβάρυνε τον ζυγόν ημών, αλλά συ ελάφρωσον αυτόν εις ημάς· ούτω θέλεις λαλήσει προς αυτούς· Ο μικρός μου δάκτυλος θέλει είσθαι παχύτερος της οσφύος του πατρός μου·
\par 11 τώρα λοιπόν ο μεν πατήρ μου επεφόρτισεν εις εσάς ζυγόν βαρύν, εγώ δε θέλω κάμει βαρύτερον τον ζυγόν σας· ο πατήρ μου σας επαίδευσε με μάστιγας, εγώ δε θέλω σας παιδεύσει με σκορπίους.
\par 12 Και ήλθεν ο Ιεροβοάμ και πας ο λαός προς τον Ροβοάμ την τρίτην ημέραν, ως είχε λαλήσει ο βασιλεύς, λέγων, Επανέλθετε προς εμέ την τρίτην ημέραν.
\par 13 Και απεκρίθη ο βασιλεύς προς αυτούς σκληρώς· και εγκατέλιπεν ο βασιλεύς Ροβοάμ την συμβουλήν των πρεσβυτέρων,
\par 14 και ελάλησε προς αυτούς κατά την συμβουλήν των νέων, λέγων, Ο πατήρ μου εβάρυνε τον ζυγόν σας, αλλ' εγώ θέλω κάμει αυτόν βαρύτερον· ο πατήρ μου σας επαίδευσε με μάστιγας, αλλ' εγώ θέλω σας παιδεύσει με σκορπίους.
\par 15 Και δεν εισήκουσεν ο βασιλεύς εις τον λαόν· διότι το πράγμα έγεινε παρά του Θεού, διά να εκτελέση ο Κύριος τον λόγον αυτού, τον οποίον ελάλησε διά του Αχιά του Σηλωνίτου προς Ιεροβοάμ τον υιόν του Ναβάτ.
\par 16 Και ιδών πας ο Ισραήλ ότι ο βασιλεύς δεν εισήκουσεν εις αυτούς, απεκρίθη ο λαός προς τον βασιλέα, λέγων, Τι μέρος έχομεν ημείς εις τον Δαβίδ; ουδεμίαν κληρονομίαν έχομεν εις τον υιόν του Ιεσσαί· εις τας σκηνάς σου έκαστος, Ισραήλ· πρόβλεψον τώρα, Δαβίδ, περί του οίκου σου. Και ανεχώρησε πας ο Ισραήλ εις τας σκηνάς αυτού.
\par 17 Περί δε των υιών Ισραήλ των κατοικούντων εν ταις πόλεσιν Ιούδα, ο Ροβοάμ εβασίλευσεν επ' αυτούς.
\par 18 Και απέστειλεν ο βασιλεύς Ροβοάμ τον Αδωράμ, τον επί των φόρων· και ελιθοβόλησαν αυτόν οι υιοί Ισραήλ με λίθους, και απέθανεν. Όθεν έσπευσεν ο βασιλεύς Ροβοάμ να αναβή εις την άμαξαν, διά να φύγη εις Ιερουσαλήμ.
\par 19 Ούτως απεστάτησεν ο Ισραήλ από του οίκου του Δαβίδ, έως της ημέρας ταύτης.

\chapter{11}

\par 1 Και ελθών ο Ροβοάμ εις Ιερουσαλήμ, συνήθροισε τον οίκον Ιούδα και Βενιαμίν, εκατόν ογδοήκοντα χιλιάδας εκλεκτών πολεμιστών, διά να πολεμήσωσι κατά του Ισραήλ, όπως επαναφέρωσι την βασιλείαν εις τον Ροβοάμ.
\par 2 Έγεινεν όμως λόγος Κυρίου προς τον Σεμαΐαν, άνθρωπον του Θεού, λέγων,
\par 3 Λάλησον προς Ροβοάμ τον υιόν του Σολομώντος, τον βασιλέα του Ιούδα, και προς πάντα τον Ισραήλ εν Ιούδα και Βενιαμίν, λέγων,
\par 4 Ούτω λέγει Κύριος· Δεν θέλετε αναβή ουδέ πολεμήσει εναντίον των αδελφών σας· επιστρέψατε έκαστος εις τον οίκον αυτού, διότι παρ' εμού έγεινε το πράγμα τούτο. Και υπήκουσαν εις τους λόγους του Κυρίου και απεστράφησαν από του να υπάγωσι κατά του Ιεροβοάμ.
\par 5 Και κατώκησεν ο Ροβοάμ εν Ιερουσαλήμ και ωκοδόμησε πόλεις οχυράς εν Ιούδα.
\par 6 Και ωκοδόμησε την Βηθλεέμ και την Ητάμ και την Θεκουέ
\par 7 και την Βαιθ-σούρ και την Σοκχώ και την Οδολλάμ
\par 8 και την Γαθ και την Μαρησά και την Ζιφ
\par 9 και την Αδωραΐμ και την Λαχείς και την Αζηκά
\par 10 και την Σαραά και την Αιαλών και την Χεβρών, αίτινες είναι εν Ιούδα και εν Βενιαμίν, πόλεις ωχυρωμέναι.
\par 11 Και ωχύρωσε τα φρούρια, και έβαλεν εις αυτά φρουράρχους και αποθήκας τροφών και ελαίου και οίνου.
\par 12 Και εις πάσαν πόλιν έβαλεν ασπίδας και λόγχας, και ωχύρωσεν αυτάς πολύ σφόδρα. Και ήσαν υπ' αυτόν ο Ιούδας και ο Βενιαμίν.
\par 13 Και οι ιερείς και οι Λευΐται οι εν παντί τω Ισραήλ συνήχθησαν προς αυτόν, εκ πάντων των ορίων αυτών.
\par 14 Διότι οι Λευΐται εγκατέλιπον τα προάστεια αυτών και τας ιδιοκτησίας αυτών, και ήλθον εις τον Ιούδαν και εις την Ιερουσαλήμ· επειδή ο Ιεροβοάμ και οι υιοί αυτού είχον αποβάλει αυτούς από του να ιερατεύωσιν εις τον Κύριον·
\par 15 και κατέστησεν εις εαυτόν ιερείς διά τους υψηλούς τόπους και διά τους δαίμονας και διά τους μόσχους, τους οποίους έκαμε·
\par 16 και μετ' αυτούς, όσοι εκ πασών των φυλών του Ισραήλ έδωκαν τας καρδίας αυτών εις το να ζητώσι Κύριον τον Θεόν του Ισραήλ, ήλθον εις Ιερουσαλήμ, διά να θυσιάσωσιν εις Κύριον τον Θεόν των πατέρων αυτών.
\par 17 Και κατίσχυσαν την βασιλείαν του Ιούδα και ισχυροποίησαν τον Ροβοάμ τον υιόν του Σολομώντος, τρία έτη· διότι τρία έτη περιεπάτησαν εν τη οδώ του Δαβίδ και του Σολομώντος.
\par 18 Έλαβε δε ο Ροβοάμ εις εαυτόν γυναίκα την Μαελέθ, θυγατέρα του Ιεριμώθ υιού του Δαβίδ, και την Αβιχαίλ, θυγατέρα του Ελιάβ υιού του Ιεσσαί·
\par 19 ήτις εγέννησεν εις αυτόν υιούς, τον Ιεούς και τον Σαμαρίαν και τον Ζαάμ.
\par 20 Και μετ' αυτήν έλαβε την Μααχά θυγατέρα του Αβεσσαλώμ, ήτις εγέννησεν εις αυτόν τον Αβιά και τον Ατθαΐ και τον Ζιζά και τον Σελωμείθ.
\par 21 Και ηγάπησεν ο Ροβοάμ την Μααχά θυγατέρα του Αβεσσαλώμ υπέρ πάσας τας γυναίκας αυτού και τας παλλακάς αυτού· διότι έλαβε δεκαοκτώ γυναίκας και εξήκοντα παλλακάς· και εγέννησεν εικοσιοκτώ υιούς και εξήκοντα θυγατέρας·
\par 22 και κατέστησεν ο Ροβοάμ άρχοντα τον Αβιά τον υιόν της Μααχά, διά να άρχη επί τους αδελφούς αυτού· διότι εστοχάζετο να κάμη αυτόν βασιλέα·
\par 23 και ποιών φρονίμως διέσπειρε πάντας τους υιούς αυτού εις πάντας τους τόπους Ιούδα και Βενιαμίν, εις πάσαν οχυράν πόλιν· και έδωκεν εις αυτούς τροφάς εν αφθονία και εζήτησε πολλάς γυναίκας.

\chapter{12}

\par 1 Και καθώς εστερεώθη η βασιλεία του Ροβοάμ και ενεδυναμώθη, εγκατέλιπε τον νόμον του Κυρίου, και πας ο Ισραήλ μετ' αυτού.
\par 2 Και εν τω πέμπτω έτει της βασιλείας του Ροβοάμ, Σισάκ ο βασιλεύς της Αιγύπτου ανέβη εναντίον της Ιερουσαλήμ, επειδή παρηνόμησαν εις τον Κύριον,
\par 3 μετά χιλίων διακοσίων αμαξών και εξήκοντα χιλιάδων ιππέων· ο δε λαός όστις ήλθε μετ' αυτού εξ Αιγύπτου ήτο αναρίθμητος, Λίβυες, Τρωγλοδύται και Αιθίοπες.
\par 4 Και κυριεύσας τας οχυράς πόλεις τας εν Ιούδα, ήλθεν έως της Ιερουσαλήμ.
\par 5 Τότε Σεμαΐας ο προφήτης ήλθε προς τον Ροβοάμ και τους άρχοντας του Ιούδα, τους συναχθέντας εν Ιερουσαλήμ διά τον φόβον του Σισάκ, και είπε προς αυτούς, Ούτω λέγει Κύριος· Σεις με εγκατελίπετε· διά τούτο σας εγκατέλιπον και εγώ εις την χείρα του Σισάκ.
\par 6 Και εταπεινώθησαν οι άρχοντες του Ισραήλ και ο βασιλεύς, και έλεγον, Δίκαιος ο Κύριος.
\par 7 Και ότε είδεν ο Κύριος ότι εταπεινώθησαν, έγεινε λόγος Κυρίου προς τον Σεμαΐαν, λέγων, Ούτοι εταπεινώθησαν· δεν θέλω εξολοθρεύσει αυτούς, αλλά θέλω χαρίσει εις αυτούς σωτηρίαν τινά· και ο θυμός μου δεν θέλει εκχυθή επί την Ιερουσαλήμ διά χειρός του Σισάκ·
\par 8 αλλ' όμως θέλουσι γείνει δούλοι αυτού, διά να γνωρίσωσι την δουλείαν την εμήν και την δουλείαν των βασιλειών της γης.
\par 9 Και ανέβη Σισάκ ο βασιλεύς της Αιγύπτου επί την Ιερουσαλήμ, και έλαβε τους θησαυρούς του οίκου του Κυρίου και τους θησαυρούς του οίκου του βασιλέως· τα πάντα έλαβεν· έλαβεν έτι τους θυρεούς τους χρυσούς, τους οποίους έκαμεν ο Σολομών.
\par 10 Και αντ' εκείνων έκαμεν ο βασιλεύς Ροβοάμ θυρεούς χαλκίνους, και παρέδωκεν αυτούς εις τας χείρας των αρχόντων των σωματοφυλάκων, οίτινες εφύλαττον την είσοδον του οίκου του βασιλέως.
\par 11 Και οπότε εισήρχετο ο βασιλεύς εις τον οίκον του Κυρίου, οι σωματοφύλακες ήρχοντο και ελάμβανον αυτούς, και πάλιν έφερον αυτούς εις το οίκημα των σωματοφυλάκων.
\par 12 Επειδή λοιπόν εταπεινώθη, απεστράφη απ' αυτού ο θυμός του Κυρίου, διά να μη αφανίση αυτούς ολοκλήρως· διότι ήσαν έτι αγαθά πράγματα εν τω Ιούδα.
\par 13 Και ενεδυναμώθη ο βασιλεύς Ροβοάμ εν Ιερουσαλήμ και εβασίλευσε· διότι ο Ροβοάμ ήτο ηλικίας τεσσαράκοντα και ενός έτους ότε εβασίλευσε, και εβασίλευσε δεκαεπτά έτη εν Ιερουσαλήμ, τη πόλει την οποίαν ο Κύριος εξέλεξεν εκ πασών των φυλών του Ισραήλ, διά να θέση το όνομα αυτού εκεί. Της δε μητρός αυτού το όνομα ήτο Νααμά η Αμμωνίτις.
\par 14 Και έπραξε πονηρά, επειδή δεν προσήλωσε την καρδίαν αυτού εις το να εκζητή τον Κύριον.
\par 15 Αι δε πράξεις του Ροβοάμ, αι πρώται και αι έσχαται, δεν είναι γεγραμμέναι εν τω βιβλίω Σεμαΐου του προφήτου και Ιδδώ του βλέποντος, εν ταις γενεαλογίαις; Ήσαν δε πάντοτε πόλεμοι μεταξύ Ροβοάμ και Ιεροβοάμ.
\par 16 Και εκοιμήθη ο Ροβοάμ μετά των πατέρων αυτού και ετάφη εν πόλει Δαβίδ· εβασίλευσε δε αντ' αυτού Αβιά ο υιός αυτού.

\chapter{13}

\par 1 Και εβασίλευσεν ο Αβιά επί τον Ιούδαν εν τω δεκάτω ογδόω έτει του βασιλέως Ιεροβοάμ.
\par 2 Τρία έτη εβασίλευσεν εν Ιερουσαλήμ. Το δε όνομα της μητρός αυτού ήτο Μιχαΐα, θυγάτηρ του Ουριήλ από Γαβαά. Και ήτο πόλεμος μεταξύ Αβιά και Ιεροβοάμ.
\par 3 Και παρετάχθη ο Αβιά εις μάχην με στράτευμα δυνατών πολεμιστών, τετρακοσίων χιλιάδων ανδρών εκλεκτών· και ο Ιεροβοάμ παρετάχθη εις μάχην εναντίον αυτού με οκτακοσίας χιλιάδας ανδρών εκλεκτών, δυνατών εν ισχύϊ.
\par 4 Και σηκωθείς ο Αβιά επί το όρος Σεμαραΐμ, το εν τω όρει Εφραΐμ, είπεν, Ακούσατέ μου, Ιεροβοάμ και πας ο Ισραήλ·
\par 5 δεν πρέπει να γνωρίσητε, ότι Κύριος ο Θεός του Ισραήλ έδωκε την βασιλείαν επί τον Ισραήλ διαπαντός εις τον Δαβίδ, εις αυτόν και εις τους υιούς αυτού, με συνθήκην άλατος;
\par 6 αλλ' ο Ιεροβοάμ ο υιός του Ναβάτ, ο δούλος του Σολομώντος υιού του Δαβίδ, εσηκώθη και επανεστάτησεν εναντίον του κυρίου αυτού·
\par 7 και συνήχθησαν προς αυτόν άνθρωποι μηδαμινοί, αχρείοι, και ενεδυναμώθησαν εναντίον του Ροβοάμ υιού του Σολομώντος, ότε ήτο ο Ροβοάμ νέος και απαλός την καρδίαν και δεν ηδύνατο να αντισταθή εις αυτούς·
\par 8 και τώρα σεις λέγετε να αντισταθήτε εις την βασιλείαν του Κυρίου, την εις τας χείρας των υιών του Δαβίδ, διότι είσθε πλήθος πολύ και έχετε μεθ' εαυτών χρυσούς μόσχους, τους οποίους ο Ιεροβοάμ έκαμεν εις εσάς διά θεούς·
\par 9 δεν απεβάλετε τους ιερείς του Κυρίου, τους υιούς του Ααρών, και τους Λευΐτας, και εκάμετε εις εαυτούς ιερείς κατά τα έθνη της γης; πας όστις προσέρχεται να ιερωθή με μόσχον βοός και επτά κριούς, γίνεται ιερεύς εις τους μη θεούς·
\par 10 αλλ' ημείς τον Κύριον έχομεν θεόν ημών, και δεν εγκατελίπομεν αυτόν· και οι ιερείς, οι λειτουργούντες εις τον Κύριον, είναι οι υιοί του Ααρών· και οι Λευΐται, επί την εργασίαν·
\par 11 και καίουσι προς τον Κύριον καθ' εκάστην πρωΐαν και καθ' εκάστην εσπέραν ολοκαυτώματα και θυμίαμα ευώδες· και διατέθουσι τους άρτους της προθέσεως επί της τραπέζης της καθαράς και την λυχνίαν την χρυσήν και τους λύχνους αυτής, διά να καίη πάσαν εσπέραν· διότι ημείς φυλάττομεν την φυλακήν Κυρίου του Θεού ημών· σεις όμως εγκατελίπετε αυτόν·
\par 12 και ιδού, ο Θεός αυτός είναι μεθ' ημών επί κεφαλής, και οι ιερείς αυτού με ηχητικάς σάλπιγγας, διά να ηχώσιν εναντίον σας. Υιοί Ισραήλ, μη πολεμείτε εναντίον Κυρίου του Θεού των πατέρων σας· διότι δεν θέλετε ευοδωθή.
\par 13 Ο δε Ιεροβοάμ έστρεψε την ενέδραν διά να περιέλθη εκ των όπισθεν αυτών· και ήσαν κατά πρόσωπον του Ιούδα, και η ενέδρα όπισθεν αυτών.
\par 14 Και ότε περιέβλεψεν ο Ιούδας, ιδού, η μάχη ήτο έμπροσθεν και όπισθεν αυτών· και εβόησαν προς τον Κύριον, και οι ιερείς εσάλπισαν με τας σάλπιγγας.
\par 15 Τότε οι άνδρες Ιούδα ηλάλαξαν· και καθώς ηλάλαξαν οι άνδρες Ιούδα, ο Θεός επάταξε τον Ιεροβοάμ και πάντα τον Ισραήλ, έμπροσθεν του Αβιά και του Ιούδα.
\par 16 Και έφυγον οι υιοί Ισραήλ απ' έμπροσθεν του Ιούδα· και παρέδωκεν αυτούς ο Θεός εις την χείρα αυτών.
\par 17 Και έκαμον ο Αβιά και ο λαός αυτού εις αυτούς σφαγήν μεγάλην· και έπεσαν τραυματίαι εκ του Ισραήλ πεντακόσιαι χιλιάδες ανδρών εκλεκτών.
\par 18 Και εταπεινώθησαν οι υιοί Ισραήλ εν τω καιρώ εκείνω, οι δε υιοί Ιούδα υπερίσχυσαν, επειδή ήλπισαν επί Κύριον τον Θεόν των πατέρων αυτών.
\par 19 Και κατεδίωξεν ο Αβιά εξ οπίσω του Ιεροβοάμ, και έλαβε παρ' αυτού πόλεις, την Βαιθήλ και τας κώμας αυτής, και την Ιεσανά και τας κώμας αυτής, και την Εφραΐν και τας κώμας αυτής.
\par 20 Και δεν ανέλαβε πλέον δύναμιν ο Ιεροβοάμ εν ταις ημέραις του Αβιά· αλλ' επάταξεν αυτόν ο Κύριος, και απέθανε.
\par 21 Και ενεδυναμώθη ο Αβιά· και έλαβεν εις εαυτόν δεκατέσσαρας γυναίκας, και εγέννησεν εικοσιδύο υιούς και δεκαέξ θυγατέρας.
\par 22 Αι δε λοιπαί πράξεις του Αβιά και αι οδοί αυτού και οι λόγοι αυτού είναι γεγραμμένοι εν τη ιστορία του προφήτου Ιδδώ.

\chapter{14}

\par 1 Και εκοιμήθη ο Αβιά μετά των πατέρων αυτού, και έθαψαν αυτόν εν πόλει Δαβίδ· εβασίλευσε δε αντ' αυτού Ασά ο υιός αυτού. Εν ταις ημέραις αυτού η γη ησύχασε δέκα έτη.
\par 2 Και έκαμεν ο Ασά το καλόν και το ευθές ενώπιον Κυρίου του Θεού αυτού·
\par 3 διότι αφήρεσε τα θυσιαστήρια των αλλοτρίων θεών και τους υψηλούς τόπους, και κατεσύντριψε τα αγάλματα και κατέκοψε τα άλση·
\par 4 και είπε προς τον Ιούδαν να εκζητώσι Κύριον τον Θεόν των πατέρων αυτών και να κάμνωσι τον νόμον και τας εντολάς.
\par 5 Αφήρεσεν έτι από πασών των πόλεων του Ιούδα τους υψηλούς τόπους και τα είδωλα· και ησύχασε το βασίλειον ενώπιον αυτού.
\par 6 Και ωκοδόμησε πόλεις οχυράς εν τω Ιούδα· διότι ησύχασεν η γη, και δεν ήτο εις αυτόν πόλεμος εν εκείνοις τοις χρόνοις, επειδή ο Κύριος έδωκεν εις αυτόν ανάπαυσιν.
\par 7 Διά τούτο είπε προς τον Ιούδαν, Ας οικοδομήσωμεν τας πόλεις ταύτας, και ας κάμωμεν περί αυτάς τείχη και πύργους, πύλας και μοχλούς, ενώ είμεθα κύριοι της γης, επειδή εξεζητήσαμεν Κύριον τον Θεόν ημών· εξεζητήσαμεν αυτόν, και έδωκεν εις ημάς ανάπαυσιν κυκλόθεν. Και ωκοδόμησαν και ευωδώθησαν.
\par 8 Είχε δε ο Ασά στράτευμα εκ του Ιούδα τριακοσίας χιλιάδας, φέροντας θυρεούς και λόγχας· εκ δε του Βενιαμίν, διακοσίας ογδοήκοντα χιλιάδας, ασπιδοφόρους και τοξότας· πάντες ούτοι ήσαν δυνατοί εν ισχύϊ.
\par 9 Εξήλθε δε εναντίον αυτών Ζερά ο Αιθίοψ, με στράτευμα εκατόν μυριάδων και με τριακοσίας αμάξας, και ήλθεν έως Μαρησά.
\par 10 Και εξήλθεν ο Ασά εναντίον αυτού, και παρετάχθησαν εις μάχην εν τη φάραγγι Σεφαθά, πλησίον της Μαρησά.
\par 11 Και εβόησεν ο Ασά προς Κύριον τον Θεόν αυτού και είπε, Κύριε, δεν είναι ουδέν παρά σοι να βοηθής τους έχοντας πολλήν ή μηδεμίαν δύναμιν· βοήθησον ημάς, Κύριε Θεέ ημών· διότι επί σε πεποίθαμεν, και εν τω ονόματί σου ερχόμεθα εναντίον του πλήθους τούτου. Κύριε, συ είσαι ο Θεός ημών· ας μη υπερισχύση άνθρωπος εναντίον σου.
\par 12 Και επάταξεν ο Κύριος τους Αιθίοπας έμπροσθεν του Ασά και έμπροσθεν του Ιούδα· και οι Αιθίοπες έφυγον.
\par 13 Ο δε Ασά και ο λαός ο μετ' αυτού κατεδίωξαν αυτούς έως Γεράρων· και έπεσον εκ των Αιθιόπων τοσούτοι, ώστε δεν ηδύναντο να αναλάβωσι πλέον· διότι συνετρίβησαν έμπροσθεν του Κυρίου και έμπροσθεν του στρατεύματος αυτού· και έλαβον λάφυρα πολλά σφόδρα.
\par 14 Και επάταξαν πάσας τας πόλεις κύκλω των Γεράρων· διότι ο φόβος του Κυρίου επέπεσεν επ' αυτούς· και ελαφυραγώγησαν πάσας τας πόλεις· διότι ήσαν εν αυταίς λάφυρα πολλά.
\par 15 Επάταξαν δε και τας επαύλεις των ποιμνίων και έλαβον πρόβατα πολλά και καμήλους, και επέστρεψαν εις Ιερουσαλήμ.

\chapter{15}

\par 1 Τότε ήλθε το πνεύμα του Θεού επί Αζαρίαν τον υιόν του Ωδήδ·
\par 2 και εξήλθεν εις συνάντησιν του Ασά και είπε προς αυτόν, Ακούσατέ μου, Ασά και πας ο Ιούδας και ο Βενιαμίν· Ο Κύριος είναι με σας, όταν σεις είσθε μετ' αυτού· και εάν εκζητήτε αυτόν, θέλει ευρεθή εις εσάς· εάν όμως εγκαταλείψητε αυτόν, θέλει σας εγκαταλείψει·
\par 3 πολύν μεν καιρόν εστάθη ο Ισραήλ χωρίς του αληθινού Θεού και χωρίς ιερέως διδάσκοντος και χωρίς νόμου·
\par 4 ότε όμως εν τη στενοχωρία αυτών επέστρεψαν εις Κύριον τον Θεόν του Ισραήλ και εξεζήτησαν αυτόν, ευρέθη εις αυτούς·
\par 5 και κατ' εκείνους τους καιρούς δεν ήτο ειρήνη εις τον εξερχόμενον και εις τον εισερχόμενον, αλλ' ήσαν μεγάλαι ταραχαί επί πάντας τους κατοίκους των τόπων·
\par 6 και εφθείρετο έθνος υπό έθνους και πόλις υπό πόλεως· διότι ο Θεός κατέθλιβεν αυτούς εν πάση στενοχωρία·
\par 7 σεις δε ενδυναμούσθε, και ας μη ήναι εκλελυμέναι αι χείρές σας· διότι θέλει είσθαι μισθός εις το έργον σας.
\par 8 Και ότε ήκουσεν ο Ασά τους λόγους τούτους και την προφητείαν Ωδήδ του προφήτου, ενεδυναμώθη και απέβαλε τα βδελύγματα εκ πάσης της γης Ιούδα και Βενιαμίν και εκ των πόλεων, τας οποίας έλαβεν εκ του όρους Εφραΐμ, και ανενέωσε το θυσιαστήριον του Κυρίου, το κατ' έμπροσθεν του προνάου του Κυρίου.
\par 9 Και συνήγαγε πάντα τον Ιούδαν και τον Βενιαμίν, και τους παροικούντας μετ' αυτών εκ του Εφραΐμ και Μανασσή και εκ του Συμεών· διότι πολλοί εκ του Ισραήλ προσεχώρησαν εις αυτόν, ιδόντες ότι Κύριος ο Θεός αυτού ήτο μετ' αυτού.
\par 10 Και συνήχθησαν εις Ιερουσαλήμ κατά τον τρίτον μήνα του δεκάτου πέμπτου έτους της βασιλείας του Ασά.
\par 11 Και προσέφεραν θυσίας εις τον Κύριον, κατά την ημέραν εκείνην, εκ των λαφύρων τα οποία έφεραν, επτακοσίους βόας και επτά χιλιάδας προβάτων.
\par 12 Και εισήλθον εις συνθήκην να εκζητήσωσι Κύριον τον Θεόν των πατέρων αυτών, εξ όλης της καρδίας αυτών και εξ όλης της ψυχής αυτών·
\par 13 και πας όστις δεν εκζητήση Κύριον τον Θεόν του Ισραήλ να θανατόνηται, από μικρού έως μεγάλου, από ανδρός έως γυναικός.
\par 14 Και ώμοσαν προς τον Κύριον εν φωνή μεγάλη και εν αλαλαγμώ και εν σάλπιγξι και εν κερατίναις.
\par 15 Και πας ο Ιούδας ευφράνθη εις τον όρκον· διότι ώμοσαν εξ όλης της καρδίας αυτών και εξεζήτησαν αυτόν μεθ' όλης της θελήσεως αυτών· και ευρέθη εις αυτούς· και έδωκεν εις αυτούς ο Κύριος ανάπαυσιν κυκλόθεν.
\par 16 Έτι δε Μααχά, την μητέρα του βασιλέως Ασά, απέβαλεν αυτήν του να ήναι βασίλισσα, επειδή έκαμεν είδωλον εις άλσος· και κατέκοψεν ο Ασά το είδωλον αυτής και συνέτριψε και έκαυσεν αυτό εις τον χείμαρρον Κέδρων.
\par 17 Οι υψηλοί όμως τόποι δεν αφηρέθησαν από του Ισραήλ· πλην η καρδία του Ασά ήτο τελεία πάσας τας ημέρας αυτού.
\par 18 Και έφερεν εις τον οίκον του Θεού τα αφιερώματα του πατρός αυτού και τα εαυτού αφιερώματα, άργυρον και χρυσόν και σκεύη.
\par 19 Και δεν έγεινε πόλεμος έως του τριακοστού πέμπτου έτους της βασιλείας του Ασά.

\chapter{16}

\par 1 Εν τω τριακοστώ έκτω έτει της βασιλείας του Ασά, ο Βαασά βασιλεύς του Ισραήλ ανέβη εναντίον του Ιούδα και ωκοδόμησε την Ραμά, διά να μη αφίνη μηδένα να εξέρχηται μηδέ να εισέρχηται προς τον Ασά βασιλέα του Ιούδα.
\par 2 Τότε ο Ασά εξέφερεν αργύριον και χρυσίον εκ των θησαυρών του οίκου του Κυρίου και του οίκου του βασιλέως, και απέστειλε προς τον Βεν-αδάδ βασιλέα της Συρίας, κατοικούντα εν Δαμασκώ, λέγων,
\par 3 Ας γείνη συνθήκη αναμέσον εμού και σου, ως ήτο και αναμέσον του πατρός μου και του πατρός σου· ιδού, απέστειλα προς σε αργύριον και χρυσίον· ύπαγε, διάλυσον την συνθήκην σου την προς Βαασά βασιλέα του Ισραήλ, διά να αναχωρήση απ' εμού.
\par 4 Και εισήκουσεν ο Βεν-αδάδ εις τον βασιλέα Ασά, και απέστειλε τους αρχηγούς των δυνάμεων αυτού εναντίον των πόλεων του Ισραήλ· και επάταξαν την Ιϊών και την Δαν και την Αβέλ-μαΐμ και πάσας τας αποθήκας των πόλεων Νεφθαλί.
\par 5 Και ως ήκουσεν ο Βαασά, έπαυσε να οικοδομή την Ραμά και κατέλιπε το έργον αυτού.
\par 6 Και παρέλαβεν ο βασιλεύς Ασά πάντα τον Ιούδαν, και εσήκωσαν τους λίθους της Ραμά και τα ξύλα αυτής, με τα οποία ωκοδόμει ο Βαασά· και με ταύτα ωκοδόμησε την Γαβαά και την Μισπά.
\par 7 Κατ' εκείνον δε τον καιρόν Ανανί ο βλέπων ήλθε προς Ασά τον βασιλέα του Ιούδα και είπε προς αυτόν, Επειδή επεστηρίχθης επί τον βασιλέα της Συρίας και δεν επεστηρίχθης επί Κύριον τον Θεόν σου, διά τούτο εξέφυγε το στράτευμα του βασιλέως της Συρίας από της χειρός σου·
\par 8 οι Αιθίοπες και οι Λίβυες δεν ήσαν στράτευμα μέγα, μετά πολυαρίθμων αμαξών και ιππέων; επειδή όμως επεστηρίχθης εις τον Κύριον, παρέδωκεν αυτούς εις την χείρα σου·
\par 9 διότι οι οφθαλμοί του Κυρίου περιτρέχουσι διά πάσης της γης, διά να δειχθή δυνατός υπέρ των εχόντων την καρδίαν αυτών τελείαν προς αυτόν· εις τούτο έπραξας αφρόνως· διά τούτο θέλεις έχει πολέμους εις το εξής.
\par 10 Και ωργίσθη ο Ασά κατά του βλέποντος και έβαλεν αυτόν εις φυλακήν· διότι ηγανάκτησεν εναντίον αυτού διά τούτο. Και κατέθλιψεν ο Ασά τινάς εκ του λαού εν εκείνω τω καιρώ.
\par 11 Και ιδού, αι πράξεις του Ασά, αι πρώται και αι έσχαται, ιδού, είναι γεγραμμέναι εν τω βιβλίω των βασιλέων του Ιούδα και του Ισραήλ.
\par 12 Ηρρώστησε δε ο Ασά τους πόδας αυτού εν τω τριακοστώ εννάτω έτει της βασιλείας αυτού, εωσού η αρρωστία αυτού έγεινε μεγίστη· αλλ' ουδέ εν τη αρρωστία αυτού εξεζήτησε τον Κύριον, αλλά τους ιατρούς.
\par 13 Και εκοιμήθη ο Ασά μετά των πατέρων αυτού· και απέθανεν εν τω τεσσαρακοστώ πρώτω έτει της βασιλείας αυτού.
\par 14 Και έθαψαν αυτόν εις τον τάφον αυτού, τον οποίον έσκαψε δι' εαυτόν εν πόλει Δαβίδ, και έθεσαν αυτόν επί κλίνης πλήρους ευωδίας και διαφόρων αρωμάτων μυρεψικών· και έκαμον εις αυτόν καύσιν μεγάλην σφόδρα.

\chapter{17}

\par 1 Εβασίλευσε δε αντ' αυτού Ιωσαφάτ ο υιός αυτού και ενεδυναμώθη κατά του Ισραήλ.
\par 2 Και έβαλε δυνάμεις εις πάσας τας οχυράς πόλεις του Ιούδα και κατέστησε φρουράς εν τη γη Ιούδα και εν ταις πόλεσι του Εφραΐμ, τας οποίας είχε κυριεύσει Ασά ο πατήρ αυτού.
\par 3 Και ήτο Κύριος μετά του Ιωσαφάτ, επειδή περιεπάτησεν εν ταις οδοίς Δαβίδ του πατρός αυτού ταις πρώταις, και δεν εξεζήτησε τους Βααλείμ·
\par 4 αλλά τον Θεόν του πατρός αυτού εξεζήτησε και εις τας εντολάς αυτού περιεπάτησε και ουχί κατά τα έργα του Ισραήλ.
\par 5 Διά τούτο εστερέωσεν ο Κύριος την βασιλείαν εν τη χειρί αυτού· και πας ο Ιούδας έδωκε δώρα εις τον Ιωσαφάτ· και απέκτησε πλούτον και δόξαν πολλήν.
\par 6 Και υψώθη η καρδία αυτού εις τας οδούς του Κυρίου· και έτι αφήρεσε τους υψηλούς τόπους και τα άλση από του Ιούδα.
\par 7 Και εν τω τρίτω έτει της βασιλείας αυτού απέστειλε τους άρχοντας αυτού, τον Βεν-αΐλ και τον Οβαδίαν και τον Ζαχαρίαν και τον Ναθαναήλ και τον Μιχαΐαν, διά να διδάσκωσιν εν ταις πόλεσι του Ιούδα.
\par 8 και μετ' αυτών τους Λευΐτας, τον Σεμαΐαν και Ναθανίαν και Ζεβαδίαν και Ασαήλ και Σεμιραμώθ και Ιωνάθαν και Αδωνίαν και Τωβίαν και Τωβ-αδωνίαν, τους Λευΐτας· και μετ' αυτών Ελισαμά και Ιωράμ, τους ιερείς·
\par 9 και εδίδασκον εν τω Ιούδα, έχοντες μεθ' εαυτών το βιβλίον του νόμου του Κυρίου, και περιήρχοντο εις πάσας τας πόλεις του Ιούδα και εδίδασκον τον λαόν.
\par 10 Και επέπεσε φόβος Κυρίου επί πάσας τας βασιλείας των πέριξ του Ιούδα τόπων· και δεν επολέμουν εναντίον του Ιωσαφάτ.
\par 11 Και από των Φιλισταίων έφερον δώρα προς τον Ιωσαφάτ και φόρον αργυρίου· οι Άραβες προσέτι έφερον προς αυτόν ποίμνια κριών επτά χιλιάδας επτακοσίους και τράγων επτά χιλιάδας επτακοσίους.
\par 12 Και προεχώρει ο Ιωσαφάτ μεγαλυνόμενος σφόδρα· και ωκοδόμησεν εν Ιούδα φρούρια και πόλεις αποθηκών.
\par 13 Και είχε πολλά έργα εν ταις πόλεσιν Ιούδα· και άνδρας πολεμιστάς, δυνατούς εν ισχύϊ, εν Ιερουσαλήμ.
\par 14 Ούτοι δε είναι οι αριθμοί αυτών, κατά τους οίκους των πατριών αυτών· εκ του Ιούδα, χιλίαρχοι, Αδνά ο αρχηγός, και μετ' αυτού δυνατοί εν ισχύϊ τριακόσιαι χιλιάδες.
\par 15 Και μετά τούτον Ιωανάν ο αρχηγός, και μετ' αυτού διακόσιαι ογδοήκοντα χιλιάδες.
\par 16 Και μετά τούτον Αμασίας ο υιός του Ζιχρί, όστις προθύμως προσέφερεν εαυτόν εις τον Κύριον· και μετ' αυτού διακόσιαι χιλιάδες δυνατοί εν ισχύϊ.
\par 17 Εκ δε του Βενιαμίν, δυνατός εν ισχύϊ, Ελιαδά και μετ' αυτού τοξόται και ασπιδοφόροι διακόσιαι χιλιάδες.
\par 18 Και μετά τούτον Ιωζαβάδ, και μετ' αυτού εκατόν ογδοήκοντα χιλιάδες ώπλισμένοι εις πόλεμον.
\par 19 Ούτοι ήσαν οι υπηρετούντες τον βασιλέα, εκτός των όσους έβαλεν ο βασιλεύς εις τας οχυράς πόλεις εν παντί τω Ιούδα.

\chapter{18}

\par 1 Και είχεν ο Ιωσαφάτ πλούτον και δόξαν πολλήν· και εσυμπενθέρευσε μετά του Αχαάβ.
\par 2 Μετά δε χρόνους κατέβη προς τον Αχαάβ εις την Σαμάρειαν Και έσφαξεν ο Αχαάβ πρόβατα και βόας εν αφθονία δι' αυτόν και διά τον λαόν τον μετ' αυτού, και κατέπεισεν αυτόν να συναναβή εις Ραμώθ-γαλαάδ.
\par 3 Και είπεν Αχαάβ ο βασιλεύς του Ισραήλ προς Ιωσαφάτ τον βασιλέα του Ιούδα, Έρχεσαι μετ' εμού εις Ραμώθ-γαλαάδ; Ο δε απεκρίθη προς αυτόν, Εγώ είμαι καθώς συ, και ο λαός μου καθώς ο λαός σου· και θέλομεν είσθαι μετά σου εν τω πολέμω.
\par 4 Και είπεν ο Ιωσαφάτ προς τον βασιλέα του Ισραήλ, Ερώτησον σήμερον, παρακαλώ, τον λόγον του Κυρίου.
\par 5 Και συνήθροισεν ο βασιλεύς του Ισραήλ τους προφήτας, τετρακοσίους άνδρας, και είπε προς αυτούς, να υπάγωμεν εις Ραμώθ-γαλαάδ, διά να πολεμήσωμεν; ή να απέχω; Οι δε είπον, Ανάβα, και θέλει παραδώσει ο Θεός αυτήν εις την χείρα του βασιλέως.
\par 6 Και είπεν ο Ιωσαφάτ, Δεν είναι ενταύθα έτι προφήτης του Κυρίου, διά να ερωτήσωμεν δι' αυτού;
\par 7 Και είπεν ο βασιλεύς του Ισραήλ προς τον Ιωσαφάτ, είναι έτι άνθρωπός τις, διά του οποίου δυνάμεθα να ερωτήσωμεν τον Κύριον· πλην εγώ μισώ αυτόν· διότι δεν προφητεύει καλόν περί εμού αλλά πάντοτε κακόν· είναι ο Μιχαΐας ο υιός του Ιεμλά. Και είπεν ο Ιωσαφάτ, Ας μη λαλή ο βασιλεύς ούτως.
\par 8 Και εκάλεσεν ο βασιλεύς του Ισραήλ ένα ευνούχον και είπε, Σπεύσον να φέρης Μιχαΐαν τον υιόν του Ιεμλά.
\par 9 Ο δε βασιλεύς του Ισραήλ και Ιωσαφάτ ο βασιλεύς του Ιούδα εκάθηντο, έκαστος επί του θρόνου αυτού, ενδεδυμένοι στολαίς, και εκάθηντο εν τόπω ανοικτώ, κατά την είσοδον της πύλης της Σαμαρείας· και πάντες οι προφήται προεφήτευον έμπροσθεν αυτών.
\par 10 Και Σεδεκίας ο υιός του Χαναανά είχε κάμει εις εαυτόν κέρατα σιδηρά και είπεν, ούτω λέγει Κύριος· Διά τούτων θέλεις κερατίσει τους Συρίους, εωσού συντελέσης αυτούς.
\par 11 Και πάντες οι προφήται προεφήτευον ούτω, λέγοντες, Ανάβα εις Ραμώθ-γαλαάδ και ευοδού· διότι ο Κύριος θέλει παραδώσει αυτήν εις την χείρα του βασιλέως.
\par 12 Και ο μηνυτής, όστις υπήγε να καλέση τον Μιχαΐαν, είπε προς αυτόν, λέγων, Ιδού, οι λόγοι των προφητών φανερόνουσιν εξ ενός στόματος καλόν περί του βασιλέως· ο λόγος σου λοιπόν ας ήναι, παρακαλώ, ως ενός εξ εκείνων, και λάλησον το καλόν.
\par 13 Ο δε Μιχαΐας είπε, Ζη Κύριος, ό,τι μοι είπη ο Θεός μου, τούτο θέλω λαλήσει.
\par 14 Ήλθε λοιπόν προς τον βασιλέα, και είπεν ο βασιλεύς προς αυτόν, Μιχαΐα, να υπάγωμεν εις Ραμώθ-γαλαάδ διά να πολεμήσωμεν; ή να απέχω; Ο δε είπεν, Ανάβητε και ευοδούσθε, διότι θέλουσι παραδοθή εις την χείρα σας.
\par 15 Και είπε προς αυτόν ο βασιλεύς, Έως ποσάκις θέλω σε ορκίζει να μη λέγης προς εμέ παρά την αλήθειαν εν ονόματι Κυρίου;
\par 16 Ο δε είπεν· είδον πάντα τον Ισραήλ διεσπαρμένον επί τα όρη, ως πρόβατα μη έχοντα ποιμένα· και είπε Κύριος, Ούτοι δεν έχουσι κύριον· ας επιστρέψωσιν έκαστος εις τον οίκον αυτού εν ειρήνη.
\par 17 Και είπεν ο βασιλεύς του Ισραήλ προς τον Ιωσαφάτ, Δεν σοι είπα ότι δεν θέλει προφητεύσει καλόν περί εμού, αλλά κακόν;
\par 18 Και ο Μιχαΐας είπεν, Ακούσατε λοιπόν τον λόγον του Κυρίου· είδον τον Κύριον καθήμενον επί του θρόνου αυτού και πάσαν την στρατιάν του ουρανού παρισταμένην εκ δεξιών αυτού και εξ αριστερών αυτού.
\par 19 Και είπε ο Κύριος, Τις θέλει απατήσει Αχαάβ τον βασιλέα του Ισραήλ, ώστε να αναβή και να πέση εν Ραμώθ-γαλαάδ; Και ο μεν ελάλησε λέγων ούτως, ο δε λέγων ούτως.
\par 20 Τότε εξήλθε το πνεύμα και εστάθη ενώπιον Κυρίου και είπεν, Εγώ θέλω απατήσει αυτόν. Και είπε Κύριος προς αυτό, Τίνι τρόπω;
\par 21 Και είπε, Θέλω εξέλθει και θέλω είσθαι πνεύμα ψεύδους εν τω στόματι πάντων των προφητών αυτού. Και είπε Κύριος, Θέλεις απατήσει και μάλιστα θέλεις κατορθώσει· έξελθε και κάμε ούτω.
\par 22 Τώρα λοιπόν, ιδού, ο Κύριος έβαλε πνεύμα ψεύδους εν τω στόματι τούτων των προφητών σου, και ελάλησε Κύριος κακόν επί σε.
\par 23 Τότε πλησιάσας Σεδεκίας ο υιός του Χαναανά, ερράπισε τον Μιχαΐαν επί την σιαγόνα και είπε, Διά ποίας οδού επέρασε το πνεύμα του Κυρίου απ' εμού, διά να λαλήση προς σε;
\par 24 Και είπεν ο Μιχαΐας, Ιδού, θέλεις ιδεί καθ' ην ημέραν θέλεις εισέρχεσθαι από ταμείου εις ταμείον, διά να κρυφθής.
\par 25 Και είπεν ο βασιλεύς του Ισραήλ, Πιάσατε τον Μιχαΐαν και επαναφέρετε αυτόν προς Αμών τον άρχοντα της πόλεως, και προς Ιωάς τον υιόν του βασιλέως,
\par 26 και είπατε, Ούτω λέγει ο βασιλεύς· Βάλετε τούτον εις την φυλακήν και τρέφετε αυτόν με άρτον θλίψεως και με ύδωρ θλίψεως, εωσού επιστρέψω εν ειρήνη.
\par 27 Και είπεν ο Μιχαΐας, Εάν τωόντι επιστρέψης εν ειρήνη, ο Κύριος δεν ελάλησε δι' εμού. Και είπεν, Ακούσατε σεις, πάντες οι λαοί.
\par 28 Και ανέβη ο βασιλεύς του Ισραήλ και Ιωσαφάτ ο βασιλεύς του Ιούδα εις Ραμώθ-γαλαάδ.
\par 29 Και είπεν ο βασιλεύς του Ισραήλ προς τον Ιωσαφάτ, Εγώ θέλω μετασχηματισθή και εισέλθει εις την μάχην· συ δε ενδύθητι την στολήν σου. Και μετεσχηματίσθη ο βασιλεύς του Ισραήλ και εισήλθον εις την μάχην.
\par 30 Ο δε βασιλεύς της Συρίας είχε προστάξει τους άρχοντας των αμαξών αυτού, λέγων, Μη πολεμείτε μήτε μικρόν μήτε μέγαν, αλλά μόνον τον βασιλέα του Ισραήλ.
\par 31 Και ως είδον οι άρχοντες των αμαξών τον Ιωσαφάτ, τότε αυτοί είπον, Ούτος είναι ο βασιλεύς του Ισραήλ· και περιεκύκλωσαν αυτόν διά να πολεμήσωσιν αυτόν· αλλ' ο Ιωσαφάτ ανεβόησε, και εβοήθησεν αυτόν ο Κύριος· και απέστρεψεν αυτούς ο Θεός απ' αυτού.
\par 32 Ιδόντες δε οι άρχοντες των αμαξών ότι δεν ήτο ο βασιλεύς του Ισραήλ, επέστρεψαν από της καταδιώξεως αυτού.
\par 33 Άνθρωπος δε τις, τοξεύσας ασκόπως, εκτύπησε τον βασιλέα του Ισραήλ μεταξύ των αρθρώσεων του θώρακος· ο δε είπεν προς τον ηνίοχον, Στρέψον την χείρα σου και έκβαλέ με εκ του στρατεύματος, διότι επληγώθην.
\par 34 Και εμεγαλύνθη η μάχη εν τη ημέρα εκείνη· ο δε βασιλεύς του Ισραήλ ίστατο επί της αμάξης αντικρύ των Συρίων έως εσπέρα, · και περί την δύσιν του ηλίου απέθανε.

\chapter{19}

\par 1 Και επέστρεψεν Ιωσαφάτ ο βασιλεύς του Ιούδα εις τον οίκον αυτού εν ειρήνη, εις Ιερουσαλήμ.
\par 2 Και εξήλθεν Ιηού ο υιός του Ανανί, ο βλέπων, εις απάντησιν αυτού, και είπε προς τον βασιλέα Ιωσαφάτ, Τον ασεβή βοηθείς και τους μισούντας τον Κύριον αγαπάς; διά τούτο οργή παρά του Κυρίου είναι επί σέ·
\par 3 πλην ευρέθησαν εν σοι καλά πράγματα, καθότι αφήρεσας τα άλση από της γης και κατεύθυνας την καρδίαν σου εις το να εκζητής τον Θεόν.
\par 4 Και κατώκησεν ο Ιωσαφάτ εν Ιερουσαλήμ· έπειτα εξήλθε πάλιν διά του λαού από Βηρ-σαβεέ έως του όρους Εφραΐμ, και επέστρεψεν αυτούς προς Κύριον τον Θεόν των πατέρων αυτών.
\par 5 Και κατέστησε κριτάς εν τη γη, εν πάσαις ταις οχυραίς πόλεσι του Ιούδα, εν εκάστη πόλει.
\par 6 Και είπε προς τους κριτάς, Ιδέτε τι κάμνετε σείς· διότι δεν κρίνετε κρίσιν ανθρώπου, αλλά του Κυρίου, όστις είναι μεθ' υμών εν τη κρισολογία·
\par 7 τώρα λοιπόν ας ήναι εφ' υμάς ο φόβος του Κυρίου· προσέχετε εις τας πράξεις σας· διότι δεν είναι αδικία παρά Κυρίω τω Θεώ ημών ουδέ προσωποληψία ουδέ δωροδοκία.
\par 8 Και εν Ιερουσαλήμ έτι κατέστησεν ο Ιωσαφάτ κριτάς εκ των Λευϊτών και των ιερέων και εκ των αρχηγών των πατριών του Ισραήλ, διά την κρίσιν του Κυρίου και διά τας διαφοράς, και προσέτρεχον εις Ιερουσαλήμ.
\par 9 Και προσέταξεν αυτούς, λέγων, Ούτω θέλετε κάμνει εν φόβω Κυρίου, εν πίστει και εν καρδία τελεία·
\par 10 και οποιαδήποτε διαφορά έλθη προς εσάς εκ των αδελφών σας, των κατοικούντων εν ταις πόλεσιν αυτών, αναμέσον αίματος και αίματος, αναμέσον νόμου και εντολής, διαταγμάτων και νομίμων, θέλετε νουθετεί αυτούς, διά να μη γίνωνται ένοχοι εις τον Κύριον, και έλθη οργή εφ' υμάς και επί τους αδελφούς υμών· ούτω κάμνετε, και δεν θέλετε γίνεσθαι ένοχοι·
\par 11 και ιδού, Αμαρίας ο ιερεύς θέλει είσθαι ο αρχηγός υμών εν πάση υποθέσει του Κυρίου, και Ζεβαδίας ο υιός του Ισραήλ, ο άρχων του οίκου Ιούδα, εν πάση υποθέσει του βασιλέως· οι δε Λευΐται θέλουσιν είσθαι επιστάται έμπροσθέν σας· ανδρίζεσθε και πράττετε, και ο Κύριος θέλει είσθαι μετά του αγαθού.

\chapter{20}

\par 1 Και μετά ταύτα ήλθον κατά του Ιωσαφάτ οι υιοί Μωάβ και οι υιοί Αμμών και μετ' αυτών άλλοι εκτός των Αμμωνιτών, διά να πολεμήσωσι.
\par 2 Και ήλθον και απήγγειλαν προς τον Ιωσαφάτ, λέγοντες, Μέγα πλήθος έρχεται εναντίον σου εκ του πέραν της θαλάσσης, εκ της Συρίας· και ιδού, είναι εν Ασασών-θαμάρ, ήτις είναι Εν-γαδδί.
\par 3 Και εφοβήθη ο Ιωσαφάτ και εδόθη εις το να εκζητή τον Κύριον, και εκήρυξε νηστείαν διά παντός του Ιούδα.
\par 4 Και συνήχθησαν οι άνδρες Ιούδα, διά να ζητήσωσι βοήθειαν παρά Κυρίου· εκ πασών έτι των πόλεων Ιούδα ήλθον διά να ζητήσωσι τον Κύριον.
\par 5 Και εστάθη ο Ιωσαφάτ εν τη συνάξει του Ιούδα και της Ιερουσαλήμ, εν τω οίκω του Κυρίου, κατά πρόσωπον της νέας αυλής,
\par 6 και είπε, Κύριε Θεέ των πατέρων ημών, δεν είσαι συ ο Θεός ο εν τω ουρανώ; και δεν είσαι συ ο κυριεύων επί πάντα τα βασίλεια των εθνών, και δεν είναι εν τη χειρί σου η δύναμις και η ισχύς, και ουδείς δύναται να αντισταθή εις σε;
\par 7 Δεν είσαι συ ο Θεός ημών, ο εκδιώξας τους κατοίκους της γης ταύτης έμπροσθεν του λαού σου Ισραήλ, και δους αυτήν εις το σπέρμα του Αβραάμ του αγαπητού σου εις τον αιώνα;
\par 8 Και κατώκησαν εν αυτή και ωκοδόμησαν εις σε αγιαστήριον εν αυτή διά το όνομά σου, λέγοντες,
\par 9 Εάν, όταν επέλθη εφ' ημάς κακόν, ρομφαία, κρίσις ή θανατικόν ή πείνα, σταθώμεν έμπροσθεν του οίκου τούτου και ενώπιόν σου, διότι το όνομά σου είναι εν τω οίκω τούτω, και βοήσωμεν προς σε εν τη θλίψει ημών, τότε θέλεις ακούσει και σώσει.
\par 10 Και τώρα, ιδού, οι υιοί Αμμών και Μωάβ και οι από του όρους Σηείρ, προς τους οποίους δεν αφήκας τον Ισραήλ να υπάγη, ότε ήρχοντο εκ γης Αιγύπτου, αλλ' εξέκλιναν απ' αυτών και δεν εξωλόθρευσαν αυτούς,
\par 11 και ιδού, πως ανταμείβουσιν ημάς, ερχόμενοι να εκβάλωσιν ημάς από της κληρονομίας σου, την οποίαν έδωκας εις ημάς να κληρονομήσωμεν.
\par 12 Θεέ ημών, δεν θέλεις κρίνει αυτούς; διότι δεν υπάρχει εις ημάς δύναμις διά να αντισταθώμεν εις τούτο το μέγα πλήθος, το οποίον έρχεται εφ' ημάς, και δεν εξεύρομεν τι να κάμωμεν· αλλ' επί σε είναι οι οφθαλμοί ημών.
\par 13 Και ίστατο πας ο Ιούδας ενώπιον του Κυρίου με τα βρέφη αυτών, τας γυναίκας αυτών και τους υιούς αυτών.
\par 14 Τότε ήλθε Πνεύμα Κυρίου επί Ιααζιήλ τον υιόν του Ζαχαρίου, υιού του Βεναΐα, υιού του Ιεϊήλ, υιού του Ματθανίου του Λευΐτου, εκ των υιών του Ασάφ, εν τω μέσω της συνάξεως.
\par 15 και είπε, Ακούσατε, πας ο Ιούδας και οι κατοικούντες Ιερουσαλήμ, και συ βασιλεύ Ιωσαφάτ· ούτω λέγει Κύριος προς υμάς· Μη φοβείσθε σεις μηδέ πτοηθήτε από προσώπου τούτου του μεγάλου πλήθους· διότι η μάχη δεν είναι υμών, αλλά του Θεού·
\par 16 κατάβητε αύριον εναντίον αυτών· ιδού, αναβαίνουσι διά της αναβάσεως Σίς· και θέλετε ευρεί αυτούς εν τω άκρω του χειμάρρου, έμπροσθεν της ερήμου Ιερουήλ·
\par 17 δεν θέλετε πολεμήσει σεις εν ταύτη τη μάχη· παρουσιάσθητε, στήτε και ιδέτε την μεθ' υμών σωτηρίαν του Κυρίου, Ιούδα και Ιερουσαλήμ· μη φοβείσθε μηδέ πτοηθήτε· αύριον εξέλθετε εναντίον αυτών· και ο Κύριος μεθ' υμών.
\par 18 Και έκυψεν ο Ιωσαφάτ επί πρόσωπον εις την γήν· και πας ο Ιούδας και οι κατοικούντες την Ιερουσαλήμ έπεσον ενώπιον του Κυρίου, προσκυνούντες τον Κύριον.
\par 19 Και εσηκώθησαν οι Λευΐται, εκ των υιών των Κααθιτών και εκ των υιών των Κοριτών, διά να υμνήσωσι Κύριον τον Θεόν του Ισραήλ εν φωνή υψωμένη σφόδρα.
\par 20 Και εξεγερθέντες το πρωΐ· εξήλθον προς την έρημον Θεκουέ· και ότε εξήλθον, εστάθη ο Ιωσαφάτ και είπεν, Ακούσατέ μου, Ιούδα και οι κατοικούντες την Ιερουσαλήμ· πιστεύσατε εις Κύριον τον Θεόν υμών, και θέλετε στερεωθή· πιστεύσατε τους προφήτας αυτού και θέλετε ευοδωθή.
\par 21 Και συμβουλευθείς μετά του λαού, διέταξε ψαλτωδούς διά να ψάλλωσιν εις τον Κύριον και να υμνώσι την μεγαλοπρέπειαν της αγιότητος αυτού, εξελθόντες έμπροσθεν του στρατεύματος, και να λέγωσι, Δοξολογείτε τον Κύριον, διότι το έλεος αυτού μένει εις τον αιώνα.
\par 22 Και ότε ήρχισαν να ψάλλωσι και να υμνώσιν, ο Κύριος έστησεν ενέδρας εναντίον των υιών Αμμών, Μωάβ και των εκ του όρους Σηείρ, των ελθόντων κατά του Ιούδα· και εκτυπήθησαν.
\par 23 Διότι εσηκώθησαν οι υιοί Αμμών και Μωάβ κατά των κατοίκων του όρους Σηείρ, διά να εξολοθρεύσωσι και να εξαλείψωσιν αυτούς· και αφού συνετέλεσαν τους κατοίκους του Σηείρ εβοήθησαν αλλήλους διά να εξολοθρευθώσιν.
\par 24 Ελθών δε ο Ιούδας εις την σκοπιάν της ερήμου, ανέβλεψε προς το πλήθος, και ιδού, ήσαν νεκρά σώματα πεπτωκότα κατά γης, και ουδείς διεσώθη.
\par 25 Και ότε ήλθον ο Ιωσαφάτ και ο λαός αυτού διά να λαφυραγωγήσωσιν αυτούς, εύρηκαν μεταξύ των νεκρών σωμάτων αυτών και πλούτη εν αφθονία και πολύτιμον αποσκευήν, και έλαβον εις εαυτούς τοσαύτα, ώστε δεν ηδύναντο να μεταφέρωσιν αυτά· και εστάθησαν τρεις ημέρας λαφυραγωγούντες, διότι τα λάφυρα ήσαν πολλά.
\par 26 Και την τετάρτην ημέραν συνήχθησαν εν τη κοιλάδι της Ευλογίας· διότι εκεί ευλόγησαν τον Κύριον· διά τούτο ωνομάσθη το όνομα του τόπου εκείνου Κοιλάς Ευλογίας έως της ημέρας ταύτης.
\par 27 Τότε πάντες οι άνδρες Ιούδα και της Ιερουσαλήμ και ο Ιωσαφάτ επί κεφαλής αυτών, εκίνησαν διά να επιστρέψωσιν εις Ιερουσαλήμ εν ευφροσύνη· διότι εύφρανεν αυτούς ο Κύριος από των εχθρών αυτών.
\par 28 Και ήλθον εις Ιερουσαλήμ εν ψαλτηρίοις και κιθάραις και σάλπιγξι, προς τον οίκον του Κυρίου.
\par 29 Και επέπεσε φόβος Θεού επί πάντα τα βασίλεια των τόπων εκείνων; ότε ήκουσαν έτι ο Κύριος επολέμησεν εναντίον των εχθρών του Ισραήλ.
\par 30 Και ησύχασεν η βασιλεία του Ιωσαφάτ· διότι ο Θεός αυτού έδωκεν εις αυτόν ανάπαυσιν κυκλόθεν.
\par 31 Και εβασίλευσεν ο Ιωσαφάτ επί τον Ιούδαν· τριάκοντα πέντε ετών ηλικίας ήτο ότε εβασίλευσε, και εβασίλευσεν εικοσιπέντε έτη εν Ιερουσαλήμ· το δε όνομα της μητρός αυτού ήτο Αζουβά θυγάτηρ του Σιλεΐ.
\par 32 Και περιεπάτησεν εν τη οδώ Ασά του πατρός αυτού και δεν εξέκλινεν απ' αυτής, πράττων το ευθές ενώπιον του Κυρίου.
\par 33 Οι υψηλοί όμως τόποι δεν αφηρέθησαν· διότι ο λαός δεν είχον έτι κατευθύνει τας καρδίας αυτών προς τον Θεόν των πατέρων αυτών.
\par 34 Αι δε λοιπαί πράξεις του Ιωσαφάτ, αι πρώται και αι έσχαται, ιδού, είναι γεγραμμέναι εν τοις λόγοις του Ιηού υιού του Ανανί, οίτινες κατεγράφησαν εν τω βιβλίω των βασιλέων του Ισραήλ.
\par 35 Μετά δε ταύτα ηνώθη ο Ιωσαφάτ ο βασιλεύς του Ιούδα μετά του Οχοζίου βασιλέως του Ισραήλ, όστις έπραξε λίαν ασεβώς.
\par 36 Ηνώθη δε μετ' αυτού, διά να κάμωσι πλοία, τα οποία να πλεύσωσιν εις Θαρσείς· και έκαμον τα πλοία εν Εσιών-γάβερ.
\par 37 Τότε Ελιέζερ ο υιός του Δωδανά από Μαρησά προεφήτευσεν εναντίον του Ιωσαφάτ, λέγων, Επειδή ηνώθης μετά του Οχοζίου, ο Κύριος έθραυσε τα έργα σου. Και συνετρίβησαν τα πλοία και δεν ηδυνήθησαν να υπάγωσιν εις Θαρσείς.

\chapter{21}

\par 1 Και εκοιμήθη ο Ιωσαφάτ μετά των πατέρων αυτού και ετάφη μετά των πατέρων αυτού εν πόλει Δαβίδ· και εβασίλευσεν αντ' αυτού Ιωράμ ο υιός αυτού.
\par 2 Και είχεν αδελφούς, υιούς του Ιωσαφάτ, τον Αζαρίαν, και Ιεχιήλ και Ζαχαρίαν και Αζαρίαν και Μιχαήλ και Σεφατίαν· πάντες ούτοι ήσαν υιοί του Ιωσαφάτ βασιλέως του Ισραήλ.
\par 3 Και ο πατήρ αυτών έδωκεν εις αυτούς δώρα πολλά αργυρίου και χρυσίου και πολυτίμων πραγμάτων, μετά πόλεων οχυρών εν Ιούδα· την βασιλείαν όμως έδωκεν εις τον Ιωράμ, επειδή ήτο ο πρωτότοκος.
\par 4 Ότε δε ο Ιωράμ υψώθη εις την βασιλείαν του πατρός αυτού και εκραταιώθη, εθανάτωσε πάντας τους αδελφούς αυτού εν ρομφαία και τινάς έτι εκ των αρχόντων του Ισραήλ.
\par 5 Τριάκοντα δύο ετών ηλικίας ήτο ο Ιωράμ ότε εβασίλευσε, και εβασίλευσεν οκτώ έτη εν Ιερουσαλήμ.
\par 6 Και περιεπάτησεν εν τη οδώ των βασιλέων του Ισραήλ, καθώς έκαμεν ο οίκος του Αχαάβ· διότι θυγάτηρ του Αχαάβ ήτο η γυνή αυτού· και έπραξε πονηρά ενώπιον Κυρίου.
\par 7 Αλλ' ο Κύριος δεν ηθέλησε να εξολοθρεύση τον οίκον του Δαβίδ, διά την διαθήκην την οποίαν έκαμε προς τον Δαβίδ, και διότι είπε να δώση λύχνον εις αυτόν και εις τους υιούς αυτού πάντοτε.
\par 8 Εν ταις ημέραις αυτού απεστάτησεν ο Εδώμ από της υποταγής του Ιούδα, και κατέστησαν βασιλέα εφ' εαυτούς.
\par 9 Και διήλθεν ο Ιωράμ μετά των αρχόντων αυτού και πάσαι αι άμαξαι μετ' αυτού· και σηκωθείς διά νυκτός, επάταξε τους Ιδουμαίους τους περικυκλούντας αυτόν και τους άρχοντας των αμαξών.
\par 10 Ούτως απεστάτησεν ο Εδώμ από της υποταγής του Ιούδα έως της ημέρας ταύτης. Τότε κατά τον αυτόν καιρόν απεστάτησε και η Λιβνά από της υποταγής αυτού, επειδή εγκατέλιπε Κύριον τον Θεόν των πατέρων αυτού.
\par 11 Αυτός ωκοδόμησεν έτι υψηλούς τόπους επί τα όρη του Ιούδα, και έκαμε τους κατοίκους της Ιερουσαλήμ να πορνεύωσι και απεπλάνησε τον Ιούδαν.
\par 12 Και ήλθε προς αυτόν έγγραφον παρά του Ηλία του προφήτου, λέγον, Ούτω λέγει Κύριος ο Θεός του Δαβίδ του πατρός σου· Επειδή δεν περιεπάτησας εν ταις οδοίς Ιωσαφάτ του πατρός σου και εν ταις οδοίς του Ασά βασιλέως του Ιούδα,
\par 13 αλλά περιεπάτησας εν τη οδώ των βασιλέων του Ισραήλ, και έκαμες τον Ιούδαν και τους κατοίκους της Ιερουσαλήμ να πορνεύσωσι κατά τας πορνείας του οίκου του Αχαάβ, έτι δε εθανάτωσας τους αδελφούς σου, τον οίκον του πατρός σου, τους καλητέρους σου,
\par 14 Ιδού, ο Κύριος θέλει πατάξει με πληγήν μεγάλην τον λαόν σου και τα τέκνα σου και τας γυναίκάς σου και πάντα τα υπάρχοντά σου·
\par 15 και συ θέλεις κτυπηθή με πολλάς αρρωστίας, με αρρωστίαν των εντοσθίων σου, εωσού εξέλθωσι τα εντόσθιά σου εκ της αρρωστίας από ημέρας εις ημέραν.
\par 16 Ο Κύριος έτι διήγειρεν εναντίον του Ιωράμ το πνεύμα των Φιλισταίων και των Αράβων, των πλησιοχώρων των Αιθιόπων·
\par 17 και ανέβησαν κατά του Ιούδα και εφώρμησαν επ' αυτόν και διήρπασαν πάντα τα υπάρχοντα τα ευρεθέντα εν τω οίκω του βασιλέως, και τους υιούς αυτού έτι και τας γυναίκας αυτού· ώστε δεν έμεινεν εις αυτόν άλλος υιός, ειμή Ιωάχαζ, ο νεώτερος των υιών αυτού.
\par 18 Μετά δε πάντα ταύτα επάταξεν αυτόν ο Κύριος εις τα εντόσθια αυτού με αρρωστίαν ανίατον·
\par 19 και προϊόντος του καιρού, μετά παρέλευσιν δύο ετών, εξήλθον τα εντόσθια αυτού, εκ της αρρωστίας αυτού, και απέθανε με πόνους σκληρούς. Ο δε λαός αυτού δεν έκαμεν εις αυτόν καύσιν, κατά την καύσιν των πατέρων αυτού.
\par 20 Τριάκοντα δύο ετών ηλικίας ήτο ότε εβασίλευσεν· εβασίλευσε δε εν Ιερουσαλήμ οκτώ έτη, και απήλθε χωρίς να ήναι ποθητός· και έθαψαν αυτόν εν πόλει Δαβίδ, πλην ουχί εν τοις τάφοις των βασιλέων.

\chapter{22}

\par 1 Και έκαμον οι κάτοικοι της Ιερουσαλήμ αντ' αυτού βασιλέα Οχοζίαν τον νεώτερον αυτού υιόν· διότι πάντας τους πρεσβυτέρους εθανάτωσαν τα τάγματα τα επελθόντα μετά των Αράβων εις το στρατόπεδον. Και εβασίλευσεν Οχοζίας ο υιός του Ιωράμ βασιλέως του Ιούδα.
\par 2 Τεσσαράκοντα δύο ετών ηλικίας ήτο ο Οχοζίας ότε εβασίλευσεν, εβασίλευσε δε εν έτος εν Ιερουσαλήμ· το δε όνομα της μητρός αυτού ήτο Γοθολία, θυγάτηρ του Αμρί.
\par 3 Και αυτός περιεπάτησεν εν ταις οδοίς του οίκου Αχαάβ· διότι η μήτηρ αυτού ήτο σύμβουλος αυτού εις το αμαρτάνειν.
\par 4 Και έπραξε πονηρά ενώπιον του Κυρίου, καθώς ο οίκος Αχαάβ· διότι μετά τον θάνατον του πατρός αυτού, αυτοί ήσαν οι σύμβουλοι αυτού διά τον αφανισμόν αυτού.
\par 5 Και διά των συμβουλών αυτών υπήγε μετά του Ιωράμ υιού του Αχαάβ βασιλέως του Ισραήλ, εις πόλεμον εναντίον του Αζαήλ βασιλέως της Συρίας εις Ραμώθ-γαλαάδ· και επάταξαν οι Σύριοι τον Ιωράμ.
\par 6 Και επέστρεψε διά να ιατρευθή εις Ιεζραέλ, εξ αιτίας των πληγών τας οποίας έλαβεν εν Ραμά, ότε επολέμει κατά του Αζαήλ βασιλέως της Συρίας. Και κατέβη Αζαρίας ο υιός του Ιωράμ, ο βασιλεύς του Ιούδα, διά να ίδη Ιωράμ τον υιόν του Αχαάβ εις Ιεζραέλ, επειδή ήτο άρρωστος.
\par 7 Και εστάθη παρά Θεού όλεθρος του Οχοζίου το να έλθη προς τον Ιωράμ· διότι, ότε ήλθεν, εξήλθε μετά του Ιωράμ εναντίον Ιηού του υιού του Νιμσί, τον οποίον έχρισεν ο Κύριος διά να εξολοθρεύση τον οίκον Αχαάβ.
\par 8 Και ότε έκαμνεν ο Ιηού την εκδίκησιν κατά του οίκου Αχαάβ, ευρών τους άρχοντας του Ιούδα και τους υιούς των αδελφών του Οχοζίου, τους υπηρετούντας τον Οχοζίαν, εθανάτωσεν αυτούς.
\par 9 Και εζήτησε τον Οχοζίαν· και συνέλαβον αυτόν κρυπτόμενον εν Σαμαρεία και έφεραν αυτόν προς τον Ιηού· και εθανάτωσαν αυτόν και έθαψαν αυτόν· διότι είπον, Υιός του Ιωσαφάτ είναι, όστις εξεζήτησε τον Κύριον εξ όλης της καρδίας αυτού. Και ο οίκος Οχοζίου δεν είχε δύναμιν να κρατήση πλέον την βασιλείαν.
\par 10 Η δε Γοθολία, η μήτηρ του Οχοζίου, ιδούσα ότι ο υιός αυτής απέθανεν, εσηκώθη και εξωλόθρευσεν άπαν το βασιλικόν σπέρμα του οίκου Ιούδα.
\par 11 Ιωσαβεέθ όμως, η θυγάτηρ του βασιλέως, λαβούσα τον Ιωάς υιόν του Οχοζίου, έκλεψεν αυτόν εκ του μέσου των υιών του βασιλέως των θανατουμένων, και έβαλεν αυτόν και την τροφήν αυτού εν τω ταμείω του κοιτώνος. Ούτως η Ιωσαβεέθ, η θυγάτηρ του βασιλέως Ιωράμ, η γυνή Ιωδαέ του ιερέως, διότι ήτο αδελφή του Οχοζίου, έκρυψεν αυτόν από προσώπου της Γοθολίας, και δεν εθανάτωσεν αυτόν.
\par 12 Και ήτο μετ' αυτών κρυπτόμενος εν τω οίκω του Θεού εξ έτη· η δε Γοθολία εβασίλευεν επί της γης.

\chapter{23}

\par 1 Εν δε τω εβδόμω έτει εκραταιώθη ο Ιωδαέ, και λαβών τους εκατοντάρχους, Αζαρίαν τον υιόν του Ιεροάμ και Ισμαήλ τον υιόν του Ιωανάν και Αζαρίαν τον υιόν του Ωβήδ και Μαασίαν τον υιόν του Αδαΐου και Ελισαφάτ τον υιόν του Ζιχρί, έκαμε συνθήκην μετ' αυτών.
\par 2 Και περιήλθον τον Ιούδαν και συνήγαγον τους Λευΐτας εκ πασών των πόλεων του Ιούδα και τους αρχηγούς των πατριών του Ισραήλ, και ήλθον εις Ιερουσαλήμ.
\par 3 Και πάσα η σύναξις έκαμε συνθήκην μετά του βασιλέως εν τω οίκω του Θεού. Και είπε προς αυτούς, Ιδού, ο υιός του βασιλέως θέλει βασιλεύσει, καθώς ελάλησε Κύριος περί των υιών του Δαβίδ.
\par 4 Τούτο είναι το πράγμα, το οποίον θέλετε κάμει· το τρίτον από σας οι εισερχόμενοι το σάββατον, εκ των ιερέων και εκ των Λευϊτών, θέλουσι φυλάττει εν ταις πύλαις·
\par 5 και το τρίτον εν τω οίκω του βασιλέως· και το τρίτον εν τη πύλη του θεμελίου· άπας δε ο λαός εν ταις αυλαίς του οίκου του Κυρίου·
\par 6 και ουδείς θέλει εισέρχεσθαι εις τον ναόν του Κυρίου, ειμή οι ιερείς και όσοι εκ των Λευϊτών λειτουργούσιν· αυτοί θέλουσιν εισέρχεσθαι, διότι είναι άγιοι άπας δε ο λαός θέλει φυλάττει την φυλακήν του Κυρίου·
\par 7 και οι Λευΐται θέλουσι περικυκλόνει τον βασιλέα κύκλω, έκαστος έχων τα όπλα αυτού εν τη χειρί αυτού· και όστις εισέλθη εις τον οίκον, ας θανατόνεται και θέλετε είσθαι μετά του βασιλέως, όταν εισέρχηται και όταν εξέρχηται.
\par 8 Και έκαμον οι Λευΐται και πας ο Ιούδας κατά πάντα όσα προσέταξεν Ιωδαέ ο ιερεύς, και έλαβον έκαστος τους άνδρας αυτού, τους εισερχομένους το σάββατον, μετά των εξερχομένων το σάββατον· διότι Ιωδαέ ο ιερεύς δεν απέλυε τας τάξεις.
\par 9 Και έδωκεν Ιωδαέ ο ιερεύς εις τους εκατοντάρχους τας λόγχας και τους θυρεούς και τας ασπίδας του βασιλέως Δαβίδ, τας εν τω οίκω του Θεού.
\par 10 Και έστησε πάντα τον λαόν, έκαστον άνδρα έχοντα τα όπλα αυτού εν τη χειρί αυτού, τα από της δεξιάς πλευράς του οίκου έως της αριστεράς πλευράς του οίκου, πλησίον του θυσιαστηρίου και του ναού, κύκλω του βασιλέως.
\par 11 Τότε εξήγαγον τον υιόν του βασιλέως, και επέθεσαν επ' αυτόν το διάδημα και το μαρτύριον, και έκαμον αυτόν βασιλέα. Και έχρισαν αυτόν ο Ιωδαέ και οι υιοί αυτού και είπον, Ζήτω ο βασιλεύς.
\par 12 Και ακούσασα η Γοθολία την φωνήν του λαού τρέχοντος και ευφημούντος τον βασιλέα, ήλθε προς τον λαόν εις τον οίκον του Κυρίου.
\par 13 Και είδε, και ιδού, ο βασιλεύς ίστατο πλησίον του στύλου αυτού εν τη εισόδω, και οι άρχοντες και αι σάλπιγγες πλησίον του βασιλέως· και πας ο λαός της γης έχαιρε και εσάλπιζον εν ταις σάλπιγξι, και οι ψαλτωδοί έψαλλον εν τοις μουσικοίς οργάνοις και όσοι ήσαν επιστήμονες εις το υμνωδείν· τότε διέρρηξεν η Γοθολία τα ιμάτια αυτής και είπε, Προδοσία. Προδοσία.
\par 14 Και εξήγαγεν Ιωδαέ ο ιερεύς τους εκατοντάρχους, τους αρχηγούς του στρατεύματος, και είπε προς αυτούς, Εκβάλετε αυτήν έξω των τάξεων· και όστις ακολουθήση αυτήν, ας θανατόνεται εν μαχαίρα. Διότι ο ιερεύς είχεν ειπεί, Μη θανατώσητε αυτήν εν τω οίκω του Κυρίου.
\par 15 Και έβαλον χείρας επ' αυτήν· και ότε ήλθεν εις την είσοδον της πύλης των ίππων, την εις τον οίκον του βασιλέως, εθανάτωσαν αυτήν εκεί.
\par 16 Και έκαμεν ο Ιωδαέ διαθήκην αναμέσον εαυτού και παντός του λαού και του βασιλέως, ότι θέλουσιν είσθαι λαός του Κυρίου.
\par 17 Και εισήλθον ο πας ο λαός εις τον οίκον του Βάαλ, και εκρήμνισαν αυτόν και τα θυσιαστήρια αυτού και τα είδωλα αυτού κατεσύντριψαν· και Ματθάν τον ιερέα του Βάαλ εθανάτωσαν έμπροσθεν των θυσιαστηρίων.
\par 18 Και έδωκεν ο Ιωδαέ την επιτήρησιν του οίκου του Κυρίου εις τας χείρας των ιερέων των Λευϊτών, τους οποίους ο Δαβίδ διήρεσεν επί του οίκου του Κυρίου, διά να προσφέρωσι ολοκαυτώματα του Κυρίου, ως είναι γραμμένον εν τω νόμω του Μωϋσέως, ευφροσύνη και εν ωδαίς, κατά την διάταξιν του Δαβίδ.
\par 19 Και έστησε τους πυλωρούς εν ταις πύλαις του οίκου του Κυρίου, διά να μη εισέρχηται μηδείς ακάθαρτος δι' οποιονδήποτε πράγμα.
\par 20 Και έλαβε τους εκατοντάρχους και τους δυνατούς και τους άρχοντας του λαού και πάντα τον λαόν της γης, και κατεβίβασε τον βασιλέα εκ του οίκου του Κυρίου· και ήλθον διά της υψηλής πύλης εις τον οίκον του βασιλέως και εκάθισαν τον βασιλέα επί του θρόνου της βασιλείας.
\par 21 Και ευφράνθη πας ο λαός της γής· και η πόλις ησύχασε· την δε Γοθολίαν εθανάτωσαν εν μαχαίρα.

\chapter{24}

\par 1 Επτά ετών ηλικίας ήτο ο Ιωάς ότε εβασίλευσεν· εβασίλευσε δε τεσσαράκοντα έτη εν Ιερουσαλήμ· το δε όνομα της μητρός αυτού ήτο Σιβιά, εκ Βηρ-σαβεέ.
\par 2 Και έπραττεν ο Ιωάς το ευθές ενώπιον Κυρίου, πάσας τας ημέρας Ιωδαέ του ιερέως.
\par 3 Και έλαβεν εις αυτόν ο Ιωδαέ δύο γυναίκας, και εγέννησεν υιούς και θυγατέρας.
\par 4 Και μετά ταύτα ήλθεν εις την καρδίαν του Ιωάς να ανακαινίση τον οίκον του Κυρίου.
\par 5 Και συναγαγών τους ιερείς και τους Λευΐτας, είπε προς αυτούς, Εξέλθετε εις τας πόλεις του Ιούδα, και συνάγετε από παντός του Ισραήλ αργύριον προς επισκευήν του οίκου του Θεού σας κατ' έτος, και επισπεύσατε το πράγμα· οι Λευΐται όμως δεν επέσπευσαν.
\par 6 Και εκάλεσεν ο βασιλεύς τον Ιωδαέ τον αρχηγόν και είπε προς αυτόν, Διά τι δεν εζήτησας παρά των Λευϊτών να εισπράξωσιν εκ του Ιούδα και εκ της Ιερουσαλήμ τον φόρον του Μωϋσέως, του δούλου του Κυρίου, και της συναγωγής του Ισραήλ, διά την σκηνήν του μαρτυρίου;
\par 7 Διότι η Γοθολία, η ασεβής, και οι υιοί αυτής κατέφθειραν τον οίκον του Θεού· και πάντα έτι τα αφιερώματα του οίκου του Κυρίου ανέθηκαν εις τους Βααλείμ.
\par 8 Έκαμον λοιπόν κατά προσταγήν του βασιλέως εν κιβώτιον, και έθεσαν αυτό εν τη πύλη του οίκου του Κυρίου έξω.
\par 9 Και διεκήρυξαν εις τον Ιούδαν και εις την Ιερουσαλήμ να εισφέρωσι προς τον Κύριον τον φόρον του Μωϋσέως του δούλου του Θεού, τον επιβληθέντα επί τον Ισραήλ εν τη ερήμω.
\par 10 Και ηυφράνθησαν πάντες οι άρχοντες και πας ο λαός, και εισέφερον και έρριπτον εις το κιβώτιον, εωσού γεμισθή.
\par 11 Ότε δε εφέρετο το κιβώτιον προς τους επιστάτας του βασιλέως διά χειρός των Λευϊτών, και ότε αυτοί έβλεπον ότι ήτο πολύ το αργύριον, ήρχετο ο γραμματεύς του βασιλέως και ο επιστάτης του ιερέως του πρώτου, και εξεκένονον το κιβώτιον και φέροντες έθετον αυτό πάλιν εις τον τόπον αυτού. Ούτως έκαμνον καθ' ημέραν και συνήγαγον αργύριον πολύ.
\par 12 Και έδιδεν αυτό ο βασιλεύς και ο Ιωδαέ εις τους ποιούντας το έργον της υπηρεσίας του οίκου του Κυρίου, και εμίσθονον κτίστας και ξυλουργούς διά να ανακαινίσωσι τον οίκον του Κυρίου· και σιδηρουργούς έτι και χαλκουργούς, διά να επισκευάσωσι τον οίκον του Κυρίου.
\par 13 Και οι εργαζόμενοι το έργον ειργάζοντο, και διά χειρός αυτών προέβη το έργον της επισκευής· και αποκατέστησαν τον οίκον του Θεού εις την προτέραν αυτού κατάστασιν και εστερέωσαν αυτόν.
\par 14 Και αφού ετελείωσαν, έφεραν το εναπολειφθέν αργύριον έμπροσθεν του βασιλέως και του Ιωδαέ, και εκ τούτου κατεσκεύασαν σκεύη διά τον οίκον του Κυρίου, σκεύη λειτουργίας και ολοκαυτώσεως και φιάλας και σκεύη χρυσά και αργυρά. Και προσέφερον ολοκαυτώματα εν τω οίκω του Κυρίου διά παντός, πάσας τας ημέρας του Ιωδαέ.
\par 15 Εγήρασε δε ο Ιωδαέ και ήτο πλήρης ημερών, και απέθανεν· εκατόν τριάκοντα ετών ηλικίας ήτο ότε απέθανε.
\par 16 Και έθαψαν αυτόν εν πόλει Δαβίδ, μετά των βασιλέων· επειδή έπραξε καλόν εν τω Ισραήλ και προς τον Θεόν και τον οίκον αυτού.
\par 17 Μετά δε τον θάνατον του Ιωδαέ ήλθον οι άρχοντες του Ιούδα και προσεκύνησαν τον βασιλέα· τότε ο βασιλεύς επήκουσεν αυτών·
\par 18 και εγκατέλιπον τον οίκον Κυρίου του Θεού των πατέρων αυτών, και ελάτρευον τα άλση και τα είδωλα· και ήλθεν οργή κατά του Ιούδα και της Ιερουσαλήμ, διά ταύτην την ανομίαν αυτών.
\par 19 Απέστειλε μεν προς αυτούς προφήτας, διά να επαναφέρωσιν αυτούς εις τον Κύριον, και διεμαρτυρήθησαν εναντίον αυτών· αλλά δεν έδωκαν ακρόασιν.
\par 20 Και περιεχύθη το Πνεύμα του Θεού επί Ζαχαρίαν τον υιόν του Ιωδαέ του ιερέως, και σταθείς επάνωθεν του λαού, είπε προς αυτούς, Ούτω λέγει ο Θεός· Διά τι παραβαίνετε σεις τας εντολάς του Κυρίου; δεν θέλετε βεβαίως ευοδωθή· επειδή σεις εγκατελίπετε τον Κύριον, και αυτός εγκατέλιπεν εσάς.
\par 21 Και συνώμοσαν κατ' αυτού· και ελιθοβόλησαν αυτόν με λίθους διά προσταγής του βασιλέως εν τη αυλή του οίκου του Κυρίου.
\par 22 Και δεν ενεθυμήθη Ιωάς ο βασιλεύς το έλεος, το οποίον έκαμεν εις αυτόν Ιωδαέ ο πατήρ αυτού, αλλ' εθανάτωσε τον υιόν αυτού· ενώ δε απέθνησκεν, είπεν, Ο Κύριος ας ίδη και ας εκζητήση.
\par 23 Και εν τω τέλει του έτους ανέβη το στράτευμα της Συρίας εναντίον αυτού· και ήλθον επί τον Ιούδαν και επί την Ιερουσαλήμ, και εξωλόθρευσαν πάντας τους άρχοντας του λαού εκ μέσου του λαού, και έστειλαν πάντα τα λάφυρα αυτών προς τον βασιλέα της Δαμασκού.
\par 24 Αν και το στράτευμα της Συρίας ήλθε μετ' ολίγων ανδρών, ο Κύριος όμως παρέδωκε στράτευμα μέγα σφόδρα εις την χείρα αυτών, επειδή εγκατέλιπον Κύριον τον Θεόν των πατέρων αυτών· και εξετέλεσαν κρίσιν κατά του Ιωάς.
\par 25 Αφού δε ανεχώρησαν απ' αυτού, αφήσαντες αυτόν εν αρρωστίαις μεγάλαις, συνώμοσαν εναντίον αυτού οι δούλοι αυτού διά το αίμα των υιών Ιωδαέ του ιερέως, και εθανάτωσαν αυτόν επί της κλίνης αυτού, και απέθανε· και έθαψαν αυτόν εν πόλει Δαβίδ, δεν έθαψαν όμως αυτόν εν τοις τάφοις των βασιλέων.
\par 26 Οι δε συνομόσαντες εναντίον αυτού ήσαν ούτοι Ζαβάδ ο υιός της Σιμεάθ της Αμμωνίτιδος και Ιωζαβάδ ο υιός της Σιμρίθ της Μωαβίτιδος.
\par 27 Περί δε των υιών αυτού και του πλήθους των υπ' αυτού φορτίων, και της επισκευής του οίκου του Θεού, ιδού, είναι γεγραμμένα εν τοις υπομνήμασι του βιβλίου των βασιλέων. Εβασίλευσε δε αντ' αυτού Αμασίας ο υιός αυτού.

\chapter{25}

\par 1 Εικοσιπέντε ετών ηλικίας εβασίλευσεν ο Αμασίας, και εβασίλευσεν εικοσιεννέα έτη εν Ιερουσαλήμ· το δε όνομα της μητρός αυτού ήτο Ιωαδάν, εξ Ιερουσαλήμ.
\par 2 Και έπραξε το ευθές ενώπιον Κυρίου, πλην ουχί εν καρδία τελεία.
\par 3 Ως δε η βασιλεία εκραταιώθη εις αυτόν, εθανάτωσε τους δούλους αυτού τους φονεύσαντας τον βασιλέα τον πατέρα αυτού·
\par 4 τα τέκνα όμως αυτών δεν εθανάτωσεν, ως είναι γεγραμμένον εν τω νόμω, εν τω βιβλίω του Μωϋσέως, όπου ο Κύριος προσέταξε, λέγων, οι πατέρες δεν θέλουσι θανατόνεσθαι διά τα τέκνα, ουδέ τα τέκνα θέλουσι θανατόνεσθαι διά τους πατέρας· αλλ' έκαστος θέλει θανατόνεσθαι διά το εαυτού αμάρτημα.
\par 5 Και συνήγαγεν ο Αμασίας τον Ιούδαν, και κατέστησεν εξ αυτών χιλιάρχους και εκατοντάρχους, κατ' οίκους πατριών, διά παντός του Ιούδα και Βενιαμίν· και ηρίθμησεν αυτούς από είκοσι ετών και επάνω, και εύρηκεν αυτούς τριακοσίας χιλιάδας, εκλεκτούς, εξερχομένους εις πόλεμον, κρατούντας λόγχην και ασπίδα.
\par 6 Εμίσθωσεν έτι εκ του Ισραήλ εκατόν χιλιάδας δυνατών εν ισχύϊ, δι' εκατόν τάλαντα αργυρίου.
\par 7 Ήλθε δε προς αυτόν άνθρωπος του Θεού, λέγων, Βασιλεύ, ας μη έλθη μετά σου το στράτευμα του Ισραήλ· διότι ο Κύριος δεν είναι μετά του Ισραήλ, μετά πάντων των υιών Εφραΐμ·
\par 8 αλλ' εάν θέλης να υπάγης, κάμε τούτο· ενδυναμώθητι διά τον πόλεμον· ο Θεός όμως θέλει σε κατατροπώσει έμπροσθεν του εχθρού· διότι ο Θεός έχει δύναμιν να βοηθήση και να κατατροπώση.
\par 9 Ο δε Αμασίας είπε προς τον άνθρωπον του Θεού, Αλλά τι θέλομεν κάμει διά τα εκατόν τάλαντα, τα οποία έδωκα εις το στράτευμα του Ισραήλ; Και ο άνθρωπος του Θεού απεκρίθη, Ο Κύριος είναι δυνατός να δώση εις σε πλειότερα τούτων.
\par 10 Τότε διεχώρισεν αυτούς ο Αμασίας, το στράτευμα το ελθόν προς αυτόν εκ του Εφραΐμ, διά να επιστρέψωσιν εις τον τόπον αυτών· και εξήφθη σφόδρα ο θυμός αυτών κατά του Ιούδα, και επέστρεψαν εις τον τόπον αυτών με έξαψιν θυμού.
\par 11 Ενεδυναμώθη δε ο Αμασίας και εξήγαγε τον λαόν αυτού και υπήγεν εις την κοιλάδα του άλατος και επάταξε τους υιούς Σηείρ δέκα χιλιάδας.
\par 12 Και δέκα χιλιάδας ζώντας ηχμαλώτισαν οι υιοί Ιούδα, και έφεραν αυτούς εις το άκρον του κρημνού και κατεκρήμνιζον αυτούς από του άκρου του κρημνού, ώστε πάντες διερράγησαν.
\par 13 Οι άνδρες όμως του στρατεύματος, το οποίον απέπεμψεν ο Αμασίας, διά να μη υπάγωσι μετ' αυτού εις πόλεμον, επέπεσον επί τας πόλεις του Ιούδα, από Σαμαρείας έως Βαιθ-ωρών, και επάταξαν τρεις χιλιάδας εξ αυτών και έλαβον λάφυρα πολλά.
\par 14 Αφού δε ο Αμασίας επέστρεψεν από της σφαγής των Ιδουμαίων, έφερε τους θεούς των υιών Σηείρ και έστησεν αυτούς εις εαυτόν θεούς και προσεκύνησεν έμπροσθεν αυτών και εθυμίασεν εις αυτούς.
\par 15 Διά τούτο εξήφθη η οργή του Κυρίου κατά του Αμασίου· και απέστειλε προς αυτόν προφήτην και είπε προς αυτόν, Διά τι εξεζήτησας τους θεούς του λαού, οίτινες δεν ηδυνήθησαν να ελευθερώσωσι τον λαόν αυτών εκ της χειρός σου;
\par 16 Και ενώ ελάλει προς αυτόν, ο βασιλεύς είπε προς αυτόν, Σύμβουλον σε έκαμον του βασιλέως; παύσον· διά τι να θανατωθής; Και έπαυσεν ο προφήτης, ειπών, Εξεύρω ότι ο Θεός εβουλεύθη να σε εξολοθρεύση, επειδή έκαμες τούτο και δεν υπήκουσας εις την συμβουλήν μου.
\par 17 Τότε συνεβουλεύθη Αμασίας ο βασιλεύς του Ιούδα και απέστειλε προς τον Ιωάς υιόν του Ιωάχαζ, υιού του Ιηού, τον βασιλέα του Ισραήλ, λέγων, Ελθέ, να ίδωμεν αλλήλους προσωπικώς.
\par 18 Και απέστειλεν Ιωάς ο βασιλεύς του Ισραήλ προς τον Αμασίαν βασιλέα του Ιούδα, λέγων, Η άκανθα η εν τω Λιβάνω απέστειλε προς την κέδρον την εν τω Λιβάνω, λέγουσα, Δος την θυγατέρα σου εις τον υιόν μου διά γυναίκα· πλην διέβη θηρίον του αγρού το εν τω Λιβάνω, και κατεπάτησε την άκανθαν.
\par 19 Συ λέγεις, ιδού, επάταξας τον Εδώμ· και η καρδία σου επήρθη εις καύχησιν· κάθου τώρα εν τω οίκω σου· διά τι εμπλέκεσαι εις κακόν, διά το οποίον ήθελες πέσει, συ και ο Ιούδας μετά σου;
\par 20 Αλλ' ο Αμασίας δεν υπήκουσε· διότι εκ Θεού ήτο τούτο, διά να παραδώση αυτούς εις την χείρα των εχθρών, επειδή εξεζήτησαν τους θεούς του Εδώμ.
\par 21 Ανέβη λοιπόν Ιωάς ο βασιλεύς του Ισραήλ· και είδον αλλήλους προσωπικώς, αυτός και Αμασίας ο βασιλεύς του Ιούδα, εν Βαιθ-σεμές, ήτις είναι του Ιούδα.
\par 22 Και εκτυπήθη ο Ιούδας έμπροσθεν του Ισραήλ, και έφυγον έκαστος εις τας σκηνάς αυτού.
\par 23 Και συνέλαβεν Ιωάς ο βασιλεύς του Ισραήλ Αμασίαν τον βασιλέα του Ιούδα, υιόν του Ιωάς υιού του Ιωάχαζ, εν Βαιθ-σεμές, και έφερεν αυτόν εις Ιερουσαλήμ και κατεδάφισε το τείχος της Ιερουσαλήμ από της πύλης Εφραΐμ έως της πύλης της γωνίας, τετρακοσίας πήχας.
\par 24 Και λαβών παν το χρυσίον και το αργύριον και πάντα τα σκεύη τα ευρεθέντα εν τω οίκω του Θεού μετά του Ωβήδ-εδώμ, και τους θησαυρούς του οίκου του βασιλέως, και ανθρώπους ενέχυρα, επέστρεψεν εις Σαμάρειαν.
\par 25 Έζησε δε Αμασίας ο υιός του Ιωάς ο βασιλεύς του Ιούδα, μετά τον θάνατον του Ιωάς υιού του Ιωάχαζ βασιλέως του Ισραήλ, δεκαπέντε έτη.
\par 26 Αι δε λοιπαί πράξεις του Αμασίου, αι πρώται και αι έσχαται, ιδού, δεν είναι γεγραμμέναι εν τω βιβλίω των βασιλέων του Ιούδα και του Ισραήλ;
\par 27 Και ύστερον αφού εστράφη ο Αμασίας από όπισθεν του Κυρίου, έκαμον συνωμοσίαν κατ' αυτού εν Ιερουσαλήμ· και έφυγεν εις Λαχείς· απέστειλαν όμως κατόπιν αυτού εις Λαχείς και εθανάτωσαν αυτόν εκεί.
\par 28 Και έφεραν αυτόν επί ίππων, και έθαψαν αυτόν μετά των πατέρων αυτού εν πόλει Ιούδα.

\chapter{26}

\par 1 Έλαβε δε πας ο λαός του Ιούδα τον Οζίαν, όντα ηλικίας δεκαέξ ετών, και έκαμον αυτόν βασιλέα αντί του πατρός αυτού Αμασίου.
\par 2 Ούτος ωκοδόμησε την Αιλώθ και επέστρεψεν αυτήν εις τον Ιούδαν, αφού ο βασιλεύς εκοιμήθη μετά των πατέρων αυτού.
\par 3 Δεκαέξ ετών ηλικίας ήτο ο Οζίας ότε εβασίλευσε, και εβασίλευσε πεντήκοντα δύο έτη εν Ιερουσαλήμ· το δε όνομα της μητρός αυτού ήτο Ιεχολία εξ Ιερουσαλήμ.
\par 4 Και έπραξε το ευθές ενώπιον Κυρίου, κατά πάντα όσα έπραξεν Αμασίας ο πατήρ αυτού.
\par 5 Και εξεζήτει τον Θεόν εν ταις ημέραις του Ζαχαρίου, του νοήμονος εις τας οράσεις του Θεού· και όσον καιρόν εξεζήτει τον Κύριον, ευώδονεν αυτόν ο Θεός.
\par 6 Και εξήλθε και επολέμησεν εναντίον των Φιλισταίων, και εκρήμνισε το τείχος της Γαθ και το τείχος της Ιαβνή και το τείχος της Αζώτου και ωκοδόμησε πόλεις εν Αζώτω και εν Φιλισταίοις.
\par 7 Και εβοήθησεν αυτόν ο Θεός εναντίον των Φιλισταίων και εναντίον των Αράβων των κατοικούντων εν Γούρ-βαάλ, και των Μεουνείμ.
\par 8 Και έδωκαν οι Αμμωνίται δώρα εις τον Οζίαν· και διεδόθη το όνομα αυτού έως της εισόδου της Αιγύπτου· διότι εκραταιώθη εις άκρον.
\par 9 Και ωκοδόμησεν ο Οζίας πύργους εν Ιερουσαλήμ, επί της πύλης της γωνίας και επί της πύλης της φάραγγος και επί των γωνιών, και ωχύρωσεν αυτούς.
\par 10 Ωικοδόμησεν έτι πύργους εν τη ερήμω και ήνοιξε πολλά φρέατα· διότι είχε κτήνη πολλά και εν τοις χαμηλοίς τόποις και εν ταις πεδιάσι και γεωργούς και αμπελουργούς εν τη ορεινή και εν τω Καρμήλω· διότι ηγάπα την γεωργίαν.
\par 11 Και είχεν ο Οζίας στράτευμα πολεμιστών, εξερχομένων εις πόλεμον κατά τάγματα, κατά τον αριθμόν της απαριθμήσεως αυτών γενομένης υπό Ιεϊήλ του γραμματέως και Μαασία του επιστάτου, υπό την οδηγίαν του Ανανίου, ενός των στρατηγών του βασιλέως.
\par 12 Πας ο αριθμός των αρχηγών των πατριών των δυνατών εν ισχύϊ ήτο δύο χιλιάδες εξακόσιοι.
\par 13 Και υπό την οδηγίαν αυτών ήτο δύναμις πολεμική, τριακόσιαι επτά χιλιάδες και πεντακόσιοι, δυνατοί και ανδρείοι εις τον πόλεμον, διά να βοηθώσι τον βασιλέα εναντίον των εχθρών.
\par 14 Και ητοίμασεν εις αυτούς ο Οζίας, εις άπαν το στράτευμα, θυρεούς και λόγχας και περικεφαλαίας και θώρακας και τόξα και σφενδόνας διά λίθους.
\par 15 Και έκαμεν εν Ιερουσαλήμ μηχανάς, εφευρημένας υπό μηχανικών, διά να ήναι επί των πύργων και επί των γωνιών, ώστε να ρίπτωσι δι' αυτών βέλη και λίθους μεγάλους· και εξήλθε το όνομα αυτού μακράν· διότι εβοηθείτο θαυμασίως, εωσού εκραταιώθη.
\par 16 Αλλ' αφού εκραταιώθη, επήρθη η καρδία αυτού εις διαφθοράν· και ησέβησεν εις Κύριον τον Θεόν αυτού και εισήλθεν εις τον ναόν του Κυρίου διά να θυμιάση επί το θυσιαστήριον του θυμιάματος.
\par 17 Και Αζαρίας ο ιερεύς εισήλθε κατόπιν αυτού, και μετ' αυτού ογδοήκοντα ιερείς του Κυρίου, άνδρες δυνατοί·
\par 18 και αντέστησαν εις τον Οζίαν τον βασιλέα και είπον προς αυτόν, Δεν ανήκει εις σε, Οζία, να θυμιάσης εις τον Κύριον, αλλ' εις τους ιερείς τους υιούς του Ααρών, τους καθιερωμένους να θυμιάζωσιν· έξελθε εκ του αγιαστηρίου· διότι ησέβησας· και τούτο δεν θέλει είσθαι προς δόξαν εις σε παρά Κυρίου του Θεού.
\par 19 Ο δε Οζίας, έχων εν τη χειρί αυτού θυμιατήριον διά να θυμιάση, εθυμώθη· και ενώ εθυμώθη προς τους ιερείς, ανέτειλεν η λέπρα εν τω μετώπω αυτού έμπροσθεν των ιερέων εν τω οίκω του Κυρίου, πλησίον του θυσιαστηρίου του θυμιάματος.
\par 20 Και ανέβλεψεν εις αυτόν Αζαρίας ο ιερεύς ο πρώτος και πάντες οι ιερείς, και ιδού, ήτο λεπρός κατά το μέτωπον αυτού· και έσπευσαν να εκβάλωσιν αυτόν εκείθεν· και αυτός μάλιστα έσπευσε να εξέλθη, διότι επάταξεν αυτόν ο Κύριος.
\par 21 Και ήτο ο Οζίας ο βασιλεύς λεπρός έως της ημέρας του θανάτου αυτού· και κατώκει εν οίκω κεχωρισμένω λεπρός· διότι απεκόπη από του οίκου του Κυρίου· ήτο δε επί του οίκου του βασιλέως Ιωθάμ ο υιός αυτού, κρίνων τον λαόν της γης.
\par 22 Αι δε λοιπαί πράξεις του Οζίου, αι πρώται και αι έσχαται, εγράφησαν υπό Ησαΐου του προφήτου υιού του Αμώς.
\par 23 Και εκοιμήθη ο Οζίας μετά των πατέρων αυτού, και έθαψαν αυτόν μετά των πατέρων αυτού εν τω πεδίω της ταφής των βασιλέων· διότι είπον, Είναι λεπρός. Και εβασίλευσεν αντ' αυτού Ιωθάμ ο υιός αυτού.

\chapter{27}

\par 1 Εικοσιπέντε ετών ηλικίας ήτο ο Ιωθάμ ότε εβασίλευσε· και εβασίλευσε δεκαέξ έτη εν Ιερουσαλήμ· το δε όνομα της μητρός αυτού ήτο Ιερουσά, θυγάτηρ του Σαδώκ.
\par 2 Και έπραξε το ευθές ενώπιον Κυρίου, κατά πάντα όσα έπραξεν Οζίας ο πατήρ αυτού· δεν εισήλθεν όμως εις τον ναόν του Κυρίου. Και ο λαός ήτο έτι διεφθαρμένος.
\par 3 Ούτος ωκοδόμησε την υψηλήν πύλην του οίκου του Κυρίου· και επί του τείχους του Οφήλ ωκοδόμησε πολλά.
\par 4 Ωικοδόμησεν έτι πόλεις εν τη ορεινή του Ιούδα, και εν τοις δρυμοίς ωκοδόμησε φρούρια και πύργους.
\par 5 Και πολεμήσας με τον βασιλέα των υιών Αμμών, υπερίσχυσεν εναντίον αυτών. Και κατ' εκείνον τον ενιαυτόν οι υιοί Αμμών έδωκαν εις αυτόν εκατόν τάλαντα αργυρίου και δέκα χιλιάδας κόρων σίτου και δέκα χιλιάδας κριθής. Τόσα επλήρωσαν εις αυτόν οι υιοί Αμμών και το δεύτερον έτος και το τρίτον.
\par 6 Και εκραταιώθη ο Ιωθάμ, επειδή κατεύθυνε τας οδούς αυτού ενώπιον Κυρίου του Θεού αυτού.
\par 7 Αι δε λοιπαί πράξεις του Ιωθάμ και πάντες οι πόλεμοι αυτού, και αι οδοί αυτού, ιδού, είναι γεγραμμένοι εν τω βιβλίω των βασιλέων του Ισραήλ και Ιούδα.
\par 8 Εικοσιπέντε ετών ηλικίας ήτο ότε εβασίλευσε, και εβασίλευσε δεκαέξ έτη εν Ιερουσαλήμ.
\par 9 Και εκοιμήθη ο Ιωθάμ μετά των πατέρων αυτού, και έθαψαν αυτόν εν πόλει Δαβίδ· εβασίλευσε δε αντ' αυτού Άχαζ ο υιός αυτού.

\chapter{28}

\par 1 Είκοσι ετών ηλικίας ήτο ο Άχαζ ότε εβασίλευσε, και εβασίλευσε δεκαέξ έτη εν Ιερουσαλήμ· δεν έπραξεν όμως το ευθές ενώπιον Κυρίου, ως ο Δαβίδ ο πατήρ αυτού·
\par 2 αλλά περιεπάτησεν εν ταις οδοίς των βασιλέων του Ισραήλ και έκαμεν έτι είδωλα χωνευτά εις τους Βααλείμ.
\par 3 Και αυτός εθυμίασεν εν τη κοιλάδι του υιού Εννόμ και διεβίβασε τα τέκνα αυτού διά του πυρός, κατά τα βδελύγματα των εθνών τα οποία εξεδίωξεν ο Κύριος απ' έμπροσθεν των υιών Ισραήλ.
\par 4 Και εθυσίαζε και εθυμίαζεν επί τους υψηλούς τόπους και επί τους λόφους και υποκάτω παντός δένδρου πρασίνου.
\par 5 Διά τούτο παρέδωκεν αυτόν Κύριος ο Θεός αυτού εις την χείρα του βασιλέως της Συρίας· και επάταξαν αυτόν, και έλαβον αιχμαλώτους μέγα πλήθος εξ αυτών και έφεραν αυτούς εις Δαμασκόν. Και παρεδόθη έτι εις την χείρα του βασιλέως του Ισραήλ, όστις επάταξεν αυτόν εν σφαγή μεγάλη.
\par 6 Διότι Φεκά ο υιός του Ρεμαλία εθανάτωσεν εκ του Ιούδα εκατόν είκοσι χιλιάδας εν μιά ημέρα, πάντας δυνατούς εν ισχύϊ, επειδή εγκατέλιπον Κύριον τον Θεόν των πατέρων αυτών.
\par 7 Και Ζιχρί, ανήρ δυνατός εκ του Εφραΐμ, εθανάτωσε Μαασίαν τον υιόν του βασιλέως και Αζρικάμ τον επιστάτην του παλατίου και Ελκανά τον δεύτερον μετά τον βασιλέα.
\par 8 Και ηχμαλώτισαν οι υιοί Ισραήλ εκ των αδελφών αυτών διακοσίας χιλιάδας, γυναίκας, υιούς και θυγατέρας, και έλαβον έτι λάφυρα πολλά εξ αυτών και έφεραν τα λάφυρα εις Σαμάρειαν.
\par 9 Ήτο δε εκεί προφήτης του Κυρίου, ονομαζόμενος Ωδήδ· και εξήλθεν εις απάντησιν του στρατεύματος του ερχομένου εις Σαμάρειαν και είπε προς αυτούς, Ιδού, επειδή Κύριος ο Θεός των πατέρων σας εθυμώθη κατά του Ιούδα, παρέδωκεν αυτούς εις την χείρα σας· και σεις εθανατώσατε αυτούς εν μανία, ήτις έφθασεν έως του ουρανού·
\par 10 και τώρα λέγετε να υποτάξητε εις εαυτούς τους υιούς Ιούδα και της Ιερουσαλήμ διά δούλους και δούλας· δεν είναι με σας, με σας μάλιστα, αμαρτίαι εναντίον Κυρίου του Θεού σας;
\par 11 τώρα λοιπόν ακούσατέ μου και επιστρέψατε τους αιχμαλώτους, τους οποίους ηχμαλωτίσατε εκ των αδελφών σας· διότι οργή θυμού Κυρίου επίκειται εις εσάς.
\par 12 Και εσηκώθησάν τινές εκ των αρχόντων των υιών Εφραΐμ, Αζαρίας ο υιός του Ιωανάν, Βαραχίας ο υιός του Μεσιλλεμώθ και Εζεκίας ο υιός του Σαλλούμ και Αμασά ο υιός του Αδλαΐ εναντίον των ερχομένων από του πολέμου,
\par 13 και είπον προς αυτούς, Δεν θέλετε εισάξει εδώ τους αιχμαλώτους· διότι ενώ ηνομήσαμεν εις Κύριον, θέλετε να προσθέσητε εις τας αμαρτίας ημών και εις τας ανομίας ημών· διότι μεγάλη είναι ανομία ημών, και οργή θυμού επίκειται επί τον Ισραήλ.
\par 14 Και αφήκαν οι πολεμισταί τους αιχμαλώτους και τα λάφυρα ενώπιον των αρχόντων και πάσης της συνάξεως.
\par 15 Και σηκωθέντες οι άνδρες, οι ονομασθέντες κατ' όνομα, έλαβον τους αιχμαλώτους και πάντας τους γυμνούς αυτών ενέδυσαν εκ των λαφύρων· και αφού ενέδυσαν αυτούς και υπεδημάτωσαν αυτούς και έδωκαν εις αυτούς να φάγωσι και να πίωσι και ήλειψαν αυτούς, και πάντας τους αδυνάτους εξ αυτών μετεκόμισαν επί όνους και έφεραν αυτούς εις Ιεριχώ, την πόλιν των φοινίκων, προς τους αδελφούς αυτών· και επέστρεψαν εις Σαμάρειαν.
\par 16 Κατά τον καιρόν εκείνον ο βασιλεύς Άχαζ απέστειλε προς τους βασιλείς της Ασσυρίας, διά να βοηθήσωσιν αυτόν.
\par 17 Διότι ελθόντες πάλιν οι Ιδουμαίοι επάταξαν τον Ιούδαν και έλαβον αιχμαλώτους.
\par 18 Και εφορμήσαντες οι Φιλισταίοι εις τας πόλεις της πεδινής και της μεσημβρινής του Ιούδα· εκυρίευσαν την Βαιθ-σεμές και την Αιαλών και την Γεδηρώθ, και την Σοκχώ και τας κώμας αυτής, και την Θαμνά και τας κώμας αυτής, και την Γιμζώ και ταις κώμας αυτής· και κατώκησαν εκεί.
\par 19 Διότι ο Κύριος εταπείνωσε τον Ιούδαν διά τον Άχαζ βασιλέα του Ισραήλ· επειδή διέφθειρε τον Ιούδαν και ησέβησε σφόδρα εις τον Κύριον.
\par 20 Και ήλθε προς αυτόν ο Θελγάθ-φελνασάρ, βασιλεύς της Ασσυρίας, και κατέθλιψεν αυτόν αντί να ενδυναμώση αυτόν.
\par 21 Διότι ο Άχαζ, λαβών τους θησαυρούς του οίκου του Κυρίου και του οίκου του βασιλέως και των αρχόντων, έδωκεν εις τον βασιλέα της Ασσυρίας· πλην ουχί εις βοήθειαν αυτού.
\par 22 Και εν τω καιρώ της στενοχωρίας αυτού έτι μάλλον παρηνόμησεν εις τον Κύριον αυτός ο βασιλεύς Άχαζ.
\par 23 Και εθυσίαζεν εις τους θεούς της Δαμασκού, τους πατάξαντας αυτόν· και έλεγεν, Επειδή οι θεοί του βασιλέως της Συρίας βοηθούσιν αυτούς, εις τούτους θέλω θυσιάσει, διά να βοηθήσωσι και εμέ. Εκείνοι όμως εστάθησαν η φθορά αυτού και παντός του Ισραήλ.
\par 24 Και συνήθροισεν ο Άχαζ τα σκεύη του οίκου του Θεού, και κατέκοψε τα σκεύη του οίκου του Θεού και έκλεισε τας θύρας του οίκου του Κυρίου, και έκαμεν εις εαυτόν θυσιαστήρια εν πάση γωνία εν Ιερουσαλήμ.
\par 25 Και εν πάση πόλει του Ιούδα έκαμεν υψηλούς τόπους, διά να θυμιάζη εις άλλους θεούς, και παρώργισε Κύριον τον Θεόν των πατέρων αυτού.
\par 26 Αι δε λοιπαί πράξεις αυτού και πάσαι αι οδοί αυτού, αι πρώται και αι έσχαται, ιδού, είναι γεγραμμέναι εν τω βιβλίω των βασιλέων του Ιούδα και Ισραήλ.
\par 27 Και εκοιμήθη ο Άχαζ μετά των πατέρων αυτού, και έθαψαν αυτόν εν τη πόλει, εν Ιερουσαλήμ· δεν έφεραν όμως αυτόν εις τους τάφους των βασιλέων του Ισραήλ· εβασίλευσε δε αντ' αυτού Εζεκίας ο υιός αυτού.

\chapter{29}

\par 1 Ο Εζεκίας εβασίλευσεν ηλικίας εικοσιπέντε ετών, και εβασίλευσεν εικοσιεννέα έτη εν Ιερουσαλήμ· το δε όνομα της μητρός αυτού ήτο Αβιά, θυγάτηρ του Ζαχαρίου.
\par 2 Και έπραξε το ευθές ενώπιον Κυρίου, κατά πάντα όσα έπραξε Δαβίδ ο πατήρ αυτού.
\par 3 Ούτος εν τω πρώτω έτει της βασιλείας αυτού, τον πρώτον μήνα, ήνοιξε τας θύρας του οίκου του Κυρίου και επεσκεύασεν αυτάς.
\par 4 Και εισήγαγε τους ιερείς και τους Λευΐτας, και συνήγαγεν αυτούς εις την ανατολικήν πλατείαν,
\par 5 και είπε προς αυτούς, Ακούσατέ μου, Λευΐται· Αγιάσθητε τώρα, και αγιάσατε τον ναόν Κυρίου του Θεού των πατέρων σας και εκβάλετε την ακαθαρσίαν εκ του αγίου τόπου.
\par 6 Διότι οι πατέρες ημών παρηνόμησαν και έπραξαν πονηρά ενώπιον Κυρίου του Θεού ημών και εγκατέλιπον αυτόν, και απέστρεψαν τα πρόσωπα αυτών από του κατοικητηρίου του Κυρίου και έστρεψαν τα νώτα·
\par 7 και έκλεισαν τας θύρας του προνάου και έσβεσαν τους λύχνους και θυμίαμα δεν εθυμίαζον και ολοκαυτώματα δεν προσέφερον εις τον Θεόν του Ισραήλ εν τω αγίω τόπω.
\par 8 Διά τούτο η οργή του Κυρίου ήλθεν επί τον Ιούδαν και την Ιερουσαλήμ, και παρέδωκεν αυτούς εις διασποράν, εις έκστασιν και εις συριγμόν, καθώς βλέπετε με τους οφθαλμούς σας.
\par 9 Διότι ιδού, οι πατέρες ημών έπεσον διά μαχαίρας· και οι υιοί ημών και αι θυγατέρες ημών και αι γυναίκες ημών είναι διά τούτο εις αιχμαλωσίαν.
\par 10 Τώρα λοιπόν έχω εν τη καρδία μου να κάμω διαθήκην προς τον Κύριον τον Θεόν του Ισραήλ, διά να αποστρέψη την οργήν του θυμού αυτού αφ' ημών.
\par 11 Τέκνα μου, μη πλανάσθε τώρα· διότι ο Κύριος σας εξέλεξε διά να παρίστασθε ενώπιον αυτού, να υπηρετήτε αυτόν και να ήσθε λειτουργοί αυτού και να θυμιάζητε.
\par 12 Τότε εσηκώθησαν οι Λευΐται, Μαάθ ο υιός του Αμασαΐ και Ιωήλ ο υιός του Αζαρίου, εκ των υιών των Κααθιτών· εκ δε των υιών του Μεραρί, Κείς ο υιός του Αβδί και Αζαρίας ο υιός του Ιαλελεήλ· και εκ των Γηρσωνιτών, Ιωάχ ο υιός του Ζιμμά και Εδέν ο υιός του Ιωάχ·
\par 13 και εκ των υιών του Ελισαφάν, Σιμρί και Ιεϊήλ· και εκ των υιών του Ασάφ, Ζαχαρίας και Ματθανίας·
\par 14 και εκ των υιών του Αιμάν, Ιεχιήλ και Σιμεΐ· και εκ των υιών του Ιεδουθούν, Σεμαΐας και Οζιήλ.
\par 15 Και συνήγαγον τους αδελφούς αυτών και ηγιάσθησαν και ήλθον, ως προσέταξεν ο βασιλεύς διά του λόγου του Κυρίου, να καθαρίσωσι τον οίκον του Κυρίου.
\par 16 Και εισήλθον οι ιερείς εις το ενδότερον του οίκου του Κυρίου, διά να καθαρίσωσιν αυτόν· και εξέβαλον πάσαν την ακαθαρσίαν την ευρεθείσαν εν τω ναώ του Κυρίου, εν τη αυλή του οίκου του Κυρίου· οι δε Λευΐται λαβόντες, έφεραν έξω εις τον χείμαρρον Κέδρων.
\par 17 Και ήρχισαν να αγιάζωσι τη πρώτη του μηνός του πρώτου, και τη ογδόη ημέρα του μηνός εισήλθον εις τον πρόναον του Κυρίου. Και ηγίασαν τον οίκον του Κυρίου εν οκτώ ημέραις, και τη δεκάτη έκτη του μηνός του πρώτου ετελείωσαν.
\par 18 Τότε εισήλθον προς Εζεκίαν τον βασιλέα και είπον, Εκαθαρίσαμεν όλον τον οίκον του Κυρίου και το θυσιαστήριον της ολοκαυτώσεως και πάντα τα σκεύη αυτού και την τράπεζαν της προθέσεως και πάντα τα σκεύη αυτής·
\par 19 και πάντα τα σκεύη, τα οποία εμίανεν ο βασιλεύς Άχαζ επί της βασιλείας αυτού, ότε απεστάτησεν, ητοιμάσαμεν και ηγιάσαμεν· και ιδού, είναι έμπροσθεν του θυσιαστηρίου του Κυρίου.
\par 20 Τότε εξηγέρθη Εζεκίας ο βασιλεύς, και συναγαγών τους άρχοντας της πόλεως, ανέβη προς τον οίκον του Κυρίου.
\par 21 Και έφεραν επτά μόσχους και επτά κριούς και επτά αρνία και επτά τράγους διά προσφοράν περί αμαρτίας υπέρ της βασιλείας και υπέρ του αγιαστηρίου και υπέρ του Ιούδα. Και είπε προς τους ιερείς, τους υιούς Ααρών, να προσφέρωσιν αυτά επί του θυσιαστηρίου του Κυρίου.
\par 22 Και έσφαξαν τους μόσχους· και παραλαβόντες οι ιερείς το αίμα, ερράντισαν επί το θυσιαστήριον· ομοίως έσφαξαν τους κριούς και ερράντισαν το αίμα επί το θυσιαστήριον· και έσφαξαν τα αρνία και ερράντισαν το αίμα επί το θυσιαστήριον.
\par 23 Έπειτα προσήγαγον τους τράγους, διά την περί αμαρτίας προσφοράν, έμπροσθεν του βασιλέως και της συνάξεως, οι δε επέθεσαν τας χείρας αυτών επ' αυτούς·
\par 24 και έσφαξαν αυτούς οι ιερείς και ερράντισαν το αίμα αυτών περί αμαρτίας επί το θυσιαστήριον, διά να κάμωσιν εξιλέωσιν υπέρ παντός του Ισραήλ· διότι ο βασιλεύς προσέταξε το ολοκαύτωμα και την περί αμαρτίας προσφοράν υπέρ παντός του Ισραήλ.
\par 25 Και έστησε τους Λευΐτας εν τω οίκω του Κυρίου εν κυμβάλοις, εν ψαλτηρίοις και εν κιθάραις, κατά το πρόσταγμα του Δαβίδ και Γαδ του βλέποντος του βασιλέως και Νάθαν του προφήτου· διότι το πρόσταγμα ήτο παρά Κυρίου διά των προφητών αυτού.
\par 26 Και εστάθησαν οι Λευΐται με τα όργανα του Δαβίδ και οι ιερείς με τας σάλπιγγας.
\par 27 Και είπεν ο Εζεκίας να προσφέρωσι την ολοκαύτωσιν επί του θυσιαστηρίου. Και ότε ήρχισεν η ολοκαύτωσις, ήρχισεν ο ύμνος του Κυρίου, με τας σάλπιγγας και με τα όργανα τα διωρισμένα παρά του Δαβίδ βασιλέως του Ισραήλ.
\par 28 Και προσεκύνει πάσα η σύναξις και έψαλλον οι ψαλτωδοί και οι σαλπιγκταί εσάλπιζον· όλον τούτο εξηκολούθει εωσού ετελείωσεν η ολοκαύτωσις.
\par 29 Και ως ετελείωσαν προσφέροντες, έκλιναν ο βασιλεύς και πάντες οι ευρεθέντες μετ' αυτού και προσεκύνησαν.
\par 30 Και είπε προς τους Λευΐτας Εζεκίας ο βασιλεύς και οι άρχοντες να υμνώσι τον Κύριον με τους λόγους του Δαβίδ και Ασάφ του βλέποντος. Και ύμνησαν εν ευφροσύνη και κύψαντες προσεκύνησαν.
\par 31 Τότε αποκριθείς ο Εζεκίας είπε, Τώρα είσθε καθιερωμένοι εις τον Κύριον· προσέλθετε και προσφέρετε θυσίας και ευχαριστηρίους προσφοράς εν τω οίκω του Κυρίου. Και προσέφερεν η σύναξις θυσίας και ευχαριστηρίους προσφοράς, και πας πρόθυμος την καρδίαν, ολοκαυτώματα.
\par 32 Και έγεινεν ο αριθμός των ολοκαυτωμάτων, τα οποία προσέφερεν η σύναξις, εβδομήκοντα μόσχοι, εκατόν κριοί, διακόσια αρνία· πάντα ταύτα ήσαν διά ολοκαύτωσιν προς τον Κύριον.
\par 33 Τα δε αφιερώματα ήσαν εξακόσιοι βόες και τρισχίλια πρόβατα.
\par 34 Οι ιερείς όμως ήσαν ολίγοι και δεν ηδύναντο να εκδέρωσι πάντα τα ολοκαυτώματα· όθεν οι αδελφοί αυτών οι Λευΐται εβοήθησαν αυτούς, εωσού η εργασία συνετελέσθη και εωσού ηγιάσθησαν οι ιερείς· διότι οι Λευΐται εστάθησαν ευθύτεροι την καρδίαν εις το να αγιασθώσι, παρά οι ιερείς.
\par 35 Έτι δε τα ολοκαυτώματα ήσαν πολλά, μετά των στεάτων των ειρηνικών προσφορών, και μετά των σπονδών δι' έκαστον ολοκαύτωμα. Ούτως αποκατεστάθη η υπηρεσία του οίκου του Κυρίου.
\par 36 Και ευφράνθη ο Εζεκίας και πας ο λαός, ότι ο Θεός προδιέθεσε τον λαόν· επειδή το πράγμα έγεινεν αιφνιδίως.

\chapter{30}

\par 1 Και απέστειλεν ο Εζεκίας προς πάντα τον Ισραήλ και Ιούδαν· έγραψεν έτι επιστολάς προς Εφραΐμ και Μανασσή, διά να έλθωσιν εις τον οίκον του Κυρίου εν Ιερουσαλήμ, να κάμωσι πάσχα εις Κύριον τον Θεόν του Ισραήλ.
\par 2 Διότι συνεβουλεύθη ο βασιλεύς και οι άρχοντες αυτού και πάσα η σύναξις εν Ιερουσαλήμ να κάμωσι το πάσχα εν τω δευτέρω μηνί.
\par 3 Επειδή δεν ηδυνήθησαν να κάμωσιν αυτό εν τω καιρώ εκείνω, διότι οι ιερείς δεν ήσαν αρκετά ηγιασμένοι και ο λαός δεν ήτο συνηγμένος εν Ιερουσαλήμ.
\par 4 Και ήρεσε το πράγμα εις τον βασιλέα και εις πάσαν την σύναξιν.
\par 5 Όθεν απεφάσισαν να διακηρύξωσι διά παντός του Ισραήλ, από Βηρ-σαβεέ έως Δαν, να έλθωσι διά να κάμωσι πάσχα εις Κύριον τον Θεόν του Ισραήλ εν Ιερουσαλήμ· διότι από πολλού χρόνου δεν είχον κάμει κατά το γεγραμμένον.
\par 6 Και υπήγαν οι ταχυδρόμοι μετά των επιστολών παρά του βασιλέως και των αρχόντων αυτού, διά παντός του Ισραήλ και Ιούδα, και κατά την προσταγήν του βασιλέως, λέγοντες, υιοί Ισραήλ, επιστρέψατε προς Κύριον τον Θεόν του Αβραάμ, Ισαάκ και Ισραήλ· και αυτός θέλει επιστρέψει εις τους εναπολειφθέντας από σας, όσοι διεσώθητε εκ χειρός των βασιλέων της Ασσυρίας·
\par 7 και μη γίνεσθε καθώς οι πατέρες σας και καθώς οι αδελφοί σας, οίτινες ησέβησαν εις Κύριον τον Θεόν των πατέρων αυτών· και παρέδωκεν αυτούς εις ερήμωσιν, ως βλέπετε·
\par 8 τώρα μη σκληρύνητε τον τράχηλόν σας, καθώς οι πατέρες σας· υποτάχθητε εις τον Κύριον και εισέλθετε εις το αγιαστήριον αυτού, το οποίον ηγίασεν εις τον αιώνα· και δουλεύσατε Κύριον τον Θεόν σας, διά να αποστρέψη την έξαψιν του θυμού αυτού αφ' υμών·
\par 9 διότι εάν επιστρέψητε προς τον Κύριον, οι αδελφοί σας και τα τέκνα σας θέλουσιν ευρεί έλεος έμπροσθεν των αιχμαλωτισάντων αυτούς, και θέλουσιν επανέλθει εις την γην ταύτην· διότι οικτίρμων και ελεήμων είναι Κύριος ο Θεός σας και δεν θέλει αποστρέψει το πρόσωπον αυτού από σας, εάν επιστρέψητε προς αυτόν.
\par 10 Και διήλθον οι ταχυδρόμοι από πόλεως εις πόλιν διά της γης του Εφραΐμ και Μανασσή και έως Ζαβουλών· πλην εκείνοι κατεγέλασαν αυτούς και εμυκτήρισαν αυτούς.
\par 11 Τινές όμως εκ του Ασήρ και Μανασσή και Ζαβουλών υπέκλιναν και ήλθον εις Ιερουσαλήμ.
\par 12 Και επί Ιούδαν ήτο χειρ Θεού, ώστε να δώση εις αυτούς καρδίαν μίαν, διά να κάμωσι την προσταγήν του βασιλέως και των αρχόντων, κατά τον λόγον του Κυρίου.
\par 13 Και συνήχθησαν εις Ιερουσαλήμ λαός πολύς, διά να κάμωσι την εορτήν των αζύμων εν τω μηνί τω δευτέρω, σύναξις μεγάλη σφόδρα.
\par 14 Και σηκωθέντες, αφήρεσαν τα θυσιαστήρια τα εν Ιερουσαλήμ· και πάντα τα θυσιαστήρια του θυμιάματος αφήρεσαν και έρριψαν αυτά εις τον χείμαρρον Κέδρων.
\par 15 Και εθυσίασαν το πάσχα τη δεκάτη τετάρτη του δευτέρου μηνός· και εντράπησαν οι ιερείς και οι Λευΐται, και αγιασθέντες εισέφεραν ολοκαυτώματα εις τον οίκον του Κυρίου.
\par 16 Και εστάθησαν εν τω τόπω αυτών, κατά την τάξιν αυτών, κατά τον νόμον Μωϋσέως του ανθρώπου του Θεού· και ερράντιζον οι ιερείς το αίμα, λαμβάνοντες εκ της χειρός των Λευϊτών.
\par 17 Διότι ήσαν πολλοί εν τη συνάξει, οι μη αγιασθέντες· διά τούτο έλαβον οι Λευΐται το φορτίον να σφάξωσι τα αρνία του πάσχα διά πάντα τον μη καθαρόν, διά να αγιάσωσιν αυτούς εις τον Κύριον.
\par 18 Επειδή μέγα μέρος εκ του λαού, πολλοί εκ του Εφραΐμ και Μανασσή, Ισσάχαρ και Ζαβουλών δεν είχον καθαρισθή, αλλ' έτρωγον το πάσχα ουχί κατά το γεγραμμένον· ο Εζεκίας όμως εδεήθη υπέρ αυτών, λέγων, Ο αγαθός Κύριος ας γείνη ίλεως εις πάντα,
\par 19 όστις κατευθύνει την καρδίαν αυτού εις το να εκζητή τον Θεόν, Κύριον τον Θεόν των πατέρων αυτού, και αν δεν εκαθαρίσθη κατά τον καθαρισμόν του αγιαστηρίου.
\par 20 Και επήκουσεν ο Κύριος του Εζεκίου και συνεχώρησε τον λαόν.
\par 21 Και έκαμον οι υιοί Ισραήλ οι ευρεθέντες εν Ιερουσαλήμ την εορτήν των αζύμων επτά ημέρας εν ευφροσύνη μεγάλη· και ύμνουν οι Λευΐται και οι ιερείς τον Κύριον καθ' εκάστην ημέραν, τον Κύριον, με όργανα δυνατά.
\par 22 Και ελάλησεν ο Εζεκίας κατά την καρδίαν πάντων των Λευϊτών των εχόντων σύνεσιν αγαθήν περί του Κυρίου· και έτρωγον εν τη εορτή επτά ημέρας, θυσιάζοντες θυσίας ειρηνικάς και δοξολογούντες Κύριον τον Θεόν των πατέρων αυτών.
\par 23 Και συνεβουλεύθη πάσα η σύναξις να κάμωσιν άλλας επτά ημέρας· και έκαμον άλλας επτά ημέρας ευφροσύνην.
\par 24 Διότι προσέφερεν Εζεκίας, ο βασιλεύς του Ιούδα, εις την σύναξιν χιλίους βόας και επτά χιλιάδας προβάτων· και οι άρχοντες προσέφεραν εις την σύναξιν χιλίους βόας και δέκα χιλιάδας προβάτων· και ηγιάσθησαν πολλοί ιερείς.
\par 25 Και ευφράνθησαν πάσα η σύναξις του Ιούδα και οι ιερείς και οι Λευΐται και πάσα η σύναξις η συνελθούσα εκ του Ισραήλ και οι ξένοι οι ελθόντες εκ της γης του Ισραήλ και οι κατοικούντες εν Ιούδα.
\par 26 Και έγεινεν ευφροσύνη μεγάλη εν Ιερουσαλήμ· διότι από των ημερών του Σολομώντος υιού του Δαβίδ βασιλέως του Ισραήλ, δεν έγεινε τοιούτον πράγμα εν Ιερουσαλήμ.
\par 27 Μετά ταύτα σηκωθέντες οι ιερείς οι Λευΐται ηυλόγησαν τον λαόν· και επηκούσθη η φωνή αυτών, και ήλθεν η προσευχή αυτών εις τον ουρανόν, το άγιον κατοικητήριον του Κυρίου.

\chapter{31}

\par 1 Αφού δε συνετελέσθησαν πάντα ταύτα, πας ο Ισραήλ οι ευρεθέντες εξήλθον εις τας πόλεις του Ιούδα και συνέτριψαν τα αγάλματα και κατέκοψαν τα άλση και εκρήμνισαν τους υψηλούς τόπους και τα θυσιαστήρια από παντός του Ιούδα και Βενιαμίν· το αυτό έκαμον και εις τον Εφραΐμ και Μανασσή, εωσού συνετέλεσαν. Τότε επέστρεψαν πάντες οι υιοί Ισραήλ, έκαστος εις την ιδιοκτησίαν αυτού, εις τας εαυτών πόλεις.
\par 2 Και διέταξεν ο Εζεκίας τας διαιρέσεις των ιερέων και των Λευϊτών, κατά τας διαιρέσεις αυτών, έκαστον κατά την υπηρεσίαν αυτού, τους ιερείς και τους Λευΐτας, διά τα ολοκαυτώματα και τας ειρηνικάς προσφοράς, διά να λειτουργώσι και να δοξολογώσι και να υμνώσιν εν ταις πύλαις των σκηνωμάτων του Κυρίου.
\par 3 Διέταξε και το μερίδιον του βασιλέως εκ των υπαρχόντων αυτού, διά τας ολοκαυτώσεις, διά τας πρωϊνάς και εσπερινάς ολοκαυτώσεις και διά τας ολοκαυτώσεις των σαββάτων και των νεομηνιών και των επισήμων εορτών, κατά το γεγραμμένον εν τω νόμω του Κυρίου.
\par 4 Είπεν έτι προς τον λαόν τον κατοικούντα εν Ιερουσαλήμ, να δίδη την μερίδα των ιερέων και Λευϊτών, διά να ενισχύωνται εν τω νόμω του Κυρίου.
\par 5 Και καθώς εξεδόθη ο λόγος, οι υιοί Ισραήλ έφεραν εν αφθονία απαρχάς σίτου, οίνου και ελαίου και μέλιτος και πάντων των γεννημάτων του αγρού· έφεραν έτι εν αφθονία τα δέκατα παντός πράγματος.
\par 6 Και οι υιοί Ισραήλ και Ιούδα, οι κατοικούντες εν τους πόλεσιν Ιούδα, και αυτοί έφεραν τα δέκατα βοών και προβάτων και τα δέκατα των αγίων πραγμάτων των αφιερουμένων εις Κύριον τον Θεόν αυτών, και έθεσαν κατά σωρούς.
\par 7 Εν τω τρίτω μηνί ήρχισαν να κάμνωσι τους σωρούς και εν τω εβδόμω μηνί ετελείωσαν.
\par 8 και ότε ήλθον ο Εζεκίας και οι άρχοντες και είδον τους σωρούς, ηυλόγησαν τον Κύριον και τον λαόν αυτού τον Ισραήλ.
\par 9 Έπειτα ηρώτησεν ο Εζεκίας τους ιερείς και τους Λευΐτας περί των σωρών.
\par 10 Και απεκρίθη προς αυτόν Αζαρίας, ο πρώτος ιερεύς, εκ του οίκου Σαδώκ, και είπεν, Αφού ήρχισαν να φέρωσι τας προσφοράς εις τον οίκον του Κυρίου, εφάγομεν εις χορτασμόν, και επερίσσευσε πλήθος· διότι ο Κύριος ηυλόγησε τον λαόν αυτού· και το εναπολειφθέν είναι η μεγάλη αύτη αφθονία.
\par 11 Τότε είπεν ο Εζεκίας να ετοιμάσωσι ταμεία εκ τω οίκω του Κυρίου· και ητοίμασαν,
\par 12 και εισέφεραν εν πίστει τας προσφοράς και τα δέκατα και τα αφιερώματα· επ' αυτών δε ήτο επιστάτης Χωνανίας ο Λευΐτης και μετ' αυτόν Σιμεΐ ο αδελφός αυτού.
\par 13 Ο δε Ιεχιήλ και Αζαζίας και Ναχάθ και Ασαήλ και Ιεριμώθ και Ιωζαβάδ και Ελιήλ και Ισμαχίας και Μαάθ και Βεναΐας ήσαν επιτηρηταί, υπό την οδηγίαν του Χωνανίου και Σιμεΐ του αδελφού αυτού, διά προσταγής Εζεκίου του βασιλέως και Αζαρίου του επιστάτου του οίκου του Θεού.
\par 14 Και Κωρή ο υιός του Ιεμνά του Λευΐτου, ο πυλωρός κατά ανατολάς, ήτο επί των προαιρετικών προσφορών του Θεού, διά να διανέμη τας προσφοράς του Κυρίου και τα αγιώτατα πράγματα.
\par 15 Και μετ' αυτόν Εδέν και Μινιαμείν και Ιησοής και Σεμαΐας, Αμαρίας και Σεχανίας, εν ταις πόλεσι των ιερέων εμπεπιστευμένοι να διανέμωσιν εις τους αδελφούς αυτών κατά τας διαιρέσεις αυτών, εξ ίσου εις τον μεγάλον και εις τον μικρόν,
\par 16 εις πάντα τον εισερχόμενον εις τον οίκον του Κυρίου, το καθημερινόν αυτού μερίδιον, διά την εις τα υπουργήματα αυτών υπηρεσίαν αυτών, κατά τας διαιρέσεις αυτών, εκτός των αρσενικών αυτών, τα οποία απηριθμήθησαν κατά γενεαλογίαν, από τριών ετών ηλικίας και επάνω·
\par 17 η απαρίθμησις δε των ιερέων και των Λευϊτών έγεινε κατά τον οίκον των πατριών αυτών, από είκοσι ετών ηλικίας και επάνω, κατά τα υπουργήματα αυτών, κατά τας διαιρέσεις αυτών·
\par 18 και εις πάντα τα τέκνα αυτών, τας γυναίκας αυτών και τους υιούς αυτών και τας θυγατέρας αυτών, εις πάσαν την σύναξιν, οίτινες απηριθμήθησαν κατά γενεαλογίαν· διότι εν πίστει ηγιάσθησαν εις τα άγια.
\par 19 Περί δε των υιών Ααρών των ιερέων εν τοις αγροίς των προαστείων των πόλεων αυτών, ήσαν εν εκάστη πόλει άνθρωποι διωρισμένοι κατ' όνομα διά να δίδωσι μερίδια εις πάντα τα αρσενικά μεταξύ των ιερέων και εις πάντα τα απαριθμηθέντα μεταξύ των Λευϊτών.
\par 20 Και ούτως έκαμεν ο Εζεκίας καθ' όλον τον Ιούδαν· και έπραξε το καλόν και ευθές και αληθινόν ενώπιον Κυρίου του Θεού αυτού.
\par 21 Και εις παν έργον, το οποίον ήρχισεν εις την υπηρεσίαν του οίκου του Θεού, και εις τον νόμον και εις τα προστάγματα, εκζητών τον Θεόν αυτού, έκαμνεν αυτό εξ όλης της καρδίας αυτού και ευωδούτο.

\chapter{32}

\par 1 Μετά τα πράγματα ταύτα και την αλήθειαν ταύτην, ήλθε Σενναχειρείμ ο βασιλεύς της Ασσυρίας, και εισήλθεν εις τον Ιούδαν και εστρατοπέδευσεν εναντίον των οχυρών πόλεων και είπε να υποτάξη αυτάς εις εαυτόν.
\par 2 Και ιδών ο Εζεκίας, ότι ο Σενναχειρείμ ήλθε και ο σκοπός αυτού ήτο να πολεμήση εναντίον της Ιερουσαλήμ,
\par 3 συνεβουλεύθη μετά των αρχόντων αυτού και μετά των δυνατών αυτού, να εμφράξη τα ύδατα των πηγών των έξω της πόλεως· και συνήργησαν μετ' αυτού.
\par 4 Και συνήχθη λαός πολύς, και ενέφραξαν πάσας τας πηγάς και τον ποταμόν τον ρέοντα διά μέσου της γης, λέγων, Διά τι ελθόντες οι βασιλείς της Ασσυρίας να εύρωσιν ύδωρ πολύ;
\par 5 Ενδυναμωθείς έτι ανωκοδόμησεν όλον το τείχος το κεχαλασμένον και ύψωσεν έως των πύργων, και άλλο τείχος έξω και επεσκεύασε την Μιλλώ της πόλεως Δαβίδ, και έκαμεν όπλα πολλά και θυρεούς.
\par 6 Και έβαλε πολεμάρχους επί τον λαόν, και συνήθροισεν αυτούς προς εαυτόν εις την πλατείαν της πύλης της πόλεως και ελάλησε κατά την καρδίαν αυτών, λέγων,
\par 7 Ενδυναμούσθε και ανδρίζεσθε, μη φοβηθήτε μηδέ πτοηθήτε από προσώπου του βασιλέως της Ασσυρίας, και από προσώπου παντός του πλήθους του μετ' αυτού· διότι πλειότεροι είναι μεθ' ημών παρά μετ' αυτού·
\par 8 μετ' αυτού είναι βραχίονες σάρκινοι μεθ' ημών δε είναι Κύριος ο Θεός ημών, διά να βοηθή ημάς και να μάχηται τας μάχας ημών. Και ενεθαρρύνθη ο λαός εις τους λόγους Εζεκίου του βασιλέως του Ιούδα.
\par 9 Μετά ταύτα απέστειλεν ο Σενναχειρείμ βασιλεύς της Ασσυρίας τους δούλους αυτού εις Ιερουσαλήμ, αυτός δε, έχων μεθ' εαυτού πάσαν την δύναμιν αυτού, επολιόρκει την Λαχείς, προς Εζεκίαν τον βασιλέα του Ιούδα, και προς πάντα τον Ιούδαν τον εν Ιερουσαλήμ, λέγων,
\par 10 Ούτω λέγει Σενναχειρείμ ο βασιλεύς της Ασσυρίας· Εις τι πεποιθότες κάθησθε, πολιορκούμενοι εν Ιερουσαλήμ;
\par 11 Δεν σας απατά ο Εζεκίας διά να σας παραδώση εις θάνατον από πείνης και από δίψης, λέγων, Κύριος ο Θεός ημών θέλει ελευθερώσει ημάς εκ της χειρός του βασιλέως της Ασσυρίας;
\par 12 Αυτός ούτος ο Εζεκίας δεν εσήκωσε τους υψηλούς αυτού τόπους και τα θυσιαστήρια αυτού και είπε προς τον Ιούδαν και προς τον Ιερουσαλήμ, λέγων, Έμπροσθεν ενός μόνον θυσιαστηρίου θέλετε προσκυνεί και επ' αυτό θέλετε θυμιάζει;
\par 13 Δεν εξεύρετε τι έπραξα εγώ και οι πατέρες μου εις πάντας τους λαούς της γης; ηδυνήθησαν οι θεοί των εθνών της γης να λυτρώσωσι τους τόπους αυτών εκ της χειρός μου;
\par 14 Τις εκ πάντων των θεών των εθνών εκείνων, τα οποία οι πατέρες μου εξωλόθρευσαν, ηδυνήθη να λυτρώση τον λαόν αυτού εκ της χειρός μου, ώστε να δυνηθή ο Θεός υμών να λυτρώση υμάς εκ της χειρός μου;
\par 15 Τώρα λοιπόν ας μη σας πλανά ο Εζεκίας, και ας μη σας απατά ούτως, και μη πιστεύετε αυτόν· διότι ουδείς θεός ουδενός έθνους ή βασιλείας ηδυνήθη να λυτρώση τον λαόν αυτού εκ της χειρός μου και εκ της χειρός των πατέρων μου· πολύ ολιγώτερον ο Θεός σας θέλει σας λυτρώσει εκ της χειρός μου.
\par 16 Και περισσότερα έτι ελάλησαν οι δούλοι αυτού εναντίον Κυρίου του Θεού και εναντίον του δούλου αυτού Εζεκίου.
\par 17 Και επιστολάς έγραψε διά να ονειδίση Κύριον τον Θεόν του Ισραήλ και να λαλήση κατ' αυτού, λέγων, Καθώς οι θεοί των εθνών της γης δεν ελύτρωσαν τον λαόν αυτών εκ της χειρός μου, ούτω και ο Θεός του Εζεκίου δεν θέλει λυτρώσει τον λαόν αυτού εκ της χειρός μου.
\par 18 Τότε εβόησαν Ιουδαϊστί, μετά φωνής μεγάλης, προς τον λαόν της Ιερουσαλήμ τον επί του τείχους, διά να φοβίσωσιν αυτούς και να ταράξωσιν αυτούς, όπως κυριεύσωσι την πόλιν·
\par 19 και ελάλησαν κατά του Θεού της Ιερουσαλήμ, καθώς κατά των θεών των λαών της γης, οίτινες είναι έργα χειρών ανθρώπων.
\par 20 Και προσευχήθη περί τούτων Εζεκίας ο βασιλεύς και Ησαΐας ο προφήτης, ο υιός του Αμώς, και εβόησαν προς τον ουρανόν.
\par 21 Και απέστειλε Κύριος άγγελον, όστις ηφάνισε πάντας τους δυνατούς εν ισχύϊ και τους άρχοντας και τους στρατηγούς εν τω στρατοπέδω του βασιλέως της Ασσυρίας. Και επέστρεψε με κατησχυμμένον πρόσωπον εις την γην αυτού. Και ότε εισήλθεν εις τον οίκον του θεού αυτού, οι εξελθόντες εκ των σπλάγχνων αυτού εθανάτωσαν αυτόν εκεί εν μαχαίρα.
\par 22 Και έσωσεν ο Κύριος τον Εζεκίαν και τους κατοίκους της Ιερουσαλήμ εκ της χειρός Σενναχειρείμ του βασιλέως της Ασσυρίας και εκ της χειρός πάντων, και ησφάλισεν αυτούς κυκλόθεν.
\par 23 Και έφεραν πολλοί δώρα προς τον Κύριον εις Ιερουσαλήμ και πολύτιμα πράγματα προς Εζεκίαν τον βασιλέα του Ιούδα· και εμεγαλύνθη έκτοτε ενώπιον πάντων των εθνών.
\par 24 Κατ' εκείνας τας ημέρας ηρρώστησεν ο Εζεκίας έως θανάτου· και προσευχήθη εις τον Κύριον· και επήκουσεν αυτού και έδωκεν εις αυτόν σημείον.
\par 25 Πλην δεν ανταπέδωκεν ο Εζεκίας κατά την εις αυτόν ευεργεσίαν· διότι επήρθη η καρδία αυτού· όθεν επήλθεν οργή επ' αυτόν και επί τον Ιούδαν και την Ιερουσαλήμ.
\par 26 Και εταπεινώθη ο Εζεκίας διά την έπαρσιν της καρδίας αυτού, αυτός και οι κάτοικοι της Ιερουσαλήμ, και δεν ήλθεν επ' αυτούς η οργή του Κυρίου εν ταις ημέραις του Εζεκίου.
\par 27 Απέκτησε δε ο Εζεκίας πλούτον και δόξαν πολλήν σφόδρα· και έκαμεν εις εαυτόν θησαυρούς αργυρίου και χρυσίου και λίθων πολυτίμων και αρωμάτων και ασπίδων και παντός είδους σκευών επιθυμητών·
\par 28 και αποθήκας διά το εισόδημα του σίτου και του οίνου και του ελαίου· και σταύλους διά παν είδος κτηνών και μάνδρας διά ποίμνια.
\par 29 Και έκαμεν εις εαυτόν πόλεις και απέκτησε πρόβατα και βόας εις πλήθος· διότι ο Θεός έδωκεν εις αυτόν περιουσίαν πολλήν σφόδρα.
\par 30 Έφραξεν έτι αυτός ο Εζεκίας την άνω έξοδον των υδάτων του Γιών, και διηύθυνεν αυτά κάτω προς δυσμάς της πόλεως Δαβίδ. Και ευωδώθη ο Εζεκίας εις πάντα τα έργα αυτού.
\par 31 Επί των πρέσβεων όμως των αρχόντων της Βαβυλώνος, οίτινες έστειλαν προς αυτόν διά να ερευνήσωσι περί του θαύματος του γενομένου εν τη γη, ο Θεός εγκατέλιπεν αυτόν, διά να δοκιμάση αυτόν, ώστε να γνωρίση πάντα τα εν τη καρδία αυτού.
\par 32 Αι δε λοιπαί πράξεις του Εζεκίου και τα ελέη αυτού, ιδού, είναι γεγραμμένα εν τη οράσει Ησαΐου του προφήτου, υιού του Αμώς, εν τω βιβλίω των βασιλέων Ιούδα και Ισραήλ.
\par 33 Και εκοιμήθη ο Εζεκίας μετά των πατέρων αυτού, και έθαψαν αυτόν εν τω υψηλοτέρω των τάφων των υιών Δαβίδ· και πας ο Ιούδας και οι κάτοικοι της Ιερουσαλήμ έκαμον εις αυτόν τιμάς εν τω θανάτω αυτού· εβασίλευσε δε αντ' αυτού Μανασσής ο υιός αυτού.

\chapter{33}

\par 1 Δώδεκα ετών ηλικίας ήτο ο Μανασσής ότε εβασίλευσε, και εβασίλευσε πεντήκοντα πέντε έτη εν Ιερουσαλήμ.
\par 2 Και έπραξε πονηρά ενώπιον του Κυρίου, κατά τα βδελύγματα των εθνών, τα οποία εξεδίωξεν ο Κύριος απ' έμπροσθεν των υιών Ισραήλ·
\par 3 και ανωκοδόμησε τους υψηλούς τόπους, τους οποίους Εζεκίας ο πατήρ αυτού κατέστρεψε, και ανήγειρε θυσιαστήρια εις τους Βααλείμ, και έκαμεν άλση και προσεκύνησε πάσαν την στρατιάν του ουρανού και ελάτρευσεν αυτά.
\par 4 Και ωκοδόμησε θυσιαστήρια εν τω οίκω του Κυρίου, περί του οποίου ο Κύριος είπεν, Εν Ιερουσαλήμ θέλει είσθαι το όνομά μου εις τον αιώνα.
\par 5 Και ωκοδόμησε θυσιαστήρια εις πάσαν την στρατιάν του ουρανού εντός των δύο αυλών του οίκου του Κυρίου.
\par 6 Και αυτός διεβίβασε τους υιούς αυτού διά του πυρός εν τη κοιλάδι του υιού του Εννόμ· και προεμάντευε καιρούς και έκαμνεν οιωνισμούς και μαγείας και εσύστησεν ανταποκριτάς δαιμονίων και επαοιδούς· πολλά πονηρά έπραξεν ενώπιον του Κυρίου, διά να παροργίση αυτόν.
\par 7 Και έστησε το γλυπτόν, την εικόνα την οποίαν έκαμεν, εν τω οίκω του Θεού, περί του οποίου ο Θεός είπε προς τον Δαβίδ και προς τον Σολομώντα τον υιόν αυτού, Εν τω οίκω τούτω και εν Ιερουσαλήμ, την οποίαν εξέλεξα από πασών των φυλών του Ισραήλ, θέλω θέσει το όνομά μου εις τον αιώνα·
\par 8 και δεν θέλω μετασαλεύσει τον πόδα του Ισραήλ από της γης, την οποίαν παρέδωκα εις τους πατέρας σας· εάν μόνον προσέξωσι να κάμνωσι πάντα όσα προσέταξα εις αυτούς, κατά πάντα τον νόμον και τα διατάγματα και τας κρίσεις τας δοθείσας διά του Μωϋσέως.
\par 9 Και επλάνησεν ο Μανασσής τον Ιούδαν και τους κατοίκους της Ιερουσαλήμ, ώστε να πράττωσι πονηρότερα παρά τα έθνη, τα οποία ο Κύριος ηφάνισεν απ' έμπροσθεν των υιών Ισραήλ.
\par 10 Και ελάλησε Κύριος προς τον Μανασσήν και προς τον λαόν αυτού· πλην δεν έδωκαν ακρόασιν.
\par 11 Διά τούτο έφερε κατ' αυτών ο Κύριος τους άρχοντας του στρατεύματος του βασιλέως της Ασσυρίας, και επίασαν τον Μανασσήν μεταξύ των θάμνων και δέσαντες αυτόν με αλύσεις, έφεραν αυτόν εις Βαβυλώνα.
\par 12 Και ενώ ήτο εν θλίψει, ικέτευσε Κύριον τον Θεόν αυτού και εταπεινώθη σφόδρα ενώπιον του Θεού των πατέρων αυτού,
\par 13 και προσηυχήθη εις αυτόν· τότε ηλέησεν αυτόν και επήκουσε της δεήσεως αυτού και επανέφερεν αυτόν εις Ιερουσαλήμ, εις το βασίλειον αυτού. Τότε εγνώρισεν ο Μανασσής έτι ο Κύριος αυτός είναι ο Θεός.
\par 14 Μετά δε ταύτα ωκοδόμησε τείχος έξω της πόλεως Δαβίδ, προς δυσμάς του Γιών, εν τη κοιλάδι, έως της εισόδου της πύλης της ιχθυϊκής, και περιεκύκλωσε το Οφήλ και ύψωσεν αυτό εις μέγα ύψος, και έβαλε πολεμάρχους εν πάσαις ταις ωχυρωμέναις πόλεσι του Ιούδα.
\par 15 Και αφήρεσε τους ξένους θεούς και την εικόνα από του οίκου του Κυρίου και πάντα τα θυσιαστήρια, τα οποία ωκοδόμησεν εν τω όρει του οίκου του Κυρίου και εν Ιερουσαλήμ· και έρριψεν αυτά έξω της πόλεως.
\par 16 Και ανώρθωσε το θυσιαστήριον του Κυρίου και εθυσίασεν επ' αυτού θυσίας ειρηνικάς και ευχαριστηρίους, και προσέταξε τον Ιούδαν να λατρεύη Κύριον τον Θεόν του Ισραήλ.
\par 17 Ο λαός όμως εθυσίαζεν έτι επί τους υψηλούς τόπους, πλην μόνον εις Κύριον τον Θεόν αυτών.
\par 18 Αι δε λοιπαί πράξεις του Μανασσή και η προσευχή αυτού η προς τον Θεόν αυτού και οι λόγοι των βλεπόντων, οίτινες ελάλησαν προς αυτόν εν ονόματι Κυρίου του Θεού Ισραήλ, ιδού, είναι γεγραμμέναι εν τοις χρονικοίς των βασιλέων του Ισραήλ.
\par 19 Και η προσευχή αυτού, και πως εισηκούσθη, και πάσαι αι αμαρτίαι αυτού και η αποστασία αυτού και τα μέρη, όπου ωκοδόμησεν υψηλούς τόπους και έστησε τα άλση και τα γλυπτά, πριν ταπεινωθή, ιδού, είναι γεγραμμένα εν τοις λόγοις των βλεπόντων.
\par 20 Και εκοιμήθη ο Μανασσής μετά των πατέρων αυτού, και έθαψαν αυτόν εν τω οίκω αυτού· εβασίλευσε δε αντ' αυτού Αμών ο υιός αυτού.
\par 21 Εικοσιδύο ετών ηλικίας ήτο ο Αμών ότε εβασίλευσε, και εβασίλευσε δύο έτη εν Ιερουσαλήμ.
\par 22 Και έπραξε πονηρά ενώπιον του Κυρίου, καθώς έπραξε Μανασσής ο πατήρ αυτού· και εθυσίαζεν ο Αμών εις πάντα τα γλυπτά, τα οποία Μανασσής ο πατήρ αυτού έκαμε, και ελάτρευεν αυτά·
\par 23 και δεν εταπεινώθη ενώπιον του Κυρίου, καθώς εταπεινώθη Μανασσής ο πατήρ αυτού· αλλ' αυτός ο Αμών ηνόμησε μάλλον και μάλλον.
\par 24 Και συνώμοσαν οι δούλοι αυτού κατ' αυτού και εθανάτωσαν αυτόν εν τω οίκω αυτού.
\par 25 Ο δε λαός της γης εθανάτωσε πάντας τους συνομόσαντας κατά του βασιλέως Αμών· και έκαμεν ο λαός της γης βασιλέα αντ' αυτού Ιωσίαν τον υιόν αυτού.

\chapter{34}

\par 1 Οκτώ ετών ηλικίας ήτο ο Ιωσίας ότε εβασίλευσε· και εβασίλευσεν εν Ιερουσαλήμ έτη τριάκοντα και εν.
\par 2 Και έπραξε το ευθές ενώπιον του Κυρίου, και περιεπάτησεν εν ταις οδοίς Δαβίδ του πατρός αυτού, και δεν εξέκλινε δεξιά η αριστερά.
\par 3 Και εν τω ογδόω έτει της βασιλείας αυτού, νέος ων έτι, ήρχισε να εκζητή τον Θεόν του Δαβίδ του πατρός αυτού· και εν τω δωδεκάτω έτει ήρχισε να καθαρίζη τον Ιούδαν και την Ιερουσαλήμ από των υψηλών τόπων και από των άλσεων και των γλυπτών και των χωνευτών.
\par 4 Και κατέστρεψαν έμπροσθεν αυτού τα θυσιαστήρια των Βααλείμ· και τα είδωλα τα υπεράνω αυτών κατεκρήμνισε· και τα άλση και τα γλυπτά και τα χωνευτά κατεσύντριψε και ελέπτυνεν εις σκόνην και έρριψεν αυτήν επί τα μνήματα των θυσιαζόντων εις αυτά.
\par 5 Και τα οστά των ιερέων έκαυσεν επί των θυσιαστηρίων αυτών και εκαθάρισε τον Ιούδαν και την Ιερουσαλήμ.
\par 6 Και έκαμε το αυτό εις τας πόλεις του Μανασσή και Εφραΐμ και Συμεών και μέχρι του Νεφθαλί, κύκλω των ηρημωμένων τόπων αυτών.
\par 7 Και αφού κατέστρεψε τα θυσιαστήρια και τα άλση και κατελέπτυνεν εις σκόνην τα γλυπτά και κατέκοψε πάντα τα είδωλα διά πάσης της γης του Ισραήλ, επέστρεψεν εις Ιερουσαλήμ.
\par 8 Εν δε τω δεκάτω ογδόω έτει της βασιλείας αυτού, αφού εκαθάρισε την γην και τον ναόν, εξαπέστειλε τον Σαφάν υιόν του Αζαλίου, και τον Μαασίαν τον άρχοντα της πόλεως, και τον Ιωάχ υιόν του Ιωάχαζ τον υπομνηματογράφον, διά να επισκευάσωσι τον οίκον Κυρίου του Θεού αυτού.
\par 9 Και ελθόντες προς Χελκίαν τον ιερέα τον μέγαν, παρέδωκαν το αργύριον το εισαχθέν εις τον οίκον του Θεού, το οποίον οι Λευΐται οι φυλάττοντες τας θύρας εσύναξαν εκ της χειρός του Μανασσή και Εφραΐμ και εκ παντός του επιλοίπου του Ισραήλ και εκ παντός του Ιούδα και Βενιαμίν· και επέστρεψαν εις Ιερουσαλήμ.
\par 10 Και έδωκαν αυτά εις την χείρα των ποιούντων τα έργα, των επιστατούντων εν τω οίκω του Κυρίου· οι δε ποιούντες τα έργα, τα οποία ειργάζοντο εν τω οίκω του Κυρίου, παρέδωκαν αυτό διά να επισκευάσωσι και να επιδιορθώσωσι τον οίκον·
\par 11 εις τους τέκτονας και οικοδόμους έδωκαν αυτό, διά ν' αγοράσωσι λίθους πελεκητούς και ξύλα διά δοκούς, και διά να στεγάσωσι τους οίκους τους οποίους κατέστρεψαν οι βασιλείς του Ιούδα.
\par 12 Και ειργάζοντο οι άνδρες το έργον εν πίστει· επιτηρηταί δε επ' αυτών ήσαν Ιαάθ και Οβαδίας, οι Λευΐται, εκ των υιών Μεραρί· και Ζαχαρίας και Μεσουλλάμ, εκ των υιών των Κααθιτών, διά να κατεπείγωσι το έργον· και εκ των Λευϊτών πάντες οι επιστήμονες μουσικών οργάνων.
\par 13 Ήσαν έτι επί των αχθοφόρων και εργοδιώκται πάντων των εργαζομένων, καθ' οποιανδήποτε υπηρεσίαν· και εκ των Λευϊτών ήσαν γραμματείς και επιστάται και θυρωροί.
\par 14 Και ενώ εξέφερον το αργύριον το εισαχθέν εις τον οίκον του Κυρίου, εύρηκε Χελκίας ο ιερεύς το βιβλίον του νόμου του Κυρίου, του δοθέντος διά χειρός του Μωϋσέως.
\par 15 Και απεκρίθη ο Χελκίας και είπε προς Σαφάν τον γραμματέα, εύρηκα βιβλίον του νόμου εν τω οίκω του Κυρίου. Και έδωκεν ο Χελκίας το βιβλίον εις τον Σαφάν.
\par 16 Και ο Σαφάν έφερε το βιβλίον προς τον βασιλέα και έπειτα έδωκε λόγον εις τον βασιλέα, λέγων, Οι δούλοί σου κάμνουσι παν το διορισθέν εις αυτούς·
\par 17 και ηρίθμησαν το αργύριον το ευρεθέν εν τω οίκω του Κυρίου, και παρέδωκαν αυτό εις την χείρα των επιστατών και εις την χείρα των ποιούντων τα έργα.
\par 18 Και απήγγειλε Σαφάν ο γραμματεύς προς τον βασιλέα, λέγων, Χελκίας ιερεύς έδωκεν εις εμέ βιβλίον. Και ανέγνωσεν αυτό ο Σαφάν ενώπιον του βασιλέως.
\par 19 Και ως ήκουσεν ο βασιλεύς τους λόγους του νόμου, διέσχισε τα ιμάτια αυτού.
\par 20 Και προσέταξεν ο βασιλεύς Χελκίαν και Αχικάμ τον υιόν του Σαφάν και Αβδών τον υιόν του Μιχαία και Σαφάν τον γραμματέα και Ασαΐαν τον δούλον του βασιλέως, λέγων,
\par 21 Υπάγετε, ερωτήσατε τον Κύριον περί εμού και περί των εναπολειφθέντων εν τω Ισραήλ και εν τω Ιούδα, περί των λόγων του βιβλίου του ευρεθέντος· διότι μεγάλη είναι η οργή του Κυρίου ήτις εξεχύθη εφ' ημάς, επειδή οι πατέρες ημών δεν εφύλαξαν τον λόγον του Κυρίου, ώστε να πράξωσι κατά πάντα τα γεγραμμένα εν τω βιβλίω τούτω.
\par 22 Τότε υπήγεν ο Χελκίας και οι παρά του βασιλέως προς Όλδαν την προφήτισσαν, την γυναίκα του Σαλλούμ υιού του Τικβά, υιού του Ασρά, του ιματιοφύλακος, κατώκει δε αύτη εν Ιερουσαλήμ, κατά το Μισνέ· και ελάλησαν προς αυτήν κατά ταύτα.
\par 23 Η δε είπε προς αυτούς· Ούτω λέγει Κύριος ο Θεός του Ισραήλ· Είπατε προς τον άνθρωπον όστις σας απέστειλε προς εμέ,
\par 24 Ούτω λέγει Κύριος· Ιδού, εγώ επιφέρω κακά επί τον τόπον τούτον και επί τους κατοίκους αυτού, πάσας τας κατάρας τας γεγραμμένας εν τω βιβλίω, το οποίον ανέγνωσαν ενώπιον του βασιλέως του Ιούδα·
\par 25 επειδή με εγκατέλιπον και εθυμίασαν εις άλλους θεούς, διά να με παροργίσωσι διά πάντα τα έργα των χειρών αυτών· διά τούτο θέλει εκχυθή ο θυμός μου επί τον τόπον τούτον και δεν θέλει σβεσθή.
\par 26 Προς δε τον βασιλέα του Ιούδα, όστις σας απέστειλε διά να ερωτήσητε τον Κύριον, ούτω θέλετε ειπεί προς αυτόν· Ούτω λέγει Κύριος ο Θεός του Ισραήλ, περί των λόγων τους οποίους ήκουσας·
\par 27 επειδή η καρδία σου ηπαλύνθη, και εταπεινώθης ενώπιον του Θεού, ότε ήκουσας τους λόγους αυτού εναντίον του τόπου τούτου και εναντίον των κατοίκων αυτού, και εταπεινώθης ενώπιόν μου και διέσχισας τα ιμάτιά σου και έκλαυσας ενώπιόν μου, διά τούτο και εγώ επήκουσα, λέγει Κύριος·
\par 28 ιδού, εγώ θέλω σε συνάξει εις τους πατέρας σου, και θέλεις συναχθή εις τον τάφον σου εν ειρήνη, και δεν θέλουσιν ιδεί οι οφθαλμοί σου πάντα τα κακά, τα οποία εγώ επιφέρω επί τον τόπον τούτον και επί τους κατοίκους αυτού. Και έφεραν απόκρισιν προς τον βασιλέα.
\par 29 Και απέστειλεν ο βασιλεύς και συνήγαγε πάντας τους πρεσβυτέρους του Ιούδα και της Ιερουσαλήμ.
\par 30 Και ανέβη ο βασιλεύς εις τον οίκον του Κυρίου, και πάντες οι άνδρες Ιούδα και οι κάτοικοι της Ιερουσαλήμ και οι ιερείς και οι Λευΐται και πας ο λαός, από μεγάλου έως μικρού· και ανέγνωσεν εις επήκοον αυτών πάντας τους λόγους του βιβλίου της διαθήκης, του ευρεθέντος εν τω οίκω του Κυρίου.
\par 31 Και σταθείς ο βασιλεύς επί του τόπου αυτού, έκαμε την διαθήκην ενώπιον του Κυρίου, να περιπατή κατόπιν του Κυρίου και να φυλάττη τας εντολάς αυτού και τα μαρτύρια αυτού και τα διατάγματα αυτού εξ όλης αυτού της καρδίας και εξ όλης αυτού της ψυχής, ώστε να εκτελή τους λόγους της διαθήκης τους γεγραμμένους εν τω βιβλίω τούτω.
\par 32 Και έκαμε πάντας τους ευρεθέντας εν Ιερουσαλήμ και τον Βενιαμίν να σταθώσιν εν τούτω. Και οι κάτοικοι της Ιερουσαλήμ έκαμον κατά την διαθήκην του Θεού, του Θεού των πατέρων αυτών.
\par 33 Και αφήρεσεν ο Ιωσίας πάντα τα βδελύγματα εκ πάντων των τόπων των υιών Ισραήλ, και έκαμε πάντας τους ευρεθέντας εν τω Ισραήλ να λατρεύωσι Κύριον τον Θεόν αυτών· κατά πάσας τας ημέρας αυτού δεν απεμακρύνθησαν από όπισθεν Κυρίου του Θεού των πατέρων αυτών.

\chapter{35}

\par 1 Ο Ιωσίας έκαμεν έτι πάσχα προς τον Κύριον εν Ιερουσαλήμ· και εθυσίασαν το πάσχα την δεκάτην τετάρτην του πρώτου μηνός.
\par 2 Και έστησε τους ιερείς εις τας φυλακάς αυτών και ενίσχυσεν αυτούς εις την υπηρεσίαν του οίκου του Κυρίου·
\par 3 και είπε προς τους Λευΐτας τους διδάσκοντας πάντα τον Ισραήλ, τους καθιερωμένους εις τον Κύριον, Θέσατε την κιβωτόν την αγίαν εν τω οίκω, τον οποίον ωκοδόμησε Σολομών ο υιός Δαβίδ του βασιλέως του Ισραήλ· δεν θέλετε βαστάζει πλέον αυτήν επ' ώμων· δουλεύετε τώρα Κύριον τον Θεόν σας και τον λαόν αυτού τον Ισραήλ·
\par 4 και ετοιμάσθητε κατά τους οίκους των πατριών σας, κατά τας διαιρέσεις σας, κατά το γεγραμμένον Δαβίδ του βασιλέως του Ισραήλ, και κατά το γεγραμμένον Σολομώντος του υιού αυτού.
\par 5 Και στήτε εν τω αγιαστηρίω κατά τας διαιρέσεις των οίκων των πατριών υπέρ των αδελφών σας των υιών του λαού, και κατά την διαίρεσιν των οίκων των πατριών των Λευϊτών.
\par 6 Και θυσιάσατε το πάσχα και αγιάσθητε και ετοιμάσατε αυτό εις τους αδελφούς σας, διά να κάμωσι κατά τον λόγον του Κυρίου, τον δοθέντα διά χειρός του Μωϋσέως.
\par 7 Και προσέφερεν ο Ιωσίας εις τον λαόν πρόβατα, αρνία και ερίφια αιγών, τα πάντα διά θυσίας του πάσχα, διά πάντας τους παρευρεθέντας, τριάκοντα χιλιάδας τον αριθμόν, και τρισχιλίους βόας· ταύτα ήσαν εκ των υπαρχόντων του βασιλέως.
\par 8 Και οι άρχοντες αυτού προσέφεραν αυτοπροαιρέτως εις τον λαόν, εις τους ιερείς, και εις τους Λευΐτας. Ο Χελκίας και ο Ζαχαρίας και ο Ιεχιήλ, οι άρχοντες του οίκου του Θεού έδωκαν εις τους ιερείς, διά τας θυσίας του πάσχα, δισχίλια και εξακόσια αρνία και ερίφια, και τριακοσίους βόας.
\par 9 Και ο Χωνανίας και Σεμαΐας και Ναθανιήλ, οι αδελφοί αυτού, και Ασαβίας και Ιεϊήλ και Ιωζαβάδ, άρχοντες των Λευϊτών, προσέφεραν εις τους Λευΐτας, διά θυσίας του πάσχα, πεντακισχίλια αρνία και ερίφια και πεντακοσίους βόας.
\par 10 Και ητοιμάσθη η υπηρεσία, και οι ιερείς εστάθησαν εν τω τόπω αυτών και οι Λευΐται εις τας διαιρέσεις αυτών, κατά την προσταγήν του βασιλέως.
\par 11 Και εθυσίασαν το πάσχα και ερράντισαν οι ιερείς το αίμα εκ της χειρός αυτών, και οι Λευΐται εξέδειραν τα θύματα.
\par 12 Και διήρεσαν τα ολοκαυτώματα, διά να δώσωσιν αυτά κατά τας διαιρέσεις των οίκων των πατριών του λαού, διά να προσφέρωσιν εις τον Κύριον, κατά το γεγραμμένον εν τω βιβλίω του Μωϋσέως· και ωσαύτως περί των βοών.
\par 13 Και έψησαν το πάσχα εν πυρί, κατά το διατεταγμένον· τα δε άγια έψησαν εις χύτρας και εις λέβητας και εις κακάβια, και διεμοίρασαν ταχέως μεταξύ παντός του λαού.
\par 14 Και έπειτα ητοίμασαν εις εαυτούς και εις τους ιερείς· διότι οι ιερείς οι υιοί Ααρών κατεγίνοντο εις το να προσφέρωσι τα ολοκαυτώματα και τα στέατα μέχρι νυκτός· διά τούτο οι Λευΐται ητοίμασαν εις εαυτούς και εις τους ιερείς τους υιούς Ααρών.
\par 15 Και οι ψαλτωδοί οι υιοί του Ασάφ ήσαν εν τω τόπω αυτών, κατά την διαταγήν του Δαβίδ και του Ασάφ και του Αιμάν και του Ιεδουθούν, του βλέποντος του βασιλέως, και οι πυλωροί εφύλαττον εν εκάστη πύλη· δεν ήτο χρεία να απομακρυνθώσιν από της υπηρεσίας αυτών· διότι οι αδελφοί αυτών οι Λευΐται ητοίμασαν δι' αυτούς.
\par 16 Και ητοιμάσθη πάσα η υπηρεσία του Κυρίου την αυτήν ημέραν, διά να κάμωσι το πάσχα και να προσφέρωσιν ολοκαυτώματα επί του θυσιαστηρίου του Κυρίου, κατά την προσταγήν του βασιλέως Ιωσία.
\par 17 Και οι υιοί Ισραήλ οι παρευρεθέντες έκαμον το πάσχα εν τω καιρώ εκείνω και την εορτήν των αζύμων επτά ημέρας.
\par 18 Και δεν έγεινε πάσχα ως εκείνο εν τω Ισραήλ, από των ημερών Σαμουήλ του προφήτου· ουδέ έκαμον πάντες οι βασιλείς του Ισραήλ ως το πάσχα, το οποίον έκαμεν ο Ιωσίας και οι ιερείς και οι Λευΐται και πας ο Ιούδας και ο Ισραήλ οι παρευρεθέντες, και οι κάτοικοι της Ιερουσαλήμ.
\par 19 Εν τω δεκάτω ογδόω έτει της βασιλείας του Ιωσία έγεινε το πάσχα τούτο.
\par 20 Μετά δε ταύτα πάντα, αφού ο Ιωσίας ητοίμασε τον οίκον, ανέβη Νεχαώ ο βασιλεύς της Αιγύπτου διά να πολεμήση εν Χαρκεμίς επί τον Ευφράτην· και εξήλθεν ο Ιωσίας εναντίον αυτού.
\par 21 Απέστειλε δε μηνυτάς προς αυτόν, λέγων, Τι είναι μεταξύ εμού και σου, βασιλεύ του Ιούδα; δεν έρχομαι σήμερον εναντίον σου, αλλ' εναντίον του οίκου, με τον οποίον έχω πόλεμον· και ο Θεός προσέταξεν εις εμέ να σπεύσω· άπεχε από του Θεού, όστις είναι μετ' εμού, και να μη σε εξολοθρεύση.
\par 22 Πλην ο Ιωσίας δεν απέστρεψε το πρόσωπον αυτού απ' αυτού· αλλά μετεσχηματίσθη, διά να πολεμήση εναντίον αυτού, και δεν εισήκουσεν εις τους λόγους του Νεχαώ, τους εκ στόματος του Θεού, και ήλθε να πολεμήση εν τη κοιλάδι Μεγιδδώ.
\par 23 Και ετόξευσαν οι τοξόται επί τον βασιλέα Ιωσίαν· και είπεν ο βασιλεύς προς τους δούλους αυτού, Εκβάλετέ με έξω, διότι επληγώθην βαρέως.
\par 24 Και εξέβαλον αυτόν οι δούλοι αυτού εκ της αμάξης, και επεβίβασαν αυτόν εις την δευτέραν αυτού άμαξαν· και έφεραν αυτόν εις Ιερουσαλήμ, και απέθανε· και ετάφη εν τοις τάφοις των πατέρων αυτού. Και πας ο Ιούδας και η Ιερουσαλήμ επένθησαν επί τον Ιωσίαν.
\par 25 Και εθρήνησεν ο Ιερεμίας διά τον Ιωσίαν· και πάντες οι ψάλται και αι ψάλτριαι αναφέρουσιν έως της σήμερον εις τους θρήνους αυτών τον Ιωσίαν, και έκαμον αυτούς νόμιμον εν τω Ισραήλ· και ιδού, είναι γεγραμμένοι εν τοις Θρήνοις.
\par 26 Αι δε λοιπαί πράξεις του Ιωσία και τα ελέη αυτού, κατά το γεγραμμένον εν τω νόμω του Κυρίου,
\par 27 και τα έργα αυτού, τα πρώτα και τα έσχατα, ιδού, είναι γεγραμμένα εν τω βιβλίω των βασιλέων του Ισραήλ και του Ιούδα.

\chapter{36}

\par 1 Και έλαβεν ο λαός της γης τον Ιωάχαζ, υιόν του Ιωσία, και έκαμον αυτόν βασιλέα εν Ιερουσαλήμ, αντί του πατρός αυτού.
\par 2 Εικοσιτριών ετών ηλικίας ήτο ο Ιωάχαζ ότε εβασίλευσε, και εβασίλευσε τρεις μήνας εν Ιερουσαλήμ.
\par 3 Καθήρεσε δε αυτόν ο βασιλεύς της Αιγύπτου εν Ιερουσαλήμ, και κατεδίκασε την γην εις πρόστιμον εκατόν ταλάντων αργυρίου και ενός ταλάντου χρυσίου.
\par 4 Και έκαμεν ο βασιλεύς της Αιγύπτου τον Ελιακείμ τον αδελφόν αυτού βασιλέα επί Ιούδαν και Ιερουσαλήμ, και μετήλλαξε το όνομα αυτού εις Ιωακείμ· τον δε Ιωάχαζ, τον αδελφόν αυτού, έλαβεν ο Νεχαώ και έφερεν αυτόν εις Αίγυπτον.
\par 5 Εικοσιπέντε ετών ηλικίας ήτο ο Ιωακείμ ότε εβασίλευσε, και εβασίλευσεν ένδεκα έτη εν Ιερουσαλήμ· και έπραξε πονηρά ενώπιον Κυρίου του Θεού αυτού.
\par 6 Ανέβη εναντίον αυτού Ναβουχοδονόσορ ο βασιλεύς της Βαβυλώνος, και έδεσεν αυτόν με αλύσεις, διά να φέρη αυτόν εις Βαβυλώνα.
\par 7 Και εκ των σκευών του οίκου του Κυρίου έφερεν ο Ναβουχοδονόσορ εις Βαβυλώνα και έθεσεν αυτά εν τω ναώ αυτού εν Βαβυλώνι.
\par 8 Αι δε λοιπαί πράξεις του Ιωακείμ και τα βδελύγματα αυτού όσα έκαμε, και όσα ευρέθησαν εν αυτώ, ιδού, είναι γεγραμμένα εν τω βιβλίω των βασιλέων του Ισραήλ και του Ιούδα· και εβασίλευσεν αντ' αυτού Ιωαχείν ο υιός αυτού.
\par 9 Δέκα οκτώ ετών ηλικίας ήτο ο Ιωαχείν ότε εβασίλευσε, και εβασίλευσε τρεις μήνας και δέκα ημέρας εν Ιερουσαλήμ· και έπραξε πονηρά ενώπιον Κυρίου.
\par 10 Εν τω τέλει δε του ενιαυτού αποστείλας ο βασιλεύς Ναβουχοδονόσορ, έφερεν αυτόν εις Βαβυλώνα, μετά των εκλεκτών σκευών του οίκου του Κυρίου· και έκαμε Σεδεκίαν τον αδελφόν αυτού βασιλέα επί τον Ιούδαν και Ιερουσαλήμ.
\par 11 Ενός και είκοσι ετών ηλικίας ήτο ο Σεδεκίας ότε εβασίλευσε, και εβασίλευσεν ένδεκα έτη εν Ιερουσαλήμ.
\par 12 Και έπραξε πονηρά ενώπιον Κυρίου του Θεού αυτού· δεν εταπεινώθη ενώπιον Ιερεμίου του προφήτου, λαλούντος εκ στόματος του Κυρίου.
\par 13 Και έτι απεστάτησεν εναντίον του βασιλέως Ναβουχοδονόσορ, όστις ώρκισεν αυτόν εις τον Θεόν· και εσκλήρυνε τον τράχηλον αυτού και επεισμάτωσε την καρδίαν αυτού, ώστε να μη επιστρέψη εις Κύριον τον Θεόν του Ισραήλ.
\par 14 Πάντες προσέτι οι πρώτοι των ιερέων και ο λαός ηθέτησαν καθ υπερβολήν κατά πάντα τα βδελύγματα των εθνών και εμίαναν τον οίκον του Κυρίου, τον οποίον ηγίασεν εν Ιερουσαλήμ.
\par 15 Και παρήγγειλεν εις αυτούς Κύριος ο Θεός των πατέρων αυτών διά χειρός των απεσταλμένων αυτού, εγειρόμενος πρωΐ και εξαποστέλλων· διότι εφείδετο του λαού αυτού και του κατοικητηρίου αυτού.
\par 16 Αλλ' αυτοί εχλεύαζον τους απεσταλμένους του Θεού και κατεφρόνουν τους λόγους αυτού και έσκωπτον τους προφήτας αυτού, εωσού η οργή του Κυρίου ανέβη κατά του λαού αυτού, ώστε δεν ήτο θεραπεία·
\par 17 διά τούτο έφερεν επ' αυτούς τον βασιλέα των Χαλδαίων, και εθανάτωσε τους νεανίσκους αυτών εν μαχαίρα εντός του οίκου του αγιαστηρίου αυτών, και δεν εφείσθη νέου ή παρθένου, γέροντος η κεκυφότος· πάντας παρέδωκεν εις την χείρα αυτού.
\par 18 Και πάντα τα σκεύη του οίκου του Θεού, μεγάλα και μικρά, και τους θησαυρούς του οίκου του Κυρίου και τους θησαυρούς του βασιλέως και των αρχόντων αυτού, τα πάντα έφερεν εις Βαβυλώνα.
\par 19 Και κατέκαυσαν τον οίκον του Θεού και κατέσκαψαν το τείχος της Ιερουσαλήμ, και πάντα τα παλάτια αυτής κατέκαυσαν εν πυρί, και πάντα τα πολύτιμα σκεύη αυτής ηφάνισαν·
\par 20 Και τους εκφυγόντας την μάχαιραν μετώκισεν εις Βαβυλώνα, όπου ήσαν δούλοι εις αυτόν και εις τους υιούς αυτού, μέχρι του καιρού της βασιλείας των Περσών·
\par 21 διά να πληρωθή ο λόγος του Κυρίου ο διά στόματος Ιερεμίου, εωσού η γη χαρή τα σάββατα αυτής· διότι πάντα τον καιρόν της ερημώσεως αυτής εφύλαττε σάββατον, εωσού συμπληρωθώσιν εβδομήκοντα έτη.
\par 22 Εν δε τω πρώτω έτει Κύρου του βασιλέως της Περσίας, διά να πληρωθή ο λόγος του Κυρίου ο διά στόματος Ιερεμίου, διήγειρεν ο Κύριος το πνεύμα του Κύρου βασιλέως της Περσίας, και διεκήρυξε διά παντός του βασιλείου αυτού, και μάλιστα εγγράφως, λέγων,
\par 23 Ούτω λέγει Κύρος ο βασιλεύς της Περσίας· πάντα τα βασίλεια της γης έδωκεν εις εμέ Κύριος ο Θεός του ουρανού· και αυτός προσέταξεν εις εμέ να οικοδομήσω εις αυτόν οίκον εν Ιερουσαλήμ, ήτις είναι εν τη Ιουδαία· τις εξ υμών είναι εκ παντός του λαού αυτού; Κύριος ο Θεός αυτού έστω μετ' αυτού, και ας αναβή.


\end{document}