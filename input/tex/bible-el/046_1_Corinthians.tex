\begin{document}

\title{1 Κορινθίους}


\chapter{1}

\par 1 Παύλος, προσκεκλημένος απόστολος Ιησού Χριστού διά θελήματος Θεού, και Σωσθένης ο αδελφός,
\par 2 προς την εκκλησίαν του Θεού την ούσαν εν Κορίνθω, τους ηγιασμένους εν Χριστώ Ιησού, τους προσκεκλημένους αγίους, μετά πάντων των επικαλουμένων εν παντί τόπω το όνομα Ιησού Χριστού του Κυρίου ημών, αυτών τε και ημών·
\par 3 χάρις είη υμίν και ειρήνη από Θεού Πατρός ημών και Κυρίου Ιησού Χριστού.
\par 4 Ευχαριστώ πάντοτε εις τον Θεόν μου διά σας, διά την χάριν του Θεού την δοθείσαν εις εσάς διά του Ιησού Χριστού·
\par 5 ότι κατά πάντα επλουτίσθητε δι' αυτού, κατά πάντα λόγον και πάσαν γνώσιν,
\par 6 καθώς η μαρτυρία του Χριστού εστηρίχθη μεταξύ σας,
\par 7 ώστε δεν μένετε οπίσω εις ουδέν χάρισμα, προσμένοντες την αποκάλυψιν του Κυρίου ημών Ιησού Χριστού·
\par 8 όστις και θέλει σας στηρίξει έως τέλους αμέμπτους εν τη ημέρα του Κυρίου ημών Ιησού Χριστού.
\par 9 Πιστός ο Θεός, διά του οποίου προσεκλήθητε εις το να ήσθε συγκοινωνοί του Υιού αυτού Ιησού Χριστού του Κυρίου ημών.
\par 10 Σας παρακαλώ δε, αδελφοί, διά του ονόματος του Κυρίου ημών Ιησού Χριστού, να λέγητε πάντες το αυτό, και να μη ήναι σχίσματα μεταξύ σας, αλλά να ήσθε εντελώς ηνωμένοι έχοντες το αυτό πνεύμα και την αυτήν γνώμην.
\par 11 Διότι εφανερώθη εις εμέ παρά των εκ της οικογενείας της Χλόης, περί υμών, αδελφοί μου, ότι είναι έριδες μεταξύ σας·
\par 12 λέγω δε τούτο, διότι έκαστος από σας λέγει· Εγώ μεν είμαι του Παύλου, εγώ δε του Απολλώ, εγώ δε του Κηφά, εγώ δε του Χριστού.
\par 13 Διεμερίσθη ο Χριστός; μήπως ο Παύλος εσταυρώθη διά σας; ή εις το όνομα του Παύλου εβαπτίσθητε;
\par 14 Ευχαριστώ εις τον Θεόν ότι ουδένα από σας εβάπτισα, ειμή Κρίσπον και Γάϊον,
\par 15 διά να μη είπη τις ότι εις το όνομά μου εβάπτισα.
\par 16 Εβάπτισα δε και τον οίκον του Στεφανά· εκτός τούτων δεν εξεύρω εάν εβάπτισα άλλον τινά.
\par 17 Διότι δεν με απέστειλεν ο Χριστός διά να βαπτίζω, αλλά διά να κηρύττω το ευαγγέλιον, ουχί εν σοφία λόγου, διά να μη ματαιωθή ο σταυρός του Χριστού.
\par 18 Διότι ο λόγος του σταυρού εις μεν τους απολλυμένους είναι μωρία, εις ημάς δε τους σωζομένους είναι δύναμις Θεού.
\par 19 Επειδή είναι γεγραμμένον· Θέλω απολέσει την σοφίαν των σοφών, και θέλω αθετήσει την σύνεσιν των συνετών.
\par 20 Που ο σοφός; που ο γραμματεύς; που ο συζητητής του αιώνος τούτου; δεν εμώρανεν ο Θεός την σοφίαν του κόσμου τούτου;
\par 21 Διότι επειδή εν τη σοφία του Θεού ο κόσμος δεν εγνώρισε τον Θεόν διά της σοφίας, ηυδόκησεν ο Θεός διά της μωρίας του κηρύγματος να σώση τους πιστεύοντας.
\par 22 Επειδή και οι Ιουδαίοι σημείον αιτούσι και οι Έλληνες σοφίαν ζητούσιν,
\par 23 ημείς δε κηρύττομεν Χριστόν εσταυρωμένον, εις μεν τους Ιουδαίους σκάνδαλον, εις δε τους Έλληνας μωρίαν,
\par 24 εις αυτούς όμως τους προσκεκλημένους, Ιουδαίους τε και Έλληνας, Χριστόν Θεού δύναμιν και Θεού σοφίαν·
\par 25 διότι το μωρόν του Θεού είναι σοφώτερον των ανθρώπων, και το ασθενές του Θεού είναι ισχυρότερον των ανθρώπων.
\par 26 Επειδή βλέπετε την πρόσκλησίν σας, αδελφοί, ότι είσθε ου πολλοί σοφοί κατά σάρκα, ου πολλοί δυνατοί, ου πολλοί ευγενείς.
\par 27 Αλλά τα μωρά του κόσμου εξέλεξεν ο Θεός διά να καταισχύνη τους σοφούς, και τα ασθενή του κόσμου εξέλεξεν ο Θεός διά να καταισχύνη τα ισχυρά,
\par 28 και τα αγενή του κόσμου και τα εξουθενημένα εξέλεξεν ο Θεός, και τα μη όντα, διά να καταργήση τα όντα,
\par 29 διά να μη καυχηθή ουδεμία σαρξ ενώπιον αυτού.
\par 30 Αλλά σεις είσθε εξ αυτού εν Χριστώ Ιησού, όστις εγενήθη εις ημάς σοφία από Θεού, δικαιοσύνη τε και αγιασμός και απολύτρωσις·
\par 31 ώστε, καθώς είναι γεγραμμένον, Ο καυχώμενος εν Κυρίω ας καυχάται.

\chapter{2}

\par 1 Και εγώ, αδελφοί, ότε ήλθον προς εσάς, ήλθον ουχί με υπεροχήν λόγου ή σοφίας κηρύττων εις εσάς την μαρτυρίαν του Θεού.
\par 2 Διότι απεφάσισα να μη εξεύρω μεταξύ σας άλλο τι ειμή Ιησούν Χριστόν, και τούτον εσταυρωμένον.
\par 3 Και εγώ ήλθον προς εσάς με ασθένειαν και με φόβον και με τρόμον πολύν,
\par 4 και ο λόγος μου και το κήρυγμά μου δεν εγίνοντο με καταπειστικούς λόγους ανθρωπίνης σοφίας, αλλά με απόδειξιν Πνεύματος και δυνάμεως,
\par 5 διά να ήναι η πίστις σας ουχί διά της σοφίας των ανθρώπων, αλλά διά της δυνάμεως του Θεού.
\par 6 Λαλούμεν δε σοφίαν μεταξύ των τελείων, σοφίαν όμως ουχί του αιώνος τούτου, ουδέ των αρχόντων του αιώνος τούτου, των φθειρομένων·
\par 7 αλλά λαλούμεν σοφίαν Θεού μυστηριώδη, την αποκεκρυμμένην, την οποίαν προώρισεν ο Θεός προ των αιώνων εις δόξαν ημών,
\par 8 την οποίαν ουδείς των αρχόντων του αιώνος τούτου εγνώρισε· διότι αν ήθελον γνωρίσει, δεν ήθελον σταυρώσει τον Κύριον της δόξης·
\par 9 αλλά καθώς είναι γεγραμμένον, Εκείνα τα οποία οφθαλμός δεν είδε και ωτίον δεν ήκουσε και εις καρδίαν ανθρώπου δεν ανέβησαν, τα οποία ο Θεός ητοίμασεν εις τους αγαπώντας αυτόν.
\par 10 Εις ημάς δε ο Θεός απεκάλυψεν αυτά διά του Πνεύματος αυτού· επειδή το Πνεύμα ερευνά τα πάντα και τα βάθη του Θεού.
\par 11 Διότι τις των ανθρώπων γινώσκει τα του ανθρώπου, ειμή το πνεύμα του ανθρώπου το εν αυτώ; Ούτω και τα του Θεού ουδείς γινώσκει ειμή το Πνεύμα του Θεού.
\par 12 Αλλ' ημείς δεν ελάβομεν το πνεύμα του κόσμου, αλλά το πνεύμα το εκ του Θεού, διά να γνωρίσωμεν τα υπό του Θεού χαρισθέντα εις ημάς.
\par 13 Τα οποία και λαλούμεν ουχί με διδακτούς λόγους ανθρωπίνης σοφίας, αλλά με διδακτούς του Πνεύματος του Αγίου, συγκρίνοντες τα πνευματικά προς τα πνευματικά.
\par 14 Ο φυσικός όμως άνθρωπος δεν δέχεται τα του Πνεύματος του Θεού· διότι είναι μωρία εις αυτόν, και δεν δύναται να γνωρίση αυτά, διότι πνευματικώς ανακρίνονται.
\par 15 Ο δε πνευματικός ανακρίνει μεν πάντα, αυτός δε υπ' ουδενός ανακρίνεται.
\par 16 Διότι τις εγνώρισε τον νούν του Κυρίου, ώστε να διδάξη αυτόν; ημείς όμως έχομεν νούν Χριστού.

\chapter{3}

\par 1 Και εγώ, αδελφοί, δεν ηδυνήθην να λαλήσω προς εσάς ως προς πνευματικούς, αλλ' ως προς σαρκικούς, ως προς νήπια εν Χριστώ.
\par 2 Γάλα σας επότισα και ουχί στερεάν τροφήν· διότι δεν ηδύνασθε έτι να δεχθήτε αυτήν. Αλλ' ουδέ τώρα δύνασθε έτι·
\par 3 επειδή έτι σαρκικοί είσθε. Διότι ενώ είναι μεταξύ σας φθόνος και έρις και διχόνοιαι, δεν είσθε σαρκικοί και περιπατείτε κατά άνθρωπον;
\par 4 Διότι όταν λέγη τις, Εγώ μεν είμαι του Παύλου, άλλος δε, Εγώ του Απολλώ· δεν είσθε σαρκικοί;
\par 5 Τις λοιπόν είναι ο Παύλος, και τις ο Απολλώς, παρά υπηρέται, διά των οποίων επιστεύσατε και, όπως ο Κύριος έδωκεν εις έκαστον;
\par 6 Εγώ εφύτευσα, ο Απολλώς επότισεν, αλλ' ο Θεός ηύξησεν·
\par 7 ώστε ούτε ο φυτεύων είναι τι ούτε ο ποτίζων, αλλ' ο Θεός ο αυξάνων.
\par 8 Ο φυτεύων δε και ο ποτίζων είναι έν· και έκαστος θέλει λάβει τον εαυτού μισθόν κατά τον κόπον αυτού.
\par 9 Διότι του Θεού είμεθα συνεργοί· σεις είσθε του Θεού αγρός, του Θεού οικοδομή.
\par 10 Εγώ κατά την χάριν του Θεού την δοθείσαν εις εμέ ως σοφός αρχιτέκτων θεμέλιον έθεσα, άλλος δε εποικοδομεί· έκαστος όμως ας βλέπη πως εποικοδομεί·
\par 11 διότι θεμέλιον άλλο ουδείς δύναται να θέση παρά το τεθέν, το οποίον είναι ο Ιησούς Χριστός.
\par 12 Εάν δε τις εποικοδομή επί το θεμέλιον τούτο χρυσόν, άργυρον, λίθους τιμίους, ξύλα, χόρτον, καλάμην·
\par 13 εκάστου το έργον θέλει φανερωθή· διότι η ημέρα θέλει φανερώσει αυτό, επειδή διά πυρός ανακαλύπτεται· και το πυρ θέλει δοκιμάσει το έργον εκάστου οποίον είναι.
\par 14 Εάν το έργον τινός, το οποίον επωκοδόμησε μένη, θέλει λάβει μισθόν·
\par 15 εάν το έργον τινός κατακαή, θέλει ζημιωθή, αυτός όμως θέλει σωθή, πλην ούτως ως διά πυρός.
\par 16 Δεν εξεύρετε ότι είσθε ναός Θεού και το Πνεύμα του Θεού κατοικεί εν υμίν;
\par 17 Εάν τις φθείρη τον ναόν του Θεού, τούτον θέλει φθείρει ο Θεός· διότι ο ναός του Θεού είναι άγιος, όστις είσθε σεις.
\par 18 Μηδείς ας μη εξαπατά εαυτόν· εάν τις μεταξύ σας νομίζη ότι είναι σοφός εν τω κόσμω τούτω, ας γείνη μωρός διά να γείνη σοφός.
\par 19 Διότι η σοφία του κόσμου τούτου είναι μωρία παρά τω Θεώ. Επειδή είναι γεγραμμένον· Όστις συλλαμβάνει τους σοφούς εν τη πανουργία αυτών·
\par 20 και πάλιν· Ο Κύριος γινώσκει τους διαλογισμούς των σοφών, ότι είναι μάταιοι.
\par 21 Ώστε μηδείς ας μη καυχάται εις ανθρώπους· διότι τα πάντα είναι υμών,
\par 22 είτε Παύλος είτε Απολλώς είτε Κηφάς είτε κόσμος είτε ζωή είτε θάνατος είτε παρόντα είτε μέλλοντα, τα πάντα είναι υμών,
\par 23 σεις δε του Χριστού, ο δε Χριστός του Θεού.

\chapter{4}

\par 1 Ούτως ας μας θεωρή πας άνθρωπος ως υπηρέτας του Χριστού και οικονόμους των μυστηρίων του Θεού.
\par 2 Το δε επίλοιπον ζητείται μεταξύ των οικονόμων, να ευρεθή έκαστος πιστός.
\par 3 Εις εμέ δε ελάχιστον είναι να ανακριθώ υφ' υμών ή υπό ανθρωπίνης κρίσεως· αλλ' ουδέ ανακρίνω εμαυτόν.
\par 4 Διότι η συνείδησίς μου δεν με ελέγχει εις ουδέν· πλην με τούτο δεν είμαι δεδικαιωμένος· αλλ' ο ανακρίνων με είναι ο Κύριος.
\par 5 Ώστε μη κρίνετε μηδέν προ καιρού, έως αν έλθη ο Κύριος, όστις και θέλει φέρει εις το φως τα κρυπτά του σκότους και θέλει φανερώσει τας βουλάς των καρδιών, και τότε ο έπαινος θέλει γείνει εις έκαστον από του Θεού.
\par 6 Ταύτα δε, αδελφοί, μετέφερα παραδειγματικώς εις εμαυτόν και εις τον Απολλώ διά σας, διά να μάθητε διά του παραδείγματος ημών να μη φρονήτε υπέρ ό,τι είναι γεγραμμένον, διά να μη επαίρησθε εις υπέρ του ενός κατά του άλλου.
\par 7 Διότι τις σε διακρίνει από του άλλου; και τι έχεις, το οποίον δεν έλαβες, εάν δε και έλαβες, τι καυχάσαι ως μη λαβών;
\par 8 Τώρα είσθε κεχορτασμένοι, τώρα επλουτήσατε, εβασιλεύσατε χωρίς ημών· και είθε να εβασιλεύητε, διά να συμβασιλεύσωμεν και ημείς με σας.
\par 9 Διότι νομίζω ότι ο Θεός απέδειξεν ημάς τους αποστόλους εσχάτους ως καταδεδικασμένους εις θάνατον· διότι εγείναμεν θέατρον εις τον κόσμον, και εις αγγέλους και εις ανθρώπους.
\par 10 Ημείς μωροί διά τον Χριστόν, σεις δε φρόνιμοι εν Χριστώ· ημείς ασθενείς, σεις δε ισχυροί· σεις ένδοξοι, ημείς δε άτιμοι.
\par 11 Έως της παρούσης ώρας και πεινώμεν και διψώμεν και γυμνητεύομεν και ραπιζόμεθα και περιπλανώμεθα
\par 12 και κοπιώμεν, εργαζόμενοι με τας ιδίας ημών χείρας· λοιδορούμενοι ευλογούμεν, διωκόμενοι υποφέρομεν,
\par 13 βλασφημούμενοι παρακαλούμεν· ως περικαθάρματα του κόσμου εγείναμεν, σκύβαλον πάντων έως της σήμερον.
\par 14 Δεν γράφω ταύτα προς εντροπήν σας, αλλ' ως τέκνα μου αγαπητά νουθετώ.
\par 15 Διότι εάν έχητε μυρίους παιδαγωγούς εν Χριστώ, δεν έχετε όμως πολλούς πατέρας· επειδή εγώ σας εγέννησα εν Χριστώ Ιησού διά του ευαγγελίου.
\par 16 Σας παρακαλώ, λοιπόν, γίνεσθε μιμηταί μου.
\par 17 Διά τούτο σας έπεμψα τον Τιμόθεον, όστις είναι τέκνον μου αγαπητόν και πιστόν εν Κυρίω, όστις θέλει σας ενθυμίσει τας οδούς μου τας εν Χριστώ, καθώς διδάσκω πανταχού εν πάση εκκλησία.
\par 18 Τινές όμως εφυσιώθησαν, ως εάν εγώ δεν έμελλον να έλθω προς εσάς·
\par 19 πλην θέλω ελθεί ταχέως προς εσάς, εάν ο Κύριος θελήση, και θέλω γνωρίσει ουχί τον λόγον των πεφυσιωμένων, αλλά την δύναμιν·
\par 20 διότι η βασιλεία του Θεού δεν είναι εν λόγω, αλλ' εν δυνάμει.
\par 21 Τι θέλετε; με ράβδον να έλθω προς εσάς, ή με αγάπην και με πνεύμα πραότητος;

\chapter{5}

\par 1 Γενικώς ακούεται ότι είναι μεταξύ σας πορνεία, και τοιαύτη πορνεία, ήτις ουδέ μεταξύ των εθνών ονομάζεται, ώστε να έχη τις την γυναίκα του πατρός αυτού.
\par 2 Και σεις είσθε πεφυσιωμένοι, και δεν επενθήσατε μάλλον, διά να εκβληθή εκ μέσου υμών ο πράξας το έργον τούτο.
\par 3 Διότι εγώ ως απών κατά το σώμα, παρών όμως κατά το πνεύμα, έκρινα ήδη ως παρών τον ούτω πράξαντα τούτο,
\par 4 εν τω ονόματι του Κυρίου ημών Ιησού Χριστού αφού συναχθήτε σεις και το εμόν πνεύμα με την δύναμιν του Κυρίου ημών Ιησού Χριστού
\par 5 να παραδώσητε τον τοιούτον εις τον Σατανάν προς όλεθρον της σαρκός, διά να σωθή το πνεύμα αυτού εν τη ημέρα του Κυρίου Ιησού.
\par 6 Δεν είναι καλόν το καύχημά σας. Δεν εξεύρετε ότι ολίγη ζύμη κάμνει όλον το φύραμα ένζυμον;
\par 7 Καθαρίσθητε λοιπόν από της παλαιάς ζύμης, διά να ήσθε νέον φύραμα, καθώς είσθε άζυμοι. Διότι το πάσχα ημών εθυσιάσθη υπέρ ημών, ο Χριστός·
\par 8 ώστε ας εορτάζωμεν ουχί με ζύμην παλαιάν, ουδέ με ζύμην κακίας και πονηρίας, αλλά με άζυμα ειλικρινείας και αληθείας.
\par 9 Σας έγραψα εν τη επιστολή να μη συναναστρέφησθε με πόρνους,
\par 10 και ουχί διόλου με τους πόρνους του κόσμου τούτου ή με τους πλεονέκτας ή άρπαγας ή ειδωλολάτρας· επειδή τότε πρέπει να εξέλθητε από του κόσμου.
\par 11 Αλλά τώρα σας έγραψα να μη συναναστρέφησθε, εάν τις αδελφός ονομαζόμενος ήναι πόρνος ή πλεονέκτης ή ειδωλολάτρης ή λοίδορος ή μέθυσος ή άρπαξ· με τον τοιούτον μηδέ να συντρώγητε.
\par 12 Διότι τι με μέλει να κρίνω και τους έξω; δεν κρίνετε σεις τους έσω;
\par 13 Τους δε έξω ο Θεός θέλει κρίνει. Όθεν εκβάλετε τον κακόν εκ μέσου υμών.

\chapter{6}

\par 1 Τολμά τις από σας, όταν έχη διαφοράν προς τον άλλον, να κρίνηται ενώπιον των αδίκων και ουχί ενώπιον των αγίων;
\par 2 Δεν εξεύρετε ότι οι άγιοι θέλουσι κρίνει τον κόσμον; και εάν ο κόσμος κρίνηται από σας, ανάξιοι είσθε να κρίνητε ελάχιστα πράγματα;
\par 3 Δεν εξεύρετε ότι αγγέλους θέλομεν κρίνει; πόσω μάλλον βιωτικά;
\par 4 Βιωτικάς λοιπόν κρίσεις εάν έχητε, τους εξουθενημένους εν τη εκκλησία τούτους καθίζετε κριτάς.
\par 5 Προς εντροπήν σας λέγω τούτο. Ούτω δεν υπάρχει μεταξύ σας ουδέ εις σοφός, όστις θέλει δυνηθή να κρίνη ανά μέσον του αδελφού αυτού,
\par 6 αλλά αδελφός κρίνεται με αδελφόν, και τούτο ενώπιον απίστων;
\par 7 Τώρα λοιπόν είναι διόλου ελάττωμα εις εσάς ότι έχετε κρίσεις μεταξύ σας. Διά τι μάλλον δεν αδικείσθε; διά τι μάλλον δεν αποστερείσθε;
\par 8 Αλλά σεις αδικείτε και αποστερείτε, και μάλιστα αδελφούς.
\par 9 Η δεν εξεύρετε ότι οι άδικοι δεν θέλουσι κληρονομήσει την βασιλείαν του Θεού; Μη πλανάσθε· ούτε πόρνοι ούτε ειδωλολάτραι ούτε μοιχοί ούτε μαλακοί ούτε αρσενοκοίται
\par 10 ούτε κλέπται ούτε πλεονέκται ούτε μέθυσοι ούτε λοίδοροι ούτε άρπαγες θέλουσι κληρονομήσει την βασιλείαν του Θεού.
\par 11 Και τοιούτοι υπήρχετέ τινες· αλλά απελούσθητε, αλλά ηγιάσθητε, αλλ' εδικαιώθητε διά του ονόματος του Κυρίου Ιησού και διά του Πνεύματος του Θεού ημών.
\par 12 Πάντα είναι εις την εξουσίαν μου, πλην πάντα δεν συμφέρουσι· πάντα είναι εις την εξουσίαν μου, αλλ' εγώ δεν θέλω εξουσιασθή υπ' ουδενός.
\par 13 Τα φαγητά είναι διά την κοιλίαν και η κοιλία διά τα φαγητά· πλην ο Θεός και ταύτην και ταύτα θέλει καταργήσει· το δε σώμα δεν είναι διά την πορνείαν, αλλά διά τον Κύριον, και ο Κύριος διά το σώμα·
\par 14 ο δε Θεός και τον Κύριον ανέστησε και ημάς θέλει αναστήσει διά της δυνάμεως αυτού.
\par 15 Δεν εξεύρετε ότι τα σώματά σας είναι μέλη του Χριστού; να λάβω λοιπόν τα μέλη του Χριστού και να κάμω αυτά μέλη πόρνης; Μη γένοιτο.
\par 16 Η δεν εξεύρετε ότι ο προσκολλώμενος με την πόρνην είναι εν σώμα; διότι θέλουσιν είσθαι, λέγει, οι δύο εις σάρκα μίαν·
\par 17 όστις όμως προσκολλάται με τον Κύριον είναι εν πνεύμα.
\par 18 Φεύγετε την πορνείαν. Παν αμάρτημα, το οποίον ήθελε πράξει ο άνθρωπος, είναι εκτός του σώματος· ο πορνεύων όμως αμαρτάνει εις το ίδιον αυτού σώμα.
\par 19 Η δεν εξεύρετε ότι το σώμα σας είναι ναός του Αγίου Πνεύματος του εν υμίν, το οποίον έχετε από Θεού, και δεν είσθε κύριοι εαυτών;
\par 20 Διότι ηγοράσθητε διά τιμής· δοξάσατε λοιπόν τον Θεόν διά του σώματός σας και διά του πνεύματός σας, τα οποία είναι του Θεού.

\chapter{7}

\par 1 Περί δε των όσων μοι εγράψατε, καλόν είναι εις τον άνθρωπον να μη εγγίση εις γυναίκα·
\par 2 διά τας πορνείας όμως ας έχη έκαστος την εαυτού γυναίκα, και εκάστη ας έχη τον εαυτής άνδρα.
\par 3 Ο ανήρ ας αποδίδη εις την γυναίκα την οφειλομένην εύνοιαν· ομοίως δε και η γυνή εις τον άνδρα.
\par 4 Η γυνή δεν εξουσιάζει το εαυτής σώμα, αλλ' ο ανήρ· ομοίως δε και ο ανήρ δεν εξουσιάζει το εαυτού σώμα, αλλ' η γυνή.
\par 5 Μη αποστερείτε αλλήλους, εκτός εάν ήναι τι εκ συμφώνου προς καιρόν, διά να καταγίνησθε εις την νηστείαν και εις την προσευχήν· και πάλιν συνέρχεσθε επί το αυτό, διά να μη σας πειράζη ο Σατανάς διά την ακράτειάν σας.
\par 6 Λέγω δε τούτο κατά συγγνώμην, ουχί κατά προσταγήν.
\par 7 Διότι θέλω πάντας τους ανθρώπους να ήναι καθώς και εμαυτόν· αλλ' έκαστος έχει ιδιαίτερον χάρισμα εκ Θεού, άλλος μεν ούτως, άλλος δε ούτως.
\par 8 Λέγω δε προς τους αγάμους και προς τας χήρας, καλόν είναι εις αυτούς εάν μείνωσι καθώς και εγώ.
\par 9 Αλλ' εάν δεν εγκρατεύωνται, ας νυμφευθώσι· διότι καλήτερον είναι να νυμφευθώσι παρά να εξάπτωνται.
\par 10 Εις δε τους νενυμφευμένους παραγγέλλω, ουχί εγώ αλλ' ο Κύριος, να μη χωρισθή η γυνή από του ανδρός αυτής·
\par 11 αλλ' εάν και χωρισθή, ας μένη άγαμος ή ας συνδιαλλαγή με τον άνδρα· και ο ανήρ να μη αφίνη την εαυτού γυναίκα.
\par 12 Προς δε τους λοιπούς εγώ λέγω, ουχί ο Κύριος· Εάν τις αδελφός έχη γυναίκα άπιστον, και αυτή συγκατανεύη να συνοική μετ' αυτού, ας μη αφίνη αυτήν·
\par 13 και γυνή ήτις έχει άνδρα άπιστον, και αυτός συγκατανεύει να συνοική μετ' αυτής, ας μη αφίνη αυτόν.
\par 14 Διότι ο ανήρ ο άπιστος ηγιάσθη διά της γυναικός, και η γυνή η άπιστος ηγιάσθη διά του ανδρός· επειδή άλλως τα τέκνα σας ήθελον είσθαι ακάθαρτα, αλλά τώρα είναι άγια.
\par 15 Εάν δε ο άπιστος χωρίζηται, ας χωρισθή. Ο αδελφός όμως ή αδελφή δεν είναι δεδουλωμένοι εις τα τοιαύτα· ο Θεός όμως προσεκάλεσεν ημάς εις ειρήνην.
\par 16 Διότι τι εξεύρεις, γύναι, αν μέλλης να σώσης τον άνδρα; ή τι εξεύρεις, άνερ, αν μέλλης να σώσης την γυναίκα;
\par 17 Αλλά καθώς ο Θεός εμοίρασεν εις έκαστον, και καθώς ο Κύριος προσεκάλεσεν έκαστον, ούτως ας περιπατή. Και ούτω διατάττω εις πάσας τας εκκλησίας.
\par 18 Προσεκλήθη τις εις την πίστιν περιτετμημένος; Ας μη καλύπτη την περιτομήν. Προσεκλήθη τις απερίτμητος; Ας μη περιτέμνηται.
\par 19 Η περιτομή είναι ουδέν, και η ακροβυστία είναι ουδέν, αλλ' η τήρησις των εντολών του Θεού.
\par 20 Έκαστος εν τη κλήσει, καθ' ην εκλήθη, εν ταύτη ας μένη.
\par 21 Εκλήθης δούλος; μη σε μέλη· αλλ' εάν δύνασαι να γείνης ελεύθερος, μεταχειρίσου τούτο καλήτερα.
\par 22 Διότι όστις δούλος εκλήθη εις τον Κύριον, είναι απελεύθερος του Κυρίου· ομοίως και όστις ελεύθερος εκλήθη, δούλος είναι του Χριστού.
\par 23 Διά τιμής ηγοράσθητε· μη γίνεσθε δούλοι ανθρώπων.
\par 24 Έκαστος, αδελφοί, εις ό,τι εκλήθη, εν τούτω ας μένη παρά τω Θεώ.
\par 25 Περί δε των παρθένων προσταγήν του Κυρίου δεν έχω· αλλά γνώμην δίδω ως ηλεημένος υπό του Κυρίου να ήμαι πιστός.
\par 26 Τούτο λοιπόν νομίζω ότι είναι καλόν διά την παρούσαν ανάγκην, ότι καλόν είναι εις τον άνθρωπον να ήναι ούτως.
\par 27 Είσαι δεδεμένος με γυναίκα; μη ζήτει λύσιν. Είσαι λελυμένος από γυναικός; μη ζήτει γυναίκα.
\par 28 Πλην και εάν νυμφευθής, δεν ημάρτησας· και εάν η παρθένος νυμφευθή, δεν ημάρτησεν· αλλ' οι τοιούτοι θέλουσιν έχει θλίψιν εν τη σαρκί· εγώ δε σας φείδομαι.
\par 29 Λέγω δε τούτο, αδελφοί, ότι ο επίλοιπος καιρός είναι σύντομος, ώστε και οι έχοντες γυναίκας να ήναι ως μη έχοντες,
\par 30 και οι κλαίοντες ως μη κλαίοντες, και οι χαίροντες ως μη χαίροντες, και οι αγοράζοντες ως μη έχοντες κατοχήν,
\par 31 και οι μεταχειριζόμενοι τον κόσμον τούτον ως μηδόλως μεταχειριζόμενοι· διότι το σχήμα του κόσμου τούτου παρέρχεται.
\par 32 Θέλω δε να ήσθε αμέριμνοι. Ο άγαμος μεριμνά τα του Κυρίου, πως να αρέση εις τον Κύριον·
\par 33 ο δε νενυμφευμένος μεριμνά τα του κόσμου, πως να αρέση εις την γυναίκα.
\par 34 Διαφέρει η γυνή και η παρθένος. Η άγαμος μεριμνά τα του Κυρίου, διά να ήναι αγία και το σώμα και το πνεύμα· η δε νενυμφευμένη μεριμνά τα του κόσμου, πως να αρέση εις τον άνδρα.
\par 35 Λέγω δε τούτο διά το συμφέρον υμών αυτών, ουχί διά να βάλω εις εσάς παγίδα, αλλά διά το σεμνοπρεπές, και διά να ήσθε προσκεκολλημένοι εις τον Κύριον χωρίς περισπασμούς.
\par 36 Αλλ' εάν τις νομίζη ότι ασχημονεί προς την παρθένον αυτού, αν παρήλθεν η ακμή αυτής, και πρέπη να γείνη ούτως, ας κάμη ό,τι θέλει· δεν αμαρτάνει· ας υπανδρεύωνται.
\par 37 Όστις όμως στέκει στερεός εν τη καρδία, μη έχων ανάγκην, έχει όμως εξουσίαν περί του ιδίου αυτού θελήματος, και απεφάσισε τούτο εν τη καρδία αυτού, να φυλάττη την εαυτού παρθένον, πράττει καλώς.
\par 38 Ώστε και όστις υπανδρεύει πράττει καλώς, αλλ' ο μη υπανδρεύων πράττει καλήτερα.
\par 39 Η γυνή είναι δεδεμένη διά του νόμου εφ' όσον καιρόν ζη ο ανήρ αυτής· εάν δε ο ανήρ αυτής αποθάνη, είναι ελευθέρα να υπανδρευθή με όντινα θέλει, μόνον να γίνηται τούτο εν Κυρίω.
\par 40 Μακαριωτέρα όμως είναι εάν μείνη ούτω, κατά την εμήν γνώμην· νομίζω δε ότι και εγώ έχω Πνεύμα Θεού.

\chapter{8}

\par 1 Περί δε των ειδωλοθύτων, εξεύρομεν ότι πάντες έχομεν γνώσιν, η γνώσις όμως φυσιοί, η δε αγάπη οικοδομεί.
\par 2 Και εάν τις νομίζη ότι εξεύρει τι, δεν έμαθεν έτι ουδέν καθώς πρέπει να μάθη·
\par 3 αλλ' εάν τις αγαπά τον Θεόν, ούτος γνωρίζεται υπ' αυτού.
\par 4 Περί της βρώσεως λοιπόν των ειδωλοθύτων, εξεύρομεν ότι το είδωλον είναι ουδέν εν τω κόσμω, και ότι δεν υπάρχει ουδείς άλλος Θεός ειμή εις.
\par 5 Διότι αν και ήναι λεγόμενοι θεοί είτε εν τω ουρανώ είτε επί της γης, καθώς και είναι θεοί πολλοί και κύριοι πολλοί,
\par 6 αλλ' εις ημάς είναι εις Θεός ο Πατήρ, εξ ου τα πάντα και ημείς εις αυτόν, και εις Κύριος Ιησούς Χριστός, δι' ου τα πάντα και ημείς δι' αυτού.
\par 7 Αλλά δεν είναι εις πάντας η γνώσις αύτη· τινές δε διά την συνείδησιν του ειδώλου έως σήμερον τρώγουσι το ειδωλόθυτον ως ειδωλόθυτον, και η συνείδησις αυτών ασθενής ούσα μολύνεται.
\par 8 το φαγητόν όμως δεν συνιστά ημάς εις τον Θεόν· διότι ούτε εάν φάγωμεν περισσεύομεν, ούτε εάν δεν φάγωμεν ελαττούμεθα.
\par 9 Πλην προσέχετε μήπως αύτη η εξουσία σας γείνη πρόσκομμα εις τους ασθενείς.
\par 10 Διότι εάν τις ίδη σε, τον έχοντα γνώσιν, ότι κάθησαι εις τράπεζαν εντός ναού ειδώλων, δεν θέλει ενθαρρυνθή η συνείδησις αυτού, ασθενούντος, εις το να τρώγη τα ειδωλόθυτα;
\par 11 Και διά την γνώσιν σου θέλει απολεσθή ο ασθενής αδελφός, διά τον οποίον ο Χριστός απέθανεν.
\par 12 Αμαρτάνοντες δε ούτως εις τους αδελφούς και προσβάλλοντες την ασθενή συνείδησιν αυτών, εις τον Χριστόν αμαρτάνετε.
\par 13 Διά τούτο, εάν το φαγητόν σκανδαλίζη τον αδελφόν μου, δεν θέλω φάγει κρέας εις τον αιώνα, διά να μη σκανδαλίσω τον αδελφόν μου.

\chapter{9}

\par 1 Δεν είμαι απόστολος; δεν είμαι ελεύθερος; δεν είδον τον Ιησούν Χριστόν τον Κύριον ημών; δεν είσθε σεις το έργον μου εν Κυρίω;
\par 2 Αν δεν ήμαι εις άλλους απόστολος, αλλ' εις εσάς τουλάχιστον είμαι· διότι η σφραγίς της αποστολής μου σεις είσθε εν Κυρίω.
\par 3 Η απολογία μου εις τους ανακρίνοντάς με είναι αύτη·
\par 4 μη δεν έχομεν εξουσίαν να φάγωμεν και να πίωμεν;
\par 5 μη δεν έχομεν εξουσίαν να συμπεριφέρωμεν αδελφήν γυναίκα, ως και οι λοιποί απόστολοι και οι αδελφοί του Κυρίου και ο Κηφάς;
\par 6 ή μόνος εγώ και ο Βαρνάβας δεν έχομεν εξουσίαν να μη εργαζώμεθα;
\par 7 Τις ποτέ εκστρατεύει με ίδια αυτού έξοδα; Τις φυτεύει αμπελώνα και δεν τρώγει εκ του καρπού αυτού; ή τις ποιμαίνει ποίμνιον και δεν τρώγει εκ του γάλακτος του ποιμνίου;
\par 8 Μήπως κατά άνθρωπον λαλώ ταύτα; ή δεν λέγει ταύτα και ο νόμος;
\par 9 διότι εν τω νόμω του Μωϋσέως είναι γεγραμμένον· Δεν θέλεις εμφράξει το στόμα βοός αλωνίζοντος. Μήπως μέλει τον Θεόν περί των βοών;
\par 10 ή δι' ημάς βεβαίως λέγει τούτο; διότι δι' ημάς εγράφη, ότι ο αροτριών με ελπίδα πρέπει να αροτριά, και ο αλωνίζων με ελπίδα να μετέχη της ελπίδος αυτού.
\par 11 Εάν ημείς εσπείραμεν εις εσάς τα πνευματικά, μέγα είναι εάν ημείς θερίσωμεν τα σαρκικά σας;
\par 12 Εάν άλλοι μετέχωσι της εφ' υμάς εξουσίας, δεν πρέπει μάλλον ημείς; Αλλά δεν μετεχειρίσθημεν την εξουσίαν ταύτην, αλλ' υποφέρομεν πάντα, διά να μη προξενήσωμεν εμπόδιόν τι εις το ευαγγέλιον του Χριστού.
\par 13 Δεν εξεύρετε ότι οι εργαζόμενοι τα ιερά εκ του ιερού τρώγουσιν, οι ενασχολούμενοι εις το θυσιαστήριον μετά του θυσιαστηρίου λαμβάνουσι μερίδιον;
\par 14 Ούτω και ο Κύριος διέταξεν, οι κηρύττοντες το ευαγγέλιον να ζώσιν εκ του ευαγγελίου.
\par 15 Πλην εγώ ουδέν τούτων μετεχειρίσθην. Ουδέ έγραψα ταύτα διά να γείνη ούτως εις εμέ· διότι καλόν είναι εις εμέ να αποθάνω μάλλον παρά να ματαιώση τις το καύχημά μου.
\par 16 Διότι εάν κηρύττω το ευαγγέλιον, δεν είναι εις εμέ καύχημα· επειδή ανάγκη επίκειται εις εμέ· ουαί δε είναι εις εμέ εάν δεν κηρύττω·
\par 17 επειδή εάν κάμνω τούτο εκουσίως, έχω μισθόν· εάν δε ακουσίως, είμαι εμπεπιστευμένος οικονομίαν.
\par 18 Τις λοιπόν είναι ο μισθός μου; το να κάμω αδάπανον το ευαγγέλιον του Χριστού διά της κηρύξεώς μου, ώστε να μη κάμνω κατάχρησιν της εξουσίας μου εν τω ευαγγελίω.
\par 19 Διότι ελεύθερος ων πάντων εις πάντας εδούλωσα εμαυτόν, διά να κερδήσω τους πλειοτέρους·
\par 20 και έγεινα εις τους Ιουδαίους ως Ιουδαίος, διά να κερδήσω τους Ιουδαίους· εις τους υπό νόμον ως υπό νόμον, διά να κερδήσω τους υπό νόμον·
\par 21 εις τους ανόμους ως άνομος, μη ων άνομος εις τον Θεόν, αλλ' έννομος εις τον Χριστόν, διά να κερδήσω ανόμους·
\par 22 έγεινα εις τους ασθενείς ως ασθενής, διά να κερδήσω τους ασθενείς· εις πάντας έγεινα τα πάντα, διά να σώσω παντί τρόπω τινάς.
\par 23 Κάμνω δε τούτο διά το ευαγγέλιον, διά να γείνω συγκοινωνός αυτού.
\par 24 Δεν εξεύρετε ότι οι τρέχοντες εν τω σταδίω πάντες μεν τρέχουσιν, εις όμως λαμβάνει το βραβείον; ούτω τρέχετε, ώστε να λάβητε αυτό.
\par 25 Πας δε ο αγωνιζόμενος εις πάντα εγκρατεύεται, εκείνοι μεν διά να λάβωσι φθαρτόν στέφανον, ημείς δε άφθαρτον.
\par 26 Εγώ λοιπόν ούτω τρέχω, ουχί ως αβεβαίως, ούτω πυγμαχώ, ουχί ως κτυπών τον αέρα,
\par 27 αλλά δαμάζω το σώμα μου και δουλαγωγώ, μήπως εις άλλους κηρύξας εγώ γείνω αδόκιμος.

\chapter{10}

\par 1 Δεν θέλω δε να αγνοήτε, αδελφοί, ότι οι πατέρες ημών ήσαν πάντες υπό την νεφέλην, και πάντες διά της θαλάσσης διήλθον,
\par 2 και πάντες εις τον Μωϋσήν εβαπτίσθησαν εν τη νεφέλη και εν τη θαλάσση,
\par 3 και πάντες την αυτήν πνευματικήν βρώσιν έφαγον,
\par 4 και πάντες το αυτό πνευματικόν ποτόν έπιον· διότι έπινον από πνευματικής πέτρας ακολουθούσης, η δε πέτρα ήτο ο Χριστός·
\par 5 αλλά δεν ευηρεστήθη ο Θεός εις τους πλειοτέρους εξ αυτών· διότι κατεστρώθησαν εν τη ερήμω.
\par 6 Ταύτα δε έγειναν παραδείγματα ημών, διά να μη ήμεθα ημείς επιθυμηταί κακών, καθώς και εκείνοι επεθύμησαν.
\par 7 Μηδέ γίνεσθε ειδωλολάτραι, καθώς τινές εξ αυτών, ως είναι γεγραμμένον· Εκάθησεν ο λαός διά να φάγη και να πίη, και εσηκώθησαν να παίζωσι.
\par 8 Μηδέ ας πορνεύωμεν, καθώς τινές αυτών επόρνευσαν και έπεσον εν μιά ημέρα εικοσιτρείς χιλιάδες.
\par 9 Μηδέ ας πειράζωμεν τον Χριστόν, καθώς και τινές αυτών επείρασαν και απωλέσθησαν υπό των όφεων.
\par 10 Μηδέ γογγύζετε, καθώς και τινές αυτών εγόγγυσαν, και απωλέσθησαν υπό του εξολοθρευτού.
\par 11 Ταύτα δε πάντα εγίνοντο εις εκείνους παραδείγματα, και εγράφησαν προς νουθεσίαν ημών, εις τους οποίους τα τέλη των αιώνων έφθασαν.
\par 12 Ώστε ο νομίζων ότι ίσταται ας βλέπη μη πέση.
\par 13 Πειρασμός δεν σας κατέλαβεν ειμή ανθρώπινος· πιστός όμως είναι ο Θεός, όστις δεν θέλει σας αφήσει να πειρασθήτε υπέρ την δύναμίν σας, αλλά μετά του πειρασμού θέλει κάμει και την έκβασιν, ώστε να δύνασθε να υποφέρητε.
\par 14 Διά τούτο, αγαπητοί μου, φεύγετε από της ειδωλολατρείας.
\par 15 Λέγω ως προς φρονίμους· κρίνατε σεις τούτο το οποίον λέγω·
\par 16 Το ποτήριον της ευλογίας, το οποίον ευλογούμεν, δεν είναι κοινωνία του αίματος του Χριστού; Ο άρτος, τον οποίον κόπτομεν, δεν είναι κοινωνία του σώματος του Χριστού;
\par 17 διότι εις άρτος, εν σώμα είμεθα οι πολλοί· επειδή πάντες εκ του ενός άρτου μετέχομεν.
\par 18 Βλέπετε τον Ισραήλ κατά σάρκα· οι τρώγοντες τας θυσίας δεν είναι κοινωνοί του θυσιαστηρίου;
\par 19 Τι λοιπόν λέγω; ότι το είδωλον είναι τι; ή ότι το ειδωλόθυτον είναι τι; ουχί·
\par 20 αλλ' ότι εκείνα, τα οποία θυσιάζουσι τα έθνη, εις τα δαιμόνια θυσιάζουσι και ουχί εις τον Θεόν· και δεν θέλω σεις να γίνησθε κοινωνοί των δαιμονίων.
\par 21 Δεν δύνασθε να πίνητε το ποτήριον του Κυρίου και το ποτήριον των δαιμονίων· δεν δύνασθε να ήσθε μέτοχοι της τραπέζης του Κυρίου και της τραπέζης των δαιμονίων.
\par 22 Η τον Κύριον θέλομεν να διεγείρωμεν εις ζηλοτυπίαν; μήπως είμεθα ισχυρότεροι αυτού;
\par 23 Πάντα είναι εις την εξουσίαν μου αλλά πάντα δεν συμφέρουσι· πάντα είναι εις την εξουσίαν μου, αλλά πάντα δεν οικοδομούσι.
\par 24 Μηδείς ας ζητή το εαυτού συμφέρον, αλλ' έκαστος τα του άλλου.
\par 25 Παν το πωλούμενον εν τω μακελλίω τρώγετε, μηδέν εξετάζοντες διά την συνείδησιν·
\par 26 διότι του Κυρίου είναι η γη και το πλήρωμα αυτής.
\par 27 Και εάν τις των απίστων σας προσκαλή και θέλετε να υπάγητε, τρώγετε ό,τι βάλλεται έμπροσθέν σας, μηδέν εξετάζοντες διά την συνείδησιν.
\par 28 Εάν δε τις σας είπη, Τούτο είναι ειδωλόθυτον, μη τρώγετε δι' εκείνον τον φανερώσαντα και διά την συνείδησιν· διότι του Κυρίου είναι η γη και το πλήρωμα αυτής.
\par 29 Συνείδησιν δε λέγω ουχί την ιδικήν σου, αλλά την του άλλου. Επειδή διά τι η ελευθερία μου κρίνεται υπό άλλης συνειδήσεως;
\par 30 Και εάν εγώ μετ' ευχαριστίας μετέχω, διά τι βλασφημούμαι δι' εκείνο, διά το οποίον εγώ ευχαριστώ;
\par 31 Είτε λοιπόν τρώγετε είτε πίνετε είτε πράττετέ τι, πάντα πράττετε εις δόξαν Θεού.
\par 32 Μη γίνεσθε πρόσκομμα μήτε εις Ιουδαίους μήτε εις Έλληνας μήτε εις την εκκλησίαν του Θεού,
\par 33 καθώς και εγώ κατά πάντα αρέσκω εις πάντας, μη ζητών το ιδικόν μου συμφέρον, αλλά το των πολλών, διά να σωθώσι

\chapter{11}

\par 1 Μιμηταί μου γίνεσθε, καθώς και εγώ του Χριστού.
\par 2 Σας επαινώ δε, αδελφοί, ότι εις πάντα με ενθυμείσθε, και κρατείτε τας παραδόσεις, καθώς παρέδωκα εις εσάς.
\par 3 Θέλω δε να εξεύρητε, ότι η κεφαλή παντός ανδρός είναι ο Χριστός, κεφαλή δε της γυναικός ο ανήρ, κεφαλή δε του Χριστού ο Θεός.
\par 4 Πας ανήρ προσευχόμενος ή προφητεύων, εάν έχη κεκαλυμμένην την κεφαλήν, καταισχύνει την κεφαλήν αυτού.
\par 5 Πάσα δε γυνή προσευχομένη ή προφητεύουσα με την κεφαλήν ασκεπή καταισχύνει την κεφαλήν εαυτής· διότι εν και το αυτό είναι με την εξυρισμένην.
\par 6 Επειδή αν δεν καλύπτηται η γυνή, ας κουρεύση και τα μαλλία αυτής· αλλ' εάν ήναι αισχρόν εις γυναίκα να κουρεύη τα μαλλία αυτής ή να ξυρίζηται, ας καλύπτηται.
\par 7 Διότι ο μεν ανήρ δεν χρεωστεί να καλύπτη την κεφαλήν αυτού, επειδή είναι εικών και δόξα του Θεού· η δε γυνή είναι δόξα του ανδρός.
\par 8 Διότι ο ανήρ δεν είναι εκ της γυναικός, αλλ' η γυνή εκ του ανδρός·
\par 9 επειδή δεν εκτίσθη ο ανήρ διά την γυναίκα, αλλ' η γυνή διά τον άνδρα.
\par 10 Διά τούτο η γυνή χρεωστεί να έχη εξουσίαν επί της κεφαλής αυτής διά τους αγγέλους.
\par 11 Πλην ούτε ο ανήρ χωρίς της γυναικός ούτε η γυνή χωρίς του ανδρός υπάρχει εν Κυρίω.
\par 12 Διότι καθώς η γυνή είναι εκ του ανδρός, ούτω και ο ανήρ είναι διά της γυναικός, τα πάντα δε εκ του Θεού.
\par 13 Κρίνατε σεις καθ' εαυτούς· είναι πρέπον γυνή να προσεύχηται εις τον Θεόν ασκεπής;
\par 14 Η ουδέ αυτή η φύσις δεν σας διδάσκει, ότι ανήρ μεν εάν έχη κόμην είναι εις αυτόν ατιμία,
\par 15 γυνή δε εάν έχη κόμην, είναι δόξα εις αυτήν; διότι η κόμη εδόθη εις αυτήν αντί καλύμματος.
\par 16 Εάν τις όμως φαίνηται ότι είναι φιλόνεικος, ημείς τοιαύτην συνήθειαν δεν έχομεν, ουδέ αι εκκλησίαι του Θεού.
\par 17 Ενώ δε παραγγέλλω τούτο, δεν επαινώ ότι συνέρχεσθε ουχί διά το καλήτερον αλλά διά το χειρότερον.
\par 18 Διότι πρώτον μεν όταν συνέρχησθε εις την εκκλησίαν, ακούω ότι υπάρχουσι σχίσματα μεταξύ σας, και μέρος τι πιστεύω·
\par 19 διότι είναι ανάγκη να υπάρχωσι και αιρέσεις μεταξύ σας, διά να γείνωσι φανεροί μεταξύ σας οι δόκιμοι.
\par 20 Όταν λοιπόν συνέρχησθε επί το αυτό, τούτο δεν είναι να φάγητε Κυριακόν δείπνον·
\par 21 διότι έκαστος λαμβάνει προ του άλλου το ίδιον εαυτού δείπνον εν τω καιρώ του τρώγειν, και άλλος μεν πεινά, άλλος δε μεθύει.
\par 22 Μη δεν έχετε οικίας διά να τρώγητε και να πίνητε; ή την εκκλησίαν του Θεού καταφρονείτε, και καταισχύνετε τους μη έχοντας; τι να σας είπω; να σας επαινέσω εις τούτο; δεν σας επαινώ.
\par 23 Διότι εγώ παρέλαβον από του Κυρίου εκείνο, το οποίον και παρέδωκα εις εσάς, ότι ο Κύριος Ιησούς εν τη νυκτί καθ' ην παρεδίδετο έλαβεν άρτον,
\par 24 και ευχαριστήσας έκοψε και είπε· Λάβετε, φάγετε· τούτο είναι το σώμα μου το υπέρ υμών κλώμενον· τούτο κάμνετε εις την ανάμνησίν μου.
\par 25 Ομοίως και το ποτήριον, αφού εδείπνησε, λέγων· Τούτο το ποτήριον είναι η καινή διαθήκη εν τω αίματί μου· τούτο κάμνετε, οσάκις πίνητε, εις την ανάμνησίν μου.
\par 26 Διότι οσάκις αν τρώγητε τον άρτον τούτον και πίνητε το ποτήριον τούτο, τον θάνατον του Κυρίου καταγγέλλετε, μέχρι της ελεύσεως αυτού.
\par 27 Ώστε όστις τρώγη τον άρτον τούτον ή πίνη το ποτήριον του Κυρίου αναξίως, ένοχος θέλει είσθαι του σώματος και αίματος του Κυρίου.
\par 28 Ας δοκιμάζη δε εαυτόν ο άνθρωπος, και ούτως ας τρώγη εκ του άρτου και ας πίνη εκ του ποτηρίου·
\par 29 διότι ο τρώγων και πίνων αναξίως τρώγει και πίνει κατάκρισιν εις εαυτόν, μη διακρίνων το σώμα του Κυρίου.
\par 30 Διά τούτο υπάρχουσι μεταξύ σας πολλοί ασθενείς και άρρωστοι, και αποθνήσκουσιν ικανοί.
\par 31 Διότι εάν διεκρίνομεν εαυτούς, δεν ηθέλομεν κρίνεσθαι·
\par 32 αλλ' όταν κρινώμεθα, παιδευόμεθα υπό του Κυρίου, διά να μη κατακριθώμεν μετά του κόσμου.
\par 33 Ώστε αδελφοί μου, όταν συνέρχησθε διά να φάγητε, περιμένετε αλλήλους·
\par 34 εάν δε τις πεινά, ας τρώγη εν τη οικία αυτού, διά να μη συνέρχησθε προς κατάκρισιν. Τα δε λοιπά, όταν έλθω, θέλω διατάξει.

\chapter{12}

\par 1 Περί δε των πνευματικών, αδελφοί, δεν θέλω να αγνοήτε.
\par 2 Εξεύρετε ότι ήσθε εθνικοί, συρόμενοι όπως εσύρεσθε προς τα είδωλα τα άφωνα.
\par 3 Διά τούτο σας γνωστοποιώ ότι ουδείς λαλών διά Πνεύματος Θεού λέγει ανάθεμα τον Ιησούν, και ουδείς δύναται να είπη Κύριον Ιησούν, ειμή διά Πνεύματος Αγίου.
\par 4 Είναι δε διαιρέσεις χαρισμάτων, το Πνεύμα όμως το αυτό·
\par 5 είναι και διαιρέσεις διακονιών, ο Κύριος όμως ο αυτός·
\par 6 είναι και διαιρέσεις ενεργημάτων, ο Θεός όμως είναι ο αυτός, ο ενεργών τα πάντα εν πάσι.
\par 7 Δίδεται δε εις έκαστον η φανέρωσις του Πνεύματος προς το συμφέρον.
\par 8 Διότι εις άλλον μεν δίδεται διά του Πνεύματος λόγος σοφίας, εις άλλον δε λόγος γνώσεως κατά το αυτό Πνεύμα,
\par 9 εις άλλον δε πίστις διά του αυτού Πνεύματος, εις άλλον δε χαρίσματα ιαμάτων διά του αυτού Πνεύματος,
\par 10 εις άλλον δε ενέργειαι θαυμάτων, εις άλλον δε προφητεία, εις άλλον δε διακρίσεις πνευμάτων, εις άλλον δε είδη γλωσσών, εις άλλον δε ερμηνεία γλωσσών.
\par 11 Πάντα δε ταύτα ενεργεί το εν και το αυτό Πνεύμα, διάνεμον ιδία εις έκαστον καθώς θέλει.
\par 12 Διότι καθώς το σώμα είναι εν και έχει μέλη πολλά, πάντα δε τα μέλη του σώματος του ενός, πολλά όντα, είναι εν σώμα, ούτω και ο Χριστός·
\par 13 διότι ημείς πάντες διά του ενός Πνεύματος εβαπτίσθημεν εις εν σώμα, είτε Ιουδαίοι είτε Έλληνες, είτε δούλοι είτε ελεύθεροι, και πάντες εις εν Πνεύμα εποτίσθημεν.
\par 14 Διότι το σώμα δεν είναι εν μέλος, αλλά πολλά.
\par 15 Εάν είπη ο πους, Επειδή δεν είμαι χειρ, δεν είμαι εκ του σώματος, διά τούτο τάχα δεν είναι εκ του σώματος;
\par 16 Και εάν είπη το ωτίον, Επειδή δεν είμαι οφθαλμός, δεν είμαι εκ του σώματος, διά τούτο δεν είναι τάχα εκ του σώματος;
\par 17 Εάν όλον το σώμα ήναι οφθαλμός, που η ακοή; Εάν όλον ακοή, που η όσφρησις;
\par 18 Αλλά τώρα ο Θεός έθεσε τα μέλη εν έκαστον αυτών εις το σώμα καθώς ηθέλησεν.
\par 19 Εάν όμως πάντα ήσαν εν μέλος, που το σώμα;
\par 20 Αλλά τώρα είναι μεν πολλά μέλη, εν όμως σώμα.
\par 21 Και δεν δύναται ο οφθαλμός να είπη προς την χείρα· Δεν έχω χρείαν σου· ή πάλιν η κεφαλή προς τους πόδας· Δεν έχω χρείαν υμών.
\par 22 Αλλά πολύ περισσότερον τα μέλη του σώματος, τα οποία φαίνονται ότι είναι ασθενέστερα, ταύτα είναι αναγκαία,
\par 23 και εκείνα τα οποία νομίζομεν ότι είναι τα ατιμότερα του σώματος, εις ταύτα αποδίδομεν τιμήν περισσοτέραν, και τα άσχημα ημών έχουσι περισσοτέραν ευσχημοσύνην·
\par 24 τα δε ευσχήμονα ημών δεν έχουσι χρείαν. Αλλ' ο Θεός συνεκέρασε το σώμα, δώσας περισσοτέραν τιμήν εις το ευτελέστερον,
\par 25 διά να μη ήναι σχίσμα εν τω σώματι, αλλά να φροντίζωσι τα μέλη το αυτό υπέρ αλλήλων.
\par 26 Και είτε πάσχει εν μέλος, πάντα τα μέλη συμπάσχουσιν· είτε τιμάται εν μέλος, πάντα τα μέλη συγχαίρουσι.
\par 27 Και σεις είσθε σώμα Χριστού και μέλη κατά μέρος.
\par 28 Και άλλους μεν έθεσεν ο Θεός εν τη εκκλησία πρώτον αποστόλους, δεύτερον προφήτας, τρίτον διδασκάλους, έπειτα θαύματα, έπειτα χαρίσματα ιαμάτων, βοηθείας, κυβερνήσεις, είδη γλωσσών.
\par 29 Μη πάντες είναι απόστολοι; μη πάντες προφήται; μη πάντες διδάσκαλοι; μη πάντες ενεργούσι θαύματα;
\par 30 μη πάντες έχουσι χαρίσματα ιαμάτων; μη πάντες λαλούσι γλώσσας; μη πάντες διερμηνεύουσι;
\par 31 Ζητείτε δε μετά ζήλου τα καλήτερα χαρίσματα. Και έτι πολύ υπερέχουσαν οδόν σας δεικνύω.

\chapter{13}

\par 1 Εάν λαλώ τας γλώσσας των ανθρώπων και των αγγέλων, αγάπην δε μη έχω, έγεινα χαλκός ηχών ή κύμβαλον αλαλάζον.
\par 2 Και εάν έχω προφητείαν και εξεύρω πάντα τα μυστήρια και πάσαν την γνώσιν, και εάν έχω πάσαν την πίστιν, ώστε να μετατοπίζω όρη, αγάπην δε μη έχω, είμαι ουδέν.
\par 3 Και εάν πάντα τα υπάρχοντά μου διανείμω, και εάν παραδώσω το σώμα μου διά να καυθώ, αγάπην δε μη έχω, ουδέν ωφελούμαι.
\par 4 Η αγάπη μακροθυμεί, αγαθοποιεί, η αγάπη δεν φθονεί, η αγάπη δεν αυθαδιάζει, δεν επαίρεται,
\par 5 δεν ασχημονεί, δεν ζητεί τα εαυτής, δεν παροξύνεται, δεν διαλογίζεται το κακόν,
\par 6 δεν χαίρει εις την αδικίαν, συγχαίρει δε εις την αλήθειαν·
\par 7 πάντα ανέχεται, πάντα πιστεύει, πάντα ελπίζει, πάντα υπομένει.
\par 8 Η αγάπη ουδέποτε εκπίπτει· τα άλλα όμως, είτε προφητείαι είναι, θέλουσι καταργηθή· είτε γλώσσαι, θέλουσι παύσει· είτε γνώσις, θέλει καταργηθή.
\par 9 Διότι κατά μέρος γινώσκομεν και κατά μέρος προφητεύομεν·
\par 10 όταν όμως έλθη το τέλειον, τότε το κατά μέρος θέλει καταργηθή.
\par 11 Ότε ήμην νήπιος, ως νήπιος ελάλουν, ως νήπιος εφρόνουν, ως νήπιος εσυλλογιζόμην· ότε όμως έγεινα ανήρ, κατήργησα τα του νηπίου.
\par 12 Διότι τώρα βλέπομεν διά κατόπτρου αινιγματωδώς, τότε δε πρόσωπον προς πρόσωπον· τώρα γνωρίζω κατά μέρος, τότε δε θέλω γνωρίσει καθώς και εγνωρίσθην.
\par 13 Τώρα δε μένει πίστις, ελπίς, αγάπη, τα τρία ταύτα. μεγαλητέρα δε τούτων είναι η αγάπη.

\chapter{14}

\par 1 Ακολουθείτε την αγάπην· και ζητείτε μετά ζήλου τα πνευματικά, μάλλον δε το να προφητεύητε.
\par 2 Διότι ο λαλών γλώσσαν αγνώριστον δεν λαλεί προς ανθρώπους, αλλά προς τον Θεόν· διότι ουδείς ακούει αυτόν, αλλά με το πνεύμα αυτού λαλεί μυστήρια·
\par 3 ο δε προφητεύων λαλεί προς ανθρώπους εις οικοδομήν και προτροπήν και παρηγορίαν.
\par 4 Ο λαλών γλώσσαν αγνώριστον εαυτόν οικοδομεί, ο δε προφητεύων την εκκλησίαν οικοδομεί.
\par 5 Θέλω δε πάντες να λαλήτε γλώσσας, μάλλον δε να προφητεύητε· διότι ο προφητεύων είναι μεγαλήτερος παρά ο λαλών γλώσσας, εκτός εάν διερμηνεύη, διά να λάβη οικοδομήν η εκκλησία.
\par 6 Και τώρα, αδελφοί, εάν έλθω προς εσάς λαλών γλώσσας, τι θέλω σας ωφελήσει, εάν δεν σας λαλήσω ή με αποκάλυψιν ή με γνώσιν ή με προφητείαν ή με διδαχήν;
\par 7 Και τα άψυχα, όσα δίδουσι φωνήν, είτε αυλός είτε κιθάρα, εάν δεν δώσωσι διακεκριμένους τους φθόγγους, πως θέλει γνωρισθή το αυλούμενον ή το κιθαριζόμενον;
\par 8 Διότι εάν η σάλπιγξ δώση φωνήν ασαφή, τις θέλει ετοιμασθή εις πόλεμον;
\par 9 Ούτω και σεις, εάν δεν δώσητε διά της γλώσσης φωνήν ακατάληπτον, πως θέλει γνωρισθή το λαλούμενον; διότι θέλετε λαλεί εις τον αέρα.
\par 10 Τόσα είδη φωνών είναι τυχόν εν τω κόσμω, και ουδέν εξ αυτών είναι ασήμαντον.
\par 11 Εάν λοιπόν δεν γνωρίσω την σημασίαν της φωνής, θέλω είσθαι προς τον λαλούντα βάρβαρος και ο λαλών βάρβαρος προς εμέ.
\par 12 Ούτω και σεις, επειδή είσθε ζηλωταί πνευματικών, ζητείτε να περισσεύητε εν αυτοίς προς την οικοδομήν της εκκλησίας.
\par 13 Διά τούτο ο λαλών γλώσσαν αγνώριστον ας προσεύχηται διά να γείνη ικανός να διερμηνεύη,
\par 14 διότι εάν προσεύχωμαι με γλώσσαν αγνώριστον, το πνεύμά μου προσεύχεται, αλλ' ο νούς μου είναι ακαρποφόρητος.
\par 15 Τι πρέπει λοιπόν; Θέλω προσευχηθή με το πνεύμα, θέλω δε προσευχηθή και με τον νούν. Θέλω ψάλλει με το πνεύμα, θέλω δε ψάλλει και με τον νούν.
\par 16 Διότι εάν δοξολογήσης με το πνεύμα, εκείνος όστις έχει τάξιν ιδιώτου πως θέλει ειπεί το αμήν εις την ευχαριστίαν σου, μη εξεύρων τι λέγεις;
\par 17 Διότι συ μεν καλώς ευχαριστείς, ο άλλος όμως δεν οικοδομείται.
\par 18 Ευχαριστώ εις τον Θεόν μου ότι λαλώ πλειοτέρας γλώσσας παρά πάντας υμάς·
\par 19 πλην εν τη εκκλησία πέντε λόγους προτιμώ να λαλήσω διά του νοός μου, διά να κατηχήσω και άλλους, παρά μυρίους λόγους με γλώσσαν αγνώριστον.
\par 20 Αδελφοί, μη γίνεσθε παιδία κατά τας φρένας, αλλά γίνεσθε νήπια μεν εις την κακίαν, τέλειοι όμως εις τας φρένας.
\par 21 Εν τω νόμω είναι γεγραμμένον ότι δι' ετερογλώσσων και διά ξένων χειλέων θέλω λαλήσει προς τον λαόν τούτον, και ουδέ ούτω θέλουσι με εισακούσει, λέγει Κύριος.
\par 22 Ώστε αι γλώσσαι είναι διά σημείον ουχί προς τους πιστεύοντας, αλλά προς τους απίστους· η προφητεία όμως είναι ουχί προς τους απίστους, αλλά προς τους πιστεύοντας.
\par 23 Εάν λοιπόν συνέλθη η εκκλησία όλη επί το αυτό και λαλώσι πάντες γλώσσας αγνωρίστους, εισέλθωσι δε ιδιώται ή άπιστοι, δεν θέλουσιν ειπεί ότι είσθε μαινόμενοι;
\par 24 Αλλ' εάν πάντες προφητεύωσιν, εισέλθη δε τις άπιστος η ιδιώτης, ελέγχεται υπό πάντων, ανακρίνεται υπό πάντων,
\par 25 και ούτω τα κρυπτά της καρδίας αυτού γίνονται φανερά· και ούτω πεσών κατά πρόσωπον θέλει προσκυνήσει τον Θεόν, κηρύττων ότι ο Θεός είναι τωόντι εν μέσω υμών.
\par 26 Τι πρέπει λοιπόν, αδελφοί; Όταν συνέρχησθε, έκαστος υμών ψαλμόν έχει, διδαχήν έχει, γλώσσαν έχει, αποκάλυψιν έχει, ερμηνείαν έχει· πάντα ας γίνωνται προς οικοδομήν.
\par 27 Εάν τις λαλή γλώσσαν αγνώριστον, ας κάμωσι τούτο ανά δύο ή το περισσότερον ανά τρεις και εκ διαδοχής, και εις ας διερμηνεύη·
\par 28 αλλ' εάν δεν ήναι διερμηνευτής, ας σιωπά εν τη εκκλησία, ας λαλή δε προς εαυτόν και προς τον Θεόν.
\par 29 Προφήται δε ας λαλώσι δύο ή τρεις, και οι άλλοι ας διακρίνωσιν·
\par 30 εάν δε έλθη αποκάλυψις εις άλλον καθήμενον, ο πρώτος ας σιωπά.
\par 31 Διότι δύνασθε ο εις μετά τον άλλον να προφητεύητε πάντες, διά να μανθάνωσι πάντες και πάντες να παρηγορώνται·
\par 32 και τα πνεύματα των προφητών υποτάσσονται εις τους προφήτας·
\par 33 διότι ο Θεός δεν είναι ακαταστασίας, αλλ' ειρήνης. Καθώς εν πάσαις ταις εκκλησίαις των αγίων.
\par 34 Αι γυναίκές σας ας σιωπώσιν εν ταις εκκλησίαις· διότι δεν είναι συγκεχωρημένον εις αυτάς να λαλώσιν, αλλά να υποτάσσωνται, καθώς και ο νόμος λέγει.
\par 35 Αλλ' εάν θέλωσι να μάθωσι τι, ας ερωτώσιν εν τω οίκω τους άνδρας αυτών· διότι αισχρόν είναι εις γυναίκας να λαλώσιν εν εκκλησία.
\par 36 Μήπως από σας εξήλθεν ο λόγος του Θεού, ή εις σας μόνους κατήντησεν;
\par 37 Εάν τις νομίζη ότι είναι προφήτης ή πνευματικός, ας μάθη εκείνα τα οποία γράφω προς εσάς, ότι είναι εντολαί του Κυρίου.
\par 38 αλλ' εάν τις αγνοή, ας αγνοή.
\par 39 Ώστε, αδελφοί, ζητείτε μετά ζήλου το προφητεύειν, και το λαλείν γλώσσας μη εμποδίζετε·
\par 40 πάντα ας γίνωνται ευσχημόνως και κατά τάξιν.

\chapter{15}

\par 1 Σας φανερόνω δε, αδελφοί, το ευαγγέλιον, το οποίον εκήρυξα προς εσάς, το οποίον και παρελάβετε, εις το οποίον και ίστασθε,
\par 2 διά του οποίου και σώζεσθε, τίνι τρόπω σας εκήρυξα αυτό, αν φυλάττητε αυτό, εκτός εάν επιστεύσατε ματαίως.
\par 3 Διότι παρέδωκα εις εσάς εν πρώτοις εκείνο, το οποίον και παρέλαβον, ότι ο Χριστός απέθανε διά τας αμαρτίας ημών κατά τας γραφάς,
\par 4 και ότι ετάφη, και ότι ανέστη την τρίτην ημέραν κατά τας γραφάς,
\par 5 και ότι εφάνη εις τον Κηφάν, έπειτα εις τους δώδεκα·
\par 6 μετά ταύτα εφάνη εις πεντακοσίους και επέκεινα αδελφούς διά μιας, εκ των οποίων οι πλειότεροι μένουσιν έως τώρα, τινές δε και εκοιμήθησαν·
\par 7 έπειτα εφάνη εις τον Ιάκωβον, έπειτα εις πάντας τους αποστόλους·
\par 8 τελευταίον δε πάντων εφάνη και εις εμέ ως εις έκτρωμα.
\par 9 Διότι εγώ είμαι ο ελάχιστος των αποστόλων, όστις δεν είμαι άξιος να ονομάζωμαι απόστολος, διότι κατεδίωξα την εκκλησίαν του Θεού·
\par 10 αλλά χάριτι Θεού είμαι ότι είμαι· και η εις εμέ χάρις αυτού δεν έγεινε ματαία, αλλά περισσότερον αυτών πάντων εκοπίασα, πλην ουχί εγώ, αλλ' η χάρις του Θεού η μετ' εμού.
\par 11 Είτε λοιπόν εγώ είτε εκείνοι, ούτω κηρύττομεν και ούτως επιστεύσατε.
\par 12 Εάν δε ο Χριστός κηρύττηται ότι ανέστη εκ νεκρών, πως τινές μεταξύ σας λέγουσιν ότι ανάστασις νεκρών δεν είναι;
\par 13 Και εάν ανάστασις νεκρών δεν ήναι, ουδ' ο Χριστός ανέστη·
\par 14 και αν ο Χριστός δεν ανέστη, μάταιον άρα είναι το κήρυγμα ημών, ματαία δε και η πίστις σας.
\par 15 Ευρισκόμεθα δε και ψευδομάρτυρες του Θεού, διότι εμαρτυρήσαμεν περί του Θεού ότι ανέστησε τον Χριστόν, τον οποίον δεν ανέστησεν, εάν καθ' υπόθεσιν δεν ανασταίνωνται νεκροί.
\par 16 Διότι εάν δεν ανασταίνωνται νεκροί, ουδ' ο Χριστός ανέστη·
\par 17 αλλ' εάν ο Χριστός δεν ανέστη, ματαία η πίστις σας· έτι είσθε εν ταις αμαρτίαις υμών.
\par 18 Άρα και οι κοιμηθέντες εν Χριστώ απωλέσθησαν.
\par 19 Εάν εν ταύτη τη ζωή μόνον ελπίζωμεν εις τον Χριστόν, είμεθα ελεεινότεροι πάντων των ανθρώπων.
\par 20 Αλλά τώρα ο Χριστός ανέστη εκ νεκρών, έγεινεν απαρχή των κεκοιμημένων.
\par 21 Διότι επειδή ο θάνατος ήλθε δι' ανθρώπου, ούτω και δι' ανθρώπου η ανάστασις των νεκρών.
\par 22 Επειδή καθώς πάντες αποθνήσκουσιν εν τω Αδάμ, ούτω και πάντες θέλουσι ζωοποιηθή εν τω Χριστώ.
\par 23 Έκαστος όμως κατά την ιδίαν αυτού τάξιν· ο Χριστός είναι η απαρχή, έπειτα όσοι είναι του Χριστού εν τη παρουσία αυτού·
\par 24 Ύστερον θέλει είσθαι το τέλος, όταν παραδώση την βασιλείαν εις τον Θεόν και Πατέρα, όταν καταργήση πάσαν αρχήν και πάσαν εξουσίαν και δύναμιν.
\par 25 Διότι πρέπει να βασιλεύη εωσού θέση πάντας τους εχθρούς υπό τους πόδας αυτού.
\par 26 Έσχατος εχθρός καταργείται ο θάνατος·
\par 27 διότι πάντα υπέταξεν υπό τους πόδας αυτού. Όταν δε είπη ότι πάντα είναι υποτεταγμένα, φανερόν ότι εξαιρείται ο υποτάξας εις αυτόν τα πάντα.
\par 28 Όταν δε υποταχθώσιν εις αυτόν τα πάντα, τότε και αυτός ο Υιός θέλει υποταχθή εις τον υποτάξαντα εις αυτόν τα πάντα, διά να ήναι ο Θεός τα πάντα εν πάσιν.
\par 29 Επειδή τι θέλουσι κάμει οι βαπτιζόμενοι υπέρ των νεκρών, εάν τωόντι οι νεκροί δεν ανασταίνωνται, διά τι και βαπτίζονται υπέρ των νεκρών;
\par 30 διά τι και ημείς κινδυνεύομεν πάσαν ώραν;
\par 31 Καθ' ημέραν αποθνήσκω, μα την εις εσάς καύχησίν μου, την οποίαν έχω εν Χριστώ Ιησού τω Κυρίω ημών.
\par 32 Εάν κατά άνθρωπον επολέμησα με θηρία εν Εφέσω, τι το όφελος εις εμέ; αν οι νεκροί δεν ανασταίνωνται, ας φάγωμεν και ας πίωμεν, διότι αύριον αποθνήσκομεν.
\par 33 Μη πλανάσθε· Φθείρουσι τα καλά ήθη αι κακαί συναναστροφαί.
\par 34 Συνέλθετε εις εαυτούς κατά το δίκαιον και μη αμαρτάνετε· διότι τινές έχουσιν αγνωσίαν Θεού· προς εντροπήν σας λέγω τούτο.
\par 35 Αλλά θέλει τις ειπεί· Πως ανασταίνονται οι νεκροί; και με ποίον σώμα έρχονται;
\par 36 Άφρον, εκείνο το οποίον συ σπείρεις, δεν ζωογονείται εάν δεν αποθάνη·
\par 37 και εκείνο το οποίον σπείρεις, δεν σπείρεις το σώμα το οποίον μέλλει να γείνη, αλλά γυμνόν κόκκον, σίτου τυχόν ή τινός των λοιπών.
\par 38 Ο δε Θεός δίδει εις αυτό σώμα καθώς ηθέλησε, και εις έκαστον των σπερμάτων το ιδιαίτερον αυτού σώμα.
\par 39 Πάσα σαρξ δεν είναι η αυτή σαρξ, αλλά άλλη μεν σαρξ των ανθρώπων, άλλη δε σαρξ των κτηνών, άλλη δε των ιχθύων και άλλη των πτηνών.
\par 40 Είναι και σώματα επουράνια και σώματα επίγεια· πλην άλλη μεν η δόξα των επουρανίων, άλλη δε η των επιγείων.
\par 41 Άλλη δόξα είναι του ηλίου, και άλλη δόξα της σελήνης, και άλλη δόξα των αστέρων· διότι αστήρ διαφέρει αστέρος κατά την δόξαν.
\par 42 Ούτω και η ανάστασις των νεκρών. Σπείρεται εν φθορά, ανίσταται εν αφθαρσία·
\par 43 σπείρεται εν ατιμία, ανίσταται εν δόξη· σπείρεται εν ασθενεία, ανίσταται εν δυνάμει·
\par 44 σπείρεται σώμα ζωϊκόν, ανίσταται σώμα πνευματικόν. Είναι σώμα ζωϊκόν, και είναι σώμα πνευματικόν.
\par 45 Ούτως είναι και γεγραμμένον· Ο πρώτος άνθρωπος Αδάμ έγεινεν εις ψυχήν ζώσαν· ο έσχατος Αδάμ εις πνεύμα ζωοποιούν.
\par 46 Πλην ουχί πρώτον το πνευματικόν, αλλά το ζωϊκόν, έπειτα το πνευματικόν.
\par 47 Ο πρώτος άνθρωπος είναι εκ της γης χοϊκός, ο δεύτερος άνθρωπος ο Κύριος εξ ουρανού.
\par 48 Οποίος ο χοϊκός, τοιούτοι και οι χοϊκοί, και οποίος ο επουράνιος, τοιούτοι και οι επουράνιοι·
\par 49 και καθώς εφορέσαμεν την εικόνα του χοϊκού, θέλομεν φορέσει και την εικόνα του επουρανίου.
\par 50 Τούτο δε λέγω, αδελφοί, ότι σαρξ και αίμα βασιλείαν Θεού δεν δύνανται να κληρονομήσωσιν, ουδέ η φθορά κληρονομεί την αφθαρσίαν.
\par 51 Ιδού, μυστήριον λέγω προς εσάς· πάντες μεν δεν θέλομεν κοιμηθή, πάντες όμως θέλομεν μεταμορφωθή,
\par 52 εν μιά στιγμή, εν ριπή οφθαλμού, εν τη εσχάτη σάλπιγγι· διότι θέλει σαλπίσει, και οι νεκροί θέλουσιν αναστηθή άφθαρτοι, και ημείς θέλομεν μεταμορφωθή.
\par 53 Διότι πρέπει το φθαρτόν τούτο να ενδυθή αφθαρσίαν, και το θνητόν τούτο να ενδυθή αθανασίαν.
\par 54 Όταν δε το φθαρτόν τούτο ενδυθή αφθαρσίαν και το θνητόν τούτο ενδυθή αθανασίαν, τότε θέλει γείνει ο λόγος ο γεγραμμένος· Κατεπόθη ο θάνατος εν νίκη.
\par 55 Που, θάνατε, το κέντρον σου; που, άδη, η νίκη σου;
\par 56 το δε κέντρον του θανάτου είναι η αμαρτία, και η δύναμις της αμαρτίας ο νόμος.
\par 57 Αλλά χάρις εις τον Θεόν, όστις δίδει εις ημάς την νίκην διά του Κυρίου ημών Ιησού Χριστού.
\par 58 Ώστε, αδελφοί μου αγαπητοί, γίνεσθε στερεοί, αμετακίνητοι, περισσεύοντες πάντοτε εις το έργον του Κυρίου, γινώσκοντες ότι ο κόπος σας δεν είναι μάταιος εν Κυρίω.

\chapter{16}

\par 1 Περί δε της συνεισφοράς της υπέρ των αγίων, καθώς διέταξα εις τας εκκλησίας της Γαλατίας, ούτω κάμετε και σεις.
\par 2 Κατά την πρώτην της εβδομάδος έκαστος υμών ας εναποθέτη παρ' εαυτώ θησαυρίζων ό,τι αν ευπορή, ώστε όταν έλθω να μη συνάγωνται τότε συνεισφοραί.
\par 3 Και όταν έλθω, οποίους εγκρίνητε, δι' επιστολών τούτους θέλω πέμψει διά να φέρωσι την δωρεάν σας εις Ιερουσαλήμ·
\par 4 και εάν ήναι άξιον να υπάγω και εγώ, θέλουσιν ελθεί μετ' εμού.
\par 5 Θέλω δε ελθεί προς εσάς, αφού διέλθω την Μακεδονίαν· διότι την Μακεδονίαν διέρχομαι·
\par 6 και ίσως θέλω παραμείνει πλησίον σας, ή και παραχειμάσει, διά να με προπέμψητε σεις όπου αν υπάγω.
\par 7 Διότι δεν θέλω να σας ίδω τώρα εν παρόδω, αλλ' ελπίζω να μείνω πλησίον σας καιρόν τινά, εάν ο Κύριος συγχωρήση τούτο.
\par 8 Θέλω δε μείνει εν Εφέσω έως της πεντηκοστής·
\par 9 διότι ηνοίχθη εις εμέ θύρα μεγάλη και ενεργητική, και είναι πολλοί εναντίοι.
\par 10 Και εάν έλθη ο Τιμόθεος, προσέχετε να ήναι άφοβος μεταξύ σας· διότι το έργον του Κυρίου εργάζεται καθώς και εγώ·
\par 11 μηδείς λοιπόν ας μη εξουθενήση αυτόν. Προπέμψατε δε αυτόν εν ειρήνη, διά να έλθη προς εμέ· διότι προσμένω αυτόν μετά των αδελφών.
\par 12 Περί δε του αδελφού Απολλώ, παρεκάλεσα αυτόν πολλά να έλθη προς εσάς μετά των αδελφών· και δεν ήθελε παντάπασι να έλθη τώρα, θέλει όμως ελθεί όταν ευκαιρήση.
\par 13 Αγρυπνείτε, στέκεσθε εν τη πίστει, ανδρίζεσθε, ενδυναμούσθε.
\par 14 Πάντα τα έργα υμών ας γίνωνται εν αγάπη.
\par 15 Σας παρακαλώ δε, αδελφοί· εξεύρετε την οικίαν του Στεφανά, ότι είναι απαρχή της Αχαΐας και αφιέρωσαν εαυτούς εις την διακονίαν των αγίων·
\par 16 να υποτάσσησθε και σεις εις τους τοιούτους και εις πάντα τον συνεργούντα και κοπιώντα.
\par 17 Χαίρω δε διά την έλευσιν του Στεφανά και Φουρτουνάτου και Αχαϊκού, διότι την έλλειψίν σας ούτοι ανεπλήρωσαν·
\par 18 επειδή ανέπαυσαν το ιδικόν μου πνεύμα και το ιδικόν σας. Τιμάτε λοιπόν τους τοιούτους.
\par 19 Σας ασπάζονται αι εκκλησίαι της Ασίας. Σας ασπάζονται πολλά εν Κυρίω ο Ακύλας και η Πρίσκιλλα μετά της κατ' οίκον αυτών εκκλησίας.
\par 20 Σας ασπάζονται οι αδελφοί πάντες. Ασπάσθητε αλλήλους εν φιλήματι αγίω.
\par 21 Ο ασπασμός εγράφη με την χείρα εμού του Παύλου.
\par 22 Όστις δεν αγαπά τον Κύριον Ιησούν Χριστόν, ας ήναι ανάθεμα. Μαράν αθά.
\par 23 Η χάρις του Κυρίου Ιησού Χριστού είη μεθ' υμών.
\par 24 Η αγάπη μου μετά πάντων υμών εν Χριστώ Ιησού· αμήν.


\end{document}