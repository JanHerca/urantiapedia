\begin{document}

\title{Αριθμοί}


\chapter{1}

\par 1 Και ελάλησε Κύριος προς τον Μωϋσήν εν τη ερήμω Σινά, εν τη σκηνή του μαρτυρίου, την πρώτην του δευτέρου μηνός, εις το δεύτερον έτος αφού εξήλθον εκ γης Αιγύπτου, λέγων,
\par 2 Λάβετε το κεφάλαιον πάσης της συναγωγής των υιών Ισραήλ κατά τας συγγενείας αυτών, κατά τους οίκους των πατέρων αυτών, απαριθμούντες κατ' όνομα παν αρσενικόν κατά κεφαλήν αυτών.
\par 3 Από είκοσι ετών και επάνω, πάντας τους δυναμένους εν τω Ισραήλ να εξέλθωσιν εις πόλεμον, συ και ο Ααρών απαριθμήσατε αυτούς κατά τα στρατεύματα αυτών.
\par 4 Και με σας θέλει είσθαι εις άνθρωπος αφ' εκάστης φυλής· έκαστος άρχων του οίκου των πατέρων αυτού θέλει είσθαι.
\par 5 Και ταύτα είναι τα ονόματα των ανδρών οίτινες θέλουσι παρασταθή με σάς· εκ του Ρουβήν, Ελισούρ ο υιός του Σεδιούρ·
\par 6 εκ του Συμεών, Σελουμήλ ο υιός του Σουρισαδαΐ.
\par 7 εκ του Ιούδα, Ναασσών ο υιός του Αμμιναδάβ·
\par 8 εκ του Ισσάχαρ, Ναθαναήλ ο υιός του Σουάρ·
\par 9 εκ του Ζαβουλών, Ελιάβ ο υιός του Χαιλών·
\par 10 εκ των υιών του Ιωσήφ, εκ μεν του Εφραΐμ, Ελισαμά ο υιός του Αμμιούδ· εκ δε του Μανασσή, Γαμαλιήλ ο υιός του Φεδασσούρ·
\par 11 εκ του Βενιαμίν, Αβειδάν ο υιός του Γιδεωνί·
\par 12 εκ του Δαν, Αχιέζερ ο υιός του Αμμισαδαΐ·
\par 13 εκ του Ασήρ, Φαγαιήλ ο υιός του Οχράν·
\par 14 εκ του Γαδ, Ελιασάφ ο υιός του Δεουήλ·
\par 15 εκ του Νεφθαλί, Αχιρά ο υιός του Αινάν.
\par 16 Ούτοι ήσαν οι εκλεκτοί της συναγωγής, άρχοντες των φυλών των πατέρων αυτών, αρχηγοί των χιλιάδων του Ισραήλ.
\par 17 Ο Μωϋσής λοιπόν και ο Ααρών έλαβον τους άνδρας τούτους τους ονομασθέντας κατ' όνομα.
\par 18 και συνεκάλεσαν πάσαν την συναγωγήν, την πρώτην του δευτέρου μηνός, και κατεγράφησαν κατά τας συγγενείας αυτών, κατά τους οίκους των πατέρων αυτών, κατά τον αριθμόν των ονομάτων, από είκοσι ετών και επάνω, κατά κεφαλήν αυτών.
\par 19 Καθώς προσέταξεν ο Κύριος εις τον Μωϋσήν, ούτως απηρίθμησεν αυτούς εν τη ερήμω Σινά.
\par 20 Και οι υιοί Ρουβήν, του πρωτοτόκου του Ισραήλ, κατά τας γενεάς αυτών, κατά τας συγγενείας αυτών, κατά τους οίκους των πατέρων αυτών, κατά τον αριθμόν των ονομάτων, κατά κεφαλήν αυτών, παν αρσενικόν από είκοσι ετών και επάνω, πάντες οι δυνάμενοι να εξέλθωσιν εις πόλεμον,
\par 21 οι απαριθμηθέντες εκ της φυλής Ρουβήν ήσαν τεσσαράκοντα εξ χιλιάδες και πεντακόσιοι.
\par 22 Εκ των υιών Συμεών, κατά τας γενεάς αυτών, κατά τας συγγενείας αυτών, κατά τους οίκους των πατέρων αυτών, οι απαριθμηθέντες κατά τον αριθμόν των ονομάτων εξ αυτών, κατά κεφαλήν αυτών, παν αρσενικόν από είκοσι ετών και επάνω, πάντες οι δυνάμενοι να εξέλθωσιν εις πόλεμον,
\par 23 οι απαριθμηθέντες εκ της φυλής Συμεών ήσαν πεντήκοντα εννέα χιλιάδες και τριακόσιοι.
\par 24 Εκ των υιών Γαδ, κατά τας γενεάς αυτών, κατά τας συγγενείας αυτών, κατά τους οίκους των πατέρων αυτών, κατά τον αριθμόν των ονομάτων, από είκοσι ετών και επάνω, πάντες οι δυνάμενοι να εξέλθωσιν εις πόλεμον,
\par 25 οι απαριθμηθέντες εκ της φυλής Γαδ ήσαν τεσσαράκοντα πέντε χιλιάδες και εξακόσιοι πεντήκοντα.
\par 26 Εκ των υιών Ιούδα, κατά τας γενεάς αυτών, κατά τας συγγενείας αυτών, κατά τους οίκους των πατέρων αυτών, κατά τον αριθμόν των ονομάτων, από είκοσι ετών και επάνω, πάντες οι δυνάμενοι να εξέλθωσιν εις πόλεμον,
\par 27 οι απαριθμηθέντες εκ της φυλής Ιούδα ήσαν εβδομήκοντα τέσσαρες χιλιάδες και εξακόσιοι.
\par 28 Εκ των υιών Ισσάχαρ, κατά τας γενεάς αυτών, κατά τας συγγενείας αυτών, κατά τους οίκους των πατέρων αυτών, κατά τον αριθμόν των ονομάτων, από είκοσι ετών και επάνω, πάντες οι δυνάμενοι να εξέλθωσιν εις πόλεμον,
\par 29 οι απαριθμηθέντες εκ της φυλής Ισσάχαρ ήσαν πεντήκοντα τέσσαρες χιλιάδες και τετρακόσιοι.
\par 30 Εκ των υιών Ζαβουλών, κατά τας γενεάς αυτών, κατά τας συγγενείας αυτών, κατά τους οίκους των πατέρων αυτών, κατά τον αριθμόν των ονομάτων, από είκοσι ετών και επάνω, πάντες οι δυνάμενοι να εξέλθωσιν εις πόλεμον,
\par 31 οι απαριθμηθέντες εκ της φυλής Ζαβουλών ήσαν πεντήκοντα επτά χιλιάδες και τετρακόσιοι.
\par 32 Εκ των υιών Ιωσήφ, εκ μεν των υιών Εφραΐμ, κατά τας γενεάς αυτών, κατά τας συγγενείας αυτών, κατά τους οίκους των πατέρων αυτών, κατά τον αριθμόν των ονομάτων, από είκοσι ετών και επάνω, πάντες οι δυνάμενοι να εξέλθωσιν εις πόλεμον,
\par 33 οι απαριθμηθέντες εκ της φυλής Εφραΐμ ήσαν τεσσαράκοντα χιλιάδες και πεντακόσιοι.
\par 34 Εκ των υιών Μανασσή, κατά τας γενεάς αυτών, κατά τας συγγενείας αυτών, κατά τους οίκους των πατέρων αυτών, κατά τον αριθμόν των ονομάτων, από είκοσι ετών και επάνω, πάντες οι δυνάμενοι να εξέλθωσιν εις πόλεμον,
\par 35 οι απαριθμηθέντες εκ της φυλής Μανασσή ήσαν τριάκοντα δύο χιλιάδες και διακόσιοι.
\par 36 Εκ των υιών Βενιαμίν, κατά τας γενεάς αυτών, κατά τας συγγενείας αυτών, κατά τους οίκους των πατέρων αυτών, κατά τον αριθμόν των ονομάτων, από είκοσι ετών και επάνω, πάντες οι δυνάμενοι να εξέλθωσιν εις πόλεμον,
\par 37 οι απαριθμηθέντες εκ της φυλής Βενιαμίν ήσαν τριάκοντα πέντε χιλιάδες και τετρακόσιοι.
\par 38 Εκ των υιών Δαν, κατά τας γενεάς αυτών, κατά τας συγγενείας αυτών, κατά τους οίκους των πατέρων αυτών, κατά τον αριθμόν των ονομάτων, από είκοσι ετών και επάνω, πάντες οι δυνάμενοι να εξέλθωσιν εις πόλεμον,
\par 39 οι απαριθμηθέντες εκ της φυλής Δαν ήσαν εξήκοντα δύο χιλιάδες και επτακόσιοι.
\par 40 Εκ των υιών Ασήρ, κατά τας γενεάς αυτών, κατά τας συγγενείας αυτών, κατά τους οίκους των πατέρων αυτών, κατά τον αριθμόν των ονομάτων, από είκοσι ετών και επάνω, πάντες οι δυνάμενοι να εξέλθωσιν εις πόλεμον,
\par 41 οι απαριθμηθέντες εκ της φυλής Ασήρ ήσαν χιλιάδες τεσσαράκοντα μία και πεντακόσιοι.
\par 42 Εκ των υιών Νεφθαλί, κατά τας γενεάς αυτών, κατά τας συγγενείας αυτών, κατά τους οίκους των πατέρων αυτών, κατά τον αριθμόν των ονομάτων, από είκοσι ετών και επάνω, πάντες οι δυνάμενοι να εξέλθωσιν εις πόλεμον,
\par 43 οι απαριθμηθέντες εκ της φυλής Νεφθαλί ήσαν πεντήκοντα τρεις χιλιάδες και τετρακόσιοι.
\par 44 Ούτοι είναι οι απαριθμηθέντες, τους οποίους απηρίθμησεν ο Μωϋσής και ο Ααρών και οι άρχοντες του Ισραήλ, οι δώδεκα άνδρες· έκαστος ήτο κατά τον οίκον των πατέρων αυτού.
\par 45 Και ήσαν πάντες οι απαριθμηθέντες εκ των υιών Ισραήλ, κατά τους οίκους των πατέρων αυτών, από είκοσι ετών και επάνω, πάντες οι δυνάμενοι μεταξύ του Ισραήλ να εξέλθωσιν εις πόλεμον,
\par 46 πάντες οι απαριθμηθέντες ήσαν εξακόσιαι τρεις χιλιάδες και πεντακόσιοι πεντήκοντα.
\par 47 Οι Λευΐται όμως, κατά την φυλήν των πατέρων αυτών, δεν απηριθμήθησαν μεταξύ αυτών.
\par 48 Διότι ο Κύριος είχε λαλήσει προς τον Μωϋσήν, λέγων,
\par 49 Μόνον την φυλήν του Λευΐ μη απαριθμήσης και το κεφάλαιον αυτών μη λάβης μετά των υιών Ισραήλ·
\par 50 αλλά δος εις τους Λευΐτας την επιστασίαν της σκηνής του μαρτυρίου και πάντων των σκευών αυτής και πάντων των ανηκόντων εις αυτήν· ούτοι θέλουσι βαστάζει την σκηνήν και πάντα τα σκεύη αυτής, και ούτοι θέλουσιν υπηρετεί εις αυτήν, και θέλουσι στρατοπεδεύει κύκλω της σκηνής.
\par 51 Και όταν η σκηνή μέλλη να σηκωθή, οι Λευΐται θέλουσι καταβιβάζει αυτήν· και όταν η σκηνή πρέπη να στηθή, οι Λευΐται θέλουσι στήνει αυτήν· και όστις ξένος πλησιάση, ας θανατόνηται.
\par 52 Και οι μεν υιοί Ισραήλ θέλουσι στρατοπεδεύει, έκαστος εν τω στρατοπέδω αυτού, και έκαστος πλησίον της σημαίας αυτού κατά τα στρατεύματα αυτών.
\par 53 Οι δε Λευΐται θέλουσι στρατοπεδεύει κύκλω της σκηνής του μαρτυρίου, διά να μη ήναι οργή επί την συναγωγήν των υιών Ισραήλ· και οι Λευΐται θέλουσι φυλάττει τας φυλακάς της σκηνής του μαρτυρίου.
\par 54 Και έκαμον οι υιοί Ισραήλ κατά πάντα όσα προσέταξεν ο Κύριος εις τον Μωϋσήν· ούτως έκαμον.

\chapter{2}

\par 1 Και ελάλησε Κύριος προς τον Μωϋσήν και προς τον Ααρών, λέγων,
\par 2 Ας στρατοπεδεύωσιν οι υιοί Ισραήλ, έκαστος πλησίον της σημαίας αυτού, μετά του σημείου του οίκου των πατέρων αυτών· κύκλω της σκηνής του μαρτυρίου κατέναντι θέλουσι στρατοπεδεύει.
\par 3 Και οι μεν προς ανατολάς στρατοπεδεύοντες θέλουσιν είσθαι οι εκ της σημαίας του στρατοπέδου Ιούδα, κατά τα τάγματα αυτών· και ο άρχων των υιών Ιούδα θέλει είσθαι Ναασσών ο υιός του Αμμιναδάβ·
\par 4 το δε στράτευμα αυτού και οι απαριθμηθέντες αυτών ήσαν εβδομήκοντα τέσσαρες χιλιάδες και εξακόσιοι.
\par 5 Και οι στρατοπεδεύοντες πλησίον αυτού θέλουσιν είσθαι η φυλή Ισσάχαρ· και ο άρχων των υιών Ισσάχαρ θέλει είσθαι Ναθαναήλ ο υιός του Σουάρ·
\par 6 το στράτευμα αυτού και οι απαριθμηθέντες αυτών ήσαν πεντήκοντα τέσσαρες χιλιάδες και τετρακόσιοι.
\par 7 Έπειτα η φυλή Ζαβουλών· και ο άρχων των υιών Ζαβουλών θέλει είσθαι Ελιάβ ο υιός του Χαιλών·
\par 8 το δε στράτευμα αυτού και οι απαριθμηθέντες αυτών ήσαν πεντήκοντα επτά χιλιάδες και τετρακόσιοι.
\par 9 Πάντες οι απαριθμηθέντες εν τω στρατοπέδω Ιούδα ήσαν εκατόν ογδοήκοντα εξ χιλιάδες και τετρακόσιοι, κατά τα τάγματα αυτών· ούτοι θέλουσι σηκόνεσθαι πρώτοι.
\par 10 Προς μεσημβρίαν δε θέλει είσθαι σημαία του στρατοπέδου Ρουβήν κατά τα τάγματα αυτών· και ο άρχων των υιών Ρουβήν θέλει είσθαι Ελισούρ ο υιός του Σεδιούρ·
\par 11 το δε στράτευμα αυτού και οι απαριθμηθέντες αυτών ήσαν τεσσαράκοντα εξ χιλιάδες και πεντακόσιοι.
\par 12 Και οι στρατοπεδεύοντες πλησίον αυτού θέλουσιν είσθαι η φυλή Συμεών· και ο άρχων των υιών Συμεών θέλει είσθαι Σελουμιήλ ο υιός του Σουρισαδαΐ·
\par 13 το δε στράτευμα αυτού και οι απαριθμηθέντες αυτών ήσαν πεντήκοντα εννέα χιλιάδες και τριακόσιοι.
\par 14 Έπειτα η φυλή Γάδ· και ο άρχων των υιών Γαδ θέλει είσθαι Ελιασάφ ο υιός του Δεουήλ·
\par 15 το δε στράτευμα αυτού και οι απαριθμηθέντες αυτών ήσαν τεσσαράκοντα πέντε χιλιάδες και εξακόσιοι πεντήκοντα.
\par 16 Πάντες οι απαριθμηθέντες εν τω στρατοπέδω Ρουβήν ήσαν χιλιάδες εκατόν πεντήκοντα μία και τετρακόσιοι πεντήκοντα, κατά τα τάγματα αυτών· ούτοι θέλουσι σηκόνεσθαι δεύτεροι.
\par 17 Έπειτα θέλει σηκόνεσθαι η σκηνή του μαρτυρίου, το στρατόπεδον των Λευϊτών εν τω μέσω των στρατοπέδων· καθώς εστρατοπέδευσαν, ούτω θέλουσι σηκόνεσθαι έκαστος εις την τάξιν αυτού πλησίον της σημαίας αυτών.
\par 18 Προς δυσμάς δε θέλει είσθαι η σημαία του στρατοπέδου του Εφραΐμ κατά τα τάγματα αυτών· και ο άρχων των υιών Εφραΐμ θέλει είσθαι Ελισαμά ο υιός του Αμμιούδ·
\par 19 το δε στράτευμα αυτού και οι απαριθμηθέντες αυτών ήσαν τεσσαράκοντα χιλιάδες και πεντακόσιοι.
\par 20 Και πλησίον αυτού η φυλή Μανασσή· και ο άρχων των υιών Μανασσή θέλει είσθαι Γαμαλιήλ ο υιός του Φεδασσούρ·
\par 21 το δε στράτευμα αυτού και οι απαριθμηθέντες αυτών ήσαν τριάκοντα δύο χιλιάδες και διακόσιοι.
\par 22 Έπειτα η φυλή Βενιαμίν· και ο άρχων των υιών Βενιαμίν θέλει είσθαι Αβειδάν ο υιός του Γιδεωνί·
\par 23 το δε στράτευμα αυτού και οι απαριθμηθέντες αυτών ήσαν τριάκοντα πέντε χιλιάδες και τετρακόσιοι.
\par 24 Πάντες οι απαριθμηθέντες του στρατοπέδου Εφραΐμ ήσαν εκατόν οκτώ χιλιάδες και εκατόν, κατά τα τάγματα αυτών· ούτοι θέλουσι σηκόνεσθαι τρίτοι.
\par 25 Προς βορράν δε θέλει είσθαι η σημαία του στρατοπέδου Δαν, κατά τα τάγματα αυτών· και ο άρχων των υιών Δαν θέλει είσθαι Αχιέζερ ο υιός του Αμμισαδαΐ·
\par 26 το δε στράτευμα αυτού και οι απαριθμηθέντες αυτών ήσαν εξήκοντα δύο χιλιάδες και επτακόσιοι.
\par 27 Και οι στρατοπεδεύοντες πλησίον αυτού θέλουσιν είσθαι η φυλή Ασήρ· και ο άρχων των υιών Ασήρ θέλει είσθαι Φαγαιήλ ο υιός του Οχράν·
\par 28 το δε στράτευμα αυτού και οι απαριθμηθέντες αυτών ήσαν χιλιάδες τεσσαράκοντα μία και πεντακόσιοι.
\par 29 Έπειτα η φυλή Νεφθαλί· και ο άρχων των υιών Νεφθαλί θέλει είσθαι Αχιρά ο υιός του Αινάν·
\par 30 το δε στράτευμα αυτού και οι απαριθμηθέντες αυτών ήσαν πεντήκοντα τρεις χιλιάδες και τετρακόσιοι.
\par 31 Πάντες οι απαριθμηθέντες του στρατοπέδου Δαν ήσαν εκατόν πεντήκοντα επτά χιλιάδες και εξακόσιοι ούτοι θέλουσι σηκόνεσθαι έσχατοι κατά τας σημαίας αυτών.
\par 32 Ούτοι είναι οι απαριθμηθέντες εκ των υιών Ισραήλ κατά τους οίκους των πατέρων αυτών· πάντες οι απαριθμηθέντες εν τοις στρατοπέδοις κατά τα τάγματα αυτών ήσαν εξακόσιαι τρεις χιλιάδες και πεντακόσιοι πεντήκοντα.
\par 33 Οι δε Λευΐται δεν συνηριθμήθησαν μεταξύ των υιών Ισραήλ, καθώς προσέταξε Κύριος εις τον Μωϋσήν.
\par 34 Και έκαμον οι υιοί Ισραήλ κατά πάντα όσα προσέταξε Κύριος εις τον Μωϋσήν· ούτως εστρατοπέδευσαν κατά τας σημαίας αυτών και ούτως εσηκώθησαν έκαστος κατά τας συγγενείας αυτού, κατά τους οίκους των πατέρων αυτού.

\chapter{3}

\par 1 Αύται δε είναι αι γενεαί του Ααρών και του Μωϋσέως, την ημέραν καθ' ην ελάλησεν ο Κύριος προς τον Μωϋσήν επί του όρους Σινά.
\par 2 Και ταύτα είναι τα ονόματα των υιών του Ααρών· Ναδάβ ο πρωτότοκος και Αβιούδ, Ελεάζαρ και Ιθάμαρ.
\par 3 Ταύτα είναι τα ονόματα των υιών του Ααρών, των ιερέων των κεχρισμένων, οίτινες καθιερώθησαν διά να ιερατεύωσιν.
\par 4 Απέθανε δε ο Ναδάβ και ο Αβιούδ ενώπιον του Κυρίου, ενώ προσέφερον πυρ ξένον ενώπιον του Κυρίου εν τη ερήμω Σινά, και τέκνα δεν είχον· και ιεράτευσεν ο Ελεάζαρ και ο Ιθάμαρ ενώπιον Ααρών του πατρός αυτών.
\par 5 Και ελάλησε Κύριος προς τον Μωϋσήν, λέγων,
\par 6 Φέρε την φυλήν του Λευΐ και παράστησον αυτούς έμπροσθεν Ααρών του ιερέως, διά να υπηρετώσιν εις αυτόν.
\par 7 Και θέλουσι φυλάττει τας φυλακάς αυτού και τας φυλακάς πάσης της συναγωγής έμπροσθεν της σκηνής του μαρτυρίου, εκτελούντες τας υπηρεσίας της σκηνής.
\par 8 Και θέλουσι φυλάττει πάντα τα σκεύη της σκηνής του μαρτυρίου και τας φυλακάς των υιών Ισραήλ, εκτελούντες τας υπηρεσίας της σκηνής.
\par 9 Και θέλεις δώσει τους Λευΐτας εις τον Ααρών και εις τους υιούς αυτού· ούτοι είναι δεδομένοι δώρον εις αυτόν εκ των υιών Ισραήλ.
\par 10 Τον δε Ααρών και τους υιούς αυτού θέλεις καταστήσει εις το να εκτελώσι τα της ιερατείας αυτών· όστις δε ξένος πλησιάση θέλει θανατόνεσθαι.
\par 11 Και ελάλησε Κύριος προς τον Μωϋσήν, λέγων,
\par 12 Ιδού, εγώ έλαβον τους Λευΐτας εκ μέσου των υιών Ισραήλ, αντί παντός πρωτοτόκου διανοίγοντος μήτραν εκ των υιών Ισραήλ· και θέλουσιν είσθαι οι Λευΐται εμού.
\par 13 Διότι παν πρωτότοκον είναι εμού· επειδή καθ' ην ημέραν επάταξα παν πρωτότοκον εν γη Αιγύπτου, ηγίασα εις εμαυτόν παν πρωτότοκον εν τω Ισραήλ, από ανθρώπου έως κτήνους· εμού θέλουσιν είσθαι. Εγώ είμαι ο Κύριος.
\par 14 Και ελάλησε Κύριος προς τον Μωϋσήν εν τη ερήμω Σινά, λέγων,
\par 15 Απαρίθμησον τους υιούς Λευΐ κατά τους οίκους των πατέρων αυτών, κατά τας συγγενείας αυτών· παν αρσενικόν από ενός μηνός και επάνω θέλεις απαριθμήσει αυτούς.
\par 16 Και απηρίθμησεν αυτούς ο Μωϋσής, κατά τον λόγον του Κυρίου, ως προσετάχθη.
\par 17 Ήσαν δε ούτοι οι υιοί Λευΐ κατά τα ονόματα αυτών· Γηρσών, και Καάθ, και Μεραρί.
\par 18 Και ταύτα ήσαν τα ονόματα των υιών Γηρσών, κατά τας συγγενείας αυτών· Λιβνί, και Σεμεΐ.
\par 19 Και οι υιοί Καάθ, κατά τας συγγενείας αυτών, Αμράμ, και Ισαάρ, Χεβρών, και Οζιήλ.
\par 20 Και οι υιοί Μεραρί, κατά τας συγγενείας αυτών, Μααλί, και Μουσί. Αύται είναι αι συγγένειαι των Λευϊτών, κατά τους οίκους των πατέρων αυτών.
\par 21 Εκ του Γηρσών ήτο η συγγένεια του Λιβνί, και η συγγένεια του Σεμεΐ· αύται είναι αι συγγένειαι των Γηρσωνιτών.
\par 22 Οι απαριθμηθέντες αυτών κατά τον αριθμόν πάντων των αρσενικών από ενός μηνός και επάνω, οι απαριθμηθέντες αυτών ήσαν επτά χιλιάδες και πεντακόσιοι.
\par 23 Αι συγγένειαι των Γηρσωνιτών θέλουσι στρατοπεδεύει όπισθεν της σκηνής προς δυσμάς.
\par 24 Και ο άρχων του πατρικού οίκου των Γηρσωνιτών θέλει είσθαι Ελιασάφ ο υιός του Λαήλ.
\par 25 Και η φυλακή των υιών Γηρσών εν τη σκηνή του μαρτυρίου θέλει είσθαι η σκηνή και η σκέπη το κάλυμμα αυτής και το καταπέτασμα της θύρας της σκηνής του μαρτυρίου,
\par 26 και τα παραπετάσματα της αυλής και το καταπέτασμα της θύρας της αυλής, τα οποία είναι διά την σκηνήν και διά το θυσιαστήριον κύκλω, και τα σχοινία αυτής διά πάσας τας υπηρεσίας αυτών.
\par 27 Και εκ του Καάθ ήτο η συγγένεια των Αμραμιτών και η συγγένεια των Ισααριτών και η συγγένεια των Χεβρωνιτών και η συγγένεια των Οζιηλιτών· αύται είναι αι συγγένειαι των Κααθιτών.
\par 28 Πάντα τα αρσενικά, από ενός μηνός και επάνω ήσαν κατά τον αριθμόν οκτώ χιλιάδες και εξακόσιοι, οίτινες εφύλαττον τας φυλακάς του αγιαστηρίου.
\par 29 Αι συγγένειαι των υιών Καάθ θέλουσι στρατοπεδεύει εις τα πλάγια της σκηνής προς μεσημβρίαν.
\par 30 Και ο άρχων του πατρικού οίκου των συγγενειών των Κααθιτών θέλει είσθαι Ελισαφάν ο υιός του Οζιήλ.
\par 31 Και η φυλακή αυτών θέλει είσθαι η κιβωτός, και η τράπεζα, και η λυχνία, και τα θυσιαστήρια, και τα σκεύη του αγιαστηρίου διά των οποίων λειτουργούσι, και το καταπέτασμα και πάντα τα προς υπηρεσίαν αυτών.
\par 32 Και Ελεάζαρ ο υιός Ααρών του ιερέως θέλει είσθαι αρχηγός επί των αρχηγών των Λευϊτών, έχων την επιστασίαν των φυλαττόντων τας φυλακάς του αγιαστηρίου.
\par 33 Εκ του Μεραρί ήτο η συγγένεια των Μααλιτών και η συγγένεια των Μουσιτών· αύται είναι αι συγγένειαι του Μεραρί.
\par 34 Και οι απαριθμηθέντες αυτών κατά τον αριθμόν πάντων των αρσενικών από ενός μηνός και επάνω ήσαν εξ χιλιάδες και διακόσιοι.
\par 35 Και ο άρχων του πατρικού οίκου των συγγενειών του Μεραρί ήτο Σουρήλ ο υιός του Αβιχαίλ· ούτοι θέλουσι στρατοπεδεύει εις τα πλάγια της σκηνής προς βορράν.
\par 36 Και υπό την επιστασίαν της φυλακής των υιών Μεραρί θέλουσιν είσθαι αι σανίδες της σκηνής και οι μοχλοί αυτής και οι στύλοι αυτής και τα υποβάσια αυτής, και πάντα τα σκεύη αυτής, και πάντα τα προς υπηρεσίαν αυτής·
\par 37 και οι στύλοι της αυλής κύκλω και τα υποβάσια αυτών και οι πάσσαλοι αυτών, και τα σχοινία αυτών.
\par 38 Οι δε στρατοπεδεύοντες κατά πρόσωπον της σκηνής προς ανατολάς, κατέναντι της σκηνής του μαρτυρίου κατά ανατολάς, θέλουσιν είσθαι Μωϋσής και Ααρών και οι υιοί αυτού, φυλάττοντες τας φυλακάς του αγιαστηρίου αντί των φυλακών των υιών Ισραήλ· και όστις ξένος πλησιάση θέλει θανατόνεσθαι.
\par 39 Πάντες οι απαριθμηθέντες των Λευϊτών, τους οποίους απηρίθμησεν ο Μωϋσής και ο Ααρών διά προσταγής του Κυρίου κατά τας συγγενείας αυτών, πάντα τα αρσενικά από ενός μηνός και επάνω, ήσαν είκοσι δύο χιλιάδες.
\par 40 Και είπε Κύριος προς τον Μωϋσήν, Απαρίθμησον πάντα τα πρωτότοκα αρσενικά των υιών Ισραήλ, από ενός μηνός και επάνω, και λάβε τον αριθμόν των ονομάτων αυτών.
\par 41 Και θέλεις λάβει τους Λευΐτας διά εμέ, εγώ είμαι ο Κύριος αντί πάντων των πρωτοτόκων των υιών Ισραήλ· και τα κτήνη των Λευϊτών αντί πάντων των πρωτοτόκων των κτηνών των υιών Ισραήλ.
\par 42 Και απηρίθμησεν ο Μωϋσής, καθώς προσέταξεν εις αυτόν ο Κύριος, πάντα τα πρωτότοκα των υιών Ισραήλ.
\par 43 Και πάντα τα πρωτότοκα αρσενικά απαριθμηθέντα κατ' όνομα από ενός μηνός και επάνω, κατά την απαρίθμησιν αυτών ήσαν είκοσι δύο χιλιάδες διακόσια εβδομήκοντα τρία.
\par 44 Και ελάλησε Κύριος προς τον Μωϋσήν, λέγων,
\par 45 Λάβε τους Λευΐτας αντί πάντων των πρωτοτόκων των υιών Ισραήλ και τα κτήνη των Λευϊτών αντί των κτηνών αυτών· και οι Λευΐται θέλουσιν είσθαι εμού. Εγώ είμαι ο Κύριος.
\par 46 Και διά εξαγοράν των διακοσίων εβδομήκοντα τριών εκ των πρωτοτόκων των υιών Ισραήλ, οίτινες υπερβαίνουσι τον αριθμόν των Λευϊτών,
\par 47 θέλεις λάβει ανά πέντε σίκλους κατά κεφαλήν, κατά τον σίκλον τον άγιον θέλεις λάβει αυτούς· ο σίκλος είναι είκοσι γερά·
\par 48 και θέλεις δώσει το αργύριον της εξαγοράς του περισσεύοντος αριθμού αυτών εις τον Ααρών και εις τους υιούς αυτού.
\par 49 Και έλαβεν ο Μωϋσής το αργύριον της εξαγοράς των υπερβαινόντων τον αριθμόν των εξαγορασθέντων εις ανταλλαγήν των Λευϊτών·
\par 50 παρά των πρωτοτόκων των υιών Ισραήλ έλαβε το αργύριον, χιλίους τριακοσίους εξήκοντα πέντε σίκλους, κατά τον σίκλον τον άγιον·
\par 51 και έδωκεν ο Μωϋσής το αργύριον της εξαγοράς των υπερβαινόντων εις τον Ααρών και εις τους υιούς αυτού, κατά τον λόγον του Κυρίου, καθώς προσέταξε Κύριος εις τον Μωϋσήν.

\chapter{4}

\par 1 Και ελάλησε Κύριος προς τον Μωϋσήν και προς τον Ααρών, λέγων,
\par 2 Λάβε το κεφάλαιον των υιών Καάθ εκ μέσου των υιών Λευΐ, κατά τας συγγενείας αυτών, κατά τους οίκους των πατέρων αυτών,
\par 3 από τριάκοντα ετών και επάνω έως πεντήκοντα ετών, πάντων των εισερχομένων εις το τάγμα διά να κάμνωσιν εργασίας εν τη σκηνή του μαρτυρίου.
\par 4 Αύτη θέλει είσθαι η υπηρεσία των υιών Καάθ εν τη σκηνή του μαρτυρίου· τα άγια των αγίων.
\par 5 Και όταν σηκόνηται το στρατόπεδον, θέλουσιν έρχεσθαι ο Ααρών και οι υιοί αυτού και θέλουσι καταβιβάσει το καλυπτήριον καταπέτασμα και θέλουσι σκεπάζει δι' αυτού την κιβωτόν του μαρτυρίου·
\par 6 και θέλουσι βάλει επ' αυτήν κάλυμμα εκ δερμάτων θώων και επάνωθεν θέλουσιν εφαπλώσει ύφασμα όλον κυανούν και θέλουσι διαπεράσει τους μοχλούς αυτής.
\par 7 Και επί της τραπέζης της προθέσεως θέλουσιν εφαπλώσει ύφασμα όλον κυανούν και θέλουσι βάλει επ' αυτήν τους δίσκους, και τα θυμιαματοδόχα, και τα λεκάνια, και τα σπονδεία, διά να σπένδωσι· και οι παντοτεινοί άρτοι θέλουσιν είσθαι επ' αυτής·
\par 8 και θέλουσιν εφαπλώσει επ' αυτά ύφασμα κόκκινον και τούτο θέλουσι σκεπάσει διά καλύμματος εκ δερμάτων θώων και θέλουσι διαπεράσει τους μοχλούς αυτής.
\par 9 Και θέλουσι λάβει ύφασμα κυανούν, και θέλουσι περισκεπάσει την λυχνίαν του φωτός, και τους λύχνους αυτής, και τα λυχνοψάλιδα αυτής, και τα υποθέματα αυτής, και πάντα τα ελαιοδόχα αγγεία αυτής, διά των οποίων εκτελούσι τας υπηρεσίας αυτής·
\par 10 και θέλουσι θέσει αυτήν μετά πάντων των σκευών αυτής εντός καλύμματος εκ δερμάτων θώων και θέλουσιν επιθέσει αυτήν επί τους μοχλούς.
\par 11 Επί δε το θυσιαστήριον το χρυσούν θέλουσιν εφαπλώσει ύφασμα κυανούν και τούτο θέλουσι σκεπάσει διά καλύμματος εκ δερμάτων θώων και θέλουσι διαπεράσει τους μοχλούς αυτού.
\par 12 και θέλουσι λάβει πάντα τα σκεύη της υπηρεσίας, διά των οποίων υπηρετούσιν εις τα άγια, και βάλει εις ύφασμα κυανούν και θέλουσι σκεπάσει αυτά διά καλύμματος εκ δερμάτων θώων και επιθέσει εις μοχλούς.
\par 13 Και θέλουσι καθαρίσει το θυσιαστήριον από της στάκτης, και θέλουσι περισκεπάσει αυτό διά υφάσματος πορφυρού·
\par 14 και θέλουσι βάλει επ' αυτό πάντα τα σκεύη αυτού, διά των οποίων εκτελούσι τας υπηρεσίας αυτού, τα θυμιατήρια, τας κρεάγρας, και τα πτυάρια, και τας λεκάνας· πάντα τα σκεύη του θυσιαστηρίου, και θέλουσιν εφαπλώσει επ' αυτό κάλυμμα εκ δερμάτων θώων και διαπεράσει τους μοχλούς αυτού.
\par 15 Και αφού τελειώσωσιν ο Ααρών και οι υιοί αυτού να περισκεπάζωσι τα άγια και πάντα τα σκεύη τα άγια, όταν μέλλη να σηκωθή το στρατόπεδον, τότε θέλουσι πλησιάσει οι υιοί του Καάθ διά να βαστάσωσιν αυτά· και δεν θέλουσιν εγγίσει τα άγια, διά να μη αποθάνωσι· ταύτα είναι τα όσα θέλουσι βαστάζει οι υιοί του Καάθ εν τη σκηνή του μαρτυρίου.
\par 16 Και η επιστασία του Ελεάζαρ υιού του Ααρών του ιερέως θέλει είσθαι το έλαιον του φωτός, και το ευώδες θυμίαμα, και η καθημερινή εξ αλφίτων προσφορά, και το έλαιον του χρίσματος, η επιστασία πάσης της σκηνής, και πάντων των εν αυτή, του αγιαστηρίου, και πάντων των σκευών αυτού.
\par 17 Και ελάλησε Κύριος προς τον Μωϋσήν και προς τον Ααρών, λέγων,
\par 18 Μη εξολοθρεύσητε την φυλήν των συγγενειών των Κααθιτών εκ μέσου των Λευϊτών·
\par 19 αλλά τούτο κάμετε εις αυτούς, διά να ζήσωσι και μη αποθάνωσι προσεγγίζοντες εις τα άγια των αγίων· ο Ααρών και οι υιοί αυτού ας εισέρχωνται και ας διορίζωσιν αυτούς έκαστον εις το έργον αυτού και εις το φορτίον αυτού·
\par 20 ας μη εισέρχωνται όμως να ίδωσιν, όταν περισκεπάζωνται τα άγια, διά να μη αποθάνωσι.
\par 21 Και ελάλησε Κύριος προς τον Μωϋσήν, λέγων,
\par 22 Λάβε το κεφάλαιον και των υιών Γηρσών, κατά τους οίκους των πατέρων αυτών, κατά τας συγγενείας αυτών·
\par 23 από τριάκοντα ετών και επάνω έως πεντήκοντα ετών θέλεις απαριθμήσει αυτούς, πάντας τους εισερχομένους εις το τάγμα, διά να κάμνωσιν εργασίας εν τη σκηνή του μαρτυρίου.
\par 24 Αύτη είναι η υπηρεσία των συγγενειών των Γηρσωνιτών, να υπηρετώσι και να βαστάζωσι·
\par 25 θέλουσι λοιπόν βαστάζει τα παραπετάσματα της σκηνής και την σκηνήν του μαρτυρίου, το κάλυμμα αυτής και το κάλυμμα το εκ δερμάτων θώων το επάνωθεν αυτής και το καταπέτασμα της θύρας της σκηνής του μαρτυρίου,
\par 26 και τα παραπετάσματα της αυλής και το καταπέτασμα της θύρας της πύλης της αυλής, τα οποία είναι διά την σκηνήν, και διά το θυσιαστήριον κύκλω, και τα σχοινία αυτών και πάντα τα σκεύη της υπηρεσίας αυτών και πάντα τα χρησιμεύοντα εις αυτά· ούτω θέλουσιν υπηρετεί.
\par 27 Κατά προσταγήν του Ααρών και των υιών αυτού θέλουσι γίνεσθαι πάσαι αι υπηρεσίαι των υιών των Γηρσωνιτών, εις πάντα τα φορτία αυτών και εις πάσας τας υπηρεσίας αυτών· και σεις θέλετε διορίζει εις αυτούς πάντα όσα οφείλουσι να βαστάζωσιν.
\par 28 Αύτη είναι η υπηρεσία των συγγενειών των υιών των Γηρσωνιτών εν τη σκηνή του μαρτυρίου· και η φυλακή αυτών θέλει είσθαι υπό την επιστασίαν του Ιθάμαρ υιού του Ααρών του ιερέως.
\par 29 Θέλεις απαριθμήσει και τους υιούς του Μεραρί κατά τας συγγενείας αυτών, κατά τους οίκους των πατέρων αυτών·
\par 30 από τριάκοντα ετών και επάνω έως πεντήκοντα ετών θέλεις απαριθμήσει αυτούς πάντας τους εισερχομένους εις το τάγμα, διά να κάμνωσιν εργασίας εν τη σκηνή του μαρτυρίου.
\par 31 Και ταύτα είναι, τα οποία οφείλουσι να βαστάζωσι καθ' όλην την υπηρεσίαν αυτών εν τη σκηνή του μαρτυρίου· αι σανίδες της σκηνής και οι μοχλοί αυτής και οι στύλοι αυτής, και τα υποβάσια αυτής,
\par 32 και οι στύλοι της αυλής κύκλω και τα υποβάσια αυτών και οι πάσσαλοι αυτών και τα σχοινία αυτών, μετά πάντων των σκευών αυτών και πάντα τα προς υπηρεσίαν αυτών· και θέλετε διορίσει κατ όνομα τα σκεύη τα οποία οφείλουσι να βαστάζωσιν.
\par 33 Αύτη είναι η υπηρεσία των συγγενειών των υιών Μεραρί, καθ' όλην την υπηρεσίαν αυτών εν τη σκηνή του μαρτυρίου, υπό την επιστασίαν του Ιθάμαρ υιού του Ααρών του ιερέως.
\par 34 Ο Μωϋσής λοιπόν και ο Ααρών και οι άρχοντες της συναγωγής απηρίθμησαν τους υιούς των Κααθιτών κατά τας συγγενείας αυτών και κατά τους οίκους των πατέρων αυτών,
\par 35 από τριάκοντα ετών και επάνω έως πεντήκοντα ετών, πάντας τους εισερχομένους εις το τάγμα, διά να κάμνωσιν εργασίας εν τη σκηνή του μαρτυρίου·
\par 36 και οι απαριθμηθέντες αυτών κατά τας συγγενείας αυτών ήσαν δύο χιλιάδες επτακόσιοι πεντήκοντα.
\par 37 Ούτοι είναι οι απαριθμηθέντες των συγγενειών των Κααθιτών, πάντες οι υπηρετούντες εν τη σκηνή του μαρτυρίου, τους οποίους απηρίθμησαν ο Μωϋσής και ο Ααρών, καθώς προσέταξε Κύριος διά χειρός του Μωϋσέως.
\par 38 Οι δε απαριθμηθέντες των υιών Γηρσών κατά τας συγγενείας αυτών και κατά τους οίκους των πατέρων αυτών,
\par 39 από τριάκοντα ετών και επάνω έως πεντήκοντα ετών, πάντες οι εισερχόμενοι εις το τάγμα διά να κάμνωσιν εργασίας εν τη σκηνή του μαρτυρίου,
\par 40 οι απαριθμηθέντες αυτών κατά τας συγγενείας αυτών, κατά τους οίκους των πατέρων αυτών, ήσαν δύο χιλιάδες εξακόσιοι τριάκοντα.
\par 41 Ούτοι είναι οι απαριθμηθέντες των συγγενειών των υιών Γηρσών, πάντες οι υπηρετούντες εν τη σκηνή του μαρτυρίου, τους οποίους απηρίθμησαν ο Μωϋσής και ο Ααρών κατά την προσταγήν του Κυρίου.
\par 42 Οι δε απαριθμηθέντες των συγγενειών των υιών Μεραρί κατά τας συγγενείας αυτών, κατά τους οίκους των πατέρων αυτών,
\par 43 από τριάκοντα ετών και επάνω έως πεντήκοντα ετών, πάντες οι εισερχόμενοι εις το τάγμα διά να κάμνωσιν εργασίας εν τη σκηνή του μαρτυρίου,
\par 44 οι απαριθμηθέντες αυτών κατά τας συγγενείας αυτών ήσαν τρεις χιλιάδες και διακόσιοι.
\par 45 Ούτοι είναι οι απαριθμηθέντες των συγγενειών των υιών Μεραρί, τους οποίους απηρίθμησαν ο Μωϋσής και ο Ααρών, καθώς προσέταξε Κύριος διά χειρός του Μωϋσέως.
\par 46 Πάντες οι απαριθμηθέντες των Λευϊτών, τους οποίους απηρίθμησαν ο Μωϋσής και ο Ααρών και οι άρχοντες του Ισραήλ, κατά τας συγγενείας αυτών και κατά τους οίκους των πατέρων αυτών,
\par 47 από τριάκοντα ετών και επάνω έως πεντήκοντα ετών, πάντες οι εισερχόμενοι διά να υπηρετώσιν υπηρεσίαν και να βαστάζωσι το φορτίον εν τη σκηνή του μαρτυρίου,
\par 48 οι απαριθμηθέντες αυτών ήσαν οκτώ χιλιάδες πεντακόσιοι ογδοήκοντα.
\par 49 Απηριθμήθησαν καθώς προσέταξεν ο Κύριος διά χειρός του Μωϋσέως, έκαστος κατά την υπηρεσίαν αυτού και κατά το φορτίον αυτού. Ούτως απηριθμήθησαν υπ' αυτού, καθώς προσέταξε Κύριος εις τον Μωϋσήν.

\chapter{5}

\par 1 Και ελάλησε Κύριος προς τον Μωϋσήν, λέγων,
\par 2 Πρόσταξον τους υιούς Ισραήλ να αποπέμψωσιν από του στρατοπέδου πάντα λεπρόν και πάντα γονόρροιον και πάντα μεμολυσμένον διά νεκρόν·
\par 3 αρσενικόν τε και θηλυκόν αποπέμψατε· έξω του στρατοπέδου αποπέμψατε αυτούς, διά να μη μολύνωσι τα στρατόπεδα αυτών, εν μέσω των οποίων εγώ κατοικώ.
\par 4 Και έκαμον ούτως οι υιοί Ισραήλ και απέπεμψαν αυτούς έξω του στρατοπέδου· καθώς είπεν ο Κύριος προς τον Μωϋσήν, ούτως έκαμον οι υιοί Ισραήλ.
\par 5 Και ελάλησε Κύριος προς τον Μωϋσήν, λέγων,
\par 6 Ειπέ προς τους υιούς Ισραήλ, Όταν ανήρ ή γυνή κάμη τι εκ των αμαρτημάτων των ανθρωπίνων, πράττων παράβασιν εις τον Κύριον, και αμαρτήση η ψυχή εκείνη,
\par 7 τότε θέλει εξομολογηθή την αμαρτίαν αυτού, την οποίαν έπραξε, και θέλει αποδώσει το αδίκημα αυτού μετά του κεφαλαίου τούτου και εις αυτό θέλει προσθέσει το πέμπτον αυτού και θέλει δώσει αυτό εις όντινα ηδίκησεν.
\par 8 Εάν δε ο άνθρωπος δεν έχη συγγενή διά να αποδοθή εις αυτόν το αδίκημα, ας αποδίδεται το αδίκημα εις τον Κύριον προς τον ιερέα, εκτός του κριού της εξιλεώσεως, διά του οποίου θέλει γείνει εξιλέωσις περί αυτού.
\par 9 Και πάσα υψουμένη προσφορά εκ πάντων των ηγιασμένων πραγμάτων των υιών Ισραήλ, την οποίαν προσφέρουσιν εις τον ιερέα, θέλει είσθαι αυτού.
\par 10 Αυτού λοιπόν θέλουσιν είσθαι τα αγιαζόμενα παντός ανθρώπου· ό,τι έκαστος δίδη εις τον ιερέα, θέλει είσθαι αυτού.
\par 11 Και ελάλησε Κύριος προς τον Μωϋσήν, λέγων,
\par 12 Λάλησον προς τους υιούς Ισραήλ και ειπέ προς αυτούς, Εάν ανθρώπου τινός η γυνή παραδρομήση και αμαρτήση εναντίον αυτού,
\par 13 και συγκοιμηθή τις μετ' αυτής, και λανθάση τους οφθαλμούς του ανδρός αυτής και κρυφθή, και αυτή μολυνθή και μάρτυς δεν υπάρχη κατ' αυτής και δεν πιασθή,
\par 14 και επέλθη εις αυτόν πνεύμα ζηλοτυπίας και ζηλοτυπήση την γυναίκα αυτού και αυτή ήναι μεμολυσμένη· ή εάν επέλθη εις αυτόν το πνεύμα της ζηλοτυπίας, και ζηλοτυπήση την γυναίκα αυτού και αυτή δεν ήναι μεμολυσμένη·
\par 15 τότε θέλει φέρει ο άνθρωπος την γυναίκα αυτού προς τον ιερέα και θέλει προσφέρει το δώρον αυτής υπέρ αυτής, το δέκατον του εφά άλευρον κρίθινον· έλαιον όμως δεν θέλει επιχύσει επ' αυτό, ουδέ λιβάνιον θέλει επιθέσει επ' αυτό· διότι είναι προσφορά ζηλοτυπίας, προσφορά ενθυμήσεως, φέρουσα εις ενθύμησιν ανομίαν.
\par 16 Και θέλει πλησιάσει αυτήν ο ιερεύς και στήσει αυτήν ενώπιον του Κυρίου·
\par 17 Έπειτα θέλει λάβει ο ιερεύς ύδωρ άγιον εις αγγείον πήλινον· και από του χώματος, το οποίον είναι επί του εδάφους της σκηνής, θέλει λάβει ο ιερεύς και βάλει εις το ύδωρ.
\par 18 Και θέλει στήσει ο ιερεύς την γυναίκα ενώπιον του Κυρίου και θέλει αποκαλύψει την κεφαλήν της γυναικός, και θέλει βάλει εις τας χείρας αυτής την προσφοράν της ενθυμήσεως, την προσφοράν της ζηλοτυπίας· εν δε τη χειρί του ιερέως θέλει είσθαι το ύδωρ το πικρόν, το οποίον φέρει την κατάραν·
\par 19 Και θέλει ορκίσει αυτήν ο ιερεύς και θέλει ειπεί προς την γυναίκα, Εάν δεν εκοιμήθη τις μετά σου και εάν δεν παρεδρόμησας διά να μολυνθής, δεχομένη άλλον αντί του ανδρός σου, ας ήσαι αβλαβής από του ύδατος τούτου του πικρού, το οποίον φέρει την κατάραν·
\par 20 εάν όμως παρεδρόμησας, δεχομένη άλλον αντί του ανδρός σου, και εμολύνθης και εκοιμήθη τις μετά σου εκτός του ανδρός σου,
\par 21 τότε ο ιερεύς θέλει ορκίσει την γυναίκα μεθ' όρκου κατάρας, και θέλει ειπεί ο ιερεύς προς την γυναίκα, Ο Κύριος να σε καταστήση κατάραν και όρκον μεταξύ του λαού σου, κάμνων ο Κύριος να σαπή ο μηρός σου και να πρησθή η κοιλία σου·
\par 22 και το ύδωρ τούτο, το οποίον φέρει την κατάραν, θέλει εισέλθει εις τα εντόσθιά σου, διά να κάμη να πρησθή η κοιλία σου και να σαπή ο μηρός σου. Και θέλει ειπεί η γυνή, Αμήν, αμήν·
\par 23 Έπειτα θέλει γράψει ο ιερεύς τας κατάρας ταύτας εν βιβλίω και θέλει εξαλείψει αυτάς διά του ύδατος του πικρού·
\par 24 και θέλει ποτίσει την γυναίκα το ύδωρ το πικρόν, το οποίον φέρει την κατάραν· και το ύδωρ, το φέρον την κατάραν, θέλει εισέλθει εις αυτήν διά πικρίαν·
\par 25 Και θέλει λάβει ο ιερεύς εκ της χειρός της γυναικός την προσφοράν της ζηλοτυπίας και θέλει κινήσει την προσφοράν ενώπιον του Κυρίου και θέλει προσφέρει αυτήν εις το θυσιαστήριον·
\par 26 και θέλει δράξει ο ιερεύς από της προσφοράς το μνημόσυνον αυτής και θέλει καύσει επί το θυσιαστήριον, και μετά ταύτα θέλει ποτίσει την γυναίκα το ύδωρ.
\par 27 Και αφού ποτίση αυτήν το ύδωρ τότε θέλει συμβή ώστε, αν ήναι μεμολυσμένη και ηδίκησε τον άνδρα αυτής, θέλει εισέλθει εις αυτήν το ύδωρ, το φέρον την κατάραν, διά πικρίαν, και η κοιλία αυτής θέλει πρησθή και ο μηρός αυτής θέλει σαπή και θέλει είσθαι η γυνή κατάρα εν μέσω του λαού αυτής.
\par 28 Εάν όμως δεν ήναι μεμολυσμένη η γυνή αλλά καθαρά, τότε θέλει μείνει αβλαβής, και θέλει συλλάβει σπέρμα.
\par 29 Ούτος είναι ο νόμος της ζηλοτυπίας, όταν γυνή τις παραδρομήση, δεχομένη άλλον αντί του ανδρός αυτής και μολυνθή·
\par 30 ή όταν έλθη το πνεύμα της ζηλοτυπίας εις άνδρα τινά και ζηλοτυπήση την γυναίκα αυτού και στήση την γυναίκα αυτού ενώπιον του Κυρίου, και ο ιερεύς κάμη εις αυτήν κατά πάντα τον νόμον τούτον·
\par 31 Τότε ο μεν ανήρ θέλει είσθαι αθώος από της ανομίας, η δε γυνή εκείνη θέλει βαστάσει την ανομίαν αυτής.

\chapter{6}

\par 1 Και ελάλησε Κύριος προς τον Μωϋσήν, λέγων,
\par 2 Λάλησον προς τους υιούς Ισραήλ και ειπέ προς αυτούς, Όταν ανήρ ή γυνή ευχηθή ευχήν Ναζηραίου, διά να αφιερωθή εις τον Κύριον,
\par 3 θέλει εγκρατεύεσθαι από οίνου και από σίκερα και δεν θέλει πίει όξος από οίνου ή όξος από σίκερα ούτε θέλει πίει ό,τι είναι κατεσκευασμένον από σταφυλής ούτε θέλει φάγει σταφυλήν πρόσφατον ουδέ σταφίδας.
\par 4 Πάσας τας ημέρας της αφιερώσεως αυτού δεν θέλει φάγει ουδέν εκ των όσα γίνονται εξ αμπέλου, από φλοιού σταφυλής έως κόκκου αυτής.
\par 5 Πάσας τας ημέρας της ευχής της αφιερώσεως αυτού δεν θέλει περάσει ξυράφιον επί της κεφαλής αυτού, εωσού πληρωθώσιν αι ημέραι, τας οποίας ευχήθη εις τον Κύριον· άγιος θέλει είσθαι, αφίνων τας τρίχας της κόμης της κεφαλής αυτού να αυξάνωσι.
\par 6 Πάσας τας ημέρας της αφιερώσεως αυτού εις τον Κύριον δεν θέλει εισέλθει εις τεθνεώτα.
\par 7 Δεν θέλει μολυνθή διά τον πατέρα αυτού ή διά την μητέρα αυτού, διά τον αδελφόν αυτού ή διά την αδελφήν αυτού, όταν αποθάνωσιν· επειδή η προς τον Θεόν αφιέρωσις αυτού είναι επί της κεφαλής αυτού.
\par 8 Πάσας τας ημέρας της αφιερώσεως αυτού είναι άγιος εις τον Κύριον.
\par 9 Και εάν τις αποθάνη εξαίφνης πλησίον αυτού και μολυνθή η κεφαλή της αφιερώσεως αυτού, τότε θέλει ξυρίσει την κεφαλήν αυτού εν τη ημέρα του καθαρισμού αυτού· εις την εβδόμην ημέραν θέλει ξυρίσει αυτήν.
\par 10 Την δε ογδόην ημέραν θέλει φέρει δύο τρυγόνας ή δύο νεοσσούς περιστερών προς τον ιερέα εις την θύραν της σκηνής του μαρτυρίου·
\par 11 και ο ιερεύς θέλει προσφέρει την μίαν εις προσφοράν περί αμαρτίας, την δε άλλην εις ολοκαύτωμα· και θέλει κάμει εξιλέωσιν υπέρ αυτού διά την περί τον νεκρόν αμαρτίαν αυτού και θέλει αγιάσει την κεφαλήν αυτού εν εκείνη τη ημέρα.
\par 12 Και θέλει αφιερώσει εις τον Κύριον τας ημέρας της αφιερώσεως αυτού, και θέλει φέρει αρνίον ενιαύσιον διά προσφοράν περί ανομίας· αι δε ημέραι αι παρελθούσαι δεν θέλουσι λογισθή, διότι εμολύνθη η αφιέρωσις αυτού.
\par 13 Ούτος δε είναι ο νόμος του Ναζηραίου αφού πληρωθώσιν αι ημέραι της αφιερώσεως αυτού· θέλει φερθή εις την θύραν της σκηνής του μαρτυρίου,
\par 14 και θέλει προσφέρει το δώρον αυτού εις τον Κύριον, εν αρνίον ενιαύσιον άμωμον εις ολοκαύτωμα και εν αρνίον θηλυκόν ενιαύσιον άμωμον εις προσφοράν περί αμαρτίας και ένα κριόν άμωμον εις προσφοράν ειρηνικήν,
\par 15 και κάνιστρον άρτων αζύμων εκ σεμιδάλεως εζυμωμένης μετά ελαίου, και λάγανα άζυμα κεχρισμένα μετά ελαίου και την εξ αλφίτων προσφοράν αυτών και την σπονδήν αυτών.
\par 16 Και ο ιερεύς θέλει προσφέρει αυτά ενώπιον του Κυρίου και θέλει κάμει την περί αμαρτίας προσφοράν αυτού και το ολοκαύτωμα αυτού.
\par 17 Και θέλει προσφέρει τον κριόν εις θυσίαν ειρηνικήν προς τον Κύριον μετά του κανίστρου των αζύμων· θέλει προσφέρει έτι ο ιερεύς την εξ αλφίτων προσφοράν αυτού και την σπονδήν αυτού.
\par 18 Και ο Ναζηραίος θέλει ξυρισθή την κεφαλήν της αφιερώσεως αυτού εις την θύραν της σκηνής του μαρτυρίου, και θέλει λάβει τας τρίχας της κεφαλής της αφιερώσεως αυτού και επιθέσει επί του πυρός του υποκάτω της ειρηνικής θυσίας.
\par 19 Και ο ιερεύς θέλει λάβει τον εψημένον ώμον του κριού και ένα άρτον άζυμον εκ του κανίστρου, και εν λάγανον άζυμον και θέλει επιθέσει αυτά επί τας χείρας του Ναζηραίου, αφού ξυρισθή τας τρίχας της αφιερώσεως αυτού.
\par 20 Και θέλει κινήσει αυτά ο ιερεύς εις κινητήν προσφοράν ενώπιον του Κυρίου· τούτο είναι άγιον εις τον ιερέα, μετά του στήθους της κινητής προσφοράς και μετά του ώμου της υψουμένης προσφοράς· και μετά ταύτα δύναται να πίη οίνον ο Ναζηραίος.
\par 21 Του Ναζηραίου, όστις έκαμεν ευχήν, ούτος είναι ο νόμος του δώρου αυτού εις τον Κύριον διά την αφιέρωσιν αυτού, εκτός του ό,τι ήθελε προσφέρει εκουσίως· συμφώνως με την ευχήν, την οποίαν ευχήθη, ούτω θέλει κάμει κατά τον νόμον της αφιερώσεως αυτού.
\par 22 Και ελάλησε Κύριος προς τον Μωϋσήν, λέγων,
\par 23 Λάλησον προς τον Ααρών και προς τους υιούς αυτού, λέγων, Ούτω θέλετε ευλογεί τους υιούς Ισραήλ, λέγοντες προς αυτούς·
\par 24 Ο Κύριος να σε ευλογήση και να σε φυλάξη·
\par 25 Ο Κύριος να επιλάμψη το πρόσωπον αυτού επί σε και να σε ελεήση·
\par 26 Ο Κύριος να υψώση το πρόσωπον αυτού επί σε και να σοι δώση ειρήνην·
\par 27 Και θέλουσιν επιθέσει το όνομά μου επί τους υιούς Ισραήλ· και εγώ θέλω ευλογήσει αυτούς.

\chapter{7}

\par 1 Και την ημέραν, καθ' ην ο Μωϋσής ετελείωσε να στήνη την σκηνήν και έχρισεν αυτήν και ηγίασεν αυτήν και πάντα τα σκεύη αυτής και το θυσιαστήριον και πάντα τα σκεύη αυτού και έχρισεν αυτά και ηγίασεν αυτά·
\par 2 τότε οι άρχοντες του Ισραήλ, οι αρχηγοί των οίκων των πατέρων αυτών οίτινες ήσαν οι άρχοντες των φυλών, οι επιστατήσαντες εις την απαρίθμησιν, έκαμον προσφοράν·
\par 3 και έφεραν τα δώρα αυτών ενώπιον του Κυρίου, εξ αμάξας σκεπαστάς και δώδεκα βόας, μίαν άμαξαν ανά δύο άρχοντες και ένα βουν έκαστος, και έφεραν αυτά έμπροσθεν της σκηνής.
\par 4 Και είπε Κύριος προς τον Μωϋσήν, λέγων,
\par 5 Λάβε ταύτα παρ' αυτών, και θέλουσιν είσθαι διά τα έργα της υπηρεσίας της σκηνής του μαρτυρίου· και θέλεις δώσει αυτά εις τους Λευΐτας, εις έκαστον κατά την υπηρεσίαν αυτού.
\par 6 Και έλαβεν ο Μωϋσής τας αμάξας και τους βόας και έδωκεν αυτά εις τους Λευΐτας.
\par 7 Τας δύο αμάξας και τους τέσσαρας βόας έδωκεν εις τους υιούς Γηρσών, κατά την υπηρεσίαν αυτών.
\par 8 Και τας τέσσαρας αμάξας και τους οκτώ βόας έδωκεν εις τους υιούς Μεραρί· κατά την υπηρεσίαν αυτών, υπό την χείρα του Ιθάμαρ υιού του Ααρών του ιερέως.
\par 9 Εις δε τους υιούς Καάθ δεν έδωκε· διότι η εν τω αγιαστηρίω υπηρεσία αυτών ήτο να βαστάζωσιν επ' ώμων.
\par 10 Και προσέφεραν οι άρχοντες διά τον εγκαινιασμόν του θυσιαστηρίου καθ' ην ημέραν εχρίσθη, και προσέφεραν οι άρχοντες τα δώρα αυτών έμπροσθεν του θυσιαστηρίου.
\par 11 Και είπε Κύριος προς τον Μωϋσήν, Εις άρχων καθ' ημέραν, θέλουσι προσφέρει τα δώρα αυτών διά τον εγκαινιασμόν του θυσιαστηρίου.
\par 12 Και ο προσφέρων το δώρον αυτού την πρώτην ημέραν ήτο Ναασσών ο υιός του Αμμιναδάβ, εκ της φυλής Ιούδα·
\par 13 και το δώρον αυτού ήτο εις αργυρούς δίσκος, βάρους εκατόν τριάκοντα σίκλων· λεκάνιον εν αργυρούν εβδομήκοντα σίκλων, κατά τον σίκλον τον άγιον· αμφότερα πλήρη σεμιδάλεως εζυμωμένης μετά ελαίου, διά προσφοράν εξ αλφίτων·
\par 14 εις θυμιαματοδόχος χρυσούς δέκα σίκλων, πλήρης θυμιάματος·
\par 15 εις μόσχος εκ βοών, εις κριός, εν αρνίον ενιαύσιον, εις ολοκαύτωμα·
\par 16 εις τράγος εξ αιγών εις προσφοράν περί αμαρτίας·
\par 17 και εις θυσίαν ειρηνικήν, δύο βόες, κριοί πέντε, τράγοι πέντε, αρνία ενιαύσια πέντε. Τούτο ήτο το δώρον του Ναασσών, υιού του Αμμιναδάβ.
\par 18 Την δευτέραν ημέραν προσέφερεν ο Ναθαναήλ ο υιός του Σουάρ, ο άρχων της φυλής Ισσάχαρ·
\par 19 και προσέφερε το δώρον αυτού ένα δίσκον αργυρούν βάρους εκατόν τριάκοντα σίκλων· λεκάνιον εν αργυρούν εβδομήκοντα σίκλων, κατά τον σίκλον τον άγιον· αμφότερα πλήρη σεμιδάλεως εζυμωμένης μετά ελαίου, διά προσφοράν εξ αλφίτων·
\par 20 ένα θυμιαματοδόχον χρυσούν δέκα σίκλων, πλήρης θυμιάματος·
\par 21 ένα μόσχον εκ βοών, ένα κριόν, εν αρνίον ενιαύσιον, εις ολοκαύτωμα·
\par 22 τράγον εξ αιγών ένα, εις προσφοράν περί αμαρτίας·
\par 23 και εις θυσίαν ειρηνικήν, δύο βόας, κριούς πέντε, τράγους πέντε, αρνία ενιαύσια πέντε. Τούτο ήτο το δώρον του Ναθαναήλ, υιού του Σουάρ.
\par 24 Την τρίτην ημέραν προσέφερεν ο άρχων των υιών Ζαβουλών, Ελιάβ ο υιός του Χαιλών·
\par 25 το δώρον αυτού ήτο εις αργυρούς δίσκος βάρους εκατόν τριάκοντα σίκλων· λεκάνιον εν αργυρούν εβδομήκοντα σίκλων, κατά τον σίκλον τον άγιον· αμφότερα πλήρη σεμιδάλεως εζυμωμένης μετά ελαίου, διά προσφοράν εξ αλφίτων·
\par 26 εις θυμιαματοδόχος χρυσούς δέκα σίκλων, πλήρης θυμιάματος·
\par 27 εις μόσχος εκ βοών, εις κριός, εν αρνίον ενιαύσιον, εις ολοκαύτωμα·
\par 28 εις τράγος εξ αιγών εις προσφοράν περί αμαρτίας·
\par 29 και εις θυσίαν ειρηνικήν, δύο βόες, κριοί πέντε, τράγοι πέντε, αρνία ενιαύσια πέντε. Τούτο ήτο το δώρον του Ελιάβ, υιού του Χαιλών.
\par 30 Την τετάρτην ημέραν προσέφερεν Ελισούρ ο υιός του Σεδιούρ, ο άρχων των υιών Ρουβήν.
\par 31 Το δώρον αυτού ήτο εις αργυρούς δίσκος βάρους εκατόν τριάκοντα σίκλων· λεκάνιον εν αργυρούν εβδομήκοντα σίκλων, κατά τον σίκλον τον άγιον· αμφότερα πλήρη σεμιδάλεως εζυμωμένης μετά ελαίου, διά προσφοράν εξ αλφίτων·
\par 32 εις θυμιαματοδόχος χρυσούς δέκα σίκλων, πλήρης θυμιάματος·
\par 33 εις μόσχος εκ βοών, εις κριός, εν αρνίον ενιαύσιον, εις ολοκαύτωμα·
\par 34 εις τράγος εξ αιγών εις προσφοράν περί αμαρτίας·
\par 35 και εις θυσίαν ειρηνικήν, δύο βόες, κριοί πέντε, τράγοι πέντε, αρνία ενιαύσια πέντε. Τούτο ήτο το δώρον του Ελισούρ, υιού του Σεδιούρ.
\par 36 Την πέμπτην ημέραν προσέφερεν ο άρχων των υιών Συμεών, Σελουμιήλ ο υιός του Σουρισαδαΐ·
\par 37 το δώρον αυτού ήτο εις δίσκος αργυρούς βάρους εκατόν τριάκοντα σίκλων· λεκάνιον εν αργυρούν εβδομήκοντα σίκλων, κατά τον σίκλον τον άγιον· αμφότερα πλήρη σεμιδάλεως εζυμωμένης μετά ελαίου, διά προσφοράν εξ αλφίτων·
\par 38 εις θυμιαματοδόχος χρυσούς δέκα σίκλων, πλήρης θυμιάματος·
\par 39 εις μόσχος εκ βοών, εις κριός, εν αρνίον ενιαύσιον, εις ολοκαύτωμα·
\par 40 εις τράγος εξ αιγών εις προσφοράν περί αμαρτίας·
\par 41 και εις θυσίαν ειρηνικήν, δύο βόες, κριοί πέντε, τράγοι πέντε, αρνία ενιαύσια πέντε. Τούτο ήτο το δώρον του Σελουμιήλ, υιού του Σουρισαδαΐ.
\par 42 Την έκτην ημέραν προσέφερεν ο άρχων των υιών Γαδ, Ελιασάφ ο υιός του Δεουήλ·
\par 43 το δώρον αυτού ήτο εις αργυρούς δίσκος βάρους εκατόν τριάκοντα σίκλων· λεκάνιον εν αργυρούν εβδομήκοντα σίκλων, κατά τον σίκλον τον άγιον· αμφότερα πλήρη σεμιδάλεως εζυμωμένης μετά ελαίου, διά προσφοράν εξ αλφίτων·
\par 44 εις θυμιαματοδόχος χρυσούς δέκα σίκλων, πλήρης θυμιάματος·
\par 45 εις μόσχος εκ βοών, εις κριός, εν αρνίον ενιαύσιον, εις ολοκαύτωμα·
\par 46 εις τράγος εξ αιγών εις προσφοράν περί αμαρτίας·
\par 47 και εις θυσίαν ειρηνικήν, δύο βόες, κριοί πέντε, τράγοι πέντε, αρνία ενιαύσια πέντε. Τούτο ήτο το δώρον του Ελιασάφ, υιού του Δεουήλ.
\par 48 Την εβδόμην ημέραν προσέφερεν ο άρχων των υιών Εφραΐμ, Ελισαμά ο υιός του Αμμιούδ·
\par 49 το δώρον αυτού ήτο εις αργυρούς δίσκος βάρους εκατόν τριάκοντα σίκλων· λεκάνιον εν αργυρούν εβδομήκοντα σίκλων, κατά τον σίκλον τον άγιον· αμφότερα πλήρη σεμιδάλεως εζυμωμένης μετά ελαίου, διά προσφοράν εξ αλφίτων·
\par 50 εις θυμιαματοδόχος χρυσούς δέκα σίκλων, πλήρης θυμιάματος·
\par 51 εις μόσχος εκ βοών, εις κριός, εν αρνίον ενιαύσιον, εις ολοκαύτωμα·
\par 52 εις τράγος εξ αιγών εις προσφοράν περί αμαρτίας·
\par 53 και εις θυσίαν ειρηνικήν, δύο βόες, κριοί πέντε, τράγοι πέντε, αρνία ενιαύσια πέντε. Τούτο ήτο το δώρον του Ελισαμά, υιού του Αμμιούδ.
\par 54 Την ογδόην ημέραν προσέφερεν ο άρχων των υιών Μανασσή, Γαμαλιήλ ο υιός του Φεδασσούρ·
\par 55 το δώρον αυτού ήτο εις αργυρούς δίσκος βάρους εκατόν τριάκοντα σίκλων· λεκάνιον εν αργυρούν εβδομήκοντα σίκλων, κατά τον σίκλον τον άγιον· αμφότερα πλήρη σεμιδάλεως εζυμωμένης μετά ελαίου, διά προσφοράν εξ αλφίτων·
\par 56 εις θυμιαματοδόχος χρυσούς δέκα σίκλων, πλήρης θυμιάματος·
\par 57 εις μόσχος εκ βοών, εις κριός, εν αρνίον ενιαύσιον, εις ολοκαύτωμα·
\par 58 εις τράγος εξ αιγών εις προσφοράν περί αμαρτίας·
\par 59 και εις θυσίαν ειρηνικήν, δύο βόες, κριοί πέντε, τράγοι πέντε, αρνία ενιαύσια πέντε. Τούτο ήτο το δώρον του Γαμαλιήλ, υιού του Φεδασσούρ.
\par 60 Την εννάτην ημέραν προσέφερεν ο άρχων των υιών Βενιαμίν, Αβειδάν ο υιός του Γιδεωνί·
\par 61 το δώρον αυτού ήτο εις αργυρούς δίσκος βάρους εκατόν τριάκοντα σίκλων· λεκάνιον εν αργυρούν εβδομήκοντα σίκλων, κατά τον σίκλον τον άγιον· αμφότερα πλήρη σεμιδάλεως εζυμωμένης μετά ελαίου, διά προσφοράν εξ αλφίτων·
\par 62 εις θυμιαματοδόχος χρυσούς δέκα σίκλων, πλήρης θυμιάματος·
\par 63 εις μόσχος εκ βοών, εις κριός, εν αρνίον ενιαύσιον, εις ολοκαύτωμα·
\par 64 εις τράγος εξ αιγών εις προσφοράν περί αμαρτίας·
\par 65 και εις θυσίαν ειρηνικήν, δύο βόες, κριοί πέντε, τράγοι πέντε, αρνία ενιαύσια πέντε. Τούτο ήτο το δώρον του Αβειδάν υιού του Γιδεωνί.
\par 66 Την δεκάτην ημέραν προσέφερεν ο άρχων των υιών Δαν, Αχιέζερ ο υιός του Αμμισαδαΐ·
\par 67 το δώρον αυτού ήτο εις αργυρούς δίσκος βάρους εκατόν τριάκοντα σίκλων· λεκάνιον εν αργυρούν εβδομήκοντα σίκλων, κατά τον σίκλον τον άγιον· αμφότερα πλήρη σεμιδάλεως εζυμωμένης μετά ελαίου, διά προσφοράν εξ αλφίτων·
\par 68 εις θυμιαματοδόχος χρυσούς δέκα σίκλων, πλήρης θυμιάματος·
\par 69 εις μόσχος εκ βοών, εις κριός, εν αρνίον ενιαύσιον, εις ολοκαύτωμα·
\par 70 εις τράγος εξ αιγών εις προσφοράν περί αμαρτίας·
\par 71 και εις θυσίαν ειρηνικήν, δύο βόες, κριοί πέντε, τράγοι πέντε, αρνία ενιαύσια πέντε. Τούτο ήτο το δώρον του Αχιέζερ, υιού του Αμμισαδαΐ.
\par 72 Την ενδεκάτην ημέραν προσέφερεν ο άρχων των υιών Ασήρ, Φαγαιήλ ο υιός του Οχράν·
\par 73 το δώρον αυτού ήτο εις αργυρούς δίσκος βάρους εκατόν τριάκοντα σίκλων· λεκάνιον εν αργυρούν εβδομήκοντα σίκλων, κατά τον σίκλον τον άγιον· αμφότερα πλήρη σεμιδάλεως εζυμωμένης μετά ελαίου, διά προσφοράν εξ αλφίτων·
\par 74 εις θυμιαματοδόχος χρυσούς δέκα σίκλων, πλήρης θυμιάματος.
\par 75 εις μόσχος εκ βοών, εις κριός, εν αρνίον ενιαύσιον, εις ολοκαύτωμα.
\par 76 εις τράγος εξ αιγών εις προσφοράν περί αμαρτίας·
\par 77 και εις θυσίαν ειρηνικήν, δύο βόες, κριοί πέντε, τράγοι πέντε, αρνία ενιαύσια πέντε. Τούτο ήτο το δώρον του Φαγαιήλ, υιού του Οχράν.
\par 78 Την δωδεκάτην ημέραν προσέφερεν ο άρχων των υιών Νεφθαλί, Αχιρά ο υιός του Αινάν·
\par 79 το δώρον αυτού ήτο εις αργυρούς δίσκος βάρους εκατόν τριάκοντα σίκλων· λεκάνιον εν αργυρούν εβδομήκοντα σίκλων· κατά τον σίκλον τον άγιον· αμφότερα πλήρη σεμιδάλεως εζυμωμένης μετά ελαίου, διά προσφοράν εξ αλφίτων·
\par 80 εις θυμιαματοδόχος χρυσούς δέκα σίκλων, πλήρης θυμιάματος·
\par 81 εις μόσχος εκ βοών, εις κριός, εν αρνίον ενιαύσιον, εις ολοκαύτωμα.
\par 82 εις τράγος εξ αιγών εις προσφοράν περί αμαρτίας·
\par 83 και εις θυσίαν ειρηνικήν, δύο βόες, κριοί πέντε, τράγοι πέντε, αρνία ενιαύσια πέντε· τούτο ήτο το δώρον του Αχιρά, υιού του Αινάν.
\par 84 Ούτος ήτο ο εγκαινιασμός του θυσιαστηρίου, την ημέραν καθ' ην εχρίσθη υπό των αρχόντων του Ισραήλ· δίσκοι αργυροί δώδεκα, λεκάνια αργυρά δώδεκα, θυμιαματοδόχοι χρυσοί δώδεκα·
\par 85 εκατόν τριάκοντα σίκλων ήτο έκαστος δίσκος αργυρούς, και εβδομήκοντα σίκλων έκαστον λεκάνιον· άπαν το αργύριον των σκευών, δύο χιλιάδων και τετρακοσίων σίκλων, κατά τον σίκλον τον άγιον·
\par 86 θυμιαματοδόχοι χρυσοί δώδεκα, πλήρεις θυμιάματος, από δέκα σίκλων ο θυμιαματοδόχος κατά τον σίκλον τον άγιον· άπαν το χρυσίον των θυμιαματοδόχων εκατόν είκοσι σίκλων.
\par 87 Πάντες οι βόες διά ολοκαύτωμα ήσαν μόσχοι δώδεκα, οι κριοί δώδεκα, τα αρνία τα ενιαύσια δώδεκα μετά των εξ αλφίτων προσφορών αυτών· και οι τράγοι εξ αιγών εις προσφοράν περί αμαρτίας, δώδεκα.
\par 88 Και πάντες οι βόες διά θυσίαν ειρηνικήν ήσαν μόσχοι εικοσιτέσσαρες, οι κριοί εξήκοντα, οι τράγοι εξήκοντα, τα αρνία τα ενιαύσια εξήκοντα. Ούτος ήτο ο εγκαινιασμός του θυσιαστηρίου, αφού εχρίσθη.
\par 89 Και ότε εισήλθεν ο Μωϋσής εις την σκηνήν του μαρτυρίου διά να λαλήση μετά του Κυρίου, τότε ήκουσε την φωνήν του λαλούντος προς αυτό, άνωθεν του ιλαστηρίου, το οποίον ήτο επί της κιβωτού του μαρτυρίου ανά μέσον των δύο χερουβείμ· και ελάλει προς αυτόν.

\chapter{8}

\par 1 Και ελάλησε Κύριος προς τον Μωϋσήν, λέγων,
\par 2 Λάλησον προς τον Ααρών, και ειπέ προς αυτόν, Όταν ανάψης τους λύχνους, οι επτά λύχνοι θέλουσι φωτίζει κατά πρόσωπον της λυχνίας.
\par 3 Και ο Ααρών έκαμεν ούτως· ήναψε κατά πρόσωπον της λυχνίας τους λύχνους αυτής, καθώς προσέταξεν ο Κύριος εις τον Μωϋσήν.
\par 4 Αύτη δε ήτο η κατασκευή της λυχνίας από χρυσίου σφυρηλάτου· και ο κορμός αυτής και τα άνθη αυτής, ήτο όλη σφυρήλατος· κατά το σχέδιον, το οποίον έδειξεν ο Κύριος εις τον Μωϋσήν, ούτως έκαμε την λυχνίαν.
\par 5 Και ελάλησε Κύριος προς τον Μωϋσήν, λέγων,
\par 6 Λάβε τους Λευΐτας εκ μέσου των υιών Ισραήλ και καθάρισον αυτούς.
\par 7 Και ούτω θέλεις κάμει εις αυτούς διά τον καθαρισμόν αυτών· ράντισον επ' αυτούς ύδωρ καθαρισμού και ας περάσωσι ξυράφιον δι' όλου του σώματος αυτών και ας πλύνωσι τα ενδύματα αυτών και ας καθαρισθώσιν.
\par 8 Έπειτα ας λάβωσιν ένα μόσχον εκ βοών μετά της εξ αλφίτων προσφοράς αυτού από σεμιδάλεως εζυμωμένης μετά ελαίου, και ένα άλλον μόσχον εκ βοών θέλεις λάβει εις προσφοράν περί αμαρτίας.
\par 9 Και θέλεις φέρει τους Λευΐτας έμπροσθεν της σκηνής του μαρτυρίου και θέλεις συνάξει όλην την συναγωγήν των υιών Ισραήλ·
\par 10 και θέλεις φέρει τους Λευΐτας έμπροσθεν του Κυρίου, και θέλουσιν επιθέσει οι υιοί Ισραήλ τας χείρας αυτών επί τους Λευΐτας.
\par 11 και ο Ααρών θέλει προσφέρει τους Λευΐτας ενώπιον του Κυρίου προσφοράν παρά των υιών Ισραήλ, διά να υπηρετώσι την υπηρεσίαν του Κυρίου.
\par 12 Και οι Λευΐται θέλουσιν επιθέσει τας χείρας αυτών επί τας κεφαλάς των μόσχων, και θέλεις προσφέρει τον ένα περί αμαρτίας και τον άλλον δι' ολοκαύτωμα εις τον Κύριον· διά να κάμης εξιλέωσιν υπέρ των Λευϊτών.
\par 13 Και θέλεις στήσει τους Λευΐτας έμπροσθεν του Ααρών και έμπροσθεν των υιών αυτού· και θέλεις προσφέρει αυτούς προσφοράν εις τον Κύριον.
\par 14 Ούτω θέλεις αποχωρίσει τους Λευΐτας εκ μέσου των υιών Ισραήλ, και οι Λευΐται θέλουσιν είσθαι εμού.
\par 15 Και μετά ταύτα θέλουσιν εισέλθει οι Λευΐται διά να υπηρετώσι την σκηνήν του μαρτυρίου· και θέλεις καθαρίσει αυτούς και θέλεις προσφέρει αυτούς προσφοράν.
\par 16 Διότι ούτοι είναι δεδομένοι δώρον εις εμέ εκ μέσου των υιών Ισραήλ· αντί των διανοιγόντων πάσαν μήτραν, πάντων των πρωτοτόκων των υιών Ισραήλ έλαβον αυτούς εις εμαυτόν.
\par 17 Διότι πάντα τα πρωτότοκα των υιών Ισραήλ είναι εμού, από ανθρώπου έως κτήνους· καθ' ην ημέραν επάταξα πάντα τα πρωτότοκα εν τη γη της Αιγύπτου, ηγίασα αυτούς εις εμαυτόν·
\par 18 και έλαβον τους Λευΐτας αντί πάντων των πρωτοτόκων των υιών Ισραήλ.
\par 19 Και έδωκα τους Λευΐτας δώρον εις τον Ααρών και εις τους υιούς αυτού, εκ μέσου των υιών Ισραήλ, διά να υπηρετώσι την υπηρεσίαν των υιών Ισραήλ εν τη σκηνή του μαρτυρίου, και διά να κάμνωσιν εξιλέωσιν υπέρ των υιών Ισραήλ· διά να μη ήναι πληγή επί τους υιούς Ισραήλ, εάν πλησιάσωσιν οι υιοί Ισραήλ εις τα άγια.
\par 20 Και έκαμον ο Μωϋσής και ο Ααρών και πάσα η συναγωγή των υιών Ισραήλ εις τους Λευΐτας, κατά πάντα όσα προσέταξε Κύριος εις τον Μωϋσήν περί των Λευϊτών· ούτως έκαμον εις αυτούς οι υιοί Ισραήλ.
\par 21 Και εκαθαρίσθησαν οι Λευΐται και έπλυναν τα ιμάτια αυτών· και προσέφερεν αυτούς ο Ααρών προσφοράν ενώπιον του Κυρίου, και ο Ααρών έκαμεν υπέρ αυτών εξιλέωσιν διά να καθαρίση αυτούς.
\par 22 Και μετά ταύτα εισήλθον οι Λευΐται διά να υπηρετώσι την υπηρεσίαν αυτών εν τη σκηνή του μαρτυρίου έμπροσθεν του Ααρών και έμπροσθεν των υιών αυτού· καθώς προσέταξεν ο Κύριος εις τον Μωϋσήν περί των Λευϊτών, ούτως έκαμον εις αυτούς.
\par 23 Και ελάλησε Κύριος προς τον Μωϋσήν, λέγων,
\par 24 Τούτο είναι το περί των Λευϊτών· από εικοσιπέντε ετών και επάνω θέλουσιν εισέρχεσθαι να εκτελώσι την υπηρεσίαν της σκηνής του μαρτυρίου·
\par 25 και από πεντήκοντα ετών θέλουσι παύεσθαι του να εκτελώσι την υπηρεσίαν και δεν θέλουσιν υπηρετεί πλέον·
\par 26 αλλά θέλουσιν υπουργεί εις τους αδελφούς αυτών εν τη σκηνή του μαρτυρίου, διά να φυλάττωσι τας φυλακάς· υπηρεσίαν όμως δεν θέλουσι κάμνει. Ούτω θέλεις κάμει εις τους Λευΐτας ως προς τας φυλακάς αυτών.

\chapter{9}

\par 1 Και ελάλησε Κύριος προς τον Μωϋσήν εν τη ερήμω Σινά, τον πρώτον μήνα του δευτέρου έτους αφού εξήλθον εκ γης Αιγύπτου, λέγων,
\par 2 Ας κάμνωσιν οι υιοί Ισραήλ το πάσχα εν τω καιρώ αυτού·
\par 3 την δεκάτην τετάρτην ημέραν τούτου του μηνός προς εσπέραν θέλετε κάμει αυτό, κατά τον καιρόν αυτού· κατά πάντα τα νόμιμα αυτού και κατά πάσας τας τελετάς αυτού θέλετε κάμει αυτό.
\par 4 Και ελάλησεν ο Μωϋσής προς τους υιούς Ισραήλ διά να κάμωσι το πάσχα.
\par 5 Και έκαμον το πάσχα την δεκάτην τετάρτην του πρώτου μηνός προς εσπέραν εν τη ερήμω Σινά· κατά πάντα όσα προσέταξε Κύριος εις τον Μωϋσήν, ούτως έκαμον οι υιοί Ισραήλ.
\par 6 Και ευρίσκοντο τινές, οίτινες ήσαν ακάθαρτοι από νεκρού σώματος ανθρώπου και δεν ηδύναντο να κάμωσι το πάσχα εκείνην την ημέραν· και ήλθον έμπροσθεν του Μωϋσέως και έμπροσθεν του Ααρών την ημέραν εκείνην.
\par 7 Και είπον οι άνδρες εκείνοι προς αυτόν, Ημείς είμεθα ακάθαρτοι από νεκρού σώματος ανθρώπου· διά τι εμποδιζόμεθα να προσφέρωμεν το δώρον του Κυρίου εν τω καιρώ αυτού μεταξύ των υιών Ισραήλ;
\par 8 Και είπε προς αυτούς ο Μωϋσής, Στήτε αυτού και θέλω ακούσει τι θέλει προστάξει ο Κύριος διά σας.
\par 9 Και ελάλησε Κύριος προς τον Μωϋσήν, λέγων,
\par 10 Ειπέ προς τους υιούς Ισραήλ λέγων, Εάν τις άνθρωπος εξ υμών ή εκ των γενεών υμών γείνη ακάθαρτος από νεκρού σώματος, ή ήναι εις οδόν μακράν, θέλει κάμει το πάσχα εις τον Κύριον·
\par 11 την δεκάτην του δευτέρου μηνός προς εσπέραν θέλουσι κάμει αυτό και μετά αζύμων και πικραλίδων θέλουσι φάγει αυτό.
\par 12 Δεν θέλουσιν αφήσει εξ αυτού μέχρι πρωΐας ουδέ θέλουσι συντρίψει οστούν εξ αυτού· κατά πάντα τα νόμιμα του πάσχα θέλουσι κάμει αυτό.
\par 13 Και ο άνθρωπος όστις καθαρός ων, και μη ευρισκόμενος εις οδόν, λείψη από του να κάμη το πάσχα, θέλει εξολοθρευθή η ψυχή εκείνη εκ του λαού αυτής· επειδή δεν προσέφερε το δώρον του Κυρίου εν τω καιρώ αυτού, ο άνθρωπος εκείνος θέλει βαστάσει την αμαρτίαν αυτού.
\par 14 Εάν δε παροική ξένος μεταξύ σας και κάμη το πάσχα εις τον Κύριον κατά τα νόμιμα του πάσχα και κατά τας τελετάς αυτού, ούτω θέλει κάμει αυτό· τον αυτόν νόμον θέλετε έχει και διά τον ξένον και διά τον αυτόχθονα.
\par 15 Και την ημέραν καθ' ην εστήθη η σκηνή, εκάλυψεν η νεφέλη την σκηνήν, τον οίκον του μαρτυρίου· και από εσπέρας έως πρωΐ ήτο επί της σκηνής ως είδος πυρός.
\par 16 Ούτως εγίνετο πάντοτε· η νεφέλη εκάλυπτεν αυτήν την ημέραν και είδος πυρός την νύκτα.
\par 17 Και ότε ανέβαινεν η νεφέλη από της σκηνής, τότε εσηκόνοντο οι υιοί Ισραήλ· και εν τω τόπω, όπου ίστατο η νεφέλη, εκεί εστρατοπέδευον οι υιοί Ισραήλ.
\par 18 Κατά την προσταγήν του Κυρίου εσηκόνοντο οι υιοί Ισραήλ και κατά την προσταγήν του Κυρίου εστρατοπέδευον· πάσας τας ημέρας καθ' ας η νεφέλη έκειτο επί της σκηνής, έμενον εστρατοπεδευμένοι.
\par 19 Και ότε η νεφέλη διέμενεν επί της σκηνής πολλάς ημέρας, τότε οι υιοί Ισραήλ εφύλαττον τας φυλακάς του Κυρίου και δεν εσηκόνοντο.
\par 20 Και οπότε μεν η νεφέλη ίστατο επί της σκηνής οσασδήποτε ημέρας, κατά την προσταγήν του Κυρίου έμενον εστρατοπεδευμένοι και κατά την προσταγήν του Κυρίου εσηκόνοντο.
\par 21 Οπότε δε η νεφέλη ίστατο από εσπέρας έως πρωΐ, το δε πρωΐ ανέβαινεν η νεφέλη, τότε αυτοί εσηκόνοντο· είτε την ημέραν είτε την νύκτα ανέβαινεν η νεφέλη, τότε αυτοί εσηκόνοντο.
\par 22 Δύο ημέρας ή ένα μήνα ή εν έτος εάν διέμενεν η νεφέλη επί της σκηνής, ισταμένη επ' αυτής, έμενον εστρατοπεδευμένοι οι υιοί Ισραήλ και δεν εσηκόνοντο· ότε δε αυτή ανέβαινεν, εσηκόνοντο.
\par 23 Κατά την προσταγήν του Κυρίου εστρατοπέδευον και κατά την προσταγήν του Κυρίου εσηκόνοντο· εφύλαττον τας φυλακάς του Κυρίου, καθώς προσέταξεν ο Κύριος διά χειρός του Μωϋσέως.

\chapter{10}

\par 1 Και ελάλησε Κύριος προς τον Μωϋσήν, λέγων,
\par 2 Κάμε εις σεαυτόν δύο σάλπιγγας αργυράς· σφυρηλάτους θέλεις κάμει αυτάς, και θέλουσιν είσθαι εις σε διά να συγκαλής την συναγωγήν, και να βάλλης εις κίνησιν τα στρατόπεδα.
\par 3 Και όταν σαλπίζωσι δι' αυτών, πάσα η συναγωγή θέλει συναθροίζεσθαι προς σε εις την θύραν της σκηνής του μαρτυρίου.
\par 4 Εάν δε σαλπίσωσι διά μιας μόνης, τότε θέλουσι συναθροίζεσθαι προς σε οι άρχοντες, οι αρχηγοί των χιλιάδων του Ισραήλ.
\par 5 Και όταν σαλπίζητε αλαλαγμόν, τότε θέλουσι σηκόνεσθαι τα στρατόπεδα τα εστρατοπεδευμένα προς ανατολάς.
\par 6 Και όταν σαλπίζητε αλαλαγμόν δεύτερον, τότε θέλουσι σηκόνεσθαι τα στρατόπεδα τα εστρατοπεδευμένα προς νότον· θέλουσι σαλπίζει αλαλαγμόν διά να σηκωθώσι.
\par 7 Όταν δε συγκαλήται η συναγωγή, θέλετε σαλπίζει, ουχί όμως αλαλαγμόν.
\par 8 Και οι υιοί του Ααρών, οι ιερείς, θέλουσι σαλπίζει διά των σαλπίγγων· και ταύτα θέλουσιν είσθαι εις εσάς νόμιμον αιώνιον εις τας γενεάς σας.
\par 9 Και εάν εξέλθητε εις μάχην εν τη γη υμών κατά του εχθρού του πολεμούντος εναντίον υμών, τότε θέλετε σαλπίζει αλαλαγμόν διά των σαλπίγγων και θέλετε ελθεί εις ενθύμησιν ενώπιον Κυρίου του Θεού υμών και θέλετε διασωθή εκ των εχθρών υμών.
\par 10 Και εις τας ημέρας της ευφροσύνης υμών και εις τας εορτάς υμών, και εις τας νεομηνίας υμών θέλετε σαλπίζει διά των σαλπίγγων επί των ολοκαυτωμάτων υμών και επί των θυσιών των ειρηνικών προσφορών υμών, και θέλουσιν είσθαι εις υμάς προς ενθύμησιν ενώπιον του Θεού υμών. Εγώ είμαι Κύριος ο Θεός υμών.
\par 11 Και την εικοστήν του δευτέρου μηνός του δευτέρου έτους ανέβη η νεφέλη από της σκηνής του μαρτυρίου.
\par 12 Και εσηκώθησαν οι υιοί Ισραήλ κατά την τάξιν της οδοιπορίας αυτών από της ερήμου Σινά, και εστάθη η νεφέλη εν τη ερήμω Φαράν.
\par 13 Και εσηκώθησαν πρώτον, καθώς προσέταξε Κύριος διά χειρός του Μωϋσέως.
\par 14 Και πρώτη εσηκώθη η σημαία του στρατοπέδου των υιών Ιούδα, κατά τα τάγματα αυτών, και επί του στρατεύματος αυτού ήτο Ναασσών ο υιός του Αμμιναδάβ.
\par 15 Και επί του στρατεύματος της φυλής των υιών Ισσάχαρ ήτο Ναθαναήλ ο υιός του Σουάρ.
\par 16 Και επί του στρατεύματος της φυλής των υιών Ζαβουλών, Ελιάβ ο υιός του Χαιλών.
\par 17 Και κατεβίβασαν την σκηνήν· και οι υιοί Γηρσών και οι υιοί Μεραρί εσηκώθησαν, βαστάζοντες την σκηνήν.
\par 18 Έπειτα εσηκώθη σημαία του στρατοπέδου του Ρουβήν κατά τα τάγματα αυτών, και επί του στρατεύματος αυτού ήτο Ελισούρ ο υιός του Σεδιούρ.
\par 19 Και επί του στρατεύματος της φυλής των υιών Συμεών ήτο Σελουμιήλ ο υιός του Σουρισαδαΐ.
\par 20 Και επί του στρατεύματος της φυλής των υιών Γαδ ήτο Ελιασάφ ο υιός του Δεουήλ.
\par 21 Και οι Κααθίται εσηκώθησαν, βαστάζοντες τα άγια, και οι άλλοι έστηνον την σκηνήν εωσού αυτοί φθάσωσι.
\par 22 Έπειτα εσηκώθη η σημαία του στρατοπέδου των υιών Εφραΐμ, κατά τα τάγματα αυτών, και επί του στρατεύματος αυτού ήτο Ελισαμά ο υιός του Αμμιούδ.
\par 23 Και επί του στρατεύματος της φυλής των υιών Μανασσή ήτο Γαμαλιήλ ο υιός του Φεδασσούρ.
\par 24 Και επί του στρατεύματος της φυλής των υιών Βενιαμίν ήτο Αβειδάν ο υιός του Γιδεωνί.
\par 25 Έπειτα εσηκώθη η σημαία του στρατοπέδου των υιών Δαν κατόπιν πάντων των στρατοπέδων κατά τα τάγματα αυτών, και επί του στρατεύματος αυτού ήτο Αχιέζερ ο υιός του Αμμισαδαΐ.
\par 26 Και επί του στρατεύματος της φυλής των υιών Ασήρ ήτο Φαγαιήλ ο υιός του Οχράν.
\par 27 Και επί του στρατεύματος της φυλής των υιών Νεφθαλί ήτο Αχιρά ο υιός του Αινάν.
\par 28 Ούτως εγίνετο η οδοιπορία των υιών Ισραήλ κατά τα τάγματα αυτών, ότε εσηκόνοντο.
\par 29 Και είπεν ο Μωϋσής προς τον Οβάβ, υιόν του Ραγουήλ του Μαδιανίτου, πενθερού του Μωϋσέως, Ημείς υπάγομεν εις τον τόπον περί του οποίου είπεν ο Κύριος, Τούτον θέλω δώσει εις εσάς· ελθέ μεθ' ημών και θέλομεν σε αγαθοποιήσει επειδή ο Κύριος ελάλησεν αγαθά περί του Ισραήλ.
\par 30 Και είπε προς αυτόν, δεν θέλω ελθεί αλλά θέλω επιστρέψει εις την γην μου και εις την γενεάν μου.
\par 31 Και είπε, Μη μας αφήσης, παρακαλώ, επειδή συ γνωρίζεις που πρέπει να στρατοπεδεύωμεν εν τη ερήμω και θέλεις είσθαι εις ημάς αντί οφθαλμών·
\par 32 και εάν έλθης μεθ' ημών, τα αγαθά εκείνα, τα οποία θέλει κάμει εις ημάς ο Κύριος, τα αυτά θέλομεν κάμει και ημείς εις σε.
\par 33 Και ώδοιπόρησαν από του όρους του Κυρίου οδόν τριών ημερών· και η κιβωτός της διαθήκης του Κυρίου προεπορεύετο έμπροσθεν αυτών οδόν τριών ημερών, διά να ζητήση τόπον αναπαύσεως δι' αυτούς.
\par 34 Και η νεφέλη του Κυρίου ήτο επάνωθεν αυτών την ημέραν, ότε εσηκόνοντο από του στρατοπέδου.
\par 35 Και ότε εσηκόνετο η κιβωτός, έλεγεν ο Μωϋσής, Ανάστα, Κύριε, και διασκορπισθήτωσαν οι εχθροί σου και φυγέτωσαν οι μισούντές σε από προσώπου σου.
\par 36 Και ότε ανεπαύετο, έλεγεν, Επίστρεψον, Κύριε, εις τας χιλιάδας των μυριάδων του Ισραήλ.

\chapter{11}

\par 1 Και εγόγγυζεν ο λαός πονηρά εις τα ώτα του Κυρίου· και ο Κύριος ήκουσε και εξήφθη η οργή αυτού· και εξεκαύθη μεταξύ αυτών πυρ Κυρίου και κατέφαγε το άκρον του στρατοπέδου.
\par 2 Και εβόησεν ο λαός προς τον Μωϋσήν· και ο Μωϋσής προσηυχήθη προς τον Κύριον και έπαυσε το πυρ.
\par 3 Και εκαλέσθη το όνομα του τόπου εκείνου Ταβερά, διότι εξεκαύθη μεταξύ αυτών πυρ Κυρίου.
\par 4 Και το σύμμικτον πλήθος το μεταξύ αυτών, επεθύμησαν επιθυμίαν· και έκλαιον πάλιν και οι υιοί Ισραήλ, και είπαν, Τις θέλει δώσει εις ημάς κρέας να φάγωμεν;
\par 5 ενθυμούμεθα τα οψάρια, τα οποία ετρώγομεν εν Αιγύπτω δωρεάν, τα αγγούρια και τα πεπόνια και τα πράσα και τα κρόμμυα και τα σκόρδα·
\par 6 τώρα δε η ψυχή ημών είναι κατάξηρος· δεν είναι εις τους οφθαλμούς ημών ουδέν άλλο παρά τούτο το μάννα.
\par 7 Το δε μάννα ήτο ως ο σπόρος του κοριάνδρου, και το χρώμα αυτού ως το χρώμα του βδελλίου.
\par 8 Ο λαός περιεφέρετο συνάγων αυτό, και ήλεθον εις μύλον ή εκοπάνιζον αυτό εις ιγδίον και έψηνον αυτό εις χύτραν και έκαμνον εγκρυφίας εξ αυτού· και η γεύσις αυτού ήτο ως γεύσις λαγάνου εξ ελαίου.
\par 9 Και ότε κατέβαινεν η δρόσος εις το στρατόπεδον την νύκτα, έπιπτε το μάννα επ' αυτής.
\par 10 Και ήκουσεν ο Μωϋσής τον λαόν κλαίοντα κατά τας συγγενείας αυτών, έκαστον εις την θύραν της σκηνής αυτού· και εξήφθη η οργή του Κυρίου σφόδρα· εφάνη δε τούτο κακόν και εις τον Μωϋσήν.
\par 11 Και είπεν ο Μωϋσής προς τον Κύριον, Διά τι εταλαιπώρησας τον δούλον σου; και διά τι δεν εύρηκα χάριν ενώπιόν σου, ώστε έβαλες επ' εμέ το φορτίον όλον του λαού τούτου;
\par 12 μήπως εγώ συνέλαβον όλον τον λαόν τούτον; ή εγώ εγέννησα αυτούς, διά να μοι λέγης, Λάβε αυτόν εις τον κόλπον σου, καθώς βαστάζει η τροφός το θηλάζον βρέφος, εις την γην την οποίαν ώμοσας προς τους πατέρας αυτών;
\par 13 πόθεν εις εμέ κρέατα να δώσω εις όλον τον λαόν τούτον; διότι κλαίουσι προς εμέ, λέγοντες, Δος εις ημάς κρέας να φάγωμεν·
\par 14 δεν δύναμαι εγώ μόνος να βαστάσω όλον τον λαόν τούτον, διότι είναι πολύ βαρύ εις εμέ·
\par 15 και αν κάμνης ούτως εις εμέ, θανάτωσόν με ευθύς, δέομαι, εάν εύρηκα χάριν ενώπιόν σου, διά να μη βλέπω την δυστυχίαν μου.
\par 16 Και είπε Κύριος προς τον Μωϋσήν, Σύναξον εις εμέ εβδομήκοντα άνδρας εκ των πρεσβυτέρων του Ισραήλ, τους οποίους γνωρίζεις ότι είναι πρεσβύτεροι του λαού και άρχοντες αυτών· και φέρε αυτούς εις την σκηνήν του μαρτυρίου, όπου θέλουσι σταθή μετά σου.
\par 17 Και θέλω καταβή και λαλήσει εκεί μετά σού· και θέλω λάβει από του πνεύματος του επί σε και θέλω επιθέσει επ' αυτούς· και θέλουσι βαστάζει το φορτίον του λαού μετά σου, διά να μη βαστάζης αυτό συ μόνος.
\par 18 Και ειπέ προς τον λαόν, Αγιάσατε εαυτούς διά την αύριον, και θέλετε φάγει κρέας· διότι εκλαύσατε εις τα ώτα του Κυρίου λέγοντες, Τις θέλει δώσει εις ημάς κρέας να φάγωμεν; διότι καλά ήμεθα εν Αιγύπτω. Διά τούτο θέλει σας δώσει κρέας ο Κύριος, και θέλετε φάγει·
\par 19 δεν θέλετε φάγει μίαν ημέραν ούτε δύο ημέρας ούτε πέντε ημέρας ούτε δέκα ημέρας, ούτε είκοσι ημέρας·
\par 20 ολόκληρον μήνα θέλετε φάγει, εωσού εξέλθη εκ των μυκτήρων σας και γείνη εις εσάς αηδία· διότι ηπειθήσατε εις τον Κύριον, όστις είναι μεταξύ σας, και εκλαύσατε ενώπιον αυτού, λέγοντες, Διά τι να αναχωρήσωμεν από της Αιγύπτου;
\par 21 Και είπεν ο Μωϋσής, Εξακόσιαι χιλιάδες πεζών είναι ο λαός, εν μέσω των οποίων εγώ είμαι και συ είπας, Θέλω δώσει εις αυτούς κρέας, διά να φάγωσιν ολόκληρον μήνα.
\par 22 Θέλουσι σφαχθή δι' αυτούς τα ποίμνια και αι αγέλαι, διά να εξαρκέσωσιν εις αυτούς; ή θέλουσι συναχθή ομού πάντα τα οψάρια της θαλάσσης δι' αυτούς, διά να εξαρκέσωσιν εις αυτούς;
\par 23 Και είπε Κύριος προς τον Μωϋσήν, Μήπως η χειρ του Κυρίου εσμικρύνθη; τώρα θέλεις ιδεί αν εκτελήται ο λόγος μου, ή ουχί.
\par 24 Και εξήλθεν ο Μωϋσής και είπε προς τον λαόν τους λόγους του Κυρίου· και συνήγαγε τους εβδομήκοντα άνδρας εκ των πρεσβυτέρων του λαού και έστησεν αυτούς κύκλω της σκηνής.
\par 25 Και κατέβη Κύριος εν νεφέλη και ελάλησε προς αυτόν, και έλαβεν από του πνεύματος του επ' αυτόν και επέθηκεν επί τους εβδομήκοντα άνδρας τους πρεσβυτέρους· και αφού εκάθησεν επ' αυτούς το πνεύμα, επροφήτευσαν αλλά δεν εξηκολούθησαν.
\par 26 Έμειναν όμως δύο άνδρες εν τω στρατοπέδω, το όνομα του ενός Ελδάδ και το όνομα του δευτέρου Μηδάδ· και εκάθησεν επ' αυτούς το πνεύμα· και ούτοι ήσαν εκ των καταγεγραμμένων, δεν εξήλθον όμως εις την σκηνήν· και επροφήτευον εν τω στρατοπέδω.
\par 27 Και έδραμε νεανίσκος τις και ανήγγειλε προς τον Μωϋσήν λέγων, Ο Ελδάδ και ο Μηδάδ προφητεύουσιν εν τω στρατοπέδω.
\par 28 Και Ιησούς ο υιός του Ναυή, ο θεράπων του Μωϋσέως, ο εκλεκτός αυτού, απεκρίθη και είπε, Κύριέ μου Μωϋσή, εμπόδισον αυτούς.
\par 29 Και είπε προς αυτόν ο Μωϋσής, Ζηλοτυπείς υπέρ εμού; είθε πας ο λαός του Κυρίου να ήσαν προφήται, και ο Κύριος να επέθετεν επ' αυτούς το πνεύμα αυτού
\par 30 Και ανεχώρησεν ο Μωϋσής εις το στρατόπεδον, αυτός και οι πρεσβύτεροι του Ισραήλ.
\par 31 Και εξήλθεν άνεμος παρά Κυρίου και έφερεν ορτύκια από της θαλάσσης και έρριψεν αυτά επί το στρατόπεδον, έως μιας ημέρας οδόν εντεύθεν και έως μιας ημέρας οδόν εντεύθεν, κύκλω του στρατοπέδου· και ήσαν έως δύο πήχας επί το πρόσωπον της γης.
\par 32 Και σηκωθείς ο λαός όλην εκείνην την ημέραν και όλην την νύκτα και όλην την ακόλουθον ημέραν, εσύναξαν τα ορτύκια· ο συνάξας το ολιγώτερον, εσύναξε δέκα χομόρ· και εξήπλονον αυτά κύκλω του στρατοπέδου δι' εαυτούς.
\par 33 Ενώ δε το κρέας ήτο έτι εις τους οδόντας αυτών, πριν μασσηθή, εξήφθη η οργή του Κυρίου επί τον λαόν· και επάταξε Κύριος τον λαόν εν πληγή μεγάλη σφόδρα.
\par 34 Και εκάλεσε το όνομα του τόπου εκείνου Κιβρώθ-αττααβά, διότι εκεί ετάφη ο λαός ο επιθυμητής.
\par 35 Και ανεχώρησεν ο λαός από Κιβρώθ-αττααβά εις Ασηρώθ και έμεινεν εν Ασηρώθ.

\chapter{12}

\par 1 Και ελάλησεν η Μαριάμ και ο Ααρών εναντίον του Μωϋσέως ένεκα της Αιθιοπίσσης την οποίαν έλαβε· διότι γυναίκα Αιθιόπισσαν έλαβε·
\par 2 και είπαν, Μήπως προς τον Μωϋσήν μόνον ελάλησεν ο Κύριος; δεν ελάλησε και προς εμάς; Και ήκουσε τούτο ο Κύριος.
\par 3 Και ο άνθρωπος ο Μωϋσής ήτο πραΰς σφόδρα υπέρ πάντας τους ανθρώπους τους επί της γης.
\par 4 Και είπε Κύριος παρευθύς προς τον Μωϋσήν και προς τον Ααρών και προς την Μαριάμ, Εξέλθετε σεις οι τρεις προς την σκηνήν του μαρτυρίου. Και εξήλθον οι τρεις.
\par 5 Και κατέβη ο Κύριος εν στύλω νεφέλης και εστάθη εις την θύραν της σκηνής του μαρτυρίου, και εκάλεσε τον Ααρών και την Μαριάμ· και εξήλθον αμφότεροι.
\par 6 Και είπεν, Ακούσατε τώρα τους λόγους μου· Εάν ήναι μεταξύ σας προφήτης, εγώ ο Κύριος δι' οπτασίας θέλω γνωρισθή εις αυτόν· καθ' ύπνον θέλω λαλήσει προς αυτόν·
\par 7 δεν είναι ούτως περί του θεράποντός μου Μωϋσέως· εν όλω τω οίκω μου ούτος είναι πιστός·
\par 8 στόμα προς στόμα θέλω λαλεί προς αυτόν και φανερώς και ουχί δι' αινιγμάτων, και το πρόσωπον του Κυρίου θέλει βλέπει· διά τι λοιπόν δεν εφοβήθητε να λαλήσητε εναντίον του θεράποντός μου Μωϋσέως;
\par 9 Και εξήφθη η οργή του Κυρίου κατ' αυτών και ανεχώρησε.
\par 10 Και η νεφέλη απεμακρύνθη από της σκηνής, και ιδού, η Μαριάμ έγεινε λεπρά ως χιών· και είδεν ο Ααρών την Μαριάμ και ιδού, ήτο λεπρά.
\par 11 Και είπεν ο Ααρών προς τον Μωϋσήν, Δέομαι, κύριέ μου, μη επιθέσης την αμαρτίαν εφ' ημάς, επειδή επράξαμεν ανοήτως και επειδή ημαρτήσαμεν·
\par 12 ας μη ήναι αυτή ως έκτρωμα, του οποίου είναι φαγωμένον το ήμισυ της σαρκός, ότε εξέρχεται εκ της μήτρας της μητρός αυτού.
\par 13 Και εβόησεν ο Μωϋσής προς τον Κύριον, λέγων, Δέομαι, Θεέ, ιάτρευσον αυτήν.
\par 14 Και είπε Κύριος προς τον Μωϋσήν, Εάν ο πατήρ αυτής μόνον έπτυεν εις το πρόσωπον αυτής, δεν ήθελεν είσθαι κατησχυμμένη επτά ημέρας; ας αποχωρισθή επτά ημέρας από του στρατοπέδου, και μετά ταύτα ας επιστρέψη.
\par 15 Και απεχωρίσθη η Μαριάμ από του στρατοπέδου επτά ημέρας· και ο λαός δεν εσηκώθη εωσού επέστρεψεν η Μαριάμ.
\par 16 Και μετά ταύτα εσηκώθη ο λαός από Ασηρώθ και εστρατοπέδευσαν εν τη ερήμω Φαράν.

\chapter{13}

\par 1 Και ελάλησε Κύριος προς τον Μωϋσήν λέγων,
\par 2 Απόστειλον άνδρας διά να κατασκοπεύσωσι την γην Χαναάν, την οποίαν εγώ δίδω εις τους υιούς Ισραήλ· από πάσης φυλής των πατέρων αυτών θέλετε αποστείλει ανά ένα άνδρα, έκαστον εξ αυτών αρχηγόν.
\par 3 Και απέστειλεν αυτούς ο Μωϋσής διά προσταγής του Κυρίου εκ της ερήμου Φαράν. Πάντες οι άνδρες ούτοι ήσαν αρχηγοί των υιών Ισραήλ.
\par 4 Και ταύτα ήσαν τα ονόματα αυτών· εκ της φυλής Ρουβήν, Σαμμουά ο υιός του Σακχούρ·
\par 5 εκ της φυλής Συμεών, Σαφάτ ο υιός του Χορρί·
\par 6 εκ της φυλής Ιούδα, Χάλεβ ο υιός του Ιεφοννή·
\par 7 εκ της φυλής Ισσάχαρ, Ιγάλ ο υιός του Ιωσήφ·
\par 8 εκ της φυλής Εφραΐμ, Αυσή ο υιός του Ναυή·
\par 9 εκ της φυλής Βενιαμίν, Φαλτί ο υιός του Ραφού·
\par 10 εκ της φυλής Ζαβουλών, Γαδιήλ ο υιός του Σουδί·
\par 11 εκ της φυλής Ιωσήφ, εκ της φυλής Μανασσή, Γαδδί ο υιός του Σουσί·
\par 12 εκ της φυλής Δαν, Αμμιήλ ο υιός του Γεμαλί·
\par 13 εκ της φυλής Ασήρ, Σεθούρ ο υιός του Μιχαήλ·
\par 14 εκ της φυλής Νεφθαλί, Νααβί ο υιός του Βαυσί·
\par 15 εκ της φυλής Γαδ, Γεουήλ ο υιός του Μαχί.
\par 16 Ταύτα είναι τα ονόματα των ανδρών, τους οποίους απέστειλεν ο Μωϋσής διά να κατασκοπεύσωσι την γήν· και επωνόμασεν ο Μωϋσής τον Αυσή, τον υιόν του Ναυή, Ιησούν.
\par 17 Και απέστειλεν αυτούς ο Μωϋσής διά να κατασκοπεύσωσι την γην Χαναάν· και είπε προς αυτούς, Ανάβητε από του μέρους τούτου της μεσημβρίας και θέλετε αναβή εις το όρος·
\par 18 και θεωρήσατε την γην, οποία είναι, και τον λαόν τον κατοικούντα εν αυτή, εάν ήναι δυνατός ή αδύνατος, ολίγοι ή πολλοί·
\par 19 και οποία είναι η γη εν ή ούτοι κατοικούσι, καλή ήναι ή κακή· και οποίαι είναι αι πόλεις, τας οποίας ούτοι κατοικούσιν, ατείχιστοι ή περιτετειχισμέναι·
\par 20 και οποία είναι η γη, παχεία ήναι ή λεπτή, εάν υπάρχωσιν εν αυτή δένδρα ή ουχί· και ανδρίζεσθε και φέρετε από των καρπών της γης. Αι δε ημέραι ήσαν αι ημέραι των πρώτων σταφυλών.
\par 21 Και αναβάντες κατεσκόπευσαν την γην από της ερήμου Σιν έως Ρεώβ, κατά την είσοδον Αιμάθ.
\par 22 Και ανέβησαν κατά το μεσημβρινόν, και ήλθον έως Χεβρών, όπου ήσαν Αχιμάν, Σεσαΐ και Θαλμαΐ, οι υιοί του Ανάκ. Η δε Χεβρών εκτίσθη επτά έτη προ της Τάνεως της Αιγύπτου.
\par 23 Και ήλθον μέχρι της φάραγγος Εσχώλ, και έκοψαν εκείθεν κλήμα αμπέλου μετά ενός βότρυος σταφυλής, και εβάσταζον αυτό δύο επί ξύλου· έφεραν και ρόδια και σύκα.
\par 24 Ο τόπος εκείνος ωνομάσθη φάραγξ Εσχώλ· διά τον βότρυν τον οποίον έκοψαν εκείθεν οι υιοί Ισραήλ.
\par 25 Και επέστρεψαν, αφού κατεσκόπευσαν την γην, μετά τεσσαράκοντα ημέρας.
\par 26 Και πορευθέντες ήλθον προς τον Μωϋσήν, και προς τον Ααρών, και προς πάσαν την συναγωγήν των υιών Ισραήλ, εκ τη ερήμω Φαράν, εις Κάδης· και έφεραν απόκρισιν προς αυτούς, και προς πάσαν την συναγωγήν, και έδειξαν εις αυτούς τον καρπόν της γης.
\par 27 Και απήγγειλαν προς αυτόν, και είπον, Ήλθομεν εις την γην, εις την οποίαν απέστειλας εμάς, και είναι τωόντι γη ρέουσα γάλα και μέλι και ιδού, ο καρπός αυτής·
\par 28 ο λαός όμως, ο κατοικών εν τη γη, είναι δυνατός, και αι πόλεις περιτετειχισμέναι, μεγάλαι σφόδρα· προς τούτοις, είδομεν εκεί και τους υιούς του Ανάκ·
\par 29 οι Αμαληκίται κατοικούσιν εν τη γη της μεσημβρίας· και οι Χετταίοι, και οι Ιεβουσαίοι, και οι Αμορραίοι, κατοικούσιν επί τα όρη· και οι Χαναναίοι κατοικούσι παρά την θάλασσαν και τας όχθας του Ιορδάνου.
\par 30 Και ο Χάλεβ κατεσίγασε τον λαόν έμπροσθεν του Μωϋσέως, και είπεν, Ας αναβώμεν ευθύς, και ας εξουσιάσωμεν αυτήν· διότι δυνάμεθα να κυριεύσωμεν αυτήν.
\par 31 Οι άνθρωποι όμως, οι συναναβάντες μετ' αυτού, είπαν, Δεν δυνάμεθα να αναβώμεν επί τον λαόν τούτον, διότι είναι δυνατώτεροι ημών.
\par 32 Και δυσφήμησαν την γην, την οποίαν κατεσκόπευσαν, προς τους υιούς Ισραήλ, λέγοντες, Η γη, την οποίαν διεπεράσαμεν διά να κατασκοπεύσωμεν αυτήν, είναι γη κατατρώγουσα τους κατοίκους αυτής· και πας ο λαός, τον οποίον είδομεν εν αυτή είναι άνδρες υπερμεγέθεις·
\par 33 και είδομεν εκεί τους γίγαντας, τους υιούς Ανάκ, του εκ των γιγάντων· και εβλέπομεν εαυτούς ως ακρίδας και τοιούτους έβλεπον ημάς αυτοί.

\chapter{14}

\par 1 Και πάσα η συναγωγή υψώσασα την φωνήν αυτής εβόησε· και έκλαυσεν ο λαός την νύκτα εκείνην.
\par 2 Και πάντες οι υιοί Ισραήλ εγόγγυζον κατά του Μωϋσέως και του Ααρών, και είπε προς αυτούς πάσα η συναγωγή, Είθε να απεθνήσκομεν εν γη Αιγύπτου· ή εν τη ερήμω ταύτη είθε να απεθνήσκομεν·
\par 3 και διά τι μας έφερεν ο Κύριος εις την γην ταύτην να πέσωμεν διά μαχαίρας, να γείνωσι διαρπαγή αι γυναίκες και τα τέκνα ημών; δεν ήτο καλήτερον εις ημάς να επιστρέψωμεν εις την Αίγυπτον;
\par 4 Και έλεγεν ο εις προς τον άλλον, Ας κάμωμεν αρχηγόν και ας επιστρέψωμεν εις την Αίγυπτον.
\par 5 Τότε έπεσεν ο Μωϋσής και ο Ααρών κατά πρόσωπον αυτών ενώπιον όλου του πλήθους της συναγωγής των υιών Ισραήλ.
\par 6 Και Ιησούς ο υιός του Ναυή και Χάλεβ ο υιός του Ιεφοννή, εκ των κατασκοπευσάντων την γην, διέσχισαν τα ιμάτια αυτών·
\par 7 και είπον προς πάσαν την συναγωγήν των υιών Ισραήλ λέγοντες, Η γη, την οποίαν διεπεράσαμεν διά να κατασκοπεύσωμεν αυτήν, είναι γη αγαθή σφόδρα σφόδρα·
\par 8 εάν ο Κύριος ευαρεστήται εις ημάς, τότε θέλει φέρει εμάς εις την γην ταύτην και θέλει δώσει αυτήν εις ημάς, γην ρέουσαν γάλα και μέλι·
\par 9 μόνον μη αποστατείτε κατά του Κυρίου μηδέ φοβείσθε τον λαόν της γής· διότι αυτοί είναι ψωμίον δι' ημάς· η σκέπη αυτών απεσύρθη επάνωθεν αυτών, και ο Κύριος είναι μεθ' ημών· μη φοβείσθε αυτούς.
\par 10 Και είπε πάσα η συναγωγή να λιθοβολήσωσιν αυτούς με λίθους· Και η δόξα του Κυρίου επεφάνη επί τη σκηνή του μαρτυρίου εις πάντας τους υιούς Ισραήλ.
\par 11 Και είπε Κύριος προς τον Μωϋσήν, Έως πότε θέλει με παροργίζει ο λαός ούτος; και έως πότε δεν θέλουσι πιστεύει εις εμέ, μετά πάντα τα σημεία τα οποία έκαμα εν μέσω αυτών;
\par 12 θέλω πατάξει αυτούς με θανατικόν και θέλω εξολοθρεύσει αυτούς, και σε θέλω κάμει εις έθνος μεγαλήτερον και δυνατώτερον αυτών.
\par 13 Και είπεν ο Μωϋσής προς τον Κύριον, Τότε η Αίγυπτος θέλει ακούσει διότι συ ανεβίβασας τον λαόν τούτον εν τη δυνάμει σου εκ μέσου αυτών·
\par 14 και θέλουσιν ειπεί τούτο προς τους κατοίκους της γης ταύτης· οίτινες ήκουσαν ότι συ, Κύριε, είσαι εν μέσω του λαού τούτου, ότι συ, Κύριε, φαίνεσαι πρόσωπον προς πρόσωπον, και η νεφέλη σου ίσταται επ' αυτούς, και συ προπορεύεσαι αυτών την ημέραν εν στύλω νεφέλης, την δε νύκτα εν στύλω πυρός.
\par 15 Εάν λοιπόν θανατώσης τον λαόν τούτον ως ένα άνθρωπον, τότε τα έθνη, τα οποία ήκουσαν το όνομά σου, θέλουσιν ειπεί λέγοντες,
\par 16 Επειδή δεν ηδύνατο ο Κύριος να φέρη τον λαόν τούτον εις την γην, την οποίαν ώμοσε προς αυτούς, διά τούτο κατέστρεψεν αυτούς εν τη ερήμω.
\par 17 Και τώρα, δέομαί σου, ας μεγαλυνθή η δύναμις του Κυρίου μου καθ' ον τρόπον είπας λέγων,
\par 18 Ο Κύριος είναι μακρόθυμος και πολυέλεος, συγχωρών ανομίαν και παράβασιν, και όστις κατ' ουδένα τρόπον δεν θέλει αθωώσει τον ένοχον, ανταποδίδων την ανομίαν των πατέρων επί τα τέκνα έως τρίτης και τετάρτης γενεάς.
\par 19 Συγχώρησον, δέομαι, την ανομίαν του λαού τούτου κατά το μέγα έλεός σου και καθώς συνεχώρησας τον λαόν τούτον από της Αιγύπτου μέχρι του νυν.
\par 20 Και είπε Κύριος, Συνεχώρησα αυτούς κατά τον λόγον σου·
\par 21 αλλά ζω εγώ, και θέλει εμπλησθή πάσα η γη από της δόξης του Κυρίου.
\par 22 Επειδή πάντες οι άνδρες, οι ιδόντες την δόξαν μου και τα σημείά μου, τα οποία έκαμον εν τη Αιγύπτω και εν τη ερήμω, με παρώργισαν ήδη δεκάκις και δεν υπήκουσαν εις την φωνήν μου,
\par 23 βεβαίως δεν θέλουσιν ιδεί την γην, την οποίαν ώμοσα προς τους πατέρας αυτών· ουδείς εκ των παροργισάντων με θέλει ιδεί αυτήν.
\par 24 Τον δε δούλον μου Χάλεβ, επειδή έχει εν εαυτώ άλλο πνεύμα και με ηκολούθησεν εντελώς, τούτον θέλω φέρει εις την γην εις την οποίαν εισήλθε, και το σπέρμα αυτού θέλει κληρονομήσει αυτήν.
\par 25 Οι δε Αμαληκίται και οι Χαναναίοι κατοικούσιν εν τη κοιλάδι. Αύριον στρέψατε και υπάγετε εις την έρημον κατά την οδόν της Ερυθράς θαλάσσης.
\par 26 Και είπε Κύριος προς τον Μωϋσήν και προς τον Ααρών λέγων,
\par 27 Έως πότε θέλω υποφέρει την συναγωγήν ταύτην την πονηράν, όσα αυτοί γογγύζουσιν εναντίον μου; ήκουσα τους γογγυσμούς των υιών Ισραήλ, τους οποίους γογγύζουσιν εναντίον μου.
\par 28 Ειπέ προς αυτούς, Ζω εγώ, λέγει ο Κύριος, καθώς σεις ελαλήσατε εις τα ώτα μου, ούτω βεβαίως θέλω κάμει εις εσάς·
\par 29 τα πτώματά σας θέλουσι πέσει εν τη ερήμω ταύτη· και πάντες οι απηριθμημένοι από σας καθ' όλον τον αριθμόν σας, από είκοσι ετών και επάνω, όσοι εγόγγυσαν εναντίον μου,
\par 30 βεβαίως δεν θέλετε εισέλθει σεις εις την γην, περί της οποίας ώμοσα να σας κατοικίσω εν αυτή, εκτός του Χάλεβ υιού του Ιεφοννή και του Ιησού υιού του Ναυή·
\par 31 αλλά τα παιδία σας, τα οποία είπετε ότι θέλουσι γίνει εις διαρπαγήν, ταύτα θέλω εισαγάγει, και θέλουσι γνωρίσει την γην την οποίαν σεις κατεφρονήσατε·
\par 32 τα δε πτώματα υμών θέλουσι πέσει εν τη ερήμω ταύτη·
\par 33 και τα τέκνα σας θέλουσι περιπλανάσθαι εν τη ερήμω τεσσαράκοντα έτη και θέλουσι φέρει εαυτά την ποινήν της πορνείας σας, εωσού διαφθαρώσι τα πτώματά σας εν τη ερήμω·
\par 34 κατά τον αριθμόν των ημερών εις τας οποίας κατεσκοπεύσατε την γην, ημέρας τεσσαράκοντα, εκάστης ημέρας λογιζομένης δι' εν έτος, τεσσαράκοντα έτη θέλετε φέρει εφ' εαυτούς τας ανομίας σας, και θέλετε γνωρίσει την εγκατάλειψίν μου.
\par 35 Εγώ ο Κύριος ελάλησα· βεβαίως θέλω κάμει τούτο εις πάσαν την συναγωγήν την πονηράν ταύτην, την επισυνηγμένην επ' εμέ· εν τη ερήμω ταύτη θέλουσιν εξολοθρευθή και εκεί θέλουσιν αποθάνει.
\par 36 Και οι άνθρωποι, τους οποίους απέστειλεν ο Μωϋσής διά να κατασκοπεύσωσι την γην, οίτινες επιστρέψαντες έκαμον πάσαν την συναγωγήν να γογγύση εναντίον αυτού, δυσφημούντες την γην,
\par 37 και οι άνθρωποι εκείνοι, οίτινες εδυσφήμησαν την γην, απέθανον εν τη πληγή ενώπιον του Κυρίου.
\par 38 Ιησούς δε ο υιός του Ναυή και Χάλεβ ο υιός του Ιεφοννή έζησαν, εκ των ανθρώπων εκείνων οίτινες υπήγαν να κατασκοπεύσωσι την γην.
\par 39 Και ελάλησεν ο Μωϋσής τους λόγους τούτους προς πάντας τους υιούς Ισραήλ· και επένθησεν ο λαός σφόδρα.
\par 40 Και σηκωθέντες ενωρίς το πρωΐ, ανέβησαν εις την κορυφήν του όρους, λέγοντες, Ιδού, ημείς, και θέλομεν αναβή εις τον τόπον τον οποίον μας υπεσχέθη ο Κύριος· διότι ημαρτήσαμεν.
\par 41 Και είπεν ο Μωϋσής, Διά τι σεις παραβαίνετε την προσταγήν του Κυρίου; τούτο βεβαίως δεν θέλει ευοδοθή·
\par 42 μη αναβαίνετε· διότι δεν είναι ο Κύριος μεθ' υμών· διά να μη κτυπηθήτε έμπροσθεν των εχθρών σας·
\par 43 διότι οι Αμαληκίται και οι Χαναναίοι είναι εκεί έμπροσθέν σας και θέλετε πέσει εν μαχαίρα· επειδή εξεκλίνατε από του Κυρίου, διά τούτο ο Κύριος δεν θέλει είσθαι μεθ' υμών.
\par 44 Αλλ' αυτοί απετόλμησαν να αναβώσιν εις την κορυφήν του όρους· η κιβωτός όμως της διαθήκης του Κυρίου και ο Μωϋσής δεν εκινήθησαν εκ μέσου του στρατοπέδου.
\par 45 Τότε οι Αμαληκίται και οι Χαναναίοι οι κατοικούντες εν τω όρει εκείνω, κατέβησαν και επάταξαν αυτούς και κατεδίωξαν αυτούς έως Ορμά.

\chapter{15}

\par 1 Και ελάλησε Κύριος προς τον Μωϋσήν λέγων,
\par 2 Λάλησον προς τους υιούς Ισραήλ και ειπέ προς αυτούς, Όταν εισέλθητε εις την γην της κατοικήσεώς σας, την οποίαν εγώ δίδω εις εσάς,
\par 3 και κάμητε προσφοράν διά πυρός προς τον Κύριον, ολοκαύτωμα, θυσίαν εις εκπλήρωσιν ευχής ή αυτοπροαιρέτως ή εις τας εορτάς σας, διά να κάμητε οσμήν ευωδίας προς τον Κύριον, είτε εκ των βοών είτε εκ των προβάτων,
\par 4 τότε ο προσφέρων το δώρον αυτού προς τον Κύριον θέλει φέρει προσφοράν εξ αλφίτων από ενός δεκάτου σεμιδάλεως, εζυμωμένης με το τέταρτον ενός ιν ελαίου·
\par 5 και οίνον διά σπονδήν, το τέταρτον ενός ιν, θέλεις προσθέσει εις το ολοκαύτωμα ή την θυσίαν, δι' έκαστον αρνίον.
\par 6 Η δι' έκαστον κριόν θέλεις προσθέσει προσφοράν εξ αλφίτων, δύο δέκατα σεμιδάλεως εζυμωμένης με το τρίτον ενός ιν ελαίου·
\par 7 και οίνον διά σπονδήν θέλεις προσφέρει, το τρίτον ενός ιν, εις οσμήν ευωδίας προς τον Κύριον.
\par 8 Εάν δε προσφέρης μόσχον εκ βοών δι' ολοκαύτωμα ή διά θυσίαν προς εκπλήρωσιν ευχής ή διά ειρηνικήν προσφοράν προς τον Κύριον,
\par 9 τότε θέλεις φέρει μετά του μόσχου εκ βοών προσφοράν εξ αλφίτων, τρία δέκατα σεμιδάλεως εζυμωμένης με εν ήμισυ ιν ελαίου·
\par 10 και θέλεις φέρει οίνον διά σπονδήν, το ήμισυ του ιν, εις προσφοράν γινομένην διά πυρός, εις οσμήν ευωδίας προς τον Κύριον.
\par 11 ούτω θέλει γίνεσθαι δι' ένα μόσχον ή δι' ένα κριόν ή δι' αρνίον ή διά τράγον.
\par 12 Κατά τον αριθμόν τον οποίον θέλετε προσφέρει, ούτω θέλετε κάμει εις έκαστον κατά τον αριθμόν αυτών.
\par 13 Πάντες οι αυτόχθονες θέλουσι κάμει ταύτα κατά τον τρόπον τούτον, προσφέροντες προσφοράν γινομένην διά πυρός εις οσμήν ευωδίας προς τον Κύριον.
\par 14 Και εάν παροική μεταξύ σας ξένος ή οποιοσδήποτε είναι μεταξύ σας εις τας γενεάς σας, και θέλη να κάμη προσφοράν γινομένην διά πυρός εις οσμήν ευωδίας προς τον Κύριον, καθώς σεις κάμνετε, ούτω θέλει κάμει·
\par 15 εις νόμος θέλει είσθαι διά σας τους εκ της συναγωγής και διά τον ξένον τον παροικούντα μεταξύ σας, νόμιμον αιώνιον εις τας γενεάς σας· καθώς σεις, ούτω θέλει είσθαι και ο ξένος ενώπιον του Κυρίου·
\par 16 εις νόμος και μία διάταξις θέλει είσθαι διά σας και διά τον ξένον τον παροικούντα μεταξύ σας.
\par 17 Και ελάλησε Κύριος προς τον Μωϋσήν, λέγων,
\par 18 Λάλησον προς τους υιούς Ισραήλ και ειπέ προς αυτούς, Όταν έλθητε εις την γην, εις την οποίαν εγώ σας φέρω,
\par 19 τότε όταν φάγητε εκ των άρτων της γης, θέλετε προσφέρει εις τον Κύριον προσφοράν υψουμένην.
\par 20 Θέλετε προσφέρει άρτον εκ της πρώτης ζύμης σας, εις προσφοράν υψουμένην· καθώς την προσφοράν την υψουμένην από του αλωνίου σας, ούτω θέλετε υψώσει αυτήν.
\par 21 Εκ του πρώτου της ζύμης σας θέλετε δώσει εις τον Κύριον προσφοράν υψουμένην εις τας γενεάς σας.
\par 22 Και εάν σφάλητε και δεν πράξητε πάντα ταύτα τα προστάγματα, τα οποία ελάλησε Κύριος προς τον Μωϋσήν,
\par 23 κατά πάντα όσα προσέταξεν ο Κύριος εις εσάς διά χειρός του Μωϋσέως αφ' ης ημέρας ο Κύριος προσέταξε και μετά ταύτα εις τας γενεάς σας·
\par 24 τότε, εάν γείνη τι εξ αγνοίας, χωρίς να εξεύρη αυτό η συναγωγή, πάσα η συναγωγή θέλει προσφέρει ένα μόσχον εκ βοών διά ολοκαύτωμα, εις οσμήν ευωδίας προς τον Κύριον, μετά της εξ αλφίτων προσφοράς αυτού και της σπονδής αυτού κατά το διατεταγμένον, και ένα τράγον εξ αιγών εις προσφοράν περί αμαρτίας·
\par 25 και θέλει κάμει ο ιερεύς εξιλέωσιν υπέρ πάσης της συναγωγής των υιών Ισραήλ, και θέλει συγχωρηθή εις αυτούς· διότι έγεινεν εξ αγνοίας· και θέλουσι φέρει την προσφοράν αυτών, θυσίαν γινομένην διά πυρός προς τον Κύριον, και την περί αμαρτίας προσφοράν αυτών, ενώπιον του Κυρίου, διά την άγνοιαν αυτών·
\par 26 και θέλει συγχωρηθή εις πάσαν την συναγωγήν των υιών Ισραήλ και εις τον ξένον τον παροικούντα μεταξύ αυτών· διότι πας ο λαός ήμαρτεν εξ αγνοίας.
\par 27 Εάν δε ψυχή τις αμαρτήση εξ αγνοίας, ούτος πρέπει να φέρη αίγα ενιαύσιον εις προσφοράν περί αμαρτίας·
\par 28 και θέλει κάμει εξιλέωσιν ο ιερεύς υπέρ της ψυχής ήτις ημάρτησεν εξ αγνοίας, όταν αμαρτήση εξ αγνοίας ενώπιον του Κυρίου, διά να κάμη εξιλέωσιν υπέρ αυτού· και θέλει συγχωρηθή εις αυτόν.
\par 29 Εις νόμος θέλει είσθαι εις εσάς διά τον αυτόχθονα μεταξύ των υιών Ισραήλ και διά τον ξένον τον παροικούντα μεταξύ αυτών, όταν αμαρτήση εξ αγνοίας.
\par 30 Η δε ψυχή ήτις πράξη αμάρτημα με χείρα υπερήφανον, είτε αυτόχθων είτε ξένος, ούτος καταφρονεί τον Κύριον· και θέλει εξολοθρευθή η ψυχή εκείνη εκ μέσου του λαού αυτής.
\par 31 Επειδή κατεφρόνησε τον λόγον του Κυρίου και παρέβη την προσταγήν αυτού, η ψυχή εκείνη εξάπαντος θέλει εξολοθρευθή· η αμαρτία αυτής θέλει είσθαι επ' αυτήν.
\par 32 Και ότε ήσαν οι υιοί Ισραήλ εν τη ερήμω, εύρον άνθρωπον συλλέγοντα ξύλα την ημέραν του σαββάτου.
\par 33 Και οι ευρόντες αυτόν συλλέγοντα ξύλα έφεραν αυτόν προς τον Μωϋσήν και τον Ααρών και προς πάσαν την συναγωγήν·
\par 34 και έβαλον αυτόν εις φύλαξιν, επειδή δεν ήτο ότι φανερόν τι έπρεπε να κάμωσιν εις αυτόν.
\par 35 Και είπε Κύριος προς τον Μωϋσήν, Ο άνθρωπος εξάπαντος θέλει θανατωθή· πάσα η συναγωγή θέλει λιθοβολήσει αυτόν με λίθους έξω του στρατοπέδου.
\par 36 Και πάσα η συναγωγή έφεραν αυτόν έξω του στρατοπέδου και ελιθοβόλησαν αυτόν με λίθους και απέθανε· καθώς προσέταξε Κύριος εις τον Μωϋσήν.
\par 37 Και ελάλησε Κύριος προς τον Μωϋσήν λέγων,
\par 38 Λάλησον προς τους υιούς Ισραήλ και ειπέ προς αυτούς να κάμωσι κράσπεδα εις τα άκρα των ιματίων αυτών, εις τας γενεάς αυτών, και να βάλωσιν εις τα κράσπεδα των άκρων ταινίαν κυανήν·
\par 39 και θέλετε έχει αυτήν εις τα κράσπεδα, διά να βλέπητε αυτήν και να ενθυμήσθε πάσας τας εντολάς του Κυρίου και να εκτελήτε αυτάς, και να μη διαστραφήτε κατόπιν των καρδιών σας και κατόπιν των οφθαλμών σας, κατόπιν των οποίων σεις εκπορνεύετε·
\par 40 διά να ενθυμήσθε και να εκτελήτε πάσας τας εντολάς μου, και να ήσθε άγιοι εις τον Θεόν σας.
\par 41 Εγώ είμαι Κύριος ο Θεός σας, όστις εξήγαγον υμάς εκ γης Αιγύπτου, διά να ήμαι Θεός σας. Εγώ είμαι Κύριος ο Θεός σας.

\chapter{16}

\par 1 Ο δε Κορέ ο υιός του Ισαάρ, υιού του Καάθ, υιού του Λευΐ, και Δαθάν και Αβειρών οι υιοί του Ελιάβ, και Ων ο υιός του Φαλέθ, υιοί Ρουβήν, εστασίασαν,
\par 2 και εσηκώθησαν εναντίον του Μωϋσέως μετά διακοσίων πεντήκοντα ανθρώπων εκ των υιών Ισραήλ, αρχηγών της συναγωγής, συγκλήτων της βουλής, ανδρών ονομαστών·
\par 3 και συνήχθησαν εναντίον του Μωϋσέως και εναντίον του Ααρών και είπον προς αυτούς, Αρκεί εις εσάς, διότι πάσα η συναγωγή, πάντες είναι άγιοι και ο Κύριος είναι μεταξύ αυτών· και διά τι υψόνεσθε υπεράνω της συναγωγής του Κυρίου;
\par 4 Ακούσας δε ο Μωϋσής έπεσε κατά πρόσωπον αυτού·
\par 5 και ελάλησε προς τον Κορέ και προς πάσαν την συνοδίαν αυτού, λέγων, Το πρωΐ θέλει φανερώσει ο Κύριος ποίοι είναι αυτού και ποίος άγιος και θέλει κάμει αυτόν να πλησιάση εις αυτόν· και όντινα εξέλεξε, τούτον θέλει κάμει να πλησιάση εις αυτόν.
\par 6 Τούτο κάμετε, Λάβετε εις εαυτούς θυμιατήρια, ο Κορέ και πάσα η συνοδία αυτού·
\par 7 και βάλετε επ' αυτά πυρ και επιθέσατε θυμίαμα επ' αυτά ενώπιον του Κυρίου αύριον· και ο άνθρωπος τον οποίον εκλέξη ο Κύριος, ούτος θέλει είσθαι άγιος. Αρκεί εις εσάς, υιοί Λευΐ.
\par 8 Και είπεν ο Μωϋσής προς τον Κορέ, Ακούσατε τώρα, υιοί Λευΐ.
\par 9 Μικρόν πράγμα είναι τούτο εις εσάς, ότι εξεχώρισεν εσάς ο Θεός του Ισραήλ από της συναγωγής του Ισραήλ, διά να σας φέρη πλησίον αυτού να κάμνητε την υπηρεσίαν της σκηνής του Κυρίου και να στέκησθε έμπροσθεν της συναγωγής, διά να υπηρετήτε εις αυτούς;
\par 10 και αφού σε έφερε πλησίον εαυτού και πάντας τους αδελφούς σου τους υιούς Λευΐ μετά σου, σεις ζητείτε και την ιερατείαν;
\par 11 ούτω κάμνεις, συ και πάσα η συνοδία σου, οίτινες είσθε συνηθροισμένοι εναντίον του Κυρίου· και ο Ααρών τις είναι, ώστε να γογγύζητε εναντίον αυτού;
\par 12 Και έστειλεν ο Μωϋσής να καλέση τον Δαθάν και τον Αβειρών τους υιούς Ελιάβ· οι δε είπον, Δεν αναβαίνομεν·
\par 13 μικρόν είναι τούτο, ότι ανήγαγες ημάς εκ γης ρεούσης γάλα και μέλι, διά να θανατώσης ημάς εν τη ερήμω, και ότι ως άρχων θέλεις να κατεξουσιάζης ημάς;
\par 14 αλλά συ δεν έφερες ημάς εις γην ρέουσαν γάλα και μέλι ουδέ έδωκας εις ημάς κληρονομίαν αγρών και αμπελώνων· τους οφθαλμούς των ανθρώπων τούτων θέλεις να εκβάλης; δεν αναβαίνομεν.
\par 15 Και εβαρυθύμησεν ο Μωϋσής σφόδρα και είπε προς τον Κύριον, Μη επιβλέψης εις την προσφοράν αυτών· ουδέ ένα όνον απ' αυτών έλαβον ουδέ έβλαψα τινά εξ αυτών.
\par 16 Και είπεν ο Μωϋσής προς τον Κορέ, Συ και πάσα η συνοδία σου να ήσθε ενώπιον του Κυρίου, συ, και αυτοί, και ο Ααρών, αύριον·
\par 17 και λάβετε έκαστος το θυμιατήριον αυτού και επιθέσατε θυμίαμα επ' αυτά και φέρετε ενώπιον του Κυρίου, έκαστος το θυμιατήριον αυτού, διακόσια πεντήκοντα θυμιατήρια· και συ, και ο Ααρών, έκαστος το θυμιατήριον αυτού.
\par 18 Και έλαβον έκαστος το θυμιατήριον αυτού και έβαλον επ' αυτά πυρ, και επέθεσαν επ' αυτά θυμίαμα και εστάθησαν εις την θύραν της σκηνής του μαρτυρίου μετά του Μωϋσέως και του Ααρών.
\par 19 Και συνήγαγεν εναντίον αυτών ο Κορέ πάσαν την συναγωγήν εις την θύραν της σκηνής του μαρτυρίου. Και η δόξα του Κυρίου εφάνη εις πάσαν την συναγωγήν.
\par 20 Και ελάλησε Κύριος προς τον Μωϋσήν και προς τον Ααρών λέγων,
\par 21 Αποχωρίσθητε εκ μέσου της συναγωγής ταύτης, διά να εξαναλώσω αυτούς διά μιας.
\par 22 Και έπεσαν κατά πρόσωπον αυτών, και είπον, Ω Θεέ, Θεέ των πνευμάτων πάσης σαρκός, εις άνθρωπος ημάρτησε και θέλεις οργισθή εναντίον πάσης της συναγωγής;
\par 23 Και ελάλησε Κύριος προς τον Μωϋσήν, λέγων,
\par 24 Λάλησον προς την συναγωγήν, λέγων, Αναχωρήσατε από της σκηνής του Κορέ, του Δαθάν και του Αβειρών κυκλόθεν.
\par 25 Και εσηκώθη ο Μωϋσής, και υπήγε προς τον Δαθάν και Αβειρών· και ηκολούθησαν αυτόν οι πρεσβύτεροι του Ισραήλ.
\par 26 Και ελάλησε προς την συναγωγήν λέγων, Αποχωρίσθητε ευθύς από των σκηνών των ασεβών τούτων ανθρώπων και μη εγγίσητε μηδέν εκ των όσα είναι αυτών, διά να μη αφανισθήτε εν μέσω πασών των αμαρτιών αυτών.
\par 27 Ανεχώρησαν λοιπόν από της σκηνής του Κορέ, του Δαθάν και του Αβειρών κυκλόθεν· και ο Δαθάν και ο Αβειρών εξήλθον, και εστάθησαν εις την θύραν των σκηνών αυτών και αι γυναίκες αυτών και οι υιοί αυτών και αι οικογένειαι αυτών.
\par 28 Και είπεν ο Μωϋσής, Εκ τούτου θέλετε γνωρίσει ότι ο Κύριος με απέστειλε διά να πράξω πάντα τα έργα ταύτα, και ότι δεν έπραξα απ' εμαυτού.
\par 29 Εάν οι άνθρωποι ούτοι αποθάνωσι τον κοινόν θάνατον πάντων των ανθρώπων, ή εάν γείνη ανταπόδοσις εις αυτούς κατά την ανταπόδοσιν πάντων των ανθρώπων, δεν με απέστειλεν ο Κύριος·
\par 30 εάν όμως ο Κύριος κάμη θαύμα, και ανοίξη η γη το στόμα αυτής και καταπίη αυτούς και πάντα τα αυτών και καταβώσι ζώντες εις τον άδην, τότε θέλετε γνωρίσει ότι παρώξυναν οι άνθρωποι ούτοι τον Κύριον.
\par 31 Και ως έπαυσε λαλών πάντας τους λόγους τούτους, εσχίσθη το έδαφος το υποκάτω αυτών.
\par 32 Και η γη ήνοιξε το στόμα αυτής και κατέπιεν αυτούς και τους οίκους αυτών και πάντας τους ανθρώπους τους μετά του Κορέ και πάσαν την περιουσίαν αυτών.
\par 33 Και κατέβησαν αυτοί και πάντα τα αυτών ζώντες εις τον άδην, και η γη εκλείσθη επάνωθεν αυτών· και ηφανίσθησαν εκ μέσου της συναγωγής.
\par 34 Και πας ο Ισραήλ ο πέριξ αυτών έφυγον εις την βοήν αυτών, λέγοντες, Μήπως καταπίη και ημάς η γη.
\par 35 Και πυρ εξήλθε παρά Κυρίου και κατέφαγε τους διακοσίους πεντήκοντα άνδρας τους προσφέροντας το θυμίαμα.
\par 36 Και ελάλησε Κύριος προς τον Μωϋσήν λέγων,
\par 37 Ειπέ προς τον Ελεάζαρ τον υιόν Ααρών του ιερέως, να λάβη τα θυμιατήρια από της πυρκαϊάς και το πυρ σκόρπισον εκεί· διότι είναι ηγιασμένα,
\par 38 τα θυμιατήρια τούτων των αμαρτησάντων εναντίον εις τας ψυχάς αυτών· και ας κάμωσιν αυτά πλάκας διά κάλυμμα του θυσιαστηρίου· επειδή ούτοι προσέφεραν αυτά ενώπιον του Κυρίου, διά τούτο είναι ηγιασμένα· και θέλουσιν είσθαι διά σημείον εις τους υιούς Ισραήλ.
\par 39 Και έλαβεν Ελεάζαρ ο ιερεύς τα χάλκινα θυμιατήρια, τα οποία προσέφεραν οι καυθέντες· και έκαμον αυτά πλάκας διά να καλύψωσι το θυσιαστήριον·
\par 40 προς μνημόσυνον εις τους υιούς Ισραήλ, ώστε μηδείς αλλογενής, μη ων εκ του σπέρματος του Ααρών, να μη πλησιάζη διά να προσφέρη θυμίαμα ενώπιον του Κυρίου, διά να μη γείνη ως ο Κορέ και ως η συνοδία αυτού, καθώς είπε Κύριος προς αυτόν διά χειρός του Μωϋσέως.
\par 41 Την δε ακόλουθον ημέραν πάσα η συναγωγή των υιών Ισραήλ εγόγγυσαν εναντίον του Μωϋσέως και του Ααρών, λέγοντες, Σεις εφονεύσατε τον λαόν του Κυρίου.
\par 42 Και ενώ η συναγωγή ήτο συνηθροισμένη εναντίον του Μωϋσέως και εναντίον του Ααρών, ανέβλεψαν προς την σκηνήν του μαρτυρίου, και ιδού, η νεφέλη εσκέπασεν αυτήν, και εφάνη η δόξα του Κυρίου.
\par 43 Και ήλθεν ο Μωϋσής και ο Ααρών έμπροσθεν της σκηνής του μαρτυρίου.
\par 44 Και ελάλησε Κύριος προς τον Μωϋσήν λέγων·
\par 45 Αποσύρθητε εκ μέσου της συναγωγής ταύτης, διά να αναλώσω αυτούς διά μιας. Και έπεσον κατά πρόσωπον αυτών.
\par 46 Και είπεν ο Μωϋσής προς τον Ααρών, Λάβε το θυμιατήριον και βάλε πυρ εις αυτό εκ του θυσιαστηρίου και επίθες θυμίαμα και ύπαγε ταχέως εις την συναγωγήν και κάμε εξιλέωσιν υπέρ αυτών· διότι εξήλθεν οργή παρά του Κυρίου· η πληγή ήρχισε.
\par 47 Και έλαβε το θυμιατήριον ο Ααρών, καθώς ελάλησεν ο Μωϋσής, και έδραμεν εις το μέσον της συναγωγής· και ιδού, η πληγή είχεν αρχίσει εν τω λαώ· και επέθεσε θυμίαμα και έκαμεν εξιλέωσιν υπέρ του λαού.
\par 48 Και εστάθη αναμέσον των αποθανόντων και των ζώντων, και έπαυσεν η θραύσις.
\par 49 Ήσαν δε οι αποθανόντες εις την θραύσιν δεκατέσσαρες χιλιάδες και επτακόσιοι, εκτός των αποθανόντων εξ αιτίας του Κορέ.
\par 50 Και επέστρεψεν ο Ααρών προς τον Μωϋσήν, εις την θύραν της σκηνής του μαρτυρίου· και έπαυσεν η θραύσις.

\chapter{17}

\par 1 Και ελάλησε Κύριος προς τον Μωϋσήν λέγων,
\par 2 Λάλησον προς τους υιούς Ισραήλ, και λάβε παρ' εκάστου αυτών ράβδον κατά τον οίκον των πατέρων αυτών, παρά πάντων των αρχόντων αυτών κατά τον οίκον των πατέρων αυτών, δώδεκα ράβδους· εκάστου το όνομα επίγραψον επί της ράβδου αυτού·
\par 3 και το όνομα του Ααρών επίγραψον επί της ράβδου του Λευΐ, επειδή μία ράβδος θέλει είσθαι δι' έκαστον αρχηγόν του οίκου των πατέρων αυτών·
\par 4 και θέλεις αποθέσει αυτάς εν τη σκηνή του μαρτυρίου έμπροσθεν του μαρτυρίου, όπου θέλω ευρίσκεσθαι μεθ' υμών·
\par 5 και η ράβδος του ανθρώπου, όντινα εκλέξω, θέλει βλαστήσει και θέλω κάμει να παύσωσιν απ' έμπροσθέν μου οι γογγυσμοί των υιών Ισραήλ, τους οποίους αυτοί γογγύζουσιν εναντίον σας.
\par 6 Και ελάλησεν ο Μωϋσής προς τους υιούς Ισραήλ· και έδωκαν εις αυτόν πάντες οι άρχοντες αυτών ανά μίαν ράβδον έκαστος άρχων κατά τους οίκους των πατέρων αυτών, δώδεκα ράβδους· και ράβδος του Ααρών ήτο μεταξύ των ράβδων αυτών.
\par 7 Και απέθηκεν ο Μωϋσής τας ράβδους ενώπιον του Κυρίου εν τη σκηνή του μαρτυρίου.
\par 8 Και την επαύριον εισήλθεν ο Μωϋσής εις την σκηνήν του μαρτυρίου· και ιδού, η ράβδος του Ααρών διά τον οίκον του Λευΐ εβλάστησε και ανεφύησε βλαστόν και εξήνθησεν άνθη και έδωκεν αμύγδαλα.
\par 9 Και έφερεν έξω ο Μωϋσής πάσας τας ράβδους απ' έμπροσθεν του Κυρίου προς πάντας τους υιούς Ισραήλ· και αυτοί είδον και έλαβον έκαστος την ράβδον αυτού.
\par 10 Και είπε Κύριος προς τον Μωϋσήν, Απόθες την ράβδον του Ααρών έμπροσθεν του μαρτυρίου, διά να φυλάττηται εις σημείον εις τους υιούς της αποστασίας· και θέλεις καταπαύσει απ' εμού τους γογγυσμούς αυτών, διά να μη αποθάνωσι.
\par 11 Και έκαμεν ο Μωϋσής καθώς προσέταξεν εις αυτόν ο Κύριος· ούτως έκαμε.
\par 12 Και είπον οι υιοί Ισραήλ προς τον Μωϋσήν λέγοντες, Ιδού, ημείς αποθνήσκομεν, αφανιζόμεθα, πάντες αφανιζόμεθα·
\par 13 πας ο πλησιάζων, ο πλησιάζων εις την σκηνήν του Κυρίου, αποθνήσκει πάντες θέλομεν εκλείψει αποθνήσκοντες;

\chapter{18}

\par 1 Και είπε Κύριος προς τον Ααρών, Συ και οι υιοί σου και ο οίκος του πατρός σου μετά σου θέλετε βαστάζει την ανομίαν του αγιαστηρίου· και συ και οι υιοί σου μετά σου θέλετε βαστάζει την ανομίαν της ιερατείας σας.
\par 2 Και έτι τους αδελφούς σου, την φυλήν του Λευΐ, την φυλήν του πατρός σου, φέρε μετά σου, διά να ήναι ηνωμένοι μετά σου και να σε υπηρετώσι· συ όμως και οι υιοί σου μετά σου θέλετε υπηρετεί έμπροσθεν της σκηνής του μαρτυρίου.
\par 3 Και θέλουσι φυλάττει τας φυλακάς σου και τας φυλακάς όλης της σκηνής· μόνον εις τα σκεύη του αγιαστηρίου και εις το θυσιαστήριον δεν θέλουσι πλησιάζει, διά να μη αποθάνωσι μήτε αυτοί μήτε σεις.
\par 4 Και θέλουσιν είσθαι ηνωμένοι μετά σου και θέλουσι φυλάττει τας φυλακάς της σκηνής του μαρτυρίου κατά πάσας τας υπηρεσίας της σκηνής· και ξένος δεν θέλει σας πλησιάζει.
\par 5 Και θέλετε φυλάττει τας φυλακάς του αγιαστηρίου και τας φυλακάς του θυσιαστηρίου, και δεν θέλει είσθαι πλέον οργή εις τους υιούς Ισραήλ.
\par 6 Και εγώ, ιδού, έλαβον τους αδελφούς σας τους Λευΐτας εκ μέσου των υιών Ισραήλ· ούτοι είναι δεδομένοι εις εσάς δώρον διά τον Κύριον, διά να εκτελώσι τας υπηρεσίας της σκηνής του μαρτυρίου.
\par 7 Συ δε και οι υιοί σου μετά σου θέλετε φυλάττει την ιερατείαν σας εις πάντα τα του θυσιαστηρίου και τα εντός του παραπετάσματος, και θέλετε κάμνει την υπηρεσίαν. Δώρον έδωκα την υπηρεσίαν της ιερατείας σας· και όστις ξένος πλησιάση, θέλει θανατόνεσθαι.
\par 8 Και είπε Κύριος προς τον Ααρών, Ιδού, εγώ έδωκα έτι εις σε την επιστασίαν των υψουμένων προσφορών μου από πάντων των ηγιασμένων παρά των υιών Ισραήλ· εις σε έδωκα αυτά διά το χρίσμα και εις τους υιούς σου εις νόμιμον αιώνιον.
\par 9 Τούτο θέλει είσθαι σου από των αγιωτάτων εκ των διά πυρός προσφερομένων· πάντα τα δώρα αυτών, πάσαι αι εξ αλφίτων προσφοραί αυτών και πάσαι αι περί αμαρτίας προσφοραί αυτών και πάσαι αι περί ανομίας προσφοραί αυτών, τας οποίας θέλουσιν αποδίδει εις εμέ, αγιώτατα θέλουσιν είσθαι διά σε και διά τους υιούς σου.
\par 10 Εν τω αγίω των αγίων θέλετε τρώγει αυτά· παν αρσενικόν θέλει τρώγει αυτά· άγια θέλουσιν είσθαι εις σε.
\par 11 Και τούτο είναι σου, η υψουμένη προσφορά εκ των δώρων αυτών, μετά πασών των κινητών προσφορών των υιών Ισραήλ· εις σε έδωκα αυτά και εις τους υιούς σου και εις τας θυγατέρας σου μετά σου εις νόμιμον αιώνιον· πας καθαρός εν τω οίκω σου θέλει τρώγει αυτά.
\par 12 Παν το εξαίρετον του ελαίου, και παν το εξαίρετον του οίνου και του σίτου, τας απαρχάς αυτών, όσα προσφέρωσιν εις τον Κύριον, εις σε έδωκα αυτά·
\par 13 Πάντα τα πρωτογεννήματα της γης, όσα φέρωσι προς τον Κύριον, σου θέλουσιν είσθαι πας καθαρός εν τω οίκω σου θέλει τρώγει αυτά.
\par 14 Παν καθιέρωμα του Ισραήλ θέλει είσθαι σου.
\par 15 Παν διανοίγον μήτραν εκ πάσης σαρκός, το οποίον προσφέρωσι προς τον Κύριον, από ανθρώπου έως κτήνους, σου θέλει είσθαι πλην τα πρωτότοκα των ανθρώπων θέλουσιν εξάπαντος εξαγοράζεσθαι και τα πρωτότοκα των κτηνών των ακαθάρτων θέλουσιν εξαγοράζεσθαι.
\par 16 Και όσα πρέπει να εξαγορασθώσιν από ηλικίας ενός μηνός, θέλουσιν εξαγοράζεσθαι κατά την εκτίμησίν σου, διά πέντε σίκλους αργυρίου, κατά τον σίκλον τον άγιον, όστις είναι είκοσι γερά.
\par 17 Τα πρωτότοκα όμως των βοών ή τα πρωτότοκα των προβάτων ή τα πρωτότοκα των αιγών δεν θέλουσιν εξαγοράζεσθαι· άγια είναι· το αίμα αυτών θέλεις ραντίζει επί το θυσιαστήριον, και το πάχος αυτών θέλεις καίει διά προσφοράν γινομένην διά πυρός εις οσμήν ευωδίας προς τον Κύριον.
\par 18 Και το κρέας αυτών θέλει είσθαι σου, καθώς το κινητόν στήθος και ο δεξιός ώμος είναι σου.
\par 19 Πάσας τας υψουμένας προσφοράς των αγίων πραγμάτων, τας οποίας οι υιοί Ισραήλ προσφέρωσιν εις τον Κύριον, έδωκα εις σε και εις τους υιούς σου και εις τας θυγατέρας σου μετά σου εις νόμιμον αιώνιον. Αύτη είναι διαθήκη άλατος παντοτεινή ενώπιον του Κυρίου εις σε και εις το σπέρμα σου μετά σου.
\par 20 Και είπε Κύριος προς τον Ααρών, Εν τη γη αυτών δεν θέλεις έχει κληρονομίαν ουδέ θέλεις έχει μερίδα μεταξύ αυτών· εγώ είμαι η μερίς σου και η κληρονομία σου εν μέσω των υιών Ισραήλ.
\par 21 Και ιδού, έδωκα εις τους υιούς Λευΐ πάντα τα δέκατα του Ισραήλ εις κληρονομίαν, διά την υπηρεσίαν αυτών την οποίαν υπηρετούσι, την υπηρεσίαν της σκηνής του μαρτυρίου·
\par 22 και δεν θέλουσι πλησιάζει του λοιπού οι υιοί Ισραήλ εις την σκηνήν του μαρτυρίου, διά να μη λάβωσιν εφ' εαυτούς αμαρτίαν και αποθάνωσιν·
\par 23 αλλ' οι Λευΐται, ούτοι θέλουσιν υπηρετεί την υπηρεσίαν της σκηνής του μαρτυρίου και ούτοι θέλουσι βαστάζει την ανομίαν αυτών· τούτο θέλει είσθαι νόμιμον αιώνιον εις τας γενεάς σας· και δεν θέλουσιν έχει μεταξύ των υιών Ισραήλ ουδεμίαν κληρονομίαν·
\par 24 διότι τα δέκατα των υιών Ισραήλ, τα οποία προσφέρωσιν υψουμένην προσφοράν προς τον Κύριον, έδωκα κληρονομίαν εις τους Λευΐτας· διά τούτο είπα περί αυτών, Εν μέσω των υιών Ισραήλ δεν θέλουσιν έχει ουδεμίαν κληρονομίαν.
\par 25 Και ελάλησε Κύριος προς τον Μωϋσήν λέγων,
\par 26 Λάλησον και προς τους Λευΐτας και ειπέ προς αυτούς, Όταν λαμβάνητε παρά των υιών Ισραήλ το δέκατον, το οποίον έδωκα εις εσάς παρ' αυτών διά κληρονομίαν σας, τότε θέλετε προσφέρει εξ αυτών προσφοράν υψουμένην εις τον Κύριον, δέκατον από του δεκάτου.
\par 27 Και αύται αι υψούμεναι προσφοραί σας θέλουσι λογίζεσθαι εις εσάς ως σίτος του αλωνίου και ως αφθονία του ληνού.
\par 28 Ούτω και σεις θέλετε προσφέρει προσφοράν υψουμένην εις τον Κύριον από πάντων των δεκάτων σας, τα οποία λαμβάνετε παρά των υιών Ισραήλ· και από τούτων θέλετε δίδει την υψουμένην προσφοράν του Κυρίου εις τον Ααρών τον ιερέα.
\par 29 Από πάντων των δώρων σας θέλετε προσφέρει πάσαν υψουμένην προσφοράν του Κυρίου, από παντός του εξαιρέτου αυτών το ηγιασμένον μέρος εξ αυτών.
\par 30 Και θέλεις ειπεί προς αυτούς, Όταν προσφέρητε εξ αυτών το εξαίρετον αυτών, τούτο θέλει λογίζεσθαι διά τους Λευΐτας ως προϊόν του αλωνίου και ως προϊόν του ληνού·
\par 31 και θέλετε τρώγει αυτά εν παντί τόπω, σεις και αι οικογένειαί σας· διότι τούτο είναι μισθός εις εσάς διά την υπηρεσίαν σας εν τη σκηνή του μαρτυρίου·
\par 32 και δεν θέλετε φέρει εφ' εαυτούς αμαρτίαν δι' αυτά, όταν προσφέρητε απ' αυτών το εξαίρετον αυτών· και δεν θέλετε βεβηλώσει τα άγια των υιών Ισραήλ, διά να μη αποθάνητε.

\chapter{19}

\par 1 Και ελάλησε Κύριος προς τον Μωϋσήν και προς τον Ααρών λέγων,
\par 2 Τούτο είναι το διάταγμα του νόμου, το οποίον ο Κύριος προσέταξε λέγων, Ειπέ προς τους υιούς Ισραήλ να φέρωσιν εις σε ξανθήν δάμαλιν άμωμον, μη έχουσαν ελάττωμα, επί της οποίας δεν επεβλήθη ζυγός·
\par 3 και θέλετε δώσει αυτήν εις τον Ελεάζαρ τον ιερέα και θέλει φέρει αυτήν έξω του στρατοπέδου· και θέλουσι σφάξει αυτήν ενώπιον αυτού.
\par 4 Και θέλει λάβει Ελεάζαρ ο ιερεύς από του αίματος αυτής διά του δακτύλου αυτού και θέλει ραντίσει επτάκις από του αίματος αυτής κατ' έμπροσθεν του προσώπου της σκηνής του μαρτυρίου.
\par 5 Και θέλουσι καύσει την δάμαλιν ενώπιον αυτού· το δέρμα αυτής και το κρέας αυτής και το αίμα αυτής μετά της κόπρου αυτής θέλουσι καή.
\par 6 Και ο ιερεύς θέλει λάβει ξύλον κέδρινον και ύσσωπον και κόκκινον και θέλει ρίψει αυτά εις το μέσον του κατακαύματος της δαμάλεως.
\par 7 Τότε θέλει πλύνει τα ιμάτια αυτού ο ιερεύς και θέλει λούσει το σώμα αυτού εν ύδατι και μετά ταύτα θέλει εισέλθει εις το στρατόπεδον και θέλει είσθαι ακάθαρτος ο ιερεύς έως εσπέρας.
\par 8 Και ο καίων αυτήν θέλει πλύνει τα ιμάτια αυτού εν ύδατι και θέλει λούσει το σώμα αυτού εν ύδατι και θέλει είσθαι ακάθαρτος έως εσπέρας.
\par 9 Και άνθρωπος καθαρός θέλει συνάξει την στάκτην της δαμάλεως και θέλει αποθέσει αυτήν έξω του στρατοπέδου εις τόπον καθαρόν· και θέλει φυλάττεσθαι διά την συναγωγήν των υιών Ισραήλ διά ύδωρ χωρισμού· τούτο είναι διά καθαρισμόν αμαρτίας.
\par 10 Και ο συνάξας την στάκτην της δαμάλεως θέλει πλύνει τα ιμάτια αυτού και θέλει είσθαι ακάθαρτος έως εσπέρας· και τούτο θέλει είσθαι εις τους υιούς Ισραήλ και εις τους ξένους τους παροικούντας μεταξύ υμών εις νόμιμον αιώνιον.
\par 11 Όστις εγγίση νεκρόν σώμα ανθρώπου, ούτος θέλει είσθαι ακάθαρτος επτά ημέρας.
\par 12 Ούτος θέλει αγνισθή διά τούτου την τρίτην ημέραν και την ημέραν την εβδόμην θέλει είσθαι καθαρός· εάν όμως δεν αγνισθή την τρίτην ημέραν, ουδέ την εβδόμην ημέραν θέλει είσθαι καθαρός.
\par 13 Όστις εγγίση νεκρόν σώμα οποιουδήποτε τεθνεώτος ανθρώπου και δεν αγνισθή, μιαίνει την σκηνήν του Κυρίου· και η ψυχή εκείνη θέλει εξολοθρευθή εκ του Ισραήλ· επειδή δεν ερραντίσθη επ' αυτόν το ύδωρ του χωρισμού, θέλει είσθαι ακάθαρτος· η ακαθαρσία αυτού μένει επ' αυτόν.
\par 14 Ούτος είναι ο νόμος όταν άνθρωπός τις αποθάνη εν σκηνή· Πάντες οι εισερχόμενοι εις την σκηνήν και πάντα τα εν τη σκηνή θέλουσιν είσθαι ακάθαρτα επτά ημέρας·
\par 15 και παν αγγείον ανοικτόν, μη έχον σκέπασμα δεδεμένον επάνωθεν αυτού, είναι ακάθαρτον.
\par 16 Και όστις εγγίση εν τη πεδιάδι πεφονευμένον τινά διά μαχαίρας ή νεκρόν σώμα ή οστούν ανθρώπου ή μνήμα, θέλει είσθαι ακάθαρτος επτά ημέρας.
\par 17 Και θέλουσι λάβει διά τον ακάθαρτον από της στάκτης της καυθείσης δαμάλεως διά καθαρισμόν της αμαρτίας, και θέλει χυθή επ' αυτήν ύδωρ ζων εις αγγείον.
\par 18 Και άνθρωπος καθαρός θέλει λάβει ύσσωπον, και εμβάψας εις το ύδωρ θέλει ραντίσει επί την σκηνήν και πάντα τα σκεύη και επί τους ανθρώπους τους ευρεθέντας εκεί και επ' εκείνον όστις ήγγισεν οστούν ή πεφονευμένον ή νεκρόν ή μνήμα.
\par 19 Και ο καθαρός θέλει ραντίσει επί τον ακάθαρτον την τρίτην ημέραν και την εβδόμην ημέραν· την δε εβδόμην ημέραν θέλει αγνίσει αυτόν· και αυτός θέλει πλύνει τα ιμάτια αυτού και θέλει λουσθή εν ύδατι· και το εσπέρας θέλει είσθαι καθαρός.
\par 20 Ο δε άνθρωπος όστις είναι ακάθαρτος και δεν αγνισθή, η ψυχή εκείνη θέλει εξολοθρευθή εκ μέσου της συναγωγής· διότι το αγιαστήριον του Κυρίου εμίανε· το ύδωρ του χωρισμού δεν ερραντίσθη επ' αυτόν· αυτός είναι ακάθαρτος.
\par 21 Και θέλει είσθαι εις αυτούς νόμιμον αιώνιον, ότι όστις ραντίση το ύδωρ του χωρισμού, θέλει πλύνει τα ιμάτια αυτού, και όστις εγγίση το ύδωρ του χωρισμού θέλει είσθαι ακάθαρτος έως εσπέρας.
\par 22 Και παν ό,τι εγγίση ο ακάθαρτος, τούτο θέλει είσθαι ακάθαρτον· και η ψυχή ήτις εγγίση αυτό, θέλει είσθαι ακάθαρτος έως εσπέρας.

\chapter{20}

\par 1 Και ήλθον οι υιοί Ισραήλ, πάσα συναγωγή, εις την έρημον Σιν, τον πρώτον μήνα· και έμεινεν ο λαός εν Κάδης· και απέθανεν εκεί η Μαριάμ και ετάφη εκεί.
\par 2 Και δεν ήτο ύδωρ διά την συναγωγήν· και συνηθροίσθησαν κατά του Μωϋσέως και κατά του Ααρών.
\par 3 Και ο λαός ελοιδόρει κατά του Μωϋσέως και είπον, λέγοντες, Είθε ν' απεθνήσκομεν, ότε οι αδελφοί ημών απέθανον ενώπιον του Κυρίου.
\par 4 Και διά τι ανεβιβάσατε την συναγωγήν του Κυρίου εις την έρημον ταύτην, διά να αποθάνωμεν εκεί ημείς και τα κτήνη ημών;
\par 5 και διά τι ανεβιβάσατε ημάς εκ της Αιγύπτου, διά να φέρητε ημάς εις τον κακόν τούτον τόπον; ούτος δεν είναι τόπος σποράς ή σύκων ή αμπέλων ή ροδίων· ουδέ ύδωρ υπάρχει διά να πίωμεν.
\par 6 Και ήλθον ο Μωϋσής και ο Ααρών απ' έμπροσθεν της συναγωγής εις την θύραν της σκηνής του μαρτυρίου και έπεσον κατά πρόσωπον αυτών· και εφάνη εις αυτούς η δόξα του Κυρίου.
\par 7 Και ελάλησε Κύριος προς τον Μωϋσήν, λέγων,
\par 8 Λάβε την ράβδον και συγκάλεσον την συναγωγήν συ και Ααρών ο αδελφός σου, και λαλήσατε προς την πέτραν ενώπιον αυτών· και θέλει δώσει τα ύδατα αυτής, και θέλεις εκβάλει εις αυτούς ύδωρ εκ της πέτρας· και θέλεις ποτίσει την συναγωγήν και τα κτήνη αυτών.
\par 9 Και έλαβεν ο Μωϋσής την ράβδον απ' έμπροσθεν του Κυρίου, καθώς προσέταξεν εις αυτόν·
\par 10 και συνεκάλεσαν Μωϋσής και ο Ααρών την συναγωγήν έμπροσθεν της πέτρας· και είπε προς αυτούς, Ακούσατε τώρα, σεις οι απειθείς· να σας εκβάλωμεν ύδωρ εκ της πέτρας ταύτης;
\par 11 Και υψώσας ο Μωϋσής την χείρα αυτού εκτύπησε με την ράβδον αυτού την πέτραν δίς· και εξήλθον ύδατα πολλά· και έπιεν η συναγωγή και τα κτήνη αυτών.
\par 12 Και είπε Κύριος προς τον Μωϋσήν και προς τον Ααρών, Επειδή δεν με επιστεύσατε, διά να με αγιάσητε έμπροσθεν των υιών Ισραήλ, διά τούτο σεις δεν θέλετε φέρει την συναγωγήν ταύτην εις την γην, την οποίαν έδωκα εις αυτούς.
\par 13 τούτο είναι το ύδωρ Μεριβά· διότι οι υιοί Ισραήλ ελοιδόρησαν κατά του Κυρίου, και αυτός ηγιάσθη εν αυτοίς.
\par 14 Και απέστειλε Μωϋσής πρέσβεις από Κάδης προς τον βασιλέα του Εδώμ, λέγων, Ταύτα λέγει ο αδελφός σου Ισραήλ· συ εξεύρεις πάσαν την ταλαιπωρίαν ήτις μας εύρηκεν·
\par 15 ότι κατέβησαν οι πατέρες ημών εις την Αίγυπτον και κατωκήσαμεν πολύν καιρόν εν Αιγύπτω· και οι Αιγύπτιοι κατεδυνάστευσαν ημάς και τους πατέρας ημών·
\par 16 και ανεβοήσαμεν προς τον Κύριον και αυτός εισήκουσε της φωνής ημών και απέστειλεν άγγελον και εξήγαγεν ημάς εκ της Αιγύπτου· και ιδού, είμεθα εν Κάδης, πόλει εις τα άκρα των ορίων σου·
\par 17 ας περάσωμεν, παρακαλώ, διά της γης σου· δεν θέλομεν περάσει διά των αγρών ή διά των αμπελώνων, ουδέ θέλομεν πίει ύδωρ εκ των φρεάτων· θέλομεν περάσει διά της βασιλικής οδού· δεν θέλομεν εκκλίνει δεξιά ή αριστερά, εωσού περάσωμεν τα όριά σου.
\par 18 Και είπε προς αυτόν ο Εδώμ, Δεν θέλεις περάσει διά της γης μου· ει δε μη, θέλω εξέλθει εν μαχαίρα εις συνάντησίν σου.
\par 19 Και οι υιοί Ισραήλ είπον προς αυτόν, Ημείς διαβαίνομεν διά της λεωφόρου· και εάν εγώ και τα κτήνη μου πίωμεν εκ του ύδατός σου, θέλω πληρώσει αυτό· θέλω διαβή μόνον επί ποδός, ουδέν άλλο.
\par 20 Ο δε είπε, Δεν θέλεις διαβή. Και εξήλθεν ο Εδώμ εναντίον αυτού μετά πολλού λαού και εν χειρί ισχυρά.
\par 21 Ούτως ηρνήθη ο Εδώμ να δώση διάβασιν εις τον Ισραήλ διά των ορίων αυτού· και εξέκλινεν ο Ισραήλ απ' αυτού.
\par 22 Και εσηκώθησαν οι υιοί Ισραήλ, πάσα η συναγωγή, από Κάδης και ήλθον εις το όρος Ωρ.
\par 23 Και ελάλησε Κύριος προς τον Μωϋσήν και προς τον Ααρών εν τω όρει Ωρ πλησίον των ορίων της γης Εδώμ, λέγων,
\par 24 Ο Ααρών θέλει προστεθή εις τον λαόν αυτού· διότι δεν θέλει εισέλθει εις την γην, την οποίαν έδωκα εις τους υιούς Ισραήλ· επειδή ηπειθήσατε εις τον λόγον μου εις το ύδωρ Μεριβά·
\par 25 λάβε τον Ααρών και Ελεάζαρ τον υιόν αυτού και αναβίβασον αυτούς εις το όρος Ωρ·
\par 26 και έκδυσον τον Ααρών την στολήν αυτού και ένδυσον αυτήν Ελεάζαρ τον υιόν αυτού· και ο Ααρών θέλει προστεθή εις τον λαόν αυτού και θέλει αποθάνει εκεί.
\par 27 Και έκαμεν ο Μωϋσής καθώς προσέταξεν ο Κύριος· και ανέβησαν εις το όρος Ωρ έμπροσθεν πάσης της συναγωγής.
\par 28 Και εξέδυσεν ο Μωϋσής τον Ααρών την στολήν αυτού και ενέδυσεν αυτήν Ελεάζαρ τον υιόν αυτού· και απέθανεν ο Ααρών εκεί επί της κορυφής του όρους· και κατέβησαν Μωϋσής και Ελεάζαρ από του όρους.
\par 29 Και είδε πάσα η συναγωγή ότι ετελεύτησεν ο Ααρών· και επένθησαν τον Ααρών τριάκοντα ημέρας πας ο οίκος Ισραήλ.

\chapter{21}

\par 1 Και ήκουσεν ο Χαναναίος ο βασιλεύς της Αράδ, ο κατοικών προς μεσημβρίαν, ότι ήλθεν ο Ισραήλ διά της οδού Αθαρείμ, και επολέμησεν εναντίον του Ισραήλ και συνέλαβεν εξ αυτών αιχμαλώτους.
\par 2 Και ηυχήθη ο Ισραήλ ευχήν προς τον Κύριον και είπεν, Εάν τωόντι παραδώσης τον λαόν τούτον εις την χείρα μου, θέλω καταστρέψει τας πόλεις αυτών.
\par 3 Και εισήκουσεν ο Κύριος της φωνής του Ισραήλ και παρέδωκε τους Χαναναίους· και κατέστρεψαν αυτούς και τας πόλεις αυτών· και εκάλεσαν το όνομα του τόπου Ορμά.
\par 4 Και εσηκώθησαν από του όρους Ωρ διά της οδού της Ερυθράς θαλάσσης, διά να περιέλθωσι την γην Εδώμ· και ωλιγοψύχησεν ο λαός εν τη οδώ.
\par 5 Και ελάλησεν ο λαός κατά του Θεού και κατά του Μωϋσέως, λέγοντες, Διά τι ανεβίβασας ημάς εξ Αιγύπτου διά ν' αποθάνωμεν εν τη ερήμω; διότι άρτος δεν είναι και ύδωρ δεν είναι και η ψυχή ημών αηδίασε τον άρτον τούτον, τον ελαφρόν.
\par 6 Και απέστειλεν ο Κύριος επί τον λαόν τους όφεις τους φλογερούς και εδάγκανον τον λαόν, και λαός πολύς εκ του Ισραήλ απέθανε.
\par 7 Και ελθών ο λαός προς τον Μωϋσήν είπον, Ημαρτήσαμεν, διότι ελαλήσαμεν κατά του Κυρίου και κατά σού· δεήθητι του Κυρίου να σηκώση τους όφεις αφ' ημών. Και εδεήθη ο Μωϋσής υπέρ του λαού.
\par 8 Και είπε Κύριος προς τον Μωϋσήν, Κάμε εις σεαυτόν όφιν φλογερόν και βάλε αυτόν επί ξύλου· και πας όστις δαγκασθή και εμβλέψη εις αυτόν, θέλει ζήσει.
\par 9 Και έκαμεν ο Μωϋσής όφιν χαλκούν και έβαλεν αυτόν επί ξύλου· και εάν όφις εδάγκανε τινά, εμβλέπων ούτος εις τον όφιν τον χαλκούν, έζη.
\par 10 Και εσηκώθησαν οι υιοί Ισραήλ και εστρατοπέδευσαν εν Ωβώθ.
\par 11 Και σηκωθέντες από Ωβώθ εστρατοπέδευσαν εις Ιϊέ-αβαρίμ, εν τη ερήμω τη κατά πρόσωπον του Μωάβ, προς ανατολάς ηλίου.
\par 12 Εκείθεν σηκωθέντες εστρατοπέδευσαν εν τη κοιλάδι Ζαρέδ.
\par 13 Εκείθεν σηκωθέντες εστρατοπέδευσαν εις το πέραν του Αρνών, όστις είναι εν τη ερήμω και εξέρχεται εκ των ορίων των Αμορραίων· διότι ο Αρνών είναι το όριον του Μωάβ, μεταξύ Μωάβ και Αμορραίων.
\par 14 Διά τούτο λέγεται εν τω βιβλίω των πολέμων του Κυρίου, Κατά τον Βαέβ εν Σουφά, και επί των ρυάκων του Αρνών,
\par 15 και επί του ρεύματος των ρυάκων, το οποίον καταβαίνει εις την πόλιν Αρ και κείται εις τα όρια του Μωάβ.
\par 16 Και εκείθεν ήλθον εις Βήρ· τούτο είναι το φρέαρ, περί του οποίου είπε Κύριος προς τον Μωϋσήν, Σύναξον τον λαόν, και θέλω δώσει ύδωρ εις αυτούς.
\par 17 Τότε έψαλεν ο Ισραήλ την ωδήν ταύτην· Ανάβα, ω φρέαρ· ψάλλετε εις αυτό·
\par 18 οι άρχοντες έσκαψαν το φρέαρ, οι ευγενείς του λαού έσκαψαν, διά προσταγής του νομοθέτου, με τας ράβδους αυτών. Και από της ερήμου ήλθον εις Ματτανά,
\par 19 και από Ματτανά εις Νααλιήλ, και από Νααλιήλ εις Βαμώθ,
\par 20 και από Βαμώθ της κοιλάδος της εν τη γη Μωάβ, επί της κορυφής Φασγά, το οποίον βλέπει προς Γεσιμών.
\par 21 Και απέστειλεν ο Ισραήλ πρέσβεις προς τον Σηών βασιλέα των Αμορραίων λέγων,
\par 22 Ας περάσωμεν διά της γης σου· δεν θέλομεν κλίνει εις τους αγρούς ούτε εις τους αμπελώνας· δεν θέλομεν πίει ύδωρ εκ των φρεάτων· αλλά διά της βασιλικής οδού θέλομεν πορευθή, εωσού περάσωμεν τα όριά σου.
\par 23 Και ο Σηών δεν αφήκε τον Ισραήλ να περάση διά των ορίων αυτού· αλλ' ο Σηών συνήγαγε πάντα τον λαόν αυτού και εξήλθε να παραταχθή εναντίον του Ισραήλ εις την έρημον· και ήλθεν εις Ιασσά και επολέμησεν εναντίον του Ισραήλ.
\par 24 Και επάταξεν ο Ισραήλ αυτόν εν στόματι μαχαίρας και κατεκυρίευσε την γην αυτού από Αρνών έως Ιαβόκ, μέχρι των υιών Αμμών· επειδή τα όρια των υιών Αμμών ήσαν οχυρά.
\par 25 Και εκυρίευσεν ο Ισραήλ πάσας τας πόλεις ταύτας· και κατώκησεν ο Ισραήλ εις πάσας τας πόλεις των Αμορραίων, εις Εσεβών και εις πάσας τας κώμας αυτής·
\par 26 επειδή η Εσεβών ήτο η πόλις του Σηών βασιλέως των Αμορραίων, όστις είχε πολεμήσει πρότερον τον βασιλέα του Μωάβ και έλαβε πάσαν την γην αυτού από της χειρός αυτού, έως Αρνών.
\par 27 Διά τούτο λέγουσιν οι παροιμιασταί, Έλθετε εις Εσεβών· Ας κτισθή και ας κατασκευασθή η πόλις του Σηών·
\par 28 διότι πυρ εξήλθεν από Εσεβών, φλόξ από της πόλεως του Σηών· κατέφαγε την Αρ του Μωάβ, και τους άρχοντας των υψηλών τόπων του Αρνών·
\par 29 ουαί εις σε, Μωάβ Απωλέσθης, λαέ του Χεμώς· έδωκε τους διασωθέντας υιούς αυτού, και τας θυγατέρας αυτού αιχμαλώτους εις τον Σηών βασιλέα των Αμορραίων·
\par 30 Ημείς κατετοξεύσαμεν αυτούς· η Εσεβών ηφανίσθη έως Δαιβών, και κατηρημώσαμεν αυτούς έως Νοφά, το οποίον εκτείνεται έως Μεδεβά.
\par 31 Και κατώκησεν ο Ισραήλ εν τη γη των Αμορραίων.
\par 32 Και απέστειλεν ο Μωϋσής να κατασκοπεύσωσι την Ιαζήρ· και εκυρίευσαν τας κώμας αυτής και εξεδίωξαν τους Αμορραίους τους κατοικούντας εκεί.
\par 33 Και στρέψαντες ανέβησαν την οδόν την εις Βασάν· και εξήλθεν ο Ωγ βασιλεύς της Βασάν εις συνάντησιν αυτών, αυτός και πας ο λαός αυτού, προς μάχην εις Εδρεΐ.
\par 34 Και είπε Κύριος προς τον Μωϋσήν, Μη φοβηθής αυτόν· διότι εις τας χείρας σου παρέδωκα αυτόν και πάντα τον λαόν αυτού και την γην αυτού· και θέλεις κάμει εις αυτόν, ως έκαμες εις τον Σηών βασιλέα των Αμορραίων τον κατοικούντα εν Εσεβών.
\par 35 Και επάταξαν αυτόν και τους υιούς αυτού, και πάντα τον λαόν αυτού, εωσού δεν εναπελείφθη εις αυτόν ουδέν· και κατεκυρίευσαν την γην αυτού.

\chapter{22}

\par 1 Και σηκωθέντες οι υιοί Ισραήλ εστρατοπέδευσαν εις τας πεδιάδας του Μωάβ, παρά τον Ιορδάνην, κατέναντι της Ιεριχώ.
\par 2 Και Βαλάκ ο υιός του Σεπφώρ είδε πάντα όσα έκαμεν ο Ισραήλ εις τους Αμορραίους.
\par 3 Και εφοβήθη ο Μωάβ τον λαόν σφόδρα, διότι ήσαν πολλοί· και ήτο ο Μωάβ εις αμηχανίαν εξ αιτίας των υιών Ισραήλ.
\par 4 Και είπεν ο Μωάβ προς τους πρεσβυτέρους του Μαδιάμ, Τώρα θέλει καταφάγει το πλήθος τούτο πάντα τα πέριξ ημών, καθώς ο βους κατατρώγει τον χόρτον της πεδιάδος. Και Βαλάκ ο υιός του Σεπφώρ ήτο βασιλεύς των Μωαβιτών κατ' εκείνον τον καιρόν.
\par 5 Και απέστειλε πρέσβεις προς τον Βαλαάμ, υιόν του Βεώρ, εις Φεθορά, κειμένην πλησίον του ποταμού της γης των υιών του λαού αυτού, διά να προσκαλέση αυτόν, λέγων, Ιδού, λαός εξήλθεν εξ Αιγύπτου· ιδού, περικαλύπτει το πρόσωπον της γης, και κάθηται εναντίον μου·
\par 6 τώρα λοιπόν ελθέ, σε παρακαλώ, καταράσθητί μοι τον λαόν τούτον, διότι είναι δυνατώτερός μου· ίσως υπερισχύσω, να πατάξωμεν αυτούς και να εκδιώξω αυτούς εκ της γής· επειδή εξεύρω, ότι όντινα ευλογήσης είναι ευλογημένος, και όντινα καταρασθής είναι κατηραμένος.
\par 7 Και υπήγαν οι πρεσβύτεροι του Μωάβ και οι πρεσβύτεροι του Μαδιάμ, φέροντες τα δώρα της μαντείας εις τας χείρας αυτών· και ήλθον προς Βαλαάμ και είπον προς αυτόν τους λόγους του Βαλάκ.
\par 8 Ο δε είπε προς αυτούς, Μείνατε ενταύθα ταύτην την νύκτα και θέλω αποκριθή εις εσάς ό,τι λαλήση ο Κύριος προς εμέ. Και έμειναν μετά του Βαλαάμ οι άρχοντες του Μωάβ.
\par 9 Και ήλθεν ο Θεός εις τον Βαλαάμ και είπε, Τι θέλουσιν οι άνθρωποι ούτοι μετά σου;
\par 10 Και είπεν ο Βαλαάμ προς τον Θεόν, Βαλάκ ο υιός του Σεπφώρ, βασιλεύς του Μωάβ, απέστειλε προς εμέ λέγων,
\par 11 Ιδού, λαός εξήλθεν εξ Αιγύπτου και κατεκάλυψε το πρόσωπον της γής· ελθέ τώρα, καταράσθητί μοι αυτόν· ίσως υπερισχύσω να νικήσω αυτόν και να εκδιώξω αυτόν.
\par 12 Και είπεν ο Θεός προς τον Βαλαάμ, Μη υπάγης μετ' αυτών· μη καταρασθής τον λαόν, διότι είναι ευλογημένος.
\par 13 Και σηκωθείς την αυγήν ο Βαλαάμ είπε προς τους άρχοντας του Βαλάκ, Υπάγετε εις την γην σας· διότι δεν μοι συγχωρεί ο Κύριος να έλθω μεθ' υμών.
\par 14 Και σηκωθέντες οι άρχοντες του Μωάβ, ήλθον προς τον Βαλάκ και είπον, Δεν θέλει ο Βαλαάμ να έλθη μεθ' ημών.
\par 15 Και ο Βαλάκ απέστειλε πάλιν άρχοντας περισσοτέρους και εντιμοτέρους τούτων·
\par 16 και ήλθον προς τον Βαλαάμ και είπον προς αυτόν, ούτω λέγει Βαλάκ ο υιός του Σεπφώρ· Μη εμποδισθής, σε παρακαλώ, να έλθης προς εμέ·
\par 17 διότι θέλω σε τιμήσει με μεγάλας τιμάς, και θέλω κάμει παν ό,τι μοι είπης· ελθέ λοιπόν, παρακαλώ, καταράσθητί μοι τον λαόν τούτον.
\par 18 Και απεκρίθη ο Βαλαάμ και είπε προς τους δούλους του Βαλάκ, Και εάν μοι δώση ο Βαλάκ την οικίαν αυτού πλήρη αργυρίου και χρυσίου, δεν δύναμαι να παραβώ τον λόγον Κυρίου του Θεού μου, διά να κάμω ολιγώτερον ή περισσότερον·
\par 19 διά τούτο μείνατε ενταύθα, παρακαλώ, και σεις την νύκτα ταύτην, διά να ίδω τι θέλει ειπεί ότι ο Κύριος προς εμέ.
\par 20 Και ήλθεν ο Θεός προς τον Βαλαάμ την νύκτα και είπε προς αυτόν, Εάν έλθωσιν οι άνθρωποι διά να σε καλέσωσι, σηκωθείς ύπαγε μετ' αυτών· πλην ό,τι σοι είπω, τούτο θέλεις κάμει.
\par 21 Και εσηκώθη ο Βαλαάμ το πρωΐ, και εσαμάρωσε την όνον αυτού και υπήγε μετά των αρχόντων του Μωάβ.
\par 22 Και εξήφθη η οργή του Θεού ότι υπήγε· και εστάθη άγγελος Κυρίου εν τη οδώ έμπροσθεν αυτού, διά να εναντιωθή εις αυτόν· αυτός δε εκάθητο επί της όνου αυτού και δύο δούλοι αυτού ήσαν μετ' αυτού·
\par 23 και ιδούσα η όνος τον άγγελον του Κυρίου ιστάμενον εν τη οδώ, και την ρομφαίαν αυτού γεγυμνωμένην εν τη χειρί αυτού, εξέκλινεν η όνος εκ της οδού και υπήγαινεν προς την πεδιάδα· και εκτύπησεν ο Βαλαάμ την όνον, διά να επαναφέρη αυτήν εις την οδόν.
\par 24 Αλλ' ο άγγελος του Κυρίου εστάθη εν μιά στενή οδώ των αμπελώνων, όπου ήτο φραγμός εντεύθεν και φραγμός εντεύθεν·
\par 25 και ιδούσα η όνος τον άγγελον του Κυρίου, προσέθλιψεν εαυτήν προς τον τοίχον και συνέθλιψε τον πόδα του Βαλαάμ εις τον τοίχον· αυτός δε εκτύπησεν αυτήν πάλιν.
\par 26 Και ο άγγελος του Κυρίου υπήγε παρεμπρός, και εστάθη εν στενώ τόπω, όπου δεν ήτο οδός διά να εκκλίνη δεξιά ή αριστερά·
\par 27 και ιδούσα η όνος τον άγγελον του Κυρίου, συνεκάθησεν υποκάτω του Βαλαάμ· και θυμωθείς ο Βαλαάμ, εκτύπησε την όνον διά της ράβδου.
\par 28 Και ήνοιξεν ο Κύριος το στόμα της όνου· και είπε προς τον Βαλαάμ, Τι σοι έκαμα, και με εκτύπησας τρίτην ταύτην φοράν;
\par 29 Και είπεν ο Βαλαάμ προς την όνον, Διότι με ενέπαιξας· είθε να είχον μάχαιραν εν τη χειρί μου, διότι τώρα ήθελον σε θανατώσει.
\par 30 Και η όνος είπε προς τον Βαλαάμ, Δεν είμαι εγώ η όνος σου, επί της οποίας εκάθιζες αφ' ου χρόνου με έχεις έως της ημέρας ταύτης; ήμην πότε συνειθισμένη να κάμνω ούτως εις σε; Ο δε είπεν, Ουχί.
\par 31 Και ήνοιξεν ο Κύριος τους οφθαλμούς του Βαλαάμ, και είδε τον άγγελον του Κυρίου ιστάμενον εν τη οδώ και την ρομφαίαν αυτού γεγυμνωμένην εν τη χειρί αυτού· και κύψας προσεκύνησεν επί πρόσωπον αυτού.
\par 32 Και είπε προς αυτόν ο άγγελος του Κυρίου, Διά τι εκτύπησας την όνον σου τρίτην ταύτην φοράν; ιδού, εγώ εξήλθον διά να σοι εναντιωθώ, διότι ο δρόμος σου είναι διεστραμμένος ενώπιόν μου·
\par 33 και ιδούσά με η όνος εξέκλινεν απ' εμού τρίτην ταύτην φοράν· άλλως, εάν δεν εξέκλινεν απ' εμού τώρα σε μεν ήθελον φονεύσει, εκείνην δε ήθελον αφήσει ζώσαν.
\par 34 Και είπεν ο Βαλαάμ προς τον άγγελον του Κυρίου, Ημάρτησα· διότι δεν ήξευρον ότι συ έστεκες εν τη οδώ εναντίον μου· όθεν τώρα, εάν δεν ήναι αρεστόν εις σε, επιστρέφω.
\par 35 Και είπεν ο άγγελος του Κυρίου προς τον Βαλαάμ, Ύπαγε μετά των ανθρώπων· πλην ό,τι σοι είπω, τούτο θέλεις λαλήσει. Και υπήγεν ο Βαλαάμ μετά των αρχόντων του Βαλάκ.
\par 36 Και ακούσας ο Βαλάκ ότι ήρχετο ο Βαλαάμ, εξήλθε να προϋπαντήση αυτόν, έως εις πόλιν τινά του Μωάβ, κειμένην εν τοις ορίοις του Αρνών, όστις είναι το έσχατον όριον.
\par 37 Και είπεν ο Βαλάκ προς τον Βαλαάμ, Δεν απέστειλα προς σε μετά σπουδής να σε καλέσω; διά τι δεν ήλθες προς εμέ; μήπως δεν είμαι ικανός να σε τιμήσω;
\par 38 Και είπεν ο Βαλαάμ προς τον Βαλάκ, Ιδού, ήλθον προς σέ· έχω τώρα την δύναμιν να λαλήσω τι; όντινα λόγον βάλη ο Θεός εις το στόμα μου, τούτον θέλω λαλήσει.
\par 39 Και υπήγεν ο Βαλαάμ μετά του Βαλάκ, και ήλθον εις Κιριάθ-ουζώθ.
\par 40 Και εθυσίασεν ο Βαλάκ βόας και πρόβατα, και έπεμψεν εξ αυτών προς τον Βαλαάμ και προς τους άρχοντας τους μετ' αυτού.
\par 41 Και το πρωΐ έλαβεν ο Βαλάκ τον Βαλαάμ, και ανεβίβασεν αυτόν επί τους υψηλούς τόπους του Βάαλ, και εκείθεν είδε την άκραν του λαού.

\chapter{23}

\par 1 Και είπεν ο Βαλαάμ προς τον Βαλάκ, Οικοδόμησόν μοι ενταύθα επτά βωμούς και ετοίμασόν μοι ενταύθα επτά μόσχους και επτά κριούς.
\par 2 Και έκαμεν ο Βαλάκ καθώς είπεν ο Βαλαάμ· και προσέφεραν ο Βαλάκ και ο Βαλαάμ μόσχον και κριόν εφ' έκαστον βωμόν.
\par 3 Και είπεν ο Βαλαάμ προς τον Βαλάκ, Στήθι πλησίον του ολοκαυτώματός σου και εγώ θέλω υπάγει ίσως φανή ο Κύριος εις συνάντησίν μου· και ό,τι δείξη εις εμέ, τούτο θέλω σοι αναγγείλει. Και υπήγεν εις τόπον υψηλόν.
\par 4 Και συνήντησεν ο Θεός τον Βαλαάμ· και είπε προς αυτόν, Ητοίμασα τους επτά βωμούς, και προσέφερα μόσχον και κριόν εφ' έκαστον βωμόν.
\par 5 Και έβαλεν ο Κύριος λόγον εις το στόμα του Βαλαάμ και είπεν, Επίστρεψον προς τον Βαλάκ, και ούτω θέλεις ειπεί.
\par 6 Και επέστρεψε προς αυτόν, και ιδού, ίστατο πλησίον του ολοκαυτώματος αυτού, αυτός και πάντες οι άρχοντες του Μωάβ.
\par 7 Και ήρχισε την παραβολήν αυτού και είπε, Βαλάκ με έφερεν εκ της Αράμ, ο βασιλεύς του Μωάβ εκ των ορέων των προς ανατολάς, λέγων, Ελθέ, καταράσθητί μοι τον Ιακώβ· και ελθέ, αναθεμάτισον τον Ισραήλ.
\par 8 Πως να καταρασθώ τον οποίον ο Θεός δεν καταράται; ή πως να αναθεματίσω τον οποίον ο Κύριος δεν ανεθεμάτισε;
\par 9 Διότι από της κορυφής των ορέων βλέπω αυτόν, και από των λόφων θεωρώ αυτόν· ιδού, λαός, όστις θέλει κατοικήσει μόνος, και δεν θέλει λογαριασθή μεταξύ των εθνών·
\par 10 τις δύναται να αριθμήση την άμμον του Ιακώβ, και τον αριθμόν του τετάρτου του Ισραήλ; είθε να αποθάνω κατά τον θάνατον των δικαίων, και το τέλος μου να ήναι όμοιον με το τέλος αυτού.
\par 11 Και είπεν ο Βαλάκ προς τον Βαλαάμ, Τι μοι έκαμες; διά να καταρασθής τους εχθρούς μου σε παρέλαβον· και ιδού, συ ευλογών ευλόγησας αυτούς.
\par 12 Ο δε αποκριθείς είπε, Δεν πρέπει να προσέξω ό,τι ο Κύριος έβαλεν εις το στόμα μου, τούτο να είπω;
\par 13 Και είπε προς αυτόν ο Βαλάκ, Ελθέ, παρακαλώ, μετ' εμού εις άλλον τόπον, όθεν θέλεις ιδεί αυτόν· μόνον την άκραν αυτού θέλεις ιδεί, το δε όλον αυτού δεν θέλεις ιδεί· και καταράσθητί μοι αυτόν εκείθεν.
\par 14 Και έφερεν αυτόν εις την πεδιάδα Ζοφίμ επί την κορυφήν του Φασγά, και ωκοδόμησεν επτά βωμούς και προσέφερε μόσχον και κριόν εφ' έκαστον βωμόν.
\par 15 Και είπε προς τον Βαλάκ, Στήθι αυτού πλησίον του ολοκαυτώματός σου και εγώ θέλω συναντήσει εκεί τον Κύριον.
\par 16 Και συνήντησεν ο Κύριος τον Βαλαάμ, και έβαλε λόγον εις το στόμα αυτού και είπεν, Επίστρεψον προς τον Βαλάκ και ειπέ ούτω.
\par 17 Και ήλθε προς αυτόν· και ιδού, αυτός ίστατο πλησίον του ολοκαυτώματος αυτού και οι άρχοντες του Μωάβ μετ' αυτού. Και είπε προς αυτόν ο Βαλάκ, Τι ελάλησεν ο Κύριος;
\par 18 Και αρχίσας την παραβολήν αυτού είπε, Σηκώθητι, Βαλάκ, και άκουσον· δος ακρόασιν εις εμέ, συ ο υιός του Σεπφώρ·
\par 19 ο Θεός δεν είναι άνθρωπος να ψευσθή, ούτε υιός ανθρώπου να μεταμεληθή· αυτός είπε και δεν θέλει εκτελέσει; ή ελάλησε και δεν θέλει εμμείνει;
\par 20 Ιδού, ευλογίαν παρέλαβον· και ευλόγησε· και εγώ δεν δύναμαι να μεταστρέψω αυτήν.
\par 21 Δεν εθεώρησεν ανομίαν εις τον Ιακώβ, ουδέ είδε διαστροφήν εις τον Ισραήλ· Κύριος ο Θεός αυτού είναι μετ' αυτού, και αλαλαγμός βασιλέως είναι μεταξύ αυτών.
\par 22 Ο Θεός εξήγαγεν αυτούς εξ Αιγύπτου· έχουσιν ως δύναμιν μονοκέρωτος.
\par 23 Βεβαίως ουδεμία γοητεία δεν ισχύει κατά του Ιακώβ, ουδέ μαντεία κατά του Ισραήλ· κατά καιρόν θέλει λαληθή περί του Ιακώβ και περί του Ισραήλ, Τι κατώρθωσεν ο Θεός.
\par 24 Ιδού, ο λαός θέλει σηκωθή ως λέων, και θέλει εγερθή ως σκύμνος· δεν θέλει κοιμηθή εωσού φάγη το θήραμα, και πίη το αίμα των πεφονευμένων.
\par 25 Και είπεν ο Βαλάκ προς τον Βαλαάμ, Μήτε να καταρασθής αυτούς διόλου μήτε να ευλογήσης αυτούς διόλου.
\par 26 Αποκριθείς δε ο Βαλαάμ είπε προς τον Βαλάκ, δεν σε ελάλησα, λέγων, Παν ό,τι μοι είπη ο Κύριος, τούτο πρέπει να κάμω;
\par 27 Και είπεν ο Βαλάκ προς τον Βαλαάμ, Ελθέ, παρακαλώ, θέλω σε φέρει εις άλλον τόπον, ίσως θέλει αρέσει εις τον Θεόν να μοι καταρασθής αυτόν εκείθεν.
\par 28 Και έφερεν ο Βαλάκ τον Βαλαάμ επί την κορυφήν του Φεγώρ, το οποίον βλέπει προς Γεσιμών.
\par 29 Και είπεν ο Βαλαάμ προς τον Βαλάκ, οικοδόμησόν μοι ενταύθα επτά βωμούς και ετοίμασόν μοι ενταύθα επτά μόσχους και επτά κριούς.
\par 30 και έκαμεν ο Βαλάκ ως είπεν ο Βαλαάμ και προσέφερε μόσχον και κριόν εφ' έκαστον βωμόν.

\chapter{24}

\par 1 Και ιδών ο Βαλαάμ ότι ήτο αρεστόν ενώπιον του Κυρίου να ευλογήση τον Ισραήλ, δεν υπήγε, καθώς άλλοτε, να ζητήση μαντείας, αλλ' έστησε το πρόσωπον αυτού προς την έρημον.
\par 2 Και ανύψωσεν ο Βαλαάμ τους οφθαλμούς αυτού και είδε τον Ισραήλ κατεσκηνωμένον κατά τας φυλάς αυτών· και ήλθεν επ' αυτόν το πνεύμα του Θεού.
\par 3 Και αρχίσας την παραβολήν αυτού είπε, Βαλαάμ ο υιός του Βεώρ είπε, και ο άνθρωπος, ο έχων ανοικτούς τους οφθαλμούς αυτού, είπεν·
\par 4 είπεν εκείνος, όστις ήκουσε τα λόγια του Θεού, Όστις είδεν όρασιν του Παντοδυνάμου, πεσών εις έκστασιν, έχων όμως ανοικτούς τους οφθαλμούς αυτού.
\par 5 Πόσον ώραίαι είναι αι κατοικίαι σου, Ιακώβ, αι σκηναί σου, Ισραήλ.
\par 6 Ως κοιλάδες είναι εξηπλωμέναι, ως παράδεισοι εις όχθας ποταμού, ως δένδρα αλόης τα οποία εφύτευσεν ο Κύριος, ως κέδροι πλησίον των υδάτων.
\par 7 Θέλει εκχέει ύδωρ εκ της αντλίας αυτού, και το σπέρμα αυτού θέλει είσθαι εις ύδατα πολλά, και ο βασιλεύς αυτού θέλει είσθαι υψηλότερος του Αγάγ, και η βασιλεία αυτού θέλει μεγαλυνθή.
\par 8 Ο Θεός εξήγαγεν αυτόν εξ Αιγύπτου· έχει ως δύναμιν μονοκέρωτος· θέλει καταφάγει τα έθνη τους πολεμίους αυτού, και θέλει συντρίψει τα οστά αυτών, και θέλει κατατοξεύσει αυτούς με τα βέλη αυτού.
\par 9 Αναπεσών, εκοιμήθη ως λέων, και ως σκύμνος λέοντος· τις θέλει εξεγείρει αυτόν; Ευλογημένος ο ευλογών σε και κατηραμένος ο καταρώμενός σε.
\par 10 Και εξήφθη ο θυμός του Βαλάκ εναντίον του Βαλαάμ και συνεκρότησε τας χείρας αυτού· και είπεν ο Βαλάκ προς τον Βαλαάμ, διά να καταρασθής τους εχθρούς μου σε εκάλεσα· και ιδού, συ ευλογών ευλόγησας αυτούς τρίτην ταύτην φοράν·
\par 11 τώρα λοιπόν φύγε εις τον τόπον σου· έλεγον να σε τιμήσω με τιμάς· αλλ' ιδού, ο Κύριος σε εστέρησε της τιμής.
\par 12 Και είπεν ο Βαλαάμ προς τον Βαλάκ, Δεν είπον και προς τους απεσταλμένους σου, τους οποίους απέστειλας προς εμέ, λέγων,
\par 13 Και αν μοι δώση ο Βαλάκ την οικίαν αυτού πλήρη αργυρίου και χρυσίου, δεν δύναμαι να παραβώ την προσταγήν του Κυρίου, ώστε να κάμω καλόν ή κακόν απ' εμαυτού, αλλ' ό,τι ο Κύριος λαλήση, τούτο θέλω ειπεί;
\par 14 και τώρα, ιδού, εγώ υπάγω προς τον λαόν μου· ελθέ λοιπόν να σοι φανερώσω τι θέλει κάμει ο λαός ούτος εις τον λαόν σου εις τας εσχάτας ημέρας.
\par 15 Και αρχίσας την παραβολήν αυτού είπε, Βαλαάμ ο υιός του Βεώρ είπε, και ο άνθρωπος, ο έχων ανοικτούς τους οφθαλμούς αυτού, είπεν·
\par 16 είπεν εκείνος, όστις ήκουσε τα λόγια του Θεού, και έλαβε την γνώσιν του Υψίστου, όστις είδεν όρασιν του Παντοδυνάμου, πεσών εις έκστασιν, έχων όμως ανοικτούς τους οφθαλμούς αυτού·
\par 17 Θέλω ιδεί αυτόν, αλλ' ουχί τώρα· θέλω θεωρήσει αυτόν, αλλ' ουχί εκ του πλησίον· θέλει ανατείλει άστρον εξ Ιακώβ, και θέλει αναστηθή σκήπτρον εκ του Ισραήλ, και θέλει πατάξει τους αρχηγούς Μωάβ, και εξολοθρεύσει πάντας τους υιούς του Σήθ·
\par 18 και ο Εδώμ θέλει είσθαι κληρονομία, και ο Σηείρ θέλει είσθαι κληρονομία εις τους εχθρούς αυτού· και ο Ισραήλ θέλει πράξει εν ισχύϊ·
\par 19 και θέλει εξέλθει εξ Ιακώβ ο εξουσιάζων, και θέλει εξολοθρεύσει τον διασωθέντα εκ της πόλεως.
\par 20 Και ιδών τον Αμαλήκ, ήρχισε την παραβολήν αυτού και είπεν, Ο Αμαλήκ είναι αρχή των εθνών· αλλ' εν τω τέλει αυτού θέλει αφανισθή.
\par 21 Και ιδών τον Κεναίον, ήρχισε την παραβολήν αυτού και είπεν, Ισχυρά είναι η κατοικία σου, και θέτεις την φωλεάν σου επί την πέτραν·
\par 22 πλην ο Κεναίος θέλει καταπορθηθή, εωσού σε φέρη αιχμάλωτον ο Ασσούρ.
\par 23 Και επανέλαβε την παραβολήν αυτού και είπεν, Ω τις θέλει ζήσει, όταν ο Θεός κάμη τούτο;
\par 24 Και πλοία θέλουσιν ελθεί από των παραλίων των Κητιαίων, και θέλουσι καταθλίψει τον Ασσούρ, και θέλουσι καταθλίψει τον Εβερ· αλλά και εκείνοι θέλουσιν εξαφανισθή.
\par 25 Και σηκωθείς ο Βαλαάμ ανεχώρησε και επέστρεψεν εις τον τόπον αυτού· ο δε Βαλάκ απήλθε και αυτός εις την οδόν αυτού.

\chapter{25}

\par 1 Και έμεινεν ο Ισραήλ εν Σιττείμ· και ήρχισεν ο λαός να πορνεύη μετά των θυγατέρων Μωάβ·
\par 2 αίτινες προσεκάλεσαν τον λαόν εις τας θυσίας των θεών αυτών· και έφαγεν ο λαός και προσεκύνησε τους θεούς αυτών.
\par 3 Και προσεκολλήθη ο Ισραήλ εις τον Βέελ-φεγώρ· και εξήφθη η οργή του Κυρίου κατά του Ισραήλ.
\par 4 Και είπε Κύριος προς τον Μωϋσήν, Λάβε πάντας τους αρχηγούς του λαού και κρέμασον αυτούς ενώπιον του Κυρίου κατέναντι του ηλίου· διά να σηκωθή από του Ισραήλ η φλογερά οργή του Κυρίου.
\par 5 Και είπεν ο Μωϋσής προς τους κριτάς του Ισραήλ, Φονεύσατε έκαστος τους ανθρώπους αυτού, τους όσοι προσεκολλήθησαν εις τον Βέελ-φεγώρ.
\par 6 Και ιδού, εις εκ των υιών Ισραήλ ήλθε φέρων εις τους αδελφούς αυτού γυναίκα Μαδιανίτιν, ενώπιον του Μωϋσέως και ενώπιον πάσης της συναγωγής των υιών Ισραήλ, ενώ έκλαιον εν τη θύρα της σκηνής του μαρτυρίου.
\par 7 Και ιδών Φινεές, ο υιός του Ελεάζαρ, υιού του Ααρών του ιερέως, εσηκώθη εκ μέσον της συναγωγής, και λαβών εις την χείρα αυτού δοράτιον,
\par 8 υπήγε κατόπιν του ανθρώπου του Ισραηλίτου εις την σκηνήν και διεπέρασεν αμφοτέρους, τον τε άνθρωπον τον Ισραηλίτην και την γυναίκα διά της κοιλίας αυτής. Και έπαυσεν η πληγή από των υιών Ισραήλ.
\par 9 Ήσαν δε οι αποθανόντες εν τη πληγή εικοσιτέσσαρες χιλιάδες.
\par 10 Και ελάλησε Κύριος προς τον Μωϋσήν λέγων,
\par 11 Ο Φινεές, ο υιός του Ελεάζαρ, υιού Ααρών του ιερέως, απέστρεψε τον θυμόν μου από των υιών Ισραήλ, δείξας ζήλον υπέρ εμού μεταξύ αυτών, όθεν δεν εξωλόθρευσα τους υιούς Ισραήλ εν τη ζηλοτυπία μου·
\par 12 διά τούτο ειπέ, Ιδού, εγώ δίδω εις αυτόν την διαθήκην μου της ειρήνης·
\par 13 και θέλει είσθαι εις αυτόν και εις το σπέρμα αυτού μετ' αυτόν διαθήκη ιερατείας αιωνίου· διότι εστάθη ζηλωτής υπέρ του Θεού αυτού, και έκαμεν εξιλέωσιν υπέρ των υιών Ισραήλ.
\par 14 Το δε όνομα του Ισραηλίτου του θανατωθέντος εκείνου, όστις εθανατώθη μετά της γυναικός της Μαδιανίτιδος, ήτο Ζιμβρί, υιός Σαλού, άρχοντος οικογενείας επισήμου μεταξύ των Συμεωνιτών.
\par 15 Και το όνομα της γυναικός της Μαδιανίτιδος της θανατωθείσης ήτο Χασβί, θυγάτηρ του Σούρ, αρχηγού λαού, εξ επισήμου οικογενείας εν Μαδιάμ.
\par 16 Και ελάλησε Κύριος προς τον Μωϋσήν λέγων,
\par 17 Πολεμείτε τους Μαδιανίτας και πατάξατε αυτούς·
\par 18 διότι αυτοί σας πολεμούσι με τας δολιότητας αυτών, με τας οποίας σας εδολιεύθησαν εις την υπόθεσιν του Φεγώρ και εις την υπόθεσιν της Χασβί θυγατρός του άρχοντος Μαδιανίτου, της αδελφής αυτών, ήτις εθανατώθη εν τη ημέρα της πληγής διά την υπόθεσιν του Φεγώρ.

\chapter{26}

\par 1 Και μετά την πληγήν ελάλησε Κύριος προς τον Μωϋσήν και προς τον Ελεάζαρ τον υιόν του Ααρών τον ιερέα, λέγων,
\par 2 Λάβετε το κεφάλαιον πάσης της συναγωγής των υιών Ισραήλ από είκοσι ετών και επάνω, κατά τους οίκους των πατέρων αυτών, πάντας τους δυναμένους εν τω Ισραήλ να εξέλθωσιν εις πόλεμον.
\par 3 Και ελάλησαν ο Μωϋσής και Ελεάζαρ ο ιερεύς προς αυτούς εις τας πεδιάδας Μωάβ παρά τον Ιορδάνην, κατέναντι της Ιεριχώ, λέγοντες,
\par 4 Αριθμήσατε αυτούς από είκοσι ετών και επάνω, καθώς προσέταξε Κύριος εις τον Μωϋσήν και εις τους υιούς Ισραήλ, τους εξελθόντας εκ γης Αιγύπτου.
\par 5 Ρουβήν, ο πρωτότοκος του Ισραήλ· οι υιοί Ρουβήν ήσαν Ανώχ, εξ ου η συγγένεια των Ανωχιτών· εκ Φαλλού, η συγγένεια των Φαλλουϊτών·
\par 6 εξ Εσρών, η συγγένεια των Εσρωνιτών· εκ του Χαρμί, η συγγένεια των Χαρμιτών.
\par 7 Αύται είναι αι συγγένειαι των Ρουβηνιτών· και η απαρίθμησις αυτών ήτο τεσσαράκοντα τρεις χιλιάδες και επτακόσιοι τριάκοντα.
\par 8 και οι υιοί του Φαλλού ήσαν Ελιάβ·
\par 9 και οι υιοί του Ελιάβ, Νεμουήλ και Δαθάν και Αβειρών. Ούτοι είναι ο Δαθάν και ο Αβειρών, οι ονομαστοί εκείνοι εν τη συναγωγή, οι στασιάσαντες κατά του Μωϋσέως και κατά του Ααρών εν τη συνοδία του Κορέ, ότε εστασίασαν κατά του Κυρίου·
\par 10 και ήνοιξεν η γη το στόμα αυτής και κατέπιεν αυτούς μετά του Κορέ, εν τω εξολοθρευμώ της συνοδίας αυτού, ότε το πυρ κατέφαγε τους διακοσίους πεντήκοντα ανθρώπους· και έγειναν εις σημείον·
\par 11 του Κορέ όμως οι υιοί δεν απέθανον.
\par 12 Οι υιοί Συμεών κατά τας οικογενείας αυτών ήσαν, εκ Νεμουήλ, η συγγένεια των Νεμουηλιτών· εξ Ιαμείν, η συγγένεια των Ιαμεινιτών· εξ Ιαχείν, η συγγένεια των Ιαχεινιτών·
\par 13 εκ Ζερά, η συγγένεια των Ζεριτών· εκ Σαούλ, η συγγένεια των Σαουλιτών.
\par 14 Αύται είναι αι συγγένειαι των Συμεωνιτών· κατά την απαρίθμησιν αυτών, εικοσιδύο χιλιάδες και διακόσιοι.
\par 15 Οι υιοί Γαδ κατά τας συγγενείας αυτών ήσαν, εκ του Σιφών, η συγγένεια των Σιφωνιτών· εξ Αγγί, η συγγένεια των Αγγιτών· εκ Σουνί, η συγγένεια των Σουνιτών·
\par 16 εξ Αζενί, η συγγένεια των Αζενιτών· εξ Ηρί, η συγγένεια των Ηριτών·
\par 17 εξ Αρόδ, η συγγένεια των Αροδιτών· εξ Αριηλί, η συγγένεια των Αριηλιτών.
\par 18 Αύται είναι αι συγγένειαι των υιών Γάδ· κατά την απαρίθμησιν αυτών, τεσσαράκοντα χιλιάδες και πεντακόσιοι.
\par 19 Οι υιοί Ιούδα ήσαν Ηρ και Αυνάν· και απέθανον ο Ηρ και ο Αυνάν εν τη γη Χαναάν.
\par 20 Και οι υιοί Ιούδα κατά τας συγγενείας αυτών ήσαν, εκ Σηλά, η συγγένεια των Σηλανιτών· εκ Φαρές, η συγγένεια των Φαρεσιτών· εκ Ζαρά, η συγγένεια των Ζαριτών·
\par 21 και οι υιοί Φαρές ήσαν εξ Εσρών, η συγγένεια των Εσρωνιτών· εξ Αμούλ, η συγγένεια των Αμουλιτών.
\par 22 Αύται είναι αι συγγένειαι Ιούδα· κατά την απαρίθμησιν αυτών, εβδομήκοντα εξ χιλιάδες και πεντακόσιοι.
\par 23 Οι υιοί Ισσάχαρ κατά τας συγγενείας αυτών ήσαν, εκ Θωλά, συγγένεια των Θωλαϊτών· εκ Φουά, η συγγένεια των Φουνιτών·
\par 24 εξ Ιασούβ, η συγγένεια των Ιασουβιτών· εκ Σιμβρών, η συγγένεια των Σιμβρωνιτών.
\par 25 Αύται είναι αι συγγένειαι Ισσάχαρ· κατά την απαρίθμησιν αυτών, εξήκοντα τέσσαρες χιλιάδες και τριακόσιοι.
\par 26 Οι υιοί Ζαβουλών κατά τας συγγενείας αυτών ήσαν, εκ Σερέδ, συγγένεια των Σερεδιτών· εξ Αιλών, η συγγένεια των Αιλωνιτών· εξ Ιαλεήλ, η συγγένεια των Ιαλεηλιτών.
\par 27 Αύται είναι αι συγγένειαι των Ζαβουλωνιτών· κατά την απαρίθμησιν αυτών, εξήκοντα χιλιάδες και πεντακόσιοι.
\par 28 Οι υιοί Ιωσήφ κατά τας συγγενείας αυτών ήσαν Μανασσής και Εφραΐμ.
\par 29 Οι υιοί Μανασσή ήσαν, εκ Μαχείρ, η συγγένεια των Μαχειριτών. Και ο Μαχείρ εγέννησε τον Γαλαάδ· εκ δε του Γαλαάδ η συγγένεια των Γαλααδιτών·
\par 30 ούτοι είναι οι υιοί Γαλαάδ· εξ Αχιέζερ, η συγγένεια των Αχιεζεριτών· εκ Χελέκ, η συγγένεια των Χελεκιτών·
\par 31 και εξ Ασριήλ, η συγγένεια των Ασριηλιτών· εκ Συχέμ, η συγγένεια των Συχεμιτών· και εκ Σεμιδά, η συγγένεια των Σεμιδαϊτών·
\par 32 και εξ Εφέρ, η συγγένεια των Εφεριτών·
\par 33 και Σαλπαάδ, ο υιός του Εφέρ, δεν είχεν υιούς, αλλά θυγατέρας· τα δε ονόματα των θυγατέρων του Σαλπαάδ ήσαν Μααλά και Νουά, Αγλά, Μελχά και Θερσά.
\par 34 Αύται είναι αι συγγένειαι Μανασσή· και η απαρίθμησις αυτών, πεντήκοντα δύο χιλιάδες και επτακόσιοι.
\par 35 Ούτοι είναι οι υιοί Εφραΐμ κατά τας συγγενείας αυτών· εκ Σουθαλά, η συγγένεια των Σουθαλαϊτών· εκ Βεχέρ, η συγγένεια των Βεχεριτών· εκ Ταχάν, η συγγένεια των Ταχανιτών·
\par 36 και ούτοι είναι οι υιοί Σουθαλά· εξ Εράν, η συγγένεια των Ερανιτών.
\par 37 Αύται είναι αι συγγένειαι των υιών Εφραΐμ· κατά την απαρίθμησιν αυτών, τριάκοντα δύο χιλιάδες και πεντακόσιοι. Ούτοι είναι οι υιοί Ιωσήφ κατά τας συγγενείας αυτών.
\par 38 Οι υιοί Βενιαμίν κατά τας συγγενείας αυτών ήσαν, εκ Βελά, η συγγένεια των Βελαϊτών· εξ Ασβήλ, η συγγένεια των Ασβηλιτών· εξ Αχιράμ, η συγγένεια των Αχιραμιτών·
\par 39 εκ Σουφάμ, η συγγένεια των Σουφαμιτών· εξ Ουφάμ, η συγγένεια των Ουφαμιτών·
\par 40 και οι υιοί Βελά ήσαν Αρέδ και Νααμάν· εξ Αρέδ, η συγγένεια των Αρεδιτών· εκ Νααμάν, συγγένεια των Νααμιτών.
\par 41 Ούτοι είναι οι υιοί Βενιαμίν κατά τας συγγενείας αυτών· και η απαρίθμησις αυτών ήτο τεσσαράκοντα πέντε χιλιάδες και εξακόσιοι.
\par 42 Ούτοι είναι οι υιοί Δαν κατά τας συγγενείας αυτών· εκ Σουάμ, η συγγένεια των Σουαμιτών· αύται είναι αι συγγένειαι Δαν κατά τας συγγενείας αυτών·
\par 43 πάσαι αι συγγένειαι των Σουαμιτών, κατά την απαρίθμησιν αυτών, ήσαν εξήκοντα τέσσαρες χιλιάδες και τετρακόσιοι.
\par 44 Οι υιοί Ασήρ κατά τας συγγενείας αυτών ήσαν, εξ Ιεμνά, η συγγένεια των Ιεμνιτών· εξ Ιεσουΐ, η συγγένεια των Ιεσουϊτών· εκ Βεριά, η συγγένεια των Βεριαϊτών·
\par 45 εκ των υιών Βεριά ήσαν, εξ Έβερ, η συγγένεια των Εβεριτών· εκ Μαλχιήλ, η συγγένεια των Μαλχιηλιτών·
\par 46 και το όνομα της θυγατρός του Ασήρ ήτο Σάρα.
\par 47 Αύται είναι αι συγγένειαι των υιών Ασήρ κατά την απαρίθμησιν αυτών, πεντήκοντα τρεις χιλιάδες και τετρακόσιοι.
\par 48 Οι υιοί Νεφθαλί κατά τας συγγενείας αυτών ήσαν, εξ Ιασιήλ, συγγένεια των Ιασιηλιτών· εκ Γουνί, η συγγένεια των Γουνιτών·
\par 49 εξ Ιεσέρ, η συγγένεια των Ιεσεριτών· εκ Σιλλήμ, η συγγένεια των Σιλλημιτών.
\par 50 Αύται είναι αι συγγένειαι Νεφθαλί κατά τας συγγενείας αυτών· και απαρίθμησις αυτών ήτο τεσσαράκοντα πέντε χιλιάδες και τετρακόσιοι.
\par 51 Αύτη είναι η απαρίθμησις των υιών Ισραήλ, εξακόσιαι χιλιάδες και χίλιοι επτακόσιοι τριάκοντα.
\par 52 Και ελάλησε Κύριος προς τον Μωϋσήν, λέγων,
\par 53 Εις τούτους θέλει μοιρασθή η γη εις κληρονομίαν κατά τον αριθμόν των ονομάτων αυτών·
\par 54 εις τους περισσοτέρους θέλεις δώσει περισσοτέραν κληρονομίαν και εις τους ολιγωτέρους θέλεις δώσει ολιγωτέραν κληρονομίαν· εις έκαστον θέλει δοθή κληρονομία αυτού κατά την απαρίθμησιν αυτού·
\par 55 και η γη θέλει μοιρασθή διά κλήρων· κατά τα ονόματα των φυλών, κατά τας πατριάς αυτών, θέλουσι κληρονομήσει·
\par 56 κατά τον κλήρον θέλει μοιρασθή η κληρονομία αυτών μεταξύ πολλών και ολίγων.
\par 57 Και αύτη είναι η απαρίθμησις των Λευϊτών, κατά τας συγγενείας αυτών· εκ Γηρσών, η συγγένεια των Γηρσωνιτών· εκ Καάθ, η συγγένεια των Κααθιτών· εκ Μεραρί, η συγγένεια των Μεραριτών.
\par 58 Αύται είναι αι συγγένειαι των Λευϊτών· η συγγένεια των Λιβνιτών, η συγγένεια των Χεβρωνιτών, η συγγένεια των Μααλιτών, συγγένεια των Μουσιτών, η συγγένεια των Κοραϊτών· και ο Καάθ εγέννησε τον Αμράμ.
\par 59 Το δε όνομα της γυναικός του Αμράμ ήτο Ιωχαβέδ, θυγάτηρ του Λευΐ, ήτις εγεννήθη εις τον Λευΐ εν Αιγύπτω· και εγέννησεν εις τον Αμράμ τον Ααρών και τον Μωϋσήν και Μαριάμ την αδελφήν αυτών.
\par 60 Και εγεννήθησαν εις τον Ααρών Ναδάβ και Αβιούδ, Ελεάζαρ και Ιθάμαρ.
\par 61 Απέθανον δε ο Ναδάβ και ο Αβιούδ, ότε προσέφεραν πυρ ξένον ενώπιον του Κυρίου.
\par 62 Και η απαρίθμησις αυτών ήτο εικοσιτρείς χιλιάδες, παν αρσενικόν από ενός μηνός και επάνω· διότι δεν απηριθμήθησαν μεταξύ των υιών Ισραήλ, επειδή δεν εδόθη εις αυτούς κληρονομία μεταξύ των υιών Ισραήλ.
\par 63 Ούτοι είναι οι απαριθμηθέντες διά του Μωϋσέως και Ελεάζαρ του ιερέως, οίτινες απηρίθμησαν τους υιούς Ισραήλ εις τας πεδιάδας Μωάβ παρά τον Ιορδάνην κατέναντι της Ιεριχώ.
\par 64 Μεταξύ δε τούτων δεν ευρίσκετο άνθρωπος εκ των απαριθμηθέντων υπό του Μωϋσέως και Ααρών του ιερέως, ότε απηρίθμησαν τους υιούς Ισραήλ εν τη ερήμω Σινά.
\par 65 Διότι ο Κύριος είπε περί αυτών, Εξάπαντος θέλουσιν αποθάνει εν τη ερήμω. Και δεν εναπελείφθη εξ αυτών ουδείς, ει μη Χάλεβ ο υιός του Ιεφοννή και Ιησούς ο υιός του Ναυή.

\chapter{27}

\par 1 Και προσήλθον αι θυγατέρες του Σαλπαάδ, υιού του Εφέρ, υιού του Γαλαάδ, υιού του Μαχείρ, υιού του Μανασσή, εκ των συγγενειών Μανασσή υιού του Ιωσήφ. Και ταύτα είναι τα ονόματα των θυγατέρων αυτού· Μααλά, Νουά και Αγλά και Μελχά και Περσά.
\par 2 Και εστάθησαν ενώπιον του Μωϋσέως και ενώπιον Ελεάζαρ του ιερέως και ενώπιον των αρχόντων και πάσης της συναγωγής εις την θύραν της σκηνής του μαρτυρίου, λέγουσαι,
\par 3 Ο πατήρ ημών απέθανεν εν τη ερήμω· και αυτός δεν ήτο εν τη συνοδία των συναθροισθέντων κατά του Κυρίου εν τη συνοδία του Κορέ, αλλ' απέθανε δι' ιδίαν αυτού αμαρτίαν και δεν είχεν υιούς·
\par 4 διά τι να εξαλειφθή το όνομα του πατρός ημών εκ μέσου της συγγενείας αυτού, διότι δεν έχει υιόν; δότε εις ημάς κληρονομίαν μεταξύ των αδελφών του πατρός ημών.
\par 5 Και έφερεν ο Μωϋσής την κρίσιν αυτών ενώπιον του Κυρίου.
\par 6 Και ελάλησε Κύριος προς τον Μωϋσήν λέγων,
\par 7 Ορθώς λαλούσιν αι θυγατέρες του Σαλπαάδ· εξάπαντος θέλεις δώσει εις αυτάς κτήμα εις κληρονομίαν μεταξύ των αδελφών του πατρός αυτών· και θέλεις διαβιβάσει εις αυτάς την κληρονομίαν του πατρός αυτών.
\par 8 Και θέλεις λαλήσει προς τους υιούς Ισραήλ λέγων, Εάν άνθρωπός τις αποθάνη και δεν έχη υιόν, τότε θέλετε διαβιβάσει την κληρονομίαν αυτού εις την θυγατέρα αυτού.
\par 9 Και εάν δεν έχη θυγατέρα, τότε θέλετε δώσει την κληρονομίαν αυτού εις τους αδελφούς αυτού.
\par 10 Και εάν δεν έχη αδελφούς, τότε θέλετε δώσει την κληρονομίαν αυτού εις τους αδελφούς του πατρός αυτού.
\par 11 Εάν δε ο πατήρ αυτού δεν έχη αδελφούς, τότε θέλετε δώσει την κληρονομίαν αυτού εις τον συγγενή αυτού τον πλησιέστερον εκ της συγγενείας αυτού, και ούτος θέλει εξουσιάζει αυτήν. Και τούτο θέλει είσθαι εις τους υιούς Ισραήλ διάταγμα κρίσεως, καθώς προσέταξε Κύριος εις τον Μωϋσήν.
\par 12 Και είπε Κύριος προς τον Μωϋσήν, Ανάβα εις τούτο το όρος Αβαρίμ και ιδέ την γην, την οποίαν έδωκα εις τους υιούς Ισραήλ·
\par 13 και αφού ίδης αυτήν, θέλεις προστεθή και συ εις τον λαόν σου, καθώς προσετέθη Ααρών ο αδελφός σου·
\par 14 διότι σεις ηναντιώθητε εις τον λόγον μου εν τη ερήμω Σιν εν τη αντιλογία της συναγωγής, ώστε να με αγιάσητε εις το ύδωρ ενώπιον αυτών. Τούτο είναι το ύδωρ Μεριβά εν Κάδης εν τη ερήμω Σιν.
\par 15 Και ελάλησεν ο Μωϋσής προς τον Κύριον, λέγων,
\par 16 Κύριος, ο Θεός των πνευμάτων πάσης σαρκός, ας διορίση άνθρωπον επί την συναγωγήν,
\par 17 όστις να εξέλθη έμπροσθεν αυτών, και όστις να εισέλθη έμπροσθεν αυτών, και όστις να εξάγη αυτούς, και όστις να εισάγη αυτούς· ώστε να μη ήναι η συναγωγή του Κυρίου ως πρόβατα μη έχοντα ποιμένα.
\par 18 Και είπε Κύριος προς τον Μωϋσήν, Λάβε μετά σου Ιησούν τον υιόν του Ναυή, άνθρωπον εις τον οποίον είναι το πνεύμα, και επίθες την χείρα σου επ' αυτόν·
\par 19 και παράστησον αυτόν ενώπιον Ελεάζαρ του ιερέως και ενώπιον πάσης της συναγωγής και δος εις αυτόν διαταγάς ενώπιον αυτών·
\par 20 και θέλεις επιθέσει επ' αυτόν από της δόξης σου, διά να υπακούωσιν εις αυτόν πάσα η συναγωγή των υιών Ισραήλ·
\par 21 και θέλει παρασταθή ενώπιον Ελεάζαρ του ιερέως, όστις θέλει ερωτήσει περί αυτού κατά την κρίσιν του Ουρίμ ενώπιον του Κυρίου· κατά τον λόγον αυτού θέλουσιν εξέρχεσθαι και κατά τον λόγον αυτού θέλουσιν εισέρχεσθαι, αυτός και πάντες οι υιοί Ισραήλ μετ' αυτού και πάσα η συναγωγή.
\par 22 Και έκαμεν ο Μωϋσής, καθώς προσέταξεν εις αυτόν ο Κύριος· και έλαβε τον Ιησούν και παρέστησεν αυτόν ενώπιον Ελεάζαρ του ιερέως, και ενώπιον πάσης της συναγωγής·
\par 23 και επέθηκε τας χείρας αυτού επ' αυτόν και έδωκεν εις αυτόν διαταγάς, καθώς προσέταξε Κύριος διά χειρός του Μωϋσέως.

\chapter{28}

\par 1 Και ελάλησε Κύριος προς τον Μωϋσήν λέγων,
\par 2 Πρόσταξον τους υιούς Ισραήλ και ειπέ προς αυτούς, Τα δώρα μου, τους άρτους μου, την θυσίαν μου γινομένην διά πυρός εις οσμήν ευωδίας προς εμέ, προσέχετε να προσφέρητε εις εμέ εν τω πρέποντι καιρώ αυτών.
\par 3 Και θέλεις ειπεί προς αυτούς, Αύτη είναι η διά πυρός γινομένη προσφορά, την οποίαν θέλετε προσφέρει προς τον Κύριον· δύο αρνία ενιαύσια άμωμα καθ' ημέραν, εις παντοτεινόν ολοκαύτωμα.
\par 4 Το εν αρνίον θέλεις προσφέρει το πρωΐ, και το άλλο αρνίον θέλεις προσφέρει το δειλινόν.
\par 5 Και διά προσφοράν εξ αλφίτων θέλετε προσφέρει σεμίδαλιν, το δέκατον του εφά, εζυμωμένην με έλαιον από ελαίας κοπανισμένας, το τέταρτον του ιν.
\par 6 Τούτο είναι παντοτεινόν ολοκαύτωμα, διωρισμένον εν τω όρει Σινά, εις οσμήν ευωδίας, θυσία γινομένη διά πυρός εις τον Κύριον.
\par 7 Και η σπονδή αυτού θέλει είσθαι το τέταρτον του ιν διά το εν αρνίον· εις το αγιαστήριον θέλεις χύσει σίκερα διά σπονδήν εις τον Κύριον.
\par 8 Και το άλλο αρνίον θέλεις προσφέρει το δειλινόν· κατά την εξ αλφίτων προσφοράν της πρωΐας και κατά την σπονδήν αυτής θέλεις προσφέρει αυτό θυσίαν γινομένην διά πυρός εις οσμήν ευωδίας προς τον Κύριον.
\par 9 και την ημέραν του σαββάτου θέλεις προσφέρει δύο αρνία ενιαύσια άμωμα, και δύο δέκατα σεμιδάλεως εζυμωμένης με έλαιον διά προσφοράν εξ αλφίτων, και την σπονδήν αυτής.
\par 10 τούτο είναι το ολοκαύτωμα εκάστου σαββάτου εκτός του παντοτεινού ολοκαυτώματος και της σπονδής αυτού.
\par 11 Και εις τας νεομηνίας σας θέλετε προσφέρει ολοκαύτωμα προς τον Κύριον, δύο μόσχους και ένα κριόν, επτά αρνία ενιαύσια, άμωμα·
\par 12 και δι' έκαστον μόσχον τρία δέκατα σεμιδάλεως εζυμωμένης με έλαιον διά προσφοράν εξ αλφίτων, και διά τον ένα κριόν δύο δέκατα σεμιδάλεως εζυμωμένης με έλαιον διά προσφοράν εξ αλφίτων·
\par 13 και ανά εν δέκατον σεμιδάλεως εζυμωμένης με έλαιον διά προσφοράν εξ αλφίτων δι' έκαστον αρνίον, προς ολοκαύτωμα, θυσίαν γινομένην διά πυρός εις οσμήν ευωδίας προς τον Κύριον.
\par 14 Και η σπονδή αυτών θέλει είσθαι οίνος, το ήμισυ του ιν διά τον μόσχον, και το τρίτον του ιν διά τον κριόν, και το τέταρτον του ιν διά το αρνίον. Τούτο είναι το ολοκαύτωμα εκάστου μηνός, κατά τους μήνας του ενιαυτού.
\par 15 Και εις τράγος εξ αιγών θέλει προσφέρεσθαι προς τον Κύριον εις προσφοράν περί αμαρτίας, εκτός του παντοτεινού ολοκαυτώματος και της σπονδής αυτού.
\par 16 Και την δεκάτην τετάρτην ημέραν του πρώτου μηνός είναι το πάσχα του Κυρίου.
\par 17 Και την δεκάτην πέμπτην του μηνός τούτον είναι εορτή· επτά ημέρας θέλουσι τρώγεσθαι άζυμα.
\par 18 Εν τη πρώτη ημέρα θέλει είσθαι συγκάλεσις αγία· δεν θέλετε κάμνει ουδέν έργον δουλευτικόν.
\par 19 Και θέλετε προσφέρει θυσίαν γινομένην διά πυρός, ολοκαύτωμα προς τον Κύριον, δύο μόσχους εκ βοών και ένα κριόν και επτά αρνία ενιαύσια· άμωμα θέλουσιν είσθαι εις εσάς.
\par 20 Και η εξ αλφίτων προσφορά αυτών θέλει είσθαι σεμίδαλις εζυμωμένη με έλαιον· τρία δέκατα θέλετε προσφέρει διά τον μόσχον και δύο δέκατα διά τον κριόν.
\par 21 Ανά εν δέκατον θέλεις προσφέρει δι' έκαστον αρνίον, κατά τα επτά αρνία·
\par 22 και ένα τράγον εις προσφοράν περί αμαρτίας, διά να γείνη εξιλέωσις διά σας.
\par 23 Εκτός του ολοκαυτώματος της πρωΐας, το οποίον είναι διά ολοκαύτωμα παντοτεινόν, θέλετε προσφέρει ταύτα.
\par 24 Ούτω θέλετε προσφέρει καθ' ημέραν εις τας επτά ημέρας, τα δώρα τα προς θυσίαν γινομένην διά πυρός εις οσμήν ευωδίας προς τον Κύριον. Τούτο θέλει προσφέρεσθαι εκτός του παντοτεινού ολοκαυτώματος και της σπονδής αυτού.
\par 25 Και εν τη ημέρα τη εβδόμη θέλετε έχει συγκάλεσιν αγίαν· δεν θέλετε κάμνει ουδέν έργον δουλευτικόν.
\par 26 Και εν τη ημέρα των απαρχών, όταν προσφέρητε νέαν εξ αλφίτων προσφοράν προς τον Κύριον, εις το τέλος των εβδομάδων σας, θέλετε έχει συγκάλεσιν αγίαν· δεν θέλετε κάμνει ουδέν έργον δουλευτικόν.
\par 27 Και θέλετε προσφέρει ολοκαύτωμα εις οσμήν ευωδίας προς τον Κύριον, δύο μόσχους εκ βοών, ένα κριόν, επτά αρνία ενιαύσια·
\par 28 και η εξ αλφίτων προσφορά αυτών θέλει είσθαι σεμίδαλις εζυμωμένη με έλαιον, τρία δέκατα δι' έκαστον μόσχον, δύο δέκατα διά τον ένα κριόν,
\par 29 ανά εν δέκατον δι' έκαστον αρνίον, κατά τα επτά αρνία·
\par 30 ένα τράγον εξ αιγών, διά να γείνη εξιλέωσις διά σας.
\par 31 Εκτός του παντοτεινού ολοκαυτώματος και της εξ αλφίτων προσφοράς αυτού, ταύτα θέλετε προσφέρει, άμωμα θέλουσιν είσθαι εις εσάς, και τας σπονδάς αυτών.

\chapter{29}

\par 1 Και εν τω μηνί τω εβδόμω, τη πρώτη του μηνός, θέλετε έχει συγκάλεσιν αγίαν· δεν θέλετε κάμνει ουδέν έργον δουλευτικόν· αύτη είναι εις εσάς ημέρα αλαλαγμού σαλπίγγων.
\par 2 Και θέλετε προσφέρει ολοκαύτωμα εις οσμήν ευωδίας προς τον Κύριον, ένα μόσχον εκ βοών, ένα κριόν, επτά αρνία ενιαύσια, άμωμα·
\par 3 και η εξ αλφίτων προσφορά αυτών θέλει είσθαι σεμίδαλις εζυμωμένη με έλαιον, τρία δέκατα διά τον μόσχον, δύο δέκατα διά τον κριόν,
\par 4 και εν δέκατον δι' έκαστον αρνίον, κατά τα επτά αρνία·
\par 5 και ένα τράγον εξ αιγών εις προσφοράν περί αμαρτίας, διά να γείνη εξιλέωσις διά σάς·
\par 6 εκτός του ολοκαυτώματος του μηνός και της εξ αλφίτων προσφοράς αυτού και του παντοτεινού ολοκαυτώματος και της εξ αλφίτων προσφοράς αυτού και των σπονδών αυτών, κατά το διατεταγμένον περί αυτών, θυσίαν γινομένην διά πυρός εις οσμήν ευωδίας προς τον Κύριον.
\par 7 Και τη δεκάτη τούτου του εβδόμου μηνός θέλετε έχει συγκάλεσιν αγίαν· και θέλετε ταπεινώσει τας ψυχάς σας· ουδεμίαν εργασίαν θέλετε κάμνει·
\par 8 και θέλετε προσφέρει ολοκαύτωμα προς τον Κύριον εις οσμήν ευωδίας, ένα μόσχον εκ βοών, ένα κριόν, επτά αρνία ενιαύσια· άμωμα θέλουσιν είσθαι εις εσάς.
\par 9 Και η εξ αλφίτων προσφορά αυτών θέλει είσθαι σεμίδαλις εζυμωμένη με έλαιον, τρία δέκατα διά τον μόσχον, δύο δέκατα διά τον ένα κριόν,
\par 10 ανά εν δέκατον δι' έκαστον αρνίον, κατά τα επτά αρνία·
\par 11 ένα τράγον εξ αιγών εις προσφοράν περί αμαρτίας, εκτός της προς εξιλέωσιν περί αμαρτίας προσφοράς και του παντοτεινού ολοκαυτώματος και της εξ αλφίτων προσφοράς αυτού και των σπονδών αυτών.
\par 12 Και τη δεκάτη πέμπτη ημέρα του εβδόμου μηνός θέλετε έχει συγκάλεσιν αγίαν· δεν θέλετε κάμνει ουδέν έργον δουλευτικόν· και θέλετε εορτάζει εορτήν εις τον Κύριον επτά ημέρας.
\par 13 Και θέλετε προσφέρει ολοκαύτωμα, θυσίαν γινομένην διά πυρός εις οσμήν ευωδίας προς τον Κύριον, δεκατρείς μόσχους, δύο κριούς, δεκατέσσαρα αρνία ενιαύσια· άμωμα θέλουσιν είσθαι.
\par 14 Και η εξ αλφίτων προσφορά αυτών θέλει είσθαι σεμίδαλις εζυμωμένη με έλαιον, τρία δέκατα δι' έκαστον μόσχον εκ των δεκατριών μόσχων, δύο δέκατα δι' έκαστον κριόν εκ των δύο κριών,
\par 15 και ανά εν δέκατον δι' έκαστον αρνίον κατά τα δεκατέσσαρα αρνία·
\par 16 και ένα τράγον εξ αιγών εις προσφοράν περί αμαρτίας, εκτός του παντοτεινού ολοκαυτώματος, της εξ αλφίτων προσφοράς αυτού και της σπονδής αυτού.
\par 17 Και τη δευτέρα ημέρα θέλετε προσφέρει δώδεκα μόσχους, δύο κριούς, δεκατέσσαρα αρνία ενιαύσια, άμωμα·
\par 18 και την εξ αλφίτων προσφοράν αυτών και τας σπονδάς αυτών, διά τους μόσχους, διά τους κριούς και διά τα αρνία, κατά τον αριθμόν αυτών, ως είναι διατεταγμένον·
\par 19 και ένα τράγον αιγών εις προσφοράν περί αμαρτίας, εκτός του παντοτεινού ολοκαυτώματος και της εξ αλφίτων προσφοράς αυτού και των σπονδών αυτών.
\par 20 Και τη τρίτη ημέρα ένδεκα μόσχους, δύο κριούς, δεκατέσσαρα αρνία ενιαύσια, άμωμα·
\par 21 και την εξ αλφίτων προσφοράν αυτών και τας σπονδάς αυτών, διά τους μόσχους, διά τους κριούς και διά τα αρνία, κατά τον αριθμόν αυτών, ως είναι διατεταγμένον·
\par 22 και ένα τράγον εις προσφοράν περί αμαρτίας, εκτός του παντοτεινού ολοκαυτώματος και της εξ αλφίτων προσφοράς αυτού και της σπονδής αυτού.
\par 23 Και τη τετάρτη ημέρα δέκα μόσχους, δύο κριούς, δεκατέσσαρα αρνία ενιαύσια, άμωμα·
\par 24 την εξ αλφίτων προσφοράν αυτών και τας σπονδάς αυτών, διά τους μόσχους, διά τους κριούς και διά τα αρνία κατά τον αριθμόν αυτών, ως είναι διατεταγμένον·
\par 25 και ένα τράγον εξ αιγών εις προσφοράν περί αμαρτίας, εκτός του παντοτεινού ολοκαυτώματος, της εξ αλφίτων προσφοράς αυτού και της σπονδής αυτού.
\par 26 Και τη πέμπτη ημέρα εννέα μόσχους, δύο κριούς, δεκατέσσαρα αρνία ενιαύσια, άμωμα·
\par 27 και την εξ αλφίτων προσφοράν αυτών και τας σπονδάς αυτών, διά τους μόσχους, διά τους κριούς και διά τα αρνία, κατά τον αριθμόν αυτών, ως είναι διατεταγμένον·
\par 28 και ένα τράγον εις προσφοράν περί αμαρτίας, εκτός του παντοτεινού ολοκαυτώματος και της εξ αλφίτων προσφοράς αυτού και της σπονδής αυτού.
\par 29 Και τη έκτη ημέρα οκτώ μόσχους, δύο κριούς, δεκατέσσαρα αρνία ενιαύσια, άμωμα·
\par 30 και την εξ αλφίτων προσφοράν αυτών και τας σπονδάς αυτών, διά τους μόσχους, διά τους κριούς και διά τα αρνία κατά τον αριθμόν αυτών, ως είναι διατεταγμένον·
\par 31 και ένα τράγον εις προσφοράν περί αμαρτίας, εκτός του παντοτεινού ολοκαυτώματος, της εξ αλφίτων προσφοράς αυτού και της σπονδής αυτού.
\par 32 Και τη εβδόμη ημέρα επτά μόσχους, δύο κριούς, δεκατέσσαρα αρνία ενιαύσια, άμωμα·
\par 33 και την εξ αλφίτων προσφοράν αυτών και τας σπονδάς αυτών, διά τους μόσχους, διά τους κριούς και διά τα αρνία κατά τον αριθμόν αυτών, ως είναι διατεταγμένον περί αυτών·
\par 34 και ένα τράγον εις προσφοράν περί αμαρτίας, εκτός του παντοτεινού ολοκαυτώματος, της εξ αλφίτων προσφοράς αυτού και της σπονδής αυτού.
\par 35 Τη ογδόη ημέρα θέλετε έχει σύναξιν επίσημον· ουδέν έργον δουλευτικόν θέλετε κάμνει·
\par 36 και θέλετε προσφέρει ολοκαύτωμα, θυσίαν γινομένην διά πυρός εις οσμήν ευωδίας προς τον Κύριον, ένα μόσχον, ένα κριόν, επτά αρνία ενιαύσια, άμωμα·
\par 37 την εξ αλφίτων προσφοράν αυτών και τας σπονδάς αυτών, διά τον μόσχον, διά την κριόν και διά τα αρνία κατά τον αριθμόν αυτών, ως είναι διατεταγμένον·
\par 38 και ένα τράγον εις προσφοράν περί αμαρτίας, εκτός του παντοτεινού ολοκαυτώματος και της εξ αλφίτων προσφοράς αυτού και της σπονδής αυτού.
\par 39 ταύτα θέλετε κάμνει προς τον Κύριον εις τας διωρισμένας εορτάς σας, εκτός των ευχών σας και των αυτοπροαιρέτων προσφορών σας, διά τα ολοκαυτώματά σας και διά τας εξ αλφίτων προσφοράς σας και διά τας σπονδάς σας και διά τας ειρηνικάς προσφοράς σας.
\par 40 Και ελάλησεν ο Μωϋσής προς τους υιούς Ισραήλ κατά πάντα όσα προσέταξεν ο Κύριος εις τον Μωϋσήν.

\chapter{30}

\par 1 Και ελάλησεν ο Μωϋσής προς τους άρχοντας των φυλών των υιών Ισραήλ, λέγων, Ούτος είναι ο λόγος τον οποίον προσέταξεν ο Κύριος·
\par 2 Όταν άνθρωπός τις κάμη ευχήν προς τον Κύριον ή ομόση όρκον, ώστε να δέση την ψυχήν αυτού με δεσμόν, δεν θέλει παραβή τον λόγον αυτού θέλει κάμει κατά πάντα όσα εξήλθον εκ του στόματος αυτού.
\par 3 Εάν δε γυνή τις κάμη ευχήν προς τον Κύριον και δέση εαυτήν με δεσμόν εν τη οικία του πατρός αυτής εις την νεότητα αυτής,
\par 4 και ακούση ο πατήρ αυτής την ευχήν αυτής και τον δεσμόν αυτής διά του οποίου έδεσε την ψυχήν αυτής και σιωπήση προς αυτήν ο πατήρ αυτής, τότε πάσαι αι ευχαί αυτής θέλουσι μένει και πας δεσμός, διά του οποίου έδεσε την ψυχήν αυτής, θέλει μένει.
\par 5 Εάν δε ο πατήρ αυτής δεν συγκατανεύση εις αυτήν, καθ' ην ημέραν ακούση, πάσαι αι ευχαί αυτής ή οι δεσμοί αυτής, διά των οποίων έδεσε την ψυχήν αυτής, δεν θέλουσι μένει και ο Κύριος θέλει συγχωρήσει αυτήν, διότι ο πατήρ αυτής δεν συγκατένευσεν εις αυτήν.
\par 6 Εάν όμως έχουσα άνδρα ηυχήθη ή επρόφερέ τι διά των χειλέων αυτής, διά του οποίου έδεσε την ψυχήν αυτής,
\par 7 και ήκουσεν ο ανήρ αυτής και εσιώπησε προς αυτήν, καθ' ην ημέραν ήκουσε, τότε αι ευχαί αυτής θέλουσι μένει και οι δεσμοί αυτής, διά των οποίων έδεσε την ψυχήν αυτής, θέλουσι μένει.
\par 8 Εάν όμως ο ανήρ αυτής δεν συγκατένευσεν εις αυτήν, καθ' ην ημέραν ήκουσε, τότε θέλει ακυρώσει την ευχήν αυτής, την οποίαν ηυχήθη, και ό,τι επρόφερε διά των χειλέων αυτής, διά του οποίου έδεσε την ψυχήν αυτής· και ο Κύριος θέλει συγχωρήσει αυτήν.
\par 9 Πάσα όμως ευχή χήρας ή γυναικός αποβεβλημένης, διά της οποίας έδεσε την ψυχήν αυτής, θέλει μένει επ' αυτήν.
\par 10 Και εάν ηυχήθη εν τη οικία του ανδρός αυτής ή έδεσε την ψυχήν αυτής με δεσμόν όρκου,
\par 11 και ήκουσεν ο ανήρ αυτής και εσιώπησε προς αυτήν και δεν ηναντιώθη εις αυτήν, τότε πάσαι αι ευχαί αυτής θέλουσι μένει και πάντες οι δεσμοί, διά των οποίων έδεσε την ψυχήν αυτής, θέλουσι μένει.
\par 12 Εάν όμως ο ανήρ αυτής ηκύρωσεν αυτά ρητώς, καθ' ην ημέραν ήκουσε, παν ό,τι εξήλθεν εκ των χειλέων αυτής περί των ευχών αυτής και περί του δεσμού της ψυχής αυτής δεν θέλει μένει· ο ανήρ αυτής ηκύρωσεν αυτά και ο Κύριος θέλει συγχωρήσει αυτήν.
\par 13 Πάσαν ευχήν και πάντα όρκον υποχρεόνοντα εις κακουχίαν ψυχής ο ανήρ αυτής δύναται να επικυρώση ή ο ανήρ αυτής δύναται να ακυρώση·
\par 14 εάν όμως ο ανήρ αυτής σιωπήση διόλου προς αυτήν από ημέρας εις ημέραν, τότε επικυρόνει πάσας τας ευχάς αυτής ή πάντας τους δεσμούς αυτής, οίτινες είναι επ' αυτήν· αυτός επεκύρωσεν αυτά, διότι εσιώπησε προς αυτήν, καθ' ην ημέραν ήκουσεν.
\par 15 Εάν όμως ηκύρωσεν αυτά ρητώς αφού ήκουσε, τότε θέλει βαστάσει την αμαρτίαν αυτής.
\par 16 Ταύτα είναι τα διατάγματα, τα οποία προσέταξε Κύριος εις τον Μωϋσήν, μεταξύ ανδρός και γυναικός αυτού και μεταξύ πατρός και θυγατρός αυτού εν τη νεότητι αυτής εν τη οικία του πατρός αυτής.

\chapter{31}

\par 1 Και ελάλησε Κύριος προς τον Μωϋσήν λέγων,
\par 2 Κάμε την εκδίκησιν των υιών Ισραήλ κατά των Μαδιανιτών· έπειτα θέλεις προστεθή εις τον λαόν σου.
\par 3 Και ελάλησεν ο Μωϋσής προς τον λαόν, λέγων, Ας οπλισθώσιν από σας άνδρες εις πόλεμον και ας υπάγωσιν εναντίον του Μαδιάμ, διά να εκδικήσωσι τον Κύριον κατά του Μαδιάμ·
\par 4 ανά χιλίους από πάσης φυλής, εκ πασών των φυλών του Ισραήλ, θέλετε αποστείλει εις τον πόλεμον.
\par 5 Και εξηριθμήθησαν εκ των χιλιάδων του Ισραήλ χίλιοι από πάσης φυλής, δώδεκα χιλιάδες ώπλισμένοι εις πόλεμον.
\par 6 Και απέστειλεν αυτούς ο Μωϋσής εις τον πόλεμον, χιλίους από πάσης φυλής, αυτούς και Φινεές τον υιόν του Ελεάζαρ του ιερέως, εις τον πόλεμον, μετά των σκευών των αγίων και μετά των σαλπίγγων του αλαλαγμού εις τας χείρας αυτού.
\par 7 Και επολέμησαν εναντίον του Μαδιάμ, καθώς προσέταξε Κύριος εις τον Μωϋσήν, και εθανάτωσαν παν αρσενικόν.
\par 8 Και εκτός των θανατωθέντων και τους βασιλείς του Μαδιάμ εθανάτωσαν, τον Ευΐ και τον Ρεκέμ και τον Σούρ και τον Ουρ και τον Ρεβά, πέντε βασιλείς του Μαδιάμ· και τον Βαλαάμ υιόν του Βεώρ εθανάτωσαν εν μαχαίρα.
\par 9 Και ηχμαλώτισαν οι υιοί Ισραήλ τας γυναίκας του Μαδιάμ και τα παιδία αυτών, και πάντα τα κτήνη αυτών και πάντα τα ποίμνια αυτών και πάντα τα υπάρχοντα αυτών ελεηλάτησαν.
\par 10 Και πάσας τας πόλεις αυτών κατά τας κατοικίας αυτών, και πάντας τους πύργους αυτών, κατέκαυσαν εν πυρί.
\par 11 Και έλαβον πάντα τα λάφυρα και πάσαν την λεηλασίαν από ανθρώπου έως κτήνους.
\par 12 Και έφεραν προς τον Μωϋσήν και προς Ελεάζαρ τον ιερέα και προς την συναγωγήν των υιών Ισραήλ τους αιχμαλώτους, και τα λάφυρα και την λεηλασίαν, εις το στρατόπεδον, εις τας πεδιάδας του Μωάβ, τας παρά τον Ιορδάνην, κατέναντι της Ιεριχώ.
\par 13 Και εξήλθον ο Μωϋσής και Ελεάζαρ ο ιερεύς και πάντες οι άρχοντες της συναγωγής εις συνάντησιν αυτών έξω του στρατοπέδου.
\par 14 Και εθυμώθη ο Μωϋσής εναντίον των αρχηγών του στρατεύματος, των χιλιάρχων και των εκατοντάρχων, των ελθόντων από της παρατάξεως του πολέμου·
\par 15 και είπε προς αυτούς ο Μωϋσής, Ζώσας αφήσατε πάσας τας γυναίκας;
\par 16 ιδού, αύται έγειναν αιτία εις τους υιούς Ισραήλ, κατά την συμβουλήν του Βαλαάμ, να ανομήσωσιν εναντίον του Κυρίου εις την υπόθεσιν του Φεγώρ, και έγεινεν η πληγή επί της συναγωγής του Κυρίου·
\par 17 και τώρα θανατώσατε εκ των παιδίων πάντα τα αρσενικά, και θανατώσατε πάσας τας γυναίκας, όσαι εγνώρισαν άνδρα, κοιμηθείσαι μετ' αυτού·
\par 18 πάντα όμως τα κοράσια, όσα δεν εγνώρισαν κοίτην ανδρός, φυλάξατε ζώντα δι' εαυτούς·
\par 19 και μείνατε έξω του στρατοπέδου επτά ημέρας· όστις εθανάτωσεν άνθρωπον, και όστις ήγγισε πεφονευμένον, καθαρίσθητε σεις και οι αιχμάλωτοί σας την τρίτην ημέραν και την εβδόμην ημέραν·
\par 20 και καθαρίσατε πάντα τα ιμάτια και πάντα τα σκεύη τα δερμάτινα και πάντα τα ειργασμένα εκ τριχών αιγός και πάντα τα ξύλινα σκεύη.
\par 21 Και είπεν Ελεάζαρ ο ιερεύς προς τους πολεμιστάς τους ερχομένους εις τον πόλεμον, Τούτο είναι το πρόσταγμα του νόμου, τον οποίον προσέταξεν ο Κύριος εις τον Μωϋσήν·
\par 22 πλην το χρυσίον και το αργύριον, τον χαλκόν, τον σίδηρον, τον κασσίτερον και τον μόλυβδον,
\par 23 παν ό,τι δύναται να εμβή εις το πυρ, θέλετε περάσει διά του πυρός και θέλει είσθαι καθαρόν· πρέπει όμως να καθαρισθή και διά του ύδατος του καθαρισμού· και παν ό,τι δεν εμβαίνει εις το πυρ, θέλετε περάσει διά του ύδατος·
\par 24 και θέλετε πλύνει τα ιμάτιά σας την εβδόμην ημέραν και θέλετε είσθαι καθαροί· και μετά ταύτα θέλετε εισέλθει εις το στρατόπεδον.
\par 25 Και ελάλησε Κύριος προς τον Μωϋσήν λέγων,
\par 26 Λάβε τον αριθμόν των λαφύρων της αιχμαλωσίας, από ανθρώπου έως κτήνους, συ και Ελεάζαρ ο ιερεύς και οι αρχηγοί των πατριών της συναγωγής·
\par 27 και διαίρεσον τα λάφυρα εις δύο μεταξύ των πολεμιστών των εξελθόντων εις τον πόλεμον και πάσης της συναγωγής·
\par 28 και αφαίρεσον διά τον Κύριον απόδομα εκ των ανδρών, των πολεμιστών, των εξελθόντων εις τον πόλεμον, ανά εν εκ πεντακοσίων, από ανθρώπων και από βοών και από όνων και από προβάτων·
\par 29 από του ημίσεως αυτών θέλετε λάβει και δώσει εις Ελεάζαρ τον ιερέα προσφοράν του Κυρίου·
\par 30 και από του ημίσεως μεριδίον των υιών Ισραήλ θέλεις λάβει εν μερίδιον από πεντήκοντα, από ανθρώπων, από βοών, από όνων και από προβάτων, από παντός κτήνους, και θέλεις δώσει αυτά εις τους Λευΐτας, τους φυλάττοντας τας φυλακάς της σκηνής του Κυρίου.
\par 31 Και έκαμεν ο Μωϋσής και Ελεάζαρ ο ιερεύς καθώς προσέταξε Κύριος εις τον Μωϋσήν.
\par 32 Και τα λάφυρα τα εναπολειφθέντα εκ της λεηλασίας, την οποίαν έκαμον οι άνδρες οι πολεμισταί, ήσαν πρόβατα εξακόσια, εβδομήκοντα πέντε χιλιάδες,
\par 33 και βόες εβδομήκοντα δύο χιλιάδες,
\par 34 και όνοι χιλιάδες εξήκοντα μία,
\par 35 και ψυχαί ανθρώπων, από των γυναικών αίτινες δεν εγνώρισαν κοίτην ανδρός, πάσαι αι ψυχαί, τριάκοντα δύο χιλιάδες.
\par 36 Και το ήμισυ, το μερίδιον των εξελθόντων εις τον πόλεμον, ήτο κατά τον αριθμόν, τα πρόβατα τριακόσια τριάκοντα επτά χιλιάδες και πεντακόσια·
\par 37 και το απόδομα του Κυρίου από των προβάτων ήτο εξακόσια εβδομήκοντα πέντε·
\par 38 και οι βόες τριάκοντα εξ χιλιάδες, και το απόδομα του Κυρίου εβδομήκοντα δύο·
\par 39 και οι όνοι τριάκοντα χιλιάδες και πεντακόσιοι, και το απόδομα του Κυρίου εις και εξήκοντα·
\par 40 και ψυχαί ανθρώπων ήσαν δεκαέξ χιλιάδες, και το απόδομα του Κυρίου τριάκοντα δύο ψυχαί.
\par 41 Και έδωκεν ο Μωϋσής το απόδομα, την προσφοράν του Κυρίου, εις Ελεάζαρ τον ιερέα, καθώς προσέταξε Κύριος εις τον Μωϋσήν.
\par 42 Και από του ημίσεως μεριδίου των υιών Ισραήλ, το οποίον ο Μωϋσής εξεχώρισεν από του μεριδίου των ανδρών των πολεμιστών·
\par 43 και τούτο το ήμισυ της συναγωγής ήτο πρόβατα τριακόσια τριάκοντα επτά χιλιάδες και πεντακόσια,
\par 44 και βόες τριάκοντα εξ χιλιάδες,
\par 45 και όνοι τριάκοντα χιλιάδες και πεντακόσιοι,
\par 46 και ψυχαί ανθρώπων δεκαέξ χιλιάδες·
\par 47 και έλαβεν ο Μωϋσής από του ημίσεως μεριδίου των υιών Ισραήλ ανά εν εκ πεντήκοντα, από ανθρώπων και από κτηνών, και έδωκεν αυτά εις τους Λευΐτας, τους φυλάττοντας τας φυλακάς της σκηνής του Κυρίου, καθώς προσέταξε Κύριος εις τον Μωϋσήν
\par 48 Και προσήλθον εις τον Μωϋσήν οι αρχηγοί οι επί των χιλιάδων του στρατεύματος, χιλίαρχοι και εκατόνταρχοι,
\par 49 και είπον προς τον Μωϋσήν, Οι δούλοί σου έλαβον τον αριθμόν των ανδρών των πολεμιστών των υπό την επιστασίαν ημών, και δεν λείπει εξ ημών ουδέ είς·
\par 50 και εφέραμε τα δώρα του Κυρίου, έκαστος ό,τι εύρηκε, σκεύη χρυσά, αλύσους και βραχιόλια, δακτυλίδια, ενώτια και περιδέραια, διά να γείνη εξιλέωσις υπέρ των ψυχών ημών ενώπιον του Κυρίου.
\par 51 Και έλαβεν ο Μωϋσής και Ελεάζαρ ο ιερεύς το χρυσίον παρ' αυτών όλον εις σκεύη ειργασμένα.
\par 52 Και παν το χρυσίον της προσφοράς των χιλιάρχων και εκατοντάρχων, το οποίον προσέφεραν εις τον Κύριον, ήτο δεκαέξ χιλιάδες επτακόσιοι πεντήκοντα σίκλοι.
\par 53 Διότι οι άνδρες οι πολεμισταί είχον λαφυραγωγήσει έκαστος δι' εαυτόν.
\par 54 Και έλαβεν ο Μωϋσής και Ελεάζαρ ο ιερεύς το χρυσίον παρά των χιλιάρχων και εκατοντάρχων και έφεραν αυτό εις την σκηνήν του μαρτυρίου εις μνημόσυνον των υιών Ισραήλ ενώπιον του Κυρίου.

\chapter{32}

\par 1 Οι δε υιοί Ρουβήν και οι υιοί Γαδ είχον πλήθος κτηνών πολύ σφόδρα· και ότε είδον την γην Ιαζήρ και την γην Γαλαάδ, ότι, ιδού, ο τόπος ήτο τόπος διά κτήνη,
\par 2 οι υιοί Γαδ και οι υιοί Ρουβήν ελθόντες είπον προς τον Μωϋσήν και προς Ελεάζαρ τον ιερέα και προς τους άρχοντας της συναγωγής, λέγοντες,
\par 3 Η Αταρώθ και Δαιβών και Ιαζήρ και Νιμρά και Εσεβών και Ελεαλή και Σεβάμ και Νεβώ και Βαιών,
\par 4 η γη την οποίαν επάταξεν ο Κύριος έμπροσθεν της συναγωγής του Ισραήλ, είναι γη κτηνοτρόφος και οι δούλοί σου έχουσι κτήνη·
\par 5 διά τούτο, είπον, εάν ευρήκαμεν χάριν έμπροσθέν σου, ας δοθή η γη αύτη εις τους δούλους σου διά ιδιοκτησίαν· μη διαβιβάσης ημάς τον Ιορδάνην.
\par 6 Και είπεν ο Μωϋσής προς τους υιούς Γαδ και προς τους υιούς Ρουβήν, Οι αδελφοί σας θέλουσιν υπάγει εις πόλεμον και σεις θέλετε μείνει εδώ;
\par 7 και διά τι δειλιάζετε την καρδίαν των υιών Ισραήλ, διά να μη περάσωσιν εις την γην, την οποίαν ο Κύριος έδωκεν εις αυτούς;
\par 8 ούτως έκαμον οι πατέρες σας, ότε απέστειλα αυτούς από Κάδης-βαρνή διά να ίδωσι την γήν·
\par 9 και ανέβησαν μέχρι της φάραγγος Εσχώλ, και ιδόντες την γην εδειλίασαν την καρδίαν των υιών Ισραήλ, διά να μη εισέλθωσιν εις την γην την οποίαν ο Κύριος έδωκεν εις αυτούς·
\par 10 και εξήφθη η οργή του Κυρίου εν τη ημέρα εκείνη, και ώμοσε λέγων,
\par 11 Δεν θέλουσιν ιδεί οι άνδρες οι αναβάντες εξ Αιγύπτου, από είκοσι ετών και επάνω, την γην την οποίαν ώμοσα προς τον Αβραάμ, προς τον Ισαάκ και προς τον Ιακώβ· διότι δεν με ηκολούθησαν εντελώς·
\par 12 εκτός Χάλεβ υιού Ιεφοννή του Κενεζίτου και Ιησού υιού του Ναυή· διότι ηκολούθησαν εντελώς τον Κύριον.
\par 13 Και εξήφθη η οργή του Κυρίου κατά του Ισραήλ, και έκαμεν αυτούς να περιπλανώνται εις την έρημον τεσσαράκοντα έτη, εωσού εξωλοθρεύθη πάσα η γενεά, ήτις είχε πράξει το κακόν ενώπιον του Κυρίου.
\par 14 Και ιδού, σεις εσηκώθητε αντί των πατέρων σας, γενεά ανθρώπων αμαρτωλών, διά να εξάψητε περισσότερον την φλόγα της οργής του Κυρίου κατά του Ισραήλ.
\par 15 Επειδή, εάν εκκλίνητε απ' αυτού, έτι πάλιν θέλει αφήσει τον Ισραήλ εν τη ερήμω και θέλετε εξολοθρεύσει πάντα τον λαόν τούτον.
\par 16 Και προσήλθον εις αυτόν και είπον, Θέλομεν οικοδομήσει ενταύθα μάνδρας διά τα κτήνη ημών και πόλεις διά τα παιδία ημών·
\par 17 ημείς δε ώπλισμένοι θέλομεν προχωρεί πρόθυμοι έμπροσθεν των υιών Ισραήλ, εωσού φέρωμεν αυτούς εις τον τόπον αυτών· τα δε παιδία ημών θέλουσι κατοικεί εις τετειχισμένας πόλεις, διά τους κατοίκους του τόπου·
\par 18 δεν θέλομεν επιστρέψει εις τας οικίας ημών, εωσού οι υιοί Ισραήλ κληρονομήσωσιν έκαστος την κληρονομίαν αυτού·
\par 19 διότι ημείς δεν θέλομεν να κληρονομήσωμεν μετ' αυτών πέραν του Ιορδάνου και επέκεινα· διότι η κληρονομία ημών έπεσεν εις ημάς εντεύθεν του Ιορδάνου προς ανατολάς.
\par 20 Και είπε προς αυτούς ο Μωϋσής, Εάν κάμητε κατά τον λόγον τούτον, εάν προχωρήτε ώπλισμένοι έμπροσθεν του Κυρίου εις πόλεμον,
\par 21 και διαβήτε πάντες, ώπλισμένοι τον Ιορδάνην έμπροσθεν του Κυρίου, εωσού εκδιώξη τους εχθρούς αυτού απ' έμπροσθεν αυτού,
\par 22 και υποταχθή η γη έμπροσθεν του Κυρίου· τότε μετά ταύτα θέλετε επιστρέψει και θέλετε είσθαι αθώοι ενώπιον του Κυρίου και ενώπιον του Ισραήλ, και θέλετε έχει την γην ταύτην ιδιοκτησίαν σας ενώπιον του Κυρίου·
\par 23 εάν όμως δεν κάμητε ούτως, ιδού, θέλετε αμαρτήσει ενώπιον του Κυρίου· και ας ήσθε βέβαιοι ότι θέλει σας ευρή η αμαρτία σας·
\par 24 οικοδομήσατε πόλεις διά τα παιδία σας και μάνδρας διά τα πρόβατά σας και κάμετε εκείνο, το οποίον εξήλθεν εκ του στόματός σας.
\par 25 Και είπον οι υιοί Γαδ και οι υιοί Ρουβήν προς τον Μωϋσήν, λέγοντες, Οι δούλοί σου θέλουσι κάμει καθώς ο κύριός μου προστάζει·
\par 26 τα παιδία ημών, αι γυναίκες ημών, τα ποίμνια ημών και πάντα τα κτήνη ημών θέλουσι μένει ενταύθα εις τας πόλεις του Γαλαάδ·
\par 27 οι δούλοί σου όμως θέλουσι διαβή πάντες ώπλισμένοι, παρατεταγμένοι ενώπιον του Κυρίου εις μάχην, καθώς λέγει ο κύριός μου.
\par 28 Τότε ο Μωϋσής έδωκε προσταγήν περί αυτών εις Ελεάζαρ τον ιερέα και εις τον Ιησούν υιόν του Ναυή και εις τους αρχηγούς των πατριών των φυλών των υιών Ισραήλ·
\par 29 και είπε προς αυτούς ο Μωϋσής, Εάν οι υιοί Γαδ και οι υιοί Ρουβήν διαβώσι με σας τον Ιορδάνην, πάντες ώπλισμένοι εις μάχην, έμπροσθεν του Κυρίου, και κατακυριευθή η γη έμπροσθέν σας, τότε θέλετε δώσει εις αυτούς την γην Γαλαάδ εις ιδιοκτησίαν·
\par 30 εάν όμως δεν θέλωσι να διαβώσιν ώπλισμένοι με σας, τότε θέλουσι λάβει κληρονομίαν μεταξύ σας εν τη γη Χαναάν.
\par 31 Και απεκρίθησαν οι υιοί Γαδ και οι υιοί Ρουβήν λέγοντες, Ως είπεν ο Κύριος εις τους δούλους σου, ούτω θέλομεν κάμει·
\par 32 ημείς θέλομεν διαβή ώπλισμένοι έμπροσθεν του Κυρίου εις την γην Χαναάν, διά να έχωμεν την ιδιοκτησίαν της κληρονομίας ημών εντεύθεν του Ιορδάνου.
\par 33 Και έδωκεν εις αυτούς ο Μωϋσής, εις τους υιούς Γαδ και εις τους υιούς Ρουβήν και εις το ήμισυ της φυλής Μανασσή υιού του Ιωσήφ, το βασίλειον του Σηών βασιλέως των Αμορραίων και το βασίλειον του Ωγ βασιλέως της Βασάν, την γην μετά των πόλεων αυτής εν τοις ορίοις, τας πόλεις της γης κύκλω.
\par 34 Και οι υιοί Γαδ ωκοδόμησαν την Δαιβών και την Αταρώθ και την Αροήρ
\par 35 και την Ατρώθ-σοφάν και την Ιαζήρ και την Ιογβεά
\par 36 και την Βαιθ-νιμρά και την Βαιθ-αράν, πόλεις οχυράς, και μάνδρας προβάτων.
\par 37 Και οι υιοί Ρουβήν ωκοδόμησαν την Εσεβών και την Ελεαλή και την Κιριαθαΐμ
\par 38 και την Νεβώ και την Βάαλ-μεών, μεταβάλλοντες τα ονόματα αυτών, και την Σιβμά· και έδωκαν άλλα ονόματα εις τας πόλεις, τας οποίας ωκοδόμησαν.
\par 39 Και οι υιοί Μαχείρ υιού του Μανασσή υπήγαν εις την Γαλαάδ και εκυρίευσαν αυτήν, εκδιώξαντες τον Αμορραίον τον εν αυτή.
\par 40 Και έδωκεν ο Μωϋσής την Γαλαάδ εις Μαχείρ τον υιόν του Μανασσή· και κατώκησεν εν αυτή.
\par 41 Και Ιαείρ ο υιός του Μανασσή υπήγε και εκυρίευσε τας μικράς πόλεις αυτής· και ωνόμασεν αυτάς Αβώθ-ιαείρ.
\par 42 Και ο Νοβά υπήγε και εκυρίευσε την Καινάθ και τα χωρία αυτής· και ωνόμασεν αυτήν Νοβά, από του ονόματος αυτού.

\chapter{33}

\par 1 Αύται είναι αι οδοιπορίαι των υιών Ισραήλ, των εξελθόντων εκ της γης Αιγύπτου κατά τα στρατεύματα αυτών διά χειρός του Μωϋσέως και του Ααρών.
\par 2 Και έγραψεν ο Μωϋσής τας εξόδους αυτών κατά τας οδοιπορίας αυτών, διά προσταγής του Κυρίου· και αύται είναι αι οδοιπορίαι αυτών κατά τας εξόδους αυτών.
\par 3 Και εσηκώθησαν από Ραμεσσή τον πρώτον μήνα, τη δεκάτη πέμπτη ημέρα του πρώτου μηνός· τη επαύριον του πάσχα εξήλθον οι υιοί Ισραήλ εν χειρί υψηλή ενώπιον πάντων των Αιγυπτίων·
\par 4 ενώ οι Αιγύπτιοι έθαπτον εκείνους, τους οποίους ο Κύριος επάταξε μεταξύ αυτών, παν πρωτότοκον· και εις τους θεούς αυτών έκαμεν ο Κύριος εκδίκησιν.
\par 5 Και σηκωθέντες οι υιοί Ισραήλ από Ραμεσσή, εστρατοπέδευσαν εν Σοκχώθ.
\par 6 Και σηκωθέντες από Σοκχώθ, εστρατοπέδευσαν εν Εθάμ, ήτις είναι εν τω άκρω της ερήμου.
\par 7 Και σηκωθέντες από Εθάμ, έστρεψαν προς Πι-αϊρώθ, ήτις είναι κατέναντι Βέελ-σεφών· και εστρατοπέδευσαν κατέναντι της Μιγδώλ.
\par 8 Και σηκωθέντες απ' έμπροσθεν της Αϊρώθ, διέβησαν διά της θαλάσσης εις την έρημον· και ώδοιπόρησαν οδόν τριών ημερών διά της ερήμου Εθάμ και εστρατοπέδευσαν εν Μερρά.
\par 9 Και σηκωθέντες από Μερρά, ήλθον εις Αιλείμ· και ήσαν εν Αιλείμ δώδεκα πηγαί υδάτων και εβδομήκοντα δένδρα φοινίκων· και εστρατοπέδευσαν εκεί.
\par 10 Και σηκωθέντες από Αιλείμ, εστρατοπέδευσαν παρά την Ερυθράν θάλασσαν.
\par 11 Και σηκωθέντες από της Ερυθράς θαλάσσης, εστρατοπέδευσαν εν τη ερήμω Σιν.
\par 12 Και σηκωθέντες από της ερήμου Σιν, εστρατοπέδευσαν εν Δοφκά.
\par 13 Και σηκωθέντες από Δοφκά, εστρατοπέδευσαν εν Αιλούς.
\par 14 Και σηκωθέντες από Αιλούς, εστρατοπέδευσαν εν Ραφιδείν, όπου δεν ήτο ύδωρ διά να πίη ο λαός.
\par 15 Και σηκωθέντες από Ραφιδείν, εστρατοπέδευσαν εν τη ερήμω Σινά.
\par 16 Και σηκωθέντες από της ερήμου Σινά, εστρατοπέδευσαν εν Κιβρώθ-αττααβά.
\par 17 Και σηκωθέντες από Κιβρώθ-αττααβά, εστρατοπέδευσαν εν Ασηρώθ.
\par 18 Και σηκωθέντες από Ασηρώθ, εστρατοπέδευσαν εν Ριθμά.
\par 19 Και σηκωθέντες από Ριθμά, εστρατοπέδευσαν εν Ριμμών-φαρές.
\par 20 Και σηκωθέντες από Ριμμών-φαρές, εστρατοπέδευσαν εν Λιβνά.
\par 21 Και σηκωθέντες από Λιβνά, εστρατοπέδευσαν εν Ρισσά.
\par 22 Και σηκωθέντες από Ρισσά, εστρατοπέδευσαν εν Κεελαθά.
\par 23 Και σηκωθέντες από Κεελαθά, εστρατοπέδευσαν εν τω όρει Σαφέρ.
\par 24 Και σηκωθέντες από του όρους Σαφέρ, εστρατοπέδευσαν εν Χαραδά.
\par 25 Και σηκωθέντες από Χαραδά, εστρατοπέδευσαν εν Μακηλώθ.
\par 26 Και σηκωθέντες από Μακηλώθ, εστρατοπέδευσαν εν Ταχάθ.
\par 27 Και σηκωθέντες από Ταχάθ, εστρατοπέδευσαν εν Θαρά.
\par 28 Και σηκωθέντες από Θαρά, εστρατοπέδευσαν εν Μιθκά.
\par 29 Και σηκωθέντες από Μιθκά, εστρατοπέδευσαν εν Ασεμωνά.
\par 30 Και σηκωθέντες από Ασεμωνά, εστρατοπέδευσαν εν Μοσηρώθ.
\par 31 Και σηκωθέντες από Μοσηρώθ, εστρατοπέδευσαν εν Βενέ-ιακάν.
\par 32 Και σηκωθέντες από Βενέ-ιακάν, εστρατοπέδευσαν εν τω όρει Γαδγάδ.
\par 33 Και σηκωθέντες από του όρους Γαδγάδ, εστρατοπέδευσαν εν Ιοτβαθά.
\par 34 Και σηκωθέντες από Ιοτβαθά, εστρατοπέδευσαν εν Εβρωνά.
\par 35 Και σηκωθέντες από Εβρωνά, εστρατοπέδευσαν εν Εσιών-γάβερ.
\par 36 Και σηκωθέντες από Εσιών-γάβερ, εστρατοπέδευσαν εν τη ερήμω Σιν, ήτις είναι η Κάδης.
\par 37 Και σηκωθέντες από Κάδης, εστρατοπέδευσαν εν τω όρει Ωρ, κατά το άκρον της γης Εδώμ.
\par 38 Και ανέβη Ααρών ο ιερεύς, διά προσταγής του Κυρίου, εις το όρος Ωρ και απέθανεν εκεί, το τεσσαρακοστόν έτος της εξόδου των υιών Ισραήλ εκ γης Αιγύπτου, τον πέμπτον μήνα· την πρώτην του μηνός.
\par 39 Και ο Ααρών ήτο εκατόν εικοσιτριών ετών, ότε απέθανεν εν τω όρει Ωρ.
\par 40 Και ήκουσεν ο Χαναναίος, βασιλεύς της Αράδ, όστις κατώκει προς μεσημβρίαν, εν γη Χαναάν, την έλευσιν των υιών Ισραήλ.
\par 41 Και σηκωθέντες από του όρους Ωρ, εστρατοπέδευσαν εν Σαλμωνά.
\par 42 Και σηκωθέντες από Σαλμωνά, εστρατοπέδευσαν εν Φυνών.
\par 43 Και σηκωθέντες από Φυνών, εστρατοπέδευσαν εν Ωβώθ.
\par 44 Και σηκωθέντες από Ωβώθ, εστρατοπέδευσαν εν Ιϊέ-αβαρίμ, κατά τα όρια του Μωάβ.
\par 45 Και σηκωθέντες από Ιείμ, εστρατοπέδευσαν εν Δαιβών-γαδ.
\par 46 Και σηκωθέντες από Δαιβών-γαδ, εστρατοπέδευσαν εν Αλμών-διβλαθαΐμ.
\par 47 Και σηκωθέντες από Αλμών-διβλαθαΐμ, εστρατοπέδευσαν εις τα όρη Αβαρίμ, κατέναντι Νεβώ.
\par 48 Και σηκωθέντες από των ορέων Αβαρίμ, εστρατοπέδευσαν εις τας πεδιάδας Μωάβ παρά τον Ιορδάνην κατέναντι της Ιεριχώ.
\par 49 Και εστρατοπέδευσαν παρά τον Ιορδάνην, από Βαιθ-ιεσιμώθ έως Αβέλ-σιττίμ, εις τας πεδιάδας Μωάβ.
\par 50 Και ελάλησε Κύριος προς τον Μωϋσήν εις τας πεδιάδας Μωάβ παρά τον Ιορδάνην κατέναντι της Ιεριχώ, λέγων,
\par 51 Λάλησον προς τους υιούς Ισραήλ και ειπέ προς αυτούς, Αφού διαβήτε τον Ιορδάνην προς την γην Χαναάν,
\par 52 θέλετε εκδιώξει πάντας τους κατοίκους της γης απ' έμπροσθέν σας και καταστρέψει πάσας τας εικόνας αυτών και καταστρέψει πάντα τα χυτά είδωλα αυτών και κατεδαφίσει πάντας τους βωμούς αυτών·
\par 53 και θέλετε κυριεύσει την γην και κατοικήσει εν αυτή· διότι εις εσάς έδωκα την γην ταύτην εις κληρονομίαν·
\par 54 και θέλετε διαμοιρασθή την γην διά κλήρων μεταξύ των συγγενειών σας· εις τους περισσοτέρους θέλετε δώσει περισσοτέραν κληρονομίαν, και εις τους ολιγωτέρους θέλετε δώσει ολιγωτέραν κληρονομίαν· εκάστου η κληρονομία θέλει είσθαι εις το μέρος όπου πέση ο κλήρος αυτού· κατά τας φυλάς των πατέρων σας θέλετε κληρονομήσει.
\par 55 Εάν όμως δεν εκδιώξητε τους κατοίκους της γης απ' έμπροσθέν σας, τότε όσους ηθέλετε αφήσει εξ αυτών να μένωσι, θέλουσιν είσθαι άκανθαι εις τους οφθαλμούς σας και κέντρα εις τας πλευράς σας και θέλουσι σας ενοχλεί εν τω τόπω όπου κατοικείτε·
\par 56 και έτι, καθώς εστοχαζόμην να κάμω εις αυτούς, ούτω θέλω κάμει εις εσάς.

\chapter{34}

\par 1 Και ελάλησε Κύριος προς τον Μωϋσήν, λέγων,
\par 2 Πρόσταξον τους υιούς Ισραήλ και ειπέ προς αυτούς, Όταν εισέλθητε εις την γην Χαναάν, την γην εκείνην ήτις θέλει σας πέσει εις κληρονομίαν, την γην Χαναάν μετά των ορίων αυτής,
\par 3 τότε το μέρος σας το προς μεσημβρίαν θέλει είσθαι από της ερήμου Σιν έως πλησίον Εδώμ· και τα μεσημβρινά όριά σας θέλουσιν είσθαι από του άκρου της αλμυράς θαλάσσης προς ανατολάς·
\par 4 και το όριόν σας θέλει στρέφει από μεσημβρίας προς την ανάβασιν Ακραββίμ, και θέλει διέρχεσθαι εις Σίν· και θέλει προχωρεί από του μεσημβρινού μέρους έως Κάδης-βαρνή, και θέλει εκβαίνει εις Ασάρ-αδδάρ, και θέλει διαβαίνει έως Ασμών·
\par 5 και θέλει στρέφεσθαι το όριον από Ασμών έως του χειμάρρου της Αιγύπτου, και θέλει καταντήσει εις την θάλασσαν.
\par 6 Δυτικόν δε όριον θέλετε έχει την θάλασσαν την μεγάλην· αύτη θέλει είσθαι το δυτικόν όριόν σας.
\par 7 Και ταύτα θέλουσιν είσθαι τα αρκτικά όριά σας· από της μεγάλης θαλάσσης θέλετε θέσει όριόν σας το όρος Ωρ·
\par 8 από του όρους Ωρ θέλετε θέσει όριόν σας έως της εισόδου Αιμάθ, και θέλει προχωρεί το όριον εις Σεδάδ·
\par 9 και θέλει προχωρεί το όριον εις Ζιφρών, και θέλει καταντήσει εις Ασάρ-ενάν· τούτο θέλει είσθαι το αρκτικόν όριόν σας.
\par 10 Και θέλετε θέσει τα ανατολικά όριά σας από Ασάρ-ενάν έως Σεπφάμ·
\par 11 και θέλει καταβαίνει το όριον από Σεπφάμ έως Ριβλά, προς ανατολάς του Αείν· και θέλει καταβαίνει το όριον και θέλει φθάνει εις το πλάγιον της θαλάσσης Χιννερώθ προς ανατολάς·
\par 12 και θέλει καταβαίνει το όριον προς τον Ιορδάνην και θέλει καταντήσει εις την αλμυράν θάλασσαν. Αύτη είναι η γη σας κατά τα όρια αυτής κύκλω.
\par 13 Και προσέταξεν ο Μωϋσής τους υιούς Ισραήλ λέγων, Αύτη είναι η γη, την οποίαν θέλετε κληρονομήσει διά κλήρων, την οποίαν ο Κύριος προσέταξε να δοθή εις τας εννέα φυλάς και εις το ήμισυ της φυλής.
\par 14 Διότι η φυλή των υιών Ρουβήν κατά τον οίκον των πατέρων αυτών και η φυλή των υιών Γαδ κατά τον οίκον των πατέρων αυτών, έλαβον την κληρονομίαν αυτών· και το ήμισυ της φυλής του Μανασσή έλαβε την κληρονομίαν αυτού.
\par 15 Αι δύο φυλαί και το ήμισυ της φυλής έλαβον την κληρονομίαν αυτών εντεύθεν του Ιορδάνου κατέναντι της Ιεριχώ, προς ανατολάς.
\par 16 Και ελάλησε Κύριος προς τον Μωϋσήν, λέγων,
\par 17 Ταύτα είναι τα ονόματα των ανδρών, οίτινες θέλουσι κληροδοτήσει εις εσάς την γήν· Ελεάζαρ ο ιερεύς και Ιησούς ο υιός του Ναυή·
\par 18 και θέλετε λάβει ανά ένα άρχοντα αφ' εκάστης φυλής, διά να κληροδοτήσωσι την γήν·
\par 19 και ταύτα είναι τα ονόματα των ανδρών· Εκ της φυλής Ιούδα, Χάλεβ ο υιός του Ιεφοννή·
\par 20 και εκ της φυλής των υιών Συμεών, Σαμουήλ ο υιός του Αμμιούδ·
\par 21 εκ της φυλής Βενιαμίν, Ελιδάδ ο υιός του Χισλών·
\par 22 και εκ της φυλής των υιών Δαν, ο άρχων Βουκκί ο υιός του Ιογλί·
\par 23 εκ των υιών Ιωσήφ, εκ της φυλής των υιών Μανασσή, ο άρχων Ανιήλ ο υιός του Εφώδ·
\par 24 και εκ της φυλής των υιών Εφραΐμ, ο άρχων Κεμουήλ ο υιός του Σιφτάν·
\par 25 και εκ της φυλής των υιών Ζαβουλών, ο άρχων Ελισαφάν ο υιός του Φαρνάχ·
\par 26 και εκ της φυλής των υιών Ισσάχαρ, ο άρχων Φαλτιήλ ο υιός του Αζάν·
\par 27 και εκ της φυλής των υιών Ασήρ, ο άρχων Αχιούδ ο υιός του Σελωμί·
\par 28 και εκ της φυλής των υιών Νεφθαλί, ο άρχων Φεδαήλ ο υιός του Αμμιούδ.
\par 29 Ούτοι είναι, τους οποίους προσέταξεν ο Κύριος να κληροδοτήσωσιν εις τους υιούς Ισραήλ εν τη γη Χαναάν.

\chapter{35}

\par 1 Και ελάλησε Κύριος προς τον Μωϋσήν εις τας πεδιάδας Μωάβ παρά τον Ιορδάνην κατέναντι της Ιεριχώ, λέγων,
\par 2 Πρόσταξον τους υιούς Ισραήλ να δώσωσιν εις τους Λευΐτας, από της κληρονομίας της ιδιοκτησίας αυτών, πόλεις διά να κατοικήσωσι· και περίχωρα θέλετε δώσει εις τους Λευΐτας διά τας πόλεις πέριξ αυτών.
\par 3 Και αι μεν πόλεις θέλουσιν είσθαι εις αυτούς διά να κατοικώσιν εις αυτάς· τα δε περίχωρα αυτών θέλουσιν είσθαι διά τα κτήνη αυτών και διά τα υπάρχοντα αυτών και διά πάντα τα ζώα αυτών.
\par 4 Και τα περίχωρα των πόλεων, τα οποία θέλετε δώσει εις τους Λευΐτας, θέλουσιν είσθαι, από του τείχους της πόλεως και έξω, χίλιαι πήχαι κύκλω.
\par 5 Και θέλετε μετρήσει από του έξω της πόλεως προς το ανατολικόν μέρος δύο χιλιάδας πηχών, και προς το μεσημβρινόν μέρος δύο χιλιάδας πηχών, και προς το δυτικόν μέρος δύο χιλιάδας πηχών, και προς το αρκτικόν μέρος δύο χιλιάδας πηχών· και η πόλις θέλει είσθαι εν τω μέσω. Ταύτα θέλουσιν είσθαι εις αυτούς τα περίχωρα των πόλεων.
\par 6 Και εκ των πόλεων, τας οποίας θέλετε δώσει εις τους Λευΐτας, εξ πόλεις θέλουσιν είσθαι διά καταφύγιον, τας οποίας θέλετε διορίσει διά να φεύγη εκεί ο φονεύς· και εις ταύτας θέλετε προσθέσει τεσσαράκοντα δύο πόλεις.
\par 7 Πάσαι αι πόλεις, τας οποίας θέλετε δώσει εις τους Λευΐτας, θέλουσιν είσθαι τεσσαράκοντα οκτώ πόλεις· ταύτας θέλετε δώσει μετά των περιχώρων αυτών.
\par 8 Και αι πόλεις, τας οποίας θέλετε δώσει, θέλουσιν είσθαι εκ της ιδιοκτησίας των υιών Ισραήλ· από των εχόντων πολλά θέλετε δώσει πολλά, και από των εχόντων ολίγα θέλετε δώσει ολίγα· έκαστος κατά την κληρονομίαν, την οποίαν εκληρονόμησε, θέλει δώσει εκ των πόλεων αυτού εις τους Λευΐτας.
\par 9 Και ελάλησε Κύριος προς τον Μωϋσήν λέγων,
\par 10 Λάλησον προς τους υιούς Ισραήλ και ειπέ προς αυτούς, Όταν διαβήτε τον Ιορδάνην προς την γην Χαναάν,
\par 11 τότε θέλετε διορίσει εις εαυτούς πόλεις, διά να ήναι εις εσάς πόλεις καταφυγίου, ώστε να φεύγη εκεί ο φονεύς, όστις εφόνευσεν άνθρωπον ακουσίως.
\par 12 Και θέλουσιν είσθαι εις εσάς πόλεις διά καταφύγιον από του εκδικούντος το αίμα· διά να μη αποθάνη ο φονεύς, εωσού παρασταθή ενώπιον της συναγωγής εις κρίσιν.
\par 13 Και εκ των πόλεων, τας οποίας θέλετε δώσει, εξ πόλεις θέλουσιν είσθαι διά καταφύγιον εις εσάς.
\par 14 Τας τρεις πόλεις θέλετε δώσει εντεύθεν του Ιορδάνου, και τας τρεις πόλεις θέλετε δώσει εν τη γη Χαναάν· πόλεις καταφυγίου θέλουσιν είσθαι.
\par 15 Αύται αι εξ πόλεις θέλουσιν είσθαι καταφύγιον διά τους υιούς Ισραήλ και διά τον ξένον και διά τον παροικούντα μεταξύ αυτών· ώστε όστις φονεύση άνθρωπον ακουσίως να φεύγη εκεί.
\par 16 Και εάν πατάξη αυτόν με όργανον σιδηρούν, ώστε να αποθάνη, είναι φονεύς· ο φονεύς εξάπαντος θέλει θανατωθή.
\par 17 Και εάν πατάξη αυτόν με λίθον εκ της χειρός, διά του οποίου δύναται να αποθάνη, και αποθάνη, είναι φονεύς· ο φονεύς εξάπαντος θέλει θανατωθή.
\par 18 Η εάν πατάξη αυτόν με ξύλινον όργανον εκ χειρός, εκ του οποίου δύναται να αποθάνη, και αποθάνη, είναι φονεύς· ο φονεύς εξάπαντος θέλει θανατωθή.
\par 19 Ο εκδικητής του αίματος, αυτός θέλει θανατόνει τον φονέα· όταν απαντήση αυτόν, θέλει θανατόνει αυτόν·
\par 20 Εάν δε δι' έχθραν ωθήση αυτόν ή παραμονεύσας ρίψη τι επ' αυτόν και αποθάνη,
\par 21 ή εχθρικώς πατάξη αυτόν με την χείρα αυτού και αποθάνη, ο πατάξας εξάπαντος θέλει θανατωθή· είναι φονεύς· ο εκδικητής του αίματος θέλει θανατόνει τον φονέα, όταν απαντήση αυτόν.
\par 22 Εάν όμως ωθήση αυτόν εξαίφνης χωρίς έχθρας ή ρίψη τι επ' αυτόν χωρίς να παραμονεύση αυτόν,
\par 23 ή λίθον τινά χωρίς να ίδη αυτόν, εκ του οποίου δύναται να αποθάνη, και ρίψη επ' αυτόν ώστε να αποθάνη, και δεν ήτο εχθρός αυτού ουδέ εζήτει να κακοποιήση αυτόν,
\par 24 τότε η συναγωγή θέλει κρίνει αναμέσον του φονέως και του εκδικούντος το αίμα κατά τας κρίσεις ταύτας·
\par 25 και η συναγωγή θέλει ελευθερώσει τον φονέα εκ της χειρός του εκδικούντος το αίμα, και η συναγωγή θέλει αποκαταστήσει αυτόν εις την πόλιν του καταφυγίου αυτού, όπου είχε φύγει· και θέλει κατοικεί εν αυτή μέχρι του θανάτου του ιερέως του μεγάλου, του κεχρισμένου διά του αγίου ελαίου.
\par 26 Εάν όμως ο φονεύς εξέλθη έξω των ορίων της πόλεως του καταφυγίου αυτού, εις την οποίαν έφυγε,
\par 27 και ο εκδικητής του αίματος εύρη αυτόν έξω των ορίων της πόλεως του καταφυγίου αυτού και ο εκδικητής του αίματος θανατώση τον φονέα, δεν θέλει είσθαι ένοχος αίματος·
\par 28 διότι έπρεπε να μένη εν τη πόλει του καταφυγίου αυτού μέχρι του θανάτου του μεγάλου ιερέως· μετά δε τον θάνατον του μεγάλου ιερέως, ο φονεύς θέλει επιστρέφει εις την γην της ιδιοκτησίας αυτού.
\par 29 Και ταύτα θέλουσιν είσθαι εις διάταγμα κρίσεως προς εσάς, εις πάσας τας γενεάς σας κατά πάσας τας κατοικίας σας.
\par 30 Όστις φονεύση τινά, ο φονεύς θέλει θανατωθή διά στόματος μαρτύρων· πλην εις μόνος μάρτυς δεν θέλει μαρτυρήσει εναντίον τινός, ώστε να θανατωθή.
\par 31 Και δεν θέλετε λαμβάνει ουδεμίαν εξαγοράν υπέρ της ζωής του φονέως, όστις είναι ένοχος θανάτου· αλλά εξάπαντος θέλει θανατωθή.
\par 32 Και δεν θέλετε λαμβάνει εξαγοράν υπέρ του φυγόντος εις την πόλιν του καταφυγίου αυτού· διά να επιστρέψη να κατοική εν τω τόπω αυτού, μέχρι του θανάτου του ιερέως.
\par 33 Και δεν θέλετε μολύνει την γην εις την οποίαν κατοικείτε· διότι το αίμα αυτό μολύνει την γήν· και η γη δεν δύναται να καθαρισθή από του αίματος του εκχυθέντος επ' αυτής, ειμή διά του αίματος εκείνου όστις έχυσεν αυτό.
\par 34 Μη μολύνετε λοιπόν την γην, εις την οποίαν θέλετε κατοικήσει, εν μέσω της οποίας εγώ κατοικώ· διότι εγώ ο Κύριος είμαι ο κατοικών εν τω μέσω των υιών Ισραήλ.

\chapter{36}

\par 1 Και προσελθόντες οι αρχηγοί των πατριών των συγγενειών των υιών Γαλαάδ, υιού του Μαχείρ, υιού του Μανασσή, εκ των συγγενειών των υιών Ιωσήφ, ελάλησαν ενώπιον του Μωϋσέως και ενώπιον των αρχόντων, οίτινες ήσαν οι αρχηγοί των πατριών των υιών Ισραήλ·
\par 2 και είπον, Ο Κύριος προσέταξεν εις τον κύριόν μου να δώση την γην διά κλήρου προς κληρονομίαν εις τους υιούς Ισραήλ, και ο κύριός μου προσετάχθη παρά του Κυρίου να δώση την κληρονομίαν Σαλπαάδ, του αδελφού ημών, εις τας θυγατέρας αυτού·
\par 3 και εάν νυμφευθώσι μετά τινός εκ των υιών των φυλών των υιών Ισραήλ, τότε η κληρονομία αυτών θέλει αφαιρεθή εκ της κληρονομίας των πατέρων ημών και θέλει προστεθή εις την κληρονομίαν της φυλής ήτις ήθελε δεχθή αυτάς· ούτω θέλει αφαιρεθή από του κλήρου της κληρονομίας ημών·
\par 4 και όταν έλθη το έτος της αφέσεως των υιών Ισραήλ, τότε η κληρονομία αυτών θέλει προστεθή εις την κληρονομίαν της φυλής, ήτις ήθελε δεχθή αυτάς· και η κληρονομία αυτών θέλει αφαιρεθή από της κληρονομίας της φυλής των πατέρων ημών.
\par 5 Και προσέταξεν ο Μωϋσής τους υιούς Ισραήλ, κατά τον λόγον του Κυρίου, λέγων, Η φυλή των υιών Ιωσήφ ελάλησεν ορθώς.
\par 6 Ούτος είναι ο λόγος, τον οποίον προσέταξεν ο Κύριος περί των θυγατέρων του Σαλπαάδ, λέγων, Ας νυμφευθώσι με όντινα αρέσκη εις αυτάς· μόνον θέλουσι νυμφευθή μετά ανδρών εκ της συγγενείας της φυλής των πατέρων αυτών·
\par 7 και δεν θέλει μεταβαίνει η κληρονομία των υιών Ισραήλ από φυλής εις φυλήν· διότι έκαστος εκ των υιών Ισραήλ θέλει είσθαι προσκεκολλημένος εις την κληρονομίαν της φυλής των πατέρων αυτού.
\par 8 Και πάσα θυγάτηρ, ήτις έχει κληρονομίαν εν φυλή τινί των υιών Ισραήλ, θέλει είσθαι γυνή ενός εκ της συγγενείας της φυλής του πατρός αυτής· διά να απολαμβάνωσιν οι υιοί Ισραήλ έκαστος την κληρονομίαν των πατέρων αυτού.
\par 9 Και δεν θέλει μεταβαίνει η κληρονομία από φυλής εις άλλην φυλήν, αλλ' έκαστος εκ των φυλών των υιών Ισραήλ θέλει είσθαι προσκεκολλημένος εις την κληρονομίαν αυτού.
\par 10 Ως προσέταξε Κύριος εις τον Μωϋσήν, ούτως έκαμον αι θυγατέρες του Σαλπαάδ·
\par 11 διότι η Μααλά, η Θερσά και η Αγλά και η Μελχά και η Νουά, αι θυγατέρες του Σαλπαάδ, ενυμφεύθησαν με τους υιούς των αδελφών του πατρός αυτών·
\par 12 ενυμφεύθησαν με άνδρας εκ των συγγενειών των υιών Μανασσή, υιού του Ιωσήφ· και η κληρονομία αυτών έμεινεν εν τη φυλή της συγγενείας του πατρός αυτών.
\par 13 Ταύτα είναι τα προστάγματα και αι κρίσεις, τας οποίας προσέταξεν ο Κύριος διά χειρός του Μωϋσέως προς τους υιούς Ισραήλ, εις τας πεδιάδας Μωάβ παρά τον Ιορδάνην κατέναντι της Ιεριχώ.


\end{document}