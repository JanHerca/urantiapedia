\begin{document}

\title{James}


\chapter{1}

\par Ιάκωβος, δούλος του Θεού και του Κυρίου Ιησού Χριστού, προς τας δώδεκα φυλάς τας διεσπαρμένας, χαίρειν.
\par 2 Πάσαν χαράν νομίσατε, αδελφοί μου, όταν περιπέσητε εις διαφόρους πειρασμούς,
\par 3 γνωρίζοντες ότι η δοκιμασία της πίστεώς σας εργάζεται υπομονήν.
\par 4 Η δε υπομονή ας έχη έργον τέλειον, διά να ήσθε τέλειοι και ολόκληροι, μη όντες εις μηδέν ελλιπείς.
\par 5 Εάν δε τις από σας ήναι ελλιπής σοφίας, ας ζητή παρά του Θεού του δίδοντος εις πάντας πλουσίως και μη ονειδίζοντος, και θέλει δοθή εις αυτόν.
\par 6 Ας ζητή όμως μετά πίστεως, χωρίς να διστάζη παντελώς· διότι ο διστάζων ομοιάζει με κύμα θαλάσσης κινούμενον υπό ανέμων και συνταραττόμενον.
\par 7 Διότι ας μη νομίζη ο άνθρωπος εκείνος ότι θέλει λάβει τι παρά του Κυρίου.
\par 8 Άνθρωπος δίγνωμος είναι ακατάστατος εν πάσαις ταις οδοίς αυτού.
\par 9 Ας καυχάται δε ο αδελφός ο ταπεινός εις το ύψος αυτού,
\par 10 ο δε πλούσιος εις την ταπείνωσιν αυτού, επειδή ως άνθος χόρτου θέλει παρέλθει.
\par 11 Διότι ανέτειλεν ο ήλιος με τον καύσωνα και εξήρανε τον χόρτον, και το άνθος αυτού εξέπεσε, και το κάλλος του προσώπου αυτού ηφανίσθη· ούτω και ο πλούσιος θέλει μαρανθή εν ταις οδοίς αυτού.
\par 12 Μακάριος ο άνθρωπος, όστις υπομένει πειρασμόν· διότι αφού δοκιμασθή, θέλει λάβει τον στέφανον της ζωής, τον οποίον υπεσχέθη ο Κύριος εις τους αγαπώντας αυτόν.
\par 13 Μηδείς πειραζόμενος ας λέγη ότι από του Θεού πειράζομαι· διότι ο Θεός είναι απείραστος κακών και αυτός ουδένα πειράζει.
\par 14 Πειράζεται δε έκαστος υπό της ιδίας αυτού επιθυμίας, παρασυρόμενος και δελεαζόμενος.
\par 15 Έπειτα η επιθυμία αφού συλλάβη, γεννά την αμαρτίαν, η δε αμαρτία εκτελεσθείσα γεννά τον θάνατον.
\par 16 Μη πλανάσθε, αδελφοί μου αγαπητοί.
\par 17 Πάσα δόσις αγαθή και παν δώρημα τέλειον είναι άνωθεν καταβαίνον από του Πατρός των φώτων, εις τον οποίον δεν υπάρχει αλλοίωσις ή σκιά μεταβολής.
\par 18 Εξ ιδίας αυτού θελήσεως εγέννησεν ημάς διά του λόγου της αληθείας, διά να ήμεθα ημείς απαρχή τις των κτισμάτων αυτού.
\par 19 Λοιπόν, αδελφοί μου αγαπητοί, ας είναι πας άνθρωπος ταχύς εις το να ακούη, βραδύς εις το να λαλή, βραδύς εις οργήν·
\par 20 διότι η οργή του ανθρώπου δεν εργάζεται την δικαιοσύνην του Θεού.
\par 21 Διά τούτο απορρίψαντες πάσαν ρυπαρίαν και περισσείαν κακίας δέχθητε μετά πραότητος τον εμφυτευθέντα λόγον τον δυνάμενον να σώση τας ψυχάς σας.
\par 22 Γίνεσθε δε εκτελεσταί του λόγου και μη μόνον ακροαταί, απατώντες εαυτούς.
\par 23 Διότι εάν τις ήναι ακροατής του λόγου και ουχί εκτελεστής, ούτος ομοιάζει με άνθρωπον, όστις θεωρεί το φυσικόν αυτού πρόσωπον εν κατόπτρω·
\par 24 διότι εθεώρησεν εαυτόν και ανεχώρησε, και ευθύς ελησμόνησεν οποίος ήτο.
\par 25 Όστις όμως εγκύψη εις τον τέλειον νόμον της ελευθερίας και επιμείνη εις αυτόν, ούτος γενόμενος ουχί ακροατής επιλήσμων, αλλ' εκτελεστής έργου, ούτος θέλει είσθαι μακάριος εις την εκτέλεσιν αυτού.
\par 26 Εάν τις μεταξύ σας νομίζη ότι είναι θρήσκος, και δεν χαλινόνη την γλώσσαν αυτού αλλ' απατά την καρδίαν αυτού, τούτου η θρησκεία είναι ματαία.
\par 27 Θρησκεία καθαρά και αμίαντος ενώπιον του Θεού και Πατρός είναι αύτη, να επισκέπτηται τους ορφανούς και τας χήρας εν τη θλίψει αυτών, και να φυλάττη εαυτόν αμόλυντον από του κόσμου.

\chapter{2}

\par Αδελφοί μου, μη έχετε με προσωποληψίας την πίστιν του δεδοξασμένου Κυρίου ημών Ιησού Χριστού.
\par 2 Διότι εάν εισέλθη εις την συναγωγήν σας άνθρωπος έχων χρυσούν δακτυλίδιον με λαμπρόν ένδυμα, εισέλθη δε και πτωχός με ρυπαρόν ένδυμα,
\par 3 και επιβλέψητε εις τον φορούντα το ένδυμα το λαμπρόν και είπητε προς αυτόν, Συ κάθου εδώ καλώς, και προς τον πτωχόν είπητε, Συ στέκε εκεί· κάθου εδώ υπό το υποπόδιόν μου,
\par 4 δεν εκάμετε άρα διάκρισιν εν εαυτοίς και εγείνετε κριταί πονηρά διαλογιζόμενοι;
\par 5 Ακούσατε, αδελφοί μου αγαπητοί, δεν εξέλεξεν ο Θεός τους πτωχούς του κόσμου τούτου πλουσίους εν πίστει και κληρονόμους της βασιλείας, την οποίαν υπεσχέθη προς τους αγαπώντας αυτόν;
\par 6 Σεις όμως ητιμάσατε τον πτωχόν. Δεν σας καταδυναστεύουσιν οι πλούσιοι και αυτοί σας σύρουσιν εις κριτήρια;
\par 7 Αυτοί δεν βλασφημούσι το καλόν όνομα, με το οποίον ονομάζεσθε;
\par 8 Εάν μεν εκτελήτε τον νόμον τον βασιλικόν κατά την γραφήν, Θέλεις αγαπά τον πλησίον σου ως σεαυτόν, καλώς ποιείτε·
\par 9 εάν όμως προσωποληπτήτε, κάμνετε αμαρτίαν και ελέγχεσθε υπό του νόμου ως παραβάται.
\par 10 Διότι όστις φυλάξη όλον τον νόμον και πταίση εις εν, έγεινεν ένοχος πάντων.
\par 11 Επειδή ο ειπών, Μη μοιχεύσης, είπε και, Μη φονεύσης· αλλ' εάν δεν μοιχεύσης, φονεύσης δε, έγεινες παραβάτης του νόμου.
\par 12 Ούτω λαλείτε και ούτω πράττετε, ως μέλλοντες να κριθήτε διά του νόμου της ελευθερίας·
\par 13 διότι η κρίσις θέλει είσθαι ανίλεως εις τον όστις δεν έκαμεν έλεος· και το έλεος καυχάται κατά της κρίσεως.
\par 14 Τι το όφελος, αδελφοί μου, εάν λέγη τις ότι έχει πίστιν, και έργα δεν έχη; μήπως η πίστις δύναται να σώση αυτόν;
\par 15 Εάν δε αδελφός ή αδελφή γυμνοί υπάρχωσι και στερώνται της καθημερινής τροφής,
\par 16 και είπη τις εξ υμών προς αυτούς, Υπάγετε εν ειρήνη, θερμαίνεσθε και χορτάζεσθε, και δεν δώσητε εις αυτούς τα αναγκαία του σώματος, τι το όφελος;
\par 17 Ούτω και η πίστις, εάν δεν έχη έργα, νεκρά είναι καθ' εαυτήν.
\par 18 Αλλά θέλει τις ειπεί· Συ έχεις πίστιν, και εγώ έχω έργα· δείξόν μοι την πίστιν σου εκ των έργων σου, και εγώ θέλω σοι δείξει εκ των έργων μου την πίστιν μου.
\par 19 Συ πιστεύεις ότι ο Θεός είναι είς· καλώς ποιείς· και τα δαιμόνια πιστεύουσι και φρίττουσι.
\par 20 Θέλεις όμως να γνωρίσης, ω άνθρωπε μάταιε, ότι η πίστις χωρίς των έργων είναι νεκρά;
\par 21 Αβραάμ ο πατήρ ημών δεν εδικαιώθη εξ έργων, ότε προσέφερεν Ισαάκ τον υιόν αυτού επί το θυσιαστήριον;
\par 22 Βλέπεις ότι η πίστις συνήργει εις τα έργα αυτού, και εκ των έργων η πίστις ετελειώθη,
\par 23 και επληρώθη η γραφή η λέγουσα· Επίστευσε δε Αβραάμ εις τον Θεόν, και ελογίσθη εις αυτόν εις δικαιοσύνην, και φίλος Θεού ωνομάσθη.
\par 24 Βλέπετε λοιπόν ότι εξ έργων δικαιούται ο άνθρωπος και ουχί εκ πίστεως μόνον.
\par 25 Ομοίως δε και Ραάβ η πόρνη δεν εδικαιώθη εξ έργων, ότε υπεδέχθη τους απεσταλμένους και εξέβαλεν αυτούς δι' άλλης οδού;
\par 26 Διότι καθώς το σώμα χωρίς πνεύματος είναι νεκρόν, ούτω και η πίστις χωρίς των έργων είναι νεκρά.

\chapter{3}

\par Μη γίνεσθε πολλοί διδάσκαλοι, αδελφοί μου, εξεύροντες ότι μεγαλητέραν κατάκρισιν θέλομεν λάβει·
\par 2 διότι εις πολλά πταίομεν άπαντες. Εάν τις δεν πταίη εις λόγον, ούτος είναι τέλειος ανήρ, δυνατός να χαλιναγωγήση και όλον το σώμα.
\par 3 Ιδού, τους χαλινούς βάλλομεν εις τα στόματα των ίππων διά να πείθωνται εις ημάς, και μεταφέρομεν όλον το σώμα αυτών.
\par 4 Ιδού, και τα πλοία, όντα τόσον μεγάλα και υπό σφοδρών ανέμων ελαυνόμενα, μεταφέρονται υπό ελαχίστου πηδαλίου, όπου αν θέλη η επιθυμία του κυβερνώντος.
\par 5 Ούτω και η γλώσσα είναι μικρόν μέλος, όμως μεγαλαυχεί. Ιδού, ολίγον πυρ πόσον μεγάλην ύλην ανάπτει·
\par 6 και η γλώσσα πυρ είναι, ο κόσμος της αδικίας. Ούτω μεταξύ των μελών ημών η γλώσσα είναι η μολύνουσα όλον το σώμα και φλογίζουσα τον τροχόν του βίου και φλογιζομένη υπό της γεέννης.
\par 7 Διότι παν είδος θηρίων και πτηνών, ερπετών και θαλασσίων δαμάζεται και εδαμάσθη υπό της ανθρωπίνης φύσεως,
\par 8 την γλώσσαν όμως ουδείς των ανθρώπων δύναται να δαμάση· είναι ακράτητον κακόν, μεστή θανατηφόρου φαρμάκου.
\par 9 Δι' αυτής ευλογούμεν τον Θεόν και Πατέρα, και δι' αυτής καταρώμεθα τους ανθρώπους τους καθ' ομοίωσιν Θεού πλασθέντας·
\par 10 εκ του αυτού στόματος εξέρχεται ευλογία και κατάρα. Δεν πρέπει, αδελφοί μου, ταύτα να γίνωνται ούτω.
\par 11 Μήπως η πηγή από της αυτής τρύπης αναβρύει το γλυκύ και το πικρόν;
\par 12 μήπως είναι δυνατόν, αδελφοί μου, η συκή να κάμη ελαίας ή η άμπελος σύκα; ούτως ουδεμία πηγή είναι δυνατόν να κάμη ύδωρ αλμυρόν και γλυκύ.
\par 13 Τις είναι μεταξύ σας σοφός και επιστήμων; ας δείξη εκ της καλής διαγωγής τα έργα εαυτού εν πραότητι σοφίας.
\par 14 Εάν όμως έχητε εν τη καρδία υμών φθόνον πικρόν και φιλονεικίαν, μη κατακαυχάσθε και ψεύδεσθε κατά της αληθείας.
\par 15 Η σοφία αύτη δεν είναι άνωθεν καταβαίνουσα, αλλ' είναι επίγειος, ζωώδης, δαιμονιώδης.
\par 16 Διότι όπου είναι φθόνος και φιλονεικία, εκεί ακαταστασία και παν αχρείον πράγμα.
\par 17 Η άνωθεν όμως σοφία πρώτον μεν είναι καθαρά, έπειτα ειρηνική, επιεικής, ευπειθής, πλήρης ελέους και καλών καρπών, αμερόληπτος και ανυπόκριτος.
\par 18 Και ο καρπός της δικαιοσύνης σπείρεται εν ειρήνη υπό των ειρηνοποιών.

\chapter{4}

\par Πόθεν προέρχονται πόλεμοι και μάχαι μεταξύ σας; ουχί εντεύθεν, εκ των ηδονών σας, αίτινες στρατεύονται εντός των μελών σας;
\par 2 Επιθυμείτε και δεν έχετε· φονεύετε και φθονείτε, και δεν δύνασθε να επιτύχητε· μάχεσθε και πολεμείτε· αλλά δεν έχετε, επειδή δεν ζητείτε·
\par 3 ζητείτε και δεν λαμβάνετε, διότι κακώς ζητείτε, διά να δαπανήσητε εις τας ηδονάς σας.
\par 4 Μοιχοί και μοιχαλίδες, δεν εξεύρετε ότι η φιλία του κόσμου είναι έχθρα του Θεού; όστις λοιπόν θελήση να ήναι φίλος του κόσμου, εχθρός του Θεού καθίσταται.
\par 5 Η νομίζετε ότι ματαίως η γραφή λέγει, Προς φθόνον επιποθεί το πνεύμα, το οποίον κατώκησεν εν ημίν;
\par 6 Αλλά μεγαλητέραν χάριν δίδει ο Θεός· όθεν λέγει· Ο Θεός εις τους υπερηφάνους αντιτάσσεται, εις δε τους ταπεινούς δίδει χάριν.
\par 7 Υποτάχθητε λοιπόν εις τον Θεόν. Αντιστάθητε εις τον διάβολον, και θέλει φύγει από σάς·
\par 8 πλησιάσατε εις τον Θεόν, και θέλει πλησιάσει εις εσάς. Καθαρίσατε τας χείρας σας, αμαρτωλοί, και αγνίσατε τας καρδίας, δίγνωμοι.
\par 9 Κακοπαθήσατε και πενθήσατε και κλαύσατε· ο γέλως σας ας μεταστραφή εις πένθος και η χαρά εις κατήφειαν.
\par 10 Ταπεινώθητε ενώπιον του Κυρίου, και θέλει σας υψώσει.
\par 11 Μη καταλαλείτε αλλήλους, αδελφοί. Όστις καταλαλεί αδελφόν και κρίνει τον αδελφόν αυτού, καταλαλεί τον νόμον και κρίνει τον νόμον και εάν κρίνης τον νόμον, δεν είσαι εκτελεστής του νόμου, αλλά κριτής.
\par 12 Εις είναι ο νομοθέτης, ο δυνάμενος να σώση και να απολέση· συ τις είσαι όστις κρίνεις τον άλλον;
\par 13 Έλθετε τώρα οι λέγοντες· Σήμερον ή αύριον θέλομεν υπάγει εις ταύτην την πόλιν και θέλομεν κάμει εκεί ένα χρόνον και θέλομεν εμπορευθή και κερδήσει·
\par 14 οίτινες δεν εξεύρετε το μέλλον της αύριον· διότι ποία είναι η ζωή σας; είναι τωόντι ατμός, όστις φαίνεται προς ολίγον και έπειτα αφανίζεται·
\par 15 αντί να λέγητε, Εάν ο Κύριος θελήση, και ζήσωμεν, θέλομεν κάμει τούτο ή εκείνο.
\par 16 Τώρα όμως καυχάσθε εις τας αλαζονείας σας· πάσα τοιαύτη καύχησις είναι κακή.
\par 17 Εις τον όστις λοιπόν εξεύρει να κάμνη το καλόν και δεν κάμνει, εις αυτόν είναι αμαρτία.

\chapter{5}

\par Έλθετε τώρα οι πλούσιοι, κλαύσατε ολολύζοντες διά τας επερχομένας ταλαιπωρίας σας.
\par 2 Ο πλούτος σας εσάπη και τα ιμάτιά σας έγειναν σκωληκόβρωτα,
\par 3 ο χρυσός σας και ο άργυρος εσκωρίασε, και η σκωρία αυτών θέλει είσθαι εις μαρτυρίαν εναντίον σας και θέλει φάγει τας σάρκας σας ως πυρ. Εθησαυρίσατε διά τας εσχάτας ημέρας.
\par 4 Ιδού, ο μισθός των εργατών των θερισάντων τα χωράφια σας, τον οποίον εστερήθησαν από σας, κράζει, και αι κραυγαί των θερισάντων εισήλθον εις τα ώτα Κυρίου Σαβαώθ.
\par 5 Ετρυφήσατε επί της γης και εσπαταλήσατε, εθρέψατε τας καρδίας σας ως εν ημέρα σφαγής.
\par 6 Κατεδικάσατε, εφονεύσατε τον δίκαιον· δεν σας αντιστέκεται.
\par 7 Μακροθυμήσατε λοιπόν, αδελφοί, έως της παρουσίας του Κυρίου. Ιδού, ο γεωργός περιμένει τον πολύτιμον καρπόν της γης και μακροθυμεί δι' αυτόν, εωσού λάβη βροχήν πρώϊμον και όψιμον·
\par 8 μακροθυμήσατε και σεις, στηρίξατε τας καρδίας σας, διότι η παρουσία του Κυρίου επλησίασε.
\par 9 Μη στενάζετε κατ' αλλήλων, αδελφοί, διά να μη κατακριθήτε· ιδού, ο κριτής ίσταται έμπροσθεν των θυρών.
\par 10 Λάβετε, αδελφοί μου, παράδειγμα της κακοπαθείας και της μακροθυμίας τους προφήτας, οίτινες ελάλησαν εν τω ονόματι του Κυρίου.
\par 11 Ιδού, μακαρίζομεν τους υπομένοντας· ηκούσατε την υπομονήν του Ιώβ και είδετε το τέλος του Κυρίου, ότι είναι πολυεύσπλαγχνος ο Κύριος και οικτίρμων.
\par 12 Προ πάντων δε, αδελφοί μου, μη ομνύετε μήτε τον ουρανόν μήτε την γην μήτε άλλον τινά όρκον· αλλ' έστω υμών το ναι ναι, και το ου, διά να μη πέσητε υπό κρίσιν.
\par 13 Κακοπαθεί τις μεταξύ σας; ας προσεύχηται· ευθυμεί τις; ας ψάλλη.
\par 14 Ασθενεί τις μεταξύ σας; ας προσκαλέση τους πρεσβυτέρους της εκκλησίας, και ας προσευχηθώσιν επ' αυτόν, αλείψαντες αυτόν με έλαιον εν τω ονόματι του Κυρίου.
\par 15 Και η μετά πίστεως ευχή θέλει σώσει τον πάσχοντα, και ο Κύριος θέλει εγείρει αυτόν· και αμαρτίας αν έπραξε, θέλουσι συγχωρηθή εις αυτόν.
\par 16 Εξομολογείσθε εις αλλήλους τα πταίσματά σας και εύχεσθε υπέρ αλλήλων, διά να ιατρευθήτε· πολύ ισχύει η δέησις του δικαίου ενθέρμως γενομένη.
\par 17 Ο Ηλίας ήτο άνθρωπος ομοιοπαθής με ημάς και προσηυχήθη ενθέρμως να μη βρέξη, και δεν έβρεξεν επί της γης έτη τρία και μήνας έξ·
\par 18 και πάλιν προσηυχήθη, και ο ουρανός έδωκε βροχήν και η γη εβλάστησε τον καρπόν αυτής.
\par 19 Αδελφοί, εάν τις μεταξύ σας αποπλανηθή από της αληθείας, και επιστρέψη τις αυτόν,
\par 20 ας εξεύρη ότι ο επιστρέψας αμαρτωλόν από της πλάνης της οδού αυτού θέλει σώσει ψυχήν εκ θανάτου και θέλει καλύψει πλήθος αμαρτιών.


\end{document}