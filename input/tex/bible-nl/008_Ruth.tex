\begin{document}

\title{Ruth}



\chapter{1}

\par 1 In de dagen, als de richters richtten, zo geschiedde het, dat er honger in het land was; daarom toog een man van Bethlehem-juda, om als vreemdeling te verkeren in de velden Moabs, hij, en zijn huisvrouw, en zijn twee zonen.
\par 2 De naam nu dezes mans was Elimelech, en de naam zijner huisvrouw Naomi, en de naam zijner twee zonen Machlon en Chiljon, Efrathers, van Bethlehem-juda; en zij kwamen in de velden Moabs, en bleven aldaar.
\par 3 En Elimelech, de man van Naomi, stierf; maar zij werd overgelaten met haar twee zonen.
\par 4 Die namen zich Moabietische vrouwen; de naam der ene was Orpa, en de naam der andere Ruth; en zij bleven aldaar omtrent tien jaren.
\par 5 En die twee, Machlon en Chiljon, stierven ook; alzo werd deze vrouw overgelaten na haar twee zonen en na haar man.
\par 6 Toen maakte zij zich op met haar schoondochters, en keerde weder uit de velden van Moab; want zij had gehoord in het land van Moab, dat de HEERE Zijn volk bezocht had, gevende hun brood.
\par 7 Daarom ging zij uit van de plaats, waar zij geweest was en haar twee schoondochters met haar. Als zij nu gingen op den weg, om weder te keren naar het land van Juda,
\par 8 Zo zeide Naomi tot haar twee schoondochters: Gaat heen, keert weder, een iegelijk tot het huis van haar moeder; de HEERE doe bij u weldadigheid, gelijk als gij gedaan hebt bij de doden, en bij mij.
\par 9 De HEERE geve u, dat gij ruste vindt, een iegelijk in het huis van haar man! En als zij haar kuste, hieven zij haar stem op en weenden;
\par 10 En zij zeiden tot haar: Wij zullen zekerlijk met u wederkeren tot uw volk.
\par 11 Maar Naomi zeide: Keert weder, mijn dochters! Waarom zoudt gij met mij gaan? Heb ik nog zonen in mijn lichaam, dat zij u tot mannen zouden zijn?
\par 12 Keert weder, mijn dochters! Gaat heen; want ik ben te oud om een man te hebben. Wanneer ik al zeide: Ik heb hoop, of ik ook in dezen nacht een man had, ja, ook zonen baarde;
\par 13 Zoudt gij daarnaar wachten, totdat zij zouden groot geworden zijn; zoudt gij daarnaar opgehouden worden, om geen man te nemen? Niet, mijn dochters! Want het is mij veel bitterder dan u; maar de hand des HEEREN is tegen mij uitgegaan.
\par 14 Toen hieven zij haar stem op, en weenden wederom; en Orpa kuste haar schoonmoeder, maar Ruth kleefde haar aan.
\par 15 Daarom zeide zij: Zie, uw zwagerin is wedergekeerd tot haar volk en tot haar goden; keer gij ook weder, uw zwagerin na.
\par 16 Maar Ruth zeide: Val mij niet tegen, dat ik u zou verlaten, om van achter u weder te keren; want waar gij zult heengaan, zal ik ook heengaan, en waar gij zult vernachten, zal ik vernachten; uw volk is mijn volk, en uw God mijn God.
\par 17 Waar gij zult sterven, zal ik sterven, en aldaar zal ik begraven worden; alzo doe mij de HEERE en alzo doe Hij daartoe, zo niet de dood alleen zal scheiding maken tussen mij en tussen u!
\par 18 Als zij nu zag, dat zij vastelijk voorgenomen had met haar te gaan, zo hield zij op tot haar te spreken.
\par 19 Alzo gingen die beiden, totdat zij te Bethlehem inkwamen; en het geschiedde, als zij te Bethlehem inkwamen, dat de ganse stad over haar beroerd werd, en zij zeiden: Is dit Naomi?
\par 20 Maar zij zeide tot henlieden: Noemt mij niet Naomi, noemt mij Mara; want de Almachtige heeft mij grote bitterheid aangedaan.
\par 21 Vol toog ik weg, maar ledig heeft mij de HEERE doen wederkeren; waarom zoudt gij mij Naomi noemen, daar de HEERE tegen mij getuigt, en de Almachtige mij kwaad aangedaan heeft?
\par 22 Alzo kwam Naomi weder, en Ruth, de Moabietische, haar schoondochter, met haar, die uit de velden Moabs wederkwam; en zij kwamen te Bethlehem in het begin van den gersteoogst.

\chapter{2}

\par 1 Naomi nu had een bloedvriend van haar man, een man, geweldig van vermogen, van het geslacht van Elimelech; en zijn naam was Boaz.
\par 2 En Ruth, de Moabietische, zeide tot Naomi: Laat mij toch in het veld gaan, en van de aren oplezen, achter dien, in wiens ogen ik genade zal vinden. En zij zeide tot haar: Ga heen, mijn dochter!
\par 3 Zo ging zij heen, en kwam en las op in het veld, achter de maaiers; en haar viel bij geval voor, een deel van het veld van Boaz, die van het geslacht van Elimelech was.
\par 4 En ziet, Boaz kwam van Bethlehem, en zeide tot de maaiers: De HEERE zij met ulieden! En zij zeiden tot hem: De HEERE zegene u!
\par 5 Daarna zeide Boaz tot zijn jongen, die over de maaiers gezet was: Wiens is deze jonge vrouw?
\par 6 En de jongen, die over de maaiers gezet was, antwoordde en zeide: Deze is de Moabietische jonge vrouw, die met Naomi wedergekomen is uit de velden Moabs;
\par 7 En zij heeft gezegd: Laat mij toch oplezen en aren bij de garven verzamelen, achter de maaiers; zo is zij gekomen en heeft gestaan van des morgens af tot nu toe; nu is haar te huis blijven weinig.
\par 8 Toen zeide Boaz tot Ruth: Hoort gij niet, mijn dochter? Ga niet, om in een ander veld op te lezen; ook zult gij van hier niet weggaan, maar hier zult gij u houden bij mijn maagden.
\par 9 Uw ogen zullen zijn op dit veld, dat zij maaien zullen, en gij zult achter haarlieden gaan; heb ik den jongens niet geboden, dat men u niet aanroere? Als u dorst, zo ga tot de vaten, en drink van hetgeen de jongens zullen geschept hebben.
\par 10 Toen viel zij op haar aangezicht, en boog zich ter aarde, en zij zeide tot hem: Waarom heb ik genade gevonden in uw ogen, dat gij mij kent, daar ik een vreemde ben?
\par 11 En Boaz antwoordde en zeide tot haar: Het is mij wel aangezegd alles, wat gij bij uw schoonmoeder gedaan hebt, na den dood uws mans, en hebt uw vader en uw moeder, en het land uwer geboorte verlaten, en zijt heengegaan tot een volk, dat gij van te voren niet kendet.
\par 12 De HEERE vergelde u uw daad en uw loon zij volkomen, van den HEERE, den God Israels, onder Wiens vleugelen gij gekomen zijt om toevlucht te nemen!
\par 13 En zij zeide: Laat mij genade vinden in uw ogen, mijn heer, dewijl gij mij getroost hebt, en dewijl gij naar het hart uwer dienstmaagd gesproken hebt, hoewel ik niet ben, gelijk een uwer dienstmaagden.
\par 14 Als het nu etenstijd was, zeide Boaz tot haar: Kom hier bij, en eet van het brood, en doop uw bete in den azijn. Zo zat zij neder aan de zijde van de maaiers, en hij langde haar geroost koren, en zij at, en werd verzadigd, en hield over.
\par 15 Als zij nu opstond, om op te lezen, zo gebood Boaz zijn jongens, zeggende: Laat haar ook tussen de garven oplezen, en beschaamt haar niet.
\par 16 Ja, laat ook allengskens van de handvollen voor haar wat vallen, en laat het liggen, dat zij het opleze, en bestraft haar niet.
\par 17 Alzo las zij op in dat veld, tot aan den avond; en zij sloeg uit, wat zij opgelezen had, en het was omtrent een efa gerst.
\par 18 En zij nam het op, en kwam in de stad; en haar schoonmoeder zag, wat zij opgelezen had; ook bracht zij voort, en gaf haar, wat zij van haar verzadiging overgehouden had.
\par 19 Toen zeide haar schoonmoeder tot haar: Waar hebt gij heden opgelezen, en waar hebt gij gewrocht? Gezegend zij, die u gekend heeft! En zij verhaalde haar schoonmoeder, bij wien zij gewrocht had, en zeide: De naam des mans, bij welken ik heden gewrocht heb, is Boaz.
\par 20 Toen zeide Naomi tot haar schoondochter: Gezegend zij hij den HEERE, Die Zijn weldadigheid niet heeft nagelaten aan de levenden en aan de doden! Voorts zeide Naomi tot haar: Die man is ons nabestaande; hij is een van onze lossers.
\par 21 En Ruth, de Moabietische, zeide: Ook, omdat hij tot mij gezegd heeft: Gij zult u houden bij de jongens, die ik heb, totdat zij den gansen oogst, die ik heb, zullen hebben voleindigd.
\par 22 En Naomi zeide tot haar schoondochter Ruth: Het is goed, mijn dochter, dat gij met zijn maagden uitgaat, opdat zij u niet tegenvallen in een ander veld.
\par 23 Alzo hield zij zich bij de maagden van Boaz, om op te lezen, totdat de gersteoogst en tarweoogst voleindigd waren; en zij bleef bij haar schoonmoeder.

\chapter{3}

\par 1 En Naomi, haar schoonmoeder, zeide tot haar: Mijn dochter! zoude ik u geen rust zoeken, dat het u welga?
\par 2 Nu dan, is niet Boaz, met wiens maagden gij geweest zijt, van onze bloedvriendschap? Zie, hij zal dezen nacht gerst op den dorsvloer wannen.
\par 3 Zo baad u, en zalf u, en doe uw klederen aan, en ga af naar den dorsvloer; maar maak u den man niet bekend, totdat hij geeindigd zal hebben te eten en te drinken.
\par 4 En het zal geschieden, als hij nederligt, dat gij de plaats zult merken, waar hij zal nedergelegen zijn; ga dan in, en sla zijn voetdeksel op, en leg u; zo zal hij u te kennen geven, wat gij doen zult.
\par 5 En zij zeide tot haar: Al wat gij tot mij zegt, zal ik doen.
\par 6 Alzo ging zij af naar den dorsvloer, en deed naar alles, wat haar schoonmoeder haar geboden had.
\par 7 Als nu Boaz gegeten en gedronken had, en zijn hart vrolijk was, zo kwam hij om neder te liggen aan het uiterste van een koren hoop. Daarna kwam zij stilletjes in, en sloeg zijn voetdeksel op, en leide zich.
\par 8 En het geschiedde te middernacht, dat die man verschrikte, en om zich greep; en ziet, een vrouw lag aan zijn voetdeksel.
\par 9 En hij zeide: Wie zijt gij? En zij zeide: Ik ben Ruth, uw dienstmaagd, breid dan uw vleugel uit over uw dienstmaagd, want gij zijt de losser.
\par 10 En hij zeide: Gezegend zijt gij den HEERE, mijn dochter! Gij hebt deze uw laatste weldadigheid beter gemaakt dan de eerste, dewijl gij geen jonge gezellen zijt nagegaan, hetzij arm of rijk.
\par 11 En nu, mijn dochter, vrees niet; al wat gij gezegd hebt, zal ik u doen; want de ganse stad mijns volks weet, dat gij een deugdelijke vrouw zijt.
\par 12 Nu dan, wel is waar, dat ik een losser ben; maar er is nog een losser, nader dan ik.
\par 13 Blijf dezen nacht over; voorts in den morgen zal het geschieden, indien hij u lost, goed, laat hem lossen; maar indien het hem niet lust u te lossen, zo zal ik u lossen, zo waarachtig als de HEERE leeft; leg u neder tot den morgen toe.
\par 14 Alzo lag zij neder aan zijn voetdeksel tot den morgen toe; en zij stond op, eer dat de een den ander kennen kon; want hij zeide: Het worde niet bekend, dat een vrouw op den dorsvloer gekomen is.
\par 15 Voorts zeide hij: Lang den sluier, die op u is, en houd dien; en zij hield hem; en hij mat zes maten gerst, en leide ze op haar; daarna ging hij in de stad.
\par 16 Zij nu kwam tot haar schoonmoeder, dewelke zeide: Wie zijt gij, mijn dochter? En zij verhaalde haar alles, wat die man haar gedaan had.
\par 17 Ook zeide zij: Deze zes maten gerst heeft hij mij gegeven; want hij zeide tot mij: Kom niet ledig tot uw schoonmoeder.
\par 18 Toen zeide zij: Zit stil, mijn dochter, totdat gij weet, hoe de zaak zal vallen; want die man zal niet rusten, tenzij dat hij heden deze zaak voleind hebbe.

\chapter{4}

\par 1 En Boaz ging op in de poort, en zette zich aldaar en ziet, de losser, van welken Boaz gesproken had, ging voorbij; zo zeide hij: Wijk herwaarts, zet u hier, gij, zulk een! En hij week derwaarts, en zette zich.
\par 2 En hij nam tien mannen van de oudsten der stad, en zeide: Zet u hier; en zij zetten zich.
\par 3 Toen zeide hij tot dien losser: Het stuk lands, dat van onzen broeder Elimelech was, heeft Naomi, die uit der Moabieten land wedergekomen is, verkocht;
\par 4 En ik heb gezegd: Ik zal het voor uw oor openbaren, zeggende: Aanvaard het in tegenwoordigheid der inwoners, en in tegenwoordigheid der oudsten mijns volks; zo gij het zult lossen, los het; en zo men het ook niet zou lossen, verklaar het mij, dat ik het wete; want er is niemand, behalve gij, die het losse, en ik na u. Toen zeide hij: Ik zal het lossen.
\par 5 Maar Boaz zeide: Ten dage, als gij het land aanvaardt van de hand van Naomi, zo zult gij het ook aanvaarden van Ruth, de Moabietische, de huisvrouw des verstorvenen, om den naam des verstorvenen te verwekken over zijn erfdeel.
\par 6 Toen zeide die losser: Ik zal het voor mij niet kunnen lossen, opdat ik mijn erfdeel niet misschien verderve; los gij mijn lossing voor u; want ik zal niet kunnen lossen.
\par 7 Nu was dit van ouds een gewoonheid in Israel, bij de lossing en bij de verwisseling, om de ganse zaak te bevestigen, zo trok de man zijn schoen uit en gaf die aan zijn naaste; en dit was tot een getuigenis in Israel.
\par 8 Zo zeide de losser tot Boaz: Aanvaard gij het voor u; en hij trok zijn schoen uit.
\par 9 Toen zeide Boaz tot de oudsten en al het volk: Gijlieden zijt heden getuigen, dat ik aanvaard heb alles, wat van Elimelech geweest is, en alles, wat van Chiljon en Machlon geweest is, van de hand van Naomi.
\par 10 Daartoe aanvaard ik mij ook Ruth, de Moabietische, de huisvrouw van Machlon, tot een vrouw, om den naam des verstorvenen over zijn erfdeel te verwekken, opdat de naam des verstorvenen niet worde uitgeroeid van zijn broederen, en van de poort zijner plaats; gijlieden zijt heden getuigen.
\par 11 En al het volk, dat in de poort was, mitsgaders de oudsten zeiden: Wij zijn getuigen; de HEERE make deze vrouw, die in uw huis komt, als Rachel en als Lea, die beiden het huis van Israel gebouwd hebben; en handel kloekelijk in Efratha, en maak uw naam vermaard in Bethlehem!
\par 12 En uw huis zij, als het huis van Perez (die Thamar aan Juda baarde), van het zaad, dat de HEERE u geven zal uit deze jonge vrouw.
\par 13 Alzo nam Boaz Ruth, en zij werd hem ter vrouwe, en hij ging tot haar in; en de HEERE gaf haar, dat zij zwanger werd en een zoon baarde.
\par 14 Toen zeiden de vrouwen tot Naomi: Geloofd zij de HEERE, Die niet heeft nagelaten u heden een losser te geven; en zijn naam worde vermaard in Israel!
\par 15 Die zal u zijn tot een verkwikker der ziel, en om uw ouderdom te onderhouden; want uw schoondochter, die u liefheeft, heeft hem gebaard, dewelke u beter is dan zeven zonen.
\par 16 En Naomi nam dat kind, en zette het op haar schoot, en werd zijn voedster.
\par 17 En de naburinnen gaven hem een naam, zeggende: Aan Naomi is een zoon geboren; en zij noemden zijn naam Obed; deze is de vader van Isai, Davids vader.
\par 18 Dit nu zijn de geboorten van Perez: Perez gewon Hezron;
\par 19 En Hezron gewon Ram; en Ram gewon Amminadab;
\par 20 En Amminadab gewon Nahesson; en Nahesson gewon Salma;
\par 21 En Salmon gewon Boaz, en Boaz gewon Obed;
\par 22 En Obed gewon Isai; en Isai gewon David.



\end{document}