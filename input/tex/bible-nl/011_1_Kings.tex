\begin{document}

\title{1 Koningen}



\chapter{1}

\par 1 De koning David nu was oud, wel bedaagd; en zij dekten hem met klederen, doch hij kreeg gene warmte.
\par 2 Toen zeiden zijn knechten tot hem: Laat ze mijn heer den koning een jonge dochter, een maagd zoeken, die voor het aangezicht des konings sta, en hem koestere; en zij slape in uw schoot, dat mijn heer de koning warm worde.
\par 3 Zo zochten zij een schone jonge dochter in alle landpalen van Israel; en zij vonden Abisag, een Sunamietische, en brachten ze tot den koning.
\par 4 En de jonge dochter was bovenmate schoon, en koesterde den koning, en diende hem; doch den koning bekende ze niet.
\par 5 Adonia nu, de zoon van Haggith, verhief zich, zeggende: Ik zal koning zijn; en hij bereidde zich wagenen en ruiteren, en vijftig mannen, lopende voor zijn aangezicht.
\par 6 En zijn vader had hem niet bedroefd van zijn dagen, zeggende: Waarom hebt gij alzo gedaan? En ook was hij zeer schoon van gedaante, en Haggith had hem gebaard na Absalom.
\par 7 En zijn raadslagen waren met Joab, den zoon van Zeruja, en met Abjathar, den priester; die hielpen, volgende Adonia.
\par 8 Maar Zadok, de priester, en Benaja, de zoon van Jojada, en Nathan, de profeet, en Simei, en Rei, en de helden, die David had, waren met Adonia niet.
\par 9 En Adonia slachtte schapen en runderen, en gemest vee bij den steen Zoheleth, die bij de fontein Rogel is; en noodde al zijn broederen, de zonen des konings, en alle mannen van Juda, des konings knechten.
\par 10 Maar Nathan, den profeet, en Benaja, en de helden, en Salomo, zijn broeder, noodde hij niet.
\par 11 Toen sprak Nathan tot Bathseba, de moeder van Salomo, zeggende: Hebt gij niet gehoord, dat Adonia, de zoon van Haggith, koning is? En onze heer David weet dat niet.
\par 12 Nu dan, kom, laat mij u toch een raad geven, dat gij uw ziel en de ziel van uw zoon Salomo redt.
\par 13 Ga heen, en treed in tot den koning David, en zeg tot hem: Hebt gij niet, mijn heer koning, uw dienstmaagd gezworen, zeggende: Voorzeker, uw zoon Salomo zal na mij koning zijn, en hij zal op mijn troon zitten! Waarom dan is Adonia koning?
\par 14 Zie, als gij daar nog met den koning spreken zult, zo zal ik na u inkomen, en zal uw woorden vervullen.
\par 15 En Bathseba ging in tot den koning in de binnenkamer; doch de koning was zeer oud, en Abisag, de Sunamietische, diende den koning.
\par 16 En Bathseba neigde het hoofd en boog zich neder voor den koning; en de koning zeide: Wat is u?
\par 17 En zij zeide tot hem: Mijn heer! gij hebt uw dienstmaagd bij den HEERE, uw God, gezworen: Voorzeker Salomo, uw zoon, zal na mij koning zijn, en hij zal op mijn troon zitten!
\par 18 En nu zie, Adonia is koning; en nu, mijn heer koning, gij weet het niet.
\par 19 En hij heeft ossen, en gemest vee, en schapen in menigte geslacht, en genood al de zonen des konings, en Abjathar, den priester, en Joab, den krijgsoverste, maar uw knecht Salomo heeft hij niet genood.
\par 20 Maar gij, mijn heer koning, de ogen van het ganse Israel zijn op u, dat gij hun zoudt te kennen geven, wie op den troon van mijn heer den koning na hem zitten zal.
\par 21 Anders zal het geschieden, als mijn heer de koning met zijn vaderen zal ontslapen zijn, dat ik en mijn zoon Salomo als zondaars zullen zijn.
\par 22 En ziet, zij sprak nog met den koning, als de profeet Nathan inkwam.
\par 23 En zij gaven den koning te kennen, zeggende: Zie, de profeet Nathan is daar; en hij kwam voor het aangezicht des konings, en boog zich voor den koning op zijn aangezicht ter aarde.
\par 24 En Nathan zeide: Mijn heer koning! hebt gij gezegd: Adonia zal na mij koning zijn, en hij zal op mijn troon zitten?
\par 25 Want hij is heden afgegaan, en heeft geslacht ossen, en gemest vee, en schapen in menigte, en heeft genood al de zonen des konings, en de oversten des heirs, en Abjathar, den priester; en zie, zij eten, en drinken voor zijn aangezicht, en zeggen: De koning Adonia leve!
\par 26 Maar mij, die uw knecht ben, en Zadok, den priester, en Benaja, den zoon van Jojada, en Salomo, uw knecht, heeft hij niet genood.
\par 27 Is deze zaak van mijn heer den koning geschied? En hebt gij uw knecht niet bekend gemaakt, wie op den troon van mijn heer den koning na hem zitten zou?
\par 28 En de koning David antwoordde en zeide: Roept mij Bathseba; en zij kwam voor het aangezicht des konings, en stond voor het aangezicht des konings.
\par 29 Toen zwoer de koning, en zeide: Zo waarachtig als de HEERE leeft, die mijn ziel uit allen nood verlost heeft;
\par 30 Voorzeker, gelijk als ik u gezworen heb bij den HEERE, den God Israels, zeggende: Voorzeker zal uw zoon Salomo na mij koning zijn, en zal op mijn troon in mijn plaats zitten; voorzeker, alzo zal ik te dezen zelfden dage doen.
\par 31 Toen neigde zich Bathseba met het aangezicht ter aarde, en boog zich neder voor den koning, en zeide: Mijn heer de koning David leve in eeuwigheid!
\par 32 En de koning David zeide: Roep mij Zadok, den priester, en Nathan, den profeet, en Benaja, den zoon van Jojada; en zij kwamen voor het aangezicht des konings.
\par 33 En de koning zeide tot hen: Neemt met u de knechten uws heren, en doet mijn zoon Salomo rijden op de muilezelin, die voor mij is; en voert hem af naar Gihon.
\par 34 En dat Zadok, de priester, met Nathan, den profeet, hem aldaar tot koning over Israel zalven. Daarna zult gij met de bazuin blazen, en zeggen: De koning Salomo leve!
\par 35 Dan zult gij achter hem optrekken, en hij zal komen, en zal op mijn troon zitten, en hij zal koning zijn in mijn plaats; want ik heb geboden, dat hij een voorganger zou zijn over Israel en over Juda.
\par 36 Toen antwoordde Benaja, de zoon van Jojada, den koning, en zeide: Amen; alzo zegge de HEERE, de God van mijn heer den koning!
\par 37 Gelijk als de HEERE met mijn heer den koning geweest is, alzo zij Hij met Salomo; en Hij make zijn troon groter dan den troon van mijn heer den koning David!
\par 38 Toen ging Zadok, de priester, af, met Nathan, den profeet, en Benaja, den zoon van Jojada, en de Krethi en de Plethi, en zij deden Salomo rijden op de muilezelin van den koning David, en geleidden hem naar Gihon.
\par 39 En Zadok, de priester, nam den oliehoorn uit de tent, en zalfde Salomo; en zij bliezen met de bazuin, en al het volk zeide: De koning Salomo leve!
\par 40 En al het volk kwam op achter hem, en het volk pijpte met pijpen, en verblijdde zich met grote blijdschap, zodat de aarde van hun geluid spleet.
\par 41 En Adonia hoorde het, en al de genoden, die met hem waren, die nu geeindigd hadden te eten; ook hoorde Joab het geluid der bazuinen, en zeide: Waarom is het geroep dier stad, die in roer is?
\par 42 Als hij nog sprak, ziet, zo kwam Jonathan, de zoon van Abjathar, den priester; en Adonia zeide: Kom in, want gij zijt een kloek man, en zult het goede boodschappen.
\par 43 En Jonathan antwoordde en zeide tot Adonia: Ja, maar onze heer, de koning David, heeft Salomo tot koning gemaakt.
\par 44 En de koning heeft met hem gezonden Zadok, den priester, en Nathan, den profeet, en Benaja, den zoon van Jojada, en de Krethi en de Plethi; en zij hebben hem doen rijden op de muilezelin des konings.
\par 45 Daartoe hebben hem Zadok, de priester, en Nathan, de profeet, in Gihon tot koning gezalfd, en zijn van daar blijde opgetogen, zodat de stad in roer is; dat is het geroep, dat gij gehoord hebt.
\par 46 En ook zit Salomo op den troon des koninkrijks.
\par 47 Zo zijn ook de knechten des konings gekomen, om onzen heer, den koning David, te zegenen, zeggende: Uw God make den naam van Salomo beter dan uw naam, en make zijn troon groter dan uw troon; en de koning heeft aangebeden op de slaapstede.
\par 48 Ja, ook heeft de koning aldus gezegd: Geloofd zij de HEERE, de God Israels, Die heden gegeven heeft een, zittende op mijn troon, dat het mijn ogen gezien hebben!
\par 49 Toen verschrikten en stonden op al de genoden, die bij Adonia waren, en gingen een iegelijk zijns weegs.
\par 50 Doch Adonia vreesde voor Salomo, en hij stond op, en ging heen, en vatte de hoornen des altaars.
\par 51 En men maakte Salomo bekend, zeggende: Zie, Adonia vreest den koning Salomo, want zie, hij heeft de hoornen des altaars gevat, zeggende: Dat de koning Salomo mij als heden zwere, dat hij zijn knecht met het zwaard niet doden zal!
\par 52 En Salomo zeide: Indien hij een vroom man zal zijn, daar zal niet van zijn haar op de aarde vallen; maar indien in hem kwaad bevonden zal worden, zo zal hij sterven.
\par 53 En de koning Salomo zond heen, en zij deden hem afgaan van het altaar; en hij kwam, en boog zich neder voor den koning Salomo. En Salomo zeide tot hem: Ga heen naar uw huis.

\chapter{2}

\par 1 Als nu de dagen van David nabij waren, dat hij sterven zou, zo gebood hij zijn zoon Salomo, zeggende:
\par 2 Ik ga heen in den weg der ganse aarde, zo wees sterk, en wees een man.
\par 3 En neem waar de wacht des HEEREN, uws Gods, om te wandelen in Zijn wegen, om te onderhouden Zijn inzettingen, en Zijn geboden, en Zijn rechten, en Zijn getuigenissen, gelijk geschreven is in de wet van Mozes; opdat gij verstandelijk handelt in al wat gij doen zult, en al waarheen gij u wenden zult;
\par 4 Opdat de HEERE bevestige Zijn woord, dat Hij over mij gesproken heeft, zeggende: Indien uw zonen hun weg bewaren, om voor Mijn aangezicht trouwelijk, met hun ganse hart en met hun ganse ziel te wandelen, zo zal geen man, zeide Hij, u afgesneden worden van den troon Israels.
\par 5 Zo weet gij ook, wat Joab, de zoon van Zeruja, mij gedaan heeft, en wat hij gedaan heeft aan de twee krijgsoversten van Israel, Abner, den zoon van Ner, en Amasa, den zoon van Jether, dien hij gedood heeft, en heeft krijgsbloed vergoten in vrede; en hij heeft krijgsbloed gedaan aan zijn gordel, die aan zijn lendenen was, en aan zijn schoenen, die aan zijn voeten waren.
\par 6 Doe dan naar uw wijsheid, dat gij zijn grauwe haar niet met vrede in het graf laat dalen.
\par 7 Maar aan de zonen van Barzillai, den Gileadiet, zult gij weldadigheid bewijzen, en zij zullen zijn onder degenen, die aan uw tafel eten; want alzo naderden zij tot mij, als ik vluchtte voor het aangezicht van uw broeder Absalom.
\par 8 En zie, bij u is Simei, de zoon van Gera, de zoon van Jemini, uit Bahurim, die mij vloekte met een geweldige vloek, ten dage als ik ging naar Mahanaim; doch hij kwam af mij tegemoet aan de Jordaan, en ik zwoer hem bij den HEERE, zeggende: Zo ik hem met het zwaard dode!
\par 9 Maar nu, houd hem niet onschuldig, dewijl gij een wijs man zijt; en gij zult weten, wat gij hem doen zult, opdat gij zijn grauwe haar met bloed in het graf doet dalen.
\par 10 En David ontsliep met zijn vaderen, en werd begraven in de stad Davids.
\par 11 De dagen nu, die David geregeerd heeft over Israel, zijn veertig jaren; zeven jaren heeft hij geregeerd in Hebron, en in Jeruzalem heeft hij drie en dertig jaren geregeerd.
\par 12 En Salomo zat op den troon van zijn vader David; en zijn koninkrijk werd zeer bevestigd.
\par 13 Toen kwam Adonia, de zoon van Haggith, tot Bathseba, de moeder van Salomo; en zij zeide: Is uw komst vrede? En hij zeide: Vrede.
\par 14 Daarna zeide hij: Ik heb een woord aan u. En zij zeide: Spreek.
\par 15 Hij zeide dan: Gij weet, dat het koninkrijk mijn was, en het ganse Israel zijn aangezicht op mij gezet had, dat ik koning zijn zou; hoewel het koninkrijk omgewend en mijns broeders geworden is; want het is van den HEERE hem geworden.
\par 16 En nu begeer ik van u een enige begeerte; wijs mijn aangezicht niet af. En zij zeide tot hem: Spreek.
\par 17 En hij zeide: Spreek toch tot den koning Salomo, want hij zal uw aangezicht niet afwijzen, dat hij mij Abisag, de Sunamietische, ter vrouwe geve.
\par 18 En Bathseba zeide: Het is goed, ik zal den koning voor u aanspreken.
\par 19 Zo kwam Bathseba tot den koning Salomo, om hem voor Adonia aan te spreken. En de koning stond op, haar tegemoet, en boog zich voor haar; daarna zat hij op zijn troon, en deed een stoel voor de moeder des konings zetten; en zij zat aan zijn rechterhand.
\par 20 Toen zeide zij: Ik begeer van u een enige kleine begeerte, wijs mijn aangezicht niet af. En de koning zeide tot haar: Begeer, mijn moeder, want ik zal uw aangezicht niet afwijzen.
\par 21 En zij zeide: Laat Abisag, de Sunamietische, aan Adonia, uw broeder, ter vrouwe gegeven worden.
\par 22 Toen antwoordde de koning Salomo, en zeide tot zijn moeder: En waarom begeert gij Abisag, de Sunamietische, voor Adonia? Begeer ook voor hem het koninkrijk (want hij is mijn broeder, die ouder is dan ik ben), ja, voor hem, en voor Abjathar, den priester, en voor Joab, den zoon van Zeruja.
\par 23 En de koning Salomo zwoer bij den HEERE, zeggende: Zo doe mij God, en zo doe Hij daartoe, voorzeker Adonia zal dat woord tegen zijn leven gesproken hebben!
\par 24 En nu, zo waarachtig als de HEERE leeft, Die mij bevestigd heeft, en mij heeft doen zitten op den troon van mijn vader David, en Die mij een huis gemaakt heeft, gelijk als Hij gesproken had; voorzeker, Adonia zal heden gedood worden!
\par 25 En de koning Salomo zond door de hand van Benaja, den zoon van Jojada; die viel op hem aan, dat hij stierf.
\par 26 En tot Abjathar, den priester, zeide de koning: Ga naar Anathoth, op uw akkers; want gij zijt een man des doods; maar op dezen dag zal ik u niet doden, omdat gij de ark des Heeren HEEREN voor het aangezicht van mijn vader David gedragen hebt, en omdat gij verdrukt zijt geweest, in alles, waarin mijn vader verdrukt was.
\par 27 Salomo dan verdreef Abjathar, dat hij des HEEREN priester niet ware, om te vervullen het woord des HEEREN, hetwelk Hij over het huis van Eli te Silo gesproken had.
\par 28 Als het gerucht tot Joab kwam (want Joab had zich gewend achter Adonia, hoewel hij zich niet had gewend achter Absalom), zo vluchtte Joab tot de tent des HEEREN, en vatte de hoornen des altaars.
\par 29 En het werd den koning Salomo aangezegd, dat Joab tot de tent des HEEREN gevloden was, en zie, hij is bij het altaar. Toen zond Salomo Benaja, den zoon van Jojada, zeggende: Ga heen, val op hem aan.
\par 30 En Benaja kwam tot de tent des HEEREN, en zeide tot hem: Zo zegt de koning: Kom uit. En hij zeide: Neen, maar hier zal ik sterven! En Benaja bracht het antwoord weder aan den koning, zeggende: Zo heeft Joab gesproken, en zo heeft hij mij geantwoord.
\par 31 En de koning zeide tot hem: Doe gelijk als hij gesproken heeft, en val op hem aan, en begraaf hem, opdat gij wegdoet, van mij en van mijns vaders huis, dat bloed, dat Joab zonder oorzaak vergoten heeft.
\par 32 Zo zal de HEERE zijn bloed op zijn hoofd doen wederkeren, omdat hij op twee mannen, rechtvaardiger en beter dan hij, aangevallen is, en die met het zwaard gedood heeft, daar het mijn vader David niet wist, Abner, den zoon van Ner, den krijgsoverste van Israel, en Amasa, den zoon van Jether, den krijgsoverste van Juda.
\par 33 Alzo zal hun bloed wederkeren op het hoofd van Joab, en op het hoofd van zijn zaad in eeuwigheid; maar David, en zijn zaad, en zijn huis, en zijn troon zal vrede hebben van den HEERE tot in eeuwigheid.
\par 34 En Benaja, de zoon van Jojada, ging op, en viel op hem aan, en doodde hem; en hij werd begraven in zijn huis, in de woestijn.
\par 35 En de koning zette Benaja, den zoon van Jojada, in zijn plaats over het heir; en Zadok, den priester, zette de koning in de plaats van Abjathar.
\par 36 Daarna zond de koning, en riep Simei, en zeide tot hem: Bouw u een huis in Jeruzalem, en woon aldaar; en ga van daar niet uit herwaarts of derwaarts.
\par 37 Want het zal geschieden ten dage van uw uitgaan, als gij over de beek Kidron zult gaan, weet voorzeker, dat gij den dood sterven zult; uw bloed zal op uw hoofd zijn.
\par 38 En Simei zeide tot den koning: Dat woord is goed; gelijk als mijn heer de koning gesproken heeft, alzo zal uw knecht doen. En Simei woonde te Jeruzalem vele dagen.
\par 39 Doch het geschiedde met het einde van drie jaren, dat twee knechten van Simei wegliepen tot Achis, den zoon van Maacha, den koning van Gath; en men gaf het Simei te kennen, zeggende: Zie, uw knechten zijn in Gath.
\par 40 Toen maakte zich Simei op, en zadelde zijn ezel, en toog heen naar Gath tot Achis, om zijn knechten te zoeken; zo toog Simei heen, en bracht zijn knechten van Gath.
\par 41 En het werd Salomo aangezegd, dat Simei uit Jeruzalem naar Gath getogen, en wedergekomen was.
\par 42 Toen zond de koning, en riep Simei, en zeide tot hem: Heb ik u niet beedigd bij den HEERE, en tegen u betuigd, zeggende: Ten dage van uw uitgaan, als gij zult herwaarts of derwaarts gaan, weet voorzeker, dat gij den dood zult sterven? En gij zeidet tot mij: Dat woord is goed, dat ik gehoord heb.
\par 43 Waarom dan hebt gij den eed des HEEREN niet gehouden, en het gebod, dat ik over u geboden had?
\par 44 Verder zeide de koning tot Simei: Gij weet al de boosheid, die uw hart weet, die gij aan mijn vader David gedaan hebt; daarom heeft de HEERE uw boosheid op uw hoofd doen wederkeren.
\par 45 Maar de koning Salomo is gezegend; en de troon van David zal bevestigd zijn voor het aangezicht des HEEREN tot in eeuwigheid.
\par 46 En de koning gebood Benaja, den zoon van Jojada; die ging uit, en viel op hem aan, dat hij stierf. Alzo is het koninkrijk bevestigd in de hand van Salomo.

\chapter{3}

\par 1 En Salomo verzwagerde zich met Farao, den koning van Egypte; en nam de dochter van Farao, en bracht ze in de stad Davids totdat hij voleind zou hebben het bouwen van zijn huis en het huis des HEEREN, en den muur van Jeruzalem rondom.
\par 2 Alleenlijk offerde het volk op de hoogten, want geen huis was den Naam des HEEREN gebouwd, tot die dagen toe.
\par 3 En Salomo had den HEERE lief, wandelende in de inzettingen van zijn vader David; alleenlijk offerde hij en rookte op de hoogten.
\par 4 En de koning ging naar Gibeon, om aldaar te offeren, omdat die hoogte groot was; duizend brandofferen offerde Salomo op dat altaar.
\par 5 Te Gibeon verscheen de HEERE aan Salomo in een droom des nachts en God zeide: Begeer wat Ik u geven zal.
\par 6 En Salomo zeide: Gij hebt aan Uw knecht David, mijn vader, grote weldadigheid gedaan, gelijk als hij voor Uw aangezicht gewandeld heeft, in waarheid, en in gerechtigheid, en in oprechtheid des harten met U; en Gij hebt hem deze grote weldadigheid gehouden, dat Gij hem gegeven hebt een zoon, zittende op zijn troon, als te dezen dage.
\par 7 Nu dan, HEERE, mijn God! Gij hebt Uw knecht koning gemaakt in de plaats van mijn vader David; en ik ben een klein jongeling, ik weet niet uit te gaan noch in te gaan.
\par 8 En Uw knecht is in het midden van Uw volk, dat Gij verkoren hebt, een groot volk, hetwelk niet kan geteld noch gerekend worden, vanwege de menigte.
\par 9 Geef dan Uw knecht een verstandig hart, om Uw volk te richten, verstandelijk onderscheidende tussen goed en kwaad; want wie zou dit Uw zwaar volk kunnen richten?
\par 10 Die zaak nu was goed in de ogen des HEEREN, dat Salomo deze zaak begeerd had.
\par 11 En God zeide tot hem: Daarom dat gij deze zaak begeerd hebt, en niet begeerd hebt, voor u vele dagen, noch voor u begeerd hebt rijkdom, noch begeerd hebt de ziel uwer vijanden; maar hebt begeerd verstand voor u, om gerichtszaken te horen;
\par 12 Zie, Ik heb gedaan naar uw woorden; zie, Ik heb u een wijs en verstandig hart gegeven, dat uws gelijke voor u niet geweest is, en uws gelijke na u niet opstaan zal.
\par 13 Zelfs ook wat gij niet begeerd hebt, heb Ik u gegeven, beide rijkdom en eer; dat uws gelijke niemand onder de koningen al uw dagen zijn zal.
\par 14 En zo gij in Mijn wegen wandelen zult, onderhoudende Mijn inzettingen en Mijn geboden, gelijk als uw vader David gewandeld heeft, zo zal Ik ook uw dagen verlengen.
\par 15 En Salomo waakte op, en ziet, het was een droom. En hij kwam te Jeruzalem, en stond voor de ark des verbonds des HEEREN, en offerde brandofferen, en bereidde dankofferen, en maakte een maaltijd voor al zijn knechten.
\par 16 Toen kwamen er twee vrouwen, die hoeren waren, tot den koning; en zij stonden voor zijn aangezicht.
\par 17 En de ene vrouw zeide: Och, mijn heer. Ik en deze vrouw wonen in een huis; en ik heb bij haar in dat huis gebaard.
\par 18 Het is nu geschied op den derden dag na mijn baren dat deze vrouw ook gebaard heeft; en wij waren te zamen, geen vreemde was met ons in dat huis, behalve ons tweeen in het huis.
\par 19 En de zoon dezer vrouw is des nachts gestorven, omdat zij op hem gelegen had.
\par 20 En zij stond ter middernacht op, en nam mijn zoon van bij mij, als uw dienstmaagd sliep, en leide hem in haar schoot, en haar doden zoon leide zij in mijn schoot.
\par 21 En ik stond in de morgen op, om mijn zoon te zogen, en zie, hij was dood; maar ik lette in den morgen op hem, en zie, het was mijn zoon niet, dien ik gebaard had.
\par 22 Toen zeide de andere vrouw: Neen, maar die levende is mijn zoon, en de dode is uw zoon; gene daarentegen zeide: Neen, maar de dode is uw zoon, en de levende is mijn zoon! Alzo spraken zij voor het aangezicht des konings.
\par 23 Toen zeide de koning: Deze zegt: Dit is mijn zoon, die leeft, maar uw zoon is het, die dood is; en die zegt: Neen, maar de dode is uw zoon, en de levende mijn zoon.
\par 24 Verder zeide de koning: Haalt mij een zwaard; en zij brachten een zwaard voor het aangezicht des konings.
\par 25 En de koning zeide: Doorsnijdt dat levende kind in tweeen, en geeft de ene een helft, en de andere een helft.
\par 26 Maar de vrouw, welker zoon de levende was, sprak tot den koning (want haar ingewand ontstak over haar zoon), en zeide: Och, mijn heer! Geef haar dat levende kind, en dood het geenszins; deze daarentegen zeide: Het zij noch het uwe noch het mijne, doorsnijdt het.
\par 27 Toen antwoordde de koning, en zeide: Geeft aan die het levende kind, en doodt het geenszins; die is zijn moeder.
\par 28 En geheel Israel hoorde dat oordeel, dat de koning geoordeeld had, en vreesde voor het aangezicht des konings; want zij zagen, dat de wijsheid Gods in hem was, om recht te doen.

\chapter{4}

\par 1 Alzo was de koning Salomo koning over gans Israel.
\par 2 En deze waren de vorsten, die hij had: Azaria, de zoon van Zadok, was opperambtman.
\par 3 Elihoref, en Ahia, de zoon van Sisa, waren schrijvers; Josafat, de zoon van Ahilud, was kanselier.
\par 4 En Benaja, de zoon van Jojada, was over het heir; en Zadok en Abjathar waren priesters.
\par 5 En Azaria, de zoon van Nathan, was over de bestelmeesters; en Zabud, de zoon van Nathan, was overambtman, des konings vriend.
\par 6 En Ahisar was hofmeester; en Adoniram, de zoon van Abda, was over de schatting.
\par 7 En Salomo had twaalf bestelmeesters over gans Israel, die den koning en zijn huis verzorgden; voor elk was een maand in het jaar om te verzorgen.
\par 8 En dit zijn hun namen: de zoon van Hur was in het gebergte van Efraim.
\par 9 De zoon van Deker in Makaz, en in Saalbim, en Beth-semes, en Elon-beth-hanan.
\par 10 De zoon van Hesed in Arubboth; hij had daartoe Socho en het ganse land Hefer.
\par 11 De zoon van Abinadab had de ganse landstreek van Dor; deze had Tafath, de dochter van Salomo, tot een vrouw.
\par 12 Baana, de zoon van Ahilud, had Taanach, en Megiddo, en het ganse Beth-sean, hetwelk is bij Zartana, beneden van Jizreel, van Beth-sean aan tot Abel-mehola, tot op gene zijde van Jokmeam.
\par 13 De zoon van Geber was te Ramoth in Gilead; hij had de dorpen van Jair, den zoon van Manasse, die in Gilead zijn; ook had hij de streek van Argob, welke is in Basan, zestig grote steden, met muren en koperen grendelen.
\par 14 Abinadab, de zoon van Iddo, was te Mahanaim.
\par 15 Ahimaaz was in Nafthali; deze nam ook Salomo's dochter, Basmath, ter vrouwe.
\par 16 Baana, de zoon van Husai, was in Aser en in Aloth.
\par 17 Josafath, de zoon van Paruah, in Issaschar.
\par 18 Simei, de zoon van Ela, in Benjamin.
\par 19 Geber, de zoon van Uri, was in het land Gilead, het land van Sihon, den koning der Amorieten, en van Og, den koning van Basan, en hij was de enige bestelmeester, die in dat land was.
\par 20 Juda nu en Israel waren velen, als zand, dat aan de zee is in menigte, etende, en drinkende, en blijde zijnde.
\par 21 En Salomo was heersende over al de koninkrijken, van de rivier tot het land der Filistijnen, en tot aan de landpale van Egypte; die brachten geschenken, en dienden Salomo al de dagen zijns levens.
\par 22 De spijze nu van Salomo was voor een dag, dertig kor meelbloem, en zestig kor meel;
\par 23 Tien vette runderen, en twintig weiderunderen, en honderd schapen; uitgenomen de herten, en reeen, en buffelen en gemeste vogelen.
\par 24 Want hij had heerschappij over al wat op deze zijde der rivier was van Thifsah tot aan Gaza, over alle koningen op deze zijde der rivier; en hij had vrede van al zijn zijden rondom.
\par 25 En Juda en Israel woonden zeker, een iegelijk onder zijn wijnstok en onder zijn vijgeboom, van Dan tot Ber-seba, al de dagen van Salomo.
\par 26 Salomo had ook veertig duizend paardenstallen tot zijn wagenen, en twaalf duizend ruiteren.
\par 27 Die bestelmeesters nu, een ieder op zijn maand, verzorgden den koning Salomo, en al degenen, die tot de tafel van den koning Salomo naderden; zij lieten geen ding ontbreken.
\par 28 De gerst nu en het stro voor de paarden, en voor de snelle kemelen, brachten zij aan de plaats, waar hij was, een iegelijk naar zijn last.
\par 29 En God gaf Salomo wijsheid en zeer veel verstand, en een wijd begrip des harten, gelijk zand, dat aan den oever der zee is.
\par 30 En de wijsheid van Salomo was groter dan de wijsheid van al die van het oosten, en dan alle wijsheid der Egyptenaren;
\par 31 Ja, hij was wijzer dan alle mensen; dan Ethan, de Ezrahiet, en Heman, en Chalcol, en Darda, de zonen van Mahol; en zijn naam was onder alle heidenen rondom.
\par 32 En hij sprak drie duizend spreuken; daartoe waren zijn liederen duizend en vijf.
\par 33 Hij sprak ook van de bomen, van den cederboom af, die op den Libanon is, tot op den hysop, die aan den wand uitwast; hij sprak ook van het vee, en van het gevogelte, en van de kruipende dieren, en van de vissen.
\par 34 En van alle volken kwamen er, om de wijsheid van Salomo te horen, van alle koningen der aarde, die van zijn wijsheid gehoord hadden.

\chapter{5}

\par 1 En Hiram, de koning van Tyrus, zond zijn knechten tot Salomo (want hij had gehoord, dat zij Salomo tot koning gezalfd hadden in zijns vaders plaats), dewijl Hiram David altijd bemind had.
\par 2 Daarna zond Salomo tot Hiram, zeggende:
\par 3 Gij weet, dat mijn vader David den Naam des HEEREN, zijns Gods, geen huis kon bouwen, vanwege de oorlogen, waarmede zij hem omsingelden, totdat de HEERE hen onder zijn voetzolen gaf.
\par 4 Maar nu heeft de HEERE, mijn God, mij van rondom rust gegeven; er is geen tegenpartijder, en geen bejegening van kwaad.
\par 5 En zie, ik denk voor den Naam van den HEERE, mijn God, een huis te bouwen; gelijk als de HEERE gesproken heeft tot mijn vader David, zeggende: Uw zoon, dien Ik in uw plaats op uw troon zetten zal, die zal Mijn Naam dat huis bouwen.
\par 6 Zo gebied nu, dat men mij cederen uit den Libanon houwe, en mijn knechten zullen met uw knechten zijn, en het loon uwer knechten zal ik u geven, naar al wat gij zeggen zult; want gij weet, dat onder ons niemand is, die weet hout te houwen, gelijk de Sidoniers.
\par 7 En het geschiedde, als Hiram de woorden van Salomo gehoord had, dat hij zich zeer verblijdde, en zeide: Gezegend zij de HEERE heden, Die David een wijzen zoon gegeven heeft over dit grote volk!
\par 8 En Hiram zond tot Salomo, zeggende: Ik heb gehoord, waarom gij tot mij gezonden hebt; ik zal al uw wil doen met het cederenhout, en met het dennenhout.
\par 9 Mijn knechten zullen het afbrengen van den Libanon aan de zee; en ik zal het op vlotten over de zee doen voeren, tot die plaats, die gij aan mij ontbieden zult, en zal het aldaar los maken, en gij zult het wegnemen; gij zult ook mijn wil doen, dat gij mijn huis spijze geeft.
\par 10 Alzo gaf Hiram aan Salomo cederenhout en dennenhout, naar al zijn wil.
\par 11 En Salomo gaf Hiram twintig duizend kor tarwe, tot spijze van zijn huis, en twintig kor gestoten olie; zulks gaf Salomo aan Hiram jaar op jaar.
\par 12 De HEERE dan gaf Salomo wijsheid, gelijk als Hij tot hem gesproken had; en er was vrede tussen Hiram en tussen Salomo, en zij beiden maakten een verbond.
\par 13 En de koning Salomo deed een uitschot opkomen uit gans Israel; en het uitschot was dertig duizend man.
\par 14 En hij zond hen naar den Libanon, tien duizend des maands bij beurten; een maand waren zij op den Libanon; twee maanden elk in zijn huis; en Adoniram was over dit uitschot.
\par 15 Daartoe had Salomo zeventig duizend, die last droegen, en tachtig duizend houwers op het gebergte.
\par 16 Behalve de oversten van Salomo's bestelden, die over dat werk waren, drie duizend en driehonderd, die heerschappij hadden over het volk, hetwelk dat werk deed.
\par 17 Als de koning het nu gebood, zo voerden zij grote stenen toe, kostelijke stenen, gehouwen stenen, om den grond van dat huis te leggen.
\par 18 En de bouwlieden van Salomo, en de bouwlieden van Hiram, en de Giblieten behieuwen ze, en bereidden het hout toe, en de stenen, om dat huis te bouwen.

\chapter{6}

\par 1 Het geschiedde nu in het vierhonderd en tachtigste jaar, na den uitgang der kinderen Israels uit Egypte, in het vierde jaar van het koninkrijk van Salomo over Israel, in de maand Ziv (deze is de tweede maand), dat hij het huis des HEEREN bouwde.
\par 2 En dat huis, hetwelk de koning Salomo den HEERE bouwde, was van zestig ellen in zijn lengte, en van twintig in zijn breedte, en van dertig ellen in zijn hoogte.
\par 3 En het voorhuis, vooraan den tempel van dat huis, was in zijn lengte van twintig ellen, naar de breedte van het huis, tien ellen in zijn breedte, vooraan het huis.
\par 4 En hij maakte vensteren aan het huis van gesloten uitzichten.
\par 5 En rondom aan den wand van het huis bouwde hij kameren, aan de wanden van het huis rondom, beide van den tempel en van de aanspraakplaats. Alzo maakte hij zijkameren rondom.
\par 6 De onderste kamer was van vijf ellen in haar breedte, en de middelste van zes ellen in haar breedte, en de derde van zeven ellen in haar breedte; want hij had aan het huis rondom buitenwaarts inkortingen gemaakt, opdat zij zich niet hielden in de wanden van het huis.
\par 7 Het huis nu, als het gebouwd werd, werd met volmaakten steen, zoals dezelve toegevoerd was, gebouwd; zodat geen hameren, noch bijl of enig ijzeren gereedschap gehoord werd in het huis, als het gebouwd werd.
\par 8 De deur der middelste zijkamer was aan de rechterzijde van het huis; en door wenteltrappen ging men tot de middelste zijkamer, en van de middelste tot de derde.
\par 9 Alzo bouwde hij het huis, en volmaakte het; en bedekte dat huis met gewelven en rijen van cederen.
\par 10 Hij bouwde ook de kameren aan het ganse huis, van vijf ellen in haar hoogte; en hij voegde ze vast aan dat huis met cederenhout.
\par 11 Toen geschiedde het woord des HEEREN tot Salomo, zeggende:
\par 12 Aangaande dit huis, dat gij bouwt, zo gij wandelt in Mijn inzettingen, en doet Mijn rechten, en onderhoudt al Mijn geboden, wandelende in dezelve; zo zal Ik Mijn woord met u bevestigen, dat Ik tot uw vader David gesproken heb;
\par 13 En Ik zal in het midden der kinderen Israels wonen; en Ik zal Mijn volk Israel niet verlaten.
\par 14 Alzo bouwde Salomo dat huis en volmaakte hetzelve.
\par 15 Ook bouwde hij de wanden van het huis van binnen met cederen planken; van den vloer des huizes tot aan het dak der wanden, beschoot hij ze van binnen met hout; en overdekte den vloer van het huis met dennen planken.
\par 16 Daartoe bouwde hij twintig ellen met cederen planken aan de zijden van het huis, van den vloer af tot de wanden; dit bouwde hij Hem van binnen tot een aanspraakplaats, tot het heilige der heiligen.
\par 17 Dat huis nu was van veertig ellen, namelijk de tempel, die vooraan was.
\par 18 En het ceder aan het huis inwendig was gesneden met knoppen en open bloemen; het was al ceder, geen steen werd gezien.
\par 19 En de aanspraakplaats bereidde hij inwaarts in het huis, om de ark des verbonds des HEEREN daar te zetten.
\par 20 En de aanspraakplaats vooraan was van twintig ellen in lengte, en van twintig ellen in breedte, en van twintig ellen in haar hoogte, en hij overtoog ze met gesloten goud; ook overtoog hij het cederen altaar.
\par 21 En Salomo overtoog het huis van binnen met gesloten goud; en hij toog voor de aanspraakplaats een voorhang henen door met gouden ketenen, en overtoog dien met goud.
\par 22 Alzo overtoog hij het ganse huis met goud, totdat het ganse huis volmaakt was; daartoe overtoog hij met goud het gehele altaar, dat voor de aanspraakplaats was.
\par 23 In de aanspraakplaats nu maakte hij twee cherubs van olieachtig hout; elks hoogte was tien ellen.
\par 24 En van vijf ellen was de ene vleugel des cherubs, en van vijf ellen de andere vleugel des cherubs; van het einde van zijn enen vleugel, tot aan het einde van zijn anderen vleugel, waren tien ellen.
\par 25 Alzo was de andere cherub van tien ellen; beide cherubs hadden enerlei maat, en enerlei snede.
\par 26 De hoogte van den enen cherub was van tien ellen, en alzo van den anderen cherub.
\par 27 En hij zette deze cherubs in het midden van het binnenste huis; en de cherubs spreidden de vleugelen uit, zodat de vleugel des enen raakte aan dezen wand, en de vleugel des anderen cherubs raakte aan den anderen wand; en hun vleugelen naar het midden van het huis raakten vleugel aan vleugel.
\par 28 En hij overtoog deze cherubs met goud.
\par 29 En al de wanden van het huis, in het ronde, graveerde hij met uitgesneden graveringen van cherubs, en van palmbomen, en open bloemen, van binnen en van buiten.
\par 30 Daartoe overtoog hij den vloer van het huis met goud van binnen en van buiten.
\par 31 En aan den ingang der aanspraakplaats maakte hij deuren van olieachtig hout; de bovendorpel met de posten was het vijfde deel des wands.
\par 32 De twee deuren ook waren van olieachtige bomen; en hij graveerde daarop graveringen van cherubs, en van palmbomen, en van open bloemen, dewelke hij met goud overtoog; ook trok hij goud over de cherubs en over de palmbomen.
\par 33 En alzo maakte hij aan de deuren des tempels posten van olieachtige bomen, uit het vierde deel van de wand.
\par 34 En de twee deuren waren van dennenhout; de twee zijden der ene deur waren omdraaiende; alzo waren de twee gegraveerde zijden der andere deur omdraaiende.
\par 35 En hij graveerde ze met cherubs, en palmbomen, en open bloemen, dewelke hij met goud overtoog, gericht naar het uitgesnedene.
\par 36 Daarna bouwde hij het binnenste voorhof van drie rijen gehouwen stenen, en een rij cederen balken.
\par 37 In het vierde jaar werd de grond van het huis des HEEREN gelegd, in de maand Ziv;
\par 38 En in het elfde jaar, in de maand Bul, welke is de achtste maand, was dit huis volmaakt, naar al zijn stukken en naar al zijn behoren; alzo heeft hij zeven jaren daaraan gebouwd.

\chapter{7}

\par 1 Maar aan zijn huis bouwde Salomo dertien jaren, en hij volmaakte zijn ganse huis.
\par 2 Hij bouwde ook het huis des wouds van Libanon, van honderd ellen in zijn lengte, en vijftig ellen in zijn breedte, en dertig ellen in zijn hoogte, op vier rijen van cederen pilaren, en cederen balken op de pilaren.
\par 3 En het was bedekt met ceder van boven op de ribben, die op vijf en veertig pilaren waren, vijftien in een rij.
\par 4 Er waren drie rijen van uitzichten, dat het ene venster was over het andere venster, in drie orden.
\par 5 Ook waren al de deuren en de posten vierkantig van enerlei uitzicht; en venster was tegenover venster, in drie orden.
\par 6 Daarna maakte hij een voorhuis van pilaren; vijftig ellen was zijn lengte, en dertig ellen zijn breedte; en het voorhuis was tegenover die, en de pilaren met de dikke balken tegenover dezelve.
\par 7 Ook maakte hij een voorhuis voor den troon, alwaar hij richtte, tot een voorhuis des gerichts, dat met ceder bedekt was, van vloer tot vloer.
\par 8 En aan zijn huis, alwaar hij woonde, was een ander voorhof, meer inwaarts dan dat voorhuis, hetwelk aan hetzelve werk gelijk was; ook maakte hij voor de dochter van Farao, die Salomo tot vrouw genomen had, een huis, aan dat voorhuis gelijk.
\par 9 Al deze dingen waren van kostelijke stenen, naar de maten gehouwen, van binnen en van buiten met de zaag gezaagd; en dat van den grondslag tot aan de neutstenen een palm breed, en van buiten tot het grote voorhof.
\par 10 Het was ook gegrondvest met kostelijke stenen, grote stenen; met stenen van tien ellen, en stenen van acht ellen.
\par 11 En bovenop kostelijke stenen, naar de winkelmaten gehouwen, en cederen.
\par 12 En het grote voorhof was rondom van drie rijen gehouwen stenen, met een rij van cederen balken. Zo was het met het binnenste voorhof, van het huis des HEEREN, en met het voorhuis van dat huis.
\par 13 En de koning Salomo zond heen, en liet Hiram van Tyrus halen.
\par 14 Hij was de zoon ener weduwvrouw, uit den stam van Nafthali, en zijn vader was een man van Tyrus geweest, een koperwerker, die vervuld was met wijsheid, en met verstand, en met wetenschap, om alle werk in het koper te maken; deze kwam tot den koning Salomo, en maakte al zijn werk.
\par 15 Want hij vormde twee koperen pilaren; de hoogte van den enen pilaar was achttien ellen, en een draad van twaalf ellen omving den anderen pilaar.
\par 16 Hij maakte ook twee kapitelen, van gegoten koper, om op de hoofden der pilaren te zetten; vijf ellen was de hoogte van het ene kapiteel, en vijf ellen de hoogte van het andere kapiteel.
\par 17 De netten waren van nettenwerk, de banden van ketenwerk voor de kapitelen, die op het hoofd der pilaren waren; zeven waren voor het ene kapiteel, en zeven voor het andere kapiteel.
\par 18 Zo maakte hij de pilaren, mitsgaders twee rijen rondom over het ene net, om de kapitelen, die boven het hoofd der granaatappelen waren, te bedekken; alzo deed hij ook aan het andere kapiteel.
\par 19 En de kapitelen, dewelke waren op het hoofd der pilaren, waren van leliewerk in het voorhuis, van vier ellen.
\par 20 De kapitelen nu waren op de twee pilaren, ja, daarboven tegenover den buik, dewelke was nevens het net; en tweehonderd granaatappelen waren in rijen rondom, ook over het andere kapiteel.
\par 21 Daarna richtte hij de pilaren op in het voorhuis des tempels; en den rechter pilaar opgericht hebbende, zo noemde hij zijn naam Jachin, en den linker pilaar opgericht hebbende, zo noemde hij zijn naam Boaz.
\par 22 En op het hoofd der pilaren was het leliewerk; alzo werd het werk der pilaren volmaakt.
\par 23 Verder maakte hij de gegotene zee; van tien ellen was zij van haar enen rand tot haar anderen rand, rondom rond, en van vijf ellen in haar hoogte, en een meetsnoer van dertig ellen omving ze rondom.
\par 24 En onder haar rand waren knoppen, dezelve rondom omsingelende, tien in een el, omringende die zee rondom; twee rijen dezer knoppen waren in haar gieting gegoten.
\par 25 Zij stond op twaalf runderen; drie ziende naar het noorden, en drie ziende naar het westen, en drie ziende naar het zuiden, en drie ziende naar het oosten; en de zee was boven op dezelve; en al hun achterdelen waren inwaarts.
\par 26 Haar dikte nu was een hand breed, en haar rand als het werk van den rand eens bekers of ener leliebloem; zij hield twee duizend bath.
\par 27 Hij maakte ook tien koperen stellingen; van vier ellen was de lengte ener stelling, en van vier ellen haar breedte, en van drie ellen haar hoogte.
\par 28 En dit was het werk der stelling; zij hadden lijsten, en de lijsten waren tussen kransen.
\par 29 En op de lijsten, die tussen de kransen waren, waren leeuwen, runderen en cherubs; en op de kransen was een voet boven henen; en onder de leeuwen en runderen bijvoegselen van uitgerekt werk.
\par 30 En een stelling had vier koperen raderen, en koperen platen; en haar vier hoeken hadden schouderen; onder het wasvat waren deze gegoten schouderen ter zijde van ieders bijvoegselen.
\par 31 En de mond daarvan was van binnen den krans, en daarboven van een el, en de mond hiervan was rond van voetwerk van een el en een halve el; en op de mond daarvan waren ook graveringen, en de lijsten daarvan waren vierkantig, niet rond.
\par 32 De vier raderen nu waren onder de lijsten, en de assen der raderen aan de stelling; en de hoogte van een rad was een el en een halve el.
\par 33 En het werk van die raderen was als het werk van een wagenrad; hun assen, en hun naven, en hun randen, en hun spaken waren alle gegoten.
\par 34 En er waren vier schouderen op de vier hoeken ener stelling; haar schouderen waren uit de stelling.
\par 35 En op het hoofd ener stelling was een ronde hoogte van een halve el rondom; ook waren op het hoofd der stelling haar handhaven, en haar lijsten uit denzelve.
\par 36 Hij sneed nu op de platen van haar handhaven, en op haar lijsten, cherubs, leeuwen, en palmbomen, naar elks ledige plaats, en bijvoegselen rondom.
\par 37 Dezen gelijk maakte hij de tien stellingen; enerlei gieting, enerlei maat, enerlei snede hadden zij allen.
\par 38 Hij maakte ook tien koperen wasvaten; een wasvat hield veertig bath; een wasvat was van vier ellen; op elke stelling van die tien stellingen was een wasvat.
\par 39 En hij zette vijf dier stellingen aan de rechterzijde van het huis, en vijf aan de linkerzijde van het huis; maar de zee zette hij aan de rechterzijde van het huis, oostwaarts tegen het zuiden.
\par 40 Daartoe maakte Hiram de wasvaten, en de schoffelen, en de besprengbekkens; en Hiram voleindde al het werk te maken, dat hij voor den koning Salomo maakte voor het huis des HEEREN;
\par 41 Te weten de twee pilaren, en bollen der kapitelen, die op het hoofd der twee pilaren waren, en de twee netten, om de twee bollen der kapitelen te bedekken, die op het hoofd der pilaren waren;
\par 42 En de vierhonderd granaatappelen tot de twee netten, namelijk twee rijen van granaatappelen tot het ene net, om de twee bollen der kapitelen te bedekken, die boven op de pilaren waren;
\par 43 Mitsgaders de tien stellingen, en de tien wasvaten op de stellingen;
\par 44 Daartoe de enige zee; en de twaalf runderen onder die zee.
\par 45 De potten ook, en de schoffelen, en de besprengbekkens, en al deze vaten, die Hiram voor den koning Salomo tot het huis des HEEREN maakte, alle van gepolijst koper.
\par 46 In de vlakte van de Jordaan goot ze de koning, in dichte aarde, tussen Sukkoth en tussen Zarthan.
\par 47 En Salomo liet al deze vaten ongewogen vanwege de zeer grote menigte; het gewicht des kopers werd niet onderzocht.
\par 48 Ook maakte Salomo al de vaten, die voor het huis des HEEREN waren; het gouden altaar, en de gouden tafel, op dewelke de toonbroden waren;
\par 49 En de kandelaren, vijf aan de rechterhand, en vijf aan de linkerhand, voor de aanspraakplaats, van gesloten goud; en de bloemen, en de lampen, en de snuiters van goud;
\par 50 Mitsgaders de schalen, en de gaffelen, en de sprengbekkens, en de rookschalen, en de wierookvaten, van gesloten goud; daartoe de herren der deuren van het binnenste huis, van het heilige der heiligen, en der deuren van het huis des tempels, van goud.
\par 51 Alzo werd al het werk volbracht, dat de koning Salomo aan het huis des HEEREN maakte. Daarna bracht Salomo de geheiligde dingen van zijn vader David; het zilver en het goud, en de vaten leide hij onder de schatten van het huis des HEEREN.

\chapter{8}

\par 1 Toen vergaderde Salomo de oudsten van Israel, en al de hoofden der stammen, de oversten der vaderen, onder de kinderen Israels, tot den koning Salomo te Jeruzalem, om de ark des verbonds des HEEREN op te brengen uit de stad Davids, dewelke is Sion.
\par 2 En alle mannen van Israel verzamelden zich tot den koning Salomo, in de maand Ethanim op het feest; die is de zevende maand.
\par 3 En al de oudsten van Israel kwamen; en de priesters namen de ark op.
\par 4 En zij brachten de ark des HEEREN en de tent der samenkomst opwaarts mitsgaders al de heilige vaten, die in de tent waren; en de priesters en de Levieten brachten dezelve opwaarts.
\par 5 De koning Salomo nu en de ganse vergadering van Israel, die bij hem vergaderd waren, waren met hem voor de ark, offerende schapen en runderen, die vanwege de menigte niet konden geteld, noch gerekend worden.
\par 6 Alzo brachten de priesteren de ark des verbonds des HEEREN tot haar plaats, tot de aanspraakplaats van het huis, tot het heilige der heiligen, tot onder de vleugelen der cherubim.
\par 7 Want de cherubim spreidden beide vleugelen over de plaats der ark; en de cherubim overdekten de ark en haar handbomen van boven.
\par 8 Daarna schoven zij de handbomen verder uit, dat de hoofden der handbomen gezien werden uit het heiligdom voor aan de aanspraakplaats, maar buiten niet gezien werden; en zij zijn aldaar tot op dezen dag.
\par 9 Er was niets in de ark, dan alleen de twee stenen tafelen, die Mozes bij Horeb daarin gelegd had, als de HEERE een verbond maakte met de kinderen Israels, toen zij uit Egypteland uitgetogen waren.
\par 10 En het geschiedde, als de priesters uit het heilige uitgingen, dat een wolk het huis des HEEREN vervulde.
\par 11 En de priesters konden niet staan om te dienen, vanwege de wolk; want de heerlijkheid des HEEREN had het huis des HEEREN vervuld.
\par 12 Toen zeide Salomo: De HEERE heeft gezegd, dat Hij in donkerheid zou wonen.
\par 13 Ik heb immers een huis gebouwd, U ter woonstede, een vaste plaats tot Uw eeuwige woning.
\par 14 Daarna wendde de koning zijn aangezicht om, en zegende de ganse gemeente van Israel; en de ganse gemeente van Israel stond.
\par 15 En hij zeide: Geloofd zij de HEERE, de God Israels, Die met Zijn mond tot mijn vader David gesproken heeft, en heeft het met Zijn hand vervuld, zeggende:
\par 16 Van dien dag af, dat Ik Mijn volk Israel uit Egypteland uitgevoerd heb, heb Ik geen stad verkoren uit alle stammen van Israel, om een huis te bouwen, dat Mijn Naam daar zou wezen; maar Ik heb David verkoren, dat hij over Mijn volk Israel wezen zou.
\par 17 Het was ook in het hart van mijn vader David, een huis den Naam van den HEERE, den God Israels, te bouwen.
\par 18 Maar de HEERE zeide tot David, mijn vader: Dewijl dat in uw hart geweest is Mijn Naam een huis te bouwen, gij hebt welgedaan, dat het in uw hart geweest is.
\par 19 Evenwel gij zult dat huis niet bouwen; maar uw zoon, die uit uw lendenen voortkomen zal, die zal Mijn Naam dat huis bouwen.
\par 20 Zo heeft de HEERE bevestigd Zijn woord, dat Hij gesproken had; want ik ben opgestaan in de plaats van mijn vader David, en ik zit op den troon van Israel, gelijk als de HEERE gesproken heeft; en ik heb een huis gebouwd den Naam des HEEREN, des Gods van Israel.
\par 21 En ik heb daar een plaats beschikt voor de ark, waarin het verbond des HEEREN is, hetwelk Hij met onze vaderen maakte, als Hij hen uit Egypteland uitvoerde.
\par 22 En Salomo stond voor het altaar des HEEREN, tegenover de ganse gemeente van Israel, en breidde zijn handen uit naar den hemel;
\par 23 En hij zeide: HEERE, God van Israel, er is geen God, gelijk Gij, boven in den hemel, noch beneden op de aarde, houdende het verbond en de weldadigheid aan Uw knechten, die voor Uw aangezicht met hun ganse hart wandelen;
\par 24 Die Uw knecht, mijn vader David, gehouden hebt, wat Gij tot hem gesproken hadt; want met Uw mond hebt Gij gesproken, en met Uw hand vervuld, gelijk het te dezen dage is.
\par 25 En nu HEERE, God van Israel, houd Uw knecht, mijn vader David, wat Gij tot hem gesproken hebt, zeggende: Geen man zal u van voor Mijn aangezicht afgesneden worden, die op den troon van Israel zitte; alleenlijk zo uw zonen hun weg bewaren, om te wandelen voor Mijn aangezicht, gelijk als gij gewandeld hebt voor Mijn aangezicht.
\par 26 Nu dan, o God van Israel, laat toch Uw woord waar worden, hetwelk Gij gesproken hebt tot Uw knecht, mijn vader David.
\par 27 Maar waarlijk, zou God op de aarde wonen? Zie, de hemelen, ja, de hemel der hemelen zouden U niet begrijpen, hoeveel te min dit huis, dat ik gebouwd heb!
\par 28 Wend U dan nog tot het gebed van Uw knecht, en tot zijn smeking, o HEERE, mijn God, om te horen naar het geroep en naar het gebed, dat Uw knecht heden voor Uw aangezicht bidt.
\par 29 Dat Uw ogen open zijn, nacht en dag, over dit huis, over deze plaats, van dewelke Gij gezegd hebt: Mijn Naam zal daar zijn; om te horen naar het gebed, hetwelk Uw knecht bidden zal in deze plaats.
\par 30 Hoor dan naar de smeking van Uw knecht, en van Uw volk Israel, die in deze plaats zullen bidden; en Gij, hoor in de plaats Uwer woning, in den hemel, ja, hoor, en vergeef.
\par 31 Wanneer iemand tegen zijn naaste zal gezondigd hebben, en hij hem een eed des vloeks opgelegd zal hebben, om zichzelven te vervloeken; en de eed des vloeks voor Uw altaar in dit huis komen zal;
\par 32 Hoor Gij dan in den hemel, en doe, en richt Uw knechten, veroordelende den ongerechtige, gevende zijn weg op zijn hoofd, en rechtvaardigende den gerechtige, gevende hem naar zijn gerechtigheid.
\par 33 Wanneer Uw volk Israel zal geslagen worden voor het aangezicht des vijands, omdat zij tegen U gezondigd zullen hebben, en zich tot U bekeren, en Uw Naam belijden, en tot U in dit huis bidden en smeken zullen;
\par 34 Hoor Gij dan in den hemel, en vergeef de zonde van Uw volk Israel, en breng hen weder in het land, dat Gij hun vaderen gegeven hebt.
\par 35 Als de hemel zal gesloten zijn, dat er geen regen is, omdat zij tegen U gezondigd zullen hebben; en zij in deze plaats bidden, en Uw Naam belijden, en van hun zonden zich bekeren zullen, als Gij hen geplaagd zult hebben;
\par 36 Hoor Gij dan in den hemel, en vergeef de zonde van Uw knechten en van Uw volk Israel, als Gij hun zult geleerd hebben den goeden weg in denwelken zij wandelen zullen; en geef regen op Uw land, dat Gij Uw volk tot een erfenis gegeven hebt.
\par 37 Als er honger in het land wezen zal, als er pest wezen zal, als er brandkoren, honigdauw, sprinkhanen, kevers wezen zullen, als zijn vijand in het land zijner poorten hem belegeren zal, of enige plage, of enige krankheid wezen zal;
\par 38 Alle gebed, alle smeking, die van enig mens, van al Uw volk Israel, geschieden zal; als zij erkennen, een ieder de plage zijns harten, en een ieder zijn handen in dit huis uitbreiden zal;
\par 39 Hoor Gij dan in den hemel, de vaste plaats Uwer woning, en vergeef, en doe, en geef een iegelijk naar al zijn wegen, gelijk Gij zijn hart kent; want Gij alleen kent het hart van alle kinderen der mensen;
\par 40 Opdat zij U vrezen al de dagen, die zij leven zullen in het land, dat Gij onzen vaderen gegeven hebt.
\par 41 Zelfs ook aangaande den vreemde, die van Uw volk Israel niet zal zijn, maar uit verren lande om Uws Naams wil komen zal;
\par 42 (Want zij zullen horen van Uw groten Naam, en van Uw sterke hand, en van Uw uitgestrekten arm) als hij komen en bidden zal in dit huis;
\par 43 Hoor Gij in den hemel, de vaste plaats Uwer woning, en doe naar alles, waarom die vreemde tot U roepen zal; opdat alle volken der aarde Uw Naam kennen, om U te vrezen, gelijk Uw volk Israel, en om te weten, dat Uw Naam genoemd wordt over dit huis, hetwelk ik gebouwd heb.
\par 44 Wanneer Uw volk in den krijg tegen zijn vijand uittrekken zal door den weg, dien Gij hen henen zenden zult, en zullen tot den HEERE bidden naar den weg dezer stad, die Gij verkoren hebt, en naar dit huis, hetwelk ik Uw Naam gebouwd heb;
\par 45 Hoor dan in den hemel hun gebed en hun smeking, en voer hun recht uit.
\par 46 Wanneer zij gezondigd zullen hebben tegen U (want geen mens is er, die niet zondigt), en Gij tegen hen vertoornd zult zijn, en hen leveren zult voor het aangezicht des vijands, dat degenen, die hen gevangen hebben, hen gevankelijk wegvoeren in des vijands land, dat verre of nabij is.
\par 47 En zij in het land, waar zij gevankelijk weggevoerd zijn, weder aan hun hart brengen zullen, dat zij zich bekeren, en tot U smeken in het land dergenen, die ze gevankelijk weggevoerd hebben, zeggende: Wij hebben gezondigd, en verkeerdelijk gedaan, wij hebben goddelooslijk gehandeld;
\par 48 En zij zich tot U bekeren, met hun ganse hart, en met hun ganse ziel, in het land hunner vijanden, die hen gevankelijk weggevoerd zullen hebben; en tot U bidden zullen naar den weg van hun land (hetwelk Gij hun vaderen gegeven hebt), naar deze stad, die Gij verkoren hebt, en naar dit huis, dat ik Uw Naam gebouwd heb;
\par 49 Hoor dan in den hemel, de vaste plaats Uwer woning, hun gebed en hun smeking en voer hun recht uit;
\par 50 En vergeef aan Uw volk, dat zij tegen U gezondigd zullen hebben, en al hun overtredingen, waarmede zij tegen U zullen overtreden hebben; en geef hun barmhartigheid voor het aangezicht dergenen, die ze gevangen houden, opdat zij zich hunner ontfermen;
\par 51 Want zij zijn Uw volk en Uw erfdeel, die Gij uitgevoerd hebt uit Egypteland, uit het midden des ijzeren ovens;
\par 52 Opdat Uw ogen open zijn tot de smeking van Uw knecht, en tot de smeking van Uw volk Israel, om naar hen te horen, in al hun roepen tot U.
\par 53 Want Gij hebt hen U tot een erfdeel afgezonderd, uit alle volken der aarde; gelijk als Gij gesproken hebt door den dienst van Mozes, Uw knecht, als Gij onze vaderen uit Egypte uitvoerdet, Heere HEERE!
\par 54 Het geschiedde nu, als Salomo voleind had dit ganse gebed, en deze smeking tot den HEERE te bidden, dat hij van voor het altaar des HEEREN opstond, van het knielen op zijn knieen, met zijn handen uitgebreid naar den hemel;
\par 55 Zo stond hij, en zegende de ganse gemeente van Israel, zeggende met luider stem:
\par 56 Geloofd zij de HEERE, Die aan Zijn volk Israel rust gegeven heeft, naar alles, wat Hij gesproken heeft! Niet een enig woord is er gevallen van al Zijn goede woorden, die Hij gesproken heeft door den dienst van Mozes, Zijn knecht.
\par 57 De HEERE, onze God, zij met ons, gelijk als Hij geweest is met onze vaderen; Hij verlate ons niet, en begeve ons niet;
\par 58 Neigende tot Zich ons hart, om in al Zijn wegen te wandelen, en om te houden Zijn geboden, en Zijn inzettingen, en Zijn rechten, dewelke Hij onzen vaderen geboden heeft.
\par 59 En dat deze mijn woorden, waarmede ik voor den HEERE gesmeekt heb, mogen nabij zijn voor den HEERE, onzen God, dag en nacht; opdat Hij het recht van Zijn knecht uitvoere, en het recht van Zijn volk Israel, elkeen dagelijks op zijn dag.
\par 60 Opdat alle volken der aarde weten, dat de HEERE die God is, niemand meer;
\par 61 En ulieder hart volkomen zij met den HEERE, onzen God, om te wandelen in Zijn inzettingen, en Zijn geboden te houden, gelijk te dezen dage.
\par 62 En de koning, en gans Israel met hem, offerden slachtofferen voor het aangezicht des HEEREN.
\par 63 En Salomo offerde ten dankoffer, dat hij den HEERE offerde, twee en twintig duizend runderen, en honderd en twintig duizend schapen. Alzo hebben zij het huis des HEEREN ingewijd, de koning en al de kinderen Israels.
\par 64 Ten zelfden dage heiligde de koning het middelste des voorhofs, dat voor het huis des HEEREN was, omdat hij aldaar het brandoffer en het spijsoffer bereid had, mitsgaders het vet der dankofferen; want het koperen altaar, dat voor het aangezicht des HEEREN was, was te klein, om de brandofferen, en de spijsofferen, en het vet der dankofferen te vatten.
\par 65 Terzelfder tijd ook hield Salomo het feest, en gans Israel met hem, een grote gemeente, van den ingang af van Hamath tot de rivier van Egypte, voor het aangezicht des HEEREN, onzes Gods, zeven dagen en zeven dagen, zijnde veertien dagen.
\par 66 Op den achtsten dag liet hij het volk gaan, en zij zegenden den koning; daarna gingen zij naar hun tenten, blijde en goedsmoeds over al het goede, dat de HEERE aan David, Zijn knecht, en aan Israel, Zijn volk, gedaan had.

\chapter{9}

\par 1 Het geschiedde nu, als Salomo voleind had te bouwen het huis des HEEREN en het huis des konings, en al de begeerte van Salomo, die hem gelust had te maken;
\par 2 Dat de HEERE ten anderen male aan Salomo verscheen, gelijk als Hij hem in Gibeon verschenen was.
\par 3 En de HEERE zeide tot hem: Ik heb uw gebed en uw smeking gehoord, die gij voor Mijn aangezicht smekende gedaan hebt; Ik heb dat huis geheiligd, hetwelk gij gebouwd hebt, opdat Ik Mijn Naam aldaar tot in eeuwigheid zette; en Mijn ogen en Mijn hart zullen daar zijn te allen dage.
\par 4 En zo gij voor Mijn aangezicht wandelen zult, gelijk als uw vader David gewandeld heeft, met volkomenheid des harten, en met oprechtheid, om te doen naar al wat Ik u geboden heb, en Mijn inzettingen en Mijn rechten houden zult;
\par 5 Zo zal Ik den troon uws koninkrijks over Israel bevestigen in eeuwigheid; gelijk als Ik gesproken heb over uw vader David, zeggende: Geen man zal u afgesneden worden van den troon van Israel.
\par 6 Maar zo gijlieden u te enen male afkeren zult, gij en uw kinderen, van Mij na te volgen, en niet houden zult Mijn geboden en Mijn inzettingen, die Ik voor uw aangezicht gegeven heb; maar heengaan, en andere goden dienen, en u voor dezelve nederbuigen zult;
\par 7 Zo zal Ik Israel uitroeien van het land, dat Ik hun gegeven heb, en dit huis, hetwelk Ik Mijn Naam geheiligd heb, zal Ik van Mijn aangezicht wegwerpen; en Israel zal tot een spreekwoord en spotrede zijn onder alle volken.
\par 8 En aangaande dit huis, dat verheven zal geweest zijn, al wie voor hetzelve zal voorbijgaan, zal zich ontzetten en fluiten; men zal zeggen: Waarom heeft de HEERE alzo gedaan aan dit land en aan dit huis?
\par 9 En men zal zeggen: Omdat zij den HEERE, hun God, verlaten hebben, Die hun vaderen uit Egypteland uitgevoerd had, en hebben zich aan andere goden gehouden, en zich voor dezelve nedergebogen, en hen gediend; daarom heeft de HEERE al dit kwaad over hen gebracht.
\par 10 En het geschiedde ten einde van twintig jaren, in dewelke Salomo die twee huizen gebouwd had, het huis des HEEREN en het huis des konings;
\par 11 (Waartoe Hiram, de koning van Tyrus, Salomo van cederbomen, en van dennenbomen, en van goud, naar al zijn lust opgebracht had), dat alstoen de koning Salomo aan Hiram twintig steden gaf in het land van Galilea.
\par 12 En Hiram toog uit van Tyrus, om de steden te bezien, die Salomo hem gegeven had, maar zij waren niet recht in zijn ogen.
\par 13 Daarom zeide hij: Wat zijn dat voor steden, mijn broeder, die gij mij gegeven hebt? En hij noemde ze het land Kabul, tot op dezen dag.
\par 14 En Hiram had den koning gezonden honderd en twintig talenten gouds.
\par 15 Dit is nu de oorzaak van het uitschot, dat de koning Salomo deed opkomen, om het huis des HEEREN te bouwen, en zijn huis, en Millo, en den muur van Jeruzalem, mitsgaders Hazor, en Megiddo, en Gezer.
\par 16 Want Farao, de koning van Egypte, was opgekomen, en had Gezer ingenomen, en haar met vuur verbrand, en de Kanaanieten, die in de stad woonden, gedood, en had haar aan zijn dochter, de huisvrouw van Salomo, tot een geschenk gegeven.
\par 17 Alzo bouwde Salomo Gezer, en het lage Beth-horon.
\par 18 En Baalath, en Tamor in de woestijn, in dat land;
\par 19 En al de schatsteden, die Salomo had, en de wagensteden, en de steden der ruiteren, en wat de begeerte van Salomo begeerde te bouwen, in Jeruzalem, en op den Libanon, en in het ganse land zijner heerschappij.
\par 20 Aangaande al het volk, dat overgebleven was van de Amorieten, Hethieten, Ferezieten, Hevieten, en Jebusieten, die niet waren van de kinderen Israels;
\par 21 Hun kinderen, die na hen in het land overgebleven waren, die de kinderen Israels niet hadden kunnen verbannen, die heeft Salomo gebracht op slaafsen uitschot tot op dezen dag.
\par 22 Doch van de kinderen Israels maakte Salomo geen slaaf; maar zij waren krijgslieden, en zijn knechten, en zijn vorsten, en zijn hoofdlieden, en de oversten zijner wagenen, en zijner ruiteren.
\par 23 Dezen waren de oversten der bestelden, die over het werk van Salomo waren, vijfhonderd en vijftig, die heerschappij hadden over het volk, dat in het werk doende was.
\par 24 Doch de dochter van Farao toog van de stad Davids op tot haar huis, hetwelk hij voor haar gebouwd had; toen bouwde hij Millo.
\par 25 En Salomo offerde driemaal des jaars brandofferen en dankofferen, op het altaar, dat hij den HEERE gebouwd had, en rookte op dat, hetwelk voor het aangezicht des HEEREN was, als hij het huis volmaakt had.
\par 26 De koning Salomo maakte ook schepen te Ezeon-geber, dat bij Eloth is, aan den oever der Schelfzee, in het land van Edom.
\par 27 En Hiram zond met die schepen zijn knechten, scheepslieden, kenners van de zee, met de knechten van Salomo.
\par 28 En zij kwamen te Ofir, en haalden van daar aan goud, vierhonderd en twintig talenten, en brachten het tot den koning Salomo.

\chapter{10}

\par 1 En toen de koningin van Scheba het gerucht van Salomo hoorde, aangaande den Naam des HEEREN, kwam zij, om hem met raadselen te verzoeken.
\par 2 En zij kwam te Jeruzalem, met een zeer zwaar heir, met kemelen, dragende specerijen, en zeer veel gouds, en kostelijk gesteente; en zij kwam tot Salomo, en sprak tot hem al wat in haar hart was.
\par 3 En Salomo verklaarde haar al haar woorden; geen ding was er verborgen voor den koning, dat hij haar niet verklaarde.
\par 4 Als nu de koningin van Scheba zag al de wijsheid van Salomo, en het huis, hetwelk hij gebouwd had,
\par 5 En de spijze zijner tafel, en het zitten zijner knechten, en het staan zijner dienaren, en hun kledingen, en zijn schenkers, en zijn opgang, waardoor hij henen opging in het huis des HEEREN, zo was in haar geen geest meer.
\par 6 En zij zeide tot den koning: Het woord is waarheid geweest, dat ik in mijn land gehoord heb, van uw zaken en van uw wijsheid.
\par 7 Ik heb die woorden niet geloofd, totdat ik gekomen ben, en mijn ogen dat gezien hebben; en zie, de helft is mij niet aangezegd; gij hebt met wijsheid en goed overtroffen het gerucht, dat ik gehoord heb.
\par 8 Welgelukzalig zijn uw mannen, welgelukzalig deze uw knechten, die gedurig voor uw aangezicht staan, die uw wijsheid horen!
\par 9 Geloofd zij de HEERE, uw God, Die behagen in u heeft gehad, om u op den troon van Israel te zetten! Omdat de HEERE Israel in eeuwigheid bemint, daarom heeft Hij u tot koning gesteld, om recht en gerechtigheid te doen.
\par 10 En zij gaf den koning honderd en twintig talenten gouds, en zeer veel specerijen, en kostelijk gesteente; als deze specerij, die de koningin van Scheba den koning Salomo gaf, is er nooit meer in menigte gekomen.
\par 11 Verder ook de schepen van Hiram, die goud uit Ofir voerden, brachten uit Ofir zeer veel almuggimhout en kostelijk gesteente.
\par 12 En de koning maakte van dit almuggimhout steunselen voor het huis des HEEREN, en voor het huis des konings, mitsgaders harpen en luiten voor de zangers. Het almuggimhout was zo niet gekomen noch gezien geweest, tot op dezen dag.
\par 13 En de koning Salomo gaf de koningin van Scheba al haar behagen, wat zij begeerde; behalve dat hij haar gaf naar het vermogen van den koning Salomo; zo keerde zij en toog in haar land, zij en haar knechten.
\par 14 Het gewicht nu van het goud, dat voor Salomo op een jaar inkwam was zeshonderd zes en zestig talenten gouds;
\par 15 Behalve dat van de kramers was, en van den handel der kruideniers, en van alle koningen van Arabie, en van de geweldigen van dat land.
\par 16 Ook maakte de koning Salomo tweehonderd rondassen van geslagen goud; zeshonderd sikkelen gouds liet hij opwegen tot elke rondas.
\par 17 Insgelijks driehonderd schilden van geslagen goud; drie pond gouds liet hij opwegen tot elk schild; en de koning leide ze in het huis des wouds van Libanon.
\par 18 Nog maakte de koning een groten elpenbenen troon, en hij overtoog denzelven met dicht goud.
\par 19 Deze troon had zes trappen, en het hoofd van den troon was van achteren rond, en aan beide zijden waren leuningen tot de zitplaats toe, en twee leeuwen stonden bij die leuningen.
\par 20 En twaalf leeuwen stonden daar op de zes trappen aan beide zijden, desgelijks is in geen koninkrijken gemaakt geweest.
\par 21 Ook waren alle drinkvaten van den koning Salomo van goud, en alle vaten van het huis des wouds van Libanon waren van gesloten goud; geen zilver was er aan; want het werd in de dagen van Salomo niet voor enig ding geacht.
\par 22 Want de koning had in zee schepen van Tharsis, met de schepen van Hiram; deze schepen van Tharsis kwamen in, eenmaal in drie jaren, brengende goud, en zilver, elpenbeen, en apen, en pauwen.
\par 23 Alzo werd de koning Salomo groter dan alle koningen der aarde, in rijkdom en in wijsheid.
\par 24 En de ganse aarde zocht het aangezicht van Salomo, om zijn wijsheid te horen, die God in zijn hart gegeven had.
\par 25 En zij brachten een ieder zijn geschenk, zilveren vaten, en gouden vaten, en klederen, en harnas, en specerijen, paarden en muilezelen, elk ding van jaar tot jaar.
\par 26 Daartoe vergaderde Salomo wagenen en ruiteren, en hij had duizend en vierhonderd wagenen, en twaalf duizend ruiteren, en leide ze in de wagensteden en bij den koning in Jeruzalem.
\par 27 En de koning maakte het zilver in Jeruzalem te zijn als stenen, en de cederen maakte hij te zijn als de wilde vijgebomen, die in de laagte zijn, in menigte.
\par 28 En het uitbrengen der paarden was hetgeen Salomo uit Egypte had; en aangaande het linnen garen, de kooplieden des konings namen het linnen garen voor den prijs.
\par 29 En een wagen kwam op, en ging uit van Egypte, voor zeshonderd sikkelen zilvers, en een paard voor honderd en vijftig; en alzo voerden ze die uit door hun hand voor alle koningen der Hethieten, en voor de koningen van Syrie.

\chapter{11}

\par 1 En de koning Salomo had veel vreemde vrouwen lief, en dat benevens de dochter van Farao: Moabietische, Ammonietische, Edomietische, Sidonische, Hethietische;
\par 2 Van die volken, waarvan de HEERE gezegd had tot de kinderen Israels: Gijlieden zult tot hen niet ingaan, en zij zullen tot u niet inkomen; zij zouden zekerlijk uw hart achter hun goden neigen; aan deze hing Salomo met liefde.
\par 3 En hij had zevenhonderd vrouwen, vorstinnen, en driehonderd bijwijven en zijn vrouwen neigden zijn hart.
\par 4 Want het geschiedde in den tijd van Salomo's ouderdom, dat zijn vrouwen zijn hart achter andere goden neigden; dat zijn hart niet volkomen was met den HEERE, zijn God, gelijk het hart van zijn vader David.
\par 5 Want Salomo wandelde Astoreth, den god der Sidoniers, na, en Milchom, het verfoeisel der Ammonieten.
\par 6 Alzo deed Salomo, dat kwaad was in de ogen des HEEREN; en volhardde niet den HEERE te volgen, gelijk zijn vader David.
\par 7 Toen bouwde Salomo een hoogte voor Kamos, het verfoeisel der Moabieten, op den berg, die voor Jeruzalem is, en voor Molech, het verfoeisel der kinderen Ammons.
\par 8 En alzo deed hij voor al zijn vreemde vrouwen, die haar goden rookten en offerden.
\par 9 Daarom vertoornde Zich de HEERE tegen Salomo, omdat hij zijn hart geneigd had van den HEERE, den God Israels, Die hem tweemaal verschenen was.
\par 10 En hem van deze zaak geboden had, dat hij andere goden niet zou nawandelen; doch hij hield niet, wat de HEERE geboden had.
\par 11 Daarom zeide de HEERE tot Salomo: Dewijl dit bij u geschied is, dat gij niet hebt gehouden Mijn verbond en Mijn inzettingen, die Ik u geboden heb; Ik zal gewisselijk dit koninkrijk van u scheuren, en datzelve uw knecht geven.
\par 12 In uw dagen nochtans zal Ik dat niet doen, om uws vaders Davids wil, van de hand uws zoons zal Ik het scheuren.
\par 13 Doch Ik zal het gehele koninkrijk niet afscheuren; een stam zal Ik uw zoon geven, om Mijns knechts Davids wil, en om Jeruzalems wil, dat Ik verkoren heb.
\par 14 Zo verwekte de HEERE Salomo een tegenpartijder, Hadad, den Edomiet; hij was van des konings zaad in Edom.
\par 15 Want het was geschied, als David in Edom was, toen Joab, de krijgsoverste, optoog, om de verslagenen te begraven, dat hij al wat mannelijk was in Edom sloeg;
\par 16 Want Joab bleef aldaar zes maanden, met het ganse Israel, totdat hij al wat mannelijk was in Edom uitgeroeid had.
\par 17 Doch Hadad was ontvloden, hij en enige Edomietische mannen uit zijns vaders knechten met hem, om in Egypte te komen; Hadad nu was een klein jongsken.
\par 18 En zij maakten zich op van Midian, en kwamen tot Paran; en namen met zich mannen van Paran, en kwamen in Egypte tot Farao, den koning van Egypte, die hem een huis gaf, en hem voeding toezeide, en hem een land gaf.
\par 19 En Hadad vond grote genade in de ogen van Farao, zodat hij hem tot een vrouw gaf de zuster zijner huisvrouw, de zuster van Tachpenes, de koningin.
\par 20 En de zuster van Tachpenes baarde hem zijn zoon Genubath, denwelken Tachpenes optoog in het huis van Farao; zodat Genubath in het huis van Farao was, onder de zonen van Farao.
\par 21 Toen nu Hadad in Egypte hoorde, dat David met zijn vaderen ontslapen, en dat Joab, de krijgsoverste, dood was, zeide Hadad tot Farao: Laat mij gaan, dat ik in mijn land trekke.
\par 22 Doch Farao zeide: Maar wat ontbreekt u bij mij, dat, zie, gij in uw land zoekt te trekken? En hij zeide: Niets, maar laat mij evenwel gaan.
\par 23 Ook verwekte God hem een wederpartijder, Rezon, den zoon van Eljada, die gevloden was van zijn heer Hadad-ezer, den koning van Zoba,
\par 24 Tegen welken hij ook mannen vergaderd had, en werd overste ener bende, als David die doodde; en getrokken zijnde naar Damaskus, woonden zij aldaar, en regeerden in Damaskus.
\par 25 En hij was Israels tegenpartijder al de dagen van Salomo, en dat benevens het kwaad, dat Hadad deed; want hij had een afkeer van Israel, en hij regeerde over Syrie.
\par 26 Daartoe Jerobeam, de zoon van Nebat, een Efrathiet van Zereda, Salomo's knecht (wiens moeders naam was Zerua, een weduwvrouw), hief ook de hand op tegen den koning.
\par 27 Dit is nu de zaak, waarom hij de hand tegen den koning ophief. Salomo bouwde Millo, en sloot de breuk der stad van zijn vader David toe.
\par 28 En de man Jerobeam was een dapper held. Toen Salomo dezen jongeling zag, dat hij arbeidzaam was, zo stelde hij hem over al den last van het huis van Jozef.
\par 29 Het geschiedde nu te dier tijd, als Jerobeam uit Jeruzalem uitging, dat de profeet Ahia, de Siloniet, hem op den weg vond, en hij zich een nieuw kleed aangedaan had, en zij beiden alleen op het veld waren;
\par 30 Zo vatte Ahia het nieuwe kleed, dat aan hem was, en scheurde het, in twaalf stukken.
\par 31 En hij zeide tot Jerobeam: Neem u tien stukken; want alzo zegt de HEERE, de God Israels: Zie, Ik zal het koninkrijk van de hand van Salomo scheuren, en u tien stammen geven.
\par 32 Maar een stam zal hij hebben, om Mijns knechts Davids wil, en om Jeruzalems wil, de stad, die Ik verkoren heb uit alle stammen van Israel.
\par 33 Daarom dat zij Mij verlaten, en zich nedergebogen hebben voor Astoreth, den god der Sidoniers, Kamos, den god der Moabieten, en Milchom, den god der kinderen Ammons; en niet gewandeld hebben in Mijn wegen, om te doen wat recht is in Mijn ogen, te weten Mijn inzettingen en Mijn rechten; gelijk zijn vader David.
\par 34 Doch niets van dit koninkrijk zal Ik uit zijn hand nemen; maar Ik stel hem tot een vorst al de dagen zijns levens, om Mijns knechts Davids wil, dien Ik verkoren heb, die Mijn geboden en Mijn inzettingen gehouden heeft.
\par 35 Maar uit de hand zijns zoons zal Ik het koninkrijk nemen; en Ik zal u daarvan tien stammen geven.
\par 36 En zijn zoon zal Ik een stam geven; opdat Mijn knecht David altijd een lamp voor Mijn aangezicht hebbe in Jeruzalem, de stad, die Ik Mij verkoren heb, om Mijn Naam daar te stellen.
\par 37 Zo zal Ik u nemen, en gij zult regeren over al wat uw ziel zal begeren; en gij zult koning zijn over Israel.
\par 38 En het zal geschieden, zo gij horen zult al wat Ik u zal gebieden, en in Mijn wegen zult wandelen, en doen wat recht in Mijn ogen is, houdende Mijn inzettingen en Mijn geboden, gelijk als Mijn knecht David gedaan heeft; dat Ik met u zal zijn, en u een bestendig huis bouwen, gelijk als Ik David gebouwd heb, en zal u Israel geven.
\par 39 En Ik zal om diens wil het zaad van David verootmoedigen; nochtans niet altijd.
\par 40 Daarom zocht Salomo Jerobeam te doden; maar Jerobeam maakte zich op, en vlood in Egypte, tot Sisak, den koning van Egypte, en was in Egypte, totdat Salomo stierf.
\par 41 Het overige nu der geschiedenissen van Salomo, en al wat hij gedaan heeft, en zijn wijsheid, is dat niet geschreven in het boek der geschiedenissen van Salomo?
\par 42 De tijd nu, dien Salomo te Jeruzalem over het ganse Israel regeerde, was veertig jaar.
\par 43 Daarna ontsliep Salomo met zijn vaderen, en werd begraven in de stad van zijn vader David; en Rehabeam, zijn zoon, werd koning in zijn plaats.

\chapter{12}

\par 1 En Rehabeam toog naar Sichem, want het ganse Israel was te Sichem gekomen, om hem koning te maken.
\par 2 Het geschiedde nu, als Jerobeam, de zoon van Nebat, dit hoorde, daar hij nog in Egypte was (want hij was van het aangezicht van den koning Salomo gevloden; en Jerobeam woonde in Egypte),
\par 3 Dat zij henen zonden, en lieten hem roepen; en Jerobeam en de ganse gemeente van Israel kwamen en spraken tot Rehabeam, zeggende:
\par 4 Uw vader heeft ons juk hard gemaakt; gij dan nu, maak uws vaders harden dienst, en zijn zwaar juk, dat hij ons opgelegd heeft, lichter, en wij zullen u dienen.
\par 5 En hij zeide tot hen: Gaat heen tot aan den derden dag, komt dan weder tot mij. En het volk ging heen.
\par 6 En de koning Rehabeam hield raad met de oudsten, die gestaan hadden voor het aangezicht van zijn vader Salomo, als hij leefde, zeggende: Hoe raadt gijlieden, dat men dit volk antwoorden zal?
\par 7 En zij spraken tot hem, zeggende: Indien gij heden knecht van dit volk wezen zult, en hen dienen, en hun antwoorden, en tot hen goede woorden spreken zult, zo zullen zij te allen dage uw knechten zijn.
\par 8 Maar hij verliet den raad der oudsten, dien zij hem geraden hadden; en hij hield raad met de jongelingen, die met hem opgewassen waren, die voor zijn aangezicht stonden.
\par 9 En hij zeide tot hen: Wat raadt gijlieden, dat wij dit volk antwoorden zullen, die tot mij gesproken hebben, zeggende: Maak het juk, dat uw vader ons opgelegd heeft, lichter.
\par 10 En de jongelingen, die met hem opgewassen waren, spraken tot hem, zeggende: Alzo zult gij zeggen tot dat volk, die tot u gesproken hebben, zeggende: Uw vader heeft ons juk zwaar gemaakt, maar maak gij het over ons lichter; alzo zult gij tot hen spreken: Mijn kleinste vinger zal dikker zijn dan mijns vaders lenden.
\par 11 Indien nu mijn vader een zwaar juk op u heeft doen laden, zo zal ik boven uw juk nog daartoe doen; mijn vader heeft u met geselen gekastijd, maar ik zal u met schorpioenen kastijden.
\par 12 Zo kwam Jerobeam en het ganse volk tot Rehabeam op den derden dag, gelijk als de koning gesproken had, zeggende: Komt weder tot mij op den derden dag.
\par 13 En de koning antwoordde het volk hardelijk; want hij verliet den raad der oudsten, dien zij hem geraden hadden.
\par 14 En hij sprak tot hen naar den raad der jongelingen, zeggende: Mijn vader heeft uw juk zwaar gemaakt, maar ik zal boven uw juk nog daartoe doen; mijn vader heeft u met geselen gekastijd, maar ik zal u met schorpioenen kastijden.
\par 15 Alzo hoorde de koning naar het volk niet; want deze omwending was van den HEERE, opdat Hij Zijn woord bevestigde, hetwelk de HEERE door den dienst van Ahia, den Siloniet, gesproken had tot Jerobeam, den zoon van Nebat.
\par 16 Toen gans Israel zag, dat de koning naar hen niet hoorde, zo gaf het volk den koning weder antwoord, zeggende: Wat deel hebben wij aan David? Ja, geen erve hebben wij aan den zoon van Isai; naar uw tenten, o Israel! Voorzie nu uw huis, o David! Zo ging Israel naar zijn tenten.
\par 17 Doch aangaande de kinderen van Israel, die in de steden van Juda woonden, over die regeerde Rehabeam ook.
\par 18 Toen zond de koning Rehabeam Adoram, die over de schatting was; en het ganse Israel stenigde hem met stenen, dat hij stierf; maar de koning Rehabeam verkloekte zich om op een wagen te klimmen, dat hij naar Jeruzalem vluchtte.
\par 19 Alzo vielen de Israelieten van het huis Davids af, tot op dezen dag.
\par 20 En het geschiedde, als gans Israel hoorde, dat Jerobeam wedergekomen was, dat zij henen zonden, en hem in de vergadering riepen, en hem over gans Israel koning maakten; niemand volgde het huis Davids, dan de stam van Juda alleen.
\par 21 Toen nu Rehabeam te Jeruzalem gekomen was, vergaderde hij het ganse huis van Juda en den stam van Benjamin, honderd en tachtig duizend uitgelezenen, geoefend ten oorlog, om tegen het huis Israels te strijden, opdat hij het koninkrijk weder aan Rehabeam, den zoon van Salomo, bracht.
\par 22 Doch het woord van God geschiedde tot Semaja, den man Gods, zeggende:
\par 23 Zeg tot Rehabeam, den zoon van Salomo, den koning van Juda, en tot het ganse huis van Juda en Benjamin, en overige des volks, zeggende:
\par 24 Zo zegt de HEERE: Gij zult niet optrekken, noch strijden tegen uw broederen, de kinderen Israels; een ieder kere weder tot zijn huis, want deze zaak is van Mij geschied. En zij hoorden het woord des HEEREN, en keerden weder, om weg te trekken naar het woord des HEEREN.
\par 25 Jerobeam nu bouwde Sichem op het gebergte van Efraim, en woonde daarin, en toog van daar uit, en bouwde Penuel.
\par 26 En Jerobeam zeide in zijn hart: Nu zal het koninkrijk weder tot het huis van David keren.
\par 27 Zo dit volk opgaan zal om offeranden te doen in het huis des HEEREN te Jeruzalem, zo zal het hart dezes volks tot hun heer, tot Rehabeam, den koning van Juda, wederkeren; ja, zij zullen mij doden, en tot Rehabeam, den koning van Juda, wederkeren.
\par 28 Daarom hield de koning een raad, en maakte twee gouden kalveren; en hij zeide tot hen: Het is ulieden te veel om op te gaan naar Jeruzalem; zie uw goden, o Israel, die u uit Egypteland opgebracht hebben.
\par 29 En hij zette het ene te Beth-el, en het andere stelde hij te Dan.
\par 30 En deze zaak werd tot zonde; want het volk ging heen voor het ene, tot Dan toe.
\par 31 Hij maakte ook een huis der hoogten; en maakte priesteren van de geringsten des volks, die niet waren uit de zonen van Levi.
\par 32 En Jerobeam maakte een feest in de achtste maand, op den vijftienden dag der maand, gelijk het feest, dat in Juda was, en offerde op het altaar; van gelijken deed hij te Beth-el, offerende den kalveren, die hij gemaakt had; hij stelde ook te Beth-el priesteren der hoogten, die hij gemaakt had.
\par 33 En hij offerde op het altaar, dat hij te Beth-el gemaakt had, op den vijftienden dag der achtste maand, der maand, dewelke hij uit zijn hart verdacht had; zo maakte hij den kinderen Israels een feest, en offerde op dat altaar, rokende.

\chapter{13}

\par 1 En ziet, een man Gods kwam uit Juda, door het woord des HEEREN tot Beth-el; en Jerobeam stond bij het altaar, om te roken.
\par 2 En hij riep tegen het altaar, door het woord des HEEREN, en zeide: Altaar, altaar, zo zegt de HEERE: Zie, een zoon zal aan het huis Davids geboren worden, wiens naam zal zijn Josia; die zal op u offeren de priesters der hoogten, die op u roken, en men zal mensenbeenderen op u verbranden.
\par 3 En hij gaf ten zelfden dage een wonderteken, zeggende: Dit is dat wonderteken, waarvan de HEERE gesproken heeft; ziet, het altaar zal vaneen gescheurd, en de as, die daarop is, afgestort worden.
\par 4 Het geschiedde nu, als de koning het woord van den man Gods hoorde, hetwelk hij tegen het altaar te Beth-el geroepen had, dat Jerobeam zijn hand van op het altaar uitstrekte, zeggende: Grijpt hem! Maar zijn hand, die hij tegen hem uitgestrekt had, verdorde, dat hij ze niet weder tot zich trekken kon.
\par 5 En het altaar werd vaneen gescheurd, en de as van het altaar afgestort, naar dat wonderteken, dat de man Gods gegeven had, door het woord des HEEREN.
\par 6 Toen antwoordde de koning, en zeide tot den man Gods: Aanbid toch het aangezicht des HEEREN, uws Gods, ernstelijk, en bid voor mij, dat mijn hand weder tot mij kome! Toen bad de man Gods het aangezicht des HEEREN ernstelijk; en de hand des konings kwam weder tot hem, en werd gelijk te voren.
\par 7 En de koning sprak tot den man Gods: Kom met mij naar huis, en sterk u, en ik zal u een geschenk geven.
\par 8 Maar de man Gods zeide tot den koning: Al gaaft gij mij de helft van uw huis, zo zou ik niet met u gaan, en ik zou in deze plaats geen brood eten, noch water drinken.
\par 9 Want zo heeft mij de HEERE geboden door Zijn woord, zeggende: Gij zult geen brood eten, noch water drinken; en gij zult niet wederkeren door den weg, dien gij gegaan zijt.
\par 10 En hij ging door een anderen weg, en keerde niet weder door den weg, door welken hij te Beth-el gekomen was.
\par 11 Een oud profeet nu woonde te Beth-el; en zijn zoon kwam, en vertelde hem al het werk, dat de man Gods te dien dage in Beth-el gedaan had, met de woorden, die hij tot den koning gesproken had; deze vertelden zij ook hun vader.
\par 12 En hun vader sprak tot hen: Wat weg is hij getogen? En zijn zonen hadden den weg gezien, welken de man Gods was getogen, die uit Juda gekomen was.
\par 13 Toen zeide hij tot zijn zonen: Zadelt mij den ezel. En zij zadelden hem den ezel, en hij reed daarop.
\par 14 En hij toog den man Gods na, en vond hem zittende onder een eik; en hij zeide tot hem: Zijt gij de man Gods, die uit Juda gekomen zijt? En hij zeide: Ik ben het.
\par 15 Toen zeide hij tot hem: Kom met mij naar huis, en eet brood.
\par 16 Doch hij zeide: Ik kan niet met u wederkeren, noch met u inkomen; ik zal ook geen brood eten, noch met u water drinken, in deze plaats.
\par 17 Want een woord is tot mij geschied door het woord des HEEREN: Gij zult aldaar noch brood eten, noch water drinken; gij zult niet wederkeren, gaande door den weg, door denwelken gij gegaan zijt.
\par 18 En hij zeide tot hem: Ik ben ook een profeet, gelijk gij, en een engel heeft tot mij gesproken door het woord des HEEREN, zeggende: Breng hem weder met u in uw huis, dat hij brood ete en water drinke. Doch hij loog hem.
\par 19 En hij keerde met hem wederom, en at brood in zijn huis, en dronk water.
\par 20 En het geschiedde, als zij aan de tafel zaten, dat het woord des HEEREN geschiedde tot den profeet, die hem had doen wederkeren;
\par 21 En hij riep tot den man Gods, die uit Juda gekomen was, zeggende: Zo zegt de HEERE: Daarom dat gij den mond des HEEREN zijt wederspannig geweest, en niet gehouden hebt het gebod, dat u de HEERE, uw God, geboden had,
\par 22 Maar zijt wedergekeerd, en hebt brood gegeten en water gedronken ter plaatse, waarvan Hij tot u gesproken had: Gij zult geen brood eten noch water drinken; zo zal uw dood lichaam in uw vaderen graf niet komen.
\par 23 En het geschiedde, nadat hij brood gegeten, en nadat hij gedronken had, dat hij hem den ezel zadelde, te weten voor den profeet, dien hij had doen wederkeren.
\par 24 Zo toog hij heen, en een leeuw vond hem op den weg, en doodde hem; en zijn dood lichaam lag geworpen op den weg, en de ezel stond daarbij; ook stond de leeuw bij het dode lichaam.
\par 25 En ziet, er gingen lieden voorbij, en zagen het dode lichaam geworpen op den weg, en den leeuw, staande bij het dode lichaam; en zij kwamen en zeiden het in de stad, waarin de oude profeet woonde.
\par 26 Als de profeet, die hem van den weg had doen wederkeren, dit hoorde, zo zeide hij: Het is de man Gods, die den mond des HEEREN wederspannig is geweest; daarom heeft de HEERE hem den leeuw overgegeven, die hem gebroken, en hem gedood heeft, naar het woord des HEEREN, dat Hij tot hem gesproken had.
\par 27 Verder sprak hij tot zijn zonen, zeggende: Zadelt mij den ezel. En zij zadelden hem.
\par 28 Toen toog hij heen, en vond zijn dood lichaam geworpen op den weg, en den ezel, en den leeuw, staande bij het dode lichaam; de leeuw had het dode lichaam niet gegeten, en den ezel niet gebroken.
\par 29 Toen nam de profeet het dode lichaam van den man Gods op, en leide dat op den ezel, en voerde het wederom; zo kwam de oude profeet in de stad om rouw te bedrijven en hem te begraven.
\par 30 En hij leide zijn dood lichaam in zijn graf; en zij maakten over hem een weeklage: Ach, mijn broeder!
\par 31 Het geschiedde nu, nadat hij hem begraven had, dat hij sprak tot zijn zonen, zeggende: Als ik zal gestorven zijn, zo begraaft mij in dat graf, waarin de man Gods begraven is, en legt mijn beenderen bij zijn beenderen.
\par 32 Want de zaak zal gewisselijk geschieden, die hij door het woord des HEEREN uitgeroepen heeft tegen het altaar, dat te Beth-el is, en tegen al de huizen der hoogten, die in de steden van Samaria zijn.
\par 33 Na deze geschiedenis keerde zich Jerobeam niet van zijn bozen weg; maar maakte wederom priesters der hoogten van de geringsten des volks; wie wilde, diens hand vulde hij, en werd een van de priesters der hoogten.
\par 34 En hij werd in deze zaak het huis van Jerobeam tot zonde, om hetzelve te doen afsnijden en te verdelgen van den aardbodem.

\chapter{14}

\par 1 Te dierzelfder tijd was Abia, de zoon van Jerobeam, krank.
\par 2 En Jerobeam zeide tot zijn huisvrouw: Maak u nu op, en verstel u, dat men niet merkte, dat gij Jerobeams huisvrouw zijt, en ga heen naar Silo, zie, daar is de profeet Ahia, die van mij gesproken heeft, dat ik koning zou zijn over dit volk.
\par 3 En neem in uw hand tien broden, en koeken, en een kruik honig, en ga tot hem; hij zal u te kennen geven, wat dezen jongen geschieden zal.
\par 4 En Jerobeams huisvrouw deed alzo, en maakte zich op, en ging naar Silo, en kwam in het huis van Ahia. Ahia nu kon niet zien, want zijn ogen stonden stijf vanwege zijn ouderdom.
\par 5 Maar de HEERE zeide tot Ahia: Zie, Jerobeams huisvrouw komt, om een zaak van u te vragen, aangaande haar zoon, want hij is krank; zo en zo zult gij tot haar spreken, en het zal zijn, als zij inkomt, dat zij zich vreemd aanstellen zal.
\par 6 En het geschiedde, als Ahia het geruis harer voeten hoorde, toen zij ter deure inkwam, dat hij zeide: Kom in, gij huisvrouw van Jerobeam! Waarom stelt gij u dus vreemd aan? Want ik ben tot u gezonden met een harde boodschap.
\par 7 Ga heen, zeg Jerobeam: Zo zegt de HEERE, de God Israels: Daarom, dat Ik u verheven heb uit het midden des volks, en u tot een voorganger over Mijn volk Israel gesteld heb;
\par 8 En het koninkrijk van het huis van David gescheurd, en dat u gegeven heb, en gij niet geweest zijt, gelijk Mijn knecht David, die Mijn geboden hield, en die Mij met zijn ganse hart navolgde, om te doen alleen wat recht is in Mijn ogen;
\par 9 Maar kwaad gedaan hebt, doende des meer dan allen, die voor u geweest zijn, en henengegaan zijt, en hebt u andere goden en gegotene beelden gemaakt, om Mij tot toorn te verwekken, en hebt Mij achter uw rug geworpen;
\par 10 Daarom, zie, Ik zal kwaad over het huis van Jerobeam brengen, en van Jerobeam uitroeien wat mannelijk is, den beslotene en verlatene in Israel; en Ik zal de nakomelingen van het huis van Jerobeam wegdoen, gelijk de drek weggedaan wordt, totdat het ganselijk vergaan zij.
\par 11 Die van Jerobeam in de stad sterft, zullen de honden eten; en die in het veld sterft, zullen de vogelen des hemels eten; want de HEERE heeft het gesproken.
\par 12 Gij dan maak u op, ga naar uw huis; als uw voeten in de stad zullen gekomen zijn, zo zal het kind sterven.
\par 13 En gans Israel zal hem beklagen, en hem begraven; want deze alleen van Jerobeam zal in het graf komen, omdat in hem wat goeds voor den HEERE, den God Israels, in het huis van Jerobeam gevonden is.
\par 14 Doch de HEERE zal Zich een koning verwekken over Israel, die het huis van Jerobeam ten zelfden dage uitroeien zal; maar wat zal het ook nu zijn?
\par 15 De HEERE zal ook Israel slaan, gelijk een riet in het water omgedreven wordt, en zal Israel uitrukken uit dit goede land, dat Hij hun vaderen gegeven heeft, en zal hen verstrooien op gene zijde der rivier; daarom dat zij hun bossen gemaakt hebben, den HEERE tot toorn verwekkende.
\par 16 En Hij zal Israel overgeven, om Jerobeams zonden wil, die gezondigd heeft, en die Israel heeft doen zondigen.
\par 17 Toen maakte zich Jerobeams vrouw op, en ging heen, en kwam te Thirza; als zij nu op den dorpel van het huis kwam, zo stierf de jongeling.
\par 18 En zij begroeven hem, en gans Israel beklaagde hem; naar het woord des HEEREN, dat Hij gesproken had door den dienst van Zijn knecht Ahia, den profeet.
\par 19 Het overige nu der geschiedenissen van Jerobeam, hoe hij gekrijgd, en hoe hij geregeerd heeft, ziet, die zijn geschreven in het boek der kronieken der koningen van Israel.
\par 20 De dagen nu, die Jerobeam heeft geregeerd, zijn twee en twintig jaren; en hij ontsliep met zijn vaderen, en Nadab, zijn zoon, regeerde in zijn plaats.
\par 21 Rehabeam nu, de zoon van Salomo, regeerde in Juda; een en veertig jaren was Rehabeam oud, als hij koning werd, en regeerde zeventien jaren te Jeruzalem, in de stad, die de HEERE verkoren had uit al de stammen van Israel, om Zijn Naam daar te zetten; en de naam zijner moeder was Naama, de Ammonietische.
\par 22 En Juda deed, wat kwaad was in de ogen des HEEREN, en zij verwekten Hem tot ijver, meer dan al hun vaderen gedaan hadden, met hun zonden, die zij zondigden.
\par 23 Want ook zij bouwden zich hoogten, en opgerichte beelden, en bossen, op allen hogen heuvel, en onder allen groenen boom.
\par 24 Er waren ook schandjongens in het land; zij deden naar al de gruwelen der heidenen, die de HEERE van het aangezicht der kinderen Israels uit de bezitting verdreven had.
\par 25 Het geschiedde nu in het vijfde jaar van den koning Rehabeam, dat Sisak, de koning van Egypte, optoog tegen Jeruzalem.
\par 26 En hij nam de schatten van het huis des HEEREN, en de schatten van het huis des konings weg, ja, hij nam alles weg; hij nam ook al de gouden schilden weg, die Salomo gemaakt had.
\par 27 En de koning Rehabeam maakte, in plaats van die, koperen schilden; en hij beval die onder de hand van de oversten der trawanten, die de deur van het huis des konings bewaarden.
\par 28 En het geschiedde, zo wanneer de koning in het huis des HEEREN ging, dat de trawanten dezelve droegen, en die wederbrachten in der trawanten wachtkamer.
\par 29 Het overige nu der geschiedenissen van Rehabeam, en al wat hij gedaan heeft, zijn die niet geschreven in het boek der kronieken der koningen van Juda?
\par 30 En er was krijg tussen Rehabeam en tussen Jerobeam, al hun dagen.
\par 31 En Rehabeam ontsliep met zijn vaderen, en werd begraven bij zijn vaderen in de stad Davids; en de naam zijner moeder was Naama, de Ammonietische; en zijn zoon Abiam regeerde in zijn plaats.

\chapter{15}

\par 1 In het achttiende jaar nu van den koning Jerobeam, den zoon van Nebat, werd Abiam koning over Juda.
\par 2 Hij regeerde drie jaren te Jeruzalem; en de naam zijner moeder was Maacha, een dochter van Abisalom.
\par 3 En hij wandelde in al de zonden zijns vaders, die hij voor hem gedaan had; en zijn hart was niet volkomen met den HEERE, zijn God, gelijk het hart van zijn vader David.
\par 4 Maar om Davids wil, gaf de HEERE, zijn God, hem een lamp in Jeruzalem, verwekkende zijn zoon na hem, en bevestigende Jeruzalem.
\par 5 Omdat David gedaan had wat recht was in de ogen des HEEREN, en niet geweken was van alles, wat Hij hem geboden had, al de dagen zijns levens, dan alleen in de zaak van Uria, den Hethiet.
\par 6 En er was krijg geweest tussen Rehabeam en tussen Jerobeam, al de dagen zijns levens.
\par 7 Het overige nu der geschiedenissen van Abiam, en alles, wat hij gedaan heeft, is dat niet geschreven in het boek der kronieken der koningen van Juda? Er was ook krijg tussen Abiam en tussen Jerobeam.
\par 8 En Abiam ontsliep met zijn vaderen, en zij begroeven hem in de stad Davids; en Asa, zijn zoon, regeerde in zijn plaats.
\par 9 In het twintigste jaar van Jerobeam, den koning van Israel, werd Asa koning over Juda.
\par 10 En hij regeerde een en veertig jaren te Jeruzalem, en de naam zijner moeder was Maacha, een dochter van Abisalom.
\par 11 En Asa deed wat recht was in de ogen des HEEREN, gelijk zijn vader David.
\par 12 Want hij nam weg de schandjongens uit het land, en deed weg al de drekgoden, die zijn vaders gemaakt hadden.
\par 13 Ja, zelfs zijn moeder Maacha zette hij ook af, dat zij geen koningin ware, omdat zij een afgrijselijken afgod in een bos gemaakt had; ook roeide Asa uit haar afgrijselijken afgod, en verbrandde hem aan de beek Kidron.
\par 14 De hoogten werden wel niet weggenomen; nochtans was het hart van Asa volkomen met den HEERE, al zijn dagen.
\par 15 En hij bracht in het huis des HEEREN de geheiligde dingen zijns vaders, en zijn geheiligde dingen, zilver, en goud, en vaten.
\par 16 En er was krijg tussen Asa en tussen Baesa, den koning van Israel, al hun dagen.
\par 17 Want Baesa, de koning van Israel, toog op tegen Juda, en bouwde Rama; opdat hij niemand toeliet uit te gaan en in te komen tot Asa, den koning van Juda.
\par 18 Toen nam Asa al het zilver en goud, dat overgebleven was in de schatten van het huis des HEEREN, en de schatten van het huis des konings, en gaf ze in de hand zijner knechten; en de koning Asa zond ze tot Benhadad, den zoon van Tabrimmon, den zoon van Hezion, den koning van Syrie, die te Damaskus woonde, zeggende:
\par 19 Er is een verbond tussen mij en tussen u, tussen mijn vader en tussen uw vader; zie, ik zend u een geschenk, zilver en goud; ga heen, maak uw verbond te niet met Baesa, den koning van Israel, dat hij aftrekke van tegen mij.
\par 20 En Benhadad hoorde naar den koning Asa, en zond de oversten der heiren, die hij had, tegen de steden van Israel; en sloeg Ijon, en Dan, en Abel Beth-maacha, en het ganse Cinneroth, met het ganse land Nafthali.
\par 21 En het geschiedde, als Baesa zulks hoorde, dat hij afliet van Rama te bouwen, en hij bleef te Thirza.
\par 22 Toen liet de koning Asa door gans Juda uitroepen (niemand was vrij), dat zij de stenen van Rama, en het hout daarvan, zouden wegdragen, waarmede Baesa gebouwd had; en de koning Asa bouwde daarmede Geba-benjamins, en Mizpa.
\par 23 Het overige nu van alle geschiedenissen van Asa, en al zijn macht, en al wat hij gedaan heeft, en de steden, die hij gebouwd heeft, zijn die niet geschreven in het boek der kronieken der koningen van Juda? Doch in den tijd zijns ouderdoms werd hij krank aan zijn voeten.
\par 24 En Asa ontsliep met zijn vaderen, en werd begraven met zijn vaderen, in de stad van zijn vader David; en zijn zoon Josafat werd koning in zijn plaats.
\par 25 Nadab nu, de zoon van Jerobeam, werd koning over Israel, in het tweede jaar van Asa, den koning van Juda; en hij regeerde twee jaren over Israel.
\par 26 En hij deed wat kwaad was in de ogen des HEEREN, en wandelde in den weg zijns vaders, en in zijn zonde, waarmede hij Israel had doen zondigen.
\par 27 En Baesa, de zoon van Ahia, van het huis van Issaschar, maakte een verbintenis tegen hem, en Baesa sloeg hem te Gibbethon, hetwelk der Filistijnen is, als Nadab en gans Israel Gibbethon belegerden.
\par 28 En Baesa doodde hem, in het derde jaar van Asa, den koning van Juda, en werd koning in zijn plaats.
\par 29 Het geschiedde nu, als hij regeerde, dat hij het ganse huis van Jerobeam sloeg; hij liet niets over van Jerobeam, wat adem had, totdat hij hem verdelgd had, naar het woord des HEEREN, dat Hij gesproken had door den dienst van Zijn knecht Ahia, den Siloniet;
\par 30 Om de zonden van Jerobeam, die zondigde, en die Israel zondigen deed, en om zijn terging, waarmede hij den HEERE, den God Israels, getergd had.
\par 31 Het overige nu der geschiedenissen van Nadab, en al wat hij gedaan heeft, is dat niet geschreven in het boek der kronieken der koningen van Israel?
\par 32 En er was oorlog tussen Asa en tussen Baesa, den koning van Israel, al hun dagen.
\par 33 In het derde jaar van Asa, koning van Juda, werd Baesa, de zoon van Ahia, koning over gans Israel, te Thirza, en regeerde vier en twintig jaren.
\par 34 En hij deed wat kwaad was in de ogen des HEEREN, en wandelde in den weg van Jerobeam, en in zijn zonde, waarmede hij Israel had doen zondigen.

\chapter{16}

\par 1 Toen geschiedde het woord des HEEREN tot Jehu, den zoon van Hanani, tegen Baesa, zeggende:
\par 2 Daarom, dat Ik u uit het stof verheven, en u tot een voorganger over Mijn volk Israel gesteld heb, en gij gewandeld hebt in den weg van Jerobeam, en Mijn volk Israel hebt doen zondigen, Mij tot toorn verwekkende door hun zonden;
\par 3 Zie, zo zal Ik de nakomelingen van Baesa, en de nakomelingen van zijn huis wegdoen; en Ik zal uw huis maken, gelijk het huis van Jerobeam, den zoon van Nebat.
\par 4 Die van Baesa in de stad sterft, zullen de honden eten, en die van hem in het veld sterft, zullen de vogelen des hemels eten.
\par 5 Het overige nu der geschiedenissen van Baesa, en wat hij gedaan heeft, en zijn macht, zijn die niet geschreven in het boek der kronieken der koningen van Israel?
\par 6 En Baesa ontsliep met zijn vaderen, en werd begraven te Thirza; en zijn zoon Ela regeerde in zijn plaats.
\par 7 Alzo geschiedde ook het woord des HEEREN, door den dienst van den profeet Jehu, den zoon van Hanani, tegen Baesa en tegen zijn huis; en dat om al het kwaad, dat hij gedaan had in de ogen des HEEREN, Hem tot toorn verwekkende door het werk zijner handen, omdat hij was gelijk het huis van Jerobeam, en omdat hij hetzelve verslagen had.
\par 8 In het zes en twintigste jaar van Asa, den koning van Juda, werd Ela, de zoon van Baesa, koning over Israel, te Thirza, en regeerde twee jaren.
\par 9 En Zimri, zijn knecht, overste van de helft der wagenen, maakte een verbintenis tegen hem, als hij te Thirza was, zich dronken drinkende in het huis van Arza, den hofmeester te Thirza;
\par 10 Zo kwam Zimri in, en sloeg hem, en doodde hem, in het zeven en twintigste jaar van Asa, den koning van Juda; en hij werd koning in zijn plaats.
\par 11 En het geschiedde, als hij regeerde, als hij op zijn troon zat, dat hij het ganse huis van Baesa sloeg; hij liet hem niet over die mannelijk was, noch zijn bloedverwanten, noch zijn vrienden.
\par 12 Alzo verdelgde Zimri het ganse huis van Baesa, naar het woord des HEEREN, dat Hij over Baesa gesproken had, door den dienst van den profeet Jehu;
\par 13 Om al de zonden van Baesa, en de zonden van Ela, zijn zoon, waarmede zij gezondigd hadden, en waarmede zij Israel hadden doen zondigen, tot toorn verwekkende den HEERE, den God Israels, door hun ijdelheden.
\par 14 Het overige nu der geschiedenissen van Ela, en al wat hij gedaan heeft, is dat niet geschreven in het boek der kronieken der koningen van Israel?
\par 15 In het zeven en twintigste jaar van Asa, den koning van Juda, regeerde Zimri zeven dagen te Thirza; en het volk had zich gelegerd tegen Gibbethon, dat der Filistijnen is.
\par 16 Het volk nu, dat zich gelegerd had, hoorde zeggen: Zimri heeft een verbintenis gemaakt, ja, heeft ook den koning verslagen; daarom maakte het ganse Israel ten zelfden dage Omri, den krijgsoverste, koning over Israel, in het leger.
\par 17 En Omri toog op, en gans Israel met hem van Gibbethon, en belegerde Thirza.
\par 18 En het geschiedde, als Zimri zag, dat de stad ingenomen was, dat hij ging in het paleis van het huis des konings, en verbrandde boven zich het huis des konings met vuur, en stierf;
\par 19 Om zijn zonden, die hij gezondigd had, doende wat kwaad was in de ogen des HEEREN, wandelende in den weg van Jerobeam, en in zijn zonde, die hij gedaan had, doende Israel zondigen.
\par 20 Het overige nu der geschiedenissen van Zimri, en zijn verbintenis, die hij gemaakt heeft, zijn die niet geschreven in het boek der kronieken der koningen van Israel?
\par 21 Toen werd het volk van Israel verdeeld in twee helften; de helft des volks volgde Tibni, den zoon van Ginath, om hem koning te maken; en de helft volgde Omri.
\par 22 Maar het volk, dat Omri volgde, was sterker dan het volk, dat Tibni, den zoon van Ginath, volgde; en Tibni stierf, en Omri regeerde.
\par 23 In het een en dertigste jaar van Asa, den koning van Juda, werd Omri koning over Israel, en regeerde twaalf jaren; te Thirza regeerde hij zes jaren.
\par 24 En hij kocht den berg Samaria van Semer, voor twee talenten zilvers, en bebouwde den berg; en noemde den naam der stad, die hij bouwde, naar den naam van Semer, den heer des bergs, Samaria.
\par 25 En Omri deed wat kwaad was in de ogen des HEEREN; ja, hij deed erger dan allen, die voor hem geweest waren.
\par 26 En hij wandelde in alle wegen van Jerobeam, den zoon van Nebat, en in zijn zonden, waarmede hij Israel had doen zondigen, verwekkende den HEERE, den God Israels, tot toorn, door hun ijdelheden.
\par 27 Het overige nu der geschiedenissen van Omri, wat hij gedaan heeft, en zijn macht die hij gepleegd heeft, zijn die niet geschreven in het boek der kronieken der koningen van Israel?
\par 28 En Omri ontsliep met zijn vaderen, en werd begraven te Samaria; en zijn zoon Achab regeerde in zijn plaats.
\par 29 En Achab, de zoon van Omri, werd koning over Israel, in het acht en dertigste jaar van Asa, den koning van Juda; en Achab, de zoon van Omri, regeerde over Israel, te Samaria, twee en twintig jaren.
\par 30 En Achab, den zoon van Omri, deed wat kwaad was in de ogen des HEEREN, meer dan allen, die voor hem geweest waren.
\par 31 En het geschiedde (was het een lichte zaak, dat hij wandelde in de zonden van Jerobeam, den zoon van Nebat?), dat hij nog ter vrouwe nam Izebel, de dochter van Eth-baal, den koning der Sidoniers, en heenging, en diende Baal, en boog zich voor hem.
\par 32 En hij richtte voor Baal een altaar op, in het huis van Baal, hetwelk hij te Samaria gebouwd had.
\par 33 Ook maakte Achab een bos, zodat Achab nog meer deed, om den HEERE, den God Israels, tot toorn te verwekken, dan alle koningen van Israel, die voor hem geweest waren.
\par 34 In zijn dagen bouwde Hiel, de Betheliet, Jericho; op Abiram, zijn eerstgeborenen zoon, heeft hij haar gegrondvest, en op Segub, zijn jongsten zoon, heeft hij haar poorten gesteld; naar het woord des HEEREN, dat Hij door den dienst van Jozua, den zoon van Nun, gesproken had.

\chapter{17}

\par 1 En Elia, de Thisbiet, van de inwoneren van Gilead, zeide tot Achab: Zo waarachtig als de HEERE, de God Israels, leeft, voor Wiens aangezicht ik sta, indien deze jaren dauw of regen zijn zal, tenzij dan naar mijn woord!
\par 2 Daarna geschiedde het woord des HEEREN tot hem, zeggende:
\par 3 Ga weg van hier, en wend u naar het oosten, en verberg u aan de beek Krith, die voor aan de Jordaan is.
\par 4 En het zal geschieden, dat gij uit de beek drinken zult; en Ik heb de raven geboden, dat zij u daar onderhouden zullen.
\par 5 Hij ging dan heen, en deed naar het woord des HEEREN; want hij ging en woonde bij de beek Krith, die voor aan de Jordaan is.
\par 6 En de raven brachten hem des morgens brood en vlees, desgelijks brood en vlees des avonds; en hij dronk uit de beek.
\par 7 En het geschiedde ten einde van vele dagen, dat de beek uitdroogde; want geen regen was in het land geweest.
\par 8 Toen geschiedde het woord des HEEREN tot hem, zeggende:
\par 9 Maak u op, ga heen naar Zarfath, dat bij Sidon is, en woon aldaar; zie, Ik heb daar een weduwvrouw geboden, dat zij u onderhoude.
\par 10 Toen maakte hij zich op, en ging naar Zarfath. Als hij nu aan de poort der stad kwam, ziet, zo was daar een weduwvrouw, hout lezende; en hij riep tot haar, en zeide: Haal mij toch een weinig waters in dit vat, dat ik drinke.
\par 11 Toen zij nu heenging om te halen, zo riep hij tot haar, en zeide: Haal mij toch ook een bete broods in uw hand.
\par 12 Maar zij zeide: Zo waarachtig als de HEERE, uw God, leeft, indien ik een koek heb, dan alleen een hand vol meels in de kruik, en een weinig olie in de fles! En zie ik heb een paar houten gelezen, en ik ga heen, en zal het voor mij en voor mijn zoon bereiden, dat wij het eten, en sterven.
\par 13 En Elia zeide tot haar: Vrees niet, ga heen, doe naar uw woord; maar maak mij vooreerst een kleinen koek daarvan, en breng mij dien hier uit; doch voor u en uw zoon zult gij daarna wat maken.
\par 14 Want zo zegt de HEERE, de God Israels: Het meel van de kruik zal niet verteerd worden, en de olie der fles zal niet ontbreken, tot op den dag, dat de HEERE regen op den aardbodem geven zal.
\par 15 En zij ging heen, en deed naar het woord van Elia; zo at zij, en hij, en haar huis, vele dagen.
\par 16 Het meel van de kruik werd niet verteerd, en de olie van de fles ontbrak niet, naar het woord des HEEREN, dat Hij gesproken had door den dienst van Elia.
\par 17 En het geschiedde na deze dingen, dat de zoon dezer vrouw, der waardin van het huis, krank werd; en zijn krankheid werd zeer sterk, totdat geen adem in hem overgebleven was.
\par 18 En zij zeide tot Elia: Wat heb ik met u te doen, gij man Gods? Zijt gij bij mij ingekomen, om mijn ongerechtigheid in gedachtenis te brengen, en om mijn zoon te doden?
\par 19 En hij zeide tot haar: Geef mij uw zoon. En hij nam hem van haar schoot, en droeg hem boven in de opperzaal, waar hij zelf woonde, en hij leide hem neder op zijn bed.
\par 20 En hij riep den HEERE aan, en zeide: HEERE, mijn God, hebt Gij dan ook deze weduwe, bij dewelke ik herberge, zo kwalijk gedaan, dat Gij haar zoon gedood hebt?
\par 21 En hij mat zich driemaal uit over dat kind, en riep den HEERE aan, en zeide: HEERE, mijn God, laat toch de ziel van dit kind in hem wederkomen.
\par 22 En de HEERE verhoorde de stem van Elia; en de ziel van het kind kwam weder in hem, dat het weder levend werd.
\par 23 En Elia nam het kind, en bracht het af van de opperzaal in het huis, en gaf het aan zijn moeder; en Elia zeide: Zie, uw zoon leeft.
\par 24 Toen zeide die vrouw tot Elia: Nu weet ik, dat gij een man Gods zijt, en dat het woord des HEEREN in uw mond waarheid is.

\chapter{18}

\par 1 En het gebeurde na vele dagen, dat het woord des HEEREN geschiedde tot Elia, in het derde jaar, zeggende: Ga heen, vertoon u aan Achab; want Ik zal regen geven op den aardbodem.
\par 2 En Elia ging heen, om zich aan Achab te vertonen. En de honger was sterk in Samaria.
\par 3 En Achab had Obadja, den hofmeester, geroepen; en Obadja was den HEERE zeer vrezende.
\par 4 Want het geschiedde, als Izebel de profeten des HEEREN uitroeide, dat Obadja honderd profeten nam, en verborg ze bij vijftig man in een spelonk, en onderhield hen met brood en water.
\par 5 En Achab had gezegd tot Obadja: Trek door het land, tot alle waterfonteinen en tot alle rivieren; misschien zullen wij gras vinden, opdat wij de paarden en de muilezelen in het leven behouden, en niets uitroeien van de beesten.
\par 6 En zij deelden het land onder zich, dat zij het doortogen; Achab ging bijzonder op een weg, en Obadja ging ook bijzonder op een weg.
\par 7 Als nu Obadja op den weg was, ziet, zo was hem Elia tegemoet; en hem kennende, zo viel hij op zijn aangezicht, en zeide: Zijt gij mijn heer Elia?
\par 8 Hij zeide: Ik ben het; ga heen, zeg uw heer: Zie, Elia is hier.
\par 9 Maar hij zeide: Wat heb ik gezondigd, dat gij uw knecht geeft in de hand van Achab, dat hij mij dode?
\par 10 Zo waarachtig als de HEERE, uw God, leeft, zo er een volk of koninkrijk is, waar mijn heer niet gezonden heeft, om u te zoeken; en als zij zeiden: Hij is hier niet; zo nam hij dat koninkrijk en dat volk een eed af; dat zij u niet hadden gevonden.
\par 11 En nu zegt gij: Ga heen, zeg uw heer: Zie, Elia is hier.
\par 12 En het mocht geschieden, wanneer ik van u zou weggegaan zijn, dat de Geest des HEEREN u wegnam, ik weet niet waarheen; en ik kwam, om dat Achab aan te zeggen, en hij vond u niet, zo zou hij mij doden; ik, uw knecht, nu vrees den HEERE van mijn jonkheid af.
\par 13 Is mijn heer niet aangezegd, wat ik gedaan heb, als Izebel de profeten des HEEREN doodde? Dat ik van de profeten des HEEREN honderd man heb verborgen, elk vijftig man in een spelonk, en die met brood en water onderhouden heb?
\par 14 En nu zegt gij: Ga heen, zeg uw heer: Zie, Elia is hier, en hij zou mij doodslaan.
\par 15 En Elia zeide: Zo waarachtig als de HEERE der heirscharen leeft, voor Wiens aangezicht ik sta, ik zal voorzeker mij heden aan hem vertonen!
\par 16 Toen ging Obadja Achab tegemoet, en zeide het hem aan; en Achab ging Elia tegemoet.
\par 17 En het geschiedde, als Achab Elia zag, dat Achab tot hem zeide: Zijt gij die beroerder van Israel?
\par 18 Toen zeide hij: Ik heb Israel niet beroerd, maar gij en uws vaders huis, daarmede, dat gijlieden de geboden des HEEREN verlaten hebt en de Baals nagevolgd zijt.
\par 19 Nu dan, zend heen, verzamel tot mij het ganse Israel op den berg Karmel, en de vierhonderd en vijftig profeten van Baal, en de vierhonderd profeten van het bos, die van de tafel van Izebel eten.
\par 20 Zo zond Achab onder alle kinderen Israels, en verzamelde de profeten op den berg Karmel.
\par 21 Toen naderde Elia tot het ganse volk, en zeide: Hoe lang hinkt gij op twee gedachten? Zo de HEERE God is, volgt Hem na, en zo het Baal is, volgt hem na! Maar het volk antwoordde hem niet een woord.
\par 22 Toen zeide Elia tot het volk: Ik ben alleen een profeet des HEEREN overgebleven, en de profeten van Baal zijn vierhonderd en vijftig mannen.
\par 23 Dat men ons dan twee varren geve, en dat zij voor zich den enen var kiezen, en denzelven in stukken delen, en op het hout leggen, maar geen vuur daaraan leggen; en ik zal den anderen var bereiden, en op het hout leggen, en geen vuur daaraan leggen.
\par 24 Roept gij daarna den naam van uw god aan, en ik zal den Naam des HEEREN aanroepen; en de God, Die door vuur antwoorden zal, Die zal God zijn. En het ganse volk antwoordde en zeide: Dat woord is goed.
\par 25 En Elia zeide tot de profeten van Baal: Kiest gijlieden voor u den enen var, en bereidt gij hem eerst, want gij zijt velen; en roept den naam uws gods aan, en legt geen vuur daaraan.
\par 26 En zij namen den var, dien hij hun gegeven had, en bereidden hem, en riepen den naam van Baal aan, van den morgen tot op den middag, zeggende: O Baal, antwoord ons! Maar er was geen stem en geen antwoorder. En zij sprongen tegen het altaar, dat men gemaakt had.
\par 27 En het geschiedde op den middag, dat Elia met hen spotte, en zeide: Roept met luider stem, want hij is een god; omdat hij in gepeins is, of omdat hij wat te doen heeft, of omdat hij een reize heeft; misschien slaapt hij en zal wakker worden.
\par 28 En zij riepen met luider stem, en zij sneden zichzelven met messen en met priemen, naar hun wijze, totdat zij bloed over zich uitstortten.
\par 29 Het geschiedde nu, als de middag voorbij was, dat zij profeteerden totdat men het spijsoffer zou offeren; maar er was geen stem, en geen antwoorder, en geen opmerking.
\par 30 Toen zeide Elia tot het ganse volk: Nadert tot mij. En al het volk naderde tot hem; en hij heelde het altaar des HEEREN, dat verbroken was.
\par 31 En Elia nam twaalf stenen, naar het getal der stammen van de kinderen Jakobs, tot welke het woord des HEEREN geschied was, zeggende: Israel zal uw naam zijn.
\par 32 En hij bouwde met die stenen het altaar in den Naam des HEEREN; daarna maakte hij een groeve rondom het altaar, naar de wijdte van twee maten zaads.
\par 33 En hij schikte het hout, en deelde den var in stukken, en leide hem op het hout.
\par 34 En hij zeide: Vult vier kruiken met water, en giet het op het brandoffer en op het hout. En hij zeide: Doet het ten tweeden male. En zij deden het ten tweeden male. Voorts zeide hij: Doet het ten derden male. En zij deden het ten derden male;
\par 35 Dat het water rondom het altaar liep; daartoe vulde hij ook de groeve met water.
\par 36 Het geschiedde nu, als men het spijsoffer offerde, dat de profeet Elia naderde, en zeide: HEERE, God van Abraham, Izak en Israel, dat het heden bekend worde, dat Gij God in Israel zijt, en ik Uw knecht; en dat ik al deze dingen naar Uw woord gedaan heb.
\par 37 Antwoord mij, HEERE, antwoord mij; opdat dit volk erkenne, dat Gij, o HEERE, die God zijt, en dat Gij hun hart achterwaarts omgewend hebt.
\par 38 Toen viel het vuur de HEEREN, en verteerde dat brandoffer, en dat hout, en die stenen, en dat stof, ja, lekte dat water op, hetwelk in de groeve was.
\par 39 Als nu het ganse volk dat zag, zo vielen zij op hun aangezichten, en zeiden: De HEERE is God, de HEERE is God!
\par 40 En Elia zeide tot hen: Grijpt de profeten van Baal, dat niemand van hen ontkome. En zij grepen ze; en Elia voerde hen af naar de beek Kison, en slachtte hen aldaar.
\par 41 Daarna zeide Elia tot Achab: Trek op, eet en drink; want er is een geruis van een overvloedigen regen.
\par 42 Alzo toog Achab op, om te eten en te drinken; maar Elia ging op naar de hoogte van Karmel, en breidde zich uit voorwaarts ter aarde; daarna leide hij zijn aangezicht tussen zijn knieen.
\par 43 En hij zeide tot zijn jongen: Ga nu op, en zie uit naar de zee. Toen ging hij op, en zag uit, en zeide: Er is niets. Toen zeide hij: Ga weder henen, zevenmaal.
\par 44 En het geschiedde op de zevende maal, dat hij zeide: Zie, een kleine wolk, als eens mans hand, gaat op van de zee. En hij zeide: Ga op, zeg tot Achab: Span aan, en kom af, dat u de regen niet ophoude.
\par 45 En het geschiedde ondertussen, dat de hemel van wolken en wind zwart werd; en er kwam een grote regen; en Achab reed weg, en toog naar Jizreel.
\par 46 En de hand des HEEREN was over Elia, en hij gordde zijn lenden, en liep voor het aangezicht van Achab henen, tot daar men te Jizreel komt.

\chapter{19}

\par 1 En Achab zeide Izebel aan al wat Elia gedaan had, en allen, die hij gedood had, te weten al de profeten, met het zwaard.
\par 2 Toen zond Izebel een bode tot Elia, om te zeggen: Zo doen mij de goden, en doen zo daartoe, voorzeker, ik zal morgen omtrent dezen tijd uw ziel stellen, als de ziel van een hunner.
\par 3 Toen hij dat zag, maakte hij zich op, en ging heen, om zijns levens wil, en kwam te Ber-seba, dat in Juda is, en liet zijn jongen aldaar.
\par 4 Maar hij zelf ging henen in de woestijn een dagreis, en kwam, en zat onder een jeneverboom; en bad, dat zijn ziel stierve, en zeide: Het is genoeg; neem nu, HEERE, mijn ziel, want ik ben niet beter dan mijn vaderen.
\par 5 En hij leide zich neder, en sliep onder een jeneverboom; en ziet, toen roerde hem een engel aan, en zeide tot hem: Sta op, eet;
\par 6 En hij zag om, en ziet, aan zijn hoofdeinde was een koek op de kolen gebakken, en een fles met water; alzo at hij, en dronk, en leide zich wederom neder.
\par 7 En de engel des HEEREN kwam ten anderen male weder, en roerde hem aan, en zeide: Sta op, eet, want de weg zou te veel voor u zijn.
\par 8 Zo stond hij op, en at, en dronk; en hij ging, door de kracht derzelver spijs, veertig dagen en veertig nachten, tot aan den berg Gods, Horeb.
\par 9 En hij kwam aldaar in een spelonk, en vernachtte aldaar; en ziet, het woord des HEEREN geschiedde tot hem, en zeide tot hem: Wat maakt gij hier, Elia?
\par 10 En hij zeide: Ik heb zeer geijverd voor den HEERE, den God der heirscharen; want de kinderen Israels hebben Uw verbond verlaten, Uw altaren afgebroken en Uw profeten met het zwaard gedood; en ik alleen ben overgebleven, en zij zoeken mijn ziel, om die weg te nemen.
\par 11 En Hij zeide: Ga uit, en sta op dezen berg, voor het aangezicht des HEEREN. En ziet, de HEERE ging voorbij, en een grote en sterke wind, scheurende de bergen, en brekende de steenrotsen, voor den HEERE henen; doch de HEERE was in den wind niet; en na dezen wind een aardbeving; de HEERE was ook in de aardbeving niet;
\par 12 En na de aardbeving een vuur; de HEERE was ook in het vuur niet; en na het vuur het suizen van een zachte stilte.
\par 13 En het geschiedde, als Elia dat hoorde, dat hij zijn aangezicht bewond met zijn mantel, en uitging, en stond in den ingang der spelonk. En ziet, een stem kwam tot hem, die zeide: Wat maakt gij hier, Elia?
\par 14 En hij zeide: Ik heb zeer geijverd voor den HEERE, den God der heirscharen; want de kinderen Israels hebben Uw verbond verlaten, Uw altaren afgebroken en Uw profeten met het zwaard gedood; en ik alleen ben overgebleven, en zij zoeken mijn ziel, om die weg te nemen.
\par 15 En de HEERE zeide tot hem: Ga, keer weder op uwe weg, naar de woestijn van Damaskus; en ga daar in, en zalf Hazael ten koning over Syrie.
\par 16 Daartoe zult gij Jehu, den zoon van Nimsi, zalven ten koning over Israel; en Elisa, den zoon van Safat, van Abel-mehola, zult gij tot profeet zalven in uw plaats.
\par 17 En het zal geschieden, dat Jehu hem, die van het zwaard van Hazael ontkomt, doden zal; en die van het zwaard van Jehu ontkomt, dien zal Elisa doden.
\par 18 Ook heb Ik in Israel doen overblijven zeven duizend, alle knieen, die zich niet gebogen hebben voor Baal, en allen mond, die hem niet gekust heeft.
\par 19 Zo ging hij van daar, en vond Elisa, den zoon van Safat; dezelve ploegde met twaalf juk runderen voor zich henen, en hij was bij het twaalfde; en Elia ging over tot hem, en wierp zijn mantel op hem.
\par 20 En hij verliet de runderen, en liep Elia na, en zeide: Dat ik toch mijn vader en mijn moeder kusse, daarna zal ik u navolgen. En hij zeide tot hem: Ga, keer weder; want wat heb ik u gedaan?
\par 21 Zo keerde hij weder van achter hem af, en nam een juk runderen, en slachtte het, en met het gereedschap der runderen zood hij hun vlees, hetwelk hij aan het volk gaf; en zij aten. Daarna stond hij op, en volgde Elia na, en diende hem.

\chapter{20}

\par 1 En Benhadad, de koning van Syrie, vergaderde al zijn macht; en twee en dertig koningen waren met hem, en paarden en wagenen; en hij toog op, en belegerde Samaria en krijgde tegen haar.
\par 2 En hij zond boden tot Achab, den koning van Israel, in de stad.
\par 3 En hij zeide hem aan: Zo zegt Benhadad: Uw zilver en uw goud, dat is mijn, daartoe uw vrouwen en uw beste kinderen, die zijn mijn.
\par 4 En de koning van Israel antwoordde en zeide: Naar uw woord, mijn heer de koning, ik ben uwe, en al wat ik heb.
\par 5 Daarna kwamen de boden weder, en zeiden: Alzo spreekt Benhadad, zeggende: Ik heb wel tot u gezonden, zeggende: Uw zilver, en uw goud, en uw vrouwen, en uw kinderen zult gij mij geven;
\par 6 Maar morgen om dezen tijd zal ik mijn knechten tot u zenden, dat zij uw huis en de huizen uwer knechten bezoeken; en het zal geschieden, dat zij al het begeerlijke uwer ogen in hun handen leggen en wegnemen zullen.
\par 7 Toen riep de koning van Israel alle oudsten des lands, en zeide: Merkt toch en ziet, dat deze het kwade zoekt; want hij had tot mij gezonden, om mijn vrouwen, en om mijn kinderen, en om mijn zilver, en om mijn goud, en ik heb het hem niet geweigerd.
\par 8 Doch al de oudsten, en het ganse volk, zeiden tot hem: Hoor niet, en bewillig niet.
\par 9 Daarom zeide hij tot de boden van Benhadad: Zegt mijn heer den koning: Alles, waarom gij in het eerst tot uw knecht gezonden hebt, zal ik doen; maar deze zaak kan ik niet doen. Zo gingen de boden heen en brachten hem bescheid weder.
\par 10 En Benhadad zond tot hem en zeide: De goden doen mij zo, en doen zo daartoe, indien het stof van Samaria genoeg zal zijn tot handvollen voor al het volk, dat mijn voetstappen volgt!
\par 11 Maar de koning van Israel antwoordde en zeide: Spreekt tot hem: Die zich aangordt, beroeme zich niet, als die zich los maakt.
\par 12 En het geschiedde, als hij dit woord hoorde, daar hij was drinkende, hij en de koningen in de tenten, dat hij zeide tot zijn knechten: Legt aan! En zij leiden aan tegen de stad.
\par 13 En ziet, een profeet trad tot Achab, den koning van Israel, en zeide: Zo zegt de HEERE: Hebt gij gezien al deze grote menigte? Zie, Ik zal ze heden in uw hand geven, opdat gij weet, dat Ik de HEERE ben.
\par 14 En Achab zeide: Door wie? En hij zeide: Zo zegt de HEERE: Door de jongens van de oversten der landschappen. En hij zeide: Wie zal den strijd aanbinden? En hij zeide: Gij.
\par 15 Toen telde hij de jongens van de oversten der landschappen, en zij waren tweehonderd twee en dertig; en na hen telde hij al het volk, al de kinderen Israels, zeven duizend.
\par 16 En zij togen uit op den middag. Benhadad nu dronk zich dronken in de tenten, hij en de koningen, de twee en dertig koningen, die hem hielpen.
\par 17 En de jongens van de oversten der landschappen togen eerst uit. Doch Benhadad zond enigen uit, en zij boodschapten hem, zeggende: Uit Samaria zijn mannen uitgetogen.
\par 18 En hij zeide: Hetzij dat zij tot vrede uitgetogen zijn, grijpt hen levend; hetzij ook, dat zij ten strijde uitgetogen zijn, grijpt hen levend.
\par 19 Zo togen deze jongens van de oversten der landschappen uit de stad, en het heir, dat hen navolgde.
\par 20 En een ieder sloeg zijn man, zodat de Syriers vloden, en Israel jaagde hen na. Doch Benhadad, de koning van Syrie, ontkwam op een paard, met enige ruiteren.
\par 21 En de koning van Israel toog uit, en sloeg paarden en wagenen, dat hij een groten slag aan de Syriers sloeg.
\par 22 Toen trad die profeet tot den koning van Israel, en zeide tot hem: Ga heen, sterk u; en bemerk, en zie, wat gij doen zult; want met de wederkomst des jaars zal de koning van Syrie tegen u optrekken.
\par 23 Want de knechten van den koning van Syrie hadden tot hem gezegd: Hun goden zijn berggoden, daarom zijn zij sterker geweest dan wij; maar zeker, laat ons tegen hen op het effen veld strijden, zo wij niet sterker zijn dan zij!
\par 24 Daarom doe deze zaak: Doe de koningen weg, elkeen uit zijn plaats, en stel landvoogden in hun plaats.
\par 25 En gij, tel u een heir, als dat heir, dat van de uwen gevallen is, en paarden, als die paarden, en wagenen, als die wagenen; en laat ons tegen hen op het effen veld strijden, zo wij niet sterker zijn dan zij! En hij hoorde naar hun stem, en deed alzo.
\par 26 Het geschiedde nu met de wederkomst des jaars, dat Benhadad de Syriers monsterde; en hij toog op naar Afek, ten krijge tegen Israel.
\par 27 De kinderen Israels werden ook gemonsterd, en waren verzorgd van leeftocht, en trokken hun tegemoet; en de kinderen Israels legerden zich tegenover hen, als twee blote geitenkudden, maar de Syriers vervulden het land.
\par 28 En de man Gods trad toe, en sprak tot den koning van Israel, en zeide: Zo zegt de HEERE: Daarom dat de Syriers gezegd hebben: De HEERE is een God der bergen, en Hij is niet een God der laagten; zo zal Ik al deze grote menigte in uw hand geven, opdat gijlieden weet, dat Ik de HEERE ben.
\par 29 En dezen waren gelegerd tegenover die, zeven dagen; het geschiedde nu op den zevenden dag, dat de strijd aanging; en de kinderen Israels sloegen van de Syriers honderd duizend voetvolks op een dag.
\par 30 En de overgeblevenen vloden naar Afek in de stad, en de muur viel op zeven en twintig duizend mannen, die overgebleven waren; ook vlood Benhadad, en kwam in de stad van kamer in kamer.
\par 31 Toen zeiden de knechten tot hem: Zie toch, wij hebben gehoord, dat de koningen van het huis Israels goedertierene koningen zijn; laat ons toch zakken om onze lenden leggen, en koorden om onze hoofden, en uitgaan tot den koning van Israel; mogelijk zal hij uw ziel in het leven behouden.
\par 32 Toen gordden zij zakken om hun lenden, en koorden om hun hoofden, en kwamen tot den koning van Israel, en zeiden: Uw knecht Benhadad zegt: Laat toch mijn ziel leven. En hij zeide: Leeft hij dan nog? Hij is mijn broeder.
\par 33 De mannen nu namen naarstiglijk waar, en vatten het haastelijk, of het van hem ware, en zeiden: Uw broeder Benhadad leeft. En hij zeide: Komt, brengt hem. Toen kwam Benhadad tot hem uit, en hij deed hem op den wagen klimmen.
\par 34 En hij zeide tot hem: De steden, die mijn vader van uw vader genomen heeft, zal ik wedergeven, en maak u straten in Damaskus, gelijk mijn vader in Samaria gemaakt heeft. En ik, antwoordde Achab, zal u met dit verbond dan laten gaan. Zo maakte hij een verbond met hem, en liet hem gaan.
\par 35 Toen zeide een man uit de zonen der profeten tot zijn naaste, door het woord des HEEREN: Sla mij toch. En de man weigerde hem te slaan.
\par 36 En hij zeide tot hem: Daarom dat gij de stem des HEEREN niet gehoorzaam zijt geweest, zie, als gij van mij weggegaan zijt, zo zal u een leeuw slaan. En als hij van bij hem weggegaan was, zo vond hem een leeuw, die hem sloeg.
\par 37 Daarna vond hij een anderen man, en zeide: Sla mij toch. En die man sloeg hem, slaande en wondende.
\par 38 Toen ging de profeet heen, en stond voor den koning op den weg; en hij verstelde zich met as boven zijn ogen.
\par 39 En het geschiedde, als de koning voorbijging, dat hij tot den koning riep, en zeide: Uw knecht was uitgegaan in het midden des strijds; en zie, een man was afgeweken, en bracht tot mij een man, en zeide: Bewaar dezen man, indien hij enigszins gemist wordt, zo zal uw ziel in de plaats zijner ziel zijn, of gij zult een talent zilvers opwegen.
\par 40 Het geschiedde nu, als uw knecht hier en daar doende was, dat hij er niet was. Toen zeide de koning van Israel tot hem: Zo is uw oordeel; gij hebt zelf het geveld.
\par 41 Toen haastte hij zich, en deed de as af van zijn ogen; en de koning van Israel kende hem, dat hij een der profeten was.
\par 42 En hij zeide tot hem: Zo zegt de HEERE: Omdat gij den man, dien Ik verbannen heb, uit de hand hebt laten gaan, zo zal uw ziel in de plaats van zijn ziel zijn, en uw volk in de plaats van zijn volk.
\par 43 En de koning van Israel toog henen, gemelijk en toornig, naar zijn huis, en kwam te Samaria.

\chapter{21}

\par 1 Het geschiedde nu na deze dingen, alzo Naboth, een Jizreeliet, een wijngaard had, die te Jizreel was, bij het paleis van Achab, den koning van Samaria.
\par 2 Dat Achab sprak tot Naboth, zeggende: Geef mij uw wijngaard, opdat hij mij zij tot een kruidhof, dewijl hij nabij mijn huis is; en ik zal u daarvoor geven een wijngaard, die beter is dan die; of, zo het goed in uw ogen is, zal ik u in geld deszelfs waarde geven.
\par 3 Maar Naboth zeide tot Achab: Dat late de HEERE verre van mij zijn, dat ik u de erve mijner vaderen geven zou!
\par 4 Toen kwam Achab in zijn huis, gemelijk en toornig over het woord, dat Naboth, de Jizreeliet, tot hem gesproken had, en gezegd: Ik zal de erve mijner vaderen niet geven. En hij leide zich neder op zijn bed, en keerde zijn aangezicht om, en at geen brood.
\par 5 Maar Izebel, zijn huisvrouw, kwam tot hem, en sprak tot hem: Wat is dit, dat uw geest dus gemelijk is, en dat gij geen brood eet?
\par 6 En hij sprak tot haar: Omdat ik tot Naboth, den Jizreeliet, gesproken en hem gezegd heb: Geef mij uw wijngaard om geld, of, zo het u behaagt, zal ik u een wijngaard in zijn plaats geven; maar hij heeft gezegd: Ik zal u mijn wijngaard niet geven.
\par 7 Toen zeide Izebel, zijn huisvrouw, tot hem: Zoudt gij nu het koninkrijk over Israel regeren? Sta op, eet brood, en uw hart zij vrolijk; ik zal u den wijngaard van Naboth, den Jizreeliet, geven.
\par 8 Zij dan schreef brieven in den naam van Achab, en verzegelde ze met zijn signet; en zond de brieven tot de oudsten en tot de edelen, die in zijn stad waren, wonende met Naboth.
\par 9 En zij schreef in die brieven, zeggende: Roept een vasten uit, en zet Naboth in de hoogste plaats des volks;
\par 10 En zet tegenover hem twee mannen, zonen Belials, die tegen hem getuigen, zeggende: Gij hebt God en den koning gezegend; en voert hem uit, en stenigt hem, dat hij sterve.
\par 11 En de mannen zijner stad, die oudsten en die edelen, die in zijn stad woonden, deden gelijk als Izebel tot hen gezonden had; gelijk als geschreven was in de brieven, die zij tot hen gezonden had.
\par 12 Zij riepen een vasten uit; en zij zetten Naboth in de hoogste plaats des volks.
\par 13 Toen kwamen de twee mannen, zonen Belials, en zetten zich tegenover hem; en de mannen Belials getuigden tegen hem, tegen Naboth, voor het volk, zeggende: Naboth heeft God en den koning gezegend. En zij voerden hem buiten de stad, en stenigden hem met stenen, dat hij stierf.
\par 14 Daarna zonden zij tot Izebel, zeggende: Naboth is gestenigd en is dood.
\par 15 Het geschiedde nu, toen Izebel hoorde, dat Naboth gestenigd en dood was, dat Izebel tot Achab zeide: Sta op, bezit den wijngaard van Naboth, den Jizreeliet, erfelijk, dien hij u weigerde om geld te geven; want Naboth leeft niet, maar is dood.
\par 16 En het geschiedde, als Achab hoorde, dat Naboth dood was, dat Achab opstond, om naar den wijngaard van Naboth, den Jizreeliet, af te gaan, om dien erfelijk te bezitten.
\par 17 Doch het woord des HEEREN geschiedde tot Elia, den Thisbiet, zeggende:
\par 18 Maak u op, ga henen af, Achab, den koning van Israel, tegemoet, die in Samaria is; zie hij is in den wijngaard van Naboth, waarhenen hij afgegaan is, om dien erfelijk te bezitten.
\par 19 En gij zult tot hem spreken, zeggende: Alzo zegt de HEERE: Hebt gij doodgeslagen, en ook een erfelijke bezitting ingenomen? Daartoe zult gij tot hem spreken, zeggende: Alzo zegt de HEERE: In plaats dat de honden het bloed van Naboth gelekt hebben, zullen de honden uw bloed lekken, ja het uwe!
\par 20 En Achab zeide tot Elia: Hebt gij mij gevonden, o, mijn vijand? En hij zeide: Ik heb u gevonden, overmits gij uzelven verkocht hebt, om te doen dat kwaad is in de ogen des HEEREN.
\par 21 Zie, Ik zal kwaad over u brengen, en uw nakomelingen wegdoen; en Ik zal van Achab uitroeien wat mannelijk is, mitsgaders den beslotene en verlatene in Israel.
\par 22 En Ik zal uw huis maken gelijk het huis van Jerobeam, den zoon van Nebat, en gelijk het huis van Baesa, den zoon van Ahia; om de terging, waarmede gij Mij getergd hebt, en dat gij Israel hebt doen zondigen.
\par 23 Verder ook over Izebel sprak de HEERE, zeggende: De honden zullen Izebel eten, aan den voorwal van Jizreel.
\par 24 Die van Achab sterft in de stad, zullen de honden eten; en die in het veld sterft, zullen de vogelen des hemels eten.
\par 25 Doch er was niemand geweest gelijk Achab, die zichzelven verkocht had, om te doen dat kwaad is in de ogen des HEEREN, dewijl Izebel, zijn huisvrouw, hem ophitste.
\par 26 En hij deed zeer gruwelijk, wandelende achter de drekgoden; naar alles, wat de Amorieten gedaan hadden, die God voor het aangezicht van de kinderen Israels uit de bezitting verdreven had.
\par 27 Het geschiedde nu, als Achab deze woorden hoorde, dat hij zijn klederen scheurde, en een zak om zijn vlees leide, en vastte; hij lag ook neder in den zak, en ging langzaam.
\par 28 En het woord des HEEREN geschiedde tot Elia, den Thisbiet, zeggende:
\par 29 Hebt gij gezien, dat Achab zich vernedert voor Mijn aangezicht? Daarom dewijl hij zich vernedert voor Mijn aangezicht, zo zal Ik dat kwaad in zijn dagen niet brengen; in de dagen zijns zoons zal Ik dat kwaad over zijn huis brengen.

\chapter{22}

\par 1 En zij zaten drie jaren stil, dat er geen krijg was tussen Syrie en tussen Israel.
\par 2 Maar het geschiedde in het derde jaar, als Josafat, de koning van Juda, tot den koning van Israel afgekomen was,
\par 3 Dat de koning van Israel tot zijn knechten zeide: Weet gij, dat Ramoth in Gilead onze is? En wij zijn stil, zonder dat te nemen uit de hand van den koning van Syrie.
\par 4 Daarna zeide hij tot Josafat: Zult gij met mij trekken in den strijd naar Ramoth in Gilead? En Josafat zeide tot den koning van Israel: Zo zal ik zijn gelijk gij zijt, zo mijn volk als uw volk, zo mijn paarden als uw paarden.
\par 5 Verder zeide Josafat tot den koning van Israel: Vraag toch als heden naar het woord des HEEREN.
\par 6 Toen vergaderde de koning van Israel de profeten, omtrent vierhonderd man, en hij zeide tot hen: Zal ik tegen Ramoth in Gilead ten strijde trekken, of zal ik het nalaten? En zij zeiden: Trek op, want de HEERE zal ze in de hand des konings geven.
\par 7 Maar Josafat zeide: Is hier niet nog een profeet des HEEREN, dat wij het van hem vragen mochten?
\par 8 Toen zeide de koning van Israel tot Josafat: Er is nog een man, om door hem den HEERE te vragen; maar ik haat hem, omdat hij over mij niets goeds profeteert, maar kwaad: Micha, de zoon van Jimla. En Josafat zeide: De koning zegge niet alzo!
\par 9 Toen riep de koning van Israel een kamerling, en hij zeide: Haal haastelijk Micha, den zoon van Jimla.
\par 10 De koning van Israel nu, en Josafat, de koning van Juda, zaten elk op zijn troon, bekleed met hun klederen, op het plein, aan de deur der poort van Samaria; en al de profeten profeteerden in hun tegenwoordigheid.
\par 11 En Zedekia, de zoon van Kenaana, had zich ijzeren horens gemaakt; en hij zeide: Zo zegt de HEERE: Met deze zult gij de Syriers stoten, totdat gij hen gans verdaan zult hebben.
\par 12 En al de profeten profeteerden alzo, zeggende: Trek op naar Ramoth in Gilead, en gij zult voorspoedig zijn; want de HEERE zal hen in de hand des konings geven.
\par 13 De bode nu, die henengegaan was, om Micha te roepen, sprak tot hem, zeggende: Zie toch, de woorden der profeten zijn uit een mond goed tot den koning; dat toch uw woord zij, gelijk als het woord van een uit hen, en spreek het goede.
\par 14 Doch Micha zeide: Zo waarachtig als de HEERE leeft, hetgeen de HEERE tot mij zeggen zal, dat zal ik spreken.
\par 15 Als hij tot den koning gekomen was, zo zeide de koning tot hem: Micha, zullen wij naar Ramoth in Gilead ten strijde trekken, of zullen wij het nalaten? En hij zeide tot hem: Trek op, en gij zult voorspoedig zijn, want de HEERE zal ze in de hand des konings geven.
\par 16 En de koning zeide tot hem: Tot hoe vele reizen zal ik u bezweren, opdat gij tot mij niet spreekt, dan alleen de waarheid, in den Naam des HEEREN?
\par 17 En hij zeide: Ik zag het ganse Israel verstrooid op de bergen, gelijk schapen, die geen herder hebben; en de HEERE zeide: Dezen hebben geen heer; een iegelijk kere weder naar zijn huis in vrede.
\par 18 Toen zeide de koning van Israel tot Josafat: Heb ik tot u niet gezegd: Hij zal over mij niets goed, maar kwaads profeteren?
\par 19 Verder zeide hij: Daarom hoort het woord des HEEREN: Ik zag den HEERE, zittende op Zijn troon, en al het hemelse heir staande nevens Hem, aan Zijn rechter hand en aan Zijn linkerhand.
\par 20 En de HEERE zeide: Wie zal Achab overreden, dat hij optrekke en valle te Ramoth in Gilead? De een nu zeide aldus, en de andere zeide alzo.
\par 21 Toen ging een geest uit, en stond voor het aangezicht des HEEREN, en zeide: Ik zal hem overreden.
\par 22 En de HEERE zeide tot hem: Waarmede? En hij zeide: Ik zal uitgaan, en een leugengeest zijn in den mond van al zijn profeten. En Hij zeide: Gij zult overreden, en zult het ook vermogen; ga uit en doe alzo.
\par 23 Nu dan, zie, de HEERE heeft een leugengeest in den mond van al deze uw profeten gegeven; en de HEERE heeft kwaad over u gesproken.
\par 24 Toen trad Zedekia, de zoon van Kenaana, toe, en sloeg Micha op het kinnebakken; en hij zeide: Door wat weg is de geest des HEEREN van mij doorgegaan, om u aan te spreken?
\par 25 En Micha zeide: Zie, gij zult het zien, op dienzelfden dag, als gij zult gaan van kamer in kamer, om u te versteken.
\par 26 De koning van Israel nu zeide: Neem Micha, en breng hem weder tot Amon, den overste der stad, en tot Joas, den zoon des konings;
\par 27 En gij zult zeggen: Zo zegt de koning: Zet dezen in het gevangenhuis, en spijst hem met brood der bedruktheid, en met water der bedruktheid, totdat ik met vrede weder kom.
\par 28 En Micha zeide: Indien gij enigszins met vrede wederkomt, zo heeft de HEERE door mij niet gesproken! Verder zeide hij: Hoort, gij volken altegaar!
\par 29 Alzo toog de koning van Israel en Josafat, de koning van Juda, op naar Ramoth in Gilead.
\par 30 En de koning van Israel zeide tot Josafat: Als ik mij versteld heb, zal ik in den strijd komen; maar gij, trek uw klederen aan. Alzo verstelde zich de koning van Israel, en kwam in den strijd.
\par 31 De koning nu van Syrie had geboden aan de oversten der wagenen, van welke hij twee en dertig had, zeggende: Gij zult noch kleinen noch groten bestrijden, maar den koning van Israel alleen.
\par 32 Het geschiedde dan, als de oversten der wagenen Josafat zagen, dat zij zeiden: Gewisselijk, die is de koning van Israel, en zij keerden zich naar hem, om te strijden; maar Josafat riep uit.
\par 33 En het geschiedde, als de oversten der wagenen zagen, dat hij de koning van Israel niet was, dat zij zich van achter hem afkeerden.
\par 34 Toen spande een man den boog in zijn eenvoudigheid, en schoot den koning van Israel tussen de gespen en tussen het pantsier. Toen zeide hij tot zijn voerman: Keer uw hand, en voer mij uit het leger, want ik ben zeer verwond.
\par 35 En de strijd nam op denzelven dag toe, en de koning werd met den wagen staande gehouden tegenover de Syriers; maar hij stierf des avonds, en het bloed der wonde vloeide in den bak des wagens.
\par 36 En er ging een uitroeping door het heirleger, als de zon onderging, zeggende: Een ieder kere naar zijn stad, en een ieder naar zijn land!
\par 37 Alzo stierf de koning, en werd naar Samaria gebracht; en zij begroeven den koning te Samaria.
\par 38 Als men nu den wagen in den vijver van Samaria spoelde, lekten de honden zijn bloed, waar de hoeren wiesen, naar het woord des HEEREN, dat Hij gesproken had.
\par 39 Het overige nu der geschiedenissen van Achab, en al wat hij gedaan heeft, en het elpenbenen huis, dat hij gebouwd heeft, en al de steden, die hij gebouwd heeft, zijn die niet geschreven in het boek der kronieken der koningen van Israel?
\par 40 Alzo ontsliep Achab met zijn vaderen; en zijn zoon Ahazia werd koning in zijn plaats.
\par 41 Josafat nu, de zoon van Asa, werd koning over Juda, in het vierde jaar van Achab, den koning van Israel.
\par 42 Josafat was vijf en dertig jaren oud, als hij koning werd, en regeerde vijf en twintig jaren te Jeruzalem; en de naam zijner moeder was Azuba, de dochter van Silchi.
\par 43 En hij wandelde in al den weg van zijn vader Asa; hij week niet daarvan, doende dat recht was in de ogen des HEEREN.
\par 44 Evenwel werden de hoogten niet weggenomen; het volk offerde en rookte nog op de hoogten.
\par 45 En Josafat maakte vrede met den koning van Israel.
\par 46 Het overige nu der geschiedenissen van Josafat, en zijn macht, die hij bewezen heeft, en hoe hij geoorloogd heeft, zijn die niet geschreven in het boek der kronieken der koningen van Juda?
\par 47 Ook deed hij uit het land weg de overige schandjongens, die in de dagen van zijn vader Asa overgebleven waren.
\par 48 Toen was er geen koning in Edom, maar een stadhouder des konings.
\par 49 En Josafat maakte schepen van Tharsis, om naar Ofir te gaan om goud; maar zij gingen niet, want de schepen werden gebroken te Ezeon-geber.
\par 50 Toen zeide Ahazia, de zoon van Achab, tot Josafat: Laat mijn knechten met uw knechten op de schepen varen; maar Josafat wilde niet.
\par 51 En Josafat ontsliep met zijn vaderen, en werd bij zijn vaderen begraven in de stad van zijn vader David; en zijn zoon Joram werd koning in zijn plaats.
\par 52 Ahazia, de zoon van Achab, werd koning over Israel te Samaria, in het zeventiende jaar van Josafat, den koning van Juda, en regeerde twee jaren over Israel.
\par 53 En hij deed dat kwaad was in de ogen des HEEREN; want hij wandelde in den weg van zijn vader, en in den weg van zijn moeder, en in den weg van Jerobeam, den zoon van Nebat, die Israel zondigen deed.
\par 54 En hij diende Baal, en boog zich voor hem, en vertoornde den HEERE, den God Israels, naar alles, wat zijn vader gedaan had.



\end{document}