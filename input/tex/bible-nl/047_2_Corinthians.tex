\begin{document}

\title{2 Korinthe}



\chapter{1}

\par 1 Paulus, een apostel van Jezus Christus, door den wil van God, en Timotheus, de broeder, aan de Gemeente Gods, die te Korinthe is, met al de heiligen, die in geheel Achaje zijn:
\par 2 Genade zij u en vrede van God, onzen Vader, en den Heere Jezus Christus.
\par 3 Geloofd zij de God en Vader van onzen Heere Jezus Christus, de Vader der barmhartigheden, en de God aller vertroosting;
\par 4 Die ons vertroost in al onze verdrukking, opdat wij zouden kunnen vertroosten degenen, die in allerlei verdrukking zijn, door de vertroosting, met welke wij zelven van God vertroost worden.
\par 5 Want gelijk het lijden van Christus overvloedig is in ons, alzo is ook door Christus onze vertroosting overvloedig.
\par 6 Doch hetzij dat wij verdrukt worden, het is tot uw vertroosting en zaligheid, die gewrocht wordt in de lijdzaamheid van hetzelfde lijden, hetwelk wij ook lijden; hetzij dat wij vertroost worden, het is tot uw vertroosting en zaligheid;
\par 7 En onze hoop van u is vast, als die weten, dat, gelijk gij gemeenschap hebt aan het lijden, gij ook alzo gemeenschap hebt aan de vertroosting.
\par 8 Want wij willen niet, broeders, dat gij onwetende zijt van onze verdrukking, die ons in Azie overkomen is, dat wij uitnemend zeer bezwaard zijn geweest boven onze macht, alzo dat wij zeer in twijfel waren, ook van het leven.
\par 9 Ja, wij hadden al zelven in onszelven het vonnis des doods, opdat wij niet op onszelven vertrouwen zouden, maar op God, Die de doden verwekt;
\par 10 Die ons uit zo groten dood verlost heeft, en nog verlost; op Welken wij hopen, dat Hij ons ook nog verlossen zal.
\par 11 Alzo gijlieden ook medearbeidt voor ons door het gebed, opdat over de gave, door vele personen aan ons teweeggebracht ook voor ons dankzegging door velen gedaan worde.
\par 12 Want onze roem is deze, namelijk de getuigenis van ons geweten, dat wij in eenvoudigheid en oprechtheid Gods, niet in vleselijke wijsheid, maar in de genade Gods, in de wereld verkeerd hebben, en allermeest bij ulieden.
\par 13 Want wij schrijven u geen andere dingen, dan die gij kent, of ook erkent; en ik hoop, dat gij ze ook tot het einde toe erkennen zult;
\par 14 Gelijkerwijs gij ook ten dele ons erkend hebt, dat wij uw roem zijn, gelijk gij ook de onze zijt, in den dag van den Heere Jezus.
\par 15 En op dit betrouwen wilde ik te voren tot u komen, opdat gij een tweede genade zoudt hebben;
\par 16 En door uw stad naar Macedonie gaan, en wederom van Macedonie tot u komen, en van ulieden naar Judea geleid worden.
\par 17 Als ik dan dit voorgenomen heb, heb ik ook lichtvaardigheid gebruikt? Of neem ik het naar het vlees voor, hetgeen ik voorneem, opdat bij mij zou wezen, ja, ja, en neen, neen?
\par 18 Doch God is getrouw, dat ons woord, hetwelk tot u is geschied, niet is geweest ja en neen.
\par 19 Want de Zoon van God, Jezus Christus, Die onder u door ons is gepredikt, namelijk door mij, en Silvanus, en Timotheus, was niet ja en neen, maar is geweest ja in Hem.
\par 20 Want zovele beloften Gods als er zijn, die zijn in Hem ja, en zijn in Hem amen, Gode tot heerlijkheid door ons.
\par 21 Maar Die ons met u bevestigt in Christus, en Die ons gezalfd heeft, is God;
\par 22 Die ons ook heeft verzegeld, en het onderpand des Geestes in onze harten gegeven.
\par 23 Doch ik aanroepe God tot een Getuige over mijn ziel, dat ik, om u te sparen, nog te Korinthe niet ben gekomen.
\par 24 Niet dat wij heerschappij voeren over uw geloof, maar wij zijn medewerkers uwer blijdschap; want gij staat door het geloof.

\chapter{2}

\par 1 Maar ik heb dit bij mijzelven voorgenomen, dat ik niet wederom in droefheid tot u komen zou.
\par 2 Want indien ik ulieden bedroef, wie is het toch, die mij zal vrolijk maken, dan degene, die van mij bedroefd is geworden?
\par 3 En ditzelfde heb ik u geschreven, opdat ik, daar komende, niet zou droefheid hebben van degenen, van welke ik moest verblijd worden; vertrouwende van u allen, dat mijn blijdschap uw aller blijdschap is.
\par 4 Want ik heb ulieden uit vele verdrukking en benauwdheid des harten, met vele tranen geschreven, niet opdat gij zoudt bedroefd worden, maar opdat gij de liefde zoudt verstaan, die ik overvloediglijk tot u heb.
\par 5 Doch indien iemand bedroefd heeft, die heeft niet mij bedroefd, maar ten dele (opdat ik hem niet bezware) ulieden allen.
\par 6 Den zodanige is deze bestraffing genoeg, die van velen geschied is.
\par 7 Alzo dat gij daarentegen hem liever moet vergeven en vertroosten, opdat de zodanige door al te overvloedige droefheid niet enigszins worde verslonden.
\par 8 Daarom bid ik u, dat gij de liefde aan hem bevestigt.
\par 9 Want daartoe heb ik ook geschreven, opdat ik uw beproeving mocht verstaan, of gij in alles gehoorzaam zijt.
\par 10 Dien gij nu iets vergeeft, dien vergeef ik ook; want zo ik ook iets vergeven heb, dien ik vergeven heb, heb ik het vergeven om uwentwil, voor het aangezicht van Christus, opdat de satan over ons geen voordeel krijge;
\par 11 Want zijn gedachten zijn ons niet onbekend.
\par 12 Voorts, als ik te Troas kwam, om het Evangelie van Christus te prediken, en als mij een deur geopend was in den Heere, zo heb ik geen rust gehad voor mijn geest, omdat ik Titus, mijn broeder, niet vond;
\par 13 Maar, afscheid van hen genomen hebbende, vertrok ik naar Macedonie.
\par 14 En Gode zij dank, Die ons allen tijd doet triomferen in Christus, en den reuk Zijner kennis door ons openbaar maakt in alle plaatsen.
\par 15 Want wij zijn Gode een goede reuk van Christus, in degenen, die zalig worden, en in degenen, die verloren gaan;
\par 16 Dezen wel een reuk des doods ten dode; maar genen een reuk des levens ten leven. En wie is tot deze dingen bekwaam?
\par 17 Want wij dragen niet, gelijk velen, het Woord Gods te koop, maar als uit oprechtheid, maar als uit God, in de tegenwoordigheid Gods, spreken wij het in Christus.

\chapter{3}

\par 1 Beginnen wij onszelven wederom u aan te prijzen? Of behoeven wij ook, gelijk sommigen, brieven van voorschrijving aan u, of brieven van voorschrijving van u?
\par 2 Gijlieden zijt onze brief, geschreven in onze harten, bekend en gelezen van alle mensen;
\par 3 Als die openbaar zijt geworden, dat gij een brief van Christus zijt, en door onzen dienst bereid, die geschreven is niet met inkt, maar door den Geest des levenden Gods, niet in stenen tafelen, maar in vlezen tafelen des harten.
\par 4 En zodanig een vertrouwen hebben wij door Christus bij God.
\par 5 Niet dat wij van onszelven bekwaam zijn iets te denken, als uit onszelven; maar onze bekwaamheid is uit God;
\par 6 Die ons ook bekwaam gemaakt heeft, om te zijn dienaars des Nieuwen Testaments, niet der letter, maar des Geestes; want de letter doodt, maar de Geest maakt levend.
\par 7 En indien de bediening des doods in letteren bestaande, en in stenen ingedrukt, in heerlijkheid is geweest, alzo dat de kinderen Israels het aangezicht van Mozes niet konden sterk aanzien, om de heerlijkheid zijns aangezichts, die te niet gedaan zou worden.
\par 8 Hoe zal niet veel meer de bediening des Geestes in heerlijkheid zijn?
\par 9 Want indien de bediening der verdoemenis heerlijkheid geweest is, veel meer is de bediening der rechtvaardigheid overvloedig in heerlijkheid.
\par 10 Want ook het verheerlijkte is zelfs niet verheerlijkt in dezen dele, ten aanzien van deze uitnemende heerlijkheid.
\par 11 Want indien hetgeen te niet gedaan wordt, in heerlijkheid was, veel meer is, hetgeen blijft, in heerlijkheid.
\par 12 Dewijl wij dan zodanige hoop hebben, zo gebruiken wij vele vrijmoedigheid in het spreken;
\par 13 En doen niet gelijkerwijs Mozes, die een deksel op zijn aangezicht leide, opdat de kinderen Israels niet zouden sterk zien op het einde van hetgeen te niet gedaan wordt.
\par 14 Maar hun zinnen zijn verhard geworden; want tot op den dag van heden blijft hetzelfde deksel in het lezen des Ouden Testaments, zonder ontdekt te worden, hetwelk door Christus te niet gedaan wordt.
\par 15 Maar tot den huidigen dag toe, wanneer Mozes gelezen wordt, ligt een deksel op hun hart.
\par 16 Doch zo wanneer het tot den Heere zal bekeerd zijn, zo wordt het deksel weggenomen.
\par 17 De Heere nu is de Geest; en waar de Geest des Heeren is, aldaar is vrijheid.
\par 18 En wij allen, met ongedekten aangezichte de heerlijkheid des Heeren als in een spiegel aanschouwende, worden naar hetzelfde beeld in gedaante veranderd, van heerlijkheid tot heerlijkheid, als van des Heeren Geest.

\chapter{4}

\par 1 Daarom dewijl wij deze bediening hebben, naar de barmhartigheid, die ons geschied is, zo vertragen wij niet;
\par 2 Maar wij hebben verworpen de bedekselen der schande, niet wandelende in arglistigheid, noch het Woord Gods vervalsende, maar door openbaring der waarheid onszelven aangenaam makende bij alle gewetens der mensen, in de tegenwoordigheid Gods.
\par 3 Doch indien ook ons Evangelie bedekt is, zo is het bedekt in degenen, die verloren gaan;
\par 4 In dewelke de god dezer eeuw de zinnen verblind heeft, namelijk der ongelovigen, opdat hen niet bestrale de verlichting van het Evangelie der heerlijkheid van Christus, Die het Beeld Gods is.
\par 5 Want wij prediken niet onszelven, maar Christus Jezus, den Heere; en onszelven, dat wij uw dienaars zijn om Jezus' wil.
\par 6 Want God, Die gezegd heeft, dat het licht uit de duisternis zou schijnen, is Degene, Die in onze harten geschenen heeft, om te geven verlichting der kennis der heerlijkheid Gods in het aangezicht van Jezus Christus.
\par 7 Maar wij hebben dezen schat in aarden vaten, opdat de uitnemendheid der kracht zij van God, en niet uit ons;
\par 8 Als die in alles verdrukt worden, doch niet benauwd; twijfelmoedig, doch niet mismoedig;
\par 9 Vervolgd, doch niet daarin verlaten; nedergeworpen, doch niet verdorven;
\par 10 Altijd de doding van den Heere Jezus in het lichaam omdragende, opdat ook het leven van Jezus in ons lichaam zou geopenbaard worden.
\par 11 Want wij, die leven, worden altijd in den dood overgegeven om Jezus' wil; opdat ook het leven van Jezus in ons sterfelijk vlees zou geopenbaard worden.
\par 12 Zo dan, de dood werkt wel in ons, maar het leven in ulieden.
\par 13 Dewijl wij nu denzelfden Geest des geloofs hebben, gelijk er geschreven is: Ik heb geloofd, daarom heb ik gesproken; zo geloven wij ook, daarom spreken wij ook;
\par 14 Wetende, dat Hij, Die den Heere Jezus opgewekt heeft, ook ons door Jezus zal opwekken, en met ulieden daar zal stellen.
\par 15 Want al deze dingen zijn om uwentwil, opdat de vermenigvuldigde genade, door de dankzegging van velen, overvloedig worde ter heerlijkheid Gods.
\par 16 Daarom vertragen wij niet; maar hoewel onze uitwendige mens verdorven wordt, zo wordt nochtans de inwendige vernieuwd van dag tot dag.
\par 17 Want onze lichte verdrukking, die zeer haast voorbij gaat, werkt ons een gans zeer uitnemend eeuwig gewicht der heerlijkheid;
\par 18 Dewijl wij niet aanmerken de dingen, die men ziet, maar de dingen, die men niet ziet; want de dingen, die men ziet, zijn tijdelijk, maar de dingen, die men niet ziet, zijn eeuwig.

\chapter{5}

\par 1 Want wij weten, dat, zo ons aardse huis dezes tabernakels gebroken wordt, wij een gebouw van God hebben, een huis niet met handen gemaakt, maar eeuwig in de hemelen.
\par 2 Want ook in dezen zuchten wij, verlangende met onze woonstede, die uit den hemel is, overkleed te worden.
\par 3 Zo wij ook bekleed en niet naakt zullen gevonden worden.
\par 4 Want ook wij, die in dezen tabernakel zijn, zuchten, bezwaard zijnde; nademaal wij niet willen ontkleed, maar overkleed worden, opdat het sterfelijke van het leven verslonden worde.
\par 5 Die ons nu tot ditzelfde bereid heeft, is God, Die ons ook het onderpand des Geestes gegeven heeft.
\par 6 Wij hebben dan altijd goeden moed, en weten, dat wij, inwonende in het lichaam, uitwonen van den Heere;
\par 7 (Want wij wandelen door geloof en niet door aanschouwen.)
\par 8 Maar wij hebben goeden moed, en hebben meer behagen om uit het lichaam uit te wonen, en bij den Heere in te wonen.
\par 9 Daarom zijn wij ook zeer begerig, hetzij inwonende, hetzij uitwonende, om Hem welbehagelijk te zijn.
\par 10 Want wij allen moeten geopenbaard worden voor den rechterstoel van Christus, opdat een iegelijk wegdrage, hetgeen door het lichaam geschiedt, naardat hij gedaan heeft, hetzij goed, hetzij kwaad.
\par 11 Wij dan, wetende den schrik des Heeren, bewegen de mensen tot het geloof, en zijn Gode openbaar geworden; doch ik hoop ook in uw gewetens geopenbaard te zijn.
\par 12 Want wij prijzen onszelven u niet wederom aan, maar wij geven u oorzaak van roem over ons, opdat gij stof zoudt hebben tegen degenen, die in het aangezicht roemen en niet in het hart.
\par 13 Want hetzij dat wij uitzinnig zijn, wij zijn het Gode; hetzij dat wij gematigd van zinnen zijn, wij zijn het ulieden.
\par 14 Want de liefde van Christus dringt ons;
\par 15 Als die dit oordelen, dat, indien Een voor allen gestorven is, zij dan allen gestorven zijn. En Hij is voor allen gestorven, opdat degenen, die leven, niet meer zichzelven zouden leven, maar Dien, Die voor hen gestorven en opgewekt is.
\par 16 Zo dan, wij kennen van nu aan niemand naar het vlees; en indien wij ook Christus naar het vlees gekend hebben, nochtans kennen wij Hem nu niet meer naar het vlees.
\par 17 Zo dan, indien iemand in Christus is, die is een nieuw schepsel; het oude is voorbijgegaan, ziet, het is alles nieuw geworden.
\par 18 En al deze dingen zijn uit God, Die ons met Zichzelven verzoend heeft door Jezus Christus, en ons de bediening der verzoening gegeven heeft.
\par 19 Want God was in Christus de wereld met Zichzelven verzoenende, hun zonden hun niet toerekenende; en heeft het woord der verzoening in ons gelegd.
\par 20 Zo zijn wij dan gezanten van Christus wege, alsof God door ons bade; wij bidden van Christus wege: laat u met God verzoenen.
\par 21 Want Dien, Die geen zonde gekend heeft, heeft Hij zonde voor ons gemaakt, opdat wij zouden worden rechtvaardigheid Gods in Hem.

\chapter{6}

\par 1 En wij, als medearbeidende, bidden u ook, dat gij de genade Gods niet tevergeefs moogt ontvangen hebben.
\par 2 Want Hij zegt: In den aangenamen tijd heb Ik u verhoord, en in den dag der zaligheid heb Ik u geholpen. Ziet, nu is het de welaangename tijd, ziet, nu is het de dag der zaligheid!
\par 3 Wij geven geen aanstoot in enig ding, opdat de bediening niet gelasterd worde.
\par 4 Maar wij, als dienaars van God, maken onszelven in alles aangenaam, in vele verdraagzaamheid, in verdrukkingen, in noden, in benauwdheden,
\par 5 In slagen, in gevangenissen, in beroerten, in arbeid, in waken, in vasten,
\par 6 In reinheid, in kennis, in lankmoedigheid, in goedertierenheid, in den Heiligen Geest, in ongeveinsde liefde.
\par 7 In het woord der waarheid, in de kracht van God, door de wapenen der gerechtigheid aan de rechter zijde en aan de linker zijde;
\par 8 Door eer en oneer, door kwaad gerucht en goed gerucht; als verleiders, en nochtans waarachtigen;
\par 9 Als onbekenden, en nochtans bekend; als stervenden, en ziet, wij leven; als getuchtigd, en niet gedood;
\par 10 Als droevig zijnde, doch altijd blijde; als arm, doch velen rijk makende; als niets hebbende, en nochtans alles bezittende.
\par 11 Onze mond is opengedaan tegen u, o Korinthiers, ons hart is uitgebreid.
\par 12 Gij zijt niet nauw in ons, maar gij zijt nauw in uw ingewanden.
\par 13 Nu, om dezelfde vergelding te doen, (ik spreek als tot mijn kinderen) zo wordt gij ook uitgebreid.
\par 14 Trekt niet een ander juk aan met de ongelovigen; want wat mededeel heeft de gerechtigheid met de ongerechtigheid, en wat gemeenschap heeft het licht met de duisternis?
\par 15 En wat samenstemming heeft Christus met Belial, of wat deel heeft de gelovige met den ongelovige?
\par 16 Of wat samenvoeging heeft de tempel Gods met de afgoden? Want gij zijt de tempel des levenden Gods; gelijkerwijs God gezegd heeft: Ik zal in hen wonen, en Ik zal onder hen wandelen; en Ik zal hun God zijn, en zij zullen Mij een Volk zijn.
\par 17 Daarom gaat uit het midden van hen, en scheidt u af, zegt de Heere, en raakt niet aan hetgeen onrein is, en Ik zal ulieden aannemen.
\par 18 En Ik zal u tot een Vader zijn, en gij zult Mij tot zonen en dochteren zijn, zegt de Heere, de Almachtige.

\chapter{7}

\par 1 Dewijl wij dan deze beloften hebben, geliefden, laat ons onszelven reinigen van alle besmetting des vleses en des geestes, voleindigende de heiligmaking in de vreze Gods.
\par 2 Geeft ons plaats; wij hebben niemand verongelijkt, wij hebben niemand verdorven, wij hebben bij niemand ons voordeel gezocht.
\par 3 Ik zeg dit niet tot uw veroordeling; want ik heb te voren gezegd, dat gij in onze harten zijt, om samen te sterven en samen te leven.
\par 4 Ik heb vele vrijmoedigheid in het spreken tegen u, ik heb veel roems over u; ik ben vervuld met vertroosting; ik ben zeer overvloedig van blijdschap in al onze verdrukking.
\par 5 Want ook, als wij in Macedonie gekomen zijn, zo heeft ons vlees geen rust gehad; maar wij waren in alles verdrukt; van buiten was strijd, van binnen vrees.
\par 6 Doch God, Die de nederigen vertroost, heeft ons getroost door de komst van Titus.
\par 7 En niet alleen door zijn komst, maar ook door de vertroosting, met welke hij over u vertroost is geweest, als hij ons verhaalde uw verlangen, uw kermen, uw ijver voor mij; alzo dat ik te meer verblijd ben geweest.
\par 8 Want hoewel ik u in den zendbrief bedroefd heb, het berouwt mij niet, hoewel het mij berouwd heeft; want ik zie, dat dezelve zendbrief, hoewel voor een kleinen tijd, u bedroefd heeft.
\par 9 Nu verblijde ik mij, niet omdat gij bedroefd zijt geweest, maar omdat gij bedroefd zijt geweest tot bekering; want gij zijt bedroefd geweest naar God, zodat gij in geen ding schade van ons geleden hebt.
\par 10 Want de droefheid naar God werkt een onberouwelijke bekering tot zaligheid; maar de droefheid der wereld werkt den dood.
\par 11 Want ziet, ditzelfde dat gij naar God zijt bedroefd geworden, hoe grote naarstigheid heeft het in u gewrocht? Ja, verantwoording, ja, onlust, ja, vrees, ja, verlangen, ja, ijver, ja, wraak; in alles hebt gij uzelven bewezen rein te zijn in deze zaak.
\par 12 Hoewel ik dan aan u geschreven heb, dat is niet om diens wil, die onrecht gedaan had, noch om diens wil, die onrecht gedaan was; maar opdat onze vlijtigheid voor u bij u openbaar zou worden, in de tegenwoordigheid Gods.
\par 13 Daarom zijn wij vertroost geworden over uw vertroosting; en zijn nog overvloediger verblijd geworden over de blijdschap van Titus, omdat zijn geest van u allen verkwikt is geworden.
\par 14 Want indien ik iets bij hem over u geroemd heb, zo ben ik niet beschaamd geworden; maar gelijk wij alles met waarheid tot u gesproken hebben, alzo is ook onze roem, dien ik bij Titus geroemd heb, waarheid geworden.
\par 15 En zijn innerlijke bewegingen zijn te overvloediger jegens u, als hij u aller gehoorzaamheid overdenkt, hoe gij hem met vreze en beven hebt ontvangen.
\par 16 Ik verblijde mij dan, dat ik in alles van u vertrouwen mag hebben.

\chapter{8}

\par 1 Voorts maken wij u bekend, broeders, de genade van God, die in de Gemeenten van Macedonie gegeven is.
\par 2 Dat in vele beproeving der verdrukking de overvloed hunner blijdschap, en hun zeer diepe armoede overvloedig geweest is tot den rijkdom hunner goeddadigheid.
\par 3 Want zij zijn naar vermogen (ik betuig het), ja, boven vermogen gewillig geweest;
\par 4 Ons met vele vermaning biddende, dat wij wilden aannemen de gave en de gemeenschap dezer bediening, die voor de heiligen geschiedt.
\par 5 En zij deden niet alleen, gelijk wij gehoopt hadden, maar gaven zichzelven eerst aan den Heere en daarna aan ons, door den wil van God.
\par 6 Alzo dat wij Titus vermaanden, dat, gelijk hij te voren begonnen had, hij ook alzo nog deze gave bij u voleinden zou.
\par 7 Zo dan, gelijk gij in alles overvloedig zijt, in geloof, en in woord, en in kennis, en in alle naarstigheid, en in uw liefde tot ons, ziet, dat gij ook in deze gave overvloedig zijt.
\par 8 Ik zeg dit niet als gebiedende, maar als door de naarstigheid van anderen ook de oprechtheid uwer liefde beproevende.
\par 9 Want gij weet de genade van onzen Heere Jezus Christus, dat Hij om uwentwil is arm geworden, daar Hij rijk was, opdat gij door Zijn armoede zoudt rijk worden.
\par 10 En ik zeg in dezen mijn mening; want dit is u oorbaar, als die niet alleen het doen, maar ook het willen van over een jaar te voren hebt begonnen.
\par 11 Maar nu voleindigt ook het doen; opdat, gelijk als er geweest is de volvaardigheid des gemoeds om te willen, er ook alzo zij het voleindigen uit hetgeen gij hebt.
\par 12 Want indien te voren de volvaardigheid des gemoeds daar is, zo is iemand aangenaam naar hetgeen hij heeft, niet naar hetgeen hij niet heeft.
\par 13 Want dit zeg ik niet, opdat anderen zouden verlichting hebben, en gij verdrukking;
\par 14 Maar opdat uit gelijkheid, in dezen tegenwoordigen tijd, uw overvloed zij om hun gebrek te vervullen; opdat ook hun overvloed zij om uw gebrek te vervullen, opdat er gelijkheid worde.
\par 15 Gelijk geschreven is: Die veel verzameld had, had niet over; en die weinig verzameld had, had niet te weinig.
\par 16 Doch Gode zij dank, Die dezelfde naarstigheid voor u in het hart van Titus gegeven heeft;
\par 17 Dat hij de vermaning heeft aangenomen, en zeer naarstig zijnde, gewillig tot u gereisd is.
\par 18 En wij hebben ook met hem gezonden den broeder, die lof heeft in het Evangelie door al de Gemeenten;
\par 19 En dat niet alleen, maar hij is ook van de Gemeenten verkoren, om met ons te reizen met deze gave, die van ons bediend wordt tot de heerlijkheid des Heeren Zelven, en de volvaardigheid uws gemoeds;
\par 20 Dit verhoedende, dat ons niemand moge lasteren in dezen overvloed, die van ons wordt bediend;
\par 21 Als die bezorgen, hetgeen eerlijk is, niet alleen voor den Heere, maar ook voor de mensen.
\par 22 Wij hebben ook met hen gezonden onzen broeder, welken wij in vele dingen dikmaals beproefd hebben, dat hij naarstig is; en nu veel naarstiger, door het groot vertrouwen, dat hij heeft tot ulieden.
\par 23 Hetzij dan Titus, hij is mijn metgezel en medearbeider bij u; hetzij onze broeders, zij zijn afgezanten der Gemeenten, en een eer van Christus.
\par 24 Bewijst dan aan hen de bewijzing uwer liefde, en van onzen roem van u, ook voor het aangezicht der Gemeenten.

\chapter{9}

\par 1 Want van de bediening, die voor de heiligen geschiedt, is mij onnodig aan u te schrijven.
\par 2 Want ik weet de volvaardigheid uws gemoeds, van welke ik roem over u bij de Macedoniers, dat Achaje van over een jaar bereid is geweest; en de ijver, van u begonnen, heeft er velen verwekt.
\par 3 Maar ik heb deze broeders gezonden, opdat onze roem, dien wij over u hebben, niet zou ijdel gemaakt worden in dezen dele; opdat (gelijk ik gezegd heb) gij bereid moogt zijn;
\par 4 En dat niet mogelijk, zo de Macedoniers met mij kwamen, en u onbereid vonden, wij (opdat wij niet zeggen, gij) beschaamd worden in dezen vasten grond der roeming.
\par 5 Ik heb dan nodig geacht deze broeders te vermanen, dat zij eerst tot u zouden komen, en voorbereiden uw te voren aangedienden zegen; opdat die gereed zij, alzo als een zegen, en niet als een vrekheid.
\par 6 En dit zeg ik: Die spaarzamelijk zaait, zal ook spaarzamelijk maaien; en die in zegeningen zaait, zal ook in zegeningen maaien.
\par 7 Een iegelijk doe, gelijk hij in zijn hart, voorneemt; niet uit droefheid, of uit nooddwang; want God heeft een blijmoedigen gever lief.
\par 8 En God is machtig alle genade te doen overvloedig zijn in u; opdat gij in alles te allen tijd, alle genoegzaamheid hebbende, tot alle goed werk overvloedig moogt zijn.
\par 9 Gelijk er geschreven is: Hij heeft gestrooid, hij heeft den armen gegeven; Zijn gerechtigheid blijft in der eeuwigheid.
\par 10 Doch Die het zaad den zaaier verleent, Die verlene ook brood tot spijze, en vermenigvuldige uw gezaaisel, en vermeerdere de vruchten uwer gerechtigheid;
\par 11 Dat gij in alles rijk wordt tot alle goeddadigheid, welke door ons werkt dankzegging tot God.
\par 12 Want de bediening van dezen dienst vervult niet alleen het gebrek der heiligen, maar is ook overvloedig door vele dankzeggingen tot God;
\par 13 Dewijl zij door de beproeving dezer bediening God verheerlijken over de onderwerping uwer belijdenis onder het Evangelie van Christus, en over de goeddadigheid der mededeling aan hen en aan allen;
\par 14 En door hun gebed voor u, welke naar u verlangen, om de uitnemende genade Gods over u.
\par 15 Doch Gode zij dank voor Zijn onuitsprekelijke gave.

\chapter{10}

\par 1 Voorts ik Paulus zelf bid u, door de zachtmoedigheid en goedertierenheid van Christus, die, tegenwoordig zijnde, wel gering ben onder u, maar afwezend stout ben tegen u;
\par 2 Ik bid dan, dat ik, tegenwoordig zijnde, niet stout moge zijn met die vrijmoedigheid, waarmede ik geacht word stoutelijk gehandeld te hebben tegen sommigen, die ons achten, alsof wij naar het vlees wandelden.
\par 3 Want wandelende in het vlees, voeren wij den krijg niet naar het vlees;
\par 4 Want de wapenen van onzen krijg zijn niet vleselijk, maar krachtig door God, tot nederwerping der sterkten;
\par 5 Dewijl wij de overleggingen ter nederwerpen, en alle hoogte, die zich verheft tegen de kennis van God, en alle gedachte gevangen leiden tot de gehoorzaamheid van Christus;
\par 6 En gereed hebbende, hetgeen dient om te wreken alle ongehoorzaamheid, wanneer uw gehoorzaamheid zal vervuld zijn.
\par 7 Ziet gij aan wat voor ogen is? Indien iemand bij zichzelven betrouwt, dat hij van Christus is, die denke dit wederom uit zichzelven, dat gelijkerwijs hij van Christus is, alzo ook wij van Christus zijn.
\par 8 Want indien ik ook iets overvloediger zou roemen van onze macht, welke de Heere ons gegeven heeft tot stichting, en niet tot uw nederwerping, zo zal ik niet beschaamd worden;
\par 9 Opdat ik niet zou schijnen, alsof ik u door de brieven wilde verschrikken.
\par 10 Want de brieven (zeggen zij) zijn wel gewichtig en krachtig; maar de tegenwoordigheid des lichaams is zwak, en de rede is verachtelijk.
\par 11 Dezulke bedenke dit, dat hoedanigen wij zijn in het woord door brieven, als wij afwezig zijn, wij ook zodanigen zijn inderdaad, als wij tegenwoordig zijn.
\par 12 Want wij durven onszelven niet rekenen of vergelijken met sommigen, die zichzelven prijzen; maar deze verstaan niet, dat zij zichzelven met zichzelven meten, en zichzelven met zichzelven vergelijken.
\par 13 Doch wij zullen niet roemen buiten de maat; maar dat wij, naar de maat des regels, welke maat ons God toegedeeld heeft, ook tot u toe zijn gekomen.
\par 14 Want wij strekken onszelven niet te wijd uit, als die tot u niet zouden komen; want wij zijn ook gekomen tot u toe, in het Evangelie van Christus;
\par 15 Niet roemende buiten de maat in anderer lieden arbeid, maar hebbende hoop, als uw geloof zal gewassen zijn, dat wij onder ulieden overvloediglijk zullen vergroot worden naar onzen regel;
\par 16 Om het Evangelie te verkondigen in de plaatsen, die op gene zijde van u gelegen zijn; niet om te roemen in eens anders regel over hetgeen alrede bereid is.
\par 17 Doch wie roemt, die roeme in den Heere.
\par 18 Want niet die zichzelven prijst, maar dien de Heere prijst, die is beproefd.

\chapter{11}

\par 1 Och, of gij mij een weinig verdroegt in de onwijsheid; ja ook, verdraagt mij!
\par 2 Want ik ben ijverig over u met een ijver Gods; want ik heb ulieden toebereid, om u als een reine maagd aan een man voor te stellen, namelijk aan Christus.
\par 3 Doch ik vrees, dat niet enigszins, gelijk de slang Eva door haar arglistigheid bedrogen heeft, alzo uw zinnen bedorven worden, om af te wijken van de eenvoudigheid, die in Christus is.
\par 4 Want indien degene, die komt, een anderen Jezus predikte, dien wij niet gepredikt hebben, of indien gij een anderen geest ontvingt, dien gij niet hebt ontvangen, of een ander Evangelie, dat gij niet hebt aangenomen, zo verdroegt gij hem met recht.
\par 5 Want ik acht, dat ik nergens minder in ben geweest dan de uitnemendste apostelen.
\par 6 En indien ik ook slecht ben in woorden, nochtans ben ik het niet in wetenschap; maar alleszins zijn wij in alle dingen onder u openbaar geworden.
\par 7 Heb ik zonde gedaan, als ik mijzelven vernederd heb, opdat gij zoudt verhoogd worden, overmits ik u het Evangelie Gods om niet verkondigd heb?
\par 8 Ik heb andere Gemeenten beroofd, bezoldiging van haar nemende, om u te bedienen; en als ik bij u tegenwoordig was en gebrek had, ben ik niemand lastig gevallen.
\par 9 Want mijn gebrek hebben de broeders vervuld, die van Macedonie kwamen; en ik heb mijzelven in alles gehouden zonder u te bezwaren, en zal mij nog alzo houden.
\par 10 De waarheid van Christus is in mij, dat deze roem in de gewesten van Achaje aan mij niet zal verhinderd worden.
\par 11 Waarom? Is het, omdat ik u niet liefheb? God weet het!
\par 12 Maar wat ik doe, dat zal ik nog doen, om de oorzaak af te snijden dengenen, die oorzaak hebben willen, opdat zij in hetgeen zij roemen, bevonden mochten worden gelijk als wij.
\par 13 Want zulke valse apostelen zijn bedriegelijke arbeiders, zich veranderende in apostelen van Christus.
\par 14 En het is geen wonder; want de satan zelf verandert zich in een engel des lichts.
\par 15 Zo is het dan niets groots, indien ook zijn dienaars zich veranderen, als waren zij dienaars der gerechtigheid; van welke het einde zal zijn naar hun werken.
\par 16 Ik zeg wederom, dat niemand mene, dat ik onwijs ben; doch zo niet, neemt mij dan aan als een onwijze, opdat ik ook een weinig moge roemen.
\par 17 Dat ik spreek, spreek ik niet naar den Heere, maar als in onwijsheid, in dezen vasten grond der roeming.
\par 18 Dewijl velen roemen naar het vlees, zo zal ik ook roemen.
\par 19 Want gij verdraagt gaarne de onwijzen, dewijl gij wijs zijt.
\par 20 Want gij verdraagt het, zo u iemand dienstbaar maakt, zo u iemand opeet, zo iemand van u neemt, zo zich iemand verheft, zo u iemand in het aangezicht slaat.
\par 21 Ik zeg dit naar oneer, gelijk of wij zwak waren geweest; maar waarin iemand stout is (ik spreek in onwijsheid), daarin ben ik ook stout.
\par 22 Zijn zij Hebreen? Ik ook. Zijn zij Israelieten? Ik ook. Zijn zij het zaad van Abraham? Ik ook.
\par 23 Zijn zij dienaars van Christus? (ik spreek onwijs zijnde) ik ben boven hen; in arbeid overvloediger, in slagen uitnemender, in gevangenissen overvloediger, in doods gevaar menigmaal.
\par 24 Van de Joden heb ik veertig slagen min een, vijfmaal ontvangen.
\par 25 Driemaal ben ik met roeden gegeseld geweest, eens ben ik gestenigd, driemaal heb ik schipbreuk geleden, een gansen nacht en dag heb ik in de diepte doorgebracht.
\par 26 In het reizen menigmaal in gevaren van rivieren, in gevaren van moordenaars, in gevaren van mijn geslacht, in gevaren van de heidenen, in gevaren in de stad, in gevaren in de woestijn, in gevaren op de zee, in gevaren onder de valse broeders;
\par 27 In arbeid en moeite, in waken menigmaal, in honger en dorst, in vasten menigmaal, in koude en naaktheid.
\par 28 Zonder de dingen, die van buiten zijn, overvalt mij dagelijks de zorg van al de Gemeenten.
\par 29 Wie is er zwak, dat ik niet zwak ben? Wie wordt er geergerd, dat ik niet brande?
\par 30 Indien men moet roemen, zo zal ik roemen de dingen mijner zwakheid.
\par 31 De God en Vader van onzen Heere Jezus Christus, Die geprezen is in der eeuwigheid, weet, dat ik niet lieg.
\par 32 De stadhouder van den koning Aretas in Damaskus, bezette de stad der Damaskenen, willende mij vangen;
\par 33 En ik werd door een venster in een mand over den muur nedergelaten, en ontvlood zijn handen.

\chapter{12}

\par 1 Te roemen is mij waarlijk niet oorbaar; want ik zal komen tot gezichten en openbaringen des Heeren.
\par 2 Ik ken een mens in Christus, voor veertien jaren (of het geschied zij in het lichaam, weet ik niet, of buiten het lichaam, weet ik niet, God weet het), dat de zodanige opgetrokken is geweest tot in den derden hemel;
\par 3 En ik ken een zodanig mens (of het in het lichaam, of buiten het lichaam geschied zij, weet ik niet, God weet het),
\par 4 Dat hij opgetrokken is geweest in het paradijs, en gehoord heeft onuitsprekelijke woorden, die het een mens niet geoorloofd is te spreken.
\par 5 Van den zodanige zal ik roemen, doch van mijzelven zal ik niet roemen, dan in mijn zwakheden.
\par 6 Want zo ik roemen wil, ik zal niet onwijs zijn, want ik zal de waarheid zeggen; maar ik houde daarvan af, opdat niemand van mij denke boven hetgeen hij ziet, dat ik ben, of dat hij uit mij hoort.
\par 7 En opdat ik mij door de uitnemendheid der openbaringen niet zou verheffen, zo is mij gegeven een scherpe doorn in het vlees, namelijk een engel des satans, dat hij mij met vuisten slaan zou, opdat ik mij niet zou verheffen.
\par 8 Hierover heb ik den Heere driemaal gebeden, opdat hij van mij zou wijken.
\par 9 En Hij heeft tot mij gezegd: Mijn genade is u genoeg; want Mijn kracht wordt in zwakheid volbracht. Zo zal ik dan veel liever roemen in mijn zwakheden, opdat de kracht van Christus in mij wone.
\par 10 Daarom heb ik een welbehagen in zwakheden, in smaadheden, in noden, in vervolgingen, in benauwdheden, om Christus' wil; want als ik zwak ben, dan ben ik machtig.
\par 11 Ik ben roemende onwijs geworden; gij hebt mij genoodzaakt, want ik behoorde van u geprezen te zijn; want ik ben in geen ding minder geweest dan de uitnemendste apostelen, hoewel ik niets ben.
\par 12 De merktekenen van een apostel zijn onder u betoond in alle lijdzaamheid, met tekenen, en wonderen, en krachten.
\par 13 Want wat is er, waarin gij minder geweest zijt dan de andere Gemeenten, anders, dan dat ik zelf u niet lastig ben geweest? Vergeeft mij dit ongelijk.
\par 14 Ziet, ik ben ten derden male gereed, om tot u te komen, en zal u niet lastig zijn; want ik zoek niet het uwe, maar u; want de kinderen moeten niet schatten vergaderen voor de ouders, maar de ouders voor de kinderen.
\par 15 En ik zal zeer gaarne de kosten doen, en voor uw zielen ten koste gegeven worden; hoewel ik, u overvloediger beminnende, weiniger bemind worde.
\par 16 Doch het zij zo, ik heb u niet bezwaard; maar alzo ik listig was, heb ik u met bedrog gevangen.
\par 17 Heb ik door iemand dergenen, die ik tot u gezonden heb, van u mijn voordeel gezocht?
\par 18 Ik heb Titus gebeden, en den broeder medegezonden; heeft ook Titus van u zijn voordeel gezocht? Hebben wij niet in denzelfden geest gewandeld? Hebben wij niet gewandeld in dezelfde voetstappen?
\par 19 Meent gij wederom, dat wij ons bij u verontschuldigen? Wij spreken in de tegenwoordigheid van God in Christus; en dit alles, geliefden, tot uw stichting.
\par 20 Want ik vrees, dat als ik gekomen zal zijn, ik u niet enigszins zal vinden zodanigen als ik wil, en dat ik van u zal gevonden worden zodanig als gij niet wilt; dat er niet enigszins zijn twisten, nijdigheden, toorn, gekijf, achterklap, oorblazingen, opgeblazenheden, beroerten;
\par 21 Opdat wederom, als ik zal gekomen zijn, mijn God mij niet vernedere bij u, en ik rouw hebbe over velen, die te voren gezondigd hebben, en die zich niet bekeerd zullen hebben van de onreinigheid, en hoererij, en ontuchtigheid, die zij gedaan hebben.

\chapter{13}

\par 1 Dit is de derde maal, dat ik tot u kom; in den mond van twee of drie getuigen zal alle woord bestaan.
\par 2 Ik heb het te voren gezegd, en zeg het te voren als tegenwoordig zijnde de tweede maal, en ik schrijf het nu afwezende aan degenen, die te voren gezondigd hebben, en aan al de anderen, dat, zo ik wederom kom, ik hen niet zal sparen;
\par 3 Dewijl gij zoekt een proeve van Christus, Die in mij spreekt, Welke in u niet zwak is, maar krachtig is onder u.
\par 4 Want hoewel Hij gekruist is door zwakheid, zo leeft Hij nochtans door de kracht Gods. Want ook wij zijn zwak in Hem, maar zullen met Hem leven door de kracht Gods in u.
\par 5 Onderzoekt uzelven, of gij in het geloof zijt, beproeft uzelven. Of kent gij uzelven niet, dat Jezus Christus in u is? tenzij dat gij enigszins verwerpelijk zijt.
\par 6 Doch ik hoop, dat gij zult verstaan, dat wij niet verwerpelijk zijn.
\par 7 En ik wens van God, dat gij geen kwaad doet; niet opdat wij beproefd zouden bevonden worden, maar opdat gij het goede zoudt doen, en wij als verwerpelijk zouden zijn.
\par 8 Want wij vermogen niets tegen de waarheid, maar voor de waarheid.
\par 9 Want wij verblijden ons, wanneer wij zwak zijn, en gij sterk zijt. En wij wensen ook dit, namelijk uw volmaking.
\par 10 Daarom schrijf ik, afwezende, deze dingen, opdat ik niet, tegenwoordig zijnde, strengheid zou gebruiken, naar de macht, die mij de Heere gegeven heeft tot opbouwing, en niet tot nederwerping.
\par 11 Voorts, broeders, zijt blijde, wordt volmaakt, zijt getroost, zijt eensgezind, leeft in vrede; en de God der liefde en des vredes zal met u zijn.
\par 12 Groet elkander met een heiligen kus.
\par 13 U groeten al de heiligen.
\par 14 De genade van den Heere Jezus Christus, en de liefde van God, en de gemeenschap des Heiligen Geestes, zij met u allen. Amen.




\end{document}