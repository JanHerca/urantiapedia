\begin{document}

\title{Judges}



\chapter{1}

\par 1 En het geschiedde na den dood van Jozua, dat de kinderen Israels den HEERE vraagden, zeggende: Wie zal onder ons het eerst optrekken naar de Kanaanieten, om tegen hen te krijgen?
\par 2 En de HEERE zeide: Juda zal optrekken; ziet, Ik heb dat land in zijn hand gegeven.
\par 3 Toen zeide Juda tot zijn broeder Simeon: Trek met mij op in mijn lot, en laat ons tegen de Kanaanieten krijgen, zo zal ik ook met u optrekken in uw lot. Alzo toog Simeon op met hem.
\par 4 En Juda toog op, en de HEERE gaf de Kanaanieten en de Ferezieten in hun hand; en zij sloegen hen bij Bezek, tien duizend man.
\par 5 En zij vonden Adoni-bezek te Bezek, en streden tegen hem; en zij sloegen de Kanaanieten en de Ferezieten.
\par 6 Doch Adoni-bezek vluchtte; en zij jaagden hem na, en zij grepen hem, en hieuwen de duimen zijner handen en zijner voeten af.
\par 7 Toen zeide Adoni-bezek: Zeventig koningen, met afgehouwen duimen van hun handen en van hun voeten, waren onder mijn tafel, de kruimen oplezende; gelijk als ik gedaan heb, alzo heeft mij God vergolden! En zij brachten hem te Jeruzalem, en hij stierf aldaar.
\par 8 Want de kinderen van Juda hadden tegen Jeruzalem gestreden, en hadden haar ingenomen, en met de scherpte des zwaards geslagen; en zij hadden de stad in het vuur gezet.
\par 9 En daarna waren de kinderen van Juda afgetogen, om te krijgen tegen de Kanaanieten, wonende in het gebergte, en in het zuiden, en in de laagte.
\par 10 En Juda was heengetogen tegen de Kanaanieten, die te Hebron woonden (de naam nu van Hebron was te voren Kirjath-arba), en zij sloegen Sesai, en Ahiman, en Thalmai.
\par 11 En van daar was hij heengetogen tegen de inwoners van Debir; de naam nu van Debir was te voren Kirjath-sefer.
\par 12 En Kaleb zeide: Wie Kirjath-sefer zal slaan, en haar innemen, dien zal ik ook mijn dochter Achsa tot een vrouw geven.
\par 13 Toen nam Othniel haar in, de zoon van Kenaz, broeder van Kaleb, die jonger was dan hij; en Kaleb gaf hem Achsa, zijn dochter, tot een vrouw.
\par 14 En het geschiedde, als zij tot hem kwam, dat zij hem aanporde, om van haar vader een veld te begeren; en zij sprong van den ezel af; toen zeide Kaleb tot haar: Wat is u?
\par 15 En zij zeide tot hem: Geef mij een zegen; dewijl gij mij een dor land gegeven hebt, geef mij ook waterwellingen. Toen gaf Kaleb haar hoge wellingen en lage wellingen.
\par 16 De kinderen van den Keniet, den schoonvader van Mozes, togen ook uit de Palmstad op, met de kinderen van Juda, naar de woestijn van Juda, die tegen het zuiden van Harad is; en zij gingen heen en woonden met het volk.
\par 17 Juda dan toog met zijn broeder Simeon, en zij sloegen de Kanaanieten, wonende te Zefat, en zij verbanden hen; en men noemde den naam dezer stad Horma.
\par 18 Daartoe nam Juda Gaza in, met haar landpale, en Askelon met haar landpale, en Ekron met haar landpale.
\par 19 En de HEERE was met Juda, dat hij de inwoners van het gebergte verdreef; maar hij ging niet voort om de inwoners des dals te verdrijven, omdat zij ijzeren wagenen hadden.
\par 20 En zij gaven Hebron aan Kaleb, gelijk als Mozes gesproken had; en hij verdreef van daar de drie zonen van Enak.
\par 21 Doch de kinderen van Benjamin hebben de Jebusieten, te Jeruzalem wonende, niet verdreven; maar de Jebusieten woonden met de kinderen van Benjamin te Jeruzalem, tot op dezen dag.
\par 22 En het huis van Jozef toog ook op naar Beth-el. En de HEERE was met hen.
\par 23 En het huis van Jozef bestelde verspieders bij Beth-el; de naam nu dezer stad was te voren Luz.
\par 24 En de wachters zagen een man, uitgaande uit de stad; en zij zeiden tot hem: Wijs ons toch den ingang der stad, en wij zullen weldadigheid bij u doen.
\par 25 En als hij hun den ingang der stad gewezen had, zo sloegen zij de stad met de scherpte des zwaards; maar dien man en zijn ganse huis lieten zij gaan.
\par 26 Toen toog deze man in het land der Hethieten, en hij bouwde een stad, en noemde haar naam Luz; dit is haar naam tot op dezen dag.
\par 27 En Manasse verdreef Beth-sean niet, noch haar onderhorige plaatsen, noch Thaanach met haar onderhorige plaatsen, noch de inwoners van Dor met haar onderhorige plaatsen, noch de inwoners van Jibleam met haar onderhorige plaatsen, noch de inwoners van Megiddo met haar onderhorige plaatsen; en de Kanaanieten wilden wonen in hetzelve land.
\par 28 En het geschiedde, als Israel sterk werd, dat hij de Kanaanieten op cijns stelde; maar hij verdreef hen niet ganselijk.
\par 29 Ook verdreef Efraim de Kanaanieten niet, die te Gezer woonden; maar de Kanaanieten woonden in het midden van hem te Gezer.
\par 30 Zebulon verdreef de inwoners van Kitron niet, noch de inwoners van Nahalol; maar de Kanaanieten woonden in het midden van hem, en waren cijnsbaar.
\par 31 Aser verdreef de inwoners van Acco niet, noch de inwoners van Sidon, noch Achlab, noch Achsib, noch Chelba, noch Afik, noch Rechob;
\par 32 Maar de Aserieten woonden in het midden der Kanaanieten, die in het land woonden; want zij verdreven hen niet.
\par 33 Nafthali verdreef de inwoners van Beth-semes niet, noch de inwoners van Beth-anath, maar woonde in het midden der Kanaanieten, die in het land woonden; doch de inwoners van Beth-semes en Beth-anath werden hun cijnsbaar.
\par 34 En de Amorieten drongen de kinderen van Dan in het gebergte; want zij lieten hun niet toe, af te komen in het dal.
\par 35 Ook wilden de Amorieten wonen op het gebergte van Heres, te Ajalon, en te Saalbim; maar de hand van het huis van Jozef werd zwaar, zodat zij cijnsbaar werden.
\par 36 En de landpale der Amorieten was van den opgang van Akrabbim, van den rotssteen, en opwaarts heen.

\chapter{2}

\par 1 En een Engel des HEEREN kwam opwaarts van Gilgal tot Bochim, en Hij zeide: Ik heb ulieden uit Egypte opgevoerd, en u gebracht in het land, dat Ik uw vaderen gezworen heb, en gezegd: Ik zal Mijn verbond met ulieden niet verbreken in eeuwigheid.
\par 2 En ulieden aangaande, gij zult geen verbond maken met de inwoners dezes lands; hun altaren zult gij afbreken. Maar gij zijt Mijner stem niet gehoorzaam geweest; waarom hebt gij dit gedaan?
\par 3 Daarom heb Ik ook gezegd: Ik zal hen voor uw aangezicht niet uitdrijven; maar zij zullen u aan de zijden zijn, en hun goden zullen u tot een strik zijn.
\par 4 En het geschiedde, als de Engel des HEEREN deze woorden tot alle kinderen Israels gesproken had, zo hief het volk zijn stem op en weende.
\par 5 Daarom noemden zij den naam dier plaats Bochim; en zij offerden aldaar den HEERE.
\par 6 Als Jozua het volk had laten gaan, zo waren de kinderen Israels heengegaan, een ieder tot zijn erfdeel, om het land erfelijk te bezitten.
\par 7 En het volk diende den HEERE, al de dagen van Jozua, en al de dagen der oudsten, die lang geleefd hadden na Jozua; die gezien hadden al dat grote werk des HEEREN, dat Hij aan Israel gedaan had.
\par 8 Maar als Jozua, de zoon van Nun, de knecht des HEEREN, gestorven was, honderd en tien jaren oud zijnde;
\par 9 En zij hem begraven hadden in de landpale zijns erfdeels, te Timnath-heres, op een berg van Efraim, tegen het noorden van den berg Gaas;
\par 10 En al datzelve geslacht ook tot zijn vaderen vergaderd was; zo stond er een ander geslacht na hen op, dat den HEERE niet kende, noch ook het werk, dat Hij aan Israel gedaan had.
\par 11 Toen deden de kinderen Israels, dat kwaad was in de ogen des HEEREN, en zij dienden de Baals.
\par 12 En zij verlieten den HEERE, hunner vaderen God, Die hen uit Egypteland had uitgevoerd, en volgden andere goden na, van de goden der volken, die rondom hen waren, en bogen zich voor die, en zij verwekten den HEERE tot toorn.
\par 13 Want zij verlieten den HEERE, en dienden den Baal en Astharoth.
\par 14 Zo ontstak des HEEREN toorn tegen Israel, en Hij gaf hen in de hand der rovers, die hen beroofden; en Hij verkocht hen in de hand hunner vijanden rondom; en zij konden niet meer bestaan voor het aangezicht hunner vijanden.
\par 15 Overal, waarheen zij uittogen, was de hand des HEEREN tegen hen, ten kwade, gelijk als de HEERE gesproken, en gelijk als de HEERE gezworen had; en hun was zeer bang.
\par 16 En de HEERE verwekte richteren, die hen verlosten uit de hand dergenen, die hen beroofden;
\par 17 Doch zij hoorden ook niet naar hun richteren, maar hoereerden andere goden na, en bogen zich voor die; haast weken zij af van den weg, dien hun vaders gewandeld hadden, horende de geboden des HEEREN; alzo deden zij niet.
\par 18 En wanneer de HEERE hun richteren verwekte, zo was de HEERE met den richter, en verloste hen uit de hand hunner vijanden, al de dagen des richters; want het berouwde den HEERE, huns zuchtens halve vanwege degenen, die hen drongen en die hen drukten.
\par 19 Maar het geschiedde met het versterven des richters, dat zij omkeerden, en verdierven het meer dan hun vaderen, navolgende andere goden, dezelve dienende, en zich voor die buigende; zij lieten niets vallen van hun werken, noch van dezen hun harden weg.
\par 20 Daarom ontstak de toorn des HEEREN tegen Israel, dat Hij zeide: Omdat dit volk Mijn verbond heeft overtreden, dat Ik hun vaderen geboden heb, en zij naar Mijn stem niet gehoord hebben;
\par 21 Zo zal Ik ook niet voortvaren voor hun aangezicht iemand uit de bezitting te verdrijven, van de heidenen, die Jozua heeft achtergelaten, als hij stierf;
\par 22 Opdat Ik Israel door hen verzoeke, of zij den weg des HEEREN zullen houden, om daarin te wandelen, gelijk als hun vaderen gehouden hebben, of niet.
\par 23 Alzo liet de HEERE deze heidenen blijven, dat Hij hen niet haastelijk uit de bezitting verdreef; die Hij in de hand van Jozua niet had overgegeven.

\chapter{3}

\par 1 Dit nu zijn de heidenen, die de HEERE liet blijven, om door hen Israel te verzoeken, allen, die niet wisten van al de krijgen van Kanaan;
\par 2 Alleenlijk, opdat de geslachten der kinderen Israels die wisten, opdat Hij hun den krijg leerde, tenminste dengenen, die daar te voren niet van wisten.
\par 3 Vijf vorsten der Filistijnen, en al de Kanaanieten, en de Sidoniers, en de Hevieten, wonende in het gebergte van den Libanon, van den berg Baal-hermon, tot daar men komt te Hamath.
\par 4 Dezen dan waren, om Israel door hen te verzoeken, opdat men wiste, of zij de geboden des HEEREN zouden horen, die Hij hun vaderen door de hand van Mozes geboden had.
\par 5 Als nu de kinderen Israels woonden in het midden der Kanaanieten, der Hethieten, en der Amorieten, en der Ferezieten, en der Hevieten, en der Jebusieten;
\par 6 Zo namen zij zich derzelver dochters tot vrouwen, en gaven hun dochters aan derzelver zonen; en zij dienden derzelver goden.
\par 7 En de kinderen Israels deden, dat kwaad was in de ogen des HEEREN, en vergaten den HEERE, hun God, en zij dienden de Baals en de bossen.
\par 8 Toen ontstak de toorn des HEEREN tegen Israel; en Hij verkocht hen in de hand van Cuschan Rischataim, koning van Mesopotamie; en de kinderen Israels dienden Cuschan Rischataim acht jaren.
\par 9 Zo riepen de kinderen Israels tot den HEERE; en de HEERE verwekte den kinderen Israels een verlosser, die hen verloste, Othniel, zoon van Kenaz, broeder van Kaleb, die jonger was dan hij.
\par 10 En de Geest des HEEREN was over hem, en hij richtte Israel, en toog uit ten strijde; en de HEERE gaf Cuschan Rischataim, den koning van Syrie, in zijn hand, dat zijn hand sterk werd over Cuschan Rischataim.
\par 11 Toen was het land veertig jaren stil, en Othniel, de zoon van Kenaz, stierf.
\par 12 Maar de kinderen Israels voeren voort te doen, dat kwaad was in de ogen des HEEREN; toen sterkte de HEERE Eglon, den koning der Moabieten, tegen Israel, omdat zij deden, wat kwaad was in de ogen des HEEREN.
\par 13 En hij vergaderde tot zich de kinderen Ammons en de Amalekieten en hij toog heen, en sloeg Israel, en zij namen de Palmstad in bezit.
\par 14 En de kinderen Israels dienden Eglon, koning der Moabieten, achttien jaren.
\par 15 Toen riepen de kinderen Israels tot den HEERE, en de HEERE verwekte hun een verlosser, Ehud, den zoon van Gera, een zoon van Jemini, een man, die links was. En de kinderen Israels zonden door zijn hand een geschenk aan Eglon, den koning der Moabieten.
\par 16 En Ehud maakte zich een zwaard, dat twee scherpten had, welks lengte een el was; en hij gordde dat onder zijn klederen, aan zijn rechterheup.
\par 17 En hij bracht aan Eglon, den koning der Moabieten, dat geschenk; Eglon nu was een zeer vet man.
\par 18 En het geschiedde, als hij geeindigd had het geschenk te leveren, zo geleidde hij het volk, die het geschenk gedragen hadden;
\par 19 Maar hij zelf keerde wederom van de gesneden beelden, die bij Gilgal waren, en zeide: Ik heb een heimelijke zaak aan u, o koning! dewelke zeide: Zwijg! En allen, die om hem stonden, gingen van hem uit.
\par 20 En Ehud kwam tot hem in, daar hij was zittende in een koele opperzaal, die hij voor zich alleen had; zo zeide Ehud: Ik heb een woord Gods aan u. Toen stond hij op van den stoel.
\par 21 Ehud dan reikte zijn linkerhand uit, en nam het zwaard van zijn rechterheup, en stak het in zijn buik;
\par 22 Dat ook het hecht achter het lemmer inging, en het vet om het lemmer toesloot (want hij trok het zwaard niet uit zijn buik), en de drek uitging.
\par 23 Toen ging Ehud uit van de voorzaal, en sloot de deuren der opperzaal voor zich toe, en deed ze in het slot.
\par 24 Als hij uitgegaan was, zo kwamen zijn knechten, en zagen toe, en ziet, de deuren der opperzaal waren in het slot gedaan; zo zeiden zij: Zeker, hij bedekt zijn voeten in de verkoelkamer.
\par 25 Als zij nu tot schamens toe gebeid hadden, ziet, zo opende hij de deuren der opperzaal niet. Toen namen zij den sleutel en deden open; en ziet, hunlieder heer lag ter aarde dood.
\par 26 En Ehud ontkwam, terwijl zij vertoefden; want hij ging voorbij de gesneden beelden, en ontkwam naar Sehirath.
\par 27 En het geschiedde, als hij aankwam, zo blies hij met de bazuin op het gebergte van Efraim; en de kinderen Israels togen met hem af van het gebergte, en hij zelf voor hun aangezicht heen.
\par 28 En hij zeide tot hen: Volgt mij na; want de HEERE heeft uw vijanden, de Moabieten, in ulieder hand gegeven. En zij togen af, hem na, en namen de veren van de Jordaan in naar Moab, en lieten niemand overgaan.
\par 29 En zij sloegen de Moabieten te dier tijd, omtrent tien duizend man, allen vette en allen strijdbare mannen, dat er niet een man ontkwam.
\par 30 Alzo werd Moab te dien dage onder Israels hand te ondergebracht; en het land was stil tachtig jaren.
\par 31 Na hem nu was Samgar, een zoon van Anath, die sloeg de Filistijnen, zeshonderd man, met een ossenstok; alzo verloste hij ook Israel.

\chapter{4}

\par 1 Maar de kinderen Israels voeren voort te doen, dat kwaad was in de ogen des HEEREN, als Ehud gestorven was.
\par 2 Zo verkocht hen de HEERE in de hand van Jabin, koning der Kanaanieten, die te Hazor regeerde; en zijn krijgsoverste was Sisera; dezelve nu woonde in Haroseth der heidenen.
\par 3 Toen riepen de kinderen Israels tot den HEERE; want hij had negenhonderd ijzeren wagenen, en hij had de kinderen Israels met geweld onderdrukt, twintig jaren.
\par 4 Debora nu, een vrouw, die een profetesse was, de huisvrouw van Lappidoth, deze richtte te dier tijd Israel.
\par 5 En zij woonde onder den palmboom van Debora, tussen Rama en tussen Beth-el, op het gebergte van Efraim; en de kinderen Israels gingen op tot haar ten gerichte.
\par 6 En zij zond heen en riep Barak, den zoon van Abinoam, van Kedes-nafthali; en zij zeide tot hem: Heeft de HEERE, de God Israels, niet geboden: Ga heen en trek op den berg Thabor, en neem met u tien duizend man, van de kinderen van Nafthali, en van de kinderen van Zebulon?
\par 7 En Ik zal aan de beek Kison tot u trekken Sisera, den krijgsoverste van Jabin, met zijn wagenen en zijn menigte; en Ik zal hem in uw hand geven?
\par 8 Toen zeide Barak tot haar: Indien gij met mij trekken zult, zo zal ik heen trekken; maar indien gij niet met mij zult trekken, zo zal ik niet trekken.
\par 9 En zij zeide: Ik zal zekerlijk met u trekken, behalve dat de eer de uwe niet zal zijn op dezen weg, dien gij wandelt; want de HEERE zal Sisera verkopen in de hand ener vrouw. Alzo maakte Debora zich op, en toog met Barak naar Kedes.
\par 10 Toen riep Barak Zebulon en Nafthali bijeen te Kedes, en hij toog op, op zijn voeten, met tien duizend man; ook toog Debora met hem op.
\par 11 Heber nu, de Keniet, had zich afgezonderd van Kain, uit de kinderen van Hobab, Mozes' schoonvader; en hij had zijn tenten opgeslagen tot aan den eik in Zaanaim, die bij Kedes is.
\par 12 Toen boodschapten zij Sisera, dat Barak, de zoon van Abinoam, op den berg Thabor getogen was.
\par 13 Zo riep Sisera al zijn wagenen bijeen, negenhonderd ijzeren wagenen, en al het volk, dat met hem was, van Haroseth der heidenen tot de beek Kison.
\par 14 Debora dan zeide tot Barak: Maak u op; want dit is de dag, in welken de HEERE Sisera in uw hand gegeven heeft; is de HEERE niet voor uw aangezicht henen uitgetogen? Zo trok Barak van den berg Thabor af, en tien duizend man achter hem.
\par 15 En de HEERE versloeg Sisera, met al zijn wagenen, en het ganse heirleger, door de scherpte des zwaards, voor het aangezicht van Barak; dat Sisera van den wagen afklom, en vluchtte op zijn voeten.
\par 16 En Barak jaagde ze na, achter de wagenen en achter het heirleger, tot aan Haroseth der heidenen. En het ganse heirleger van Sisera viel door de scherpte des zwaards, dat er niet overbleef tot een toe.
\par 17 Maar Sisera vluchtte op zijn voeten naar de tent van Jael, de huisvrouw van Heber, den Keniet; want er was vrede tussen Jabin, den koning van Hazor, en tussen het huis van Heber, den Keniet.
\par 18 Jael nu ging uit, Sisera tegemoet, en zeide tot hem: Wijk in, mijn heer, wijk in tot mij, vrees niet! En hij week tot haar in de tent, en zij bedekte hem met een deken.
\par 19 Daarna zeide hij tot haar: Geef mij toch een weinig waters te drinken, want mij dorst. Toen opende zij een melkfles, en gaf hem te drinken, en dekte hem toe.
\par 20 Ook zeide hij tot haar: Sta in de deur der tent; en het zij, zo iemand zal komen, en u vragen, en zeggen: Is hier iemand? dat gij zegt: Niemand.
\par 21 Daarna nam Jael, de huisvrouw van Heber, een nagel der tent, en greep een hamer in haar hand, en ging stilletjes tot hem in, en dreef den nagel in den slaap zijns hoofds, dat hij in de aarde vast werd; hij nu was met een diepen slaap bevangen en vermoeid, en stierf.
\par 22 En ziet, Barak vervolgde Sisera; en Jael ging uit hem tegemoet, en zeide tot hem: Kom, en ik zal u den man wijzen, dien gij zoekt. Zo kwam hij tot haar in, en ziet, Sisera lag dood, en de nagel was in den slaap zijns hoofds.
\par 23 Alzo heeft God te dien dage Jabin, den koning van Kanaan, te ondergebracht, voor het aangezicht der kinderen Israels.
\par 24 En de hand der kinderen Israels ging steeds voort, en werd hard over Jabin, den koning van Kanaan, totdat zij Jabin, den koning van Kanaan, hadden uitgeroeid.

\chapter{5}

\par 1 Voorts zong Debora, en Barak, de zoon van Abinoam, ten zelven dage, zeggende:
\par 2 Looft den HEERE, van het wreken der wraken in Israel, van dat het volk zich gewillig heeft aangeboden.
\par 3 Hoort, gij koningen, neemt ter oren, gij vorsten! Ik, den HEERE zal ik zingen, ik zal den HEERE, den God Israels, psalmzingen.
\par 4 HEERE! toen Gij voorttoogt van Seir, toen Gij daarheen traadt van het veld van Edom, beefde de aarde, ook droop de hemel, ook dropen de wolken van water.
\par 5 De bergen vervloten van het aangezicht des HEEREN; zelfs Sinai van het aangezicht des HEEREN, des Gods van Israel.
\par 6 In de dagen van Samgar, den zoon van Anath, in de dagen van Jael, hielden de wegen op, en die op paden wandelden, gingen kromme wegen.
\par 7 De dorpen hielden op in Israel, zij hielden op; totdat ik, Debora, opstond, dat ik opstond, een moeder in Israel.
\par 8 Verkoos hij nieuwe goden, dan was er krijg in de poorten; werd er ook een schild gezien, of een spies, onder veertig duizend in Israel?
\par 9 Mijn hart is tot wetgevers van Israel, die zich gewillig aangeboden hebben onder het volk; looft den HEERE!
\par 10 Gij, die op witte ezelinnen rijdt, gij, die aan het gerichte zit, en gij, die over weg wandelt, spreekt er van!
\par 11 Van het gedruis der schutters, tussen de plaatsen, waar men water schept, spreekt aldaar te zamen van de gerechtigheid des HEEREN, van de gerechtigheden, bewezen aan zijn dorpen in Israel; toen ging des HEEREN volk af tot de poorten.
\par 12 Waak op, waak op, Debora, waak op, waak op, spreek een lied! maak u op, Barak! en leid uw gevangenen gevangen, gij zoon van Abinoam.
\par 13 Toen deed Hij de overgeblevenen heersen over de heerlijken onder het volk; de HEERE doet mij heersen over de geweldigen.
\par 14 Uit Efraim was hun wortel tegen Amalek. Achter u was Benjamin onder uw volken. Uit Machir zijn de wetgevers afgetogen, en uit Zebulon, trekkende door den staf des schrijvers.
\par 15 Ook waren de vorsten in Issaschar met Debora; en gelijk Issaschar, alzo was Barak; op zijn voeten werd hij gezonden in het dal. In Rubens gedeelten waren de inbeeldingen des harten groot.
\par 16 Waarom bleeft gij zitten tussen de stallingen, om te horen het geblaat der kudden? De gedeelten van Ruben hadden grote onderzoekingen des harten.
\par 17 Gilead bleef aan gene zijde van de Jordaan; en Dan, waarom onthield hij zich in schepen! Aser zat aan de zeehaven, en bleef in zijn gescheurde plaatsen.
\par 18 Zebulon, het is een volk, dat zijn ziel versmaad heeft ter dood, insgelijks Nafthali, op de hoogten des velds.
\par 19 De koningen kwamen, zij streden; toen streden de koningen van Kanaan, te Thaanach aan de wateren van Megiddo; zij brachten geen gewin des zilvers daarvan.
\par 20 Van den hemel streden zij, de sterren uit haar loopplaatsen streden tegen Sisera.
\par 21 De beek Kison wentelde hen weg, de beek Kedumin, de beek Kison; vertreed, o mijn ziel! de sterken.
\par 22 Toen werden de paardenhoeven verpletterd, van het rennen, het rennen zijner machtigen.
\par 23 Vloekt Meroz, zegt de Engel des HEEREN, vloekt haar inwoners geduriglijk; omdat zij niet gekomen zijn tot de hulp des HEEREN, tot de hulp des HEEREN, met de helden.
\par 24 Gezegend zij boven de vrouwen Jael, de huisvrouw van Heber, den Keniet; gezegend zij ze boven de vrouwen in de tent!
\par 25 Water eiste hij, melk gaf zij; in een herenschaal bracht zij boter.
\par 26 Haar hand sloeg zij aan den nagel, en haar rechterhand aan den hamer der arbeidslieden; en zij klopte Sisera; zij streek zijn hoofd af, als zij zijn slaap had doorgenageld en doorgedrongen.
\par 27 Tussen haar voeten kromde hij zich, viel henen, lag daar neder; tussen haar voeten kromde hij zich; hij viel; alwaar hij zich kromde, daar lag hij geheel geschonden!
\par 28 De moeder van Sisera keek uit door het venster, en schreeuwde door de tralien: Waarom vertoeft zijn wagen te komen! Waarom blijven de gangen zijner wagenen achter?
\par 29 De wijsten harer staatsvrouwen antwoordden; ook beantwoordde zij haar redenen aan zichzelve:
\par 30 Zouden zij dan den buit niet vinden en delen? een liefje, of twee liefjes, voor iegelijken man? Voor Sisera een buit van verscheidene verven, een buit van verscheidene verven, gestikt; van verscheiden verf aan beide zijden gestikt, voor de buithalzen?
\par 31 Alzo moeten omkomen al Uw vijanden, o HEERE! die Hem daarentegen liefhebben, moeten zijn, als wanneer de zon opgaat in haar kracht. En het land was stil, veertig jaren.

\chapter{6}

\par 1 Maar de kinderen Israels deden, dat kwaad was in de ogen des HEEREN; zo gaf hen de HEERE in de hand der Midianieten, zeven jaren.
\par 2 Als nu de hand der Midianieten sterk werd over Israel, maakten zich de kinderen Israels, vanwege de Midianieten, de holen, die in de bergen zijn, en de spelonken, en de vestingen.
\par 3 Want het geschiedde, als Israel gezaaid had, zo kwamen de Midianieten op, en de Amalekieten, en die van het oosten kwamen ook op tegen hen.
\par 4 En zij legerden zich tegen hen, en verdierven de opkomst des lands, tot daar gij komt te Gaza; en zij lieten geen leeftocht overig in Israel, noch klein vee, noch os, noch ezel.
\par 5 Want zij kwamen op met hun vee en hun tenten; zij kwamen gelijk de sprinkhanen in menigte, dat men hen en hun kemelen niet tellen kon; en zij kwamen in het land, om dat te verderven.
\par 6 Alzo werd Israel zeer verarmd, vanwege de Midianieten. Toen riepen de kinderen Israels tot den HEERE.
\par 7 En het geschiedde, als de kinderen Israels tot den HEERE riepen, ter oorzake van de Midianieten;
\par 8 Zo zond de HEERE een man, die een profeet was, tot de kinderen Israels; die zeide tot hen: Alzo zegt de HEERE, de God Israels: Ik heb u uit Egypte doen opkomen, en u uit het diensthuis uitgevoerd;
\par 9 En Ik heb u verlost van de hand der Egyptenaren, en van de hand van allen, die u drukten; en Ik heb hen voor uw aangezicht uitgedreven, en u hun land gegeven;
\par 10 En Ik zeide tot ulieden: Ik ben de HEERE, uw God; vreest de goden der Amorieten niet, in welker land gij woont; maar gij zijt Mijner stem niet gehoorzaam geweest.
\par 11 Toen kwam een Engel des HEEREN, en zette Zich onder den eik, die te Ofra is, welke aan Joas, den Abi-ezriet, toekwam; en zijn zoon Gideon dorste tarwe bij de pers, om die te vluchten voor het aangezicht der Midianieten.
\par 12 Toen verscheen hem de Engel des HEEREN, en zeide tot hem: De HEERE is met u, gij, strijdbare held!
\par 13 Maar Gideon zeide tot Hem: Och, mijn Heer! zo de HEERE met ons is, waarom is ons dan dit alles wedervaren? en waar zijn al Zijn wonderen, die onze vaders ons verteld hebben, zeggende: Heeft ons de HEERE niet uit Egypte opgevoerd? Doch nu heeft ons de HEERE verlaten, en heeft ons in der Midianieten hand gegeven.
\par 14 Toen keerde zich de HEERE tot hem, en zeide: Ga heen in deze uw kracht, en gij zult Israel uit der Midianieten hand verlossen; heb Ik u niet gezonden?
\par 15 En hij zeide tot Hem: Och, mijn Heer! waarmede zal ik Israel verlossen? Zie, mijn duizend is het armste in Manasse, en ik ben de kleinste in mijns vaders huis.
\par 16 En de HEERE zeide tot hem: Omdat Ik met u zal zijn, zo zult gij de Midianieten slaan, als een enigen man.
\par 17 En hij zeide tot Hem: Indien ik nu genade gevonden heb in Uw ogen, zo doe mij een teken, dat Gij het zijt, Die met mij spreekt.
\par 18 Wijk toch niet van hier, totdat ik tot U kome, en mijn geschenk uitbrenge, en U voorzette. En Hij zeide: Ik zal blijven, totdat gij wederkomt.
\par 19 En Gideon ging in, en bereidde een geitenbokje, en ongezuurde koeken van een efa meels; het vlees leide hij in een korf, en het sop deed hij in een pot; en hij bracht het tot Hem uit, tot onder den eik, en zette het neder.
\par 20 Doch de Engel Gods zeide tot hem: Neem het vlees en de ongezuurde koeken, en leg ze op dien rotssteen, en giet het sop uit; en hij deed alzo.
\par 21 En de Engel des HEEREN stak het uiterste van den staf uit, die in Zijn hand was, en roerde het vlees en de ongezuurde koeken aan; toen ging er vuur op uit de rots, en verteerde het vlees en de ongezuurde koeken. En de Engel des HEEREN bekwam uit zijn ogen.
\par 22 Toen zag Gideon, dat het een Engel des HEEREN was; en Gideon zeide: Ach, Heere, HEERE! daarom, omdat ik een Engel des HEEREN gezien heb van aangezicht tot aangezicht.
\par 23 Doch de HEERE zeide tot hem: Vrede zij u, vrees niet, gij zult niet sterven.
\par 24 Toen bouwde Gideon aldaar den HEERE een altaar, en noemde het: De HEERE is vrede! het is nog tot op dezen dag in Ofra der Abi-ezrieten.
\par 25 En het geschiedde in dienzelven nacht, dat de HEERE tot hem zeide: Neem een var van de ossen, die van uw vader zijn, te weten, den tweeden var, van zeven jaren; en breek af het altaar van Baal, dat van uw vader is, en houw af het bos, dat daarbij is.
\par 26 En bouw den HEERE, uw God, een altaar, op de hoogte dezer sterkte, in een bekwame plaats; en neem den tweeden var, en offer een brandoffer met het hout der hage, die gij zult hebben afgehouwen.
\par 27 Toen nam Gideon tien mannen uit zijn knechten, en deed, gelijk als de HEERE tot hem gesproken had. Doch het geschiedde, dewijl hij zijns vaders huis en de mannen van die stad vreesde, van het te doen bij dag, dat hij het deed bij nacht.
\par 28 Als nu de mannen van die stad des morgens vroeg opstonden, ziet, zo was het altaar van Baal omgeworpen, en de haag, die daarbij was, afgehouwen, en die tweede var was op het gebouwde altaar geofferd.
\par 29 Zo zeiden zij, de een tot den ander: Wie heeft dit stuk gedaan? En als zij onderzochten en navraagden, zo zeide men: Gideon, de zoon van Joas, heeft dit stuk gedaan.
\par 30 Toen zeiden de mannen van die stad tot Joas: Breng uw zoon uit, dat hij sterve, omdat hij het altaar van Baal heeft omgeworpen, en omdat hij de haag, die daarbij was, afgehouwen heeft.
\par 31 Joas daarentegen zeide tot allen, die bij hem stonden: Zult gij voor den Baal twisten; zult gij hem verlossen? Die voor hem zal twisten, zal nog dezen morgen gedood worden! Indien hij een god is, hij twiste voor zichzelven, omdat men zijn altaar heeft omgeworpen.
\par 32 Daarom noemde hij hem te dien dage Jerubbaal, zeggende: Baal twiste tegen hem, omdat hij zijn altaar heeft omgeworpen.
\par 33 Alle Midianieten nu, en Amalekieten, en de kinderen van het oosten, waren samenvergaderd, en zij trokken over, en legerden zich in het dal van Jizreel.
\par 34 Toen toog de Geest des HEEREN Gideon aan, en hij blies met de bazuin, en de Abi-ezrieten werden achter hem bijeengeroepen.
\par 35 Ook zond hij boden in gans Manasse, en die werden ook achter hem bijeengeroepen; desgelijks zond hij boden in Aser, en in Zebulon, en in Nafthali; en zij kwamen op, hun tegemoet.
\par 36 En Gideon zeide tot God: Indien Gij Israel door mijn hand zult verlossen, gelijk als Gij gesproken hebt;
\par 37 Zie, ik zal een wollen vlies op den vloer leggen; indien er dauw op het vlies alleen zal zijn, en droogte op de ganse aarde, zo zal ik weten, dat Gij Israel door mijn hand zult verlossen, gelijk als Gij gesproken hebt.
\par 38 En het geschiedde alzo; want hij stond des anderen daags vroeg op, en drukte het vlies uit, en hij wrong den dauw uit het vlies, een schaal vol waters.
\par 39 En Gideon zeide tot God: Uw toorn ontsteke niet tegen mij, dat ik alleenlijk ditmaal spreke; laat mij toch alleenlijk ditmaal met het vlies verzoeken; er zij toch droogte op het vlies alleen, en op de ganse aarde zij dauw.
\par 40 En God deed alzo in denzelven nacht; want de droogte was op het vlies alleen, en op de ganse aarde was dauw.

\chapter{7}

\par 1 Toen stond Jerubbaal (dewelke is Gideon) vroeg op, en al het volk, dat met hem was; en zij legerden zich aan de fontein van Harod; dat hij het heirleger der Midianieten had tegen het noorden, achter den heuvel More, in het dal.
\par 2 En de HEERE zeide tot Gideon: Des volks is te veel, dat met u is, dan dat Ik de Midianieten in hun hand zou geven; opdat zich Israel niet tegen Mij beroeme, zeggende: Mijn hand heeft mij verlost.
\par 3 Nu dan, roep nu uit voor de oren des volks, zeggende: Wie blode en versaagd is, die kere weder, en spoede zich naar het gebergte van Gilead! Toen keerden uit het volk weder twee en twintig duizend, dat er tien duizend overbleven.
\par 4 En de HEERE zeide tot Gideon: Nog is des volks te veel; doe hen afgaan naar het water, en Ik zal ze u aldaar beproeven; en het zal geschieden, van welken Ik tot u zeggen zal: Deze zal met u trekken, die zal met u trekken; maar al degene, van welken Ik zeggen zal: Deze zal niet met u trekken, die zal niet trekken.
\par 5 En hij deed het volk afgaan naar het water. Toen zeide de HEERE tot Gideon: Al wie met zijn tong uit het water zal lekken, gelijk als een hond zou lekken, dien zult gij alleen stellen; desgelijks al wie op zijn knieen zal bukken om te drinken.
\par 6 Toen was het getal dergenen, die met hun hand tot hun mond gelekt hadden, driehonderd man; maar alle overigen des volks hadden op hun knieen gebukt, om water te drinken.
\par 7 En de HEERE zeide tot Gideon: Door deze driehonderd mannen, die gelekt hebben, zal Ik ulieden verlossen, en de Midianieten in uw hand geven; daarom laat al dat volk weggaan, een ieder naar zijn plaats.
\par 8 En het volk nam den teerkost in hun hand, en hun bazuinen; en hij liet al die mannen van Israel gaan, een iegelijk naar zijn tent; maar die driehonderd man behield hij. En hij had het heirleger der Midianieten beneden in het dal.
\par 9 En het geschiedde in denzelven nacht, dat de HEERE tot hem zeide: Sta op, ga henen af in het leger, want Ik heb het in uw hand gegeven.
\par 10 Vreest gij dan nog af te gaan, zo ga af, gij, en Pura, uw jongen, naar het leger.
\par 11 En gij zult horen, wat zij zullen spreken, en daarna zullen uw handen gesterkt worden, dat gij aftrekken zult in het leger. Toen ging hij af, met Pura, zijn jongen, tot het uiterste der schildwachten, die in het leger waren.
\par 12 En de Midianieten, en Amalekieten, en al de kinderen van het oosten, lagen in het dal, gelijk sprinkhanen in menigte, en hun kemelen waren ontelbaar, gelijk het zand, dat aan den oever der zee is, in menigte.
\par 13 Toen nu Gideon aankwam, ziet, zo was er een man, die zijn metgezel een droom vertelde, en zeide: Zie, ik heb een droom gedroomd, en zie, een geroost gerstebrood wentelde zich in het leger der Midianieten, en het kwam tot aan de tent, en sloeg haar, dat zij viel, en keerde haar om, het onderste boven, dat de tent er lag.
\par 14 En zijn metgezel antwoordde, en zeide: Dit is niet anders, dan het zwaard van Gideon, den zoon van Joas, den Israelietischen man; God heeft de Midianieten en dit ganse leger in zijn hand gegeven.
\par 15 En het geschiedde, als Gideon de vertelling dezes drooms, en zijn uitlegging hoorde, zo aanbad hij; en hij keerde weder tot het leger van Israel, en zeide: Maakt u op, want de HEERE heeft het leger der Midianieten in ulieder hand gegeven.
\par 16 En hij deelde de driehonderd man in drie hopen; en hij gaf een iegelijk een bazuin in zijn hand, en ledige kruiken, en fakkelen in het midden der kruiken.
\par 17 En hij zeide tot hen: Ziet naar mij en doet alzo; en ziet, als ik zal komen aan het uiterste des legers, zo zal het geschieden, gelijk als ik zal doen, alzo zult gij doen.
\par 18 Als ik met de bazuin zal blazen, ik en allen, die met mij zijn, dan zult gijlieden ook met de bazuin blazen, rondom het ganse leger, en gij zult zeggen: Voor den HEERE en voor Gideon!
\par 19 Alzo kwam Gideon, en honderd mannen, die met hem waren, in het uiterste des legers, in het begin van de middelste nachtwaak, als zij maar even de wachters gesteld hadden; en zij bliezen met de bazuinen, ook sloegen zij de kruiken, die in hun hand waren, in stukken.
\par 20 Alzo bliezen de drie hopen met de bazuinen, en braken de kruiken; en zij hielden met de linkerhand de fakkelen, en met hun rechterhand de bazuinen om te blazen; en zij riepen: Het zwaard van den HEERE, en van Gideon!
\par 21 En zij stonden, een iegelijk in zijn plaats, rondom het leger. Toen verliep het ganse leger, en zij schreeuwden en vloden.
\par 22 Als de driehonderd met de bazuinen bliezen, zo zette de HEERE het zwaard des een tegen den anderen, en dat in het ganse leger; en het leger vluchtte tot Beth-sitta toe naar Tseredath, tot aan de grens van Abel-mehola, boven Tabbath.
\par 23 Toen werden de mannen van Israel bijeengeroepen, uit Nafthali, en uit Aser, en uit gans Manasse; en zij jaagden de Midianieten achterna.
\par 24 Ook zond Gideon boden in het ganse gebergte van Efraim, zeggende: Komt af den Midianieten tegemoet, en beneemt hunlieden de wateren, tot aan Beth-bara, te weten de Jordaan; alzo werd alle man van Efraim bijeengeroepen, en zij benamen hun de wateren tot aan Beth-bara, en de Jordaan.
\par 25 En zij vingen twee vorsten der Midianieten, Oreb en Zeeb, en doodden Oreb op den rotssteen Oreb, en Zeeb doodden zij in de perskuip van Zeeb, en vervolgden de Midianieten; en zij brachten de hoofden van Oreb en Zeeb tot Gideon, over de Jordaan.

\chapter{8}

\par 1 Toen zeiden de mannen van Efraim tot hem: Wat stuk is dit, dat gij ons gedaan hebt, dat gij ons niet riept, toen gij heentoogt om te strijden tegen de Midianieten? En zij twistten sterk met hem.
\par 2 Hij daarentegen zeide tot hen: Wat heb ik nu gedaan, gelijk gijlieden; zijn niet de nalezingen van Efraim beter dan de wijnoogst van Abi-ezer?
\par 3 God heeft de vorsten der Midianieten, Oreb en Zeeb, in uw hand gegeven; wat heb ik dan kunnen doen, gelijk gijlieden? Toen liet hun toorn van hem af, als hij dit woord sprak.
\par 4 Als nu Gideon gekomen was aan de Jordaan, ging hij over, met de driehonderd mannen, die bij hem waren, zijnde moede, nochtans vervolgende.
\par 5 En hij zeide tot de lieden van Sukkoth: Geeft toch enige bollen broods aan het volk, dat mijn voetstappen volgt, want zij zijn moede; en ik jaag Zebah en Tsalmuna, de koningen der Midianieten, achterna.
\par 6 Maar de oversten van Sukkoth zeiden: Is dan de handpalm van Zebah en Tsalmuna alrede in uw hand, dat wij aan uw heir brood zouden geven?
\par 7 Toen zeide Gideon: Daarom, als de HEERE Zebah en Tsalmuna in mijn hand geeft, zo zal ik uw vlees dorsen met doornen der woestijn, en met distelen.
\par 8 En hij toog van daar op naar Pnuel, en sprak tot hen desgelijks. En de lieden van Pnuel antwoordden hem, gelijk als de lieden van Sukkoth geantwoord hadden.
\par 9 Daarom sprak hij ook tot de lieden van Pnuel, zeggende: Als ik met vrede wederkome, zal ik deze toren afwerpen.
\par 10 Zebah nu en Tsalmuna waren te Karkor, en hun legers met hen, omtrent vijftien duizend, al de overgeblevenen van het ganse leger der kinderen van het oosten; en de gevallenen waren honderd en twintig duizend mannen, die het zwaard uittrokken.
\par 11 En Gideon toog opwaarts, den weg dergenen, die in tenten wonen, tegen het oosten van Nobah en Jogbeha; en hij sloeg dat leger, want het leger was zorgeloos.
\par 12 En Zebah en Tsalmuna vloden; doch hij jaagde hen na; en hij ving de beide koningen der Midianieten, Zebah en Tsalmuna, en verschrikte het ganse leger.
\par 13 Toen nu Gideon, de zoon van Joas, van den strijd wederkwam, voor den opgang der zon,
\par 14 Zo ving hij een jongen van de lieden te Sukkoth, en ondervraagde hem; die schreef hem op de oversten van Sukkoth, en hun oudsten, zeven en zeventig mannen.
\par 15 Toen kwam hij tot de lieden van Sukkoth, en zeide: Ziet daar Zebah en Tsalmuna, van dewelke gij mij smadelijk verweten hebt, zeggende: Is de handpalm van Zebah en Tsalmuna alrede in uw hand, dat wij aan uw mannen, die moede zijn, brood zouden geven?
\par 16 En hij nam de oudsten dier stad, en doornen der woestijn, en distelen, en deed het den lieden van Sukkoth door dezelve verstaan.
\par 17 En den toren van Pnuel wierp hij af, en doodde de lieden der stad.
\par 18 Daarna zeide hij tot Zebah en Tsalmuna: Wat waren het voor mannen, die gij te Thabor doodsloegt? En zij zeiden: Gelijk gij, alzo waren zij, enerlei, van gedaante als koningszonen.
\par 19 Toen zeide hij: Het waren mijn broeders, zonen mijner moeder; zo waarlijk als de HEERE leeft, zo gij hen hadt laten leven, ik zou ulieden niet doden!
\par 20 En hij zeide tot Jether, zijn eerstgeborene: Sta op, dood hen; maar de jongeling trok zijn zwaard niet uit, want hij vreesde, dewijl hij nog een jongeling was.
\par 21 Toen zeiden Zebah en Tsalmuna: Sta gij op, en val op ons aan, want naar dat de man is, zo is zijn macht. Zo stond Gideon op, en doodde Zebah en Tsalmuna, en nam de maantjes, die aan de halzen hunner kemelen waren.
\par 22 Toen zeiden de mannen van Israel tot Gideon: Heers over ons, zo gij als uw zoon en uws zoons zoon, dewijl gij ons van der Midianieten hand verlost hebt.
\par 23 Maar Gideon zeide tot hen: Ik zal over u niet heersen; ook zal mijn zoon over u niet heersen; de HEERE zal over u heersen.
\par 24 Voorts zeide Gideon tot hen: Een begeerte zal ik van u begeren: geeft mij maar een iegelijk een voorhoofdsiersel van zijn roof; want zij hadden gouden voorhoofdsierselen gehad, dewijl zij Ismaelieten waren.
\par 25 En zij zeiden: Wij zullen ze gaarne geven; en zij spreidden een kleed uit, en wierpen daarop een iegelijk een voorhoofdsiersel van zijn roof.
\par 26 En het gewicht der gouden voorhoofdsierselen, die hij begeerd had, was duizend en zevenhonderd sikkelen gouds, zonder de maantjes, en ketenen, en purperen klederen, die de koningen der Midianieten aangehad hadden, en zonder de halsbanden, die aan de halzen hunner kemelen geweest waren.
\par 27 En Gideon maakte daarvan een efod, en stelde die in zijn stad, te Ofra; en gans Israel hoereerde aldaar denzelven na; en het werd Gideon en zijn huis tot een valstrik.
\par 28 Alzo werden de Midianieten te ondergebracht voor het aangezicht der kinderen Israels, en hieven hun hoofd niet meer op. En het land was stil veertig jaren, in de dagen van Gideon.
\par 29 En Jerubbaal, de zoon van Joas, ging henen en woonde in zijn huis.
\par 30 Gideon nu had zeventig zonen, die uit zijn heupe voortgekomen waren; want hij had vele vrouwen.
\par 31 En zijn bijwijf, hetwelk te Sichem was, baarde hem ook een zoon; en hij noemde zijn naam Abimelech.
\par 32 En Gideon, de zoon van Joas, stierf in goeden ouderdom; en hij werd begraven in het graf van zijn vader Joas, te Ofra, des Abi-ezriets.
\par 33 En het geschiedde, als Gideon gestorven was, dat de kinderen Israels zich omkeerden, en de Baals nahoereerden; en zij stelden zich Baal-berith tot een God.
\par 34 En de kinderen Israels dachten niet aan den HEERE, hun God, Die hen gered had van de hand van al hun vijanden van rondom.
\par 35 En zij deden geen weldadigheid bij het huis van Jerubbaal, dat is Gideon, naar al het goede, dat hij bij Israel gedaan had.

\chapter{9}

\par 1 Abimelech nu, de zoon van Jerubbaal, ging henen naar Sichem, tot de broeders zijner moeder; en hij sprak tot hen, en tot het ganse geslacht van het huis van den vader zijner moeder, zeggende:
\par 2 Spreekt toch voor de oren van alle burgers van Sichem: Wat is u beter, dat zeventig mannen, alle zonen van Jerubbaal, over u heersen, of dat een man over u heerse? Gedenkt ook, dat ik uw been en uw vlees ben.
\par 3 Toen spraken de broeders zijner moeder van hem, voor de oren van alle burgers van Sichem, al dezelve woorden; en hun hart neigde zich naar Abimelech; want zij zeiden: Hij is onze broeder.
\par 4 En zij gaven hem zeventig zilverlingen, uit het huis van Baal-berith; en Abimelech huurde daarmede ijdele en lichtvaardige mannen, die hem navolgden.
\par 5 En hij kwam in zijns vaders huis te Ofra, en doodde zijn broederen, de zonen van Jerubbaal, zeventig mannen, op een steen; doch Jotham, de jongste zoon van Jerubbaal werd overgelaten, want hij had zich verstoken.
\par 6 Toen vergaderden zich alle burgeren van Sichem, en het ganse huis van Millo, en gingen heen en maakten Abimelech ten koning, bij den hogen eik, die bij Sichem is.
\par 7 Als zij dit Jotham aanzeiden, zo ging hij heen, en stond op de hoogte des bergs Gerizim, en verhief zijn stem, en riep, en hij zeide tot hen: Hoort naar mij, gij, burgers van Sichem! en God zal naar ulieden horen.
\par 8 De bomen gingen eens heen, om een koning over zich te zalven, en zij zeiden tot den olijfboom: Wees gij koning over ons.
\par 9 Maar de olijfboom zeide tot hen: Zoude ik mijn vettigheid verlaten, die God en de mensen in mij prijzen? En zoude ik heengaan om te zweven over de bomen?
\par 10 Toen zeiden de bomen tot den vijgeboom: Kom gij, wees koning over ons.
\par 11 Maar de vijgeboom zeide tot hen: Zou ik mijn zoetigheid en mijn goede vrucht verlaten? En zou ik heengaan om te zweven over de bomen?
\par 12 Toen zeiden de bomen tot den wijnstok: Kom gij, wees koning over ons.
\par 13 Maar de wijnstok zeide tot hen: Zou ik mijn most verlaten, die God en mensen vrolijk maakt? En zou ik heengaan om te zweven over de bomen?
\par 14 Toen zeiden al de bomen tot den doornenbos: Kom gij, wees koning over ons.
\par 15 En de doornenbos zeide tot de bomen: Indien gij mij in waarheid tot een koning over u zalft, zo komt, vertrouwt u onder mijn schaduw; maar indien niet, zo ga vuur uit den doornenbos, en vertere de cederen van den Libanon.
\par 16 Alzo nu, indien gij het in waarheid en oprechtheid gedaan hebt, dat gij Abimelech koning gemaakt hebt, en indien gij welgedaan hebt bij Jerubbaal en bij zijn huis, en indien gij hem naar de verdienste zijner handen gedaan hebt.
\par 17 (Want mijn vader heeft voor ulieden gestreden, en hij heeft zijn ziel verre weggeworpen, en u uit der Midianieten hand gered;
\par 18 Maar gij zijt heden opgestaan tegen het huis mijns vaders, en hebt zijn zonen, zeventig mannen, op een steen gedood; en gij hebt Abimelech, een zoon zijner dienstmaagd, koning gemaakt over de burgers van Sichem, omdat hij uw broeder is);
\par 19 Indien gij dan in waarheid en in oprechtheid bij Jerubbaal en bij zijn huis te dezen dage gehandeld hebt, zo weest vrolijk over Abimelech, en hij zij ook vrolijk over ulieden.
\par 20 Maar indien niet, zo ga vuur uit van Abimelech, en vertere de burgers van Sichem, en het huis van Millo; en vuur ga uit van de burgers van Sichem, en van het huis van Millo, en vertere Abimelech!
\par 21 Toen vlood Jotham, en vluchtte, en ging naar Beer; en hij woonde aldaar vanwege zijn broeder Abimelech.
\par 22 Als nu Abimelech drie jaren over Israel geheerst had,
\par 23 Zo zond God een bozen geest tussen Abimelech en tussen de burgers van Sichem; en de burgers van Sichem handelden trouweloos tegen Abimelech;
\par 24 Opdat het geweld, gedaan aan de zeventig zonen van Jerubbaal, kwame, en opdat hun bloed gelegd wierd op Abimelech, hun broeder, die hen gedood had, en op de burgers van Sichem, die zijn handen gesterkt hadden om zijn broeders te doden.
\par 25 En de burgers van Sichem bestelden tegen hem, die op de hoogten der bergen lagen leiden, en al wie voorbij hen op den weg doorging, beroofden zij; en het werd Abimelech aangezegd.
\par 26 Gaal, de zoon van Ebed, kwam ook met zijn broederen, en zij gingen over in Sichem; en de burgeren van Sichem verlieten zich op hem.
\par 27 En zij togen uit in het veld, en lazen hun wijnbergen af, en traden de druiven, en maakten lofliederen; en zij gingen in het huis huns gods, en aten en dronken, en vloekten Abimelech.
\par 28 En Gaal, de zoon van Ebed, zeide: Wie is Abimelech, en wat is Sichem, dat wij hem dienen zouden? is hij niet een zoon van Jerubbaal? en Zebul zijn bevelhebber? dient liever de mannen van Hemor, den vader van Sichem; want waarom zouden wij hem dienen?
\par 29 Och, dat dit volk in mijn hand ware! ik zoude Abimelech wel verdrijven. En tot Abimelech zeide hij: Vermeerder uw heir, en trek uit.
\par 30 Als Zebul, de overste der stad, de woorden van Gaal, den zoon van Ebed, hoorde, zo ontstak zijn toorn.
\par 31 En hij zond listiglijk boden tot Abimelech, zeggende: Zie, Gaal, de zoon van Ebed, en zijn broeders zijn te Sichem gekomen, en zie, zij, met deze stad, handelen vijandiglijk tegen u.
\par 32 Zo maak u nu op bij nacht, gij en het volk, dat met u is, en leg lagen in het veld.
\par 33 En het geschiede in den morgen, als de zon opgaat, zo maak u vroeg op, en overval deze stad; en zie, zo hij en het volk, dat met hem is, tot u uittrekken, zo doe hem, gelijk als uw hand vinden zal.
\par 34 Abimelech dan maakte zich op, en al het volk, dat met hem was, bij nacht; en zij leiden lagen op Sichem, met vier hopen.
\par 35 En Gaal, de zoon van Ebed, ging uit, en stond aan de deur van de stadspoort; en Abimelech rees op, en al het volk, dat met hem was, uit de achterlage.
\par 36 Als Gaal dat volk zag, zo zeide hij tot Zebul: Zie, er komt volk af van de hoogten der bergen. Zebul daarentegen zeide tot hem: Gij ziet de schaduw der bergen voor mensen aan.
\par 37 Maar Gaal voer wijders voort te spreken en zeide: Zie daar volk, afkomende uit het midden des lands, en een hoop komt van den weg van den eik Meonenim.
\par 38 Toen zeide Zebul tot hem: Waar is nu uw mond, waarmede gij zeidet: Wie is Abimelech, dat wij hem zouden dienen? is niet dit het volk, dat gij veracht hebt? trek toch nu uit en strijd tegen hem!
\par 39 En Gaal trok uit voor het aangezicht der burgeren van Sichem, en hij streed tegen Abimelech.
\par 40 En Abimelech jaagde hem na, want hij vlood voor zijn aangezicht; en er vielen vele verslagenen tot aan de deur der stads poort.
\par 41 Abimelech nu bleef te Aruma; en Zebul verdreef Gaal en zijn broederen, dat zij te Sichem niet mochten wonen.
\par 42 En het geschiedde des anderen daags dat het volk uittrok in het veld, en zij zeiden het Abimelech aan.
\par 43 Toen nam hij het volk, en deelde hen in drie hopen, en hij leide lagen in het veld; en hij zag toe, en ziet, het volk trok uit de stad, zo maakte hij zich tegen hen op, en sloeg hen.
\par 44 Want Abimelech en de hopen, die bij hem waren, overvielen hen, en bleven staan aan de deur der stadspoort; en de twee andere hopen overvielen allen, die in het veld waren, en sloegen hen.
\par 45 Voorts streed Abimelech tegen de stad dienzelven gansen dag, en nam de stad in, en doodde het volk, dat daarin was; en hij brak de stad af, en bezaaide haar met zout.
\par 46 Als alle burgers des torens van Sichem dat hoorden, zo gingen zij in de sterkte, in het huis van den god Berith.
\par 47 En het werd Abimelech aangezegd, dat alle burgeren des torens van Sichem zich verzameld hadden.
\par 48 Zo ging Abimelech op den berg Zalmon, hij en al het volk, dat met hem was; en Abimelech nam een bijl in zijn hand, en hieuw een tak van de bomen, en nam hem op, en leide hem op zijn schouder; en hij zeide tot het volk, dat bij hem was: Wat gij mij hebt zien doen, haast u, doet als ik.
\par 49 Zo hieuw ook al het volk een iegelijk zijn tak af, en zij volgden Abimelech na, en leiden ze aan de sterkte, en verbrandden daardoor de sterkte met vuur; dat ook alle lieden des torens van Sichem stierven, omtrent duizend mannen en vrouwen.
\par 50 Voorts toog Abimelech naar Thebez, en hij legerde zich tegen Thebez, en nam haar in.
\par 51 Doch er was een sterke toren in het midden der stad; zo vloden daarheen al de mannen en de vrouwen, en alle burgers van de stad, en sloten voor zich toe; en zij klommen op het dak des torens.
\par 52 Toen kwam Abimelech tot aan den toren, en bestormde dien; en hij genaakte tot aan de deur des torens, om dien met vuur te verbranden.
\par 53 Maar een vrouw wierp een stuk van een molensteen op Abimelechs hoofd; en zij verpletterde zijn hersenpan.
\par 54 Toen riep hij haastelijk den jongen, die zijn wapenen droeg, en zeide tot hem: Trek uw zwaard uit, en dood mij, opdat zij niet van mij zeggen: Een vrouw heeft hem gedood. En zijn jongen doorstak hem, dat hij stierf.
\par 55 Als nu de mannen van Israel zagen, dat Abimelech dood was, zo gingen zij een iegelijk naar zijn plaats.
\par 56 Alzo deed God wederkeren het kwaad van Abimelech, dat hij aan zijn vader gedaan had, dodende zijn zeventig broederen.
\par 57 Desgelijks al het kwaad der lieden van Sichem deed God wederkeren op hun hoofd; en de vloek van Jotham, den zoon van Jerubbaal, kwam over hen.

\chapter{10}

\par 1 Na Abimelech nu stond op, om Israel te behouden, Thola, een zoon van Pua, zoon van Dodo, een man van Issaschar; en hij woonde te Samir, op het gebergte van Efraim.
\par 2 En hij richtte Israel drie en twintig jaren; en hij stierf, en werd begraven te Samir.
\par 3 En na hem stond op Jair, de Gileadiet; en hij richtte Israel twee en twintig jaren.
\par 4 En hij had dertig zonen, rijdende op dertig ezelveulens, en die hadden dertig steden, die zij noemden Havvoth-jair, tot op dezen dag, dewelke in het land van Gilead zijn.
\par 5 En Jair stierf, en werd begraven te Kamon.
\par 6 Toen voeren de kinderen Israels voort te doen, dat kwaad was in de ogen des HEEREN, en dienden de Baals, en Astharoth, en de goden van Syrie, en de goden van Sidon, en de goden van Moab, en de goden der kinderen Ammons, mitsgaders de goden der Filistijnen; en zij verlieten den HEERE, en dienden Hem niet.
\par 7 Zo ontstak de toorn des HEEREN tegen Israel; en Hij verkocht hen in de hand der Filistijnen, en in de hand der kinderen Ammons.
\par 8 En zij onderdrukten en vertraden de kinderen Israels in datzelve jaar; achttien jaren, onderdrukten zij al de kinderen Israels, die aan gene zijde van de Jordaan waren, in het land der Amorieten, dat in Gilead is.
\par 9 Daartoe togen de kinderen Ammons over de Jordaan, om te krijgen, zelfs tegen Juda, en tegen Benjamin, en tegen het huis van Efraim; zodat het Israel zeer bang werd.
\par 10 Toen riepen de kinderen Israels tot den HEERE, zeggende: Wij hebben tegen U gezondigd, zo omdat wij onzen God hebben verlaten, als dat wij de Baals gediend hebben.
\par 11 Maar de HEERE zeide tot de kinderen Israels: Heb Ik u niet van de Egyptenaren, en van de Amorieten, en van de kinderen Ammons, en van de Filistijnen,
\par 12 En de Sidoniers, en Amalekieten, en Maonieten, die u onderdrukten, toen gij tot Mij riept, alsdan uit hun hand verlost?
\par 13 Nochtans hebt gij Mij verlaten, en andere goden gediend; daarom zal Ik u niet meer verlossen.
\par 14 Gaat henen, roept tot de goden, die gij verkoren hebt; laten die u verlossen, ter tijd uwer benauwdheid.
\par 15 Maar de kinderen Israels zeiden tot den HEERE: Wij hebben gezondigd; doe Gij ons, naar alles, wat goed is in Uw ogen; alleenlijk verlos ons toch te dezen dage!
\par 16 En zij deden de vreemde goden uit hun midden weg, en dienden den HEERE. Toen werd Zijn ziel verdrietig over den arbeid van Israel.
\par 17 En de kinderen Ammons werden bijeengeroepen, en legerden zich in Gilead; daarentegen werden de kinderen Israels vergaderd, en legerden zich te Mizpa.
\par 18 Toen zeide het volk, de oversten van Gilead, de een tot den ander: Wie is de man, die beginnen zal te strijden tegen de kinderen Ammons? die zal tot een hoofd zijn over alle inwoners van Gilead.

\chapter{11}

\par 1 Jeftha nu, de Gileadiet, was een strijdbaar held, maar hij was een hoerekind; doch Gilead had Jeftha gegenereerd.
\par 2 Gileads huisvrouw baarde hem ook zonen; en de zonen dezer vrouw, groot geworden zijnde, stieten Jeftha uit, en zeiden tot hem: Gij zult in het huis onzes vaders niet erven, want gij zijt een zoon van een andere vrouw.
\par 3 Toen vlood Jeftha voor het aangezicht zijner broederen, en woonde in het land Tob; en ijdele mannen vergaderden zich tot Jeftha, en togen met hem uit.
\par 4 En het geschiedde, na enige dagen, dat de kinderen Ammons tegen Israel krijgden.
\par 5 Zo geschiedde het, als de kinderen Ammons tegen Israel krijgden, dat de oudsten van Gilead heengingen, om Jeftha te halen uit het land van Tob.
\par 6 En zij zeiden tot Jeftha: Kom, en wees ons tot een overste, opdat wij strijden tegen de kinderen Ammons.
\par 7 Maar Jeftha zeide tot de oudsten van Gilead: Hebt gijlieden mij niet gehaat, en mij uit mijn vaders huis verstoten? waarom zijt gij dan nu tot mij gekomen, terwijl gij in benauwdheid zijt?
\par 8 En de oudsten van Gilead zeiden tot Jeftha: Daarom zijn wij nu tot u wedergekomen, dat gij met ons trekt, en tegen de kinderen Ammons strijdt; en gij zult ons tot een hoofd zijn, over alle inwoners van Gilead.
\par 9 Toen zeide Jeftha tot de oudsten van Gilead: Zo gijlieden mij wederhaalt, om te strijden tegen de kinderen Ammons, en de HEERE hen voor mijn aangezicht geven zal, zal ik u dan tot een hoofd zijn?
\par 10 En de oudsten van Gilead zeiden tot Jeftha: De HEERE zij toehoorder tussen ons, indien wij niet alzo naar uw woord doen.
\par 11 Alzo ging Jeftha met de oudsten van Gilead, en het volk stelde hem tot een hoofd en overste over zich. En Jeftha sprak al zijn woorden voor het aangezicht des HEEREN te Mizpa.
\par 12 Voorts zond Jeftha boden tot den koning der kinderen Ammons, zeggende: Wat hebben ik en gij met elkander te doen, dat gij tot mij gekomen zijt, om tegen mijn land te krijgen?
\par 13 En de koning der kinderen Ammons zeide tot de boden van Jeftha: Omdat Israel, als hij uit Egypte optoog, mijn land genomen heeft, van de Arnon af tot aan de Jabbok, en tot aan de Jordaan; zo geef mij dat nu weder met vrede.
\par 14 Maar Jeftha voer wijders voort, en zond boden tot den koning der kinderen Ammons.
\par 15 En hij zeide tot hem: Zo zegt Jeftha: Israel heeft het land der Moabieten, en het land der kinderen Ammons niet genomen;
\par 16 Want als zij uit Egypte optogen, zo wandelde Israel door de woestijn tot aan de Schelfzee, en kwam te Kades.
\par 17 En Israel zond boden tot den koning der Edomieten, zeggende: Laat mij toch door uw land doortrekken; maar de koning der Edomieten gaf geen gehoor. En hij zond ook tot den koning der Moabieten, die ook niet wilde. Alzo bleef Israel in Kades.
\par 18 Daarna wandelde hij in de woestijn, en toog om het land der Edomieten en het land der Moabieten, en kwam van den opgang der zon aan het land der Moabieten, en zij legerden zich op gene zijde van de Arnon; maar zij kwamen niet binnen de landpale der Moabieten; want de Arnon is de landpale der Moabieten.
\par 19 Maar Israel zond boden tot Sihon, den koning der Amorieten, koning van Hesbon, en Israel zeide tot hem: Laat ons toch door uw land doortrekken tot aan mijn plaats.
\par 20 Doch Sihon betrouwde Israel niet door zijn landpale door te trekken; maar Sihon verzamelde al zijn volk, en zij legerden zich te Jaza; en hij streed tegen Israel.
\par 21 En de HEERE, de God Israels, gaf Sihon met al zijn volk in de hand van Israel, dat zij hen sloegen; alzo nam Israel erfelijk in het ganse land der Amorieten, die in datzelve land woonden.
\par 22 En zij namen erfelijk in de ganse landpale der Amorieten, van de Arnon af tot aan de Jabbok, en van de woestijn tot aan de Jordaan.
\par 23 Zo heeft nu de HEERE, de God Israels, de Amorieten voor het aangezicht van Zijn volk Israel uit de bezitting verdreven; en zoudt gij hunlieder erfgenaam zijn?
\par 24 Zoudt gij niet dengene erven, dien uw god Kamos voor u uit de bezitting verdreef? Alzo zullen wij al dengene erven, dien de HEERE, onze God, voor ons aangezicht uit de bezitting verdrijft.
\par 25 Nu voorts, zijt gij veel beter dan Balak, de zoon van Zippor, de koning der Moabieten? heeft hij ooit met Israel getwist? heeft hij ook ooit tegen hem gekrijgd?
\par 26 Terwijl Israel driehonderd jaren gewoond heeft in Hesbon, en in haar stedekens, en in Aroer en in de stedekens, en in al de steden, die aan de zijde van de Arnon zijn; waarom hebt gij het dan in dien tijd niet gered?
\par 27 Ook heb ik tegen u niet gezondigd, maar gij doet kwalijk bij mij, dat gij tegen mij krijgt; de HEERE, Die Rechter is, richte heden tussen de kinderen Israels en tussen de kinderen Ammons!
\par 28 Maar de koning der kinderen Ammons hoorde niet naar de woorden van Jeftha, die hij tot hem gezonden had.
\par 29 Toen kwam de Geest des HEEREN op Jeftha, dat hij Gilead en Manasse doortrok; want hij trok door tot Mizpa in Gilead, en van Mizpa in Gilead trok hij door tot de kinderen Ammons.
\par 30 En Jeftha beloofde den HEERE een gelofte, en zeide: Indien Gij de kinderen Ammons ganselijk in mijn hand zult geven;
\par 31 Zo zal het uitgaande, dat uit de deur van mijn huis mij tegemoet zal uitgaan, als ik met vrede van de kinderen Ammons wederkom, dat zal des HEEREN zijn, en ik zal het offeren ten brandoffer.
\par 32 Alzo trok Jeftha door naar de kinderen Ammons, om tegen hen te strijden; en de HEERE gaf hen in zijn hand.
\par 33 En hij sloeg hen van Aroer af tot daar gij komt te Minnith, twintig steden, en tot aan Abel-keramim, met een zeer groten slag. Alzo werden de kinderen Ammons te ondergebracht voor het aangezicht der kinderen Israels.
\par 34 Toen nu Jeftha te Mizpa bij zijn huis kwam, ziet, zo ging zijn dochter uit hem tegemoet, met trommelen en met reien. Zij nu was alleen, een enig kind; hij had uit zich anders geen zoon of dochter.
\par 35 En het geschiedde, als hij haar zag, zo verscheurde hij zijn klederen, en zeide: Ach, mijn dochter! gij hebt mij ganselijk nedergebogen, en gij zijt onder degenen, die mij beroeren; want ik heb mijn mond opengedaan tot den HEERE, en ik zal niet kunnen teruggaan.
\par 36 En zij zeide tot hem: Mijn vader! hebt gij uw mond opengedaan tot den HEERE, doe mij, gelijk als uit uw mond gegaan is; naardien u de HEERE volkomene wraak gegeven heeft van uw vijanden, van de kinderen Ammons.
\par 37 Voorts zeide zij tot haar vader: Laat deze zaak aan mij geschieden: Laat twee maanden van mij af, dat ik heenga, en ga tot de bergen, en bewene mijn maagdom, ik en mijn gezellinnen.
\par 38 En hij zeide: Ga heen; en hij liet haar twee maanden gaan. Toen ging zij heen met haar gezellinnen, en beweende haar maagdom op de bergen.
\par 39 En het geschiedde ten einde van twee maanden dat zij tot haar vader wederkwam, die aan haar volbracht zijn gelofte, die hij beloofd had; en zij heeft geen man bekend. Voorts werd het een gewoonheid in Israel,
\par 40 Dat de dochteren Israels van jaar tot jaar heengingen, om de dochter van Jeftha, den Gileadiet, aan te spreken, vier dagen in het jaar.

\chapter{12}

\par 1 Toen werden de mannen van Efraim bijeengeroepen, en trokken over naar het noorden; en zij zeiden tot Jeftha: Waarom zijt gij doorgetogen om te strijden tegen de kinderen Ammons, en hebt ons niet geroepen, om met u te gaan? wij zullen uw huis met u met vuur verbranden.
\par 2 En Jeftha zeide tot hen: Ik en mijn volk waren zeer twistig met de kinderen Ammons; en ik heb ulieden geroepen, maar gij hebt mij uit hun hand niet verlost.
\par 3 Als ik nu zag, dat gij niet verlostet, zo stelde ik mijn ziel in mijn hand, en toog door tot de kinderen Ammons, en de HEERE gaf hen in mijn hand; waarom zijt gij dan te dezen dage tot mij opgekomen, om tegen mij te strijden?
\par 4 En Jeftha vergaderde alle mannen van Gilead, en streed met Efraim; en de mannen van Gilead sloegen Efraim, want de Gileadieten, zijnde tussen Efraim en tussen Manasse, zeiden: Gijlieden zijt vluchtelingen van Efraim.
\par 5 Want de Gileadieten namen den Efraimieten de veren van de Jordaan af; en het geschiedde, als de vluchtelingen van Efraim zeiden: Laat mij overgaan; zo zeiden de mannen van Gilead tot hem: Zijt gij een Efraimiet? wanneer hij zeide: Neen;
\par 6 Zo zeiden zij tot hem: Zeg nu Schibboleth; maar hij zeide: Sibbolet, en kon het alzo niet recht spreken; zo grepen zij hem, en versloegen hem aan de veren van de Jordaan, dat te dier tijd van Efraim vielen twee en veertig duizend.
\par 7 Jeftha nu richtte Israel zes jaren; en Jeftha, de Gileadiet, stierf, en werd begraven in de steden van Gilead.
\par 8 En na hem richtte Israel Ebzan, van Bethlehem.
\par 9 En hij had dertig zonen; en hij zond dertig dochteren naar buiten, en bracht dertig dochteren van buiten in voor zijn zonen; en hij richtte Israel zeven jaren.
\par 10 Toen stierf Ebzan, en werd begraven te Bethlehem.
\par 11 En na hem richtte Israel Elon, de Zebuloniet, en hij richtte Israel tien jaren.
\par 12 En Elon, de Zebuloniet, stierf, en werd begraven te Ajalon, in het land van Zebulon.
\par 13 En na hem richtte Israel Abdon, een zoon van Hillel, de Pirhathoniet.
\par 14 En hij had veertig zonen, en dertig zoons zonen, rijdende op zeventig ezelveulens; en hij richtte Israel acht jaren.
\par 15 Toen stierf Abdon, een zoon van Hillel, de Pirhathoniet; en hij werd begraven te Pirhathon, in het land van Efraim, op den berg van den Amalekiet.

\chapter{13}

\par 1 En de kinderen Israels voeren voort te doen, dat kwaad was in de ogen des HEEREN; zo gaf de HEERE hen in de hand der Filistijnen veertig jaren.
\par 2 En er was een man van Zora, uit het geslacht van een Daniet, wiens naam was Manoach; en zijn huisvrouw was onvruchtbaar en baarde niet.
\par 3 En een Engel des HEEREN verscheen aan deze vrouw, en Hij zeide tot haar: Zie nu, gij zijt onvruchtbaar, en hebt niet gebaard; maar gij zult zwanger worden, en een zoon baren.
\par 4 Zo wacht u toch nu, en drink geen wijn noch sterken drank, en eet niets onreins.
\par 5 Want zie, gij zult zwanger worden, en een zoon baren, op wiens hoofd geen scheermes zal komen; want dat knechtje zal een Nazireer Gods zijn, van moeders buik af; en hij zal beginnen Israel te verlossen uit der Filistijnen hand.
\par 6 Toen kwam deze vrouw in, en sprak tot haar man, zeggende: Er kwam een Man Gods tot mij, Wiens aangezicht was als het aangezicht van een Engel Gods, zeer vreselijk; en ik vraagde Hem niet, van waar Hij was, en Zijn naam gaf Hij mij niet te kennen.
\par 7 Maar Hij zeide tot mij: Zie, gij zult zwanger worden, en een zoon baren; zo drink nu geen wijn noch sterken drank, en eet niets onreins; want dat knechtje zal een Nazireer Gods zijn, van moeders buik af tot op den dag zijns doods.
\par 8 Toen aanbad Manoach den HEERE vuriglijk, en zeide: Och, HEERE! dat toch de Man Gods, Dien Gij gezonden hebt, weder tot ons kome, en ons lere, wat we dat knechtje doen zullen, dat geboren zal worden.
\par 9 En God verhoorde de stem van Manoach; en de Engel Gods kwam wederom tot de vrouw. Zij nu zat in het veld, doch haar man Manoach was niet bij haar.
\par 10 Zo haastte de vrouw, en liep, en gaf het haar man te kennen; en zij zeide tot hem: Zie, die Man is mij verschenen, Welke op dien dag tot mij kwam.
\par 11 Toen stond Manoach op, en ging zijn huisvrouw na; en hij kwam tot dien Man, en zeide tot Hem: Zijt gij die Man, Dewelke tot deze vrouw gesproken hebt? En Hij zeide: Ik ben het.
\par 12 Toen zeide Manoach: Nu, dat Uw woorden komen; maar wat zal des knechtjes wijze en zijn werk zijn?
\par 13 En de Engel des HEEREN zeide tot Manoach: Van alles, wat Ik tot de vrouw gezegd heb, zal zij zich wachten.
\par 14 Zij zal niet eten van iets, dat van den wijnstok des wijns voortkomt; en wijn en sterken drank zal zij niet drinken, noch iets onreins eten; al wat Ik haar geboden heb, zal zij onderhouden.
\par 15 Toen zeide Manoach tot den Engel des HEEREN: Laat ons U toch ophouden, en een geitenbokje voor Uw aangezicht bereiden.
\par 16 Maar de Engel des HEEREN zeide tot Manoach: Indien gij Mij zult ophouden, Ik zal van uw brood niet eten; en indien gij een brandoffer zult doen, dat zult gij den HEERE offeren. Want Manoach wist niet, dat het een Engel des HEEREN was.
\par 17 En Manoach zeide tot den Engel des HEEREN: Wat is Uw naam, opdat wij U vereren, wanneer Uw woord zal komen.
\par 18 En de Engel des HEEREN zeide tot hem: Waarom vraagt gij dus naar Mijn naam? Die is toch Wonderlijk.
\par 19 Toen nam Manoach een geitenbokje, en het spijsoffer, en offerde het op den rotssteen, den HEERE. En Hij handelde wonderlijk in Zijn doen; en Manoach en zijn huisvrouw zagen toe.
\par 20 En het geschiedde, als de vlam van het altaar opvoer naar den hemel, zo voer de Engel des HEEREN op in de vlam des altaars. Als Manoach en zijn huisvrouw dat zagen, zo vielen zij op hun aangezichten ter aarde.
\par 21 En de Engel des HEEREN verscheen niet meer aan Manoach, en aan zijn huisvrouw. Toen bekende Manoach, dat het een Engel des HEEREN was.
\par 22 En Manoach zeide tot zijn huisvrouw: Wij zullen zekerlijk sterven, omdat wij God gezien hebben.
\par 23 Maar zijn huisvrouw zeide tot hem: Zo de HEERE lust had ons te doden, Hij had het brandoffer en spijsoffer van onze hand niet aangenomen, noch ons dit alles getoond, noch ons om dezen tijd laten horen, zulks als dit is.
\par 24 Daarna baarde deze vrouw een zoon, en zij noemde zijn naam Simson; en dat knechtje werd groot, en de HEERE zegende het.
\par 25 En de Geest des HEEREN begon hem bij wijlen te drijven in het leger van Dan, tussen Zora en tussen Esthaol.

\chapter{14}

\par 1 En Simson ging af naar Thimnath, en gezien hebbende een vrouw te Thimnath, van de dochteren der Filistijnen,
\par 2 Zo ging hij opwaarts, en gaf het zijn vader en zijn moeder te kennen, en zeide: Ik heb een vrouw gezien te Thimnath, van de dochteren der Filistijnen; nu dan, neem mij die tot een vrouw.
\par 3 Maar zijn vader zeide tot hem, mitsgaders zijn moeder: Is er geen vrouw onder de dochteren uwer broeders, en onder al mijn volk, dat gij heengaat, om een vrouw te nemen van de Filistijnen, die onbesnedenen? En Simson zeide tot zijn vader: Neem mij die, want zij is bevallig in mijn ogen.
\par 4 Zijn vader nu en zijn moeder wisten niet, dat dit van den HEERE was, dat hij gelegenheid zocht van de Filistijnen; want de Filistijnen heersten te dier tijd over Israel.
\par 5 Alzo ging Simson, met zijn vader en zijn moeder, henen af naar Thimnath. Als zij nu kwamen tot aan de wijngaarden van Thimnath, ziet daar, een jonge leeuw, brullende hem tegemoet.
\par 6 Toen werd de Geest des HEEREN vaardig over hem, dat hij hem van een scheurde, gelijk men een bokje van een scheurt, en er was niets in zijn hand; doch hij gaf zijn vader en zijn moeder niet te kennen, wat hij gedaan had.
\par 7 En hij kwam af, en sprak tot de vrouw; en zij beviel in Simsons ogen.
\par 8 En na sommige dagen kwam hij weder, om haar te nemen; toen week hij af, om het aas van den leeuw te bezien, en ziet, een bijenzwerm was in het lichaam van den leeuw, met honig.
\par 9 En hij nam dien in zijn handen, en ging voort, al gaande en etende; en hij ging tot zijn vader en tot zijn moeder, en gaf hun daarvan, en zij aten; doch hij gaf hun niet te kennen, dat hij den honig uit het lichaam van den leeuw genomen had.
\par 10 Als nu zijn vader afgekomen was tot die vrouw, zo maakte Simson aldaar een bruiloft, want alzo plachten de jongelingen te doen.
\par 11 En het geschiedde, als zij hem zagen, zo namen zij dertig metgezellen, die bij hem zouden zijn.
\par 12 Simson dan zeide tot hen: Ik zal nu ulieden een raadsel te raden geven; indien gij mij dat in de zeven dagen dezer bruiloft wel zult verklaren en uitvinden, zo zal ik ulieden geven dertig fijne lijnwaadsklederen, en dertig wisselklederen.
\par 13 En indien gij het mij niet zult kunnen verklaren, zo zult gijlieden mij geven dertig fijne lijnwaadsklederen, en dertig wisselklederen. En zij zeiden tot hem: Geef uw raadsel te raden, en laat het ons horen.
\par 14 En hij zeide tot hen: Spijze ging uit van den eter, en zoetigheid ging uit van den sterke. En zij konden dat raadsel in drie dagen niet verklaren.
\par 15 Daarna geschiedde het op den zevenden dag, dat zij tot de huisvrouw van Simson zeiden: Overreed uw man, dat hij ons dat raadsel verklare, opdat wij niet misschien u, en het huis uws vaders, met vuur verbranden. Hebt gijlieden ons genodigd, om het onze te bezitten; is het zo niet?
\par 16 En Simsons huisvrouw weende voor hem en zeide: Gij haat mij maar, en hebt mij niet lief; gij hebt den kinderen mijns volks een raadsel te raden gegeven, en hebt het mij niet verklaard. En hij zeide tot haar: Zie, ik heb het mijn vader en mijn moeder niet verklaard, zou ik het u dan verklaren?
\par 17 En zij weende voor hem, op den zevenden der dagen in dewelke zij deze bruiloft hadden; zo geschiedde het op den zevenden dag, dat hij het haar verklaarde, want zij perste hem; en zij verklaarde dat raadsel den kinderen haars volks.
\par 18 Toen zeiden de mannen der stad tot hem, op den zevenden dag, eer de zon onderging: Wat is zoeter dan honig? en wat is sterker dan een leeuw? En hij zeide tot hen: Zo gij met mijn kalf niet hadt geploegd, gij zoudt mijn raadsel niet hebben uitgevonden.
\par 19 Toen werd de Geest des HEEREN vaardig over hem, en hij ging af naar de Askelonieten, en sloeg van hen dertig man; en hij nam hun gewaad, en gaf de wisselklederen aan degenen, die dat raadsel verklaard hadden. Doch zijn toorn ontstak, en hij ging op in zijns vaders huis.
\par 20 En de huisvrouw van Simson werd zijns metgezels, die hem vergezelschapt had.

\chapter{15}

\par 1 En het geschiedde na sommige dagen, in de dagen van den tarweoogst, dat Simson zijn huisvrouw bezocht met een geitenbokje, en hij zeide: Laat mij tot mijn huisvrouw ingaan in de kamer; maar haar vader liet hem niet toe in te gaan.
\par 2 Want haar vader zeide: Ik sprak zeker, dat gij haar ganselijk haattet, zo heb ik haar aan uw metgezel gegeven. Is niet haar kleinste zuster schoner dan zij? Laat ze u toch zijn in de plaats van haar.
\par 3 Toen zeide Simson tot henlieden: Ik ben ditmaal onschuldig van de Filistijnen, wanneer ik aan hen kwaad doe.
\par 4 En Simson ging heen, en ving driehonderd vossen; en hij nam fakkelen, en keerde staart aan staart, en deed een fakkel tussen twee staarten in het midden.
\par 5 En hij stak de fakkelen aan met vuur, en liet ze lopen in het staande koren der Filistijnen; en hij stak in brand zowel de korenhopen als het staande koren, zelfs tot de wijngaarden en olijfbomen toe.
\par 6 Toen zeiden de Filistijnen: Wie heeft dit gedaan? En men zeide: Simson, de schoonzoon van den Thimniet, omdat hij zijn huisvrouw heeft genomen, en heeft haar aan zijn metgezel gegeven. Toen kwamen de Filistijnen op, en verbrandden haar en haar vader met vuur.
\par 7 Toen zeide Simson tot hen: Zoudt gij alzo doen? Zeker, als ik mij aan u gewroken heb, zo zal ik daarna ophouden.
\par 8 En hij sloeg hen, den schenkel en de heup, met een groten slag; en hij ging af, en woonde op de hoogte van de rots Etam.
\par 9 Toen togen de Filistijnen op, en legerden zich tegen Juda, en breidden zich uit in Lechi.
\par 10 En de mannen van Juda zeiden: Waarom zijt gijlieden tegen ons opgetogen? En zij zeiden: Wij zijn opgetogen om Simson te binden, om hem te doen, gelijk als hij ons gedaan heeft.
\par 11 Toen kwamen drie duizend mannen af uit Juda tot het hol der rots Etam, en zeiden tot Simson: Wist gij niet, dat de Filistijnen over ons heersen? Waarom hebt gij ons dan dit gedaan? En hij zeide tot hen: Gelijk als zij mij gedaan hebben, alzo heb ik hunlieden gedaan.
\par 12 En zij zeiden tot hem: Wij zijn afgekomen om u te binden, om u over te geven in de hand der Filistijnen. Toen zeide Simson tot hen: Zweert mij, dat gijlieden op mij niet zult aanvallen.
\par 13 En zij spraken tot hem, zeggende: Neen, maar wij zullen u wel binden, en u in hunlieder hand overgeven; doch wij zullen u geenszins doden. En zij bonden hem met twee nieuwe touwen, en voerden hem op van de rots.
\par 14 Als hij kwam tot Lechi, zo juichten de Filistijnen hem tegemoet; maar de Geest des HEEREN werd vaardig over hem; en de touwen, die aan zijn armen waren, werden als linnen draden, die van het vuur gebrand zijn, en zijn banden versmolten van zijn handen.
\par 15 En hij vond een vochtig ezelskinnebakken, en hij strekte zijn hand uit, en nam het, en sloeg daarmede duizend man.
\par 16 Toen zeide Simson: Met een ezelskinnebakken, een hoop, twee hopen, met een ezelskinnebakken heb ik duizend man geslagen.
\par 17 En het geschiedde, als hij geeindigd had te spreken, zo wierp hij het kinnebakken uit zijn hand, en hij noemde dezelve plaats Ramath-lechi.
\par 18 Als hem nu zeer dorstte, zo riep hij tot den HEERE, en zeide: Gij hebt door de hand van Uw knecht dit grote heil gegeven; zou ik dan nu van dorst sterven, en vallen in de hand dezer onbesnedenen?
\par 19 Toen kloofde God de holle plaats, die in Lechi is, en er ging water uit van dezelve, en hij dronk. Toen kwam zijn geest weder, en hij werd levend. Daarom noemde hij haar naam: De fontein des aanroepers, die in Lechi is, tot op dezen dag.
\par 20 En hij richtte Israel, in de dagen der Filistijnen, twintig jaren.

\chapter{16}

\par 1 Simson nu ging heen naar Gaza; en hij zag aldaar een vrouw, die een hoer was; en hij ging tot haar in.
\par 2 Toen werd den Gazieten gezegd: Simson is hier ingekomen; zo gingen zij rondom, en leiden hem den gansen nacht lagen in de stadspoort; doch zij hielden zich den gansen nacht stil, zeggende: Tot aan het morgenlicht, dan zullen wij hem doden.
\par 3 Maar Simson lag tot middernacht toe; toen stond hij op ter middernacht, en hij greep de deuren der stadspoort met de beide posten, en nam ze weg met den grendelboom, en leide ze op zijn schouderen, en droeg ze opwaarts op de hoogte des bergs, die in het gezicht van Hebron is.
\par 4 En het geschiedde daarna, dat hij een vrouw lief kreeg, aan de beek Sorek, welker naam was Delila.
\par 5 Toen kwamen de vorsten der Filistijnen tot haar op, en zeiden tot haar: Overreed hem, en zie, waarin zijn grote kracht zij, en waarmede wij hem zouden machtig worden, en hem binden, om hem te plagen; zo zullen wij u geven, een iegelijk, duizend en honderd zilverlingen.
\par 6 Delila dan zeide tot Simson: Verklaar mij toch, waarin uw grote kracht zij, en waarmede gij zoudt kunnen gebonden worden, dat men u plage.
\par 7 En Simson zeide tot haar: Indien zij mij bonden met zeven verse zelen, die niet verdroogd zijn, zo zou ik zwak worden, en wezen als een ander mens.
\par 8 Toen brachten de vorsten der Filistijnen tot haar op zeven verse zelen, die niet verdroogd waren; en zij bond hem daarmede.
\par 9 De achterlage nu zat bij haar in een kamer. Zo zeide zij tot hem: De Filistijnen over u, Simson! Toen verbrak hij de zelen, gelijk als een snoertje van grof vlas verbroken wordt, als het vuur riekt. Alzo werd zijn kracht niet bekend.
\par 10 Toen zeide Delila tot Simson: Zie, gij hebt met mij gespot, en leugenen tot mij gesproken; verklaar mij toch nu, waarmede gij zoudt kunnen gebonden worden?
\par 11 En hij zeide tot haar: Indien zij mij vastbonden met nieuwe touwen, met dewelke geen werk gedaan is, zo zou ik zwak worden, en wezen als een ander mens.
\par 12 Toen nam Delila nieuwe touwen, en bond hem daarmede, en zeide tot hem: De Filistijnen over u, Simson! (De achterlage nu was zittende in een kamer.) Toen verbrak hij ze van zijn armen als een draad.
\par 13 En Delila zeide tot Simson: Tot hiertoe hebt gij met mij gespot, en leugenen tot mij gesproken; verklaar mij toch nu, waarmede gij zoudt kunnen gebonden worden. En hij zeide tot haar: Indien gij de zeven haarlokken mijns hoofds vlochtet aan een weversboom.
\par 14 En zij maakte ze vast met een pin, en zeide tot hem: De Filistijnen over u, Simson! Toen waakte hij op uit zijn slaap, en nam weg de pin der gevlochten haarlokken, en den weversboom.
\par 15 Toen zeide zij tot hem: Hoe zult gij zeggen: Ik heb u lief, daar uw hart niet met mij is? Gij hebt nu driemaal met mij gespot, en mij niet verklaard, waarin uw grote kracht zij.
\par 16 En het geschiedde, als zij hem alle dagen met haar woorden perste, en hem moeilijk viel, dat zijn ziel verdrietig werd tot stervens toe;
\par 17 Zo verklaarde hij haar zijn ganse hart, en zeide tot haar: Er is nooit een scheermes op mijn hoofd gekomen, want ik ben een Nazireer Gods van mijn moeders buik af; indien ik geschoren wierd, zo zou mijn kracht van mij wijken, en ik zou zwak worden, en wezen als alle de mensen.
\par 18 Als nu Delila zag, dat hij haar zijn ganse hart verklaard had, zo zond zij heen, en riep de vorsten der Filistijnen, zeggende: Komt ditmaal op, want hij heeft mij zijn ganse hart verklaard. En de vorsten der Filistijnen kwamen tot haar op, en brachten dat geld in hun hand.
\par 19 Toen deed zij hem slapen op haar knieen, en riep een man en liet hem de zeven haarlokken zijns hoofds afscheren, en zij begon hem te plagen; en zijn kracht week van hem.
\par 20 En zij zeide: De Filistijnen over u, Simson! En hij ontwaakte uit zijn slaap, en zeide: Ik zal ditmaal uitgaan, als op andere malen, en mij uitschudden; want hij wist niet, dat de HEERE van hem geweken was.
\par 21 Toen grepen hem de Filistijnen, en groeven zijn ogen uit; en zij voerden hem af naar Gaza, en bonden hem met twee koperen ketenen, en hij was malende in het gevangenhuis.
\par 22 En het haar zijns hoofds begon weder te wassen, gelijk toen hij geschoren werd.
\par 23 Toen verzamelden zich de vorsten der Filistijnen, om hun god Dagon een groot offer te offeren, en tot vrolijkheid; en zij zeiden: Onze god heeft onze vijand Simson in onze hand gegeven.
\par 24 Desgelijks als hem het volk zag, loofden zij hun god, want zij zeiden: Onze god heeft in onze hand gegeven onzen vijand, en die ons land verwoestte, en die onzer verslagenen velen maakte!
\par 25 En het geschiedde, als hun hart vrolijk was, dat zij zeiden: Roept Simson, dat hij voor ons spele. En zij riepen Simson uit het gevangenhuis; en hij speelde voor hun aangezichten, en zij deden hem staan tussen de pilaren.
\par 26 Toen zeide Simson tot den jongen, die hem bij de hand hield: Laat mij gaan, dat ik de pilaren betaste, op dewelke het huis gevestigd is, dat ik daaraan leune.
\par 27 Het huis nu was vol mannen en vrouwen; ook waren daar alle vorsten der Filistijnen; en op het dak waren omtrent drie duizend mannen en vrouwen, die toezagen, als Simson speelde.
\par 28 Toen riep Simson tot den HEERE, en zeide: Heere, HEERE! gedenk toch mijner, en sterk mij toch alleenlijk ditmaal, o God! dat ik mij met een wrake voor mijn twee ogen aan de Filistijnen wreke.
\par 29 En Simson vatte de twee middelste pilaren, op dewelke het huis was gevestigd, en waarop het steunde, den enen met zijn rechterhand, en den anderen met zijn linkerhand;
\par 30 En Simson zeide: Mijn ziel sterve met de Filistijnen; en hij boog zich met kracht, en het huis viel op de vorsten, en op al het volk, dat daarin was. En de doden, die hij in zijn sterven gedood heeft, waren meer, dan die hij in zijn leven gedood had.
\par 31 Toen kwamen zijn broeders af, en het ganse huis zijns vaders, en namen hem op, en brachten hem opwaarts, en begroeven hem tussen Zora en tussen Esthaol, in het graf van zijn vader Manoach; hij nu had Israel gericht twintig jaren.

\chapter{17}

\par 1 En er was een man van het gebergte van Efraim, wiens naam was Micha.
\par 2 Die zeide tot zijn moeder: De duizend en honderd zilverlingen, die u ontnomen zijn, om dewelke gij gevloekt hebt, en ook voor mijn oren gesproken hebt, zie, dat geld is bij mij, ik heb dat genomen. Toen zeide zijn moeder: Gezegend zij mijn zoon den HEERE!
\par 3 Alzo gaf hij aan zijn moeder de duizend en honderd zilverlingen weder. Doch zijn moeder zeide: Ik heb dat geld den HEERE ganselijk geheiligd van mijn hand, voor mijn zoon, om een gesneden beeld en een gegoten beeld te maken; zo zal ik het u nu wedergeven.
\par 4 Maar hij gaf dat geld aan zijn moeder weder. En zijn moeder nam tweehonderd zilverlingen, en gaf ze den goudsmid, die maakte daarvan een gesneden beeld en een gegoten beeld; dat was in het huis van Micha.
\par 5 En de man Micha had een godshuis; en hij maakte een efod, en terafim, en vulde de hand van een uit zijn zonen, dat hij hem tot een priester ware.
\par 6 In diezelve dagen was er geen koning in Israel; een iegelijk deed, wat recht was in zijn ogen.
\par 7 Nu was er een jongeling van Bethlehem-juda, van het geslacht van Juda; deze was een Leviet, en verkeerde aldaar als vreemdeling.
\par 8 En deze man was uit die stad, uit Bethlehem-juda getogen, om te verkeren, waar hij gelegenheid zou vinden. Als hij nu kwam aan het gebergte van Efraim tot aan het huis van Micha, om zijn weg te gaan,
\par 9 Zo zeide Micha tot hem: Van waar komt gij? En hij zeide tot hem: Ik ben een Leviet, van Bethlehem-juda, en ik wandel, om te verkeren, waar ik gelegenheid zal vinden.
\par 10 Toen zeide Micha tot hem: Blijf bij mij, en wees mij tot een vader en tot een priester; en ik zal u jaarlijks geven tien zilverlingen, en orde van klederen, en uw leeftocht; alzo ging de Leviet met hem.
\par 11 En de Leviet bewilligde bij dien man te blijven; en de jongeling was hem als een van zijn zonen.
\par 12 En Micha vulde de hand van den Leviet, dat hij hem tot een priester wierd; alzo was hij in het huis van Micha.
\par 13 Toen zeide Micha: Nu weet ik, dat de HEERE mij weldoen zal, omdat ik dezen Leviet tot een priester heb.

\chapter{18}

\par 1 In die dagen was er geen koning in Israel; en in dezelve dagen zocht de stam der Danieten voor zich een erfenis om te wonen; want hun was tot op dien dag onder de stammen van Israel niet genoegzaam ter erfenis toegevallen.
\par 2 Zo zonden de kinderen van Dan uit hun geslacht vijf mannen uit hun einden, mannen, die strijdbaar waren, van Zora en van Esthaol, om het land te verspieden, en dat te doorzoeken; en zij zeiden tot hen: Gaat, doorzoekt het land. En zij kwamen aan het gebergte van Efraim, tot aan het huis van Micha, en vernachtten aldaar.
\par 3 Zijnde bij het huis van Micha, zo kenden zij de stem van den jongeling, den Leviet; en zij weken daarheen, en zeiden tot hem: Wie heeft u hier gebracht, en wat doet gij alhier, en wat hebt gij hier?
\par 4 En hij zeide tot hen: Zo en zo heeft Micha mij gedaan; en hij heeft mij gehuurd, en ik ben hem tot een priester.
\par 5 Toen zeiden zij tot hem: Vraag toch God, dat wij mogen weten, of onze weg, op welken wij wandelen, voorspoedig zal zijn.
\par 6 En de priester zeide tot hen: Gaat in vrede; uw weg, welken gij zult heentrekken, is voor den HEERE.
\par 7 Toen gingen die vijf mannen heen, en kwamen te Lais; en zij zagen het volk, hetwelk in derzelver midden was, zijnde gelegen in zekerheid, naar de wijze der Sidoniers, stil en zeker zijnde; en daar was geen erfheer, die iemand om enige zaak schande aandeed in dat land; ook waren zij verre van de Sidoniers, en hadden niets te doen met enigen mens.
\par 8 En zij kwamen tot hun broederen te Zora en te Esthaol, en hun broeders zeiden tot hen: Wat zegt gijlieden?
\par 9 En zij zeiden: Maakt u op, en laat ons tot hen optrekken; want wij hebben dat land bezien, en ziet, het is zeer goed; zoudt gij dan stil zijn? Weest niet lui om te trekken, dat gij henen inkomt, om dat land in erfelijke bezitting te nemen;
\par 10 (Als gij daarhenen komt, zo zult gij komen tot een zorgeloos volk, en dat land is wijd van ruimte) want God heeft het in uw hand gegeven; een plaats, alwaar geen gebrek is van enig ding, dat op de aarde is.
\par 11 Toen reisden van daar uit het geslacht der Danieten, van Zora en van Esthaol, zeshonderd man, aangegord met krijgswapenen.
\par 12 En zij togen op, en legerden zich bij Kirjath-jearim, in Juda; daarom noemden zij deze plaats, Machane-dan, tot op dezen dag; ziet, het is achter Kirjath-jearim.
\par 13 En van daar togen zij door naar het gebergte van Efraim, en zij kwamen tot aan het huis van Micha.
\par 14 Toen antwoordden de vijf mannen, die gegaan waren om het land van Lais te verspieden, en zeiden tot hun broederen: Weet gijlieden ook, dat in die huizen een efod is, en terafim, en een gesneden en een gegoten beeld? Zo weet nu, wat u te doen zij.
\par 15 Toen weken zij daarheen, en kwamen aan het huis van den jongeling, den Leviet, ten huize van Micha; en zij vraagden hem naar vrede.
\par 16 En de zeshonderd mannen, die van de kinderen van Dan waren, met hun krijgswapenen aangegord, bleven staan aan de deur van de poort.
\par 17 Maar de vijf mannen, die gegaan waren om het land te verspieden, gingen op, kwamen daarhenen in, en namen weg het gesneden beeld, en den efod, en de terafim, en het gegoten beeld; de priester nu bleef staan aan de deur van de poort, met de zeshonderd mannen, die met krijgswapenen aangegord waren.
\par 18 Als die nu ten huize van Micha waren ingegaan, en het gesneden beeld, den efod, en de terafim, en het gegoten beeld weggenomen hadden, zo zeide de priester tot hen: Wat doet gijlieden?
\par 19 En zij zeiden tot hem: Zwijg, leg uw hand op uw mond, en ga met ons, en wees ons tot een vader en tot een priester! Is het beter, dat gij een priester zijt voor het huis van een man, of dat gij een priester zijt voor een stam, en een geslacht in Israel?
\par 20 Toen werd het hart van den priester vrolijk, en hij nam den efod, en de terafim, en het gesneden beeld, en hij kwam in het midden des volks.
\par 21 Alzo keerden zij zich, en togen voort; en zij stelden de kinderkens, en het vee, en de bagage voor zich.
\par 22 Als zij nu verre van Micha's huis gekomen waren, zo werden de mannen, zijnde in de huizen, die bij het huis van Micha waren, bijeengeroepen, en zij achterhaalden de kinderen van Dan.
\par 23 En zij riepen de kinderen van Dan na; dewelke hun aangezichten omkeerden, en zeiden tot Micha: Wat is u, dat gij bijeengeroepen zijt?
\par 24 Toen zeide hij: Gijlieden hebt mijn goden, die ik gemaakt had, weggenomen, mitsgaders den priester, en zijt weggegaan; wat heb ik nu meer? Wat is het dan, dat gij tot mij zegt: Wat is u?
\par 25 Maar de kinderen van Dan zeiden tot hem: Laat uw stem bij ons niet horen, opdat niet misschien mannen, van bitteren gemoede, op u aanvallen, en gij uw leven verliest, en het leven van uw huis.
\par 26 Alzo gingen de kinderen van Dan huns weegs; en Micha, ziende, dat zij sterker waren dan hij, zo keerde hij om, en kwam weder tot zijn huis.
\par 27 Zij dan namen wat Micha gemaakt had, en den priester, die hij gehad had, en kwamen te Lais, tot een stil en zeker volk, en sloegen hen met de scherpte des zwaards, en de stad verbrandden zij met vuur.
\par 28 En er was niemand, die hen verloste; want zij was verre van Sidon, en zij hadden niets met enigen mens te doen; en zij lag in het dal, dat bij Beth-rechob is. Daarna herbouwden zij de stad, en woonden daarin.
\par 29 En zij noemden den naam der stad Dan, naar den naam huns vaders Dan, die aan Israel geboren was; hoewel de naam dezer stad te voren Lais was.
\par 30 En de kinderen van Dan richtten voor zich dat gesneden beeld op; en Jonathan, de zoon van Gersom, den zoon van Manasse, hij en zijn zonen waren priesters voor den stam der Danieten, tot den dag toe, dat het land gevankelijk is weggevoerd.
\par 31 Alzo stelden zij onder zich het gesneden beeld van Micha, dat hij gemaakt had, al de dagen, dat het huis Gods te Silo was.

\chapter{19}

\par 1 Het geschiedde ook in die dagen, als er geen koning was in Israel, dat er een Levietisch man was, verkerende als vreemdeling aan de zijden van het gebergte van Efraim, die zich een vrouw, een bijwijf, nam van Bethlehem-juda.
\par 2 Maar zijn bijwijf hoereerde, bij hem zijnde, en toog van hem weg naar haars vaders huis, tot Bethlehem-juda; en zij was aldaar enige dagen, te weten vier maanden.
\par 3 En haar man maakte zich op, en toog haar na, om naar haar hart te spreken, om haar weder te halen; en zijn jongen was bij hem, en een paar ezels. En zij bracht hem in het huis haars vaders. En als de vader van de jonge vrouw hem zag, werd hij vrolijk over zijn ontmoeting.
\par 4 En zijn schoonvader, de vader van de jonge vrouw, behield hem, dat hij drie dagen bij hem bleef; en zij aten en dronken, en vernachtten aldaar.
\par 5 Op den vierden dag nu geschiedde het, dat zij des morgens vroeg op waren, en hij opstond om weg te trekken; toen zeide de vader van de jonge dochter tot zijn schoonzoon: Sterk uw hart met een bete broods, en daarna zult gijlieden wegtrekken.
\par 6 Zo zaten zij neder, en zij beiden aten te zamen, en dronken. Toen zeide de vader van de jonge vrouw tot den man: Bewillig toch en vernacht, en laat uw hart vrolijk zijn.
\par 7 Maar de man stond op, om weg te trekken. Toen drong hem zijn schoonvader, dat hij aldaar wederom vernachtte.
\par 8 Als hij op den vijfden dag des morgens vroeg op was, om weg te trekken, zo zeide de vader van de jonge vrouw: Sterk toch uw hart. En zij vertoefden, totdat de dag zich neigde; en zij beiden aten te zamen.
\par 9 Toen maakte zich de man op, om weg te trekken, hij, en zijn bijwijf, en zijn jongen; en zijn schoonvader, de vader van de jonge vrouw, zeide: Zie toch, de dag heeft afgenomen, dat het avond zal worden, vernacht toch; zie, de dag legert zich, vernacht hier, en laat uw hart vrolijk zijn, en maak u morgen vroeg op uws weegs, en ga naar uw tent.
\par 10 Doch de man wilde niet vernachten, maar stond op, en trok weg, en kwam tot tegenover Jebus (dewelke is Jeruzalem), en met hem het paar gezadelde ezelen; ook was zijn bijwijf met hem.
\par 11 Als zij nu bij Jebus waren, zo was de dag zeer gedaald; en de jongen zeide tot zijn heer: Trek toch voort, en laat ons in deze stad der Jebusieten wijken, en daarin vernachten.
\par 12 Maar zijn heer zeide tot hem: Wij zullen herwaarts niet wijken tot een vreemde stad, die niet is van de kinderen Israels; maar wij zullen voorttrekken tot Gibea toe.
\par 13 Voorts zeide hij tot zijn jongen: Ga voort, dat wij tot een van die plaatsen naderen, en te Gibea of te Rama vernachten.
\par 14 Alzo togen zij voort, en wandelden; en de zon ging hun onder bij Gibea, dewelke Benjamins is;
\par 15 En zij weken daarheen, dat zij inkwamen, om in Gibea te vernachten. Toen hij nu inkwam, zat hij neder in een straat der stad, want er was niemand, die hen in huis nam, om te vernachten.
\par 16 En ziet, een oud man kwam van zijn werk van het veld in den avond, welke man ook was van het gebergte van Efraim, doch als vreemdeling verkeerde te Gibea; maar de lieden dezer plaats waren kinderen van Jemini.
\par 17 Als hij nu zijn ogen ophief, zo zag hij dien reizenden man op de straat der stad; en de oude man zeide: Waar trekt gij henen, en van waar komt gij?
\par 18 En hij zeide tot hem: Wij trekken door van Bethlehem-juda tot aan de zijden van het gebergte van Efraim, van waar ik ben; en ik was naar Bethlehem-juda getogen, maar ik trek nu naar het huis des HEEREN; en er is niemand, die mij in huis neemt.
\par 19 Daar toch onze ezelen zowel stro als voeder hebben, en ook brood en wijn is voor mij, en voor uw dienstmaagd, en voor den jongen, die bij uw knechten is; er is aan geen ding gebrek.
\par 20 Toen zeide de oude man: Vrede zij u! al wat u ontbreekt, is toch bij mij; alleenlijk vernacht niet op de straat.
\par 21 En hij bracht hem in zijn huis, en gaf aan de ezelen voeder; en hun voeten gewassen hebbende, zo aten en dronken zij.
\par 22 Toen zij nu hun hart vrolijk maakten, ziet, zo omringden de mannen van die stad (mannen, die Belials kinderen waren) het huis, kloppende op de deur; en zij spraken tot den ouden man, den heer des huizes, zeggende: Breng den man, die in uw huis gekomen is, uit, opdat wij hem bekennen.
\par 23 En de man, de heer des huizes, ging tot hen uit, en zeide tot hen: Niet, mijn broeders, doet toch zo kwalijk niet; naardien deze man in mijn huis gekomen is, zo doet zulke dwaasheid niet.
\par 24 Ziet, mijn dochter die maagd is, en zijn bijwijf, die zal ik nu uitbrengen, dat gij die schendt, en haar doet, wat goed is in uw ogen; maar doet aan dezen man zulk een dwaas ding niet.
\par 25 Maar de mannen wilden naar hem niet horen. Toen greep de man zijn bijwijf, en bracht haar uit tot hen daarbuiten; en zij bekenden haar, en waren met haar bezig den gansen nacht tot aan den morgen, en lieten haar gaan, als de dageraad oprees.
\par 26 En deze vrouw kwam tegen het aanbreken van den morgenstond, en viel neder voor de deur van het huis des mans, waarin haar heer was, totdat het licht werd.
\par 27 Als nu haar heer des morgens opstond en de deuren van het huis opendeed, en uitging om zijns weegs te gaan, ziet, zo lag de vrouw, zijn bijwijf, aan de deur van het huis, en haar handen op den dorpel.
\par 28 En hij zeide tot haar: Sta op, en laat ons trekken; maar niemand antwoordde. Toen nam hij haar op den ezel, en de man maakte zich op, en toog naar zijn plaats.
\par 29 Als hij nu in zijn huis kwam, zo nam hij een mes, en greep zijn bijwijf, en deelde haar met haar beenderen in twaalf stukken; en hij zond ze in alle landpalen van Israel.
\par 30 En het geschiedde, dat al wie het zag, zeide: Zulks is niet geschied noch gezien, van dien dag af, dat de kinderen Israels uit Egypteland zijn opgetogen, tot op dezen dag; legt uw hart daarop, geeft raad en spreekt!

\chapter{20}

\par 1 Toen togen alle kinderen Israels uit, en de vergadering verzamelde zich, als een enig man, van Dan af tot Ber-seba toe, ook het land van Gilead, tot den HEERE te Mizpa.
\par 2 En uit de hoeken des gansen volks stelden zich al de stammen van Israel in de vergadering van het volk Gods, vierhonderd duizend man te voet, die het zwaard uittrokken.
\par 3 (De kinderen Benjamins nu hoorden, dat de kinderen Israels opgetogen naar Mizpa.) En de kinderen Israels zeiden: Spreekt, hoe is dit kwaad geschied?
\par 4 Toen antwoordde de Levietische man, de man van de vrouw, die gedood was, en zeide: Ik kwam met mijn bijwijf te Gibea, dewelke Benjamins is, om te vernachten.
\par 5 En de burgers van Gibea maakten zich tegen mij op, en omringden tegen mij het huis bij nacht; zij dachten mij te doden, en mijn bijwijf hebben zij geschonden, dat zij gestorven is.
\par 6 Toen greep ik mijn bijwijf, en deelde haar, en zond haar in het ganse land der erfenis van Israel, omdat zij een schandelijke daad en dwaasheid in Israel gedaan hadden.
\par 7 Ziet, gij allen zijt kinderen Israels, geeft hier voor ulieden woord en raad!
\par 8 Toen maakte zich al het volk op, als een enig man, zeggende: Wij zullen niet gaan, een ieder naar zijn tent, noch wijken, een ieder naar zijn huis.
\par 9 Maar nu, dit is de zaak, die wij aan Gibea zullen doen: tegen haar bij het lot!
\par 10 En wij zullen tien mannen nemen van honderd, van alle stammen Israels, en honderd van duizend, en duizend van tienduizend, om teerkost te nemen voor het volk, opdat zij, komende te Gibea-benjamins, haar doen naar al de dwaasheid, die zij in Israel gedaan heeft.
\par 11 Alzo werden alle mannen van Israel verzameld tot deze stad, verbonden als een enig man.
\par 12 En de stammen van Israel zonden mannen door den gansen stam van Benjamin, zeggende: Wat voor een kwaad is dit, dat onder ulieden geschied is?
\par 13 Zo geeft nu die mannen, die kinderen Belials, die te Gibea zijn, dat wij hen doden, en het kwaad uit Israel wegdoen. Doch de kinderen van Benjamin wilden niet horen naar de stem van hun broederen, de kinderen Israels.
\par 14 Maar de kinderen van Benjamin verzamelden zich uit de steden naar Gibea, om uit te trekken ten strijde tegen de kinderen Israels.
\par 15 En de kinderen van Benjamin werden te dien dage geteld uit de steden, zes en twintig duizend mannen, die het zwaard uittrokken, behalve dat de inwoners van Gibea geteld werden, zevenhonderd uitgelezene mannen.
\par 16 Onder al dit volk waren zevenhonderd uitgelezene mannen, welke links waren; deze allen slingerden met een steen op een haar, dat het hun niet miste.
\par 17 En de mannen van Israel werden geteld, behalve Benjamin, vierhonderd duizend mannen, die het zwaard uittrokken; deze allen waren mannen van oorlog.
\par 18 En de kinderen Israels maakten zich op, en togen opwaarts ten huize Gods, en vraagden God, en zeiden: Wie zal onder ons vooreerst optrekken ten strijde tegen de kinderen van Benjamin? En de HEERE zeide: Juda vooreerst.
\par 19 Alzo maakten zich de kinderen Israels in den morgenstond op, en legerden zich tegen Gibea.
\par 20 En de mannen van Israel togen uit ten strijde tegen Benjamin; voorts schikten de mannen Israels den strijd tegen hen bij Gibea.
\par 21 Toen togen de kinderen van Benjamin uit van Gibea, en zij vernielden ter aarde op dien dag van Israel twee en twintig duizend man.
\par 22 Doch het volk versterkte zich, te weten de mannen van Israel, en zij beschikten den strijd wederom ter plaatse, waar zij dien des vorigen daags geschikt hadden.
\par 23 En de kinderen Israels togen op, en weenden voor het aangezicht des HEEREN tot op den avond, en vraagden den HEERE zeggende: Zal ik weder genaken ten strijde tegen de kinderen van Benjamin, mijn broeder? En de HEERE zeide: Trekt tegen hem op.
\par 24 Zo naderden de kinderen Israels tot de kinderen van Benjamin, des anderen daags.
\par 25 En die van Benjamin trokken uit hun tegemoet, uit Gibea, op den tweeden dag, en velden van de kinderen Israels nog achttien duizend man neder ter aarde; die allen trokken het zwaard uit.
\par 26 Toen togen alle kinderen Israels en al het volk op, en kwamen ten huize Gods, en weenden, en bleven aldaar voor het aangezicht des HEEREN, en vastten dien dag tot op den avond; en zij offerden brandofferen en dankofferen voor het aangezicht des HEEREN.
\par 27 En de kinderen Israels vraagden den HEERE, want aldaar was de ark des verbonds van God in die dagen.
\par 28 En Pinehas, de zoon van Eleazar, den zoon van Aaron, stond voor Zijn aangezicht, in die dagen, zeggende: Zal ik nog meer uittrekken ten strijde tegen de kinderen van Benjamin, mijn broeder, of zal ik ophouden? en de HEERE zeide: Trekt op, want morgen zal Ik hem in uw hand geven.
\par 29 Toen bestelde Israel achterlagen op Gibea rondom.
\par 30 En de kinderen Israels togen op, aan den derden dag, tegen de kinderen van Benjamin; en zij schikten den strijd op Gibea, als op de andere malen.
\par 31 Toen togen de kinderen van Benjamin uit, het volk tegemoet, en werden van de stad afgetrokken; en zij begonnen te slaan van het volk, en te doorsteken, gelijk de andere malen, op de straten, waarvan de een opgaat naar het huis Gods, en de ander naar Gibea, in het veld, omtrent dertig man van Israel.
\par 32 Toen zeiden de kinderen van Benjamin: Zij zijn voor ons aangezicht geslagen, als te voren; maar de kinderen Israels zeiden: Laat ons vlieden, en hen van de stad aftrekken naar de straten.
\par 33 Toen maakten zich alle mannen van Israel op uit hun plaatsen, en schikten den strijd te Baal-thamar; ook brak Israels achterlage op uit haar plaats, na de ontbloting van Gibea.
\par 34 En tien duizend uitgelezen mannen van gans Israel kwamen van tegenover Gibea, en de strijd werd zwaar; doch zij wisten niet, dat het kwaad hen treffen zou.
\par 35 Toen sloeg de HEERE Benjamin voor Israels aangezicht; dat de kinderen Israels op dien dag van Benjamin vernielden vijf en twintig duizend en honderd mannen; die allen trokken het zwaard uit.
\par 36 En de kinderen van Benjamin zagen, dat zij geslagen waren; want de mannen van Israel gaven den Benjaminieten plaats, omdat zij vertrouwden op de achterlage, die zij tegen Gibea gesteld hadden.
\par 37 En de achterlage haastte, en brak voorwaarts naar Gibea toe; ja, de achterlage trok recht door, en sloeg de ganse stad met de scherpte des zwaards.
\par 38 En de mannen van Israel hadden een bestemden tijd met de achterlage, wanneer zij een grote verheffing van rook van de stad zouden doen opgaan.
\par 39 Zo keerden zich de mannen van Israel om in den strijd; en Benjamin had begonnen te slaan en te doorsteken van de mannen van Israel omtrent dertig man; want zij zeiden: Immers is hij zekerlijk voor ons aangezicht geslagen, als in den vorigen strijd.
\par 40 Toen begon de verheffing op te gaan van de stad, als een pilaar van rook; als nu Benjamin achter zich omzag, ziet, zo ging de brand der stad op naar den hemel.
\par 41 En de mannen van Israel keerden zich om; en de mannen van Benjamin werden verbaasd, want zij zagen, dat het kwaad hen treffen zou.
\par 42 Zo wendden zij zich voor het aangezicht der mannen van Israel naar den weg der woestijn; maar de strijd kleefde hen aan, en die uit de steden vernielden ze in het midden van hen.
\par 43 Zij omringden Benjamin, zij vervolgden hem, zij vertraden hem gemakkelijk, tot voor Gibea, tegen den opgang der zon.
\par 44 En er vielen van Benjamin achttien duizend mannen; deze allen waren strijdbare mannen.
\par 45 Toen keerden zij zich, en vloden naar de woestijn, tot den rotssteen van Rimmon; maar zij deden een nalezing onder hen op de straten, van vijf duizend man; voorts kleefden zij hen achteraan tot aan Gideom, en sloegen van hen twee duizend man.
\par 46 Alzo waren allen, die op dien dag van Benjamin vielen, vijf en twintig duizend mannen, die het zwaard uittrokken; die allen waren strijdbare mannen.
\par 47 Doch zeshonderd mannen keerden zich, en vloden naar de woestijn, tot den rotssteen van Rimmon, en bleven in den rotssteen van Rimmon, vier maanden.
\par 48 En de mannen van Israel keerden weder tot de kinderen van Benjamin, en sloegen hen met de scherpte des zwaards, die van de gehele stad tot de beesten toe, ja, al wat gevonden werd; ook zetten zij alle steden, die gevonden werden, in het vuur.

\chapter{21}

\par 1 De mannen van Israel nu hadden te Mizpa gezworen, zeggende: Niemand van ons zal zijn dochter aan de Benjaminieten ter vrouwe geven.
\par 2 Zo kwam het volk tot het huis Gods, en zij bleven daar tot op den avond, voor Gods aangezicht; en zij hieven hun stem op en weenden met groot geween.
\par 3 En zeiden: O HEERE, God van Israel! Waarom is dit geschied in Israel, dat er heden een stam van Israel gemist wordt?
\par 4 En het geschiedde des anderen daags, dat zich het volk vroeg opmaakte, en bouwde aldaar een altaar; en zij offerden brandofferen en dankofferen.
\par 5 En de kinderen Israels zeiden: Wie is er, die niet is opgekomen in de vergadering uit al de stammen van Israel tot den HEERE? Want er was een grote eed geschied aangaande dengene, die niet opkwam tot den HEERE te Mizpa, zeggende: Hij zal zekerlijk gedood worden.
\par 6 En het berouwde den kinderen Israels over Benjamin, hun broeder; en zij zeiden: Heden is een stam van Israel afgesneden.
\par 7 Wat zullen wij, belangende de vrouwen, doen aan degenen, die overgebleven zijn? Want wij hebben bij den HEERE gezworen, dat wij hun van onze dochteren geen tot vrouwen zullen geven.
\par 8 En zij zeiden: Is er iemand van de stammen van Israel, die niet opgekomen is tot den HEERE te Mizpa? En ziet, van Jabes in Gilead was niemand opgekomen in het leger, tot de gemeente.
\par 9 Want het volk werd geteld, en ziet, er was niemand van de inwoners van Jabes in Gilead.
\par 10 Toen zond de vergadering daarheen twaalf duizend mannen, van de strijdbaarste; en zij geboden hun, zeggende: Trekt heen, en slaat met de scherpte des zwaards de inwoners van Jabes in Gilead, met de vrouwen en de kinderkens.
\par 11 Doch dit is de zaak, die ge doen zult; al wat mannelijk is, en alle vrouwen, die de bijligging eens mans bekend hebben, zult gij verbannen.
\par 12 En zij vonden onder de inwoners van Jabes in Gilead vierhonderd jonge dochters, die maagden waren, die geen man bekend hadden in bijligging des mans; en zij brachten die in het leger te Silo, dewelke is in het land Kanaan.
\par 13 Toen zond de ganse vergadering heen, en sprak tot de kinderen van Benjamin, die in den rotssteen van Rimmon waren, en zij riepen hun vrede toe.
\par 14 Alzo kwamen de Benjaminieten ter zelfder tijd weder; en zij gaven hun de vrouwen, die zij in het leven behouden hadden van de vrouwen van Jabes in Gilead; maar alzo waren er nog niet genoeg voor hen.
\par 15 Toen berouwde het den volke over Benjamin, omdat de HEERE een scheur gemaakt had in de stammen van Israel.
\par 16 En de oudsten der vergadering zeiden: Wat zullen wij, belangende de vrouwen, doen aan degenen, die overgebleven zijn? Want de vrouwen zijn uit Benjamin verdelgd.
\par 17 Wijders zeiden zij: De erfenis dergenen, die ontkomen zijn, is van Benjamin, en er moet geen stam uitgedelgd worden uit Israel.
\par 18 Maar wij zullen hun geen vrouwen van onze dochteren kunnen geven; want de kinderen Israels hebben gezworen, zeggende: Vervloekt zij, die den Benjaminieten een vrouw geeft!
\par 19 Toen zeiden zij: Ziet, er is een feest des HEEREN te Silo, van jaar tot jaar, dat gehouden wordt tegen het noorden van het huis Gods, tegen den opgang der zon, aan den hogen weg, die opgaat van het huis Gods naar Sichem, en tegen het zuiden van Lebona.
\par 20 En zij geboden den kinderen van Benjamin, zeggende: Gaat heen, en loert in de wijngaarden.
\par 21 En let er op, en zie, als de dochters van Silo zullen uitgegaan zijn om met reien te dansen, zo komt gij voort uit de wijngaarden, en schaakt u, een ieder zijn huisvrouw, uit de dochteren van Silo; en gaat heen in het land van Benjamin.
\par 22 En het zal geschieden, wanneer haar vaders of haar broeders zullen komen, om voor ons te rechten, dat wij tot hen zullen zeggen: Zijt hun om onzentwil genadig, omdat wij geen huisvrouw voor een ieder van hen in dezen krijg genomen hebben; want gijlieden hebt ze hun niet gegeven, dat gij te dezer tijd schuldig zoudt zijn.
\par 23 En de kinderen van Benjamin deden alzo, en voerden naar hun getal vrouwen weg, van de reiende dochters, die zij roofden, en zij togen heen, en keerden weder tot hun erfenis, en herbouwden de steden, en woonden daarin.
\par 24 Ook togen de kinderen Israels te dier tijd van daar, een iegelijk naar zijn stam en naar zijn geslacht; alzo togen zij uit van daar, een iegelijk naar zijn erfenis.
\par 25 In die dagen was er geen koning in Israel; een iegelijk deed, wat recht was in zijn ogen.


\end{document}