\begin{document}

\title{Job}



\chapter{1}

\par 1 Er was een man in het land Uz, zijn naam was Job; en dezelve man was oprecht, en vroom, en godvrezende, en wijkende van het kwaad.
\par 2 En hem werden zeven zonen en drie dochteren geboren.
\par 3 Daartoe was zijn vee zeven duizend schapen, en drie duizend kemelen, en vijfhonderd juk ossen, en vijfhonderd ezelinnen; ook was zijn dienstvolk zeer veel; zodat deze man groter was dan al die van het oosten.
\par 4 En zijn zonen gingen, en maakten maaltijden in ieders huis op zijn dag; en zij zonden henen, en nodigden hun drie zusteren, om met hen te eten en te drinken.
\par 5 Het geschiedde dan, als de dagen der maaltijden omgegaan waren, dat Job henenzond, en hen heiligde en des morgens vroeg opstond, en brandofferen offerde naar hun aller getal; want Job zeide: Misschien hebben mijn kinderen gezondigd, en God in hun hart gezegend. Alzo deed Job al die dagen.
\par 6 Er was nu een dag, als de kinderen Gods kwamen, om zich voor den HEERE te stellen, dat de satan ook in het midden van hen kwam.
\par 7 Toen zeide de HEERE tot den satan: Van waar komt gij? En de satan antwoordde den HEERE, en zeide: Van om te trekken op de aarde, en van die te doorwandelen.
\par 8 En de HEERE zeide tot den satan: Hebt gij ook acht geslagen op Mijn knecht Job? Want niemand is op de aarde gelijk hij, een man oprecht en vroom, godvrezende en wijkende van het kwaad.
\par 9 Toen antwoordde de satan den HEERE, en zeide: Is het om niet, dat Job God vreest?
\par 10 Hebt Gij niet een betuining gemaakt voor hem, en voor zijn huis, en voor al wat hij heeft rondom? Het werk zijner handen hebt Gij gezegend, en zijn vee is in menigte uitgebroken in den lande.
\par 11 Maar toch strek nu Uw hand uit, en tast aan alles, wat hij heeft; zo hij U niet in Uw aangezicht zal zegenen?
\par 12 En de HEERE zeide tot den satan: Zie, al wat hij heeft, zij in uw hand; alleen aan hem strek uw hand niet uit. En de satan ging uit van het aangezicht des HEEREN.
\par 13 Er was nu een dag, als zijn zonen en zijn dochteren aten, en wijn dronken in het huis van hun broeder, den eerstgeborene.
\par 14 Dat een bode tot Job kwam, en zeide: De runderen waren ploegende, en de ezelinnen weidende aan hun zijden.
\par 15 Doch de Sabeers deden een inval, en namen ze, en sloegen de jongeren met de scherpte des zwaards; en ik ben maar alleen ontkomen, om het u aan te zeggen.
\par 16 Als deze nog sprak, zo kwam een ander, en zeide: Het vuur Gods viel uit den hemel, en ontstak onder de schapen en onder de jongeren, en verteerde ze; en ik ben maar alleen ontkomen, om het u aan te zeggen.
\par 17 Als deze nog sprak, zo kwam een ander, en zeide: De Chaldeen stelden drie hopen, en vielen op de kemelen aan, en namen ze, en sloegen de jongeren met de scherpte des zwaards; en ik ben maar alleen ontkomen, om het u aan te zeggen.
\par 18 Als deze nog sprak, zo kwam een ander, en zeide: Uw zonen en uw dochteren aten, en dronken wijn, in het huis van hun broeder, den eerstgeborene;
\par 19 En zie, een grote wind kwam van over de woestijn, en stiet aan de vier hoeken van het huis, en het viel op de jongelingen, dat ze stierven; en ik ben maar alleen ontkomen, om het u aan te zeggen.
\par 20 Toen stond Job op, en scheurde zijn mantel, en schoor zijn hoofd, en viel op de aarde, en boog zich neder;
\par 21 En hij zeide: Naakt ben ik uit mijner moeders buik gekomen, en naakt zal ik daarhenen wederkeren. De HEERE heeft gegeven, en de HEERE heeft genomen; de Naam des HEEREN zij geloofd!
\par 22 In dit alles zondigde Job niet, en schreef Gode niets ongerijmds toe.

\chapter{2}

\par 1 Wederom was er een dag, als de kinderen Gods kwamen, om zich voor den HEERE te stellen, dat de satan ook in het midden van hen kwam, om zich voor den HEERE te stellen.
\par 2 Toen zeide de HEERE tot den satan: Van waar komt gij? En de satan antwoordde den HEERE, en zeide: Van om te trekken op de aarde, en van die te doorwandelen.
\par 3 En de HEERE zeide tot den satan: Hebt gij ook acht geslagen op Mijn knecht Job? Want niemand is op de aarde gelijk hij, een man, oprecht en vroom, godvrezende en wijkende van het kwaad; en hij houdt nog vast aan zijn oprechtigheid, hoewel gij Mij tegen hem opgehitst hebt, om hem te verslinden zonder oorzaak.
\par 4 Toen antwoordde de satan den HEERE, en zeide: Huid voor huid, en al wat iemand heeft, zal hij geven voor zijn leven.
\par 5 Doch strek nu Uw hand uit, en tast zijn gebeente en zijn vlees aan; zo hij U niet in Uw aangezicht zal zegenen!
\par 6 En de HEERE zeide tot den satan: Zie, hij zij in uw hand, doch verschoon zijn leven.
\par 7 Toen ging de satan uit van het aangezicht des HEEREN, en sloeg Job met boze zweren, van zijn voetzool af tot zijn schedel toe.
\par 8 En hij nam zich een potscherf, om zich daarmede te schrabben, en hij zat neder in het midden der as.
\par 9 Toen zeide zijn huisvrouw tot hem: Houdt gij nog vast aan uw oprechtigheid? Zegen God, en sterf.
\par 10 Maar hij zeide tot haar: Gij spreekt als een der zottinnen spreekt; ja, zouden wij het goede van God ontvangen, en het kwade niet ontvangen? In dit alles zondigde Job met zijn lippen niet.
\par 11 Als nu de drie vrienden van Job gehoord hadden al dit kwaad, dat over hem gekomen was, kwamen zij, ieder uit zijn plaats, Elifaz, de Themaniet, en Bildad, de Suhiet, en Zofar, de Naamathiet; en zij waren het eens geworden, dat zij kwamen om hem te beklagen, en om hem te vertroosten.
\par 12 En toen zij hun ogen van verre ophieven, kenden zij hem niet, en hieven hun stem op, en weenden; daartoe scheurden zij een ieder zijn mantel, en strooiden stof op hun hoofden naar den hemel.
\par 13 Alzo zaten zij met hem op de aarde, zeven dagen en zeven nachten; en niemand sprak tot hem een woord, want zij zagen, dat de smart zeer groot was.

\chapter{3}

\par 1 Daarna opende Job zijn mond, en vervloekte zijn dag.
\par 2 Want Job antwoordde en zeide:
\par 3 De dag verga, waarin ik geboren ben, en de nacht, waarin men zeide: Een knechtje is ontvangen;
\par 4 Diezelve dag zij duisternis; dat God naar hem niet vrage van boven; en dat geen glans over hem schijne;
\par 5 Dat de duisternis en des doods schaduw hem verontreinigen; dat wolken over hem wonen; dat hem verschrikken de zwarte dampen des dags!
\par 6 Diezelve nacht, donkerheid neme hem in; dat hij zich niet verheuge onder de dagen des jaars; dat hij in het getal der maanden niet kome!
\par 7 Ziet, diezelve nacht zij eenzaam; dat geen vrolijk gezang daarin kome;
\par 8 Dat hem vervloeken de vervloekers des dags, die bereid zijn hun rouw te verwekken;
\par 9 Dat de sterren van zijn schemertijd verduisterd worden; hij wachte naar het licht, en het worde niet; en hij zie niet de oogleden des dageraads!
\par 10 Omdat hij niet toegesloten heeft de deuren mijns buiks, noch verborgen de moeite van mijn ogen.
\par 11 Waarom ben ik niet gestorven van de baarmoeder af, en heb den geest gegeven, als ik uit den buik voortkwam?
\par 12 Waarom zijn mij de knieen voorgekomen, en waartoe de borsten, opdat ik zuigen zou?
\par 13 Want nu zou ik nederliggen, en stil zijn; ik zou slapen, dan zou voor mij rust wezen;
\par 14 Met de koningen en raadsheren der aarde, die voor zich woeste plaatsen bebouwden;
\par 15 Of met de vorsten, die goud hadden, die hun huizen met zilver vervulden.
\par 16 Of als een verborgene misdracht, zou ik niet zijn; als de kinderkens, die het licht niet gezien hebben.
\par 17 Daar houden de bozen op van beroering, en daar rusten de vermoeiden van kracht;
\par 18 Daar zijn de gebondenen te zamen in rust; zij horen de stem des drijvers niet.
\par 19 De kleine en de grote is daar; en de knecht vrij van zijn heer.
\par 20 Waarom geeft Hij den ellendigen het licht, en het leven den bitterlijk bedroefden van gemoed?
\par 21 Die verlangen naar den dood, maar hij is er niet; en graven daarnaar meer dan naar verborgene schatten;
\par 22 Die blijde zijn tot opspringens toe, en zich verheugen, als zij het graf vinden;
\par 23 Aan den man, wiens weg verborgen is, en dien God overdekt heeft?
\par 24 Want voor mijn brood komt mijn zuchting; en mijn brullingen worden uitgestort als water.
\par 25 Want ik vreesde een vreze, en zij is mij aangekomen; en wat ik schroomde, is mij overkomen.
\par 26 Ik was niet gerust; en was niet stil, en rustte niet; en de beroering is gekomen.

\chapter{4}

\par 1 Toen antwoordde Elifaz, de Themaniet, en zeide:
\par 2 Zo wij een woord opnemen tegen u, zult gij verdrietig zijn? Nochtans wie zal zich van woorden kunnen onthouden?
\par 3 Zie, gij hebt velen onderwezen, en gij hebt slappe handen gesterkt;
\par 4 Uw woorden hebben den struikelende opgericht, en de krommende knieen hebt gij vastgesteld;
\par 5 Maar nu komt het aan u, en gij zijt verdrietig; het raakt tot u, en gij wordt beroerd.
\par 6 Was niet uw vreze Gods uw hoop, en de oprechtheid uwer wegen uw verwachting?
\par 7 Gedenk toch, wie is de onschuldige, die vergaan zij; en waar zijn de oprechten verdelgd?
\par 8 Maar gelijk als ik gezien heb: die ondeugd ploegen, en moeite zaaien, maaien dezelve.
\par 9 Van den adem Gods vergaan zij, en van het geblaas van Zijn neus worden zij verdaan.
\par 10 De brulling des leeuws, en de stem des fellen leeuws, en de tanden der jonge leeuwen worden verbroken.
\par 11 De oude leeuw vergaat, omdat er geen roof is, en de jongens eens oudachtigen leeuws worden verstrooid.
\par 12 Voorts is tot mij een woord heimelijk gebracht, en mijn oor heeft een weinigje daarvan gevat;
\par 13 Onder de gedachten van de gezichten des nachts, als diepe slaap valt op de mensen;
\par 14 Kwam mij schrik en beving over, en verschrikte de veelheid mijner beenderen.
\par 15 Toen ging voorbij mijn aangezicht een geest; hij deed het haar mijns vleses te berge rijzen.
\par 16 Hij stond, doch ik kende zijn gedaante niet; een beeltenis was voor mijn ogen; er was stilte, en ik hoorde een stem, zeggende:
\par 17 Zou een mens rechtvaardiger zijn dan God? Zou een man reiner zijn dan zijn Maker?
\par 18 Zie, op Zijn knechten zou Hij niet vertrouwen; hoewel Hij in Zijn engelen klaarheid gesteld heeft.
\par 19 Hoeveel te min op degenen, die lemen huizen bewonen, welker grondslag in het stof is? Zij worden verbrijzeld voor de motten.
\par 20 Van den morgen tot den avond worden zij vermorzeld; zonder dat men er acht op slaat, vergaan zij in eeuwigheid.
\par 21 Verreist niet hun uitnemendheid met hen? Zij sterven, maar niet in wijsheid.

\chapter{5}

\par 1 Roep nu, zal er iemand zijn, die u antwoorde? En tot wien van de heiligen zult gij u keren?
\par 2 Want den dwaze brengt de toornigheid om, en de ijver doodt den slechte.
\par 3 Ik heb gezien een dwaas wortelende; doch terstond vervloekte ik zijn woning.
\par 4 Verre waren zijn zonen van heil; en zij werden verbrijzeld in de poort, en er was geen verlosser.
\par 5 Wiens oogst de hongerige verteerde, dien hij ook tot uit de doornen gehaald had; de struikrover slokte hun vermogen in.
\par 6 Want uit het stof komt het verdriet niet voort, en de moeite spruit niet uit de aarde;
\par 7 Maar de mens wordt tot moeite geboren; gelijk de spranken der vurige kolen zich verheffen tot vliegen.
\par 8 Doch ik zou naar God zoeken, en tot God mijn aanspraak richten;
\par 9 Die grote dingen doet, die men niet doorzoeken kan; wonderen, die men niet tellen kan;
\par 10 Die den regen geeft op de aarde, en water zendt op de straten;
\par 11 Om de vernederden te stellen in het hoge; dat de rouwdragenden door heil verheven worden.
\par 12 Hij maakt te niet de gedachten der arglistigen; dat hun handen niet een ding uitrichten.
\par 13 Hij vangt de wijzen in hun arglistigheid; dat de raad der verdraaiden gestort wordt.
\par 14 Des daags ontmoeten zij de duisternis, en gelijk des nachts tasten zij in den middag.
\par 15 Maar Hij verlost den behoeftige van het zwaard, van hun mond, en van de hand des sterken.
\par 16 Zo is voor den arme verwachting; en de boosheid stopt haar mond toe.
\par 17 Zie, gelukzalig is de mens, denwelken God straft; daarom verwerp de kastijding des Almachtigen niet.
\par 18 Want Hij doet smart aan, en Hij verbindt; Hij doorwondt, en Zijn handen helen.
\par 19 In zes benauwdheden zal Hij u verlossen, en in de zevende zal u het kwaad niet aanroeren.
\par 20 In den honger zal Hij u verlossen van den dood, en in den oorlog van het geweld des zwaards.
\par 21 Tegen den gesel der tong zult gij verborgen wezen, en gij zult niet vrezen voor de verwoesting, als zij komt.
\par 22 Tegen de verwoesting en tegen den honger zult gij lachen, en voor het gedierte der aarde zult gij niet vrezen.
\par 23 Want met de stenen des velds zal uw verbond zijn, en het gedierte des velds zal met u bevredigd zijn.
\par 24 En gij zult bevinden, dat uw tent in vrede is; en gij zult uw woning verzorgen, en zult niet feilen.
\par 25 Ook zult gij bevinden, dat uw zaad menigvuldig wezen zal, en uw spruiten als het kruid der aarde.
\par 26 Gij zult in ouderdom ten grave komen, gelijk de korenhoop te zijner tijd opgevoerd wordt.
\par 27 Zie dit, wij hebben het doorzocht, het is alzo; hoor het, en bemerk gij het voor u.

\chapter{6}

\par 1 Maar Job antwoordde en zeide:
\par 2 Och, of mijn verdriet recht gewogen wierd, en men mijn ellende samen in een weegschaal ophief!
\par 3 Want het zou nu zwaarder zijn dan het zand der zeeen; daarom worden mijn woorden opgezwolgen.
\par 4 Want de pijlen des Almachtigen zijn in mij, welker vurig venijn mijn geest uitdrinkt; de verschrikkingen Gods rusten zich tegen mij.
\par 5 Rochelt ook de woudezel bij het jonge gras? Loeit de os bij zijn voeder?
\par 6 Wordt ook het onsmakelijke gegeten zonder zout? Is er smaak in het witte des dooiers?
\par 7 Mijn ziel weigert uw woorden aan te roeren; die zijn als mijn laffe spijze.
\par 8 Och, of mijn begeerte kwame, en dat God mijn verwachting gave;
\par 9 En dat het Gode beliefde, dat Hij mij verbrijzelde, Zijn hand losliet, en een einde met mij maakte!
\par 10 Dat zou nog mijn troost zijn, en zou mij verkwikken in den weedom, zo Hij niet spaarde; want ik heb de redenen des Heiligen niet verborgen gehouden.
\par 11 Wat is mijn kracht, dat ik hopen zou? Of welk is mijn einde, dat ik mijn leven verlengen zou?
\par 12 Is mijn kracht stenen kracht? Is mijn vlees staal?
\par 13 Is dan mijn hulp niet in mij, en is de wijsheid uit mij verdreven?
\par 14 Aan hem, die versmolten is, zou van zijn vriend weldadigheid geschieden; of hij zou de vreze des Almachtigen verlaten.
\par 15 Mijn broeders hebben trouwelooslijk gehandeld als een beek; als de storting der beken gaan zij door;
\par 16 Die verdonkerd zijn van het ijs, en in dewelke de sneeuw zich verbergt.
\par 17 Ten tijde, als zij van hitte vervlieten, worden zij uitgedelgd; als zij warm worden, verdwijnen zij uit haar plaats.
\par 18 De gangen haars wegs wenden zich ter zijde af; zij lopen op in het woeste, en vergaan.
\par 19 De reizigers van Thema zien ze, de wandelaars van Scheba wachten op haar.
\par 20 Zij worden beschaamd, omdat elkeen vertrouwde; als zij daartoe komen, zo worden zij schaamrood.
\par 21 Voorwaar, alzo zijt gijlieden mij nu niets geworden; gij hebt gezien de ontzetting, en gij hebt gevreesd.
\par 22 Heb ik gezegd: Brengt mij, en geeft geschenken voor mij van uw vermogen?
\par 23 Of bevrijdt mij van de hand des verdrukkers, en verlost mij van de hand der tirannen?
\par 24 Leert mij, en ik zal zwijgen, en geeft mij te verstaan, waarin ik gedwaald heb.
\par 25 O, hoe krachtig zijn de rechte redenen! Maar wat bestraft het bestraffen, dat van ulieden is?
\par 26 Zult gij, om te bestraffen, woorden bedenken, en zullen de redenen des mismoedigen voor wind zijn?
\par 27 Ook werpt gij u op een wees; en gij graaft tegen uw vriend.
\par 28 Maar nu, belieft het u, wendt u tot mij, en het zal voor ulieder aangezicht zijn, of ik liege.
\par 29 Keert toch weder, laat er geen onrecht wezen, ja, keert weder; nog zal mijn gerechtigheid daarin zijn.
\par 30 Zou onrecht op mijn tong wezen? Zou mijn gehemelte niet de ellenden te verstaan geven?

\chapter{7}

\par 1 Heeft niet de mens een strijd op de aarde, en zijn zijn dagen niet als de dagen des dagloners?
\par 2 Gelijk de dienstknecht hijgt naar de schaduw, en gelijk de dagloner verwacht zijn werkloon;
\par 3 Alzo zijn mij maanden der ijdelheid ten erve geworden, en nachten der moeite zijn mij voorbereid.
\par 4 Als ik te slapen lig, dan zeg ik: Wanneer zal ik opstaan, en Hij den avond afgemeten hebben? En ik word zat van woelingen tot aan den schemertijd.
\par 5 Mijn vlees is met het gewormte en met het gruis des stofs bekleed; mijn huid is gekliefd en verachtelijk geworden.
\par 6 Mijn dagen zijn lichter geweest dan een weversspoel, en zijn vergaan zonder verwachting.
\par 7 Gedenk, dat mijn leven een wind is; mijn oog zal niet wederkomen, om het goede te zien.
\par 8 Het oog desgenen, die mij nu ziet, zal mij niet zien; uw ogen zullen op mij zijn; maar ik zal niet meer zijn.
\par 9 Een wolk vergaat en vaart henen; alzo die in het graf daalt, zal niet weder opkomen.
\par 10 Hij zal niet meer wederkeren tot zijn huis, en zijn plaats zal hem niet meer kennen.
\par 11 Zo zal ik ook mijn mond niet wederhouden, ik zal spreken in benauwdheid mijns geestes; ik zal klagen in bitterheid mijner ziel.
\par 12 Ben ik dan een zee, of walvis, dat Gij om mij wachten zet?
\par 13 Wanneer ik zeg: Mijn bedstede zal mij vertroosten, mijn leger zal van mijn klacht wat wegnemen;
\par 14 Dan ontzet Gij mij met dromen, en door gezichten verschrikt Gij mij;
\par 15 Zodat mijn ziel de verworging kiest; den dood meer dan mijn beenderen.
\par 16 Ik versmaad ze, ik zal toch in der eeuwigheid niet leven; houd op van mij, want mijn dagen zijn ijdelheid.
\par 17 Wat is de mens, dat Gij hem groot acht, en dat Gij Uw hart op hem zet?
\par 18 En dat Gij hem bezoekt in elken morgenstond; dat Gij hem in elken ogenblik beproeft?
\par 19 Hoe lang keert Gij U niet af van mij, en laat niet van mij af, totdat ik mijn speeksel inzwelge?
\par 20 Heb ik gezondigd, wat zal ik U doen, o Mensenhoeder? Waarom hebt Gij mij U tot een tegenloop gesteld, dat ik mijzelven tot een last zij?
\par 21 En waarom vergeeft Gij niet mijn overtreding, en doet mijn ongerechtigheid niet weg? Want nu zal ik in het stof liggen; en Gij zult mij vroeg zoeken, maar ik zal niet zijn.

\chapter{8}

\par 1 Toen antwoordde Bildad, de Suhiet, en zeide:
\par 2 Hoe lang zult gij deze dingen spreken, en de redenen uws monds een geweldige wind zijn?
\par 3 Zou dan God het recht verkeren, en zou de Almachtige de gerechtigheid verkeren?
\par 4 Indien uw kinderen gezondigd hebben tegen Hem, Hij heeft hen ook in de hand hunner overtreding geworpen.
\par 5 Maar indien gij naar God vroeg zoekt, en tot den Almachtige om genade bidt;
\par 6 Zo gij zuiver en recht zijt, gewisselijk zal Hij nu opwaken, om uwentwil, en Hij zal de woning uwer gerechtigheid volmaken.
\par 7 Uw beginsel zal wel gering zijn; maar uw laatste zal zeer vermeerderd worden.
\par 8 Want vraag toch naar het vorige geslacht, en bereid u tot de onderzoeking hunner vaderen.
\par 9 Want wij zijn van gisteren en weten niet; dewijl onze dagen op de aarde een schaduw zijn.
\par 10 Zullen die u niet leren, tot u spreken, en uit hun hart redenen voortbrengen?
\par 11 Verheft zich de bieze zonder slijk? Groeit het rietgras zonder water?
\par 12 Als het nog in zijn groenigheid is, hoewel het niet afgesneden wordt, nochtans verdort het voor alle gras.
\par 13 Alzo zijn de paden van allen, die God vergeten; en de verwachting des huichelaars zal vergaan.
\par 14 Van denwelke zijn hoop walgen zal; en zijn vertrouwen zal zijn een huis der spinnekop.
\par 15 Hij zal op zijn huis leunen, maar het zal niet bestaan; hij zal zich daaraan vasthouden, maar het zal niet staande blijven.
\par 16 Hij is sappig voor de zon, en zijn scheuten gaan over zijn hof uit.
\par 17 Zijn wortelen worden bij de springader ingevlochten; hij ziet een stenige plaats.
\par 18 Maar als God hem verslindt uit zijn plaats, zo zal zij hem loochenen, zeggende: Ik heb u niet gezien.
\par 19 Zie, dat is de vreugde zijns wegs; en uit het stof zullen anderen voortspruiten.
\par 20 Zie, God zal den oprechte niet verwerpen; Hij vat ook de boosdoeners niet bij de hand;
\par 21 Totdat Hij uw mond met gelach vervulle, en uw lippen met gejuich.
\par 22 Uw haters zullen met schaamte bekleed worden; en de tent der goddelozen zal niet meer zijn.

\chapter{9}

\par 1 Maar Job antwoordde en zeide:
\par 2 Waarlijk, ik weet, dat het zo is; want hoe zou de mens rechtvaardig zijn bij God?
\par 3 Zo Hij lust heeft, om met hem te twisten, niet een uit duizend zal hij Hem beantwoorden.
\par 4 Hij is wijs van hart, en sterk van kracht; wie heeft zich tegen Hem verhard, en vrede gehad?
\par 5 Die de bergen verzet, dat zij het niet gewaar worden, Die ze omkeert in Zijn toorn;
\par 6 Die de aarde beweegt uit haar plaats, dat haar pilaren schudden;
\par 7 Die de zon gebiedt, en zij gaat niet op; en verzegelt de sterren;
\par 8 Die alleen de hemelen uitbreidt, en treedt op de hoogten der zee;
\par 9 Die den Wagen maakt, den Orion, en het Zevengesternte, en de binnenkameren van het Zuiden;
\par 10 Die grote dingen doet, die men niet doorzoeken kan; en wonderen, die men niet tellen kan.
\par 11 Zie, Hij zal voor mij henengaan, en ik zal Hem niet zien; en Hij zal voorbijgaan, en ik zal Hem niet merken.
\par 12 Zie, Hij zal roven, wie zal het Hem doen wedergeven? Wie zal tot Hem zeggen: Wat doet Gij?
\par 13 God zal Zijn toorn niet afkeren; onder Hem worden gebogen de hovaardige helpers.
\par 14 Hoeveel te min zal ik Hem antwoorden, en mijn woorden uitkiezen tegen Hem?
\par 15 Denwelken ik, zo ik rechtvaardig ware, niet zou antwoorden; mijn Rechter zal ik om genade bidden.
\par 16 Indien ik roep, en Hij mij antwoordt; ik zal niet geloven, dat Hij mijn stem ter ore genomen heeft.
\par 17 Want Hij vermorzelt mij door een onweder, en vermenigvuldigt mijn wonden zonder oorzaak.
\par 18 Hij laat mij niet toe mijn adem te verhalen; maar Hij verzadigt mij met bitterheden.
\par 19 Zo het aan de kracht komt, zie, Hij is sterk; en zo het aan het recht komt, wie zal mij dagvaarden?
\par 20 Zo ik mij rechtvaardig, mijn mond zal mij verdoemen; ben ik oprecht, Hij zal mij toch verkeerd verklaren.
\par 21 Ben ik oprecht, zo acht ik toch mijn ziel niet; ik versmaad mijn leven.
\par 22 Dat is een ding, daarom zeg ik: Den oprechte en den goddeloze verdoet Hij.
\par 23 Als de gesel haastelijk doodt, bespot Hij de verzoeking der onschuldigen.
\par 24 De aarde wordt gegeven in de hand des goddelozen; Hij overdekt het aangezicht harer rechteren; zo niet, wie is Hij dan?
\par 25 En mijn dagen zijn lichter geweest dan een loper; zij zijn weggevloden, zij hebben het goede niet gezien.
\par 26 Zij zijn voorbijgevaren met jachtschepen; gelijk een arend naar het aas toevliegt.
\par 27 Indien mijn zeggen is: Ik zal mijn klacht vergeten, en ik zal mijn gebaar laten varen, en mij verkwikken;
\par 28 Zo schroom ik voor al mijn smarten; ik weet, dat Gij mij niet onschuldig zult houden.
\par 29 Ik zal toch goddeloos zijn; waarom dan zal ik ijdellijk arbeiden?
\par 30 Indien ik mij wasse met sneeuwwater, en mijn handen zuivere met zeep;
\par 31 Dan zult Gij mij in de gracht induiken, en mijn klederen zullen van mij gruwen.
\par 32 Want Hij is niet een man, als ik, dien ik antwoorden zou, zo wij te zamen in het gericht kwamen.
\par 33 Er is geen scheidsman tussen ons, die zijn hand op ons beiden leggen mocht.
\par 34 Dat Hij van op mij Zijn roede wegdoe, en dat Zijn verschrikking mij niet verbaasd make;
\par 35 Zo zal ik spreken, en Hem niet vrezen; want zodanig ben ik niet bij mij.

\chapter{10}

\par 1 Mijn ziel is verdrietig over mijn leven; ik zal mijn klacht op mij laten; ik zal spreken in bitterheid mijner ziel.
\par 2 Ik zal tot God zeggen: Verdoem mij niet; doe mij weten, waarover Gij met mij twist.
\par 3 Is het U goed, dat Gij verdrukt, dat Gij verwerpt den arbeid Uwer handen, en over den raad der goddelozen schijnsel geeft?
\par 4 Hebt Gij vleselijke ogen, ziet Gij, gelijk een mens ziet?
\par 5 Zijn Uw dagen als de dagen van een mens? Zijn Uw jaren als de dagen eens mans?
\par 6 Dat Gij onderzoekt naar mijn ongerechtigheid, en naar mijn zonde verneemt?
\par 7 Het is Uw wetenschap, dat ik niet goddeloos ben; nochtans is er niemand, die uit Uw hand verlosse.
\par 8 Uw handen doen mij smart aan, hoewel zij mij gemaakt hebben, te zamen rondom mij zijn zij, en Gij verslindt mij.
\par 9 Gedenk toch, dat Gij mij als leem bereid hebt, en mij tot stof zult doen wederkeren.
\par 10 Hebt Gij mij niet als melk gegoten, en mij als een kaas doen runnen?
\par 11 Met vel en vlees hebt Gij mij bekleed; met beenderen ook en zenuwen hebt Gij mij samengevlochten;
\par 12 Benevens het leven hebt Gij weldadigheid aan mij gedaan, en Uw opzicht heeft mijn geest bewaard.
\par 13 Maar deze dingen hebt Gij verborgen in Uw hart; ik weet, dat dit bij U geweest is.
\par 14 Indien ik zondig, zo zult Gij mij waarnemen, en van mijn misdaad zult Gij mij niet onschuldig houden.
\par 15 Zo ik goddeloos ben, wee mij! En ben ik rechtvaardig, ik zal mijn hoofd niet opheffen; ik ben zat van schande, maar aanzie mijn ellende.
\par 16 Want zij verheft zich; gelijk een felle leeuw jaagt Gij mij; Gij keert weder en stelt U wonderlijk tegen mij.
\par 17 Gij vernieuwt Uw getuigen tegenover mij, en vermenigvuldigt Uw toorn tegen mij; verwisselingen, ja, een heirleger, zijn tegen mij.
\par 18 En waarom hebt Gij mij uit de baarmoeder voortgebracht? Och, dat ik den geest gegeven had, en geen oog mij gezien had!
\par 19 Ik zou zijn, alsof ik niet geweest ware; van moeders buik zou ik tot het graf gebracht zijn geweest.
\par 20 Zijn mijn dagen niet weinig? Houd op, zet van mij af, dat ik mij een weinig verkwikke;
\par 21 Eer ik henenga (en niet wederkom) in een land der duisternis en der schaduwe des doods;
\par 22 Een stikdonker land, als de duisternis zelve, de schaduwe des doods, en zonder ordeningen, en het geeft schijnsel als de duisternis.

\chapter{11}

\par 1 Toen antwoordde Zofar, de Naamathiet, en zeide:
\par 2 Zou de veelheid der woorden niet beantwoord worden, en zou een klapachtig man recht hebben?
\par 3 Zouden uw leugenen de lieden doen zwijgen, en zoudt gij spotten, en niemand u beschamen?
\par 4 Want gij hebt gezegd: Mijn leer is zuiver, en ik ben rein in uw ogen.
\par 5 Maar gewisselijk, och, of God sprak, en Zijn lippen tegen u opende;
\par 6 En u bekend maakte de verborgenheden der wijsheid, omdat zij dubbel zijn in wezen! Daarom weet, dat God voor u vergeet van uw ongerechtigheid.
\par 7 Zult gij de onderzoeking Gods vinden? Zult gij tot de volmaaktheid toe den Almachtige vinden?
\par 8 Zij is als de hoogten der hemelen, wat kunt gij doen? Dieper dan de hel, wat kunt gij weten?
\par 9 Langer dan de aarde is haar maat, en breder dan de zee.
\par 10 Indien Hij voorbijgaat, opdat Hij overlevere of vergadere, wie zal dan Hem afkeren?
\par 11 Want Hij kent de ijdele lieden, en Hij ziet de ondeugd; zou Hij dan niet aanmerken?
\par 12 Dan zal een verstandeloos man kloekzinnig worden; hoewel de mens als het veulen eens woudezels geboren is.
\par 13 Indien gij uw hart bereid hebt, zo breid uw handen tot Hem uit.
\par 14 Indien er ondeugd in uw hand is, doe die verre weg; en laat het onrecht in uw tenten niet wonen.
\par 15 Want dan zult gij uw aangezicht opheffen uit de gebreken, en zult vast wezen, en niet vrezen.
\par 16 Want gij zult de moeite vergeten, en harer gedenken als der wateren, die voorbijgegaan zijn.
\par 17 Ja, uw tijd zal klaarder dan de middag oprijzen; gij zult uitvliegen, als de morgenstond zult gij zijn.
\par 18 En gij zult vertrouwen, omdat er verwachting zal zijn; en gij zult graven, gerustelijk zult gij slapen;
\par 19 En gij zult nederliggen, en niemand zal u verschrikken; en velen zullen uw aangezicht smeken.
\par 20 Maar de ogen der goddelozen zullen bezwijken, en de toevlucht zal van hen vergaan; en hun verwachting zal zijn de uitblazing der ziel.

\chapter{12}

\par 1 Maar Job antwoordde en zeide:
\par 2 Trouwens, omdat gijlieden het volk zijt, zo zal de wijsheid met ulieden sterven!
\par 3 Ik heb ook een hart even als gijlieden, ik zwicht niet voor u; en bij wien zijn niet dergelijke dingen?
\par 4 Ik ben het, die zijn vriend een spot is, maar roepende tot God, Die hem verhoort; de rechtvaardige en oprechte is een spot.
\par 5 Hij is een verachte fakkel, naar de mening desgenen, die gerust is; hij is gereed met den voet te struikelen.
\par 6 De tenten der verwoesters hebben rust, en die God tergen, hebben verzekerdheden, om hetgene God met Zijn hand toebrengt.
\par 7 En waarlijk, vraag toch de beesten, en elkeen van die zal het u leren; en het gevogelte des hemels, dat zal het u te kennen geven.
\par 8 Of spreek tot de aarde, en zij zal het u leren; ook zullen het u de vissen der zee vertellen.
\par 9 Wie weet niet uit alle deze, dat de hand des HEEREN dit doet?
\par 10 In Wiens hand de ziel is van al wat leeft, en de geest van alle vlees des mensen.
\par 11 Zal niet het oor de woorden proeven, gelijk het gehemelte voor zich de spijze smaakt?
\par 12 In de stokouden is de wijsheid, en in de langheid der dagen het verstand.
\par 13 Bij Hem is wijsheid en macht; Hij heeft raad en verstand.
\par 14 Ziet, Hij breekt af, en het zal niet herbouwd worden; Hij besluit iemand, en er zal niet opengedaan worden.
\par 15 Ziet, Hij houdt de wateren op, en zij drogen uit; ook laat Hij ze uit, en zij keren de aarde om.
\par 16 Bij Hem is kracht en wijsheid; Zijns is de dwalende, en die doet dwalen.
\par 17 Hij voert de raadsheren beroofd weg, en de rechters maakt Hij uitzinnig,
\par 18 Den band der koningen maakt Hij los, en Hij bindt den gordel aan hun lenden.
\par 19 Hij voert de oversten beroofd weg, en de machtigen keert Hij om.
\par 20 Hij beneemt den getrouwen de spraak, en der ouden oordeel neemt Hij weg.
\par 21 Hij giet verachting over de prinsen uit, en Hij verslapt den riem der geweldigen.
\par 22 Hij openbaart de diepten uit de duisternis, en des doods schaduwe brengt Hij voort in het licht.
\par 23 Hij vermenigvuldigt de volken, en verderft ze; Hij breidt de volken uit, en leidt ze.
\par 24 Hij neemt het hart van de hoofden des volks der aarde weg, en doet hen dwalen in het woeste, waar geen weg is.
\par 25 Zij tasten in de duisternis, waar geen licht is; en Hij doet hen dwalen, als een dronkaard.

\chapter{13}

\par 1 Ziet, dat alles heeft mijn oog gezien, mijn oor gehoord en verstaan.
\par 2 Gelijk gijlieden het weet, weet ik het ook; ik zwicht niet voor u.
\par 3 Maar ik zal tot den Almachtige spreken, en ben belust mij te verdedigen voor God.
\par 4 Want gewisselijk, gij zijt leugenstoffeerders; gij allen zijt nietige medicijnmeesters.
\par 5 Och, of gij gans stilzweegt! Dat zou ulieden voor wijsheid wezen.
\par 6 Hoort toch mijn verdediging, en merkt op de twistingen mijner lippen.
\par 7 Zult gij voor God onrecht spreken, en zult gij voor Hem bedriegerij spreken?
\par 8 Zult gij Zijn aangezicht aannemen? Zult gij voor God twisten?
\par 9 Zal het goed zijn, als Hij u zal onderzoeken? Zult gij met Hem spotten, gelijk men met een mens spot?
\par 10 Hij zal u gewisselijk bestraffen, zo gij in het verborgene het aangezicht aanneemt.
\par 11 Zal u niet Zijn hoogheid verschrikken, en Zijn vreze over u vallen?
\par 12 Uw gedachtenissen zijn gelijk as, uw hoogten als hoogten van leem.
\par 13 Houdt stil van mij, opdat ik spreke, en er ga over mij, wat het zij.
\par 14 Waarom zou ik mijn vlees in mijn tanden nemen, en mijn ziel in mijn hand stellen?
\par 15 Ziet, zo Hij mij doodde, zou ik niet hopen? Evenwel zal ik mijn wegen voor Zijn aangezicht verdedigen.
\par 16 Ook zal Hij mij tot zaligheid zijn; maar een huichelaar zal voor Zijn aangezicht niet komen.
\par 17 Hoort naarstiglijk mijn rede, en mijn aanwijzing met uw oren.
\par 18 Ziet nu, ik heb het recht ordentelijk gesteld; ik weet, dat ik rechtvaardig zal verklaard worden.
\par 19 Wie is hij, die met mij twist? Wanneer ik nu zweeg, zo zou ik den geest geven.
\par 20 Alleenlijk doe twee dingen niet met mij; dan zal ik mij van Uw aangezicht niet verbergen.
\par 21 Doe Uw hand verre van op mij, en Uw verschrikking make mij niet verbaasd.
\par 22 Roep dan, en ik zal antwoorden; of ik zal spreken, en geef mij antwoord.
\par 23 Hoeveel misdaden en zonden heb ik? Maak mijn overtreding en mijn zonden mij bekend.
\par 24 Waarom verbergt Gij Uw aangezicht, en houdt mij voor Uw vijand?
\par 25 Zult Gij een gedreven blad verbrijzelen, en zult Gij een drogen stoppel vervolgen?
\par 26 Want Gij schrijft tegen mij bittere dingen; en Gij doet mij erven de misdaden mijner jonkheid.
\par 27 Gij legt ook mijn voeten in den stok, en neemt waar al mijn paden; Gij drukt U in de wortelen mijner voeten,
\par 28 En hij veroudert als een verrotting, als een kleed, dat de mot opeet.

\chapter{14}

\par 1 De mens, van een vrouw geboren, is kort van dagen, en zat van onrust.
\par 2 Hij komt voort als een bloem, en wordt afgesneden; ook vlucht hij als een schaduw, en bestaat niet.
\par 3 Nog doet Gij Uw ogen over zulk een open; en Gij betrekt mij in het gericht met U.
\par 4 Wie zal een reine geven uit den onreine? Niet een.
\par 5 Dewijl zijn dagen bestemd zijn, het getal zijner maanden bij U is, en Gij zijn bepalingen gemaakt hebt, die hij niet overgaan zal;
\par 6 Wend U van hem af, dat hij rust hebbe, totdat hij als een dagloner aan zijn dag een welgevallen hebbe.
\par 7 Want voor een boom, als hij afgehouwen wordt, is er verwachting, dat hij zich nog zal veranderen, en zijn scheut niet zal ophouden.
\par 8 Indien zijn wortel in de aarde veroudert, en zijn stam in het stof versterft;
\par 9 Hij zal van den reuk der wateren weder uitspruiten, en zal een tak maken, gelijk een plant.
\par 10 Maar een man sterft, als hij verzwakt is, en de mens geeft den geest, waar is hij dan?
\par 11 De wateren verlopen uit een meer, en een rivier droogt uit en verdort;
\par 12 Alzo ligt de mens neder, en staat niet op; totdat de hemelen niet meer zijn, zullen zij niet opwaken, noch uit hun slaap opgewekt worden.
\par 13 Och, of Gij mij in het graf verstaakt, mij verborgt, totdat Uw toorn zich afkeerde; dat Gij mij een bepaling steldet, en mijner gedachtig waart!
\par 14 Als een man gestorven is, zal hij weder leven? Ik zou al de dagen mijns strijds hopen, totdat mijn verandering komen zou.
\par 15 Dat Gij zoudt roepen, en ik U zou antwoorden, dat Gij tot het werk Uwer handen zoudt begerig zijn.
\par 16 Maar nu telt Gij mijn treden; Gij bewaart mij niet om mijner zonden wil.
\par 17 Mijn overtreding is in een bundeltje verzegeld, en Gij pakt mijn ongerechtigheid opeen.
\par 18 En voorwaar, een berg vallende vergaat, en een rots wordt versteld uit haar plaats;
\par 19 De wateren vermalen de stenen, het stof der aarde overstelpt het gewas, dat van zelf daaruit voortkomt; alzo verderft Gij de verwachting des mensen.
\par 20 Gij overweldigt hem in eeuwigheid, en hij gaat heen; veranderende zijn gelaat, zo zendt Gij hem weg.
\par 21 Zijn kinderen komen tot eer, en hij weet het niet; of zij worden klein, en hij let niet op hen.
\par 22 Maar zijn vlees, nog aan hem zijnde, heeft smart; en zijn ziel, in hem zijnde, heeft rouw.

\chapter{15}

\par 1 Toen antwoordde Elifaz, de Themaniet, en zeide:
\par 2 Zal een wijs man winderige wetenschap voor antwoord geven, en zal hij zijn buik vullen met oostenwind?
\par 3 Bestraffende door woorden, die niet baten, en door redenen, met dewelke hij geen voordeel doet?
\par 4 Ja, gij vernietigt de vreze, en neemt het gebed voor het aangezicht Gods weg.
\par 5 Want uw mond leert uw ongerechtigheid, en gij hebt de tong der arglistigen verkoren.
\par 6 Uw mond verdoemt u, en niet ik; en uw lippen getuigen tegen u.
\par 7 Zijt gij de eerste een mens geboren? Of zijt gij voor de heuvelen voortgebracht?
\par 8 Hebt gij den verborgen raad Gods gehoord, en hebt gij de wijsheid naar u getrokken?
\par 9 Wat weet gij, dat wij niet weten? Wat verstaat gij, dat bij ons niet is?
\par 10 Onder ons is ook een grijze, ja, een stokoude, meerder van dagen dan uw vader.
\par 11 Zijn de vertroostingen Gods u te klein, en schuilt er enige zaak bij u?
\par 12 Waarom rukt uw hart u weg, en waarom wenken uw ogen?
\par 13 Dat gij uw geest keert tegen God, en zulke redenen uit uw mond laat uitgaan.
\par 14 Wat is de mens, dat hij zuiver zou zijn, en die geboren is van een vrouw, dat hij rechtvaardig zou zijn?
\par 15 Zie, op Zijn heiligen zou Hij niet vertrouwen, en de hemelen zijn niet zuiver in Zijn ogen.
\par 16 Hoeveel te meer is een man gruwelijk en stinkende, die het onrecht indrinkt als water?
\par 17 Ik zal u wijzen, hoor mij aan, en hetgeen ik gezien heb, dat zal ik vertellen;
\par 18 Hetwelk de wijzen verkondigd hebben, en men voor hun vaderen niet verborgen heeft;
\par 19 Denwelken alleen het land gegeven was, en door welker midden niemand vreemds doorging.
\par 20 Te allen dage doet de goddeloze zichzelven weedom aan; en weinige jaren in getal zijn voor den tiran weggelegd.
\par 21 Het geluid der verschrikkingen is in zijn oren; in den vrede zelven komt de verwoester hem over.
\par 22 Hij gelooft niet uit de duisternis weder te keren, maar dat hij beloerd wordt ten zwaarde.
\par 23 Hij zwerft heen en weder om brood, waar het zijn mag; hij weet, dat bij zijn hand gereed is de dag der duisternis.
\par 24 Angst en benauwdheid verschrikken hem; zij overweldigt hem, gelijk een koning, bereid ten strijde.
\par 25 Want hij strekt tegen God zijn hand uit, en tegen den Almachtige stelt hij zich geweldiglijk aan.
\par 26 Hij loopt tegen Hem aan met den hals, met zijn dikke, hoog verhevene schilden;
\par 27 Omdat hij zijn aangezicht met zijn vet bedekt heeft, en rimpelen gemaakt om de weekdarmen;
\par 28 En heeft bewoond verdelgde steden, en huizen, die men niet bewoonde, die gereed waren tot steen hopen te worden.
\par 29 Hij zal niet rijk worden, en zijn vermogen zal niet bestaan; en hun volmaaktheid zal zich niet uitbreiden op de aarde.
\par 30 Hij zal van de duisternis niet ontwijken, de vlam zal zijn scheut verdrogen; hij zal wijken door het geblaas zijns monds.
\par 31 Hij betrouwe niet op ijdelheid, waardoor hij verleid wordt; want ijdelheid zal zijn vergelding wezen.
\par 32 Als zijn dag nog niet is, zal hij vervuld worden; want zijn tak zal niet groenen.
\par 33 Men zal zijn onrijpe druiven afrukken, als van een wijnstok, en zijn bloeisel afwerpen, als van een olijfboom.
\par 34 Want de vergadering der huichelaren wordt eenzaam, en het vuur verteert de tenten der geschenken.
\par 35 Zijn ontvangen moeite, en baren ijdelheid, en hun buik richt bedrog aan.

\chapter{16}

\par 1 Maar Job antwoordde en zeide:
\par 2 Ik heb vele dergelijke dingen gehoord; gij allen zijt moeilijke vertroosters.
\par 3 Zal er een einde zijn aan de winderige woorden? Of wat stijft u, dat gij alzo antwoordt?
\par 4 Zou ik ook, als gijlieden, spreken, indien uw ziel ware in mijner ziele plaats? Zou ik woorden tegen u samenhopen, en zou ik over u met mijn hoofd schudden?
\par 5 Ik zou u versterken met mijn mond, en de beweging mijner lippen zou zich inhouden.
\par 6 Zo ik spreek, mijn smart wordt niet ingehouden; en houd ik op, wat gaat er van mij weg?
\par 7 Gewisselijk, Hij heeft mij nu vermoeid; Gij hebt mijn ganse vergadering verwoest.
\par 8 Dat Gij mij rimpelachtig gemaakt hebt, is tot een getuige; en mijn magerheid staat tegen mij op, zij getuigt in mijn aangezicht.
\par 9 Zijn toorn verscheurt, en Hij haat mij; Hij knerst over mij met Zijn tanden; mijn wederpartijder scherpt zijn ogen tegen mij.
\par 10 Zij gapen met hun mond tegen mij; zij slaan met smaadheid op mijn kinnebakken; zij vervullen zich te zamen aan mij.
\par 11 God heeft mij den verkeerde overgegeven, en heeft mij afgewend in de handen der goddelozen.
\par 12 Ik had rust, maar Hij heeft mij verbroken, en bij mijn nek gegrepen, en mij verpletterd; en Hij heeft mij Zich tot een doelwit opgericht.
\par 13 Zijn schutters hebben mij omringd; Hij heeft mijn nieren doorspleten, en niet gespaard; Hij heeft mijn gal op de aarde uitgegoten.
\par 14 Hij heeft mij gebroken met breuk op breuk; Hij is tegen mij aangelopen als een geweldige.
\par 15 Ik heb een zak over mijn huid genaaid; ik heb mijn hoorn in het stof gedaan.
\par 16 Mijn aangezicht is gans bemodderd van wenen, en over mijn oogleden is des doods schaduw.
\par 17 Daar toch geen wrevel in mijn handen is, en mijn gebed zuiver is.
\par 18 O, aarde! bedek mijn bloed niet; en voor mijn geroep zij geen plaats.
\par 19 Ook nu, zie, in den hemel is mijn Getuige, en mijn Getuige in de hoogten.
\par 20 Mijn vrienden zijn mijn bespotters; doch mijn oog druipt tot God.
\par 21 Och, mocht men rechten voor een man met God, gelijk een kind des mensen voor zijn vriend.
\par 22 Want weinige jaren in getal zullen er nog aankomen, en ik zal het pad henengaan, waardoor ik niet zal wederkeren.

\chapter{17}

\par 1 Mijn geest is verdorven, mijn dagen worden uitgeblust, de graven zijn voor mij.
\par 2 Zijn niet bespotters bij mij, en overnacht niet mijn oog in hunlieder verbittering?
\par 3 Zet toch bij, stel mij een borg bij U; wie zal hij zijn? Dat in mijn hand geklapt worde.
\par 4 Want hun hart hebt Gij van kloek verstand verborgen; daarom zult Gij hen niet verhogen.
\par 5 Die met vleiing den vrienden wat aanzegt, ook zijner kinderen ogen zullen versmachten.
\par 6 Doch Hij heeft mij tot een spreekwoord der volken gesteld; zodat ik een trommelslag ben voor ieders aangezicht.
\par 7 Daarom is mijn oog door verdriet verdonkerd, en al mijn ledematen zijn gelijk een schaduw.
\par 8 De oprechten zullen hierover verbaasd zijn, en de onschuldige zal zich tegen den huichelaar opmaken;
\par 9 En de rechtvaardige zal zijn weg vasthouden, en die rein van handen is, zal in sterkte toenemen.
\par 10 Maar toch gij allen, keert weder, en komt nu; want ik vind onder u geen wijze.
\par 11 Mijn dagen zijn voorbijgegaan; uitgerukt zijn mijn gedachten, de bezittingen mijns harten.
\par 12 Den nacht verstellen zij in den dag; het licht is nabij den ondergang vanwege de duisternis.
\par 13 Zo ik wacht, het graf zal mijn huis wezen; in de duisternis zal ik mijn bed spreiden.
\par 14 Tot de groeve roep ik: Gij zijt mijn vader! Tot het gewormte: Mijn moeder, en mijn zuster!
\par 15 Waar zou dan nu mijn verwachting wezen? Ja, mijn verwachting, wie zal ze aanschouwen?
\par 16 Zij zullen ondervaren met de handbomen des grafs, als er rust te zamen in het stof wezen zal.

\chapter{18}

\par 1 Toen antwoordde Bildad, de Suhiet, en zeide:
\par 2 Hoe lang is het, dat gijlieden een einde van woorden zult maken? Merkt op, en daarna zullen wij spreken.
\par 3 Waarom worden wij geacht als beesten, en zijn onrein in ulieder ogen?
\par 4 O gij, die zijn ziel verscheurt door zijn toorn! Zal om uwentwil de aarde verlaten worden, en zal een rots versteld worden uit haar plaats?
\par 5 Ja, het licht der goddelozen zal uitgeblust worden, en de vonk zijns vuurs zal niet glinsteren.
\par 6 Het licht zal verduisteren in zijn tent, en zijn lamp zal over hem uitgeblust worden.
\par 7 De treden zijner macht zullen benauwd worden, en zijn raad zal hem nederwerpen.
\par 8 Want met zijn voeten zal hij in het net geworpen worden, en zal in het wargaren wandelen.
\par 9 De strik zal hem bij de verzenen vatten; de struikrover zal hem overweldigen.
\par 10 Zijn touw is in de aarde verborgen, en zijn val op het pad.
\par 11 De beroeringen zullen hem rondom verschrikken, en hem verstrooien op zijn voeten.
\par 12 Zijn macht zal hongerig wezen, en het verderf is bereid aan zijn zijde.
\par 13 De eerstgeborene des doods zal de grendelen zijner huid verteren, zijn grendelen zal hij verteren.
\par 14 Zijn vertrouwen zal uit zijn tent uitgerukt worden; zulks zal hem doen treden tot den koning der verschrikkingen.
\par 15 Zij zal wonen in zijn tent, waar zij de zijne niet is; zijn woning zal met zwavel overstrooid worden.
\par 16 Van onder zullen zijn wortelen verdorren, en van boven zal zijn tak afgesneden worden.
\par 17 Zijn gedachtenis zal vergaan van de aarde, en hij zal geen naam hebben op de straten.
\par 18 Men zal hem stoten van het licht in de duisternis, en men zal hem van de wereld verjagen.
\par 19 Hij zal geen zoon, noch neef hebben onder zijn volk; en niemand zal in zijn woningen overig zijn.
\par 20 Over zijn dag zullen de nakomelingen verbaasd zijn, en de ouden met schrik bevangen worden.
\par 21 Gewisselijk, zodanige zijn de woningen des verkeerden, en dit is de plaats desgenen die God niet kent.

\chapter{19}

\par 1 Maar Job antwoordde en zeide:
\par 2 Hoe lang zult gijlieden mijn ziel bedroeven, en mij met woorden verbrijzelen?
\par 3 Gij hebt nu tienmaal mij schande aangedaan; gij schaamt u niet, gij verhardt u tegen mij.
\par 4 Maar ook het zij waarlijk, dat ik gedwaald heb, mijn dwaling zal bij mij vernachten.
\par 5 Indien gijlieden waarlijk u verheft tegen mij, en mijn smaad tegen mij drijft;
\par 6 Weet nu, dat God mij heeft omgekeerd, en mij met Zijn net omsingeld.
\par 7 Ziet, ik roep, geweld! doch word niet verhoord; ik schreeuw, doch er is geen recht.
\par 8 Hij heeft mijn weg toegemuurd, dat ik niet doorgaan kan, en over mijn paden heeft Hij duisternis gesteld.
\par 9 Mijn eer heeft Hij van mij afgetrokken, en de kroon mijns hoofds heeft Hij weggenomen.
\par 10 Hij heeft mij rondom afgebroken, zodat ik henenga, en heeft mijn verwachting als een boom weggerukt.
\par 11 Daartoe heeft Hij Zijn toorn tegen mij ontstoken, en mij bij Zich geacht als Zijn vijanden.
\par 12 Zijn benden zijn te zamen aangekomen, en hebben tegen mij haar weg gebaand, en hebben zich gelegerd rondom mijn tent.
\par 13 Mijn broeders heeft Hij verre van mij gedaan; en die mij kennen, zekerlijk, zij zijn van mij vervreemd.
\par 14 Mijn nabestaanden houden op, en mijn bekenden vergeten mij.
\par 15 Mijn huisgenoten en mijn dienstmaagden achten mij voor een vreemde; een uitlander ben ik in hun ogen.
\par 16 Ik riep mijn knecht, en hij antwoordde niet; ik smeekte met mijn mond tot hem.
\par 17 Mijn adem is mijn huisvrouw vreemd; en ik smeek om der kinderen mijns buiks wil.
\par 18 Ook versmaden mij de jonge kinderen; sta ik op, zo spreken zij mij tegen.
\par 19 Alle mensen mijns heimelijken raads hebben een gruwel aan mij; en die ik liefhad, zijn tegen mij gekeerd.
\par 20 Mijn gebeente kleeft aan mijn huid en aan mijn vlees; en ik ben ontkomen met de huid mijner tanden.
\par 21 Ontfermt u mijner, ontfermt u mijner, o gij, mijn vrienden! want de hand Gods heeft mij aangeraakt.
\par 22 Waarom vervolgt gij mij als God, en wordt niet verzadigd van mijn vlees?
\par 23 Och, of nu mijn woorden toch opgeschreven wierden. Och, of zij in een boek ook wierden ingetekend!
\par 24 Dat zij met een ijzeren griffie en lood voor eeuwig in een rots gehouwen wierden!
\par 25 Want ik weet: mijn Verlosser leeft, en Hij zal de laatste over het stof opstaan;
\par 26 En als zij na mijn huid dit doorknaagd zullen hebben, zal ik uit mijn vlees God aanschouwen;
\par 27 Denwelken ik voor mij aanschouwen zal, en mijn ogen zien zullen, en niet een vreemde; mijn nieren verlangen zeer in mijn schoot.
\par 28 Voorwaar, gij zoudt zeggen: Waarom vervolgen wij hem? Nademaal de wortel der zaak in mij gevonden wordt.
\par 29 Schroomt u vanwege het zwaard; want de grimmigheid is over de misdaden des zwaards; opdat gij weet, dat er een gericht zij.

\chapter{20}

\par 1 Toen antwoordde Zofar, de Naamathiet, en zeide:
\par 2 Daarom doen mijn gedachten mij antwoorden, en over zulks is mijn verhaasten in mij.
\par 3 Ik heb aangehoord een bestraffing, die mij schande aandoet; maar de geest zal uit mijn verstand voor mij antwoorden.
\par 4 Weet gij dit? Van altoos af, van dat God den mens op de wereld gezet heeft,
\par 5 Dat het gejuich de goddelozen van nabij geweest is, en de vreugde des huichelaars voor een ogenblik?
\par 6 Wanneer zijn hoogheid tot den hemel toe opklomme, en zijn hoofd tot aan de wolken raakte;
\par 7 Zal hij, gelijk zijn drek, in eeuwigheid vergaan; die hem gezien hadden, zullen zeggen: Waar is hij?
\par 8 Hij zal wegvlieden als een droom, dat men hem niet vinden zal, en hij zal verjaagd worden als een gezicht des nachts.
\par 9 Het oog, dat hem zag, zal het niet meer doen; en zijn plaats zal hem niet meer aanschouwen.
\par 10 Zijn kinderen zullen zoeken den armen te behagen; en zijn handen zullen zijn vermogen moeten weder uitkeren.
\par 11 Zijn beenderen zullen vol van zijn verborgene zonden zijn; van welke elkeen met hem op het stof nederliggen zal.
\par 12 Indien het kwaad in zijn mond zoet is, hij dat verbergt, onder zijn tong,
\par 13 Hij dat spaart, en hetzelve niet verlaat, maar dat in het midden van zijn gehemelte inhoudt;
\par 14 Zijn spijze zal in zijn ingewand veranderd worden; gal der adderen zal zij in het binnenste van hem zijn.
\par 15 Hij heeft goed ingeslokt, maar zal het uitspuwen; God zal het uit zijn buik uitdrijven.
\par 16 Het vergif der adderen zal hij zuigen; de tong der slang zal hem doden.
\par 17 De stromen, rivieren, beken van honig en boter zal hij niet zien.
\par 18 Den arbeid zal hij wedergeven en niet inslokken; naar het vermogen zijner verandering, zo zal hij van vreugde niet opspringen.
\par 19 Omdat hij onderdrukt heeft, de armen verlaten heeft, een huis geroofd heeft, dat hij niet opgebouwd had;
\par 20 Omdat hij geen rust in zijn buik gekend heeft, zo zal hij van zijn gewenst goed niet uitbehouden.
\par 21 Er zal niets overig zijn, dat hij ete; daarom zal hij niet wachten naar zijn goed.
\par 22 Als zijn genoegzaamheid zal vol zijn, zal hem bang zijn; alle hand des ellendigen zal over hem komen.
\par 23 Er zij wat om zijn buik te vullen; God zal over hem de hitte Zijns toorns zenden, en over hem regenen op zijn spijze.
\par 24 Hij zij gevloden van de ijzeren wapenen, de stalen boog zal hem doorschieten.
\par 25 Men zal het zwaard uittrekken, het zal uit het lijf uitgaan, en glinsterende uit zijn gal voortkomen; verschrikkingen zullen over hem zijn.
\par 26 Alle duisternis zal verborgen zijn in zijn schuilplaatsen; een vuur, dat niet opgeblazen is, zal hem verteren; den overigen in zijn tent zal het kwalijk gaan.
\par 27 De hemel zal zijn ongerechtigheid openbaren, en de aarde zal zich tegen hem opmaken.
\par 28 De inkomste van zijn huis zal weggevoerd worden; het zal al henenvloeien in den dag Zijns toorns.
\par 29 Dit is het deel des goddelozen mensen van God, en de erve zijner redenen van God.

\chapter{21}

\par 1 Maar Job antwoordde en zeide:
\par 2 Hoort aandachtelijk mijn rede, en laat dit zijn uw vertroostingen.
\par 3 Verdraagt mij, en ik zal spreken; en nadat ik gesproken zal hebben, spot dan.
\par 4 Is (mij aangaande) mijn klacht tot den mens? Doch of het zo ware, waarom zou mijn geest niet verdrietig zijn?
\par 5 Ziet mij aan, en wordt verbaasd, en legt de hand op den mond.
\par 6 Ja, wanneer ik daaraan gedenk, zo word ik beroerd, en mijn vlees heeft een gruwen gevat.
\par 7 Waarom leven de goddelozen, worden oud, ja, worden geweldig in vermogen?
\par 8 Hun zaad is bestendig met hen voor hun aangezicht, en hun spruiten zijn voor hun ogen.
\par 9 Hun huizen hebben vrede zonder vreze, en de roede Gods is op hen niet.
\par 10 Zijn stier bespringt, en mist niet; zijn koe kalft, en misdraagt niet.
\par 11 Hun jonge kinderen zenden zij uit als een kudde, en hun kinderen huppelen.
\par 12 Zij heffen op met de trommel en de harp, en zij verblijden zich op het geluid des orgels.
\par 13 In het goede verslijten zij hun dagen; en in een ogenblik dalen zij in het graf.
\par 14 Nochtans zeggen zij tot God: Wijk van ons, want aan de kennis Uwer wegen hebben wij geen lust.
\par 15 Wat is de Almachtige, dat wij Hem zouden dienen? En wat baat zullen wij hebben, dat wij Hem aanlopen zouden?
\par 16 Doch ziet, hun goed is niet in hun hand; de raad der goddelozen is verre van mij.
\par 17 Hoe dikwijls geschiedt het, dat de lamp der goddelozen uitgeblust wordt, en hun verderf hun overkomt; dat God hun smarten uitdeelt in Zijn toorn!
\par 18 Dat zij gelijk stro worden voor den wind, en gelijk kaf, dat de wervelwind wegsteelt;
\par 19 Dat God Zijn geweld weglegt voor Zijn kinderen, hem vergeldt, dat hij het gewaar wordt;
\par 20 Dat zijn ogen zijn ondergang zien, en hij drinkt van de grimmigheid des Almachtigen!
\par 21 Want wat lust zou hij na zich aan zijn huis hebben, als het getal zijner maanden afgesneden is?
\par 22 Zal men God wetenschap leren, daar Hij de hogen richt?
\par 23 Deze sterft in de kracht zijner volkomenheid, daar hij gans stil en gerust was;
\par 24 Zijn melkvaten waren vol melk, en het merg zijner benen was bevochtigd.
\par 25 De ander daarentegen sterft met een bittere ziel, en hij heeft van het goede niet gegeten.
\par 26 Zij liggen te zamen neder in het stof, en het gewormte overdekt ze.
\par 27 Ziet, ik weet ulieder gedachten, en de boze verdichtselen, waarmede gij tegen mij geweld doet.
\par 28 Want gij zult zeggen: Waar is het huis van den prins, en waar is de tent van de woningen der goddelozen?
\par 29 Hebt gijlieden niet gevraagd de voorbijgaanden op den weg, en kent gij hun tekenen niet?
\par 30 Dat de boze onttrokken wordt ten dage des verderfs; dat zij ten dage der verbolgenheden ontvoerd worden.
\par 31 Wie zal hem in het aangezicht zijn weg vertonen? Als hij wat doet, wie zal hem vergelden?
\par 32 Eindelijk wordt hij naar de graven gebracht, en is gedurig in den aardhoop.
\par 33 De kluiten des dals zijn hem zoet, en hij trekt na zich alle mensen; en dergenen, die voor hem geweest zijn, is geen getal.
\par 34 Hoe vertroost gij mij dan met ijdelheid, dewijl in uw antwoorden overtreding overig is?

\chapter{22}

\par 1 Toen antwoordde Elifaz, de Themaniet, en zeide:
\par 2 Zal ook een man Gode voordelig zijn? Maar voor zichzelven zal de verstandige voordelig zijn.
\par 3 Is het voor den Almachtige nuttigheid, dat gij rechtvaardig zijt; of gewin, dat gij uw wegen volmaakt?
\par 4 Is het om uw vreze, dat Hij u bestraft, dat Hij met u in het gericht komt?
\par 5 Is niet uw boosheid groot, en uwer ongerechtigheden geen einde?
\par 6 Want gij hebt uw broederen zonder oorzaak pand afgenomen, en de klederen der naakten hebt gij uitgetogen.
\par 7 Den moede hebt gij geen water te drinken gegeven, en van den hongerige hebt gij het brood onthouden.
\par 8 Maar was er een man van geweld, voor dien was het land, en een aanzienlijk persoon woonde daarin.
\par 9 De weduwen hebt gij ledig weggezonden, en de armen der wezen zijn verbrijzeld.
\par 10 Daarom zijn strikken rondom u, en vervaardheid heeft u haastelijk beroerd.
\par 11 Of gij ziet de duisternis niet, en des water overvloed bedekt u.
\par 12 Is niet God in de hoogte der hemelen? Zie toch het opperste der sterren aan, dat zij verheven zijn.
\par 13 Daarom zegt gij: Wat weet er God van? Zal Hij door de donkerheid oordelen?
\par 14 De wolken zijn Hem een verberging, dat Hij niet ziet; en Hij bewandelt den omgang der hemelen.
\par 15 Hebt gij het pad der eeuw waargenomen, dat de ongerechtige lieden betreden hebben?
\par 16 Die rimpelachtig gemaakt zijn, als het de tijd niet was; een vloed is over hun grond uitgestort;
\par 17 Die zeiden tot God: Wijk van ons! En wat had de Almachtige hun gedaan?
\par 18 Hij had immers hun huizen met goed gevuld; daarom is de raad der goddelozen verre van mij.
\par 19 De rechtvaardigen zagen het, en waren blijde, en de onschuldige bespotte hen;
\par 20 Dewijl onze stand niet verdelgd is, maar het vuur hun overblijfsel verteerd heeft.
\par 21 Gewen u toch aan Hem, en heb vrede; daardoor zal u het goede overkomen.
\par 22 Ontvang toch de wet uit Zijn mond, en leg Zijn redenen in uw hart.
\par 23 Zo gij u bekeert tot den Almachtige, gij zult gebouwd worden; doe het onrecht verre van uw tenten.
\par 24 Dan zult gij het goud op het stof leggen, en het goud van Ofir bij den rotssteen der beken;
\par 25 Ja, de Almachtige zal uw overvloedig goud zijn, en uw krachtig zilver zijn;
\par 26 Want dan zult gij u over den Almachtige verlustigen, en gij zult tot God uw aangezicht opheffen.
\par 27 Gij zult tot Hem ernstiglijk bidden, en Hij zal u verhoren; en gij zult uw geloften betalen.
\par 28 Als gij een zaak besluit, zo zal zij u bestendig zijn; en op uw wegen zal het licht schijnen.
\par 29 Als men iemand vernederen zal, en gij zeggen zult: Het zij verhoging; dan zal God den nederige van ogen behouden.
\par 30 Ja, Hij zal dien bevrijden, die niet onschuldig is, want hij wordt bevrijd door de zuiverheid uwer handen.

\chapter{23}

\par 1 Maar Job antwoordde en zeide:
\par 2 Ook heden is mijn klacht wederspannigheid; mijn plage is zwaar boven mijn zuchten.
\par 3 Och, of ik wist, dat ik Hem vinden zou, ik zou tot Zijn stoel komen;
\par 4 Ik zou het recht voor Zijn aangezicht ordentelijk voorstellen, en mijn mond zou ik met verdedigingen vervullen.
\par 5 Ik zou de redenen weten, die Hij mij antwoorden zou; en verstaan, wat Hij mij zeggen zou.
\par 6 Zou Hij naar de grootheid Zijner macht met mij twisten? Neen; maar Hij zou acht op mij slaan.
\par 7 Daar zou de oprechte met Hem pleiten; en ik zou mij in eeuwigheid van mijn Rechter vrijmaken.
\par 8 Zie, ga ik voorwaarts, zo is Hij er niet, of achterwaarts, zo verneem ik Hem niet.
\par 9 Als Hij ter linkerhand werkt, zo aanschouw ik Hem niet; bedekt Hij Zich ter rechterhand, zo zie ik Hem niet.
\par 10 Doch Hij kent den weg, die bij mij is; Hij beproeve mij; als goud zal ik uitkomen.
\par 11 Aan Zijn gang heeft mijn voet vastgehouden; Zijn weg heb ik bewaard, en ben niet afgeweken.
\par 12 Het gebod Zijner lippen heb ik ook niet weggedaan; de redenen Zijns monds heb ik meer dan mijn bescheiden deel weggelegd.
\par 13 Maar is Hij tegen iemand, wie zal dan Hem afkeren? Wat Zijn ziel begeert, dat zal Hij doen.
\par 14 Want Hij zal volbrengen, dat over mij bescheiden is; en diergelijke dingen zijn er vele bij Hem.
\par 15 Hierom word ik voor Zijn aangezicht beroerd; aanmerk het, en vrees voor Hem;
\par 16 Want God heeft mijn hart week gemaakt, en de Almachtige heeft mij beroerd;
\par 17 Omdat ik niet uitgedelgd ben voor de duisternis, en dat Hij van mijn aangezicht de donkerheid bedekt heeft.

\chapter{24}

\par 1 Waarom zouden van den Almachtige de tijden niet verborgen zijn, dewijl zij, die Hem kennen, Zijn dagen niet zien?
\par 2 Zij tasten de landpalen aan; de kudden roven zij, en weiden ze.
\par 3 Den ezel der wezen drijven zij weg; den os ener weduwe nemen zij te pand.
\par 4 Zij doen de nooddruftigen wijken van den weg; te zamen versteken zich de ellendigen des lands.
\par 5 Ziet, zij zijn woudezels in de woestijn; zij gaan uit tot hun werk, makende zich vroeg op ten roof; het vlakke veld is hem tot spijs, en den jongeren.
\par 6 Op het veld maaien zij zijn voeder, en den wijnberg des goddelozen lezen zij af.
\par 7 Den naakten laten zij vernachten zonder kleding, die geen deksel heeft tegen de koude.
\par 8 Van den stroom der bergen worden zij nat, en zonder toevlucht zijnde, omhelzen zij de steenrotsen.
\par 9 Zij rukken het weesje van de borst, en dat over den arme is, nemen zij te pand.
\par 10 Den naakte doen zij weggaan zonder kleed, en hongerig, die garven dragen.
\par 11 Tussen hun muren persen zij olie uit, treden de wijnpersen, en zijn dorstig.
\par 12 Uit de stad zuchten de lieden, en de ziel der verwonden schreeuwt uit; nochtans beschikt God niets ongerijmds.
\par 13 Zij zijn onder de wederstrevers des lichts; zij kennen Zijn wegen niet, en zij blijven niet op Zijn paden.
\par 14 Met het licht staat de moorder op, doodt den arme en den nooddruftige; en des nachts is hij als een dief.
\par 15 Ook neemt het oog des overspelers de schemering waar, zeggende: Geen oog zal mij zien; en hij legt een deksel op het aangezicht.
\par 16 In de duisternis doorgraaft hij de huizen, die zij zich des daags afgetekend hadden; zij kennen het licht niet.
\par 17 Want de morgenstond is hun te zamen de schaduw des doods; als men hen kent, zijn zij in de strikken van des doods schaduw.
\par 18 Hij is licht op het vlakke der wateren; vervloekt is hun deel op de aarde; hij wendt zich niet tot den weg der wijngaarden.
\par 19 De droogte mitsgaders de hitte nemen de sneeuwwateren weg; alzo het graf dergenen, die gezondigd hebben.
\par 20 De baarmoeder vergeet hem, het gewormte is hem zoet, zijns wordt niet meer gedacht; en het onrecht wordt gebroken als een hout.
\par 21 De onvruchtbare, die niet baart, teert hij af, en aan de weduwe doet hij niets goeds.
\par 22 Ook trekt hij de machtigen door zijn kracht; staat hij op, zo is men des levens niet zeker.
\par 23 Stelt hem God in gerustigheid, zo steunt hij daarop; nochtans zijn Zijn ogen op hun wegen.
\par 24 Zij zijn een weinig tijds verheven, daarna is er niemand van hen; zij worden nedergedrukt; gelijk alle anderen worden zij besloten; en gelijk de top ener aar worden zij afgesneden.
\par 25 Indien het nu zo niet is, wie zal mij leugenachtig maken, en mijn rede tot niet brengen?

\chapter{25}

\par 1 Toen antwoordde Bildad, de Suhiet, en zeide:
\par 2 Heerschappij en vreze zijn bij Hem, Hij maakt vrede in Zijn hoogten.
\par 3 Is er een getal Zijner benden? En over wien staat Zijn licht niet op?
\par 4 Hoe zou dan een mens rechtvaardig zijn bij God, en hoe zou hij zuiver zijn, die van een vrouw geboren is?
\par 5 Zie, tot de maan toe, en zij zal geen schijnsel geven; en de sterren zijn niet zuiver in Zijn ogen.
\par 6 Hoeveel te min de mens, die een made is, en des mensen kind, die een worm is!

\chapter{26}

\par 1 Maar Job antwoordde en zeide:
\par 2 Hoe hebt gij geholpen dien, die zonder kracht is, en behouden den arm, die zonder sterkte is?
\par 3 Hoe hebt gij hem geraden, die geen wijsheid heeft, en de zaak, alzo zij is, ten volle bekend gemaakt?
\par 4 Aan wien hebt gij die woorden verhaald? En wiens geest is van u uitgegaan?
\par 5 De doden zullen geboren worden van onder de wateren, en hun inwoners.
\par 6 De hel is naakt voor Hem, en geen deksel is er voor het verderf.
\par 7 Hij breidt het noorden uit over het woeste; Hij hangt de aarde aan een niet.
\par 8 Hij bindt de wateren in Zijn wolken; nochtans scheurt de wolk daaronder niet.
\par 9 Hij houdt het vlakke Zijns troons vast; Hij spreidt Zijn wolk daarover.
\par 10 Hij heeft een gezet perk over het vlakke der wateren rondom afgetekend, tot aan de voleinding toe des lichts met de duisternis.
\par 11 De pilaren des hemels sidderen, en ontzetten zich voor Zijn schelden.
\par 12 Door Zijn kracht klieft Hij de zee, en door Zijn verstand verslaat Hij haar verheffing.
\par 13 Door Zijn Geest heeft Hij de hemelen versierd; Zijn hand heeft de langwemelende slang geschapen.
\par 14 Ziet, dit zijn maar uiterste einden Zijner wegen; en wat een klein stukje der zaak hebben wij van Hem gehoord? Wie zou dan den donder Zijner mogendheden verstaan?

\chapter{27}

\par 1 En Job ging voort zijn spreuk op te heffen, en zeide:
\par 2 Zo waarachtig als God leeft, Die mijn recht weggenomen heeft, en de Almachtige, Die mijner ziel bitterheid heeft aangedaan!
\par 3 Zo lang als mijn adem in mij zal zijn, en het geblaas Gods in mijn neus;
\par 4 Indien mijn lippen onrecht zullen spreken, en indien mijn tong bedrog zal uitspreken!
\par 5 Het zij verre van mij, dat ik ulieden rechtvaardigen zou; totdat ik den geest zal gegeven hebben, zal ik mijn oprechtigheid van mij niet wegdoen.
\par 6 Aan mijn gerechtigheid zal ik vasthouden, en zal ze niet laten varen; mijn hart zal die niet versmaden van mijn dagen.
\par 7 Mijn vijand zij als de goddeloze, en die zich tegen mij opmaakt, als de verkeerde.
\par 8 Want wat is de verwachting des huichelaars, als hij zal gierig geweest zijn, wanneer God zijn ziel zal uittrekken?
\par 9 Zal God zijn geroep horen, als benauwdheid over hem komt?
\par 10 Zal hij zich verlustigen in den Almachtige? Zal hij God aanroepen te aller tijd?
\par 11 Ik zal ulieden leren van de hand Gods; wat bij den Almachtige is, zal ik niet verhelen.
\par 12 Ziet, gij zelve allen hebt het gezien; en waarom wordt gij dus door ijdelheid verijdeld?
\par 13 Dit is het deel des goddelozen mensen bij God, en de erve der tirannen, die zij van den Almachtige ontvangen zullen.
\par 14 Indien zijn kinderen vermenigvuldigen, het is ten zwaarde; en zijn spruiten zullen van brood niet verzadigd worden.
\par 15 Zijn overgeblevenen zullen in den dood begraven worden, en zijn weduwen zullen niet wenen.
\par 16 Zo hij zilver opgehoopt zal hebben als stof, en kleding bereid als leem;
\par 17 Hij zal ze bereiden, maar de rechtvaardige zal ze aantrekken, en de onschuldige zal het zilver delen.
\par 18 Hij bouwt zijn huis als een motte, en als een hoeder de hutte maakt.
\par 19 Rijk ligt hij neder, en wordt niet weggenomen; doet hij zijn ogen open, zo is hij er niet.
\par 20 Verschrikkingen zullen hem als wateren aangrijpen; des nachts zal hem een wervelwind wegstelen.
\par 21 De oostenwind zal hem wegvoeren, dat hij henengaat, en zal hem wegstormen uit zijn plaats.
\par 22 En God zal dit over hem werpen, en niet sparen; van Zijn hand zal hij snellijk vlieden.
\par 23 Een ieder zal over hem met zijn handen klappen, en over hem fluiten uit zijn plaats.

\chapter{28}

\par 1 Gewisselijk, er is voor het zilver een uitgang, en een plaats voor het goud, dat zij smelten.
\par 2 Het ijzer wordt uit stof genomen, en uit steen wordt koper gegoten.
\par 3 Het einde, dat God gesteld heeft voor de duisternis, en al het uiterste onderzoekt hij; het gesteente der donkerheid en der schaduw des doods.
\par 4 Breekt er een beek door, bij dengene, die daar woont, de wateren vergeten zijnde van den voet, worden van den mens uitgeput, en gaan weg.
\par 5 Uit de aarde komt het brood voort, en onder zich wordt zij veranderd, alsof zij vuur ware.
\par 6 Haar stenen zijn de plaats van den saffier, en zij heeft stofjes van goud.
\par 7 De roofvogel heeft het pad niet gekend, en het oog der kraai heeft het niet gezien.
\par 8 De jonge hoogmoedige dieren hebben het niet betreden, de felle leeuw is daarover niet heengegaan.
\par 9 Hij legt zijn hand aan de keiachtige rots, hij keert de bergen van den wortel om.
\par 10 In de rotsstenen houwt hij stromen uit, en zijn oog ziet al het kostelijke.
\par 11 Hij bindt de rivier toe, dat niet een traan uitkomt, en het verborgene brengt hij uit in het licht.
\par 12 Maar de wijsheid, van waar zal zij gevonden worden? En waar is de plaats des verstands?
\par 13 De mens weet haar waarde niet, en zij wordt niet gevonden in het land der levenden.
\par 14 De afgrond zegt: Zij is in mij niet; en de zee zegt: Zij is niet bij mij.
\par 15 Het gesloten goud kan voor haar niet gegeven worden, en met zilver kan haar prijs niet worden opgewogen.
\par 16 Zij kan niet geschat worden tegen fijn goud van Ofir, tegen den kostelijken Schoham, en den Saffier.
\par 17 Men kan het goud of het kristal haar niet gelijk waarderen; ook is zij niet te verwisselen voor een kleinood van dicht goud.
\par 18 De Ramoth en Gabisch zal niet gedacht worden; want de trek der wijsheid is meerder dan der Robijnen.
\par 19 Men kan de Topaas van Morenland haar niet gelijk waarderen; en bij het fijn louter goud kan zij niet geschat worden.
\par 20 Die wijsheid dan, van waar komt zij, en waar is de plaats des verstands?
\par 21 Want zij is verholen voor de ogen aller levenden, en voor het gevogelte des hemels is zij verborgen.
\par 22 Het verderf en de dood zeggen: Haar gerucht hebben wij met onze oren gehoord.
\par 23 God verstaat haar weg, en Hij weet haar plaats.
\par 24 Want Hij schouwt tot aan de einden der aarde, Hij ziet onder al de hemelen.
\par 25 Als Hij den wind het gewicht maakte, en de wateren opwoog in mate;
\par 26 Als Hij den regen een gezette orde maakte, en een weg voor het weerlicht der donderen;
\par 27 Toen zag Hij haar, en vertelde ze; Hij schikte ze, en ook doorzocht Hij ze.
\par 28 Maar tot den mens heeft Hij gezegd: Zie, de vreze des HEEREN is de wijsheid, en van het kwade te wijken is het verstand.

\chapter{29}

\par 1 En Job ging voort zijn spreuk op te heffen, en zeide:
\par 2 Och, of ik ware, gelijk in de vorige maanden, gelijk in de dagen, toen God mij bewaarde!
\par 3 Toen Hij Zijn lamp deed schijnen over mijn hoofd, en ik bij Zijn licht de duisternis doorwandelde;
\par 4 Gelijk als ik was in de dagen mijner jonkheid, toen Gods verborgenheid over mijn tent was;
\par 5 Toen de Almachtige nog met mij was, en mijn jongens rondom mij;
\par 6 Toen ik mijn gangen wies in boter, en de rots bij mij oliebeken uitgoot;
\par 7 Toen ik uitging naar de poort door de stad, toen ik mijn stoel op de straat liet bereiden.
\par 8 De jongens zagen mij, en verstaken zich, en de stokouden rezen op en stonden.
\par 9 De oversten hielden de woorden in, en leiden de hand op hun mond.
\par 10 De stem der vorsten verstak zich, en hun tong kleefde aan hun gehemelte.
\par 11 Als een oor mij hoorde, zo hield het mij gelukzalig; als mij een oog zag, zo getuigde het van mij.
\par 12 Want ik bevrijdde den ellendige, die riep, en den wees, die geen helper had.
\par 13 De zegen desgenen, die verloren ging, kwam op mij; en het hart der weduwe deed ik vrolijk zingen.
\par 14 Ik bekleedde mij met gerechtigheid, en zij bekleedde mij; mijn oordeel was als een mantel en vorstelijke hoed.
\par 15 Den blinden was ik tot ogen, en den kreupelen was ik tot voeten.
\par 16 Ik was den nooddruftigen een vader; en het geschil, dat ik niet wist, dat onderzocht ik.
\par 17 En ik verbrak de baktanden des verkeerden, en wierp den roof uit zijn tanden.
\par 18 En ik zeide: Ik zal in mijn nest den geest geven, en ik zal de dagen vermenigvuldigen als het zand.
\par 19 Mijn wortel was uitgebreid aan het water, en dauw vernachtte op mijn tak.
\par 20 Mijn heerlijkheid was nieuw bij mij, en mijn boog veranderde zich in mijn hand.
\par 21 Zij hoorden mij aan, en wachtten, en zwegen op mijn raad.
\par 22 Na mijn woord spraken zij niet weder, en mijn rede drupte op hen.
\par 23 Want zij wachtten naar mij, gelijk naar den regen, en sperden hun mond open, als naar den spaden regen.
\par 24 Lachte ik hun toe, zij geloofden het niet; en het licht mijns aangezichts deden zij niet nedervallen.
\par 25 Verkoos ik hun weg, zo zat ik bovenaan, en woonde als een koning onder de benden, als een, die treurigen vertroost.

\chapter{30}

\par 1 Maar nu lachen over mij minderen dan ik van dagen, welker vaderen ik versmaad zou hebben, om bij de honden mijner kudde te stellen.
\par 2 Waartoe zou mij ook geweest zijn de krachten hunner handen? Zij was door ouderdom in hen vergaan.
\par 3 Die door gebrek en honger eenzaam waren, vliedende naar dorre plaatsen, in het donkere, woeste en verwoeste.
\par 4 Die ziltige kruiden plukten bij de struiken, en welker spijze was de wortel der jeneveren.
\par 5 Zij werden uit het midden uitgedreven; (men jouwde over hen, als over een dief),
\par 6 Opdat zij wonen zouden in de kloven der dalen, de holen des stofs en der steenrotsen.
\par 7 Zij schreeuwden tussen de struiken; onder de netelen vergaderden zij zich.
\par 8 Zij waren kinderen der dwazen, en kinderen van geen naam; zij waren geslagen uit den lande.
\par 9 Maar nu ben ik hun een snarenspel geworden, en ik ben hun tot een klapwoord.
\par 10 Zij hebben een gruwel aan mij, zij maken zich verre van mij, ja, zij onthouden het speeksel niet van mijn aangezicht.
\par 11 Want Hij heeft mijn zeel losgemaakt, en mij bedrukt; daarom hebben zij den breidel voor mijn aangezicht afgeworpen.
\par 12 Ter rechterhand staat de jeugd op, stoten mijn voeten uit, en banen tegen mij hun verderfelijke wegen.
\par 13 Zij breken mijn pad af, zij bevorderen mijn ellende; zij hebben geen helper van doen.
\par 14 Zij komen aan, als door een wijde breuk; onder de verwoesting rollen zij zich aan.
\par 15 Men is met verschrikkingen tegen mij gekeerd; elk een vervolgt als een wind mijn edele ziel, en mijn heil is als een wolk voorbijgegaan.
\par 16 Daarom stort zich nu mijn ziel in mij uit; de dagen des druks grijpen mij aan.
\par 17 Des nachts doorboort Hij mijn beenderen in mij, en mijn polsaderen rusten niet.
\par 18 Door de veelheid der kracht is mijn kleed veranderd; Hij omgordt mij als de kraag mijns roks.
\par 19 Hij heeft mij in het slijk geworpen, en ik ben gelijk geworden als stof en as.
\par 20 Ik schrei tot U, maar Gij antwoordt mij niet; ik sta, maar Gij acht niet op mij.
\par 21 Gij zijt veranderd in een wrede tegen mij; door de sterkte Uwer hand wederstaat Gij mij hatelijk.
\par 22 Gij heft mij op in den wind; Gij doet mij daarop rijden, en Gij versmelt mij het wezen.
\par 23 Want ik weet, dat Gij mij ter dood brengen zult, en tot het huis der samenkomst aller levenden.
\par 24 Maar Hij zal tot den aardhoop de hand niet uitsteken; is er bij henlieden geschrei in zijn verdrukking?
\par 25 Weende ik niet over hem, die harde dagen had? Was mijn ziel niet beangst over den nooddruftige?
\par 26 Nochtans toen ik het goede verwachtte, zo kwam het kwade; toen ik hoopte naar het licht, zo kwam de donkerheid.
\par 27 Mijn ingewand ziedt, en is niet stil; de dagen der verdrukking zijn mij voorgekomen.
\par 28 Ik ga zwart daarheen, niet van de zon; opstaande schreeuw ik in de gemeente.
\par 29 Ik ben den draken een broeder geworden, en een metgezel der jonge struisen.
\par 30 Mijn huid is zwart geworden over mij, en mijn gebeente is ontstoken van dorrigheid.
\par 31 Hierom is mijn harp tot een rouwklage geworden, en mijn orgel tot een stem der wenenden.

\chapter{31}

\par 1 Ik heb een verbond gemaakt met mijn ogen; hoe zou ik dan acht gegeven hebben op een maagd?
\par 2 Want wat is het deel Gods van boven, of de erve des Almachtigen uit de hoogten?
\par 3 Is niet het verderf voor den verkeerde, ja, wat vreemds voor de werkers der ongerechtigheid?
\par 4 Ziet Hij niet mijn wegen, en telt Hij niet al mijn treden?
\par 5 Zo ik met ijdelheid omgegaan heb, en mijn voet gesneld heeft tot bedriegerij;
\par 6 Hij wege mij op, in een rechte weegschaal, en God zal mijn oprechtigheid weten.
\par 7 Zo mijn gang uit den weg geweken is, en mijn hart mijn ogen nagevolgd is, en aan mijn handen iets aankleeft;
\par 8 Zo moet ik zaaien, maar een ander eten, en mijn spruiten moeten uitgeworteld worden!
\par 9 Zo mijn hart verlokt is geweest tot een vrouw, of ik aan mijns naasten deur geloerd heb;
\par 10 Zo moet mijn huisvrouw met een ander malen, en anderen zich over haar krommen!
\par 11 Want dat is een schandelijke daad, en het is een misdaad bij de rechters.
\par 12 Want dat is een vuur, hetwelk tot de verderving toe verteert, en al mijn inkomen uitgeworteld zou hebben.
\par 13 Zo ik versmaad heb het recht mijns knechts, of mijner dienstmaagd, als zij geschil hadden met mij;
\par 14 (Want wat zou ik doen, als God opstond? En als Hij bezoeking deed, wat zou ik Hem antwoorden?
\par 15 Heeft Hij niet, Die mij in den buik maakte, hem ook gemaakt en Een ons in de baarmoeder bereid?)
\par 16 Zo ik den armen hun begeerte onthouden heb, of de ogen der weduwe laten versmachten;
\par 17 En mijn bete alleen gegeten heb, zodat de wees daarvan niet gegeten heeft;
\par 18 (Want van mijn jonkheid af is hij bij mij opgetogen, als bij een vader, en van mijner moeders buik af heb ik haar geleid;)
\par 19 Zo ik iemand heb zien omkomen, omdat hij zonder kleding was, en dat de nooddruftige geen deksel had;
\par 20 Zo zijn lenden mij niet gezegend hebben, toen hij van de vellen mijner lammeren verwarmd werd;
\par 21 Zo ik mijn hand tegen den wees bewogen heb, omdat ik in de poort mijn hulp zag;
\par 22 Mijn schouder valle van het schouderbeen, en mijn arm breke van zijn pijp af!
\par 23 Want het verderf Gods was bij mij een schrik, en ik vermocht niet vanwege Zijn hoogheid.
\par 24 Zo ik het goud tot mijn hoop gezet heb, of tot het fijn goud gezegd heb: Gij zijt mijn vertrouwen;
\par 25 Zo ik blijde ben geweest, omdat mijn vermogen groot was, en omdat mijn hand geweldig veel verkregen had;
\par 26 Zo ik het licht aangezien heb, wanneer het scheen, of de maan heerlijk voortgaande;
\par 27 En mijn hart verlokt is geweest in het verborgen, dat mijn hand mijn mond gekust heeft;
\par 28 Dat ware ook een misdaad bij den rechter; want ik zou den God van boven verzaakt hebben.
\par 29 Zo ik verblijd ben geweest in de verdrukking mijns haters, en mij opgewekt heb, als het kwaad hem vond;
\par 30 (Ook heb ik mijn gehemelte niet toegelaten te zondigen, mits door een vloek zijn ziel te begeren).
\par 31 Zo de lieden mijner tent niet hebben gezegd: Och, of wij van zijn vlees hadden, wij zouden niet verzadigd worden;
\par 32 De vreemdeling overnachtte niet op de straat; mijn deuren opende ik naar den weg;
\par 33 Zo ik, gelijk Adam, mijn overtredingen bedekt heb, door eigenliefde mijn misdaad verbergende!
\par 34 Zeker, ik kon wel een grote menigte geweldiglijk onderdrukt hebben; maar de verachtste der huisgezinnen zou mij afgeschrikt hebben; zodat ik gewezen zou hebben, en ter deure niet uitgegaan zijn.
\par 35 Och, of ik een hadde, die mij hoorde! Zie, mijn oogmerk is, dat de Almachtige mij antwoorde, en dat mijn tegenpartij een boek schrijve.
\par 36 Zou ik het niet op mijn schouder dragen? Ik zou het op mij binden als een kroon.
\par 37 Het getal mijner treden zou ik hem aanwijzen; als een vorst zou ik tot hem naderen.
\par 38 Zo mijn land tegen mij roept, en zijn voren te zamen wenen;
\par 39 Zo ik zijn vermogen gegeten heb zonder geld, en de ziel zijner akkerlieden heb doen hijgen;
\par 40 Dat voor tarwe distelen voortkomen, en voor gerst stinkkruid! De woorden van Job hebben een einde.

\chapter{32}

\par 1 Toen hielden de drie mannen op van Job te antwoorden, dewijl hij in zijn ogen rechtvaardig was.
\par 2 Zo ontstak de toorn van Elihu, den zoon van Baracheel, den Buziet, van het geslacht van Ram; tegen Job werd zijn toorn ontstoken, omdat hij zijn ziel meer rechtvaardigde dan God.
\par 3 Zijn toorn ontstak ook tegen zijn drie vrienden, omdat zij, geen antwoord vindende, nochtans Job verdoemden.
\par 4 Doch Elihu had gewacht op Job in het spreken, omdat zij ouder van dagen waren dan hij.
\par 5 Als dan Elihu zag, dat er geen antwoord was in den mond van die drie mannen, ontstak zijn toorn.
\par 6 Hierom antwoordde Elihu, de zoon van Baracheel, den Buziet, en zeide: Ik ben minder van dagen, maar gijlieden zijt stokouden; daarom heb ik geschroomd en gevreesd, ulieden mijn gevoelen te vertonen.
\par 7 Ik zeide: Laat de dagen spreken, en de veelheid der jaren wijsheid te kennen geven.
\par 8 Zekerlijk de geest, die in den mens is, en de inblazing des Almachtigen, maakt henlieden verstandig.
\par 9 De groten zijn niet wijs, en de ouden verstaan het recht niet.
\par 10 Daarom zeg ik: Hoor naar mij; ik zal mijn gevoelen ook vertonen.
\par 11 Ziet, ik heb gewacht op ulieder woorden; ik heb het oor gewend tot ulieder aanmerkingen, totdat gij redenen uitgezocht hadt.
\par 12 Als ik nu acht op u gegeven heb, ziet, er is niemand, die Job overreedde, die uit ulieden zijn redenen beantwoordde;
\par 13 Opdat gij niet zegt: Wij hebben de wijsheid gevonden; God heeft hem nedergestoten, geen mens.
\par 14 Nu heeft hij tegen mij geen woorden gericht, en met ulieder woorden zal ik hem niet beantwoorden.
\par 15 Zij zijn ontzet, zij antwoorden niet meer; zij hebben de woorden van zich verzet.
\par 16 Ik heb dan gewacht, maar zij spreken niet; want zij staan stil; zij antwoorden niet meer.
\par 17 Ik zal mijn deel ook antwoorden, ik zal mijn gevoelen ook vertonen.
\par 18 Want ik ben der woorden vol; de geest mijns buiks benauwt mij.
\par 19 Ziet, mijn buik is als de wijn, die niet geopend is; gelijk nieuwe lederen zakken zou hij bersten.
\par 20 Ik zal spreken, opdat ik voor mij lucht krijge; ik zal mijn lippen openen, en zal antwoorden.
\par 21 Och, dat ik niemands aangezicht aanneme, en tot den mens geen bijnamen gebruike!
\par 22 Want ik weet geen bijnamen te gebruiken; in kort zou mijn Maker mij wegnemen.

\chapter{33}

\par 1 En gewisselijk, o Job! hoor toch mijn redenen, en neem al mijn woorden ter ore.
\par 2 Zie nu, ik heb mijn mond opengedaan; mijn tong spreekt onder mijn gehemelte.
\par 3 Mijn redenen zullen de oprechtigheid mijns harten, en de wetenschap mijner lippen, wat zuiver is, uitspreken.
\par 4 De Geest Gods heeft mij gemaakt, en de adem des Almachtigen heeft mij levend gemaakt.
\par 5 Zo gij kunt, antwoord mij; schik u voor mijn aangezicht, stel u.
\par 6 Zie, ik ben Godes, gelijk gij; uit het leem ben ik ook afgesneden.
\par 7 Zie, mijn verschrikking zal u niet beroeren, en mijn hand zal over u niet zwaar zijn.
\par 8 Zeker, gij hebt gezegd voor mijn oren, en ik heb de stem der woorden gehoord;
\par 9 Ik ben rein, zonder overtreding; ik ben zuiver, en heb geen misdaad.
\par 10 Zie, Hij vindt oorzaken tegen mij, Hij houdt mij voor Zijn vijand.
\par 11 Hij legt mijn voeten in den stok; Hij neemt al mijn paden waar.
\par 12 Zie, hierin zijt gij niet rechtvaardig, antwoord ik u; want God is meerder dan een mens.
\par 13 Waarom hebt gij tegen Hem getwist? Want Hij antwoordt niet van al Zijn daden.
\par 14 Maar God spreekt eens of tweemaal; doch men let niet daarop.
\par 15 In den droom, door het gezicht des nachts, als een diepe slaap op de lieden valt, in de sluimering op het leger;
\par 16 Dan openbaart Hij het voor het oor der lieden, en Hij verzegelt hun kastijding;
\par 17 Opdat Hij den mens afwende van zijn werk, en van den man de hovaardij verberge;
\par 18 Dat Hij zijn ziel van het verderf afhoude; en zijn leven, dat het door het zwaard niet doorga.
\par 19 Ook wordt hij gestraft met smart op zijn leger, en de sterke menigte zijner beenderen;
\par 20 Zodat zijn leven het brood zelf verfoeit, en zijn ziel de begeerlijke spijze;
\par 21 Dat zijn vlees verdwijnt uit het gezicht, en zijn beenderen, die niet gezien werden, uitsteken;
\par 22 En zijn ziel nadert ten verderve, en zijn leven tot de dingen, die doden.
\par 23 Is er dan bij Hem een Gezant, een Uitlegger, een uit duizend, om den mens zijn rechten plicht te verkondigen;
\par 24 Zo zal Hij hem genadig zijn, en zeggen: Verlos hem, dat hij in het verderf niet nederdale, Ik heb verzoening gevonden.
\par 25 Zijn vlees zal frisser worden dan het was in de jeugd; hij zal tot de dagen zijner jonkheid wederkeren.
\par 26 Hij zal tot God ernstiglijk bidden, Die in hem een welbehagen nemen zal, en zijn aangezicht met gejuich aanzien; want Hij zal den mens zijn gerechtigheid wedergeven.
\par 27 Hij zal de mensen aanschouwen, en zeggen: Ik heb gezondigd, en het recht verkeerd, hetwelk mij niet heeft gebaat;
\par 28 Maar God heeft mijn ziel verlost, dat zij niet voere in het verderf, zodat mijn leven het licht aanziet.
\par 29 Zie, dit alles werkt God twee maal of driemaal met een man;
\par 30 Opdat hij zijn ziel afkere van het verderf, en hij verlicht worde met het licht der levenden.
\par 31 Merk op, o Job! Hoor naar mij; zwijg, en ik zal spreken.
\par 32 Zo er redenen zijn, antwoord mij; spreek, want ik heb lust u te rechtvaardigen.
\par 33 Zo niet, hoor naar mij; zwijg, en ik zal u wijsheid leren.

\chapter{34}

\par 1 Verder antwoordde Elihu, en zeide:
\par 2 Hoort, gij wijzen, mijn woorden, en gij verstandigen, neigt de oren naar mij.
\par 3 Want het oor proeft de woorden, gelijk het gehemelte de spijze smaakt.
\par 4 Laat ons kiezen voor ons, wat recht is; laat ons kennen onder ons wat goed is.
\par 5 Want Job heeft gezegd: Ik ben rechtvaardig, en God heeft mijn recht weggenomen.
\par 6 Ik moet liegen in mijn recht; mijn pijl is smartelijk zonder overtreding.
\par 7 Wat man is er, gelijk Job? Hij drinkt de bespotting in als water;
\par 8 En gaat over weg in gezelschap met de werkers der ongerechtigheid, en wandelt met goddeloze lieden.
\par 9 Want hij heeft gezegd: Het baat een man niet, als hij welbehagen heeft aan God.
\par 10 Daarom, gij, lieden van verstand, hoort naar mij: Verre zij God van goddeloosheid, en de Almachtige van onrecht!
\par 11 Want naar het werk des mensen vergeldt Hij hem, en naar eens ieders weg doet Hij het hem vinden.
\par 12 Ook waarlijk, God handelt niet goddelooslijk, en de Almachtige verkeert het recht niet.
\par 13 Wie heeft Hem gesteld over de aarde, en wie heeft de ganse wereld geschikt?
\par 14 Indien Hij Zijn hart tegen hem zette, Zijn geest en Zijn adem zou Hij tot Zich vergaderen;
\par 15 Alle vlees zou tegelijk den geest geven, en de mens zou tot stof wederkeren.
\par 16 Zo er dan verstand bij u is, hoor dit; neig de oren tot de stem mijner woorden.
\par 17 Zou Hij ook, Die het recht haat, den gewonde verbinden, en zoudt gij den zeer Rechtvaardige verdoemen?
\par 18 Zou men tot een koning zeggen: Gij Belial; tot de prinsen: Gij goddelozen!
\par 19 Hoe dan tot Dien, Die het aangezicht der vorsten niet aanneemt, en den rijke voor den arme niet kent? Want zij zijn allen Zijner handen werk.
\par 20 In een ogenblik sterven zij; zelfs ter middernacht wordt een volk geschud, dat het doorga; en de machtige wordt weggenomen zonder hand.
\par 21 Want Zijn ogen zijn op ieders wegen, en Hij ziet al zijn treden.
\par 22 Er is geen duisternis, en er is geen schaduw des doods, dat aldaar de werkers der ongerechtigheid zich verbergen mochten.
\par 23 Gewisselijk, Hij legt den mens niet te veel op, dat hij tegen God in het gericht zou mogen treden.
\par 24 Hij vermorzelt de geweldigen, dat men het niet doorzoeken kan, en stelt anderen in hun plaats.
\par 25 Daarom dat Hij hun werken kent, zo keert Hij hen des nachts om, en zij worden verbrijzeld.
\par 26 Hij klopt hen samen als goddelozen, in een plaats, waar aanschouwers zijn;
\par 27 Daarom dat zij van achter Hem afgeweken zijn, en geen Zijner wegen verstaan hebben;
\par 28 Opdat Hij op hem het geroep des armen brenge, en het geroep der ellendigen verhore.
\par 29 Als Hij stilt, wie zal dan beroeren? Als Hij het aangezicht verbergt, wie zal Hem dan aanschouwen, zowel voor een volk, als voor een mens alleen?
\par 30 Opdat de huichelachtige mens niet meer regere, en geen strikken des volks zijn.
\par 31 Zekerlijk heeft hij tot God gezegd: Ik heb Uw straf verdragen, ik zal het niet verderven.
\par 32 Behalve wat ik zie, leer Gij mij; heb ik onrecht gewrocht, ik zal het niet meer doen.
\par 33 Zal het van u zijn, hoe Hij iets vergelden zal, dewijl gij Hem versmaadt? Zoudt gij dan verkiezen, en niet ik? Wat weet gij dan? Spreek.
\par 34 De lieden van verstand zullen met mij zeggen, en een wijs man zal naar mij horen:
\par 35 Dat Job niet met wetenschap gesproken heeft, en zijn woorden niet met kloek verstand geweest zijn.
\par 36 Mijn Vader, laat Job beproefd worden tot het einde toe, om zijner antwoorden wil onder de ongerechtige lieden.
\par 37 Want tot zijn zonde zou hij nog overtreding bijvoegen; hij zou onder ons in de handen klappen, en hij zou zijn redenen vermenigvuldigen tegen God.

\chapter{35}

\par 1 Elihu antwoordde verder, en zeide:
\par 2 Houdt gij dat voor recht, dat gij gezegd hebt: Mijn gerechtigheid is meerder dan Gods?
\par 3 Want gij hebt gezegd: Wat zou zij u baten? Wat meer voordeel zal ik daarmede doen, dan met mijn zonde?
\par 4 Ik zal u antwoord geven, en uw vrienden met u.
\par 5 Bemerk den hemel en zie; en aanschouw de bovenste wolken, zij zijn hoger dan gij.
\par 6 Indien gij zondigt, wat bedrijft gij tegen Hem? Indien uw overtredingen menigvuldig zijn, wat doet gij Hem?
\par 7 Indien gij rechtvaardig zijt, wat geeft gij Hem, of wat ontvangt Hij uit uw hand?
\par 8 Uw goddeloosheid zou zijn tegen een man, gelijk gij zijt, en uw gerechtigheid voor eens mensen kind.
\par 9 Vanwege hun grootheid doen zij de onderdrukten roepen; zij schreeuwen vanwege den arm der groten.
\par 10 Maar niemand zegt: Waar is God, mijn Maker, die de psalmen geeft in den nacht?
\par 11 Die ons geleerder maakt dan de beesten der aarde, en ons wijzer maakt dan het gevogelte des hemels?
\par 12 Daar roepen zij; maar Hij antwoordt niet, vanwege den hoogmoed der bozen.
\par 13 Gewisselijk zal God de ijdelheid niet verhoren, en de Almachtige zal die niet aanschouwen.
\par 14 Dat gij ook gezegd hebt: Gij zult Hem niet aanschouwen; er is nochtans gericht voor Zijn aangezicht, wacht gij dan op Hem.
\par 15 Maar nu, dewijl het niets is, dat Zijn toorn Job bezocht heeft, en Hij hem niet zeer in overvloed doorkend heeft;
\par 16 Zo heeft Job in ijdelheid zijn mond geopend, en zonder wetenschap woorden vermenigvuldigd.

\chapter{36}

\par 1 Elihu ging nog voort, en zeide:
\par 2 Verbeid mij een weinig, en ik zal u aanwijzen, dat er nog redenen voor God zijn.
\par 3 Ik zal mijn gevoelen van verre ophalen, en mijn Schepper gerechtigheid toewijzen.
\par 4 Want voorwaar, mijn woorden zullen geen valsheid zijn; een, die oprecht is van gevoelen, is bij u.
\par 5 Zie, God is geweldig, nochtans versmaadt Hij niet; geweldig is Hij in kracht des harten.
\par 6 Hij laat den goddeloze niet leven, en het recht der ellendigen beschikt Hij.
\par 7 Hij onttrekt Zijn ogen niet van den rechtvaardige, maar met de koningen zijn zij in den troon; daar zet Hij hen voor altoos, en zij worden verheven.
\par 8 En zo zij, gebonden zijnde in boeien, vast gehouden worden met banden der ellende;
\par 9 Dan geeft Hij hun hun werk te kennen, en hun overtredingen, omdat zij de overhand genomen hebben;
\par 10 En Hij openbaart het voor hunlieder oor ter tucht, en zegt, dat zij zich van de ongerechtigheid bekeren zouden.
\par 11 Indien zij horen, en Hem dienen, zo zullen zij hun dagen eindigen in het goede, en hun jaren in liefelijkheden.
\par 12 Maar zo zij niet horen, zo gaan zij door het zwaard door, en zij geven den geest zonder kennis.
\par 13 En die met het hart huichelachtig zijn, leggen toorn op; zij roepen niet, als Hij hen gebonden heeft.
\par 14 Hun ziel zal in de jonkheid sterven, en hun leven onder de schandjongens.
\par 15 Hij zal den ellendige in zijn ellende vrijmaken, en in de onderdrukking zal Hij het voor hunlieder oor openbaren.
\par 16 Alzo zou Hij ook u afgekeerd hebben van den mond des angstes tot de ruimte, onder dewelke geen benauwing zou geweest zijn; en het gerecht uwer tafel zou vol vettigheid geweest zijn.
\par 17 Maar gij hebt het gericht des goddelozen vervuld; het gericht en het recht houden u vast.
\par 18 Omdat er grimmigheid is, wacht u, dat Hij u misschien niet met een klop wegstote; zodat u een groot rantsoen er niet zou afbrengen.
\par 19 Zou Hij uw rijkdom achten, dat gij niet in benauwdheid zoudt zijn; of enige versterkingen van kracht?
\par 20 Haak niet naar dien nacht, als de volken van hun plaats opgenomen worden.
\par 21 Wacht u, wend u niet tot ongerechtigheid; overmits gij ze in dezen verkoren hebt, uit oorzake van de ellende.
\par 22 Zie, God verhoogt door Zijn kracht; wie is een Leraar, gelijk Hij?
\par 23 Wie heeft Hem gesteld over Zijn weg? Of wie heeft gezegd: Gij hebt onrecht gedaan?
\par 24 Gedenk, dat gij Zijn werk groot maakt, hetwelk de lieden aanschouwen.
\par 25 Alle mensen zien het aan; de mens schouwt het van verre.
\par 26 Zie, God is groot, en wij begrijpen het niet; er is ook geen onderzoeking van het getal Zijner jaren.
\par 27 Want Hij trekt de druppelen der wateren op, die den regen na zijn damp uitgieten;
\par 28 Welke de wolken uitgieten, en over den mens overvloediglijk afdruipen.
\par 29 Kan men ook verstaan de uitbreidingen der wolken, en de krakingen Zijner hutte?
\par 30 Zie, Hij breidt over hem Zijn licht uit, en de wortelen der zee bedekt Hij.
\par 31 Want daardoor richt Hij de volken; Hij geeft spijze ten overvloede.
\par 32 Met handen bedekt Hij het licht, en doet aan hetzelve verbod door dengene, die tussen doorkomt.
\par 33 Daarvan verkondigt Zijn geklater, en het vee; ook van den opgaanden damp.

\chapter{37}

\par 1 Ook beeft hierover mijn hart, en springt op uit zijn plaats.
\par 2 Hoort met aandacht de beweging Zijner stem, en het geluid, dat uit Zijn mond uitgaat!
\par 3 Dat zendt Hij rechtuit onder den gansen hemel, en Zijn licht over de einden der aarde.
\par 4 Daarna brult Hij met de stem; Hij dondert met de stem Zijner hoogheid, en vertrekt die dingen niet, als Zijn stem zal gehoord worden.
\par 5 God dondert met Zijn stem zeer wonderlijk; Hij doet grote dingen, en wij begrijpen ze niet.
\par 6 Want Hij zegt tot de sneeuw: Wees op de aarde; en tot den plasregen des regens; dan is er de plasregen Zijner sterke regenen.
\par 7 Dan zegelt Hij de hand van ieder mens toe, opdat Hij kenne al de lieden Zijns werks.
\par 8 En het gedierte gaat in de loerplaatsen, en blijft in zijn holen.
\par 9 Uit de binnenkamer komt de wervelwind, en van de verstrooiende winden de koude.
\par 10 Door zijn geblaas geeft God de vorst, zodat de brede wateren verstijfd worden.
\par 11 Ook vermoeit Hij de dikke wolken door klaarheid; Hij verstrooit de wolk Zijns lichts.
\par 12 Die keert zich dan naar Zijn wijzen raad door ommegangen, dat zij doen al wat Hij ze gebiedt, op het vlakke der wereld, op de aarde.
\par 13 Hetzij dat Hij die tot een roede, of tot Zijn land, of tot weldadigheid beschikt.
\par 14 Neem dit, o Job, ter ore; sta, en aanmerk de wonderen Gods.
\par 15 Weet gij, wanneer God over dezelve orde stelt, en het licht Zijner wolk laat schijnen?
\par 16 Hebt gij wetenschap van de opwegingen der dikke wolken; de wonderheden Desgenen, Die volmaakt is in wetenschappen?
\par 17 Hoe uw klederen warm worden, als Hij de aarde stil maakt uit het zuiden?
\par 18 Hebt gij met Hem de hemelen uitgespannen, die vast zijn, als een gegoten spiegel?
\par 19 Onderricht ons, wat wij Hem zeggen zullen; want wij zullen niets ordentelijk voorstellen kunnen vanwege de duisternis.
\par 20 Zal het Hem verteld worden, als ik zo zou spreken? Denkt iemand dat, gewisselijk, hij zal verslonden worden.
\par 21 En nu ziet men het licht niet als het helder is in den hemel, als de wind doorgaat, en dien zuivert;
\par 22 Als van het noorden het goud komt; maar bij God is een vreselijke majesteit!
\par 23 Den Almachtige, Dien kunnen wij niet uitvinden; Hij is groot van kracht; doch door gericht en grote gerechtigheid verdrukt Hij niet.
\par 24 Daarom vrezen Hem de lieden; Hij ziet geen wijzen van harte aan.

\chapter{38}

\par 1 Daarna antwoordde de HEERE Job uit een onweder, en zeide:
\par 2 Wie is hij, die den raad verduistert met woorden zonder wetenschap?
\par 3 Gord nu, als een man, uw lenden, zo zal Ik u vragen, en onderricht Mij.
\par 4 Waar waart gij, toen Ik de aarde grondde? Geef het te kennen, indien gij kloek van verstand zijt.
\par 5 Wie heeft haar maten gezet, want gij weet het; of wie heeft over haar een richtsnoer getrokken?
\par 6 Waarop zijn haar grondvesten nedergezonken, of wie heeft haar hoeksteen gelegd?
\par 7 Toen de morgensterren te zamen vrolijk zongen, en al de kinderen Gods juichten.
\par 8 Of wie heeft de zee met deuren toegesloten, toen zij uitbrak, en uit de baarmoeder voortkwam?
\par 9 Toen Ik de wolk tot haar kleding stelde, en de donkerheid tot haar windeldoek;
\par 10 Toen Ik voor haar met Mijn besluit de aarde doorbrak, en zette grendel en deuren;
\par 11 En zeide: Tot hiertoe zult gij komen, en niet verder, en hier zal hij zich stellen tegen den hoogmoed uwer golven.
\par 12 Hebt gij van uw dagen den morgenstond geboden? Hebt gij den dageraad zijn plaats gewezen;
\par 13 Opdat hij de einden der aarde vatten zou; en de goddelozen uit haar uitgeschud zouden worden?
\par 14 Dat zij veranderd zou worden gelijk zegelleem, en zij gesteld worden als een kleed?
\par 15 En dat van de goddelozen hun licht geweerd worde, en de hoge arm worde gebroken?
\par 16 Zijt gij gekomen tot aan de oorsprongen der zee, en hebt gij in het onderste des afgronds gewandeld?
\par 17 Zijn u de poorten des doods ontdekt, en hebt gij gezien de poorten van de schaduw des doods?
\par 18 Zijt gij met uw verstand gekomen tot aan de breedte der aarde? Geef het te kennen, indien gij dit alles weet.
\par 19 Waar is de weg, daar het licht woont? En de duisternis, waar is haar plaats?
\par 20 Dat gij dat brengen zoudt tot zijn pale, en dat gij merken zoudt de paden zijns huizes?
\par 21 Gij weet het, want gij waart toen geboren, en uw dagen zijn veel in getal.
\par 22 Zijt gij gekomen tot de schatkameren der sneeuw, en hebt gij de schatkameren des hagels gezien?
\par 23 Dien Ik ophoude tot den tijd der benauwdheid, tot den dag des strijds en des oorlogs!
\par 24 Waar is de weg, daar het licht verdeeld wordt, en de oostenwind zich verstrooit op de aarde?
\par 25 Wie deelt voor den stortregen een waterloop uit, en een weg voor het weerlicht der donderen?
\par 26 Om te regenen op het land, waar niemand is, op de woestijn, waarin geen mens is;
\par 27 Om het woeste en het verwoeste te verzadigen, en om het uitspruitsel der grasscheutjes te doen wassen.
\par 28 Heeft de regen een vader, of wie baart de druppelen des dauws?
\par 29 Uit wiens buik komt het ijs voort, en wie baart den rijm des hemels?
\par 30 Als met een steen verbergen zich de wateren, en het vlakke des afgrond wordt omvat.
\par 31 Kunt gij de liefelijkheden van het Zevengesternte binden, of de strengen des Orions losmaken?
\par 32 Kunt gij de Mazzaroth voortbrengen op haar tijd, en den Wagen met zijn kinderen leiden?
\par 33 Weet gij de verordeningen des hemels, of kunt gij deszelfs heerschappij op de aarde bestellen?
\par 34 Kunt gij uw stem tot de wolken opheffen, opdat een overvloed van water u bedekke?
\par 35 Kunt gij de bliksemen uitlaten, dat zij henenvaren, en tot u zeggen: Zie, hier zijn wij?
\par 36 Wie heeft de wijsheid in het binnenste gezet? Of wie heeft den zin het verstand gegeven?
\par 37 Wie kan de wolken met wijsheid tellen, en wie kan de flessen des hemels nederleggen?
\par 38 Als het stof doorgoten is tot vastigheid, en de kluiten samenkleven?

\chapter{39}

\par 1 Zult gij voor den ouden leeuw roof jagen, of de graagheid der jonge leeuwen vervullen?
\par 2 Als zij nederbukken in de holen, en in den kuil zitten, ter loering?
\par 3 Wie bereidt de raaf haar kost, als haar jongen tot God schreeuwen, als zij dwalen, omdat er geen eten is?
\par 4 Weet gij den tijd van het baren der steengeiten? Hebt gij waargenomen den arbeid der hinden?
\par 5 Zult gij de maanden tellen, die zij vervullen, en weet gij den tijd van haar baren?
\par 6 Als zij zich krommen, haar jongen met versplijting voortbrengen, haar smarten uitwerpen?
\par 7 Haar jongen worden kloek, worden groot door het koren; zij gaan uit, en keren niet weder tot dezelve.
\par 8 Wie heeft den woudezel vrij henengezonden, en wie heeft de banden des wilden ezels gelost?
\par 9 Dien Ik de wildernis tot zijn huis besteld heb, en het ziltige tot zijn woningen.
\par 10 Hij belacht het gewoel der stad; het menigerlei getier des drijvers hoort hij niet.
\par 11 Dat hij uitspeurt op de bergen, is zijn weide; en hij zoekt allerlei groensel na.
\par 12 Zal de eenhoorn u willen dienen? Zal hij vernachten aan uw kribbe?
\par 13 Zult gij den eenhoorn met zijn touw aan de voren binden? Zal hij de laagten achter u eggen?
\par 14 Zult gij op hem vertrouwen, omdat zijn kracht groot is, en zult gij uw arbeid op hem laten?
\par 15 Zult gij hem geloven, dat hij uw zaad zal wederbrengen, en vergaderen tot uw dorsvloer?
\par 16 Zijn van u de verheugelijke vleugelen der pauwen? Of de vederen des ooievaars, en des struisvogels?
\par 17 Dat zij haar eieren in de aarde laat, en in het stof die verwarmt.
\par 18 En vergeet, dat de voet die drukken kan, en de dieren des velds die vertrappen kunnen?
\par 19 Zij verhardt zich tegen haar jongen, alsof zij de hare niet waren; haar arbeid is te vergeefs, omdat zij zonder vreze is.
\par 20 Want God heeft haar van wijsheid ontbloot, en heeft haar des verstands niet medegedeeld.
\par 21 Als het tijd is, verheft zij zich in de hoogte; zij belacht het paard en zijn rijder.
\par 22 Zult gij het paard sterkte geven? Kunt gij zijn hals met donder bekleden?
\par 23 Zult gij het beroeren als een sprinkhaan? De pracht van zijn gesnuif is een verschrikking.
\par 24 Het graaft in den grond, en het is vrolijk in zijn kracht; en trekt uit, den geharnaste tegemoet.
\par 25 Het belacht de vreze, en wordt niet ontsteld, en keert niet wederom vanwege het zwaard.
\par 26 Tegen hem ratelt de pijlkoker, het vlammig ijzer des spies en der lans.
\par 27 Met schudding en beroering slokt het de aarde op, en gelooft niet, dat het is het geluid der bazuin.
\par 28 In het volle geklank der bazuin, zegt het: Heah! en ruikt den krijg van verre, den donder der vorsten en het gejuich.
\par 29 Vliegt de sperwer door uw verstand, en breidt hij zijn vleugelen uit naar het zuiden?
\par 30 Is het naar uw bevel, dat de arend zich omhoog verheft, en dat hij zijn nest in de hoogte maakt?
\par 31 Hij woont en vernacht in de steenrots, op de scherpte der steenrots en der vaste plaats.
\par 32 Van daar speurt hij de spijze op; zijn ogen zien van verre af.
\par 33 Ook zuipen zijn jongen bloed; en waar verslagenen zijn, daar is hij.
\par 34 En de HEERE antwoordde Job, en zeide:
\par 35 Is het twisten met den Almachtige onderrichten? Wie God bestraft, die antwoorde daarop.
\par 36 Toen antwoordde Job den HEERE, en zeide:
\par 37 Zie, ik ben te gering; wat zou ik U antwoorden? Ik leg mijn hand op mijn mond.
\par 38 Eenmaal heb ik gesproken, maar zal niet antwoorden; of tweemaal, maar zal niet voortvaren.

\chapter{40}

\par 1 En de HEERE antwoordde Job uit een onweder, en zeide:
\par 2 Gord nu als een man uw lenden; Ik zal u vragen, en onderricht Mij.
\par 3 Zult gij ook Mijn oordeel te niet maken? Zult Gij Mij verdoemen, opdat gij rechtvaardig zijt?
\par 4 Hebt gij een arm gelijk God? En kunt gij, gelijk Hij, met de stem donderen?
\par 5 Versier u nu met voortreffelijkheid en hoogheid, en bekleed u met majesteit en heerlijkheid!
\par 6 Strooi de verbolgenheden uws toorns uit, en zie allen hoogmoedige, en verneder hem!
\par 7 Zie allen hoogmoedige, en breng hem ten onder; en verpletter de goddelozen in hun plaats!
\par 8 Verberg hen te zamen in het stof; verbind hun aangezichten in het verborgen!
\par 9 Dan zal Ik ook u loven, omdat uw rechterhand u zal verlost hebben.
\par 10 Zie nu Behemoth, welken Ik gemaakt heb nevens u; hij eet hooi, gelijk een rund.
\par 11 Zie toch, zijn kracht is in zijn lenden, en zijn macht in den navel zijns buiks.
\par 12 Als het hem lust, zijn staart is als een ceder; de zenuwen zijner schaamte zijn doorvlochten.
\par 13 Zijn beenderen zijn als vast koper; zijn gebeenten zijn als ijzeren handbomen.
\par 14 Hij is een hoofdstuk der wegen Gods; die hem gemaakt heeft, heeft hem zijn zwaard aangehecht.
\par 15 Omdat de bergen hem voeder voortbrengen, daarom spelen al de dieren des velds aldaar.
\par 16 Onder schaduwachtige bomen ligt hij neder, in een schuilplaats des riets en des slijks.
\par 17 De schaduwachtige bomen bedekken hem, elkeen met zijn schaduw; de beekwilgen omringen hem.
\par 18 Zie, hij doet de rivier geweld aan, en verhaast zich niet; hij vertrouwt, dat hij de Jordaan in zijn mond zou kunnen intrekken.
\par 19 Zou men hem voor zijn ogen kunnen vangen? Zou men hem met strikken den neus doorboren kunnen?
\par 20 Zult gij den Leviathan met den angel trekken, of zijn tong met een koord, dat gij laat nederzinken?
\par 21 Zult gij hem een bieze in den neus leggen, of met een doorn zijn kaak doorboren?
\par 22 Zal hij aan u veel smekingen maken? Zal hij zachtjes tot u spreken?
\par 23 Zal hij een verbond met u maken? Zult gij hem aannemen tot een eeuwigen slaaf?
\par 24 Zult gij met hem spelen gelijk met een vogeltje, of zult gij hem binden voor uw jonge dochters?
\par 25 Zullen de metgezellen over hem een maaltijd bereiden? Zullen zij hem delen onder de kooplieden?
\par 26 Zult gij zijn huis met haken vullen, of met een visserskrauwel zijn hoofd?
\par 27 Leg uw hand op hem, gedenk des strijds, doe het niet meer.
\par 28 Zie, zijn hoop zal feilen; zal hij ook voor zijn gezicht nedergeslagen worden?

\chapter{41}

\par 1 Niemand is zo koen, dat hij hem opwekken zou; wie is dan hij, die zich voor Mijn aangezicht stellen zou?
\par 2 Wie heeft Mij voorgekomen, dat Ik hem zou vergelden? Wat onder den gansen hemel is, is het Mijne.
\par 3 Ik zal zijn leden niet verzwijgen, noch het verhaal zijner sterkte, noch de bevalligheid zijner gestaltenis.
\par 4 Wie zou het opperste zijns kleeds ontdekken? Wie zou met zijn dubbelen breidel hem aankomen?
\par 5 Wie zou de deuren zijns aangezichts opendoen? Rondom zijn tanden is verschrikking.
\par 6 Zeer uitnemend zijn zijn sterke schilden, elkeen gesloten als met een nauwdrukkend zegel.
\par 7 Het een is zo na aan het andere, dat de wind daar niet kan tussen komen.
\par 8 Zij kleven aan elkander, zij vatten zich samen, dat zij zich niet scheiden.
\par 9 Elk een zijner niezingen doet een licht schijnen; en zijn ogen zijn als de oogleden des dageraads.
\par 10 Uit zijn mond gaan fakkelen, vurige vonken raken er uit.
\par 11 Uit zijn neusgaten komt rook voort, als uit een ziedenden pot en ruimen ketel.
\par 12 Zijn adem zou kolen doen vlammen, en een vlam komt uit zijn mond voort.
\par 13 In zijn hals herbergt de sterkte; voor hem springt zelfs de droefheid van vreugde op.
\par 14 De stukken van zijn vlees kleven samen; elkeen is vast in hem, het wordt niet bewogen.
\par 15 Zijn hart is vast gelijk een steen; ja, vast gelijk een deel van den ondersten molensteen.
\par 16 Van zijn verheffen schromen de sterken; om zijner doorbrekingen wille ontzondigen zij zich.
\par 17 Raakt hem iemand met het zwaard, dat zal niet bestaan, spies, schicht noch pantsier.
\par 18 Hij acht het ijzer voor stro, en het staal voor verrot hout.
\par 19 De pijl zal hem niet doen vlieden, de slingerstenen worden hem in stoppelen veranderd.
\par 20 De werpstenen worden van hem geacht als stoppelen, en hij belacht de drilling der lans.
\par 21 Onder hem zijn scherpe scherven; hij spreidt zich op het puntachtige, als op slijk.
\par 22 Hij doet de diepte zieden gelijk een pot; hij stelt de zee als een apothekerskokerij.
\par 23 Achter zich verlicht hij het pad; men zou den afgrond voor grijzigheid houden.
\par 24 Op de aarde is niets met hem te vergelijken, die gemaakt is om zonder schrik te wezen.
\par 25 Hij aanziet alles, wat hoog is, hij is een koning over alle jonge hoogmoedige dieren.

\chapter{42}

\par 1 Toen antwoordde Job den HEERE, en zeide:
\par 2 Ik weet, dat Gij alles vermoogt, en dat geen van Uw gedachten kan afgesneden worden.
\par 3 Wie is hij, zegt Gij, die den raad verbergt zonder wetenschap? Zo heb ik dan verhaald, hetgeen ik niet verstond, dingen, die voor mij te wonderbaar waren, die ik niet wist.
\par 4 Hoor toch, en ik zal spreken; ik zal U vragen, en onderricht Gij mij.
\par 5 Met het gehoor des oors heb ik U gehoord; maar nu ziet U mijn oog.
\par 6 Daarom verfoei ik mij, en ik heb berouw in stof en as.
\par 7 Het geschiedde nu, nadat de HEERE die woorden tot Job gesproken had, dat de HEERE tot Elifaz, den Themaniet, zeide: Mijn toorn is ontstoken tegen u, en tegen uw twee vrienden, want gijlieden hebt niet recht van Mij gesproken, gelijk Mijn knecht Job.
\par 8 Daarom neemt nu voor ulieden zeven varren en zeven rammen, en gaat henen tot Mijn knecht Job, en offert brandoffer voor ulieden, en laat Mijn knecht Job voor ulieden bidden; want zekerlijk, Ik zal zijn aangezicht aannemen, opdat Ik aan ulieden niet doe naar uw dwaasheid; want gijlieden hebt niet recht van Mij gesproken, gelijk Mijn knecht Job.
\par 9 Toen gingen Elifaz, de Themaniet, en Bildad, de Suhiet, en Zofar, de Naamathiet, henen, en deden, gelijk als de HEERE tot hen gesproken had; en de HEERE nam het aangezicht van Job aan.
\par 10 En de HEERE wendde de gevangenis van Job, toen hij gebeden had voor zijn vrienden; en de HEERE vermeerderde al hetgeen Job gehad had tot dubbel zoveel.
\par 11 Ook kwamen tot hem al zijn broeders, en al zijn zusters, en allen, die hem te voren gekend hadden, en aten brood met hem in zijn huis, en beklaagden hem, en vertroostten hem over al het kwaad, dat de HEERE over hem gebracht had; en zij gaven hem een iegelijk een stuk gelds, een iegelijk ook een gouden voorhoofdsiersel.
\par 12 En de HEERE zegende Jobs laatste meer dan zijn eerste; want hij had veertien duizend schapen, en zes duizend kemelen, en duizend juk runderen, en duizend ezelinnen.
\par 13 Daartoe had hij zeven zonen en drie dochteren.
\par 14 En hij noemde den naam der eerste Jemima, en den naam der tweede Kezia, en den naam der derde Keren-happuch.
\par 15 En er werden zo schone vrouwen niet gevonden in het ganse land, als de dochteren van Job; en haar vader gaf haar erfdeel onder haar broederen.
\par 16 En Job leefde na dezen honderd en veertig jaren, dat hij zag zijn kinderen, en de kinderen zijner kinderen, tot in vier geslachten.
\par 17 En Job stierf, oud en der dagen zat.



\end{document}