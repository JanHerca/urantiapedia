\begin{document}

\title{Genesis}




\chapter{1}

\par 1 In den beginne schiep God den hemel en de aarde.
\par 2 De aarde nu was woest en ledig, en duisternis was op den afgrond; en de Geest Gods zweefde op de wateren.
\par 3 En God zeide: Daar zij licht! en daar werd licht.
\par 4 En God zag het licht, dat het goed was; en God maakte scheiding tussen het licht en tussen de duisternis.
\par 5 En God noemde het licht dag, en de duisternis noemde Hij nacht. Toen was het avond geweest, en het was morgen geweest, de eerste dag.
\par 6 En God zeide: Daar zij een uitspansel in het midden der wateren; en dat make scheiding tussen wateren en wateren!
\par 7 En God maakte dat uitspansel, en maakte scheiding tussen de wateren, die onder het uitspansel zijn, en tussen de wateren, die boven het uitspansel zijn. En het was alzo.
\par 8 En God noemde het uitspansel hemel. Toen was het avond geweest, en het was morgen geweest, de tweede dag.
\par 9 En God zeide: Dat de wateren van onder den hemel in een plaats vergaderd worden, en dat het droge gezien worde! En het was alzo.
\par 10 En God noemde het droge aarde, en de vergadering der wateren noemde Hij zeeen; en God zag, dat het goed was.
\par 11 En God zeide: Dat de aarde uitschiete grasscheutjes, kruid zaadzaaiende, vruchtbaar geboomte, dragende vrucht naar zijn aard, welks zaad daarin zij op de aarde! En het was alzo.
\par 12 En de aarde bracht voort grasscheutjes, kruid zaadzaaiende naar zijn aard, en vruchtdragend geboomte, welks zaad daarin was, naar zijn aard. En God zag, dat het goed was.
\par 13 Toen was het avond geweest, en het was morgen geweest, de derde dag.
\par 14 En God zeide: Dat er lichten zijn in het uitspansel des hemels, om scheiding te maken tussen den dag en tussen den nacht; en dat zij zijn tot tekenen en tot gezette tijden, en tot dagen en jaren!
\par 15 En dat zij zijn tot lichten in het uitspansel des hemels, om licht te geven op de aarde! En het was alzo.
\par 16 God dan maakte die twee grote lichten; dat grote licht tot heerschappij des daags, en dat kleine licht tot heerschappij des nachts; ook de sterren.
\par 17 En God stelde ze in het uitspansel des hemels, om licht te geven op de aarde.
\par 18 En om te heersen op den dag, en in den nacht, en om scheiding te maken tussen het licht en tussen de duisternis. En God zag, dat het goed was.
\par 19 Toen was het avond geweest, en het was morgen geweest, de vierde dag.
\par 20 En God zeide: Dat de wateren overvloediglijk voortbrengen een gewemel van levende zielen; en het gevogelte vliege boven de aarde, in het uitspansel des hemels!
\par 21 En God schiep de grote walvissen, en alle levende wremelende ziel, welke de wateren overvloediglijk voortbrachten, naar haar aard; en alle gevleugeld gevogelte naar zijn aard. En God zag, dat het goed was.
\par 22 En God zegende ze, zeggende: Zijt vruchtbaar, en vermenigvuldigt, en vervult de wateren in de zeeen; en het gevogelte vermenigvuldige op de aarde!
\par 23 Toen was het avond geweest, en het was morgen geweest, de vijfde dag.
\par 24 En God zeide: De aarde brenge levende zielen voort, naar haar aard, vee, en kruipend, en wild gedierte der aarde, naar zijn aard! En het was alzo.
\par 25 En God maakte het wild gedierte der aarde naar zijn aard, en het vee naar zijn aard, en al het kruipend gedierte des aardbodems naar zijn aard. En God zag, dat het goed was.
\par 26 En God zeide: Laat Ons mensen maken, naar Ons beeld, naar Onze gelijkenis; en dat zij heerschappij hebben over de vissen der zee, en over het gevogelte des hemels, en over het vee, en over de gehele aarde, en over al het kruipend gedierte, dat op de aarde kruipt.
\par 27 En God schiep den mens naar Zijn beeld; naar het beeld van God schiep Hij hem; man en vrouw schiep Hij ze.
\par 28 En God zegende hen, en God zeide tot hen: Weest vruchtbaar, en vermenigvuldigt, en vervult de aarde, en onderwerpt haar, en hebt heerschappij over de vissen der zee, en over het gevogelte des hemels, en over al het gedierte, dat op de aarde kruipt!
\par 29 En God zeide: Ziet, Ik heb ulieden al het zaadzaaiende kruid gegeven, dat op de ganse aarde is, en alle geboomte, in hetwelk zaadzaaiende boomvrucht is; het zij u tot spijze!
\par 30 Maar aan al het gedierte der aarde, en aan al het gevogelte des hemels, en aan al het kruipende gedierte op de aarde, waarin een levende ziel is, heb Ik al het groene kruid tot spijze gegeven. En het was alzo.
\par 31 En God zag al wat Hij gemaakt had, en ziet, het was zeer goed. Toen was het avond geweest, en het was morgen geweest, de zesde dag.

\chapter{2}

\par 1 Alzo zijn volbracht de hemel en de aarde, en al hun heir.
\par 2 Als nu God op den zevenden dag volbracht had Zijn werk, dat Hij gemaakt had, heeft Hij gerust op den zevenden dag van al Zijn werk, dat Hij gemaakt had.
\par 3 En God heeft den zevenden dag gezegend, en dien geheiligd; omdat Hij op denzelven gerust heeft van al Zijn werk, hetwelk God geschapen had, om te volmaken.
\par 4 Dit zijn de geboorten des hemels en der aarde, als zij geschapen werden; ten dage als de HEERE God de aarde en den hemel maakte.
\par 5 En allen struik des velds, eer hij in de aarde was, en al het kruid des velds, eer het uitsproot; want de HEERE God had niet doen regenen op de aarde, en er was geen mens geweest, om den aardbodem te bouwen.
\par 6 Maar een damp was opgegaan uit de aarde, en bevochtigde den gansen aardbodem.
\par 7 En de HEERE God had den mens geformeerd uit het stof der aarde, en in zijn neusgaten geblazen den adem des levens; alzo werd de mens tot een levende ziel.
\par 8 Ook had de HEERE God een hof geplant in Eden, tegen het oosten, en Hij stelde aldaar den mens, dien Hij geformeerd had.
\par 9 En de HEERE God had alle geboomte uit het aardrijk doen spruiten, begeerlijk voor het gezicht, en goed tot spijze; en den boom des levens in het midden van den hof, en de boom der kennis des goeds en des kwaads.
\par 10 En een rivier was voortgaande uit Eden, om dezen hof te bewateren; en werd van daar verdeeld, en werd tot vier hoofden.
\par 11 De naam der eerste rivier is Pison; deze is het, die het ganse land van Havila omloopt, waar het goud is.
\par 12 En het goud van dit land is goed; daar is ook bedolah, en de steen sardonix.
\par 13 En de naam der tweede rivier is Gihon; deze is het, die het ganse land Cusch omloopt.
\par 14 En de naam der derde rivier is Hiddekel; deze is gaande naar het oosten van Assur. En de vierde rivier is Frath.
\par 15 Zo nam de HEERE God den mens, en zette hem in den hof van Eden, om dien te bouwen, en dien te bewaren.
\par 16 En de HEERE God gebood den mens, zeggende: Van allen boom dezes hofs zult gij vrijelijk eten;
\par 17 Maar van den boom der kennis des goeds en des kwaads, daarvan zult gij niet eten; want ten dage, als gij daarvan eet, zult gij den dood sterven.
\par 18 Ook had de HEERE God gesproken: Het is niet goed, dat de mens alleen zij; Ik zal hem een hulpe maken, die als tegen hem over zij.
\par 19 Want als de HEERE God uit de aarde al het gedierte des velds, en al het gevogelte des hemels gemaakt had, zo bracht Hij die tot Adam, om te zien, hoe hij ze noemen zou; en zo als Adam alle levende ziel noemen zoude, dat zou haar naam zijn.
\par 20 Zo had Adam genoemd de namen van al het vee, en van het gevogelte des hemels, en van al het gedierte des velds; maar voor den mens vond hij geen hulpe, die als tegen hem over ware.
\par 21 Toen deed de HEERE God een diepen slaap op Adam vallen, en hij sliep; en Hij nam een van zijn ribben, en sloot derzelver plaats toe met vlees.
\par 22 En de HEERE God bouwde de ribbe, die Hij van Adam genomen had, tot een vrouw, en Hij bracht haar tot Adam.
\par 23 Toen zeide Adam: Deze is ditmaal been van mijn benen, en vlees van mijn vlees! Men zal haar Manninne heten, omdat zij uit den man genomen is.
\par 24 Daarom zal de man zijn vader en zijn moeder verlaten, en zijn vrouw aankleven; en zij zullen tot een vlees zijn.
\par 25 En zij waren beiden naakt, Adam en zijn vrouw; en zij schaamden zich niet.

\chapter{3}

\par 1 De slang nu was listiger dan al het gedierte des velds, hetwelk de HEERE God gemaakt had; en zij zeide tot de vrouw: Is het ook, dat God gezegd heeft: Gijlieden zult niet eten van allen boom dezes hofs?
\par 2 En de vrouw zeide tot de slang: Van de vrucht der bomen dezes hofs zullen wij eten;
\par 3 Maar van de vrucht des booms, die in het midden des hofs is, heeft God gezegd: Gij zult van die niet eten, noch die aanroeren, opdat gij niet sterft.
\par 4 Toen zeide de slang tot de vrouw: Gijlieden zult den dood niet sterven;
\par 5 Maar God weet, dat, ten dage als gij daarvan eet, zo zullen uw ogen geopend worden, en gij zult als God wezen, kennende het goed en het kwaad.
\par 6 En de vrouw zag, dat die boom goed was tot spijze, en dat hij een lust was voor de ogen, ja, een boom, die begeerlijk was om verstandig te maken; en zij nam van zijn vrucht en at; en zij gaf ook haar man met haar, en hij at.
\par 7 Toen werden hun beider ogen geopend, en zij werden gewaar, dat zij naakt waren; en zij hechtten vijgeboombladeren samen, en maakten zich schorten.
\par 8 En zij hoorden de stem van den HEERE God, wandelende in den hof, aan den wind des daags. Toen verborg zich Adam en zijn vrouw voor het aangezicht van den HEERE God, in het midden van het geboomte des hofs.
\par 9 En de HEERE God riep Adam, en zeide tot hem: Waar zijt gij?
\par 10 En hij zeide: Ik hoorde Uw stem in den hof, en ik vreesde; want ik ben naakt; daarom verborg ik mij.
\par 11 En Hij zeide: Wie heeft u te kennen gegeven, dat gij naakt zijt? Hebt gij van dien boom gegeten, van welken Ik u gebood, dat gij daarvan niet eten zoudt?
\par 12 Toen zeide Adam: De vrouw, die Gij bij mij gegeven hebt, die heeft mij van dien boom gegeven, en ik heb gegeten.
\par 13 En de HEERE God zeide tot de vrouw: Wat is dit, dat gij gedaan hebt? En de vrouw zeide: De slang heeft mij bedrogen, en ik heb gegeten.
\par 14 Toen zeide de HEERE God tot die slang: Dewijl gij dit gedaan hebt, zo zijt gij vervloekt boven al het vee, en boven al het gedierte des velds! Op uw buik zult gij gaan, en stof zult gij eten, al de dagen uws levens.
\par 15 En Ik zal vijandschap zetten tussen u en tussen deze vrouw, en tussen uw zaad en tussen haar zaad; datzelve zal u den kop vermorzelen, en gij zult het de verzenen vermorzelen.
\par 16 Tot de vrouw zeide Hij: Ik zal zeer vermenigvuldigen uw smart, namelijk uwer dracht; met smart zult gij kinderen baren; en tot uw man zal uw begeerte zijn, en hij zal over u heerschappij hebben.
\par 17 En tot Adam zeide Hij: Dewijl gij geluisterd hebt naar de stem uwer vrouw, en van dien boom gegeten, waarvan Ik u gebood, zeggende: Gij zult daarvan niet eten; zo zij het aardrijk om uwentwil vervloekt; en met smart zult gij daarvan eten al de dagen uws levens.
\par 18 Ook zal het u doornen en distelen voortbrengen, en gij zult het kruid des velds eten.
\par 19 In het zweet uws aanschijns zult gij brood eten, totdat gij tot de aarde wederkeert, dewijl gij daaruit genomen zijt; want gij zijt stof, en gij zult tot stof wederkeren.
\par 20 Voorts noemde Adam den naam zijner vrouw Heva, omdat zij een moeder aller levenden is.
\par 21 En de HEERE God maakte voor Adam en zijn vrouw rokken van vellen, en toog ze hun aan.
\par 22 Toen zeide de HEERE God: Ziet, de mens is geworden als Onzer een, kennende het goed en het kwaad! Nu dan, dat hij zijn hand niet uitsteke, en neme ook van den boom des levens, en ete, en leve in eeuwigheid.
\par 23 Zo verzond hem de HEERE God uit den hof van Eden, om den aardbodem te bouwen, waaruit hij genomen was.
\par 24 En Hij dreef den mens uit; en stelde cherubim tegen het oosten des hofs van Eden, en een vlammig lemmer eens zwaards, dat zich omkeerde, om te bewaren den weg van den boom des levens.

\chapter{4}

\par 1 En Adam bekende Heva, zijn huisvrouw, en zij werd zwanger, en baarde Kain, en zeide: Ik heb een man van den HEERE verkregen!
\par 2 En zij voer voort te baren zijn broeder Habel; en Habel werd een schaapherder, en Kain werd een landbouwer.
\par 3 En het geschiedde ten einde van enige dagen, dat Kain van de vrucht des lands den HEERE offer bracht.
\par 4 En Habel bracht ook van de eerstgeborenen zijner schapen, en van hun vet. En de HEERE zag Habel en zijn offer aan;
\par 5 Maar Kain en zijn offer zag Hij niet aan. Toen ontstak Kain zeer, en zijn aangezicht verviel.
\par 6 En de HEERE zeide tot Kain: Waarom zijt gij ontstoken, en waarom is uw aangezicht vervallen?
\par 7 Is er niet, indien gij weldoet, verhoging? en zo gij niet weldoet, de zonde ligt aan de deur. Zijn begeerte is toch tot u, en gij zult over hem heersen.
\par 8 En Kain sprak met zijn broeder Habel; en het geschiedde, als zij in het veld waren, dat Kain tegen zijn broeder Habel opstond, en sloeg hem dood.
\par 9 En de HEERE zeide tot Kain: Waar is Habel, uw broeder? En hij zeide: Ik weet het niet; ben ik mijns broeders hoeder?
\par 10 En Hij zeide: Wat hebt gij gedaan? daar is een stem des bloeds van uw broeder, dat tot Mij roept van den aardbodem.
\par 11 En nu zijt gij vervloekt van den aardbodem, die zijn mond heeft opengedaan, om uws broeders bloed van uw hand te ontvangen.
\par 12 Als gij den aardbodem bouwen zult, hij zal u zijn vermogen niet meer geven; gij zult zwervende en dolende zijn op aarde.
\par 13 En Kain zeide tot den HEERE: Mijn misdaad is groter, dan dat zij vergeven worde.
\par 14 Zie, Gij hebt mij heden verdreven van den aardbodem, en ik zal voor Uw aangezicht verborgen zijn; en ik zal zwervende en dolende zijn op de aarde, en het zal geschieden, dat al wie mij vindt, mij zal doodslaan.
\par 15 Doch de HEERE zeide tot hem: Daarom, al wie Kain doodslaat, zal zevenvoudig gewroken worden! En de HEERE stelde een teken aan Kain; opdat hem niet versloeg al wie hem vond.
\par 16 En Kain ging uit van het aangezicht des HEEREN; en hij woonde in het land Nod, ten oosten van Eden.
\par 17 En Kain bekende zijn huisvrouw, en zij werd bevrucht en baarde Henoch; en hij bouwde een stad, en noemde den naam dier stad naar den naam zijns zoons, Henoch.
\par 18 En aan Henoch werd Hirad geboren; en Hirad gewon Mechujael; en Mechujael gewon Methusael; en Methusael gewon Lamech.
\par 19 En Lamech nam zich twee vrouwen; de naam van de eerste was Ada, en de naam van de andere Zilla.
\par 20 En Ada baarde Jabal; deze is geweest een vader dergenen, die tenten bewoonden, en vee hadden.
\par 21 En de naam zijns broeders was Jubal; deze was de vader van allen, die harpen en orgelen handelen.
\par 22 En Zilla baarde ook Tubal-kain, een leermeester van allen werker in koper en ijzer; en de zuster van Tubal-kain was Naema.
\par 23 En Lamech zeide tot zijn vrouwen Ada en Zilla: Hoort mijn stem, gij vrouwen van Lamech! neemt ter ore mijn rede! Voorwaar, ik sloeg wel een man dood, om mijn wonde, en een jongeling, om mijn buile!
\par 24 Want Kain zal zevenvoudig gewroken worden, maar Lamech zeventigmaal zevenmaal.
\par 25 En Adam bekende wederom zijn huisvrouw, en zij baarde een zoon, en zij noemde zijn naam Seth; want God heeft mij, sprak zij, een ander zaad gezet voor Habel; want Kain heeft hem doodgeslagen.
\par 26 En denzelven Seth werd ook een zoon geboren, en hij noemde zijn naam Enos. Toen begon men den Naam des HEEREN aan te roepen.

\chapter{5}

\par 1 Dit is het boek van Adams geslacht. Ten dage als God den mens schiep, maakte Hij hem naar de gelijkenis Gods.
\par 2 Man en vrouw schiep Hij hen, en zegende ze, en noemde hun naam Mens, ten dage als zij geschapen werden.
\par 3 En Adam leefde honderd en dertig jaren, en gewon een zoon naar zijn gelijkenis, naar zijn evenbeeld, en noemde zijn naam Seth.
\par 4 En Adams dagen, nadat hij Seth gewonnen had, zijn geweest achthonderd jaren; en hij gewon zonen en dochteren.
\par 5 Zo waren al de dagen van Adam, die hij leefde, negenhonderd jaren, en dertig jaren; en hij stierf.
\par 6 En Seth leefde honderd en vijf jaren, en hij gewon Enos.
\par 7 En Seth leefde, nadat hij Enos gewonnen had, achthonderd en zeven jaren; en hij gewon zonen en dochteren.
\par 8 Zo waren al de dagen van Seth negenhonderd en twaalf jaren; en hij stierf.
\par 9 En Enos leefde negentig jaren, en hij gewon Kenan.
\par 10 En Enos leefde, nadat hij Kenan gewonnen had, achthonderd en vijftien jaren; en hij gewon zonen en dochteren.
\par 11 Zo waren al de dagen van Enos negenhonderd en vijf jaren; en hij stierf.
\par 12 En Kenan leefde zeventig jaren, en hij gewon Mahalal-el.
\par 13 En Kenan leefde, nadat hij Mahalal-el gewonnen had, achthonderd en veertig jaren; en hij gewon zonen en dochteren.
\par 14 Zo waren al de dagen van Kenan negenhonderd en tien jaren; en hij stierf.
\par 15 En Mahalal-el leefde vijf en zestig jaren, en hij gewon Jered.
\par 16 En Mahalal-el leefde, nadat hij Jered gewonnen had, achthonderd en dertig jaren; en hij gewon zonen en dochteren.
\par 17 Zo waren al de dagen van Mahalal-el achthonderd vijf en negentig jaren; en hij stierf.
\par 18 En Jered leefde honderd twee en zestig jaren, en hij gewon Henoch.
\par 19 En Jered leefde, nadat hij Henoch gewonnen had, achthonderd jaren; en hij gewon zonen en dochteren.
\par 20 Zo waren al de dagen van Jered negenhonderd twee en zestig jaren; en hij stierf.
\par 21 En Henoch leefde vijf en zestig jaren, en hij gewon Methusalach.
\par 22 En Henoch wandelde met God, nadat hij Methusalach gewonnen had, driehonderd jaren; en hij gewon zonen en dochteren.
\par 23 Zo waren al de dagen van Henoch driehonderd vijf en zestig jaren.
\par 24 Henoch dan wandelde met God; en hij was niet meer; want God nam hem weg.
\par 25 En Methusalach leefde honderd zeven en tachtig jaren, en hij gewon Lamech.
\par 26 En Methusalach leefde, nadat hij Lamech gewonnen had, zevenhonderd twee en tachtig jaren; en hij gewon zonen en dochteren.
\par 27 Zo waren al de dagen van Methusalach negenhonderd negen en zestig jaren; en hij stierf.
\par 28 En Lamech leefde honderd twee en tachtig jaren, en hij gewon een zoon.
\par 29 En hij noemde zijn naam Noach, zeggende: Deze zal ons troosten over ons werk, en over de smart onzer handen, vanwege het aardrijk, dat de HEERE vervloekt heeft!
\par 30 En Lamech leefde, nadat hij Noach gewonnen had, vijfhonderd vijf en negentig jaren; en hij gewon zonen en dochteren.
\par 31 Zo waren al de dagen van Lamech zevenhonderd zeven en zeventig jaren; en hij stierf.
\par 32 En Noach was vijfhonderd jaren oud; en Noach gewon Sem, Cham en Jafeth.

\chapter{6}

\par 1 En het geschiedde, als de mensen op den aardbodem begonnen te vermenigvuldigen, en hun dochters geboren werden,
\par 2 Dat Gods zonen de dochteren der mensen aanzagen, dat zij schoon waren, en zij namen zich vrouwen uit allen, die zij verkozen hadden.
\par 3 Toen zeide de HEERE: Mijn Geest zal niet in eeuwigheid twisten met den mens, dewijl hij ook vlees is; doch zijn dagen zullen zijn honderd en twintig jaren.
\par 4 In die dagen waren er reuzen op de aarde, en ook daarna, als Gods zonen tot de dochteren der mensen ingegaan waren, en zich kinderen gewonnen hadden; deze zijn de geweldigen, die van ouds geweest zijn, mannen van name.
\par 5 En de HEERE zag, dat de boosheid des mensen menigvuldig was op de aarde, en al het gedichtsel der gedachten zijns harten te allen dage alleenlijk boos was.
\par 6 Toen berouwde het den HEERE, dat Hij den mens op de aarde gemaakt had, en het smartte Hem aan Zijn hart.
\par 7 En de HEERE zeide: Ik zal den mens, dien Ik geschapen heb, verdelgen van den aardbodem, van den mens tot het vee, tot het kruipend gedierte, en tot het gevogelte des hemels toe; want het berouwt Mij, dat Ik hen gemaakt heb.
\par 8 Maar Noach vond genade in de ogen des HEEREN.
\par 9 Dit zijn de geboorten van Noach. Noach was een rechtvaardig, oprecht man in zijn geslachten. Noach wandelde met God.
\par 10 En Noach gewon drie zonen: Sem, Cham en Jafeth.
\par 11 Maar de aarde was verdorven voor Gods aangezicht; en de aarde was vervuld met wrevel.
\par 12 Toen zag God de aarde, en ziet, zij was verdorven; want al het vlees had zijn weg verdorven op de aarde.
\par 13 Daarom zeide God tot Noach: Het einde van alle vlees is voor Mijn aangezicht gekomen; want de aarde is door hen vervuld met wrevel; en zie, Ik zal hen met de aarde verderven.
\par 14 Maak u een ark van goferhout; met kameren zult gij deze ark maken; en gij zult die bepekken van binnen en van buiten met pek.
\par 15 En aldus is het, dat gij haar maken zult: driehonderd ellen zij de lengte der ark, vijftig ellen haar breedte, en dertig ellen haar hoogte.
\par 16 Gij zult een venster aan de ark maken, en zult haar volmaken tot een elle van boven; en de deur der ark zult gij in haar zijde zetten; gij zult ze met onderste, tweede en derde verdiepingen maken.
\par 17 Want Ik, zie, Ik breng een watervloed over de aarde, om alle vlees, waarin een geest des levens is, van onder den hemel te verderven; al wat op de aarde is, zal den geest geven.
\par 18 Maar met u zal Ik Mijn verbond oprichten; en gij zult in de ark gaan, gij, en uw zonen, en uw huisvrouw, en de vrouwen uwer zonen met u.
\par 19 En gij zult van al wat leeft, van alle vlees, twee van elk, doen in de ark komen, om met u in het leven te behouden: mannetje en wijfje zullen zij zijn;
\par 20 Van het gevogelte naar zijn aard, en van het vee naar zijn aard, van al het kruipend gedierte des aardbodems naar zijn aard, twee van elk zullen tot u komen, om die in het leven te behouden.
\par 21 En gij, neem voor u van alle spijze, die gegeten wordt, en verzamel ze tot u, opdat zij u en hun tot spijze zij.
\par 22 En Noach deed het; naar al wat God hem geboden had, zo deed hij.

\chapter{7}

\par 1 Daarna zeide de HEERE tot Noach: Ga gij, en uw ganse huis in de ark; want u heb Ik gezien rechtvaardig voor Mijn aangezicht in dit geslacht.
\par 2 Van alle rein vee zult gij tot u nemen zeven en zeven, het mannetje en zijn wijfje; maar van het vee, dat niet rein is, twee, het mannetje en zijn wijfje.
\par 3 Ook van het gevogelte des hemels zeven en zeven, het mannetje en het wijfje, om zaad levend te houden op de ganse aarde.
\par 4 Want over nog zeven dagen zal Ik doen regenen op de aarde veertig dagen, en veertig nachten; en Ik zal van den aardbodem verdelgen al wat bestaat, dat Ik gemaakt heb.
\par 5 En Noach deed, naar al wat de HEERE hem geboden had.
\par 6 Noach nu was zeshonderd jaren oud, als de vloed der wateren op de aarde was.
\par 7 Zo ging Noach, en zijn zonen, en zijn huisvrouw, en de vrouwen zijner zonen met hem in de ark, vanwege de wateren des vloeds.
\par 8 Van het reine vee, en van het vee, dat niet rein was, en van het gevogelte, en al wat op den aardbodem kruipt,
\par 9 Kwamen er twee en twee tot Noach in de ark, het mannetje en het wijfje, gelijk als God Noach geboden had.
\par 10 En het geschiedde na die zeven dagen, dat de wateren des vloeds op de aarde waren.
\par 11 In het zeshonderdste jaar des levens van Noach, in de tweede maand, op den zeventienden dag der maand, op dezen zelfden dag zijn alle fonteinen des groten afgronds opengebroken, en de sluizen des hemels geopend.
\par 12 En een plasregen was op de aarde veertig dagen en veertig nachten.
\par 13 Even op dienzelfden dag ging Noach, en Sem, en Cham, en Jafeth, Noachs zonen, desgelijks ook Noachs huisvrouw, en de drie vrouwen zijner zonen met hem in de ark;
\par 14 Zij, en al het gedierte naar zijn aard, en al het vee naar zijn aard, en al het kruipend gedierte, dat op de aarde kruipt, naar zijn aard, en al het gevogelte naar zijn aard, alle vogeltjes van allerlei vleugel.
\par 15 En van alle vlees, waarin een geest des levens was, kwamen er twee en twee tot Noach in de ark.
\par 16 En die er kwamen, die kwamen mannetje en wijfje, van alle vlees, gelijk als hem God bevolen had. En de HEERE sloot achter hem toe.
\par 17 En die vloed was veertig dagen op de aarde, en de wateren vermeerderden, en hieven de ark op, zodat zij oprees boven de aarde.
\par 18 En de wateren namen de overhand, en vermeerderden zeer op de aarde; en de ark ging op de wateren.
\par 19 En de wateren namen gans zeer de overhand op de aarde, zodat alle hoge bergen, die onder den gansen hemel zijn, bedekt werden.
\par 20 Vijftien ellen omhoog namen de wateren de overhand, en de bergen werden bedekt.
\par 21 En alle vlees, dat zich op de aarde roerde, gaf den geest, van het gevogelte, en van het vee, en van het wild gedierte, en van al het kruipend gedierte, dat op de aarde kroop, en alle mens.
\par 22 Al wat een adem des geestes des levens in zijn neusgaten had, van alles wat op het droge was, is gestorven.
\par 23 Alzo werd verdelgd al wat bestond, dat op den aardbodem was, van den mens aan tot het vee, tot het kruipend gedierte, en tot het gevogelte des hemels, en zij werden verdelgd van de aarde; doch Noach alleen bleef over, en wat met hem in de ark was.
\par 24 En de wateren hadden de overhand boven de aarde, honderd en vijftig dagen.

\chapter{8}

\par 1 En God gedacht aan Noach, en aan al het gedierte, en aan al het vee, dat met hem in de ark was; en God deed een wind over de aarde doorgaan, en de wateren werden stil.
\par 2 Ook werden de fonteinen des afgronds, en de sluizen des hemels gesloten, en de plasregen van den hemel werd opgehouden.
\par 3 Daartoe keerden de wateren weder van boven de aarde, heen en weder vloeiende, en de wateren namen af ten einde van honderd en vijftig dagen.
\par 4 En de ark rustte in de zevende maand, op den zeventienden dag der maand, op de bergen van Ararat.
\par 5 En de wateren waren gaande, en afnemende tot de tiende maand; in de tiende maand, op den eersten der maand, werden de toppen der bergen gezien.
\par 6 En het geschiedde, ten einde van veertig dagen, dat Noach het venster der ark, die hij gemaakt had, opendeed.
\par 7 En hij liet een raaf uit, die dikwijls heen en weder ging, totdat de wateren van boven de aarde verdroogd waren.
\par 8 Daarna liet hij een duif van zich uit, om te zien, of de wateren gelicht waren van boven den aardbodem.
\par 9 Maar de duif vond geen rust voor het hol van haar voet; zo keerde zij weder tot hem in de ark; want de wateren waren op de ganse aarde; en hij stak zijn hand uit, en nam haar, en bracht haar tot zich in de ark.
\par 10 En hij verbeidde nog zeven andere dagen; toen liet hij de duif wederom uit de ark.
\par 11 En de duif kwam tot hem tegen den avondtijd; en ziet, een afgebroken olijfblad was in haar bek; zo merkte Noach, dat de wateren van boven de aarde gelicht waren.
\par 12 Toen vertoefde hij nog zeven andere dagen; en hij liet de duif uit; maar zij keerde niet meer weder tot hem.
\par 13 En het geschiedde in het zeshonderd en eerste jaar, in de eerste maand, op den eersten derzelver maand, dat de wateren droogden van boven de aarde; toen deed Noach het deksel der ark af, en zag toe, en ziet, de aardbodem was gedroogd.
\par 14 En in de tweede maand, op den zeven en twintigsten dag der maand, was de aarde opgedroogd.
\par 15 Toen sprak God tot Noach, zeggende:
\par 16 Ga uit de ark, gij, en uw huisvrouw, en uw zonen, en de vrouwen uwer zonen met u.
\par 17 Al het gedierte, dat met u is, van alle vlees, aan gevogelte, en aan vee, en aan al het kruipend gedierte, dat op de aarde kruipt, doe met u uitgaan; en dat zij overvloediglijk voorttelen op de aarde, en vruchtbaar zijn, en vermenigvuldigen op de aarde.
\par 18 Toen ging Noach uit, en zijn zonen, en zijn huisvrouw, en de vrouwen zijner zonen met hem.
\par 19 Al het gedierte, al het kruipende, en al het gevogelte, al wat zich op de aarde roert, naar hun geslachten, gingen uit de ark.
\par 20 En Noach bouwde den HEERE een altaar; en hij nam van al het reine vee, en van al het rein gevogelte, en offerde brandofferen op dat altaar.
\par 21 En de HEERE rook dien liefelijken reuk, en de HEERE zeide in Zijn hart: Ik zal voortaan den aardbodem niet meer vervloeken om des mensen wil; want het gedichtsel van 's mensen hart is boos van zijn jeugd aan; en Ik zal voortaan niet meer al het levende slaan, gelijk als Ik gedaan heb.
\par 22 Voortaan al de dagen der aarde zullen zaaiing en oogst, en koude en hitte, en zomer en winter, en dag en nacht, niet ophouden.

\chapter{9}

\par 1 En God zegende Noach en zijn zonen, en Hij zeide tot hen: Zijt vruchtbaar en vermenigvuldigt, en vervult de aarde!
\par 2 En uw vrees, en uw verschrikking zij over al het gedierte der aarde, en over al het gevogelte des hemels; in al wat zich op den aardbodem roert, en in alle vissen der zee; zij zijn in uw hand overgegeven.
\par 3 Al wat zich roert, dat levend is, zij u tot spijze; Ik heb het u al gegeven, gelijk het groene kruid.
\par 4 Doch het vlees met zijn ziel, dat is zijn bloed, zult gij niet eten.
\par 5 En voorwaar, Ik zal uw bloed, het bloed uwer zielen eisen; van de hand van alle gedierte zal Ik het eisen; ook van de hand des mensen, van de hand eens iegelijken zijns broeders zal Ik de ziel des mensen eisen.
\par 6 Wie des mensen bloed vergiet, zijn bloed zal door den mens vergoten worden; want God heeft den mens naar Zijn beeld gemaakt.
\par 7 Maar gijlieden, weest vruchtbaar, en vermenigvuldigt; teelt overvloediglijk voort op de aarde, en vermenigvuldigt op dezelve.
\par 8 Voorts zeide God tot Noach, en tot zijn zonen met hem, zeggende:
\par 9 Maar Ik, ziet, Ik richt Mijn verbond op met u, en met uw zaad na u;
\par 10 En met alle levende ziel, die met u is, van het gevogelte, van het vee, en van alle gedierte der aarde met u; van allen, die uit de ark gegaan zijn, tot al het gedierte der aarde toe.
\par 11 En Ik richt Mijn verbond op met u, dat niet meer alle vlees door de wateren des vloeds zal worden uitgeroeid; en dat er geen vloed meer zal zijn, om de aarde te verderven.
\par 12 En God zeide: Dit is het teken des verbonds, dat Ik geef tussen Mij en tussen ulieden, en tussen alle levende ziel, die met u is, tot eeuwige geslachten.
\par 13 Mijn boog heb Ik gegeven in de wolken; die zal zijn tot een teken des verbonds tussen Mij en tussen de aarde.
\par 14 En het zal geschieden, als Ik wolken over de aarde brenge, dat deze boog zal gezien worden in de wolken;
\par 15 Dan zal Ik gedenken aan Mijn verbond, hetwelk is tussen Mij en tussen u, en tussen alle levende ziel van alle vlees; en de wateren zullen niet meer wezen tot een vloed, om alle vlees te verderven.
\par 16 Als deze boog in de wolken zal zijn, zo zal Ik hem aanzien, om te gedenken aan het eeuwig verbond tussen God en tussen alle levende ziel, van alle vlees, dat op de aarde is.
\par 17 Zo zeide dan God tot Noach: Dit is het teken des verbonds, dat Ik opgericht heb tussen Mij en tussen alle vlees, dat op de aarde is.
\par 18 En de zonen van Noach, die uit de ark gingen, waren Sem, en Cham, en Jafeth; en Cham is de vader van Kanaan.
\par 19 Deze drie waren de zonen van Noach; en van dezen is de ganse aarde overspreid.
\par 20 En Noach begon een akkerman te zijn, en hij plantte een wijngaard.
\par 21 En hij dronk van dien wijn, en werd dronken; en hij ontblootte zich in het midden zijner tent.
\par 22 En Cham, Kanaans vader, zag zijns vaders naaktheid, en hij gaf het zijn beiden broederen daar buiten te kennen.
\par 23 Toen namen Sem en Jafeth een kleed, en zij leiden het op hun beider schouderen, en gingen achterwaarts, en bedekten de naaktheid huns vaders; en hun aangezichten waren achterwaarts gekeerd, zodat zij de naaktheid huns vaders niet zagen.
\par 24 En Noach ontwaakte van zijn wijn; en hij merkte wat zijn kleinste zoon hem gedaan had.
\par 25 En hij zeide: Vervloekt zij Kanaan; een knecht der knechten zij hij zijn broederen!
\par 26 Voorts zeide hij: Gezegend zij de HEERE, de God van Sem; en Kanaan zij hem een knecht!
\par 27 God breide Jafeth uit, en hij wone in Sems tenten! en Kanaan zij hem een knecht!
\par 28 En Noach leefde na den vloed driehonderd en vijftig jaren.
\par 29 Zo waren al de dagen van Noach negenhonderd en vijftig jaren; en hij stierf.

\chapter{10}

\par 1 Dit nu zijn de geboorten van Noachs zonen: Sem, Cham, en Jafeth; en hun werden zonen geboren na den vloed.
\par 2 De zonen van Jafeth zijn: Gomer, en Magog, en Madai, en Javan, en Tubal, en Mesech, en Thiras.
\par 3 En de zonen van Gomer zijn: Askenaz, en Rifath, en Togarma.
\par 4 En de zonen van Javan zijn: Elisa, en Tarsis; de Chittieten en Dodanieten.
\par 5 Van dezen zijn verdeeld de eilanden der volken in hun landschappen, elk naar zijn spraak, naar hun huisgezinnen, onder hun volken.
\par 6 En de zonen van Cham zijn: Cusch en Mitsraim, en Put, en Kanaan.
\par 7 En de zonen van Cusch zijn: Seba en Havila, en Sabta, en Raema, en Sabtecha. En de zonen van Raema zijn: Scheba en Dedan.
\par 8 En Cusch gewon Nimrod; deze begon geweldig te zijn op de aarde.
\par 9 Hij was een geweldig jager voor het aangezicht des HEEREN; daarom wordt gezegd: Gelijk Nimrod, een geweldig jager voor het aangezicht des HEEREN.
\par 10 En het beginsel zijns rijks was Babel, en Erech, en Accad, en Calne in het land Sinear.
\par 11 Uit ditzelve land is Assur uitgegaan, en heeft gebouwd Nineve, en Rehoboth, Ir, en Kalach.
\par 12 En Resen, tussen Nineve en tussen Kalach; deze is die grote stad.
\par 13 En Mitsraim gewon de Ludieten, en de Anamieten, en de Lehabieten, en de Naftuchieten,
\par 14 En de Pathrusieten, en de Casluchieten, van waar de Filistijnen uitgekomen zijn, en de Caftorieten.
\par 15 En Kanaan gewon Sidon, zijn eerstgeborene, en Heth,
\par 16 En den Jebusiet, en den Amoriet, en den Girgasiet,
\par 17 En den Hivviet, en den Arkiet, en den Siniet,
\par 18 En den Arvadiet, en den Tsemariet, en den Hamathiet; en daarna zijn de huisgezinnen der Kanaanieten verspreid.
\par 19 En de landpale der Kanaanieten was van Sidon, daar gij gaat naar Gerar tot Gaza toe; daar gij gaat naar Sodom en Gomorra, en Adama, en Zoboim, tot Lasa toe.
\par 20 Deze zijn zonen van Cham, naar hun huisgezinnen, naar hun spraken, in hun landschappen, in hun volken.
\par 21 Voorts zijn Sem zonen geboren; dezelve is ook de vader aller zonen van Heber, broeder van Jafeth, den grootste.
\par 22 Sems zonen waren Elam, en Assur, en Arfachsad, en Lud, en Aram.
\par 23 En Arams zonen waren Uz, en Hul, en Gether, en Maz.
\par 24 En Arfachsad gewon Selah, en Selah gewon Heber.
\par 25 En Heber werden twee zonen geboren; des enen naam was Peleg; want in zijn dagen is de aarde verdeeld; en zijns broeders naam was Joktan.
\par 26 En Joktan gewon Almodad, en Selef, en Hatsarmaveth, en Jarach,
\par 27 En Hadoram, en Usal, en Dikla,
\par 28 En Obal, en Abimael, en Scheba,
\par 29 En Ofir, en Havila, en Jobab; deze allen waren zonen van Joktan.
\par 30 En hun woning was van Mescha af, daar gij gaat naar Sefar, het gebergte van het oosten.
\par 31 Deze zijn zonen van Sem, naar hun huisgezinnen, naar hun spraken, in hun landschappen, naar hun volken.
\par 32 Deze zijn de huisgezinnen der zonen van Noach, naar hun geboorten, in hun volken; en van dezen zijn de volken op de aarde verdeeld na den vloed.

\chapter{11}

\par 1 En de ganse aarde was van enerlei spraak en enerlei woorden.
\par 2 Maar het geschiedde, als zij tegen het oosten togen, dat zij een laagte vonden in het land Sinear; en zij woonden aldaar.
\par 3 En zij zeiden een ieder tot zijn naaste: Kom aan, laat ons tichelen strijken, en wel doorbranden! En de tichel was hun voor steen, en het lijm was hun voor leem.
\par 4 En zij zeiden: Kom aan, laat ons voor ons een stad bouwen, en een toren, welks opperste in den hemel zij, en laat ons een naam voor ons maken, opdat wij niet misschien over de ganse aarde verstrooid worden!
\par 5 Toen kwam de HEERE neder, om te bezien de stad en den toren, die de kinderen der mensen bouwden.
\par 6 En de HEERE zeide: Ziet, zij zijn enerlei volk, en hebben allen enerlei spraak; en dit is het, dat zij beginnen te maken; maar nu, zoude hun niet afgesneden worden al wat zij bedacht hebben te maken?
\par 7 Kom aan, laat Ons nedervaren, en laat Ons hun spraak aldaar verwarren, opdat iegelijk de spraak zijns naasten niet hore.
\par 8 Alzo verstrooide hen de HEERE van daar over de ganse aarde; en zij hielden op de stad te bouwen.
\par 9 Daarom noemde men haar naam Babel; want aldaar verwarde de HEERE de spraak der ganse aarde, en van daar verstrooide hen de HEERE over de ganse aarde.
\par 10 Deze zijn de geboorten van Sem: Sem was honderd jaren oud, en gewon Arfachsad, twee jaren na den vloed.
\par 11 En Sem leefde, nadat hij Arfachsad gewonnen had, vijfhonderd jaren; en hij gewon zonen en dochteren.
\par 12 En Arfachsad leefde vijf en dertig jaren, en hij gewon Selah.
\par 13 En Arfachsad leefde, nadat hij Selah gewonnen had, vierhonderd en drie jaren; en hij gewon zonen en dochteren.
\par 14 En Selah leefde dertig jaren, en hij gewon Heber.
\par 15 En Selah leefde, nadat hij Heber gewonnen had, vierhonderd en drie jaren, en hij gewon zonen en dochteren.
\par 16 En Heber leefde vier en dertig jaren, en gewon Peleg.
\par 17 En Heber leefde, nadat hij Peleg gewonnen had, vierhonderd en dertig jaren; en hij gewon zonen en dochteren.
\par 18 En Peleg leefde dertig jaren, en hij gewon Rehu.
\par 19 En Peleg leefde, nadat hij Rehu gewonnen had, tweehonderd en negen jaren; en hij gewon zonen en dochteren.
\par 20 En Rehu leefde twee en dertig jaren, en hij gewon Serug.
\par 21 En Rehu leefde, nadat hij Serug gewonnen had, tweehonderd en zeven jaren; en hij gewon zonen en dochteren.
\par 22 En Serug leefde dertig jaren, en gewon Nahor.
\par 23 En Serug leefde, nadat hij Nahor gewonnen had, tweehonderd jaren; en hij gewon zonen en dochteren.
\par 24 En Nahor leefde negen en twintig jaren, en gewon Terah.
\par 25 En Nahor leefde, nadat hij Terah gewonnen had, honderd en negentien jaren; en hij gewon zonen en dochteren.
\par 26 En Terah leefde zeventig jaren, en gewon Abram, Nahor en Haran.
\par 27 En deze zijn de geboorten van Terah: Terah gewon Abram, Nahor en Haran; en Haran gewon Lot.
\par 28 En Haran stierf voor het aangezicht zijns vaders Terah, in het land zijner geboorte, in Ur der Chaldeen.
\par 29 En Abram en Nahor namen zich vrouwen; de naam van Abrams huisvrouw was Sarai, en de naam van Nahors huisvrouw was Milka, een dochter van Haran, vader van Milka, en vader van Jiska.
\par 30 En Sarai was onvruchtbaar; zij had geen kind.
\par 31 En Terah nam Abram, zijn zoon, en Lot, Harans zoon, zijns zoons zoon, en Sarai, zijn schoondochter, de huisvrouw van zijn zoon Abram, en zij togen met hen uit Ur der Chaldeen, om te gaan naar het land Kanaan; en zij kwamen tot Haran, en woonden aldaar.
\par 32 En de dagen van Terah waren tweehonderd en vijf jaren, en Terah stierf te Haran.

\chapter{12}

\par 1 De HEERE nu had tot Abram gezegd: Ga gij uit uw land, en uit uw maagschap, en uit uws vaders huis, naar het land, dat Ik u wijzen zal.
\par 2 En Ik zal u tot een groot volk maken, en u zegenen, en uw naam groot maken; en wees een zegen!
\par 3 En Ik zal zegenen, die u zegenen, en vervloeken, die u vloekt; en in u zullen alle geslachten des aardrijks gezegend worden.
\par 4 En Abram toog heen, gelijk de HEERE tot hem gesproken had; en Lot toog met hem; en Abram was vijf en zeventig jaren oud, toen hij uit Haran ging.
\par 5 En Abram nam Sarai, zijn huisvrouw, en Lot, zijns broeders zoon, en al hun have, die zij verworven hadden, en de zielen, die zij verkregen hadden in Haran; en zij togen uit, om te gaan naar het land Kanaan, en zij kwamen in het land Kanaan.
\par 6 En Abram is doorgetogen in dat land, tot aan de plaats Sichem, tot aan het eikenbos More; en de Kanaanieten waren toen ter tijd in dat land.
\par 7 Zo verscheen de HEERE aan Abram, en zeide: Aan uw zaad zal Ik dit land geven. Toen bouwde hij aldaar een altaar den HEERE, Die hem verschenen was.
\par 8 En hij brak op van daar naar het gebergte, tegen het oosten van Beth-el, en hij sloeg zijn tent op, zijnde Beth-el tegen het westen, en Ai tegen het oosten; en hij bouwde daar den HEERE een altaar, en riep den Naam des HEEREN aan.
\par 9 Daarna vertrok Abram, gaande en trekkende naar het zuiden.
\par 10 En er was honger in dat land; zo toog Abram af naar Egypte, om daar als een vreemdeling te verkeren, dewijl de honger zwaar was in dat land.
\par 11 En het geschiedde, als hij naderde, om in Egypte te komen, dat hij zeide tot Sarai, zijn huisvrouw: Zie toch, ik weet, dat gij een vrouw zijt, schoon van aangezicht.
\par 12 En het zal geschieden, als u de Egyptenaars zullen zien, zo zullen zij zeggen: Dat is zijn huisvrouw; en zij zullen mij doden, en u in het leven behouden.
\par 13 Zeg toch: Gij zijt mijn zuster; opdat het mij wel ga om u, en mijn ziel om uwentwil leve.
\par 14 En het geschiedde, als Abram in Egypte kwam, dat de Egyptenaars deze vrouw zagen, dat zij zeer schoon was.
\par 15 Ook zagen haar de vorsten van Farao, en prezen haar bij Farao; en die vrouw werd weggenomen naar het huis van Farao.
\par 16 En hij deed Abram goed, om harentwil; zodat hij had schapen, en runderen, en ezelen, en knechten, en maagden, en ezelinnen, en kemelen.
\par 17 Maar de HEERE plaagde Farao met grote plagen, ook zijn huis, ter oorzake van Sarai, Abrams huisvrouw.
\par 18 Toen riep Farao Abram, en zeide: Wat is dit, dat gij mij gedaan hebt? waarom hebt gij mij niet te kennen gegeven, dat zij uw huisvrouw is?
\par 19 Waarom hebt gij gezegd: Zij is mijn zuster; zodat ik haar mij tot een vrouw zoude genomen hebben? en nu, zie, daar is uw huisvrouw; neem haar en ga henen!
\par 20 En Farao gebood zijn mannen vanwege hem, en zij geleidden hem, en zijn huisvrouw, en alles wat hij had.

\chapter{13}

\par 1 Alzo toog Abram op uit Egypte naar het zuiden, hij en zijn huisvrouw, en al wat hij had, en Lot met hem.
\par 2 En Abram was zeer rijk, in vee, in zilver, en in goud.
\par 3 En hij ging, volgens zijn reizen, van het zuiden tot Beth-el toe, tot aan de plaats, waar zijn tent in het begin geweest was, tussen Beth-el, en tussen Ai;
\par 4 Tot de plaats des altaars, dat hij in het eerst daar gemaakt had; en Abram heeft aldaar den Naam des HEEREN aangeroepen.
\par 5 En Lot, die met Abram toog, had ook schapen, en runderen, en tenten.
\par 6 En dat land droeg hen niet, om samen te wonen; want hun have was vele, zodat zij samen niet konden wonen.
\par 7 En er was twist tussen de herders van Abrams vee, en tussen de herders van Lots vee. Ook woonden toen de Kanaanieten en Ferezieten in dat land.
\par 8 En Abram zeide tot Lot: Laat toch geen twisting zijn tussen mij en tussen u, en tussen mijn herders en tussen uw herders; want wij zijn mannen broeders.
\par 9 Is niet het ganse land voor uw aangezicht? Scheid u toch van mij; zo gij de linkerhand kiest, zo zal ik ter rechterhand gaan; en zo gij de rechterhand, zo zal ik ter linkerhand gaan.
\par 10 En Lot hief zijn ogen op, en hij zag de ganse vlakte der Jordaan, dat zij die geheel bevochtigde; eer de HEERE Sodom en Gomorra verdorven had, was zij als de hof des HEEREN, als Egypteland, als gij komt te Zoar.
\par 11 Zo koos Lot voor zich de ganse vlakte der Jordaan, en Lot trok tegen het oosten; en zij werden gescheiden, de een van den ander.
\par 12 Abram dan woonde in het land Kanaan; en Lot woonde in de steden der vlakte, en sloeg tenten tot aan Sodom toe.
\par 13 En de mannen van Sodom waren boos, en grote zondaars tegen den HEERE.
\par 14 En de HEERE zeide tot Abram, nadat Lot van hem gescheiden was: Hef uw ogen op, en zie van de plaats, waar gij zijt noordwaarts en zuidwaarts, en oostwaarts en westwaarts.
\par 15 Want al dit land, dat gij ziet, zal Ik u geven, en aan uw zaad, tot in eeuwigheid.
\par 16 En Ik zal uw zaad stellen als het stof der aarde, zodat, indien iemand het stof der aarde zal kunnen tellen, zal ook uw zaad geteld worden.
\par 17 Maak u op, wandel door dit land, in zijn lengte en in zijn breedte; want Ik zal het u geven.
\par 18 En Abram sloeg tenten op, en kwam en woonde aan de eikenbossen van Mamre, die bij Hebron zijn; en hij bouwde aldaar den HEERE een altaar.

\chapter{14}

\par 1 En het geschiedde in de dagen van Amrafel, den koning van Sinear, van Arioch, den koning van Ellasar, van Kedor-laomer, den koning van Elam, en van Tideal, den koning der volken;
\par 2 Dat zij krijg voerden met Bera, koning van Sodom, en met Birsa, koning van Gomorra, Sinab, koning van Adama, en Semeber, koning van Zeboim, en den koning van Bela, dat is Zoar.
\par 3 Deze allen voegden zich samen in het dal Siddim, dat is de Zoutzee.
\par 4 Twaalf jaren hadden zij Kedor-laomer gediend; maar in het dertiende jaar vielen zij af.
\par 5 Zo kwam Kedor-laomer in het veertiende jaar, en de koningen, die met hem waren, en sloegen de Refaieten in Asteroth-karnaim, en de Zuzieten in Ham, en de Emieten in Schave-kiriathaim;
\par 6 En de Horieten op hun gebergte Seir, tot aan het effen veld van Paran, hetwelk aan de woestijn is.
\par 7 Daarna keerden zij wederom, en kwamen tot En-mispat, dat is Kades, en sloegen al het land der Amalekieten, en ook den Amoriet, die te Hazezon-thamar woonde.
\par 8 Toen toog de koning van Sodom uit, en de koning van Gomorra, en de koning van Adama, en de koning van Zeboim, en de koning van Bela, dat is Zoar; en zij stelden tegen hen slagorden in het dal Siddim,
\par 9 Tegen Kedor-laomer, den koning van Elam, en Tideal, den koning der volken, en Amrafel, den koning van Sinear, en Arioch, den koning van Ellasar; vier koningen tegen vijf.
\par 10 Het dal nu van Siddim was vol lijmputten; en de koningen van Sodom en Gomorra vluchtten, en vielen aldaar; en de overgeblevenen vluchtten naar het gebergte.
\par 11 En zij namen al de have van Sodom en Gomorra, en al hun spijze, en trokken weg.
\par 12 Ook namen zij Lot, den zoon van Abrams broeder, en zijn have, en trokken weg; want hij woonde in Sodom.
\par 13 Toen kwam er een, die ontkomen was, en boodschapte het aan Abram, den Hebreer, die woonachtig was aan de eikenbossen van Mamre, den Amoriet, broeder van Eskol, en broeder van Aner, welke Abrams bondgenoten waren.
\par 14 Als Abram hoorde, dat zijn broeder gevangen was, zo wapende hij zijn onderwezenen, de ingeborenen van zijn huis, driehonderd en achttien, en hij jaagde hen na tot Dan toe.
\par 15 En hij verdeelde zich tegen hen des nachts, hij en zijn knechten, en sloeg ze; en hij jaagde hen na tot Hoba toe, hetwelk is ter linkerhand van Damaskus.
\par 16 En hij bracht alle have weder, en ook Lot zijn broeder en deszelfs have bracht hij weder, als ook de vrouwen, en het volk.
\par 17 En de koning van Sodom toog uit, hem tegemoet (nadat hij wedergekeerd was van het slaan van Kedor-laomer, en van de koningen, die met hem waren), tot het dal Schave, dat is, het dal des konings.
\par 18 En Melchizedek, koning van Salem, bracht voort brood en wijn; en hij was een priester des allerhoogsten Gods.
\par 19 En hij zegende hem, en zeide: Gezegend zij Abram Gode, de Allerhoogste, Die hemel en aarde bezit!
\par 20 En gezegend zij de allerhoogste God, Die uw vijanden in uw hand geleverd heeft! En hij gaf hem de tiende van alles.
\par 21 En de koning van Sodom zeide tot Abram: Geef mij de zielen; maar neem de have voor u.
\par 22 Doch Abram zeide tot den koning van Sodom: Ik heb mijn hand opgeheven tot den HEERE, den allerhoogsten God, Die hemel en aarde bezit;
\par 23 Zo ik van een draad aan tot een schoenriem toe, ja, zo ik van alles, dat het uwe is, iets neme! opdat gij niet zegt: Ik heb Abram rijk gemaakt!
\par 24 Het zij buiten mij; alleen wat de jongelingen verteerd hebben, en het deel dezer mannen, die met mij getogen zijn, Aner, Eskol en Mamre, laat die hun deel nemen!

\chapter{15}

\par 1 Na deze dingen geschiedde het woord des HEEREN tot Abram in een gezicht, zeggende: Vrees niet, Abram! Ik ben u een Schild, uw Loon zeer groot.
\par 2 Toen zeide Abram: Heere, HEERE! wat zult Gij mij geven, daar ik zonder kinderen heenga en de bezorger van mijn huis is deze Damaskener Eliezer?
\par 3 Voorts zeide Abram: Zie, mij hebt Gij geen zaad gegeven, en zie, de zoon van mijn huis zal mijn erfgenaam zijn!
\par 4 En ziet, het woord des HEEREN was tot hem, zeggende: Deze zal uw erfgenaam niet zijn; maar die uit uw lijf voortkomen zal, die zal uw erfgenaam zijn.
\par 5 Toen leidde Hij hem uit naar buiten, en zeide: Zie nu op naar den hemel, en tel de sterren, indien gij ze tellen kunt; en Hij zeide tot hem: Zo zal uw zaad zijn!
\par 6 En hij geloofde in den HEERE; en Hij rekende het hem tot gerechtigheid.
\par 7 Voorts zeide Hij tot hem: Ik ben de HEERE, Die u uitgeleid heb uit Ur der Chaldeen, om u dit land te geven, om dat erfelijk te bezitten.
\par 8 En hij zeide: Heere, HEERE! waarbij zal ik weten, dat ik het erfelijk bezitten zal?
\par 9 En Hij zeide tot hem: Neem Mij een driejarige vaars, en een driejarige geit, en een driejarigen ram, en een tortelduif, en een jonge duif.
\par 10 En hij bracht Hem deze alle, en hij deelde ze middendoor, en hij leide elks deel tegen het andere over; maar het gevogelte deelde hij niet.
\par 11 En het wild gevogelte kwam neder op het aas; maar Abram joeg het weg.
\par 12 En het geschiedde, als de zon was aan het ondergaan, zo viel een diepe slaap op Abram; en ziet, een schrik, en grote duisternis viel op hem.
\par 13 Toen zeide Hij tot Abram: Weet voorzeker, dat uw zaad vreemd zal zijn in een land, dat het hunne niet is, en zij zullen hen dienen, en zij zullen hen verdrukken vierhonderd jaren.
\par 14 Doch Ik zal het volk ook rechten, hetwelk zij zullen dienen; en daarna zullen zij uittrekken met grote have.
\par 15 En gij zult tot uw vaderen gaan met vrede; gij zult in goeden ouderdom begraven worden.
\par 16 En het vierde geslacht zal herwaarts wederkeren; want de ongerechtigheid der Amorieten is tot nog toe niet volkomen.
\par 17 En het geschiedde, dat de zon onderging en het duister werd, en ziet, daar was een rokende oven en vurige fakkel, die tussen die stukken doorging.
\par 18 Ten zelfden dage maakte de HEERE een verbond met Abram, zeggende: Aan uw zaad heb Ik dit land gegeven, van de rivier van Egypte af, tot aan die grote rivier, de rivier Frath:
\par 19 Den Keniet, en den Keniziet, en den Kadmoniet,
\par 20 En den Hethiet, en den Fereziet, en de Refaieten,
\par 21 En den Amoriet, en den Kanaaniet, en den Girgaziet, en den Jebusiet.

\chapter{16}

\par 1 Doch Sarai, Abrams huisvrouw, baarde hem niet; en zij had een Egyptische dienstmaagd, welker naam was Hagar.
\par 2 Zo zeide Sarai tot Abram: Zie toch, de HEERE heeft mij toegesloten, dat ik niet bare; ga toch in tot mijn dienstmaagd, misschien zal ik uit haar gebouwd worden. En Abram hoorde naar de stem van Sarai.
\par 3 Zo nam Sarai, Abrams huisvrouw, de Egyptische Hagar, haar dienstmaagd, ten einde van tien jaren, welke Abram in het land Kanaan gewoond had, en zij gaf haar aan Abram, haar man, hem tot een vrouw.
\par 4 En hij ging in tot Hagar, en zij ontving. Als zij nu zag, dat zij ontvangen had, zo werd haar vrouw veracht in haar ogen.
\par 5 Toen zeide Sarai tot Abram: Mijn ongelijk is op u; ik heb mijn dienstmaagd in uw schoot gegeven; nu zij ziet, dat zij ontvangen heeft, zo ben ik veracht in haar ogen; de HEERE rechte tussen mij en tussen u!
\par 6 En Abram zeide tot Sarai: Zie uw dienstmaagd is in uw hand; doe haar, wat goed is in uw ogen. En Sarai vernederde haar, en zij vluchtte van haar aangezicht.
\par 7 En de Engel des HEEREN vond haar aan een waterfontein in de woestijn, aan de fontein op den weg van Sur.
\par 8 En hij zeide: Hagar, gij, dienstmaagd van Sarai! van waar komt gij, en waar zult gij heengaan? En zij zeide: Ik ben vluchtende van het aangezicht mijner vrouw Sarai!
\par 9 Toen zeide de Engel des HEEREN tot haar: Keer weder tot uw vrouw, en verneder u onder haar handen.
\par 10 Voorts zeide de Engel des HEEREN tot haar: Ik zal uw zaad grotelijks vermenigvuldigen, zodat het vanwege de menigte niet zal geteld worden.
\par 11 Ook zeide des HEEREN Engel tot haar: Zie, gij zijt zwanger, en zult een zoon baren, en gij zult zijn naam Ismael noemen, omdat de HEERE uw verdrukking aangehoord heeft.
\par 12 En hij zal een woudezel van een mens zijn; zijn hand zal tegen allen zijn, en de hand van allen tegen hem; en hij zal wonen voor het aangezicht van al zijn broederen.
\par 13 En zij noemde den Naam des HEEREN, Die tot haar sprak: Gij, God des aanziens! want zij zeide: Heb ik ook hier gezien naar Dien, Die mij aanziet?
\par 14 Daarom noemde men dien put, den put Lachai-roi; ziet, hij is tussen Kades en tussen Bered.
\par 15 En Hagar baarde Abram een zoon; en Abram noemde den naam zijns zoons, die Hagar gebaard had, Ismael.
\par 16 En Abram was zes en tachtig jaren oud, toen Hagar Ismael aan Abram baarde.

\chapter{17}

\par 1 Als nu Abram negen en negentig jaren oud was, zo verscheen de HEERE aan Abram, en zeide tot hem: Ik ben God, de Almachtige! Wandel voor Mijn aangezicht, en zijt oprecht!
\par 2 En Ik zal Mijn verbond stellen tussen Mij en tussen u, en Ik zal u gans zeer vermenigvuldigen.
\par 3 Toen viel Abram op zijn aangezicht, en God sprak met hem, zeggende:
\par 4 Mij aangaande, zie, Mijn verbond is met u; en gij zult tot een vader van menigte der volken worden!
\par 5 En uw naam zal niet meer genoemd worden Abram; maar uw naam zal wezen Abraham; want Ik heb u gesteld tot een vader van menigte der volken.
\par 6 En Ik zal u gans zeer vruchtbaar maken, en Ik zal u tot volken stellen, en koningen zullen uit u voortkomen.
\par 7 En Ik zal Mijn verbond oprichten tussen Mij en tussen u, en tussen uw zaad na u in hun geslachten, tot een eeuwig verbond, om u te zijn tot een God, en uw zaad na u.
\par 8 En Ik zal u, en uw zaad na u, het land uwer vreemdelingschappen geven, het gehele land Kanaan, tot eeuwige bezitting; en Ik zal hun tot een God zijn.
\par 9 Voorts zeide God tot Abraham: Gij nu zult Mijn verbond houden, gij, en uw zaad na u, in hun geslachten.
\par 10 Dit is Mijn verbond, dat gijlieden houden zult tussen Mij, en tussen u, en tussen uw zaad na u: dat al wat mannelijk is, u besneden worde.
\par 11 En gij zult het vlees uwer voorhuid besnijden; en dat zal tot een teken zijn van het verbond tussen Mij en tussen u.
\par 12 Een zoontje dan van acht dagen zal u besneden worden, al wat mannelijk is in uw geslachten: de ingeborene van het huis, en de gekochte met geld van allen vreemde, welke niet is van uw zaad;
\par 13 De ingeborene van uw huis, en de gekochte met uw geld zal zekerlijk besneden worden; en Mijn verbond zal zijn in ulieder vlees, tot een eeuwig verbond.
\par 14 En wat mannelijk is, de voorhuid hebbende, wiens voorhuids vlees niet zal besneden worden, dezelve ziel zal uit haar volken uitgeroeid worden; hij heeft Mijn verbond gebroken.
\par 15 Nog zeide God tot Abraham: Gij zult den naam van uw huisvrouw Sarai, niet Sarai noemen; maar haar naam zal zijn Sara.
\par 16 Want Ik zal haar zegenen, en u ook uit haar een zoon geven; ja, Ik zal haar zegenen, zodat zij tot volken worden zal: koningen der volken zullen uit haar worden!
\par 17 Toen viel Abraham op zijn aangezicht, en hij lachte; en hij zeide in zijn hart: Zal een, die honderd jaren oud is, een kind geboren worden; en zal Sara, die negentig jaren oud is, baren?
\par 18 En Abraham zeide tot God: Och, dat Ismael mocht leven voor Uw aangezicht!
\par 19 En God zeide: Voorwaar, Sara, uw huisvrouw, zal u een zoon baren, en gij zult zijn naam noemen Izak; en Ik zal Mijn verbond met hem oprichten, tot een eeuwig verbond zijn zade na hem.
\par 20 En aangaande Ismael heb Ik u verhoord; zie, Ik heb hem gezegend, en zal hem vruchtbaar maken, en hem gans zeer vermenigvuldigen; twaalf vorsten zal hij gewinnen, en Ik zal hem tot een groot volk stellen;
\par 21 Maar Mijn verbond zal Ik met Izak oprichten, die u Sara op dezen gezetten tijd in het andere jaar baren zal.
\par 22 En Hij eindigde met hem te spreken, en God voer op van Abraham.
\par 23 Toen nam Abraham zijn zoon Ismael, en al de ingeborenen van zijn huis, en alle gekochten met zijn geld, al wat mannelijk was onder de lieden van het huis van Abraham, en hij besneed het vlees hunner voorhuid, even ten zelfden dage, gelijk als God met hem gesproken had.
\par 24 En Abraham was oud negen en negentig jaren, als hem het vlees zijner voorhuid besneden werd.
\par 25 En Ismael, zijn zoon, was dertien jaren oud, als hem het vlees zijner voorhuid besneden werd.
\par 26 Even op dezen zelfden dag werd Abraham besneden, en Ismael, zijn zoon.
\par 27 En alle mannen van zijn huis, de ingeborenen des huizes, en de gekochten met geld, van den vreemde af, werden met hem besneden.

\chapter{18}

\par 1 Daarna verscheen hem de HEERE aan de eikenbossen van Mamre, als hij in de deur der tent zat, toen de dag heet werd.
\par 2 En hij hief zijn ogen op en zag; en ziet, daar stonden drie mannen tegenover hem; als hij hen zag, zo liep hij hun tegemoet van de deur der tent, en boog zich ter aarde.
\par 3 En hij zeide: Heere! heb ik nu genade gevonden in Uw ogen, zo gaat toch niet aan Uw knecht voorbij.
\par 4 Dat toch een weinig waters gebracht worde, en wast Uw voeten, en leunt onder dezen boom.
\par 5 En ik zal een bete broods langen, dat Gij Uw hart sterkt; daarna zult Gij voortgaan, daarom omdat Gij tot Uw knecht overgekomen zijt. En zij zeiden: Doe zo als gij gesproken hebt.
\par 6 En Abraham haastte zich naar de tent tot Sara, en hij zeide: Haast u; kneed drie maten meelbloem, en maak koeken.
\par 7 En Abraham liep tot de runderen, en hij nam een kalf, teder en goed, en hij gaf het aan den knecht, die haastte, om dat toe te maken.
\par 8 En hij nam boter en melk, en het kalf, dat hij toegemaakt had, en hij zette het hun voor, en stond bij hen onder dien boom, en zij aten.
\par 9 Toen zeiden zij tot hem: Waar is Sara, uw huisvrouw? En hij zeide: Ziet, in de tent.
\par 10 En Hij zeide: Ik zal voorzeker weder tot u komen, omtrent dezen tijd des levens; en zie, Sara, uw huisvrouw, zal een zoon hebben! En Sara hoorde het aan de deur der tent, welke achter Hem was.
\par 11 Abraham nu en Sara waren oud, en wel bedaagd; het had Sara opgehouden te gaan naar de wijze der vrouwen.
\par 12 Zo lachte Sara bij zichzelve, zeggende: Zal ik wellust hebben, nadat ik oud geworden ben, en mijn heer oud is?
\par 13 En de HEERE zeide tot Abraham: Waarom heeft Sara gelachen, zeggende: Zou ik ook waarlijk baren, nu ik oud geworden ben?
\par 14 Zou iets voor den HEERE te wonderlijk zijn? Ter gezetter tijd zal Ik tot u wederkomen, omtrent dezen tijd des levens, en Sara zal een zoon hebben!
\par 15 En Sara loochende het, zeggende: Ik heb niet gelachen; want zij vreesde. En Hij zeide: Neen! maar gij hebt gelachen.
\par 16 Toen stonden die mannen op van daar, en zagen naar Sodom toe; en Abraham ging met hen, om hen te geleiden.
\par 17 En de HEERE zeide: Zal Ik voor Abraham verbergen, wat Ik doe?
\par 18 Dewijl Abraham gewisselijk tot een groot en machtig volk worden zal, en alle volken der aarde in hem gezegend zullen worden?
\par 19 Want Ik heb hem gekend, opdat hij zijn kinderen en zijn huis na hem zoude bevelen, en zij den weg des HEEREN houden, om te doen gerechtigheid en gerichte; opdat de HEERE over Abraham brenge, hetgeen Hij over hem gesproken heeft.
\par 20 Voorts zeide de HEERE: Dewijl het geroep van Sodom en Gomorra groot is, en dewijl haar zonde zeer zwaar is,
\par 21 Zal Ik nu afgaan en bezien, of zij naar hun geroep, dat tot Mij gekomen is, het uiterste gedaan hebben, en zo niet, Ik zal het weten.
\par 22 Toen keerden die mannen het aangezicht van daar, en gingen naar Sodom; maar Abraham bleef nog staande voor het aangezicht des HEEREN.
\par 23 En Abraham trad toe, en zeide: Zult Gij ook den rechtvaardige met den goddeloze ombrengen?
\par 24 Misschien zijn er vijftig rechtvaardigen in de stad; zult Gij hen ook ombrengen, en de plaats niet sparen, om de vijftig rechtvaardigen, die binnen haar zijn?
\par 25 Het zij verre van U, zulk een ding te doen, te doden den rechtvaardige met den goddeloze! dat de rechtvaardige zij gelijk de goddeloze, verre zij het van U! zou de Rechter der ganse aarde geen recht doen?
\par 26 Toen zeide de HEERE: Zo Ik te Sodom binnen de stad vijftig rechtvaardigen zal vinden, zo zal Ik de ganse plaats sparen om hunnentwil.
\par 27 En Abraham antwoordde en zeide: Zie toch; ik heb mij onderwonden te spreken tot den Heere, hoewel ik stof en as ben!
\par 28 Misschien zullen aan de vijftig rechtvaardigen vijf ontbreken; zult Gij dan om vijf de ganse stad verderven? En Hij zeide: Ik zal haar niet verderven, zo Ik er vijf en veertig zal vinden.
\par 29 En hij voer voort nog tot Hem te spreken, en zeide: Misschien zullen aldaar veertig gevonden worden! En Hij zeide: Ik zal het niet doen om der veertigen wil.
\par 30 Voorts zeide hij: Dat toch de Heere niet ontsteke, dat ik spreke; misschien zullen aldaar dertig gevonden worden! En Hij zeide: Ik zal het niet doen, zo Ik aldaar dertig zal vinden.
\par 31 En hij zeide: Zie toch, ik heb mij onderwonden te spreken tot de Heere; misschien zullen er twintig gevonden worden! En Hij zeide: Ik zal haar niet verderven om der twintigen wil.
\par 32 Nog zeide hij: Dat toch de Heere niet ontsteke, dat ik alleenlijk ditmaal spreke: misschien zullen er tien gevonden worden. En Hij zeide: Ik zal haar niet verderven om der tienen wil.
\par 33 Toen ging de HEERE weg, als Hij geeindigd had tot Abraham te spreken; en Abraham keerde weder naar zijn plaats.

\chapter{19}

\par 1 En die twee engelen kwamen te Sodom in den avond; en Lot zat in de poort te Sodom; en als Lot hen zag, stond hij op hun tegemoet, en boog zich met het aangezicht ter aarde.
\par 2 En hij zeide: Ziet nu, mijne heren! keert toch in ten huize van uw knecht, en vernacht, en wast uw voeten; en gij zult vroeg opstaan, en gaan uws weegs. En zij zeiden: Neen, maar wij zullen op de straat vernachten.
\par 3 En hij hield bij hen zeer aan, zodat zij tot hem inkeerden, en kwamen in zijn huis; en hij maakte hun een maaltijd, en bakte ongezuurde koeken, en zij aten.
\par 4 Eer zij zich te slapen leiden, zo hebben de mannen dier stad, de mannen van Sodom, van den jongste tot den oudste toe, dat huis omsingeld, het ganse volk, van het uiterste einde af.
\par 5 En zij riepen Lot toe, en zeiden tot hem: Waar zijn die mannen, die deze nacht tot u gekomen zijn? breng hen uit tot ons, opdat wij ze bekennen.
\par 6 Toen ging Lot uit tot hen aan de deur, en hij sloot de deur achter zich toe;
\par 7 En hij zeide: Mijn broeders! doet toch geen kwaad!
\par 8 Ziet toch, ik heb twee dochters, die geen man bekend hebben; ik zal haar nu tot u uitbrengen, en doet haar, zoals het goed is in uw ogen; alleenlijk doet dezen mannen niets; want daarom zijn zij onder de schaduw mijns daks ingegaan.
\par 9 Toen zeiden zij: Kom verder aan! Voorts zeiden zij: Deze ene is gekomen, om als vreemdeling hier te wonen, en zoude hij alleszins rechter zijn? Nu zullen wij u meer kwaads doen, dan hun. En zij drongen zeer op den man, op Lot, en zij traden toe om de deur open te breken.
\par 10 Doch die mannen staken hun hand uit, en deden Lot tot zich inkomen in het huis, en sloten de deur toe.
\par 11 En zij sloegen de mannen, die aan de deur van het huis waren, met verblindheden, van den kleinste tot aan den grootste, zodat zij moede werden, om de deur te vinden.
\par 12 Toen zeiden die mannen tot Lot: Wien hebt gij hier nog meer? een schoonzoon, of uw zonen, of uw dochteren, en allen, die gij hebt in deze stad, breng uit deze plaats;
\par 13 Want wij gaan deze plaats verderven, omdat haar geroep groot geworden is voor het aangezicht des HEEREN, en de HEERE ons uitgezonden heeft, om haar te verderven.
\par 14 Toen ging Lot uit, en sprak tot zijn schoonzonen, die zijn dochteren nemen zouden, en zeide: Maakt u op, gaat uit deze plaats; want de HEERE gaat deze stad verderven. Maar hij was in de ogen zijner schoonzonen als jokkende.
\par 15 En als de dageraad opging, drongen de engelen Lot aan, zeggende: Maak u op, neem uw huisvrouw, en uw twee dochteren, die voorhanden zijn, opdat gij in de ongerechtigheid dezer stad niet omkomt.
\par 16 Maar hij vertoefde; zo grepen dan die mannen zijn hand, en de hand zijner vrouw, en de hand zijner twee dochteren, om de verschoning des HEEREN over hem; en zij brachten hem uit, en stelden hem buiten de stad.
\par 17 En het geschiedde als zij hen uitgebracht hadden naar buiten, zo zeide Hij: behoud u om uws levens wil; zie niet achter u om, en sta niet op deze ganse vlakte; behoud u naar het gebergte heen, opdat gij niet omkomt.
\par 18 En Lot zeide tot hen: Neen toch, Heere!
\par 19 Zie toch, Uw knecht heeft genade gevonden in Uw ogen, en Gij hebt Uw weldadigheid groot gemaakt, die Gij aan mij gedaan hebt, om mijn ziel te behouden bij het leven; maar ik zal niet kunnen behouden worden naar het gebergte heen, opdat mij niet misschien dat kwaad aankleve, en ik sterve!
\par 20 Ziet toch, deze stad is nabij, om derwaarts te vluchten, en zij is klein; laat mij toch derwaarts behouden worden (is zij niet klein?) opdat mijn ziel leve.
\par 21 En Hij zeide tot hem: Zie, Ik heb uw aangezicht opgenomen ook in deze zaak, dat Ik deze stad niet omkere waarvan gij gesproken hebt.
\par 22 Haast, behoud u derwaarts; want Ik zal niets kunnen doen, totdat gij daarhenen ingekomen zijt. Daarom noemde men den naam dezer stad Zoar.
\par 23 De zon ging op boven de aarde, als Lot te Zoar inkwam.
\par 24 Toen deed de HEERE zwavel en vuur over Sodom en Gomorra regenen, van den HEERE uit den hemel.
\par 25 En Hij keerde deze steden om, en die ganse vlakte, en alle inwoners dezer steden, ook het gewas des lands.
\par 26 En zijn huisvrouw zag om van achter hem; en zij werd een zoutpilaar.
\par 27 En Abraham maakte zich deszelven morgens vroeg op, naar de plaats, waar hij voor het aangezicht des HEEREN gestaan had.
\par 28 En hij zag naar Sodom en Gomorra toe, en naar het ganse land van die vlakte; en hij zag, en ziet, er ging een rook van het land op, gelijk de rook eens ovens.
\par 29 En het geschiedde, toen God de steden dezer vlakte verdierf, dat God aan Abraham gedacht, en Hij leidde Lot uit het midden dezer omkering, in het omkeren dier steden, in welke Lot gewoond had.
\par 30 En Lot toog op uit Zoar, en woonde op den berg, en zijn twee dochters met hem; want hij vreesde binnen Zoar te wonen. En hij woonde in een spelonk, hij en zijn twee dochters.
\par 31 Toen zeide de eerstgeborene tot de jongste: Onze vader is oud, en er is geen man in dit land, om tot ons in te gaan, naar de wijze der ganse aarde.
\par 32 Kom, laat ons onze vader wijn te drinken geven, en bij hem liggen, opdat wij van onze vader zaad in het leven behouden.
\par 33 En zij gaven dien nacht haar vader wijn te drinken; en de eerstgeborene kwam, en lag bij haar vader, en hij werd het niet gewaar in haar nederliggen, noch in haar opstaan.
\par 34 En het geschiedde des anderen daags, dat de eerstgeborene zeide tot de jongste: Zie, ik heb gisteren nacht bij mijn vader gelegen; laat ons ook dezen nacht hem wijn te drinken geven; ga dan in, lig bij hem, opdat wij van onzen vader zaad in het leven behouden.
\par 35 En zij gaven haar vader ook dien nacht wijn te drinken, en de jongste stond op, en lag bij hem. En hij werd het niet gewaar in haar nederliggen, noch in haar opstaan.
\par 36 En de twee dochters van Lot werden bevrucht van haar vader.
\par 37 En de eerstgeborene baarde een zoon, en noemde zijn naam Moab; deze is de vader der Moabieten, tot op dezen dag.
\par 38 En de jongste baarde ook een zoon, en noemde zijn naam Ben-ammi; deze is de vader der kinderen Ammons, tot op dezen dag.

\chapter{20}

\par 1 En Abraham reisde van daar naar het land van het zuiden, en woonde tussen Kades en tussen Sur; en hij verkeerde als vreemdeling te Gerar.
\par 2 Als nu Abraham van Sara, zijn huisvrouw, gezegd had: Zij is mijn zuster, zo zond Abimelech, de koning van Gerar, en nam Sara weg.
\par 3 Maar God kwam tot Abimelech in een droom des nachts, en Hij zeide tot hem: Zie, gij zijt dood om der vrouwe wil, die gij weggenomen hebt; want zij is met een man getrouwd.
\par 4 Doch Abimelech was tot haar niet genaderd; daarom zeide hij: Heere! zult Gij dan ook een rechtvaardig volk doden?
\par 5 Heeft hij zelf mij niet gezegd: Zij is mijn zuster? en zij, ook zij heeft gezegd: Hij is mijn broeder. In oprechtheid mijns harten en in reinheid mijner handen, heb ik dit gedaan.
\par 6 En God zeide tot hem in den droom: Ik heb ook geweten, dat gij dit in oprechtheid uws harten gedaan hebt, en Ik heb u ook belet van tegen Mij te zondigen; daarom heb Ik u niet toegelaten, haar aan te roeren.
\par 7 Zo geef dan nu dezes mans huisvrouw weder; want hij is een profeet, en hij zal voor u bidden, opdat gij leeft; maar zo gij haar niet wedergeeft, weet, dat gij voorzeker sterven zult, gij, en al wat uwes is!
\par 8 Toen stond Abimelech des morgens vroeg op, en riep al zijn knechten, en sprak al deze woorden voor hun oren. En die mannen vreesden zeer.
\par 9 En Abimelech riep Abraham, en zeide tot hem: Wat hebt gij ons gedaan? en wat heb ik tegen u gezondigd, dat gij over mij en over mijn koninkrijk een grote zonde gebracht hebt? gij hebt daden met mij gedaan, die niet zouden gedaan worden.
\par 10 Voorts zeide Abimelech tot Abraham: Wat hebt gij gezien, dat gij deze zaak gedaan hebt?
\par 11 En Abraham zeide: Want ik dacht: alleen is de vreze Gods in deze plaats niet, zodat zij mij om mijner huisvrouw wil zullen doden.
\par 12 En ook is zij waarlijk mijn zuster; zij is mijns vaders dochter, maar niet mijner moeder dochter; en zij is mij ter vrouwe geworden.
\par 13 En het is geschied, als God mij uit mijns vaders huis deed dwalen, zo sprak ik tot haar: Dit zij uw weldadigheid, die gij bij mij doen zult; aan alle plaatsen waar wij komen zullen, zeg van mij: Hij is mijn broeder!
\par 14 Toen nam Abimelech schapen en runderen, ook dienstknechten en dienstmaagden, en gaf dezelve aan Abraham; en hij gaf hem Sara zijn huisvrouw weder.
\par 15 En Abimelech zeide: Zie, mijn land is voor uw aangezicht; woon, waar het goed is in uw ogen.
\par 16 En tot Sara zeide hij: Zie, ik heb uw broeder duizend zilverlingen gegeven; zie, hij zij u een deksel der ogen, allen, die met u zijn, ja, bij allen, en wees geleerd.
\par 17 En Abraham bad tot God; en God genas Abimelech, en zijn huisvrouw, en zijn dienstmaagden, zodat zij baarden.
\par 18 Want de HEERE had al de baarmoeders van het huis van Abimelech ganselijk toegesloten, ter oorzake van Sara, Abrahams huisvrouw.

\chapter{21}

\par 1 En de HEERE bezocht Sara, gelijk als Hij gezegd had; en de HEERE deed aan Sara, gelijk als Hij gesproken had.
\par 2 En Sara werd bevrucht, en baarde Abraham een zoon in zijn ouderdom, ter gezetter tijd, dien hem God gezegd had.
\par 3 En Abraham noemde den naam zijns zoons, die hem geboren was, dien hem Sara gebaard had, Izak.
\par 4 En Abraham besneed zijn zoon Izak, zijnde acht dagen oud, gelijk als hem God geboden had.
\par 5 En Abraham was honderd jaren oud, als hem Izak zijn zoon geboren werd.
\par 6 En Sara zeide: God heeft mij een lachen gemaakt; al die het hoort, zal met mij lachen.
\par 7 Voorts zeide zij: Wie zou Abraham gezegd hebben: Sara heeft zonen gezoogd? want ik heb een zoon gebaard in zijn ouderdom.
\par 8 En het kind werd groot, en werd gespeend; toen maakte Abraham een groten maaltijd op den dag, als Izak gespeend werd.
\par 9 En Sara zag den zoon van Hagar, de Egyptische, dien zij Abraham gebaard had, spottende.
\par 10 En zij zeide tot Abraham: Drijf deze dienstmaagd en haar zoon uit; want de zoon dezer dienstmaagd zal met mijn zoon, met Izak, niet erven.
\par 11 En dit woord was zeer kwaad in Abrahams ogen, ter oorzake van zijn zoon.
\par 12 Maar God zeide tot Abraham: Laat het niet kwaad zijn in uw ogen, over den jongen, en over uw dienstmaagd; al wat Sara tot u zal zeggen, hoor naar haar stem; want in Izak zal uw zaad genoemd worden.
\par 13 Doch Ik zal ook den zoon dezer dienstmaagd tot een volk stellen, omdat hij uw zaad is.
\par 14 Toen stond Abraham des morgens vroeg op, en nam brood, en een fles water, en gaf ze aan Hagar, die leggende op haar schouder; ook gaf hij haar het kind, en zond haar weg. En zij ging voort, en dwaalde in de woestijn Ber-seba.
\par 15 Als nu het water van de fles uit was, zo wierp zij het kind onder een van de struiken.
\par 16 En zij ging en zette zich tegenover, afgaande zo verre, als die met den boog schieten; want zij zeide: Dat ik het kind niet zie sterven; en zij zat tegenover, en hief haar stem op, en weende.
\par 17 En God hoorde de stem van den jongen; en de Engel Gods riep Hagar toe uit den hemel, en zeide tot haar: Wat is u, Hagar? Vrees niet; want God heeft naar des jongens stem gehoord, ter plaatse, waar hij is.
\par 18 Sta op, hef den jongen op, en houd hem vast met uwe hand; want Ik zal hem tot een groot volk stellen.
\par 19 En God opende haar ogen, dat zij een waterput zag; en zij ging, en vulde de fles met water, en gaf den jongen te drinken.
\par 20 En God was met den jongen; en hij werd groot, en hij woonde in de woestijn, en werd een boogschutter.
\par 21 En hij woonde in de woestijn Paran; en zijn moeder nam hem een vrouw uit Egypteland.
\par 22 Voorts geschiedde het ter zelfder tijd, dat Abimelech, mitsgaders Pichol, zijn krijgsoverste, tot Abraham sprak, zeggende: God is met u in alles, wat gij doet.
\par 23 Zo zweer mij nu hier bij God: Zo gij mij, of mijn zoon, of mijn neef liegen zult! naar de weldadigheid, die ik bij u gedaan heb, zult gij doen bij mij, en bij het land, waarin gij als vreemdeling verkeert.
\par 24 En Abraham zeide: Ik zal zweren.
\par 25 En Abraham berispte Abimelech ter oorzake van een waterput, die Abimelechs knechten met geweld genomen hadden.
\par 26 Toen zeide Abimelech: Ik heb niet geweten, wie dit stuk gedaan heeft; en ook hebt gij het mij niet aangezegd, en ik heb er ook niet van gehoord, dan heden.
\par 27 En Abraham nam schapen en runderen, en gaf die aan Abimelech; en die beiden maakten een verbond.
\par 28 Doch Abraham stelde zeven ooilammeren der kudde bijzonder.
\par 29 Zo zeide Abimelech tot Abraham: Wat zullen hier deze zeven ooilammeren, die gij bijzonder gesteld hebt?
\par 30 En hij zeide: Dat gij de zeven ooilammeren van mijn hand nemen zult, opdat het mij tot een getuigenis zij, dat ik dezen put gegraven heb.
\par 31 Daarom noemde men die plaats Ber-seba, omdat die beiden daar gezworen hadden.
\par 32 Alzo maakten zij een verbond te Ber-seba. Daarna stond Abimelech op, en Pichol, zijn krijgsoverste, en zij keerden wederom naar het land der Filistijnen.
\par 33 En hij plantte een bos in Ber-seba, en riep aldaar den Naam des HEEREN, des eeuwigen Gods, aan.
\par 34 En Abraham woonde als vreemdeling vele dagen in het land der Filistijnen.

\chapter{22}

\par 1 En het geschiedde na deze dingen, dat God Abraham verzocht; en Hij zeide tot hem: Abraham! En hij zeide: Zie, hier ben ik!
\par 2 En Hij zeide: Neem nu uw zoon, uw enige, dien gij liefhebt, Izak, en ga heen naar het land Moria, en offer hem aldaar tot een brandoffer, op een van de bergen, dien Ik u zeggen zal.
\par 3 Toen stond Abraham des morgens vroeg op, en zadelde zijn ezel, en nam twee van zijn jongeren met zich, en Izak zijn zoon; en hij kloofde hout tot het brandoffer, en maakte zich op, en ging naar de plaats, die God hem gezegd had.
\par 4 Aan den derden dag, toen hief Abraham zijn ogen op, en zag die plaats van verre.
\par 5 En Abraham zeide tot zijn jongeren: Blijft gij hier met den ezel, en ik en de jongen zullen heengaan tot daar; als wij aangebeden zullen hebben, dan zullen wij tot u wederkeren.
\par 6 En Abraham nam het hout des brandoffers, en legde het op Izak, zijn zoon; en hij nam het vuur en het mes in zijn hand, en zij beiden gingen samen.
\par 7 Toen sprak Izak tot Abraham, zijn vader, en zeide: Mijn vader! En hij zeide: Zie, hier ben ik, mijn zoon! En hij zeide: Zie het vuur en het hout; maar waar is het lam tot het brandoffer?
\par 8 En Abraham zeide: God zal Zichzelven een lam ten brandoffer voorzien, mijn zoon! Zo gingen zij beiden samen.
\par 9 En zij kwamen ter plaatse, die hem God gezegd had; en Abraham bouwde aldaar een altaar, en hij schikte het hout, en bond zijn zoon Izak, en leide hem op het altaar boven op het hout.
\par 10 En Abraham strekte zijn hand uit, en nam het mes om zijn zoon te slachten.
\par 11 Maar de Engel des HEEREN riep tot hem van den hemel, en zeide: Abraham, Abraham! En hij zeide: Zie, hier ben ik!
\par 12 Toen zeide Hij: Strek uw hand niet uit aan den jongen, en doe hem niets! want nu weet Ik, dat gij God vrezende zijt, en uw zoon, uw enige, van Mij niet hebt onthouden.
\par 13 Toen hief Abraham zijn ogen op, en zag om, en ziet, achter was een ram in de verwarde struiken vast met zijn hoornen; en Abraham ging, en nam dien ram, en offerde hem ten brandoffer in zijns zoons plaats.
\par 14 En Abraham noemde den naam van die plaats: De HEERE zal het voorzien! Waarom heden ten dage gezegd wordt: Op den berg des HEEREN zal het voorzien worden!
\par 15 Toen riep de Engel des HEEREN tot Abraham ten tweeden male van den hemel;
\par 16 En zeide: Ik zweer bij Mijzelven, spreekt de HEERE; daarom dat gij deze zaak gedaan hebt, en uw zoon, uw enige, niet onthouden hebt;
\par 17 Voorzeker zal Ik u grotelijks zegenen, en uw zaad zeer vermenigvuldigen, als de sterren des hemels, en als het zand, dat aan den oever der zee is; en uw zaad zal de poort zijner vijanden erfelijk bezitten.
\par 18 En in uw zaad zullen gezegend worden alle volken der aarde, naardien gij Mijn stem gehoorzaam geweest zijt.
\par 19 Toen keerde Abraham weder tot zijn jongeren, en zij maakten zich op, en zij gingen samen naar Ber-seba; en Abraham woonde te Ber-seba.
\par 20 En het geschiedde na deze dingen, dat men Abraham boodschapte, zeggende: Zie, Milka heeft ook Nahor, uw broeder, zonen gebaard:
\par 21 Uz, zijn eerstgeborene, en Buz, zijn broeder, en Kemuel, den vader van Aram,
\par 22 En Chesed, en Hazo, en Pildas, en Jidlaf, en Bethuel;
\par 23 (En Bethuel gewon Rebekka) deze acht baarde Milka aan Nahor, den broeder van Abraham.
\par 24 En zijn bijwijf, welker naam was Reuma, diezelve baarde ook Tebah, en Gaham, en Tahas, en Maacha.

\chapter{23}

\par 1 En het leven van Sara was honderd zeven en twintig jaren; dit waren de jaren des levens van Sara.
\par 2 En Sara stierf te Kiriath-arba, dat is Hebron, in het land Kanaan; en Abraham kwam om Sara te beklagen, en haar te bewenen.
\par 3 Daarna stond Abraham op van het aangezicht van zijner dode, en hij sprak tot de zonen Heths, zeggende:
\par 4 Ik ben een vreemdeling en inwoner bij u; geeft mij een erfbegrafenis bij u, opdat ik mijn dode van voor mijn aangezicht begrave.
\par 5 En de zonen Heths antwoordden Abraham, zeggende tot hem:
\par 6 Hoor ons, mijn heer! gij zijt een vorst Gods in het midden van ons; begraaf uw dode in de keure onzer graven; niemand van ons zal zijn graf voor u weren, dat gij uw dode niet zoudt begraven.
\par 7 Toen stond Abraham op, en boog zich neder voor het volk des lands, voor de zonen Heths;
\par 8 En hij sprak met hen, zeggende: Is het met uw wil, dat ik mijn dode begrave van voor mijn aangezicht; zo hoort mij, en spreekt voor mij bij Efron, den zoon van Zohar,
\par 9 Dat hij mij geve de spelonk van Machpela, die hij heeft, die in het einde van zijn akker is, dat hij dezelve mij om het volle geld geve, tot een erfbegrafenis in het midden van u.
\par 10 Efron nu zat in het midden van de zonen Heths; en Efron de Hethiet antwoordde Abraham, voor de oren van de zonen Heths, van al degenen, die ter poorte zijner stad ingingen, zeggende:
\par 11 Neen, mijn heer! hoor mij; den akker geef ik u; ook de spelonk, die daarin is, die geef ik u; voor de ogen van de zonen mijns volks geef ik u die; begraaf uw dode.
\par 12 Toen boog zich Abraham neder voor het aangezicht van het volk des lands;
\par 13 En hij sprak tot Efron, voor de oren van het volk des lands, zeggende: Trouwens, zijt gij het? lieve, hoor mij; ik zal het geld des akkers geven; neem het van mij, zo zal ik mijn dode aldaar begraven.
\par 14 En Efron antwoordde Abraham, zeggende tot hem:
\par 15 Mijn heer! hoor mij; een land van vierhonderd sikkelen zilvers, wat is dat tussen mij en tussen u? begraaf slechts uw dode.
\par 16 En Abraham luisterde naar Efron; en Abraham woog Efron het geld, waarvan hij gesproken had voor de oren van de zonen Heths, vierhonderd sikkelen zilvers, onder den koopman gangbaar.
\par 17 Alzo werd de akker van Efron, die in Machpela was, dat tegenover Mamre lag, de akker en de spelonk, die daarin was, en al het geboomte, dat op den akker stond, dat rondom in zijn ganse landpale was gevestigd,
\par 18 Aan Abraham tot een bezitting, voor de ogen van de zonen Heths, bij allen, die tot zijn stadspoort ingingen.
\par 19 En daarna begroef Abraham zijn huisvrouw Sara in de spelonk des akkers van Machpela, tegenover Mamre, hetwelk is Hebron, in het land Kanaan.
\par 20 Alzo werd die akker, en de spelonk die daarin was, aan Abraham gevestigd tot een erfbegrafenis van de zonen Heths.

\chapter{24}

\par 1 Abraham nu was oud en wel bedaagd; en de HEERE had Abraham in alles gezegend.
\par 2 Zo sprak Abraham tot zijn knecht, den oudste van zijn huis, regerende over alles, wat hij had: Leg toch uw hand onder mijn heup,
\par 3 Opdat ik u doe zweren bij den HEERE, den God des hemels, en den God der aarde, dat gij voor mijn zoon geen vrouw nemen zult van de dochteren der Kanaanieten, in het midden van welke ik woon;
\par 4 Maar dat gij naar mijn land, en naar mijn maagschap trekken, en voor mijn zoon Izak een vrouw nemen zult.
\par 5 En die knecht zeide tot hem: Misschien zal die vrouw mij niet willen volgen in dit land; zal ik dan uw zoon moeten wederbrengen in het land, waar gij uitgetogen zijt?
\par 6 En Abraham zeide tot hem: Wacht u, dat gij mijn zoon niet weder daarheen brengt!
\par 7 De HEERE, de God des hemels, Die mij uit mijns vaders huis en uit het land mijner maagschap genomen heeft, en Die tot mij gesproken heeft, en Die mij gezworen heeft, zeggende: Aan uw zaad zal Ik dit land geven! Die Zelf zal Zijn Engel voor uw aangezicht zenden, dat gij voor mijn zoon van daar een vrouw neemt.
\par 8 Maar indien de vrouw u niet volgen wil, zo zult gij rein zijn van dezen mijn eed; alleenlijk breng mijn zoon daar niet weder heen.
\par 9 Toen legde de knecht zijn hand onder de heup van Abraham, zijn heer, en hij zwoer hem over deze zaak.
\par 10 En die knecht nam tien kemelen van zijns heren kemelen, en toog heen; en al het goed zijns heren was in zijn hand; en hij maakte zich op, en toog heen naar Mesopotamie, naar de stad van Nahor.
\par 11 En hij deed de kemelen nederknielen buiten de stad, bij een waterput, des avondtijds, ten tijde, als de putsters uitkwamen.
\par 12 En hij zeide: HEERE! God van mijn heer Abraham! doe haar mij toch heden ontmoeten, en doe weldadigheid bij Abraham, mijn heer.
\par 13 Zie, ik sta bij de waterfontein, en de dochteren der mannen dezer stad zijn uitgaande om water te putten;
\par 14 Zo geschiede, dat die jonge dochter, tot welke ik zal zeggen: Neig toch uw kruik, dat ik drinke; en zij zal zeggen: Drink, en ik zal ook uw kemelen drenken; diezelve zij, die Gij Uw knecht Izak toegewezen hebt, en dat ik daaraan bekenne, dat Gij weldadigheid bij mijn heer gedaan hebt.
\par 15 En het geschiedde, eer hij geeindigd had te spreken, ziet, zo kwam Rebekka uit, welke aan Bethuel geboren was, de zoon van Milka, de huisvrouw van Nahor, de broeder van Abraham; en zij had haar kruik op haar schouder.
\par 16 En die jonge dochter was zeer schoon van aangezicht, een maagd, en geen man had haar bekend; en zij ging af naar de fontein, en vulde haar kruik, en ging op.
\par 17 Toen liep die knecht haar tegemoet, en hij zeide: Laat mij toch een weinig waters uit uw kruik drinken.
\par 18 En zij zeide: Drink, mijn heer! en zij haastte zich en liet haar kruik neder op haar hand, en gaf hem te drinken.
\par 19 Als zij nu voleindigd had van hem drinken te geven, zeide zij: Ik zal ook voor uw kemelen putten, totdat zij voleindigd hebben te drinken.
\par 20 En zij haastte zich, en goot haar kruik uit in de drinkbak, en liep weder naar den put om te putten, en zij putte voor al zijn kemelen.
\par 21 En de man ontzette zich over haar, stilzwijgende, om te merken, of de HEERE zijn weg voorspoedig gemaakt had, of niet.
\par 22 En het geschiedde, als de kemelen voleindigd hadden te drinken, dat die man een gouden voorhoofdsiersel nam, welks gewicht was een halve sikkel, en twee armringen aan haar handen, welker gewicht was tien sikkelen gouds.
\par 23 Want hij had gezegd: Wiens dochter zijt gij? geef het mij toch te kennen; is er ook ten huize uws vaders plaats voor ons, om te vernachten?
\par 24 En zij had tot hem gezegd: Ik ben de dochter van Bethuel, den zoon van Milka, die zij Nahor gebaard heeft.
\par 25 Voorts had zij tot hem gezegd: Ook is er stro en veel voeders bij ons, ook plaats om te vernachten.
\par 26 Toen neigde die man zijn hoofd, en aanbad den HEERE;
\par 27 En hij zeide: Geloofd zij de HEERE, de God van mijn heer Abraham, Die Zijn weldadigheid en waarheid niet nagelaten heeft van mijn heer; aangaande mij, de HEERE heeft mij op dezen weg geleid, ten huize van mijns heren broederen.
\par 28 En die jonge dochter liep, en gaf ten huize harer moeder te kennen, gelijk deze zaken waren.
\par 29 En Rebekka had een broeder, wiens naam was Laban; en Laban liep tot dien man naar buiten tot de fontein.
\par 30 En het geschiedde, als hij dat voorhoofdsiersel gezien had, en de armringen aan de handen zijner zuster; en als hij gehoord had de woorden zijner zuster Rebekka, zeggende: Alzo heeft die man tot mij gesproken, zo kwam hij tot dien man, en ziet, hij stond bij de kemelen, bij de fontein.
\par 31 En hij zeide: Kom in, gij, gezegende des HEEREN! waarom zoudt gij buiten staan? want ik heb het huis bereid, en de plaats voor de kemelen.
\par 32 Toen kwam die man naar het huis toe, en men ontgordde de kemelen, en men gaf den kemelen stro en voeder; en water om zijn voeten te wassen, en de voeten der mannen, die bij hem waren.
\par 33 Daarna werd hem te eten voorgezet; maar hij zeide: Ik zal niet eten, totdat ik mijn woorden gesproken heb. En hij zeide: Spreek!
\par 34 Toen zeide hij: Ik ben een knecht van Abraham;
\par 35 En de HEERE heeft mijn heer zeer gezegend, zodat hij groot geworden is; en Hij heeft hem gegeven schapen, en runderen, en zilver, en goud, en knechten, en maagden, en kemelen, en ezelen.
\par 36 En Sara, de huisvrouw van mijn heer, heeft mijn heer een zoon gebaard, nadat zij oud geworden was; en hij heeft hem gegeven alles, wat hij heeft.
\par 37 En mijn heer heeft mij doen zweren, zeggende: Gij zult voor mijn zoon geen vrouw nemen van de dochteren der Kanaanieten, in welker land ik wone;
\par 38 Maar gij zult trekken naar het huis mijns vaders, en naar mijn geslacht, en zult voor mijn zoon een vrouw nemen!
\par 39 Toen zeide ik tot mijn heer: Misschien zal mij de vrouw niet volgen.
\par 40 En hij zeide tot mij: De HEERE, voor Wiens aangezicht ik gewandeld heb, zal Zijn Engel met u zenden, en Hij zal uw weg voorspoedig maken, dat gij voor mijn zoon een vrouw neemt, uit mijn geslacht en uit mijns vaders huis.
\par 41 Dan zult gij van mijn eed rein zijn, wanneer gij tot mijn geslacht zult gegaan zijn; en indien zij haar u niet geven, zo zult gij rein zijn van mijn eed.
\par 42 En ik kwam heden aan de fontein; en ik zeide: O, HEERE! God van mijn heer Abraham! zo Gij nu mijn weg voorspoedig maken zult, op welke ik ga;
\par 43 Zie, ik sta bij de waterfontein; zo geschiede, dat de maagd, die uitkomen zal om te putten, en tot welke ik zeggen zal: Geef mij toch een weinig waters te drinken uit uw kruik;
\par 44 En zij tot mij zal zeggen: Drink gij ook, en ik zal ook uw kemelen putten; dat deze die vrouw zij, die de HEERE aan den zoon van mijn heer heeft toegewezen.
\par 45 Eer ik geeindigd had te spreken in mijn hart, ziet, zo kwam Rebekka uit, en had haar kruik op haar schouder, en zij kwam af tot de fontein en putte; en ik zeide tot haar: Geef mij toch te drinken!
\par 46 Zo haastte zij zich en liet haar kruik van zich neder, en zeide: Drink gij, en ik zal ook uw kemelen drenken; en ik dronk, en zij drenkte ook de kemelen.
\par 47 Toen vraagde ik haar, en zeide: Wiens dochter zijt gij? En zij zeide: De dochter van Bethuel, den zoon van Nahor, welken Milka hem gebaard heeft. Zo leide ik het voorhoofdsiersel op haar aangezicht, en de armringen aan haar handen;
\par 48 En ik neigde mijn hoofd, en aanbad den HEERE; en ik loofde den HEERE, den God van mijn heer Abraham, Die mij op den rechten weg geleid had, om de dochter des broeders van mijn heer voor zijn zoon te nemen.
\par 49 Nu dan, zo gijlieden weldadigheid en trouw aan mijn heer doen zult, geeft het mij te kennen; en zo niet, geeft het mij ook te kennen, opdat ik mij ter rechter hand of ter linkerhand wende.
\par 50 Toen antwoordde Laban, en Bethuel, en zeiden: Van den HEERE is deze zaak voortgekomen; wij kunnen kwaad noch goed tot u spreken.
\par 51 Zie, Rebekka is voor uw aangezicht; neem haar en trek henen; zij zij de vrouw van den zoon uws heren, gelijk de HEERE gesproken heeft!
\par 52 En het geschiedde, als Abrahams knecht hun woorden hoorde, zo boog hij zich ter aarde voor den HEERE.
\par 53 En de knecht langde voort zilveren kleinoden, en gouden kleinoden, en klederen, en hij gaf die aan Rebekka; hij gaf ook aan haar moeder kostelijkheden.
\par 54 Toen aten en dronken zij, hij en de mannen, die bij hem waren; en zij vernachtten, en zij stonden des morgens op, en hij zeide: Laat mij trekken tot mijn heer!
\par 55 Toen zeide haar broeder, en haar moeder: Laat de jonge dochter enige dagen, of tien, bij ons blijven; daarna zult gij gaan.
\par 56 Maar hij zeide tot hen: Houdt mij niet op, dewijl de HEERE mijn weg voorspoedig gemaakt heeft! laat mij trekken, dat ik tot mijn heer ga.
\par 57 Toen zeiden zij: Laat ons de jonge dochter roepen, en haar mond vragen.
\par 58 En zij riepen Rebekka, en zeiden tot haar: Zult gij met dezen man trekken? En zij antwoordde: Ik zal trekken.
\par 59 Toen lieten zij Rebekka, hun zuster, en haar voedster trekken, mitsgaders Abrahams knecht en zijn mannen.
\par 60 En zij zegenden Rebekka, en zeiden tot haar: O, onze zuster! wordt gij tot duizenden millioenen, en uw zaad bezitte de poort zijner haters!
\par 61 En Rebekka maakte zich op met haar jonge dochteren, en zij reden op kemelen, en volgden den man; en die knecht nam Rebekka, en toog heen.
\par 62 Izak nu kwam, van daar men komt tot den put Lachai-roi; en hij woonde in het zuiderland.
\par 63 En Izak was uitgegaan om te bidden in het veld, tegen het naken van den avond; en hij hief zijn ogen op en zag toe, en ziet, de kemelen kwamen!
\par 64 Rebekka hief ook haar ogen op, en zij zag Izak; en zij viel van den kemel af.
\par 65 En zij zeide tot den knecht: Wie is die man, die ons in het veld tegemoet wandelt? En de knecht zeide: Dat is mijn heer! Toen nam zij den sluier, en bedekte zich.
\par 66 En de knecht vertelde aan Izak al de zaken, die hij gedaan had.
\par 67 En Izak bracht haar in de tent van zijn moeder Sara; en hij nam Rebekka, en zij werd hem ter vrouw, en hij had haar lief. Alzo werd Izak getroost na zijner moeders dood.

\chapter{25}

\par 1 En Abraham voer voort, en nam een vrouw, wier naam was Ketura.
\par 2 En zij baarde hem Zimran en Joksan, en Medan en Midian, en Jisbak en Suah.
\par 3 En Joksan gewon Seba en Dedan; en de zonen van Dedan waren de Assurieten, en Letusieten, en Leummieten.
\par 4 En de zonen van Midian waren Efa en Efer, en Henoch en Abida, en Eldaa. Deze allen waren zonen van Ketura.
\par 5 Doch Abraham gaf aan Izak al wat hij had.
\par 6 Maar aan de zonen der bijwijven, die Abraham had, gaf Abraham geschenken; en zond hen weg van zijn zoon Izak, terwijl hij nog leefde, oostwaarts naar het land van het Oosten.
\par 7 Dit nu zijn de dagen der jaren des levens van Abraham, welke hij geleefd heeft, honderd vijf en zeventig jaren.
\par 8 En Abraham gaf den geest en stierf, in goeden ouderdom, oud en des levens zat, en hij werd tot zijn volken verzameld.
\par 9 En Izak en Ismael, zijn zonen, begroeven hem, in de spelonk van Machpela, in den akker van Efron, den zoon van Zohar, den Hethiet, welke tegenover Mamre is;
\par 10 In den akker, dien Abraham van de zonen Heths gekocht had, daar is Abraham begraven, en Sara, zijn huisvrouw.
\par 11 En het geschiedde na Abrahams dood, dat God Izak, zijn zoon, zegende; en Izak woonde bij den put Lachai-roi.
\par 12 Dit nu zijn de geboorten van Ismael, den zoon van Abraham, dien Hagar, de Egyptische, dienstmaagd van Sara, Abraham gebaard heeft.
\par 13 En dit zijn de namen der zonen van Ismael, met hun namen naar hun geboorten. De eerstgeborene van Ismael, Nabajoth; daarna Kedar, en Adbeel, en Mibsam,
\par 14 En Misma, en Duma, en Massa,
\par 15 Hadar en Thema, Jetur, Nafis en Kedma.
\par 16 Deze zijn de zonen van Ismael, en dit zijn hun namen, in hun dorpen en paleizen, twaalf vorsten naar hun volken.
\par 17 En dit zijn de jaren des levens van Ismael, honderd zeven en dertig jaren; en hij gaf den geest, en stierf, en hij werd verzameld tot zijn volken.
\par 18 En zij woonden van Havila tot Sur toe, hetwelk tegenover Egypte is, daar gij gaat naar Assur; hij heeft zich nedergeslagen voor het aangezicht van al zijn broederen.
\par 19 Dit nu zijn de geboorten van Izak, den zoon van Abraham: Abraham gewon Izak.
\par 20 En Izak was veertig jaren oud, als hij Rebekka, de dochter van Bethuel, den Syrier, uit Paddan-aram, de zuster van Laban, den Syrier, zich ter vrouw nam.
\par 21 En Izak bad den HEERE zeer in de tegenwoordigheid van zijn huisvrouw; want zij was onvruchtbaar; en de HEERE liet zich van hem verbidden, zodat Rebekka, zijn huisvrouw, zwanger werd.
\par 22 En de kinderen stieten zich samen in haar lichaam. Toen zeide zij: Is het zo? waarom ben ik dus? en zij ging om den HEERE te vragen.
\par 23 En de HEERE zeide tot haar: Twee volken zijn in uw buik, en twee natien zullen zich uit uw ingewand van een scheiden; en het ene volk zal sterker zijn dan het andere volk; en de meerdere zal den mindere dienen.
\par 24 Als nu haar dagen vervuld waren om te baren, ziet, zo waren tweelingen in haar buik.
\par 25 En de eerste kwam uit, ros; hij was geheel als een haren kleed; daarom noemden zij zijn naam Ezau.
\par 26 En daarna kwam zijn broeder uit, wiens hand Ezau's verzenen hield; daarom noemde men zijn naam Jakob. En Izak was zestig jaren oud, als hij hen gewon.
\par 27 Als nu deze jongeren groot werden, werd Ezau een man, verstandig op de jacht, een veldman; maar Jakob werd een oprecht man, wonende in tenten.
\par 28 En Izak had Ezau lief; want het wildbraad was naar zijn mond; maar Rebekka had Jakob lief.
\par 29 En Jakob had een kooksel gekookt; en Ezau kwam uit het veld, en was moede.
\par 30 En Ezau zeide tot Jakob: Laat mij toch slorpen van dat rode, dat rode daar, want ik ben moede; daarom heeft men zijn naam genoemd Edom.
\par 31 Toen zeide Jakob: Verkoop mij op dezen dag uw eerstgeboorte.
\par 32 En Ezau zeide: Zie, ik ga sterven; en waartoe mij dan de eerstgeboorte?
\par 33 Toen zeide Jakob: Zweer mij op dezen dag! en hij zwoer hem; en hij verkocht aan Jakob zijn eerstgeboorte.
\par 34 En Jakob gaf aan Ezau brood, en het linzenkooksel; en hij at en dronk, en hij stond op en ging heen; alzo verachtte Ezau de eerstgeboorte.

\chapter{26}

\par 1 En er was honger in dat land, behalve den eersten honger, die in de dagen van Abraham geweest was; daarom toog Izak tot Abimelech, de koning der Filistijnen, naar Gerar.
\par 2 En de HEERE verscheen hem en zeide: Trek niet af naar Egypte; woon in het land, dat Ik u aanzeggen zal;
\par 3 Woon als vreemdeling in dat land, en Ik zal met u zijn, en zal u zegenen; want aan u en uw zaad zal Ik al deze landen geven, en Ik zal den eed bevestigen, dien Ik Abraham uw vader gezworen heb.
\par 4 En Ik zal uw zaad vermenigvuldigen, als de sterren des hemels, en zal aan uw zaad al deze landen geven; en in uw zaad zullen gezegend worden alle volken der aarde,
\par 5 Daarom dat Abraham Mijn stem gehoorzaam geweest is, en heeft onderhouden Mijn bevel, Mijn geboden, Mijn inzettingen en Mijn wetten.
\par 6 Alzo woonde Izak te Gerar.
\par 7 En als de mannen van die plaats hem vraagden van zijn huisvrouw, zeide hij: Zij is mijn zuster; want hij vreesde te zeggen, mijn huisvrouw; opdat mij misschien, zeide hij de mannen dezer plaats niet doden, om Rebekka; want zij was schoon van aangezicht.
\par 8 En het geschiedde, als hij een langen tijd daar geweest was, dat Abimelech, de koning der Filistijnen, ten venster uitkeek, en hij zag, dat, ziet, Izak was jokkende met Rebekka zijn huisvrouw.
\par 9 Toen riep Abimelech Izak, en zeide: Voorwaar, zie, zij is uw huisvrouw! hoe hebt gij dan gezegd: Zij is mijn zuster? En Izak zeide tot hem: Want ik zeide: Dat ik niet misschien om harentwil sterve.
\par 10 En Abimelech zeide: Wat is dit, dat gij ons gedaan hebt? Lichtelijk had een van dit volk bij uw huisvrouw gelegen, zodat gij een schuld over ons zoudt gebracht hebben.
\par 11 En Abimelech gebood het ganse volk, zeggende: Zo wie deze man of zijn huisvrouw aanroert, zal voorzeker gedood worden!
\par 12 En Izak zaaide in datzelve land, en hij vond in datzelve jaar honderd maten; want de HEERE zegende hem.
\par 13 En die man werd groot, ja, hij werd doorgaans groter, totdat hij zeer groot geworden was.
\par 14 En hij had bezitting van schapen, en bezitting van runderen, en groot gezin; zodat hem de Filistijnen benijdden.
\par 15 En al de putten, die de knechten van zijn vader, in de dagen van zijn vader Abraham, gegraven hadden, die stopten de Filistijnen, en vulden dezelve met aarde.
\par 16 Ook zeide Abimelech tot Izak: Trek van ons; want gij zijt veel machtiger geworden, dan wij.
\par 17 Toen toog Izak van daar, en hij legerde zich in het dal van Gerar, en woonde aldaar.
\par 18 Als nu Izak wedergekeerd was, groef hij die waterputten op, die zij ten tijde van Abraham, zijn vader, gegraven, en die de Filistijnen na Abrahams dood toegestopt hadden; en hij noemde derzelver namen naar de namen, waarmede zijn vader die genoemd had.
\par 19 De knechten van Izak dan groeven in dat dal, en zij vonden aldaar een put van levend water.
\par 20 En de herders van Gerar twistten met Izaks herders, zeggende: Dit water hoort ons toe! Daarom noemde hij den naam van dien put Esek, omdat zij met hem gekeven hadden.
\par 21 Toen groeven zij een anderen put, en daar twistten zij ook over; daarom noemde hij deszelfs naam Sitna.
\par 22 En hij brak op van daar, en groef een anderen put, en zij twistten over dien niet; daarom noemde hij deszelfs naam Rehoboth, en zeide: Want nu heeft ons de HEERE ruimte gemaakt, en wij zijn gewassen in dit land.
\par 23 Daarna toog hij van daar op naar Ber-seba.
\par 24 En de HEERE verscheen hem in denzelven nacht, en zeide: Ik ben de God van Abraham, uw vader; vrees niet; want Ik ben met u; en Ik zal u zegenen, en uw zaad vermenigvuldigen, om Abrahams, Mijns knechts, wil.
\par 25 Toen bouwde hij daar een altaar, en riep den Naam des HEEREN aan. En hij sloeg aldaar zijn tent op; en Izaks knechten groeven daar een put.
\par 26 En Abimelech trok tot hem van Gerar, met Ahuzzat, zijn vriend, en Pichol, zijn krijgsoverste.
\par 27 En Izak zeide tot hen: Waarom zijt gij tot mij gekomen, daar gij mij haat, en hebt mij van u weggezonden?
\par 28 En zij zeiden: Wij hebben merkelijk gezien, dat de HEERE met u is; daarom hebben wij gezegd: Laat toch een eed tussen ons zijn, tussen ons en tussen u, en laat ons een verbond met u maken:
\par 29 Zo gij bij ons kwaad doet, gelijk als wij u niet aangeroerd hebben, en gelijk als wij bij u alleenlijk goed gedaan hebben, en hebben u in vrede laten trekken! Gij zijt nu de gezegende des HEEREN!
\par 30 Toen maakte hij hun een maaltijd, en zij aten en dronken.
\par 31 En zij stonden des morgens vroeg op, en zwoeren de een den ander; daarna liet Izak hen gaan, en zij togen van hem in vrede.
\par 32 En het geschiedde ten zelfden dage, dat Izaks knechten kwamen, en boodschapten hem van de zaak des puts, dien zij gegraven hadden, en zij zeiden hem: Wij hebben water gevonden.
\par 33 En hij noemde denzelven Seba; daarom is de naam dier stad Ber-seba, tot op dezen dag.
\par 34 Als nu Ezau veertig jaren oud was, nam hij tot een vrouw Judith, de dochter van Beeri, den Hethiet, en Basmath, de dochter van Elon, den Hethiet.
\par 35 En deze waren voor Izak en Rebekka een bitterheid des geestes.

\chapter{27}

\par 1 En het geschiedde, als Izak oud geworden was, en zijn ogen donker geworden waren, en hij niet zien kon; toen riep hij Ezau, zijn grootsten zoon, en zeide tot hem: Mijn zoon! En hij zeide tot hem: Zie, hier ben ik!
\par 2 En hij zeide: Zie nu, ik ben oud geworden, ik weet den dag mijns doods niet.
\par 3 Nu dan, neem toch uw gereedschap, uw pijlkoker en uw boog, en ga uit in het veld, en jaag mij een wildbraad;
\par 4 En maak mij smakelijke spijzen, zo als ik die gaarne heb, en breng ze mij, dat ik ete; opdat mijn ziel u zegene, eer ik sterve.
\par 5 Rebekka nu hoorde toe, als Izak tot zijn zoon Ezau sprak; en Ezau ging in het veld, om een wildbraad te jagen, dat hij het inbracht.
\par 6 Toen sprak Rebekka tot Jakob, haar zoon, zeggende: Zie, ik heb uw vader tot Ezau, uw broeder, horen spreken, zeggende:
\par 7 Breng mij een wildbraad, en maak mij smakelijke spijzen toe, dat ik ete; en ik zal u zegenen voor het aangezicht des HEEREN, voor mijn dood.
\par 8 Nu dan, mijn zoon! hoor mijn stem in hetgeen ik u gebiede.
\par 9 Ga nu heen tot de kudde, en haal mij van daar twee goede geitenbokjes; en ik zal die voor uw vader maken tot smakelijke spijzen, gelijk als hij gaarne heeft.
\par 10 En gij zult ze tot uw vader brengen, en hij zal eten, opdat hij u zegene voor zijn dood.
\par 11 Toen zeide Jakob tot Rebekka, zijn moeder: Zie, mijn broeder Ezau is een harig man, en ik ben een glad man.
\par 12 Misschien zal mij mijn vader betasten, en ik zal in zijn ogen zijn als een bedrieger; zo zoude ik een vloek over mij halen, en niet een zegen.
\par 13 En zijn moeder zeide tot hem: Uw vloek zij op mij, mijn zoon! hoor alleen naar mijn stem, en ga, haal ze mij.
\par 14 Toen ging hij, en hij haalde ze, en bracht ze zijn moeder; en zijn moeder maakte smakelijke spijzen, gelijk als zijn vader gaarne had.
\par 15 Daarna nam Rebekka de kostelijke klederen van Ezau, haar grootsten zoon, die zij bij zich in huis had, en zij trok ze Jakob, haar kleinsten zoon, aan.
\par 16 En de vellen van de geitenbokjes trok zij over zijn handen, en over de gladdigheid van zijn hals.
\par 17 En zij gaf de smakelijke spijzen, en het brood, welke zij toegemaakt had, in de hand van Jakob, haar zoon.
\par 18 En hij kwam tot zijn vader, en zeide: Mijn vader! En hij zeide: Zie, hier ben ik; wie zijt gij, mijn zoon?
\par 19 En Jakob zeide tot zijn vader: Ik ben Ezau uw eerstgeborene; ik heb gedaan, gelijk als gij tot mij gesproken hadt; sta toch op, zit, en eet van mijn wildbraad, opdat uw ziel mij zegene.
\par 20 Toen zeide Izak tot zijn zoon: Hoe is dit, dat gij het zo haast gevonden hebt, mijn zoon? En hij zeide: Omdat de HEERE uw God dat heeft doen ontmoeten voor mijn aangezicht.
\par 21 En Izak zeide tot Jakob: Nader toch, dat ik u betaste, mijn zoon! of gij mijn zoon Ezau zelf zijt, of niet.
\par 22 Toen kwam Jakob bij, tot zijn vader Izak, die hem betastte; en hij zeide: De stem is Jakobs stem, maar de handen zijn Ezau's handen.
\par 23 Doch hij kende hem niet, omdat zijn handen harig waren, gelijk zijns broeders Ezau's handen; en hij zegende hem.
\par 24 En hij zeide: Zijt gij mijn zoon Ezau zelf? En hij zeide: Ik ben het!
\par 25 Toen zeide hij: Stel het nabij mij, dat ik van het wildbraad mijns zoons ete, opdat mijn ziel u zegene. En hij stelde het nabij hem, en hij at; hij bracht hem ook wijn, en hij dronk.
\par 26 En zijn vader Izak zeide tot hem: Kom toch bij, en kus mij, mijn zoon!
\par 27 En hij kwam bij, en hij kuste hem; toen rook hij de reuk zijner klederen, en zegende hem; en hij zeide: Zie, de reuk mijns zoons is als de reuk des velds, hetwelk de HEERE gezegend heeft.
\par 28 Zo geve u dan God van den dauw des hemels, en de vettigheid der aarde, en menigte van tarwe en most.
\par 29 Volken zullen u dienen, en natien zullen zich voor u nederbuigen; wees heer over uw broederen, en de zonen uwer moeder zullen zich voor u nederbuigen! Vervloekt moet hij zijn, wie u vervloekt; en wie u zegent, zij gezegend!
\par 30 En het geschiedde, als Izak voleindigd had Jakob te zegenen, zo geschiedde het, toen Jakob maar even van het aangezicht van zijn vader Izak uitgegaan was, dat Ezau, zijn broeder, van zijn jacht kwam.
\par 31 Hij nu maakte smakelijke spijzen toe, en bracht die tot zijn vader; en hij zeide tot zijn vader: Mijn vader sta op en ete van het wildbraad zijns zoons, opdat uw ziel mij zegene.
\par 32 En Izak, zijn vader, zeide tot hem: Wie zijt gij? En hij zeide: Ik ben uw zoon, uw eerstgeborene, Ezau.
\par 33 Toen verschrikte Izak met zeer grote verschrikking, gans zeer, en zeide: Wie is hij dan, die het wildbraad gejaagd en tot mij gebracht heeft? en ik heb van alles gegeten, eer gij kwaamt, en heb hem gezegend; ook zal hij gezegend wezen.
\par 34 Als Ezau de woorden zijns vaders hoorde, zo schreeuwde hij met een groten en bitteren schreeuw, gans zeer; en hij zeide tot zijn vader: Zegen mij, ook mij, mijn vader!
\par 35 En hij zeide: Uw broeder is gekomen met bedrog, en heeft uw zegen weggenomen.
\par 36 Toen zeide hij: Is het niet omdat men zijn naam noemt Jakob, dat hij mij nu twee reizen heeft bedrogen? mijn eerstgeboorte heeft hij genomen, en zie, nu heeft hij mijn zegen genomen! Voorts zeide hij: Hebt gij dan geen zegen voor mij uitbehouden?
\par 37 Toen antwoordde Izak, en zeide tot Ezau: Zie, ik heb hem tot een heer over u gezet, en al zijn broeders heb ik hem tot knechten gegeven; en ik heb hem met koorn en most ondersteund; wat zal ik u dan nu doen, mijn zoon?
\par 38 En Ezau zeide tot zijn vader: Hebt gij maar dezen enen zegen, mijn vader? Zegen mij, ook mij, mijn vader! En Ezau hief zijn stem op, en weende.
\par 39 Toen antwoordde zijn vader Izak en zeide tot hem: Zie, de vettigheden der aarde zullen uw woningen zijn, en van den dauw des hemels van boven af zult gij gezegend zijn.
\par 40 En op uw zwaard zult gij leven, en zult uw broeder dienen; doch het zal geschieden, als gij heersen zult, dan zult gij zijn juk van uw hals afrukken.
\par 41 En Ezau haatte Jakob om dien zegen, waarmede zijn vader hem gezegend had; en Ezau zeide in zijn hart: De dagen van den rouw mijns vaders naderen, en ik zal mijn broeder Jakob doden.
\par 42 Toen aan Rebekka deze woorden van Ezau, haar grootsten zoon, geboodschapt werden, zo zond zij heen, en ontbood Jakob, haar kleinsten zoon, en zeide tot hem: Zie, uw broeder Ezau troost zich over u, dat hij u doden zal.
\par 43 Nu dan, mijn zoon! hoor naar mijn stem, en maak u op, vlied gij naar Haran, tot Laban, mijn broeder.
\par 44 En blijf bij hem enige dagen, totdat de hittige gramschap uws broeders kere;
\par 45 Totdat de toorn uws broeders van u afkere, en hij vergeten hebbe, hetgeen gij hem gedaan hebt; dan zal ik zenden, en u van daar nemen; waarom zoude ik ook van u beiden beroofd worden op een dag?
\par 46 En Rebekka zeide tot Izak: Ik heb verdriet aan mijn leven vanwege de dochteren Heths! Indien Jakob een vrouw neemt van de dochteren Heths, gelijk deze zijn, van de dochteren dezes lands, waartoe zal mij het leven zijn?

\chapter{28}

\par 1 En Izak riep Jakob, en zegende hem; en gebood hem, en zeide tot hem: Neem geen vrouw van de dochteren van Kanaan.
\par 2 Maak u op, ga naar Paddan-aram, ten huize van Bethuel, den vader uwer moeder, en neem u van daar een vrouw, van de dochteren van Laban, uwer moeders broeder.
\par 3 En God almachtig zegene u, en make u vruchtbaar, en vermenigvuldige u, dat gij tot een hoop volken wordt.
\par 4 En Hij geve u den zegen van Abraham; aan u, en uw zaad met u, opdat gij erfelijk bezit het land uwer vreemdelingschappen, hetwelk God aan Abraham gegeven heeft.
\par 5 Alzo zond Izak Jakob weg, dat hij toog naar Paddan-aram, tot Laban, den zoon van Bethuel, den Syrier, den broeder van Rebekka, Jakobs en Ezau's moeder.
\par 6 Als nu Ezau zag, dat Izak Jakob gezegend, en hem naar Paddan-aram weggezonden had om zich van daar een vrouw te nemen; en als hij hem zegende, dat hij hem geboden had, zeggende: Neem geen vrouw van de dochteren van Kanaan;
\par 7 En dat Jakob zijn vader en zijn moeder gehoorzaam geweest was, en naar Paddan-aram getrokken was;
\par 8 En dat Ezau zag, dat de dochteren van Kanaan kwaad waren in de ogen van Izak, zijn vader;
\par 9 Zo ging Ezau tot Ismael, en nam zich tot een vrouw boven zijn vrouwen, Mahalath, de dochter van Ismael, den zoon van Abraham, de zuster van Nebajoth.
\par 10 Jakob dan toog uit van Ber-seba, en ging naar Haran.
\par 11 En hij geraakte op een plaats, waar hij vernachtte; want de zon was ondergegaan; en hij nam van de stenen dier plaats, en maakte zijn hoofdpeluw, en leide zich te slapen te dierzelver plaats.
\par 12 En hij droomde; en ziet, een ladder was gesteld op de aarde, welker opperste aan den hemel raakte; en ziet, de engelen Gods klommen daarbij op en neder.
\par 13 En ziet, de HEERE stond op dezelve en zeide: Ik ben de HEERE, de God van uw vader Abraham, en de God van Izak; dit land, waarop gij ligt te slapen, zal Ik aan u geven, en aan uw zaad.
\par 14 En uw zaad zal wezen als het stof der aarde, en gij zult uitbreken in menigte, westwaarts en oostwaarts, en noordwaarts en zuidwaarts; en in u, en in uw zaad zullen alle geslachten des aardbodems gezegend worden.
\par 15 En zie, Ik ben met u, en Ik zal u behoeden overal, waarheen gij trekken zult, en Ik zal u wederbrengen in dit land; want Ik zal u niet verlaten, totdat Ik zal gedaan hebben, hetgeen Ik tot u gesproken heb.
\par 16 Toen nu Jakob van zijn slaap ontwaakte, zeide hij: Gewisselijk is de HEERE aan deze plaats, en ik heb het niet geweten!
\par 17 En hij vreesde, en zeide: Hoe vreselijk is deze plaats! Dit is niet dan een huis Gods, en dit is de poort des hemels!
\par 18 Toen stond Jakob des morgens vroeg op, en hij nam dien steen, dien hij tot zijn hoofdpeluw gelegd had, en zette hem tot een opgericht teken, en goot daar olie boven op.
\par 19 En hij noemde den naam dier plaats Beth-el; daar toch de naam dier stad te voren was Luz.
\par 20 En Jakob beloofde een gelofte, zeggende: Wanneer God met mij geweest zal zijn, en mij behoed zal hebben op dezen weg, dien ik reize, en mij gegeven zal hebben brood om te eten, en klederen om aan te trekken;
\par 21 En ik ten huize mijns vaders in vrede zal wedergekeerd zijn; zo zal de HEERE mij tot een God zijn!
\par 22 En deze steen, dien ik tot een opgericht teken gezet heb, zal een huis Gods wezen, en van alles, wat Gij mij geven zult, zal ik U voorzeker de tienden geven!

\chapter{29}

\par 1 Toen hief Jakob zijn voeten op, en ging naar het land der kinderen van het Oosten.
\par 2 En hij zag toe, en ziet, er was een put in het veld; en ziet, er waren drie kudden schapen nevens dien nederliggende; want uit dien put drenkten zij de kudden; en er was een grote steen op den mond van dien put.
\par 3 En derwaarts werden al de kudden verzameld, en zij wentelden den steen van den mond des puts, en drenkten de schapen, en legden den steen weder op den mond van dien put, op zijn plaats.
\par 4 Toen zeide Jakob tot hen: Mijn broeders! van waar zijt gij? En zij zeiden: Wij zijn van Haran.
\par 5 En hij zeide tot hen: Kent gij Laban, den zoon van Nahor? En zij zeiden: Wij kennen hem.
\par 6 Voorts zeide hij tot hen: Is het wel met hem? En zij zeiden: Het is wel; en zie, Rachel, zijn dochter, komt met de schapen.
\par 7 En hij zeide: Ziet, het is nog hoog dag, het is geen tijd, dat het vee verzameld worde; drenkt de schapen, en gaat heen, weidt dezelve.
\par 8 Toen zeiden zij: Wij kunnen niet, totdat al de kudden samen zullen vergaderd zijn, en dat men den steen van den mond des puts afwentele, opdat wij de schapen drenken.
\par 9 Als hij nog met hen sprak, zo kwam Rachel met de schapen, die haar vader toebehoorden; want zij was een herderin.
\par 10 En het geschiedde, als Jakob Rachel zag, de dochter van Laban, zijner moeders broeder, en de schapen van Laban, zijner moeders broeder, dat Jakob toetrad, en wentelde den steen van den mond des puts, en drenkte de schapen van Laban, zijner moeders broeder.
\par 11 En Jakob kuste Rachel; en hij hief zijn stem op en weende.
\par 12 En Jakob gaf Rachel te kennen, dat hij een broeder van haar vader, en dat hij de zoon van Rebekka was. Toen liep zij heen, en gaf het aan haar vader te kennen.
\par 13 En het geschiedde, als Laban die tijding hoorde van Jakob, zijner zusters zoon, zo liep hij hem tegemoet, en omhelsde hem, en kuste hem, en bracht hem tot zijn huis. En hij vertelde Laban al deze dingen.
\par 14 Toen zeide Laban tot hem: Voorwaar, gij zijt mijn gebeente en mijn vlees! En hij bleef bij hem een volle maand.
\par 15 Daarna zeide Laban tot Jakob: Omdat gij mijn broeder zijt, zoudt gij mij derhalve om niet dienen? verklaar mij, wat zal uw loon zijn?
\par 16 En Laban had twee dochters: de naam der grootste was Lea; en de naam der kleinste was Rachel.
\par 17 Doch Lea had tedere ogen; maar Rachel was schoon van gedaante, en schoon van aangezicht.
\par 18 En Jakob had Rachel lief; en hij zeide: Ik zal u zeven jaren dienen, om Rachel, uw kleinste dochter.
\par 19 Toen zeide Laban: Het is beter, dat ik haar aan u geve, dan dat ik haar aan een anderen man geve; blijf bij mij.
\par 20 Alzo diende Jakob om Rachel zeven jaren; en die waren in zijn ogen als enige dagen, omdat hij haar liefhad.
\par 21 Toen zeide Jakob tot Laban: Geef mijn huisvrouw, want mijn dagen zijn vervuld, dat ik tot haar inga.
\par 22 Zo verzamelde Laban al de mannen dier plaats, en maakte een maaltijd.
\par 23 En het geschiedde des avonds, dat hij zijn dochter Lea nam, en bracht haar tot hem; en hij ging tot haar in.
\par 24 En Laban gaf haar Zilpa, zijn dienstmaagd, aan Lea, zijn dochter, tot een dienstmaagd.
\par 25 En het geschiedde des morgens, en ziet, het was Lea. Daarom zeide hij tot Laban: Wat is dit, dat gij mij gedaan hebt; heb ik niet bij u gediend om Rachel? waarom hebt gij mij dan bedrogen?
\par 26 En Laban zeide: Men doet alzo niet te dezer onzer plaatse, dat men de kleinste uitgeve voor de eerstgeborene.
\par 27 Vervul de week van deze; dan zullen wij u ook die geven, voor den dienst, dien gij nog andere zeven jaren bij mij dienen zult.
\par 28 En Jakob deed alzo; en hij vervulde de week van deze. Toen gaf hij hem Rachel, zijn dochter, hem tot een vrouw.
\par 29 En Laban gaf aan zijn dochter Rachel zijn dienstmaagd Bilha, haar tot een dienstmaagd.
\par 30 En hij ging ook in tot Rachel, en had ook Rachel liever dan Lea; en hij diende bij hem nog andere zeven jaren.
\par 31 Toen nu de HEERE zag, dat Lea gehaat was, opende Hij haar baarmoeder; maar Rachel was onvruchtbaar.
\par 32 En Lea werd bevrucht, en baarde een zoon, en zij noemde zijn naam Ruben; want zij zeide: Omdat de HEERE mijn verdrukking heeft aangezien, daarom zal mijn man mij nu liefhebben.
\par 33 En zij werd wederom bevrucht, en baarde een zoon, en zeide: Dewijl de HEERE gehoord heeft, dat ik gehaat was, zo heeft Hij mij ook dezen gegeven; en zij noemde zijn naam Simeon.
\par 34 En zij werd nog bevrucht, en baarde een zoon, en zeide: Nu zal zich ditmaal mijn man bij mij voegen, dewijl ik hem drie zonen gebaard heb; daarom noemde zij zijn naam Levi.
\par 35 En zij werd wederom bevrucht, en baarde een zoon, en zeide: Ditmaal zal ik den HEERE loven; daarom noemde zij zijn naam Juda. En zij hield op van baren.

\chapter{30}

\par 1 Als nu Rachel zag, dat zij Jakob niet baarde, zo benijdde Rachel haar zuster; en zij zeide tot Jakob: Geef mij kinderen! of indien niet, zo ben ik dood.
\par 2 Toen ontstak Jakobs toorn tegen Rachel, en hij zeide: Ben ik dan in plaats van God, Die de vrucht des buiks van u geweerd heeft?
\par 3 En zij zeide: Zie, daar is mijn dienstmaagd Bilha, ga tot haar in; dat zij op mijn knieen bare, en ik ook uit haar gebouwd worde.
\par 4 Zo gaf zij hem haar dienstmaagd Bilha tot een vrouw; en Jakob ging tot haar in.
\par 5 En Bilha werd zwanger, en baarde Jakob een zoon.
\par 6 Toen zeide Rachel: God heeft mij gericht, en ook mijn stem verhoord, en heeft mij een zoon gegeven; daarom noemde zij zijn naam Dan.
\par 7 En Bilha, Rachels dienstmaagd, werd wederom bevrucht, en baarde Jakob den tweeden zoon.
\par 8 Toen zeide Rachel: Ik heb worstelingen Gods met mijn zuster geworsteld; ook heb ik de overhand gehad; en zij noemde zijn naam Nafthali.
\par 9 Toen nu Lea zag, dat zij ophield van baren, nam zij ook haar dienstmaagd Zilpa, en gaf die aan Jakob tot een vrouw.
\par 10 En Zilpa, Lea's dienstmaagd, baarde Jakob een zoon.
\par 11 Toen zeide Lea: Er komt een hoop! en zij noemde zijn naam Gad.
\par 12 Daarna baarde Zilpa, Lea's dienstmaagd, Jakob een tweeden zoon.
\par 13 Toen zeide Lea: Tot mijn geluk! want de dochters zullen mij gelukkig achten; en zij noemde zijn naam Aser.
\par 14 En Ruben ging in de dagen van de tarweoogst, en hij vond Dudaim in het veld, en hij bracht die tot zijn moeder Lea. Toen zeide Rachel tot Lea: Geef mij toch van uws zoons Dudaim.
\par 15 En zij zeide tot haar: Is het weinig, dat gij mijn man genomen hebt, dat gij ook mijns zoons Dudaim nemen zult? Toen zeide Rachel: Daarom zal hij dezen nacht voor uws zoons Dudaim bij u liggen.
\par 16 Als nu Jakob des avonds uit het veld kwam, ging Lea uit hem tegemoet, en zeide: Gij zult tot mij inkomen; want ik heb u om loon zekerlijk gehuurd voor mijns zoons Dudaim; en hij lag dien nacht bij haar.
\par 17 En God verhoorde Lea; en zij werd bevrucht, en baarde Jakob den vijfden zoon.
\par 18 Toen zeide Lea: God heeft mijn loon gegeven, nadat ik mijn dienstmaagd aan mijn man gegeven heb; en zij noemde zijn naam Issaschar.
\par 19 En Lea werd wederom bevrucht, en zij baarde Jakob den zesden zoon.
\par 20 En Lea zeide: God heeft mij, mij heeft Hij begiftigd met een goede gift; ditmaal zal mijn man mij bijwonen; want ik heb hem zes zonen gebaard; en zij noemde zijn naam Zebulon.
\par 21 En zij baarde daarna een dochter; en zij noemde haar naam Dina.
\par 22 God dacht ook aan Rachel; en God verhoorde haar, en opende haar baarmoeder.
\par 23 En zij werd bevrucht, en baarde een zoon; en zij zeide: God heeft mijn smaadheid weggenomen!
\par 24 En zij noemde zijn naam Jozef, zeggende: De HEERE voege mij een anderen zoon daartoe.
\par 25 En het geschiedde, dat Rachel Jozef gebaard had, dat Jakob tot Laban zeide: Laat mij vertrekken, dat ik ga tot mijn plaats, en naar mijn land.
\par 26 Geef mijn vrouwen, en mijn kinderen, om welke ik u gediend heb, dat ik vertrek; want gij weet mijn dienst, dien ik u gediend heb.
\par 27 Toen zeide Laban tot hem: Zo ik nu genade gevonden heb in uw ogen; ik heb waargenomen, dat de HEERE mij om uwentwil gezegend heeft.
\par 28 Hij zeide dan: Noem mij uitdrukkelijk uw loon, dat ik geven zal.
\par 29 Toen zeide hij tot hem: Gij weet, hoe ik u gediend heb, en hoe uw vee bij mij geweest is.
\par 30 Want het weinige, dat gij voor mij gehad hebt, dat is tot een menigte uitgebroken; en de HEERE heeft u gezegend bij mijn voet; nu dan, wanneer zal ik ook werken voor mijn huis?
\par 31 En hij zeide: Wat zal ik u geven? Toen zeide Jakob: Gij zult mij niet met al geven, indien gij mij deze zaak doen zult; ik zal wederom uw kudden weiden, en bewaren.
\par 32 Ik zal heden door uw ganse kudde gaan, daarvan afzonderende al het gespikkelde en geplekte vee, en al het bruine vee onder de lammeren, en het geplekte en gespikkelde onder de geiten; en zulks zal mijn loon zijn.
\par 33 Zo zal mijn gerechtigheid op den dag van morgen met mij betuigen, als gij komen zult over mijn loon, voor uw aangezicht; al wat niet gespikkeld en geplekt is onder de geiten en bruin onder de lammeren, dat zij bij mij gestolen.
\par 34 Toen zeide Laban: Zie, och ja, het zij naar uw woord!
\par 35 En hij zonderde af ten zelfden dage de gesprenkelde en geplekte bokken en al de gespikkelde en geplekte geiten, al waar wit aan was, en al het bruine onder de lammeren; en hij gaf dezelve in de hand zijner zonen.
\par 36 En hij stelde een weg van drie dagen tussen hem, en tussen Jakob; en Jakob weidde de overige kudde van Laban.
\par 37 Toen nam zich Jakob roeden van groen populierenhout, en van hazelaar, en van kastanje; en hij schilde daarin witte strepen, ontblotende het wit, hetwelk aan die roeden was.
\par 38 En hij leide deze roeden, die hij geschild had, in de goten, en in de drinkbakken van het water, waar de kudde kwam drinken, tegenover de kudde; en zij werden verhit, als zij kwamen om te drinken.
\par 39 Als dan de kudde verhit werd bij de roeden, zo lammerde de kudde gesprenkelde, gespikkelde, en geplekte.
\par 40 Toen scheidde Jakob de lammeren, en hij wendde het gezicht der kudde op het gesprenkelde, en al het bruine onder Labans kudde; en hij stelde zijn kudden alleen, en hij zette ze niet bij de kudde van Laban.
\par 41 En het geschiedde, telkens als de kudde der vroegelingen verhit werd, zo stelde Jakob de roeden voor de ogen der kudde in de goten, opdat zij hittig werden bij de roeden.
\par 42 Maar als de kudde spade hittig werd, zo stelde hij ze niet, zodat de spadelingen Laban, en de vroegelingen Jakob toekwamen.
\par 43 En die man brak gans zeer uit in menigte, en hij had vele kudden, en dienstmaagden, en dienstknechten, en kemelen, en ezelen.

\chapter{31}

\par 1 Toen hoorde hij de woorden der zonen van Laban, zeggende: Jakob heeft genomen alles, wat onzes vaders was, en van hetgeen, dat onzes vaders was, heeft hij al deze heerlijkheid gemaakt.
\par 2 Jakob zag ook het aangezicht van Laban aan, en ziet, het was jegens hem niet als gisteren en eergisteren.
\par 3 En de HEERE zeide tot Jakob: Keer weder tot het land uwer vaderen, en tot uw maagschap, en Ik zal met u zijn.
\par 4 Toen zond Jakob heen, en riep Rachel en Lea, op het veld tot zijn kudde;
\par 5 En hij zeide tot haar: Ik zie het aangezicht uws vaders, dat het jegens mij niet is als gisteren en eergisteren; doch de God mijns vaders is bij mij geweest.
\par 6 En gijlieden weet, dat ik met al mijn macht uw vader gediend heb.
\par 7 Maar uw vader heeft bedriegelijk met mij gehandeld, en heeft mijn loon tien malen veranderd; doch God heeft hem niet toegelaten, om mij kwaad te doen.
\par 8 Wanneer hij aldus zeide: De gespikkelde zullen uw loon zijn, zo lammerden al de kudden gespikkelde; en wanneer hij alzo zeide: De gesprenkelde zullen uw loon zijn, zo lammerden al de kudden gesprenkelde.
\par 9 Alzo heeft God uw vader het vee ontrukt, en aan mij gegeven.
\par 10 En het geschiedde ten tijde, als de kudde hittig werd, dat ik mijn ogen ophief, en ik zag in den droom; en ziet, de bokken, die de kudden beklommen, waren gesprenkeld, gespikkeld, en hagelvlakkig.
\par 11 En de Engel Gods zeide tot mij in den droom: Jakob! En ik zeide: Zie, hier ben ik!
\par 12 En Hij zeide: Hef toch uw ogen op, en zie! alle bokken, die de kudde beklimmen, zijn gesprenkeld, gespikkeld, en hagelvlakkig; want Ik heb gezien alles, wat Laban u doet.
\par 13 Ik ben die God van Beth-el, alwaar gij het opgerichte teken gezalfd hebt, waar gij Mij een gelofte beloofd hebt; nu, maak u op, vertrek uit dit land, en keer weder in het land uwer maagschap.
\par 14 Toen antwoordden Rachel en Lea, en zeiden tot hem: Is er nog voor ons een deel of erfenis, in het huis onzes vaders?
\par 15 Zijn wij niet vreemden van hem geacht? Want hij heeft ons verkocht, en hij heeft ook steeds ons geld verteerd.
\par 16 Want al de rijkdom, welke God onzen vader heeft ontrukt, die is onze, en van onze zonen; nu dan, doe alles, wat God tot u gezegd heeft.
\par 17 Toen maakte zich Jakob op, en laadde zijn zonen en zijn vrouwen op kemelen.
\par 18 En hij voerde al zijn vee weg, en al zijn have, die hij gewonnen had, het vee, dat hij bezat, hetwelk hij in Paddan-aram geworven had, om te komen tot Izak, zijn vader, naar het land Kanaan.
\par 19 Laban nu was gegaan, om zijn schapen te scheren; zo stal Rachel de terafim, die haar vader had.
\par 20 En Jakob ontstal zich aan het hart van Laban, den Syrier, overmits hij hem niet te kennen gaf, dat hij vlood.
\par 21 En hij vlood, en al wat het zijne was, en hij maakte zich op, en voer over de rivier, en hij zette zijn aangezicht naar het gebergte Gilead.
\par 22 En ten derden dage werd aan Laban geboodschapt, dat Jakob gevloden was.
\par 23 Toen nam hij zijn broeders met zich, en jaagde hem achterna, een weg van zeven dagen, en hij kreeg hem op het gebergte van Gilead.
\par 24 Doch God kwam tot Laban, den Syrier, in een droom des nachts, en Hij zeide tot hem: Wacht u, dat gij met Jakob spreekt, noch goed, noch kwaad.
\par 25 En Laban achterhaalde Jakob; Jakob nu had zijn tent geslagen op dat gebergte; ook sloeg Laban met zijn broederen de zijne op het gebergte van Gilead.
\par 26 Toen zeide Laban tot Jakob: Wat hebt gij gedaan, dat gij u aan mijn hart ontstolen hebt, en mijn dochteren ontvoerd hebt, als gevangenen met het zwaard?
\par 27 Waarom zijt gij heimelijk gevloden, en hebt u aan mij ontstolen? en hebt het mij niet aangezegd, dat ik u geleid had met vreugde, en met gezangen, met trommel en met harp?
\par 28 Ook hebt gij mij niet toegelaten mijn zonen en mijn dochteren te kussen; nu, gij hebt dwaselijk gedaan zo doende.
\par 29 Het ware in de macht mijner hand aan ulieden kwaad te doen; maar de God van ulieder vader heeft tot mij gisteren nacht gesproken, zeggende: Wacht u, van met Jakob te spreken, of goed, of kwaad.
\par 30 En nu, gij hebt immers willen vertrekken, omdat gij zo zeer begerig waart naar uws vaders huis; waarom hebt gij mijn goden gestolen?
\par 31 Toen antwoordde Jakob, en zeide tot Laban: Omdat ik vreesde; want ik zeide: Opdat gij niet misschien uw dochteren mij ontweldigdet!
\par 32 Bij wien gij uw goden vinden zult, laat hem niet leven! Onderken gij voor onze broederen, wat bij mij is, en neem het tot u. Want Jakob wist niet, dat Rachel dezelve gestolen had.
\par 33 Toen ging Laban in de tent van Jakob, en in de tent van Lea, en in de tent van de beide dienstmaagden, en hij vond niets; en als hij uit de tent van Lea gegaan was, kwam hij in de tent van Rachel.
\par 34 Maar Rachel had de terafim genomen, en zij had die in een kemels zadeltuig gelegd, en zij zat op dezelve. En Laban betastte die ganse tent, en hij vond niets.
\par 35 En zij zeide tot haar vader: Dat de toorn niet ontsteke in mijns heren ogen, omdat ik voor uw aangezicht niet kan opstaan; want het gaat mij naar der vrouwen wijze; en hij doorzocht; maar hij vond de terafim niet.
\par 36 Toen ontstak Jakob, en twistte met Laban; en Jakob antwoordde en zeide tot Laban: Wat is mijn overtreding, wat is mijn zonde, dat gij mij zo hittiglijk hebt nagejaagd?
\par 37 Als gij al mijn huisraad betast hebt, wat hebt gij gevonden van al het huisraad uws huizes! Leg het hier voor mijn broederen en uw broederen, en laat hen richten tussen ons beiden.
\par 38 Deze twintig jaren ben ik bij u geweest; uw ooien en uw geiten hebben niet misdragen, en de rammen uwer kudde heb ik niet gegeten.
\par 39 Het verscheurde heb ik tot u niet gebracht; ik heb het geboet; gij hebt het van mijn hand geeist, het ware des daags gestolen, of des nachts gestolen.
\par 40 Ik ben geweest, dat mij bij dag de hitte verteerde, en bij nacht de vorst, en dat mijn slaap van mijn ogen week.
\par 41 Ik ben nu twintig jaren in uw huis geweest; ik heb u veertien jaren gediend om uw beide dochteren, en zes jaren om uw kudde; en gij hebt mijn loon tien malen veranderd.
\par 42 Ten ware de God van mijn vader, de God van Abraham, en de Vreze van Izak, bij mij geweest was, zekerlijk, gij zoudt mij nu ledig weggezonden hebben! God heeft mijn ellende, en den arbeid mijner handen aangezien, en heeft u gisteren nacht bestraft.
\par 43 Toen antwoordde Laban en zeide tot Jakob: Deze dochters zijn mijn dochters, en deze zonen zijn mijn zonen, en deze kudde is mijn kudde, ja, al wat gij ziet, dat is mijn; en wat zoude ik aan deze mijn dochteren heden doen? of aan haar zonen, die zij gebaard hebben?
\par 44 Nu dan, kom, laat ons een verbond maken, ik en gij, dat het tot een getuigenis zij tussen mij en tussen u!
\par 45 Toen nam Jakob een steen, en hij verhoogde dien tot een opgericht teken.
\par 46 En Jakob zeide tot zijn broederen: Vergadert stenen! En zij namen stenen, en maakten een hoop; en zij aten aldaar op dien hoop.
\par 47 En Laban noemde hem Jegar-sahadutha; maar Jakob noemde denzelven Gilead.
\par 48 Toen zeide Laban: Deze hoop zij heden een getuige tussen mij en tussen u! Daarom noemde men zijn naam Gilead,
\par 49 En Mizpa; omdat hij zeide: Dat de HEERE opzicht neme tussen mij en tussen u, wanneer wij de een van den ander zullen verborgen zijn!
\par 50 Zo gij mijn dochteren beledigt, en zo gij vrouwen neemt boven mijn dochteren, niemand is bij ons; zie toe, God zal getuige zijn tussen mij en tussen u!
\par 51 Laban zeide voorts tot Jakob: Zie, daar is deze zelfde hoop, en zie, daar is dit opgericht teken, hetwelk ik opgeworpen heb tussen mij en tussen u;
\par 52 Deze zelfde hoop zij getuige, en dit opgericht teken zij getuige, dat ik tot u voorbij dezen hoop niet komen zal, en dat gij tot mij, voorbij dezen hoop en dit opgericht teken, niet komen zult ten kwade!
\par 53 De God van Abraham, en de God van Nahor, de God huns vaders richte tussen ons! En Jakob zwoer bij de Vreze zijn vaders Izaks.
\par 54 Toen slachtte Jakob een slachting op dat gebergte, en hij nodigde zijn broederen, om brood te eten; en zij aten brood, en vernachtten op dat gebergte.
\par 55 En Laban stond des morgens vroeg op, en kuste zijn zonen, en zijn dochteren, en zegende hen; en Laban trok heen, en keerde weder tot zijn plaats.

\chapter{32}

\par 1 Jakob toog ook zijns weegs; en de engelen Gods ontmoetten hem.
\par 2 En Jakob zeide, met dat hij hen zag: Dit is een heirleger Gods! en hij noemde den naam derzelver plaats Mahanaim.
\par 3 En Jakob zond boden uit voor zijn aangezicht tot Ezau, zijn broeder, naar het land Seir, de landstreek van Edom.
\par 4 En hij gebood hun, zeggende: Zo zult gij zeggen tot mijn heer, tot Ezau: Zo zegt Jakob, uw knecht: Ik heb als vreemdeling gewoond bij Laban, en heb er tot nu toe vertoefd;
\par 5 En ik heb ossen en ezelen, schapen en knechten en maagden; en ik heb gezonden om mijn heer aan te zeggen, opdat ik genade vinde in uw ogen.
\par 6 En de boden kwamen weder tot Jakob, zeggende: Wij zijn gekomen tot uw broeder, tot Ezau; en ook trekt hij u tegemoet, en vierhonderd mannen met hem.
\par 7 Toen vreesde Jakob zeer, en hem was bange; en hij verdeelde het volk, dat met hem was, en de schapen, en de runderen, en de kemels, in twee heiren;
\par 8 Want hij zeide: Indien Ezau op het ene heir komt, en slaat het, zo zal het overgeblevene heir ontkomen.
\par 9 Voorts zeide Jakob: O, God mijns vaders Abrahams, en God mijns vaders Izaks, o HEERE! Die tot mij gezegd hebt: Keer weder tot uw land, en tot uw maagschap, en Ik zal wel bij u doen!
\par 10 Ik ben geringer dan al deze weldadigheden, en dan al deze trouw, die Gij aan Uw knecht gedaan hebt; want ik ben met mijn staf over deze Jordaan gegaan, en nu ben ik tot twee heiren geworden!
\par 11 Ruk mij toch uit mijns broeders hand, uit Ezau's hand; want ik vreze hem, dat hij niet misschien kome, en mij sla, de moeder met de zonen!
\par 12 Gij hebt immers gezegd: Ik zal gewisselijk bij u weldoen, en Ik zal uw zaad stellen als het zand der zee, dat vanwege de menigte niet geteld kan worden!
\par 13 En hij vernachtte aldaar dienzelfden nacht; en hij nam van hetgeen, dat hem in zijn hand kwam, een geschenk voor Ezau zijn broeder;
\par 14 Tweehonderd geiten en twintig bokken, tweehonderd ooien en twintig rammen;
\par 15 Dertig zogende kemelinnen met haar veulens, veertig koeien en tien varren, twintig ezelinnen en tien jonge ezels.
\par 16 En hij gaf die in de hand zijner knechten, elke kudde bijzonder; en hij zeide tot zijn knechten: Gaat gijlieden door, voor mijn aangezicht, en stelt ruimte tussen kudde en tussen kudde.
\par 17 En hij gebood de eerste, zeggende: Wanneer Ezau, mijn broeder, u ontmoeten zal, en u vragen, zeggende: Wiens zijt gij? en waarheen gaat gij? en wiens zijn deze voor uw aangezicht?
\par 18 Zo zult gij zeggen: Dat is een geschenk van uw knecht Jakob, gezonden tot mijn heer, tot Ezau, en zie, hij zelf is ook achter ons!
\par 19 En hij gebood ook den tweede, ook den derde, ook allen, die de kudden nagingen, zeggende: Naar ditzelfde woord zult gij spreken tot Ezau, als gij hem vinden zult.
\par 20 En gij zult ook zeggen: Zie, uw knecht Jakob is achter ons! Want hij zeide: Ik zal zijn aangezicht verzoenen met dit geschenk, dat voor mijn aangezicht gaat, en daarna zal ik zijn aangezicht zien; misschien zal hij mijn aangezicht aannemen.
\par 21 Alzo ging dat geschenk heen voor zijn aangezicht; doch hij zelf vernachtte dienzelfden nacht in het leger.
\par 22 En hij stond op in dienzelfden nacht, en hij nam zijn twee vrouwen, en zijn twee dienstmaagden, en zijn elf kinderen, en hij toog over het veer van de Jabbok.
\par 23 En hij nam ze, en deed hen over die beek trekken; en hij deed overtrekken hetgeen hij had.
\par 24 Doch Jakob bleef alleen over; en een man worstelde met hem, totdat de dageraad opging.
\par 25 En toen Hij zag, dat Hij hem niet overmocht, roerde Hij het gewricht zijner heup aan, zodat het gewricht van Jakobs heup verwrongen werd, als Hij met hem worstelde.
\par 26 En Hij zeide: Laat Mij gaan, want de dageraad is opgegaan. Maar hij zeide: Ik zal U niet laten gaan, tenzij dat Gij mij zegent.
\par 27 En Hij zeide tot hem: Hoe is uw naam? En hij zeide: Jakob.
\par 28 Toen zeide Hij: Uw naam zal voortaan niet Jakob heten, maar Israel; want gij hebt u vorstelijk gedragen met God en met de mensen, en hebt overmocht.
\par 29 En Jakob vraagde, en zeide: Geef toch Uw naam te kennen. En Hij zeide: Waarom is het, dat gij naar Mijn naam vraagt? En Hij zegende hem aldaar.
\par 30 En Jakob noemde den naam dier plaats Pniel: Want, zeide hij ik heb God gezien van aangezicht tot aangezicht, en mijn ziel is gered geweest.
\par 31 En de zon rees hem op, als hij door Pniel gegaan was; en hij was hinkende aan zijn heup.
\par 32 Daarom eten de kinderen Israels de verrukte zenuw niet, die op het gewricht der heup is, tot op dezen dag, omdat Hij het gewricht van Jakobs heup aangeroerd had, aan de verrukte zenuw.

\chapter{33}

\par 1 En Jakob hief zijn ogen op en zag; en ziet, Ezau kwam, en vierhonderd mannen met hem. Toen verdeelde hij de kinderen onder Lea, en onder Rachel, en onder de twee dienstmaagden.
\par 2 En hij stelde de dienstmaagden en haar kinderen vooraan; en Lea en haar kinderen meer achterwaarts; maar Rachel en Jozef de achterste.
\par 3 En hij ging voorbij hun aangezicht heen, en hij boog zich zeven malen ter aarde, totdat hij bij zijn broeder kwam.
\par 4 Toen liep Ezau hem tegemoet, en nam hem in den arm, en viel hem aan den hals, en kuste hem; en zij weenden.
\par 5 Daarna hief hij zijn ogen op, en zag die vrouwen en die kinderen, en zeide: Wie zijn deze bij u? En hij zeide: De kinderen, die God aan uw knecht genadiglijk verleend heeft.
\par 6 Toen traden de dienstmaagden toe, zij en haar kinderen, en zij bogen zich neder.
\par 7 En Lea trad ook toe, met haar kinderen, en zij bogen zich neder; en daarna trad Jozef toe en Rachel, en zij bogen zich neder.
\par 8 En hij zeide: Voor wien is u al dit heir, dat ik ontmoet heb? En hij zeide: Om genade te vinden in de ogen mijns heren!
\par 9 Maar Ezau zeide: Ik heb veel, mijn broeder! het zij het uwe, wat gij hebt!
\par 10 Toen zeide Jakob: Och neen! indien ik nu genade in uw ogen gevonden heb, zo neem mijn geschenk van mijn hand; daarom, omdat ik uw aangezicht gezien heb, als had ik Gods aangezicht gezien, en gij welgevallen aan mij genomen hebt.
\par 11 Neem toch mijn zegen, die u toegebracht is, dewijl het God mij genadiglijk verleend heeft, en dewijl ik alles heb; en hij hield bij hem aan, zodat hij het nam.
\par 12 En hij zeide: Laat ons reizen en voorttrekken; en ik zal voor u trekken.
\par 13 Maar hij zeide tot hem: Mijn heer weet, dat deze kinderen teder zijn, en dat ik zogende schapen en koeien bij mij heb; indien men dezelve maar een dag afdrijft, zo zal de gehele kudde sterven.
\par 14 Mijn heer trekke toch voorbij, voor het aangezicht van zijn knecht; en ik zal mij op mijn gemak als leidsman voegen, naar den gang van het werk, hetwelk voor mijn aangezicht is, en naar den gang dezer kinderen, totdat ik bij mijn heer te Seir kome.
\par 15 En Ezau zeide: Laat mij toch van dit volk, dat met mij is, u bijstellen. En hij zeide: Waartoe dat? laat mij genade vinden in mijns heren ogen!
\par 16 Alzo keerde Ezau dien dag wederom zijns weegs naar Seir toe.
\par 17 Maar Jakob reisde naar Sukkoth, en bouwde een huis voor zich, en maakte hutten voor zijn vee; daarom noemde hij den naam dier plaats Sukkoth.
\par 18 En Jakob kwam behouden tot de stad Sichem, welke is in het land Kanaan, als hij kwam van Paddan-aram; en hij legerde zich in het gezicht der stad.
\par 19 En hij kocht een deel des velds, waarop hij zijn tent gespannen had, van de hand der zonen van Hemor, den vader van Sichem, voor honderd stukken gelds.
\par 20 En hij richte aldaar een altaar op, en noemde het: De God Israels is God!

\chapter{34}

\par 1 En Dina, de dochter van Lea, die zij Jakob gebaard had, ging uit, om de dochteren van dat land te bezien.
\par 2 Sichem nu, de zoon van Hemor den Heviet, den landvorst, zag haar, en hij nam ze, en lag bij haar, en verkrachtte ze.
\par 3 En zijn ziel kleefde aan Dina, Jakobs dochter; en hij had de jonge dochter lief, en sprak naar het hart van de jonge dochter.
\par 4 Sichem sprak ook tot zijn vader Hemor, zeggende: Neem mij deze dochter tot een vrouw.
\par 5 Toen Jakob hoorde, dat hij zijn dochter Dina verontreinigd had, zo waren zijn zonen met het vee in het veld; en Jakob zweeg, totdat zij kwamen.
\par 6 En Hemor, de vader van Sichem, ging uit tot Jakob, om met hem te spreken.
\par 7 En de zonen van Jakob kwamen van het veld, als zij dit hoorden; en het smartte dezen mannen, en zij ontstaken zeer, omdat hij dwaasheid in Israel gedaan had, Jakobs dochter beslapende, hetwelk alzo niet zoude gedaan worden.
\par 8 Toen sprak Hemor met hen, zeggende: Mijns zoons Sichems ziel is verliefd op ulieder dochter; geeft hem haar toch tot een vrouw.
\par 9 En verzwagert u met ons; geeft ons uw dochteren; en neemt voor u onze dochteren;
\par 10 En woont met ons; en het land zal voor uw aangezicht zijn; woont, en handelt daarin, en stelt u tot bezitters daarin.
\par 11 En Sichem zeide tot haar vader, en tot haar broederen: Laat mij genade vinden in uw ogen; en wat gij tot mij zeggen zult, zal ik geven.
\par 12 Vergroot zeer over mij den bruidschat en het geschenk; en ik zal geven, gelijk als gij tot mij zult zeggen; geef mij slechts de jonge dochter tot een vrouw.
\par 13 Toen antwoordden Jakobs zonen aan Sichem en Hemor, zijn vader, bedriegelijk, en spraken (overmits dat hij Dina, hun zuster, verontreinigd had);
\par 14 En zij zeiden tot hen: Wij zullen deze zaak niet kunnen doen, dat wij onze zuster aan een man geven zouden, die de voorhuid heeft; want dat ware ons een schande.
\par 15 Doch hierin zullen wij u te wille zijn, zo gij wordt gelijk als wij, dat onder u besneden worde al wat mannelijk is.
\par 16 Dan zullen wij u onze dochteren geven, en uw dochteren zullen wij ons nemen, en wij zullen met u wonen, en wij zullen tot een volk zijn.
\par 17 Maar zo gij naar ons niet zult horen, om besneden te worden, zo zullen wij onze dochter nemen, en wegtrekken.
\par 18 En hun woorden waren goed in de ogen van Hemor, en in de ogen van Sichem, Hemors zoon.
\par 19 En de jongeling vertoogde niet, deze zaak te doen; want hij had lust in Jakobs dochter; en hij was geeerd boven al zijns vaders huis.
\par 20 Zo kwam Hemor en Sichem, zijn zoon, tot hunner stadspoort; en zij spraken tot de mannen hunner stad, zeggende:
\par 21 Deze mannen zijn vreedzaam met ons; daarom laat hen in dit land wonen, en daarin handelen, en het land (ziet het is wijd van begrip) voor hun aangezicht zijn; wij zullen ons hun dochteren tot vrouwen nemen, en wij zullen onze dochteren aan hen geven.
\par 22 Doch hierin zullen deze mannen ons te wille zijn, dat zij met ons wonen, om tot een volk te zijn; als al wat mannelijk is onder ons besneden wordt, gelijk als zij besneden zijn.
\par 23 Hun vee, en hun bezitting, en al hun beesten, zullen die niet onze zijn? Alleen laat ons hun te wille zijn, en zij zullen met ons wonen.
\par 24 En zij hoorden naar Hemor, en naar Sichem, zijn zoon, allen, die ter zijner stadspoort uitgingen; en zij werden besneden, al wat mannelijk was, allen, die ter zijner stadspoort uitgingen.
\par 25 En het geschiedde ten derden dage, toen zij in de smart waren, zo namen de twee zonen van Jakob, Simeon en Levi, broeders van Dina, een iegelijk zijn zwaard, en kwamen stoutelijk in de stad, en doodden al wat mannelijk was.
\par 26 Zij sloegen ook Hemor, en zijn zoon Sichem, dood met de scherpte des zwaards; en zij namen Dina uit Sichems huis, en gingen van daar.
\par 27 De zonen van Jakob kwamen over de verslagenen, en plunderden de stad, omdat zij hun zuster verontreinigd hadden.
\par 28 Hun schapen, en hun runderen, en hun ezelen, en hetgeen dat in de stad, en hetgeen dat in het veld was, namen zij.
\par 29 En al hun vermogen, en al hun kleine kinderen, en hun vrouwen, voerden zij gevankelijk weg, en plunderden dezelven, en al wat binnenshuis was.
\par 30 Toen zeide Jakob tot Simeon en tot Levi: Gij hebt mij beroerd, mits mij stinkende te maken onder de inwoners dezes lands, onder de Kanaanieten, en onder de Ferezieten; en ik ben weinig volks in getal; zo zij zich tegen mij verzamelen, zo zullen zij mij slaan, en ik zal verdelgd worden, ik en mijn huis.
\par 31 En zij zeiden: Zou hij dan met onze zuster als met een hoer doen?

\chapter{35}

\par 1 Daarna zeide God tot Jakob: Maak u op, trek op naar Beth-el, en woon aldaar; en maak daar een altaar dien God, Die u verscheen, toen gij vluchttet voor het aangezicht van uw broeder Ezau.
\par 2 Toen zeide Jakob tot zijn huisgezin, en tot allen, die bij hem waren: Doet weg de vreemde goden, die in het midden van u zijn, en reinigt u, en verandert uw klederen;
\par 3 En laat ons ons opmaken, en optrekken naar Beth-el; en ik zal daar een altaar maken dien God, Die mij antwoordt ten dage mijner benauwdheid, en met mij geweest is op den weg, die ik gewandeld heb.
\par 4 Toen gaven zij Jakob al die vreemde goden, die in hun hand waren, en de oorsierselen, die aan hun oren waren, en Jakob verborg ze onder den eikeboom, die bij Sichem is.
\par 5 En zij reisden heen; en Gods verschrikking was over de steden, die rondom hen waren, zodat zij de zonen van Jakob niet achterna jaagden.
\par 6 Alzo kwam Jakob te Luz, hetwelk is in het land Kanaan (dat is Beth-el), hij en al het volk, dat bij hem was.
\par 7 En hij bouwde aldaar een altaar, en noemde die plaats El Beth-el; want God was hem aldaar geopenbaard geweest, als hij voor zijns broeders aangezicht vlood.
\par 8 En Debora, de voedster van Rebekka, stierf, en zij werd begraven onder aan Beth-el; onder dien eik, welks naam hij noemde Allon-bachuth.
\par 9 En God verscheen Jakob wederom, als hij van Paddan-aram gekomen was; en Hij zegende hem.
\par 10 En God zeide tot hem: Uw naam is Jakob, uw naam zal voortaan niet Jakob genoemd worden, maar Israel zal uw naam zijn; en Hij noemde zijn naam Israel.
\par 11 Voorts zeide God tot hem: Ik ben God de Almachtige! wees vruchtbaar, en vermenigvuldig! Een volk, ja, een hoop der volken zal uit u worden, en koningen zullen uit uw lenden voortkomen.
\par 12 En dit land, dat Ik aan Abraham en Izak gegeven heb, dat zal Ik u geven; en aan uw zaad na u zal Ik dit land geven.
\par 13 Toen voer God van hem op in die plaats, waar Hij met hem gesproken had.
\par 14 En Jakob stelde een opgericht teken op in die plaats, waar Hij met hem gesproken had, een stenen opgericht teken; en hij stortte daarop drankoffer, en goot olie daarover.
\par 15 En Jakob noemde den naam dier plaats, alwaar God met hem gesproken had, Beth-el.
\par 16 En zij reisden van Beth-el; en er was nog een kleine streek lands om tot Efrath te komen; en Rachel baarde, en zij had het hard in haar baren.
\par 17 En het geschiedde, als zij het hard had in haar baren, zo zeide de vroedvrouw tot haar: Vrees niet; want dezen zoon zult gij ook hebben!
\par 18 En het geschiedde, als haar ziel uitging (want zij stierf), dat zij zijn naam noemde Ben-oni; maar zijn vader noemde hem Benjamin.
\par 19 Alzo stierf Rachel; en zij werd begraven aan den weg naar Efrath, hetwelk is Bethlehem.
\par 20 En Jakob richtte een gedenkteken op boven haar graf, dit is het gedenkteken van Rachels graf tot op dezen dag.
\par 21 Toen verreisde Israel, en hij spande zijn tent op gene zijde van Migdal-eder.
\par 22 En het geschiedde, als Israel in dat land woonde, dat Ruben heenging, en lag bij Bilha, zijns vaders bijwijf; en Israel hoorde het. En de zonen van Jakob waren twaalf.
\par 23 De zonen van Lea waren: Ruben, Jakobs eerstgeborene, daarna Simeon, en Levi, en Juda, en Issaschar, en Zebulon.
\par 24 De zonen van Rachel: Jozef en Benjamin.
\par 25 En de zonen van Bilha, Rachels dienstmaagd: Dan en Nafthali.
\par 26 En de zonen van Zilpa, Lea's dienstmaagd: Gad en Aser. Deze zijn de zonen van Jakob, die hem geboren zijn in Paddan-aram.
\par 27 En Jakob kwam tot Izak, zijn vader, in Mamre, te Kirjath-arba, hetwelk is Hebron, waar Abraham als vreemdeling had verkeerd, en Izak.
\par 28 En de dagen van Izak waren honderd jaren, en tachtig jaren.
\par 29 En Izak gaf den geest en stierf, en werd verzameld tot zijn volken, oud en zat van dagen; en zijn zonen Ezau en Jakob begroeven hem.

\chapter{36}

\par 1 Dit nu zijn de geboorten van Ezau, welke is Edom.
\par 2 Ezau nam zijn vrouwen uit de dochteren van Kanaan, Ada, de dochter van Elon, den Hethiet, en Aholibama, de dochter van Ana, de dochter van Zibeon, den Heviet;
\par 3 En Basmath, de dochter van Ismael, zuster van Nebajoth.
\par 4 Ada nu baarde aan Ezau Elifaz, en Basmath baarde Rehuel.
\par 5 En Aholibama baarde Jehus, en Jaelam, en Korah. Dit zijn de zonen van Ezau, die hem geboren zijn in het land Kanaan.
\par 6 Ezau nu had genomen zijn vrouwen, en zijn zonen, en zijn dochters, en al de zielen zijns huizes, en zijn vee, en al zijn beesten, en al zijn bezitting, die hij in het land Kanaan geworven had, en was vertrokken naar een ander land, van het aangezicht van zijn broeder Jakob.
\par 7 Want hun have was te veel, om samen te wonen; en het land hunner vreemdelingschappen kon ze niet dragen vanwege hun vee.
\par 8 Derhalve woonde Ezau op het gebergte Seir. Ezau is Edom.
\par 9 Dit nu zijn de geboorten van Ezau, den vader der Edomieten, op het gebergte van Seir.
\par 10 Dit zijn de namen der zonen van Ezau: Elifaz, de zoon van Ada, Ezau's huisvrouw; Rehuel, de zoon van Basmath, Ezau's huisvrouw.
\par 11 En de zonen van Elifaz waren: Teman, Omar, Zefo, en Gaetam, en Kenaz.
\par 12 En Timna was een bijwijf van Elifaz, den zoon van Ezau, en zij baarde aan Elifaz Amalek; dit zijn de zonen van Ada, Ezau's huisvrouw.
\par 13 En dit zijn de zonen van Rehuel: Nahath, en Zerah, Samma en Mizza; dat zijn geweest de zonen van Basmath, Ezau's huisvrouw.
\par 14 En dit zijn geweest de zonen van Aholibama, dochter van Ana, dochter van Zibeon, Ezau's huisvrouw; en zij baarde aan Ezau Jehus, en Jaelam, en Korah.
\par 15 Dit zijn de vorsten der zonen van Ezau: de zonen van Elifaz, den eerstgeborene van Ezau, waren: de vorst Teman, de vorst Omar, de vorst Zefo, de vorst Kenaz.
\par 16 De vorst Korah, de vorst Gaetam, de vorst Amalek; dat zijn de vorsten van Elifaz in het land Edom; dat zijn de zonen van Ada.
\par 17 En dit zijn de zonen van Rehuel, den zoon van Ezau: de vorst Nahath, de vorst Zerah, de vorst Samma, de vorst Mizza; dat zijn de vorsten van Rehuel in het land Edom; dat zijn de zonen van Basmath, de huisvrouw van Ezau.
\par 18 En dit zijn de zonen van Aholibama, de huisvrouw van Ezau: de vorst Jehus, de vorst Jaelam, de vorst Korah; dat zijn de vorsten van Aholibama, de dochter van Ana, de huisvrouw van Ezau.
\par 19 Dat zijn de zonen van Ezau, en dat zijn hunlieder vorsten; hij is Edom.
\par 20 Dit zijn de zonen van Seir, den Horiet, inwoners van dat land: Lotan, en Sobal, en Zibeon, en Ana,
\par 21 En Dison, en Ezer, en Disan; dat zijn de vorsten der Horieten, zonen van Seir, in het land van Edom.
\par 22 En de zonen van Lotan waren Hori en Hemam; en Lotans zuster was Timna.
\par 23 En dit zijn de zonen van Sobal: Alvan en Manahath, en Ebal, en Sefo, en Onam.
\par 24 En dit zijn de zonen van Zibeon: Aja en Ana, hij is die Ana, die de muilen in de woestijn gevonden heeft, toen hij de ezels van zijn vader Zibeon weidde.
\par 25 En dit zijn de zonen van Ana: Dison; en Aholibama was de dochter van Ana.
\par 26 En dit zijn de zonen van Dison: Hemdan, en Esban, en Ithran, en Cheran.
\par 27 Dit zijn de zonen van Ezer: Bilhan, en Zaavan, en Akan.
\par 28 Dit zijn de zonen van Disan: Uz en Aran.
\par 29 Dit zijn de vorsten der Horieten: de vorst Lotan, de vorst Sobal, de vorst Zibeon, de vorst Ana.
\par 30 De vorst Dison, de vorst Ezer, de vorst Disan; dit zijn de vorsten der Horieten, naar hun vorsten in het land Seir.
\par 31 En dit zijn koningen, die geregeerd hebben in het land Edom, eer een koning regeerde over de kinderen Israels.
\par 32 Bela dan, de zoon van Beor, regeerde in Edom, en de naam zijner stad was Dinhaba.
\par 33 En Bela stierf, en Jobab, de zoon van Zerah, van Bozra, regeerde in zijn plaats.
\par 34 En Jobab stierf, en Husam, uit der Temanieten land, regeerde in zijn plaats.
\par 35 En Husam stierf, en in zijn plaats regeerde Hadad, de zoon van Bedad, die Midian versloeg in het veld van Moab; en de naam zijner stad was Avith.
\par 36 En Hadad stierf, en Samla, van Masreka, regeerde in zijn plaats.
\par 37 En Samla stierf, en Saul van Rehoboth, aan de rivier, regeerde in zijn plaats.
\par 38 En Saul stierf, en Baal-hanan, de zoon van Achbor, regeerde in zijn plaats.
\par 39 En Baal-hanan, de zoon van Achbor, stierf, en Hadar regeerde in zijn plaats; en de naam zijner stad was Pahu; en de naam zijner huisvrouw was Mechetabeel, een dochter van Matred, de dochter van Me-zahab.
\par 40 En dit zijn de namen der vorsten van Ezau, naar hun geslachten, naar hun plaatsen, met hun namen: de vorst Timna, de vorst Alva, de vorst Jetheth,
\par 41 De vorst Aholibama, de vorst Ela, de vorst Pinon,
\par 42 De vorst Kenaz, de vorst Teman, de vorst Mibzar,
\par 43 De vorst Magdiel, de vorst Iram; dit zijn de vorsten van Edom, naar hun woningen, in het land hunner bezitting; hij is Ezau, de vader van Edom.

\chapter{37}

\par 1 En Jakob woonde in het land der vreemdelingschappen zijns vaders, in het land Kanaan.
\par 2 Dit zijn Jakobs geschiedenissen. Jozef, zijnde een zoon van zeventien jaren, weidde de kudde met zijn broeders (en hij was een jongeling), met de zonen van Bilha, en de zonen van Zilpa, zijns vaders vrouwen; en Jozef bracht hun kwaad gerucht tot hun vader.
\par 3 En Israel had Jozef lief, boven al zijn zonen; want hij was hem een zoon des ouderdoms; en hij maakte hem een veelvervigen rok.
\par 4 Als nu zijn broeders zagen, dat hun vader hem boven al zijn broederen liefhad, haatten zij hem, en konden hem niet vredelijk toespreken.
\par 5 Ook droomde Jozef een droom, dien hij aan zijn broederen vertelde; daarom haatten zij hem nog te meer.
\par 6 En hij zeide tot hen: Hoort toch dezen droom, dien ik gedroomd heb.
\par 7 En ziet, wij waren schoven bindende in het midden des velds; en ziet, mijn schoof stond op, en bleef ook staande; en ziet, uw schoven kwamen rondom, en bogen zich neder voor mijn schoof.
\par 8 Toen zeiden zijn broeders tot hem: Zult gij dan ganselijk over ons regeren: zult gij dan ganselijk over ons heersen? Zo haatten zij hem nog te meer, om zijn dromen en om zijn woorden.
\par 9 En hij droomde nog een anderen droom, en verhaalde dien aan zijn broederen; en hij zeide: Ziet, ik heb nog een droom gedroomd, en ziet, de zon, en de maan, en elf sterren bogen zich voor mij neder.
\par 10 En als hij het aan zijn vader en aan zijn broederen verhaalde, bestrafte hem zijn vader, en zeide tot hem: Wat is dit voor een droom, dien gij gedroomd hebt; zullen wij dan ganselijk komen, ik, en uw moeder, en uw broeders, om ons voor u ter aarde te buigen?
\par 11 Zijn broeders dan benijdden hem; doch zijn vader bewaarde deze zaak.
\par 12 En zijn broeders gingen heen, om de kudde van hun vader te weiden bij Sichem.
\par 13 Zo zeide Israel tot Jozef: Weiden uw broeders niet bij Sichem? Kom, dat ik u tot hen zende. En hij zeide tot hem: Zie, hier ben ik!
\par 14 En hij zeide tot hem: Ga toch heen, zie naar den welstand van uw broederen, en naar den welstand van de kudde, en breng mij een woord wederom. Zo zond hij hem uit het dal Hebron, en hij kwam te Sichem.
\par 15 En een man vond hem (want ziet, hij was dwalende in het veld); zo vraagde hem deze man, zeggende: Wat zoekt gij?
\par 16 En hij zeide: Ik zoek mijn broederen; geef mij toch te kennen, waar zij weiden.
\par 17 Zo zeide die man: Zij zijn van hier gereisd; want ik hoorde hen zeggen: Laat ons naar Dothan gaan. Jozef dan ging zijn broederen na, en vond hen te Dothan.
\par 18 En zij zagen hem van verre; en eer hij tot hen naderde, sloegen zij tegen hem een listigen raad, om hem te doden.
\par 19 En zij zeiden de een tot den ander: Ziet, daar komt die meesterdromer aan!
\par 20 Nu komt dan, en laat ons hem doodslaan, en hem in een dezer kuilen werpen; en wij zullen zeggen: een boos dier heeft hem opgegeten; zo zullen wij zien, wat van zijn dromen worden zal.
\par 21 Ruben hoorde dat, en verloste hem uit hun hand; en hij zeide: Laat ons hem niet aan het leven slaan.
\par 22 Ook zeide Ruben tot hen: Vergiet geen bloed; werpt hem in dezen kuil die in de woestijn is, en legt de hand niet aan hem; opdat hij hem uit hun hand verloste, om hem tot zijn vader weder te brengen.
\par 23 En het geschiedde, als Jozef tot zijn broederen kwam, zo togen zij Jozef zijn rok uit, den veelvervigen rok, dien hij aanhad.
\par 24 En zij namen hem, en wierpen hem in den kuil; doch de kuil was ledig; er was geen water in.
\par 25 Daarna zaten zij neder om brood te eten, en hieven hun ogen op, en zagen, en ziet, een reisgezelschap van Ismaelieten kwam uit Gilead; en hun kemelen droegen specerijen en balsem, en mirre, reizende, om dat af te brengen naar Egypte.
\par 26 Toen zeide Juda tot zijn broederen: Wat gewin zal het zijn, dat wij onzen broeder doodslaan, en zijn bloed verbergen?
\par 27 Komt, en laat ons hem aan deze Ismaelieten verkopen, en onze hand zij niet aan hem; want hij is onze broeder, ons vlees, en zijn broederen hoorden hem.
\par 28 Als nu de Midianietische kooplieden voorbijtogen, zo trokken en hieven zij Jozef op uit den kuil, en verkochten Jozef aan deze Ismaelieten voor twintig zilverlingen; die brachten Jozef naar Egypte.
\par 29 Als nu Ruben tot den kuil wederkeerde, ziet, zo was Jozef niet in den kuil; toen scheurde hij zijn klederen.
\par 30 En hij keerde weder tot zijn broederen, en zeide: De jongeling is er niet; en ik, waar zal ik heengaan?
\par 31 Toen namen zij Jozefs rok, en zij slachtten een geitenbok, en zij doopten den rok in het bloed.
\par 32 En zij zonden den veelvervigen rok, en deden hem tot hun vader brengen, en zeiden: Dezen hebben wij gevonden; beken toch, of deze uws zoons rok zij, of niet.
\par 33 En hij bekende hem, en zeide: Het is mijns zoons rok! een boos dier heeft hem opgegeten! voorzeker is Jozef verscheurd!
\par 34 Toen scheurde Jakob zijn klederen, en legde een zak om zijn lenden; en hij bedreef rouw over zijn zoon vele dagen.
\par 35 En al zijn zonen, en al zijn dochteren maakten zich op, om hem te troosten; maar hij weigerde zich te laten troosten, en zeide: Want ik zal, rouw bedrijvende, tot mijn zoon in het graf nederdalen. Alzo beweende hem zijn vader.
\par 36 En de Midianieten verkochten hem in Egypte, aan Potifar, een hoveling van Farao, overste der trawanten.

\chapter{38}

\par 1 En het geschiedde ten zelven tijde, dat Juda van zijn broederen aftoog, en hij keerde in tot een man van Adullam, wiens naam was Hira.
\par 2 En Juda zag aldaar de dochter van een Kanaanietisch man, wiens naam was Sua; en hij nam haar, en ging tot haar in.
\par 3 En zij werd bevrucht, en baarde een zoon, en hij noemde zijn naam Er.
\par 4 Daarna werd zij weder bevrucht, en baarde een zoon, en zij noemde zijn naam Onan.
\par 5 En zij voer nog voort, en baarde een zoon, en noemde zijn naam Sela; doch hij was te Chezib, toen zij hem baarde.
\par 6 Juda nu nam een vrouw voor Er, zijn eerstgeborene, en haar naam was Thamar.
\par 7 Maar Er, de eerstgeborene van Juda, was kwaad in des HEEREN ogen; daarom doodde hem de HEERE.
\par 8 Toen zeide Juda tot Onan: Ga in tot uws broeders huisvrouw, en trouw haar in uws broeders naam, en verwek uw broeder zaad.
\par 9 Doch Onan, wetende, dat dit zaad voor hem niet zoude zijn, zo geschiedde het, als hij tot zijns broeders huisvrouw inging, dat hij het verdierf tegen de aarde, om zijn broeder geen zaad te geven.
\par 10 En het was kwaad in des HEEREN ogen, wat hij deed; daarom doodde Hij hem ook.
\par 11 Toen zeide Juda tot Thamar, zijn schoondochter: Blijf weduwe in uws vaders huis, totdat mijn zoon Sela groot wordt; want hij zeide: Dat niet misschien ook deze sterve, gelijk zijn broeders! Zo ging Thamar heen, en bleef in haar vaders huis.
\par 12 Als nu vele dagen verlopen waren, stierf de dochter van Sua, de huisvrouw van Juda; daarna troostte zich Juda, en ging op tot zijn schaapscheerders naar Timna toe, hij en Hira, zijn vriend, de Adullamiet.
\par 13 En men gaf Thamar te kennen, zeggende: Zie, uw schoonvader gaat op naar Timna, om zijn schapen te scheren.
\par 14 Toen leide zij de klederen van haar weduwschap van zich af, en zij bedekte zich met een sluier, en bewond zich, en zette zich aan den ingang der twee fonteinen, die op den weg naar Timna is; want zij zag, dat Sela groot geworden was, en zij hem niet ter vrouw was gegeven.
\par 15 Als Juda haar zag, zo hield hij haar voor een hoer, overmits zij haar aangezicht bedekt had.
\par 16 En hij week tot haar naar den weg, en zeide: Kom toch, laat mij tot u ingaan; want hij wist niet, dat zij zijn schoondochter was. En zij zeide: Wat zult gij mij geven, dat gij tot mij ingaat?
\par 17 En hij zeide: Ik zal u een geitenbok van de kudde zenden. En zij zeide: Zo gij pand zult geven, totdat gij hem zendt.
\par 18 Toen zeide hij: Wat pand is het, dat ik u geven zal? En zij zeide: Uw zegelring en uw snoer en uw staf, die in uw hand is; hetwelk hij haar gaf, en ging tot haar in; en zij ontving bij hem.
\par 19 En zij maakte zich op, en ging heen, en leide haar sluier van zich af, en zij trok aan de klederen van haar weduwschap.
\par 20 En Juda zond den geitenbok door de hand van zijn vriend, den Adullamiet, om het pand uit de hand der vrouw te nemen; maar hij vond haar niet.
\par 21 En hij vraagde de lieden van haar plaats, zeggende: Waar is de hoer, die bij deze twee fonteinen aan den weg was? En zij zeiden: Hier is geen hoer geweest.
\par 22 En hij keerde weder tot Juda, en zeide: Ik heb haar niet gevonden; en ook zeiden de lieden van die plaats: Hier is geen hoer geweest.
\par 23 Toen zeide Juda: Zij neme het voor zich, opdat wij misschien niet tot verachting worden; zie, ik heb dezen bok gezonden; maar gij hebt haar niet gevonden.
\par 24 En het geschiedde omtrent na drie maanden, dat men Juda te kennen gaf, zeggende: Thamar, uw schoondochter, heeft gehoereerd, en ook zie, zij is zwanger van hoererij. Toen zeide Juda: Breng ze hervoor, dat zij verbrand worde!
\par 25 Als zij voorgebracht werd, schikte zij tot haar schoonvader, om te zeggen: Bij den man, wiens deze dingen zijn, ben ik zwanger; en zij zeide: Beken toch, wiens deze zegelring, en deze snoeren, en deze staf zijn.
\par 26 En Juda kende ze, en zeide: Zij is rechtvaardiger dan ik, daarom, omdat ik haar aan mijn zoon Sela niet gegeven heb. En hij bekende haar voortaan niet meer.
\par 27 En het geschiedde ten tijde, als zij baren zou, ziet, zo waren tweelingen in haar buik.
\par 28 En het geschiedde, als zij baarde, dat een de hand uitgaf; en de vroedvrouw nam dezelve, en zij bond een scharlaken draad om zijn hand, zeggende: Deze komt het eerst uit.
\par 29 Maar het geschiedde, als hij zijn hand weder intoog, ziet, zo kwam zijn broeder uit; en zij zeide: Hoe zijt gij doorgebroken? op u is de breuke! en men noemde zijn naam Perez.
\par 30 En daarna kwam zijn broeder uit, om wiens hand de scharlaken draad was; en men noemde zijn naam Zera.

\chapter{39}

\par 1 Jozef nu werd naar Egypte afgevoerd; en Potifar, een hoveling van Farao, een overste der trawanten, een Egyptisch man, kocht hem uit de hand der Ismaelieten, die hem derwaarts afgevoerd hadden.
\par 2 En de HEERE was met Jozef, zodat hij een voorspoedig man was; en hij was in het huis van zijn heer, den Egyptenaar.
\par 3 Als nu zijn heer zag, dat de HEERE met hem was, en dat de HEERE al wat hij deed, door zijn hand voorspoedig maakte;
\par 4 Zo vond Jozef genade in zijn ogen, en diende hem; en hij stelde hem over zijn huis; en al wat hij had, gaf hij in zijn hand.
\par 5 En het geschiedde van toen af, dat hij hem over zijn huis, en over al wat het zijne was, gesteld had, dat de HEERE des Egyptenaars huis zegende, om Jozefs wil; ja, de zegen des HEEREN was in alles, wat hij had, in het huis en in het veld.
\par 6 En hij liet alles, wat hij had, in Jozefs hand, zodat hij met hem van geen ding kennis had, behalve van het brood, dat hij at. En Jozef was schoon van gedaante, en schoon van aangezicht.
\par 7 En het geschiedde na deze dingen, dat de huisvrouw zijns heren haar ogen op Jozef wierp; en zij zeide: lig bij mij!
\par 8 Maar hij weigerde het, en zeide tot de huisvrouw zijns heren: Zie, mijn heer heeft geen kennis met mij, wat er in het huis is; en al wat hij heeft, dat heeft hij in mijn hand gegeven.
\par 9 Niemand is groter in dit huis dan ik, en hij heeft voor mij niets onthouden, dan u, daarin dat gij zijn huisvrouw zijt; hoe zoude ik dan dit een zo groot kwaad doen, en zondigen tegen God!
\par 10 En het geschiedde, als zij Jozef dag op dag aansprak, en hij naar haar niet hoorde, om bij haar te liggen, en bij haar te zijn;
\par 11 Zo gebeurde het op zulk een dag, dat hij in het huis kwam, om zijn werk te doen; en niemand van de lieden des huizes was daar binnenshuis.
\par 12 En zij greep hem bij zijn kleed, zeggende: Lig bij mij! En hij liet zijn kleed in haar hand, en vluchtte, en ging uit naar buiten.
\par 13 En het geschiedde, als zij zag, dat hij zijn kleed in haar hand gelaten had, en naar buiten gevlucht was;
\par 14 Zo riep zij de lieden van haar huis, en sprak tot hen, zeggende: Ziet, hij heeft ons den Hebreeuwsen man ingebracht, om met ons te spotten; hij is tot mij gekomen, om bij mij te liggen, en ik heb geroepen met luider stem;
\par 15 En het geschiedde, als hij hoorde, dat ik mijn stem verhief, en riep, zo verliet hij zijn kleed bij mij, en vluchtte, en ging uit naar buiten.
\par 16 En zij leide zijn kleed bij zich, totdat zijn heer in zijn huis kwam.
\par 17 Toen sprak zij tot hem naar diezelfde woorden, zeggende: De Hebreeuwse knecht, dien gij ons hebt ingebracht, is tot mij gekomen, om met mij te spotten.
\par 18 En het is geschied, als ik mijn stem verhief, en riep, dat hij zijn kleed bij mij liet, en vluchtte naar buiten.
\par 19 En het geschiedde, als zijn heer de woorden zijner huisvrouw hoorde, die zij tot hem sprak, zeggende: Naar deze zelfde woorden heeft mij uw knecht gedaan, zo ontstak zijn toorn.
\par 20 En Jozefs heer nam hem, en leverde hem in het gevangenhuis, ter plaatse, waar des konings gevangenen gevangen waren; alzo was hij daar in het gevangenhuis.
\par 21 Doch de HEERE was met Jozef, en wende Zijn goedertierenheid tot hem; en gaf hem genade in de ogen van den overste van het gevangenhuis.
\par 22 En de overste van het gevangenhuis gaf al de gevangenen, die in het gevangenhuis waren, in Jozefs hand; en al wat zij daar deden, deed hij.
\par 23 De overste van het gevangenhuis zag gans op geen ding, dat in zijn hand was, overmits dat de HEERE met hem was; en wat hij deed, dat deed de HEERE wel gedijen.

\chapter{40}

\par 1 En het geschiedde na deze dingen, dat de schenker des konings van Egypte, en de bakker, zondigden tegen hun heer, tegen den koning van Egypte.
\par 2 Zodat Farao zeer toornig werd op zijn twee hovelingen, op den overste der schenkers, en op den overste der bakkers.
\par 3 En hij leverde hen in bewaring, ten huize van den overste der trawanten, in het gevangenhuis, ter plaatse, waar Jozef gevangen was.
\par 4 En de overste der trawanten bestelde Jozef bij hen, dat hij hen diende; en zij waren sommige dagen in bewaring.
\par 5 Zij droomden nu beiden een droom, elk zijn droom, in een nacht, elk naar de uitlegging zijns drooms, de schenker en de bakker, die des konings van Egypte waren, die gevangen waren in het gevangenhuis.
\par 6 En Jozef kwam des morgens tot hen, en hij zag hen aan, en ziet, zij waren ontsteld.
\par 7 Toen vraagde hij de hovelingen van Farao, die bij hem waren in hechtenis van het huis zijns heren, zeggende: Waarom zijn uw aangezichten heden kwalijk gesteld?
\par 8 En zij zeiden tot hem: Wij hebben een droom gedroomd, en er is niemand, die hem uitlegge. En Jozef zeide tot hen: Zijn de uitleggingen niet van God? Vertelt ze mij toch.
\par 9 Toen vertelde de overste der schenkers Jozef zijn droom, en zeide tot hem: In mijn droom, zie, zo was een wijnstok voor mijn aangezicht;
\par 10 En aan den wijnstok waren drie ranken; en hij was als bottende, zijn bloeisel ging op, zijn trossen brachten rijpe druiven voort.
\par 11 En Farao's beker was in mijn hand; en ik nam die druiven, en drukte ze uit in Farao's beker, en ik gaf den beker op Farao's hand.
\par 12 Toen zeide Jozef tot hem: Dit is zijn uitlegging: de drie ranken zijn drie dagen.
\par 13 Binnen nog drie dagen zal Farao uw hoofd verheffen, en zal u in uw staat herstellen; en gij zult Farao's beker in zijn hand geven, naar de vorige wijze, toen gij zijn schenker waart.
\par 14 Doch gedenk mijner bij uzelven, wanneer het u wel gaan zal, en doe toch weldadigheid aan mij, en doe van mij melding bij Farao, en maak, dat ik uit dit huis kome.
\par 15 Want ik ben diefelijk ontstolen uit het land der Hebreen; en ook heb ik hier niets gedaan, dat zij mij in dezen kuil gezet hebben.
\par 16 Toen de overste der bakkers zag, dat hij een goede uitlegging gedaan had, zo zeide hij tot Jozef: Ik was ook in mijn droom, en zie, drie getraliede korven waren op mijn hoofd.
\par 17 En in den oppersten korf was van alle spijze van Farao, die bakkerswerk is; en het gevogelte at dezelve uit den korf, van boven mijn hoofd.
\par 18 Toen antwoordde Jozef, en zeide: Dit is zijn uitlegging: de drie korven zijn drie dagen.
\par 19 Binnen nog drie dagen zal Farao uw hoofd verheffen van boven u, en hij zal u aan een hout hangen, en het gevogelte zal uw vlees van boven u eten.
\par 20 En het geschiedde op den derden dag, den dag van Farao's geboorte, dat hij voor al zijn knechten een maaltijd maakte; en hij verhief het hoofd van den overste der schenkers, en het hoofd van den overste der bakkers, in het midden zijner knechten.
\par 21 En hij deed den overste der schenkers wederkeren tot zijn schenkambt, zodat hij den beker op Farao's hand gaf.
\par 22 Maar den overste der bakkers hing hij op; gelijk Jozef hun uitgelegd had.
\par 23 Doch de overste der schenkers gedacht aan Jozef niet, maar vergat hem.

\chapter{41}

\par 1 En het geschiedde ten einde van twee volle jaren, dat Farao droomde, en ziet, hij stond aan de rivier.
\par 2 En ziet, uit de rivier kwamen op zeven koeien, schoon van aanzien, en vet van vlees, en zij weidden in het gras.
\par 3 En ziet, zeven andere koeien kwamen na die op uit de rivier, lelijk van aanzien, en dun van vlees; en zij stonden bij de andere koeien aan den oever der rivier.
\par 4 En die koeien, lelijk van aanzien, en dun van vlees, aten op die zeven koeien, schoon van aanzien en vet. Toen ontwaakte Farao.
\par 5 Daarna sliep hij en droomde andermaal; en ziet, zeven aren rezen op, in een halm, vet en goed.
\par 6 En ziet, zeven dunne en van den oostenwind verzengde aren schoten na dezelve uit.
\par 7 En de dunne aren verslonden de zeven vette en volle aren. Toen ontwaakte Farao, en ziet, het was een droom.
\par 8 En het geschiedde in den morgenstond, dat zijn geest verslagen was, en hij zond heen, en riep al de tovenaars van Egypte, en al de wijzen, die daarin waren; en Farao vertelde hun zijn droom; maar er was niemand, die ze aan Farao uitlegde.
\par 9 Toen sprak de overste der schenkers tot Farao, zeggende: Ik gedenk heden aan mijn zonden.
\par 10 Farao was zeer vertoornd op zijn dienaars, en leverde mij in bewaring ten huize van den overste der trawanten, mij en den overste der bakkers.
\par 11 En in een nacht droomden wij een droom, ik en hij; wij droomden elk naar de uitlegging zijns drooms.
\par 12 En aldaar was bij ons een Hebreeuws jongeling, een knecht van den overste der trawanten; en wij vertelden ze hem, en hij leide ons onze dromen uit; een ieder leide hij ze uit, naar zijn droom.
\par 13 En gelijk hij ons uitleide, alzo is het geschied; mij heeft hij hersteld in mijn staat, en hem gehangen.
\par 14 Toen zond Farao en riep Jozef en zij deden hem haastelijk uit den kuil komen; en men schoor hem, en men veranderde zijn klederen; en hij kwam tot Farao.
\par 15 En Farao sprak tot Jozef: Ik heb een droom gedroomd, en er is niemand, die hem uitlegge; maar ik heb van u horen zeggen, als gij een droom hoort, dat gij hem uitlegt.
\par 16 En Jozef antwoordde Farao, zeggende: Het is buiten mij! God zal Farao's welstand aanzeggen.
\par 17 Toen sprak Farao tot Jozef: Zie, in mijn droom stond ik aan den oever der rivier;
\par 18 En zie, uit de rivier kwamen op zeven koeien, vet van vlees en schoon van gedaante, en zij weidden in het gras.
\par 19 En zie, zeven andere koeien kwamen op na deze, mager en zeer lelijk van gedaante, rank van vlees; ik heb dergelijke van lelijkheid niet gezien in het ganse Egypteland.
\par 20 En die ranke en lelijke koeien aten die eerste zeven vette koeien op;
\par 21 Dewelke in haar buik inkwamen; maar men merkte niet, dat ze in haar buik ingekomen waren; want haar aanzien was lelijk, gelijk als in het begin. Toen ontwaakte ik.
\par 22 Daarna zag ik in mijn droom, en zie, zeven aren rezen op in een halm, vol en goed.
\par 23 En zie, zeven dorre, dunne en van den oostenwind verzengde aren, schoten na dezelve uit;
\par 24 En de zeven dunne aren verslonden die zeven goede aren. En ik heb het den tovenaars gezegd; maar er was niemand, die het mij verklaarde.
\par 25 Toen zeide Jozef tot Farao: De droom van Farao is een; hetgeen God is doende, heeft Hij Farao te kennen gegeven.
\par 26 Die zeven schone koeien zijn zeven jaren; die zeven schone aren zijn ook zeven jaren; de droom is een.
\par 27 En die zeven ranke en lelijke koeien, die na gene opkwamen, zijn zeven jaren; en die zeven ranke van den oostenwind verzengde aren zullen zeven jaren des hongers wezen.
\par 28 Dit is het woord, hetwelk ik tot Farao gesproken heb: hetgeen God is doende, heeft Hij Farao vertoond.
\par 29 Zie, de zeven aankomende jaren, zal er grote overvloed in het ganse land van Egypte zijn.
\par 30 Maar na dezelve zullen er opstaan zeven jaren des hongers; dan zal in het land van Egypte al die overvloed vergeten worden; en de honger zal het land verteren.
\par 31 Ook zal de overvloed in het land niet gemerkt worden, vanwege dienzelven honger, die daarna wezen zal; want hij zal zeer zwaar zijn.
\par 32 En aangaande, dat die droom aan Farao ten tweeden maal is herhaald, is, omdat de zaak van God vastbesloten is, en dat God haast, om dezelve te doen.
\par 33 Zo zie nu Farao naar een verstandigen en wijzen man, en zette hem over het land van Egypte.
\par 34 Farao doe zo, en bestelle opzieners over het land; en neme het vijfde deel des lands van Egypte in de zeven jaren des overvloeds.
\par 35 En dat zij alle spijze van deze aankomende goede jaren verzamelen, en koren opleggen, onder de hand van Farao, tot spijze in de steden, en bewaren het.
\par 36 Zo zal de spijze zijn tot voorraad voor het land, voor zeven jaren des hongers, die in Egypteland wezen zullen; opdat het land van honger niet verga.
\par 37 En dit woord was goed in de ogen van Farao, en in de ogen van al zijn knechten.
\par 38 Zo zeide Farao tot zijn knechten: Zouden wij wel een man vinden als dezen, in welken Gods Geest is?
\par 39 Daarna zeide Farao tot Jozef: Naardien dat God u dit alles heeft verkondigd, zo is er niemand zo verstandig en wijs, als gij.
\par 40 Gij zult over mijn huis zijn, en op uw bevel zal al mijn volk de hand kussen; alleen dezen troon zal ik groter zijn dan gij.
\par 41 Voorts sprak Farao tot Jozef: Zie, ik heb u over gans Egypteland gesteld.
\par 42 En Farao nam zijn ring van zijn hand af, en deed hem aan Jozefs hand, en liet hem fijne linnen klederen aantrekken, en leide hem een gouden keten aan zijn hals;
\par 43 En hij deed hem rijden op den tweeden wagen, dien hij had; en zij riepen voor zijn aangezicht: Knielt! Alzo stelde hij hem over gans Egypteland.
\par 44 En Farao zeide tot Jozef: Ik ben Farao! doch zonder u zal niemand zijn hand of zijn voet opheffen in gans Egypteland.
\par 45 En Farao noemde Jozefs naam Zafnath Paaneah, en gaf hem Asnath, de dochter van Potifera, overste van On, tot een vrouw; en Jozef toog uit door het land van Egypte.
\par 46 Jozef nu was dertig jaren oud, als hij stond voor het aangezicht van Farao, koning van Egypte; en Jozef ging uit van Farao's aangezicht, en hij toog door gans Egypteland.
\par 47 En het land bracht voort, in de zeven jaren des overvloeds, bij handvollen.
\par 48 En hij vergaderde alle spijze der zeven jaren, die in Egypteland was, en deed de spijze in de steden; de spijze van het veld van elke stad, hetwelk rondom haar was, deed hij daar binnen.
\par 49 Alzo bracht Jozef zeer veel koren bijeen, als het zand der zee, totdat men ophield te tellen: want daarvan was geen getal.
\par 50 En Jozef werden twee zonen geboren, eer er een jaar des hongers aankwam, die Asnath, de dochter van Potifera, overste van On, hem baarde.
\par 51 En Jozef noemde den naam des eerstgeborenen Manasse; want, zeide hij God heeft mij doen vergeten al mijn moeite, en het ganse huis mijns vaders.
\par 52 En den naam des tweeden noemde hij Efraim; want, zeide hij God heeft mij doen wassen in het land mijner verdrukking.
\par 53 Toen eindigden de zeven jaren des overvloeds, die in Egypte geweest was.
\par 54 En de zeven jaren des hongers begonnen aan te komen, gelijk als Jozef gezegd had. En er was honger in al de landen; maar in gans Egypteland was brood.
\par 55 Als nu gans Egypteland hongerde, riep het volk tot Farao om brood; en Farao zeide tot alle Egyptenaren: Gaat tot Jozef, doet wat hij u zegt.
\par 56 Als dan honger over het ganse land was, zo opende Jozef alles, waarin iets was, en verkocht aan de Egyptenaren; want de honger werd sterk in Egypteland.
\par 57 En alle landen kwamen in Egypte tot Jozef, om te kopen; want de honger was sterk in alle landen.

\chapter{42}

\par 1 Toen Jakob zag, dat er koren in Egypte was, zo zeide Jakob tot zijn zonen: Waarom ziet gij op elkander?
\par 2 Voorts zeide hij: Ziet, ik heb gehoord, dat er koren in Egypte is; trekt daarhenen af, en koopt ons koren van daar, opdat wij leven en niet sterven.
\par 3 Toen togen Jozefs tien broederen af, om koren uit Egypte te kopen.
\par 4 Doch Benjamin, Jozefs broeder, zond Jakob niet met zijn broederen; want hij zeide: Opdat hem niet misschien het verderf ontmoete!
\par 5 Alzo kwamen Israels zonen om te kopen onder degenen, die daar kwamen; want de honger was in het land Kanaan.
\par 6 Jozef nu was regent over dat land; hij verkocht aan al het volk des lands; en Jozefs broederen kwamen, en bogen zich voor hem, met de aangezichten ter aarde.
\par 7 Als Jozef zijn broederen zag, zo kende hij hen; maar hij hield zich vreemd jegens hen, en sprak hard met hen, en zeide tot hen: Van waar komt gij? En zij zeiden: Uit het land Kanaan; om spijze te kopen.
\par 8 Jozef dan kende zijn broederen; maar zij kenden hem niet.
\par 9 Toen gedacht Jozef aan de dromen, die hij van hen gedroomd had; en hij zeide tot hen: Gij zijt verspieders, gij zijt gekomen om te bezichtigen, waar het land bloot is.
\par 10 En zij zeiden tot hem: Neen, mijn heer! maar uw knechten zijn gekomen, om spijze te kopen.
\par 11 Wij allen zijn eens mans zonen; wij zijn vroom; uw knechten zijn geen verspieders.
\par 12 En hij zeide tot hen: Neen, maar gij zijt gekomen, om te bezichtigen, waar het land bloot is.
\par 13 En zij zeiden: Wij, uw knechten, waren twaalf gebroeders, eens mans zonen, in het land Kanaan; en zie, de kleinste is heden bij onzen vader; doch de een is niet meer.
\par 14 Toen zeide Jozef tot hen: Dat is het, wat ik tot u gesproken heb, zeggende: Gij zijt verspieders!
\par 15 Hierin zult gij beproefd worden: zo waarlijk als Farao leeft! indien gij van hier zult uitgaan, tenzij dan, wanneer uw kleinste broeder herwaarts zal gekomen zijn!
\par 16 Zendt een uit u, die uw broeder hale; maar weest gijlieden gevangen, en uw woorden zullen beproefd worden, of de waarheid bij u zij; en indien niet, zo waarlijk als Farao leeft, zo zijt gij verspieders!
\par 17 En hij zette hen samen drie dagen in bewaring.
\par 18 En ten derden dage zeide Jozef tot hen: Doet dit, zo zult gij leven; ik vrees God.
\par 19 Zo gij vroom zijt, zo zij een uwer broederen gebonden in het huis uwer bewaring; en gaat gij heen, brengt het koren voor den honger uwer huizen.
\par 20 En brengt uw kleinsten broeder tot mij, zo zullen uw woorden waargemaakt worden; en gij zult niet sterven. En zij deden alzo.
\par 21 Toen zeiden zij de een tot den ander: Voorwaar, wij zijn schuldig aan onzen broeder, wiens benauwdheid der ziele wij zagen, toen hij ons om genade bad; maar wij hoorden niet! daarom komt deze benauwdheid over ons.
\par 22 En Ruben antwoordde hun, zeggende: Heb ik het tot u niet gezegd, toen ik zeide: Zondigt niet aan dezen jongeling! maar gij hoordet niet; en ook zijn bloed, ziet, het wordt gezocht!
\par 23 En zij wisten niet, dat het Jozef hoorde; want daar was een taalman tussen hen.
\par 24 Toen wendde hij zich om, van hen af, en weende; daarna keerde hij weder tot hen, en sprak tot hen, en nam Simeon van hen, en bond hem voor hun ogen.
\par 25 En Jozef gebood, dat men hun zakken met koren vullen zou, en dat men hun geld wederkeerde, een iegelijk in zijn zak, en dat men hun teerkost gave tot den weg; en men deed hun alzo.
\par 26 En zij laadden hun koren op hun ezels, en togen van daar.
\par 27 Toen een zijn zak opendeed, om zijn ezel voeder te geven in de herberg, zo zag hij zijn geld; want ziet, het was in den mond van zijn zak.
\par 28 En hij zeide tot zijn broederen: Mijn geld is wedergekeerd; daartoe ook, ziet, het is in mijn zak! Toen ontging hun het hart, en zij verschrikten, de een tot den ander zeggende: Wat is dit, dat ons God gedaan heeft?
\par 29 En zij kwamen in het land Kanaan, tot Jakob, hun vader; en zij gaven hem te kennen al hun wedervaren, zeggende:
\par 30 Die man, de heer van dat land, heeft hard met ons gesproken; en hij heeft ons gehouden voor verspieders des lands.
\par 31 Maar wij zeiden tot hem: Wij zijn vroom; wij zijn geen verspieders.
\par 32 Wij waren twaalf gebroeders, zonen van onzen vader; de een is niet meer, en de kleinste is heden bij onzen vader in het land Kanaan.
\par 33 En die man, de heer van dat land, zeide tot ons: Hieraan zal ik bekennen, dat gijlieden vroom zijt; laat een uwer broederen bij mij, en neemt voor den honger uwer huizen, en trekt heen.
\par 34 En brengt uw kleinsten broeder tot mij; zo zal ik weten, dat gij geen verspieders zijt, maar dat gij vroom zijt; uw broeder zal ik u wedergeven, en gij zult in dit land handelen.
\par 35 En het geschiedde, als zij hun zakken ledigden, ziet, zo had een iegelijk den bundel zijns gelds in zijn zak; en zij zagen de bundelen huns gelds, zij en hun vader, en zij waren bevreesd.
\par 36 Toen zeide Jakob, hun vader, tot hen: Gij berooft mij van kinderen! Jozef is er niet, en Simeon is er niet; nu zult gij Benjamin wegnemen! al deze dingen zijn tegen mij!
\par 37 Toen sprak Ruben tot zijn vader, zeggende: Dood twee mijner zonen, zo ik hem tot u niet wederbreng; geef hem in mijn hand, en ik zal hem weder tot u brengen!
\par 38 Maar hij zeide: Mijn zoon zal met ulieden niet aftrekken; want zijn broeder is dood, en hij is alleen overgebleven; zo hem een verderf ontmoette op den weg, dien gij zult gaan, zo zoudt gij mijn grauwe haren met droefenis ten grave doen nederdalen.

\chapter{43}

\par 1 De honger nu werd zwaar in dat land;
\par 2 Zo geschiedde het, als zij den leeftocht, dien zij uit Egypte gebracht hadden, opgegeten hadden, dat hun vader tot hen zeide: Keert wederom, koopt ons een weinig spijze.
\par 3 Toen sprak Juda tot hem, zeggende: Die man heeft ons op het hoogste betuigd, zeggende: Gij zult mijn aangezicht niet zien, tenzij dat uw broeder met u is.
\par 4 Indien gij onzen broeder met ons zendt, wij zullen aftrekken, en u spijze kopen;
\par 5 Maar indien gij hem niet zendt, wij zullen niet aftrekken; want die man heeft tot ons gezegd: Gij zult mijn aangezicht niet zien, tenzij dat uw broeder met u is.
\par 6 En Israel zeide: Waarom hebt gij zo kwalijk aan mij gedaan, dat gij dien man te kennen gaaft, of gij nog een broeder hadt?
\par 7 En zij zeiden: Die man vraagde zeer nauw naar ons, en naar onze maagschap, zeggende: Leeft uw vader nog; hebt gij nog een broeder? Zo gaven wij het hem te kennen, volgens diezelfde woorden; hebben wij juist geweten, dat hij zeggen zou: Brengt uw broeder af?
\par 8 Toen zeide Juda tot Israel, zijn vader: Zend den jongeling met mij, zo zullen wij ons opmaken en reizen, opdat wij leven en niet sterven, noch wij, noch gij, noch onze kinderkens.
\par 9 Ik zal borg voor hem zijn; van mijn hand zult gij hem eisen; indien ik hem tot u niet breng en hem voor uw aangezicht stel, zo zal ik alle dagen tegen u gezondigd hebben!
\par 10 Want hadden wij niet gezuimd, voorwaar, wij waren alreeds tweemaal wedergekomen.
\par 11 Toen zeide Israel, hun vader, tot hen: Is het nu alzo, zo doet dit; neemt van het loffelijkste dezes lands in uwe vaten, en brengt dien man een geschenk henen af: een weinig balsem, en een weinig honig, specerijen en mirre, terpentijnnoten en amandelen.
\par 12 En neemt dubbel geld in uw hand; en brengt het geld, hetwelk in den mond uwer zakken wedergekeerd is, weder in uw hand; misschien is het een feil.
\par 13 Neemt ook uw broeder mede, en maakt u op, keert weder tot dien man.
\par 14 En God, de Almachtige, geve u barmhartigheid voor het aangezicht van dien man, dat hij uw anderen broeder en Benjamin met u late gaan! En mij aangaande, als ik van kinderen beroofd ben, zo ben ik beroofd!
\par 15 En die mannen namen dat geschenk, en namen dubbel geld in hun hand, en Benjamin; en zij maakten zich op, en togen af naar Egypte, en zij stonden voor Jozefs aangezicht.
\par 16 Als Jozef Benjamin met hen zag, zo zeide hij tot dengene, die over zijn huis was: Breng deze mannen naar het huis toe, en slacht slachtvee, en maak het gereed; want deze mannen zullen te middag met mij eten.
\par 17 De man nu deed, gelijk Jozef gezegd had; en de man bracht deze mannen in het huis van Jozef.
\par 18 Toen vreesden deze mannen, omdat zij in het huis van Jozef gebracht werden, en zeiden: Ter oorzake van het geld, dat in het begin in onze zakken wedergekeerd is, worden wij ingebracht, opdat hij ons overrompele en ons overvalle, en ons tot slaven neme, met onze ezelen.
\par 19 Daarom naderden zij tot dien man, die over het huis van Jozef was, en zij spraken tot hem aan de deur van het huis.
\par 20 En zij zeiden: Och, mijn heer! wij waren in het begin gewisselijk afgekomen, om spijze te kopen.
\par 21 Het is nu geschied, als wij in de herberg gekomen waren, en wij onze zakken opendeden, zie, zo was ieders mans geld in den mond van zijn zak, ons geld in zijn gewicht; en wij hebben hetzelve wedergebracht in onze hand.
\par 22 Wij hebben ook ander geld in onze hand afgebracht, om spijze te kopen; wij weten niet, wie ons geld in onze zakken gelegd heeft.
\par 23 En hij zeide: Vrede zij ulieden, vreest niet! Uw God en de God uws vaders heeft u een schat in uw zakken gegeven; uw geld is tot mij gekomen. En hij bracht Simeon tot hen uit.
\par 24 Daarna bracht de man deze mannen in het huis van Jozef, en hij gaf water; en zij wiesen hun voeten; hij gaf ook aan hun ezelen voeder.
\par 25 En zij bereidden het geschenk, totdat Jozef kwam op den middag; want zij hadden gehoord, dat zij aldaar brood eten zouden.
\par 26 Als nu Jozef te huis gekomen was, zo brachten zij hem het geschenk, hetwelk in hun hand was, in het huis, en zij bogen zich voor hem ter aarde.
\par 27 En hij vraagde hun naar hun welstand, en zeide: Is het wel met uw vader, den oude, waarvan gij zeidet? Leeft hij nog?
\par 28 En zij zeiden: Het is wel met uw knecht, onzen vader, hij leeft nog; en zij neigden het hoofd en bogen zich neder.
\par 29 En hij hief zijn ogen op, en zag Benjamin, zijn broeder, den zoon zijner moeder, en zeide: Is dit uw kleinste broeder, waarvan gij tot mij zeidet? Daarna zeide hij: Mijn zoon? God zij u genadig!
\par 30 En Jozef haastte zich; want zijn ingewand ontstak jegens zijn broeder, en hij zocht te wenen; en hij ging in een kamer, en weende aldaar.
\par 31 Daarna wies hij zijn aangezicht en kwam uit; en hij bedwong zichzelven, en zeide: Zet brood op.
\par 32 En zij richtten voor hem aan in het bijzonder, en voor hen in het bijzonder; en voor de Egyptenaren, die met hem aten, in het bijzonder; want de Egyptenaars mogen geen brood eten met de Hebreen, dewijl zulks den Egyptenaren een gruwel is.
\par 33 En zij aten voor zijn aangezicht, de eerstgeborene naar zijn eerstgeboorte, en de jongere naar zijn jonkheid; dies verwonderden zich de mannen onder elkander.
\par 34 En hij langde hun van de gerechten, die voor hem waren; maar Benjamins gerecht was vijfmaal groter, dan de gerechten van hen allen. En zij dronken, en zij werden dronken met hem.

\chapter{44}

\par 1 En hij gebood dengene, die over zijn huis was, zeggende: Vul de zakken dezer mannen met spijze, naar dat zij zullen kunnen dragen, en leg ieders mans geld in den mond van zijn zak;
\par 2 En mijn beker, den zilveren beker, zult gij leggen in den mond van den zak des kleinsten, met het geld van zijn koren. En hij deed naar Jozefs woord, hetwelk hij gesproken had.
\par 3 Des morgens, als het licht werd, zo liet men deze mannen trekken, hen en hun ezelen.
\par 4 Zij zijn ter stad uitgegaan; zij waren niet verre gekomen, als Jozef tot dengene, die over zijn huis was, zeide: Maak u op, en jaag die mannen achterna; en als gij hen zult achterhaald hebben, zo zult gij tot hen zeggen: Waarom hebt gij kwaad voor goed vergolden?
\par 5 Is het deze niet, waaruit mijn heer drinkt? en waarbij hij iets zekerlijk waarnemen zal? Gij hebt kwalijk gedaan, wat gij gedaan hebt.
\par 6 En hij achterhaalde hen, en sprak tot hen diezelfde woorden.
\par 7 En zij zeiden tot hem: Waarom spreekt mijn heer zulke woorden? Het zij verre van uw knechten, dat zij zodanig een ding doen zouden.
\par 8 Zie, het geld, dat wij in den mond onzer zakken vonden, hebben wij tot u uit het land Kanaan wedergebracht; hoe zouden wij dan uit het huis uws heren zilver of goud stelen?
\par 9 Bij wien van uw knechten hij gevonden zal worden, dat hij sterve; en ook zullen wij mijn heer tot slaven zijn!
\par 10 En hij zeide: Dit zij nu ook alzo, naar uw woorden! Bij wien hij gevonden wordt, die zij mijn slaaf; maar gijlieden zult onschuldig zijn.
\par 11 En zij haastten, en iegelijk zette zijn zak af op de aarde, en iegelijk opende zijn zak.
\par 12 En hij doorzocht, beginnende met den grootste, en voleindigende met den kleinste; en die beker werd gevonden in den zak van Benjamin.
\par 13 Toen scheurden zij hun klederen; en ieder man laadde zijn ezel op, en zij keerden weder naar de stad.
\par 14 En Juda kwam met zijn broederen in het huis van Jozef; want hij was nog zelf aldaar; en zij vielen voor zijn aangezicht neder ter aarde.
\par 15 En Jozef zeide tot hen: Wat daad is dit, die gij gedaan hebt? Weet gij niet, dat zulk een man als ik dat zekerlijk waarnemen zoude?
\par 16 Toen zeide Juda: Wat zullen wij tot mijn heer zeggen, wat zullen wij spreken, en wat zullen wij ons rechtvaardigen? God heeft de ongerechtigheid uwer knechten gevonden; zie, wij zijn mijns heren slaven, zo wij, als hij, in wiens hand de beker gevonden is.
\par 17 Maar hij zeide: Het zij verre van mij zulks te doen! de man, in wiens hand de beker gevonden is, die zal mijn slaaf zijn; doch trekt gijlieden op in vrede tot uw vader.
\par 18 Toen naderde Juda tot hem, en zeide: Och, mijn heer! laat toch uw knecht een woord spreken voor mijns heren oren, en laat uw toorn tegen uw knecht niet ontsteken; want gij zijt even gelijk Farao!
\par 19 Mijn heer vraagde zijn knechten, zeggende: Hebt gijlieden een vader, of broeder?
\par 20 Zo zeiden wij tot mijn heer: Wij hebben een ouden vader, en een jongeling des ouderdoms, den kleinsten, wiens broeder dood is, en hij is alleen van zijn moeder overgebleven, en zijn vader heeft hem lief.
\par 21 Toen zeidet gij tot uw knechten: Brengt hem af tot mij, dat ik mijn oog op hem sla.
\par 22 En wij zeiden tot mijn heer: Die jongeling zal zijn vader niet kunnen verlaten; indien hij zijn vader verlaat, zo zal hij sterven.
\par 23 Toen zeidet gij tot uw knechten: Indien uw kleinste broeder met u niet afkomt, zo zult gij mijn aangezicht niet meer zien.
\par 24 En het is geschied, als wij tot uw knecht, mijn vader, opgetrokken zijn, en wij hem de woorden mijns heren verhaald hebben;
\par 25 En dat onze vader gezegd heeft: Keert weder. koopt ons een weinig spijze;
\par 26 Zo hebben wij gezegd: Wij zullen niet mogen aftrekken; indien onze kleinste broeder bij ons is, zo zullen wij aftrekken; want wij zullen het aangezicht van dien man niet mogen zien, zo deze onze kleinste broeder niet bij ons is.
\par 27 Toen zeide uw knecht, mijn vader, tot ons: Gijlieden weet, dat mijn huisvrouw er mij twee gebaard heeft.
\par 28 En de een is van mij uitgegaan, en ik heb gezegd: Voorwaar, hij is gewisselijk verscheurd geworden! en ik heb hem niet gezien tot nu toe.
\par 29 Indien gij nu dezen ook van mijn aangezicht wegneemt, en hem een verderf ontmoette, zo zoudt gij mijn grauwe haren met jammer ten grave doen nederdalen!
\par 30 Nu dan, als ik tot uw knecht, mijn vader, kome, en de jongeling is niet bij ons (alzo zijn ziel aan de ziel van dezen gebonden is),
\par 31 Zo zal het geschieden, als hij ziet, dat de jongeling er niet is, dat hij sterven zal; en uw knechten zullen de grauwe haren van uw knecht, onzen vader, met droefenis ten grave doen nederdalen.
\par 32 Want uw knecht is voor deze jongeling borg bij mijn vader, zeggende: Zo ik hem tot u niet wederbreng, zo zal ik tegen mijn vader alle dagen gezondigd hebben!
\par 33 Nu dan, laat toch uw knecht voor dezen jongeling slaaf van mijn heer blijven, en laat den jongeling met zijn broederen optrekken!
\par 34 Want hoe zoude ik optrekken tot mijn vader, indien de jongeling niet met mij was, opdat ik den jammer niet zie, welke mijn vader overkomen zou.

\chapter{45}

\par 1 Toen kon zich Jozef niet bedwingen voor allen, die bij hem stonden, en hij riep: Doet alle man van mij uitgaan! En er stond niemand bij hem, als Jozef zich aan zijn broederen bekend maakte.
\par 2 En hij verhief zijn stem met wenen, zodat het de Egyptenaren hoorden, en dat het Farao's huis hoorde.
\par 3 En Jozef zeide tot zijn broederen: Ik ben Jozef! leeft mijn vader nog? En zijn broeders konden hem niet antwoorden; want zij waren verschrikt voor zijn aangezicht.
\par 4 En Jozef zeide tot zijn broederen: Nadert toch tot mij! En zij naderden. Toen zeide hij: Ik ben Jozef, uw broeder, dien gij naar Egypte verkocht hebt.
\par 5 Maar nu, weest niet bekommerd, en de toorn ontsteke niet in uw ogen, omdat gij mij hierheen verkocht hebt; want God heeft mij voor uw aangezicht gezonden, tot behoudenis des levens.
\par 6 Want het zijn nu twee jaren des hongers in het midden des lands; en er zijn nog vijf jaren, in welke geen ploeging noch oogst zijn zal.
\par 7 Doch God heeft mij voor uw aangezicht henen gezonden, om u een overblijfsel te stellen op de aarde, en om u bij het leven te behouden, door een grote verlossing.
\par 8 Nu dan, gij hebt mij herwaarts niet gezonden, maar God Zelf, Die mij tot Farao's vader gesteld heeft, en tot een heer over zijn ganse huis, en regeerder in het ganse land van Egypte.
\par 9 Haast u en trekt op tot mijn vader, en zegt het hem: Alzo zegt uw zoon Jozef: God heeft mij tot een heer over gans Egypteland gesteld; kom af tot mij, en vertoef niet.
\par 10 En gij zult in het land Gosen wonen, en nabij mij wezen, gij en uw zonen, en de zonen uwer zonen, en uw schapen, en uw runderen, en al wat gij hebt.
\par 11 En ik zal u aldaar onderhouden; want er zullen nog vijf jaren des hongers zijn, opdat gij niet verarmt, gij en uw huis, en alles wat gij hebt!
\par 12 En ziet, uw ogen zien het, en de ogen van mijn broeder Benjamin, dat mijn mond tot u spreekt.
\par 13 En boodschapt mijn vader al mijn heerlijkheid in Egypte, en alles wat gij gezien hebt; en haast u, en brengt mijn vader herwaarts af.
\par 14 En hij viel aan den hals van Benjamin, zijn broeder, en weende; en Benjamin weende aan zijn hals.
\par 15 En hij kuste al zijn broederen, en hij weende over hen; en daarna spraken zijn broeders met hem.
\par 16 Als dit gerucht in het huis van Farao gehoord werd, dat men zeide: Jozefs broeders zijn gekomen! was het goed in de ogen van Farao, en in de ogen van zijn knechten.
\par 17 En Farao zeide tot Jozef: Zeg tot uw broederen: Doet dit, laadt uw beesten, en trekt heen, gaat naar het land Kanaan;
\par 18 En neemt uw vader en uw huisgezinnen, en komt tot mij, en ik zal u het beste van Egypteland geven, en gij zult het vette dezes lands eten.
\par 19 Gij zijt toch gelast: doet dit, neemt u uit Egypteland wagenen voor uw kinderkens, en voor uw vrouwen, en voert uw vader, en komt.
\par 20 En uw oog verschone uw huisraad niet; want het beste van gans Egypteland, dat zal het uwe zijn.
\par 21 En de zonen van Israel deden alzo. Zo gaf Jozef hun wagenen, naar Farao's bevel; ook gaf hij hun teerkost op den weg.
\par 22 Hij gaf hun allen, ieder een, wisselklederen; maar Benjamin gaf hij driehonderd zilverlingen, en vijf wisselklederen.
\par 23 En zijn vader desgelijks zond hij tien ezelen, dragende van het beste van Egypte, en tien ezelinnen, dragende koren, en brood, en spijze voor zijn vader op den weg.
\par 24 En hij zond zijn broeders heen; en zij vertrokken; en hij zeide tot hen: Verstoort u niet op den weg.
\par 25 En zij trokken op uit Egypte, en zij kwamen in het land Kanaan tot hun vader Jakob.
\par 26 Toen boodschapten zij hem, zeggende: Jozef leeft nog, ja, ook is hij regeerder in gans Egypteland! Toen bezweek zijn hart, want hij geloofde hen niet.
\par 27 Maar als zij tot hem gesproken hadden al de woorden van Jozef, die hij tot hen gesproken had, en dat hij de wagenen zag, die Jozef gezonden had om hem te voeren, zo werd de geest van Jakob hun vader, levendig.
\par 28 En Israel zeide: Het is genoeg! mijn zoon Jozef leeft nog! ik zal gaan, en hem zien, eer ik sterve!

\chapter{46}

\par 1 En Israel verreisde met al wat hij had, en hij kwam te Ber-seba, en hij offerde offeranden aan den God van zijn vader Izak.
\par 2 En God sprak tot Israel in gezichten des nachts, en zeide: Jakob, Jakob! En hij zeide: Zie, hier ben ik!
\par 3 En Hij zeide: Ik ben die God, uws vaders God; vrees niet van af te trekken naar Egypte; want Ik zal u aldaar tot een groot volk zetten.
\par 4 Ik zal met u aftrekken naar Egypte en Ik zal u doen weder optrekken, mede optrekkende; en Jozef zal zijn hand op uw ogen leggen.
\par 5 Toen maakte zich Jakob op van Ber-seba; en de zonen van Israel voerden Jakob, hun vader, en hun kinderen, en hun vrouwen, op de wagenen, die Farao gezonden had, om hem te voeren.
\par 6 En zij namen hun vee, en hun have, die zij in het land Kanaan geworven hadden, en zij kwamen in Egypte, Jakob en al zijn zaad met hem;
\par 7 Zijn zonen, en de zonen zijner zonen met hem; zijn dochteren, en zijner zonen dochteren, en al zijn zaad bracht hij met zich in Egypte.
\par 8 En dit zijn de namen der zonen van Israel, die in Egypte kwamen: Jakob en zijn zonen. De eerstgeborene van Jakob: Ruben.
\par 9 En de zonen van Ruben: Hanoch, en Pallu, en Hezron, en Karmi.
\par 10 En de zonen van Simeon: Jemuel, en Jamin, en Ohad, en Jachin, en Zohar, en Saul, de zoon ener Kanaanietische vrouw.
\par 11 En de zonen van Levi: Gerson, Kehath en Merari.
\par 12 En de zonen van Juda: Er, en Onan, en Sela, en Perez, en Zerah. Doch Er en Onan waren gestorven in het land van Kanaan; en de zonen van Perez waren Hezron en Hamul.
\par 13 En de zonen van Issaschar: Tola, en Puwa, en Job, en Simron.
\par 14 En de zonen van Zebulon: Sered, en Elon, en Jahleel.
\par 15 Dit zijn de zonen van Lea, die zij Jakob gebaard heeft in Paddan-aram, met Dina zijn dochter; al de zielen zijner zonen en zijner dochteren waren drie en dertig.
\par 16 En de zonen van Gad: Zifjon en Haggi, Schuni en Ezbon, Eri en Arodi, en Areli.
\par 17 En de zonen van Aser: Jimna, en Jisva, en Jisvi, en Berija, en Sera, hun zuster; en de zonen van Berija: Heber en Malchiel.
\par 18 Dit zijn de zonen van Zilpa, die Laban aan zijn dochter Lea gegeven had; en zij baarde Jakob deze zestien zielen.
\par 19 De zonen van Rachel, Jakobs huisvrouw: Jozef en Benjamin.
\par 20 En Jozef werden geboren in Egypteland, Manasse en Efraim, die hem Asnath, de dochter van Potifera, den overste te On, baarde.
\par 21 En de zonen van Benjamin: Bela, Becher en Asbel, Gera en Naaman, Echi en Ros, Muppim en Huppim, en Ard.
\par 22 Dit zijn de zonen van Rachel, die Jakob geboren zijn, al te zamen veertien zielen.
\par 23 En de zonen van Dan: Chusim.
\par 24 En de zonen van Nafthali: Jahzeel, en Guni, en Jezer, en Sillem.
\par 25 Dit zijn de zonen van Bilha, die Laban aan zijn dochter Rachel gegeven had; en zij baarde dezelve Jakob, zij waren allen zeven zielen.
\par 26 Al de zielen, die met Jakob in Egypte kwamen, uit zijn heup gesproten, uitgenomen de vrouwen van de zonen van Jakob, waren allen zes en zestig zielen.
\par 27 En de zonen van Jozef, die hem in Egypte geboren zijn, waren twee zielen. Al de zielen van het huis van Jakob, die in Egypte kwamen, waren zeventig.
\par 28 En hij zond Juda voor zijn aangezicht heen tot Jozef, om voor zijn aangezicht aanwijzing te doen naar Gosen; en zij kwamen in het land Gosen.
\par 29 Toen spande Jozef zijn wagen aan, en toog op, zijn vader Israel tegemoet naar Gosen; en als hij zich aan hem vertoonde, zo viel hij hem aan zijn hals, en weende lang aan zijn hals.
\par 30 En Israel zeide tot Jozef: Dat ik nu sterve, nadat ik uw aangezicht gezien heb, dat gij nog leeft!
\par 31 Daarna zeide Jozef tot zijn broederen, en tot zijns vaders huis: Ik zal optrekken en Farao boodschappen, en tot hem zeggen: Mijn broeders en het huis mijns vaders, die in het land Kanaan waren, zijn tot mij gekomen.
\par 32 En die mannen zijn schaapherders; want het zijn mannen, die met vee omgaan; en zij hebben hun schapen, en hun runderen, en al wat zij hebben, medegebracht.
\par 33 Wanneer het nu geschieden zal, dat Farao ulieden zal roepen, en zeggen: Wat is uw hantering?
\par 34 Zo zult gij zeggen: Uw knechten zijn mannen, die van onze jeugd af tot nu toe met vee omgegaan hebben, zo wij als onze vaders; opdat gij in het land Gosen moogt wonen; want alle schaapherder is den Egyptenaren een gruwel.

\chapter{47}

\par 1 Toen kwam Jozef en boodschapte Farao, en zeide: Mijn vader en mijn broeders, en hun schapen, en hun runderen, met alles wat zij hebben, zijn gekomen uit het land Kanaan; en zie, zij zijn in het land Gosen.
\par 2 En hij nam een deel zijner broederen, te weten vijf mannen, en hij stelde hen voor Farao's aangezicht.
\par 3 Toen zeide Farao tot zijn broederen: Wat is uw hantering? En zij zeiden tot Farao: Uw knechten zijn schaapherders, zo wij als onze vaders.
\par 4 Voorts zeiden zij tot Farao: Wij zijn gekomen, om als vreemdelingen in dit land te wonen; want er is geen weide voor de schapen, die uw knechten hebben, dewijl de honger zwaar is in het land Kanaan; en nu, laat toch uw knechten in het land Gosen wonen!
\par 5 Toen sprak Farao tot Jozef, zeggende: Uw vader en uw broeders zijn tot u gekomen;
\par 6 Egypteland is voor uw aangezicht; doe uw vader en uw broeders in het beste van het land wonen; laat hen in het land Gosen wonen, en zo gij weet, dat er onder hen kloeke mannen zijn, zo zet hen tot veemeesters over hetgeen ik heb.
\par 7 En Jozef bracht zijn vader Jakob mede, en stelde hem voor Farao's aangezicht; en Jakob zegende Farao.
\par 8 En Farao zeide tot Jakob: Hoe vele zijn de dagen der jaren uws levens!
\par 9 En Jakob zeide tot Farao: De dagen der jaren mijner vreemdelingschappen zijn honderd en dertig jaren; weinig en kwaad zijn de dagen der jaren mijns levens geweest, en hebben niet bereikt de dagen van de jaren des levens mijner vaderen, in de dagen hunner vreemdelingschappen.
\par 10 En Jakob zegende Farao, en ging uit van Farao's aangezicht.
\par 11 En Jozef bestelde voor Jakob en zijn broederen woningen, en hij gaf hun een bezitting in Egypteland, in het beste van het land, in het land Rameses, gelijk als Farao geboden had.
\par 12 En Jozef onderhield zijn vader, en zijn broeders, en het ganse huis zijns vaders, met brood, tot den mond der kinderkens toe.
\par 13 En er was geen brood in het ganse land; want de honger was zeer zwaar: zodat het land van Egypte en het land Kanaan raasden vanwege dien honger.
\par 14 Toen verzamelde Jozef al het geld, dat in Egypteland en in het land Kanaan gevonden werd, voor het koren, dat zij kochten; en Jozef bracht dat geld in Farao's huis.
\par 15 Als nu het geld uit Egypteland en uit het land Kanaan verdaan was, kwamen al de Egyptenaars tot Jozef, zeggende: Geef ons brood; want waarom zouden wij in uw tegenwoordigheid sterven? want het geld ontbreekt;
\par 16 En Jozef zeide: Geeft uw vee, zo zal ik het u geven voor uw vee, indien het geld ontbreekt.
\par 17 Toen brachten zij hun vee tot Jozef; en Jozef gaf hun brood voor paarden en voor het vee der schapen, en voor het vee der runderen, en voor ezels; en hij voedde hen met brood, datzelve jaar, voor al hun vee.
\par 18 Toen datzelve jaar voleind was, zo kwamen zij tot hem in het tweede jaar, en zeiden tot hem: Wij zullen het voor mijn heer niet verbergen, alzo het geld verdaan is, en de bezitting der beesten gekomen aan mijn heer, zo is er niets anders overgebleven voor het aangezichts mijns heren, dan ons lichaam en ons land.
\par 19 Waarom zullen wij voor uw ogen sterven, zo wij als ons land? Koop ons en ons land voor brood; zo zullen wij en ons land Farao dienstbaar zijn; en geef zaad, opdat wij leven en niet sterven, en het land niet woest worde!
\par 20 Alzo kocht Jozef het gehele land van Egypte voor Farao; want de Egyptenaars verkochten een ieder zijn akker, dewijl de honger sterk over hen geworden was; zo werd het land Farao's eigen.
\par 21 En aangaande het volk, dat zette hij over in de steden, van het ene uiterste der palen van Egypte, tot het andere uiterste deszelven.
\par 22 Alleen het land der priesteren kocht hij niet, want de priesters hadden een bescheiden deel van Farao, en zij aten hun bescheiden deel, hetwelk hun Farao gegeven had; daarom verkochten zij hun land niet.
\par 23 Toen zeide Jozef tot het volk: Ziet, ik heb heden u en uw land gekocht voor Farao; ziet, daar is zaad voor u, opdat gij het land bezaait.
\par 24 Doch met de inkomsten zal het geschieden, dat gij aan Farao het vijfde deel zult geven, en de vier delen zullen voor u zijn, tot zaad des velds , en tot uw spijze en van degenen, die in uw huizen zijn, en om te eten voor uw kinderkens.
\par 25 En zij zeiden: Gij hebt ons leven behouden; laat ons genade vinden in de ogen mijns heren, en wij zullen Farao's knechten zijn.
\par 26 Jozef dan stelde ditzelve in tot een wet, tot dezen dag, over het land van Egypte, dat Farao het vijfde deel zou hebben; behalve dat alleen het land der priesteren van Farao niet werd.
\par 27 Zo woonde Israel in het land van Egypte, in het land Gosen; en zij stelden zich tot bezitters daarin, en zij werden vruchtbaar en vermeerderden zeer.
\par 28 En Jakob leefde in het land van Egypte zeventien jaar; zodat de dagen van Jakob, de jaren zijns levens, geweest zijn honderd zeven en veertig jaren.
\par 29 Als nu de dagen van Israel naderden, dat hij sterven zou, zo riep hij zijn zoon Jozef, en zeide tot hem: Indien ik nu genade gevonden heb in uw ogen, zo leg toch uw hand onder mijn heup, en doe weldadigheid en trouw aan mij, en begraaf mij toch niet in Egypte;
\par 30 Maar dat ik bij mijn vaderen ligge; hierom zult gij mij uit Egypte voeren, en mij in hun graf begraven. En hij zeide: Ik zal doen naar uw woord!
\par 31 En hij zeide: Zweer mij! en hij zwoer hem. En Israel boog zich ten hoofde van het bed.

\chapter{48}

\par 1 Het geschiedde nu na deze dingen, dat men Jozef zeide: Zie, uw vader is krank! Toen nam hij zijn twee zonen met zich, Manasse en Efraim!
\par 2 En men boodschapte Jakob, en men zeide: Zie, uw zoon Jozef komt tot u! Zo versterkte zich Israel, en zat op het bed.
\par 3 Daarna zeide Jakob tot Jozef: God de Almachtige, is mij verschenen te Luz, in het land Kanaan, en Hij heeft mij gezegend;
\par 4 En Hij heeft tot mij gezegd: Zie, Ik zal u vruchtbaar maken, en u vermenigvuldigen, en u tot een hoop van volken stellen; en Ik zal aan uw zaad na u dit land tot een eeuwige bezitting geven.
\par 5 Nu dan, uw twee zonen, die u in Egypteland geboren waren, eer ik in Egypte tot u gekomen ben, zijn mijne; Efraim en Manasse zullen mijne zijn, als Ruben en Simeon.
\par 6 Maar uw geslacht, dat gij na hen zult gewinnen, zullen uwe zijn; zij zullen naar hunner broederen naam genoemd worden in hun erfdeel.
\par 7 Toen ik nu van Paddan kwam, zo is Rachel bij mij gestorven in het land Kanaan, op den weg, als het nog een kleine streek lands was, om tot Efrath te komen; en ik begroef haar aldaar aan den weg van Efrath, welke is Bethlehem.
\par 8 En Israel zag de zonen van Jozef, en zeide: Wiens zijn deze?
\par 9 En Jozef zeide tot zijn vader: Zij zijn mijn zonen, die mij God hier gegeven heeft. En hij zeide: Breng hen toch tot mij, dat ik hen zegene!
\par 10 Doch de ogen van Israel waren zwaar van ouderdom; hij kon niet zien; en hij deed hen naderen tot zich; toen kuste hij hen, en omhelsde hen.
\par 11 En Israel zeide tot Jozef: Ik had niet gemeend uw aangezicht te zien; maar zie, God heeft mij ook uw zaad doen zien!
\par 12 Toen deed hen Jozef uitgaan van zijn knieen; en hij boog zich voor zijn aangezicht neder ter aarde.
\par 13 En Jozef nam die beiden, Efraim met zijn rechterhand, tegenover Israels linkerhand, en Manasse met zijn linkerhand, tegenover Israels rechterhand, en hij deed hen naderen tot hem.
\par 14 Maar Israel strekte zijn rechterhand uit, en leide die op het hoofd van Efraim, hoewel hij de minste was, en zijn linkerhand op het hoofd van Manasse; hij bestierde zijn handen verstandelijk; want Manasse was de eerstgeborene.
\par 15 En hij zegende Jozef, en zeide: De God, voor Wiens aangezicht mijn vaders, Abraham en Izak, gewandeld hebben, die God, Die mij gevoed heeft, van dat ik was, tot op dezen dag;
\par 16 Die Engel, Die mij verlost heeft van alle kwaad, zegene deze jongeren, en dat in hen mijn naam genoemd worde, en de naam mijner vaderen, Abraham en Izak, en dat zij vermenigvuldigen als vissen in menigte, in het midden des lands!
\par 17 Toen Jozef zag, dat zijn vader zijn rechterhand op het hoofd van Efraim leide, zo was het kwaad in zijn ogen, en hij ondervatte zijns vaders hand, om die van het hoofd van Efraim op het hoofd van Manasse af te brengen.
\par 18 En Jozef zeide tot zijn vader: Niet alzo, mijn vader! want deze is de eerstgeborene; leg uw rechterhand op zijn hoofd.
\par 19 Maar zijn vader weigerde het, en zeide: Ik weet het, mijn zoon! ik weet het; hij zal ook tot een volk worden, en hij zal ook groot worden; maar nochtans zal zijn kleinste broeder groter worden dan hij, en zijn zaad zal een volle menigte van volkeren worden.
\par 20 Alzo zegende hij ze te dien dage, zeggende: In u zal Israel zegenen, zeggende: God zette u als Efraim en als Manasse! En hij zette Efraim voor Manasse.
\par 21 Daarna zeide Israel tot Jozef: Zie, ik sterf; maar God zal met ulieden wezen, en Hij zal u wederbrengen in het land uwer vaderen.
\par 22 En ik heb u een stuk lands gegeven boven uw broederen; hetwelk ik, met mijn zwaard en met mijn boog, uit de hand der Amorieten genomen heb.

\chapter{49}

\par 1 Daarna riep Jakob zijn zonen, en hij zeide: Verzamelt u, en ik zal u verkondigen, hetgeen u in de navolgende dagen wedervaren zal.
\par 2 Komt samen en hoort, gij, zonen van Jakob! en hoort naar Israel, uw vader.
\par 3 Ruben! gij zijt mijn eerstgeborene, mijn kracht, en het begin mijner macht; de voortreffelijkste in hoogheid, en de voortreffelijkste in sterkte!
\par 4 Snelle afloop als der wateren, gij zult de voortreffelijkste niet zijn! want gij hebt uws vaders leger beklommen; toen hebt gij het geschonden; hij heeft mijn bed beklommen!
\par 5 Simeon en Levi zijn gebroeders! hun handelingen zijn werktuigen van geweld!
\par 6 Mijn ziel kome niet in hun verborgen raad; mijn eer worde niet verenigd met hun vergadering! want in hun toorn hebben zij de mannen doodgeslagen, en in hun moedwil hebben zij de ossen weggerukt.
\par 7 Vervloekt zij hun toorn, want hij is heftig; en hun verbolgenheid, want zij is hard! ik zal hen verdelen onder Jakob, en zal hen verstrooien onder Israel.
\par 8 Juda! gij zijt het, u zullen uw broeders loven; uw hand zal zijn op den nek uwer vijanden; voor u zullen zich uws vaders zonen nederbuigen.
\par 9 Juda is een leeuwenwelp! gij zijt van den roof opgeklommen, mijn zoon! Hij kromt zich, hij legt zich neder als een leeuw, en als een oude leeuw; wie zal hem doen opstaan?
\par 10 De schepter zal van Juda niet wijken, noch de wetgever van tussen zijn voeten, totdat Silo komt, en Denzelven zullen de volken gehoorzaam zijn.
\par 11 Hij bindt zijn jongen ezel aan den wijnstok, en het veulen zijner ezelin aan den edelsten wijnstok; hij wast zijn kleed in den wijn, en zijn mantel in wijndruivenbloed.
\par 12 Hij is roodachtig van ogen door den wijn, en wit van tanden door de melk.
\par 13 Zebulon zal aan de haven der zeeen wonen, en hij zal aan de haven der schepen wezen; en zijn zijde zal zijn naar Sidon.
\par 14 Issaschar is een sterk gebeende ezel, nederliggende tussen twee pakken.
\par 15 Toen hij de rust zag, dat zij goed was, en het land, dat het lustig was, zo boog hij zijn schouder om te dragen, en was dienende onder cijns.
\par 16 Dan zal zijn volk richten, als een der stammen Israels.
\par 17 Dan zal een slang zijn aan den weg, een adderslang nevens het pad, bijtende des paards verzenen, dat zijn rijder achterover valle.
\par 18 Op uw zaligheid wacht ik, HEERE!
\par 19 Aangaande Gad, een bende zal hem aanvallen; maar hij zal haar aanvallen in het einde.
\par 20 Van Aser, zijn brood zal vet zijn; en hij zal koninklijke lekkernijen leveren.
\par 21 Nafthali is een losgelaten hinde; hij geeft schone woorden.
\par 22 Jozef is een vruchtbare tak, een vruchtbare tak aan een fontein; elk der takken loopt over den muur.
\par 23 De schutters hebben hem wel bitterheid aangedaan, en beschoten, en hem gehaat;
\par 24 Maar zijn boog is in stijvigheid gebleven, en de armen zijner handen zijn gesterkt geworden, door de handen van den Machtige Jakobs; daarvan is hij een herder, een steen Israels;
\par 25 Van uws vaders God, Die u zal helpen, en van den Almachtige, Die u zal zegenen, met zegeningen des hemels van boven, met zegeningen des afgronds, die daaronder ligt, met zegeningen der borsten en der baarmoeder!
\par 26 De zegeningen uws vaders gaan te boven de zegeningen mijner voorvaderen, tot aan het einde van de eeuwige heuvelen; die zullen zijn op het hoofd van Jozef, en op den hoofdschedel des afgezonderden zijner broederen!
\par 27 Benjamin zal als een wolf verscheuren; des morgens zal hij roof eten, en des avonds zal hij buit uitdelen.
\par 28 Al deze stammen van Israel zijn twaalf; en dit is het, wat hun vader tot hen sprak, als hij hen zegende; hij zegende hen, een iegelijk naar zijn bijzonderen zegen.
\par 29 Daarna gebood hij hun, en zeide tot hen: Ik word verzameld tot mijn volk: begraaft mij bij mijn vaders, in de spelonk, die is in den akker van Efron, den Hethiet;
\par 30 In de spelonk, welke is op den akker van Machpela, die tegenover Mamre is, in het land Kanaan, die Abraham met dien akker gekocht heeft van Efron, den Hethiet, tot een erfbegrafenis.
\par 31 Aldaar hebben zij Abraham begraven, en Sara, zijn huisvrouw; daar hebben zij Izak begraven, en Rebekka, zijn huisvrouw; en daar heb ik Lea begraven.
\par 32 De akker, en de spelonk, die daarin is, is gekocht van de zonen Heths.
\par 33 Als Jakob voleind had aan zijn zonen bevelen te geven, zo leide hij zijn voeten samen op het bed, en hij gaf den geest, en hij werd verzameld tot zijn volken.

\chapter{50}

\par 1 Toen viel Jozef op zijns vaders aangezicht, en hij weende over hem, en kuste hem.
\par 2 En Jozef gebood zijn knechten, den medicijnmeesters, dat zij zijn vader balsemen zouden; en de medicijnmeesters balsemden Israel.
\par 3 En veertig dagen werden aan hem vervuld; want alzo werden vervuld de dagen dergenen, die gebalsemd werden; en de Egyptenaars beweenden hem zeventig dagen.
\par 4 Als nu de dagen zijns bewenens over waren, zo sprak Jozef tot het huis van Farao, zeggende: Indien ik nu genade gevonden heb in uw ogen, spreekt toch voor de oren van Farao, zeggende:
\par 5 Mijn vader heeft mij doen zweren, zeggende: Zie, ik sterf; in mijn graf, dat ik mij in het land Kanaan gegraven heb, daar zult gij mij begraven! Nu dan, laat mij toch optrekken, dat ik mijn vader begrave, dan zal ik wederkomen.
\par 6 En Farao zeide: Trek op en begraaf uw vader, gelijk als hij u heeft doen zweren.
\par 7 En Jozef toog op, om zijn vader te begraven; en met hem togen op alle Farao's knechten, de oudsten van zijn huis, en al de oudsten des lands van Egypte;
\par 8 Daartoe het ganse huis van Jozef, en zijn broeders, en het huis zijns vaders; alleen hun kleine kinderen, en hun schapen, en hun runderen lieten zij in het land Gosen.
\par 9 En met hem togen op, zo wagenen als ruiteren; en het was een zeer zwaar heir.
\par 10 Toen zij nu aan het plein van het doornbos kwamen, dat aan gene zijde van de Jordaan is, hielden zij daar een grote en zeer zware rouwklage; en hij maakte zijn vader een rouw van zeven dagen.
\par 11 Als de inwoners des lands, de Kanaanieten, dien rouw zagen op het plein van het doornbos, zo zeiden zij: Dit is een zware rouw der Egyptenaren; daarom noemde men haar naam Abel-mizraim, die aan het veer van de Jordaan is.
\par 12 En zijn zonen deden hem, gelijk als hij hun geboden had;
\par 13 Want zijn zonen voerden hem in het land Kanaan, en begroeven hem in de spelonk des akkers van Machpela, welke Abraham met den akker gekocht had tot een erfbegrafenis van Efron, den Hethiet, tegenover Mamre.
\par 14 Daarna keerde Jozef weder in Egypte, hij en zijn broeders, en allen, die met hem opgetogen waren, om zijn vader te begraven, nadat hij zijn vader begraven had.
\par 15 Toen Jozefs broeders zagen, dat hun vader dood was, zo zeiden zij: Misschien zal ons Jozef haten, en hij zal ons gewisselijk vergelden al het kwaad, dat wij hem aangedaan hebben.
\par 16 Daarom ontboden zij aan Jozef, zeggende: Uw vader heeft bevolen voor zijn dood, zeggende:
\par 17 Zo zult gij tot Jozef zeggen: Ei, vergeef toch de overtreding uwer broederen, en hun zonde; want zij hebben u kwaad aangedaan; maar nu vergeef toch de overtreding der dienaren van den God uws vaders! En Jozef weende, als zij tot hem spraken.
\par 18 Daarna kwamen ook zijn broeders, en vielen voor hem neder, en zeiden: Zie, wij zijn u tot knechten!
\par 19 En Jozef zeide tot hen: Vreest niet; want ben ik in de plaats van God?
\par 20 Gijlieden wel, gij hebt kwaad tegen mij gedacht; doch God heeft dat ten goede gedacht; opdat Hij deed, gelijk het te dezen dage is, om een groot volk in het leven te behouden.
\par 21 Nu dan, vreest niet! Ik zal u en uw kleine kinderen onderhouden. Zo troostte hij hen, en sprak naar hun hart.
\par 22 Jozef dan woonde in Egypte, hij en het huis zijns vaders; en Jozef leefde honderd en tien jaren.
\par 23 En Jozef zag van Efraim kinderen, van het derde gelid; ook werden de zonen van Machir, den zoon van Manasse, op Jozefs knieen geboren.
\par 24 En Jozef zeide tot zijn broederen: Ik sterf; maar God zal u gewisselijk bezoeken, en Hij zal u doen optrekken uit dit land, in het land, hetwelk hij aan Abraham, Izak en Jakob gezworen heeft.
\par 25 En Jozef deed de zonen van Israel zweren, zeggende: God zal u gewisselijk bezoeken, zo zult gij mijn beenderen van hier opvoeren!
\par 26 En Jozef stierf, honderd en tien jaren oud zijnde; en zij balsemden hem, en men leide hem in een kist in Egypte.




\end{document}