\begin{document}

\title{Nehemiah}


Neh 1:1  De geschiedenissen van Nehemia, zoon van Hachalja. En het geschiedde in de maand Chisleu, in het twintigste jaar, als ik te Susan in het paleis was;
Neh 1:2  Zo kwam Hanani, een van mijn broederen, hij en sommige mannen uit Juda, en ik vraagde hen naar de Joden, die ontkomen waren (die overgebleven waren van de gevangenis), en naar Jeruzalem.
Neh 1:3  En zij zeiden tot mij: De overgeblevenen, die van de gevangenis aldaar in het landschap zijn overgebleven, zijn in grote ellende en in versmaadheid; en Jeruzalems muur is verscheurd, en haar poorten zijn met vuur verbrand.
Neh 1:4  En het geschiedde, als ik deze woorden hoorde, zo zat ik neder, en weende, en bedreef rouw, enige dagen; en ik was vastende en biddende voor het aangezicht van den God des hemels.
Neh 1:5  En ik zeide: Och, HEERE, God des hemels, Gij, grote en vreselijke God! Die het verbond en de goedertierenheid houdt dien, die Hem liefhebben, en Zijn geboden houden.
Neh 1:6  Laat toch Uw oor opmerkende, en Uw ogen open zijn, om te horen naar het gebed Uws knechts, dat ik heden voor Uw aangezicht bid, dag en nacht, voor de kinderen Israels, Uw knechten; en ik doe belijdenis over de zonden der kinderen Israels, die wij tegen U gezondigd hebben; ook ik en mijns vaders huis, wij hebben gezondigd.
Neh 1:7  Wij hebben het ganselijk tegen U verdorven; en wij hebben niet gehouden de geboden, noch de inzettingen, noch de rechten, die Gij Uw knecht Mozes geboden hebt.
Neh 1:8  Gedenk toch des woords, dat Gij Uw knecht Mozes geboden hebt, zeggende: Gijlieden zult overtreden, Ik zal u onder de volken verstrooien.
Neh 1:9  En gij zult u tot Mij bekeren, en Mijn geboden houden, en die doen; al waren uw verdrevenen aan het einde des hemels, Ik zal hen vandaar verzamelen, en zal ze brengen tot de plaats, die Ik verkoren heb, om Mijn Naam aldaar te doen wonen.
Neh 1:10  Zij zijn toch Uw knechten en Uw volk, dat Gij verlost hebt door Uw grote kracht en door Uw sterke hand.
Neh 1:11  Och, HEERE, laat toch Uw oor opmerkende zijn op het gebed Uws knechts, en op het gebed Uwer knechten, die lust hebben Uw Naam te vrezen; en doe het toch Uw knecht heden wel gelukken, en geef hem barmhartigheid voor het aangezicht dezes mans. Ik nu was des konings schenker.
Neh 2:1  Toen geschiedde het in de maand Nisan, in het twintigste jaar van den koning Arthahsasta, als er wijn voor zijn aangezicht was, dat ik den wijn opnam, en gaf hem den koning; nu was ik nooit treurig geweest voor zijn aangezicht.
Neh 2:2  Zo zeide de koning tot mij: Waarom is uw aangezicht treurig, zo gij toch niet krank zijt? Dit is niet dan treurigheid des harten. Toen vreesde ik gans zeer.
Neh 2:3  En ik zeide tot de koning: De koning leve in eeuwigheid! Hoe zou mijn aangezicht niet treurig zijn, daar de stad, de plaats der begrafenissen mijner vaderen, woest is, en haar poorten met vuur verteerd zijn?
Neh 2:4  En de koning zeide tot mij: Wat verzoekt gij nu? Toen bad ik tot God van den hemel.
Neh 2:5  En ik zeide tot den koning: Zo het den koning goeddunkt, en zo uw knecht voor uw aangezicht aangenaam is, dat gij mij zendt naar Juda, naar de stad der begrafenissen mijner vaderen, dat ik ze bouwe.
Neh 2:6  Toen zeide de koning tot mij, daar de koningin nevens hem zat: Hoe lang zal uw reis wezen, en wanneer zult gij wederkomen? En het behaagde den koning, dat hij mij zond, als ik hem zekeren tijd gesteld had.
Neh 2:7  Voorts zeide ik tot den koning: Zo het den koning goeddunkt, dat men mij brieven geve aan de landvoogden aan gene zijde der rivier, dat zij mij overgeleiden, totdat ik in Juda zal gekomen zijn;
Neh 2:8  Ook een brief aan Asaf, den bewaarder van den lusthof, denwelken de koning heeft, dat hij mij hout geve om te zolderen de poorten van het paleis, dat aan het huis is, en tot den stadsmuur, en tot het huis, waar ik intrekken zal. En de koning gaf ze mij, naar de goede hand mijns Gods over mij.
Neh 2:9  Toen kwam ik tot de landvoogden aan gene zijde der rivier, en gaf hun de brieven des konings. En de koning had oversten des heirs en ruiteren met mij gezonden.
Neh 2:10  Toen nu Sanballat, de Horoniet, en Tobia, de Ammonietische knecht dat hoorden, mishaagde het hun met groot mishagen, dat er een mens gekomen was, om wat goeds te zoeken voor de kinderen Israels.
Neh 2:11  En ik kwam te Jeruzalem, en was daar drie dagen.
Neh 2:12  Daarna maakte ik mij des nachts op, ik en weinig mannen met mij, en ik gaf geen mens te kennen, wat mijn God in mijn hart gegeven had, om aan Jeruzalem te doen; en er was geen dier met mij, dan het dier, waarop ik reed.
Neh 2:13  En ik trok uit bij nacht door de Dalpoort, en voorbij de Drakenfontein, en naar de Mistpoort, en ik brak aan de muren van Jeruzalem, dewelke verscheurd waren, en haar poorten met vuur verteerd.
Neh 2:14  En ik ging voort naar de Fonteinpoort, en naar des konings vijver; doch daar was geen plaats voor het dier, om onder mij voort te gaan.
Neh 2:15  Toen ging ik op, des nachts, door de beek, en ik brak aan den muur; en ik keerde weder, en kwam in door de Dalpoort; alzo keerde ik wederom.
Neh 2:16  En de overheden wisten niet, waar ik heengegaan was, en wat ik deed; want ik had tot nog toe den Joden, en den priesteren, en den edelen, en overheden, en den anderen, die het werk deden, niets te kennen gegeven.
Neh 2:17  Toen zeide ik tot hen: Gijlieden ziet de ellende, waarin wij zijn, dat Jeruzalem woest is, en haar poorten met vuur verbrand zijn; komt, en laat ons Jeruzalems muur opbouwen; opdat wij niet meer een versmaadheid zijn.
Neh 2:18  En ik gaf hun te kennen de hand mijns Gods, Die goed over mij geweest was, als ook de woorden des konings, die hij tot mij gesproken had. Toen zeiden zij: Laat ons op zijn, dat wij bouwen; en zij sterkten hun handen ten goede.
Neh 2:19  Als nu Sanballat, de Horoniet, en Tobia, de Ammonietische knecht, en Gesem, de Arabier, dit hoorden, zo bespotten zij ons, en verachtten ons; en zij zeiden: Wat is dit voor een ding, dat gijlieden doet? Wilt gijlieden tegen den koning rebelleren?
Neh 2:20  Toen gaf ik hun tot antwoord, en zeide tot hen: God van den hemel, Die zal het ons doen gelukken, en wij, Zijn knechten, zullen ons opmaken en bouwen; maar gijlieden hebt geen deel, noch gerechtigheid, noch gedachtenis in Jeruzalem.
Neh 3:1  En Eljasib, de hogepriester, maakte zich op met zijn broederen, de priesteren, en zij bouwden de Schaapspoort; zij heiligden ze, en richtten haar deuren op; ja, zij heiligden ze tot aan den toren Mea, tot aan den toren Hananeel.
Neh 3:2  En aan zijn hand bouwden de mannen van Jericho; ook bouwde aan zijn hand Zacchur, de zoon van Imri.
Neh 3:3  De Vispoort nu bouwden de kinderen van Senaa; zij zolderden die, en richtten haar deuren op, met haar sloten en haar grendelen.
Neh 3:4  En aan hun hand verbeterde Meremoth, de zoon van Uria, den zoon van Koz; en aan hun hand verbeterde Mesullam, de zoon van Berechja, den zoon van Mesezabeel; en aan hun hand verbeterde Zadok, zoon van Baena.
Neh 3:5  Voorts aan hun hand verbeterden de Thekoieten; maar hun voortreffelijken brachten hun hals niet tot den dienst huns Heeren.
Neh 3:6  En de Oude poort verbeterden Jojada, de zoon van Paseah, en Mesullam, de zoon van Besodja; deze zolderden zij, en richtten haar deuren op, met haar sloten en haar grendelen.
Neh 3:7  En aan hun hand verbeterden Melatja, de Gibeoniet, en Jadon, de Meronothiet, de mannen van Gibeon en van Mizpa; tot aan den stoel des landvoogds aan deze zijde der rivier.
Neh 3:8  Aan zijn hand verbeterde Uzziel, de zoon van Harhoja, een der goudsmeden, en aan zijn hand verbeterde Hananja, de zoon van een der apothekers; en zij lieten Jeruzalem tot aan den breden muur.
Neh 3:9  En aan hun hand verbeterde Refaja, de zoon van Hur, overste des halven deels van Jeruzalem.
Neh 3:10  Voorts aan hun hand verbeterde Jedaja, de zoon van Harumaf, en tegenover zijn huis; en aan zijn hand verbeterde Hattus, de zoon van Hasabneja.
Neh 3:11  De andere mate verbeterden Malchia, de zoon van Harim, en Hassub, de zoon van Pahath-moab; daartoe den Bakoventoren.
Neh 3:12  En aan zijn hand verbeterde Sallum, de zoon van Lohes, overste van het andere halve deel van Jeruzalem, hij en zijn dochteren.
Neh 3:13  De Dalpoort verbeterden Hanun, en de inwoners van Zanoah; zij bouwden die, en richtten haar deuren op, met haar sloten en haar grendelen; daartoe duizend ellen aan den muur, tot aan de Mistpoort.
Neh 3:14  De Mistpoort nu verbeterde Malchia, de zoon van Rechab, overste van het deel Beth-cherem; hij bouwde ze, en richtte haar deuren op, met haar sloten en haar grendelen.
Neh 3:15  En de Fonteinpoort verbeterde Sallum, de zoon van Kol-hoze, overste van het deel van Mizpa; hij bouwde ze, en overdekte ze, en richtte haar deuren op, met haar sloten en haar grendelen; daartoe den muur des vijvers Schelah bij des konings hof, en tot aan de trappen, die afgaan van Davids stad.
Neh 3:16  Na hem verbeterde Nehemia, de zoon van Azbuk, overste van het halve deel van Beth-zur, tot tegenover Davids graven, en tot aan den gemaakten vijver, en tot aan het huis der helden.
Neh 3:17  Na hem verbeterden de Levieten, Rehum, de zoon van Bani; aan zijn hand verbeterde Hasabja, de overste van het halve deel van Kehila, in zijn deel.
Neh 3:18  Na hem verbeterden hun broederen, Bavai, de zoon van Henadad, de overste van het andere halve deel van Kehila.
Neh 3:19  Aan zijn hand verbeterde Ezer, de zoon van Jesua, de overste van Mizpa, een andere maat; tegenover den opgang naar het wapenhuis, aan den hoek.
Neh 3:20  Na hem verbeterde zeer vuriglijk Baruch, de zoon van Zabbai, een andere maat; van den hoek tot aan de deur van het huis van Eljasib, den hogepriester.
Neh 3:21  Na hem verbeterde Meremoth, de zoon van Uria, den zoon van Koz, een andere maat; van de huisdeur van Eljasib af, tot aan het einde van Eljasibs huis.
Neh 3:22  En na hem verbeterden de priesteren, wonende in de vlakke velden.
Neh 3:23  Daarna verbeterden Benjamin, en Hassub, tegenover hun huis; na hem verbeterde Azaria, de zoon van Maaseja, den zoon van Hananja, bij zijn huis.
Neh 3:24  Na hem verbeterde Binnui, de zoon van Henadad, een andere maat; van het huis van Azaria tot aan den hoek en tot aan het punt;
Neh 3:25  Palal, de zoon van Uzai, tegen den hoek, en den hogen toren over, die van des konings huis uitsteekt, die bij den voorhof der gevangenis is; na hem Pedaja, de zoon van Paros;
Neh 3:26  De Nethinim nu, die in Ofel woonden, tot tegenover de Waterpoort aan het oosten, en den uitstekenden toren.
Neh 3:27  Daarna verbeterden de Thekoieten een andere maat; tegenover den groten uitstekenden toren, en tot aan den muur van Ofel.
Neh 3:28  Van boven de Paardenpoort verbeterden de priesteren, een iegelijk tegenover zijn huis.
Neh 3:29  Daarna verbeterde Zadok, de zoon van Immer, tegenover zijn huis. En na hem verbeterde Semaja, de zoon van Sechanja, de bewaarder van de Oostpoort.
Neh 3:30  Na hem verbeterden Hananja, de zoon van Selemja, en Hanun, de zoon van Zalaf, de zesde, een andere maat. Na hem verbeterde Mesullam, de zoon van Berechja, tegenover zijn kamer.
Neh 3:31  Na hem verbeterde Malchia, de zoon eens goudsmids, tot aan het huis der Nethinim en der kruideniers, tegenover de poort van Mifkad, en tot de opperzaal van het punt.
Neh 3:32  En tussen de opperzaal van het punt tot de Schaapspoort toe, verbeterden de goudsmeden en de kruideniers.
Neh 4:1  Maar het geschiedde, als Sanballat gehoord had, dat wij den muur bouwden, zo ontstak hij, en werd zeer toornig; en hij bespotte de Joden.
Neh 4:2  En sprak in de tegenwoordigheid zijner broederen en van het heir van Samaria, en zeide: Wat doen deze amechtige Joden? Zal men hen laten geworden? Zullen zij offeren? Zullen zij het in een dag voleinden? Zullen zij de steentjes uit de stofhopen levend maken, daar zij verbrand zijn?
Neh 4:3  En Tobia, de Ammoniet, was bij hem, en zeide: Al is het, dat zij bouwen, zo er een vos opkwame, hij zou hun stenen muur wel verscheuren.
Neh 4:4  Hoor, o onze God! dat wij zeer veracht zijn, en keer hun versmaadheid weder op hun hoofd, en geef hen over tot een roof in een land der gevangenis.
Neh 4:5  En dek hun ongerechtigheid niet toe; en hun zonde worde niet uitgedelgd van voor Uw aangezicht, want zij hebben U getergd, staande tegenover de bouwlieden.
Neh 4:6  Doch wij bouwden den muur, zodat de ganse muur samengevoegd werd tot zijn helft toe; want het hart des volks was om te werken.
Neh 4:7  En het geschiedde, als Sanballat, en Tobia, en de Arabieren, en de Ammonieten, en de Asdodieten hoorden, dat de verbetering aan de muren van Jeruzalem toenam, dat de scheuren begonnen gestopt te worden, zo ontstaken zij zeer;
Neh 4:8  En zij maakten allen te zamen een verbintenis, dat zij zouden komen om tegen Jeruzalem te strijden, en een verbijstering daarin te maken.
Neh 4:9  Maar wij baden tot onzen God, en zetten wacht tegen hen, dag en nacht, hunnenthalve.
Neh 4:10  Toen zeide Juda: De kracht der dragers is vervallen, en des stofs is veel, zodat wij aan den muur niet zullen kunnen bouwen.
Neh 4:11  Nu hadden onze vijanden gezegd: Zij zullen het niet weten, noch zien, totdat wij in het midden van hen komen, en slaan hen dood; alzo zullen wij het werk doen ophouden.
Neh 4:12  En het geschiedde, als de Joden, die bij hen woonden, kwamen, dat zij het ons wel tienmaal zeiden, uit al de plaatsen, door dewelke gij tot ons wederkeert.
Neh 4:13  Daarom zette ik in de benedenste plaatsen achter den muur, en op de hoogten, en ik zette het volk naar de geslachten, met hun zwaarden, hun spiesen en hun bogen.
Neh 4:14  En ik zag toe, en maakte mij op, en zeide tot de edelen, en tot de overheden, en tot het overige des volks: Vreest niet voor hun aangezicht; denkt aan dien groten en vreselijken HEERE, en strijdt voor uw broederen, uw zonen en uw dochteren, uw vrouwen en uw huizen.
Neh 4:15  Daarna geschiedde het, als onze vijanden hoorden, dat het ons bekend was geworden, en God hun raad te niet gemaakt had, zo keerden wij allen weder tot den muur, een iegelijk tot zijn werk.
Neh 4:16  En het geschiedde van dien dag af, dat de helft mijner jongens doende waren aan het werk, en de helft van hen hielden de spiesen, en de schilden, en de bogen, en de pantsiers; en de oversten waren achter het ganse huis van Juda.
Neh 4:17  Die aan den muur bouwden, en die den last droegen, en die oplaadden, waren een ieder met zijn ene hand doende aan het werk, en de andere hield het geweer.
Neh 4:18  En de bouwers hadden een iegelijk zijn zwaard aan zijn lenden gegord, en bouwden; maar die met de bazuin blies, was bij mij.
Neh 4:19  En ik zeide tot de edelen, en tot de overheden, en tot het overige des volks: Het werk is groot en wijd; en wij zijn op den muur afgezonderd, de een ver van den ander;
Neh 4:20  Ter plaatse, waar gij het geluid der bazuin zult horen, daarheen zult gij u tot ons verzamelen; onze God zal voor ons strijden.
Neh 4:21  Alzo waren wij doende aan het werk; en de helft van hen hielden de spiesen, van het opgaan des dageraads tot het voortkomen der sterren toe.
Neh 4:22  Ook zeide ik te dier tijd tot het volk: Een iegelijk vernachte met zijn jongen binnen Jeruzalem, opdat zij ons des nachts ter wacht zijn, en des daags aan het werk.
Neh 4:23  Voorts noch ik, noch mijn broederen, noch mijn jongelingen, noch de mannen van de wacht, die achter mij waren, wij trokken onze klederen niet uit; een iegelijk had zijn geweer en water.
Neh 5:1  Maar het geroep des volks en hunner vrouwen was groot, tegen hun broederen, de Joden.
Neh 5:2  Want er waren, die zeiden: Onze zonen, en onze dochteren, wij zijn velen; daarom hebben wij koren opgenomen, opdat wij eten en leven.
Neh 5:3  Ook waren er, die zeiden: Wij verpanden onze akkers, en onze wijngaarden, en onze huizen, opdat wij in dezen honger koren mogen opnemen.
Neh 5:4  Desgelijks waren er, die zeiden: Wij hebben geld ontleend tot des konings cijns, op onze akkers en onze wijngaarden.
Neh 5:5  Nu is toch ons vlees als het vlees onzer broederen, onze kinderen zijn als hun kinderen; en ziet, wij onderwerpen onze zonen en onze dochteren tot dienstknechten; ja, er zijn enige van onze dochteren onderworpen, dat zij in de macht onzer handen niet zijn; en anderen hebben onze akkers en onze wijngaarden.
Neh 5:6  Toen ik nu hun geroep en deze woorden hoorde, ontstak ik zeer.
Neh 5:7  En mijn hart beraadslaagde in mij; daarna twistte ik met de edelen, en met de overheden, en zeide tot hen: Gijlieden vordert een last, een iegelijk van zijn broeder. Voorts belegde ik een grote vergadering tegen hen.
Neh 5:8  En ik zeide tot hen: Wij hebben onze broederen, de Joden, die aan de heidenen verkocht waren, naar ons vermogen wedergekocht; en zoudt gijlieden ook uw broederen verkopen, of zouden zij aan ons verkocht worden? Toen zwegen zij, en vonden geen antwoord.
Neh 5:9  Voorts zeide ik: De zaak is niet goed, die gijlieden doet; zoudt gij niet wandelen in de vreze onzes Gods, om de versmading der heidenen, onze vijanden?
Neh 5:10  Ik, mijn broederen, en mijn jongens, vorderen wij ook geld en koren van hen? Laat ons toch dezen last nalaten.
Neh 5:11  Geeft hun toch als heden weder hun akkers, hun wijngaarden, hun olijfgaarden en hun huizen; en het honderdste deel van het geld, en van het koren, den most en de olie, die gij hun hebt afgevorderd.
Neh 5:12  Toen zeiden zij: Wij zullen het wedergeven, en van hen niets zoeken; wij zullen alzo doen, als gij zegt. En ik riep de priesteren, en deed hen zweren, dat zij doen zouden naar dit woord.
Neh 5:13  Ook schudde ik mijn boezem uit, en zeide: Alzo schudde God uit allen man, die dit woord niet zal bevestigen, uit zijn huis en uit zijn arbeid, en hij zij alzo uitgeschud en ledig. En de ganse gemeente zeide: Amen! En zij prezen de HEERE. En het volk deed naar dit woord.
Neh 5:14  Ook van dien dag af, dat hij mij bevolen heeft hun landvoogd te zijn in het land Juda, van het twintigste jaar af, tot het twee en dertigste jaar van den koning Arthahsasta, zijnde twaalf jaren, heb ik, met mijn broederen, het des landvoogds niet gegeten.
Neh 5:15  En de vorige landvoogden, die voor mij geweest zijn, hebben het volk bezwaard, en van hen genomen aan brood en wijn, daarna veertig zilveren sikkelen; ook heersten hun jongens over het volk; maar ik heb alzo niet gedaan, om der vreze Gods wil.
Neh 5:16  Daartoe heb ik ook aan het werk dezes muurs verbeterd, en wij hebben geen land gekocht; en al mijn jongens zijn aldaar verzameld geweest tot het werk.
Neh 5:17  Ook zijn van de Joden en van de overheden honderd en vijftig man, en die van de heidenen, die rondom ons zijn, tot ons kwamen, aan mijn tafel geweest.
Neh 5:18  En wat voor een dag bereid werd, was een os en zes uitgelezen schapen; ook werden mij vogelen bereid, en binnen tien dagen van allen wijn zeer veel; nog heb ik bij dezen het brood des landvoogds niet gezocht, omdat de dienstbaarheid zwaar was over dit volk.
Neh 5:19  Gedenk mijner, mijn God, ten goede, alles, wat ik aan dit volk gedaan heb.
Neh 6:1  Voorts is het geschied, als van Sanballat, en Tobia, en van Gesem, den Arabier, en van onze andere vijanden gehoord was, dat ik den muur gebouwd had, en dat geen scheur daarin was overgelaten; ook had ik tot dezen tijd toe de deuren niet opgezet in de poorten;
Neh 6:2  Zo zond Sanballat, en Gesem, tot mij, om te zeggen: Kom en laat ons te zamen vergaderen in de dorpen, in het dal Ono. Maar zij dachten mij kwaad te doen.
Neh 6:3  En ik zond boden tot hen, om te zeggen: Ik doe een groot werk, zodat ik niet zal kunnen afkomen; waarom zou dit werk ophouden, terwijl ik het zou nalaten, en tot ulieden afkomen?
Neh 6:4  Zij zonden nu wel viermaal tot mij, op dezelfde wijze. En ik antwoordde hun op dezelfde wijze.
Neh 6:5  Toen zond Sanballat tot mij op dezelfde wijze, ten vijfden male, zijn jongen, met een open brief in zijn hand.
Neh 6:6  Daarin was geschreven: Het is onder de volken gehoord, en Gasmu zegt: Gij en de Joden denkt te rebelleren, daarom bouwt gij den muur, en gij zult hun ten koning zijn; naar dat deze zaken zijn.
Neh 6:7  Dat gij ook profeten hebt besteld, om van u te Jeruzalem uit te roepen, zeggende: Hij is koning in Juda. Nu zal het van den koning gehoord worden, naar dat deze zaken zijn; kom dan nu, en laat ons te zamen raadslaan.
Neh 6:8  Doch ik zond tot hem, om te zeggen: Er is van al zulke zaken, als gij zegt, niets geschied; maar gij versiert ze uit uw hart.
Neh 6:9  Want zij allen zochten ons vreesachtig te maken, zeggende: Hun handen zullen van het werk aflaten, dat het niet zal gedaan worden; nu dan, sterk mijn handen!
Neh 6:10  Als ik nu kwam in het huis van Semaja, den zoon van Delaja, den zoon van Mehetabeel (hij nu was besloten), zo zeide hij: Laat ons samenkomen in het huis Gods, in het midden des tempels, en laat ons de deuren des tempels toesluiten; want zij zullen komen om u te doden, ja, bij nacht zullen zij komen, om u te doden.
Neh 6:11  Maar ik zeide: Zou een man, als ik, vlieden? En wie is er, zijnde als ik, die in den tempel zou gaan, dat hij levend bleve? Ik zal er niet ingaan.
Neh 6:12  Want ik merkte, en ziet, God had hem niet gezonden; maar hij sprak deze profetie tegen mij, omdat Tobia en Sanballat hem gehuurd hadden.
Neh 6:13  Daarom was hij gehuurd, opdat ik zou vrezen, en alzo doen, en zondigen; opdat zij iets zouden hebben tot een kwaden naam, opdat zij mij zouden honen.
Neh 6:14  Gedenk, mijn God, aan Tobia en aan Sanballat, naar deze zijn werken; en ook aan de profetes Noadja, en aan de andere profeten, die mij gezocht hebben vreesachtig te maken.
Neh 6:15  De muur nu werd volbracht, op den vijf en twintigsten van Elul, in twee en vijftig dagen.
Neh 6:16  En het geschiedde, als al onze vijanden dit hoorden, zo vreesden al de heidenen, die rondom ons waren, en zij vervielen zeer in hun ogen; want zij merkten, dat dit werk van onzen God gedaan was.
Neh 6:17  Ook schreven in die dagen edelen van Juda vele brieven, die naar Tobia gingen; en die van Tobia kwamen tot hen.
Neh 6:18  Want velen in Juda hadden hem gezworen, omdat hij was een schoonzoon van Sechanja, den zoon van Arah; en zijn zoon Johanan had genomen de dochter van Mesullam, den zoon van Berechja.
Neh 6:19  Ook verhaalden zij zijn goeddadigheden voor mijn aangezicht, en mijn woorden brachten zij uit tot hem. Tobia dan zond brieven, om mij vreesachtig te maken.
Neh 7:1  Voorts geschiedde het, als de muur gebouwd was, dat ik de deuren oprichtte, en de poortiers, en de zangers, en de Levieten werden besteld.
Neh 7:2  En ik gaf bevel aan mijn broeder Hanani, en aan Hananja, den overste van den burg te Jeruzalem, want hij was als een man van getrouwheid, en godvrezende boven velen.
Neh 7:3  En ik zeide tot hen: Laat de poorten van Jeruzalem niet geopend worden, totdat de zon heet wordt, en terwijl zij daarbij staan, laat hen de deuren sluiten, betast gij ze dan; en dat men wachten zette, inwoners van Jeruzalem, een iegelijk op zijn wacht, en een iegelijk tegenover zijn huis.
Neh 7:4  De stad nu was wijd van ruimte en groot; doch des volks was weinig daarbinnen; en de huizen waren niet gebouwd.
Neh 7:5  Zo gaf mijn God in mijn hart, dat ik de edelen, en de overheden, en het volk verzamelde, om de geslachten te rekenen; en ik vond het geslachtsregister dergenen, die in het eerst waren opgetogen, en vond daarin geschreven aldus:
Neh 7:6  Dit zijn de kinderen van dat landschap, die optogen uit de gevangenis der weggevoerden, die Nebukadnezar, koning van Babel, weggevoerd had, en die wedergekeerd zijn naar Jeruzalem en naar Juda, een iegelijk tot zijn stad;
Neh 7:7  Dewelke kwamen met Zerubbabel, Jesua, Nehemia, Azaria, Raamja, Nahamani, Mordechai, Bilsan, Mispereth, Bigvai, Nehum en Baena. Dit is het getal der mannen van het volk van Israel.
Neh 7:8  De kinderen van Parhos waren twee duizend, honderd twee en zeventig;
Neh 7:9  De kinderen van Sefatja, driehonderd twee en zeventig;
Neh 7:10  De kinderen van Arach, zeshonderd twee en vijftig;
Neh 7:11  De kinderen van Pahath-moab, van de kinderen van Jesua en Joab, twee duizend, achthonderd en achttien;
Neh 7:12  De kinderen van Elam, duizend, tweehonderd vier en vijftig;
Neh 7:13  De kinderen van Zatthu, achthonderd vijf en veertig;
Neh 7:14  De kinderen van Zakkai, zevenhonderd en zestig;
Neh 7:15  De kinderen van Binnui, zeshonderd acht en veertig;
Neh 7:16  De kinderen van Bebai, zeshonderd acht en twintig;
Neh 7:17  De kinderen van Azgad, twee duizend, driehonderd twee en twintig;
Neh 7:18  De kinderen van Adonikam, zeshonderd zeven en zestig;
Neh 7:19  De kinderen van Bigvai, twee duizend, zeven en zestig;
Neh 7:20  De kinderen van Adin, zeshonderd vijf en vijftig;
Neh 7:21  De kinderen van Ater, van Hizkia, acht en negentig;
Neh 7:22  De kinderen van Hassum, driehonderd acht en twintig;
Neh 7:23  De kinderen van Bezai, driehonderd vier en twintig;
Neh 7:24  De kinderen van Harif, honderd en twaalf;
Neh 7:25  De kinderen van Gibeon, vijf en negentig;
Neh 7:26  De mannen van Bethlehem en Netofa, honderd acht en tachtig;
Neh 7:27  De mannen van Anathoth, honderd acht en twintig;
Neh 7:28  De mannen van Beth-azmaveth, twee en veertig;
Neh 7:29  De mannen van Kirjath-jearim, Cefira en Beeroth, zevenhonderd drie en veertig;
Neh 7:30  De mannen van Rama en Gaba, zeshonderd en twintig;
Neh 7:31  De mannen van Michmas, honderd twee en twintig;
Neh 7:32  De mannen van Beth-el en Ai, honderd drie en twintig;
Neh 7:33  De mannen van het andere Nebo, twee en vijftig;
Neh 7:34  De kinderen des anderen Elams, duizend, tweehonderd vier en vijftig;
Neh 7:35  De kinderen van Harim, driehonderd en twintig;
Neh 7:36  De kinderen van Jericho, driehonderd vijf en veertig;
Neh 7:37  De kinderen van Lod, Hadid en Ono, zevenhonderd een en twintig;
Neh 7:38  De kinderen van Senaa, drie duizend, negenhonderd en dertig;
Neh 7:39  De priesters: de kinderen van Jedaja, van het huis van Jesua, negenhonderd drie en zeventig;
Neh 7:40  De kinderen van Immer, duizend twee en vijftig;
Neh 7:41  De kinderen van Pashur, duizend, tweehonderd zeven en veertig;
Neh 7:42  De kinderen van Harim, duizend en zeventien;
Neh 7:43  De Levieten: de kinderen van Jesua, van Kadmiel, van de kinderen van Hodeva, vier en zeventig;
Neh 7:44  De zangers: de kinderen van Asaf, honderd acht en veertig;
Neh 7:45  De poortiers: de kinderen van Sallum, de kinderen van Ater, de kinderen van Talmon, de kinderen van Akkub, de kinderen van Hatita, de kinderen van Sobai, honderd acht en dertig;
Neh 7:46  De Nethinim: de kinderen van Ziha, de kinderen van Hasufa, de kinderen van Tabbaoth;
Neh 7:47  De kinderen van Keros, de kinderen van Sia, de kinderen van Padon;
Neh 7:48  De kinderen van Lebana, de kinderen van Hagaba, de kinderen van Salmai;
Neh 7:49  De kinderen van Hanan, de kinderen van Giddel, de kinderen van Gahar;
Neh 7:50  De kinderen van Reaja, de kinderen van Rezin, de kinderen van Nekoda;
Neh 7:51  De kinderen van Gazzam, de kinderen van Uzza, de kinderen van Paseah;
Neh 7:52  De kinderen van Bezai, de kinderen van Meunim, de kinderen van Nefussim;
Neh 7:53  De kinderen van Bakbuk, de kinderen van Hakufa, de kinderen van Harhur;
Neh 7:54  De kinderen van Bazlith, de kinderen van Mehida, de kinderen van Harsa;
Neh 7:55  De kinderen van Barkos, de kinderen van Sisera, de kinderen van Thamah;
Neh 7:56  De kinderen van Neziah, de kinderen van Hatifa;
Neh 7:57  De kinderen der knechten van Salomo; de kinderen van Sotai, de kinderen van Sofereth, de kinderen van Perida;
Neh 7:58  De kinderen van Jaela, de kinderen van Darkon, de kinderen van Giddel;
Neh 7:59  De kinderen van Sefatja, de kinderen van Hattil, de kinderen van Pochereth van Zebaim, de kinderen van Amon;
Neh 7:60  Al de Nethinim, en de kinderen der knechten van Salomo, waren driehonderd twee en negentig.
Neh 7:61  Ook togen dezen op van Thel-melah, Thel-harsa, Cherub, Addon en Immer; maar zij konden hunner vaderen huis, en hun zaad niet tonen, of zij uit Israel waren;
Neh 7:62  De kinderen van Delaja, de kinderen van Tobia, de kinderen van Nekoda, zeshonderd twee en veertig.
Neh 7:63  En van de priesteren, de kinderen van Habaja, de kinderen van Koz, de kinderen van Barzillai, die een vrouw van de dochteren van Barzillai, den Gileadiet, genomen had, en naar hun naam genoemd was.
Neh 7:64  Dezen zochten hun geschrift, willende hun geslacht rekenen, maar het werd niet gevonden; daarom werden zij als onreinen van het priesterdom geweerd.
Neh 7:65  En Hattirsatha zeide tot hen, dat zij van de heiligste dingen niet zouden eten, totdat er een priester stond met urim en thummim.
Neh 7:66  Deze ganse gemeente te zamen was twee en veertig duizend, driehonderd en zestig;
Neh 7:67  Behalve hun knechten en hun maagden, die waren zeven duizend, driehonderd zeven en dertig; en zij hadden tweehonderd vijf en veertig zangers en zangeressen.
Neh 7:68  Hun paarden, zevenhonderd zes en dertig; hun muildieren, tweehonderd vijf en veertig;
Neh 7:69  Kemelen, vierhonderd vijf en dertig; ezelen, zes duizend, zevenhonderd en twintig.
Neh 7:70  Een deel nu van de hoofden der vaderen gaven tot het werk. Hattirsatha gaf tot den schat, aan goud, duizend drachmen, vijftig sprengbekkens, vijfhonderd en dertig priesterrokken.
Neh 7:71  En anderen van de hoofden der vaderen gaven tot den schat des werks, aan goud, twintig duizend drachmen, en aan zilver, twee duizend en tweehonderd ponden.
Neh 7:72  En wat de overigen des volks gaven, was aan goud, twintig duizend drachmen, en aan zilver, twee duizend mijnen, en zeven en zestig priesterrokken.
Neh 7:73  En de priesters, en de Levieten, en de poortiers, en de zangers, en sommigen van het volk, en de Nethinim, en gans Israel, woonden in hun steden.
Neh 8:1  Als nu de zevende maand aankwam, en de kinderen Israels in hun steden waren,
Neh 8:2  Zo verzamelde zich al het volk als een enig man op de straat voor de Waterpoort; en zij zeiden tot Ezra, den schriftgeleerde, dat hij het boek der wet van Mozes zou halen, die de HEERE Israel geboden had.
Neh 8:3  En Ezra, de priester, bracht de wet voor de gemeente, beiden mannen en vrouwen, en allen, die verstandig waren om te horen, op den eersten dag der zevende maand.
Neh 8:4  En hij las daarin voor de straat, die voor de Waterpoort is, van het morgen licht aan tot op den middag, voor de mannen en vrouwen, en de verstandigen; en de oren des gansen volks waren naar het wetboek.
Neh 8:5  En Ezra, de schriftgeleerde, stond op een hogen houten stoel, dien zij tot die zaak gemaakt hadden, en nevens hem stond Mattithja, en Sema, en Anaja, en Uria, en Hilkia, en Maaseja, aan zijn rechterhand; en aan zijn linkerhand Pedaja, en Misael, en Malchia, en Hasum, en Hasbaddana, Zacharja en Mesullam.
Neh 8:6  En Ezra opende het boek voor de ogen des gansen volks, want hij was boven al het volk; en als hij het opende, stond al het volk.
Neh 8:7  En Ezra loofde den HEERE, den groten God; en al het volk antwoordde: Amen, amen! met opheffing hunner handen, en neigden zich, en aanbaden den HEERE, met de aangezichten ter aarde.
Neh 8:8  Jesua nu, en Bani, en Serebja, Jamin, Akkub, Sabbethai, Hodia, Maaseja, Kelita, Azaria, Jozabad, Hanan, Pelaja, en de Levieten onderwezen het volk in de wet. En het volk stond op zijn standplaats.
Neh 8:9  En zij lazen in het boek, in de wet Gods, duidelijk; en den zin verklarende, zo maakten zij, dat men het verstond in het lezen.
Neh 8:10  En Nehemia (dezelve is Hattirsatha) en Ezra, de priester, de schriftgeleerde, en de Levieten, die het volk onderwezen, zeiden tot al het volk: Deze dag is den HEERE, uw God, heilig; bedrijft dan geen rouw, en weent niet; want al het volk weende, als zij de woorden der wet hoorden.
Neh 8:11  Voorts zeide hij tot hen: Gaat, eet het vette, en drinkt het zoete, en zendt delen dengenen, voor welken niets bereid is, want deze dag is onzen HEERE heilig; zo bedroeft u niet, want de blijdschap des HEEREN, die is uw sterkte.
Neh 8:12  En de Levieten stilden al het volk, zeggende: Zwijgt, want deze dag is heilig, daarom bedroeft u niet.
Neh 8:13  Toen ging al het volk henen om te eten, en om te drinken, en om delen te zenden, en om grote blijdschap te maken; want zij hadden de woorden verstaan, die men hun had bekend gemaakt.
Neh 8:14  En des anderen daags verzamelden zich de hoofden der vaderen van het ganse volk, de priesters en de Levieten, tot Ezra, den schriftgeleerde, en dat, om verstand te bekomen in de woorden der wet.
Neh 8:15  En zij vonden in de wet geschreven, dat de HEERE door de hand van Mozes geboden had, dat de kinderen Israels in loofhutten zouden wonen, op het feest in de zevende maand;
Neh 8:16  En dat zij het zouden luidbaar maken, en een stem laten doorgaan door al hun steden, en te Jeruzalem, zeggende: Gaat uit op het gebergte, en haalt takken van olijfbomen, en takken van andere olieachtige bomen, en takken van mirtebomen, en takken van palmbomen, en takken van andere dichte bomen, om loofhutten te maken, als er geschreven is.
Neh 8:17  Alzo ging het volk uit en haalden ze, en maakten zich loofhutten, een iegelijk op zijn dak, en in hun voorhoven, en in de voorhoven van Gods huis, en op de straat der Waterpoort, en op de straat van Efraimspoort.
Neh 8:18  En de ganse gemeente dergenen, die uit de gevangenis waren wedergekomen, maakten loofhutten, en woonden in die loofhutten; want de kinderen Israels hadden alzo niet gedaan sinds de dagen van Jesua, den zoon van Nun, tot op dezen dag toe; en er was zeer grote blijdschap.
Neh 8:19  En men las in het wetboek Gods dag bij dag, van den eersten dag tot den laatsten dag. En zij hielden het feest zeven dagen, en op den achtsten dag den verbodsdag, naar het recht.
Neh 9:1  Voorts op den vier en twintigsten dag dezer maand verzamelden zich de kinderen Israels met vasten en met zakken, en aarde was op hen.
Neh 9:2  En het zaad Israels scheidde zich af van alle vreemden. En zij stonden, en deden belijdenis van hun zonden en hunner vaderen ongerechtigheden.
Neh 9:3  Want als zij opgestaan waren op hun standplaats, zo lazen zij in het wetboek des HEEREN, huns Gods, een vierendeel van den dag; en op een ander vierendeel deden zij belijdenis, en aanbaden den HEERE, hun God.
Neh 9:4  Jesua nu, en Bani, Kadmiel, Sebanja, Bunni, Serebja, Bani en Chenani, stonden op het hoge gestoelte der Levieten, en riepen met luider stem tot den HEERE, hun God;
Neh 9:5  En de Levieten, Jesua, en Kadmiel, Bani, Hasabneja; Serebja, Hodia, Sebanja, Petahja, zeiden: Staat op, looft den HEERE, uw God, van eeuwigheid tot in eeuwigheid; en men love den Naam Uwer heerlijkheid, die verhoogd is boven allen lof en prijs!
Neh 9:6  Gij zijt die HEERE alleen, Gij hebt gemaakt den hemel, den hemel der hemelen, en al hun heir, de aarde en al wat daarop is, de zeeen en al wat daarin is, en Gij maakt die allen levend; en het heir der hemelen aanbidt U.
Neh 9:7  Gij zijt die HEERE, de God, Die Abram hebt verkoren, en hem uit Ur der Chaldeen uitgevoerd; en Gij hebt zijn naam gesteld Abraham.
Neh 9:8  En Gij hebt zijn hart getrouw gevonden voor Uw aangezicht, en hebt een verbond met hem gemaakt, dat Gij zoudt geven het land der Kanaanieten, der Hethieten, der Amorieten, en der Ferezieten, en der Jebusieten, en der Girgasieten, dat Gij het zijn zade zoudt geven; en Gij hebt Uw woorden bevestigd, omdat Gij rechtvaardig zijt.
Neh 9:9  En Gij hebt aangezien onzer vaderen ellende in Egypte, en Gij hebt hun geroep gehoord aan de Schelfzee;
Neh 9:10  En Gij hebt tekenen en wonderen gedaan aan Farao, en aan al zijn knechten, en aan al het volk zijns lands; want Gij wist, dat zij trotselijk tegen hen handelden; en Gij hebt U een Naam gemaakt, als het is te dezen dage.
Neh 9:11  En Gij hebt de zee voor hun aangezicht gekliefd, dat zij in het midden der zee op het droge zijn doorgegaan; en hun vervolgers hebt Gij in de diepten geworpen, als een steen in sterke wateren.
Neh 9:12  En Gij hebt ze des daags geleid met een wolkkolom, en des nachts met een vuurkolom, om hen te lichten op den weg, waarin zij zouden wandelen.
Neh 9:13  En Gij zijt neergedaald op den berg Sinai, en hebt met hen gesproken uit den hemel; en Gij hebt hun gegeven rechtmatige rechten, en getrouwe wetten, goede inzettingen en geboden.
Neh 9:14  En Gij hebt Uw heiligen sabbat bekend gemaakt; en Gij hebt hun geboden, en inzettingen en een wet bevolen, door de hand van Uw knecht Mozes.
Neh 9:15  En Gij hebt hun brood uit den hemel gegeven voor hun honger, en hun water uit de steenrots voortgebracht voor hun dorst; en Gij hebt tot hen gezegd, dat zij zouden ingaan om te erven het land, waarover Gij Uw hand ophieft, dat Gij het hun zoudt geven.
Neh 9:16  Maar zij en onze vaders hebben trotselijk gehandeld, en zij hebben hun nek verhard, en niet gehoord naar Uw geboden;
Neh 9:17  En zij hebben geweigerd te horen, en niet gedacht aan Uw wonderen, die Gij bij hen gedaan hadt, en hebben hun nek verhard, en in hun wederspannigheid een hoofd gesteld, om weder te keren tot hun dienstbaarheid. Doch Gij, een God van vergevingen, genadig en barmhartig, lankmoedig, en groot van weldadigheid, hebt hen evenwel niet verlaten.
Neh 9:18  Zelfs, als zij zich een gegoten kalf gemaakt hadden, en gezegd: Dit is uw God, Die u uit Egypte heeft opgevoerd; en grote lasteren gedaan hadden;
Neh 9:19  Hebt Gij hen nochtans door Uw grote barmhartigheid niet verlaten in de woestijn; de wolkkolom week niet van hen des daags, om hen op den weg te leiden, noch de vuurkolom des nachts, om hen te lichten, en dat, op den weg, waarin zij zouden wandelen.
Neh 9:20  En Gij hebt Uw goeden Geest gegeven om hen te onderwijzen; en Uw Manna hebt Gij niet geweerd van hun mond, en water hebt Gij hun gegeven voor hun dorst.
Neh 9:21  Alzo hebt Gij hen veertig jaren onderhouden in de woestijn; zij hebben geen gebrek gehad; hun klederen zijn niet veroud, en hun voeten niet gezwollen.
Neh 9:22  Voorts hebt Gij hun koninkrijken en volken gegeven, en hebt hen verdeeld in hoeken. Alzo hebben zij erfelijk bezeten het land van Sihon, te weten, het land des konings van Hesbon, en het land van Og, koning van Basan.
Neh 9:23  Gij hebt ook hun kinderen vermenigvuldigd, als de sterren des hemels; en Gij hebt hen gebracht in het land, waarvan Gij tot hun vaderen hadt gezegd, dat zij zouden ingaan om het erfelijk te bezitten.
Neh 9:24  Alzo zijn de kinderen daarin gekomen, en hebben dat land erfelijk ingenomen; en Gij hebt de inwoners des lands, de Kanaanieten, voor hun aangezicht ten ondergebracht, en hebt hen in hun hand gegeven, mitsgaders hun koningen en de volken des lands, om daarmede te doen naar hun welgevallen.
Neh 9:25  En zij hebben vaste steden en een vet land ingenomen, en erfelijk bezeten, huizen, vol van alle goed, uitgehouwen bornputten, wijngaarden, olijfgaarden en bomen van spijze, in menigte; en zij hebben gegeten, en zijn zat en vet geworden, en hebben in wellust geleefd, door Uw grote goedigheid.
Neh 9:26  Maar zij zijn wederspannig geworden, en hebben tegen U gerebelleerd, en Uw wet achter hun rug geworpen, en Uw profeten gedood die tegen hen betuigden, om hen te doen wederkeren tot U; alzo hebben zij grote lasteren gedaan.
Neh 9:27  Daarom hebt Gij hen gegeven in de hand hunner benauwers, die hen benauwd hebben; maar als zij in den tijd hunner benauwdheid tot U riepen, hebt Gij van den hemel gehoord, en hun naar Uw grote barmhartigheden verlossers gegeven, die hen uit de hand hunner benauwers verlosten.
Neh 9:28  Maar als zij rust hadden, keerden zij weder om kwaad te doen voor Uw aangezicht; zo verliet Gij hen in de hand hunner vijanden, dat zij over hen heersten; als zij zich dan bekeerden, en U aanriepen, zo hebt Gij hen van den hemel gehoord, en hebt hen naar Uw barmhartigheden tot vele tijden uitgerukt.
Neh 9:29  En Gij hebt tegen hen betuigd, om hen te doen wederkeren tot Uw wet; maar zij hebben trotselijk gehandeld, en niet gehoord naar Uw geboden, en tegen Uw rechten, tegen dezelve hebben zij gezondigd, door dewelke een mens, die ze doet, leven zal; en zij hebben hun schouder teruggetogen, en hun nek verhard, en niet gehoord.
Neh 9:30  Doch Gij vertoogt het vele jaren over hen, en betuigdet tegen hen door Uw Geest, door den dienst Uwer profeten, maar zij neigden het oor niet; daarom hebt Gij hen gegeven in de hand van de volken der landen.
Neh 9:31  Doch door Uw grote barmhartigheden hebt Gij hen niet vernield, noch hen verlaten; want Gij zijt een genadig en barmhartig God.
Neh 9:32  Nu dan, o onze God, Gij grote, Gij machtige, en Gij vreselijke God, Die het verbond en de weldadigheid houdt; laat voor Uw aangezicht niet gering zijn al de moeite, die ons getroffen heeft, onze koningen, onze vorsten, en onze priesteren; en onze profeten, en onze vaderen, en Uw ganse volk, van de dagen der koningen van Assur af tot op dezen dag.
Neh 9:33  Doch Gij zijt rechtvaardig, in alles, wat ons overkomen is; want Gij hebt trouwelijk gehandeld, maar wij hebben goddelooslijk gehandeld.
Neh 9:34  En onze koningen, onze vorsten, onze priesters en onze vaders hebben Uw wet niet gedaan; en zij hebben niet geluisterd naar Uw geboden, en naar Uw getuigenissen, die Gij tegen hen betuigdet.
Neh 9:35  Want zij hebben U niet gediend in hun koninkrijk, en in Uw menigvuldig goed, dat Gij hun gaaft, en in dat wijde en dat vette land, dat Gij voor hun aangezicht gegeven hadt; en zij hebben zich niet bekeerd van hun boze werken.
Neh 9:36  Zie, wij zijn heden knechten; ja, het land, dat Gij onzen vaderen gegeven hebt, om de vrucht daarvan, en het goede daarvan te eten, zie, daarin zijn wij knechten.
Neh 9:37  En het vermenigvuldigt zijn inkomste voor den koningen, die Gij over ons gesteld hebt, om onzer zonden wil; en zij heersen over onze lichamen en over onze beesten, naar hun welgevallen; alzo zijn wij in grote benauwdheid.
Neh 9:38  En in dit alles maken wij een vast verbond en schrijven het; en onze vorsten, onze Levieten en onze priesteren zullen het verzegelen.
Neh 10:1  Tot de verzegelingen nu waren: Nehemia Hattirsatha, zoon van Hachalja, en Zidkia,
Neh 10:2  Seraja, Azarja, Jeremia,
Neh 10:3  Pashur, Amarja, Malchia,
Neh 10:4  Hattus, Sebanja, Malluch,
Neh 10:5  Harim, Meremoth, Obadja,
Neh 10:6  Daniel, Ginnethon, Baruch,
Neh 10:7  Mesullam, Abia, Mijamin,
Neh 10:8  Maazia, Bilgai, Semaja. Dit waren de priesters.
Neh 10:9  En de Levieten, namelijk: Jesua, zoon van Azanja, Binnui; van de zonen van Henadad, Kadmiel;
Neh 10:10  En hun broederen: Sebanja, Hodia, Kelita, Pelaja, Hanan,
Neh 10:11  Micha, Rehob, Hasabja,
Neh 10:12  Zakkur, Serebja, Sebanja,
Neh 10:13  Hodia, Bani, Beninu;
Neh 10:14  De hoofden des volks: Parhos, Pahath-moab, Elam, Zatthu, Bani,
Neh 10:15  Bunni, Azgad, Bebai,
Neh 10:16  Adonia, Bigvai, Adin,
Neh 10:17  Ater, Hizkia, Azzur,
Neh 10:18  Hodia, Hasum, Bezai,
Neh 10:19  Harif, Anathoth, Nebai,
Neh 10:20  Magpias, Mesullam, Hezir,
Neh 10:21  Mesezabeel, Zadok, Jaddua,
Neh 10:22  Pelatja, Hanan, Anaja,
Neh 10:23  Hosea, Hananja, Hassub,
Neh 10:24  Hallohes, Pilha, Sobek,
Neh 10:25  Rehum, Hasabna, Maaseja,
Neh 10:26  En Ahia, Hanan, Anan,
Neh 10:27  Malluch, Harim, Baana.
Neh 10:28  En het overige des volks, de priesteren, de Levieten, de poortiers, de zangers, de Nethinim, en al wie zich van de volken der landen had afgescheiden tot Gods wet, hun vrouwen, hun zonen en hun dochteren, al wie wetenschap en verstand had;
Neh 10:29  Die hielden zich aan hun broederen, hun voortreffelijken, en kwamen in den vloek en in den eed, dat zij zouden wandelen in de wet Gods, die gegeven is door de hand van den knecht Gods, Mozes; en dat zij zouden houden, en dat zij zouden doen al de geboden des HEEREN, onzes Heeren, en Zijn rechten en Zijn inzettingen;
Neh 10:30  En dat wij onze dochteren niet zouden geven aan de volken des lands, noch hun dochteren nemen voor onze zonen.
Neh 10:31  Ook als de volken des lands waren en alle koren op den sabbatdag ten verkoop brengen, dat wij op den sabbat, of op een anderen heiligen dag van hen niet zouden nemen; en dat wij het zevende jaar zouden vrij laten, mitsgaders allerhande bezwaarnis.
Neh 10:32  Voorts zetten wij ons geboden op, ons opleggende een derde deel van een sikkel in het jaar, tot den dienst van het huis onzes Gods;
Neh 10:33  Tot het brood der toerichting, en het gedurig spijsoffer, en tot het gedurig brandoffer, der sabbatten, der nieuwe maanden, tot de gezette hoogtijden, en tot de heilige dingen, en tot de zondofferen, om verzoening te doen over Israel; en tot alle werk van het huis onzes Gods.
Neh 10:34  Ook wierpen wij de loten, onder de priesters, de Levieten en het volk, over het offer van het hout, dat men brengen zou ten huize onzes Gods, naar het huis onzer vaderen, op bestemde tijden, jaar op jaar, om te branden op het altaar des HEEREN, onzes Gods, gelijk het in de wet geschreven is;
Neh 10:35  Dat wij ook de eerstelingen onzes lands en de eerstelingen van alle vrucht van al het geboomte, jaar op jaar, zouden brengen ten huize des HEEREN;
Neh 10:36  En de eerstgeborenen onzer zonen en onzer beesten, gelijk het in de wet geschreven is; en dat wij de eerstgeborenen onzer runderen en onzer schapen zouden brengen ten huize onzes Gods, tot de priesteren, die in het huis onzes Gods dienen.
Neh 10:37  En dat wij de eerstelingen onzes deegs, en onze hefofferen, en de vrucht aller bomen, most en olie, zouden brengen tot de priesteren, in de kameren van het huis onzes Gods, en de tienden onzes lands tot de Levieten; en dat dezelfde Levieten de tienden zouden hebben in alle steden onzer landbouwerij;
Neh 10:38  En dat er een priester, een zoon van Aaron, bij de Levieten zou zijn, als de Levieten de tienden ontvangen; en dat de Levieten de tienden zouden opbrengen ten huize onzes Gods, in de kameren van het schathuis.
Neh 10:39  Want de kinderen Israels en de kinderen van Levi moeten hefoffer van koren, most en olie in die kameren brengen, omdat aldaar de vaten des heiligdoms zijn, en de priesteren, die dienen, en de poortiers, en de zangers; dat wij alzo het huis onzes Gods niet zouden verlaten.
Neh 11:1  Voorts woonden de oversten des volks te Jeruzalem; maar het overige des volks wierpen loten, om uit tien een uit te brengen, die in de heilige stad Jeruzalem zou wonen, en negen delen in de andere steden.
Neh 11:2  En het volk zegende al de mannen, die vrijwilliglijk aanboden te Jeruzalem te wonen.
Neh 11:3  En dit zijn de hoofden van het landschap, die te Jeruzalem woonden; (maar in de steden van Juda woonden, een iegelijk op zijn bezitting, in hun steden, Israel, de priesters, en de Levieten, en de Nethinim, en de kinderen der knechten van Salomo).
Neh 11:4  Te Jeruzalem dan woonden sommigen van de kinderen van Juda, en van de kinderen van Benjamin. Van de kinderen van Juda: Athaja, de zoon van Uzzia, den zoon van Zacharja, den zoon van Amarja, den zoon van Sefatja, den zoon van Mahalaleel, van de kinderen van Perez;
Neh 11:5  En Maaseja, de zoon van Baruch, den zoon van Kol-hose, den zoon van Hazaja, den zoon van Adaja, den zoon van Jojarib, den zoon van Zacharja, den zoon van Siloni.
Neh 11:6  Alle kinderen van Perez, die te Jeruzalem woonden, waren vierhonderd acht en zestig dappere mannen.
Neh 11:7  En dit zijn de kinderen van Benjamin: Sallu, de zoon van Mesullam, den zoon van Joed, den zoon van Pedaja, den zoon van Kolaja, den zoon van Maaseja, den zoon van Ithiel, den zoon van Jesaja;
Neh 11:8  En na hem Gabbai, Sallai; negenhonderd acht en twintig.
Neh 11:9  En Joel, de zoon van Zichri, was opziener over hen; en Juda, de zoon van Senua, was de tweede over de stad.
Neh 11:10  Van de priesteren: Jedaja, de zoon van Jojarib, Jachin;
Neh 11:11  Seraja, de zoon van Hilkia, den zoon van Mesullam, den zoon van Zadok, den zoon van Merajoth, den zoon van Ahitub, was voorganger van Gods huis;
Neh 11:12  En hun broederen, die het werk in het huis deden, waren achthonderd twee en twintig. En Adaja, de zoon van Jeroham, den zoon van Pelalja, den zoon van Amzi, den zoon van Zacharja, den zoon van Pashur, den zoon van Malchia;
Neh 11:13  En zijn broederen, hoofden der vaderen, waren tweehonderd twee en veertig. En Amassai, de zoon van Azareel, den zoon van Achzai, den zoon van Mesillemoth, den zoon van Immer;
Neh 11:14  En hun broederen, dappere helden, waren honderd acht en twintig; en opziener over hen was Zabdiel, de zoon van Gedolim.
Neh 11:15  En van de Levieten: Semaja, de zoon van Hassub, den zoon van Azrikam, den zoon van Hasabja, den zoon van Buni.
Neh 11:16  En Sabbethai, en Jozabad, van de hoofden der Levieten, waren over het buitenwerk van het huis Gods.
Neh 11:17  En Matthanja, de zoon van Micha, den zoon van Zabdi, den zoon van Asaf, was het hoofd, die de dankzegging begon in het gebed, en Bakbukja was de tweede van zijn broederen; en Abda, de zoon van Sammua, den zoon van Galal, den zoon van Jeduthun.
Neh 11:18  Al de Levieten in de heilige stad waren tweehonderd vier en tachtig.
Neh 11:19  En de poortiers: Akkub, Talmon, met hun broederen, die wacht hielden in de poorten, waren honderd twee en zeventig.
Neh 11:20  Het overige nu van Israel, van de priesters en de Levieten, was in alle steden van Juda, een iegelijk in zijn erfdeel.
Neh 11:21  En de Nethinim woonden in Ofel; en Ziha en Gispa waren over de Nethinim.
Neh 11:22  En der Levieten opziener te Jeruzalem was Uzzi, de zoon van Bani, den zoon van Hasabja, den zoon van Matthanja, den zoon van Micha; van de kinderen van Asaf waren de zangers tegenover het werk van Gods huis.
Neh 11:23  Want er was een gebod des konings van hen, te weten, een zeker onderhoud voor de zangers, van elk dagelijks op zijn dag.
Neh 11:24  En Petahja, de zoon van Mesezabeel, van de kinderen van Zerah, den zoon van Juda, was aan des konings hand, in alle zaken tot het volk.
Neh 11:25  In de dorpen nu op hun akkers woonden sommigen van de kinderen van Juda, in Kirjath-arba en haar onderhorige plaatsen, en in Dibon en haar onderhorige plaatsen, en in Jekabzeel en haar dorpen;
Neh 11:26  En te Jesua, en te Molada, en te Beth-pelet,
Neh 11:27  En te Hazar-sual, en in Ber-seba, en haar onderhorige plaatsen,
Neh 11:28  En te Ziklag, en in Mechona en haar onderhorige plaatsen,
Neh 11:29  En te En-rimmon, en te Zora, en te Jarmuth,
Neh 11:30  Zanoah, Adullam en haar dorpen, Lachis en haar akkers, Azeka en haar onderhorige plaatsen; en zij legerden zich van Ber-seba af tot aan het dal Hinnom.
Neh 11:31  De kinderen van Benjamin nu van Geba woonden in Michmas, en Aja, en Beth-el, en haar onderhorige plaatsen,
Neh 11:32  Anathoth, Nob, Ananja,
Neh 11:33  Hazor, Rama, Gitthaim,
Neh 11:34  Hadid, Zeboim, Neballat,
Neh 11:35  Lod, en Ono, in het dal der werkmeesters.
Neh 11:36  Van de Levieten nu, woonden sommigen in de verdelingen van Juda, en van Benjamin.
Neh 12:1  Dit nu zijn de priesters en de Levieten, die met Zerubbabel, den zoon van Sealthiel, en Jesua, optogen: Seraja, Jeremia, Ezra,
Neh 12:2  Amarja, Malluch, Hattus,
Neh 12:3  Sechanja, Rehum, Meremoth,
Neh 12:4  Iddo, Ginnethoi, Abia,
Neh 12:5  Mijamin, Maadja, Bilga,
Neh 12:6  Semaja, en Jojarib, Jedaja,
Neh 12:7  Sallu, Amok, Hilkia, Jedaja; dat waren de hoofden der priesteren, en hun broederen, in de dagen van Jesua.
Neh 12:8  En de Levieten waren: Jesua, Binnui, Kadmiel, Serebja, Juda, Matthanja; hij en zijn broederen waren over de dankzeggingen.
Neh 12:9  En Bakbukja, en Unni, hun broederen, waren tegen hen over in de wachten.
Neh 12:10  Jesua nu gewon Jojakim, en Jojakim gewon Eljasib, en Eljasib gewon Jojada,
Neh 12:11  En Jojada gewon Jonathan, en Jonathan gewon Jaddua.
Neh 12:12  En in de dagen van Jojakim waren priesters, hoofden der vaderen: van Seraja was Meraja; van Jeremia, Hananja;
Neh 12:13  Van Ezra, Mesullam; van Amarja, Johanan;
Neh 12:14  Van Melichu, Jonathan; van Sebanja, Jozef;
Neh 12:15  Van Harim, Adna; van Merajoth, Helkai;
Neh 12:16  Van Iddo, Zacharia; van Ginnethon, Mesullam;
Neh 12:17  Van Abia, Zichri; van Minjamin, van Moadja, Piltai;
Neh 12:18  Van Bilga, Sammua; van Semaja, Jonathan;
Neh 12:19  En van Jojarib, Matthenai; van Jedaja, Uzzi;
Neh 12:20  Van Sallai, Kallai; van Amok, Heber;
Neh 12:21  Van Hilkia, Hasabja; van Jedaja, Nethaneel.
Neh 12:22  Van de Levieten werden in de dagen van Eljasib, Jojada, en Johanan, en Jaddua, de hoofden der vaderen beschreven; mitsgaders de priesteren, tot het koninkrijk van Darius, den Perziaan.
Neh 12:23  De kinderen van Levi, de hoofden der vaderen, werden beschreven in het boek der kronieken, tot de dagen van Johanan, den zoon van Eljasib, toe.
Neh 12:24  De hoofden dan der Levieten waren Hasabja, Serebja, en Jesua, de zoon van Kadmiel, en hun broederen tegen hen over, om te prijzen en te danken, naar het gebod van David, den man Gods, wacht tegen wacht.
Neh 12:25  Matthanja en Bakbukja, Obadja, Mesullam, Talmon en Akkub, waren poortiers, de wacht waarnemende bij de schatkamers der poorten.
Neh 12:26  Dezen waren in de dagen van Jojakim, den zoon van Jesua, den zoon van Jozadak, en in de dagen van Nehemia, den landvoogd, en van den priester Ezra, den schriftgeleerde.
Neh 12:27  In de inwijding nu van Jeruzalems muur, zochten zij de Levieten uit al hun plaatsen, dat zij hen te Jeruzalem brachten, om de inwijding te doen met vreugde, en met dankzeggingen, en met gezang, cimbalen, luiten, en met harpen.
Neh 12:28  Alzo werden de kinderen der zangers verzameld, zo uit het vlakke veld rondom Jeruzalem, als uit de dorpen van de Netofathieten;
Neh 12:29  En uit het huis van Gilgal, en uit de velden van Geba en Asmaveth; want de zangers hadden zich dorpen gebouwd rondom Jeruzalem.
Neh 12:30  En de priesters en de Levieten reinigden zichzelven; daarna reinigden zij het volk, en de poorten, en den muur.
Neh 12:31  Toen deed ik de vorsten van Juda opgaan op den muur; en ik stelde twee grote dankkoren en omgangen, een ter rechterhand op den muur, naar de Mistpoort toe.
Neh 12:32  En achter hen ging Hosaja, en de helft der vorsten van Juda.
Neh 12:33  En Azarja, Ezra, en Mesullam,
Neh 12:34  Juda, en Benjamin, en Semaja, en Jeremia;
Neh 12:35  En van de priesters kinderen met trompetten: Zacharja, de zoon van Jonathan, den zoon van Semaja, den zoon van Matthanja, den zoon van Michaja, den zoon van Zakkur, den zoon van Asaf;
Neh 12:36  En zijn broeders, Semaja, en Azareel, Milalai, Gilalai, Maai, Nethaneel, en Juda, Hanani, met muziekinstrumenten van David, den man Gods; en Ezra, de schriftgeleerde, ging voor hun aangezicht heen.
Neh 12:37  Voorts naar de Fonteinpoort, en tegen hen over, gingen zij op bij de trappen van Davids stad, door den opgang des muurs, boven Davids huis, tot aan de Waterpoort, tegen het oosten.
Neh 12:38  Het tweede dankkoor nu ging tegenover, en ik achter hetzelve, met de helft des volks, op den muur, van boven den Bakoventoren, tot aan den breden muur;
Neh 12:39  En van boven de poort van Efraim, en boven de Oude poort, en boven de Vispoort, en den toren Hananeel, en den toren Mea, tot aan de Schaapspoort, en zij bleven staan in de Gevangenpoort.
Neh 12:40  Daarna stonden de beide dankkoren in Gods huis; ook ik en de helft der overheden met mij.
Neh 12:41  En de priesters, Eljakim, Maaseja, Minjamin, Michaja, Eljoenei, Zacharja, Hananja, met trompetten;
Neh 12:42  Voorts Maaseja, en Semaja, en Eleazar, en Uzzi, en Johanan, en Malchia, en Elam, en Ezer; ook lieten zich de zangers horen, met Jizrahja, den opziener.
Neh 12:43  En zij offerden deszelven daags grote slachtofferen, en waren vrolijk; want God had hen vrolijk gemaakt met grote vrolijkheid; en ook waren de vrouwen en de kinderen vrolijk; zodat de vrolijkheid van Jeruzalem tot van verre gehoord werd.
Neh 12:44  Ook werden ten zelfden dage mannen gesteld over de kameren, tot de schatten, tot de hefofferen, tot de eerstelingen en tot de tienden, om daarin uit de akkers der steden te verzamelen de delen der wet, voor de priesteren en voor de Levieten; want Juda was vrolijk over de priesteren en over de Levieten, die daar stonden.
Neh 12:45  En de wacht huns Gods waarnamen, en de wacht der reiniging, ook de zangers, en de poortiers, naar het gebod van David en zijn zoon Salomo.
Neh 12:46  Want in de dagen van David en Asaf, van ouds, waren er hoofden der zangers, en des lofgezangs, en der dankzeggingen tot God.
Neh 12:47  Daarom gaf gans Israel, in de dagen van Zerubbabel, en in de dagen van Nehemia, de delen der zangers en der poortiers, van elk dagelijks op zijn dag; en zij heiligden voor de Levieten, en de Levieten heiligden voor de kinderen van Aaron.
Neh 13:1  Te dien dage werd er gelezen in het boek van Mozes, voor de oren des volks; en daarin werd geschreven gevonden, dat de Ammonieten en Moabieten niet zouden komen in de gemeente Gods, tot in eeuwigheid;
Neh 13:2  Omdat zij den kinderen Israels niet waren tegengekomen met brood en met water, ja, Bileam tegen hen gehuurd hadden, om hen te vloeken, hoewel onze God den vloek omkeerde in een zegen.
Neh 13:3  Zo geschiedde het, als zij deze wet hoorden, dat zij alle vermengeling van Israel afscheidden.
Neh 13:4  Eljasib nu, de priester, die gesteld was over de kamer van het huis onzes Gods, was voor dezen nabestaande van Tobia geworden.
Neh 13:5  En hij had hem een grote kamer gemaakt, alwaar zij te voren henenleiden het spijsoffer, den wierook en de vaten, en de tienden van koren, van most en van olie, die bevolen waren voor de Levieten, en de zangers, en de poortiers, mitsgaders het hefoffer der priesteren.
Neh 13:6  Doch in dit alles was ik niet te Jeruzalem; want in het twee en dertigste jaar van Arthahsasta, koning van Babel, kwam ik tot den koning; maar ten einde van sommige dagen verkreeg ik weder verlof van den koning.
Neh 13:7  En ik kwam te Jeruzalem, en verstond van het kwaad, dat Eljasib voor Tobia gedaan had, makende hem een kamer in de voorhoven van Gods huis.
Neh 13:8  En het mishaagde mij zeer; zo wierp ik al het huisraad van Tobia buiten, uit de kamer.
Neh 13:9  Voorts gaf ik bevel, en zij reinigden de kameren; en ik bracht daar weder in de vaten van Gods huis, met het spijsoffer en den wierook.
Neh 13:10  Ook vernam ik, dat der Levieten deel hun niet gegeven was; zodat de Levieten en de zangers, die het werk deden, gevloden waren, een iegelijk naar zijn akker.
Neh 13:11  En ik twistte met de overheden, en zeide: Waarom is het huis Gods verlaten? Doch ik vergaderde hen, en herstelde ze in hun stand.
Neh 13:12  Toen bracht gans Juda de tienden van het koren, en van den most, en van de olie, in de schatten.
Neh 13:13  En ik stelde tot schatmeesters over de schatten, Selemja, den priester, en Zadok, den schrijver, en Pedaja, uit de Levieten; en aan hun hand Hanan, den zoon van Zakkur, den zoon van Matthanja; want zij werden getrouw geacht, en hun werd opgelegd aan hun broederen uit te delen.
Neh 13:14  Gedenk mijner, mijn God, in dezen; en delg mijn weldadigheden niet uit, die ik aan het huis mijns Gods en aan Zijn wachten gedaan heb.
Neh 13:15  In dezelfde dagen zag ik in Juda, die persen traden op den sabbat, en die garven inbrachten, die zij op ezels laadden; als ook wijn, druiven en vijgen, en allen last, dien zij te Jeruzalem inbrachten op den sabbatdag; en ik betuigde tegen hen ten dage, als zij eetwaren verkochten.
Neh 13:16  Daar waren ook Tyriers binnen, die vis aanbrachten, en alle koopwaren, die zij op den sabbat verkochten aan de kinderen van Juda en te Jeruzalem.
Neh 13:17  Zo twistte ik met de edelen van Juda, en zeide tot hen: Wat voor een boos ding is dit, dat gijlieden doet, en ontheiligt den sabbatdag?
Neh 13:18  Deden niet uw vaders alzo, en onze God bracht al dit kwaad over ons en over deze stad? En gijlieden maakt de hittige gramschap nog meer over Israel, ontheiligende den sabbat.
Neh 13:19  Het geschiedde nu, als de poorten van Jeruzalem schaduw gaven, voor den sabbat, dat ik bevel gaf, en de deuren werden gesloten; en ik beval, dat zij ze niet zouden opendoen tot na den sabbat; en ik stelde van mijn jongens aan de poorten, opdat er geen last zou inkomen op den sabbatdag.
Neh 13:20  Toen vernachtten de kramers, en de verkopers van alle koopwaren, buiten voor Jeruzalem, eens of tweemaal.
Neh 13:21  Zo betuigde ik tegen hen, en zeide tot hen: Waarom vernacht gijlieden tegenover den muur? Zo gij het weder doet, zal ik de hand aan u slaan. Van dien tijd af kwamen zij niet op den sabbat.
Neh 13:22  Voorts zeide ik tot de Levieten, dat zij zich zouden reinigen, en de poorten komen wachten, om den sabbatdag te heiligen. Gedenk mijner ook in dezen, mijn God! en verschoon mij naar de veelheid Uwer goedertierenheid.
Neh 13:23  Ook zag ik in die dagen Joden, die Asdodische, Ammonietische en Moabietische vrouwen bij zich hadden doen wonen.
Neh 13:24  En hun kinderen spraken half Asdodisch, en zij konden geen Joods spreken, maar naar de taal eens iegelijken volks.
Neh 13:25  Zo twistte ik met hen, en vloekte hen, en sloeg sommige mannen van hen, en plukte hun het haar uit; en ik deed hen zweren bij God: Indien gij uw dochteren hun zonen zult geven, en indien gij van hun dochteren voor uw zonen of voor u zult nemen!
Neh 13:26  Heeft niet Salomo, de koning van Israel, daarin gezondigd, hoewel er onder vele heidenen geen koning was, gelijk hij, en hij zijn God lief was, en God hem ten koning over gans Israel gesteld had? Ook hem deden de vreemde vrouwen zondigen.
Neh 13:27  Zouden wij dan naar ulieden horen, dat gij al dit grote kwaad zoudt doen, overtredende tegen onzen God, doende vreemde vrouwen bij u wonen?
Neh 13:28  Ook was er een van de kinderen van Jojada, den zoon van Eljasib, den hogepriester, schoonzoon geworden van Sanballat, den Horoniet; daarom jaagde ik hem van mij weg.
Neh 13:29  Gedenk aan hen, mijn God, omdat zij het priesterdom hebben verontreinigd, ja, het verbond des priesterdoms en der Levieten.
Neh 13:30  Alzo reinigde ik hen van alle vreemden; en ik bestelde de wachten der priesteren en der Levieten, elk op zijn werk;
Neh 13:31  Ook tot het offer des houts, op bestemde tijden, en tot de eerstelingen. Gedenk mijner, mijn God, ten goede.



\end{document}