\begin{document}

\title{Exodus}



\chapter{1}

\par 1 Dit nu zijn de namen der zonen van Israel, die in Egypte gekomen zijn, met Jakob; zij kwamen er in, elk met zijn huis.
\par 2 Ruben, Simeon, Levi, en Juda;
\par 3 Issaschar, Zebulon, en Benjamin;
\par 4 Dan en Nafthali, Gad en Aser.
\par 5 Al de zielen nu, die uit Jakobs heup voortgekomen zijn, waren zeventig zielen; doch Jozef was in Egypte.
\par 6 Toen nu Jozef gestorven was, en al zijn broeders, en al dat geslacht,
\par 7 Zo werden de kinderen Israels vruchtbaar en wiesen overvloedig, en zij vermeerderden, en werden gans zeer machtig, zodat het land met hen vervuld werd.
\par 8 Daarna stond een nieuwe koning op over Egypte, die Jozef niet gekend had;
\par 9 Die zeide tot zijn volk: Ziet, het volk der kinderen Israels is veel, ja, machtiger dan wij.
\par 10 Komt aan, laat ons wijselijk tegen hetzelve handelen, opdat het niet vermenigvuldige, en het geschiede, als er enige krijg voorvalt, dat het zich ook niet vervoege tot onze vijanden, en tegen ons strijde, en uit het land optrekke.
\par 11 En zij zetten oversten der schattingen over hetzelve, om het te verdrukken met hun lasten; want men bouwde voor Farao schatsteden, Pitom en Raamses.
\par 12 Maar hoe meer zij het verdrukten, hoe meer het vermeerderde, en hoe meer het wies; zodat zij verdrietig waren vanwege de kinderen Israels.
\par 13 En de Egyptenaars deden de kinderen Israels dienen met hardigheid;
\par 14 Zodat zij hun het leven bitter maakten met harden dienst, in leem en in tichelstenen, en met allen dienst op het veld, met al hun dienst, dien zij hen deden dienen met hardigheid.
\par 15 Daarenboven sprak de koning van Egypte tot de vroedvrouwen der Hebreinnen, welker ener naam Sifra, en de naam der andere Pua was;
\par 16 En zeide: Wanneer gij de Hebreinnen in het baren helpt, en ziet haar op de stoelen; is het een zoon, zo doodt hem; maar is het een dochter, zo laat haar leven!
\par 17 Doch de vroedvrouwen vreesden God, en deden niet, gelijk als de koning van Egypte tot haar gesproken had, maar zij behielden de knechtjes in het leven.
\par 18 Toen riep de koning van Egypte de vroedvrouwen, en zeide tot haar: Waarom hebt gijlieden deze zaak gedaan, dat gij de knechtjes in het leven behouden hebt?
\par 19 En de vroedvrouwen zeiden tot Farao: Omdat de Hebreinnen niet zijn gelijk de Egyptische vrouwen; want zij zijn sterk; eer de vroedvrouw tot haar komt, zo hebben zij gebaard.
\par 20 Daarom deed God aan de vroedvrouwen goed; en dat volk vermeerderde, en het werd zeer machtig.
\par 21 En het geschiedde, dewijl de vroedvrouwen God vreesden, zo bouwde Hij haar huizen.
\par 22 Toen gebood Farao aan al zijn volk, zeggende: Alle zonen, die geboren worden, zult gij in de rivier werpen, maar al de dochteren in het leven behouden.

\chapter{2}

\par 1 En een man van het huis van Levi ging, en nam een dochter van Levi.
\par 2 En de vrouw werd zwanger, en baarde een zoon. Toen zij hem zag, dat hij schoon was, zo verborg zij hem drie maanden.
\par 3 Doch als zij hem niet langer verbergen kon, zo nam zij voor hem een kistje van biezen, en belijmde het met lijm en met pek; en zij leide het knechtje daarin, en leide het in de biezen, aan den oever der rivier.
\par 4 En zijn zuster stelde zich van verre, om te weten, wat hem gedaan zou worden.
\par 5 En de dochter van Farao ging af, om zich te wassen in de rivier; en haar jonkvrouwen wandelden aan den kant der rivier; toen zij het kistje in het midden van de biezen zag, zo zond zij haar dienstmaagd heen, en liet het halen.
\par 6 Toen zij het open deed, zo zag zij dat knechtje; en ziet, het jongsken weende; en zij werd met barmhartigheid bewogen over hetzelve, en zij zeide: Dit is een van de knechtjes der Hebreen!
\par 7 Toen zeide zijn zuster tot Farao's dochter: Zal ik heengaan, en u een voedstervrouw uit de Hebreinnen roepen, die dat knechtje voor u zoge?
\par 8 En de dochter van Farao zeide tot haar: Ga heen. En de jonge maagd ging, en riep des knechtjes moeder.
\par 9 Toen zeide Farao's dochter tot haar: Neem dit knechtje heen, en zoog het mij; ik zal u uw loon geven. En de vrouw nam het knechtje en zoogde het.
\par 10 En toen het knechtje groot geworden was, zo bracht zij het tot Farao's dochter, en het werd haar ten zoon; en zij noemde zijn naam Mozes, en zeide: Want ik heb hem uit het water getogen.
\par 11 En het geschiedde in die dagen, toen Mozes groot geworden was, dat hij uitging tot zijn broederen, en bezag hun lasten; en hij zag, dat een Egyptisch man een Hebreeuwsen man uit zijn broederen sloeg.
\par 12 En hij zag herwaarts en gindswaarts; en toen hij zag, dat er niemand was, zo versloeg hij den Egyptenaar, en verborg hem in het zand.
\par 13 Des anderen daags ging hij wederom uit, en ziet, twee Hebreeuwse mannen twistten; en hij zeide tot den ongerechte: Waarom slaat gij uw naaste?
\par 14 Hij dan zeide: Wie heeft u tot een overste en rechter over ons gezet? Zegt gij dit, om mij te doden, gelijk gij den Egyptenaar gedood hebt? Toen vreesde Mozes, en zeide: Voorwaar, deze zaak is bekend geworden!
\par 15 Als nu Farao deze zaak hoorde, zo zocht hij Mozes te doden; doch Mozes vlood voor Farao's aangezicht, en woonde in het land Midian, en hij zat bij een waterput.
\par 16 En de priester in Midian had zeven dochters, die kwamen om te putten, en vulden de drinkbakken, om de kudde haars vaders te drenken.
\par 17 Toen kwamen de herders, en zij dreven haar van daar; doch Mozes stond op, en verloste ze, en drenkte haar kudden.
\par 18 En toen zij tot haar vader Rehuel kwamen, zo sprak hij: Waarom zijt gij heden zo haast wedergekomen?
\par 19 Toen zeiden zij: Een Egyptisch man heeft ons verlost uit de hand der herderen; en hij heeft ook overvloedig voor ons geput, en de kudde gedrenkt.
\par 20 En hij zeide tot zijn dochters: Waar is hij toch, waarom liet gij den man nu gaan? roept hem, dat hij brood ete.
\par 21 En Mozes bewilligde bij den man te wonen; en hij gaf Mozes zijn dochter Zippora;
\par 22 Die baarde een zoon; en hij noemde zijn naam Gersom; want hij zeide: Ik ben een vreemdeling geworden in een vreemd land.
\par 23 En het geschiedde na vele dezer dagen, als de koning van Egypte gestorven was, dat de kinderen Israels zuchtten en schreeuwden over den dienst; en hun gekrijt over hun dienst kwam op tot God.
\par 24 En God hoorde hun gekerm, en God gedacht aan Zijn verbond met Abraham, met Izak, en met Jakob.
\par 25 En God zag de kinderen Israels aan, en God kende hen.

\chapter{3}

\par 1 En Mozes hoedde de kudde van Jethro, zijn schoonvader, den priester in Midian; en hij leidde de kudde achter de woestijn, en hij kwam aan den berg Gods, aan Horeb.
\par 2 En de Engel des HEEREN verscheen hem in een vuurvlam uit het midden van een braambos; en hij zag, en ziet, het braambos brandde in het vuur, en het braambos werd niet verteerd.
\par 3 En Mozes zeide: Ik zal mij nu daarheen wenden, en bezien dat grote gezicht, waarom het braambos niet verbrandt.
\par 4 Toen de HEERE zag, dat hij zich daarheen wendde, om te bezien, zo riep God tot hem uit het midden van het braambos, en zeide: Mozes, Mozes! En hij zeide: Zie, hier ben ik!
\par 5 En Hij zeide: Nader hier niet toe; trek uw schoenen uit van uw voeten; want de plaats, waarop gij staat, is heilig land.
\par 6 Hij zeide voorts: Ik ben de God uws vaders, de God van Abraham, de God van Izak en de God van Jakob. En Mozes verborg zijn aangezicht, want hij vreesde God aan te zien.
\par 7 En de HEERE zeide: Ik heb zeer wel gezien de verdrukking Mijns volks, hetwelk in Egypte is, en heb hun geschrei gehoord, vanwege hun drijvers; want Ik heb hun smarten bekend.
\par 8 Daarom ben Ik nedergekomen, dat Ik het verlosse uit de hand der Egyptenaren, en het opvoere uit dit land, naar een goed en ruim land, naar een land, vloeiende van melk en honig, tot de plaats der Kanaanieten, en der Hethieten, en der Amorieten, en der Ferezieten, en der Hevieten, en der Jebusieten.
\par 9 En nu, zie, het geschrei der kinderen Israels is tot Mij gekomen; en ook heb Ik gezien de verdrukking, waarmede de Egyptenaars hen verdrukken.
\par 10 Zo kom nu, en Ik zal u tot Farao zenden, opdat gij Mijn volk (de kinderen Israels) uit Egypte voert.
\par 11 Toen zeide Mozes tot God: Wie ben ik, dat ik tot Farao zou gaan; en dat ik de kinderen Israels uit Egypte zou voeren?
\par 12 Hij dan zeide: Ik zal voorzeker met u zijn, en dit zal u een teken zijn, dat Ik u gezonden heb: wanneer gij dit volk uit Egypte geleid hebt, zult gijlieden God dienen op dezen berg.
\par 13 Toen zeide Mozes tot God: Zie, wanneer ik kom tot de kinderen Israels, en zeg tot hen: De God uwer vaderen heeft mij tot ulieden gezonden; en zij mij zeggen: Hoe is Zijn naam? wat zal ik tot hen zeggen?
\par 14 En God zeide tot Mozes: IK ZAL ZIJN,, Die IK ZIJN ZAL! Ook zeide Hij: Alzo zult gij tot de kinderen Israels zeggen: IK ZAL ZIJN heeft mij tot ulieden gezonden!
\par 15 Toen zeide God verder tot Mozes: Aldus zult gij tot de kinderen Israels zeggen: De HEERE, de God uwer vaderen, de God van Abraham, de God van Izak, en de God van Jakob, heeft mij tot ulieden gezonden; dat is Mijn Naam eeuwiglijk, en dat is Mijn gedachtenis van geslacht tot geslacht.
\par 16 Ga heen, en verzamel de oudsten van Israel, en zeg tot hen: De HEERE, de God uwer vaderen, is mij verschenen, de God van Abraham, Izak en Jakob, zeggende: Ik heb ulieden getrouwelijk bezocht, en hetgeen ulieden in Egypte is aangedaan;
\par 17 Daarom heb Ik gezegd: Ik zal ulieden uit de verdrukking van Egypte opvoeren, tot het land der Kanaanieten, en der Hethieten, en der Amorieten, en der Ferezieten, en der Hevieten, en der Jebusieten; tot het land, vloeiende van melk en honig.
\par 18 En zij zullen uw stem horen; en gij zult gaan, gij en de oudsten van Israel, tot den koning van Egypte, en gijlieden zult tot hem zeggen: De HEERE, de God der Hebreen, is ons ontmoet; zo laat ons nu toch gaan den weg van drie dagen in de woestijn, opdat wij den HEERE, onzen God, offeren!
\par 19 Doch Ik weet, dat de koning van Egypte ulieden niet zal laten gaan, ook niet door een sterke hand.
\par 20 Want Ik zal Mijn hand uitstrekken, en Egypte slaan met al Mijn wonderen, die Ik in het midden van hetzelve doen zal; daarna zal hij ulieden laten vertrekken.
\par 21 En Ik zal dit volk genade geven in de ogen der Egyptenaren; en het zal geschieden, wanneer gijlieden uitgaan zult, zo zult gij niet ledig uitgaan.
\par 22 Maar elke vrouw zal van haar naburin, en van de waardin haars huizes, eisen zilveren vaten, en gouden vaten, en klederen; die zult gijlieden op uw zonen, en op uw dochteren leggen, en gij zult Egypte beroven.

\chapter{4}

\par 1 Toen antwoordde Mozes, en zeide: Maar zie, zij zullen mij niet geloven, noch mijn stem horen; want zij zullen zeggen: De HEERE is u niet verschenen!
\par 2 En de HEERE zeide tot hem: Wat is er in uw hand? En hij zeide: Een staf.
\par 3 En Hij zeide: Werp hem ter aarde. En hij wierp hem ter aarde! Toen werd hij tot een slang; en Mozes vlood van haar.
\par 4 Toen zeide de HEERE tot Mozes: Strek uw hand uit, en grijp haar bij haar staart! Toen strekte hij zijn hand uit, en vatte haar, en zij werd tot een staf in zijn hand.
\par 5 Opdat zij geloven, dat u verschenen is de HEERE, de God hunner vaderen, de God van Abraham, de God van Izak, en de God van Jakob.
\par 6 En de HEERE zeide verder tot hem: Steek nu uw hand in uw boezem. En hij stak zijn hand in zijn boezem; daarna trok hij ze uit, en ziet, zijn hand was melaats, wit als sneeuw.
\par 7 En Hij zeide: Steek uw hand wederom in uw boezem. En hij stak zijn hand wederom in zijn boezem; daarna trok hij ze uit zijn boezem, en ziet, zij was weder als zijn ander vlees.
\par 8 En het zal geschieden, zo zij u niet geloven, noch naar de stem van het eerste teken horen, zo zullen zij de stem van het laatste teken geloven.
\par 9 En het zal geschieden, zo zij ook deze twee tekenen niet geloven, noch naar uw stem horen, zo neem van de wateren der rivier, en giet ze op het droge; zo zullen de wateren, die gij uit de rivier zult nemen, diezelve zullen tot bloed worden op het droge.
\par 10 Toen zeide Mozes tot den HEERE: Och Heere! ik ben geen man wel ter tale, noch van gisteren, noch van eergisteren, noch van toen af, toen Gij tot Uw knecht gesproken hebt; want ik ben zwaar van mond, en zwaar van tong.
\par 11 En de HEERE zeide tot hem: Wie heeft den mens den mond gemaakt, of wie heeft den stomme, of dove, of ziende, of blinde gemaakt? Ben Ik het niet, de HEERE?
\par 12 En nu ga henen, en Ik zal met uw mond zijn, en zal u leren, wat gij spreken zult.
\par 13 Doch hij zeide: Och, Heere! zend toch door de hand desgenen, dien Gij zoudt zenden.
\par 14 Toen ontstak de toorn des HEEREN over Mozes, en Hij zeide: is niet Aaron, de Leviet, uw broeder? Ik weet, dat hij zeer wel spreken zal, en ook, zie, hij zal uitgaan u tegemoet; wanneer hij u ziet, zo zal hij in zijn hart verblijd zijn.
\par 15 Gij dan zult tot hem spreken, en de woorden in zijn mond leggen; en Ik zal met uw mond, en met zijn mond zijn; en Ik zal ulieden leren, wat gij doen zult.
\par 16 En hij zal voor u tot het volk spreken; en het zal geschieden, dat hij u tot een mond zal zijn, en gij zult hem tot een god zijn.
\par 17 Neem dan dezen staf in uw hand, waarmede gij die tekenen doen zult.
\par 18 Toen ging Mozes heen, en keerde weder tot Jethro, zijn schoonvader, en zeide tot hem: Laat mij toch gaan, dat ik wederkere tot mijn broederen, die in Egypte zijn, en zie, of zij nog leven. Jethro dan zeide tot Mozes: Ga in vrede!
\par 19 Ook zeide de HEERE tot Mozes in Midian: Ga heen, keer weder in Egypte, want al de mannen zijn dood, die uw ziel zochten.
\par 20 Mozes dan nam zijn vrouw, en zijn zonen, en voerde hen op een ezel, en keerde weder in Egypteland; en Mozes nam den staf Gods in zijn hand.
\par 21 En de HEERE zeide tot Mozes: Terwijl gij heentrekt, om weder in Egypte te keren, zie toe, dat gij al de wonderen doet voor Farao, die Ik in uw hand gesteld heb; doch Ik zal zijn hart verstokken, dat hij het volk niet zal laten gaan.
\par 22 Dan zult gij tot Farao zeggen: Alzo zegt de HEERE: Mijn zoon, Mijn eerstgeborene, is Israel.
\par 23 En Ik heb tot u gezegd: Laat Mijn zoon trekken, dat hij Mij diene! maar gij hebt geweigerd hem te laten trekken; zie, Ik zal uw zoon, uw eerstgeborene doden!
\par 24 En het geschiedde op den weg, in de herberg, dat de HEERE hem tegenkwam, en zocht hem te doden.
\par 25 Toen nam Zippora een stenen mes en besneed de voorhuid haars zoons, en wierp die voor zijn voeten, en zeide: Voorwaar, gij zijt mij een bloedbruidegom!
\par 26 En Hij liet van hem af. Toen zeide zij: Bloedbruidegom! vanwege de besnijdenis.
\par 27 De HEERE zeide ook tot Aaron: Ga Mozes tegemoet in de woestijn. En hij ging, en ontmoette hem aan den berg Gods, en hij kuste hem.
\par 28 En Mozes gaf Aaron te kennen al de woorden des HEEREN, Die hem gezonden had, en al de tekenen, die Hij hem bevolen had.
\par 29 Toen ging Mozes en Aaron, en zij verzamelden al de oudsten der kinderen Israels.
\par 30 En Aaron sprak al de woorden, die de HEERE tot Mozes gesproken had; en hij deed de tekenen voor de ogen des volks.
\par 31 En het volk geloofde, en zij hoorden, dat de HEERE de kinderen Israels bezocht, en dat Hij hun verdrukking zag, en zij neigden hun hoofden, en aanbaden.

\chapter{5}

\par 1 En daarna gingen Mozes en Aaron heen, en zeiden tot Farao: Alzo zegt de HEERE, de God van Israel: Laat Mijn volk trekken, dat het Mij een feest houde in de woestijn!
\par 2 Maar Farao zeide: Wie is de HEERE, Wiens stem ik gehoorzamen zou, om Israel te laten trekken? Ik ken den HEERE niet, en ik zal ook Israel niet laten trekken.
\par 3 Zij dan zeiden: De God der Hebreen is ons ontmoet; zo laat ons toch heentrekken, den weg van drie dagen in de woestijn, en den HEERE, onzen God, offeren, dat Hij ons niet overkome met pestilentie, of met het zwaard.
\par 4 Toen zeide de koning van Egypte tot hen: Gij, Mozes en Aaron! waarom trekt gij het volk af van hun werken? Gaat heen tot uw lasten.
\par 5 Verder zeide Farao: Ziet, het volk des lands is alreeds te veel; en zoudt gijlieden hen doen rusten van hun lasten?
\par 6 Daarom beval Farao, ten zelfden dage, aan de aandrijvers onder het volk, en deszelfs ambtlieden, zeggende:
\par 7 Gij zult voortaan aan deze lieden geen stro meer geven, tot het maken der tichelstenen, als gisteren en eergisteren; laat hen zelven heengaan, en stro voor zichzelven verzamelen.
\par 8 En het getal der tichelstenen, die zij gisteren en eergisteren gemaakt hebben, zult gij hun opleggen; gij zult daarvan niet verminderen; want zij gaan ledig; daarom roepen zij, zeggende: Laat ons gaan, laat ons onzen God offeren!
\par 9 Men verzware den dienst over deze mannen, dat zij daaraan te doen hebben, en zich niet vergapen aan leugenachtige woorden.
\par 10 Toen gingen de aandrijvers des volks uit, en deszelfs ambtlieden, en spraken tot het volk, zeggende: Zo zegt Farao: Ik zal ulieden geen stro geven.
\par 11 Gaat gij zelve heen, haalt u stro, waar gij het vindt; doch van uw dienst zal niet verminderd worden.
\par 12 Toen verstrooide zich het volk in het ganse land van Egypte, dat het stoppelen verzamelde, voor stro.
\par 13 En de aandrijvers drongen aan, zeggende: Voleindigt uw werken, elk dagwerk op zijn dag, gelijk toen er stro was.
\par 14 En de ambtlieden der kinderen Israels, die Farao's aandrijvers over hen gesteld hadden, werden geslagen, en men zeide: Waarom hebt gijlieden uw gezette werk niet voleindigd, in het maken der tichelstenen, gelijk te voren, alzo ook gisteren en heden?
\par 15 Derhalve gingen de ambtlieden der kinderen Israels, en schreeuwden tot Farao, zeggende: Waarom doet gij uw knechten alzo?
\par 16 Aan uw knechten wordt geen stro gegeven, en zij zeggen tot ons: Maakt de tichelstenen; en ziet, uw knechten worden geslagen, doch de schuld is uws volks!
\par 17 Hij dan zeide: Gijlieden gaat ledig, ledig gaat gij; daarom zegt gij: Laat ons gaan, laat ons den HEERE offeren!
\par 18 Zo gaat nu heen, arbeidt; doch stro zal u niet gegeven worden; evenwel zult gij het getal der tichelstenen leveren.
\par 19 Toen zagen de ambtlieden der kinderen Israels, dat het kwalijk met hen stond, dewijl men zeide: Gij zult niet minderen van uw tichelstenen, van het dagwerk op zijn dag.
\par 20 En zij ontmoetten Mozes en Aaron, die tegen hen over stonden, toen zij van Farao uitgingen.
\par 21 En zeiden tot hen: De HEERE zie op u, en richte het, dewijl dat gij onzen reuk hebt stinkende gemaakt voor Farao, en voor zijn knechten, gevende een zwaard in hun handen, om ons te doden.
\par 22 Toen keerde Mozes weder tot den HEERE, en zeide: Heere! waarom hebt Gij dit volk kwaad gedaan, waarom hebt Gij mij nu gezonden?
\par 23 Want van toen af, dat ik tot Farao ben ingegaan, om in Uw naam te spreken, heeft hij dit volk kwaad gedaan; en Gij hebt Uw volk geenszins verlost.
\par 24 Toen zeide de HEERE tot Mozes: Nu zult gij zien, wat Ik aan Farao doen zal; want door een machtige hand zal hij hen laten trekken, ja, door een machtige hand zal hij hen uit zijn land drijven.

\chapter{6}

\par 1 Verder sprak God tot Mozes, en zeide tot hem: Ik ben de HEERE,
\par 2 En Ik ben aan Abraham, Izak, en Jakob verschenen, als God de Almachtige; doch met Mijn Naam HEERE ben Ik hun niet bekend geweest.
\par 3 En ook heb Ik Mijn verbond met hen opgericht, dat Ik hun geven zou het land Kanaan, het land hunner vreemdelingschappen, waarin zij vreemdelingen geweest zijn.
\par 4 En ook heb Ik gehoord het gekerm der kinderen Israels, die de Egyptenaars in dienstbaarheid houden, en Ik heb aan Mijn verbond gedacht.
\par 5 Derhalve zeg tot de kinderen Israels: Ik ben de HEERE! en Ik zal ulieden uitleiden van onder de lasten der Egyptenaren, en Ik zal u redden uit hun dienstbaarheid, en zal u verlossen door een uitgestrekten arm, en door grote gerichten;
\par 6 En Ik zal ulieden tot Mijn volk aannemen, en Ik zal u tot een God zijn; en gijlieden zult bekennen, dat Ik de HEERE uw God ben, Die u uitleide van onder de lasten der Egyptenaren.
\par 7 En Ik zal ulieden brengen in dat land, waarover Ik Mijn hand opgeheven heb, dat Ik het aan Abraham, Izak, en Jakob geven zou; en Ik zal het ulieden geven tot een erfdeel, Ik, de HEERE!
\par 8 En Mozes sprak alzo tot de kinderen Israels; doch zij hoorden naar Mozes niet, vanwege de benauwdheid des geestes, en vanwege de harde dienstbaarheid.
\par 9 Verder sprak de HEERE tot Mozes, zeggende:
\par 10 Ga heen, spreek tot Farao, den koning van Egypte, dat hij de kinderen Israels uit zijn land trekken late.
\par 11 Doch Mozes sprak voor den HEERE, zeggende: Zie, de kinderen Israels hebben naar mij niet gehoord; hoe zou mij dan Farao horen? daartoe ben ik onbesneden van lippen.
\par 12 Evenwel sprak de HEERE tot Mozes en tot Aaron, en gaf hun bevel aan de kinderen Israels, en aan Farao, den koning van Egypte, om de kinderen Israels uit Egypteland te leiden.
\par 13 Dit zijn de hoofden van ieder huis hunner vaderen: de zonen van Ruben, den eerstgeborene van Israel, zijn Hanoch en Pallu, Hezron en Charmi; dit zijn de huisgezinnen van Ruben.
\par 14 En de zonen van Simeon: Jemuel, en Jamin, en Ohad, en Jachin, en Zohar, en Saul, de zoon ener Kanaanietische; dit zijn de huisgezinnen van Simeon.
\par 15 Dit nu zijn de namen der zonen van Levi, naar hun geboorten: Gerson, en Kehath, en Merari. En de jaren des levens van Levi waren honderd zeven en dertig jaren.
\par 16 De zonen van Gerson: Libni en Simei, naar hun huisgezinnen.
\par 17 En de zonen van Kehath: Amram, en Jizhar, en Hebron, en Uzziel, en de jaren des levens van Kehath waren honderd drie en dertig jaren.
\par 18 En de zonen van Merari: Machli en Musi; dit zijn de huisgezinnen van Levi, naar hun geboorten.
\par 19 En Amram nam Jochebed, zijn moei, zich tot een huisvrouw, en zij baarde hem Aaron en Mozes; en de jaren des levens van Amram waren honderd zeven en dertig jaren.
\par 20 En de zonen van Jizhar: Korah, en Nefeg, en Zichri.
\par 21 En de zonen van Uzziel: Misael, en Elzafan, en Sithri.
\par 22 En Aaron nam zich tot een vrouw Eliseba, dochter van Amminadab, zuster van Nahesson; en zij baarde hem Nadab en Abihu, Eleazar en Ithamar.
\par 23 En de zonen van Korah waren: Assir, en Elkana, en Abiasaf; dat zijn de huisgezinnen der Korachieten.
\par 24 En Eleazar, de zoon van Aaron, nam voor zich een van de dochteren van Putiel tot een vrouw; en zij baarde hem Pinehas. Dit zijn de hoofden van de vaderen der Levieten, naar hun huisgezinnen.
\par 25 Dit is Aaron en Mozes, tot welke de HEERE zeide: Leidt de kinderen Israels uit Egypteland, naar hun heiren.
\par 26 Dezen zijn het, die tot Farao, den koning van Egypte, spraken, opdat zij de kinderen Israels uit Egypte leidden; dit is Mozes en Aaron.
\par 27 En het geschiedde te dien dage, als de HEERE tot Mozes sprak in Egypteland;
\par 28 Zo sprak de HEERE tot Mozes, zeggende: Ik ben de HEERE! spreek tot Farao, den koning van Egypte, alles, wat Ik tot u spreek.
\par 29 Toen zeide Mozes voor het aangezicht des HEEREN: Zie, ik ben onbesneden van lippen; hoe zal dan Farao naar mij horen?

\chapter{7}

\par 1 Toen zeide de HEERE tot Mozes: Zie, Ik heb u tot een God gezet over Farao; en Aaron, uw broeder, zal uw profeet zijn.
\par 2 Gij zult spreken alles, wat Ik u gebieden zal; en Aaron, uw broeder, zal tot Farao spreken, dat hij de kinderen Israels uit zijn land trekken laat.
\par 3 Doch Ik zal Farao's hart verharden; en Ik zal Mijn tekenen en Mijn wonderheden in Egypteland vermenigvuldigen.
\par 4 Farao nu zal naar ulieden niet horen, en Ik zal Mijn hand aan Egypte leggen, en voeren Mijn heiren, Mijn volk, de kinderen Israels, uit Egypteland, door grote gerichten.
\par 5 Dan zullen de Egyptenaars weten, dat Ik de HEERE ben, wanneer Ik Mijn hand over Egypte uitstrekke, en de kinderen Israels uit het midden van hen uitleide.
\par 6 Toen deed Mozes en Aaron, als hun de HEERE geboden had, alzo deden zij.
\par 7 En Mozes was tachtig jaar oud, en Aaron was drie en tachtig jaar oud, toen zij tot Farao spraken.
\par 8 En de HEERE sprak tot Mozes en tot Aaron, zeggende:
\par 9 Wanneer Farao tot ulieden spreken zal, zeggende: Doet een wonderteken voor ulieden; zo zult gij tot Aaron zeggen: Neem uw staf, en werp hem voor Farao's aangezicht neder; hij zal tot een draak worden.
\par 10 Toen ging Mozes en Aaron tot Farao henen in, en deden alzo, gelijk de HEERE geboden had; en Aaron wierp zijn staf neder voor Farao's aangezicht, en voor het aangezicht zijner knechten; en hij werd tot een draak.
\par 11 Farao nu riep ook de wijzen en de guichelaars; en de Egyptische tovenaars deden ook alzo met hun bezweringen.
\par 12 Want een iegelijk wierp zijn staf neder, en zij werden tot draken; maar Aarons staf verslond hun staven.
\par 13 Doch Farao's hart verstokte, zodat hij naar hen niet hoorde, gelijk de HEERE gesproken had.
\par 14 Toen zeide de HEERE tot Mozes: Farao's hart is zwaar; hij weigert het volk te laten trekken.
\par 15 Ga heen tot Farao in den morgenstond; zie, hij zal uitgaan naar het water toe, zo stel u tegen hem over aan den oever der rivier, en den staf, die in een slang is veranderd geweest, zult gij in uw hand nemen.
\par 16 En gij zult tot hem zeggen: de HEERE, de God der Hebreen, heeft mij tot u gezonden, zeggende: Laat Mijn volk trekken, dat het Mij diene in de woestijn; doch zie, gij hebt tot nu toe niet gehoord.
\par 17 Zo zegt de HEERE: Daaraan zult gij weten, dat Ik de HEERE ben; zie, ik zal met dezen staf, die in mijn hand is, op het water, dat in deze rivier is, slaan, en het zal in bloed veranderd worden.
\par 18 En de vis in de rivier zal sterven, zodat de rivier zal stinken; en de Egyptenaars zullen vermoeid worden, dat zij het water uit de rivier drinken mogen.
\par 19 Verder zeide de HEERE tot Mozes: zeg tot Aaron: Neem uw staf, en steek uw hand uit over de wateren der Egyptenaren, over hun stromen, over hun rivieren, en over hun poelen, en over alle vergadering hunner wateren, dat zij bloed worden; en er zij bloed in het ganse Egypteland, beide in houten en in stenen vaten.
\par 20 Mozes nu en Aaron deden alzo, gelijk de HEERE geboden had; en hij hief den staf op, en sloeg het water, dat in de rivier was, voor de ogen van Farao, en voor de ogen van zijn knechten; en al het water in de rivier werd in bloed veranderd.
\par 21 En de vis, die in de rivier was, stierf; en de rivier stonk, zodat de Egyptenaars het water uit de rivier niet drinken konden; en er was bloed in het ganse Egypteland.
\par 22 Doch de Egyptische tovenaars deden ook alzo met hun bezweringen; zodat Farao's hart verstokte, en hij hoorde naar hen niet, gelijk als de HEERE gesproken had.
\par 23 En Farao keerde zich om, en ging naar zijn huis; en hij zette zijn hart daar ook niet op.
\par 24 Doch alle Egyptenaars groeven rondom de rivier, om water te drinken; want zij konden van het water der rivier niet drinken.
\par 25 Alzo werden zeven dagen vervuld, nadat de HEERE de rivier geslagen had.

\chapter{8}

\par 1 Daarna zeide de HEERE tot Mozes: Ga in tot Farao, en zeg tot hem: Zo zegt de HEERE: Laat Mijn volk trekken, dat zij Mij dienen.
\par 2 En indien gij het weigert te laten trekken, zie, zo zal ik uw ganse landpale met vorsen slaan;
\par 3 Dat de rivier van vorsen zal krielen, die zullen opkomen, en in uw huis komen, en in uw slaapkamer, ja, op uw bed; ook in de huizen uwer knechten, en op uw volk, en in uw bakovens, en in uw baktroggen.
\par 4 En de vorsen zullen opkomen, op u, en op uw volk, en op al uw knechten.
\par 5 Verder zeide de HEERE tot Mozes: Zeg tot Aaron: Strek uw hand uit met uw staf, over de stromen, en over de rivieren, en over de poelen; en doe vorsen opkomen over Egypteland.
\par 6 En Aaron strekte zijn hand uit over de wateren van Egypte, en er kwamen vorsen op en bedekten Egypteland.
\par 7 Toen deden de tovenaars ook alzo, met hun bezweringen; en zij deden vorsen over Egypteland opkomen.
\par 8 En Farao riep Mozes en Aaron, en zeide: Bidt vuriglijk tot den HEERE, dat Hij de vorsen van mij en van mijn volk wegneme; zo zal ik het volk trekken laten, dat zij den HEERE offeren.
\par 9 Doch Mozes zeide tot Farao: Heb de eer boven mij! Tegen wanneer zal ik voor u, en voor uw knechten, en voor uw volk, vuriglijk bidden, om deze vorsen van u en van uw huizen te verdelgen, dat zij alleen in de rivier overblijven?
\par 10 Hij dan zeide: Tegen morgen. En hij zeide: Het zij naar uw woord, opdat gij weet, dat er niemand is, gelijk de HEERE, onze God.
\par 11 Zo zullen de vorsen van u, en van uw huizen, en van uw knechten, en van uw volk wijken; zij zullen alleen in de rivier overblijven.
\par 12 Toen ging Mozes en Aaron uit van Farao; en Mozes riep tot den HEERE, ter oorzake der vorsen, die Hij Farao had opgelegd.
\par 13 En de HEERE deed naar het woord van Mozes; en de vorsen stierven, uit de huizen, uit de voorzalen, en uit de velden.
\par 14 En zij vergaderden ze samen bij hopen, en het land stonk.
\par 15 Toen nu Farao zag, dat er verademing was, verzwaarde hij zijn hart, dat hij naar hen niet hoorde, gelijk als de HEERE gesproken had.
\par 16 Verder zeide de HEERE tot Mozes: Zeg tot Aaron: Strek uw staf uit, en sla het stof der aarde, dat het tot luizen worde, in het ganse Egypteland.
\par 17 En zij deden alzo; want Aaron strekte zijn hand uit met zijn staf, en sloeg het stof der aarde, en er werden vele luizen aan de mensen, en aan het vee; al het stof der aarde werd luizen, in het ganse Egypteland.
\par 18 De tovenaars deden ook alzo met hun bezweringen, opdat zij luizen voortbrachten; doch zij konden niet; zo waren de luizen aan de mensen, en aan het vee.
\par 19 Toen zeiden de tovenaars tot Farao: Dit is Gods vinger! Doch Farao's hart verstijfde, zodat hij naar hen niet hoorde, gelijk de HEERE gesproken had.
\par 20 Verder zeide de HEERE tot Mozes: Maak u morgen vroeg op, en stel u voor Farao's aangezicht; zie, hij zal aan het water uitgaan, en zeg tot hem: Zo zegt de HEERE: Laat Mijn volk trekken, dat zij Mij dienen;
\par 21 Want zo gij Mijn volk niet laat trekken, zie, zo zal Ik een vermenging van ongedierte zenden op u, en op uw knechten, en op uw volk, en in uw huizen; alzo dat de huizen der Egyptenaren met deze vermenging zullen vervuld worden, en ook het aardrijk, waarop zij zijn.
\par 22 En Ik zal te dien dage het land Gosen, waarin Mijn volk woont, afzonderen, dat daar geen vermenging van ongedierte zij, opdat gij weet, dat Ik, de HEERE, in het midden dezes lands ben.
\par 23 En Ik zal een verlossing zetten tussen Mijn volk en tussen uw volk; tegen morgen zal dit teken geschieden!
\par 24 En de HEERE deed alzo; en er kwam een zware vermenging van ongedierte in het huis van Farao, en in de huizen van zijn knechten, en over het ganse Egypteland; het land werd verdorven van deze vermenging.
\par 25 Toen riep Farao Mozes en Aaron, en zeide: Gaat heen, en offert uwen God in dit land.
\par 26 Mozes dan zeide: Het is niet recht, dat men alzo doe; want wij zouden der Egyptenaren gruwel den HEERE, onzen God, mogen offeren; zie, indien wij der Egyptenaren gruwel voor hun ogen offerden, zouden zij ons niet stenigen?
\par 27 Laat ons den weg van drie dagen in de woestijn gaan, dat wij den HEERE onzen God offeren, gelijk Hij tot ons zeggen zal.
\par 28 Toen zeide Farao: Ik zal u trekken laten, dat gijlieden den HEERE, uwen God, offert in de woestijn; alleen, dat gijlieden in het gaan geenszins te verre trekt! Bidt vuriglijk voor mij.
\par 29 Mozes nu zeide: Zie, ik ga van u, en zal tot den HEERE vuriglijk bidden, dat deze vermenging van ongedierte van Farao, van zijn knechten, en van zijn volk morgen wegwijke! Alleen, dat Farao niet meer bedriegelijk handele, dit volk niet latende gaan, om den HEERE te offeren.
\par 30 Toen ging Mozes uit van Farao, en bad vuriglijk tot den HEERE.
\par 31 En de HEERE deed naar het woord van Mozes, en de vermenging van ongedierte week van Farao, van zijn knechten, en van zijn volk; er bleef niet een over.
\par 32 Doch Farao verzwaarde zijn hart ook op ditmaal, en hij liet het volk niet trekken.

\chapter{9}

\par 1 Daarna zeide de HEERE tot Mozes: Ga in tot Farao, en spreek tot hem: Alzo zegt de HEERE, de God der Hebreen: Laat Mijn volk trekken, dat het Mij diene.
\par 2 Want zo gij hen weigert te laten trekken, en gij hen nog met geweld ophoudt,
\par 3 Zie, de hand des HEEREN zal zijn over uw vee, dat in het veld is, over de paarden, over de ezelen, over de kemelen, over de runderen, en over het klein vee, door een zeer zware pestilentie.
\par 4 En de HEERE zal een afzondering maken tussen het vee der Israelieten, en tussen het vee der Egyptenaren, dat er niets sterve van al wat van de kinderen Israels is.
\par 5 En de HEERE bestemde een zekeren tijd, zeggende: Morgen zal de HEERE deze zaak in dit land doen.
\par 6 En de HEERE deed deze zaak des anderen daags; en al het vee der Egyptenaren stierf; maar van het vee der kinderen Israels stierf niet een.
\par 7 En Farao zond er heen, en ziet, van het vee van Israel was niet tot een toe gestorven. Doch het hart van Farao werd verzwaard, en hij liet het volk niet trekken.
\par 8 Toen zeide de HEERE tot Mozes en tot Aaron: Neemt gijlieden uw vuisten vol as uit den oven; en Mozes strooie die naar den hemel voor de ogen van Farao.
\par 9 En zij zal tot klein stof worden over het ganse Egypteland; en zij zal aan de mensen, en aan het vee worden tot zweren, uitbrekende met blaren, in het ganse Egypteland.
\par 10 En zij namen as uit den oven, en stonden voor Farao's aangezicht; en Mozes strooide die naar den hemel; toen werden er zweren, uitbrekende met blaren, aan de mensen en aan het vee;
\par 11 Alzo dat de tovenaars voor Mozes niet staan konden, vanwege de zweren; want aan de tovenaars waren zweren, en aan al de Egyptenaren.
\par 12 Doch de HEERE verstokte Farao's hart, dat hij naar hen niet hoorde, gelijk de HEERE tot Mozes gesproken had.
\par 13 Toen zeide de HEERE tot Mozes: Maak u morgen vroeg op, en stel u voor Farao's aangezicht, en zeg tot hem: Zo zegt de HEERE, de God der Hebreen: Laat Mijn volk trekken, dat zij Mij dienen.
\par 14 Want ditmaal zal Ik al Mijn plagen in uw hart zenden, en over uw knechten, en over uw volk, opdat gij weet, dat er niemand is gelijk Ik, op de ganse aarde.
\par 15 Want nu heb Ik Mijn hand uitgestrekt, opdat Ik u en uw volk met de pestilentie zou slaan, en dat gij van de aarde zoudt verdelgd worden.
\par 16 Maar waarlijk, daarom heb Ik u verwekt, opdat Ik Mijn kracht aan u betoonde, en opdat men Mijn Naam vertelle op de ganse aarde.
\par 17 Verheft gij uzelven nog tegen Mijn volk, dat gij het niet wilt laten trekken?
\par 18 Zie, Ik zal morgen omtrent dezen tijd een zeer zwaren hagel doen regenen, desgelijks in Egypte niet geweest is van dien dag af, dat het gegrond is, tot nu toe.
\par 19 En nu, zend heen, vergader uw vee, en alles wat gij op het veld hebt; alle mens en gedierte, dat op het veld gevonden zal worden, en niet in huis verzameld zal zijn, als deze hagel op hen vallen zal, zo zullen zij sterven.
\par 20 Wie onder Farao's knechten des HEEREN woord vreesde, die deed zijn knechten en zijn vee in de huizen vlieden;
\par 21 Doch die zijn hart niet zette tot des HEEREN woord, die liet zijn knechten en zijn vee op het veld.
\par 22 Toen zeide de HEERE tot Mozes: Strek uw hand uit naar den hemel, en er zal hagel zijn in het ganse Egypteland; over de mensen, en over het vee, en over al het kruid des velds in Egypteland.
\par 23 Toen strekte Mozes zijn staf naar den hemel; en de HEERE gaf donder en hagel, en het vuur schoot naar de aarde; en de HEERE liet hagel regenen over Egypteland.
\par 24 En er was hagel, en vuur in het midden des hagels vervangen; hij was zeer zwaar; desgelijks is in het ganse Egypteland nooit geweest, sedert het tot een volk geweest is.
\par 25 En de hagel sloeg, in het ganse Egypteland, alles wat op het veld was, van de mensen af tot de beesten toe; ook sloeg de hagel al het kruid des velds, en verbrak al het geboomte des velds.
\par 26 Alleen in het land Gosen, waar de kinderen Israels waren, daar was geen hagel.
\par 27 Toen schikte Farao heen, en hij riep Mozes en Aaron, en zeide tot hen: Ik heb mij ditmaal verzondigd; de HEERE is rechtvaardig; ik daarentegen en mijn volk zijn goddelozen!
\par 28 Bidt vuriglijk tot den HEERE (want het is genoeg), dat geen donder Gods noch hagel meer zij; dan zal ik ulieden trekken laten, en gij zult niet langer blijven.
\par 29 Toen zeide Mozes tot hem: Wanneer ik ter stad uitgegaan zal zijn, zo zal ik mijn handen uitbreiden voor den HEERE; de donder zal ophouden, en de hagel zal niet meer zijn; opdat gij weet, dat de aarde des HEEREN is!
\par 30 Nochtans u en uw knechten aangaande, weet ik, dat gijlieden voor het aangezicht van den HEERE God nog niet vrezen zult.
\par 31 Het vlas nu, en de gerst werd geslagen; want de gerst was in de aar, en het vlas was in den halm.
\par 32 Maar de tarwe en de spelt werden niet geslagen; want zij waren bedekt.
\par 33 Zo ging Mozes van Farao ter stad uit, en breidde zijn handen tot den HEERE; de donder en de hagel hielden op, en de regen werd niet meer uitgegoten op de aarde.
\par 34 Toen Farao zag, dat de regen en hagel, en de donder ophielden, zo verzondigde hij zich verder, en hij verzwaarde zijn hart, hij en zijn knechten.
\par 35 Alzo werd Farao's hart verstokt, dat hij de kinderen Israels niet trekken liet, gelijk als de HEERE gesproken had door Mozes.

\chapter{10}

\par 1 Daarna zeide de HEERE tot Mozes: Ga in tot Farao; want Ik heb zijn hart verzwaard, ook het hart zijner knechten, opdat Ik deze Mijn tekenen in het midden van hen zette;
\par 2 En opdat gij voor de oren uwer kinderen en uwer kindskinderen moogt vertellen, wat Ik in Egypte uitgericht heb, en Mijn tekenen, die Ik onder hen gesteld heb; opdat gijlieden weet, dat Ik de HEERE ben.
\par 3 Zo gingen Mozes en Aaron tot Farao, en zeiden tot hem: Zo zegt de HEERE, de God der Hebreen: Hoe lang weigert gij u voor Mijn aangezicht te verootmoedigen? Laat Mijn volk trekken, dat zij Mij dienen.
\par 4 Want indien gij weigert Mijn volk te laten trekken, zie, zo zal Ik morgen sprinkhanen in uw landpale brengen.
\par 5 En zij zullen het gezicht des lands bedekken, alzo dat men de aarde niet zal kunnen zien; en zij zullen afeten het overige van hetgeen ontkomen is, hetgeen ulieden overgebleven was van den hagel; zij zullen ook al het geboomte afeten, dat ulieden uit het veld voortkomt.
\par 6 En zij zullen vervullen uw huizen, en de huizen van al uw knechten, en de huizen van alle Egyptenaren; dewelke uw vaders, noch de vaderen uwer vaders gezien hebben, van dien dag af, dat zij op den aardbodem geweest zijn, tot op dezen dag. En hij keerde zich om, en ging uit van Farao.
\par 7 En de knechten van Farao zeiden tot hem: Hoe lang zal ons deze tot een strik zijn, laat de mannen trekken, dat zij den HEERE hun God dienen! weet gij nog niet, dat Egypte verloren is?
\par 8 Toen werden Mozes en Aaron weder tot Farao gebracht, en hij zeide tot hen: Gaat henen, dient den HEERE, uw God! wie en wie zijn zij, die gaan zullen?
\par 9 En Mozes zeide: Wij zullen gaan met onze jonge en met onze oude lieden; met onze zonen en met onze dochteren, met onze schapen en met onze runderen zullen wij gaan; want wij hebben een feest des HEEREN.
\par 10 Toen zeide hij tot hen: De HEERE zij alzo met ulieden, gelijk ik u en uw kleine kinderen zal trekken laten: ziet toe, want er is kwaad voor ulieder aangezicht!
\par 11 Niet alzo gij, mannen, gaat nu heen, en dient den HEERE; want dat hebt gijlieden verzocht! En men dreef hen uit van Farao's aangezicht.
\par 12 Toen zeide de HEERE tot Mozes: Strek uw hand uit over Egypteland, om de sprinkhanen, dat zij opkomen over Egypteland, en al het kruid des lands opeten, al wat de hagel heeft over gelaten.
\par 13 Toen strekte Mozes zijn staf over Egypteland, en de HEERE bracht een oostenwind in dat land, dien gehele dag en dien gansen nacht; het geschiedde des morgens, dat de oostenwind de sprinkhanen opbracht.
\par 14 En de sprinkhanen kwamen op over het ganse Egypteland, en lieten zich neder aan al de palen der Egyptenaren, zeer zwaar; voor dezen zijn dergelijke sprinkhanen, als deze, nooit geweest, en na dezen zullen er zulke niet wezen;
\par 15 Want zij bedekten het gezicht des gansen lands, alzo dat het land verduisterd werd; en zij aten al het kruid des lands op, en al de vruchten der bomen, die de hagel had over gelaten; en er bleef niets groens aan de bomen, noch aan de kruiden des velds, in het ganse Egypteland.
\par 16 Toen haastte Farao, om Mozes en Aaron te roepen, en zeide: Ik heb gezondigd tegen den HEERE, uw God, en tegen ulieden.
\par 17 En nu vergeeft mij toch mijn zonde alleen ditmaal, en bidt vuriglijk tot den HEERE, uw God, dat Hij slechts dezen dood van mij wegneme.
\par 18 En hij ging uit van Farao, en bad vuriglijk tot den HEERE.
\par 19 Toen keerde de HEERE een zeer sterken westenwind, die hief de sprinkhanen op, en wierp ze in de Schelfzee; er bleef niet een sprinkhaan over in al de landpalen van Egypte.
\par 20 Doch de HEERE verstokte Farao's hart, dat hij de kinderen Israels niet liet trekken.
\par 21 Toen zeide de HEERE tot Mozes: Strek uw hand uit naar den hemel, en er zal duisternis komen over Egypteland, dat men de duisternis tasten zal.
\par 22 Als Mozes zijn hand uitstrekte naar den hemel, werd er een dikke duisternis in het ganse Egypteland, drie dagen.
\par 23 Zij zagen de een den ander niet; er stond ook niemand op van zijn plaats, in drie dagen; maar bij al de kinderen Israels was het licht in hun woningen.
\par 24 Toen riep Farao Mozes, en zeide: Gaat heen, dient den HEERE! alleen uw schapen en uw runderen zullen vast blijven; ook zullen uw kinderkens met u gaan.
\par 25 Doch Mozes zeide: Ook zult gij slachtofferen en brandofferen in onze handen geven, die wij den HEERE, onzen God, doen mogen;
\par 26 En ons vee zal ook met ons gaan, er zal niet een klauw achterblijven; want van hetzelve zullen wij nemen, om den HEERE, onzen God, te dienen; want wij weten niet, waarmede wij den HEERE, onzen God, dienen zullen, totdat wij daar komen.
\par 27 Doch de HEERE verhardde Farao's hart; en hij wilde hen niet laten trekken.
\par 28 Maar Farao zeide tot hem: Ga van mij! wacht u, dat gij niet meer mijn aangezicht ziet; want op welken dag gij mijn aangezicht zult zien, zult gij sterven!
\par 29 Mozes nu zeide: Gij hebt recht gesproken; ik zal niet meer uw aangezicht zien!

\chapter{11}

\par 1 (Want de HEERE had tot Mozes gesproken: Ik zal nog een plaag over Farao, en over Egypte brengen, daarna zal hij ulieden van hier laten trekken; als hij u geheellijk zal laten trekken, zo zal hij u haastelijk van hier uitdrijven.
\par 2 Spreek nu voor de oren des volks, dat ieder man van zijn naaste, en iedere vrouw van haar naaste zilveren vaten en gouden vaten eise.
\par 3 En de HEERE gaf het volk genade in de ogen der Egyptenaren; ook was de man Mozes zeer groot in Egypteland voor de ogen van Farao's knechten, en voor de ogen des volks.)
\par 4 Verder zeide Mozes: Zo heeft de HEERE gezegd: Omtrent middernacht zal Ik uitgaan door het midden van Egypte;
\par 5 En alle eerstgeborenen in Egypteland zullen sterven, van Farao's eerstgeborene af, die op zijn troon zitten zou, tot den eerstgeborene der dienstmaagd, die achter den molen is, en alle eerstgeborenen van het vee.
\par 6 En er zal een groot geschrei zijn in het ganse Egypteland, desgelijke nooit geweest is, en desgelijke niet meer wezen zal.
\par 7 Maar bij alle kinderen Israels zal niet een hond zijn tong verroeren, van de mensen af tot de beesten toe; opdat gijlieden weet, dat de HEERE tussen de Egyptenaren en tussen de Israelieten een afzondering maakt.
\par 8 Dan zullen al deze uw knechten tot mij afkomen, en zich voor mij neigen, zeggende: Trek uit, gij en al het volk, dat uw voetstappen volgt; en daarna zal ik uitgaan. En hij ging uit van Farao in hitte des toorns.
\par 9 De HEERE dan had tot Mozes gesproken: Farao zal naar ulieden niet horen, opdat Mijn wonderen in Egypteland vermenigvuldigd worden.
\par 10 En Mozes en Aaron hebben al deze wonderen gedaan voor Farao's aangezicht; doch de HEERE verhardde Farao's hart, dat hij de kinderen Israels uit zijn land niet trekken liet.

\chapter{12}

\par 1 De HEERE nu had tot Mozes en tot Aaron in Egypteland gesproken, zeggende:
\par 2 Deze zelfde maand zal ulieden het hoofd der maanden zijn; zij zal u de eerste van de maanden des jaars zijn.
\par 3 Spreekt tot de ganse vergadering van Israel, zeggende: Aan den tienden dezer maand neme een iegelijk een lam, naar de huizen der vaderen, een lam voor een huis.
\par 4 Maar indien een huis te klein is voor een lam, zo neme hij het en zijn nabuur, de naaste aan zijn huis, naar het getal der zielen, een iegelijk naar dat hij eten kan; gij zult rekening maken naar het lam.
\par 5 Gij zult een volkomen lam hebben, een manneken, een jaar oud; van de schapen of van de geitenbokken zult gij het nemen.
\par 6 En gij zult het in bewaring hebben tot den veertienden dag dezer maand; en de ganse gemeente der vergadering van Israel zal het slachten tussen twee avonden.
\par 7 En zij zullen van het bloed nemen, en strijken het aan de beide zijposten, en aan den bovendorpel, aan de huizen, in welke zij het eten zullen.
\par 8 En zij zullen het vlees eten in denzelfden nacht, aan het vuur gebraden, met ongezuurde broden; zij zullen het met bittere saus eten.
\par 9 Gij zult daarvan niet rauw eten, ook geenszins in water gezoden; maar aan het vuur gebraden, zijn hoofd met zijn schenkelen en met zijn ingewand.
\par 10 Gij zult daarvan ook niet laten overblijven tot den morgen; maar hetgeen daarvan overblijft tot den morgen, zult gij met vuur verbranden.
\par 11 Aldus nu zult gij het eten: uw lenden zullen opgeschort zijn, uw schoenen aan uw voeten, en uw staf in uw hand; en gij zult het met haast eten; het is des HEEREN pascha.
\par 12 Want Ik zal in dezen nacht door Egypteland gaan, en alle eerstgeborenen in Egypteland slaan, van de mensen af tot de beesten toe; en Ik zal gerichten oefenen aan al de goden der Egyptenaren, Ik, de HEERE!
\par 13 En dat bloed zal ulieden tot een teken zijn aan de huizen, waarin gij zijt; wanneer Ik het bloed zie, zal Ik ulieden voorbijgaan; en er zal geen plaag onder ulieden ten verderve zijn, wanneer Ik Egypteland slaan zal.
\par 14 En deze dag zal ulieden wezen ter gedachtenis, en gij zult hem den HEERE tot een feest vieren; gij zult hem vieren onder uw geslachten tot een eeuwige inzetting.
\par 15 Zeven dagen zult gijlieden ongezuurde broden eten; maar aan den eersten dag zult gij het zuurdeeg wegdoen uit uw huizen; want wie het gedesemde eet, van den eersten dag af tot op den zevenden dag, diezelve ziel zal uitgeroeid worden uit Israel.
\par 16 En op den eersten dag zal er een heilige verzameling zijn; ook zult gij een heilige verzameling hebben op den zevenden dag; er zal geen werk op denzelven gedaan worden; maar wat van iedere ziel gegeten zal worden, datzelve alleen mag van ulieden toegemaakt worden.
\par 17 Zo onderhoudt dan de ongezuurde broden, dewijl Ik even aan denzelfden dag ulieder heiren uit Egypteland geleid zal hebben; daarom zult gij dezen dag houden, onder uw geslachten, tot een eeuwige inzetting.
\par 18 In de eerste maand, aan den veertienden dag der maand, in den avond, zult gij ongezuurde broden eten, tot den een en twintigsten dag der maand, in den avond.
\par 19 Dat er zeven dagen lang geen zuurdesem in uw huizen gevonden worde, want al wie het gedesemde eten zal, dezelve ziel zal uit de vergadering van Israel uitgeroeid worden, hij zij een vreemdeling of een ingeborene des lands.
\par 20 Gij zult niets eten, dat gedesemd is; in al uw woningen zult gij ongezuurde broden eten.
\par 21 Mozes dan riep al de oudsten van Israel, en zeide tot hen: Leest uit, en neemt u lammeren voor uw huisgezinnen, en slacht het pascha.
\par 22 Neemt dan een bundelken hysop, en doopt het in het bloed, dat in een bekken zal wezen; en strijkt aan den bovendorpel, en aan de beide zijposten van dat bloed, hetwelk in het bekken zijn zal; doch u aangaande, niemand zal uitgaan uit de deur van zijn huis, tot aan den morgen.
\par 23 Want de HEERE zal doorgaan, om de Egyptenaren te slaan; doch wanneer Hij het bloed zien zal aan den bovendorpel en aan de twee zijposten, zo zal de HEERE de deur voorbijgaan, en den verderver niet toelaten in uw huizen te komen om te slaan.
\par 24 Onderhoudt dan deze zaak, tot een inzetting voor u en voor uw kinderen, tot in eeuwigheid.
\par 25 En het zal geschieden, als gij in dat land komt, dat u de HEERE geven zal, gelijk Hij gesproken heeft, zo zult gij dezen dienst onderhouden.
\par 26 En het zal geschieden, wanneer uw kinderen tot u zullen zeggen: Wat hebt gij daar voor een dienst?
\par 27 Zo zult gij zeggen: Dit is den HEERE een paasoffer, Die voor de huizen der kinderen Israels voorbijging in Egypte, toen Hij de Egyptenaars sloeg, en onze huizen bevrijdde! Toen boog zich het volk en neigde zich.
\par 28 En de kinderen Israels gingen en deden het, gelijk als de HEERE Mozes en Aaron geboden had, alzo deden zij.
\par 29 En het geschiedde ter middernacht, dat de HEERE al de eerstgeborenen in Egypteland sloeg, van den eerstgeborene van Farao af, die op zijn troon zitten zou, tot op den eerstgeborene van den gevangene, die in het gevangenhuis was, en alle eerstgeborenen der beesten.
\par 30 En Farao stond op bij nacht, hij en al zijn knechten, en al de Egyptenaars; en er was een groot geschrei in Egypte; want er was geen huis, waarin niet een dode was.
\par 31 Toen riep hij Mozes en Aaron in den nacht, en zeide: Maakt u op, trekt uit het midden van mijn volk, zo gijlieden als de kinderen van Israel; en gaat heen, dient den HEERE, gelijk gijlieden gesproken hebt.
\par 32 Neemt ook met u uw schapen en uw runderen, zoals gijlieden gesproken hebt, en gaat heen, en zegent mij ook.
\par 33 En de Egyptenaars hielden sterk aan bij het volk, haastende, om die uit het land te drijven; want zij zeiden: Wij zijn allen dood!
\par 34 En het volk nam zijn deeg op, eer het gedesemd was, hun deegklompen, gebonden in hun klederen, op hun schouderen.
\par 35 De kinderen Israels nu hadden gedaan naar het woord van Mozes, en hadden van de Egyptenaren geeist zilveren vaten, en gouden vaten, en klederen.
\par 36 Daartoe had de HEERE het volk genade gegeven in de ogen der Egyptenaren, dat zij hun hun begeerte deden; en zij beroofden de Egyptenaren.
\par 37 Alzo reisden de kinderen Israels uit van Rameses naar Sukkoth, omtrent zeshonderd duizend te voet, mannen alleen, behalve de kinderkens.
\par 38 En veel vermengd volk trok ook met hen op, en schapen, en runderen, gans veel vee.
\par 39 En zij bakten van het deeg, dat zij uit Egypte gebracht hadden, ongezuurde koeken; want het was niet gedesemd; overmits zij uit Egypte uitgedreven werden, zodat zij niet vertoeven konden, noch ook tering voor zich bereiden.
\par 40 De tijd nu der woning, dien de kinderen Israels in Egypte gewoond hebben, is vierhonderd jaren en dertig jaren.
\par 41 En het geschiedde ten einde van de vierhonderd en dertig jaren, zo is het even op denzelfden dag geschied, dat al de heiren des HEEREN uit Egypteland gegaan zijn.
\par 42 Dezen nacht zal men den HEERE op het vlijtigst houden, omdat Hij hen uit Egypteland geleid heeft; deze is de nacht des HEEREN, die op het vlijtigst moet gehouden worden, van al de kinderen Israels, onder hun geslachten.
\par 43 Voorts zeide de HEERE tot Mozes en Aaron: Dit is de inzetting van het pascha: geen zoon eens vreemdelings zal daarvan eten.
\par 44 Doch alle knecht van iedereen, die voor geld gekocht is, nadat gij hem zult besneden hebben, dan zal hij daarvan eten.
\par 45 Geen uitlander noch huurling zal er van eten.
\par 46 In een huis zal het gegeten worden; gij zult van het vlees niet buiten uit het huis dragen, en gij zult geen been daaraan breken.
\par 47 De ganse vergadering van Israel zal het doen.
\par 48 Als nu een vreemdeling bij u verkeert, en den HEERE het pascha houden zal, dat alles, wat mannelijk is, bij hem besneden worde, en dan kome hij daartoe, om dat te houden, en hij zal wezen als een ingeborene des lands; maar geen onbesnedene zal daarvan eten.
\par 49 Enerlei wet zij voor den ingeborene, en den vreemdeling, die als vreemdeling in het midden van u verkeert.
\par 50 En alle kinderen Israels deden het; gelijk als de HEERE Mozes en Aaron geboden had, alzo deden zij.
\par 51 En het geschiedde even tenzelfden dage, dat de HEERE de kinderen Israels uit Egypteland leidde, naar hun heiren.

\chapter{13}

\par 1 Toen sprak de HEERE tot Mozes, zeggende:
\par 2 Heilig Mij alle eerstgeborenen; wat enige baarmoeder opent onder de kinderen Israels, van mensen en van beesten, dat is Mijn.
\par 3 Verder zeide Mozes tot het volk: Gedenkt aan dezen zelfden dag, op welken gijlieden uit Egypte, uit het diensthuis, gegaan zijt; want de HEERE heeft u door een sterke hand van hier uitgevoerd; daarom zal het gedesemde niet gegeten worden.
\par 4 Heden gaat gijlieden uit, in de maand Abib.
\par 5 En het zal geschieden, als u de HEERE zal gebracht hebben in het land der Kanaanieten, en der Hethieten, en der Amorieten, en der Hevieten, en der Jebusieten, hetwelk Hij uw vaderen gezworen heeft u te geven, een land vloeiende van melk en honig; zo zult gij dezen dienst houden in deze maand.
\par 6 Zeven dagen zult gij ongezuurde broden eten, en aan den zevenden dag zal den HEERE een feest zijn.
\par 7 Zeven dagen zullen ongezuurde broden gegeten worden, en het gedesemde zal bij u niet gezien worden, ja, er zal geen zuurdeeg bij u gezien worden, in al uw palen.
\par 8 En gij zult uw zoon te kennen geven te dienzelven dage, zeggende: Dit is om hetgeen de HEERE mij gedaan heeft, toen ik uit Egypte uittoog.
\par 9 En het zal u zijn tot een teken op uw hand, en tot een gedachtenis tussen uw ogen, opdat de wet des HEEREN in uw mond zij, omdat u de HEERE door een sterke hand uit Egypte uitgevoerd heeft.
\par 10 Daarom onderhoudt deze inzetting ter bestemder tijd, van jaar tot jaar.
\par 11 Het zal ook geschieden, wanneer u de HEERE in het land der Kanaanieten zal gebracht hebben, gelijk Hij u en uw vaderen gezworen heeft, en Hij het u zal gegeven hebben;
\par 12 Zo zult gij tot den HEERE doen overgaan alles, wat de baarmoeder opent; ook alles, wat de baarmoeder opent van de vrucht der beesten, die gij hebben zult; de mannetjes zullen des HEEREN zijn.
\par 13 Doch al wat de baarmoeder der ezelin opent, zult gij lossen met een lam; wanneer gij het nu niet lost, zo zult gij het den nek breken; maar alle eerstgeborenen des mensen onder uw zonen zult gij lossen.
\par 14 Wanneer het geschieden zal, dat uw zoon u morgen zal vragen, zeggende: Wat is dat, zo zult gij tot hem zeggen: De HEERE heeft ons door een sterke hand uit Egypte, uit het diensthuis, uitgevoerd.
\par 15 Want het geschiedde, toen Farao zich verhardde ons te laten trekken, zo doodde de HEERE alle eerstgeborenen in Egypteland, van des mensen eerstgeborene af, tot den eerstgeborene der beesten; daarom offer ik den HEERE de mannetjes van alles, wat de baarmoeder opent; doch alle eerstgeborenen mijner zonen los ik.
\par 16 En het zal tot een teken zijn op uw hand, en tot voorhoofdspanselen tussen uw ogen; want de HEERE heeft door een sterke hand ons uit Egypte uitgevoerd.
\par 17 En het is geschied, toen Farao het volk had laten trekken, zo leidde hen God niet op den weg van het land der Filistijnen, hoewel die nader was; want God zeide: Dat het den volke niet rouwe, als zij den strijd zien zouden, en wederkeren naar Egypte.
\par 18 Maar God leidde het volk om, langs den weg van de woestijn der Schelfzee. De kinderen Israels nu togen bij vijven uit Egypteland.
\par 19 En Mozes nam de beenderen van Jozef met zich; want hij had met een zwaren eed de kinderen Israels bezworen, zeggende: God zal ulieden voorzeker bezoeken; voert dan mijn beenderen met ulieden op van hier!
\par 20 Alzo reisden zij uit Sukkoth; en zij legerden zich in Etham, aan het einde der woestijn.
\par 21 En de HEERE toog voor hun aangezicht, des daags in een wolkkolom, dat Hij hen op den weg leidde, en des nachts in een vuurkolom, dat Hij hen lichtte, om voort te gaan dag en nacht.
\par 22 Hij nam de wolkkolom des daags, noch de vuurkolom des nachts niet weg van het aangezicht des volks.

\chapter{14}

\par 1 Toen sprak de HEERE tot Mozes, zeggende:
\par 2 Spreek tot de kinderen Israels, dat zij wederkeren, en zich legeren voor Pi-hachiroth, tussen Migdol en tussen de zee, voor Baal-zefon; daar tegenover zult gij u legeren aan de zee.
\par 3 Farao dan zal zeggen van de kinderen Israels: Zij zijn verward in het land; die woestijn heeft hen besloten.
\par 4 En Ik zal Farao's hart verstokken, dat hij hen najage; en Ik zal aan Farao en aan al zijn heir verheerlijkt worden, alzo dat de Egyptenaars zullen weten, dat Ik de HEERE ben. En zij deden alzo.
\par 5 Toen nu den koning van Egypte werd geboodschapt, dat het volk vluchtte, zo is het hart van Farao en van zijn knechten veranderd tegen het volk, en zij zeiden: Waarom hebben wij dat gedaan, dat wij Israel hebben laten trekken, dat zij ons niet dienden?
\par 6 En hij spande zijn wagen aan, en nam zijn volk met zich.
\par 7 En hij nam zeshonderd uitgelezene wagens, ja, al de wagens van Egypte, en de hoofdlieden over die allen.
\par 8 Want de HEERE verstokte het hart van Farao, den koning van Egypte, dat hij de kinderen Israels najaagde; doch de kinderen Israels waren door een hoge hand uitgegaan.
\par 9 En de Egyptenaars jaagden hen na, en achterhaalden hen, daar zij zich gelegerd hadden aan de zee; al de paarden, de wagens van Farao en zijn ruiters, en zijn heir; nevens Pi-hachiroth, voor Baal-zefon.
\par 10 Als Farao nabij gekomen was, zo hieven de kinderen Israels hun ogen op, en ziet, de Egyptenaars togen achter hen; en zij vreesden zeer; toen riepen de kinderen Israels tot den HEERE.
\par 11 En zij zeiden tot Mozes: Hebt gij ons daarom, omdat er in Egypte gans geen graven waren, weggenomen, opdat wij in deze woestijn sterven zouden? Waarom hebt gij ons dat gedaan, dat gij ons uit Egypte uitgevoerd hebt?
\par 12 Is dit niet het woord, dat wij in Egypte tot u spraken, zeggende: Houd af van ons, en laat ons de Egyptenaren dienen? Want het ware ons beter geweest de Egyptenaren te dienen, dan in deze woestijn te sterven.
\par 13 Doch Mozes zeide tot het volk: Vreest niet, staat vast, en ziet het heil des HEEREN, dat Hij heden aan ulieden doen zal, want de Egyptenaars, die gij heden gezien hebt, zult gij niet weder zien in eeuwigheid.
\par 14 De HEERE zal voor ulieden strijden, en gij zult stil zijn.
\par 15 Toen zeide de HEERE tot Mozes: Wat roept gij tot Mij? Zeg den kinderen Israels, dat zij voorttrekken.
\par 16 En gij, hef uw staf op, en strek uw hand uit over de zee, en klief dezelve, dat de kinderen Israels door het midden der zee gaan op het droge.
\par 17 En Ik, zie, Ik zal het hart der Egyptenaren verstokken, dat zij na hen daarin gaan; en Ik zal verheerlijkt worden aan Farao en aan al zijn heir, aan zijn wagenen en aan zijn ruiteren.
\par 18 En de Egyptenaars zullen weten, dat Ik de HEERE ben, wanneer Ik verheerlijkt zal worden aan Farao, aan zijn wagenen en aan zijn ruiteren.
\par 19 En de Engel Gods, Die voor het heir van Israel ging, vertrok, en ging achter hen; de wolkkolom vertrok ook van hun aangezicht, en stond achter hen.
\par 20 En zij kwam tussen het leger der Egyptenaren, en tussen het leger van Israel; en de wolk was te gelijk duisternis en verlichtte den nacht; zodat de een tot den ander niet naderde den gansen nacht.
\par 21 Toen Mozes zijn hand uitstrekte over de zee, zo deed de HEERE de zee weggaan, door een sterken oostenwind, dien gansen nacht, en maakte de zee droog, en de wateren werden gekliefd.
\par 22 En de kinderen Israels zijn ingegaan in het midden van de zee, op het droge; en de wateren waren hun een muur, aan hun rechter hand en aan hun linkerhand.
\par 23 En de Egyptenaars vervolgden hen, en gingen in, achter hen, al de paarden van Farao, zijn wagenen en zijn ruiteren, in het midden van de zee.
\par 24 En het geschiedde in dezelfde morgenwake, dat de HEERE, in de kolom des vuurs en der wolk, zag op het leger der Egyptenaren; en Hij verschrikte het leger der Egyptenaren.
\par 25 En Hij stiet de raderen hunner wagenen weg, en deed ze zwaarlijk voortvaren. Toen zeiden de Egyptenaars: Laat ons vlieden van het aangezicht van Israel, want de HEERE strijdt voor hen tegen de Egyptenaars.
\par 26 En de HEERE zeide tot Mozes: Strek uw hand uit over de zee, dat de wateren wederkeren over de Egyptenaars, over hun wagenen en over hun ruiters.
\par 27 Toen strekte Mozes zijn hand uit over de zee; en de zee kwam weder, tegen het naken van den morgenstond, tot haar kracht; en de Egyptenaars vluchtten die tegemoet; en de HEERE stortte de Egyptenaars in het midden der zee.
\par 28 Want als de wateren wederkeerden, zo bedekten zij de wagenen en de ruiters van het ganse heir van Farao, dat hen nagevolgd was in de zee; er bleef niet een van hen over.
\par 29 Maar de kinderen Israels gingen op het droge, in het midden der zee; en de wateren waren hun een muur, aan hun rechter hand en aan hun linkerhand.
\par 30 Alzo verloste de HEERE Israel aan dien dag uit de hand der Egyptenaren; en Israel zag de Egyptenaren dood aan den oever der zee.
\par 31 Ook zag Israel de grote hand, die de HEERE aan de Egyptenaren bewezen had; en het volk vreesde den HEERE, en geloofde in den HEERE, en aan Mozes, Zijn knecht.

\chapter{15}

\par 1 Toen zong Mozes en de kinderen Israels den HEERE dit lied, en spraken, zeggende: Ik zal den HEERE zingen; want Hij is hogelijk verheven! Het paard en zijn ruiter heeft Hij in de zee geworpen.
\par 2 De HEERE is mijn Kracht en Lied, en Hij is mij tot een Heil geweest; deze is mijn God; daarom zal ik Hem een liefelijke woning maken; Hij is mijns vaders God, dies zal ik Hem verheffen!
\par 3 De HEERE is een krijgsman; HEERE is Zijn Naam!
\par 4 Hij heeft Farao's wagenen en zijn heir in de zee geworpen; en de keure zijner hoofdlieden zijn verdronken in de Schelfzee.
\par 5 De afgronden hebben hen bedekt; zij zijn in de diepten gezonken als een steen.
\par 6 O HEERE! Uw rechterhand is verheerlijkt geworden in macht; Uw rechterhand, o HEERE! heeft den vijand verbroken!
\par 7 En door Uw grote hoogheid hebt Gij, die tegen U opstonden, omgeworpen; Gij hebt Uw brandenden toorn uitgezonden, die hen verteerd heeft als een stoppel.
\par 8 En door het geblaas van Uw neus zijn de wateren opgehoopt geworden; de stromen hebben overeind gestaan, als een hoop; de afgronden zijn stijf geworden in het hart der zee.
\par 9 De vijand zeide: Ik zal vervolgen, ik zal achterhalen, ik zal den buit delen, mijn ziel zal van hen vervuld worden, ik zal mijn zwaard uittrekken, mijn hand zal hen uitroeien.
\par 10 Gij hebt met Uw wind geblazen; de zee heeft hen gedekt, zij zonken onder als lood in geweldige wateren!
\par 11 O HEERE! wie is als Gij onder de goden? wie is als Gij, verheerlijkt in heiligheid, vreselijk in lofzangen, doende wonder?
\par 12 Gij hebt Uw rechterhand uitgestrekt, de aarde heeft hen verslonden!
\par 13 Gij leiddet door Uw weldadigheid dit volk, dat Gij verlost hebt; Gij voert hen zachtkens door Uw sterkte tot de liefelijke woning Uwer heiligheid.
\par 14 De volken hebben het gehoord, zij zullen sidderen; weedom heeft de ingezetenen van Palestina bevangen.
\par 15 Dan zullen de vorsten van Edom verbaasd wezen; beving zal de machtigen der Moabieten bevangen; al de ingezetenen van Kanaan zullen versmelten!
\par 16 Verschrikking en vrees zal op hen vallen; door de grootheid van Uw arm zullen zij verstommen, als een steen, totdat Uw volk, HEERE! henen doorkome; totdat dit volk henen doorkome, dat Gij verworven hebt.
\par 17 Die zult Gij inbrengen, en planten hen op den berg Uwer erfenis, ter plaatse, welke Gij, o HEERE! gemaakt hebt tot Uw woning, het heiligdom, hetwelk Uw handen gesticht hebben, o HEERE!
\par 18 De HEERE zal in eeuwigheid en geduriglijk regeren!
\par 19 Want Farao's paard, met zijn wagen, met zijn ruiters, zijn in de zee gekomen, en de HEERE heeft de wateren der zee over hen doen wederkeren; maar de kinderen Israels zijn op het droge in het midden van de zee gegaan.
\par 20 En Mirjam, de profetes, Aarons zuster, nam een trommel in haar hand; en al de vrouwen gingen uit, haar na, met trommelen en met reien.
\par 21 Toen antwoordde Mirjam hunlieden: Zingt den HEERE; want Hij is hogelijk verheven! Hij heeft het paard met zijn ruiter in de zee gestort!
\par 22 Hierna deed Mozes de Israelieten voortreizen van de Schelfzee af; en zij trokken uit tot in de woestijn Sur, en zij gingen drie dagen in de woestijn, en vonden geen water.
\par 23 Toen kwamen zij te Mara; doch zij konden het water van Mara niet drinken, want het was bitter; daarom werd derzelver naam genoemd Mara.
\par 24 Toen murmureerde het volk tegen Mozes, zeggende: Wat zullen wij drinken?
\par 25 Hij dan riep tot den HEERE; en de HEERE wees hem een hout, dat wierp hij in dat water; toen werd het water zoet. Aldaar stelde Hij het volk een inzetting en recht, en aldaar verzocht Hij hetzelve,
\par 26 En zeide: Is het, dat gij met ernst naar de stem des HEEREN uws Gods horen zult, en doen, wat recht is in Zijn ogen, en uw oren neigt tot Zijn geboden, en houdt al Zijn inzettingen; zo zal Ik geen van de krankheden op u leggen, die Ik op Egypteland gelegd heb; want Ik ben de HEERE, uw Heelmeester!
\par 27 Toen kwamen zij te Elim, en daar waren twaalf waterfonteinen, en zeventig palmbomen; en zij legerden zich aldaar aan de wateren.

\chapter{16}

\par 1 Toen zij van Elim gereisd waren, zo kwam de ganse vergadering der kinderen Israels in de woestijn Sin, welke is tussen Elim en tussen Sinai, aan den vijftienden dag der tweede maand, nadat zij uit Egypteland uitgegaan waren.
\par 2 En de ganse vergadering der kinderen Israels murmureerde tegen Mozes en tegen Aaron, in de woestijn.
\par 3 En de kinderen Israels zeiden tot hen: Och, dat wij in Egypteland gestorven waren door de hand des HEEREN, toen wij bij de vleespotten zaten, toen wij tot verzadiging brood aten! Want gijlieden hebt ons uitgeleid in deze woestijn, om deze ganse gemeente door den honger te doden.
\par 4 Toen zeide de HEERE tot Mozes: Zie, Ik zal voor ulieden brood uit den hemel regenen; en het volk zal uitgaan, en verzamelen elke dagmaat op haar dag; opdat Ik het verzoeke, of het in Mijn wet ga, of niet.
\par 5 En het zal geschieden op den zesden dag, dat zij bereiden zullen hetgeen zij ingebracht zullen hebben; dat zal dubbel zijn boven hetgeen zij dagelijks zullen verzamelen.
\par 6 Toen zeiden Mozes en Aaron tot al de kinderen Israels: Aan den avond, dan zult gij weten, dat u de HEERE uit Egypteland uitgeleid heeft;
\par 7 En morgen, dan zult gij des HEEREN heerlijkheid zien, dewijl Hij uw murmureringen tegen den HEERE gehoord heeft; want wat zijn wij, dat gij tegen ons murmureert?
\par 8 Voorts zeide Mozes: Als de HEERE ulieden aan den avond vlees te eten zal geven, en aan den morgen brood tot verzadiging, het zal zijn, omdat de HEERE uw murmureringen gehoord heeft, die gij tegen Hem murmureert; want wat zijn wij? Uw murmureringen zijn niet tegen ons, maar tegen den HEERE.
\par 9 Daarna zeide Mozes tot Aaron: Zeg tot de ganse vergadering der kinderen Israels: Nadert voor het aangezicht des HEEREN, want Hij heeft uw murmureringen gehoord.
\par 10 En het geschiedde, als Aaron tot de ganse vergadering der kinderen Israels sprak, en zij zich naar de woestijn keerden, zo ziet, de heerlijkheid des HEEREN verscheen in de wolk.
\par 11 Ook heeft de HEERE tot Mozes gesproken, zeggende:
\par 12 Ik heb de murmureringen van de kinderen Israels gehoord; spreek tot hen, zeggende: Tussen de twee avonden zult gij vlees eten, en aan den morgen zult gij met brood verzadigd worden; en gij zult weten, dat Ik de HEERE uw God ben.
\par 13 En het geschiedde aan den avond, dat er kwakkelen opkwamen, en het leger bedekten; en aan den morgen lag de dauw rondom het leger.
\par 14 Als nu de liggende dauw opgevaren was, zo ziet, over de woestijn was een klein rond ding, klein als de rijm, op de aarde.
\par 15 Toen het de kinderen Israels zagen, zo zeiden zij, de een tot den ander: Het is Man, want zij wisten niet wat het was. Mozes dan zeide tot hen: Dit is het brood, hetwelk de HEERE ulieden te eten gegeven heeft.
\par 16 Dit is het woord, dat de HEERE geboden heeft: Verzamelt daarvan een ieder naar dat hij eten mag, een gomer voor een hoofd, naar het getal van uw zielen; ieder zal nemen voor degenen, die in zijn tent zijn.
\par 17 En de kinderen Israels deden alzo, en verzamelden, de een veel en de ander weinig.
\par 18 Doch als zij het met den gomer maten, zo had hij, die veel verzameld had, niets over, en dien, die weinig verzameld had, ontbrak niet; een iegelijk verzamelde zoveel, als hij eten mocht.
\par 19 En Mozes zeide tot hen: Niemand late daarvan over tot den morgen.
\par 20 Doch zij hoorden niet naar Mozes, maar sommige mannen lieten daarvan over tot den morgen. Toen wiesen er wormen in, en het werd stinkende; dies werd Mozes zeer toornig op hen.
\par 21 Zij nu verzamelden het allen morgen, een iegelijk naardat hij eten mocht; want als de zon heet werd, zo versmolt het.
\par 22 En het geschiedde op den zesden dag, dat zij dubbel brood verzamelden, twee gomers voor een; en al de oversten der vergadering kwamen en verkondigden het aan Mozes.
\par 23 Hij dan zeide tot hen: Dit is het, dat de HEERE gesproken heeft: Morgen is de rust, de heilige sabbat des HEEREN! wat gij bakken zoudt, bakt dat, en ziedt, wat gij zieden zoudt; en al wat over blijft, legt het op voor u in bewaring tot den morgen.
\par 24 En zij leiden het op tot den morgen, gelijk als Mozes geboden had; en het stonk niet, en er was geen worm in.
\par 25 Toen zeide Mozes: Eet dat heden, want het is heden de sabbat des HEEREN; gij zult het heden op het veld niet vinden.
\par 26 Zes dagen zult gij het verzamelen; doch op den zevenden dag is het sabbat, op denzelven zal het niet zijn.
\par 27 En het geschiedde aan den zevenden dag, dat sommigen van het volk uitgingen, om te verzamelen; doch zij vonden niet.
\par 28 Toen zeide de HEERE tot Mozes: Hoe lang weigert gijlieden te houden Mijn geboden en Mijn wetten?
\par 29 Ziet, omdat de HEERE ulieden den sabbat gegeven heeft, daarom geeft Hij u aan den zesden dag voor twee dagen brood; een ieder blijve in zijn plaats! dat niemand uit zijn plaats ga op den zevenden dag!
\par 30 Alzo rustte het volk op den zevenden dag.
\par 31 En het huis Israels noemde deszelfs naam Man; en het was als korianderzaad, wit, en de smaak daarvan was als honigkoeken.
\par 32 Voorts zeide Mozes: Dit is het woord, hetwelk de HEERE bevolen heeft: Vul een gomer daarvan tot bewaring voor uw geslachten, opdat zij zien het brood, dat Ik ulieden heb te eten gegeven in deze woestijn, toen Ik u uit Egypteland uitleidde.
\par 33 Ook zeide Mozes tot Aaron: Neem een kruik, en doe een gomer vol Man daarin; en zet die voor het aangezicht des HEEREN, tot bewaring voor uw geslachten.
\par 34 Gelijk als de HEERE aan Mozes geboden had, alzo zette ze Aaron voor de getuigenis tot bewaring.
\par 35 En de kinderen Israels aten Man veertig jaren, totdat zij in een bewoond land kwamen; zij aten Man, totdat zij kwamen aan de pale van het land Kanaan.
\par 36 Een gomer nu is het tiende deel van een efa.

\chapter{17}

\par 1 Daarna toog de ganse vergadering van de kinderen Israels, naar hun dagreizen, uit de woestijn Sin, op het bevel des HEEREN, en zij legerden zich te Rafidim. Daar nu was geen water voor het volk om te drinken.
\par 2 Toen twistte het volk met Mozes, en zeide: Geeft gijlieden ons water, dat wij drinken! Mozes dan zeide tot hen: Wat twist gij met mij? Waarom verzoekt gij den HEERE?
\par 3 Toen nu het volk aldaar dorstte naar water, zo murmureerde het volk tegen Mozes, en het zeide: Waartoe hebt gij ons nu uit Egypte doen optrekken, opdat gij mij, en mijn kinderen, en mijn vee, van dorst deedt sterven?
\par 4 Zo riep Mozes tot den HEERE, zeggende: Wat zal ik dit volk doen? Er feilt niet veel aan, of zij zullen mij stenigen.
\par 5 Toen zeide de HEERE tot Mozes: Ga heen voor het aangezicht des volks, en neem met u uit de oudsten van Israel; en neem uw staf in uw hand, waarmede gij de rivier sloegt, en ga heen.
\par 6 Zie, Ik zal aldaar voor uw aangezicht op de rotssteen in Horeb staan; en gij zult op den rotssteen slaan, zo zal er water uitgaan, dat het volk drinke. Mozes nu deed alzo voor de ogen der oudsten van Israel.
\par 7 En hij noemde den naam dier plaats Massa en Meriba, om den twist der kinderen Israels, en omdat zij den HEERE verzocht hadden, zeggende: Is de HEERE in het midden van ons, of niet?
\par 8 Toen kwam Amalek en streed tegen Israel in Rafidim.
\par 9 Mozes dan zeide tot Jozua: Kies ons mannen, en trek uit, strijd tegen Amalek; morgen zal ik op de hoogte des heuvels staan, en de staf Gods zal in mijn hand zijn.
\par 10 Jozua nu deed, als Mozes hem gezegd had, strijdende tegen Amalek; doch Mozes, Aaron en Hur klommen op de hoogte des heuvels.
\par 11 En het geschiedde, terwijl Mozes zijn hand ophief, zo was Israel de sterkste; maar terwijl hij zijn hand nederliet, zo was Amalek de sterkste.
\par 12 Doch de handen van Mozes werden zwaar; daarom namen zij een steen, en legden dien onder hem, dat hij daarop zat; en Aaron en Hur onderstutten zijn handen, de een op deze, de ander op de andere zijde; alzo waren zijn handen gewis, totdat de zon onderging.
\par 13 Alzo dat Jozua Amalek en zijn volk krenkte, door de scherpte des zwaards.
\par 14 Toen zeide de HEERE tot Mozes: Schrijf dit ter gedachtenis in een boek, en leg het in de oren van Jozua, dat Ik de gedachtenis van Amalek geheel uitdelgen zal van onder den hemel.
\par 15 En Mozes bouwde een altaar; en hij noemde deszelfs naam: De HEERE is mijn Banier!
\par 16 En hij zeide: Dewijl de hand op den troon des HEEREN is, zo zal de oorlog des HEEREN tegen Amalek zijn, van geslacht tot geslacht!

\chapter{18}

\par 1 Toen Jethro, priester van Midian, schoonvader van Mozes, hoorde al wat God aan Mozes, en aan Israel, Zijn volk, gedaan had: dat de HEERE Israel uit Egypte uitgevoerd had;
\par 2 Zo nam Jethro, Mozes' schoonvader, Zippora, Mozes' huisvrouw (nadat hij haar wedergezonden had),
\par 3 Met haar twee zonen, welker enes naam was Gersom (want hij zeide: Ik ben een vreemdeling geweest in een vreemd land);
\par 4 En de naam des anderen was Eliezer, want, zeide hij, de God mijns vaders is tot mijn Hulpe geweest, en heeft mij verlost van Farao's zwaard.
\par 5 Toen nu Jethro, Mozes' schoonvader, met zijn zonen en zijn huisvrouw, tot Mozes kwam, in de woestijn, aan den berg Gods, waar hij zich gelegerd had,
\par 6 Zo zeide hij tot Mozes: Ik, uw schoonvader Jethro, kom tot u, met uw huisvrouw, en haar beide zonen met haar.
\par 7 Toen ging Mozes uit, zijn schoonvader tegemoet, en hij boog zich, en kuste hem; en zij vraagden de een den ander naar den welstand, en zij gingen naar de tent.
\par 8 En Mozes vertelde zijn schoonvader alles, wat de HEERE aan Farao en aan de Egyptenaren gedaan had, om Israels wil; al de moeite, die hun op dien weg ontmoet was, en dat hen de HEERE verlost had.
\par 9 Jethro nu verheugde zich over al het goede, hetwelk de HEERE Israel gedaan had; dat Hij het verlost had uit de hand der Egyptenaren.
\par 10 En Jethro zeide: Gezegend zij de HEERE, Die ulieden verlost heeft uit de hand der Egyptenaren, en uit Farao's hand; Die dit volk van onder de hand der Egyptenaren verlost heeft!
\par 11 Nu weet ik, dat de HEERE groter is dan alle goden; want in de zaak, waarin zij trotselijk gehandeld hebben, was Hij boven hen.
\par 12 Toen nam Jethro, de schoonvader van Mozes, Gode brandoffer en slachtofferen; en Aaron kwam, en al de oversten van Israel, om brood te eten met den schoonvader van Mozes, voor het aangezicht Gods.
\par 13 Doch het geschiedde des anderen daags, zo zat Mozes om het volk te richten, en het volk stond voor Mozes, van den morgen tot den avond.
\par 14 Als de schoonvader van Mozes alles zag, wat hij het volk deed, zo zeide hij: Wat ding is dit, dat gij het volk doet? Waarom zit gij zelf alleen, en al het volk staat voor u, van den morgen tot den avond?
\par 15 Toen zeide Mozes tot zijn schoonvader: Omdat dit volk tot mij komt, om God raad te vragen.
\par 16 Wanneer zij een zaak hebben, zo komt het tot mij, dat ik richte tussen den man en tussen zijn naaste; en dat ik hun bekend make Gods instellingen en Zijn wetten.
\par 17 Doch de schoonvader van Mozes zeide tot hem: De zaak is niet goed, die gij doet.
\par 18 Gij zult geheel vervallen, zo gij, als dit volk, hetwelk bij u is; want deze zaak is te zwaar voor u, gij alleen kunt het niet doen.
\par 19 Hoor nu mijn stem, ik zal u raden, en God zal met u zijn; wees gij voor het volk bij God, en breng gij de zaken voor God;
\par 20 En verklaar hun de instellingen en de wetten, en maak hun bekend den weg, waarin zij wandelen zullen, en het werk, dat zij doen zullen.
\par 21 Doch zie gij om, onder al het volk, naar kloeke mannen, God vrezende, waarachtige mannen, de gierigheid hatende; stel ze over hen, oversten der duizenden, oversten der honderden, oversten der vijftigen, oversten der tienen.
\par 22 Dat zij dit volk te allen tijde richten; doch het geschiede, dat zij alle grote zaken aan u brengen, maar dat zij alle kleine zaken richten; verlicht alzo uzelven, en laat hen met u dragen.
\par 23 Indien gij deze zaak doet, en God het u gebiedt, zo zult gij kunnen bestaan; zo zal ook al dit volk in vrede aan zijn plaats komen.
\par 24 Mozes nu hoorde naar de stem van zijn schoonvader, en hij deed alles, wat hij gezegd had.
\par 25 En Mozes verkoos kloeke mannen, uit gans Israel, en maakte hen tot hoofden over het volk; oversten der duizenden, oversten der honderden, oversten der vijftigen, en oversten der tienen;
\par 26 Dat zij het volk te allen tijde richtten, de harde zaak tot Mozes brachten, maar zij alle kleine zaak richtten.
\par 27 Toen liet Mozes zijn schoonvader trekken; en hij ging naar zijn land.

\chapter{19}

\par 1 In de derde maand, na het uittrekken der kinderen Israels uit Egypteland, ten zelfden dage kwamen zij in de woestijn Sinai.
\par 2 Want zij togen uit Rafidim, en kwamen in de woestijn Sinai, en zij legerden zich in de woestijn; Israel nu legerde zich aldaar tegenover dien berg.
\par 3 En Mozes klom op tot God. En de HEERE riep tot hem van den berg, zeggende: Aldus zult gij tot het huis van Jakob spreken, en den kinderen Israels verkondigen:
\par 4 Gijlieden hebt gezien, wat Ik den Egyptenaren gedaan heb; hoe Ik u op vleugelen der arenden gedragen, en u tot Mij gebracht heb.
\par 5 Nu dan, indien gij naarstiglijk Mijner stem zult gehoorzamen, en Mijn verbond houden, zo zult gij Mijn eigendom zijn uit alle volken, want de ganse aarde is Mijn;
\par 6 En gij zult Mij een priesterlijk koninkrijk, en een heilig volk zijn. Dit zijn de woorden, die gij tot de kinderen Israels spreken zult.
\par 7 En Mozes kwam en riep de oudsten des volks, en stelde voor hun aangezichten al deze woorden, die de HEERE hem geboden had.
\par 8 Toen antwoordde al het volk gelijkelijk, en zeide: Al wat de HEERE gesproken heeft, zullen wij doen! En Mozes bracht de woorden des volks weder tot den HEERE.
\par 9 En de HEERE zeide tot Mozes: Zie, Ik zal tot u komen in een dikke wolk, opdat het volk hore, als Ik met u spreek, en dat zij ook eeuwiglijk aan u geloven. Want Mozes had den HEERE de woorden des volks verkondigd.
\par 10 Ook zeide de HEERE tot Mozes: Ga tot het volk, en heilig hen heden en morgen, en dat zij hun klederen wassen,
\par 11 En bereid zijn tegen den derden dag; want op den derden dag zal de HEERE voor de ogen van al het volk afkomen, op den berg Sinai.
\par 12 En bepaal het volk rondom, zeggende: Wacht u op den berg te klimmen, en deszelfs einde aan te roeren; al wie den berg aanroert, zal zekerlijk gedood worden.
\par 13 Geen hand zal hem aanroeren, maar hij zal zekerlijk gestenigd, of zekerlijk doorschoten worden; hetzij een beest, hetzij een man, hij zal niet leven. Als de ramshoorn langzaam gaat, zullen zij op den berg klimmen.
\par 14 Toen ging Mozes van den berg af tot het volk, en hij heiligde het volk; en zij wiesen hun klederen.
\par 15 En hij zeide tot het volk: Weest gereed tegen den derden dag, en nadert niet tot de vrouw.
\par 16 En het geschiedde op den derden dag, toen het morgen was, dat er op den berg donderen en bliksemen waren, en een zware wolk, en het geluid ener zeer sterke bazuin, zodat al het volk verschrikte, dat in het leger was.
\par 17 En Mozes leidde het volk uit het leger, Gode tegemoet; en zij stonden aan het onderste des bergs.
\par 18 En de ganse berg Sinai rookte, omdat de HEERE op denzelven nederkwam in vuur; en zijn rook ging op, als de rook van een oven; en de ganse berg beefde zeer.
\par 19 Toen het geluid der bazuin gaande was, en zeer sterk werd, sprak Mozes; en God antwoordde hem met een stem.
\par 20 Als de HEERE nedergekomen was op den berg Sinai, op de spits des bergs, zo riep de HEERE Mozes op de spits des bergs; en Mozes klom op.
\par 21 En de HEERE zeide tot Mozes: Ga af, betuig dit volk, dat zij niet doorbreken tot den HEERE, om te zien, en velen van hen vallen.
\par 22 Daartoe zullen ook de priesters, die tot den HEERE naderen, zich heiligen, dat de HEERE niet tegen hen uitbreke.
\par 23 Toen zeide Mozes tot den HEERE: Het volk zal op den berg Sinai niet kunnen klimmen, want Gij hebt ons betuigd, zeggende: Bepaal den berg, en heilig hem.
\par 24 De HEERE dan zeide tot hem: Ga heen, klim af, daarna zult gij, en Aaron met u, opklimmen; doch dat de priesters en het volk niet doorbreken, om op te klimmen tot den HEERE, dat Hij tegen hen niet uitbreke.
\par 25 Toen klom Mozes af tot het volk, en zeide het hun aan.

\chapter{20}

\par 1 Toen sprak God al deze woorden, zeggende:
\par 2 Ik ben de HEERE uw God, Die u uit Egypteland, uit het diensthuis, uitgeleid heb.
\par 3 Gij zult geen andere goden voor Mijn aangezicht hebben.
\par 4 Gij zult u geen gesneden beeld, noch enige gelijkenis maken, van hetgeen boven in den hemel is, noch van hetgeen onder op de aarde is, noch van hetgeen in de wateren onder de aarde is.
\par 5 Gij zult u voor die niet buigen, noch hen dienen; want Ik, de HEERE uw God, ben een ijverig God, Die de misdaad der vaderen bezoek aan de kinderen, aan het derde, en aan het vierde lid dergenen, die Mij haten;
\par 6 En doe barmhartigheid aan duizenden dergenen, die Mij liefhebben, en Mijn geboden onderhouden.
\par 7 Gij zult den naam des HEEREN uws Gods niet ijdellijk gebruiken; want de HEERE zal niet onschuldig houden, die Zijn naam ijdellijk gebruikt.
\par 8 Gedenkt den sabbatdag, dat gij dien heiligt.
\par 9 Zes dagen zult gij arbeiden en al uw werk doen;
\par 10 Maar de zevende dag is de sabbat des HEEREN uws Gods; dan zult gij geen werk doen, gij, noch uw zoon, noch uw dochter, noch uw dienstknecht, noch uw dienstmaagd, noch uw vee, noch uw vreemdeling, die in uw poorten is;
\par 11 Want in zes dagen heeft de HEERE den hemel en de aarde gemaakt, de zee en al wat daarin is, en Hij rustte ten zevenden dage; daarom zegende de HEERE den sabbatdag, en heiligde denzelven.
\par 12 Eert uw vader en uw moeder, opdat uw dagen verlengd worden in het land, dat u de HEERE uw God geeft.
\par 13 Gij zult niet doodslaan.
\par 14 Gij zult niet echtbreken.
\par 15 Gij zult niet stelen.
\par 16 Gij zult geen valse getuigenis spreken tegen uw naaste.
\par 17 Gij zult niet begeren uws naasten huis; gij zult niet begeren uws naasten vrouw, noch zijn dienstknecht, noch zijn dienstmaagd, noch zijn os, noch zijn ezel, noch iets, dat uws naasten is.
\par 18 En al het volk zag de donderen, en de bliksemen, en het geluid der bazuin, en den rokenden berg; toen het volk zulks zag, weken zij af, en stonden van verre;
\par 19 En zij zeiden tot Mozes: Spreek gij met ons, en wij zullen horen; en dat God met ons niet spreke, opdat wij niet sterven!
\par 20 En Mozes zeide tot het volk: Vreest niet, want God is gekomen, opdat Hij u verzocht, en opdat Zijn vreze voor uw aangezicht zou zijn, dat gij niet zondigdet.
\par 21 En het volk stond van verre; maar Mozes naderde tot de donkerheid, alwaar God was.
\par 22 Toen zeide de HEERE tot Mozes: Aldus zult gij tot de kinderen Israels zeggen: Gij hebt gezien, dat Ik met ulieden van den hemel gesproken heb.
\par 23 Gij zult nevens Mij niet maken zilveren goden, en gouden goden zult gij u niet maken.
\par 24 Maakt Mij een altaar van aarde, en offert daarop uw brandofferen, en uw dankofferen, uw schapen, en uw runderen; aan alle plaats, waar Ik Mijns Naams gedachtenis stichten zal, zal Ik tot u komen, en zal u zegenen.
\par 25 Maar indien gij Mij een stenen altaar zult maken, zo zult gij dit niet bouwen van gehouwen steen; zo gij uw houwijzer daarover verheft, zo zult gij het ontheiligen.
\par 26 Gij zult ook niet met trappen tot Mijn altaar opklimmen, opdat uw schaamte voor hetzelve niet ontdekt worde.

\chapter{21}

\par 1 Dit nu zijn de rechten, die gij hun zult voorstellen.
\par 2 Als gij een Hebreeuwsen knecht kopen zult, die zal zes jaren dienen; maar in het zevende zal hij voor vrij uitgaan, om niet.
\par 3 Indien hij met zijn lijf ingekomen zal zijn, zo zal hij met zijn lijf uitgaan; indien hij een getrouwd man was, zo zal zijn vrouw met hem uitgaan.
\par 4 Indien hem zijn heer een vrouw gegeven, en zij hem zonen of dochteren gebaard zal hebben, zo zal de vrouw en haar kinderen haars heren zijn, en hij zal met zijn lijf uitgaan.
\par 5 Maar indien de knecht ronduit zeggen zal: Ik heb mijn heer, mijn vrouw en mijn kinderen lief, ik wil niet vrij uitgaan;
\par 6 Zo zal hem zijn heer tot de goden brengen, daarna zal hij hem aan de deur, of aan den post brengen; en zijn heer zal hem met een priem zijn oor doorboren, en hij zal hem eeuwiglijk dienen.
\par 7 Wanneer nu iemand zijn dochter zal verkocht hebben tot een dienstmaagd, zo zal zij niet uitgaan, gelijk de knechten uitgaan.
\par 8 Indien zij kwalijk bevalt in de ogen haars heren, dat hij haar niet ondertrouwd heeft, zo zal hij haar doen lossen; aan een vreemd volk haar te verkopen zal hij niet vermogen, dewijl hij trouweloos met haar gehandeld heeft.
\par 9 Maar indien hij haar aan zijn zoon ondertrouwt, zo zal hij met haar doen naar het recht der dochteren.
\par 10 Indien hij voor zich een andere neemt, zo zal hij aan deze haar spijs, haar deksel, en haar huwelijksplicht niet onttrekken.
\par 11 En indien hij haar deze drie dingen niet doet, zo zal zij om niet uitgaan, zonder geld.
\par 12 Wie iemand slaat, dat hij sterft, die zal zekerlijk gedood worden.
\par 13 Doch die hem niet nagesteld heeft, maar God heeft hem zijn hand doen ontmoeten, zo zal Ik u een plaats bestellen, waar hij henen vliede.
\par 14 Maar indien iemand tegen zijn naaste moedwillig gehandeld heeft, om hem met list te doden, zo zult gij denzelven van voor Mijn altaar nemen, dat hij sterve.
\par 15 Zo wie zijn vader of zijn moeder slaat, die zal zekerlijk gedood worden.
\par 16 Verder, zo wie een mens steelt, hetzij dat hij dien verkocht heeft, of dat hij in zijn hand gevonden wordt, die zal zekerlijk gedood worden.
\par 17 Wie ook zijn vader of zijn moeder vloekt, die zal zekerlijk gedood worden.
\par 18 En wanneer mannen twisten, en de een slaat den ander met een steen, of met een vuist, en hij sterft niet, maar valt te bedde;
\par 19 Indien hij weder opstaat, en op straat gaat bij zijn stok, zo zal hij, die hem sloeg, onschuldig zijn; alleen zal hij geven hetgeen hij verzuimd heeft, en hij zal hem volkomen laten helen.
\par 20 Wanneer ook iemand zijn dienstknecht of zijn dienstmaagd met een stok slaat, dat hij onder zijn hand sterft, die zal zekerlijk gewroken worden.
\par 21 Zo hij nochtans een dag of twee dagen overeind blijft, zo zal hij niet gewroken worden; want hij is zijn geld.
\par 22 Wanneer nu mannen kijven, en slaan een zwangere vrouw, dat haar de vrucht afgaat, doch geen dodelijk verderf zij, zo zal hij zekerlijk gestraft worden, gelijk als hem de man der vrouw oplegt, en hij zal het geven door de rechters.
\par 23 Maar indien er een dodelijk verderf zal zijn, zo zult gij geven ziel voor ziel,
\par 24 Oog voor oog, tand voor tand, hand voor hand, voet voor voet.
\par 25 Brand voor brand, wond voor wond, buil voor buil.
\par 26 Wanneer ook iemand het oog van zijn dienstknecht, of het oog van zijn dienstmaagd slaat, en verderft het, hij zal hem vrij laten gaan voor zijn oog.
\par 27 En indien hij een tand van zijn dienstknecht, of een tand van zijn dienstmaagd uitslaat, zo zal hij hem vrijlaten voor zijn tand.
\par 28 En wanneer een os een man of een vrouw stoot, dat hij sterft, zal de os zekerlijk gestenigd worden, en zijn vlees zal niet gegeten worden; maar de heer van den os zal onschuldig zijn.
\par 29 Maar indien de os te voren stotig geweest is, en zijn heer is daarvan overtuigd geweest, en hij hem niet bewaard heeft, en hij doodt een man of een vrouw, zo zal die os gestenigd worden, en zijn heer zal ook gedood worden.
\par 30 Indien hem losgeld opgelegd wordt, zo zal hij tot lossing zijner ziel geven naar alles, wat hem zal opgelegd worden;
\par 31 Hetzij dat hij een zoon gestoten heeft, of een dochter gestoten heeft, naar dat recht zal hem gedaan worden.
\par 32 Indien de os een knecht of een dienstmaagd stoot, hij zal zijn heer dertig zilverlingen geven, en de os zal gestenigd worden.
\par 33 En wanneer iemand een kuil opent, of wanneer iemand een kuil graaft, en hij dekt hem niet toe, en een os of ezel valt daarin;
\par 34 De heer des kuils zal het vergelden; hij zal aan deszelfs heer het geld wederkeren; doch dat dode zal zijns wezen.
\par 35 Wanneer nu iemands os den os van zijn naaste kwetst, dat hij sterft, zo zal men den levenden os verkopen, en het geld daarvan half en half delen, en den dode zal men ook half en half delen.
\par 36 Of is het kennelijk geweest, dat die os van te voren stotig was, en zijn heer heeft hem niet bewaard, zo zal hij in alle manier os voor os vergelden; doch de dode zal zijns wezen.

\chapter{22}

\par 1 Wanneer iemand een os, of klein vee steelt, en slacht het, of verkoopt het, die zal vijf runderen voor een os wedergeven, en vier schapen voor een stuk klein vee.
\par 2 Indien een dief gevonden wordt in het doorgraven, en hij wordt geslagen, dat hij sterft, het zal hem geen bloedschuld zijn.
\par 3 Indien de zon over hem opgegaan is, zo zal het hem een bloedschuld zijn; hij zal het volkomen wedergeven; heeft hij niet, zo zal hij verkocht worden voor zijn dieverij.
\par 4 Indien de diefstal levend in zijn hand voorzeker gevonden wordt, hetzij os, of ezel, of klein vee, hij zal het dubbel wedergeven.
\par 5 Wanneer iemand een veld, of een wijngaard laat afweiden, en hij zijn beest daarin drijft, dat het in eens anders veld weidt, die zal het van het beste zijns velds en van het beste zijns wijngaards wedergeven.
\par 6 Wanneer een vuur uitgaat, en vat de doornen, zodat de koornhoop verteerd wordt, of het staande koorn, of het veld; hij, die den brand heeft aangestoken, zal het volkomen wedergeven.
\par 7 Wanneer iemand zijn naaste geld of vaten te bewaren geeft, en het wordt uit diens mans huis gestolen; indien de dief gevonden wordt, hij zal het dubbel wedergeven.
\par 8 Indien de dief niet gevonden wordt, zo zal de heer des huizes tot de goden gebracht worden, of hij niet zijn hand aan zijns naasten have gelegd heeft.
\par 9 Over alle zaak van onrecht, over een os, over een ezel, over klein vee, over kleding, over al het verlorene, hetwelk iemand zegt, dat het zijn is, beider zaak zal voor de goden komen; wien de goden verwijzen, die zal het aan zijn naaste dubbel wedergeven.
\par 10 Wanneer iemand aan zijn naaste een ezel, of os, of klein vee, of enig beest te bewaren geeft, en het sterft, of het wordt verzeerd, of weggedreven, dat het niemand ziet;
\par 11 Zo zal des HEEREN eed tussen hen beiden zijn, of hij niet zijn hand aan zijns naasten have geslagen heeft; en derzelver heer zal dien aannemen; en hij zal het niet wedergeven.
\par 12 Maar indien het van hem zekerlijk gestolen is, hij zal het zijn heer wedergeven.
\par 13 Is het gewisselijk verscheurd, dat hij het brenge tot getuige, zo zal hij het verscheurde niet wedergeven.
\par 14 En wanneer iemand van zijn naaste wat begeert, en het wordt beschadigd, of het sterft; zijn heer daar niet bij zijnde, zal hij het volkomen wedergeven.
\par 15 Indien zijn heer daarbij geweest is, hij zal het niet wedergeven; indien het gehuurd is, zo is het voor zijn huur gekomen.
\par 16 Wanneer nu iemand een maagd verlokt, die niet ondertrouwd is, en hij ligt bij haar, die zal haar zonder uitstel een bruidschat geven, dat zij hem ter vrouwe zij.
\par 17 Indien haar vader ganselijk weigert haar aan hem te geven, zo zal hij geld geven naar den bruidschat der maagden.
\par 18 De toveres zult gij niet laten leven.
\par 19 Al wie bij een beest ligt, die zal zekerlijk gedood worden.
\par 20 Wie den goden offert, behalve den HEERE alleen, die zal verbannen worden.
\par 21 Gij zult ook den vreemdeling geen overlast doen, noch hem onderdrukken; want gij zijt vreemdelingen geweest in Egypteland.
\par 22 Gij zult geen weduwe noch wees beledigen.
\par 23 Indien gij hen enigszins beledigt, en indien zij enigszins tot Mij roepen, Ik zal hun geroep zekerlijk verhoren;
\par 24 En Mijn toorn zal ontsteken, en Ik zal ulieden met het zwaard doden; en uw vrouwen zullen weduwen, en uw kinderen zullen wezen worden.
\par 25 Indien gij Mijn volk, dat bij u arm is, geld leent, zo zult gij tegen hetzelve niet zijn, als een woekeraar; gij zult op hetzelve geen woeker leggen.
\par 26 Indien gij enigszins uws naasten kleed te pand neemt, zo zult gij het hem wedergeven, eer de zon ondergaat;
\par 27 Want dat alleen is zijn deksel, het is zijn kleed over zijn huid; waarin zou hij liggen? Het zal dan geschieden, wanneer hij tot Mij roept, dat Ik het zal horen; want Ik ben genadig!
\par 28 De goden zult gij niet vloeken, en de oversten in uw volk zult gij niet lasteren.
\par 29 Uw volheid en uw tranen zult gij niet uitstellen; den eerstgeborene uwer zonen zult gij Mij geven.
\par 30 Desgelijks zult gij doen met uw ossen en met uw schapen; zeven dagen zullen zij bij hun moeder zijn, op den achtsten dag zult gij ze Mij geven.
\par 31 Gij nu zult Mij heilige lieden zijn; daarom zult gij geen vlees eten, dat op het veld gescheurd is, gij zult het den hond voorwerpen.

\chapter{23}

\par 1 Gij zult geen vals gerucht opnemen; en stelt uw hand niet bij den goddeloze, om een getuige tot geweld te zijn.
\par 2 Gij zult de menigte tot boze zaken niet volgen; en gij zult niet spreken in een twistige zaak, dat gij u neigt naar de menigte, om het recht te buigen.
\par 3 Ook zult gij den geringe niet voortrekken en zijn twistige zaak.
\par 4 Wanneer gij uws vijands os, of zijn dwalenden ezel, ontmoet, gij zult hem denzelven ganselijk wederbrengen.
\par 5 Wanneer gij uws haters ezel onder zijn last ziet liggen, zult gij dan nalatig zijn, om het uwe te verlaten voor hem? Gij zult het in alle manier met hem verlaten.
\par 6 Gij zult het recht uws armen niet buigen in zijn twistige zaak.
\par 7 Zijt verre van valse zaken; en den onschuldige en gerechtige zult gij niet doden; want Ik zal den goddeloze niet rechtvaardigen.
\par 8 Ook zult gij geen geschenk nemen; want het geschenk verblindt de zienden, en het verkeert de zaak der rechtvaardigen.
\par 9 Gij zult ook den vreemdeling niet onderdrukken; want gij kent het gemoed des vreemdelings, dewijl gij vreemdelingen geweest zijt in Egypteland.
\par 10 Gij zult ook zes jaar uw land bezaaien, en deszelfs inkomst verzamelen;
\par 11 Maar in het zevende zult gij het rusten en stil liggen laten, dat de armen uws volks mogen eten, en het overige daarvan de beesten des velds eten mogen; alzo zult gij ook doen met uw wijngaard, en met uw olijfbomen.
\par 12 Zes dagen zult gij uw werken doen; maar op den zevenden dag zult gij rusten; opdat uw os en uw ezel ruste, en dat de zoon uwer dienstmaagd en de vreemdeling adem scheppe.
\par 13 In alles, wat Ik tot ulieden gezegd heb, zult gij op uw hoede zijn; en den naam van andere goden zult gij niet gedenken; uit uw mond zal hij niet gehoord worden!
\par 14 Drie reizen in het jaar zult gij Mij feest houden.
\par 15 Het feest van de ongezuurde broden zult gij houden; zeven dagen zult gij ongezuurde broden eten (gelijk Ik u geboden heb), ter bestemder tijd in de maand Abib, want in dezelve zijt gij uit Egypte getogen; doch men zal niet ledig voor Mijn aangezicht verschijnen.
\par 16 En het feest des oogstes, der eerste vruchten van uw arbeid, die gij op het veld gezaaid zult hebben. En het feest der inzameling, op den uitgang des jaars, wanneer gij uw arbeid uit het veld zult ingezameld hebben.
\par 17 Drie malen des jaars zullen al uw mannen voor het aangezicht des Heeren HEEREN verschijnen.
\par 18 Gij zult het bloed Mijns offers met geen gedesemde broden offeren; ook zal het vette Mijns feestes tot op den morgen niet vernachten.
\par 19 De eerstelingen der eerste vruchten uws lands zult gij in het huis des HEEREN uws Gods brengen. Gij zult het bokje niet koken in de melk zijner moeder.
\par 20 Ziet, Ik zende een Engel voor uw aangezicht, om u te behoeden op dezen weg, en om u te brengen tot de plaats, die Ik bereid heb.
\par 21 Hoedt u voor Zijn aangezicht, en weest Zijner stem gehoorzaam, en verbittert Hem niet; want Hij zal ulieder overtredingen niet vergeven; want Mijn Naam is in het binnenste van Hem.
\par 22 Maar zo gij Zijner stem naarstiglijk gehoorzaamt, en doet al wat Ik spreken zal, zo zal Ik uwer vijanden vijand, en uwer wederpartijders wederpartij zijn.
\par 23 Want Mijn Engel zal voor uw aangezicht gaan, en Hij zal u inbrengen tot de Amorieten, en Hethieten, en Ferezieten, en Kanaanieten, Hevieten, en Jebusieten; en Ik zal hen verdelgen.
\par 24 Gij zult u voor hun goden niet buigen, noch hen dienen; ook zult gij naar hun werken niet doen; maar gij zult ze geheel afbreken, en hun opgerichte beelden ganselijk vermorzelen.
\par 25 En gij zult den HEERE uw God dienen, zo zal Hij uw brood en uw water zegenen; en Ik zal de krankheden uit het midden van u weren.
\par 26 Er zal geen misdrachtige, noch onvruchtbare in uw land zijn; Ik zal het getal uwer dagen vervullen.
\par 27 Ik zal Mijn schrik voor uw aangezicht zenden, en al het volk, tot hetwelk gij komt, versaagd maken; en Ik zal maken, dat al uw vijanden u den nek toekeren.
\par 28 Ik zal ook horzelen voor uw aangezicht zenden; die zullen van voor uw aangezicht uitstoten de Hevieten, de Kanaanieten en de Hethieten.
\par 29 Ik zal hen in een jaar van uw aangezicht niet uitstoten, opdat het land niet woest worde, en het wild gedierte boven u niet vermenigvuldigd worde.
\par 30 Ik zal hen allengskens van uw aangezicht uitstoten, totdat gij gewassen zijt en het land erft.
\par 31 En Ik zal uw landpalen zetten van de zee Suf tot aan de zee der Filistijnen, en van de woestijn tot aan de rivier; want Ik zal de inwoners van dat land in uw hand geven, dat gij hen voor uw aangezicht uitstoot.
\par 32 Gij zult met hen, noch met hun goden, een verbond maken.
\par 33 Zij zullen in uw land niet wonen, opdat zij u tegen Mij niet doen zondigen; indien gij hun goden dient, het zal u voorzeker tot een valstrik zijn.

\chapter{24}

\par 1 Daarna zeide Hij tot Mozes: Klim op tot den HEERE, gij en Aaron, Nadab en Abihu, en zeventig van de oudsten van Israel; en buigt u neder van verre!
\par 2 En dat Mozes alleen zich nadere tot den HEERE, maar dat zij niet naderen; en het volk klimme ook niet op met hem.
\par 3 Als Mozes kwam en verhaalde aan het volk al de woorden des HEEREN, en al de rechten, toen antwoordde al het volk met een stem, en zij zeiden: Al deze woorden, die de HEERE gesproken heeft, zullen wij doen.
\par 4 Mozes nu beschreef al de woorden des HEEREN, en hij maakte zich des morgens vroeg op, en hij bouwde een altaar onder aan den berg, en twaalf kolommen, naar de twaalf stammen van Israel.
\par 5 En hij zond de jongelingen van de kinderen Israels, die brandofferen offerden, en den HEERE dankofferen offerden, van jonge ossen.
\par 6 En Mozes nam de helft van het bloed, en zette het in bekkens; en de helft van het bloed sprengde hij op het altaar.
\par 7 En hij nam het boek des verbonds, en hij las het voor de oren des volks; en zij zeiden: Al wat de HEERE gesproken heeft, zullen wij doen en gehoorzamen.
\par 8 Toen nam Mozes dat bloed, en sprengde het op het volk; en hij zeide: Ziet, dit is het bloed des verbonds, hetwelk de HEERE met ulieden gemaakt heeft over al die woorden.
\par 9 Mozes nu en Aaron klommen opwaarts, ook Nadab en Abihu, en zeventig van de oudsten van Israel.
\par 10 En zij zagen den God van Israel, en onder Zijn voeten als een werk van saffierstenen, en als de gestaltenis des hemels in Zijn klaarheid.
\par 11 Doch Hij strekte Zijn hand niet tot de afgezonderden van de kinderen Israels; maar zij aten en dronken, nadat zij God gezien hadden.
\par 12 Toen zeide de HEERE tot Mozes: Kom tot Mij op den berg, en wees aldaar; en Ik zal u stenen tafelen geven, en de wet, en de geboden, die Ik geschreven heb, om hen te onderwijzen.
\par 13 Toen maakte zich Mozes op, met Jozua, zijn dienaar; en Mozes klom op den berg Gods.
\par 14 En hij zeide tot de oudsten: Blijft gij ons hier, totdat wij weder tot u komen; en ziet, Aaron en Hur zijn bij u; wie enige zaken heeft, zal tot dezelve komen.
\par 15 Toen Mozes op den berg geklommen was, zo heeft een wolk den berg bedekt.
\par 16 En de heerlijkheid des HEEREN woonde op den berg Sinai, en de wolk bedekte hem zes dagen, en op den zevenden dag riep Hij Mozes uit het midden der wolk.
\par 17 En het aanzien der heerlijkheid des HEEREN was als een verterend vuur, op het opperste diens bergs, in de ogen der kinderen Israels.
\par 18 En Mozes ging in het midden der wolk, nadat hij op den berg geklommen was; en Mozes was op dien berg veertig dagen en veertig nachten.

\chapter{25}

\par 1 Toen sprak de HEERE tot Mozes, zeggende:
\par 2 Spreek tot de kinderen Israels, dat zij voor Mij een hefoffer nemen. Van alle man, wiens hart zich vrijwillig bewegen zal, zult gij Mijn hefoffer nemen.
\par 3 Dit nu is het hefoffer, hetwelk gij van hen nemen zult: goud, en zilver, en koper;
\par 4 Als ook hemelsblauw, en purper, en scharlaken, en fijn linnen, en geiten haar.
\par 5 En roodgeverfde ramsvellen, en dassenvellen; en sittimhout;
\par 6 Olie tot den luchter, specerijen ter zalfolie, en tot roking welriekende specerijen;
\par 7 Sardonixstenen, en vervullende stenen tot den efod, en tot den borstlap.
\par 8 En zij zullen Mij een heiligdom maken, dat Ik in het midden van hen wone.
\par 9 Naar al wat Ik u tot een voorbeeld dezes tabernakels, en een voorbeeld van al deszelfs gereedschap wijzen zal, even alzo zult gijlieden dat maken.
\par 10 Zo zullen zij een ark van sittimhout maken; twee ellen en een halve zal haar lengte zijn, en anderhalve el haar breedte, en anderhalve el haar hoogte.
\par 11 En gij zult ze met louter goud overtrekken, van binnen en van buiten zult gij ze overtrekken; en gij zult op dezelve een gouden krans maken rondom heen.
\par 12 En giet voor haar vier gouden ringen, en zet die aan haar vier hoeken, alzo dat twee ringen op de ene zijde derzelve zijn, en twee ringen op haar andere zijde.
\par 13 En maak handbomen van sittimhout, en overtrek ze met goud.
\par 14 En steek de handbomen in de ringen, die aan de zijde der ark zijn, dat men de ark daarmede drage.
\par 15 De draagbomen zullen in de ringen der ark zijn; zij zullen er niet uitgetogen worden.
\par 16 Daarna zult gij in de ark leggen de getuigenis, die Ik u geven zal.
\par 17 Gij zult ook een verzoendeksel maken van louter goud; twee ellen en een halve zal deszelfs lengte zijn, en anderhalve el deszelfs breedte.
\par 18 Gij zult ook twee cherubim van goud maken; van dicht goud zult gij ze maken, uit de beide einden des verzoendeksels.
\par 19 En maak u een cherub uit het ene einde aan deze zijde, en den anderen cherub uit het andere einde aan gene zijde; uit het verzoendeksel zult gijlieden de cherubim maken, uit de beide einden van hetzelve.
\par 20 En de cherubim zullen hun beide vleugelen omhoog uitbreiden, bedekkende met hun vleugelen het verzoendeksel; en hun aangezichten zullen tegenover elkander zijn; de aangezichten der cherubim zullen naar het verzoendeksel zijn.
\par 21 En gij zult het verzoendeksel boven op de ark zetten, nadat gij in de ark de getuigenis, die Ik u geven zal, zult gelegd hebben.
\par 22 En aldaar zal Ik bij u komen, en Ik zal met u spreken van boven het verzoendeksel af, van tussen de twee cherubim, die op de ark der getuigenis zijn zullen, alles, wat Ik u gebieden zal aan de kinderen Israels.
\par 23 Gij zult ook een tafel maken van sittimhout; twee ellen zal haar lengte zijn, en een el haar breedte, en een el en een halve zal haar hoogte zijn.
\par 24 En gij zult ze met louter goud overtrekken; gij zult ook een gouden krans daaraan maken, rondom heen.
\par 25 Gij zult ook een lijst rondom daaraan maken, een hand breed; en gij zult een gouden krans rondom derzelver lijst maken.
\par 26 Ook zult gij vier gouden ringen daaraan maken; en gij zult de ringen zetten aan de vier hoeken, die aan derzelver vier voeten zijn zullen.
\par 27 Tegenover de lijst zullen de ringen zijn, tot plaatsen voor de handbomen, om de tafel te dragen.
\par 28 Deze handbomen nu zult gij van sittimhout maken, en gij zult dezelve met goud overtrekken; en de tafel zal daaraan gedragen worden.
\par 29 Gij zult ook maken haar schotelen, en haar rookschalen, en haar platelen, en haar kroezen (met welke zij bedekt zal worden); van louter goud zult gij ze maken.
\par 30 En gij zult op deze tafel altijd het toonbrood voor Mijn aangezicht leggen.
\par 31 Gij zult ook een kandelaar van louter goud maken. Van dicht werk zal deze kandelaar gemaakt worden, zijn schacht, en zijn rietjes; zijn schaaltjes, zijn knopen, en zijn bloemen zullen uit hem zijn.
\par 32 En zes rieten zullen uit zijn zijden uitgaan; drie rieten des kandelaars uit zijn ene zijde, en drie rieten des kandelaars uit zijn andere zijde.
\par 33 In het ene riet zullen drie schaaltjes zijn, gelijk amandelnoten, een knoop en een bloem; en drie schaaltjes, gelijk amandelnoten in een ander riet, een knoop en een bloem; alzo zullen die zes rieten zijn, die uit den kandelaar gaan.
\par 34 Maar aan den kandelaar zelven zullen vier schaaltjes zijn, gelijk amandelnoten, met zijn knopen, en met zijn bloemen.
\par 35 En daar zal een knoop zijn onder twee rieten, uit denzelven uitgaande; ook een knoop onder twee rieten, uit denzelven uitgaande; nog een knoop onder twee rieten, uit denzelven uitgaande; alzo zal het zijn met de zes rieten, die uit den kandelaar uitgaan.
\par 36 Hun knopen en hun rieten zullen uit hem zijn; het zal altemaal een enig dicht werk van louter goud zijn.
\par 37 Gij zult hem ook zeven lampen maken, en men zal zijn lampen aansteken, en doen lichten aan zijn zijden.
\par 38 Zijn snuiters en zijn blusvaten zullen louter goud zijn.
\par 39 Uit een talent louter goud zal men dat maken, met al dit gereedschap.
\par 40 Zie dan toe, dat gij het maakt naar hun voorbeeld, hetwelk u op den berg getoond is.

\chapter{26}

\par 1 Den tabernakel nu zult gij maken van tien gordijnen, van fijn getweernd linnen, en hemelsblauw, en purper, en scharlaken, met cherubim; van het allerkunstelijkste werk zult gij ze maken.
\par 2 De lengte van een gordijn zal van acht en twintig ellen zijn, en de breedte ener gordijn van vier ellen; al deze gordijnen zullen een maat hebben.
\par 3 Er zullen vijf gordijnen samengevoegd zijn, de een aan de andere; wederom zullen er vijf gordijnen samengevoegd zijn, de een aan de andere.
\par 4 En gij zult hemelsblauwe striklisjes maken aan den kant van de ene gordijn, aan het uiterste, in de samenvoeging; alzo zult gij ook doen aan den uitersten kant der gordijn, aan de tweede samenvoegende.
\par 5 Vijftig striklisjes zult gij aan de ene gordijn maken, en vijftig striklisjes zult gij maken aan het uiterste der gordijn, dat aan de tweede samenvoegende is; deze striklisjes zullen het ene aan het andere samenvatten.
\par 6 Gij zult ook vijftig gouden haakjes maken, en zult de gordijnen samenvoegen, de ene aan de andere, met deze haakjes, opdat het een tabernakel zij.
\par 7 Ook zult gij gordijnen uit geiten haar maken tot een tent over den tabernakel; van elf gordijnen zult gij die maken.
\par 8 De lengte ener gordijn zal dertig ellen zijn, en de breedte ener gordijn vier ellen; deze elf gordijnen zullen een maat hebben.
\par 9 En gij zult vijf dezer gordijnen aan elkander bijzonder voegen, en zes dezer gordijnen bijzonder; en de zesde dezer gordijnen zult gij dubbel maken, recht voorop de tent.
\par 10 En gij zult vijftig striklisjes maken aan den kant van de ene gordijn, het uiterste in de samenvoeging, en vijftig striklisjes aan den kant van de gordijn, die de tweede samenvoegende is.
\par 11 Gij zult ook vijftig koperen haakjes maken, en gij zult de haakjes in de striklisjes doen, en gij zult de tent samenvoegen, dat zij een zij.
\par 12 Het overige nu, dat overschiet aan de gordijnen der tent, de helft der gordijn, die overschiet, zal overhangen, aan de achterste delen des tabernakels.
\par 13 En een el van deze, en een el van gene zijde van hetgeen, dat overig zijn zal aan de lengte van de gordijnen der tent, zal overhangen aan de zijden des tabernakels, aan deze en aan gene zijde, om dien te bedekken.
\par 14 Gij zult ook voor de tent een deksel maken van roodgeverfde ramsvellen, en daarover een deksel van dassenvellen.
\par 15 Gij zult ook tot den tabernakel staande berderen maken, van sittimhout.
\par 16 De lengte van een berd zal tien ellen zijn, en een el en een halve el zal de breedte van elk berd zijn.
\par 17 Twee houvasten zal een berd hebben, als sporten in een ladder gezet, het ene nevens het andere; alzo zult gij het met al de berderen des tabernakels maken.
\par 18 En de berderen tot den tabernakel zult gij aldus maken: twintig berderen naar de zuidzijde zuidwaarts.
\par 19 Gij zult ook veertig zilveren voeten maken onder de twintig berderen; twee voeten onder een berd, aan zijn twee houvasten, en twee voeten onder een ander berd, aan zijn twee houvasten.
\par 20 Er zullen ook twintig berderen zijn aan de andere zijde des tabernakels, aan den noorderhoek,
\par 21 Met hun veertig zilveren voeten; twee voeten onder een berd, en twee voeten onder een ander berd.
\par 22 Doch aan de zijde des tabernakels tegen het westen zult gij zes berderen maken.
\par 23 Ook zult gij twee berderen maken tot de hoekberderen des tabernakels, aan de beide zijden.
\par 24 En zij zullen van beneden als tweelingen samengevoegd zijn; zij zullen ook als tweelingen aan het oppereinde deszelven samengevoegd zijn, met een ring; alzo zal het met de twee berderen zijn; tot twee hoekberderen zullen zij zijn.
\par 25 Alzo zullen de acht berderen zijn met hun zilveren voeten, zijnde zestien voeten; twee voeten onder een berd, wederom twee voeten onder een berd.
\par 26 Gij zult ook richelen maken van sittimhout; vijf aan de berderen van de ene zijde des tabernakels;
\par 27 En vijf richelen aan de berderen van de andere zijde des tabernakels; alsook vijf richelen aan de berderen van de zijde des tabernakels, aan de beide zijden westwaarts.
\par 28 En de middelste richel zal midden aan de berderen zijn, doorschietende van het ene einde tot het andere einde.
\par 29 En gij zult de berderen met goud overtrekken, en hun ringen (de plaatsen voor de richelen) zult gij van goud maken; de richelen zult gij ook met goud overtrekken.
\par 30 Dan zult gij den tabernakel oprichten naar zijn wijze, die u op den berg getoond is.
\par 31 Daarna zult gij een voorhang maken, van hemelsblauw, en purper, en scharlaken, en fijn getweernd linnen; van het allerkunstelijkste werk zal men dien maken, met cherubim.
\par 32 En gij zult hem hangen aan vier pilaren van sittim hout, met goud overtogen; hun haken zullen van goud zijn; staande op vier zilveren voeten.
\par 33 En gij zult den voorhang onder de haakjes hangen, en gij zult de ark der getuigenis aldaar binnen den voorhang brengen; en deze voorhang zal ulieden een scheiding maken tussen het heilige, en tussen het heilige der heiligen.
\par 34 En gij zult het verzoendeksel zetten op de ark der getuigenis, in het heilige der heiligen.
\par 35 De tafel nu zult gij zetten buiten den voorhang, en den kandelaar tegen de tafel over, aan de ene zijde des tabernakels, zuidwaarts; maar de tafel zult gij zetten aan de noordzijde.
\par 36 Gij zult ook aan de deur der tent een deksel maken, van hemelsblauw, en purper, en scharlaken, en fijn getweernd linnen, geborduurd werk.
\par 37 En gij zult tot dit deksel vijf pilaren van sittim hout maken, en die met goud overtrekken; hun haken zullen van goud zijn; en gij zult hun vijf koperen voeten gieten.

\chapter{27}

\par 1 Gij zult ook een altaar maken van sittimhout; vijf ellen zal de lengte zijn, en vijf ellen de breedte (vierkant zal dit altaar zijn), en drie ellen zijn hoogte.
\par 2 En gij zult zijn hoornen maken op zijn vier hoeken; uit hetzelve zullen zijn hoornen zijn, en gij zult het met koper overtrekken.
\par 3 Gij zult het ook potten maken, om zijn as te ontvangen, ook zijn schoffelen, en zijn besprengbekkens, en zijn krauwelen, en zijn koolpannen; al zijn gereedschap zult gij van koper maken.
\par 4 Gij zult het een rooster maken van koperen netwerk; en gij zult aan dat net vier koperen ringen maken aan zijn vier einden.
\par 5 En gij zult het onder den omloop des altaars van beneden opleggen, alzo dat het net tot het midden des altaars zij.
\par 6 Gij zult ook handbomen maken tot het altaar, handbomen van sittimhout; en gij zult ze met koper overtrekken.
\par 7 En de handbomen zullen in de ringen gedaan worden, alzo dat de handbomen zijn aan beide zijden des altaars, als men het draagt.
\par 8 Gij zult hetzelve hol van planken maken; gelijk als Hij u op den berg gewezen heeft, alzo zullen zij doen.
\par 9 Gij zult ook den voorhof des tabernakels maken; aan den zuidhoek zuidwaarts, zullen aan den voorhof behangselen zijn van fijn getweernd linnen; de lengte ener zijde zal honderd ellen zijn.
\par 10 Ook zullen zijn twintig pilaren, en derzelver twintig voeten, van koper zijn; de haken dezer pilaren, en hun banden zullen van zilver zijn.
\par 11 Alzo zullen ook aan den noorderhoek, in de lengte, de behangsels honderd ellen lang zijn; en zijn twintig pilaren, en derzelver twintig voeten, van koper; de haken der pilaren, en derzelver banden zullen van zilver zijn.
\par 12 En in de breedte des voorhofs, aan den westerhoek, zullen behangselen zijn van vijftig ellen; hun pilaren tien, en derzelver voeten tien.
\par 13 Van gelijken zal de breedte des voorhofs, aan den oosterhoek oostwaarts, van vijftig ellen zijn.
\par 14 Alzo dat er vijftien ellen der behangselen op de ene zijde zijn; hun pilaren drie, en hun voeten drie;
\par 15 En vijftien ellen der behangselen aan de andere zijde; hun pilaren drie, en hun voeten drie.
\par 16 In de poort nu des voorhofs zal een deksel zijn van twintig ellen, hemelsblauw, en purper, en scharlaken, en fijn getweernd linnen, geborduurd werk; de pilaren vier, en hun voeten vier.
\par 17 Al de pilaren des voorhofs zullen rondom met zilveren banden bezet zijn; hun haken zullen van zilver zijn, maar hun voeten zullen van koper zijn.
\par 18 De lengte des voorhofs zal honderd ellen zijn, en de breedte doorgaans vijftig, en de hoogte vijf ellen, van fijn getweernd linnen; maar hun voeten zullen van koper zijn.
\par 19 Aangaande al het gereedschap des tabernakels, in al deszelfs dienst, ja, al zijn pennen, en al de pennen des voorhofs, zullen van koper zijn.
\par 20 Gij nu zult den kinderen Israels gebieden, dat zij tot u brengen reine olie van olijven, gestoten tot den luchter, dat men geduriglijk de lampen aansteke.
\par 21 In de tent der samenkomst, van buiten den voorhang, die voor de getuigenis is, zal ze Aaron en zijn zonen toerichten, van den avond tot den morgen, voor het aangezicht des HEEREN; dit zal een eeuwige inzetting zijn voor hun geslachten, vanwege de kinderen Israels.

\chapter{28}

\par 1 Daarna zult gij uw broeder Aaron, en zijn zonen met hem, tot u doen naderen uit het midden der kinderen Israels, om Mij het priesterambt te bedienen: namelijk Aaron, Nadab en Abihu, Eleazar en Ithamar, de zonen van Aaron.
\par 2 En gij zult voor uw broeder Aaron heilige klederen maken, tot heerlijkheid en tot sieraad.
\par 3 Gij zult ook spreken tot allen, die wijs van hart zijn, die Ik met den geest der wijsheid vervuld heb, dat zij voor Aaron klederen maken, om hem te heiligen, dat hij Mij het priesterambt bediene.
\par 4 Dit nu zijn de klederen, die zij maken zullen: een borstlap, en een efod, en een mantel, en een rok vol oogjes, een hoed en een gordel; zij zullen dan voor uw broeder Aaron heilige klederen maken, en voor zijn zonen, om Mij het priesterambt te bedienen.
\par 5 Zij zullen ook het goud, en hemelsblauw, en purper, en scharlaken, en fijn linnen nemen;
\par 6 En zullen den efod maken van goud, hemelsblauw, en purper, scharlaken en fijn getweernd linnen, van het allerkunstelijkste werk.
\par 7 Hij zal twee samenvoegende schouderbanden hebben aan zijn beide einden, waarmede hij samengevoegd zal worden.
\par 8 En de kunstelijke riem zijns efods, die op hem is, zal zijn gelijk zijn werk, van hetzelfde, van goud, hemelsblauw en purper, en scharlaken, en fijn getweernd linnen.
\par 9 En gij zult twee sardonixstenen nemen, en de namen der zonen van Israel daarop graveren.
\par 10 Zes van hun namen op een steen, en de zes overige namen op den anderen steen, naar hun geboorten;
\par 11 Naar steensnijderswerk, gelijk men de zegelen graveert, zult gij deze twee stenen graveren, met de namen der zonen van Israel; gij zult ze maken, dat zij omvat zijn in gouden kastjes.
\par 12 En gij zult de twee stenen aan de schouderbanden des efods zetten, zijnde stenen ter gedachtenis voor de kinderen Israels; en Aaron zal hun namen op zijn beide schouders dragen, ter gedachtenis, voor het aangezicht des HEEREN.
\par 13 Gij zult ook gouden kastjes maken,
\par 14 En twee ketentjes van louter goud; gelijk-eindigende zult gij die maken, gedraaid werk; en de gedraaide ketentjes zult gij aan de kastjes hechten.
\par 15 Gij zult ook een borstlap des gerichts maken, van het allerkunstelijkste werk, gelijk het werk des efods zult gij hem maken; van goud, hemelsblauw, en purper, en scharlaken, en van fijn getweernd linnen zult gij hem maken.
\par 16 Vierkant zal hij zijn, en verdubbeld; een span zal zijn lengte zijn, en een span zijn breedte.
\par 17 En gij zult vervullende stenen daarin vullen, vier rijen stenen, een rij van een Sardis, een Topaas en een Karbonkel; dit is de eerste rij.
\par 18 En de tweede rij van een Smaragd, een Saffier, en een Diamant.
\par 19 En de derde rij, een Hyacint, Agaat en Amethist.
\par 20 En de vierde rij van een Turkoois, en een Sardonix, en een Jaspis; zij zullen met goud ingevat zijn in hun vullingen.
\par 21 En deze stenen zullen zijn met de twaalf namen der zonen van Israel, met hun namen; zij zullen als zegelen gegraveerd worden, elk met zijn naam; voor de twaalf stammen zullen zij zijn.
\par 22 Gij zult ook aan den borstlap gelijk-eindigende ketentjes van gedraaid werk uit louter goud maken.
\par 23 Gij zult ook aan den borstlap twee gouden ringen maken; en gij zult de twee ringen aan de twee einden van den borstlap zetten.
\par 24 Dan zult gij de twee gedraaide gouden ketentjes in de twee ringen doen, aan de einden van den borstlap.
\par 25 Maar de twee einden der twee gedraaide ketentjes zult gij aan die twee kastjes doen; en gij zult ze zetten aan de schouderbanden van den efod, recht op de voorste zijde van dien.
\par 26 Gij zult nog twee gouden ringen maken, en zult ze aan de twee einden des borstlaps zetten; inwendig aan zijn rand, die aan de zijde van den efod zijn zal.
\par 27 Nog zult gij twee gouden ringen maken, die gij zetten zult aan de twee schouderbanden van den efod, beneden aan de voorste zijde, tegenover zijn voege, boven den kunstelijken riem des efods.
\par 28 En zij zullen den borstlap met zijn ringen aan de ringen van den efod opwaarts binden, met een hemelsblauw snoer, dat hij op den kunstelijken riem van den efod zij; en de borstlap zal van den efod niet afgescheiden worden.
\par 29 Alzo zal Aaron de namen der zonen van Israel dragen aan den borstlap des gerichts, op zijn hart, als hij in het heilige zal gaan, ter gedachtenis voor het aangezicht des HEEREN geduriglijk.
\par 30 Gij zult ook in den borstlap des gerichts de Urim en de Thummim zetten, dat zij op het hart van Aaron zijn, als hij voor het aangezicht des HEEREN ingaan zal; alzo zal Aaron dat gericht der kinderen Israels geduriglijk op zijn hart dragen, voor het aangezicht des HEEREN.
\par 31 Gij zult ook den mantel des efods geheel van hemelsblauw maken.
\par 32 En het hoofdgat deszelven zal in het midden daarvan zijn; dit gat zal een boord rondom hebben van geweven werk; als het gat eens pantsiers zal het daaraan zijn, dat het niet gescheurd worde.
\par 33 En aan deszelfs zomen zult gij granaatappelen maken van hemelsblauw, en van purper, en van scharlaken, aan zijn zomen rondom, en gouden schelletjes rondom tussen dezelve.
\par 34 Dat er een gouden schelletje, daarna een granaatappel zij; wederom een gouden schelletje, en een granaatappel, aan de zomen des mantels rondom.
\par 35 En Aaron zal denzelven aanhebben, om te dienen; opdat zijn geluid gehoord worde, als hij in het heilige, voor het aangezicht des HEEREN, ingaat, en als hij uitgaat, opdat hij niet sterve.
\par 36 Verder zult gij een plaat maken van louter goud, en gij zult daarin graveren, gelijk men de zegelen graveert: De HEILIGHEID DES HEEREN!
\par 37 En gij zult dezelve aanhechten met een hemelsblauw snoer, alzo dat zij aan den hoed zij; aan de voorste zijde des hoeds zal zij zijn.
\par 38 En zij zal op het voorhoofd van Aaron zijn, opdat Aaron drage de ongerechtigheid der heilige dingen, welke de kinderen Israels zullen geheiligd hebben, in alle gaven hunner geheiligde dingen; en zij zal geduriglijk aan zijn voorhoofd zijn, om henlieden voor het aangezicht des HEEREN aangenaam te maken.
\par 39 Gij zult ook een rok vol oogjes maken, van fijn linnen; gij zult ook den hoed van fijn linnen maken; maar den gordel zult gij van geborduurd werk maken.
\par 40 Voor de zonen van Aaron zult gij ook rokken maken, en gij zult voor hen gordels maken; ook zult gij voor hen mutsen maken, tot heerlijkheid en sieraad.
\par 41 En gij zult die uw broeder Aaron en ook zijn zonen aantrekken; en gij zult hen zalven, en hun hand vullen, en hen heiligen, dat zij Mij het priesterambt bedienen.
\par 42 Maak hun ook linnen onderbroeken, om het vlees der schaamte te bedekken; zij zullen zijn van de lenden tot de dijen.
\par 43 Aaron nu en zijn zonen zullen die aanhebben, als zij in de tent der samenkomst gaan, of als zij tot het altaar treden zullen, om in het heilige te dienen; opdat zij geen ongerechtigheid dragen en sterven. Dit zal een eeuwige inzetting zijn, voor hem, en zijn zaad na hem.

\chapter{29}

\par 1 Dit nu is de zaak, die gij hun doen zult, om hen te heiligen, dat zij Mij het priesterambt bedienen: neem een var, het jong eens runds, en twee volkomen rammen;
\par 2 En ongezuurd brood, en ongezuurde koeken, met olie gemengd, en ongezuurde vladen, met olie bestreken; van tarwemeelbloem zult gij dezelve maken.
\par 3 En gij zult ze in een korf leggen, en zult ze in den korf toebrengen, met den var en de twee rammen.
\par 4 Alsdan zult gij Aaron en zijn zonen doen naderen aan de deur van de tent der samenkomst; en gij zult hen met water wassen.
\par 5 Daarna zult gij de klederen nemen, en Aaron den rok, en den mantel des efods, en den efod, en den borstlap aandoen; en gij zult hem omgorden met den kunstelijken riem des efods.
\par 6 En gij zult den hoed op zijn hoofd zetten; de kroon der heiligheid zult gij aan den hoed zetten.
\par 7 En gij zult de zalfolie nemen, en op zijn hoofd gieten; alzo zult gij hem zalven.
\par 8 Daarna zult gij zijn zonen doen naderen, en zult hen de rokken doen aantrekken.
\par 9 En gij zult hen met den gordel omgorden, namelijk Aaron en zijn zonen; en gij zult hun de mutsen opbinden, opdat zij het priesterambt hebben tot een eeuwige inzetting. Voorts zult gij de hand van Aaron vullen, en de hand zijner zonen.
\par 10 En gij zult den var nabij brengen voor de tent der samenkomst; en Aaron en zijn zonen zullen hun handen op het hoofd van den var leggen.
\par 11 En gij zult den var slachten voor het aangezicht des HEEREN, voor de deur van de tent der samenkomst.
\par 12 Daarna zult gij van het bloed des vars nemen, en met uw vinger op de hoornen des altaars doen; en al het bloed zult gij uitgieten aan den bodem des altaars.
\par 13 Gij zult ook al het vet nemen, hetwelk het ingewand bedekt, en het net over de lever, en beide nieren en het vet, dat aan dezelve is, en gij zult ze aansteken op het altaar.
\par 14 Maar het vlees des vars, en zijn vel, en zijn drek, zult gij met vuur verbranden, buiten het leger; het is een zondoffer.
\par 15 Daarna zult gij den enen ram nemen, en Aaron en zijn zonen zullen hun handen op het hoofd des rams leggen;
\par 16 En gij zult den ram slachten, en gij zult zijn bloed nemen, en rondom op het altaar sprengen.
\par 17 En den ram zult gij in zijn delen delen; en gij zult zijn ingewand en zijn schenkelen wassen, en op zijn delen, en op zijn hoofd leggen.
\par 18 Alzo zult gij den gehelen ram aansteken op het altaar; het is een brandoffer den HEERE, tot een liefelijken reuk, het is een vuuroffer den HEERE.
\par 19 Daarna zult gij den anderen ram nemen, en Aaron en zijn zonen zullen hun handen op des rams hoofd leggen;
\par 20 En gij zult den ram slachten, en van zijn bloed nemen, en doen het op het rechter oorlapje van Aaron, en op het rechteroorlapje van zijn zonen, desgelijks op den duim hunner rechterhand, en op den groten teen huns rechtervoets; en dat bloed zult gij op het altaar sprengen, rondom heen.
\par 21 Dan zult gij nemen van het bloed, dat op het altaar is, en van de zalfolie, en gij zult op Aaron en op zijn klederen sprengen, en op zijn zonen en op de klederen zijner zonen met hem; opdat hij geheiligd zij, en zijn klederen, ook zijn zonen, en de klederen zijner zonen met hem.
\par 22 Daarna zult gij van den ram nemen het vet mitsgaders den staart, ook het vet, dat het ingewand bedekt, en het net der lever en de beide nieren, met het vet, dat aan dezelve is, en den rechterschouder; want het is een ram der vulofferen;
\par 23 En een broodbol, en een koek geolied brood, en een vlade, uit den korf der ongezuurde broden, die voor het aangezicht des HEEREN zijn zal;
\par 24 En leg ze alle op de handen van Aaron, en op de handen zijner zonen, en beweeg ze ten beweegoffer voor het aangezicht des HEEREN.
\par 25 Neem ze daarna van hun hand, en steek ze aan op het altaar, op het brandoffer, tot een liefelijken reuk voor het aangezicht des HEEREN; het is een vuuroffer den HEERE.
\par 26 En neem de borst van den ram der vulofferen, die van Aaron is, en beweeg hem ten beweegoffer voor het aangezicht des HEEREN; en het zal u ten dele zijn.
\par 27 En gij zult de borst des beweegoffers heiligen, en den schouder des hefoffers, die bewogen, en die opgeheven zal zijn van den ram des vuloffers, van hetgeen dat Aarons, en van hetgeen dat zijner zonen is.
\par 28 En het zal voor Aaron en zijn zonen zijn tot een eeuwige inzetting vanwege de kinderen Israels; want het is een hefoffer; en het hefoffer vanwege de kinderen Israels zal zijn van hun dankofferen; hun hefoffer zal voor den HEERE zijn.
\par 29 De heilige klederen nu, die van Aaron zullen geweest zijn, zullen van zijn zonen na hem zijn, opdat men hen in dezelve zalve, en dat men hun hand in dezelve vulle.
\par 30 Zeven dagen zal hij ze aantrekken, die uit zijn zonen in zijn plaats priester zal worden, die in de tent der samenkomst gaan zal, om in het heilige te dienen.
\par 31 Gij zult den ram der vulling nemen, en gij zult zijn vlees in de heilige plaats zieden.
\par 32 Aaron nu en zijn zonen zullen het vlees van dezen ram eten, en het brood, dat in den korf zal zijn, bij de deur van de tent der samenkomst.
\par 33 En zij zullen die dingen eten, met welke de verzoening zal gedaan zijn, om hun hand te vullen, en om hen te heiligen; maar een vreemde zal ze niet eten, want ze zijn heilig.
\par 34 En indien er wat overblijven zal van het vlees der vulofferen, of van dit brood, tot aan den morgen, zo zult gij het overgeblevene met vuur verbranden; het zal niet gegeten worden, want het is heilig.
\par 35 Gij zult dan aan Aaron en aan zijn zonen alzo doen, naar alles, wat Ik u geboden heb; zeven dagen zult gij hun hand vullen.
\par 36 Gij zult ook des daags een var des zondoffers bereiden, tot de verzoeningen, en gij zult het altaar ontzondigen, mits doende de verzoening over hetzelve; en gij zult het zalven, om het te heiligen.
\par 37 Zeven dagen zult gij verzoening doen voor het altaar, en zult het heiligen; alsdan zal dat altaar een heiligheid der heiligheden zijn; al wat het altaar aanroert, zal heilig zijn.
\par 38 Dit nu is het, wat gij op het altaar bereiden zult: twee lammeren, die eenjarig zijn, des daags, geduriglijk.
\par 39 Het ene lam zult gij des morgens bereiden; maar het andere lam zult gij bereiden tussen de twee avonden.
\par 40 Met een tiende deel meelbloem, gemengd met een vierendeel van een hin gestoten olie; en tot drankoffer een vierde deel van een hin wijn, tot het ene lam.
\par 41 Het andere lam nu zult gij bereiden tussen de twee avonden; gij zult daarmede doen gelijk met het morgenspijsoffer, en gelijk met het drankoffer deszelven, tot een liefelijken reuk; het is een vuuroffer den HEERE.
\par 42 Het zal een gedurig brandoffer zijn bij uw geslachten, aan de deur van de tent der samenkomst, voor het aangezicht des HEEREN; aldaar zal Ik met ulieden komen, dat Ik aldaar met u spreke.
\par 43 En daar zal Ik komen tot de kinderen Israels; opdat zij geheiligd worden door Mijn heerlijkheid.
\par 44 En Ik zal de tent der samenkomst heiligen, mitsgaders het altaar; Ik zal ook Aaron en zijn zonen heiligen, opdat zij Mij het priesterambt bedienen.
\par 45 En Ik zal in het midden der kinderen Israels wonen, en Ik zal hun tot een God zijn.
\par 46 En zij zullen weten, dat Ik de HEERE hun God ben, Die hen uit Egypteland uitgevoerd heb, opdat Ik in het midden van hen wonen zou; Ik ben de HEERE, hun God.

\chapter{30}

\par 1 Gij zult ook een reukaltaar des reukwerks maken; van sittimhout zult gij het maken.
\par 2 Een el zal zijn lengte zijn, en een el zijn breedte, vierkant zal het zijn, maar twee ellen deszelfs hoogte; uit hetzelve zullen zijn hoornen zijn.
\par 3 En gij zult het met louter goud overtrekken, zijn dak en deszelfs wanden rondom, als ook zijn hoornen; en gij zult het een gouden krans rondom maken.
\par 4 Gij zult ook twee gouden ringen daaraan maken, onder zijn krans; aan zijn twee zijden zult gij dezelve maken, aan zijn beide zijden; en zij zullen zijn tot plaatsen voor de handbomen, dat men het daarmede drage.
\par 5 De draagbomen nu zult gij van sittimhout maken, en gij zult die met goud overtrekken.
\par 6 En gij zult het zetten voor den voorhang, die voor de ark der getuigenis zijn zal; voor het verzoendeksel, hetwelk zijn zal boven de getuigenis, waarheen Ik met u samenkomen zal.
\par 7 En Aaron zal daarop aansteken welriekende specerijen; allen morgen, als hij de lampen wel zal toegericht hebben, zal hij dezelve aansteken.
\par 8 En als Aaron de lampen aansteken zal, tussen de twee avonden, zal hij dat aansteken; het zal een gedurig reukwerk zijn, voor het aangezicht des HEEREN, bij uw geslachten.
\par 9 Gij zult geen vreemd reukwerk op hetzelve aansteken, noch brandoffer, noch spijsoffer; gij zult ook geen drankoffer daarop gieten.
\par 10 En Aaron zal eens in het jaar over deszelfs hoornen verzoening doen, met het bloed des zondoffers der verzoeningen; eens in het jaar zal hij verzoening daarop doen bij uw geslachten; het is heiligheid der heiligheden den HEERE!
\par 11 Verder sprak de HEERE tot Mozes, zeggende:
\par 12 Als gij de som van de kinderen Israels opnemen zult, naar de getelden onder hen, zo zullen zij een iegelijk de verzoening zijner ziel den HEERE geven, als gij hen tellen zult; opdat onder hen geen plage zij, als gij hen tellen zult.
\par 13 Dit zullen zij geven, al die tot de getelden overgaat, de helft eens sikkels, naar den sikkel des heiligdoms (deze sikkel is twintig gera); de helft eens sikkels is een hefoffer den HEERE.
\par 14 Al wie overgaat tot de getelden, van twintig jaren oud en daarboven, zal het hefoffer des HEEREN geven.
\par 15 De rijke zal het niet vermeerderen, en de arme zal niet verminderen van de helft des sikkels, als gij het hefoffer des HEEREN geeft om voor uw zielen verzoening te doen.
\par 16 Gij dan zult het geld der verzoeningen van de kinderen Israels nemen, en zult het leggen tot den dienst van de tent der samenkomst; en het zal den kinderen Israels ter gedachtenis zijn, voor het aangezicht des HEEREN, om voor uw zielen verzoening te doen.
\par 17 En de HEERE sprak tot Mozes, zeggende:
\par 18 Gij zult ook een koperen wasvat maken, met zijn koperen voet, om te wassen; en gij zult het zetten tussen de tent der samenkomst, en tussen het altaar, en gij zult water daarin doen;
\par 19 Dat Aaron en zijn zonen zich daaruit wassen, hun handen en voeten.
\par 20 Wanneer zij in de tent der samenkomst zullen gaan, zo zullen zij zich met water wassen, opdat zij niet sterven; of wanneer zij tot het altaar naderen, om te dienen, dat zij het vuuroffer den HEERE aansteken;
\par 21 Zij zullen dan hun handen en voeten wassen, opdat zij niet sterven; en dit zal hun een eeuwige inzetting zijn, voor hem en zijn zaad, bij hun geslachten.
\par 22 Verder sprak de HEERE tot Mozes, zeggende:
\par 23 Gij nu, neem u de voornaamste specerijen, de zuiverste mirre, vijfhonderd sikkels, en specerijkaneel, half zoveel namelijk tweehonderd en vijftig sikkels, ook specerijkalmus, tweehonderd en vijftig sikkels;
\par 24 Ook kassie, vijfhonderd, naar den sikkel des heiligdoms, en olie van olijfbomen een hin;
\par 25 En maak daarvan een olie der heilige zalving, een zalf, heel kunstiglijk gemaakt, naar apothekerswerk; het zal een olie der heilige zalving zijn.
\par 26 En met dezelve zult gij zalven de tent der samenkomst, en de ark der getuigenis.
\par 27 En de tafel met al haar gereedschap, en den kandelaar met zijn gereedschap, en het reukaltaar;
\par 28 En het altaar des brandoffers, met al zijn gereedschap, en het wasvat met zijn voet.
\par 29 Gij zult ze alzo heiligen, dat zij heiligheid der heiligheden zijn; al wat ze aanroert, zal heilig zijn.
\par 30 Gij zult ook Aaron en zijn zonen zalven, en gij zult hen heiligen, om Mij het priesterambt te bedienen.
\par 31 En gij zult tot de kinderen Israels spreken, zeggende: Dit zal Mij een olie der heilige zalving zijn bij uw geslachten.
\par 32 Op geens mensen vlees zal men ze gieten; gij zult ook naar haar maaksel geen dergelijke maken; het is heiligheid, zij zal ulieden heiligheid zijn.
\par 33 De man, die zulk een zalf maken zal als deze, of die daarvan op wat vreemds doet, die zal uitgeroeid worden uit zijn volken.
\par 34 Verder zeide de HEERE tot Mozes: Neem tot u welriekende specerijen, mirresap, en oniche, en galban, deze welriekende specerijen, en zuiveren wierook; dat elk bijzonder zij.
\par 35 En gij zult een reukwerk ener zalf daaruit maken, naar het werk des apothekers, gemengd, rein, heilig.
\par 36 En gij zult van hetzelve heel klein pulver stoten, en gij zult daarvan leggen voor de getuigenis in de tent der samenkomst, waarheen Ik tot u komen zal; het zal ulieden heiligheid der heiligheden zijn.
\par 37 Doch naar het maaksel dezes reukwerks, hetwelk gij gemaakt zult hebben, zult gijlieden voor uzelven geen maken; het zal u heiligheid zijn voor den HEERE.
\par 38 De man, die dergelijke maken zal, om daaraan te rieken, die zal uitgeroeid worden uit zijn volken.

\chapter{31}

\par 1 Daarna sprak de HEERE tot Mozes, zeggende:
\par 2 Zie, Ik heb met name geroepen Bezaleel, den zoon van Uri, den zoon van Hur, van den stam van Juda.
\par 3 En Ik heb hem vervuld met den Geest Gods, met wijsheid, en met verstand, en met wetenschap, namelijk in alle handwerk;
\par 4 Om te bedenken vernuftigen arbeid; te werken in goud, en in zilver, en in koper,
\par 5 En in kunstige steensnijding, om in te zetten, en in kunstige houtsnijding, om te werken in alle handwerk.
\par 6 En Ik, zie, Ik heb hem bijgevoegd Aholiab, den zoon van Ahisamach, van den stam van Dan; en in het hart van een iegelijk, die wijs van hart is, heb Ik wijsheid gegeven; en zij zullen maken al wat Ik u geboden heb.
\par 7 Namelijk de tent der samenkomst, en de ark der getuigenis, en het verzoendeksel, dat daarop zal zijn, en al het gereedschap der tent;
\par 8 En de tafel, met haar gereedschap; en den louteren kandelaar, met al zijn gereedschap; en het reukaltaar;
\par 9 Ook des brandoffers altaar, met al zijn gereedschap; en het wasvat met zijn voet;
\par 10 En de ambtsklederen, en de heilige klederen van den priester Aaron, en de klederen van zijn zonen, om het priesterambt te bedienen;
\par 11 Ook de zalfolie, en het reukwerk van welriekende specerijen voor het heiligdom; naar alles, wat Ik u geboden heb, zullen zij het maken.
\par 12 Verder sprak de HEERE tot Mozes, zeggende:
\par 13 Gij nu, spreek tot de kinderen Israels, zeggende: Gij zult evenwel mijn sabbatten onderhouden; want dit is een teken tussen Mij en tussen ulieden, bij uw geslachten; opdat men wete, dat Ik de HEERE ben, Die u heilige.
\par 14 Onderhoudt dan den sabbat, dewijl hij ulieden heilig is! Wie hem ontheiligt, zal zekerlijk gedood worden; want een ieder, die op denzelven enig werk doet, die ziel zal uitgeroeid worden uit het midden harer volken.
\par 15 Zes dagen zal men het werk doen; doch op den zevenden dag is den sabbat der rust, een heiligheid des HEEREN! Wie op den sabbatdag arbeid doet, zal zekerlijk gedood worden.
\par 16 Dat dan de kinderen Israels den sabbat houden, den sabbat onderhoudende in hun geslachten, tot een eeuwig verbond.
\par 17 Hij zal tussen Mij en tussen de kinderen Israels een teken in eeuwigheid zijn; dewijl de HEERE, in zes dagen, den hemel en de aarde gemaakt, en op den zevenden dag gerust en Zich verkwikt heeft.
\par 18 En Hij gaf aan Mozes, als Hij met hem op den berg Sinai te spreken geeindigd had, de twee tafelen der getuigenis, tafelen van steen, beschreven met den vinger Gods.

\chapter{32}

\par 1 Toen het volk zag, dat Mozes vertoog van den berg af te komen, zo verzamelde zich het volk tot Aaron, en zij zeiden tot hem: Sta op, maak ons goden, die voor ons aangezicht gaan; want dezen Mozes, dien man, die ons uit Egypteland uitgevoerd heeft, wij weten niet, wat hem geschied zij.
\par 2 Aaron nu zeide tot hen: Rukt af de gouden oorsierselen, die in de oren uwer vrouwen, uwer zonen, en uwer dochteren zijn; en brengt ze tot mij.
\par 3 Toen rukte het ganse volk de gouden oorsierselen af, die in hun oren waren; en zij brachten ze tot Aaron.
\par 4 En hij nam ze uit hun hand, en hij bewierp het met een griffie, en hij maakte een gegoten kalf daaruit. Toen zeiden zij: Dit zijn uw goden, Israel! die u uit Egypteland opgevoerd hebben.
\par 5 Als Aaron dat zag, zo bouwde hij een altaar voor hetzelve; en Aaron riep uit, en zeide: Morgen zal den HEERE een feest zijn!
\par 6 En zij stonden des anderen daags vroeg op, en offerden brandoffer, en brachten dankoffer daartoe; en het volk zat neder om te eten en te drinken; daarna stonden zij op, om te spelen.
\par 7 Toen sprak de HEERE tot Mozes: Ga heen, klim af! want uw volk, dat gij uit Egypteland opgevoerd hebt, heeft het verdorven.
\par 8 En zij zijn haast afgeweken van den weg, dien Ik hun geboden had, zij hebben zich een gegoten kalf gemaakt; en zij hebben zich voor hetzelve gebogen, en hebben het offerande gedaan, en gezegd: Dit zijn uw goden, Israel, die u uit Egypteland opgevoerd hebben.
\par 9 Verder zeide de HEERE tot Mozes: Ik heb dit volk gezien, en zie, het is een hardnekkig volk!
\par 10 En nu, laat Mij toe, dat Mijn toorn tegen hen ontsteke, en hen vertere; zo zal Ik u tot een groot volk maken.
\par 11 Doch Mozes aanbad het aangezicht des HEEREN zijns Gods, en hij zeide: O HEERE! waarom zou Uw toorn ontsteken tegen Uw volk, hetwelk Gij met grote kracht, en met een sterke hand, uit Egypteland uitgevoerd hebt?
\par 12 Waarom zouden de Egyptenaars spreken, zeggende: In kwaadheid heeft Hij hen uitgevoerd, opdat Hij hen doodde op de bergen, en opdat Hij hen vernielde van den aardbodem? Keer af van de hittigheid Uws toorns, en laat het U over het kwaad Uws volks berouwen.
\par 13 Gedenk aan Abraham, aan Izak en aan Israel, Uw knechten, aan welke Gij bij Uzelven gezworen hebt, en hebt tot hen gesproken: Ik zal uw zaad vermenigvuldigen als de sterren des hemels; en dit gehele land, waarvan Ik gezegd heb, zal Ik aan ulieder zaad geven, dat zij het erfelijk bezitten in eeuwigheid.
\par 14 Toen berouwde het den HEERE over het kwaad, hetwelk Hij gesproken had Zijn volk te zullen doen.
\par 15 En Mozes wendde zich om, en klom van den berg af, met de twee tafelen der getuigenis in zijn hand; deze tafelen waren op haar beide zijden beschreven, zij waren op de ene en op de andere zijde beschreven.
\par 16 En diezelfde tafelen waren Gods werk; het geschrift was ook Gods geschrift zelf, in de tafelen gegraveerd.
\par 17 Toen nu Jozua des volks stem hoorde, als het juichte, zo zeide hij tot Mozes: Er is een krijgsgeschrei in het leger.
\par 18 Maar hij zeide: Het is geen stem des geroeps van overwinning, het is ook geen stem des geroeps van nederlaag; ik hoor een stem van zingen bij beurte.
\par 19 En het geschiedde, als hij aan het leger naderde, en het kalf, en de reien zag, dat de toorn van Mozes ontstak, en dat hij de tafelen uit zijn handen wierp, en dezelve beneden aan den berg verbrak.
\par 20 En hij nam dat kalf, dat zij gemaakt hadden, en verbrandde het in het vuur, en vermaalde het, totdat het klein werd, en strooide het op het water, en deed het den kinderen Israels drinken.
\par 21 En Mozes zeide tot Aaron: Wat heeft u dit volk gedaan, dat gij zulk een grote zonde over hetzelve gebracht hebt?
\par 22 Toen zeide Aaron: De toorn mijns heren ontsteke niet! gij kent dit volk, dat het in den boze ligt.
\par 23 Zij dan zeiden tot mij: Maak ons goden, die voor ons aangezicht gaan, want dezen Mozes, dien man, die ons uit Egypteland opgevoerd heeft, wij weten niet, wat hem geschied zij.
\par 24 Toen zeide ik tot hen: Wie goud heeft, die rukke het af, en geve het mij; en ik wierp het in het vuur, en dit kalf is er uit gekomen.
\par 25 Als Mozes zag, dat het volk ontbloot was, (want Aaron had het ontbloot tot verkleining onder degenen, die tegen hen hadden mogen opstaan),
\par 26 Zo bleef Mozes staan in de poort des legers, en zeide: Wie den HEERE toebehoort, kome tot mij! Toen verzamelden zich tot hem al de zonen van Levi.
\par 27 En hij zeide tot hen: Alzo zegt de HEERE, de God van Israel: Een ieder doe zijn zwaard aan zijn heup; gaat door en keert weder, van poort tot poort in het leger, en een iegelijk dode zijn broeder, en elk zijn vriend, en elk zijn naaste!
\par 28 En de zonen van Levi deden naar het woord van Mozes; en er vielen van het volk, op dien dag, drie duizend man.
\par 29 Want Mozes had gezegd: Vult heden uw handen den HEERE; want elk zal zijn tegen zijn zoon, en tegen zijn broeder; en dit, opdat Hij heden een zegen over ulieden geve!
\par 30 En het geschiedde des anderen daags, dat Mozes tot het volk zeide: Gijlieden hebt een grote zonde gezondigd; doch nu, ik zal tot den HEERE opklimmen; misschien zal ik een verzoening doen voor uw zonde.
\par 31 Zo keerde Mozes weder tot den HEERE, en zeide: Och, dit volk heeft een grote zonde gezondigd, dat zij zich gouden goden gemaakt hebben.
\par 32 Nu dan, indien Gij hun zonden vergeven zult! doch zo niet, zo delg mij nu uit Uw boek, hetwelk Gij geschreven hebt.
\par 33 Toen zeide de HEERE tot Mozes: Dien zou Ik uit Mijn boek delgen, die aan Mij zondigt.
\par 34 Doch ga nu heen, leid dit volk, waarheen Ik u gezegd heb; zie, Mijn Engel zal voor uw aangezicht gaan! doch ten dage Mijns bezoekens, zo zal Ik hun zonde over hen bezoeken!
\par 35 Aldus plaagde de HEERE dit volk, omdat zij dat kalf gemaakt hadden, hetwelk Aaron gemaakt had.

\chapter{33}

\par 1 Voorts sprak de HEERE tot Mozes: Ga heen, trek op van hier, gij en het volk, dat gij uit Egypteland opgevoerd hebt, naar het land, dat Ik Abraham, Izak en Jakob gezworen heb, zeggende: Aan uw zaad zal Ik het geven;
\par 2 En Ik zal een Engel voor uw aangezicht zenden (en Ik zal uitdrijven de Kanaanieten, de Amorieten, en de Hethieten, en de Ferezieten, de Hevieten, en de Jebusieten),
\par 3 Naar het land, dat van melk en honig is vloeiende; want Ik zal in het midden van u niet optrekken; want gij zijt een hardnekkig volk; dat Ik u op dezen weg niet vertere.
\par 4 Toen het volk dit kwade woord hoorde, zo droegen zij leed; en niemand van hen deed zijn versiersel aan zich.
\par 5 En de HEERE had tot Mozes gezegd: Zeg tot de kinderen Israels: Gij zijt een hardnekkig volk; in een ogenblik zou Ik in het midden van ulieden optrekken, en zou u vernielen; doch nu, legt uw sieraad van u af, en Ik zal weten, wat Ik u doen zal.
\par 6 De kinderen Israels dan beroofden zichzelven van hun versierselen, verre van den berg Horeb.
\par 7 En Mozes nam de tent, en spande ze zich buiten het leger, ver van het leger afwijkende; en hij noemde ze de Tent der samenkomst. En het geschiedde, dat al wie den HEERE zocht, uitging tot de tent der samenkomst, die buiten het leger was.
\par 8 En het geschiedde, wanneer Mozes uitging naar de tent, stond al het volk op, en een ieder stelde zich in de deur zijner tent; en zij zagen Mozes na, totdat hij de tent ingegaan was.
\par 9 En het geschiedde, als Mozes de tent ingegaan was, zo kwam de wolkkolom nederwaarts, en stond in de deur der tent, en Hij sprak met Mozes.
\par 10 Als het volk de wolkkolom zag staan in de deur der tent, zo stond al het volk op, en zij bogen zich, een ieder in de deur zijner tent.
\par 11 En de HEERE sprak tot Mozes aangezicht aan aangezicht, gelijk een man met zijn vriend spreekt; daarna keerde hij weder tot het leger; doch zijn dienaar Jozua, de zoon van Nun, de jongeling, week niet uit het midden der tent.
\par 12 En Mozes zeide tot den HEERE: Zie, Gij zegt tot mij: Voer dit volk op! maar Gij laat mij niet weten, wien Gij met mij zult zenden; daar Gij gezegd hebt: Ik ken u bij name! en ook: Gij hebt genade gevonden in Mijn ogen!
\par 13 Nu dan, ik bidde, indien ik genade gevonden heb in Uw ogen, zo laat mij nu Uw weg weten, en ik zal U kennen, opdat ik genade vinde in Uw ogen; en zie aan, dat deze natie Uw volk is!
\par 14 Hij dan zeide: Zou Mijn aangezicht moeten medegaan, om u gerust te stellen?
\par 15 Toen zeide hij tot Hem: Indien Uw aangezicht niet medegaan zal, doe ons van hier niet optrekken!
\par 16 Want waarbij zou nu bekend worden, dat ik genade gevonden heb in Uw ogen, ik en Uw volk? Is het niet daarbij, dat Gij met ons gaat? Alzo zullen wij afgezonderd worden, ik en Uw volk, van alle volk, dat op den aardbodem is.
\par 17 Toen zeide de HEERE tot Mozes: Ook deze zelfde zaak, die gij gesproken hebt, zal Ik doen, dewijl gij genade gevonden hebt in Mijn ogen, en Ik u bij name ken.
\par 18 Toen zeide hij: Toon mij nu Uw heerlijkheid!
\par 19 Doch Hij zeide: Ik zal al Mijn goedigheid voorbij uw aangezicht laten gaan, en zal den Naam des HEEREN uitroepen voor uw aangezicht; maar Ik zal genadig zijn, wien Ik zal genadig zijn, en Ik zal Mij ontfermen, over wien Ik Mij ontfermen zal.
\par 20 Hij zeide verder: Gij zoudt Mijn aangezicht niet kunnen zien; want Mij zal geen mens zien, en leven.
\par 21 De HEERE zeide verder: Zie, er is een plaats bij Mij; daar zult gij u op de steenrots stellen.
\par 22 En het zal geschieden, wanneer Mijn heerlijkheid voorbij zal gaan, zo zal Ik u in een kloof der steenrots zetten; en Ik zal u met Mijn hand overdekken, totdat Ik zal voorbijgegaan zijn.
\par 23 En wanneer Ik Mijn hand zal weggenomen hebben, zo zult gij Mijn achterste delen zien; maar Mijn aangezicht zal niet gezien worden!

\chapter{34}

\par 1 Toen zeide de HEERE tot Mozes: Houw u twee stenen tafelen, gelijk de eerste waren, zo zal Ik op de tafelen schrijven dezelfde woorden, die op de eerste tafelen geweest zijn, die gij gebroken hebt.
\par 2 En wees bereid tegen den morgenstond; dat gij in den morgenstond op den berg Sinai klimt, en stel u aldaar voor Mij, op den top des bergs.
\par 3 En niemand zal met u opklimmen; dat er ook niemand gezien worde op den gansen berg; ook het kleine vee, noch runderen zullen tegenover dezen berg niet weiden.
\par 4 Toen hieuw hij twee stenen tafelen, gelijk de eerste; en Mozes stond des morgens vroeg op, en klom op den berg Sinai, gelijk als hem de HEERE geboden had; en hij nam de twee stenen tafelen in zijn hand.
\par 5 De HEERE nu kwam nederwaarts in een wolk, en stelde Zich aldaar bij hem; en Hij riep uit den Naam des HEEREN.
\par 6 Als nu de HEERE voor zijn aangezicht voorbijging, zo riep Hij: HEERE, HEERE, God, barmhartig en genadig, lankmoedig en groot van weldadigheid en waarheid.
\par 7 Die de weldadigheid bewaart aan vele duizenden, Die de ongerechtigheid, en overtreding, en zonde vergeeft; Die den schuldige geenszins onschuldig houdt, bezoekende de ongerechtigheid der vaderen aan de kinderen, en aan de kindskinderen, in het derde en vierde lid.
\par 8 Mozes nu haastte zich en neigde het hoofd ter aarde, en hij boog zich.
\par 9 En hij zeide: Heere! indien ik nu genade gevonden heb in Uw ogen, zo ga nu de Heere in het midden van ons, want dit is een hardnekkig volk; doch vergeef onze ongerechtigheid en onze zonde, en neem ons aan tot een erfdeel!
\par 10 Toen zeide Hij: Zie, Ik maak een verbond; voor uw ganse volk zal Ik wonderen doen, die niet geschapen zijn op de ganse aarde, noch onder enige volken; alzo dat dit ganse volk, in welks midden gij zijt, des HEEREN werk zien zal, dat het schrikkelijk is, hetwelk Ik met u doe.
\par 11 Onderhoudt gij hetgeen Ik u heden gebiede! zie, Ik zal voor uw aangezicht uitdrijven de Amorieten, en de Kanaanieten, en de Hethieten, en de Ferezieten, en de Hevieten, en de Jebusieten.
\par 12 Wacht u, dat gij toch geen verbond maakt met den inwoner des lands, waarin gij komen zult; dat hij misschien niet tot een strik worde in het midden van u.
\par 13 Maar hun altaren zult gijlieden omwerpen, en hun opgerichte beelden zult gij verbreken, en hun bossen zult gij afhouwen.
\par 14 (Want gij zult u niet buigen voor een anderen god; want des HEEREN Naam is Ijveraar! een ijverig God is Hij!)
\par 15 Opdat gij misschien geen verbond maakt met den inwoner van dat land; en zij hun goden niet nahoereren, noch hun goden offerande doen, en hij u nodigende, gij van hun offerande etet.
\par 16 En gij voor uw zonen vrouwen neemt van hun dochteren; en hun dochteren, haar goden nahoererende, maken, dat ook uw zonen haar goden nahoereren.
\par 17 Gij zult u geen gegoten goden maken.
\par 18 Het feest der ongezuurde broden zult gij houden; zeven dagen zult gij ongezuurde broden eten, gelijk Ik u geboden heb, ter gezetter tijd der maand Abib; want in de maand Abib zijt gij uit Egypte uitgegaan.
\par 19 Al wat de baarmoeder opent, is Mijn; ja, al uw vee, dat mannelijk zal geboren worden, openende de baarmoeder van het grote en kleine vee.
\par 20 Doch den ezel, die de baarmoeder opent, zult gij met een stuk klein vee lossen; maar indien gij hem niet zult lossen, zo zult gij hem den nek breken. Al de eerstgeborenen uwer zonen zult gij lossen, en men zal voor Mijn aangezicht niet ledig verschijnen.
\par 21 Zes dagen zult gij arbeiden, maar op den zevenden dag zult gij rusten; in den ploegtijd en in den oogst zult gij rusten.
\par 22 Het feest der weken zult gij ook houden, zijnde het feest der eerstelingen van den tarweoogst, en het feest der inzameling, als het jaar om is.
\par 23 Al wat mannelijk is onder u zal driemaal in het jaar verschijnen voor het aangezicht des Heeren HEEREN, den God van Israel.
\par 24 Wanneer Ik de volken voor uw aangezicht uit de bezitting zal verdrijven, en uw landpalen verwijden, dan zal niemand uw land begeren, terwijl gij henen opgaan zult, om te verschijnen voor het aangezicht des HEEREN uws Gods, driemaal in het jaar.
\par 25 Gij zult het bloed van Mijn slachtoffer niet offeren met gedesemd brood; het slachtoffer van het paasfeest zal ook niet vernachten tot den morgen.
\par 26 De eerstelingen van de eerste vruchten uws lands zult gij in het huis des HEEREN uws Gods brengen. Gij zult het bokje in de melk zijner moeder niet koken.
\par 27 Verder zeide de HEERE tot Mozes: Schrijf u deze woorden; want naar luid dezer woorden heb Ik een verbond met u en met Israel gemaakt.
\par 28 En hij was aldaar met den HEERE, veertig dagen en veertig nachten; hij at geen brood, en hij dronk geen water; en Hij schreef op de tafelen de woorden des verbonds, de tien woorden.
\par 29 En het geschiedde, toen Mozes van den berg Sinai afging (de twee tafelen der getuigenis nu waren in de hand van Mozes, als hij van den berg afging), zo wist Mozes niet, dat het vel zijns aangezichts glinsterde, toen Hij met hem sprak.
\par 30 Als nu Aaron en al de kinderen Israels Mozes aanzagen, ziet, zo glinsterde het vel zijns aangezichts; daarom vreesden zij tot hem toe te treden.
\par 31 Toen riep Mozes hen; en Aaron, en al de oversten in de vergadering keerden weder tot hem; en Mozes sprak tot hen.
\par 32 En daarna traden al de kinderen Israels toe; en hij gebood hun al wat de HEERE met hem gesproken had op den berg Sinai.
\par 33 Alzo eindigde Mozes met hen te spreken, en hij had een deksel op zijn aangezicht gelegd.
\par 34 Doch als Mozes voor het aangezicht des HEEREN kwam, om met Hem te spreken, zo nam hij het deksel af, totdat hij uitging; en nadat hij uitgegaan was, zo sprak hij tot de kinderen Israels, wat hem geboden was.
\par 35 Zo zagen dan de kinderen Israels het aangezicht van Mozes, dat het vel van het aangezicht van Mozes glinsterde; derhalve deed Mozes het deksel weder op zijn aangezicht, totdat hij inging om met Hem te spreken.

\chapter{35}

\par 1 Toen deed Mozes de ganse vergadering der kinderen Israels verzamelen, en zeide tot hen: Dit zijn de woorden, die de HEERE geboden heeft, dat men ze doe.
\par 2 Zes dagen zal men het werk doen; maar op den zevenden dag zal ulieden heiligheid zijn, een sabbat der rust den HEERE; al wie daarop werk doet, zal gedood worden.
\par 3 Gij zult geen vuur aansteken in enige uwer woningen op den sabbatdag.
\par 4 Verder sprak Mozes tot de ganse vergadering der kinderen Israels, zeggende: Dit is het woord, dat de HEERE geboden heeft, zeggende:
\par 5 Neemt van hetgeen, dat gijlieden hebt, een hefoffer den HEERE; een ieder, wiens hart vrijwillig is, zal het brengen, ten hefoffer des HEEREN: goud, en zilver, en koper;
\par 6 Als ook hemelsblauw, en purper, en scharlaken, en fijn linnen, en geiten haar;
\par 7 En roodgeverfde ramsvellen, en dassenvellen, en sittimhout;
\par 8 En olie tot den luchter, en specerijen ter zalfolie, en tot roking welriekende specerijen;
\par 9 En sardonixstenen, en vervullende stenen, tot den efod en tot den borstlap.
\par 10 En allen, die wijs van hart zijn onder ulieden, zullen komen, en maken alles, wat de HEERE geboden heeft:
\par 11 De tabernakel, zijn tent en zijn deksel, zijn haakjes en zijn berderen, zijn richelen, zijn pilaren, en zijn voeten;
\par 12 De ark en haar handbomen, het verzoendeksel en den voorhang des deksels;
\par 13 De tafel en haar handbomen, en al haar gereedschap, en de toonbroden;
\par 14 En den kandelaar tot het licht, en zijn gereedschap, en zijn lampen, en de olie tot het licht;
\par 15 En het reukaltaar, en zijn handbomen, en de zalfolie, en het reukwerk van welriekende specerijen; en het deksel der deur aan de deur des tabernakels;
\par 16 Het altaar des brandoffers, en den koperen rooster, dien het hebben zal, zijn handbomen, en al zijn gereedschappen; het wasvat en zijn voet.
\par 17 De behangselen des voorhofs, zijn pilaren en zijn voeten; en het deksel van de poort des voorhofs;
\par 18 De nagelen des tabernakels, en de pennen des voorhofs, met derzelver zelen;
\par 19 De ambtsklederen om in het heilige te dienen, de heilige klederen van den priester Aaron, en de klederen zijner zonen, om het priesterambt te bedienen.
\par 20 Toen ging de ganse vergadering der kinderen Israels uit van voor het aangezicht van Mozes.
\par 21 En zij kwamen, alle man, wiens hart hem bewoog, en een ieder, wiens geest hem vrijwillig maakte, die brachten des HEEREN hefoffer tot het werk van de tent der samenkomst, en tot al haar dienst, en tot de heilige klederen.
\par 22 Zo kwamen dan de mannen met de vrouwen, alle vrijwilligen van hart; zij brachten haken, en oorsierselen, en ringen, en spanselen, alle gouden vaten; en alle man, die een gouden beweegoffer den HEERE offerde,
\par 23 En alle man, bij wien gevonden werd hemelsblauw, en purper, en scharlaken, en fijn linnen, en geiten haar, en roodgeverfde ramsvellen, en dassenvellen, die brachten ze.
\par 24 Allen, die een hefoffer van zilver of koper offerden, die brachten het ten hefoffer des HEEREN; en allen, bij welke sittimhout gevonden werd, brachten het tot alle werk van den dienst.
\par 25 En alle vrouwen, die wijs van hart waren, sponnen met haar handen, en zij brachten het gesponnene, de hemelsblauwe zijde, en het purper, het scharlaken, en het fijn linnen.
\par 26 En alle vrouwen, welker hart haar bewoog in wijsheid, die sponnen het geiten haar.
\par 27 De oversten nu brachten sardonixstenen en vulstenen, tot den efod en tot den borstlap;
\par 28 En specerijen en olie, tot den luchter en tot de zalfolie, en tot roking welriekende specerijen.
\par 29 Alle man en vrouw, welker hart hen vrijwillig bewoog te brengen tot al het werk, hetwelk de HEERE geboden had te maken door de hand van Mozes; dat brachten de kinderen Israels tot een vrijwillig offer den HEERE.
\par 30 Daarna zeide Mozes tot de kinderen Israels: Ziet, de HEERE heeft met name geroepen Bezaleel, den zoon van Uri, den zoon van Hur, van den stam van Juda.
\par 31 En de Geest Gods heeft hem vervuld met wijsheid, met verstand, en met wetenschap, namelijk in alle handwerk;
\par 32 En om te bedenken vernuftigen arbeid, te werken in goud, en in zilver, en in koper,
\par 33 En in kunstige steensnijding, om in te zetten, en in kunstige houtsnijding; om te werken in alle vernuftige handwerk.
\par 34 Hij heeft hem ook in zijn hart gegeven anderen te onderwijzen, hem en Aholiab, den zoon van Ahisamach, van den stam van Dan.
\par 35 Hij heeft hen vervuld met wijsheid des harten, om te maken alle werk eens werkmeesters, en des allervernuftigsten handwerkers, en des borduurders en hemelsblauw, en in purper, in scharlaken, en in fijn linnen, en des wevers; makende alle werk, en bedenkende vernuftigen arbeid.

\chapter{36}

\par 1 Toen wrocht Bezaleel en Aholiab, en alle man, die wijs van hart was, in denwelken de HEERE wijsheid en verstand gegeven had, om te weten, hoe zij maken zouden alle werk ten dienste des heiligdoms naar alles, dat de HEERE geboden had.
\par 2 Want Mozes had geroepen Bezaleel en Aholiab, en alle man, die wijs van hart was, in wiens hart God wijsheid gegeven had, al wiens hart hem bewogen had, dat hij toetrad tot het werk, om dat te maken.
\par 3 Zij dan namen van voor het aangezicht van Mozes het ganse hefoffer, hetwelk de kinderen Israels gebracht hadden, tot het werk van den dienst des heiligdoms, om dat te maken; doch zij brachten tot hem nog allen morgen vrijwillig offer.
\par 4 Derhalve kwamen alle wijzen, die al het werk des heiligdoms maakten, ieder man van zijn werk, hetwelk zij maakten;
\par 5 En zij spraken tot Mozes, zeggende: Het volk brengt te veel, meer dan genoeg is ten dienste des werks, hetwelk de HEERE te maken geboden heeft.
\par 6 Toen gebood Mozes, dat men een stem zoude laten gaan door het leger, zeggende: Man noch vrouw make geen werk meer ten hefoffer des heiligdoms! Alzo werd het volk teruggehouden van meer te brengen.
\par 7 Want der stoffe was denzelven genoeg tot het gehele werk, dat te maken was; ja, er was over.
\par 8 Alzo maakte een ieder wijze van hart, onder degenen, die het werk maakten, den tabernakel van tien gordijnen, van getweernd fijn linnen, en hemelsblauw, en purper, en scharlaken met cherubim; van het allerkunstelijkste werk maakte hij ze.
\par 9 De lengte ener gordijn was van acht en twintig ellen, en de breedte ener gordijn van vier ellen; al deze gordijnen hadden een maat.
\par 10 En hij voegde vijf gordijnen, de ene aan de andere; en hij voegde andere vijf gordijnen, de ene aan de andere.
\par 11 Daarna maakte hij striklisjes van hemelsblauw aan den kant ener gordijn, aan het uiterste in de samenvoeging; hij deed het ook aan den uitersten kant der tweede samenvoegende gordijn.
\par 12 Vijftig striklisjes maakte hij aan de ene gordijn, en vijftig striklisjes maakte hij aan het uiterste der gordijn; dat aan de tweede samenvoegende was; deze striklisjes vatten de ene aan de andere.
\par 13 Hij maakte ook vijftig gouden haakjes, en voegde de gordijnen samen, de ene aan de andere, met deze haakjes, dat het een tabernakel werd.
\par 14 Verder maakte hij gordijnen van geiten haar, tot een tent over den tabernakel; van elf gordijnen maakte hij ze.
\par 15 De lengte ener gordijn was dertig ellen, en vier ellen de breedte ener gordijn; deze elf gordijnen hadden een maat.
\par 16 En hij voegde vijf gordijnen samen bijzonder; wederom zes dezer gordijnen bijzonder.
\par 17 En hij maakte vijftig striklisjes aan den kant van de gordijn, de uiterste in de samenvoeging; hij maakte ook vijftig striklisjes aan den kant van de gordijn der andere samenvoeging.
\par 18 Hij maakte ook vijftig koperen haakjes, om de tent samen te voegen, dat zij een ware.
\par 19 Ook maakte hij voor de tent een deksel van roodgeverfde ramsvellen, en daarover een deksel van dassenvellen.
\par 20 Hij maakte ook aan den tabernakel berderen van staand sittimhout.
\par 21 De lengte van een berd was tien ellen, en ene el en ene halve el was de breedte van elk berd.
\par 22 Twee houvasten had een berd, als sporten in een ladder gezet, het ene nevens het andere; alzo maakte hij het met al de berderen des tabernakels.
\par 23 Hij maakte ook de berderen tot den tabernakel; twintig berderen naar de zuidzijde zuidwaarts.
\par 24 En hij maakte veertig zilveren voeten onder de twintig berderen; twee voeten onder een berd, aan zijn twee houvasten, en twee voeten onder een ander berd, aan zijn twee houvasten.
\par 25 Hij maakte ook twintig berderen aan de andere zijde des tabernakels, aan den noorderhoek.
\par 26 Met hun veertig zilveren voeten; twee voeten onder een berd, en twee voeten onder een ander berd.
\par 27 Doch aan de zijde des tabernakels tegen het westen, maakte hij zes berderen.
\par 28 Ook maakte hij twee berderen tot hoekberderen des tabernakels, aan de beide zijden.
\par 29 En zij waren van beneden als tweelingen samengevoegd, zij waren ook als tweelingen aan deszelfs oppereinde samengevoegd met een ring; alzo deed hij met die beide, aan de twee hoeken.
\par 30 Alzo waren er acht berderen met hun zilveren voeten, zijnde zestien voeten: twee voeten onder elk berd.
\par 31 Hij maakte ook richelen van sittimhout; vijf aan de berderen der ene zijde des tabernakels;
\par 32 En vijf richelen aan de berderen van de andere zijde des tabernakels; alsook vijf richelen aan de berderen des tabernakels, aan de beide zijden westwaarts.
\par 33 En hij maakte de middelste richel doorschietende in het midden der berderen, van het ene einde tot het andere einde.
\par 34 En hij overtrok de berderen met goud, en hun ringen (de plaatsen voor de richelen) maakte hij van goud; de richelen overtrok hij ook met goud.
\par 35 Daarna maakte hij een voorhang van hemelsblauw, en purper, en scharlaken, en fijn getweernd linnen; van het allerkunstelijkste werk maakte hij denzelven, met cherubim.
\par 36 En hij maakte daartoe vier pilaren van sittim hout, die hij overtrok met goud; hun haken waren van goud, en hij goot hun vier zilveren voeten.
\par 37 Hij maakte ook aan de deur der tent een deksel van hemelsblauw, en purper, en scharlaken, en fijn getweernd linnen, geborduurd werk;
\par 38 En de vijf pilaren daarvan, en hun haken; en hij overtrok hun hoofden en derzelver banden met goud; en hun vijf voeten waren van koper.

\chapter{37}

\par 1 Alzo maakte Bezaleel de ark van sittimhout; twee ellen en een halve was haar lengte, en anderhalve el haar breedte, en anderhalve el haar hoogte.
\par 2 En hij overtrok ze met louter goud, van binnen en van buiten; en hij maakte ze een gouden krans rondom.
\par 3 En hij goot voor dezelve vier gouden ringen, aan haar vier hoeken, alzo dat twee ringen op derzelver ene zijde waren, en twee ringen op haar andere zijde.
\par 4 En hij maakte handbomen van sittimhout, en hij overtrok ze met goud.
\par 5 En hij stak de handbomen in de ringen, aan de zijden der ark, om de ark te dragen.
\par 6 Hij maakte ook een verzoendeksel van louter goud; twee ellen en een halve was deszelfs lengte, en anderhalve el deszelfs breedte.
\par 7 Ook maakte hij twee cherubim van goud; van dicht werk maakte hij ze, uit de beide einden des verzoendeksels.
\par 8 Een cherub uit het ene einde aan deze zijde, en den anderen cherub uit het andere einde aan gene zijde; uit het verzoendeksel maakte hij de cherubim, uit deszelfs beide einden.
\par 9 En de cherubim waren de beide vleugelen omhoog uitbreidende, bedekkende met hun vleugelen het verzoendeksel; en hun aangezichten waren tegenover elkander; de aangezichten der cherubim waren naar het verzoendeksel.
\par 10 Hij maakte ook een tafel van sittimhout; twee ellen was haar lengte, en een el haar breedte; en een el en een halve haar hoogte.
\par 11 En hij overtrok ze met louter goud; en hij maakte een gouden krans daaraan, rondom.
\par 12 Hij maakte daaraan ook een lijst rondom, een hand breed; en hij maakte een gouden krans rondom derzelver lijst.
\par 13 Hij goot ook vier gouden ringen daaraan; en hij zette de ringen aan de vier hoeken, die aan derzelver vier voeten waren.
\par 14 Tegenover de lijst waren de ringen tot plaatsen voor de handbomen, om de tafel te dragen.
\par 15 Hij maakte ook de handbomen van sittimhout; en hij overtrok ze met goud, om de tafel te dragen.
\par 16 En hij maakte het gereedschap, dat op de tafel zijn zoude, haar schotelen, en haar reukschalen, en haar kroezen, en haar platelen (met welke zij bedekt zoude worden), van louter goud.
\par 17 Hij maakte ook een kandelaar van louter goud. Van dicht werk maakte hij dezen kandelaar, zijn schacht, en zijn rieten; zijn schaaltjes, zijn knopen, en zijn bloemen waren uit hem.
\par 18 Zes rieten nu gingen uit zijn zijden; drie rieten des kandelaars uit zijn ene zijde, en drie rieten des kandelaars uit zijn andere zijde.
\par 19 In het ene riet waren drie schaaltjes, gelijk amandelnoten, een knoop en een bloem; en drie schaaltjes, gelijk amandelnoten in een ander riet, een knoop en een bloem; alzo waren die zes rieten, die uit den kandelaar gingen.
\par 20 Maar aan den kandelaar zelven waren vier schaaltjes, gelijk amandelnoten, met zijn knopen, en met zijn bloemen.
\par 21 En daar was een knoop onder twee rieten, uit denzelven uitgaande; ook een knoop onder twee rieten, uit denzelven uitgaande; nog een knoop onder twee rieten, uit denzelven uitgaande; alzo was het met de zes rieten, die uit denzelven uitgingen.
\par 22 Hun knopen en rieten waren uit hem; het was altemaal een enig dicht werk van louter goud.
\par 23 En hij maakte hem zeven lampen; zijn snuiters en zijn blusvaten waren van louter goud.
\par 24 Hij maakte denzelven uit een talent louter goud, met al zijn vaten.
\par 25 En hij maakte het reukaltaar van sittimhout; een el was zijn lengte en een el zijn breedte, vierkant, maar twee ellen zijn hoogte; uit hetzelve waren zijn hoornen.
\par 26 En hij overtrok het met louter goud, zijn dak, en zijn wanden rondom, alsook zijn hoornen; en hij maakte het een gouden krans rondom.
\par 27 Hij maakte ook twee gouden ringen daaraan, onder zijn krans, aan zijn twee hoeken, aan zijn beide zijden, tot plaatsen voor de handbomen, dat men het daarmede droeg.
\par 28 En hij maakte de handbomen van sittimhout, en hij overtrok ze met goud.
\par 29 Hij maakte ook de heilige zalfolie, en het reukwerk der zuiverste welriekende specerijen, naar apothekerswerk.

\chapter{38}

\par 1 Hij maakte ook het brandofferaltaar van sittimhout; vijf ellen was deszelfs lengte, en vijf ellen zijn breedte, vierkant, en drie ellen zijn hoogte.
\par 2 En hij maakte deszelfs hoornen op zijn vier hoeken; uit hetzelve waren zijn hoornen; en hij overtrok het met koper.
\par 3 Hij maakte ook al het gereedschap des altaars, de potten, en de schoffelen, en de besprengbekkens, en de krauwelen, en de koolpannen; al zijn vaten maakte hij van koper.
\par 4 Ook maakte hij aan het altaar een rooster van koperen netwerk, onder zijn omloop, van beneden tot zijn midden toe.
\par 5 En hij goot vier ringen aan de vier einden des koperen roosters, tot plaatsen voor de handbomen.
\par 6 En hij maakte de handbomen van sittimhout, en hij overtrok ze met koper.
\par 7 En hij deed de handbomen in de ringen, aan de zijden des altaars, dat men het met dezelve droeg; hij maakte hetzelve hol van planken.
\par 8 Hij maakte ook het koperen wasvat, met zijn koperen voet, van de spiegels der te hoop komende vrouwen, die te hoop kwamen voor de deur van de tent der samenkomst.
\par 9 Hij maakte ook den voorhof, aan den zuidhoek zuidwaarts; de behangselen tot den voorhof waren van fijn getweernd linnen, van honderd ellen.
\par 10 Hun twintig pilaren en derzelver twintig voeten, waren van koper; de haken dezer pilaren en hun banden waren van zilver.
\par 11 En aan den noorderhoek honderd ellen, hun twintig pilaren en derzelver twintig voeten waren van koper; de haken der pilaren en derzelver banden waren van zilver.
\par 12 En aan den westerhoek waren behangselen van vijftig ellen, hun pilaren tien en derzelver voeten tien; de haken der pilaren en hun banden waren van zilver.
\par 13 En aan den oosterhoek tegen den opgang waren vijftig ellen.
\par 14 De behangselen aan deze zijde waren vijftien ellen, derzelver pilaren drie en hun voeten drie.
\par 15 En aan de andere zijde van de deur des voorhofs, van hier en van daar, waren behangselen van vijftien ellen; hun pilaren drie en derzelver voeten drie.
\par 16 Al de behangselen des voorhofs waren rondom van fijn getweernd linnen.
\par 17 De voeten nu der pilaren waren van koper, de haken der pilaren, en hun banden waren van zilver, en het overdeksel hunner hoofden was van zilver, en al de pilaren des voorhofs waren met zilver omtogen.
\par 18 En het deksel van de poort des voorhofs was van geborduurd werk, van hemelsblauw, en purper, en scharlaken, en fijn getweernd linnen; en twintig ellen was de lengte, en de hoogte in de breedte was vijf ellen, tegenover de behangselen des voorhofs.
\par 19 En hun vier pilaren en derzelver vier voeten waren van koper, hun haken waren van zilver; ook was het overdeksel hunner hoofden en hun banden van zilver.
\par 20 En al de pennen des tabernakels en des voorhofs rondom waren van koper.
\par 21 Dit zijn de getelde dingen van den tabernakel, van den tabernakel der getuigenis, die geteld zijn naar den mond van Mozes, ten dienste der Levieten, door de hand van Ithamar, den zoon van den priester Aaron.
\par 22 Bezaleel nu, de zoon van Uri, den zoon van Hur, van den stam van Juda, maakte al, dat de HEERE aan Mozes geboden had.
\par 23 En met hem Aholiab, de zoon van Ahisamach, van den stam van Dan, een werkmeester en vernuftig kunstenaar, en een borduurder in hemelsblauw, en in purper, en in scharlaken, en in fijn linnen.
\par 24 Al het goud, dat tot het werk verarbeid is, in het ganse werk des heiligdoms, te weten, het goud des beweegoffers, was negen en twintig talenten, en zevenhonderd en dertig sikkelen, naar den sikkel des heiligdoms.
\par 25 Het zilver nu van de getelden der vergadering was honderd talenten, en duizend zevenhonderd vijf en zeventig sikkelen, naar den sikkel des heiligdoms.
\par 26 Een beka voor elk hoofd, dat is een halve sikkel, naar den sikkel des heiligdoms, van een ieder, die overging tot de getelden, van twintig jaren oud en daarboven, namelijk zeshonderd drie duizend, vijfhonderd en vijftig.
\par 27 En er waren honderd talenten zilver, om te gieten de voeten des heiligdoms, en de voeten des voorhangs; tot honderd voeten waren honderd talenten, een talent tot een voet.
\par 28 Maar uit de duizend zevenhonderd vijf en zeventig sikkelen maakte hij de haken aan de pilaren, en hij overtrok hun hoofden, en omtoog ze met banden.
\par 29 Het koper nu des beweegoffers was zeventig talenten, en twee duizend vierhonderd sikkelen.
\par 30 En hij maakte daarvan de voeten der deur van de tent der samenkomst, en het koperen altaar, en den koperen rooster, dien het had, en al het gereedschap des altaars.
\par 31 En de voeten des voorhofs rondom, en de voeten van de poort des voorhofs, ook al de pennen des tabernakels, en al de pennen des voorhofs rondom.

\chapter{39}

\par 1 Zij maakten ook ambtsklederen, om in het heilige te dienen, van hemelsblauw, en purper, en scharlaken; ook maakten zij de heilige klederen, die voor Aaron waren, gelijk de HEERE aan Mozes geboden had.
\par 2 Aldus maakte hij den efod, van goud, hemelsblauw en purper, en scharlaken, en fijn getweernd linnen.
\par 3 En zij rekten uit de dunne platen van goud, en sneden het tot draden, om te doen in het midden van het hemelsblauw, en in het midden van het purper, en in het midden van het scharlaken, en in het midden van het fijn linnen, van het allerkunstelijkste werk.
\par 4 Zij maakten samenvoegende schouderbanden daaraan; aan deszelfs beide einden werd hij samengevoegd.
\par 5 En de kunstelijke riem zijns efods, die daarop was, was gelijk zijn werk, van hetzelfde, van goud, van hemelsblauw, en purper, en scharlaken, en fijn getweernd linnen, gelijk als de HEERE aan Mozes bevolen had.
\par 6 Zij bereidden ook de sardonixstenen, omvat in gouden kastjes, als zegelgravering gegraveerd, met de namen der zonen van Israel.
\par 7 En hij zette ze op de schouderbanden des efods, tot stenen der gedachtenis voor de kinderen Israels, gelijk de HEERE aan Mozes geboden had.
\par 8 Hij maakte ook den borstlap van het allerkunstelijkste werk, gelijk het werk des efods, van goud, hemelsblauw, en purper, en scharlaken, en fijn getweernd linnen.
\par 9 Hij was vierkant; zij maakten den borstlap dubbel; een span was zijn lengte, en een span was zijn breedte, dubbel zijnde.
\par 10 En zij vulden daarin vier rijen stenen: een rij van een Sardis, een Topaas en een Karbonkel; dit is de eerste rij.
\par 11 En de tweede rij van een Smaragd, een Saffier en een Diamant.
\par 12 En de derde rij van een Hyacint, Agaat, en Amethist.
\par 13 En de vierde rij van een Turkoois, en een Sardonix, en een Jaspis; omvat in gouden kastjes in hun vullingen.
\par 14 Deze stenen nu, met de namen der zonen van Israel, waren twaalf, met hun namen, met zegelgravering; ieder met zijn naam, naar de twaalf stammen.
\par 15 Zij maakten ook aan den borstlap gelijk-eindigende ketentjes, van gedraaid werk, uit louter goud.
\par 16 En zij maakten twee gouden kastjes, en twee gouden ringen; en zij zetten die twee ringen aan de beide einden des borstlaps.
\par 17 En zij zetten de twee gedraaide gouden ketentjes aan de twee ringen, aan de einden van den borstlap.
\par 18 Doch de twee andere einden der twee gedraaide ketenen zetten zij aan de twee kastjes, en zij zetten ze aan de schouderbanden des efods, recht op de voorste zijde van dien.
\par 19 Zij maakten ook twee gouden ringen, die zij aan de twee andere einden des borstlaps zetten, inwendig aan zijn boord, die aan de zijde des efods is.
\par 20 Nog maakten zij twee gouden ringen, die zij zetten aan de twee schouderbanden van den efod, beneden, aan deszelfs voorste zijde, tegenover zijn andere voege, boven den kunstelijken riem des efods.
\par 21 En zij bonden den borstlap met zijn ringen aan de ringen van den efod, met een hemelsblauw snoer, dat hij op den kunstelijken riem van den efod was; opdat de borstlap van den efod niet afgescheiden wierd, gelijk als de HEERE aan Mozes geboden had.
\par 22 En hij maakte den mantel des efods van geweven werk, geheel van hemelsblauw.
\par 23 En het gat des mantels was in deszelfs midden, als het gat eens pantsiers; dit gat had een boord rondom, dat het niet gescheurd wierd.
\par 24 En aan de zomen des mantels maakten zij granaatappelen van hemelsblauw, en purper, en scharlaken, getweernd.
\par 25 Zij maakten ook schelletjes van louter goud, en zij stelden de schelletjes tussen de granaatappelen, aan de zomen des mantels rondom, tussen de granaatappelen;
\par 26 Dat er een schelletje, daarna een granaatappel was; wederom een schelletje, en een granaatappel; aan de zomen des mantels rondom; om te dienen, gelijk als de HEERE aan Mozes geboden had.
\par 27 Zij maakten ook de rokken van fijn linnen, van geweven werk, voor Aaron en voor zijn zonen;
\par 28 En den hoed van fijn linnen, en de sierlijke mutsen van fijn linnen, en de linnen onderbroeken van fijn getweernd linnen;
\par 29 En den gordel van fijn getweernd linnen, en van hemelsblauw, en purper, en scharlaken, van geborduurd werk, gelijk als de HEERE aan Mozes geboden had.
\par 30 Zij maakten ook de plaat van de kroon der heiligheid van louter goud, en zij schreven daarop een schrift, met zegelgravering: De HEILIGHEID DES HEEREN.
\par 31 En zij hechtten een snoer van hemelsblauw daaraan, om aan den hoed van boven te hechten, gelijk als de HEERE aan Mozes geboden had.
\par 32 Aldus werd al het werk des tabernakels, van de tent der samenkomst voleind; en de kinderen Israels hadden het gemaakt naar alles, wat de HEERE aan Mozes geboden had; alzo hadden zij het gemaakt.
\par 33 Daarna brachten zij den tabernakel tot Mozes, de tent, en al haar gereedschap, haar haakjes, haar berderen, haar richelen, en haar pilaren, en haar voeten;
\par 34 En het deksel van roodgeverfde ramsvellen, en het deksel van dassenvellen, en den voorhang van het deksel;
\par 35 De ark der getuigenis, en haar handbomen, en het verzoendeksel;
\par 36 De tafel, met al haar gereedschap, en de toonbroden;
\par 37 Den louteren kandelaar met zijn lampen, de lampen, die men toerichten moest, en al deszelfs gereedschap, en de olie tot het licht;
\par 38 Verder het gouden altaar, en de zalfolie, en het reukwerk van welriekende specerijen, en het deksel van de deur der tent.
\par 39 Het koperen altaar, en den koperen rooster, dien het heeft, deszelfs handbomen, en al zijn gereedschap; het wasvat en zijn voet;
\par 40 De behangselen des voorhofs, zijn pilaren en zijn voeten, en het deksel van de poort des voorhofs, zijn zelen, en zijn pennen, en al het gereedschap van den dienst des tabernakels, tot de tent der samenkomst;
\par 41 De ambtsklederen, om in het heiligdom te dienen, de heilige klederen van den priester Aaron, en de klederen van zijn zonen, om het priesterambt te bedienen.
\par 42 Naar alles, wat de HEERE aan Mozes geboden had, alzo hadden de kinderen Israels het ganse werk gemaakt.
\par 43 Mozes nu bezag het ganse werk, en ziet, zij hadden het gemaakt, gelijk als de HEERE geboden had; alzo hadden zij het gemaakt. Toen zegende Mozes hen.

\chapter{40}

\par 1 Verder sprak de HEERE tot Mozes, zeggende:
\par 2 Op den dag der eerste maand, te weten op den eersten der maand, zult gij den tabernakel, de tent der samenkomst, oprichten.
\par 3 En gij zult aldaar zetten de ark der getuigenis; en gij zult de ark met den voorhang bedekken.
\par 4 Daarna zult gij de tafel daarin brengen, en gij zult schikken wat daarop te schikken is; gij zult ook den kandelaar daarin brengen, en zijn lampen aansteken.
\par 5 En gij zult het gouden altaar ten reukwerk voor de ark der getuigenis zetten; dan zult gij het deksel van de deur des tabernakels ophangen.
\par 6 Gij zult ook het altaar des brandoffers zetten voor de deur van den tabernakel, van de tent der samenkomst.
\par 7 En gij zult het wasvat zetten tussen de tent der samenkomst, en tussen het altaar; en gij zult water daar in doen.
\par 8 Daarna zult gij den voorhof rondom zetten, en gij zult het deksel ophangen aan de poort des voorhofs.
\par 9 Dan zult gij de zalfolie nemen en zalven den tabernakel, en al wat daarin is; en gij zult dezelven heiligen, met al zijn gereedschap, en het zal een heiligheid zijn.
\par 10 Gij zult ook het altaar des brandoffers zalven, en al zijn gereedschap; en gij zult het altaar heiligen, en het altaar zal heiligheid der heiligheden zijn.
\par 11 Dan zult gij het wasvat zalven, en deszelfs voet; en gij zult het heiligen.
\par 12 Gij zult ook Aaron en zijn zonen doen naderen, tot de deur van de tent der samenkomst; en gij zult hen met water wassen.
\par 13 En gij zult Aaron de heilige klederen aantrekken; en gij zult hem zalven, en hem heiligen, dat hij Mij het priesterambt bediene.
\par 14 Gij zult ook zijn zonen doen naderen, en zult hun de rokken aantrekken.
\par 15 En gij zult hen zalven, gelijk als gij hun vader zult gezalfd hebben, dat zij Mij het priesterambt bedienen. En het zal geschieden, dat hun hun zalving zal zijn tot een eeuwig priesterdom bij hun geslachten.
\par 16 Mozes nu deed het naar alles, wat hem de HEERE geboden had; alzo deed hij.
\par 17 En het geschiedde in de eerste maand, in het tweede jaar, op den eersten der maand, dat de tabernakel opgericht werd.
\par 18 Want Mozes richtte den tabernakel op, en zette zijn voeten, en stelde zijn berderen, en zette zijn richelen daaraan, en hij richtte deszelfs pilaren op.
\par 19 En hij spreidde de tent uit over den tabernakel, en hij zette het deksel der tent daar bovenop, gelijk als de HEERE aan Mozes geboden had.
\par 20 Voorts nam hij, en leide de getuigenis in de ark, en deed de handbomen aan de ark, en hij zette het verzoendeksel boven op de ark.
\par 21 En hij bracht de ark in den tabernakel, en hij hing den voorhang van het deksel op, en bedekte de ark der getuigenis, gelijk als de HEERE aan Mozes geboden had.
\par 22 Hij zette ook de tafel in de tent der samenkomst, aan de zijde des tabernakels tegen het noorden, buiten den voorhang.
\par 23 En hij schikte daarop het brood in orde, voor het aangezicht des HEEREN, gelijk als de HEERE aan Mozes geboden had.
\par 24 Hij zette ook den kandelaar in de tent der samenkomst, recht over de tafel, aan de zijde des tabernakels, zuidwaarts.
\par 25 En hij stak de lampen aan voor het aangezicht des HEEREN, gelijk als de HEERE aan Mozes geboden had.
\par 26 En hij zette het gouden altaar in de tent der samenkomst, voor den voorhang.
\par 27 En hij stak daarop aan reukwerk van welriekende specerijen, gelijk als de HEERE aan Mozes geboden had.
\par 28 Hij hing ook het deksel van de deur des tabernakels.
\par 29 En hij zette het altaar des brandoffers aan de deur des tabernakels, van de tent der samenkomst; en hij offerde daarop brandoffer, en spijsoffer, gelijk de HEERE aan Mozes geboden had.
\par 30 Hij zette ook het wasvat tussen de tent der samenkomst, en tussen het altaar; en hij deed water daarin om te wassen.
\par 31 En Mozes en Aaron, en zijn zonen wiesen daaruit hun handen en hun voeten.
\par 32 Als zij ingingen tot de tent der samenkomst, en als zij tot het altaar naderden, zo wiesen zij zich, gelijk als de HEERE aan Mozes geboden had.
\par 33 Hij richtte ook den voorhof op, rondom den tabernakel en het altaar, en hij hing het deksel van de poort des voorhofs op. Alzo voleindigde Mozes het werk.
\par 34 Toen bedekte de wolk de tent der samenkomst; en de heerlijkheid des HEEREN vervulde den tabernakel.
\par 35 Zodat Mozes niet kon ingaan in de tent der samenkomst, dewijl de wolk daarop bleef, en de heerlijkheid des HEEREN den tabernakel vervulde.
\par 36 Als nu de wolk opgeheven werd van boven den tabernakel, zo reisden de kinderen Israels voort in al hun reizen.
\par 37 Maar als de wolk niet opgeheven werd, zo reisden zij niet tot op den dag, dat zij opgeheven werd.
\par 38 Want de wolk des HEEREN was op den tabernakel bij dag, en het vuur was er bij nacht op, voor de ogen van het ganse huis Israels in al hun reizen.



\end{document}