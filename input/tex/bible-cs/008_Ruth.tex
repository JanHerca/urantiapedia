\begin{document}

\title{Rút}

\chapter{1}

\par 1 Stalo se pak za casu soudcu, že byl hlad v zemi. I odšel clovek z Betléma Judova, a bydlil pohostinu v krajine Moábské s manželkou svou a se dvema syny svými.
\par 2 Jméno pak muže toho Elimelech, a jméno ženy jeho Noémi, jméno také dvou synu jeho Mahalon a Chelion, Efratejští z Betléma Judova; a prišedše do krajiny Moábské, bydlili tam.
\par 3 Umrel pak Elimelech, muž Noémi; i pozustala ona s obema syny svými.
\par 4 Kteríž pojali sobe ženy Moábské; jméno jedné bylo Orfa a jméno druhé Rut. I bydlili tam témer deset let.
\par 5 Umreli také oni oba dva, Mahalon i Chelion, a zustala žena ta po dvou synech svých a manželu svém.
\par 6 Tedy vstavši ona s nevestami svými, navracela se z krajiny Moábské; nebo slyšela v krajine Moábské, že Hospodin navštívil lid svuj, dav jemu chléb.
\par 7 Vyšedši tedy z místa, na nemž bydlila, a obe nevesty její s ní, daly se na cestu, aby se navrátily do zeme Judské.
\par 8 Rekla pak Noémi obema nevestám svým: Jdete, navratte se jedna každá do domu matky své. Uciniž Hospodin s vámi milosrdenství, jakož jste i vy cinili s mrtvými syny mými i se mnou.
\par 9 Dejž vám Hospodin, abyste nalezly odpocinutí, jedna každá v dome muže svého. I políbila jich, ony pak pozdvihše hlasu svého, plakaly.
\par 10 A rekly jí: Obrátíme se radeji s tebou k lidu tvému.
\par 11 I rekla Noémi: Navratte se, dcerky mé. Proc chcete jíti se mnou? Zdaliž ješte budu míti syny, aby byli vaši muži?
\par 12 Navratte se, dcerky mé, a odejdete, nebot jsem již stará k vdání; nýbrž bych i rekla, že mám nadeji, bych i noci této se vdala a syny zrodila,
\par 13 Zdaž byste na ne cekaly, až by dorostli? Zdaliž tou prícinou meškati se budete, abyste se nemely vdáti? Ne tak, mé dcerky, nebo mé trápení vetší jest nežli vaše, proto že ruka Hospodinova jest proti mne.
\par 14 Ony pak pozdvihše hlasu, opet plakaly. A Orfa políbivši svegruši svou, odešla, Rut pak prídržela se jí.
\par 15 Kteréžto rekla Noémi: Aj, prítelkyne tvá navrátila se k lidu svému a k bohum svým, navratiž se také za ní.
\par 16 Ale Rut rekla: Nenut mne, abych te opustiti a od tebe odjíti mela. Nebo kamž se koli obrátíš, pujdu, a kdekoli bydliti budeš, i já bydliti budu; lid tvuj lid muj, a Buh tvuj Buh muj.
\par 17 Kdekoli umreš, umru, a tu pochována budu. Toto mi ucin Hospodin, a toto pridej, že toliko smrt rozdelí mne s tebou.
\par 18 Tedy viduci, že se na tom ustavila, aby šla s ní, prestala k ní mluviti.
\par 19 I šly obe dve spolu, až prišly k Betlému. I stalo se, že když prišly do Betléma, roznesla se povest o nich po všem meste, a pravili: To-li jest ta Noémi?
\par 20 Jimž ona rekla: Nenazývejte mne Noémi, ríkejte mi Mara; nebo horkostí velikou naplnil mne Všemohoucí.
\par 21 Vyšla jsem plná, ted pak prázdnou mne zase Hospodin privedl. Procež tedy nazýváte mne Noémi, ponevadž mne Hospodin ssoužil, a Všemohoucí mne znuzil?
\par 22 A navrátila se Noémi s Rut Moábskou, nevestou svou; navrátila se pak z krajiny Moábské. I prišly do Betléma, když pocínali žíti jecmene.

\chapter{2}

\par 1 Mela pak Noémi prítele po manželu svém, muže mocného z celedi Elimelechovy, jménem Bóza.
\par 2 I rekla Rut Moábská Noémi: Necht medle jdu na pole sbírati klasu za tím, kdož by mi toho prál. Jížto ona rekla: Jdi, dcero má.
\par 3 Šla tedy, a prišedši, sbírala klasy na poli za ženci. Prihodilo se pak, že prišla na díl pole toho, kteréž prináleželo Bózovi, jenž byl z celedi Elimelechovy.
\par 4 A v tom prišel Bóz z Betléma, a rekl žencum: Hospodin s vámi. Kteríž odpovedeli jemu: Požehnejž tobe Hospodin.
\par 5 I rekl Bóz služebníku svému, kterýž postaven byl nad ženci: Cí jest tato mladice?
\par 6 Odpovedel služebník ten, kterýž postaven byl nad ženci, a rekl: Jest mladice Moábská, kteráž prišla s Noémi z zeme Moábské.
\par 7 A rekla mi: Prosím, necht sbírám a shromažduji klasy mezi snopy za ženci. A prišedši, trvá od jitra až dosavad, krome že na chvilku doma pobyla.
\par 8 Tedy rekl Bóz k Rut: Slyš, dcero má, nechod sbírati na jiné pole, aniž odcházej odsud, ale prídrž se ted devek mých.
\par 9 Zustávej na tom poli, na nemž budou žíti, a chod za nimi, nebo jsem prikázal služebníkum svým, aby se tebe žádný nedotýkal. Bude-lit se chtíti píti, jdi k nádobám, a napí se té vody, kteréž by navážili služebníci moji.
\par 10 Tedy ona padla na tvár svou, a schýlivši se k zemi, rekla jemu: Odkudž mi to, že jsem nalezla milost pred tebou, abys se známil ke mne, kteráž jsem cizozemka.
\par 11 Odpovedel Bóz a rekl: O všemt mi oznámeno jest, co jsi koli cinila svegruši své po smrti muže svého, a že opustivši otce svého a matku svou, i zemi, v kteréžs se narodila, šla jsi mezi lid, kteréhožs prvé neznala.
\par 12 Odplatiž tobe Hospodin za skutek tvuj, a budiž mzda tvá dokonalá od Hospodina Boha Izraelského, ponevadž jsi prišla, abys pod krídly jeho doufala.
\par 13 Kterážto rekla: Nalezla jsem milost pred tebou, pane muj, ponevadž jsi mne potešil, a mluvils k srdci devky své, ješto nejsem podobná jedné z devek tvých.
\par 14 Tedy rekl jí Bóz: Když bude cas jísti, pristup sem, a pojez chleba, a omoc skyvu svou v octe. I posadila se pri žencích, a podal jí pražmy; ona pak jedla až do sytosti, a ješte jí zbylo.
\par 15 I vstala, aby sbírala. Prikázal pak Bóz služebníkum svým, rka: Byt pak i mezi snopy sbírala, nezbranujte jí.
\par 16 Nýbrž naschvál jí upouštejte z snopu a nechávejte, at sbírá, a nedomlouvejte jí.
\par 17 Sbírala tedy na poli tom až do vecera, a což sebrala, to vymlátila; i byla témer míra efi jecmene.
\par 18 Kterýž vzavši, prišla do mesta, a videla svegruše její to, což nasbírala. Vynala také a dala jí to, což pozustalo po nasycení jejím.
\par 19 I rekla jí svegruše její: Kdes sbírala dnes, a kdes pracovala? Budiž požehnaný ten, kterýž te prijal. Tedy oznámila svegruši své, u koho pracovala, rkuci: Jméno muže, u kteréhož jsem pracovala dnes, jest Bóz.
\par 20 I rekla Noémi neveste své: Požehnanýt jest od Hospodina, že neprestal milosrdenství svého nad živými i mrtvými. I to ješte k ní rekla Noémi: Blízký prítel náš a z príbuzných našich jest muž ten.
\par 21 Rekla jí také Rut Moábská: I to mi ješte rekl: Celádky mé prídrž se, dokavadž by všeho, což mého jest, nedožali.
\par 22 Tedy rekla Noémi Rut neveste své: Dobré jest tedy, dcero má, abys vycházela s deveckami jeho, at by na jiném poli neco neprekazili.
\par 23 A tak se prídržela Rut devek Bózových, a sbírala klasy, dokudž nesžali jecmene a pšenice, a bydlela u svegruše své.

\chapter{3}

\par 1 Rekla jí potom Noémi svegruše její: Má dcero, nemám-liž pohledati tobe odpocinutí, aby tobe dobre bylo?
\par 2 Anobrž zdaliž Bóz ten príbuzný náš, s jehož jsi deveckami byla, nebude víti jecmene na humne noci této?
\par 3 Protož umej se a pomaž, roucho své také oblec, a jdi na humno, však tak, aby nebylo známé muži tomu, prvé než by prestal jísti a píti.
\par 4 A když pujde ležeti, znamenej místo, na kterémž lehne, a prijduc, pozdvihneš plášte u noh jeho, a tu se položíš; on pak oznámí tobe, co bys mela ciniti.
\par 5 Jížto Rut rekla: Cokoli mi rozkážeš, uciním.
\par 6 Šla tedy na to humno, a ucinila všecko, což jí rozkázala svegruše její.
\par 7 Když pak pojedl Bóz a napil se, a rozveselilo se srdce jeho, šel spáti vedlé stohu; prišla i ona tiše, a pozdvihši plášte u noh jeho, položila se.
\par 8 A když bylo o pul noci, ulekl se muž ten a zchopil se, a aj, žena leží u noh jeho.
\par 9 I rekl: Kdo jsi ty? A ona odpovedela: Já jsem Rut, devka tvá. Vztáhni krídlo plášte svého na devku svou, nebo príbuzný jsi.
\par 10 A on rekl: Požehnaná jsi ty od Hospodina, dcero má. Vetší jsi nyní pobožnosti dokázala, nežli prvé, že jsi nehledala mládencu bohatých aneb chudých.
\par 11 Protož nyní, dcero má, neboj se; vše, cehož žádáš, uciním tobe, nebo vít všecko mesto lidu mého, že jsi ty žena šlechetná.
\par 12 A také jest to pravé, že jsem príbuzný tvuj, ale jestit príbuzný bližší nežli já.
\par 13 Odpociniž tu pres noc, a když bude ráno, jestližet on bude chtíti práva príbuznosti k tobe užiti, dobre, necht užive. Paklit nebude chtíti práva užiti k tobe, já právem príbuznosti pojmu tebe, živt jest Hospodin. Spiž tu až do jitra.
\par 14 A tak spala u noh jeho až do jitra. Potom vstala prvé, nežli by kdo poznati mohl bližního svého; nebo pecoval Bóz, aby nekdo nezvedel, že prišla žena ta na humno.
\par 15 A rekl: Prines loktušku, kterouž se odíváš, a drž ji. A když ji držela, nameriv jí šest mer jecmene, vložil na ni. I vešla do mesta.
\par 16 A prišla k svegruši své. Kteráž rekla: Kdo jsi ty, dcero má? I vypravovala jí všecko, což jí ucinil muž ten.
\par 17 A rekla: Šest mer techto jecmene dal mi, nebo rekl ke mne: Nenavrátíš se prázdná k svegruši své.
\par 18 I rekla jí Noémi: Pocekej, dcero má, až porozumíš, jak to padne; nebot neobleví muž ten, až tu vec dnes k místu privede.

\chapter{4}

\par 1 Tedy Bóz všed do brány, posadil se tam. A aj, príbuzný ten, o nemž on byl mluvil, šel tudy. I rekl jemu: Pod sem, a posed tuto. Kterýž zastaviv se, sedl.
\par 2 A vzav Bóz deset mužu z starších mesta toho, rekl: Posadte se tuto. I posadili se.
\par 3 Tedy rekl príbuznému tomu: Díl pole, kteréž bylo bratra našeho Elimelecha, prodala Noémi, kteráž se navrátila z krajiny Moábské.
\par 4 A já jsem umínil netajiti toho pred tebou, a pravímt: Ujmi pole to pred prísedícími temito a staršími lidu mého. Jestliže chceš koupiti, kup; pakli nekoupíš, oznam mi. Nebo vím, že krome tebe není žádného, kterýž by mel právo koupiti je, a já jsem po tobe. Tedy on rekl: Já koupím.
\par 5 I rekl Bóz: Když ujmeš pole to od Noémi, tedy i Rut Moábskou, manželku mrtvého, sobe pojmeš, abys vzbudil jméno mrtvého v dedictví jeho.
\par 6 Odpovedel príbuzný ten: Nemohut koupiti, abych snad nezahladil dedictví svého. Uživ ty práva príbuznosti mé, nebo já ho nemohu užiti.
\par 7 (Byl pak ten obycej od starodávna v Izraeli pri koupi a smenách, ku potvrzení všelijakého jednání, že szul jeden obuv svou, a dal ji druhému. A to bylo na svedectví té veci v Izraeli.)
\par 8 Protož rekl príbuzný Bózovi: Ujmi ty. I szul obuv svou.
\par 9 Tedy rekl Bóz starším tem a všemu lidu: Vy svedkové jste dnes, že jsem ujal všecko, což bylo Elimelechovo, i všecko to, což bylo Chelionovo a Mahalonovo, od Noémi.
\par 10 Ano i Rut Moábskou, ženu Mahalonovu, vzal jsem sobe za manželku, abych vzbudil jméno mrtvého v dedictví jeho, a aby nebylo zahlazeno jméno mrtvého z bratrí jeho, a z brány místa jeho. Vy svedkové jste dnes toho.
\par 11 I rekl všecken lid, kterýž byl v bráne mesta, i starší: Svedkové jsme. Dejž Hospodin, aby žena vcházející do domu tvého byla jako Ráchel a jako Lía, kteréžto dve vzdelaly dum Izraelský. Pocínejž sobe zmužile v Efrate, a dosáhni jména v Betléme.
\par 12 A at jest dum tvuj jako dum Fáresa, (kteréhož porodila Támar Judovi,) z semene toho, kteréž by dal tobe Hospodin s mladicí touto.
\par 13 A tak Bóz pojal sobe Rut, a byla manželkou jeho. A když všel k ní, dal jí to Hospodin, že pocala a porodila syna.
\par 14 I rekly ženy Noémi: Požehnaný Hospodin, kterýž nedopustil toho, abys mela zbavena býti príbuzného v tento cas, tak aby trvalo v Izraeli jméno jeho.
\par 15 Ont ocerství duši tvou, a chovati te bude v starosti tvé; nebo nevesta tvá, kteráž te miluje, porodila ho, kteráž tobe lepší jest, nežli sedm synu.
\par 16 Tedy vzavši Noémi díte, položila je na klín svuj, a byla pestounkou jeho.
\par 17 Daly mu pak jméno sousedy, kteréž pravily: Narodil se syn Noémi, a nazvaly ho jménem Obéd. Ont jest otec Izai, otce Davidova.
\par 18 A tito jsou rodové Fáresovi: Fáres zplodil Ezrona;
\par 19 Ezron pak zplodil Rama, Ram pak zplodil Aminadaba;
\par 20 Aminadab pak zplodil Názona, Názon pak zplodil Salmona;
\par 21 Salmon pak zplodil Bóza, Bóz pak zplodil Obéda;
\par 22 Obéd pak zplodil Izai, Izai pak zplodil Davida.

\end{document}