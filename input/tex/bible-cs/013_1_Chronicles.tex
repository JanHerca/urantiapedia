\begin{document}

\title{1 Chronicles}

\chapter{1}

\par 1 Adam, Set, Enos,
\par 2 Kainan, Mahalaleel, Járed,
\par 3 Enoch, Matuzalém, Lámech,
\par 4 Noé, Sem, Cham a Jáfet.
\par 5 Synové Jáfetovi: Gomer, Magog, Madai, Javan, Tubal, Mešech a Tiras.
\par 6 Synové pak Gomerovi: Ascenez, Difat a Togorma.
\par 7 Synové pak Javanovi: Elisa, Tarsis, Cetim a Rodanim.
\par 8 Synové Chamovi: Chus, Mizraim, Put a Kanán.
\par 9 A synové Chusovi: Sába, Evila, Sabata, Regma, Sabatacha. Synové pak Regmovi: Sába a Dedan.
\par 10 Zplodil také Chus Nimroda; ten pocal mocným býti na zemi.
\par 11 Mizraim pak zplodil Ludim, Anamim, Laabim a Neftuim,
\par 12 Fetruzim také a Chasluim, (odkudž pošli Filistinští), a Kafturim.
\par 13 Kanán pak zplodil Sidona, prvorozeného svého, a Het,
\par 14 A Jebuzea, Amorea a Gergezea,
\par 15 A Hevea, Aracea a Sinea,
\par 16 A Aradia, Samarea a Amatea.
\par 17 Synové Semovi: Elam, Assur, Arfaxad, Lud, Aram, Hus a Hul, Geter a Mas.
\par 18 A Arfaxad zplodil Sále, Sále pak zplodil Hebera.
\par 19 Heberovi pak narodili se dva synové, z nichž jednoho jméno Peleg, proto že za dnu jeho rozdelena byla zeme, jméno pak bratra jeho Jektan.
\par 20 Kterýžto Jektan zplodil Elmodada, Salefa, Azarmota a Járe,
\par 21 A Adoráma, Uzala a Dikla,
\par 22 A Ebale, Abimahele a Sebai,
\par 23 A Ofira, Evila a Jobaba. Všickni ti byli synové Jektanovi.
\par 24 Sem, Arfaxad, Sále,
\par 25 Heber, Peleg, Réhu,
\par 26 Sárug, Náchor, Táre,
\par 27 Abram, ten jest Abraham.
\par 28 Synové Abrahamovi: Izák a Izmael.
\par 29 Tito jsou rodové jejich: Prvorozený Izmaeluv Nabajot, Cedar, Adbeel a Mabsan,
\par 30 Masma, Dumah, Massa, Hadad a Tema,
\par 31 Jetur, Nafis a Cedma. Ti jsou synové Izmaelovi.
\par 32 Synové pak Cetury, ženiny Abrahamovy: Ta porodila Zamrana, Jeksana, Madana, Madiana, Jezbocha a Suecha. Synové pak Jeksanovi: Sába a Dedan.
\par 33 Synové pak Madianovi: Efa, Efer, Enoch, Abida a Helda. Všickni ti synové Cetury.
\par 34 Zplodil tedy Abraham Izáka. Synové pak Izákovi: Ezau a Izrael.
\par 35 Synové Ezau: Elifaz, Rahuel, Jehus, Jhelom a Kore.
\par 36 Synové Elifazovi: Teman, Omar, Sefi, Gatam, Kenaz a syn Tamny, totiž Amalech.
\par 37 Synové Rahuelovi: Nahat, Zára, Samma a Méza.
\par 38 Synové pak Seir: Lotan, Sobal, Sebeon, Ana, Dison, Eser a Dízan.
\par 39 Synové pak Lotanovi: Hori a Homam. Sestra pak Lotanova: Tamna.
\par 40 Synové Sobalovi: Alian, Manáhat, Ebal, Sefi a Onam. Synové pak Sebeonovi: Aia a Ana.
\par 41 Synové Anovi: Dison. A synové Disonovi: Hamran, Eseban, Jetran a Charan.
\par 42 Synové Eser: Balaan, Závan a Jakan. Synové Dízonovi: Hus a Aran.
\par 43 Tito pak jsou králové, kteríž kralovali v zemi Idumejské, prvé než kraloval který král z synu Izraelských: Béla syn Beoruv, jehožto mesto jméno melo Denaba.
\par 44 A když umrel Béla, kraloval na míste jeho Jobab, syn Záre z Bozra.
\par 45 A když umrel Jobab, kraloval místo neho Husam z zeme Temanské.
\par 46 A když umrel Husam, kraloval místo neho Adad syn Badaduv, kterýž porazil Madianské v krajine Moábské; jehož mesto jméno melo Avith.
\par 47 A když umrel Adad, kraloval na míste jeho Semla z Masreka.
\par 48 A když umrel Semla, kraloval místo neho Saul z Rohobot reky.
\par 49 A když umrel Saul, kraloval místo neho Bálanan, syn Achoboruv.
\par 50 A když umrel Bálanan, kraloval místo neho Adad, jehož mesto recené Pahu; jméno pak ženy jeho Mehetabel, dcera Matredy, dcery Mezábovy.
\par 51 A když umrel Adad, byli vývodové Idumejští: Vývoda Tamna, vývoda Alja, vývoda Jetet,
\par 52 Vývoda Olibama, vývoda Ela, vývoda Finon,
\par 53 Vývoda Kenaz, vývoda Teman, vývoda Mabsar,
\par 54 Vývoda Magdiel, vývoda Híram. Ti byli vývodové Idumejští.

\chapter{2}

\par 1 Tito jsou synové Izraelovi: Ruben, Simeon, Léví, Juda, Izachar a Zabulon,
\par 2 Dan, Jozef, Beniamin, Neftalím, Gád a Asser.
\par 3 Synové Judovi: Her, Onan a Séla. Ti tri narodili se jemu z dcery Suovy Kananejské. Ale Her, prvorozený Juduv, byl zlý pred ocima Hospodinovýma, protož zabil ho.
\par 4 Támar pak nevesta jeho porodila mu Fáresa a Záru. Všech synu Judových pet.
\par 5 Synové Fáresovi: Ezron a Hamul.
\par 6 Synové pak Záre: Zamri, Etan, Héman, Kalkol a Dára, všech tech pet.
\par 7 A synové Zamri: Charmi, vnuk Achar, kterýž zkormoutil Izraele, zhrešiv pri veci proklaté.
\par 8 Synové pak Etanovi: Azariáš.
\par 9 Synové pak Ezronovi, kteríž se mu zrodili: Jerachmeel, Ram a Chelubai.
\par 10 Ram pak zplodil Aminadaba, a Aminadab zplodil Názona, kníže synu Juda.
\par 11 Názon pak zplodil Salmona, a Salmon zplodil Bóza.
\par 12 A Bóz zplodil Obéda, a Obéd zplodil Izai.
\par 13 Izai pak zplodil prvorozeného svého Eliaba, a Abinadaba druhého, a Sammu tretího,
\par 14 Natanaele ctvrtého, Raddaia pátého,
\par 15 Ozema šestého, Davida sedmého,
\par 16 A sestry jejich: Sarvii a Abigail. Synové pak Sarvie byli: Abizai, Joáb, Azael, tri.
\par 17 Abigail pak porodila Amazu, otec pak Amazuv byl Jeter Izmaelitský.
\par 18 Kálef pak syn Ezronuv zplodil s Azubou manželkou a s Jeriotou syny. Jehož tito synové byli: Jeser, Sobab a Ardon.
\par 19 Když pak umrela Azuba, pojal sobe Kálef Efratu, kteráž mu porodila Hura.
\par 20 A Hur zplodil Uri, a Uri zplodil Bezeleele.
\par 21 Potom všel Ezron k dceri Machira otce Galádova, kterouž on pojal, když byl v šedesáti letech. I porodila jemu Seguba.
\par 22 Segub pak zplodil Jaira, kterýž mel trimecítma mest v zemi Galád.
\par 23 Nebo vzal Gessurejským a Assyrským vsi Jairovy, i Kanat s mestecky jeho, šedesáte mest.To všecko pobrali synové Machirovi, otce Galádova.
\par 24 Též i po smrti Ezronove, když již pojal byl Kálef Efratu, manželka Ezronova Abia porodila jemu také Ashura, otce Tekoa.
\par 25 Byli pak synové Jerachmeele prvorozeného Ezronova: Prvorozený Ram, po nem Buna a Oren, a Ozem s Achia.
\par 26 Mel také manželku druhou Jerachmeel, jménem Atara. Ta jest matka Onamova.
\par 27 Byli pak synové Ramovi prvorozeného Jerachmeele: Maaz a Jamin a Eker.
\par 28 Též synové Onamovi byli: Sammai a Jáda. A synové Sammai: Nádab a Abisur.
\par 29 Jméno pak manželky Abisurovy Abichail; ktéráž porodila jemu Achbana a Molida.
\par 30 A synové Nádabovi: Seled a Appaim. Ale umrel Seled bez detí.
\par 31 Synové pak Appaimovi: Jesi; a synové Jesi: Sesan; a dcera Sesanova: Achlai.
\par 32 Synové pak Jády, bratra Sammaiova: Jeter a Jonatan. Ale umrel Jeter bez detí.
\par 33 Synové pak Jonatanovi: Felet a Záza. Ti byli synové Jerachmeelovi.
\par 34 Nemel pak Sesan synu, ale dceru. A mel Sesan služebníka Egyptského jménem Jarchu.
\par 35 Protož dal Sesan dceru svou Jarchovi služebníku svému za manželku, kteráž porodila mu Attaie.
\par 36 Attai pak zplodil Nátana, a Nátan zplodil Zabada.
\par 37 Zabad pak zplodil Eflale, a Eflal zplodil Obéda.
\par 38 Obéd pak zplodil Jéhu, a Jéhu zplodil Azariáše.
\par 39 Azariáš pak zplodil Cheleza, a Chelez zplodil Elasu.
\par 40 Elasa pak zplodil Sismaie, a Sismai zplodil Salluma.
\par 41 Sallum pak zplodil Jekamiáše, a Jekamiáš zplodil Elisama.
\par 42 Synové pak Kálefa, bratra Jerachmeelova: Mésa prvorozený jeho. On byl otec Zifejských i synu Marese, otce Hebronova.
\par 43 Synové pak Hebronovi: Chóre a Tapuach, a Rekem a Sema.
\par 44 Sema pak zplodil Rachama otce Jorkeamova, a Rekem zplodil Sammaie.
\par 45 Syn pak Sammai byl Maon; kterýžto Maon byl otec Betsurských.
\par 46 Efa také, ženina Kálefova, porodila Chárana a Mozu a Gazeza. A Cháran zplodil Gazeza.
\par 47 Synové pak Johedai: Regem, Jotam, Gesan, Felet, Efa a Saaf.
\par 48 S ženinou Maachou Kálef zplodil Sebera a Tirchana.
\par 49 Porodila pak Saafa otce Madmanejských, Sévu otce Makbenejských a otce Gibejských. Též dcera Kálefova Axa.
\par 50 Ti byli synové Kálefovi, syna Hur prvorozeného Efraty: Sobal otec Kariatjeharimských,
\par 51 Salma otec Betlémských, Charef otec celedi Betgaderských.
\par 52 Mel pak syny Sobal otec Kariatjeharimských: Haroe otce obyvatelu dílu Menuchotských.
\par 53 A celedi Kariatjeharimských, Jeterských, Putských, Sumatských a Misraiských. Z tech pošli Zaratští a Estaolští.
\par 54 Synové Salmy: Betlémští, Netofatští, Atarotští z celedi Joábovy, a Zarští, kteríž užívali dílu Menuchotských,
\par 55 A celedi písaru obývajících v Jábezu, Tiratských, Simatských, Suchatských. Ti jsou Cinejští príchozí z Amata, otce celedi Rechabovy.

\chapter{3}

\par 1 Tito jsou pak synové Davidovi, kteríž se jemu zrodili v Hebronu: Prvorozený Amnon z Achinoam Jezreelské, druhý Daniel z Abigail Karmelské;
\par 2 Tretí Absolon syn Maachy, dcery Tolmai, krále Gessur, ctvrtý Adoniáš, syn Haggit;
\par 3 Pátý Sefatiáš z Abitál, šestý Jetram, z Egly manželky jeho.
\par 4 Šest se mu jich zrodilo v Hebronu, kdež kraloval sedm let a šest mesícu; tridceti pak a tri kraloval v Jeruzaléme.
\par 5 Potom tito se jemu zrodili v Jeruzaléme: Sammua, Sobab, Nátan a Šalomoun, ctyri, z Betsabé, dcery Amielovy;
\par 6 Též Ibchar, Elisama a Elifelet;
\par 7 A Noga, Nefeg a Jafia;
\par 8 A Elisama, Eliada a Elifelet, tech devet.
\par 9 Všickni ti synové Davidovi krom synu ženin, a Támar sestra jejich.
\par 10 Syn pak Šalomounuv Roboám, Abiam syn jeho, Aza syn jeho, Jozafat syn jeho,
\par 11 Joram syn jeho, Ochoziáš syn jeho, Joas syn jeho,
\par 12 Amaziáš syn jeho, Azariáš syn jeho, Jotam syn jeho,
\par 13 Achas syn jeho, Ezechiáš syn jeho, Manasses syn jeho,
\par 14 Amon syn jeho, Joziáš syn jeho.
\par 15 Synové pak Joziášovi: Prvorozený Jochanan, druhý Joakim, tretí Sedechiáš, ctvrtý Sallum.
\par 16 Synové pak Joakimovi: Jekoniáš syn jeho, Sedechiáš syn jeho.
\par 17 Synové pak Jekoniáše vezne: Salatiel syn jeho.
\par 18 Toho pak Malkiram, Pedai, Senazar, Jekamia, Hosama a Nedabia.
\par 19 Synové pak Pedaiovi: Zorobábel a Semei. A syn Zorobábeluv: Mesullam, Chananiáš, a Selomit sestra jejich.
\par 20 Toho pak Chasuba, Ohel, Berechiáš, Chasadiáš a Jusabchesed, tech pet.
\par 21 Syn pak Chananiášuv: Pelatia a Izaiáš. Synové Refaie, synové Arnanovi, synové Abdiášovi, synové Sechaniovi.
\par 22 A synové Sechaniovi: Semaiáš. A synové Semaiášovi: Chattus, Igal, Bariach, Neariáš a Safat, šest.
\par 23 A syn Neariáše: Elioenai, Ezechiáš a Azrikam, ti tri.
\par 24 Též synové Elioenai: Hodaviáš, Eliasib, Pelaiáš, Akkub, Jochanan, Delaiáš a Anani, tech sedm.

\chapter{4}

\par 1 Synové Judovi: Fáres, Ezron, Charmi, Hur a Sobal.
\par 2 Reaiáš pak syn Sobaluv zplodil Jachata, Jachat pak zplodil Ahumai a Laad. Ti jsou rodové Zarati.
\par 3 A tito z otce Etama: Jezreel, Isma a Idbas; a jméno sestry jejich Zelelfoni.
\par 4 Fanuel pak otec Gedor, a Ezer otec Chusuv. Ti jsou synové Hur prvorozeného Efraty, otce Betlémských.
\par 5 Ashur pak otec Tekoe mel dve manželky, Chélu a Naaru.
\par 6 I porodila jemu Naara Achuzama, Hefera, Temana a Achastara. Ti jsou synové Naary.
\par 7 Synové pak Chéle: Zeret, Jezochar a Etnan.
\par 8 Kóz pak zplodil Anuba, Hazobeba, a rodiny Acharchele syna Harumova.
\par 9 Byl pak Jábez slavnejší nad bratrí své, a matka jeho nazvala jméno jeho Jábez, rkuci: Nebo jsem ho porodila s bolestí.
\par 10 Kterýžto Jábez vzýval Boha Izraelského, rka: Jestliže štedre požehnáš mi, a rozšíríš meze mé, a bude ruka tvá se mnou, a vysvobodíš mne od zlého, abych bolesti netrpel. I ucinil Buh, zacež žádal.
\par 11 Chelub pak, bratr Sucha, zplodil Mechiru. Ont jest otec Estonuv.
\par 12 Eston pak zplodil Betrafa, Paseacha a Techinna, otce mesta Náchas. Ti jsou muži Rechy.
\par 13 Synové pak Cenezovi: Otoniel a Saraiáš. Synové pak Otonielovi: Chatat.
\par 14 Meonatai pak zplodil Ofru, Saraiáš pak zplodil Joába, otce bydlících v údolí remeslníku; nebo tam remeslníci byli.
\par 15 Synové pak Kálefa, syna Jefonova: Iru, Ela a Naam. Syn pak Ela: Cenez.
\par 16 Synové pak Jehalleleelovi: Zif, Zifa, Tiriáš a Asarel.
\par 17 A synové Ezry: Jeter, Mered, Efer a Jalon. Porodila také Miriama, Sammai a Jezba otce Estemo.
\par 18 Manželka pak jeho Jehudia porodila Jereda otce Gedor, a Hebera otce Socho, a Jekutiele otce Zanoe. A ti jsou synové Betie dcery Faraonovy, kterouž pojal Mered.
\par 19 Synové pak manželky Hodia sestry Nachamovy, otce Cejly: Garmi a Estemo Maachatský.
\par 20 Synové pak Simonovi: Amnon, Rinna, Benchanan a Tilon. A synové Jesi: Zochet a Benzochet.
\par 21 Synové Séla syna Judova: Her otec Lechuv, a Lada otec Maresuv, a celedi domu tech, jenž delali díla kmentová v dome Asbea,
\par 22 A Jokim a muži Chozeby, Joas a Saraf, kteríž panovali v Moáb, a Jasubi Lechem. Ale ty veci jsou starodávní.
\par 23 Tot jsou ti hrncíri obyvatelé v štepnicích a ohradách u krále, prícinou díla jeho tam bydlíce.
\par 24 Synové Simeonovi: Namuel, Jamin, Jarib, Zára a Saul.
\par 25 Sallum syn jeho, Mabsam syn jeho, Masma syn jeho.
\par 26 Synové pak Masmovi: Hamuel syn jeho, Zakur syn jeho, Semei syn jeho.
\par 27 Ten Semei mel synu šestnácte a dcer šest. Bratrí pak jejich nemeli mnoho synu, tak že vší rodiny jejich nebylo tak mnoho, jako synu Judových.
\par 28 Bydlili pak v Bersabé a Molada a v Azarsual,
\par 29 A v Bála, v Esem a v Tolad,
\par 30 A v Betueli, v Horma a v Sicelechu,
\par 31 A v Betmarchabot, a v Azarsusim, v Betberi a v Saraim. Ta byla mesta jejich, dokudž kraloval David.
\par 32 Vsi také jejich pri Etam, Ain, Remmon, Tochen, Asan, peti mestech.
\par 33 A tak všecky vesnice jejich, kteréž byly vukol tech mest až do Baal, ta byla obydlé jejich vedlé rodu jejich.
\par 34 A Mesobab, Jamlech a Josa syn Amazuv;
\par 35 A Joel, a Jéhu syn Jozabiáše, syna Saraiášova, syna Azielova;
\par 36 A Elioenai, Jákoba, Jesochaiáš, Asaiáš, Adiel a Jesimeel a Benaiáš;
\par 37 A Ziza syn Sifi, syna Allonova, syna Jedaiášova, syna Simri, syna Semaiášova.
\par 38 Tito ze jména vyctení ustaveni jsou za knížata v celedech svých, a celedi otcovské jejich rozmnožily se náramne.
\par 39 A protož brali se, kudyž se vchází k Gedor, až k východu po údolí tom, aby hledali pastev dobytku svému.
\par 40 I nalezli pastvu hojnou a výbornou, zemi pak prostrannou, bezpecnou a pokojnou, a že z Chama byli ti, kteríž bydlili tam pred tím.
\par 41 Protož prišedše ti napsaní ze jména ve dnech Ezechiáše krále Judského, poborili stany jejich a príbytky, kteríž tam nalezeni byli, a zmordovali je, tak že jich do tohoto dne není, a sami osedli místo nich; nebo meli tu pastvu dobytku svému.
\par 42 Nekterí pak z tech synu Simeonových odebrali se na horu Seir, mužu pet set, jichž Pelatia, Neariáš, Refaiáš a Uziel, synové Jesi, byli vudcové.
\par 43 I vyplénili ostatek tech, kteríž ušli z Amalechitských, a bydlili tam až do tohoto dne.

\chapter{5}

\par 1 Synové pak Rubenovi prvorozeného Izraelova: (nebo on byl prvorozený, ale když poškvrnil lože otce svého, dáno jest prvorozenství jeho synum Jozefovým, syna Izraelova, však jemu není pricteno prvorozenství.
\par 2 Nebo Judas byl nejsilnejší z bratrí svých, a kníže mezi nimi, ale prvorozenství Jozefovi náleželo).
\par 3 Synové, pravím, Rubenovi, prvorozeného Izraelova: Enoch, Fallu, Ezron a Charmi.
\par 4 Synové Joelovi: Semaiáš syn jeho, Gog syn jeho, Semei syn jeho;
\par 5 Mícha syn jeho, Reaiáš syn jeho, Bál syn jeho;
\par 6 Béra syn jeho, jehož zavedl Tiglatfalazar král Assyrský. Ten byl knížetem pokolení Rubenova.
\par 7 Bratrí pak jeho po celedech jejich, když vycteni byli po rodinách svých, knížetem byl Jehiel a Zachariáš.
\par 8 A Béla syn Azaza, syna Semy, syna Joelova. Ten bydlil v Aroer až do Nébo a Balmeon.
\par 9 Potom i na východ bydlil, až kudy se vchází na poušt od reky Eufrates; nebo stáda jejich rozmnožila se v zemi Galádské.
\par 10 Procež ve dnech Saulových bojovali s Agarenskými, kteríž poraženi jsou od ruky jejich. A tak bydlili v staních jejich po vší krajine východní zeme Galádské.
\par 11 Synové pak Gádovi naproti nim bydlili v zemi Bázan, až do Sálechy.
\par 12 Joel byl kníže jejich, a Safan druhý. Ale Janai a Safat v Bázan zustali.
\par 13 Bratrí pak jejich po domích otcu svých: Michael, Mesullam, Seba, Jorai, Jachan, Zia, Heber, tech sedm.
\par 14 (Ti byli synové Abichailovi, synové Hurovi, synové Jaroachovi, synové Galádovi, synové Michaelovi, synové Jesisovi, synové Jachdovi, synové Buzovi.)
\par 15 Též Ahi syn Abdiele, syna Gunova, kníže v dome otcu jich.
\par 16 I bydlili v Galád, v Bázan a v vesnicech jeho, a ve všech predmestích Sáron, až do hranic jejich.
\par 17 Všickni tito vycteni byli ve dnech Jotama krále Judského, a ve dnech Jeroboáma krále Izraelského.
\par 18 Synu Rubenových a Gádových a polovice pokolení Manassesova, mužu silných, nosících štít a mec, a natahujících lucište, a umelých v bitve, ctyridceti a ctyri tisíce, sedm set a šedesát vycházejících k boji.
\par 19 I bojovali s Agarenskými, Iturejskými, Nafejskými a Nodabskými.
\par 20 A meli pomoc proti nim. I dáni jsou v ruku jich Agarenové i všecko, což meli. Nebo k Bohu volali v boji, a vyslyšel je, nebo doufali v neho.
\par 21 I zajali stáda jejich, velbloudu jejich padesáte tisícu, a dobytka dve ste a padesáte tisícu, a oslu dva tisíce, a lidí sto tisíc osob.
\par 22 Zranených také množství padlo, nebo od Boha byla porážka ta. I bydlili na míste jejich až do prestehování svého.
\par 23 Synové pak polovice pokolení Manassesova bydlili v té zemi od Bázanu až do Balhermon i Sanir, totiž hory Hermon; nebo i oni rozmnoženi byli.
\par 24 Tato pak byla knížata v dome otcu jejich: Efer, Jesi, Eliel, Azriel, Jeremiáš, Hodaviáš a Jachdiel, muži udatní a silní, muži slovoutní, knížata domu otcu svých.
\par 25 Ale když prestoupili proti Bohu otcu svých, a smilnili, následujíce bohu národu zeme té, kteréž shladil Buh od tvári jejich:
\par 26 Vzbudil Buh Izraelský ducha Fule krále Assyrského, a ducha Tiglatfalazara krále Assyrského, kterýž prenesl pokolení Rubenovo a Gádovo a polovici pokolení Manassesova, a dovedl je do Chelach a do Chabor, až do Hara a k rece Gozan až do dnešního dne.

\chapter{6}

\par 1 Synové Léví: Gerson, Kahat a Merari.
\par 2 Synové pak Kahat: Amram, Izar, Hebron a Uziel.
\par 3 Synové pak Amramovi: Aron, Mojžíš, a Maria. Synové pak Aronovi: Nádab, Abiu, Eleazar a Itamar.
\par 4 Eleazar zplodil Fínesa, Fínes zplodil Abisua.
\par 5 Abisua pak zplodil Bukki, Bukki pak zplodil Uzi.
\par 6 Uzi pak zplodil Zerachiáše, Zerachiáš pak zplodil Meraiota.
\par 7 Meraiot zplodil Amariáše, Amariáš pak zplodil Achitoba.
\par 8 Achitob pak zplodil Sádocha, Sádoch pak zplodil Achimaasa.
\par 9 Achimaas pak zplodil Azariáše, Azariáš pak zplodil Jochanana.
\par 10 Jochanan pak zplodil Azariáše. Ont jest užíval knežství v dome, jejž ustavel Šalomoun v Jeruzaléme.
\par 11 Zplodil pak Azariáš Amariáše, Amariáš pak zplodil Achitoba.
\par 12 Achitob zplodil Sádocha, Sádoch pak zplodil Salluma.
\par 13 Sallum pak zplodil Helkiáše, Helkiáš pak zplodil Azariáše.
\par 14 Azariáš pak zplodil Saraiáše, Saraiáš pak zplodil Jozadaka.
\par 15 Jozadak pak odšel, když prevedl Hospodin Judu a Jeruzalém skrze Nabuchodonozora.
\par 16 Synové Léví: Gersom, Kahat a Merari.
\par 17 Tato pak jsou jména synu Gersomových: Lebni a Semei.
\par 18 Synové pak Kahat: Amram, Izar, Hebron a Uziel.
\par 19 Synové Merari: Moholi a Musi. A tak ty jsou celedi Levítu po otcích jejich.
\par 20 Gersomovi: Lebni syn jeho, Jachat syn jeho, Zimma syn jeho,
\par 21 Joach syn jeho, Iddo syn jeho, Zára syn jeho, Jetrai syn jeho.
\par 22 Synové Kahat: Aminadab syn jeho, Chóre syn jeho, Assir syn jeho,
\par 23 Elkána syn jeho, a Abiazaf syn jeho, Assir syn jeho,
\par 24 Tachat syn jeho, Uriel syn jeho, Uziáš syn jeho, Saul syn jeho.
\par 25 Synové pak Elkánovi: Amasai a Achimot.
\par 26 Elkána: Synové Elkánovi: Zofai syn jeho, a Nachat syn jeho,
\par 27 Eliab syn jeho, Jerocham syn jeho, Elkána syn jeho.
\par 28 Synové pak Samuelovi: Prvorozený Vasni a Abia.
\par 29 Synové Merari: Moholi, Lebni syn jeho, Semei syn jeho, Uza syn jeho,
\par 30 Sima syn jeho, Aggia syn jeho, Azaiáš syn jeho.
\par 31 Tito jsou pak, kteréž ustanovil David k zpívání v dome Hospodinove, když tam postavena truhla,
\par 32 Kteríž prisluhovali pred príbytkem stánku úmluvy zpíváním, dokudž neustavel Šalomoun domu Hospodinova v Jeruzaléme, a stáli podlé porádku svého v prisluhování svém.
\par 33 Tito jsou pak, kteríž stáli, i synové jejich, z synu Kahat: Héman kantor, syn Joele, syna Samuelova,
\par 34 Syna Elkánova, syna Jerochamova, syna Elielova, syna Tohu,
\par 35 Syna Sufova, syna Elkánova, syna Machatova, syna Amasai,
\par 36 Syna Elkánova, syna Joelova, syna Azariášova, syna Sofoniášova,
\par 37 Syna Tachatova, syna Assirova, syna Abiazafova, syna Chóre,
\par 38 Syna Izarova, syna Kahatova, syna Léví, syna Izraelova.
\par 39 A bratr jeho Azaf, kterýž stával po pravici jeho. Azaf pak byl syn Berechiáše, syna Simova,
\par 40 Syna Michaelova, syna Baaseiášova, syna Malkiášova,
\par 41 Syna Etni, syna Záry, syna Adaiova,
\par 42 Syna Etanova, syna Zimmova, syna Semeiova,
\par 43 Syna Jachatova, syna Gersomova, syna Léví.
\par 44 Synové pak Merari, bratrí jejich, stávali po levici: Etan, syn Kísi, syna Abdova, syna Malluchova,
\par 45 Syna Chasabiášova, syna Amaziášova, syna Helkiášova,
\par 46 Syna Amzova, syna Bánova, syna Semerova,
\par 47 Syna Moholi, syna Musi, syna Merari, syna Léví.
\par 48 Bratrí pak jejich Levítové jiní oddáni jsou ke všelikému prisluhování príbytku domu Božího.
\par 49 Ale Aron a synové jeho pálili na oltári zápalu a na oltári kadení, pri všelikém prisluhování svatyne svatých, a k ocištování Izraele podlé všeho toho, jakož prikázal Mojžíš služebník Boží.
\par 50 Tito pak jsou synové Aronovi: Eleazar syn jeho, Fínes syn jeho, Abisua syn jeho,
\par 51 Bukki syn jeho, Uzi syn jeho, Zerachiáš syn jeho,
\par 52 Meraiot syn jeho, Amariáš syn jeho, Achitob syn jeho,
\par 53 Sádoch syn jeho, Achimaas syn jeho.
\par 54 A tato obydlé jejich, po príbytcích jejich, v mezech jejich, synu Aronových po celedi Kahatských; nebo jejich byl los.
\par 55 A protož dali jim Hebron v zemi Judské, a predmestí jeho vukol neho.
\par 56 Pole však mestská a vsi jejich dali Kálefovi synu Jefonovu.
\par 57 Synum pak Aronovým dali z mest Judských mesta útocištná: Hebron a Lebno a predmestí jeho, a Jeter i Estemo a predmestí jeho,
\par 58 A Holon i predmestí jeho, a Dabir i predmestí jeho,
\par 59 Též Asan a predmestí jeho, a Betsemes a predmestí jeho.
\par 60 Z pokolení pak Beniamin: Gaba a predmestí jeho, a Allemet i predmestí jeho, i Anatot a predmestí jeho, všech mest jejich trinácte mest po celedech jejich.
\par 61 Synum též Kahatovým ostatním z celedi toho pokolení dáno v polovici pokolení Manassesova losem mest deset.
\par 62 Synum pak Gersonovým po celedech jejich v pokolení Izachar a v pokolení Asser, a v pokolení Neftalím, a v pokolení Manassesovu v Bázan mest trináct.
\par 63 Synum Merari po celedech jejich v pokolení Ruben, a v pokolení Gád, a v pokolení Zabulon losem mest dvanáct.
\par 64 Dali synové Izraelští Levítum ta mesta a predmestí jejich.
\par 65 A dali je losem v pokolení synu Judových, a v pokolení synu Simeonových, a v pokolení synu Beniaminových, mesta ta, kteráž jmenovali ze jména.
\par 66 A kteríž byli z celedi synu Kahat, (byla pak mesta a hranice jejich v pokolení Efraimovu),
\par 67 Tem dali mesta útocištná: Sichem a predmestí jeho na hore Efraim, a Gázer a predmestí jeho,
\par 68 A Jekmaam i predmestí jeho, a Betoron i predmestí jeho,
\par 69 Též i Aialon a predmestí jeho, a Getremmon s predmestím jeho.
\par 70 A v polovici pokolení Manassesova: Aner a predmestí jeho, Balám a predmestí jeho, celedem synu Kahat ostatním.
\par 71 Synum pak Gersonovým v celedi polovice pokolení Manassesova dali Golan v Bázan s predmestím jeho, a Astarot i predmestí jeho.
\par 72 V pokolení pak Izachar: Kádes s predmestím jeho, Daberet a predmestí jeho,
\par 73 Rámot také s predmestím jeho, a Anem i predmestí jeho.
\par 74 V pokolení pak Asser: Masal s predmestím jeho, a Abdon i predmestí jeho,
\par 75 Též Hukok s predmestím jeho, Rohob také i predmestí jeho.
\par 76 V pokolení pak Neftalímovu: Kádes v Galilei a predmestí jeho, Hamon a predmestí jeho, a Kariataim i predmestí jeho.
\par 77 Synum Merari ostatním v pokolení Zabulonovu dali Remmon s predmestím jeho, Tábor a predmestí jeho.
\par 78 A za Jordánem u Jericha, k východní strane Jordánu, v pokolení Rubenovu: Bozor na poušti s predmestím jeho, a Jasa i predmestí jeho.
\par 79 Kedemot také s predmestím jeho, a Mefat i predmestí jeho.
\par 80 V pokolení pak Gád: Rámot v Galád s predmestím jeho, a Mahanaim i predmestí jeho,
\par 81 I Ezebon s predmestím jeho, a Jazer i predmestí jeho.

\chapter{7}

\par 1 Synové pak Izacharovi: Tola, Fua, Jasub a Simron, ctyri.
\par 2 Synové pak Tolovi: Uzi, Refaia, Jeriel, Jachmai, Jipsam, Samuel, knížata po domích otcu jejich, pošlí od Toly, muži udatní v pokoleních svých. Pocet jejich ve dnech Davidových byl dvamecítma tisícu a šest set.
\par 3 Synové Uzovi: Izrachiáš. Synové pak Izrachiášovi: Michael, Abdiáš, Joel a Isia, všech pet knížat.
\par 4 A s nimi v pokoleních jejich, po celedech jejich otcovských, mužu válecných tridceti šest tisícu; nebo mnoho meli žen a synu.
\par 5 Bratrí také jejich po všech celedech Izachar, mužu udatných osmdesáte sedm tisícu, všech vyctených.
\par 6 Synové Beniaminovi: Béla, Becher, Jediael, tri.
\par 7 Synové pak Bélovi: Ezbon, Uzi, Uziel, Jerimot a Iri, pet knížat celedí otcovských, muži udatní; nacteno jich dvamecítma tisícu, tridceti a ctyri.
\par 8 Potom synové Becherovi: Zemira, Joas, Eliezer, Elioenai, Amri, Jeremot, Abiáš, Anatot a Alemet, všickni synové Becherovi.
\par 9 Kterýchž pocet po pokoleních jejich, a po knížatech v dome celedí otcovských, mužu udatných, dvadceti tisícu a dve ste.
\par 10 Synové také Jediaelovi: Bilan. Synové pak Bilanovi: Jeus, Beniamin, Ahod, Kenan, Zetan, Tarsis a Achisachar.
\par 11 Všech tech synu Jediaelových po knížatech celedí, mužu udatných, sedmnáct tisíc a dve ste, vycházejících na vojnu k bitve,
\par 12 Krome Suppim a Chuppim, synu doma zrozených, a Chusim, synu vne zplozených.
\par 13 Synové Neftalímovi: Jasiel, Guni, Jezer a Sallum, synové Bály.
\par 14 Synové Manassesovi: Asriel, kteréhož mu manželka porodila. (Ženina též jeho Syrská porodila Machira, otce Galád.
\par 15 Machir pak vzal manželku Chuppimovu a Suppimovu, a jméno sestry jeho Maacha.) Jméno pak druhého Salfad, a mel Salfad dcery.
\par 16 Porodila pak Maacha manželka Machirova syna, kteréhož nazvala Fáres, a jméno bratra jeho Sáres, synové pak jeho Ulam a Rekem.
\par 17 Synové pak Ulamovi: Bedan. Tit jsou synové Galád syna Machirova, syna Manassesova.
\par 18 Sestra pak jeho Molechet porodila Ishoda a Abiezera a Machla.
\par 19 Byli pak synové Semidovi: Achian, Sechem, Likchi a Aniam.
\par 20 Synové pak Efraimovi: Sutelach, a Bered syn jeho, Tachat syn jeho, Elada syn jeho, Tachat syn jeho,
\par 21 Též Zabad syn jeho, Sutelach syn jeho, Ezer a Elad. I zbili je muži Gát, kteríž zrozeni byli v zemi té; nebo sstoupili byli, aby zajali dobytky jejich.
\par 22 Protož kvílil Efraim otec jejich za mnohé dny, a prišli bratrí jeho, aby ho tešili.
\par 23 Potom všel k manželce své, kteráž pocala a porodila syna, a nazval jméno jeho Beria, že byl v zámutku pro rodinu svou.
\par 24 Dceru také Seeru, kteráž vystavela Betoron dolní i horní, a Uzen Seera.
\par 25 A Refacha syna jeho, Resefa, Telecha, a Tachana syna jeho,
\par 26 Ladana syna jeho, Amiuda syna jeho, Elisama syna jeho,
\par 27 Non syna jeho, Jozue syna jeho.
\par 28 Vládarství pak jejich a bydlení jejich Bethel s vesnicemi svými, a k východu Náran, a k západu Gázer a vesnice jeho, Sichem s vesnicemi svými, až do Gázy a vesnic jeho.
\par 29 A v místech naproti synum Manassesovým: Betsan s vesnicemi svými, Tanach s vesnicemi svými, Mageddo s vesnicemi svými, Dor s vesnicemi svými. V tech bydlili synové Jozefa syna Izraelova.
\par 30 Synové Asser: Jemna, Jesua, Jesui, Beria, a Serach sestra jejich.
\par 31 Synové pak Beriovi: Heber, Melchiel. Ont jest otec Birzavituv.
\par 32 Heber pak zplodil Jafleta, Somera, Chotama, a Suu sestru jejich.
\par 33 Synové pak Jafletovi: Pasach, Bimhal, a Asvat. Ti jsou synové Jafletovi.
\par 34 Synové pak Somerovi: Achi, Rohaga, Jehubba a Aram.
\par 35 Synové pak Helema, bratra jeho: Zofach, Jimna, Seles a Amal.
\par 36 Synové Zofachovi: Suach, Charnefer, Sual, Beri a Jimra,
\par 37 Bezer, Hod, Samma, Silsa, Jitran a Béra.
\par 38 Synové Jeterovi: Jefunne, Fispa a Ara.
\par 39 Synové pak Ulla: Arach, Haniel a Riziáš.
\par 40 Všickni ti synové Asser, knížata domu otcovských, vybraní, udatní, prední z knížat, kteríž vycteni do vojska k bitve, v poctu šest a dvadceti tisíc mužu.

\chapter{8}

\par 1 Beniamin pak zplodil Bélu, prvorozeného svého, Asbele druhého, Achracha tretího,
\par 2 Nocha ctvrtého, Rafa pátého.
\par 3 Béla pak mel syny: Addara, Geru, Abiuda,
\par 4 Abisua, Námana, Achoacha,
\par 5 A Geru, Sefufana a Churama.
\par 6 Ti jsou synové Echudovi, ti jsou knížata celedí otcovských, bydlících v Gabaa, kteríž je uvedli do Manáhat,
\par 7 Totiž: Náman, a Achia a Gera. On prestehoval je; zplodil pak Uza a Achichuda.
\par 8 Sacharaim pak zplodil v krajine Moábské, když onen byl propustil je, s Chusimou a Bárou manželkami svými.
\par 9 Zplodil s Chodes manželkou svou Jobaba, Sebia, Mésa a Malkama,
\par 10 Jehuza, Sachia a Mirma. Ti jsou synové jeho, knížata celedí otcovských.
\par 11 S Chusimou pak byl zplodil Abitoba a Elpále.
\par 12 Synové pak Elpálovi: Heber, Misam a Semer. Ten vystavel Ono a Lod, i vsi jeho.
\par 13 A Beria a Sema. Ti jsou knížata celedí otcovských, bydlících v Aialon; ti zahnali obyvatele Gát.
\par 14 Achio pak, Sasák a Jeremot,
\par 15 Zebadiáš, Arad a Ader,
\par 16 Michael, Ispa a Jocha synové Beria.
\par 17 A Zebadiáš, Mesullam, Chiski, Heber,
\par 18 Ismerai, Izliáš a Jobab synové Elpálovi.
\par 19 A Jakim, Zichri a Zabdi.
\par 20 Elienai, Ziletai a Eliel,
\par 21 Adaiáš, Baraiáš a Simrat synové Simei.
\par 22 Ispan a Heber a Eliel,
\par 23 Abdon, Zichri a Chanan,
\par 24 Chananiáš, Elam a Anatotiáš,
\par 25 Ifdaiáš a Fanuel synové Sasákovi.
\par 26 Samserai, Sechariáš a Ataliáš,
\par 27 Jaresiáš, Eliáš a Zichri synové Jerochamovi.
\par 28 Ta jsou knížata otcovských celedí po rodinách svých, kterážto knížata bydlila v Jeruzaléme.
\par 29 V Gabaon pak bydlilo kníže Gabaon, a jméno manželky jeho Maacha.
\par 30 A syn jeho prvorozený Abdon, Zur, Cis, Bál a Nádab,
\par 31 Ale Gedor, Achio, Zecher.
\par 32 A Miklot zplodil Simea. I ti také naproti bratrím svým bydlili v Jeruzaléme s bratrími svými.
\par 33 Ner pak zplodil Cisa, a Cis zplodil Saule. Saul pak zplodil Jonatu, Melchisua, Abinadaba a Ezbále.
\par 34 Syn pak Jonatuv Meribbál, Meribbál pak zplodil Mícha.
\par 35 Synové pak Míchovi: Piton, Melech, Tarea a Achaz.
\par 36 Achaz pak zplodil Jehoadu, Jehoada pak zplodil Alemeta, Azmaveta a Zimru. Zimri pak zplodil Mozu.
\par 37 Moza pak zplodil Bina. Ráfa syn jeho, Elasa syn jeho, Azel syn jeho.
\par 38 Azel pak mel šest synu, jichž tato jsou jména: Azrikam, Bochru, Izmael, Seariáš a Abdiáš a Chanan. Všickni ti synové Azelovi.
\par 39 Synové pak Ezeka, bratra jeho: Ulam prvorozený jeho, Jehus druhý, a Elifelet tretí.
\par 40 A byli synové Ulamovi muži udatní a strelci umelí, kteríž meli mnoho synu a vnuku až do sta a padesáti. Všickni ti byli z synu Beniaminových.

\chapter{9}

\par 1 A tak všickni Izraelští secteni byli, a aj, zapsáni jsou v knize králu Izraelských a Judských, a preneseni jsou do Babylona pro prestoupení své.
\par 2 Ti pak, kteríž bydlili prvé v vládarství svém v mestech svých, totiž Izraelští, kneží, Levítové a Netinejští,
\par 3 Bydlili v Jeruzaléme, z synu Judových, z synu Beniaminových, ano i z synu Efraimových a Manassesových:
\par 4 Uttai syn Amiuda, syna Amri, syna Imri, syna Bani, z synu Fáresových, syna Judova.
\par 5 A z celedi Silonovy: Azaiáš prvorozený a synové jeho.
\par 6 A z synu Zerachových: Jehuel, a príbuzných jejich šest set a devadesát.
\par 7 Z synu pak Beniaminových: Sallu syn Mesullama, syna Hodavia, syna Hasenuova,
\par 8 A Ibneiáš syn Jerochamuv, Ela syn Uzi, syna Michri, a Mesullam syn Sefatiáše, syna Rehuelova, syna Ibniášova.
\par 9 Bratrí také jejich po pokoleních jejich, devet set padesáte a šest; všickni ti muži knížata celedí po domích otcu svých.
\par 10 Z kneží také Jedaiáš, Jehoiarib a Jachin,
\par 11 Azariáš syn Helkiáše, syna Mesullamova, syna Sádochova, syna Meraiotova, syna Achitobova, kníže v dome Božím.
\par 12 A Adaiáš syn Jerochama, syna Paschurova, syna Malchiášova, a Masai syn Adiele, syna Jachzery, syna Mesullamova, syna Mesillemitova, syna Immerova.
\par 13 A bratrí jejich knížata po domích otcu svých, tisíc sedm set a šedesát mužu udatných v práci prisluhování domu Božího.
\par 14 A z Levítu: Semaiáš syn Chasuba, syna Azrikamova, syna Chasabiášova, z synu Merari.
\par 15 A Bakbakar Cheres, a Galal, a Mataniáš syn Míchy, syna Zichri, syna Azafova.
\par 16 A Abdiáš syn Semaiáše, syna Galalova, syna Jedutunova, a Berechiáš syn Asy, syna Elkánova, kterýž prebýval ve vsech Netofatských.
\par 17 Též vrátní: Sallum a Akkub, a Talmon, Achiman, i bratrí jejich, z nichž byl Sallum kníže.
\par 18 Kterýž až po dnes v bráne královské stával k východu, onino pak vrátnými byli po houfích synu Léví.
\par 19 Ale Sallum syn Chóre, syna Abiazafova, syna Chóre, a bratrí jeho z domu otce jeho, Chorejští, nad pracemi prisluhování, ostríhali prahu pri stánku, tak jako otcové jejich nad vojskem Hospodinovým, kteríž ostríhali vcházení.
\par 20 Nad kterýmiž Fínes syn Eleazaruv byl nekdy knížetem, a Hospodin byl s ním.
\par 21 Zachariáš pak syn Meselemiášuv vrátným byl u dverí stánku úmluvy.
\par 22 Všickni ti voleni za vrátné k verejím, osob dve ste a dvanáct. Ti ve vsech svých vycteni jsou, kteréž narídil David a Samuel vidoucí, pro jejich vernost,
\par 23 Aby oni i synové jejich byli pri branách domu Hospodinova, v dome stánku po strážích.
\par 24 Po ctyrech stranách byli vrátní, k východu, k západu, k pulnoci a ku poledni.
\par 25 A bratrí jejich, po vsech svých bydlící, aby každého sedmého dne casy svými pricházeli s nimi.
\par 26 Nebo Levítové byli pod spravou tech ctyr predních vrátných, a byli nad komorami a nad poklady domu Božího,
\par 27 A aby vukol domu Božího ponocovali; nebo jim porucena stráž, a otvírati každý den ráno.
\par 28 Z tech také ustaveni nad nádobami prisluhování; nebo v poctu vnášeli je, a v jistém poctu též vynášeli.
\par 29 Z nichž opet nekterí postaveni byli nad nádobami jinými a nade všemi nádobami posvátnými, a nad belí, vínem, olejem, kadidlem a vonnými vecmi.
\par 30 Nekterí také synové knežští strojili masti z tech vonných vecí.
\par 31 Matatiáš pak z Levítu, prvorozený Salluma Choritského, správcí byl tech vecí, kteréž se na pánvi smažily.
\par 32 A z synu Kahat, z bratrí jejich, byli ustanoveni nad chlebem predložení, aby jej pripravovali na každou sobotu.
\par 33 Z tech také byli zpeváci, prednejší z celedí otcovských mezi Levíty bydlíce, nejsouce zamestknaní necím jiným; nebo ve dne i v noci k své povinnosti státi musili.
\par 34 Ti jsou prední z celedí otcovských mezi Levíty v celedech svých, a ti prední jsouce, bydlili v Jeruzaléme.
\par 35 V Gabaon pak bydlili, otec Gabaonitských Jehiel, a jméno manželky jeho Maacha.
\par 36 Syn pak jeho prvorozený Abdon, Zur, Cis, Bál, Ner a Nádab.
\par 37 Ale Gedor, Achio, Zachariáš a Miklot,
\par 38 (Miklot pak zplodil Simam), ti také naproti bratrím svým bydlili v Jeruzaléme s bratrími svými.
\par 39 Ner pak zplodil Cisa, a Cis zplodil Saule, Saul pak zplodil Jonatu, Melchisua, Abinadaba a Ezbále.
\par 40 Syn pak Jonatuv Meribbál, a Meribbál zplodil Mícha.
\par 41 Synové pak Míchovi: Piton, Melech, Tarea.
\par 42 Achaz pak zplodil Járu, Jára pak zplodil Alemeta, Azmaveta a Zimri. Zimri pak zplodil Mozu.
\par 43 Moza pak zplodil Bina. Refaiáš syn jeho, Elasa syn jeho, Azel syn jeho.
\par 44 Azel pak mel šest synu, jichžto tato jsou jména: Azrikam, Bochru, Izmael, Seariáš, Abdiáš a Chanan. Ti jsou synové Azelovi.

\chapter{10}

\par 1 Když pak bojovali Filistinští s Izraelem, utíkali muži Izraelští pred Filistinskými, a padli, zbiti jsouce na hore Gelboe.
\par 2 I stihali Filistinští Saule a syny jeho, a zabili Filistinští Jonatu, Abinadaba a Melchisua, syny Saulovy.
\par 3 A když se zsilila bitva proti Saulovi, trefili na nej strelci s luky, a postrelen jest od strelcu.
\par 4 Protož rekl Saul odenci svému: Vytrhni mec svuj a probodni mne jím, aby prijdouce ti neobrezanci, necinili sobe ze mne posmechu. Ale nechtel odenec jeho, nebo se bál velmi. A pochytiv Saul mec, nalehl na nej.
\par 5 Tedy vida odenec jeho, že umrel Saul, nalehl i on na mec a umrel.
\par 6 A tak umrel Saul i tri synové jeho, a všecka celed jeho spolu zemreli.
\par 7 Když pak uzreli všickni muži Izraelští, kteríž bydlili za tím údolím, že utekli Izraelští, a že zemreli Saul i synové jeho, opustivše mesta svá, též utíkali. I prišli Filistinští a bydlili v nich.
\par 8 Potom nazejtrí prišli Filistinští, aby zloupili pobité. I nalezli Saule a syny jeho, ležící na hore Gelboe.
\par 9 A svlékše ho, vzali hlavu jeho i odení jeho, a poslali po zemi Filistinské vukol, aby to ohlášeno bylo modlám jejich i lidu.
\par 10 Složili také i odení jeho v chráme boha svého, hlavu pak jeho pribili v chráme Dágon.
\par 11 Tedy uslyšavše všickni v Jábes Galád všecko, což ucinili Filistinští Saulovi,
\par 12 Zdvihli se všickni muži silní a vzali telo Saulovo, a tela synu jeho, a prinesli je doJábes. A pochovavše kosti jejich pod jedním dubem v Jábes, postili se sedm dní.
\par 13 A tak umrel Saul pro prestoupení své, jímž byl prestoupil proti Hospodinu, totiž proti slovu Hospodinovu, jehož neostríhal, a že se také radil s duchem veštím, doptávaje se ho,
\par 14 A nedoptával se Hospodina. Protož zabil jej, a prenesl království na Davida syna Izai.

\chapter{11}

\par 1 Nebo shromáždivše se všickni Izraelští k Davidovi do Hebronu, rekli: Aj, my kost tvá a telo tvé jsme.
\par 2 Ano predešlých casu, když byl Saul králem, ty jsi vyvodil i privodil Izraele, a nadto rekl Hospodin Buh tvuj tobe: Ty pásti budeš lid muj Izraelský, a ty budeš vývoda nad lidem mým Izraelským.
\par 3 Prišli také všickni starší Izraelští k králi do Hebronu, a ucinil s nimi David smlouvu v Hebronu pred Hospodinem. I pomazali Davida za krále nad Izraelem, podlé slova Hospodinova skrze Samuele.
\par 4 Táhl pak David a všecken lid Izraelský k Jeruzalému, (jenž bylo Jebus, nebo tam byli Jebuzejští obyvatelé té zeme).
\par 5 I mluvili obyvatelé Jebus Davidovi: Nevejdeš sem. Ale David vzal hrad Sion, to jest mesto Davidovo.
\par 6 Nebo byl rekl David: Kdož by koli nejprvé porazil Jebuzea, bude predním a knížetem. Protož vstoupil nejprvé Joáb syn Sarvie, a ucinen predním.
\par 7 Potom bydlil David na tom hrade, procež nazvali jej mestem Davidovým.
\par 8 I vystavel mesto vukol a vukol, od Mello až do okolku, Joáb pak opravil ostatek mesta.
\par 9 A tak David cím dále tím více prospíval a rostl; nebo Hospodin zástupu byl s ním.
\par 10 Tito pak jsou prední z udatných, kteréž mel David, ježto se zmužile pricinili s ním o království jeho se vším Izraelem, aby ho za krále vyzdvihli podlé slova Hospodinova nad Izraelem.
\par 11 Tento jest pocet silných, kteréž mel David: Jasobam syn Chachmonuv, prední z vudcu. Ten pozdvihl kopí svého proti trem stum, a zbil je pojednou.
\par 12 Po nem též Eleazar syn Dodi Achochitského. On byl jeden z tech trí udatných.
\par 13 Ten byl s Davidem v Pasdammim, když se Filistinští sebrali k boji. A byl tu díl rolí poseté jecmenem, a lid byl utekl pred Filistinskými.
\par 14 I zastavili se u prostred toho dílu, a obdrželi jej, porazivše Filistinské. A vysvobodil Hospodin lid vysvobozením velikým.
\par 15 Ti také tri ze tridcíti predních sstoupili k skále k Davidovi do jeskyne Adulam, když vojsko Filistinských leželo v údolí Refaim.
\par 16 (Nebo David tehdáž bydlil v pevnosti své, a osazený lid Filistinských byl tehdáž u Betléma.)
\par 17 Pohnul se pak David žádostí, a rekl: Ó by mi nekdo dal píti vody z cisterny Betlémské, kteráž jest u brány.
\par 18 A protož probivše se ti tri skrze vojsko Filistinských, navážili vody z cisterny Betlémské, kteráž jest u brány, a nabravše, prinesli k Davidovi. David pak nechtel jí píti, ale vylil ji v obet Hospodinu,
\par 19 A rekl: Nedejž mi toho, Bože muj, abych to uciniti mel. Zdali krev mužu tech píti budu, kteríž se opovážili života svého? Nebo s opovážením života svého prinesli ji. I nechtel jí píti. To ucinili ti tri silní.
\par 20 Potom Abizai bratr Joábuv byl prední mezi trmi, a ten také pozdvihl kopí svého proti trem stum, i pobil je, a byl z tech trí nejslovoutnejší.
\par 21 Z tech trí nad druhé dva jsa nejvzácnejší, byl knížetem jejich, a však onem prvním nebyl rovný.
\par 22 Banaiáš také syn Joiaduv, syn muže udatného, velikých cinu, z Kabsael, ten zabil dva reky Moábské. Tentýž sstoupiv, zabil lva v jáme, když byl sníh.
\par 23 Ten také zabil muže Egyptského zvýší peti loket. A ackoli mel Egyptský ten v ruce kopí, jako vratidlo tkadlcovské, však šel k nemu s holí, a vytrh kopí z ruky Egyptského, zabil jej kopím jeho.
\par 24 To ucinil Banaiáš syn Joiaduv, kterýž také slovoutný byl mezi temi trmi silnými.
\par 25 A ac byl mezi tridcíti slavný, však onem trem se nevrovnal. I ustanovil ho David nad drabanty svými.
\par 26 Udatní rytíri také i tito: Azael bratr Joábuv, Elchanan syn Doduv Betlémský,
\par 27 Sammot Charodský, Chelez Pelonský,
\par 28 Híra syn Ikeš Tekoitský, Abiezer Anatotský,
\par 29 Sibbechai Chusatský, Ilai Achochský,
\par 30 Maharai Netofatský, Cheleb syn Baany Netofatský,
\par 31 Ittai syn Ribai z Gabaa synu Beniaminových, Banaiáš Faratonský,
\par 32 Churai od potoku Gás, Abiel Arbatský,
\par 33 Azmavet Bacharomský, Eliachba Salbonský,
\par 34 Synové Chasem Gizonského, Jonatan syn Sage Hararského,
\par 35 Achiam syn Sacharuv Hararský, Elifal syn Uruv,
\par 36 Hefer Mecheratský, Achiáš Pelonský,
\par 37 Chezro Karmelský, Narai syn Ezbai,
\par 38 Joel bratr Nátanuv, Mibchar syn Geri,
\par 39 Zelek Ammonský, Nacharai Berotský, odenec Joába syna Sarvie,
\par 40 Híra Itrejský, Gareb Itrejský,
\par 41 Uriáš Hetejský, Zabad syn Achlai,
\par 42 Adina syn Sizuv Rubenský, kníže nad Rubenskými, a s ním jiných tridceti,
\par 43 Chanan syn Maachuv, a Jozafat Mitnejský,
\par 44 Uziáš Asteratský, Sama a Johiel, synové Chotama Aroerského,
\par 45 Jediael, syn Simri, a Jocha bratr jeho Tizejský,
\par 46 Eliel Machavimský, a Jeribai a Josaviáš synové Elnámovi, a Itma Moábský,
\par 47 Eliel, a Obéd, a Jaasiel z Mezobaia.

\chapter{12}

\par 1 Tito pak jsou, kteríž prišli k Davidovi do Sicelechu, když se ješte kryl pred Saulem synem Cis, a ti byli mezi udatnými, pomocníci boje,
\par 2 Nosíce lucište, a bojujíce pravicí i levicí kamením i strelami z lucište, a byli z bratrí Saulových, z pokolení Beniaminova:
\par 3 Kníže Achiezer a Joas, synové Semmaa Gabatského, a Jeziel a Felet, synové Azmavetovi, Beracha, a Jéhu Anatotský,
\par 4 A Ismaiáš Gabaonitský, silný mezi tridcíti, kterýž byl nad tridcíti, Jeremiáš, Jachaziel, Jochanan a Jozabad Gederatský,
\par 5 A Eluzai, Jerimot, Bealiáš, Semariáš, a Sefatiáš Charufský,
\par 6 Elkána, Isiáš, Azareel, Joezer a Jasobam, Chorejští,
\par 7 A Joela a Zebadiáš synové Jerochamovi z Gedor.
\par 8 Též i z pokolení Gádova utekli k Davidovi k pevnosti na poušt, muži udatní, muži zpusobní k boji, užívajíce štítu a pavézy, jichžto tvári jako tvári lvové, a podobní srnám na horách v rychlosti:
\par 9 Ezer první, Abdiáš druhý, Eliab tretí,
\par 10 Mismanna ctvrtý, Jeremiáš pátý,
\par 11 Attai šestý, Eliel sedmý,
\par 12 Jochanan osmý, Elzabad devátý,
\par 13 Jeremiáš desátý, Machbannai jedenáctý.
\par 14 Ti byli z synu Gádových, knížata vojska, jeden nad stem menší, a vetší nad tisícem.
\par 15 Ti jsou, kteríž prepravili se pres Jordán mesíce prvního, ackoli se byl vylil ze všech brehu svých, a zahnali všecky z údolí na východ i na západ.
\par 16 Prišli také nekterí z synu Beniaminových a Judových k pevnosti Davidove.
\par 17 I vyšel jim David vstríc, a mluve, rekl jim: Jestliže z príciny pokoje jdete ke mne, abyste mi pomáhali, i mé srdce také s vámi se sjednotí; pakli k vyzrazení mne neprátelum mým, (ješto není nepravosti pri mne), popatriž Buh otcu našich a tresci.
\par 18 Duch pak posilnil Amazu predního mezi hejtmany, i rekl: Tobe, ó Davide, a tem, jenž s tebou jsou, synu Izai, pokoj; pokoj tobe, pokoj i pomocníkum tvým. Tobet zajisté pomáhá Buh tvuj. A tak prijal je David, a postavil je mezi knížaty houfu.
\par 19 Nadto i z pokolení Manassesova odstoupili k Davidovi, když táhl s Filistinskými k boji proti Saulovi, ale nepomáhali jim. Nebo uradivše se, propustili ho zase knížata Filistinská, rkouce: S nebezpecenstvím hrdel našich odstoupil by ku pánu svému Saulovi.
\par 20 Takž když táhl do Sicelechu, odstoupili k nemu nekterí z pokolení Manassesova:Adnach, Jozabad, Jediael, Michael, Jozabad, Elihu, a Zilletai, hejtmané nad tisíci v pokolení Manassesovu.
\par 21 Ti také pomáhali Davidovi s houfy jeho; nebo udatní byli všickni, procež byli knížaty v jeho vojšte.
\par 22 Anobrž každého dne pribývalo jich Davidovi ku pomoci, až bylo vojsko veliké, jako vojsko Boží.
\par 23 Tento pak jest pocet vývod zpusobných k boji, kteríž prišli k Davidovi do Hebronu, aby obrátili království Saulovo k nemu podlé slova Hospodinova.
\par 24 Z synu Judových, nosících pavézy a kopí, šest tisíc a osm set zpusobných k boji.
\par 25 Z synu Simeonových, udatných mužu k boji, sedm tisíc a sto.
\par 26 Z synu Léví ctyri tisíce a šest set.
\par 27 Joiada také vývoda synu Aronových, a s ním tri tisíce a sedm set.
\par 28 A Sádoch mládenec rek udatný, a z domu otce jeho knížat dvamecítma.
\par 29 A z synu Beniaminových, bratrí Saulových tri tisíce; nebo ješte množství jiných drželi stráž domu Saulova.
\par 30 Z synu též Efraimových dvadceti tisíc a osm set. Ti byli rekové udatní, muži slovoutní v celedech otcu svých.
\par 31 Z polovice pak pokolení Manassesova osmnáct tisíc, kteríž vycteni byli ze jména, aby prišli a ustanovili Davida za krále.
\par 32 Z synu také Izacharových, znajících a rozumejících casum, tak že vedeli, co by mel ciniti lid Izraelský, knížat jejich dve ste, a všickni bratrí jejich cinili podlé rcení jejich.
\par 33 Z synu Zabulon vycházejících na vojnu, vycvicených v bitve všelijakými nástroji válecnými, padesáte tisíc, a ku potýkání se v šiku bez choulostivosti srdce.
\par 34 Z Neftalímova pak pokolení knížat tisíc, a s nimi pavézníku a kopidlníku tridceti a sedm tisíc.
\par 35 Z pokolení Dan, zpusobných k boji, osm a dvadceti tisíc a šest set.
\par 36 A z pokolení Asserova, kteríž zpusobní byli k boji, a umelí v šikování se k bitve, ctyridceti tisíc.
\par 37 A z Zajordání, totiž z Rubenských a Gádských, a z polovice pokolení Manassesova,prišli se všemi nástroji válecnými sto a dvadceti tisíc.
\par 38 Všickni ti muži bojovní, umelí v šiku, úmyslem uprímým prišli do Hebronu, aby ustanovili Davida za krále nade vším lidem Izraelským. Nýbrž i všickni ostatní Izraelští srdce jednoho byli, aby za krále ustanovili Davida.
\par 39 I byli tu s Davidem tri dni, jedouce a pijíce, nebo jim byli pripravili bratrí jejich.
\par 40 Ano i ti, kteríž jim blízcí byli až k Izachar a Zabulon a Neftalím, prinášeli chleba na oslích a na velbloudích, i na mezcích a na volích, potravy, mouky, fíku a hroznu sušených, vína, oleje, volu, a ovcí v hojnosti. Nebo radost byla v lidu Izraelském.

\chapter{13}

\par 1 David pak poradil se s hejtmany nad tisíci, s setníky a se všemi vývodami.
\par 2 A rekl David všemu shromáždení Izraelskému: Jestliže se vám líbí, a jestli to od Hospodina Boha našeho, rozešleme posly k bratrím našim pozustalým do všech zemí Izraelských, a též knežím a Levítum do mest a predmestí jejich, a necht se shromáždí k nám,
\par 3 Abychom zase k nám privezli truhlu Boha našeho; nebo jsme jí nehledali ve dnech Saulových.
\par 4 I reklo všecko množství, aby se tak stalo; nebo líbila se ta vec všemu lidu.
\par 5 Protož shromáždil David všecken lid Izraelský od Nílu Egyptského, až kudy se vchází do Emat, aby privezli truhlu Hospodinovu z Kariatjeharim.
\par 6 A tak vstoupil David a všecken lid Izraelský do Bála, v Kariatjeharim, kteréž jest v Judstvu, aby prenesli odtud truhlu Boha Hospodina, sedícího nad cherubíny, jehož se jméno vzývá.
\par 7 I vstavili truhlu Boží na nový vuz, vzavše ji z domu Abinadabova, Uza pak a Achio spravovali vuz.
\par 8 Ale David a všecken lid Izraelský hrali pred Bohem ze vší síly, v zpevích na harfy, na loutny, na bubny, na cymbály a na trouby.
\par 9 A když prišli až k humnu Kídon, vztáhl Uza ruku svou, aby pozdržel truhly; nebo uchýlili se volové.
\par 10 Protož rozhneval se Hospodin na Uzu a zabil jej, proto že vztáhl ruku svou k truhle; a umrel tu pred Bohem.
\par 11 Tedy zkormoutil se David, proto že se Hospodin tak prísne oboril na Uzu. I nazval to místo Perez Uza až do tohoto dne.
\par 12 A boje se David Boha v ten den, rekl: Kterakž mám k sobe privezti truhlu Boží?
\par 13 Procež neprenesl David truhly k sobe do mesta Davidova, ale obrátil ji do domu Obededoma Gittejského.
\par 14 I pozustala truhla Boží mezi celedí Obededomovou, v dome jeho za tri mesíce, a požehnal Hospodin domu Obededomovu a všem vecem jeho.

\chapter{14}

\par 1 Potom poslal Chíram král Tyrský posly k Davidovi, a dríví cedrového a zedníky i tesare, aby staveli jemu dum.
\par 2 I poznal David, že ho potvrdil Hospodin za krále nad Izraelem, a že zvýšil království jeho pro lid svuj Izraelský.
\par 3 Pojal pak David ješte ženy v Jeruzaléme, a zplodil David více synu a dcer.
\par 4 A tato jsou jména tech, kteríž se jemu zrodili v Jeruzaléme: Sammua, Sobab, Nátan a Šalomoun,
\par 5 Též Ibchar, Elisua, Elfelet,
\par 6 Za tím Noga, Nefeg, Jafia,
\par 7 Elisama, Beeliada a Elifelet.
\par 8 V tom uslyšavše Filistinští, že by pomazán byl David za krále nade vším Izraelem, vytáhli všickni Filistinští hledati Davida. O cemž uslyšav David, vytáhl proti nim.
\par 9 Nebo když Filistinští prišli, a rozprostreli se v údolí Refaim,
\par 10 Radil se David s Bohem, rka: Potáhnu-li proti Filistinským, a dáš-li je v ruku mou? Odpovedel jemu Hospodin: Táhni, a dám je v ruku tvou.
\par 11 Tedy vtrhli do Balperazim. I porazil je tam David a rekl: Protrhlt jest Buh neprátely mé rukou mou, jako vody protrhují brehy. A protož nazváno jméno místa toho Balperazim.
\par 12 Nebo nechali tam bohu svých. I prikázal David, aby je spálili.
\par 13 Ale Filistinští sebravše se znovu, rozprostreli se v tom údolí.
\par 14 Procež David radil se opet s Bohem. I rekl jemu Buh: Nepristupuj k nim po zadu; odvrat se od nich, abys na ne trefil naproti moruším.
\par 15 A když uslyšíš, že šustí vrchové moruší, tedy vytáhneš k bitve; nebo vyšel Buh pred tebou, aby porazil vojska Filistinská.
\par 16 I ucinil David tak, jakž mu byl prikázal Buh, a porazili vojska Filistinská od Gabaon až do Gázera.
\par 17 A tak rozešla se povest o Davidovi do všech zemí, a zpusobil Hospodin to, že se ho báli všickni národové.

\chapter{15}

\par 1 Když pak sobe nastavel domu v meste Davidove, a pripravil místo pro truhlu Boží, a roztáhl jí stánek,
\par 2 Tehdy rekl David: Nemát nositi žádný truhly Boží krome Levítu, ty zajisté vyvolil Hospodin, aby nosili truhlu Boží, a aby prisluhovali jemu až na veky.
\par 3 Protož shromáždil David všecken lid Izraelský do Jeruzaléma, aby prenesl truhlu Hospodinovu na místo její, kteréž jí byl pripravil.
\par 4 Shromáždil také David syny Aronovy a Levíty.
\par 5 Z synu Kahat byli Uriel kníže, a bratrí jeho sto a dvadceti.
\par 6 Z synu Merari Asaiáš kníže, a bratrí jeho dve ste a dvadceti.
\par 7 Z synu Gersomových Joel kníže, a bratrí jeho sto a tridceti.
\par 8 Z synu Elizafanových Semaiáš kníže, a bratrí jeho dve ste.
\par 9 Z synu Hebronových Eliel kníže, a bratrí jeho osmdesát.
\par 10 Z synu Uzielových Aminadab kníže, a bratrí jeho sto a dvanáct.
\par 11 Tedy povolal David Sádocha a Abiatara, kneží, též i Levítu: Uriele, Asaiáše, Joele, Semaiáše, Eliele a Aminadaba,
\par 12 A rekl jim: Vy jste prední z otcovských celedí mezi Levíty, posvette sebe i bratrí svých, abyste vnesli truhlu Hospodina Boha Izraelského tu, kdež jsem jí pripravil.
\par 13 Nebo že spocátku ne vy jste spravovali toho, oboril se Hospodin Buh náš na nás; nebo jsme ho nehledali náležite.
\par 14 I posvetili se kneží i Levítové, aby prenesli truhlu Hospodina Boha Izraelského.
\par 15 A nesli synové Levítu truhlu Boží, jakož byl prikázal Mojžíš, podlé slova Hospodinova, na ramenou svých na sochorích.
\par 16 Rekl také David predním z Levítu, aby ustanovili z bratrí svých zpeváky s nástroji muzickými, loutnami, harfami a cymbály, aby zvuceli, povyšujíce hlasu s radostí.
\par 17 Takž ustanovili Levítové Hémana syna Joelova, a z bratrí jeho Azafa syna Berechiášova, a z synu Merari bratrí jejich Etana syna Kusaiova.
\par 18 A s nimi bratrí jejich z druhého porádku: Zachariáše, Béna, Jaaziele, Semiramota, Jechiele, Unni, Eliaba, Benaiáše, Maaseiáše, Mattitiáše, Elifele, Mikneiáše, Obededoma a Jehiele, vrátné.
\par 19 Nebo zpeváci Héman, Azaf a Etan hrali hlasite na cymbálích medených,
\par 20 A Zachariáš, Aziel, Semiramot, Jechiel, Unni, Eliab, Maaseiáš a Benaiáš na loutnách, pri zpevu vysokém.
\par 21 A Mattitiáš, Elifele, Mikneiáš, Obededom, Jehiel a Azaziáš hrali na harfách pri zpevu nízkém.
\par 22 Chenaniáš pak, prední z Levítu nesoucích truhlu, spravoval, jak by nésti meli; nebo byl umelý.
\par 23 Berechiáš pak a Elkána byli vrátní u truhly.
\par 24 Sebaniáš také a Jozafat, Natanael, Amazai, Zachariáš, Benaiáš a Eliezer kneží, troubili na trouby pred truhlou Boží; ale Obededom a Jechiáš byli též vrátní u truhly.
\par 25 A tak vypravil se David a starší Izraelští a hejtmané, aby prenesli truhlu smlouvy Hospodinovy z domu Obededomova s veselím.
\par 26 I stalo se, ponevadž Buh pomáhal Levítum nesoucím truhlu smlouvy Hospodinovy, že obetovali sedm volu a sedm beranu.
\par 27 David pak odín byl pláštem kmentovým, tolikéž všickni Levítové, kteríž nesli truhlu, i zpeváci, i Chenaniáš, správce nesoucích, mezi zpeváky. Mel také David na sobe efod lnený.
\par 28 Takž všecken lid Izraelský provázeli truhlu smlouvy Hospodinovy s plésáním a zvukem trouby, a pozaunu a cymbálu, a hrali na loutny a na harfy.
\par 29 Když pak truhla smlouvy Hospodinovy vcházela do mesta Davidova, Míkol dcera Saulova vyhlédla z okna, a viduci krále Davida poskakujícího a plésajícího, pohrdla jím v srdci svém.

\chapter{16}

\par 1 A když prinesli truhlu Boží a postavili ji u prostred stánku, kterýž jí byl rozbil David, tedy obetovali obeti zápalné a obeti pokojné pred Bohem.
\par 2 Zatím dokonav David obetování obetí zápalných a pokojných, dal požehnání lidu ve jménu Hospodinovu.
\par 3 Rozdelil také všechnem mužum Izraelským, od muže až do ženy, jednomu každému po pecnu chleba a kusu masa, a í láhvici.
\par 4 Potom postavil pred truhlou Hospodinovou služebníky z Levítu k pripomínání, k vyznávání a k chválení Hospodina Boha Izraelského.
\par 5 Azaf byl prední, a druhý po nem Zachariáš, Jehiel, Semiramot, Jechiel, Mattitiáš, Eliab, Benaiáš, Obededom a Jehiel. Ti na nástrojích, na loutnách a harfách, ale Azaf na cymbálích hral.
\par 6 Benaiáš pak a Jachaziel kneží s trubami byli ustavicne pred truhlou smlouvy Boží.
\par 7 Teprv toho dne ponejprvé narídil David, aby slaven byl Hospodin zpevem tímto od Azafa a bratrí jeho:
\par 8 Slavte Hospodina, zvestujte jméno jeho, a oznamujte mezi národy skutky jeho.
\par 9 Zpívejte a žalmy prozpevujte jemu, rozmlouvejte o všech divných skutcích jeho.
\par 10 Chlubte se v svatém jménu jeho, vesel se srdce tech, jenž hledají Hospodina.
\par 11 Hledejte Hospodina i síly jeho, hledejte tvári jeho ustavicne.
\par 12 Rozpomínejte se na divné skutky jeho, kteréž cinil, na zázraky jeho, i na soudy úst jeho.
\par 13 Ó síme Izraele, služebníka jeho, ó synové Jákobovi, vyvolení jeho,
\par 14 Ont jest Hospodin Buh náš, na vší zemi soudové jeho.
\par 15 Rozpomínejte se ustavicne na smlouvu jeho, na slovo, kteréž prikázal až do tisíce pokolení,
\par 16 Kterouž ucinil s Abrahamem, a na prísahu jeho Izákovi.
\par 17 A vystavil ji Jákobovi za ustanovení, Izraelovi za smlouvu vecnou,
\par 18 Prave: Tobe dám zemi Kananejskou za provazec vládarství vašeho,
\par 19 Ackoli vás byl malý pocet, a malicko byli jste v ní pohostinu.
\par 20 A precházeli od národu do národu, a z království k jinému lidu.
\par 21 Nedopustil žádnému ublížiti jim, ano i krále pro ne trestal, rka:
\par 22 Nedotýkejte se pomazaných mých, a prorokum mým necinte nic zlého.
\par 23 Zpívejte Hospodinu všecka zeme, zvestujte den po dni spasení jeho.
\par 24 Vypravujte mezi pohany slávu jeho, a mezi všemi národy divy jeho.
\par 25 Nebo veliký jest Hospodin, a chvalitebný náramne, hroznejší nade všecky bohy.
\par 26 Všickni zajisté bohové národu jsou modly, Hospodin pak nebesa ucinil.
\par 27 Sláva a jasnost pred ním, síla a veselé na míste jeho.
\par 28 Vzdejte Hospodinu celedi národu, vzdejte Hospodinu slávu i moc.
\par 29 Vzdejte Hospodinu cest jména jeho,prineste dary a pridte pred oblícej jeho,a sklánejte se pred Hospodinem v okrase svatosti.
\par 30 Bojte se oblíceje jeho všickni obyvatelé zeme, a budet upevnen okršlek zeme, aby se nepohnul.
\par 31 Veseliti se budou nebesa, a plésati bude zeme, a reknou mezi pohany: Hospodin kraluje.
\par 32 Zvuk vydá more, i což v nem jest, veseliti se bude pole i vše, což jest na nem.
\par 33 Tedy prozpevovati bude dríví lesní pred Hospodinem, nebot se bére, aby soudil zemi.
\par 34 Oslavujte Hospodina, neb dobrý jest, nebo na veky milosrdenství jeho.
\par 35 A rcete: Zachovej nás, Bože spasení našeho, a shromažd nás, a vytrhni nás z pohanu, abychom slavili svaté jméno tvé, a chlubili se v chvále tvé.
\par 36 Požehnaný Hospodin Buh Izraelský od veku a až na veky. I rekl všecken lid: Amen, i Halelujah.
\par 37 I nechal tu David pred truhlou smlouvy Hospodinovy Azafa a bratrí jeho, aby prisluhovali pred truhlou ustavicne podlé povinnosti dne každého.
\par 38 Též i Obededoma s bratrími jejich, osob šedesáte osm, Obededoma, pravím, syna Jedutunova, a Chosi, aby vrátní byli.
\par 39 Sádocha také kneze a bratrí jeho za kneží nechal pred príbytkem Hospodinovým na výsosti, kteráž byla v Gabaon,
\par 40 Aby obetovali zápaly Hospodinu na oltári zápalu ustavicne, ráno i vecer, podlé všeho, což psáno jest v zákone Hospodinove, jejž vydal Izraelovi.
\par 41 A s nimi nechal Hémana a Jedutuna a jiných vybraných, kteríž vycteni byli zejména, aby vzdávali chválu Hospodinu, proto že na veky trvá milosrdenství jeho.
\par 42 Tem také, totiž Hémanovi a Jedutunovi, nechal trub a cymbálu, aby zvuceli, i jiných nástroju muziky Boží, syny pak Jedutunovy postavil u vrat.
\par 43 A tak rozešel se všecken lid, jeden každý do domu svého; David též navrátil se, aby požehnání dal domu svému.

\chapter{17}

\par 1 I stalo se, když bydlil David v dome svém, že rekl Nátanovi proroku: Aj, já prebývám v dome cedrovém, truhla pak smlouvy Hospodinovy jest pod kortýnami.
\par 2 I rekl Nátan Davidovi: Cožkoli jest v srdci tvém, ucin; nebo Buh s tebou jest.
\par 3 Potom té noci stalo se slovo Boží k Nátanovi, rkoucí:
\par 4 Jdi a rci Davidovi služebníku mému: Toto dí Hospodin: Ne ty staveti mi budeš dum k bydlení,
\par 5 Ponevadž jsem nebydlil v žádném dome od toho dne, jakž jsem vyvedl syny Izraelské, až do dne tohoto, ale procházel jsem se z stánku do stánku, také i vne krome príbytku.
\par 6 Nadto kudyž jsem koli chodil se vším Izraelem, zdali jsem slovo rekl kterému z soudcu Izraelských, (jimž jsem prikázal, aby pásli lid muj), rka: Proc jste mi neustaveli domu cedrového?
\par 7 Protož nyní toto díš služebníku mému Davidovi: Takto praví Hospodin zástupu: Já jsem te vzal z ovcince, když jsi chodil za stádem, abys byl vývodou lidu mého Izraelského.
\par 8 A býval jsem s tebou všudy, kamž jsi koli se obrátil, všecky také neprátely tvé vyhladil jsem pred tvárí tvou, a ucinilt jsem jméno veliké, jako jméno vznešených na zemi.
\par 9 Ano i lidu svému Izraelskému zpusobil jsem místo, a vštípil jej tu. I bude bydliti na míste svém, a nepohne se více, aniž na nej dotírati budou lidé nešlechetní, jako prvé,
\par 10 Hned od toho casu, jakž jsem ustanovil soudce nad lidem svým Izraelským, až jsem ponížil všech neprátel tvých; nýbrž oznamujit, že Hospodin sám vystaví tobe dum.
\par 11 Nebo když se vyplní dnové tvoji, abys šel za otci svými, vzbudím síme tvé po tobe, kteréž bude z synu tvých, a utvrdím království jeho.
\par 12 Ont mi ustaví dum, a já utvrdím trun jeho až na veky.
\par 13 Já budu jemu otcem, a on mi bude synem, a milosrdenství svého neodejmu od neho, jako jsem je odjal od toho, kterýž byl pred tebou.
\par 14 Ale postavím jej v dome svém a v království svém až na veky, a trun jeho bude nepohnutelný až na veky.
\par 15 Podlé všech slov techto, a podlé všeho videní tohoto, tak mluvil Nátan Davidovi.
\par 16 Tedy všed král David, posadil se pred Hospodinem, a rekl: Kdož jsem já, ó Hospodine Bože, a jaký jest dum muj, že jsi mne tak zvýšil?
\par 17 Anobrž i to jsi za málo u sebe položil, ó Bože, procež jsi zamluvil se o domu služebníka svého i na dlouhé casy, a popatril jsi na mne, jako na osobu cloveka vzácného, Hospodine Bože.
\par 18 Což ješte více mluviti má David pred tebou o zvelebení služebníka tvého? Ty zajisté znáš služebníka svého.
\par 19 Hospodine, pro služebníka svého a podlé srdce svého ciníš velikou vec tuto, abys v známost uvedl všecky preveliké veci.
\par 20 Hospodine, nenít tobe rovného, anobrž není žádného Boha krome tebe, podlé toho všeho, jakž jsme slýchali ušima svýma.
\par 21 Nebo kde jest který národ na zemi, jako lid tvuj Izraelský, jehož by Buh šel, aby vykoupil sobe lid a dobyl sobe jména, cine veliké a hrozné veci, vyháneje pred tvárí lidu svého, kterýž jsi vykoupil z Egypta, pohany?
\par 22 Zvolil jsi zajisté lid svuj Izraelský sobe za lid až na veky, a ty, Hospodine, sám jsi jejich Bohem.
\par 23 Nyní tedy, Hospodine, slovo to, jímž jsi zamluvil se služebníku svému a domu jeho, budiž jisté až na veky, a ucin tak, jakž jsi mluvil.
\par 24 Budiž, pravím, jisté, tak aby velebeno bylo jméno tvé až na veky, a ríkáno: Hospodin zástupu, Buh Izraelský, jest Buh nad Izraelem, a dum Davida služebníka tvého at jest nepohnutelný pred oblícejem tvým.
\par 25 Nebo ty, Bože muj, zjevil jsi služebníku svému, že mu ustavíš dum, a protož smel služebník tvuj modliti se pred tebou.
\par 26 A tak, ó Hospodine, ty jsi sám Buh, a mluvil jsi o služebníku svém dobré veci tyto.
\par 27 Nyní tedy rácil jsi požehnati domu služebníka svého, aby trval na veky pred oblícejem tvým; nebo jsi ty, Hospodine, požehnal, i budet požehnaný na veky.

\chapter{18}

\par 1 Stalo se potom, že porazil David Filistinské, a zemdlil je, a vzal Gát i vesnice jeho z ruky Filistinských.
\par 2 Porazil také i Moábské, a ucineni jsou Moábští služebníci Davidovi, a dávali jemu plat.
\par 3 Porazil též David Hadarezera krále Soba v Emat, když byl vytáhl, aby opanoval reku Eufrates.
\par 4 A pobral mu David tisíc vozu, a sedm tisíc jízdných, a dvadceti tisíc mužu peších, a zpodrezoval David žily všechnem konum vozníkum. Toliko zanechal z nich ke stu vozum.
\par 5 Pritáhli pak byli Syrští od Damašku na pomoc Hadarezerovi králi Soba, ale David porazil z Syrských dvamecítma tisíc mužu.
\par 6 Tedy osadil David Syrii Damašskou. I ucineni jsou Syrští služebníci Davidovi, dávajíce plat; nebo zachovával Hospodin Davida, kamž se koli obrátil.
\par 7 Pobral také David štíty zlaté, kteréž meli služebníci Hadarezerovi, a prinesl je do Jeruzaléma.
\par 8 Z Tibchat též a z Chun, mest Hadarezerových, nabral David medi velmi mnoho, z kteréž potom slil Šalomoun more medené, a sloupy i nádobí medené.
\par 9 A když uslyšel Tohu král Emat, že porazil David všecko vojsko Hadarezera krále Soba,
\par 10 Poslal Adorama syna svého k králi Davidovi, aby ho pozdravil prátelsky, a spolu se s ním radoval z toho, že bojoval s Hadarezerem, a porazil ho; (nebo válcil Tohu s Hadarezerem). Kterýžto prinesl všelijaké nádoby zlaté a stríbrné i medené.
\par 11 Ty také obetoval král David Hospodinu s stríbrem a zlatem, kteréhož byl nabral ze všech národu, z Idumejských, z Moábských, z synu Ammon, z Filistinských, i z Amalechitských.
\par 12 Abizai také syn Sarvie porazil Idumejských v údolí solnatém osmnácte tisíc.
\par 13 Protož i nad Idumejskými postavil stráž, a ucineni jsou všickni Idumejští služebníci Davidovi; nebo zachovával Hospodin Davida, kamž se koli obrátil.
\par 14 A tak kraloval David nade vším Izraelem, a cinil soud a spravedlnost všemu lidu svému.
\par 15 Joáb pak syn Sarvie byl nad vojskem, a Jozafat syn Achiluduv byl kanclérem.
\par 16 Sádoch také syn Achitobuv a Abimelech syn Abiataruv byli knežími, a Susa byl písarem.
\par 17 Banaiáš pak syn Joiaduv byl nad Cheretejskými a Peletejskými, a synové Davidovi knížaty pri králi.

\chapter{19}

\par 1 Stalo se zatím, že umrel Náhas král Ammonitský, a kraloval syn jeho místo neho.
\par 2 I rekl David: Uciním milosrdenství s Chanunem synem Náhasovým, nebo ucinil otec jeho milosrdenství nade mnou. Tedy poslal David posly, aby ho potešili pro otce jeho. I prišli služebníci Davidovi do zeme synu Ammon k Chanunovi, aby ho tešili.
\par 3 Tedy rekla knížata Ammonitská Chanunovi: Cožt se zdá, že David ciní poctivost otci tvému, že poslal k tobe, kteríž by te potešili? Zdali ne proto, aby shlédli a vyšpehovali, i podvrátili zemi tuto, prišli služebníci jeho k tobe?
\par 4 Procež Chanun vzav služebníky Davidovy, oholil je a zustrihoval roucha jejich polovici až do rozkroku, a propustil je.
\par 5 Tedy odešli nekterí, a oznámili Davidovi o tech mužích. I poslal proti nim, (nebo byli muži ti zlehceni velice), a rozkázal jim král: Pobudte v Jerichu, dokudž neobrostou brady vaše; potom se navrátíte.
\par 6 Vidouce pak Ammonitští, že se zošklivili Davidovi, poslal Chanun a synové Ammon tisíc hriven stríbra, aby sobe najali ze mzdy z Mezopotamie a z Syrie Maacha a z Soba vozu a jezdcu.
\par 7 A najali sobe ze mzdy tridceti a dva tisíce vozu, a krále Maachu i lid jeho. Kterížto pritáhše, položili se s vojskem naproti Medaba. Ammonitští také shromáždivše se z mest svých, pritáhli k té bitve.
\par 8 Což uslyšav David, poslal Joába se vším vojskem udatných.
\par 9 A tak vytáhše Ammonitští, sšikovali se k boji u brány mesta toho. Králové pak, kteríž byli pritáhli, byli zvlášte na poli.
\par 10 A protož vida Joáb proti sobe sšikovaný lid z predu i z zadu, vybrav nekteré ze všech výborných Izraelských, sšikoval je také proti Syrským.
\par 11 Ostatek pak lidu dal pod správu Abizaie bratra svého, a sšikovali se proti Ammonitským.
\par 12 I rekl Joáb: Jestliže Syrští budou mne silnejší, prispeješ mi na pomoc; pakli Ammonitští silnejší budou tebe, já pomohu tobe.
\par 13 Posiliž se, a zmužile se mejme, bojujíce za lid náš a za mesta Boha našeho, Hospodin však, což mu se dobre líbí, necht uciní.
\par 14 Takž pristoupil Joáb i lid, kterýž pri nem byl, k bitve proti Syrským. Ale oni utekli pred ním.
\par 15 V tom Ammonitští vidouce, že utíkají Syrští, utekli také i oni pred Abizaiem bratrem jeho, a ušli do mesta. Joáb též navrátil se do Jeruzaléma.
\par 16 A tak vidouce Syrští, že jsou poraženi od Izraelských, poslali posly, a vyvedli Syrské, kteríž bydlejí za rekou, a Sofach kníže vojska Hadarezerova vedl je.
\par 17 I oznámeno to Davidovi. Kterýžto shromáždiv všecken lid Izraelský, prepravil se pres Jordán, a pritáhl k nim, a sšikoval vojsko proti nim. A když sšikoval David vojsko proti Syrským k bitve, bojovali s ním.
\par 18 Tedy utekli Syrští pred Izraelem. I porazil David z Syrských sedm tisíc vozu, a ctyridceti tisíc lidu pešího, až i Sofacha hejtmana vojska toho zabil.
\par 19 Procež když videli služebníci Hadarezerovi, že jsou poraženi od Izraele, vešli v pokoj s Davidem, a sloužili jemu. A nechteli více Syrští táhnouti na pomoc Ammonitským.

\chapter{20}

\par 1 I stalo se po roce, když králové vyjíždívají na vojnu, že vedl Joáb lid válecný, a hubil zemi synu Ammon. A když pritáhl, oblehl Rabbu; (ale David zustal v Jeruzaléme). I dobyl Joáb Rabby, a rozboril ji.
\par 2 Tedy snal David korunu krále jejich s hlavy jeho, a našel, že vážila hrivnu zlata. A bylo v ní kamení drahé, i vstavena byla na hlavu Davidovu. Vyvezl též koristi mesta velmi veliké.
\par 3 Lid pak, kterýž v nem byl, vyvedl, a dal je pod pily a brány železné i sekery. A tak ucinil David všechnem mestum Ammonitským. I navrátil se David se vším lidem do Jeruzaléma.
\par 4 Potom pak když trvala válka v Gázer s Filistinskými, tehdy zabil Sibbechai Chusatský Sippaje, kterýž byl zplozený z obru, i sníženi jsou.
\par 5 Byla ješte i jiná válka s Filistinskými, kdežto zabil Elchanan, syn Jairuv, Lachmi bratra Goliáše Gittejského, u jehož kopí bylo drevo jako vratidlo tkadlcovské.
\par 6 Opet byla jiná válka v Gát, a byl tam muž veliké postavy, kterýž mel po šesti prstech, všech ctyrmecítma, a byl i on zplozený z téhož obra.
\par 7 Ten když hanel Izraele, zabil ho Jonata syn Semmaa, bratra Davidova.
\par 8 Ti byli synové jednoho obra v Gát, kteríž padli od ruky Davidovy a od ruky služebníku jeho.

\chapter{21}

\par 1 Satan pak povstal proti Izraelovi, a ponukl Davida, aby sectl lid Izraelský.
\par 2 Protož rekl David Joábovi a knížatum lidu: Jdete, sectete lid Izraelský od Bersabé až do Dan, a oznamte mi, abych vedel pocet jich.
\par 3 Ale Joáb rekl: Pridejž Hospodin lidu svého, což ho koli, stokrát více. I zdaliž, pane muj králi, nejsou všickni oni pána mého služebníci? Proc toho vyhledává pán muj? Proc má býti uvedena pokléska na Izraele?
\par 4 Ale rec královská premohla Joába. A tak vyšed Joáb, prošel všecken lid Izraelský; potom navrátil se do Jeruzaléma.
\par 5 I dal Joáb pocet lidu secteného Davidovi. A bylo všeho lidu Izraelského jedenáctekrát sto tisíc mužu bojovných, lidu pak Judského ctyrikrát sto tisíc, a sedmdesáte tisíc mužu bojovných.
\par 6 Pokolení pak Léví a Beniaminova nepocítal mezi ne; nebo v ošklivosti mel Joáb rozkázaní královo.
\par 7 Ovšem nelíbila se Bohu ta vec, protož ranil Izraele.
\par 8 I rekl David Bohu: Zhrešil jsem težce, že jsem to ucinil, ale nyní odejmi, prosím, nepravost služebníka svého, nebo jsem velmi bláznive ucinil.
\par 9 V tom mluvil Hospodin k Gádovi, proroku Davidovu, rka:
\par 10 Jdi a rci Davidovi: Toto praví Hospodin: Trojít veci podávám, vyvol sobe jednu z nich, kteroužt bych ucinil.
\par 11 Tedy prišel Gád k Davidovi a rekl jemu: Toto praví Hospodin: Vol sobe,
\par 12 Budto hlad za tri léta, bud abys za tri mesíce kažen byl od protivníku svých, když by mec neprátel tvých dotekl se tebe, aneb aby za tri dni mec Hospodinuv a mor byl v zemi, a andel Hospodinuv aby hubil po všech koncinách Izraelských. Již tedy viz, co mám odpovedíti tomu, kterýž mne poslal.
\par 13 I rekl David Gádovi: Úzko mi náramne; necht prosím, upadnu v ruce Hospodinovy, nebot jsou mnohá slitování jeho, jediné at v ruce lidské neupadám.
\par 14 A tak uvedl Hospodin mor na lid Izraelský, a padlo jich z Izraele sedmdesáte tisíc mužu.
\par 15 Poslal také Buh andela i na Jeruzalém, aby hubil jej. A když hubil, popatril Hospodin, a zželelo mu se toho zlého. I rekl andelu, kterýž hubil: Dostit jest, zdrž ruku svou. Andel pak stál podlé humna Ornana Jebuzejského.
\par 16 Mezi tím pozdvih David ocí svých, uzrel andela Hospodinova, stojícího mezi zemí a nebem, a mec dobytý v ruce jeho, vztažený proti Jeruzalému. I padl David, ano i starší, odíni jsouce žínemi, na tvári své.
\par 17 A rekl David Bohu: Zdaliž jsem já nerozkázal císti lidu? A ját jsem sám ten, kterýž jsem zhrešil, a prevelmi jsem zle ucinil, tyto pak ovce co ucinily? Hospodine, Bože muj, necht jest, prosím, ruka tvá proti mne, a proti domu otce mého, ale proti lidu tvému necht není rána.
\par 18 Zatím andel Hospodinuv mluvil k Gádovi, aby rekl Davidovi, aby vstoupe, vzdelal oltár Hospodinu na humne Ornana Jebuzejského.
\par 19 I vstoupil David podlé reci Gádovy, kterouž byl mluvil ve jménu Hospodinovu.
\par 20 A obrátiv se Ornan, uzrel toho andela, a ctyri synové jeho s ním skryli se. Ornan pak mlátil pšenici.
\par 21 V tom prišel David k Ornanovi. A pohledev Ornan, uzrel Davida, a vyšed z humna toho, klanel se Davidovi až k zemi.
\par 22 Tedy rekl David Ornanovi: Dej mi to místo humna, at vzdelám na nem oltár Hospodinu; za slušné peníze dej mi je, i prestane rána v lidu.
\par 23 I rekl Ornan Davidovi: Necht sobe vezme, a uciní pán muj král, což se mu za dobré vidí. Hle, pridám i voly tyto k obeti zápalné, a smyky na drva, i pšenici k obeti suché, to všecko dávám.
\par 24 Král pak David rekl Ornanovi: Nikoli, ale radeji koupím je od tebe za slušné peníze; nebot nevezmu, což tvého jest, Hospodinu, aniž budu obetovati obeti zápalné darem dané.
\par 25 I dal David Ornanovi za to místo zlata ztíží šesti set lotu.
\par 26 A vzdelal tu David oltár Hospodinu, a obetoval zápaly a obeti pokojné, a vzýval Hospodina. Kterýžto vyslyšel ho, spustiv ohen s nebe na oltár zápalu.
\par 27 I rekl Hospodin andelu, aby obrátil mec svuj do pošvy jeho.
\par 28 Toho casu když uzrel David, že jej vyslyšel Hospodin na humne Ornana Jebuzejského, obetovával tu obeti.
\par 29 Nebo stánek Hospodinuv, kterýž byl ucinil Mojžíš na poušti, a oltár k zápalu toho casu byl na výsosti v Gabaon.
\par 30 David pak nemohl tam choditi k nemu, aby hledal Boha, proto že se zhrozil mece andela Hospodinova.

\chapter{22}

\par 1 I rekl David: Totot jest místo domu Hospodina Boha, a toto jest místo oltári k zápalu Izraelovi.
\par 2 Protož prikázal David, aby shromáždili cizozemce prebývající v zemi Izraelské, a ustanovil z nich kameníky, aby tesali kamení k stavení domu Božího.
\par 3 Železa také mnoho na hreby, a na dvére k branám i k spojováním, pripravil David, i medi mnoho bez váhy.
\par 4 Též i dríví cedrového bez poctu; nebo priváželi Sidonští a Tyrští dríví cedrového množství Davidovi.
\par 5 Nebo rekl byl David: Šalomoun syn muj mládencek jest malý, dum pak vystaven býti má Hospodinu veliký, znamenitý a slovoutný po všech zemích, a protož pripravím mu nyní potreb. A tak pripravil David množství toho pred smrtí svou.
\par 6 Potom povolav Šalomouna syna svého, prikázal mu, aby vystavel dum Hospodinu Bohu Izraelskému.
\par 7 A rekl David Šalomounovi: Synu muj, uložil jsem byl v srdci svém vystaveti dum jménu Hospodina Boha svého.
\par 8 Ale stala se ke mne rec Hospodinova, rkoucí: Mnohou jsi krev vylil, a boje veliké jsi vedl; nebudeš staveti domu jménu mému, proto že jsi mnoho krve vylil na zem prede mnou.
\par 9 Aj, syn narodí se tobe, tent bude muž pokojný. Odpocinutí zajisté dám jemu vukol prede všemi neprátely jeho, procež Šalomoun slouti bude; nebo pokoj a odpocinutí dám Izraelovi za dnu jeho.
\par 10 Ont ustaví dum jménu mému, a on bude mi za syna, a já jemu za otce, a upevním trun království jeho nad Izraelem až na veky.
\par 11 Protož, synu muj, Hospodin bude s tebou, a štastnet se povede, a vystavíš dum Hospodina Boha svého, jakož mluvil o tobe.
\par 12 A však dejž tobe Hospodin rozum a moudrost, a ustanoviž te nad Izraelem, abys ostríhal zákona Hospodina Boha svého.
\par 13 A tehdyt se štastne povede, když ostríhati a ciniti budeš ustanovení a soudy, kteréž prikázal Hospodin skrze Mojžíše lidu Izraelskému. Posilniž se a zmocni, neboj se, ani lekej.
\par 14 A aj, já v nevolech svých pripravil jsem k domu Hospodinovu sto tisícu centnéru zlata, a stríbra tisíc tisícu centnéru, medi pak a železa bez váhy; nebo toho mnoho jest. Dríví také i kamení pripravil jsem, a k tomu ostatek pridáš.
\par 15 Pres to máš u sebe hojne delníku, kameníku a zedníku, i tesaru i jiných zbehlých ve všelijakém díle.
\par 16 Zlata, stríbra, medi a železa není poctu; snažiž se a delej, a Hospodin budiž s tebou.
\par 17 Prikázal také David všechnem knížatum Izraelským, aby pomáhali Šalomounovi synu jeho, rka:
\par 18 Zdaliž Hospodin Buh váš není s vámi, kterýž vám zpusobil vukol a vukol odpocinutí? Nebo dal v ruku mou obyvatele zeme této, a podmanena jest zeme Hospodinu a lidu jeho.
\par 19 Nyní tedy vydejte se srdcem svým a duší svou k hledání Hospodina Boha svého, a priciníce se, vystavejte svatyni Hospodinu Bohu, abyste tam vnesli truhlu smlouvy Hospodinovy, a nádobí Bohu posvecená do domu vystaveného jménu Hospodinovu.

\chapter{23}

\par 1 Sstarav se pak David, a jsa pln dnu, ustanovil králem Šalomouna syna svého nad Izraelem,
\par 2 A shromáždil všecka knížata Izraelská, i kneží a Levíty.
\par 3 I secteni jsou Levítové od tridcítiletých a výše, a byl pocet jich, vedlé jmen a osob jejich, tridceti a osm tisícu.
\par 4 Z kterýchž postaveno bylo nad dílem domu Hospodinova ctyrmecítma tisícu, vladaru pak a soudcu šest tisícu,
\par 5 A vrátných ctyri tisíce, a ctyri tisíce chválících Hospodina na nástrojích, kterýchž nadelal k chválení Boha.
\par 6 I narídil David porádku mezi syny Léví, totiž mezi syny Gerson, Kahat a Merari.
\par 7 Z Gersona byli Ladan a Semei.
\par 8 Synové Ladan: Kníže Jehiel, Zetam a Joel, ti tri.
\par 9 Synové Semei: Selomit, Oziel a Háran, ti tri. Ta jsou knížata otcovských celedí Ladanských.
\par 10 Synové pak Semei: Jachat, Zina, Jehus a Beria. Ti ctyri jsou synové Semei.
\par 11 Byl pak Jachat kníže, a Ziza druhý, ale Jehus a Beria ne mnoho meli synu, a protož v celedi otcovské za jedny byli pocítáni.
\par 12 Synové Kahat: Amram, Izar, Hebron a Uziel, ti ctyri.
\par 13 Synové Amramovi: Aron a Mojžíš. Byl pak oddelen Aron, aby sloužil v svatyni svatých, on i synové jeho na veky, a aby kadili pred Hospodinem, a sloužili jemu, i dobrorecili ve jménu jeho až na veky.
\par 14 Ale synové Mojžíše, muže Božího, pocteni jsou v pokolení Léví.
\par 15 Synové Mojžíšovi: Gersom a Eliezer.
\par 16 Synové Gersomovi: Sebuel kníže.
\par 17 A synové Eliezerovi: Rechabiáš kníže. Nemel pak Eliezer více synu, ale synové Rechabiášovi rozmnožili se velmi.
\par 18 Synové Izarovi: Selomit kníže.
\par 19 Synové Hebronovi: Jeriáš kníže, Amariáš druhé, Jachaziel tretí, a Jekmaam ctvrté.
\par 20 Synové Uzielovi: Mícha kníže, a Jezia druhý.
\par 21 Synové Merari: Moholi a Musi. Synové Moholi: Eleazar a Cis.
\par 22 Umrel pak Eleazar, a nemel synu, než toliko dcery, kteréž pojali synové Cis, bratrí jejich.
\par 23 Synové Musi: Moholi a Eder a Jerimot, ti tri.
\par 24 Ti jsou synové Léví po celedech svých, knížata celedí, kteríž vycteni byli vedlé poctu jmen a osob, konající dílo v prisluhování domu Hospodinova od dvadcítiletých a výše.
\par 25 Nebo rekl David: Odpocinutí dal Hospodin Buh Izraelský lidu svému, a bydliti bude v Jeruzaléme až na veky.
\par 26 Ano i Levítové nebudou více nositi stánku, ani kterých nádob jeho k prisluhování jeho.
\par 27 A tak podlé narízení Davidova nejposlednejšího byvše secteni synové Léví, pocna od dvadcítiletých a výše,
\par 28 Postaveni byli, aby byli ku pomoci synu Aronových, v prisluhování domu Hospodinova v síncích, v pokojích a pri ocištování všeliké veci svaté, i pri práci služebnosti domu Božího,
\par 29 Též pri chlebích posvátných, a pri beli k obetem suchým, pri kolácích presných, a pri pánvicích a rendlících, i pri všeliké míre a merení,
\par 30 A aby stáli každého jitra k slavení a chválení Hospodina, tolikéž i u vecer,
\par 31 A pri všeliké obeti zápalu Hospodinových ve dny sobotní, a na novmesíce a v svátky výrocní v jistém poctu, vedlé rádu jejich ustavicne pred Hospodinem,
\par 32 A tak aby drželi stráž stánku úmluvy, a stráž svatyne, i stráž synu Aronových, bratrí svých v službe domu Hospodinova.

\chapter{24}

\par 1 Synu pak Aronových tato jsou zporádaní: Synové Aronovi: Nádab, Abiu, Eleazar a Itamar.
\par 2 Ale že umrel Nádab a Abiu pred otcem svým, a nemeli synu, protož konali úrad knežský Eleazar a Itamar.
\par 3 Kteréž zporádal David, totiž Sádocha z synu Eleazarových, a Achimelecha z synu Itamarových, vedlé poctu a rádu jejich v prisluhováních jejich.
\par 4 Nalezeno pak synu Eleazarových více predních mužu, než synu Itamarových, a rozdelili je. Z synu Eleazarových bylo predních po domích otcovských šestnáct, a z synu Itamarových po celedech otcovských osm.
\par 5 I rozdeleni jsou losem jedni od druhých, ackoli byli knížata nad vecmi svatými, a knížata Boží z synu Eleazarových a z synu Itamarových.
\par 6 I popsal je Semaiáš syn Natanaeluv, písar z pokolení Léví pred králem a knížaty, a Sádochem knezem a Achimelechem synem Abiatarovým i knížaty celedí otcovských mezi knežími a Levíty, tak že dum otcovský jeden zaznamenán Eleazarovi, tolikéž druhý zaznamenán Itamarovi.
\par 7 Padl pak los první na Jehoiariba, na Jedaiáše druhý,
\par 8 Na Charima tretí, na Seorima ctvrtý,
\par 9 Na Malkiáše pátý, na Miamin šestý,
\par 10 Na Hakkoza sedmý, na Abiáše osmý,
\par 11 Na Jesua devátý, na Sechaniáše desátý,
\par 12 Na Eliasiba jedenáctý, na Jakima dvanáctý,
\par 13 Na Chuppa trináctý, na Jesebaba ctrnáctý,
\par 14 Na Bilgu patnáctý, na Immera šestnáctý,
\par 15 Na Chezira sedmnáctý, na Happizeza osmnáctý,
\par 16 Na Petachiáše devatenáctý, na Ezechiele dvadcátý,
\par 17 Na Jachina jedenmecítmý, na Gamule dvamecítmý,
\par 18 Na Delaiáše trímecítmý, na Maaseiáše ctyrmecítmý.
\par 19 Ti jsou, jenž zrízeni byli v prisluhováních svých, aby vcházeli do domu Hospodinova podlé rádu svého, pod spravou Arona otce jejich, jakož mu byl prikázal Hospodin Buh Izraelský.
\par 20 Z synu Léví ostatních, z synu Amramových Subael, z synu Subael Jechdeiáš.
\par 21 Z Rechabiáše, z synu Rechabiášových kníže Iziáš.
\par 22 Z Izara Selomot, z synu Selomotových Jachat.
\par 23 Synové pak Jeriášovi: Amariáš druhý, Jachaziel tretí, Jekamam ctvrtý.
\par 24 Syn Uzieluv Mícha, z synu Míchy Samir.
\par 25 Bratr Míchuv Iziáš, a syn Iziášuv Zachariáš.
\par 26 Synové Merari: Moholi a Musi, synové Jaaziášovi: Beno.
\par 27 Synové Merari z Jaaziáše: Beno, Soham, Zakur a Ibri.
\par 28 Z Moholi Eleazar, kterýž nemel synu.
\par 29 Z Cisa synové Cisovi: Jerachmeel.
\par 30 Synové pak Musi: Moholi, Eder a Jerimot. Ti jsou synové Levítu po domích otcu svých.
\par 31 I ti také metali losy naproti bratrím svým, synum Aronovým, pred Davidem králem, Sádochem a Achimelechem, i knížaty otcovských celedí z kneží a Levítu, z celedí otcovských, každý prednejší naproti bratru svému mladšímu.

\chapter{25}

\par 1 I oddelil David a knížata vojska k službe syny Azafovy a Hémanovy a Jedutunovy, kteríž by prorokovali pri harfách, pri loutnách a pri cimbálích. Byl pak pocet jejich, totiž mužu tech, jenž práci vedli v prisluhování svém,
\par 2 Z synu Azafových: Zakur, Jozef, Netaniáš a Asarela, synové Azafovi, pod spravou Azafavou byli, kterýž prorokoval k rozkazu královu.
\par 3 Z Jedutuna synu Jedutunových bylo šest: Godoliáš, Zeri, Izaiáš, Chasabiáš, Mattitiáš a Simei, pod spravou otce jejich Jedutuna, kterýž prorokoval pri harfe k slavení a chválení Hospodina.
\par 4 Z Hémana synové Hémanovi: Bukkiáš, Mataniáš, Uziel, Sebuel, Jerimot, Chananiáš, Chanani, Eliata, Giddalti, Romantiezer, Jazbekasa, Malloti, Hotir a Machaziot.
\par 5 Všickni ti synové Hémanovi, proroka králova v slovích Božských, k vyvyšování moci. A dal Buh Hémanovi synu ctrnácte a dcery tri.
\par 6 Všickni ti byli pod spravou otce svého, pri zpívání v dome Hospodinove na cymbálích, loutnách a harfách, k službe v dome Božím vedlé porucení králova Azafovi, Jedutunovi a Hémanovi.
\par 7 Byl pak pocet jich s bratrími jejich, temi, kteríž byli vycvicení v zpevích Hospodinových, všech mistru dve ste osmdesát osm.
\par 8 Tedy metali losy, houf držících stráž naproti druhému, jakž malý, tak veliký, mistr i ucedlník.
\par 9 I padl první los v celedi Azaf na Jozefa, na Godoliáše s bratrími a syny jeho druhý, jichž bylo dvanáct.
\par 10 Tretí na Zakura, synum jeho a bratrím jeho dvanácti.
\par 11 Ctvrtý na Izara, synum jeho a bratrím jeho dvanácti.
\par 12 Pátý na Netaniáše, synum jeho a bratrím jeho dvanácti.
\par 13 Šestý na Bukkiáše, synum jeho a bratrím jeho dvanácti.
\par 14 Sedmý na Jesarele, synum jeho a bratrím jeho dvanácti.
\par 15 Osmý na Izaiáše, synum jeho a bratrím jeho dvanácti.
\par 16 Devátý na Mataniáše, synum a bratrím jeho dvanácti.
\par 17 Desátý na Simei, synum a bratrím jeho dvanácti.
\par 18 Jedenáctý na Azarele, synum a bratrím jeho dvanácti.
\par 19 Dvanáctý na Chasabiáše, synum a bratrím jeho dvanácti.
\par 20 Trináctý na Subaele, synum a bratrím jeho dvanácti.
\par 21 Ctrnáctý na Mattitiáše, synum a bratrím jeho dvanácti.
\par 22 Patnáctý na Jerimota, synum a bratrím jeho dvanácti.
\par 23 Šestnáctý na Chananiáše, synum a bratrím jeho dvanácti.
\par 24 Sedmnáctý na Jazbekasa, synum a bratrím jeho dvanácti.
\par 25 Osmnáctý na Chanani, synum a bratrím jeho dvanácti.
\par 26 Devatenáctý na Malloti, synum a bratrím jeho dvanácti.
\par 27 Dvadcátý na Eliatu, synum a bratrím jeho dvanácti.
\par 28 Jedenmecítmý na Hotira, synum a bratrím jeho dvanácti.
\par 29 Dvamecítmý na Giddalta, synum a bratrím jeho dvanácti.
\par 30 Trímecítmý na Machaziota, synum a bratrím jeho dvanácti.
\par 31 Ctyrmecítmý na Romantiezera, synum a bratrím jeho dvanácti.

\chapter{26}

\par 1 Zporádání pak vrátných takové bylo: Z Chorejských Meselemiáš syn Chóre, z synu Azafových.
\par 2 A z Meselemiášových synu: Zachariáš prvorozený, Jediael druhý, Zebadiáš tretí, Jatniel ctvrtý,
\par 3 Elam pátý, Jochanan šestý, Elioenai sedmý.
\par 4 A z Obededomových synu: Semaiáš prvorozený, Jozabad druhý, Joach tretí, Sachar ctvrtý a Natanael pátý,
\par 5 Amiel šestý, Izachar sedmý, Pehulletai osmý; nebo požehnal mu Buh.
\par 6 Semaiášovi pak synu jeho zrodili se synové, kteríž panovali v dome otce svého; nebo muži udatní byli.
\par 7 Synové Semaiášovi: Otni a Refael, Obéd a Elzabad, jehož bratrí byli muži udatní; též Elihu a Semachiáš.
\par 8 Všickni ti z potomku Obededomových, oni sami i synové jejich a bratrí jejich, jeden každý muž udatný a zpusobný k službe, šedesáte a dva všech z Obededoma.
\par 9 Též synu a bratrí Meselemiášových, mužu silných, osmnáct.
\par 10 Z Chosových pak, kterýž byl z synu Merari, synové byli: Simri kníže. Ackoli nebyl prvorozený, však postavil ho otec jeho za predního.
\par 11 Helkiáš druhý, Tebaliáš tretí, Zachariáš ctvrtý. Všech synu a bratrí Chosových trinácte.
\par 12 Tem rozdeleny jsou povinnosti, aby byli vrátnými po mužích predních, držíce stráž naproti bratrím svým pri službe v dome Hospodinove.
\par 13 Nebo metali losy, jakož malý, tak veliký, po domích svých otcovských, k jedné každé bráne.
\par 14 I padl los k východu Selemiášovi. Zachariášovi také synu jeho, rádci opatrnému, uvrhli losy, i padl los jeho na pulnoci.
\par 15 Obededomovi pak na poledne, ale synum jeho na dum pokladu.
\par 16 Suppimovi a Chosovi na západ s branou Salléchet, na ceste podlážené vzhuru jdoucí. Stráž byla naproti stráži.
\par 17 K východu Levítu šest, k pulnoci na den ctyri, ku poledni na den ctyri, a pri dome pokladu dva a dva,
\par 18 V strane zevnitrní k západu, po ctyrech k príkopu, po dvou k strane zevnitrní.
\par 19 Ta jsou zporádání vrátných synu Chóre a synu Merari.
\par 20 Tito také Levítové: Achiáš byl nad poklady domu Božího, totiž nad poklady vecí posvátných.
\par 21 Z synu Ladanových synové Gersunských, z Ladana knížata otcovských celedí, z Ladana totiž Gersunského Jechiel.
\par 22 Synové Jechielovi, Zetam a Joel bratr jeho, byli nad poklady domu Hospodinova.
\par 23 Z Amramských, z Izarských z Hebronských a z Ozielských.
\par 24 Sebuel pak syn Gersoma, syna Mojžíšova, prední nad poklady.
\par 25 Ale bratrí jeho z Eliezera: Rechabiáš syn jeho, a Izaiáš syn jeho, a Joram syn jeho, a Zichri syn jeho, a Selomit syn jeho.
\par 26 Ten Selomit a bratrí jeho byli nade všemi poklady vecí posvátných, kterýchž byl posvetil David král a knížata celedí otcovských, i hejtmané a setníci s vývodami vojska.
\par 27 Nebo z boju a z koristí obetovávali k oprave domu Hospodinova,
\par 28 A cehožkoli byl posvetil Samuel prorok, a Saul syn Cis, a Abner syn Neruv, a Joáb syn Sarvie. Kdokoli posvecoval ceho, dával do rukou Selomita a bratrí jeho.
\par 29 Z Izarských Chenaniáš a synové jeho, nad dílem, kteréž vne deláno, byli v Izraeli za úredníky a soudce.
\par 30 Z Hebronských Chasabiáš a bratrí jeho, mužu silných tisíc a sedm set bylo v prednosti nad Izraelem, za Jordánem k západu, ve všelikém díle Hospodinove a v službe královské.
\par 31 Mezi kterýmiž Hebronskými Jeriáš kníže byl nad Hebronskými v pokolení jejich, po celedech otcovských; nebo léta ctyridcátého kralování Davidova vyhledáváni byli, a nalezeni jsou mezi nimi muži udatní v Jazar Galádské,
\par 32 A bratrí jeho, mužu silných, dva tisíce a sedm set, knížat otcovských celedí. Kteréž ustanovil David král nad Rubenskými a Gádskými, a nad polovicí pokolení Manassesova, ve všech vecech Božských i vecech královských.

\chapter{27}

\par 1 Synové pak Izraelští podlé poctu svého, knížata celedí otcovských a hejtmané a setníci a úredníci nad temi, kteríž prisluhovali králi ve všech vecech jedni po druhých, vcházejíce a vycházejíce na každý mesíc, pres všecky mesíce roku: V jednom každém houfe bylo jich ctyrmecítma tisícu.
\par 2 Nad houfem prvním první mesíc byl Jasobam, syn Zabdieluv, a v houfe jeho bylo ctyrmecítma tisícu.
\par 3 Z synu Fáresových bylo to kníže všech knížat nad vojsky, mesíce prvního.
\par 4 Zatím nad houfem na mesíc druhý byl Dodai Achochitský i s houfem svým, potom Miklot vývoda, a v houfe jeho ctyrmecítma tisícu.
\par 5 Kníže vojska tretího na tretí mesíc Banaiáš syn Joiady, nejvyššího kneze, a v houfe jeho ctyrmecítma tisícu.
\par 6 Ten Banaiáš byl silný mezi tridcíti a nad tridcíti, a v houfe jeho Amizabad syn jeho.
\par 7 Ctvrtého houfu kníže na ctvrtý mesíc Azael, bratr Joábuv, a Zebadiáš syn jeho po nem, a v houfe jeho ctyrmecítma tisícu.
\par 8 Pátého na pátý mesíc kníže Samhut Izrachitský, a v houfe jeho ctyrmecítma tisícu.
\par 9 Šestého na šestý mesíc Híra, syn Ikeše Tekoitského, a v houfe jeho ctyrmecítma tisícu.
\par 10 Sedmého na sedmý mesíc Chelez Pelonský z synu Efraimových, a v houfe jeho ctyrmecítma tisícu.
\par 11 Osmého na osmý mesíc Sibbechai Chusatský z Zarchejských, a v houfe jeho ctyrmecítma tisícu.
\par 12 Devátého na devátý mesíc Abiezer Anatotský z Beniaminských, a v houfe jeho ctyrmecítma tisícu.
\par 13 Desátého na mesíc desátý Maharai Netofatský z Zarchejských, a v houfe jeho ctyrmecítma tisícu.
\par 14 Jedenáctého na jedenáctý mesíc Banaiáš Faratonský z synu Efraimových, a v houfe jeho ctyrmecítma tisícu.
\par 15 Dvanáctého na dvanáctý mesíc Cheldai Netofatský z Otoniele, a v houfe jeho ctyrmecítma tisícu.
\par 16 Mimo to byli nad pokoleními Izraelskými, nad Rubenskými vývoda Eliezer syn Zichruv, nad Simeonskými Sefatiáš syn Maachuv,
\par 17 Nad pokolením Léví Chasabiáš syn Kemueluv, nad Aronovým Sádoch,
\par 18 Nad Judovým Elihu z bratrí Davidových, nad Izacharovým Amri syn Michaeluv,
\par 19 Nad Zabulonovým Izmaiáš syn Abdiášuv, nad Neftalímovým Jerimot syn Azrieluv,
\par 20 Nad syny Efraimovými Ozeáš syn Azaziášuv, nad polovicí pokolení Manasse Joel syn Pedaiášuv,
\par 21 Nad druhou pak polovicí Manasse v Gálad Iddo syn Zachariášuv, nad Beniaminovým Jaasiel syn Abneruv,
\par 22 Nad Danovým Azarel syn Jerochamuv. Ta jsou knížata pokolení Izraelských.
\par 23 Nesectl jich pak David všech od dvadcítiletých a níže; nebo byl rekl Hospodin, že rozmnoží Izraele jako hvezdy nebeské.
\par 24 A ackoli Joáb syn Sarvie pocal jich pocítati, však nedokonal; nebo proto prišel hnev Boží na Izraele, aniž jest vložen pocet ten v knihu o králi Davidovi.
\par 25 Nad poklady pak královskými byl Azmavet syn Adieluv, a nad duchody z polí, z mest a ze vsí i z zámku byl Jonatan syn Uziášuv.
\par 26 A nad delníky na poli, pri delání rolí, Ezri syn Chelubuv.
\par 27 A nad vinicemi Semeiáš Ramatský, a nad úrodami z vinic i nad sklepy vinnými Zabdi Sifmejský.
\par 28 A nad olivovím a planým fíkovím, kteréž jest na polích, Baalchanan Gederský, a nad špižírnami olejnými Joas.
\par 29 A nad skoty, kteríž se pásli v Sáron, Sitrai Sáronský, a nad skoty v údolích Safat syn Adlai.
\par 30 A nad velbloudy Obil Izmaelský, nad oslicemi Jechdeiáš Meronotský,
\par 31 Nad bravy pak Jazim Agarejský. Všickni ti úredníci byli nad statkem krále Davida.
\par 32 Ale Jonatan strýc Daviduv byl rada, muž rozumný a kanclér. On a Jechiel syn Chachmonuv býval s syny královskými.
\par 33 Achitofel též rada králova, a Chusai Architský prítel králuv.
\par 34 Po Achitofelovi potom byl Joiada syn Banaiášuv, a Abiatar, a kníže vojska králova Joáb.

\chapter{28}

\par 1 Shromáždil pak David všecka knížata Izraelská, knížata jednoho každého pokolení, a knížata houfu sloužících králi, i hejtmany a setníky, i úredníky nade vším statkem a jmením královým i synu jeho s komorníky, i se všemi vzácnými a udatnými lidmi, do Jeruzaléma.
\par 2 A povstav David král na nohy své, rekl: Slyšte mne, bratrí moji a lide muj. Já jsem uložil v srdci svém vystaveti dum k odpocinutí truhle úmluvy Hospodinovy, a ku podnoži noh Boha našeho, a pripravil jsem byl potreby k stavení.
\par 3 Ale Buh mi rekl: Nebudeš staveti domu jménu mému, proto že jsi muž válkami zamestknaný a krev jsi vyléval.
\par 4 Vyvolil pak Hospodin Buh Izraelský mne ze vší celedi otce mého, abych byl králem nad Izraelem na veky; nebo z Judy vybral vývodu a z domu Judova celed otce mého, a z synu otce mého mne rácil za krále ustanoviti nade vším Izraelem.
\par 5 Tolikéž ze všech synu mých, (nebo mnoho synu dal mi Hospodin), vybral Šalomouna syna mého, aby sedel na stolici království Hospodinova nad Izraelem,
\par 6 A rekl mi: Šalomoun syn tvuj, ten mi vzdelá dum muj a síne mé; nebo jsem jej sobe zvolil za syna, a já budu jemu za otce.
\par 7 I utvrdím království jeho až na veky, bude-li stálý v ostríhání prikázaní mých a soudu mých, jako i nynejšího casu.
\par 8 Nyní tedy pri prítomnosti všeho Izraele, shromáždení Hospodinova, an slyší Buh náš, napomínám vás: Zachovávejte a dotazujte se na všecka prikázaní Hospodina Boha svého, abyste vládli zemí dobrou, a v dedictví její uvedli i syny své po sobe až na veky.
\par 9 Ty také, Šalomoune, synu muj, znej Boha otce svého, a služ jemu celým srdcem a myslí ochotnou. Nebo všecka srdce zpytuje Hospodin a všeliká mysli tanutí zná. Budeš-li ho hledati, nalezneš jej; pakli ho opustíš, zavrže te na veky.
\par 10 A tak viziž, že te Hospodin zvolil, abys vystavel dum svatyne; posilniž se a delej.
\par 11 Dal pak David Šalomounovi synu svému formu sínce i pokoju jejích, a sklepu i palácu a komor jejích vnitrních, i domu pro slitovnici,
\par 12 A formu všeho toho, což byl složil v mysli své o síních domu Hospodinova, i o všech komorách vukol chrámu pro poklady domu Božího, i pro poklady vecí posvátných,
\par 13 I pro houfy kneží a Levítu, a pro všecko dílo služby domu Hospodinova, a pro všecko nádobí k prisluhování v dome Hospodinove.
\par 14 Dal též zlata v jisté váze na nádobí zlaté, na všelijaké nádobí k jedné každé službe, též stríbra na všecky nádoby stríbrné v jisté váze na všelijaké nádobí k jedné každé službe,
\par 15 Totiž váhu na svícny zlaté, a lampy jejich zlaté podlé váhy jednoho každého svícnu i lamp jeho, na svícny pak stríbrné podlé váhy svícnu každého a lamp jeho, jakž potrebí bylo každému svícnu.
\par 16 Zlata též váhu na stoly predložení na jeden každý stul, i stríbra na stoly stríbrné,
\par 17 I na vidlicky a na kotlíky, i na prikryvadla zlata ryzího, a na medenice zlaté váhu na jednu každou medenici, tolikéž na medenice stríbrné jistou váhu na jednu každou medenici.
\par 18 Také na oltár k kadení zlata ryzího váhu, zlata i k udelání vozu cherubínu, kteríž by roztaženými krídly zastírali truhlu úmluvy Hospodinovy.
\par 19 Všecko to skrze vypsání z ruky Hospodinovy mne došlo, kterýž mi to dal, abych vyrozumel všemu dílu formy té.
\par 20 A tak rekl David Šalomounovi synu svému: Posilniž se a zmocni a delej; neboj se, ani lekej. Nebo Hospodin Buh, Buh muj s tebou bude, nenechát tebe samého, aniž te opustí, až i dokonáno bude všecko dílo služby domu Hospodinova.
\par 21 Hle, i houfové kneží a Levítu ke všeliké službe domu Božího s tebou také budou pri všelikém díle, jsouce všickni ochotní a prozretelní v moudrosti pri všeliké práci, knížata také i všecken lid ke všechnem slovum tvým.

\chapter{29}

\par 1 Potom rekl David král všemu shromáždení: Šalomouna syna mého jediného vyvolil Buh ješte malického a mladého, dílo pak toto veliké jest; nebo ne cloveku palác ten, ale Hospodinu Bohu býti má.
\par 2 Já zajisté podlé své nejvyšší možnosti nachystal jsem potreb k domu Boha svého, zlata k nádobám zlatým, a stríbra k stríbrným, medi k medeným, železa k železným a dríví k dreveným, kamení onychinového i kamení k vsazování, i karbunkulového, a rozlicných barev, a všelijakého kamení drahého, i kamení mramorového hojne.
\par 3 Nadto z veliké lásky k domu Boha svého, maje zvláštní poklad zlata a stríbra, i ten dávám k domu Boha svého mimo všecko to, což jsem pripravil k domu svatyne,
\par 4 Totiž tri tisíce centnéru zlata, zlata z Ofir, a sedm tisíc centnéru stríbra precišteného k potažení sten domu svatých,
\par 5 Zlata k nádobám zlatým a stríbra k stríbrným i ke všelikému dílu remeslnému. A jestliže jest kdo více, ješto by co chtel dobrovolne obetovati dnes Hospodinu?
\par 6 Tedy knížata celedí otcovských a knížata pokolení Izraelských, i hejtmané a setníci i úredníci nad dílem královským dobrovolne obetovali.
\par 7 A dali k službe domu Božího zlata pet tisíc centnéru, a deset tisíc zlatých, stríbra pak deset tisíc centnéru, a medi osmnácte tisíc centnéru, a železa sto tisíc centnéru.
\par 8 Kdožkoli meli kamení drahé, dávali do pokladu domu Hospodinova do rukou Jechiele Gersunského.
\par 9 I veselil se lid, že tak ochotne obetovali; nebo celým srdcem dobrovolne obetovali Hospodinu. Nýbrž i král David radoval se radostí velikou.
\par 10 A protož dobrorecil David Hospodinu pred oblícejem všeho shromáždení, a rekl David: Požehnaný jsi ty Hospodine, Bože Izraele otce našeho, od veku až na veky.
\par 11 Tvát jest, ó Hospodine, velebnost i moc i sláva, i vítezství i cest, ano i všecko, což jest v nebi i v zemi. Tvé jest, ó Hospodine, království, a ty jsi vyšší nad všelikou vrchnost.
\par 12 I bohatství i sláva od tebe jest, a ty panuješ nade vším, a v ruce tvé jest moc a síla, a v ruce tvé také jest i to, kohož chceš zvelebiti a upevniti.
\par 13 Nyní tedy, Bože náš, dekujeme tobe, a chválíme jméno slávy tvé.
\par 14 Nebo kdo jsem já, a co jest lid muj, abychom mohli míti moc k tak dobrovolnému obetování tobe? Od tebet jest zajisté všecko, a i to z ruky tvé dali jsme tobe.
\par 15 Nebo my príchozí jsme pred tebou a hosté, jako i všickni otcové naši; dnové naši jsou jako stín bežící po zemi beze vší zástavy.
\par 16 Hospodine, Bože náš, všecka ta hojnost, kterouž jsme pripravili k stavení domu tvého, jménu svatému tvému, z ruky tvé jest, a tvé jsou všecky veci.
\par 17 Známt pak, Bože muj, že ty zpytuješ srdce, a uprímnost oblibuješ; protož já z uprímnosti srdce svého ochotne obetoval jsem toto všecko. Ano i lid tvuj, kterýž se ted shledal, videl jsem, jak s radostí chtive obetuje tobe.
\par 18 Hospodine, Bože Abrahama, Izáka a Izraele, otcu našich, zachovejž to na veky,totiž snažnost takovou srdce lidu svého,a nastrojuj srdce jejich sobe.
\par 19 Šalomounovi také synu mému dej srdce uprímé, aby ostríhal prikázaní tvých, svedectví tvých a ustanovení tvých, a aby cinil všecko, a vystavel dum ten, k nemuž jsem potreby pripravil.
\par 20 Potom rekl David všemu shromáždení: Dobrorecte nyní Hospodinu Bohu svému. I dobrorecilo všecko shromáždení Hospodinu Bohu otcu svých, a prisehnuvše hlavy, poklonili se Hospodinu i králi.
\par 21 Zatím obetovali Hospodinu obeti, obetovali také zápaly Hospodinu nazejtrí, volu tisíc, skopcu tisíc, beranu tisíc, s mokrými obetmi jejich, a jiných obetí množství za všecken lid Izraelský.
\par 22 I jedli a pili pred Hospodinem v ten den s radostí velikou. Potom za krále ustanovili po druhé Šalomouna syna Davidova, a pomazali jej Hospodinu za vývodu, a Sádocha za kneze.
\par 23 A tak dosedl Šalomoun na stolici Hospodinovu, aby byl králem místo Davida otce svého. I vedlo se mu štastne, a poslouchal ho všecken Izrael.
\par 24 Tolikéž i všecka knížata a znamenití, i všickni synové krále Davida poddali se Šalomounovi králi.
\par 25 I zvelebil Hospodin Šalomouna náramne pred ocima všeho Izraele, a dal mu slávu královskou, jakéž nemel pred ním žádný král v Izraeli.
\par 26 Kraloval pak byl David syn Izai nade vším Izraelem.
\par 27 A dnu, v nichž kraloval David nad Izraelem, bylo ctyridceti let. V Hebronu kraloval sedm let, a v Jeruzaléme kraloval tridceti a tri léta.
\par 28 I umrel v starosti dobré, pln jsa dnu, bohatství a slávy, a kraloval Šalomoun syn jeho místo neho.
\par 29 Cinové pak Davida krále první i poslední popsáni jsou v knize Samuele proroka, a v knize Nátana proroka, též v knize Gáda proroka,
\par 30 Se vším kralováním i silou jeho, i se všemi príbehy casu pri nem i pri lidu Izraelském, i pri všech královstvích zemských.

\end{document}