\begin{document}

\title{Exodus}

\chapter{1}

\par 1 Tato jsou pak jména synu Izraelských, kteríž vešli do Egypta s Jákobem; každý s celedí svou prišel:
\par 2 Ruben, Simeon, Léví a Juda;
\par 3 Izachar, Zabulon a Beniamin;
\par 4 Dan a Neftalím, Gád a Asser.
\par 5 A bylo všech osob pošlých z bedr Jákobových sedmdesáte duší; Jozef pak byl v Egypte.
\par 6 I umrel Jozef a všickni bratrí jeho, a všecken ten rod.
\par 7 Synové pak Izraelští rozmnožili se a rodili se v hojnosti; a rozhojnovali se, i zsilili se náramne velmi, a naplnena jest jimi zeme.
\par 8 V tom povstal král nový v Egypte, kterýž neznal Jozefa.
\par 9 Ten rekl lidu svému: Aj, lid synu Izraelských jest mnohý a silnejší nad nás.
\par 10 Nuže, opatrne sobe pocínejme pred ním, aby se nerozmohl; a když by prišla válka, aby se nepripojil i on k neprátelum našim, a nebojoval proti nám, a nevyšel z zeme.
\par 11 Protož ustanovili nad ním úredníky, kteríž by plat vybírali, aby je trápili bremeny svými. I vystavel lid Izraelský Faraonovi mesta skladu, Fiton a Ramesses.
\par 12 Ale cím více trápili jej, tím více rostl a tím se více rozmáhal. I vzali sobe syny Izraelské v ošklivost.
\par 13 A tak podrobovali Egyptští syny Izraelské v službu težkou.
\par 14 A k horkosti privodili život jejich robotami težkými, v hline a cihlách a ve všelijakém díle na poli, mimo všelikou potrebu svou, k níž práce jejich užívali nenáležite a bez lítosti.
\par 15 I porucil král Egyptský babám Hebrejským, z nichž jedna sloula Sefora a druhá Fua,
\par 16 A rekl: Když budete pomáhati ženám Hebrejským pri porodu, a uzríte, že již rodí, byl-li by syn, zabíte ho, pakli dcera, tedy at jest živa.
\par 17 Bály se pak ty baby Boha, a necinily tak, jakž jim porucil král Egyptský, ale živých nechávaly pacholíku.
\par 18 Povolav tedy bab král Egyptský, mluvil jim: Proc jste to ucinily, že jste živé zachovaly pacholíky?
\par 19 I odpovedely baby Faraonovi: Nejsou ženy Hebrejské jako ženy Egyptské; nebo ony jsou silnejšího prirození. Dríve než prijde k nim baba, ony porodí.
\par 20 I ucinil dobre Buh tem babám. A rozmnožen jest lid, a zsilili se velmi.
\par 21 Stalo se pak proto, že se bály baby ty Boha, vzdelal jim domy.
\par 22 I prikázal Farao všemu lidu svému, rka: Každého syna, kterýž se narodí, do reky uvrzte; každé pak dcery nechte živé.

\chapter{2}

\par 1 Odšed pak muž jeden z domu Léví, vzal dceru z pokolení Léví.
\par 2 I pocala žena ta, a porodila syna; a viduci, že jest krásný, kryla ho za tri mesíce.
\par 3 A když ho nemohla déle tajiti, vzala mu ošitku z sítí, a omazala ji klím a smolou; a vložila do ní to díte, a vyložila do rákosí u brehu reky.
\par 4 A postavila sestru jeho zdaleka, aby zvedela, co se s ním díti bude.
\par 5 A v tom sešla dcera Faraonova, aby se myla v rece; a devecky její procházely se po brehu reky. A uzrevši ošitku mezi rákosím, poslala devecku svou, a vzala ji.
\par 6 A když otevrela, uzrela díte; a aj, plakalo pacholátko. A slitovavši se nad ním, rekla: Z detí Hebrejských jest toto.
\par 7 I rekla sestra jeho k dceri Faraonove: Mám-li jíti a zavolati tobe chuvy z žen Hebrejských, kteráž by odchovala tobe díte?
\par 8 Odpovedela dcera Faraonova: Jdi. Tedy šla devecka a zavolala matky toho dítete.
\par 9 I rekla jí dcera Faraonova: Vezmi toto díte, a odchovej mi je; a ját dám mzdu tvou. I vzala žena díte, a chovala je.
\par 10 A když odrostlo pachole, dovedla je k dceri Faraonove, kteráž jej mela za syna; a nazvala jméno jeho Mojžíš, rkuci: Nebo jsem ho z vody vytáhla.
\par 11 I stalo se ve dnech tech, když vyrostl Mojžíš, že vyšel k bratrím svým, a hledel na trápení jejich. Uzrel také muže Egyptského, an tepe muže Hebrejského, jednoho z bratrí jeho. A sem i tam se ohlédna, vida, že žádného tu není, zabil Egyptského, a zahrabal jej v písku.
\par 12 Vyšed potom druhého dne, a aj, dva muži Hebrejští vadili se. I rekl tomu, kterýž krivdu cinil: Proc tepeš bližního svého?
\par 13 Kterýžto odpovedel: Kdo te ustanovil knížetem a soudcí nad námi? Zdali zabiti mne myslíš, jako jsi zabil Egyptského? Protož ulekl se Mojžíš a rekl: Jiste známá jest ta vec.
\par 14 A uslyšav Farao tu vec, hledal zabiti Mojžíše. Ale Mojžíš utekl od tvári Faraonovy, a bydlil v zemi Madianské; i usadil se podlé studnice.
\par 15 Kníže pak Madianské melo sedm dcer. Kteréžto prišedše, vážily vodu, a nalívaly do koryt, aby napájely dobytek otce svého.
\par 16 I prišli pastýri, a odehnali je. Tedy Mojžíš vstav, pomohl jim a napojil dobytek jejich.
\par 17 A když se navrátily k Raguelovi, otci svému, rekl on: Jakž jste to dnes tak brzo prišly?
\par 18 Odpovedely: Muž Egyptský vysvobodil nás z ruky pastýru; ano také ochotne navážil nám vody, a napojil dobytek.
\par 19 I rekl dcerám svým: Kdež pak jest? Procež jste pustily muže toho? Povolejte ho, at pojí chleba.
\par 20 A svolil Mojžíš k tomu, aby bydlil s mužem tím. Kterýžto dal Zeforu, dceru svou, Mojžíšovi.
\par 21 I porodila syna, a nazval jméno jeho Gerson; nebo rekl: Príchozí jsem byl v zemi cizí.
\par 22 Stalo se pak po mnohých casích, že umrel král Egyptský; a synové Izraelští úpeli pro roboty, a kriceli. I vstoupil k Bohu krik jejich pro roboty.
\par 23 A uslyšel Buh naríkání jejich, a rozpomenul se Buh na smlouvu svou s Abrahamem, Izákem a Jákobem.
\par 24 I vzhlédl Buh na syny Izraelské, a poznal Buh.

\chapter{3}

\par 1 Mojžíš pak pásl dobytek Jetry tchána svého, kneze Madianského, a hnav stádo po poušti, prišel až k hore Boží Oréb.
\par 2 Tedy ukázal se mu andel Hospodinuv v plameni ohne z prostredku kre. I videl, a aj, ker horel ohnem, a však neshorel.
\par 3 Protož rekl Mojžíš: Pujdu nyní, a spatrím videní toto veliké, proc neshorí ker.
\par 4 Vida pak Hospodin, že jde, aby pohledel, zavolal nan Buh z prostredku kre, a rekl: Mojžíši, Mojžíši! Kterýžto odpovedel: Aj, ted jsem.
\par 5 I rekl: Nepristupuj sem, szuj obuv svou s noh svých; nebo místo, na kterémž ty stojíš, zeme svatá jest.
\par 6 A rekl: Já jsem Buh otce tvého, Buh Abrahamuv, Buh Izákuv, a Buh Jákobuv. I zakryl Mojžíš tvár svou, (nebo se bál), aby nepatril na Boha.
\par 7 Jemužto rekl Hospodin: Zretelne videl jsem trápení lidu mého, kterýž jest v Egypte; a krik jejich pro prísnost úredníku jeho slyšel jsem; nebo znám bolesti jeho.
\par 8 Protož jsem sstoupil, abych vysvobodil jej z ruky Egyptských, a vyvedl jej z zeme té do zeme dobré a prostranné, do zeme oplývající mlékem a strdí, na místa Kananejského a Hetejského, a Amorejského a Ferezejského, a Hevejského a Jebuzejského.
\par 9 Nebo nyní, aj, krik synu Izraelských prišel ke mne; videl jsem také i ssoužení, jímž je ssužují Egyptští.
\par 10 Protož, nyní pod a pošli te k Faraonovi; a vyvedeš lid muj, syny Izraelské z Egypta.
\par 11 I rekl Mojžíš Bohu: Kdo jsem já, abych šel k Faraonovi, a abych vyvedl syny Izraelské z Egypta?
\par 12 I odpovedel: Však budu s tebou; a toto budeš míti znamení, že jsem já te poslal: Když vyvedeš lid ten z Egypta, sloužiti budete Bohu na hore této.
\par 13 I rekl Mojžíš Bohu: Aj, já pujdu k synum Izraelským a dím jim: Buh otcu vašich poslal mne k vám. Reknou-li mi: Které jest jméno jeho? co jim odpovím?
\par 14 I rekl Buh Mojžíšovi: JSEM, KTERÝŽ JSEM. Rekl dále: Takto díš synum Izraelským: JSEM poslal mne k vám.
\par 15 Rekl ješte Buh Mojžíšovi: Takto díš synum Izraelským: Hospodin, Buh otcu vašich, Buh Abrahamuv, Buh Izákuv, a Buh Jákobuv poslal mne k vám; tot jest jméno mé na vecnost, a tat jest památka má po všecky veky.
\par 16 Jdi, a shromážde starší Izraelské, mluv jim: Hospodin Buh otcu vašich ukázal mi se, Buh Abrahamuv, Buh Izákuv, a Buh Jákobuv, rka: Rozpomínaje, rozpomenul jsem se na vás, a na to, co se vám dálo v Egypte.
\par 17 Protož jsem rekl: Vyvedu vás z trápení Egyptského do zeme Kananejského, a Hetejského, a Amorejského, a Ferezejského, a Hevejského, a Jebuzejského, do zeme oplývající mlékem a strdí.
\par 18 I poslechnou hlasu tvého. Pujdeš pak ty a starší Izraelští k králi Egyptskému, a díte jemu: Hospodin Buh Hebrejský potkal se s námi; protož nyní, necht medle jdeme cestou trí dnu na poušt, abychom obetovali Hospodinu Bohu našemu.
\par 19 Ale já vím, žet vám nedopustí král Egyptský jíti; lec v ruce silné.
\par 20 Protož vztáhnu ruku svou, a bíti budu Egypt divnými vecmi svými, kteréž ciniti budu u prostred neho; a potom propustí vás.
\par 21 A dám milost lidu tomuto pred ocima Egyptských. I stane se, že když pujdete, neodejdete prázdní.
\par 22 Ale vypujcí žena od sousedy své, a od hospodyne domu svého klínotu stríbrných, a klínotu zlatých a roucha; i vložíte to na syny a na dcery své, a tak obloupíte Egypt.

\chapter{4}

\par 1 Odpovedel pak Mojžíš, a rekl: Aj, neuverí mi, ani uposlechnou hlasu mého; nebo reknou: Neukázalt se tobe Hospodin.
\par 2 Tedy rekl jemu Hospodin: Co jest to v ruce tvé? Odpovedel: Hul.
\par 3 I rekl: Vrz ji na zem. I povrhl ji na zem, a obrácena jest v hada; a utíkal Mojžíš pred ním.
\par 4 Tedy rekl Hospodin Mojžíšovi: Vztáhni ruku svou, a chyt ho za ocas. Kterýžto vztáh ruku svou, chytil jej, a obrácen jest v hul v rukou jeho.
\par 5 Aby verili, že se ukázal tobe Hospodin, Buh otcu jejich, Buh Abrahamuv, Buh Izákuv a Buh Jákobuv.
\par 6 Potom zase rekl jemu Hospodin: Vlož nyní ruku svou za nadra svá. I vložil ruku svou za nadra svá; a vynal ji, a aj, ruka jeho byla malomocná, bílá jako sníh.
\par 7 Rekl opet: Vlož ruku svou zase v nadra svá. Kterýž vložil ruku svou zase v nadra svá; a vynal ji z nader svých, a aj, ucinena jest zase jako jiné telo jeho.
\par 8 I budet, jestliže neuverí tobe, a neposlechnou hlasu a znamení prvního, uverí hlasu a znamení druhému.
\par 9 A pakli neuverí ani tem dvema znamením, a neuposlechnou hlasu tvého, tedy nabereš vody z reky, a vyliješ ji na zem; a promení se vody, kteréž vezmeš z reky, a obrátí se v krev na zemi.
\par 10 I rekl Mojžíš Hospodinu: Prosím, Pane, nejsem muž výmluvný, aniž prvé, ani jakž jsi mluvil s služebníkem svým; nebo zpozdilých úst a neohbitého jazyku jsem.
\par 11 Jemuž odpovedel Hospodin: Kdo dal ústa cloveku? Aneb kdo muže uciniti nemého, neb hluchého, vidoucího, neb slepého? Zdali ne já Hospodin?
\par 12 Nyní tedy jdi, a já budu v ústech tvých, a naucím te, co bys mluviti mel.
\par 13 I rekl: Slyš mne, Pane, pošli, prosím, toho, kteréhož poslati máš.
\par 14 A rozhnevav se velmi Hospodin na Mojžíše, rekl: Zdaliž nemáš Arona bratra svého z pokolení Léví? Vím, že on výmluvný jest; ano aj, sám vyjde v cestu tobe, a vida tebe, radovati se bude v srdci svém.
\par 15 Ty mluviti budeš k nemu, a vložíš slova v ústa jeho; a já budu v ústech tvých a v ústech jeho, a naucím vás, co byste meli ciniti.
\par 16 A on mluviti bude za tebe k lidu; a bude tobe on za ústa, a ty budeš jemu za Boha.
\par 17 Hul pak tuto vezmeš v ruku svou, kterouž ciniti budeš ta znamení.
\par 18 Tedy odšed Mojžíš, navrátil se k Jetrovi tchánu svému, a rekl jemu: Necht jdu nyní, a navrátím se k bratrím svým, kteríž jsou v Egypte, a pohledím, jsou-li ješte živi. I rekl Jetro Mojžíšovi: Jdi v pokoji.
\par 19 Nebo rekl byl Hospodin Mojžíšovi v zemi Madianské: Jdi, navrat se do Egypta; nebo zemreli jsou všickni muži, kteríž hledali bezživotí tvého.
\par 20 A vzav Mojžíš ženu svou, a syny své, vsadil je na osla, aby se navrátil do zeme Egyptské; vzal také Mojžíš hul Boží v ruku svou.
\par 21 I rekl Hospodin Mojžíšovi: Když pujdeš a navrátíš se do Egypta, hled, abys všecky zázraky, kteréž jsem složil v ruce tvé, cinil pred Faraonem. Ját pak zatvrdím srdce jeho, aby nepropustil lidu.
\par 22 Protož díš Faraonovi: Toto praví Hospodin: Syn muj, prvorozený muj jest Izrael.
\par 23 I rekl jsem tobe: Propust syna mého, at slouží mi; a nechtel jsi ho propustiti. Aj, já zabiji syna tvého, prvorozeného tvého.
\par 24 I stalo se, když byl Mojžíš na ceste v hospode, že se oboril na nej Hospodin, a hledal ho usmrtiti.
\par 25 Tedy vzala Zefora nuž ostrý, a obrezala neobrízku syna svého, kteroužto vrhla k nohám jeho, rkuci: Zajisté ženich krví jsi mi.
\par 26 I nechal ho. Ona pak nazvala ho tehdáž ženichem krví pro obrezání.
\par 27 Rekl také Hospodin Aronovi: Jdi vstríc Mojžíšovi na poušt. I šel a potkal se s ním na hore Boží, a políbil ho.
\par 28 A vypravoval Mojžíš Aronovi všecka slova Hospodinova, kterýž ho poslal, i o všech znameních, kteráž prikázal jemu.
\par 29 Tedy šel Mojžíš s Aronem, a shromáždili všecky starší synu Izraelských.
\par 30 I mluvil Aron všecka slova, kteráž byl mluvil Hospodin k Mojžíšovi, a cinil znamení pred ocima lidu.
\par 31 A uveril lid, když uslyšeli, že navštívil Hospodin syny Izraelské, a že videl ssoužení jejich. A sklonivše se, poklonu ucinili.

\chapter{5}

\par 1 Potom pak prišli Mojžíš s Aronem, a rekli Faraonovi: Takto praví Hospodin, Buh Izraelský: Propust lid muj, at mi slaví svátky na poušti.
\par 2 Odpovedel Farao: Kdo jest Hospodin, abych poslechl hlasu jeho a propustil Izraele? Hospodina neznám, Izraele také nepropustím.
\par 3 I rekli: Buh Hebrejský potkal se s námi. Necht medle jdeme cestou trí dní na poušt, a obetujeme Hospodinu Bohu našemu, aby nedopustil na nás moru neb mece.
\par 4 I rekl jim král Egyptský: Proc ty Mojžíši a Arone, odtrhujete lid od prací jejich? Jdete k robotám svým.
\par 5 Rekl také Farao: Hle, již nyní mnoho jest lidu toho v zemi, a vy odvozujete je od robot jejich.
\par 6 I prikázal Farao v ten den úredníkum nad lidem a šafárum jeho, rka:
\par 7 Nedávejte již více slámy lidu k delání cihel jako prvé; nechat jdou sami a sbírají sobe slámu.
\par 8 Však touž summu cihel, kterouž udelávali prvé, uložte na ne, nic neujímejte z ní; nebot zahálejí, a protož volají, rkouce: Podme, obetujme Bohu našemu.
\par 9 Necht se pritíží robot mužum tem; a necht pracují v nich, aby se neohlédali na slova lživá.
\par 10 Vyšedše tedy úredníci nad lidem a šafári jeho, mluvili k lidu, rkouce: Takto praví Farao: Já nebudu vám dávati slámy.
\par 11 Sami jdete, berte sobe slámu, kdekoli naleznete; ale nic nebude ujato díla vašeho.
\par 12 I rozbehl se lid po vší zemi Egyptské, aby trhal strnište místo slámy.
\par 13 A úredníci nutili je, rkouce: Vyplnte díla svá, úkol denní v den jeho, jako když sláma byla.
\par 14 I biti jsou šafári synu Izraelských, kteréž ustanovili nad nimi úredníci Faraonovi, a mluveno k nim: Proc jste nevyplnili úkolu svého v díle cihel, jako prvé, ani vcera ani dnes?
\par 15 I prišli šafári synu Izraelských, a volali k Faraonovi, rkouce: Proc tak deláš služebníkum svým?
\par 16 Slámy se nedává služebníkum tvým, a ríkají nám: Delejte cihly; a hle, služebníci tvoji biti bývají, a hreší lid tvuj.
\par 17 I rekl: Zahálíte, zahálíte, a protož mluvíte: Podme, obetujme Hospodinu.
\par 18 Protož nyní jdete, delejte; slámy se vám dávati nebude, ale vy summu cihel vyplnujte.
\par 19 Vidouce šafári synu Izraelských, že zle s nimi, ponevadž receno: Neujmete poctu cihel vašich z úkolu denního v den jeho,
\par 20 Potkali se s Mojžíšem a Aronem, kteríž stáli, aby jim vstríc vyšli, když by se vraceli od Faraona.
\par 21 A mluvili jim: Pohlediž Hospodin na vás, a sud; nebo zošklivili jste nás pred Faraonem a služebníky jeho, a dali jste mec v ruku jejich, aby nás zamordovali.
\par 22 I navrátil se Mojžíš k Hospodinu a rekl: Pane, proc jsi tyto zlé veci uvedl na lid tento? Proc jsi mne sem poslal?
\par 23 Nebo od té chvíle, jakž jsem všel k Faraonovi, abych mluvil jménem tvým, hure nakládá s lidem tímto, a ty jsi vždy nevysvobodil lidu svého.

\chapter{6}

\par 1 Odpovedel Hospodin Mojžíšovi: Nyní uzríš, co uciním Faraonovi; nebo v ruce silné propustí je, a v ruce mocné vyžene je z zeme své.
\par 2 Mluvil ješte Buh k Mojžíšovi a rekl jemu: Já jsem Hospodin.
\par 3 Ukázalt jsem se zajisté Abrahamovi, Izákovi a Jákobovi v tom, že jsem Buh silný všemohoucí; ale v jménu svém, Hospodin, nejsem poznán od nich.
\par 4 K tomu utvrdil jsem smlouvu svou s nimi, že jim dám zemi Kananejskou, zemi putování jejich, v níž pohostinu byli.
\par 5 Nad to, já slyšel jsem krik synu Izraelských, kteréž Egyptští v službu podrobují, a rozpomenul jsem se na smlouvu svou.
\par 6 Protož povez synum Izraelským: Já jsem Hospodin, a vyvedu vás z robot Egyptských, a vytrhnu vás z služby jejich, a vysvobodím vás v ruce vztažené a skrze soudy veliké.
\par 7 A vezmu vás sobe za lid, a budu vám za Boha; a zvíte, že jsem Hospodin Buh váš, vysvobozující vás z robot Egyptských.
\par 8 Uvedu vás také do zeme, o níž jsem, zdvihna ruku svou, prisáhl, že ji dám Abrahamovi, Izákovi a Jákobovi; a dám ji vám v dedictví; Já Hospodin.
\par 9 I mluvil tak Mojžíš synum Izraelským; ale neslyšeli Mojžíše pro úzkost ducha a službu pretežkou.
\par 10 Protož mluvil Hospodin Mojžíšovi, rka:
\par 11 Vejdi, mluv Faraonovi králi Egyptskému, at propustí syny Izraelské z zeme své.
\par 12 I mluvil Mojžíš pred Hospodinem, rka: Hle, synové Izraelští neposlechli mne, kterakž tedy poslechne mne Farao, a já jsem zpozdilý v reci?
\par 13 I mluvil Hospodin Mojžíšovi a Aronovi, a prikázaní dal jim k synum Izraelským a k Faraonovi králi Egyptskému, aby vyvedli syny Izraelské z zeme Egyptské.
\par 14 Tito jsou prední v celedech otcu svých. Synové Rubenovi, prvorozeného Izraelova: Enoch, Fallu, Ezron a Charmi. Ty jsou celedi Rubenovy.
\par 15 Synové pak Simeonovi: Jamuel, Jamin, Ahod, Jachin, Sohar a Saul, syn Kananejské. Ty jsou celedi Simeonovy.
\par 16 A tato jsou jména synu Léví v rodech jejich: Gerson, Kahat a Merari. Let pak života Léví bylo sto tridceti a sedm let.
\par 17 Synové Gerson: Lebni a Semei po celedech svých.
\par 18 A synové Kahat: Amram, Izar, Hebron a Uziel. Let pak života Kahat bylo sto tridceti a tri léta.
\par 19 A synové Merari: Moholi a Musi. Ty jsou celedi Léví v rodech svých.
\par 20 Pojal pak Amram ženu Jochebed, tetu svou, sobe za manželku, kterážto porodila mu Arona a Mojžíše. A let života Amramova bylo sto tridceti a sedm let.
\par 21 Synové také Izarovi: Chore, Nefega Zechri.
\par 22 A synové Uzielovi: Mizael, Elzafan a Sethri.
\par 23 Pojal pak Aron Alžbetu, dceru Aminadabovu, sestru Názonovu, sobe za manželku; kterážto porodila jemu Nádaba, Abiu, Eleazara a Itamara.
\par 24 Synové pak Chore: Asser, Elkana a Abiazaf. Ty jsou celedi Choritských.
\par 25 Eleazar pak syn Aronuv vzal jednu ze dcer Putielových sobe za manželku; kteráž mu porodila Fínesa. Ti jsou prední z otcu Levítských po celedech svých.
\par 26 To jest ten Aron a Mojžíš, jimž rekl Hospodin: Vyvedte syny Izraelské z zeme Egyptské po houfích jejich.
\par 27 Tito jsou, kteríž mluvili Faraonovi, králi Egyptskému, aby vyvedli syny Izraelské z Egypta; tot jest ten Mojžíš a Aron.
\par 28 Stalo se pak, když mluvil Hospodin k Mojžíšovi v zemi Egyptské,
\par 29 Že mu rekl takto: Já Hospodin; mluv k Faraonovi, králi Egyptskému, všecko, což já mluvím tobe.
\par 30 A rekl Mojžíš pred Hospodinem: Aj, já jsem zpozdilý v reci, kterakž tedy poslouchati mne bude Farao?

\chapter{7}

\par 1 I rekl Hospodin Mojžíšovi: Aj, ustanovil jsem te za Boha Faraonovi; Aron pak bratr tvuj bude prorokem tvým.
\par 2 Ty mluviti budeš všecko, což tobe prikáži; Aron pak bratr tvuj mluviti bude k Faraonovi, aby propustil syny Izraelské z zeme své.
\par 3 Ale ját zatvrdím srdce Faraonovo, a množiti budu znamení svá a zázraky své v zemi Egyptské.
\par 4 Aniž poslechne vás Farao. I vzložím ruku svou na Egypt, a vyvedu vojska svá, lid svuj, syny Izraelské, z zeme Egyptské skrze soudy veliké.
\par 5 I zvedít Egyptští, že já jsem Hospodin, když vztáhnu ruku svou na Egypt; a vyvedu syny Izraelské z prostredku jich.
\par 6 Tedy ucinil Mojžíš a Aron tak; jakž prikázal jim Hospodin, tak ucinili.
\par 7 A byl Mojžíš v osmdesáti, Aron pak v osmdesáti a trech letech, když mluvili s Faraonem.
\par 8 I rekl Hospodin Mojžíšovi a Aronovi takto:
\par 9 Když mluviti k vám bude Farao, rka: Ukažte od sebe zázrak, tedy díš Aronovi: Vezmi hul svou, a povrz pred Faraonem, i obrátí se v hada.
\par 10 Tedy všel Mojžíš s Aronem k Faraonovi, a ucinili tak, jakž prikázal Hospodin; a povrhl Aron hul svou pred Faraonem i pred služebníky jeho, a obrácena jest v hada.
\par 11 Povolal pak také Farao mudrcu a carodejníku; a ucinili i ti carodejníci Egyptští skrze cáry své tolikéž.
\par 12 Nebo povrhl každý z nich hul svou, a obráceny jsou v hady; ale požrela hul Aronova hole jejich.
\par 13 I posililo se srdce Faraonovo, a neuposlechl jich, tak jakž byl mluvil Hospodin.
\par 14 Protož rekl Hospodin Mojžíšovi: Obtížilo se srdce Faraonovo; nechce propustiti lidu toho.
\par 15 Jdi k Faraonovi ráno, aj, pujde ven k vode, a stuj naproti nemu pri brehu reky; a hul, kteráž obrácena byla v hada, vezmeš do ruky své.
\par 16 A díš mu: Hospodin Buh Hebrejský poslal mne k tobe, atbych rekl: Propust lid muj, aby sloužili mi na poušti; a aj, neuposlechls až dosavad.
\par 17 Protož takto praví Hospodin: Po tomto poznáš, že já jsem Hospodin: Aj, já uderím holí, kteráž jest v ruce mé, na vody, kteréž jsou v rece, a obráceny budou v krev.
\par 18 A ryby, kteréž jsou v rece, pomrou; i nasmradí se reka, a ustávati budou Egyptští, hledajíce vody, kterouž by pili z reky.
\par 19 Protož rekl Hospodin Mojžíšovi: Rci Aronovi: Vezmi hul svou, a vztáhni ruku svou na vody Egyptské, na reky jejich, na potoky jejich, i na jezera jejich, a na všecka shromáždení vod jejich, aby se obrátily v krev; i bude krev po vší zemi Egyptské, tak v nádobách drevených, jako kamenných.
\par 20 Tedy ucinili tak Mojžíš a Aron, jakž byl prikázal Hospodin; a zdvihna hul, uderil v vodu, kteráž byla v rece, pred ocima Faraonovýma a pred ocima služebníku jeho; i obráceny jsou všecky vody, kteréž byly v rece, v krev.
\par 21 Ryby pak, kteréž byly v rece, pomrely, a nasmradila se reka, tak že nemohli Egyptští píti vody z reky; a byla krev po vší zemi Egyptské.
\par 22 To též ucinili i carodejníci Egyptští skrze cáry své. I zsililo se srdce Faraonovo, aby neuposlechl jich, tak jakž byl mluvil Hospodin.
\par 23 A odvrátiv se Farao, prišel do domu svého; a ani k tomu nepriložil srdce svého.
\par 24 Kopali pak všickni Egyptští vukol reky, hledajíce vody ku pití; nebo nemohli píti vody z reky.
\par 25 A vyplnilo se dní sedm, jakž ranil Hospodin reku.

\chapter{8}

\par 1 I mluvil Hospodin k Mojžíšovi: Vejdi k Faraonovi a rci jemu: Takto praví Hospodin: Propust lid muj, at mi slouží.
\par 2 Pakli nebudeš chtíti propustiti, aj, já raním všecky krajiny tvé žabami.
\par 3 A vydá reka množství žab, kteréž vystoupí a polezou do domu tvého a do pokoje, v nemž líháš, a na ložce tvé, a do domu služebníku tvých i lidu tvého, a do pecí tvých a do testa tvého.
\par 4 I na tebe a na lid tvuj, i na všecky služebníky tvé polezou žáby.
\par 5 I rekl Hospodin Mojžíšovi: Rci Aronovi: Vztáhni ruku svou s holí svou na reky, na potoky a na jezera, a vyved žáby na zemi Egyptskou.
\par 6 I vztáhl Aron ruku svou na vody Egyptské; a vystoupily žáby a prikryly zemi Egyptskou.
\par 7 A ucinili tolikéž carodejníci skrze své cáry; a udelali, že vyšly žáby na zemi Egyptskou.
\par 8 Tedy Farao povolav Mojžíše a Arona, rekl: Modlte se Hospodinu, at odejme žáby ode mne a od lidu mého; a propustím ten lid, aby obetovali Hospodinu.
\par 9 I rekl Mojžíš Faraonovi: Poctím te tím, a povez, kdy bych se mel modliti za te a za služebníky tvé, a za lid tvuj, aby vypléneny byly žáby od tebe, i z domu tvých; toliko v rece zustanou.
\par 10 Kterýžto odpovedel: Zítra. A Mojžíš rekl: Podlé slova tvého necht jest, abys vedel, že žádného takového není, jako Hospodin Buh náš.
\par 11 I odejdou žáby od tebe a od domu tvých, i od služebníku tvých a od lidu tvého; toliko v rece zustanou.
\par 12 Tedy vyšel Mojžíš s Aronem od Faraona. I volal Mojžíš k Hospodinu, aby odjaty byly žáby, kteréž byl dopustil na Faraona.
\par 13 I ucinil Hospodin podlé slova Mojžíšova; a vymrely žáby z domu, ze vsí i z polí.
\par 14 I shrnuli je na hromady; a nasmradila se zeme.
\par 15 Vida pak Farao, že by dáno bylo oddechnutí, více zatvrdil se v srdci svém, a neuposlechl jich, jakož byl mluvil Hospodin.
\par 16 I rekl Hospodin Mojžíšovi: Rci k Aronovi: Vztáhni hul svou, a uder v prach zeme, aby obrátil se v stenice na vší zemi Egyptské.
\par 17 Kteríž ucinili tak. Nebo vztáhl Aron ruku svou s holí svou, a uderil prach zeme. I byly stenice na lidech i hovadech; všecken prach zeme obrátil se v stenice ve vší zemi Egyptské.
\par 18 Delali také tak carodejníci skrze cáry své, aby vyvedli stenice, ale nemohli. A byly stenice na lidech i hovadech.
\par 19 Tedy rekli carodejníci Faraonovi: Prst Boží toto jest. A posililo se srdce Faraonovo, aniž poslechl jich, jakož mluvil Hospodin.
\par 20 Rekl pak Hospodin Mojžíšovi: Vstan ráno a stuj pred Faraonem. Hle vyjde k vode, a díš k nemu: Takto praví Hospodin: Propust lid muj, at mi slouží.
\par 21 Pakli nepropustíš lidu mého, hle, já pošli na te a na služebníky tvé, a na lid tvuj, a na domy tvé smesici všelikých škodlivých žížal; a naplneni budou domové Egyptští temi žížalami, nad to i zeme ta, na níž oni jsou.
\par 22 A oddelím v ten den zemi Gesen, v níž lid muj zustává, aby tam nebylo smesice té; abys vedel, že jsem já Hospodin u prostred zeme.
\par 23 A vysvobozením rozdíl uciním mezi lidem svým a lidem tvým. Zítra bude znamení toto.
\par 24 I ucinil Hospodin tak. Nebo prišla težká smesice škodlivých žížal na dum Faraonuv,a do domu služebníku jeho i na všecku zemi Egyptskou; a nakazila se zeme od té smesice.
\par 25 Povolal pak Farao Mojžíše a Arona, a rekl: Jdete, obetujte Bohu svému tu v zemi.
\par 26 I rekl Mojžíš: Nenáleží nám tak ciniti; nebo ohavnost Egyptských obetovali bychom Hospodinu Bohu našemu. A jestliže bychom obetovali to, což jest ohavnost pred ocima Egyptských, zdaž by nás neukamenovali?
\par 27 Cestou trí dnu pujdeme na poušt, a obetovati budeme Hospodinu Bohu našemu, jakž nám rozkázal.
\par 28 I rekl Farao: Já propustím vás, abyste obetovali Hospodinu Bohu svému na poušti, však dále abyste nikoli neodcházeli. Modltež se za mne.
\par 29 Odpovedel Mojžíš: Aj, já vycházím od tebe, a modliti se budu Hospodinu, aby odešla ta smesice od tebe, od služebníku tvých i od lidu tvého zítra; avšak at Farao více nezklamává, nepropoušteje lidu, aby obetovali Hospodinu.
\par 30 A vyšed Mojžíš od Faraona, modlil se Hospodinu.
\par 31 I ucinil Hospodin podlé slova Mojžíšova, a odjal tu smesici od Faraona, od služebníku jeho i od lidu jeho, tak že ani jedné žížaly nezustalo.
\par 32 Ale Farao ztížil srdce své také i tehdáž, a nepropustil lidu.

\chapter{9}

\par 1 Tedy rekl Hospodin Mojžíšovi: Vejdi k Faraonovi, a mluv k nemu: Takto praví Hospodin, Buh Hebrejský: Propust lid muj, at mi slouží.
\par 2 Pakli nebudeš chtíti propustiti, než predce držeti je budeš:
\par 3 Aj, ruka Hospodinova bude na dobytku tvém, kterýž jest na poli, na koních, na oslích, na velbloudích, na volích a na ovcech, mor težký velmi.
\par 4 A uciní Hospodin rozdíl mezi dobytky Izraelských a mezi dobytky Egyptských, aby nic neumrelo ze všeho, což jest synu Izraelských.
\par 5 A uložil Hospodin cas jistý, rka: Zítra uciní Hospodin vec takovou na zemi.
\par 6 I ucinil Hospodin tu vec na zejtrí, a pomrel všecken dobytek Egyptským; z dobytku pak synu Izraelských ani jedno neumrelo.
\par 7 I poslal Farao, a aj, neumrelo z dobytku Izraelských ani jedno. Ale obtíženo jest srdce Faraonovo, a nepropustil lidu.
\par 8 I rekl Hospodin Mojžíšovi a Aronovi: Vezmete sobe plné hrsti své popela z peci, a at jej sype Mojžíš k nebi pred ocima Faraonovýma.
\par 9 I obrátí se v prach po vší zemi Egyptské, a budou z neho na lidech i na hovadech vredové prýštící se neštovicemi po vší zemi Egyptské.
\par 10 Nabravše tedy popela z peci, stáli pred Faraonem, a sypal jej Mojžíš k nebi. I byli vredové plní neštovic, prýštící se na lidech i na hovadech.
\par 11 Aniž mohli carodejníci státi pred Mojžíšem pro vredy; nebo byli vredové na carodejnících i na všech Egyptských.
\par 12 I zsilil Hospodin srdce Faraonovo, a neposlechl jich, tak jakž byl mluvil Hospodin k Mojžíšovi.
\par 13 Tedy rekl Hospodin Mojžíšovi: Vstana ráno, postav se pred Faraonem, a rci k nemu: Takto praví Hospodin, Buh Hebrejský: Propust lid muj, at mi slouží.
\par 14 Nebo já ted již pošli všecky rány své na srdce tvé, i na služebníky tvé a na lid tvuj, abys vedel, žet není podobného mne na vší zemi.
\par 15 Nebo nyní, když jsem vztáhl ruku svou, byl bych tebe také ranil i lid tvuj morem tím; a tak bys byl vyhlazen z zeme.
\par 16 Ale však proto jsem te zachoval, abych ukázal na tobe moc svou, a aby vypravovali jméno mé na vší zemi.
\par 17 Ješte ty pozdvihuješ se proti lidu mému, nechteje ho propustiti?
\par 18 Aj, já dštíti budu zítra v tentýž cas krupobitím težkým náramne, jakéhož nebylo v Egypte od toho dne, jakž založen jest, až do tohoto casu.
\par 19 Protož nyní pošli, shromažd dobytek svuj a cokoli máš na poli. Na všecky lidi i hovada, kteráž by nalezena byla na poli, a nebyla by shromáždena do domu, spadne krupobití, a pomrou.
\par 20 Kdo tedy z služebníku Faraonových ulekl se slova Hospodinova, svolal hbite služebníky své i dobytek svuj do domu.
\par 21 Ale kdož nepriložil srdce svého k slovu Hospodinovu, nechal služebníku svých a dobytka svého na poli.
\par 22 I rekl Hospodin Mojžíšovi: Vztáhni ruku svou k nebi, at jest krupobití po vší zemi Egyptské, na lidi i na hovada i na všelikou bylinu polní v zemi Egyptské.
\par 23 Tedy vztáhl Mojžíš hul svou k nebi, a Hospodin vydal hrímání a krupobití. I sstoupil ohen na zem, a dštil Hospodin krupobitím na zemi Egyptskou.
\par 24 I bylo krupobití a ohen smíšený s krupobitím težký velmi, jakéhož nebylo nikdy ve vší zemi Egyptské, jakž v ní bydliti lidé zacali.
\par 25 I ztloukly kroupy po vší zemi Egyptské, cožkoli bylo na poli od cloveka až do hovada; všecku také bylinu polní potloukly kroupy, i všecko stromoví na poli zprerážely.
\par 26 Toliko v zemi Gesen, v níž byli synové Izraelští, nebylo krupobití.
\par 27 Poslav tedy Farao, povolal Mojžíše a Arona a rekl jim: Zhrešil jsem i nyní. Hospodint jest spravedlivý, ale já a lid muj bezbožní jsme.
\par 28 Modlte se Hospodinu, (nebo dosti jest), at není hrímání Božího a krupobití. Tedy propustím vás, aniž déle zustávati budete.
\par 29 I rekl jemu Mojžíš: Když vyjdu ven z mesta, rozprostru ruce své k Hospodinu, a hrímání prestane, i krupobití více nebude, abys poznal, že Hospodinova jest zeme.
\par 30 Ale vím, že ani ty, ani služebníci tvoji ješte se nebudete báti tvári Hospodina Boha.
\par 31 I potlucen jest len a jecmen; nebo jecmen se byl vymetal, len také byl v hlávkách.
\par 32 Ale pšenice a špalda nebyla ztlucena, nebo pozdní byla.
\par 33 Tedy Mojžíš vyšed od Faraona z mesta, rozprostrel ruce své k Hospodinu. I prestalo hrímání a krupobití, a ani déšt nelil se na zemi.
\par 34 Uzrev pak Farao, že prestal déšt a krupobití a hrímání, opet hrešil; a více obtížil srdce své, on i služebníci jeho.
\par 35 I zsililo se srdce Faraonovo, a nepropustil synu Izraelských, tak jakž byl mluvil Hospodin skrze Mojžíše.

\chapter{10}

\par 1 I rekl Hospodin Mojžíšovi: Vejdi k Faraonovi, ackoli jsem já obtížil srdce jeho, a srdce služebníku jeho, abych ucinil divy tyto své u prostred nich;
\par 2 A abys ty vypravoval v uši synu svých i vnuku svých, co jsem ucinil v Egypte, a znamení má, kteráž jsem prokázal na nich; abyste vedeli, že já jsem Hospodin.
\par 3 I všel Mojžíš s Aronem k Faraonovi, a rekli jemu: Takto praví Hospodin Buh Hebrejský: Dokavadž nechceš se ponížiti prede mnou? Propust lid muj, at mi slouží.
\par 4 Pakli nechceš propustiti lidu mého, aj, já uvedu zítra kobylky na krajinu tvou.
\par 5 A prikryjí svrchek zeme, aby jí nebylo videti, a snedí ostatky pozustalé, kteríž vám zanecháni jsou po krupobití; zhryzou vám také každý strom pucící se na poli.
\par 6 A naplní domy tvé, i domy všech služebníku tvých, a domy všech Egyptských; cehož nevideli otcové tvoji a otcové otcu tvých, od pocátku bytu svého na zemi až do dne tohoto. A odvrátiv se, vyšel od Faraona.
\par 7 Rekli pak služebníci Faraonovi k nemu: Dokavadž tento bude nám osídlem? Propust ty muže, at slouží Hospodinu Bohu svému. Zdaž ješte nevíš, že zkažen jest Egypt?
\par 8 I zavolán jest Mojžíš s Aronem pred Faraona. Jimž rekl: Jdete, služte Hospodinu Bohu svému. Kdo jsou ti, kteríž jíti mají?
\par 9 A odpovedel Mojžíš: S dítkami i s starými našimi pujdeme, s syny i s dcerami našimi, s ovcemi a s vetším dobytkem naším odejdeme; nebo slavnost Hospodinovu držeti máme.
\par 10 Tedy rekl jim: Nechat jest tak Hospodin s vámi, jako já propustím vás i dítky vaše. Hledte, nebo zlé jest pred tvári vaší.
\par 11 Nebudet tak. Jdete vy sami muži, a služte Hospodinu, nebo toho vy toliko žádáte. I vyhnáni jsou od tvári Faraonovy.
\par 12 Tedy rekl Hospodin Mojžíšovi: Vztáhni ruku svou na zemi Egyptskou pro kobylky, at vystoupí na zemi Egyptskou, a sežerou všelikou bylinu zeme té, cožkoli zustalo po krupobití.
\par 13 I vztáhl Mojžíš hul svou na zemi Egyptskou; a Hospodin uvedl vítr východní na zemi, aby vál celého toho dne a celou noc. A když bylo ráno, vítr východní prinesl kobylky.
\par 14 A vystoupily kobylky na všecku zemi Egyptskou, a pripadly na všecky konciny Egyptské nescíslne. Pred temi nebylo takových kobylek, aniž po tech takové budou.
\par 15 I prikryly veškeren svrchek zeme, tak že pro ne nebylo lze znáti zeme; a sežraly všelikou bylinu zeme, a všeliké ovoce na stromích, kteréž zustalo po krupobití; a nepozustalo nic zeleného na stromích a bylinách polních ve vší zemi Egyptské.
\par 16 Tedy Farao spešne povolav Mojžíše s Aronem, rekl: Zhrešil jsem proti Hospodinu Bohu vašemu, i proti vám.
\par 17 Ale nyní, odpust, prosím, hrích muj aspon tento, a modlte se Hospodinu Bohu vašemu, at jen tuto smrt odejme ode mne.
\par 18 Protož vyšed Mojžíš od Faraona, modlil se Hospodinu.
\par 19 I obrátil Hospodin vítr západní tuhý velmi, kterýžto zachvátiv kobylky, uvrhl je do more Rudého, tak že nezustalo žádné kobylky ve vší krajine Egyptské.
\par 20 Ale obtížil Hospodin srdce Faraonovo, a nepropustil synu Izraelských.
\par 21 I rekl Hospodin Mojžíšovi: Vztáhni ruku svou k nebi, a bude tma na zemi Egyptské, a makati ji budou.
\par 22 I vztáhl Mojžíš ruku svou k nebi, a byla tma prehustá po vší zemi Egyptské za tri dni.
\par 23 Aniž videl jeden druhého, a aniž kdo vstal z místa svého za tri dni; ale synové Izraelští všickni meli svetlo v príbytcích svých.
\par 24 Potom povolav Farao Mojžíše, rekl: Jdete, služte Hospodinu. Toliko ovce vaše a vetší dobytek váš nechat zustane, také dítky vaše pujdou s vámi.
\par 25 Odpovedel Mojžíš: Dáš také v ruce naše obeti a zápaly, kteréž bychom obetovali Hospodinu Bohu našemu.
\par 26 A protož také dobytek náš pujde s námi, a nezustane ani kopyta; nebo z nich vezmeme ku pocte Hospodinu Bohu našemu. My pak nevíme, cím sloužiti máme Hospodinu, dokudž neprijdeme tam.
\par 27 Zatvrdil pak Hospodin srdce Faraonovo, tak že nechtel propustiti jich.
\par 28 I rekl mu Farao: Odejdi ode mne, a varuj se, abys více nevidel tvári mé; nebo v který den uzríš tvár mou, umreš.
\par 29 Odpovedel Mojžíš: Dobre jsi rekl; neuzrímt více tvári tvé.

\chapter{11}

\par 1 Rekl pak byl Hospodin Mojžíšovi: Ješte ránu jednu uvedu na Faraona a na Egypt, potom propustí vás odsud; propustí docela, anobrž vypudí vás odsud.
\par 2 Mluv nyní v uši lidu, at vypujcí jeden každý od bližního svého, a každá od bližní své klínotu stríbrných a klínotu zlatých.
\par 3 A dal Hospodin milost lidu pred ocima Egyptských. (Sám také Mojžíš veliký byl velmi v zemi Egyptské, pred ocima služebníku Faraonových i pred ocima lidu.)
\par 4 I rekl Mojžíš: Takto praví Hospodin: O pulnoci já pujdu prostredkem Egypta.
\par 5 A pomre všecko prvorozené v zemi Egyptské, od prvorozeného Faraonova, jenž sedeti mel na stolici jeho, až do prvorozeného devky, kteráž jest pri žernovu, i všecko prvorozené hovad.
\par 6 I bude krik veliký po vší zemi Egyptské, jakéhož nebylo prvé, a jakéhož nikdy nebude více.
\par 7 U synu pak Izraelských nikdež nehne pes jazykem svým, ovšem pak ani clovek ani hovado, abyste vedeli, že rozdíl ucinil Hospodin mezi Egyptskými a Izraelskými.
\par 8 I sstoupí všickni tito služebníci tvoji ke mne, a skláneti mi se budou, rkouce: Vyjdi, ty i všecken lid, kterýž jest pod správou tvou; a potom vyjdu. A vyšel od Faraona s velikým hnevem.
\par 9 I rekl Hospodin Mojžíšovi: Neposlechnet vás Farao, abych rozmnožil zázraky své v zemi Egyptské.
\par 10 Ale Mojžíš a Aron cinili všecky ty zázraky pred Faraonem; Hospodin pak zatvrdil srdce Faraonovo, tak že nepropustil synu Izraelských z zeme své.

\chapter{12}

\par 1 Mluvil pak Hospodin k Mojžíšovi a k Aronovi v zemi Egyptské, rka:
\par 2 Tento mesíc pocátek mesícu vám bude; první vám bude mezi mesíci rocními.
\par 3 Mluvte ke všemu shromáždení Izraelskému, rkouce: Desátého dne mesíce tohoto vezmete sobe jeden každý beránka po celedech, beránka na každý dum.
\par 4 Byl-li by pak dum tak malý, že by s beránka býti nemohl, privezme souseda svého, kterýž jest blízký domu jeho, podlé poctu duší; jeden každý pocte tolik osob, kolikž by jich snísti mohlo beránka.
\par 5 Beránka bez vady, samce rocního míti budete, kteréhož z ovcí aneb z koz vezmete.
\par 6 A chovati ho budete až do ctrnáctého dne mesíce tohoto; a zabije ho všecko množství shromáždení Izraelského k vecerou.
\par 7 A vezmouce krve, pomaží obou verejí a nade dvermi u domu, v nichž jej jísti budou.
\par 8 I budou jísti noci té maso pecené ohnem, s chleby presnými; s bylinami horkými jísti jej budou.
\par 9 Nebudete jísti z neho nic surového ani v vode vareného, ale pecené ohnem, s hlavou jeho i s nohami a droby.
\par 10 Nezanecháte z neho nicehož do jitra; pakli by co pozustalo z neho až do jitra, ohnem spálíte.
\par 11 Takto jej pak jísti budete: Bedra svá prepásaná míti budete, obuv svou na nohách svých a hul svou v ruce své, a jísti budete s chvátáním; nebo Jití jest Hospodinovo.
\par 12 V tu noc zajisté pujdu po zemi Egyptské, a budu bíti všecko prvorozené v zemi Egyptské, od cloveka až do hovada, a nade všemi bohy Egyptskými uciním soud: Já Hospodin.
\par 13 Krev pak ta na domích, v nichž budete, budet vám na znamení; a když uzrím krev, pominu vás, a nebude mezi vámi rána zahubující, když bíti budu prvorozené v zemi Egyptské.
\par 14 A budet vám den ten na památku, a slaviti jej budete slavný Hospodinu po rodech svých; právem vecným slaviti jej budete.
\par 15 Za sedm dní presné chleby jísti budete, a hned prvního dne vyprázdníte kvas z domu vašich; nebo kdožkoli jedl by co kvašeného od prvního až do sedmého dne, vyhlazena bude duše ta z Izraele.
\par 16 A v den první budet shromáždení svaté; dne také sedmého shromáždení svaté míti budete. Žádného díla nebude deláno v nich; toliko cehož se užívá k jídlu od každého, to samo pripraveno bude od vás.
\par 17 A ostríhati budete presnic, nebo v ten den vyvedl jsem vojska vaše z zeme Egyptské; protož zachovávati budete den ten po rodech svých právem vecným.
\par 18 Prvního mesíce, ctrnáctého dne téhož mesíce, u vecer jísti budete chleby presné, až do dne jedenmecítmého téhož mesíce k vecerou.
\par 19 Za sedm dní nebude nalezeno kvasu v domích vašich; nebo kdo by koli jedl neco kvašeného, vyhlazena bude duše ta z shromáždení Izraelského, tak príchozí jako zrozený v zemi.
\par 20 Nic kvašeného jísti nebudete, ale ve všech príbytcích vašich jísti budete chleby presné.
\par 21 Tedy svolal Mojžíš všecky starší Izraelské, a rekl jim: Vyberte a vezmete sobe beránka po celedech svých, a zabíte Fáze.
\par 22 Vezmete také svazcek yzopu, a omocíte v krvi, kteráž bude v medenici, a pomažete nade dvermi a na obou verejích tou krví, kteráž bude v nádobe; z vás pak žádný nevycházej ze dverí domu svého až do jitra.
\par 23 Nebot pujde Hospodin, aby bil Egypt, a kde uzrí krev nade dvermi a na obou verejích, preskocí Hospodin ty dvére, aniž dopustí zhoubci vjíti do domu vašich k hubení.
\par 24 Protož ostríhati budete veci této za ustanovení tobe i synum tvým až na veky.
\par 25 A když vejdete do zeme, kterouž dá Hospodin vám, jakž zaslíbil, zachovávati budete službu tuto.
\par 26 Když by pak rekli vám synové vaši: Jaká jest to služba vaše?
\par 27 Tedy díte: Obet Fáze toto jest Hospodinu, kterýž pominul domu synu Izraelských v Egypte, když bil Egypt, domy pak naše vysvobodil. A lid sklonivše hlavy, poklonu ucinili.
\par 28 A rozšedše se synové Izraelští, ucinili, jakž byl Hospodin prikázal Mojžíšovi a Aronovi; tak a nejinak ucinili.
\par 29 Stalo se pak o pulnoci, pobil Hospodin všecko prvorozené v zemi Egyptské, od prvorozeného Faraonova, kterýž sedeti mel na stolici jeho, až do prvorozeného vezne, kterýž byl v žalári, i všecko prvorozené hovad.
\par 30 Tedy vstal Farao noci té, a všickni služebníci jeho i všickni Egyptští, a vzešel krik veliký v Egypte; nebo žádného nebylo domu, v nemž by nebylo neceho mrtvého.
\par 31 A povolav Mojžíše a Arona v noci, rekl: Vstante, vyjdete z prostredku lidu mého, i vy i synové Izraelští, a odejdouce, služte Hospodinu, jakž jste mluvili.
\par 32 Ovce také vaše i voly vaše vezmete, jakž jste žádali, a jdete; a dejte mi také požehnání.
\par 33 I nutkali Egyptští lid, aby co nejrychleji vyšli z zeme; nebo pravili: Všickni již ted zemreme.
\par 34 Protož vzal lid testo své, prvé než zkysalo, obaliv je v šaty své, na ramena svá.
\par 35 Ucinili pak synové Izraelští podlé rozkazu Mojžíšova; nebo vyžádali byli od Egyptských klínotu stríbrných a zlatých, i šatu.
\par 36 A Hospodin dal milost lidu pred ocima Egyptských, tak že pujcovali jim. I obloupili Egyptské.
\par 37 Tedy táhli synové Izraelští z Ramesses k Sochot, okolo šestkrát sto tisíc peších, mužu toliko krome detí.
\par 38 Ano také jiného lidu mnoho vyšlo s nimi, ovec také a volu, dobytka velmi mnoho.
\par 39 I napekli z testa, kteréž vynesli z Egypta, kolácu nekvašených; nebo ješte bylo nezkynulo, proto že vypuzeni byli z Egypta, a nemohli prodlévati, a ani pokrmu na cestu nepripravili sobe.
\par 40 Cas pak bydlení synu Izraelských, kteríž byli v Egypte, byl ctyri sta a tridceti let.
\par 41 A když se vyplnilo ctyri sta a tridceti let, práve toho dne vyšla všecka vojska Hospodinova z zeme Egyptské.
\par 42 Noc tato pilne ostríhána býti má Hospodinu, v níž vyvedl je z zeme Egyptské; tat tedy noc Hospodinova ostríhána bude ode všech synu Izraelských po národech jejich.
\par 43 I rekl Hospodin Mojžíšovi a Aronovi: Tentot bude rád pri slavnosti Fáze: Žádný cizozemec nebude jísti z neho.
\par 44 Každý pak služebník váš za stríbro koupený, když by obrezán byl, teprv jísti bude z neho.
\par 45 Príchozí a nájemník nebude jísti z neho.
\par 46 V témž dome jísti jej budeš, nevyneseš z domu ven masa jeho; a kostí v nem nezlámete.
\par 47 Všecko shromáždení Izraelské tak s ním uciní.
\par 48 Jestliže by pak cizozemec bydlil s tebou pohostinu, a slaviti by chtel Fáze Hospodinu, prvé obrezán bude každý pohlaví mužského; a tehdy pristoupí k slavení jeho, a bude jako tu v zemi zrozený; žádný pak neobrezaný nebude jísti z neho.
\par 49 Jednostejné právo bude tu zrodilému a príchozímu, kterýž jest pohostinu u prostred vás.
\par 50 Tedy ucinili všickni synové Izraelští, jakž prikázal Hospodin Mojžíšovi a Aronovi; tak ucinili.
\par 51 A tak stalo se práve toho dne, že vyvedl Hospodin syny Izraelské z zeme Egyptské s vojsky jejich.

\chapter{13}

\par 1 I mluvil Hospodin k Mojžíšovi, rka:
\par 2 Posvet mi všeho prvorozeného, cožkoli otvírá každý život mezi syny Izraelskými, tak z lidí jako z hovad, nebo mé jest.
\par 3 Protož rekl Mojžíš lidu: Pamatujte na den tento, v kterémž jste vyšli z Egypta, z domu služby; nebo v silné ruce vyvedl vás odsud Hospodin, aniž kdo jez co kvašeného.
\par 4 Dnes vycházíte vy, mesíce Abib.
\par 5 Když tedy uvede te Hospodin do zeme Kananejských, Hetejských, Amorejských, Hevejských a Jebuzejských, tak jakž prisáhl otcum tvým, a dá tobe zemi oplývající mlékem a strdí: tedy vykonávati budeš službu tuto v tento mesíc.
\par 6 Za sedm dní jísti budeš chleby presné, dne pak sedmého slavnost bude Hospodinova.
\par 7 Presní chlebové jedeni budou za dnu sedm, aniž spatríno bude u tebe co kvašeného, aniž se uhlédá u tebe kvas ve všech koncinách tvých.
\par 8 A vypravovati budeš synu svému v ten den, rka: Proto, což mi ucinil Hospodin, když jsem vycházel z Egypta.
\par 9 A budet tobe to jako nejaké znamení na ruce tvé, a jako památka pred ocima tvýma, aby zákon Hospodinuv byl v ústech tvých; nebo v ruce silné vyvedl te Hospodin z Egypta.
\par 10 Protož zachovávati budeš ustanovení toto v cas jistý, rok po roce.
\par 11 A když by te uvedl Hospodin do zeme Kananejských, tak jakž prisáhl tobe a otcum tvým, a dal by ji tobe:
\par 12 Tedy všecko, což otvírá život, oddelíš Hospodinu, i každý plod hovada tvého otvírající život, což by koli bylo samcu, Hospodinovo jest.
\par 13 Každé pak prvorozené osle vyplatíš hovádkem; pakli bys nevyplatil, zlom jemu šíji; každého také prvorozeného cloveka mezi syny svými vyplatíš.
\par 14 A když by se tebe vzeptal syn tvuj potom, a rekl: Co jest to? tedy povíš jemu: V ruce silné vyvedl nás Hospodin z Egypta, z domu služebnosti.
\par 15 Nebo když se byl zatvrdil Farao, a nechtel nás propustiti, pobil Hospodin všecko prvorozené v zemi Egyptské, od prvorozeného z lidí, až do prvorozeného z hovad; i tou prícinou já obetuji Hospodinu všecky samce otvírající život, ale všecko prvorozené z synu svých vyplacuji.
\par 16 Mejž to tedy jako znamení na ruce své, a jako nácelník mezi ocima svýma, že v ruce silné vyvedl nás Hospodin z Egypta.
\par 17 Stalo se pak, když pustil Farao lid, že nevedl jich Buh cestou zeme Filistinské, ackoli bližší byla; nebo rekl Buh: Aby nepykal lid, když by uzrel, an válka nastává, a nevrátili se do Egypta.
\par 18 Ale obvedl Buh lid cestou pres poušt, kteráž jest pri mori Rudém. A vojensky zporádaní vyšli synové Izraelští z zeme Egyptské.
\par 19 Vzal také Mojžíš kosti Jozefovy s sebou; nebo byl prísahou zavázal syny Izraelské, rka: Jistotne navštíví vás Buh, protož vyneste odsud kosti mé s sebou.
\par 20 Vytáhše tedy z Sochot, položili se v Etam pri kraji poušte.
\par 21 Hospodin pak predcházel je ve dne v sloupu oblakovém, aby je vedl cestou, v noci pak v sloupu ohnivém, aby svítil jim, aby ve dne i v noci jíti mohli.
\par 22 Neodjal sloupu oblakového ve dne, ani ohnivého sloupu v noci od tvári toho lidu.

\chapter{14}

\par 1 Mluvil pak Hospodin k Mojžíšovi, rka:
\par 2 Mluv k synum Izraelským, at navrátíce se, rozbijí stany pred Fiarot, mezi Magdalem a morem, proti Belsefon; naproti nemu rozbijete stany pri mori.
\par 3 A dí Farao o synech Izraelských: Ssouženi jsou na zemi, sevrela je poušt.
\par 4 I zatvrdím srdce Faraonovo, a honiti je bude, a oslaven budu v Faraonovi a ve všem vojsku jeho; a zvedí Egyptští, že já jsem Hospodin. I ucinili tak.
\par 5 Povedíno pak bylo králi Egyptskému, že by lid utíkal. I obráceno jest srdce Faraonovo a služebníku jeho proti lidu, a rekli: Co jsme to ucinili, že jsme propustili Izraele, aby nesloužil nám?
\par 6 Protož zapráhl do svého vozu, a lid svuj vzal s sebou.
\par 7 A vzal šest set vozu vybraných, i všecky vozy Egyptské, nad nimiž nade všemi byli hejtmané.
\par 8 I zatvrdil Hospodin srdce Faraona krále Egyptského, tak že honil syny Izraelské; synové pak Izraelští vyšli v ruce vyvýšené.
\par 9 I honili je Egyptští, a postihli je, když se položili pri mori, všickni vozové Faraonovi, a jezdci jeho i vojsko jeho podlé Fiarot, pred Belsefon.
\par 10 A když se priblížil Farao, pozdvihli synové Izraelští ocí svých, a aj, Egyptští táhnou za nimi. I báli se velmi, a volali synové Izraelští k Hospodinu.
\par 11 A rekli Mojžíšovi: Zdali proto, že nebylo hrobu v Egypte, vyvedl jsi nás, abychom zemreli na poušti? Co jsi nám to ucinil, že jsi vyvedl nás z Egypta?
\par 12 Zdali jsme toho nemluvili tobe ješte v Egypte, rkouce: Nech nás, at sloužíme Egyptským? Nebo lépe bylo nám sloužiti Egyptským, než zemríti na poušti.
\par 13 I rekl Mojžíš lidu: Nebojte se, stujte a vizte spasení Hospodinovo, kteréž vám zpusobí dnes; nebo Egyptských, kteréž jste videli dnes, neuzríte nikdy více až na veky.
\par 14 Hospodin bojovati bude za vás, a vy mlceti budete.
\par 15 I rekl Hospodin Mojžíšovi: Co voláš ke mne? Mluv synum Izraelským, at jdou predce.
\par 16 Ty pak zdvihni hul svou, a vztáhni ruku svou na more, a rozdel je; a nechat jdou synové Izraelští prostredkem more po suše.
\par 17 Ját pak, aj, já zatvrdím srdce Egyptských, a vejdou za nimi; i budu oslaven v Faraonovi, a ve všem vojsku jeho, v vozích jeho i v jezdcích jeho.
\par 18 A zvedí Egyptští, že já jsem Hospodin, když oslaven budu v Faraonovi, v vozích jeho a v jezdcích jeho.
\par 19 I bral se andel Boží, kterýž byl prvé predcházel vojsko Izraelské, a šel z zadu za nimi; nebo hnul se sloup oblakový, kterýž byl pred nimi, a stál z zadu za nimi.
\par 20 A prišed mezi vojska Egyptských a vojska Izraelská, byl Egyptským oblakem a tmou, Izraelským pak osvecoval noc, tak aby nepriblížili se jedni k druhým pres celou noc.
\par 21 I vztáhl Mojžíš ruku svou na more, a Hospodin rozdelil more vetrem východním prudce vejícím pres celou noc; a ucinil more v suchost, když se rozstoupily vody.
\par 22 Tedy šli synové Izraelští prostredkem more po suše, a vody jim byly jako zed po pravé i po levé strane.
\par 23 A honíce je Egyptští, vešli za nimi do prostred more, všecka jízda Faraonova, vozové i jízdní jeho.
\par 24 Stalo se pak v bdení jitrním, že pohledel Hospodin na vojska Egyptských z sloupu ohne a oblaku, a zmátl vojsko Egyptské.
\par 25 A odjal kola vozu jejich, aby je težce táhli. I rekli Egyptští: Utecme pred Izraelem, nebo Hospodin bojuje za ne proti Egyptským.
\par 26 Tedy rekl Hospodin Mojžíšovi: Vztáhni ruku svou na more, at se zase vrátí vody na Egyptské, na vozy jejich a na jezdce jejich.
\par 27 I vztáhl Mojžíš ruku svou na more, a navrátilo se more ráno k moci své, a Egyptští utíkali proti nemu; a vrazil Hospodin Egyptské do prostred more.
\par 28 A navrátivše se vody, zatopily vozy i jezdce se vším vojskem Faraonovým, což jich koli vešlo za nimi do more, tak že nezustal z nich ani jeden.
\par 29 Ale synové Izraelští šli po suchu prostredkem more, a vody jim byly místo zdi po pravé i po levé strane.
\par 30 A tak vysvobodil Hospodin v ten den Izraele z ruky Egyptských; a videl Izrael Egyptské mrtvé na brehu morském.
\par 31 Videl také Izrael moc velikou, kterouž prokázal Hospodin na Egyptských. I bál se lid Hospodina, a verili Hospodinu i Mojžíšovi, služebníku jeho.

\chapter{15}

\par 1 Tehdy zpíval Mojžíš a synové Izraelští písen tuto Hospodinu, a rekli takto: Zpívati budu Hospodinu, nebot jest slavne zveleben; kone i s jezdcem uvrhl do more.
\par 2 Síla má a písen jest Hospodin, nebo vysvobodil mne. Ont jest Buh muj silný, protož stánek vzdelám jemu; ont jest Buh otce mého, protož vyvyšovati ho budu.
\par 3 Hospodin jest udatný bojovník, Hospodin jméno jeho.
\par 4 Vozy Faraonovy i vojsko jeho uvrhl do more; a nejprednejší hejtmané jeho ztopeni jsou v mori Rudém.
\par 5 Propasti prikryly je; vpadli do hlubiny jako kámen.
\par 6 Pravice tvá, Hospodine, zvelebena jest v síle, pravice tvá, ó Hospodine, potrela neprítele.
\par 7 A ve mnohé vyvýšenosti své podvrátil jsi povstávající proti tobe; pustils hnev svuj, kterýžto sežral je jako strnište.
\par 8 A duchem chrípí tvých shromáždeny jsou vody, stály tekuté vody jako hromada, ssedly se propasti u prostred more.
\par 9 Rekl neprítel: Honiti budu, dohoním se, budu deliti loupeže, nasytí se jimi duše má, vytrhnu mec svuj, zahladí je ruka má.
\par 10 Povanul jsi vetrem svým, i prikrylo je more; pohlceni jsou jako olovo v prudkých vodách.
\par 11 Kdo podobný tobe mezi silnými, ó Hospodine? Kdo jest tak, jako ty, velebný v svatosti, hrozný v chvalách, cinící divy?
\par 12 Vztáhls pravici svou, i požrela je zeme.
\par 13 Zprovodíš v milosrdenství svém lid tento, kterýž jsi vykoupil; laskave povedeš jej v síle své k príbytku svatosti své.
\par 14 Uslyší lidé, bouriti se budou; bolest zachvátí obyvatele Filistinské.
\par 15 Tedy zkormoucena budou knížata Idumejská, silné Moábské podejme strach, rozplynou se všickni obyvatelé Kananejští.
\par 16 Pripadne na ne strach a lekání, pro velikost ramene tvého; mlceti budou jako kámen, dokudž neprejde lid tvuj, ó Hospodine, dokudž neprejde lid ten, kteréhožs sobe dobyl.
\par 17 Uvedeš je, a štípíš je na hore dedictví svého, na míste, kteréž jsi k príbytku svému pripravil, Hospodine, v svatyni, kterouž utvrdí ruce tvé, Pane.
\par 18 Hospodin kralovati bude na veky veku.
\par 19 Nebo vešli koni Faraonovi s vozy jeho i s jezdci jeho do more, a obrátil na ne Hospodin vody morské, synové pak Izraelští šli po suše u prostred more.
\par 20 I vzala Maria prorokyne, sestra Aronova, buben v ruku svou, a vyšly za ní všecky ženy s bubny a s píštalami.
\par 21 I odpovídala jim Maria: Zpívejte Hospodinu, ponevadž slavne zveleben jest; kone i s jezdcem uvrhl do more.
\par 22 Hnul pak Mojžíš lidem Izraelským od more Rudého, a táhli na poušt Sur. I šli tri dni po poušti, a nenalezli vod.
\par 23 A prišedše do Marah, nemohli píti vod z Marah, nebo byly horké; protož nazváno jest jméno jeho Marah.
\par 24 Z té príciny reptal lid na Mojžíše, rka: Co budeme píti?
\par 25 I volal k Hospodinu, a ukázal mu Hospodin drevo, kteréž jakž uvrhl do vod, ucineny jsou sladké vody. Tu vydal jemu práva a soudy, a tu ho zkusil.
\par 26 A rekl: Jestliže skutecne poslouchati budeš hlasu Hospodina Boha svého, a ciniti budeš, což spravedlivého jest pred ocima jeho, a nakloníš uší k prikázaním jeho, a ostríhati budeš všech ustavení jeho: žádné nemoci, kterouž jsem dopustil na Egypt, nedopustím na tebe; nebo já jsem Hospodin, kterýž te uzdravuji.
\par 27 I prišli do Elim, kdež bylo dvanácte studnic vod a sedmdesáte palm; i rozbili tu stany pri vodách.

\chapter{16}

\par 1 Když se pak hnuli z Elim, prišlo všecko množství synu Izraelských na poušt Sin, kteráž jest mezi Elim a Sinai, v patnáctý den druhého mesíce po vyjití z zeme Egyptské.
\par 2 I reptalo všecko shromáždení synu Izraelských proti Mojžíšovi a proti Aronovi na poušti.
\par 3 A mluvili jim synové Izraelští: Ó bychom byli zemreli od ruky Hospodinovy v zemi Egyptské, když jsme sedávali nad hrnci masa, když jsme se najídali chleba do sytosti! A ted vyvedli jste nás na tuto poušt, abyste zmorili všecko shromáždení toto hladem.
\par 4 I rekl Hospodin Mojžíšovi: Aj, já dám vám chleba s nebe jako déšt, a vycházeti bude lid a sbírati, což by postacilo na každý den, abych ho zkusil, bude-li choditi v zákone mém, ci nebude.
\par 5 V den pak šestý pristrojí sobe to, co prinesou; a bude toho dvakrát více, než toho, což sbírati mají na každý den.
\par 6 Tedy mluvil Mojžíš a Aron všechnem synum Izraelským: U vecer poznáte, že Hospodin vyvedl vás z zeme Egyptské.
\par 7 A ráno uzríte slávu Hospodinovu; nebot jest slyšel reptání vaše proti Hospodinu. My zajisté co jsme, že repcete proti nám?
\par 8 Mluvil dále Mojžíš: Z toho, pravím, poznáte, když vám dá Hospodin u vecer masa, abyste se najedli, a chleba ráno do sytosti, ponevadž slyšel Hospodin reptání vaše, jimiž jste na nej reptali. Nebo my co jsme? Ne proti námt jsou reptání vaše, ale proti Hospodinu.
\par 9 I rekl Mojžíš Aronovi: Mluv ke všemu shromáždení synu Izraelských: Pristuptež pred oblícej Hospodinuv; nebot jest slyšel reptání vaše.
\par 10 Stalo se pak, když mluvil Aron ke všemu shromáždení synu Izraelských, že se obrátili tvárí k poušti, a aj, sláva Hospodinova ukázala se v oblaku.
\par 11 (A již byl mluvil Hospodin k Mojžíšovi, rka:
\par 12 Slyšelt jsem reptání synu Izraelských. Mluviž jim a povez: K vecerou jísti budete maso, a ráno chlebem nasyceni budete, abyste poznali, že já jsem Hospodin Buh váš.)
\par 13 Tedy stalo se u vecer, že priletely krepelky a prikryly tábor; ráno pak spadla rosa okolo táboru.
\par 14 A když prestalo padání rosy, aj, ukázalo se po vrchu poušte drobného cosi a okrouhlého, drobného jako jíní na zemi.
\par 15 Což vidouce synové Izraelští, rekli jeden druhému: Man jest toto. Nebo nevedeli, co by bylo. Tedy rekl jim Mojžíš: To jest ten chléb, kterýž vám dal Hospodin ku pokrmu.
\par 16 To jest, o cemž prikázal Hospodin: Nasbírejte sobe toho každý k svému pokrmu; gomer na jednoho cloveka vedlé poctu osob vašich, každý na ty, kteríž jsou v stanu jeho, vezmete.
\par 17 I ucinili tak synové Izraelští, a nasbírali jiní více, jiní méne.
\par 18 Potom merili na gomer. A nezbylo tomu, kdo nasbíral mnoho, a ten, kdo nasbíral málo, nemel nedostatku; ale každý, což mohl snísti, nasbíral.
\par 19 I rekl jim Mojžíš: Žádný at nic z toho nepozustavuje k jitru.
\par 20 Ale neuposlechli Mojžíše. Nebo nekterí zanechali díl z toho až do jitra; i zcervivelo a zsmradilo se. Procež rozhneval se na ne Mojžíš.
\par 21 Tak tedy sbírali to každého jitra, každý což snísti mohl. A když horké bylo slunce, tedy se ta manna rozpouštela.
\par 22 Když pak bylo v den šestý, nasbírali toho chleba dvojnásobne, po dvou gomer na každého; protož prišla všecka knížata toho shromáždení, a povedeli Mojžíšovi.
\par 23 Kterýžto rekl jim: Tot jest, což mluvil Hospodin: Odpocinutí soboty svaté Hospodinu bude zítra. Což byste meli péci, pecte, a což byste variti meli, varte dnes; což pak koli zbude, nechte sobe a schovejte to k jitru.
\par 24 Protož schovali to do rána, jakž prikázal Mojžíš; a nezsmradilo se, ani v nem cervu nebylo.
\par 25 I rekl Mojžíš: Jeztež to dnes, ponevadž sobota jest dnes Hospodinu; dnes toho nenaleznete na poli.
\par 26 Po šest dní budete to sbírávati, den pak sedmý sobota jest; nebude bývati manny v ní.
\par 27 Stalo se pak dne sedmého, že vyšli nekterí z lidu sbírat, a nenašli.
\par 28 Tedy rekl Hospodin Mojžíšovi: I dokudž zpecovati se budete prikázaní mých ostríhati a zákonu mých?
\par 29 Viztež, žet Hospodin vám dal sobotu, a proto on vám dává v den šestý chleba na dva dni. Zustante každý v svém, aniž kdo vycházej z místa svého v den sedmý.
\par 30 I odpocinul lid v den sedmý.
\par 31 Nazval pak lid Izraelský jméno toho chleba man; kterýž byl jako síme koliandrové, a bílý, a chut jeho jako koláce s medem.
\par 32 Rekl také Mojžíš: Tot jest, což prikázal Hospodin: Napln gomer tou mannou, aby chována byla na budoucí veky vaše, aby videli pokrm, kterýž jsem vám dával jísti na poušti, když jsem vás vyvedl z zeme Egyptské.
\par 33 I rekl Mojžíš Aronovi: Vezmi jedno vederce, a vsyp do neho plné gomer manny; a nech jí pred tvárí Hospodinovou, aby chována byla na budoucí veky vaše.
\par 34 A protož jakž byl prikázal Hospodin Mojžíšovi, nechal jí Aron pred svedectvím, aby tu chována byla.
\par 35 Jedli pak synové Izraelští mannu za ctyridceti let, dokudž nevešli do zeme, v níž bydliti meli; mannu jedli, dokudž neprišli k koncinám zeme Kananejské.
\par 36 Gomer pak jest desátý díl efi.

\chapter{17}

\par 1 Tedy když se odebralo všecko shromáždení synu Izraelských z poušte Sin, po stanovištích svých vedlé rozkazu Hospodinova, položili se v Rafidim, kdež nebylo vod, kteréž by lid píti mohl.
\par 2 Protož domlouval se lid na Mojžíše, prave: Dejte nám vody, abychom pili. Jimž odpovedel Mojžíš: Proc se na mne domlouváte? Proc pokoušíte Hospodina?
\par 3 I žíznil tu lid pro nedostatek vod, a reptal na Mojžíše a mluvil: Proc jsi vyvedl nás z Egypta, abys mne s syny i dobytky mými žízní zmoril?
\par 4 Tedy volal Mojžíš k Hospodinu, rka: Což mám ciniti s tím lidem? Však již tudíž ukamenují mne.
\par 5 I rekl Hospodin Mojžíšovi: Jdiž pred lidem, pojma s sebou nekteré z starších Izraelských; hul také svou kterouž jsi uderil v vodu, vezmi do ruky své a jdi.
\par 6 Aj, já státi budu pred tebou tam na skále, na Orébe; i uderíš v skálu, a vyjdou z ní vody, kteréž bude píti lid. I ucinil tak Mojžíš pred ocima starších Izraelských.
\par 7 A dal jméno místu tomu Massah a Meribah, pro reptání synu Izraelských, a že pokoušeli Hospodina, rkouce: Jest-li Hospodin u prostred nás, ci není?
\par 8 Pritáhl pak Amalech, a bojoval s Izraelem v Rafidim.
\par 9 I rekl Mojžíš k Jozue: Vyber nám nekteré muže, a vytáhna, bojuj s Amalechem; já zítra státi budu na vrchu hory, a hul Boží v ruce své míti budu.
\par 10 Tedy Jozue udelal tak, jakž mu porucil Mojžíš, a bojoval s Amalechem; Mojžíš pak, Aron a Hur vstoupili na vrch hory.
\par 11 A dokudž Mojžíš vzhuru držel ruce své, vítezil Izrael; ale jakž opouštel ruku svou, premáhal Amalech.
\par 12 Ale že ruce Mojžíšovy byly obtíženy, protož vzavše kámen, podložili pod neho, a on sedl na nem; Aron pak a Hur podpírali ruce jeho, jeden z jedné, druhý z druhé strany. I byly obe ruce jeho vztažené až do západu slunce.
\par 13 A tak porazil Jozue Amalecha i lid jeho mecem.
\par 14 Mluvil potom Hospodin k Mojžíšovi: Vpiš to do knih na památku, a pilne to vkládej v uši Jozue, že do konce vyhladím památku Amalechovu všudy pod nebem.
\par 15 Tedy vzdelal Mojžíš oltár a nazval jméno jeho: Hospodin korouhev má.
\par 16 Nebo rekl: Tak má jmenován býti, proto že ruka nad trunem Hospodinovým osvedcuje boj Hospodinuv proti Amalechovi od národu až do národu.

\chapter{18}

\par 1 Uslyšel pak Jetro, kníže Madianské, test Mojžíšuv, o všech vecech, kteréž ucinil Buh Mojžíšovi a Izraelovi, lidu svému, že vyvedl Hospodin Izraele z Egypta.
\par 2 A vzal Jetro, test Mojžíšuv, Zeforu manželku Mojžíšovu, kterouž byl odeslal,
\par 3 A dva syny její, z nichž jméno jednoho Gerson; nebo rekl: Príchozí jsem byl v zemi cizí;
\par 4 Jméno pak druhého Eliezer; nebo rekl: Buh otce mého spomocník muj byl, a vytrhl mne od mece Faraonova.
\par 5 I prišel Jetro, test Mojžíšuv, s syny jeho i s ženou jeho k Mojžíšovi na poušt, kdež se byl položil pri hore Boží.
\par 6 A vzkázal Mojžíšovi: Já, test tvuj Jetro, jdu k tobe, i žena tvá a oba synové její s ní.
\par 7 I vyšel Mojžíš proti testi svému, a pokloniv se, políbil ho. I ptal se jeden druhého, jak se má; potom vešli do stanu.
\par 8 A vypravoval Mojžíš testi svému všecko, což ucinil Hospodin Faraonovi a Egyptským pro Izraele, a o všech nevolech, kteréž pricházely na ne na ceste, a jak je vysvobodil Hospodin.
\par 9 I radoval se Jetro ze všeho dobrého, což ucinil Hospodin Izraelovi, a že vytrhl jej z ruky Egyptských.
\par 10 A rekl Jetro: Požehnaný Hospodin, kterýž vytrhl vás z ruky Egyptských a z ruky Faraonovy, kterýž vytrhl ten lid z poroby Egyptské.
\par 11 Nyní jsem poznal, že vetší jest Hospodin nade všecky bohy; nebo touž vecí, kterouž se vyvyšovali, on je prevýšil.
\par 12 A vzal Jetro, test Mojžíšuv, zápal a obeti, kteréž obetoval Bohu. Potom prišel Aron a všickni starší Izraelští, aby jedli chléb s tchánem Mojžíšovým pred Bohem.
\par 13 Nazejtrí pak posadil se Mojžíš, aby soudil lid; a stál lid pred Mojžíšem od jitra až do vecera.
\par 14 Vida pak test Mojžíšuv všecku práci jeho pri lidu, rekl: Co jest to, což deláš s lidem? Proc ty sám sedíš, a všecken lid stojí pred tebou od jitra až do vecera?
\par 15 Odpovedel Mojžíš tchánu svému: Prichází ke mne lid raditi se s Bohem.
\par 16 Když mají o nejakou vec ciniti, pricházejí ke mne, a soud ciním mezi stranami, a oznamuji rady Boží a ustanovení jeho.
\par 17 I rekl jemu test Mojžíšuv: Nedobre deláš.
\par 18 Tudíž tak ustaneš i ty i lid, kterýž s tebou jest. Nad možnost tvou težká jest tato vec, nebudeš jí moci sám dosti uciniti.
\par 19 Protož nyní uposlechni reci mé; poradím tobe, a bude Buh s tebou. Stuj ty za lid pred Bohem, a donášej veci nesnadné k Bohu.
\par 20 A vysvetluj jim rády a zákony, a oznamuj jim cestu, po níž by šli, a co by delati meli.
\par 21 Vyhledej také ze všeho lidu muže statecné, bohabojné, muže pravdomluvné, kteríž v nenávisti mají lakomství, a ustanov z nich knížata nad tisíci, setníky, padesátníky a desátníky.
\par 22 Oni at soudí lid každého casu. Což bude vetšího, vznesou na tebe, a menší pre sami necht soudí; a tak lehceji bude tobe, když jiní ponesou bríme s tebou.
\par 23 Jestliže to uciníš a rozkážet Buh, budeš moci trvati; také i všecken lid tento navracovati se bude k místum svým pokojne.
\par 24 Tedy uposlechl Mojžíš reci tchána svého, a ucinil všecko, což on rekl.
\par 25 A vybral Mojžíš muže statecné ze všeho Izraele, a ustanovil je hejtmany nad lidem, knížata nad tisíci, setníky, padesátníky a desátníky,
\par 26 Kteríž soudili lid každého casu. Nesnadnejší veci vznášeli na Mojžíše, všecky pak menší pre sami soudili.
\par 27 Potom propustil Mojžíš tchána svého; i odšel do zeme své.

\chapter{19}

\par 1 Mesíce tretího po vyjití synu Izraelských z zeme Egyptské, v ten den prišli na poušt Sinai.
\par 2 Nebo hnuvše se z Rafidim, prišli až na poušt Sinai a položili se na té poušti; a tu rozbili Izraelští stany naproti hore.
\par 3 Mojžíš pak vstoupil k Bohu. A mluvil hlasem k nemu Hospodin s té hory, rka: Takto díš domu Jákobovu, a oznámíš synum Izraelským:
\par 4 Sami jste videli, co jsem ucinil Egyptským, a jak jsem vás nesl na krídlách orlicích, a privedl jsem vás k sobe.
\par 5 Protož nyní, jestliže skutecne poslouchati budete hlasu mého, a ostríhati smlouvy mé, budete mi lid zvláštní mimo všecky lidi, ackoli má jest všecka zeme.
\par 6 A vy budete mi království knežské a národ svatý. Tat jsou slova, kteráž mluviti budeš synum Izraelským.
\par 7 Protož prišel Mojžíš a svolav starší lidu, predložil jim všecka slova ta, kteráž mu prikázal Hospodin.
\par 8 Odpovedel pak všecken lid spolecne, a rekl: Cožkoli mluvil Hospodin, budeme ciniti. A oznámil zas Mojžíš Hospodinu slova lidu.
\par 9 I rekl Hospodin Mojžíšovi: Aj, já pujdu k tobe v hustém oblaku, aby slyšel lid, když mluviti budu s tebou, ano také, aby tobe veril na veky. Nebo byl oznámil Mojžíš Hospodinu slova lidu.
\par 10 Rekl dále Hospodin Mojžíšovi: Jdi k lidu, a posvet jich dnes a zítra; a necht svá roucha zeperou.
\par 11 A at jsou hotovi ke dni tretímu, nebo v den tretí sstoupí Hospodin pred ocima všeho lidu na horu Sinai.
\par 12 Uložíš pak lidu meze všudy vukol a povíš: Varujte se, abyste nevstupovali na horu, ani nedotýkali se kraju jejích. Kdož by se koli dotkl hory, smrtí umre.
\par 13 Nedotknet se ho ruka, ale ukamenován neb zastrelen bude; bud že by hovado bylo, bud clovek, nebudet živ. Když se zdlouha troubiti bude, teprv oni vstoupí na horu.
\par 14 Sstoupiv tedy Mojžíš s hury k lidu, posvetil ho; a oni zeprali roucha svá.
\par 15 I mluvil k lidu: Budtež hotovi ke dni tretímu; nepristupujte k manželkám svým.
\par 16 I stalo se dne tretího, když bylo ráno, že bylo hrímání s blýskáním a oblak hustý na té hore, zvuk také trouby velmi tuhý, až se zhrozil všecken lid, kterýž byl v ležení.
\par 17 Tedy Mojžíš vyvedl lid z ležení vstríc Bohu; a lid stál dole pod horou.
\par 18 Hora pak Sinai všecka se kourila, proto že sstoupil na ni Hospodin v ohni, a vystupoval dým její jako dým z vápenice, a trásla se všecka hora velmi hrube.
\par 19 Zvuk také trouby více se rozmáhal, a silil se náramne. Mojžíš mluvil, a Buh mu odpovídal hlasem.
\par 20 Sstoupil pak Hospodin na horu Sinai, na vrch hory; a když povolal Hospodin Mojžíše na vrch hory, vstoupil Mojžíš.
\par 21 I rekl Hospodin Mojžíšovi: Sstup, osvedc lidu, at se nevytrhují k Hospodinu, chtejíce ho videti, aby nepadlo jich množství;
\par 22 Nýbrž ani sami kneží, kteríž, majíce pristupovati k Hospodinu, posvecují se, aby se neoboril na ne Hospodin.
\par 23 Mojžíš pak rekl Hospodinu: Nebudet moci lid vstoupiti na horu Sinai, ponevadž jsi ty osvedcil nám, rka: Obmez horu a posvet ji.
\par 24 I rekl jemu Hospodin: Jdi, sstup, a potom vstup ty a Aron s tebou. Kneží pak a lid at se nepokoušejí vstoupiti k Hospodinu, aby se na ne neoboril.
\par 25 I sešel Mojžíš k lidu, a to jim oznámil.

\chapter{20}

\par 1 I mluvil Buh všecka slova tato, rka:
\par 2 Já jsem Hospodin Buh tvuj, kterýž jsem te vyvedl z zeme Egyptské, z domu služby.
\par 3 Nebudeš míti bohu jiných prede mnou.
\par 4 Neuciníš sobe rytiny, ani jakého podobenství tech vecí, kteréž jsou na nebi svrchu, ani tech, kteréž na zemi dole, ani tech, kteréž u vodách pod zemí.
\par 5 Nebudeš se jim klaneti, ani jich ctíti. Nebo já jsem Hospodin Buh tvuj, Buh silný, horlivý, navštevující nepravost otcu na synech do tretího i ctvrtého pokolení tech, kteríž nenávidí mne,
\par 6 A cinící milosrdenství nad tisíci temi, kteríž mne milují, a ostríhají prikázaní mých.
\par 7 Nevezmeš jména Hospodina Boha svého nadarmo; nebot nenechá bez pomsty Hospodin toho, kdož by bral jméno jeho nadarmo.
\par 8 Pomni na den sobotní, abys jej svetil.
\par 9 Šest dní pracovati budeš, a delati všeliké dílo své;
\par 10 Ale dne sedmého odpocinutí jest Hospodina Boha tvého. Nebudeš delati žádného díla, ty i syn tvuj i dcera tvá, služebník tvuj i devka tvá, hovado tvé i príchozí, kterýž jest v branách tvých.
\par 11 Nebo v šesti dnech ucinil Hospodin nebe a zemi, more a všecko, což v nich jest, a odpocinul dne sedmého; protož požehnal Hospodin dne sobotního, a posvetil ho.
\par 12 Cti otce svého i matku svou, at se prodlejí dnové tvoji na zemi, kterouž Hospodin Buh tvuj dá tobe.
\par 13 Nezabiješ.
\par 14 Nesesmilníš.
\par 15 Nepokradeš.
\par 16 Nepromluvíš proti bližnímu svému krivého svedectví.
\par 17 Nepožádáš domu bližního svého, aniž požádáš manželky bližního svého, ani služebníka jeho, ani devky jeho, ani vola jeho, ani osla jeho, ani cožkoli jest bližního tvého.
\par 18 Veškeren pak lid videl hrímání to a blýskání, a zvuk trouby, a horu kourící se. To když videl lid, pohnuli se a stáli zdaleka.
\par 19 A rekli Mojžíšovi: Mluv ty s námi, a poslouchati budeme; a necht nemluví s námi Buh, abychom nezemreli.
\par 20 Odpovedel Mojžíš lidu: Nebojte se; nebo pro zkušení vás sám Buh prišel, aby bázen jeho byla mezi vámi, abyste nehrešili.
\par 21 Tedy stál lid zdaleka; Mojžíš pak pristoupil k mrákote, kdež byl Buh.
\par 22 I rekl Hospodin Mojžíšovi: Tak povíš synum Izraelským: Vy jste sami videli, že s nebe mluvil jsem s vámi.
\par 23 Protož nevyzdvihujte nicehož ku pocte se mnou; bohu stríbrných a bohu zlatých neuciníte sobe.
\par 24 Oltár z zeme udeláš mi a obetovati budeš na nem zápaly své, a pokojné obeti své, ovce své a voly své. Na kterémkoli míste rozkáži slaviti památku jména svého, prijdu k tobe a požehnám tobe.
\par 25 Jestliže mi pak vzdeláš oltár kamenný, nedelej ho z kamene tesaného; nebo jestliže pozdvihneš železa na nej, poškvrníš ho.
\par 26 Aniž po stupních vstupovati budeš k oltári mému, aby hanba tvá u neho odkryta nebyla.

\chapter{21}

\par 1 Tito jsou pak soudové, kteréž jim predložíš:
\par 2 Jestliže koupíš k službe Žida, šest let sloužiti bude, a sedmého odejde svobodný darmo.
\par 3 Prišel-li by sám toliko, sám také odejde; pakli mel ženu, vyjde s ním i žena jeho.
\par 4 Jestliže pán jeho dá mu ženu, a ona zrodí jemu syny neb dcery: žena ta i deti její budou pána jeho, on pak sám toliko odejde.
\par 5 Pakli by rekl služebník: Miluji pána svého, manželku svou a syny své, nevyjdu svobodný:
\par 6 Tedy postaví ho pán jeho pred soudci, a privede ho ke dverím neb k vereji,a probodne pán jeho ucho jemu špicí; i zustanet služebníkem jeho na veky.
\par 7 Když by pak prodal nekdo dceru svou, aby byla devkou, nevyjdet tak, jako vycházejí služebníci.
\par 8 Nelíbila-li by se pánu svému, kterýž jí sobe ješte nezasnoubil, dopustí ji vyplatiti. Lidu cizímu nebude míti práva prodati ji, ponevadž zhrešil proti ní.
\par 9 Pakli by synu svému ji zasnoubil, ucinít jí tak, jakž obycej jest ciniti dcerám.
\par 10 A dal-li by mu jinou, z stravy její, odevu jejího, a prívetivosti manželské nic této neujme.
\par 11 Neudelal-li by nic z toho trojího, vyjde darmo bez stríbra.
\par 12 Kdo by ubil cloveka, až by od toho umrel, smrtí umre.
\par 13 Když by pak neukládal o bezživotí jeho, než Buh dal by jej v ruce jeho: tedy uložím tobe místo, do nehož by takový mohl uteci.
\par 14 Pakli by kdo tak pyšne sobe pocínal proti bližnímu svému, že by ho lstive zabil, i od oltáre mého odtrhneš jej, aby umrel.
\par 15 Kdo by otce svého neb matku svou bil, smrtí at umre.
\par 16 Kdo by pak, ukradna nekoho, prodal jej, a nalezen by byl v ruce jeho, smrtí at umre.
\par 17 I ten, kdož by zlorecil otci svému neb materi své, smrtí at umre.
\par 18 Když by se svadili muži, a urazil by který bližního svého kamenem neb pestí, a ten by neumrel, než složil se na luži;
\par 19 A potom by povstal a chodil vne o holi své: již nebude vinen ten, kdož urazil; toliko co zatím obmeškal, to jemu nahradí, a na vyhojení jeho naloží.
\par 20 Když by pak ubil kdo služebníka svého neb devku svou kyjem, tak že by umrel mu v ruce jeho: pomstou pomšteno bude nad takovým.
\par 21 A však jestliže by den neb dva preckal, neponese pomsty, nebo jej zaplatil.
\par 22 Když by se svadili muži, a urazili ženu tehotnou, tak že by vyšel z ní plod její, však by se zhouba nestala: pokutován bude, jakž by uložil nan muž té ženy, a dá vedlé uznání soudcu.
\par 23 Paklit by smrt prišla, tedy dáš život za život,
\par 24 Oko za oko, zub za zub, ruku za ruku, nohu za nohu,
\par 25 Spáleninu za spáleninu, ránu za ránu, modrinu za modrinu.
\par 26 Jestliže by kdo urazil služebníka svého v oko, aneb devku svou v oko, tak že by jej o ne pripravil: svobodného jej propustí za oko jeho.
\par 27 Pakli by zub služebníku svému neb zub devce své vyrazil, svobodného jej propustí za zub jeho.
\par 28 Jestliže by vul utrkl muže neb ženu, tak že by umrel clovek: ukamenován bude ten vul, aniž jedeno bude maso jeho, však pán vola toho bez viny bude.
\par 29 Než byl-lit by vul trkavý prvé, a bylo by to osvedceno pánu jeho, on pak nezavrel by ho, a v tom zabil by muže neb ženu: vul ten ukamenován bude, a pán jeho také umre.
\par 30 Paklit mu bude uloženo, aby se vyplatil: tedy dá výplatu za život svuj, jakážkoli na nej uložena bude.
\par 31 Bud že by syna utrkl, bud dceru, podlé soudu toho stane se jemu.
\par 32 Jestliže by služebníka vul ztrkal neb devku, tridceti lotu stríbra dá pánu jeho, a vul ten bude ukamenován.
\par 33 Kdyby kdo odhradil studnici, a neb vykopal nekdo studnici, a zase jí neprikryl, a vpadl by tam vul neb osel:
\par 34 Pán té studnice nahradí to, a peníze položí pánu jeho, a což se zabilo, to sobe míti bude.
\par 35 A ustrcil-li by vul necí vola sousedova, že byl umrel: tedy prodadí vola toho živého, a podelí se penezi jeho; i s zabitým volem také se rozdelí.
\par 36 Pakli vedíno bylo, že vul byl trkavý prvé, a nezavrel ho pán jeho: bez výmluvy at dá vola za vola, a zabitý at mu zustane.

\chapter{22}

\par 1 Jestliže by kdo ukradl vola aneb dobytce, a zabil by je neb prodal: pet volu navrátí za toho vola, a ctvero dobytcat za to dobytce.
\par 2 (Jestliže by zlodej zastižen byl pri podkopávání a ubit jsa, umrel by: ten, kdo ho ranil, nebude vinen smrtí.
\par 3 Pakli by to ve dne ucinil, smrtí vinen bude.) Bez prodlévání at navrátí; pakli nemá co, prodán bude pro zlodejství své.
\par 4 Jestliže nalezena bude v rukou jeho krádež, bud vul, neb osel, bud dobytce ješte živé, dvénásobne navrátí.
\par 5 Jestliže by kdo spásl pole neb vinici, a vpustil hovado své, aby se páslo na cizím poli: což nejlepšího má na poli svém neb vinici své, tím tu škodu nahradí.
\par 6 Vyšel-li by ohen, a chytilo by se trní, a shorel by stoh neb stojaté obilí neb pole: nahradí ten, kdož zapálil, to, což shorelo.
\par 7 Kdyby nekdo dal schovati bližnímu svému peníze neb nádoby, a bylo by ukradeno z domu muže toho, jestliže nalezen bude zlodej, dvojnásobne navrátí.
\par 8 Pakli nebude zlodej nalezen, tedy postaven bude pán domu toho pred soudce, a prisáhne, že nevztáhl ruky své na vec bližního svého.
\par 9 O všelijakou vec, o niž by byla nesnáz, bud o vola neb osla, dobytce neb roucho, pro všelikou vec ztracenou, když by kdo pravil, že to jest: pred soudce prijde pre obou dvou; ten, kohož oni vinného usoudí, dvojnásobne navrátí bližnímu svému.
\par 10 Jestliže by kdo dal bližnímu svému k chování osla neb vola, neb dobytce a jakékoli hovado, a umrelo by neb ochromelo, neb zajato bylo, že žádný nevidel:
\par 11 Prísaha Hospodinova vkrocí mezi oba, že nevztáhl ruky své k veci bližního svého; a prijme jej v tom pán veci té, a onen nebude povinen navraceti.
\par 12 Pakli by krádeží vzato bylo od neho, navrátiti zase má pánu jeho.
\par 13 Pakli by udáveno bylo, postaví svedka a nebude povinen upláceti toho, což udáveno jest.
\par 14 Kdyby pak nekdo vypujcil neceho od bližního svého, a ochromelo by aneb umrelo v neprítomnosti pána jeho, bez výmluvy navrátí zase.
\par 15 Pakli by pán jeho byl s ním, není povinen platiti, ponevadž bylo za peníze najaté, a prišlo za mzdu svou.
\par 16 Jestliže by kdo namluvil pannu, kteráž není zasnoubena, a spal by s ní: dát jí veno, a vezme ji sobe za manželku.
\par 17 Pakli by otec její nikoli nechtel jí dáti jemu, odváží stríbra podlé obyceje vena panenského.
\par 18 Carodejnici nedáš živu býti.
\par 19 Kdo by koli scházel se s hovadem, smrtí at umre.
\par 20 Kdo by obetoval bohum, krome samému Hospodinu, jako proklatý vyhlazen bude.
\par 21 Príchozímu neuciníš krivdy, aniž utiskneš ho; nebo príchozí byli jste v zemi Egyptské.
\par 22 Žádné vdovy neb sirotka trápiti nebudete.
\par 23 Pakli bez lítosti trápiti je budete, a oni by volali ke mne, vezte, že vyslyším krik jejich.
\par 24 A rozhnevá se prchlivost má, i zbiji vás mecem; a budou ženy vaše vdovy a deti vaši sirotci.
\par 25 Pujcíš-li penez lidu mému chudému, kterýž jest s tebou: nebudeš jemu jako lichevník, aniž ho lichvou obtížíš.
\par 26 Pakli v základu vezmeš roucho bližního svého, do západu slunce jemu je navrátíš.
\par 27 Nebo ten jediný má odev, to jest roucho, jímž prikrývá telo své, a na nemž spí. Když bude volati ke mne, tedy uslyším, nebo jsem milosrdný.
\par 28 Soudcum nebudeš utrhati, a knížeti lidu svého zloreciti nebudeš.
\par 29 Z hojnosti obilí, a tekutých vecí svých neobmeškáš prvotin obetovati. Prvorozeného z synu svých mne dáš.
\par 30 Tak uciníš s volem svým a s dobytkem svým: Sedm dní bude s matkou svou, dne pak osmého mne je dáš.
\par 31 Lid svatý budete mi, a nebudete jísti masa z udáveného na poli; psu je vržete.

\chapter{23}

\par 1 Nebudeš vynášeti povesti lživé. Neklad s bezbožným ruky své, abys mel býti s ním svedek nepravý.
\par 2 Nepostoupíš po množství ke zlému,a nebudeš se primlouvati k rozepri, tak abys se uchýlil po vetším poctu k prevrácení soudu.
\par 3 Ani chudého šanovati nebudeš v pri jeho.
\par 4 Trefil-li bys na vola neprítele svého neb osla jeho, an bloudí, obrátíš a dovedeš ho k nemu.
\par 5 Uzrel-li bys an osel toho, jenž te má v nenávisti, leží pod bremenem svým, zdaž se zdržíš, abys mu nemel pomoci? Nýbrž opravdove pomužeš jemu, spolu s tím, kdož te v nenávisti má.
\par 6 Neprevrátíš soudu chudého svého v jeho pri.
\par 7 Od slova lživého vzdálíš se. Nevinného a spravedlivého nezabiješ, nebo já neospravedlním bezbožného.
\par 8 Aniž bráti budeš daru, nebo dar oslepuje i prozretelné, a prevrací slova spravedlivých.
\par 9 Príchozího nebudeš ssužovati; nebo sami znáte, jaký jest život príchozích, ponevadž pohostinu jste byli v zemi Egyptské.
\par 10 Po šest let osívati budeš zemi svou, a shromaždovati úrodu její;
\par 11 Sedmého pak léta ponecháš jí, at odpocine, aby jedli chudí lidu tvého. Co pak zustane po nich, pojí zver polní. Tak udeláš s vinicí svou i s olivovím svým.
\par 12 Šest dní budeš delati díla svá, dne pak sedmého prestaneš, aby odpocinul vul tvuj i osel tvuj, a oddechl syn devky tvé i príchozí.
\par 13 Ve všech tech vecech, kteréž mluvil jsem vám, ostríhati se budete. Jména bohu cizích ani pripomínati nebudete, aniž bude slyšáno z úst tvých.
\par 14 Trikrát slaviti mi budeš svátek na každý rok.
\par 15 Slavnosti presnic ostríhati budeš. Sedm dní jísti budeš chleby presné, jakž jsem prikázal tobe, v cas vymerený mesíce Abib; nebo v ten vyšel jsi z Egypta. Aniž se ukážete prede mnou prázdní.
\par 16 A držeti budeš slavnost žne, když mi obetovati budeš prvotiny prací svých z toho, což jsi vsel na poli. Slavnost také sklizení držeti budeš pri vyjití roku, když sklidíš práce své z pole.
\par 17 Trikrát v roce ukáže se každý z tvých pohlaví mužského pred tvárí Panovníka Hospodina.
\par 18 Nebudeš obetovati krve z obeti mé, dokavadž u tebe kvas jest, aniž zustane tuk slavnosti mé do jitra.
\par 19 Prvotiny prvních úrod zeme své prinášeti budeš do domu Hospodina Boha svého. Nebudeš variti kozelce v mléku matere jeho.
\par 20 Aj, já pošli andela pred tebou, aby ostríhal tebe na ceste, a privedl te na místo, kteréž jsem pripravil.
\par 21 Šetrne se mej pred ním, a poslouchej hlasu jeho. Nepopouzej ho, nebot nepromine prestoupení vašeho, ponevadž jméno mé jest u prostred neho.
\par 22 Nebo budeš-li verne poslouchati hlasu jeho, a ciniti, cožt bych koli rekl: tedy neprítelem budu neprátel tvých, a trápiti budu ty, jenž tebe trápí.
\par 23 Nebo pujde andel muj pred tebou, a uvede te do zeme Amorejského a Hetejského, Ferezejského a Kananejského, Hevejského a Jebuzejského, kteréž vyhladím.
\par 24 Nebudeš se klaneti bohum jejich, ani jim sloužiti, aniž delati budeš tak, jako oni delají; ale z gruntu vyvrátíš je, a obrazy jejich na kusy stroskoceš.
\par 25 Sloužiti pak budete Hospodinu Bohu svému, a požehnát chlebu tvému i vodám tvým; a odejmu nemoc z prostredku tvého.
\par 26 Nebudet, která by potratila, ani neplodná v zemi tvé; pocet dnu tvých doplním.
\par 27 Strach svuj pustím pred tebou, a predesím všeliký lid, proti kterémuž vyjdeš, a zpusobím to, aby všickni neprátelé tvoji utíkali pred tebou.
\par 28 Pošli i sršne pred tebou, aby vyhnali Hevea, a Kananea a Hetea pred tvárí tvou.
\par 29 Nevyženu ho od tvári tvé v jednom roce, aby se zeme neobrátila v poušt, a nerozmnožily se proti tobe šelmy divoké.
\par 30 Pomalu vyháneti jej budu od tvári tvé, až bys ty se rozplodil, a dedicne mohl ujíti zemi.
\par 31 Položím pak meze tvé od more Rudého až k mori Filistinskému, a od poušte až k rece; nebo v ruce vaše dám obyvatele zeme, a vyženeš je od tvári své.
\par 32 Neuciníš s nimi a bohy jejich smlouvy.
\par 33 Nebudou bydliti v zemi tvé, aby nepripravili te k hríchu proti mne, když bys ctil bohy jejich; nebo by to bylo tobe osídlem.

\chapter{24}

\par 1 Mojžíšovi pak rekl: Vstup k Hospodinu ty a Aron, Nádab a Abiu, a sedmdesáte z starších Izraelských, a klaneti se budete zdaleka.
\par 2 Sám pak toliko Mojžíš vstoupí k Hospodinu, ale oni se nepriblíží; aniž lid vstoupí s ním.
\par 3 Tedy prišel Mojžíš, a vypravoval lidu všecka slova Hospodinova a všecky soudy. I odpovedel všecken lid jedním hlasem, a rekli: Všecka slova, kteráž mluvil Hospodin, uciníme.
\par 4 Napsal pak Mojžíš všecka slova Hospodinova, a vstav ráno, vzdelal oltár pod horou, a dvanácte sloupu podlé poctu dvanáctera pokolení Izraelského.
\par 5 A poslal mládence z synu Izraelských, kteríž obetovali zápaly; a obetovali obeti pokojné Hospodinu, totiž voly.
\par 6 I vzav Mojžíš polovici krve, vlil do medenic, a polovici druhou vylil na oltár.
\par 7 Vzav také knihu smlouvy, cetl v uších lidu. Kteríž rekli: Cožkoli mluvil Hospodin, ciniti a poslouchati budeme.
\par 8 Vzal také Mojžíš krev a pokropil lidu a rekl: Aj, krev smlouvy, kterouž ucinil s vámi Hospodin pri všech techto vecech.
\par 9 Potom vstoupili Mojžíš a Aron, Nádab a Abiu a sedmdesáte z starších Izraelských,
\par 10 A videli Boha Izraelského. A pod nohami jeho bylo jako dílo z kamene zafirového, a jako nebe, když jest jasné.
\par 11 Na knížata pak synu Izraelských nevztáhl ruky své, ackoli videli Boha, a potom jedli i pili.
\par 12 I rekl Hospodin Mojžíšovi: Vstup ke mne na horu a bud tam; a dám tobe tabule kamenné, zákon i prikázaní, kteráž jsem napsal, abys je ucil.
\par 13 Tedy vstal Mojžíš a Jozue služebník jeho; i vstoupil Mojžíš na horu Boží.
\par 14 Starším pak rekl: Zustante tuto, dokudž se nenavrátíme k vám. A ted Aron a Hur jsou s vámi, kdož by mel rozepri, k nim at jde.
\par 15 Tedy vstoupil Mojžíš na horu, a prikryl oblak horu.
\par 16 I prebývala sláva Hospodinova na hore Sinai, a prikryl ji oblak za šest dní; a dne sedmého zavolal na Mojžíše z prostred oblaku.
\par 17 A tvárnost slávy Hospodinovy byla jako spalující ohen na vrchu hory, pred ocima synu Izraelských.
\par 18 I všel Mojžíš do prostred oblaku a vstoupil na horu. A byl Mojžíš na hore ctyridceti dní a ctyridceti nocí.

\chapter{25}

\par 1 I mluvil Hospodin k Mojžíšovi, rka:
\par 2 Mluv k synum Izraelským, at mi vybírají obet vzhuru pozdvižení. Od každého cloveka, kterýž by ji z srdce dobrovolne dal, prijmete takovou obet mou.
\par 3 Tatot pak jest obet pozdvižení, kterouž budete bráti od nich: Zlato, a stríbro, a med,
\par 4 Postavec modrý, šarlat, a cervec dvakrát barvený, bílé hedbáví a srsti kozí;
\par 5 Též kuže skopcové na cerveno barvené, a kuže jezevcí, a dríví setim,
\par 6 Olej k svícení, vonné veci na olej ku pomazování, a pro kadení vonné veci;
\par 7 Kamení onychinové, a jiné kamení k vsazování do náramníku a náprsníku.
\par 8 I udelajít mi svatyni, abych bydlil uprostred nich.
\par 9 Vedlé všeho, jakž já ukazuji tobe podobenství stánku a podobenství všech nádob jeho, tak udeláte.
\par 10 Udelají také truhlu z dríví setim. Pul tretího lokte bude dlouhost její, pul druhého lokte širokost její, pul druhého také lokte vysokost její.
\par 11 A obložíš ji zlatem cistým, vnitr i zevnitr obložíš ji; a udeláš nad ní vukol korunu zlatou.
\par 12 Sliješ k ní také ctyri kruhy zlaté, kteréž prideláš ke ctyrem úhlum jejím, dva totiž kruhy po jedné strane její, a dva kruhy po druhé strane její.
\par 13 Udeláš k tomu i sochory z dríví setim, a obložíš je zlatem.
\par 14 I uvleceš sochory do kruhu po stranách té truhly, aby na nich nošena byla truhla.
\par 15 V kruzích té truhly budou bývati sochorové; nebudou vytahováni z nich.
\par 16 A dáš do truhly svedectví, kteréž dám tobe.
\par 17 Udeláš i slitovnici z zlata cistého. Pul tretího lokte bude dlouhost její, pul druhého pak lokte širokost její.
\par 18 Udeláš také dva cherubíny zlaté, z taženého zlata udeláš je na dvou koncích slitovnice.
\par 19 Udeláš pak cherubína jednoho na jednom konci, a cherubína druhého na druhém konci; na slitovnici udeláte cherubíny na obou koncích jejích.
\par 20 A budou míti cherubínové krídla vztažená svrchu, zastírajíce krídly svými slitovnici, a tvári jejich obráceny budou jednoho k druhému; k slitovnici budou tvári cherubínu.
\par 21 Dáš pak slitovnici svrchu na truhlu, a do truhly vložíš svedectví, kteréž dám tobe.
\par 22 A tam budu pricházeti k tobe, a s tebou z té slitovnice, z prostredku dvou cherubínu, kteríž jsou nad truhlou svedectví, mluviti o všecko, cožt bych porouceti chtel k synum Izraelským.
\par 23 Udeláš také stul z dríví setim. Dvou loket bude dlouhost jeho, a na loket širokost jeho, pul druhého pak lokte vysokost jeho.
\par 24 A obložíš jej zlatem cistým, a udeláš mu okolek zlatý vukol.
\par 25 Udeláš také okolo neho lištu ctyr prstu zšírí; a okolek zlatý udeláš okolo té lišty.
\par 26 Udeláš u neho i ctyri kruhy zlaté, kteréž vpustíš do ctyr úhlu, kteríž jsou ve ctyrech nohách jeho.
\par 27 Pod tou lištou budou kruhové, skrze než provlacováni budou sochorové k nošení stolu.
\par 28 Ty pak sochory udeláš z dríví setim, a obložíš je zlatem; i bude stul nošen na nich.
\par 29 Udeláš také misy jeho, a lžice jeho, a prikryvadla jeho, a koflíky jeho, k prikrývání; z cistého zlata nadeláš toho.
\par 30 A klásti budeš na ten stul chleby predložení pred tvár mou ustavicne.
\par 31 Udeláš také svícen z zlata cistého, z taženého zlata at jest ten svícen; sloupec jeho i prutové jeho, misky jeho a koule jeho, i kvetové jeho z neho budou.
\par 32 A šest prutu vycházeti bude z boku jeho, tri prutové svícnu s jedné strany jeho,a tri prutové s druhé strany jeho.
\par 33 Tri misky udelané na zpusob pecky mandlové at jsou na prutu jednom, a koule a kvet, a tri misky udelané na zpusob pecky mandlové na prutu druhém, a koule a kvet; takž i na jiných šesti prutech z svícnu vycházejících.
\par 34 Na svícnu také budou ctyri misky udelané na zpusob mandlové pecky, a koule jeho, i kvetové jeho.
\par 35 A bude koule pode dvema pruty z neho, koule též pod druhými dvema pruty z neho, koule opet pod jinými dvema pruty z neho; a tak pod šesti pruty vycházejícími z svícna.
\par 36 Koule jejich i prutové jejich z neho budou; všecko to z cela kované z zlata cistého.
\par 37 Udeláš i sedm lamp na nej; a bude je rozsvecovati knez, aby svítily po stranách jeho.
\par 38 I uteradla jeho, i nádoby k oharkum jeho z zlata cistého.
\par 39 Z centnére zlata cistého udeláno bude to se vším tím nádobím.
\par 40 Hlediž pak, abys udelal podlé podobenství toho, kteréž tobe ukázáno jest na hore.

\chapter{26}

\par 1 Príbytek pak udeláš z desíti calounu, kteríž budou z bílého hedbáví soukaného, a z postavce modrého, a z šarlatu, a z cervce dvakrát barveného; a cherubíny dílem remeslným udeláš.
\par 2 Dlouhost calounu jednoho osm a dvadceti loket, a širokost calounu jednoho ctyri lokty; míra jedna bude všech calounu.
\par 3 Pet calounu spolu spojeno bude jeden s druhým, a pet druhých calounu též spolu spojeno bude jeden s druhým.
\par 4 A nadeláš i ok z hedbáví modrého po kraji calounu jednoho na konci, kde se má spojovati s druhým; a tolikéž udeláš na kraji calounu druhého na konci v spojení druhém.
\par 5 Padesáte ok udeláš na calounu jednom, a padesáte ok udeláš po kraji calounu, kterýmž má pripojen býti k druhému; oko jedno proti druhému aby bylo.
\par 6 Udeláš také padesáte haklíku zlatých a spojíš calouny jeden s druhým haklíky temi; a tak bude príbytek jeden.
\par 7 Nadto udeláš houní z srstí kozích na stánek k pristírání príbytku po vrchu; jedenácte takových houní udeláš.
\par 8 Dlouhost houne jedné tridceti loktu, a širokost houne jedné ctyr loktu; jednostejná míra tech jedenácti houní bude.
\par 9 A spojíš pet houní obzvláštne, a šest houní opet obzvláštne, a prehneš na dvé houni šestou napred v cele stánku.
\par 10 Udeláš pak padesáte ok po kraji houne jedné na konci, kdež se spojovati má, a padesáte ok po kraji houne k spojení druhému.
\par 11 Udeláš také haklíku medených padesát, kteréž vepneš do ok, a spojíš stánek, aby byl jedno.
\par 12 Co pak zbývá houní po prikrytí stánku, totiž pul houne presahující, previsne pri zadní strane príbytku.
\par 13 A loket s jedné a loket s druhé strany, zbývající na dýl z houní stánku, previsne po stranách príbytku sem i tam, aby jej prikrýval.
\par 14 Udeláš také prikrytí na stánek z koží skopových na cerveno barvených, prikrytí také z koží jezevcích svrchu.
\par 15 Nadeláš k príbytku i desk stojatých z dríví setim.
\par 16 Desíti loktu dlouhost dsky, a pul druhého lokte širokost dsky jedné.
\par 17 Dva cepy dska jedna míti bude, podobne jako stupne u schodu zporádané, jeden proti druhému; tak udeláš u všech desk príbytku.
\par 18 Zdeláš pak desky k príbytku, dvadceti desk k strane polední, k vetru polednímu.
\par 19 (A ctyridceti podstavku stríbrných udeláš pod dvadceti desk; dva podstavky pod jednu dsku ke dvema cepum jejím, a dva podstavky pod dsku druhou pro dva cepy její.)
\par 20 Na druhé pak strane príbytku k strane pulnocní dvadceti desk,
\par 21 A ctyridceti podstavku jejich stríbrných; dva podstavky pod jednu dsku a dva podstavky pod dsku druhou.
\par 22 Na strane také príbytku k západu šest udeláš desk.
\par 23 A dve dsky udeláš v obou dvou úhlech príbytku;
\par 24 Kteréž budou spojené pozpodu, a tolikéž spojené svrchu k jednomu kruhu; tak bude pri dvou tech, ve dvou úhlech budou.
\par 25 A tak bude osm desk, a podstavkové jejich stríbrní, šestnácte podstavku; dva podstavkové pod dskou jednou a dva podstavkové pod dskou druhou.
\par 26 Nadeláš také svlaku z dríví setim. Pet jich bude dskám k strane príbytku jedné,
\par 27 A pet svlaku dskám pri strane príbytku druhé, a pet svlaku dskám k strane západní príbytku dosahující k obema úhlum.
\par 28 Ale prostrední svlak u prostred desk provlece se od jednoho konce k druhému.
\par 29 Ty pak dsky obložíš zlatem, a kruhy k nim udeláš zlaté, do nichž by svlakové byli uvlacováni; a obložíš i svlaky zlatem.
\par 30 A tak vyzdvihneš príbytek podlé zpusobu toho, kterýž tobe ukázán na hore.
\par 31 Udeláš i oponu z postavce modrého, a z šarlatu, a z cervce dvakrát barveného, a z bílého hedbáví soukaného; dílem remeslným udeláš ji s cherubíny.
\par 32 A zavesíš ji na ctyrech sloupích z dríví setim, obložených zlatem, (hákové jejich zlatí), na ctyrech podstavcích stríbrných.
\par 33 A dáš oponu na háky, a vneseš vnitr za oponu truhlu svedectví; a oddelovati vám bude ta opona svatyni od svatyne svatých.
\par 34 Položíš také slitovnici na truhlu svedectví v svatyni svatých.
\par 35 A postavíš stul vne pred oponou, svícen pak naproti stolu v strane príbytku polední, a stul dáš na stranu pulnocní.
\par 36 A udeláš zastrení dverí stánku z postavce modrého a z šarlatu, a z cervce dvakrát barveného, a z bílého hedbáví presukovaného, dílem vyšívaným.
\par 37 K zastrení pak tomu udeláš pet sloupu z dríví setim, kteréž obložíš zlatem, a hákové jejich zlatí; a sleješ k nim pet podstavku medených.

\chapter{27}

\par 1 Udeláš také oltár z dríví setim peti loket zdýlí a peti loket zšírí; ctverhranatý bude oltár, a trí loket zvýší bude.
\par 2 A zdeláš mu rohy na ctyrech úhlech jeho; z neho budou rohové jeho; a obložíš jej medí.
\par 3 Nadeláš také k nemu hrncu, do kterýchž by popel bral, a lopat a kotlíku, a vidlicek a nádob k uhlí. Všecka nádobí jeho z medi udeláš.
\par 4 Udeláš mu i rošt mrežovaný medený, a u té mríže ctyri kruhy medené na ctyrech rozích jejích.
\par 5 A dáš ji pod okolek oltáre do vnitrku; a bude ta mríže až do polu oltáre.
\par 6 Udeláš k tomu oltári i sochory z dríví setim, a medí je okuješ.
\par 7 A ti sochorové provleceni budou skrze ty kruhy; a budou sochorové ti na obou stranách oltáre, když nošen bude.
\par 8 Udeláš jej z desk, aby byl vnitr prázdný; jakž ukázáno tobe na hore, tak udelají.
\par 9 Udeláš také sín príbytku k strane polední; koltry síne té ockovaté budou z bílého hedbáví soukaného; sto loket zdélí at má kraj jeden.
\par 10 Sloupu pak bude k nim dvadcet, a podstavku k nim medených dvadcet; háky na sloupích a prepásaní jich stríbrné.
\par 11 A tak i strana pulnocní na dýl at má koltry ockovaté sto loket zdýlí, a sloupu svých dvadceti, a podstavku k nim medených dvadceti; háky na sloupích a prepásaní jich stríbrné.
\par 12 Na šír pak té síne k strane západní budou koltry ockovaté padesáti loktu zdýlí, a sloupu k nim deset, a podstavku jejich deset.
\par 13 A širokost síne v strane prední na východ bude padesáti loktu.
\par 14 Patnácti loktu budou koltry ockovaté k strane jedné, sloupové k nim tri, a podstavkové jejich tri.
\par 15 A k strane druhé patnácti loktu zdýlí budou koltry ockovaté, sloupové jejich tri, a podstavkové jejich tri.
\par 16 K bráne pak té síne udeláno bude zastrení dvadcíti loktu z postavce modrého a šarlatu, a z cervce dvakrát barveného, a bílého hedbáví soukaného, dílem krumpérským, sloupové k nemu ctyri, a podstavkové jejich ctyri.
\par 17 Všickni sloupové síne vukol prepásani budou stríbrem; hákové pak jejich budou stríbrní, a podstavkové jejich medení.
\par 18 Dlouhost síne bude sto loket, a širokost padesáte, všudy jednostejná, vysokost pak peti loktu, z bílého hedbáví soukaného, a podstavkové budou medení.
\par 19 Všecka nádobí príbytku, ke vší službe jeho, a všickni kolíkové jeho, i všickni kolíkové síne z medi budou.
\par 20 Ty také prikaž synum Izraelským, at nanesou oleje olivového cistého, vytlaceného k svícení, aby lampy vždycky rozsvecovány byly.
\par 21 V stánku úmluvy pred oponou, kteráž zastírati bude svedectví, spravovati je budou Aron a synové jeho od vecera až do jitra pred Hospodinem. Tot bude rád vecný, kterýž zachovávati budou potomci jejich mezi syny Izraelskými.

\chapter{28}

\par 1 Ty pak prijmi k sobe Arona bratra svého s syny jeho z prostredku synu Izraelských, aby úrad knežský konali prede mnou: Aron, Nádab, Abiu, Eleazar a Itamar, synové Aronovi.
\par 2 A udeláš roucha svatá Aronovi bratru svému k sláve a k ozdobe.
\par 3 Ty také mluviti budeš se všechnemi umelými remeslníky, kteréž jsem naplnil duchem moudrosti, aby delali roucha Aronovi ku posvecení jeho, v nichž by úrad knežský konal prede mnou.
\par 4 Tato pak jsou roucha, kteráž udelají, náprsník, náramenník, plášt a sukni s oky, cepici a pás. Takové šaty svaté udelají Aronovi bratru tvému a synum jeho, aby úrad knežský konali prede mnou.
\par 5 A vezmou remeslníci zlato a modrý postavec, a šarlat a cervec dvakrát barvený a kment.
\par 6 Udelají pak náramenník z zlata, z postavce modrého a šarlatu, z cervce dvakrát barveného a hedbáví bílého, presukovaného dílem remeslným.
\par 7 Dva vrchní kraje spojená míti bude na dvou koncích svých, a tak se spolu držeti bude.
\par 8 Prepásaní pak pres ten náramenník, kteréž na nem bude, podobné bude dílu jeho; z týchž vecí bude, totiž z zlata, z postavce modrého a z šarlatu, a z cervce dvakrát barveného, a z hedbáví bílého presukovaného.
\par 9 Vezmeš také dva kameny onychinové, a vyryješ na nich jména synu Izraelských.
\par 10 Šest jmen jejich na kameni jednom, a jmen šest ostatních na kameni druhém, podlé porádku narození jejich.
\par 11 Dílem remeslníka, kterýž reže na kameni, a kterýž vyrývá peceti, vyryješ na tech dvou kameních jména synu Izraelských; do zlata je vsadíš.
\par 12 A položíš dva kameny ty na vrchních krajích náramenníku, kameny pro pamet na syny Izraelské; a nositi bude Aron jména jejich pred Hospodinem na obou ramenách svých na památku.
\par 13 Udeláš i haklíky zlaté.
\par 14 A dva retízky z zlata cistého jednostejne udeláš dílem toceným, a zavesíš retízky ty stocené na ty haklíky.
\par 15 Udeláš také náprsník soudu dílem remeslným, takovým dílem, jako náramenník udeláš jej z zlata, z postavce modrého, z šarlatu, z cervce dvakrát barveného, a z bílého hedbáví presukovaného.
\par 16 Ctverhraný bude a dvojnásobní; pídi dlouhost, a pídi širokost jeho bude.
\par 17 A vysadíš jej všudy kamením drahým. Ctyrmi rady at jest kamení, porádkem tímto: Sardius, topazius a smaragdus v prvním radu;
\par 18 V druhém pak zporádaní karbunkulus, zafir a jaspis;
\par 19 A v radu tretím linkurius, achates a ametyst;
\par 20 V ctvrtém radu chryzolit, onychin a beryl. Vsazeni budou do zlata v svém porádku.
\par 21 Tech pak kamenu s jmény synu Izraelských bude dvanácte, podlé jmen jejich, dílem režících peceti; jeden každý podlé jména svého, pro dvanáctero pokolení budou.
\par 22 Udeláš i k náprsníku retízky jednostejné dílem toceným z zlata cistého.
\par 23 Udeláš také k náprsníku dva kroužky zlaté, a dáš je na dva kraje náprsníka.
\par 24 A prostrcíš dva retízky zlaté skrze dva kroužky po krajích náprsníka.
\par 25 A druhé konce tech dvou retízku pripneš k haklíkum, a dáš na vrchní kraje náramenníka po predu.
\par 26 Udeláš i dva kroužky zlaté, kteréž dáš na dva kraje náprsníka, na tu obrubu jeho, kteráž jest po té strane k náramenníku do vnitrku.
\par 27 Udeláš ješte dva jiné kroužky zlaté, kteréž dáš na dve strany náramenníka zespod po predu naproti spojení jeho, svrchu nad prepásaním náramenníka.
\par 28 Tak svíží náprsník ten, kroužky jeho s kroužky náramenníka, tkanicí z postavce modrého, aby byl nad prepásaním náramenníka, a neodevstával náprsník od náramenníka.
\par 29 I bude nositi Aron jména synu Izraelských na náprsníku soudu na srdci svém, když vcházeti bude do svatyne, na památku pred Hospodinem ustavicne.
\par 30 Položíš pak do náprsníku soudu urim a thumim, aby bylo na srdci Aronove, když vcházeti bude pred Hospodina; a nositi bude Aron soud synu Izraelských na srdci svém pred Hospodinem vždycky.
\par 31 Udeláš také plášt pod náramenník, všecken z postavce modrého.
\par 32 A bude na vrchu v prostred neho díra; okolek její všudy vukol dílem tkaným, jako obojek u pancíre bude, aby se neroztrhl.
\par 33 Udeláš i na podolku jeho jablka zrnatá z hedbáví modrého, z šarlatu a z cervce dvakrát barveného, na podolku jeho vukol, a zvonecky zlaté mezi nimi vukol.
\par 34 Zvoncek zlatý a jablko zrnaté, opet za tím zvoncek zlatý a jablko zrnaté na podolku plášte vukol.
\par 35 A bude to míti na sobe Aron pri službách, aby slyšán byl zvuk jeho, když vcházeti bude do svatyne pred Hospodina, i když vycházeti bude, aby neumrel.
\par 36 Udeláš také plech z zlata cistého, a vyryješ na nem dílem vyrývajících peceti: Svatost Hospodinu.
\par 37 Kterýž dáš na tkanici z modrého postavce, a bude na cepici; napred na cepici bude.
\par 38 I bude nad celem Aronovým, aby nesl Aron nepravosti posvecených vecí, kterýchž by posvetili synové Izraelští pri všech darích posvecených vecí svých; a bude nad celem jeho vždycky, aby príjemné je cinil pred Hospodinem.
\par 39 Udeláš také sukni z hedbáví bílého vázanou s oky; udeláš i cepici z hedbáví bílého; pás také udeláš dílem krumpérským.
\par 40 Synum také Aronovým zdeláš sukne; pasy také jim udeláš a klobouky k sláve a k ozdobe.
\par 41 A obleceš v ne Arona bratra svého a syny jeho s ním, a pomažeš jich, a naplníš ruce jejich, a posvetíš mi jich, aby úrad knežský konali prede mnou.
\par 42 Nadelej jim i košilek lnených k zakrytí nahoty tela; od bedr až do stehen budou.
\par 43 A at je na sobe mají Aron i synové, když vcházeti budou do stánku úmluvy, aneb když pristupovati budou k oltári, aby sloužili v svatyni; a neponesou nepravosti, aniž zemrou. Rád tento bude vecný jemu i potomkum jeho po nem.

\chapter{29}

\par 1 Toto také uciníš jim ku posvecení jich, aby úrad knežský konali prede mnou: Vezmi volka jednoho ješte mladého a skopce dva bez vady.
\par 2 Chleby též presné a koláce presné s olejem smíšené, a oplatky presné polité olejem; z beli pšenicné nadeláš toho.
\par 3 A vklada to do koše jednoho, obetovati to budeš v koši, spolu s tím volkem a dvema skopci.
\par 4 Potom Aronovi a synum jeho pristoupiti kážeš ke dverím stánku úmluvy, a umyješ je vodou.
\par 5 A vezma roucha, obleceš Arona v sukni, a v plášt náležící pod náramenník, a v náramenník, a v náprsník, a prepášeš ho pasem náramenníka.
\par 6 Vstavíš i cepici na hlavu jeho, a korunu svatosti vstavíš na cepici.
\par 7 Naposledy vezmeš olej pomazání, a vyleje na hlavu jeho, pomažeš ho.
\par 8 Potom synum jeho pristoupiti kážeš, a zoblácíš je v sukne.
\par 9 A zopasuješ je pasy, Arona i syny jeho, a vstavíš jim cepicky na hlavu. I budout míti knežství rádem vecným; a posvetíš ruky Aronovy a ruky synu jeho.
\par 10 Privedeš také volka pred stánek úmluvy, i vloží Aron a synové jeho ruce své na hlavu volka.
\par 11 A zabiješ volka pred Hospodinem u dverí stánku úmluvy.
\par 12 A nabera krve z volka, pomažeš na rozích oltáre prstem svým, a všecku krev vyleješ k spodku oltáre.
\par 13 Vezmeš pak všecken tuk prikrývající droby a branici s jater, a dve ledviny s tukem, kterýž jest na nich, a zapálíš to na oltári.
\par 14 Maso pak z toho volka, a kuži s lejny jeho spálíš ohnem vne za stany; nebo obet za hríchy jest.
\par 15 Skopce také jednoho vezmeš, a vloží Aron a synové jeho ruce své na hlavu toho skopce.
\par 16 A zabiješ toho skopce, a nabera krve jeho, pokropíš oltáre na vrchu vukol.
\par 17 Skopce pak rozsekáš na kusy, a vymyje droby jeho i nohy, vkladeš je na ty kusy z neho a na hlavu jeho.
\par 18 A potom všeho skopce zapálíš na oltári; nebo zápal ten jest Hospodinu vune príjemná, obet ohnivá jest Hospodinu.
\par 19 Vezmeš také skopce druhého; i vloží Aron a synové jeho ruce své na hlavu téhož skopce.
\par 20 A zabiješ skopce toho, a vezma krev jeho, pomažeš jí konce ucha Aronova, a konce pravého ucha synu jeho, i palce na pravé ruce jejich, a palce na pravé noze jejich; a vykropíš tu krev na oltár vukol.
\par 21 A vezma krve, kteráž bude na oltári, a oleje pomazání, pokropíš Arona a roucha jeho i synu jeho a roucha jejich s ním; a budet posvecen on i roucho jeho, i synové jeho, a roucho synu jeho s ním.
\par 22 Potom vezmeš z skopce tuk a ocas, a tuk prikrývající droby, a branici s jater, a dve ledviny s tukem, kterýž jest na nich, a plece pravé, (nebo skopec naplnení jest),
\par 23 A jeden pecník chleba, a jeden kolác chleba s olejem, a oplatek jeden z koše chlebu presných, kterýž jest pred Hospodinem.
\par 24 A dáš to vše v ruce Aronovy a v ruce synu jeho, a obraceti to budeš sem i tam, aby byla obet obracení pred Hospodinem.
\par 25 Potom vezma to z rukou jejich, zapálíš na oltári v zápal, k vuni príjemné pred Hospodinem. Tot jest obet ohnivá Hospodinu.
\par 26 Vezmeš také hrudí z skopce posvecení, kterýž bude Aronuv, a obraceti je budeš sem i tam, aby byla obet obracení pred Hospodinem; a budet na tvuj díl.
\par 27 Posvetíš tedy hrudí obracení, a plece pozdvižení, kteréž obracíno a kteréž pozdvihováno bylo z skopce posvecení, z toho, kterýž bude Aronuv, a z toho, kterýž bude synu jeho.
\par 28 A bude to Aronovi a synum jeho právem vecným od synu Izraelských, když obet pozdvižení bude. Nebo pozdvižení bude od synu Izraelských, z obetí jejich pokojných; pozdvižení jejich náleží Hospodinu.
\par 29 Roucha pak svatá, kteráž jsou Aronova, zustanou synum jeho po nem, aby pomazováni byli v nich, a aby posvecovány byly v nich ruce jejich.
\par 30 Sedm dní bude v nich choditi knez, kterýž bude na jeho míste z synu jeho, kterýž vcházeti bude do stánku úmluvy, aby sloužil v svatyni.
\par 31 Skopce pak posvecení vezma, uvaríš maso jeho na míste svatém.
\par 32 A budet jísti Aron s syny svými maso toho skopce a chléb, kterýž jest v koši u dverí stánku úmluvy.
\par 33 Jísti budou to ti, za než ocištení se stalo ku posvecení rukou jejich, aby posveceni byli. Cizí pak nebude jísti, nebo svatá vec jest.
\par 34 Zustalo-li by co masa posvecení a chleba až do jitra, spálíš ostatky ohnem; nebude jedeno, nebo svatá vec jest.
\par 35 Tak tedy udeláš s Aronem a syny jeho vedlé všeho, což jsem prikázal tobe; za sedm dní posvecovati budeš rukou jejich.
\par 36 A volka za hrích obetovati budeš na každý den na ocištení, a krví za hrích pokropíš oltáre, cine ocištení na nem, a pomažeš ho ku posvecení jeho.
\par 37 Za sedm dní ocištování konati budeš na oltári, a posvetíš ho, a ten oltár bude nejsvetejší. Cokoli dotkne se oltáre, posveceno bude.
\par 38 A toto jest, což obetovati budeš na oltári, beránky rocní dva, na každý den ustavicne.
\par 39 Jednoho beránka obetovati budeš ráno, a beránka druhého obetovati budeš k vecerou.
\par 40 A desátý díl efi beli smíšené s olejem vytlaceným, jehož by bylo s ctvrtý díl hin, a k obeti mokré ctvrtý díl hin vína na jednoho beránka.
\par 41 Tolikéž beránka druhého obetovati budeš k vecerou. Jako pri obeti suché ranní, a jako pri mokré obeti její, tak pri této uciníš, aby byla vune príjemná, obet ohnivá Hospodinu.
\par 42 Zápalná obet tato ustavicná at jest po všecky veky vaše u dverí stánku úmluvy pred Hospodinem, kdež pricházeti budu k vám, abych tam s tebou mluvil.
\par 43 A tam pricházeti budu k synum Izraelským, a posveceno bude místo slávou mou.
\par 44 Nebo posvetím stánku úmluvy i oltáre; Arona také a synu jeho posvetím, aby mi úrad knežský konali.
\par 45 A bydliti budu u prostred synu Izraelských, a budu jim za Boha.
\par 46 A zvedít, že já jsem Hospodin Buh jejich, kterýž jsem je vyvedl z zeme Egyptské, abych prebýval u prostred nich, já Hospodin Buh jejich.

\chapter{30}

\par 1 Udeláš i oltár, na nemž by se kadilo; z dríví setim udeláš jej.
\par 2 Lokte zdélí, a lokte zšírí, ctverhraný bude, a dvou loket zvýší; z neho budou rohy jeho.
\par 3 Obložíš pak jej zlatem cistým, svrchek jeho i po stranách vukol i rohy jeho; a udeláš mu korunu zlatou vukol.
\par 4 Po dvou také kruzích zlatých udeláš u neho, pod korunou ve dvou úhlech jeho, po obou stranách jeho; a skrze ne provleceš sochory, aby nošen byl na nich.
\par 5 Ty pak sochory udeláš z dríví setim, a obložíš je zlatem.
\par 6 A postavíš jej pred oponou, za kterouž jest truhla svedectví, pred slitovnicí, kteráž jest nad svedectvím, kdež pricházeti budu k tobe.
\par 7 I kaditi bude na nem Aron kadidlem z vonných vecí; každého jitra, když spraví svetla, kaditi bude.
\par 8 Tolikéž když rozsvítí Aron lampy k vecerou, kaditi bude kadením vonných vecí ustavicne pred Hospodinem po rodech vašich.
\par 9 Nevložíte na nej kadidla cizího, ani zápalu, ani obeti suché, ani obeti mokré obetovati budete na nem.
\par 10 Toliko ocištení vykoná nad rohy jeho Aron jednou v roce, krví obeti za hrích v den ocištování; jednou v roce ocištení vykoná na nem po rodech vašich; svatosvaté jest Hospodinu.
\par 11 Mluvil pak Hospodin k Mojžíšovi, rka:
\par 12 Když vyzdvihneš hlavní summu synu Izraelských náležejících ku poctu, dá jeden každý výplatu duše své Hospodinu, když je pocítati budeš, aby nebyla na nich rána, když secteni budou.
\par 13 Toto pak dají: Každý z tech, kterí jdou v pocet, pul lotu dá, podlé lotu svatyne. (Dvadceti penez platí ten lot.) Pul lotu obet pozdvižení bude Hospodinu.
\par 14 Kdožkoli jde v pocet od dvadcíti let a výše, tu obet pozdvižení at dá Hospodinu.
\par 15 Bohatý nedá více, a chudý nedá méne, než pul lotu, když dávati budou obet pozdvižení Hospodinu k ocištení duší vašich.
\par 16 A vezma stríbro ocištení od synu Izraelských, dáš je na potreby k službe stánku úmluvy; a bude to synum Izraelským na památku pred Hospodinem k ocištení duší vašich.
\par 17 Mluvil také Hospodin k Mojžíšovi, rka:
\par 18 Udeláš i umyvadlo medené k umývání a podstavek jeho medený, a postavíš je mezi stánkem úmluvy a oltárem, a naleješ do neho vody.
\par 19 Umývati budou z neho Aron i synové jeho ruce své i nohy své.
\par 20 Když vcházeti budou do stánku úmluvy, umývati se budou vodou, aby nezemreli, aneb když by meli pristupovati k oltári, aby sloužili, a zapalovali obet ohnivou Hospodinu.
\par 21 I budou umývati ruce i nohy své, aby nezemreli. A bude jim ustanovení toto vecné, Aronovi i semeni jeho po rodech jejich.
\par 22 Mluvil také Hospodin k Mojžíšovi, rka:
\par 23 Ty pak vezmi sobe vonných vecí predních: Mirry nejcistší pet set lotu, a skorice vonné polovici toho, totiž dve ste a padesát, a prustvorce vonného dve ste a padesát;
\par 24 Kasie pak pet set lotu na váhu svatyne, oleje olivového míru hin.
\par 25 A udeláš z toho olej pomazání svatého, mast nejvýbornejší dílem apatykárským. Olej pomazání svatého bude.
\par 26 Pomažeš jím stánku úmluvy i truhly svedectví,
\par 27 A stolu i všech nádob k nemu, svícnu i všech nádob jeho, i oltáre, na nemž se kadí;
\par 28 Oltáre také, na nemž se pálí obeti, a všech nádob jeho, i umyvadla s podstavkem jeho.
\par 29 Tak posvetíš jich, aby byly nejsvetejší. Cožkoli dotkne se jich, posveceno bude.
\par 30 Arona také a synu jeho pomažeš, a posvetíš jich, aby úrad knežský konali prede mnou.
\par 31 K synum pak Izraelským mluviti budeš, rka: Tento olej pomazání svatého bude mi v národech vašich.
\par 32 Telo cloveka nebude mazáno jím, a podlé složení jeho neudeláte podobného. Svatýt jest, svatý vám bude.
\par 33 Kdo by koli udelal mast podobnou, a neb mazal by jí cizího, vyhlazen bude z lidu svého.
\par 34 I rekl Hospodin Mojžíšovi: Vezmi sobe vonných vecí, balsamu, onychi, galbanu vonného, a kadidla cistého; jednostejná váha toho bude.
\par 35 A udeláš z toho kadidlo, složení dílem apatykárským smíšené, cisté a svaté.
\par 36 A ztluka to drobne, klásti budeš z neho pred svedectvím v stánku úmluvy, kdež pricházeti budu k tobe. Nejsvetejší to vám bude.
\par 37 Neudeláte sobe kadidla podlé složení tohoto, kteréž pripravíš; za svatou vec tobe bude pro Hospodina.
\par 38 Kdo by koli delal co podobného k vuni sobe, vyhlazen bude z lidu svého.

\chapter{31}

\par 1 I mluvil Hospodin k Mojžíšovi, rka:
\par 2 Hle, povolal jsem ze jména Bezeleele, syna Uri, syna Hur, z pokolení Judova.
\par 3 A naplnil jsem ho duchem Božím, moudrostí a rozumností, i umením všelijakého remesla,
\par 4 Aby vtipne smysliti umel, což by koli remeslne udeláno býti mohlo z zlata a z stríbra i z medi.
\par 5 I v rezání kamení drahého k vsazování, i v umelém vysazování na dreve aby delal všelijaké dílo.
\par 6 A aj, já pridal jsem jemu Aholiaba, syna Achisamechova, z pokolení Dan. A v srdci každého vtipného složil jsem moudrost, aby spravili vše, což jsem prikázal tobe:
\par 7 Stánek úmluvy a truhlu svedectví, a slitovnici, kteráž má býti na ní, i všelijaké nádobí stánku;
\par 8 Stul také a nádoby k nemu, i svícen cistý se všemi nádobami jeho, a oltár pro kadení;
\par 9 Též oltár k zápalum se všemi nádobami jeho, a umyvadlo s podstavkem jeho;
\par 10 I roucha k službe, i roucha svatá Aronovi knezi, i roucha synu jeho, aby mi úrad knežský konali;
\par 11 I olej pomazání a kadidlo vonné do svatyne. Všecko tak, jakž jsem prikázal tobe, udelají.
\par 12 Mluvil také Hospodin k Mojžíšovi, rka:
\par 13 Ty pak mluv k synum Izraelským a rci: A však sobot mých ostríhati budete. Nebo to znamením jest mezi mnou a vámi po rodech vašich, aby známo bylo, že já jsem Hospodin, kterýž vás posvecuji.
\par 14 Protož ostríhati budete soboty, nebo svatá jest vám. Kdož by ji poškvrnil, smrtí umre; a kdo by koli delal v ní dílo, vyhlazena bude ta duše z prostredku lidu svého.
\par 15 Šest dní deláno bude dílo, ale v den sedmý sobota odpocinutí jest, svatost Hospodinu. Každý, kdož by delal dílo v den sobotní, smrtí umre.
\par 16 Protož ostríhati budou synové Izraelští soboty, tak aby svetili sobotu po rodech svých smlouvou vecnou.
\par 17 Mezi mnou a syny Izraelskými za znamení jest na vecnost; nebo šest dní cinil Hospodin nebe i zemi, v den pak sedmý prestal a odpocinul.
\par 18 I dal Pán Mojžíšovi po dokonání techto recí s ním na hore Sinai dve dsky svedectví, dsky kamenné, psané prstem Božím.

\chapter{32}

\par 1 Vida pak lid, že by prodléval Mojžíš sstoupiti s hury, sebrali se proti Aronovi a rekli jemu: Vstan, udelej nám bohy, kteríž by šli pred námi; nebo Mojžíšovi, muži tomu, kterýž vyvedl nás z zeme Egyptské, nevíme, co se prihodilo.
\par 2 I rekl jim Aron: Odejmete náušnice zlaté, kteréž jsou na uších žen vašich, synu vašich i dcer vašich, a prineste ke mne.
\par 3 Tedy strhl všecken lid náušnice zlaté, kteréž byly na uších jejich, a prinesli k Aronovi.
\par 4 Kteréžto vzav z rukou jejich, dal je do formy, a udelal z nich tele slité. I rekli: Tito jsou bohové tvoji, Izraeli, kteríž te vyvedli z zeme Egyptské.
\par 5 Což vida Aron, vzdelal oltár pred ním. I volal Aron, a rekl: Slavnost Hospodinova zítra bude.
\par 6 A nazejtrí vstavše velmi ráno, obetovali zápaly, a privedli obeti pokojné. I sedl lid, aby jedl a pil, potom vstali, aby hrali.
\par 7 Mluvil pak Hospodin k Mojžíšovi: Jdi, sstup, nebo porušil se lid tvuj, kterýž jsi vyvedl z zeme Egyptské.
\par 8 Sešli brzo s cesty, kterouž jsem prikázal jim. Udelali sobe tele slité, a klaneli se mu, a obetovali jemu, rkouce: Tito jsou bohové tvoji, Izraeli, kteríž te vyvedli z zeme Egyptské.
\par 9 Rekl také Hospodin Mojžíšovi: Videl jsem lid tento, a aj, lid jest tvrdé šíje.
\par 10 Protož nyní nech mne, abych v hneve prchlivosti své vyhladil je, tebe pak uciním v národ veliký.
\par 11 I modlil se Mojžíš Hospodinu Bohu svému, a rekl: Procež, ó Hospodine, roznecuje se prchlivost tvá na lid tvuj, kterýž jsi vyvedl z zeme Egyptské v síle veliké a v ruce mocné?
\par 12 A proc mají mluviti Egyptští, rkouce: Lstive je vyvedl, aby zmordoval je na horách, a aby vyhladil je se svrchku zeme? Odvrat se od hnevu prchlivosti své, a lituj zlého, kteréžs uložil uvésti na lid svuj.
\par 13 Rozpomen se na Abrahama, Izáka a Izraele, služebníky své, jimž jsi zaprisáhl skrze sebe samého a mluvil jsi jim: Rozmnožím síme vaše jako hvezdy nebeské, a všecku zemi tuto, o kteréž jsem mluvil, dám semeni vašemu, a dedicne obdržíte ji na veky.
\par 14 I litoval Hospodin zlého, kteréž rekl, že uciní lidu svému.
\par 15 A obrátiv se Mojžíš, sstoupil s hury, dve dsky svedectví maje v rukou svých, dsky po obou stranách psané; s jedné i s druhé strany byly popsané.
\par 16 A dsky ty dílo Boží byly; písmo také písmo Boží bylo vyryté na dskách.
\par 17 Uslyšev pak Jozue hlas lidu kricícího, rekl Mojžíšovi: Hrmot boje v táboru jest.
\par 18 Kterýžto odpovedel: Není to krik vítezících, ani krik poražených, hlas zpívajících já slyším.
\par 19 I stalo se, když se priblížil k stanum, že uzrel tele a tance. A rozhnevav se Mojžíš velmi, povrhl z rukou svých dsky, a rozrazil je pod Horou.
\par 20 Vzal také tele, kteréž byli udelali, a spálil je v ohni, a setrel je až na prach, a vsypav na vodu, dal píti synum Izraelským.
\par 21 A rekl Mojžíš Aronovi: Cot ucinil lid tento, že jsi uvedl na nej hrích veliký?
\par 22 Odpovedel Aron: Nehnevej se, pane muj. Ty víš, že lid tento k zlému naklonen jest.
\par 23 Nebo rekli mi: Udelej nám bohy, kteríž by šli pred námi; nebo Mojžíšovi, muži tomu, kterýž vyvedl nás z zeme Egyptské, nevíme, co se stalo.
\par 24 Jimž jsem odpovedel: Kdo má zlato, strhnete je s sebe. I dali mi, a uvrhl jsem je do ohne, a udelalo se to tele.
\par 25 A vida Mojžíš lid obnažený, že obnažil jej Aron ku potupe pred neprátely, kteríž povstati meli proti nim,
\par 26 Stoje v bráne táboru, rekl: Kdo jest Hospodinuv, pristup ke mne. I shromáždili se k nemu všickni synové Léví.
\par 27 Jimž rekl: Tak praví Hospodin Buh Izraelský: Pripaš jeden každý mec svuj k boku svému; prejdete sem i tam od brány táboru k bráne, a zabí jeden každý bratra svého, a každý prítele svého i bližního svého.
\par 28 I ucinili synové Léví podlé reci Mojžíšovy, a padlo jich v ten den z lidu na tri tisíce mužu.
\par 29 Nebo rekl byl Mojžíš: Posvettež dnes rukou svých Hospodinu, jeden každý na synu svém a na bratru svém, aby vám dal dnes požehnání.
\par 30 A když bylo nazejtrí, rekl Mojžíš lidu: Vy jste zhrešili hríchem velikým. Protož nyní vstoupím k Hospodinu, zda bych ho ukrotil pro hrích váš.
\par 31 Tedy navrátiv se Mojžíš k Hospodinu, rekl: Prosím, zhrešilt jest lid ten hríchem velikým; nebo udelali sobe bohy zlaté.
\par 32 Nyní pak neb odpust hrích jejich, a pakli nic, vymaž mne, prosím, z knihy své, kteroužs psal.
\par 33 I rekl Hospodin Mojžíšovi: Kdo zhrešil proti mne, toho vymaži z knihy své.
\par 34 Protož nyní jdi, ved lid tento, kamž jsem rozkázal tobe. Aj, andel muj pujde pred tebou; v den pak navštívení mého navštívím i na nich hrích jejich.
\par 35 I bil Hospodin lid, proto že ucinili tele, kteréž byl udelal Aron.

\chapter{33}

\par 1 I mluvil Hospodin k Mojžíšovi: Jdi, vstup odsud, ty i lid, kterýž jsi vyvedl z zeme Egyptské do zeme, kterouž jsem prisáhl Abrahamovi, Izákovi a Jákobovi, rka: Semeni tvému dám ji,
\par 2 (A pošli pred tebou andela, a vyženu Kananea, Amorea, Hetea, Ferezea, Hevea a Jebuzea,)
\par 3 Do zeme oplývající mlékem a strdí. Nebot sám nevstoupím s tebou, proto že lid tvrdé šíje jsi, abych nezahubil tebe na ceste.
\par 4 A uslyšav lid rec tuto prezlou, zámutek nesli, aniž vzal kdo okrasy své na sebe.
\par 5 Nebo byl rekl Hospodin Mojžíšovi: Mluv synum Izraelským: Vy jste lid tvrdé šíje; jakž jen jedinou vstoupím mezi vás, zahladím vás. Protož již, slož okrasu svou s sebe, a zvím, co uciniti mám s tebou.
\par 6 I svlékli s sebe synové Izraelští okrasy své u hory Oréb.
\par 7 Mojžíš pak vzav stánek, rozbil jej sobe vne za stany, vzdáliv se od táboru, a nazval jej stánkem úmluvy. Tedy kdokoli hledal Hospodina, ven choditi musil k stánku úmluvy, kterýž byl vne za stany.
\par 8 K tomu také, když vycházel Mojžíš k stánku, povstával všecken lid a stál každý u dverí stanu svého, a hledeli za Mojžíšem, dokudž nevšel do stánku.
\par 9 Bývalo pak toto, že když vcházíval Mojžíš do stánku, sstupoval sloup oblakový, a stával u dverí stánku, a mluvil s Mojžíšem.
\par 10 A všecken lid vida sloup oblakový, an stojí u dverí stánku, povstávali všickni, a klaneli se každý u dverí stanu svého.
\par 11 A mluvíval Hospodin k Mojžíšovi tvárí v tvár, tak jako mluví clovek s prítelem svým. Potom navracel se do táboru, ale služebník jeho Jozue, syn Nun, mládenec, neodcházel z stánku.
\par 12 I rekl Mojžíš Hospodinu: Pohled, ty velíš mi, abych vedl lid tento, a neoznámils mi, koho pošleš se mnou, ještos pravil: Znám te ze jména, k tomu také nalezl jsi milost prede mnou.
\par 13 Již tedy, jestliže jsem jen nalezl milost pred tebou, oznam mi, prosím, cestu svou, abych te poznal, a abych nalezl milost pred tebou; a pohled, že národ tento jest lid tvuj.
\par 14 I odpovedel: Tvár má predcházeti vás bude, a dámt odpocinutí.
\par 15 I rekl: Nemá-lit predcházeti nás tvár tvá, nevyvozuj nás odsud.
\par 16 Nebo po cem poznáno bude zde, že jsem nalezl milost pred tebou, já i lid tvuj? Zdali ne po tom, když pujdeš s námi, a když oddeleni budeme, já a lid tvuj, ode všeho lidu, kterýž jest na tvári zeme?
\par 17 I rekl Hospodin Mojžíšovi: I tu také vec, kterouž jsi pravil, uciním; nebo jsi nalezl milost prede mnou, a znám te ze jména.
\par 18 Rekl opet: Okažiž mi, prosím, slávu svou.
\par 19 Kterýž odpovedel: Já zpusobím to, aby šlo mimo tebe pred tvárí tvou všecko dobré mé, a zavolám ze jména: Hospodin pred tvárí tvou. Smiluji se, nad kýmž se smiluji, a slituji se, nad kýmž se slituji.
\par 20 Rekl také: Nebudeš moci videti tvári mé; nebot neuzrí mne clovek, aby živ zustal.
\par 21 I to rekl Hospodin: Aj, místo u mne, a staneš na skále.
\par 22 A když tudy pujde sláva má, postavím te v rozsedline skály, a prikryji te rukou svou, dokudž neprejdu.
\par 23 Potom odejmu ruku svou, i uzríš hrbet muj, ale tvár má nebude spatrína.

\chapter{34}

\par 1 I rekl Hospodin k Mojžíšovi: Vyteš sobe dve dsky kamenné podobné prvním, a napíši na dskách tech slova, kteráž byla na dskách prvních, kteréž jsi rozrazil.
\par 2 Budiž tedy hotov ráno, a vstoupíš v jitre na horu Sinai, a staneš prede mnou na vrchu hory té.
\par 3 Žádný at nevstupuje s tebou, aniž také kdo vidín bude na vší hore; ani ovce neb volové pásti se budou naproti hore této.
\par 4 Tedy Mojžíš vytesal dve dsky kamenné podobné prvním, a vstav ráno, vstoupil na horu Sinai, jakž mu prikázal Hospodin, a vzal v ruku svou dve dsky kamenné.
\par 5 I sstoupil Hospodin v oblaku, a stál s ním tam, a zavolal ze jména: HOSPODIN.
\par 6 Nebo pomíjeje Hospodin tvár jeho, volal: Hospodin, Hospodin, Buh silný, lítostivý a milostivý, dlouhocekající a hojný v milosrdenství a pravde,
\par 7 Milosrdenství cine tisícum, odpoušteje nepravost a prestoupení i hrích, a kterýž nikoli neospravedlnuje vinného, navštevuje nepravost otcu na synech, a na synech synu do tretího i ctvrtého pokolení.
\par 8 Mojžíš pak rychle sklonil hlavu k zemi, a poklonu ucinil.
\par 9 A rekl: Prosím, našel-li jsem milost v ocích tvých, Pane, necht jde, prosím, Pán u prostred nás, nebo lid jest tvrdé šíje, a milostiv bud nepravosti naší a hríchu našemu, a mej nás za dedictví.
\par 10 Kterýžto rekl: Aj, já uciním smlouvu prede vším lidem tvým. Uciním divné veci, kteréž nejsou ucineny na vší zemi a ve všech národech, a videti bude všecken lid, (mezi nimiž jsi,) skutky Hospodinovy; nebo hrozné bude to, což já uciním s tebou.
\par 11 Zachovej to, což já dnes tobe prikazuji. Aj, já vyženu pred tvárí tvou Amorea a Kananea, Hetea a Ferezea, Hevea a Jebuzea.
\par 12 Varuj se pak, abys necinil smlouvy s obyvateli zeme té, do kteréž vejdeš, at by nebyli osídlem u prostred tebe.
\par 13 Ale zboríte oltáre jejich, a modly jejich polámete, a jejich háje posekáte.
\par 14 Nebo nebudeš se klaneti Bohu jinému, proto že Hospodin jest, jméno má horlivý, Buh silný, horlivý jest.
\par 15 Nevcházej v smlouvu s obyvateli zeme té, aby když by smilnili, jdouce po bozích svých, a obetovali bohum svým, nepovolali te, a jedl bys z obeti jejich.
\par 16 A abys nebral ze dcer jeho synum svým, i smilnily by dcery jejich, jdouce po bozích svých, a naucily by smilniti syny tvé, jdouce po bozích svých.
\par 17 Bohu slitých neudeláš sobe.
\par 18 Slavnost presnic zachovávati budeš. Za sedm dní jísti budeš chleby nekvašené, jakž jsem prikázal tobe, v cas vymerený mesíce Abib; nebo mesíce toho vyšel jsi z Egypta.
\par 19 Všecko což otvírá život, mé jest, i všeliký samec v dobytku tvém, prvorozený z volu a ovcí.
\par 20 Ale prvorozené osle vyplatíš dobytcetem; pakli bys nevyplatil, šíji zlomíš jemu. Každého prvorozeného z synu svých vyplatíš, aniž ukáží se prede mnou prázdní.
\par 21 Šest dní pracovati budeš, dne pak sedmého prestaneš; v cas orání i žne prestaneš.
\par 22 A uciníš sobe slavnost téhodnu, svátek prvotin žne pšenicné a slavnost klizení po vyjití každého roku.
\par 23 Trikrát v roce ukáže se každý z vás pohlaví mužského pred oblícejem Panovníka Hospodina, Boha Izraelského.
\par 24 Nebo vyvrhu národy od tvári tvé a rozšírím meze tvé, aniž kdo sáhne na zemi tvou, když vstoupíš, abys se ukázal pred Hospodinem Bohem svým trikrát v roce.
\par 25 Nebudeš obetovati krve obeti mé, dokavadž u tebe kvas jest, aniž zustane do jitra obet slavnosti Fáze.
\par 26 Prvotiny prvních úrod zeme své prinášeti budeš do domu Hospodina Boha svého. Nebudeš variti kozelce v mléce matky jeho.
\par 27 I rekl Hospodin Mojžíšovi: Napiš sobe slova tato; nebo podlé slov tech ucinil jsem smlouvu s tebou a s Izraelem.
\par 28 Byl pak tam s Hospodinem ctyridceti dní a ctyridceti nocí, chleba nejedl a vody nepil; a napsal na dskách slova té smlouvy, totiž deset slov.
\par 29 I stalo se, když sstupoval Mojžíš s hory Sinai, (a mel dve dsky svedectví v rukou svých, když sstupoval s hory), nevedel, že by se stkvela kuže tvári jeho, když mluvil s ním.
\par 30 A videl Aron i všickni synové Izraelští Mojžíše, a aj, stkvela se kuže tvári jeho, a nesmeli pristoupiti k nemu.
\par 31 Ale Mojžíš zavolal jich, a navrátili se k nemu Aron i všecka knížata shromáždení toho, a mluvil Mojžíš s nimi.
\par 32 Potom prišli také k nemu všickni synové Izraelští, jimžto prikázal všecko, což s ním mluvil Hospodin na hore Sinai.
\par 33 Dokudž pak mluvil Mojžíš s nimi, mel zásteru na tvári své.
\par 34 Ale když vcházel Mojžíš pred tvár Hospodina, aby mluvil s ním, odjímal zásteru, dokudž nevyšel. Vyšed pak, mluvil synum Izraelským, což mu bylo rozkázáno.
\par 35 Tedy videli synové Izraelští tvár Mojžíšovu, že se stkvela kuže tvári jeho. A kladl zase Mojžíš zásteru na tvár svou, dokudž nevcházel, aby mluvil s ním.

\chapter{35}

\par 1 Tedy svolal Mojžíš všecko shromáždení synu Izraelských, a rekl jim: Tato jsou slova, kteráž prikázal vám Hospodin, abyste je cinili.
\par 2 Šest dní deláno bude dílo, ale v sedmý den mejte svátek, sobotu odpocinutí Hospodinova. Kdo by delal v nem dílo, umre.
\par 3 Nezanítíte ohne nikdež v príbytcích svých v den sobotní.
\par 4 Mluvil také Mojžíš ke všemu shromáždení synu Izraelských takto: Toto jest slovo, kteréž prikázal Hospodin, rka:
\par 5 Seberte z sebe obet pozdvižení Hospodinu. Každý, kdož jest ochotný v srdci svém, prinese tu obet Hospodinu: Zlato, stríbro a med,
\par 6 A postavec modrý, a šarlat, a cervec dvakrát barvený, a bílé hedbáví, a kozí srsti;
\par 7 Též kuže skopcu na cerveno barvené, a kuže jezevcí, a dríví setim,
\par 8 A olej k svícení a vonné veci na olej ku pomazání a pro kadení vonné,
\par 9 A kamení onychinové, a jiné kamení k vsazování do náramenníku a náprsníku.
\par 10 A všickni, kdož jsou vtipní mezi vámi, prijdou a delati budou, cožkoli prikázal Hospodin:
\par 11 Príbytek, stánek jeho i prikrytí jeho a háky jeho, dsky jeho, svlaky jeho, sloupy jeho i podstavky jeho;
\par 12 Truhlu s sochory jejími, slitovnici a oponu zastrení;
\par 13 Stul i sochory k nemu se všemi nádobami jeho, i chléb predložení;
\par 14 A svícen k svícení s nádobami jeho i lampy jeho, a olej k svícení;
\par 15 Též oltár pro kadení a sochory jeho, i olej pomazání a kadidlo z vonných vecí, i zastrení dverí v príbytku;
\par 16 Oltár k zápalu a rošt jeho medený a sochory jeho, i všecka nádobí jeho, i umyvadlo s podstavkem jeho;
\par 17 Koltry ockovaté k síni, sloupy její a podstavky její, i zastrení brány do síne;
\par 18 Kolíky k príbytku, a kolíky síne s provázky jejich;
\par 19 Roucha k službe, k prisluhování v svatyni, i roucho svaté Arona kneze, i roucho synu jeho k konání úradu knežského.
\par 20 Vyšlo tedy všecko shromáždení synu Izraelských od tvári Mojžíšovy,
\par 21 A prišli, každý muž, kteréhož ponuklo srdce jeho, a každý, v nemž duch jeho byl dobrovolný, a prinesli obet pozdvižení Hospodinu, k dílu stánku úmluvy a ke vší službe jeho, i k rouchu svatému.
\par 22 Pricházeli muži i ženy, každý, kdož byl ochotný v srdci, a prinášeli spinadla, a náušnice, a prsteny, a záponky z pravých rukou, všelijaké nádobí zlaté, a kdožkoli obetoval obet zlata Hospodinu.
\par 23 Každý, kdož mel postavec modrý a šarlat, a cervec dvakrát barvený, a bílé hedbáví a kozí srsti, a kuže skopcu na cerveno barvené, a kuže jezevcí, prinesl to.
\par 24 Kdokoli obetoval obet stríbra a medi, prinášeli to za obet pozdvižení Hospodinu; každý také, kdo mel dríví setim, ke všelikému dílu služebnosti, prinášeli je.
\par 25 Ano i všecky ženy vtipné rukama svýma predly, a prinesly, co napredly, postavec modrý a šarlat, a cervec dvakrát barvený a bílé hedbáví.
\par 26 Všecky pak ženy, kterýchž ponuklo srdce jejich, aby predly umele, predly srsti kozí.
\par 27 Knížata pak prinášeli kamení onychinové, a jiné kamení k vsazování do náramenníku a náprsníku,
\par 28 Též vonné veci a olej k svícení, a k oleji pomazání, a k vonnému kadidlu.
\par 29 Každý muž i žena, v nichž ochotné srdce jejich bylo k tomu, aby prinášeli potreby ke všelikému dílu, kteréž byl prikázal delati Hospodin skrze Mojžíše, prinášeli synové Izraelští obet dobrovolnou Hospodinu.
\par 30 Tedy rekl Mojžíš synum Izraelským: Pohledte, povolal ze jména Hospodin Bezeleele, syna Uri, syna Hur, z pokolení Judova.
\par 31 A naplnil ho duchem Božím, moudrostí, rozumností i umením všelijakého remesla,
\par 32 Aby vtipne smysliti umel, jak by se co delati melo na zlatu a stríbru a medi,
\par 33 I v remeslném strojení kamenu k vsazování, i v díle od dreva, aby delal všelijakým remeslem vtipným.
\par 34 Dal nadto v srdce jeho i to, aby uciti mohl on i Aholiab, syn Achisamechuv z pokolení Dan.
\par 35 Naplnil je moudrostí srdce, aby delali všelijaké dílo tesarské a remeslné, i krumpérské a vytkávané z postavce modrého a šarlatu, a cervce dvakrát barveného, a bílého hedbáví, a aby delali všelijaké dílo a vymýšleli vtipné veci.

\chapter{36}

\par 1 Tedy delal Bezeleel a Aholiab i všeliký muž vtipný, jimž dal Hospodin moudrost a rozumnost, aby umeli delati všeliké dílo k službe svatyne, všecko, což prikázal Hospodin.
\par 2 Povolal pak Mojžíš Bezeleele a Aholiaba i každého muže vtipného, v jehož srdce dal Hospodin moudrost, každého také, kohož ponoukalo srdce jeho, aby pristoupil ku práci díla toho.
\par 3 A vzali od Mojžíše všecky dary, kteréž prinesli synové Izraelští k dílu služebnému svatyne, aby delali je. Ale oni vždy predce prinášeli k nemu každého jitra dary dobrovolné.
\par 4 Tedy prišli všickni vtipní delníci díla svatyne, každý od díla svého, kteréž delali,
\par 5 A mluvili k Mojžíšovi temi slovy: Mnohem více prináší lid, nežli potrebí jest k delání díla, kteréž prikázal Hospodin delati.
\par 6 I rozkázal Mojžíš, aby provoláno bylo v vojšte takto: Muž ani žena, žádný neprinášej více obeti k svatyni. I zbráneno jest lidu, aby nenosili.
\par 7 Nebo meli potreb hojne dosti k delání všelikého díla, tak že zbývalo.
\par 8 I delal každý vtipný z delníku tech dílo to, príbytek z desíti calounu, kteríž byli z bílého hedbáví presukovaného, a z postavce modrého a šarlatu, a z cervce dvakrát barveného, cherubíny dílem remeslným udelal na nich.
\par 9 Dlouhost calounu jednoho osm a dvadceti loktu, a ctyr loktu širokost calounu jednoho; všickni calounové byli jednostejné míry.
\par 10 Potom spojil pet calounu jeden s druhým, a pet jiných calounu spojil jeden s druhým.
\par 11 Nadelal i ok z hedbáví modrého po kraji calounu jednoho na konci, kde se spojovati má s druhým, a tolikéž udelal na kraji calounu druhého na konci v spojení druhém.
\par 12 Padesáte ok udelal na calounu jednom, a padesáte ok udelal po kraji calounu, kterýmž pripojen byl k druhému; oko jedno proti druhému bylo.
\par 13 Udelal i padesáte haklíku zlatých a spojil calouny jeden s druhým haklíky temi; a tak udelán jest príbytek jeden.
\par 14 Nadto nadelal houní z srstí kozích na stánek, k pristírání príbytku po vrchu; jedenácte houní udelal.
\par 15 Dlouhost houne jedné tridceti loktu, a širokost houne jedné ctyr loktu; jednostejná míra byla tech jedenácti houní.
\par 16 A spojil pet houní obzvláštne, a šest houní obzvláštne.
\par 17 Udelal také padesáte ok po kraji houne na konci, kdež se spojovati má, a padesáte ok udelal po kraji houne v spojení druhém.
\par 18 Udelal k tomu haklíku medených padesáte k spojení stánku, aby byl jedno.
\par 19 Nadto udelal prikrytí stánku z koží skopcových na cerveno barvených, a prikrytí z koží jezevcích svrchu.
\par 20 Nadelal k príbytku i desk z dríví setim stojatých.
\par 21 Desíti loktu dlouhost dsky, a pul druhého širokost dsky každé.
\par 22 Dva cepy mela dska jedna, podobne jako stupne u schodu sporádané, jeden proti druhému; tak udelal u všech desk príbytku.
\par 23 Zdelal i dsky k príbytku, dvadceti desk k strane polední, k vetru polednímu.
\par 24 A ctyridceti podstavku stríbrných udelal pode dvadceti desk, dva podstavky pod jednu dsku ke dvema cepum jejím, a dva podstavky pod dsku druhou pro dva cepy její.
\par 25 Na druhé pak strane príbytku k strane pulnocní udelal dvadceti desk,
\par 26 A ctyridceti podstavku jejich stríbrných, dva podstavky pod jednu dsku a dva podstavky pod dsku druhou.
\par 27 Na strane pak príbytku k západu udelal šest desk.
\par 28 Dve dsky udelal v úhlech po obou stranách príbytku.
\par 29 A byly spojené po spodu a tolikéž spojené svrchu k jednomu kruhu; tak udelal po obou stranách ve dvou úhlech.
\par 30 A tak bylo osm desk a podstavku jejich stríbrných šestnácte, dva podstavkové pod každou dskou.
\par 31 Nadelal i svlaku z drívi setim, pet ke dskám strane príbytku jedné,
\par 32 A pet svlaku ke dskám druhé strany príbytku, a pet svlaku ke dskám strany príbytku západní, dosahujících k obema úhlum.
\par 33 A svlak prostrední udelal, aby šel po prostredku desk od jednoho kraje k druhému.
\par 34 Dsky pak obložil zlatem a kruhy jejich udelal z zlata, aby v nich svlaky byly; a obložil svlaky zlatem.
\par 35 Udelal také oponu z postavce modrého a šarlatu, a z cervce dvakrát barveného, a z bílého hedbáví presukovaného; dílem remeslným udelal ji s figurami cherubínu.
\par 36 A udelal pro ni ctyri sloupy z dríví setim, a obložil je zlatem; hákové pak jejich byli zlatí, a slil k nim ctyri podstavky stríbrné.
\par 37 Udelal také zastrení ke dverum stánku z postavce modrého, z šarlatu a z cervce dvakrát barveného a z bílého hedbáví presukovaného, dílem krumpérským,
\par 38 A sloupu k tomu zastrení pet s háky jejich, (obložil pak makovice a prepásaní jich zlatem), a podstavku pet medených.

\chapter{37}

\par 1 Udelal také Bezeleel truhlu z dríví setim, jejíž dlouhost byla pul tretího lokte, a pul druhého lokte širokost, vysokost také pul druhého lokte.
\par 2 A obložil ji zlatem cistým vnitr i zevnitr, a udelal jí korunu zlatou vukol.
\par 3 Slil jí také ctyri kruhy zlaté ke ctyrem úhlum jejím, dva totiž kruhy po jedné strane její, a dva kruhy po druhé strane její.
\par 4 Zdelal i sochory z dríví setim a obložil je zlatem.
\par 5 A uvlékl sochory do kruhu po stranách truhly; aby na nich nošena byla truhla.
\par 6 Udelal také slitovnici z zlata cistého. Pul tretího lokte byla dlouhost její, a pul druhého lokte širokost její.
\par 7 Udelal i dva cherubíny z zlata, z taženého zlata udelal je na dvou koncích slitovnice,
\par 8 Cherubína jednoho na jednom konci a cherubína druhého na druhém konci; na slitovnici udelal cherubíny, na obou koncích jejích.
\par 9 Ti pak cherubínové meli krídla vztažená svrchu nad ní, zastírajíce krídly svými slitovnici; a tvári jejich byly obráceny jedna k druhé, k slitovnici byly obráceny tvári cherubínu.
\par 10 Udelal i stul z dríví setim. Dvou loket byla dlouhost jeho a na loket širokost jeho, pul druhého pak lokte vysokost.
\par 11 A obložil jej zlatem cistým, i korunu zlatou udelal mu vukol.
\par 12 Udelal mu i lištu dlani zšírí vukol, a udelal korunu zlatou okolo té lišty.
\par 13 Slil také k nemu ctyri kruhy zlaté a vpustil je do ctyr úhlu, kteríž byli na ctyrech nohách jeho.
\par 14 Proti té lište byli kruhové, skrze než by provlacováni byli sochorové k nošení stolu.
\par 15 Udelal i sochory z dríví setim, kteréžto obložil zlatem k nošení stolu.
\par 16 A zdelal nádoby, kteréž byly na stole, misy jeho a lžice jeho, a koflíky jeho, a prikryvadla k prikrývání z cistého zlata.
\par 17 Udelal také svícen z zlata cistého, z taženého zlata udelal svícen; sloupec jeho i prutové jeho, misky jeho, koule jeho i kvetové jeho z neho byli.
\par 18 Šest pak prutu bylo po stranách jeho, tri prutové s jedné strany jeho, a tri prutové s druhé strany jeho.
\par 19 Tri misky na zpusob pecky mandlové udelány byly na prutu jednom, a koule a kvet, a tri misky na zpusob pecky mandlové udelány na prutu druhém, a koule a kvet; tak i na jiných šesti prutech z svícnu vycházejících.
\par 20 Na svícnu také byly ctyri misky, udelané na zpusob mandlové pecky, a koule jeho i kvetové jeho.
\par 21 A byla koule pode dvema pruty z neho, a koule druhá pode dvema pruty z neho, a koule tretí pode dvema pruty z neho, a tak pod šesti pruty vycházejícími z neho.
\par 22 Koule jejich i prutové jejich z neho byli, všecko hned jedním tažením z zlata cistého.
\par 23 Udelal i lamp jeho sedm, i uteradla jeho, i nádoby k oharkum jeho z zlata cistého.
\par 24 Z centnére zlata cistého udelal jej se vším tím nádobím.
\par 25 Udelal také oltár pro kadení z dríví setim, lokte zdélí a lokte zšírí, ctverhraný, dvou pak loket zvýší; z neho byli rohové jeho.
\par 26 A obložil jej zlatem cistým, svrchek jeho i po stranách vukol, i rohy jeho; a udelal mu korunu zlatou vukol.
\par 27 Po dvou podobne kruzích zlatých udelal u neho, pod korunou ve dvou úhlech jeho, po dvou stranách jeho, skrze než by provlacováni byli sochorové, aby nošen byl na nich.
\par 28 A zdelal ty sochory z dríví setim, a obložil je zlatem.
\par 29 Nadelal také oleje pomazání svatého, a kadidla z vonných vecí, cistého, dílem apatykárským.

\chapter{38}

\par 1 Udelal také oltár k zápalu z dríví setim, peti loket zdélí a peti loket zšírí, ctverhraný, a trí loket zvýší.
\par 2 A zdelal mu rohy na ctyrech úhlech jeho; z neho byli rohové jeho, a obložil jej medí.
\par 3 Nadelal také všelijakých nádob k oltári, hrncu a lopat, a kotlíku a vidlicek, a nádob jeho k uhlí; všecka nádobí jeho udelal medená.
\par 4 Udelal k oltári i rošt mrežovaný, medený, pod okolkem oltáre dole, až do prostred neho.
\par 5 A slil ctyri kruhy na ctyrech krajích k roštu medenému, v nichž by sochorové bývali.
\par 6 Sochory pak udelal z dríví setim a obložil je medí.
\par 7 A uvlékl ty sochory do tech kruhu po obou stranách oltáre k nošení jeho na nich; prázdný z prken udelal jej.
\par 8 Udelal též umyvadlo medené, a podstavek jeho medený z zrcadel houfne pricházejících žen, kteréž pricházely ke dverím stánku úmluvy.
\par 9 Udelal také sín k strane polední; koltry ockovaté síne té z bílého hedbáví presukovaného na sto loket,
\par 10 Sloupu jejich dvadceti, a k nim podstavku dvadceti z medi, háky na sloupích, a prepásaní jejich z stríbra.
\par 11 Tolikéž k strane pulnocní koltry na sto loket, sloupu k nim dvadceti a podstavku jejich dvadceti z medi, háky na sloupích a prepásaní jejich z stríbra.
\par 12 K strane pak západní koltry ockovaté na padesáte loket, sloupu k nim deset a podstavku jejich deset, háky na sloupích a prepásaní jejich z stríbra.
\par 13 A v strane prední na východ padesáti loktu,
\par 14 Koltry ockovaté patnácti loktu byly pri strane jedné, sloupové k nim tri, a podstavkové jejich tri.
\par 15 A k strane druhé, u brány síne té, jakž tam, tak tuto, koltry ockovaté patnácti loktu, sloupové k nim tri a podstavkové jejich tri.
\par 16 Všecky koltry síne vukol ockovaté z bílého hedbáví presukovaného.
\par 17 Podstavkové pak sloupu z medi, hákové na sloupích a prepásaní jich z stríbra, a obložení makovic jejich z stríbra; všickni také sloupové síne prepásáni byli stríbrem.
\par 18 Zastrení pak brány sínce dílem krumpérským z postavce modrého a šarlatu, a cervce dvakrát barveného a hedbáví bílého presukovaného; dlouhost jeho dvadceti loktu, výsost pak šírky peti loktu, jako i jiných koltr síne ockovatých.
\par 19 A sloupové k ní ctyri a podstavkové jejich ctyri z medi, hákové jejich stríbrní, a obložení makovic jejich a prepásaní jejich z stríbra.
\par 20 Všickni pak kolíkové príbytku a síne vukol byli z medi.
\par 21 Tyto jsou veci vyctené k príbytku, príbytku svedectví, kteréž jsou vycteny podlé rozkázaní Mojžíšova, skrze Itamara, syna Arona kneze, k službe Levítu.
\par 22 A Bezeleel, syn Uri, syna Hur, z pokolení Juda, udelal všecky tyto veci, kteréž prikázal Hospodin Mojžíšovi;
\par 23 A s ním Aholiab, syn Achisamechuv z pokolení Dan, tesar a vtipný remeslník, a krumpér na modrém postavci a šarlatu, a cervci dvakrát barveném a kmentu.
\par 24 Všeho zlata vynaloženého na samo dílo, na všecko dílo svatyne, (bylo pak zlato obetované), devet a dvadceti centnéru, a sedm set tridceti lotu podlé váhy svatyne.
\par 25 Stríbra pak od tech, jenž náležejí ku poctu shromáždení, sto centnéru, a tisíc sedm set sedmdesáte pet lotu podlé váhy svatyne.
\par 26 Pul lotu z každé hlavy podlé váhy svatyne, ode všech jdoucích v pocet od dvadcíti let a výše, jichž bylo šestkrát sto tisíc, tri tisíce pet set a padesáte.
\par 27 A bylo sto centnéru stríbra k slévání podstavku svatyne a podstavku opony; sto podstavku ze sta centnéru, centnér do podstavku.
\par 28 A z tisíce sedmi set sedmdesáti peti lotu udelal háky na sloupy, a obložil makovice jejich a prepásal je.
\par 29 Medi pak obetované bylo sedmdesáte centnéru a dva tisíce a ctyri sta lotu.
\par 30 A udelal z ní podstavky ke dverum stánku svedectví a oltár medený, a rošt medený k nemu, a všecky nádoby oltáre,
\par 31 A podstavky síne vukol a podstavky brány síne, všecky také kolíky príbytku a všecky kolíky sínce vukol.

\chapter{39}

\par 1 Z modrého pak postavce a šarlatu a cervce dvakrát barveného udelali roucha k službe, k prisluhování v svatyni. Udelali i roucho svaté, kteréž by bylo Aronovi, jakž byl prikázal Hospodin Mojžíšovi.
\par 2 A udelal náramenník z zlata, postavce modrého a šarlatu, a cervce dvakrát barveného, a bílého hedbáví presukovaného.
\par 3 I nadelali plíšku zlatých, a nastríhali z nich nití, aby jimi províjeli skrze modrý postavec a šarlat, a cervec dvakrát barvený a bílé hedbáví, dílem remeslným.
\par 4 Náramky u neho udelali tak, aby se jeden s druhým spojiti mohl; na dvou krajích svých spojoval se.
\par 5 Prepásaní také náramenníka, kteréž bylo na nem, z týchž vecí bylo a týmž dílem, z zlata, postavce modrého a šarlatu a cervce dvakrát barveného, a bílého hedbáví presukovaného, jakž byl prikázal Hospodin Mojžíšovi.
\par 6 Prisadili i kamení onychinové, vložené a vsazené do zlata, rezané tak, jako ryty bývají peceti, s jmény synu Izraelských.
\par 7 A vložil je na vrchní kraje náramenníku, aby byli kamenové pro pamet na syny Izraelské, jakž byl prikázal Hospodin Mojžíšovi.
\par 8 Udelal i náprsník dílem remeslným, takovým dílem jako náramenník, z zlata, postavce modrého a šarlatu, a cervce dvakrát barveného, a bílého hedbáví presukovaného.
\par 9 Ctverhranatý byl; dvojnásobní udelali náprsník, na píd zdélí, a na píd zšírí, dvojnásobní.
\par 10 A vysadili jej ctyrmi rady kamení drahého porádkem tímto: Sardius, topazius a smaragdus v radu prvním;
\par 11 V radu pak druhém karbunkulus, zafir a jaspis;
\par 12 A v radu tretím linkurius, achates a ametyst;
\par 13 A v ctvrtém radu chrysolit, onychin a beryl, vložení a vsazení do zlata v svém porádku.
\par 14 Tech pak kamenu s jmény synu Izraelských bylo dvanácte vedlé jmen jejich, vyrytých, jako pecet ryta bývá, každý vedlé jména svého pro dvanáctero pokolení.
\par 15 Udelali i k náprsníku retízky jednostejné, dílem toceným z zlata cistého.
\par 16 Udelali také dva haklíky zlaté, a dva kroužky zlaté, a pripjali ty dva kroužky na dvou krajích náprsníku.
\par 17 A prostrcili dva retízky zlaté skrze dva kroužky na krajích náprsníku.
\par 18 Druhé pak dva konce dvou retízku vpjali do dvou tech haklíku, a dali je na vrchní kraje náramenníku po predu.
\par 19 Udelali též dva kroužky zlaté, kteréž dali na dva kraje náprsníku, na té obrube jeho, kteráž byla po strane náramenníka po spodu.
\par 20 Udelali ješte dva jiné kroužky zlaté, kteréž dali na dve strany náramenníka zespod po predu, proti spojení jeho, kteréž jest nad prepásaním náramenníka.
\par 21 I privázali náprsník od kroužku jeho k kroužkum toho náramenníka tkanicí hedbáví modrého, aby byl nad prepásaním náramenníka, a aby neodevstával náprsník od náramenníka, jakž byl prikázal Hospodin Mojžíšovi.
\par 22 Udelal také plášt náležející k náramenníku, dílem vytkávaným, všecken z postavce modrého,
\par 23 A díru u prostred plášte, jako díra v pancíri; okolek byl po kraji jejím vukol, aby se neroztrhl.
\par 24 Udelali také na podolku plášte jablka zrnatá z postavce modrého a šarlatu, a cervce dvakrát barveného, a bílého hedbáví presukovaného.
\par 25 Nadelali i zvonecku z zlata cistého, a zzavešovali zvonecky ty mezi jablky zrnatými u podolku plášte vukol, u prostred mezi jablky zrnatými;
\par 26 Zvoncek a jablko zrnaté, opet za tím zvoncek a jablko zrnaté u podolku plášte vukol, k užívání toho pri službe, jakž byl prikázal Hospodin Mojžíšovi.
\par 27 Potom udelali sukni z bílého hedbáví dílem vytkávaným, Aronovi a synum jeho.
\par 28 I cepici z bílého hedbáví, a klobouky ozdobné z bílého hedbáví, a košilky tenké z bílého hedbáví presukovaného,
\par 29 Pás také z bílého hedbáví presukovaného, a postavce modrého a šarlatu, a cervce dvakrát barveného, dílem krumpérským, jakž byl prikázal Hospodin Mojžíšovi,
\par 30 Udelali i plech koruny svaté z zlata cistého, a napsali na nem písmo dílem vyrývajících peceti: Svatost Hospodinu.
\par 31 A dali do neho tkanici z hedbáví modrého, aby privázán byl k cepici na hore, jakž byl prikázal Hospodin Mojžíšovi.
\par 32 A tak dokonáno jest všecko dílo príbytku stánku úmluvy; a ucinili synové Izraelští všecko, jakž prikázal Hospodin Mojžíšovi, tak ucinili,
\par 33 A prinesli príbytek ten k Mojžíšovi, stánek i všecka nádobí jeho, háky jeho, dsky jeho, svlaky jeho, i sloupy jeho a podstavky jeho,
\par 34 Prikrytí také z koží skopových na cerveno barvených, a prikrytí z koží jezevcích i oponu zastrení,
\par 35 Truhlu svedectví s sochory jejími i slitovnici,
\par 36 Stul, všecka nádobí jeho i chléb predložení,
\par 37 Svícen cistý, lampy jeho, lampy zporádané, i všecka nádobí jeho, i olej k svícení,
\par 38 Též oltár zlatý a olej pomazání a kadidlo z vonných vecí a zastrení ke dverum stánku,
\par 39 Oltár medený a rošt jeho medený, sochory jeho a všecka nádobí jeho, i umyvadlo a podstavek jeho,
\par 40 Ockovaté koltry síne a sloupy k nim s podstavky jejich, i zastrení k bráne té síne, provazy také její a kolíky i všecka nádobí k službe príbytku, k stánku úmluvy,
\par 41 Roucha k službe, k prisluhování v svatyni, roucho svaté Arona kneze i roucho synu jeho k konání úradu knežského.
\par 42 Vedlé všeho, což prikázal Hospodin Mojžíšovi, tak udelali synové Izraelští všecko to dílo.
\par 43 A videl Mojžíš všecko to dílo, a aj, udelali je, jakž byl prikázal Hospodin, tak udelali. I požehnal jim Mojžíš.

\chapter{40}

\par 1 Potom mluvil Hospodin k Mojžíšovi, rka:
\par 2 V den mesíce prvního, prvního dne téhož mesíce vyzdvihneš príbytek, stánek úmluvy.
\par 3 A postavíš tam truhlu svedectví a zastreš ji oponou.
\par 4 Vneseš i stul a zrídíš rád jeho, vneseš také svícen a rozsvítíš lampy jeho.
\par 5 Postavíš též oltár zlatý pro kadení naproti truhle svedectví, a zavesíš zastrení ve dverích príbytku.
\par 6 Potom postavíš oltár k zápalum prede dvermi príbytku stánku úmluvy.
\par 7 Postavíš také umyvadlo mezi stánkem úmluvy a mezi oltárem, do nehož naleješ vody.
\par 8 Naposledy vyzdvihneš sín vukol a zavesíš zastrení brány síne.
\par 9 Tedy vezmeš olej pomazání a pomažeš príbytku a všech vecí, kteréž v nem jsou, a posvetíš ho i všech nádob jeho, a bude svatý.
\par 10 Pomažeš i oltáre zápalu a všech nádob jeho a posvetíš oltáre, a budet oltár svatý.
\par 11 Pomažeš také umyvadla a podstavku jeho, a posvetíš ho.
\par 12 A pristoupiti kážeš Aronovi i synum jeho ke dverím stánku svedectví, a umyješ je vodou.
\par 13 A obleceš Arona v roucha svatá a pomažeš ho a posvetíš ho, aby úrad knežský konal prede mnou.
\par 14 Synum také jeho pristoupiti kážeš, a zoblácíš je v sukne.
\par 15 A pomažeš jich, tak jako jsi pomazal otce jejich, aby úrad knežský konali prede mnou, aby jim bylo pomazání jejich toto k knežství vecnému po rodech jejich.
\par 16 I ucinil Mojžíš tak. Všecko, jakž mu rozkázal Hospodin, tak ucinil.
\par 17 I stalo se mesíce prvního léta druhého, prvního dne mesíce, že vyzdvižen jest príbytek.
\par 18 Mojžíš tedy vyzdvihl príbytek a podložil podstavky jeho a postavil dsky jeho, a provlékl svlaky jeho, a vyzdvihl sloupy jeho.
\par 19 Potom postavil stánek v príbytku a dal prikrytí stánku svrchu na nej, jakož mu byl prikázal Hospodin.
\par 20 A vzav svedectví, vložil je do truhly, uvlékl také sochory k truhle a dal slitovnici svrchu na truhlu.
\par 21 I vnesl truhlu do príbytku a zavesil oponu zastrení, a zastrel truhlu svedectví, jakož byl prikázal Hospodin Mojžíšovi.
\par 22 Postavil i stul v stánku úmluvy k strane príbytku pulnocní, vne pred oponou.
\par 23 A zrídil na nem zporádaní chlebu pred Hospodinem, jakž prikázal Hospodin Mojžíšovi.
\par 24 A postavil svícen v stánku úmluvy naproti stolu, v strane príbytku ku poledni.
\par 25 A rozsvítil lampy pred Hospodinem, jakož prikázal Hospodin Mojžíšovi.
\par 26 Postavil také oltár zlatý v stánku úmluvy pred oponou,
\par 27 A kadil na nem kadidlem vonným, jakž prikázal Hospodin Mojžíšovi.
\par 28 Zavesil také zastrení dverí príbytku.
\par 29 A oltár zápalu postavil ke dverum príbytku stánku úmluvy, a obetoval na nem obeti zápalné a suché, jakož prikázal Hospodin Mojžíšovi.
\par 30 A postavil umyvadlo mezi stánkem úmluvy a mezi oltárem, do nehož nalil vody k umývání.
\par 31 A umýval z neho Mojžíš, Aron a synové jeho ruce své i nohy své.
\par 32 Když vcházeli do stánku úmluvy, a když pristupovali k oltári, umývali se, jakož prikázal Hospodin Mojžíšovi.
\par 33 Naposledy vyzdvihl sín vukol príbytku a oltáre, a zavesil zastrení brány síne. A tak dokonal Mojžíš dílo to.
\par 34 Tedy prikryl oblak stánek úmluvy, a sláva Hospodinova naplnila príbytek.
\par 35 A nemohl Mojžíš vjíti do stánku úmluvy; nebo byl nad ním oblak, a sláva Hospodinova naplnila príbytek.
\par 36 Když pak odnášel se oblak s príbytku, brali se synové Izraelští po všech taženích svých.
\par 37 Pakli se neodnášel oblak, nehýbali se až do dne, v nemž se zdvihl.
\par 38 A byl oblak Hospodinuv nad príbytkem ve dne, a ohen býval v noci na nem, pred ocima všeho domu Izraelského ve všech taženích jejich.

\end{document}