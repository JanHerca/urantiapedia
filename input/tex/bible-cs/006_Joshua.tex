\begin{document}

\title{Jozue}

\chapter{1}

\par 1 Stalo se pak po smrti Mojžíše, služebníka Hospodinova, že mluvil Hospodin k Jozue, synu Nun, služebníku Mojžíšovu, rka:
\par 2 Mojžíš, služebník muj, umrel; protož nyní vstan, prejdi Jordán tento, ty i všecken lid tento, a jdi do zeme, kterouž já dávám synum Izraelským.
\par 3 Každé místo, po kterémž šlapati budete nohama svýma, dal jsem vám, jakož jsem mluvil k Mojžíšovi.
\par 4 Od poušte a Libánu toho až k rece veliké, rece Eufrates, všecka zeme Hetejská až do more velikého na západ slunce bude pomezí vaše.
\par 5 Neostojí žádný pred tebou po všecky dny života tvého; jakož jsem byl s Mojžíšem, tak budu s tebou; nenechám tebe samého, aniž te opustím.
\par 6 Posilniž se a zmužile se mej, nebo ty uvedeš v dedictví lidu tomuto zemi, kterouž jsem s prísahou zaslíbil otcum jejich, že ji dám jim.
\par 7 Toliko posiln se a udatne sobe pocínej, abys ostríhal a cinil všecko podlé zákona, kterýž prikázal tobe Mojžíš, služebník muj; neuchyluj se od neho na pravo ani na levo, abys byl opatrný ve všem, k cemuž se obrátíš.
\par 8 Neodejdet kniha zákona tohoto od úst tvých, ale premyšlovati budeš o nem dnem i nocí, abys ostríhal a cinil všecko podlé toho, což psáno jest v nem; nebo tehdáž štastný budeš na cestách svých, a tehdáž opatrný budeš.
\par 9 Zdaliž jsem neprikázal tobe, rka: Posilni se a zmužile se mej, neboj se, ani lekej, nebo s tebou jest Hospodin Buh tvuj, kamž se koli obrátíš.
\par 10 Tedy prikázal Jozue správcum lidu, rka:
\par 11 Projdete skrze tábor a prikažte lidu, rkouce: Nachystejte sobe potravy, nebo po trech dnech pujdete pres Jordán tento, abyste vejdouce, opanovali zemi, kterouž Hospodin Buh váš dává vám k dedicnému vládarství.
\par 12 Rubenovu pak pokolení a Gádovu, a polovici pokolení Manassesova mluvil Jozue, rka:
\par 13 Pomnete na to, co vám prikázal Mojžíš služebník Hospodinuv, když rekl: Hospodin Buh váš zpusobil vám odpocinutí, že vám dal zemi tuto.
\par 14 Ženy vaše, dítky vaše i dobytek váš necht zustanou v zemi, kterouž dal vám Mojžíš s této strany Jordánu, vy pak jdete vojensky zporádaní pred bratrími svými, kteríkoli jste muži silní, a pomáhejte jim,
\par 15 Dokudž nedá odpocinutí Hospodin bratrím vašim jako i vám, a neobdrží dedicne také i oni zeme, kterouž Hospodin Buh váš dává jim. Potom navrátíte se do zeme dedictví svého, kterouž dal vám Mojžíš, služebník Hospodinuv, s této strany Jordánu, k východu slunce, a dedicne vlásti jí budete.
\par 16 I odpovedeli k Jozue, rkouce: Všecko, což jsi nám rozkázal, uciníme, a kamžkoli pošleš nás, pujdeme.
\par 17 Rovne jakž jsme poslouchali Mojžíše, tak poslouchati budeme tebe; jediné necht jest Hospodin Buh tvuj s tebou, jako byl s Mojžíšem.
\par 18 Kdo by koli odporný byl rozkázaní tvému, a neposlouchal by recí tvých ve všech vecech, kteréž bys prikázal jemu, umret; toliko posiln se a zmužile se mej.

\chapter{2}

\par 1 Poslal pak byl Jozue, syn Nun, z Setim dva muže špehére tajne, rka: Jdete, shlédnete zemi, zvlášt Jericho. I šli a vešli do domu ženy nevestky, jejíž jméno bylo Raab, a odpocinuli tu.
\par 2 Ale jakž oznámeno bylo králi Jericha a povedíno: Aj, muži prišli sem této noci z synu Izraelských, aby shlédli zemi,
\par 3 Tedy poslal král Jericha k Raab, rka: Vyved muže, kteríž prišli k tobe, a vešli do domu tvého, nebo k vyšpehování vší zeme prišli.
\par 4 (Pojavši pak žena ty dva muže, skryla je.) Kteráž odpovedela: Pravda jest, prišlit jsou ke mne muži, ale nevedela jsem, odkud jsou.
\par 5 A když bránu zavírati meli v soumrak, vyšli muži ti, a nevím, kam jsou šli; honte je rychle, nebo dostihnete jich.
\par 6 Ale ona rozkázala jim vstoupiti na strechu, a skryla je v pazderí lneném, kteréž byla skladla na streše.
\par 7 Muži pak vyslaní honili je cestou Jordánskou až k brodum; a zavrína jest brána, jakž vyšli ti, kteríž je honili.
\par 8 Prvé pak, než usnuli špehéri, vstoupila k nim ona na strechu,
\par 9 A rekla mužum tem: Vím, že Hospodin dal vám zemi tuto, nebo pripadl na nás strach váš, tak že oslábli všickni obyvatelé zeme pred tvárí vaší.
\par 10 Slyšeli jsme zajisté, jak vysušil Hospodin vody more Rudého pred tvárí vaší, když jste vyšli z Egypta, a co jste ucinili dvema králum Amorejským, kteríž byli za Jordánem, Seonovi a Ogovi, kteréž jste zahladili jako proklaté.
\par 11 Což když jsme uslyšeli, osláblo srdce naše, aniž zustává více v kom duše pred tvárí vaší, nebo Hospodin Buh váš jest Buh na nebi svrchu, i na zemi dole.
\par 12 Nyní tedy, prosím, prisáhnete mi skrze Hospodina, ponevadž jsem ucinila vám milosrdenství, že i vy uciníte s domem otce mého milosrdenství, a dáte mi znamení jisté,
\par 13 Že budete živiti otce mého i matku mou, bratrí mé i sestry mé a všecko, cožkoli jejich jest, a vysvobodíte duše naše od smrti.
\par 14 I rekli jí muži ti: Duše naše za vás necht jsou na smrt; jestliže však nepronesete reci naší této, takt se jiste stane, že když nám dá Hospodin zemi tuto, tehdy uciníme s tebou milosrdenství a pravdu.
\par 15 Protož spustila je oknem po provazu; nebo dum její byl pri zdi mestské, a na zdi ona bydlila.
\par 16 I rekla jim: Jdete k této hore, aby se nepotkali s vámi, kteríž vás honí, a krejte se tam za tri dni, až se oni zase navrátí, a potom pujdete cestou svou.
\par 17 Tedy rekli muži k ní: Prosti budeme od této prísahy tvé, kterouž jsi zavázala nás,
\par 18 Jestliže, když vejdeme do zeme, neuvážeš provázku tohoto z nití hedbáví cerveného dvakrát barveného v tomto okne, jímž jsi nás spustila, také otce svého a matky své, i bratrí svých, a všeho domu otce svého neshromáždíš-li k sobe do domu.
\par 19 Také kdo by vyšel ven ze dverí domu tvého, krev jeho bude na hlavu jeho, ale my budeme bez viny; a každého toho, kdož bude s tebou v dome, jestliže by se kdo rukou dotekl, krev jeho na hlavy naše pripadni.
\par 20 Jestliže pak proneseš tuto rec naši, budeme prosti od prísahy tvé, kterouž jsi nás zavázala.
\par 21 Odpovedela ona: Necht jest tak, jakž jste rekli. I propustila je a odešli; i uvázala provázek z hedbáví cerveného dvakrát barveného v tom okne.
\par 22 Odšedše pak, prišli na horu, a pobyli tam za tri dni, dokudž se nenavrátili, kteríž je honili; nebo jich hledali ti, kteríž je honili po všech cestách, ale nic nenalezli.
\par 23 Ti pak dva muži sšedše s hory, navrátili se a preplavili se pres Jordán; i prišli k Jozue, synu Nun, a vypravovali jemu všecko, co se s nimi dálo.
\par 24 A rekli k Jozue: Dalt jest Hospodin v ruce naše všecku zemi, nebo se zdesili všickni obyvatelé zeme tvári naší.

\chapter{3}

\par 1 Vstal pak Jozue velmi ráno, a hnuvše se z Setim, prišli až k Jordánu, on i všickni synové Izraelští, a prenocovali tu, prvé nežli šli pres nej.
\par 2 I stalo se, že tretího dne správcové šli prostredkem stanu,
\par 3 A prikazovali lidu, rkouce: Když uzríte truhlu smlouvy Hospodina Boha svého a kneží Levítské, ani ji nesou, vy také hnete se s místa svého a pujdete za ní,
\par 4 (A však místo mezi vámi a mezi ní bude okolo dvou tisíc loket míry obecné, nepribližujte se k ní), abyste videli cestu, kterouž byste jíti meli; nebo nešli jste tou cestou prvé.
\par 5 Rekl pak byl Jozue lidu: Posvettež se, zítra zajisté uciní Hospodin divné veci mezi vámi.
\par 6 Potom rekl Jozue knežím temi slovy: Vezmete truhlu smlouvy, a jdete pred lidem. I vzali truhlu smlouvy a brali se pred lidem.
\par 7 Nebo rekl byl Hospodin k Jozue: V tento den zacnu tebe zvelebovati pred ocima všeho Izraele, aby poznali, že jakož jsem byl s Mojžíšem, tak budu s tebou.
\par 8 Protož ty prikaž knežím, kteríž nosí truhlu smlouvy, a rci: Když vejdete na kraj vody Jordánské, zastavte se v Jordáne.
\par 9 Rekl také Jozue synum Izraelským: Pristupte sem, a slyšte slova Hospodina Boha vašeho.
\par 10 I rekl Jozue: Po tomto poznáte, že Buh silný živý jest u prostred vás, a že konecne vyžene od tvári vaší Kananejského, Hetejského, Hevejského, Ferezejského, Gergezejského, Amorejského a Jebuzejského:
\par 11 Aj, truhla smlouvy Panovníka vší zeme pujde pred vámi pres Jordán.
\par 12 Protož nyní vezmete sobe dvanácte mužu z pokolení Izraelských, po jednom muži z každého pokolení.
\par 13 I bude, že hned jakž se zastaví kneží, nesoucí truhlu Hospodina Panovníka vší zeme, u vode Jordánské, vody Jordánu rozdelí se, a vody tekoucí s vrchu zustanou v jedné hromade.
\par 14 Stalo se tedy, když se bral lid z stanu svých, aby šli pres Jordán, a kneží, kteríž nesli truhlu smlouvy, pred nimi,
\par 15 A když ti, kteríž nesli truhlu, prišli až k Jordánu, a kneží, nesoucí truhlu, omocili nohy v kraji vod: (Jordán pak rozvodnuje se a vystupuje ze všech brehu svých v každý cas žne),
\par 16 Že se zastavily vody, kteréž s hury pricházely, a shrnuly se v hromadu jednu velmi daleko od Adam mesta, kteréž leží k strane Sartan, kteréž pak odcházely k mori dolu, k mori slanému sešly a sbehly, a lid prešel naproti Jerichu.
\par 17 Stáli pak kneží, kteríž nesli truhlu smlouvy Hospodinovy na suše u prostred Jordánu, nehýbajíce se, (a všecken Izrael šel po suše), až se lid všecken prepravil pres Jordán.

\chapter{4}

\par 1 I stalo se, když již všecken národ prešel Jordán, (nebo byl rekl Hospodin k Jozue, rka:
\par 2 Vezmete sobe z lidu dvanácte mužu, po jednom muži z každého pokolení,
\par 3 A prikažte jim, rkouce: Vezmete sobe odsud z prostred Jordánu, s místa, na nemž stály nohy knežské nepohnute, dvanácte kamenu, a vyneste je s sebou, a skladte v ležení, v kterémž pres tuto noc pozustanete),
\par 4 Že povolav Jozue dvanácti mužu, kteréž k tomu zrídil z synu Izraelských, po jednom muži z každého pokolení,
\par 5 I rekl jim Jozue: Jdete pred truhlou Hospodina Boha svého do prostred Jordánu a vezmete sobe každý kámen jeden na rameno své, vedlé poctu pokolení synu Izraelských,
\par 6 Aby to bylo na znamení mezi vámi. Když by potom tázali se synové otcu svých, rkouce: K cemu jsou vám ti kamenové?
\par 7 Odpovíte jim, že se rozdelily vody Jordánské pred truhlou smlouvy Hospodinovy, (když, pravím, šla pres Jordán, rozdelily se vody Jordánské), i zustávají kamenové tito na památku synum Izraelským až na veky.
\par 8 I ucinili synové Izraelští tak, jakž prikázal Jozue, a vzali dvanácte kamenu z prostredku Jordánu, jakož mluvil Hospodin k Jozue, vedlé poctu pokolení synu Izraelských, a prinesli je s sebou na první stanovište, a tu je složili.
\par 9 Jozue také vyzdvihl dvanácte kamenu u prostred Jordánu na míste, kdež stály nohy kneží nesoucích truhlu smlouvy, a byli tam až do tohoto dne.
\par 10 Kneží pak, jenž nesli truhlu, stáli u prostred Jordánu, dokudž se nevyplnilo všeliké slovo, kteréž prikázal Hospodin Jozue, aby oznámil lidu podlé všeho, jakž byl prikázal Mojžíš Jozue. I pospíšil lid a prešli.
\par 11 Když pak již prešel všecken lid, prešla také truhla Hospodinova i kneží, an lid na to hledí.
\par 12 Prešli i synové Rubenovi a synové Gádovi, a polovice pokolení Manassesova, vojensky zporádaní pred syny Izraelskými, jakož byl mluvil k nim Mojžíš.
\par 13 Okolo ctyridcíti tisícu odených bojovníku šlo pred Hospodinem k boji na roviny Jericha.
\par 14 V ten den zvelebil Hospodin Jozue pred ocima všeho Izraele. I báli se ho, jako se báli Mojžíše po všecky dny života jeho.
\par 15 Mluvil pak Hospodin k Jozue, rka:
\par 16 Prikaž knežím, kteríž nesou truhlu svedectví, at vystoupí z Jordánu.
\par 17 I prikázal Jozue knežím, rka: Vystupte z Jordánu.
\par 18 Když pak vystupovali kneží, jenž nesli truhlu smlouvy Hospodinovy, z prostred Jordánu, jakž jen vytrhli kneží nohy na sucho, navrátily se vody Jordánské k místu svému, a tekly predce jako i prvé ve všech brezích svých.
\par 19 Vyšed pak lid z Jordánu desátého dne mesíce prvního, položili se v Galgala k strane východní Jericha;
\par 20 A dvanácte kamenu tech, kteréž vynesli z Jordánu, postavil Jozue v Galgala.
\par 21 A mluvil k synum Izraelským takto: Když by se otázali potom synové vaši otcu svých, rkouce: Co ti kamenové znamenají?
\par 22 Oznámíte synum svým a díte: Po suše prešel Izrael Jordán tento.
\par 23 Nebo vysušil Hospodin Buh váš vody Jordánské pred tvárí vaší, až jste prešli, jakož ucinil Hospodin Buh váš mori Rudému, kteréž vysušil pred tvárí naší, až jsme prešli,
\par 24 Aby poznali všickni národové zeme ruku Hospodinovu, že silná jest, a abyste se báli Hospodina Boha vašeho po všecky dny.

\chapter{5}

\par 1 Stalo se pak, když uslyšeli všickni králové Amorejští, kteríž bydlili za Jordánem k západu, a všickni králové Kananejští, kteríž bydlili pri mori, že vysušil Hospodin vody Jordánské pred syny Izraelskými, dokudž ho neprešli: zemdlelo srdce jejich, aniž zustalo více v nich duše pred tvárí synu Izraelských.
\par 2 Toho casu rekl Hospodin k Jozue: Udelej sobe nože ostré, a obrež opet syny Izraelské po druhé.
\par 3 I udelal sobe Jozue nože ostré, a obrezal syny Izraelské na pahrbku Aralot.
\par 4 Tato pak jest prícina, pro kterouž je obrezal Jozue, že všecken lid, kterýž byl vyšel z Egypta, pohlaví mužského, všickni muži bojovní zemreli byli na poušti na ceste po vyjití z Egypta.
\par 5 Nebo obrezán byl všecken lid, kterýž byl vyšel, ale žádného z toho lidu, kterýž se zrodil na poušti, na ceste po vyjití z Egypta, neobrezali.
\par 6 (Nebo ctyridceti let chodili synové Izraelští po poušti, dokudž nezahynul všecken národ mužu bojovných, kteríž byli vyšli z Egypta, ješto neposlouchali hlasu Hospodinova, jimžto zaprisáhl Hospodin, že neukáže jim zeme, kterouž s prísahou zaslíbil dáti otcum jejich, že ji nám dá, zemi oplývající mlékem a strdí.)
\par 7 Ale syny jejich, kteréž postavil na místo jejich, ty obrezal Jozue, že byli neobrezaní; nebo žádný jich neobrezoval na ceste.
\par 8 Když pak byl všecken lid obrezán, zustali na míste svém v ležení, dokudž se nezhojili.
\par 9 I rekl Hospodin k Jozue: Dnes jsem odjal pohanení Egyptské od vás. A nazval jméno místa toho Galgal až do tohoto dne.
\par 10 Když pak ležení meli synové Izraelští v Galgala, slavili velikunoc ctrnáctého dne toho mesíce u vecer, na rovinách Jericha.
\par 11 I jedli z úrod té zeme na zejtrí po velikonoci chleby nekvašené, a pražmu téhož dne.
\par 12 I prestala manna na zejtrí, když jedli z obilé té zeme, a již více nemeli synové Izraelští manny, ale jedli z úrod zeme Kananejské toho roku.
\par 13 Stalo se pak, když byl Jozue na poli Jericha, že pozdvihl ocí svých a videl, an muž stojí naproti nemu, maje v ruce mec dobytý. I šel Jozue k nemu, a rekl jemu: Jsi-li náš, cili neprátel našich?
\par 14 I odpovedel: Nikoli, ale já jsem kníže vojska Hospodinova, a nyní jsem prišel. I padl Jozue tvárí svou na zem, a pokloniv se, rekl jemu: Co pán muj chce mluviti služebníku svému?
\par 15 Tedy odpovedel kníže vojska Hospodinova k Jozue: Szuj obuv svou s noh svých, nebo místo, na nemž stojíš, svaté jest. I ucinil tak Jozue.

\chapter{6}

\par 1 Jericho pak velmi pilne zavríno bylo pro strach synu Izraelských, a žádný nevycházel ven, ani tam nevcházel.
\par 2 I rekl Hospodin k Jozue: Aj, dal jsem v ruku tvou Jericho, a krále jeho s silnými muži jeho.
\par 3 Protož obcházeti budete mesto všickni muži bojovní, okolo mesta chodíce, jednou za den; tak uciníš po šest dní.
\par 4 Kneží pak sedm at nesou sedm trub z rohu beraních pred truhlou; dne pak sedmého obejdete mesto sedmkrát, a kneží troubiti budou v trouby.
\par 5 A když zdlouha troubiti budou na roh beraní, jakž nejprvé uslyšíte hlas trouby, zkrikne všecken lid krikem velikým, i oborí se zed mestská na míste svém, a vejde lid do mesta, jeden každý proti místu, kdež stál.
\par 6 Tedy povolav Jozue, syn Nun, kneží, rekl jim: Vezmete truhlu smlouvy, a sedm kneží at vezmou sedm trub beraních pred truhlou Hospodinovou.
\par 7 Rekl také lidu: Jdete a obejdete mesto, a zbrojní at jdou pred truhlou Hospodinovou.
\par 8 A když to oznámil Jozue lidu, sedm kneží, nesouce sedm trub beraních, šli pred truhlou Hospodinovou, a troubili v trouby; truhla také smlouvy Hospodinovy brala se za nimi.
\par 9 Zbrojní pak šli pred knežími, kteríž troubili na trouby, a ostatní šli za truhlou, jdouce a v trouby troubíce.
\par 10 (Lidu pak byl prikázal Jozue, rka: Nebudete kriceti, ani slyšán bude hlas váš, ani vyjde slovo z úst vašich až do dne toho, v nemž reknu vám: Kricte, i budete kriceti.)
\par 11 Tedy obešla truhla Hospodinova mesto vukol jednou, a navrátili se do stanu a zustali v nich.
\par 12 Opet povstal Jozue ráno a kneží nesli truhlu Hospodinovu.
\par 13 Sedm pak kneží, nesouce sedm trub beraních, predcházeli truhlu Hospodinovu, jdouce, a troubili v trouby; a zbrojní šli pred nimi, a ostatní šli za truhlou Hospodinovou, když kneží šli, v trouby troubíce.
\par 14 I obešli mesto druhého dne opet, a navrátili se do stanu. Tak cinili po šest dní.
\par 15 V den pak sedmý vstali, jakž zasvitávalo, a obešli mesto týmž zpusobem sedmkrát; toliko toho dne obešli mesto sedmkrát.
\par 16 Stalo se pak, když po sedmé obcházeli, a kneží v trouby troubili, rekl Jozue lidu: Krictež již, dalt jest Hospodin vám mesto.
\par 17 A budiž to mesto proklaté, ono i všecky veci, kteréž v nem jsou, Hospodinu; toliko Raab nevestka at jest živa, ona i všickni, kteríž by s ní byli v dome, nebo skryla posly, kteréž jsme byli poslali.
\par 18 Avšak vystríhejte se od proklatého, abyste i vy nebyli ucineni proklatí, berouce z proklatých vecí, a uvedli byste stany Izraelské v prokletí, a zkormoutili byste je.
\par 19 Všecko pak stríbro a zlato, a nádoby medené a železné, svaté bude Hospodinu; na poklad Hospodinu složeno bude.
\par 20 Tedy kricel lid, když zatroubili v trouby. Nebo když slyšel lid hlas trub, kriceli i oni krikem velikým, i oborila se zed na míste svém. Tedy všel lid do mesta, jeden každý proti místu, kdež stál. I vzali je.
\par 21 A pohubili ostrostí mece jako proklaté všecko, což bylo v meste, od muže až do ženy, od dítete až do starce, a až do vola, dobytcete i osla.
\par 22 Dvema pak mužum, kteríž shlédli zemi, rekl Jozue: Vejdete do domu ženy nevestky, a vyvedte ji odtud, i všecky veci, kteréž má, jakož jste jí prisáhli.
\par 23 I všedše mládenci špehéri, vyvedli Raab a otce jejího, i matku její a bratrí její i všecko, což mela, a všecku rodinu její vyvedli, a nechali jich vne za stany Izraelskými.
\par 24 Mesto pak spálili ohnem, i všecko, což v nem bylo, stríbro však a zlato a nádoby medené a železné složili na poklad v dome Hospodinove.
\par 25 Raab také nevestku, a dum otce jejího i všecko, což mela, živé zustavil Jozue; a bydlila v lidu Izraelském až do tohoto dne, nebo skryla posly, kteréž poslal Jozue k shlédnutí Jericha.
\par 26 Toho casu vydal klatbu Jozue, rka: Zlorecený bud pred Hospodinem muž ten, kterýž by povstal, aby stavel mesto Jericho. V prvorozeném svém založí je, a v nejmenším postaví brány jeho.
\par 27 Byl pak Hospodin s Jozue, a rozhlásila se povest o nem po vší zemi.

\chapter{7}

\par 1 Zhrešili pak synové Izraelští prestoupením pri vecech proklatých; nebo Achan, syn Charmi, syna Zabdi, syna Záre, z pokolení Juda, vzal neco z vecí proklatých. Procež rozpálila se prchlivost Hospodinova na syny Izraelské.
\par 2 Nebo když poslal Jozue muže nekteré od Jericha do Hai, kteréž bylo blízko Betaven, k východní strane Bethel, a mluvil k nim, rka: Vstupte a shlédnete zemi, vstoupili tedy muži ti, a shlédli Hai.
\par 3 Kterížto navrátivše se k Jozue, rekli jemu: Necht nevstupuje všecken lid tento, okolo dvou tisíc mužu neb okolo trí tisícu mužu necht vstoupí, a dobudout Hai; neobtežuj všeho lidu, vysílaje jej tam, nebo málo jest onechno.
\par 4 Tedy vytáhlo jich z lidu tam okolo trí tisíc mužu, a utekli pred muži mesta Hai.
\par 5 I zabili z nich obyvatelé Hai okolo tridcíti a šesti mužu, a honili je od brány až k Sabarim, a porazili je, když s vrchu utíkali. I rozpustilo se srdce lidu, a bylo jako voda.
\par 6 Tedy Jozue roztrhl roucho své, a padl tvárí svou na zem pred truhlou Hospodinovou, a ležel až do vecera, ano i starší Izraelští, a sypali prach na hlavy své.
\par 7 I rekl Jozue: Ach, Panovníce Hospodine, proc jsi kdy prevedl lid tento pres Jordán, abys nás vydal v ruku Amorejského, tak aby nás zahubil? Ó kdybychom byli radeji zustali za Jordánem.
\par 8 Ó Pane, což mám ríci, když již lid Izraelský utíká pred neprátely svými?
\par 9 Nebo uslyšíce Kananejští a všickni obyvatelé zeme, obklící nás vukol, a vyhladí jméno naše z zeme. I což to uciníš jménu svému velikému?
\par 10 I rekl Hospodin k Jozue: Vstan. Proc jsi padl na tvár svou?
\par 11 Zhrešil Izrael, tak že smlouvu mou prestoupili, kterouž jsem jim prikázal; nebo jsou i vzali z vecí proklatých, k tomu i ukradli, také i sklamali, nad to i odložili mezi nádobí svá.
\par 12 Protož nebudou moci synové Izraelští ostáti pred neprátely svými, utíkati budou pred neprátely svými, nebo poškvrnili se vecí proklatou. Nebudut více s vámi, lec vyhladíte prokletí to z prostredku svého.
\par 13 Vstan, posvet lidu a rci: Posvette se k zítrku; nebo takto praví Hospodin Buh Izraelský: Vec proklatá jest u prostred tebe, Izraeli, nebudeš moci ostáti pred neprátely svými, dokudž neodejmeš prokletí toho z prostredku svého.
\par 14 Protož pristupovati budete ráno po pokoleních svých, a pokolení, kteréž ukáže Hospodin, pristupovati bude po rodech, takž rod, kterýž ukáže Hospodin, pristupovati bude po domích, a dum, kterýž ukáže Hospodin, pristupovati bude po osobách.
\par 15 Kdož pak postižen bude vinný vecí proklatou, ohnem spálen bude, on i všecko, což jeho jest, proto že prestoupil smlouvu Hospodinovu, a že ucinil nešlechetnost v Izraeli.
\par 16 Vstav tedy Jozue ráno, kázal pristupovati Izraelovi po pokoleních jejich. I postiženo jest pokolení Judovo.
\par 17 Tedy když kázal pristupovati celedem Juda, postižena jest celed Záre. Potom kázal pristupovati celedi Záre po osobách, a postižen jest Zabdi.
\par 18 I kázal pristupovati domu jeho po osobách, i postižen jest Achan, syn Charmi, syna Zabdi, syna Záre z pokolení Judova.
\par 19 I rekl Jozue Achanovi: Synu muj, dej, prosím, chválu Hospodinu Bohu Izraelskému, a vyznej se jemu, a oznam mi aspon již, co jsi ucinil, netaj již toho prede mnou.
\par 20 Tedy odpovídaje Achan k Jozue, rekl: Pravdat jest, já jsem zhrešil proti Hospodinu Bohu Izraelskému, tak že to a toto jsem ucinil.
\par 21 Videl jsem mezi loupeží plášt jeden Babylonský pekný, a dve ste lotu stríbra, a prut zlatý jeden, padesáte lotu ztíží, cehož požádav, vzal jsem to, a aj, jsou ty veci skryté v zemi prostred stanu mého, a stríbro pod tím.
\par 22 Tedy poslal Jozue posly, kteríž beželi do stanu, a aj, bylo to skryto v stanu jeho, a stríbro pod tím.
\par 23 A vzavše to z stanu, prinesli k Jozue a ke všechnem synum Izraelským, a položili ty veci pred Hospodinem.
\par 24 Vzav tedy Jozue a všecken Izrael s ním Achana, syna Záre, a stríbro i plášt, i prut zlatý, i syny jeho, i dcery jeho, voly a osly, dobytek i stan jeho i všecko, což mel, vyvedli je do údolí Achor.
\par 25 Kdež rekl Jozue: Proc jsi zkormoutil nás? Zkormutiž tebe Hospodin v tento den. I uházel jej všecken lid kamením, a spálili je ohnem, ukamenovavše je kamením.
\par 26 Potom nametali nan hromadu kamení velikou, kteráž trvá až do tohoto dne; a tak odvrácen jest Hospodin od hnevu prchlivosti své. Protož nazváno jest jméno místa toho údolí Achor, až do dnešního dne.

\chapter{8}

\par 1 I rekl Hospodin k Jozue: Neboj se, ani se strachuj; pojmi s sebou všecken lid bojovný, a vstana, táhni k Hai. Aj, dal jsem v ruku tvou krále Hai a lid jeho, mesto i zemi jeho.
\par 2 A uciníš Hai a králi jeho, jako jsi ucinil Jerichu a králi jeho, loupež však a dobytky jeho rozbitujete mezi sebe. Zdelejž sobe zálohy k mestu po zadu.
\par 3 Tedy vstal Jozue a všecken lid bojovný, aby táhli k Hai. I vybral Jozue tridcet tisícu mužu velmi silných, a predeslal je v noci.
\par 4 A prikázal jim, rka: Šetrtež vy, kteríž udeláte zálohy mestu po zadní strane mesta, abyste nebyli príliš daleko od neho, ale budte všickni pohotove.
\par 5 Já pak i všecken lid, kterýž se mnou jest, pritáhneme k mestu. A když oni nám vyjdou vstríc, jako prvé, utíkati budeme pred nimi.
\par 6 Tedy honiti budou nás, až bychom poodvedli jich od mesta, (nebo reknou: Utíkají pred námi, jako prvé), utíkajíce pred nimi.
\par 7 Vy mezi tím vyskocíte z záloh, a vyženete ostatní obyvatele mesta, nebo dá je Hospodin Buh váš v ruku vaši.
\par 8 A když vezmete mesto, zapálíte je ohnem, podlé slova Hospodinova uciníte; šetrtež toho, což jsem prikázal vám.
\par 9 I poslal Jozue, a oni šli k zálohám, a zustali mezi Bethel a Hai, od západní strany Hai; Jozue pak zustal té noci u prostred lidu.
\par 10 Potom vstav Jozue velmi ráno, sectl lid, i bral se napred, on a starší Izraelští pred lidem k Hai.
\par 11 Všecken také lid bojovný, kterýž byl s ním, táhnouce, priblížili se, až prišli naproti mestu, a položili se po strane pulnocní Hai; údolí pak bylo mezi nimi a mezi Hai.
\par 12 Vzal pak byl okolo peti tisíc mužu, kteréž postavil v zálohách mezi Bethel a Hai, od západní stany mesta.
\par 13 I priblížil se lid, totiž všecko vojsko, kteréž bylo od pulnocní strany mesta, a kteríž byli v zálohách jeho od západní strany mesta; a tak vtáhl Jozue noci té do prostred údolí.
\par 14 I stalo se, že když je uzrel král Hai, pospíšili, a ráno vstavše, vyšli lidé mesta vstríc Izraelovi k boji, on i všecken lid jeho toho casu pred rovinu; nevedel pak, že zálohy udelány byly jemu po zadu mesta.
\par 15 I postoupil Jozue a všecken Izrael pred nimi, a utíkali cestou poušte.
\par 16 I svolán jest všecken lid v meste, aby je honili. I honili Jozue, a vzdálili se od mesta svého,
\par 17 Tak že nezustal žádný z obyvatelu Hai a Bethel, kdo by nevyšel, aby honil Izraele; a nechali mesta otevreného, a honili Izraele.
\par 18 Rekl pak Hospodin k Jozue: Zdvihni korouhev, kterouž máš v rukou svých, proti Hai, nebo v ruce tvé dám je. I zdvihl Jozue korouhev, kterouž mel v ruce své, proti mestu.
\par 19 Tedy, kteríž byli v zálohách, rychle vyskocili z místa svého, a beželi, když pozdvihl ruky své, a všedše do mesta, vzali je, a spešne zapálili mesto ohnem.
\par 20 Muži pak mesta Hai ohlédše se nazpet, uzreli, a aj, vstupoval dým mesta k nebi, a nemeli místa k utíkání sem ani tam; nebo lid, kterýž utíkati pocal k poušti, obrátil se na ty, kteríž je honili.
\par 21 Jozue zajisté a všecken Izrael, když videli, že z záloh vzali mesto, a že se vznáší dým mesta, obrátili se, a bili muže Hai.
\par 22 Onino také vyšli z mesta proti nim, a obklícil Izrael neprátely své, jedni odsud, druzí od onud; a zmordovali je, tak že žádný živ nezustal ani neušel.
\par 23 Ale krále Hai jali živého, a privedli ho k Jozue.
\par 24 Když pak pomordoval Izrael všecky obyvatele Hai v poli, totiž na poušti, kamž je honili, a padli ti všickni od ostrosti mece, až i zahlazeni jsou: navrátil se všecken Izrael do Hai, a zmordovali ostatky jeho mecem.
\par 25 A bylo všech, kteríž padli v ten den, od muže až do ženy, dvanácte tisícu; všickni ti byli z Hai.
\par 26 Ale Jozue nespustil ruky své, kterouž vyzdvihl korouhev, dokudž nebyli zmordováni všickni obyvatelé Hai.
\par 27 Toliko hovada a loupež mesta toho rozbitovali mezi sebou synové Izraelští podlé slova Hospodinova, kteréž on prikázal Jozue.
\par 28 Tedy vypálil Jozue Hai, a položil je v hromadu vecnou a pustinu, až do tohoto dne.
\par 29 Krále pak Hai obesil na dreve, a nechal ho tam až do vecera. A když zapadlo slunce, rozkázal Jozue, aby složili telo jeho s dreva, a povrhli je u brány mesta, a nametali na ne hromadu kamení velikou, kteráž trvá až do dnešního dne.
\par 30 Tedy vzdelal Jozue oltár Hospodinu Bohu Izraelskému na hore Hébal,
\par 31 (Jakož prikázal Mojžíš, služebník Hospodinuv, synum Izraelským, jakož psáno jest v knize zákona Mojžíšova,) oltár z kamení celého, nad nímž nebylo zdviženo železo, a obetovali na nem obeti zápalné Hospodinu, obetovali i obeti pokojné.
\par 32 Napsal také tam na kameních výpis zákona Mojžíšova, kterýž psal pred syny Izraelskými.
\par 33 Všecken pak Izrael a starší jeho, i správcové i soudcové jeho stáli po obou stranách truhly pred knežími Levítskými, kteríž nosili truhlu smlouvy Hospodinovy, tak cizí jako doma zrozený, polovice jich proti hore Garizim, a polovice proti hore Hébal, jakož byl prvé prikázal Mojžíš, služebník Hospodinuv, aby dobrorecil lidu Izraelskému nejprvé.
\par 34 A potom cetl všecka slova zákona, požehnání i zlorecení, tak jakž psáno jest v knize zákona.
\par 35 Nebylo ani slova ze všeho, což prikázal Mojžíš, jehož by necetl Jozue prede vším shromáždením Izraelským, i ženami i detmi i príchozími, kteríž šli u prostred nich.

\chapter{9}

\par 1 To když uslyšeli všickni králové, kteríž bydlili za Jordánem, na horách i na rovinách, a na všem brehu more velikého naproti Libánu: Hetejský, Amorejský, Kananejský, Ferezejský, Hevejský a Jebuzejský,
\par 2 Sebrali se spolu, aby bojovali proti Jozue a proti Izraeli jednomyslne.
\par 3 Ale obyvatelé Gabaon uslyšavše, co ucinil Jozue Jerichu a Hai,
\par 4 Ucinili i oni chytre. Nebo odšedše, ukázali se, jako by zdaleka poslové byli, a vzali pytle staré na osly své, a kožené láhvice vinné vetché, zedrané a zzašívané,
\par 5 Obuv také starou a splácenou na nohy své, též šaty otrelé na sebe; a všecken chléb, kterýž sobe na cestu vzali, vyschlý byl a zdrobený.
\par 6 I šli k Jozue do ležení v Galgala, a rekli jemu i mužum Izraelským: Z zeme daleké jsme prišli, protož nyní ucinte s námi smlouvu.
\par 7 Tedy odpovedeli muži Izraelští k Hevejským: Snad u prostred nás vy bydlíte, i kterakž bychom s vámi ucinili smlouvu?
\par 8 Ale oni odpovedeli Jozue: Služebníci tvoji jsme. Jimž rekl Jozue: Kdo pak jste, a odkud jste prišli?
\par 9 Odpovedeli jemu: Z zeme velmi daleké prišli služebníci tvoji ve jménu Hospodina Boha tvého; nebo jsme slyšeli povest jeho a všecky veci, kteréž cinil v Egypte.
\par 10 A všecko, což ucinil dvema králum Amorejským, kteríž bydlili za Jordánem, Seonovi, králi Ezebon, a Ogovi, králi Bázan, kterýž byl v Astarot.
\par 11 I rekli nám starší naši a všickni obyvatelé zeme naší temito slovy: Naberte sobe potravy na cestu, a jdete jim vstríc, a rcete jim: Služebníci vaši jsme, protož nyní ucinte s námi smlouvu.
\par 12 Totot jest chléb náš; horký jsme na cestu vzali z domu svých v ten den, když jsme vyšli, abychom k vám šli, a hle, nyní již vyschl a zdrobil se.
\par 13 A tyto kožené láhvice vinné nové byly, když jsme je naplnili, a již hle, potrhané jsou; tolikéž tento odev náš a obuv naše zvetšela, pro prílišnou cesty dalekost.
\par 14 I prijali to muži ti z pokrmu jejich, a úst Hospodinových neotázali se.
\par 15 A ucinil s nimi Jozue pokoj, a všel s nimi v smlouvu, aby jich pri životu zanechal; také i knížata shromáždení prísahu jim ucinili.
\par 16 Po trech pak dnech, po té smlouve s nimi ucinené, uslyšeli, že by velmi blízko jich byli, a že by u prostred nich bydlili.
\par 17 Jdouce tedy synové Izraelští, pritáhli k mestum jejich tretího dne; mesta pak jejich byla Gabaon a Kafira a Berot a Kariatjeharim.
\par 18 Ale nezbili jich synové Izraelští, proto že prísahu jim ucinili knížata shromáždení skrze Hospodina Boha Izraelského. A reptalo všecko shromáždení proti knížatum.
\par 19 Tedy rekla všecka knížata všemu shromáždení: My jsme jim prísahu ucinili skrze Hospodina Boha Izraelského, protož nyní nemužeme se jich dotknouti.
\par 20 Toto jim uciníme: Zanecháme jich pri životu, aby nebylo na nás rozhnevání pro prísahu, kterouž jsme jim ucinili.
\par 21 Rekli jim knížata i to: Necht jsou živi a drva sekají, a vodu nosí všemu shromáždení. I prestali na tom, jakož jim mluvila knížata.
\par 22 Povolal jich pak Jozue a mluvil k nim, rka: Proc jste nás podvedli, rkouce: Velmi jsme dalecí od vás? A vy u prostred nás bydlíte.
\par 23 Protož nyní zlorecení jste, a neprestanou z vás služebníci, a kteríž by dríví sekali a vodu nosili do domu Boha mého.
\par 24 Kteríž odpovídajíce Jozue, rekli: Za vec jistou oznámeno bylo služebníkum tvým, kterak prikázal Hospodin Buh tvuj Mojžíšovi služebníku svému, dáti vám zemi tuto, a vyhladiti všecky obyvatele zeme této pred tvárí vaší; protož strachovali jsme se vás, bojíce se za životy své pred vámi, a ucinili jsme to.
\par 25 A hle, nyní v ruce tvé jsme; cožt se dobrého a spravedlivého vidí uciniti s námi, to ucin.
\par 26 I ucinil jim tak, a vysvobodil je z rukou synu Izraelských, aby jich nepobili.
\par 27 A ustanovil je Jozue v ten den, aby dríví sekali a vodu nosili všemu shromáždení, i k oltári Hospodinovu, až do tohoto dne, na míste, kteréž by vyvolil.

\chapter{10}

\par 1 Uslyšev pak Adonisedech, král Jeruzalémský, že Jozue vzal mesto Hai, a jako proklaté zkazil je, (nebo jakž ucinil Jerichu a králi jeho, tak ucinil Hai a králi jeho,) a že pokoj ucinili obyvatelé Gabaon s Izraelem, a bydlí u prostred neho,
\par 2 I bál se velmi, proto že mesto veliké bylo Gabaon, jako jedno z mest královských, a že bylo vetší než Hai, a všickni muži jeho udatní.
\par 3 Protož poslal Adonisedech, král Jeruzalémský, k Ohamovi, králi Hebron, a k Faramovi, králi Jarmut, a k Jafiovi, králi Lachis, a k Dabirovi, králi Eglon, rka:
\par 4 Sjedte se ke mne, a pomozte mi, abychom dobyli Gabaon, proto že pokoj ucinili s Jozue a s syny Izraelskými.
\par 5 I shromáždilo se a vytáhlo pet králu Amorejských, král Jeruzalémský, král Hebron, král Jarmut, král Lachis, král Eglon, oni i všecka vojska jejich, a položivše se u Gabaon, dobývali ho.
\par 6 Tedy poslali muži Gabaon k Jozue do ležení v Galgala, rkouce: Neodjímejž ruky své od služebníku svých; pritáhni rychle k nám, zachovej nás a spomoz nám; nebo sebrali se proti nám všickni králové Amorejští bydlící na horách.
\par 7 I táhl Jozue z Galgala, on i všecken lid bojovný s ním, všickni muži udatní.
\par 8 (Nebo byl rekl Hospodin k Jozue: Neboj se jich, v ruce tvé zajisté dal jsem je, neostojít žádný z nich pred oblícejem tvým.)
\par 9 I pripadl na ne Jozue v náhle, nebo celou noc táhl z Galgala.
\par 10 A potrel je Hospodin pred Izraelem, kterýžto pobil je ranou velikou u Gabaon, a honil je cestou, kudy se jde k Betoron, a bil je až do Azeka a až do Maceda.
\par 11 I stalo se, když utíkali pred tvárí Izraele, sstupujíce do Betoron, že Hospodin metal na ne kamení veliké s nebe, až k Azeku, a mreli. Více jich zemrelo od krupobití kamenného, než jich pobili synové Izraelští mecem.
\par 12 Tedy mluvil Jozue k Hospodinu v den, v kterýž dal Hospodin Amorejského v moc synum Izraelským, a rekl pred syny Izraelskými: Slunce v Gabaon zastav se, a mesíc v údolí Aialon.
\par 13 I zastavilo se slunce, a stál mesíc, dokudž nepomstil se lid nad neprátely svými. Zdali není toho napsáno v knize Uprímého? Tedy stálo slunce u prostred nebe, a nepospíchalo k západu, jako za jeden celý den.
\par 14 Nebylo dne takového prvé ani potom, jako když vyslyšel Hospodin hlas cloveka; nebo Hospodin bojoval za Izraele.
\par 15 I navracel se Jozue i všecken Izrael s ním do táboru v Galgala.
\par 16 Uteklo pak bylo tech králu pet, a skryli se v jeskyni pri Maceda.
\par 17 I oznámeno bylo Jozue temi slovy: Nalezeno jest pet králu, kteríž se skryli v jeskyni pri Maceda.
\par 18 I rekl Jozue: Privalte kamení veliké k díre té jeskyne, a osadte ji muži, aby ostríhali jich.
\par 19 Vy pak nezastavujte se, honte neprátely své, a bíte je po zadu, a nedejte jim vjíti do mest jejich, nebo dal je Hospodin Buh váš v ruku vaši.
\par 20 Když pak prestali Jozue a synové Izraelští bíti jich porážkou velikou velmi, až i zahlazeni byli; a kteríž živi pozustali z nich, utekli do mest hrazených:
\par 21 Navrátil se všecken lid do ležení k Jozue do Maceda ve zdraví; nepohnul proti synum Izraelským žádný jazykem svým.
\par 22 Potom rekl Jozue: Odhradte díru u jeskyne, a vyvedte ke mne pet králu tech z jeskyne.
\par 23 I ucinili tak, a vyvedli k nemu pet králu tech z jeskyne, krále Jeruzalémského, krále Hebron, krále Jarmut, krále Lachis, krále Eglon.
\par 24 A když vyvedli ty krále k Jozue, svolav všecky muže Izraelské, rekl vývodám mužu bojovných, kteríž s ním v tažení tom byli: Pristupte sem, a šlapejte nohama svýma na hrdla tech králu. Kterížto pristoupivše, šlapali nohama svýma na hrdla jejich.
\par 25 I rekl jim Jozue: Nebojte se ani strachujte, posilnte se a budte zmužilí, nebo tak uciní Hospodin všechnem neprátelum vašim, proti nimž bojujete.
\par 26 Potom bil je Jozue a zmordoval, a povesil je na peti drevích; a viseli na drevích až do vecera.
\par 27 Když pak zapadlo slunce, k rozkazu Jozue složili je s drev, a uvrhli je do jeskyne, v kteréž se byli skryli, a privalili kamení veliké k díre jeskyne, kteréž jest tu až do tohoto dne.
\par 28 Téhož dne také dobyl Jozue Maceda, a pohubil je mecem, i krále jejich pobil spolu s nimi, a všecky lidi, kteríž byli v nem. Neživil žádného, ale ucinil králi Maceda, jako ucinil králi Jericha.
\par 29 Táhl potom Jozue a všecken Izrael s ním z Maceda k Lebnu, a dobýval Lebna.
\par 30 I vydal je také Hospodin v ruku Izraelovi i krále jeho, a pobil je mecem, a všecky duše, kteréž byly v nem; nenechal tam žádného živého, a ucinil králi jeho, jako ucinil králi Jericha.
\par 31 Táhl také Jozue a všecken Izrael s ním z Lebna do Lachis, a položivše se u neho, dobývali ho.
\par 32 I dal Hospodin Lachis v ruku Izraele, a dobyl ho druhého dne, a pohubil je mecem i všecky lidi, kteríž byli v nem, rovne tak, jakž ucinil Lebnu.
\par 33 Tehdy pritáhl Horam, král Gázer, aby pomoc dal Lachis. I porazil jej Jozue i lid jeho, tak že nepozustavil žádného živého.
\par 34 Táhl potom Jozue a všecken Izrael s ním z Lachis do Eglon, a položivše se naproti, bojovali proti nemu.
\par 35 A vzali je toho dne, a pobili je mecem, a všecky duše, kteréž byly v nem, dne toho pomordovali, rovne tak, jakž ucinili Lachis.
\par 36 Vstoupil pak Jozue a všecken Izrael s ním z Eglon do Hebron, a dobývali ho.
\par 37 I vzali je a zhubili mecem i krále jeho, i všecka mesta jeho, i každého cloveka, kterýž byl v nem; nepozustavil žádného živého, podobne tak, jakož ucinil Eglon. I zahladil je i všelikou duši, kteráž byla v nem.
\par 38 Potom navracuje se Jozue a všecken Izrael s ním, prišli do Dabir, a dobýval ho.
\par 39 I vzal je a krále jeho i všecka mesta jeho, a pohubili je mecem, a všelikou duši, kteráž byla v nem, jako proklaté vyhladili. Nepozustavili žádného živého; jakož ucinil Hebron, tak ucinil Dabir i králi jeho, a jakož ucinil Lebnu a králi jeho.
\par 40 A tak pohubil Jozue všecku zemi po horách, i polední stranu, i roviny a údolí, i všecky krále jejich. Nepozustavil žádného živého, ale všelikou duši vyhladil, jakož prikázal Hospodin Buh Izraelský.
\par 41 Pohubil tedy Jozue všecko od Kádesbarne až do Gázy, a všecku zemi Gosen až do Gabaon.
\par 42 A všecky krále ty a zemi jejich vzal Jozue pojednou, nebo Hospodin Buh Izraelský bojoval za Izraele.
\par 43 Potom navrátil se Jozue a všecken Izrael s ním do ležení, kteréž bylo v Galgala.

\chapter{11}

\par 1 To když uslyšel Jabín, král Azor, poslal k Jobábovi, králi Mádon, a k králi Simron, a k králi Achzaf,
\par 2 I k králum, kteríž bydlili k strane pulnocní na horách i na rovinách ku polední Ceneret, i na rovinách a v krajinách Dor na západ,
\par 3 K Kananejskému na východ i na západ, a k Amorejskému, Hetejskému, Ferezejskému a Jebuzejskému na horách, a k Hevejskému pod horou Hermon v zemi Masfa.
\par 4 I vytáhli všickni ti a všecka vojska jejich s nimi, lid mnohý jako písek, kterýž jest na brehu morském nescíslný, i konu i vozu velmi mnoho.
\par 5 A smluvivše se všickni králové ti, pritáhli a položili se spolu pri vodách Merom, aby bojovali proti Izraelovi.
\par 6 Rekl pak Hospodin k Jozue: Neboj se jich, nebo zítra o této chvíli vydám všecky tyto k zabití Izraelovi; konum jejich žily podrežeš, a vozy jejich ohnem spálíš.
\par 7 Tedy vytáhl Jozue a s ním všecken lid válecný proti nim k vodám Merom rychle, a uderili na ne.
\par 8 I dal je Hospodin v ruku Izraelovi, a porazili je. I honili je až k Sidonu velikému, a až k vodám Maserefot, a k údolí Masfe na východ, a pobili je, tak že nepozustavili žádného živého.
\par 9 A ucinil jim Jozue, jakož byl rozkázal jemu Hospodin, konum jejich žily zpodrezoval, a vozy jejich ohnem popálil.
\par 10 Potom navrátiv se Jozue téhož casu, dobyl Azor, a krále jeho zabil mecem. Azor pak bylo prvé nejznamenitejší mezi všemi království temi.
\par 11 Pobili také všecko, cožkoli v nem živo bylo, mordujíce je mecem, tak že nezustalo žádného živého; Azor pak spálil ohnem.
\par 12 Takž podobne ucinil všechnem mestum králu tech, a všecky krále jejich zjímal Jozue, a zbiv je mecem, vyhladil je, jakož prikázal Mojžíš, služebník Hospodinuv.
\par 13 A však žádného mesta z tech, kteráž ješte zustala v ohrade své, nespálil Izrael, krome samého Azor, kteréž spálil Jozue.
\par 14 A všecky loupeže mest tech i hovada rozdelili mezi sebou synové Izraelští; toliko všecky lidi zbili mecem, dokudž nevyhladili jich, nenechavše žádného živého.
\par 15 Jakož byl prikázal Hospodin Mojžíšovi, služebníku svému, tak prikázal Mojžíš Jozue, a Jozue tak cinil; nepominul niceho toho, což byl prikázal Hospodin Mojžíšovi.
\par 16 A tak vzal Jozue všecku tu zemi, hory i všecku stranu polední, i všecku zemi Gosen, i roviny, i pole, totiž hory Izraelské i roviny jejich,
\par 17 Od hory Halak, kteráž se táhne do Seir, až k Balgad na rovine Libánské, pod horou Hermon; všecky také krále jejich zjímal, a zbil je i zmordoval.
\par 18 Po mnohé dny Jozue vedl válku se všechnemi temi králi.
\par 19 A nebylo mesta, ješto by v pokoj vešlo s syny Izraelskými, krome Hevejských obyvatelu v Gabaon; jiná všecka válkou vzali.
\par 20 Nebo od Hospodina to bylo, že jsou zatvrdili byli srdce své, tak aby vyšli válecne proti Izraelovi, a aby je v prokletí vydal, a neucinil jim milosti, ale aby je zahubil, jakož prikázal Hospodin Mojžíšovi.
\par 21 Toho pak casu pritáhl Jozue a vyplénil Enaky z tech hor, totiž z Hebronu, z Dabir a z Anab, i ze všeliké hory Judské, a ze všeliké hory Izraelské; spolu s mesty jejich vyhladil je Jozue.
\par 22 Nezustal žádný z Enaku v zemi synu Izraelských; toliko v Gáze, v Gát a v Azotu zustali.
\par 23 Tak tedy vzal Jozue všecku tu zemi, jakž byl rozkázal Hospodin Mojžíšovi, a dal ji Jozue v dedictví Izraelovi vedlé dílu jejich, po pokoleních jejich. I odpocinula zeme od válek.

\chapter{12}

\par 1 Tito pak jsou králové té zeme, kteréž pobili synové Izraelští, a opanovali zemi jejich, za Jordánem k východu slunce, od potoku Arnon až k hore Hermon i všecky roviny k východu:
\par 2 Seon, král Amorejský, kterýž bydlil v Ezebon, a panoval od Aroer, kteréž leží pri brehu potoka Arnon, a u prostred potoka toho, a polovici Galád, až do potoka Jabok, kterýž jest na pomezí synu Ammon,
\par 3 A od rovin až k mori Ceneret k východu, a až k mori poušte, jenž jest more slané k východu, kudyž se jde k Betsimot, a od polední strany ležící pod horou Fazga.
\par 4 Pomezí také Oga, krále Bázan, z ostatku Refaimských, kterýž bydlil v Astarot a v Edrei,
\par 5 A kterýž panoval na hore Hermon a v Sálecha, i ve vší krajine Bázan až ku pomezí Gessuri a Machati, a nad polovicí Galád, ku pomezí Seona, krále Ezebon.
\par 6 Mojžíš, služebník Hospodinuv, a synové Izraelští pobili je; a dal ji Mojžíš služebník Hospodinuv k vládarství pokolení Rubenovu, Gádovu a polovici pokolení Manassesova.
\par 7 Tito pak jsou králové zeme té, kteréž pobil Jozue a synové Izraelští za Jordánem k západu, od Balgad, kteréž jest na poli Libánském, až k hore lysé, kteráž se táhne až do Seir, a dal ji Jozue pokolením Izraelským k vládarství po dílích jejich,
\par 8 Na horách i na rovinách, i po polích, i v údolích, i na poušti a na poledne, zemi Hetejského, Amorejského, Kananejského, Ferezejského, Hevejského a Jebuzejského:
\par 9 Král Jericha jeden, král Hai, kteréž bylo na strane Bethel, jeden;
\par 10 Král Jeruzalémský jeden, král Hebron jeden;
\par 11 Král Jarmut jeden, král Lachis jeden;
\par 12 Král Eglon jeden, král Gázer jeden;
\par 13 Král Dabir jeden, král Gader jeden;
\par 14 Král Horma jeden, král Arad jeden;
\par 15 Král Lebna jeden, král Adulam jeden;
\par 16 Král Maceda jeden, král Bethel jeden;
\par 17 Král Tafua jeden, král Chefer jeden;
\par 18 Král Afek jeden, král Sáron jeden;
\par 19 Král Mádon jeden, král Azor jeden;
\par 20 Král Simron Meron jeden, král Achzaf jeden;
\par 21 Král Tanach jeden, král Mageddo jeden;
\par 22 Král Kedes jeden, král Jekonam z Karmelu jeden;
\par 23 Král Dor z krajiny Dor jeden, král z Goim v Galgal jeden;
\par 24 Král Tersa jeden. Všech králu tridceti a jeden.

\chapter{13}

\par 1 Jozue pak již byl starý a sešlý vekem. I rekl jemu Hospodin: Tys se již sstaral, a jsi sešlého veku, zeme pak zustává velmi mnoho k opanování.
\par 2 Tato jest zeme, kteráž zustává: Všecky konciny Filistinské a všecka Gessuri,
\par 3 Od Níle, kterýž jest naproti Egyptu, až ku pomezí Akaron na pulnoci, kterážto krajina Kananejským se pricítá, v níž jest patero knížetství Filistinských: Gazejské, Azotské, Aškalonitské, Getejské a Akaronitské, a to bylo Hevejské;
\par 4 Na poledne pak všecka zeme Kananejská a Mára, kteréž jest Sidonských až do Afeka, a až ku pomezí Amorejského;
\par 5 Též zeme Giblická, a všecken Libán k východu slunce, od Balgad pod horou Hermon, až kde se vchází do Emat.
\par 6 Všecky obyvatele té hory, od Libánu až k vodám Maserefot, všecky Sidonské já vyženu od tvári synu Izraelských; toliko ty rozdel ji losem lidu Izraelskému v dedictví, jakožt jsem prikázal.
\par 7 Protož nyní rozdel zemi tu v dedictví devateru pokolení, a polovici pokolení Manassesova,
\par 8 Ponevadž druhá polovice a pokolení Rubenovo a Gádovo vzali díl svuj, kterýž jim dal Mojžíš pred Jordánem k východu, jakož dal jim Mojžíš služebník Hospodinuv,
\par 9 Od Aroer, kteréž jest pri brehu potoka Arnon, i mesta u prostred potoka, i všecky roviny Medaba až do Dibon,
\par 10 I všecka mesta Seona, krále Amorejského, kterýž kraloval v Ezebon, až ku pomezí synu Ammon,
\par 11 A Galád, i pomezí Gessuri a Machati, všecku horu Hermon, i všecken Bázan až do Sálecha,
\par 12 Všecko království Oga v Bázan, kterýž kraloval v Astarot a v Edrei, kterýž byl pozustal z ostatku Refaim, když je pobil Mojžíš, a zahladil je.
\par 13 Nevyhnali pak synové Izraelští Gessuri a Machati, protož bydlil Gessura a Machata u prostred Izraele až do dnešního dne.
\par 14 Toliko pokolení Léví nedal dedictví; obeti ohnivé Hospodina Boha Izraelského jsou dedictví jeho, jakož mluvil jemu.
\par 15 Dal pak Mojžíš pokolení synu Ruben po celedech jejich dedictví.
\par 16 A bylo jejich pomezí od Aroer, kteréž jest pri brehu potoka Arnon, i mesto, kteréž jest u prostred potoka, i všecky roviny, kteréž jsou pri Medaba,
\par 17 Ezebon i všecka mesta jeho, kteráž byla na rovinách, Dibon a Bamotbal a Betbalmeon,
\par 18 Jasa a Kedemot a Mefat,
\par 19 A Kariataim a Sabma, a Saratazar na hore údolí,
\par 20 A Betfegor i Assedot, Fazga a Betsimot.
\par 21 Všecka také mesta v kraji, i všecko království Seona, krále Amorejského, kterýž kraloval v Ezebon, jehožto zabil Mojžíš i s knížaty Madianskými, Evi i Rekem, Sur, Hur a Rebe, vývodami Seonovými, obyvateli té zeme.
\par 22 Ano i Baláma, syna Beorova, veštce, zabili synové Izraelští mecem, s jinými, kteríž zbiti od nich.
\par 23 I bylo pomezí synu Rubenových Jordán s mezemi svými. To jest dedictví synu Rubenových po celedech jejich, mesta i vsi jejich.
\par 24 Dal také Mojžíš pokolení Gád, synum Gádovým, po celedech jejich dedictví.
\par 25 A bylo jejich pomezí Jazer i všecka mesta Galád, a polovice zeme synu Ammon až do Aroer, kteréž jest naproti Rabba,
\par 26 A od Ezebon až do Rámot, Masfe a Betonim, a od Mahanaim až ku pomezí Dabir;
\par 27 V údolí také Betaram a Betnemra, a Sochot, a Sefon, ostatek království Seona, krále Ezebon, i Jordán s pomezím svým až k kraji more Ceneret za Jordánem na východ.
\par 28 To jest dedictví synu Gád po celedech jejich, mesta i vsi jejich.
\par 29 Dal také Mojžíš polovici pokolení Manassesova dedictví, i bylo polovice pokolení synu Manassesových, po celedech jejich.
\par 30 Pomezí jejich od Mahanaim, všecken Bázan, všecko království Oga, krále Bázan, i všecky vesnice Jair, kteréž jsou v Bázan, šedesáte mest,
\par 31 A polovice Galád, a Astarot, a Edrei, mesta království Oga v Bázan. To dal synum Machirovým, syna Manassesova, polovici synu Machirových, po celedech jejich.
\par 32 Ta jsou dedictví, kteráž rozdelil Mojžíš na rovinách Moábských pred Jordánem proti Jerichu k východu.
\par 33 Pokolení pak Léví nedal Mojžíš dedictví; Hospodin Buh Izraelský jest dedictví jejich, jakož mluvil jim.

\chapter{14}

\par 1 Toto pak jest, což dedicne obdrželi synové Izraelští v zemi Kanán, což uvedli jim právem dedicným k vládarství Eleazar knez, a Jozue, syn Nun, a prední z otcu, v pokolení synu Izraelských,
\par 2 Losem delíce dedictví jejich, jakož prikázal Hospodin skrze Mojžíše, aby dal devateru pokolení a polovici pokolení.
\par 3 Nebo byl dal Mojžíš dedictví pultretímu pokolení pred Jordánem, Levítum pak nedal dedictví u prostred nich.
\par 4 Nebo synu Jozefových bylo dvoje pokolení, Manassesovo a Efraimovo; a nedali dílu Levítum v zemi, krome mest k bydlení, a podmestí jejich pro dobytek a stáda jejich.
\par 5 Jakož prikázal Hospodin Mojžíšovi, tak ucinili synové Izraelští, a rozdelili zemi.
\par 6 Pristoupili pak synové Juda k Jozue v Galgala, i promluvil k nemu Kálef, syn Jefonuv, Cenezejský: Ty víš, co jest mluvil Hospodin k Mojžíšovi, muži Božímu, z príciny mé a tvé v Kádesbarne.
\par 7 Ve ctyridceti letech byl jsem, když mne poslal Mojžíš, služebník Hospodinuv, z Kádesbarne k spatrení zeme, a oznámil jsem jemu tu vec, jakž bylo v srdci mém.
\par 8 Ale bratrí moji, kteríž šli se mnou, zkormoutili srdce lidu, já pak cele krácel jsem za Hospodinem Bohem svým.
\par 9 I prisáhl Mojžíš toho dne, rka: Jiste že zeme, po kteréž jsi chodil nohama svýma, bude v dedictví tobe i synum tvým až na veky, proto že jsi cele následoval Hospodina Boha mého.
\par 10 A nyní, aj, propujcil mi Hospodin života, jakož zaslíbil. Již ctyridceti a pet let jest od toho casu, jakž toto mluvil Hospodin k Mojžíšovi, a jakž chodil Izrael po poušti, a aj, již dnes jsem v osmdesáti peti letech,
\par 11 A ješte nyní jsem pri síle jako tehdáž, když poslal mne Mojžíš. Jaká byla síla má tehdáž, taková i nyní síla má jest k boji, a k vycházení i k vcházení.
\par 12 Protož nyní dej mi horu tuto, o níž mluvil Hospodin onoho dne, nebo ty slyšel jsi toho dne, že Enakim jsou tam, a mesta veliká a pevne hrazená. Bude-li Hospodin se mnou, vyhladím je, jakož mluvil Hospodin.
\par 13 I požehnal mu Jozue, a dal Hebron Kálefovi, synu Jefone, v dedictví.
\par 14 Protož dostal se Hebron Kálefovi, synu Jefona Cenezejského, v dedictví až do tohoto dne, proto že cele krácel za Hospodinem Bohem Izraelským.
\par 15 Sloulo pak Hebron prvé mesto Arbe, kterýžto Arbe byl clovek veliký mezi Enakim. I odpocinula zeme od boju.

\chapter{15}

\par 1 Tento pak byl los pokolení synu Juda po celedech jejich, pri pomezí Edomském a poušti Tsin ku poledni, k strane polední.
\par 2 I byla jejich meze polední kraj more slaného od zátoky, kteráž se chýlí ku poledni.
\par 3 Odkudž jda na poledne k vrchu Akrabim, prechází Tsin, a táhne se od poledne k Kádesbarne, i prichází až do Ezron, a odtud tocí se k Addar, a obchází Karkaha.
\par 4 Odtud jde do Asmona, a vychází ku potoku Egyptskému, a prichází meze ta až k západu. To budete míti pomezí na poledne.
\par 5 Meze pak na východ jest more slané, až k kraji Jordánu, a meze strany pulnocní jest od zátoky more a od kraje Jordánu.
\par 6 Odkudž jde meze ta do Betogla, a táhne se od pulnoci do Betaraba; a odtud prichází k kameni Bohana syna Rubenova.
\par 7 A vstupuje ta meze do Dabir od údolí Achor, a na pulnoci chýlí se k Galgala, kteréž jest naproti vcházení do Adomim, jenž jest údolí tomu ku poledni, a prechází k vodám Ensemes, a skonává se u studnice Rogel.
\par 8 Odtud jde ta meze pres údolí synu Hinnom k strane Jebus od poledne, jenž jest Jeruzalém, odkudž vstupuje táž meze k vrchu hory, kteráž jest naproti údolí Hinnom na západ, a kteráž jest na konci údolí Refaim na pulnoci.
\par 9 Obchází pak ta meze od vrchu té hory k studnici vody Neftoa, a vychází k mestum hory Efron; a odtud jde ta meze do Bála, jenž jest Kariatjeharim.
\par 10 Potom tocí se ta meze od Bála na západ k hore Seir, a odtud jde k strane hory Jeharimských od pulnoci, jenž jest Cheslon, a sstupuje do Betsemes, a prichází do Tamna.
\par 11 A vychází ta meze v stranu Akaron na pulnoci, a tocí se vukol k Sechronu, a prechází až k hore Bála, a odtud táhne se do Jebnael, i dochází ta meze k mori.
\par 12 Potom západní pomezí jest pri mori velikém a mezech jeho. To jest pomezí synu Juda vukol, po celedech jejich.
\par 13 Kálefovi pak, synu Jefone, dal díl u prostred synu Juda, podlé reci Hospodinovy k Jozue, mesto Arbe, otce Enakova, jenž jest Hebron.
\par 14 I vyhnal odtud Kálef tri syny Enakovy: Sesai a Achimana a Tolmai, rodinu Enakovu.
\par 15 A odtud vstoupil k obyvatelum Dabir, kteréž prvé sloulo Kariatsefer.
\par 16 I rekl Kálef: Kdo by dobyl Kariatsefer a vzal je, dám jemu Axu dceru svou za manželku.
\par 17 Dobyl ho pak Otoniel syn Cenezuv, príbuzný Kálefuv, i dal jemu Axu dceru svou za manželku.
\par 18 I stalo se, že když prišla k nemu, ponukla ho, aby prosil otce jejího za pole; protož ssedla s osla. I rekl jí Kálef: Což te?
\par 19 A ona odpovedela: Dej mi dar, ponevadž jsi mi dal zemi suchou, dejž mi také studnice vod. I dal jí studnice horní a studnice dolní.
\par 20 To jest dedictví pokolení synu Juda po celedech jejich.
\par 21 Tato pak jsou mesta v koncinách pokolení synu Juda, podlé pomezí Edom na poledne: Kabsael, Eder a Jagur;
\par 22 A Cina, a Dimona, a Adada;
\par 23 A Kedes, a Azor, a Jetnan;
\par 24 Zif a Telem, a Balot;
\par 25 Též Azor, Chadat a Kariot, Ezron, jenž jest Azor;
\par 26 Amam a Sama, a Molada;
\par 27 A Azar Gadda, a Esmon, a Betfelet;
\par 28 Též Azarsual, a Bersabé, a Baziothia;
\par 29 Bála a Im, a Esem;
\par 30 A Eltolad, a Chesil, a Horma;
\par 31 A Sicelech, a Medemena, a Sensenna;
\par 32 A Lebaot, a Selim, též Ain a Remmon; všech mest dvadceti a devet i vsi jejich.
\par 33 Na rovinách pak: Estaol a Zaraha, a Asna;
\par 34 A Zanoe, a Engannim, Tafua a Enaim;
\par 35 Jarmut, Adulam, Socho a Azeka;
\par 36 A Saraim, Aditaim, a Gedera, a Gederotaim, mest ctrnácte i vsi jejich;
\par 37 Senan a Adassa, a Magdalgad;
\par 38 Delean a Masfa, a Jektehel;
\par 39 Lachis, Baskat a Eglon;
\par 40 Chebon, Lemam a Cetlis;
\par 41 Gederot, Betdagon, a Naama, i Maceda, mest šestnáct a vsi jejich;
\par 42 Lebna, Eter a Asan;
\par 43 Jefta, Asna a Nesib;
\par 44 Ceila, Achzib a Maresa, mest devet i vsi jejich;
\par 45 Akaron a mestecka, i vsi jeho;
\par 46 Od Akaron až k mori všecka mesta, kteráž se chýlí k Azotu, i vsi jejich;
\par 47 Azot, mestecka jeho i vsi jeho; Gáza, mestecka jeho i vsi jeho až ku potoku Egyptskému, i more veliké s pomezím svým.
\par 48 A na horách: Samir, Jeter a Socho;
\par 49 Danna a mesto Sanna, jenž jest Dabir;
\par 50 Anab, Estemo a Anim;
\par 51 Gosen, Holon a Gilo, mest jedenácte i vsi jejich;
\par 52 Arab, Duma a Esan;
\par 53 Janum, Bettafua a Afeka;
\par 54 Též Atmata a Kariatarbe, jenž jest Hebron, a Sior, mest devet a vsi jejich.
\par 55 Maon, Karmel a Zif, a Juta;
\par 56 Jezreel a Jukadam, a Zanoe;
\par 57 Kain, Gabaa a Tamna, mest deset i vsi jejich;
\par 58 Alul, Betsur a Gedor;
\par 59 Maret, Betanot a Eltekon, mest šest i vsi jejich;
\par 60 Kariatbaal, kteréž jest Kariatjeharim, a Rebba, mesta dve i vsi jejich;
\par 61 Na poušti: Betaraba, Middin a Sechacha;
\par 62 A Nibsam, a mesto solné, a Engadi, mest šest i vsi jejich.
\par 63 Jebuzejských pak obyvatelu Jeruzaléma nemohli synové Juda vypléniti, protož bydlil Jebuzejský s syny Judskými v Jeruzaléme až do tohoto dne.

\chapter{16}

\par 1 Potom padl los synum Jozefovým, od Jordánu proti Jerichu pri vodách Jerišských k východu, poušt, kteráž se zacíná od Jericha pres hory Bethel.
\par 2 A od Bethel vychází do Luza, a prichází ku pomezí Archi do Atarot.
\par 3 Potom jde k mori ku pomezí Jefleti, až ku pomezí Betoron dolního, a až k Gázer, a skonává se pri mori.
\par 4 I vzali dedictví své synové Jozefovi, Manasses a Efraim.
\par 5 Byla pak meze synu Efraimových po celedech jejich, ta byla meze dedictví jejich na východ, od Atarot Addar až do Betoron vrchního.
\par 6 A vychází meze ta k mori pri Michmetat od pulnocní strany, a obchází meze k východu Tanatsilo, a prechází ji od východu k Janoe.
\par 7 A sstupuje z Janoe do Atarot a Nárat, a prichází do Jericha, a vychází k Jordánu.
\par 8 Od Tafue meze jde k mori ku potoku Kána, a skonává se pri mori. To jest dedictví pokolení synu Efraim po celedech jejich.
\par 9 Mesta pak oddelená synum Efraimovým byla u prostred dedictví synu Manassesových, všecka mesta s vesnicemi svými.
\par 10 A nevyplénili Kananea bydlícího v Gázer. I bydlil Kananejský u prostred Efraima až do dnes, dávaje plat.

\chapter{17}

\par 1 A tento byl los Manassesuv (nebo on jest prvorozený Jozefuv): Machirovi prvorozenému Manassesovu, otci Gálad, proto že byl muž bojovný, dostal se Galád a Bázan.
\par 2 Dostalo se také jiným synum Manassesovým po celedech jejich, synum Abiezer, a synum Helek, a synum Asriel, i synum Sechem, a synum Hefer, a synum Semida. (Nebo ti jsou synové Manassesovi, syna Jozefova, muži po rodech svých.
\par 3 Ale Salfad, syn Hefer, syna Galád, syna Machir, syna Manasse, nemel synu, než dcery toliko, jejichž jsou tato jména: Mahla a Noa, Hegla, Melcha a Tersa.
\par 4 Kteréžto pristoupivše pred Eleazara kneze, a pred Jozue, syna Nun, i pred knížata, rekly: Hospodin prikázal Mojžíšovi, aby nám dal dedictví u prostred bratrí našich. I dal jim Jozue podlé rozkázaní Hospodinova dedictví u prostred bratrí otce jejich.)
\par 5 Dostalo se provazcu Manassesovi deset, krom zeme Galád a Bázan, kteráž byla pred Jordánem.
\par 6 Nebo dcery Manassesovy obdržely dedictví mezi syny jeho, zeme pak Galád prišla jiným synum Manassesovým.
\par 7 A byla meze Manassesova od Asser, Michmetat, jenž jest pred Sichem, a táhne se na pravou stranu k obyvatelum Entafue.
\par 8 (Manassesova zajisté byla zeme Tafue, ale Tafue podlé pomezí Manassesova jest synu Efraimových.)
\par 9 Odkudž sstupuje pomezí ku potoku Kána, na poledne tomu potoku, a tu jsou mesta Efraimova u prostred mest Manassesových; pomezí pak Manassesovo jest na pulnoci toho potoka, a skonává se pri mori.
\par 10 Na poledne jest díl Efraimuv, a na pulnoci Manassesuv, more pak jest pomezí jejich; a v pokolení Asser sbíhají se na pulnoci, v pokolení pak Izachar na východ.
\par 11 Nebo dedictví Manassesovo jest mezi Izacharovým a Asserovým, Betsan i mestecka jeho, a Jibleam a mestecka jeho; též obyvatelé Dor a mestecka jeho, a obyvatelé Endor a mestecka jeho; také obyvatelé Tanach a mestecka jeho, i obyvatelé Mageddo a mestecka jeho; tri ty krajiny.
\par 12 Ale synové Manassesovi nemohli vypléniti obyvatelu tech mest; protož smeleji pocal Kananejský bydliti v zemi té.
\par 13 Když se pak zsilili synové Izraelští, uvedli sobe Kananejské pod plat, a nevyhladili jich do konce.
\par 14 Tedy mluvili synové Jozefovi k Jozue, rkouce: Proc jsi nám dal dedictví toliko los jeden a provazec jeden, ponevadž jsme lid mnohý; nebo až dosavad žehnal nám Hospodin.
\par 15 I rekl jim Jozue: Ponevadž jsi lid tak mnohý, vejdi do lesa, a vyplan sobe tam v zemi Ferezejské a Refaimské, jestližet jest malá hora Efraim.
\par 16 Jemuž odpovedeli synové Jozefovi: I tak nám nepostací ta hora; pres to vozy železné mají všickni Kananejští, kteríž bydlejí v luzích tech, i ti, kteríž jsou v Betsan a v mesteckách jeho, i ti, kteríž jsou v údolí Jezreel.
\par 17 I rekl Jozue domu Jozefovu, Efraimovu a Manassesovu, rka: Lid mnohý a silný jsi, nebudeš míti toliko dílu jednoho,
\par 18 Ale horu budeš míti. Jestližet prekáží les, tedy vysekáš jej, a obdržíš konciny její; nebo vyhladíš Kananejského, ackoli má vozy železné a jest silný.

\chapter{18}

\par 1 Shromáždilo se pak všecko množství synu Izraelských do Sílo, a tu postavili stánek úmluvy, když již zeme od nich podmanena byla.
\par 2 Pozustalo pak z synu Izraelských, jimž ješte nebylo rozdeleno dedictví jejich, sedmero pokolení.
\par 3 I rekl Jozue synum Izraelským: I dokudž zanedbáváte vjíti, abyste se uvázali v zemi, kterouž vám dal Hospodin Buh otcu vašich?
\par 4 Vydejte z sebe z každého pokolení tri muže, at je pošli, aby vstanouce, zchodili zemi, a popsali ji vedlé dedictví svých; potom navrátí se ke mne.
\par 5 I rozdelí ji na sedm dílu. Juda zustane v koncinách svých od poledne, a celedi Jozefovy zustanou v koncinách svých od pulnoci.
\par 6 Vy pak, když popíšete zemi na sedm dílu, prinesete sem ke mne; tedy uvrhu vám losy zde pred Hospodinem Bohem naším.
\par 7 Nebo nemají dílu Levítové u prostred vás, proto že knežství Hospodinovo jest dedictví jejich; Gád pak a Ruben, a polovice pokolení Manassesova, vzali dedictví své pred Jordánem na východ, kteréž dal jim Mojžíš, služebník Hospodinuv.
\par 8 Protož vstavše muži ti, odešli. A prikázal Jozue tem, kteríž šli, aby popsali zemi, rka: Jdete a projdete zemi, a popište ji; potom navratte se ke mne, a uvrhu zde losy pred Hospodinem v Sílo.
\par 9 Tedy odešli muži, a prošedše zemi, popsali ji po mestech na sedm dílu na knize, a navrátili se k Jozue do stanu v Sílo.
\par 10 I uvrhl jim losy Jozue v Sílo pred Hospodinem, a rozdelil tu Jozue zemi synum Izraelským vedlé dílu jejich.
\par 11 Padl pak los pokolení synu Beniaminových po celedech jejich, a prišla meze dílu jejich mezi syny Juda a syny Jozefovy.
\par 12 A byla meze jejich k strane pulnocní od Jordánu, a šla pri strane pulnocní Jericha, a táhla se na hory k mori, a skonávala se pri poušti Betaven.
\par 13 A odtud prechází ta meze do Luz, pri strane Luz polední, (jenž jest Bethel,) a chýlí se ta meze do Atarot Addar vedlé hory, kteráž jest od poledne Betoron dolního.
\par 14 Odkudž obchází vukol k strane more na poledne od hory, kteráž jest proti Betoron ku polední, a konec její jest Kariatbaal, (jenž jest Kariatjeharim), mesto synu Juda. Ta jest strana západní.
\par 15 Strana pak ku poledni od konce Kariatjeharim, a vychází ta meze k mori, a prichází až k studnici vod Neftoa.
\par 16 A táhne se táž meze k konci hory, kteráž jest naproti údolí synu Hinnom, a jest v údolí Refaim na pulnoci, a beží skrze údolí Hinnom po strane Jebuzea na poledne, a prichází k studnici Rogel.
\par 17 Potom tocí se od pulnoci, a dochází k Ensemes, a vychází do Gelilot, kteréž jest naproti místu, kudy se jde do Adomim, a sstupuje k kameni Bohana, syna Rubenova.
\par 18 Odtud prechází k strane, kteráž jest naproti rovinám strany pulnocní, a táhne se do Araba.
\par 19 Odtud jde k strane Betogla na pulnoci, a skonává se meze ta pri zátoce more slaného od pulnocní strany, tu kdež vpadá Jordán do more na polední strane. To jest pomezí ku poledni.
\par 20 Jordán také je oddeluje k strane východní. To jest dedictví synu Beniaminových s mezemi svými vukol a vukol po celedech jejich.
\par 21 A byla mesta tato pokolení synu Beniaminových po celedech jejich: Jericho, Betogla a údolí Kasis;
\par 22 A Betaraba, Semaraim a Bethel;
\par 23 A Avim, též Afara a Ofra;
\par 24 A Cefer, Hamona, Ofni a Gaba, mest dvanácta vsi jejich;
\par 25 Gabaon, Ráma a Berot;
\par 26 Masfa, Kafira a Mosa;
\par 27 Rekem, Jarefel a Tarela;
\par 28 A Sela, Elef a Jebus, (jenž jest Jeruzalém,) Gibat, Kariat, mest ctrnácte i vsi jejich. To jest dedictví synu Beniaminových po celedech jejich.

\chapter{19}

\par 1 Potom padl los druhý Simeonovi, pokolení synu Simeonových, po celedech jejich, a bylo dedictví jejich u prostred dedictví synu Juda.
\par 2 A prišlo jim v dedictví jejich Bersabé, Seba a Molada;
\par 3 Azarsual a Bala, též Esem;
\par 4 Eltolad a Betul, a Horma;
\par 5 Sicelech a Betmarchabot, a Azarsusa;
\par 6 Betlebaot a Sarohem, mest trinácte i vsi jejich;
\par 7 Ain, Remmon, též Eter a Asan, mesta ctyri i vsi jejich;
\par 8 I všecky vsi, kteréž byly vukolí mest tech, až do Balatber a Rámat poledního. To jest dedictví pokolení synu Simeonových po celedech jejich.
\par 9 Z podílu synu Judových bylo dedictví synu Simeonových; nebo díl synu Judových byl jim príliš veliký, protož u prostred dedictví jejich vzali dedictví synové Simeonovi.
\par 10 Potom prišel tretí los synum Zabulon po celedech jejich, a jest meze dedictví jejich až do Sarid.
\par 11 Odkudž vstupuje meze jejich podlé more k Merala, a prichází až do Debaset, a beží až ku potoku, kterýž jest proti Jekonam.
\par 12 Obrací se pak od Sarid nazpátek k východu slunce, ku pomezí Chazelet Tábor, a odtud táhne se k Daberet, a vstupuje do Jafie.
\par 13 Odtud prechází zase k východu, do Gethefer a do Itakasin, odkudž vychází do Remmon, a tocí se k Nea.
\par 14 Tocí se také táž meze od pulnoci do Anaton, a dochází až k údolí Jeftael;
\par 15 A Katet, Naalol, Simron, Idala a Betlém, mest dvanácte i vsi jejich.
\par 16 To jest dedictví synu Zabulonových po celedech jejich, ta mesta i vsi jejich.
\par 17 Izacharovi také padl los ctvrtý, totiž synum Izacharovým po celedech jejich.
\par 18 A meze jejich: Jezreel, Kasalat a Sunem;
\par 19 Hafaraim, Sion, též Anaharat;
\par 20 Rabbot, Kesion a Abez,
\par 21 Ramet, Engannim a Enhada, i Betfeses.
\par 22 Odkudž pribíhá meze k Táboru a k Sehesima a k Betsemes, a dochází až k Jordánu, mest šestnácte i vsi jejich.
\par 23 To jest dedictví pokolení synu Izacharových po celedech jejich, ta mesta i vsi jejich.
\par 24 Padl také los pátý pokolení synu Asser po celedech jejich.
\par 25 A byla meze jejich: Helkat a Chali, a Beten, a Achzaf;
\par 26 Elmelech, též Amaad a Mesal, a pribíhá až na Karmel k mori, a do Sichor Libnat;
\par 27 A obrací se k východu slunce do Betdagon, a dosahá k losu Zabulonovu, a do údolí Jeftael k pulnoci, a do Betemek a Nehiel, a táhne se do Kábul na levou stranu,
\par 28 A do Ebron a Rohob, a Hamon a Kána až do Sidonu velikého.
\par 29 Odtud se navrací ta meze do Ráma až k mestu hrazenému Zor; tu se obrací do Chosa, a skonává se pri mori podlé vymerení v Achziba.
\par 30 K tomu prísluší Afek a Rohob, mest dvamecítma i vsi jejich.
\par 31 To jest dedictví pokolení synu Asser po celedech jejich, ta mesta i vsi jejich.
\par 32 Synum Neftalímovým padl los šestý, po celedech jejich.
\par 33 A byla meze jejich od Helef a od Elon do Sananim, a Adami, Nekeb a Jebnael, až do Lekum, a skonává se u Jordánu.
\par 34 Potom navracuje se meze na západ k Azanot Tábor, a odtud jde do Hukuka, a vpadá k  Zabulonovu od poledne, a k Asserovu pribíhá od západu, a k Judovu pri Jordánu na východ slunce.
\par 35 Mesta pak hrazená jsou: Assedim, Ser a Emat, Rechat a Ceneret;
\par 36 Adama, Ráma a Azor;
\par 37 Kedes, Edrei a Enazor;
\par 38 Jeron, Magdalel, Horem, Betanat a Betsemes, mest devatenácte i vsi jejich.
\par 39 To jest dedictví pokolení synu Neftalím po celedech jejich, ta mesta s vesnicemi svými.
\par 40 Na pokolení synu Dan po celedech jejich padl los sedmý.
\par 41 A byla meze dedictví jejich: Zaraha a Estaol, a Hirsemes;
\par 42 Salbin, Aialon a Jetela;
\par 43 Elon, Tamna a Ekron;
\par 44 Elteke, Gebbeton a Baalat;
\par 45 Jehud, Beneberak a Getremmon;
\par 46 Mehaiarkon a Rakon s pomezím, kteréž jest naproti Joppe.
\par 47 Prišlo pak pomezí synum Dan príliš malé. Protož vstoupili synové Dan, a bojovali proti Lesen, a dobyvše ho, pobili obyvatele ostrostí mece, a vzavše je v dedictví, bydlili tam, a prezdeli Lesenu Dan, vedlé jména Dan otce svého.
\par 48 To jest dedictví pokolení synu Dan po celedech jejich, ta mesta i vsi jejich.
\par 49 Když pak prestali deliti se zemí po mezech jejích, dali synové Izraelští dedictví Jozue, synu Nun, mezi sebou.
\par 50 Podlé rozkázaní Hospodinova dali jemu mesto, kteréhož žádal, Tamnatsára, na hore Efraim. I vystavel mesto, a prebýval v nem.
\par 51 Ta jsou dedictví, kteráž dali k vládarství Eleazar knez a Jozue syn Nun, i prední z otcu pokolení synu Izraelských, losem v Sílo pred Hospodinem, u dverí stánku úmluvy, a tak dokonali rozdelování zeme.

\chapter{20}

\par 1 I mluvil Hospodin k Jozue, rka:
\par 2 Mluv k synum Izraelským, a rci jim:Oddejte z tech mest vašich mesta útocištná, o nichž jsem mluvil vám skrze Mojžíše,
\par 3 Aby tam utekl vražedlník, kterýž by zabil cloveka nechte a z nevedomí. I budou vám útocište pred prítelem zabitého.
\par 4 A když by utekl do jednoho z tech mest, stane u vrat v bráne mesta, a oznámí starším mesta toho pri svou; i prijmou ho do mesta k sobe, a dají mu místo, i bydliti bude u nich.
\par 5 Když by jej pak honil prítel toho zabitého, nevydají vražedlníka v ruce jeho; nebo nechte uderil bližního svého, nemev ho prvé v nenávisti.
\par 6 A bydliti bude v meste tom, dokudž nestane pred shromáždením k soudu, až do smrti kneze velikého, kterýž by byl toho casu; nebo tehdáž navrátí se vražedlník, a prijde do mesta svého a do domu svého, do mesta, odkudž utekl.
\par 7 I oddelili Kádes v Galilei na hore Neftalím, a Sichem na hore Efraim, a mesto Arbe, (jenž jest Hebron,) na hore Juda.
\par 8 Pred Jordánem pak proti Jerichu k východu oddelili Bozor, kteréž leží na poušti v rovine pokolení Rubenova, a Rámot v Galád z pokolení Gádova, a Golan v Bázan z pokolení Manassesova.
\par 9 Ta mesta byla útocištná všechnem synum Izraelským, i príchozímu, kterýž pohostinu jest mezi nimi, aby utekl tam, kdo by koli zabil nekoho nechte, a nesešel od ruky prítele toho zabitého prvé, než by stál pred shromáždením.

\chapter{21}

\par 1 Pristoupili pak prední z otcu Levítského pokolení k Eleazarovi knezi, a k Jozue, synu Nun, a k predním z otcu pokolení synu Izraelských,
\par 2 A mluvili k nim v Sílo, v zemi Kananejské, rkouce: Hospodin prikázal skrze Mojžíše, abyste nám dali mesta k prebývání, i podmestí jejich pro dobytky naše.
\par 3 Dali tedy synové Izraelští Levítum z dedictví svého, vedlé rozkázaní Hospodinova, mesta tato i podmestí jejich.
\par 4 Padl pak los celedem Kahat, i dáno losem synum Arona kneze, Levítum, z pokolení Juda, a z pokolení Simeon, i z pokolení Beniaminova mest trinácte.
\par 5 A jiným synum Kahat, z celedí pokolení Efraimova, a z pokolení Danova, a z polovice pokolení Manassesova losem dáno mest deset.
\par 6 Synum pak Gerson, z celedí pokolení Izacharova, a z pokolení Asserova, též z pokolení Neftalímova, a z polovice pokolení Manassesova v Bázan losem dáno mest trinácte.
\par 7 Synum Merari po celedech jejich, z pokolení Rubenova a z pokolení Gádova, též z pokolení Zabulonova mest dvanácte.
\par 8 Dali tedy synové Izraelští Levítum ta mesta i predmestí jejich, (jakož prikázal Hospodin skrze Mojžíše,) losem.
\par 9 A tak z pokolení synu Judových, a z pokolení synu Simeonových dána jsou mesta, kterýchž tuto jména jsou položena.
\par 10 A dostal se první díl synum Aronovým, z celedí Kahat, z synu Léví; nebo jim padl los první.
\par 11 Dáno jest tedy jim mesto Arbe, otce Enakova, (jenž jest Hebron,) na hore Juda, a predmestí jeho vukol neho.
\par 12 Ale pole mesta toho i vsi jeho dali Kálefovi synu Jefone k vládarství jeho.
\par 13 Synum tedy Arona kneze dali mesto útocištné vražedlníku, Hebron i predmestí jeho, a Lebno i predmestí jeho;
\par 14 A Jeter s predmestím jeho, též Estemo a predmestí jeho;
\par 15 Holon i predmestí jeho, a Dabir s podmestím jeho;
\par 16 Také Ain s predmestím jeho, a Juta s podmestím jeho, i Betsemes a predmestí jeho, mest devet z toho dvojího pokolení.
\par 17 Z pokolení pak Beniaminova dali Gabaon a predmestí jeho, a Gaba s predmestím jeho;
\par 18 Též Anatot a podmestí jeho, i Almon s predmestím jeho, mesta ctyri.
\par 19 Všech mest synu Aronových kneží trinácte mest s predmestími jejich.
\par 20 Celedem pak synu Kahat, Levítum, kteríž pozustali z synu Kahat, (byla pak mesta losu jejich z pokolení Efraim,)
\par 21 Dali jim mesto útocištné vražedlníku, Sichem i predmestí jeho, na hore Efraim, a Gázer s predmestím jeho.
\par 22 Též Kibsaim a predmestí jeho, a Betoron s predmestím jeho, mesta ctyri.
\par 23 Z pokolení pak Dan: Elteke a predmestí jeho, a Gebbeton s predmestím jeho;
\par 24 Též Aialon a predmestí jeho, a Getremmon s podmestím jeho, mesta ctyri.
\par 25 Z polovice pak pokolení Manassesova: Tanach a podmestí jeho, a Getremmon s predmestím jeho, mesta dve.
\par 26 Všech mest deset s predmestími jejich celedem synum Kahat ostatním.
\par 27 Synum také Gersonovým, z celedí Levítských, z polovice pokolení Manassesova dali mesto útocištné vražedlníku, Golan v Bázan a predmestí jeho, a Bozran s predmestím jeho, mesta dve.
\par 28 Z pokolení Izacharova: Kesion a podmestí jeho, a Daberet s predmestím jeho;
\par 29 Jarmut a predmestí jeho, a Engannim s predmestím jeho, mesta ctyri.
\par 30 Z pokolení pak Asserova: Mesal a predmestí jeho, též Abdon a podmestí jeho;
\par 31 Helkat s predmestím jeho, a Rohob s podmestím jeho, mesta ctyri.
\par 32 Z pokolení také Neftalímova: Mesto útocištné vražedlníku, Kedes v Galilei a predmestí jeho, a Hamotdor s predmestím jeho, a Kartam s predmestím jeho, mesta tri.
\par 33 Všech mest Gersonitských po celedech jejich, trinácte mest s predmestími jejich.
\par 34 Celedem pak synu Merari, Levítum ostatním, dali z pokolení Zabulonova Jekonam a predmestí jeho, Karta a predmestí jeho;
\par 35 Damna a predmestí jeho, Naalol a predmestí jeho, mesta ctyri.
\par 36 Z pokolení pak Rubenova: Bozor a predmestí jeho, a Jasa a predmestí jeho;
\par 37 Kedemot a predmestí jeho, a Mefat s podmestím jeho, mesta ctyri.
\par 38 A z pokolení Gádova: Mesto útocištné vražedlníku, Rámot v Galád a predmestí jeho, a Mahanaim s predmestím jeho;
\par 39 Ezebon a podmestí jeho, Jazer s podmestím jeho, mesta ctyri.
\par 40 Všech mest synu Merari po celedech jejich, kteríž ostatní byli z celedí Levítských, bylo podlé losu jejich mest dvanácte.
\par 41 A tak všech mest Levítských u prostred vládarství synu Izraelských mest ctyridceti osm s predmestími svými.
\par 42 Melo pak to mesto jedno každé obzvláštne svá predmestí vukol sebe, a taková byla všecka ta mesta.
\par 43 Dal tedy Hospodin Izraelovi všecku tu zemi, kterouž s prísahou zaslíbil dáti otcum jejich; i opanovali ji dedicne, a bydlili v ní.
\par 44 Dal také Hospodin jim odpocinutí se všech stran podlé všeho, jakž byl s prísahou zaslíbil otcum jejich, aniž kdo byl, ješto by ostál proti nim ze všech neprátel jejich; všecky neprátely jejich dal Hospodin v ruku jejich.
\par 45 Nepominulo ani jedno slovo ze všelikého slova dobrého, kteréž mluvil Hospodin k domu Izraelskému, ale všecko se tak stalo.

\chapter{22}

\par 1 Toho casu povolav Jozue Rubenitských, Gáditských a polovice pokolení Manassesova,
\par 2 Rekl jim: Vy jste ostríhali všeho, což prikázal vám Mojžíš, služebník Hospodinuv, a poslouchali jste hlasu mého ve všem, což jsem prikázal vám.
\par 3 Neopustili jste bratrí svých již za dlouhý cas až do dne tohoto, ale bedlive jste ostríhali prikázaní Hospodina Boha vašeho.
\par 4 Nyní pak již odpocinutí dal Hospodin Buh váš bratrím vašim, jakož mluvil jim; již tedy navratte se a berte se do stanu svých, do zeme vládarství svého, kteréž vám dal Mojžíš, služebník Hospodinuv, pred Jordánem.
\par 5 Toliko hledte pilne zachovávati a plniti prikázaní a zákon, kterýž prikázal vám Mojžíš, služebník Hospodinuv, a milovati Hospodina Boha svého a choditi po všech cestách jeho, a zachovávajíce prikázaní jeho, prídržeti se ho, a sloužiti jemu celým srdcem svým a celou duší svou.
\par 6 A požehnav jim Jozue, propustil je; i odešli do stanu svých.
\par 7 (Polovici pak pokolení Manassesova dal byl Mojžíš dedictví v Bázan, a druhé polovici jeho dal Jozue s bratrími jejich za Jordánem k západu.) A když je propouštel Jozue do stanu jejich, požehnal jim,
\par 8 A mluvil k nim, rka: S bohatstvím velikým navracujete se do stanu svých a s dobytky velmi mnohými, s stríbrem a zlatem, s medí a železem a rouchem velmi mnohým; rozdeltež se loupeží neprátel svých s bratrími svými.
\par 9 Tedy navracujíce se, odešli synové Rubenovi a synové Gádovi a polovice pokolení Manassesova od synu Izraelských z Sílo, jenž jest v zemi Kananejské, aby šli do zeme Galád, do zeme vládarství svého, v kteréž dedictví obdrželi, vedlé reci Hospodinovy skrze Mojžíše.
\par 10 A prišedše ku pomezí pri Jordánu, kteréž jest v zemi Kananejské, i vzdelali tu synové Rubenovi, a synové Gádovi, a polovice pokolení Manassesova oltár nad Jordánem, oltár veliký ku podivení.
\par 11 Uslyšeli pak synové Izraelští, že praveno bylo: Aj, vystaveli synové Rubenovi a synové Gádovi, a polovice pokolení Manassesova oltár naproti zemi Kananejské, pri pomezi u Jordánu, kdež prešli synové Izraelští.
\par 12 Uslyševše, pravím, synové Izraelští, sešlo se všecko množství jejich do Sílo, aby táhli proti nim k boji.
\par 13 I poslali synové Izraelští k synum Rubenovým a k synum Gádovým a ku polovici pokolení Manassesova do zeme Galád Fínesa, syna Eleazara kneze,
\par 14 A deset knížat s ním, po jednom knížeti z každého domu otcovského, ze všech pokolení Izraelských. (Každý pak z nich byl prední v dome otcu svých v tisících Izraele.)
\par 15 Ti prišli k synum Rubenovým a k synum Gádovým a ku polovici pokolení Manassesova do zeme Galád, a mluvili s nimi, rkouce:
\par 16 Toto praví všecko shromáždení Hospodinovo: Jaké jest to prestoupení, jímž jste prestoupili proti Bohu Izraelskému, odvrátivše se dnes, abyste nešli za Hospodinem, vzdelavše sobe oltár, abyste se protivili dnes Hospodinu?
\par 17 Ješte-liž se nám malá zdá nepravost modly Fegor, od níž nejsme ocišteni až do dnes, procež byla rána v shromáždení Hospodinovu,
\par 18 Že vy pres to odvracujete se dnes, abyste nešli za Hospodinem? I stane se, ponevadž vy dnes odporujete Hospodinu, že on zítra na všecko shromáždení Izraelské rozhevá se.
\par 19 Jestližet jest necistá zeme vládarství vašeho, prejdete do zeme vládarství Hospodinova, v níž prebývá stánek Hospodinuv, a dedictví vezmete mezi námi; toliko proti Hospodinu se nepostavujte, a nebudte odporní nám, stavejíce sobe oltár mimo oltár Hospodina Boha našeho.
\par 20 Zdali Achan syn Záre nedopustil se prestoupení pri veci proklaté? A prišlo rozhnevání na všecko shromáždení Izraelské, a on sám jeden zhrešiv, nezahynul pro svou nepravost sám.
\par 21 I odpovedeli synové Rubenovi a synové Gádovi a polovice pokolení Manassesova, a mluvili s knížaty tisícu Izraelských:
\par 22 Silný Buh Hospodin, silný Buh Hospodin, ont ví, ano sám Izrael pozná, žet jsme ne z zpoury a všetecnosti proti Hospodinu to ucinili, jinác nezachovávejž nás ani dne tohoto.
\par 23 Jiste žet jsme nestaveli sobe oltáre k tomu, abychom se odvrátiti meli a nejíti za Hospodinem, ani k obetování na nem zápalu a obetí suchých, a k obetování na nem obetí pokojných, sic jinác Hospodin sám at to vyhledává.
\par 24 Anobrž radeji obávajíce se té veci, ucinili jsme to, myslíce: Potom mluviti budou synové vaši synum našim, rkouce: Co vám do Hospodina Boha Izraelského?
\par 25 Ponevadž meze položil Hospodin mezi námi a vámi, ó synové Rubenovi a synové Gádovi, Jordán tento, nemáte vy dílu v Hospodinu. I odvrátí synové vaši syny naše od bázne Hospodinovy.
\par 26 Protož jsme rekli: Pricinme se a vzdelejme oltár, ne pro zápaly a obeti,
\par 27 Ale aby byl svedkem mezi námi a vámi, a mezi potomky našimi po nás, k vykonávání služby Hospodinu pred ním zápaly našimi, a obetmi našimi a pokojnými obetmi našimi, a aby nerekli potom synové vaši synum našim: Nemáte dílu v Hospodinu.
\par 28 Protož jsme rekli: Jestliže by potom mluvili nám neb potomkum našim, tedy odpovíme: Vizte podobenství oltáre Hospodinova, kterýž ucinili otcové naši, ne pro zápaly ani obeti, ale aby byl na svedectví mezi námi a vámi.
\par 29 Odstup to od nás, abychom odporovati meli Hospodinu, a odvraceti se dnes a nejíti za Hospodinem, stavejíce oltár k zápalum, k obetem suchým a jiným obetem, mimo oltár Hospodina Boha našeho, kterýž jest pred stánkem jeho.
\par 30 Uslyšev pak Fínes knez a knížata shromáždení a prední z tisícu Izraelských, kteríž s ním byli, slova, kteráž mluvili synové Rubenovi a synové Gádovi a synové Manassesovi, líbilo se jim.
\par 31 I rekl Fínes, syn Eleazara kneze, synum Rubenovým a synum Gádovým, i synum Manassesovým: Nyní jsme poznali, že u prostred nás jest Hospodin, a že jste se nedopustili proti Hospodinu prestoupení toho, a tak vysvobodili jste syny Izraelské z ruky Hospodinovy.
\par 32 Tedy navrátil se Fínes, syn Eleazara kneze, i ta knížata od synu Rubenových a od synu Gádových z zeme Galád do zeme Kananejské k synum Izraelským a oznámili jim tu vec.
\par 33 I líbilo se to synum Izraelským, a dobrorecili Boha synové Izraelští, a již více nemluvili o to, aby táhli proti nim k boji a zkazili zemi, v kteréž synové Rubenovi a synové Gádovi bydlili.
\par 34 Nazvali pak synové Rubenovi a synové Gádovi oltár ten Ed, rkouce: Nebo svedkem bude mezi námi, že Hospodin jest Buh.

\chapter{23}

\par 1 I stalo se po mnohých dnech, když dal Hospodin odpocinutí Izraelovi ode všech neprátel jejich vukol, a Jozue se sstaral a sešel vekem,
\par 2 Že povolav Jozue všeho Izraele, starších jeho i predních jeho soudcu i správcu jeho, rekl jim: Já jsem se sstaral a sešel vekem.
\par 3 A vy jste videli všecko to, co jest ucinil Hospodin Buh váš všechnem národum tem pred oblícejem vaším; nebo Hospodin Buh váš ont bojoval za vás.
\par 4 Pohledte, losem rozdelil jsem vám ty národy pozustalé v dedictví po pokoleních vašich od Jordánu, i všecky národy, kteréž jsem vyplénil až k mori velikému na západ slunce.
\par 5 Hospodin pak Buh váš ont je vyžene od tvári vaší, a vypudí je pred oblícejem vaším, a vy dedicne obdržíte zemi jejich, jakož mluvil vám Hospodin Buh váš.
\par 6 Protož zmužile se mejte, abyste ostríhali a cinili všecko, což psáno jest v knize zákona Mojžíšova, a neodstupovali od neho na pravo ani na levo.
\par 7 Nesmešujte se s temi národy, kteríž pozustávají s vámi, a jména bohu jejich nepripomínejte, ani skrze ne prisahejte, aniž jim služte, ani se jim klanejte.
\par 8 Ale prídržte se Hospodina Boha svého, jakož jste cinili až do tohoto dne.
\par 9 I budet, že jakož jest vyhnal Hospodin od tvári vaší národy veliké a silné, aniž ostál kdo pred oblícejem vaším až do tohoto dne:
\par 10 Tak jeden z vás honiti bude tisíc, nebo Hospodin Buh váš ont bojuje za vás, jakož mluvil vám.
\par 11 Toho tedy nejpilnejší budte, abyste milovali Hospodina Boha svého.
\par 12 Nebo jestliže predce se odvrátíte, a pripojíte se k tem pozustalým národum, k tem, kteríž zustávají mezi vámi, a vejdete s nimi v príbuzenství a smísíte se s nimi a oni s vámi:
\par 13 Tedy jistotne vezte, že Hospodin Buh váš nebude více vyháneti všech národu tech od tvári vaší, ale budou vám osídlem a úrazem i bicem na bocích vašich, a trním v ocích vašich, dokudž nezahynete z zeme této výborné, kterouž vám dal Hospodin Buh váš.
\par 14 A hle, já nyní odcházím podlé zpusobu všech lidí, poznejtež tedy vším srdcem svým a celou duší svou, že nepochybilo ani jedno slovo ze všech slov nejlepších, kteráž mluvil Hospodin Buh váš o vás; všecko se naplnilo vám, ani jediné slovo z nich nepominulo.
\par 15 Jakož tedy naplnilo se vám dnes každé slovo dobré, kteréž mluvil k vám Hospodin Buh váš, takt uvede na vás každé slovo zlé, dokudž nevyhladí vás z zeme této výborné, kterouž dal vám Hospodin Buh váš,
\par 16 Jestliže prestoupíte smlouvu Hospodina Boha svého, kterouž prikázal vám, a jdouce, sloužiti budete bohum cizím, a klaneti se jim. I rozhevá se prchlivost Hospodinova na vás, a zahynete rychle z zeme této výborné, kterouž vám dal.

\chapter{24}

\par 1 Shromáždil pak Jozue všecka pokolení Izraelská v Sichem a svolal starší Izraelské a prední z nich, i soudce a správce jejich, i postavili se pred oblícejem Božím.
\par 2 I rekl Jozue všemu lidu: Toto praví Hospodin Buh Izraelský: Za rekou bydlili otcové vaši za starodávna, jmenovite Táre, otec Abrahamuv a otec Náchoruv, a sloužili bohum cizím.
\par 3 I vzal jsem otce vašeho Abrahama z místa, kteréž jest za rekou, a provedl jsem jej skrze všecku zemi Kananejskou, a rozmnožil jsem síme jeho, dav jemu Izáka.
\par 4 A Izákovi dal jsem Jákoba a Ezau. I dal jsem Ezau horu Seir, aby vládl jí; Jákob pak a synové jeho sstoupili do Egypta.
\par 5 I poslal jsem Mojžíše a Arona, a zbil jsem Egypt, a když jsem to ucinil u prostred neho, potom vyvedl jsem vás.
\par 6 A vyvedl jsem otce vaše z Egypta, i prišli jste k mori, a honili Egyptští otce vaše s vozy a jezdci až k mori Rudému.
\par 7 Tedy volali k Hospodinu, kterýž postavil mrákotu mezi vámi a Egyptskými, a uvedl na ne more, i zatopilo je; anobrž videly oci vaše i to, co jsem ucinil pri Egyptu, a bydlili jste na poušti za dlouhý cas.
\par 8 Potom uvedl jsem vás do zeme Amorejského, kterýž bydlil za Jordánem, a bojovali s vámi; i dal jsem je v ruku vaši, a opanovali jste zemi jejich, a zahladil jsem je pred tvárí vaší.
\par 9 Povstal pak Balák syn Seforuv, král Moábský, a bojoval s Izraelem; i poslav, povolal Baláma syna Beorova, aby zlorecil vám.
\par 10 Ale nechtel jsem slyšeti Baláma, procež stále dobrorecil vám, a tak vysvobodil jsem vás z ruky jeho.
\par 11 A když jste prešli Jordán, prišli jste k Jerichu; i bojovali proti vám muži Jericha, Amorejští, a Ferezejští, a Kananejští, a Hetejští, a Gergezejští, a Hevejští, a Jebuzejští, a dal jsem je v ruku vaši.
\par 12 Poslal jsem pred vámi i sršne, kterížto vyhnali je od tvári vaší, dva krále Amorejské; ne mecem ani lucištem svým tys je vyhnal.
\par 13 A dal jsem vám zemi, v níž jste nepracovali, a mesta, kterýchž jste nestaveli, v nichž bydlíte; vinic a olivoví, kteréhož jste neštípili, užíváte.
\par 14 Protož nyní bojte se Hospodina, a služte jemu v dokonalosti a v pravde, a odvrzte bohy, jimž sloužili otcové vaši za rekou a v Egypte, a služte Hospodinu.
\par 15 Pakli se vám zdá zle sloužiti Hospodinu, vyvolte sobe dnes, komu byste sloužili, bud bohy, jimž sloužili otcové vaši, kteríž byli za rekou, bud bohy Amorejských, v jichž zemi bydlíte; ját pak a dum muj sloužiti budeme Hospodinu.
\par 16 Jemuž odpovedel lid, rka: Odstup to od nás, abychom opustiti meli Hospodina a sloužiti bohum cizím.
\par 17 Nebo Hospodin Buh náš ont jest, kterýž vyvedl nás i otce naše z zeme Egyptské, z domu služebnosti, a kterýž cinil pred ocima našima znamení ta veliká, a choval nás na vší ceste, po níž jsme šli, a mezi všemi národy, skrze než jsme prošli.
\par 18 A vypudil Hospodin všecky národy, zvlášte Amorejského, obyvatele zeme této, pred tvárí naší; my také sloužiti budeme Hospodinu, nebo on jest Buh náš.
\par 19 I rekl Jozue lidu: Nebudete moci sloužiti Hospodinu, nebo Buh svatý jest, Buh silný horlivý jest, nesnese nepravostí vašich a hríchu vašich.
\par 20 Jestliže byste opustíce Hospodina, sloužili bohum cizozemcu, obrátí se a zle uciní vám, a zkazí vás, ponevadž prvé dobre ucinil vám.
\par 21 Jemuž odpovedel lid: Nikoli, ale Hospodinu sloužiti budeme.
\par 22 I rekl Jozue lidu: Svedkové budete sami proti sobe, že jste sobe vyvolili Hospodina, abyste jemu sloužili. Tedy odpovedeli: Svedkové jsme.
\par 23 Jimž on rekl: Odvrztež tedy nyní bohy cizozemcu, kteríž jsou u prostred vás, a naklonte srdcí svých k Hospodinu Bohu Izraelskému.
\par 24 Odpovedel lid Jozue: Hospodinu Bohu našemu sloužiti a hlasu jeho poslouchati budeme.
\par 25 A tak ucinil Jozue toho dne smlouvu s lidem, a predložil jim ustanovení a soudy v Sichem.
\par 26 A zapsal Jozue slova ta do knihy zákona Božího; vzav také kámen veliký, postavil jej tu pod dubem, kterýž byl u svatyne Hospodinovy.
\par 27 I rekl Jozue všemu lidu: Aj, kámen tento bude mezi námi na svedectví, nebo on slyšel všecka slova Hospodinova, kteráž mluvil s námi, a bude na svedectví proti vám, abyste snad neklamali proti Bohu svému.
\par 28 A rozpustil Jozue lid, jednoho každého do dedictví jeho.
\par 29 I stalo se po vykonání tech vecí, že umrel Jozue syn Nun, služebník Hospodinuv, jsa ve stu a desíti letech.
\par 30 A pochovali ho v krajine dedictví jeho, v Tamnatsáre, kteréž jest na hore Efraim k strane pulnocní hory Gás.
\par 31 I sloužil Izrael Hospodinu po všecky dny Jozue, i po všecky dny starších, kteríž dlouho živi byli po Jozue, a kteríž povedomi byli všech skutku Hospodinových, kteréž ucinil Izraelovi.
\par 32 Kosti pak Jozefovy, kteréž byli vynesli synové Izraelští z Egypta, pochovali v Sichem, v dílu pole, kteréž koupil Jákob od synu Emora otce Sichemova za sto penez; i byly u synu Jozefových v dedictví jejich.
\par 33 Eleazar také, syn Aronuv, umrel, a pochovali jej na pahrbku Fínesa syna jeho, kterýž dán byl jemu na hore Efraim.

\end{document}