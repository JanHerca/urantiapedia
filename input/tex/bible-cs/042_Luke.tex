\begin{document}

\title{Lukáše}

\chapter{1}

\par 1 Ponevadž mnozí usilovali sepsati porádne vypravování tech vecí, kteréž jsou u nás jisté,
\par 2 Jakž jsou nám vydali ti, kterížto od pocátku sami videli, a služebníci toho Slova byli,
\par 3 Videlo se i mne, kterýž jsem toho všeho pravé povedomosti došel, tobe z gruntu o tom porádne vypsati, výborný Theofile,
\par 4 Abys zvedel jistotu tech vecí, jimž jsi vyucován.
\par 5 Byl za dnu Herodesa krále Judského knez nejaký, jménem Zachariáš, z trídy Abiášovy, a manželka jeho ze dcer Aronových, a jméno její Alžbeta.
\par 6 Byli pak oba spravedliví pred oblicejem Božím, chodíce ve všech prikázáních a spravedlnostech Páne bez úhony.
\par 7 A nemeli plodu, protože Alžbeta byla neplodná, a oba se byli zstarali ve dnech svých.
\par 8 I stalo se, když on úrad knežský konal v porádku trídy své pred Bohem,
\par 9 Že vedle obyceje úradu knežského los nan prišel, aby položil zápal, vejda do chrámu Páne.
\par 10 A všecko množství lidu bylo vne, modlíce se v hodinu zápalu.
\par 11 Tedy ukázal se jemu andel Páne, stoje na pravé strane oltáre zápalu.
\par 12 A uzrev jej Zachariáš, zstrašil se, a bázen pripadla na nej.
\par 13 I rekl jemu andel: Neboj se, Zachariáši, nebot jest uslyšána modlitba tvá, a Alžbeta manželka tvá porodí tobe syna, a nazuveš jméno jeho Jan.
\par 14 Z cehož budeš míti radost a veselé, a mnozí z jeho narození budou se radovati.
\par 15 Bude zajisté veliký pred oblicejem Páne, a vína i nápoje opojného nebudet píti, a Duchem svatým bude naplnen hned od života matky své.
\par 16 A mnohé z synu Izraelských obrátí ku Pánu Bohu jejich.
\par 17 Nebot, on predejde pred oblicejem jeho v duchu a v moci Eliášove, aby obrátil srdce otcu k synum, a neverící k opatrnosti spravedlivých, aby tak pripravil Pánu lid hotový.
\par 18 I rekl Zachariáš k andelu: Po cemž to poznám? Nebo já starý jsem, a manželka má zstarala se ve dnech svých.
\par 19 Odpovedev andel, rekl jemu: Ját jsem Gabriel, kterýž stojím pred oblicejem Božím, a poslán jsem, abych mluvil s tebou, a tyto veci veselé tobe zvestoval.
\par 20 A aj, budeš nemý, a nebudeš moci mluviti až do toho dne, v kterémž se tyto veci stanou, protože jsi neuveril recem mým, kteréž se naplní casem svým.
\par 21 Lid pak ocekával Zachariáše, a divili se, že on tak prodléval v chráme.
\par 22 Vyšed pak, nemohl mluviti k nim. I srozumeli, že videní videl v chráme. Nebo on náveští jim dával, a zustal nemý.
\par 23 I stalo se, když se vyplnili dnové konání úradu jeho, odšel do domu svého.
\par 24 A po tech dnech pocala Alžbeta manželka jeho, a tajila se za pet mesícu, rkuci:
\par 25 Že tak mi ucinil Pán ve dnech, v nichžto vzezrel, aby odjal mé pohanení mezi lidmi.
\par 26 V mesíci pak šestém poslán jest andel Gabriel od Boha do mesta Galilejského, kterémuž jméno Nazarét,
\par 27 Ku panne zasnoubené muži, kterémuž jméno bylo Jozef, z domu Davidova, a jméno panny Maria.
\par 28 I všed k ní andel, dí: Zdráva milosti došlá, Pán Buh s tebou, požehnaná ty mezi ženami.
\par 29 Ona pak uzrevši ho, zarmoutila se nad recí jeho, a myslila, jaké by to bylo pozdravení.
\par 30 I rekl jí andel: Neboj se, Maria, nebo jsi nalezla milost u Boha.
\par 31 A pocneš v živote a porodíš syna, a nazuveš jméno jeho Ježíš.
\par 32 Tent bude veliký, a Syn Nejvyššího slouti bude, a dát jemu Pán Buh stolici Davida otce jeho.
\par 33 A kralovati bude v dome Jákobove na veky, a království jeho nebude konce.
\par 34 I rekla Maria k andelu: Kterak se to stane, ponevadž já muže nepoznávám?
\par 35 A odpovedev andel, rekl jí: Duch svatý sstoupí v te, a moc Nejvyššího zastíní tobe; a protož, což se z tebe svatého narodí, slouti bude Syn Boží.
\par 36 A aj, Alžbeta, príbuzná tvá, i ona pocala syna v starosti své, a tento jest jí šestý mesíc, kteráž sloula neplodná.
\par 37 Nebot nebude nemožné u Boha všeliké slovo.
\par 38 I rekla Maria: Aj, služebnice Páne, staniž mi se podle slova tvého. I odšel od ní andel.
\par 39 Tedy povstavši Maria v tech dnech, odešla na hory s chvátáním do mesta Judova.
\par 40 I vešla do domu Zachariášova, a pozdravila Alžbety.
\par 41 I stalo se, jakž uslyšela pozdravení Marie Alžbeta, zplésalo nemluvnátko v živote jejím, a naplnena jest Duchem svatým Alžbeta.
\par 42 I zvolala hlasem velikým a rekla: Požehnaná ty mezi ženami, a požehnaný plod života tvého.
\par 43 A odkud mi to, aby prišla matka Pána mého ke mne?
\par 44 Nebo aj, jakž se stal hlas pozdravení tvého v uších mých, zplésalo radostne nemluvnátko v živote mém.
\par 45 A blahoslavená, kteráž uverila, nebot dokonány budou ty veci, kteréž jsou povedíny jí ode Pána.
\par 46 Tedy rekla Maria: Velebí duše má Hospodina,
\par 47 A veselí se duch muj v Bohu, Spasiteli mém,
\par 48 Že jest vzezrel na ponížení služebnice své; neb aj, od této chvíle blahoslaviti mne budou všickni národové.
\par 49 Nebot mi ucinil veliké veci ten, jenž mocný jest, a svaté jméno jeho,
\par 50 A jehož milosrdenství od pokolení až do pokolení bojícím se jeho.
\par 51 Dokázal moci ramenem svým, rozptýlil pyšné myšlením srdce jejich.
\par 52 Sházel mocné s stolic, a povýšil ponížených.
\par 53 Lacné nakrmil dobrými vecmi, a bohaté pustil prázdné.
\par 54 Prijal Izraele, služebníka svého, pametliv jsa na milosrdenství své
\par 55 (Jakož mluvil k otcum našim, k Abrahamovi a semeni jeho) na veky.
\par 56 I zustala Maria s ní asi za tri mesíce, a potom navrátila se do domu svého.
\par 57 Alžbete pak naplnil se cas, aby porodila; i porodila syna.
\par 58 A uslyšeli sousedé a prátelé její, že Hospodin veliké ucinil s ní milosrdenství své, i radovali se spolu s ní.
\par 59 Stalo se pak v den osmý, prišli obrezovati dítete, a nazývali jej jménem otce jeho Zachariášem.
\par 60 Ale odpovedevši matka jeho, rekla: Nikoli, ale slouti bude Jan.
\par 61 I rekli k ní: Však nižádného není v rodu tvém, kterýž by sloul jménem tím.
\par 62 I dávali náveští otci jeho, jak by ho chtel nazývati.
\par 63 A on požádav dešticky, napsal rka: Jan jest jméno jeho. I divili se všickni.
\par 64 A ihned otevrela se ústa jeho a jazyk jeho, i mluvil, velebe Boha.
\par 65 Tedy prišla bázen na všecky sousedy jejich, a po všech horách Judských rozhlásána jsou všecka ta slova.
\par 66 A všickni, kteríž o tom slyšeli, skládali to v srdci svém, rkouce: I kteraké díte toto bude? A ruka Páne byla s ním.
\par 67 Zachariáš pak otec jeho naplnen jest Duchem svatým, a prorokoval rka:
\par 68 Požehnaný Pán Buh Izraelský, že jest navštívil, a ucinil vykoupení lidu svému,
\par 69 A vyzdvihl nám roh spasení v domu Davida, služebníka svého,
\par 70 Jakož mluvil skrze ústa proroku svých svatých, kteríž byli od veku,
\par 71 O vysvobození z neprátel našich, a z ruky všech, kteríž nás nenávideli,
\par 72 Aby ucinil milosrdenství s otci našimi, a rozpomenul se na smlouvu svou svatou,
\par 73 Na prísahu, kterouž jest prisáhl Abrahamovi, otci našemu, že jiste nám to dá,
\par 74 Abychom bez strachu, z ruky neprátel našich jsouce vysvobozeni, sloužili jemu,
\par 75 V svatosti a v spravedlnosti pred oblicejem jeho, po všecky dny života našeho.
\par 76 Ty pak, díte, prorokem Nejvyššího slouti budeš, nebo predejdeš pred tvárí Páne pripravovati cesty jeho,
\par 77 Aby bylo dáno umení spasitelné lidu jeho na odpuštení hríchu jejich,
\par 78 Skrze srdecné milosrdenství Boha našeho, v nemžto navštívil nás, vyšed z výsosti,
\par 79 Aby se ukázal sedícím v temnostech a v stínu smrti, k spravení noh našich na cestu pokoje.
\par 80 Díte pak rostlo a posilovalo se v duchu, a bylo na poušti až do dne zjevení svého lidu Izraelskému.

\chapter{2}

\par 1 I stalo se v tech dnech, vyšlo jest vyrcení od císare Augusta, aby byl popsán všecken svet.
\par 2 (To popsání nejprve stalo se, když vladarem Syrským byl Cyrenius.)
\par 3 I šli všickni, aby zapsáni byli, jeden každý do svého mesta.
\par 4 Vstoupil pak i Jozef od Galilee z mesta Nazarétu do Judstva, do mesta Davidova, kteréž slove Betlém, (protože byl z domu a z celedi Davidovy,)
\par 5 Aby zapsán byl s Marijí, zasnoubenou sobe manželkou, tehotnou.
\par 6 I stalo se, když tam byli, naplnili se dnové Marie, aby porodila.
\par 7 I porodila Syna svého prvorozeného, a plénkami ho obvinula, a položila jej v jeslech, neb nemeli jinde místa v hospode.
\par 8 A pastýri byli v krajine té, ponocujíce a stráž nocní držíce nad svým stádem.
\par 9 A aj, andel Páne postavil se podle nich, a sláva Páne osvítila je. I báli se bázní velikou.
\par 10 Tedy rekl jim andel: Nebojtež se; nebo aj, zvestuji vám radost velikou, kteráž bude všemu lidu.
\par 11 Nebo narodil se vám dnes Spasitel, jenž jest Kristus Pán, v meste Davidove.
\par 12 A toto vám bude za znamení: Naleznete nemluvnátko plénkami obvinuté, a ležící v jeslech.
\par 13 A hned s andelem zjevilo se množství rytírstva nebeského, chválících Boha a rkoucích:
\par 14 Sláva na výsostech Bohu, a na zemi pokoj, lidem dobrá vule.
\par 15 I stalo se, jakž odešli od nich andelé do nebe, že ti lidé, totiž pastýri, rekli vespolek: Pojdme až do Betléma a vizme tu vec, jenž se stala, o níž Pán oznámil nám.
\par 16 I prišli, chvátajíce, a nalezli Mariji a Jozefa, a nemluvnátko ležící v jeslech.
\par 17 A videvše, rozhlašovali to, což jim povedíno bylo o tom díteti.
\par 18 I divili se všickni, kteríž slyšeli o tom, což bylo mluveno od pastýru k nim.
\par 19 Ale Maria zachovávala všecka slova tato, skládajici je v srdci svém.
\par 20 I navrátili se pastýri, velebíce a chválíce Boha ze všeho, což slyšeli a videli, tak jakž bylo povedíno jim.
\par 21 A když se naplnilo dní osm, aby obrezáno bylo díte, nazváno jest jméno jeho Ježíš, kterýmž bylo nazváno od andela, prve než se v živote pocalo.
\par 22 A když se naplnili dnové ocištování Marie podle Zákona Mojžíšova, prinesli jej do Jeruzaléma, aby ho postavili prede Pánem,
\par 23 (Jakož psáno jest v Zákone Páne, že každý pacholík, otvíraje život, svatý Pánu slouti bude,)
\par 24 A aby dali obet, jakož povedíno jest v Zákone Páne, dve hrdlicky anebo dvé holoubátek.
\par 25 A aj, clovek jeden byl v Jeruzaléme, jemuž jméno Simeon. A clovek ten byl spravedlivý a nábožný, ocekávající potešení Izraelského, a Duch svatý byl v nem.
\par 26 A bylo jemu zjeveno od Ducha svatého, že neuzrí smrti, až by prve uzrel Krista Páne.
\par 27 Ten prišel, ponuknut jsa od Ducha Páne, do chrámu. A když uvodili díte Ježíše rodicové, aby ucinili podle obyceje Zákona za nej,
\par 28 Tedy on vzal jej na lokty své, i chválil Boha a rekl:
\par 29 Nyní propouštíš služebníka svého, Pane, podle slova svého v pokoji.
\par 30 Nebot jsou videly oci mé spasení tvé,
\par 31 Kteréž jsi pripravil pred oblicejem všech lidí,
\par 32 Svetlo k zjevení národum a slávu lidu tvého Izraelského.
\par 33 Otec pak a matka jeho divili se tem vecem, kteréž praveny byly o nem.
\par 34 I požehnal jim Simeon, a rekl k Mariji, matce jeho: Aj, položen jest tento ku pádu a ku povstání mnohým v Izraeli, a na znamení, kterémužto bude odpíráno,
\par 35 (A tvou vlastní duši pronikne mec,) aby zjevena byla z mnohých srdcí myšlení.
\par 36 Byla také Anna prorokyne, dcera Fanuelova z pokolení Aser. Ta se byla zstarala ve dnech mnohých, a živa byla s mužem svým sedm let od panenství svého.
\par 37 A ta vdova byla, mající let okolo osmdesáti a ctyr, kteráž nevycházela z chrámu, posty a modlitbami sloužeci Bohu dnem i nocí.
\par 38 A ta v touž hodinu prišedši, chválila Pána, a mluvila o nem všechnem, kteríž cekali vykoupení v Jeruzaléme.
\par 39 Oni pak, jakž vykonali všecko podle Zákona Páne, vrátili se do Galilee, do mesta svého Nazaréta.
\par 40 Díte pak rostlo a posilovalo se v duchu, plné moudrosti, a milost Boží byla v nem.
\par 41 I chodívali rodicové jeho každého roku do Jeruzaléma na den slavný velikonocní.
\par 42 A když byl ve dvanácti letech, a oni vstupovali do Jeruzaléma, podle obyceje toho dne svátecního,
\par 43 A když vykonali dni, a již se navracovali, zustalo díte Ježíš v Jeruzaléme, a nevedeli o tom Jozef a matka jeho.
\par 44 Domnívajíce se pak o nem, že by byl v zástupu, ušli den cesty. I hledali ho mezi príbuznými a známými.
\par 45 A nenalezše jeho, navrátili se do Jeruzaléma, hledajíce ho.
\par 46 I stalo se po trech dnech, že nalezli jej v chráme, an sedí mezi doktory, poslouchaje jich a otazuje se jich.
\par 47 A desili se všickni, kteríž jej slyšeli, nad rozumností a odpovedmi jeho.
\par 48 A uzrevše ho, ulekli se. I rekla matka jeho k nemu: Synu, proc jsi nám tak ucinil? Aj, otec tvuj a já s bolestí hledali jsme tebe.
\par 49 I rekl k nim: Co jest, že jste mne hledali? Zdaliž jste nevedeli, že v tech vecech, kteréž jsou Otce mého, musím já býti?
\par 50 Ale oni nesrozumeli tem slovum, kteráž k nim mluvil.
\par 51 I šel s nimi, a prišel do Nazarétu, a byl poddán jim. Matka pak jeho zachovávala všecka slova ta v srdci svém.
\par 52 A Ježíš prospíval moudrostí, a vekem, a milostí, u Boha i u lidí.

\chapter{3}

\par 1 Léta pak patnáctého císarství Tiberia císare, když Pontský Pilát spravoval Judstvo, a Herodes byl ctvrtákem v Galilei, Filip pak bratr jeho byl ctvrtákem Iturejské a Trachonitidské krajiny, a Lyzaniáš ctvrtákem Abilinským,
\par 2 Za nejvyššího kneze Annáše a Kaifáše, stalo se slovo Páne nad Janem synem Zachariášovým na poušti.
\par 3 I chodil po vší krajine ležící pri Jordánu a kázal krest pokání na odpuštení hríchu,
\par 4 Jakož psáno jest v knihách proroctví Izaiáše proroka, rkoucího: Hlas volajícího na poušti: Pripravujte cestu Páne, prímé cinte stezky jeho.
\par 5 Každé údolí bude vyplneno, a každá hora a pahrbek bude ponížen; i budou krivé veci spraveny a ostré cesty budou hladké.
\par 6 A uzrít všeliké telo spasení Boží.
\par 7 Pravil pak k zástupum vycházejícím, aby se krtili od neho: Plemeno ještercí, kdo je vám ukázal, abyste utekli pred budoucím hnevem?
\par 8 Protož cinte ovoce hodné pokání, a neríkejtež u sebe: Otce máme Abrahama. Nebot pravím vám, žet jest mocen Buh z kamení tohoto vzbuditi syny Abrahamovi.
\par 9 A jižt jest i sekera k korenu stromu priložena. Protož každý strom, jenž nenese ovoce dobrého, vytat a na ohen uvržen bude.
\par 10 I tázali se ho zástupové, rkouce: Což tedy ciniti budeme?
\par 11 A on odpovídaje, pravil jim: Kdo má dve sukne, dej jednu nemajícímu, a kdo má pokrmy, tolikéž ucin.
\par 12 Prišli pak i celní krtíti se, i rekli jemu: Mistre, co budeme ciniti?
\par 13 A on rekl k nim: Nic více nevybírejte než to, což jest ustaveno.
\par 14 I tázali se ho také i žoldnéri, rkouce: A my což ciniti budeme? I rekl jim: Nižádného neutiskujte, ani podvodne cinte, a dosti mejte na svých žoldích.
\par 15 A když pak lid ocekával, a myslili všickni v srdcích svých o Janovi, nebyl-li by snad on Kristus,
\par 16 Odpovedel Jan všechnem, a rka: Ját zajisté krtím vás vodou, ale jdet silnejší nežli já, kteréhožto nejsem hoden rozvázati reménka u obuvi jeho. Tent vás krtíti bude Duchem svatým a ohnem.
\par 17 Jehožto vejecka v ruce jeho, a vycistít humno své, a shromáždí pšenici do obilnice své, ale plevy páliti bude ohnem neuhasitelným.
\par 18 A tak mnohé jiné veci, napomínaje, zvestoval lidu.
\par 19 Herodes pak ctvrták, když od neho byl trestán pro Herodiadu manželku Filipa bratra svého, i ze všech zlých vecí, kteréž cinil Herodes,
\par 20 Pridal i toto nade všecko, že jest vsadil Jana do žaláre.
\par 21 I stalo se, když se krtil všecken lid, a když se pokrtil i Ježíš, a modlil se, že otevrelo se nebe,
\par 22 A sstoupil Duch svatý v telesné zpusobe jako holubice na nej, a stal se hlas s nebe, rkoucí: Ty jsi Syn muj milý, v tobet mi se zalíbilo.
\par 23 Ježíš pak pocínal býti jako ve tridcíti letech, jakž domnín byl syn Jozefuv, kterýž byl syn Heli,
\par 24 Kterýž byl Matatuv, kterýž byl Léví, kterýž byl Melchuv, kterýž byl Jannuv, kterýž byl Jozefuv,
\par 25 Kterýž byl Matatiášuv, kterýž byl Amosuv, kterýž byl Naum, kterýž byl Esli, kterýž byl Nagge,
\par 26 Kterýž byl Mahatuv, kterýž byl Matatiášuv, kterýž byl Semei, kterýž byl Jozefuv, kterýž byl Juduv,
\par 27 Kterýž byl Johannuv, kterýž byl Resuv, kterýž byl Zorobábeluv, kterýž byl Salatieluv, kterýž byl Neriuv,
\par 28 Kterýž byl Melchiuv, kterýž byl Addiuv, kterýž byl Kozamuv, kterýž byl Elmódamuv, kterýž byl Eruv,
\par 29 Kterýž byl Józuv, kterýž byl Eliezeruv, kterýž byl Jórimuv, kterýž byl Matatuv, kterýž byl Léví,
\par 30 Kterýž byl Simeonuv, kterýž byl Juduv, kterýž byl Jozefuv, kterýž byl Jónamuv, kterýž byl Eliachimuv,
\par 31 Kterýž byl Meleuv, kterýž byl Ménamuv, kterýž byl Matatanuv, kterýž byl Nátanuv, kterýž byl Daviduv,
\par 32 Kterýž byl Jesse, kterýž byl Obéduv, kterýž byl Bózuv, kterýž byl Salmonuv, kterýž byl Názonuv,
\par 33 Kterýž byl Aminadabuv, kterýž byl Aramuv, kterýž byl Ezromuv, kterýž byl Fáresuv, kterýž byl Juduv, kterýž byl Jákobuv,
\par 34 Kterýž byl Izákuv, kterýž byl Abrahamuv, kterýž byl Táre, kterýž byl Náchoruv,
\par 35 Kterýž byl Sáruchuv, kterýž byl Ragauv, kterýž byl Fálekuv, kterýž byl Heberuv, kterýž byl Sále,
\par 36 Kterýž byl Kainanuv, kterýž byl Arfaxaduv, kterýž byl Semuv, kterýž byl Noé, kterýž byl Lámechuv,
\par 37 Kterýž byl Matuzalémuv, kterýž byl Enochuv, kterýž byl Járeduv, kterýž byl Malaleheluv, kterýž byl Kainanuv,
\par 38 Kterýž byl Enosuv, kterýž byl Setuv, kterýž byl Adamuv, kterýž byl Boží.

\chapter{4}

\par 1 Ježíš pak, pln jsa Ducha svatého, vrátil se od Jordánu, a puzen jest v Duchu na poušt.
\par 2 A za ctyridceti dní pokoušín byl od dábla, a nic nejedl v tech dnech. A když se skonali dnové ti, potom zlacnel.
\par 3 I rekl jemu dábel: Jestliže jsi Syn Boží, rci kamenu tomuto, at jest chléb.
\par 4 I odpovedel jemu Ježíš: Psánot jest: Že ne samým chlebem živ bude clovek, ale každým slovem Božím.
\par 5 I vedl jej dábel na horu vysokou, a ukázal mu všecka království okršlku zeme pojednou.
\par 6 A rekl jemu dábel: Tobet dám tuto všecku moc i slávu techto království, nebo mne dána jest, a komuž bych koli chtel, dám ji.
\par 7 Protož ty pokloníš-li se prede mnou, budet všecko tvé.
\par 8 I odpovedev Ježíš, rekl jemu: Jdi pryc ode mne, satanáši; nebot psáno jest: Pánu Bohu svému budeš se klaneti a jemu samému sloužiti.
\par 9 Tedy vedl jej do Jeruzaléma, a postavil ho na vrchu chrámu, a rekl mu: Jsi-li Syn Boží, pust se odtud dolu.
\par 10 Nebo psáno jest: Že andelum svým prikáže o tobe, aby te ostríhali,
\par 11 A že te na ruce uchopí, abys neurazil o kámen nohy své.
\par 12 A odpovídaje, dí mu Ježíš: Povedínot jest: Nebudeš pokoušeti Pána Boha svého.
\par 13 A dokonav všecka pokušení dábel, odšel od neho až do casu.
\par 14 I navrátil se Ježíš v moci Ducha do Galilee, a vyšla povest o nem po vší té okolní krajine.
\par 15 A on ucil v školách jejich, a slaven byl ode všech.
\par 16 I prišel do Nazaréta, kdež byl vychován, a všel podle obyceje svého v den sobotní do školy. I vstal, aby cetl.
\par 17 I dána jemu kniha Izaiáše proroka. A otevrev knihu, nalezl místo, kdež bylo napsáno:
\par 18 Duch Páne nade mnou, protože pomazal mne, kázati evangelium chudým poslal mne, a uzdravovati zkroušené srdcem, zvestovati jatým propuštení a slepým videní, a propustiti soužené v svobodu,
\par 19 A zvestovati léto Páne vzácné.
\par 20 A zavrev knihu a vrátiv služebníku, posadil se. A všech v škole oci byly obráceny nan.
\par 21 I pocal mluviti k nim: Že dnes naplnilo se písmo toto v uších vašich.
\par 22 A všickni jemu posvedcovali, a divili se libým slovum, pocházejícím z úst jeho, a pravili: Zdaliž tento není syn Jozefuv?
\par 23 I dí k nim: Zajisté díte mi toto podobenství: Lékari, uzdrav se sám. Které veci slyšeli jsme, žes cinil v Kafarnaum, ucin také i zde v své vlasti.
\par 24 I rekl jim: Amen pravím vám, žet žádný prorok není vzácen v vlasti své.
\par 25 Ale v pravde pravím vám, že mnoho vdov bylo za dnu Eliáše v lidu Izraelském, kdyžto zavríno bylo nebe za tri léta a za šest mesícu, takže byl hlad veliký po vší zemi,
\par 26 Však Eliáš k nižádné z nich není poslán, než toliko do Sarepty Sidonské k žene vdove.
\par 27 A mnoho malomocných bylo v lidu Izraelském za Elizea proroka, a žádný z nich není ocišten, než Náman Syrský.
\par 28 I naplneni jsou všickni v škole hnevem, slyšíce to.
\par 29 A povstavše, vyvedli jej ven z mesta, a vedli ho až na vrch hory, na nížto mesto jejich bylo vzdeláno, aby jej dolu sstrcili.
\par 30 Ale on bera se prostredkem jich, ušel.
\par 31 I sstoupil do Kafarnaum, mesta Galilejského, a tu ucil je ve dny sobotní.
\par 32 I divili se velmi ucení jeho, nebo mocná byla rec jeho.
\par 33 Byl pak tu v škole clovek jeden, mající ducha dábelství necistého. I zvolal hlasem velikým,
\par 34 Rka: Ach, což jest tobe do nás, Ježíši Nazaretský? Prišel jsi zatratiti nás? Znám te, kdo jsi, že jsi ten svatý Boží.
\par 35 I primluvil jemu Ježíš, rka: Umlkniž a vyjdi od neho. A povrha jej dábel mezi ne, vyšel od neho, a nic mu neuškodil.
\par 36 I prišel strach na všecky, a rozmlouvali vespolek, rkouce: Jaké jest toto slovo, že v moci a síle prikazuje necistým duchum, a vycházejí?
\par 37 I rozcházela se o nem povest po všem okolí té krajiny.
\par 38 Vstav pak Ježíš ze školy, všel do domu Šimonova. Svegruše pak Šimonova trápena byla težkou zimnicí. I prosili ho za ni.
\par 39 Tedy stoje nad ní, primluvil zimnici, i prestala jí. A ona hned vstavši, posluhovala jim.
\par 40 Pri západu pak slunce všickni, kteríž meli nemocné rozlicnými neduhy, vodili je k nemu, a on na jednoho každého z nich ruce vzkládav, uzdravoval je.
\par 41 Od mnohých také dábelství vycházela, kricící a ríkající: Ty jsi Kristus, Syn Boží. Ale on primlouvaje, nedopouštel jim mluviti; nebo vedeli, že jest on Kristus.
\par 42 A jakž byl den, vyšed, bral se na pusté místo. I hledali ho zástupové, a prišli až k nemu, a zdržovali ho, aby neodcházel od nich.
\par 43 On pak rekl jim: I jinýmt mestum musím zvestovati království Boží; nebo na to poslán jsem.
\par 44 I kázal v školách Galilejských.

\chapter{5}

\par 1 Stalo se pak, když se zástup na nej valil, aby slyšeli slovo Boží, že on stál podle jezera Genezaretského.
\par 2 I uzrel dve lodí, any stojí u jezera, rybári pak sstoupivše z nich, vypírali síti.
\par 3 I vstoupiv na jednu z tech lodí, kteráž byla Šimonova, prosil ho, aby od zeme odvezl malicko. A posadiv se, ucil z lodí zástupy.
\par 4 A když prestal mluviti, dí k Šimonovi: Vez na hlubinu, a rozestrete síti své k lovení ryb.
\par 5 I odpovedev Šimon, rekl jemu: Mistre, pres celou noc pracovavše, nic jsme nepopadli, ale k slovu tvému rozestru sít.
\par 6 A když to ucinili, zahrnuli množství veliké ryb, takže se trhala sít jejich.
\par 7 I ponukli tovaryšu, kteríž byli na druhé lodí, aby prišli a pomohli jim. I prišli a naplnili obe lodí, takže se témer pohrižovaly.
\par 8 To uzrev Šimon Petr, padl k nohám Ježíšovým, rka: Odejdi ode mne, Pane, nebot jsem clovek hríšný.
\par 9 Hruza zajisté byla jej obklícila, i všecky, kteríž s ním byli, nad tím lovením ryb, kteréž byli popadli,
\par 10 A též Jakuba a Jana, syny Zebedeovy, kteríž byli tovaryši Šimonovi. I dí Šimonovi Ježíš: Nebojž se. Již od tohoto casu lidi živé budeš loviti.
\par 11 A privezše k brehu lodí, a všecko opustivše, šli za ním.
\par 12 I stalo se, když byl v jednom meste, a aj, byl tam muž plný malomocenství. A uzrev Ježíše, padl na tvár, a prosil ho, rka: Pane, kdybys chtel, mužeš mne ocistiti.
\par 13 I vztáh Ježíš ruku, dotekl se ho, rka: Chci, bud cist. A hned odešlo od neho malomocenství.
\par 14 I prikázal jemu, aby žádnému nepravil, ale rekl mu: Odejda, ukaž se knezi, a obetuj za ocištení své, jakož prikázal Mojžíš, na svedectví jim.
\par 15 Tedy rozhlašovala se více rec o nem, a scházeli se zástupové mnozí, aby jej slyšeli, a uzdravováni byli od neho v svých nemocech.
\par 16 On pak odcházel na poušte, a modlil se.
\par 17 I stalo se v jeden den, že on sedel uce, a sedeli také tu i farizeové a Zákona ucitelé, kteríž se byli sešli z každého mestecka Galilejského a Judského i z Jeruzaléma, a moc Páne prítomná byla k uzdravování jich.
\par 18 A aj, muži nesli na loži cloveka, kterýž byl šlakem poražený, i hledali vnésti ho a položiti pred nej.
\par 19 A nenalezše, kterou by jej stranou vnesli pro zástup, vstoupili na dum, a skrze podlahu spustili jej s ložem uprostred pred Ježíše.
\par 20 Kterýžto videv víru jejich, rekl mu: Clovece, odpuštenit jsou tobe hríchové tvoji.
\par 21 Tedy pocali premyšlovati zákoníci a farizeové, rkouce: Kdo jest tento, jenž mluví rouhání? Kdo muže odpustiti hríchy, jediné sám Buh?
\par 22 Poznav pak Ježíš myšlení jejich, odpovídaje, rekl k nim: Co tak utrhave premyšlujete v srdcích vašich?
\par 23 Co jest snáze ríci: Odpouštejí se tobe hríchové tvoji, cili ríci: Vstan a chod?
\par 24 Ale abyste vedeli, že Syn cloveka má moc na zemi odpoušteti hríchy, (rekl dnou zlámanému:) Tobet pravím: Vstan, a vezma lože své, jdi do domu svého.
\par 25 A on hned vstav pred nimi, vzal lože, na nemž ležel, i odšel do domu svého, velebe Boha.
\par 26 I užasli se všickni, i velebili Boha, a naplneni jsou bázní, rkouce: Že jsme videli dnes divné veci.
\par 27 A potom vyšel Ježíš a uzrel celného, jménem Léví, sedícího na cle. I rekl jemu: Pojd za mnou.
\par 28 A on opustiv všecko, vstav, šel za ním.
\par 29 I ucinil jemu hody veliké Léví v domu svém, a byl tu zástup veliký publikánu i jiných, kteríž s ním stolili.
\par 30 Tedy reptali zákoníci a farizeové, rkouce ucedlníkum jeho: Proc s publikány a hríšníky jíte a pijete?
\par 31 I odpovedev Ježíš, rekl k nim: Nepotrebujít zdraví lékare, ale nemocní.
\par 32 Neprišelt jsem volati spravedlivých, ale hríšných ku pokání.
\par 33 A oni rekli jemu: Proc ucedlníci Janovi postí se casto a modlí se, též podobne i farizejští, tvoji pak jedí a pijí?
\par 34 On pak rekl k nim: Zdali mužete synum ženicha, dokudž s nimi jest ženich, kázati se postiti?
\par 35 Ale prijdout dnové, a když odjat bude od nich ženich, tehdážt se budou postiti v tech dnech.
\par 36 Pravil pak i podobenství k nim: Že žádný záplaty roucha nového neprišívá k rouchu vetchému; sic jinak i nové roztrhuje, a vetchému neprísluší záplata z nového.
\par 37 A žádný nevlévá vína nového do nádob starých; sic jinak víno nové rozpucí nádoby, a samo vytece, a nádoby se pokazí.
\par 38 Ale víno nové v nádoby nové má lito býti, a obé bude zachováno.
\par 39 A aniž kdo, když pije staré, hned chce nového, ale dít: Staré lepší jest.

\chapter{6}

\par 1 I stalo se v druhou sobotu, že šel Ježíš skrze obilí. I trhali ucedlníci jeho klasy, a rukama vymínajíce, jedli.
\par 2 Tehdy nekterí z farizeu rekli jim: Proc to ciníte, cehož nesluší ciniti v svátky?
\par 3 I odpovedev Ježíš, rekl jim: Což jste ani toho nectli, co jest ucinil David, když lacnel, on i ti, kteríž s ním byli?
\par 4 Kterak všel do domu Božího, a chleby posvátné vzal a jedl, a dal i tem, kteríž s ním byli, jichžto nenáleží jísti než toliko samým knežím?
\par 5 I rekl jim: Že jest Syn cloveka pánem také i dne svátecního.
\par 6 Stalo se pak i v jiný den svátecní, že všel do školy Ježíš, a ucil. A byl tu clovek, jehož pravá ruka byla uschlá.
\par 7 I šetrili ho zákoníci a farizeové, bude-li v sobotu uzdravovati, aby nalezli, cím by jej obžalovali.
\par 8 Ale on znal premyšlování jich. I dí cloveku, kterýž mel ruku uschlou: Vstan, a stuj v prostredku. A on vstav, i stál.
\par 9 Tedy rekl k nim Ježíš: Otíži se vás na jednu vec: Sluší-li v sobotu dobre ciniti? duši zachovati, cili zatratiti?
\par 10 A pohledev na ne na všecky vukol, dí cloveku: Vztáhni ruku svou. A on ucinil tak. I navrácena jest k zdraví ruka jeho a byla jako druhá.
\par 11 Oni pak naplneni jsou hnevivou nemoudrostí, a rozmlouvali mezi sebou, co by uciniti meli Ježíšovi.
\par 12 I stalo se v tech dnech, vyšel Ježíš na horu k modlení. I byl tam pres noc na modlitbe Boží.
\par 13 A když byl den, povolal ucedlníku svých, a vyvolil z nich dvanácte, kteréž i apoštoly nazval.
\par 14 (Šimona, kterémuž také dal jméno Petr, a Ondreje bratra jeho, Jakuba a Jana, Filipa a Bartolomeje,
\par 15 Matouše a Tomáše, Jakuba syna Alfeova, a Šimona, kterýž slove Zelótes,
\par 16 Judu bratra Jakubova, a Jidáše Iškariotského, kterýž pak byl zrádce.)
\par 17 I sstoupiv s nimi s hory, stál na míste polním, a zástup ucedlníku jeho, a množství veliké lidu ze všeho Judstva i z Jeruzaléma, i z Týru i z Sidonu, jenž pri mori jsou, kteríž byli prišli, aby jej slyšeli a uzdraveni byli od neduhu svých,
\par 18 I kteríž trápeni byli od duchu necistých. A byli uzdravováni.
\par 19 A všecken zástup hledal se ho dotknouti; nebo moc z neho vycházela, a uzdravovala všecky.
\par 20 A on pozdvih ocí svých na ucedlníky, pravil: Blahoslavení chudí, nebo vaše jest království Boží.
\par 21 Blahoslavení, kteríž nyní lacníte, nebo nasyceni budete. Blahoslavení, kteríž nyní placete, nebo smáti se budete.
\par 22 Blahoslavení budete, když vás nenávideti budou lidé, a když vás vyobcují, a haneti budou, a vyvrhou jméno vaše jakožto zlé, pro Syna cloveka.
\par 23 Radujte se v ten den a veselte se, nebo aj, odplata vaše mnohá jest v nebesích. Takt jsou zajisté cinívali prorokum otcové jejich.
\par 24 Ale beda vám bohatým, nebo vy již máte potešení své.
\par 25 Beda vám, kteríž jste nasyceni, nebo lacneti budete. Beda vám, kteríž se nyní smejete, nebo kvíliti a plakati budete.
\par 26 Beda vám, když by dobre o vás mluvili všickni lidé; nebo tak jsou cinívali falešným prorokum otcové jejich.
\par 27 Ale vámt pravím, kteríž slyšíte: Milujte neprátely vaše, dobre cinte tem, kteríž vás nenávidí,
\par 28 Dobrorecte tem, kteríž vás proklínají, a modlte se za ty, kteríž vám bezpráví ciní.
\par 29 A uderil-li by tebe kdo v líce jedno, nasad mu i druhého, a tomu, kterýž tobe odjímá plášt, také i sukne nebran.
\par 30 Každému pak prosícímu tebe dej, a od toho, jenž bére tvé veci, zase nežádej.
\par 31 A jakž chcete, aby vám lidé cinili, i vy jim též podobne cinte.
\par 32 Nebo jestliže milujete ty, kteríž vás milují, jakou míti budete milost? Nebo i hríšníci milují ty, od nichž milováni bývají.
\par 33 A budete-li dobre ciniti tem, kteríž vám dobre ciní, jakou máte milost? Však i hríšníci totéž ciní.
\par 34 A budete-li pujcovati tem, od kterýchž se nadejete zase vzíti, jakou máte milost? Však i hríšníci hríšníkum pujcují, aby tolikéž zase vzali.
\par 35 Protož milujte neprátely vaše, a dobre cinte, a pujcujte, nic se odtud nenadejíce, a budet odplata vaše mnohá, a budete synové Nejvyššího. Nebo on dobrotivý jest i k nevdecným a zlým.
\par 36 Protož budte milosrdní, jako i Otec váš milosrdný jest.
\par 37 Nesudte, a nebudete souzeni. Nepotupujte, a nebudete potupeni. Odpouštejte, a budet vám odpušteno.
\par 38 Dávejte, a budet vám dáno. Míru dobrou, natlacenou, a natresenou, a osutou dadít v luno vaše; touž zajisté merou, kterouž meríte, bude vám odmereno.
\par 39 Povedel jim také i podobenství: Zdali muže slepý slepého vésti? Zdaž oba do jámy neupadnou?
\par 40 Nenít ucedlník nad mistra svého, ale dokonalý bude každý, bude-li jako mistr jeho.
\par 41 Což pak vidíš mrvu v oku bratra svého, a brevna, kteréž jest v tvém vlastním oku, neznamenáš?
\par 42 Aneb kterak mužeš ríci bratru svému: Bratre, nechat vyvrhu mrvu z oka tvého, sám v oku svém brevna nevida? Pokrytce, vyvrz prve brevno z oka svého, a tehdy prohlédneš, abys vynal mrvu, kteráž jest v oku bratra tvého.
\par 43 Nebot není ten strom dobrý, kterýž nese ovoce zlé, aniž jest strom zlý, kterýž nese ovoce dobré.
\par 44 Každý zajisté strom po svém vlastním ovoci bývá poznán; nebo nesbírají s trní fíku, ani s hloží sbírají hroznu.
\par 45 Dobrý clovek z dobrého pokladu srdce svého vynáší dobré, a zlý clovek ze zlého pokladu srdce svého vynáší zlé. Nebo z hojnosti srdce mluví ústa jeho.
\par 46 Co pak mi ríkáte: Pane, Pane, a neciníte, což pravím?
\par 47 Každý kdož prichází ke mne, a slyší slovo mé, a zachovává je, ukáži vám, komu by podoben byl.
\par 48 Podoben jest cloveku stavejícímu dum, kterýž kopal hluboko, a založil grunty v skále. A když se stala povoden, oborila se reka na dum ten, ale nemohla jím pohnouti, nebo byl založen na skále.
\par 49 Ale kdož slyší a neciní, podoben jest cloveku, kterýž staví dum svuj na zemi bez gruntu. Na kterýžto oborila se reka, a on hned padl, i stal se pád domu toho veliký.

\chapter{7}

\par 1 A když vykonal všecka slova svá pri prítomnosti lidu, všel do Kafarnaum.
\par 2 Setníka pak jednoho služebník nemocen jsa, k smrti se približoval, kteréhož on sobe mnoho vážil.
\par 3 I uslyšav o Ježíšovi, poslal k nemu starší z Židu, prose ho, aby prišel a uzdravil služebníka jeho.
\par 4 A oni prišedše k Ježíšovi, prosili ho snažne, rkouce: Hoden jest, abys jemu to ucinil.
\par 5 Nebo miluje národ náš, a školu on nám vystavel.
\par 6 Tedy Ježíš šel s nimi. A když již nedaleko byl od domu, poslal k nemu setník prátely, rka jemu: Pane, nepridávej sobe práce. (Nejsem zajisté hoden, abys všel pod strechu mou.
\par 7 A protožt jsem i sebe samého za nehodného položil, abych prišel k tobe.) Ale rci slovem, a budet uzdraven služebník muj.
\par 8 Nebo i já jsem clovek pod mocí postavený, maje pod sebou žoldnére, a dím tomuto: Jdi, a jde, a jinému: Prijd, a prijde, a služebníku svému: Ucin toto, a uciní.
\par 9 Tedy uslyšav to Ježíš, podivil se jemu, a obrátiv se k zástupu, kterýž za ním šel, rekl: Pravím vám, že ani v Izraeli nenalezl jsem tak veliké víry.
\par 10 Vrátivše se pak do domu ti, kteríž posláni byli, nalezli služebníka, kterýž nemocen byl, zdravého.
\par 11 I stalo se potom, šel Ježíš do mesta, kteréž slove Naim, a šli s ním ucedlníci jeho mnozí a zástup veliký.
\par 12 A když se priblížil k bráne mesta, aj, mrtvý byl nesen ven, syn jediný matky své, a ta vdova byla, a zástup mesta mnohý s ní.
\par 13 Kteroužto uzrev Pán, milosrdenstvím hnut jsa nad ní, rekl jí: Neplaciž.
\par 14 A pristoupiv, dotekl se már. (Ti pak, kteríž nesli, zastavili se.) I rekl: Mládence, tobet pravím, vstan.
\par 15 I pozdvih se mrtvý, sedl, a pocal mluviti. I dal jej materi jeho.
\par 16 Tedy podjala všecky bázen, i velebili Boha, rkouce: Že prorok veliký povstal mezi námi, a že Buh navštívil lid svuj.
\par 17 I vyšla rec ta o nem po všem Judstvu i po vší okolní krajine.
\par 18 I zvestovali Janovi ucedlníci jeho o všech techto vecech. A zavolav kterýchs dvou z ucedlníku svých Jan,
\par 19 Poslal k Ježíšovi, rka: Ty-li jsi ten, kterýž prijíti má, cili jiného cekati máme?
\par 20 Prišedše pak k nemu muži ti, rekli: Jan Krtitel poslal nás k tobe, rka: Ty-li jsi ten, kterýž prijíti má, cili jiného cekati máme?
\par 21 A v touž hodinu mnohé uzdravil od neduhu, od nemocí a duchu zlých, a slepým mnohým zrak dal.
\par 22 Odpovedev pak Ježíš, rekl jim: Jdouce, povezte Janovi, co jste videli a slyšeli, že slepí vidí, kulhaví chodí, malomocní ocištení prijímají, hluší slyší, mrtví z mrtvých vstávají, chudým se zvestuje evangelium.
\par 23 A blahoslavený jest, kdož by se na mne nezhoršil.
\par 24 A když odešli poslové Janovi, pocal praviti k zástupum o Janovi: Co jste vyšli na poušt spatrovati? Trtinu-li, kteráž se vetrem klátí?
\par 25 Anebo nac jste hledeti vyšli? Na cloveka-li mekkým rouchem odeného? Aj, kteríž v rouše slavném a v rozkoši jsou, v domích královských jsou.
\par 26 Aneb co jste vyšli videti? Proroka-li? Ovšem pravím vám, i více nežli proroka.
\par 27 Tentot jest, o kterémž jest psáno: Aj, já posílám andela svého pred tvárí tvou, jenž pripraví cestu tvou pred tebou.
\par 28 Nebo pravím vám, vetšího proroka mezi syny ženskými nad Jana Krtitele není žádného, ale kdož jest menší v království Božím, vetšít jest nežli on.
\par 29 Tedy všecken lid slyše to i publikáni, velebili Boha, byvše pokrteni krtem Janovým.
\par 30 Ale farizeové a zákoníci pohrdli radou Boží sami proti sobe, nebyvše pokrteni od neho.
\par 31 I rekl Pán: Komu tedy prirovnám lidi pokolení tohoto a cemu podobni jsou?
\par 32 Podobni jsou detem, jenž na rynku sedí a jedni na druhé volají, ríkajíce: Pískali jsme vám, a neskákali jste; žalostne jsme naríkali vám, a neplakali jste.
\par 33 Nebo prišel Jan Krtitel, nejeda chleba, ani pije vína, a pravíte: Dábelství má.
\par 34 Prišel Syn cloveka, jeda a pije, a pravíte: Aj, clovek žrác a pijan vína, prítel publikánu a hríšníku.
\par 35 Ale ospravedlnena jest moudrost ode všech synu svých.
\par 36 Prosil ho pak jeden z farizeu, aby jedl s ním. Procež všed do domu toho farizea, posadil se za stul.
\par 37 A aj, žena jedna v meste, kteráž byla hríšnice, zvedevši, že by sedel za stolem v dome farizea, prinesla nádobu alabastrovou masti.
\par 38 A stojeci zzadu u noh jeho, s plácem pocala slzami smáceti nohy jeho, a vlasy hlavy své vytírala, a líbala nohy jeho, a mastí mazala.
\par 39 Uzrev pak to farizeus, kterýž ho byl pozval, rekl sám v sobe: Byt tento byl prorok, vedelt by, která a jaká jest to žena, kteráž se ho dotýká; nebo hríšnice jest.
\par 40 I odpovedev Ježíš, dí k nemu: Šimone, mámt neco povedíti. A on rekl: Mistre, povez.
\par 41 I rekl Ježíš: Dva dlužníky mel jeden veritel. Jeden dlužen byl pet set penez, a druhý padesát.
\par 42 A když nemeli, odkud by zaplatili, odpustil obema. Poveziž tedy, který z nich více jej bude milovati?
\par 43 I odpovedev Šimon, rekl: Mám za to, že ten, kterémuž více odpustil. A on rekl jemu: Práve jsi rozsoudil.
\par 44 A obrátiv se k žene, rekl Šimonovi: Vidíš tuto ženu? Všel jsem do domu tvého, vody nohám mým nepodal jsi, ale tato slzami smácela nohy mé, a vlasy hlavy své vytrela.
\par 45 Nepolíbil jsi mne, ale tato, jakž jsem všel, neprestala líbati noh mých.
\par 46 Olejem hlavy mé nepomazal jsi, ale tato mastí mazala nohy mé.
\par 47 Protož pravím tobe: Odpuštenit jsou jí hríchové mnozí, nebot jest milovala mnoho. Komut se pak málo odpouští, málo miluje.
\par 48 I rekl k ní: Odpuštenit jsou tobe hríchové.
\par 49 Tedy pocali, kteríž tu spolu sedeli za stolem, ríci sami mezi sebou: Kdo jest tento, kterýž i hríchy odpouští?
\par 50 I rekl k žene: Víra tvá tebe k spasení privedla. Jdiž u pokoji.

\chapter{8}

\par 1 I stalo se potom, že on chodil po mestech a po mesteckách, káže a zvestuje království Boží, a dvanácte s ním,
\par 2 I ženy nekteré, kteréž byly uzdraveny od duchu zlých a od nemocí: Maria, jenž slove Magdaléna, z nížto bylo sedm dáblu vyšlo,
\par 3 A Johanna manželka Chuzova, úredníka Herodesova, a Zuzanna, a jiné mnohé, kteréž posluhovaly jemu z statku svých.
\par 4 Když se pak scházel zástup mnohý, a z okolních mest hrnuli se k nemu, mluvil jim v podobenství:
\par 5 Vyšel rozsevac, aby rozsíval síme své. A když on rozsíval, jedno padlo podle cesty, i pošlapáno jest, a ptáci nebeští szobali je.
\par 6 A jiné padlo na skálu, a vzešlé uvadlo, nebo nemelo vláhy.
\par 7 Jiné pak padlo mezi trní, a spolu vzrostlé trní udusilo je.
\par 8 A jiné padlo v zemi dobrou, a když vzešlo, ucinilo užitek stý. To povedev, volal: Kdo má uši k slyšení, slyš.
\par 9 I otázali se ho ucedlníci jeho, rkouce: Jaké jest to podobenství?
\par 10 A on rekl: Vám dáno jest znáti tajemství království Božího, ale jiným v podobenství, aby hledíce, nevideli, a slyšíce, nerozumeli.
\par 11 Jestit pak podobenství toto: Síme jest slovo Boží.
\par 12 A kteréž padlo podle cesty, jsou ti, kteríž slyší, a potom prichází dábel, a vynímá slovo z srdce jejich, aby neveríce, spaseni nebyli.
\par 13 Ale kteríž na skálu, ti když slyší, s radostí príjímají slovo, a tit korenu nemají; ti na cas verí, a v cas pokušení odstupují.
\par 14 Kteréž pak mezi trní padlo, tit jsou, kteríž slyšíce, a po pecování a zboží a rozkošech života jdouce, bývají udušeni, a neprinášejí užitku.
\par 15 Ale kteréž padlo v zemi dobrou, ti jsou, kterížto v srdci ctném a dobrém, slyšíce slovo, zachovávají je, a užitek prinášejí v trpelivosti.
\par 16 Nižádný pak rozsvíte svíci, neprikrývá jí nádobou, ani staví pod postel, ale na svícen staví, aby ti, kteríž vcházejí, svetlo videli.
\par 17 Nebo nic není tajného, což by nemelo býti zjeveno, ani co ukrytého, což by nemelo poznáno býti a na svetlo vyjíti.
\par 18 Protož vizte, jak slyšíte. Nebo kdož má, tomu bude dáno, a kdo nemá, i to, což domnívá se míti, bude odjato od neho.
\par 19 Tedy prišli k nemu matka a bratrí jeho, ale nemohli ho dojíti pro zástup.
\par 20 I povedeli mu, rkouce: Matka tvá a bratrí tvoji stojí vne, chtíce tebe videti.
\par 21 A on odpovedev, rekl k nim: Matka má a bratrí moji jsou ti, kteríž slovo Boží slyší a plní je.
\par 22 Stalo se pak v jeden den, že on vstoupil na lodí i ucedlníci jeho. I rekl k nim: Preplavme se pres jezero. I odstrcili lodí od brehu.
\par 23 A když se plavili, usnul. Tedy prišla boure tuhého vetru na jezero, a vlny lodí naplnovaly, takže v nebezpecenství byli.
\par 24 I pristoupivše, zbudili ho, rkouce: Mistre, Mistre, hyneme. A on procítiv, primluvil vetru a zdutí vod. I prestala boure, a stalo se utišení.
\par 25 I rekl jim: Kde je víra vaše? Kterížto bojíce se, podivili se, vespolek rkouce: I kdo jest tento, že vetrum prikazuje i vodám, a poslouchají ho?
\par 26 I plavili se do krajiny Gadarenské, kteráž jest proti Galilei.
\par 27 A když z lodí vystoupil na zemi, potkal jej muž jeden z mesta, kterýž mel dábelství od mnoha casu, a rouchem se neodíval, ani v domu býval, ale v hrobích.
\par 28 Ten uzrev Ježíše zkrikl a padl pred ním, a hlasem velikým rekl: Co je tobe do mne, Ježíši, Synu Boha nejvyššího? Prosím tebe, netrap mne.
\par 29 Nebo prikazoval duchu necistému, aby vyšel z toho cloveka. Po mnohé zajisté casy jím lomcoval, a býval ukován retezy a v poutech ostríhán, ale on polámal okovy a býval od dábelství puzen na poušt.
\par 30 I otázal se Ježíš, rka: Jakt ríkají? A on rekl: Tma. Neb bylo mnoho dáblu vešlo do neho.
\par 31 Tedy prosili ho, aby jim neprikazoval jíti do propasti.
\par 32 Bylo pak tu veliké stádo vepru, kteríž se pásli na hore. I prosili ho dáblové, aby jim dopustil do nich vjíti. I dopustil jim.
\par 33 I vyšedše dáblové z cloveka, vešli do vepru, a hned beželo stádo s chvátáním s vrchu do jezera, i ztonulo.
\par 34 A videvše pastýri, co se stalo, utekli pryc; a šedše, vypravovali to v meste i po vsech.
\par 35 I vyšli lidé, aby videli, co se stalo. I prišli k Ježíšovi, a nalezli cloveka toho, z kteréhož dáblové vyšli, odeného a majícího rozum, an sedí u noh Ježíšových. I báli se.
\par 36 A vypravovali jim také ti, kteríž byli videli, kterak jest zdráv ucinen ten, jenž mel dábelství.
\par 37 I prosilo ho to všecko množství té okolní krajiny Gadarenských, aby odšel od nich; nebo bázní velikou naplneni byli. A on vstoupiv na lodí, navrátil se.
\par 38 Prosil ho pak muž ten, z kteréhož dáblové vyšli, aby s ním byl. Ale Ježíš propustil ho, rka:
\par 39 Navrat se do domu svého, a vypravuj, kterak veliké veci ucinil tobe Buh. I odšel, po všem meste vypravuje, jak veliké veci ucinil jemu Ježíš.
\par 40 Stalo se pak, když se navrátil Ježíš, že prijal jej zástup; nebo všickni ocekávali ho.
\par 41 A aj, prišel muž, kterémuž jméno bylo Jairus, a ten byl kníže školy Židovské. I padna k nohám Ježíšovým, prosil ho, aby všel do domu jeho.
\par 42 Nebo mel dceru tu jedinou, kteréž bylo okolo dvanácti let, a ta umírala. A když šel, tiskl jej zástup.
\par 43 Tedy žena jedna, jenž nemoc svou trpela od let dvanácti, (kterážto byla na lékare vynaložila všecken statek, a od žádného nemohla uzdravena býti,)
\par 44 Pristoupivši pozadu, dotkla se podolka roucha jeho, a hned prestala nemoc její.
\par 45 I rekl Ježíš: Kdo jest, jenž se mne dotekl? A když všickni zapírali, rekl Petr, a kteríž s ním byli: Mistre, zástupové tebe tisknou a tlací, a ty pravíš: Kdo se mne dotekl?
\par 46 I rekl Ježíš: Dotekl se mne nekdo, nebo poznal jsem já, že jest moc ode mne vyšla.
\par 47 A viduci žena, že by tajno nebylo, tresuci se, pristoupila a padla pred ním, a pro kterou prícinu dotkla se ho, povedela prede vším lidem, a kterak jest hned uzdravena.
\par 48 A on rekl jí: Dobré mysli bud, dcero, víra tvá tebe uzdravila. Jdiž u pokoji.
\par 49 A když on ješte mluvil, prišel jeden od knížete školy, rka jemu: Již umrela dcera tvá, nezamestnávej Mistra.
\par 50 Ale Ježíš uslyšav to, odpovedel jemu: Nebojž se, ver toliko, a zdrávat bude.
\par 51 A všed do domu, nedopustil s sebou vjíti žádnému než Petrovi a Jakubovi a Janovi, a otci a materi té devecky.
\par 52 Plakali jí pak všickni a kvílili. A on rekl: Neplactež. Neumrelat, ale spít.
\par 53 I posmívali se jemu, vedouce, že jest umrela.
\par 54 On pak vyhnav ven všecky, a ujav ruku její, zavolal, rka: Devecko, vstan!
\par 55 I navrátil se duch její, a vstala hned. I kázal jí dáti jísti.
\par 56 I divili se náramne rodicové její. A on jim kázal, aby žádnému nepravili o tom, co se bylo stalo.

\chapter{9}

\par 1 I svolav Ježíš dvanácte ucedlníku svých, dal jim sílu a moc nad všelikým dábelstvím, a aby neduhy uzdravovali.
\par 2 I poslal je, aby kázali království Boží, a uzdravovali nemocné.
\par 3 A rekl jim: Nic neberte na cestu, ani hulky, ani mošny, ani chleba, ani penez, ani po dvou sukních mívejte.
\par 4 A do kteréhožkoli domu vešli byste, tu zustante, a odtud vyjdete.
\par 5 A kteríž by vás koli neprijali, vyjdouce z mesta toho, také i ten prach z noh vašich vyrazte na svedectví proti nim.
\par 6 I vyšedše, chodili po mesteckách vukol, zvestujíce evangelium, a uzdravujíce všudy.
\par 7 Uslyšel pak Herodes ctvrták o všech vecech, kteréž se dály od neho. I rozjímal to v mysli své, protože bylo praveno od nekterých, že by Jan vstal z mrtvých,
\par 8 A od jiných, že by se Eliáš zjevil, od nekterých pak, že by jeden z proroku starých vstal.
\par 9 I rekl Herodes: Jana jsem já stal. Kdož pak jest tento, o kterémž já slyším takové veci? I žádostiv byl ho videti.
\par 10 Vrátivše se pak apoštolé, vypravovali jemu, cožkoli cinili. A pojav je, odšel soukromí na místo pusté mesta receného Betsaida.
\par 11 To když zvedeli zástupové, šli za ním; i prijal je, a mluvil jim o království Božím, a ty, kteríž uzdravení potrebovali, uzdravoval.
\par 12 Den pak pocal se nachylovati. I pristoupivše dvanácte ucedlníku, rekli jemu: Rozpust zástupy, at rozejdouce se do mestecek okolních a do vesnic, jdou a hledají pokrmu, nebo jsme tuto na míste pustém.
\par 13 I rekl jim: Dejte vy jim jísti. A oni rekli: Nemámet víc než pet chlebu a dve rybe, lec bychom my snad šli a nakoupili na tento všecken lid pokrmu?
\par 14 Nebo bylo mužu okolo peti tisícu. I rekl ucedlníkum svým: Rozkažte se jim posaditi v každém radu po padesáti.
\par 15 I ucinili tak, a posadili se všickni.
\par 16 A vzav tech pet chlebu a dve rybe, vzhlédl v nebe a dobrorecil jim, i lámal, a rozdával ucedlníkum, aby kladli pred zástup.
\par 17 I jedli, a nasyceni jsou všickni. A sebráno jest, což jim bylo ostalo drobtu, dvanácte košu.
\par 18 I stalo se, když se on modlil obzvláštne, že byli s ním ucedlníci. I otázal se jich, rka: Kým mne praví býti zástupové?
\par 19 A oni odpovedevše, rekli: Janem Krtitelem, a jiní Eliášem, jiní pak, že prorok jeden z starých vstal.
\par 20 I rekl jim: Vy pak kým mne býti pravíte? Odpovedev Petr, rekl: Krista toho Božího.
\par 21 A on pohroziv jim, rozkázal, aby toho žádnému nepravili,
\par 22 Prave: Že Syn cloveka musí mnoho trpeti, a potupen býti od starších a od predních kneží i od zákoníku, a zamordován býti, a tretího dne z mrtvých vstáti.
\par 23 I pravil ke všechnem: Chce-li kdo prijíti za mnou, zapri sám sebe, a ber svuj kríž na každý den, a následuj mne.
\par 24 Nebo kdož bude chtíti duši svou zachovati, ztratít ji; a kdož ztratí duši svou pro mne, zachovát ji.
\par 25 Nebo co jest platno cloveku, by všecken svet získal, kdyby sám sebe zatratil, nebo sám sebe zmrhal?
\par 26 Neb kdož by se za mne stydel a za mé reci, za tohot se Syn cloveka stydeti bude, když prijde v sláve své a Otce svého i svatých andelu.
\par 27 Ale pravímt vám jiste: Jsout nekterí z tech, jenž tuto stojí, kteríž neokusí smrti, až i uzrí království Boží.
\par 28 I stalo se po tech recech, jako po osmi dnech, že Ježíš vzav s sebou Petra a Jakuba a Jana, vstoupil na horu, aby se modlil.
\par 29 A když se modlil, ucinena jest tvár jeho promenená, a odev jeho bílý a stkvoucí.
\par 30 A aj, dva muži mluvili s ním, a ti byli Mojžíš a Eliáš.
\par 31 Kteríž okázavše se v sláve, vypravovali o smrti jeho, kterouž mel podstoupiti v Jeruzaléme.
\par 32 Petr pak a ti, kteríž s ním byli, obtíženi byli snem, a procítivše, videli slávu jeho a dva muže, ani stojí s ním.
\par 33 I stalo se, když oni odešli od neho, rekl Petr k Ježíšovi: Mistre, dobrét jest nám zde býti. Protož udelejme tuto tri stánky, tobe jeden, a Mojžíšovi jeden, a Eliášovi jeden, neveda, co mluví.
\par 34 A když on to mluvil, stal se oblak, i zastínil je. Báli se pak ucedlníci, když oni vcházeli do oblaku.
\par 35 I stal se hlas z oblaku rkoucí: Tentot jest Syn muj milý, jeho poslouchejte.
\par 36 A když se ten hlas stal, nalezen jest Ježíš sám. A oni mlceli, a nepravili žádnému v tech dnech nicehož z tech vecí, kteréž jsou videli.
\par 37 Stalo se pak druhého dne, když sstupovali s hory, potkal jej zástup mnohý.
\par 38 A aj, muž z zástupu zvolal, rka: Mistre, prosím tebe, vzhlédni na syna mého, nebt jediného toho mám.
\par 39 A aj, duch jej napadá, a on ihned kricí, a sliní se, a dábel lomcuje jím slinícím se, a nesnadne odchází od neho, sápaje jím.
\par 40 I prosil jsem ucedlníku tvých, aby jej vyvrhli, ale nemohli.
\par 41 I odpovedev Ježíš, rekl: Ó pokolení neverné a prevrácené, dokudž budu u vás a dokud vás snášeti budu? Prived sem syna svého.
\par 42 A v tom, když on pricházel, porazil jej dábel a lomcoval jím. I primluvil duchu necistému Ježíš, a uzdravil mládence, a navrátil jej otci jeho.
\par 43 I divili se náramne všickni velikomocnosti Božské. A když se všickni divili všem vecem, kteréž cinil Ježíš, rekl ucedlníkum svým:
\par 44 Složte vy v uších vašich reci tyto, nebot Syn cloveka bude vydán v ruce lidské.
\par 45 Ale oni nesrozumeli slovu tomu, a bylo pred nimi skryto, aby nevyrozumeli jemu. A báli se ho otázati o tom slovu.
\par 46 I vznikla mezi nimi hádka o to, kdo by z nich byl vetší.
\par 47 Ježíš pak videv premyšlování srdce jejich, vzav díte, postavil je podle sebe,
\par 48 A rekl jim: Kdožkoli prijal by díte toto ve jménu mém, mnet prijímá; a kdož by koli mne prijal, prijímá toho, kterýž mne poslal. Nebo kdožt jest nejmenší mezi všemi vámi, tent bude veliký.
\par 49 I odpovedev Jan, rekl: Mistre, videli jsme jednoho, an ve jménu tvém dábly vymítá; i bránili jsme mu, protože nechodí s námi.
\par 50 I dí jemu Ježíš: Nebrantež. Nebo kdot není proti nám, s námit jest.
\par 51 I stalo se, když se doplnili dnové vzetí jeho vzhuru, a on se byl již na tom ustavil, aby šel do Jeruzaléma,
\par 52 Že poslal posly pred sebou. A oni jdouce, vešli do mestecka Samaritánského, aby jemu zjednali hospodu.
\par 53 I neprijali ho, protože oblicej jeho byl obrácen k jití do Jeruzaléma.
\par 54 A videvše to ucedlníci jeho, Jakub a Jan, rekli: Pane, chceš-li, at díme, aby ohen sstoupil s nebe a spálil je, jako i Eliáš ucinil?
\par 55 A obrátiv se Ježíš, potrestal jich, rka: Nevíte, cího jste vy duchu.
\par 56 Syn zajisté cloveka neprišel zatracovati duší lidských, ale aby je spasil. I odešli do jiného mestecka.
\par 57 Stalo se pak, když šli cestou, rekl jemu jeden: Pane, pujdu za tebou, kam se koli obrátíš.
\par 58 I rekl jemu Ježíš: Lišky doupata mají a ptáci nebeští hnízda, ale Syn cloveka nemá, kde by hlavu sklonil.
\par 59 I rekl k jinému: Pojd za mnou. A on rekl: Pane, dopust mi prve jíti a pochovati otce mého.
\par 60 I dí jemu Ježíš: Nech, at mrtví pochovávají mrtvé své, ale ty jda, zvestuj království Boží.
\par 61 I rekl opet jiný : Pujdu za tebou, Pane, ale prve dopust mi, at se rozžehnám s temi, kteríž jsou v domu mém.
\par 62 Rekl jemu Ježíš: Žádný, kdož vztáhna ruku svou k pluhu, ohlídal by se nazpet, není zpusobný k království Božímu.

\chapter{10}

\par 1 Potom pak vyvolil Pán i jiných sedmdesát, a poslal je po dvou pred tvárí svou do každého mesta i místa, kamž mel sám prijíti.
\par 2 A pravil jim: Žen zajisté jest mnohá, ale delníku málo. Protož proste Pána žni, at vypudí delníky na žen svou.
\par 3 Jdetež. Aj, já posílám vás jako berany mezi vlky.
\par 4 Nenostež s sebou pytlíka, ani mošny, ani obuvi, a žádného na ceste nepozdravujte.
\par 5 A do kteréhožkoli domu vejdete, nejprve rcete: Pokoj tomuto domu.
\par 6 A bude-lit tu který syn pokoje, odpocinet na nem pokoj váš; pakli nic, k vámt se navrátí.
\par 7 A v témž domu ostante, jedouce a pijíce, což u nich jest. Nebo hoden jest delník mzdy své. Nechodtež z domu do domu.
\par 8 Ale do kteréhožkoli mesta vešli byste a prijali by vás, jezte, což pred vás predloží.
\par 9 A uzdravujte nemocné, kteríž by v nem byli, a rcete jim: Priblížilot se k vám království Boží.
\par 10 A do kteréhožkoli mesta vešli byste, a neprijali by vás, vyjdouce na ulice jeho, rcetež:
\par 11 Také i ten prach, kterýž se prichytil nás z mesta vašeho, vyrážíme na vás. Ale však to vezte, žet se jest priblížilo k vám království Boží.
\par 12 Pravím zajisté vám, že Sodomským v onen den lehceji bude nežli tomu mestu.
\par 13 Beda tobe Korozaim, beda tobe Betsaido. Nebo kdyby v Týru a v Sidonu cineni byli divové ti, kteríž v vás cineni jsou, dávno by v žíni a v popele sedíce, pokání cinili.
\par 14 A protož Týru a Sidonu lehceji bude na soudu nežli vám.
\par 15 A ty Kafarnaum, které jsi až do nebe zvýšeno, až do pekla sníženo budeš.
\par 16 Kdož vás slyší, mne slyší; a kdo vámi pohrdá, mnou pohrdá; kdož pak mnou pohrdá, pohrdát tím, kdož mne poslal.
\par 17 Potom navrátilo se s radostí tech sedmdesáte, rkouce: Pane, také i dáblové se nám poddávají ve jménu tvém.
\par 18 I rekl jim: Videl jsem satana jako blesk padajícího s nebe.
\par 19 Aj, dávámt vám moc šlapati na hady a na štíry i na všelikou moc neprítele, a nic vám neuškodí.
\par 20 Avšak z toho se neradujte, žet se vám poddávají duchové, ale radeji se radujte, že jména vaše napsána jsou v nebesích.
\par 21 V tu hodinu rozveselil se v duchu Ježíš, a rekl: Chválím te, Otce, Pane nebe i zeme, že jsi tyto veci skryl pred moudrými a opatrnými, a zjevils je malickým. Ovšem, Otce, neb tak se líbilo pred tebou.
\par 22 Všecky veci dány jsou mi od Otce mého, a žádný neví, kdo by byl Syn, jediné Otec, a kdo by byl Otec, jediné Syn, a komuž by chtel Syn zjeviti.
\par 23 A obrátiv se k ucedlníkum obzvláštne, rekl: Blahoslavené oci, kteréž vidí, co vy vidíte.
\par 24 Nebo pravím vám, že mnozí proroci i králové chteli videti, což vy vidíte, a nevideli, a slyšeti, což vy slyšíte, a neslyšeli.
\par 25 A aj, jeden zákoník vstal, pokoušeje ho, a rka: Mistre, co cine, život vecný dedicne obdržím?
\par 26 A on rekl k nemu: V Zákone co jest psáno? Kterak cteš?
\par 27 A on odpovedev, rekl: Milovati budeš Pána Boha svého ze všeho srdce svého, a ze vší duše své, a ze vší síly své, i ze vší mysli své, a bližního svého jako sebe samého.
\par 28 I rekl mu Ježíš: Práve jsi odpovedel. To cin, a živ budeš.
\par 29 On pak chteje se sám ospravedlniti, dí Ježíšovi: A kdo jest muj bližní?
\par 30 I odpovedev Ježíš, rekl: Clovek jeden šel z Jeruzaléma do Jericho, i upadl mezi lotry. Kteríž obloupivše jej a zranivše, odešli, odpolu živého nechavše.
\par 31 I prihodilo se, že knez jeden šel touž cestou, a uzrev jej, pominul.
\par 32 Též i Levíta až k tomu místu prišed, a uzrev jej, pominul.
\par 33 Samaritán pak jeden, cestou se bera, prišel až k nemu, a uzrev jej, milosrdenstvím hnut jest.
\par 34 A pristoupe, uvázal rány jeho, naliv oleje a vína, a vloživ jej na hovado své, vedl do hospody, a péci o nej mel.
\par 35 Druhého pak dne odjíti maje, vynav dva peníze, dal hospodári, a rekl: Mej o nej péci, a cožkoli nad to vynaložíš, já když se vrátím, zaplatím tobe.
\par 36 Kdo tedy z tech trí zdá se tobe bližním býti tomu, kterýž upadl mezi lotry?
\par 37 A on rekl: Ten, kterýž ucinil milosrdenství nad ním. I rekl jemu Ježíš: Jdi, i ty ucin též.
\par 38 I stalo se, když šli, že on všel do jednoho mestecka. Žena pak jedna, jménem Marta, prijala jej do domu svého.
\par 39 A ta mela sestru, jménem Mariji, kterážto sedeci u noh Ježíšových, poslouchala slova jeho.
\par 40 Ale Marta pecliva byla pri mnohé službe Pánu. Kterážto pristoupivši, rekla: Pane, nemáš-liž o to péce, že sestra má nechala mne samé sloužiti? Protož rci jí, at mi pomuž.
\par 41 A odpovedev, rekl jí Ježíš: Marta, Marta, peclivá jsi, a rmoutíš se pri mnohých vecech.
\par 42 Ale jednohot jest potrebí. Mariat dobrou stránku vyvolila, kterážto nebude odjata od ní.

\chapter{11}

\par 1 I stalo se, když byl na jednom míste, modle se, že když prestal, rekl k nemu jeden z ucedlníku jeho: Pane, nauc nás modliti se, jako i Jan ucil ucedlníky své.
\par 2 I rekl jim: Když se modlíte; ríkejte: Otce náš, jenž jsi v nebesích, posvet se jméno tvé. Prijd království tvé. Bud vule tvá, jako v nebi tak i na zemi.
\par 3 Chléb náš vezdejší dávej nám každého dne.
\par 4 I odpust nám hríchy naše, nebo i my odpouštíme všelikému vinníku našemu. A neuvod nás v pokušení, ale zbav nás od zlého.
\par 5 I rekl k nim: Kdo z vás bude míti prítele, a pujde k nemu o pulnoci, a dí jemu: Príteli, pujc mi trí chlebu.
\par 6 Nebo prítel muj prišel s cesty ke mne, a nemám, co bych predložil pred nej.
\par 7 A on vnitr jsa, odpovedel by, rka: Necin mi nevole, neb jsou již dvere zavríny, a dítky mé se mnou jsou v pokoji. Nemohut vstáti a dáti tobe.
\par 8 Pravím vám: Act nedá jemu, vstana, protože jest prítel jeho, ale však pro nezbednost jeho vstana, dá jemu, kolikožkoli potrebuje.
\par 9 I ját pravím vám: Proste, a budet vám dáno; hledejte, a naleznete; tlucte, a budet vám otevríno.
\par 10 Neb každý, kdož prosí, bére; a kdož hledá, nalézá; a tomu, kdož tluce, bude otevríno.
\par 11 Kterého pak z vás otce prosil by syn za chléb, zdali kamene podá jemu? Aneb za rybu, zdali místo ryby dá jemu hada?
\par 12 Aneb prosil-li by za vejce, zdali podá jemu štíra?
\par 13 Ponevadž tedy vy, zlí jsouce, umíte dobré dary dávati detem svým, cím více Otec váš nebeský dá Ducha svatého tem, kteríž ho prosí?
\par 14 I vymítal Ježíš dábelství, a to bylo nemé. Stalo se pak, když vyšlo dábelství, že mluvil nemý. I divili se zástupové.
\par 15 Ale nekterí z nich pravili: V Belzebubu, knížeti dábelském, vymítá dábly.
\par 16 A jiní pokoušejíce ho, znamení s nebe hledali od neho.
\par 17 Ale on znaje myšlení jejich, rekl jim: Každé království samo v sobe rozdelené pustne, a dum na dum padá.
\par 18 Jestližet jest pak i satan proti sobe rozdelen, kterakž stane království jeho? Nebo pravíte, že já v Belzebubu vymítám dábly.
\par 19 Jestliže já v Belzebubu vymítám dábly, synové vaši v kom vymítají? Protož oni soudcové vaši budou.
\par 20 Paklit prstem Božím vymítám dábly, jistet jest prišlo k vám království Boží.
\par 21 Když silný odenec ostríhá síne své, v pokoji jsou všecky veci, kteréž má.
\par 22 Pakli by silnejší než on prijda, premohl jej, všecka odení jeho odejme, v než úfal, a loupeže jeho rozdelí.
\par 23 Kdožt není se mnou, proti mne jest; a kdož neshromažduje se mnou, rozptylujet.
\par 24 Když necistý duch vyjde od cloveka, chodí po místech suchých, hledaje odpocinutí. A nenalezna, dí: Vrátím se do domu svého, odkudž jsem vyšel.
\par 25 A prijda, nalezne jej vymetený a ozdobený.
\par 26 I jde, a prijme k sobe jiných sedm duchu horších sebe, a vejdouce, prebývají tam. I jsou poslední veci cloveka toho horší nežli první.
\par 27 I stalo se, když on to mluvil, pozdvihši hlasu jedna žena z zástupu, rekla jemu: Blahoslavený život, kterýž tebe nosil, a prsy, kterýchž jsi požíval.
\par 28 A on rekl: Ovšem pak blahoslavení, kteríž slyší slovo Boží a ostríhají jeho.
\par 29 A když se zástupové scházeli, pocal praviti: Pokolení toto nešlechetné jest. Znamení vyhledává, a znamení jemu nebude dáno, než znamení Jonáše proroka.
\par 30 Nebo jakož Jonáš ucinen byl znamením Ninivitským, takt bude i Syn cloveka pokolení tomuto.
\par 31 Královna od poledne stane na soudu s muži pokolení tohoto, a odsoudí je. Nebo prijela od koncin zeme, aby slyšela moudrost Šalomounovu, a aj, více než Šalomoun tuto!
\par 32 Muži Ninivitští povstanou na soudu s pokolením tímto, a odsoudí je. Nebo cinili pokání k kázání Jonášovu, a aj, více nežli Jonáš tuto!
\par 33 Žádný rozsvíte svíci, nepostaví jí do skrýše, ani pod kbelec, ale na svícen, aby ti, kteríž vcházejí, svetlo videli.
\par 34 Svíce tela tvého jest oko tvé. Když by tedy oko tvé sprostné bylo, i telo tvé všecko bude svetlé; a paklit bude nešlechetné, takét i telo tvé tmavé bude.
\par 35 Viziž tedy, aby svetlo, kteréž jest v tobe, nebylo tmou.
\par 36 Pakli celé telo tvé svetlé bude, nemaje žádné cástky tmavé, budet všecko tak svetlé, že te jako svíce bleskem osvítí.
\par 37 A mezi tím když on mluvil, prosil ho jeden farizeus, aby obedval u neho. A všed, posadil se za stul.
\par 38 Farizeus pak videv to, podivil se, že se neumyl pred obedem.
\par 39 I rekl Pán k nemu: Nyní vy farizeové povrchu konvice a mísy cistíte, ale to, což vnitr jest v vás, plno jest loupeže a nešlechetností.
\par 40 Blázni, zdaliž ten, kterýž ucinil, což zevnitr jest, neucinil také i toho, což jest vnitr?
\par 41 Ale však i z toho, což máte, dávejte almužnu, a aj, všecky veci vaše cisté budou.
\par 42 Ale beda vám farizeum, kteríž desátky dáváte z máty a z routy a ze všeliké byliny, ale opouštíte soud a lásku Boží; ješto tyto veci meli jste ciniti, a onech neopoušteti.
\par 43 Beda vám farizeum, nebo milujete první místa v školách a pozdravování na trzích.
\par 44 Beda vám, zákoníci a farizeové pokrytci, nebo jste jako hrobové nepatrní, po nichž lidé chodíce, nevedí o tom, co tam jest.
\par 45 I odpovedev jeden z zákoníku, rekl jemu: Mistre, tyto veci mluve, i nám také lehkost ciníš.
\par 46 A on rekl: I vám zákoníkum beda, nebo obtežujete lidi bremeny nesnesitelnými, a sami se tech bremen jedním prstem nedotýkáte.
\par 47 Beda vám, jenž vzdeláváte hroby prorocké, kteréž otcové vaši zmordovali.
\par 48 A tak osvedcujete a potvrzujete skutku otcu vašich. Nebo oni zajisté zmordovali jsou je, vy pak vzdeláváte hroby jejich.
\par 49 Protož i Moudrost Boží rekla: Pošlit k nim proroky a apoštoly, a z tech nekteré mordovati budou, a jiné vyháneti,
\par 50 Aby požádáno bylo od tohoto pokolení krve všech proroku, kteráž vylita jest od ustanovení sveta,
\par 51 Od krve Abelovy až do krve Zachariášovy, kterýž zahynul mezi oltárem a chrámem. Jiste, pravím vám, požádáno bude od pokolení tohoto.
\par 52 Beda vám zákoníkum, nebo jste vzali klíc umení; sami jste nevešli, a tem, kteríž vcházeli, zbránili jste.
\par 53 A když jim to mluvil, pocali zákoníci a farizeové prísne jemu odpírati, a k mnohým recem príciny jemu dávati,
\par 54 Ukládajíce o nem, a hledajíce popadnouti neco z úst jeho, aby jej obžalovali.

\chapter{12}

\par 1 A vtom, když mnozí zástupové scházeli se, takže jedni druhé velmi tlacili, pocal mluviti k ucedlníkum svým: Nejpredneji se varujte od kvasu farizeu, jenž jest pokrytství.
\par 2 Nebot nic není skrytého, což by nemelo býti zjeveno; ani jest co tajného, ješto by nemelo býti zvedíno.
\par 3 Protož to, co jste pravili ve tmách, bude na svetle slyšáno, a co jste sobe v uši šeptali v pokojích, hlásánot bude na strechách.
\par 4 Pravím pak vám prátelum svým: Nestrachujte se tech, jenž telo zabíjejí, a potom nemají, co by více ucinili.
\par 5 Ale ukážit vám, koho se máte báti: Bojte se toho, kterýžto, když zabije, má moc uvrci do pekelného ohne. Jiste, pravím vám, toho se bojte.
\par 6 Zdaliž neprodávají pet vrabcu za dva halére? Avšak ani jeden z nich není v zapomenutí pred Bohem.
\par 7 Nýbrž i vlasové hlavy vaší všickni zecteni jsou. Protož nebojtež se, mnohemt vy vrabce prevyšujete.
\par 8 Pravímt pak vám: Každý kdož by koli vyznal mne pred lidmi, i Syn cloveka vyzná jej pred andely Božími.
\par 9 Kdož by mne pak zaprel pred lidmi, zaprínt bude pred andely Božími.
\par 10 A každý kdož dí slovo proti Synu cloveka, bude mu odpušteno, ale tomu, kdož by se Duchu svatému rouhal, nebudet odpušteno.
\par 11 Když pak vás voditi budou do škol a k vladarum a k mocným, nepecujte, kterak aneb co byste odpovídali, aneb co byste mluvili.
\par 12 Duch svatý zajisté naucí vás v tu hodinu, co byste meli mluviti.
\par 13 I rekl jemu jeden z zástupu: Mistre, rci bratru mému, at rozdelí se mnou dedictví.
\par 14 A on rekl jemu: Clovece, kdo mne ustavil soudcí aneb delicem nad vámi?
\par 15 I rekl k nim: Viztež a vystríhejte se od lakomství; nebot ne v rozhojnení statku necího život jeho záleží.
\par 16 Povedel jim také i podobenství, rka: Cloveka jednoho bohatého hojné úrody pole prineslo.
\par 17 I premyšloval sám v sobe, rka: Co uciním, že nemám, kde bych shromáždil úrody své?
\par 18 I rekl: Toto uciním: Zborím stodoly své a vetších nadelám, a tu shromáždím všecky své úrody i zboží svá.
\par 19 A dím duši své: Duše, máš mnoho statku složeného za mnohá léta, odpocívej, jez, pij, mej dobrou vuli.
\par 20 I rekl jemu Buh: Ó blázne, této noci požádají duše tvé od tebe, a to, cožs pripravil, cí bude?
\par 21 Takt jest každý, kdož sobe shromažduje, a není v Bohu bohatý.
\par 22 Rekl pak ucedlníkum svým: Protož pravím vám: Nebudtež pecliví o život svuj, co byste jedli, ani o telo, cím byste se odívali.
\par 23 Život vetší jest nežli pokrm, a telo vetší nežli odev.
\par 24 Patrte na havrany, žet nesejí, ani žnou, a nemají špižírny, ani stodoly, a Buh krmí je. I cím v vetší vážnosti jste vy než ptactvo?
\par 25 A kdož pak z vás peclive o to mysle, muž pridati ku postave své loket jeden?
\par 26 Ponevadž tedy nemužete s to býti, což nejmenšího jest, proc o jiné veci se staráte?
\par 27 Patrte na kvítí polní, kterak rostou, nedelají, ani predou, a pravímt vám, že ani Šalomoun ve vší sláve své nebyl tak odín, jako jedno z techto.
\par 28 A ponevadž trávu, kteráž dnes na poli jest, a zítra do peci uvržena bývá, Buh tak odívá, cím více vás, ó malé víry?
\par 29 I vy nestarejte se o to, co byste jedli, aneb co byste pili, aniž o to tak velmi pecujte.
\par 30 Nebo tech všech vecí národové sveta tohoto hledají. Vít pak Otec váš, že tech vecí potrebujete.
\par 31 Ale radeji hledejte království Božího, a tyto všecky veci budou vám pridány.
\par 32 Neboj se, ó malické stádce, nebot se zalíbilo Otci vašemu dáti vám království.
\par 33 Prodávejte statky vaše, a dávejte almužnu. Delejte sobe pytlíky, kteríž nevetšejí, poklad, kterýž nehyne, v nebesích, kdežto zlodej dojíti nemuž, a kdež mol nekazí.
\par 34 Nebo kdež jest poklad váš, tut bude i srdce vaše.
\par 35 Budtež bedra vaše prepásaná, a svíce horící.
\par 36 A vy podobni budte lidem ocekávajícím Pána svého, až by se vrátil z svadby, aby hned, jakž by prišel a potloukl, otevreli jemu.
\par 37 Blaze služebníkum tem, kteréž prijda Pán, nalezl by, a oni bdí. Amen pravím vám, že prepáše se, a káže jim sednouti za stul, a chode, bude jim sloužiti.
\par 38 A prišel-lit by v druhé bdení, a paklit by v tretí bdení prišel, a tak je nalezl, blahoslavení jsou služebníci ti.
\par 39 Toto pak vezte, že byt vedel hospodár, v kterou by hodinu mel zlodej prijíti, bdel by zajisté, a nedal by podkopati domu svého.
\par 40 Protož i vy budte hotovi, nebo v kterou hodinu nenadejete se, Syn cloveka prijde.
\par 41 I rekl jemu Petr: Pane, nám-li pravíš toto podobenství, cili všechnem?
\par 42 I dí Pán: Aj kdo jest verný šafár a opatrný, jehož by ustanovil pán nad celedí svou, aby jim v cas dával vymerený pokrm,
\par 43 Blahoslavený služebník ten, kteréhož, když by prišel pán jeho, nalezne, an tak ciní.
\par 44 Vpravde pravím vám, že nade vším statkem svým ustanoví jej.
\par 45 Pakli by rekl služebník ten v srdci svém: Prodlévá prijíti pán muj, i pocal by bíti služebníky a služebnice, a jísti a píti i opíjeti se,
\par 46 Prijdet pán služebníka toho v den, v kterýž se nenadeje, a v hodinu, kteréž neví. I oddelít jej, a díl jeho položí s nevernými.
\par 47 Služebník pak ten, kterýž by znal vuli pána svého a nepripravoval se, a necinil podle vule jeho, bit bude velmi.
\par 48 Ale kterýž neznal, a hodné veci trestání cinil, bit bude ne tak velmi. Každému pak, komuž jest mnoho dáno, mnoho bude od neho požádáno; a komut jsou mnoho porucili, vícet požádají od neho.
\par 49 Ohen prišel jsem pustiti na zemi, a co chci, jestliže již horí?
\par 50 Ale krtem mám krten býti, a kterak jsem soužen, dokudž se nevykoná!
\par 51 A což se domníváte, že bych prišel pokoj dáti na zemi? Nikoli, pravím vám, ale rozdelení.
\par 52 Nebo již od této chvíle bude jich pet v jednom domu rozdeleno, tri proti dvema, a dva proti trem.
\par 53 Bude rozdelen otec proti synu, a syn proti otci, máte proti dceri, a dcera proti materi, svegruše proti neveste své, a nevesta proti svegruši své.
\par 54 Pravil také i k zástupum: Když vídáte oblak, an vzchodí od západu, hned pravíte: Príval jde, a tak bývá.
\par 55 A když od poledne vítr veje, ríkáte: Bude horko, a bývát.
\par 56 Pokrytci, zpusob nebe a zeme umíte souditi, a kterakž pak tohoto casu nepoznáváte?
\par 57 Ano proc i sami od sebe nesoudíte, což spravedlivého jest?
\par 58 Když pak jdeš s protivníkem svým k vrchnosti, na ceste pricin se o to, abys byl zprošten od neho, aby snad netáhl tebe k soudci, a soudce dal by tebe birici, a biric vsadil by te do žaláre.
\par 59 Pravím tobe: Nevyjdeš odtud, dokudž bys i toho posledního halére nenavrátil.

\chapter{13}

\par 1 Byli pak tu prítomní casu toho nekterí, vypravujíce jemu o Galilejských, kterýchžto krev Pilát smísil s obetmi jejich.
\par 2 I odpovedev Ježíš, rekl jim: Co mníte, že jsou ti Galilejští byli vetší hríšníci nežli všickni jiní Galilejští, že takové veci trpeli?
\par 3 Nikoli, pravímt vám. Ale nebudete-li pokání ciniti, všickni též zahynete.
\par 4 Anebo onech osmnácte, na kteréžto upadla veže v Siloe, a zbila je, zdali se domníváte, že by oni vinni byli nad všecky lidi prebývající v Jeruzaléme?
\par 5 Nikoli, pravím vám. Ale nebudete-li pokání ciniti, všickni též zahynete.
\par 6 Povedel pak toto podobenství: Clovek jeden mel strom fíkový štípený na vinici své. I prišel, hledaje ovoce na nem, ale nenalezl.
\par 7 I rekl k vinari: Aj, po tri léta již pricházím, hledaje ovoce na tom fíku, a nenalézám. Vytniž jej! Proc i tu zemi darmo kazí?
\par 8 On pak odpovedev, rekl jemu: Pane, ponechejž ho i tohoto léta, ažt jej okopám a ohnojím,
\par 9 Zdali by nesl ovoce. Paklit neponese, potom vytneš jej.
\par 10 Ucil pak v jedné škole jejich v den svátecní.
\par 11 A aj, byla tu žena, kteráž mela ducha nemoci osmnácte let, a byla sklícena, a nijakž se nemohla zprostiti.
\par 12 A uzrev ji Ježíš, zavolal jí k sobe, a rekl jí: Ženo, zproštena jsi od nemoci své.
\par 13 I vložil na ni ruce, a ihned zdvihla se, a velebila Boha.
\par 14 Tedy kníže školní odpovedev, hnevaje se proto, že v den svátecní uzdravoval Ježíš, rekl k zástupu: Šest dní jest, v nichž náleží delati; protož v tech dnech pricházejíce, budte uzdravováni, a ne v den sobotní.
\par 15 I odpovedev jemu Pán, rekl: Pokrytce, zdali jeden každý z vás v den svátecní neodvazuje vola svého nebo osla od jeslí, a nevodí napájeti?
\par 16 Tato pak dcera Abrahamova, kterouž byl svázal satan již osmnácte let, což nemela býti rozvázána od svazku v den svátecní?
\par 17 A když on to povedel, zastydeli se všickni protivníci jeho, ale všecken lid radoval se ze všech tech slavných skutku, kteríž se dáli od neho.
\par 18 I rekl Ježíš: Cemu podobno jest království Boží a k cemu je prirovnám?
\par 19 Podobno jest zrnu horcicnému, kteréžto vzav clovek, uvrhl do zahrady své. I rostlo, a ucineno jest v strom veliký, a ptactvo nebeské hnízda sobe delali na ratolestech jeho.
\par 20 A opet rekl: K cemu pripodobním království Boží?
\par 21 Podobno jest kvasu, kterýžto vzavši žena, zadelala ve trech mericích mouky, až zkysalo všecko.
\par 22 I chodil po mestech a mesteckách uce, bera se do Jeruzaléma.
\par 23 I rekl jemu jeden: Pane, tuším, že jest málo tech, kteríž spaseni býti mají? On pak rekl k nim:
\par 24 Snažujte se vcházeti tesnou branou; nebot (pravím vám) mnozí hledati budou vjíti, a nebudou moci,
\par 25 Totiž když vejde hospodár, a zavre dvere, a pocnete vne státi a tlouci na dvere, rkouce: Pane, Pane, otevri nám, a on odpovídaje, dít vám: Neznám vás, odkud jste:
\par 26 Tedy pocnete ríci: Jídali jsme a píjeli pred tebou, a na ulicech našich jsi ucil.
\par 27 I dí: Pravím vám, žet vás neznám, odkud jste. Odejdetež ode mne všickni cinitelé nepravosti.
\par 28 Tamt bude plác a škripení zubu, když uzríte Abrahama a Izáka a Jákoba a všecky proroky v království Božím, sami pak sebe vyhnané ven.
\par 29 I prijdout mnozí od východu, a od západu, a od pulnoci, i od poledne, a budou stoliti v království Božím.
\par 30 A aj, jsout poslední, kteríž budou první, a jsou první, kteríž budou poslední.
\par 31 A v ten den pristoupili nekterí z farizeu, rkouce jemu: Vyjdi, a odejdi odsud, nebo Herodes chce te zamordovati.
\par 32 I rekl jim: Jdouce, povezte lišce té: Aj, vymítám dábly, a uzdravuji dnes a zítra, a tretího dne dokonám.
\par 33 Ale však musím dnes a zítra i pozejtrí choditi; nebot jest nelze proroku zahynouti jinde krome Jeruzaléma.
\par 34 Jeruzaléme, Jeruzaléme, ješto morduješ proroky a kamenuješ ty, kteríž k tobe bývají posláni, kolikrát jsem chtel shromážditi syny tvé, jako slepice kurátka svá pod krídla? A nechteli jste.
\par 35 Aj, opušten bude dum váš a zanechán vám pustý. Ale jiste pravím vám, že nikoli mne neuzríte, ažt prijde cas, když díte: Požehnaný, jenž se bére ve jménu Páne.

\chapter{14}

\par 1 I stalo se, když všel Ježíš do domu jednoho knížete farizejského v sobotu, aby jedl chléb, že oni šetrili ho.
\par 2 A aj, clovek jeden vodnotelný byl pred ním.
\par 3 I odpovedev Ježíš, dí zákoníkum a farizeum, rka: Sluší-li v sobotu uzdravovati?
\par 4 A oni mlceli. Tedy on dosáh jeho, uzdravil a propustil.
\par 5 A odpovedev k nim rekl: Cí z vás osel anebo vul upadl by do studnice, a ne ihned by ho vytáhl v den sobotní?
\par 6 I nemohli jemu na to odpovedíti.
\par 7 Povedel také i ku pozvaným podobenství, (spatriv to, kterak sobe prední místa vyvolovali,) rka k nim:
\par 8 Kdybys byl od nekoho pozván na svadbu, nesedej na predním míste, at by snad vzácnejší nežli ty nebyl pozván od neho.
\par 9 A prijda ten, kterýž tebe i onoho pozval, rekl by tobe: Dej tomuto místo. A tehdy pocal bys s hanbou na posledním míste sedeti.
\par 10 Ale když bys byl pozván, jda, posad se na posledním míste. A kdyby prišel ten, kterýž tebe pozval, aby rekl tobe: Príteli, posedni výše, tedy budeš míti chválu pred spolustolícími.
\par 11 Nebo každý, kdož se povyšuje, bude ponížen; a kdož se ponižuje, bude povýšen.
\par 12 Pravil také i tomu, kterýž ho byl pozval: Když ciníš obed nebo veceri, nezov prátel svých, ani bratrí svých, ani sousedu bohatých, at by snad i oni zase nezvali tebe, a mel bys odplatu.
\par 13 Ale když ciníš hody, povolej chudých, chromých, kulhavých, slepých,
\par 14 A blahoslavený budeš. Nebot nemají, odkud by odplatili tobe, ale budet odplaceno pri vzkríšení spravedlivých.
\par 15 I uslyšav to jeden z prísedících, rekl jemu: Blahoslavený jest, kdož jí chléb v království Božím.
\par 16 On pak rekl jemu: Clovek jeden ucinil veceri velikou, a pozval mnohých.
\par 17 I poslal služebníka svého v hodinu vecere, aby rekl pozvaným: Pojdte, nebo již pripraveno jest všecko.
\par 18 I pocali se všickni spolu vymlouvati. První rekl jemu: Ves jsem koupil, a musím vyjíti a ohledati jí; prosím tebe, vymluv mne.
\par 19 A druhý rekl: Patero sprežení volu koupil jsem, a jdu, abych jich zkusil; prosím tebe, vymluv mne.
\par 20 A jiný dí: Ženu jsem pojal, a protož nemohu prijíti.
\par 21 I navrátiv se služebník, zvestoval tyto veci pánu svému. Tedy rozhnevav se hospodár, rekl služebníku svému: Vyjdi rychle na rynky a na ulice mesta, a chudé, i chromé, i kulhavé, a slepé uved sem.
\par 22 I rekl služebník: Pane, stalo se, jakož jsi rozkázal, a ještet místo jest.
\par 23 Tedy rekl pán služebníku: Vyjdiž na cesty a mezi ploty, a prinut vjíti, at se naplní dum muj.
\par 24 Nebo pravímt vám, že žádný z mužu tech, kteríž pozváni byli, neokusí vecere mé.
\par 25 Šli pak mnozí zástupové s ním. A on obrátiv se, rekl jim:
\par 26 Jde-li kdo ke mne, a nemá-li v nenávisti otce svého, i matere, i ženy, i detí, i bratrí, i sestr, ano i té duše své, nemuž býti mým ucedlníkem.
\par 27 A kdožkoli nenese kríže svého, a jde za mnou, nemuž býti mým ucedlníkem.
\par 28 Nebo kdo z vás jest, chteje staveti veži, aby prve sedna, nepocetl nákladu, bude-li míti dosti k dokonání toho díla?
\par 29 Aby snad, když by položil grunt, a nemohl dokonati, neprišlo na to, že všickni to vidouce, pocali by se jemu posmívati,
\par 30 Rkouce: Tento clovek pocal staveti, a nemohl dokonati.
\par 31 Anebo který král bera se k boji proti jinému králi, zdaliž prve nesedne, aby se poradil, mohl-li by s desíti tisíci potkati se s tím, kterýž s dvadcíti tisíci táhne proti nemu?
\par 32 Sic jinak, když ješte podál od neho jest, pošle posly k nemu, žádaje toho, což by bylo ku pokoji.
\par 33 Tak zajisté každý z vás, kdož se neodrekne všech vecí, kterýmiž vládne, nemuž býti mým ucedlníkem.
\par 34 Dobrát jest sul. Pakli sul bude zmarena, cím bude napravena?
\par 35 Ani do zeme, ani do hnoje se nehodí; ale vyvržena bude ven. Kdo má uši k slyšení, slyš.

\chapter{15}

\par 1 Približovali se pak k nemu všickni publikáni a hríšníci, aby ho slyšeli.
\par 2 I reptali farizeové a zákoníci, rkouce: Tento hríšníky prijímá a jí s nimi.
\par 3 I povedel jim podobenství toto, rka:
\par 4 Kdyby nekdo z vás mel sto ovec, a ztratil by jednu z nich, zdaliž by nenechal devadesáti devíti na poušti, a nešel k té, kteráž zahynula, až by i nalezl ji?
\par 5 A nalezna, jiste by ji vložil na ramena svá s radostí.
\par 6 A prijda domu, svolal by prátely a sousedy, rka jim: Spolu radujte se se mnou, neb jsem nalezl ovci svou, kteráž byla zahynula.
\par 7 Pravímt vám, že tak jest radost v nebi nad jedním hríšníkem pokání cinícím vetší, nežli nad devadesáti devíti spravedlivými, kteríž nepotrebují pokání.
\par 8 Aneb žena nekterá mající grošu deset, ztratila-li by jeden groš, zdaliž nezažže svíce, a nemete domu, a nehledá pilne, dokudž nenalezne?
\par 9 A když nalezne, svolá prítelkyne a sousedy, rkuci: Spolu radujte se se mnou, neb jsem nalezla groš, kterýž jsem byla ztratila.
\par 10 Takt pravím vám, že jest radost pred andely Božími nad jedním hríšníkem pokání cinícím.
\par 11 Rekl také Ježíš: Clovek jeden mel dva syny.
\par 12 Z nichž mladší rekl otci: Otce, dej mi díl statku, kterýž mne náleží. I rozdelil jim statek.
\par 13 A po nemnohých dnech, shromáždiv sobe všecko mladší syn, odšel do daleké krajiny, a tam rozmrhal statek svuj, živ jsa prostopášne.
\par 14 A když všecko utratil, stal se hlad veliký v krajine té, a on pocal nouzi trpeti.
\par 15 I všed, prídržel se jednoho meštenína krajiny té; a on jej poslal do vsi své, aby pásl vepre.
\par 16 I žádal nasytiti bricho své mlátem, kteréž svine jedly, a žádný nedával jemu.
\par 17 On pak prišed sám k sobe, rekl: Aj, jak mnozí celedínové u otce mého hojnost mají chleba, a já tuto hladem mru!
\par 18 Vstana, pujdu k otci svému, a dím jemu: Otce, zhrešil jsem proti nebi a pred tebou,
\par 19 A již více nejsem hoden slouti syn tvuj. Ale ucin mne jako jednoho z celedínu svých.
\par 20 I vstav, šel k otci svému. A když ješte opodál byl, uzrel jej otec jeho, a milosrdenstvím hnut jsa, pribeh, padl na šíji jeho, a políbil ho.
\par 21 I rekl jemu syn: Otce, zhrešil jsem proti nebi a pred tebou, a jižt nejsem hoden slouti syn tvuj.
\par 22 I rekl otec služebníkum svým: Prineste roucho to první, a oblecte jej, a dejte prsten na ruku jeho a obuv na nohy.
\par 23 A privedouce tele tucné, zabijte, a hodujíce, budme veseli.
\par 24 Nebo tento syn muj byl umrel, a zase ožil; byl zahynul, a nalezen jest. I pocali veseli býti.
\par 25 Byl pak syn jeho starší na poli. A jda, když se približoval k domu, uslyšel zpívání a hluk veselících se.
\par 26 I povolav jednoho z služebníku svých, otázal se ho, co by to bylo.
\par 27 A on rekl jemu: Bratr tvuj prišel, i zabil otec tvuj tucné tele, že ho zdravého prijal.
\par 28 I rozhneval se on, a nechtel tam vjíti. Otec pak jeho vyšed, prosil ho.
\par 29 A on odpovedev, rekl otci: Aj, tolik let sloužím tobe, a nikdy jsem prikázání tvého neprestoupil, avšak nikdy jsi mi nedal ani kozelce, abych také s práteli svými vesel pobyl.
\par 30 Ale když syn tvuj tento, kterýž prožral statek tvuj s nevestkami, prišel, zabils jemu tele tucné.
\par 31 A on rekl mu: Synu, ty vždycky se mnou jsi, a všecky veci mé jsou tvé.
\par 32 Ale hodovati a radovati se náleželo. Nebo bratr tvuj tento byl umrel, a zase ožil; zahynul byl, a nalezen jest.

\chapter{16}

\par 1 Pravil pak i k ucedlníkum svým: Clovek jeden byl bohatý, kterýž mel šafáre; a ten obžalován jest pred ním, jako by mrhal statek jeho.
\par 2 I povolav ho, rekl jemu: Což to slyším o tobe? Vydej pocet z vladarství svého, neb již nebudeš moci déle vládnouti.
\par 3 I dí vladar sám v sobe: Co uciním? Ted pán muj odjímá ode mne vladarství. Kopati nemohu, žebrati se stydím.
\par 4 Vím, co uciním, aby, když budu zbaven vladarství, prijali mne do svých domu.
\par 5 I zavolav jednoho každého dlužníka pána svého, rekl prvnímu: Jaks mnoho dlužen pánu mému?
\par 6 A on rekl: Sto tun oleje. I rekl mu: Vezmi rejistra svá, a sedna rychle, napiš padesát.
\par 7 Potom druhému rekl: Ty pak jaks mnoho dlužen? Kterýž rekl: Sto korcu pšenice. I dí mu: Vezmi rejistra svá, a napiš osmdesát.
\par 8 I pochválil pán vladare nepravého, že sobe opatrne ucinil. Nebo synové tohoto sveta opatrnejší jsou, nežli synové svetla v svých vecech.
\par 9 I ját pravím vám: Cinte sobe prátely z mamony nepravosti, aby, když byste zhynuli, prijali vás do vecných stanu.
\par 10 Kdožt jest verný v mále, i ve mnozet verný bude. A kdož v mále jest nepravý, i ve mnozet nepravý jest.
\par 11 Ponevadž tedy v mamone nepravé verní jste nebyli, spravedlivého zboží kdo vám sverí?
\par 12 A když jste v cizím verní nebyli, což vašeho jest, kdo vám dá?
\par 13 Nižádný služebník nemuž dvema pánum sloužiti. Nebt zajisté jednoho nenávideti bude, a druhého milovati, aneb jednoho prídržeti se bude, a druhým pohrdne. Nemužte Bohu sloužiti a mamone.
\par 14 Slyšeli pak toto všecko i farizeové, kteríž byli lakomí, a posmívali se jemu.
\par 15 I dí jim: Vy jste, ješto se sami spravedliví ciníte pred lidmi, ale Buht zná srdce vaše; nebo což jest u lidí vysokého, ohavnost jest pred Bohem.
\par 16 Zákon a Proroci až do Jana, a od té chvíle království Boží zvestuje se, a každý se do neho násilne tiskne.
\par 17 Snázet jest zajisté nebi a zemi pominouti, nežli v Zákone jednomu tytlíku zahynouti.
\par 18 Každý, kdož propustí manželku svou, a jinou pojímá, cizoloží; a kdož propuštenou od muže pojímá, cizoloží.
\par 19 Byl pak clovek jeden bohatý, a oblácel se v šarlat a v kment, a hodoval na každý den stkvostne.
\par 20 A byl také jeden žebrák, jménem Lazar, kterýžto ležel u vrat jeho vredovitý,
\par 21 Žádaje nasycen býti z drobtu, kteríž padali z stolu bohatce. Ale i psi pricházejíce, lízali vredy jeho.
\par 22 I stalo se, že ten žebrák umrel, a nesen jest od andelu do luna Abrahamova. Umrel pak i bohatec, a pohrben jest.
\par 23 Potom v pekle pozdvih ocí svých, v mukách jsa, uzrel Abrahama zdaleka, a Lazara v lunu jeho.
\par 24 I zvolav bohatec, rekl: Otce Abrahame, smiluj se nade mnou, a pošli Lazara, at omocí konec prstu svého v vode, a svlaží jazyk muj; nebo se mucím v tomto plameni.
\par 25 I rekl mu Abraham: Synu, rozpomen se, žes ty již vzal dobré veci své v živote svém, a Lazar též zlé. Nyní pak tento se již teší, ale ty se mucíš.
\par 26 A nadto nade všecko mezi námi a vámi propast veliká utvrzena jest, aby ti, kteríž chtí odsud k vám jíti, nemohli, ani odonud k nám prijíti.
\par 27 I rekl: Ale prosím tebe, Otce, abys ho poslal do domu otce mého.
\par 28 Nebot mám pet bratru. At jim svedcí, aby i oni neprišli do tohoto místa muk.
\par 29 I rekl jemu Abraham: Majít Mojžíše a Proroky, necht jich poslouchají.
\par 30 A on rekl: Nic, otce Abrahame, ale kdyby kdo z mrtvých šel k nim, budou pokání ciniti.
\par 31 I rekl mu: Ponevadž Mojžíše a Proroku neposlouchají, aniž byt kdo z mrtvých vstal, uverí jemu.

\chapter{17}

\par 1 Tedy rekl ucedlníkum: Není možné, aby neprišla pohoršení, ale beda tomu, skrze kohož pricházejí.
\par 2 Lépe by mu bylo, aby žernov oslicí vložen byl na hrdlo jeho, a uvržen byl do more, nežli by pohoršil jednoho z techto malických.
\par 3 Šetrte se. Zhrešil-li by pak proti tobe bratr tvuj, potresci ho, a bude-lit toho želeti, odpust mu.
\par 4 A byt pak sedmkrát za den zhrešil proti tobe, a sedmkrát za den obrátil se k tobe, rka: Žel mi toho, odpust mu.
\par 5 I rekli apoštolé Pánu: Prispor nám víry.
\par 6 I dí Pán: Kdybyste meli víru jako zrno horcicné, rekli byste této moruši: Vykoren se a presad se do more, a uposlechla by vás.
\par 7 Nebo kdo jest z vás, maje služebníka, ješto ore aneb pase dobytek, aby jemu, když by se s pole navrátil, hned rekl: Pojd a sed za stul?
\par 8 Ale zdali radeji nedí jemu: Priprav, at povecerím, a opáše se, služ mi, až se najím a napím, a potom i ty jez a pij?
\par 9 Zdali dekuje služebníku tomu, že ucinil to, což mu rozkázal? Nezdá mi se.
\par 10 Tak i vy, když uciníte všecko, což vám prikázáno, rcete: Služebníci neužitecní jsme. Což jsme povinni byli uciniti, ucinili jsme.
\par 11 I stalo se, když se bral do Jeruzaléma, že šel skrze Samarí a Galilei.
\par 12 A když vcházel do jednoho mestecka, potkalo se s ním deset mužu malomocných, kterížto stáli zdaleka.
\par 13 A pozdvihše hlasu, rekli: Ježíši prikazateli, smiluj se nad námi.
\par 14 Kteréžto on uzrev, rekl jim: Jdouce, ukažte se knežím. I stalo se, když šli, že ocišteni jsou.
\par 15 Jeden pak z nich uzrev, že jest uzdraven, navrátil se s velikým hlasem, velebe Boha.
\par 16 A padl na tvár k nohám jeho, díky cine jemu. A ten byl Samaritán.
\par 17 I odpovedev Ježíš, rekl: Zdaliž jich deset není ocišteno? A kdež jest jich devet?
\par 18 Nenalezli se k tomu, aby prijdouce, chválu Bohu vzdali, jediné cizozemec tento?
\par 19 I rekl jemu: Vstana, jdi, víra tvá te uzdravila.
\par 20 Otázán pak jsa od zákoníku, kdy prijde království Boží, odpovedel jim a rekl: Neprijdet království Boží patrne.
\par 21 Aniž reknou: Aj, tuto, aneb aj, tamto. Nebo aj, království Boží jestit mezi vámi.
\par 22 I rekl ucedlníkum: Prijdou dnové, že budete žádati videti jeden den Syna cloveka, a neuzríte.
\par 23 A dejít vám: Aj, zde, hle, tamto. Nechodte, ani následujte.
\par 24 Nebo jakožto blesk blýskající se z jedné krajiny, kteráž pod nebem jest, až do druhé, kteráž též pod nebem jest, svítí, tak bude i Syn cloveka ve dni svém.
\par 25 Ale nejprve musí mnoho trpeti, a potupen býti od národu tohoto.
\par 26 A jakož se dálo za dnu Noé, tak bude i za dnu Syna cloveka.
\par 27 Jedli, pili, ženili se, vdávaly se až do toho dne, v kterémžto Noé všel do korábu; i prišla potopa, a zahladila všecky.
\par 28 A též podobne, jako se stalo ve dnech Lotových: Jedli, pili, kupovali, prodávali, štepovali, staveli.
\par 29 Ale dne toho, když vyšel Lot z Sodomy, pršel ohen s sirou s nebe, a zahladil všecky.
\par 30 Takt nápodobne bude v ten den, když se Syn cloveka zjeví.
\par 31 V ten cas kdo by byl na streše, a nádobí jeho v domu, nesstupuj, aby je pobral; a kdo na poli, též nevracuj se zase.
\par 32 Pomnete na Lotovu ženu.
\par 33 Nebo kdož by koli hledal život svuj zachovati, ztratít jej; a kdož by jej koli ztratil, obživít jej.
\par 34 Pravímt vám: V tu noc budou dva na loži jednom; jeden bude vzat, a druhý opušten.
\par 35 Dve budou mleti spolu; jedna bude vzata, a druhá opuštena.
\par 36 Dva budou na poli; jeden bude vzat, a druhý opušten.
\par 37 I odpovedevše, rekli jemu: Kde, Pane? On pak rekl jim: Kdežt bude telo, tamt se shromáždí i orlice.

\chapter{18}

\par 1 Povedel jim také i podobenství, kterak by potrebí bylo vždycky se modliti a neoblevovati,
\par 2 Rka: Byl jeden soudce v meste jednom, kterýž se Boha nebál a cloveka nestydel.
\par 3 Byla pak vdova jedna v témž meste. I prišla k nemu, rkuci: Pomsti mne nad protivníkem mým.
\par 4 A on nechtel za dlouhý cas. Ale potom rekl sám v sobe: Ac se Boha nebojím, a cloveka nestydím,
\par 5 Však že mi pokoje nedá tato vdova, pomstím jí, aby naposledy prijduci, neuhanela mne.
\par 6 I dí Pán: Slyšte, co praví soudce nepravý.
\par 7 A což by pak Buh nepomstil volených svých, volajících k nemu dnem i nocí, ackoli i prodlévá jim?
\par 8 Pravímt vám, žet jich brzo pomstí. Ale když prijde Syn cloveka, zdaliž nalezne víru na zemi?
\par 9 I rekl také k nekterým, kteríž v sebe doufali, že by spravedliví byli, jiných sobe za nic nevážíce, podobenství toto:
\par 10 Dva muži vstupovali do chrámu, aby se modlili, jeden farizeus a druhý publikán.
\par 11 Farizeus stoje soukromí, takto se modlil: Bože, dekuji tobe, že nejsem jako jiní lidé, dráci, nespravedliví, cizoložníci, aneb jako i tento publikán.
\par 12 Postím se dvakrát do téhodne, desátky dávám ze všech vecí, kterýmiž vládnu.
\par 13 Publikán pak zdaleka stoje, nechtel ani ocí k nebi pozdvihnouti, ale bil se v prsy své, rka: Bože, bud milostiv mne hríšnému.
\par 14 Pravímt vám: Odšel tento, ospravedlnen jsa, do domu svého, a ne onen. Nebo každý, kdož se povyšuje, bude ponížen; a kdož se ponižuje, bude povýšen.
\par 15 Prinášeli také k nemu i nemluvnátka, aby se jich dotýkal. To videvše ucedlníci, primlouvali jim.
\par 16 Ale Ježíš svolav je, rekl: Nechte dítek, at jdou ke mne, a nebrante jim, nebo takovýcht jest království Boží.
\par 17 Amen pravím vám: Kdož by koli neprijal království Božího jako díte, nevejdet do neho.
\par 18 I otázalo se ho jedno kníže, rka: Mistre dobrý, co cine, život vecný obdržím?
\par 19 I rekl jemu Ježíš: Co mne nazýváš dobrým? Žádný není dobrý, než sám toliko Buh.
\par 20 Však umíš prikázání: Nezcizoložíš, nezabiješ, nepokradeš, nepromluvíš krivého svedectví, cti otce svého i matku svou.
\par 21 On pak rekl: Toho všeho ostríhal jsem od své mladosti.
\par 22 Slyšav to Ježíš, rekl mu: Ještet se jednoho nedostává. Všecko, což máš, prodej, a rozdej chudým, a budeš míti poklad v nebi, a pojd, následuj mne.
\par 23 On pak uslyšav to, zarmoutil se; byl zajisté bohatý velmi.
\par 24 A videv jej Ježíš zarmouceného, rekl: Aj, jak nesnadne ti, kdož statky mají, do království Božího vejdou!
\par 25 Snáze jest zajisté velbloudu skrze jehelnou dírku projíti, nežli bohatému vjíti do království Božího.
\par 26 Tedy rekli ti, kteríž to slyšeli: I kdož muže spasen býti?
\par 27 A on dí jim: Což jest nemožného u lidí, možné jest u Boha.
\par 28 I rekl Petr: Aj, my opustili jsme všecko, a šli jsme za tebou.
\par 29 On pak rekl jim: Amen pravím vám, že není žádného, kterýž by opustil dum, neb rodice, neb bratrí, neb manželku, nebo dítky, pro království Boží,
\par 30 Aby nevzal v tomto casu mnohem více, v budoucím pak veku míti bude život vecný.
\par 31 Tedy pojav Ježíš dvanácte, rekl jim: Aj, vstupujeme do Jeruzaléma, a naplnít se všecko to, což psáno jest skrze Proroky o Synu cloveka.
\par 32 Nebo vydán bude pohanum, a bude posmíván, a zlehcen i uplván.
\par 33 A ubicujíce, zamordují jej, ale on tretího dne z mrtvých vstane.
\par 34 Oni pak tomu nic nerozumeli, a bylo slovo to skryto pred nimi, aniž vedeli, co se pravilo.
\par 35 I stalo se, když se približoval k Jericho, slepý jeden sedel podle cesty, žebre.
\par 36 A slyšev zástup pomíjející, otázal se, co by to bylo?
\par 37 I oznámili jemu, že Ježíš Nazaretský tudy jde.
\par 38 I zvolal, rka: Ježíši, synu Daviduv, smiluj se nade mnou!
\par 39 A ti, kteríž napred šli, primlouvali mu, aby mlcal. Ale on mnohem více volal: Synu Daviduv, smiluj se nade mnou!
\par 40 Tedy zastaviv se Ježíš, rozkázal ho k sobe privésti. A když se približoval, otázal se ho,
\par 41 Rka: Co chceš, at tobe uciním? On pak dí: Pane, at vidím.
\par 42 A Ježíš rekl jemu: Prohlédni. Víra tvá te uzdravila.
\par 43 A ihned prohlédl, a šel za ním, velebe Boha. A všecken lid videv to, vzdal chválu Bohu.

\chapter{19}

\par 1 A všed Ježíš, bral se pres Jericho.
\par 2 A aj, muž, jménem Zacheus, a ten byl hejtman nad celnými, a byl bohatý.
\par 3 I žádostiv byl videti Ježíše, kdo by byl; a nemohl pro zástup, nebo postavy malé byl.
\par 4 A predbeh napred, vstoupil na strom planého fíku, aby jej videl; neb tudy mel jíti.
\par 5 A když prišel k tomu místu, pohledev zhuru Ježíš, uzrel jej, i rekl jemu: Zachee, spešne sstup dolu, nebo dnes v domu tvém musím zustati.
\par 6 I sstoupil rychle, a prijal jej radostne.
\par 7 A videvše to všickni, reptali, rkouce: K cloveku hríšnému se obrátil.
\par 8 Stoje pak Zacheus, rekl ku Pánu: Aj, polovici statku svého, Pane, dávám chudým, a oklamal-li jsem v cem koho, navracuji to ctvernásob.
\par 9 I dí jemu Ježíš: Dnes spasení stalo se domu tomuto, protože i on jest syn Abrahamuv.
\par 10 Nebo prišel Syn cloveka, aby hledal a spasil, což bylo zahynulo.
\par 11 Toho když oni poslouchali, promluvil k nim dále podobenství, protože byl blízko od Jeruzaléma a že se oni domnívali, že by se hned melo zjeviti království Boží.
\par 12 I rekl: Clovek jeden rodu znamenitého odšel do daleké krajiny, aby prijal království a zase se navrátil.
\par 13 I povolav desíti služebníku svých, dal jim deset hriven, a rekl jim: Kupctež, dokudž neprijdu.
\par 14 Meštané pak jeho nenávideli ho, a poslali poselství za ním, rkouce: Nechcemet, aby tento kraloval nad námi.
\par 15 I stalo se, když se navrátil, prijav království, rozkázal zavolati tech svých služebníku, kterýmž byl dal peníze, aby zvedel, jak kdo mnoho získal.
\par 16 I prišel první, rka: Pane, hrivna tvá deset hriven získala.
\par 17 I rekl jemu: To dobre, služebníce dobrý. Že jsi nad málem byl verný, mejž moc nad desíti mesty.
\par 18 A druhý prišel, rka: Pane, hrivna tvá získala pet hriven.
\par 19 I tomu rekl: I ty budiž nad peti mesty.
\par 20 A jiný prišel, rka: Pane, aj, ted hrivna tvá, kterouž jsem mel složenou v šátku.
\par 21 Nebo jsem se bál tebe, ješto jsi clovek prísný; béreš, ceho jsi nepoložil, a žneš, ceho jsi nerozsíval.
\par 22 I rekl jemu: Z úst tvých soudím tebe, služebníce zlý. Vedel jsi, že jsem já clovek prísný, bera, což jsem nepoložil, a žna, cehož jsem nerozsíval.
\par 23 I proces tedy nedal penez mých na stul, a já prijda, byl bych je vzal i s požitky?
\par 24 I rekl tem, kteríž tu stáli: Vezmete od neho hrivnu, a dejte tomu, kterýž má deset hriven.
\par 25 I rekli jemu: Pane, mát deset hriven.
\par 26 I dí jim: Jiste pravím vám, že každému, kdož má, bude dáno, ale od toho, kterýž nemá, i to, což má, bude odjato.
\par 27 Ty pak neprátely mé, kteríž nechteli, abych nad nimi kraloval, privedte sem a zmordujte prede mnou.
\par 28 To povedev, šel predce, vstupuje k Jeruzalému.
\par 29 I stalo se; když se priblížil k Betfagi a k Betany, k hore, kteráž slove Olivetská, poslal dva ucedlníky své,
\par 30 Rka: Jdete do mestecka, kteréž proti vám jest. Do kteréhožto vejdouce, naleznete oslátko privázané, na nemžto nikdy žádný z lidí nesedel. Odvežtež je, a privedte ke mne.
\par 31 A optal-lit by se vás kdo, proc je odvazujete, tak jemu díte: Protože Pán ho potrebuje.
\par 32 Tedy odšedše ti, kteríž byli posláni, nalezli tak, jakž jim byl povedel.
\par 33 A když odvazovali oslátko, rekli páni jeho k nim: Proc odvazujete oslátko?
\par 34 A oni rekli: Pán ho potrebuje.
\par 35 I privedli je k Ježíšovi, a vloživše roucha svá na oslátko, vsadili na ne Ježíše.
\par 36 A když on jel, stlali roucha svá na ceste.
\par 37 Když se pak již približoval k místu tomu, kudyž scházejí s hory Olivetské, pocalo všecko množství ucedlníku radostne chváliti Boha hlasem velikým ze všech divu, kteréž byli videli,
\par 38 Rkouce: Požehnaný král, jenž se bére ve jménu Páne. Pokoj na nebi, a sláva na výsostech.
\par 39 Ale nekterí z farizeu, kteríž tu byli v zástupu, rekli k nemu: Mistre, potresci ucedlníku svých.
\par 40 I odpovedev, rekl jim: Pravímt vám: Budou-li tito mlceti, kamenít bude volati.
\par 41 A když se priblížil, uzrev mesto, plakal nad ním,
\par 42 Rka: Ó kdybys poznalo i ty, a to aspon v takový tento den tvuj, které by veci ku pokoji tobe byly; ale skrytot jest to nyní od ocí tvých.
\par 43 Nebo prijdou na te dnové, v nichž obklící te neprátelé tvoji valem, a oblehnou tebe, a ssouží te se všech stran.
\par 44 A s zemí srovnají te, i syny tvé, kteríž v tobe jsou, a nenechajít v tobe kamene na kameni, protože jsi nepoznalo casu navštívení svého.
\par 45 A všed do chrámu, pocal vymítati prodavace a kupce z neho,
\par 46 Rka jim: Psáno jest: Dum muj dum modlitby jest, vy jste jej pak ucinili peleší lotrovskou.
\par 47 I ucil na každý den v chráme. Prední pak kneží a zákoníci i prední v lidu hledali ho zahladiti.
\par 48 A nenalezli, co by jemu ucinili. Nebo všecken lid jej sobe liboval, poslouchaje ho.

\chapter{20}

\par 1 I stalo se v jeden den, když on ucil lid v chráme a kázal evangelium, že prišli k tomu prední kneží a zákoníci s staršími,
\par 2 I rekli jemu: Povez nám, jakou mocí tyto veci ciníš, aneb kdo tobe tuto moc dal?
\par 3 I odpovedev, rekl jim: Otížit se i já vás o jedné veci a odpoveztež mi:
\par 4 Krest Januv s nebe-li byl, cili z lidí?
\par 5 Oni pak uvažovali to mezi sebou, rkouce: Jestliže bychom rekli: S nebe, dít nám: Procež jste tedy neuverili jemu?
\par 6 Pakli díme: Z lidí, lid všecken ukamenuje nás; neb oni cele tak drží, že Jan jest prorok.
\par 7 I odpovedeli: Že nevedí, odkud jest byl.
\par 8 I rekl jim Ježíš: Aniž já vám povím, jakou mocí toto ciním.
\par 9 I pocal lidu praviti podobenství toto: Clovek jeden štípil vinici, a pronajal ji vinarum, a sám odšed podál byl tam za mnohé casy.
\par 10 A v cas slušný poslal k vinarum služebníka, aby z ovoce vinice dali jemu. Vinari pak zmrskavše jej, pustili ho prázdného.
\par 11 A on poslal druhého služebníka. Oni pak i toho zmrskavše a zohavivše, pustili prázdného.
\par 12 I poslal tretího. Ale oni i toho zranivše, vystrcili ven.
\par 13 Tedy rekl Pán vinice: Co uciním? Pošli svého milého Syna. Snad když toho uzrí, ostýchati se budou.
\par 14 Uzrevše pak vinari, rozmlouvali mezi sebou, rkouce: Tentot jest dedic; pojdte, zabijme jej, aby naše bylo dedictví.
\par 15 A vystrcivše jej ven z vinice, zamordovali ho. Což tedy uciní jim Pán vinice?
\par 16 Prijde a vyhladí vinare ty, a dá vinici jiným. To uslyšavše, rekli: Odstup to.
\par 17 A on pohledev na ne, rekl: Co jest pak to, což napsáno jest: Kámen, kterýmž pohrdli delníci, ten ucinen jest v hlavu úhelní.
\par 18 Každý, kdož padne na ten kámen, rozrazí se; a na kohož by on upadl, potret jej.
\par 19 I hledali prední kneží a zákoníci, jak by nan vztáhli ruce v tu hodinu, ale báli se lidu. Nebo porozumeli, že by na ne mluvil podobenství to.
\par 20 Tedy ukládajíce o nem, poslali špehére, kteríž by se spravedlivými cinili, aby ho polapili v reci, a potom jej vydali vrchnosti a v moc hejtmanu.
\par 21 I otázali se ho oni, rkouce: Mistre víme, že práve mluvíš a ucíš, a neprijímáš osoby, ale v pravde ceste Boží ucíš.
\par 22 Sluší-li nám dan dávati císari, cili nic?
\par 23 Ale Ježíš rozumeje chytrosti jejich, dí jim: Co mne pokoušíte?
\par 24 Ukažte mi peníz. Cí má obraz a nápis? I odpovedevše, rekli: Císaruv.
\par 25 On pak rekl jim: Dejtež tedy, co jest císarova, císari, a což jest Božího, Bohu.
\par 26 I nemohli ho za slovo popadnouti pred lidem, a divíce se odpovedi jeho, umlkli.
\par 27 Pristoupivše pak nekterí z saduceu, (kteríž odpírají býti vzkríšení,) otázali se ho,
\par 28 Rkouce: Mistre, Mojžíš napsal nám: Kdyby bratr necí umrel, maje manželku, a umrel by bez detí, aby ji pojal bratr jeho za manželku, a vzbudil síme bratru svému.
\par 29 I bylo sedm bratrí, a první pojav ženu, umrel bez detí.
\par 30 I pojal ji druhý, a umrel i ten bez detí.
\par 31 A tretí pojal ji, též i všech tech sedm, a nezustavivše semene, zemreli.
\par 32 Nejposléze po všech umrela i žena.
\par 33 Protož pri vzkríšení kterého z nich bude manželka, ponevadž sedm jich melo ji za manželku?
\par 34 A odpovídaje, rekl jim Ježíš: Synové tohoto sveta žení se a vdávají.
\par 35 Ale ti, kteríž hodni jmíni budou dosáhnouti onoho veku a vzkríšení z mrtvých, ani se ženiti budou ani vdávati.
\par 36 Nebo ani umírati více nebudou moci, andelum zajisté rovni budou. A jsou synové Boží, ponevadž jsou synové vzkríšení.
\par 37 A že mrtví vstanou z mrtvých, i Mojžíš ukázal pri onom kri, když nazývá Pána Bohem Abrahamovým a Bohem Izákovým a Bohem Jákobovým.
\par 38 Buht pak není mrtvých, ale živých, nebo všickni jsou jemu živi.
\par 39 Tedy odpovedevše nekterí z zákoníku, rekli: Mistre, dobre jsi povedel.
\par 40 I neodvážili se jeho na nic více tázati.
\par 41 On pak rekl jim: Kterak nekterí praví Krista býti synem Davidovým?
\par 42 A sám David praví v knihách Žalmových: Rekl Pán Pánu mému: Sed na pravici mé,
\par 43 Ažt i položím neprátely tvé v podnož noh tvých.
\par 44 Ponevadž David jej Pánem nazývá, i kterakž syn jeho jest?
\par 45 I rekl ucedlníkum svým prede vším lidem:
\par 46 Varujte se od zákoníku, kteríž rádi chodí v krásném rouše a milují pozdravování na trzích a prední stolice v školách a první místo na vecerích,
\par 47 Kteríž zžírají domy vdovské pod zámyslem dlouhé modlitby. Tit vezmou težší odsouzení.

\chapter{21}

\par 1 A pohledev, uzrel lidi bohaté, kteríž metali dary své do pokladnice.
\par 2 Uzrel pak i jednu vdovu chudickou, ana uvrhla dva šarty.
\par 3 I rekl: Vpravde pravím vám, že vdova tato chudá více uvrhla nežli všickni jiní.
\par 4 Nebo všickni tito z toho, což jim zbývalo, dali dary Bohu, tato pak z své chudoby všecku živnost svou, kterouž mela, uvrhla.
\par 5 A když nekterí pravili o chrámu, kterak by kamením pekným i jinými okrasami ozdoben byl, rekl:
\par 6 Nacež se to díváte? Prijdout dnové, v nichžto nebude zustaven kámen na kameni, kterýž by nebyl zboren.
\par 7 I otázali se ho, rkouce: Mistre, kdy to bude? A které znamení, když se to bude míti státi?
\par 8 On pak rekl: Vizte, abyste nebyli svedeni. Nebo mnozí prijdou ve jménu mém, rkouce: Že já jsem Kristus, a cas se blíží. Protož nepostupujte po nich.
\par 9 Když pak uslyšíte o válkách a ruznicech, nestrachujte se; nebot musí to prve býti, ale ne ihned konec.
\par 10 Tehdy pravil jim: Povstanet národ proti národu, a království proti království.
\par 11 A zeme tresení veliká budou po místech, a hladové, a morové, hruzy i zázrakové s nebe velicí.
\par 12 Ale pred tím prede vším vztáhnou ruce své na vás, a protiviti se vám budou, vydávajíce vás do škol a žaláru, vodíce k králum a k vladarum pro jméno mé.
\par 13 A tot se vám díti bude na svedectví.
\par 14 Protož složtež to v srdcích vašich, abyste se nestarali, kterak byste odpovídati meli.
\par 15 Ját zajisté dám vám ústa a moudrost, kteréžto nebudou moci odolati, ani proti vám ostáti všickni protivníci vaši.
\par 16 Budete pak zrazováni i od rodicu a od bratru, od príbuzných i od prátel, a zmordují nekteré z vás.
\par 17 A budete v nenávisti všechnem pro jméno mé.
\par 18 Ale vlas s hlavy vaší nezahyne.
\par 19 V trpelivosti vaší vládnete dušemi vašimi.
\par 20 Když pak uzríte obležený od vojska Jeruzalém, tedy vezte, žet se priblížilo zkažení jeho.
\par 21 A tehdy ti, kdož jsou v Judstvu, utíkejte k horám, a kdo uprostred neho, vyjdete, a kterí v koncinách, nevcházejte do neho.
\par 22 Nebot budou dnové pomsty, aby se naplnilo všecko, což psáno jest.
\par 23 Ale beda tehotným a tem, kteréž kojí v tech dnech. Nebot bude dav veliký v této zemi, a hnev Boží nad lidem tímto.
\par 24 I padati budou ostrostí mece, a zjímaní vedeni budou mezi všecky národy, a Jeruzalém tlacen bude od pohanu, dokudž se nenaplní casové pohanu.
\par 25 A budout znamení na slunci a na mesíci i na hvezdách, a na zemi soužení národu, nevedoucích se kam díti, když zvuk vydá more a vlnobití,
\par 26 Takže zmrtvejí lidé pro strach a pro ocekávání tech vecí, kteréž prijdou na všecken svet. Nebo moci nebeské pohybovati se budou.
\par 27 A tehdyt uzrí Syna cloveka, an se bére v oblace s mocí a slavou velikou.
\par 28 A když se toto pocne díti, pohledtež a pozdvihnetež hlav vašich,protože se približuje vykoupení vaše.
\par 29 I povedel jim podobenství: Patrte na fíkový strom i na všecka stromoví.
\par 30 Když se již pucí, vidouce to, sami to znáte, že blízko jest léto.
\par 31 Takž i vy, když uzríte, ano se tyto veci dejí, veztež, že blízko jest království Boží.
\par 32 Amen pravím vám, že nepomine vek tento, ažt se toto všecko stane.
\par 33 Nebe a zeme pominou, ale slova má nepominou.
\par 34 Pilne se pak varujte, aby snad nebyla obtížena srdce vaše obžerstvím a opilstvím a pecováním o tento život, a vnáhle prikvacil by vás ten den.
\par 35 Nebo jako osidlo prijde na všecky, jenž prebývají na tvári vší zeme.
\par 36 Protož bdete všelikého casu, modléce se, abyste hodni byli ujíti všech tech vecí, kteréž se budou díti, a postaviti se pred Synem cloveka.
\par 37 I býval ve dne v chráme, uce, ale v noci vycházeje, prebýval na hore, kteráž slove Olivetská.
\par 38 A všecken lid na úsvite pricházel k nemu do chrámu, aby ho poslouchal.

\chapter{22}

\par 1 Približoval se pak svátek presnic, jenž slove velikanoc.
\par 2 I hledali prední kneží a zákoníci, kterak by jej vyhladili; ale obávali se lidu.
\par 3 Tedy dábel vstoupil do Jidáše, kterýž sloul Iškariotský, jednoho z poctu dvanácti.
\par 4 A on odšed, mluvil s predními knežími, a s úredníky nad chrámem, kterak by ho jim zradil.
\par 5 I zradovali se, a smluvili s ním, že mu chtí peníze dáti.
\par 6 A on také jim prirekl. I hledal príhodného casu, aby ho jim zradil bez zástupu.
\par 7 Tedy prišel den presnic, v kterémžto zabit mel býti beránek.
\par 8 I poslal Ježíš Petra a Jana, rka: Jdouce, pripravte nám beránka, abychom jedli.
\par 9 A oni rekli mu: Kde chceš, at pripravíme?
\par 10 On pak rekl k nim: Aj, když vcházeti budete do mesta, potkát vás clovek, dcbán vody nesa. Jdetež za ním do domu, do kteréhož vejde.
\par 11 A díte hospodári toho domu: Vzkazuje tobe Mistr: Kde jest sín, kdežto budu jísti beránka s ucedlníky svými?
\par 12 A ont vám ukáže veceradlo veliké podlážené. Tam pripravte.
\par 13 I odšedše, nalezli, jakž jim povedel, a pripravili beránka.
\par 14 A když prišel cas vecere, posadil se za stul, a dvanácte apoštolu s ním.
\par 15 I rekl jim: Žádostí žádal jsem tohoto beránka jísti s vámi, prve než bych trpel.
\par 16 Nebo pravímt vám, žet ho již více nebudu jísti, ažt se naplní v království Božím.
\par 17 A vzav kalich, a díky ciniv, rekl: Vezmete jej a delte mezi sebou.
\par 18 Nebot pravím vám, žet nebudu píti z plodu vinného korene, ažt království Boží prijde.
\par 19 A vzav chléb, díky ciniv, lámal a dal jim, rka: To jest telo mé, kteréž se za vás dává. To cinte na mou památku.
\par 20 Takž také dal jim i kalich, když bylo po veceri, rka: Tento kalich jest nová smlouva v mé krvi, kteráž se za vás vylévá.
\par 21 Ale aj, ruka zrádce mého se mnou jest za stolem.
\par 22 A Syn zajisté cloveka jde, tak jakž jest uloženo o nem, ale beda cloveku tomu, kterýž ho zrazuje.
\par 23 Tedy oni pocali vyhledávati mezi sebou, kdo by z nich byl, kterýž by to mel uciniti.
\par 24 Stal se pak i svár mezi nimi, kdo by z nich zdál se býti vetší.
\par 25 On pak rekl jim: Králové národu panují nad nimi, a kteríž moc mají nad nimi, dobrodincové slovou.
\par 26 Ale vy ne tak. Nýbrž kdož vetší jest mezi vámi, budiž jako nejmenší, a kdož vudce jest, budiž jako sloužící.
\par 27 Nebo kdo vetší jest, ten-li, kterýž sedí, cili ten, kterýž slouží? Zdali ne ten, kterýž sedí? Ale já mezi vámi jsem jako ten, kterýž slouží.
\par 28 Vy pak jste ti, kteríž jste v mých pokušeních se mnou zustali.
\par 29 A ját vám zpusobuji, jakož mi zpusobil Otec muj, království,
\par 30 Abyste jedli a pili za stolem mým v království mém, a sedeli na stolicích, soudíce dvanáctero pokolení Izraelské.
\par 31 I rekl Pán: Šimone, Šimone, aj, satan vyprosil vás, aby vás tríbil jako pšenici.
\par 32 Ale ját jsem prosil za tebe, aby nezhynula víra tvá. A ty nekdy obráte se, potvrzuj bratrí svých.
\par 33 A on rekl jemu: Pane, s tebou hotov jsem i do žaláre i na smrt jíti.
\par 34 On pak dí: Pravím tobe, Petre, nezazpívát dnes kohout, až prve trikrát zapríš, že neznáš mne.
\par 35 I rekl jim: Když jsem vás posílal bez pytlíka, a bez mošny, a bez obuvi, zdali jste v cem nedostatek meli? A oni rekli: V nicemž.
\par 36 Tedy dí jim: Ale nyní, kdo má pytlík, vezmi jej, a též i mošnu; a kdož nemá, prodej sukni svou, a kup sobe mec.
\par 37 Nebo pravím vám, že se ješte to musí naplniti na mne, což psáno: A s nešlechetnými pocten jest. Nebo ty veci, kteréž psány jsou o mne, konec berou.
\par 38 Oni pak rekli: Pane, aj, dva mece ted. A on rekl jim: Dostit jest.
\par 39 A vyšed podle obyceje svého, šel na horu Olivovou, a šli za ním i ucedlníci jeho.
\par 40 A když prišel na místo, rekl jim: Modlte se, abyste nevešli v pokušení.
\par 41 A sám vzdáliv se od nich, jako by mohl kamenem dohoditi, a poklek na kolena, modlil se,
\par 42 Rka: Otce, chceš-li, prenes kalich tento ode mne, ale však ne má vule, ale tvá stan se.
\par 43 I ukázal se jemu andel s nebe, posiluje ho.
\par 44 A jsa v boji, horliveji se modlil. I ucinen jest pot jeho jako krupe krve tekoucí na zemi.
\par 45 A vstav od modlitby, a prišed k ucedlníkum, nalezl je, ani spí zámutkem.
\par 46 I rekl jim: Co spíte? Vstante a modlte se, abyste nevešli v pokušení.
\par 47 A když on ješte mluvil, aj, zástup, a ten, kterýž sloul Jidáš, jeden ze dvanácti, šel napred, a priblížil se k Ježíšovi, aby jej políbil.
\par 48 Ježíš pak rekl jemu: Jidáši, políbením Syna cloveka zrazuješ?
\par 49 A vidouce ti, kteríž pri nem byli, k cemu se chýlí, rekli jemu: Pane, budeme-liž bíti mecem?
\par 50 I uderil jeden z nich služebníka nejvyššího kneze, a utal ucho jeho pravé.
\par 51 A Ježíš odpovedev, rekl: Nechtež až potud. A dotkna se ucha jeho, uzdravil jej.
\par 52 I dí Ježíš tem, kteríž prišli k nemu, predním knežím a úredníkum chrámu a starším: Jako na lotra vyšli jste s meci a s kyjmi?
\par 53 Ješto na každý den býval jsem s vámi v chráme, a nevztáhli jste rukou na mne. Ale totot jest ta vaše hodina a moc temnosti.
\par 54 A oni javše jej, vedli ho, a uvedli do domu nejvyššího kneze. Petr pak šel za ním zdaleka.
\par 55 A když zanítili ohen uprostred síne a posadili se vukol, sedl Petr mezi ne.
\par 56 A uzrevši ho jedna devecka, an sedí u ohne, a pilne nan pohledevši, rekla: I tento byl s ním.
\par 57 A on zaprel ho, rka: Ženo, neznám ho.
\par 58 A po malé chvíli jiný, vida jej, rekl: I ty z nich jsi. Petr pak rekl: Ó clovece, nejsem.
\par 59 A potom asi po jedné hodine jiný potvrzoval, rka: V pravde i tento s ním byl, neb i Galilejský jest.
\par 60 I rekl Petr: Clovece, nevím, co pravíš. A hned, když on ješte mluvil, kohout zazpíval.
\par 61 I obrátiv se Pán, pohledel na Petra. I rozpomenul se Petr na slovo Páne, kterak jemu byl rekl: Že prve než kohout zazpívá, trikrát mne zapríš.
\par 62 I vyšed ven Petr, plakal horce.
\par 63 Muži pak ti, kteríž drželi Ježíše, posmívali se jemu, tepouce ho.
\par 64 A zakrývajíce ho, bili jej v tvár, a tázali se ho, rkouce: Prorokuj, kdo jest, kterýž tebe uderil?
\par 65 A jiného mnoho, rouhajíce se, mluvili proti nemu.
\par 66 A když byl den, sešli se starší lidu a prední kneží a zákoníci, a vedli ho do rady své,
\par 67 Rkouce: Jsi-li ty Kristus? Povez nám! I dí jim: Povím-li vám, nikoli neuveríte.
\par 68 A pakli se vás co otíži, neodpovíte mi, ani propustíte.
\par 69 Ale od této chvíle Syn cloveka sedne na pravici moci Boží.
\par 70 I rekli všickni: Tedy jsi ty Syn Boží? On pak rekl jim: Vy pravíte, že já jsem.
\par 71 A oni rekli: Což ješte potrebujeme svedectví? Však jsme sami slyšeli z úst jeho.

\chapter{23}

\par 1 Tehdy povstavši všecko to množství jich, vedli jej ku Pilátovi.
\par 2 A pocali nan žalovati, rkouce: Tohoto jsme nalezli, an prevrací lid, a brání dane dávati císari, prave se býti Kristem králem.
\par 3 Pilát pak otázal se ho, rka: Ty-li jsi král Židovský? A on odpovedev, rekl jemu: Ty pravíš.
\par 4 I dí Pilát k predním knežím a k zástupum: Žádné viny nenalézám na tomto cloveku.
\par 5 Oni pak více se rozmáhali v kriku, rkouce: Bourít lid, uce po všem Judstvu, pocav od Galilee až sem.
\par 6 Tedy Pilát uslyšav o Galilei, otázal se, byl-li by clovek Galilejský.
\par 7 A když zvedel, že by byl z panství Herodesova, poslal jej k Herodesovi, kterýž také v Jeruzaléme byl v ty dni.
\par 8 Herodes pak uzrev Ježíše, zradoval se velmi. Nebo dávno žádal videti jej, protože mnoho o nem slýchal, a nadál se, že nejaký div od neho ucinený uzrí.
\par 9 I tázal se ho mnohými recmi, ale on jemu nic neodpovídal.
\par 10 Stáli pak tu prední kneží a zákoníci, tuze na nej žalujíce.
\par 11 A pohrdna jím Herodes s svým rytírstvem, a naposmívav se jemu, oblékl jej v roucho bílé, i odeslal zase ku Pilátovi.
\par 12 I ucineni jsou prátelé Herodes s Pilátem v ten den. Nebo pred tím neprátelé byli vespolek.
\par 13 Pilát pak svolav prední kneží a úradné osoby i lid,
\par 14 Rekl k nim: Dali jste mi toho cloveka, jako by lid odvracel po sobe, a aj, já pred vámi vyptávaje se ho, žádné viny jsem nenalezl na tom cloveku ve všem tom, což vy na nej žalujete.
\par 15 Ano ani Herodes; nebo odeslal jsem vás k nemu, a aj, nic hodného smrti nestalo se jemu.
\par 16 Protož potresce ho, propustím jej.
\par 17 Musíval pak propoušteti jim v svátek jednoho vezne.
\par 18 Protož zkrikli spolu všecko množství, rkouce: Zahlad tohoto, a propust nám Barabbáše.
\par 19 Kterýž byl pro svádu nejakou v meste ucinenou a pro vraždu vsazen do žaláre.
\par 20 Tedy Pilát opet mluvil k nim, chteje propustiti Ježíše.
\par 21 A oni vždy volali, rkouce: Ukrižuj ho, ukrižuj!
\par 22 A on po tretí rekl k nim: Což jest pak zlého ucinil tento? Ját žádné príciny smrti nenalézám na nem. Protož potresce ho, propustím.
\par 23 Oni pak predce dotírali na nej krikem velikým, žádajíce, aby byl ukrižován. A rozmáhali se hlasové jejich a predních kneží.
\par 24 Pilát pak prisoudil, aby se naplnila žádost jejich.
\par 25 I propustil jim toho, kterýž pro bourku a vraždu vsazen byl do žaláre, za nehož prosili, ale Ježíše vydal k vuli jejich.
\par 26 A když jej vedli, chytivše Šimona nejakého Cyrenenského, jdoucího s pole, vložili na nej kríž, aby jej nesl za Ježíšem.
\par 27 I šlo za ním veliké množství lidu i žen, kteréžto plakaly a kvílily ho.
\par 28 A obrátiv se k nim Ježíš, dí: Dcery Jeruzalémské, neplactež nade mnou, ale radeji samy nad sebou placte a nad svými detmi.
\par 29 Nebo aj, prijdou dnové, v nichžto reknou: Blahoslavené neplodné, a bricha, kteráž nerodila, a prsy, kteréž nekrmily.
\par 30 Tehdyt pocnou ríci k horám: Padnete na nás, a pahrbkum: Prikrejte nás!
\par 31 Nebo ponevadž na zeleném dreve toto se deje, i co pak bude na suchém?
\par 32 Vedeni pak byli s ním i jiní dva zlocinci, aby spolu s ním byli ukrižováni.
\par 33 A když prišli na místo, kteréž slove popravištné, tu jej ukrižovali, i ty zlocince, jednoho na pravici, druhého pak na levici.
\par 34 Tedy Ježíš rekl: Otce, odpust jim, nebot nevedí, co ciní. A rozdelivše roucho jeho, metali o ne los.
\par 35 I stál lid, dívaje se. A posmívali se jemu knížata s nimi, rkouce: Jinýmt jest spomáhal, nechat nyní pomuže sám sobe, jestliže on jest Kristus, ten Boží zvolený.
\par 36 Posmívali se také jemu žoldnéri, pristupujíce a octa podávajíce jemu,
\par 37 A rkouce: Jsi-li ty ten král Židovský, spomoziž sám sobe.
\par 38 A byl také i nápis napsaný nad ním, literami Reckými a Latinskými a Židovskými: Tento jest král Židovský.
\par 39 Jeden pak z tech zlocincu, kteríž s ním viseli, rouhal se jemu, rka: Jsi-li ty Kristus, spomoziž sobe i nám.
\par 40 A odpovedev druhý, trestal ho, rka: Ty se ani Boha nebojíš, ješto jsi v témž odsouzení?
\par 41 Myt zajisté spravedlive trpíme, nebo hodnou pomstu za skutky naše béreme, ale tento nic zlého neucinil.
\par 42 I dí Ježíšovi: Pane, rozpomen se na mne, když prijdeš do království svého.
\par 43 I rekl mu Ježíš: Amen pravím tobe, dnes budeš se mnou v ráji.
\par 44 A bylo okolo hodiny šesté. I stala se tma po vší zemi až do hodiny deváté.
\par 45 I zatmelo se slunce, a opona chrámová roztrhla se na poly.
\par 46 A zvolav Ježíš hlasem velikým, rekl: Otce, v ruce tvé poroucím ducha svého. A to povedev, umrel.
\par 47 A vida centurio, co se stalo, velebil Boha, rka: Jiste clovek tento spravedlivý byl.
\par 48 A všickni zástupové, prítomní tomu divadlu, hledíce na to, co se dálo, tepouce prsy své, navracovali se.
\par 49 Stáli pak všickni známí jeho zdaleka, i ženy, kteréž byly prišly za ním od Galilee, hledíce na to.
\par 50 A aj, muž jménem Jozef, jeden z úradu, muž dobrý a spravedlivý,
\par 51 Kterýž byl nepovolil rade a skutku jejich, z Arimatie mesta Judského, kterýžto také ocekával království Božího,
\par 52 Ten pristoupiv ku Pilátovi, vyprosil telo Ježíšovo,
\par 53 A složiv je s kríže, obvinul v kment a pochoval je v hrobe vytesaném v skále, v kterémž ješte nebyl žádný pochován.
\par 54 A byl den pripravování, a sobota se zacínala.
\par 55 Šedše pak také za ním ženy, kteréž byly s Ježíšem prišly od Galilee, pohledely na hrob, a kterak pochováno bylo telo jeho.
\par 56 Vrátivše se pak, pripravily vonné veci a masti, ale v sobotu odpocinuly, podle prikázání.

\chapter{24}

\par 1 První pak den po sobote, velmi ráno vyšedše, prišly k hrobu, nesouce vonné veci, kteréž byly pripravily, a nekteré jiné byly spolu s nimi.
\par 2 I nalezly kámen odvalený od hrobu.
\par 3 A všedše tam, nenalezly tela Pána Ježíše.
\par 4 I stalo se, když ony se toho užasly, aj, muži dva postavili se podle nich, v rouše stkvoucím.
\par 5 Když se pak ony bály, a sklonily tvári své k zemi, rekli k nim: Co hledáte živého s mrtvými?
\par 6 Nenít ho tuto, ale vstalt jest. Rozpomente se, kterak mluvil vám, když ješte v Galilei byl,
\par 7 Rka: Že Syn cloveka musí vydán býti v ruce hríšných lidí, a ukrižován býti, a tretí den z mrtvých vstáti.
\par 8 I rozpomenuly se na slova jeho.
\par 9 A navrátivše se od hrobu, zvestovaly to všecko tem jedenácti ucedlníkum i jiným všechnem.
\par 10 Byly pak ženy ty: Maria Magdaléna a Johanna a Maria matka Jakubova, a jiné nekteré s nimi, kteréž vypravovaly to apoštolum.
\par 11 Ale oni meli za bláznovství slova jejich, a neverili jim.
\par 12 Tedy Petr vstav, bežel k hrobu, a pohledev do neho, uzrel prosteradla, ana sama leží. I odšel, dive se sám v sobe, co se to stalo.
\par 13 A aj, dva z nich šli toho dne do mestecka, kteréž bylo vzdálí od Jeruzaléma honu šedesáte, jemuž jméno Emaus.
\par 14 A rozmlouvali vespolek o tech všech vecech, kteréž se byly staly.
\par 15 I stalo se, když rozmlouvali a sebe se otazovali, že i Ježíš, priblíživ se k nim, šel s nimi.
\par 16 Ale oci jejich držány byly, aby ho nepoznali.
\par 17 I rekl k nim: Které jsou to veci, o nichž rozjímáte vespolek, jdouce, a proc jste smutní?
\par 18 A odpovedev jeden, kterémuž jméno Kleofáš, rekl jemu: Ty sám jeden jsi z príchozích do Jeruzaléma, ještos nezvedel, co se stalo v nem techto dnu?
\par 19 Kterýmžto on rekl: I co? Oni pak rekli jemu: O Ježíšovi Nazaretském, kterýž byl muž prorok, mocný v slovu i v skutku, pred Bohem i prede vším lidem,
\par 20 A kterak jej vydali prední kneží a knížata naše na odsouzení k smrti, i ukrižovali jej.
\par 21 My pak jsme se nadáli, že by on mel vykoupiti lid Izraelský. Ale nyní tomu všemu tretí den jest dnes, jakž se to stalo.
\par 22 Ale i ženy nekteré z našich zdesily nás, kteréž ráno byly u hrobu,
\par 23 A nenalezše tela jeho, prišly, pravíce, že jsou také videní andelské videly, kterížto praví, že by živ byl.
\par 24 I chodili nekterí z našich k hrobu, a nalezli tak, jakž pravily ženy, ale jeho nevideli.
\par 25 Tedy on rekl k nim: Ó blázni a zpozdilí srdcem k verení všemu tomu, což mluvili Proroci.
\par 26 Zdaliž nemusil tech vecí trpeti Kristus a tak vjíti v slávu svou?
\par 27 A pocav od Mojžíše a všech Proroku, vykládal jim všecka ta písma, kteráž o nem byla.
\par 28 A vtom priblížili se k mestecku, do kteréhož šli, a on potrh se, jako by chtel dále jíti.
\par 29 Ale oni prinutili ho, rkouce: Zustan s námi, nebo se již pripozdívá, a den se nachýlil. I všel, aby s nimi zustal.
\par 30 I stalo se, když sedel s nimi za stolem, vzav chléb, dobrorecil, a lámaje, podával jim.
\par 31 I otevríny jsou oci jejich, a poznali ho. On pak zmizel od ocí jejich.
\par 32 I rekli vespolek: Zdaliž srdce naše v nás nehorelo, když mluvil nám na ceste a otvíral nám písma?
\par 33 A vstavše v tu hodinu, vrátili se do Jeruzaléma, a nalezli shromáždených jedenácte, a ty, kteríž s nimi byli,
\par 34 Ani praví: Že vstal Pán práve, a ukázal se Šimonovi.
\par 35 I vypravovali oni také to, co se stalo na ceste, a kterak ho poznali v lámání chleba.
\par 36 A když oni o tom rozmlouvali, postavil se Ježíš uprostred nich, a rekl jim: Pokoj vám.
\par 37 Oni pak zhrozivše se a prestrašeni byvše, domnívali se, že by ducha videli.
\par 38 I dí jim: Co se strašíte a myšlení vstupují na srdce vaše?
\par 39 Vizte ruce mé i nohy mé, žet v pravde já jsem. Dotýkejte se a vizte; nebot duch tela a kostí nemá, jako mne vidíte míti.
\par 40 A povedev to, ukázal jim ruce i nohy.
\par 41 Když pak oni ješte neverili pro radost, ale divili se, rekl jim: Máte-li tu neco, ješto by se pojedlo?
\par 42 A oni podali jemu kusu ryby pecené a plástu strdi.
\par 43 A vzav to, pojedl pred nimi,
\par 44 A rekl jim: Tatot jsou slova, kteráž jsem mluvil vám, ješte byv s vámi: Že se musí naplniti všecko, což psáno jest v Zákone Mojžíšove a v Prorocích i v Žalmích o mne.
\par 45 Tedy otevrel jim mysl, aby rozumeli Písmum.
\par 46 A rekl jim: Že tak psáno jest a tak musil Kristus trpeti, a tretího dne z mrtvých vstáti,
\par 47 A aby bylo kázáno ve jménu jeho pokání a odpuštení hríchu mezi všemi národy, pocna od Jeruzaléma.
\par 48 Vy jste pak svedkové toho.
\par 49 A aj, já pošli zaslíbení Otce svého na vás. Vy pak cekejte v meste Jeruzaléme, dokudž nebudete obleceni mocí s výsosti.
\par 50 I vyvedl je ven až do Betany, a pozdvih rukou svých, dal jim požehnání.
\par 51 I stalo se, když jim žehnal, bral se od nich, a nesen jest do nebe.
\par 52 A oni poklonivše se jemu, navrátili se do Jeruzaléma s radostí velikou.
\par 53 A byli vždycky v chráme, chválíce a dobrorecíce Boha. Amen.


\end{document}