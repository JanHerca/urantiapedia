\begin{document}

\title{James}

\chapter{1}

\par 1 Jakub, Boží a Pána Jezukrista služebník, dvanácteru pokolení rozptýlenému pozdravení vzkazuje.
\par 2 Za nejvetší radost mejte, bratrí moji, kdyžkoli pokušeními obklicováni býváte rozlicnými,
\par 3 Vedouce, že zkušení víry vaší pusobí trpelivost.
\par 4 Trpelivost pak at má dokonalý skutek, abyste byli dokonalí a celí, v nicemž nemajíce nedostatku.
\par 5 Jestliže pak komu z vás nedostává se moudrosti, žádejž jí od Boha, kterýž všechnem dává ochotne a neomlouvá; i budet jemu dána.
\par 6 Žádejž pak duverne, nic nepochybuje. Nebo kdož pochybuje, podoben jest vlnám morským, kteréž vítr sem i tam žene, a jimi zmítá.
\par 7 Nedomnívej se zajisté clovek ten, by co vzíti mel ode Pána,
\par 8 Jakožto muž dvojí mysli a neustavicný ve všech cestách svých.
\par 9 Chlubiž se pak bratr ponížený v povýšení svém,
\par 10 A bohatý v ponížení se; nebo jako kvet byliny pomine.
\par 11 Nebo jakož slunce vzešlé s horkostí usušilo bylinu, a kvet její spadl, i ušlechtilost postavy jeho zhynula, takt i bohatý v svých cestách usvadne.
\par 12 Ale blahoslavený muž, kterýž snáší pokušení, nebo když bude zkušen, vezme korunu života, kterouž zaslíbil Pán tem, jenž ho milují.
\par 13 Žádný, když bývá pokoušín, neríkej, že by od Boha pokoušín byl; nebot Buh nemuže pokoušín býti ve zlém, aniž on koho pokouší.
\par 14 Ale jeden každý pokoušín bývá, od svých vlastních žádostí jsa zachvacován a oklamáván.
\par 15 Potom žádost když pocne, porodí hrích, hrích pak vykonaný zplozuje smrt.
\par 16 Nebludtež, bratrí moji milí.
\par 17 Všeliké dání dobré a každý dar dokonalý shury jest sstupující od Otce svetel, u nehožto není promenení, ani jakého pro obrácení se nekam jinam zastínení.
\par 18 On proto, že chtel, zplodil nás slovem pravdy, k tomu, abychom byli prvotiny nejaké stvorení jeho.
\par 19 A tak, bratrí moji milí, budiž každý clovek rychlý k slyšení, ale zpozdilý k mluvení, zpozdilý k hnevu.
\par 20 Nebo hnev muže spravedlnosti Boží nepusobí.
\par 21 Protož odvrhouce všelikou necistotu, a ohyzdnost zlosti, s tichostí prijímejte vsáté slovo, kteréž muže spasiti duše vaše.
\par 22 Budtež pak cinitelé slova, a ne posluchaci toliko, oklamávajíce sami sebe.
\par 23 Nebo byl-li by kdo posluchac slova, a ne cinitel, ten podoben jest muži spatrujícímu oblicej prirozený svuj v zrcadle.
\par 24 Vzhlédl se zajisté, i odšel, a hned zapomenul, jaký by byl.
\par 25 Ale kdož by se vzhlédl v dokonalý zákon svobody a zustával by v nem, ten nejsa posluchac zapominatelný, ale cinitel skutku, ten, pravím, blahoslavený bude v skutku svém.
\par 26 Zdá-li se pak komu z vás, že jest nábožný, avšak v uzdu nepojímá jazyka svého, ale svodí srdce své, takového marné jest náboženství.
\par 27 Náboženství cisté a neposkvrnené pred Bohem a Otcem totot jest: Navštevovati sirotky a vdovy v souženích jejich a ostríhati sebe neposkvrneného od sveta.

\chapter{2}

\par 1 Bratrí moji, nepripojujtež prijímání osob k víre slavného Pána našeho Jezukrista.
\par 2 Nebo kdyby prišel do shromáždení vašeho muž, maje prsten zlatý, v drahém rouše, a všel by také i chudý v chaterném odevu,
\par 3 A popatrili byste k tomu, jenž drahé roucho má, a rekli byste jemu: Ty sedni tuto pekne; chudému pak rekli byste: Ty stuj tamto, aneb sedni tuto, pod podnoží noh mých;
\par 4 Zdaliž jste již neucinili rozdílu mezi sebou a ucineni jste rozeznavatelé v myšleních zlých?
\par 5 Slyšte, bratrí moji milí, zdaliž Buh nevyvolil chudých na tomto svete, aby bohatí byli u víre a dedicové království, kteréž zaslíbil tem, jenž jej milují?
\par 6 Ale vy jste neuctili chudého. Zdaliž ne ti, jenž bohatí jsou, mocí utiskují vás, a onit vás i k soudum privozují?
\par 7 Zdali oni nerouhají se tomu slavnému jménu, kteréž vzýváno jest nad vámi?
\par 8 Jestliže pak plníte Zákon královský podle Písem: Milovati budeš bližního svého, jako sebe samého, dobre ciníte.
\par 9 Pakli osoby prijímáte, hrešíte, a Zákon vás tresce jako prestupníky.
\par 10 Nebo kdo by koli celého Zákona ostríhal, prestoupil by pak v jediném, ucinen jest všemi vinen.
\par 11 Ten zajisté, kterýž rekl: Nesesmilníš, takét jest rekl: Nezabiješ. Pakli bys nesesmilnil, ale zabil bys, ucinen jsi prestupníkem Zákona.
\par 12 Tak mluvte a tak cinte, jakožto ti, kteríž podle zákona svobody souzeni býti máte.
\par 13 Nebo odsouzení bez milosrdenství stane se tomu, kdož neciní milosrdenství, ale chlubít se milosrdenstvím proti odsudku.
\par 14 Co prospeje, bratrí moji, praví-li se kdo víru míti, a nemá-li skutku? Zdaliž jej ta víra muže spasiti?
\par 15 A kdyby bratr neb sestra neodení byli, a opuštení z strany každodenního pokrmu,
\par 16 Rekl by pak jim nekdo z vás: Jdete v pokoji a zhrejte se, a najezte se, avšak nedali byste jim potreby telesné, což to platno bude?
\par 17 Takž i víra, nemá-li skutku, mrtvát jest sama v sobe.
\par 18 Ale dí nekdo: Ty víru máš, a já mám skutky. Ukažiž ty mi víru svou z skutku svých, a ját tobe ukáži víru svou z skutku svých.
\par 19 Ty veríš, že jest jeden Buh. Dobre ciníš. I dáblovét tomu verí, avšak tresou se.
\par 20 Ale chceš-liž vedeti, ó clovece marný, že víra bez skutku jest mrtvá?
\par 21 Abraham otec náš zdali ne z skutku ospravedlnen jest, obetovav syna svého Izáka na oltár?
\par 22 Vidíš-li, že víra napomáhala skutkum jeho a z skutku víra dokonalá byla?
\par 23 A tak naplneno jest Písmo, rkoucí: I uveril Abraham Bohu, a pocteno jest jemu to za spravedlnost, a prítelem Božím nazván jest.
\par 24 Vidíte-liž tedy, že z skutku ospravedlnen bývá clovek a ne z víry toliko?
\par 25 Též podobne i Raab nevestka zdali ne z skutku ospravedlnena jest, prijavši posly a jinou cestou pryc je vypustivši?
\par 26 Nebo jakož telo bez duše jest mrtvé, takt i víra bez skutku jest mrtvá.

\chapter{3}

\par 1 Nebudtež mnozí mistri, bratrí moji, vedouce, že bychom težší odsouzení vzali.
\par 2 V mnohém zajisté klesáme všickni. Kdožt neklesá v slovu, tent jest dokonalý muž, mohoucí jako uzdou spravovati všecko telo.
\par 3 An my kone v uzdu pojímáme, aby nám povolni byli, a vším telem jejich vládneme.
\par 4 An i lodí tak veliké jsouce a prudkými vetry hnány bývajíce, však i nejmenším veslem bývají sem i tam obracíny, kamžkoli líbí se tomu, kdož je spravuje.
\par 5 Tak i jazyk malý úd jest, avšak veliké veci provodí. Aj, malický ohen, kterak veliký les zapálí!
\par 6 A jestit jazyk jako ohen a svet nepravosti. Takt jest, pravím, postaven jazyk mezi údy našimi, nanecištující celé telo, a rozpalující kolo narození našeho, jsa roznecován od ohne pekelného.
\par 7 Všeliké zajisté prirození i zveri, i ptactva, i hadu, i morských potvor bývá zkroceno, a jest okroceno od lidí;
\par 8 Ale jazyka žádný z lidí zkrotiti nemuže; tak jest nezkrotitelné zlé, pln jsa jedu smrtelného.
\par 9 Jím dobrorecíme Bohu Otci a jím zlorecíme lidem, ku podobenství Božímu stvoreným.
\par 10 Z jednech a týchž úst pochází dobrorecení i zlorecení. Ne takt býti má, bratrí moji.
\par 11 Zdaliž studnice jedním pramenem vydává sladkou i horkou vodu?
\par 12 Zdaliž muže, bratrí moji, fíkový strom nésti olivky, aneb vinný kmen fíky? Takt žádná studnice slané a sladké vody spolu vydávati nemuže.
\par 13 Kdo jest moudrý a umelý mezi vámi? Ukažiž dobrým obcováním skutky své v krotké moudrosti.
\par 14 Paklit máte mezi sebou horkou závist a zdráždení v srdci svém, nechlubte se a neklamejte proti pravde.
\par 15 Nenít zajisté ta moudrost shury sstupující, ale jest zemská, hovadná a dábelská.
\par 16 Nebo kdežt jest závist a rozdráždení, tu jest roztržka a všeliké dílo zlé.
\par 17 Ale moudrost, kteráž jest shury, nejprve zajisté jest cistotná, potom pokojná, mírná, povolná, plná milosrdenství a ovoce dobrého, bez rozsuzování a bez pokrytství.
\par 18 Ovoce pak spravedlnosti v pokoji rozsívá se tem, kteríž pokoj pusobí.

\chapter{4}

\par 1 Odkud pocházejí bojové a vády mezi vámi? Zdali ne odtud, totiž z libostí vašich, kteréž ryterují v údech vašich?
\par 2 Žádáte, a nemáte; závidíte sobe, a dychtíte po tom, což sobe zalibujete, a nemužete dosáhnouti; bojujete a válcíte, avšak toho, oc usilujete, nemáte, protože neprosíte.
\par 3 Prosíte, a nebérete, protože zle prosíte, abyste na své libosti vynakládali.
\par 4 Cizoložníci a cizoložnice, což nevíte, že prízen sveta jest neprítelkyne Boží? A protož kdo by koli chtel býti prítelem tohoto sveta, neprítelem Božím ucinen bývá.
\par 5 Což mníte, že nadarmo dí Písmo: Zdali k závisti naklonuje duch ten, kterýž prebývá v nás?
\par 6 Nýbrž hojnejší dává milost. Nebo dí: Buh se pyšným protiví, ale pokorným dává milost.
\par 7 Poddejtež se tedy Bohu, a zepretež se dáblu, i utecet od vás.
\par 8 Približte se k Bohu, a priblížít se k vám. Umejte ruce, hríšníci, a ocistte srdce vy, jenž jste dvojité mysli.
\par 9 Souženi budte, a kvelte, a placte; smích váš obratiž se v kvílení, a radost v zámutek.
\par 10 Ponižte se pred oblicejem Páne, a povýšít vás.
\par 11 Neutrhejtež jedni druhým, bratrí. Kdož utrhá bratru a soudí bratra svého, utrhá Zákonu a soudí Zákon. Soudíš-li pak Zákon, nejsi plnitel Zákona, ale soudce.
\par 12 Jedent jest vydavatel Zákona, kterýž muže spasiti i zatratiti. Ty kdo jsi, jenž soudíš jiného?
\par 13 Ale nuže vy, kteríž ríkáte: Dnes nebo zítra vypravíme se do onoho mesta, a pobudeme tam pres celý rok, a budeme kupciti, a neco zíšteme;
\par 14 (Ješto nevíte, co zítra bude. Nebo jakýt jest život váš? Pára zajisté jest, kteráž se na malicko ukáže, a potom zmizí.)
\par 15 Místo toho, co byste meli ríci: Bude-li Buh chtíti, a budeme-li živi, i uciníme toto nebo onono.
\par 16 Vy pak chlubíte se v pýše své. Všeliká taková chlouba zlá jest.
\par 17 A protož kdo umí dobre ciniti, a neciní, hrích má.

\chapter{5}

\par 1 Nuže nyní, boháci, placte, kvílíce nad bídami svými, kteréž prijdou.
\par 2 Zboží vaše shnilo a roucho vaše zmolovatelo.
\par 3 Zlato vaše a stríbro zerzavelo, a rez jejich bude na svedectví proti vám, a zžíret tela vaše jako ohen. Shromáždili jste poklad ku posledním dnum.
\par 4 Aj, mzda delníku, kteríž žali krajiny vaše, pri vás zadržaná, kricí, a hlas volání žencu v uši Pána zástupu vešel.
\par 5 Rozkoš jste provodili na zemi a zbujneli jste; vykrmili jste srdce vaše jakožto ke dni zabití.
\par 6 Odsoudili jste a zamordovali spravedlivého, a neodpíral vám.
\par 7 A protož trpeliví budte, bratrí, až do príchodu Páne. Aj, orác ocekává drahého užitku zemského, trpelive nan ocekávaje, až by prijal podzimní i jarní déšt.
\par 8 Budtež i vy trpeliví a potvrzujte srdcí vašich; nebot se približuje príští Páne.
\par 9 Nevzdychejtež k Bohu jedni proti druhým, bratrí, abyste nebyli odsouzeni. Aj, Soudce již prede dvermi stojí.
\par 10 Ku príkladu snášení protivenství a dlouhocekání, bratrí moji, vezmete proroky, kteríž mluvívali ve jménu Páne.
\par 11 Aj, blahoslavíme ty trpelivé. O trpelivosti Jobove slýchali jste, a dokonání Páne videli jste; nebo velmi jest milosrdný Pán a lítostivý.
\par 12 Prede všemi pak vecmi, bratrí moji, neprisahejte, ani skrze nebe, ani skrze zemi, ani kteroukoli jinou prísahou, ale bud rec vaše: Jiste, jiste, nikoli, nikoli, abyste neupadli v odsouzení.
\par 13 Jest-li kdo z vás zkormoucený? Modliž se. Pakli jest kdo mysli dobré? Prozpevuj Pánu.
\par 14 Stune-li kdo z vás? Zavolej starších sboru, a oni modltež se za nej, mažíce jej olejem ve jménu Páne.
\par 15 A modlitba víry uzdraví neduživého, a pozdvihnet ho Pán; a jestliže jest co prohrešil, budet jemu odpušteno.
\par 16 Vyznávejtež se pak jedni druhým z hríchu svých, a modlte se jedni za druhé, abyste uzdraveni byli. Mnohot zajisté muže modlitba spravedlivého opravdová.
\par 17 Eliáš clovek byl týmž bídám jako i my poddaný, a modlitbou modlil se, aby nepršelo, i nepršel déšt na zemi za tri léta a za šest mesícu.
\par 18 A zase modlil se, i vydalo nebe déšt, a zeme zplodila ovoce své.
\par 19 Bratrí, jestliže by kdo z vás pobloudil od pravdy, a nekdo by jej napravil,
\par 20 Veziž, že ten, kdož by odvrátil hríšníka od bludné cesty jeho, vysvobodí duši jeho od smrti a prikryje množství hríchu.


\end{document}