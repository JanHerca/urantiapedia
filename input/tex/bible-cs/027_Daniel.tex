\begin{document}

\title{Daniel}

\chapter{1}

\par 1 Léta tretího kralování Joakima krále Judského, pritáhl Nabuchodonozor král Babylonský k Jeruzalému, a oblehl jej.
\par 2 I vydal Pán v ruku jeho Joakima krále Judského, a neco nádobí domu Božího. Kterýž zavezl je do zeme Sinear, do domu boha svého, a nádobí to dal vnésti do domu pokladu boha svého.
\par 3 Rozkázal také král Ašpenazovi, správci dvoranu svých, aby privedl z synu Izraelských, z semene královského a z knížat,
\par 4 Mládence, na nichž by nebylo žádné poškvrny, a krásného oblíceje, a vtipné ke vší moudrosti, a schopné k umení i k nabývání jeho, a v kterýchž by byla síla, aby stávali na palácu královském, a ucili se liternímu umení a jazyku Kaldejskému.
\par 5 I narídil jim král odmerený pokrm na každý den z stolu královského, i vína, kteréž on sám pil, a aby je tak choval za tri léta, a po dokonání jich aby stávali pred králem.
\par 6 Byli pak mezi nimi z synu Juda: Daniel, Chananiáš, Mizael a Azariáš.
\par 7 I dal jim správce dvoranu jména. Nazval Daniele Baltazarem, Chananiáše pak Sidrachem, a Mizaele Mizachem, a Azariáše Abdenágem.
\par 8 Ale Daniel uložil v srdci svém, aby se nepoškvrnoval pokrmem z stolu královského, a vínem, kteréž král pil. Procež hledal toho u správce nad dvorany, aby se nemusil poškvrnovati.
\par 9 I zpusobil Buh Danielovi milost a lásku u správce nad dvorany.
\par 10 A rekl správce nad dvorany Danielovi: Já se bojím pána svého krále, kterýž vymeril pokrm váš a nápoj váš, tak že uzrel-li by, že tváre vaše opadlejší jsou, nežli mládencu tech, kteríž podobne jako i vy chování býti mají, zpusobíte mi to u krále, že prijdu o hrdlo.
\par 11 I rekl Daniel služebníku, kteréhož ustanovil správce dvoranu nad Danielem, Chananiášem, Mizaelem a Azariášem:
\par 12 Zkus, prosím, služebníku svých za deset dní, a necht se nám varení dává, kteréž bychom jedli, a voda, kterouž bychom pili.
\par 13 A potom necht se spatrí pred tebou tváre naše a tváre mládencu, kteríž jídají pokrm z stolu královského, a jakž uhlédáš, ucin s služebníky svými.
\par 14 I uposlechl jich v té veci, a zkusil jich za deset dní.
\par 15 Po skonání pak desíti dnu spatríno jest, že tváre jejich byly krásnejší, a byli tlustší na tele než všickni mládenci, kteríž jídali pokrm z stolu královského.
\par 16 Protož služebník brával ten vymerený pokrm jejich, a víno nápoje jejich, a dával jim varení.
\par 17 Mládence pak ty ctyri obdaril Buh povedomostí a rozumností ve všelikém literním umení a moudrostí; nadto Danielovi dal, aby rozumel všelikému videní a snum.
\par 18 A když se dokonali dnové, po kterýchž rozkázal král, aby je privedli, privedl je správce dvoranu pred Nabuchodonozora.
\par 19 I mluvil s nimi král. Ale není nalezen mezi všemi temi, jako Daniel, Chananiáš, Mizael a Azariáš. I stávali pred králem.
\par 20 A ve všelikém slovu moudrosti a rozumnosti, na kteréž se jich doptával král, nalezl je desetkrát zbehlejší nade všecky mudrce a hvezdáre, kteríž byli ve všem království jeho.
\par 21 I zustával tu Daniel až do léta prvního Cýra krále.

\chapter{2}

\par 1 Léta pak druhého kralování Nabuchodonozora mel Nabuchodonozor sen, a desil se duch jeho, až se tudy i ze sna protrhl.
\par 2 I rozkázal král svolati mudrce, a hvezdáre i kouzedlníky a Kaldejské, aby oznámili králi sen jeho. Kterížto prišli, a postavili se pred králem.
\par 3 Tedy rekl jim král: Mel jsem sen, a predesil se duch muj, tak že nevím, jaký to byl sen.
\par 4 I mluvili Kaldejští králi Syrsky: Králi, na veky bud živ. Povez sen služebníkum svým, a oznámímet výklad.
\par 5 Odpovedel král a rekl Kaldejským: Ten sen mi již z pameti vyšel. Neoznámíte-li mi snu i výkladu jeho, na kusy rozsekáni budete, a domové vaši v záchody obráceni budou.
\par 6 Pakli mi sen i výklad jeho oznámíte, daru, odplaty a slávy veliké dujdete ode mne. A protož sen i výklad jeho mi oznamte.
\par 7 Odpovedeli po druhé a rekli: Nechat král sen poví služebníkum svým, a výklad jeho oznámíme.
\par 8 Odpovedel král a rekl: Jistotne rozumím tomu, že naschvál odtahujete, vidouce, že mi vyšel z pameti ten sen.
\par 9 Neoznámíte-li mi toho snu, jistý jest ten úsudek o vás. Nebo rec lživou a chytrou smyslili jste sobe, abyste mluvili prede mnou, ažby se cas promenil. A protož sen mi povezte, a zvím, budete-li mi moci i výklad jeho oznámiti.
\par 10 Odpovedeli Kaldejští králi a rekli: Není cloveka na zemi, kterýž by tu vec králi oznámiti mohl. Nadto žádný král, kníže neb potentát takové veci se nedoptával na žádném mudrci a hvezdári aneb Kaldeovi.
\par 11 Nebo ta vec, na niž se král ptá, nesnadná jest, a není jiného, kdo by ji oznámiti mohl králi, krome bohu, kteríž bydlení s lidmi nemají.
\par 12 Z té príciný rozlítil se král a rozhneval velmi, a prikázal, aby zhubili všecky mudrce Babylonské.
\par 13 A když vyšel ortel, a mudrci mordováni byli, hledali i Daniele a tovaryšu jeho, aby zmordováni byli.
\par 14 Tedy Daniel odpovedel moudre a opatrne Ariochovi, hejtmanu nad žoldnéri královskými, kterýž vyšel, aby mordoval mudrce Babylonské.
\par 15 A odpovídaje, rekl Ariochovi, hejtmanu královskému: Proc ta výpoved náhle vyšla od krále? I oznámil tu vec Arioch Danielovi.
\par 16 Procež Daniel všed, prosil krále, aby jemu prodlel casu k oznámení výkladu toho králi.
\par 17 A odšed Daniel do domu svého, oznámil tu vec Chananiášovi, Mizaelovi a Azariášovi, tovaryšum svým,
\par 18 Aby se za milosrdenství modlili Bohu nebeskému prícinou té veci tajné, a nebyli zahubeni Daniel a tovaryši jeho s pozustalými mudrci Babylonskými.
\par 19 I zjevena jest Danielovi u videní nocním ta vec tajná. Procež Daniel dobrorecil Bohu nebeskému.
\par 20 Mluvil pak Daniel a rekl: Bud jméno Boží požehnáno od veku až na veky, nebo moudrost a síla jeho jest.
\par 21 A on promenuje casy i chvíle; ssazuje krále, i ustanovuje krále; dává moudrost moudrým a umení majícím rozum.
\par 22 On zjevuje veci hluboké a skryté; zná to, což jest v temnostech, a svetlo s ním prebývá.
\par 23 Ó Bože otcu mých, tet oslavuji a chválím, že jsi mne moudrostí a silou obdaril, ovšem nyní, že jsi mi oznámil to, cehož jsme žádali od tebe; nebo vec královu oznámil jsi nám.
\par 24 A protož Daniel všel k Ariochovi, kteréhož ustanovil král, aby zhubil mudrce Babylonské. A prišed, takto rekl jemu: Mudrcu Babylonských nezahlazuj; uved mne pred krále a výklad ten oznámím.
\par 25 Tedy Arioch s chvátáním uvedl Daniele pred krále, a takto rekl jemu: Našel jsem muže z zajatých synu Judských, kterýž výklad ten králi oznámí.
\par 26 Odpovedel král a rekl Danielovi, jehož jméno bylo Baltazar: Budeš-liž ty mi moci oznámiti sen, kterýž jsem videl, i výklad jeho?
\par 27 Odpovedel Daniel králi a rekl: Té veci tajné, na niž se král doptává, nikoli nemohou mudrci, hvezdári, veštci a hadaci králi oznámiti.
\par 28 Ale však jest Buh na nebi, kterýž zjevuje tajné veci, a on ukázal králi Nabuchodonozorovi, co se díti bude v potomních dnech. Sen tvuj, a což jsi ty videl na ložci svém, toto jest:
\par 29 Tobe, ó králi, na mysl pricházelo na ložci tvém, co bude potom, a ten, kterýž zjevuje tajné veci, ukázalt to, což budoucího jest.
\par 30 Strany pak mne, ne skrze moudrost, kteráž by pri mne byla nade všecky lidi, ta vec tajná mne zjevena jest, ale skrze modlitbu, aby ten výklad králi oznámen byl, a ty myšlení srdce svého abys zvedel.
\par 31 Ty králi, videl jsi, a aj, obraz nejaký veliký, (obraz ten byl znamenitý a blesk jeho náramný), stál naproti tobe, kterýž na pohledení byl hrozný.
\par 32 Toho obrazu hlava byla z zlata výborného, prsy jeho a ramena jeho z stríbra, a bricho jeho i bedra jeho z medi,
\par 33 Hnátové jeho z železa, nohy jeho z cástky z železa a z cástky z hliny.
\par 34 Hledels na to, až se utrhl kámen, kterýž nebýval v rukou, a uderil obraz ten v nohy jeho železné a hlinené, a potrel je.
\par 35 A tak potríno jest spolu železo, hlina, med, stríbro i zlato, a bylo to všecko jako plevy z placu letního, a zanesl to vítr, tak že místa jejich není nalezeno. Kámen pak ten, kterýž uderil v obraz, ucinen jest horou velikou, a naplnil všecku zemi.
\par 36 Tot jest ten sen. Výklad jeho také povíme králi:
\par 37 Ty králi, jsi král králu; nebo Buh nebeský dal tobe království, moc a sílu i slávu.
\par 38 A všeliké místo, na nemž prebývají synové lidští, zver polní i ptactvo nebeské dal v ruku tvou, a pánem te ustavil nade všemi temi vecmi. Ty jsi ta hlava zlatá.
\par 39 Ale po tobe povstane království jiné, nižší než tvé, a jiné království tretí medené, kteréž panovati bude po vší zemi.
\par 40 Království pak ctvrté bude tvrdé jako železo. Nebo jakož železo drobí a zemdlévá všecko, tak, pravím, jako železo, kteréž potírá všecko, i ono potre a potríská všecko.
\par 41 Že jsi pak videl nohy a prsty z cástky z hliny hrncírské a z cástky z železa, království rozdílné znamená, v nemž bude neco mocnosti železa, tak jakž jsi videl železo smíšené s hlinou nicemnou.
\par 42 Ale prstové noh z cástky z železa a z cástky z hliny znamenají království z cástky mocné a z cástky ku potrení snadné.
\par 43 A že jsi videl železo smíšené s hlinou nicemnou, ukazuje, že se prízniti budou vespolek lidé, a však nebude se prídržeti jeden druhého, tak jako železo nedrží se z hlinou.
\par 44 Za dnu pak tech králu vzbudí Buh nebeský království, kteréž na veky nebude zkaženo, a království to na žádného jiného nespadne, ale ono potre a konec uciní všechnem tem královstvím, samo pak státi bude na veky,
\par 45 Tak jakž jsi videl, že se s hory utrhl kámen, kterýž nebýval v rukou, a potrel železo, med, hlinu, stríbro a zlato. Buh veliký oznámil králi, co býti má potom, a pravý jest sen ten i verný výklad jeho.
\par 46 Tedy král Nabuchodonozor padl na tvár svou, a poklonil se Danielovi, a rozkázal, aby obeti a vune libé obetovali jemu.
\par 47 A odpovídaje král Danielovi, rekl: V pravde že Buh váš jest Buh bohu a Pán králu, kterýž zjevuje skryté veci, ponevadž jsi mohl vyjeviti tajnou vec tuto.
\par 48 Tedy král zvelebil Daniele, a dary veliké a mnohé dal jemu, a pánem ho ucinil nade vší krajinou Babylonskou, a knížetem nad vývodami, a nade všemi mudrci Babylonskými.
\par 49 Daniel pak vyžádal na králi, aby predstavil krajine Babylonské Sidracha, Mizacha a Abdenágo. Ale Daniel býval v bráne královské.

\chapter{3}

\par 1 Potom Nabuchodonozor král udelav obraz zlatý, jehož výška byla šedesáti loket, šírka pak šesti loket, postavil jej na poli Dura v krajine Babylonské.
\par 2 I poslal Nabuchodonozor král, aby shromáždili knížata, vývody a vudce, starší, správce nad poklady, v právích zbehlé, úredníky a všecky, kteríž panovali nad krajinami, aby prišli ku posvecování obrazu, kterýž postavil Nabuchodonozor král.
\par 3 Tedy shromáždili se knížata, vývodové a vudcové, starší, správcové nad poklady, v právích zbehlí, úredníci a všickni, kteríž panovali nad krajinami ku posvecování obrazu toho, kterýž postavil Nabuchodonozor král, a stáli pred obrazem, kterýž postavil Nabuchodonozor.
\par 4 Biric pak volal ze vší síly: Vám se to praví lidem, národum a jazykum,
\par 5 Jakž uslyšíte zvuk trouby, píštalky, citary, huslí, loutny, zpívání a všelijaké muziky, padnete a klanejte se obrazu zlatému, kterýž postavil Nabuchodonozor král.
\par 6 Kdož by pak nepadl a neklanel se, té hodiny uvržen bude do prostred peci ohnivé rozpálené.
\par 7 A protož hned, jakž uslyšeli všickni lidé zvuk trouby, píštalky, citary, huslí, loutny a všelijaké muziky, padli všickni lidé, národové a jazykové, klanejíce se obrazu zlatému, kterýž postavil Nabuchodonozor král.
\par 8 A hned téhož casu pristoupili muži Kaldejští, a s krikem žalovali na Židy,
\par 9 A mluvíce, rekli Nabuchodonozorovi králi: Králi, na veky bud živ.
\par 10 Ty králi, vynesls výpoved, aby každý clovek, kterýž by slyšel zvuk trouby, píštalky, citary, huslí, loutny, zpívání a všelijaké muziky, padl a klanel se obrazu zlatému,
\par 11 A kdož by nepadl a neklanel se, aby uvržen byl do prostred peci ohnivé rozpálené.
\par 12 Našli se pak nekterí Židé, kteréž jsi predstavil krajine Babylonské, totiž Sidrach, Mizach a Abdenágo, kterížto muži nedbali na tvé, ó králi, narízení. Bohu tvých nectí, a obrazu zlatému, kterýž jsi postavil, se neklanejí.
\par 13 Tedy Naduchodonozor v hneve a v prchlivosti rozkázal privésti Sidracha, Mizacha a Abdenágo. I privedeni jsou muži ti pred krále.
\par 14 I mluvil Nabuchodonozor a rekl jim: Zoumyslne-li, Sidrachu, Mizachu a Abdenágo, bohu mých nectíte, a obrazu zlatému, kterýž jsem postavil, se neklaníte?
\par 15 Protož nyní, jste-liž hotovi, abyste hned, jakž uslyšíte zvuk trouby, píštalky, citary, huslí, loutny, zpívání a všelijaké muziky, padli a klaneli se obrazu tomu, kterýž jsem ucinil? Pakli se klaneti nebudete, té hodiny uvrženi budete do prostred peci ohnivé rozpálené, a který jest ten Buh, ješto by vás vytrhl z ruky mé?
\par 16 Odpovedeli Sidrach, Mizach a Abdenágo, a rekli králi: My se nestaráme o to, ó Nabuchodonozore, co bychom meli odpovedíti tobe.
\par 17 Nebo aj, budto že Buh, jehož my ctíme, (kterýž mocen jest vytrhnouti nás z peci ohnivé rozpálené, a tak z ruky tvé, ó králi), vytrhne nás.
\par 18 Bud že nevytrhne, známo bud tobe, ó králi, žet bohu tvých ctíti a obrazu zlatému, kterýž jsi postavil, klaneti se nebudeme.
\par 19 Tedy Nabuchodonozor naplnen jsa prchlivostí, tak že oblícej tvári jeho se promenil proti Sidrachovi, Mizachovi a Abdenágovi, a odpovídaje, rozkázal rozpáliti pec sedmkrát více, než obycej meli ji rozpalovati.
\par 20 A mužum silným, kteríž byli mezi rytíri jeho, rozkázal, aby svížíce Sidracha, Mizacha a Abdenágo, uvrhli do peci ohnivé rozpálené.
\par 21 Tedy svázali muže ty v pláštích jejich, v košilkách jejich, i v kloboucích jejich a v odevu jejich, a uvrhli je do prostred peci ohnivé rozpálené.
\par 22 Že pak rozkaz královský náhlý byl, a pec velmi rozpálená, z té príciny muže ty, kteríž uvrhli Sidracha, Mizacha a Abdenágo, zadusil plamen ohne.
\par 23 Ale ti tri muži, Sidrach, Mizach a Abdenágo, padli do prostred peci ohnivé rozpálené svázaní.
\par 24 Tedy Nabuchodonozor král zdesil se, a vstal s chvátáním, a promluviv, rekl hejtmanum svým: Zdaliž jsme neuvrhli trí mužu do prostred peci svázaných? Odpovedeli a rekli králi: Pravda jest, králi.
\par 25 On pak odpovídaje, rekl: Aj, vidím ctyri muže rozvázané, procházející se u prostred ohne, a není žádného porušení pri nich, a ctvrtý na pohledení podobný jest synu Božímu.
\par 26 A pristoupiv Nabuchodonozor k celisti peci ohnivé rozpálené, mluvil a rekl: Sidrachu, Mizachu a Abdenágo, služebníci Boha nejvyššího, vyjdete a podte sem. I vyšli Sidrach, Mizach a Abdenágo z prostredku ohne.
\par 27 Shromáždivše se pak knížata, vývodové a vudcové a hejtmané královští, hledeli na ty muže, an žádné moci nemel ohen pri telích jejich, ani vlas hlavy jejich nepriškvrkl, ani pláštové jejich se nezmenili, aniž co ohnem páchli.
\par 28 I mluvil Nabuchodonozor a rekl: Požehnaný Buh jejich, totiž Sidrachuv, Mizachuv a Abdenáguv, kterýž poslal andela svého, a vytrhl služebníky své, kteríž doufali v neho, až i rozkazu královského neuposlechli, ale tela svá vydali, aby nesloužili a neklaneli se žádnému bohu, krome Bohu svému.
\par 29 A protož toto já prikazuji, aby každý ze všelikého lidu, národu a jazyku, kdož by koli co rouhavého rekl proti Bohu Sidrachovu, Mizachovu a Abdenágovu, na kusy rozsekán byl, a dum jeho v záchod obrácen, proto že není Boha jiného, kterýž by mohl vytrhovati, jako tento.
\par 30 Tedy zvelebil zase král Sidracha, Mizacha a Abdenága v krajine Babylonské.
\par 31 Nabuchodonozor král všechnem lidem, národum i jazykum, kteríž bydlí na vší zemi: Pokoj váš rozmnožen bud.
\par 32 Znamení a divy, kteréž ucinil pri mne Buh nejvyšší, videlo mi se za slušné, abych vypravoval.
\par 33 Znamení jeho jak veliká jsou, a divové jeho jak mocní jsou, království jeho království vecné, a panování jeho od národu do pronárodu.

\chapter{4}

\par 1 Já Nabuchodonozor, užívaje pokoje v dome svém, a kveta na palácu svém,
\par 2 Mel jsem sen, kterýž mne zhrozil, a myšlení na ložci svém, a to, což jsem videl, znepokojilo mne.
\par 3 A protož vyšla ode mne ta výpoved, aby uvedeni byli prede mne mudrci Babylonští, kteríž by mi výklad toho snu oznámili.
\par 4 Tedy predstoupili mudrci, hvezdári, Kaldejští a hadaci. I povedel jsem jim sen, a však výkladu jeho nemohli mi oznámiti.
\par 5 Až naposledy predstoupil prede mne Daniel, jehož jméno Baltazar, podlé jména boha mého, a v nemž jest duch bohu svatých. Jemuž jsem oznámil sen:
\par 6 Baltazare, kníže mudrcu, já vím, že duch bohu svatých jest v tobe, a nic tajného není tobe nesnadného. Videní snu mého, kterýž jsem mel, i výklad jeho oznam.
\par 7 U videní pak, kteréž jsem videl na ložci svém, videl jsem, a aj, strom u prostred zeme, jehož vysokost byla veliká.
\par 8 Veliký byl strom ten a mocný, a výsost jeho dosahovala až k nebi, a patrný byl až do koncin vší zeme.
\par 9 Lístí jeho bylo pekné, a ovoce jeho hojné, všechnem za pokrm, pod stínem pak jeho byla zver polní, a na ratolestech jeho bydlili ptáci nebeští, a z neho potravu mel všeliký živocich.
\par 10 Videl jsem také u videních svých na ložci svém, a aj, hlásný a svatý s nebe sstoupiv,
\par 11 Volal ze vší síly, a tak pravil: Podetnete strom ten, a osekejte ratolesti jeho, a otlucte lístí jeho, a rozmecte ovoce jeho; nechat se vzdálí zver od neho, a ptáci z ratolestí jeho.
\par 12 A však kmene korenu jeho v zemi zanechejte, a v poutech železných a ocelivých na tráve polní, aby rosou nebeskou smácín byl, a díl jeho s zverí v byline zemské.
\par 13 Srdce jeho od lidského at jest rozdílné, a srdce zvírecí necht jest dáno jemu, ažby sedm let vyplnilo se pri nem.
\par 14 Usouzení hlásných a rec žádosti svatých stane se, ažby k tomu prišlo, aby poznali lidé, že Nejvyšší panuje nad královstvím lidským, a že komuž chce, dává je, a toho, kterýž jest ponížený mezi lidmi, ustanovuje nad ním.
\par 15 Tento sen videl jsem já Nabuchodonozor král, ty pak, Baltazare, oznam výklad jeho. Nebo všickni mudrci v království mém nemohli mi výkladu oznámiti, ale ty mužeš, proto že duch bohu svatých jest v tobe.
\par 16 Tedy Daniel, jemuž jméno Baltazar, predešený stál za jednu hodinu, a myšlení jeho strašila ho. A odpovedev král, rekl: Baltazare, sen ani výklad jeho necht nestraší tebe. Odpovedel Baltazar a rekl: Pane muj, sen tento prid na ty, kteríž te v nenávisti mají, a výklad jeho na tvé neprátely.
\par 17 Strom ten, kterýž jsi videl, veliký a mocný, jehož výsost dosahovala až k nebi, a kterýž patrný byl po vší zemi.
\par 18 A lístí jeho pekné, a ovoce jeho hojné, a z nehož všickni pokrm meli, pod nímž byla zver polní, a na ratolestech jeho bydlili ptáci nebeští,
\par 19 Ty jsi ten, ó králi, kterýž jsi rozšíril se a zmocnil, a velikost tvá vzrostla a vznesla se až k nebi, a panování tvé až do konce zeme.
\par 20 Že pak videl král hlásného a svatého sstupujícího s nebe, kterýž rekl: Podetnete strom ten a zkazte jej, a však kmene korenu jeho v zemi zanechejte, a v okovách železných a ocelivých at jest na tráve polní, aby rosou nebeskou smácín byl, a s zverí polní díl jeho, ažby sedm let vyplnilo se pri nem:
\par 21 Tent jest výklad, ó králi, a ortel Nejvyššího, kterýž vyšel na pána mého krále.
\par 22 Nebo zaženou te lidé od sebe, a s zverí polní bude bydlení tvé, a bylinu jako volum tobe jísti dávati budou, a rosou nebeskou smácín budeš, až se vyplní sedm let pri tobe, dokudž bys nepoznal, že panuje Nejvyšší nad královstvím lidským, a že komuž chce, dává je.
\par 23 Že pak rekli, aby zanechán byl kmen korenu toho stromu, království tvé tobe zustane, jakž jen poznáš, že nebesa panují.
\par 24 Protož ó králi, prijmi radu mou, a hríchy své spravedlností pretrhuj, a nepravosti své milostivostí k ssouženým, zdaby prodloužen byl pokoj tvuj.
\par 25 Všecko to prišlo na krále Nabuchodonozora.
\par 26 Nebo po dokonání dvanácti mesícu, procházeje se po palácu královském v Babylone,
\par 27 Mluvil král a rekl: Zdaliž toto není ten Babylon veliký, kterýž jsem já vystavel mocí síly své, aby byl stolicí království k ozdobe slávy mé?
\par 28 Ješte ta rec byla v ústech krále, a aj, hlas s nebe prišel: Tobet se praví, Nabuchodonozore králi, že království odešlo od tebe,
\par 29 Nýbrž te lidé i z sebe vyvrhou, a s zverí polní bydliti budeš. Bylinu jako volum tobe jísti dávati budou, ažby sedm let vyplnilo se pri tobe, dokudž bys nepoznal, že panuje Nejvyšší nad královstvím lidským, a že komuž chce, dává je.
\par 30 V touž hodinu rec ta naplnila se pri Nabuchodonozorovi. Nebo z spolku lidí vyvržen byl, a bylinu jako vul jedl, a rosou nebeskou telo jeho smácíno bylo, až na nem vlasy zrostly jako perí orlicí, a nehty jeho jako pazoury ptací.
\par 31 Pri skonání pak tech dnu já Nabuchodonozor pozdvihl jsem ocí svých k nebi, a rozum muj ke mne se zase navrátil. I dobrorecil jsem Nejvyššímu, a živého na veky chválil jsem a oslavoval; nebo panování jeho jest panování vecné, a království jeho od národu do pronárodu.
\par 32 A všickni obyvatelé zeme jako za nic pocteni jsou, a podlé vule své ciní mezi vojskem nebeským i obyvateli zeme, aniž jest kdo, ješto by mu pres ruku dáti mohl, a ríci jemu: Co to deláš?
\par 33 Téhož casu rozum muj navrátil se ke mne, a k sláve království mého ozdoba má, i dustojnost má navrátila se ke mne; nadto i hejtmané moji a knížata má hledali mne, a zmocnen jsem v království svém, a velebnost vetší jest mi pridána.
\par 34 Nyní tedy já Nabuchodonozor chválím, vyvyšuji a oslavuji krále nebeského, jehož všickni skutkové jsou pravda, a stezky jeho soud, a kterýž chodící v pýše muže snižovati.

\chapter{5}

\par 1 Balsazar král ucinil hody veliké tisíci knížatum svým, a pred nimi víno pil.
\par 2 A když pil víno Balsazar, rozkázal prinésti nádobí zlaté a stríbrné, kteréž vynesl Nabuchodonozor otec jeho z chrámu Jeruzalémského, aby z neho pili král i knížata jeho, ženy jeho i ženiny jeho.
\par 3 I prineseno jest nádobí zlaté, kteréž vynesli z chrámu domu Božího, kterýž byl v Jeruzaléme, a pili z neho král i knížata jeho, ženy jeho i ženiny jeho.
\par 4 Pili víno, a chválili bohy zlaté a stríbrné, medené, železné, drevené a kamenné.
\par 5 V touž hodinu vyšli prstové ruky lidské, a psali naproti svícnu na stene paláce královského, a král hledel na cástky ruky, kteráž psala.
\par 6 Tedy jasnost královská zmenila se, a myšlení jeho zkormoutila ho, a pasové bedr jeho rozpásali se, i kolena jeho jedno o druhé se tlouklo.
\par 7 A zkrikl král ze vší síly, aby privedeni byli hvezdári, Kaldejští a hadaci. I mluvil král a rekl mudrcum Babylonským: Kdokoli precte psání toto, a výklad jeho mi oznámí, šarlatem odín bude, a retez zlatý na hrdlo jeho, a tretím v království po mne bude.
\par 8 I predstoupili všickni mudrci královští, ale nemohli písma toho císti, ani výkladu oznámiti králi.
\par 9 Procež král Balsazar velmi predešen byl, a jasnost jeho zmenila se na nem, ano i knížata jeho zkormouceni byli.
\par 10 Královna pak, prícinou té veci královské a knížat jeho, do domu tech hodu vešla, a promluvivši královna, rekla: Králi, na veky živ bud. Necht te nedesí myšlení tvá, a jasnost tvá necht se nemení.
\par 11 Jest muž v království tvém, v nemž jest duch bohu svatých, v kterémž za dnu otce tvého osvícení, rozumnost a moudrost, jako moudrost bohu, nalezena, jehož král Nabuchodonozor otec tvuj knížetem mudrcu, hvezdáru, Kaldejských a hadacu ustanovil, otec tvuj, ó králi,
\par 12 Proto že duch znamenitý, i umení a rozumnost vykládání snu a oznámení pohádek, i rozvázání vecí nesnadných nalezeno pri Danielovi, jemuž král jméno dal Baltazar. Nechat nyní zavolán jest Daniel, a oznámít výklad ten.
\par 13 Tedy priveden jest Daniel pred krále. I mluvil král a rekl Danielovi: Ty-li jsi ten Daniel, jeden z synu zajatých Judských, kteréhož privedl král otec muj z Judstva?
\par 14 Slyšel jsem zajisté o tobe, že duch bohu svatých jest v tobe, a osvícení i rozumnost a moudrost znamenitá nalezena jest v tobe.
\par 15 A nyní privedeni jsou prede mne mudrci a hvezdári, aby mi písmo toto prectli, a výklad jeho oznámili, a však nemohli výkladu veci té oznámiti.
\par 16 Já pak slyšel jsem o tobe, že mužeš to, což jest nesrozumitelného, vykládati, a což nesnadného, rozvázati. Protož nyní, budeš-li moci písmo to precísti, a výklad jeho mne oznámiti, v šarlat oblecen budeš, a retez zlatý na hrdlo tvé, a tretím v království po mne budeš.
\par 17 Tedy odpovedel Daniel a rekl pred králem: Darové tvoji necht zustávají tobe, a odplatu svou dej jinému, a však písmo prectu králi, a výklad oznámím jemu.
\par 18 Ty králi, slyš: Buh nejvyšší královstvím a dustojností i slávou a okrasou obdaril Nabuchodonozora otce tvého,
\par 19 A pro dustojnost, kterouž ho obdaril, všickni lidé, národové a jazykové trásli a báli se pred ním. Kohokoli chtel, zabil, a kterékoli chtel, bil, kteréž chtel, povyšoval, a kteréž chtel, ponižoval.
\par 20 Když se pak bylo pozdvihlo srdce jeho, a duch jeho zmocnil se v pýše, ssazen byl z stolice království svého, a slávu odjali od neho.
\par 21 Ano i z spolku synu lidských vyvržen byl, a srdce jeho zvírecímu podobné ucineno bylo, a s divokými osly bylo bydlení jeho. Bylinu jako volum dávali jemu jísti, a rosou nebeskou telo jeho smácíno bylo, dokudž nepoznal, že panuje Buh nejvyšší nad královstvím lidským, a že kohož chce, ustanovuje nad ním.
\par 22 Ty také, synu jeho Balsazare, neponížil jsi srdce svého, ackolis o tom o všem vedel.
\par 23 Ale pozdvihls se proti Pánu nebes; nebo nádobí domu jeho prinesli pred tebe, a ty i knížata tvá, ženy tvé i ženiny tvé pili jste víno z neho. Nadto bohy stríbrné a zlaté, medené, železné, drevené a kamenné, kteríž nevidí, ani slyší, aniž co vedí, chválil jsi, Boha pak, v jehož ruce jest dýchání tvé i všecky cesty tvé, neoslavoval jsi.
\par 24 Protož nyní od neho poslána jest cástka ruky této, a písmo to napsáno jest.
\par 25 A totot jest písmo napsané: Mene, mene, tekel, ufarsin, totiž: Zcetl jsem, zcetl, zvážil a rozdeluji.
\par 26 Tento pak jest výklad slov: Mene, zcetl Buh království tvé, a k konci je privedl.
\par 27 Tekel, zvážen jsi na váze, a nalezen jsi lehký.
\par 28 Peres, rozdeleno jest království tvé, a dáno jest Médským a Perským.
\par 29 Tedy z rozkazu Balsazarova oblékli Daniele v šarlat, a retez zlatý dali na hrdlo jeho, a rozhlašovali o nem, že má býti pánem tretím v království.
\par 30 V touž noc zabit jest Balsazar král Kaldejský.
\par 31 Darius pak Médský ujal království v letech okolo šedesáti a dvou.

\chapter{6}

\par 1 Líbilo se pak Dariovi, aby ustanovil nad královstvím úredníku sto a dvadceti, kteríž by byli po všem království.
\par 2 Nad temi pak hejtmany tri, z nichžto Daniel prední byl, kterýmž by úredníci onino vydávali pocet, aby se králi škoda nedála.
\par 3 Tedy Daniel prevyšoval ty hejtmany a úredníky, proto že duch znamenitejší v nem byl. Procež král myslil ustanoviti jej nade vším královstvím.
\par 4 Tedy hejtmané a úredníci hledali príciny proti Danielovi s strany království, a však žádné príciny ani vady nemohli najíti; nebo verný byl, aniž jaký omyl neb vada nalézala se pri nem.
\par 5 Protož muži ti rekli: Nenajdeme proti Danielovi tomuto žádné príciny, jediné lec bychom našli neco proti nemu s strany zákona Boha jeho.
\par 6 Tedy hejtmané a úredníci ti shromáždivše se k králi, takto mluvili k nemu: Darie králi, na veky bud živ.
\par 7 Uradili se všickni hejtmané království, vývodové, úredníci, správcové a vudcové, abys ustanovil narízení královské, a utvrdil zápoved: Kdož by koli vložil žádost na kteréhokoli boha neb cloveka do tridcíti dnu, krome na tebe, králi, aby uvržen byl do jámy lvové.
\par 8 Nyní tedy, ó králi, potvrd zápovedi této, a vydej mandát, kterýž by nemohl zmenen býti podlé práva Médského a Perského, kteréž jest nepromenitelné.
\par 9 Procež král Darius vydal mandát a zápoved.
\par 10 Daniel pak, když se dovedel, že jest vydán mandát, všel do domu svého, kdež otevrená byla okna v pokoji jeho proti Jeruzalému, a trikrát za den klekal na kolena svá, a modlíval se a vyznával se Bohu svému, tak jakož prvé to ciníval.
\par 11 Tedy muži ti shromáždivše se a nalezše Daniele, an se modlí a pokorne prosí Boha svého,
\par 12 Tedy pristoupili a mluvili k králi o zápovedi královské: Zdaliž jsi nevydal mandátu, aby každý clovek, kdož by koli neceho žádal od kterého boha neb cloveka až do tridcíti dnu, krome od tebe, králi, uvržen byl do jámy lvové? Odpovedev král, rekl: Pravét jest slovo to, podlé práva Médského a Perského, kteréž jest nepromenitelné.
\par 13 Tedy odpovídajíce, rekli králi: Daniel ten, kterýž jest z zajatých synu Judských, nechtel dbáti na tvé, ó králi, narízení, ani na mandát tvuj, kterýž jsi vydal, ale trikrát za den modlívá se modlitbou svou.
\par 14 Tedy král, jakž uslyšel tu rec, velmi se zarmoutil nad tím, a uložil král v mysli své vysvoboditi Daniele, a až do západu slunce usiloval ho vytrhnouti.
\par 15 Ale muži ti shromáždivše se k králi, mluvili jemu: Vez, králi, že jest takové právo u Médských a Perských, aby každá výpoved a narízení, kteréž by král ustanovil, nepromenitelné bylo.
\par 16 I rekl král, aby privedli Daniele, a uvrhli jej do jámy lvové. Mluvil pak král a rekl Danielovi: Buh tvuj, kterémuž sloužíš ustavicne, on vysvobodí tebe.
\par 17 A prinesen jest kámen jeden, a položen na díru té jámy, a zapecetil ji král prstenem svým a prsteny knížat svých, aby nebyl zmenen ortel pri Danielovi.
\par 18 I odšel král na palác svuj, a šel ležeti, nic nejeda, a nicímž se obveseliti nedal, tak že i sen jeho vzdálen byl od neho.
\par 19 Tedy král hned ráno vstav na úsvite, s chvátáním šel k jáme lvové.
\par 20 A jakž se priblížil k jáme, hlasem žalostným zavolal na Daniele, a promluviv král, rekl Danielovi: Danieli, služebníce Boha živého, Buh tvuj, kterémuž ty sloužíš ustavicne, mohl-liž te vysvoboditi od lvu?
\par 21 Tedy Daniel mluvil s králem, rka: Králi, na veky bud živ.
\par 22 Buh muj poslal andela svého, kterýž zavrel ústa lvu, aby mi neuškodili; nebo pred ním nevina nalezena jest pri mne, nýbrž ani proti tobe, králi, nic zlého jsem neucinil.
\par 23 Tedy král velmi se z toho zradoval, a rozkázal Daniele vytáhnouti z jámy. I vytažen byl Daniel z jámy, a žádného úrazu není nalezeno na nem; nebo veril v Boha svého.
\par 24 I rozkázal král, aby privedeni byli muži ti, kteríž osocili Daniele, a uvrženi jsou do jámy lvové, oni i synové jejich i ženy jejich, a prvé než dopadli dna té jámy, zmocnili se jich lvové, a všecky kosti jejich zetreli.
\par 25 Tedy Darius král napsal všechnem lidem, národum a jazykum, kteríž bydlili po vší zemi: Pokoj váš rozmnožen bud.
\par 26 Ode mne vyšlo narízení toto, aby na všem panství království mého trásli a báli se pred Bohem Danielovým; nebo on jest Buh živý a zustávající na veky, a království jeho nebude zrušeno, ani panování jeho až do konce.
\par 27 Vysvobozuje a vytrhuje, a ciní znamení a divy na nebi i na zemi, kterýž vysvobodil Daniele z moci lvu.
\par 28 Danielovi pak štastne se vedlo v království Dariovu, a v království Cýra Perského.

\chapter{7}

\par 1 Léta prvního Balsazara krále Babylonského Daniel mel sen a videní svá na ložci svém, i napsal ten sen krátkými slovy.
\par 2 Mluvil Daniel a rekl? Videl jsem u videní svém v noci, a aj, ctyri vetrové nebeští bojovali na mori velikém.
\par 3 A ctyri šelmy veliké vystupovaly z more, jedna od druhé rozdílná.
\par 4 První podobná lvu, a krídla orlicí mela. Hledel jsem, až vytrhána byla krídla její, jimiž se vznášela od zeme, tak že na nohách jako clovek státi musila, a srdce lidské dáno jest jí.
\par 5 A aj, jiná šelma druhá podobná nedvedu, kteráž panství jedno vyzdvihla, a tri žebra v ústech jejích, mezi zuby jejími, a tak mluveno bylo k ní: Vstan, nažer se hojne masa.
\par 6 Potom jsem videl, a aj, jiná podobná pardovi, kteráž mela ctyri krídla ptací na hrbete svém, a ctyrhlavá byla šelma ta, jíž moc dána byla.
\par 7 Potom videl jsem u videních nocních, a aj, šelma ctvrtá strašlivá a hrozná a velmi silná, mající zuby železné veliké, kteráž zžírala a potírala, ostatek pak nohama svýma pošlapávala; a ta byla rozdílná ode všech šelm, kteréž byly pred ní, a mela rohu deset.
\par 8 Pilne jsem šetril tech rohu, a hle, roh poslední malý vyrostal mezi nimi, a tri z tech rohu prvních vyvráceni jsou pred ním; a aj, oci podobné ocím lidským v rohu tom, a ústa mluvící pyšne.
\par 9 Hledel jsem, až trunové ti svrženi byli, a Starý dnu posadil se, jehož roucho jako sníh bílé, a vlasové hlavy jeho jako vlna cistá, trun jako jiskry ohne, kola jeho jako ohen horící.
\par 10 Potok ohnivý tekl a vycházel od neho, tisícové tisícu sloužili jemu, a desetkrát tisíckrát sto tisícu stálo pred ním; soud zasedl, a knihy otevríny byly.
\par 11 Patril jsem tehdáž, hned jakž se zacal zvuk té reci pyšné, kterouž roh mluvil; patril jsem, dokudž ta šelma nebyla zabita, a vyhlazeno telo její, a dáno k spálení ohni.
\par 12 A i ostatkum šelm odjali panství; nebo dlouhost života jim odmerena byla až do casu, a to uloženého casu.
\par 13 Videl jsem u videní nocním, a aj, s oblaky nebeskými podobný Synu cloveka pricházel; potom až k Starému dnu prišel, a pred nej postaven byl.
\par 14 I dáno jest jemu panství a sláva i království, aby všickni lidé, národové a jazykové sloužili jemu; jehož panství jest panství vecné, kteréž nepomíjí, a království jeho, kteréž se neruší.
\par 15 I zhrozil se duch muj ve mne Danielovi u prostred tela, a videní má predesila mne.
\par 16 Tedy pristoupil jsem k jednomu z prístojících, a ptal jsem se ho na jistotu vší té veci. I povedel mi, a výklad recí mi oznámil:
\par 17 Ty šelmy veliké, kteréž jsou ctyry, jsou ctyri králové, kteríž povstanou z zeme,
\par 18 A ujmou království svatých výsostí, kteríž obdržeti mají království až na veky, a až na veky veku.
\par 19 Tedy žádostiv jsem byl zprávy o šelme ctvrté, kteráž rozdílná byla ode všech jiných, hrozná velmi; zubové její železní, a pazoury její ocelivé, kteráž zžírala, potírala, ostatek pak nohama svýma pošlapávala.
\par 20 Tolikéž o rozích desíti, kteríž byli na hlave její, a o posledním, kterýž vyrostl, a pred ním spadli tri; o tom rohu, pravím, kterýž mel oci a ústa mluvící pyšne, a byl na pohledení vetší než jiní.
\par 21 Videl jsem, an roh ten válku vedl s svatými, a premáhal je,
\par 22 Až prišel Starý dnu, a oddán jest soud svatým výsostí, a cas prišel, aby to království svatí obdrželi.
\par 23 Rekl takto: Šelma ctvrtá znamená království ctvrté na zemi, kteréž rozdílné bude ode všech království, a zžíre všecku zemi, a zmlátí ji a potre ji.
\par 24 Rohu pak deset znamená, že z království toho deset králu povstane, a poslední povstane po nich, kterýž bude rozdílný od prvních, a poníží trí králu.
\par 25 A slova proti Nejvyššímu mluviti bude, a svaté výsostí potre; nadto pomýšleti bude, aby promenil casy i práva, když vydáni budou v ruku jeho, až do casu a casu, i do cástky casu.
\par 26 V tom bude soud osazen, a panství jeho odejmou, vypléní a vyhladí je docela.
\par 27 Království pak i panství, a dustojnost královská pode vším nebem dána bude lidu svatých výsostí; jehož království bude království vecné, a všickni páni jemu sloužiti a jeho poslouchati budou.
\par 28 Až potud konec té reci. Mne pak Daniele myšlení má velice zkormoutila, a krása má promenila se pri mne, slovo však toto v srdci svém zachoval jsem.

\chapter{8}

\par 1 Léta tretího kralování Balsazara krále ukázalo mi se videní, mne Danielovi po onom, kteréž se mi ukázalo na pocátku.
\par 2 I videl jsem u videní, (tehdáž pak, když jsem videl, byl jsem v Susan, na hrade, kterýž jest v krajine Elam), videl jsem, pravím, byv u potoka Ulai.
\par 3 A pozdvih ocí svých, videl jsem, a aj, u toho potoka stál skopec jeden, kterýž mel dva rohy. A ti dva rohové byli vysocí, a však jeden vyšší než druhý, ale ten vyšší zrostl posléze.
\par 4 Videl jsem skopce toho, an trkal k západu, pulnoci a poledni, jemuž žádná šelma odolati nemohla, aniž kdo co mohl vytrhnouti z moci jeho; procež cinil podlé vule své, a to veci veliké.
\par 5 A když jsem to rozvažoval, aj, kozel pricházel od západu na svrchek vší zeme, a žádný se ho nedotýkal na zemi, a ten kozel mel roh znamenitý mezi ocima svýma.
\par 6 A prišel až k tomu skopci majícímu dva rohy, kteréhož jsem byl videl stojícího u potoka, a pribehl k nemu v prchlivosti síly své.
\par 7 Videl jsem také, an dotrel na toho skopce, a rozlítiv se proti nemu, uderil jej, tak že zlámal oba rohy jeho, a nebylo síly v skopci k odpírání jemu. A poraziv ho na zemi, pošlapal jej, aniž byl, kdo by vytrhl skopce z moci jeho.
\par 8 Kozel pak velikým ucinen jest velmi. A když se ssilil, zlámal se roh ten veliký, i zrostli znamenití ctyri místo neho, na ctyri strany sveta.
\par 9 Z tech pak jednoho vyšel roh jeden malický, a zrostl velmi ku poledni a východu, a k zemi Judské.
\par 10 A zpjal se až k vojsku nebeskému, a svrhl na zemi nekteré z vojska toho i z hvezd, a pošlapal je.
\par 11 Anobrž až k vojska toho knížeti zpjal se, nebo od neho zastavena byla ustavicná obet, a zavržen príbytek svatyne Boží,
\par 12 Tak že vojsko to vydáno v prevrácenost proti ustavicné obeti, a povrhlo pravdu na zemi, a což cinilo, štastne mu se darilo.
\par 13 Tedy slyšel jsem jednoho svatého mluvícího, a rekl ten svatý tomu, kterýž tajné veci v poctu maje, mluví: Dokudž toto videní o obeti ustavicné, a prevrácenost na zpuštení privodící trvati bude, a svaté služby vydávány budou i vojsko v pošlapání?
\par 14 A rekli mi: Až do dvou tisíc a trí set veceru a jiter, a prijdou k obnovení svému svaté služby.
\par 15 Stalo se pak, že když jsem já Daniel hledel na to videní, a ptal jsem se na rozum jeho, aj, postavil se podlé mne na pohledení jako muž.
\par 16 Slyšel jsem také hlas lidský mezi Ulaiem, kterýžto zavolav, rekl: Gabrieli, vylož tomuto videní to.
\par 17 I prišel ke mne, kdež jsem stál, a když prišel, zhrozil jsem se, a padl jsem na tvár svou. I rekl ke mne: Pozoruj, synu clovecí; nebo v casu uloženém videní toto se naplní.
\par 18 Když pak on mluvil se mnou, usnul jsem tvrde, leže tvárí svou na zemi. I dotekl se mne, a postavil mne tu, kdež jsem byl stál,
\par 19 A rekl: Aj, já oznámím tobe to, což se díti bude až do vykonání hnevu toho; nebo v uloženém casu konec bude.
\par 20 Skopec ten, kteréhož jsi videl, an mel dva rohy, jsou králové Médský a Perský.
\par 21 Kozel pak ten chlupatý jest král Recký, a roh ten veliký, kterýž jest mezi ocima jeho, jest král první.
\par 22 Že pak zlámán jest, a povstali ctyri místo neho, ctvero království z svého národu povstane, ale ne s takovou silou.
\par 23 Pri dokonání pak království jejich, když na vrch vzejdou nešlechetníci, povstane král nestydatý a chytrý.
\par 24 Jehož síla zmocní se, ackoli ne jeho silou, tak že ku podivení hubiti bude, a štastne se mu povede, až i vše vykoná; nebo hubiti bude silné i lid svatý.
\par 25 A obmyslností svou štastne svede lest v predsevzetí svém, a v srdci svém zvelebí sebe, a v cas pokoje zhubí mnohé; nadto i proti knížeti knížat se postaví, a však bez rukou potrín bude.
\par 26 Videní pak to vecerní a jitrní, o nemž povedíno, jest jistá pravda; procež ty zavri to videní, nebo jest mnohých dnu.
\par 27 Tedy já Daniel zchuravel jsem, a nemocen jsem byl nekolik dnu. Potom povstav, konal jsem povinnost od krále porucenou, byv predešen nad tím videním, cehož však žádný na mne neseznal.

\chapter{9}

\par 1 Léta prvního Daria syna Asverova z semene Médského, kteréhož ucinen jest králem v království Kaldejském,
\par 2 Léta prvního kralování jeho já Daniel porozumel jsem z knih poctu let, o nichž se stalo slovo Hospodinovo k Jeremiášovi proroku, že se vyplní zpuštení Jeruzaléma sedmdesátého léta.
\par 3 A obrátil jsem tvár svou ku Pánu Bohu, hledaje ho modlitbou a pokornými prosbami, v postu, v žíni a popele.
\par 4 I modlil jsem se Hospodinu Bohu svému, a vyznávaje se, rekl jsem: Prosím , Pane Bože silný, veliký a všeliké cti hodný, ostríhající smlouvy a dobrotivosti k tem, kteríž te milují, a ostríhají prikázaní tvých.
\par 5 Zhrešilit jsme a prevrácene jsme cinili, bezbožnost jsme páchali, a protivili jsme se, a odvrátili od prikázaní tvých a soudu tvých.
\par 6 Aniž jsme poslouchali služebníku tvých proroku, kteríž mluvívali ve jménu tvém králum našim, knížatum našim a otcum našim, i všemu lidu zeme.
\par 7 Tobet, ó Pane, prísluší spravedlnost, nám pak zahanbení tvári, jakž se to deje nyní mužum Judským a obyvatelum Jeruzalémským, a všechnem Izraelským, blízkým i dalekým, ve všech zemích, kamž jsi je zahnal pro prestoupení jejich, jímž prestupovali proti tobe.
\par 8 Námt, ó Hospodine, sluší zahanbení tvári, králum našim, knížatum našim a otcum našim, nebot jsme zhrešili proti tobe,
\par 9 Pánu Bohu pak našemu milosrdenství a slitování, ponevadž jsme se protivili jemu,
\par 10 A neposlouchali jsme hlasu Hospodina Boha našeho, abychom chodili v nauceních jeho, kteréž predkládal pred oci naše skrze služebníky své proroky.
\par 11 Nýbrž všickni Izraelští prestoupili zákon tvuj a odvrátili se, aby neposlouchali hlasu tvého; protož vylito jest na nás to prokletí a klatba, kteráž jest zapsána v zákone Mojžíše služebníka Božího, nebo jsme zhrešili proti nemu.
\par 12 Procež splnil slovo své, kteréž mluvil proti nám a proti soudcum našim, kteríž nás soudili, a uvedl na nás toto zlé veliké, jehož se nestalo pode vším nebem, jakéž se stalo v Jeruzaléme.
\par 13 Tak jakž zapsáno jest v zákone Mojžíšove, všecko to zlé prišlo na nás, a však ani tak jsme se nekorili pred tvárí hnevivou Hospodina Boha našeho, abychom se odvrátili od nepravostí svých, a šetrili pravdy jeho.
\par 14 Protož neobmeškal Hospodin s tím zlým, ale uvedl je na nás; nebo spravedlivý jest Hospodin Buh náš ve všech skutcích svých, kteréž ciní, jehož hlasu poslušni jsme nebyli.
\par 15 Nyní však, ó Pane Bože náš, kterýž jsi vyvedl lid svuj z zeme Egyptské rukou silnou, a zpusobils sobe jméno, jakéž jest dnešního dne, zhrešili jsme, bezbožne jsme cinili.
\par 16 Ó Pane, podlé vší tvé dobrotivosti necht se, prosím, odvrátí hnev tvuj a prchlivost tvá od mesta tvého Jeruzaléma, hory svatosti tvé; nebo pro hríchy naše a pro nepravosti otcu našich Jeruzalém a lid tvuj v pohanení jest u všech, kteríž jsou vukol nás.
\par 17 Nyní tedy, ó Bože náš, vyslyš modlitbu služebníka svého, a pokorné prosby jeho, a zasvet tvár svou nad svatyní svou spuštenou, pro Pána.
\par 18 Naklon, Bože muj, ucha svého a slyš, otevri oci své a viz zpuštení naše i mesta, kteréž jest nazváno od jména tvého; nebo ne pro nejaké naše spravedlnosti padajíce, pokorne prosíme tebe, ale pro milosrdenství tvá mnohá.
\par 19 Vyslyšiž, ó Pane, Pane, odpust, Pane, pozoruj a ucin; neprodlévejž pro sebe samého, muj Bože, nebo od jména tvého nazváno jest mesto toto i lid tvuj.
\par 20 Ješte jsem mluvil a modlil se, a vyznával hrích svuj i hrích lidu svého Izraelského, a padna, pokorne jsem se modlil pred tvárí Hospodina Boha svého, za horu svatosti Boha svého,
\par 21 Ješte, pravím, mluvil jsem pri modlitbe, a muž ten Gabriel, kteréhož jsem videl v tom videní na pocátku, rychle priletev, dotekl se mne v cas obeti vecerní.
\par 22 A slouže mi k srozumení, mluvil se mnou a rekl: Danieli, nyní jsem vyšel, abych te naucil vyrozumívati tajemstvím.
\par 23 Pri pocátku pokorných proseb tvých vyšlo slovo, a já jsem prišel, atbych je oznámil, nebo jsi velmi milý; procež pozoruj slova toho, a rozumej videní tomu.
\par 24 Sedmdesáte téhodnu odecteno jest lidu tvému a mestu svatému tvému k zabránení prevrácenosti a k zapecetení hríchu, i vycištení nepravosti a k privedení spravedlnosti vecné, a k zapecetení videní i proroctví, a ku pomazání Svatého svatých.
\par 25 Veziž tedy a rozumej, že od vyjití výpovedi o navrácení a o vystavení Jeruzaléma až do Mesiáše vývody bude téhodnu sedm, potom téhodnu šedesáte dva, když již zase vzdelána bude ulice a príkopa, a ti casové budou prenesnadní.
\par 26 Po téhodnech pak tech šedesáti a dvou zabit bude Mesiáš, však jemu to nic neuškodí; nýbrž to mesto i tu svatyni zkazí, i lid ten svuj budoucí, tak že skoncení jeho bude hrozné, ano i do vykonání boje bude boj stálý všelijak do vyplénení.
\par 27 Utvrdí však smlouvu mnohým v téhodni posledním, u prostred pak toho téhodne uciní konec obeti zápalné i obeti suché; a skrze vojsko ohavné, až do posledního a uloženého poplénení hubící, na poplénené vylito bude zpuštení.

\chapter{10}

\par 1 Léta tretího Cýra krále Perského, zjeveno bylo slovo Danielovi, kterýž sloul jménem Baltazar, a pravé bylo slovo to, i uložený cas dlouhý, a rozum toho slova i smysl zjeven jemu u videní.
\par 2 V tech dnech já Daniel kvílil jsem za tri téhodny dnu.
\par 3 Pokrmu pochotného jsem nejedl, ani maso ani víno nevešlo do úst mých, aniž jsem se mastí mazal, až se vyplnili dnové trí téhodnu.
\par 4 Dne pak dvadcátého ctvrtého mesíce prvního, když jsem byl na brehu reky veliké, to jest Hiddekel.
\par 5 Pozdvih ocí svých, videl jsem, a aj, muž jeden odený v roucho lnené, a bedra jeho prepásaná byla zlatem ryzím z Ufaz.
\par 6 Telo pak jeho jako tarsis, a oblícej jeho na pohledení jako blesk, a oci jeho podobné pochodním horícím, a ramena jeho i nohy jeho na pohledení jako med vypulerovaná, a zvuk slov jeho podobný zvuku množství.
\par 7 Videl jsem pak já Daniel sám videní to, ale muži ti, kteríž se mnou byli, nevideli toho videní, než hruza veliká pripadla na ne, až i utekli, aby se skryli.
\par 8 Procež já zustal jsem sám, a videl jsem videní to veliké, ale nezustalo i ve mne síly, a krása má zmenila se, a porušila na mne, aniž jsem mohl zadržeti síly.
\par 9 Tedy slyšel jsem zvuk slov jeho, a uslyšav zvuk slov jeho, usnul jsem tvrde na tvári své, na tvári své na zemi.
\par 10 V tom aj, ruka dotkla se mne, a pozdvihla mne na kolena má a na dlane rukou mých.
\par 11 I rekl mi: Danieli, muži velmi milý, pozoruj slov, kteráž já mluviti budu tobe, a stuj na míste svém, nebo nyní poslán jsem k tobe. A když promluvil ke mne slovo to, stál jsem, tresa se.
\par 12 Procež rekl mi: Nebojž se, Danieli; nebo od prvního dne, jakž jsi priložil srdce své, abys rozumel, a trápil se pred Bohem svým, vyslyšána jsou slova tvá, a já prišel jsem prícinou slov tvých.
\par 13 Ale kníže království Perského postavovalo se proti mne za jedenmecítma dnu, až aj, Michal, jeden prední z knížat, prišel mi na pomoc; protož jsem já zustával tam pri králích Perských.
\par 14 Již pak prišel jsem, atbych oznámil, co potkati má lid tvuj v potomních dnech; nebo ješte videní bude o tech dnech.
\par 15 A když mluvil ke mne ta slova, sklopiv tvár svou k zemi, onemel jsem.
\par 16 A aj, jako podobnost cloveka dotkla se rtu mých, a otevrev ústa svá, mluvil jsem a rekl tomu, kterýž stál naproti mne: Pane muj, prícinou toho videní obrátili se bolesti mé na mne, a aniž jsem síly zadržeti mohl.
\par 17 Jakž tedy bude moci služebník Pána mého takový mluviti se Pánem mým takovým, ponevadž ve mne od toho casu, ve mne, pravím, nezustalo síly, ani dchnutí nepozustalo ve mne?
\par 18 Procež opet dotkl se mne na pohledení jako clovek, a posilnil mne.
\par 19 A rekl: Neboj se, muži velmi milý, pokoj tobe, posiln se, posiln se, pravím. Když pak on mluvil se mnou, posilnen jsa, rekl jsem: Necht mluví Pán muj, nebo jsi mne posilnil.
\par 20 I rekl: Víš-liž, proc jsem prišel k tobe? Nebo již navrátím se, abych bojoval s knížetem Perským. A já odcházím, a aj, kníže Recké pritáhne.
\par 21 Ale oznámímt to, což jest zapsáno v psání pravdomluvném; nebo ani jednoho není, ješto by sobe zmužile pocínal se mnou v tech vecech, krome Michala knížete vašeho.

\chapter{11}

\par 1 A tak já léta prvního Daria Médského postavil jsem se, abych ho zmocnoval a posiloval.
\par 2 Již pak oznámímt pravdu: Aj, ješte tri králové kralovati budou v Perské zemi; potom ctvrtý zbohatne bohatstvím velikým nade všecky, a když se zmocní v bohatství svém, vzbudí všecky proti království Reckému.
\par 3 I povstane král mocný, kterýž bude míti panství široké, a bude ciniti podlé vule své.
\par 4 Když se pak zmocní, potríno bude království jeho, a rozdeleno bude na ctyri strany sveta, však ne mezi potomky jeho, aniž bude panství jeho takové, jakéž bylo; nebo vykoreneno bude království jeho, a jiným mimo ne se dostane.
\par 5 Procež posilní se král polední, ano i jedno z knížat jeho, a mocnejší bude nad neho, a panovati bude; panství široké bude panství jeho.
\par 6 Po nekterých pak letech sprízní se; nebo dcera krále poledního dostane se za krále pulnocního, aby ucinila prímerí. Ale neobdrží síly ramene, aniž on ostojí s ramenem svým, ale vydána bude ona i ti, kteríž ji privedou, i syn její, i ten, kterýž ji posilnoval v ty casy.
\par 7 Potom povstane z výstrelku korenu jejích na místo jeho, kterýž pritáhne s vojskem svým, a uderí na pevnost krále pulnocního, a priciní se, aby se jich zmocnil.
\par 8 Nadto i bohy jejich s knížaty jejich, s nádobami drahými jejich, stríbrem a zlatem v zajetí zavede do Egypta, a bude bezpecen za mnoho let pred králem pulnocním.
\par 9 A tak prijde do království král polední, a navrátí se do zeme své.
\par 10 Ale synové onoho válciti budou, a seberou množství vojsk velikých. A nenadále prijda, jako povoden procházeti bude, a navracuje se, válkou dotírati bude až k jeho pevnostem.
\par 11 Procež rozdrážden jsa král polední, vytáhne, a bojovati bude s ním, s králem pulnocním, a sšikuje množství veliké, i bude vydáno množství to v ruku jeho.
\par 12 I pozdvihne se množství to, a povýší se srdce jeho, a ackoli porazí na tisíce, a však se nezmocní.
\par 13 Potom navráte se král pulnocní, sšikuje množství vetší než prvé, a po dokonání casu nekterých let, nenadále prijde s vojskem velikým a s dostatkem hojným.
\par 14 V tech casích mnozí se postaví proti králi polednímu, ale synové nešlechetní z lidu tvého zhubeni budou, a pro stvrzení videní tohoto padnou.
\par 15 Nebo pritáhne král pulnocní, a vzdelaje náspy, dobude mest hrazených, tak že ramena poledního neostojí, ani lid vybraný jeho, aniž budou míti síly k odpírání.
\par 16 A pritáhna proti nemu, bude ciniti podlé vule své, a nebude žádného, ješto by se postavil proti nemu. Postaví se také v zemi Judské, kterouž docela zkazí rukou svou.
\par 17 Potom obrátí tvár svou, aby pritáhna s mocí všeho království svého, a ukazuje se jako by vše upríme jednal, dovede neceho. Nebo dá jemu krásnou pannu, aby ho zahubil skrze ni, ale ona nedostojí aniž bude držeti s ním.
\par 18 Zatím obrátí tvár svou k ostrovum, a dobude mnohých, ale vudce prítrž uciní pohanení jeho, anobrž to hanení jeho na nej obrátí.
\par 19 Procež obrátí tvár svou k pevnostem zeme své, ale klesne a padne, i zahyne.
\par 20 I povstane na místo jeho v sláve královské ten, kterýž rozešle výbercí, ale ten po nemnohých dnech potrín bude, a to ne v hneve, ani v boji.
\par 21 Na míste tohoto postaví se nevzácný, ackoli nevloží na nej ozdoby královské, a však prijda pokojne, ujme království skrze úlisnost.
\par 22 A rameny jako povodní zachváceni budou pred oblícejem jeho mnozí, a potríni budou jako i ten vudce, kterýž s ním smlouvu ucinil.
\par 23 Nebo v tovaryšství s ním vejda, prokáže nad ním lest, a prijeda, zmocní se království s malým poctem lidu.
\par 24 Bezpecne také i do nejúrodnejších míst té krajiny vpadne, a ciniti bude to, cehož necinili otcové jeho, ani otcové otcu jeho; loupež a koristi a zboží jim rozdelí, ano i proti pevnostem chytrosti své vymýšleti bude, a to do casu.
\par 25 Potom vzbudí sílu svou a srdce své proti králi polednímu s vojskem velikým, s nímž král polední vojensky se potýkati bude, s vojskem velikým a velmi silným, ale neostojí, proto že vymyslí proti nemu chytrost.
\par 26 Nebo kteríž jídají pokrm, potrou jej, když vojsko onoho se rozvodní; i padnou, zbiti jsouce mnozí.
\par 27 Tehdáž obou tech králu srdce bude ciniti zlé, a za jedním a týmž stolem lež mluviti budou, ale nepodarí se, proto že cíl uložený na jiný ješte cas odložen.
\par 28 A protož navrátí se do zeme své s zbožím velikým, a srdce jeho bude proti smlouve svaté. Což ucine, navrátí se do zeme své.
\par 29 V uložený cas navráte se, potáhne na poledne, ale to nebude podobné prvnímu ani poslednímu.
\par 30 Nebo prijdou proti nemu lodí Citejské, procež bude jej to boleti, tak že opet zlobiti se bude proti smlouve svaté. Což ucine, navrátí se, a srozumení míti bude s temi, kteríž opustili smlouvu svatou.
\par 31 A vojska veliká podlé neho státi budou, a poškvrní svatyne a pevnosti; odejmou také obet ustavicnou, a postaví ohavnost zpuštení,
\par 32 Tak aby ty, kteríž se bezbožne proti smlouve chovati budou, v pokrytství posiloval úlisnostmi, lid pak, kterýž zná Boha svého, aby jímali. Což i uciní.
\par 33 Procež vyucující lid, vyucující mnohé, padati budou od mece a ohne, zajetí a loupeže za mnohé dny.
\par 34 A když padati budou, malou pomoc míti budou; nebo pripojí se k nim mnozí pochlebne.
\par 35 Z tech pak, kteríž jiné vyucují, padati budou, aby prubováni a cišteni a bíleni byli až do casu jistého; nebot to ješte potrvá až do casu uloženého.
\par 36 Král zajisté ten bude ciniti podlé vule své, a pozdvihne se a zvelebí nad každého boha, i proti Bohu nade všemi bohy nejsilnejšímu mluviti bude divné veci; a štastne se mu povede až do vykonání prchlivosti, ažby se to, což uloženo jest, vykonalo.
\par 37 Ani k bohum otcu svých se nenakloní, ani k milování žen, aniž k komu z bohu se nakloní, proto že se nade všecko velebiti bude.
\par 38 A na míste Boha nejsilnejšího ctíti bude boha, kteréhož neznali otcové jeho; ctíti bude zlatem a stríbrem, a kamením drahým a klénoty.
\par 39 A tak dovede toho, že pevnosti Nejsilnejšího budou boha cizího, a kteréž se mu videti bude, poctí slávou, a zpusobí, aby panovali nad mnohými, a zemi rozdelí místo mzdy.
\par 40 Pri dokonání pak toho casu trkati se s ním bude král polední, ale král pulnocní oborí se na nej s vozy a s jezdci a lodími mnohými, a pritáhna do zemí, jako povoden projde.
\par 41 Potom pritáhne do zeme Judské, a mnohé zeme padnou. Tito pak ujdou ruky jeho, Idumejští a Moábští, a prvotiny synu Ammon.
\par 42 A když ruku svou vztáhne na zeme, ani zeme Egyptská nebude moci jeho zniknouti.
\par 43 Nebo opanuje poklady zlata a stríbra, a všecky klénoty Egyptské; Lubimští také a Mourenínové za ním pujdou.
\par 44 V tom noviny od východu a od pulnoci prestraší jej; procež vytáhne s prchlivostí velikou, aby hubil a mordoval mnohé.
\par 45 I rozbije stany paláce svého mezi moremi, na hore okrasy svatosti; a když prijde k skonání svému, nebude míti žádného spomocníka.

\chapter{12}

\par 1 Toho casu postaví se Michal, kníže veliké, kterýž zastává synu lidu tvého, a bude cas ssoužení, jakéhož nebylo, jakž jest národ, až do toho casu; toho, pravím, casu vysvobozen bude lid tvuj, kdožkoli nalezen bude zapsaný v knize.
\par 2 Tut mnozí z tech, kteríž spí v prachu zeme, procítí, jedni k životu vecnému, druzí pak ku pohanení a ku potupe vecné.
\par 3 Ale ti, kteríž jiné vyucují, stkvíti se budou jako blesk oblohy, a kteríž k spravedlnosti privozují mnohé, jako hvezdy na vecné veky.
\par 4 Ty pak Danieli, zavri slova tato, a zapecet knihu tuto až do casu jistého. Mnozít budou pilne zpytovati, a rozmnoženo bude umení.
\par 5 Zatím videl jsem já Daniel, a aj, jiní dva stáli, jeden z této strany brehu reky, a druhý z druhé strany brehu též reky.
\par 6 A rekl muži tomu oblecenému v roucho lnené, kterýž stál nad vodou té reky: Když bude konec tem divným vecem?
\par 7 I slyšel jsem muže toho obleceného v roucho lnené, kterýž stál nad vodou té reky, an zdvihl pravici svou i levici svou k nebi, a prisáhl skrze Živého na veky, že po uloženém casu, a uložených casích, i pul casu, a když do cela rozptýlí násilí lidu svatého, dokonají se všecky tyto veci.
\par 8 A když jsem já slyše, nerozumel, rekl jsem: Pane muj, jaký konec bude tech vecí?
\par 9 Tedy rekl: Odejdi, Danieli, nebo zavrína jsou a zapecetena slova ta až do casu jistého.
\par 10 Precištováni a bíleni a prubováni budou mnozí; bezbožní zajisté bezbožnost páchati budou, aniž co porozumejí kterí z nich, ale moudrí porozumejí.
\par 11 Od toho pak casu, v nemž odjata bude obet ustavicná, a postavena ohavnost hubící, bude dnu tisíc, dve ste a devadesát.
\par 12 Blahoslavený, kdož doceká a prijde ke dnum tisíci, trem stum, tridcíti a peti.
\par 13 Ty pak odejdi k místu svému, a odpocívati budeš, a zustaneš v losu svém na skonání dnu.

\end{document}