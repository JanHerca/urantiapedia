\begin{document}

\title{Zechariah}

\chapter{1}

\par 1 Mesíce osmého, léta druhého Dariova, stalo se slovo Hospodinovo k Zachariášovi synu Barachiáše, syna Iddova, proroku, rkoucí:
\par 2 Rozhneval se Hospodin na otce vaše velice.
\par 3 Protož rci temto: Takto praví Hospodin zástupu: Navratte se ke mne, praví Hospodin zástupu, a navrátím se k vám, praví Hospodin zástupu.
\par 4 Nebudte jako otcové vaši, na než volávali proroci onino predešlí, ríkajíce: Takto praví Hospodin zástupu: Navratte se nyní od cest svých zlých, i od skutku vašich zlých, ale neuposlechli, ani pozorovali mne, praví Hospodin.
\par 5 Otcové vaši kde jsou? A proroci ti zdali na veky živi jsou?
\par 6 Ale však slova má a soudové moji, kteréž jsem prikázal služebníkum svým prorokum, zdali nepostihli otcu vašich? tak že obrátivše se, rekli: Jakž uložil Hospodin zástupu uciniti nám podlé cest našich, a podlé skutku našich, tak ucinil nám.
\par 7 Dne ctyrmecítmého, jedenáctého mesíce, kterýž jest mesíc Šebat, léta druhého Dariova, stalo se slovo Hospodinovo k Zachariášovi synu Barachiáše, syna Iddova, proroku, rkoucí:
\par 8 Videl jsem v noci, a aj, muž sedí na koni ryzím, kterýž stál mezi myrtovím, kteréž bylo v doline, za ním pak kone ryzí, strakaté a bílé.
\par 9 I rekl jsem: Kdo jsou tito, Pane muj? Rekl mi andel ten, kterýž mluvil se mnou: Já ukáži tobe, kdo jsou tito.
\par 10 Tedy odpovídaje muž ten, kterýž stál mezi myrtovím, rekl: Tito jsou, kteréž poslal Hospodin, aby zchodili zemi.
\par 11 I odpovedeli andelu Hospodinovu tomu, kterýž stál mezi myrtovím, a rekli: Zchodili jsme zemi, a aj, všecka zeme bezpecne bydlí, a pokoje užívá.
\par 12 Tedy odpovedel andel Hospodinuv a rekl: Ó Hospodine zástupu, až dokudž ty se nesmiluješ nad Jeruzalémem a nad mesty Judskými, na kteréž jsi hneval se již sedmdesáte let?
\par 13 I odpovedel Hospodin andelu tomu, kterýž mluvil se mnou, slovy dobrými, slovy potešitelnými.
\par 14 Tedy rekl mi andel, kterýž mluvil ke mne: Volej a rci: Takto praví Hospodin zástupu: Horlím pro Jeruzalém a Sion horlením velikým.
\par 15 A hnevám se náramne na ty národy, kteríž mají pokoj; nebo když jsem já se málo rozhneval, oni pomáhali k zlému.
\par 16 Protož takto praví Hospodin: Navrátil jsem se k Jeruzalému milosrdenstvím, dum muj staven bude v nem, praví Hospodin zástupu, a pravidlo vztaženo bude na Jeruzalém.
\par 17 Ješte volej a rci: Takto praví Hospodin zástupu: Ještet se rozsadí mesta má pro hojnost dobrého; nebot uteší ješte Hospodin Sion, a vyvolí ješte Jeruzalém.
\par 18 Tedy pozdvihl jsem ocí svých, a uzrel jsem, a aj, ctyri rohové.
\par 19 I rekl jsem andelu, kterýž mluvil se mnou: Co jest toto? I rekl mi: To jsou ti rohové, kteríž zmítali Judou, Izraelem a Jeruzalémem.
\par 20 Ukázal mi také Hospodin ctyri kováre.
\par 21 I rekl jsem: Co jdou delati tito? I mluvil, rka: Tito jsou rohové, kteríž zmítali Judou, tak že žádný nemohl pozdvihnouti hlavy své. Protož prišli tito, aby je prestrašili, a srazili rohy tech národu, kteríž pozdvihli rohu proti zemi Judské, aby ní zmítali.

\chapter{2}

\par 1 Opet pozdvihl jsem ocí svých, a uzrel jsem, a aj muž, v jehož ruce byla šnura míry.
\par 2 I rekl jsem: Kam jdeš? I rekl mi: Meriti Jeruzaléma, abych videl, jak veliká širokost jeho, a jak veliká dlouhost jeho.
\par 3 A aj, když andel ten, kterýž mluvil se mnou, vycházel, jiný andel vycházel jemu vstríc.
\par 4 A rekl jemu: Bež, mluv k mládenci tomu, rka: Po vsech bydliti budou Jeruzalémští, pro množství lidu a dobytka u prostred neho.
\par 5 A já budu, praví Hospodin, jeho zdí ohnivou vukol, a slávou budu u prostred neho.
\par 6 Nuže, nuže, utectež již z zeme pulnocní, praví Hospodin, ponevadž se ctyr stran sveta volný pruchod ucinil jsem vám, praví Hospodin.
\par 7 Nuže, Sione, kterýž prebýváš u dcery Babylonské, vydobud se.
\par 8 Nebo takto praví Hospodin zástupu: Po sláve poslal mne proti národum tem, kteríž zloupili vás; nebo kdož se dotýká vás, dotýká se zrítelnice oka mého.
\par 9 Nebo aj, já zdvihnu ruku svou proti nim, a budou loupež služebníkum svým, i zvíte, že Hospodin zástupu poslal mne.
\par 10 Prozpevuj a vesel se, dcero Sionská; nebo aj, já prijdu a budu bydliti u prostred tebe, praví Hospodin.
\par 11 I pripojí se národové mnozí k Hospodinu v ten den, a budou mým lidem, a budu bydliti u prostred tebe, i zvíš, že Hospodin zástupu poslal mne k tobe.
\par 12 Tehdy dedicne ujme Hospodin Judu, díl svuj, v zemi svaté, a vyvolí zase Jeruzalém.
\par 13 Umlkniž všeliké telo pred oblícejem Hospodinovým, nebot procítí z príbytku svatosti své.

\chapter{3}

\par 1 Potom mi ukázal Jozue kneze nejvyššího, stojícího pred andelem Hospodinovým, a satana stojícího po pravici jeho, aby se mu protivil.
\par 2 Ale Hospodin rekl satanu: Potresciž te Hospodin, satane, potresciž te, pravím, Hospodin, kterýž vyvoluje Jeruzalém. Zdaliž tento není jako hlavne vychvácená z ohne?
\par 3 Jozue pak oblecen byl v roucha zmazaná, a stál pred andelem.
\par 4 I odpovedel a rekl tem, kteríž stáli pred ním, rka: Vezmete roucho to zmazané s neho. A rekl jemu: Pohled, prenesl jsem s tebe nepravost tvou, a oblékl jsem te v roucha promenná.
\par 5 Opet rekl: Necht vstaví cepici peknou na hlavu jeho. I vstavili cepici peknou na hlavu jeho, a oblékli ho v roucha. Andel pak Hospodinuv tu stál.
\par 6 A osvedcil andel Hospodinuv Jozue, rka:
\par 7 Takto praví Hospodin zástupu: Jestliže po cestách mých choditi budeš, a jestliže stráž mou držeti budeš, budeš-li také souditi dum muj, a budeš-li ostríhati síní mých: dámt zajisté to, abys chodil mezi temito prístojícími.
\par 8 Slyš nyní, Jozue, kneže nejvyšší, ty i tovaryši tvoji, kteríž sedí pred tebou: Ackoli muži ti jsou za zázrak, aj, já však privedu služebníka svého, Výstrelek.
\par 9 Nebo aj, totot jest ten kámen, kterýž kladu pred Jozue, na kámen jeden sedm ocí; aj, já vyreži na nem rezbu, praví Hospodin zástupu, a odejmu nepravost té zeme jednoho dne.
\par 10 V ten den, praví Hospodin zástupu, povoláte jeden každý bližního svého pod vinný kmen a pod fík.

\chapter{4}

\par 1 Potom navrátil se andel, kterýž mluvil se mnou, a zbudil mne jako muže, kterýž zbuzen bývá ze sna svého.
\par 2 I rekl mi: Co vidíš? Jemuž jsem rekl:Vidím, že aj, svícen zlatý všecken, a olejný dcbán na vrchu jeho, a sedm lamp jeho na nem, a sedm nálevek k tem sedmi lampám, kteréž jsou na vrchu jeho.
\par 3 A dve olivy pri nem, jedna po pravé strane dcbánu olejného, a druhá po levé strane jeho.
\par 4 Tehdy odpovedel jsem a rekl jsem andelu tomu, kterýž mluvil ke mne, rka: Co ty veci jsou, pane muj?
\par 5 A odpovídaje andel, kterýž mluvil se mnou, rekl mi: Což nevíš, co ty veci jsou? I rekl jsem: Nevím, pane muj.
\par 6 Tedy odpovídaje, mluvil ke mne, rka: Toto jest slovo Hospodinovo k Zorobábelovi, rkoucí: Ne silou, ani mocí, ale Duchem mým, praví Hospodin zástupu.
\par 7 Co jsi ty, ó horo veliká, pred Zorobábelem? Rovina. Nebo doloží nejvyšší kámen s hlucným prokrikováním: Milost, milost jemu.
\par 8 I stalo se slovo Hospodinovo ke mne, rkoucí:
\par 9 Ruce Zorobábelovy založily dum tento, a ruce jeho dokonají. I zvíš, že Hospodin zástupu poslal mne k vám.
\par 10 Nebo kdož by pohrdal dnem malých zacátku, ponevadž se veselí, hledíce na ten kámen, totiž na závaží v ruce Zorobábelove, tech sedm ocí Hospodinových, procházejících všecku zemi?
\par 11 Tedy odpovídaje jemu, rekl jsem? Co ty dve olivy po pravé strane toho svícnu, i po levé strane jeho?
\par 12 Opet odpovídaje jemu, rekl jsem: Co ty dve olivky, kteréž jsou mezi dvema trubicemi zlatými, kteréž vylévají z sebe zlato?
\par 13 I mluvil ke mne, rka: Víš-liž, co ty veci jsou? I rekl jsem: Nevím, pane muj.
\par 14 Tedy rekl: To jsou ty dve olivy, kteréž jsou u Panovníka vší zeme.

\chapter{5}

\par 1 Potom opet pozdvihna ocí svých, uzrel jsem, a aj, kniha letela.
\par 2 I rekl mi: Co vidíš? Jemuž jsem rekl: Vidím knihu letící, jejíž dlouhost byla dvadcíti loktu a širokost desíti loktu.
\par 3 Tedy rekl mi: Toto prokletí vyjde na širokost vší zeme. Nebo každý zlodej podlé neho jako i ona vyhlazen, a každý prisahající podlé neho jako i ona vyhlazen bude.
\par 4 Vynesu je, praví Hospodin zástupu, aby došlo na dum zlodeje, a na dum prisahajícího skrze jméno mé falešne; anobrž bude bydliti u prostred domu jeho, a docela zkazí jej, i dríví jeho i kamení jeho.
\par 5 Tedy vyšel andel ten, kterýž mluvil se mnou, a rekl mi: Pozdvihni nyní ocí svých a viz, co jest to, což pochází.
\par 6 I rekl jsem: Co jest? Kterýž odpovedel: Toto jest efi pocházející. Rekl také: Toto jest oko její prohlédající všecku zemi.
\par 7 A aj, plech tlustý olovený nesen byl, a pritom žena jedna, kteráž sedela u prostred té efi.
\par 8 I rekl: Toto jest ta bezbožnost. I uvrhl ji do prostred té efi, uvrhl i ten plech olovený na vrch její.
\par 9 A pozdvihna ocí svých, uzrel jsem, a aj, dve ženy vycházely, majíce vítr v krídlách svých. Mely pak krídla podobná krídlum cápím, a vyzdvihly tu efi mezi zemi a mezi nebe.
\par 10 I rekl jsem andelu tomu, kterýž mluvil se mnou: Kam ony nesou tu efi?
\par 11 Kterýž rekl mi: Aby sobe vystavela dum v zemi Sinear, kdež by utvrzena byla a postavena na podstavku svém.

\chapter{6}

\par 1 Potom opet pozdvihna ocí svých, videl jsem, a aj, ctyri vozové vycházeli z prostredku dvou hor, hory pak ty byly hory ocelivé.
\par 2 V prvním voze byli koni ryzí, a v druhém voze byli koni vraní,
\par 3 V voze pak tretím koni bílí, a v voze ctvrtém koni strakatí a hnedí.
\par 4 Tedy odpovídaje, rekl jsem andelu, kterýž mluvil se mnou: Co tyto veci, pane muj?
\par 5 I odpovedel andel ten a rekl mi: To jsou ctyri vetrové nebeští, vycházející odtud, kdež stáli pred Panovníkem vší zeme.
\par 6 Koni vraní zaprežení vycházejí do zeme pulnocní, a bílí vycházejí za nimi, strakatí pak vycházejí do zeme polední.
\par 7 Hnedí také vyšedše, chteli jíti, aby schodili zemi. Protož rekl: Jdete, schodte zemi. I schodili zemi.
\par 8 A povolav mne, mluvil ke mne, rka: Aj ti, kteríž vyšli do zeme pulnocní, spokojili ducha mého v zemi pulnocní.
\par 9 I stalo se slovo Hospodinovo ke mne, rkoucí:
\par 10 Vezmi od zajatých, od Cheldaje a od Tobiáše a od Jedaiáše, (a prijdeš ty téhož dne, a vejdeš do domu Joziáše syna Sofoniášova), kteríž jdou z Babylona,
\par 11 Vezmi, pravím, stríbro a zlato, a udelej koruny, a vstav na hlavu Jozue syna Jozadakova, kneze nejvyššího.
\par 12 A mluviti budeš k nemu, rka: Takto praví Hospodin zástupu, rka: Aj muž, jehož jméno jest Výstrelek, kterýž z místa svého puciti se bude, ten vystaví chrám Hospodinuv.
\par 13 Nebo ten má vystaveti chrám Hospodinuv, a tentýž zase prinese slávu, a sedeti a panovati bude na trunu svém, a bude knezem na trunu svém, a rada pokoje bude mezi nimi obema.
\par 14 Budou pak ty koruny Chelemovi, a Tobiášovi, a Jedaiášovi, a Chenovi synu Sofoniášovu na památku v chráme Hospodinove.
\par 15 Nebo dalecí prijdou, a budou staveti chrám Hospodinuv, i zvíte, že Hospodin zástupu poslal mne k vám. A to se stane, jestliže skutecne poslouchati budete hlasu Hospodina Boha svého.

\chapter{7}

\par 1 Potom stalo se léta ctvrtého Daria krále, stalo se slovo Hospodinovo k Zachariášovi, ctvrtého dne mesíce devátého, kterýž jest Kislef,
\par 2 Když poslal do domu Božího Sarezer a Regemmelech, i muži jeho, aby se korili tvári Hospodinove,
\par 3 A aby mluvili k knežím, kteríž byli v dome Hospodina zástupu, i k prorokum, rkouce: Budeme-li plakati mesíce pátého, oddelujíce se, jako jsme cinili již po mnoho let?
\par 4 I stalo se slovo Hospodina zástupu ke mne, rkoucí:
\par 5 Rci všemu lidu této zeme i knežím takto: Když jste se postívali a kvílili, pátého a sedmého mesíce, a to po sedmdesáte let, zdaliž jste se opravdu mne, mne, pravím, postili?
\par 6 A když jíte aneb pijete, zdaliž ne pro sebe jíte a ne pro sebe pijete?
\par 7 Zdaliž tato nejsou slova, kteráž prohlásil Hospodin skrze proroky predešlé, když ješte Jeruzalém sedel bezpecne a užíval pokoje, i mesta jeho vukol neho, a lid v strane polední i na rovinách bydlil?
\par 8 I stalo se slovo Hospodinovo k Zachariášovi, rkoucí:
\par 9 Takto mluvíval Hospodin zástupu, rka: Soud pravý vynášejte, a milosrdenství a lítosti dokazujte jeden každý k bližnímu svému.
\par 10 A vdovy ani sirotka, príchozího ani chudého neutiskejte, a zlého žádný bližnímu svému neobmýšlejte v srdci svém.
\par 11 Ale nechteli pozorovati, a nastavili ramene urputného, a uši své obtížili, aby neslyšeli.
\par 12 A srdce své ucinili kámen pretvrdý, aby neslyšeli zákona toho a slov tech, kteráž posílal Hospodin zástupu Duchem svým skrze proroky predešlé. Procež prišel hnev veliký od Hospodina zástupu.
\par 13 Nebo stalo se, že jakož volajícího neslyšeli, tak když volali, neslyšel jsem, praví Hospodin zástupu.
\par 14 A vichricí rozptýlil jsem je mezi všecky ty národy, kterýchž neznali, a zeme tato spustla po nich, tak že nebylo žádného, kdo by tudy chodil, a tak privedli zemi žádoucí na spuštení.

\chapter{8}

\par 1 Opet stalo se slovo Hospodina zástupu, rkoucí:
\par 2 Takto praví Hospodin zástupu: Horlil jsem pro Sion horlením velikým, nýbrž rozhneváním velikým horlil jsem pro nej.
\par 3 Takto praví Hospodin: Navrátil jsem se k Sionu, a bydlím u prostred Jeruzaléma, aby sloul Jeruzalém mestem verným, a hora Hospodina zástupu horou svatosti.
\par 4 Takto praví Hospodin zástupu: Ještet sedati budou starci i baby na ulicích Jeruzalémských, maje každý z nich hul v ruce své pro sešlost veku.
\par 5 Ulice také mesta plné budou pacholat a devcat, hrajících na ulicích jeho.
\par 6 Takto praví Hospodin zástupu: Zdali, že se to nepodobné zdá pred ocima ostatku lidu tohoto dnu techto, také pred ocima mýma nepodobné bude? praví Hospodin zástupu.
\par 7 Takto praví Hospodin zástupu: Aj, já vysvobozuji lid svuj z zeme východní, a z zeme na západ slunce,
\par 8 A privedu je zase. I budou bydliti u prostred Jeruzaléma, a budou lidem mým, a já budu jejich Bohem v pravde a v spravedlnosti.
\par 9 Takto praví Hospodin zástupu: Posilntež se ruce vaše, kteríž jste slyšeli techto dnu slova tato z úst proroku, od toho dne, v nemž založen jest dum Hospodina zástupu, že má chrám dostaven býti.
\par 10 Nebo pred temito dny práce lidská, ani práce hovádek se nenahražovala, nýbrž ani vycházejícímu ani vcházejícímu nebylo pokoje pro neprítele, nebo já spustil jsem všecky lidi jedny s druhými.
\par 11 Nyní pak, ne jako ve dnech tech predešlých, ciním ostatkum lidu tohoto, praví Hospodin zástupu.
\par 12 Ale rozsívání máte pokojné, vinný kmen vydává ovoce své, a zeme vydává úrodu svou, nebesa také dávají rosu svou, a to všecko dávám v dedictví ostatkum lidu tohoto.
\par 13 A tak stane se, že jakož jste byli zlorecením mezi pohany, ó dome Judský a dome Izraelský, tak zase vás chrániti budu, a budete požehnáním. Nebojtež se, posilnte se ruce vaše.
\par 14 Nebo takto praví Hospodin zástupu: Jakož jsem byl myslil zle uciniti vám, když mne hnevali otcové vaši, praví Hospodin zástupu, aniž jsem litoval:
\par 15 Tak obráte se, myslím v techto dnech dobre ciniti Jeruzalému a domu Judskému. Nebojtež se.
\par 16 Tyto pak veci jsou, kteréž ciniti budete: Mluvte pravdu každý s bližním svým, pravý a pokojný soud vynášejte v branách svých.
\par 17 Aniž kdo bližnímu svému zlého obmýšlej v srdci svém, též prísahy falešné nemilujte; nebo všecko to jest, cehož nenávidím, praví Hospodin.
\par 18 I stalo se slovo Hospodina zástupu ke mne, rkoucí:
\par 19 Takto praví Hospodin zástupu: Pust ctvrtého mesíce, a pust pátého, a pust sedmého, a pust desátého obrátí se domu Judskému v radost a veselí, i v slavnosti rozkošné, ale pravdu a pokoj milujte.
\par 20 Takto praví Hospodin zástupu: Ještet budou pricházeti národové a obyvatelé mest mnohých,
\par 21 Pricházeti budou, pravím, obyvatelé jednoho k druhému, rkouce: Podmež s ochotností koriti se tvári Hospodinove, a hledati Hospodina zástupu. Pujdu i já.
\par 22 A tak prijdou lidé mnozí a národové nescíslní, aby hledali Hospodina zástupu v Jeruzaléme, a korili se tvári Hospodinove.
\par 23 Takto praví Hospodin zástupu: V tech dnech chopí se deset mužu ze všech jazyku tech národu, chopí se, pravím, podolka jednoho Žida, rkouce: Pujdeme s vámi; nebo slyšíme, že jest Buh s vámi.

\chapter{9}

\par 1 Bríme slova Hospodinova proti zemi, kteráž jest v vukolí tvém, a Damašek bude odpocinutí jeho. Nebo k Hospodinu zretel cloveka i všech pokolení Izraelských.
\par 2 Ano i do Emat dosáhne, do Týru i Sidonu, ackoli jest moudrý velmi.
\par 3 Vystavelte sobe Týrus pevnost, a nashromáždil stríbra jako prachu, a ryzího zlata jako bláta na ulicích.
\par 4 Aj, Pán vyžene jej, a vrazí do more sílu jeho, i on od ohne sežrán bude.
\par 5 Vida to Aškalon, báti se bude, i Gáza velikou bolest míti bude, též i Akaron, proto že jej zahanbilo ocekávání jeho. I zahyne král z Gázy, a Aškalon neosedí.
\par 6 A bude bydliti pankhart v Azotu, a tak vypléním pýchu Filistinských.
\par 7 Když pak odejmu vraždu jednoho každého od úst jeho, a ohavnosti jeho od zubu jeho, pripojen bude také i on Bohu našemu, aby byl jako vývoda v Judstvu, a Akaron jako Jebuzejský.
\par 8 A položím se vojensky u domu svého pro vojsko a pro ty, kteríž tam i zase jdou; aniž pujde skrze ne více násilník, proto že se tak nyní vidí ocím mým.
\par 9 Plésej velice, dcerko Sionská, prokrikuj, dcerko Jeruzalémská. Aj, král tvuj prijde tobe spravedlivý a spasení plný, chudý a sedící na oslu, totiž na oslátku mladém.
\par 10 Nebo vypléním vozy z Efraima a kone z Jeruzaléma, a vyplénena budou lucište válecná; nadto rozhlásí pokoj národum, a panování jeho od more až k mori, a od reky až do koncin zeme.
\par 11 Anobrž ty, pro krev smlouvy své vypustil jsem vezne tvé z jámy, v níž není žádné vody.
\par 12 Navratež se k ohrade, ó veznové v nadeji postavení. A tak v ten den, jakžt oznamuji, dvojnásobne nahradím tobe,
\par 13 Když sobe napnu Judu a lucište naplním Efraimem, a vzbudím syny tvé, ó Sione, proti synum tvým, ó Javane, a nastrojím te jako mec udatného.
\par 14 Nebo Hospodin proti nim se ukáže, a vynikne jako blesk strela jeho. Panovník, pravím, Hospodin trubou troubiti bude, a pobére se s vichricemi poledními.
\par 15 Hospodin zástupu chrániti bude lidu svého, aby zmocníce se kamením z praku, jedli a pili, prokrikujíce jako od vína, a naplní jakož cíši tak i rohy oltáre.
\par 16 A tak je vysvobodí v ten den Hospodin Buh jejich, jakožto stádce lid svuj, a vystaveno bude kamení pekne tesané místo korouhví v zemi jeho.
\par 17 Nebo aj, jaké blahoslavenství jeho, a jak veliká okrasa jeho! Obilé mládence a mest panny uciní mluvné.

\chapter{10}

\par 1 Žádejte od Hospodina dešte v cas príhodný, i zpusobí Hospodin pršku, a dá vám déšt hojný, a každému bylinu na poli.
\par 2 Nebo obrazové mluví marnost, a veštci prorokují faleš, a sny marné mluví, marností potešují. Protož šli jako stádo, ztrestáni jsou, proto že žádného nebylo pastýre.
\par 3 Proti pastýrum takovým zažžen jest hnev muj, a kozly ty trestati budu, ale stádo své, dum Judský, navštíví Hospodin zástupu, a uciní je jako kone ozdobného k boji.
\par 4 Od neho úhel, od neho hreb, od neho lucište válecné, od neho též pojde všeliký úredník.
\par 5 A budou podobni rekum, pošlapávajíce do bláta na ulicích v boji, když bojovati budou; nebo Hospodin s nimi, a zahanbí ty, kteríž jedou na koních.
\par 6 Posilním zajisté domu Judova, a dum Jozefuv vysvobodím, a bezpecne je osadím. Nebo lítost mám nad nimi, i budou, jako bych jich nezahnal; nebo já jsem Hospodin Buh jejich, a vyslyším je.
\par 7 I budou Efraimští jako silný rek, a veseliti se bude srdce jejich jako od vína. I jejich synové vidouce to, rozveselí se, a zpléše srdce jejich v Hospodinu.
\par 8 Šeptati jim budu, a tak je shromáždím; nebo je vykoupím, a rozmnoženi budou, jakož rozmnoženi byli.
\par 9 Nadto rozseji je mezi národy, aby na místech dalekých rozpomínali se na mne, a živi jsouce s syny svými, navrátili se.
\par 10 A tak je zase privedu z zeme Egyptské, i z Assyrské shromáždím je, a do zeme Galád a k Libánu privedu je, ale nepostací jim.
\par 11 Protož pro tesnost prejde pres more, a prorazí na mori vlnobití, i vyschnou všecky hlubiny reky, budet snížena i pýcha Assyrie, a berla Egypta odjata bude.
\par 12 A posilním jich v Hospodinu, aby ve jménu jeho ustavicne chodili, praví Hospodin.

\chapter{11}

\par 1 Otevri, Libáne, vrata svá, at zžíre ohen cedry tvé.
\par 2 Kvel jedle, nebo padl cedr, nebo znamenití vypléneni jsou; kvelte dubové Bázanští, nebo klesl les ohražený.
\par 3 Hlas kvílení pastýru, proto že popléneno dustojenství jejich; hlas rvání lvu, proto že poplénena pýcha Jordánu.
\par 4 Takto praví Hospodin Buh muj: Pas ovce tyto k zbití oddané,
\par 5 Kteríž držitelé jejich mordují, aniž bývají obvinováni, a kdož je prodávají, ríkají: Požehnaný Hospodin, že jsme zbohatli, a kteríž je pasou, nemají lítosti nad nimi.
\par 6 Protož nebudu míti lítosti více nad obyvateli této zeme, praví Hospodin, ale aj, já uvedu ty lidi jednoho druhému v ruku, a v ruku krále jejich. I budou potírati zemi tuto, a nevytrhnu jí z ruky jejich.
\par 7 Nebo jsem pásl ovce k zbití oddané, totiž vás chudé toho stáda, a vzav sobe dve hole, nazval jsem jednu utešením, a druhou jsem nazval svazujících. A pásl jsem, pravím, ty ovce,
\par 8 A zahladil jsem tri pastýre mesíce jednoho, ale duše má stýskala sobe s nimi, proto že duše jejich nenávidela mne.
\par 9 Procež rekl jsem: Nebudu vás pásti. Kteráž umríti má, necht umre, a kteráž má vyhlazena býti, necht jest vyhlazena, a jiné nechažt jedí maso jedna druhé.
\par 10 Protož vzav hul svou utešení, posekal jsem ji, zrušiv smlouvu svou, kterouž jsem ucinil se vším tím lidem.
\par 11 A v ten den, když zrušena byla, práve poznali chudí toho stáda, kteríž na mne pozor meli, že rec Hospodinova jest.
\par 12 Nebo jsem rekl jim: Jestliže se vám vidí, dejte mzdu mou; pakli nic, nechte tak. I odvážili mzdu mou tridceti stríbrných.
\par 13 I rekl mi Hospodin: Povrz je pred hrncíre. Znamenitá mzda, kterouž jsem tak draze šacován od nich. A tak vzav tridceti stríbrných, uvrhl jsem je v dome Hospodinove pred hrncíre.
\par 14 Potom posekal jsem hul svou druhou svazujících, zrušiv bratrství mezi Judou a mezi Izraelem.
\par 15 I rekl mi Hospodin: Vezmi sobe ješte oruží pastýre bláznivého.
\par 16 Nebo aj, já vzbudím pastýre v této zemi. Pobloudilých nebude navštevovati, ani jehnátka hledati, ani což polámaného jest, léciti, ani toho, což se zastavuje, nositi, ale maso toho, což tucnejšího jest, jísti bude, a kopyta jejich poláme.
\par 17 Beda pastýri tomu nicemnému, kterýž opouští stádo. Mec na rameno jeho a na oko pravé jeho, ráme jeho docela uschne, a oko pravé jeho naprosto zatmí se.

\chapter{12}

\par 1 Bríme slova Hospodinova prícinou Izraele. Praví Hospodin, kterýž roztáhl nebesa, a založil zemi, a sformoval ducha cloveka, kterýž jest v nem:
\par 2 Aj, já postavím Jeruzalém jako cíši, kteráž ku potácení privede všecky národy vukol, kteríž budou proti Judovi v obležení, i proti Jeruzalému.
\par 3 Nýbrž stane se v ten den, že položím Jeruzalém jako kámen pretežký všechnem národum, jejž kdožkoli zdvihati budou, velmi se urazí, byt se pak shromáždili proti nemu všickni národové zeme.
\par 4 V ten den, praví Hospodin, raním všelikého kone strnutím, a jezdce jeho zbláznením, ale na dum Judský otevru oci své, a všecky kone národu raním slepotou.
\par 5 I dejí vudcové Judští v srdci svém: Mámet sílu, i obyvatelé Jeruzalémští, v Hospodinu zástupu, Bohu svém.
\par 6 V ten den uciním vudce Judské podobné ohni zanícenému mezi drívím, a pochodni horící mezi snopy, i zžíre na pravo i na levo všecky národy vukol, a ostojí Jeruzalém ješte na míste svém v Jeruzaléme.
\par 7 Zachová Hospodin i stánky Judské prvé, aby se nevelebila ozdoba domu Davidova a ozdoba, prebývajících v Jeruzaléme nad Judu.
\par 8 V ten den chrániti bude Hospodin obyvatelu Jeruzalémských, a bude nejnestatecnejší z nich v ten den podobný Davidovi, a dum Daviduv podobný bohum, podobný andelu Hospodinovu pred nimi.
\par 9 Nebo stane se v ten den, že shledám všecky národy, kteríž pritáhnou proti Jeruzalému, abych je zahladil.
\par 10 A vyleji na dum Daviduv a na obyvatele Jeruzalémské Ducha milosti a pokorných proseb. I obrátí zretel ke mne, kteréhož jsou bodli, a kvíliti budou nad ním jako kvílením nad jednorozeným; horce, pravím, plakati budou nad ním, jako horce plací nad prvorozeným.
\par 11 V ten den bude veliké kvílení v Jeruzaléme, jako kvílení v Adadremmon na poli Mageddo.
\par 12 Nebo kvíliti bude zeme, každá celed obzvláštne, celed domu Davidova obzvláštne, a ženy jejich obzvláštne, celed domu Nátanova obzvláštne, a ženy jejich obzvláštne,
\par 13 Celed domu Léví obzvláštne, a ženy jejich obzvláštne, celed Semei obzvláštne, a ženy jejich obzvláštne,
\par 14 I všecky celedi jiné, každá celed obzvláštne, a ženy jejich obzvláštne.

\chapter{13}

\par 1 V ten den bude studnice otevrená domu Davidovu a obyvatelum Jeruzalémským, k obmytí hrícha i necistoty.
\par 2 Stane se také v ten den, praví Hospodin zástupu, že vyhladím jména modl z zeme, tak že nebudou pripomínány více, nýbrž také i ty proroky a ducha necistoty vyprázdním z zeme.
\par 3 I stane se, prorokoval-li by nekdo více, že jemu reknou otec jeho a matka jeho, kteríž jej zplodili: Nebudeš živ, proto že jsi lež mluvil ve jménu Hospodinovu. I probodnou jej otec jeho a matka jeho, kteríž jej zplodili, že prorokoval.
\par 4 A tak stane se v ten den, že se budou stydeti proroci ti, každý za videní své, když by prorokovali, aniž oblekou sukne z srstí, aby klamali.
\par 5 Ale dí každý: Nejsem já prorok. Muž, kterýž zemi delá, jsem já; nebo mne ucil tomu clovek od detinství mého.
\par 6 A dí-li kdo jemu: Jaké to máš rány na rukou svých? I odpoví: Jimiž jsem zbit v dome tech, kteríž mne milují.
\par 7 Ó meci, procit na pastýre mého, a na muže bližního mého, praví Hospodin zástupu. Bí pastýre, a rozprchnou se ovce, ale zase obrátím ruku svou k malickým.
\par 8 Nebo stane se po vší zemi této, praví Hospodin, že dve cástky vyhlazeny budou v ní a zemrou, a tretí v ní zanechána bude.
\par 9 A i tu tretí uvedu do ohne, a preženu je, jako se prehání stríbro, a zprubuji je, jako prubováno bývá zlato. Každý vzývati bude jméno mé, a já vyslyším jej. Reknu: Lid muj jest: a on dí: Hospodin jest Buh muj.

\chapter{14}

\par 1 Aj, den Hospodinuv prichází, a rozdeleny budou koristi tvé u prostred tebe.
\par 2 Nebo shromáždím všecky národy proti Jeruzalému k boji, i bude dobyto mesto, a domové zloupeni, a ženy zhanobeny budou. A když vyjde díl mesta v zajetí, ostatek lidu nebude vyhlazeno z mesta.
\par 3 Nebo Hospodin vytáhna, bude bojovati proti tem národum, jakž bojuje v den potýkání.
\par 4 I stanou nohy jeho v ten den na hore Olivetské, kteráž jest naproti Jeruzalému od východu, a rozdvojí se hora Olivetská napoly k východu a k západu údolím velmi velikým, a odstoupí díl té hory na pulnoci, a díl její na poledne.
\par 5 I budete utíkati pred údolím hor; nebo dosáhne údolí hor až k Azal. Budete, pravím, utíkati, jako jste utíkali pred zeme tresením za dnu Uziáše krále Judského, když prijde Hospodin Buh muj, a všickni svatí s ním.
\par 6 I stane se v ten den, že nebude svetla drahého, ani tmy husté.
\par 7 A tak bude den jeden, kterýž jest znám Hospodinu, aniž bude den, ani noc; a však stane se, že v cas vecera bude svetlo.
\par 8 Stane se také v ten den, že vycházeti budou vody živé z Jeruzaléma, díl jich k mori východnímu, a díl jich k mori nejdalšímu. V léte i v zime bude.
\par 9 A bude Hospodin králem nade vší zemí; v ten den bude Hospodin jediný, a jméno jeho jedno.
\par 10 A ucinena bude všecka tato zeme jako rovina od Gaba k Remmon na polední strane Jeruzalému, kterýž vyvýšen jsa, státi bude na míste svém, od brány Beniaminovy až k místu brány první, a až k bráne úhlové, a od veže Chananeel až k presu královskému.
\par 11 A budou bydliti v nem, a nebude více v prokletí; mesto Jeruzalém zajisté bezpecne sedeti bude.
\par 12 Tato pak bude rána, kterouž raní Hospodin všecky národy, kteríž by bojovali proti Jeruzalému: Usuší telo jednoho každého, stojícího na nohách svých, a oci jednoho každého usvadnou v derách svých, a jazyk jednoho každého usvadne v ústech jejich.
\par 13 I stane se v ten den, že bude znepokojení Hospodinovo veliké mezi nimi, tak že uchopí jeden druhého ruku, a vztažena bude ruka jednoho na ruku druhého.
\par 14 Také i ty, Judo, bojovati budeš v Jeruzaléme, a shromáždeno bude zboží všech národu vukol, zlato a stríbro, i roucha velmi mnoho.
\par 15 A podobná bude rána koní, mezku, velbloudu a oslu i všech hovad, kteráž budou v tom ležení, podobná ráne té.
\par 16 I stane se, že kdožkoli pozustane ze všech národu, kteríž by pritáhli proti Jeruzalému, pricházeti budou z rok do roka klaneti se králi Hospodinu zástupu a slaviti slavnost stánku.
\par 17 I stane se, kdo z celedí zeme nebude pricházeti do Jeruzaléma klaneti se králi Hospodinu zástupu, že nebude na ne pršeti déšt.
\par 18 A jestliže celed Egyptská nevstoupí, ani prijde, jakkoli na ne nepršívá, prijde však táž rána, kterouž raní Hospodin národy, kteríž by nepricházeli k slavení slavnosti stánku.
\par 19 Tat bude pokuta pro hrích Egyptských a pokuta pro hrích všech národu, kteríž by nechodili k slavení slavnosti stánku.
\par 20 V ten den bude na zvoncích konských: Svatost Hospodinu; a bude hrncu v dome Hospodinove jako cíší pred oltárem.
\par 21 Nýbrž bude všeliký hrnec v Jeruzaléme a v Judstvu svatost Hospodinu zástupu; a pricházejíce všickni, kteríž obetovati mají, budou je bráti a variti v nich. Aniž bude Kananejský více v dome Hospodina zástupu v ten den.

\end{document}