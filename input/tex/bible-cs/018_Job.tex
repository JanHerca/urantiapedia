\begin{document}

\title{Jób}

\chapter{1}

\par 1 Byl muž v zemi Uz, jménem Job, a muž ten byl sprostný a uprímý, boje se Boha, a vystríhaje se zlého.
\par 2 Kterémuž se narodilo sedm synu a tri dcery.
\par 3 A mel dobytka sedm tisíc ovec, tri tisíce velbloudu, pet set sprežení volu, a pet set oslic, a celedi služebné velmi mnoho, a byl muž ten vznešenejší nade všecky lidi východní.
\par 4 I scházívali se synové jeho, a strojívali hody po domích, každý ve dni svém. Posílávali také, a zvávali své tri sestry, aby jedly a pily s nimi.
\par 5 A když vyporádali dny hodu, posílával Job, a posvecoval jich, a vstávaje ráno, obetoval zápaly podlé poctu všech jich. Nebo ríkával Job: Snad zhrešili synové moji, aneb zlorecili Bohu v srdci svém. Tak ciníval \par po všecky ty dny.
\par 6 Jednoho pak dne, když prišli synové Boží, aby se postavili pred Hospodinem, prišel také i Satan mezi ne.
\par 7 Tedy rekl Hospodin Satanovi: Odkud jdeš? I odpovedel Satan Hospodinu, rka: Procházel jsem zemi, a obcházel jsem ji.
\par 8 I rekl Hospodin Satanovi: Spatril-lis služebníka mého Joba, že není jemu rovného na zemi, a že jest muž sprostný a uprímý, bojící se Boha a varující se zlého.
\par 9 A odpovídaje Satan Hospodinu, rekl: Zdaliž se \par darmo bojí Boha?
\par 10 Zdaž jsi ty ho neohradil i domu jeho a všeho, což má, se všech stran? Dílu rukou jeho požehnal jsi, a dobytek jeho rozmnožil se na zemi.
\par 11 Ale vztáhni nyní ruku svou, a dotkni se všeho, což má, nebude-lit zloreciti v oci.
\par 12 Tedy rekl Hospodin Satanovi: Aj, cožkoli má, v moci tvé bud, toliko na nej nevztahuj ruky své. I vyšel Satan od tvári Hospodinovy.
\par 13 Jednoho pak dne synové a dcery jeho jedli, a pili víno v dome bratra svého prvorozeného.
\par 14 I prišel posel k Jobovi, a rekl: Když volové orali, a oslice se pásly podlé nich,
\par 15 Vpád ucinivše Sabejští, zajali je, a služebníky zbili ostrostí mece, a utekl jsem toliko já sám, abych oznámil tobe.
\par 16 A když on ješte mluvil, prišed druhý, rekl: Ohen Boží spadl s nebe, a rozpáliv se na dobytek i na služebníky, sehltil je, já pak utekl jsem toliko sám, abych oznámil tobe.
\par 17 A když ten ješte mluvil, jiný prišed, rekl: Kaldejští sšikovavše tri houfy, pripadli na velbloudy, a zajali je, a služebníky zbili ostrostí mece, a utekl jsem toliko já sám, abych oznámil tobe.
\par 18 A když ten ješte mluvil, jiný prišel a rekl: Synové tvoji a dcery tvé jedli, a pili víno v dome bratra svého prvorozeného.
\par 19 A aj, vítr veliký strhl se z té strany od poušte, a uderil na ctyri úhly domu, tak že se oboril na deti, i zemreli, a utekl jsem toliko já sám, abych oznámil tobe.
\par 20 Tedy \par vstav, roztrhl roucho své, a oholil hlavu svou, a padna na zem, poklonu ucinil.
\par 21 A rekl: Nahý jsem vyšel z života matky své, nahý se také zase tam navrátím. Hospodin dal, Hospodin též odjal. Bud požehnáno jméno Hospodinovo.
\par 22 V tom ve všem nezhrešil Job, a neprivlastnil Bohu nic nemoudrého.

\chapter{2}

\par 1 I stalo se opet jednoho dne, že když prišli synové Boží, aby se postavili pred Hospodinem, prišel také i Satan mezi ne, aby se postavil pred Hospodinem.
\par 2 Tedy rekl Hospodin Satanovi: Odkud jdeš? I odpovedel Satan Hospodinu, rka: Procházel jsem zemi, a obcházel jsem ji.
\par 3 I rekl Hospodin Satanovi: Spatril-lis služebníka mého Joba, že není jemu rovného na zemi, že jest muž sprostný a uprímý, bojící se Boha, a varující se zlého, a že po dnes trvá v uprímnosti své, ackoli jsi ty mne popudil proti nemu, abych jej hubil bez príciny.
\par 4 A odpovídaje Satan Hospodinu, rekl: Kuži za kuži, a všecko, což má clovek, dá za sebe samého.
\par 5 Ale vztáhni nyní ruku svou, a dotkni se kostí jeho, a masa jeho, nebude-lit v oci zloreciti tobe.
\par 6 Tedy rekl Hospodin Satanovi: Aj, v moci tvé bud, a však zachovej ho pri životu.
\par 7 Protož vyšed Satan od tvári Hospodinovy, ranil Joba nežitem nejhorším, od zpodku nohy jeho až do vrchu hlavy jeho,
\par 8 Tak že vzal strepinu, aby se jí drbal, usadiv se v popele.
\par 9 I rekla jemu žena jeho: Ješte vždy trváš v své uprímnosti? Zlorec Bohu a umri.
\par 10 Jížto rekl: Mluvíš, jako jedna z bláznivých mluvívá. Dobré-liž jen veci bráti budeme od Boha, zlých pak nebudeme prijímati? V tom ve všem nezhrešil \par rty svými.
\par 11 Když pak uslyšeli tri prátelé Jobovi o všem tom zlém, kteréž prišlo na nej, prišli jeden každý z místa svého: Elifaz Temanský, a Bildad Suchský a Zofar Naamatský, na tom zustavše spolu, aby prijdouce k nemu, politovali ho a tešili jej.
\par 12 Kteríž pozdvihše ocí svých zdaleka, nepoznali ho. Potom pozdvihše hlasu svého, plakali, a roztrhše jeden každý roucho své, házeli prachem nad hlavy své zhuru.
\par 13 A sedeli s ním na zemi sedm dní a sedm nocí, a žádný k nemu nepromluvil slova; nebo videli, že se velmi rozmohla bolest jeho.

\chapter{3}

\par 1 Potom otevrev \par ústa svá, zlorecil dni svému.
\par 2 Nebo mluve Job, rekl:
\par 3 Ó by byl zahynul ten den, v nemž jsem se naroditi mel, i noc, v níž bylo receno: Pocat jest pacholík.
\par 4 Ten den ó by byl obrácen v temnost, aby ho byl nevyhledával Buh shury, a nebyl osvícen svetlem.
\par 5 Ó by jej byly zachvátily tmy a stín smrti, a aby jej byla prikvacila mracna, a predesila horkost denní.
\par 6 Ó by noc tu mrákota byla opanovala, aby nebyla pripojena ke dnum roku, a v pocet mesícu neprišla.
\par 7 Ó by noc ta byla osamela, a zpevu aby nebylo v ní.
\par 8 Ó by jí byli zlorecili ti, kteríž proklínají den, hotovi jsouce vzbuditi velryba.
\par 9 Ó by se byly hvezdy zatmely v soumraku jejím, a ocekávajíc svetla, aby ho nebyla docekala, ani spatrila záre jitrní.
\par 10 Nebo nezavrela dverí života mého, ani skryla trápení od ocí mých.
\par 11 Proc jsem neumrel v matce, aneb vyšed z života, proc jsem nezahynul?
\par 12 Proc jsem vzat byl na klín, a proc jsem prsí požíval?
\par 13 Nebo bych nyní ležel a odpocíval, spal bych a mel bych pokoj,
\par 14 S králi a radami zeme, kteríž sobe vzdelávali místa pustá,
\par 15 Aneb s knížaty, kteríž meli zlato, a domy své naplnovali stríbrem.
\par 16 Aneb jako nedochudce nezretelné proc jsem nebyl, a jako nemluvnátka, kteráž svetla nevidela?
\par 17 Tamt bezbožní prestávají bouriti, a tamt odpocívají ti, jenž v práci ustali.
\par 18 Také i veznové pokoj mají, a neslyší více hlasu násilníka.
\par 19 Malý i veliký tam jsou rovni sobe, a služebník jest prost pána svého.
\par 20 Proc Buh dává svetlo zbedovanému a život tem, kteríž jsou ducha truchlivého?
\par 21 Kteríž ocekávají smrti, a není jí, ackoli jí hledají pilneji než skrytých pokladu?
\par 22 Kteríž by se veselili s plésáním a radovali, když by nalezli hrob?
\par 23 Cloveku, jehož cesta skryta jest, a jehož Buh pristrel?
\par 24 Nebo pred pokrmem mým vzdychání mé prichází, a rozchází se jako voda rvání mé.
\par 25 To zajisté, cehož jsem se lekal, stalo se mi, a cehož jsem se obával, prišlo na mne.
\par 26 Nemel jsem pokoje, aniž jsem se ubezpecil, ani odpocíval, až i prišlo pokušení toto.

\chapter{4}

\par 1 Jemuž odpovídaje Elifaz Temanský, rekl:
\par 2 Pocneme-li mluviti s tebou, neponeseš-liž toho težce? Ale kdož by se zdržeti mohl, aby nemel mluviti?
\par 3 Aj, ucívals mnohé, a rukou opuštených jsi posiloval.
\par 4 Padajícího pozdvihovals recmi svými, a kolena zemdlená jsi zmocnoval.
\par 5 Nyní pak, jakž toto prišlo na tebe, težce to neseš, a jakž te dotklo, predešen jsi.
\par 6 Nebylo-liž náboženství tvé nadejí tvou, a uprímost cest tvých ocekáváním tvým?
\par 7 Rozpomen se, prosím, kdo jest kdy nevinný zahynul? Aneb kde uprímí vyhlazeni jsou?
\par 8 Jakož jsem já vídal ty, kteríž orali nepravost, a rozsívali prevrácenost, že ji i žali.
\par 9 Od dchnutí Božího hynou, a duchem prchlivosti jeho v nic obracíni bývají.
\par 10 Rvání lva a hlas lvice a zubové mladých lvícat setríni bývají.
\par 11 Hyne lev, že nemá loupeže, a lvícata mladá rozptýlena bývají.
\par 12 Nebo i tajne doneslo se mne slovo, a pochopilo ucho mé neco malicko toho.
\par 13 V premyšlováních z videní nocních, když pripadá tvrdý sen na lidi,
\par 14 Strach pripadl na mne a lekání, kteréž predesilo všecky kosti mé.
\par 15 Duch zajisté pred tvárí mou šel, tak že vlasové vstávali na tele mém.
\par 16 Zastavil se, ale neznal jsem tvári jeho; tvárnost jen byla pred ocima mýma. Mezi tím mlce, slyšel jsem hlas:
\par 17 Zdaliž muže clovek spravedlivejším býti než Buh, aneb muž cistším nad toho, kterýž ho ucinil?
\par 18 Ano mezi služebníky jeho není dokonalosti, a pri andelích svých zanechal nedostatku.
\par 19 Cím více pri tech, kteríž bydlejí v domích hlinených, jejichž základ jest na prachu, a setríni bývají snáze než mol.
\par 20 Od jitra až do vecera stíráni bývají, a kdož toho nerozvažují, na veky zahynou.
\par 21 Zdaliž nepomíjí sláva jejich s nimi? Umírají, ale ne v moudrosti.

\chapter{5}

\par 1 Zavolejž tedy, dá-lit kdo odpoved? A k kterému se z svatých obrátíš?
\par 2 Pakli k bláznu, zahubí ho rozhnevání, a nesmyslného zabije prchlivost.
\par 3 Ját jsem videl blázna, an se vkorenil, ale hned jsem zle tušil príbytku jeho, rka:
\par 4 Vzdálenit jsou synové jeho od spasení; nebo potríni budou v bráne, aniž bude, kdo by je vytrhl.
\par 5 Obilé jeho zžíre hladovitý, a i z prostred trní je vychvátí; nadto sehltí násilník statek takových.
\par 6 Nebot nepochází z prachu trápení, aniž se z zeme pucí bída.
\par 7 Ale clovek rodí se k bíde, tak jako jiskry z uhlí zhuru létají.
\par 8 Jiste žet bych já hledal Boha silného, a jemu bych predložil pri svou,
\par 9 Kterýž ciní veci veliké, nezpytatelné, divné, a jimž poctu není,
\par 10 Kterýž dává déšt na zemi, a spouští vody na pole,
\par 11 Kterýž sází opovržené na míste vysokém, a žalostící vyvyšuje spasením,
\par 12 Kterýž v nic obrací myšlení chytráku, tak aby nemohli k skutku privésti ruce jejich niceho,
\par 13 Kterýž lapá moudré v chytrosti jejich; nebo rada prevrácených bláznová bývá.
\par 14 Ve dne motají se jako ve tmách, a jako v noci šámají o poledni.
\par 15 Kterýž zachovává od mece a od úst jejich, a chudého od ruky násilníka.
\par 16 Mát zajisté nuzný nadeji, ale nepravost musí zacpati ústa svá.
\par 17 Aj, jak blahoslavený jest clovek, kteréhož tresce Buh! A protož káráním Všemohoucího nepohrdej.
\par 18 Ont zajisté uráží, on i obvazuje; raní, ruka jeho také lécí.
\par 19 Z šesti úzkostí vysvobodil by tebe, ano i v sedmi nedotklo by se tebe zlé.
\par 20 V hladu vykoupil by te od smrti, a v boji od moci mece.
\par 21 Když utrhá jazyk, byl bys skryt, aniž bys se bál zhouby, když by prišla.
\par 22 Zhouba a hlad bud tobe za smích, a nestrachuj se ani líté zveri zemské.
\par 23 Nebo s kamením polním prímerí tvé, a zver lítá polní pokoj zachová k tobe.
\par 24 A shledáš, žet stánek tvuj bude bezpecný, a navrátíš se zase k príbytku svému, a nezhrešíš.
\par 25 Shledáš také, žet se rozmnoží síme tvé, a potomci tvoji jako bylina zemská.
\par 26 Vejdeš v šedinách do hrobu, tak jako odnášíno bývá zralé obilí casem svým.
\par 27 Aj, tot jsme vyhledali, a takt jest; poslechniž toho, a schovej sobe to.

\chapter{6}

\par 1 Odpovídaje pak Job, rekl:
\par 2 Ó kdyby pilne zváženo bylo horekování mé, a bída má na váze aby spolu vyzdvižena byla.
\par 3 Jiste že by se nad písek morský težší ukázala, procež mi se i slov nedostává.
\par 4 Nebo strely Všemohoucího vezí ve mne, jejichž jed vysušil ducha mého, a hruzy Boží bojují proti mne.
\par 5 Zdaliž rve divoký osel nad mladistvou travou? Rve-liž vul nad picí svou?
\par 6 Zdaliž jedí to, což neslaného jest, bez soli? Jest-liž chut v veci slzké?
\par 7 Ach, kterýchž se ostýchala dotknouti duše má, ty jsou již bolesti tela mého.
\par 8 Ó by se naplnila žádost má, a aby to, cehož ocekávám, dal Buh,
\par 9 Totiž, aby se líbilo Bohu setríti mne, vztáhnouti ruku svou, a zahladiti mne.
\par 10 Nebot mám ješte, cím bych se potešoval, (ackoli horím bolestí, aniž mne Buh co lituje), že jsem netajil recí Nejsvetejšího.
\par 11 Nebo jaká jest síla má, abych potrvati mohl? Aneb jaký konec muj, abych prodlel života svého?
\par 12 Zdali síla má jest síla kamenná? Zdali telo mé ocelivé?
\par 13 Zdaliž pak obrany mé není pri mne? Aneb zdravý soud vzdálen jest ode mne,
\par 14 Proti tomu, jehož lítostivost k bližnímu mizí, a kterýž bázen Všemohoucího opustil?
\par 15 Bratrí moji zmýlili mne jako potok, pominuli jako prudcí potokové,
\par 16 Kteríž kalní bývají od ledu, a v nichž se kryje sníh.
\par 17 V cas horka vysychají; když sucho bývá, mizejí z místa svého.
\par 18 Sem i tam roztekají se od toku svého obecného, v nic se obracejí a hynou.
\par 19 To vidouce houfové jdoucích z Tema, zástupové Sabejských, jenž nadeji meli v nich,
\par 20 Zastydeli se, že v nich doufali; nebo prišedše až k nim, oklamáni jsou.
\par 21 Tak zajisté i vy byvše, nejste; vidouce potrení mé, desíte se.
\par 22 Zdali jsem rekl: Prineste mi, aneb z zboží svého udelte daru pro mne?
\par 23 Aneb: Vysvobodte mne z ruky neprítele, a z ruky násilníku vykupte mne?
\par 24 Poucte mne, a budu mlceti, a v cem bych bloudil, poslužte mi k srozumení.
\par 25 Ó jak jsou pronikavé reci uprímé! Ale co vzdelá obvinování vaše?
\par 26 Zdali jen z slov mne viniti myslíte, a prevívati reci choulostivého?
\par 27 Také i na sirotka se oborujete, anobrž jámu kopáte príteli svému.
\par 28 A protož nyní chtejtež popatriti na mne, a sudte, klamám-lit pred oblícejem vaším.
\par 29 Napravte se, prosím, necht není nepravostí; napravte se, pravím, a tak poznáte, žet jest spravedlnost v té reci mé.
\par 30 A jest-li na jazyku mém nepravost, nemel-liž bych, citedlen býti bíd?

\chapter{7}

\par 1 Zdaliž nemá vymereného casu clovek na zemi? A dnové jeho jako dnové nájemníka.
\par 2 Jako služebník, kterýž touží po stínu, a jako nájemník, jenž ocekává skonání díla svého:
\par 3 Tak jsou mi dedicne privlastneni mesícové marní, a noci plné trápení jsou mi odecteny.
\par 4 Jestliže ležím, ríkám: Kdy vstanu? A  pomine noc? Tak pln bývám myšlení až do svitání.
\par 5 Telo mé odíno jest cervy a strupem i prachem, kuže má puká se a rozpouští.
\par 6 Dnové moji rychlejší byli nežli clunek tkadlce, nebo stráveni jsou bez prodlení.
\par 7 Rozpomen se, ó Pane, že jako vítr jest život muj, a oko mé že více neuzrí dobrých vecí,
\par 8 Aniž mne spatrí oko, jenž mne vídalo. Oci tvé budou ke mne, a mne již nebude.
\par 9 Jakož oblak hyne a mizí, tak ten, kterýž sstupuje do hrobu, nevystoupí zase,
\par 10 Aniž se opet navrátí do domu svého, aniž ho již více pozná místo jeho.
\par 11 Protož nemohut já zdržeti úst svých, mluvím v ssoužení ducha svého, naríkám v horkosti duše své.
\par 12 Zdali jsem já morem cili velrybem, že jsi mne stráží osadil?
\par 13 Když myslím: Poteší mne luže mé, poodejme naríkání mého postel má:
\par 14 Tedy mne strašíš sny, a videními desíš mne,
\par 15 Tak že sobe zvoluje zaškrcení duše má, a smrt nad život.
\par 16 Mrzí mne, nebudut déle živ. Poodstupiž ode mne, nebo marní jsou dnové moji.
\par 17 Co jest clovek, že ho sobe tak vážíš, a že tak o nej pecuješ?
\par 18 A že ho navštevuješ každého jitra, a každé chvíle jej zkušuješ?
\par 19 Dokudž se neodvrátíš ode mne, a nedáš mi aspon polknouti mé sliny?
\par 20 Zhrešil jsem, což mám uciniti, ó strážce lidský? Proc jsi mne položil za cíl sobe, tak abych sám sobe byl bremenem?
\par 21 Nýbrž proc neodejmeš prestoupení mého, a neodpustíš nepravosti mé? Nebo již v zemi lehnu. Potom bys mne i pilne hledal, nebude mne.

\chapter{8}

\par 1 Tedy odpovídaje Bildad Suchský, rekl:
\par 2 Dokudž mluviti budeš takové veci, a slova úst tvých budou jako vítr násilný?
\par 3 Což by Buh silný nepráve soudil,a Všemohoucí což by prevracel spravedlnost?
\par 4 Synové zajisté tvoji že zhrešili proti nemu, proto pustil je po nepravosti jejich.
\par 5 Kdybys ty opravdove hledal Boha silného, a Všemohoucímu se modlil,
\par 6 A byl cistý a uprímý: jiste žet by se hned probudil k tobe, a napravil by príbytek spravedlnosti tvé.
\par 7 A byly by první veci tvé špatné, poslední pak rozmnožily by se náramne.
\par 8 Nebo vzeptej se, prosím, veku starého, a nastroj se k zpytování otcu jejich.
\par 9 (Myt zajisté vcerejší jsme, aniž jsme ceho povedomi; k tomu dnové naši jsou jako stín na zemi.)
\par 10 Zdaliž te oni nenaucí, a nepovedí tobe, a z srdce svého nevynesou-liž slov?
\par 11 Zdali roste trtí bez bahna? Roste-liž rákosí bez vody?
\par 12 Nýbrž ješte za zelena, dríve než vytrháno bývá, ano prvé než jaká jiná tráva, usychá.
\par 13 Tak stezky všech zapomínajících se na Boha silného, tak, pravím, nadeje pokrytce zahyne.
\par 14 Klesne nadeje jeho, a doufání jeho jako dum pavouka.
\par 15 Spolehne-li na dum svuj, neostojí; chytí-li se ho, nezdrží.
\par 16 Vláhu má pred sluncem, tak že z zahrady jeho výstrelkové jeho vynikají.
\par 17 Pri vrchovišti korenové jeho huste rostou, i na místech skalnatých rozkládá se.
\par 18 A však bývá-li zachvácen z místa svého, až by se ho i odectlo, rka: Nevidelo jsem te:
\par 19 Tožt ta radost života jeho, a z zeme jiný vykvetá.
\par 20 Aj, Buh silný nepohrdá uprímým, ale nešlechetným ruky nepodává:
\par 21 Až i naplní smíchem ústa tvá, a rty tvé plésáním,
\par 22 Když nenávidící tebe v hanbu obleceni budou, a stánku lidí bezbožných nikdež nebude.

\chapter{9}

\par 1 Odpovedev pak Job, rekl:
\par 2 I ovšem vím, žet tak jest; nebo jak by mohl clovek spravedliv býti pred Bohem silným?
\par 3 A chtel-li by se hádati s ním, nemohl by jemu odpovedíti ani na jedno z tisíce slov.
\par 4 Moudrého jest srdce a silný v moci. Kdo zatvrdiv se proti nemu, pokoje užil?
\par 5 On prenáší hory, než kdo shlédne, a podvrací je v prchlivosti své.
\par 6 On pohybuje zemí z místa jejího, tak že se tresou sloupové její.
\par 7 On když zapovídá slunci, nevychází, a hvezdy zapecetuje.
\par 8 On roztahuje nebe sám, a šlapá po vlnách morských.
\par 9 On ucinil Arktura, Oriona, Kurátka a hvezdy skryté na poledne.
\par 10 On ciní veci veliké, a to nevystižitelné a divné, jimž není poctu.
\par 11 Ano jde-li mimo mne, tedy nevidím; ovšem když pomíjí, neznamenám ho.
\par 12 Tolikéž jestliže co uchvátí, kdo mu to rozkáže navrátiti? Kdo dí jemu: Co ciníš?
\par 13 Nezdržel-li by Buh hnevu svého, klesli by pred ním spolu spuntovaní, jakkoli mocní.
\par 14 Jakž bych já tedy jemu odpovídati, a jaká slova svá proti nemu vyhledati mohl?
\par 15 Kterémuž, bych i spravedliv byl, nebudu odpovídati, ale pred soudcím svým pokoriti se budu.
\par 16 Ac bych pak i volal, a on mi se ozval, neuverím, aby vyslyšel hlas muj,
\par 17 Ponevadž vichricí setrel mne, rozmnožil rány mé bez príciny.
\par 18 Aniž mi dá oddechnouti, ale sytí mne horkostmi.
\par 19 Obrátil-li bych se k moci, aj, ont jest nejsilnejší; pakli k soudu, kdo mi rok složí?
\par 20 Jestliže se za spravedlivého staveti budu, ústa má potupí mne; pakli za uprímého, prevráceného mne býti ukáží.
\par 21 Jsem-li uprímý, nebudu vedeti toho; nenávideti budu života svého.
\par 22 Jediná jest vec, procež jsem to mluvil, že uprímého jako bezbožného on zahlazuje.
\par 23 Jestliže bicem náhle usmrcuje, zkušování nevinných se posmívá;
\par 24 Zeme dána bývá v ruku bezbožného, tvár soudcu jejich zakrývá: jestliže ne on, kdož jiný jest?
\par 25 Dnové pak moji rychlejší byli nežli posel; utekli, aniž videli dobrých vecí.
\par 26 Pominuli jako prudké lodí, jako orlice letící na pastvu.
\par 27 Dím-li: Zapomenu se na své naríkání, zanechám horlení svého, a posilím se:
\par 28 Lekám se všech bolestí svých, vida, že mne jich nezprostíš.
\par 29 Jestli jsem bezbožný, procež bych nadarmo pracoval?
\par 30 Ano bych se i umyl vodou snežnou, a ocistil mýdlem ruce své,
\par 31 Tedy v jáme pohrížíš mne, tak že se ode mne zprzní i to roucho mé.
\par 32 Nebo Buh není clovekem jako já, jemuž bych odpovídati mohl, a abychom vešli spolu v soud.
\par 33 Aniž máme prostredníka mezi sebou, kterýž by rozhodl nás oba.
\par 34 Kdyby odjal ode mne prut svuj, a strach jeho aby mne nekormoutil,
\par 35 Tehdáž bych mluvil, a nebál bych se, ponevadž není toho tak pri mne.

\chapter{10}

\par 1 Stýšte se duši mé v živote mém, vypustím nad sebou naríkání své, mluviti budu v horkosti duše své.
\par 2 Dím Bohu: Neodsuzuj mne, oznam mi, proc se nesnadníš se mnou?
\par 3 Jaký máš na tom užitek, že mne ssužuješ, že pohrdáš dílem rukou svých, a radu bezbožných osvecuješ?
\par 4 Zdali oci telesné máš? Zdali tak, jako hledí clovek, ty hledíš?
\par 5 Zdaž jsou jako dnové cloveka dnové tvoji, a léta tvá podobná dnum lidským,
\par 6 Že vyhledáváš nepravosti mé, a na hrích muj se vyptáváš?
\par 7 Ty víš, žet nejsem bezbožný, ackoli není žádného, kdo by mne vytrhl z ruky tvé.
\par 8 Ruce tvé sformovaly mne, a ucinily mne, a ted pojednou všudy vukol hubíš mne.
\par 9 Pametliv bud, prosím, že jsi mne jako hlinu ucinil, a že v prach zase obrátíš mne.
\par 10 Zdalis mne jako mléka neslil, a jako syrení neshustil?
\par 11 Kuží a masem priodel jsi mne, a kostmi i žilami spojils mne.
\par 12 Života z milosrdenství udelil jsi mi, presto navštevování tvé ostríhalo dýchání mého.
\par 13 Ale toto skryl jsi v srdci svém; vím, žet jest to pri tobe.
\par 14 Jakž zhreším, hned mne šetríš, a od nepravosti mé neocištuješ mne.
\par 15 Jestliže jsem bezbožný, beda mne; paklit jsem spravedlivý, ani tak nepozdvihnu hlavy, nasycen jsa hanbou, a vida trápení své,
\par 16 Kteréhož vždy více pribývá. Honíš mne jako lev, a jedno po druhém divne se mnou zacházíš.
\par 17 Obnovuješ svedky své proti mne, a rozmnožuješ rozhnevání své na mne; vojska jedna po druhých jsou proti mne.
\par 18 Proc jsi jen z života vyvedl mne? Ó bych byl zahynul, aby mne bylo ani oko nevidelo,
\par 19 A abych byl, jako by mne nikdy nebylo, z života do hrobu abych byl vnesen.
\par 20 Zdaliž jest mnoho dnu mých? Ponechejž tedy a popust mne, abych malicko pookrál,
\par 21 Prvé než odejdu tam, odkudž se zase nenavrátím, do krajiny tmavé, anobrž stínu smrti,
\par 22 Do krajiny, pravím, tmavé, kdež jest sama mrákota stínu smrti, a kdež není žádných promen, ale sama pouhá mrákota.

\chapter{11}

\par 1 A odpovídaje Zofar Naamatský, rekl:
\par 2 Zdaliž k mnohým slovum nemá odpovedíno býti? Aneb zdali clovek mnohomluvný práv zustane?
\par 3 Žváním svým lidi zamestknáváš, a posmíváš se, aniž jest, kdo by te zahanbil.
\par 4 Nebo jsi rekl: Cisté jest ucení mé, a cist jsem, ó Bože, pred ocima tvýma.
\par 5 Ješto, ó kdyby Buh mluvil, a otevrel rty své proti tobe,
\par 6 Atby oznámil tajemství moudrosti, že dvakrát vetšího trestání zasloužil jsi. A vez, že se Buh zapomnel na tebe pro nepravost tvou.
\par 7 Zdaliž ty stižitelnosti Boží dosáhneš, aneb dokonalost Všemohoucího vystihneš?
\par 8 Vyšší jest nebes, což uciníš? Hlubší než peklo, jakž porozumíš?
\par 9 Delší jest míra její než zeme, a širší než more.
\par 10 Bude-li pléniti neb zavírati aneb ssužovati, kdo se na nej bude domlouvati?
\par 11 Ponevadž zná lidskou marnost, a vidí nepravost, což by tomu rozumeti nemel?
\par 12 Tak aby muž nesmyslný nabyl rozumu, ackoli clovek jest jako hrebec z divokého osla zplozený.
\par 13 Jestliže ty nastrojíš srdce své, a ruce své k nemu vztáhneš;
\par 14 Byla-li by nepravost v ruce tvé, vzdal ji od sebe, aniž dopouštej bydliti v staních svých nešlechetnosti:
\par 15 Tedy jiste pozdvihneš tvári své z poškvrny, a budeš nepohnutý, aniž se báti budeš.
\par 16 Nebo se na težkost zapomeneš, na niž jako na vody, kteréž pominuly, zpomínati budeš.
\par 17 K tomu nad poledne jasný nastanet cas; zatmíš-li se pak, jitru podobný budeš.
\par 18 Budeš i mysli doufanlivé, maje nadeji; stánek roztáhneš, i bezpecne spáti budeš.
\par 19 A tak v pokoji budeš, aniž te kdo predesí, a mnozí tvári tvé koriti se budou.
\par 20 Oci pak bezbožných zkaženy budou, a utíkání jim zhyne; nadto nadeje jejich bude jako dchnutí cloveka.

\chapter{12}

\par 1 Odpovedev pak Job, rekl:
\par 2 V pravde, že jste vy lidé, a že s vámi umre moudrost.
\par 3 I ját mám srdce jako vy, aniž jsem zpozdilejší než vy, anobrž pri komž toho není?
\par 4 Za posmech príteli svému jsem, kteréhož, když volá, vyslýchá Buh; v posmechut jest spravedlivý a uprímý.
\par 5 Pochodne zavržená jest (podlé smýšlení cloveka pokoje užívajícího) ten, kterýž jest blízký pádu.
\par 6 Pokojné a bezpecné príbytky mají loupežníci ti, kteríž popouzejí Boha silného, jimž on uvodí dobré veci v ruku jejich.
\par 7 Ano zeptej se trebas hovad, a naucí te, aneb ptactva nebeského, a oznámí tobe.
\par 8 Aneb rozmluv s zemí, a poucí te, ano i ryby morské vypravovati budou tobe.
\par 9 Kdo nezná ze všeho toho, že ruka Hospodinova to ucinila?
\par 10 V jehož ruce jest duše všelikého živocicha, a duch každého tela lidského.
\par 11 Zdaliž ucho slov rozeznávati nebude, tak jako dásne pokrmu okoušejí?
\par 12 Pri starcích jest moudrost, a pri dlouhovekých rozumnost.
\par 13 Nadto pak u Boha moudrost a síla, jehot jest rada a rozumnost.
\par 14 Jestliže on borí, nemuže zase stavíno býti; zavírá-li cloveka, nemuže býti otevríno.
\par 15 Hle, tak zastavuje vody, až i vysychají, a tak je vypouští, že podvracejí zemi.
\par 16 U neho jest síla a bytnost, jeho jest ten, kterýž bloudí, i kterýž v blud uvodí.
\par 17 On uvodí rádce v nemoudrost, a z soudcu blázny ciní.
\par 18 Svazek králu rozvazuje, a pasem prepasuje bedra jejich.
\par 19 On uvodí knížata v nemoudrost, a mocné vyvrací.
\par 20 On odjímá rec výmluvným, a soud starcum bére.
\par 21 On vylévá potupu na urozené, a sílu mocných zemdlívá.
\par 22 On zjevuje hluboké veci z temností, a vyvodí na svetlo stín smrti.
\par 23 On rozmnožuje národy i hubí je, rozširuje národy i zavodí je.
\par 24 On odjímá srdce predním z lidu zeme, a v blud je uvodí na poušti bezcestné,
\par 25 Aby šámali ve tme bez svetla. Summou, ciní, aby bloudili jako opilý.

\chapter{13}

\par 1 Aj, všecko to videlo oko mé, slyšelo ucho mé, a srozumelo tomu.
\par 2 Jakož vy znáte to, znám i já, nejsem zpozdilejší než vy.
\par 3 Jiste žet já s Všemohoucím mluviti, a s Bohem silným o svou pri jednati budu.
\par 4 Nebo vy jste skladatelé lži, a lékari marní všickni vy.
\par 5 Ó kdybyste aspon mlceli, a bylo by vám to za moudrost.
\par 6 Slyštež medle odpory mé, a duvodu rtu mých pozorujte.
\par 7 Zdali zastávajíce Boha silného, mluviti máte nepravost? Aneb za neho mluviti máte lest?
\par 8 Zdaliž osobu jeho prijímati budete, a o Boha silného se zasazovati?
\par 9 Zdaž vám to k dobrému bude, když na prubu vezme vás, že jakož clovek oklamán bývá, oklamati jej chcete?
\par 10 V pravde žet vám prísne domlouvati bude, budete-li povrchne osoby jeho šetriti.
\par 11 Což ani dustojnost jeho vás nepredešuje, ani strach jeho neprikvacuje vás?
\par 12 Všecka vzácnost vaše podobná jest popelu, a hromadám bláta vyvýšení vaše.
\par 13 Postrptež mne, nechat já mluvím, prid na mne cokoli.
\par 14 Procež bych trhati mel maso své zuby svými, a duši svou klásti v ruku svou?
\par 15 By mne i zabil, což bych v neho nedoufal? A však cesty své pred oblícej jeho predložím.
\par 16 Ont sám jest spasení mé; nebo pred oblícej jeho pokrytec neprijde.
\par 17 Poslouchejte pilne reci mé, a zprávu mou pustte v uši své.
\par 18 Aj, jižt zacínám pre své vésti, vím, že zustanu spravedliv.
\par 19 Kdo jest, ješto by mi odpíral, tak abych nyní umlknouti a umríti musil?
\par 20 Toliko té dvoji veci, ó Bože, necin mi, a tehdy pred tvárí tvou nebudu se skrývati:
\par 21 Ruku svou vzdal ode mne, a hruza tvá necht mne nedesí.
\par 22 Zatím povolej mne, a budut odpovídati; aneb nechat já mluvím, a odpovídej mi.
\par 23 Jak mnoho jest mých nepravostí a hríchu? Prestoupení mé a hrích muj ukaž mi.
\par 24 Proc tvár svou skrýváš, a pokládáš mne sobe za neprítele?
\par 25 Zdaliž list vetrem se zmítající potríti chceš, a stéblo suché stihati budeš?
\par 26 Že zapisuješ proti mne horkosti, a dáváš mi v dedictví nepravosti mladosti mé,
\par 27 A dáváš do klady nohy mé, a šetríš všech stezek mých, na paty noh mých našlapuješ;
\par 28 Ješto clovek jako hnis kazí se, a jako roucho, kteréž jí mol.

\chapter{14}

\par 1 Clovek narozený z ženy jest krátkého veku a plný lopotování.
\par 2 Jako kvet vychází a podtat bývá, a utíká jako stín, a netrvá.
\par 3 A však i na takového otvíráš oko své, a mne uvodíš k soudu s sebou.
\par 4 Kdo toho dokáže, aby cistý z necistého pošel? Ani jeden.
\par 5 Ponevadž vymereni jsou dnové jeho, pocet mesícu jeho u tebe, a cíles jemu položil, kterýchž by neprekracoval:
\par 6 Odvrat se od neho, at oddechne sobe, a zatím aby precekal jako nájemník den svuj.
\par 7 O stromu zajisté jest nadeje, by i podtat byl, že se zase zotaví, a výstrelek jeho nevyhyne,
\par 8 By se pak i sstaral v zemi koren jeho, a v prachu již jako umrel pen jeho:
\par 9 Avšak jakž pocije vláhy, zase se pucí, a zahustí jako ker.
\par 10 Ale clovek umírá, mdlobou premožen jsa, a když vypustí duši clovek, kam se podel?
\par 11 Jakož ucházejí vody z jezera, a reka opadá a vysychá:
\par 12 Tak clovek, když lehne, nevstává zase dotud, dokudž nebes stává. Nebývajít vzbuzeni lidé, aniž se probuzují ze sna svého.
\par 13 Ó kdybys mne v hrobe schoval, a skryl mne, dokudž by nebyl odvrácen hnev tvuj, ulože mi cíl, abys se rozpomenul na mne.
\par 14 Když umre clovek, zdaliž zase ožive? Po všecky tedy dny vymereného casu svého ocekávati budu, až prijde promena pri mne.
\par 15 Zavoláš, a já se ohlásím tobe, díla rukou svých budeš žádostiv,
\par 16 Ackoli nyní kroky mé pocítáš, aniž shovíváš hríchum mým,
\par 17 Ale zapecetené maje jako v pytlíku prestoupení mé, ješte prikládáš k nepravosti mé.
\par 18 Jiste že jako hora padnuc, rozdrobuje se, a skála odsedá z místa svého,
\par 19 Jako kamení stírá voda, a povodní zachvacuje, což z prachu zemského samo od sebe roste: tak i ty nadeji cloveka v nic obracíš.
\par 20 Premáháš jej ustavicne, tak aby odjíti musil; promenuješ tvár jeho, a propouštíš jej.
\par 21 Budou-li slavní synové jeho, nic neví; pakli v potupe, nic o ne nepecuje.
\par 22 Toliko telo jeho, dokudž živ jest, bolestí okouší, a duše jeho v nem kvílí.

\chapter{15}

\par 1 Tedy odpovídaje Elifaz Temanský, rekl:
\par 2 Zdali moudrý vynášeti má umení povetrné, aneb naplnovati východním vetrem bricho své,
\par 3 Hádaje se slovy neprospešnými, aneb recmi neužitecnými?
\par 4 Anobrž vyprazdnuješ i bázen Boží, a modliteb k Bohu ciniti se zbranuješ.
\par 5 Osvedcujít zajisté nepravost tvou ústa tvá, ac jsi koli sobe zvolil jazyk chytrých.
\par 6 Potupují te ústa tvá, a ne já, a rtové tvoji svedcí proti tobe.
\par 7 Zdaliž ty nejprv z lidí zplozen jsi, aneb prvé než pahrbkové sformován?
\par 8 Zdaliž jsi tajemství Boží slyšel, že u sebe zavíráš moudrost?
\par 9 Co víš, cehož bychom nevedeli? Cemu rozumíš, aby toho pri nás nebylo?
\par 10 I šedivýt i starec mezi námi jest, ano i starší vekem než otec tvuj.
\par 11 Zdali malá jsou tobe potešování Boha silného, cili neco je zastenuje tobe?
\par 12 Tak-liž te jalo srdce tvé, a tak-liž blíkají oci tvé,
\par 13 Že smíš odpovídati Bohu silnému tak pyšne, a vypoušteti z úst svých ty reci?
\par 14 Nebo což jest clovek, aby se mohl ocistiti, aneb spravedliv býti narozený z ženy?
\par 15 An pri svatých jeho není dokonalosti, a nebesa nejsou cistá pred ocima jeho,
\par 16 Nadto ohavný a neužitecný clovek, kterýž pije nepravost jako vodu.
\par 17 Já oznámím tobe, poslyš mne; to zajisté, což jsem videl, vypravovati budu,
\par 18 Což moudrí vynesli a nezatajili, slýchavše od predku svých.
\par 19 Jimž samým dána byla zeme, aniž prejíti mohl cizí prostredkem jejich.
\par 20 Po všecky své dny bezbožný sám se bolestí trápí, po všecka, pravím léta, skrytá pred ukrutníkem.
\par 21 Zvuk strachu jest v uších jeho, že i v cas pokoje zhoubce pripadne na nej.
\par 22 Neverí, by se mel navrátiti z temností, ustavicne ocekávaje na sebe mece.
\par 23 Bývá i tulákem, chleba hledaje, kde by byl, cíte, že pro nej nastrojen jest den temností.
\par 24 Desí jej nátisk a ssoužení, kteréž se silí proti nemu, jako král s vojskem sšikovaným.
\par 25 Nebo vztáhl proti Bohu silnému ruku svou, a proti Všemohoucímu postavil se.
\par 26 Útok ucinil na nej, na šíji jeho s množstvím zdvižených štítu svých.
\par 27 Nebo priodíl tvár svou tukem svým, tak že se mu nadelalo faldu na slabinách.
\par 28 A bydlil v mestech zkažených, a v domích, v nichž žádný nebydlil, kteráž v hromady rumu obrácena byla.
\par 29 Avšak nezbohatnet, aniž stane moc jeho, aniž se rozšírí na zemi dokonalost takových.
\par 30 Nevyjde z temností, mladistvou ratolest jeho usuší plamen, a tak zahyne od ducha úst svých.
\par 31 Ale neverí, že v marnosti jest ten, jenž bloudí, a že marnost bude smena jeho.
\par 32 Pred casem svým vytat bude, a ratolest jeho nebude se zelenati.
\par 33 Zmarí, jako vinný kmen nezralý hrozen svuj, a svrže kvet svuj jako oliva.
\par 34 Nebo shromáždení pokrytce spustne, a ohen spálí stany oslepených dary.
\par 35 Kterížto když pocali ssužování, a porodili nepravost, hned bricho jejich strojí jinou lest.

\chapter{16}

\par 1 A odpovídaje Job, rekl:
\par 2 Slyšel jsem již podobných vecí mnoho; všickni vy nepríjemní jste tešitelé.
\par 3 Bude-liž kdy konec slovum povetrným? Aneb co te popouzí, že tak mluvíš?
\par 4 Zdaliž bych já tak mluviti mohl, jako vy, kdybyste byli na míste mém? Shromaždoval-li bych proti vám slova, aneb potrásal na vás hlavou svou?
\par 5 Nýbrž posiloval bych vás ústy svými, a otvírání rtu mých krotilo by bolest.
\par 6 Bud že mluvím, neumenšuje se bolesti mé, bud že tak nechám, neodchází ode mne.
\par 7 Ale ustavicne zemdlívá mne; nebo jsi mne, ó Bože, zbavil všeho shromáždení mého.
\par 8 A vrásky jsi mi zdelal; což mám za svedka, ano patrná na mne hubenost má na tvári mé to osvedcuje.
\par 9 Prchlivost jeho zachvátila mne, a vzal mne v nenávist, škripe na mne zuby svými; jako neprítel muj zaostril oci své na mne.
\par 10 Rozedreli na mne ústa svá, potupne mne polickujíce, proti mne se shromáždivše.
\par 11 Vydal mne Buh silný nešlechetníku, a v ruce bezbožných uvedl mne.
\par 12 Pokoje jsem užíval, však potrel mne, a uchopiv mne za šíji mou, roztríštil mne, a vystavil mne sobe za cíl.
\par 13 Obklícili mne strelci jeho, roztal ledví má beze vší lítosti, a vylil na zem žluc mou.
\par 14 Ranil mne ranou na ránu, outok ucinil na mne jako silný.
\par 15 Žíni jsem ušil na zjízvenou kuži svou, a zohavil jsem v prachu sílu svou.
\par 16 Tvár má oduravela od pláce, a na víckách mých stín smrti jest.
\par 17 Ne pro nejaké bezpraví v rukou mých; nebo i modlitba má cistá jest.
\par 18 Ó zeme, neprikrývej krve mé, a necht nemá místa volání mé.
\par 19 Aj, nyní jestit i v nebesích svedek muj, svedek muj, pravím, jest na výsostech.
\par 20 Ó mudráci moji, prátelé moji, k Bohut slzí oko mé.
\par 21 Ó by lze bylo muži v hádku s ním se vydati, jako synu cloveka s prítelem svým.
\par 22 Nebo léta mne odectená pricházejí, a cestou, kterouž se zase nenavrátím, již se beru.

\chapter{17}

\par 1 Dýchání mé ruší se, dnové moji hynou, hrobu blízký jsem.
\par 2 Jiste posmevaci jsou u mne, a pro jejich mne kormoucení neprichází ani sen na oci mé.
\par 3 Postav mi, prosím, rukojme za sebe; kdo jest ten, necht mi na to ruky podá.
\par 4 Nebo srdce jejich prikryl jsi, aby nerozumeli, a protož jich nepovýšíš.
\par 5 Kdož pochlebuje bližním, oci synu jeho zhynou.
\par 6 Jiste vystavil mne za prísloví lidem, a za divadlo všechnem,
\par 7 Tak že pro žalost pošly oci mé, a oudové moji všickni stínu jsou podobni.
\par 8 Užasnout se nad tím uprímí, a však nevinný proti pokrytci vždy se zsilovati bude.
\par 9 Prídržeti se bude, pravím, spravedlivý cesty své, a ten, jenž jest cistých rukou, posilní se více.
\par 10 Tolikéž i vy všickni obratte se, a podte, prosím; nebot nenacházím mezi vámi moudrého.
\par 11 Dnové moji pomíjejí, myšlení má mizejí, premyšlování, pravím, srdce mého.
\par 12 Noc mi obracejí v den, a svetla denního ukracují pro prítomnost temností.
\par 13 Abych pak ceho i ocekával, hrob bude dum muj, ve tme usteli ložce své.
\par 14 Jámu nazovu otcem svým, matkou pak a sestrou svou cervy.
\par 15 Kdež jest tedy ocekávání mé? A kdo to, cím bych se troštoval, spatrí?
\par 16 Do skrýší hrobu sstoupí, ponevadž jest všechnem v prachu zeme odpocívati.

\chapter{18}

\par 1 Odpovídaje pak Bildad Suchský, rekl:
\par 2 Dokudž neuciníte konce recem? Pomyslte na to, a potom mluviti budeme.
\par 3 Proc jsme pocteni za hovada? Oškliví jsme jemu, jakž sami vidíte.
\par 4 Ó ty, jenž hubíš život svuj zurením svým, zdaliž pro tebe opuštena bude zeme, a odsedne skála z místa svého?
\par 5 Anobrž svetlo bezbožných uhašeno bude, aniž se blyšteti bude jiskra ohne jejich.
\par 6 Svetlo se zatmí v stánku jeho, a lucerna jeho v nem zhasne.
\par 7 Ssouženi budou krokové síly jeho, a porazí jej rada jeho.
\par 8 Nebo zapleten jest do síti nohami svými, a v zamotání chodí.
\par 9 Chytí ho za patu osídlo, a zmocní se ho násilník.
\par 10 Skrytat jest pri zemi smecka jeho, a lécka jeho na stezce.
\par 11 Odevšad hruzy jej desiti budou a dotírati na nohy jeho.
\par 12 Hladovitá bude síla jeho, a bída pohotove pri boku jeho.
\par 13 Zžíre žily kuže jeho, zžíre oudy jeho kníže smrti.
\par 14 Uchváceno bude z stánku jeho doufání jeho, a to jej privede k králi strachu.
\par 15 V stánku jeho hruza bydleti bude, ac nebyl jeho; posypáno bude obydlí jeho sirou.
\par 16 Od zpodku korenové jeho uschnou, a svrchu osekány budou ratolesti jeho.
\par 17 Památka jeho zahyne z zeme, aniž jméno jeho slýcháno bude na ulicích.
\par 18 Vyženou ho z svetla do tmy, anobrž z okršlku zemského vypudí jej.
\par 19 Nepozustaví ani syna ani vnuka v lidu svém, ani jakého ostatku v príbytcích svých.
\par 20 Nade dnem jeho zdesí se potomci, a prítomní strachem podjati budou.
\par 21 Takovýt jest zajisté zpusob nešlechetného, a takový cíl toho, kterýž nezná Boha silného.

\chapter{19}

\par 1 Tedy odpovedev Job, rekl:
\par 2 Dokudž trápiti budete duši mou, a dotírati na mne recmi svými?
\par 3 Již na desetkrát zhaneli jste mne, aniž se stydíte, že se zatvrzujete proti mne.
\par 4 Ale necht jest tak, že jsem zbloudil, pri mne zustane blud muj.
\par 5 Jestliže se pak vždy proti mne siliti chcete, a obvinujíc mne, za pomoc sobe bráti proti mne potupu mou:
\par 6 Tedy vezte, že Buh podvrátil mne, a sítí svou otáhl mne.
\par 7 Nebo aj, volám-li pro nátisk, nemám vyslyšení; kricím-li, není rozsouzení.
\par 8 Cestu mou zapletl tak, abych nikoli projíti nemohl, a stezky mé temnostmi zastrel.
\par 9 Slávu mou se mne strhl, a snal korunu s hlavy mé.
\par 10 Zpodvracel mne všudy vukol, abych zahynul, a vyvrátil jako strom nadeji mou.
\par 11 Nadto zažžel proti mne prchlivost svou, a prictl mne mezi neprátely své.
\par 12 Procež pritáhše houfové jeho, ucinili sobe ke mne cestu, a vojensky se položili okolo stanu mého.
\par 13 Bratrí mé ode mne vzdálil, a známí moji všelijak se mne cizí.
\par 14 Opustili mne príbuzní moji, a známí moji zapomenuli se na mne.
\par 15 Podruhové domu mého a devky mé za cizího mne mají, cizozemec jsem pred ocima jejich.
\par 16 Na služebníka svého volám, ale neozývá se, i když ho ústy svými pekne prosím.
\par 17 Dýchání mého štítí se manželka má, ackoli pokorne jí prosím, pro dítky života mého.
\par 18 Nadto i ti nejšpatnejší pohrdají mnou; i když povstanu, utrhají mi.
\par 19 V ošklivost mne sobe vzali všickni rádcové moji, a ti, kteréž miluji, obrátili se proti mne.
\par 20 K kuži mé jako k masu mému prilnuly kosti mé, kuže pri zubích mých toliko v cele zustala.
\par 21 Slitujte se nade mnou, slitujte se nade mnou, vy prátelé moji; nebo ruka Boží se mne dotkla.
\par 22 Proc mi se protivíte tak jako Buh silný, a masem mým nemužte se nasytiti?
\par 23 Ó kdyby nyní sepsány byly reci mé, ó kdyby v knihu vepsány byly,
\par 24 Anobrž rafijí železnou a olovem na vecnost na skále aby vyryty byly.
\par 25 Ackoli já vím, že vykupitel muj živ jest, a že v den nejposlednejší nad prachem se postaví.
\par 26 A ac by kuži mou i telo cervi zvrtali, však vždy v tele svém uzrím Boha.
\par 27 Kteréhož já uzrím sobe, a oci mé spatrí jej, a ne jiný, jakkoli zhynula ledví má u vnitrnosti mé.
\par 28 Ješto byste ríci meli: I procež ho trápíme? ponevadž základ dobré pre pri mne se nalézá.
\par 29 Bojte se mece, nebo pomsta za nepravosti jest mec, a vezte, žet bude soud.

\chapter{20}

\par 1 Odpovídaje pak Zofar Naamatský, rekl:
\par 2 Z príciny té myšlení má k odpovídání tobe nutí mne, a to abych rychle ucinil,
\par 3 Že kárání k zahanbení svému slyším, procež duch muj osvícený nutí mne, atbych odpovídal.
\par 4 Zdaž nevíš o tom, že od veku, a jakž postavil Buh cloveka na zemi,
\par 5 Plésání bezbožných krátké jest, a veselí pokrytce jen na chvílku?
\par 6 Byt pak vstoupila až k nebi pýcha jeho, a hlava jeho oblaku by se dotkla,
\par 7 Však jako lejno jeho na veky zahyne. Ti, kteríž jej vídali, reknou: Kam se podel?
\par 8 Jako sen pomine, aniž ho naleznou; nebo utece jako videní nocní.
\par 9 Oko, kteréž ho vídalo, již ho nikdy neuzrí, aniž více patriti bude na nej místo jeho.
\par 10 Synové jeho budou prízne u nuzných hledati, a ruce jeho musejí zase vraceti loupež svou.
\par 11 Kosti jeho naplneny jsou hríchy mladosti jeho, a s ním v prachu lehnou.
\par 12 A ackoli zlost sladne v ústech jeho, a chová ji pod jazykem svým;
\par 13 Kochá se v ní, a nepouští jí, ale zdržuje ji u prostred dásní svých:
\par 14 Však pokrm ten ve strevách jeho promení se; bude jako žluc hadu nejlítejších u vnitrnostech jeho.
\par 15 Zboží nahltané vyvrátí, z bricha jeho Buh silný je vyžene.
\par 16 Jed hadu lítých ssáti bude, zabije ho jazyk ještercí.
\par 17 Neuzrí pramenu potoku a rek medu a másla.
\par 18 Navrátí úsilé cizí, a nezažive ho, vedlé nátisku svého rozlicného; nebude na ne vesel.
\par 19 Nebo utiskal a opouštel nuzné, dum zloupil a nestavel ho.
\par 20 Procež nesezná nic pokojného v živote svém, aniž které nejrozkošnejší své veci bude moci zachovati.
\par 21 Nic mu nepozustane z pokrmu jeho, tak že nebude míti, cím by se troštoval.
\par 22 Byt pak i dovršena byla hojnost jeho, ssoužení míti bude; každá ruka trapice oborí se na nej.
\par 23 By mel cím naplniti bricho své, dopustí na nej Buh prchlivost hnevu svého, kterouž na nej dštíti bude i na pokrm jeho.
\par 24 Když utíkati bude pred zbrojí železnou, prostrelí ho lucište ocelivé.
\par 25 Strela vynata bude z toulu a vystrelena, nadto mec pronikne žluc jeho; a když odcházeti bude, prikvací jej hruzy.
\par 26 Všeliká neštestí jsou polecena v skrýších jeho, zžíre jej ohen nerozdmýchaný, zle se povede i pozustalému v stanu jeho.
\par 27 Odkryjí nebesa nepravost jeho, a zeme povstane proti nemu.
\par 28 Rozptýlena bude úroda domu jeho, rozplyne se v den hnevu jeho.
\par 29 Tent jest podíl cloveka bezbožného od Boha, to, pravím, dedictví vyrcené jemu od Boha silného.

\chapter{21}

\par 1 A odpovídaje Job, rekl:
\par 2 Poslouchejte pilne reci mé, a bude mi to za potešení od vás.
\par 3 Postrpte mne, abych i já mluvil, a když odmluvím, posmívejž se.
\par 4 Zdaliž já pred clovekem naríkám? A ponevadž jest proc, jakž nemá býti ssoužen duch muj?
\par 5 Pohledte na mne, a užasnete se, a položte prst na ústa.
\par 6 Ano já sám, když rozvažuji své bídy, tedy se desím, a spopadá telo mé hruza.
\par 7 Proc bezbožní živi jsou, k veku starému pricházejí, též i bohatnou?
\par 8 Síme jejich stálé jest pred oblícejem jejich s nimi, a rodina jejich pred ocima jejich.
\par 9 Domové jejich bezpecni jsou pred strachem, aniž metla Boží na nich.
\par 10 Býk jejich pripouštín bývá, ale ne na prázdno; kráva jejich rodí, a nepotracuje plodu.
\par 11 Vypouštejí jako stádo malické své, a synové jejich poskakují.
\par 12 Povyšují hlasu pri bubnu a harfe, a veselí se k zvuku muziky.
\par 13 Tráví v štestí dny své, a v okamžení do hrobu sstupují.
\par 14 Kteríž ríkají Bohu silnému: Odejdi od nás, nebo známosti cest tvých neoblibujeme.
\par 15 Kdo jest Všemohoucí, abychom sloužili jemu? A jaký toho zisk, že bychom se modlili jemu?
\par 16 Ale pohled, že není v moci jejich štestí jejich, procež rada bezbožných vzdálena jest ode mne.
\par 17 Casto-liž svíce bezbožných hasne? Prichází-liž na ne bída jejich? Podeluje-liž je bolestmi Buh v hneve svém?
\par 18 Bývají-liž jako plevy pred vetrem, a jako drtiny, kteréž zachvacuje vicher?
\par 19 Odkládá-liž Buh synum bezbožníka nepravost jeho? Odplacuje-liž jemu tak, aby to znáti mohl,
\par 20 A aby videly oci jeho neštestí jeho, a prchlivost Všemohoucího že by pil?
\par 21 O dum pak jeho po nem jaká jest péce jeho, když pocet mesícu jeho bude umenšen?
\par 22 Zdali Boha silného kdo uciti bude umení, kterýž sám vysokosti soudí?
\par 23 Tento umírá v síle dokonalosti své, všelijak bezpecný a pokojný.
\par 24 Prsy jeho plné jsou mléka, a mozk kostí jeho svlažován bývá.
\par 25 Jiný pak umírá v horkosti ducha, kterýž nikdy nejídal s potešením.
\par 26 Jednostejne v prachu lehnou, a cervy se rozlezou.
\par 27 Aj, známt myšlení vaše, a chytrosti, kteréž proti mne nepráve vymýšlíte.
\par 28 Nebo pravíte: Kde jest dum urozeného? A kde stánek príbytku bezbožných?
\par 29 Což jste se netázali jdoucích cestou? Zkušení-liž aspon jejich nepovolíte,
\par 30 Že v den neštestí ochranu mívá bezbožný, v den, pravím, rozhnevání pristrín bývá?
\par 31 Kdo jemu oznámí zjevne cestu jeho? Aneb za to, co cinil, kdo jemu odplatí?
\par 32 A však i on k hrobu vyprovozen bude, a tam zustane.
\par 33 Sladnou jemu hrudy údolí, nadto za sebou všecky lidi táhne, tech pak, kteríž ho predešli, není poctu.
\par 34 Hle, jak vy mne marne troštujete, nebo v odpovedech vašich nezustává než faleš.

\chapter{22}

\par 1 A odpovídaje Elifaz Temanský, rekl:
\par 2 Zdaliž Bohu silnému co prospešný býti muže clovek, když sobe nejmoudreji pocíná?
\par 3 Zdaliž se kochá Všemohoucí v tom, že ty se ospravedlnuješ? Aneb má-liž zisk, když bys dokonalé ukázal býti cesty své?
\par 4 Zdali, že by se tebe bál, tresce te, mste nad tebou?
\par 5 Zdali zlost tvá není mnohá? Anobrž není konce nepravostem tvým.
\par 6 Nebo jsi brával základ od bratrí svých bez príciny, a roucha z nahých jsi svlácel.
\par 7 Vody ustalému jsi nepodal, a hladovitému zbranovals chleba.
\par 8 Ale muži boháci prál jsi zeme, tak aby ten, jehož osoba vzácná, v ní sedel.
\par 9 Vdovy pak pouštel jsi prázdné, ackoli ramena sirotku potrína byla.
\par 10 A protož obklicuji te osídla, a desí te strach nenadálý,
\par 11 Aneb tma, abys nevidel, anobrž rozvodnení prikrývá te.
\par 12 Ríkáš: Zdaž Buh není na výsosti nebeské? Ano shlédni vrch hvezd, jak jsou vysoké.
\par 13 Protož pravíš: Jak by vedel Buh silný? Skrze mrákotu-liž by soudil?
\par 14 Oblakové jsou skrýše jeho, tak že nevidí; nebo okršlek nebeský obchází.
\par 15 Šetríš-liž stezky veku predešlého, kterouž kráceli lidé marní?
\par 16 Kteríž vypléneni jsou pred casem, potok vylit jest na základ jejich.
\par 17 Kteríž ríkali Bohu silnému: Odejdi od nás. Což by tedy jim uciniti mel Všemohoucí?
\par 18 On zajisté domy jejich naplnil dobrými vecmi, (ale rada bezbožných vzdálena jest ode mne).
\par 19 Což vidouce spravedliví, veselí se, a nevinný posmívá se jim,
\par 20 Zvlášt když není vypléneno jmení naše, ostatky pak jejich sežral ohen.
\par 21 Privykejž medle s ním choditi, a pokojneji se míti, skrze to prijde tobe všecko dobré.
\par 22 Prijmi, prosím, z úst jeho zákon, a slož reci jeho v srdci svém.
\par 23 Navrátíš-li se k Všemohoucímu, vzdelán budeš, a vzdálíš-li nepravost od stanu svých,
\par 24 Tedy nakladeš na zemi zlata, a místo kamení potocního zlata z Ofir.
\par 25 Nebo bude Všemohoucí nejcistším zlatem tvým, a stríbrem i silou tvou.
\par 26 A tehdáž v Všemohoucím kochati se budeš, a pozdvihna k Bohu tvári své,
\par 27 Pokorne modliti se budeš jemu, a vyslyší te; procež sliby své plniti budeš.
\par 28 Nebo cožkoli zacneš, budet se dariti, anobrž na cestách tvých svítiti bude svetlo.
\par 29 Když jiní sníženi budou, tedy díš: Ját jsem povýšen. Nebo toho, kdož jest ocí ponížených, Buh spasena uciní.
\par 30 Vysvobodí i toho, kterýž není bez viny, vysvobodí, pravím, cistotou rukou tvých.

\chapter{23}

\par 1 Tedy odpovedel \par a rekl:
\par 2 Což vždy predce naríkání mé za zpouru jmíno bude, ješto bída má težší jest nežli lkání mé?
\par 3 Ó bych vedel, kde ho najíti, šel bych až k trunu jeho.
\par 4 Porádne bych pred ním vedl pri, a ústa svá naplnil bych duvody.
\par 5 Zvedel bych, jakými slovy by mne odpovedel, a porozumel bych, co by mi rekl.
\par 6 Zdaliž by podlé veliké síly své rozepri vedl se mnou? Nikoli, nýbrž on sám dal by mi sílu.
\par 7 Tut by uprímý hádati se mohl s ním, a byl bych osvobozen všelijak od soudce svého.
\par 8 Ale aj, pujdu-li uprímo dále, tam ho není; pakli nazpet, nepostihnu ho.
\par 9 By i cím zamestknán byl na levo, predce ho nespatrím; zastre-li se na pravo, ovšem ho neuzrím.
\par 10 Nebo on zná cestu, kteráž jest pri mne; bude-li mne zkušovati, jako zlato se ukáži.
\par 11 Šlepejí zajisté jeho prídržela se noha má, cesty jeho šetril jsem, abych se s ní neuchyloval.
\par 12 Aniž od prikázaní rtu jeho uchýlil jsem se, nýbrž ustaviv se na tom, schované jsem mel reci úst jeho.
\par 13 On pak jestliže pri cem stojí, kdo jej odvrátí? Ano duše jeho cehož jen žádá, toho hned dovodí.
\par 14 A vykoná uložení své o mne; nebo takových príkladu mnoho jest pri nem.
\par 15 Procež pred tvárí jeho desím se; když to rozvažuji, lekám se ho.
\par 16 Buh zajisté zemdlil srdce mé, a Všemohoucí predesil mne,
\par 17 Tak že sotva jsem nezahynul v tech temnostech; nebo pred tvárí mou nezakryl mrákoty.

\chapter{24}

\par 1 Proc by od Všemohoucího nemeli býti skryti casové, a ti, kteríž jej znají, nemeli nevideti dnu jeho?
\par 2 Bezbožnít nyní mezníky prenášejí, a stádo, kteréž mocí zajali, pasou.
\par 3 Osla sirotku zajímají, v zástave berou vola od vdovy.
\par 4 Strkají nuznými s cesty, musejí se vubec skrývati chudí na svete.
\par 5 Aj, oni jako divocí oslové na poušti, vycházejí jako ku práci své, ráno privstávajíce k loupeži; poušt jest chléb jejich i detí jejich.
\par 6 Na cizím poli žnou, a z vinic bezbožníci sbírají.
\par 7 Nahé privodí k tomu, aby nocovati musili bez roucha, nemajíce se cím priodíti na zime.
\par 8 A prívalem v horách moknouce, nemajíce obydlí, k skále se privinouti musejí.
\par 9 Loupí sirotky, kteríž jsou pri prsích, a od chudého základ berou.
\par 10 Nahého opouštejí , že musí choditi bez odevu, a ti, kteríž snášejí snopy, v hladu zustávati.
\par 11 Ti, jenž mezi zdmi jejich olej vytlacují, a presy tlací, žíznejí.
\par 12 Lidé v mestech lkají, a duše zranených volají, Buh pak prítrže tomu neciní.
\par 13 Onit jsou ti, kteríž odporují svetlu, a neznají cest jeho, aniž chodí po stezkách jeho.
\par 14 Na úsvite povstávaje vražedlník, morduje chudého a nuzného, a v noci jest jako zlodej.
\par 15 Tolikéž oko cizoložníka šetrí soumraku, ríkaje: Nespatrít mne žádný, a tvár zakrývá.
\par 16 Podkopávají potme i domy, kteréž sobe ve dne znamenali; nebo nenávidí svetla.
\par 17 Ale hned v jitre prichází na ne stín smrti; když jeden druhého poznati muže, strachu stínu smrti okoušejí.
\par 18 Lehcí jsou na svrchku vody, zlorecené jest jmení jejich na zemi, aniž odcházejí cestou svobodnou.
\par 19 Jako sucho a horko uchvacuje vody snežné, tak hrob ty, jenž hrešili.
\par 20 Zapomíná se na nej život matky, sladne cervum, nebývá více pripomínán, a tak polámána bývá nepravost jako strom.
\par 21 Pripojuje mu neplodnou, kteráž nerodí, a vdove dobre neciní.
\par 22 Zachvacuje silné mocí svou; ostojí-li kdo z nich, bojí se o život svuj.
\par 23 Dává jemu, na cemž by bezpecne spolehnouti mohl, však oci jeho šetrí cest jejich.
\par 24 Bývají zvýšeni ponekud, ale hned jich není; tak jako jiní všickni sníženi, vypléneni, a jako vrškové klasu stínáni bývají.
\par 25 Zdaliž není tak? Kdo na mne dokáže klamu, a v nic obrátí rec mou?

\chapter{25}

\par 1 Tedy odpovídaje Bildad Suchský, rekl:
\par 2 Panování a hruza Boží pusobí pokoj na výsostech jeho.
\par 3 Zdaliž jest pocet vojskum jeho? A nad kým nevzchází svetlo jeho?
\par 4 Jakž by tedy spravedliv býti mohl bídný clovek pred Bohem silným, aneb jak cist býti narozený z ženy?
\par 5 Hle, ani mesíc nesvítil by, ani hvezdy nebyly by cisté pred ocima jeho,
\par 6 Nadto pak smrtelný clovek, jsa jako cerv, a syn cloveka, jako hmyz.

\chapter{26}

\par 1 A odpovídaje Job, rekl:
\par 2 Komu jsi napomohl? Tomu-li, kterýž nemá síly? Toho-lis retoval, kterýž jest bez moci?
\par 3 Komu jsi rady udelil? Nemoudrému-li? Hned jsi základu dostatecne poucil?
\par 4 Komužs ty reci zvestoval? A cí duch vyšel z tebe?
\par 5 Však i mrtvé veci pod vodami a obyvateli jejich sformovány bývají.
\par 6 Odkryta jest propast pred ním, i zahynutí není zakryto.
\par 7 Onte roztáhl pulnocní stranu nad prázdnem, zavesil zemi na nicemž.
\par 8 Zavazuje vody v oblacích svých, aniž se trhá oblak pod nimi.
\par 9 On sám zdržuje stále trun svuj, a roztahuje na nem oblaky své.
\par 10 Cíl vymeril rozlévání se vodám, až do skonání svetla a tmy.
\par 11 Sloupové nebeští tresou se a pohybují od žehrání jeho.
\par 12 Mocí svou rozdelil more, a rozumností svou dutí jeho.
\par 13 Duchem svým nebesa ozdobil, a ruka jeho sformovala hada dlouhého.
\par 14 Aj, tot jsou jen cástky cest jeho, a jak nestižitelné jest i to malicko, což jsme slyšeli o nem. Hrímání pak moci jeho kdo srozumí?

\chapter{27}

\par 1 Potom dále \par vedl rec svou a rekl:
\par 2 Živt jest Buh silný, kterýž zavrhl pri mou, a Všemohoucí, kterýž horkostí naplnil duši mou,
\par 3 Že nikoli, dokudž duše má ve mne bude a duch Boží v chrípích mých,
\par 4 Nebudou mluviti rtové moji nepravosti, a jazyk muj vynášeti lsti.
\par 5 Odstup ode mne, abych vás za spravedlivé vysvedcil; dokudž dýchati budu, neodložím uprímosti své od sebe.
\par 6 Spravedlnosti své držím se, aniž se jí pustím; nezahanbít mne srdce mé nikdy.
\par 7 Bude jako bezbožník neprítel muj, a povstávající proti mne jako nešlechetník.
\par 8 Nebo jaká jest nadeje pokrytce, by pak lakomel, když Buh vytrhne duši jeho?
\par 9 Zdaliž volání jeho vyslyší Buh silný, když na nej prijde ssoužení?
\par 10 Zdaliž v Všemohoucím kochati se bude? Bude-liž vzývati Boha každého casu?
\par 11 Ale já ucím vás, v kázni Boha silného jsa, a jak se mám k Všemohoucímu, netajím.
\par 12 Aj, vy všickni to vidíte, procež vždy tedy takovou marnost vynášíte?
\par 13 Ten má podíl clovek bezbožný u Boha silného, a to dedictví ukrutníci od Všemohoucího prijímají:
\par 14 Rozmnoží-li se synové jeho, rozmnoží se pod mec, a rodina jeho nenasytí se chlebem.
\par 15 Pozustalí po nem v smrti pohrbeni budou, a vdovy jeho nebudou ho plakati.
\par 16 Nashromáždí-li jako prachu stríbra, a jako bláta najedná-li šatu:
\par 17 Co najedná, to spravedlivý oblece, a stríbro nevinný rozdelí.
\par 18 Vystaví-li jako Arktura dum svuj, bude však jako bouda, kterouž udelal strážný.
\par 19 Bohatý když umre, nebude pochován; pohledí nekdo, ant ho není.
\par 20 Postihnou jej hruzy jako vody, v noci kradmo zachvátí ho vicher.
\par 21 Pochytí jej východní vítr, a odejde, nebo vichricí uchvátí jej z místa jeho.
\par 22 Takové veci na nej dopustí Buh bez lítosti, ackoli pred rukou jeho prudce utíkati bude.
\par 23 Tleskne nad ním každý rukama svýma, a ckáti bude z místa svého.

\chapter{28}

\par 1 Mát zajisté stríbro prameny své, a zlato místo k prehánení.
\par 2 Železo z zeme vzato bývá, a kámen rozpuštený dává med.
\par 3 Cíl ukládá temnostem, a všelikou dokonalost clovek vystihá, kámen mrákoty a stínu smrti.
\par 4 Protrhuje se reka na obyvatele, tak že ji nemuže žádný prebresti, a svozována bývá umením smrtelného cloveka, i odchází.
\par 5 Z zeme vychází chléb, ackoli pod ní jest neco rozdílného, podobného k ohni.
\par 6 V nekteré zemi jest kamení zafirové a prach zlatý,
\par 7 K cemuž stezky nezná žádný pták, aniž ji spatrilo oko lunáka,
\par 8 Kteréž nešlapala mladá zver, aniž šel po ní lev.
\par 9 K škremeni vztahuje ruku svou, a z korene prevrací hory.
\par 10 Z skálí vyvodí potucky, a všecko, což jest drahého, spatruje oko jeho.
\par 11 Vylévati se rekám zbranuje, a tak cožkoli skrytého jest, na svetlo vynáší.
\par 12 Ale moudrost kde nalezena bývá? A kde jest místo rozumnosti?
\par 13 Neví smrtelný clovek ceny její, aniž bývá nalezena v zemi živých.
\par 14 Propast praví: Není ve mne, more také dí: Není u mne.
\par 15 Nedává se zlata cistého za ni, aniž odváženo bývá stríbro za smenu její.
\par 16 Nemuže býti cenena za zlato z Ofir, ani za onychin drahý a zafir.
\par 17 Nevrovná se jí zlato ani drahý kámen, aniž smenena býti muže za nádobu z ryzího zlata.
\par 18 Korálu pak a perel se nepripomíná; nebo nabytí moudrosti dražší jest nad klénoty.
\par 19 Není jí rovný v cene smaragd z Mourenínské zeme, aniž za cisté zlato muže cenena býti.
\par 20 Odkudž tedy moudrost prichází? A kde jest místo rozumnosti?
\par 21 Ponevadž skryta jest pred ocima všelikého živého, i pred nebeským ptactvem ukryta jest.
\par 22 Zahynutí i smrt praví: Ušima svýma slyšely jsme povest o ní.
\par 23 Sám Buh rozumí ceste její, a on ví místo její.
\par 24 Nebo on konciny zeme spatruje, a všecko, což jest pod nebem, vidí,
\par 25 Tak že vetru váhu dává, a vody v míru odvažuje.
\par 26 On též vymeruje dešti právo, i cestu blýskání hromu.
\par 27 Hned tehdáž videl ji, a rozhlásil ji,pripravil ji, a vystihl ji.
\par 28 Cloveku pak rekl: Aj, bázen Páne jest moudrost, a odstoupiti od zlého rozumnost.

\chapter{29}

\par 1 Ješte dále \par vedl rec svou, a rekl:
\par 2 Ó bych byl jako za casu predešlých, za dnu, v nichž mne Buh zachovával,
\par 3 Dokudž svítil svící svou nad hlavou mou, pri jehož svetle chodíval jsem v temnostech,
\par 4 Tak jako jsem byl za dnu mladosti své, dokudž prívetivost Boží byla v stanu mém,
\par 5 Dokudž ješte Všemohoucí byl se mnou, a všudy vukol mne dítky mé,
\par 6 Když šlepeje mé máslem oplývaly, a skála vylévala mi prameny oleje,
\par 7 Když jsem vycházel k bráne skrze mesto, a na ulici strojíval sobe stolici svou.
\par 8 Jakž mne spatrovali mládenci, skrývali se, starci pak povstávali a stáli.
\par 9 Knížata choulili se v recech, anobrž ruku kladli na ústa svá.
\par 10 Hlas vývod se tratil, a jazyk jejich lnul k dásním jejich.
\par 11 Nebo ucho slyše, blahoslavilo mne, a oko vida, posvedcovalo mi,
\par 12 Že vysvobozuji chudého volajícího, a sirotka, i toho, kterýž nemá spomocníka.
\par 13 Požehnání hynoucího pricházelo na mne, a srdce vdovy k plésání jsem vzbuzoval.
\par 14 V spravedlnost jsem se oblácel, a ona ozdobovala mne; jako plášt a koruna byl soud muj.
\par 15 Místo ocí býval jsem slepému, a místo noh kulhavému.
\par 16 Byl jsem otcem nuzných, a na pri, jíž jsem nebyl povedom, vyptával jsem se.
\par 17 A tak vylamoval jsem trenovní zuby nešlechetníka, a z zubu jeho vyrážel jsem loupež.
\par 18 A protož jsem ríkal: V hnízde svém umru, a jako písek rozmnožím dny.
\par 19 Koren muj rozloží se pri vodách, a rosa nocovati bude na ratolestech mých.
\par 20 Sláva má mladnouti bude pri mne, a lucište mé v ruce mé obnovovati se.
\par 21 Poslouchajíce, cekali na mne, a prestávali na rade mé.
\par 22 Po slovu mém nic nemenili, tak na ne dštila rec má.
\par 23 Nebo ocekávali mne jako dešte, a ústa svá otvírali jako k prívalu žádostivému.
\par 24 Žertoval-li jsem s nimi, neverili; procež u vážnosti mne míti neoblevovali.
\par 25 Prišel-li jsem kdy k nim, sedal jsem na predním míste, a tak bydlil jsem jako král v vojšte, když smutných potešuje.

\chapter{30}

\par 1 Nyní pak posmívají se mi mladší mne, jejichž bych otcu nechtel byl postaviti se psy stáda svého.
\par 2 Ac síla rukou jejich k cemu by mi byla? Zmarena jest pri nich starost jejich.
\par 3 Nebo chudobou a hladem znuzeni, utíkali na planá, tmavá, soukromná a pustá místa.
\par 4 Kteríž trhali zeliny po chrastinách, ano i korení, a jalovec za pokrm byl jim.
\par 5 Z prostred lidí vyháníni byli; povolávali za nimi, jako za zlodejem,
\par 6 Tak že musili bydliti v výmolích potoku, v derách zeme a skálí.
\par 7 V chrastinách rvali, pod trní se shromaždovali,
\par 8 Lidé nejnešlechetnejší, nýbrž lidé bez poctivosti, menší váhy i než ta zeme.
\par 9 Nyní, pravím, jsem jejich písnickou, jsa jim ucinen za prísloví.
\par 10 V ošklivosti mne mají, vzdalují se mne, a na tvár mou nestydí se plvati.
\par 11 Nebo Buh mou vážnost odjal, a ssoužil mne; procež uzdu pred prítomností mou svrhli.
\par 12 Po pravici mládež povstává, nohy mi podrážejí, tak že šlapáním protreli ke mne stezky nešlechetnosti své.
\par 13 Mou pak stezku zkazili, k bíde mé pridali, ac jim to nic nepomuže.
\par 14 Jako širokou mezerou vskakují, a k vyplénení mému valí se.
\par 15 Obrátily se na mne hruzy, stihají jako vítr ochotnost mou, nebo jako oblak pomíjí zdraví mé.
\par 16 A již ve mne rozlila se duše má, pochytili mne dnové trápení mého,
\par 17 Kteréž v noci vrtá kosti mé ve mne; procež ani nervové moji neodpocívají.
\par 18 Odev muj mení se pro násilnou moc bolesti, kteráž mne tak jako obojek sukne mé svírá.
\par 19 Uvrhl mne do bláta, tak že jsem již podobný prachu a popelu.
\par 20 Volám k tobe, ó Bože, a neslyšíš mne; postavuji se, ale nehledíš na mne.
\par 21 Obrátils mi se v ukrutného neprítele, silou ruky své mi odporuješ.
\par 22 Vznášíš mne u vítr, sázíš mne na nej, a k rozplynutí mi privodíš zdravý soud.
\par 23 Nebo vím, že mne k smrti odkážeš, a do domu, do nehož se shromažduje všeliký živý.
\par 24 Jiste žet nevztáhne Buh do hrobu ruky, by pak, když je stírá, i volali.
\par 25 Zdaliž jsem neplakal nad tím, kdož okoušel zlých dnu? Duše má kormoutila se nad nuzným.
\par 26 Když jsem dobrého cekal, prišlo mi zlé; nadál jsem se svetla, ale prišla mrákota.
\par 27 Vnitrností mé zevrely, tak že se ješte neupokojily; predstihli mne dnové trápení.
\par 28 Chodím osmahlý, ne od slunce, povstávaje, i mezi mnohými kricím.
\par 29 Bratrem ucinen jsem draku, a tovaryšem mladých pstrosu.
\par 30 Kuže má zcernala na mne, a kosti mé vyprahly od horkosti.
\par 31 A protož v kvílení obrátila se harfa má, a píštalka má v hlas placících.

\chapter{31}

\par 1 Smlouvu jsem ucinil s ocima svýma, a proc bych hledel na pannu?
\par 2 Nebo jaký jest díl od Boha s hury, aneb dedictví od Všemohoucího s výsosti?
\par 3 Zdaliž zahynutí nešlechetnému a pomsta zázracná cinitelum nepravosti pripravena není?
\par 4 Zdaliž on nevidí cest mých, a všech kroku mých nepocítá?
\par 5 Obíral-li jsem se s neuprímostí, a chvátala-li ke lsti noha má:
\par 6 Necht mne zváží na váze spravedlnosti, a prezví Buh uprímost mou.
\par 7 Uchýlil-li se krok muj s cesty, a za ocima mýma odešlo-li srdce mé, a rukou mých chytila-li se jaká poškvrna:
\par 8 Tedy co naseji, necht jiný sní, a výstrelkové moji at jsou vykoreneni.
\par 9 Jestliže se dalo privábiti srdce mé k žene, a u dverí bližního svého cinil-li jsem úklady:
\par 10 Necht mele jinému žena má, a nad ní at se schylují jiní.
\par 11 Nebot jest to nešlechetnost, a nepravost odsudku hodná.
\par 12 Ohen ten zajisté by až do zahynutí žral, a všecku úrodu mou vykorenil.
\par 13 Nechtel-li jsem státi k soudu s služebníkem svým aneb devkou svou v rozepri jejich se mnou?
\par 14 Nebo co bych cinil, kdyby povstal Buh silný? A kdyby vyhledával, co bych odpovedel jemu?
\par 15 Zdali ten, kterýž mne v briše ucinil, neucinil i jeho? A nesformoval nás hned v živote jeden a týž?
\par 16 Odeprel-li jsem žádosti nuzných, a oci vdovy jestliže jsem kormoutil?
\par 17 A jedl-li jsem skyvu svou sám, a nejedl-li i sirotek z ní?
\par 18 Ponevadž od mladosti mé rostl se mnou jako u otce, a od života matky své býval jsem vdove za vudce.
\par 19 Díval-li jsem se na koho, že by hynul, nemaje šatu, a nuzný že by nemel odevu?
\par 20 Nedobrorecila-li mi bedra jeho, že rounem beranu mých se zahrel?
\par 21 Opráhl-li jsem na sirotka rukou svou, když jsem v bráne videl pomoc svou:
\par 22 Lopatka má od svých plecí necht odpadne, a ruka má z kloubu svého at se vylomí.
\par 23 Nebo jsem se bál, aby mne Buh nesetrel, jehož bych velebnosti nikoli neznikl.
\par 24 Skládal-li jsem v zlate nadeji svou, aneb hrude zlata ríkal-li jsem: Doufání mé?
\par 25 Veselil-li jsem se z toho, že bylo rozmnoženo zboží mé, a že ho množství nabyla ruka má?
\par 26 Hledel-li jsem na svetlost slunce svítícího, a na mesíc spanile chodící,
\par 27 Tak že by se tajne dalo svésti srdce mé, a že by líbala ústa má ruku mou?
\par 28 I tot by byla nepravost odsudku hodná; nebot bych tím zapíral Boha silného nejvyššího.
\par 29 Radoval-li jsem se z neštestí toho, kterýž mne nenávidel, a plésal-li jsem, když se mu zle vedlo?
\par 30 Nedopustilt jsem zajisté hrešiti ani ústum svým, abych zlorecení žádal duši jeho.
\par 31 Jestliže neríkala celádka má: Ó by nám dal nekdo masa toho; nemužeme se ani najísti?
\par 32 Nebo vne nenocoval host, dvére své pocestnému otvíral jsem.
\par 33 Prikrýval-li jsem jako jiní lidé prestoupení svá, skrývaje v skrýši své nepravost svou?
\par 34 A ac bych byl mohl škoditi množství velikému, ale pohanení rodu desilo mne; protož jsem mlcel, nevycházeje ani ze dverí.
\par 35 Ó bych mel toho, kterýž by mne vyslyšel. Ale aj, totot jest znamení mé: Všemohoucí sám bude odpovídati za mne, a kniha, kterouž sepsal odpurce muj.
\par 36 Vít Buh, nenosil-li bych ji na rameni svém, neotocil-li bych ji sobe místo koruny.
\par 37 Pocet kroku svých oznámil bych jemu, jako kníže priblížil bych se k nemu.
\par 38 Jestliže proti mne zeme má volala, tolikéž i záhonové její plakali,
\par 39 Jídal-li jsem úrody její bez penez, a duši držitelu jejich privodil-li jsem k vzdychání:
\par 40 Místo pšenice necht vzejde trní, a místo jecmene koukol.
\par 41 Skonávají se slova Jobova.

\chapter{32}

\par 1 A když prestali ti tri muži odpovídati Jobovi, proto že se spravedlivý sobe zdál,
\par 2 Tedy rozpáliv se hnevem Elihu, syn Barachele Buzitského z rodu Syrského, na Joba, rozhneval se, proto že spravedlivejší pravil býti duši svou nad Boha.
\par 3 Ano i na ty tri z prátel jeho roznítil se hnev jeho, proto že nenalézajíce odpovedi, však potupovali Joba.
\par 4 Nebo Elihu ocekával na Joba a na ne s recí, proto že starší byli vekem než on.
\par 5 Ale vida Elihu, že nebylo žádné odpovedi v ústech tech trí mužu, zažhl se v hneve svém.
\par 6 I mluvil Elihu syn Barachele Buzitského, rka: Já jsem nejmladší, vy pak jste starci, procež ostýchaje se, nesmel jsem vám oznámiti zdání svého.
\par 7 Myslil jsem: Starí mluviti budou, a mnoho let mající v známost uvedou moudrost.
\par 8 Ale vidím, že Duch Boží v cloveku a nadšení Všemohoucího ciní lidi rozumné.
\par 9 Slavní ne vždycky jsou moudrí, aniž starci vždycky rozumejí soudu.
\par 10 A protož pravím: Poslouchejte mne, oznámím i já také zdání své.
\par 11 Aj, ocekával jsem na slova vaše, poslouchal jsem duvodu vašich dotud, dokudž jste vyhledávali reci,
\par 12 A bedlive vás soude, spatril jsem, že žádného není, kdo by Joba premohl, není z vás žádného, ješto by odpovídal recem jeho.
\par 13 Ale díte snad: Nalezli jsme moudrost, Buh silný stihá jej, ne clovek.
\par 14 Odpovím: Ac \par neobracel proti mne reci, a však slovy vašimi nebudu jemu odpovídati.
\par 15 Bojí se, neodpovídají více, zavrhli od sebe slova.
\par 16 Cekal jsem zajisté, však ponevadž nemluví, ale mlcí, a neodpovídají více,
\par 17 Odpovím i já také za sebe, oznámím zdání své i já.
\par 18 Nebo pln jsem recí, tesno ve mne duchu života mého.
\par 19 Aj, bricho mé jest jako mest nemající pruduchu, jako sudové noví rozpuklo by se.
\par 20 Mluviti budu, a vydchnu sobe, otevru rty své, a odpovídati budu.
\par 21 Nebudut pak šetriti osoby žádného, a k cloveku bez promenování jména mluviti budu.
\par 22 Nebo neumím jmen promenovati, nebo tudíž by mne zachvátil stvoritel muj.

\chapter{33}

\par 1 Slyšiž tedy, prosím, Jobe, reci mé, a všech slov mých ušima svýma pozoruj.
\par 2 Aj, jižt otvírám ústa svá, mluví jazyk muj v ústech mých.
\par 3 Uprímost srdce mého a umení vynesou rtové moji.
\par 4 Duch Boha silného ucinil mne, a dchnutí Všemohoucího dalo mi život.
\par 5 Mužeš-li, odpovídej mi, priprav se proti mne, a postav se.
\par 6 Aj, já podlé žádosti tvé budut místo Boha silného; z bláta sformován jsem i já.
\par 7 Procež strach ze mne nepredesí te, a ruka má nebudet k obtížení.
\par 8 Rekl jsi pak prede mnou, a hlas ten recí tvých slyšel jsem:
\par 9 Cist jsem, bez prestoupení, nevinný jsem, a nepravosti pri mne není.
\par 10 Aj, príciny ku potrení mne shledal Buh, klade mne sobe za neprítele,
\par 11 Svírá poutami nohy mé, streže všech stezek mých.
\par 12 Aj, tím nejsi spravedliv, odpovídám tobe, nebo vetší jest Buh nežli clovek.
\par 13 Oc se s ním nesnadníš? Žet všech svých vecí nezjevuje?
\par 14 Ano jednou mluví Buh silný, i dvakrát, a nešetrí toho clovek.
\par 15 Skrze sny u videní nocním, když pripadá hluboký sen na lidi ve spaní na ložci,
\par 16 Tehdáž odkrývá ucho lidem, a cemu je ucí, to zpecetuje,
\par 17 Aby odtrhl cloveka od skutku zlého, a pýchu od muže vzdálil,
\par 18 A zachoval duši jeho od jámy, a život jeho aby netrefil na mec.
\par 19 Tresce i bolestí na luži jeho, a všecky kosti jeho násilnou nemocí,
\par 20 Tak že sobe život jeho oškliví pokrm, a duše jeho krmi nejlahodnejší.
\par 21 Hyne telo jeho patrne, a vyhlédají kosti jeho, jichž prvé nebylo vídati.
\par 22 A tak bývá blízká hrobu duše jeho, a život jeho smrtelných ran.
\par 23 Však bude-li míti andela vykladace jednoho z tisíce, kterýž by za cloveka oznámil pokání jeho:
\par 24 Tedy smiluje se nad ním, a dí: Vyprost jej, at nesstoupí do porušení, oblíbilt jsem mzdu vyplacení.
\par 25 I odmladne telo jeho nad dítecí, a navrátí se ke dnum mladosti své.
\par 26 Koriti se bude Bohu, a zamiluje jej, a patriti bude na nej tvárí ochotnou; nadto navrátí cloveku spravedlnost jeho.
\par 27 Kterýž hlede na lidi, rekne: Zhrešilt jsem byl, a to, což pravého bylo, prevrátil jsem, ale nebylo mi to prospešné.
\par 28 Buh však vykoupil duši mou, aby nešla do jámy, a život muj, aby svetlo spatroval.
\par 29 Aj, všeckot to delá Buh silný dvakrát i trikrát pri cloveku,
\par 30 Aby odvrátil duši jeho od jámy, a aby osvícen byl svetlem živých.
\par 31 Pozoruj, Jobe, poslouchej mne, mlc, at já mluvím.
\par 32 Jestliže máš slova, odpovídej mi, nebo bych chtel ospravedlniti tebe.
\par 33 Pakli nic, ty mne poslouchej; mlc, a poucím te moudrosti.

\chapter{34}

\par 1 Ješte mluvil Elihu, a rekl:
\par 2 Poslouchejte, moudrí, recí mých, a rozumní, ušima pozorujte.
\par 3 Nebo ucho recí zkušuje, tak jako dásne okoušejí pokrmu.
\par 4 Soud sobe zvolme, a vyhledejme mezi sebou, co by bylo dobrého.
\par 5 Nebo rekl Job: Spravedliv jsem, a Buh silný zavrhl pri mou.
\par 6 Své-liž bych pre ukrývati mel? Preplnena jest bolestí rána má bez provinení.
\par 7 Který muž jest podobný Jobovi, ješto by pil posmech jako vodu?
\par 8 A že by všel v tovaryšství s ciniteli nepravosti, a chodil by s lidmi nešlechetnými?
\par 9 Nebo rekl: Neprospívá to cloveku líbiti se Bohu.
\par 10 A protož, muži rozumní, poslouchejte mne. Odstup od Boha silného nešlechetnost a od Všemohoucího nepravost.
\par 11 Nebo on podlé skutku cloveka odplací, a podlé toho, jaká jest cí cesta, pusobí, aby to nalézal.
\par 12 A naprosto Buh silný neciní nic nešlechetne, a Všemohoucí neprevrací soudu.
\par 13 Kdo sveril jemu zemi? A kdo zporádal všecken okršlek?
\par 14 Kdyby se na nej obrátil, a ducha jeho i duši jeho k sobe vzal,
\par 15 Umrelo by všeliké telo pojednou, a tak by clovek do prachu se navrátil.
\par 16 Máš-li tedy rozum, poslyš toho, pust v uši své hlas recí mých.
\par 17 Ješto ten, kterýž by v nenávisti mel soud, zdaliž by panovati mohl? Cili toho, jenž jest svrchovane spravedlivý, za nešlechetného vyhlásíš?
\par 18 Zdaliž sluší králi ríci: Ó nešlechetný, a šlechticum: Ó bezbožní?
\par 19 Mnohem méne tomu, kterýž neprijímá osob knížat, aniž u neho má prednost urozený pred nuzným; nebo dílo rukou jeho jsou všickni.
\par 20 V okamžení umírají, trebas o pul noci postrceni bývají lidé, a pomíjejí, a zachvácen bývá silný ne rukou lidskou.
\par 21 Nebo oci jeho hledí na cesty cloveka, a všecky kroky jeho on spatruje.
\par 22 Nenít žádných temností, ani stínu smrti, kdež by se skryli cinitelé nepravosti.
\par 23 Aniž zajisté vzkládá na koho více, tak aby se s Bohem silným souditi mohl.
\par 24 Pyšné stírá bez poctu, a postavuje jiné na místa jejich.
\par 25 Nebo zná skutky jejich; procež na ne obrací noc, a potríni bývají.
\par 26 Jakožto bezbožné rozráží je na míste patrném,
\par 27 Proto že odstoupili od neho, a žádných cest jeho nešetrili,
\par 28 Aby dokázal, že pripouští k sobe krik nuzného, a volání chudých že vyslýchá.
\par 29 (Nebo když on spokojí, kdo znepokojí? A když skryje tvár svou, kdo jej spatrí?)Tak celý národ, jako i každého cloveka jednostejne,
\par 30 Aby nekraloval clovek pokrytý, aby nebylo lidem ourazu.
\par 31 Jiste žet k Bohu silnému radeji toto mluveno býti má: Ponesut, nezruším.
\par 32 Mimo to, nevidím-li ceho, ty vyuc mne; jestliže jsem nepravost páchal, neuciním toho víc.
\par 33 Nebo zdali vedlé tvého zdání odplacovati má, že bys ty toho neliboval, že bys ono zvoloval, a ne on? Pakli co víš jiného, mluv.
\par 34 Muži rozumní se mnou reknou, i každý moudrý poslouchaje mne,
\par 35 Že \par hloupe mluví, a slova jeho nejsou rozumná.
\par 36 Ó by zkušen byl \par dokonale, pro odmlouvání nám jako lidem nepravým,
\par 37 Ponevadž k hríchu svému pridává i nešlechetnost, mezi námi také jen chloubu svou vynáší, a rozmnožuje reci své proti Bohu.

\chapter{35}

\par 1 Ješte mluvil Elihu, a rekl:
\par 2 Domníváš-liž se, že jsi to s soudem rekl: Spravedlnost má prevyšuje Boží?
\par 3 Nebo jsi rekl: Co mi prospeje, jaký užitek budu míti, bych i nehrešil?
\par 4 Já odpovím tobe místne, i tovaryšum tvým s tebou.
\par 5 Pohled na nebe a viz, anobrž spatr oblaky, vyšší, než-lis ty.
\par 6 Jestliže bys hrešil, co svedeš proti nemu? A byt se i rozmnožily nešlechetnosti tvé, co mu uškodíš?
\par 7 Budeš-li spravedlivý, ceho mu udelíš? Aneb co z ruky tvé vezme?
\par 8 Každémut cloveku bezbožnost jeho uškodí, a synu cloveka spravedlnost jeho prospeje.
\par 9 Z množství nátisk trpících, kteréž k tomu privodí, aby úpeli a kriceli pro ukrutnost povýšených,
\par 10 Žádný neríká: Kde jest Buh stvoritel muj? Ješto on dává zpev i v noci.
\par 11 On vyucuje nás nad hovada zemská, a nad ptactvo nebeské moudrejší nás ciní.
\par 12 Tehdáž volají-li pro pýchu zlých, nebývají vyslyšáni.
\par 13 A jiste žet ošemetnosti nevyslýchá Buh silný, a Všemohoucí nepatrí na ni.
\par 14 Mnohem méne, jestliže díš: Nepatríš na to. Sám s sebou vejdi v soud pred ním, a doufej v neho.
\par 15 Ale nyní ponevadž nic není tech vecí, navštívil jej hnev jeho; nebo nechce znáti hojnosti této veliké.
\par 16 A protož marne \par otvírá ústa svá, hloupe rozmnožuje reci své.

\chapter{36}

\par 1 Zatím pridal Elihu, a rekl:
\par 2 Postrp mne malicko, a oznámímt šíre; nebot mám ješte, co bych za Boha mluvil.
\par 3 Vynesu smysl svuj zdaleka, a stvoriteli svému privlastním spravedlnost.
\par 4 V pravde, žet nebudou lživé reci mé; zdrave smýšlejícího máš mne s sebou.
\par 5 Aj, Buh silný mocný jest, aniž svých zamítá; silný jest, a srdce udatného.
\par 6 Neobživuje bezbožného, chudým pak k soudu dopomáhá.
\par 7 Neodvrací od spravedlivého ocí svých, nýbrž s králi na stolici sází je na veky, i bývají zvýšeni.
\par 8 Pakli by poutami sevríni byli, zapleteni jsouce provazy ssoužení:
\par 9 Tudy jim v známost uvodí hrích jejich, a že prestoupení jejich se ssilila.
\par 10 A tak otvírá sluch jejich, aby se napravili, anobrž mluví jim, aby se navrátili od nepravosti.
\par 11 Uposlechnou-li a budou-li jemu sloužiti, stráví dny své v dobrém, a léta svá v potešení.
\par 12 Pakli neuposlechnou, od mece sejdou, a pozdychají bez umení.
\par 13 Nebo kteríž jsou necistého srdce, privetšují hnevu, aniž k nemu volají, když by je ssoužil.
\par 14 Protož umírá v mladosti duše jejich, a život jejich s smilníky.
\par 15 Vytrhuje, pravím, ssouženého z jeho ssoužení, a ty, jejichž sluch otvírá, v trápení.
\par 16 A tak by i tebe prenesl z prostredku úzkosti na širokost, kdež není stesnení, a byl by pokojný stul tvuj tukem oplývající.
\par 17 Ale ty zasloužils, abys jako bezbožný souzen byl; soud a právo na te dochází.
\par 18 Jiste strach, aby te neuvrhl Buh u vetší ránu, tak že by jakkoli veliká výplaty mzda, tebe nevyprostila.
\par 19 Zdaliž by sobe co vážil bohatství tvého? Jiste ani nejvýbornejšího zlata, ani jakékoli síly neb moci tvé.
\par 20 Nechvátejž tedy k noci, v kterouž odcházejí lidé na místo své.
\par 21 Hled, abys se neohlédal na marnost, zvoluje ji radeji, nežli ssoužení.
\par 22 Aj, Buh silný nejvyšší jest mocí svou. Kdo jemu podobný ucitel?
\par 23 Kdo jemu vymeril cestu jeho? Kdo jemu smí ríci: Ciníš nepravost?
\par 24 Pametliv bud radeji, abys vyvyšoval dílo jeho, kteréž spatrují lidé,
\par 25 Kteréž, pravím, všickni lidé vidí, na než clovek patrí zdaleka.
\par 26 Nebo Buh silný tak jest veliký, že ho nemužeme poznati, pocet let jeho jest nevystižitelný.
\par 27 On zajisté vyvodí krupeje vod, kteréž vylévají déšt z oblaku jeho,
\par 28 Když se rozpouštejí oblakové, a kropí na mnohé lidi.
\par 29 (Anobrž vyrozumí-li kdo roztažení oblaku, a zvuku stánku jeho,
\par 30 Jak rozprostírá nad ním svetlo své, aneb všecko more prikrývá?
\par 31 Skrze ty veci zajisté tresce lidi, a též dává pokrmu hojnost.
\par 32 Oblaky zakrývá svetlo, a prikazuje mu ukrývati se za to, co je potkává.)
\par 33 Ohlašuje o nem zvuk jeho, též dobytek, a to hned, když pára zhuru vstupuje.
\par 34 Takét se i nad tím desí srdce mé, až se pohybuje z místa svého.

\chapter{37}

\par 1 Poslouchejte pilne hrmotného hlasu jeho, a zvuku z úst jeho pocházejícího.
\par 2 Pode všecka nebesa jej rozprostírá, a svetlo své k krajum zeme.
\par 3 Za nímž zvucí hlukem, a hrímá hlasem dustojnosti své, aniž mešká s jinými vecmi, když se slýchá hlas jeho.
\par 4 Buh silný hrímá hlasem svým predivne, ciní veliké veci, a však nemužeme rozumeti, jak.
\par 5 Snehu zajisté ríká: Bud na zemi, tolikéž pršce deštové, ano i prívalu násilnému.
\par 6 Zavírá ruku všelikého cloveka, aby žádný z lidí nemohl konati díla svého.
\par 7 Tehdáž i zver vchází do skrýše, a v peleších svých obývá.
\par 8 Z skrýše vychází vichrice, a od pulnocní strany zima.
\par 9 Dchnutím Buh silný dává mráz, až se široké vody zavírají.
\par 10 Také i pri svlažování zeme pohybuje oblakem, a rozhání mracno svetlem svým.
\par 11 A tentýž sem i tam obrací se moudrostí jeho, aby cinil, což by mu koli prikázal na tvári okršlku zemského.
\par 12 Bud k trestání, neb pro zemi svou, bud k prokazování dobrotivosti, spraví to, že se postaví.
\par 13 Pozorujž toho, Jobe, zastav se a podívej se divum Boha silného.
\par 14 Víš-li, kdy Buh ukládá co o tech vecech, aneb kdy chce osvecovati svetlem oblaky své?
\par 15 Znáš-li, jak se vznášejí oblakové, a jiné divy dokonalého v umeních?
\par 16 A že te roucho tvé zahrívati bude, když Buh zemi pokojnou ciní vetry poledními?
\par 17 Roztahoval-li jsi s ním nebesa trvánlivá, k zrcadlu slitému podobná?
\par 18 Poukaž nám, co bychom rekli jemu; nebo nemužeme ani reci zporádati pro temnost.
\par 19 Zdaž jemu kdo oznámí, co bych já mluvil? Pakli by kdo za mne mluvil, jiste že by byl sehlcen.
\par 20 Ano nyní nemohou patriti lidé na svetlo, když jest jasné na oblacích, když je vítr prochází a vycištuje,
\par 21 Od pulnocní strany s jasnem jako zlato pricházeje, ale v Bohu hroznejší jest sláva.
\par 22 Všemohoucí, jehož vystihnouti nemužeme, ac jest veliký v moci, však soudem a prísnou spravedlností netrápí.
\par 23 Protož bojí se ho lidé; neohlédá se na žádného z tech, kdož jsou moudrého srdce.

\chapter{38}

\par 1 Tedy odpovedel Hospodin Jobovi z vichru, a rekl:
\par 2 Kdož jest to, jenž zatemnuje radu recmi neumelými?
\par 3 Prepaš nyní jako muž bedra svá, a nac se tebe tázati budu, oznam mi.
\par 4 Kdes byl, když jsem zakládal zemi? Povez, jestliže máš rozum.
\par 5 Kdo rozmeril ji, víš-li? Aneb kdo vztáhl pravidlo na ni?
\par 6 Na cem podstavkové její upevneni jsou? Aneb kdo založil úhelný kámen její,
\par 7 Když prozpevovaly spolu hvezdy jitrní, a plésali všickni synové Boží?
\par 8 Aneb kdo zavrel jako dvermi more, když vyšlo z života, a zjevilo se?
\par 9 Když jsem mu položil oblak za odev, a mrákotu místo plének jeho,
\par 10 Když jsem jemu uložil úsudek svuj, pristaviv závory a dvére,
\par 11 I rekl jsem: Až potud vycházeti budeš, a dále nic, tu, pravím, skládati budeš dutí vlnobití svého.
\par 12 Zdaž jsi kdy za dnu svých rozkázal jitru? Ukázal-lis zári jitrní místo její,
\par 13 Aby uchvacovala kraje zeme, a bezbožní aby z ní vymítáni byli?
\par 14 Tak aby promenu prijímala jako vosk pecetní, oni pak aby nedlouho stáli jako roucho,
\par 15 A aby bezbožným zbranováno bylo svetla jejich, a ráme vyvýšené zlámáno bylo?
\par 16 Prišel-lis až k hlubinám morským? A u vnitrnosti propasti chodil-lis?
\par 17 Jsou-li tobe zjeveny brány smrti? A brány stínu smrti videl-lis?
\par 18 Shlédl-lis širokosti zeme? Oznam, jestliže ji znáš všecku.
\par 19 Která jest cesta k obydlí svetla, a které místo temností,
\par 20 Že bys je pojal v meze jeho, ponevadž bys srozumíval stezkám domu jeho?
\par 21 Vedel-lis tehdáž, že jsi mel se naroditi, a pocet dnu tvých jak veliký býti má?
\par 22 Prišel-lis až ku pokladum snehu? A poklady krupobití videl-lis,
\par 23 Kteréž chovám k casu ssoužení, ke dni bitvy a boje?
\par 24 Kterými se cestami rozdeluje svetlo, kteréž rozhání východní vítr po zemi?
\par 25 Kdo rozdelil povodní tok, a cestu blýskání hromovému,
\par 26 Tak aby pršel déšt i na tu zemi, kdež není lidí, na poušt, kdež není cloveka,
\par 27 Aby zapájel místa planá a pustá, a k zrustu privodil trávu mladistvou?
\par 28 Má-liž déšt otce? A kdo plodí krupeje rosy?
\par 29 Z cího života vychází mráz? A jíní nebeské kdo plodí?
\par 30 Až i vody jako v kámen se promenují, a svrchek propasti zamrzá.
\par 31 Zdali zavázati mužeš rozkoše Kurátek, aneb stahování Orionovo rozvázati?
\par 32 Mužeš-li vyvoditi hvezdy polední v cas jistý, aneb Arktura s syny jeho povedeš-li?
\par 33 Znáš-li rád nebes? Mužeš-li spravovati panování jejich na zemi?
\par 34 Mužeš-li pozdvihnouti k oblaku hlasu svého, aby hojnost vod prikryla tebe?
\par 35 Ty-liž vypustíš blýskání, aby vycházela? Zdaliž reknou tobe: Aj ted jsme?
\par 36 Kdo složil u vnitrnostech lidských moudrost? Aneb kdo dal rozumu stižitelnost?
\par 37 Kdo vypravovati bude o nebesích moudre? A láhvice nebeské kdo nastrojuje,
\par 38 Aby svlažená zeme zase stuhnouti mohla, a hrudy se v hromade držely?
\par 39 Honíš-liž ty lvu loupež? A hltavost lvícat naplnuješ-liž,
\par 40 Když se stulují v peleších svých, ustavicne z skrýší cihajíce?
\par 41 Kdo pripravuje krkavci pokrm jeho, když mladí jeho k Bohu silnému volají, a toulají se sem i tam pro nedostatek pokrmu?

\chapter{39}

\par 1 Víš-li, kterého casu rodí kamsíkové, a lan ku porodu pracující spatril-lis?
\par 2 Máš-li v poctu mesíce, kteréž vyplnují? Znáš-li, pravím, cas porodu jejich?
\par 3 Jak se kladou, plod svuj utiskají, a s bolestí ho pozbývají?
\par 4 Jak se zmocnují mladí jejich, i odchovávají picí polní, a vycházejíce, nenavracují se k nim?
\par 5 Kdo propustil zver, aby byla svobodná? A remení divokého osla kdo rozvázal?
\par 6 Jemuž jsem dal pustinu místo domu jeho, a místo príbytku jeho zemi slatinnou.
\par 7 Posmívá se hluku mestskému, a na krikání toho, kdož by jej honil, nic nedbá.
\par 8 To, což nachází v horách, jest pastva jeho; nebo toliko zeliny hledá.
\par 9 Svolí-liž jednorožec, aby tobe sloužil, a u jeslí tvých aby nocoval?
\par 10 Pripráhneš-liž provazem jednorožce k orání? Bude-liž vláceti brázdy za tebou?
\par 11 Zdaž se na nej ubezpecíš, proto že jest veliká síla jeho, a porucíš jemu svou práci?
\par 12 Zdaž se jemu doveríš, že sveze semeno tvé, a na humno tvé shromáždí?
\par 13 Ty-lis dal pávum krídlo pekné, aneb péro cápu neb pstrosu?
\par 14 A že opouští na zemi vejce svá, ackoli je v prachu osedí,
\par 15 Nic nemysle, že by je noha potlaciti, aneb zver polní pošlapati mohla?
\par 16 Tak se zatvrzuje k mladým svým, jako by jich nemel; jako by neužitecná byla práce jeho, tak jest bez starosti.
\par 17 Nebo nedal jemu Buh moudrosti, aniž mu udelil rozumnosti.
\par 18 Casem svým zhuru se vznášeje, posmívá se koni i jezdci jeho.
\par 19 Zdaž ty dáti mužeš koni sílu? Ty-li ozdobíš šíji jeho rehtáním?
\par 20 Zdali jej zastrašíš jako kobylku? Anobrž frkání chrípí jeho strašlivé jest.
\par 21 Kopá dul, a pléše v síle své, vycházeje vstríc i zbroji.
\par 22 Smeje se strachu, aniž se leká, aniž ustupuje zpátkem pred ostrostí mece,
\par 23 Ac i toul na nem chrestí, a blyští se drevce a kopí.
\par 24 S hrmotem a s hnevem kopá zemi, aniž pokojne stojí k zvuku trouby.
\par 25 Anobrž k zvuku trouby rehce, a zdaleka cítí boj, hluk knížat a prokrikování.
\par 26 Zdali podlé rozumu tvého létá jestráb, roztahuje krídla svá na poledne?
\par 27 Zdali k rozkazu tvému zhuru se vznáší orlice, a vysoko se hnízdí?
\par 28 Na skále prebývá, prebývá na špicaté skále jako na hrade,
\par 29 Odkudž hledá pokrmu, kterýž z daleka ocima svýma spatruje.
\par 30 Ano i mladí její strebí krev, a kde tela mrtvá, tu i ona jest.
\par 31 A tak odpovídaje Hospodin Jobovi, rekl:
\par 32 Zdali hádající se s Všemohoucím obviní jej? Kdo chce viniti Boha, necht odpoví na to.
\par 33 Tehdy odpovedel \par Hospodinu a rekl:
\par 34 Aj, chaternýt jsem, což bych odpovídal tobe? Ruku svou kladu na ústa svá.
\par 35 Jednou jsem mluvil, ale nebudu již odmlouvati, nýbrž i podruhé, ale nebudu více pridávati.

\chapter{40}

\par 1 Ješte odpovídaje Hospodin z vichru Jobovi, i rekl:
\par 2 Prepaš nyní jako muž bedra svá, a nac se tebe tázati budu, oznam mi.
\par 3 Zdaliž pak i soud muj zrušiti chceš? Což mne odsoudíš, jen abys se sám ospravedlnil?
\par 4 Cili máš ráme jako Buh silný, a hlasem jako on hrímáš?
\par 5 Ozdobiž se nyní vyvýšeností a dustojností, v slávu a okrasu oblec se.
\par 6 Rozprostri prchlivost hnevu svého, a pohled na všelikého pyšného, a sniž ho.
\par 7 Pohled, pravím, na všelikého pyšného, a sehni jej, anobrž setri bezbožné na míste jejich.
\par 8 Skrej je v prachu spolu, tvár jejich zavež v skryte.
\par 9 A tak i já budu te oslavovati, že te zachovává pravice tvá.
\par 10 Aj, hle slon, jejž jsem jako i tebe ucinil, trávu jí jako vul.
\par 11 Aj, hle moc jeho v bedrách jeho, a síla jeho v pupku bricha jeho.
\par 12 Jak chce, ohání ocasem svým, ackoli jest jako cedr; žily luna jeho jako ratolesti jsou spletené.
\par 13 Kosti jeho jako trouby medené, hnátové jeho jako sochor železný.
\par 14 Ont jest prední z úcinku Boha silného, ucinitel jeho sám na nej doložiti muže mec svuj.
\par 15 Hory zajisté prinášejí mu pastvu, a všecka zver polní hrá tam.
\par 16 V stínu léhá, v soukromí mezi trtím a bahnem.
\par 17 Dríví stín dávající stínem svým jej prikrývá, a vrbí potocní obklicuje jej.
\par 18 Aj, zadržuje reku tak, že nemuže pospíchati; tuší sobe, že požre Jordán v ústa svá.
\par 19 Zdaž kdo pred ocima jeho polapí jej, aneb provazy protáhne chrípe jeho?
\par 20 Vytáhneš-liž velryba udicí, aneb provazem pohríženým až k jazyku jeho?
\par 21 Zdali dáš kroužek na chrípe jeho, aneb hákem probodneš celist jeho?
\par 22 Zdaž se obrátí k tobe s prosbami, aneb mluviti bude tobe lahodne?
\par 23 Uciní-liž smlouvu s tebou? Prijmeš-liž jej za služebníka vecného?
\par 24 Zdaž budeš s ním hráti jako s ptáckem, aneb privážeš jej detem svým?
\par 25 Pristrojí-liž sobe hody z neho spolecníci, a rozdelí-liž jej mezi kupce?
\par 26 Zdaž naplníš háky kuži jeho, a vidlicemi rybárskými hlavu jeho?
\par 27 Vztáhni jen na nej ruku svou, a neuciníš zmínky o boji.
\par 28 Aj, nadeje o polapení jeho mylná jest. Zdaž i k spatrení jeho clovek nebývá poražen?

\chapter{41}

\par 1 Není žádného tak smelého, kdo by jej zbudil, kdož tedy postaví se prede mnou?
\par 2 Kdo mne cím predšel, abych se jemu odplacel? Cožkoli jest pode vším nebem, mé jest.
\par 3 Nebudu mlceti o údech jeho, a o síle výborného sformování jeho.
\par 4 Kdo odkryl svrchek odevu jeho? S dvojitými udidly svými kdo k nemu pristoupí?
\par 5 Vrata úst jeho kdo otevre? Okolo zubu jeho jest hruza.
\par 6 Šupiny jeho pevné jako štítové sevrené velmi tuze.
\par 7 Jedna druhé tak blízko jest, že ani vítr nevchází mezi ne.
\par 8 Jedna druhé se prídrží, a nedelí se.
\par 9 Od kýchání jeho zažžehá se svetlo, a oci jeho jsou jako záre svitání.
\par 10 Z úst jeho jako pochodne vycházejí, a jiskry ohnivé vyskakují.
\par 11 Z chrípí jeho vychází dým, jako z kotla vroucího aneb hrnce.
\par 12 Dýchání jeho uhlí rozpaluje, a plamen z úst jeho vychází.
\par 13 V šíji jeho prebývá síla, a pred ním utíká žalost.
\par 14 Kusové masa jeho drží se spolu; celistvé jest v nem, aniž se rozdrobuje.
\par 15 Srdce jeho tuhé jest jako kámen, tak tuhé, jako úlomek zpodního žernovu.
\par 16 Vyskýtání jeho bojí se nejsilnejší, až se strachem i vycištují.
\par 17 Mec stihající jej neostojí, ani kopí, šíp neb i pancír.
\par 18 Pokládá železo za plevy, ocel za drevo shnilé.
\par 19 Nezahání ho strela, v stéblo obrací se jemu kamení prakové.
\par 20 Za stéblo pocítá strelbu, a posmívá se šermování kopím.
\par 21 Pod ním ostré strepiny, stele sobe na veci špicaté jako na bláte.
\par 22 Pusobí, aby vrelo v hlubine jako v kotle, a kormoutilo se more jako v moždíri.
\par 23 Za sebou patrnou ciní stezku, až sezdá, že propast má šediny.
\par 24 Žádného není na zemi jemu podobného, aby tak ucinen byl bez strachu.
\par 25 Cokoli vysokého jest, za nic pokládá, jest králem nade všemi šelmami.

\chapter{42}

\par 1 Tedy odpovídaje \par Hospodinu, rekl:
\par 2 Vím, že všecko mužeš, a že nemuže prekaženo býti tvému myšlení.
\par 3 Kdo jest to ten, ptáš se, ješto zatemnuje radu Boží tak hloupe? Protož priznávám se, že jsem tomu nerozumel. Divnejšít jsou ty veci nad mou stižitelnost, anižt jich mohu poznati.
\par 4 Vyslýchejž, prosím, když bych koli mluvil; když bych se tebe tázal, oznamuj mi.
\par 5 Tolikot jsem slýchal o tobe, nyní pak i oko mé te vidí.
\par 6 Procež mrzí mne to, a želím toho v prachu a v popele.
\par 7 Stalo se pak, když odmluvil Hospodin slova ta k Jobovi, že rekl Hospodin Elifazovi Temanskému: Rozpálil se hnev muj proti tobe, a proti dvema prátelum tvým, proto že jste nemluvili o mne toho, což pravého jest, tak jako služebník muj Job.
\par 8 Protož nyní vezmete sobe sedm volku, a sedm skopcu, a jdete k služebníku mému Jobovi, abyste dali obetovati obet za sebe, a služebník muj Job, aby se modlil za vás. Nebo jiste oblícej jeho prijmu, abych neucinil s vámi podlé bláznovství vašeho; nebo nemluvili jste toho, což pravého jest, o mne, tak jako služebník muj Job.
\par 9 A tak odšedše Elifaz Temanský a Bildad Suchský a Zofar Naamatský, ucinili, jakž jim byl prikázal Hospodin, a prijal Hospodin oblícej Jobuv.
\par 10 Navrátil také Hospodin to, což odjato bylo Jobovi, když se modlil za prátely své, tak že což mel Job, rozmnožil to Hospodin dvénásobne.
\par 11 A sšedše se k nemu všickni príbuzní, a všecky príbuzné jeho, a všickni známí jeho prvnejší, jedli s ním chléb v dome jeho, a lítost majíce nad ním, potešovali ho nade vším tím zlým, kteréž byl uvedl Hospodin na nej. A dali jemu jeden každý peníz jeden, a jeden každý náušnici zlatou jednu.
\par 12 A tak požehnal Hospodin Jobovi k posledku více nežli v pocátku jeho. Nebo mel ctrnácte tisíc ovcí, a šest tisíc velbloudu, a tisíc sprežení volu, a tisíc oslic.
\par 13 Mel také sedm synu a tri dcery,
\par 14 Z nichž první dal jméno Jemima, jméno pak druhé Keciha, a jméno tretí Kerenhappuch.
\par 15 Aniž se nacházely ženy tak krásné, jako dcery Jobovy, ve vší té krajine; kterýmž dal otec jejich dedictví mezi bratrími jejich.
\par 16 Byl pak živ \par potom sto a ctyridceti let, a videl syny své, a syny synu svých, až do ctvrtého pokolení.
\par 17 I umrel Job, stár jsa a pln dnu.

\end{document}