\begin{document}

\title{2 Korintským}

\chapter{1}

\par 1 Pavel, apoštol Ježíše Krista skrze vuli Boží, a Timoteus bratr, církvi Boží, kteráž jest v Korintu, se všemi svatými, kteríž jsou ve vší Achaii:
\par 2 Milost vám a pokoj od Boha Otce našeho a Pána Jezukrista.
\par 3 Požehnaný Buh a Otec Pána našeho Jezukrista, Otec milosrdenství, a Buh všelikého potešení,
\par 4 Kterýž teší nás ve všelikém ssoužení našem, abychom i my mohli potešovati tech, kteríž by byli v jakémkoli ssoužení, a to tím potešením, kterýmž i my potešeni jsme od Boha.
\par 5 Nebo jakož se rozhojnují utrpení Kristova na nás, tak skrze Krista rozhojnuje se i potešení naše.
\par 6 Nebo budto že souženi jsme, pro vaše potešení a spasení souženi jsme, kteréž se pusobí v snášení týchž trápení, kteráž i my trpíme; budto že potešováni býváme, pro vaše potešení a spasení potešováni býváme. A nadeje naše pevná jest o vás.
\par 7 Ponevadž víme, že jakož jste úcastníci utrpení, také i potešení.
\par 8 Nechcemet zajisté, abyste nevedeli, bratrí, o soužení našem, kteréž jsme meli v Azii, že jsme nad míru pretíženi byli a nad možnost, tak že jsme již o životu svém byli pochybili.
\par 9 Nýbrž sami v sobe již jsme byli tak usoudili, že nebylo lze než umríti, abychom nedoufali sami v sobe, ale v Bohu, jenž i mrtvé krísí.
\par 10 Kterýž z takového nebezpecenství smrti vytrhl nás, a vytrhuje, v nehož doufáme, že i ješte vytrhne,
\par 11 Když i vy nám pomáhati budete modlitbami za nás, aby z daru toho, prícinou mnohých osob nám daného, od mnohých dekováno bylo Bohu za nás.
\par 12 Nebo chlouba naše tato jest, svedectví svedomí našeho, že v sprostnosti a v uprímosti Boží, ne v moudrosti telesné, ale v milosti Boží obcovali jsme na tomto svete, zvlášte pak u vás.
\par 13 Nebot nepíšeme vám nic jiného, nežli to, což ctete, aneb což prvé znáte. A nadeji mám, že až do konce tak znáti budete.
\par 14 Jakož jste již i z stránky poznali nás, žet jsme chlouba vaše, podobne jako i vy naše, v den Pána Ježíše.
\par 15 A v tomt doufání chtel jsem k vám prijíti nejprve, abyste druhou milost meli,
\par 16 A skrze vás jíti do Macedonie, a zase z Macedonie prijíti k vám, a potom od vás abych byl doprovozen do Judstva.
\par 17 O tom pak když jsem premyšloval, zdali jsem co lehkomyslne cinil? Aneb což premyšluji, zdali podle tela premyšluji, tak aby bylo pri mne: Jest, jest, není, není?
\par 18 Ale vít to verný Buh, že rec naše, kteráž byla mluvena k vám, nebyla: Jest, a není.
\par 19 Nebo Syn Boží Ježíš Kristus, kterýž mezi vámi kázán jest skrze nás, totiž skrze mne a Silvána a Timotea, nebyl: Jest, a není, ale bylo v kázání o nem: Jest,
\par 20 (Nebo kolikžkoli jest zaslíbení Božích, v nemt jsou: Jest, a v nemt také jest Amen,) k sláve Bohu skrze nás.
\par 21 Ten pak, kterýž potvrzuje nás s vámi v Kristu, a kterýž pomazal nás, Buh jest.
\par 22 Kterýž i znamenal nás, a dal závdavek Ducha svatého v srdce naše.
\par 23 Já pak Boha za svedka beru na svou duši, že lituje vás, ješte jsem neprišel do Korintu.
\par 24 Ne jako bychom panovali nad verou vaší, ale pomocníci jsme radosti vaší; nebo verou stojíte.

\chapter{2}

\par 1 Toto jsem pak sobe uložil, abych k vám zase s zámutkem neprišel.
\par 2 Nebo jestliže já vás zarmoutím, i kdo jest, ješto by mne obveselil, než ten, kterýž jest ode mne zarmoucen?
\par 3 A protot jsem vám to napsal, abych prijda k vám, nemel zámutku z tech, z nichž bych se mel radovati, doufaje o všech o vás, že radost má jest všech vás radost.
\par 4 Nebo z velikého ssoužení a bolesti srdce psal jsem vám, s mnohými slzami, ne abyste zarmouceni byli, ale abyste poznali lásku, kteroužto k vám velikou mám.
\par 5 Jestližet jest pak kdo zarmoutil, ne mnet zarmoutil, ne toliko ponekud, abych jím neobtežoval všech vás.
\par 6 Dostit má takový na tom trestání, kteréž mel od mnohých,
\par 7 Tak abyste jemu již radeji zase odpustili a potešili ho, aby on snad prílišným zámutkem neprišel na zahynutí.
\par 8 Protož prosím vás, abyste utvrdili k nemu lásku.
\par 9 Nebo i proto psal jsem vám, abych zkušením zvedel, jste-li ve všem poslušni.
\par 10 Komuž pak vy co odpouštíte, i já odpouštím. Nebo i já, jestliže jsem co odpustil, komuž jsem odpustil, pro vás odpustil jsem, pred oblicejem Kristovým, abychom nebyli oklamáni od satana.
\par 11 Nebo nejsou nám neznámá myšlení jeho.
\par 12 Když jsem pak prišel do Troady, k zvestování evangelium Kristova, ackoli dvere mi otevríny byly skrze Pána, však nemel jsem žádného upokojení v duchu svém, protože jsem nenalezl Tita, bratra svého.
\par 13 Ale požehnav jich, odšel jsem do Macedonie.
\par 14 Bohu pak budiž díka, kterýž vždycky dává nám vítezství v Kristu, a vuni známosti své zjevuje skrze nás na každém míste.
\par 15 Nebot jsme Kristova vune dobrá Bohu v tech, jenž k spasení pricházejí, i v tech, kteríž hynou,
\par 16 Tem zajisté jsme vune smrtelná k smrti, onem pak vune života k životu. Ale k tomu kdo jest zpusobný?
\par 17 Nebot nejsme, jako mnozí, cizoložící slovo Boží, ale jako z uprímnosti, a jako z Boha, pred oblicejem Božím, o Kristu mluvíme.

\chapter{3}

\par 1 Zacínáme opet sami sebe chváliti? Zdaliž potrebujeme, jako nekterí, schvalujících listu k vám, nebo od vás k jiným?
\par 2 List náš vy jste, napsaný v srdcích našich, kterýž znají a ctou všickni lidé.
\par 3 Nebo to zjevné jest, že jste vy list Kristuv, zpravený skrze prisluhování naše, napsaný ne cernidlem, ale Duchem Boha živého, ne na dskách kamenných, ale na dskách srdce masitých.
\par 4 Doufání pak takové máme skrze Krista k Bohu,
\par 5 Ne že bychom dostatecní byli mysliti neco sami z sebe, jakožto sami z sebe, ale dostatecnost naše z Boha jest.
\par 6 Kterýžto i hodné nás ucinil služebníky Nového Zákona, ne litery, ale Ducha. Nebo litera zabíjí, ale duch obživuje.
\par 7 A ponevadž prisluhování smrti, literami vyryté na dskách kamenných, bylo slavné, tak že nemohli patriti synové Izraelští v tvár Mojžíšovu, pro slávu oblíceje jeho, kteráž pominouti mela,
\par 8 I kterakž by tedy ovšem prisluhování Ducha nemelo býti slavné?
\par 9 Nebo ponevadž prisluhování pomsty slavné bylo, mnohemt se více rozhojnuje v sláve prisluhování spravedlnosti.
\par 10 Nebo to, což oslaveno bylo, aniž oslaveno bylo v té cástce, u prirovnání prevýšené slávy nového prisluhování.
\par 11 Nebo ponevadž to pomíjející slavné bylo, mnohemt více to, což zustává, jestit slavné.
\par 12 Protož majíce takovou nadeji, mnohé svobody užíváme,
\par 13 A ne jako Mojžíš, kterýž kladl zástrení na tvár svou, aby nepatrili synové Izraelští k cíli té veci pomíjející.
\par 14 Ale ztupeni jsou smyslové jejich. Nebo až do dnešního dne to zastrení v cítání Starého Zákona zustává neodkryté; nebo skrze Krista toliko se odnímá.
\par 15 Protož až do dnešního dne, když se ctou knihy Mojžíšovy, zastrení jest položeno na jejich srdci.
\par 16 Než jakž by se obrátilo ku Pánu, odnato bude zástrení.
\par 17 Nebo Pán Duch jest, a kdež jest Duch Páne, tut i svoboda.
\par 18 My pak všickni odkrytou tvárí slávu Páne jakožto v zrcadle spatrujíce, v týž obraz promeneni býváme od slávy v slávu, jakožto od Ducha Páne.

\chapter{4}

\par 1 Protož majíce toto prisluhování, tak jakž jsme milosrdenství došli, neoblevujeme v nem.
\par 2 Ale odmítáme ukrývání neslušnosti, nechodíce v chytrosti, aniž se lstive obírajíce s slovem Božím, ale zjevováním pravdy v príjemnost uvodíce sebe u každého svedomí lidského pred oblicejem Božím.
\par 3 Paklit zakryté jest evangelium naše, pred temi, kteríž hynou, zakryté jest.
\par 4 V nichžto Buh sveta tohoto oslepil mysli, totiž v neverných, aby se jim nezasvítilo svetlo evangelium slávy Kristovy, kterýž jest obraz Boží.
\par 5 Nebot ne sami sebe kážeme, ale Krista Ježíše Pána, sami pak o sobe pravíme, že jsme služebníci vaši pro Ježíše.
\par 6 Buh zajisté, kterýž rozkázal, aby se z temností svetlo zablesklo, tent jest se osvítil v srdcích našich, k osvícení známosti slávy Boží v tvári Ježíše Krista.
\par 7 Mámet pak poklad tento v nádobách hlinených, aby vyvýšenost moci byla Boží, a ne z nás.
\par 8 Myt se všech stran úzkost máme, ale nebýváme cele potlaceni; v divných jsme nesnadnostech, ale nebýváme v tom pohlceni;
\par 9 Protivenství trpíme, ale nebýváme opušteni; býváme opovrženi, ale nehyneme.
\par 10 Vždycky mrtvení Pána Ježíše na tele svém nosíme, aby i život Ježíšuv na tele našem zjeven byl.
\par 11 Vždycky zajisté my, kteríž živi jsme, na smrt býváme vydáváni pro Ježíše, aby i život Ježíšuv zjeven byl na smrtelném tele našem.
\par 12 A tak smrt v nás moc provodí, ale v vás život.
\par 13 Majíce tedy téhož ducha víry, podle toho, jakž psáno jest: Uveril jsem, protož jsem mluvil, i myt veríme, protož i mluvíme,
\par 14 Vedouce, že ten, kterýž vzkrísil Pána Ježíše, i nás skrze Ježíše vzkrísí a postaví s vámi.
\par 15 Nebo to všecko pro vás se deje, aby ta prehojná milost skrze díku cinení od mnohých verných rozmohla se k sláve Boží.
\par 16 Protož neoblevujeme, ale ackoli zevnitrní clovek náš ruší se, však ten vnitrní obnovuje se den ode dne.
\par 17 Nebo toto nynejší lehoucké ssoužení naše prevelmi veliké vecné slávy bríme nám pusobí,
\par 18 Když nepatríme na ty veci, kteréž se vidí, ale na ty, kteréž se nevidí. Nebo ty veci, kteréž se vidí, jsou casné, ale které se nevidí, jsou vecné.

\chapter{5}

\par 1 Víme zajisté, že byl-li by tohoto našeho zemského prebývání stánek zboren, stavení od Boha máme, príbytek ne rukou udelaný, vecný v nebesích.
\par 2 Procež i v tomto stánku vzdycháme, v príbytek náš, kterýž jest s nebe, obleceni býti žádajíce,
\par 3 Jestliže však oblecení a ne nazí nalezeni budeme.
\par 4 Nebo kteríž jsme v tomto stánku, lkáme, jsouce obtíženi, ponevadž bychom nechteli svleceni býti, ale priodíni, aby pohlcena byla smrtelnost od života.
\par 5 Ten pak, kdož nás tak k tomu zpusobil, Buht jest, kterýž i dal nám závdavek Ducha svého.
\par 6 Protož doufanlivé mysli vždycky jsouce, a vedouce, že dokudž pohostinu jsme v tomto tele, vzdáleni jsme ode Pána,
\par 7 (Nebo skrze víru chodíme, a ne skrze videní tvári Páne,)
\par 8 Doufanlivét pak mysli jsme, a oblibujeme radeji vyjíti z tela a prijíti ku Pánu.
\par 9 Protož i usilujeme bud v tele pohostinu jsouce, budto z tela se berouce, jemu se líbiti.
\par 10 Všickni my zajisté ukázati se musíme pred soudnou stolicí Kristovou, aby prijal jeden každý odplatu za to, což skrze telo pusobil, podle toho, jakž práce cí byla, budto v dobrém, nebo ve zlém.
\par 11 Protož znajíce tu hruzu Páne, lidem k víre sloužíme, Bohut pak známi jsme. A nadejit mám, že i svedomí vašemu známi jsme.
\par 12 Nebot ne sami sebe opet vám chválíme, ale prícinu vám dáváme, abyste se chlubili námi, a abyste to meli proti tem, kteríž se v tvárnosti chlubí, a ne v srdci.
\par 13 Nebo budto že nesmyslní jsme, Bohu nesmyslní jsme; bud že jsme rozumní, vám rozumní jsme.
\par 14 Láska zajisté Kristova víže nás,
\par 15 Jakožto ty, kteríž tak soudíme, že ponevadž jeden za všecky umrel, tedyt všickni zemreli, a že za všecky umrel, aby ti, kteríž živi jsou, již ne sami sobe živi byli, ale tomu, kterýž za ne umrel i z mrtvých vstal.
\par 16 A tak my již od toho casu žádného neznáme podle tela. A ackoli jsme poznali Krista podle tela, ale nyní již více neznáme.
\par 17 Protož jest-li kdo v Kristu, nové stvorení jest. Staré veci pominuly, aj, nové všecko ucineno jest.
\par 18 To pak všecko jest z Boha, kterýž smíril nás s sebou skrze Jezukrista, a dal nám služebnost smírení tohoto.
\par 19 Nebo Buh byl v Kristu, smíruje svet s sebou, nepocítaje jim hríchu jejich, a složil v nás to slovo smírení.
\par 20 Protož my na místo Krista poselství dejeme; a jako by Buh skrze nás žádal vás, prosíme na míste Kristove, smirte se s Bohem.
\par 21 Nebo toho, kterýž hríchu nepoznal, za nás ucinil hríchem, abychom my ucineni byli spravedlností Boží v nem.

\chapter{6}

\par 1 Protož napomáhajíce, i napomínáme vás, abyste milosti Boží nadarmo nebrali,
\par 2 (Nebot praví Buh: V cas príhodný uslyšel jsem te a v den spasení spomohl jsem tobe. Aj, nynít jest cas príhodný, aj, nyní dnové spasení.)
\par 3 Žádného v nicemž nedávajíce pohoršení, aby byla bez úhony služba naše;
\par 4 Ale ve všem se chovajíce jakožto Boží služebníci, ve mnohé trpelivosti, v utišteních, v nedostatcích, v úzkostech,
\par 5 V ranách, v žalárích, v nepokojích, v pracech, v bdení, v postech,
\par 6 V cistote, v umení, v dlouhocekání, v dobrotivosti, v Duchu svatém, v lásce neošemetné,
\par 7 V slovu pravdy, v moci Boží, skrze odení spravedlnosti, napravo i nalevo,
\par 8 Skrze slávu i pohanení, skrze zlou i dobrou povest, jakožto bludní, a jsouce pravdomluvní,
\par 9 Jakožto neznámí, a jsouce známí, jakožto umírající, a aj, živi jsme, jakožto potrestaní, a nezmordovaní,
\par 10 Jako smutní, avšak vždycky se radujíce, jako chudí, a mnohé zbohacujíce, jako nic nemajíce, avšak všemi vecmi vládnouce.
\par 11 Ústa naše otevrína jsou k vám, ó Korinští, srdce naše rozšíreno jest.
\par 12 Nejste v nás souženi, než souženi jste v srdcích vašich.
\par 13 O takovéžt odplaty žádám od vás, jakožto synum pravím: Rozširte se i vy.
\par 14 A netáhnete jha s neverícími. Nebo jaký jest spolek spravedlnosti s nepravostí? A jaké obcování svetla s temnostmi?
\par 15 A jaké srovnání Krista s Beliálem? Aneb jaký díl verícímu s neverícím?
\par 16 A jaké spolcení chrámu Božího s modlami? Nebo vy jste chrám Boha živého, jakož povedel Buh: Že prebývati budu v nich, a procházeti se, a budu jejich Bohem, a oni budou mým lidem.
\par 17 A protož vyjdetež z prostredku jejich a oddelte se od nich, praví Pán; a necistého se nedotýkejte, a já prijmu vás.
\par 18 A budu vám za Otce, a vy mi budete za syny a za dcery, praví Pán všemohoucí.

\chapter{7}

\par 1 Taková tedy majíce zaslíbení, nejmilejší, ocištujmež se od všeliké poskvrny tela i ducha, konajíce posvecení naše v bázni Boží.
\par 2 Prijmetež nás. Žádnémut jsme neublížili, žádnému neuškodili, žádného neoklamali.
\par 3 Nepravím toho ku potupe vaší, ponevadž jsem napred povedel, že v srdcích našich jste, tak abychom hotovi byli spolu s vámi zemríti i spolu živi býti.
\par 4 Mnohét jsem k vám duvernosti, mnohot se vámi chlubím; naplnen jsem potešením, a rozhojnujit se v radosti ve všelikém soužení našem.
\par 5 Nebo i když jsme byli prišli do Macedonie, žádného odpocinutí nemelo telo naše, ale ve všem souženi jsme byli; meli jsme zevnitr boje, vnitr strachy.
\par 6 Ale ten, jenž teší ponížené, potešil nás, Buh, skrze Tituv príchod.
\par 7 A netoliko príchodem jeho, ale také i potešením tím, kteréž on mel z vás, vypravovav nám o vaší veliké žádosti nás,  o vašem kvílení, a vaší ke mne horlivé milosti, takže jsem se velmi zradoval.
\par 8 A ackoli zarmoutil jsem vás listem, nelituji toho, ac jsem byl litoval. Nebo vidím, že ten list, ackoli na cas, zarmoutil vás.
\par 9 Ale nyní raduji se, ne že jste zarmouceni byli, ale že jste se ku pokání zarmoutili. Zarmoutili jste se zajisté podle Boha, takže jste k žádné škode neprišli skrze nás.
\par 10 Nebo zámutek, kterýž jest podle Boha, ten pokání k spasení pusobí takové, jehož nikdy líto nebude, ale zámutek sveta zpusobuje smrt.
\par 11 Ano hle, to samo, že jste podlé Boha zarmouceni byli, kterakou v vás zpusobilo snažnost, nýbrž jakou omluvu, nýbrž zažhnutí hnevu, nýbrž bázen, nýbrž žádost vroucí, nýbrž horlivost, anobrž pomstu? A Všelijak ukázali jste se nevinni býti v té príhode.
\par 12 A ac psal jsem vám, však ne pro toho jsem psal, kterýž tu nepravost spáchal, ani pro toho, komuž se krivda stala, ale aby vám zjevena byla pilnost naše o vás pred oblicejem Božím.
\par 13 Protož potešenit jsme z potešení vašeho. Ale mnohem hojneji zradovali jsme se z radosti Titovy, že poocerstven jest duch jeho ode všech vás,
\par 14 A že chlubil-li jsem se v cem vámi pred ním, nebyl jsem zahanben, ale jakož všecko mluvili jsme vám v pravde, tak i chlouba naše, kteráž byla pred Titem, pravá jest shledána.
\par 15 A hojneji i srdce jeho jest k vám obráceno, nebo rozpomíná se na poslušenství všech vás, a kterak jste ho s bázní a s strachem prijali.
\par 16 Raduji se pak, že ve všem mám o vás doufání.

\chapter{8}

\par 1 Známut pak vám ciníme, bratrí, milost Boží, danou zborum Macedonským,
\par 2 Že v mnohém zkušení rozlicných soužení rozhojnilá radost jejich a preveliká chudoba jejich rozhojnena jest v bohatství uprímnosti jejich.
\par 3 Nebo svedectví jim vydávám, že podle možnosti, ba i nad možnost hotovi byli sdeliti se,
\par 4 Mnohými žádostmi prosíce nás, abychom té milosti jejich a obcování služebnost (v rozdelování toho svatým) na se prijali.
\par 5 A netoliko tak ucinili, jakž jsme se nadáli, ale sami sebe nejprve dali Pánu, a i nám také u vuli Boží,
\par 6 Tak že jsme musili napomenouti Tita, aby jakož byl prve zapocal, tak také i dokonal pri vás milost tuto.
\par 7 A protož jakž ve všech vecech jste hojní, totiž u víre i v reci i v známosti i ve všeliké snažnosti i v lásce vaší k nám, tak i v této milosti hojní budte.
\par 8 Ne jako rozkazuje, toto pravím, ale prícinou jiných snažnosti i vaši uprímou lásku zkušenou ukázati chteje.
\par 9 Nebo znáte milost Pána našeho Jezukrista, že pro vás ucinen jest chudý, jsa bohatý, abyste vy jeho chudobou zbohatli.
\par 10 A k tomut vám radu dávám; nebo jest vám to užitecné, kteríž jste netoliko ciniti, ale i chtíti prve zacali léta predešlého.
\par 11 Protož nyní již to skutkem vykonejte, aby jakož hotové bylo chtení, tak také bylo i vykonání z toho, což máte.
\par 12 Nebo jest-lit prve vule hotová, podle toho, což kdo má, vzácná jest Bohu, ne podle toho, cehož nemá.
\par 13 Nebo ne proto vás ponoukám, aby jiným bylo polehcení, a vám soužení, ale rovnost at jest; nyní prítomne vaše hojnost spomoziž jejich chudobe,
\par 14 Aby potom také jejich hojnost vaší chudobe byla ku pomoci, aby tak byla rovnost;
\par 15 Jakož psáno: Kdo mnoho nasbíral, nic mu nezbývalo, a kdo málo, nemel nedostatku.
\par 16 Ale díka Bohu, kterýž takovouž snažnost k službe vám dal v srdce Titovo,
\par 17 Takže to napomenutí naše ochotne prijal, anobrž jsa v lásce k vám opravdový, sám z své dobré vule šel k vám.
\par 18 Poslalit jsme pak s ním bratra toho, kterýž má velikou chválu v evangelium po všech sborích,
\par 19 (A netoliko to, ale také losem vyvolen jest od církví za tovaryše putování našeho, s touto milostí, kteroužto sloužíme k sláve Pánu a k vyplnení vule vaší,)
\par 20 Varujíce se toho, aby nám nekdo neutrhal pro tu hojnost, kterouž my prisluhujeme,
\par 21 Dobré opatrujíce netoliko pred oblicejem Páne, ale i pred lidmi.
\par 22 Poslali jsme pak s nimi bratra našeho, kteréhož jsme mnohokrát ve mnohých vecech zkusili, že jest pilný, a nyní mnohem pilnejší bude pro mnohé doufání mé o vás.
\par 23 Z strany Tita víte, že jest tovaryš muj, a mezi vámi pomocník muj; a z strany bratrí našich, že jsou poslové církví a sláva Kristova.
\par 24 Protož jistoty lásky vaší a chlouby naší o vás, k nim dokažte, pred oblicejem církví.

\chapter{9}

\par 1 Nebo o pomoci, kteráž se sbírá na svaté, jest zbytecné psáti vám.
\par 2 Vím zajisté o vaší hotovosti, pro kteroužto chlubím se vámi u Macedonských, kterak by Achaia hotova byla k tomu od predešlého léta. A ta z vás pošlá horlivost mnohými k témuž pohnula.
\par 3 I poslal jsem k vám tyto bratrí, aby chlouba naše vámi nebyla marná v té stránce, ale abyste, (jakž jsem rekl,) byli hotovi,
\par 4 Abychom snad, prišli-li by se mnou Macedonští, a nalezli vás nehotové, nemusili se stydeti, (at nedím vy) za takovou chloubu.
\par 5 Protož videlo mi se za potrebné techto bratrí napomenouti, aby predešli mne k vám, a pripravili prve opovedenou sbírku vaši, aby byla hotová jako dobrovolná sbírka, a ne jako bezdecná.
\par 6 Ale totot pravím: Kdo skoupe rozsívá, skoupe i žíti bude; a kdož rozsívá ochotne, ochotne i žíti bude.
\par 7 Jeden každý jakž uložil v srdci svém, tak ucin, ne s neochotnou myslí anebo z mušení. Nebot ochotného dárce miluje Buh.
\par 8 Mocent jest pak Buh všelikou milost rozhojniti v vás, abyste ve všem všudy všelikou dostatecnost majíce, hojní byli ke všelikému skutku dobrému,
\par 9 Jakož psáno jest: Rozsypal a dal chudým, spravedlnost jeho zustává na veky.
\par 10 Ten pak, kterýž dává síme rozsívajícímu, dejž i vám chléb k jedení, a rozmnožiž síme vaše, a prisporiž úrody spravedlnosti vaší,
\par 11 Abyste všelijak zbohaceni byli ke všeliké sprostnosti, kterážto pusobí skrze nás, aby díky cineny byly Bohu.
\par 12 Nebo služba obeti této netoliko doplnuje nedostatky svatých, ale také rozhojnuje se v mnohé díku cinení Bohu, prícinou schválení služby této,
\par 13 Když chválí Boha z jednomyslné poddanosti vaší k evangelium Kristovu, a z uprímé pomoci jim i všechnem ucinené,
\par 14 A modlí se za vás ti, kteríž vás prevelice milují pro vyvýšenou milost Boží v vás.
\par 15 Díka pak budiž Bohu z nevymluvného daru jeho.

\chapter{10}

\par 1 Já pak sám Pavel prosím vás skrze tichost a dobrotivost Kristovu, kterýžto v prítomnosti u vás jsem pokorný, ale v neprítomnosti smele doverný jsem k vám.
\par 2 Prosím pak vás za to, abych prítomen jsa, nemusil doufanlivý býti tou smelostí, kterouž jsem jmín, že bych smelý byl proti nekterým, kteríž za to mají, že bychom my podle tela chodili.
\par 3 V tele zajisté chodíce, ne podle tela ryterujeme,
\par 4 Nebo odení ryterování našeho není telesné, ale mocné v Bohu k vyvrácení ohrad,
\par 5 Jímžto podvracíme rady, i všelikou vysokost, povyšující se proti umení Božímu, jímajíce všelikou mysl v poddanost Kristu,
\par 6 A nahotove majíce pomstu proti každému neposlušenství, když naplneno bude vaše poslušenství.
\par 7 A což toliko na to, co pred ocima jest, hledíte? Má-li kdo tu nadeji o sobe, že by Kristuv byl, pomysliž také na to sám u sebe, že jakož on jest Kristuv, tak i my Kristovi jsme.
\par 8 Nebo bycht se pak i ješte hojneji chlubil mocí naší, kterouž nám dal Pán vzdelání a ne k zkažení vašemu, nebudut zahanben;
\par 9 Abych se nezdál listy strašiti vás.
\par 10 Nebo listové jeho (dí nekdo) jsou težcí a mocní, ale prítomnost osobná jest mdlá, a rec chaterná.
\par 11 Toto nechat myslí takový, že jacíž jsme v slovu skrze psání, vzdáleni jsouce od vás, takoví také budeme i v skutku, prijdouce k vám.
\par 12 Nebot my se neodvažujeme primísiti aneb prirovnati k nekterým, kteríž sami sebe chválí. Ale ti nerozumejí, že sami sebou sebe merí a prirovnávají sebe sobe.
\par 13 My pak nebudeme se nad to, což nám není odmereno, chlubiti, ale podle míry pravidla, kteroužto míru odmeril nám Buh, chlubiti se budeme, totiž že jsme dosáhli až k vám.
\par 14 Nebo ne jako bychom nedosáhli až k vám, rozširujeme se nad míru. Až i k vám zajisté prišli jsme v evangelium Kristovu.
\par 15 Protož my se nad to, což nám není odmereno, nechlubíme, totiž cizími pracemi, ale nadeji máme, když víra vaše rusti bude v vás, že my se rozšíríme dále podle toho, jakž nám odmereno,
\par 16 Totiž, abych ješte i na tech místech, kteráž jsou za vámi dále, kázal evangelium, a ne abych v tom, což jinému odmereno jest, a již jest hotové, se chlubil.
\par 17 Ale kdo se chlubí, v Pánu se chlub.
\par 18 Nebo ne ten, kdož se sám chválí, zkušený jest, ale ten, kohož Pán chválí.

\chapter{11}

\par 1 Ó byste mne malicko posnesli v nemoudrosti mé, nýbrž posneste mne.
\par 2 Nebot miluji vás Božím milováním. Zasnoubilt jsem zajisté vás jako cistou pannu oddati jednomu muži, Kristu.
\par 3 Ale bojímt se, aby snad, jakož had svedl Evu chytrostí svou, tak nebyly porušeny mysli vaše, abyste se totiž neuchýlili od sprostnosti, kteráž jest v Kristu.
\par 4 Nebo kdyby nekdo prijda, jiného Ježíše vám kázal, kteréhož jsme my nekázali, aneb kdybyste jiného ducha prijali, kteréhož jste prve neprijali, anebo jiné evangelium, kteréhož jste od nás nevzali, slušne byste to snášeli.
\par 5 Nebot za to mám, že jsem nic menší nebyl velikých apoštolu.
\par 6 Jestližet pak jsem nedospelý v reci, však ne v umení, ale ve všem všudy otevrení jsme vám.
\par 7 Zdali jsem zhrešil, ponižuje se, abyste vy povýšeni byli, a že jsem darmo evangelium Boží kázal vám?
\par 8 Jiné jsem církve loupil, bera od nich plat k službe vaší. A byv u vás, jsa potreben, neobtežoval jsem žádného.
\par 9 Nebo ten nedostatek muj doplnili bratrí, prišedše z Macedonie. A ve všech vecech varoval jsem se, a varovati budu, abych vás neobtežoval.
\par 10 Jestit pravda Kristova ve mne, že chlouba tato nebude mi zmarena v krajinách Achaiských.
\par 11 Z které príciny? Snad že vás nemiluji? Buht ví.
\par 12 Ale což ciním, ještet ciniti budu, abych odnal prícinu tem, kteríž hledají príciny k tomu, aby v tom, v cemž se chlubí, nalezeni byli takoví jako i my.
\par 13 Nebo takoví falešní apoštolé jsou delníci lstiví, promenujíce se v apoštoly Kristovy.
\par 14 A není div. Nebo i satan promenuje se v andela svetlosti.
\par 15 Protož nenít to tak veliká vec, jestliže i služebníci jeho promenují se, aby se zdáli býti služebníci spravedlnosti, jichžto konec bude podle skutku jejich.
\par 16 Opet pravím, aby mne nekdo nemel za nemoudrého; nýbrž i jako nemoudrého prijmete mne, at i já se malicko neco pochlubím.
\par 17 Což mluvím, nemluvímt jako ode Pána, ale jako v nemoudrosti z strany této chlouby.
\par 18 Kdyžt se mnozí chlubí podle tela, i ját se pochlubím.
\par 19 Rádi zajisté snášíte nemoudré, jsouce sami moudrí.
\par 20 Nebo snesete i to, by vás kdo v službu podrobil, by kdo zžíral, by kdo bral, by se kdo pozdvihoval, by vás kdo v tvár bil.
\par 21 K zahanbení vašemu pravím, rovne jako bychom my nejací špatní byli. Nýbrž v cem kdo smí, (v nemoudrosti mluvím,) smímt i já.
\par 22 Židé jsou? Jsem i já Žid. Izraelští jsou? Jsem i já. Síme Abrahamovo jsou? I já.
\par 23 Služebníci Kristovi jsou? (Nemoudre dím:) Nadto já. V pracech býval jsem hojneji, v ranách prílišne, v žalárích hojneji, v smrtech castokrát.
\par 24 Od Židu petkrát ctyridceti ran bez jedné trpel jsem.
\par 25 Trikrát metlami mrskán jsem, jednou jsem byl ukamenován, trikrát jsem na mori tonul, ve dne i v noci v hlubokosti morské byl jsem.
\par 26 Na cestách casto, v nebezpecenství na rekách, v nebezpecenství od lotru, v nebezpecenství od svého pokolení, v nebezpecenství od pohanu, v nebezpecenství v meste, v nebezpecenství na poušti, v nebezpecenství na mori, v nebezpecenství mezi falešnými bratrími;
\par 27 V práci a v ustání, v bdeních casto, v hladu a v žízni, v postech castokrát, na zime a v nahote.
\par 28 Krome toho, což zevnitr jest, dotírá na mne ten houf na každý den povstávající proti mne, to jest péce o všecky sbory.
\par 29 Kdo umdlévá, abych já s ním nemdlel? Kdo se uráží, abych já se nepálil?
\par 30 Jestližet se mám chlubiti, nemocmi svými se chlubiti budu.
\par 31 Buh a Otec Pána našeho Jezukrista, kterýž jest požehnaný na veky, ví, žet nelhu.
\par 32 Hejtman v Damašku lidu Aréty krále, ostríhal mesta Damašku, chteje mne do vezení vzíti.
\par 33 Ale já oknem po provaze spušten jsem v koši pres zed, i ušel jsem rukou jeho.

\chapter{12}

\par 1 Ale chlubiti mi se není dobré, nebo prišel bych k vypravování o videních a zjeveních Páne.
\par 2 Znám cloveka v Kristu pred lety ctrnácti, (v tele-li, nevím, cili krom tela, nevím, Buht ví,) kterýž byl vtržen až do tretího nebe.
\par 3 A vím takového cloveka, (bylo-li v tele, cili krom tela, nevím, Buh ví),
\par 4 Že jest byl vtržen do ráje, a slyšel nevypravitelná slova, kterýchž nesluší cloveku mluviti.
\par 5 Takovým budu se chlubiti, ale sám sebou nechci se chlubiti, než toliko nemocmi svými.
\par 6 Nebo budu-li se chtíti chlubiti, nebudut proto nemoudrým, nebo pravdu povím; ale uskrovnímt, aby nekdo nesmýšlel více, nežli vidí pri mne, aneb slyší ode mne.
\par 7 A abych se vysokostí zjevení nad míru nepozdvihl, dán mi jest osten do tela, totiž andel satan, aby mne polickoval, abych se nad míru nepovyšoval.
\par 8 Za to trikrát jsem Pána prosil, aby to odstoupilo ode mne.
\par 9 Ale rekl mi: Dosti máš na mé milosti, nebot moc má v nemoci dokonává se. Nejradeji tedy chlubiti se budu nemocmi svými, aby ve mne prebývala moc Kristova.
\par 10 Protož libost mám v nemocech svých, v pohaneních, v nedostatcích, v protivenstvích, a v úzkostech, pro Krista. Nebo když mdlím, tedy silen jsem.
\par 11 Ucinen jsem nemoudrým, chlube se; vy jste mne k tomu prinutili. Neb já od vás mel jsem chválen býti; nebot jsem nic menší nebyl velikých apoštolu, ackoli nic nejsem.
\par 12 Znamení zajisté apoštolství mého prokázána jsou mezi vámi ve vší trpelivosti, i v divích a v zázracích, a v mocech.
\par 13 Neb co jest, ceho byste vy méne meli nežli jiné církve, lec to, že jsem já vás neobtežoval? Odpusttež mi to bezpráví.
\par 14 Aj, již potretí hotov jsem prijíti k vám, a nebudut vás obtežovati. Nebo nehledám toho, což jest vašeho, ale vás. Nebot nemají synové rodicum pokladu shromaždovati, ale rodicové synum.
\par 15 Ját pak velmi rád náklad uciním, i sám se vynaložím za duše vaše, ackoli velmi vás miluje, málo jsem milován.
\par 16 Ale nechtž jest tak, že jsem já vás neobtežoval, než chytrý jsa, lstí jsem vás zjímal.
\par 17 Zdali skrze nekoho z tech, kteréž jsem poslal k vám, obloupil jsem vás?
\par 18 Dožádal jsem se Tita, a poslal jsem s ním bratra toho. Zdali vás Titus podvedl? Zdaliž jsme jedním duchem nechodili? Zdaliž ne jednemi šlépejemi?
\par 19 A zase domníváte-li se, že my se vymlouváme pred vámi? Pred oblicejemt Božím v Kristu mluvíme, a to všecko, nejmilejší, k vašemu vzdelání.
\par 20 Nebot se bojím, abych snad prijda k vám, nenalezl vás takových, jakýchž bych nechtel, a já nebyl nalezen od vás, jakéhož byste vy nechteli, aby snad nebyli mezi vámi svárové, závistí, hnevové, vády, utrhání, reptání, nadýmání, ruznice,
\par 21 Aby mne opet, když bych prišel, neponížil Buh muj u vás, a plakal bych mnohých z tech, kteríž jsou prve hrešili, a necinili pokání z necistoty, a z smilstva, a z nestydatosti, kterouž páchali.

\chapter{13}

\par 1 Toto již po tretí jdu k vám, a v ústech dvou neb trí svedku stanet každé slovo.
\par 2 Predpovedelt jsem, a predpovídám po druhé jako prítomný, a neprítomný nyní píši tem, kteríž prve hrešili, i jiným všechnem, že prijdu-lit opet znovu, neodpustím,
\par 3 Ponevadž zkusiti hledáte toho, kterýž skrze mne mluví, Krista, kterýžto k vám není nemocný, ale mocný jest v vás.
\par 4 Nebo ackoli ukrižován jest jako nemocný, ale živ jest z moci Boží. A tak i my mdlí jsme s ním, ale živi budeme s ním, z moci Boží vztahující se až k vám.
\par 5 Sami sebe zkušujte, jste-li u víre; sami sebe ohledujte. Cili sami sebe neznáte, že Ježíš Kristus jest v vás? Lec jste snad zavrženi.
\par 6 Ale nadeji mám, žet poznáte, žet my nejsme zavrženi.
\par 7 Modlímt se pak Bohu, abyste nic zlého necinili, ne proto, abychom my se dokonalí ukázali, ale abyste vy to, což jest dobrého, cinili, my pak jako zavržení abychom byli.
\par 8 Nebot nic nemužeme proti pravde, ale k pravde.
\par 9 Radujeme se zajisté, že ac jsme mdlí, ale vy jste silní, a za tot se i modlíme, abyste vy byli dokonalí.
\par 10 Protož toto neprítomný jsa, píši, abych snad potom prítomen jsa, nemusil býti prísný, podle moci, kterouž mi dal Pán k vzdelání, a ne k zkáze.
\par 11 Naposledy, bratrí mejtež se dobre, dokonalí budte, potešujte se, jednostejne smyslte, pokoj mejte. A Buh lásky a pokoje budet s vámi.
\par 12 Pozdravtež jedni druhých políbením svatým. Pozdravují vás všickni svatí.
\par 13 Milost Pána Jezukrista, a láska Boží, a úcastenství Ducha svatého budiž se všemi vámi. Amen. Druhý list k Korintským psán byl z Filippis, Mesta Macedonského, po Titovi a Lukášovi.


\end{document}