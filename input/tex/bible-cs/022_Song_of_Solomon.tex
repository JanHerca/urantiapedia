\begin{document}

\title{Song of Solomon}

\chapter{1}

\par 1 Písen nejprednejší z písní Šalomounových.
\par 2 Ó by mne políbil políbením úst svých; nebo lepší jsou milosti tvé nežli víno.
\par 3 Pro vuni masti tvé jsou výborné, mast rozlitá jméno tvé; protož te mladice milují.
\par 4 Táhniž mne, a pobehnem za tebou. Uvedlte mne král do pokoju svých, plésati a veseliti se v tobe budeme, a vychvalovati milosti tvé více než víno; uprímí milují te.
\par 5 Jsemt cerná, ale milostná, ó dcery Jeruzalémské, tak jako stanové Cedarští, jako opony Šalomounovy.
\par 6 Nehledte na mne, žet jsem snedá, nebo jsem obhorela od slunce. Synové matky mé rozpálivše se proti mne, postavili mne, abych ostríhala vinic, a vinice své nehlídala jsem.
\par 7 Oznam mi ty, kteréhož miluje duše má, kde paseš? Kde dáváš odpocinutí o poledni? Nebo proc mám býti tak jako pobehlá pri stádích tovaryšu tvých?
\par 8 Jestliže nevíš, ó nejkrašší mezi ženami, vyjdi po šlepejích ovcí, a pas kozlátka svá podlé obydlí pastýru.
\par 9 Jízde v vozích Faraonových pripodobnuji te, ó milostnice má.
\par 10 Líce tvá okrášlena jsou ozdobami, a hrdlo tvé halžemi.
\par 11 Ozdob zlatých nadeláme tobe s promenami stríbrnými.
\par 12 Dotud, dokudž král stolí, nardus muj vydává vuni svou.
\par 13 Svazcek mirry jest mi milý muj, na prsech mých odpocívaje.
\par 14 Milý muj jest mi hrozen cyprový na vinicích v Engadi.
\par 15 Aj, jak jsi ty krásná, prítelkyne má, aj, jak jsi krásná! Oci tvé jako holubicí.
\par 16 Aj, jak jsi ty krásný, milý muj, jak utešený! I to luže naše zelená se.
\par 17 Trámové domu našich jsou z cedru, a pavlace naše z boroví.

\chapter{2}

\par 1 Já jsem ruže Sáronská, a lilium pri dolinách.
\par 2 Jako lilium mezi trním, tak prítelkyne má mezi pannami.
\par 3 Jako jablon mezi drívím lesním, tak milý muj mezi mládenci. V stínu jeho žádostiva jsem byla sedeti, a sedímt; nebo ovoce jeho sladké jest ústum mým.
\par 4 Uvedl mne na hody, maje za korouhev lásku ke mne.
\par 5 Ocerstvetež mne temi flašemi, posilnte mne temi jablky, nebo umdlévám milostí,
\par 6 Levice jeho pod hlavou mou, a pravicí svou objímá mne.
\par 7 Zavazujit vás prísahou, dcery Jeruzalémské, skrze srny a lane polní, abyste nebudily a nevyrážely ze sna milého mého, dokudž by nechtel.
\par 8 Hlas milého mého, aj, ont se bére, skáce po tech horách, poskakuje na tech pahrbcích.
\par 9 Podobný jest milý muj srne aneb mladému jelenu; aj, on stojí za stenou naší, vyhlédá z oken, patrí skrze mríži.
\par 10 Ozval se milý muj, a rekl mi: Vstan, prítelkyne má, krásná má, a pod.
\par 11 Nebo aj, zima pominula, prška prestala a odešla.
\par 12 Kvítícko se ukazuje po zemi, cas prozpevování prišel, a hlas hrdlicky slyší se v krajine naší.
\par 13 Fík vypustil holicky své, a réví rozkvetlé vydalo vuni. Vstaniž, prítelkyne má, krásná má, a pod.
\par 14 Holubicko má, v rozsedlinách skalních, v skrýši príkré, ukaž mi oblícej svuj, nechat slyším hlas tvuj; nebo hlas tvuj libý jest, a oblícej tvuj žádostivý.
\par 15 Zlapejte nám lišky, lišky malické, ješto škodu delají na vinicích, ponevadž vinice naše kvete.
\par 16 Milý muj jest muj, a já jeho, jenž pase mezi lilium.
\par 17 Ažby zavítal ten den, a utekli by stínové ti, navratiž se, pripodobni se, milý muj, srne neb mladému jelenu na horách Beter.

\chapter{3}

\par 1 Na ložci svém v noci hledala jsem toho, kteréhož miluje duše má. Hledala jsem ho, ale nenašla jsem ho.
\par 2 Již tedy vstanu, a zchodím mesto; po ryncích i po ulicech hledati budu toho, kteréhož miluje duše má. Hledala jsem ho, ale nenašla jsem ho.
\par 3 Našli mne ponocní, kteríž chodí po meste. Videli-liž jste toho, kteréhož miluje duše má?
\par 4 A jakž jsem jich jen pominula, takž jsem našla toho, kteréhož miluje duše má. Chopila jsem ho, aniž ho pustím, až ho uvedu do domu matky své, a do pokojíka rodicky své.
\par 5 Prísahou vás zavazuji, dcery Jeruzalémské, skrze srny a lane polní, abyste nebudily a nevyrážely ze sna milého mého, dokudž by sám nechtel.
\par 6 Která jest to, jenž vstupuje z poušte jako sloupové dymu, okourena jsuc mirrou a kadidlem, dražším nad všelijaký prach apatekárský?
\par 7 Aj, lože Šalomounovo, okolo nehož šedesáte udatných z nejsilnejších Izraelských,
\par 8 Vše vládnoucích mecem, vycvicených v boji, z nichž jeden každý má svuj mec pri boku svém z príciny strachu nocního.
\par 9 Schranu vystavel sobe král Šalomoun z dríví Libánského,
\par 10 Pri níž udelal sloupy stríbrné, dno zlaté a ponebí šarlatové, vnitrek pak jeho postlaný milostí dcer Jeruzalémských.
\par 11 Vyjdete a pohledte, dcery Sionské, na krále Šalomouna v korune, kterouž ho korunovala matka jeho v den oddávání jeho a v den veselí srdce jeho.

\chapter{4}

\par 1 Aj, jak jsi ty krásná, prítelkyne má, aj, jak jsi krásná! Oci tvé jako holubicí mezi kaderi tvými, vlasy tvé jako stáda koz, kteréž vídati na hore Galád.
\par 2 Zubové tvoji podobní stádu ovcí jednostejných, když vycházejí z kupadla, z nichž každá mívá po dvém, a mezi nimiž není žádné neplodné.
\par 3 Jako provázek z hedbáví cerveného dvakrát barveného rtové tvoji, a rec tvá ozdobná; jako kus jablka zrnatého židoviny tvé mezi kaderi tvými.
\par 4 Hrdlo tvé jest jako veže Davidova, vystavená k chování zbroje, v níž na tisíce pavéz visí, vše štítu mužu udatných.
\par 5 Oba tvé prsy jako dvé telátek bližencu srních, jenž se pasou v kvítí.
\par 6 Ažby zavítal ten den, a utekli by stínové, poodejdu k hore mirrové a pahrbku kadidlovému.
\par 7 Všecka jsi krásná, prítelkyne má, a není na tobe poškvrny.
\par 8 Se mnou z Libánu, ó choti má, se mnou z Libánu pujdeš, a pohledíš s vrchu hory Amana, s vrchu Senir a Hermon, z peleší lvových a s hor pardových.
\par 9 Jala jsi srdce mé, sestro má choti, jala jsi srdce mé jedním okem svým, a jedinou tocenicí hrdla svého.
\par 10 Jak utešené jsou milosti tvé, sestro má choti! Jak vzácnejší jsou milosti tvé než víno, a vune mastí tvých nade všecky vonné veci.
\par 11 Strdí tekou rtové tvoji, ó choti, med a mléko pod jazykem tvým, a vune roucha tvého jako vune Libánu.
\par 12 Zahrada zamcená jsi, sestro má choti, vrchovište zamcené, studnice zapecetená.
\par 13 Výstrelkové tvoji jsou zahrada stromu jablek zrnatých s ovocem rozkošným cypru a nardu,
\par 14 Nardu s šafránem, prustvorce s skoricí, a s každým stromovím kadidlo vydávajícím, mirry a aloes, i s všelijakými zvláštními vecmi vonnými.
\par 15 Ó ty sám vrchovište zahradní, studnice vod živých, a tekoucích z Libánu.
\par 16 Vej, vetrícku pulnocní, a prid, vetrícku polední, provej zahradu mou, at tekou vonné veci její, a at príjde milý muj do zahrady své, a jí rozkošné ovoce své.

\chapter{5}

\par 1 Prišelt jsem do zahrady své, sestro má choti, sbírám mirru svou i vonné veci své, jím plást svuj i med svuj, pijí víno své a mléko své. Jezte, prátelé, píte a hojne se napíte, moji milí.
\par 2 Spávámt, a však srdce mé bdí. Hlas milého mého, tlukoucího: Otevri mi, sestro má, prítelkyne má, holubicko má, uprímá má; nebo hlava má plná jest rosy, a kadere mé krupejí nocních.
\par 3 Svlékla jsem sukni svou, kterakž ji obleku? Umyla jsem nohy své, což je mám káleti?
\par 4 Milý muj sáhl rukou svou skrze dvére, a vnitrnosti mé pohnuly se ve mne.
\par 5 I vstala jsem, abych otevrela milému svému, a aj, z rukou mých kapala mirra, i z prstu mých, mirra tekutá na rukovetech závory.
\par 6 Otevrelat jsem byla milému svému, ale milý muj již byl ušel, a pominul. Duše má byla vyšla, když on promluvil. Hledala jsem ho, ale nenašla jsem ho; volala jsem ho, ale neozval se mi.
\par 7 Nalezše mne strážní, kteríž chodí po meste, zbili mne, ranili mne, vzali i rouchu mou se mne strážní zdí mestských.
\par 8 Zavazuji vás prísahou dcery Jeruzalémské, jestliže byste našly milého mého, co jemu povíte? Že jsem nemocná milostí.
\par 9 Což má milý tvuj mimo jiné milé, ó nejkrásnejší z žen? Co má milý tvuj nad jiné milé, že nás tak prísahou zavazuješ?
\par 10 Milý muj jest bílý a cervený, znamenitejší nežli deset tisícu jiných.
\par 11 Hlava jeho jako ryzí zlato, vlasy jeho kaderavé, cerné jako havran.
\par 12 Oci jeho jako holubic nad stoky vod, jako v mléce umyté, stojící v slušnosti.
\par 13 Líce jeho jako záhonkové vonných vecí, jako kvetové vonných vecí; rtové jeho jako lilium prýštící mirru tekutou.
\par 14 Ruce jeho prstenové zlatí, vysazení kamením drahým jako postavcem modrým; bricho jeho stkvelost slonové kosti zafiry obložené.
\par 15 Hnátové jeho sloupové mramoroví, na podstavcích zlata nejcistšího založení; oblícej jeho jako Libán, výborný jako cedrové.
\par 16 Ústa jeho presladká, a všecken jest prežádostivý. Takovýt jest milý muj, takový jest prítel muj, ó dcery Jeruzalémské.
\par 17 Kamže odšel milý tvuj, ó nejkrásnejší z žen? Kam se obrátil milý tvuj? A hledati ho budeme s tebou.

\chapter{6}

\par 1 Milý muj sstoupil do zahrady své, k záhonkum veci vonných, aby pásl v zahradách, a aby zbíral lilium.
\par 2 Já jsem milého mého, a milý muj jest muj, kterýž pase mezi lilium.
\par 3 Krásná jsi, prítelkyne má, jako Tersa, pekná jako Jeruzalém, hrozná jako vojsko s praporci.
\par 4 Odvrat oci své od patrení na mne, nebo ony mne posilují. Vlasy tvé jsou jako stáda koz, kteréž vídati v Galád.
\par 5 Zubové tvoji jsou podobní stádu ovcí, když vycházejí z kupadla, z nichž každá mívá po dvém, a mezi nimiž není žádné neplodné.
\par 6 Jako kus jablka zrnatého židoviny tvé mezi kaderi tvými.
\par 7 Šedesáte jest královen, a osmdesáte ženin, a mladic bez poctu;
\par 8 Jediná jest však má holubice, má uprímá, jedinká pri matce své, nepoškvrnená pri své rodicce. Spatrivše ji dcery, blahoslavily ji, královny a ženiny chválily ji, rkouce:
\par 9 Která jest to, kterouž videti jako dennici, krásná jako mesíc, cistá jako slunce, hrozná jako vojsko s praporci?
\par 10 Do zahrady vypravené sstoupila jsem, abych spatrila ovoce údolí, abych videla, kvete-li vinný kmen, a pucí-li se jablone zrnaté.
\par 11 Nezvedela jsem, a žádost má ponukla mne na vuz prednejších z lidu mého.
\par 12 Navrat se, navrat, ó Sulamitská, navrat se, navrat, at na te patríme. Co uzríte na Sulamitské? Jako zástup vojenský.

\chapter{7}

\par 1 Jak jsou krásné nohy tvé v strevících, dcero knížecí! Okolek bedr tvých jako zápony, dílo ruku výborného remeslníka.
\par 2 Pupek tvuj koflík okrouhlý, ne bez nápoje; bricho tvé jako stoh pšenice obrostlý kvítím.
\par 3 Oba tvé prsy jako dvé telátek bližencu srních.
\par 4 Hrdlo tvé jako veže z kostí slonových, oci tvé rybníci v Ezebon podlé brány Batrabbim, nos tvuj veže Libánská patrící k Damašku.
\par 5 Hlava tvá na tobe jako Karmel, a vlasy hlavy tvé jako šarlat; i král privázán by byl na pavlacích.
\par 6 Jak jsi ty krásná, a jak utešená, ó milosti prerozkošná!
\par 7 Ta postava tvá podobna jest palme, a prsy tvé hroznum.
\par 8 Rekl jsem: Vstoupím na palmu, dosáhnu vrchu jejích. Nechažt tedy jsou prsy tvé jako hroznové vinného kmene, a vune chrípí tvých jako jablek vonných.
\par 9 Ústa tvá jako víno výborné, milá pro uprímnost, pusobící, aby i tech, jenž spí, rtové mluvili.
\par 10 Já jsem milého svého, a ke mne jest žádost jeho.
\par 11 Pod, milý muj, vyjdeme na pole, prenocujme ve vsech.
\par 12 Ráno privstaneme, na vinice pohledíme, kvete-li vinný kmen, již-li se ukázal zacátek hroznu, kvetou-li jablka zrnatá, a tut dám tobe milosti své.
\par 13 Pekná jablecka vydala vuni, a na dverech našich všecky rozkoše nové i staré, milý muj, zachovala jsem tobe.

\chapter{8}

\par 1 Ó bys byl jako bratr muj požívající prsí matky mé, abych te naleznuc vne, políbila, a nebyla zahanbena.
\par 2 Vedla bych te, a uvedla do domu matky své, a tu bys mne vyucoval; a ját bych dala píti vína strojeného, a mstu z jablek zrnatých.
\par 3 Levice jeho pod hlavou mou, a pravicí svou objímá mne.
\par 4 Prísahou vás zavazuji, dcery Jeruzalémské, abyste nebudily a nevyrážely ze sna milého mého, dokudž by sám nechtel.
\par 5 Která jest to, jenž vstupuje z poušte, zpolehši na milého svého? Pod jabloní vzbudila jsem te; tut tebe pocala matka tvá, tut te pocala rodicka tvá.
\par 6 Položiž mne jako pecet na srdce své, jako pecetní prsten na ruku svou. Nebo silné jest jako smrt milování, tvrdá jako hrob horlivost; uhlí její uhlí reravé, plamen nejprudší.
\par 7 Vody mnohé nemohly by uhasiti tohoto milování, aniž ho reky zatopí. Kdyby nekdo dáti chtel všecken statek domu svého za takovou milost, se vším tím pohrdnut by byl.
\par 8 Sestru máme malickou, kteráž ješte nemá prsí. Co uciníme s sestrou svou v den, v kterýž bude rec o ní?
\par 9 Jestliže jest zed, vzdeláme na ní palác stríbrný; pakli jest dvermi, obložíme je dskami cedrovými.
\par 10 Já jsem zed, a prsy mé jako veže; takž jsem byla pred ocima jeho, jako nacházející pokoj.
\par 11 Vinici mel Šalomoun v Balhamon, kteroužto pronajal strážným, aby jeden každý prinášel za ovoce její tisíc stríbrných.
\par 12 Ale vinice má, kterouž mám, prede mnou jest. Mejž sobe ten tisíc, ó Šalomoune, a dve ste ti, kteríž ostríhají ovoce jejího.
\par 13 Ó ty, kteráž bydlíš v zahradách, prátelét pozorují hlasu tvého, ohlašujž mi se.
\par 14 Pospeš, milý muj, a pripodobni se k srne nebo mladému jelenu, na horách vonných vecí.

\end{document}