\begin{document}

\title{Genesis}

\chapter{1}

\par 1 Na pocátku stvoril Buh nebe a zemi.
\par 2  Zeme pak byla neslicná a pustá, a tma byla nad propastí, a Duch Boží vznášel se nad vodami.
\par 3  I rekl Buh: Bud svetlo! I bylo svetlo.
\par 4  A videl Buh svetlo, že bylo dobré; i oddelil Buh svetlo od tmy.
\par 5  A nazval Buh svetlo dnem, a tmu nazval nocí. I byl vecer a bylo jitro, den první.
\par 6  Rekl také Buh: Bud obloha u prostred vod, a del vody od vod!
\par 7  I ucinil Buh tu oblohu, a oddelil vody, kteréž jsou pod oblohou, od vod, kteréž jsou nad oblohou. A stalo se tak.
\par 8  I nazval Buh oblohu nebem. I byl vecer a bylo jitro, den druhý.
\par 9  Rekl také Buh: Shromaždte se vody, kteréž jsou pod nebem, v místo jedno, a ukaž se místo suché! A stalo se tak.
\par 10  I nazval Buh místo suché zemí, shromáždení pak vod nazval morem. A videl Buh, že to bylo dobré.
\par 11  Potom rekl Buh: Zplod zeme trávu, a bylinu vydávající síme, a strom plodný, nesoucí ovoce podlé pokolení svého, v nemž by bylo síme jeho na zemi. A stalo se tak.
\par 12  Nebo zeme vydala trávu, a bylinu nesoucí semeno podlé pokolení svého, i strom prinášející ovoce, v nemž bylo síme jeho, podlé pokolení jeho. A videl Buh, že to bylo dobré.
\par 13  I byl vecer a bylo jitro, den tretí.
\par 14  Opet rekl Buh: Budte svetla na obloze nebeské, aby oddelovala den od noci, a byla na znamení a rozmerení casu, dnu a let.
\par 15  A aby svítila na obloze nebeské, a osvecovala zemi. A stalo se tak.
\par 16  I ucinil Buh dve svetla veliká, svetlo vetší, aby správu drželo nade dnem, a svetlo menší, aby správu drželo nad nocí; též i hvezdy.
\par 17  A postavil je Buh na obloze nebeské, aby osvecovala zemi;
\par 18  A aby správu držela nade dnem a nocí, a oddelovala svetlo od tmy. A videl Buh, že to bylo dobré.
\par 19  I byl vecer a bylo jitro, den ctvrtý.
\par 20  Rekl ješte Buh: Vydejte vody hmyz duše živé v hojnosti, a
\par 21  I stvoril Buh velryby veliké a všelijakou duši živou, hýbající se, kteroužto v hojnosti vydaly vody podlé pokolení jejich, a všeliké ptactvo krídla mající, podlé pokolení jeho. A videl Buh, že to bylo dobré.
\par 22  I požehnal jim Buh, rka: Plodtež se a množte se, a naplnte vody morské; též ptactvo at se rozmnožuje na zemi!
\par 23  I byl vecer a bylo jitro, den pátý.
\par 24  Rekl též Buh: Vydej zeme duši živou, jednu každou podlé pokolení jejího, hovada a zemeplazy, i zver zemskou, podlé pokolení jejího. A stalo se tak.
\par 25  I ucinil Buh zver zemskou podlé pokolení jejího, též hovada vedlé pokolení jejich, i všeliký zemeplaz podlé pokolení jeho. A videl Buh, že bylo dobré.
\par 26  Rekl opet Buh: Ucinme cloveka k obrazu našemu, podlé podobenství našeho, a at panují nad rybami morskými, a nad ptactvem nebeským, i nad hovady, a nade vší zemí, i nad všelikým zemeplazem hýbajícím se na zemi.
\par 27  I stvoril Buh cloveka k obrazu svému, k obrazu Božímu stvoril jej, muže a ženu stvoril je.
\par 28  A požehnal jim Buh, a rekl jim Buh: Plodtež se a rozmnožujte se, a naplnte zemi, a podmante ji, a panujte nad rybami morskými, a nad ptactvem nebeským, i nad všelikým živocichem hýbajícím se na zemi.
\par 29  Rekl ješte Buh: Aj, dal jsem vám všelikou bylinu, vydávající síme, kteráž jest na tvári vší zeme, a všeliké stromoví, (na nemž jest ovoce stromu), nesoucí síme; to bude vám za pokrm.
\par 30  Všechnem pak živocichum zemským, i všemu ptactvu nebeskému, a všemu tomu, což se hýbe na zemi, v cemž jest duše živá, všelikou bylinu zelenou dal jsem ku pokrmu. I stalo se tak.
\par 31  A videl Buh vše, což ucinil, a aj, bylo velmi dobré. I byl vecer a bylo jitro, den šestý.

\chapter{2}

\par 1 A tak dokonána jsou nebesa a zeme, i všecko vojsko jejich.
\par 2 A dokonal Buh dne sedmého dílo své, kteréž delal; a odpocinul v den sedmý ode všeho díla svého, kteréž byl delal.
\par 3 I požehnal Buh dni sedmému a posvetil ho; nebo v nem odpocinul Buh ode všeho díla svého, kteréž byl stvoril, aby ucineno bylo.
\par 4 Tit jsou rodové nebes a zeme, (když stvorena jsou v den, v nemž ucinil Hospodin Buh zemi i nebe),
\par 5 I každé chrastiny polní, dríve než byla na zemi, i všeliké byliny polní, prvé než vzcházela; nebo ješte byl nedštil Hospodin Buh na zemi, aniž byl který clovek, ješto by delal zemi.
\par 6 A aniž pára vystupovala z zeme, aby svlažovala všecken svrchek zeme.
\par 7 I ucinil Hospodin Buh cloveka z prachu zeme, a vdechl v chrípe jeho dchnutí života, i byl clovek v duši živou.
\par 8 Štípil pak byl Hospodin Buh ráj v Eden na východ, a postavil tam cloveka, jehož byl ucinil.
\par 9 A vyvedl Hospodin Buh z zeme všeliký strom na pohledení libý, a ovoce k jídlu chutné; též strom života u prostred ráje, i strom vedení dobrého a zlého.
\par 10 (A reka vycházela z Eden, k svlažování ráje, a odtud delila se, a byla ve ctyri hlavní reky.
\par 11 Jméno jedné Píson, ta obchází všecku zemi Hevilah, kdež jest zlato.
\par 12 A zlato zeme té jest výborné; tam jest i bdelium, a kámen onychin.
\par 13 Jméno pak druhé reky Gihon, ta obchází všecku zemi Chus.
\par 14 A jméno reky tretí Hiddekel, kteráž tece k východní strane Assyrské zeme. A reka ctvrtá jest Eufrates).
\par 15 Pojav tedy Hospodin Buh cloveka, postavil jej v ráji v zemi Eden, aby jej delal a ostríhal ho.
\par 16 I zapovedel Hospodin Buh cloveku, rka: Z každého stromu rajského svobodne jísti budeš;
\par 17 Ale z stromu vedení dobrého a zlého nikoli nejez; nebo v který bys koli den z neho jedl, smrtí umreš.
\par 18 Rekl byl také Hospodin Buh: Není dobré cloveku býti samotnému; uciním jemu pomoc, kteráž by pri nem byla.
\par 19 (Nebo když byl ucinil Hospodin Buh z zeme všelikou zver polní, i všecko ptactvo nebeské, privedl je k Adamovi, aby pohledel na ne, jaké by jméno kterému dáti mel; a jak by koli nazval Adam kterou duši živou, tak aby jmenována byla.
\par 20 I dal Adam jména všechnem hovadum, i ptactvu nebeskému, a všeliké zveri polní; Adamovi pak není nalezena pomoc, kteráž by pri nem byla.)
\par 21 Protož uvedl Hospodin Buh tvrdý sen na Adama, i usnul; a vynal jedno z žeber jeho, a to místo vyplnil telem.
\par 22 A z toho žebra, kteréž vynal z Adama, vzdelal Hospodin Buh ženu, a privedl ji k Adamovi.
\par 23 I rekl Adam: Ted tato jest kost z kostí mých a telo z tela mého; tato slouti bude mužatka, nebo z muže vzata jest.
\par 24 Z té príciny opustí muž otce svého i matku svou, a prídržeti se bude manželky své, i budou v jedno telo.
\par 25 Byli pak oba dva nazí, Adam i žena jeho, a nestydeli se.

\chapter{3}

\par 1 Had pak byl nejchytrejší ze všech živocichu polních, kteréž byl ucinil Hospodin Buh. A ten rekl žene: Tak-liž jest, že vám Buh rekl: Nebudete jísti z každého stromu rajského?
\par 2 I rekla žena hadu: Ovoce stromu rajských jíme;
\par 3 Ale o ovoci stromu, kterýž jest u prostred ráje, rekl Buh: Nebudete ho jísti, aniž se ho dotknete, abyste nezemreli.
\par 4 I rekl had žene: Nikoli nezemrete smrtí!
\par 5 Ale ví Buh, že v kterýkoli den z neho jísti budete, otevrou se oci vaše; a budete jako bohové, vedouce dobré i zlé.
\par 6 Viduci tedy žena, že dobrý jest strom k jídlu i príjemný ocima, a k nabytí rozumnosti strom žádostivý, vzala z ovoce jeho a jedla; dala také i muži svému s sebou, a on jedl.
\par 7 Tedy otevríny jsou oci obou dvou, a poznali, že jsou nazí; i navázali lístí fíkového a nadelali sobe veníku.
\par 8 A v tom uslyšeli hlas Hospodina Boha chodícího po ráji k vetru dennímu; i skryl se Adam a žena jeho pred tvárí Hospodina Boha, u prostred stromoví rajského.
\par 9 I povolal Hospodin Buh Adama, a rekl jemu: Kdež jsi?
\par 10 Kterýžto rekl: Hlas tvuj slyšel jsem v ráji a bál jsem se, že jsem nahý; protož skryl jsem se.
\par 11 I rekl Buh: Kdožt oznámil, že jsi nahý? Nejedl-lis ale z toho stromu, z nehožt jsem jísti zapovedel?
\par 12 I rekl Adam: Žena, kterouž jsi mi dal, aby byla se mnou, ona mi dala z stromu toho, a jedl jsem.
\par 13 I rekl Hospodin Buh žene: Což jsi to ucinila? I rekla žena: Had mne podvedl, i jedla jsem.
\par 14 Tedy rekl Hospodin Buh hadu: Že jsi to ucinil, zlorecený budeš nade všecka hovada a nade všecky živocichy polní; po briše svém plaziti se budeš, a prach žráti budeš po všecky dny života svého.
\par 15 Nad to, neprátelství položím mezi tebou a mezi ženou, i mezi semenem tvým a semenem jejím; ono potre tobe hlavu, a ty potreš jemu patu.
\par 16 Žene pak rekl: Velice rozmnožím bolesti tvé a pocínání tvá, s bolestí roditi budeš deti, a pod mocí muže tvého bude žádost tvá, a on panovati bude nad tebou.
\par 17 Adamovi také rekl: Že jsi uposlechl hlasu ženy své, a jedl jsi z stromu toho, kterýžt jsem zapovedel, rka: Nebudeš jísti z neho; zlorecená zeme pro tebe, s bolestí jísti budeš z ní po všecky dny života svého.
\par 18 Trní a bodlácí tobe ploditi bude, i budeš jísti byliny polní.
\par 19 V potu tvári své chléb jísti budeš, dokavadž se nenavrátíš do zeme, ponevadž jsi z ní vzat. Nebo prach jsi a v prach se navrátíš.
\par 20 Dal pak byl Adam jméno žene své Eva, proto že ona byla máte všech živých.
\par 21 I zdelal Hospodin Buh Adamovi a žene jeho odev kožený, a priodel je.
\par 22 Tedy rekl Hospodin Buh: Aj, clovek ucinen jest jako jeden z nás, veda dobré i zlé; procež nyní, aby nevztáhl ruky své, a nevzal také z stromu života, a jedl by, i byl by živ na veky, vyženme jej.
\par 23 I vypustil jej Hospodin Buh z zahrady Eden, aby delal zemi, z níž vzat byl.
\par 24 A tak vyhnal cloveka a osadil zahradu Eden cherubíny k východní strane s mecem plamenným blýskajícím se, aby ostríhali cesty k stromu života.

\chapter{4}

\par 1 Adam pak poznal Evu ženu svou, kterážto pocavši, porodila Kaina a rekla: Obdržela jsem muže na Hospodinu.
\par 2 A opet porodila bratra jeho Abele. I byl Abel pastýr ovcí, a Kain byl orác.
\par 3 Po mnohých pak dnech stalo se, že obetoval Kain z úrody zemské obet Hospodinu.
\par 4 Ano i Abel také obetoval z prvorozených vecí stáda svého, a z tuku jejich. I vzhlédl Hospodin na Abele a na obet jeho.
\par 5 Na Kaina pak a na obet jeho nevzhlédl. Protož rozlítil se Kain náramne, a opadla tvár jeho.
\par 6 I rekl Hospodin Kainovi: Proc jsi se tak rozpálil hnevem? A proc jest opadla tvár tvá?
\par 7 Zdaliž nebudeš príjemný, budeš-li dobre ciniti? Pakli nebudeš dobre ciniti, hrích ve dverích leží; a pod mocí tvou bude žádost jeho, a ty panovati budeš nad ním.
\par 8 I mluvil Kain k Abelovi bratru svému. Stalo se pak, když byli na poli, že povstav Kain proti Abelovi bratru svému, zabil jej.
\par 9 I rekl Hospodin Kainovi: Kdež jest Abel bratr tvuj? Kterýž odpovedel: Nevím. Zdaliž jsem já strážným bratra svého?
\par 10 I rekl Buh: Co jsi ucinil? Hlas krve bratra tvého volá ke mne z zeme.
\par 11 Protož nyní zlorecený budeš i od té zeme, kteráž otevrela ústa svá, aby prijala krev bratra tvého z ruky tvé.
\par 12 Když budeš delati zemi, nebude více vydávati moci své tobe; tulákem a behounem budeš na zemi.
\par 13 I rekl Kain Hospodinu: Vetšít jest nepravost má, než aby mi odpuštena býti mohla.
\par 14 Aj, vyháníš mne dnes z zeme této, a pred tvárí tvou skrývati se budu, a budu tulákem a behounem na zemi. I prijde na to, že kdo mne koli nalezne, zabije mne.
\par 15 I rekl mu Hospodin: Zajisté kdo by koli zabil Kaina, nad tím sedmnásobne mšteno bude. Procež vložil Hospodin znamení naKaina, aby ho žádný nezabil, kdo by jej koli nalezl.
\par 16 Tedy odšed Kain od tvári Hospodinovy, bydlil v zemi Nód, k východní strane naproti Eden.
\par 17 Poznal pak Kain ženu svou, kterážto pocala a porodila Enocha. I stavel mesto, a nazval jméno mesta toho jménem syna svého Enoch.
\par 18 I narodil se Enochovi Irád, a Irád zplodil Maviaele, Maviael pak zplodil Matuzaele, a Matuzael zplodil Lámecha.
\par 19 Vzal sobe pak Lámech dve ženy; jméno jedné Ada, a jméno druhé Zilla.
\par 20 I porodila Ada Jábale, kterýž byl otec prebývajících v staních a stádo pasoucích.
\par 21 A jméno bratra jeho Jubal; ten byl otec všech hrajících na harfu a nástroje hudebné.
\par 22 A Zilla také porodila Tubalkaina, kterýž byl remeslník všelikého díla od medi a od železa. Sestra pak Tubalkainova byla Noéma.
\par 23 I rekl Lámech ženám svým, Ade a Zille: Slyšte hlas muj, ženy Lámechovy, poslouchejte reci mé, že jsem zabil muže k ráne své a mládence k zsinalosti své.
\par 24 Jestližet sedmnásobne pomšteno bude pro Kaina, tedy pro Lámecha sedmdesátekrát sedmkrát.
\par 25 Poznal pak ješte Adam ženu svou, i porodila syna a nazvala jméno jeho Set; nebo rekla: Dal mi Buh jiné síme místo Abele, kteréhož zabil Kain.
\par 26 Setovi pak také narodil se syn, a nazval jméno jeho Enos. Tehdáž zacalo se vzývání jména Hospodinova.

\chapter{5}

\par 1 Tato jest kniha rodu Adamových. V ten den, v kterémž stvoril Buh cloveka, ku podobenství Božímu ucinil ho.
\par 2 Muže a ženu stvoril je a požehnal jim, a nazval jméno jejich Adam v ten den, když stvoreni jsou.
\par 3 Byl pak Adam ve stu a tridcíti letech, když zplodil syna ku podobenství svému a k obrazu svému, a nazval jméno jeho Set.
\par 4 I bylo dnu Adamových po zplození Seta osm set let, a plodil syny a dcery.
\par 5 A tak bylo všech dnu Adamových, v kterýchž byl živ, devet set a tridceti let, i umrel.
\par 6 Set pak byl ve stu a peti letech, když zplodil Enosa.
\par 7 A po zplození Enosa živ byl Set osm set a sedm let, a plodil syny a dcery.
\par 8 I bylo všech dnu Setových devet set a dvanácte let, i umrel.
\par 9 Byl pak Enos v devadesáti letech, když zplodil Kainana.
\par 10 A po zplození Kainana živ byl Enos osm set a patnácte let, a plodil syny a dcery.
\par 11 I bylo všech dnu Enosových devet set a pet let, i umrel.
\par 12 Kainan pak byl v sedmdesáti letech, když zplodil Mahalaleele.
\par 13 A po zplození Mahalaleele živ byl Kainan osm set a ctyridceti let, a plodil syny a dcery.
\par 14 I bylo všech dnu Kainanových devet set a deset let, i umrel.
\par 15 Mahalaleel pak byl v šedesáti a peti letech, když zplodil Járeda.
\par 16 A po zplození Járeda živ byl Mahalaleel osm set a tridceti let, a plodil syny a dcery.
\par 17 I bylo všech dnu Mahalaleelových osm set devadesáte a pet let, i umrel.
\par 18 Járed pak byl ve stu šedesáti a dvou letech, když zplodil Enocha.
\par 19 A po zplození Enocha živ byl Járed osm set let, a plodil syny a dcery.
\par 20 I bylo všech dnu Járedových devet set šedesáte a dve léte, i umrel.
\par 21 Enoch pak byl v šedesáti a peti letech, když zplodil Matuzaléma.
\par 22 A chodil Enoch stále s Bohem po zplození Matuzaléma tri sta let, a plodil syny a dcery.
\par 23 I bylo všech dnu Enochových tri sta šedesáte a pet let.
\par 24 A chodil Enoch stále s Bohem a nebyl více vidín; nebo vzal ho Buh.
\par 25 Matuzalém pak byl ve stu osmdesáti sedmi letech, když zplodil Lámecha.
\par 26 A po zplození Lámecha živ byl Matuzalém sedm set osmdesáte a dve léte, a plodil syny a dcery.
\par 27 I bylo všech dnu Matuzalémových devet set šedesáte a devet let, i umrel.
\par 28 Lámech pak byl ve stu osmdesáti a dvou letech, když zplodil syna,
\par 29 Jehož jméno nazval Noé, rka: Tento nám odpocinutí zpusobí od díla našeho,od práce rukou našich, kterouž máme s zemí, jížto zlorecil Hospodin.
\par 30 A živ byl Lámech potom, když zplodil Noé, pet set devadesáte a pet let, a plodil syny a dcery.
\par 31 I bylo všech dnu Lámechových sedm set sedmdesáte a sedm let, i umrel.
\par 32 A když byl Noé v peti stech letech, zplodil Sema, Chama a Jáfeta.

\chapter{6}

\par 1 Stalo se pak, když se pocali množiti lidé na zemi, a dcery se jim zrodily,
\par 2 Že vidouce synové Boží dcery lidské, any krásné jsou, brali sobe ženy ze všech, kteréž oblibovali.
\par 3 Procež rekl Hospodin: Nebude se nesnaditi duch muj s clovekem na veky, proto že také telo jest, a bude dnu jeho sto a dvadceti let.
\par 4 Obrové pak byli na zemi v tech dnech; ano i potom, když vcházeli synové Boží k dcerám lidským, ony rodily jim. To jsou ti mocní, kteríž zdávna byli, muži na slovo vzatí.
\par 5 Ale když videl Hospodin, an se rozmnožuje zlost lidská na zemi, a že by všeliké myšlení srdce jejich nebylo než zlé po všecken cas,
\par 6 Litoval Hospodin, že ucinil cloveka na zemi, a bolest mel v srdci svém.
\par 7 Tedy rekl Hospodin: Vyhladím z zeme cloveka, kteréhož jsem stvoril, od cloveka až do hovada, až do zemeplazu, až i do ptactva nebeského; nebo líto mi, že jsem je ucinil.
\par 8 Ale Noé našel milost pred Hospodinem.
\par 9 Tito jsou príbehové Noé: Noé muž spravedlivý, dokonalý byl za svého veku, s Bohem ustavicne chodil Noé.
\par 10 (Zplodil pak Noé tri syny: Sema, Chama a Jáfeta.)
\par 11 Ale zeme byla porušena pred Bohem, a naplnena byla zeme nepravostí.
\par 12 Videl tedy Buh zemi, a aj, porušena byla, nebo bylo porušilo všeliké telo cestu svou na zemi.
\par 13 Protož rekl Buh k Noé: Konec všelikého tela prichází prede mne, nebo naplnena jest zeme nepravostí od nich; z té príciny, hle, již zkazím je s zemí.
\par 14 Ucin sobe koráb z dríví gofer; príhrady zdeláš v tom korábu, a oklejuješ jej vnitr i zevnitr klím.
\par 15 A na tento zpusob udeláš jej: Trí set loktu bude dlouhost toho korábu, padesáti loktu širokost jeho a tridceti loktu vysokost jeho.
\par 16 Okno udeláš v korábu, a svrchkem na loket vysokým zavreš jej; dvére také korábu v boku jeho postavíš, a pokoje spodní, druhé i tretí zdeláš v nem.
\par 17 Já pak, aj, já uvedu potopu vod na zemi, aby zkaženo bylo všeliké telo, v nemž jest duch života pod nebem. Cožkoli bude na zemi, umre.
\par 18 S tebou však uciním smlouvu svou; a vejdeš do korábu, ty i synové tvoji, žena tvá i ženy synu tvých s tebou.
\par 19 A ze všech živocichu všelikého tela, po dvém z každého uvedeš do korábu, abys je živé zachoval s sebou; samec a samice budou.
\par 20 Z ptactva podlé pokolení jeho, a z hovad podlé pokolení jejich, ze všelikého také zemeplazu podlé pokolení jeho, po dvém z každého vejdou k tobe, aby živi zustali.
\par 21 Ty pak naber s sebou všeliké potravy, kteráž se jísti muže, a shromažd sobe, aby byla tobe i jim ku pokrmu.
\par 22 I ucinil Noé podlé všeho, jakž mu rozkázal Buh, tak ucinil.

\chapter{7}

\par 1 Potom rekl Hospodin k Noé: Vejdiž ty i všecka celed tvá do korábu; nebo jsem te videl spravedlivého pred sebou v národu tomto.
\par 2 Ze všech hovad cistých vezmeš sobe sedmero a sedmero, samce a samici jeho, ale z hovad necistých dvé a dvé, samce a samici jeho.
\par 3 Z ptactva také nebeského sedmero a sedmero, samce a samici, aby živé zachováno bylo síme na vší zemi.
\par 4 Nebo po dnech ješte sedmi já dštíti budu na zemi za ctyridceti dnu a ctyridceti nocí; a vyhladím se svrchku zeme všelikou podstatu, kterouž jsem ucinil.
\par 5 Tedy ucinil Noé všecko tak, jakž mu prikázal Hospodin.
\par 6 (Byl pak Noé v šesti stech letech, když ta potopa prišla na zemi.)
\par 7 A protož prišel Noé a synové jeho, i žena jeho, i ženy synu jeho s ním k korábu, pro vody potopy.
\par 8 Z hovad také cistých i z hovad necistých, i z ptactva a ze všeho, což se hýbe na zemi,
\par 9 Po dvém vešli k Noé do korábu, samec a samice, tak jakž byl rozkázal Buh Noé.
\par 10 Stalo se pak po sedmi dnech, že vody potopy prišly na zemi.
\par 11 Léta šestistého veku Noé, druhého mesíce, sedmnáctého dne téhož mesíce, v ten den protrženy jsou všecky studnice propasti veliké, a pruduchové nebeští otevríni jsou.
\par 12 I byl príval na zemi ctyridceti dní a ctyridceti nocí.
\par 13 Toho dne všel Noé, Sem a Cham i Jáfet, synové Noé, žena Noé, a tri ženy synu jeho s ním do korábu.
\par 14 Oni i všeliký živocich podlé pokolení svého, i všeliké hovado podlé pokolení svého, a všeliký zemeplaz, kterýž se hýbe na zemi, podlé pokolení svého, i všeliké ptactvo vedlé pokolení svého, všelijací ptáci, všecko, což krídla má,
\par 15 Vešli k Noé do korábu, po dvém ze všelikého tela, v nemž byl duch života.
\par 16 A což jich vešlo, samec a samice ze všelikého tela vešli, tak jakž byl prikázal jemu Buh, a zavrel Hospodin po nem.
\par 17 A když byla potopa za ctyridceti dnu na zemi, tedy rozmnoženy jsou vody, až i vyzdvihly koráb, a vznesly jej od zeme.
\par 18 Nebo zmohly se vody a rozmnoženy jsou velmi nad zemí, i zplýval koráb na vodách.
\par 19 A tak náramne rozmohly se vody nad zemí, že prikryty jsou všecky hory nejvyšší, kteréž byly pode vším nebem.
\par 20 Patnácte loktu zvýší rozmohly se vody, když prikryty jsou hory.
\par 21 I umrelo všeliké telo, kteréž se hýbe na zemi, tak z ptactva, jako z hovad a živocichu, i všelikého hmyzu, kterýž se plazí po zemi, i každého cloveka.
\par 22 Všecko, což melo dýchání ducha života v chrípích svých, ze všeho, což bylo na suše, pomrelo.
\par 23 A tak vyhladil Buh všelikou podstatu, kteráž byla na tvári zeme, od cloveka až do hovada, až do zemeplazu, a až do ptactva nebeského, vyhlazeno jest, pravím, z zeme; a zustal toliko Noé, a kteríž s ním byli v korábu.
\par 24 I trvaly vody nad zemí za sto a padesáte dnu.

\chapter{8}

\par 1 Rozpomenul se pak Buh na Noé, i všecky živocichy a všecka hovada, kteráž byla s ním v korábu; procež uvedl Buh vítr na zemi, i zastavily se vody.
\par 2 A zavríny jsou studnice propasti i pruduchové nebeští, a zastaven jest príval s nebe.
\par 3 I navrátily se vody se svrchku zeme, odcházejíce zase, a opadly vody po stu a padesáti dnech,
\par 4 Tak že odpocinul koráb sedmého mesíce, v sedmnáctý den toho mesíce na horách Ararat.
\par 5 Když pak vody odcházely a opadaly až do desátého mesíce, prvního dne téhož desátého mesíce ukázali se vrchové hor.
\par 6 I stalo se po ctyridcíti dnech, otevrev Noé okno v korábu, kteréž byl udelal,
\par 7 Vypustil krkavce. Kterýžto vyletuje zase se vracoval, dokudž nevyschly vody na zemi.
\par 8 Potom vypustil holubici od sebe, aby vedel, již-li by opadly vody se svrchku zeme.
\par 9 Kterážto když nenašla, kde by odpocinula noha její, navrátila se k nemu do korábu; nebo vody byly po vší zemi. On pak vztáhna ruku svou, vzal ji, a vnesl k sobe do korábu.
\par 10 A pocekal ješte sedm dní jiných, a opet vypustil holubici z korábu.
\par 11 I priletela k nemu holubice k vecerou, a aj, list olivový utržený v ústech jejích. Tedy poznal Noé, že opadly vody se svrchku zeme.
\par 12 I cekal ješte sedm dní jiných, a opet vypustil holubici, kterážto nevrátila se k nemu více.
\par 13 I stalo se šestistého prvního léta, v první den mesíce prvního, že vyschly vody na zemi. I odjal Noé prikrytí korábu a uzrel, ano již oschl svrchek zeme.
\par 14 Druhého pak mesíce, v dvadcátý sedmý den téhož mesíce oschla zeme.
\par 15 I mluvil Buh k Noé, rka:
\par 16 Vyjdi z korábu, ty i žena tvá, a synové tvoji, i ženy synu tvých s tebou.
\par 17 Všecky živocichy, kteríž jsou s tebou ze všelikého tela, tak z ptactva jako z hovad a všelikého zemeplazu, kterýž se hýbe na zemi, vyved s sebou; at se v hojnosti rozplozují na zemi, a rostou a množí se na zemi.
\par 18 I vyšel Noé a synové jeho, i žena jeho a ženy synu jeho s ním;
\par 19 Každý živocich, každý zemeplaza všecko ptactvo, všecko, což se hýbe na zemi, po pokoleních svých vyšlo z korábu.
\par 20 Tedy vzdelal Noé oltár Hospodinu, a vzav ze všech hovad cistých i ze všeho ptactva cistého, obetoval zápaly na tom oltári.
\par 21 I zachutnal Hospodin vuni tu príjemnou, a rekl Hospodin v srdci svém: Nebudu více zloreciti zemi pro cloveka, proto že myšlení srdce lidského zlé jest od mladosti jeho; aniž budu více bíti všeho, což živo jest, jako jsem ucinil.
\par 22 Nýbrž dokavadž zeme trvati bude, setí a žen, studeno i horko, léto a zima, den také a noc neprestanou.

\chapter{9}

\par 1 Tedy požehnal Buh Noé i synum jeho a rekl jim: Plodtež se a rozmnožujte se, a naplnte zemi.
\par 2 Strach váš a hruza vaše bud na všeliký živocich zeme, a na všecko ptactvo nebeské. Všecko, což se hýbe na zemi, a všecky ryby morské v ruce vaše dány jsou.
\par 3 Všecko, což se hýbe a jest živo, bude vám za pokrm; jako i bylinu zelenou, dal jsem vám to všecko.
\par 4 A však masa s duší jeho, kteráž jest krev jeho, nebudete jísti.
\par 5 A zajisté krve vaší, duší vašich vyhledávati budu; z rukou každého hovada vyhledávati jí budu, i z ruky cloveka, ano i z ruky každého bratra jeho budu vyhledávati duše cloveka.
\par 6 Kdo by koli vylil krev cloveka, skrze cloveka vylita bude krev jeho; nebo k obrazu svému ucinil Buh cloveka.
\par 7 Vy pak plodte a množte se; v hojnosti se rozplodte na zemi, a rozmnoženi budte na ní.
\par 8 I mluvil Buh k Noé a synum jeho s ním, rka:
\par 9 Já zajisté vcházím v smlouvu svou s vámi, i s semenem vaším po vás,
\par 10 A se všelikou duší živou, kteráž jest s vámi, z ptactva, z hovad a ze všech živocichu zemských, kteríž jsou s vámi, ode všech, kteríž vyšli z korábu, až do všelikého živocicha zemského.
\par 11 Protož utvrzuji smlouvu svou s vámi, že nebude vyhlazeno více všeliké telo vodami potopy; aniž bude více potopa k zkažení zeme.
\par 12 I rekl Buh: Totot bude znamení smlouvy, kteréž já dávám, mezi mnou a mezi vámi, a mezi všelikou duší živou, kteráž jest s vámi, po všecky veky.
\par 13 Duhu svou postavil jsem na oblaku, a bude na znamení smlouvy mezi mnou a mezi zemí.
\par 14 A budet, když uvedu mracný oblak nad zemí, a ukáže se duha na oblaku,
\par 15 Že se rozpomenu na smlouvu svou, kteráž jest mezi mnou a mezi vámi a mezi všelikou duší živou v každém tele; a nebudou více vody ku potope, aby zahladily všeliké telo.
\par 16 Nebo když bude duha ta na oblaku, popatrím na ni, abych se rozpomenul na smlouvu vecnou mezi Bohem a mezi všelikou duší živou v každém tele, kteréž jest na zemi.
\par 17 I rekl Buh k Noé: Tot jest znamení smlouvy, kterouž jsem utvrdil mezi sebou a mezi všelikým telem, kteréž jest na zemi.
\par 18 Byli pak synové Noé, kteríž vyšli z korábu: Sem, Cham a Jáfet; a Cham byl otec Kanánuv.
\par 19 Ti tri jsou synové Noé, a ti se rozprostreli po vší zemi.
\par 20 Noé pak obíraje se s zemí, zacal delati vinice.
\par 21 A pije víno, opil se, a obnažil se u prostred stanu svého.
\par 22 Videl pak Cham, otec Kanánuv, hanbu otce svého, a povedel obema bratrím svým vne.
\par 23 Tedy vzali Sem a Jáfet odev, kterýžto oba položili na ramena svá, a jdouce zpátkem, zakryli hanbu otce svého; tvári pak jich byly odvráceny, a hanby otce svého nevideli.
\par 24 Procítiv pak Noé po svém víne, zvedel, co mu ucinil syn jeho mladší.
\par 25 I rekl: Zlorecený Kanán, služebník služebníku bude bratrím svým.
\par 26 Rekl také: Požehnaný Hospodin, Buh Semuv, a bud Kanán služebníkem jejich.
\par 27 Rozširiž Buh milostive Jáfeta, aby bydlil v stáncích Semových, a bud Kanán služebníkem jejich.
\par 28 Živ pak byl Noé po potope tri sta a padesáte let.
\par 29 A tak bylo všech dnu Noé devet set a padesáte let; i umrel jest.

\chapter{10}

\par 1 Tito jsou pak rodové synu Noé, Sema, Chama a Jáfeta, jimž se tito synové zrodili po potope.
\par 2 Synové Jáfetovi: Gomer a Magog, a Madai, a Javan, a Tubal, a Mešech, a Tiras.
\par 3 Synové pak Gomerovi: Ascenez, Rifat, a Togorma.
\par 4 Synové pak Javanovi: Elisa a Tarsis, Cetim a Dodanim.
\par 5 Od tech rozdeleni jsou ostrovové národu po krajinách jejich, každý podlé jazyku svého, vedlé celedi své, v národech svých.
\par 6 Synové pak Chamovi: Chus a Mizraim a Put a Kanán.
\par 7 A synové Chusovi: Sába, Evila, a Sabata, a Regma, a Sabatacha. Synové pak Regmovi: Sába a Dedan.
\par 8 Zplodil také Chus Nimroda; ont jest pocal býti mocným na zemi.
\par 9 To byl silný lovec pred Hospodinem; protož se ríká: Jako Nimrod silný lovec pred Hospodinem.
\par 10 Pocátek pak jeho království byl Babylon a Erech, Achad a Chalne, v zemi Sinear.
\par 11 Z zeme té vyšel do Assur, kdežto vystavel Ninive, a Rohobot mesto, a Chále,
\par 12 A Rezen mezi Ninive a mezi Chále; tot jest mesto veliké.
\par 13 Mizraim pak zplodil Ludim a Anamim, a Laabim, a Neftuim,
\par 14 A Fetruzim, a Chasluim, (odkudž pošli Filistinští) a Kafturim.
\par 15 Kanán pak zplodil Sidona prvorozeného svého, a Het,
\par 16 A Jebuzea, a Amorea, a Gergezea,
\par 17 A Hevea, a Aracea, a Sinea,
\par 18 A Aradia, a Samarea, a Amatea; a potom odtud rozprostrely se celedi Kananejských.
\par 19 A bylo pomezí Kananejských od Sidonu, když jdeš k Gerar až do Gázy; a odtud když jdeš k Sodome a Gomore, a Adama a Seboim až do Lázy.
\par 20 Ti jsou synové Chamovi po celedech svých, vedlé jazyku svých, po krajinách svých, v národech svých.
\par 21 Semovi také, otci všech synu Heber, bratru Jáfeta staršího zrozeni jsou synové.
\par 22 A tito jsou synové Semovi: Elam, a Assur, a Arfaxad, a Lud, a Aram.
\par 23 Synové pak Aramovi: Hus, a Hul, a Geter, a Mas.
\par 24 Potom Arfaxad zplodil Sále; a Sále zplodil Hebera.
\par 25 Heberovi také narodili se dva synové; jméno jednoho Peleg, proto že za dnu jeho rozdelena byla zeme, a jméno bratra jeho Jektan.
\par 26 Jektan pak zplodil Elmodada, a Salefa, a Azarmota, a Járe,
\par 27 A Adoráma, a Uzala, a Dikla,
\par 28 A Obale, a Abimahele, a Sebai,
\par 29 A Ofira, a Evila, a Jobaba; všickni ti jsou synové Jektanovi.
\par 30 A bylo bydlení jejich od Mesa, když jdeš k Sefar hore na východ slunce.
\par 31 Tit jsou synové Semovi po celedech svých, vedlé jazyku svých, po krajinách svých, v národech svých.
\par 32 Ty jsou celedi synu Noé po rodech svých, v národech svých; a od tech rozdelili se národové na zemi po potope.

\chapter{11}

\par 1 Byla pak všecka zeme jazyku jednoho a reci jedné.
\par 2 I stalo se, když se brali od východu, nalezli pole v zemi Sinear, a bydlili tam.
\par 3 A rekli jeden druhému: Nuže, nadelejme cihel, a vypalme je ohnem. I meli cihly místo kamení, a zemi lepkou místo vápna.
\par 4 Nebo rekli: Nuže, vystavejme sobe mesto a veži, jejíž by vrch dosahal k nebi; a tak ucinme sobe jméno, abychom nebyli rozptýleni po vší zemi.
\par 5 Sstoupil pak Hospodin, aby videl to mesto a veži, kterouž staveli synové lidští.
\par 6 A rekl Hospodin: Aj, lid jeden a jazyk jeden všechnech techto, a tot jest zacátek díla jejich; nyní pak nedadí sobe v tom prekaziti, což umínili delati.
\par 7 Protož sstupme a zmetme tam jazyk jejich, aby jeden druhého jazyku nerozumel.
\par 8 A tak rozptýlil je Hospodin odtud po vší zemi; i prestali staveti mesta toho.
\par 9 Protož nazváno jest jméno jeho Bábel; nebo tu zmátl Hospodin jazyk vší zeme; a odtud rozptýlil je Hospodin po vší zemi.
\par 10 Titot jsou rodové Semovi: Sem, když byl ve stu letech, zplodil Arfaxada ve dvou letech po potope.
\par 11 A byl živ Sem po zplození Arfaxada pet set let; a plodil syny a dcery.
\par 12 Arfaxad pak živ byl pet a tridceti let, a zplodil Sále.
\par 13 A po zplození Sále živ byl Arfaxad ctyri sta a tri léta; a plodil syny a dcery.
\par 14 Sále také živ byl tridceti let, a zplodil Hebera.
\par 15 A živ byl Sále po zplození Hebera ctyri sta a tri léta; a plodil syny a dcery.
\par 16 Živ pak byl Heber ctyri a tridceti let, a zplodil Pelega.
\par 17 A živ byl Heber po zplození Pelega ctyri sta a tridceti let; a plodil syny a dcery.
\par 18 Peleg pak živ byl tridceti let, a zplodil Réhu.
\par 19 A živ byl Peleg po zplození Réhu dve ste a devet let; a plodil syny a dcery.
\par 20 Réhu také živ byl tridceti a dve léte, a zplodil Sáruga.
\par 21 A po zplození Sáruga živ byl Réhu dve ste a sedm let; a plodil syny a dcery.
\par 22 Živ pak byl Sárug tridceti let, a zplodil Náchora.
\par 23 A byl živ Sárug po zplození Náchora dve ste let; a plodil syny a dcery.
\par 24 Náchor pak živ byl dvadceti a devet let, a zplodil Táre.
\par 25 A živ byl Náchor po zplození Táre sto a devatenácte let; a plodil syny a dcery.
\par 26 Živ pak byl Táre sedmdesáte let, a zplodil Abrama, Náchora a Hárana.
\par 27 Tito jsou pak rodové Táre: Táre zplodil Abrama, Náchora a Hárana; Háran pak zplodil Lota.
\par 28 Umrel pak Háran prvé než Táre otec jeho v zemi narození svého, totiž v Ur Kaldejských.
\par 29 I zpojímali sobe ženy Abram a Náchor; jméno ženy Abramovy Sarai, a jméno ženy Náchorovy Melcha, dcera Háranova, kterýž byl otec Melchy a Jeschy.
\par 30 Byla pak Sarai neplodná, a nemela detí.
\par 31 I vzal Táre Abrama syna svého, a Lota syna Háranova, vnuka svého, a Sarai nevestu svou, ženu Abrama syna svého, a vyšli spolu z Ur Kaldejských, aby se brali do zeme Kananejské, a prišli až do Cháran, a bydlili tam.
\par 32 A byli dnové Táre dve ste a pet let; i umrel Táre v Cháran.

\chapter{12}

\par 1 Nebo byl rekl Hospodin Abramovi: Vyjdi z zeme své a z príbuznosti své, i z domu otce svého do zeme, kterouž ukáži tobe.
\par 2 A uciním te v národ veliký, a požehnám tobe, a zvelebím jméno tvé, a budeš požehnání.
\par 3 Požehnám také dobrorecícím tobe, a zlorecícím tobe zloreciti budu; ano požehnány budou v tobe všecky celedi zeme.
\par 4 I vyšel Abram, tak jakž mu byl mluvil Hospodin, a šel s ním Lot. (Byl pak Abram v sedmdesáti peti letech, když vyšel z Cháran.)
\par 5 A vzal Abram Sarai manželku svou, a Lota syna bratra svého, a všecko zboží své, kteréhož nabyli, i duše, kterýchž dosáhli v Cháran. A vyšedše, brali se do zeme Kananejské, až i prišli do ní.
\par 6 I prošel Abram tu zemi až k místu Sichem, to jest až k rovine More. A tehdáž Kananejští byli v zemi.
\par 7 I ukázal se Hospodin Abramovi a rekl: Semeni tvému dám zemi tuto. Tedy vzdelal tu oltár Hospodinu, kterýž se byl ukázal jemu.
\par 8 A odtud podal se k hore, kteráž leží na východ od Bethel, kdežto rozbil stan svuj, tak že mu Bethel byl na západ, Hai pak na východ; i vzdelal tam oltár Hospodinu, a vzýval jméno Hospodinovo.
\par 9 Potom hnul se Abram dále, a odebral se odtud ku poledni.
\par 10 Byl pak hlad v té zemi; protož sstoupil Abram do Egypta, aby tam byl pohostinu; nebo veliký byl hlad v té zemi.
\par 11 I stalo se, že když pricházel blízko k Egyptu, rekl k Sarai manželce své: Aj, nyní znám, že jsi žena krásné tvári.
\par 12 A stane se, že když te uzrí Egyptští, reknou: To jest manželka jeho; i zabijí mne, tebe pak živé nechají.
\par 13 Prav medle, že jsi sestra má, aby mi dobre bylo prícinou tvou, a živa zustala duše má pro tebe.
\par 14 I stalo se, když všel Abram do Egypta, videli Egyptští ženu, že krásná byla náramne.
\par 15 A vidouce ji knížata Faraonova, schválili ji pred ním; i vzata jest žena do domu Faraonova.
\par 16 Kterýžto i Abramovi dobre ucinil pro ni; a mel ovce a voly a osly, i služebníky a devky, též oslice a velbloudy.
\par 17 Ale Hospodin trápil Faraona ranami velikými, i dum jeho, pro Sarai manželku Abramovu.
\par 18 Protož povolal Farao Abrama a rekl: Cos mi to ucinil? Procežs mi neoznámil, že ona manželka tvá jest?
\par 19 Proc jsi rekl: Sestra má jest? A vzal jsem ji sobe za ženu. Protož nyní, ted máš manželku svou, vezmi a jdi.
\par 20 I porucil o nem Farao mužum, a propustili ho, i manželku jeho i všecko, což mel.

\chapter{13}

\par 1 Vstoupil tedy Abram z Egypta on i žena jeho i všecko, což mel, a Lot s ním, ku poledni.
\par 2 (Byl pak Abram bohatý velmi na dobytek, na stríbro i na zlato.)
\par 3 A šel cestami svými od poledne až do Bethel, až k místu tomu, kdež prvé byl stánek jeho, mezi Bethel a Hai,
\par 4 K místu oltáre, kterýž tam byl prvé vzdelal, kdežto vzýval Abram jméno Hospodinovo.
\par 5 Také i Lot, kterýž s Abramem chodil, mel ovce a voly i stany.
\par 6 A nemohla jim zeme postacovati, aby spolu bydlili, proto že zboží jich bylo veliké, tak že nemohli spolu bydliti.
\par 7 Odkudž vznikla nesnáz mezi pastýri stáda Abramova a mezi pastýri stáda Lotova; nebo Kananejští a Ferezejští tehdáž bydlili v zemi té.
\par 8 Rekl tedy Abram k Lotovi: Nechžt, prosím, není nesnáze mezi mnou a tebou, a mezi pastýri mými a pastýri tvými, ponevadž muži bratrí jsme.
\par 9 Zdaliž není pred tebou všecka zeme? Oddel se, prosím, ode mne. Pujdeš-li na levo, já na pravo se držeti budu; pakli pujdeš na pravo, na levo se držeti budu.
\par 10 Pozdvih tedy Lot ocí svých, spatril všecku rovinu vukol Jordánu, kteráž pred tím, než Hospodin zkazil Sodomu a Gomoru, všecka až k Ségor svlažována byla, jako zahrada Hospodinova, a jako zeme Egyptská.
\par 11 I zvolil sobe Lot všecku rovinu Jordánskou, a bral se k východu; a tak oddelili se jeden od druhého.
\par 12 Abram bydlil v zemi Kananejské, ale Lot prebýval v mestech té roviny, podav stanu až k Sodome.
\par 13 Lidé pak Sodomští byli zlí, a hríšníci pred Hospodinem velicí.
\par 14 I rekl Hospodin Abramovi, když se oddelil od neho Lot: Pozdvihni nyní ocí svých, a pohled z místa, na nemž jsi, na pulnoci a na poledne, i na východ a na západ.
\par 15 Nebo všecku zemi, kterouž vidíš, tobe dám a semeni tvému až na veky.
\par 16 A rozmnožím síme tvé jako prach zeme; nebo jestliže kdo bude moci scísti prach zeme, tedy i síme tvé secteno bude.
\par 17 Vstan, projdi tu zemi na dýl i na šír její; nebo tobe ji dám.
\par 18 Tedy Abram hnuv se s stanem, prišel a bydlil v rovinách Mamre, kteréž jsou pri Hebronu, kdežto vzdelal oltár Hospodinu.

\chapter{14}

\par 1 Stalo se pak ve dnech tech, že Amrafel král Sinearský, Arioch král Elasarský, Chedorlaomer král Elamitský, a Thádal král Goimský,
\par 2 Vyzdvihli válku proti Bérovi králi Sodomskému, a proti Bersovi králi Gomorskému, a Senábovi králi Adamatskému, a Semeberovi králi Seboimskému, a králi Bélamskému, to jest Ségorskému.
\par 3 Všickni tito sjeli se do údolí Siddim, to jest již more solné.
\par 4 Dvanácte let sloužili Chedorlaomerovi, trináctého pak léta zprotivili se.
\par 5 Protož léta ctrnáctého pritáhl Chedorlaomer a králové, kteríž byli s ním, a pobili Refaimské v Astarotu Karnaimských, a Zuzimské v Cham, a Eminské na rovinách Kariataimských,
\par 6 A Horejské na hore jich Seir, až k rovine Fáran, kteráž leží nad pouští.
\par 7 A vracejíce se, pritáhli k En Misfat, kteráž již jest Kádes, a pohubili všecku krajinu Amalechitského, také i Amorejského, bydlícího v Hasesontamar.
\par 8 Protož vytáhl král Sodomský, a král Gomorský, a král Adamatský, a král Seboimský, a král Bélamský, to jest Ségorský, a sšikovali se proti nim k bitve v údolí Siddim,
\par 9 Proti Chedorlaomerovi králi Elamitskému, a Thádalovi králi Goimskému, a Amrafelovi králi Sinearskému, i Ariochovi králi Elasarskému, ctyri králové proti peti.
\par 10 (V údolí pak Siddim bylo mnoho studnic klejovatých.) I utíkajíce král Sodomský a Gomorský, padli tam; a kterí pozustali, utekli na hory.
\par 11 A pobravše všecko zboží Sodomských a Gomorských, a všecky potravy jich, odtáhli.
\par 12 Vzali také Lota, a zboží jeho, syna bratra Abramova, a odjeli; nebo on bydlil v Sodome.
\par 13 Prišel pak jeden, kterýž byl utekl, a zvestoval Abramovi Hebrejskému, kterýž tehdáž bydlil v rovinách Mamre Amorejského, bratra Eškolova a bratra Anerova; nebo ti meli smlouvu s Abramem.
\par 14 Uslyšev tedy Abram, že by zajat byl bratr jeho, vypravil zpusobných k boji a v dome svém zrozených služebníku tri sta a osmnácte, a honil je až k Dan.
\par 15 A oddeliv se, pripadl na ne v noci, on i služebníci jeho, a porazil je; a stihal je až k Chobah, kteréž leží na levo Damašku.
\par 16 I odjal zase všecko zboží; také i Lota bratra svého s statkem jeho zase privedl, ano i ženy a lid.
\par 17 Tedy vyšel král Sodomský proti nemu, když se navracoval od pobití Chedorlaomera a králu, kteríž byli s ním, k údolí Sáveh, kteréž jest údolí královské.
\par 18 Melchisedech také král Sálem, vynesl chléb a víno; a ten byl knez Boha silného nejvyššího.
\par 19 I požehnal mu a rekl: Požehnaný Abram Bohu silnému nejvyššímu, kterýž vládne nebem a zemí;
\par 20 A požehnaný Buh silný nejvyšší, kterýž dal neprátely tvé v ruce tvé. I dal mu Abram desátky ze všech vecí.
\par 21 Král pak Sodomský rekl Abramovi: Dej mi lid, a zboží vezmi sobe.
\par 22 I rekl Abram králi Sodomskému: Pozdvihl jsem ruky své k Hospodinu, Bohu silnému nejvyššímu, kterýž vládne nebem i zemí,
\par 23 Že nevezmu od niti až do reménka obuvi ze všech vecí, kteréž jsou tvé, abys nerekl: Já jsem obohatil Abrama,
\par 24 Krome toliko toho, což snedli bojovníci, a dílu mužu, kteríž se mnou šli, totiž Aner, Eškol a Mamre; oni nechat vezmou díl svuj.

\chapter{15}

\par 1 Když pak ty veci pominuly, stalo se slovo Hospodinovo k Abramovi u videní, rkoucí: Neboj se, Abrame; já budu pavéza tvá, a odplata tvá velmi veliká.
\par 2 Jemužto rekl Abram: Panovníce Hospodine, což mi dáš, ponevadž já scházím bez detí, a ten, jemuž zanechám domu svého, bude Damašský Eliezer?
\par 3 Rekl ješte Abram: Aj, mne jsi nedal semene; a aj, schovanec muj bude mým dedicem.
\par 4 A aj, slovo Hospodinovo k nemu, rkuci: Nebudet ten dedicem tvým, ale kterýž vyjde z života tvého, ten dedicem tvým bude.
\par 5 I vyvedl jej ven a rekl: Vzhlédniž nyní k nebi, a secti hvezdy, budeš-li je však moci scísti? Rekl mu ješte: Tak bude síme tvé.
\par 6 I uveril Hospodinu, a pocteno mu to za spravedlnost.
\par 7 (Nebo byl rekl jemu: Já jsem Hospodin, kterýž jsem te vyvedl z Ur Kaldejských, atbych dal zemi tuto k dedicnému vládarství.
\par 8 I rekl: Panovníce Hospodine, po cem poznám, že ji dedicne obdržím?
\par 9 I odpovedel jemu: Vezmi mne jalovici tríletou, a kozu tríletou, a skopce tríletého, hrdlicku také a holoubátko.
\par 10 Kterýžto vzav ty všecky veci, zroztínal je na poly, a rozložil na dve strane, jednu polovici proti druhé; ptáku pak nezroztínal.
\par 11 Ptáci pak sedali na ta mrtvá tela, a Abram je shánel.
\par 12 I stalo se, když slunce zapadalo, že drímota težká pripadla na Abrama; a aj, hruza a tma veliká obklícila jej).
\par 13 Rekl tedy Buh Abramovi: To zajisté vez, že pohostinu bude síme tvé v zemi cizí, a v službu je podrobí, a trápiti je budou za ctyri sta let.
\par 14 Však národ, jemuž sloužiti budou, já souditi budu; a potom vyjdou s velikým zbožím.
\par 15 Ty pak pujdeš k otcum svým v pokoji; a pohrben budeš v starosti dobré.
\par 16 A ctvrté pokolení sem se navrátí; nebt ješte není doplnena nepravost Amorejských.
\par 17 I stalo se, když zapadlo slunce, a tma bylo, a aj, ukázala se pec kourící se, a pochodne ohnivá, kteráž šla mezi díly temi.
\par 18 V ten den ucinil Hospodin smlouvu s Abramem, rka: Semeni tvému dám zemi tuto, od reky Egyptské až do reky té veliké, reky Eufraten:
\par 19 Cinejské, Cenezejské, Cethmonské,
\par 20 A Hetejské, Ferezejské, a Refaimské,
\par 21 Amorejské, i Kananejské také, a Gergezejské a Jebuzejské.

\chapter{16}

\par 1 Sarai pak manželka Abramova jemu nerodila; a mela devku Egyptskou, jménem Agar.
\par 2 I rekla Sarai Abramovi: Aj, nyní Hospodin zavrel život muj, abych nerodila; vejdi, prosím, k devce mé, zda bych aspon z ní mohla míti syny. I povolil Abram reci Sarai.
\par 3 Tedy vzavši Sarai manželka Abramova Agar Egyptskou devku svou, po desíti letech, jakž bydliti pocal Abram v zemi Kananejské, dala ji Abramovi muži svému za ženu.
\par 4 I všel k Agar, kterážto pocala. Viduci pak ona, že pocala, zlehcila sobe paní svou.
\par 5 I rekla Sarai Abramovi: Krivdou mou tys vinen; já jsem dala devku svou v luno tvé, kterážto viduci, že pocala, zlehcila mne sobe. Sudiž Hospodin mezi mnou a mezi tebou.
\par 6 I rekl Abram k Sarai: Aj, devka tvá v moci tvé; ucin s ní, cožt se za dobré vidí. Tedy trápila ji Sarai, a ona utekla od ní.
\par 7 Našel ji pak andel Hospodinuv u studnice vody na poušti, u studnice té, kteráž jest pri ceste Sur.
\par 8 A rekl: Agar, devko Sarai, odkud jdeš, a kam se béreš? I rekla: Od tvári Sarai paní své já utíkám.
\par 9 Tedy rekl jí andel Hospodinuv: Navrat se ku paní své, a pokor se pod ruku její.
\par 10 Opet rekl andel Hospodinuv: Velice rozmnožím síme tvé, aniž bude moci secteno býti pro množství.
\par 11 Potom také rekl andel Hospodinuv: Aj, ty jsi tehotná, a tudíž porodíš syna, a nazuveš jméno jeho Izmael; nebo uslyšel Hospodin trápení tvé.
\par 12 Budet pak lítý clovek; ruce jeho proti všechnem, a ruce všech proti nemu; a pred tvárí všech bratrí svých bydliti bude.
\par 13 I nazvala Agar jméno Hospodinovo, kterýž mluvil jí: Ty jsi silný Buh videní; nebo rekla: Zdaliž ted také nevidím po tom, kterýž mne videl?
\par 14 Protož nazvala studnici tu studnicí Živého vidoucího mne. Aj, ta jest mezi Kádes a Barad.
\par 15 Porodila pak Agar Abramovi syna; a nazval Abram jméno syna svého, kteréhož porodila Agar, Izmael.
\par 16 Abram pak byl v osmdesáti šesti letech, když mu porodila Agar Izmaele.

\chapter{17}

\par 1 Když pak Abram byl v devadesáti devíti letech, ukázal se mu Hospodin, a rekl jemu: Já jsem Buh silný všemohoucí; chodiž ustavicne prede mnou a budiž dokonalým.
\par 2 A uciním smlouvu svou mezi sebou a tebou a rozmnožím te náramne velmi.
\par 3 Padl pak Abram na tvár svou; i mluvil Buh s ním, rka:
\par 4 Ját jsem, aj, smlouva má s tebou, a budeš otcem národu mnohých.
\par 5 Aniž více slouti bude jméno tvé Abram, ale bude jméno tvé Abraham; nebo otcem mnohých národu ucinil jsem te.
\par 6 A uciním, abys se rozplodil náramne velmi, a rozšírím te v národy; i králové z tebe vyjdou.
\par 7 Utvrdím také smlouvu svou mezi sebou a tebou, i mezi semenem tvým po tobe, po rodech jejich, za smlouvu vecnou, totiž abych byl Bohem tvým i semene tvého po tobe.
\par 8 Nadto dám tobe i semeni tvému po tobe zemi, v nížto obýváš pohostinu, všecku zemi Kananejskou k vládarství vecnému; a budu jejich Bohem.
\par 9 Rekl ješte Buh Abrahamovi: Ty pak ostríhati budeš smlouvy mé, ty i síme tvé po tobe, po rodech svých.
\par 10 Tatot jest smlouva má mezi mnou a mezi vámi, i mezi semenem tvým po tobe, kteréž ostríhati budete: Aby obrezán byl mezi vámi každý pohlaví mužského.
\par 11 Obrežete pak telo hanby své; a to bude znamením smlouvy mezi mnou a mezi vámi.
\par 12 Každý tedy pohlaví mužského osmého dne obrezán bude mezi vámi po rodech vašich, doma narozený i koupený za stríbro, z kterých by koli cizozemcu byl, jenž není z semene tvého.
\par 13 Konecne at jest obrezán narozený v dome tvém, i koupený za peníze tvé; a budet smlouva má na tele vašem za smlouvu vecnou.
\par 14 Neobrezaný pak pacholík, kterýž by neobrezal tela neobrízky své, vyhlazena zajisté bude duše ta z lidu svého; nebo smlouvu mou zrušil.
\par 15 Rekl také Buh Abrahamovi: Sarai manželce své nebudeš ríkati Sarai, ale Sára bude jméno její.
\par 16 Nebo požehnám jí a dámt z ní syna; požehnámt jí, a bude v národy; králové národu z ní vyjdou.
\par 17 Tedy padl Abraham na tvár svou, a zasmáv se, rekl v srdci svém: Zdali stoletému narodí se syn? A zdali Sára v devadesáti letech porodí?
\par 18 I rekl Abraham Bohu: Ó byt jen Izmael živ byl pred tebou!
\par 19 Jemužto rekl Buh: Nýbrž Sára manželka tvá porodí tobe syna, a nazuveš jméno jeho Izák; i utvrdím smlouvu svou s ním za smlouvu vecnou, i s semenem jeho po nem.
\par 20 Také o Izmaele uslyšel jsem te; a aj, požehnám jemu, a uciním to, aby se rozplodil, a rozmnožím ho náramne velmi; dvanáctero knížat zplodí, a rozšírím jej v národ veliký.
\par 21 Ale smlouvu svou utvrdím s Izákem, kteréhožt porodí Sára po roce, pri tomto casu.
\par 22 A když dokonal Buh rec svou s ním, vstoupil od Abrahama.
\par 23 Vzal tedy Abraham Izmaele syna svého, i všecky zrozené v dome svém, i všecky koupené za stríbro své, každého, kdož byl pohlaví mužského, z domácích svých, a obrezal telo neobrízky jejich hned v ten den, jakž s ním Buh mluvil.
\par 24 A byl Abraham v devadesáti devíti letech, když obrezáno bylo telo neobrízky jeho.
\par 25 Izmael pak syn jeho byl v trinácti letech, když obrezáno bylo telo neobrízky jeho.
\par 26 Jednoho a téhož dne obrezáni jsou, Abraham a Izmael syn jeho.
\par 27 I všickni domácí jeho, doma zrození i za stríbro od cizozemce koupení, obrezáni jsou s ním.

\chapter{18}

\par 1 Ukázal se pak jemu Hospodin v rovine Mamre; a on sedel u dverí stanu, když veliké horko na den bylo.
\par 2 A když pozdvihl ocí svých, videl, a aj, tri muži stáli naproti nemu. Kteréžto jakž uzrel, bežel jim vstríc ode dverí stanu, a sklonil se až k zemi.
\par 3 A rekl: Pane muj, jestliže jsem nyní nalezl milost pred ocima tvýma, prosím, nepomíjej služebníka svého.
\par 4 Prineseno bude trochu vody, a umyjete nohy své, a odpocinete pod stromem.
\par 5 Zatím prinesu kus chleba, a posilníte srdce svého; potom pujdete, ponevadž mimo služebníka svého jdete. I rekli: Tak ucin, jakž jsi mluvil.
\par 6 Tedy pospíšil Abraham do stanu k Sáre, a rekl: Spešne tri míry mouky belné zadelej, a napec podpopelných chlebu.
\par 7 Abraham pak bežel k stádu; a vzav tele mladé a dobré, dal služebníku, kterýžto pospíšil pripraviti je.
\par 8 Potom vzav másla a mléka, i tele, kteréž pripravil, položil pred ne; sám pak stál pri nich pod stromem, i jedli.
\par 9 Rekli pak jemu: Kde jest Sára manželka tvá? Kterýžto odpovedel: Ted v stanu.
\par 10 A rekl: Jistotne se navrátím k tobe vedlé casu života, a aj, syna míti bude Sára manželka tvá. Ale Sára poslouchala u dverí stanu, kteréž byly za ním.
\par 11 Abraham pak i Sára byli starí a sešlého veku, a prestal byl Sáre beh ženský.
\par 12 I smála se Sára sama v sobe, rkuci: Teprv když jsem se sstarala, v rozkoše se vydám? A ješte i pán muj se sstaral.
\par 13 Tedy rekl Hospodin Abrahamovi: Proc jest se smála Sára, rkuci: Zdaliž opravdu ješte roditi budu, a já se sstarala?
\par 14 Zdaliž co skrytého bude pred Hospodinem? K casu urcitému navrátím se k tobe vedlé casu života, a Sára bude míti syna.
\par 15 Zaprela pak Sára a rekla: Nesmála jsem se; nebo se bála. I rekl Hospodin: Nenít tak, ale smála jsi se.
\par 16 Tedy vstavše odtud muži ti, obrátili se k Sodome; Abraham pak šel s nimi, aby je provodil.
\par 17 A rekl Hospodin: Zdali já zatajím pred Abrahamem, což delati budu?
\par 18 Ponevadž Abraham jistotne bude v národ veliký a silný, a požehnáni budou v nem všickni národové zeme.
\par 19 Nebo znám jej; protož prikáže synum svým a domu svému po sobe, aby ostríhali cesty Hospodinovy, a cinili spravedlnost a soud, atby naplnil Hospodin Abrahamovi, což mu zaslíbil.
\par 20 I rekl Hospodin: Proto že rozmnožen jest krik Sodomských a Gomorských, a hrích jejich že obtížen jest náramne:
\par 21 Sstoupím již a pohledím, jestliže podle kriku jejich, kterýž prišel ke mne, cinili, dujde na ne setrení; a pakli toho není, zvím.
\par 22 A obrátivše se odtud muži, šli do Sodomy; Abraham pak ješte stál pred Hospodinem.
\par 23 V tom pristoupiv Abraham, rekl: Zdali také zahladíš spravedlivého s bezbožným?
\par 24 Bude-li padesáte spravedlivých v tom meste, zdali predce zahubíš, a neodpustíš místu pro padesáte spravedlivých, kteríž jsou v nem?
\par 25 Odstup to od tebe, abys takovou vec uciniti mel, abys usmrtil spravedlivého s bezbožným; takt by byl spravedlivý jako bezbožný. Odstup to od tebe; zdaliž soudce vší zeme neuciní soudu?
\par 26 I rekl Hospodin: Jestliže naleznu v Sodome, v meste tom, padesáte spravedlivých, odpustím všemu tomu místu pro ne.
\par 27 A odpovídaje Abraham, rekl: Aj, nyní chtel bych mluviti ku Pánu svému, ackoli jsem prach a popel.
\par 28 Co pak, nedostane-li se ku padesáti spravedlivým peti, zdali zkazíš pro tech pet všecko mesto? I rekl: Nezahladím, jestliže najdu tam ctyridceti pet.
\par 29 Opet mluvil Abraham a rekl: Snad nalezeno bude tam ctyridceti? A odpovedel: Neuciním pro tech ctyridceti.
\par 30 I rekl Abraham: Prosím, necht se nehnevá Pán muj, že mluviti budu: Snad se jich nalezne tam tridceti? Odpovedel: Neuciním, jestliže naleznu tam tridceti.
\par 31 A opet rekl: Aj, nyní pocal jsem mluviti ku Pánu svému: Snad se nalezne tam dvadceti? Odpovedel: Nezahladím i pro tech dvadceti.
\par 32 Rekl ješte: Prosím, at se nehnevá Pán muj, jestliže jednou ješte mluviti budu: Snad se jich najde tam deset? Odpovedel: Nezahladím i pro tech deset.
\par 33 I odšel Hospodin, když dokonal rec k Abrahamovi; Abraham pak navrátil se k místu svému.

\chapter{19}

\par 1 Prišli pak dva andelé do Sodomy u vecer, a Lot sedel v bráne Sodomské. Kteréžto když uzrel, vstav, šel jim v cestu a sklonil se tvárí až k zemi.
\par 2 A rekl: Aj, prosím páni moji, uchylte se nyní do domu služebníka svého, a zustante pres noc; umyjete také nohy své a ráno vstanouce, pujdete cestou svou. Oni pak odpovedeli: Nikoli, ale prenocujeme na ulici.
\par 3 Ale když on je velmi nutil, obrátivše se k nemu, vešli do domu jeho. I udelal jim hody, a napekl chlebu presných, i jedli.
\par 4 Prvé pak než lehli, muži mesta toho, muži Sodomští, osuli se vukol domu toho, od mladého až do starého, všecken lid odevšad.
\par 5 I volali na Lota, a rekli jemu: Kde jsou ti muži, kteríž prišli k tobe v noci? Vyved je k nám, at je poznáme.
\par 6 I vyšel k nim Lot ven, a zavrel po sobe dvére.
\par 7 A rekl: Prosím, bratrí moji, necinte zlého.
\par 8 Aj, mám ted dve dcery, kteréžto nepoznaly muže; vyvedu je nyní k vám, cinte s nimi, jak se vám líbí; toliko mužum temto nic necinte, ponevadž vešli pod stín strechy mé.
\par 9 I rekli: Odejdi tam! A mluvili: Sám se dostal sem pohostinu, a chce nás souditi? Nyní tobe hur udeláme, než jim. I oborili se násilne na muže toho, totiž na Lota, a pristoupili, aby vylomili dvére.
\par 10 Tedy muži ti vztáhli ven ruku svou, a uvedli Lota k sobe do domu, a dvére zavreli.
\par 11 A ty muže, kteríž byli prede dvermi domu, ranili slepotou velikou, od nejmenšího až do nejvetšího, tak že ustali, hledajíce dverí.
\par 12 I rekli muži k Lotovi: Máš-li ješte zde koho, bud zete neb syny své, neb dcery své, i všecko, což máš v meste, vyved z místa tohoto.
\par 13 Nebo zkazíme místo toto, proto že se velmi rozmohl krik jejich pred Hospodinem, a poslal nás Hospodin, abychom zkazili je.
\par 14 Vyšed tedy Lot, mluvil k zetum svým, kteríž již meli pojímati dcery jeho, a rekl: Vstante, vyjdete z místa tohoto, nebo zkazí Hospodin mesto toto. Ale zdálo se zetum jeho, jako by žertoval.
\par 15 A když zasvitávalo, nutili andelé Lota, rkouce: Vstan, vezmi ženu svou a dve dcery své, kteréž tu jsou, abys nezahynul v pomste mesta.
\par 16 A když prodléval, chopili muži ruku jeho, a ruku ženy jeho, a ruku dvou dcer jeho, nebo se slitoval nad ním Hospodin; i vyvedli jej, a pustili za mestem.
\par 17 A stalo se, když je vedli ven, rekl jeden: Zachovejž život svuj, neohlédej se zpet, ani se zastavuj na vší této rovine; ujdi na horu, abys nezahynul.
\par 18 I rekl jim Lot: Ne tak, prosím, páni moji.
\par 19 Aj, nyní nalezl služebník tvuj milost pred ocima tvýma, a veliké jest milosrdenství tvé, kteréž jsi ucinil se mnou, když jsi zachoval duši mou; ale ját nebudu moci ujíti na tu horu, aby mne nepostihlo to zlé, a umrel bych.
\par 20 Hle, ted jest toto mesto blízko, do nehož bych utekl, a tot jest malé; prosím, necht tam ujdu; však pak neveliké jest, a živa bude duše má.
\par 21 I rekl k nemu: Aj, uslyšel jsem žádost tvou i v této veci, abych nepodvrátil mesta toho, o nemž jsi mluvil.
\par 22 Pospešiž, ujdi tam; nebot nebudu moci uciniti nicehož, dokudž tam nedojdeš. A z té príciny nazváno jest jméno mesta toho Ségor.
\par 23 Slunce vzcházelo nad zemi, když Lot všel do Ségor.
\par 24 A Hospodin dštil na Sodomu a naGomoru sirou a ohnem od Hospodina s nebe.
\par 25 A podvrátil ta mesta i všecku tu rovinu, všecky také obyvatele tech mest, i všecko, což roste z zeme.
\par 26 I ohlédla se žena jeho, jduc za ním, a obrácena jest v sloup solný.
\par 27 Vstav pak Abraham ráno, pospíšil k místu tomu, kdež byl stál pred Hospodinem.
\par 28 A pohledev k Sodome a Gomore,i na všecku zemi té roviny, uzrel, a aj, vystupoval dým z zeme té, jako dým z vápenice.
\par 29 Stalo se tedy, když kazil Buh mesta té roviny, že se rozpomenul Buh na Abrahama, a vytrhl Lota z prostredku podvrácení, když podvracel mesta, v nichž bydlil Lot.
\par 30 Potom vyšel Lot z Ségor, a bydlil na hore té, a obe dve dcery jeho s ním; nebo nesmel bydliti v Ségor. I bydlil v jeskyni s obema dcerami svými.
\par 31 I rekla prvorozená k mladší: Otec náš jest již starý, a není žádného muže na zemi, ješto by všel k nám podlé obyceje vší zeme.
\par 32 Pod, dejme píti otci našemu vína, a speme s ním, abychom zachovaly z otce našeho síme.
\par 33 I daly píti otci svému vína té noci; a všedši prvorozená, spala s otcem svým, kterýžto necítil, ani když lehla, ani když vstala.
\par 34 Nazejtrí pak rekla prvorozená k mladší: Aj, spala jsem vcerejší noci s otcem svým; dejme mu píti vína ješte této noci; potom vejduc, spi s ním, a zachovejme síme z otce našeho.
\par 35 I daly píti ješte té noci otci svému vína; a vstala ta mladší, a spala s ním; on pak necítil, ani když ona lehla, ani když vstala.
\par 36 A tak pocaly obe dcery Lotovy z otce svého.
\par 37 I porodila prvorozená syna, a nazvala jméno jeho Moáb; ont jest otec Moábských až do dnešního dne.
\par 38 I mladší také porodila syna, a nazvala jméno jeho Ben Ammon; ont jest otcem Ammonitských až do dnešního dne.

\chapter{20}

\par 1 Odtud bral se Abraham do zeme polední, aby bydlil mezi Kádes a Sur; i byl pohostinu v Gerar;
\par 2 Kdežto pravil Abraham o Sáre manželce své: Sestra má jest. Tedy poslav Abimelech, král Gerarský, vzal Sáru.
\par 3 Ale prišed Buh k Abimelechovi ve snách v noci, rekl jemu: Aj, ty již umreš pro ženu, kterouž jsi vzal, ponevadž jest vdaná za muže.
\par 4 Abimelech pak nepriblížil se k ní; protož rekl: Pane, zdaž také spravedlivý národ zabiješ?
\par 5 Zdaliž mi sám nepravil: Sestra má jest? A ona též pravila: Bratr muj jest. V uprímnosti srdce svého a v nevinnosti rukou svých ucinil jsem to.
\par 6 I rekl jemu Buh ve snách: Ját také vím, že v uprímnosti srdce svého ucinil jsi to, a já také zdržel jsem te, abys nezhrešil proti mne; protož nedalt jsem se jí dotknouti.
\par 7 Nyní tedy, navrat ženu muži tomu; nebo prorok jest, a modliti se bude za tebe, a živ budeš. Pakli jí nenavrátíš, vez, že smrtí umreš ty i všecko, což tvého jest.
\par 8 A vstav Abimelech ráno, svolal všecky služebníky své, a vypravoval všecka slova ta v uši jejich. I báli se ti muži velmi.
\par 9 Potom povolav Abimelech Abrahama, rekl jemu: Co jsi nám to ucinil? A co jsem zhrešil proti tobe, že jsi uvedl na mne a na království mé hrích veliký? Ucinils mi, cehož jsi uciniti nemel.
\par 10 A rekl opet Abimelech Abrahamovi: Cos myslil, žes takovou vec ucinil?
\par 11 Odpovedel Abraham: Rekl jsem: Jiste že není bázne Boží na míste tomto, a zabijí mne pro ženu mou.
\par 12 A také v pravde jest sestra má, dcera otce mého, však ne dcera matky mé; a pojal jsem ji sobe za manželku.
\par 13 Když pak vyvedl mne Buh z domu otce mého, abych pohostinu bydlil, tedy rekl jsem jí: Toto mi dobrodiní uciníš: Na každém míste, kamž pujdeme, prav o mne: Bratr muj jest.
\par 14 Tedy vzav Abimelech ovce a voly, služebníky také a devky, dal je Abrahamovi; a navrátil mu Sáru manželku jeho.
\par 15 A rekl Abimelech: Aj, zeme má pred tebou; kdežt se koli príhodné býti vidí, tu prebývej.
\par 16 Sáre pak rekl: Aj, dal jsem tisíc stríbrných bratru tvému, hle, ont jest tobe zásterou ocí u všech, kteríž jsou s tebou. A všemi temito vecmi Sára poucena byla.
\par 17 I modlil se Abraham Bohu, a uzdravil Buh Abimelecha, a ženu jeho, a devky jeho; i rodily.
\par 18 Nebo byl zavrel Hospodin každý život ženský v dome Abimelechove, pro Sáru manželku Abrahamovu.

\chapter{21}

\par 1 Navštívil pak Hospodin Sáru, tak jakž byl rekl; a ucinil Hospodin Sáre, jakož byl mluvil.
\par 2 Nebo pocala a porodila Sára Abrahamovi syna v starosti jeho, v ten cas, kterýž predpovedel Buh.
\par 3 A nazval Abraham jméno syna svého, kterýž se mu narodil, jehož porodila Sára, Izák.
\par 4 A obrezal Abraham syna svého Izáka, když byl v osmi dnech, tak jakž mu byl prikázal Buh.
\par 5 Byl pak Abraham ve stu letech, když se mu narodil Izák syn jeho.
\par 6 I rekla Sára: Radost mi ucinil Buh; kdokoli uslyší, radovati se bude spolu se mnou.
\par 7 A pridala: Kdo by byl rekl Abrahamovi, že bude Sára deti kojiti? A však jsem porodila syna v starosti jeho.
\par 8 I rostlo díte a ostaveno jest. Tedy ucinil Abraham veliké hody v ten den, v nemž ostaven byl Izák.
\par 9 Videla pak Sára, že syn Agar Egyptské, kteréhož porodila Abrahamovi, jest posmevac.
\par 10 I rekla Abrahamovi: Vyvrz devku tuto i syna jejího; nebot nebude dedicem syn devky té s synem mým Izákem.
\par 11 Ale Abraham velmi težce nesl tu rec, pro syna svého.
\par 12 I rekl Buh Abrahamovi: Nestežuj sobe o díteti a o devce své; cožkoli rekla tobe Sára, povol reci její, nebo v Izákovi nazváno bude tobe síme.
\par 13 A však i syna devky uciním v národ; nebo tvé síme jest.
\par 14 Vstal tedy Abraham velmi ráno, a vzav chléb a láhvici vody, dal Agar a vložil na rameno její, a pustil ji od sebe i s dítetem. Kterážto odešla a chodila po poušti Bersabé.
\par 15 A když nebylo vody v láhvici, povrhla díte pod jedním stromem.
\par 16 A odšedši, sedla naproti tak daleko, jako by mohl z lucište dostreliti; nebo pravila: Nebudu se dívati na smrt dítete. Sedela tedy naproti, a pozdvihši hlasu svého, plakala.
\par 17 I uslyšel Buh hlas dítete; a andel Boží s nebe zavolal na Agar, a rekl jí: Cot jest, Agar? Neboj se; nebo Buh uslyšel hlas dítete z místa, na kterémž jest.
\par 18 Vstan, vezmi díte, a ujmi je rukou svou; nebo v národ veliký uciním je.
\par 19 A otevrel Buh oci její, aby uzrela studnici vody. I šla a naplnila láhvici vodou, a napojila díte.
\par 20 A Buh byl s dítetem, kteréžto zrostlo a bydlilo na poušti, a byl z neho strelec.
\par 21 Bydlil pak na poušti Fáran; i vzala mu matka jeho ženu z zeme Egyptské.
\par 22 Stalo se pak toho casu, že mluvil Abimelech a Fikol, kníže vojska jeho, k Abrahamovi temito slovy: Buh s tebou jest ve všech vecech, kteréž ty ciníš.
\par 23 Protož nyní, prisáhni mi ted skrze Boha: Toto at se stane, jestliže mi sklamáš, neb synu mému, aneb vnuku mému; vedlé milosrdenství, kteréž jsem já ucinil s tebou, i ty že uciníš se mnou a s zemí, v níž jsi byl pohostinu.
\par 24 I rekl Abraham: A já prisáhnu.
\par 25 (A pritom domlouval se Abraham na Abimelecha o studnici vody, kterouž mu mocí odjali služebníci Abimelechovi.
\par 26 I rekl Abimelech: Nevím, kdo by ucinil takovou vec; a aniž jsi ty mi oznámil, aniž jsem já také co slyšel, až dnes.)
\par 27 Vzav tedy Abraham ovce i voly dal Abimelechovi; a vešli oba dva v smlouvu.
\par 28 A postavil Abraham sedm jehnic stáda obzvlášt.
\par 29 I rekl Abimelech Abrahamovi: K cemu jest techto sedm jehnic, kteréž jsi postavil obzvlášt?
\par 30 Odpovedel: Že sedm tech jehnic vezmeš z ruky mé, aby mi to bylo na svedectví, že jsem kopal studnici tuto.
\par 31 Procež nazváno jest to místo Bersabé, že tu oba dva prisáhli.
\par 32 A tak ucinili smlouvu v Bersabé. Vstav pak Abimelech a Fikol, kníže vojska jeho, navrátili se do zeme Filistinské.
\par 33 I nasázel stromoví v Bersabé, a vzýval tam jméno Hospodina, Boha silného, vecného.
\par 34 A bydlil Abraham v zemi Filistinské za mnoho dní.

\chapter{22}

\par 1 Když pakty veci pominuly, zkusil Buh Abrahama, a rekl k nemu: Abrahame! Kterýžto odpovedel: Ted jsem.
\par 2 I rekl: Vezmi nyní syna svého, toho jediného svého, kteréhož miluješ, Izáka, a jdi do zeme Moria; a obetuj ho tam v obet zápalnou na jedné hore, o níž povím tobe.
\par 3 Tedy vstav Abraham velmi ráno, osedlal osla svého a vzal dva služebníky své s sebou, a Izáka syna svého; a nasekav dríví k obeti zápalné, vstal a bral se k místu, o nemž povedel mu Buh.
\par 4 Tretího pak dne pozdvihl Abraham ocí svých, a uzrel to místo zdaleka.
\par 5 A rekl Abraham služebníkum svým: Pozustante vy tuto s oslem, já pak a díte pujdeme tamto; a pomodlíce se, navrátíme se k vám.
\par 6 Tedy vzal Abraham dríví k zápalné obeti, a vložil je na Izáka syna svého; sám pak nesl v ruce své ohen a mec. I šli oba spolu.
\par 7 Mluve pak Izák Abrahamovi otci svému, rekl: Otce muj! Kterýž odpovedel: Co chceš, synu muj? A rekl: Aj, ted ohen a dríví, a kdež hovádko k zápalné obeti?
\par 8 Odpovedel Abraham: Buh opatrí sobe hovádko k obeti zápalné,synu muj. A šli predce oba spolu.
\par 9 A když prišli k místu, o nemž mu byl mluvil Buh, udelal tu Abraham oltár, a srovnal dríví; a svázav syna svého, vložil ho na oltár na dríví.
\par 10 I vztáhl Abraham ruku svou, a vzal mec, aby zabil syna svého.
\par 11 Tedy zavolal na neho andel Hospodinuv s nebe a rekl: Abrahame, Abrahame! Kterýžto odpovedel: Aj, já.
\par 12 I rekl jemu: Nevztahuj ruky své na díte, aniž mu co cin; nebot jsem již poznal, že se Boha bojíš, když jsi neodpustil synu svému, jedinému svému pro mne.
\par 13 A pozdvih Abraham ocí svých, videl, a hle, skopec za ním vezel v trní za rohy své. I šel Abraham a vzal skopce toho, a obetoval jej v obet zápalnou místo syna svého.
\par 14 A nazval Abraham jméno místa toho: Hospodin opatrí. Odkudž ríká se do dnes: Na hore Hospodinove opatrí se.
\par 15 Zvolal pak andel Hospodinuv na Abrahama podruhé s nebe,
\par 16 A rekl: Skrze sebe samého prisáhl jsem, praví Hospodin, ponevadž jsi ucinil tu vec, že jsi neodpustil synu svému, jedinému svému:
\par 17 Požehnám velmi tobe, a velice rozmnožím síme tvé jako hvezdy nebeské, a jako písek, kterýž jest na brehu morském; nadto dedicne vládnouti bude síme tvé branami neprátel svých.
\par 18 Ano požehnáni budou v semeni tvém všickni národové zeme, proto že jsi uposlechl hlasu mého.
\par 19 Tedy navrátil se Abraham k služebníkum svým; a vstavše, šli spolu do Bersabé; nebo bydlil Abraham v Bersabé.
\par 20 A když se tyto veci staly, zvestováno jest Abrahamovi v tato slova: Aj, porodila také Melcha syny Náchorovi, bratru tvému:
\par 21 Husa prvorozeného svého, a Buza bratra jeho, a Chamuele, otce Aramova;
\par 22 A Kazeda a Azana, a Feldasa, a Jidlafa i Bathuele.
\par 23 Bathuel pak zplodil Rebeku. Osm techto porodila Melcha Náchorovi, bratru Abrahamovu.
\par 24 Ale i ženina jeho, jejíž jméno bylo Réma, porodila také ona Tábe a Gahama, Thása a Máchu.

\chapter{23}

\par 1 Živa pak byla Sára sto a sedmmecítma let; ta jsou léta života Sáry.
\par 2 A umrela v meste Arbe, kteréž slove Hebron, v zemi Kananejské. I prišel Abraham, aby kvílil nad Sárou, a plakal jí.
\par 3 Potom vstav Abraham od mrtvého svého, mluvil k synum Het, rka:
\par 4 Hostem a príchozím jsem u vás; dejte mi místo ku pohrbu u vás, abych pochoval mrtvého svého od tvári své.
\par 5 A odpovídajíce synové Het Abrahamovi, rekli jemu:
\par 6 Slyš nás, pane milý! Kníže Boží jsi u prostred nás, v nejprednejších hrobích našich pochovej mrtvého svého; žádný z nás hrobu svého nebude zbranovati tobe, abys nemel pochovati v nem mrtvého svého.
\par 7 Abraham pak vstav, poklonil se lidu zeme té, totiž synum Het.
\par 8 A mluvil s nimi, rka: Jestliže se vám líbí, abych pochoval mrtvého svého od tvári své, slyšte mne, a primluvte se za mne k Efronovi synu Sohar,
\par 9 At mi dá jeskyni Machpelah, kterouž má na konci pole svého; za slušné peníze at mi ji dá u prostred vás, k dedicnému pohrbu.
\par 10 (Ten pak Efron sedel u prostred synu Het.) Tedy odpovedel Efron Hetejský Abrahamovi pri prítomnosti synu Het, prede všemi, kteríž vcházejí do brány mesta jeho, rka:
\par 11 Nikoli, pane muj, ale slyš mne: Pole to dávám tobe, dávámt také i jeskyni, kteráž na nem jest; pred ocima synu lidu svého dávám ji tobe; pochovejž mrtvého svého.
\par 12 Tedy poklonil se Abraham pred lidem zeme té,
\par 13 A mluvil k Efronovi v prítomnosti lidu zeme, rka: A však jestliže ty jsi ten, prosím, vyslyš mne! Dámt stríbro za pole, vezmi je ode mne, a pochovám mrtvého svého tam.
\par 14 A odpovídaje Efron Abrahamovi, rekl jemu:
\par 15 Muj pane, poslechni mne: Zeme ta za ctyri sta lotu stríbrných jest; ale mezi mnou a tebou co jest o to? Mrtvého svého pochovej.
\par 16 I uposlechl Abraham Efrona, a odvážil mu stríbra, jakž oznámil pri prítomnosti synu Het, ctyri sta lotu stríbrných, bežných mezi kupci.
\par 17 A odvedeno jest pole Efronovo, kteréž bylo v Machpelah, proti Mamre, pole a jeskyne na nem, a všecko stromoví, což ho na tom poli i na všech mezech jeho vukol,
\par 18 Abrahamovi v držení, pred ocima synu Het, a všech, kteríž vcházejí do brány mesta toho.
\par 19 A potom pochoval Abraham Sáru,manželku svou, v jeskyni pole Machpelah, proti Mamre, (to slove Hebron), v zemi Kananejské.
\par 20 Protož odevzdáno jest pole to i jeskyne, kteráž byla na nem, Abrahamovi k dedicnému pohrbu od synu Het.

\chapter{24}

\par 1 Abraham pak již byl starý a sešlého veku; a Hospodin požehnal mu ve všech vecech.
\par 2 Tedy rekl Abraham služebníku svému staršímu v dome svém, kterýž spravoval všecky veci jeho: Vlož nyní ruku svou pod bedro mé,
\par 3 Abych te zavázal prísahou skrze Hospodina, Boha nebe a Boha zeme, abys nebral manželky synu mému ze dcer Kananejských, mezi nimiž já bydlím.
\par 4 Ale pujdeš do zeme mé a príbuznosti mé; a odtud vezmeš manželku synu mému Izákovi.
\par 5 I rekl jemu služebník ten: Kdyby pak nechtela žena ta se mnou jíti do zeme této, mám-li zase uvésti syna tvého do zeme, z níž jsi vyšel?
\par 6 Kterýžto odpovedel: Hled, abys zase neuvodil syna mého tam.
\par 7 Hospodin Buh nebe, kterýž vzal mne z domu otce mého a z vlasti mé, a mluvil mi, a kterýž mi prisáhl, rka: Semeni tvému dám zemi tuto, ont pošle andela svého pred tebou, a vezmeš manželku synu mému odtud.
\par 8 Jestliže by pak žena ta nechtela jíti s tebou, svoboden budeš od této prísahy mé; toliko syna mého neuvod tam zase.
\par 9 Tedy vložil služebník ten ruku svou pod bedro Abrahama, pána svého, a prisáhl jemu na tu rec.
\par 10 A vzal služebník ten deset velbloudu z velbloudu pána svého, aby se bral; (nebo mel všecken statek pána svého v rukou svých.) I vstav, bral se k Aram Naharaim do mesta Náchor.
\par 11 A zastavil se s velbloudy pred mestem, u studnice vody k vecerou, casu toho, když vycházejí vážiti vody.
\par 12 A rekl: Hospodine, Bože pána mého Abrahama, dejž, prosím, at se potká se mnou dnes to, cehož hledám; a ucin milosrdenství se pánem mým Abrahamem.
\par 13 Aj, já stojím u studnice vody, a dcery mužu mesta tohoto vycházejí, aby vážily vodu.
\par 14 Devecka tedy, kteréž bych rekl: Nachyl medle vederce svého, at se napiji, a ona by rekla: Pí, také i velbloudy tvé napojím, ta aby byla, kterouž jsi zpusobil služebníku svému Izákovi; a po tomt poznám, že jsi milosrdenství ucinil se pánem mým.
\par 15 I stalo se prvé, než on prestal mluviti, aj, Rebeka, kteráž se narodila Bathuelovi, synu Melchy, manželky Náchora bratra Abrahamova, vycházela, a vederce její bylo na rameni jejím.
\par 16 A byla devecka na pohledení velmi krásná, panna, a muž nepoznal jí; kterážto sešla k studnici té, a naplnivši vederce své, šla vzhuru.
\par 17 Tedy bežel služebník ten proti ní, a rekl: Dej mi píti, prosím, malicko vody z vederce svého.
\par 18 I rekla: Napí se, pane muj. A rychle složila vederce své na ruku svou a dala mu píti.
\par 19 A davši mu píti, rekla: Také velbloudum tvým navážím, dokudž se nenapijí.
\par 20 Tedy rychle vylila z vederce svého do koryta, a bežela ješte k studnici, aby vážila vodu; a vážila všechnem velbloudum jeho.
\par 21 Muž pak ten s užasnutím divil se jí mlce, rozjímaje, aby zvedel, zdaril-li Hospodin cestu jeho, cili nic.
\par 22 I stalo se, když prestali píti velbloudové, vynav muž náušnici zlatou, ztíží pul lotu, a dve náramnice, dal na ruce její, kteréž vážily deset lotu zlata.
\par 23 A rekl: Cí jsi dcera? Povez mi, prosím, jest-li v dome otce tvého nám místo, kdež bychom pres noc zustali?
\par 24 Jemužto odpovedela: Dcera jsem Bathuele, syna Melchy, kteréhož porodila Náchorovi.
\par 25 Rekla ješte k nemu: Slámy také a potravy hojne u nás jest, ano i místo, kdež byste pres noc zustali.
\par 26 Tedy skloniv hlavu, poklonu ucinil Hospodinu.
\par 27 A rekl: Požehnaný Hospodin, Buh pána mého Abrahama, kterýž neodjal milosrdenství svého a pravdy své od pána mého, mne také na ceste vedl Hospodin k domu bratrí pána mého.
\par 28 I bežela devecka, a oznámila v dome matky své tak, jakž se stalo.
\par 29 A mela Rebeka bratra, jménem Lábana; ten bežel k muži tomu ven k studnici.
\par 30 Nebo uzrev náušnici a náramnice na rukou sestry své, a slyšev slova Rebeky sestry své, ana praví: Tak mluvil ke mne clovek ten, prišel k muži, a on stál pri velbloudích u studnice.
\par 31 Jemuž rekl: Vejdi, požehnaný Hospodinuv. Proc bys stál vne, kdyžt jsem pripravil dum a místo velbloudum?
\par 32 Tedy všel muž ten do domu, a odsedlal velbloudy, a dal Lában slámy a potravy velbloudum, i vody k umytí noh jeho a noh mužu tech, kterí s ním byli.
\par 33 A položil preden chléb, aby jedl. Ale on rekl: Nebudu jísti, dokudž nevypravím recí svých. I rekl Lában: Mluv.
\par 34 Tedy rekl: Služebník Abrahamuv jsem já.
\par 35 A Hospodin požehnal pánu mému velice, tak že veliký ucinen jest; nebo dal mu ovce a voly, stríbro a zlato, služebníky a devky, velbloudy a osly.
\par 36 A porodila Sára, manželka pána mého, pánu mému v starosti své syna, jemuž dal, cožkoli má.
\par 37 I zavázal mne prísahou pán muj, rka: Nevezmeš manželky synu mému ze dcer Kananejských, v jejichž zemi já bydlím.
\par 38 Ale k domu otce mého pujdeš, a k rodine mé, a odtud vezmeš manželku synu mému.
\par 39 A když jsem rekl pánu svému: Snad nepujde žena ta se mnou?
\par 40 Odpovedel mi: Hospodin, pred jehož oblícejem ustavicne jsem chodil, pošle andela svého s tebou, a štastnou zpusobí cestu tvou, a vezmeš manželku synu mému z rodiny mé, a z domu otce mého.
\par 41 A budeš svoboden od prísahy mé, když bys prišel k rodine mé; a jestliže by nedali tobe, budeš prost od prísahy mé.
\par 42 Protož dnes prišed k studnici, rekl jsem: Hospodine, Bože pána mého Abrahama, jestliže ty nyní štastne spravuješ cestu mou, po níž já jdu,
\par 43 Aj, já stojím u studnice vody; protož,nechžt panna, kteráž vyjde vážiti vody, když bych jí rekl: Dej mi píti nyní malicko vody z vederce svého,
\par 44 A ona by mi odpovedela: I ty pí, i velbloudum tvým také navážím, ta at jest manželka, kterouž zpusobil Hospodin synu pána mého.
\par 45 Prvé pak než jsem já prestal mluviti v srdci svém, aj, Rebeka vycházela mající vederce své na rameni svém, a sšedši k studnici, vážila. I rekl jsem jí: Prosím, dej mi píti.
\par 46 Ona pak rychle složila vederce své s sebe, a rekla: Pí, také i velbloudy tvé napojím. Tedy pil jsem, a napojila také velbloudy.
\par 47 I ptal jsem se jí, a rekl jsem: Cí jsi dcera? Ona odpovedela: Dcera Bathuele, syna Náchorova, jehož mu porodila Melcha. Tedy dal jsem náušnici na tvár její, a náramnice na ruce její.
\par 48 A skloniv hlavu, poklonu jsem ucinil Hospodinu, a dobrorecil jsem Hospodina, Boha pána mého Abrahama, kterýž vedl mne po ceste prímé, abych dceru bratra pána svého vzal synu jeho.
\par 49 Protož nyní, ciníte-li milosrdenství, a pravdu se pánem mým, oznamte mi; pakli nic, také mi oznamte, a obrátím se na pravo neb na levo.
\par 50 Tedy odpovedeli Lában a Bathuel, rkouce: Od Hospodina vyšla jest vec tato; nemužemet odepríti v nicemž.
\par 51 Aj, Rebeka pred tebou, vezmiž ji a jed; a necht jest manželkou synu pána tvého, jakož mluvil Hospodin.
\par 52 I stalo se, že jakž uslyšel služebník Abrahamuv slova jejich, poklonil se Hospodinu až k zemi.
\par 53 A vynav služebník ten nádoby stríbrné a nádoby zlaté a odev, dal Rebece; drahé také dary dal bratru jejímu a matce její.
\par 54 Tedy jedli a pili, on i muži, kteríž byli s ním, a zustali tu pres noc. Ráno pak když vstali, rekl: Propustte mne ku pánu mému.
\par 55 I odpovedel bratr její a matka: Nechat pozustane devecka s námi za nekterý den, aspon za deset; potom pujdeš.
\par 56 On pak rekl jim: Nezdržujte mne, ponevadž Hospodin štastnou zpusobil cestu mou; propusttež mne, at jdu ku pánu svému.
\par 57 I rekli: Zavolejme devecky a zeptejme se, co dí k tomu.
\par 58 Tedy zavolali Rebeky, a rekli jí: Chceš-li jíti s mužem tímto? I rekla: Pujdu.
\par 59 A propouštejíce Rebeku sestru svou a chovacku její, služebníka také Abrahamova s muži jeho,
\par 60 Požehnali Rebeky a mluvili jí: Sestro naše, ty bud v tisíce tisícu, a síme tvé dedicne obdrž brány neprátel svých.
\par 61 Tedy vstala Rebeka a devecky její, a vsedše na velbloudy, jely za mužem tím. A tak vzal služebník ten Rebeku, a odjel.
\par 62 Izák pak šel, navracuje se od studnice Živého vidoucího mne; nebo bydlil v zemi polední.
\par 63 A vyšel Izák k premyšlování na pole, když se chýlilo k vecerou; a pozdvih ocí svých, uzrel, an velbloudové jdou.
\par 64 Pozdvihla také Rebeka ocí svých, a uzrevši Izáka, ssedla s velblouda.
\par 65 (Nebo rekla byla služebníku: Kdo jest ten muž, kterýž jde po poli proti nám? Odpovedel služebník: To jest pán muj.) I vzala rouchu, a pristrela se.
\par 66 Tedy vypravoval služebník Izákovi vše, což pusobil.
\par 67 I uvedl ji Izák do stanu Sáry matky své, a vzal Rebeku, a mel ji za manželku, a miloval ji. I potešil se Izák po smrti matky své.

\chapter{25}

\par 1 Abraham pak opet pojal ženu jménem Ceturu.
\par 2 Kterážto porodila jemu Zamrana, a Jeksana, a Madana, a Madiana, Jezbocha a Suecha.
\par 3 Jeksan potom zplodil Sábu a Dedana. Synové pak Dedanovi byli: Assurim, a Latuzim, a Laomim.
\par 4 Ale synové Madianovi: Efa, a Efer, a Enoch, a Abida, a Helda; všickni ti synové byli Cetury.
\par 5 I dal Abraham Izákovi všecko, což mel.
\par 6 Synum pak ženin svých dal Abraham dary, a odeslal je od Izáka syna svého, ješte živ jsa, k východu do zeme východní.
\par 7 Tito pak jsou dnové let života Abrahamova, v nichž byl živ: Sto sedmdesáte a pet let.
\par 8 I skonal a umrel Abraham v starosti dobré, stár jsa a plný dnu; a pripojen jest k lidu svému.
\par 9 Tedy pochovali ho Izák a Izmael synové jeho v jeskyni Machpelah, na poli Efrona, syna Sohar Hetejského, naproti Mamre,
\par 10 Na tom poli, kteréž byl koupil Abraham od synu Het; tu pochován jest Abraham i Sára manželka jeho.
\par 11 Po smrti pak Abrahamove požehnal Buh Izákovi synu jeho, a bydlil Izák u studnice Živého vidoucího mne.
\par 12 Tito jsou pak rodové Izmaele syna Abrahamova, jehož porodila Agar Egyptská, devka Sárina, Abrahamovi.
\par 13 A tato jsou jména synu Izmaelových, jimiž se jmenují po rodech svých: Prvorozený Izmaeluv Nabajot, po nem Cedar, a Adbeel a Mabsan,
\par 14 A Masma, a Dumah a Massa,
\par 15 Hadar a Tema, Jetur, Nafis a Cedma.
\par 16 Ti jsou synové Izmaelovi, a ta jména jejich, po vsech jejich, a po mestech jejich, dvanáctero knížat po celedech jejich.
\par 17 (Bylo pak života Izmaelova sto tridceti a sedm let, i skonal; a umrev, pripojen jest k lidu svému.)
\par 18 A bydlili od Hevilah až do Sur, jenž jest proti Egyptu, když jdeš do Assyrie; pred tvárí všech bratrí svých položil se.
\par 19 Tito jsou také rodové Izáka syna Abrahamova: Abraham zplodil Izáka.
\par 20 Byl pak Izák ve ctyridcíti letech, když sobe vzal za manželku Rebeku, dceru Bathuele Syrského, z Pádan Syrské, sestru Lábana Syrského.
\par 21 I modlil se Izák pokorne Hospodinu za manželku svou; nebo byla neplodná. A uslyšel jej Hospodin; a tak pocala Rebeka manželka jeho.
\par 22 A když se deti potiskali v živote jejím, rekla: Má-lit tak býti, proc jsem já pocala? Šla tedy, aby se otázala Hospodina.
\par 23 I rekl jí Hospodin: Dva národové jsou v živote tvém, a dvuj lid z života tvého se rozdelí; lid pak jeden nad druhý bude silnejší, a vetší sloužiti bude menšímu.
\par 24 A když se naplnili dnové její, aby porodila, a aj, blíženci byli v živote jejím.
\par 25 I vyšel první ryšavý všecken, a jako odev chlupatý; i nazvali jméno jeho Ezau.
\par 26 Potom pak vyšel bratr jeho, a rukou svou držel Ezau za patu; procež nazváno jest jméno jeho Jákob. A byl Izák v šedesáti letech, když ona je porodila.
\par 27 A když dorostli ti deti, byl Ezau lovec umelý, chode po polích; Jákob pak byl muž prostý a v staních bydlil.
\par 28 I byl Izák laskav na Ezau, proto že z lovu jeho míval pokrm; ale Rebeka laskava byla na Jákoba.
\par 29 Uvaril pak Jákob krmicku. Tedy Ezau prišel z pole zemdlený,
\par 30 A rekl Jákobovi: Dej mi, prosím, jísti krme té cervené, nebo jsem umdlel. (Protož nazváno jest jméno jeho Edom.)
\par 31 Odpovedel Jákob: Prodej mi dnes hned prvorozenství své.
\par 32 I rekl Ezau: Aj, já k smrti se blížím, k cemuž mi tedy to prvorozenství?
\par 33 Dí Jákob: Prisáhni mi dnes hned. I prisáhl mu, a prodal prvorozenství své Jákobovi.
\par 34 Tedy Jákob dal Ezauchovi chleba a té krme z šocovice. Kterýžto jedl a pil, a vstav, odšel; a tak pohrdl Ezau prvorozenstvím.

\chapter{26}

\par 1 Byl pak opet hlad na zemi, mimo hlad první, kterýž byl za dnu Abrahamových. Tedy odebral se Izák k Abimelechovi králi Filistinskému do Gerar.
\par 2 Nebo ukázal se jemu Hospodin a rekl: Nesstupuj do Egypta; ale bydli v zemi, kterouž oznámím tobe.
\par 3 Budiž tedy pohostinu v zemi této, a budu s tebou, a požehnám tobe; nebo tobe a semeni tvému dám všecky zeme tyto, a utvrdím prísahu, kterouž jsem prisáhl Abrahamovi, otci tvému.
\par 4 Rozmnožím také síme tvé jako hvezdy nebeské, a dám semeni tvému všecky zeme tyto, a požehnáni budou v semeni tvém všickni národové zeme;
\par 5 Protože uposlechl Abraham hlasu mého, a ostríhal narízení mých, prikázaní mých, ustanovení mých a zákonu mých.
\par 6 Bydlil tedy Izák v Gerar.
\par 7 Ptali se pak muži místa toho o manželce jeho. Kterýžto odpovedel: Sestra má jest; nebo nesmel ríci: Manželka má jest, mysle sobe: Aby mne nezabili muži místa toho pro Rebeku. Nebo byla krásná na pohledení.
\par 8 I prihodilo se, když již cas nejaký tam bydlil, že vyhlédal Abimelech král Filistinský z okna a uzrel Izáka, an pohrává s Rebekou manželkou svou.
\par 9 Protož povolav Abimelech Izáka, rekl: Aj, v pravde manželka tvá to jest. Jakž to, že jsi pravil: Sestra má jest? I odpovedel jemu Izák: Nebo jsem rekl sám u sebe: Abych snad neumrel pro ni.
\par 10 I rekl Abimelech: Což jsi to ucinil nám? O málo, že by byl spal nekdo z lidu s manželkou tvou, a ty byl bys uvedl na nás vinu.
\par 11 I prikázal Abimelech všemu lidu, rka: Kdo by se dotkl cloveka toho, aneb manželky jeho, smrtí umre.
\par 12 Sel pak Izák v zemi té, a shledal v tom roce sto mer; nebo požehnal mu Hospodin.
\par 13 I rostl muž ten, a prospíval vždy více v zrostu, až i zrostl velmi.
\par 14 Nebo mel stáda ovcí i stáda volu, a celedi mnoho; procež závideli mu Filistinští.
\par 15 A všecky studnice, kteréž vykopali služebníci otce jeho za dnu Abrahama otce jeho, zarítili Filistinští, zasypavše je prstí.
\par 16 I rekl Abimelech Izákovi: Odejdi od nás; nebo mnohem mocnejší jsi než my.
\par 17 Tedy odšel odtud Izák, a rozbil stany v údolí Gerar, a bydlil tu.
\par 18 A kopal zase Izák studnice vod, kteréž byli vykopali za dnu Abrahama otce jeho, a kteréž zarítili Filistinští po smrti Abrahamove; a nazval je temi jmény, kterýmiž je jmenoval otec jeho.
\par 19 I kopali služebníci Izákovi v tom údolí, a nalezli tam studnici vody živé.
\par 20 Vadili se pak pastýri Gerarští s pastýri Izákovými, pravíce: Naše jest voda. Procež nazval jméno studnice té Esek, že se vadili s ním.
\par 21 Vykopali také jinou studnici, a nesnáz byla i o tu; procež dal jí jméno Sitnah.
\par 22 I hnul se odtud, a kopal jinou studnici, o kterouž se nevadili; protož nazval jméno její Rehobot. Nebo rekl: Nyní uprostrannil nám Hospodin, a vzrostli jsme na zemi.
\par 23 Vstoupil pak odtud do Bersabé.
\par 24 A ukázal se mu Hospodin v tu noc, a rekl: Já jsem Buh Abrahama otce tvého; neboj se, nebo s tebou já jsem, a požehnám tobe, a rozmnožím síme tvé pro Abrahama služebníka svého.
\par 25 I vzdelal tu oltár, a vzýval jméno Hospodinovo, a rozbil tu stan svuj; a služebníci Izákovi vykopali tam studnici.
\par 26 Abimelech pak prijel k nemu z Gerar, a Ochozat, prítel jeho, a Fikol, kníže vojska jeho.
\par 27 I rekl jim Izák: Z jaké príciny prišli jste ke mne? Ponevadž vy nenávideli jste mne, a vybyli jste mne od sebe.
\par 28 Kterížto odpovedeli: Patrne jsme to shledali, že jest Hospodin s tebou, i rekli jsme: Ucinme nyní prísahu mezi sebou, mezi námi a mezi tebou; a uciníme smlouvu s tebou:
\par 29 Že neuciníš nám nic zlého, jako i my nedotkli jsme se tebe, a jakž jsme my toliko dobre cinili tobe, a propustili jsme te v pokoji; ty nyní tedy povol tomu, požehnaný Hospodinuv.
\par 30 Tedy ucinil jim hody, i jedli a pili.
\par 31 A vstavše velmi ráno, prisáhli jeden druhému. I propustil je Izák, a oni odešli od neho v pokoji.
\par 32 Toho dne prišli služebníci Izákovi, a oznámili mu o studnici, kterouž kopali, rkouce: Nalezli jsme vodu.
\par 33 I nazval ji Seba. Protož jméno mesta toho jest Bersabé až do dnešního dne.
\par 34 Ezau pak jsa v letech ctyridcíti, pojal ženu Judit, dceru Béry Hetejského, a Bazematu, dceru Elona Hetejského.
\par 35 A kormoutily Izáka a Rebeku.

\chapter{27}

\par 1 Když se pak sstaral Izák, a pošly byly oci jeho, tak že nevidel, povolal Ezau syna svého staršího, a rekl jemu: Synu muj. Kterýžto odpovedel: Aj, ted jsem.
\par 2 I dí: Aj, já jsem se již sstaral, a nevím dne smrti své.
\par 3 Protož nyní vezmi medle nástroje své, toul svuj a lucište své, a vyjda do pole, ulov mi zverinu.
\par 4 A pristroj mi krmi chutnou, jakž já rád jídám, a prines mi, a budu jísti, atby požehnala duše má, prvé než umru.
\par 5 (Slyšela pak Rebeka, když mluvil Izák s Ezau synem svým.) I odšel Ezau na pole, aby ulovil zverinu a prinesl.
\par 6 Tedy rekla Rebeka Jákobovi synu svému takto: Aj, slyšela jsem, když otec tvuj mluvil k Ezau bratru tvému, a pravil:
\par 7 Prines mi neco z lovu, a pristroj mi krme chutné, abych jedl, a požehnámt pred Hospodinem, prvé než umru.
\par 8 Nyní tedy, synu muj, poslechni hlasu mého v tom, což já prikazuji tobe.
\par 9 Jdi medle k stádu, a odtud mi vezmi dva kozelce výborné, abych pristrojila z nich krme chutné otci tvému, jakž on rád jídá.
\par 10 A prineseš otci svému, a on jísti bude, na to, aby požehnal tobe, prvé než umre.
\par 11 I rekl Jákob Rebece matce své: Víš, že Ezau bratr muj jest clovek chlupatý, a já jsem clovek hladký.
\par 12 Jestliže omaká mne otec muj, tedy zustanu u neho za podvodného; a tak uvedu na sebe zlorecenství, a ne požehnání.
\par 13 Jemužto odpovedela matka: Necht jest na mne zlorecenství tvé, synu muj; jen ty poslechni hlasu mého, a jdi, prines mi.
\par 14 Tedy odšed, vzal, a prinesl matce své. I pripravila matka jeho krme chutné, jakž otec jeho rád jídal.
\par 15 A vzala Rebeka nejlepší šaty Ezausyna svého staršího, kteréž mela u sebe v dome, a oblékla Jákoba, syna svého mladšího.
\par 16 A kožkami kozelcími obvinula ruce jeho, a díl hladký hrdla jeho.
\par 17 I dala chléb a krme chutné, kteréž pripravila, v ruce Jákoba syna svého.
\par 18 A on prišed k otci svému, rekl: Otce muj! Kterýžto odpovedel: Aj, ted jsem. Kdož jsi ty, synu muj?
\par 19 I dí Jákob otci svému: Já jsem Ezau,prvorozený tvuj. Ucinil jsem, jakž jsi mi porucil. Vstan, prosím, sedni a jez z lovu mého, aby mi požehnala duše tvá.
\par 20 Tedy rekl Izák synu svému: Což to? Brzys to nalezl, synu muj. A on dí: Nebo Hospodin Buh tvuj zpusobil to, aby mi se pojednou nahodilo.
\par 21 I rekl Izák Jákobovi: Pristupiž, at omakám te, synu muj, ty-li jsi syn muj Ezau, ci nejsi.
\par 22 Tedy pristoupil Jákob k Izákovi otci svému; a on omakal ho, i rekl: Hlas jest hlas Jákobuv, ale ruce tyto ruce Ezau.
\par 23 A nepoznal ho; nebo byly ruce jeho, jako ruce Ezau bratra jeho, chlupaté. I požehnal mu.
\par 24 A rekl: Ty-liž jsi pak syn muj Ezau? Odpovedel: Já.
\par 25 I rekl: Podejž mi, at jím z lovu syna svého, aby tobe požehnala duše má. Tedy podal mu, a on jedl. Prinesl mu také vína, a on pil.
\par 26 I rekl jemu Izák otec jeho: Pristupiž nyní, a polib mne, synu muj.
\par 27 I pristoupil a políbil ho. A jakž ucítil Izák vuni roucha jeho, požehnal mu, rka: Pohled, vune syna mého jest jako vune pole, jemuž požehnal Hospodin.
\par 28 Dejž tobe Buh z rosy nebeské, a z tucnosti zemské, i hojnost obilé a vína.
\par 29 Nechažt slouží tobe lidé, a sklánejí se pred tebou národové. Budiž pánem bratrí svých, a at se sklánejí pred tebou synové matky tvé. Kdož by zlorecili tobe, necht jsou zlorecení, a kdo by dobrorecili tobe, požehnaní.
\par 30 A když prestal Izák požehnání dávati Jákobovi, a sotva že vyšel Jákob od Izáka otce svého, Ezau bratr jeho prišel z lovu svého.
\par 31 A pripraviv i on krme chutné, prinesl otci svému a rekl mu: Nechžt vstane otec muj, a jí z lovu syna svého, aby mi požehnala duše tvá.
\par 32 I rekl jemu Izák otec jeho: Kdo jsi ty? Dí on: Já jsem syn tvuj, prvorozený tvuj Ezau.
\par 33 Tedy zhrozil se Izák hruzou velikou náramne, a rekl: Kdo pak a kde jest ten, ješto uloviv zverinu, prinesl mi? A já jsem jedl ze všeho, prvé než jsi ty prišel, a požehnal jsem mu, a budet požehnaným.
\par 34 Uslyšev Ezau slova otce svého, zkrikl hlasem velikým, a horkostí naplnen jest náramne, a rekl otci svému: Požehnejž mne, i mne také, muj otce!
\par 35 Kterýžto rekl: Prišel bratr tvuj podvodne a uchvátil požehnání tvé.
\par 36 Tedy rekl: Právete nazváno jméno jeho Jákob; nebo již po dvakráte mne podvedl, prvorozenství mé odjal, a ted nyní uchvátil požehnání mé. Rekl ješte: Zdali jsi nezachoval i mne požehnání?
\par 37 Odpovedel Izák a rekl k Ezau: Aj, ustavil jsem ho pánem nad tebou, a všecky bratrí jeho dal jsem mu za služebníky; obilím také a vínem opatril jsem ho. Což tedy již tobe uciniti mohu, synu muj?
\par 38 I rekl Ezau otci svému: Zdaliž to jedno toliko máš požehnání, otce muj? Požehnejž mne, i mne také, muj otce! I povýšil Ezau hlasu svého a plakal.
\par 39 Tedy odpovedel Izák otec jeho, a rekl k nemu: Aj, v tucnostech zeme bude bydlení tvé a v rose nebeské s hury;
\par 40 A v meci svém živ budeš, a bratru svému sloužiti budeš; ale prijde cas, že budeš panovati a svržeš jho jeho s šíje své.
\par 41 Protož v nenávisti mel Ezau Jákoba pro požehnání, jímž požehnal mu otec jeho, a rekl v srdci svém: Priblížit se dnové smutku otce mého, a zabiji Jákoba, bratra svého.
\par 42 I oznámena jsou Rebece slova Ezau, staršího syna jejího. Procež ona poslavši, povolala Jákoba syna svého mladšího, a rekla jemu: Hle, Ezau bratr tvuj teší se tím, že te zabije.
\par 43 Protož nyní, synu muj, poslechni hlasu mého, a vstana, utec k Lábanovi, bratru mému do Cháran.
\par 44 A pobud s ním za nekterý cas, až by se odvrátila prchlivost bratra tvého,
\par 45 A prestalo rozhnevání bratra tvého na tebe, a zapomenul by na to, což jsi mu ucinil. Potom já pošli a vezmu te odtud. Proc mám zbavena býti obou synu jednoho dne?
\par 46 I rekla Rebeka Izákovi: Stýšte mi se živu býti pro dcery Het. Vezme-li Jákob ženu ze dcer Het, jako tyto jsou ze dcer zeme této, k cemu mi život?

\chapter{28}

\par 1 Povolal pak Izák Jákoba, a požehnal jemu, a prikázal mu, rka: Nepojímej ženy ze dcer Kananejských.
\par 2 Ale vstana, jdi do Pádan Syrské do domu Bathuele, otce matky své, a pojmi sobe odtud manželku ze dcer Lábana ujce svého.
\par 3 A Buh silný všemohoucí požehnejž tobe, a dejžt zrust, a rozmnožiž te, abys byl v zástup mnohého lidu.
\par 4 A dejž tobe požehnání Abrahamovo, tobe i semeni tvému s tebou, abys dedicne obdržel zemi, v níž pohostinu jsi, kterouž dal Buh Abrahamovi.
\par 5 I odeslal Izák Jákoba, kterýžto šel do Pádan Syrské k Lábanovi synu Bathuele Syrského, bratru Rebeky matky Jákobovy a Ezau.
\par 6 Vida pak Ezau, že požehnání dal Izák Jákobovi, a že ho odeslal do Pádan Syrské, aby sobe odtud vzal manželku, a že, když mu požehnání dával, prikázal mu, rka: Nepojmeš ženy ze dcer Kananejských;
\par 7 A že by uposlechl Jákob otce svého a matky své a odšel do Pádan Syrské;
\par 8 Vida také Ezau, že dcery Kananejské težké byly v ocích Izákovi otci jeho:
\par 9 Tedy odšel Ezau k Izmaelovi, a mimo prvnejší ženy své, pojal sobe za ženu Mahalat, dceru Izmaele, syna Abrahamova, sestru Nabajotovu.
\par 10 Vyšed pak Jákob z Bersabé, šel do Cháran.
\par 11 I trefil na jedno místo, na kterémžto zustal pres noc, (nebo slunce již bylo zapadlo,) a nabrav kamení na míste tom, položil pod hlavu svou, a spal na témž míste.
\par 12 I videl ve snách, a aj, žebrík stál na zemi, jehožto vrch dosahal nebe; a aj, andelé Boží vstupovali a sstupovali po nem.
\par 13 A aj, Hospodin stál nad ním, a rekl: Já jsem Hospodin, Buh Abrahama otce tvého, a Buh Izákuv; zemi tu, na kteréž ty spíš, tobe dám a semeni tvému.
\par 14 A bude síme tvé jako prach zeme; nebo rozmužeš se k západu, i k východu, na pulnoci, i ku poledni; nad to požehnány budou v tobe všecky celedi zeme, a v semeni tvém.
\par 15 A aj, já jsem s tebou, a ostríhati te budu, kamžkoli pujdeš, a privedu te zase do zeme této; nebo neopustím tebe, až i uciním, což jsem mluvil tobe.
\par 16 Procítiv pak Jákob ze sna svého, rekl: V pravde Hospodin jest na míste tomto, a já jsem nevedel.
\par 17 (Nebo zhroziv se, rekl: Jak hrozné jest místo toto! Není jiného,jediné dum Boží, a tu jest brána nebeská.)
\par 18 Vstav pak Jákob ráno, vzal kámen, kterýž byl podložil pod hlavu svou, a postavil jej na znamení pametné, a polil jej svrchu olejem.
\par 19 Protož nazval jméno místa toho Bethel, ješto prvé to mesto sloulo Luza.
\par 20 Zavázal se také Jákob slibem, rka: Jestliže Buh bude se mnou, a ostríhati mne bude na ceste této, kterouž já jdu; a dá-li mi chléb ku pokrmu a roucho k odevu,
\par 21 A navrátím-li se v pokoji do domu otce svého, a bude mi Hospodin za Boha:
\par 22 Kámen tento, kterýž jsem postavil na památku, bude domem Božím; a ze všech vecí, kteréž mi dáš, desátky spravedlive tobe dám.

\chapter{29}

\par 1 Tedy Jákob vstav, odšel do zeme východní.
\par 2 A pohledev, uzrel studnici v poli a tri stáda ovcí, ana se složila pri ní; nebo z té studnice napájeli stáda; a kámen veliký byl navrchu studnice.
\par 3 Shánívána pak tam bývala všecka stáda, a teprv odvalovali kámen ten od vrchu té studnice, a napájeli dobytky; potom zas privalovali kámen na vrch studnice, na místo jeho.
\par 4 Tedy rekl jim Jákob: Bratrí moji, odkud jste? Kterížto odpovedeli: Jsme z Cháran.
\par 5 I rekl jim: Znáte-li Lábana, syna Náchorova? A oni rekli: Známe.
\par 6 I dí opet k nim: Dobre-li se má? Odpovedeli: Dobre, a aj, Ráchel dcera jeho tamto jde s dobytkem.
\par 7 Tedy rekl: Však ješte daleko do vecera, aniž ješte cas, aby dobytek byl sehnán; napojte stáda, a jdete, paste.
\par 8 A oni odpovedeli: Nemužeme, než až se všecka stáda seženou, a odvalen bude kámen od svrchku studnice, abychom napojili ovce.
\par 9 Když on ješte mluvil s nimi, prišla k tomu Ráchel s stádem otce svého; neb ona pásla stádo.
\par 10 Tedy Jákob uzrev Ráchel, dceru Lábana ujce svého, a stádo jeho, pristoupil a odvalil ten kámen od vrchu studnice, a napojil stádo Lábana ujce svého.
\par 11 A políbil Jákob Ráchel, a povýšiv hlasu svého, plakal.
\par 12 Oznámil jí pak byl Jákob, že jest bratr otce jejího, a že jest syn Rebeky; a ona pribehši, oznámila to otci svému.
\par 13 I stalo se, když Lában uslyšel povest o Jákobovi, synu sestry své, že vybehl proti nemu, a objav ho, políbil, a uvedl do domu svého; on pak vypravoval Lábanovi o všech tech vecech.
\par 14 Jemužto odpovedel Lában: Jiste kost má a telo mé jsi. I zustal s ním pres celý mesíc.
\par 15 Tedy rekl Lában Jákobovi: Zdali proto, že jsi bratr muj, darmo mi sloužiti budeš? Povez mi, jaká má býti mzda tvá?
\par 16 (Mel pak Lában dve dcery; jméno starší Lía, a jméno mladší Ráchel.
\par 17 Než Lía mela oci mdlé, ale Ráchel byla pekné postavy a krásné tvári.
\par 18 I miloval Jákob Ráchel.) Rekl tedy: Budut sloužiti sedm let za Ráchel, dceru tvou mladší.
\par 19 Odpovedel Lában: Lépet jest, abych ji tobe dal, nežli bych ji dal muži jinému; zustaniž se mnou.
\par 20 Takž Jákob sloužil za Ráchel sedm let; a bylo pred ocima jeho jako neco málo dnu, proto že laskav byl na ni.
\par 21 Potom rekl Jákob k Lábanovi: Dejž mi ženu mou; nebo vyplneni jsou dnové moji, abych všel k ní.
\par 22 Tedy sezval Lában všecky muže místa toho, a ucinil hody.
\par 23 U vecer pak vzav Líu, dceru svou, uvedl ji k nemu; a on všel k ní.
\par 24 Dal také Lában Zelfu devku svou Líe, dceri své, za služebnici.
\par 25 A když bylo ráno, poznal, že jest Lía. I rekl k Lábanovi: Což jsi mi to ucinil? Zdaliž jsem za Ráchel nesloužil u tebe? Procežs mne tedy oklamal?
\par 26 Odpovedel Lában: Nebývá toho v kraji našem, aby vdávána byla mladší dríve, než prvorozená.
\par 27 Vypln týden této; dámet potom i tuto za službu, kterouž sloužiti budeš u mne ješte sedm let jiných.
\par 28 I udelal tak Jákob, a vyplnil týden její. Potom dal mu Ráchel dceru svou za manželku.
\par 29 Dal také Lában Ráchel, dceri své, Bálu, devku svou za služebnici.
\par 30 Tedy všel také k Ráchel, a miloval ji více než Líu; i sloužil u neho ješte sedm let jiných.
\par 31 Vida pak Hospodin, že Lía nemá lásky, otevrel život její; a Ráchel nechal neplodné.
\par 32 Tedy pocala Lía a porodila syna, a nazvala jméno jeho Ruben; nebo rekla: Jiste videl Hospodin trápení mé; jižt nyní milovati mne bude muž muj.
\par 33 I pocala opet a porodila syna, a rekla: Jiste uslyšel Hospodin, že jsem v nenávisti, protož dal mi i tohoto. A nazvala jméno jeho Simeon.
\par 34 A opet pocala a porodila syna, a rekla: Již nyní pripojí se ke mne muž muj, nebo jsem mu porodila tri syny. Z té príciny nazváno jest jméno jeho Léví.
\par 35 I pocala ješte a porodila syna, a rekla: Již nyní chváliti budu Hospodina. Procež nazvala jméno jeho Juda; i prestala roditi.

\chapter{30}

\par 1 Viduci pak Ráchel, že by nerodila Jákobovi, závidela sestre své, a rekla Jákobovi: Dej mi syny; pakli nedáš, umru.
\par 2 Procež rozhneval se velmi Jákob na Ráchel, a rekl: Zdali já jsem za Boha, kterýžt nedal plodu života?
\par 3 Rekla ona: Hle, devka má Bála; vejdi k ní, aby rodila na kolena má, a budu míti já také syny z ní.
\par 4 I dala mu Bálu devku svou za ženu; a všel k ní Jákob.
\par 5 Tedy pocavši Bála, porodila Jákobovi syna.
\par 6 I rekla Ráchel: Soudil Buh pri mou, a uslyšel také hlas muj, a dal mi syna. Protož nazvala jméno jeho Dan.
\par 7 Opet pocavši Bála, devka Ráchel, porodila syna druhého Jákobovi.
\par 8 I rekla Ráchel: Tuhé jsem odpory mela s sestrou svou, a všakt jsem premohla. A nazvala jméno jeho Neftalím.
\par 9 Viduci pak Lía, že by prestala roditi, vzala Zelfu devku svou, a dala ji Jákobovi za ženu.
\par 10 A porodila Zelfa, devka Líe, Jákobovi syna.
\par 11 Protož rekla Lía: Již prišel zástup. A nazvala jméno jeho Gád.
\par 12 Porodila také Zelfa devka Líe syna druhého Jákobovi.
\par 13 A rekla Lía: To na mé štestí; nebo štastnou mne nazývati budou ženy. A nazvala jméno jeho Asser.
\par 14 Vyšel pak Ruben v cas žne pšenicné, a nalezl pekná jablecka na poli, a prinesl je Líe matce své. I rekla Ráchel Líe: Dej mi, prosím, tech jablecek syna svého.
\par 15 Jížto ona odpovedela: Málot se snad zdá, že jsi vzala muže mého; chceš také užívati jablecek syna mého? I rekla Ráchel: Nechažt tedy spí s tebou této noci za jablecka syna tvého.
\par 16 Když pak navracoval se Jákob s pole vecer, vyšla Lía proti nemu, a rekla: Ke mne vejdeš; nebo ze mzdy najala jsem te za jablecka syna svého. I spal s ní té noci.
\par 17 A uslyšel Buh Líu; kterážto pocala a porodila Jákobovi syna pátého.
\par 18 I rekla Lía: Dal mi Buh mzdu mou, i potom, když jsem dala devku svou muži svému. Procež nazvala jméno jeho Izachar.
\par 19 A pocala opet Lía, a porodila šestého syna Jákobovi.
\par 20 I rekla Lía: Obdaril mne Buh darem dobrým; již nyní bydliti bude se mnou muž muj, nebo porodila jsem mu šest synu. A nazvala jméno jeho Zabulon.
\par 21 Potom porodila dceru; a nazvala jméno její Dína.
\par 22 A rozpomenuv se Buh na Ráchel,uslyšel jí, a otevrel život její.
\par 23 Tedy pocala a porodila syna, a rekla: Odjal Buh pohanení mé.
\par 24 A nazvala jméno jeho Jozef, rkuci: Pridejž mi Hospodin syna jiného.
\par 25 Stalo se pak, když porodila Ráchel Jozefa, rekl Jákob Lábanovi: Propust mne, at odejdu na místo své a do zeme své.
\par 26 Dej mi ženy mé a dítky mé, za kteréž jsem sloužil tobe, at odejdu; nebo ty znáš službu mou, kterouž jsem sloužil tobe.
\par 27 I rekl mu Lában: Jestliže nyní nalezl jsem milost pred ocima tvýma, zustan se mnou, nebo v skutku jsem poznal, že požehnal mi Hospodin pro tebe.
\par 28 Rekl také: Oznam mi ze jména mzdu svou a dámt ji.
\par 29 Jemužto odpovedel: Ty víš, jak jsem sloužil tobe, a jaký byl dobytek tvuj pri mne.
\par 30 Nebo to málo, kteréž jsi mel prede mnou, zrostlo velmi, a požehnalt Hospodin, jakž jsem k tobe nohou vkrocil. A nyní, kdy pak já své hospodárství opatrovati budu?
\par 31 A rekl: Cot mám dáti? Odpovedel Jákob: Nedávej mi nic. Jestliže mi uciníš toto, zase pásti budu a ostríhati dobytka tvého:
\par 32 Projdu skrze všecka stáda tvá dnes, vymešuje z nich každé dobytce peresté a strakaté, a každé dobytce nacernalé mezi ovcemi, a strakaté a peresté mezi kozami; a takové budou mzda má.
\par 33 A osvedcena potom bude spravedlnost má pred tebou, když prijde na mzdu mou: Cožkoli nebude perestého, neb strakatého mezi kozami, a nacernalého mezi ovcemi, za krádež bude mi to pocteno.
\par 34 I rekl Lában: Hle, ó by tak bylo, jakž jsi mluvil.
\par 35 A odloucil toho dne kozly prepásané na nohách a strakaté, a všecky kozy peresté a strakaté, všecko, na cemž byla místa bílá, všecko také nacernalé mezi dobytkem, a dal v ruce synu svých.
\par 36 Uložil pak mezi sebou a Jákobem místo vzdálí trí dní cesty; a Jákob pásl ostatek dobytka Lábanova.
\par 37 Nabral pak sobe Jákob prutu topolových zelených, a lískových a kaštanových; a poobloupil s nich po místech kuru až do belosti, kteráž byla na prutech.
\par 38 A nakladl tech prutu tak obloupených do žlábku a koryt, (v nichž bývá voda, k nimž pricházel dobytek, aby pil), proti dobytku, aby pocínaly, když by pricházely píti.
\par 39 I pocínaly ovce, hledíce na ty pruty, a rodily jehnata prepásaná na nohách, a perestá i strakatá.
\par 40 Potom ta jehnata odloucil Jákob, a dobytek stáda Lábanova obrátil tvárí k tem prepásaným na nohách, a ke všemu nacernalému; a své stádo postavil obzvlášt, a neobrátil ho k stádu Lábanovu.
\par 41 A bylo, že kdyžkoli silnejší pripouštíny bývaly, kladl Jákob ty pruty pred oci ovcem do koryt, aby pocínaly, hledíce na pruty.
\par 42 Když pak pozdní dobytek pripouštín býval, nekladl jich; a tak býval pozdní Lábanuv a ranný Jákobuv.
\par 43 Vzrostl tedy muž ten náramne velmi, a mel dobytka mnoho, devek i služebníku, velbloudu i oslu.

\chapter{31}

\par 1 Když pak uslyšel Jákob slova synu Lábanových, ani praví: Pobral Jákob všecko, co mel otec náš, a z tech vecí, kteréž byly otce našeho, zpusobil sobe tu všecku slávu;
\par 2 A videl tvár Lábanovu, a aj, nebyla k nemu tak jako prvé:
\par 3 I rekl mu Hospodin: Navrat se do zeme otcu svých, a k príbuznosti své, a budu s tebou.
\par 4 Protož poslav Jákob, vyvolal Ráchele a Líe na pole k stádu svému.
\par 5 A rekl jim: Vidím tvár otce vašeho, že není ke mne tak jako prvé, ješto Buh otce mého byl se mnou.
\par 6 A vy samy víte, že vší svou silou sloužil jsem otci vašemu.
\par 7 Ale otec váš oklamal mne, a na desetkráte zmenil mzdu mou; však nedopustil mu Buh, aby mi zle ucinil.
\par 8 Když takto rekl: Co bude perestého,to necht jest mzda tvá, tedy všecky ovce rodily perestý plod. Když pak takto rekl: Co prepásaného na nohách, to bude mzda tvá, tedy všecky ovce rodily prepásané na nohách.
\par 9 A tak odjal Buh stádo otce vašeho a dal mne.
\par 10 Nebo takto bylo: Tehdáž, když se dobytek bežel, já jsem pozdvihl ocí svých, a videl jsem ve snách, a aj, berani scházející se s ovcemi byli prepásaní na nohách, perestí a skropení.
\par 11 Tedy rekl mi andel Boží ve snách: Jákobe! A já jsem odpovedel: Ted jsem.
\par 12 I rekl: Pozdvihni nyní ocí svých a pohled, že všickni berani scházející se s ovcemi jsou prepásaní na nohách, perestí a skropení; nebo jsem videl všecko, co tobe Lában delá.
\par 13 Já jsem ten silný Buh, kterýž jsem se ukázal tobe v Bethel, kdež jsi pomazal kamene, kdež jsi mi se slibem zavázal; vstaniž nyní, vyjdi z zeme této, a navrat se do zeme príbuznosti své.
\par 14 Tedy odpovedely mu Ráchel a Lía, rkouce: Zdaž ješte máme díl jaký a dedictví v dome otce našeho?
\par 15 Zdaliž nejsme pocteny pred ním za cizí? Nebo prodal nás, ano i peníze naše do cista utratil.
\par 16 Všecko zajisté bohatství, kteréž odjal Buh otci našemu, naše jest a synu našich; protož nyní ucin vše, což mluvil tobe Buh.
\par 17 Vstav tedy Jákob, vsadil syny své a ženy své na velbloudy.
\par 18 A pobral všecken dobytek svuj, a všecko jmení své, kteréhož byl dosáhl, dobytek svého vychování, kteréhož nabyl v Pádan Syrské, aby se navrátil k Izákovi otci svému do zeme Kananejské.
\par 19 (Lában pak byl odšel, aby strihl ovce své. V tom ukradla Ráchel modly, kteréž mel otec její.
\par 20 I ušel Jákob tajne od Lábana Syrského, neoznámiv mu, že jde pryc.)
\par 21 Takž ušel se vším, což mel; a vstav, prepravil se pres reku, a bral se prímo k hore Galádské.
\par 22 I povedíno jest Lábanovi dne tretího, že utekl Jákob.
\par 23 Tedy on vzav bratrí své s sebou, honil ho za dnu sedm; a postihl ho na hore Galádské.
\par 24 (Prišed pak Buh k Lábanovi Syrskému ve snách noci té, rekl mu: Varuj se, abys nemluvil s Jákobem nic jinác než prátelsky.)
\par 25 Dohonil tedy Lában Jákoba; a Jákob již byl rozbil stan svuj na té hore; Lában také položil se s bratrími svými na též hore Galádské.
\par 26 I rekl Lában k Jákobovi: Cos to ucinil? Nebo jsi ušel tajne, a odvedls dcery mé jako zjímané mecem.
\par 27 Proc jsi tajne utekl, a vykradl se ode mne, a nepovedels mi, ješto bych te byl vesele sprovodil s zpevy, s bubnem a s harfou?
\par 28 A nedopustils mi, abych políbil synu svých a dcer svých? Nemoudres jiste udelal, cine tak.
\par 29 Melt bych dosti s to moci, abych vám zle ucinil, ale Buh otce vašeho mluvil mi noci pominulé, rka: Hled, abys nemluvil s Jákobem nic jinác, než prátelsky.
\par 30 Ale již to tam, ponevadž jsi predce odšel, roztouživ se po domu otce svého; než proc jsi ukradl bohy mé?
\par 31 Tedy Jákob odpovídaje Lábanovi, rekl: Ušel jsem tajne; nebo jsem se bál a rekl jsem, že bys snad mocí odjal dcery své.
\par 32 Nalezneš-li pak u koho bohy své,nechat ten umre; pred bratrími našimi ohledej sobe, jest-li co u mne tvého, a vezmi sobe. Ale nevedel Jákob, že Ráchel je ukradla.
\par 33 Všed tedy Lában do stanu Jákobova, a do stanu Líe, a do stanu obou devek, nic nenalezl. A vyšed z stanu Líe, všel do stanu Ráchel.
\par 34 Ráchel pak vzavši modly, vložila je pod sedlo, kteréž na velbloudu bývá, a sedla na ne. I premetal Lában všecken stan, ale nic nenalezl.
\par 35 A ona rekla otci svému: Necht to není proti mysli, pane muj, že nemohu povstati proti tobe; nebo vedlé behu ženského nyní mi se prihodilo. A všecko prehledav, nenalezl modl.
\par 36 Protož rozhnevav se Jákob, tuze se domlouval na Lábana. I odpovedel Jákob, a rekl Lábanovi: Jaké jest prestoupení mé? Jaký hrích muj, že rozpáliv se, honíš mne?
\par 37 Nu již jsi premetal všecky mé veci, a cos nalezl ze všech vecí domu svého? Polož ted pred bratrími mými a bratrími svými, necht rozeznají mezi námi dvema.
\par 38 Byl jsem již dvadceti let s tebou; ovce tvé ani kozy tvé nikdy nezmetaly; a skopcu stáda tvého nejedl jsem.
\par 39 Co od zveri roztrháno, toho jsem neodvodil tobe; sám jsem tu škodu nahražoval, a ty jsi z ruky mé vyhledával toho, jako i toho, co bylo ukradeno ve dne aneb v noci.
\par 40 Bývalo tak, že ve dne trápilo mne horko, a v noci mráz, tak že odcházel i sen muj od ocí mých.
\par 41 Již dvadceti let, jakž jsem v dome tvém; sloužil jsem tobe ctrnácte let za dve dcery tvé, a šest let za dobytek tvuj, a zmenils mzdu mou na desetkrát.
\par 42 A kdyby Buh otce mého, Buh Abrahamuv, a strach Izákuv nebyl se mnou, jiste bys ty byl nyní pustil mne prázdného; ale trápení mé, a práci rukou mých videl Buh, protož te pominulé noci trestal.
\par 43 Odpovídaje pak Lában, dí Jákobovi: Dcery tyto jsou dcery mé a synové tito jsou synové moji, i dobytek tento muj dobytek jest, ano cokoli vidíš, mé jest; ale což mám již uciniti dcerám temto svým aneb synum jich, kteréž zrodily?
\par 44 Protož pod, vejdeme v smlouvu já a ty, aby byla na svedectví mezi mnou a tebou.
\par 45 Tedy Jákob vzal kámen, a postavil jej vzhuru na znamení.
\par 46 A rekl bratrím svým: Nasbírejte kamení. A nabravše kamení, udelali hromadu, a jedli tu na té hromade.
\par 47 I nazval ji Lában Jegar Sahadutha, a Jákob nazval ji Gál Ed.
\par 48 Nebo rekl Lában: Tato hromada nechžt jest svedkem od dnešku mezi mnou a tebou. Protož nazval jméno její Gál Ed,
\par 49 A Mispah; nebo rekl Lában: Nechat Hospodin hledí na mne a na tebe, když se rozejdeme od sebe.
\par 50 Jestliže bys trápil dcery mé, a uvedl bys jiné ženy na dcery mé, žádného cloveka není s námi; hlediž, Buh jest svedek mezi mnou a tebou.
\par 51 A rekl ješte Lában Jákobovi: Aj, hromada tato, a aj, sloup, kterýž jsem postavil mezi sebou a tebou,
\par 52 Svedkem at jest hromada tato, svedkem také sloup tento, já že nepujdu dále k tobe za hromadu tuto, a ty tolikéž že nepujdeš dále ke mne za hromadu tuto a sloup tento, k cinení zlého.
\par 53 Buh Abrahamuv, a Buh Náchoruv, Buh otce jejich necht soudí mezi námi. Prisáhl tedy Jákob skrze strach otce svého Izáka.
\par 54 Nabil také Jákob hovad na té hore, a pozval bratrí svých, aby hodovali; i hodovali a zustali pres noc na též hore.
\par 55 I vstal Lában velmi ráno, a políbiv synu svých a dcer svých, požehnal jich; i odšel a navrátil se k místu svému.

\chapter{32}

\par 1 Jákob pak odšel cestou svou, a potkali se s ním andelé Boží.
\par 2 I rekl Jákob, když je videl: Vojsko Boží jest toto. A nazval jméno místa toho Mahanaim.
\par 3 Poslal pak Jákob posly pred sebou k bratru svému Ezau, do zeme Seir, do kraje Idumejského.
\par 4 A prikázal jim, rka: Takto povezte pánu mému Ezau: Totot vzkazuje služebník tvuj Jákob: U Lábana jsem byl pohostinu, a zustával až do tohoto casu.
\par 5 A mám voly a osly, ovce, a služebníky i devky; a poslal jsem, abych se ohlásil pánu svému, a našel milost pred ocima tvýma.
\par 6 I navrátili se poslové k Jákobovi, rkouce: Prišli jsme k bratru tvému Ezau, kterýž také jde proti tobe, a ctyri sta mužu s ním.
\par 7 Jákob pak bál se velmi, a rmoutil se náramne. Tedy rozdelil lid, kterýž s sebou mel, ovce také a voly, a velbloudy na dva houfy.
\par 8 Nebo rekl: Jestliže by prišel Ezau k houfu jednomu, a pobil by jej, bude aspon zadní houf zachován.
\par 9 I rekl Jákob: Bože otce mého Abrahama, a Bože otce mého Izáka, Hospodine, kterýž jsi mi rekl: Navrat se do zeme své, a k príbuznosti své, a dobre uciním tobe,
\par 10 Menší jsem všech milosrdenství a vší pravdy, kterouž jsi ucinil s služebníkem svým; nebo s holí svou prešel jsem Jordán tento, nyní pak dva houfy mám.
\par 11 Vytrhni mne, prosím, z ruky bratra mého, z ruky Ezau; nebt se ho bojím, aby prijda, nepohubil mne i matky s detmi.
\par 12 Však jsi ty rekl: Dobre uciním tobe, a rozmnožím síme tvé jako písek morský, kterýžto pro množství secten býti nemuže.
\par 13 I zustal tu noci té; a vzal z toho, což bylo pred rukama, poctu bratru svému Ezau:
\par 14 Totiž dve ste koz, a kozlu dvadceti, ovec dve ste, a beranu dvadceti,
\par 15 Velbloudu s mladými jich tridceti, krav ctyridceti, volu deset, oslic dvadceti, a oslátek deset.
\par 16 A porucil je služebníkum svým, každé stádo obzvláštne, a rekl služebníkum svým: Jdete prede mnou, a stádo od stáda at jde opodál.
\par 17 I porucil prednímu, rka: Když se potká s tebou Ezau bratr muj, a optá se tebe, rka Cí jsi? a kam jdeš? a cí jest to stádo pred tebou?
\par 18 Rekneš: Jsem služebníka tvého Jákoba, a dar tento jest poslán pánu mému Ezau; a ted i sám jde za námi.
\par 19 Porucil také druhému i tretímu, a všechnem jdoucím za temi stády, rka: V táž slova mluvte k Ezau, když byste nan trefili.
\par 20 A díte také: Aj, služebník tvuj Jákob za námi; nebo rekl: Ukrotím tvár jeho darem, kterýž jde prede mnou, a potom uzrím tvár jeho; snad prijme tvár mou.
\par 21 A tak predšel dar pred ním; on pak zustal tu noc pri houfu.
\par 22 A vstav ješte v noci, vzal obe ženy své, a dve devky své, a jedenácte synu svých, a prešel pres brod Jabok.
\par 23 Vzav tedy je, prepravil je pres tu reku; prepravil také i vše, což mel.
\par 24 A zustal Jákob sám; a tu zápasil s ním muž až do svitání.
\par 25 A vida, že ho nepremuže, obrazil jej v príhbí vrchní stehna jeho; i vyvinulo se príhbí stehna Jákobova, když zápasil s ním.
\par 26 A rekl: Pust mne, nebt zasvitává. I rekl: Nepustím te, lec mi požehnáš.
\par 27 I rekl jemu: Jaké jest jméno tvé? Odpovedel: Jákob.
\par 28 I dí: Nebude více nazýváno jméno tvé toliko Jákob, ale také Izrael; nebo jsi statecne zacházel s Bohem i lidmi, a premohls.
\par 29 I otázal se Jákob, rka: Oznam, prosím, jméno své. Kterýžto odpovedel: Proc se ptáš na jméno mé? I dal mu tu požehnání.
\par 30 Tedy nazval Jákob jméno místa toho Fanuel; nebo jsem prý videl Boha tvárí v tvár, a zachována jest duše má.
\par 31 I vzešlo mu slunce, když pominul místa toho Fanuel, a kulhal na nohu svou.
\par 32 Protož nejedí synové Izraelští až do tohoto dne té žily krátké, kteráž jest v vrchním príhbí stehna, proto že obrazil príhbí stehna Jákobova na žile krátké.

\chapter{33}

\par 1 Pozdvih pak ocí svých Jákob, uzrel, an Ezau jde, a ctyri sta mužu s ním; i rozdelil syny Líe zvlášt a Ráchele zvlášt, a obou devek zvlášt.
\par 2 Postavil, pravím, devky s syny jejich napred, potom Líu s syny jejími za nimi, Ráchel pak a Jozefa nejzáze.
\par 3 A sám šel pred nimi, a poklonil se až k zemi po sedmkrát, až práve prišel k bratru svému.
\par 4 I bežel Ezau proti nemu, a objal ho; a pad na šíji jeho, líbal ho. I plakali.
\par 5 Pozdvih pak ocí svých, a spatriv ženy s detmi, rekl: Kdo jsou onino s tebou? Odpovedel: Jsou dítky, kteréž Buh dal z milosti služebníku tvému.
\par 6 Mezi tím priblížily se devky s syny svými, i poklonili se.
\par 7 Priblížila se také Lía s syny svými, a poklonili se; a potom priblížil se Jozef a Ráchel, a také se poklonili.
\par 8 I rekl Ezau: K cemu jest všecken ten houf, kterýž jsem potkal? Odpovedel: Abych nalezl milost pred ocima pána mého.
\par 9 Tedy rekl Ezau: Mám hojne, bratre muj; nech sobe, což tvého jest.
\par 10 I rekl Jákob: Nezbranuj se, prosím; jestliže jsem nyní nalezl milost pred ocima tvýma, prijmi dar muj z ruky mé, ponevadž jsem videl tvár tvou, jako bych videl tvár Boží, a laskave jsi mne prijal.
\par 11 Prijmi, prosím, dar muj obetovaný tobe, ponevadž štedre obdaril mne Buh, a mám všeho dosti. Takž ho prinutil, a on vzal.
\par 12 I dí: Pod, a jdeme; já pujdu s tebou.
\par 13 I rekl k nemu Jákob: Ví pán muj, že deti jsou outlí, a mám s sebou ovce i krávy brezí; kteréžto budou-li hnány pres moc jeden den, pomre mi všecken dobytek.
\par 14 Nechžt medle jde pán muj pred služebníkem svým, já pak poznenáhlu se poberu, tak jakž bude moci jíti stádo, kteréž pred sebou mám, a jakž postaciti budou moci deti, až tak dojdu ku pánu svému do Seir.
\par 15 I rekl Ezau: Necht ale pozustavím neco lidu, kterýž mám s sebou. I odpovedel: K cemu to? Necht toliko naleznu milost pred ocima pána svého.
\par 16 Tedy Ezau toho dne navrátil se cestou svou do Seir.
\par 17 Jákob pak bral se do Sochot, a ustavel sobe dum, a dobytku svému zdelal stáje; protož nazval jméno místa toho Sochot.
\par 18 A tak Jákob navracuje se z Pádan Syrské, prišel ve zdraví k mestu Sichem, kteréž jest v zemi Kananejské, a položil se pred mestem.
\par 19 I koupil díl pole toho, na nemž byl rozbil stan svuj, od synu Emora, otce Sichemova, za sto ovec.
\par 20 A postavil tu oltár, kterémužto dal jméno Buh silný, Buh Izraelský.

\chapter{34}

\par 1 Vyšla pak Dína, dcera Líe, kterouž porodila Jákobovi, aby se dívala na dcery té zeme.
\par 2 Kteroužto uzrev Sichem, syn Emora Hevejského, knížete v krajine té, vzal ji, i ležel s ní, a ponížil jí.
\par 3 I pripojila se duše jeho k Díne, dceri Jákobove; a zamilovav devecku, mluvil k srdci jejímu.
\par 4 Mluvil potom Sichem k Emorovi, otci svému, temito slovy: Vezmi mi devecku tuto za manželku.
\par 5 Uslyšev pak Jákob, že poškvrnil Díny dcery jeho, (a synové jeho byli s stádem na poli), mlcel, až oni prišli.
\par 6 Tedy vyšel Emor, otec Sichemuv, k Jákobovi, aby mluvil s ním o to.
\par 7 A v tom synové Jákobovi prišli s pole; a uslyšavše o tom, bolestí naplneni jsou muži ti, a rozhnevali se velmi, proto že hanebnou vec ucinil v Izraeli, ležav se dcerou Jákobovou, cehož ciniti nenáleželo.
\par 8 I mluvil Emor s nimi na tento zpusob: Sichem, syn muj, horí milostí k vaší dceri; prosím, dejte mu ji za manželku.
\par 9 A spríznete se s námi: Dcery své dávejte nám, a naše dcery pojímejte sobe.
\par 10 A bydlete s námi, nebo všecka zeme bude pred vámi; osadte se a obchod vedte v ní, a vládnete jí.
\par 11 Mluvil i Sichem otci jejímu, a bratrím jejím: Necht naleznu milost pred ocima vašima, dám, co mi koli díte.
\par 12 Jmenujte mi veno i dary jak chcete veliké, dám, jak mi koli reknete; jen mi tu devecku dejte za manželku.
\par 13 Odpovídajíce pak synové Jákobovi Sichemovi a Emorovi, otci jeho, lstive mluvili, proto že poškvrnil Díny sestry jich.
\par 14 A rekli jim: Nemužeme uciniti toho, abychom dali sestru svou za muže neobrezaného; nebo to ohavnost jest u nás.
\par 15 Než na tento zpusob vám povolíme: Jestliže se chcete srovnati s námi, aby obrezán byl každý z vás pohlaví mužského:
\par 16 Tedy budeme dávati dcery své vám, a dcery vaše bráti sobe; a budeme bydliti s vámi, a budeme lid jeden.
\par 17 Pakli neuposlechnete nás, abyste se obrezali, vezmeme zase dceru svou a odejdeme.
\par 18 Tedy líbila se rec jejich Emorovi i Sichemovi, synu Emorovu.
\par 19 A nemeškal mládenec uciniti toho; nebo se mu zalíbila dcera Jákobova. A on byl nejvzácnejší ze všech v dome otce svého.
\par 20 I prišel Emor a Sichem, syn jeho, k bráne mesta svého; a mluvili mužum mesta svého, rkouce:
\par 21 Muži tito pokojne se mají k nám, necht tedy bydlí v zemi této, a obchod vedou v ní, (nebo zeme jest dosti široká a prostranná pred nimi;) dcery jejich budeme sobe bráti za manželky, a dcery své budeme dávati jim.
\par 22 Než na tento zpusob privolí nám ti muži k tomu, aby bydlili s námi, a abychom byli jeden lid: Jestliže obrezán bude každý pohlaví mužského mezi námi, tak jako oni jsou obrezáni.
\par 23 Dobytek jejich a statek jejich, i všecka hovada jejich, zdaliž nebudou naše? Toliko v tom jim povolme, a budou bydliti s námi.
\par 24 I uposlechli Emora a Sichema, syna jeho, všickni vycházející branou mesta jeho; a obrezali se všickni pohlaví mužského, což jich koli vycházelo z brány mesta jeho.
\par 25 A tot dne tretího, když oni nejvetší bolest meli, dva synové Jákobovi, Simeon a Léví, bratrí Díny, vzav každý z nich mec svuj, vpadli do mesta smele, a pomordovali všecky pohlaví mužského.
\par 26 Emora také a Sichema, syna jeho, zamordovali mecem, a vzavše Dínu z domu Sichemova, odešli.
\par 27 Potom synové Jákobovi prišedše na zbité, vzebrali mesto, proto že poškvrnili sestry jejich.
\par 28 Stáda jejich, a voly i osly jejich, a což bylo v meste i po poli, pobrali.
\par 29 K tomu i všecko jmení jejich, a všecky malé dítky jejich, a ženy jejich zajali, a vybrali, co kde v domích bylo.
\par 30 Rekl pak Jákob Simeonovi a Léví: Zkormoutili jste mne, a zošklivili jste mne u obyvatelu krajiny této, u Kananejských a Ferezejských, a já jsem s malým poctem lidí. Seberou-li se na mne, zbijí mne, a tak vyhlazen budu já i dum muj.
\par 31 A oni odpovedeli: A což meli jako nevestky zle užívati sestry naší?

\chapter{35}

\par 1 Potom mluvil Buh k Jákobovi: Vstana, vstup do Bethel, a bydli tam; a udelej tam oltár Bohu silnému, kterýž se ukázal tobe, kdyžs utíkal pred Ezau bratrem svým.
\par 2 Tedy rekl Jákob celádce své, a všechnem, kteríž s ním byli: Odvrzte bohy cizí, kteréž máte mezi sebou, a ocistte se, a zmente roucha svá.
\par 3 A vstanouce, vstupme do Bethel, a udelám tam oltár silnému Bohu, kterýž vyslyšel mne v den ssoužení mého, a byl se mnou na ceste, kterouž jsem šel.
\par 4 Tedy dali Jákobovi všecky bohy cizí, kteréž meli, i náušnice, kteréž byly na uších jejich; i zakopal je Jákob pod tím dubem, kterýž byl u Sichem.
\par 5 I brali se odtud. (A byl strach Boží na mestech, kteráž byla vukol nich, a nehonili synu Jákobových.)
\par 6 Tedy prišel Jákob do Luz, kteréž jest v zemi Kananejské, (to již slove Bethel,) on i všecken lid, kterýž byl s ním.
\par 7 I vzdelal tu oltár, a nazval to místo Buh silný Bethel; nebo tu se mu byl zjevil Buh, když utíkal pred bratrem svým.
\par 8 Tehdy umrela Debora, chovacka Rebeky, a pochována jest pod Bethel, pod dubem; i nazval jméno jeho Allon Bachuth.
\par 9 Ukázal se pak opet Buh Jákobovi, když se navracoval z Pádan Syrské, a požehnal mu.
\par 10 I rekl jemu Buh: Jméno tvé jest Jákob. Nebude více nazývano jméno tvé toliko Jákob, ale Izrael také bude jméno tvé. Protož nazval jméno jeho Izrael.
\par 11 Rekl ješte Buh jemu: Já jsem Buh silný všemohoucí; rostiž a množ se; národ, nýbrž množství národu bude z tebe, i králové z bedr tvých vyjdou.
\par 12 A zemi tu, kterouž jsem dal Abrahamovi a Izákovi, tobe ji dám; semeni také tvému po tobe dám tu zemi.
\par 13 I vstoupil od neho Buh z místa, na kterémž mluvil s ním.
\par 14 Jákob pak vyzdvihl znamení pametné na míste tom, na kterémž mluvil s ním, sloup kamenný; a pokropil ho skropením, a svrchu polil jej olejem.
\par 15 A nazval Jákob jméno místa toho, na kterémž mluvil s ním Buh, Bethel.
\par 16 I brali se z Bethel, a bylo již nedaleko do Efraty. I porodila Ráchel, a težkosti trpela rodeci.
\par 17 A když s težkostí rodila, rekla jí baba: Neboj se, nebo také tohoto syna míti budeš.
\par 18 I stalo se, když k smrti pracovala, (nebo umrela), nazvala jméno jeho Ben Oni; ale otec jeho nazval ho Beniaminem.
\par 19 I umrela Ráchel, a pochována jest na ceste k Efrate, jenž jest Betlém.
\par 20 A postavil Jákob znamení pametné nad hrobem jejím; tot jest znamení hrobu Ráchel až do dnešního dne.
\par 21 I odebral se odtud Izrael, a rozbil stan svuj za veží Eder.
\par 22 Stalo se pak také, když bydlil Izrael v té krajine, že Ruben šel, a spal s Bálou, ženinou otce svého; o cemž uslyšel Izrael. Bylo pak synu Jákobových dvanácte.
\par 23 Synové pak Líe: Prvorozený Jákobuv Ruben, potom Simeon, a Léví, a Juda, a Izachar, a Zabulon.
\par 24 Synové Ráchel: Jozef a Beniamin.
\par 25 A synové Bály, devky Ráchel: Dan a Neftalím.
\par 26 A synové Zelfy, devky Líe: Gád a Asser. Tit jsou synové Jákobovi, kteríž mu zrozeni jsou v Pádan Syrské.
\par 27 Tedy prišel Jákob k Izákovi otci svému do Mamre, do mesta Arbe, jenž jest Hebron, kdežto bydlil pohostinu Abraham a Izák.
\par 28 A bylo dnu Izákových sto osmdesáte let.
\par 29 I dokonal Izák, a umrel, a pripojen jest k lidu svému, stár jsa a plný dnu; i pochovali ho Ezau a Jákob, synové jeho.

\chapter{36}

\par 1 Tito jsou pak rodové Ezau, kterýž mel prijmí Edom.
\par 2 Ezau pojal sobe ženy ze dcer Kananejských, Adu, dceru Elona Hetejského, a Olibamu, dceru Anovu, dceru Sebeona Hevejského,
\par 3 A Bazematu, dceru Izmaelovu, sestru Nabajotovu.
\par 4 I porodila Ada Ezauchovi Elifaza, a Bazemat porodila Rahuele.
\par 5 Olibama pak porodila Jehusa, a Jheloma, a Koré. Ti jsou synové Ezau, kteríž se mu zrodili v zemi Kananejské.
\par 6 I pobral Ezau ženy své, i syny své, a dcery své, a všecku celed svou, i dobytek svuj, a všecka hovada svá, i všecko jmení své, kteréhož nabyl v zemi Kananejské, a odebral se do zeme Seir, pred príchodem Jákoba bratra svého.
\par 7 Nebo zboží jejich bylo tak veliké, že nemohli bydliti spolu; aniž ta zeme, v níž oni pohostinu byli, mohla jich snésti, pro dobytky jejich.
\par 8 Protož bydlil Ezau na hore Seir; Ezau pak ten jest Edom.
\par 9 A tak tito jsou rodové Ezau, otce Idumejských, na hore Seir.
\par 10 Tato byla jména synu Ezau: Elifaz, syn Ady, ženy Ezau; Rahuel, syn Bazematy, ženy Ezau.
\par 11 Synové pak Elifazovi byli: Teman, Omar, Sefo, a Gatam a Kenaz.
\par 12 Tamna pak byla ženina Elifaza, syna Ezau. Ta porodila Elifazovi Amalecha. Ti jsou synové Ady, ženy Ezau.
\par 13 Synové pak Rahuelovi tito: Nahat, Zára, Samma a Méza. To byli synové Bazematy, ženy Ezau.
\par 14 A tito byli synové Olibamy, dcery Anovy, dcery Sebeonovy, ženy Ezau, kteráž porodila Ezauchovi Jehusa, Jheloma a Koré.
\par 15 Tito všickni byli knížata synu Ezau. Synové Elifaza prvorozeného Ezau: Kníže Teman, kníže Omar, kníže Sefo, kníže Kenaz,
\par 16 Kníže Koré, kníže Gatam, kníže Amalech. Ta jsou knížata pošlá z Elifaza v zemi Idumejské; ti jsou synové Ady.
\par 17 Tito pak jsou synové Rahuele, syna Ezau: Kníže Nahat, kníže Zára, kníže Samma, kníže Méza. Ta knížata pošla z Rahuele, v zemi Idumejské; to jsou synové Bazematy, ženy Ezau.
\par 18 A synové Olibamy, ženy Ezau, tito jsou: Kníže Jehus, kníže Jhelom, kníže Koré. To jsou knížata z Olibamy, dcery Anovy, ženy Ezau.
\par 19 Tit jsou synové Ezau, a ta knížata jejich; ont jest Edom.
\par 20 Tito pak jsou synové Seir, totiž Horejští, obyvatelé zeme té: Lotan, Sobal, Sebeon a Ana,
\par 21 A Dison, Eser a Dízan. Ta jsou knížata Horejská, synové Seir, v zemi Idumejské.
\par 22 Synové Lotanovi byli: Hori a Hemam; a sestra Lotanova Tamna.
\par 23 Synové pak Sobalovi tito: Alvan, Manáhat, Ebal, Sefo a Onam.
\par 24 A tito synové Sebeonovi: Aia a Ana. To jest ten Ana, kterýž nalezl mezky na poušti, když pásl osly Sebeona otce svého.
\par 25 Tito pak deti Anovi: Dison a Olibama, dcera Anova.
\par 26 A synové Dízanovi tito: Hamdan, Eseban, Jetran a Charan.
\par 27 Synové Eser tito: Balaan, Závan a Achan.
\par 28 Tito synové Dízanovi: Hus a Aran.
\par 29 Tato jsou knížata Horejská: Kníže Lotan, kníže Sobal, kníže Sebeon, kníže Ana,
\par 30 Kníže Dison, kníže Eser, kníže Dízan. To byla knížata Horejská, po knížetstvích svých v zemi Seir.
\par 31 Tito pak byli králové, kteríž kralovali v zemi Idumejské, prvé než kraloval král nad syny Izraelskými.
\par 32 Kraloval tedy v Edom Béla, syn Beoruv, a jméno mesta jeho Denaba.
\par 33 I umrel Béla, a kraloval místo neho Jobab, syn Záre z Bozra.
\par 34 I umrel Jobab, a kraloval na míste jeho Husam z zeme Temanské.
\par 35 Umrel i Husam, a kraloval místo neho Adad, syn Badaduv, kterýž porazil Madianské v krajine Moábské; a jméno mesta jeho Avith.
\par 36 Když pak umrel Adad, kraloval místo neho Semla z Masreka.
\par 37 A po smrti Semlove kraloval na míste jeho Saul z Rohobot od reky.
\par 38 Umrel i Saul, a kraloval místo neho Bálanan, syn Achoboruv.
\par 39 A když i Bálanan umrel, syn Achoboruv, kraloval na míste jeho Adar; a jméno mesta jeho Pahu; jméno pak ženy jeho Mehetabel, dcera Matredy, dcery Mezábovy.
\par 40 Ta jsou jména knížat pošlých z Ezau, po celedech jejich, po místech jejich, vedlé jmen jejich: Kníže Tamna, kníže Alva, kníže Jetet,
\par 41 Kníže Olibama, kníže Ela, kníže Finon.
\par 42 Kníže Kenaz, kníže Teman, kníže Mabsar.
\par 43 Kníže Magdiel, kníže Híram. Tat jsou knížata Idumejská, tak jakž kterí bydlili v zemi dedictví svého. Tot jest ten Ezau, otec Idumejských.

\chapter{37}

\par 1 Jákob pak bydlil v zemi putování otce svého, v zemi Kananejské.
\par 2 Tito jsou príbehové Jákobovi: Jozef, když byl v sedmnácti letech, pásl s bratrími svými dobytek, (a byl mládenecek), s syny Bály a Zelfy, žen otce svého. A oznamoval Jozef zlou povest o nich otci svému.
\par 3 Izrael pak miloval Jozefa nad všecky syny své; nebo v starosti své zplodil jej. A udelal mu sukni promenných barev.
\par 4 A když spatrili bratrí jeho, že ho miluje otec jejich nad všecky bratrí jeho, nenávideli ho, aniž mohli pokojne k nemu promluviti.
\par 5 Mel pak Jozef sen, a vypravoval jej bratrím svým; procež v vetší nenávisti ho meli.
\par 6 Nebo pravil jim: Slyšte, prosím, sen, kterýž jsem mel.
\par 7 Hle, vázali jsme snopy na poli, a aj, povstal snop muj, a stál. Vukol také stáli snopové vaši, a klaneli se snopu mému.
\par 8 Jemužto odpovedeli bratrí jeho: Zdaliž kralovati budeš nad námi, aneb pánem naším budeš? Z té príciny ješte více nenávideli ho pro sny jeho, a pro slova jeho.
\par 9 Potom ješte mel jiný sen, a vypravoval jej bratrím svým, rka: Hle, opet jsem mel sen, a aj, slunce a mesíc, a jedenácte hvezd klanelo mi se.
\par 10 I vypravoval otci svému a bratrím svým. A domlouval mu otec jeho, a rekl jemu: Jakýž jest to sen, kterýž jsi mel? Zdaliž prijdeme, já a matka tvá i bratrí tvoji, abychom se klaneli pred tebou až k zemi?
\par 11 Tedy závideli mu bratrí jeho; ale otec jeho mel pozor na tu vec.
\par 12 Odešli pak bratrí jeho, aby pásli dobytek otce svého v Sichem.
\par 13 A rekl Izrael Jozefovi: Zdaliž nepasou bratrí tvoji v Sichem: Pod, a pošli te k nim. Kterýžto odpovedel: Aj, ted jsem.
\par 14 I rekl jemu: Jdi nyní, zvez, jak se mají bratrí tvoji, a co se deje s dobytkem; a zase mi povíš o tom. A tak poslal ho z údolí Hebron, a on prišel do Sichem.
\par 15 Našel ho pak muž nejaký, an bloudí po poli. I zeptal se ho muž ten, rka: Ceho hledáš?
\par 16 Odpovedel: Bratrí svých hledám; povez mi, prosím, kde oni pasou?
\par 17 I rekl muž ten: Odešli odsud; nebo slyšel jsem je, ani praví: Podme do Dothain. Tedy šel Jozef za bratrími svými, a našel je v Dothain.
\par 18 Kterížto, jakž ho uzreli zdaleka, prvé než k nim došel, ukládali o nem, aby jej zahubili.
\par 19 Nebo rekli jeden druhému: Ej, mistr snu ted jde.
\par 20 Nyní tedy podte, a zabíme jej, a uvržeme ho do nekteré cisterny, a díme: Zver lítá sežrala jej. I uzríme, nac jemu vyjdou snové jeho.
\par 21 A uslyšev to Ruben, aby ho vytrhl z ruky jejich, (nebo rekl: Neodjímejme mu hrdla,)
\par 22 Rekl jim Ruben: Nevylévejte krve. Vrzte jej do této cisterny, kteráž jest na poušti, a nevztahujte ruky na nej. Ale on chtel vysvoboditi ho z ruky jejich, a pomoci mu, aby se navrátil k otci svému.
\par 23 A když prišel Jozef k bratrím svým, strhli s neho sukni jeho, sukni promenných barev, kterouž mel na sobe.
\par 24 A pochopivše, uvrhli jej do cisterny. Cisterna pak ta byla prázdná, v níž nebylo vody.
\par 25 I usadili se, aby jedli chléb. A pozdvihše ocí svých, uzreli, a aj, množství Izmaelitských pricházejících z Galád, kterížto na velbloudích svých nesli vonné veci a kadidlo a mirru do Egypta.
\par 26 I rekl Juda bratrím svým: Jaký zisk míti budeme, zabijeme-li bratra svého, a zatajíme-li krve jeho?
\par 27 Podte, prodejme ho Izmaelitským, a nevztahujme na nej rukou svých, nebo bratr náš, telo naše jest. I uposlechli ho bratrí jeho.
\par 28 Když pak mimo ne jeli muži ti, kupci Madianští, vytáhli a vyvedli Jozefa z té cisterny, a prodali jej Izmaelitským za dvadceti stríbrných. Ti zavedli Jozefa do Egypta.
\par 29 A navrátil se Ruben k cisterne, a aj, již nebylo Jozefa v ní. I roztrhl roucha svá.
\par 30 A navrátiv se k bratrím svým, rekl: Pacholete není, a já kam se mám podíti?
\par 31 Tedy vzali sukni Jozefovu, a zabivše kozla, smocili sukni tu ve krvi.
\par 32 A poslali sukni tu promenných barev, a dali ji donésti k otci svému, aby rekli: Tuto jsme nalezli; pohled nyní, jest-li sukne syna tvého, ci není?
\par 33 A on poznav ji, rekl: Sukne syna mého jest; zver lítá sežrala jej, konecne roztrhán jest Jozef.
\par 34 I roztrhl Jákob roucha svá, a vloživ žíni na bedra svá, zámutek nesl po synu svém za mnoho dní.
\par 35 Sešli se pak všickni synové jeho, a všecky dcery jeho, aby ho tešili. Ale on nedal se potešiti, a rekl: Nýbrž já tak v zámutku sstoupím za synem svým do hrobu. A plakal ho otec jeho.
\par 36 Mezi tím Madianští prodali Jozefa do Egypta Putifarovi, dvoreninu Faraonovu, hejtmanu žoldnéru.

\chapter{38}

\par 1 Stalo se pak v ten cas, že sstupuje Juda od bratrí svých, uchýlil se k muži Odolamitskému, jehož jméno bylo Híra.
\par 2 I uzrel tam Juda dceru muže Kananejského, kterýž sloul Sua; a pojav ji, všel k ní.
\par 3 Kterážto pocala a porodila syna; i nazval jméno jeho Her.
\par 4 A pocavši opet, porodila syna; a nazvala jméno jeho Onan.
\par 5 Porodila pak ješte syna, jehož jméno nazvala Séla. A byl Juda v Chezib, když ona jej porodila.
\par 6 I dal Juda Herovi prvorozenému svému manželku, jménem Támar.
\par 7 A byl Her, prvorozený Juduv, zlý pred ocima Hospodina; i zabil jej Hospodin.
\par 8 Tedy rekl Juda Onanovi: Vejdi k žene bratra svého, a podlé príbuznosti pojmi ji, abys vzbudil síme bratru svému.
\par 9 Vedel pak Onan, že to síme nebude jeho. Protož kdykoli vcházel k žene bratra svého, vypouštel síme na zem, aby nezplodil synu bratru svému.
\par 10 I nelíbilo se Hospodinu to, co delal Onan; protož ho také zabil.
\par 11 Tedy rekl Juda k Támar neveste své: Pobud v dome otce svého vdovou, dokudž nedoroste Séla, syn muj. (Nebo rekl: Aby on také neumrel, jako i bratrí jeho.) I odešla Támar, a bydlila v dome otce svého.
\par 12 A po mnohých dnech umrela dcera Suova, manželka Judova. Kterýžto potešiv se zas, šel do Tamnas k tem, kteríž strihli ovce jeho, a s ním Híra Odolamitský, prítel jeho.
\par 13 I oznámeno jest Támar temito slovy: Aj, tchán tvuj vstupuje do Tamnas, aby strihl ovce své.
\par 14 Tedy ona složivši s sebe šaty své vdovské, zavila se v rouchu, a odela se a sedela na rozcestí, kudy se jde do Tamnas. Nebo videla, že Séla dorostl, a že není dána jemu za manželku.
\par 15 Kteroužto uzrev Juda, za to mel, že jest nevestka; nebo zakryla tvár svou.
\par 16 Protož uchýliv se k ní s cesty, rekl: Dopust medle, at vejdu k tobe. (Nebo nevedel, aby nevesta jeho byla.) I rekla: Co mi dáš, jestliže vejdeš ke mne?
\par 17 Odpovedel: Pošlit kozelce z stáda. Dí ona: Kdybys neco zastavil, dokavadž nepošleš.
\par 18 I rekl: Cožt mám dáti v zástave? Odpovedela ona: Pecetní prsten svuj, a halži svou, a hul svou, kterouž máš v ruce své. I dal jí, a všel k ní; a pocala z neho.
\par 19 Tedy vstavši, odešla, a snavši rouchu svou s sebe, oblékla se v šaty své vdovské.
\par 20 I poslal Juda kozelce po príteli svém Odolamitském, aby vzal zase základ od ženy. A nenalezl jí.
\par 21 I ptal se mužu toho místa, rka: Kde jest ta nevestka, kteráž sedela na rozcestí této cesty? Rekli oni: Nebylote zde nevestky.
\par 22 Tedy navrátiv se k Judovi, rekl: Nenašel jsem jí; a také muži místa toho pravili: Nebylo zde nevestky.
\par 23 I rekl Juda: Necht sobe to má, abychom neuvedli se v lehkost. Ját jsem poslal toho kozelce, ale tys jí nenalezl.
\par 24 I stalo se okolo trí mesícu, že oznámili Judovi, rkouce: Dopustila se smilství Támar nevesta tvá, a jest již i tehotná z smilství. I rekl Juda: Vyvedte ji, aby byla upálena.
\par 25 Když pak vedena byla, poslala k tchánu svému, rkuci: Z muže, jehož tyto veci jsou, tehotná jsem. A pri tom rekla: Pohled, prosím, cí jsou tyto veci, pecetní prsten, halže a hul tato?
\par 26 Tedy pohledev na to Juda, rekl: Spravedlivejšít jest než já, ponevadž jsem nedal jí Sélovi synu svému. A více jí nepoznával.
\par 27 Stalo se pak, že když prišel cas porodu jejího, aj, bliženci v živote jejím.
\par 28 A když rodila, jeden z nich vyskytl ruku, kteroužto chytivši baba, uvázala na ni cervenou nitku, rkuci: Ten vyjde prvé.
\par 29 Když pak vtáhl ruku svou zase, hle, vyšel bratr jeho. I rekla: Jak jsi protrhl? Tvét jest protržení. I nazváno jest jméno jeho Fáres.
\par 30 A potom vyšel bratr jeho, kterýž mel na ruce nitku cervenou. I nazváno jest jméno jeho Zára.

\chapter{39}

\par 1 Jozef pak doveden byl do Egypta; a koupil ho Putifar, dvorenin Faraonuv, nejvyšší nad drabanty, muž Egyptský, od Izmaelitských, kteríž ho tam dovedli.
\par 2 Byl pak Hospodin s Jozefem, a všecko se mu štastne vedlo, a bydlil v dome pána svého toho Egyptského.
\par 3 A videl pán jeho, že Hospodin byl s ním, a že všecko, což cinil, Hospodin k prospechu privedl v rukou jeho.
\par 4 Tedy nalezl Jozef milost pred ocima jeho, a sloužil mu. I predstavil ho domu svému, a všecko, což mel, dal v ruku jeho.
\par 5 A hned, jakž ustanovil ho nad domem svým, a nade vším, což mel, požehnal Hospodin domu Egyptského toho pro Jozefa. A bylo požehnání Hospodinovo na všech vecech, kteréž mel doma i na poli.
\par 6 Všech tedy vecí, kteréž mel, zanechal v rukou Jozefových; aniž o cem, tak jako on, vedel, jediné o chlebe, kterýž jedl. Byl pak Jozef ušlechtilé postavy a krásného vzezrení.
\par 7 I stalo se potom, že vzhlédala žena pána jeho ocima svýma na Jozefa, a rekla: Spi se mnou.
\par 8 Kterýžto odpíraje, rekl žene pána svého: Aj, pán muj neví tak jako já, co jest v dome, a všecko, což má, dal v ruce mé.
\par 9 Není žádného prednejšího nade mne v dome tomto, aniž co vynal z správy mé, krome tebe, jelikož jsi ty manželka jeho. Jak bych tedy ucinil takovou nešlechetnost, a hrešil i proti Bohu?
\par 10 A když mluvila ona Jozefovi den po dni, nepovolil jí, aby spal s ní, ani aby býval s ní.
\par 11 Tedy dne jednoho, když prišel do domu k práci své, a nebylo tu žádného z domácích v dome,
\par 12 Chytila jej ona za roucho jeho, rkuci: Lež se mnou. On pak nechav roucha svého v rukou jejích, utekl, a vyšel ven.
\par 13 A ona viduci, že nechal roucha svého v rukou jejích a vybehl ven,
\par 14 Svolala domácí své, a rekla k nim takto: Pohledte, privedl nám muže Hebrejského, kterýž by mel posmech z nás; nebo prišel ke mne, aby ležel se mnou; i kricela jsem hlasem velikým.
\par 15 A když uslyšel, že jsem hlasu svého pozdvihla a kricela, nechav roucha svého u mne, utekl a vyšel ven.
\par 16 Tedy schovala roucho jeho u sebe, až prišel pán jeho do domu svého.
\par 17 K nemuž mluvila v tato slova, rkuci: Prišel ke mne služebník ten Hebrejský, kteréhožs privedl nám, aby mi lehkost ucinil.
\par 18 A když jsem hlasu svého pozdvihla a kricela, tedy nechal roucha svého u mne, a utekl ven.
\par 19 I stalo se, že, když uslyšel pán jeho slova ženy své, kteráž mluvila mu, praveci: Tak mi ucinil služebník tvuj, rozhneval se velmi.
\par 20 Protož vzal ho pán jeho, a dal jej do veže žalárné, v to místo, kdež veznové královští sedeli; i byl tam v žalári.
\par 21 Byl pak Hospodin s Jozefem, a naklonil se k nemu milosrdenstvím; a dal jemu milost u vládare nad žalárem.
\par 22 I dal vládar žaláre v moc Jozefovi všecky vezne, kteríž byli v veži žalárné; a cožkoli tam ciniti meli, on to spravoval.
\par 23 Aniž vládar žaláre k cemu dohlídal,což jemu sveril; proto že Hospodin byl s ním, a což on cinil, Hospodin tomu prospech dával.

\chapter{40}

\par 1 Stalo se potom, že šenkýr krále Egyptského a pekar provinili proti pánu svému, králi Egyptskému.
\par 2 I rozhneval se Farao na oba úredníky své, na vládare nad šenkýri, a na vládare nad pekari.
\par 3 A dal je do vezení v dome nejvyššího nad drabanty, do veže žalárné, v místo, v nemž Jozef veznem byl.
\par 4 I postavil jim nejvyšší nad drabanty Jozefa k službe; a byli drahne dní u vezení.
\par 5 I meli sen oba dva, každý z nich sen svuj noci jedné, každý podlé vyložení sna svého, šenkýr i pekar krále Egyptského, kteríž sedeli v veži.
\par 6 Tedy prišel k nim Jozef ráno, a hledel na ne; a aj, byli smutní.
\par 7 I optal se tech úredníku Faraonových, kteríž s ním byli v vezení v dome pána jeho, rka: Proc jsou dnes tvári vaše smutnejší?
\par 8 Kterížto odpovedeli jemu: Meli jsme sen, a nemáme, kdo by jej vyložil. I rekl jim Jozef: Zdaliž Boží nejsou výkladové? Pravte mi medle.
\par 9 Tedy správce nad šenkýri vypravoval sen svuj Jozefovi, a rekl jemu: Zdálo se mi ve snách, že jsem videl pred sebou vinný kmen,
\par 10 A na tom kmenu tri ratolesti; a ten kmen jako by pupence pouštel, a vycházel kvet jeho, až k sezrání prišli hroznové jeho.
\par 11 A já maje koflík Faraonuv v ruce své, bral jsem hrozny, a vytlacoval je do koflíka Faraonova, a podával jsem koflíka Faraonovi do rukou.
\par 12 I rekl jemu Jozef: Toto jest vyložení jeho: Ti tri révové jsou tri dnové.
\par 13 Po trech dnech povýší Farao hlavy tvé, a k úradu tvému te navrátí; i budeš podávati koflíka Faraonova do ruky jeho podlé obyceje prvního, když jsi byl šenkýrem jeho.
\par 14 Ale mejž mne v své pameti, kdyžt se dobre povede; a ucin, prosím, se mnou to milosrdenství, abys zmínku ucinil o mne pred Faraonem, a vysvobodil mne z domu tohoto.
\par 15 Nebo kradmo jsem vzat z zeme Židovské; a zde jsem niceho neucinil, procež by mne do tohoto vezení dali.
\par 16 Vida pak správce nad pekari, že dobre vyložil, rekl Jozefovi: Mne také zdálo se ve snách, ano tri košové pletení na hlave mé.
\par 17 A v koši vrchním byli všelijací pokrmové Faraonovi dílem pekarským strojení, a ptáci jedli je z koše nad hlavou mou.
\par 18 I odpovedel Jozef a rekl: Toto jest vyložení jeho: Tri košové jsou tri dnové.
\par 19 Po trech dnech odejme tobe Farao hlavu tvou, a obesí te na dreve; i budou jísti ptáci maso tvé s tebe.
\par 20 Tedy stalo se v den tretí, v den pamatný narození Faraonova, že ucinil hody všechnem služebníkum svým; i pocítal hlavu vládare nad šenky, i hlavu vládare nad pekari, mezi služebníky svými.
\par 21 A navrátil nejvyššího nad šenky k místu jeho, aby podával koflíku Faraonovi do ruky.
\par 22 Vládare pak nad pekari obesil, tak jakž jim byl sen vyložil Jozef.
\par 23 A nezpomenul správce nad šenky na Jozefa, ale zapomenul na nej.

\chapter{41}

\par 1 Stalo se pak po dvou letech, mel Farao sen. Zdálo mu se, že stál nad potokem.
\par 2 A aj, z toho potoku vycházelo sedm krav, pekných na pohledení a tlustých, kteréžto pásly se na mokrinách.
\par 3 A aj, sedm krav jiných vycházelo za nimi z potoku, šeredných na pohledení a hubených, kteréžto stály podlé onech krav pri brehu potoka.
\par 4 A ty krávy na pohledení šeredné a hubené sežraly onech sedm krav na pohledení pekných a tlustých. I procítil Farao.
\par 5 A když usnul zase, zdálo se jemu podruhé. A aj, sedm klasu vyrostlo z stébla jednoho, plných a pekných.
\par 6 A aj, sedm klasu tenkých a východním vetrem usvadlých vzcházelo za nimi.
\par 7 A ti klasové tencí pohltili sedm onech klasu zdarilých a plných. I procítiv Farao, a aj, byl sen.
\par 8 Když pak bylo ráno, zkormoucena byla mysl jeho; a poslav, svolal všecky hadace Egyptské, a všecky mudrce jejich. I vypravoval jim Farao sny své; a nebylo žádného, kdo by je vyložil Faraonovi.
\par 9 Tedy mluvil nejvyšší šenk Faraonovi takto: Na provinení své rozpomínám se dnes.
\par 10 Farao rozhnevav se na služebníky své, dal mne byl do vezení v dome nejvyššího nad drabanty, mne a správce nad pekari.
\par 11 Meli jsme pak sen jedné noci, on i já, jeden každý podlé vyložení snu svého.
\par 12 A byl tam s námi mládenec Hebrejský, služebník nejvyššího nad drabanty, jemuž když jsme vypravovali, vykládal nám sny naše; jednomu každému podlé snu jeho vykládal.
\par 13 A stalo se, že jakž vykládal nám, tak bylo: Já jsem navrácen k úradu svému, a on obešen.
\par 14 Tedy poslav Farao, povolal Jozefa, a rychle vypustili ho z žaláre. Kterýžto oholiv se, a zmeniv roucho své, prišel k Faraonovi.
\par 15 I rekl Farao Jozefovi: Mel jsem sen, a není, kdo by jej vyložil; o tobe pak slyšel jsem to, že když uslyšíš sen, umíš jej vyložiti.
\par 16 Odpovedel Jozef Faraonovi, rka: Není to má vec; Buh oznámí štastné veci Faraonovi.
\par 17 Tedy rekl Farao Jozefovi: Zdálo mi se ve snách, že jsem stál na brehu potoka.
\par 18 A aj, z potoka toho vystupovalo sedm krav tlustých a pekných, kteréžto pásly se na mokrinách.
\par 19 A aj, sedm jiných krav vystupovalo za nimi churavých a šeredných velmi a hubených; nevidel jsem tak šeredných ve vší zemi Egyptské.
\par 20 A sežraly krávy ty hubené a šeredné sedm krav prvnejších tlustých.
\par 21 A ac dostaly se do bricha jejich, však nebylo znáti, by se dostaly v streva jejich; nebo na pohledení byly mrzké, jako i pred tím. I procítil jsem.
\par 22 Videl jsem také ve snách, ano sedm klasu vyrostlo z stébla jednoho plných a pekných.
\par 23 A aj, sedm klasu drobných, tenkých a východním vetrem usvadlých vycházelo za nimi.
\par 24 I pohltili klasové ti drobní sedm klasu pekných. Což když jsem vypravoval hadacum, nebyl, kdo by mi vyložil.
\par 25 Odpovedel Jozef Faraonovi: Sen Faraonuv jednostejný jest. Což Buh ciniti bude, to ukázal Faraonovi.
\par 26 Sedm krav pekných jest sedm let, a sedm klasu pekných tolikéž jest sedm let; sen jest jednostejný.
\par 27 Sedm pak hubených krav a šeredných, vystupujících za nimi, sedm let jest; a sedm klasu drobných a vetrem východním usvadlých bude sedm let hladu.
\par 28 Tot jest, což jsem mluvil Faraonovi: Což Buh ciniti bude, ukazuje Faraonovi.
\par 29 Aj, sedm let nastane, v nichž hojnost veliká bude ve vší zemi Egyptské.
\par 30 A po nich nastane sedm let hladu,v nichž v zapomenutí prijde všecka ta hojnost v zemi Egyptské; a zhubí hlad zemi.
\par 31 Aniž poznána bude hojnost ta v zemi, pro hlad, kterýž prijde potom; nebo velmi veliký bude.
\par 32 Že pak opetován jest sen Faraonovi podvakrát, znamená, že jistá vec jest od Boha, a že tím spíše Buh vykoná to.
\par 33 Protož nyní at vyhledá Farao muže opatrného a moudrého, kteréhož by ustanovil nad zemí Egyptskou.
\par 34 To at uciní Farao, a postaví úredníky nad zemí, a bére pátý díl z úrod zeme Egyptské, po sedm let hojných.
\par 35 At shromáždí všeliké potravy tech úrodných let nastávajících, a sklidí obilí k ruce Faraonovi; a potravy v mestech at se chovají pilne.
\par 36 A budou pokrmové ti za poklad zemi této k sedmi letum hladu, kteráž budou v zemi Egyptské, aby nebyla zkažena zeme tato hladem.
\par 37 I líbila se rec ta Faraonovi i všechnem služebníkum jeho.
\par 38 Tedy rekl Farao služebníkum svým: Najdeme-liž podobného tomuto muži, v nemž by byl Duch Boží?
\par 39 Jozefovi pak rekl: Ponevadž Buh dal znáti tobe všecko toto, nenít žádného tak rozumného a moudrého, jako ty jsi.
\par 40 Ty budeš nad domem mým, a líbati bude tvár tvou všecken lid muj; stolicí toliko královskou vyšší nad tebe budu.
\par 41 Rekl také Farao Jozefovi: Aj, ustanovil jsem te nade vší zemi Egyptskou.
\par 42 A snav Farao prsten svuj s ruky své, dal jej na ruku Jozefovu, a oblékl ho v roucho kmentové, a vložil zlatý retez na hrdlo jeho.
\par 43 A dal ho voziti na svém druhém voze, a volali pred ním: Klanejte se! I ustanovil ho nade vší zemi Egyptskou.
\par 44 A rekl Farao Jozefovi: Já jsem Farao, a bez dopuštení tvého nepozdvihne žádný ruky své ani nohy své ve vší zemi Egyptské.
\par 45 A dal Farao jméno Jozefovi Safenat Paneach, a dal mu Asenat dceru Putifera, knížete On, za manželku. I vyšel Jozef na zemi Egyptskou.
\par 46 (Jozef pak byl ve tridcíti letech, když stál pred Faraonem králem Egyptským.) A vyšed od tvári Faraonovy, projel všecku zemi Egyptskou.
\par 47 A vydala zeme po sedm let úrodných obilí hojnost.
\par 48 I nahromáždil všelijakých potrav v tech sedmi letech hojných v zemi Egyptské, a složil potravu tu v mestech; úrody polní jednoho každého mesta, kteréž byly okolo neho, složil v nem.
\par 49 A tak nahromáždil Jozef obilí velmi mnoho, jako jest písku morského, tak že prestali pocítati; nebo mu nebylo poctu.
\par 50 Jozefovi pak narodili se dva synové, prvé než prišel rok hladu, kteréž mu porodila Asenat, dcera Putifera, knížete On.
\par 51 A nazval Jozef jméno prvorozeného Manasses, rka: Nebo zpusobil to Buh, abych zapomenul na všecky práce své, a na všecken dum otce svého.
\par 52 Jméno pak druhého nazval Efraim, rka: Nebo dal mi Buh zrust v zemi trápení mého.
\par 53 Tedy pominulo sedm let hojných v zemi Egyptské;
\par 54 A pocalo sedm let hladu pricházeti, jakž byl predpovedel Jozef. I byl hlad po všech krajinách, ale po vší zemi Egyptské byl chléb.
\par 55 Potom také nedostatek trpela všecka zeme Egyptská, a volal lid k Faraonovi o chléb. I rekl Farao všechnem Egyptským: Jdete k Jozefovi, což vám rozkáže, uciníte.
\par 56 A byl hlad na tvári vší zeme. Tedy otevrel Jozef všecky obilnice, v nichž obilí bylo, a prodával Egyptským; nebo rozmohl se hlad v zemi Egyptské.
\par 57 A všickni obyvatelé zeme pricházeli do Egypta k Jozefovi, aby kupovali; nebo rozmohl se byl hlad po vší zemi.

\chapter{42}

\par 1 Vida pak Jákob, že by potrava byla v Egypte, rekl synum svým: Co hledíte jeden na druhého?
\par 2 I mluvil jim: Aj, slyšel jsem, že mají potravu v Egypte; jdete tam, a kupte nám odtud, abychom živi byli a nezemreli.
\par 3 Tedy šlo deset bratru Jozefových, aby nakoupili obilí v Egypte.
\par 4 Ale Beniamina, bratra Jozefova, neposlal Jákob s bratrími jeho, nebo rekl: Aby se mu tam neco zlého neprihodilo.
\par 5 I šli synové Izraelovi spolu s jinými, aby kupovali; nebo byl hlad v zemi Kananejské.
\par 6 Jozef pak byl nejvyšší správce v zemi té; on prodával obilí všemu lidu zeme. Tedy prišli bratrí Jozefovi, a skláneli se pred ním tvárí až k zemi.
\par 7 A uzrev Jozef bratrí své, poznal je; a ukázal se k nim jako cizí, a tvrde mluvil k nim,rka jim: Odkud jste prišli? I odpovedeli: Z zeme Kananejské, abychom nakoupili potrav.
\par 8 Poznal, pravím, Jozef bratrí své, ale oni nepoznali ho.
\par 9 Tedy zpomenul Jozef na sny, kteréž mel o nich, a rekl jim: Špehéri jste, a prišli jste, abyste shlédli nepevná místa zeme.
\par 10 Kterížto odpovedeli jemu: Nikoli, pane muj, ale služebníci tvoji prišli, aby nakoupili pokrmu.
\par 11 Všickni my synové jednoho muže jsme, uprímí jsme; nikdyt jsou nebyli služebníci tvoji špehéri.
\par 12 Jimž zase rekl: Není tak, ale prišli jste, abyste shlédli nepevná místa zeme.
\par 13 Odpovedeli oni: Dvanácte nás bratrí služebníku tvých bylo, synu muže jednoho v zemi Kananejské; a aj, nejmladší s otcem naším nyní jest doma, a jednoho není.
\par 14 I rekl jim Jozef: Tot jest, což jsem mluvil vám, když jsem rekl: Špehéri jste.
\par 15 Touto vecí zkušeni budete: Živt jest Farao, že nevyjdete odsud, až prijde sem bratr váš mladší.
\par 16 Vyšlete z sebe jednoho, at vezma, privede bratra vašeho; vy pak u vezení zustante, a zkušena budou vaše slova, pravdu-li jste mluvili. Pakli nic, živt jest Farao, že jste špehéri.
\par 17 Tedy dal je všecky spolu do vezení za tri dni.
\par 18 Tretího pak dne rekl jim Jozef: Toto ucinte, abyste živi byli; nebot já se bojím Boha.
\par 19 Jste-li šlechetní muži, jeden bratr váš at jest ukován v žalári, v nemž jste byli; vy pak jdete, a odneste obilí k zapuzení hladu domu vašich.
\par 20 Bratra pak svého mladšího privedete ke mne; a pravdomluvná prokázána budou vaše slova, a nezemrete. Tedy ucinili tak.
\par 21 I mluvil jeden k druhému: Jiste provinili jsme proti bratru svému. Nebo videli jsme ssoužení duše jeho, když nás pokorne prosil, a nevyslyšeli jsme ho; protož prišlo na nás ssoužení toto.
\par 22 Odpovedel pak jim Ruben, rka: Zdaliž jsem tehdy vám nepravil temito slovy: Nehrešte proti pacholeti. Ale neposlechli jste; procež také krve jeho, hle, vyhledává se.
\par 23 A nevedeli oni, že by rozumel Jozef; nebo skrze tlumace mluvil jim.
\par 24 A odvrátiv se od nich, plakal. Potom navrátiv se k nim, mluvil s nimi, a vzav Simeona z nich, svázal ho pred ocima jejich.
\par 25 Prikázal pak Jozef, aby naplneni byli pytlové jejich obilím, a navráceny peníze jejich jednomu každému do pytle jeho, a aby dána jim byla potrava na cestu. I stalo se tak.
\par 26 A vloživše obilí svá na osly své, odešli odtud.
\par 27 A rozvázav jeden z nich pytel svuj, aby dal obrok oslu svému v hospode, uzrel peníze své, kteréž byly na vrchu v pytli jeho.
\par 28 I rekl bratrím svým: Navráceny jsou mi peníze mé, a aj, jsou v pytli mém. Tedy užasli se, a predešeni jsouce, mluvili jeden k druhému: Což nám to ucinil Buh?
\par 29 Navrátivše se pak k Jákobovi otci svému do zeme Kananejské, vypravovali jemu všecko, co se jim prihodilo, pravíce:
\par 30 Muž ten, pán zeme, mluvil k nám tvrde, a dal nás do vezení, jako špehére zeme.
\par 31 A rekli jsme jemu: Uprímí jsme, nikdy jsme nebyli špehéri.
\par 32 Dvanácte bylo nás bratrí, synu otce našeho, z nichž jednoho není, a mladší nyní jest s otcem naším v zemi Kananejské.
\par 33 I rekl nám muž ten, pán zeme té: Po tomto poznám, že uprímí jste: Bratra vašeho jednoho zanechte u mne, a obilí k zapuzení hladu od domu vašich vezmouce, odejdete.
\par 34 A privedte bratra svého mladšího ke mne, abych poznal, že nejste špehéri, ale uprímí; tehdy bratra vašeho vrátím vám, a budete moci v zemi této obchod vésti.
\par 35 I stalo se, že, když vyprazdnovali pytle své, a aj, jeden každý mel uzlík penez svých v pytli svém. Vidouce pak oni i otec jejich uzlíky penez svých, báli se.
\par 36 I rekl jim Jákob otec jejich: Mne jste zbavili synu: Jozefa není, Simeona nemám, a Beniamina vezmete. Na mnet jsou se tyto všecky veci svalily.
\par 37 Tedy rekl Ruben otci svému temito slovy: Dva syny mé zabí, jestliže ho neprivedu zase k tobe; poruc ho v ruce mé, a já zase privedu ho k tobe.
\par 38 I rekl: Nesstoupít syn muj s vámi. Nebo bratr jeho umrel, a on sám pozustal; a prihodilo-li by se mu co zlého na té ceste, kterouž pujdete, uvedli byste šediny mé s bolestí do hrobu.

\chapter{43}

\par 1 Byl pak hlad veliký v krajine té.
\par 2 I stalo se, když vytrávili obilí, kteréž prinesli z Egypta, že rekl k nim otec jejich: Jdete zase, a nakupte nám neco potravy.
\par 3 I mluvil k nemu Juda temito slovy: Velice se zarekl muž ten, rka: Neuzríte tvári mé, nebude-li bratr váš s vámi.
\par 4 Jestliže pošleš bratra našeho s námi, pujdeme a nakoupíme tobe potravy;
\par 5 Pakli nepošleš, nepujdeme. Nebo povedel nám muž ten: Neuzríte tvári mé, nebude-li bratr váš s vámi.
\par 6 I rekl Izrael: Proc jste mi tak zle ucinili, oznámivše muži tomu, že máte ješte bratra?
\par 7 Odpovedeli: Pilne vyptával se muž ten na nás, i na rod náš, mluve: Jest-li živ ješte otec váš? Máte-li bratra? A dali jsme mu zprávu na ta slova. Zdaž jsme to jak vedeti mohli, že dí: Privedte bratra svého?
\par 8 I rekl Juda Izraelovi, otci svému: Pošli to pachole se mnou, a vstanouce, pujdeme, abychom živi byli, a nezemreli, i my, i ty, i maliccí naši.
\par 9 Já slibuji za nej; z ruky mé vyhledávej ho. Jestliže neprivedu ho k tobe, a nepostavím ho pred tebou, vinen budu hríchem tobe po všecky dny.
\par 10 A kdybychom byli neprodlévali, jiste již bychom se byli dvakrát vrátili.
\par 11 I rekl jim Izrael otec jejich: Jestližet tak býti musí, ucintež toto: Naberte nejvzácnejších užitku zeme do nádob svých, a doneste muži tomu dar, neco kadidla, a trochu strdi, a vonných vecí a mirry, daktylu a mandlu.
\par 12 Peníze také dvoje vezmete v ruce své, a peníze vložené na vrch do pytlu vašich zase doneste v rukou svých; snad z omýlení to prišlo.
\par 13 Bratra svého také vezmete, a vstanouce, jdete zase k muži tomu.
\par 14 A Buh silný všemohoucí dejž vám najíti milost pred mužem tím, at propustí vám onoho bratra vašeho i tohoto Beniamina. Ját pak zbaven jsa synu, jako osirelý budu.
\par 15 Tedy vzali muži ti dar ten, a dvoje peníze vzali v ruce své, a Beniamina; a vstavše, sstoupili do Egypta, a postavili se pred Jozefem.
\par 16 Vida pak Jozef Beniamina s nimi, rekl tomu, kterýž spravoval dum jeho: Uved tyto muže do domu, a zabí hovado a priprav; nebo se mnou jísti budou muži ti o poledni.
\par 17 I ucinil muž ten, jakž rozkázal Jozef, a uvedl ty lidi do domu Jozefova.
\par 18 Báli se pak muži ti, když uvedeni byli do domu Jozefova, a rekli: Pro ty peníze, kteréž prvé vloženy byly do pytlu našich, sem uvedeni jsme, aby obvine, oboril se na nás, a vzal nás za služebníky i osly naše.
\par 19 A pristoupivše k muži tomu, kterýž spravoval v dome Jozefove, mluvili k nemu ve dverích domu,
\par 20 A rekli: Slyš mne, pane muj. Prišli jsme byli ponejprvé kupovati potrav.
\par 21 I prihodilo se, když jsme do hospody prišli, a rozvazovali pytle své, a aj, peníze jednoho každého byly svrchu v pytli jeho, peníze naše podlé váhy své; a prinesli jsme je zase v rukou svých.
\par 22 Jiné také peníze prinesli jsme v rukou svých, abychom nakoupili potravy; nevíme, kdo jest zase vložil peníze naše do pytlu našich.
\par 23 A on odpovedel: Mejte o to pokoj, nebojte se. Buh váš, a Buh otce vašeho dal vám poklad do pytlu vašich; penízet jsem vaše já prijal. I vyvedl k nim Simeona.
\par 24 Uved tedy muž ten lidi ty do domu Jozefova, dal jim vody, aby umyli nohy své, dal také obrok oslum jejich.
\par 25 Mezi tím pripravili dar ten, dokudž neprišel Jozef v poledne; nebo slyšeli, že by tu meli jísti chléb.
\par 26 Tedy prišel Jozef domu. I prinesli mu dar, kterýž meli v rukou svých, a klaneli se jemu až k zemi.
\par 27 I ptal se jich, jak se mají, a rekl: Zdráv-liž jest otec váš starý, o nemž jste pravili? Živ-li jest ješte?
\par 28 Kterížto odpovedeli: Zdráv jest služebník tvuj otec náš, a ješte živ jest. A sklánejíce hlavy, poklonu mu cinili.
\par 29 Pozdvih pak ocí svých, videl Beniamina bratra svého, syna matky své, a rekl: Tento-li jest bratr váš mladší, o nemž jste mi pravili? I rekl: Uciniž Buh milost s tebou, synu muj!
\par 30 Tedy pospíšil Jozef, (nebo pohnula se streva jeho nad bratrem jeho,) a hledal, kde by mohl plakati; a všed do pokoje, plakal tam.
\par 31 Potom umyv tvár svou, vyšel zase, a zdržoval se, a rekl: Kladte chléb.
\par 32 I kladli jemu zvlášte, a jim obzvlášte, Egyptským také, kteríž s ním jídali, obzvláštne; nebo nemohou Egyptští jísti s Židy chleba, proto že to ohavnost jest Egyptským.
\par 33 Tedy sedeli proti nemu, prvorozený podlé prvorozenství svého, a mladší podlé mladšího veku svého. I divili se muži ti vespolek.
\par 34 A bera jídlo pred sebou, podával jim; Beniaminovi pak dostalo se petkrát více než jiným. I hodovali a hojne se s ním napili.

\chapter{44}

\par 1 Rozkázal pak tomu, kterýž spravoval dum jeho, rka: Napln pytle mužu tech potravou, co by jen unésti mohli; a peníze každého polož zas do pytle jeho na vrch.
\par 2 A koflík muj, koflík stríbrný, vlož na vrch do pytle mladšího s penezi jeho za obilí. I ucinil podlé reci Jozefovy, kterouž mluvil.
\par 3 Ráno pak propušteni jsou ti muži, oni i oslové jejich.
\par 4 A když vyšli z mesta, a nedaleko ješte byli, rekl Jozef správci domu svého: Vstan, hon muže ty, a dohone se jich, mluv k nim: Procež jste se odplatili zlým za dobré?
\par 5 Zdaliž to není ten koflík, z kteréhož píjí pán muj? A z tohot on jistým zkušením pozná, jací jste vy. Zle jste ucinili, co jste ucinili.
\par 6 Tedy dohoniv se jich, mluvil jim slova ta.
\par 7 Kterížto odpovedeli jemu: Proc mluví pán muj taková slova? Odstup od služebníku tvých, aby co takového ucinili.
\par 8 A my ty peníze, kteréž jsme našli na vrchu v pytlích svých, prinesli jsme tobe zase z zeme Kananejské; jakž bychom tedy krásti meli z domu pána tvého stríbro neb zlato?
\par 9 U koho z služebníku tvých nalezeno bude, nechžt umre ten; a my také budeme pána tvého služebníci.
\par 10 I rekl: Nu dobre, necht jest podlé reci vaší. U koho se nalezne, ten bude mým služebníkem, a vy budete bez viny.
\par 11 Protož rychle každý složil pytel svuj na zem, a rozvázal každý pytel svuj.
\par 12 I prehledával, od staršího pocal, a na mladším prestal; i nalezen jest koflík v pytli Beniaminovu.
\par 13 Tedy oni roztrhše roucha svá, vložil každý bríme na osla svého, a vrátili se do mesta.
\par 14 I prišel Juda s bratrími svými do domu Jozefova, (on pak ješte tam byl,) a padli pred ním na zemi.
\par 15 I dí jim Jozef: Jakýž jest to skutek, který jste ucinili? Zdaž nevíte, že takový muž, jako jsem já, umí poznati?
\par 16 Tedy rekl Juda: Což díme pánu svému? co mluviti budeme? a cím se ospravedlníme? Buht jest našel nepravost služebníku tvých. Aj, služebníci jsme pána svého, i my i ten, u nehož nalezen jest koflík.
\par 17 Odpovedel Jozef: Odstup ode mne, abych to ucinil. Muž, u nehož nalezen jest koflík, ten bude mým služebníkem; vy pak jdete u pokoji k otci svému.
\par 18 I pristoupil k nemu Juda a rekl: Slyš mne, pane muj. Prosím, nechažt promluví služebník tvuj slovo v uši pána svého, a nehnevej se na služebníka svého; nebo jsi ty jako sám Farao.
\par 19 Pán muj ptal se služebníku svých, rka: Máte-li otce, neb bratra?
\par 20 A odpovedeli jsme pánu mému: Máme otce starého, a pachole v starosti jeho zplozené malé, jehož bratr umrel, a on sám pozustal po materi své, a otec jeho miluje jej.
\par 21 I rekl jsi služebníkum svým: Privedte ho ke mne, a pohledím na nej.
\par 22 A rekli jsme pánu mému: Nemužet pachole opustiti otce svého; nebo opustí-li otce svého, on umre.
\par 23 Ty pak rekl jsi služebníkum svým: Neprijde-li bratr váš mladší s vámi, nepokoušejte se více videti tvári mé.
\par 24 I stalo se, když jsme se vrátili k služebníku tvému, otci mému, a jemu vypravovali slova pána svého,
\par 25 Že rekl otec náš: Jdete zase, nakupte nám neco potravy.
\par 26 Odpovedeli jsme: Nemužeme jíti,než bude-li bratr náš mladší s námi, tedy pujdeme; nebo bychom nemohli videti tvári toho muže, nebyl-li by bratr náš nejmladší s námi.
\par 27 I rekl nám služebník tvuj, otec muj: Vy víte, že dva toliko syny porodila mi žena má.
\par 28 A vyšel jeden ode mne, o nemž jsem pravil: Jiste roztrhán jest, a nevidel jsem ho dosavad.
\par 29 Vezmete-li i tohoto ode mne, a prišlo by na nej neco zlého, tedy uvedete šediny mé s trápením do hrobu.
\par 30 Tak tedy, když prijdu k služebníku tvému, otci svému, a pacholete nebude s námi, (ješto duše jeho spojena jest s duší tohoto):
\par 31 Prijde na to, když uzrí, že pacholete není, umre; a uvedou služebníci tvoji šediny služebníka tvého, otce svého, s žalostí do hrobu.
\par 32 Nebo služebník tvuj slíbil za pachole, abych je vzal od otce svého, rka: Jestliže ho neprivedu zase k tobe, tedy vinen budu hríchem otci mému po všecky dny.
\par 33 Protož nyní necht zustane, prosím, služebník tvuj místo pacholete tohoto za služebníka pánu svému, a pachole at vstoupí s bratry svými.
\par 34 Nebo jak bych já vstoupil k otci svému, kdyby tohoto pacholete nebylo se mnou? Lec bych chtel videti trápení, kteréž by prišlo na otce mého.

\chapter{45}

\par 1 Jozef pak nemoha se déle zdržeti, prede všemi prístojícími zvolal: Kažte všechnem ven! I nezustal žádný s nimi, když se známil Jozef s bratrími svými.
\par 2 Potom pozdvihl hlasu svého s plácem; a slyšeli to Egyptští, slyšel také dum Faraonuv.
\par 3 I rekl Jozef bratrím svým: Já jsem Jozef. Ješte-li jest živ otec muj? A nemohli mu odpovedíti bratrí jeho; nebo se ho velmi ulekli.
\par 4 Tedy rekl Jozef bratrím svým: Pristuptež medle ke mne. I pristoupili. A rekl: Já jsem Jozef bratr váš, kteréhož jste prodali do Egypta.
\par 5 Protož nyní nermutte se, a neztežujte sobe toho, že jste mne sem prodali; nebo pro zachování života vašeho poslal mne Buh pred vámi.
\par 6 Nebo dve léte již hlad jest v zemi, a ješte pet let prijde, v nichž nebudou orati, ani žíti.
\par 7 Poslal mne, pravím, Buh pred vámi, pro zachování vás ostatku na zemi, a pro zachování životu vašich vysvobozením velikým.
\par 8 Tak tedy ne vy jste mne poslali sem, ale Buh, kterýž mne dal za otce Faraonovi, a za pána všemu domu jeho, a panovníka po vší zemi Egyptské.
\par 9 Pospešte a vstupte k otci mému, a rcete jemu: Toto praví syn tvuj Jozef: Ucinil mne Buh pánem všeho Egypta; prijdiž ke mne, neprodlévej.
\par 10 A bydliti budeš v zemi Gesen; a budeš blízko mne, ty i synové tvoji, i vnukové tvoji, stáda tvá, a volové tvoji, a cožkoli máš.
\par 11 A budu te chovati tam, (nebo ješte pet let hlad bude), abys snad pro hlad nezahynul, ty i dum tvuj, a cožkoli máš.
\par 12 A aj, oci vaše vidí, i oci bratra mého Beniamina, že ústa má mluví vám.
\par 13 Povíte také otci mému o vší sláve mé v Egypte, a což jste koli videli; pospeštež tedy, a privedte otce mého sem.
\par 14 Tedy padl na šíji Beniamina bratra svého, a plakal; Beniamin také plakal na šíji jeho.
\par 15 A políbiv všech bratrí svých, plakal nad nimi; a potom mluvili bratrí jeho s ním.
\par 16 Slyšána pak jest v dome Faraonove povest tato: Prišli bratrí Jozefovi. I bylo to velmi vdecné Faraonovi i všechnem služebníkum jeho.
\par 17 A rekl Farao Jozefovi: Rci bratrím svým: Toto ucinte: Naložíce na hovada svá, jdete, a navratte se do zeme Kananejské.
\par 18 A vezmouce otce svého a celádky své, pridte ke mne, a dám vám dobré místo v zemi Egyptské, a jísti budete tuk zeme této.
\par 19 Ty pak rozkaž jim: Toto ucinte: Vezmete sobe z zeme Egyptské vozy pro deti a ženy své, a vezmete otce svého, a pridte.
\par 20 Aniž se ohlédejte na nábytky své; nebo nejlepší místo ve vší zemi Egyptské vaše bude.
\par 21 I ucinili tak synové Izraelovi. A dal jim Jozef vozy podlé rozkázaní Faraonova; dal také jim pokrmy na cestu.
\par 22 Všechnem jim dal, každému dvoje šaty; ale Beniaminovi dal tri sta stríbrných, a patery šaty jiné a jiné.
\par 23 Otci pak svému poslal tyto veci: Deset oslu, kteríž nesli z nejlepších vecí Egyptských, a deset oslic, kteréž nesly obilí, a chléb a pokrmy otci jeho na cestu.
\par 24 A propoušteje bratrí své, aby odešli, rekl jim: Nevadtež se na ceste.
\par 25 Tedy brali se z Egypta, a prišli do zeme Kananejské k Jákobovi otci svému.
\par 26 A zvestovali jemu, rkouce: Jozef ješte živ jest, ano i panuje ve vší zemi Egyptské. I omdlelo srdce jeho; nebo slyšev to neveril jim.
\par 27 Tedy vypravovali jemu všecka slova Jozefova, kteráž mluvil jim; a vida vozy, kteréž poslal Jozef pro neho, okrál duch Jákoba otce jejich.
\par 28 I rekl Izrael: Dostit jest, když ješte syn muj živ jest; pujdu a uzrím ho, prvé než umru.

\chapter{46}

\par 1 Tedy bral se Izrael se vším, což mel; a prišed do Bersabé, obetoval obeti Bohu otce svého Izáka.
\par 2 I mluvil Buh Izraelovi u videní nocním, rka: Jákobe, Jákobe! Kterýžto odpovedel: Aj, ted jsem.
\par 3 I rekl: Já jsem ten Buh silný, Buh otce tvého. Neboj se sstoupiti do Egypta, nebo v národ veliký tam tebe uciním.
\par 4 Já sstoupím s tebou do Egypta, a já te také i zase privedu; a Jozef položí ruku svou na oci tvé.
\par 5 Vstal tedy Jákob z Bersabé; a synové Izraelovi vzali Jákoba otce svého, a deti své s ženami svými na vozy, kteréž poslal pro neho Farao.
\par 6 Pobrali také dobytek svuj, a zboží své, kteréhož nabyli v zemi Kananejské; a prišli do Egypta, Jákob i všecko síme jeho s ním.
\par 7 Syny i vnuky, dcery i vnucky své, a všecku rodinu svou uvedl s sebou do Egypta.
\par 8 A tato jsou jména synu Izraelových, kteríž vešli do Egypta: Jákob a synové jeho. Prvorozený Jákobuv Ruben.
\par 9 A synové Rubenovi: Enoch, Fallu, Ezron a Charmi.
\par 10 Synové pak Simeonovi: Jamuel, Jamin, Ahod, Jachin, Sohar a Saul, syn jedné ženy Kananejské.
\par 11 Synové Léví: Gerson, Kahat a Merari.
\par 12 Synové Judovi: Her, Onan, Séla, Fáres a Zára. (Ale umrel Her a Onan v zemi Kananejské.) Fáres pak mel syny: Ezrona a Hamule.
\par 13 Synové Izacharovi: Tola, Fua, Job a Simron.
\par 14 A synové Zabulonovi: Sared, Elon a Jahelel.
\par 15 Tit jsou synové Líe, kteréž porodila Jákobovi v Pádan Syrské, a Dínu, dceru jeho. Všech duší synu i dcer jeho bylo tridceti a tri.
\par 16 Synové Gád: Sefon, Aggi, Suni, Esebon, Heri, Arodi a Areli.
\par 17 Synové Asser: Jemna, Jesua, Jesui, Beria, a Serach sestra jejich. Synové pak Beriovi: Heber a Melchiel.
\par 18 To jsou synové Zelfy, kterouž Lában dal Líe dceri své; a ty porodila Jákobovi, šestnácte duší.
\par 19 Synové pak Ráchel, manželky Jákobovy: Jozef a Beniamin.
\par 20 A Jozefovi narodili se v zemi Egyptské z Asenat, dcery Putifera knížete On, Manasses a Efraim.
\par 21 Ale synové Beniaminovi: Béla, Becher, Asbel, Gera, Náman, Echi, Roz, Mufim, Chuppim a Ared.
\par 22 Tit jsou synové Ráchel, kteréž porodila Jákobovi; všech duší ctrnáct.
\par 23 A syn Danuv: Chusim.
\par 24 Synové pak Neftalím: Jaziel, Guni, Jezer a Sallem.
\par 25 Ti jsou synové Bály, kterouž Lában dal Ráchel dceri své, a ty porodila Jákobovi; všech duší sedm.
\par 26 Všech duší, kteréž vešly s Jákobem do Egypta, což jich pošlo z bedr jeho, krome žen synu Jákobových, všech duší bylo šedesáte a šest.
\par 27 K tomu synové Jozefovi, kteríž se jemu narodili v Egypte, dva. A tak všech duší domu Jákobova, kteréž vešly do Egypta, bylo sedmdesáte.
\par 28 Poslal pak Judu napred k Jozefovi, aby oznámil jemu prvé, než prišel do Gesen. A tak prišli do zeme Gesen.
\par 29 Jozef pak zapráh do svého vozu, vyjel vstríc Izraelovi otci svému do Gesen; a jakž ho Jákob uzrel, padl na jeho šíji, a plakal dlouho na šíji jeho.
\par 30 I rekl Izrael Jozefovi: Necht již umru, když jsem videl tvár tvou; nebo ty ješte jsi živ.
\par 31 Jozef pak rekl bratrím svým a domu otce svého: Pojedu a zvestuji Faraonovi, a dím jemu: Bratrí moji a dum otce mého, kteríž bydlili v zemi Kananejské, prišli ke mne.
\par 32 Ale jsou pastýri stáda, nebo s dobytkem se obírají; protož ovce své a voly, i cožkoli mají, prihnali.
\par 33 A když by povolal vás Farao, a rekl:Jaký jest obchod váš?
\par 34 Odpovíte: Dobytkem se živili služebníci tvoji od mladosti své až do této chvíle, i my i otcové naši; abyste bydlili v zemi Gesen; nebo v mrzkosti mají Egyptští všecky pastýre stáda.

\chapter{47}

\par 1 A protož prišed Jozef, oznámil Faraonovi, a rekl: Otec muj a bratrí moji, s drobným i vetším dobytkem svým i se vším, což mají, prišli z zeme Kananejské, a aj, jsou v zemi Gesen.
\par 2 A vzav z poctu bratrí svých pet mužu, postavil je pred Faraonem.
\par 3 I rekl Farao bratrím jeho: Jaký jest obchod váš? Kterížto odpovedeli Faraonovi: Pastýri ovcí jsou služebníci tvoji, i my, i otcové naši.
\par 4 Rekli ješte Faraonovi: Abychom pohostinu byli v zemi této, prišli jsme; nebo není pastvy dobytku, kterýž mají služebníci tvoji, nebo hlad veliký jest v zemi Kananejské; protož nyní prosíme, nechat bydlí služebníci tvoji v zemi Gesen.
\par 5 I mluvil Farao Jozefovi, rka: Otec tvuj a bratrí tvoji prišli k tobe.
\par 6 Zeme Egyptská pred tebou jest; v nejlepším kraji zeme této osad otce svého a bratrí své, necht bydlí v zemi Gesen. A srozumíš-li, že jsou mezi nimi muži rozšafní, ustanovíš je úredníky nad dobytkem, kterýž mám.
\par 7 Uvedl také Jozef Jákoba otce svého, a postavil ho pred Faraonem; a pozdravil Jákob Faraona.
\par 8 Tedy rekl Farao k Jákobovi: Kolik jest let života tvého?
\par 9 Odpovedel Jákob Faraonovi: Dnu let putování mého sto a tridceti let jest; nemnozí a zlí byli dnové let života mého, a nedošli dnu let života otcu mých, v nichž živi byli.
\par 10 A požehnav Jákob Faraona, vyšel od neho.
\par 11 I osadil Jozef otce svého a bratrí své, a dal jim vládarství v zemi Egyptské v kraji výborném, v zemi Ramesses, jakž rozkázal Farao.
\par 12 A opatroval Jozef otce svého, a bratrí své, a všecken dum jeho chlebem, až do nejmenších.
\par 13 A chleba nebylo ve vší zemi; nebo veliký hlad byl velmi, a trápení veliké bylo na zemi Egyptské a zemi Kananejské od hladu.
\par 14 Shromáždil pak Jozef všecky peníze, což jich nalezeno v zemi Egyptské a v zemi Kananejské, za potravy, kteréž kupovali; a vnesl Jozef peníze do domu Faraonova.
\par 15 A když utratili peníze z zeme Egyptské a z zeme Kananejské, pricházeli všickni Egyptští k Jozefovi, rkouce: Dej nám chleba; nebo proc mríti máme pred tebou pro nedostatek penez?
\par 16 I rekl Jozef: Dejte dobytky své, a dám vám chleba za dobytky vaše, ponevadž se vám penez nedostává.
\par 17 Tedy privedli dobytky své k Jozefovi; i dal jim Jozef potrav za kone a za stáda ovcí, a za stáda volu i za osly; a prechoval je chlebem, za všecky dobytky jejich, toho roku.
\par 18 A po roce tom prišli k nemu léta druhého, a rekli mu: Nebudeme tajiti pred pánem svým, že jsme všecky peníze utratili, i stáda dobytku jsou u pána našeho; nezustávají nám pred pánem naším krome tela naše a dediny naše.
\par 19 I proc máme mríti pred ocima tvýma? I my i rolí naše hyne. Kup nás i rolí naši za chléb, a budeme my i rolí naše ve službe Faraonovi; a dej nám semena, abychom živi byli a nezemreli, a rolí aby nespustla.
\par 20 Tedy koupil Jozef všecku zemi Egyptskou Faraonovi; nebo prodali Egyptští jeden každý pole své, proto že se rozmohl mezi nimi hlad. I dostala se zeme Faraonovi.
\par 21 Lid pak prevedl do mest, od jednoho pomezí Egyptského až do druhého.
\par 22 Rolí toliko knežských nekoupil. Nebo kneží uloženou potravu meli od Faraona, a jedli z uložených pokrmu svých, kteréž dával jim Farao; protož neprodali rolí svých.
\par 23 I rekl Jozef lidu: Aj, koupil jsem vás dnes i rolí vaše Faraonovi; ted máte semeno, osívejtež tedy ji.
\par 24 A když se urodí, dáte pátý díl Faraonovi, a ctyri díly zustavíte sobe k semenu a ku pokrmu svému a tem, kteríž jsou v domích vašich, i ku pokrmu dítkám svým.
\par 25 Tedy rekli: Zachoval jsi životy naše. Nechat nalezneme milost v ocích pána svého, a budeme služebníci Faraonovi.
\par 26 I ustanovil to Jozef za právo až do tohoto dne, po vší zemi Egyptské, aby dáván byl Faraonovi pátý díl; toliko samy rolí knežské nebyly Faraonovy.
\par 27 A tak bydlil Izrael v zemi Egyptské v krajine Gesen; a osadili se v ní, a rozplodili se, a rozmnoženi jsou velmi.
\par 28 Živ pak byl Jákob v zemi Egyptské sedmnácte let; a bylo dnu Jákobových, a let života jeho, sto ctyridceti sedm let.
\par 29 I priblížili se dnové Izraelovi, aby umrel. A povolav syna svého Jozefa, rekl jemu: Jestliže jsem nalezl milost v ocích tvých, vlož, prosím, ruku svou pod bedro mé, a ucin se mnou milosrdenství a pravdu. Prosím, nepochovávej mne v Egypte.
\par 30 Když spáti budu s otci svými, vyneseš mne z Egypta, a pochováš mne v hrobe jejich. Tedy rekl jemu: Já uciním podlé slova tvého.
\par 31 I rekl Jákob: Prisáhni mi. Tedy prisáhl jemu. I sklonil se Izrael k hlavám luže.

\chapter{48}

\par 1 Stalo se pak potom, že povedíno jest Jozefovi: Aj, otec tvuj nemocen jest. I vzal s sebou dva syny své, Manasse a Efraima.
\par 2 Tedy oznámeno jest Jákobovi a povedíno: Aj, syn tvuj Jozef prišel k tobe.
\par 3 A posilniv se Izrael, usadil se na ložci a rekl Jozefovi: Buh silný všemohoucí ukázav mi se v Luza v zemi Kananejské, požehnal mi.
\par 4 A rekl ke mne: Aj, já rozplodím te a rozmnožím tebe, a uciním te v zástupy lidí; dám také zemi tuto semeni tvému po tobe za dedictví vecné.
\par 5 Protož nyní, dva synové tvoji, kterížt jsou se zrodili v zemi Egyptské, prvé než jsem prišel k tobe do Egypta, moji jsou; Efraim a Manasses budou mi jako Ruben a Simeon.
\par 6 Ale deti, kteréž po techto zplodíš, tvoji budou; jménem bratrí svých jmenováni budou v dedictvích svých.
\par 7 Nebo když jsem se vracel z Pádan, umrela mi Ráchel v zemi Kananejské na ceste, když již nedaleko bylo do Efraty; a pochoval jsem ji tam u cesty k Efrate, jenž jest Betlém.
\par 8 Uzrev potom Izrael syny Jozefovy, rekl: Kdo jsou onino?
\par 9 Odpovedel Jozef otci svému: Synové moji jsou, kteréž dal mi Buh zde. I rekl:Prived je medle ke mne, a požehnám jim
\par 10 (Oci pak Izraelovy mdlé byly pro starost, a nemohl dobre videti.) I privedl je k nemu, a on líbal a objímal je.
\par 11 I rekl Izrael Jozefovi: Nemyslilt jsem já, abych mel kdy videti tvár tvou, a aj, dal mi Buh, abych videl i síme tvé.
\par 12 Tedy vzav je Jozef z klína jeho, sklonil se tvárí až k zemi.
\par 13 A vzav oba, Efraima na pravou stranu sobe, Izraelovi pak na levou, a Manassesa na levou sobe, Izraelovi pak na pravou, postavil je pred ním.
\par 14 Tedy vztáh Izrael pravici svou, vložil ji na hlavu Efraimovu, kterýž byl mladší, levici pak svou na hlavu Manassesovu, naschvál preloživ ruce, ackoli Manasses byl prvorozený.
\par 15 I požehnal Jozefovi, rka: Buh, pred jehož oblícejem ustavicne chodili otcové moji Abraham a Izák, Buh, kterýž mne spravoval po všecken život muj až do dne tohoto;
\par 16 Andel ten, kterýž vytrhl mne ze všeho zlého, požehnejž dítek techto; a at slovou synové moji a synové otcu mých Abrahama a Izáka; a at se jako hmyz rozmnoží u prostred zeme.
\par 17 Vida pak Jozef, že vložil otec jeho ruku svou pravou na hlavu Efraimovu, nerád byl tomu. I zdvihl ruku otce svého, aby ji prenesl s hlavy Efraimovy na hlavu Manassesovu.
\par 18 A rekl jemu: Ne tak, otce muj; nebo toto jest prvorozený, vložiž pravici svou na hlavu jeho.
\par 19 Ale nepovolil otec jeho, a rekl: Vím, synu muj, vím; takét i on bude v lid, a také i on vzroste; však bratr jeho mladší více poroste než on, a síme jeho bude u veliké množství národu.
\par 20 I požehnal jim v ten den, rka: Skrze tebe požehnání bude dávati Izrael takto: Uciniž tobe Buh jako Efraimovi a jako Manassesovi. I predstavil Efraima Manassesovi.
\par 21 Rekl také Izrael Jozefovi: Aj, já umírám, a budet Buh s vámi, a zase vás privede do zeme otcu vašich.
\par 22 Já pak dal jsem tobe jeden díl výš nad bratrí tvé, kteréhož jsem mecem svým a lucištem svým dosáhl z ruky Amorejského.

\chapter{49}

\par 1 Povolal pak Jákob synu svých, a rekl: Sejdete se, a oznámím vám, co se s vámi díti bude potomních dnu.
\par 2 Shromaždte se a slyšte synové Jákobovi, poslyšte Izraele otce svého.
\par 3 Ruben, prvorozený muj jsi ty, síla má, a pocátek moci mé, vyvýšenost dustojenství a vyvýšenost síly.
\par 4 Prudce sbehneš jako voda. Nebudeš vyvýšen, proto že jsi vstoupil na luže otce svého; a jakž jsi poškvrnil postele mé, zmizelo vyvýšení tvé.
\par 5 Simeon a Léví bratrí, nástrojové nepravosti jsou v príbytcích jejich.
\par 6 Do tajné rady jejich nevcházej duše má, k shromáždení jejich nepripojuj se slávo má; nebo v prchlivosti své zbili muže, a svévolne vyvrátili zed.
\par 7 Zlorecená prchlivost jejich, nebo neustupná; i hnev jejich, nebo zatvrdilý jest. Rozdelím je v Jákobovi, a rozptýlím je v Izraeli.
\par 8 Judo, ty jsi, tebe chváliti budou bratrí tvoji; ruka tvá bude na šíji neprátel tvých; klaneti se budou tobe synové otce tvého.
\par 9 Lvíce Juda, z loupeže, synu muj, vrátil jsi se; schýliv se, ležel jako lev a jako lvice; kdo zbudí ho?
\par 10 Nebude odjata berla od Judy, ani vydavatel zákona od noh jeho, dokudž neprijde Sílo; a k nemu se shromáždí národové.
\par 11 Uváže k vinnému kmenu osle své, a k výbornému kmenu oslátko oslice své. Práti bude u víne roucho své, a v cerveném víne odev svuj.
\par 12 Cervenejších ocí bude nad víno, a zubu belejších nad mléko.
\par 13 Zabulon bydliti bude na brehu morském, a na prístavu lodí, a pomezí jeho až k Sidonu.
\par 14 Izachar osel silný, ležící mezi dvema bremeny.
\par 15 A vida odpocinutí, že jest dobré, a zemi, že jest rozkošná, sehne rameno své k nošení, a dávati bude dane.
\par 16 Dan souditi bude lid svuj, jako jedno z pokolení Izraelských.
\par 17 Budet Dan jako had podlé cesty, jako had rohatý podlé stezky, štípaje kopyta kone, aby spadl jezdec jeho zpet.
\par 18 Spasení tvého ocekávám, ó Hospodine!
\par 19 Gád, vojsko premuže jej, však on svítezí potom.
\par 20 Asser, tucný bude pokrm jeho, a ont vydávati bude rozkoše královské.
\par 21 Neftalím jako lan vypuštená, vydávaje výmluvnosti krásné.
\par 22 Ratolest rostoucí Jozef, ratolest rostoucí podlé studnice; ratolesti vycházely nad zed.
\par 23 Ackoli horkostí naplnili jej, a stríleli na nej, a v tajné nenávisti meli ho strelci:
\par 24 Však zustalo v síle lucište jeho, a zsilila se ramena rukou jeho z rukou mocného Jákobova; odkudž byl pastýr a kámen Izraeluv;
\par 25 Od silného Boha, jemuž sloužil otec tvuj, kterýž spomáhá tobe, a od všemohoucího, kterýž požehná tobe požehnáními nebeskými s hury, požehnáními propasti ležící hluboko, požehnáním prsu a života.
\par 26 Požehnání otce tvého silnejší budou nad požehnání predku mých, až k koncinám pahrbku vecných; budou nad hlavou Jozefovou, a na vrchu hlavy Nazarejského mezi bratrími jeho.
\par 27 Beniamin, vlk dravý, ráno bude jísti loupež, a vecer rozdelí koristi.
\par 28 Všech techto pokolení Izraelských jest dvanácte; a to jest, což mluvil jim otec jejich; požehnal jim také, jednomu každému vedlé požehnání jeho požehnal.
\par 29 A porouceje jim, rekl: Já pripojen budu k lidu svému; pochovejte mne s otci mými v jeskyni té, kteráž jest na poli Efrona Hetejského,
\par 30 V jeskyni, kteráž jest na poli Machpelah, jenž jest naproti Mamre v zemi Kananejské, kterouž koupil Abraham spolu s polem tím od Efrona Hetejského k dedicnému pohrbu.
\par 31 Tam pochovali Abrahama a Sáru ženu jeho; tam pochovali Izáka a Rebeku ženu jeho; tam také pochovali Líu.
\par 32 Koupeno pak bylo pole a jeskyne, kteráž na nem, od synu Het.
\par 33 A když prestal Jákob prikazovati synum svým, složil nohy své na ložci a umrel; a pripojen jest k lidu svému.

\chapter{50}

\par 1 Tedy padl Jozef na tvár otce svého, a plakal nad ním, líbaje ho.
\par 2 A porucil služebníkum svým lékarum, aby vonnými vecmi pomazali otce jeho. I pomazali lékari vonnými vecmi Izraele.
\par 3 A vyplnilo se pri nem ctyridceti dní; (nebo tak vyplnují se dnové tech, kteríž mazáni bývají vonnými vecmi). I plakali ho Egyptští za sedmdesáte dní.
\par 4 Když pak dnové pláce toho pominuli, mluvil Jozef k domu Faraonovu, rka: Jestliže jsem nyní nalezl milost pred ocima vašima, mluvte, prosím, v uši Faraonovy, a rcete:
\par 5 Otec muj prísahou mne zavázal, rka: Aj, já umírám; v hrobe mém, kterýž jsem sobe vykopal v zemi Kananejské, tam mne pochovej. Nyní tedy, prosím, necht vstoupím a pochovám otce svého, a navrátím se zase.
\par 6 I rekl Farao: Vstup a pochovej otce svého tak, jakž te prísahou zavázal.
\par 7 Tedy vstoupil Jozef, aby pochoval otce svého, a vstoupili s ním všickni služebníci Faraonovi, starší domu jeho a všickni starší zeme Egyptské,
\par 8 Všecken také dum Jozefuv, a všickni bratrí jeho i dum otce jeho; toliko detí svých, ovec a volu nechali v zemi Gesen.
\par 9 Vstoupili s ním také i vozové a jezdci, a bylo vojsko to veliké velmi.
\par 10 I prišli až k místu Atád, kteréž jest pri brodu Jordánském, a kvílili tam kvílením velikým a velmi žalostným; i držel tu zámutek po otci svém za dnu sedm.
\par 11 Vidouce pak obyvatelé zeme té, totiž Kananejští, zámutek na míste tom Atád, rekli: Težký to mají zámutek Egyptští. Protož nazváno jest jméno jeho Abel Mizraim, a to jest pri brodu Jordánském.
\par 12 Ucinili tedy s ním synové jeho tak, jakž jim byl porucil.
\par 13 A donesše ho do zeme Kananejské, pochovali ho v jeskyni na poli Machpelah, kterouž byl koupil Abraham s tím polem k dedicnému pohrbu od Efrona Hetejského, naproti Mamre.
\par 14 A když pochoval Jozef otce svého, navrátil se do Egypta s bratrími svými a se všemi, kteríž byli vstoupili s ním, aby pochovali otce jeho.
\par 15 Vidouce pak bratrí Jozefovi, že umrel otec jejich, rekli: Snad v nenávisti nás míti bude Jozef, a vrchovate nahradí nám všecko zlé, kteréž jsme jemu cinili.
\par 16 Protož vzkázali Jozefovi, rkouce: Otec tvuj ješte pred smrtí svou prikázal, rka:
\par 17 Takto díte Jozefovi: Prosím, odpust již bratrím svým prestoupení a hrích jejich; nebo zle ucinili tobe. Protož již odpust, prosím, prestoupení služebníkum Boha, jehož ctil otec tvuj. I rozplakal se Jozef, když k nemu mluvili.
\par 18 Pristoupili potom také bratrí jeho, a padše pred ním, rekli: Aj, my jsme služebníci tvoji.
\par 19 Jimž odpovedel Jozef: Nebojte se; nebo zdaliž jsem já vám za Boha?
\par 20 Vy zajisté skládali jste proti mne zlé; ale Buh obrátil to v dobré, aby ucinil to, což vidíte nyní, a pri životu zachoval lid mnohý.
\par 21 Protož nebojte se již; já chovati vás budu i deti vaše. A tak tešil je, a mluvil k srdci jejich.
\par 22 Bydlil pak Jozef v Egypte, on i dum otce jeho, a živ byl Jozef sto a deset let.
\par 23 A videl Jozef syny Efraimovy až do tretího pokolení; ano i synové Machira, syna Manassesova, vychováni jsou u Jozefa.
\par 24 Mluvil potom Jozef bratrím svým: Já tudíž umru; Buh pak jistotne navštíví vás, a vyvede vás z zeme této do zeme, kterouž prisáhl Abrahamovi, Izákovi a Jákobovi.
\par 25 Protož prísahou zavázal Jozef syny Izraelovy, rka: Když navštíví vás Buh, vynestež kosti mé odsud.
\par 26 I umrel Jozef, když byl ve stu a v desíti letech; a pomazán jsa vonnými vecmi, vložen jest do truhly v Egypte.

\end{document}