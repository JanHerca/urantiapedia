\begin{document}

\title{Jonah}

\chapter{1}

\par 1 I stalo se slovo Hospodinovo k Jonášovi synu Amaty, rkoucí:
\par 2 Vstan, jdi do Ninive mesta toho velikého, a volej proti nemu; nebot jest vstoupila nešlechetnost jejich pred oblícej muj.
\par 3 Ale Jonáš vstal, aby utekl do Tarsu od tvári Hospodinovy. A prišed do Joppe, našel lodí, ana jde do Tarsu, a zaplativ od ní, vstoupil na ni, aby se plavil s nimi do Tarsu od tvári Hospodinovy.
\par 4 Ale Hospodin vzbudil vítr veliký na mori. I stala se boure veliká na mori, až se domnívali, že se lodí ztroskoce.
\par 5 A bojíce se plavci, volali jeden každý k bohu svému, a vyhazovali to, což meli na lodí, do more, aby sobe tím polehcili. Jonáš pak byl sešel k bokum lodí, a položiv se, spal tvrde.
\par 6 Tedy prišed k nemu správce lodí, rekl jemu: Což ty deláš, ospalce? Vstan, volej k Bohu svému. Snad ten Buh rozpomene se na nás, abychom nezahynuli.
\par 7 I rekli jeden druhému: Podte, vrzme losy, abychom zvedeli, pro koho to zlé prišlo na nás. Tedy metali losy, a padl los na Jonáše.
\par 8 I rekli jemu: Povez nám medle, pro koho toto zlé na nás? Jaký jest obchod tvuj, a odkud jdeš? Z které jsi zeme a z kterého národu?
\par 9 I rekl jim: Hebrejský jsem, a Hospodina Boha nebes, kterýž ucinil more i zemi, já ctím.
\par 10 Procež báli se ti muži bázní velikou,a dovedevše se muži ti, že od tvári Hospodinovy utíká, (nebo jim byl oznámil), rekli jemu: Což jsi to ucinil?
\par 11 Rekli ješte k nemu: Což máme uciniti s tebou, aby se more spokojilo? (Nebo more vždy více a více bourilo se.)
\par 12 Jimžto rekl: Vezmete mne, a uvrzte mne do more, a utichne more pred vámi; nebo já vím, že prícinou mou boure tato veliká jest proti vám.
\par 13 Ale muži ti statecne táhli, chtíce k brehu pristati, však nemohli; nebo more vždy více a více se bourilo proti nim.
\par 14 I zvolali k Hospodinu, rkouce: Prosímet, ó Hospodine, abychom nezahynuli pro smrt cloveka tohoto, aniž na nás vyhledávej krve nevinné; nebo ty, ó Hospodine, jakž chceš, tak ciníš.
\par 15 Tedy vzavše Jonáše, uvrhli ho do more. I prestalo more bouriti se.
\par 16 Procež báli se muži ti bázní velikou Hospodina, a obetovali obet Hospodinu, a sliby cinili.

\chapter{2}

\par 1 Nastrojil pak byl Hospodin rybu velikou, aby požrela Jonáše. I byl Jonáš v strevách té ryby tri dni a tri noci.
\par 2 I modlil se Jonáš Hospodinu Bohu svému v strevách té ryby,
\par 3 A rekl: Z ssoužení svého volal jsem k Hospodinu, a ozval se mi; z bricha hrobu kricel jsem, a vyslyšel jsi hlas muj.
\par 4 Nebo jsi mne uvrhl do hlubiny, do prostred more, a reka obklícila mne; všecka vlnobití tvá i rozvodnení tvá na mne se svalila.
\par 5 Bylt jsem již rekl: Vyhnán jsem od ocí tvých, ale ještet pohledím na tvuj svatý chrám.
\par 6 Obklícily mne vody až k duši, propast obklícila mne, lekno otocilo se okolo hlavy mé.
\par 7 Až k spodkum hor dostal jsem se, zeme závorami svými zalehla mi na vecnost, ty jsi však vysvobodil od porušení život muj, ó Hospodine Bože muj.
\par 8 Když se svírala ve mne duše má, na Hospodina jsem se rozpomínal, i prišla k tobe modlitba má do chrámu svatého tvého.
\par 9 Kteríž ostríhají marností pouhých, dobroty Boží se zbavují.
\par 10 Já pak s hlasem díkcinení obetovati budu tobe; což jsem slíbil, splním. Hojné vysvobození jest u Hospodina.
\par 11 Rozkázal pak byl Hospodin rybe té, i vyvrátila Jonáše na breh.

\chapter{3}

\par 1 I stalo se slovo Hospodinovo k Jonášovi podruhé, rkoucí:
\par 2 Vstan, jdi do Ninive mesta toho velikého, a kaž proti nemu to, což já poroucím tobe.
\par 3 Tedy vstav Jonáš, šel do Ninive podlé slova Hospodinova. (Bylo pak Ninive mesto velmi veliké, cesty trí dnu.)
\par 4 A jakž pocal byl Jonáš jíti po meste cestou dne jednoho a volati, prave: Po ctyridcíti dnech Ninive vyvráceno bude,
\par 5 Tedy uverili Ninivitští Bohu, a vyhlásivše pust, oblékli se v žíne, od nejvetšího z nich až do nejmenšího z nich.
\par 6 Nebo jakž došla ta rec krále Ninivitského, vstav s trunu svého, složil s sebe odev svuj, a priodev se žíní, sedel v popele.
\par 7 A dal provolati a oznámiti v Ninive z usouzení královského i knížat svých, takto rka: Lidé i hovada, volové i ovce, neokoušejte niceho, nepaste se, ani vody nepíte.
\par 8 Ale priodejte se žínemi lidé i hovada, a volejte k Bohu horlive, a odvrat se jeden každý od cesty své zlé, i loupeže, kteráž jest v rukou jeho.
\par 9 Kdo ví, neobrátí-li se a nebude-li želeti toho Buh; neodvrátí-li se, pravím, od prchlivosti hnevu svého, abychom nezahynuli.
\par 10 I videl Buh skutky jejich, že se odvrátili od cesty své zlé, a lítost mel Buh nad tím zlým, kteréž rekl uciniti jim. A neucinil.

\chapter{4}

\par 1 I mrzelo to Jonáše velmi, a rozpálen byl hnev jeho.
\par 2 Procež modlil se Hospodinu a rekl: Prosím, Hospodine, zdaliž jsem toho nerekl, když jsem ješte byl v zemi své? Protož jsem pospíšil uteci do Tarsu; nebo jsem vedel, že jsi ty Buh milostivý a lítostivý, dlouhocekající a hojný v milosrdenství, a kterýž lituješ zlého.
\par 3 Nyní tedy, ó Hospodine, vezmi, prosím, duši mou ode mne; nebo lépe jest mi umríti nežli živu býti.
\par 4 I rekl Hospodin: Jest-liž to dobre, že tak horlíš?
\par 5 Nebo vyšel byl Jonáš z mesta, a sedel na východ proti mestu, a udelav sobe tu boudu, sedel pod ní v stínu, ažby videl, co se bude díti s tím mestem.
\par 6 Pristrojil pak byl Hospodin Buh brectan, kterýž vyrostl nad Jonáše, aby zastenoval hlavu jeho, a chránil ho pred horkem. I radoval se Jonáš z toho brectanu radostí velikou.
\par 7 V tom nazejtrí v svitání nastrojil Buh cerva, kterýž ranil ten brectan, tak že uschl.
\par 8 I stalo se, že když vzešlo slunce, nastrojil Buh vítr východní žhoucí, a bilo slunce na hlavu Jonášovu, tak že umdléval, a žádal sobe, aby umrel, rka: Lépet mi jest umríti nežli živu býti.
\par 9 I rekl Buh Jonášovi: Jest-liž to dobre, že se tak hneváš pro ten brectan? Kterýžto rekl: An dobre jest, že se hnevám až na smrt.
\par 10 Jemuž rekl Hospodin: Ty lituješ toho brectanu, o nemž jsi nepracoval, aniž jsi ho k zrostu privedl, kterýž za jednu noc zrostl, a jedné noci zahynul,
\par 11 A já abych nelitoval Ninive mesta tak velikého, v nemž jest více nežli sto a dvadceti tisíc lidí, kteríž neznají rozdílu mezi pravicí svou a levicí svou, a dobytka mnoho?

\end{document}