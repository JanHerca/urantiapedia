\begin{document}

\title{Zjevení Janovo}

\chapter{1}

\par 1 Zjevení Ježíše Krista, kteréž dal jemu Buh, aby ukázal služebníkum svým, které veci mely by se díti brzo, on pak zjevil, poslav je skrze andela svého, služebníku svému Janovi,
\par 2 Kterýž osvedcil slovo Boží a svedectví Jezukristovo, a cožkoli videl.
\par 3 Blahoslavený, kdož cte, i ti, kteríž slyší slova proroctví tohoto a ostríhají toho, což napsáno jest v nem; nebo cas blízko jest.
\par 4 Jan sedmi církvím, kteréž jsou v Azii: Milost vám a pokoj od toho, Jenž jest, a Kterýž byl, a Kterýž prijíti má, a od sedmi Duchu, kteríž pred oblicejem trunu jeho jsou,
\par 5 A od Ježíše Krista, jenž jest svedek verný, ten prvorozený z mrtvých, a kníže králu zeme, kterýžto zamiloval nás a umyl nás od hríchu našich krví svou,
\par 6 A ucinil nás krále a kneží Bohu a Otci svému, jemužto bud sláva a moc na veky veku. Amen.
\par 7 Aj, bére se s oblaky, a uzrít jej všeliké oko, i ti, kteríž ho bodli; a budou plakati pro nej všecka pokolení zeme, jiste, Amen.
\par 8 Ját jsem Alfa i Omega, pocátek i konec, praví Pán, Kterýž jest, a Kterýž byl, a Kterýž prijíti má, ten všemohoucí.
\par 9 Já Jan, kterýž jsem i bratr váš, i spoluúcastník v soužení, i v království, i v trpelivosti Ježíše Krista, byl jsem na ostrove, kterýž slove Patmos, pro slovo Boží a svedectví Ježíše Krista.
\par 10 A byl jsem u vytržení ducha v den Páne, i uslyšel jsem za sebou hlas veliký jako trouby,
\par 11 An mi dí: Já jsem Alfa i Omega, ten první i poslední, a rekl: Co vidíš, piš do knihy, a pošli sborum, kteríž jsou v Azii, do Efezu, a do Smyrny, a do Pergamu, a do Tyatiru, a do Sardy, a do Filadelfie, i do Laodicie.
\par 12 I obrátil jsem se, abych videl ten hlas, kterýž mluvil se mnou, a obrátiv se, uzrel jsem sedm svícnu zlatých,
\par 13 A uprostred sedmi svícnu podobného Synu cloveka, obleceného v dlouhé roucho a prepásaného na prsech pasem zlatým.
\par 14 Hlava pak jeho a vlasové byli bílí, jako bílá vlna, jako sníh, a oci jeho jako plamen ohne.
\par 15 Nohy pak jeho podobné mosazi, jako v peci horící; a hlas jeho jako hlas vod mnohých.
\par 16 A mel v pravé ruce své sedm hvezd, a z úst jeho mec s obou stran ostrý vycházel; a tvár jeho jako slunce, když jasne svítí.
\par 17 A jakž jsem jej uzrel, padl jsem k nohám jeho, jako mrtvý. I položil ruku svou pravou na mne, rka ke mne: Neboj se. Já jsem ten první i poslední,
\par 18 A živý, ješto jsem byl mrtvý, a aj, živý jsem na veky veku. Amen. A mám klíce pekla i smrti.
\par 19 Napiš ty veci, kteréž jsi videl, a kteréž jsou, a kteréž se mají díti potom.
\par 20 Tajemství sedmi hvezd, kteréž jsi videl v pravici mé, a sedm svícnu zlatých, jest toto: Sedm hvezd jsou andelé sedmi církví; a sedm svícnu, kteréžs videl, jest tech sedm církví.

\chapter{2}

\par 1 Andelu Efezské církve piš: Totot praví ten, kterýž drží tech sedm hvezd v pravici své, jenž se prochází uprostred tech sedmi svícnu zlatých:
\par 2 Známt skutky tvé, a práci tvou, i trpelivost tvou, a že nemužeš trpeti zlých; a zkusil jsi tech, kteríž se praví býti apoštolé, ale nejsou, a shledals je, že jsou lhári.
\par 3 A snášel jsi, a trpelivost máš, a pro jméno mé pracovals a neustal.
\par 4 Ale mámt neco proti tobe, totiž že jsi tu první lásku svou opustil.
\par 5 Protož pomni, odkud jsi vypadl, a cin pokání a první skutky cin. Paklit toho nebude, prijdu na tebe rychle a pohnut svícnem tvým z místa jeho, nebudeš-li pokání ciniti.
\par 6 Ale toto dobré do sebe máš, že nenávidíš skutku Mikulášenské roty, kterýchž i já nenávidím.
\par 7 Kdo má uši, slyš, co Duch praví církvím: Tomu, kdož zvítezí, dám jísti z dreva života, kteréž jest uprostred ráje Božího.
\par 8 Andelu pak Smyrnenské církve piš: Toto praví ten první i poslední, kterýž byl mrtvý, a ožil:
\par 9 Vím o skutcích tvých, i o soužení, a chudobe tvé, (ale bohatý jsi), i o rouhání tech, kteríž se praví býti Židé, a nejsou, ale jsou sber satanova.
\par 10 Nebojž se nic toho, co trpeti máš. Aj, uvržet dábel nekteré z vás do vezení, abyste zkušeni byli, a budete míti úzkost za deset dní. Budiž verný až do smrti, a dámt korunu života.
\par 11 Kdo má uši, slyš, co Duch praví církvím: Kdo zvítezí, nebude uražen od smrti druhé.
\par 12 Andelu pak Pergamenské církve piš: Totot praví ten, kterýž má ten mec s obou stran ostrý:
\par 13 Vím skutky tvé, i kde bydlíš, totiž tu, kdež jest stolice satanova, a že držíš se jména mého, a nezaprel jsi víry mé, ani v tech dnech, když Antipas, svedek muj verný, zamordován jest u vás, tu, kdež bydlí satan.
\par 14 Ale mámt proti tobe neco málo, že tam máš ty, kteríž drží ucení Balámovo, jenž ucil Baláka pohoršení klásti pred oblicejem synu Izraelských, aby jedli modlám obetované a smilnili.
\par 15 Tak i ty máš nekteré, kteríž drží ucení Mikulášencu, což já v nenávisti mám.
\par 16 Ciniž pokání. Pakli nebudeš, prijdu na tebe brzy, a bojovatit budu s nimi mecem úst svých.
\par 17 Kdo má uši, slyšiž, co Duch praví církvím: Tomu, jenž vítezí, dám jísti mannu skrytou a dám jemu kamének bílý, a na tom kaménku jméno nové napsané, kteréhož žádný neví, než ten, kdož je prijímá.
\par 18 Andelu pak Tyatirské církve piš: Totot praví Syn Boží, jenž má oci jako plamen ohne, a nohy jeho podobné jsou mosazi:
\par 19 Známt skutky tvé, i lásku, i prisluhování, i vernost, i trpelivost tvou, a skutky tvé, i ty poslední, kteríž vetší jsou nežli první.
\par 20 Ale mámt proti tobe neco málo, že dopouštíš žene Jezábel, kteráž se býti praví prorokyní, uciti a v blud uvoditi služebníky mé, aby smilnili a jedli modlám obetované.
\par 21 Ale dalt jsem jí cas, aby pokání cinila z smilstva svého, avšak necinila pokání.
\par 22 Aj, já uvrhu ji na lože, i ty, kteríž cizoloží s ní, v soužení preveliké, jestliže nebudou ciniti pokání z skutku svých.
\par 23 A syny její zmorduji smrtí; i zvedít všecky církve, žet jsem já ten, kterýž zpytuji ledví a srdce, a odplatím jednomu každému z vás podle skutku vašich.
\par 24 Vámt pak pravím i jiným Tyatirským, kterížkoli nemají ucení tohoto, a kteríž nepoznali hlubokosti satanovy, jakž oni ríkají: Nevzložímt na vás jiného bremene.
\par 25 Avšak to, což máte, držte, dokavadž neprijdu.
\par 26 Kdož by pak vítezil a ostríhal až do konce skutku mých, dám jemu moc nad pohany.
\par 27 I budet je spravovati prutem železným, a jako nádoba hrncírova strískáni budou, jakž i já vzal jsem od Otce svého.
\par 28 A dám jemu hvezdu jitrní.
\par 29 Kdo má uši, slyš, co Duch praví církvím.

\chapter{3}

\par 1 Andelu pak církve Sardinské piš: Toto praví ten, jenž má sedm duchu Božích a sedm hvezd: Vím skutky tvé: máš jméno, že jsi živ, ale jsi mrtvý.
\par 2 Budiž bedlivý, a potvrdiž toho, což melo umríti. Nebot jsem nenalezl skutku tvých plných pred Bohem.
\par 3 Pomniž tedy, co jsi prijal a slyšel, a ostríhej toho, a cin pokání. Pakli bdíti nebudeš, prijdut na te jako zlodej, a nezvíš, v kterou hodinu na te prijdu.
\par 4 Ale máš nekteré osoby i v Sardis, kteréž neposkvrnily roucha svého, protož budout se procházeti se mnou v bílém rouše; nebo jsou hodni.
\par 5 Kdo zvítezí, tent bude odín rouchem bílým; a nevymažit jména jeho z knih života, ale vyznámt jméno jeho pred oblicejem Otce svého i pred andely jeho.
\par 6 Kdo má uši, slyš, co Duch praví církvím.
\par 7 Andelu pak Filadelfitské církve piš: Totot praví ten svatý a pravý, jenž má klíc Daviduv, kterýžto otvírá, a žádný nezavírá, a zavírá, žádný pak neotvírá:
\par 8 Vím skutky tvé. Aj, postavil jsem pred tebou dvere otevrené, a žádnýt jich nemuže zavríti. Nebo máš ac nevelikou moc, a ostríhal jsi slova mého, a nezaprels jména mého.
\par 9 Aj, dávámt nekteré z sberi satanovy, kteríž se praví býti Židé, a nejsou, ale klamají. Aj, pravím, zpusobím to, žet prijdou a budou se klaneti pred nohama tvýma, a poznajít, že jsem já te miloval.
\par 10 Nebo jsi ostríhal slova trpelivosti mé, i ját tebe ostríhati budu od hodiny pokušení, kteréž prijíti má na všecken svet, aby zkušeni byli obyvatelé zeme.
\par 11 Aj, prijdut brzy. Držiž se toho, co máš, aby žádný nevzal koruny tvé.
\par 12 Kdo zvítezí, uciním jej sloupem v chráme Boha svého, a nevyjdet již více ven; a napíšit na nem jméno Boha svého a jméno mesta Boha mého, nového Jeruzaléma sstupujícího s nebe od Boha mého, i jméno své nové.
\par 13 Kdo má uši, slyš, co Duch praví církvím.
\par 14 Andelu pak církve Laodicenské piš: Toto praví Amen, svedek ten verný a pravý, pocátek stvorení Božího:
\par 15 Vímt skutky tvé, že ani jsi studený, ani horký. Ó bys chutne studený byl anebo horký.
\par 16 A tak, že jsi vlažný, a ani studený, ani horký, vyvrhu te z úst svých.
\par 17 Nebo pravíš: Bohatý jsem, a zbohatl jsem, a žádného nepotrebuji, a nevíš, že jsi bídný, mizerný, a chudý, a slepý, i nahý.
\par 18 Radím tobe, abys sobe koupil ode mne zlata ohnem zprubovaného, abys byl bohatý, a v roucho bílé abys oblecen byl, a neokazovala se hanba nahoty tvé. A ocí svých pomaž kollyrium, abys videl.
\par 19 Já kteréžkoli miluji, kárám a tresci; rozhorliž se tedy, a cin pokání.
\par 20 Aj, stojímt u dverí, a tluku. Jestližet by kdo uslyšel hlas muj a otevrel dvere, vejdut k nemu, a budu s ním vecereti, a on se mnou.
\par 21 Kdož zvítezí, dám jemu sedeti s sebou na trunu svém, jako i já zvítezil jsem, a sedím s Otcem svým na trunu jeho.
\par 22 Kdo má uši, slyš, co Duch praví církvím.

\chapter{4}

\par 1 Potom jsem videl, a aj, dvere otevríny byly v nebi, a hlas první, kterýž jsem byl slyšel jako trouby, an mluví se mnou, rkoucí: Vstup sem, a ukážit, co se má díti potom.
\par 2 A hned byl jsem u vytržení ducha, a aj, trun postaven byl na nebi, a na trunu jeden sedel.
\par 3 A ten, jenž sedel, podoben byl oblicejem kameni jaspidu a sardiovi; a vukol toho trunu byla duha, na pohledení podobná smaragdu.
\par 4 A okolo toho trunu bylo stolic ctyrmecítma, a na tech stolicích videl jsem ctyrmecítma starcu sedících, oblecených v roucha bílá. A meli na hlavách svých koruny zlaté.
\par 5 A z toho trunu pocházelo blýskání, a hromobití, a hlasové. A sedm lamp ohnivých horících pred trunem, jenž jsou sedm Duchu Božích.
\par 6 A pred trunem bylo more sklené, podobné krištálu, a uprostred trunu a vukol trunu ctvero zvírat, plných ocí zpredu i zzadu.
\par 7 A zvíre první podobné bylo lvu, a druhé zvíre podobné teleti, a tretí zvíre mající tvárnost jako clovek, a ctvrté zvíre podobné orlu letícímu.
\par 8 A jedno každé z tech ctyr zvírat melo šest krídel okolo, a vnitr plná byla ocí, a nemela odpocinutí dnem i nocí, ríkajíce: Svatý, svatý, svatý Pán Buh všemohoucí, kterýž byl, a jest, i prijíti má.
\par 9 A když ta zvírata vzdávala slávu a cest i díku cinení sedícímu na trunu, živému na veky veku,
\par 10 Padlo tech ctyrmecítma starcu pred oblicejem sedícího na trunu, a klanelo se živému na veky veku, a metali koruny své pred trunem, rkouce:
\par 11 Hoden jsi, Pane, prijíti slávu, a cest, i moc; nebo ty jsi stvoril všecky veci, a pro vuli tvou trvají, i stvoreny jsou.

\chapter{5}

\par 1 I videl jsem po pravici sedícího na trunu knihy popsané vnitr i zevnitr, zapecetené sedmi pecetmi
\par 2 A videl jsem andela silného, volajícího hlasem velikým: Kdo jest hoden otevríti knihy tyto a zrušiti peceti jejich?
\par 3 I nemohl žádný, ani na nebi, ani na zemi, ani pod zemí, otevríti knih, ani pohledeti do nich.
\par 4 Procež já plakal jsem velmi, že není nalezen žádný, kterýž by hoden byl otevríti a císti tu knihu, ani pohledeti do ní.
\par 5 Tedy jeden z starcu dí mi: Neplac. Aj, zvítezilt Lev, ten, kterýž jest z pokolení Judova, koren Daviduv, aby otevrel tu knihu a zrušil sedm pecetí jejích.
\par 6 I videl jsem, a aj, mezi trunem a ctyrmi zvíraty a mezi starci Beránek stojí jako zabitý, maje sedm rohu a sedm ocí, jenž jsou sedm Duchu Božích, poslaných na všecku zemi.
\par 7 I prišel a vzal knihy z pravice toho, kterýž sedel na trunu.
\par 8 A jakž vzal knihy, ihned tech ctvero zvírat a tech ctyrmecítma starcu padlo pred Beránkem, majíce jeden každý z nich harfu a báne zlaté plné vune, jenž jsou modlitby svatých.
\par 9 A zpívali písen novou, rkouce: Hoden jsi vzíti knihy a otevríti peceti jejich. Nebo jsi zabit, a vykoupils nás Bohu krví svou ze všelikého pokolení a jazyku a lidu i národu.
\par 10 A ucinil jsi nás Bohu našemu krále a kneží, a budeme kralovati na zemi.
\par 11 I videl jsem a slyšel hlas andelu mnohých okolo trunu a zvírat a starcu, a byl pocet jejich stokrát tisíc tisícu a desetkrát sto tisícu,
\par 12 Rkoucích velikým hlasem: Hodent jest ten zabitý Beránek vzíti moc, a bohatství, i moudrost, i sílu, i cest, i slávu, i požehnání.
\par 13 Též všecko stvorení, kteréž jest na nebi, i na zemi, i pod zemí, i v mori, i všecko, což v nich jest, slyšel jsem rkoucí: Sedícímu na trunu a Beránkovi požehnání, cest a sláva i síla na veky veku.
\par 14 Ctvero pak zvírat reklo: Amen. A tech ctyrmecítma starcu padlo, a klaneli se živému na veky veku.

\chapter{6}

\par 1 I videl jsem, když otevrel Beránek jednu z tech pecetí, a slyšel jsem jedno ze ctyr zvírat, ano praví, jako hlasem hromovým: Pojd a viz.
\par 2 I pohledel jsem, a aj, kun bílý, a ten, jenž sedel na nem, mel lucište. A dána jest jemu koruna, i vyjel, premáhaje, a aby premáhal.
\par 3 A když otevrel pecet druhou, slyšel jsem druhé zvíre, rkoucí: Pojd a viz.
\par 4 I vyšel jiný kun ryzí. A sedícímu na nem dáno jest, aby pokoj vyzdvihl z zeme a aby se lidé vespolek mordovali; a dán jemu mec veliký.
\par 5 A když otevrel pecet tretí, slyšel jsem tretí zvíre, rkoucí: Pojd a viz. I pohledel jsem, a aj, kun vraný, a ten, jenž na nem sedel, mel váhu v ruce své.
\par 6 I slyšel jsem hlas z prostredku ctyr zvírat, an praví: Merice pšenice za peníz, a tri merice jecmene za peníz, oleji pak a vínu neškod.
\par 7 A když otevrel pecet ctvrtou, slyšel jsem hlas ctvrtého zvírete, rkoucí: Pojd a viz.
\par 8 I pohledel jsem, a aj, kun plavý, a toho, kterýž sedel na nem, jméno smrt, a peklo šlo za ním. I dána jest jim moc nad ctvrtým dílem zeme, aby mordovali mecem, a hladem, a morem, a šelmami zemskými.
\par 9 A když otevrel pecet pátou, videl jsem pod oltárem duše zmordovaných pro slovo Boží a pro svedectví, kteréž vydávali.
\par 10 I volali hlasem velikým, rkouce: Až dokud, Pane svatý a pravý, nesoudíš a nemstíš krve naší nad temi, kteríž prebývají na zemi?
\par 11 Tedy dáno jednomu každému z nich roucho bílé, a receno jest jim, aby odpocívali ješte za malý cas, až by se naplnil pocet spoluslužebníku jejich a bratrí jejich, kteríž zmordováni býti mají, jako i oni.
\par 12 I pohledel jsem, když otevrel pecet šestou, a aj, zeme tresení veliké stalo se, a slunce ucineno jest cerné jako pytel žínený, a mesíc všecken byl jako krev.
\par 13 A hvezdy nebeské padaly na zemi, podobne jako drevo fíkové smítá s sebe ovoce své, když od velikého vetru kláceno bývá.
\par 14 A nebe se schovalo jako knihy zavrené, a všeliká hora i ostrovové z místa svého pohnuli se.
\par 15 A králové zeme, i knížata, i bohatí, a úredníci, a mocní, i každý služebník, i všeliký svobodný, skryli se v jeskyních a v skalí hor,
\par 16 A rekli horám a skalí: Padnete na nás a skrejte nás pred tvárí toho, jenž sedí na trunu, a pred hnevem Beránka.
\par 17 Nebt jest prišel den veliký hnevu jeho. I kdo bude moci ostáti?

\chapter{7}

\par 1 Potom jsem videl ctyri andely, stojící na ctyrech úhlech zeme, držící ctyri vetry zemské, aby nevál vítr na zemi, ani na more, ani na žádný strom.
\par 2 A videl jsem jiného andela, vystupujícího od východu slunce, majícího pecet Boha živého. Kterýž zvolal hlasem velikým na ty ctyri andely, jimž dáno, aby škodili zemi a mori,
\par 3 Rka: Neškodte zemi, ani mori, ani stromum, dokudž neznamenáme služebníku Boha našeho na celích jejich.
\par 4 I slyšel jsem pocet znamenaných, sto ctyridceti a ctyri tisíce jich zapeceteno ze všech pokolení synu Izraelských.
\par 5 Z pokolení Juda dvanácte tisícu znamenaných, z pokolení Ruben dvanácte tisícu znamenaných, z pokolení Gád dvanácte tisícu znamenaných;
\par 6 Z pokolení Aser dvanácte tisícu znamenaných, z pokolení Neftalím dvanácte tisícu znamenaných, z pokolení Manases dvanácte tisícu znamenaných;
\par 7 Z pokolení Simeon dvanácte tisícu znamenaných, z pokolení Léví dvanácte tisícu znamenaných, z pokolení Izachar dvanácte tisícu znamenaných;
\par 8 Z pokolení Zabulon dvanácte tisícu znamenaných, z pokolení Jozef dvanácte tisícu znamenaných, z pokolení Beniamin dvanácte tisícu znamenaných.
\par 9 Potom pohledel jsem, a aj, zástup veliký, kteréhož by žádný precísti nemohl, ze všech národu a pokolení a lidí i jazyku, ani stojí pred trunem a pred oblicejem Beránka, obleceni jsouce v bílé roucho, a palmy v rukou jejich.
\par 10 A volali hlasem velikým, rkouce: Spasení jest od Boha našeho na trunu sedícího a od Beránka.
\par 11 A všickni andelé stáli okolo trunu a starcu a ctyr zvírat. I padli pred trunem na tvári své a klaneli se Bohu,
\par 12 Rkouce: Amen. Požehnání, a sláva, a moudrost, a díku cinení, a cest, a moc, i síla Bohu našemu na veky veku. Amen.
\par 13 I promluvil ke mne jeden z tech starcu, a rekl mi: Kdo jsou tito, kteríž obleceni jsou v bílé roucho? A odkud prišli?
\par 14 I rekl jsem jemu: Pane, ty víš. I rekl mi: To jsou ti, kteríž prišli z velikého soužení, a umyli roucha svá, a zbílili je ve krvi Beránkove.
\par 15 Protož jsou pred trunem Božím, a slouží jemu dnem i nocí v chráme jeho; a ten, jenž sedí na trunu, prebývati bude s nimi.
\par 16 Nebudout lacneti ani žízniti více, a nebude bíti na ne slunce, ani žádné horko.
\par 17 Nebo Beránek, kterýž jest uprostred trunu, pásti je bude, a dovedet je k živým studnicím vod, a setre Buh všelikou slzu s ocí jejich.

\chapter{8}

\par 1 A když otevrel pecet sedmou, stalo se na nebi mlcení, jako za pul hodiny.
\par 2 I videl jsem sedm andelu, kteríž stojí pred oblicejem Božím, jimž dáno sedm trub.
\par 3 A jiný andel prišel a postavil se pred oltárem, maje kadidlnici zlatou. I dáni jsou jemu zápalové mnozí, aby je obetoval s modlitbami všechnech svatých na oltári zlatém, kterýž jest pred trunem.
\par 4 I vstoupil dým zápalu s modlitbami svatých z ruky andela až pred oblicej Boží.
\par 5 I vzal andel kadidlnici, a naplnil ji ohnem s oltáre, a svrhl ji na zem. I stalo se hromobití, a hlasové, a blýskání, i zeme tresení.
\par 6 A tech sedm andelu, kteríž meli sedm trub, pripravili se, aby troubili.
\par 7 I zatroubil první andel, a stalo se krupobití a ohen smíšený se krví, a svrženo jest to na zem. A tretí díl stromu shorel, a všecka tráva zelená spálena jest.
\par 8 Potom zatroubil druhý andel, a uvržena jest do more jako hora veliká ohnem horící. I ucinen jest krví tretí díl more.
\par 9 A zemrel v mori tretí díl stvorení tech, kteráž mela duši, a tretí díl lodí zhynul.
\par 10 A vtom tretí andel zatroubil, i spadla s nebe hvezda veliká, horící jako pochodne, a padla na tretí díl rek a do studnic vod.
\par 11 A jméno hvezdy té bylo Pelynek. I obrátil se tretí díl vod v pelynek, a mnoho lidí zemrelo od tech vod, nebo byly zhorkly.
\par 12 Potom ctvrtý andel zatroubil, i uderena jest tretina slunce, a tretina mesíce, a tretí díl hvezd, takže se tretí díl jich zatmel, a tretina dne nesvítila, a též podobne i noci.
\par 13 I videl jsem a slyšel andela jednoho, an letí po prostredku nebe, a praví hlasem velikým: Beda, beda, beda tem, kteríž prebývají na zemi, pro jiné hlasy trub trí andelu, kteríž mají troubiti.

\chapter{9}

\par 1 Potom andel pátý zatroubil, i videl jsem, ano hvezda s nebe spadla na zem; a dán jest tomu andelu klíc od studnice propasti.
\par 2 Kterýžto otevrel studnici propasti. I vyšel dým z studnice jako dým z peci veliké, i zatmelo se slunce i povetrí pro dým studnice.
\par 3 A z toho dýmu vyšly kobylky na zemi, jimžto dána jest moc taková, jakouž moc mají štírové zemští.
\par 4 Ale receno jim, aby neškodily tráve zemské, ani cemu zelenému, ani kterémukoli stromoví, než toliko lidem, kteríž nemají znamení Božího na celích svých.
\par 5 Dáno jest pak jim, ne aby je zabíjely, ale trápily za pet mesícu, a aby trápení jejich bylo jako trápení od štíra, když by uštknul cloveka.
\par 6 A v tech dnech hledati budou lidé smrti, ale nenaleznou jí, a žádati budou zemríti, ale smrt utece od nich.
\par 7 Zpusob pak tech kobylek podobný byl konum pripraveným k boji, a na hlavách jejich byly jako koruny podobné zlatu, a tvári jejich jako tvári lidské.
\par 8 A mely vlasy jako vlasy ženské, a zuby jejich byly jako zubové lvu.
\par 9 Mely také pancíre jako pancíre železné, a zvuk krídel jejich jako zvuk vozu, když množství koní beží k boji.
\par 10 A mely ocasy podobné štírum, a žihadla v ocasích jejich byla, a moc jejich byla škoditi lidem za pet mesícu.
\par 11 A mely nad sebou krále, andela propasti, jemuž jméno Židovsky Abaddon, a recky Apollyon.
\par 12 Bída jedna pominula, a aj, prijdou ješte dve bídy potom.
\par 13 Tedy šestý andel zatroubil, i slyšel jsem hlas jeden ze ctyr rohu oltáre zlatého, kterýž jest pred oblicejem Božím,
\par 14 Rkoucí k šestému andelu, kterýž mel troubu: Rozvež ty ctyri andely, kteríž jsou u vezení pri veliké rece Eufrates.
\par 15 I rozvázáni jsou ti ctyri andelé, kteríž byli pripraveni k hodine, a ke dni, a k mesíci, a k roku, aby zmordovali tretí díl lidí.
\par 16 A byl pocet vojska lidu jízdného dvestekrát tisíc tisícu; nebo slyšel jsem pocet jejich.
\par 17 A též videl jsem u videní, kone a ti, kteríž sedeli na nich, meli pancíre ohnivé a hyacintové a z síry. Hlavy pak tech konu byly jako hlavy lvové, a z úst jejich vycházel ohen, a dým, a síra.
\par 18 A tímto trojím zmordována jest tretina lidu, totiž ohnem, a dýmem, a sirou, kteréž vycházely z úst jejich.
\par 19 Nebo moc jejich jest v ústech jejich a v ocasích jejich; ocasové zajisté jejich jsou podobni hadum, majíce hlavy, jimiž škodí.
\par 20 Jiní pak lidé, kteríž nejsou zmordováni temi ranami, necinili pokání z skutku rukou svých, aby se neklanely dáblum, a modlám zlatým, a stríbrným, a medeným, i kamenným, i dreveným, kteréžto ani hledeti nemohou, ani slyšeti, ani choditi.
\par 21 Aniž cinili pokání z vražd svých, ani z trávení svých, ani z smilstva svého, ani z krádeží svých.

\chapter{10}

\par 1 Videl jsem jiného andela silného, sstupujícího s nebe, odeného oblakem, a duha na hlave jeho byla, a tvár jeho jako slunce, a nohy jeho jako sloupové ohniví.
\par 2 A v ruce své mel knížky otevrené. I postavil nohu svou pravou na mori a levou na zemi.
\par 3 A volal hlasem velikým, jako by lev rval. A když dokonal volání, mluvilo sedm hromu hlasy své.
\par 4 A když odmluvilo sedm hromu hlasy své, byl bych to psal. Ale slyšel jsem hlas s nebe rkoucí ke mne: Zapecet to, co mluvilo sedm hromu, než nepiš toho.
\par 5 Tedy andel, kteréhož jsem videl stojícího na mori a na zemi, pozdvihl ruky své k nebi,
\par 6 A prisáhl skrze živého na veky veku, kterýž stvoril nebe i to, což v nem jest, a zemi i to, což na ní jest, a more i to, což v nem jest, že již více casu nebude.
\par 7 Ale ve dnech hlasu sedmého andela, když bude troubiti, dokonánot bude tajemství Boží, jakož zvestoval služebníkum svým prorokum.
\par 8 A ten hlas, kterýž jsem byl slyšel s nebe, opet mluvil se mnou a rekl: Jdi, a vezmi ty knížky otevrené z ruky andela, stojícího na mori a na zemi.
\par 9 I šel jsem k andelu, a rekl jsem jemu: Dej mi ty knížky. I rekl mi: Vezmi, a požri je, a ucinít horkost v briše tvém, ale v ústech tvých budout sladké jako med.
\par 10 I vzal jsem knížky z ruky andela, a požrel jsem je. I byly v ústech mých sladké jako med, ale když jsem je požrel, horko mi bylo v briše mém.
\par 11 I dí mi: Musíš opet prorokovati lidem, a národum, a jazykum, i králum mnohým.

\chapter{11}

\par 1 I dána mi trtina, podobná prutu, a postavil se tu andel, a rekl mi: Vstan, a zmer chrám Boží, i oltár, i ty, kteríž se modlí v nem.
\par 2 Ale sín, kteráž vne jest, vyvrz ven, a nemer jí. Nebot jest dána pohanum, a budout tlaciti mesto svaté za ctyridceti a dva mesíce.
\par 3 Ale dám jej dvema svedkum svým, kteríž budou prorokovati tisíc dve ste a šedesáte dnu, obleceni jsouce v pytle.
\par 4 Tit jsou dve olivy a dva svícnové, stojící pred oblicejem Boha Pána vší zeme.
\par 5 Jimžto jestliže by kdo chtel ubližovati, ohen vyjde z úst jejich, a sžíre neprátely jejich; a takt musí zabit býti, kdož by koli jim škoditi chtel.
\par 6 Tit mají moc zavríti nebe, aby nepršel déšt za dnu proroctví jejich; a mají moc nad vodami, obrátiti je v krev, a bíti zemi všelikou ranou, kolikrát by koli chteli.
\par 7 Ale když dokonají svedectví své, šelma vystupující z propasti válku povede proti nim, a zvítezí nad nimi, i zmorduje je.
\par 8 A ležeti budou tela jejich mrtvá na ryncích mesta velikého, kteréž slove duchovne Sodoma a Egypt, kdežto i Pán náš ukrižován jest.
\par 9 A vídati budou mnozí z pokolení, a z lidu, a z jazyku, i z národu tela jejich mrtvá za pul ctvrta dne, ale tel jejich mrtvých nedadí pochovati v hrobích.
\par 10 Anobrž ti, kteríž prebývají na zemi, radovati se budou nad nimi a veseliti se, a dary budou posílati jedni druhým; nebo ti dva proroci trápili ty, jenž prebývají na zemi.
\par 11 Ale po pul ctvrtu dni duch života od Boha poslaný vstoupil do nich, i postavili se na nohách svých, a bázen veliká spadla na ty, kteríž je videli.
\par 12 Potom slyšeli hlas veliký s nebe, rkoucí k nim: Vstupte sem! I vstoupili na nebe v oblace, a hledeli na ne neprátelé jejich.
\par 13 A v tu hodinu stalo se zeme tresení veliké, a desátý díl mesta padl, a zbito jest v tom zeme tresení sedm tisícu lidu, a jiní zstrašeni jsou, a vzdali slávu Bohu nebeskému.
\par 14 Bída druhá pominula, a aj, tretí bída prijde rychle.
\par 15 A sedmý andel zatroubil, i stali se hlasové velicí v nebi, rkoucí: Ucinenat jsou království všeho sveta Pána našeho a Krista jeho, a kralovatit bude na veky veku.
\par 16 Tedy tech ctyrmecítma starcu, kteríž pred oblicejem Božím sedí na stolicích svých, padli na tvári své, a klaneli se Bohu,
\par 17 Rkouce: Díky ciníme tobe, Pane Bože všemohoucí, Kterýž jsi, a Kterýžs byl, a Kterýž prijíti máš; nebo jsi prijal moc svou velikou a království ujal.
\par 18 I rozhnevali se národové, a prišel hnev tvuj, a cas mrtvých, aby souzeni byli, a aby dána byla odplata služebníkum tvým prorokum, i jiným svatým i všechnem, jenž se bojí jména tvého, malým i velikým, a aby zkaženi byli ti, kteríž nakažují zemi.
\par 19 I otevrín jest chrám Boží na nebi, a vidína jest truhla smlouvy jeho v chráme jeho. I stalo se blýskání, a hlasové, a hromobití, a zeme tresení, i krupobití veliké.

\chapter{12}

\par 1 I ukázal se div veliký na nebi: Žena odená sluncem,pod jejímiž nohama byl mesíc a na jejíž hlave byla koruna dvanácti hvezd.
\par 2 A jsuci tehotná, kricela, pracujici ku porodu a trápeci se, aby porodila.
\par 3 I vidín jest jiný div na nebi. Nebo aj, drak veliký ryšavý ukázal se, maje hlav sedm a rohu deset, a na tech hlavách svých sedm korun.
\par 4 Jehož ocas strhl tretinu hvezd s nebe, a svrhl je na zem. Ten pak drak stál pred ženou, kteráž mela poroditi, aby hned, jakž by porodila, syna jejího sežral.
\par 5 I porodila syna pacholíka, kterýž mel spravovati všecky národy prutem železným. I vytržen jest syn její k Bohu a k trunu jeho.
\par 6 A žena utekla na poušt, kdež mela místo od Boha pripravené, aby ji tam živili za dnu tisíc dve ste a šedesát.
\par 7 I stal se boj na nebi: Michal a andelé jeho bojovali s drakem, a drak bojoval i andelé jeho.
\par 8 Ale nic neobdrželi, aniž jest nalezeno více místo jejich na nebi.
\par 9 I svržen jest drak ten veliký, had starý, jenž slove dábel a satanáš, kterýž svodí všecken okršlek sveta; svržen jest, pravím, na zem, i andelé jeho s ním svrženi jsou.
\par 10 A uslyšel jsem hlas veliký, rkoucí na nebi: Nyní stalo se spasení, a moc, i království Boha našeho, a moc Krista jeho; nebo svržen jest žalobník bratrí našich, kterýž žaloval na ne pred oblicejem Boha našeho dnem i nocí.
\par 11 Ale oni zvítezili nad ním skrze krev Beránka, a skrze slovo svedectví svého, a nemilovali duší svých až do smrti.
\par 12 Protož veselte se nebesa, a kteríž prebýváte v nich. Beda tem, jenž prebývají na zemi a na mori; nebo dábel sstoupil k vám, maje hnev veliký, veda, že krátký cas má.
\par 13 A když uzrel drak, že jest svržen na zem, honil ženu tu, kteráž porodila toho pacholíka.
\par 14 Ale dána jsou žene dve krídla orlice veliké, aby letela od tvári hada na poušt, na místo své, kdež by ji živili do casu a casu, a do pul casu.
\par 15 I vypustil had z úst svých po žene vodu jako reku, aby ji reka zachvátila.
\par 16 Ale zeme pomohla žene; neb otevrela zeme ústa svá a požrela reku, kterouž vypustil drak z úst svých.
\par 17 I rozhneval se drak na tu ženu, a šel bojovati s jinými z semene jejího, kteríž ostríhají prikázání Božích a mají svedectví Ježíše Krista.
\par 18 I stál jsem na písku morském.

\chapter{13}

\par 1 I videl jsem šelmu vystupující z more, kteráž mela sedm hlav a deset rohu. A na rozích jejích deset korun, a na hlavách jejích jméno rouhání.
\par 2 Byla pak šelma ta, kterouž jsem videl, podobná pardovi, a nohy její jako nedvedí, a ústa její jako ústa lvová. I dal jí drak sílu svou, a trun svuj, a moc velikou.
\par 3 A videl jsem, ano jedna z hlav jejích jako zbitá byla až na smrt, ale rána její smrtelná uzdravena jest. Tedy divivši se všecka zeme, šla za tou šelmou.
\par 4 A klaneli se drakovi, kterýž dal moc šelme; a klaneli se šelme, rkouce: Kdo jest podobný té šelme? Kdo bude moci bojovati s ní?
\par 5 I dána jsou jí ústa mluvící veliké veci a rouhání, a dána jí moc vládnouti za ctyridceti a dva mesíce.
\par 6 Protož otevrela ústa svá k rouhání se Bohu, aby se rouhala jménu jeho, i stánku jeho, i tem, kteríž prebývají na nebi.
\par 7 Dáno jí i bojovati s svatými a premáhati je; a dána jí moc nad všelikým pokolením, nad rozlicným jazykem, i nad národem.
\par 8 Protož klaneti se jí budou všickni, kteríž prebývají na zemi, jichžto jména nejsou napsaná v knihách života Beránka, toho zabitého od pocátku sveta.
\par 9 Má-li kdo uši, slyš.
\par 10 Jestliže kdo do vezení povede, do vezení pujde; zabije-li kdo mecem, musí mecem zabit býti. Zdet jest trpelivost a víra svatých.
\par 11 Potom videl jsem jinou šelmu vystupující z zeme, a mela dva rohy, podobné rohum Beránkovým, ale mluvila jako drak.
\par 12 Kterážto všecku moc první šelmy provozuje pred tvárí její; a pusobí to, že zeme i ti, kteríž prebývají na ní, klanejí se šelme té první, jejížto rána smrtelná uzdravena byla.
\par 13 A ciní divy veliké, takže i ohni rozkazuje sstupovati s nebe na zem pred oblicejem mnohých lidí.
\par 14 A svodí ty, jenž prebývají na zemi, temi divy, kteréžto dáno jí ciniti pred oblicejem šelmy, ríkající obyvatelum zeme, aby udelali obraz šelme, kteráž mela ránu od mece, a ožila zase.
\par 15 I dáno jí, aby dáti mohla ducha tomu obrazu šelmy, aby i mluvil obraz šelmy, a aby to zpusobila, kterížkoli neklaneli by se obrazu šelmy, aby byli zmordováni.
\par 16 A rozkazuje všechnem, malým i velikým, bohatým i chudým, svobodným i v službu podrobeným, aby meli znamení na pravé ruce své, aneb na celích svých,
\par 17 A aby žádný nemohl kupovati ani prodávati, než ten, kdož má znamení aneb jméno té šelmy, anebo pocet jména jejího.
\par 18 Tut jest moudrost. Kdo má rozum, sectiž pocet šelmy. Nebo jest pocet cloveka, a jestit pocet ten šest set šedesáte a šest.

\chapter{14}

\par 1 Tedy pohledel jsem, a aj, Beránek stál na hore Sion, a s ním sto ctyridceti a ctyri tisícu majících jméno Otce jeho napsané na celích svých.
\par 2 A slyšel jsem hlas s nebe, jako hlas vod mnohých, a jako hlas hromu velikého. A hlas slyšel jsem, jako tech, kteríž hrají na harfy své.
\par 3 A zpívali jakožto písen novou, pred trunem, a pred ctyrmi zvíraty, a pred starci, a žádný nemohl se nauciti té písni, jediné tech sto ctyridceti a ctyri tisíce, kteríž jsou koupeni z zeme.
\par 4 Totot jsou ti, kteríž s ženami nejsou poskvrneni; nebo panicové jsou. Tit jsou, jenž následují Beránka, kamž by koli šel; tit jsou koupeni ze všech lidí, prvotiny Bohu a Beránkovi.
\par 5 A v ústech jejich není nalezena lest; nebot jsou bez úhony pred trunem Božím.
\par 6 I videl jsem jiného andela letícího po prostredku nebe, majícího evangelium vecné, aby je zvestoval tem, jenž bydlí na zemi, a všelikému národu, i pokolení, i jazyku, i lidu,
\par 7 Rkoucího velikým hlasem: Bojte se Boha, a vzdejte jemu chválu, nebot prišla hodina soudu jeho; a klanejte se tomu, kterýž ucinil nebe i zemi i more i studnice vod.
\par 8 A jiný andel letel za ním, rka: Padl, padl Babylon, mesto to veliké, nebo vínem hnevu smilství svého napájelo všecky národy.
\par 9 A tretí andel letel za nimi, prave velikým hlasem: Bude-li se kdo klaneti šelme a obrazu jejímu, a vezme-li znamení její na celo své aneb na ruku svou,
\par 10 I tent také bude píti víno hnevu Božího, víno, pravím, kteréž jest vlito do kalichu hnevu jeho; a trápen bude ohnem a sirou pred oblicejem andelu svatých a pred oblicejem Beránka.
\par 11 A dým muk jejich vstoupít na veky veku, a nebudout míti odpocinutí dnem i nocí ti, kteríž se klanejí šelme a obrazu jejímu, a jestliže kdo prijme znamení jména jejího.
\par 12 Tut jest trpelivost svatých, tu jsou ti, kteríž ostríhají prikázání Božích a víry Ježíšovy
\par 13 I slyšel jsem hlas s nebe, rkoucí ke mne: Piš: Blahoslavení jsou od této chvíle mrtví, kteríž v Pánu umírají. Duch zajisté dí jim, aby odpocinuli od prací svých, nebo skutkové jejich následují jich.
\par 14 I pohledel jsem, a aj, oblak belostkvoucí, a na tom oblaku sedel podobný Synu cloveka, maje na hlave své korunu zlatou a v ruce své srp ostrý.
\par 15 A jiný andel vyšel z chrámu, volaje hlasem velikým na toho, kterýž sedel na oblaku: Pust srp svuj a žni; nebt jest prišla tobe hodina žni, nebo již dozrala žen zeme.
\par 16 I spustil ten, kterýž sedel na oblaku, srp svuj na zem, a požata jest zeme.
\par 17 A jiný andel vyšel z chrámu toho, kterýž jest na nebi, maje i on srp ostrý.
\par 18 Opet vyšel jiný andel z oltáre, kterýž mel moc nad ohnem, a volal krikem velikým na toho, jenž mel srp ostrý, rka: Pust srp svuj ostrý, a zber hrozny vinice zemské; nebot jsou uzrali hroznové její.
\par 19 I spustil andel srp svuj na zemi, a sebral vinici zeme, a vmetal hrozny do jezera velikého hnevu Božího.
\par 20 I tlaceno jest jezero pred mestem, a vyšla krev z jezera až do udidl konum za tisíc a za šest set honu.

\chapter{15}

\par 1 Potom videl jsem jiný zázrak na nebi veliký a predivný: Sedm andelu majících sedm ran posledních, v nichž má dokonán býti hnev Boží.
\par 2 A videl jsem jako more sklené, smíšené s ohnem, a ty, kteríž zvítezili nad šelmou a obrazem jejím a nad charakterem jejím i nad poctem jména jejího, ani stojí nad tím morem skleným, majíce harfy Boží,
\par 3 A zpívají písen Mojžíše, služebníka Božího, a písen Beránkovu, rkouce: Velicí a predivní jsou skutkové tvoji, Pane Bože všemohoucí, spravedlivé a pravé jsou cesty tvé, ó Králi všech svatých.
\par 4 Kdož by se nebál tebe, Pane, a nezveleboval jména tvého? Nebo ty sám svatý jsi. Všickni zajisté národové prijdou a klaneti se budou pred oblicejem tvým; nebo soudové tvoji zjeveni jsou.
\par 5 Potom pak videl jsem, a aj, otevrín jest chrám stánku svedectví na nebi.
\par 6 I vyšlo sedm tech andelu z chrámu, majících sedm ran, obleceni jsouce rouchem lneným, cistým a belostkvoucím, a prepásáni na prsech pasy zlatými.
\par 7 A jedno ze ctyr zvírat dalo sedmi andelum sedm koflíku zlatých, plných hnevu Boha živého na veky veku.
\par 8 I naplnen jest chrám dýmem, pocházejícím od slávy Boží a od moci jeho, a žádný nemohl vjíti do chrámu, dokudž se nevykonalo sedm ran tech sedmi andelu.

\chapter{16}

\par 1 I slyšel jsem hlas veliký z chrámu, rkoucí sedmi andelum: Jdete, vylijte sedm koflíku hnevu Božího na zemi.
\par 2 I šel první andel a vylil koflík svuj na zemi, i vyvrhli se vredové škodliví a zlí na lidech, majících znamení šelmy, a na tech, kteríž se klaneli obrazu jejímu.
\par 3 Za tím druhý andel vylil koflík svuj na more, a ucineno jest jako krev umrlého, a všeliká duše živá v mori umrela.
\par 4 Potom tretí andel vylil koflík svuj na reky a na studnice vod, i obráceny jsou v krev.
\par 5 I slyšel jsem andela vod, rkoucího: Spravedlivý jsi, Pane, Kterýž jsi, a Kterýž jsi byl, a Svatý, žes to usoudil.
\par 6 Nebot jsou krev svatých a proroku vylévali, i dal jsi jim krev píti; hodnit jsou zajisté toho.
\par 7 I slyšel jsem jiného andela od oltáre, kterýž rekl: Jiste, Pane Bože všemohoucí, praví jsou a spravedliví soudové tvoji.
\par 8 Potom ctvrtý andel vylil koflík svuj na slunce, i dáno jest jemu trápiti lidi ohnem.
\par 9 I pálili se lidé horkem velikým, a rouhali se jménu Boha toho, kterýž má moc nad temito ranami, avšak necinili pokání, aby vzdali slávu jemu.
\par 10 Tedy pátý andel vylil koflík svuj na stolici té šelmy, i ucineno jest království její tmavé, i kousali lidé jazyky své pro bolest.
\par 11 A rouhali se Bohu nebeskému pro bolesti své a pro vredy své, avšak necinili pokání z skutku svých.
\par 12 Šestý pak andel vylil koflík svuj na tu velikou reku Eufrates, i vyschla voda její, aby pripravena byla cesta králum od východu slunce.
\par 13 A videl jsem, ano z úst draka a z úst šelmy a z úst falešného proroka vyšli tri duchové necistí, podobní žabám.
\par 14 Nebo jsout duchové dábelští, ješto ciní divy a chodí mezi krále zemské a všeho sveta, aby je shromáždili k boji, k tomu velikému dni Boha všemohoucího.
\par 15 Aj, pricházímt jako zlodej. Blahoslavený, kdož bdí a ostríhá roucha svého, aby nah nechodil, aby nevideli hanby jeho.
\par 16 I shromáždil je na místo, kteréž slove Židovsky Armageddon.
\par 17 Tedy sedmý andel vylil koflík svuj na povetrí, i vyšel hlas veliký z chrámu nebeského, od trunu, rkoucí: Stalo se.
\par 18 I stali se zvukové a hromobití a blýskání, i zeme tresení stalo se veliké, jakéhož nikdy nebylo, jakž jsou lidé na zemi, totiž zeme tresení tak velikého.
\par 19 I roztrhlo se to veliké mesto na tri strany, a mesta národu padla. A Babylon veliký prišel na pamet pred oblicejem Božím, aby dal jemu kalich vína prchlivosti hnevu svého.
\par 20 A všickni ostrovové pominuli, a hory nejsou nalezeny.
\par 21 A kroupy veliké jako centnérové pršely s nebe na lidi. I rouhali se Bohu lidé pro ránu tech krup; nebo velmi veliká byla ta rána.

\chapter{17}

\par 1 I prišel jeden z sedmi andelu, kteríž meli sedm koflíku, a mluvil se mnou, rka ke mne: Pojd, ukážit odsouzení nevestky veliké, kteráž sedí na vodách mnohých,
\par 2 S kteroužto smilnili králové zeme a zpili se vínem smilství jejího obyvatelé zeme.
\par 3 I odnesl mne na poušt v duchu, a videl jsem ženu sedící na šelme brunátné; a ta šelma plná byla jmen rouhání, a mela sedm hlav a deset rohu.
\par 4 Žena pak odína byla šarlatem a brunátným rouchem, a ozdobena zlatem a kamením drahým i perlami, mající koflík zlatý v ruce své, plný ohavností a necistoty smilstva svého.
\par 5 A na cele jejím napsané jméno: Tajemství, Babylon veliký, Máte smilstva a ohavností zeme.
\par 6 A videl jsem ženu tu opilou krví svatých a krví mucedlníku Ježíšových, a videv ji, divil jsem se divením velikým.
\par 7 I rekl mi andel: Co se divíš? Já tobe povím tajemství té ženy i šelmy, kteráž ji nese, mající hlav sedm a rohu deset.
\par 8 Šelma, kterous videl, byla, a již není, a mát vystoupiti z propasti a na zahynutí jíti. I diviti se budou bydlitelé zeme, (ti, kterýchžto jména nejsou napsána v knihách života od ustanovení sveta,) vidouce šelmu, kteráž byla, a není, avšak jest.
\par 9 Tentot pak jest smysl toho, a mát zavrenou v sobe moudrost: Sedm hlav jestit sedm hor, na kterýchž ta žena sedí.
\par 10 A králu sedm jest. Pet jich padlo, jeden jest, a jiný ješte neprišel; a když prijde, na malou chvíli musí trvati.
\par 11 A šelma, kteráž byla a není, onat jest osmý, a z sedmit jest, i na zahynutí jde.
\par 12 Deset pak rohu, kteréžs videl, jestit deset králu, kteríž ješte království neprijali, ale prijmout moc jako králové, jedné hodiny spolu s šelmou.
\par 13 Tit jednu radu mají, a sílu i moc svou šelme dadí.
\par 14 Tit bojovati budou s Beránkem, a Beránek zvítezí nad nimi, nebot Pán pánu jest a Král králu, i ti, kteríž jsou s ním, povolaní, a vyvolení, a verní.
\par 15 I rekl mi: Vody, kteréžs videl, kdež nevestka sedí, jsout lidé, a zástupové, a národové, a jazykové.
\par 16 Deset pak rohu, kteréžs videl na šelme, ti v nenávist vezmou nevestku, a uciní ji opuštenou a nahou, a telo její jísti budou, a ji páliti budou ohnem.
\par 17 Nebot dal Buh v srdce jejich, aby cinili vuli jeho a aby se sjednomyslnili, a toliko dotud království své šelme dali, dokudž by nebyla vykonána slova Boží.
\par 18 A žena, kteroužs videl, jestit mesto to veliké, kteréž má království nad králi zeme.

\chapter{18}

\par 1 Potom pak videl jsem andela sstupujícího s nebe, majícího moc velikou, a zeme osvícena byla od slávy jeho.
\par 2 I zkrikl silne hlasem velikým, rka: Padl, padl Babylon ten veliký, a ucinen jest príbytkem dáblu, a stráží každého ducha necistého, a stráží všelikého ptactva necistého a ohyzdného.
\par 3 Nebo z vína hnevu smilstva jeho pili všickni národové, a králové zemští smilnili s ním, a kupci zemští z zboží rozkoší jeho zbohatli.
\par 4 I slyšel jsem jiný hlas s nebe, rkoucí: Vyjdete z neho, lide muj, abyste se nepriúcastnovali hríchum jeho a abyste neprijali z jeho ran.
\par 5 Nebot dosáhli hríchové jeho až k nebi a rozpomenul se Buh na nepravosti jeho.
\par 6 Dejtež jemu, jakož i on dával vám, a dejte jemu dvénásob podle skutku jeho; v koflík, kterýž naléval vám, nalijte jemu to dvénásob.
\par 7 Jakž se mnoho chlubil a zbujnel byl, tak mnoho dejte jemu muk a pláce. Nebot v srdci svém praví: Sedím královna, a nejsemt vdovou, a pláce neuzrím.
\par 8 Protož v jeden den prijdout rány jeho, smrt a plác i hlad, a ohnem spáleno bude; nebo silný jest Pán, kterýž je odsoudí.
\par 9 I plakati ho budou a kvíliti nad ním králové zeme, kteríž s ním smilnili a svou rozkoš meli, když uzrí dým zapálení jeho,
\par 10 Zdaleka stojíce, pro bázen muk jeho, a rkouce: Beda, beda, veliké mesto Babylon, to mesto silné, že jest jedné hodiny prišel odsudek tvuj.
\par 11 Ano i kupci zemští budou plakati a kvíliti nad ním; nebo koupí jejich žádný nebude kupovati více,
\par 12 Koupe od zlata, a stríbra, a drahého kamene, i perel, i kmentu, šarlatu, i hedbáví, i brunátného roucha, i všelikého dreva tyinového, a všech nádob z kostí slonových, i všelikého nádobí z nejdražšího dríví, i z medi, i z železa, i z mramoru;
\par 13 A kupectví skorice, i vonných vecí, a mastí, i kadidla, i vína, i oleje, i belí, i pšenice, i dobytka, i ovec, i koní, a vozu, i služebníku, i duší lidských;
\par 14 I ovoce žádostivá duši tvé odešla od tebe, a všecko tucné a krásné odešlo od tebe, a aniž toho již více nalezneš.
\par 15 Ti, pravím, kupci, kteríž v tom kupcili a jím zbohatli, zdaleka stanou pro strach muk jeho, placíce, a kvílíce,
\par 16 A rkouce: Beda, beda, ó mesto veliké, kteréž odíno kmentem, a šarlatem, a brunátným rouchem, a ozdobeno bylo zlatem, a kamením drahým, i perlami! nebo v jednu hodinu zahynula tak veliká bohatství.
\par 17 Ano i všeliký správce lodí morských, i všecko množství lidí, kteríž jsou na lodech, i plavci, i ti, kteríž svou živnost na mori mají, zdaleka stanou,
\par 18 A zkriknou, vidouce dým zapálení jeho, rkouce: Které mesto bylo podobné tomuto velikému mestu!
\par 19 A sypouce prach na hlavy své, kriceti budou, placíce, a kvílíce, a rkouce: Beda, beda, mesto veliké, v nemžto zbohatli všickni, kteríž meli lodí na mori, ze mzdy jeho; nebo jedné hodiny zpustlo.
\par 20 Ale raduj se nad ním nebe i svatí apoštolé i proroci; nebo pomstilt vás Buh nad ním.
\par 21 I zdvihl jeden silný andel kámen jako žernov veliký a hodil jím do more, rka: Tak prudce uvržen bude Babylon, to mesto veliké, a již více nebude nalezeno.
\par 22 A hlas na harfy hrajících, a zpeváku, a pištcu, a trubacu nebude více v tobe slyšán, a žádný remeslník žádného remesla nebude v tobe více nalezen, a zvuk žernovu nebude v tobe více slyšán.
\par 23 Ani svetlo svíce nebude v tobe více svítiti, a hlas ženicha ani nevesty nebude v tobe více slyšán, ješto kupci tvoji bývala knížata zemská, a tráveními tvými v blud uvedeni byli všickni národové.
\par 24 Ale v nem nalezena jest krev proroku a svatých i všech zmordovaných na zemi.

\chapter{19}

\par 1 Potom jsem slyšel veliký hlas velikého zástupu na nebi, rkoucího: Haleluiah. Spasení, a sláva, a cest, i moc Pánu Bohu našemu.
\par 2 Nebot jsou praví a spravedliví soudové jeho, kterýž odsoudil nevestku tu velikou, jenž byla porušila zemi smilstvem svým, a pomstil krve služebníku svých z ruky její.
\par 3 I rekli po druhé: Haleluiah. A dým její vstupoval na veky veku.
\par 4 I padlo ctyrmecítma starcu a ctvero zvírat, a klaneli se Bohu sedícímu na trunu, rkouce: Amen, Haleluiah.
\par 5 I vyšel z trunu hlas, rkoucí: Chvalte Boha našeho všickni služebníci jeho, a kteríž se bojíte jeho, i malí i velicí.
\par 6 A slyšel jsem hlas jako zástupu velikého, a jako hlas vod mnohých, a jako zvuk hromu silných, rkoucích: Haleluiah. Kralovalt jest Pán Buh náš všemohoucí.
\par 7 Radujme se, a veselme se, a chválu vzdejme jemu. Nebot jest prišla svadba Beránkova, a manželka jeho pripravila se.
\par 8 A dáno jest jí, aby se oblékla v kment cistý a stkvoucí, a ten kment jsout ospravedlnování svatých.
\par 9 I rekl mi: Piš: Blahoslavení, kteríž jsou k veceri svadby Beránkovy povoláni. A rekl mi: Tatot slova Boží jsou pravá.
\par 10 I padl jsem k nohám jeho, klaneti se jemu chteje. Ale rekl mi: Hled, abys toho necinil. Nebot jsem spoluslužebník tvuj i bratrí tvých, tech, jenž mají svedectví Ježíšovo. Bohu se klanej! Svedectví pak Ježíšovo jestit duch proroctví.
\par 11 I videl jsem nebe otevrené, a aj, kun bílý, a ten, jenž sedel na nem, sloul Verný a Pravý, a kterýž v spravedlnosti soudí i bojuje.
\par 12 Oci pak jeho byly jako plamen ohne, a na hlave jeho korun množství, a mel jméno napsané, kteréhož žádný neví, než on sám.
\par 13 A byl odín v roucho pokropené krví, a slovet jméno jeho Slovo Boží.
\par 14 A rytírstvo nebeské jelo za ním na bílých koních, jsouce odíni v kment bílý a cistý.
\par 15 A z úst jeho vycházel mec ostrý, aby jím bil všecky národy. Nebo on bude je spravovati prutem železným; ont i pres vína hnevu a prchlivosti Boha všemohoucího tlaciti bude.
\par 16 A mát na rouchu a na bedrách svých napsané jméno: Král králu a Pán pánu.
\par 17 I videl jsem jednoho andela, an stojí v slunci, a zkrikl hlasem velikým, rka všechnem ptákum, kteríž létali po prostred nebes: Pojdte a shromaždte se k veceri velikého Boha,
\par 18 Abyste jedli tela králu, a tela hejtmanu, a tela silných, a tela konu, i tech, kteríž na nich sedí, a tela všech svobodných i služebníku, i malých i velikých.
\par 19 I videl jsem šelmu a krále zeme a vojska jejich, ani se sjeli, aby bojovali s tím, kterýž sedel na koni, a s rytírstvem jeho.
\par 20 I jata jest šelma, a s ní falešný prorok ten, kterýž ciníval divy pred ní, jimiž svodil ty, kteríž prijali znamení šelmy a kteríž se klaneli obrazu jejímu. I uvrženi jsou oba za živa do jezera ohnivého, horícího sirou.
\par 21 A jiní zbiti jsou mecem toho, kterýž sedel na koni, vycházejícím z úst jeho. A všickni ptáci nasyceni jsou tely jejich.

\chapter{20}

\par 1 I videl jsem andela sstupujícího s nebe, majícího klíc od propasti a retez veliký v ruce své.
\par 2 I chopil draka, hada toho starého, jenž jest dábel a satan, i svázal jej za tisíc let.
\par 3 A uvrhl ho do propasti, i zavrel jej tam a svrchu nad ním zapecetil, aby nesvodil více národu, až by se vyplnilo tisíc let; nebot potom musí býti propušten na malý cas.
\par 4 I videl jsem stolice, a posadili se na nich, i dán jest jim soud, a videl jsem duše stínaných pro svedectví Ježíšovo a pro slovo Boží, a kteríž se neklaneli šelme, ani obrazu jejímu, a aniž prijali znamení jejího na cela svá, anebo na ruce své. A ožili a kralovali s Kristem tisíc let.
\par 5 Jiní pak mrtví neožili, dokudž by se nevyplnilo tisíc let. A tot jest první vzkríšení.
\par 6 Blahoslavený a svatý, kdož má díl v prvním vzkríšení. Nad temit ta druhá smrt nemá moci, ale budou kneží Boží a Kristovi, a kralovati s ním budou tisíc let.
\par 7 A když se vyplní tisíc let, propušten bude satanáš z žaláre svého.
\par 8 I vyjde, aby svodil národy, kteríž jsou na ctyrech stranách zeme, Goga a Magoga, a aby je shromáždil k boji, kterýchžto pocet jest jako písku morského.
\par 9 I vstoupili na širokost zeme a obklícili stany svatých, i to mesto milé, ale sstoupil ohen od Boha s nebe a spálil je.
\par 10 A dábel, kterýž je svodil, uvržen jest do jezera ohne a síry, kdež jest i šelma, i falešný prorok, a budout muceni dnem i nocí na veky veku.
\par 11 I videl jsem trun veliký bílý, a sedícího na nem, pred jehož tvárí utekla zeme i nebe, a místo jim není nalezeno.
\par 12 I videl jsem mrtvé, malé i veliké, stojící pred oblicejem Božím, a knihy otevríny jsou. A jiné knihy také jsou otevríny, to jest knihy života, i souzeni jsou mrtví podle toho, jakž psáno bylo v knihách, totiž podle skutku svých.
\par 13 A vydalo more mrtvé, kteríž byli v nem, tolikéž smrt i peklo vydali ty, kteríž v nich byli, i souzeni jsou jeden každý podle skutku svých.
\par 14 Smrt pak a peklo uvrženi jsou do jezera ohnivého, a tot jest smrt druhá.
\par 15 I ten, kdož není nalezen v knihách života, uvržen jest do jezera ohnivého.

\chapter{21}

\par 1 Potom videl jsem nebe nové a zemi novou. Nebo první nebe a první zeme byla pominula, a more již nebylo.
\par 2 A já Jan videl jsem mesto svaté, Jeruzalém nový, sstupující od Boha s nebe, pripravený jako nevestu okrášlenou muži svému.
\par 3 I slyšel jsem hlas veliký s nebe, rkoucí: Aj, stánek Boží s lidmi, a bydlitit bude s nimi, a oni budou lid jeho, a on Buh s nimi bude, jsa jejich Bohem.
\par 4 A setret Buh všelikou slzu s ocí jejich, a smrti již více nebude, ani kvílení, ani kriku, ani bolesti nebude více; nebo první veci pominuly.
\par 5 I rekl ten, kterýž sedel na trunu: Aj, nové ciním všecko. I rekl mi: Napiš to. Nebot jsou tato slova verná a pravá.
\par 6 I dí mi: Již se stalo. Ját jsem Alfa i Omega, pocátek i konec. Ját žíznivému dám z studnice vody živé darmo.
\par 7 Kdož zvítezí, obdržít dedicne všecko, a budut jemu Bohem, a on mi bude synem.
\par 8 Strašlivým pak, a neverným, a ohyzdným, a vražedlníkum, a smilníkum, a carodejníkum, a modlárum, i všechnem lhárum, pripraven jest díl jejich v jezere, kteréž horí ohnem a sirou, jenž jest smrt druhá.
\par 9 I prišel ke mne jeden z sedmi andelu, kteríž meli sedm koflíku plných sedmi ran nejposlednejších, a mluvil se mnou, rka: Pojd, ukážit nevestu, manželku Beránkovu.
\par 10 I vnesl mne v duchu na horu velikou a vysokou, a ukázal mi mesto veliké, ten svatý Jeruzalém, sstupující s nebe od Boha,
\par 11 Mající slávu Boží. Jehož svetlost byla podobná k kameni nejdražšímu, jako k kameni jaspidu, kterýž by byl zpusobu krištálového,
\par 12 A melo zed velikou a vysokou, v níž bylo dvanácte bran, a na tech branách dvanácte andelu, a jména napsaná, kterážto jména jsou dvanáctera pokolení synu Izraelských.
\par 13 Od východu slunce brány tri, od pulnoci brány tri, od poledne brány tri, od západu brány tri.
\par 14 A zed mestská mela základu dvanácte, a na nich jména dvanácti apoštolu Beránkových.
\par 15 A ten, kterýž mluvil se mnou, mel trtinu zlatou, aby zmeril mesto i brány jeho i zed jeho.
\par 16 Položení mesta toho ctverhrané jest, jehož dlouhost tak veliká jest jako i širokost. I zmeril to mesto trtinou a vymeril dvanácte tisícu honu; dlouhost pak jeho, i širokost, i vysokost jednostejná jest.
\par 17 I zmeril zed jeho, a nameril sto ctyridceti a ctyri loktu, merou cloveka, kteráž jest míra andela.
\par 18 A bylo stavení zdi jeho jaspis, mesto pak samo zlato cisté, podobné sklu cistému.
\par 19 A základové zdi mestské všelikým kamenem drahým ozdobeni byli. Základ první byl jaspis, druhý zafir, tretí chalcedon, ctvrtý smaragd,
\par 20 Pátý sardonyx, šestý sardius, sedmý chryzolit, osmý beryllus, devátý topazion, desátý chryzoprassus, jedenáctý hyacint, dvanáctý ametyst.
\par 21 Dvanácte pak bran dvanácte perel jest, a jedna každá brána jest z jedné perly; a rynk mesta zlato cisté jako sklo, kteréž se naskrze prohlédnouti muže.
\par 22 Ale chrámu jsem v nem nevidel; nebo Pán Buh všemohoucí chrám jeho jest a Beránek.
\par 23 A to mesto nepotrebuje slunce ani mesíce, aby svítily v nem; nebo sláva Boží je osvecuje, a svíce jeho jest Beránek.
\par 24 A národové lidí k spasení prišlých, v svetle jeho procházeti se budou, a králové zemští prenesou slávu a cest svou do neho.
\par 25 A brány jeho nebudou zavírány ve dne; noci zajisté tam nebude.
\par 26 A snesou do neho slávu a cest národu.
\par 27 A nevejdet do neho nic poskvrnujícího, anebo pusobícího ohyzdnost a lež, než toliko ti, kteríž napsaní jsou v knihách života Beránkova.

\chapter{22}

\par 1 I ukázal mi potok cistý vody živé, svetlý jako krištál, tekoucí z trunu Božího a Beránkova.
\par 2 Uprostred pak rynku jeho a s obou stran potoka bylo drevo života, prinášející dvanáctero ovoce, na každý mesíc vydávající ovoce své, a listí své k zdraví národu.
\par 3 A niceho zloreceného již více nebude, ale trun Boží a Beránkuv bude v nem, a služebníci jeho sloužiti jemu budou.
\par 4 A tvár jeho videti budou, a jméno jeho budet na celích jejich.
\par 5 A noci tam nebude, aniž budou potrebovati svíce, ani svetla slunecného; nebo Pán Buh je osvecuje, a kralovati budou na veky veku.
\par 6 I rekl mi: Slova tato jsout verná a pravá, a Pán, jenž jest Buh svatých proroku, poslal andela svého, aby ukázal služebníkum svým, co se díti musí brzo.
\par 7 Aj, prijdut rychle. Blahoslavený, kdož ostríhá slov proroctví knihy této.
\par 8 Já pak Jan videl jsem a slyšel tyto veci. A když jsem slyšel a videl, padl jsem, klaneti se chteje pred nohama andela toho, kterýž mi tyto veci ukazoval.
\par 9 Ale rekl mi: Hled, abys toho necinil. Nebot jsem spoluslužebník tvuj a bratrí tvých proroku, a tech, jenž ostríhají slov knihy této. Bohu se klanej.
\par 10 Potom rekl mi: Nezapecetuj slov proroctví knihy této; nebot jest blízko cas.
\par 11 Kdo škodí, škodiž ješte; a kdo smrdí, smrdiž ješte; a kdo jest spravedlivý, ospravedlniž se ješte; a svatý posvetiž se ješte.
\par 12 A aj, prijdut brzo, a odplata má se mnou, abych odplatil jednomu každému podle skutku jeho.
\par 13 Ját jsem Alfa i Omega, pocátek i konec, první i poslední.
\par 14 Blahoslavení, kteríž zachovávají prikázání jeho, aby meli právo k drevu života a aby branami vešli do mesta.
\par 15 Vne pak budou psi a carodejníci, a smilníci, a vražedlníci, a modlári, i každý, kdož miluje a ciní lež.
\par 16 Já Ježíš poslal jsem andela svého, aby vám svedcil o techto vecech v církvích. Já jsem koren a rod Daviduv, a hvezda jasná a jitrní.
\par 17 A Duch i nevesta rkou: Pojd. A kdož slyší, rciž: Prijd. A kdož žízní, prijdiž, a kdo chce, naber vody života darmo.
\par 18 Osvedcujit pak každému, kdož by slyšel slova proroctví knihy této, jestliže by kdo pridal k temto vecem, žet jemu pridá Buh ran napsaných v knize této.
\par 19 A jestliže by kdo ujal neco od slov proroctví tohoto, odejmet Buh díl jeho z knihy života, a z mesta svatého, a z tech vecí, kteréž jsou napsány v knize této.
\par 20 Takt praví ten, kterýž svedectví vydává o techto vecech: Jistet prijdu brzo. Amen. Prijdiž tedy, Pane Ježíši.
\par 21 Milost Pána našeho Jezukrista se všemi vámi. Amen.


\end{document}