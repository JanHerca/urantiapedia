\begin{document}

\title{Deuteronomy}

\chapter{1}

\par 1 Tato jsou slova, kteráž mluvil Mojžíš ke všemu lidu Izraelskému pred Jordánem na poušti, na rovinách proti mori Rudému, mezi Fáran a Tofel a Lában a Hazerot a Dizahab.
\par 2 Jedenácte dní cesty jest od Oréb pres hory Seir až do Kádesbarne.
\par 3 Stalo se pak ctyridcátého léta, jedenáctého mesíce, v první den téhož mesíce, že mluvil Mojžíš synum Izraelským všecky veci, kteréž jemu byl prikázal Hospodin oznámiti jim,
\par 4 Kdyžto již byl zabil Seona, krále Amorejského, kterýž bydlil v Ezebon, a Oga, krále Bázan, kterýž bydlil v Astarot, zabil v Edrei.
\par 5 Pred Jordánem, v zemi Moábské, pocal Mojžíš vysvetlovati zákona tohoto, rka:
\par 6 Hospodin Buh náš mluvil k nám na Orébe, rka: Dosti jste již na hore této bydlili.
\par 7 Obratte se, táhnete a jdete k hore Amorejských, na všecko vukolí její, bud na roviny, na hory, na údolí, na poledne, i na brehy morské, k zemi Kananejské a k Libánu, až k rece veliké, k rece Eufrates.
\par 8 Ej, ukázal jsem vám tu zemi; vejdetež do ní, a dedicne vládnete jí, kterouž s prísahou zaslíbil dáti Hospodin otcum vašim, Abrahamovi, Izákovi a Jákobovi, i semeni jejich po nich.
\par 9 A mluvil jsem k vám toho casu, rka: Nemohut sám nésti vás.
\par 10 Hospodin Buh váš rozmnožil vás, a hle, rozmnoženi jste dnes jako hvezdy nebeské.
\par 11 (Hospodin Buh otcu vašich rozmnožiž vás nad to, jakž jste nyní, tisíckrát více, a požehnej vám, jakož jest mluvil vám.)
\par 12 Kterak bych nesl sám práci vaši, bríme vaše a nesnáze vaše?
\par 13 Vydejte z sebe muže moudré a opatrné, a zkušené z pokolení svých, abych je vám predstavil.
\par 14 I odpovedeli jste mi a rekli jste: Dobrát jest ta vec, kterouž jsi uciniti rozkázal.
\par 15 Vzav tedy prední z pokolení vašich, muže moudré a zkušené, ustanovil jsem je knížaty nad vámi, hejtmany nad tisíci, setníky, padesátníky, desátníky a správce v pokoleních vašich.
\par 16 Prikázal jsem také soudcum vašim toho casu, rka: Vyslýchejte pre mezi bratrími svými, a sudte spravedlive mezi mužem a bratrem jeho, i mezi príchozím jeho.
\par 17 Nebudete prijímati osoby v soudu; jakž malého tak i velikého slyšeti budete, nebudete se báti žádného, nebo Boží soud jest. Jestliže byste pak meli jakou vec nesnadnou, vznesete na mne, a vyslyším ji.
\par 18 Prikázal jsem vám, pravím, toho casu všecko, co byste ciniti meli.
\par 19 Potom pak hnuvše se z Oréb, prešli jsme všecku poušt tuto velikou a hroznou, kterouž jste videli, jdouce cestou k hore Amorejských, jakož nám byl prikázal Hospodin Buh náš, a prišli jsme až do Kádesbarne.
\par 20 I rekl jsem vám: Prišli jste až k hore Amorejské, kterouž Hospodin Buh náš dává nám.
\par 21 Ej, dal Hospodin Buh tvuj tu zemi tobe; vstupiž a vládni jí, jakož rekl Hospodin Buh otcu tvých tobe; neboj se, aniž se strachuj.
\par 22 Vy pak všickni pristoupili jste ke mne a rekli jste: Pošleme muže pred sebou, kteríž by nám shlédli zemi, a oznámili by nám neco o ceste, kterouž bychom vstoupiti meli, i mesta, do nichž bychom prijíti meli.
\par 23 Kterážto rec líbila se mne, a vzal jsem z vás dvanácte mužu, jednoho muže z každého pokolení.
\par 24 A oni obrátivše se a vstoupivše na horu, prišli až k údolí Eškol a shlédli zemi.
\par 25 Nabrali také s sebou ovoce zeme té, a prinesli nám, a oznámili nám o tech vecech, rkouce: Dobrát jest zeme, kterouž Hospodin Buh náš dává nám.
\par 26 A však jste nechteli jíti, ale odpírali jste reci Hospodina Boha svého.
\par 27 A reptali jste v staních svých, rkouce: Proto že nás v nenávisti mel Hospodin, vyvedl nás z zeme Egyptské, aby nás vydal v ruce Amorejského, a zahladil nás.
\par 28 Kam bychom šli? Bratrí naši zstrašili srdce naše, pravíce: Lid ten jest vetší a vyšší nežli my, mesta veliká a hrazená až k nebi, ano i syny Enakovy tam jsme videli.
\par 29 I rekl jsem vám: Nebojte se, ani se strachujte jich.
\par 30 Hospodin Buh váš, kterýž jde pred vámi, ont bojovati bude za vás rovne tak, jakž ucinil s vámi v Egypte, pred ocima vašima.
\par 31 Ano i na poušti videl jsi, kterak nesl tebe Hospodin Buh tvuj, jako nosí clovek syna svého, a to po vší ceste, kterouž jste šli, až jste prišli na toto místo.
\par 32 A ani tak uverili jste Hospodinu Bohu svému,
\par 33 Kterýž k vyhledání vám místa, na kterémž byste se klásti meli, v noci predcházel vás cestou v ohni, aby vám ukázal cestu, kterouž byste meli jíti, a v oblace ve dne.
\par 34 Uslyšel pak Hospodin hlas recí vašich, a rozhneval se, a prisáhl, rka:
\par 35 Jiste že nižádný z lidí techto pokolení zlého neuzrí zeme té dobré, kterouž jsem s prísahou zaslíbil dáti otcum vašim,
\par 36 Krome Kálefa, syna Jefonova; tent ji uzrí, a jemu dám zemi, po níž chodil, i synum jeho, proto že cele následoval Hospodina.
\par 37 Ano i na mne rozhneval se Hospodin prícinou vaší, rka: Také ani ty nevejdeš tam.
\par 38 Jozue, syn Nun, kterýž stojí pred tebou, ont vejde tam, jeho posiln, nebo on rozdelí ji losem Izraelovi.
\par 39 A dítky vaše, o kterýchž jste pravili, že v loupež budou, a synové vaši, kteríž ješte neznají dobrého ani zlého, oni vejdou do ní, a jim dám ji; oni dedicne ji obdrží.
\par 40 Vy pak obrátíce se, jdete na poušt cestou k mori Rudému.
\par 41 A odpovedevše, rekli jste ke mne: Zhrešilit jsme Hospodinu. My vstoupíme a budeme bojovati podlé toho všeho, jakž rozkázal nám Hospodin Buh náš. A vzavše všickni odení svá válecná na sebe, hotovi jste byli vstoupiti na horu.
\par 42 Hospodin pak rekl mi: Rci jim: Nevstupujte a nebojujte, nebot nejsem u prostred vás, abyste nebyli poraženi pred neprátely svými.
\par 43 A když jsem vám to mluvil, neuposlechli jste, nýbrž odporni jste byli reci Hospodinove, a všetecne vstoupili jste na horu.
\par 44 Tedy vytáhl Amorejský, kterýž bydlil na té hore, proti vám, a honili vás, jako cinívají vcely, a potreli vás na hore Seir až do Horma.
\par 45 A navrátivše se, plakali jste pred Hospodinem, ale neuslyšel Hospodin hlasu vašeho, a uší svých nenaklonil k vám.
\par 46 I zustali jste v Kádes za mnohé dny, podlé poctu dnu, v nichž jste tam byli.

\chapter{2}

\par 1 Potom obrátivše se, táhli jsme na tu poušt cestou k mori Rudému, jakož mluvil Hospodin ke mne, a obcházeli jsme horu Seir za dlouhý cas,
\par 2 Až mi rekl Hospodin takto:
\par 3 Dosti jste již obcházeli horu tuto, obrattež se k strane pulnocní.
\par 4 A lidu prikaž, rka: Pujdete pres pomezí bratrí svých synu Ezau, kteríž bydlí v Seir. Ackoli budou se vás báti, hledte však pilne,
\par 5 Abyste jich nedráždili; nebo nedám vám z zeme jejich ani šlepeje nožné, proto že v dedictví Ezau dal jsem horu Seir.
\par 6 Pokrmu koupíte od nich za peníze, a jísti budete, i vody také ku pití za peníze od nich jednati budete.
\par 7 Nebo Hospodin Buh tvuj požehnal tobe pri všeliké práci rukou tvých, a zná, že jdeš pres poušt velikou tuto; již ctyridceti let Hospodin Buh tvuj byl s tebou, aniž jsi mel v cem nedostatku.
\par 8 A šli jsme od bratrí našich, synu Ezau, kteríž bydlí v Seir, s cesty polní od Elat a od Aziongaber, a uchýlili jsme se, abychom šli cestou po poušti Moábské.
\par 9 I rekl mi Hospodin: Neškod Moábským, ani jich popouzej k boji, nebo nedám tobe v zemi jejich dedictví, ponevadž synum Lot dal jsem Ar v dedictví.
\par 10 (Emim prvé bydlili v ní, lid veliký a mnohý, a vysokého zrostu, jako Enakim.
\par 11 Oni také za obry držáni byli jako Enakim, ale Moábští ríkali jim Emim.
\par 12 V Seir pak bydlili prvé Horejští, kteréž synové Ezau vyhnali, a zahladili je pred tvárí svou, a bydlili tu místo nich, jako ucinil Izrael v zemi vládarství svého, kterouž jim dal Hospodin.)
\par 13 Nyní vstanouce, prejdete potok Záred. I prešli jsme potok Záred.
\par 14 Casu pak, v nemž jsme šli z Kádesbarne, až jsme prešli potok Záred, bylo let tridceti osm, dokavadž nebyl vyhlazen všecken vek mužu bojovných z prostred stanu, jakož jim prisáhl Hospodin.
\par 15 Nebo ruka Hospodinova byla proti nim k setrení jich z prostredku stanu, dokudž nevyhladil jich.
\par 16 I stalo se, když všickni muži ti bojovní vyhynuli z prostredku lidu,
\par 17 Že mluvil Hospodin ke mne, rka:
\par 18 Ty prejdeš dnes pomezí Moábské k mestu Ar,
\par 19 A priblížíš se k synum Ammon. Nessužujž jich a nepopouzej jich k boji, nebo nedám tobe v zemi synu Ammon dedictví, ponevadž synum Lotovým dal jsem ji k vládarství.
\par 20 (I ona také držána byla za zemi obru; nebo obrové pred tím bydlili v ní, kterýmž Ammoninští ríkali Zamzomim,
\par 21 Lid veliký a mnohý, a vysokého zrostu, jako Enakim. Ale zahladil je Hospodin pred tvárí jejich, a vešli v dedictví jejich, a bydlili na míste jejich,
\par 22 Jakož ucinil synum Ezau bydlícím v Seir, pro než vyhladil Horejské pred tvárí jejich, i vešli v dedictví jejich, a bydlili na míste jejich až do dnešního dne.
\par 23 Hevejské také, kteríž bydlili v Azerim až do Gáza, Kaftorejští, kteríž vyšli z Kaftor, zahladili je, a bydlili na míste jejich.)
\par 24 Vstanouce, berte se a prejdete potok Arnon. Hle, dal jsem v ruce tvé Seona, krále Ezebon Amorejského, a zemi jeho; zacniž jí vládnouti, a bojuj válecne proti nemu.
\par 25 Dnes pocnu poušteti strach a lekání se tebe na lidi, kteríž jsou pode vším nebem, tak že kterížkoli uslyší povest o tobe, trásti a lekati se budou tvári tvé.
\par 26 I poslal jsem posly z poušte Kedemot k Seonovi, králi Ezebon, s slovy pokojnými, rka:
\par 27 Necht projdu skrze zemi tvou, prímo cestou pujdu, neuchýlím se ani na pravo ani na levo.
\par 28 Pokrmu za peníze prodáš mi, abych jedl, vody také za peníze dáš mi, a píti budu; toliko pešky projdu,
\par 29 Jakož mi ucinili synové Ezau, kteríž bydlí v Seir, a Moábští, kteríž bydlí v Ar, dokudž neprejdu Jordánu, jda k zemi, kterouž Hospodin Buh náš dává nám.
\par 30 Ale nechtel dopustiti Seon, král Ezebon, abychom prošli zemi jeho; nebo byl zatvrdil Hospodin Buh tvuj ducha jeho, a ztužil srdce jeho, aby dal jej v ruce tvé, jakož se vidí podnes.
\par 31 (Nebo rekl mi byl Hospodin: Aj, již jsem pocal v moc dávati tobe Seona i zemi jeho; zacniž jí vládnouti, abys dedicne obdržel zemi jeho.)
\par 32 A vytáhl byl Seon proti nám, on i všecken lid jeho k boji do Jasa.
\par 33 I dal jej nám Hospodin Buh náš, a porazili jsme ho s syny jeho i se vším lidem jeho.
\par 34 Vzali jsme také toho casu všecka mesta jeho, a dali jsme v prokletí lid tech všech mest, i ženy i deti, žádného nepozustavivše.
\par 35 Toliko hovada rozebrali jsme sobe, a koristi z mest, kterýchž jsme dobyli.
\par 36 Od Aroer jenž jest na brehu potoka Arnon, a od mesta, kteréž jest v údolí, až do Galád nebylo mesta, kteréž by ostáti mohlo pred námi; všecka nám dal Hospodin Buh náš.
\par 37 Toliko k zemi synu Ammon nepriblížils se, ani k žádné krajine ležící pri potoku Jabok, ani k mestum na horách, a k žádnému místu, kteréž zapovedel Hospodin Buh náš.

\chapter{3}

\par 1 Potom obrátivše se, táhli jsme cestou k Bázan. I vytáhl Og, král Bázan, proti nám, on i všecken lid jeho k bitve do Edrei.
\par 2 I rekl mi Hospodin: Neboj se ho, nebo v ruce tvé dal jsem jej, i všecken lid jeho i zemi jeho, a uciníš jemu tak, jako jsi ucinil Seonovi, králi Amorejskému, kterýž bydlil v Ezebon.
\par 3 Dal tedy Hospodin Buh náš v ruce naše i Oga, krále Bázan, a všecken lid jeho, i porazili jsme jej, tak že jsme nepozustavili po nem žádného živého.
\par 4 Dobyli jsme také téhož casu všech mest jeho; nebylo mesta, kteréhož bychom jim neodjali, šedesáte mest, všecku krajinu Argob, království Oga v Bázan.
\par 5 Všecka ta mesta byla ohražená zdmi vysokými, branami a závorami, krome mest otevrených, jichž bylo velmi mnoho.
\par 6 A vyplénili jsme je, jako jsme ucinili Seonovi, králi Ezebon, vyhladivše všecka mesta, muže, ženy i dítky.
\par 7 Všecka pak hovada a koristi mest rozebrali jsme sobe.
\par 8 Vzali jsme také téhož casu zemi z ruky dvou králu Amorejských, kteráž byla pred Jordánem, od potoku Arnon až k hore Hermon,
\par 9 (Sidonští ríkají Hermonu Sarion, a Amorejští ríkají jemu Sanir),
\par 10 Všecka mesta v kraji, a všecken Galád, a všecken Bázan až do Sálecha a Edrei, kteráž byla mesta království Og v Bázan.
\par 11 Sám toliko Og, král v Bázan, z jiných obru byl pozustal. Aj, luže jeho, luže železné, ješte zustává v Rabbat synu Ammon, devíti loktu zdýlí, a ctyr loket zšírí, jakž jest loket muže.
\par 12 Když tedy zemi tu obdrželi jsme dedicne toho casu, krajinu od Aroer, jenž jest pri potoku Arnon, a polovici hory Galád i mesta její, dal jsem pokolení Ruben a Gád.
\par 13 Ostatek pak Galád a všecku zemi Bázan, království Oga, dal jsem polovici pokolení Manassesova, totiž všecku krajinu Argob, všecku Bázan, kteráž sloula zeme obru.
\par 14 Jair, syn Manasse, vzal všecku krajinu Argob, až ku pomezí Gessuri a Machati; procež nazval zemi Bázan od jména svého Havot Jair až do dnešního dne.
\par 15 Machirovi pak dal jsem Galád.
\par 16 A Rubenovu a Gádovu pokolení dal jsem krajinu od Galád až ku potoku Arnon, polovici potoka s pomezím až ku potoku Jabok, kdež jsou hranice synu Ammonitských,
\par 17 A roviny tyto i Jordán s pomezím od Ceneret až k mori pustému, jenž jest more slané, ležící pod horou Fazga k východu.
\par 18 A prikázal jsem vám toho casu, rka: Hospodin Buh váš dal vám zemi tuto, abyste ji dedicne obdrželi; vezmouce odení na sebe, pujdete pred bratrími vašimi, syny Izraelskými, kterížkoli silní jste.
\par 19 Toliko ženy vaše a dítky vaše, a dobytek váš, (nebo vím, že mnoho dobytka máte,) zustanou v mestech vašich, kteráž jsem dal vám,
\par 20 Dokudž by nedal odpocinutí Hospodin bratrím vašim jako i vám, aby i oni dedicne obdrželi zemi, kterouž Hospodin Buh váš dává jim za Jordánem; tedy navrátíte se jeden každý k dedictví svému, kteréž jsem dal vám.
\par 21 Také i Jozue prikázal jsem toho casu, rka: Oci tvé vidí všecko, co ucinil Hospodin Buh váš tem dvema králum; takt uciní Hospodin všechnem královstvím, do kterýchž ty pujdeš.
\par 22 Nebojtež se jich, nebo Hospodin Buh váš, ont jest, kterýž bojuje za vás.
\par 23 A tehdáž prosil jsem Hospodina, rka:
\par 24 Panovníce Hospodine, ty jsi pocal ukazovati služebníku svému velikost svou a ruku svou presilnou; nebo kdo jest Buh silný na nebi aneb na zemi, ješto by ciniti mohl skutky podobné tvým, a moci podobné tobe?
\par 25 Prosím, necht vejdu a uzrím zemi tu výbornou, kteráž jest za Jordánem, horu tu výbornou i Libán.
\par 26 Pohnul se pak Hospodin na mne pro vás a neuslyšel mne, ale rekl mi: Dosti máš, nemluv více o to se mnou.
\par 27 Vstup na vrch hory Fazga, a pozdvihna ocí svých k západu a k pulnoci, ku poledni i k východu, hled ocima svýma; nebo neprejdeš Jordánu tohoto.
\par 28 Ale prikaž Jozue, a posiln ho, i potvrd ho; nebo pujde pred lidem tímto, a on uvede jim v dedictví zemi, kterouž uzríš.
\par 29 I zustali jsme v údolí naproti Betfegor.

\chapter{4}

\par 1 Nyní tedy, ó Izraeli, slyš ustanovení a soudy, kteréž já ucím vás ciniti, abyste živi byli, a vejdouce, dedicne vládli zemí, kterouž Hospodin Buh otcu vašich vám dává.
\par 2 Nepridáte nic k slovu, kteréž já prikazuji vám, aniž co ujmete od neho, abyste tak prikázaní Hospodina Boha svého, kteréž já prikazuji vám, zachovali.
\par 3 Oci vaše videly, co ucinil Hospodin pro Belfegor, jak všecky lidi, kteríž odešli po Belfegor, vyhladil Hospodin Buh tvuj z prostredku tvého.
\par 4 Vy pak, prídržející se Hospodina Boha vašeho, živi jste do dnešního dne všickni.
\par 5 Viztež, ucilt jsem vás ustanovením a soudum, jakž mi prikázal Hospodin Buh muj, abyste tak cinili v zemi, do kteréž vejdete k dedicnému držení jí.
\par 6 Ostríhejtež tedy a cinte je, nebo to jest moudrost vaše a opatrnost vaše pred ocima národu, kteríž, slyšíce všecka ustanovení tato, reknou: Jiste lid moudrý a rozumný národ veliký tento jest.
\par 7 Nebo který národ tak veliký jest, kterýž by mel bohy sobe tak blízké, jako jest Hospodin Buh váš ve všem volání našem k nemu?
\par 8 A který jest národ tak veliký, kterýž by mel ustanovení a soudy spravedlivé, jako jest všecken zákon tento, kterýž já vám dnes predkládám?
\par 9 A však hled se a bedlive ostríhej duše své, abys nezapomenul na ty veci, kteréž videly oci tvé, a aby nevyšly z srdce tvého po všecky dny života tvého; a v známost je uvedeš synum i vnukum svým.
\par 10 A nezapomínej, že jsi onoho dne stál pred Hospodinem Bohem svým na Orébe, když mi byl rekl Hospodin: Shromažd mi lid, at jim predložím slova má, z nichž by se ucili mne báti po všecky dny, dokudž živi budou na zemi, a témuž aby syny své ucili.
\par 11 Tedy pristoupili jste, a stáli jste pod horou; (hora pak ta horela ohnem až do samého nebe, a byly na ní tmy, oblak a mrákota.)
\par 12 I mluvil k vám Hospodin z prostredku ohne. Hlas slov slyšeli jste, ale obrazu žádného jste nevideli krome hlasu,
\par 13 Jímž vyhlásil vám smlouvu svou, kteréžto prikázal vám ostríhati, totiž desíti slov, a napsal je na dvou dskách kamenných.
\par 14 Mne také prikázal Hospodin toho casu, abych ucil vás ustanovením a soudum, abyste cinili je v zemi, do kteréž jdete k dedicnému držení jí.
\par 15 Protož pilne pecujte o duše své, (nebo nevideli jste žádného obrazu toho dne, když k vám mluvil Hospodin na Orébe z prostredku ohne),
\par 16 Abyste neporušili cesty své, a neucinili sobe rytiny aneb podobenství nejakého obrazu, tvárnosti muže neb ženy,
\par 17 Podobenství nejakého hovada, kteréž jest na zemi, aneb podobenství jakéhokoli ptáka krídla majícího, kterýž létá v povetrí,
\par 18 Podobenství jakéhokoli zemeplazu na zemi, aneb podobenství jakékoli ryby, kteráž jest u vode pod zemí.
\par 19 Ani pozdvihuj ocí svých k nebi, abys, vida slunce a mesíc, i hvezdy se vším zástupem nebeským a ponuknut jsa, klanel bys se jim a sloužil bys jim, ješto ty veci oddal Hospodin Buh tvuj všechnem lidem pode vším nebem,
\par 20 Vás pak vzal Hospodin, a vyvedl vás jako z peci železné z Egypta, abyste byli jeho lid dedicný, jako jste dne tohoto.
\par 21 Ale na mne rozhneval se Hospodin prícinou vaší, a prisáhl, že neprejdu Jordánu, ani nevejdu do zeme té výborné, kterouž Hospodin Buh tvuj dává tobe v dedictví.
\par 22 Nebo já umru v zemi této, a neprejdu Jordánu, vy pak prejdete, a dedicne obdržíte zemi tu výbornou.
\par 23 Hledtež, abyste se nezapomínali na smlouvu Hospodina Boha svého, kterouž ucinil s vámi, a necinili sobe rytiny, aneb obrazu jakékoli veci, jakož prikázal tobe Hospodin Buh tvuj.
\par 24 Nebo Hospodin Buh tvuj jest ohen sžírající, Buh silný, horlivý.
\par 25 Když zplodíš syny a vnuky, a zstaráte se v zemi té, jestliže porušíte cestu svou, a udeláte rytinu ku podobenství jakékoli veci, aneb uciníte neco zlého pred ocima Hospodina Boha svého, popouzejíce ho:
\par 26 Osvedcuji proti vám dnes pred nebem i zemí, že hrozne a rychle vyhlazeni budete z zeme, do kteréž pujdete pres Jordán, abyste vládli jí. Nedlouho bydliti budete v ní, ale do konce vyhlazeni budete.
\par 27 A rozptýlí vás Hospodin mezi národy, a malický pocet vás zustane mezi pohany, do kterýchž zavede vás Hospodin,
\par 28 A sloužiti tam budete bohum, dílu rukou lidských, drevu a kameni, ješto nevidí ani slyší, ani jedí, ani cijí.
\par 29 Jestliže pak i tam hledati budeš Hospodina Boha svého, tedy nalezneš, budeš-li ho hledati z celého srdce svého a z celé duše své.
\par 30 Kdyžt úzko bude, a prijdou na te všecky ty veci, naposledy však, jestliže bys se navrátil k Hospodinu Bohu svému, a poslouchal bys hlasu jeho,
\par 31 (Ponevadž Hospodin Buh tvuj jest Buh silný, milosrdný,) neopustí tebe, ani te nezkazí, ani zapomene se na smlouvu otcu tvých, kterouž s prísahou utvrdil jim.
\par 32 Nebo ptej se nyní na dni staré, kteríž byli pred tebou, od toho dne, v kterémž stvoril Buh cloveka na zemi, a od jednoho kraje nebe až do druhého, stala-li se kdy vec podobná této tak veliké, aneb slýcháno-li kdy co takového?
\par 33 Zdali kdy slyšel který lid hlas Boha mluvícího z prostredku ohne, jako jsi ty slyšel a živ zustal?
\par 34 Aneb zdali se kdy který Buh pokusil, aby prijda, vzal sobe národ nekterý z jiného národu s zkušováním, znameními a s zázraky, skrze boje a ruku silnou, v rameni vztaženém a v hruzi veliké, jako ucinil všecko toto pro vás Hospodin Buh váš v Egypte pred ocima vašima?
\par 35 Tobet jest to ukázáno, abys vedel, že Hospodin jest Buh, a že není jiného krome neho.
\par 36 Dalt s nebe slyšeti hlas svuj, aby vyucil tebe, a na zemi ukázal tobe ohen svuj veliký, a slova jeho slyšel jsi z prostredku ohne.
\par 37 Proto že miloval otce tvé, vyvolil síme jejich po nich, a vyvedl te pred sebou mocí svou velikou z Egypta,
\par 38 Aby vyžena národy veliké a silnejší, než jsi ty, pred tvárí tvou, uvedl tebe a dal tobe zemi jejich v dedictví, jakož dnes vidíš.
\par 39 Veziž tedy dnes a obnov to v srdci svém, že Hospodin jest Buh na nebi svrchu i na zemi dole, a není žádného jiného.
\par 40 Protož ostríhej po všecky dny ustanovení a prikázaní jeho, kteráž já dnes prikazuji tobe, aby dobre bylo tobe i synum tvým po tobe, a abys prodlil dnu v zemi, kterouž Hospodin Buh tvuj dá tobe.
\par 41 Tehdy oddelil Mojžíš tri mesta pred Jordánem k východu slunce.
\par 42 Aby utekl do nich vražedlník, kterýž by zabil bližního svého nechte, a nemel by ho pred tím v nenávisti; když by utekl do jednoho z tech mest, aby živ zustal:
\par 43 Bozor na poušti, na rovinách v kraji Rubenitských, a Rámot v Galád, v pokolení Gád, a Golan v Bázan, v pokolení Manasse.
\par 44 Ten jest zákon, kterýž predložil Mojžíš synum Izraelským,
\par 45 A tato jsou svedectví a ustanovení i soudové, kteréž predložil Mojžíš synum Izraelským, když vyšli z Egypta,
\par 46 Pred Jordánem v údolí naproti Betfegor, v zemi Seona, krále Amorejského, kterýž bydlil v Ezebon, jejž porazil Mojžíš a synové Izraelští, když vyšli z Egypta,
\par 47 A uvázali se dedicne v zemi jeho, a v zemi Oga, krále Bázan, dvou králu Amorejských, v zemi, kteráž jest pred Jordánem k východu slunce,
\par 48 Od Aroer, (jenž leží pri brehu potoka Arnon), až k hore Sion, kteráž jest Hermon.
\par 49 I ve všecku rovinu pred Jordánem k východu až k mori pustému, ležící pod horou Fazga.

\chapter{5}

\par 1 I svolal Mojžíš všecken lid Izraelský, a rekl jim: Slyš, Izraeli, ustanovení a soudy, kteréž já dnes mluvím v uši vaše; naucte se jim, a v skutku jich ostríhejte.
\par 2 Hospodin Buh náš ucinil s námi smlouvu na Orébe.
\par 3 Ne s otci našimi ucinil Hospodin tu smlouvu, ale s námi, kteríž zde jsme nyní my všickni živí.
\par 4 Tvárí v tvár mluvil Hospodin s vámi na té hore z prostredku ohne,
\par 5 (Já jsem pak stál mezi Hospodinem a mezi vámi toho casu, abych oznámil vám rec Hospodinovu; nebo jste se báli ohne, a nevstoupili jste na horu), rka:
\par 6 Já jsem Hospodin Buh tvuj, kterýž jsem te vyvedl z zeme Egyptské z domu služby.
\par 7 Nebudeš míti bohu jiných prede mnou.
\par 8 Neuciníš sobe rytiny , ani jakého podobenství tech vecí, kteréž jsou na nebi svrchu, ani tech, kteréž na zemi dole, ani tech, kteréž jsou u vodách pod zemí.
\par 9 Nebudeš se jim klaneti, ani jich ctíti. Nebo já jsem Hospodin Buh tvuj, Buh silný, horlivý, navštevující nepravost otcu na synech do tretího i ctvrtého pokolení tech, kteríž nenávidí mne,
\par 10 A cinící milosrdenství nad tisíci tech, kteríž mne milují, a ostríhají prikázaní mých.
\par 11 Nevezmeš jména Hospodina Boha svého nadarmo, nebot nenechá bez pomsty Hospodin toho, kdož by bral jméno jeho nadarmo.
\par 12 Ostríhej dne sobotního, abys jej svetil, jakož prikázal tobe Hospodin Buh tvuj.
\par 13 Šest dní pracovati budeš, a delati všeliké dílo své;
\par 14 Ale dne sedmého odpocinutí jest Hospodina Boha tvého. Nebudeš delati žádného díla, ty i syn tvuj i dcera tvá, i služebník tvuj i devka tvá, vul i osel tvuj i všeliké hovado tvé, i príchozí tvuj, kterýž jest v branách tvých, aby odpocinul služebník tvuj a devka tvá jako i ty.
\par 15 A pamatuj, že jsi byl služebníkem v zemi Egyptské, a vyvedl te Hospodin Buh tvuj odtud v ruce silné, a v rameni vztaženém. Protož prikázal tobe Hospodin Buh tvuj, abys svetil den svátecní.
\par 16 Cti otce svého i matku svou, jakož prikázal tobe Hospodin Buh tvuj, aby se prodleli dnové tvoji, a aby tobe dobre bylo na zemi, kterouž Hospodin Buh tvuj dá tobe.
\par 17 Nezabiješ.
\par 18 Nesesmilníš.
\par 19 Nepokradeš.
\par 20 Nepromluvíš proti bližnímu svému krivého svedectví.
\par 21 Nepožádáš manželky bližního svého, aniž požádáš domu bližního svého, pole jeho, neb služebníka jeho, aneb devky jeho, vola jeho neb osla jeho, aneb cehokoli z tech vecí, kteréž jsou bližního tvého.
\par 22 Ta slova mluvil Hospodin ke všemu shromáždení vašemu na hore z prostredku ohne, oblaku a mrákoty, hlasem velikým, a nepridal nic více, a napsal je na dvou dskách kamenných, kteréž mne dal.
\par 23 Vy pak když jste uslyšeli hlas z prostredku tmy, (nebo hora ohnem horela), pristoupili jste ke mne všickni vudcové pokolení vašich a starší vaši,
\par 24 A rekli jste: Ej, ukázal nám Hospodin Buh náš slávu svou a velikost svou, a slyšeli jsme hlas jeho z prostredku ohne; dnešního dne videli jsme, že Buh mluvil s clovekem, a on živ zustal.
\par 25 Protož nyní proc máme zemríti? Nebo sežral by nás ohen veliký tento; jestli více slyšeti budeme hlas Hospodina Boha našeho, zemreme.
\par 26 Nebo co jest všeliké telo, aby slyše hlas Boha živého mluvícího z prostredku ohne, jako my, melo živo býti?
\par 27 Pristup ty a slyš všecky veci, kteréž mluviti bude Hospodin Buh náš; potom ty mluviti budeš nám, což by koli rekl tobe Hospodin Buh náš, a my slyšeti i ciniti budeme.
\par 28 Uslyšev pak Hospodin hlas recí vašich, když jste mluvili ke mne, rekl mi Hospodin: Slyšel jsem hlas reci lidu tohoto, kterouž mluvili tobe. Cožkoli mluvili, dobret jsou mluvili.
\par 29 Ó kdyby bylo jejich srdce takové, aby se báli mne a ostríhali prikázaní mých po všeliký cas, aby jim dobre bylo i synum jejich na veky!
\par 30 Jdi, rci jim: Navratte se k stanum vašim.
\par 31 Ty pak stuj tuto pri mne, a oznámím tobe všecka prikázaní, ustanovení i soudy, kterýmž je uciti budeš, aby je cinili v zemi, kterouž já dávám jim, aby jí dedicne vládli.
\par 32 Hledtež tedy, abyste cinili, jakž prikázal vám Hospodin Buh váš; neuchylujte se na pravo ani na levo.
\par 33 Po vší té ceste, kterouž vám prikázal Hospodin Buh váš, choditi budete, abyste živi byli, a dobre bylo vám, a abyste prodlili dnu na zemi, kterouž dedicne obdržíte.

\chapter{6}

\par 1 Toto pak jest prikázaní, ustanovení a soudové, kteréž prikázal Hospodin Buh váš, abych ucil vás, abyste cinili je v zemi, do kteréž jdete k dedicnému držení jí,
\par 2 Abys se bál Hospodina Boha svého, ostríhaje všech ustanovení jeho a prikázaní jeho, kteráž já prikazuji tobe, ty i syn tvuj i vnuk tvuj, po všecky dny života svého, aby se prodlili dnové tvoji.
\par 3 Slyšiž tedy, Izraeli, a hled tak skutecne ciniti, aby tobe dobre bylo, a abyste se velmi rozmnožili, (jakož mluvil Hospodin Buh otcu tvých tobe,) v zemi oplývající mlékem a strdí.
\par 4 Slyš, Izraeli, Hospodin Buh náš, Hospodin jeden jest.
\par 5 Protož milovati budeš Hospodina Boha svého z celého srdce svého, a ze vší duše své, a ze vší síly své.
\par 6 A budou slova tato, kteráž já prikazuji tobe dnes, v srdci tvém.
\par 7 A budeš je casto opetovati synum svým, a mluviti o nich, když sedneš v dome svém, když pujdeš cestou, a léhaje i vstávaje.
\par 8 Uvážeš je za znamení na ruce své, a jako nácelník mezi ocima svýma.
\par 9 Napíšeš je také na verejích domu svého a na branách svých.
\par 10 A když te uvede Hospodin Buh tvuj do zeme, kterouž s prísahou zaslíbil otcum tvým, Abrahamovi, Izákovi a Jákobovi, že ji tobe dá, i mesta veliká a výborná, kterýchžs nestavel,
\par 11 A domy plné všech dobrých vecí, kterýchž jsi nenaplnil, a studnice vykopané, kterýchž jsi nekopal, a vinice i olivoví, jichž jsi neštípil, a jedl bys a nasytil se:
\par 12 Varuj se, abys nezapomenul na Hospodina, kterýž te vyvedl z zeme Egyptské, z poroby težké.
\par 13 Hospodina Boha svého báti se budeš, a jemu sloužiti, a ve jméno jeho prisahati.
\par 14 Neodejdeš po bozích cizích, z bohu jiných národu, kteríž vukol vás jsou,
\par 15 (Nebo Buh silný, horlivý, Hospodin tvuj u prostred tebe jest,) aby se neroznítila prchlivost Hospodina Boha tvého na tebe, a shladil by te se svrchku zeme.
\par 16 Nebudete pokoušeti Hospodina Boha svého, jako jste pokoušeli v Massah.
\par 17 Pilne ostríhejte prikázaní Hospodina Boha svého, a svedectví jeho, i ustanovení jeho, kteráž prikázal tobe,
\par 18 A cin to, což pravého a dobrého jest pred ocima Hospodinovýma, aby tobe dobre bylo, a vejda, abys dedicne obdržel zemi výbornou, kterouž s prísahou zaslíbil Hospodin otcum tvým,
\par 19 Aby vypudil všecky neprátely tvé od tvári tvé, jakož mluvil Hospodin.
\par 20 Když by se potom syn tvuj otázal tebe, rka: Co jsou to za svedectví a ustanovení i soudy, kteréž prikázal Hospodin Buh náš vám?
\par 21 Tedy díš synu svému: Služebníci jsme byli Faraonovi v Egypte, i vyvedl nás Hospodin z Egypta v ruce silné.
\par 22 A cinil Hospodin znamení a zázraky veliké a škodlivé v Egypte proti Faraonovi, a proti všemu domu jeho pred ocima našima,
\par 23 Nás pak vyvedl odtud, aby uvedl nás, a dal nám zemi, kterouž s prísahou zaslíbil otcum našim.
\par 24 Protož prikázal nám Hospodin, abychom ostríhali všech ustanovení techto, bojíce se Hospodina Boha svého, aby nám dobre bylo po všecky dny, a aby zachoval nás pri životu, jakž to ciní i v dnešní den.
\par 25 A spravedlnost míti budeme, když ostríhati budeme a ciniti všecka prikázaní tato pred Hospodinem Bohem svým, jakož prikázal nám.

\chapter{7}

\par 1 Když pak tebe uvede Hospodin Buh tvuj do zeme, do kteréž ty již vcházíš, abys vládl jí, a vypléní národy mnohé od tvári tvé, Hetea, Gergezea, Amorea, Kananea, Ferezea, Hevea a Jebuzea, sedm národu vetších a silnejších, nežli jsi ty,
\par 2 A dá je Hospodin Buh tvuj tobe, abys je pobil: jako proklaté vypléníš je, nevejdeš s nimi v smlouvu, aniž slituješ se nad nimi.
\par 3 Nikoli nesprízníš se s nimi; dcery své nedáš synu jejich, a dcery jejich nevezmeš synu svému.
\par 4 Nebot by odvedla syna tvého od následování mne, a sloužili by bohum cizím, procež popudila by se prchlivost Hospodinova na vás, a zahladila by te rychle.
\par 5 Ale radeji toto jim ucinte: Oltáre jejich zborte, modly jejich stroskotejte, háje také posekejte, a rytiny jejich ohnem spalte.
\par 6 Nebo ty lid svatý jsi Hospodinu Bohu svému; tebe vyvolil Hospodin Buh tvuj, abys jemu byl lidem zvláštním, mimo všecky národy, kteríž jsou na zemi.
\par 7 Ne proto, že by vás více bylo nad jiné národy, pripojil se k vám Hospodin, a vyvolil vás, (nebo menší vás pocet byl nežli jiných národu,)
\par 8 Ale proto, že miloval vás Hospodin, a splniti chtel prísahu, kterouž prisáhl otcum vašim, vyvedl vás v ruce silné, a vysvobodil vás z domu služby, z ruky Faraona, krále Egyptského.
\par 9 I zvíš, že Hospodin Buh tvuj jest Buh, Buh silný a pravdomluvný, ostríhající smlouvy a milosrdenství tem, kteríž ho milují a ostríhají prikázaní jeho, až do tisícího kolena,
\par 10 Odplacující tomu, kterýž ho nenávidí, v tvár jeho, tak aby zahladil jej. Nebudet prodlévati; kdož ho nenávidí, v tvár jeho odplatí jemu.
\par 11 Protož ostríhej prikázaní a ustanovení i soudu, kteréž já tobe dnes prikazuji, abys je cinil.
\par 12 I budet to, že když poslouchati budete soudu techto a ostríhati i ciniti je, také Hospodin Buh tvuj ostríhati bude tobe smlouvy a milosrdenství, kteréž s prísahou zaslíbil otcum tvým.
\par 13 A bude te milovati, i požehná tobe a rozmnoží tebe. Nebo požehná plodu života tvého a úrodám zeme tvé, obilí tvému, vínu tvému a oleji tvému, plodu skotu tvých i stádum bravu tvých v zemi, kterouž s prísahou zaslíbil otcum tvým, že ji tobe dá.
\par 14 Požehnaný budeš nad všecky národy; nebude u tebe neplodný aneb neplodná, ani mezi hovady tvými.
\par 15 Vzdálí také od tebe Hospodin všeliký neduh, a všecky zlé nemoci Egyptské, kteréž znáš; nevzloží jich na tebe, ale vzloží je na všecky, kteríž te nenávidí.
\par 16 A shladíš všecky národy, kteréž Hospodin Buh tvuj dá tobe. Neslituje se nad nimi oko tvé, aniž sloužiti budeš bohum jejich, nebo to bylo by tobe osídlem.
\par 17 Rekl-li bys v srdci svém: Vetší jsou národové tito nežli já, kterak budu moci vyhnati je?
\par 18 Neboj se jich, ale pilne pamatuj na to, co ucinil Hospodin Buh tvuj Faraonovi a všechnem Egyptským,
\par 19 Na pokušení veliká, kteráž videly oci tvé, i znamení a zázraky, a ruku silnou a ráme vztažené, v kterémž vyvedl te Hospodin Buh tvuj. Takt uciní Hospodin Buh tvuj všechnem národum, kterýchž bys se obával.
\par 20 Nadto sršne pošle Hospodin Buh tvuj na ne, dokudž by nezhynuli, kteríž by pozustali, a kteríž by se skryli pred tebou.
\par 21 Nelekejž se strachu jejich, nebo Hospodin Buh tvuj jest u prostred tebe, Buh silný, veliký a hrozný.
\par 22 I vypléní Hospodin Buh tvuj národy ty od tvári tvé pomalu; nebudeš moci pojednou jich shladiti, aby se nerozmnožila proti tobe zver polní.
\par 23 A však dá je Hospodin Buh tvuj tobe, a setre je setrením velikým, dokudž nebudou vyhlazeni.
\par 24 Vydá i krále jejich v ruce tvé, a vyhladíš jméno jejich pod nebem; neostojít žádný pred tebou, až je i vyhladíš.
\par 25 Ryté bohy jejich ohnem popálíš; nepožádáš stríbra a zlata, kteréž jest na nich, aniž ho sobe vezmeš, aby nebylo tobe osídlem, nebo ohavnost jest Hospodinu Bohu tvému.
\par 26 Aniž vneseš ohavnosti do domu svého, abys nebyl proklatý, jako i ona; všelijak v ohyzdnosti a v ohavnosti budeš míti ji, nebo proklatá jest.

\chapter{8}

\par 1 Všelikého prikázaní, kteréž já prikazuji tobe dnes, skutecne ostríhejte, abyste živi byli a rozmnožili se, a vešli k dedicnému obdržení zeme, kterouž s prísahou zaslíbil Hospodin otcum vašim.
\par 2 A rozpomínati se budeš na všecku cestu, kterouž te vedl Hospodin Buh tvuj již ted ctyridceti let po poušti, aby ponížil tebe a zkusil te, aby známé bylo, co jest v srdci tvém, budeš-li ostríhati prikázaní jeho, cili nic.
\par 3 I ponížil te a dopustil na tebe hlad, potom te krmil mannou, kteréž jsi ty neznal, ani otcové tvoji, aby známé ucinil tobe, že ne samým chlebem živ bude clovek, ale vším tím, což vychází z úst Hospodinových, živ bude clovek.
\par 4 Roucho tvé nevetšelo na tobe, a noha tvá se neodhnetla, již od ctyridceti let.
\par 5 Znejž tedy v srdci svém, že jakož cvicí clovek syna svého, tak Hospodin Buh tvuj cvicí tebe.
\par 6 A ostríhej prikázaní Hospodina Boha svého, chode po cestách jeho, a boje se jeho.
\par 7 Nebo Hospodin Buh tvuj uvozuje te do zeme výborné, zeme, v níž jsou potokové vod, studnice a propasti prýštící se po údolích i po horách,
\par 8 Do zeme hojné na pšenici a jecmen, na vinice a fíky a jablka zrnatá, do zeme, v níž jest hojnost olivoví olej prinášejícího a medu.
\par 9 Zeme, v níž bez nedostatku chléb jísti budeš, a v nicemž nouze trpeti nebudeš, zeme, jejíž kamení jest železo, a z hor jejích med sekati budeš.
\par 10 Kdyžkoli jísti budeš a nasytíš se, dobroreciti budeš Hospodina Boha svého za zemi výbornou, kterouž dal tobe.
\par 11 Aniž se kdy toho dopouštej, abys se mel zapomenouti na Hospodina Boha svého, a neostríhati prikázaní a soudu jeho i ustanovení jeho, kteráž já dnes prikazuji tobe,
\par 12 Aby, když bys jedl a nasycen byl, a domu krásných nastaveje, v nich bys bydlil,
\par 13 A volové i ovce tvé rozmnoženy byly by, stríbra také a zlata mel bys mnoho, a hojnost ve všech vecech svých,
\par 14 Nepozdvihlo se srdce tvé, a zapomenul bys na Hospodina Boha svého, kterýž te vyvedl z zeme Egyptské, z domu služby,
\par 15 A vedl te pres poušt velikou a hroznou, na níž byli hadové ohniví a štírové, poušt žíznivou, na níž nebylo žádné vody, a vyvedl tobe vodu z pretvrdé skály,
\par 16 Kterýž te krmil na poušti mannou, o kteréž nevedeli otcové tvoji, aby te potrápil a zkusil tebe, naposledy však aby dobre ucinil tobe.
\par 17 Aniž ríkej v srdci svém: Moc má a síla ruky mé zpusobila mi tato zboží,
\par 18 Ale pamatuj na Hospodina Boha svého, nebo on dává tobe moc k dobývání zboží, aby utvrdil smlouvu svou, kterouž s prísahou ucinil s otci tvými, jakož to ukazuje dnešní den.
\par 19 Pakli zapomena se na Hospodina Boha svého, postoupil bys po bozích cizích, a sloužil bys jim a klanel bys se jim: osvedcuji proti vám dnes, že konecne zahynete.
\par 20 Jako pohané, kteréž Hospodin zahladil pred tvárí vaší, tak zahynete, proto že jste neposlouchali hlasu Hospodina Boha svého.

\chapter{9}

\par 1 Slyš, Izraeli, ty prejdeš dnes pres Jordán, abys vejda, dedicne vládl národy vetšími a silnejšími, než jsi ty, mesty velikými a ohrazenými až k nebi,
\par 2 Lidem velikým a vysokým, syny Enakovými, o kterýchž víš, a o kterýchž jsi slyšel praviti: Kdo se postaví proti synum Enakovým?
\par 3 Protož veziž dnes, že Hospodin Buh tvuj, kterýž jde pred tebou, jest jako ohen spalující. On vyhladí je, a on sníží je pred tebou; i vyženeš je a vyhladíš je rychle, jakož mluvil tobe Hospodin.
\par 4 Neríkejž v srdci svém, když by je zapudil Hospodin Buh tvuj od tvári tvé, rka: Pro spravedlnost mou uvedl mne Hospodin, abych dedicne obdržel zemi tuto, tak jako pro bezbožnost národu tech Hospodin vyhnal je od tvári tvé.
\par 5 Ne pro spravedlnost svou a pravost srdce svého ty jdeš, abys dedicne obdržel zemi jejich, ale pro bezbožnost národu tech Hospodin Buh tvuj vyhání je od tvári tvé, a aby splnil slovo, kteréž s prísahou zaslíbil otcum tvým, Abrahamovi, Izákovi a Jákobovi.
\par 6 Protož veziž, že ne pro spravedlnost tvou Hospodin Buh tvuj dává tobe zemi tu výbornou, abys dedicne držel ji, ponevadž jsi lid tvrdé šíje.
\par 7 Pamatujž a nezapomínej, že jsi k hnevu popouzel Hospodina Boha svého na poušti; hned od toho dne, když jsi vyšel z zeme Egyptské, až jste prišli na místo toto, odporni jste byli Hospodinu.
\par 8 Také i na Orébe popudili jste k hnevu Hospodina, a rozhneval se na vás Hospodin, aby vás shladil.
\par 9 Když jsem vstoupil na horu, abych vzal dsky kamenné, dsky smlouvy, kterouž ucinil s vámi Hospodin, tehdáž jsem trval na hore ctyridceti dní a ctyridceti nocí, chleba nejeda a vody nepije.
\par 10 I dal mi Hospodin dve dsky kamenné, psané prstem Božím, na nichž byla všecka ta slova, kteráž mluvil vám Hospodin na hore z prostredku ohne v den shromáždení.
\par 11 Po skonání pak ctyridceti dnu a ctyridceti nocí dal mi Hospodin dve dsky kamenné, dsky smlouvy.
\par 12 I rekl mi Hospodin: Vstana, sstup rychle odsud, nebo poškvrnil se lid tvuj, kterýž jsi vyvedl z Egypta; sešli brzy s cesty, kterouž jsem jim prikázal, a ucinili sobe slitinu.
\par 13 Mluvil také Hospodin ke mne, rka: Videl jsem lid ten, a jiste lid tvrdošijný jest.
\par 14 Pust mne, at je setru, a zahladím jméno jejich pod nebem, tebe pak uciním v národ vetší a silnejší, nežli jest tento.
\par 15 Tedy obrátiv se, sstoupil jsem s hory, (hora pak horela ohnem), dve dsky smlouvy maje v obou rukou svých.
\par 16 Když jsem pak pohledel a uzrel jsem, že jste zhrešili Hospodinu Bohu vašemu, udelavše sobe tele lité, (brzy jste byli sešli s cesty, kterouž prikázal vám Hospodin):
\par 17 Pochytiv ty dve dsky, povrhl jsem je z obou rukou svých, a polámal jsem je pred ocima vašima.
\par 18 A padna, ležel jsem pred Hospodinem jako i prvé, ctyridceti dní a ctyridceti nocí, chleba nejeda a vody nepije, pro všecky hríchy vaše, kterýmiž jste byli zhrešili, ciníce to, což zlého jest pred ocima Hospodinovýma, a popouzejíce jeho.
\par 19 Nebo bál jsem se prchlivosti a hnevu, kterýmž se byl popudil Hospodin proti vám, aby zahladil vás, a uslyšel mne Hospodin i tehdáž.
\par 20 Na Arona též rozhneval se byl Hospodin náramne, tak že ho zahladiti chtel; tedy modlil jsem se také za Arona téhož casu.
\par 21 Hrích pak váš, kterýž jste byli ucinili, totiž tele, vzav, spálil jsem je ohnem a zdrobil jsem je, tluka je dobre, dokudž nebylo setríno na prach. Potom vsypal jsem prach jeho do potoka, kterýž tekl s té hory.
\par 22 Ano i v Tabbera a v Massah a v Kibrot Hattáve popouzeli jste Hospodina k hnevu.
\par 23 A když vás poslal Hospodin z Kádesbarne, rka: Vstupte a opanujte zemi tu, kterouž jsem vám dal, odporni jste byli reci Hospodina Boha svého, a neverili jste jemu, aniž jste uposlechli hlasu jeho.
\par 24 Odporni jste byli Hospodinu od toho dne, jakž jsem vás poznal.
\par 25 A padna, ležel jsem pred Hospodinem ctyridceti dní a ctyridceti nocí, v nichž jsem rozprostíral se; nebo rekl Hospodin, že vás chce zahladiti.
\par 26 I modlil jsem se Hospodinu, rka: Panovníce Hospodine, nezatracujž lidu svého, a to dedictví svého, kteréž jsi vykoupil velikomocností svou, kteréž jsi vyvedl z Egypta v ruce silné.
\par 27 Rozpomeniž se na služebníky své, Abrahama, Izáka a Jákoba; nepatriž na tvrdost lidu tohoto, na bezbožnost jeho a na hríchy jeho,
\par 28 Aby nerekli obyvatelé zeme té, z níž jsi nás vyvedl: Proto že nemohl Hospodin uvésti jich do zeme, kterouž zaslíbil jim, aneb že je mel v nenávisti, vyvedl je, aby je pobil na poušti.
\par 29 Však oni jsou lid tvuj a dedictví tvé, kteréž jsi vyvedl v síle své veliké a v rameni svém vztaženém.

\chapter{10}

\par 1 Toho casu rekl mi Hospodin: Vyhlad sobe dve dsky kamenné, podobné prvním, a vstup ke mne na horu, a udelej sobe truhlu drevenou.
\par 2 I napíši na dskách tech slova, kteráž byla na dskách prvních, kteréž jsi rozrazil, a vložíš je do té truhly.
\par 3 Tedy udelal jsem truhlu z dríví setim, a vyhladiv dve dsky kamenné, podobné prvním, vstoupil jsem na horu, nesa ty dve dsky v rukou svých.
\par 4 I napsal na tech dskách, tak jakž prvé byl napsal, deset slov, kteráž mluvil k vám Hospodin na hore z prostredku ohne v den shromáždení onoho, a dal je Hospodin mne.
\par 5 A obrátiv se, sstoupil jsem s hory té, a vložil jsem ty dsky do truhly, kterouž jsem byl udelal, a byly tam, jakož mi prikázal Hospodin.
\par 6 Synové pak Izraelští hnuli se z Beerot synu Jakan do Moserah. Tam umrel Aron, a tu jest pochován; i konal úrad knežský na míste jeho Eleazar, syn jeho.
\par 7 Odtud táhli do Gadgad, a z Gadgad do Jotbata, do zeme vod tekutých.
\par 8 Toho casu oddelil Hospodin pokolení Léví, aby nosili truhlu smlouvy Hospodinovy, a aby stáli pred Hospodinem k službe jemu, a k dobrorecení ve jménu jeho až do dnešního dne.
\par 9 Procež nemelo pokolení Léví dílu a dedictví s bratrími svými, nebo Hospodin jest dedictví jeho, jakož mluvil k nemu Hospodin Buh tvuj.
\par 10 Já pak zustal jsem na té hore jako dnu prvních, ctyridceti dní a ctyridceti nocí, a uslyšel mne Hospodin i tehdáž, a nechtel shladiti tebe.
\par 11 Potom rekl mi Hospodin: Vstan, jdi, predcházeje lid, aby vešli a vládli zemí, kterouž jsem s prísahou zaslíbil otcum jejich, že jim ji dám.
\par 12 Nyní tedy, Izraeli, ceho žádá Hospodin Buh tvuj od tebe? Jediné abys se bál Hospodina Boha svého, a chodil po všech cestách jeho, a abys miloval ho, a sloužil Hospodinu Bohu svému v celém srdci svém, a ve vší duši své,
\par 13 Ostríhaje prikázaní Hospodinových a ustanovení jeho, kteráž já prikazuji tobe dnes k tvému dobrému.
\par 14 Aj, Hospodina Boha tvého jest nebe i nebesa nebes, zeme a všecky veci, kteréž jsou na ní.
\par 15 Však toliko v otcích tvých zalíbilo se Hospodinu, aby je zamiloval, a vyvolil síme jejich po nich, vás totiž ze všech národu, jakož dnes vidíš.
\par 16 Protož obrežtež neobrízku srdce svého, a šíje své nezatvrzujte více.
\par 17 Nebo Hospodin Buh váš, on jest Buh bohu, a Pán pánu, Buh silný, veliký, mocný a hrozný, kterýž neprijímá osoby, ani bére daru,
\par 18 Cine soud sirotku a vdove, miluje také príchozího, dávaje mu chléb a odev.
\par 19 Protož milujte hoste, nebo jste byli hosté v zemi Egyptské.
\par 20 Hospodina Boha svého báti se budeš, jemu sloužiti, a jeho se prídržeti, a ve jménu jeho prisahati budeš.
\par 21 Ont jest chvála tvá, a ont jest Buh tvuj, kterýž ucinil s tebou tyto veliké a hrozné veci, kteréž videly oci tvé.
\par 22 V sedmdesáti dušech sstoupili otcové tvoji do Egypta, nyní pak rozmnožil te Hospodin Buh tvuj, abys byl v množství jako hvezdy nebeské.

\chapter{11}

\par 1 Milujž tedy Hospodina Boha svého a ostríhej narízení jeho, ustanovení a soudu jeho, i prikázaní jeho po všecky dny.
\par 2 A znejtež dnes (nebo ne k samým synum vašim mluvím, kteríž neznali toho, ani nevideli), trestání Hospodina Boha svého, dustojnost jeho, ruku jeho silnou a ráme jeho vztažené,
\par 3 A znamení i skutky jeho, kteréž cinil u prostred Egypta Faraonovi, králi Egyptskému, i vší zemi jeho,
\par 4 A co ucinil vojsku Egyptskému, konum i vozum jeho, kterýž uvedl vody more Rudého na ne, když vás honili, a shladil je Hospodin až do dnešního dne;
\par 5 Také co ucinil vám na poušti, dokudž jste neprišli až k místu tomuto,
\par 6 A co ucinil Dátanovi a Abironovi, synum Eliaba, syna Rubenova, když zeme otevrela ústa svá, a požrela je i celedi jejich, stany jejich se vším statkem jejich, kterýž pri sobe meli, u prostred všeho Izraele.
\par 7 Ale oci vaše videly všecky skutky Hospodinovy veliké, kteréž cinil.
\par 8 Protož ostríhejte všech prikázaní, kteráž já dnes prikazuji vám, abyste zmocneni byli, a vejdouce, dedicne obdrželi zemi, do kteréž vy jdete k dedicnému jí obdržení,
\par 9 A aby se prodlili dnové vaši na zemi, kterouž s prísahou zaslíbil dáti Hospodin otcum vašim i semeni jejich, zemi mlékem a strdí oplývající.
\par 10 Nebo zeme, do kteréž ty již vcházíš, abys ji dedicne obdržel, není jako zeme Egyptská, z níž jsi vyšel, v kteréž jsi rozsíval síme své, a svlažoval ji do ustání noh svých jako zahradu bylinnou:
\par 11 Ale zeme, do kteréž vy jdete, abyste jí dedicne vládli, jest zeme hornatá, mající i údolí, kteráž z dešte nebeského svlažována bývá vodou,
\par 12 Zeme, o kterouž Hospodin Buh tvuj pecuje, a vždycky oci Hospodina Boha tvého obráceny jsou na ni, od pocátku roku až do konce jeho.
\par 13 Protož jestliže opravdove poslouchati budete prikázaní mých, kteráž já vám dnes prikazuji, milujíce Hospodina Boha svého a sloužíce jemu z celého srdce svého a ze vší duše své,
\par 14 Dám déšt zemi vaší casem svým, ranní i pozdní, a sklízeti budeš obilé své, víno své i olej svuj.
\par 15 Dám i pastvu na poli tvém pro hovada tvá, a budeš jísti až do sytosti.
\par 16 Vystríhejtež se tedy, aby nebylo svedeno srdce vaše, abyste odstupujíce, nesloužili bohum cizím, a neklaneli se jim.
\par 17 Procež Hospodin velice by se na vás rozhneval, a zavrel by nebe, aby dešte nedávalo, a zeme aby nedávala úrody své; i zahynuli byste rychle z zeme výborné, kterouž Hospodin vám dává.
\par 18 Ale složte tato slova má v srdci svém a v mysli své, a uvažte je sobe za znamení na rukou svých, a budou jako nácelník mezi ocima vašima.
\par 19 A vyucujte jim syny své, rozmlouvajíce o nich, když sedneš v dome svém, aneb když pujdeš cestou, když lehneš i když vstaneš.
\par 20 Napíšeš je také na verejích domu svého i na branách svých,
\par 21 Aby byli rozmnoženi dnové vaši a dnové synu vašich na zemi, kterouž s prísahou zaslíbil Hospodin otcum vašim, že ji dá jim, dokudž nebe trvá nad zemí.
\par 22 Nebo jestliže bedlive ostríhati budete všech prikázaní techto, kteráž já prikazuji vám, abyste je cinili, milujíce Hospodina Boha svého, a chodíce po všech cestách jeho, a prídržejíce se jeho:
\par 23 Tedy vyžene Hospodin všecky ty národy od tvári vaší, a vládnouti budete dedicne národy vetšími i silnejšími, nežli jste vy.
\par 24 Všeliké místo, na kteréž by vstoupila noha vaše, vaše bude; od poušte a od Libánu, a od reky Eufraten až k mori nejdalšímu bude pomezí vaše.
\par 25 Nepostaví se žádný proti vám; strach a bázen vás pustí Hospodin Buh váš na tvár vší zeme, po kteréž choditi budete, jakož jest mluvil vám.
\par 26 Hle, já predkládám vám dnes požehnání i zlorecení,
\par 27 Požehnání, budete-li poslouchati prikázaní Hospodina Boha svého, kteráž já dnes prikazuji vám,
\par 28 Zlorecení pak, jestliže byste neposlouchali prikázaní Hospodina Boha svého, ale sešli byste s cesty, o kteréž já dnes prikazuji vám, následujíce bohu cizích, kterýchž neznáte.
\par 29 A když by te uvedl Hospodin Buh tvuj do zeme, do kteréž ty jdeš, abys dedicne vládl jí, tedy dáš požehnání toto na hore Garizim, a zlorecení na hore Hébal,
\par 30 Kteréž jsou za Jordánem, za cestou chýlící se k západu slunce, v zemi Kananejského, jenž bydlí na rovinách naproti Galgala, blízko rovin More.
\par 31 Nebo vy pujdete pres Jordán, abyste vešli a dedicne vládli zemí, kterouž Hospodin Buh váš dá vám; i obdržíte ji dedicne, a budete v ní bydliti.
\par 32 Hledtež tedy, abyste cinili všecka ustanovení a soudy, kteréž já dnes vám predkládám.

\chapter{12}

\par 1 Tato jsou ustanovení a soudové, kterýchž ostríhati budete, ciníce je v zemi, kterouž Hospodin Buh otcu tvých dá tobe, abys dedicne vládl jí po všecky dny, v nichž živi budete na zemi.
\par 2 Zkazíte do gruntu všecka místa, na nichž sloužili národové, (kteréž vy dedicne opanujete), bohum svým, na vrších vysokých a na pahrbcích, a pod každým stromem ratolestným.
\par 3 Oltáre jejich zborte, a obrazy jejich ztlucte, háje také jejich ohnem spalte a rytiny bohu jejich stroskotejte, a vyhladte jméno jejich z místa toho.
\par 4 Neuciníte tak Hospodinu Bohu svému,
\par 5 Ale kteréž by místo vyvolil Hospodin Buh váš ze všech pokolení vašich, aby položil jméno své tu, a bydlil na nem, hledati ho, a tam choditi budete.
\par 6 Tam také prinášeti budete zápaly své a obeti i desátky své, obeti rukou svých i sliby své, dobrovolné obeti své, i prvorozené z skotu a bravu svých,
\par 7 A jísti budete tam pred Hospodinem Bohem svým, a veseliti se budete v každé veci, k níž priciníte ruky své, vy i domové vaši,v nichž požehnal tobe Hospodin Buh tvuj.
\par 8 Nebudete delati tak, jakž my nyní zde ciníme, jeden každý což se mu dobrého vidí.
\par 9 Nebo až dosavad neprišli jste k odpocinutí a dedictví, kteréž Hospodin Buh tvuj dává tobe.
\par 10 Ale když prejdouce Jordán, bydliti budete v zemi, kterouž Hospodin Buh váš dá vám, abys jí vládl právem dedicným, a odpocinutí dá vám ode všech vukol neprátel vašich, a bydliti budete bezpecne:
\par 11 Tedy k místu tomu, kteréž by vyvolil Hospodin Buh váš, aby tam prebývalo jméno jeho, prinášeti budete všecky veci, kteréž já prikazuji vám, zápaly své, obeti své a desátky své, a obeti rukou svých a všecko, což predního jest v slibích vašich, kteréž byste cinili Hospodinu.
\par 12 A veseliti se budete pred Hospodinem Bohem svým, vy i synové vaši i dcery vaše, služebníci vaši i devky vaše, i Levíta, kterýž jest v branách tvých, nebo nemají dílu a dedictví s vámi.
\par 13 Vystríhej se, abys neobetoval zápalných obetí svých na žádném míste, kteréž bys sobe obhlédl.
\par 14 Ale na tom míste, kteréž by vyvolil Hospodin v jednom z pokolení tvých, tam obetovati budeš zápaly své, a tam uciníš všecko, což já prikazuji tobe.
\par 15 A však jestliže by se kdy zalíbilo duši tvé, zabiješ sobe, a jísti budeš maso vedlé požehnání Hospodina Boha svého, kteréž by dal tobe ve všech mestech tvých; necistý i cistý jísti je bude, jako srnu i jelena.
\par 16 Krve toliko jísti nebudete, na zemi vycedíte ji jako vodu.
\par 17 Nebudeš moci jísti v meste svém desátku obilí svého, vína a oleje svého, i prvorozených skotu a bravu svých, i všeho, k cemuž se slibem zavážeš, dobrovolných obetí svých a obetí rukou svých.
\par 18 Ale pred Hospodinem Bohem svým jísti je budeš na míste, kteréž vyvolil Hospodin Buh tvuj, ty i syn tvuj i dcera tvá, služebník tvuj a devka tvá, i Levíta, kterýž jest v branách tvých, a veseliti se budeš pred Hospodinem Bohem svým ve všech vecech, k nimž bys pricinil ruky své.
\par 19 Vystríhejž se, abys neopouštel Levítu v zemi své po všecky dny své.
\par 20 Když rozšírí Hospodin Buh tvuj pomezí tvé, jakož mluvil tobe, a rekl bys sám v sobe: Budu jísti maso, proto že žádá duše tvá jísti maso, vedlé vší líbosti své jísti budeš maso.
\par 21 Jestliže by daleko bylo od tebe místo, kteréž vyvolí Hospodin Buh tvuj, aby prebývalo tam jméno jeho, zabiješ hovado z skotu neb bravu svých, kteréž by dal Hospodin tobe, jakžt jsem prikázal tobe, a jísti budeš v meste svém, vedlé vší líbosti duše své.
\par 22 Jakž se jídá srna a jelen, tak je jísti budeš; necistý i cistý bude je jísti.
\par 23 Toliko bud stálý, abys krve nejedl; nebo krev jest duše, protož nebudeš jísti duše s masem jejím.
\par 24 Nejeziž jí, ale na zem ji vyced jako vodu.
\par 25 Nebudeš jísti jí, aby tobe dobre bylo i synum tvým po tobe, když bys cinil, což dobrého jest pred ocima Hospodinovýma.
\par 26 Ale posvecené veci své, kteréž bys mel, a což bys slíbil, vezmeš, a doneseš na místo, kteréž vyvolí Hospodin,
\par 27 A obetovati budeš obeti své zápalné, maso a krev, na oltár Hospodina Boha svého; ale krev jiných obetí tvých vylita bude na oltár Hospodina Boha tvého, maso pak jísti budeš.
\par 28 Ostríhej a poslouchej všech slov techto, kteráž já prikazuji tobe, aby tobe dobre bylo i synum tvým po tobe až na veky, když bys cinil, což dobrého a pravého jest, pred ocima Hospodina Boha svého.
\par 29 Když by pak vyplénil Hospodin Buh tvuj od tvári tvé národy ty, k nimž ty tudíž vejdeš, abys dedicne vládl jimi, a dedicne je opanoval, a bydlil v zemi jejich,
\par 30 Vystríhej se, abys neuvázl v osídle, jda za nimi, když by již vyhlazeni byli od tvári tvé. Nevyptávej se na bohy jejich, ríkaje: Jak jsou sloužili národové ti bohum svým, tak i já uciním.
\par 31 Neuciníš tak Hospodinu Bohu svému, nebo všecko, což v ohavnosti má Hospodin, a cehož nenávidí, cinili bohum svým; také i syny své a dcery své ohnem pálili ke cti bohum svým.
\par 32 Cožkoli já prikazuji vám, ostríhati budete, ciníce to; nepridáte k tomu, aniž co z toho ujmete.

\chapter{13}

\par 1 Povstal-li by u prostred tebe prorok, aneb nekdo, ješto by míval sny, a ukázal by tobe znamení aneb zázrak,
\par 2 Byt se pak i stalo to znamení aneb zázrak, a mluvil by tobe, rka: Podme, následujme bohu cizích, (kterýchž neznáš,) a služme jim:
\par 3 Neuposlechneš slov proroka toho, ani snáre, nebo zkušuje vás Hospodin Buh váš, aby známé bylo, milujete-li Hospodina Boha svého z celého srdce svého a z celé duše své.
\par 4 Hospodina Boha svého následujte, a jeho se bojte, prikázaní jeho ostríhejte, hlasu jeho poslouchejte, jemu služte a jeho se prídržte.
\par 5 Prorok pak ten aneb snár zamordován bude, nebo mluvil to, címž by odvrátil vás od Hospodina Boha vašeho, (kterýž vyvedl vás z zeme Egyptské, a vykoupil te z domu služby,) aby te srazil s cesty, kterouž prikázal tobe Hospodin Buh tvuj, abys chodil po ní; a tak odejmeš zlé z prostredku svého.
\par 6 Jestliže by te tajne nabádal bratr tvuj, syn matky tvé, aneb syn tvuj, aneb dcera tvá, aneb manželka, kteráž jest v lune tvém, aneb prítel tvuj, kterýžt by milý byl jako duše tvá, rka: Podme a služme bohum cizím, (kterýchž jsi neznal ty ani otcové tvoji),
\par 7 Z bohu pohanských, kteríž vukol vás jsou, bud blízko aneb daleko od tebe, od jednoho konce zeme až do druhého konce jejího,
\par 8 Nepovoluj jemu, aniž ho poslouchej;neodpustít mu oko tvé, aniž se slituješ, aniž zatajíš ho,
\par 9 Ale bez milosti zabiješ jej; ruka tvá nejprv na nej bude vztažena, abys ho zabil, a potom ruce všeho lidu.
\par 10 A ukamenuješ jej až do smrti, nebo chtel odvésti tebe od Hospodina Boha tvého, kterýž te vyvedl z zeme Egyptské, z domu služby,
\par 11 Aby všecken lid Izraelský, uslyšíce to, báli se, a necinili více neco podobného veci této nejhorší u prostred tebe.
\par 12 Uslyšel-li bys o nekterém meste svém, kteréž Hospodin Buh tvuj dá tobe, abys tam bydlil, ani praví:
\par 13 Vyšli muži, lidé nešlechetní z prostredku tvého, kteríž svedli spoluobyvatele své, rkouce: Podme, služme bohum cizím, kterýchž neznáte:
\par 14 Tedy vyhledej a vyzvez a dobre se na to vyptej, a jestliže jest pravda a jistá vec, že ohavnost se stala u prostred tebe,
\par 15 Zbiješ obyvatele mesta toho mecem, a jako vec proklatou zkazíš je i všecko, což by v nem bylo, hovada také jeho mecem pobiješ.
\par 16 A všecky koristi jeho shromáždíš u prostred ulice jeho, a spálíš ohnem to mesto i všecky koristi jeho docela Hospodinu Bohu svému, aby byla hromada rumu vecná; a nebude více staveno.
\par 17 A nevezmeš nicehož z vecí proklatých, aby Hospodin odvrátil se od hnevu prchlivosti své, a ucinil tobe milosrdenství svá, a smiloval se nad tebou, i rozmnožil te, jakož s prísahou zaslíbil otcum tvým.
\par 18 Protož poslouchati budeš hlasu Hospodina Boha svého, ostríhaje všech prikázaní jeho, kteráž já dnes prikazuji tobe, abys cinil, což pravého jest, pred ocima Hospodina Boha svého.

\chapter{14}

\par 1 Synové jste Hospodina Boha vašeho, protož nebudete se rezati, aniž sobe udeláte lysiny mezi ocima vašima nad mrtvým.
\par 2 Nebo lid svatý jsi Hospodinu Bohu svému, a tebe vyvolil Hospodin, abys jemu byl za lid zvláštní ze všech národu, kteríž jsou na tvári zeme.
\par 3 Nebudeš jísti žádné veci ohavné.
\par 4 Tato jsou hovada, kteráž jísti budete: Voly, ovce a kozy,
\par 5 Jelena, dannele, srnu, kamsíka, jezevce, buvola a losa.
\par 6 Každé hovado, kteréž má kopyta rozdelená, tak aby rozdvojená byla, a prežívá mezi hovady, jísti je budete.
\par 7 A však ne všech prežívajících, aneb tech, kteráž kopyta rozdelená mají, budete jísti, jako velblouda, zajíce a králíka; nebo ac prežívají, však kopyta rozdeleného nemají, necistá jsou vám.
\par 8 Též svine, nebo rozdelené majíc kopyto, neprežívá, necistá vám bude; masa jejího jísti nebudete, a mrchy její se nedotknete.
\par 9 Ze všech pak živocichu, kteríž u vodách jsou, tyto jísti budete: Cožkoli má plejtvy a šupiny, jísti budete.
\par 10 Což pak nemá plejtví a šupin, toho jísti nebudete; necisté vám bude.
\par 11 Všecko ptactvo cisté jísti budete.
\par 12 Techto pak jísti nebudete: Orla, noha, orlice morské,
\par 13 A sokola, supa a lunáka vedlé pokolení jeho,
\par 14 A žádného krkavce vedlé pokolení jeho,
\par 15 Pstrosa, sovy, vodní káne a krahulce vedlé pokolení jeho,
\par 16 Raroha, kalousa a labuti,
\par 17 Pelikána, porfiriána a krehare,
\par 18 Cápa, volavky vedlé pokolení jejího, dedka a netopýre.
\par 19 A všeliký zemeplaz létající necistý bude vám, nebudete ho jísti.
\par 20 Každého ptáka cistého jísti budete.
\par 21 Žádné umrliny jísti nebudete; príchozímu, kterýž jest v branách tvých, dáš ji, a jísti ji bude, aneb prodáš cizozemci, nebo lid svatý jsi Hospodinu Bohu svému. Nebudeš variti kozelce v mléce matky jeho.
\par 22 Ochotne dávati budeš desátky ze všech užitku semene svého, kterížt by prišli s pole každého roku.
\par 23 A jísti budeš pred Hospodinem Bohem svým, (na míste, kteréž by vyvolil, aby tam prebývalo jméno jeho,) desátky z obilí, vína i oleje svého, a prvorozené z volu svých a drobného dobytka svého, abys se ucil báti Hospodina Boha svého po všecky dny.
\par 24 Jestliže by pak daleká byla cesta, a nemohl bys donésti toho, proto že daleko jest od tebe to místo, kteréž by vyvolil Hospodin Buh tvuj k prebývání tam jména svého, když požehná tobe Hospodin Buh tvuj:
\par 25 Tedy zpenežíš je, a svázané peníze vezma v ruku svou, pujdeš k místu, kteréž by vyvolil Hospodin Buh tvuj,
\par 26 A vynaložíš ty peníze na všecko, cehož žádá duše tvá, na voly, na ovce, na víno, aneb jiný nápoj silný, a na všecko, cehož by sobe žádala duše tvá, a jísti budeš tam pred Hospodinem Bohem svým, a veseliti se budeš ty i dum tvuj.
\par 27 Levíty pak, kterýž by v branách tvých bydlil, neopustíš, nebo nemá dílu a dedictví s tebou.
\par 28 Každého léta tretího oddelíš všecky desátky z užitku svých toho léta, a složíš je v branách svých.
\par 29 I prijde Levíta, (nebo nemá dílu a dedictví s tebou), a host a sirotek i vdova, kteríž jsou v branách tvých, i budou jísti a nasytí se, aby požehnal tobe Hospodin Buh tvuj pri všelikém díle rukou tvých, kteréž bys delal.

\chapter{15}

\par 1 Každého léta sedmého odpoušteti budeš.
\par 2 Tento pak bude zpusob odpuštení, aby odpustil každý veritel, kterýž rukou svou pujcil to, cehož pujcil bližnímu svému; nebude upomínati bližního svého aneb bratra svého, nebo vyhlášeno jest odpuštení Hospodinovo.
\par 3 Cizozemce upomínati budeš, ale to, což bys mel u bratra svého, propustí ruka tvá.
\par 4 Toliko aby nuzným nekdo nebyl prícinou tvou, ponevadž hojne požehná tobe Hospodin v zemi, kterouž Hospodin Buh tvuj dá tobe v dedictví, abys jí dedicne vládl:
\par 5 Jestliže však pilne poslouchati budeš hlasu Hospodina Boha svého, tak abys hledel ciniti každé prikázaní toto, kteréž já tobe dnes prikazuji.
\par 6 Hospodin zajisté Buh tvuj požehná tobe, jakož mluvil tobe, tak že budeš moci pujcovati národum mnohým, tobe pak nebude potreba vypujcovati; i budeš panovati nad národy mnohými, ale oni nad tebou nebudou panovati.
\par 7 Byl-li by u tebe nuzný nekdo z bratrí tvých, v nekterém meste tvém, v zemi tvé, kterouž Hospodin Buh tvuj dá tobe, nezatvrdíš srdce svého, a nezavreš ruky své pred nuzným bratrem svým:
\par 8 Ale štedre otevreš jemu ruku svou, a ochotne pujcíš jemu, jakž by mnoho potreboval toho, v cemž by nouzi mel.
\par 9 Vystríhej se, aby nebylo neco nepravého v srdci tvém, a rekl bys: Blíží se rok sedmý, jenž jest rok odpuštení, a bylo by nešlechetné oko tvé k bratru tvému nuznému, tak že bys neudelil jemu, procež by volal proti tobe k Hospodinu, a byl by na tobe hrích:
\par 10 Ale ochotne dáš jemu, a nebude srdce tvé neuprímé, když bys dával jemu; nebo tou prícinou požehná tobe Hospodin Buh tvuj ve všech skutcích tvých a ve všem díle, k kterémuž bys vztáhl ruku svou.
\par 11 Nebo nebudete bez chudých v zemi vaší; protož prikazuji tobe, rka: Abys ochotne otvíral ruku svou bratru svému, chudému svému a nuznému svému v zemi své.
\par 12 Jestliže by prodán byl tobe bratr tvuj Žid aneb Židovka, a sloužil by tobe za šest let, sedmého léta propustíš jej od sebe svobodného.
\par 13 A když jej propustíš svobodného od sebe, nepustíš ho prázdného.
\par 14 Štedre darovati jej budeš dary z dobytka svého, z stodoly a z vinice své; v cemž požehnal tobe Hospodin Buh tvuj, z toho dáš jemu.
\par 15 A pamatuj, že jsi služebníkem byl v zemi Egyptské, a že te vykoupil Hospodin Buh tvuj, protož já to dnes tobe prikazuji.
\par 16 Paklit by rekl: Nepujdu od tebe, proto že by te miloval i dum tvuj, a že mu dobre u tebe,
\par 17 Tedy vezma špici, probodneš ucho jeho na dverích, a bude u tebe služebníkem na veky. Takž podobne i devce své uciníš.
\par 18 Necht není za težké pred ocima tvýma, když bys ho svobodného propustil od sebe, nebo dvojnásob více, než ze mzdy nájemník, sloužil tobe šest let; i požehná tobe Hospodin ve všech vecech, kteréž ciniti budeš.
\par 19 Všeho prvorozeného, což se narodí z skotu tvých neb z bravu tvých, samce posvetíš Hospodinu Bohu svému. Nebudeš delati prvorozeným volkem svým, a nebudeš holiti prvorozených ovec svých.
\par 20 Pred Hospodinem Bohem svým budeš je jísti na každý rok na míste, kteréž by vyvolil Hospodin, ty i celed tvá.
\par 21 Paklit by na nem byla vada, že by kulhavé aneb slepé bylo, aneb melo by jakoukoli škodlivou vadu, nebudeš ho obetovati Hospodinu Bohu svému.
\par 22 V branách svých budeš je jísti, budto cistý neb necistý, rovne jako srnu aneb jelena;
\par 23 Toliko krve jeho nebudeš jísti, ale na zem vycedíš ji jako vodu.

\chapter{16}

\par 1 Ostríhej mesíce Abib, abys slavil Fáze Hospodinu Bohu svému, nebo toho mesíce Abib vyvedl te Hospodin Buh tvuj z Egypta v noci.
\par 2 A obetovati budeš Fáze Hospodinu Bohu svému z bravu i skotu na míste, kteréž by vyvolil Hospodin, aby tam prebývalo jméno jeho.
\par 3 Nebudeš v nem jísti nic kvašeného. Za sedm dní jísti budeš presnice, chléb trápení, (nebo s chvátáním vyšel jsi z zeme Egyptské), abys pamatoval na den, v kterémž jsi vyšel z zeme Egyptské, po všecky dny života svého.
\par 4 Aniž spatrín bude u tebe kvas ve všech koncinách tvých za sedm dní, a nezustane nic pres noc masa toho, kteréž bys obetoval u vecer toho dne prvního, až do jitra.
\par 5 Nebudeš moci obetovati Fáze v každém meste svém, kteréž Hospodin Buh tvuj dává tobe,
\par 6 Ale na míste, kteréž by vyvolil Hospodin Buh tvuj, aby prebývalo tam jméno jeho, tu obetovati budeš Fáze, a to u vecer, pri západu slunce v jistý cas vyjití tvého z Egypta.
\par 7 Péci pak budeš a jísti na míste, kteréž by vyvolil Hospodin Buh tvuj, a ráno navracuje se, pujdeš do stanu svých.
\par 8 Šest dní jísti budeš presnice, dne pak sedmého bude slavnost Hospodinu Bohu tvému; nicehož v ní delati nebudeš.
\par 9 Sedm téhodnu secteš sobe; od zacátku žne zacneš pocítati sedm téhodnu.
\par 10 Tehdy slaviti budeš slavnost téhodnu Hospodinu Bohu svému; sec budeš moci býti, to dáš dobrovolne vedlé toho, jakžt by požehnal Hospodin Buh tvuj.
\par 11 I veseliti se budeš pred Hospodinem Bohem svým, ty i syn tvuj i dcera tvá, služebník tvuj a devka tvá, i Levíta, kterýž by byl v branách tvých, a príchozí, sirotek i vdova, kteríž by byli u prostred tebe, na míste, kteréž by vyvolil Hospodin Buh tvuj k prebývání tam jména svého.
\par 12 A tak rozpomínati se budeš, že jsi byl služebníkem v Egypte, když ostríhati a vykonávati budeš ustanovení tato.
\par 13 Slavnost stanu svetiti budeš za sedm dní, když shromáždíš s pole svého a z vinice své.
\par 14 I budeš se veseliti v slavnosti své, ty i syn tvuj i dcera tvá, služebník tvuj a služebnice tvá, Levíta i príchozí, sirotek i vdova, kteríž by byli v branách tvých.
\par 15 Sedm dní svátek svetiti budeš Hospodinu Bohu svému na míste, kteréž by vyvolil Hospodin, když požehná tobe Hospodin Buh tvuj ve všech úrodách tvých, a ve všeliké práci rukou tvých, a tak budeš vesel.
\par 16 Trikrát v roce postaví se každý pohlaví mužského pred Hospodinem Bohem tvým na míste, kteréž by vyvolil, na slavnost presnic, na slavnost téhodnu a na slavnost stanu, a neukážet se pred Hospodinem prázdný;
\par 17 Každý podlé daru sobe daného, vedlé požehnání Hospodina Boha tvého, jehož on udelil tobe.
\par 18 Soudce a správce ustanovíš sobe ve všech mestech svých, kteráž Hospodin Buh tvuj dá tobe v každém pokolení tvém, kteríž souditi budou lid soudem spravedlivým.
\par 19 Neuchýlíš soudu, a nebudeš šetriti osoby, aniž prijmeš daru; nebo dar oslepuje oci moudrých, a prevrací slova spravedlivých.
\par 20 Spravedlive spravedlnosti následovati budeš, abys živ byl, a dedicne vládl zemí,kterouž Hospodin Buh tvuj dává tobe.
\par 21 Nevysadíš sobe háje jakýmkoli drívím u oltáre Hospodina Boha svého, kterýž udeláš sobe,
\par 22 Ani vyzdvihneš sobe modly, což v ohavnosti má Hospodin Buh tvuj.

\chapter{17}

\par 1 Nebudeš obetovati Hospodinu Bohu svému vola aneb dobytcete, na nemž by vada byla, aneb jakákoli vec zlá, nebo ohavnost jest Hospodinu Bohu tvému.
\par 2 Bude-li nalezen u prostred tebe v nekterém meste tvém, kteráž Hospodin Buh tvuj dá tobe, muž aneb žena, ješto by se dopouštel zlého pred ocima Hospodina Boha tvého, prestupuje smlouvu jeho,
\par 3 A odejda, sloužil by bohum cizím, a klanel by se jim, bud slunci neb mesíci, aneb kterému rytírstvu nebeskému, cehož jsem neprikázal,
\par 4 A bylo by povedíno tobe, a slyšel bys o tom, tedy vyptáš se pilne na to, a jestliže bude pravda a vec jistá, že by se stala taková ohavnost v Izraeli:
\par 5 Bez lítosti vyvedeš muže toho aneb ženu tu, kteríž to zlé páchali, k branám svým, muže toho aneb ženu, a kamením je uházíš, at zemrou.
\par 6 V ústech dvou aneb trí svedku zabit bude ten, kdož umríti má, nebudet pak zabit podlé vyznání svedka jednoho.
\par 7 Ruka svedku nejprv bude proti nemu k zabití jeho, a potom ruce všeho lidu, a tak odejmeš zlé z prostredku svého.
\par 8 Bylo-li by pri soudu neco nesnadného, mezi krví a krví, mezi prí a prí, mezi ranou a ranou, v jakékoli rozepri v branách tvých, tedy vstana, pujdeš k místu, kteréž by vyvolil Hospodin Buh tvuj,
\par 9 A prijdeš k knežím Levítského pokolení aneb k soudci, kterýž by tehdáž byl, i budeš se jich ptáti, a oznámít výpoved soudu.
\par 10 Uciníš tedy vedlé naucení, kteréž by vynesli z místa toho, kteréž by vyvolil Hospodin, a hled, abys všecko, cemuž by te ucili, tak vykonal.
\par 11 Podlé vyrcení zákona, kterémuž by te naucili, a podlé rozsudku, kterýžt by vypovedeli, uciníš; neuchýlíš se od slova sobe oznámeného ani na pravo ani na levo.
\par 12 Jestliže by pak kdo v zpouru se vydal, tak že by neposlechl kneze postaveného tam k službe pred Hospodinem Bohem tvým, aneb soudce, tedy at umre clovek ten, a odejmeš zlé z Izraele,
\par 13 Aby všecken lid uslyšíce, báli se, a více v zpouru se nevydávali.
\par 14 Když vejdeš do zeme, kterouž Hospodin Buh tvuj dá tobe, abys ji dedicne obdržel, a budeš v ní bydliti, a rekneš: Ustanovím nad sebou krále, jako i jiní národové mají, kteríž jsou vukol mne:
\par 15 Toho toliko ustanovíš nad sebou krále, kteréhož by vyvolil Hospodin Buh tvuj. Z prostredku bratrí svých ustanovíš nad sebou krále; nebudeš moci ustanoviti nad sebou cloveka cizozemce, kterýž by nebyl bratr tvuj.
\par 16 A však at nemívá mnoho konu, a at neobrací zase lidu do Egypta, z príciny rozmnožení konu, zvlášte ponevadž Hospodin vám rekl: Nevracujte se zase cestou touto více.
\par 17 Nebudet také míti mnoho žen, aby se neodvrátilo srdce jeho; stríbra také aneb zlata at sobe príliš nerozmnožuje.
\par 18 Když pak dosedne na stolici království svého, vypíše sobe pripomenutý zákon tento do knihy z té, kteráž bude pred oblícejem kneží Levítských.
\par 19 A bude jej míti pri sobe, a císti jej bude po všecky dny života svého, aby se naucil báti Hospodina Boha svého, a aby ostríhal všech slov zákona tohoto i ustanovení tech, a cinil je,
\par 20 Aby nepozdvihlo se srdce jeho nad bratrí jeho, a neuchýlilo se od prikázaní na pravo aneb na levo, aby dlouho živ byl v království svém, on i synové jeho u prostred Izraele.

\chapter{18}

\par 1 Nebudou míti kneží Levítští a všecko pokolení Levítské dílu a dedictví s jiným lidem Izraelským, obeti ohnivé Hospodinovy a dedictví jeho jísti budou.
\par 2 Dedictví nebudou míti u prostred bratrí svých, Hospodin jest dedictví jejich, jakož jim mluvil.
\par 3 Toto pak náležeti bude knežím od lidu, od obetujících obet, bud vola neb dobytce: Dáno bude knezi plece, a líce a žaludek.
\par 4 Prvotiny obilé svého, vína a oleje svého, i prvotiny vlny z ovec svých dáš jemu.
\par 5 Nebo jej vyvolil Hospodin Buh tvuj ze všech pokolení tvých, aby stál k službe ve jménu Hospodina, on i synové jeho po všecky dny.
\par 6 Když by pak prišel který Levíta z nekterého mesta tvého, ze všeho Izraele, kdež bydlil, a prišel by s velikou žádostí duše své na místo, kteréž by vyvolil Hospodin:
\par 7 Sloužiti bude ve jménu Hospodina Boha svého, jako i jiní bratrí jeho Levítové, kteríž tu stojí pred Hospodinem.
\par 8 Díl rovný jísti budou, (bez škody toho, což by každý z nich prodal,) vedlé celedi otcu.
\par 9 Když vejdeš do zeme, kterouž Hospodin Buh tvuj dává tobe, neuc se ciniti podlé ohavností národu tech.
\par 10 Nebude nalezen v tobe, kdož by vedl syna svého aneb dceru svou skrz ohen, ani veštec, ani planétník, ani carodejník, ani kouzedlník.
\par 11 Ani losník, ani zaklinac, ani hadac, ani cernoknežník.
\par 12 Nebo ohavnost jest Hospodinu, kdožkoli ciní to, a pro takové ohavnosti Hospodin Buh tvuj vymítá je od tvári tvé.
\par 13 Dokonalý budeš pred Hospodinem Bohem svým.
\par 14 Nebo národové ti, kterýmiž vládnouti budeš, planetáru a veštcu poslouchají, tobe pak toho nedopouští Hospodin Buh tvuj.
\par 15 Proroka z prostredku tvého, z bratrí tvých, jako já jsem, vzbudí tobe Hospodin Buh tvuj; jeho poslouchati budete,
\par 16 Podlé všeho toho, jakž jsi žádal od Hospodina Boha svého, na Orébe, v den shromáždení, rka: At více neslyším hlasu Hospodina Boha svého, ani více vidím ohne toho velikého, abych neumrel.
\par 17 Procež mi rekl Hospodin: Dobre mluvili, což jsou mluvili.
\par 18 Proroka vzbudím jim z prostredku bratrí jejich, jako jsi ty, a položím slova má v ústa jeho, a bude jim mluviti všecko, což mu prikáži.
\par 19 I budet, že kdo by koli neposlouchal slov mých, kteráž mluviti bude ve jménu mém, já vyhledávati budu na nem.
\par 20 Prorok pak, kterýž by pyšne se vystavoval, mluve slovo ve jménu mém, kteréhož jsem mluviti jemu neprikázal, a kterýž by mluvil ve jménu bohu cizích, takový prorok at umre.
\par 21 Jestliže pak díš v srdci svém: Kterak poznáme slovo to, jehož by nemluvil Hospodin?
\par 22 Když by mluvil neco prorok ve jménu Hospodinovu, a však by se toho nestalo, aniž by prišlo, tot jest to slovo, jehož nemluvil Hospodin; z pychut jest to mluvil prorok ten, nebojž se ho.

\chapter{19}

\par 1 Když by vyhladil Hospodin Buh tvuj národy, jejichžto zemi Hospodin Buh tvuj dává tobe, a opanoval bys je dedicne, a bydlil bys v mestech jejich i v domích jejich:
\par 2 Oddelíš sobe tri mesta u prostred zeme své, kterouž Hospodin Buh tvuj dává tobe, abys ji dedicne obdržel.
\par 3 Spravíš sobe cesty, a rozdelíš na tri díly všecku krajinu zeme své, kterouž v dedictví dá tobe Hospodin Buh tvuj, i bude utíkati tam každý vražedlník:
\par 4 (Totot pak bude právo vražedlníka, kterýž by tam utekl, aby živ byl: Kdož by zabil bližního svého nechte, aniž by ho nenávidel prvé.
\par 5 Jako vejda nekdo s bližním svým do lesa sekati dríví, rozvedl by rukou svou sekeru, aby utal drevo, a ona by spadla s toporište, a trefila by bližního jeho, tak že by umrel, ten utece do nekterého z mest techto, a živ bude.)
\par 6 Aby prítel zabitého, stihaje vražedlníka toho, když by se rozpálilo srdce jeho, nedohonil ho na daleké ceste, a ranil smrtedlne, ješto by nebyl hoden smrti, ponevadž ho prvé nemel v nenávisti.
\par 7 Protož já prikazuji tobe, rka: Tri mesta oddelíš sobe.
\par 8 Pakli by rozšíril Hospodin Buh tvuj konciny tvé, jakož s prísahou zaslíbil otcum tvým, a dal by tobe všecku tu zemi, kterouž rekl dáti otcum tvým:
\par 9 (Když bys ostríhal všech prikázaní techto, cine je, kteráž já dnes prikazuji tobe, abys miloval Hospodina Boha svého, a chodil po cestách jeho po všecky dny), tedy pridáš sobe ješte tri mesta mimo tri onano,
\par 10 Aby nebyla vylita krev nevinná u prostred zeme tvé, kterouž Hospodin Buh tvuj dá tobe v dedictví, a aby nebyla na tobe krev.
\par 11 Ale byl-li by kdo, maje v nenávisti bližního svého, a cinil by úklady jemu, a povstana proti nemu, ranil by ho smrtedlne, tak že by umrel, a utekl by do nekterého z tech mest:
\par 12 Tedy pošlí starší mesta toho, a vezmou jej odtud, a dají v ruce prítele toho zabitého, aby umrel.
\par 13 Neodpustí jemu oko tvé, ale svedeš krev nevinnou z Izraele, i bude dobre tobe.
\par 14 Nepreneseš meze bližního svého, kterouž vymezili predkové v dedictví tvém, kteréž obdržíš v zemi, jižto Hospodin Buh tvuj dává tobe, abys ji dedicne obdržel.
\par 15 Nepovstane svedek jeden proti nekomu z príciny jakékoli nepravosti, a jakéhokoli hríchu ze všech hríchu, kterýmiž by kdo hrešil; v ústech dvou svedku aneb v ústech trí svedku stane slovo.
\par 16 Povstal-li by svedek falešný proti nekomu, aby svedcil proti nemu, že odstoupil od Boha,
\par 17 Tedy postaví se ti dva muži, kteríž tu rozepri mají pred Hospodinem, pred knežími aneb pred soudci, kteríž by za dnu tech byli.
\par 18 A když se pilne vyptají soudcové, poznají-li, že svedek ten jest svedek lživý, mluve falešné svedectví proti bližnímu svému,
\par 19 Tak jemu ucinte, jakž on myslil uciniti bratru svému. I odejmeš zlé z prostredku svého.
\par 20 A kteríž zustanou, slyšíce to, báti se budou, a nikoli více nedopustí se takových zlých vecí u prostred tebe.
\par 21 A neslituje se oko tvé; život za život, oko za oko, zub za zub, ruka za ruku, noha za nohu bude.

\chapter{20}

\par 1 Když bys vytáhl na vojnu proti neprátelum svým, a uzrel bys kone a vozy, a lid vetší, nežli ty máš, neboj se jich, nebo Hospodin Buh tvuj s tebou jest, kterýž te vyvedl z zeme Egyptské.
\par 2 A když byste se již potýkati meli, pristoupí knez, a mluviti bude k lidu,
\par 3 A dí jim: Slyš, Izraeli, vy jdete dnes k boji proti neprátelum svým, nebudiž strašlivé srdce vaše, nebojte se a nestrachujte, ani se jich lekejte.
\par 4 Nebo Hospodin Buh váš, kterýž jde s vámi, bojovati bude za vás proti neprátelum vašim, aby zachoval vás.
\par 5 Potom mluviti budou hejtmané k lidu, rkouce: Jest-li kdo z vás, ješto vystavel dum nový, a ješte nepocal bydliti v nem, odejdi a navrat se do domu svého, aby nezahynul v bitve, a nekdo jiný aby nezacal bydliti v nem.
\par 6 A kdo jest, ješto štípil vinici, a ješte neprišla k obecnému užívání, odejdi a navrat se k domu svému, aby snad nezahynul v bitve, aby nekdo jiný k obecnému užívání neprivedl jí.
\par 7 A kdo jest, ješto by mel zasnoubenou manželku, a ješte by jí nepojal, odejdi a navrat se do domu svého, aby neumrel v boji, a nekdo jiný nevzal jí.
\par 8 Pridadí i toto hejtmané, mluvíce k lidu, a reknou: Kdo jest bázlivý a lekavého srdce, odejdi a navrat se do domu svého, aby nezemdlil srdce bratrí svých, jako jest srdce jeho.
\par 9 A když prestanou hejtmané mluviti k lidu, tedy predstaví lidu vudce houfu.
\par 10 Když pritáhneš k nekterému mestu, abys ho dobýval, podáš jemu pokoje.
\par 11 Jestliže pokoj sobe podaný prijmou a otevrou tobe, všecken lid, kterýž by nalezen byl v nem, pod plat uvedeni jsouce, tobe sloužiti budou.
\par 12 Pakli by v pokoj s tebou nevešli, ale bojovali proti tobe, oblehneš je;
\par 13 A když by je Hospodin Buh tvuj dal v ruku tvou, tedy zbiješ v nem ostrostí mece všecky pohlaví mužského.
\par 14 Ženy pak, dítky a hovada, i cožkoli bylo by v meste, všecky koristi jeho rozbituješ sobe, a užívati budeš koristí neprátel svých, kteréž by dal tobe Hospodin Buh tvuj.
\par 15 Tak uciníš všechnem mestum daleko vzdáleným od tebe, kteráž nejsou z mest národu techto.
\par 16 Z mest pak lidu toho, kterýž Hospodin Buh tvuj dává tobe v dedictví, žádné duše živiti nebudeš,
\par 17 Ale dokonce vyhladíš je: Hetea, Amorea, Kananea, Ferezea, Hevea a Jebuzea, jakož prikázal tobe Hospodin Buh tvuj,
\par 18 Aby vás neucili ciniti vedlé všech ohavností svých, kteréž ciní bohum svým, i hrešili byste proti Hospodinu Bohu svému.
\par 19 Když oblehneš mesto nekteré, za dlouhý cas dobývaje ho, abys je vzal, nezkazíš stromu jeho, sekerou je vytínaje, nebo z nich ovoce jísti budeš; protož jich nevysekáš, (nebo potrava cloveka jest strom polní), chteje užívati jich k obrane své.
\par 20 A však stromoví, kteréž znáš, že nenese ovoce ku pokrmu, pohubíš a posekáš, a vzdeláš ohrady proti mestu tomu, kteréž s tebou bojuje, dokudž ho sobe nepodmaníš.

\chapter{21}

\par 1 Když by nalezen byl zabitý (v zemi, kterouž Hospodin Buh tvuj dává tobe, abys dedicne vládl jí), ležící na poli, a nebylo by vedíno, kdo by ho zabil,
\par 2 Tedy vyjdou starší tvoji a soudcové tvoji, a meriti budou k mestum, kteráž jsou vukol toho zabitého.
\par 3 A když nalezeno bude mesto nejbližší toho zabitého, tedy vezmou starší mesta toho jalovici z stáda, kteréž ješte nebylo užíváno, a kteráž netáhla ve jhu.
\par 4 I uvedou starší toho mesta jalovici tu do údolí pustého, kteréž nikdy nebylo deláno aneb oseto, a setnou šíji jalovice v tom údolí.
\par 5 Potom pristoupí kneží synové Léví; (nebo je Hospodin Buh tvuj vyvolil, aby prisluhovali jemu, a požehnání dávali ve jménu Hospodinovu, vedlé jejichž výpovedi stane všeliká rozepre a každá rána.)
\par 6 Všickni také starší toho mesta, kteríž jsou nejbližší toho zabitého, umyjí ruce své nad jalovicí statou v tom údolí,
\par 7 A osvedcovati budou, rkouce: Nevylilyt jsou ruce naše krve té, aniž oci naše videly vražedlníka.
\par 8 Ocist lid svuj Izraelský, kterýž jsi vykoupil, Hospodine, a nepricítej krve nevinné lidu svého Izraelskému. I bude snata s nich vina té krve.
\par 9 Ty pak odejmeš krev nevinnou z prostredku svého, když uciníš, což pravého jest pred ocima Hospodinovýma.
\par 10 Když bys vytáhl na vojnu proti neprátelum svým, a dal by je Hospodin Buh tvuj v ruce tvé, a zajal bys z nich mnohé;
\par 11 Uzríš-li mezi zajatými ženu krásného oblíceje, a zamiluješ ji, tak abys ji sobe vzal za manželku:
\par 12 Uvedeš ji do domu svého, i oholí sobe hlavu, a obreže nehty své;
\par 13 A složíc roucho své s sebe, v kterémž jest jata, zustane v dome tvém, a plakati bude otce svého i matky své za celý mesíc; a potom vejdeš k ní, a budeš její manžel, a ona bude manželka tvá.
\par 14 Paklit by se nelíbila, tedy propustíš ji svobodnou; ale nikoli neprodávej jí za peníze, ani nekupc jí, ponevadž jsi jí ponížil.
\par 15 Mel-li by kdo dve ženy, jednu v milosti a druhou v nenávisti, a zplodily by mu syny, milá i nemilá, a byl-li by syn prvorozený nemilé:
\par 16 Tedy když dedice ustavovati bude nad tím, což by mel, nebude moci dáti práva prvorozenství synu milé pred synem prvorozeným nemilé,
\par 17 Ale prvorozeného syna nemilé pri prvorozenství zustaví, a dá jemu dva díly ze všeho, což by mel; nebo on jest pocátek síly jeho, jeho jest právo prvorozenství.
\par 18 Mel-li by kdo syna zpurného a protivného, ješto by neposlouchal hlasu otce svého a hlasu matky své, a jsa trestán, neuposlechl by jich:
\par 19 Tedy vezmouce ho otec i matka jeho, vyvedou jej k starším mesta svého, k bráne místa prebývání svého,
\par 20 A reknou starším mesta svého: Syn náš tento, jsa zpurný a protivný, neposlouchá hlasu našeho, žrác a opilec jest.
\par 21 Tehdy všickni lidé mesta toho uházejí jej kamením a umret; a tak odejmeš zlé z prostredku svého, a všecken Izrael uslyšíce, báti se budou.
\par 22 Když by kdo zhrešil, že by hoden byl smrti, a byl by odsouzen k ní a povesil bys ho na dreve:
\par 23 Nezustane pres noc telo jeho na dreve, ale hned v ten den pochováš jej, nebo zlorecený jest pred Bohem povešený; protož nepoškvrnuj zeme své, kterouž Hospodin Buh tvuj dává tobe v dedictví.

\chapter{22}

\par 1 Jestliže bys uzrel vola aneb dobytce bratra svého, an bloudí, nepomineš jich, ale privedeš je až k bratru svému.
\par 2 Byt pak nebyl blízký bratr tvuj a neznal bys ho, uvedeš je však do domu svého, a bude u tebe, dokudž by se po nem neptal bratr tvuj, a navrátíš mu je.
\par 3 Tolikéž uciníš s oslem jeho, s odevem jeho, také i se všelikou vecí ztracenou bratra svého, kteráž by mu zhynula; když bys ji nalezl, nepomineš jí.
\par 4 Vida osla bratra svého aneb vola jeho pod bremenem ležící na ceste, nepomineš jich, ale i hned ho s ním pozdvihneš.
\par 5 Žena nebude nositi odevu mužského, aniž se obláceti bude muž v roucho ženské, nebo ohavnost pred Hospodinem Bohem tvým jest, kdožkoli ciní to.
\par 6 Když bys našel hnízdo ptací pred sebou na ceste, na jakémkoli stromu aneb na zemi, s mladými neb vejci, a matka sedela by na mladých aneb na vejcích: nevezmeš matky s mladými,
\par 7 Ale hned pustíš matku a mladé vezmeš sobe, aby tobe dobre bylo, a abys prodlil dnu svých.
\par 8 Když bys stavel dum nový, udelej zabradla vukol strechy své, abys neuvedl viny krve na dum svuj, když by kdo upadl s neho.
\par 9 Neposeješ vinice své smesicí rozlicného semene, aby nebyl poškvrnen užitek semene, kteréž jsi vsel, i ovoce vinice.
\par 10 Nebudeš orati volem a oslem spolu.
\par 11 Neobleceš roucha z rozdílných vecí, z vlny a lnu setkaného.
\par 12 Prýmy zdeláš sobe na ctyrech rozích odevu svého, jímž se odíváš.
\par 13 Když by pojal nekdo ženu, a všel by k ní, a potom mel by ji v nenávisti,
\par 14 A dal by prícinu k recem o ní, v zlou povest ji obláceje a mluve: Ženu tuto vzal jsem, a všed k ní, nenalezl jsem jí pannou:
\par 15 Tedy otec devecky a matka její vezmouce prinesou znamení panenství devecky k starším mesta svého k bráne.
\par 16 A dí otec devecky k starším: Dceru svou dal jsem muži tomuto za manželku, kterýž ji v nenávisti má.
\par 17 A hle, sám prícinu dal recem o ní, mluve: Nenalezl jsem pri dceri tvé panenství, a ted hle, znamení panenství dcery mé. I roztáhnou roucho to pred staršími mesta.
\par 18 Tedy starší mesta toho jmou muže a trestati ho budou,
\par 19 A uloží jemu pokutu sto lotu stríbra, kteréž dají otci devecky, proto že vynesl zlou povest proti panne Izraelské. I bude ji míti za manželku, kteréž nebude moci propustiti po všecky dny své.
\par 20 Jinác byla-li by pravá žaloba ta, a nebylo by nalezeno panenství pri devecce:
\par 21 Tedy vyvedou devecku ke dverím otce jejího, a uházejí ji lidé mesta toho kamením, a umre; nebo dopustila se nešlechetnosti v Izraeli, smilnivši v dome otce svého. Tak odejmeš zlé z prostredku sebe.
\par 22 Kdyby kdo postižen byl, že obýval s ženou manželkou cizí, tedy umrou oni oba dva, muž, kterýž obýval s tou ženou, i žena také; i odejmeš zlé z Izraele.
\par 23 Byla-li by devecka panna zasnoubená muži, a nalezna ji nekdo v meste, obýval by s ní:
\par 24 Vyvedete oba dva k bráne mesta toho, a uházíte je kamením, a umrou, devecka, proto že nekricela, jsuci v meste, a muž proto, že ponížil ženy bližního svého; a odejmeš zlé z prostredku svého.
\par 25 Pakli na poli nalezl by muž devecku zasnoubenou, a násilí jí ucine, obýval by s ní: smrtí umre muž ten sám,
\par 26 Devecce pak nic neuciníš. Nedopustila se hríchu hodného smrti; nebo jakož povstává nekdo proti bližnímu svému a morduje život jeho, tak i pri této veci.
\par 27 Na poli zajisté nalezl ji; kricela devecka zasnoubená, a žádný tu nebyl, kdo by ji vysvobodil.
\par 28 Jestliže by nalezl nekdo devecku pannu, kteráž by zasnoubena nebyla, a vezma ji, obýval by s ní, a byli by postiženi:
\par 29 Tedy dá muž ten, kterýž by obýval s ní, otci devecky padesáte stríbrných, a bude jeho manželka, protože ponížil jí; nebude moci jí propustiti po všecky dny své.
\par 30 Nevezme žádný manželky otce svého, a neodkryje podolka otce svého.

\chapter{23}

\par 1 Nevejde do shromáždení Hospodinova, kdož by mel stlucené aneb odnaté luno.
\par 2 Aniž vejde do shromáždení Hospodinova syn postranní; také i desáté koleno jeho nevejde do shromáždení Hospodinova.
\par 3 Ammonitský tolikéž ani Moábský nevejde do shromáždení Hospodinova, ani desáté koleno jejich nevejde do shromáždení Hospodinova až na veky.
\par 4 Proto že proti vám nevyšli s chlebem a s vodou na ceste, když jste šli z Egypta, a že ze mzdy najal proti tobe Baláma, syna Beor, z Petor Mezopotamie Syrské, aby zlorecil tobe:
\par 5 (Ackoli nechtel Hospodin Buh tvuj slyšeti Baláma, ale obrátil Hospodin Buh tvuj tobe zlorecení v požehnání, nebo miloval tebe Hospodin Buh tvuj.)
\par 6 Nebudeš hledati pokoje jejich ani dobrého jejich po všecky dny své na veky.
\par 7 Nebudeš míti v ohavnosti Idumejského, nebo bratr tvuj jest, aniž Egyptského v ohavnosti míti budeš, nebo jsi byl pohostinu v zemi jeho.
\par 8 Synové, kteríž se jim zrodí v tretím kolenu, vejdou do shromáždení Hospodinova.
\par 9 Když bys vytáhl vojensky proti neprátelum svým, vystríhej se od všeliké zlé veci.
\par 10 Bude-li mezi vámi kdo, ješto by poškvrnen byl príhodou nocní, vyjde ven z stanu, a nevejde do nich;
\par 11 Ale když bude k vecerou, umyje se vodou, a po západu slunce vejde do stanu.
\par 12 Také místo budeš míti vne za stany, abys tam chodíval ven;
\par 13 A budeš míti kolík mezi jinými nástroji svými, a když bys chtel sednouti vne, vykopáš jím dulek, a obráte se, zahrabeš necistotu svou.
\par 14 Nebo Hospodin Buh tvuj chodí u prostred stanu tvých, aby te vysvobodil, a dal tobe neprátely tvé; protož at jest príbytek tvuj svatý, tak aby nespatril pri tobe mrzkosti nejaké, procež by se odvrátil od tebe.
\par 15 Nevydáš služebníka pánu jeho, kterýž k tobe utekl od pána svého.
\par 16 S tebou bydliti bude u prostred tebe na míste, kteréž by vyvolil v nekterém meste tvém, kdežkoli jemu se líbiti bude; nebudeš ho mocí utiskati.
\par 17 Nebude nevestka žádná z dcer Izraelských, ani necistý smilník z synu Izraelských.
\par 18 Neprineseš mzdy nevestky, a mzdy psa do domu Hospodina Boha svého z jakéhokoli slibu, nebo to obé ohavnost jest Hospodinu Bohu tvému.
\par 19 Nedáš na lichvu bratru svému ani penez, ani pokrmu, ani jakékoli veci, kteráž se dává na lichvu.
\par 20 Cizímu pujcíš na lichvu, ale bratru svému nedáš na lichvu, aby požehnal tobe Hospodin Buh tvuj pri všech vecech, k kterýmž bys vztáhl ruku svou v zemi, do níž vejdeš, abys dedicne obdržel ji.
\par 21 Když bys ucinil slib Hospodinu Bohu svému, neprodlévej splniti ho; nebo konecne toho vyhledávati bude Hospodin Buh tvuj od tebe, a byl by na tobe hrích.
\par 22 Pakli nebudeš slibovati, nebude na tobe hríchu.
\par 23 Což jednou vyšlo z úst tvých, to splníš, a uciníš, jakž jsi slíbil Hospodinu Bohu svému dobrovolne, což jsi vynesl ústy svými.
\par 24 Všel-li bys do vinice bližního svého, jísti budeš hrozny podlé žádosti své do sytosti své, ale do nádoby své nevložíš.
\par 25 Všel-li bys do obilí bližního svého, natrháš sobe klasu rukou svou, ale srpem nebudeš žíti obilí bližního svého.

\chapter{24}

\par 1 Pojal-li by muž ženu a byl by manželem jejím, prihodilo by se pak, že by nenašla milosti pred ocima jeho pro nejakou mrzkost, kterouž by nalezl na ní, i napsal by jí lístek zapuzení a dal v ruku její, a vyhnal by ji z domu svého;
\par 2 A vyjduci z domu jeho, odešla by a vdala se za druhého muže;
\par 3 A ten také muž poslední v nenávisti maje ji, napsal by lístek zapuzení a dal v ruce její, a vyhnal by ji z domu svého; aneb umrel by muž její poslední, kterýž vzal ji sobe za manželku:
\par 4 Nebude moci manžel její první, kterýž ji vyhnal, zase ji vzíti sobe za manželku, když již prícinou jeho poškvrnena jest; nebo ohavnost jest pred Hospodinem. Protož nedopouštej hrešiti lidu zeme, kterouž Hospodin Buh tvuj dává tobe v dedictví.
\par 5 Když by nekdo v nove pojal ženu, nevyjde k boji, aniž na nej vzkládána bude jaká obecní práce; svoboden bude v dome svém za jeden rok, a veseliti se bude s manželkou svou, kterouž pojal.
\par 6 Žádný nevezme v zástave svrchního i spodního žernovu, nebo takový bral by duši v základu.
\par 7 Byl-li by postižen nekdo, že ukradl cloveka z bratrí svých synu Izraelských, a k zisku by sobe jej privedl aneb prodal jej: umre zlodej ten, a odejmeš zlé z prostredku svého.
\par 8 Šetr se pri ráne malomocenství, abys ostríhal pilne a cinil všecko, jakž uciti budou vás kneží Levítové; jakož prikázal jsem jim, ostríhati toho budete a tak ciniti.
\par 9 Pomni na ty veci, které ucinil Hospodin Buh tvuj Marii na ceste, když jste vyšli z Egypta.
\par 10 Pujcil-li bys bližnímu svému neceho, nevejdeš do domu jeho, abys vzal neco v zástave od neho.
\par 11 Ale vne staneš, a clovek, jemuž jsi pujcil, vynese tobe základ svuj ven.
\par 12 Jestliže by pak byl clovek chudý, nebudeš spáti s základem jeho.
\par 13 Bez prodlévání navrátíš jemu zastavenou vec jeho pri západu slunce, aby leže v šatech svých, dobrorecil tobe, a bude to za spravedlnost tobe pred Hospodinem Bohem tvým.
\par 14 Neutiskneš nájemníka chudého a nuzného, tak z bratrí svých jako z príchozích, kteríž jsou v zemi tvé v branách tvých.
\par 15 Na každý den dáš jemu mzdu jeho, prvé nežli by slunce zapadlo; nebo chudý jest, a tím se živí, aby neúpel proti tobe k Hospodinu, a byl by na tobe hrích.
\par 16 Nebudou na hrdle trestáni otcové za syny, ani synové trestáni budou na hrdle za otce, jeden každý za svuj hrích umre.
\par 17 Neprevrátíš soudu príchozímu neb sirotku, ani vezmeš v základu roucha vdovy,
\par 18 Ale pamatuj, že jsi byl služebníkem v Egypte, a že te vykoupil Hospodin Buh tvuj odtud; protož prikazujit, abys cinil toto.
\par 19 Když bys žal obilí své na poli svém, a zapomenul bys tam nekterého snopu, nenavrátíš se, abys jej vzal; príchozímu, sirotku a vdove to bude, aby požehnal tobe Hospodin Buh tvuj pri všelikém díle rukou tvých.
\par 20 Když bys trásl olivy své, nebudeš shledávati po každé ratolesti za sebou; príchozímu, sirotku a vdove to zustane.
\par 21 Když bys sbíral víno na vinici své, nebudeš paberovati jahodek za sebou; príchozímu, sirotku a vdove to bude.
\par 22 Pamatuj, že jsi byl služebníkem v zemi Egyptské; protož prikazujit, abys to cinil.

\chapter{25}

\par 1 Vznikla-li by jaká nesnáz mezi nekterými, a pristoupili by k soudu, aby je rozsoudili, tedy spravedlivého ospravedlní, a nepravého odsoudí.
\par 2 Bude-li pak hoden mrskání nepravý, tedy káže ho položiti soudce a mrskati pred sebou, vedlé nepravosti jeho v jistý pocet ran.
\par 3 Ctyridcetikrát káže ho mrštiti, aniž pridá více, aby, jestliže by jej nadto mrskal ranami mnohými, nebyl príliš zlehcen bratr tvuj pred ocima tvýma.
\par 4 Nezavížeš úst vola mlátícího.
\par 5 Když by bratrí spolu bydlili, a umrel by jeden z nich, nemaje syna, nevdá se ven žena toho mrtvého za jiného muže; bratr jeho vejde k ní, a vezme ji sobe za manželku, a právem švagrovství prižení se k ní.
\par 6 Prvorozený pak, kteréhož by porodila, nazván bude jménem bratra jeho mrtvého, aby nebylo vyhlazeno jméno jeho z Izraele.
\par 7 Nechtel-li by pak muž ten pojíti príbuzné své, tedy prijde príbuzná jeho k bráne pred starší a rekne: Nechce príbuzný muj vzbuditi bratru svému jména v Izraeli, a nechce podlé práva švagrovství pojíti mne.
\par 8 Tedy povolají ho starší mesta toho, a mluviti budou s ním; a stoje, rekl-li by: Nechci jí pojíti,
\par 9 Pristoupí príbuzná jeho k nemu pred staršími, a szuje strevíc jeho s nohy jeho, a pline mu na tvár a odpoví, rkuci: Tak se má státi muži tomu, kterýž by nechtel vzdelati domu bratra svého.
\par 10 I bude nazváno jméno jeho v Izraeli: Dum bosého.
\par 11 Když by se svadili nekterí spolu jeden s druhým, a pristoupila by žena jednoho, aby vysvobodila muže svého z ruky bijícího jej, i vztáhla by ruku svou a uchopila by ho za luno:
\par 12 Tedy utneš ruku její, neslituje se nad ní oko tvé.
\par 13 Nebudeš míti v pytlíku svém nejednostejného kamene, vetšího a menšího.
\par 14 Aniž budeš míti v dome svém nejednostejného korce, vetšího a menšího.
\par 15 Váhu celou a spravedlivou míti budeš, též míru celou a spravedlivou budeš míti, aby se prodlili dnové tvoji v zemi, kterouž Hospodin Buh tvuj dává tobe.
\par 16 Nebo v ohavnosti jest Hospodinu Bohu tvému, kdožkoli ciní ty veci, všeliký cinící nepravost.
\par 17 Pamatuj na to, cot ucinil Amalech na ceste, když jste šli z Egypta:
\par 18 Kterak vyšed tobe v cestu, zadní houf tvuj všech mdlých, kteríž šli za tebou, pobil, když jsi ty byl zemdlený a ustalý, a nebál se Boha.
\par 19 Protož když byl Hospodin Buh tvuj dal tobe odpocinutí ode všech neprátel tvých vukol v zemi, kterouž Hospodin Buh tvuj dává tobe v dedictví, abys dedicne obdržel ji, vyhladíš památku Amalecha pod nebem; nezapomínejž na to.

\chapter{26}

\par 1 Když pak vejdeš do zeme, kterouž Hospodin Buh tvuj dává tobe v dedictví, a opanuje ji, bydliti v ní budeš:
\par 2 Vezmeš prvotin všeho ovoce zeme té, kteréž obetovati budeš z zeme své, již Hospodin Buh tvuj dává tobe, a vlože do koše, pujdeš k místu, kteréž by Hospodin Buh tvuj k prebývání tam jménu svému vyvolil.
\par 3 A prijda k knezi, kterýž tech dnu bude, díš jemu: Vyznávám dnes Hospodinu Bohu svému, že jsem všel do zeme, kterouž s prísahou zaslíbil Hospodin otcum našim, že ji nám dá.
\par 4 I vezma knez z ruky tvé koš, postaví jej pred oltárem Hospodina Boha tvého.
\par 5 A mluviti budeš pred Hospodinem Bohem svým, rka: Syrský chudý otec muj sstoupil do Egypta s nemnohými osobami, a byv tam pohostinu, vzrostl v národ veliký, silný a mnohý.
\par 6 A když zle nakládali s námi Egyptští, trápíce nás, a vzkládajíce na nás službu težkou,
\par 7 Volali jsme k Hospodinu Bohu otcu našich, a vyslyšev Hospodin hlas náš, popatril na trápení naše, práci naši a ssoužení naše.
\par 8 I vyvedl nás Hospodin z Egypta v ruce silné a v rameni vztaženém, v strachu velikém, a znameních i zázracích.
\par 9 A uvedl nás na toto místo, a dal nám zemi tuto oplývající mlékem a strdí.
\par 10 Protož nyní, hle, prinesl jsem prvotiny úrod zeme, kterouž jsi mi dal, ó Hospodine. I necháš toho pred Hospodinem Bohem svým, a pokloníš se pred ním.
\par 11 I veseliti se budeš ve všech dobrých vecech, kteréž by tobe dal Hospodin Buh tvuj, i domu tvému, ty i Levíta i príchozí, kterýž jest u prostred tebe.
\par 12 Když bys pak vyplnil všecky desátky ze všech úrod svých léta tretího, jenž rok desátku jest, a dal bys Levítovi, príchozímu, sirotku i vdove, a jedli by v branách tvých a nasyceni byli:
\par 13 Tedy díš pred Hospodinem Bohem svým: Vynesl jsem, což posveceného bylo, z domu svého, a dal jsem také Levítovi, príchozímu, sirotku a vdove vedlé všelikého prikázaní tvého, kteréž jsi mi prikázal; neprestoupil jsem žádného z prikázaní tvých, aniž jsem zapomenul na ne.
\par 14 Nejedl jsem v zámutku svém z toho, a neujal jsem z toho k veci obecné, aniž jsem dal neco odtud ku pohrbu; poslechl jsem hlasu Hospodina Boha svého, ucinil jsem podlé všeho, což jsi mi prikázal.
\par 15 Popatriž z príbytku svatého svého s nebe, a požehnej lidu svému Izraelskému a zemi, kterouž jsi dal nám, jakož jsi s prísahou zaslíbil otcum našim, zemi oplývající mlékem a strdí.
\par 16 Dnes Hospodin Buh tvuj prikazuje tobe, abys ostríhal ustanovení techto a soudu; ostríhejž tedy a cin je z celého srdce svého a ze vší duše své.
\par 17 Dnes i ty pripovedels se k Hospodinu, že jej budeš míti za Boha, a choditi budeš po cestách jeho, a ostríhati ustanovení jeho, a prikázaní i soudu jeho, a poslouchati hlasu jeho.
\par 18 Hospodin také pripovedel se k tobe dnes, že te bude míti za lid zvláštní, jakož mluvil tobe, abys ostríhal všech príkázaní jeho,
\par 19 A že te vyvýší nade všecky národy, kteréž ucinil, abys byl vzácnejší, slovoutnejší a slavnejší nad ne, a tak lid svatý Hospodinu Bohu svému, jakož jest mluvil.

\chapter{27}

\par 1 I prikázal Mojžíš a starší Izraelští lidu, rkouce: Ostríhejž každého prikázaní, kteréž já prikazuji vám dnes.
\par 2 A když prejdeš pres Jordán do zeme, kterouž Hospodin Buh tvuj dává tobe, vyzdvihneš sobe kameny veliké, a obvržeš je vápnem.
\par 3 A napíšeš na nich všecka slova zákona tohoto, když prejdeš, abys všel do zeme, kterouž Hospodin Buh tvuj dává tobe, do zeme oplývající mlékem a strdí, jakož jest mluvil Hospodin Buh otcu tvých tobe.
\par 4 Když tedy prejdeš Jordán a vyzdvihneš ty kameny, kteréž já prikazuji vám dnes, na hore Hébal, a obvržeš je vápnem,
\par 5 A vzdeláš tam oltár Hospodinu Bohu svému: oltár z kamenu, jichž nebudeš tesati železem,
\par 6 Z kamení celého vzdeláš oltár Hospodinu Bohu svému, abys na nem obetoval obeti zápalné Hospodinu Bohu svému.
\par 7 Obetovati budeš i obeti pokojné, a jísti tu a veseliti se pred Hospodinem Bohem svým.
\par 8 Napíšeš pak na tech kameních všecka slova zákona toho dobre a zretelne.
\par 9 I mluvil Mojžíš a kneží Levítští ke všemu Izraelovi, rkouce: Pozoruj a slyš, Izraeli, dnes ucinen jsi lidem Hospodina Boha svého.
\par 10 Protož poslouchej hlasu Hospodina Boha svého, a zachovávej prikázaní jeho i ustanovení jeho, kteráž já tobe dnes prikazuji.
\par 11 I prikázal Mojžíš v ten den lidu, rka:
\par 12 Tito stanou, aby dobrorecili lidu na hore Garizim, když byste prešli Jordán: Simeon, Léví, Juda, Izachar, Jozef a Beniamin.
\par 13 Tito pak stanou, aby zlorecili na hore Hébal: Ruben, Gád, Asser, Zabulon, Dan a Neftalím.
\par 14 I budou osvedcovati Levítové, a reknou ke všechnem mužum Izraelským vysokým hlasem:
\par 15 Zlorecený clovek, kterýž by udelal rytinu aneb vec slitou, ohavnost Hospodinu, dílo rukou remeslníka, by ji pak i do skrýše odložil. I odpoví všecken lid a rekne: Amen.
\par 16 Zlorecený, kdož sobe zlehcuje otce svého a matku svou; i rekne všecken lid: Amen.
\par 17 Zlorecený, kdož prenáší mezník bližního svého; i rekne všecken lid: Amen.
\par 18 Zlorecený, kdož zavodí slepého, aby bloudil po ceste; i rekne všecken lid: Amen.
\par 19 Zlorecený, kdož prevrací spravedlnost príchozího, sirotka a vdovy; a odpoví všecken lid: Amen.
\par 20 Zlorecený, kdož by obýval s manželkou otce svého, nebo odkryl podolek otce svého; i rekne všecken lid: Amen.
\par 21 Zlorecený, kdož by obýval s kterýmkoli hovadem; i dí všecken lid: Amen.
\par 22 Zlorecený, kdož by obýval s sestrou svou, dcerou otce svého, aneb dcerou matky své; i rekne všecken lid: Amen.
\par 23 Zlorecený, kdož by obýval s svegruší svou; i odpoví všecken lid: Amen.
\par 24 Zlorecený, kdož by zbil bližního svého tajne; i rekne všecken lid: Amen.
\par 25 Zlorecený, kdož by vzal dary, aby zabil cloveka nevinného; i dí všecken lid: Amen.
\par 26 Zlorecený, kdož by nezustal v recech zákona tohoto a necinil jich; a rekne všecken lid: Amen.

\chapter{28}

\par 1 Jestliže pak opravdove poslušen budeš hlasu Hospodina Boha svého, ostríhaje a cine všecka prikázaní jeho, kteráž já dnes prikazuji tobe, vyvýší te Hospodin Buh tvuj nade všecky národy zeme.
\par 2 A prijdou na te všecka požehnání tato, a vyplní se pri tobe, když jen poslušen budeš hlasu Hospodina Boha svého.
\par 3 Požehnaný budeš v meste, požehnaný i na poli.
\par 4 Požehnaný plod života tvého, úrody zeme tvé, i plod dobytka tvého, prvorozené skotu tvých i stáda bravu tvých.
\par 5 Požehnaný koš tvuj i díže tvá.
\par 6 Požehnaný budeš vcházeje, požehnaný i vycházeje.
\par 7 A uciní Hospodin, že neprátelé tvoji, kteríž by povstali proti tobe, poraženi budou pred tebou; jednou cestou vytáhnou proti tobe, a sedmi cestami pred tebou utíkati budou.
\par 8 Prikáže Hospodin požehnání svému, aby s tebou bylo v špižírnách tvých a pri všem, k cemu bys koli pricinil ruku svou, a požehná tobe v zemi, kterouž Hospodin Buh tvuj dává tobe.
\par 9 Vystaví te sobe Hospodin za lid svatý, jakož zaprisáhl tobe, když ostríhati budeš prikázaní Hospodina Boha svého, a choditi po cestách jeho.
\par 10 I uzrí všickni národové zeme, že jméno Hospodinovo vzýváno jest nad tebou, a budou se báti tebe.
\par 11 Uciní také Hospodin, že hojnost míti budeš všeho dobrého, plodu života svého, i plodu dobytku svých, i úrod zemských v zemi, kterouž s prísahou zaslíbil otcum tvým, že ji tobe dá.
\par 12 Otevre Hospodin tobe poklad svuj výborný, nebe, aby vydalo déšt zemi tvé casem svým, a požehná všelikému dílu ruky tvé, tak že mnohým národum pujcovati budeš, sám pak nic nevypujcíš.
\par 13 I ustanoví te Hospodin za hlavu a ne za ocas, a budeš vždycky vyšší, a nikdy nižší, když poslouchati budeš prikázaní Hospodina Boha svého, kteráž já dnes tobe prikazuji, abys ostríhal a cinil je.
\par 14 A neuchýlíš se od žádného slova, kteráž já dnes prikazuji tobe, ani na pravo ani na levo, odcházeje po bozích cizích, abys jim sloužil.
\par 15 Jestliže pak hlasu Hospodina Boha svého poslouchati, a všech prikázaní a ustanovení jeho, kteráž já dnes prikazuji tobe, ostríhati a ciniti nebudeš, prijdou na te všecka zlorecenství tato a postihnou te.
\par 16 Zlorecený budeš v meste, zlorecený i na poli.
\par 17 Zlorecený koš tvuj, a zlorecená díže tvá.
\par 18 Zlorecený plod života tvého i úrody zeme tvé, prvorozené skotu tvých i stáda bravu tvých.
\par 19 Zlorecený budeš vcházeje, zlorecený i vycházeje.
\par 20 Pošle Hospodin na te zlorecení, zkormoucení a bídu pri všem, k cemuž bys koli pricinil ruky své a což bys koli delal, dokudž nebudeš vyhlazen, a nezahyneš v náhle pro zlé skutky tvé, skrze než jsi opustil mne.
\par 21 Dopustí Hospodin, aby se prídržely tebe morní bolesti, až te i vypléní z zeme, do níž se béreš, abys ji dedicne opanoval.
\par 22 Bíti te bude Hospodin souchotinami, zimnicí, pálivostí, horkem, mecem, suchem a rudou, a budou te stíhati, až te i zkazí.
\par 23 I to nebe, kteréž jest nad hlavou tvou, bude medené, a zeme, kteráž jest pod tebou, železná.
\par 24 Dá Hospodin zemi tvé místo dešte prach a popel, a tot s nebe sstoupí na te, dokudž bys nebyl vyhlazen.
\par 25 Uciní i to Hospodin, že poražen budeš od neprátel svých; jednou cestou vytáhneš proti nim, a sedmi cestami utíkati budeš od tvári jejich, a musíš se smýkati po všech královstvích zeme.
\par 26 I budou tela vaše mrtvá za pokrm všemu ptactvu nebeskému, a šelmám zemským, a nebude, kdo by je odehnal.
\par 27 Raní te Hospodin vredem Egyptským, neduhy na zadku, prašivinami a svrabem nezhojitelným.
\par 28 Raní te Hospodin pominutím smyslu, slepotou a tupostí srdce,
\par 29 Tak že o poledni makati budeš, jako maká slepý ve tme, a nebudeš míti prospechu na cestách svých; k tomu také utiskán budeš, a loupen po všecky dny, a nebude, kdo by te vysvobodil.
\par 30 Manželku sobe zasnoubíš, a jiný s ní obývati bude; dum vystavíš, a nebudeš bydliti v nem; vinici štípíš, a sbírati na ní nebudeš.
\par 31 Vul tvuj pred tvýma ocima zabit bude, a ty jeho jísti nebudeš; osel tvuj uchvácen bude pred tvárí tvou, anižt se zase navrátí; dobytek tvuj vydán bude neprátelum tvým, a žádný ho nevysvobodí.
\par 32 Synové tvoji a dcery tvé cizímu národu vydáni budou, a oci tvé na to hledíce, umdlívati budou pro ne celého dne, a nebude síly v ruce tvé.
\par 33 Úrody zeme tvé i všecko úsilí tvé sžíre národ, kteréhož ty neznáš, a nebudeš než potlacený a potrený po všecky dny.
\par 34 A omámený budeš nad temi vecmi, kteréž videti budou oci tvé.
\par 35 Raní te Hospodin vredem nejhorším na kolenou i na lýtkách, tak že se nebudeš moci zhojiti, od spodku nohy tvé až do vrchu hlavy.
\par 36 Zavede te Hospodin i krále tvého, kteréhož ustanovíš nad sebou, do národu, kteréhož jsi ty neznal, ani predkové tvoji, a sloužiti tam budeš bohum cizím, drevu a kameni.
\par 37 A budeš k užasnutí a prísloví i v rozprávku všechnem národum, mezi kteréž zavede te Hospodin.
\par 38 Mnoho semene vyneseš na pole k rozsívání, a málo shromáždíš, nebo sžerou to kobylky.
\par 39 Vinice štípíš a delati je budeš, ale vína píti ani sbírati nebudeš, nebo cerv sžíre je.
\par 40 Olivoví hojnost míti budeš ve všech koncinách svých, a však olejem se pomazovati nebudeš, nebo sprchne ovoce s olivy tvé.
\par 41 Synu a dcer naplodíš, a nebudeš jich míti, nebo zajati budou.
\par 42 Všecko stromoví tvé i úrody zeme tvé kobylky zkazí.
\par 43 Cizozemec, kterýž s tebou prebývá, vzroste nad tebe, ty pak velice ponižovati se musíš.
\par 44 On pujcovati bude tobe, a ty nebudeš míti, co bys pujcil jemu; on bude prednejší, a ty poslednejší.
\par 45 A prijdou na tebe všecka zlorecenství tato a stíhati te budou, a obklící te, až i zahyneš, jestliže bys neuposlechl hlasu Hospodina Boha svého, a neostríhal prikázaní a ustanovení jeho, kteráž prikázal tobe.
\par 46 A budou rány tyto znamením a zázrakem na tobe i semeni tvém až na veky,
\par 47 Proto že jsi nesloužil Hospodinu Bohu svému s potešením a veselím srdce, maje hojnost všech vecí.
\par 48 A protož nepríteli svému, kteréhož poslal na tebe Hospodin, sloužiti musíš v hladu, žízni, v nahote a v nedostatku všech vecí; a vloží na šíji tvou jho železné, dokudž te nesetre.
\par 49 Privede Hospodin na tebe národ z daleka, od nejdalších koncin zeme, jako letí orlice, národ, jehož jazyku nerozumíš,
\par 50 Národ nestydatý, kterýž ani starce nebude šanovati, a nad dítetem se neslituje.
\par 51 A sžíre plod dobytku tvých i úrody zeme tvé, dokudž nebudeš vyhlazen; a nezanechá tobe obilí, vína mladého a oleje, prvorozeného z skotu tvých, ani stáda bravu tvých, až te i vyhladí.
\par 52 A oblehne te ve všech mestech tvých, dokudž by nepadly zdi tvé vysoké a pevné, v nichž ty doufáš po vší zemi své; obležen, pravím, budeš ve všech mestech svých, po vší zemi své, kterouž Hospodin Buh tvuj dal tobe,
\par 53 Tak že v obležení a ssoužení, jímž ssouží te neprítel tvuj, jísti budeš plod života svého, maso synu svých a dcer svých, kteréž by dal tobe Hospodin Buh tvuj.
\par 54 Clovek mezi vámi rozmazaný a v rozkoši schovaný závideti bude bratru svému, i vlastní žene své, i ostatním synum svým, kterýchž ješte zanechal,
\par 55 Tak že neudelí žádnému z nich masa synu svých, kteréž jísti bude, proto že nezustalo jemu nic jiného v obležení a v ssoužení, jímž ssouží te neprítel tvuj ve všech mestech tvých.
\par 56 Rozmazaná mezi vámi a v rozkoši schovaná žena, kteráž rozmazaností a rozkoší velikou ledva nohou zeme se dotkla, vlastnímu muži svému a synu svému i dceri své,
\par 57 Také i lužka svého, kteréž z ní vychází pri porodu, ano i synu svých, kteréž zplodí, závideti bude; nebo jísti je bude tajne pro nedostatek všech vecí v obležení a ssoužení, jímž ssouží te neprítel tvuj v mestech tvých.
\par 58 Nebudeš-li ostríhati a ciniti všech slov zákona tohoto, kteráž psána jsou v knize této, abys se bál toho veleslavného a hrozného jména Hospodina Boha svého:
\par 59 Rozmnoží ku podivení Hospodin rány tvé, a rány semene tvého, rány veliké a trvánlivé, i nemoci težké a dlouhé.
\par 60 A obrátí na tebe všecky neduhy Egyptské, jichžs se strašil, a prichytí se tebe.
\par 61 Všelijaký také neduh a všelikou ránu, kteráž není psána v knize zákona tohoto, uvede Hospodin na tebe, dokudž nebudeš vyhlazen.
\par 62 A zustane vás malicko, ješto vás prvé bylo mnoho, jako hvezd nebeských, proto že jsi neposlouchal hlasu Hospodina Boha svého.
\par 63 I stane se, že jakož se veselil Hospodin nad vámi, dobre vám cine a rozmnožuje vás, tak veseliti se bude Hospodin nad vámi, když vás zkazí a vyhladí, a vypléneni budete z zeme, do kteréž jdete, abyste dedicne vládli jí.
\par 64 A rozptýlí te Hospodin mezi všecky národy, od jednoho konce zeme až do druhého, a budeš tam sloužiti bohum cizím, kterýchž ty neznáš, ani otcové tvoji, drevu a kameni.
\par 65 A mezi národy temi neoddechneš, aniž bude míti odpocinutí spodek nohy tvé; tam také dá Hospodin tobe srdce lekavé, a oci blíkavé, a truchlost mysli.
\par 66 I bude život tvuj nejistý pred tebou, a strašiti se budeš v noci i ve dne, a nikdež nebudeš jist svým životem.
\par 67 Ráno díš: Ó by již byl vecer! a vecer díš: Ó by již bylo jitro! pro strach srdce svého, jímž se lekáš, a pro ty veci, na než ocima svýma hledeti musíš.
\par 68 A zavede te Hospodin do Egypta na lodech, cestou, o níž jsem rekl tobe: Nebudeš jí videti více; a tam prodávati se budete neprátelum svým za služebníky a devky, a nebude, kdo by koupil.

\chapter{29}

\par 1 Tato jsou slova smlouvy, kteráž prikázal Hospodin Mojžíšovi uciniti s syny Izraelskými v zemi Moábské, mimo onu smlouvu, kterouž ucinil s nimi na Orébe.
\par 2 I svolav Mojžíš všecken lid Izraelský, rekl jim: Vy sami videli jste všecky veci, kteréž ucinil Hospodin pred ocima vašima v zemi Egyptské, Faraonovi, i všechnem služebníkum jeho, i vší zemi jeho,
\par 3 Zkušování veliká, kteráž videly oci tvé, znamení i zázraky ty veliké.
\par 4 A však nedal vám Hospodin srdce k srozumení, a ocí k videní, a uší k slyšení až do tohoto dne.
\par 5 A vedl jsem vás ctyridceti let po poušti, nezvetšela roucha vaše na vás, a obuv vaše neztrhala se na nohách vašich.
\par 6 Chleba jste nejedli, vína a nápoje opojného jste nepili, abyste poznali, že já jsem Hospodin Buh váš.
\par 7 A když jste prišli na toto místo, vytáhl Seon, král Ezebon, a Og, král Bázan, proti nám k boji, a porazili jsme je,
\par 8 A vzali jsme zemi jejich, i dali jsme ji v dedictví pokolení Rubenovu a Gádovu, a polovici pokolení Manassesova.
\par 9 Ostríhejtež tedy slov smlouvy této a cinte je, aby se vám štastne vedlo všecko, což byste cinili.
\par 10 Vy všickni dnes stojíte pred Hospodinem Bohem svým, knížata vaše v pokoleních vašich, starší vaši a úredníci vaši, všickni muži Izraelští,
\par 11 Dítky vaše i ženy vaše, i príchozí vaši, kteríž bydlejí u prostred stanu vašich, i ten, kterýž dríví seká, i ten, kterýž váží vodu,
\par 12 Abyste vešli v smlouvu Hospodina Boha svého, a v prísahu jeho, v kteroužto smlouvu Hospodin Buh tvuj dnes vchází s tebou,
\par 13 Aby te sobe dnes postavil za lid, a on byl tobe za Boha, jakož mluvil tobe, a jakož s prísahou zaslíbil otcum tvým, Abrahamovi, Izákovi a Jákobovi.
\par 14 A ne s vámi samými ciním smlouvu tuto a prísahu tuto,
\par 15 Ale i s každým tím, jenž tuto dnes s námi stojí pred Hospodinem Bohem naším, i s tím, jehož není tuto dnes s námi.
\par 16 Nebo vy víte, kterak jsme bydlili v zemi Egyptské, a kterak jsme šli prostredkem tech národu, až jsme naskrze prošli.
\par 17 A videli jste ohavnosti jejich i modly jejich, drevo i kámen, stríbro i zlato, kteréž jest pri nich.
\par 18 Hledtež, at nebývá mezi vámi muže, aneb ženy, aneb celedi, aneb pokolení, jehož by srdce odvrátilo se dnes od Hospodina Boha našeho, a šel by sloužiti bohum tech národu; at nebývá mezi vámi korene plodícího jed a horkost.
\par 19 I stalo by se, že uslyše takový slova zlorecení tohoto, dobrorecil by sobe v srdci svém, rka: Pokoj míti budu, bycht pak i chodil podlé žádosti srdce svého; i pridal by opilou žíznivé.
\par 20 Nechcet Hospodin odpustiti takovému, nebo tehdáž rozpálí se prchlivost Hospodinova a zurivost jeho proti takovému cloveku, tak že pripadne na nej všeliké zlorecení, o kterémž psáno jest v knize této; i vyhladí Hospodin jméno jeho pod nebem.
\par 21 A odloucí jej Hospodin s jeho zlým ode všech pokolení Izraelských, vedlé všech zlorecení smlouvy zapsané v této knize zákona,
\par 22 Tak že rekne vek potomní, synové vaši, kteríž povstanou po vás, i cizozemec, kterýž prijde z zeme daleké, (když uzrí rány zeme této, i neduhy její, kteréž uvedl na ni Hospodin,
\par 23 Sirou a solí vypálenou všecku tu zemi, a že se nemuž síti, ani co vzcházeti, ani jaké byliny rusti na ní, rovne jako na míste, kdež jest podvrácena Sodoma a Gomora, Adama a Seboim, kteréž podvrátil Hospodin v hneve svém a v prchlivosti své),
\par 24 Reknou všickni národové: Proc jest tak ucinil Hospodin zemi této? Kteraký jest to hnev prchlivosti jeho náramné?
\par 25 A bude odpovedíno: Proto že opustili smlouvu Hospodina Boha otcu svých, kterouž ucinil s nimi, když je vyvedl z zeme Egyptské.
\par 26 Nebo odcházejíce, sloužili bohum cizím a klaneli se jim, bohum, kterýchž neznali, kteríž se s nimi také nicímž dobrým nezdelili.
\par 27 I rozhnevala se prchlivost Hospodinova na tu zemi, tak že uvedl na ni všecko zlorecení zapsané v knize této.
\par 28 Protož vyplénil je Hospodin z zeme jejich v hneve, v rozpálení a v prchlivosti veliké, a vyvrhl je do zeme jiné, jakž to ukazuje dnešní den.
\par 29 Veci skryté jsou Hospodina Boha našeho, veci pak zjevené ty jsou naše a synu našich, abychom plnili všecka slova zákona tohoto.

\chapter{30}

\par 1 Když pak prijdou na te všecka slova tato, požehnání i zlorecenství, kterážt jsem predložil, a rozpomeneš se v srdci svém, kdež bys koli byl mezi národy, do kterýchž by te rozehnal Hospodin Buh tvuj,
\par 2 A obráte se k Hospodinu Bohu svému, poslouchal bys hlasu jeho ve všem, jakž já prikazuji tobe dnes, ty i synové tvoji, z celého srdce svého a ze vší duše své:
\par 3 Tehdy privede zase Hospodin Buh tvuj zajaté tvé a smiluje se nad tebou, a obráte se, shromáždí te ze všech národu, mezi než rozptýlil te Hospodin Buh tvuj.
\par 4 Byt pak nekdo z tvých i na konec sveta zahnán byl, odtud zase shromáždí te Hospodin Buh tvuj, a odtud pujme te,
\par 5 A uvede te zase Hospodin Buh tvuj do zeme, kterouž dedicne vládli otcové tvoji, a opanuješ ji, a dobret uciní a rozmnoží te více nežli otce tvé.
\par 6 A obreže Hospodin Buh tvuj srdce tvé a srdce semene tvého, abys miloval Hospodina Boha svého z celého srdce svého, a ze vší duše své, abys živ byl.
\par 7 Všecka pak zlorecenství tato obrátí Hospodin Buh tvuj na neprátely tvé a na ty, jenž v nenávisti meli tebe, a protivili se tobe.
\par 8 Ty tedy obráte se, když poslouchati budeš hlasu Hospodina Boha svého, a ciniti všecka prikázaní jeho, kteráž já tobe dnes prikazuji:
\par 9 Dát Hospodin Buh tvuj prospech pri všem díle rukou tvých, v plodu života tvého i v plodu dobytka tvého, a v úrodách zeme tvé, tobe k dobrému. Nebo zase veseliti se bude Hospodin z tebe, dobre cine tobe, jakož veselil se z otcu tvých,
\par 10 Jestliže bys však poslouchal hlasu Hospodina Boha svého, a ostríhal prikázaní jeho a ustanovení jeho, napsaných v knize zákona tohoto, když bys se obrátil k Hospodinu Bohu svému celým srdcem svým a celou duší svou.
\par 11 Nebo prikázaní toto, kteréž prikazuji tobe dnes, není skryté pred tebou, ani vzdálené od tebe.
\par 12 Není na nebi, abys rekl: Kdo nám vstoupí do nebe, aby vezma, prinesl a oznámil je nám, abychom je plnili?
\par 13 Ani za morem jest, abys rekl: Kdo se nám preplaví za more, aby je prinesl a oznámil nám, abychom plnili je?
\par 14 Ale velmi blízko tebe jest slovo, v ústech tvých a v srdci tvém, abys cinil to:
\par 15 (Hle, predložil jsem tobe dnes život a dobré, smrt i zlé),
\par 16 Což já prikazuji tobe dnes, abys miloval Hospodina Boha svého, chode po cestách jeho a ostríhaje prikázaní jeho, ustanovení a soudu jeho, abys živ jsa, rozmnožen byl, a požehnal tobe Hospodin Buh tvuj v zemi, do kteréž jdeš, abys ji dedicne obdržel.
\par 17 Pakli se odvrátí srdce tvé, a nebudeš poslouchati, ale priveden jsa k tomu, klaneti se budeš bohum cizím a jim sloužiti:
\par 18 Ohlašuji vám dnes, že jistotne zahynete, aniž prodlíte dnu svých v zemi, do kteréž se pres Jordán béreš, abys ji dedicne obdržel.
\par 19 Osvedcuji proti tobe dnes nebem a zemí, žet jsem život i smrt predložil, požehnání i zlorecenství; vyvoliž sobe tedy život, abys živ byl ty i síme tvé,
\par 20 A miloval Hospodina Boha svého, poslouchaje hlasu jeho a prídrže se jeho, (nebo on jest život tvuj, a dlouhost dnu tvých), abys bydlil v zemi, kterouž s prísahou zaslíbil Hospodin otcum tvým Abrahamovi, Izákovi a Jákobovi, že jim ji dá.

\chapter{31}

\par 1 Prišed tedy Mojžíš, mluvil slova ta ke všemu Izraelovi,
\par 2 A rekl jim: Ve stu a ve dvadcíti letech jsem dnes, nemohut již více vycházeti a vcházeti; a také rekl Hospodin ke mne: Neprejdeš Jordánu tohoto.
\par 3 Hospodin Buh tvuj predcházející tebe vyhladí ty národy od tvári tvé, a ty dedicne je opanuješ, a Jozue tento pujde pred tebou,jakož rekl Hospodin.
\par 4 I uciní Hospodin jim, jakož ucinil Seonovi a Ogovi králum Amorejským a zemi jejich, kteréž vyhladil.
\par 5 Protož když je dá Hospodin vám, tedy uciníte jim vedlé každého prikázaní, kteréž jsem prikázal vám.
\par 6 Budte silní a zmužile se mejte, nebojte se, ani se lekejte tvári jejich, nebo Hospodin Buh tvuj, on jde s tebou, neopustít a nezanechá tebe.
\par 7 Tedy povolal Mojžíš Jozue a rekl jemu pred ocima všeho Izraele: Budiž silný a zmužile se mej, nebo ty vejdeš s lidem tímto do zeme, kterouž s prísahou zaslíbil Hospodin otcum jejich, že ji dá jim, a ty ji rozdelíš jim v dedictví.
\par 8 Hospodin pak predcházející te, ont bude s tebou, neopustí te, aniž te zanechá; neboj se, ani se strachuj.
\par 9 I napsal Mojžíš zákon tento a dal jej knežím, synum Léví, kteríž nosili truhlu smlouvy Hospodinovy, a všechnem starším Izraelským,
\par 10 A prikázal jim Mojžíš, rka: Každého léta sedmého, v jistý cas léta odpuštení, v svátek stanu,
\par 11 Když prijde všecken Izrael a postaví se pred Hospodinem Bohem tvým na míste, kteréž by vyvolil, císti budeš zákon tento prede vším Izraelem v uších jejich,
\par 12 Shromážde lid, muže i ženy, i deti, i príchozí své, kteríž jsou v branách tvých, aby slyšíce, ucili se a báli se Hospodina Boha vašeho, a hledeli ciniti všecka slova zákona tohoto.
\par 13 Synové také jejich, kteríž ješte toho nepoznali, at slyší a ucí se báti Hospodina Boha vašeho po všecky dny, v nichž živi budete na zemi, do níž, prejdouce Jordán, vejdete, abyste ji dedicne obdrželi.
\par 14 I rekl Hospodin Mojžíšovi: Aj, již se priblížili dnové tvoji, abys umrel. Povolej Jozue, a postavte se oba v stánku úmluvy, a prikáži jemu. Šel tedy Mojžíš a Jozue, a postavili se v stánku úmluvy.
\par 15 I ukázal se Hospodin v stánku v sloupe oblakovém, a postavil se sloup oblakový nade dvermi stánku.
\par 16 Rekl pak Hospodin Mojžíšovi: Aj, ty již usneš s otci svými, potom lid tento vstana, smilniti bude, následuje bohu cizozemcu zeme této, do níž vchází, aby bydlil u prostred ní; tu mne opustí a zruší smlouvu mou, kterouž jsem ucinil s ním.
\par 17 Protož rozhnevá se na nej prchlivost má v ten den, a opustím je, i skryji tvár svou od nich, a bude sežrán, i prijdou na nej mnohé veci zlé a úzkosti. I rekne v ten den: Zdaliž ne proto prišly na mne tyto zlé veci, že Buh muj není u prostred mne?
\par 18 Já pak daleko skryji tvár svou v ten den pro všecko to zlé, kteréž páchal, jakž se obrátil k bohum cizím.
\par 19 Protož nyní napište sobe písen tuto, a uc jí syny Izraelské; vlož ji v ústa jejich, aby ta písen byla mi za svedka proti synum Izraelským.
\par 20 Nebo uvedu jej do zeme, kterouž jsem s prísahou zaslíbil otcum jeho, oplývající mlékem a strdí, kdežto bude jísti a nasytí se a ztucní. Tehdy obrátí se k bohum cizím a sloužiti bude jim, i popudí mne a zruší smlouvu mou.
\par 21 A když prijdou na nej mnohé zlé veci a úzkosti, tehdy bude jemu tato písen na svedectví, (nebo neprijde v zapomenutí, aniž odejde od úst semene jeho). Známt zajisté myšlení jeho, a co on ješte dnes ciniti bude, prvé nežli jej uvedu do zeme, kterouž jsem s prísahou zaslíbil.
\par 22 Tedy napsal Mojžíš tu písen v ten den a ucil jí syny Izraelské.
\par 23 Potom dal prikázaní Jozue, synu Nun, rka: Budiž silný, a mej se zmužile, nebo ty uvedeš syny Izraelské do zeme, kterouž jsem jim s prísahou zaslíbil, a já budu s tebou.
\par 24 Stalo se pak, když napsal Mojžíš slova zákona tohoto v knize, a dokonal je,
\par 25 Že prikázal Levítum, kteríž nosili truhlu smlouvy Hospodinovy, rka:
\par 26 Vezmete tu knihu zákona a vložte ji v strane truhly smlouvy Hospodina Boha vašeho, at jest tam proti vám na svedectví.
\par 27 Nebo já znám zpouru tvou a zatvrdilost šíje tvé. Aj, ponevadž, když jsem já ješte živ nyní s vámi, zpurní jste bývali Hospodinu, cím tedy více, když já umru?
\par 28 Shromaždte ke mne všecky starší pokolení svých a správce vaše, abych mluvil v uši jejich slova tato, a osvedcil proti nim nebem i zemí.
\par 29 Nebo vím, že po mé smrti velmi porušíte se a vystoupíte z cesty, kterouž jsem prikázal vám; procež prijde na vás toto zlé v posledních dnech, když byste cinili to, což se nelíbí Hospodinu, popouzejíce ho dílem rukou svých.
\par 30 Mluvil tedy Mojžíš v uši všeho shromáždení Izraelského slova písne této, až je i dokonal.

\chapter{32}

\par 1 Pozorujte nebesa a mluviti budu, poslyš i zeme výmluvností úst mých.
\par 2 Sstupiž jako déšt naucení mé, spadniž jako rosa výmluvnost má, jako tichý déšt na mladistvou trávu, a jako príval na odrostlou bylinu,
\par 3 Nebo jméno Hospodinovo slaviti budu. Vzdejtež velebnost Bohu našemu,
\par 4 Skále té, jejíž skutkové jsou dokonalí, nebo všecky cesty jeho jsou spravedlivé. Buh silný, pravdomluvný, a není nepravosti v nem, spravedlivý a prímý jest.
\par 5 Ale pokolení prevrácené a zavilé zproneverilo se jemu mrzkostí vlastní svou,jakáž vzdálena jest od synu jeho.
\par 6 Tím-liž jste se odplacovati meli Hospodinu, lide bláznivý a nemoudrý? Zdaliž on není otec tvuj, kterýž te sobe dobyl? On ucinil tebe a utvrdil te.
\par 7 Rozpomen se na dny staré, považte let každého veku; vzeptej se otce svého, a oznámí tobe, starcu svých, a povedí tobe.
\par 8 Když dedictví rozdeloval Nejvyšší národum, když rozsadil syny Adamovy, rozmeril meze národum vedlé poctu synu Izraelských.
\par 9 Nebo díl Hospodinuv jest lid jeho, Jákob provazec dedictví jeho.
\par 10 Nalezl jej v zemi pusté, a na poušti veliké a hrozné; vukol vedl jej, vyucil jej, a ostríhal ho, jako zrítedlnice oka svého.
\par 11 Jako orlice ponouká orlicátek svých, sedí na mladých svých, roztahuje krídla svá, bére je, a nosí je na krídlách svých:
\par 12 Tak Hospodin sám vedl jej, a nebylo s ním boha cizozemcu.
\par 13 Zprovodil jej na vysoká místa zeme, aby jedl úrody polní, a ucinil, aby ssál med z skály, a olej z kamene pretvrdého,
\par 14 Máslo od krav a mléko od ovcí, s nejtucnejšími beránky a skopci z Bázan a kozly s jádrem zrn pšenicných, a cervené víno výborné aby pil.
\par 15 Takž Izrael ztucnev, zpícil se; vytyl jsi a ztlustl, tukem jsi obrostl; i opustil Boha stvoritele svého, a zlehcil sobe Boha spasení svého.
\par 16 K horlení popudili ho cizími bohy, ohavnostmi zdráždili jej.
\par 17 Obetovali dáblum, ne Bohu, bohum, jichž neznali, novým, kteríž z blízka prišli, jichžto se nic nestrašili otcové vaši.
\par 18 Na skálu, kteráž zplodila te, zapomenul jsi; zapomnel jsi na Boha silného, stvoritele svého.
\par 19 To když videl Hospodin, popudil se hnevem proti synum a dcerám svým,
\par 20 A rekl: Skryji pred nimi tvár svou, podívám se posledním vecem jejich; nebo národ prevrácený jest, synové, v nichž není žádné víry.
\par 21 Onit jsou mne popudili k horlení skrze to, což není Buh silný, rozhnevali mne svými marnostmi. I ját popudím jich k závisti skrze ty, kteríž nejsou lid, skrze národ bláznivý k hnevu jich popudím.
\par 22 Nebo ohen zápalen jest v prchlivosti mé, a horeti bude až do nejhlubšího pekla, a sžíre zemi i úrody její, a zapálí základy hor.
\par 23 Shromáždím na ne zlé veci, strely své vystrílím na ne.
\par 24 Hladem usvadnou, a neduhy pálcivými a nakažením morním prehorkým žráni budou; také zuby šelm pošli na ne s jedem hadu zemských.
\par 25 Mec uvede sirobu vne, a v pokojích bude strach, tak na mládenci jako panne, díteti prsí požívajícím i muži šedivém.
\par 26 Ano, rekl bych:Rozptýlím je po koutech, rozkáži prestati mezi lidmi pameti jejich,
\par 27 Bych se pýchy neprítele neobával, aby spatríce to neprátelé jejich, nerekli: Ruka naše nepremožená byla, neucinilt jest Hospodin niceho z tech vecí.
\par 28 Nebo národ ten nesmyslný jest a nemající rozumnosti.
\par 29 Ó by moudrí byli, rozumelit by tomu, prohlédali by na poslední veci své.
\par 30 Jak by jich jeden honiti mohl tisíc, a dva pryc zahnati deset tisícu? Jediné že Buh skála jejich prodal je, a Hospodin vydal je.
\par 31 Nebo Buh skála naše není jako skála jejich, což neprátelé naši sami souditi mohou.
\par 32 Nebo z kmene Sodomského kmen jejich, a z réví Gomorských hroznové jejich, hroznové jedovatí, zrní horkosti plné.
\par 33 Jed draku víno jejich, a jed lítý nejjedovatejšího hada.
\par 34 Zdaliž to není schováno u mne? zapeceteno v pokladnicích mých?
\par 35 Mát jest pomsta a odplata, casemt svým klesne noha jejich; nebo blízko jest den zahynutí jejich, a budoucí veci zlé rychle pripadnou na ne.
\par 36 Souditi zajisté bude Hospodin lid svuj, a nad služebníky svými lítost míti bude, když uzrí, že odešla síla, a že jakož zajatý tak i zanechaný s nic býti nemuže.
\par 37 I dí: Kde jsou bohové jejich? skála, v níž nadeji meli?
\par 38 Z jejichžto obetí tuk jídali, a víno z obetí jejich mokrých píjeli? Nechat vstanou a spomohou vám, necht vám skrejší jest skála ta.
\par 39 Pohledtež již aspon, že já jsem, já jsem sám, a že není Boha krome mne. Já mohu usmrtiti i obživiti, já raniti i uzdraviti, a není žádného, kdo by vytrhl z ruky mé.
\par 40 Nebo já pozdvihám k nebi ruky své, a pravím: Živ jsem já na veky.
\par 41 Jakž nabrousím ostrí mece svého, a uchopí soud ruka má, uciním pomstu nad neprátely svými, a tem, jenž v nenávisti mne meli, odplatím.
\par 42 Opojím strely své krví, a mec muj sžere maso, a to krví ranených a zajatých, jakž jen zacnu pomsty uvoditi na neprátely.
\par 43 Veselte se pohané s lidem jeho, nebot pomstí krve služebníku svých, a uvede pomstu na neprátely své, a ocistí zemi svou a lid svuj.
\par 44 I prišel Mojžíš, a mluvil všecka slova písne této v uši lidu, on a Jozue, syn Nun.
\par 45 A když dokonal Mojžíš mluvení všech slov tech ke všemu množství Izraelskému,
\par 46 Rekl jim: Priložtež srdce svá ke všechnem slovum, kteráž já dnes osvedcuji vám,a prikažte to synum svým, aby ostríhali všech slov zákona tohoto, a cinili je.
\par 47 Nebo není daremné slovo, abyste jím pohrdnouti meli, ale jest život váš; a v slovu tom prodlíte dnu svých na zemi, kterouž abyste dedicne obdrželi, pujdete pres Jordán.
\par 48 Téhož dne mluvil Hospodin k Mojžíšovi, rka:
\par 49 Vstup na horu tuto Abarim, na vrch Nébo, kteráž jest v zemi Moábské naproti Jerichu, a spatr zemi Kananejskou, kterouž já dávám synum Izraelským právem dedicným.
\par 50 A umreš na vrchu, na kterýž vejdeš, a pripojen budeš k lidu svému, jako umrel Aron, bratr tvuj, na hore recené Hor, a pripojen jest k lidu svému.
\par 51 Nebo jste zhrešili proti mne u prostred synu Izraelských, pri vodách odpírání v Kádes, na poušti Tsin, proto že jste neposvetili mne u prostred synu Izraelských.
\par 52 Pred sebou zajisté uzríš zemi tu, ale tam nevejdeš do zeme té, kterouž dávám synum Izraelským.

\chapter{33}

\par 1 Toto pak jest požehnání, kterýmž požehnal Mojžíš, muž Boží, synu Izraelských pred svou smrtí,
\par 2 A rekl: Hospodin z Sinai prišel, a vzešel jim z Seir, zastkvel se s hory Fáran, a prišel s desíti tisíci svatých, z jehožto pravice ohen zákona svítil jim.
\par 3 Jak velice miluješ lidi! Všickni svatí jeho jsou v ruce tvé, oni také privinuli se k nohám tvým, vezmout prospech z výmluvností tvých.
\par 4 Zákon vydal nám Mojžíš, dedicný shromáždení Jákobovu,
\par 5 (Nebo byl v Izraeli jako král), když prední z lidu shromáždili se, i všecka pokolení Izraelská.
\par 6 Bud živ Ruben a neumírej, a muži jeho at jsou bez poctu.
\par 7 Tolikéž požehnal i Judovi a rekl: Vyslyš, Hospodine, hlas Juduv, a k lidu svému jej sprovozuj; ruce jeho budou bojovati za nej, a ty jemu spomáhati budeš proti neprátelum jeho.
\par 8 O Léví také rekl: Thumim tvé a urim tvé bylo, Pane, pri muži svatém tvém, kteréhož jsi zkusil v pokušení, a kterýž podlé tebe mel nesnáz pri vodách Meribah,
\par 9 Kterýž rekl otci svému a matce své: Neohlédám se na vás; a bratrí svých neznal, a o synech svých nevedel; nebo ostríhají výmluvností tvých, a smlouvu tvou zachovávají.
\par 10 Vyucovati budou soudum tvým Jákoba, a zákonu tvému Izraele, a klásti budou kadidlo pred tvárí tvou, a obet zápalnou na oltári tvém.
\par 11 Požehnejž, Hospodine, ryterování jeho, a v práci rukou jeho zalib se tobe. Zlomuj ledví neprátel jeho a tech, kteríž ho nenávidí, aby nepovstali.
\par 12 A o Beniaminovi rekl: Milý jest Hospodinu, bezpecne s ním bydliti bude; ochranovati jej bude každého dne, a mezi rameny jeho prebývati.
\par 13 O Jozefovi pak rekl: Požehnaná zeme jeho od Hospodina pro nejlepší veci nebeské, pro rosu a vrchovište zespod se prýštící,
\par 14 A pro nejlepší úrody sluncem vyzralé, a pro nejlepší veci casem mesícu došlé,
\par 15 I pro rozkoše hor nejstarších, a pro rozkoše pahrbku vecných,
\par 16 Pro nejlepší veci zeme i plnost její, vyplývající z milosti ve kri prebývajícího. Prijdiž to požehnání na hlavu Jozefovu, a na vrch hlavy oddeleného z jiných bratrí jeho.
\par 17 Prvorozeného, vola toho krása veliká bude, a rohové jednorožce rohové jeho, jimiž on trkati bude národy naporád až do koncin zeme. A tot jsou mnozí tisícové Efraimovi a tisícové Manassesovi.
\par 18 O Zabulonovi také rekl: Vesel se Zabulon u vycházení svém, a Izachar v staních svých.
\par 19 Lidi na horu Boží svolají, a budou obetovati obeti spravedlnosti; nebo hojnost morskou ssáti budou, a skryté poklady v písku.
\par 20 Gádovi pak rekl: Požehnaný, kterýž rozširuje Gáda. Ont jakožto lev bydliti bude, a uchvátí rameno i s hlavou.
\par 21 Kterýž opatril ho prvotinami; nebo tam podílem skrze vydavatele zákona ubezpecen jest. Protož pobéret se s knížaty lidu, a spravedlnost Hospodinovu i soudy jeho s Izraelem vykonávati bude.
\par 22 O Danovi pak rekl: Dan jako lvíce lvové vyskakovati bude z Bázan.
\par 23 A Neftalímovi rekl: Ó Neftalíme, sytý prízní Páne, plný požehnání Hospodinova, západní a polední stranu prijmi za dedictví.
\par 24 O Asserovi také rekl: Asser požehnaný nad jiné syny, budet milý bratrím svým, omocí v oleji nohu svou.
\par 25 Železo a med pod šlepejemi tvými; pokudž trvati budou dnové tvoji, slovoutný budeš.
\par 26 Nenít žádného, jako Buh silný, ó Izraeli, kterýž se vznáší na nebesích ku pomoci tobe, a u velebnosti své na oblacích nejvyšších.
\par 27 Ochrana tvá bud Buh vecný, a zespod ramena vecnosti, kterýž vyžene neprátely pred tebou, aneb rekne: Vyhlad je,
\par 28 Aby sám bezpecne bydlil Izrael, rodina Jákobova, a to v zemi obilím a vínem oplývající, jehož nebesa také i rosu vydávati budou.
\par 29 Blahoslavený jsi, Izraeli. Kdo jest podobný tobe, lide vysvobozený skrze Hospodina, jenž jest pavéza spomožení tvého a mec dustojnosti tvé? Tvoji zajisté neprátelé poníženi budou, ale ty po všech vyvýšenostech jejich šlapati budeš.

\chapter{34}

\par 1 Tedy vstoupil Mojžíš z rovin Moábských na horu Nébo, na vrch hory, kteráž jest proti Jerichu, a ukázal jemu Hospodin všecku zemi Galád až do Dan,
\par 2 I všecku Neftalím, a zemi Efraim a Manasse, a všecku zemi Juda až k mori nejdalšímu,
\par 3 Polední také stranu a roviny údolí Jericha, mesta palmovím osazeného, až do Segor.
\par 4 A rekl jemu Hospodin: Tato jest zeme, kterouž s prísahou zaslíbil jsem Abrahamovi, Izákovi a Jákobovi, rka: Semeni tvému dám ji. Zpusobil jsem to, abys ji videl ocima svýma, však do ní nevejdeš.
\par 5 I umrel tam Mojžíš, služebník Hospodinuv, v zemi Moábské, vedlé reci Hospodinovy.
\par 6 A pochoval jej v Gai, v zemi Moábské naproti Betfegor, a žádný nezvedel o jeho hrobu až do tohoto dne.
\par 7 Byl pak Mojžíš ve stu a dvadcíti letech, když umrel, a nepošly oci jeho, aniž síla odešla od neho.
\par 8 I plakali synové Izraelští Mojžíše na rovinách Moábských tridceti dní, a vyplneni jsou dnové pláce a kvílení nad Mojžíšem.
\par 9 Jozue pak, syn Nun, naplnen jest duchem moudrosti; nebo byl vložil Mojžíš ruce své na nej. I poslouchali ho synové Izraelští, a cinili, jakož prikázal Hospodin skrze Mojžíše.
\par 10 Ale nepovstal více prorok v Izraeli podobný Mojžíšovi, (kteréhož by tak znal Hospodin tvárí v tvár),
\par 11 Ve všech znameních a zázracích, pro než poslal jej Hospodin, aby cinil je v zemi Egyptské, pred Faraonem a prede všechnemi služebníky jeho, a vší zemí jeho,
\par 12 Také i ve všech skutcích ruky silné, a ve všeliké hruzi veliké, kteréžto veci cinil Mojžíš pred ocima všeho Izraele.


\end{document}