\begin{document}

\title{1 John}

\chapter{1}

\par 1 Což bylo od pocátku, což jsme slýchali, co jsme ocima svýma videli, a co jsme pilne spatrili, a ceho se ruce naše dotýkaly, o slovu života,
\par 2 (Nebo ten život zjeven jest, a my jsme videli jej, a svedcíme, i zvestujeme vám ten život vecný, kterýž byl u Otce, a zjeven jest nám,)
\par 3 Což jsme videli a slyšeli, to vám zvestujeme, abyste i vy s námi obecenství meli, a obecenství naše aby bylo s Otcem i s Synem jeho Jezukristem.
\par 4 A totot píšeme vám, aby radost vaše byla plná.
\par 5 Tot jest tedy zvestování to, kteréž jsme slýchali od neho, a zvestujeme vám: Že Buh jest svetlo, a tmy v nem nižádné není.
\par 6 Díme-li, že s ním obecenství máme, a ve tme chodíme, lžeme a neciníme pravdy.
\par 7 Paklit chodíme v svetle, jako on jest v svetle, obecenství máme vespolek, a krev Ježíše Krista Syna jeho ocištuje nás od všelikého hríchu.
\par 8 Paklit díme, že hríchu nemáme, sami se svodíme, a pravdy v nás není.
\par 9 Jestliže pak budeme vyznávati hríchy své, vernýt jest Buh a spravedlivý, aby nám odpustil hríchy, a ocistil nás od všeliké nepravosti.
\par 10 Díme-li, že jsme nehrešili, ciníme jej lhárem, a nenít v nás slova jeho.

\chapter{2}

\par 1 Synáckové moji, totot vám píši, abyste nehrešili. Paklit by kdo zhrešil, prímluvci máme u Otce, Ježíše Krista spravedlivého.
\par 2 A ont jest obet slitování za hríchy naše, a netoliko za naše, ale i za hríchy všeho sveta.
\par 3 A po tomt známe, že jsme jej poznali, jestliže prikázání jeho ostríháme.
\par 4 Dí-lit kdo: Znám jej, a prikázání jeho neostríhá, lhárt jest, a pravdy v nem není.
\par 5 Ale kdožt ostríhá slova jeho, v pravdet láska Boží v tom jest dokonalá. Po tomt známe, že jsme v nem.
\par 6 Kdo praví, že v nem zustává, mát, jakž on chodil, i tento tak choditi.
\par 7 Bratrí, ne nejaké nové prikázání vám píši, ale prikázání staré, kteréž jste meli od pocátku. A to prikázání staré jestit slovo to, kteréž jste slyšeli od pocátku.
\par 8 A zase prikázání nové píši vám, kteréžto jest pravé i v nem i v vás. Nebo tma pomíjí, a svetlo to pravé již svítí.
\par 9 Kdo praví, že jest v svetle, a bratra svého nenávidí, v temnostit jest až posavad.
\par 10 Kdož miluje bratra svého, v svetle zustává, a pohoršení v nem není.
\par 11 Ale kdož nenávidí bratra svého, v temnosti jest, a v temnosti chodí, a neví, kam jde; nebo temnost oslepila oci jeho.
\par 12 Píši vám, synáckové, žet jsou vám odpušteni hríchové pro jméno jeho.
\par 13 Píši vám, otcové, že jste poznali toho, kterýž jest od pocátku. Píši vám, mládenci, že jste zvítezili nad zlým.
\par 14 Píši vám, dítky, že jste poznali Otce. Psal jsem vám, otcové, že jste poznali toho, kterýž jest od pocátku. Psal jsem vám, mládenci, že silní jste, a slovo Boží v vás zustává, a že jste zvítezili nad zlým.
\par 15 Nemilujtež sveta, ani tech vecí, kteréž na svete jsou. Miluje-lit kdo svet, není lásky Otcovy v nem.
\par 16 Nebo všecko, což jest na svete, jako žádost tela, a žádost ocí, a pýcha života, tot není z Otce, ale jest z sveta.
\par 17 A svet hyne i žádost jeho, ale kdož ciní vuli Boží, ten trvá na veky.
\par 18 Dítky, poslední hodina jest, a jakož jste slýchaly, že antikrist prijíti má, i nynít antikristové jsou mnozí. Odkudž známe, že poslední hodina jest.
\par 19 Z nást jsou vyšli, ale nebyli z nás. Nebo byt byli z nás, bylit by zustali s námi, ale vyšli z nás, aby zjeveni byli, že nejsou všickni z nás.
\par 20 Vy pak máte pomazání od Svatého, a znáte všecko.
\par 21 Nepsal jsem vám proto, že byste neznali pravdy, ale že ji znáte, a že všeliká lež není z pravdy.
\par 22 A kdo jest lhár, nežli ten, kdož zapírá, že Ježíš není Kristus? Tent jest antikrist; kdot zapírá Otce, zapírát i Syna.
\par 23 Každý kdož zapírá Syna, nemát ani Otce.
\par 24 A protož zustaniž v vás to, což jste slýchali od pocátku. Zustane-lit v vás to, co jste slýchali od pocátku, i vy také v Synu i v Otci zustanete.
\par 25 A tot jest to zaslíbení, kteréž nám zaslíbil, totiž ten život vecný.
\par 26 Tyto veci psal jsem vám o tech, kteríž vás svodí.
\par 27 Ale pomazání to, kteréž jste vzali od neho, v vás zustává, a aniž potrebujete, aby kdo ucil vás, ale jakž pomazání to ucí vás o všech vecech, a pravét jest a neoklamavatelné, a jakž naucilo vás, tak v nem zustávejte.
\par 28 A nyní, synáckové, zustávejtež v nem, abychom, když by se ukázal, smelé doufání meli, a nebyli zahanbeni od neho v cas príchodu jeho.
\par 29 Ponevadž víte, že on spravedlivý jest, znejtež také, že každý, kdož ciní spravedlnost, z neho jest narozen.

\chapter{3}

\par 1 Pohledte, jakou lásku dal nám Otec, totiž abychom synové Boží slouli. Protot svet nezná nás, že jeho nezná.
\par 2 Nejmilejší, nyní synové Boží jsme, ale ještet se neokázalo, co budeme. Vímet pak, že když se okáže, podobni jemu budeme; nebo videti jej budeme tak, jakž jest.
\par 3 A každý, kdož má takovou nadeji v nem, ocištuje se, jakož i on cistý jest.
\par 4 Každý, kdož ciní hrích, cinít proti zákonu; nebo hrích jest prestoupení zákona.
\par 5 A víte, že on se okázal proto, aby hríchy naše snal, a hríchu v nem není.
\par 6 Každý tedy, kdož v nem zustává, nehreší; ale každý, kdož hreší, nevidel ho, aniž ho poznal.
\par 7 Synáckové, nižádný vás nesvod. Kdož ciní spravedlnost, spravedlivý jest, jakož i on jest spravedlivý.
\par 8 Kdož ciní hrích, z dábla jest; nebo dábel hned od pocátku hreší. Na tot jest zjeven Syn Boží, aby kazil skutky dáblovy.
\par 9 Každý, kdož se narodil z Boha, hríchu neciní; nebo síme jeho v nem zustává, aniž muže hrešiti, nebo z Boha narozen jest.
\par 10 Po tomtot zjevní jsou synové Boží a synové dáblovi. Každý, kdož neciní spravedlnosti, nenít z Boha, a kdož nemiluje bratra svého.
\par 11 Nebo tot jest to zvestování, kteréž jste slýchali od pocátku, abychom milovali jedni druhé.
\par 12 Ne jako Kain, kterýž z toho zlostníka byl a zamordoval bratra svého. A pro kterou prícinu zamordoval ho? Protože skutkové jeho byli zlí, bratra pak jeho byli spravedliví.
\par 13 Nedivtež se, bratrí moji, jestliže vás svet nenávidí.
\par 14 My víme, že jsme preneseni z smrti do života, nebo milujeme bratrí. Kdož nemiluje bratra, zustávát v smrti.
\par 15 Každý, kdož nenávidí bratra svého, vražedlník jest, a víte, že žádný vražedlník nemá života vecného v sobe zustávajícího.
\par 16 Po tomto jsme poznali lásku, že on duši svou za nás položil, i myt tedy máme za bratrí duše své klásti.
\par 17 Kdo by pak mel statek tohoto sveta, a videl by bratra svého, an nouzi trpí, a zavrel by srdce své pred ním, kterak láska Boží zustává v nem?
\par 18 Synáckové moji, nemilujmež slovem, ani jazykem toliko, ale skutkem a pravdou.
\par 19 A po tomt poznáváme, že z pravdy jsme, a pred oblicejem jeho spokojíme srdce svá.
\par 20 Nebo obvinovalo-lit by nás srdce naše, ovšemt Buh, kterýž jest vetší nežli srdce naše a zná všecko.
\par 21 Nejmilejší, jestližet by nás srdce naše neobvinovalo, smelou doufanlivost máme k Bohu.
\par 22 A zacež ho koli prosíme, béreme od neho; nebo prikázání jeho ostríháme, a to, což jest libého pred oblicejem jeho, ciníme.
\par 23 A totot jest to prikázání jeho, abychom verili jménu Syna jeho Jezukrista a milovali jedni druhé, jakož nám dal prikázání.
\par 24 Nebo kdož ostríhá prikázání jeho, v nemt zustává, a on také v nem. A po tomt poznáváme, že zustává v nás, totiž po Duchu, kteréhož dal nám.

\chapter{4}

\par 1 Nejmilejší, ne každému duchu verte, ale zkušujte duchu, jsou-li z Boha; nebo mnozí falešní proroci vyšli na svet.
\par 2 Po tomto znejte Ducha Božího: Všeliký duch, kterýž vyznává Jezukrista v tele prišlého, z Boha jest.
\par 3 Ale všeliký duch, kterýž nevyznává Jezukrista v tele prišlého, není z Boha; nýbrž tot jest ten duch antikristuv, o kterémž jste slýchali, že prijíti má, a jižt jest nyní na svete.
\par 4 Vy pak z Boha jste, synáckové, a zvítezili jste nad nimi; nebo vetšít jest ten, kterýž jest v vás, nežli ten, kterýž jest v svete.
\par 5 Oni z sveta jsou, a protož o svetu mluví, a svet jich poslouchá.
\par 6 My z Boha jsme. Kdo zná Boha, poslouchát nás; kdož pak není z Boha, neposlouchát nás. A po tomt poznáváme ducha pravdy a ducha bludu.
\par 7 Nejmilejší, milujmež jedni druhé; nebo láska z Boha jest, a každý, kdož miluje, z Boha se narodil, a znát Boha.
\par 8 Kdož nemiluje, nezná Boha; nebo Buh láska jest.
\par 9 V tomt zjevena jest láska Boží k nám, že Syna svého toho jednorozeného poslal Buh na svet, abychom živi byli skrze neho.
\par 10 V tomt jest láska, ne že bychom my Boha milovali, ale že on miloval nás, a poslal Syna svého obet slitování za hríchy naše.
\par 11 Nejmilejší, ponevadž tak miloval nás Buh, i myt máme jedni druhé milovati.
\par 12 Boha žádný nikdy nespatril, ale milujeme-lit jedni druhé, Buh v nás prebývá, a láska jeho dokonalá jest v nás.
\par 13 Po tomtot poznáváme, že v nem prebýváme, a on v nás, že z Ducha svého dal nám.
\par 14 A myt jsme videli, a svedcíme, že Otec poslal Syna svého spasitele sveta.
\par 15 Kdož by koli vyznával, že Ježíš jest Syn Boží, Buh v nem prebývá, a on v Bohu.
\par 16 A myt jsme poznali, a uverili o lásce, kterouž Buh má k nám. Buh láska jest, a kdož v lásce prebývá, v Bohu prebývá, a Buh v nem.
\par 17 V tomtot jest k dokonání svému prišla láska Boží s námi, abychom bezpecné doufání meli v den soudný, kdyžto, jakýž jest on, takovíž i my jsme na tomto svete.
\par 18 Báznet není v lásce, ale láska dokonalá ven vyhání bázen; nebo bázen trápení má, kdož se pak bojí, není dokonalý v lásce.
\par 19 My milujeme jej, nebo on prve miloval nás.
\par 20 Rekl-li by kdo: Miluji Boha, a bratra svého nenávidel by, lhár jest. Nebo kdož nemiluje bratra svého, kteréhož videl, Boha, kteréhož nevidel, kterak muže milovati?
\par 21 A totot prikázání máme od neho, aby ten, kdož miluje Boha, miloval i bratra svého.

\chapter{5}

\par 1 Každý, kdož verí, že Ježíš jest Kristus, z Boha se narodil; a každý, kdož miluje toho, kterýž zplodil, milujet i toho, kterýž zplozen jest z neho.
\par 2 Po tomt poznáváme, že milujeme syny Boží, když Boha milujeme a prikázání jeho ostríháme.
\par 3 Nebo tot jest láska Boží, abychom prikázaní jeho ostríhali; a prikázání jeho nejsou težká.
\par 4 Všecko zajisté, což se narodilo z Boha, premáhá svet; a tot jest to vítezství, kteréž premáhá svet, víra naše.
\par 5 Kdo jest, ješto premáhá svet, jediné, kdož verí, že Ježíš jest Syn Boží?
\par 6 Tot jest ten, kterýž prišel vyznamenán jsa skrze vodu a krev, totiž Ježíš Kristus, ne u vode toliko, ale u vode a ve krvi. A Duch jest, jenž svedectví vydává, že Duch jest pravda.
\par 7 Nebo tri jsou, kteríž svedectví vydávají na nebi: Otec, Slovo, a Duch Svatý, a ti tri jedno jsou.
\par 8 A tri jsou, jenž svedectví vydávají na zemi: Duch, a Voda, a Krev, a ti tri jedno jsou.
\par 9 Ponevadž svedectví lidské prijímáme, svedectvít Boží vetší jest. Nebo totot svedectví jest Boží, kteréž vysvedcil o Synu svém.
\par 10 Kdož verí v Syna Božího, mát svedectví sám v sobe. Kdož pak neverí Bohu, lhárem jej ucinil; nebo neuveril tomu svedectví, kteréž vysvedcil Buh o Synu svém.
\par 11 A totot jest svedectví to, že život vecný dal nám Buh, a ten život v Synu jeho jest.
\par 12 Kdožt má Syna Božího, mát život; kdož nemá Syna Božího, života nemá.
\par 13 Tyto veci psal jsem vám verícím ve jméno Syna Božího, abyste vedeli, že máte vecný život, a abyste verili ve jméno Syna Božího.
\par 14 A totot jest to smelé doufání, kteréž máme k nemu, že zacež bychom koli prosili podle vule jeho, slyší nás.
\par 15 A kdyžt víme, že nás slyší, prosili-li bychom zac, tedyt víme, že máme nekteré prosby naplnené, kteréž jsme predkládali jemu.
\par 16 Videl-li by kdo bratra svého hrešícího hríchem ne k smrti, modliž se za nej, a dát jemu Buh život, totiž hrešícím ne k smrti. Jestit hrích k smrti; ne za ten, pravím, aby se modlil.
\par 17 Každá nepravost jestit hrích, ale jestit hrích ne k smrti.
\par 18 Víme, že každý, kdož se narodil z Boha, nehreší; ale ten, jenž narozen jest z Boha, ostríhá sebe samého, a ten zlostník se ho nedotýká.
\par 19 Víme, že z Boha jsme, ale svet všecken ve zlém leží.
\par 20 A vímet, že Syn Boží prišel, a dal nám smysl, abychom poznali Toho Pravého, a jsmet v tom Pravém, i v Synu jeho Ježíši Kristu. Ont jest ten pravý Buh a život vecný.
\par 21 Synáckové, vystríhejte se modl. Amen.


\end{document}