\begin{document}

\title{Judges}

Jdg 1:1  Sesudah Yosua meninggal, umat Israel bertanya kepada TUHAN, "TUHAN, suku yang mana dari kami harus pertama-tama pergi menyerang orang Kanaan?"
Jdg 1:2  TUHAN menjawab, "Suku Yehuda. Mereka sudah Kuberikan kekuasaan untuk merebut negeri itu."
Jdg 1:3  Lalu kata suku Yehuda kepada suku Simeon, "Bantulah kami memerangi orang Kanaan dan merebut wilayah yang ditentukan TUHAN untuk kami. Nanti kami pun akan membantu kalian merebut wilayah yang ditentukan TUHAN untuk kalian." Maka suku Simeon
Jdg 1:4  dan Yehuda pergi bertempur bersama-sama. Dan TUHAN memberikan kepada mereka kemenangan atas orang Kanaan dan orang Feris. Sepuluh ribu orang tentara musuh tewas di Bezek.
Jdg 1:5  Mereka menemukan Adoni-Bezek, raja negeri itu, dan menyerang dia.
Jdg 1:6  Ia lari, tetapi mereka mengejar dan menangkap dia lalu memotong ibu jari tangan dan kakinya.
Jdg 1:7  Kata Adoni-Bezek, "Dahulu tujuh puluh raja sudah kupotong ibu jari tangan dan kakinya, lalu kubiarkan mereka makan sisa-sisa makanan yang jatuh dari mejaku. Sekarang Allah memperlakukan aku sama seperti aku memperlakukan mereka." Adoni-Bezek kemudian diangkut ke Yerusalem, dan di sanalah ia meninggal.
Jdg 1:8  Orang Yehuda menyerang Yerusalem, lalu merebut dan membakar kota itu serta membunuh penduduknya.
Jdg 1:9  Setelah itu mereka pergi lagi menyerang orang Kanaan yang tinggal di daerah pegunungan, dan di lereng-lereng bukit; juga yang tinggal di daerah kering di sebelah selatan,
Jdg 1:10  dan di kota Hebron, yaitu kota yang dahulu dinamakan Kiryat-Arba. Di situ orang Yehuda mengalahkan kaum Sesai, Ahiman dan Talmai.
Jdg 1:11  Dari Hebron, orang Yehuda pergi menyerbu kota Debir, yang pada waktu itu bernama Kiryat-Sefer.
Jdg 1:12  Salah seorang dari suku Yehuda itu, yang bernama Kaleb, berkata, "Barangsiapa dapat merebut Kiryat-Sefer, ia boleh kawin dengan anak perempuan saya Akhsa."
Jdg 1:13  Otniel, anak dari adik Kaleb yang bernama Kenas, merebut kota itu. Maka Kaleb memberikan Akhsa, anaknya, menjadi istri Otniel.
Jdg 1:14  Pada hari perkawinan mereka, Otniel mendesak Akhsa supaya meminta sebidang tanah dari ayahnya. Maka turunlah Akhsa dari keledainya, lalu Kaleb menanyakan kepadanya apa yang diinginkannya.
Jdg 1:15  Akhsa menjawab, "Saya ingin beberapa sumber air, sebab tanah yang ayah berikan kepada saya itu berada di daerah yang kering." Maka Kaleb memberikan kepadanya sumber air di bagian hulu dan sumber air di bagian hilir.
Jdg 1:16  Keturunan ayah mertua Musa, yaitu orang-orang Keni, mengikuti orang Yehuda keluar dari Yerikho, kota kurma, dan pergi ke daerah yang gersang di sebelah selatan Arad di Yehuda. Di sana mereka menetap di tengah-tengah bangsa Amalek.
Jdg 1:17  Kemudian bersama-sama dengan suku Simeon, suku Yehuda itu pergi menyerang dan mengalahkan orang Kanaan yang tinggal di kota Zefat. Mereka mengutuki kota itu, kemudian memusnahkannya lalu menamakannya Horma.
Jdg 1:18  TUHAN menolong orang Yehuda sehingga mereka dapat merebut daerah pegunungan. Tetapi mereka tidak dapat merebut Gaza, Askelon ataupun Ekron dengan daerah-daerah di sekitarnya, karena bangsa-bangsa yang tinggal di sepanjang pesisir mempunyai kereta-kereta perang dari besi; karena itu orang Yehuda tidak dapat mengusir mereka.
Jdg 1:20  Seperti yang telah diperintahkan oleh Musa, Hebron diberikan kepada Kaleb, sebab dahulu tiga golongan kaum keturunan Enak sudah diusir Kaleb dari kota itu.
Jdg 1:21  Suku Benyamin tidak dapat mengusir orang Yebus keluar dari kota Yerusalem. Itu sebabnya orang Yebus masih tinggal di sana bersama-sama dengan orang Benyamin.
Jdg 1:22  Suku Efraim dan Manasye keturunan Yusuf, pergi menyerang kota Betel, yang ketika itu bernama Lus. Dan TUHAN menolong mereka. Pada waktu mata-mata diutus menuju ke kota itu,
Jdg 1:24  mereka bertemu dengan seorang laki-laki yang datang dari kota. Kata mereka kepadanya, "Kalau engkau mau menolong kami supaya kami dapat masuk ke dalam kota ini, kami akan memperlakukanmu dengan baik."
Jdg 1:25  Maka orang itu menunjukkan kepada mereka jalan masuk ke kota itu. Lalu orang-orang Efraim dan Manasye memasuki kota itu serta membunuh seluruh penduduknya, kecuali orang itu dengan sanak-keluarganya.
Jdg 1:26  Ia kemudian pindah ke negeri orang Het, lalu membangun sebuah kota di situ dan menamakannya Lus. Kota itu masih tetap bernama demikian.
Jdg 1:27  Suku Manasye tidak mengusir penduduk kota Bet-Sean, Taanakh, Dor, Yibleam dan Megido dengan kampung-kampung di dekatnya. Jadi orang Kanaan tetap tinggal di situ.
Jdg 1:28  Memang setelah orang Israel bertambah kuat, mereka memaksa orang Kanaan bekerja untuk mereka, tetapi mereka juga tidak mengusir orang-orang Kanaan itu sama sekali dari situ.
Jdg 1:29  Di kota Gezer, suku Efraim tidak mengusir orang Kanaan, penduduk kota itu. Jadi, orang Kanaan masih tetap tinggal disitu bersama-sama dengan mereka.
Jdg 1:30  Penduduk kota Kitron dan Nahalol tidak pula diusir oleh suku Zebulon. Itu sebabnya bangsa Kanaan masih tetap tinggal di situ bersama-sama dengan mereka, tetapi orang-orang Kanaan itu dipaksa bekerja untuk mereka.
Jdg 1:31  Penduduk kota Ako, Sidon, Ahlab, Akhzib, Helba, Afek, dan Rehob juga tidak diusir oleh suku Asyer.
Jdg 1:32  Itu sebabnya orang Kanaan, penduduk asli negeri itu masih tinggal di situ bersama-sama dengan orang Asyer.
Jdg 1:33  Penduduk kota Bet-Semes dan Bet-Anat pun tidak diusir oleh suku Naftali. Itu sebabnya orang Kanaan, penduduk asli negeri itu masih tinggal di situ bersama-sama dengan orang Naftali. Tetapi, mereka dipaksa bekerja untuk orang Naftali.
Jdg 1:34  Bangsa Amori masih tetap tinggal di Ayalon, Saalbim dan Gunung Heres, sebab suku Dan didesak oleh mereka ke daerah pegunungan, dan tidak dibiarkan turun ke daerah dataran rendah. Tetapi bangsa Amori itu selalu ditekan oleh suku Efraim dan Manasye serta dipaksa bekerja untuk kedua suku itu.
Jdg 1:36  Di sebelah utara Sela, batas daerah orang-orang Amori melewati Pendakian Akrabim.
Jdg 2:1  Malaikat TUHAN pergi dari Gilgal ke Bokhim dan berkata kepada orang-orang Israel, "Aku sudah mengeluarkan kamu dari Mesir, dan membawa kamu ke negeri yang telah Kujanjikan kepada nenek moyangmu. Sudah Kukatakan pula, 'Ikatan janji antara Aku dengan kamu tak akan Kuputuskan.
Jdg 2:2  Sebab itu kamu tak boleh membuat sesuatu ikatan dengan penduduk negeri ini. Mezbah-mezbah mereka harus kamu runtuhkan.' Tetapi kamu tidak melakukan apa yang telah Kuperintahkan itu. Malah kamu telah melakukan yang sebaliknya!
Jdg 2:3  Jadi, sekarang dengarkan! Aku tak akan mengusir orang-orang negeri ini apabila kamu bertempur dengan mereka. Mereka akan menjadi musuh-musuhmu, dan kamu akan terjerat, sehingga kamu turut menyembah dewa-dewa mereka."
Jdg 2:4  Setelah malaikat itu selesai menyampaikan berita itu, seluruh bangsa Israel menangis dengan keras.
Jdg 2:5  Itu sebabnya tempat itu dinamakan Bokhim. Lalu di tempat itu orang-orang Israel mempersembahkan kurban kepada TUHAN.
Jdg 2:6  Setelah Yosua melepas orang-orang Israel pergi, mereka pun berangkat untuk menduduki tanah yang menjadi bagian mereka masing-masing.
Jdg 2:7  Selama Yosua hidup, orang Israel mengabdi kepada TUHAN. Setelah Yosua meninggal pun mereka tetap mengabdi kepada TUHAN selama mereka dipimpin oleh orang-orang yang telah melihat sendiri segala keajaiban yang dilakukan TUHAN untuk orang Israel.
Jdg 2:8  Yosua anak Nun--hamba TUHAN itu--meninggal dunia pada usia seratus sepuluh tahun.
Jdg 2:9  Ia dimakamkan di tanah miliknya sendiri, yaitu di Timnat-Serah di daerah pegunungan Efraim sebelah utara Gunung Gaas.
Jdg 2:10  Kemudian seluruh angkatan itu meninggal juga. Dan angkatan yang berikutnya tidak mengenal TUHAN, karena mereka tidak mengalami apa yang telah dilakukan TUHAN untuk bangsa Israel.
Jdg 2:11  Kemudian orang Israel berdosa terhadap TUHAN. Mereka tidak lagi menyembah TUHAN, Allah nenek moyang mereka, yaitu Allah yang telah membawa mereka keluar dari Mesir. Mereka mulai menyembah Baal dan Asytoret serta dewa-dewa lain yang disembah oleh bangsa-bangsa di sekeliling mereka. Maka TUHAN marah kepada mereka dan membiarkan gerombolan menyerang dan merampok mereka. Mereka tidak berdaya menghadapi musuh-musuh di sekeliling mereka.
Jdg 2:15  Setiap kali mereka pergi bertempur, TUHAN melawan mereka, seperti yang sudah diperingatkan-Nya kepada mereka. Maka mereka mengalami kesulitan yang sangat besar.
Jdg 2:16  Berulang-ulang TUHAN memberikan kepada mereka pemimpin yang berjuang untuk melepaskan mereka dari gerombolan perampok.
Jdg 2:17  Tetapi mereka tidak menghiraukan para pemimpin itu. Mereka menyembah ilah-ilah lain, dan tidak setia kepada TUHAN. Mereka tidak mengikuti teladan nenek moyang mereka yang taat kepada perintah-perintah TUHAN.
Jdg 2:18  Apabila TUHAN memberikan seorang pemimpin kepada bangsa Israel, TUHAN selalu menolong pemimpin itu. Dan selama pemimpin itu masih hidup, TUHAN selalu melepaskan mereka dari musuh-musuh mereka. Ia kasihan kepada mereka apabila Ia mendengar mereka merintih karena dianiaya dan ditekan.
Jdg 2:19  Tetapi apabila pemimpin itu sudah meninggal, bangsa itu berkelakuan lebih buruk dari angkatan yang sebelumnya. Mereka mengabdi kepada ilah-ilah lain dan menyembah ilah-ilah itu. Mereka kembali kepada kebiasaan-kebiasaan mereka yang jahat dan tidak mau mengubah kelakuan mereka.
Jdg 2:20  Sebab itu TUHAN sangat marah kepada umat Israel, dan berkata, "Aku sudah memerintahkan nenek moyang bangsa ini supaya memegang perjanjian antara Aku dengan mereka. Tetapi bangsa ini memutuskan ikatan janji itu; mereka melanggar perintah-Ku.
Jdg 2:21  Sekarang Aku tidak mau lagi mengusir bangsa manapun juga yang masih ada di negeri ini ketika Yosua meninggal.
Jdg 2:22  Aku akan memakai bangsa-bangsa itu untuk menguji bangsa Israel, supaya Aku tahu apakah umat Israel mau mengikuti perintah-perintah-Ku seperti nenek moyang mereka, atau tidak."
Jdg 2:23  Itulah sebabnya TUHAN tidak memberi kemenangan kepada Yosua atas bangsa-bangsa itu. Sebaliknya TUHAN membiarkan mereka tinggal di negeri itu, dan tidak segera mengusir mereka.
Jdg 3:1  Untuk menguji orang-orang Israel yang tidak pernah mengalami peperangan di Kanaan, TUHAN membiarkan beberapa bangsa tinggal di negeri itu.
Jdg 3:2  TUHAN melakukan itu hanya dengan maksud mengajar setiap generasi Israel berperang, khususnya mereka yang tidak pernah ikut bertempur.
Jdg 3:3  Bangsa-bangsa yang masih tinggal di negeri itu ialah: bangsa yang diam di dalam kelima kota Filistin, semua bangsa Kanaan, bangsa Sidon, dan bangsa Hewi yang tinggal di pegunungan Libanon, mulai dari Gunung Baal-Hermon sampai ke jalan yang menuju Hamat.
Jdg 3:4  Mereka berada di sana untuk menjadi batu ujian bagi bangsa Israel. Dengan demikian dapatlah diketahui apakah bangsa Israel menuruti perintah-perintah yang telah diberikan TUHAN kepada nenek moyang mereka melalui Musa.
Jdg 3:5  Itulah sebabnya orang-orang Kanaan, Het, Amori, Feris, Hewi dan Yebus tetap tinggal di negeri itu bersama-sama dengan umat Israel.
Jdg 3:6  Umat Israel kawin campur dengan bangsa-bangsa itu, dan turut menyembah dewa-dewa mereka.
Jdg 3:7  Umat Israel berdosa kepada TUHAN, Allah mereka. Mereka lupa kepada-Nya sehingga mereka beribadat kepada berhala-berhala Baal dan Asyera.
Jdg 3:8  Maka TUHAN marah kepada umat Israel, dan membiarkan mereka dikalahkan oleh Raja Kusyan-Risyataim dari Mesopotamia. Delapan tahun lamanya raja itu menguasai mereka.
Jdg 3:9  Kemudian umat Israel berseru minta tolong kepada TUHAN, lalu TUHAN memberikan Otniel untuk membebaskan mereka. (Otniel adalah anak dari Kenas, adik Kaleb.)
Jdg 3:10  Roh TUHAN menguasai Otniel sehingga ia menjadi pemimpin umat Israel. Ia pergi berperang, lalu TUHAN memberikan kemenangan kepadanya atas raja Mesopotamia.
Jdg 3:11  Maka tentramlah negeri itu empat puluh tahun lamanya. Kemudian meninggallah Otniel.
Jdg 3:12  Umat Israel berdosa lagi kepada TUHAN, sehingga TUHAN membuat Eglon, raja Moab, menjadi lebih kuat dari Israel.
Jdg 3:13  Raja Eglon bergabung dengan bangsa Amon dan Amalek, lalu mengalahkan orang Israel, kemudian merebut Yerikho, kota kurma itu.
Jdg 3:14  Delapan belas tahun lamanya umat Israel dijajah oleh Eglon.
Jdg 3:15  Tetapi ketika umat Israel berseru minta tolong kepada TUHAN, Ia memberikan Ehud untuk membebaskan mereka. Ehud seorang yang kidal; ia anak Gera dari suku Benyamin, dan ialah yang biasanya mengantarkan kepada Raja Eglon upeti yang harus dibayar oleh umat Israel.
Jdg 3:16  Pada suatu hari Ehud membuat sebilah pedang bermata dua yang panjangnya hampir setengah meter. Ia menyandangkan pedang itu pada pinggangnya sebelah kanan di dalam bajunya,
Jdg 3:17  lalu pergi menyerahkan upeti dari umat Israel kepada Raja Eglon. Raja itu sangat gemuk.
Jdg 3:18  Setelah menyerahkan upeti itu, Ehud menyuruh orang-orang yang memikul uang emas itu pulang.
Jdg 3:19  Tetapi Ehud sendiri berhenti ketika sampai di batu-batu berukir dekat Gilgal, lalu kembali kepada Eglon. Pada waktu itu Eglon berada di kamar yang sejuk di tingkat atas istananya. Kata Ehud kepada Eglon, "Paduka Yang Mulia! Hamba membawa berita rahasia untuk Tuanku." Maka raja berkata, "Tunggu dulu," lalu ia memerintahkan semua hambanya supaya keluar dan meninggalkan dia dan Ehud sendirian. Kemudian, Ehud mendekati raja yang sedang duduk, lalu berkata, "Ada berita dari Allah untuk Tuanku!" Mendengar itu, raja berdiri.
Jdg 3:21  Langsung Ehud mencabut pedangnya dengan tangan kirinya dari pinggangnya sebelah kanan, lalu menikamkannya dalam-dalam ke perut raja. Seluruh mata pedang itu sampai ke gagangnya pun masuk ke dalam lemak-lemak di perut raja itu, sampai tembus ke belakang. Ehud membiarkan pedang itu di situ,
Jdg 3:23  lalu keluar serta menutup dan mengunci pintu kamar tersebut,
Jdg 3:24  kemudian pergi. Ketika para hamba raja datang dan menemukan pintu-pintu kamar itu terkunci, mereka menyangka raja masih berada di dalam dan sedang ke belakang.
Jdg 3:25  Setelah menunggu cukup lama dan raja tidak juga membuka pintu, mereka kebingungan lalu mengambil kunci dan membuka pintu itu. Ternyata raja mereka tergeletak mati di lantai.
Jdg 3:26  Sementara mereka masih menunggu-nunggu, Ehud sudah melarikan diri ke Seira lewat batu-batu berukir.
Jdg 3:27  Segera setelah tiba di daerah pegunungan Efraim, ia meniup trompet untuk memanggil orang Israel bertempur. Mereka datang lalu ia memimpin mereka turun dari daerah pegunungan itu.
Jdg 3:28  "Ikut saya," katanya kepada mereka, "TUHAN telah mengalahkan orang-orang Moab, musuhmu itu untuk kalian." Maka di bawah pimpinan Ehud, turunlah orang-orang Israel itu dari pegunungan. Mereka merebut tempat penyeberangan orang Moab di Sungai Yordan. Tidak seorang pun dari musuh itu dibiarkan menyeberang dengan selamat.
Jdg 3:29  Dalam pertempuran itu mereka membunuh kira-kira 10.000 prajurit Moab yang terbaik; tidak ada yang lolos.
Jdg 3:30  Hari itu Moab kalah terhadap Israel, maka tentramlah negeri itu delapan puluh tahun lamanya.
Jdg 3:31  Pemimpin yang berikut ialah Samgar anak Anat. Ia pun berjuang dan membebaskan orang Israel. Dengan tongkat penggiring lembu, ia membunuh 600 orang Filistin.
Jdg 4:1  Setelah Ehud meninggal, bangsa Israel berdosa lagi kepada TUHAN.
Jdg 4:2  Maka TUHAN membiarkan mereka dikalahkan oleh Yabin, raja Kanaan yang berkedudukan di kota Hazor. Panglima angkatan perangnya ialah Sisera yang tinggal di Haroset-Hagoyim.
Jdg 4:3  Yabin mempunyai 900 kereta perang dari besi. Dua puluh tahun lamanya ia menjajah orang Israel dengan kejam. Kemudian mereka berseru meminta tolong kepada TUHAN.
Jdg 4:4  Pada masa itu yang memimpin Israel adalah seorang nabi wanita bernama Debora, istri Lapidot.
Jdg 4:5  Debora biasanya duduk di bawah pohon kurma, antara Rama dan Betel, di daerah pegunungan Efraim. Dan di situlah orang-orang Israel datang untuk minta nasihat kepadanya.
Jdg 4:6  Pada suatu hari ia menyuruh orang memanggil Barak, anak Abinoam, dari kota Kedes di daerah Naftali. Lalu ia berkata kepada Barak, "Ini perintah TUHAN, Allah Israel, kepadamu: 'Kumpulkanlah 10.000 orang laki-laki dari suku Naftali dan Zebulon, dan bawalah mereka ke Gunung Tabor.
Jdg 4:7  Nanti Aku membuat Sisera, panglima angkatan perang Yabin, mendatangi kamu di Sungai Kison dengan tentaranya dan kereta-kereta perangnya untuk memerangi kamu. Tetapi Aku akan memberi kemenangan kepadamu dalam pertempuran itu.'"
Jdg 4:8  Barak berkata kepada Debora, "Kalau engkau ikut, saya mau pergi. Tetapi kalau tidak, saya juga tidak mau pergi."
Jdg 4:9  Debora menyahut, "Baik, saya akan ikut! Tetapi ingat, bukan kau yang nanti mendapat kehormatan. Melalui seorang wanitalah TUHAN akan mengalahkan Sisera." Maka berangkatlah Debora bersama-sama Barak ke Kedes.
Jdg 4:10  Dan Barak mengerahkan suku Zebulon dan Naftali ke Kedes. Yang mengikuti Barak dan Debora ketika itu ada 10.000 orang.
Jdg 4:11  Pada waktu itu Heber, seorang Keni, telah memisahkan diri dari bangsanya. (Bangsa Keni adalah keturunan Hobab, ipar Musa.) Heber sudah pergi memasang kemah-kemahnya dekat pohon-pohon terpentin di Zaanaim, tidak jauh dari Kedes.
Jdg 4:12  Segera setelah Sisera mendengar bahwa Barak sudah pergi ke Gunung Tabor,
Jdg 4:13  ia mengumpulkan kesembilan ratus kereta perangnya yang dari besi itu, bersama-sama dengan seluruh pasukannya lalu membawa mereka dari Haroset-Hagoyim ke Sungai Kison.
Jdg 4:14  Lalu kata Debora kepada Barak, "Ayo, maju! Tuhanlah yang memimpinmu. Hari ini Ia memberi kemenangan kepadamu atas Sisera." Maka turunlah Barak dari Gunung Tabor bersama-sama dengan kesepuluh ribu orangnya,
Jdg 4:15  kemudian menyerang Sisera. Dan TUHAN mengacaukan seluruh pasukan Sisera dan kereta-kereta perangnya, sehingga Sisera turun dari keretanya, lalu melarikan diri dengan berjalan kaki.
Jdg 4:16  Barak mengejar kereta-kereta perang dan pasukan musuh itu sampai ke Haroset-Hagoyim lalu seluruh pasukan Sisera itu dibunuhnya; tidak seorang pun yang dibiarkan hidup.
Jdg 4:17  Sementara itu Sisera telah lari ke kemah Yael, istri Heber, orang Keni itu. Sebab, Raja Yabin dari Hazor bersahabat dengan kaum keluarga Heber itu.
Jdg 4:18  Ketika Sisera datang, Yael menyambut dia dan berkata, "Mari Tuan, masuk saja ke dalam kemah saya; tak usah takut." Maka Sisera masuk ke dalam kemah itu lalu Yael menutupinya dengan kain gorden supaya ia tersembunyi.
Jdg 4:19  Kata Sisera kepada Yael, "Tolong berikan saya sedikit air untuk minum; saya haus." Maka Yael membuka kantong kulit, tempat menyimpan susu, lalu memberi minum kepada Sisera, kemudian menutupinya lagi dengan kain gorden.
Jdg 4:20  Lalu Sisera berkata kepada Yael, "Pergilah berdiri di muka pintu kemah. Kalau orang datang dan bertanya kepadamu apakah ada orang di sini, katakan tidak ada."
Jdg 4:21  Sisera begitu lelah sehingga ia mengantuk lalu tertidur dengan nyenyak. Yael mengambil sebuah martil dan sebatang pasak untuk kemah, lalu dengan diam-diam ia mendekati Sisera. Kemudian ia menancapkan pasak itu ke dalam pelipisnya sampai tembus ke tanah. Begitulah Yael membunuh Sisera.
Jdg 4:22  Ketika Barak datang mencari Sisera, Yael menyambut Barak dan berkata kepadanya, "Mari, saya akan menunjukkan kepadamu orang yang kaucari itu." Barak masuk bersama Yael, lalu melihat Sisera sudah mati, tergeletak di tanah dengan pasak kemah tertancap di dalam pelipisnya.
Jdg 4:23  Hari itu Tuhan memberi kemenangan kepada orang Israel atas Yabin, raja Kanaan.
Jdg 4:24  Raja itu semakin lama semakin ditekan oleh orang Israel sampai akhirnya ia binasa.
Jdg 5:1  Pada hari itu bernyanyilah Debora dan Barak anak Abinoam. Begini katanya:
Jdg 5:2  Pujilah TUHAN, sebab pahlawan-pahlawan Israel telah bertekad untuk berjuang, dan rakyat dengan rela pergi berperang.
Jdg 5:3  Dengarlah hai para raja dan pangeran, aku bernyanyi bagi Allah Israel, yaitu TUHAN.
Jdg 5:4  Tatkala Kauberangkat dari Seir, ya TUHAN, dan wilayah Edom Kautinggalkan, bumi bergetar, saluran-saluran langit pun terbuka, dan awan-awan mencurahkan airnya.
Jdg 5:5  Gunung-gunung bergoncang di depan penguasa Sinai, di depan TUHAN, Allah Israel.
Jdg 5:6  Di masa Samgar, anak Anat, dan di masa Yael juga, tiada kafilah di negeri, tiada kelana di jalan raya.
Jdg 5:7  Kota-kota Israel tiada penghuninya, sunyi sepi hingga kaudatang, Debora; kaudatang bagaikan ibu yang tercinta.
Jdg 5:8  Ketika Israel mulai menyembah dewa, pecahlah perang di negeri mereka; di antara empat puluh ribu pahlawan bangsa, tak seorang pun mengangkat senjata.
Jdg 5:9  Kepada para panglima Israel yang menawarkan diri dengan sukarela, kusampaikan penghargaanku yang sedalam-dalamnya. Pujilah TUHAN!
Jdg 5:10  Hai kamu yang mengendarai keledai putih, kamu yang duduk di atas permadani, dan kamu semua yang berjalan kaki, kabarkanlah kisah kemenangan ini!
Jdg 5:11  Di dekat sumur-sumur terdengar orang ramai berdendang, mengisahkan kemenangan TUHAN berulang-ulang, kemenangan Israel, umat TUHAN. Maka berbarislah mereka menuju ke gerbang kota.
Jdg 5:12  Bangkitlah Debora, dan angkatlah nyanyian. Bangunlah Barak, giringlah para tahanan!
Jdg 5:13  Ksatria yang tersisa datang kepada pemimpinnya, mereka datang, dan siap untuk berjuang bagi-Nya.
Jdg 5:14  Mereka berjalan dari Efraim ke lembah, mengikuti pasukan Benyamin yang berbaris di depan mereka. Dari Makhir datanglah para panglima, dan dari Zebulon bangkit para perwira.
Jdg 5:15  Para pemuka Isakhar datang bersama Debora, ya, Isakhar datang, dan Barak turut juga; mereka mengikuti dia menuruni lembah. Tetapi suku Ruben tak dapat menentukan sikap sebab di antara mereka tak terdapat kata sepakat.
Jdg 5:16  Mengapa mereka tinggal di antara domba-domba? Apakah untuk mendengarkan gembala memanggil ternaknya? Memang suku Ruben tak dapat menentukan sikap sebab di antara mereka tak terdapat kata sepakat.
Jdg 5:17  Sementara Zebulon dan Naftali berjuang sengit di medan perang, suku Dan bercokol di kapal, dan suku Gad tetap tinggal di timur Yordan, sedangkan suku Asyer tenang-tenang dan tak mau menyingkir dari tempat mereka di sepanjang pesisir.
Jdg 5:19  Di Taanakh, dekat sungai di Megido, raja-raja berperang, raja-raja Kanaan berjuang, tapi perak tak ada yang mereka bawa sebagai jarahan.
Jdg 5:20  Bintang-bintang juga turut berjuang melawan Sisera, dari tempat peredaran mereka di angkasa.
Jdg 5:21  Sungai Kison meluap-luap menghanyutkan seteru. Maju, majulah dengan perkasa, hai jiwaku!
Jdg 5:22  Dengan laju menderaplah telapak-telapak kuda yang dipacu oleh pengendaranya.
Jdg 5:23  "Kutuklah Meros, dan penghuninya," kata Malaikat TUHAN "sebab mereka tidak datang memberi bantuan sebagai pahlawan yang berjuang untuk TUHAN."
Jdg 5:24  Beruntunglah Yael, istri Heber orang Keni; dari semua penghuni kemah, dialah yang paling diberkati.
Jdg 5:25  Sisera minta air, tapi ia memberikan susu kepadanya, disajikannya susu dalam mangkuk yang pantas untuk raja-raja.
Jdg 5:26  Dengan pasak dan martil di tangannya Yael memukul Sisera sampai hancur kepalanya.
Jdg 5:27  Di kaki Yael ia rebah tak bernyawa tergeletak di tanah menemui ajalnya.
Jdg 5:28  Dari ruji-ruji jendela ibu Sisera memandang, dan berkata, "Mengapa keretanya, tak kunjung tiba, mengapa tak terdengar derap kudanya?"
Jdg 5:29  Yang terarif di antara dayang-dayangnya memberi jawaban yang telah ditemukannya sendiri dalam hatinya:
Jdg 5:30  "Mereka sedang membagi-bagi barang rampasan, untuk tiap prajurit, satu atau dua perawan. Untuk Sisera, bahan baju yang paling berharga, untuk ratu, selendang bersulam aneka warna."
Jdg 5:31  Semoga musuh-musuh-Mu, ya TUHAN, tewas seperti Sisera, tetapi semua sahabat-Mu jaya bagaikan surya di angkasa! Setelah itu, tentramlah negeri itu empat puluh tahun lamanya.
Jdg 6:1  Sekali lagi umat Israel berdosa kepada TUHAN, sehingga ia membiarkan bangsa Midian menguasai mereka tujuh tahun lamanya.
Jdg 6:2  Orang Israel begitu takut kepada orang Midian, sehingga mereka bersembunyi di dalam gua-gua dan di tempat-tempat lain yang aman di gunung-gunung.
Jdg 6:3  Tambahan pula setiap kali petani Israel bercocok tanam, datanglah orang Midian bersama-sama dengan orang Amalek dan orang-orang lainnya dari padang pasir menyerang orang Israel.
Jdg 6:4  Mereka berkemah di daerah Israel dan merusak semua tanaman di sana sampai ke daerah di sekeliling Gaza sebelah selatan. Juga ternak domba, sapi dan keledai umat Israel dibawa lari. Tak ada sesuatu pun yang mereka tinggalkan sehingga umat Israel tidak punya makanan sama sekali.
Jdg 6:5  Orang-orang itu datang berbondong-bondong dengan ternak dan kemah mereka dalam jumlah yang sangat besar. Mereka dan unta-untanya terlalu banyak untuk dapat dihitung. Mereka datang merusak negeri itu,
Jdg 6:6  dan umat Israel tak dapat berbuat apa-apa terhadap mereka.
Jdg 6:7  Kemudian umat Israel berseru meminta TUHAN menolong mereka melawan orang Midian.
Jdg 6:8  TUHAN mengutus seorang nabi kepada mereka. Nabi itu berkata, "Beginilah kata TUHAN, Allah Israel: 'Kamu telah Kulepaskan dari perbudakan di Mesir.
Jdg 6:9  Kamu telah Kuselamatkan dari orang Mesir dan dari orang-orang negeri ini yang memerangi kamu. Mereka telah Kuusir dari negeri ini ketika kamu menyerang mereka, lalu Kuberikan tanah mereka kepadamu.
Jdg 6:10  Sudah Kukatakan kepadamu bahwa Akulah TUHAN Allahmu, dan bahwa kamu tak boleh menyembah dewa-dewa bangsa Amori yang negerinya kamu diami sekarang ini. Tetapi kamu tidak menuruti perintah-Ku.'"
Jdg 6:11  Pada suatu hari malaikat TUHAN datang ke kota Ofra, lalu duduk di bawah pohon terpentin, milik Yoas, seorang dari golongan Kaum Abiezer. Anaknya yang bernama Gideon, sedang menebah gandum di tempat pemerasan anggur. Ia melakukan itu dengan sembunyi-sembunyi supaya tidak dilihat oleh orang Midian.
Jdg 6:12  Malaikat TUHAN itu menampakkan diri kepadanya dan berkata, "Hai, pemuda yang perkasa! TUHAN besertamu!"
Jdg 6:13  Sahut Gideon, "Maaf, tuan! Kalau betul TUHAN menyertai kami, mengapa segala hal ini menimpa kami? Di manakah segala keajaiban yang pernah dilakukan TUHAN dahulu dan yang diceritakan oleh nenek moyang kami tentang bagaimana TUHAN mengeluarkan mereka dari Mesir? Sekarang TUHAN sudah meninggalkan kami dan membiarkan kami dikuasai oleh orang-orang Midian."
Jdg 6:14  Kemudian TUHAN memberikan perintah ini kepada Gideon, "Dengan seluruh kekuatanmu pergilah melepaskan orang Israel dari kekuasaan orang Midian. Akulah yang mengutusmu."
Jdg 6:15  Gideon menjawab, "Mengapa saya, TUHAN? Mana mungkin saya melepaskan umat Israel dari kekuasaan orang Midian. Di dalam suku Manasye, kaum sayalah yang paling lemah. Dan di dalam keluarga saya pun sayalah pula yang paling tak berarti."
Jdg 6:16  "Pasti kau bisa!" kata TUHAN, "sebab Akulah yang akan membantumu. Orang-orang Midian akan dapat kauhancurkan dengan mudah, seolah-olah kau hanya menghadapi satu orang saja."
Jdg 6:17  Jawab Gideon, "Kalau saya benar-benar disukai TUHAN, cobalah berikan suatu tanda bahwa memang Tuhanlah yang memberi perintah itu kepada saya.
Jdg 6:18  Saya mohon sudilah TUHAN menunggu dahulu di sini sampai saya menyajikan makanan kepada TUHAN." "Baik," jawab TUHAN, "Aku akan menunggu sampai kau kembali."
Jdg 6:19  Maka pulanglah Gideon lalu memasak seekor kambing yang muda, dan mengambil sepuluh kilogram tepung, kemudian membuat roti yang tidak beragi. Setelah itu daging itu ditaruhnya di dalam keranjang, dan kuahnya di dalam periuk, lalu semuanya itu dibawa dan disajikan kepada malaikat TUHAN yang sedang menunggu di bawah pohon terpentin itu.
Jdg 6:20  Malaikat TUHAN itu berkata, "Letakkan daging dan roti itu di atas batu ini lalu tuanglah kuah daging itu ke atasnya." Gideon melakukan apa yang diperintahkan kepadanya.
Jdg 6:21  Kemudian malaikat TUHAN itu mengulurkan tangannya lalu dengan ujung kayu yang sedang dipegangnya ia menyentuh daging dan roti itu. Batu itu menyala lalu habislah terbakar daging dan roti itu. Kemudian lenyaplah malaikat itu.
Jdg 6:22  Gideon menyadari bahwa yang telah datang kepadanya itu adalah malaikat TUHAN sendiri. Maka dengan takut berkatalah Gideon, "Celakalah saya, ya TUHAN Yang Mahatinggi! Sebab saya telah berhadapan muka dengan malaikat-Mu!"
Jdg 6:23  Tetapi TUHAN berkata kepadanya, "Tenanglah! Jangan takut. Engkau tidak akan mati."
Jdg 6:24  Lalu Gideon membangun sebuah mezbah di sana dan menamakannya "TUHAN Sumber Sejahtera". Mezbah itu masih ada di Ofra, yaitu kota orang Abiezer.
Jdg 6:25  Pada malam itu TUHAN memberi perintah ini kepada Gideon, "Ambillah sapi jantan ayahmu, yaitu sapi yang terbaik yang berumur tujuh tahun. Bongkarlah mezbah yang didirikan ayahmu untuk Baal, lalu robohkanlah tiang perlambang Dewi Asyera yang di sebelah mezbah itu.
Jdg 6:26  Dirikanlah di atas tumpukan itu sebuah mezbah yang telah dirancang dengan baik. Kemudian sapi jantan itu seluruhnya, harus kaubakar utuh sebagai kurban untuk TUHAN. Pakailah tiang perlambang Dewi Asyera yang telah kaurobohkan itu sebagai kayu bakarnya."
Jdg 6:27  Maka Gideon memilih sepuluh orang hambanya lalu pergi melaksanakan apa yang diperintahkan TUHAN kepadanya. Tetapi karena ia terlalu takut kepada keluarganya dan kepada orang-orang di desanya itu, ia tidak berani melakukan hal itu siang-siang. Jadi ia melakukannya pada waktu malam.
Jdg 6:28  Ketika penduduk kota itu bangun keesokan paginya, mereka melihat mezbah untuk Baal dan tiang perlambang Dewi Asyera sudah roboh. Mereka melihat juga bahwa di tempat itu telah dibangun sebuah mezbah dan telah pula dibakar sapi jantan sebagai kurban.
Jdg 6:29  Bertanyalah mereka seorang kepada yang lain, "Perbuatan siapakah ini?" Setelah diselidiki, mereka mendapati bahwa Gideon anak Yoaslah yang melakukannya.
Jdg 6:30  Lalu kata mereka kepada Yoas, "Anakmu telah membongkar mezbah Baal dan merobohkan tiang perlambang Dewi Asyera yang di sebelah mezbah itu. Karena itu, bawalah anakmu itu keluar ke sini. Dia harus mati!"
Jdg 6:31  Tetapi Yoas menjawab kepada semua penantangnya itu, "Apakah kalian berpihak kepada Baal dan mau membela dia? Siapa yang berpihak kepada dia, akan dibunuh hari ini! Kalau Baal memang Allah, biarlah ia sendiri membela dirinya, sebab yang dibongkar itu adalah mezbahnya."
Jdg 6:32  Karena Yoas berkata, "Biar Baal sendiri melawan Gideon, sebab ia telah membongkar mezbahnya," maka semenjak itu orang menamakan Gideon "Yerubaal".
Jdg 6:33  Pada waktu itu semua orang Midian, Amalek dan suku-suku bangsa lain di padang pasir sebelah timur Yordan telah berkumpul. Mereka menyeberangi Sungai Yordan, dan berkemah di Lembah Yizreel.
Jdg 6:34  Tetapi Gideon dikuasai oleh Roh TUHAN, sehingga ia meniup trompet untuk memanggil orang-orang kaum Abiezer supaya datang bergabung dengan dia.
Jdg 6:35  Ia juga mengirim utusan kepada suku Asyer, Zebulon, Naftali dan suku Manasye--baik yang di sebelah timur maupun yang di sebelah barat Sungai Yordan, untuk minta mereka bergabung dengan dia. Maka mereka semua datang untuk membantu Gideon.
Jdg 6:36  Lalu Gideon berkata kepada Allah, "TUHAN, Engkau telah mengatakan bahwa Engkau mau memakai saya untuk membebaskan umat Israel.
Jdg 6:37  Jika memang begitu, maka di atas tanah ini, tempat kami biasanya melepaskan gandum dari tangkainya telah saya letakkan guntingan bulu domba. Kalau besok hanya bulu domba ini saja yang dibasahi embun, dan tanahnya tidak, maka itulah buktinya bahwa Engkau akan memakai saya untuk membebaskan umat Israel."
Jdg 6:38  Maka terjadilah demikian: Ketika Gideon bangun keesokan paginya dan memeras bulu domba itu, ia mendapat air embun semangkuk.
Jdg 6:39  Lalu ia berkata kepada Allah, "Tuhan, janganlah kiranya marah kalau saya berbicara sekali lagi. Ijinkanlah saya membuat satu percobaan lagi dengan guntingan bulu domba ini. Tetapi kali ini hendaknya bulu domba ini saja yang tetap kering, sedangkan tanahnya menjadi basah."
Jdg 6:40  Malam itu Allah melakukan apa yang diminta Gideon itu, sehingga keesokan paginya bulu domba itu ternyata kering, tapi tanahnya telah basah karena embun.
Jdg 7:1  Suatu hari Gideon dengan semua anak buahnya bangun pagi-pagi sekali. Mereka pergi dan berkemah di dekat sumber air Harod. Perkemahan orang Midian berada di sebelah utara dari mereka, di lembah dekat Bukit More.
Jdg 7:2  TUHAN berkata kepada Gideon, "Anak buahmu terlalu banyak. Aku tak mau memberikan kemenangan kepada mereka atas orang Midian, sebab nanti mereka pikir mereka menang karena kekuatan sendiri, sehingga mereka tidak memuji Aku.
Jdg 7:3  Jadi, umumkanlah kepada anak buahmu, 'Siapa yang merasa takut, harus cepat-cepat meninggalkan Gunung Gilead ini dan kembali ke rumahnya.'" Maka ada 22.000 orang yang pulang; dan hanya 10.000 yang tinggal.
Jdg 7:4  Lalu TUHAN berkata lagi kepada Gideon, "Anak buahmu masih terlalu banyak. Bawalah mereka ke sungai. Di sana Aku akan menyaring mereka untuk engkau. Jika Aku berkata kepadamu, 'Orang ini harus pergi dengan engkau,' ia harus pergi. Tetapi jika Aku berkata, 'Orang ini tidak boleh pergi dengan engkau,' ia tidak boleh pergi."
Jdg 7:5  Maka Gideon membawa orang-orang itu ke sungai, lalu TUHAN berkata kepada Gideon, "Orang yang menjilat air seperti anjing, harus kaupisahkan dari orang yang berlutut untuk minum."
Jdg 7:6  Maka ada 300 orang yang menjilat air dari tangannya; semua yang lain berlutut untuk minum.
Jdg 7:7  Kemudian TUHAN berkata kepada Gideon, "Dengan ketiga ratus orang yang menjilat air itu, Aku akan membebaskan kamu dan memberikan kemenangan kepadamu atas orang Midian. Suruhlah yang lainnya pulang."
Jdg 7:8  Karena itu Gideon menyuruh semua orang Israel yang lain pulang kecuali yang tiga ratus itu. Semua bekal dan trompet diambil dari orang-orang yang pulang itu. Orang Midian berada di lembah di bawah tempat orang Israel berkemah.
Jdg 7:9  Malam itu TUHAN berkata kepada Gideon, "Bangun, Gideon! Pergilah menyerang perkemahan orang Midian. Aku memberikan kemenangan kepadamu atas mereka.
Jdg 7:10  Tetapi kalau kau takut, ajaklah Pura, hambamu itu pergi bersama-sama ke perkemahan musuh.
Jdg 7:11  Nanti kau akan mendengar apa yang dipercakapkan musuh, dan hal itu akan menjadikan engkau berani menyerang." Lalu pergilah Gideon dengan Pura, ke perbatasan perkemahan musuh.
Jdg 7:12  Orang Midian, Amalek dan orang-orang lainnya dari padang pasir, tersebar di mana-mana di lembah itu. Kelihatannya seperti belalang yang berkerumun. Unta-unta mereka banyak sekali seperti pasir di pinggir pantai.
Jdg 7:13  Ketika Gideon tiba di sana, ia mendengar seseorang sedang menceritakan mimpinya kepada kawannya. Orang itu berkata begini, "Saya bermimpi ada seketul roti terguling-guling masuk ke perkemahan kita lalu melanggar sebuah kemah. Kemah itu roboh sampai menjadi serata tanah."
Jdg 7:14  Kawannya menjawab, "Wah, itu pasti pedang Gideon anak Yoas, orang Israel itu! Berarti Allah sudah memberikan kemenangan kepadanya atas orang Midian dan seluruh angkatan perang kita!"
Jdg 7:15  Ketika Gideon mendengar tentang mimpi orang itu dan artinya, ia pun sujud menyembah Allah. Kemudian ia kembali ke perkemahan Israel dan berkata, "Bersiap-siaplah! TUHAN memberikan kemenangan kepadamu atas angkatan perang Midian!"
Jdg 7:16  Lalu Gideon membagi ketiga ratus anak buahnya menjadi tiga regu. Setiap orang diberi sebuah trompet dan sebuah kendi yang berisi obor.
Jdg 7:17  Kata Gideon kepada mereka, "Kalau saya tiba di perbatasan perkemahan orang Midian, perhatikan saya baik-baik, dan lakukanlah apa yang saya lakukan.
Jdg 7:18  Apabila saya dan regu saya meniup trompet, kalian juga harus meniup trompetmu. Lalu kalian harus berteriak di sekeliling perkemahan itu, 'Untuk TUHAN dan untuk Gideon!'"
Jdg 7:19  Menjelang tengah malam, Gideon dan orang-orangnya tiba di perbatasan perkemahan musuh. Pada waktu itu orang yang mengawal perkemahan itu, baru saja berganti jaga. Lalu Gideon dan regunya meniup trompet dan memecahkan kendi-kendi mereka.
Jdg 7:20  Kedua regu yang lainnya berbuat begitu juga. Lalu mereka semuanya mengangkat obor-obor mereka dengan tangan kiri, dan meniup trompet yang ada di tangan kanan mereka. Mereka berteriak, "Pedang untuk TUHAN dan untuk Gideon!"
Jdg 7:21  Mereka semuanya tetap berdiri masing-masing di tempatnya di sekeliling perkemahan itu. Maka seluruh angkatan perang musuh itu lari terbirit-birit sambil berteriak-teriak.
Jdg 7:22  Ketika Gideon dan orang-orangnya sedang membunyikan trompet, TUHAN membuat angkatan perang musuh saling menyerang satu sama lain dengan pedang. Mereka lari ke arah Zerera sampai sejauh Bet-Sita dan kota Abel-Mehola dekat Tabat.
Jdg 7:23  Maka suku Naftali, Asyer, dan suku Manasye baik yang di sebelah timur maupun yang di sebelah barat Sungai Yordan dipanggil juga. Lalu mereka mengejar tentara Midian.
Jdg 7:24  Juga ke seluruh daerah pegunungan Efraim, Gideon mengirim berita panggilan ini, "Mari turun memerangi orang Midian. Pertahankanlah Sungai Yordan dan semua anak sungai sampai sejauh Bet-Bara. Jangan biarkan orang Midian menyeberang dari sana." Maka orang-orang Efraim pun dikerahkan, lalu mereka mempertahankan Sungai Yordan dan semua anak sungai sampai sejauh Bet-Bara.
Jdg 7:25  Dua tokoh bangsa Midian, yaitu Oreb dan Zeeb ditangkap oleh mereka. Oreb dibunuh di Batu Oreb dan Zeeb dibunuh di tempat pemerasan anggur Zeeb. Orang-orang Efraim itu terus mengejar orang Midian, lalu membawa kepala Oreb dan Zeeb kepada Gideon di seberang Yordan bagian timur.
Jdg 8:1  Orang Efraim berkata kepada Gideon, "Mengapa kami tidak diajak waktu engkau berangkat untuk memerangi orang Midian? Apa sebab kami kauperlakukan demikian?" Begitulah mereka mengomel terhadap Gideon.
Jdg 8:2  Tetapi Gideon menjawab, "Apa yang saya lakukan tidaklah begitu berarti bila dibandingkan dengan apa yang sudah kalian lakukan. Walaupun yang kalian lakukan hanya sedikit, tetapi yang sedikit itu jauh lebih berharga daripada semua yang telah dilakukan oleh seluruh pasukan kaum saya.
Jdg 8:3  Kalianlah yang mendapat kuasa dari Allah sehingga dapat membunuh Oreb dan Zeeb, kedua tokoh Midian itu! Dibandingkan dengan itu, apakah jasa saya?" Setelah Gideon berkata demikian, mereka tidak marah lagi.
Jdg 8:4  Pada saat itu Gideon dengan ketiga ratus anak buahnya sampai ke seberang Sungai Yordan. Mereka lelah sekali tetapi masih terus saja mengejar musuh.
Jdg 8:5  Kemudian mereka sampai di Sukot. Di situ Gideon berkata kepada penduduk kota, "Tolonglah berikan kepada anak buahku sedikit makanan. Mereka lelah sekali sedangkan kami masih mengejar Zebah dan Salmuna, raja-raja Midian itu."
Jdg 8:6  Tetapi pemimpin-pemimpin di Sukot itu menjawab, "Ah, kalian belum lagi menangkap Zebah dan Salmuna, kalian sudah minta makanan! Untuk apa anak buahmu harus diberi makan?"
Jdg 8:7  Maka kata Gideon, "Baiklah! Nanti kalau TUHAN telah memberikan kepadaku kemenangan atas Zebah dan Salmuna, kalian akan kucambuk dengan duri dan onak dari padang pasir!"
Jdg 8:8  Kemudian Gideon meneruskan perjalanan lalu sampai di Pnuel. Di sana juga ia minta makanan untuk anak buahnya. Tetapi jawaban orang Pnuel sama dengan jawaban orang Sukot.
Jdg 8:9  Oleh karena itu Gideon berkata kepada mereka, "Nanti saya akan kembali dengan selamat. Dan pada saat itu saya akan merobohkan menara kota ini!"
Jdg 8:10  Sementara itu Zebah dan Salmuna dengan seluruh pasukan mereka sedang berada di Karkor. Dari seluruh pasukan orang-orang padang pasir, hanya ada kira-kira 15.000 orang saja yang sisa; 120.000 tentaranya sudah tewas.
Jdg 8:11  Gideon menyusuri perbatasan padang pasir di sebelah timur Nobah dan Yogbeha lalu menyerang musuh secara mendadak.
Jdg 8:12  Zebah dan Salmuna, kedua raja Midian itu lari, tetapi Gideon mengejar dan menangkap mereka, sehingga seluruh angkatan perang mereka menjadi panik.
Jdg 8:13  Kemudian dalam perjalanan pulang dari pertempuran, Gideon mengambil jalan yang melewati pendakian Heres.
Jdg 8:14  Di situ ia menangkap seorang muda dari Sukot, lalu ia menanyai orang itu tentang nama-nama para pemuka di Sukot. Maka orang itu menuliskan nama-nama itu untuk Gideon--semuanya ada 77 orang.
Jdg 8:15  Sesudah itu Gideon pergi menemui pemuka-pemuka di kota itu lalu berkata, "Pasti kalian masih ingat bahwa kalian pernah menolak untuk menolong saya, karena saya belum menangkap Zebah dan Salmuna, bukan? Kalian tak mau memberi makanan kepada anak buah saya; padahal mereka sedang lelah pada waktu itu. Lihat, ini Zebah dan Salmuna."
Jdg 8:16  Sesudah itu Gideon mengambil duri dan onak dari padang pasir, lalu menghajar para pemuka kota Sukot itu.
Jdg 8:17  Di Pnuel pun ia merobohkan menara dan membunuh semua orang laki-laki di kota itu.
Jdg 8:18  Sesudah itu Gideon bertanya kepada Zebah dan Salmuna, "Bagaimana rupa orang-orang yang kalian bunuh di Tabor itu?" Zebah dan Salmuna menjawab, "Mereka semuanya mirip engkau, seperti keturunan bangsawan."
Jdg 8:19  "Itu saudara-saudara saya--anak-anak ibu saya," kata Gideon. "Seandainya kalian tidak membunuh mereka, sungguh mati saya tidak akan membunuh kalian juga."
Jdg 8:20  Lalu kata Gideon kepada Yeter, anak sulungnya, "Bunuhlah mereka!" Tetapi Yeter masih muda. Jadi, ia takut sehingga ia tidak mencabut pedangnya untuk membunuh.
Jdg 8:21  Kata Zebah dan Salmuna kepada Gideon, "Ayo, kalau kau laki-laki, bunuhlah kami dengan tanganmu sendiri!" Maka Gideon pun membunuh mereka lalu mengambil semua perhiasan yang ada pada leher unta mereka.
Jdg 8:22  Kemudian berkatalah umat Israel kepada Gideon, "Kau sudah melepaskan kami dari kekuasaan orang Midian. Sebab itu hendaklah kau menjadi pemimpin kami--kau dan keturunanmu."
Jdg 8:23  Sahut Gideon, "Tidak. Saya tidak akan menjadi pemimpinmu; anak saya pun tidak. Tuhanlah pemimpinmu."
Jdg 8:24  Tetapi kemudian Gideon berkata lagi, "Hanya satu permintaan saya; anting-anting yang kalian rampas dari musuh, hendaklah kalian serahkan semuanya kepada saya." (Orang Midian biasanya memakai anting-anting emas; sama seperti orang-orang lainnya yang tinggal di padang pasir.)
Jdg 8:25  "Baik," jawab mereka, "dengan senang hati kami akan memberikannya kepadamu." Lalu mereka membentangkan sehelai kain, kemudian setiap orang datang dan meletakkan di situ anting-anting yang telah dirampasnya dari musuh.
Jdg 8:26  Semua anting-anting emas yang diserahkan kepada Gideon pada waktu itu beratnya hampir dua puluh kilogram, belum termasuk perhiasan-perhiasa kalung, kain ungu yang dipakai raja-raja Midian, dan kalung yang terdapat pada leher unta-unta mereka.
Jdg 8:27  Dari emas itu, Gideon membuat sebuah patung berhala, lalu menempatkannya di desanya, yaitu di Ofra. Semua orang Israel pergi ke sana untuk menyembah patung berhala itu, sehingga mereka tidak lagi memperhatikan Allah. Dan hal itu menjadi seperti jerat bagi Gideon dan keluarganya.
Jdg 8:28  Orang Midian tak dapat berkutik lagi; mereka telah dikalahkan oleh orang Israel. Maka amanlah negeri itu empat puluh tahun lamanya selama Gideon masih hidup.
Jdg 8:29  Setelah pertempuran itu, Gideon kembali ke kampung halamannya dan tinggal di sana.
Jdg 8:30  Ia mempunyai tujuh puluh orang anak laki-laki, sebab istrinya banyak.
Jdg 8:31  Ada juga selirnya di Sikhem; dari selirnya ini ia mendapat seorang anak laki-laki yang dinamainya Abimelekh.
Jdg 8:32  Setelah lanjut sekali usianya barulah Gideon anak Yoas itu meninggal. Ia dimakamkan di kuburan Yoas, ayahnya. Kuburan itu terletak di Ofra, yaitu kota kaum Abiezer.
Jdg 8:33  Setelah Gideon meninggal, orang Israel tidak setia lagi kepada Allah. Mereka menyembah dewa-dewa Baal dan menjadikan Baal-Berit dewa mereka.
Jdg 8:34  Mereka tidak beribadat lagi kepada Allah, TUHAN mereka, yang telah melepaskan mereka dari kekuasaan semua musuh di sekeliling mereka.
Jdg 8:35  Mereka juga tidak membalas budi kepada keluarga Gideon atas segala jasa Gideon bagi umat Israel.
Jdg 9:1  Suatu hari Abimelekh anak Gideon pergi ke kota Sikhem tempat kaum keluarga ibunya. Di sana ia berbicara dengan mereka, katanya,
Jdg 9:2  "Coba tanyakan kepada seluruh penduduk Sikhem, mana yang lebih baik: Diperintah oleh ketujuh puluh anak Gideon, atau oleh satu orang saja? Ingatkan mereka juga bahwa saya masih ada hubungan darah dengan mereka."
Jdg 9:3  Maka pergilah kaum keluarga ibu Abimelekh itu kepada orang-orang Sikhem dan menyampaikan pesan Abimelekh kepada mereka. Lalu penduduk Sikhem itu memutuskan untuk mengikuti Abimelekh, sebab menurut mereka ia memang masih ada hubungan darah dengan mereka.
Jdg 9:4  Kemudian mereka memberikan kepadanya tujuh puluh uang perak yang diambil dari rumah penyembahan Baal-Berit. Dengan uang itu Abimelekh mengupah segerombolan orang-orang kurang ajar untuk bergabung dengannya,
Jdg 9:5  lalu pergi ke kampung halaman ayahnya di Ofra. Di sana ia membunuh ketujuh puluh saudaranya itu di atas satu batu. Tetapi Yotam, anak bungsu Gideon tidak turut terbunuh karena ia bersembunyi.
Jdg 9:6  Maka berkumpullah seluruh penduduk Sikhem dan Bet-Milo, lalu pergi ke pohon terpentin keramat di Sikhem. Di sana mereka menobatkan Abimelekh menjadi raja.
Jdg 9:7  Setelah Yotam mendengar tentang hal itu, ia pergi berdiri di puncak Gunung Gerizim, lalu berteriak ke arah orang-orang itu, "Dengarkan saya, hai orang-orang Sikhem, supaya Allah juga mendengarkan kalian!
Jdg 9:8  Sekali peristiwa berkumpullah pohon-pohon hendak memilih raja untuk mereka. Maka berkatalah mereka kepada pohon zaitun, 'Jadilah raja kami.'
Jdg 9:9  Jawab pohon zaitun, 'Ah, saya tidak mau. Sebab kalau saya menjadi rajamu, saya harus berhenti menghasilkan minyak zaitun yang dipakai untuk menghormati dewa-dewa dan manusia.'
Jdg 9:10  Lalu kata pohon-pohon kepada pohon ara, 'Kau saja menjadi raja kami.'
Jdg 9:11  Tetapi pohon ara menjawab, 'Saya tidak mau, sebab untuk menjadi rajamu saya harus berhenti menghasilkan buah ara yang manis-manis.'
Jdg 9:12  Setelah itu, berkatalah pohon-pohon kepada pokok anggur, 'Nah, kaulah yang menjadi raja kami.'
Jdg 9:13  Tetapi pokok anggur menjawab, 'Masakan saya harus berhenti menghasilkan anggur yang menyenangkan hati dewa-dewa dan manusia, hanya untuk menjadi rajamu!'
Jdg 9:14  Karena itu berkatalah semua pohon kepada semak duri, 'Kau sajalah menjadi raja kami.'
Jdg 9:15  Semak duri itu menjawab, 'Kalau kalian sungguh-sungguh mau menjadikan saya rajamu, mari berlindung di bawah naungan saya. Kalau kalian tidak mau, maka dari cabang-cabang saya yang berduri ini akan keluar api yang membakar habis pohon-pohon aras di Libanon.'"
Jdg 9:16  "Nah," kata Yotam selanjutnya, "Apakah dengan menobatkan Abimelekh menjadi raja, kalian telah bertindak dengan tulus dan ikhlas? Sudahkah kalian melakukan yang sepatutnya terhadap Gideon, ayahku, dan kaum keluarganya? Sudahkah kalian membalas jasa-jasa ayahku
Jdg 9:17  yang telah berjuang untuk kalian? Ingat, ia telah mempertaruhkan nyawanya untuk melepaskan kalian dari kekuasaan orang-orang Midian.
Jdg 9:18  Tetapi sekarang kalian mengkhianati keluarga ayahku. Anak-anaknya telah kalian bunuh semuanya--tujuh puluh orang sekaligus di atas satu batu. Abimelekh adalah anak ayahku dari hambanya. Tetapi kalian menobatkan dia menjadi raja Sikhem, hanya karena ia masih ada hubungan darah dengan kalian.
Jdg 9:19  Kalau tindakanmu hari ini terhadap ayahku dan keluarganya itu tulus dan ikhlas, silakan bersenang-senang dengan Abimelekh, dan biar dia bersenang-senang dengan kalian.
Jdg 9:20  Tetapi kalau tidak demikian halnya, semoga dari Abimelekh keluarlah api yang membakar habis penduduk Sikhem serta Bet-Milo. Dan semoga dari penduduk Sikhem serta Bet-Milo pun keluarlah api yang membakar habis Abimelekh."
Jdg 9:21  Setelah itu Yotam melarikan diri lalu tinggal di Beer, karena ia takut kepada Abimelekh, saudaranya itu.
Jdg 9:22  Tiga tahun lamanya Abimelekh memerintah Israel.
Jdg 9:23  Lalu Tuhan menimbulkan permusuhan antara Abimelekh dengan orang-orang Sikhem sehingga mereka melawan dia.
Jdg 9:24  Hal ini terjadi supaya terbalaslah kejahatan Abimelekh dan orang-orang Sikhem yang menghasut dia untuk membunuh ketujuh puluh anak-anak Gideon.
Jdg 9:25  Orang-orang Sikhem itu menempatkan penghadang-penghadang di puncak-puncak gunung untuk menangkap Abimelekh. Setiap orang yang lewat di situ dirampok oleh penghadang-penghadang itu. Maka hal itu diberitahukan kepada Abimelekh.
Jdg 9:26  Pada waktu itu Gaal anak Ebed telah datang bersama-sama dengan saudara-saudaranya untuk tinggal di Sikhem; dan orang-orang Sikhem percaya kepadanya.
Jdg 9:27  Mereka pergi memetik anggur di kebun-kebun anggur, lalu membuat minuman anggur, dan mengadakan pesta. Mereka makan minum di tempat penyembahan dewa mereka sambil menyumpah-nyumpahi Abimelekh.
Jdg 9:28  "Kita ini siapa sehingga kita harus tunduk kepada Abimelekh?" kata Gaal. "Dia itu siapa sebenarnya? Bukankah dia anak Gideon yang dengan Zebul wakilnya itu telah menjadi hamba Hemor, ayah Sikhem, leluhur kita!"
Jdg 9:29  "Seandainya saya yang memimpin orang-orang Sikhem ini, pasti sudah habislah riwayat Abimelekh itu! Akan saya katakan kepadanya, 'Ayo maju! Kerahkan tentaramu dan marilah keluar berperang!'"
Jdg 9:30  Zebul, penguasa kota Sikhem, menjadi marah ketika mendengar omongan Gaal itu.
Jdg 9:31  Ia mengirim utusan kepada Abimelekh di Aruma untuk menyampaikan pesan ini, "Gaal anak Ebed bersama-sama dengan saudara-saudaranya telah datang ke Sikhem dan menghasut penduduknya supaya melawan engkau.
Jdg 9:32  Karena itu, hendaklah engkau dan orang-orangmu pergi bersembunyi di ladang pada waktu malam.
Jdg 9:33  Besok pagi apabila matahari terbit, serbulah kota itu dengan mendadak. Dan jika Gaal dengan orang-orangnya keluar melawan engkau, itulah kesempatanmu untuk menggempur dia!"
Jdg 9:34  Maka Abimelekh dengan semua anak buahnya bergerak pada waktu malam. Mereka berpencar menjadi empat pasukan lalu pergi bersembunyi di sekeliling kota Sikhem.
Jdg 9:35  Segera setelah Abimelekh dan orang-orangnya melihat Gaal keluar dan berdiri di gerbang kota, mereka semuanya keluar dari tempat persembunyian.
Jdg 9:36  Gaal melihat mereka, lalu berkata kepada Zebul, "Lihat! Ada orang-orang turun kemari dari puncak-puncak gunung!" "Ah, itu bukan orang," sahut Zebul, "itu hanya bayang-bayangan di gunung."
Jdg 9:37  Lalu kata Gaal lagi, "Benar! Ada orang turun dari lereng bukit, dan satu kelompok lagi sedang menuju kemari dari arah pohon terpentin keramat!"
Jdg 9:38  Berkatalah Zebul kepada Gaal, "Nah, di mana sekarang mulut besarmu itu? Dulu katamu untuk apa kita harus tunduk kepada si Abimelekh itu? Inilah orang-orang yang kauhina itu. Ayo maju sekarang menyerang mereka!"
Jdg 9:39  Lalu Gaal membawa keluar orang-orang Sikhem dan melawan Abimelekh.
Jdg 9:40  Tetapi Abimelekh menyerbu dia sampai ia melarikan diri. Korban berjatuhan sampai di depan pintu gerbang kota.
Jdg 9:41  Akhirnya Abimelekh kembali ke Aruma, dan menetap di sana. Zebul mengusir Gaal dengan saudara-saudaranya keluar dari Sikhem, sehingga mereka tidak dapat tinggal di sana lagi.
Jdg 9:42  Besoknya orang-orang Sikhem mulai meninggalkan kota, hendak ke ladang dan hal itu dilaporkan kepada Abimelekh.
Jdg 9:43  Lalu Abimelekh mengumpulkan orang-orangnya, dan membagi mereka menjadi tiga pasukan, lalu mereka pergi bersembunyi di ladang-ladang. Ketika ia melihat penduduk kota itu menuju ke ladang, ia keluar dari persembunyiannya lalu membunuh mereka.
Jdg 9:44  Sementara Abimelekh dengan pasukannya cepat-cepat maju menyerbu gerbang kota, kedua pasukannya yang lain mengejar dan membunuh orang-orang yang lari ke ladang-ladang.
Jdg 9:45  Abimelekh bertempur terus sepanjang hari dan menduduki kota itu; penduduknya dibunuh, lalu kotanya dihancurkannya kemudian ditaburi garam.
Jdg 9:46  Ketika orang-orang yang berada di dalam benteng Sikhem mendengar tentang hal itu, mereka masuk ke dalam lubang di bawah kuil El-Berit untuk berlindung.
Jdg 9:47  Lalu orang memberitahukan kepada Abimelekh bahwa orang-orang di benteng Sikhem itu telah berkumpul di sana.
Jdg 9:48  Karena itu, Abimelekh dengan orang-orangnya pergi ke Gunung Zalmon. Di sana ia memotong dahan-dahan kayu dengan kapak, lalu memikul dahan-dahan itu di atas pundaknya, kemudian menyuruh orang-orangnya juga cepat-cepat melakukan seperti yang telah dilakukannya.
Jdg 9:49  Maka mereka masing-masing memotong dahan-dahan kayu, lalu berjalan mengikuti Abimelekh. Semua dahan kayu itu mereka timbun di mulut lubang kuil itu, lalu membakarnya, sementara orang-orang Sikhem itu masih berada di dalamnya. Karena itu matilah semua penduduk Sikhem itu, kira-kira seribu orang pria dan wanita.
Jdg 9:50  Setelah itu Abimelekh pergi ke Tebes. Ia mengepung kota itu lalu mendudukinya.
Jdg 9:51  Di tengah-tengah kota itu ada sebuah menara yang kuat. Maka seluruh penduduk kota itu lari ke sana. Mereka masuk ke dalam menara itu dan mengunci pintunya, lalu naik ke atas.
Jdg 9:52  Kemudian datanglah Abimelekh ke situ untuk menyerang menara itu. Ketika ia tiba di pintu menara itu dan hendak membakarnya,
Jdg 9:53  seorang wanita menjatuhkan sebuah batu penggilingan ke atas kepalanya sehingga retak batok kepalanya itu.
Jdg 9:54  Cepat-cepat Abimelekh memanggil pemuda yang memikul senjatanya dan berkata, "Cabutlah pedangmu dan bunuhlah saya; saya tidak mau dikatakan orang bahwa saya dibunuh oleh wanita." Maka pemuda itu menikam dia sampai mati.
Jdg 9:55  Ketika orang-orang Israel melihat bahwa Abimelekh sudah mati, mereka semuanya pulang.
Jdg 9:56  Demikianlah caranya Allah membalas Abimelekh atas kejahatan yang dilakukannya terhadap ayahnya, ketika ia membunuh ketujuh puluh orang saudara-saudaranya.
Jdg 9:57  Dan Tuhan menghukum juga orang-orang Sikhem karena kejahatan mereka. Hal itu tepat seperti yang telah dikatakan Yotam anak Gideon tentang mereka ketika ia menyumpahi mereka.
Jdg 10:1  Setelah Abimelekh meninggal, Tola anak Pua, menjadi pemimpin untuk membebaskan orang Israel. Ia adalah cucu Dodo dari suku Isakhar. Ia tinggal di Samir di daerah pegunungan Efraim.
Jdg 10:2  Dua puluh tiga tahun lamanya ia memimpin umat Israel, kemudian ia meninggal lalu dimakamkan di Samir.
Jdg 10:3  Setelah Tola, Israel dipimpin oleh Yair dua puluh dua tahun lamanya. Yair berasal dari Gilead.
Jdg 10:4  Anaknya yang laki-laki ada tiga puluh orang. Mereka mempunyai tiga puluh ekor keledai tunggangan dan tiga puluh desa di daerah Gilead. Desa-desa itu masih disebut desa-desa Yair.
Jdg 10:5  Yair pun meninggal, dan dimakamkan di Kamon.
Jdg 10:6  Umat Israel berdosa lagi kepada TUHAN. Mereka tidak menghiraukan TUHAN dan tidak menyembah Dia lagi. Mereka menyembah dewa-dewa Baal dan Asytoret, serta dewa-dewa bangsa Siria, Sidon, Moab, Amon dan Filistin.
Jdg 10:7  Karena itu TUHAN marah kepada umat Israel, sehingga Ia membiarkan bangsa Filistin dan bangsa Amon menguasai mereka.
Jdg 10:8  Delapan belas tahun lamanya kedua bangsa itu menekan semua orang Israel yang tinggal di Gilead, yaitu daerah orang Amori sebelah timur Sungai Yordan. Mereka memperlakukan orang-orang Israel dengan kejam.
Jdg 10:9  Orang Amon malah menyeberangi Sungai Yordan untuk pergi memerangi suku Yehuda, Benyamin dan Efraim sehingga umat Israel sangat menderita.
Jdg 10:10  Lalu umat Israel berseru kepada TUHAN, katanya, "Kami telah berdosa kepada-Mu, karena kami tidak menghiraukan Engkau, ya Allah kami. Kami menyembah dewa-dewa Baal."
Jdg 10:11  TUHAN menjawab, "Dahulu kamu dijajah oleh bangsa Mesir, Amori, Amon, Filistin, Sidon, Amalek dan bangsa Maon. Lalu pada waktu itu kamu berseru meminta tolong kepada-Ku, bukan? Coba ingat, apakah Aku tidak menyelamatkan kamu dari kekuasaan mereka?
Jdg 10:13  Meskipun begitu, kamu bahkan masih juga membelakangi Aku, lalu pergi menyembah dewa-dewa. Sekarang Aku tak mau lagi membebaskan kamu.
Jdg 10:14  Pergilah saja kepada dewa-dewa yang kamu pilih itu. Berserulah kepada mereka! Biarlah mereka yang menolong pada waktu kamu dalam kesukaran."
Jdg 10:15  Tetapi umat Israel berkata kepada TUHAN, "Kami telah berdosa, TUHAN. Lakukanlah apa saja yang Engkau kehendaki, asal Engkau menolong kami hari ini."
Jdg 10:16  Kemudian mereka membuang semua patung berhala, lalu menyembah Allah. Maka Allah pun terharu hatinya melihat mereka dalam kesukaran.
Jdg 10:17  Pada waktu itu tentara Amon sedang bersiap-siap untuk berperang. Mereka berkemah di Gilead. Umat Israel pun sudah berkumpul dan berkemah di Gilead.
Jdg 10:18  Di sana rakyat dan para pemimpin dari suku-suku Israel itu saling bertanya apakah ada di antara mereka yang berani mengangkat senjata melawan bangsa Amon. Orang itu akan dijadikan pemimpin seluruh penduduk Gilead.
Jdg 11:1  Yefta adalah seorang prajurit yang gagah berani. Ia tinggal di Gilead. Ayahnya bernama Gilead tetapi ibunya seorang pelacur.
Jdg 11:2  Ayahnya itu mempunyai juga anak-anak lelaki yang lain dari istrinya yang sah. Setelah anak-anak itu dewasa, mereka mengusir Yefta dari rumah. Kata mereka kepadanya, "Kau tidak akan mendapat warisan apa-apa dari ayah kami, sebab kau anak dari wanita lain."
Jdg 11:3  Karena itu Yefta berpisah dari saudara-saudaranya lalu pergi dan tinggal di daerah Tob. Di sana segerombolan penjahat bergabung dengan dia lalu mereka pergi merampok.
Jdg 11:4  Beberapa waktu kemudian orang Israel diserang oleh orang Amon.
Jdg 11:5  Ketika hal itu terjadi, para pemuka Israel di Gilead pergi dan mengajak Yefta meninggalkan daerah Tob.
Jdg 11:6  Mereka berkata, "Pimpinlah kami melawan orang Amon."
Jdg 11:7  Tetapi Yefta menjawab, "Bukankah kalian sangat membenci saya? Kalian telah mengusir saya dari rumah ayah saya. Mengapa sekarang pada waktu kalian dalam kesukaran, kalian datang kepada saya?"
Jdg 11:8  Mereka menjawab, "Kami datang karena kami ingin supaya kau memimpin seluruh penduduk Gilead untuk memerangi bangsa Amon sekarang juga."
Jdg 11:9  Sahut Yefta, "Kalau kalian mengajak saya kembali ke Gilead untuk memerangi bangsa Amon, dan TUHAN memberi kemenangan kepada kita, sayalah yang akan menjadi penguasa atas kalian."
Jdg 11:10  Mereka menjawab, "Kami setuju. Tuhanlah saksinya."
Jdg 11:11  Maka berangkatlah Yefta bersama-sama dengan para pemimpin Gilead. Kemudian penduduk Gilead mengangkatnya menjadi penguasa dan pemimpin. Dan di hadapan TUHAN di Mizpa, Yefta mengajukan tuntutan-tuntutan kepada penduduk Gilead.
Jdg 11:12  Setelah itu Yefta mengutus orang kepada raja Amon untuk mengatakan begini, "Apa kesalahan kami sehingga kalian memerangi negeri kami?"
Jdg 11:13  Raja bangsa Amon menjawab utusan-utusan itu, "Ketika umat Israel datang dari Mesir, mereka mengambil tanah kami, mulai dari Sungai Arnon sampai ke Sungai Yabok dan Sungai Yordan. Sekarang kembalikanlah itu secara damai."
Jdg 11:14  Lalu Yefta mengutus lagi orang-orang kepada raja Amon
Jdg 11:15  untuk menjawab begini, "Umat Israel sama sekali tidak merampas tanah orang Moab atau tanah orang Amon.
Jdg 11:16  Ketika orang Israel meninggalkan Mesir, mereka menuju ke Teluk Akaba melalui padang pasir lalu sampai di Kades.
Jdg 11:17  Kemudian mereka mengutus orang kepada raja Edom untuk minta izin melewati negerinya. Tetapi ia tidak mengizinkan mereka. Kemudian mereka meminta hal yang sama kepada raja Moab, tetapi ia pun tidak mengizinkan mereka melewati daerahnya. Karena itu umat Israel tinggal di Kades.
Jdg 11:18  Kemudian mereka meneruskan perjalanan melalui padang pasir, tetapi tidak melewati daerah Edom dan Moab, melainkan mengambil jalan keliling sampai tiba di sebelah timur Moab, di seberang Sungai Arnon. Mereka berkemah di sana, tetapi tidak menyeberangi sungai itu, karena daerah itu termasuk wilayah Moab.
Jdg 11:19  Sesudah itu umat Israel mengutus orang kepada Sihon, raja Amori di Hesybon untuk minta izin melewati daerah Amori, karena mereka hendak ke daerah mereka sendiri.
Jdg 11:20  Tetapi Sihon tidak mengizinkan mereka sebab ia tidak percaya bahwa mereka hanya mau lewat saja. Ia malah mengerahkan seluruh angkatan perangnya lalu bermarkas di Yahas kemudian menyerang Israel.
Jdg 11:21  Tetapi TUHAN, Allah Israel, memberikan kepada umat Israel kemenangan atas Sihon beserta angkatan perangnya. Demikianlah umat Israel mendapatkan semua tanah-tanah orang Amori yang tinggal di negeri itu.
Jdg 11:22  Wilayah Amori itu mereka duduki mulai dari Arnon di sebelah selatan, sampai ke sebelah utara Sungai Yabok, dan dari padang pasir di sebelah timur, sampai ke Sungai Yordan sebelah barat.
Jdg 11:23  Jadi TUHAN, Allah Israel, itulah yang mengusir orang-orang Amori untuk kepentingan umat TUHAN.
Jdg 11:24  Dan sekarang ini apakah engkau mau mengambilnya kembali? Tanah yang diberikan oleh Kamos, dewamu itu, kepadamu bolehlah tetap kalian miliki. Tetapi apa yang telah diberikan TUHAN, Allah kami kepada kami, akan kami pertahankan.
Jdg 11:25  Kau kira kau lebih baik dari Balak anak Zipor, raja Moab? Balak tidak pernah menantang atau berperang dengan kami.
Jdg 11:26  Tiga ratus tahun lamanya Israel menduduki Hesybon dan Aroer serta desa-desa di sekelilingnya, dan semua kota-kota di tepi Sungai Arnon. Dan mengapa selama itu kau tidak mengambilnya kembali?
Jdg 11:27  Tidak, kami tidak bersalah kepadamu. Engkaulah yang bersalah karena menyerang kami. Tuhanlah hakim yang hari ini memutuskan perkara ini antara bangsa Israel dan bangsa Amon."
Jdg 11:28  Tetapi raja Amon tidak menghiraukan pesan dari Yefta itu.
Jdg 11:29  Kemudian Roh TUHAN menguasai Yefta. Maka pergilah Yefta mengunjungi daerah Gilead dan Manasye kemudian kembali ke Mizpa di Gilead. Dari situ ia meneruskan perjalanannya ke wilayah bangsa Amon.
Jdg 11:30  Yefta membuat janji ini kepada TUHAN, "Kalau TUHAN mengizinkan saya mengalahkan orang Amon,
Jdg 11:31  dan saya kembali dengan selamat, maka siapa pun yang pertama-tama keluar dari rumah saya untuk menyambut saya, akan saya persembahkan sebagai kurban bakaran kepada TUHAN."
Jdg 11:32  Lalu Yefta menyeberangi Sungai Yordan untuk memerangi orang Amon, dan TUHAN memberikan kemenangan kepadanya.
Jdg 11:33  Yefta menggempur mereka habis-habisan dari Aroer sampai ke daerah sekitar Minit--seluruhnya dua puluh kota--dan sampai sejauh Abel-Keramim. Banyak sekali orang Amon yang mati, sehingga kalahlah mereka terhadap orang Israel.
Jdg 11:34  Ketika Yefta kembali ke Mizpa, anak gadisnya datang menyambut dia dengan menari sambil memukul rebana. Itulah anaknya yang satu-satunya.
Jdg 11:35  Begitu Yefta melihatnya, Yefta menjadi sangat sedih sehingga ia mengoyak-ngoyak bajunya, sambil berkata, "Aduh anakku, hancurlah hatiku! Mengapakah harus kau yang menjadikan hatiku pedih? Aku telah bersumpah kepada TUHAN, dan aku tak dapat lagi menariknya kembali!"
Jdg 11:36  Lalu kata gadis itu kepada Yefta, "Ayah sudah bersumpah kepada TUHAN, dan TUHAN telah memperkenankan Ayah membalas kejahatan orang-orang Amon, musuh Ayah itu. Jadi, apa yang telah Ayah janjikan, hendaklah ayah jalankan."
Jdg 11:37  "Hanya," kata gadis itu selanjutnya, "saya mohon satu hal: Berilah saya waktu dua bulan untuk jalan-jalan di pegunungan bersama-sama dengan kawan-kawan saya. Di sana saya akan menangisi nasib saya, sebab saya akan meninggal semasa masih perawan."
Jdg 11:38  Ayahnya mengizinkannya, lalu melepaskannya pergi. Gadis itu dan kawan-kawannya pergi ke pegunungan untuk bersedih hati di sana, karena ia akan meninggal sebelum menikah dan mempunyai anak.
Jdg 11:39  Setelah lewat dua bulan, ia kembali kepada ayahnya, lalu ayahnya melakukan apa yang telah dijanjikannya kepada TUHAN. Maka meninggallah gadis itu semasa masih perawan. Itulah asal mulanya mengapa di kalangan orang Israel, biasanya
Jdg 11:40  anak-anak gadis pergi selama empat hari setiap tahun untuk bersedih hati mengenangkan anak Yefta di Gilead.
Jdg 12:1  Pada waktu itu suku Efraim bersiap-siap untuk bertempur, lalu pergi menyeberangi Sungai Yordan menuju ke Zafon. Mereka berkata kepada Yefta, "Mengapa engkau pergi memerangi bangsa Amon tanpa mengajak kami? Sekarang kami akan membakar engkau dan seisi rumahmu."
Jdg 12:2  Berkatalah Yefta, "Antara orang Amon dan kami orang Gilead ada perselisihan yang hebat. Kalian sudah saya panggil untuk membantu membebaskan saya dari orang-orang Amon itu, tetapi kalian tidak datang.
Jdg 12:3  Jadi, setelah saya yakin bahwa kalian tidak akan datang membantu saya, saya nekad pergi memerangi orang-orang Amon itu, sekalipun saya harus mati. Lalu TUHAN memberikan kepada saya kemenangan atas mereka. Mengapa sekarang kalian datang menyerang saya?"
Jdg 12:4  Kemudian Yefta mengumpulkan semua orang Gilead, lalu mereka memerangi dan mengalahkan orang Efraim. (Orang Gilead yang tinggal di Efraim dan Manasye telah dituduh oleh orang Efraim bahwa mereka telah memisahkan diri dari Efraim, sukunya.)
Jdg 12:5  Orang-orang Gilead yang sedang berperang itu, menduduki tempat-tempat penyeberangan di Sungai Yordan, supaya dapat mencegat orang-orang Efraim yang mau lari. Pada waktu seorang Efraim, yang hendak melarikan diri, minta izin untuk menyeberang, orang Gilead bertanya kepadanya, "Apakah kau orang Efraim?" Kalau ia menjawab, "Bukan,"
Jdg 12:6  mereka menyuruh dia mengucapkan kata "Syibolet". Ia akan mengatakan "Sibolet" karena tidak dapat mengucapkan kata itu dengan tepat. Lalu mereka akan menangkapnya dan membunuhnya di salah satu tempat penyeberangan itu. Maka terbunuhlah 42.000 orang Efraim pada waktu itu.
Jdg 12:7  Enam tahun lamanya Yefta memimpin orang Israel. Kemudian ia meninggal lalu dimakamkan di kampung halamannya yaitu di kota Mizpa di Gilead.
Jdg 12:8  Setelah Yefta, umat Israel dipimpin oleh Ebzan dari Betlehem.
Jdg 12:9  Ebzan mempunyai tiga puluh anak lelaki dan tiga puluh anak perempuan. Anak-anaknya yang perempuan itu dikawinkannya dengan orang-orang dari suku lain, dan ia mengambil tiga puluh gadis dari suku lain untuk anak-anaknya yang lelaki. Tujuh tahun lamanya Ebzan memimpin umat Israel,
Jdg 12:10  kemudian ia meninggal dan dimakamkan di Betlehem.
Jdg 12:11  Setelah Ebzan, umat Israel dipimpin oleh Elon dari suku Zebulon sepuluh tahun lamanya.
Jdg 12:12  Kemudian ia meninggal lalu dimakamkan di Ayalon di wilayah Zebulon.
Jdg 12:13  Setelah Elon, umat Israel dipimpin oleh Abdon anak Hilel dari Piraton.
Jdg 12:14  Anaknya yang lelaki ada empat puluh orang, dan cucu-cucunya yang lelaki ada tiga puluh orang. Mereka masing-masing memiliki seekor keledai tunggangan. Delapan tahun lamanya umat Israel dipimpin oleh Abdon.
Jdg 12:15  Kemudian ia meninggal lalu dimakamkan di Piraton di wilayah Efraim di daerah pegunungan orang Amalek.
Jdg 13:1  Orang Israel berdosa lagi kepada TUHAN, lalu TUHAN membiarkan orang Filistin menguasai mereka selama empat puluh tahun.
Jdg 13:2  Pada masa itu ada seorang laki-laki di kota Zora. Namanya Manoah, dari suku Dan. Istrinya mandul.
Jdg 13:3  Tetapi pada suatu hari malaikat TUHAN menampakkan diri kepadanya dan berkata, "Selama ini kau tidak dapat mempunyai anak, tetapi tidak lama lagi kau akan hamil, dan mendapat seorang anak laki-laki.
Jdg 13:4  Sebab itu jagalah dirimu baik-baik. Jangan minum anggur atau minuman keras, atau makan sesuatu yang haram.
Jdg 13:5  Setelah anakmu itu lahir, jangan sekali-kali memotong rambutnya, sebab sejak dalam kandungan ia sudah ditentukan untuk menjadi orang nazir. Dialah yang akan berjuang untuk membebaskan orang Israel dari kekuasaan orang Filistin."
Jdg 13:6  Setelah itu istri Manoah pergi kepada suaminya dan berkata, "Kanda, ada utusan Allah datang kepada saya; rupanya seperti malaikat Allah, sehingga saya ketakutan. Saya tidak menanyakan dari mana ia datang, dan ia pun tidak memberitahukan namanya kepada saya.
Jdg 13:7  Tetapi ia berkata bahwa saya akan hamil dan mendapat seorang anak laki-laki. Ia berkata juga bahwa saya tidak boleh minum anggur atau minuman keras, atau makan sesuatu yang haram, sebab anak itu harus diserahkan kepada Allah untuk menjadi seorang nazir seumur hidupnya."
Jdg 13:8  Mendengar itu, berdoalah Manoah kepada TUHAN, begini, "TUHAN, sudilah kiranya Engkau mengutus hamba-Mu itu lagi kepada kami untuk memberitahukan apa yang harus kami perbuat dengan anak itu setelah ia lahir."
Jdg 13:9  Allah mendengarkan permintaan Manoah. Malaikat TUHAN itu datang lagi ketika istri Manoah sedang duduk-duduk di ladang. Pada waktu itu suaminya tidak berada di situ.
Jdg 13:10  Jadi, ia lari memberitahukan hal itu kepadanya, "Kanda, lihatlah, orang yang dahulu datang kepada saya itu, datang lagi!"
Jdg 13:11  Manoah segera mengikuti istrinya. Ia mendekati orang itu dan bertanya, "Tuankah yang membawa berita kepada istri saya?" "Benar," jawabnya.
Jdg 13:12  Lalu Manoah bertanya, "Kalau apa yang Tuan katakan itu sudah terjadi nanti, apa yang harus dilakukan oleh anak itu? Bagaimana hidupnya nanti?"
Jdg 13:13  Jawab malaikat TUHAN itu, "Istrimu tidak boleh minum anggur atau minuman keras, juga tidak boleh makan sesuatu yang berasal dari pohon anggur, atau sesuatu yang haram. Ia harus melaksanakan semua yang telah kukatakan kepadanya, dan menjaga agar tidak ada yang tidak dilaksanakannya."
Jdg 13:15  Lalu kata Manoah kepada-Nya, "Sudilah Tuan menunggu sebentar. Kami akan memasakkan dahulu kambing muda untuk Tuan."
Jdg 13:16  Tetapi malaikat TUHAN itu berkata, "Sekalipun aku menunggu di sini, aku tidak akan makan apa yang kausajikan itu nanti. Tetapi, kalau engkau mau juga menyajikannya, sajikanlah itu sebagai kurban bakaran, dan persembahkanlah kepada TUHAN." Manoah belum menyadari bahwa yang berbicara dengan dia itu adalah malaikat TUHAN. Jadi ia berkata, "Kalau begitu, sudilah kiranya Tuan memberitahukan nama Tuan, supaya kalau apa yang Tuan katakan itu telah terjadi nanti, kami dapat memberi penghormatan kami kepada Tuan."
Jdg 13:18  Malaikat TUHAN itu bertanya, "Apa sebab kau ingin tahu namaku? Namaku itu nama yang ajaib."
Jdg 13:19  Maka Manoah mengambil seekor kambing muda dengan sedikit gandum, lalu menaruhnya di atas batu dan mempersembahkannya kepada TUHAN, yaitu TUHAN yang melakukan hal-hal yang ajaib.
Jdg 13:20  Sementara api menyala ke atas dari mezbah itu, Manoah dan istrinya melihat malaikat TUHAN itu naik ke atas dalam nyala api itu, menuju ke langit. Barulah Manoah menyadari bahwa itu malaikat TUHAN; maka ia dan istrinya pun sujud menyembah. Tidak pernah mereka melihat malaikat itu lagi.
Jdg 13:22  Lalu kata Manoah kepada istrinya, "Kita pasti akan mati sebab kita sudah melihat Allah!"
Jdg 13:23  Istrinya menjawab, "Kalau TUHAN memang mau membunuh kita, pasti Ia tidak mau menerima kurban yang kita persembahkan kepada-Nya itu; Ia juga tidak akan menunjukkan semuanya ini kepada kita atau memberitahukan hal-hal yang telah dikatakan-Nya itu kepada kita."
Jdg 13:24  Berbulan-bulan kemudian istri Manoah itu melahirkan seorang anak laki-laki lalu ia memberi nama Simson kepadanya. Anak itu menjadi besar serta diberkati TUHAN;
Jdg 13:25  dan ketika ia berada di perkemahan suku Dan, yang terletak di antara Zora dan Esytaol, Roh TUHAN mulai memberikan kekuatan kepadanya.
Jdg 14:1  Pada suatu hari Simson pergi ke Timna, dan melihat seorang gadis Filistin di sana.
Jdg 14:2  Lalu ia pulang dan berkata kepada orang tuanya, "Saya tertarik kepada seorang gadis Filistin di Timna. Saya mohon ayah dan ibu pergi meminang dia."
Jdg 14:3  Tetapi orang tuanya berkata, "Mengapa harus pergi kepada orang Filistin untuk mendapatkan istri, sedangkan mereka tidak tergolong umat TUHAN? Apakah di dalam kaum kita sendiri atau di antara seluruh bangsa kita tidak ada seorang gadis yang cocok?" Jawab Simson kepada ayahnya, "Tapi gadis Filistin itulah yang saya sukai. Dan saya harap ayah mau meminang dia untuk saya."
Jdg 14:4  Orang tua Simson tidak tahu bahwa Tuhanlah yang membuat Simson melakukan hal itu. Sebab, TUHAN sedang mencari kesempatan untuk memerangi orang Filistin. Pada waktu itu orang Filistin menguasai orang Israel.
Jdg 14:5  Maka pergilah Simson ke Timna bersama-sama dengan orang tuanya. Sementara mereka melalui sebidang kebun anggur, Simson bertemu dengan seekor singa muda. Singa itu mengaum,
Jdg 14:6  dan tiba-tiba Simson menjadi kuat oleh kuasa TUHAN. Lalu ia mencabik-cabik singa itu dengan tangannya, seolah-olah binatang itu hanya seekor kambing. Hal itu tidak diceritakannya kepada orang tuanya.
Jdg 14:7  Kemudian Simson mengunjungi gadis itu serta bercakap-cakap dengan dia, dan Simson suka kepadanya.
Jdg 14:8  Setelah beberapa waktu lamanya, Simson pergi lagi untuk kawin dengan gadis itu. Di dalam perjalanan, ia membelok untuk melihat bangkai singa yang telah dibunuhnya itu. Ketika sampai di situ ia melihat banyak sekali lebah di dalam bangkai itu, dan ada juga madu.
Jdg 14:9  Ia mengeruk madu itu ke dalam tangannya, lalu berjalan terus sambil makan madu itu. Kemudian ia pergi kepada orang tuanya serta memberikan juga sebagian dari madu itu kepada mereka, dan mereka memakannya. Tetapi Simson tidak memberitahukan bahwa madu itu diambilnya dari bangkai singa.
Jdg 14:10  Setelah ayah Simson pergi ke rumah gadis itu, Simson mengadakan pesta di sana, karena demikianlah kebiasaan orang-orang muda.
Jdg 14:11  Ketika orang Filistin melihat dia, mereka memilih tiga puluh pemuda untuk menemani dia.
Jdg 14:12  Lalu Simson berkata kepada mereka, "Saya punya teka-teki. Kalau kalian dapat menebaknya, kalian masing-masing akan saya beri sehelai kain lenan yang halus dan satu setel pakaian yang bagus. Kalian saya beri waktu tujuh hari selama pesta perkawinan ini, untuk menebaknya.
Jdg 14:13  Tetapi kalau kalian tidak dapat menebaknya, kalianlah masing-masing yang harus memberikan kepada saya kain lenan halus dan satu setel pakaian yang bagus." Lalu orang-orang Filistin itu berkata, "Boleh! Sekarang sebutkanlah teka-teki itu."
Jdg 14:14  Maka Simson berkata, "Dari yang makan, keluar yang dimakan; dari yang kuat, keluar yang manis." Setelah lewat tiga hari mereka belum juga dapat menebak teka-teki itu.
Jdg 14:15  Jadi, pada hari keempat berkatalah mereka kepada istri Simson, "Kau harus membujuk suamimu supaya ia mau memberitahukan kepada kami arti teka-teki itu. Kalau tidak, kami akan membakar engkau dengan seisi rumah ayahmu. Rupanya kau mengundang kami untuk menghabiskan harta kami!"
Jdg 14:16  Karena itu, istri Simson pergi kepada Simson sambil menangis, lalu berkata, "Saya tahu kau tidak mencintai saya. Mengapa kau memberikan teka-teki kepada kawan-kawan saya, tetapi kau tidak memberitahukan artinya kepada saya? Tentu kau membenci saya!" Kata Simson, "Tunggu dulu! Kepada orang tua saya sendiri pun saya tidak beritahukan, mengapa kau harus diberitahukan?"
Jdg 14:17  Selama tujuh hari pesta itu istri Simson menangis terus. Dan karena ia terus merengek-rengek, maka pada hari yang ketujuh, Simson memberitahukan arti teka-teki itu kepadanya. Lalu istri Simson itu memberitahukannya kepada orang-orang Filistin.
Jdg 14:18  Jadi, pada hari yang ketujuh sebelum matahari terbenam, orang-orang kota itu berkata kepada Simson, "Apakah yang lebih manis daripada madu? Dan apakah pula yang lebih kuat daripada singa?" Simson menjawab, "Kalau kalian tidak memperalat istriku, kalian tidak akan dapat menebak teka-teki itu!"
Jdg 14:19  Tiba-tiba Simson menjadi kuat oleh kuasa TUHAN, lalu ia pergi ke Askelon. Di sana ia membunuh tiga puluh orang laki-laki yang berpakaian bagus-bagus. Ia mengambil pakaian mereka itu dan memberikannya kepada ketiga puluh orang yang telah menebak teka-tekinya. Setelah itu, ia pulang ke rumahnya dengan hati yang kesal.
Jdg 14:20  Lalu ayah mertuanya memberikan istri Simson itu kepada orang yang menjadi pengiring Simson pada waktu pernikahannya.
Jdg 15:1  Beberapa waktu kemudian, ketika musim panen gandum, Simson pergi mengunjungi istrinya. Ia membawa seekor kambing muda untuk istrinya itu. Berkatalah Simson kepada ayah mertuanya, "Saya ingin ke kamar untuk bertemu dengan istri saya." Tetapi ayah mertuanya tidak mengizinkannya.
Jdg 15:2  Katanya kepada Simson, "Saya kira kau benar-benar membenci dia, jadi, ia sudah kuberikan kepada kawanmu. Tetapi kau boleh mengambil adiknya. Ia lebih cantik."
Jdg 15:3  Berkatalah Simson, "Apa yang akan kulakukan terhadap orang Filistin kali ini, bukanlah kesalahanku!"
Jdg 15:4  Lalu ia pergi menangkap tiga ratus serigala. Buntut serigala-serigala itu diikatnya berdua-dua. Pada simpul ikatan itu, ia menaruh sebuah obor.
Jdg 15:5  Setelah itu ia menyalakan obor-obor itu lalu melepaskan serigala-serigala itu ke dalam ladang-ladang orang Filistin. Terbakarlah semua gandum yang telah dituai di sana. Bahkan tanaman gandum dan kebun-kebun anggur pun ikut terbakar juga.
Jdg 15:6  Orang-orang Filistin pergi menyelidiki siapa yang melakukan hal itu. Lalu orang memberitahukan kepada mereka bahwa Simsonlah yang melakukannya, karena ayah mertuanya di Timna telah memberikan istri Simson kepada seorang kawan Simson. Karena itu orang-orang Filistin membakar wanita itu serta seisi rumah ayahnya sampai habis.
Jdg 15:7  Lalu kata Simson kepada orang-orang Filistin itu, "O, begitukah caranya kalian bertindak? Saya bersumpah tidak akan tinggal diam kalau saya belum membalas hal ini kepada kalian!"
Jdg 15:8  Kemudian ia menyerang mereka dengan kejam dan membunuh banyak di antara mereka. Setelah itu, ia pergi lalu tinggal di dalam gua di sebuah bukit batu di Etam.
Jdg 15:9  Maka orang Filistin datang ke daerah Yehuda lalu berkemah di situ, kemudian menyerbu kota Lehi.
Jdg 15:10  Orang Yehuda bertanya kepada orang Filistin, "Mengapa kalian menyerang kami?" Mereka menjawab, "Kami datang untuk menangkap Simson supaya kami dapat memperlakukan dia seperti ia memperlakukan kami."
Jdg 15:11  Mendengar itu, pergilah 3.000 orang Yehuda ke gua di bukit batu di Etam. Di sana mereka berkata, "Simson, apakah kau tidak tahu bahwa orang Filistin berkuasa atas kita? Apa ini yang telah kaulakukan terhadap kami!" Simson menyahut, "Saya hanya membalas apa yang mereka lakukan terhadap saya."
Jdg 15:12  Kata mereka, "Kami datang untuk mengikat dan menyerahkan engkau kepada orang Filistin." Simson berkata, "Berjanjilah dahulu bahwa kalian sendiri tidak akan bunuh saya."
Jdg 15:13  "Baik," kata mereka, "memang kami hanya mau mengikat dan menyerahkan engkau kepada orang Filistin. Kami tidak akan membunuhmu." Lalu mereka mengikat dia dengan dua buah tali yang masih baru, dan membawa dia pergi dari gua itu.
Jdg 15:14  Ketika Simson dibawa sampai ke Lehi, orang-orang Filistin berlari mendatangi dia sambil menyorakinya. Tiba-tiba ia menjadi kuat oleh kuasa TUHAN. Tali-tali yang mengikat tangannya itu diputuskannya begitu saja seolah-olah tali itu tali yang hangus.
Jdg 15:15  Kemudian ia menemukan sebuah tulang rahang yang masih baru dari seekor keledai. Ia memungut tulang itu, lalu memakainya untuk memukul 1.000 orang sampai mati.
Jdg 15:16  Maka bernyanyilah Simson, "Dengan rahang keledai, kuhajar seribu orang; dengan rahang keledai, kutumpuk mayat-mayat mereka."
Jdg 15:17  Setelah itu, ia membuang tulang rahang itu. Itu sebabnya tempat tersebut dinamakan Ramat-Lehi.
Jdg 15:18  Kemudian Simson merasa haus sekali, sehingga ia berseru kepada TUHAN. Ia berkata, "TUHAN, baru saja Engkau memberikan kemenangan yang besar ini kepada saya. Sekarang haruskah saya mati kehausan, sehingga ditangkap oleh orang-orang Filistin ini yang tidak mengenal Engkau?"
Jdg 15:19  Maka Allah membuka lubang di bukit batu di Lehi itu, lalu keluarlah air dari situ. Simson minum, kemudian merasa segar kembali. Itu sebabnya sumber air itu disebut Sumber Air Hakor. Sumber air ini masih ada di Lehi.
Jdg 15:20  Dua puluh tahun lamanya Simson memimpin umat Israel waktu mereka dikuasai orang Filistin.
Jdg 16:1  Pada suatu hari Simson pergi ke Gaza. Di sana ia bertemu dengan seorang pelacur lalu ia pergi tidur dengan wanita itu.
Jdg 16:2  Maka penduduk Gaza mendengar bahwa Simson ada di sana. Jadi, mereka mengepung tempat itu dan menunggu dia sepanjang malam di gerbang kota. Tapi mereka tidak berbuat apa-apa pada malam itu karena mereka berpikir, "Nanti apabila matahari terbit, baru kita membunuh dia."
Jdg 16:3  Tetapi Simson tidak tidur di situ sampai pagi. Tengah malam ia bangun lalu pergi ke pintu gerbang kota. Ia mencabut pintu itu seluruhnya dengan tiang-tiang serta palang-palangnya sekaligus. Kemudian ia mengangkat pintu itu di atas pundaknya, dan membawanya ke puncak bukit di seberang Hebron.
Jdg 16:4  Setelah itu Simson jatuh cinta kepada seorang wanita dari Lembah Sorek. Nama wanita itu Delila.
Jdg 16:5  Maka datanglah lima orang penguasa Filistin kepada wanita itu dan berkata, "Bujuklah Simson supaya ia mau memberitahukan kepadamu apa rahasia kekuatannya, dan bagaimana caranya untuk menghilangkan kekuatannya itu. Dengan demikian kami dapat datang dan mengikat dia untuk mengalahkan dia. Nanti kami masing-masing akan memberikan kepadamu 1.100 uang perak."
Jdg 16:6  Maka kata Delila kepada Simson, "Simson, apa sebab kau begitu kuat? Dengan apa engkau harus diikat supaya hilang kekuatanmu itu? Coba beritahukan rahasianya!"
Jdg 16:7  Simson menjawab, "Kalau saya diikat dengan tujuh tali busur baru yang belum kering betul, saya akan lemah dan menjadi seperti orang biasa."
Jdg 16:8  Lalu penguasa-penguasa Filistin itu memberikan kepada Delila tujuh tali busur baru yang belum kering betul. Kemudian, sementara orang Filistin menunggu di ruangan yang lain, Delila mengikat Simson dengan tali-tali itu lalu berseru, "Simson! Orang Filistin datang!" Langsung Simson memutuskan tali-tali busur itu. Tali-tali itu putus seperti benang yang putus kena api. Rahasia kekuatan Simson belum bocor.
Jdg 16:10  Lalu kata Delila kepada Simson, "Ah, kau bohong! Kau hanya mempermainkan saya. Ayolah, ceritakan kepada saya dengan apa kau dapat diikat!"
Jdg 16:11  Sahut Simson, "Kalau saya diikat dengan tali baru yang belum pernah dipakai, saya akan lemah dan menjadi seperti orang biasa."
Jdg 16:12  Sementara orang Filistin menunggu di ruangan lain, Delila mengambil beberapa tali yang baru, lalu mengikat Simson. Kemudian ia berteriak, "Simson! Orang Filistin datang!" Tetapi Simson memutuskan tali-tali itu dari tangannya seperti orang memutuskan benang.
Jdg 16:13  Lalu kata Delila kepada Simson, "Ah, kau mempermainkan saya lagi. Kau bohong. Ayolah ceritakan dengan apa kaudapat diikat!" Simson menjawab, "Kalau tujuh untaian rambut saya dijalin pada benang di perkakas tenun, kemudian diikat pada kayu perkakas tenun itu, saya akan lemah dan menjadi seperti orang biasa."
Jdg 16:14  Maka Delila membuat Simson tertidur lalu ia menjalin tujuh untaian rambut Simson pada benang di perkakas tenun. Setelah itu ia mengikatnya pada kayu perkakas tenun itu lalu berseru, "Simson! Orang Filistin datang!" Simson pun bangun lalu menarik rambutnya hingga lepas dari perkakas tenun itu.
Jdg 16:15  Kemudian kata Delila kepadanya, "Ah, kau katakan kaucinta kepada saya padahal hatimu jauh dari saya! Sudah tiga kali kau mempermainkan saya, dan sampai sekarang pun kau tak mau memberitahukan mengapa kau begitu kuat."
Jdg 16:16  Setiap hari Delila terus saja merengek-rengek dan mendesak, sehingga Simson bosan mendengarnya.
Jdg 16:17  Akhirnya Simson menceritakan rahasianya. "Rambut saya belum pernah dipotong," kata Simson. "Saya telah diserahkan kepada Allah menjadi miliknya yang khusus. Sejak lahir saya sudah menjadi seorang Nazir. Kalau rambut saya dipotong, kekuatan saya akan hilang; saya akan lemah dan menjadi seperti orang biasa."
Jdg 16:18  Delila merasa bahwa Simson telah memberitahukan rahasia hatinya. Sebab itu ia mengirim berita ini kepada penguasa-penguasa Filistin, "Datanglah lagi sekali ini saja. Simson telah menceritakan rahasianya kepadaku." Mereka pun datang dengan membawa uang untuk Delila.
Jdg 16:19  Sesudah itu Delila membelai Simson sampai tertidur di pangkuannya, lalu ia memanggil orang dan menyuruhnya memotong tujuh untaian rambut Simson. Kemudian ia menyiksa Simson, karena Simson sudah kehilangan kekuatannya.
Jdg 16:20  Lalu ia berteriak, "Simson! Orang Filistin datang!" Simson bangun dan menyangka ia dapat lolos seperti pada waktu-waktu yang lalu. Ia tidak tahu bahwa TUHAN telah meninggalkan dia.
Jdg 16:21  Maka orang-orang Filistin menangkap dia dan mencungkil biji matanya. Mereka membawa dia ke Gaza dan mengikat dia dengan dua rantai tembaga, lalu memasukkannya ke dalam penjara. Di sana ia disuruh bekerja menggiling.
Jdg 16:22  Tetapi selama itu pula rambutnya mulai tumbuh lagi.
Jdg 16:23  Pada suatu hari para penguasa Filistin berkumpul hendak berpesta dan mempersembahkan kurban secara besar-besaran untuk Dagon, dewa mereka. Mereka bernyanyi, "Dewa kita telah memberikan kepada kita kemenangan atas Simson, musuh kita!"
Jdg 16:24  Sementara mereka bersuka ria mereka berkata, "Bawalah Simson ke mari supaya ia menjadi bahan tertawaan kita!" Maka mereka mengambil Simson dari penjara, dan ia menjadi bahan tertawaan mereka. Mereka menyuruh dia berdiri di antara dua tiang yang besar. Ketika orang banyak itu melihat Simson, mereka memuji Dagon dan bernyanyi, "Dewa kita telah memberikan kepada kita kemenangan atas musuh yang telah merusak negeri kita dan membunuh begitu banyak orang kita!"
Jdg 16:26  Kata Simson kepada anak laki-laki yang menuntunnya, "Lepaskanlah tangan saya supaya saya dapat memegang tiang-tiang yang menyangga gedung ini, karena saya ingin bersandar di situ."
Jdg 16:27  Gedung itu penuh sesak dengan orang. Kelima penguasa Filistin pun ada di situ. Dan di tingkat atas gedung itu ada kira-kira 3.000 orang. Mereka semuanya menonton dan mengejek Simson.
Jdg 16:28  Pada saat itu Simson berdoa; ia berkata, "Ya, TUHAN Yang Mahatinggi, sudilah Engkau mengingat saya. Berikanlah kiranya kekuatan lagi kepada saya sekali ini saja, ya Allah, supaya dengan tindakan ini saya dapat membalas orang Filistin yang telah mencungkil kedua biji mata saya."
Jdg 16:29  Setelah itu Simson memegang kedua tiang di tengah yang menyangga gedung itu. Lalu sambil bertopang dengan tangan kirinya pada tiang yang satu dan tangan kanannya pada tiang yang lain,
Jdg 16:30  ia berseru, "Biar aku mati bersama orang-orang Filistin ini!" Dan dengan sekuat tenaganya ia mendorong tiang-tiang itu sampai roboh dan menimpa kelima penguasa Filistin bersama semua orang lainnya yang berada di situ. Simson meninggal juga. Tetapi, pada waktu kematiannya itu ia telah membunuh orang lebih banyak daripada ketika masa hidupnya.
Jdg 16:31  Kemudian datanglah sanak saudaranya dan semua kaum keluarganya untuk mengambil jenazahnya. Mereka membawa jenazahnya itu pulang, dan memakamkannya di dalam kuburan Manoah, ayahnya. Kuburan itu terletak di antara Zora dan Esytaol. Simson menjadi pemimpin Israel dua puluh tahun lamanya.
Jdg 17:1  Adalah seorang laki-laki bernama Mikha. Ia tinggal di daerah pegunungan di wilayah Efraim.
Jdg 17:2  Pada suatu hari berkatalah ia kepada ibunya, "Ibu, ketika ibu kehilangan seribu seratus uang perak, saya dengar ibu mengutuki pencurinya. Ini, Bu, uangnya! Sayalah yang mencurinya." Ibunya menjawab, "Semoga engkau diberkati TUHAN, nak!"
Jdg 17:3  Mikha mengembalikan uang itu kepada ibunya, lalu ibunya berkata, "Supaya engkau, anakku, terlepas dari kutukan yang telah saya ucapkan itu, saya mempersembahkan perak ini kepada TUHAN untuk dipakai membuat patung kayu yang dilapisi perak. Jadi uang itu saya kembalikan kepadamu."
Jdg 17:4  Maka setelah Mikha mengembalikan uang itu kepada ibunya, ibunya itu mengambil dua ratus keping dari uang itu, dan memberikannya kepada seorang tukang perak. Lalu ia menyuruh tukang itu membuat sebuah patung kayu yang dilapis dengan perak itu. Kemudian patung itu ditaruh di rumah Mikha.
Jdg 17:5  Mikha sendiri sudah mempunyai sebuah tempat penyembahan. Ia membuat beberapa patung berhala dan sehelai efod lalu mengangkat salah seorang anaknya yang laki-laki menjadi imam.
Jdg 17:6  Pada waktu itu bangsa Israel tidak mempunyai raja; setiap orang melakukan apa yang dianggapnya benar.
Jdg 17:7  Pada masa itu juga ada seorang pemuda dari suku Lewi. Dahulu ia tinggal di Betlehem di wilayah Yehuda.
Jdg 17:8  Tapi kini ia telah meninggalkan kota itu dan sedang berjalan mencari tempat tinggal yang lain. Di dalam perjalanannya itu ia tiba di rumah Mikha di daerah pegunungan wilayah Efraim.
Jdg 17:9  Mikha bertanya kepadanya, "Saudara orang mana?" Orang itu menjawab, "Saya orang Lewi dari Betlehem di Yehuda. Saya sedang mencari tempat tinggal."
Jdg 17:10  Mikha berkata, "Tinggallah saja dengan saya di sini, supaya kau menjadi penasihat dan imam kami. Nanti saya memberi kepadamu sepuluh uang perak setiap tahun, dan juga pakaian serta makanan."
Jdg 17:11  Pemuda Lewi itu setuju, lalu ia diangkat oleh Mikha menjadi imamnya. Ia tinggal di rumah Mikha serta diperlakukan seperti anak kandung Mikha sendiri.
Jdg 17:13  Maka kata Mikha, "Nah, sekarang saya yakin TUHAN akan berbuat baik kepada saya sebab sudah ada seorang Lewi yang menjadi imam bagi saya."
Jdg 18:1  Pada masa itu di Israel tidak ada raja. Suku Dan masih mencari-cari daerah untuk menjadi wilayah mereka, sebab mereka belum juga mendapat tanah bersama suku-suku lainnya di dalam bangsa Israel.
Jdg 18:2  Karena itu mereka memilih lima orang yang gagah dari antara semua kaum dalam suku mereka yang tinggal di Zora dan Esytaol, lalu memberi tugas kepada kelima orang itu untuk menyelidiki negeri itu. Maka berangkatlah orang-orang itu, dan tiba di daerah pegunungan di wilayah Efraim. Di sana mereka menginap di rumah Mikha.
Jdg 18:3  Ketika hendak memasuki rumah itu mereka mendengar pemuda Lewi itu berbicara, dan mereka mengenali logatnya. Jadi, mereka mendatangi dia dan bertanya, "Mengapa kau di sini? Apa urusanmu? Siapa yang membawamu?"
Jdg 18:4  Orang muda itu menjawab, "Saya di sini karena Mikha. Dialah yang mengangkat saya menjadi imamnya, dan memberi gaji kepada saya."
Jdg 18:5  Lalu kata mereka kepadanya, "Kalau begitu, tolong tanyakan kepada Allah apakah perjalanan kami ini akan berhasil atau tidak."
Jdg 18:6  Jawab imam itu, "Jangan khawatir. TUHAN telah merestui perjalanan kalian ini."
Jdg 18:7  Setelah itu berangkatlah kelima orang itu lalu pergi ke Lais. Di situ mereka memperhatikan penduduknya. Nyatalah bahwa cara hidup penduduk Lais itu seperti cara hidup bangsa Sidon. Walaupun mereka terpencil, dan jauh dari bangsa Sidon, mereka hidup aman dan tentram, tidak mempunyai musuh, makmur, serta tidak kurang sesuatu apa pun.
Jdg 18:8  Ketika kelima orang itu kembali kepada orang-orang sesukunya di Zora dan Esytaol, mereka diminta memberikan laporan.
Jdg 18:9  Mereka pun berkata, "Mari kita menyerang Lais. Kami sudah ke sana, dan melihat bahwa tanahnya baik, luas dan makmur, segalanya ada. Jangan tetap saja di sini dan tidak berbuat apa-apa. Cepatlah ke sana dan merebut negeri itu. Kalian akan mendapati bahwa penduduknya sama sekali tidak menyangka akan diserang. Kami yakin Allah telah memberikan negeri itu kepada kalian."
Jdg 18:11  Karena itu, berangkatlah dari Zora dan Esytaol 600 orang suku Dan. Dengan senjata lengkap,
Jdg 18:12  mereka pergi ke bagian barat Kiryat-Yearim di wilayah Yehuda lalu berkemah di situ. Itulah sebabnya tempat itu masih dinamakan Perkemahan Dan.
Jdg 18:13  Dari sana mereka berjalan terus ke daerah pegunungan Efraim sampai ke rumah Mikha.
Jdg 18:14  Kelima orang mata-mata yang dahulu diutus itu berkata kepada kawan-kawannya, "Tahukah kalian bahwa di salah satu rumah ini ada patung yang berlapis perak? Ada juga patung-patung lain dan efod. Coba pikirkan apa yang harus kita buat."
Jdg 18:15  Lalu mereka pergi ke rumah Mikha, tempat pemuda Lewi itu tinggal, dan menanyakan keadaannya.
Jdg 18:16  Pada waktu itu keenam ratus prajurit suku Dan, yang bersenjata lengkap itu, sedang berdiri di depan pintu gerbang.
Jdg 18:17  Kelima mata-mata itu langsung masuk ke dalam rumah, lalu mengambil patung kayu yang berlapis perak itu bersama dengan patung-patung lainnya, dan efod. Pemuda Lewi itu sedang berdiri dengan keenam ratus prajurit itu di pintu gerbang.
Jdg 18:18  Ketika kelima orang itu masuk ke rumah Mikha dan mengambil efod serta patung-patung berhala itu, pemuda Lewi itu bertanya, "Apa ini yang kalian lakukan?"
Jdg 18:19  Mereka menjawab, "Diam! Jangan bertanya. Ikut saja dengan kami, nanti engkau menjadi imam dan penasihat kami. Daripada menjadi imam untuk satu keluarga saja, lebih baik menjadi imam untuk satu suku!"
Jdg 18:20  Pemuda Lewi itu senang dengan saran itu. Jadi, ia mengambil efod dan patung-patung berhala itu lalu ikut dengan mereka.
Jdg 18:21  Kemudian rombongan itu meneruskan perjalanan mereka, didahului oleh anak-anak, ternak dan barang-barang mereka.
Jdg 18:22  Ketika mereka sudah jauh, Mikha memanggil tetangga-tetangganya lalu bersama-sama mengejar orang-orang suku Dan itu. Setelah dekat dengan mereka,
Jdg 18:23  Mikha dan orang-orangnya itu berteriak-teriak kepada mereka. Orang-orang suku Dan itu menoleh dan bertanya kepada Mikha, "Ada apa? Mengapa datang dengan begitu banyak orang?"
Jdg 18:24  Mikha menjawab, "Kamu melarikan patung-patung saya dan imamnya, sehingga saya tidak mempunyai apa-apa lagi, lalu kamu berkata, 'Ada apa!'"
Jdg 18:25  Kata orang-orang suku Dan itu, "Lebih baik tutup mulut; jangan sampai orang-orang ini menjadi marah lalu menyerang engkau. Nanti engkau dan seluruh keluargamu mati."
Jdg 18:26  Setelah mengatakan demikian, orang-orang suku Dan itu meneruskan perjalanan mereka. Mikha melihat bahwa mereka jauh lebih kuat dari dia, karena itu ia membalik lalu pulang.
Jdg 18:27  Sesudah orang-orang suku Dan itu melarikan imam itu serta barang-barang ibadat buatan Mikha itu, mereka pergi menyerang Lais, kota yang penduduknya hidup aman dan tentram. (Kota itu berada dalam satu lembah dengan kota Bet-Rehob). Orang-orang Dan itu membunuh seluruh penduduk Lais lalu membakar kota itu. Tidak ada yang datang menolong orang-orang Lais itu karena tempat mereka terpencil dan jauh dari Sidon. Setelah itu orang-orang Dan membangun kembali kota itu kemudian tinggal di sana.
Jdg 18:29  Nama kota itu mereka ganti menjadi Dan, menurut nama bapak leluhur mereka, yaitu Dan, anak Yakub.
Jdg 18:30  Patung berhala yang mereka bawa itu, mereka dirikan di kota itu lalu mengangkat Yonatan, anak Gersom, cucu Musa, menjadi imam mereka. Keturunan Gersom ini tetap menjadi imam suku Dan itu sampai masa umat Israel diangkut dan dibuang ke negeri lain.
Jdg 18:31  Patung berhala buatan Mikha itu tetap di Lais selama Kemah Tuhan berada di Silo.
Jdg 19:1  Pada waktu Israel belum mempunyai raja, ada seorang Lewi tinggal di pedalaman, di daerah pegunungan Efraim. Ia mengambil seorang wanita dari Betlehem Yehuda menjadi selirnya.
Jdg 19:2  Tetapi kemudian wanita itu marah kepadanya, sehingga ia kembali ke rumah ayahnya di Betlehem. Setelah wanita itu tinggal di sana empat bulan lamanya,
Jdg 19:3  orang Lewi itu pergi menyusul dia untuk membujuknya kembali kepadanya. Ia pergi dengan membawa seorang hamba dan dua ekor keledai. Setelah tiba di sana, wanita itu membawa dia kepada ayahnya. Ketika ayah wanita itu melihat dia, ia menyambutnya dengan gembira,
Jdg 19:4  dan mendesak supaya ia jangan pulang. Karena itu, ia tinggal selama tiga hari dan makan minum serta bermalam di situ bersama selirnya itu.
Jdg 19:5  Pada hari keempat, mereka bangun pagi-pagi sekali lalu bersiap-siap untuk berangkat. Tetapi ayah wanita itu berkata kepada orang Lewi itu, "Makan dulu supaya kuat. Kalian bisa berangkat kemudian."
Jdg 19:6  Maka duduklah orang Lewi itu lalu makan dan minum bersama-sama dengan ayah wanita itu. Tetapi ketika ia berdiri hendak berangkat, ayah wanita itu membujuknya supaya jangan pergi. "Bersenang-senanglah dulu," katanya, "dan bermalamlah di sini." Maka tinggallah orang Lewi itu di situ semalam lagi.
Jdg 19:8  Pada hari kelima, pagi-pagi sekali, ketika ia hendak berangkat, ayah wanita itu berkata lagi, "Ayo, makanlah dulu; sebentar baru berangkat." Maka mereka berdua makan bersama dan berlambat-lambat sampai matahari mulai terbenam.
Jdg 19:9  Kemudian, ketika orang Lewi itu bersama selirnya dan hambanya hendak berangkat, ayah itu berkata, "Hari sudah hampir malam, sebentar lagi gelap. Lebih baik menginap saja di sini dan bersenang-senang. Besok kalian boleh bangun pagi-pagi untuk berangkat dan pulang."
Jdg 19:10  Tetapi orang Lewi itu tidak mau bermalam lagi di situ. Ia dan selirnya berangkat, bersama-sama dengan hambanya dan kedua keledainya yang berpelana. Ketika hari hampir malam, mereka sampai di dekat Yebus (yaitu Yerusalem). Lalu hamba itu berkata kepada tuannya, "Tuan, marilah kita singgah dan bermalam di kota orang Yebus ini saja."
Jdg 19:12  Tetapi tuannya menjawab, "Tidak! Kita tidak akan berhenti di tempat yang penduduknya bukan orang Israel. Mari kita meneruskan perjalanan sedikit lagi dan nanti bermalam di Gibea atau Rama."
Jdg 19:14  Karena itu mereka berjalan terus dan tidak berhenti di Yebus. Matahari telah terbenam ketika mereka sampai di Gibea di wilayah suku Benyamin.
Jdg 19:15  Mereka singgah di sana untuk bermalam, lalu masuk ke kota dan pergi duduk di alun-alun. Tetapi tak ada seorang pun yang mengundang mereka ke rumahnya untuk bermalam.
Jdg 19:16  Sementara mereka duduk di alun-alun kota, datanglah seorang tua yang baru kembali dari pekerjaannya di ladang. Ia sebenarnya berasal dari daerah pegunungan Efraim tetapi sekarang tinggal di Gibea. (Penduduk lainnya di Gibea adalah orang Benyamin.)
Jdg 19:17  Ketika orang tua itu melihat orang Lewi itu di alun-alun, ia bertanya, "Saudara dari mana, dan mau ke mana?"
Jdg 19:18  Orang Lewi itu menjawab, "Kami baru datang dari Betlehem di Yehuda, dan sedang dalam perjalanan pulang ke rumah di pedalaman di pegunungan Efraim. Saya berasal dari sana. Tidak ada seorang pun dari penduduk kota ini mau menerima kami di rumahnya untuk bermalam.
Jdg 19:19  Kami tidak memerlukan apa-apa yang lain, bekal kami cukup. Ada jerami dan makanan untuk keledai kami; ada roti dan anggur untuk saya bersama istri saya ini dan hamba saya."
Jdg 19:20  Lalu kata orang tua itu, "Jangan khawatir, nanti saya tolong. Tidak usah bermalam di alun-alun ini."
Jdg 19:21  Kemudian ia membawa mereka ke rumahnya lalu memberi makanan kepada keledai-keledai mereka. Dan setelah tamu-tamunya itu mencuci kaki, mereka makan.
Jdg 19:22  Sementara mereka bersenang-senang, tiba-tiba rumah itu dikepung orang-orang bejat dari kota itu. Mereka menggedor-gedor pintu dan berkata kepada orang tua itu, "Serahkan kepada kami orang laki-laki yang kaubawa ke rumahmu. Kami mau pakai dia!"
Jdg 19:23  Orang tua itu keluar dan berkata kepada mereka, "Jangan, saudara-saudara! Saya mohon, janganlah melakukan hal yang jahat dan tak patut seperti itu, sebab dia tamu saya.
Jdg 19:24  Nanti saya berikan saja selirnya dan anak gadis saya kepada kalian. Bolehlah kalian memuaskan diri dengan mereka. Tapi, janganlah melakukan apa-apa yang jahat terhadap tamu saya itu!"
Jdg 19:25  Tetapi karena orang-orang bejat itu tetap berkeras, maka orang Lewi itu mendorong selirnya itu keluar untuk mereka. Lalu mereka memperkosa wanita itu tanpa henti-hentinya sampai pagi.
Jdg 19:26  Pada waktu subuh wanita itu kembali, tetapi di depan pintu rumah orang tua itu, tempat suaminya menginap, ia pingsan. Setelah matahari terbit ia masih berada di situ.
Jdg 19:27  Pagi itu, ketika suaminya bangun dan membuka pintu untuk berangkat, ia mendapati selirnya tergeletak di depan rumah itu dengan tangannya berpegang pada pintu.
Jdg 19:28  "Bangunlah," katanya kepadanya, "mari kita berangkat." Tetapi wanita itu tidak menjawab karena sudah mati. Lalu, ia mengangkat wanita itu dan menaruhnya di atas keledai, kemudian berangkat untuk pulang ke rumahnya.
Jdg 19:29  Sampai di rumah, ia masuk dan mengambil pisau kemudian memotong-motong mayat selirnya itu menjadi dua belas potong. Setelah itu, ia mengirim potongan-potongan itu kepada kedua belas suku bangsa Israel, masing-masing satu potong.
Jdg 19:30  Semua orang yang melihat itu berkata, "Belum pernah hal semacam ini terjadi, sejak umat Israel keluar dari Mesir sampai hari ini. Kita tidak boleh tinggal diam. Kita harus berunding untuk bertindak!"
Jdg 20:1  Maka datanglah seluruh umat Israel dari Dan di utara sampai Bersyeba di selatan, dan Gilead di timur, untuk memenuhi panggilan tersebut. Mereka berkumpul dan bersatu di Mizpa di hadapan TUHAN.
Jdg 20:2  Seluruhnya ada 400.000 prajurit dari pasukan berjalan kaki. Pemimpin-pemimpin dari setiap suku bangsa Israel hadir juga di sana.
Jdg 20:3  Sementara itu orang-orang suku Benyamin sudah mendengar bahwa semua orang Israel yang lain telah berkumpul di Mizpa. Maka bertanyalah orang-orang Israel itu kepada orang Lewi yang selirnya terbunuh itu, "Coba ceritakan kepada kami bagaimana kejahatan itu terjadi!" Orang Lewi itu menjawab, "Saya dengan selir saya singgah di Gibea di wilayah Benyamin dan bermalam di sana.
Jdg 20:5  Kemudian orang-orang Gibea datang pada waktu malam untuk mencari saya. Mereka mengepung rumah tempat saya menginap dengan maksud untuk membunuh saya, tetapi kemudian mereka memperkosa selir saya sampai mati.
Jdg 20:6  Maka saya mengambil mayatnya lalu memotong-motongnya, kemudian mengirimnya kepada kedua belas suku dalam bangsa kita masing-masing satu potong. Saya melakukan hal itu karena orang-orang Gibea itu telah melakukan sesuatu yang jahat dan memalukan di antara umat Israel.
Jdg 20:7  Kalian semua yang berada di sini adalah orang Israel. Jadi sekarang saya minta pertimbangan kalian!"
Jdg 20:8  Dengan serentak orang-orang yang berkumpul di tempat itu berdiri dan berkata, "Kita semua tidak ada yang akan pulang ke kemah atau ke rumah.
Jdg 20:9  Inilah yang akan kita lakukan: Kita membuang undi untuk menentukan orang-orang yang harus menyerang Gibea.
Jdg 20:10  Sepersepuluh dari semua orang laki-laki Israel harus menyediakan makanan untuk mereka yang bertempur. Yang lain pergi menyerang Gibea untuk menghukum mereka karena kejahatan yang telah mereka lakukan di antara umat Israel."
Jdg 20:11  Demikianlah umat Israel bersatu untuk menyerang Gibea.
Jdg 20:12  Setelah umat Israel mengutus orang-orang ke seluruh wilayah suku Benyamin untuk menyampaikan berita ini: "Kejahatan apa yang terjadi pada kalian?
Jdg 20:13  Serahkanlah orang-orang bejat dari Gibea itu kepada kami supaya kami bunuh mereka dan dengan demikian kami membasmi kejahatan itu dari tengah-tengah Israel." Tetapi orang Benyamin tidak menghiraukan perkataan orang-orang Israel yang lainnya itu.
Jdg 20:14  Malah dari semua kota di wilayah Benyamin itu orang-orang Benyamin datang ke Gibea untuk memerangi orang-orang Israel yang lainnya.
Jdg 20:15  Hari itu mereka mengumpulkan dari kota-kota mereka sebanyak 26.000 prajurit. Dan di samping itu pula penduduk Gibea sendiri pun telah mengumpulkan 700 orang terpilih. Semuanya orang kidal yang ahli dalam hal melontarkan batu dengan memakai umban. Kalau mereka mengumban, tidak pernah meleset sedikit pun.
Jdg 20:17  Pada pihak orang-orang Israel lainnya sudah terkumpul sebanyak 400.000 prajurit terlatih.
Jdg 20:18  Tentara umat Israel itu pergi ke tempat ibadat di Betel, dan bertanya kepada Allah, "Ya TUHAN, suku manakah dari antara kami yang harus pertama-tama pergi menyerang orang Benyamin?" TUHAN menjawab, "Suku Yehuda."
Jdg 20:19  Besok paginya berangkatlah mereka dan berkemah di dekat kota Gibea.
Jdg 20:20  Mereka menempatkan barisan perang mereka berhadapan dengan kota itu, lalu menyerang.
Jdg 20:21  Maka angkatan perang Benyamin keluar menyerang pula dan membunuh 22.000 tentara Israel pada hari itu juga.
Jdg 20:22  Karena kekalahan itu, pergilah umat Israel ke tempat ibadat lalu menangis kepada TUHAN di sana sampai malam. Mereka bertanya, "TUHAN, haruskah kami pergi lagi memerangi saudara-saudara kami orang-orang Benyamin itu?" TUHAN menjawab, "Ya, pergilah." Oleh karena itu angkatan perang Israel menjadi bersemangat lagi, sehingga mereka mengatur kembali pasukan mereka seperti hari pertama.
Jdg 20:24  Untuk kedua kalinya mereka pergi menyerang angkatan perang Benyamin itu.
Jdg 20:25  Dan untuk kedua kalinya pula orang-orang Benyamin itu keluar menyerang. Kali ini mereka membunuh 18.000 prajurit-prajurit terlatih dari angkatan perang Israel.
Jdg 20:26  Maka pergilah lagi umat Israel ke Betel dan menangis kepada TUHAN. Di sana mereka berpuasa sampai malam dan mempersembahkan kurban perdamaian dan kurban bakaran kepada TUHAN.
Jdg 20:27  Pada masa itu Peti Perjanjian Allah ada di Betel, dan Pinehas anak Eleazar, cucu Harun, bertugas sebagai imam. Maka bertanyalah umat Israel kepada TUHAN, "TUHAN, haruskah kami pergi lagi menyerang saudara-saudara kami orang-orang Benyamin itu, atau haruskah kami menghentikan pertempuran ini?" TUHAN menjawab, "Seranglah mereka! Besok Aku akan memberikan kemenangan kepada kalian."
Jdg 20:29  Mendengar jawaban TUHAN itu, orang-orang Israel menempatkan tentara mereka di tempat-tempat tersembunyi sekitar Gibea.
Jdg 20:30  Lalu pada hari yang ketiga mereka menyerang angkatan perang Benyamin dengan susunan pasukan menghadap Gibea seperti yang telah mereka lakukan pada hari-hari sebelumnya.
Jdg 20:31  Orang-orang Benyamin menyerbu juga, dan terpancing untuk keluar dari kota. Seperti yang sudah-sudah, mereka mulai membunuh orang-orang Israel di daerah terbuka di luar kota pada jalan yang menuju ke Betel dan pada jalan yang menuju ke Gibea. Tiga puluh orang Israel terbunuh.
Jdg 20:32  Karena itu orang-orang Benyamin menyangka mereka telah mengalahkan orang-orang Israel itu lagi seperti semula. Tetapi tentara Israel sudah merencanakan untuk mundur supaya memancing tentara Benyamin itu ke jalan-jalan raya, jauh dari kota.
Jdg 20:33  Ketika pasukan inti Israel mundur, dan menyusun barisan lagi di Baal-Tamar, orang-orang mereka yang ditempatkan di sekitar Gibea, segera keluar dari tempat-tempat persembunyian mereka di daerah luar kota.
Jdg 20:34  Mereka itu orang-orang terpilih dari seluruh Israel; jumlahnya 10.000 orang. Terjadilah pertempuran yang sengit ketika 10.000 orang itu menyerang Gibea. Orang-orang Benyamin sama sekali tidak menyangka bahwa mereka tak lama lagi akan kalah.
Jdg 20:35  TUHAN memberikan kepada orang Israel kemenangan atas orang Benyamin, sehingga pada hari itu orang Israel membunuh 25.100 orang tentara Benyamin.
Jdg 20:36  Barulah orang Benyamin menyadari bahwa mereka sudah kalah. Pasukan inti Israel dengan sengaja telah melarikan diri dari orang-orang Benyamin karena mereka mengandalkan anak buah mereka di tempat-tempat persembunyian di sekitar Gibea.
Jdg 20:37  Orang-orang yang bersembunyi itu cepat-cepat masuk menyerbu Gibea; mereka menyebar ke mana-mana di dalam kota dan membunuh semua orang di situ.
Jdg 20:38  Tetapi mereka sudah sepakat lebih dulu bahwa kalau ada asap tebal mengepul ke atas dari tengah-tengah kota,
Jdg 20:39  pasukan inti yang mundur harus berbalik dan menyerang musuh. Pada saat itu orang-orang Benyamin sudah membunuh 30 tentara Israel dan menyangka mereka telah mengalahkan orang-orang Israel lagi seperti semula.
Jdg 20:40  Tetapi tiba-tiba asap mulai mengepul ke atas dari tengah-tengah kota Gibea. Orang-orang Benyamin terkejut ketika mereka menoleh dan melihat seluruh kota mereka terbakar.
Jdg 20:41  Dan pada saat itu juga pasukan inti Israel balik menyerang, sehingga tentara Benyamin itu menjadi panik karena mereka menyadari bahwa mereka pasti akan kalah.
Jdg 20:42  Jadi, mereka lari dari orang-orang Israel menuju ke daerah padang gurun, tetapi mereka tidak dapat lolos. Mereka terkepung di antara pasukan inti Israel dan pasukan yang pada saat itu sedang datang dari jurusan kota. Orang Benyamin menderita kekalahan yang amat hebat.
Jdg 20:43  Orang-orang Israel mengepung dan mengejar mereka terus-menerus sampai ke suatu tempat di sebelah timur Gibea. Sambil mengejar, mereka membunuh
Jdg 20:44  18.000 prajurit terlatih dari angkatan perang Benyamin.
Jdg 20:45  Tentara Benyamin lainnya balik lalu lari lewat daerah padang gurun menuju ke gunung-gunung batu di Rimon. Lima ribu orang di antara mereka dibunuh di tengah jalan. Yang lainnya dikejar terus sampai ke Gideom. Dari orang-orang itu ada 2.000 yang dibunuh.
Jdg 20:46  Jadi, tentara Benyamin yang dibunuh hari itu ada 25.000 orang--semuanya prajurit yang gagah berani.
Jdg 20:47  Tetapi dari antara mereka yang lari lewat daerah padang gurun menuju ke gunung-gunung batu di Rimon ada 600 orang yang lolos. Mereka tinggal di Rimon empat bulan lamanya.
Jdg 20:48  Kemudian orang Israel kembali lagi lalu menyerang dan membunuh semua sisa-sisa orang Benyamin--pria, wanita, anak-anak dan hewan juga. Semua kota di daerah itu mereka bakar.
Jdg 21:1  Pada waktu orang Israel berkumpul di Mizpa, mereka telah berjanji kepada TUHAN bahwa tidak seorang pun dari mereka yang akan mengizinkan anak gadisnya menjadi istri orang Benyamin.
Jdg 21:2  Sekarang mereka pergi ke Betel dan duduk bersedih hati sampai malam di hadapan Tuhan. Mereka menangis dan berseru,
Jdg 21:3  "Ya TUHAN, Allah Israel, mengapa semua ini harus terjadi? Suku Benyamin, salah satu suku bangsa kami, hampir lenyap!"
Jdg 21:4  Keesokan harinya, pagi-pagi sekali, umat Israel mendirikan sebuah mezbah di situ, lalu mempersembahkan kepada TUHAN kurban bakaran dan kurban perdamaian.
Jdg 21:5  Kemudian mereka menyelidiki di antara suku-suku bangsa Israel apakah ada yang tidak pergi berkumpul menghadap TUHAN di Mizpa dahulu. (Pada waktu itu mereka sudah bersumpah bahwa orang yang tidak memenuhi panggilan ke Mizpa harus dihukum mati.)
Jdg 21:6  Umat Israel kasihan kepada saudara-saudara mereka suku Benyamin; mereka berkata, "Hari ini Israel kehilangan salah satu sukunya.
Jdg 21:7  Kita harus berbuat apa supaya orang Benyamin yang sisa ini bisa mempunyai istri, karena kita sudah bersumpah kepada TUHAN bahwa kita tidak akan mengizinkan anak-anak gadis kita menjadi istri orang Benyamin?"
Jdg 21:8  Maka mereka menyelidiki apakah ada satu golongan dari suku bangsa Israel yang tidak pergi berkumpul di Mizpa. Dan mereka mendapati bahwa dari orang-orang Yabes di Gilead, tidak ada seorang pun yang telah pergi ke Mizpa.
Jdg 21:9  Karena, ketika diperiksa daftar hadir, ternyata tidak terdapat di situ seorang pun dari Yabes.
Jdg 21:10  Oleh sebab itu, orang-orang yang berkumpul di Betel itu mengirim 12.000 orang yang paling berani di antara mereka untuk membunuh setiap orang di Yabes, termasuk wanita dan anak-anak.
Jdg 21:11  Mereka diperintahkan untuk membunuh semua orang laki-laki, dan juga setiap wanita yang bukan perawan lagi.
Jdg 21:12  Maka di antara orang-orang yang tinggal di Yabes itu, mereka mendapati 400 gadis yang masih perawan. Mereka membawa gadis-gadis itu ke perkemahan di Silo yang terletak di negeri Kanaan.
Jdg 21:13  Kemudian seluruh umat Israel mengirim utusan kepada orang-orang Benyamin yang berada di gunung-gunung batu di Rimon untuk menawarkan perdamaian.
Jdg 21:14  Oleh sebab itu orang-orang Benyamin itu kembali, lalu umat Israel memberikan kepada mereka gadis-gadis dari Yabes yang tidak dibunuh itu. Tetapi jumlah gadis-gadis itu tidak cukup untuk orang-orang Benyamin itu.
Jdg 21:15  Umat Israel merasa kasihan kepada orang Benyamin, karena TUHAN sudah meretakkan kesatuan suku-suku bangsa Israel.
Jdg 21:16  Sebab itu, tokoh-tokoh yang berkumpul di Betel itu berkata, "Di dalam suku Benyamin tidak ada lagi wanita. Tetapi, apa boleh buat; kita tidak bisa mengizinkan anak-anak gadis kita kawin dengan mereka, karena kita sudah menyumpahi setiap orang dari antara kita yang mengizinkan anak gadisnya kawin dengan seorang Benyamin! Jadi, apakah yang harus kita lakukan untuk mereka yang sisa ini supaya mereka bisa mempunyai istri juga? Jangan sampai Israel kehilangan salah satu dari kedua belas sukunya. Kita harus mencari jalan supaya suku Benyamin dapat mempunyai keturunan seterusnya dan tetap memiliki seluruh tanahnya."
Jdg 21:19  "Begini," kata mereka, "ada satu jalan keluar, yaitu pada waktu perayaan tahunan untuk TUHAN di Silo." (Di sebelah selatan Silo ini terdapat kota Betel, dan di sebelah baratnya terdapat jalan raya Betel ke Sikhem, sedangkan di sebelah utaranya terdapat kota Lebona.)
Jdg 21:20  Maka tokoh-tokoh pertemuan di Betel itu berkata kepada orang-orang Benyamin yang sisa itu, "Pergilah kalian bersembunyi di kebun-kebun anggur di Silo,
Jdg 21:21  dan berjaga-jagalah di situ. Nanti kalau gadis-gadis Silo keluar untuk menari selama perayaan tahunan untuk TUHAN, kalian harus keluar dari kebun-kebun anggur itu. Lalu masing-masing harus menangkap seorang gadis dari antara mereka dan melarikannya ke wilayah Benyamin untuk dijadikan istrimu.
Jdg 21:22  Kalau ayah atau saudaranya yang laki-laki datang dan menuntut dia, kalian dapat mengatakan, 'Kami mohon, biarkanlah mereka pada kami. Kami tidak merampas mereka dalam pertempuran. Dan karena kalian tidak memberikan mereka kepada kami, kalian tidak melanggar janji, jadi tidak bersalah.'"
Jdg 21:23  Maka orang-orang Benyamin itu melakukan apa yang dikatakan kepada mereka. Mereka masing-masing melarikan seorang gadis dari antara gadis-gadis yang menari di Silo. Kemudian mereka kembali ke wilayah mereka sendiri dan membangun kembali kota-kotanya lalu menetap di sana.
Jdg 21:24  Orang-orang Israel yang lainnya itu pun berangkat pula. Mereka kembali ke tanah milik mereka di dalam wilayah mereka masing-masing, menurut kaum dan suku-sukunya.
Jdg 21:25  Pada zaman itu belum ada raja di Israel. Setiap orang melakukan apa yang dianggapnya benar.


\end{document}