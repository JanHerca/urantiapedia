\begin{document}

\title{Matius}


\chapter{1}

\par 1 Inilah daftar nenek moyang Yesus Kristus, keturunan Daud, keturunan Abraham. Dari Abraham sampai Daud, nama-nama nenek moyang Yesus sebagai berikut:
\par 2 Abraham, Ishak, Yakub, Yehuda dan saudara-saudaranya, Peres dan Zerah (ibu mereka bernama Tamar), Hezron, Ram, Aminadab, Nahason, Salmon, Boas (ibunya adalah Rahab), Obed (ibunya ialah Rut), Isai,
\par 3 [1:2]
\par 4 [1:2]
\par 5 [1:2]
\par 6 dan Raja Daud. Dari Daud sampai pada masa bangsa Israel dibuang ke Babel tercatat nama-nama berikut ini: Salomo (ibunya adalah bekas istri Uria),
\par 7 Rehabeam, Abia, Asa, Yosafat, Yoram, Uzia, Yotam, Ahas, Hizkia, Manasye, Amon, Yosia, Yekhonya dan saudara-saudaranya.
\par 8 [1:7]
\par 9 [1:7]
\par 10 [1:7]
\par 11 [1:7]
\par 12 Dari masa bangsa Israel dibuang ke Babel sampai kelahiran Yesus tercatat nama-nama berikut ini: Yekhonya, Sealtiel, Zerubabel, Abihud, Elyakim, Azur, Zadok, Akhim, Eliud, Eleazar, Matan, Yakub, Yusuf suami Maria. Dan dari Maria itulah lahir Yesus yang disebut Kristus.
\par 13 [1:12]
\par 14 [1:12]
\par 15 [1:12]
\par 16 [1:12]
\par 17 Jadi dari Abraham sampai Daud, semuanya ada empat belas generasi. Dari Daud sampai masa bangsa Israel dibuang ke Babel ada empat belas generasi juga. Dari masa bangsa Israel dibuang ke Babel sampai kelahiran Kristus ada pula empat belas generasi.
\par 18 Beginilah kisah tentang kelahiran Yesus Kristus. Ibu-Nya yaitu Maria, bertunangan dengan Yusuf. Tetapi sebelum mereka menikah, ternyata Maria sudah mengandung. Yusuf tidak tahu bahwa Maria mengandung karena kuasa Roh Allah.
\par 19 Yusuf, tunangannya itu, adalah seorang yang selalu mentaati hukum agama. Jadi ia mau memutuskan pertunangannya, tetapi dengan diam-diam, supaya Maria tidak mendapat malu di muka umum.
\par 20 Sementara Yusuf menimbang-nimbang hal itu, ia bermimpi. Dalam mimpinya itu, ia melihat seorang malaikat Tuhan yang berkata kepadanya, "Yusuf, keturunan Daud, jangan takut menikah dengan Maria; sebab anak yang di dalam kandungannya itu terjadi oleh kuasa Roh Allah.
\par 21 Maria akan melahirkan seorang anak laki-laki. Anak itu harus engkau beri nama Yesus, karena Ia akan menyelamatkan umat-Nya dari dosa mereka."
\par 22 Semuanya itu terjadi demikian supaya terlaksana apa yang dikatakan Tuhan melalui nabi-Nya, yaitu,
\par 23 "Seorang perawan akan mengandung dan melahirkan seorang anak laki-laki. Anak itu akan dinamakan Imanuel." (Imanuel adalah kata Ibrani yang berarti, "Allah ada bersama kita".)
\par 24 Sesudah Yusuf bangun, ia melakukan apa yang dikatakan malaikat Tuhan itu kepadanya. Ia menikah dengan Maria.
\par 25 Tetapi selama Maria belum melahirkan anaknya itu, Yusuf tidak bercampur dengan Maria. Dan sesudah anak itu lahir, Yusuf menamakan-Nya Yesus.

\chapter{2}

\par 1 Yesus dilahirkan di kota Betlehem di negeri Yudea pada masa pemerintahan Raja Herodes. Pada waktu itu beberapa ahli ilmu bintang dari Timur datang ke Yerusalem.
\par 2 Mereka bertanya di mana-mana, "Di manakah Anak itu, yang lahir untuk menjadi raja orang Yahudi? Kami melihat bintang-Nya terbit di sebelah timur, dan kami datang untuk menyembah Dia."
\par 3 Ketika Raja Herodes mendengar hal itu, ia terkejut sekali, begitu juga semua orang di Yerusalem.
\par 4 Maka ia menyuruh semua imam kepala dan guru-guru agama bangsa Yahudi datang berkumpul. Lalu ia bertanya kepada mereka, "Di manakah akan lahir Raja yang dijanjikan Allah?"
\par 5 Mereka menjawab, "Di kota Betlehem di negeri Yudea. Sebab beginilah ditulis oleh seorang nabi,
\par 6 'Engkau Betlehem, di negeri Yehuda, engkau sekali-kali bukanlah yang terkecil di antara kota-kota utama di Yehuda. Karena dari engkau akan datang seorang pemimpin yang akan memimpin umat-Ku Israel.'"
\par 7 Sesudah mendapat keterangan itu, Herodes memanggil ahli-ahli bintang dari Timur itu secara diam-diam. Lalu ia bertanya kepada mereka kapan tepatnya bintang itu mulai kelihatan.
\par 8 Sesudah itu ia menyuruh mereka ke Betlehem dengan pesan ini, "Pergilah, carilah Anak itu dengan teliti. Dan kalau kalian menemukan Dia, beritahukanlah kepadaku, supaya aku juga pergi menyembah Dia."
\par 9 Lalu pergilah mereka. Mereka melihat lagi bintang yang mereka lihat dahulu di sebelah timur. Alangkah gembiranya mereka melihat bintang itu! Bintang itu mendahului mereka, lalu berhenti tepat di atas tempat Anak itu.
\par 10 [2:9]
\par 11 Mereka masuk ke dalam rumah itu dan melihat Anak itu dengan Maria, ibu-Nya. Mereka sujud dan menyembah Anak itu, lalu membuka tempat harta mereka, dan mempersembahkan kepada-Nya emas, kemenyan, dan mur.
\par 12 Allah memperingatkan mereka di dalam mimpi supaya jangan kembali kepada Herodes. Jadi mereka pulang melalui jalan yang lain.
\par 13 Setelah ahli-ahli bintang itu berangkat, malaikat Tuhan menampakkan diri kepada Yusuf di dalam mimpi. Malaikat itu berkata, "Herodes bermaksud mencari Anak itu untuk membunuh Dia. Karena itu bangunlah, bawalah Anak itu dengan ibu-Nya mengungsi ke Mesir. Tinggallah di sana sampai Aku berbicara lagi kepadamu."
\par 14 Yusuf bangun, dan malam itu juga ia membawa Anak itu dengan ibu-Nya mengungsi ke Mesir.
\par 15 Mereka tinggal di sana sampai Herodes meninggal. Demikian terjadilah apa yang dikatakan Tuhan melalui nabi-Nya, begini, "Aku memanggil Anak-Ku dari Mesir."
\par 16 Ketika Herodes tahu bahwa ahli-ahli bintang dari Timur itu menipunya, ia marah sekali. Lalu ia memerintahkan untuk membunuh semua anak laki-laki yang berumur dua tahun ke bawah di Betlehem dan sekitarnya. Itu cocok dengan keterangan yang didapatnya dari ahli-ahli bintang itu tentang saatnya bintang itu kelihatan.
\par 17 Demikianlah terjadi apa yang pernah dikatakan oleh Nabi Yeremia, yaitu,
\par 18 "Di Rama terdengar suara ratapan, keluhan serta tangisan. Rahel meratapi anak-anaknya; ia tak mau dihibur sebab mereka sudah tiada."
\par 19 Sesudah Herodes meninggal, ketika Yusuf masih di Mesir seorang malaikat Tuhan menampakkan diri lagi kepada Yusuf di dalam mimpi.
\par 20 Malaikat itu berkata, "Orang-orang yang mau membunuh Anak itu sudah meninggal. Karena itu bangunlah, ambil Anak itu dengan ibu-Nya dan kembalilah ke Israel."
\par 21 Yusuf pun bangun, lalu membawa Anak itu dengan Maria kembali ke Israel.
\par 22 Tetapi kemudian Yusuf mendengar bahwa Arkelaus, putra Herodes sudah menggantikan ayahnya menjadi raja negeri Yudea. Jadi Yusuf takut pergi ke sana. Sesudah ia mendapat petunjuk Tuhan lebih lanjut dalam mimpi, ia pergi ke daerah Galilea.
\par 23 Di situ ia tinggal di kota yang bernama Nazaret. Dengan demikian terjadilah apa yang dikatakan oleh nabi-nabi mengenai Anak itu: "Ia akan disebut Orang Nazaret."

\chapter{3}

\par 1 Pada waktu itu datanglah Yohanes Pembaptis di padang pasir di Yudea dan mulai berkhotbah.
\par 2 "Bertobatlah dari dosa-dosamu," katanya, "karena Allah akan segera memerintah sebagai Raja!"
\par 3 Yohanes inilah orang yang dimaksudkan oleh Nabi Yesaya dalam kata-katanya ini, "Ada orang berseru-seru di padang pasir, 'Siapkanlah jalan untuk Tuhan; ratakanlah jalan bagi Dia.'"
\par 4 Yohanes memakai pakaian dari bulu unta. Ikat pinggangnya dari kulit, dan makanannya belalang dan madu hutan.
\par 5 Banyak orang dari Yerusalem, dari seluruh negeri Yudea dan dari daerah-daerah di sekitar Sungai Yordan datang kepada Yohanes.
\par 6 Mereka mengaku dosa-dosa mereka, dan Yohanes membaptis mereka di Sungai Yordan.
\par 7 Banyak juga orang Farisi dan Saduki datang kepada Yohanes untuk dibaptis. Tetapi waktu ia melihat mereka datang, ia berkata kepada mereka, "Kamu orang jahat! Siapa yang mengatakan bahwa kamu dapat luput dari hukuman Allah yang akan datang?
\par 8 Tunjukkanlah dengan perbuatanmu bahwa kamu sudah bertobat dari dosa-dosamu.
\par 9 Jangan sangka kamu dapat lolos dari hukuman dengan berkata bahwa Abraham adalah nenek moyangmu. Ingat, dari batu-batu ini pun, Allah sanggup membuat keturunan untuk Abraham!
\par 10 Kapak sudah siap untuk menebang pohon sampai ke akar-akarnya. Setiap pohon yang tidak menghasilkan buah yang baik akan ditebang dan dibuang ke dalam api.
\par 11 Saya membaptis kamu dengan air untuk menyatakan bahwa kamu sudah bertobat dari dosa-dosamu; tetapi yang akan datang sesudah saya, akan membaptis kamu dengan Roh Allah dan api. Ia lebih besar daripada saya. Untuk membawa sepatu-Nya pun saya tidak layak.
\par 12 Di tangan-Nya ada nyiru untuk menampi semua gandum-Nya sampai bersih. Gandum akan dikumpulkan-Nya di dalam lumbung, tetapi semua sekam akan dibakar-Nya di dalam api yang tidak bisa padam."
\par 13 Pada waktu itu Yesus pergi dari Galilea ke Sungai Yordan. Di sana Ia datang pada Yohanes dan minta dibaptis.
\par 14 Tetapi Yohanes mencoba menolak permintaan-Nya itu. Yohanes berkata, "Sayalah yang seharusnya dibaptis oleh Bapak. Sekarang malah Bapak yang datang kepada saya."
\par 15 Tetapi Yesus menjawab, "Biar saja untuk saat ini. Sebab dengan demikian kita melakukan semua yang dikehendaki Allah." Jadi Yohanes membaptis Yesus.
\par 16 Sesudah dibaptis, Yesus segera keluar dari air di sungai itu. Tiba-tiba langit terbuka dan Yesus melihat Roh Allah turun seperti burung merpati ke atas-Nya.
\par 17 Kemudian terdengar suara Allah mengatakan, "Inilah Anak-Ku yang Kukasihi. Ia menyenangkan hati-Ku."

\chapter{4}

\par 1 Kemudian Yesus dibimbing oleh Roh Allah ke padang gurun untuk dicobai oleh Iblis.
\par 2 Empat puluh hari empat puluh malam Yesus tidak makan. Lalu Ia merasa lapar.
\par 3 Iblis datang dan berkata, "Engkau Anak Allah, bukan? Nah, suruhlah batu-batu ini menjadi roti."
\par 4 Yesus menjawab, "Di dalam Alkitab tertulis: Manusia tidak dapat hidup dari roti saja, tetapi juga dari setiap perkataan yang diucapkan oleh Allah."
\par 5 Sesudah itu Iblis membawa Yesus ke Yerusalem, kota suci, dan menaruh Dia di atas puncak Rumah Tuhan.
\par 6 Lalu Iblis berkata kepada-Nya, "Engkau Anak Allah, bukan? Kalau begitu, terjunlah ke bawah; sebab di dalam Alkitab ada tertulis begini, 'Allah akan menyuruh malaikat-malaikat-Nya melindungi Engkau, mereka akan menyambut Engkau dengan tangan mereka, supaya kaki-Mu pun tidak tersentuh pada batu.'"
\par 7 Yesus menjawab, "Tetapi di dalam Alkitab tertulis juga, 'Jangan engkau mencobai Tuhan Allahmu.'"
\par 8 Kemudian Iblis membawa Yesus lagi ke gunung yang tinggi sekali dan menunjukkan kepada-Nya semua kerajaan di dunia dengan segala kekayaannya.
\par 9 Lalu Iblis berkata kepada Yesus, "Semua ini akan saya berikan kepada-Mu, kalau Engkau sujud menyembah saya."
\par 10 Yesus menjawab, "Pergi kau, hai Penggoda! Dalam Alkitab tertulis: Sembahlah Tuhan, Allahmu, dan layanilah Dia saja!"
\par 11 Akhirnya Iblis meninggalkan Yesus, dan malaikat-malaikat pun datang melayani Dia.
\par 12 Ketika mendengar bahwa Yohanes dimasukkan ke dalam penjara, Yesus menyingkir ke Galilea.
\par 13 Ia meninggalkan Nazaret, lalu tinggal di Kapernaum di pinggir Danau Galilea di daerah Zebulon dan Naftali.
\par 14 Karena Yesus melakukan hal itu, terjadilah apa yang dikatakan oleh Nabi Yesaya:
\par 15 "Tanah Zebulon dan tanah Naftali, di jalan ke danau, seberang Sungai Yordan, Galilea tanah orang bukan Yahudi!
\par 16 Bangsa yang hidup di dalam kegelapan telah melihat terang yang cemerlang! Bagi yang diam dalam negeri kegelapan maut telah terbit cahaya terang!"
\par 17 Sejak waktu itu Yesus mulai mewartakan, "Bertobatlah dari dosa-dosamu, karena Allah akan segera memerintah sebagai Raja!"
\par 18 Ketika Yesus sedang berjalan di pantai Danau Galilea, Ia melihat dua nelayan, yaitu Simon (yang dinamai juga Petrus) dengan adiknya, Andreas. Mereka sedang menangkap ikan di danau itu dengan jala.
\par 19 Yesus berkata kepada mereka, "Mari ikutlah Aku. Aku akan mengajar kalian menjala orang."
\par 20 Langsung mereka meninggalkan jala mereka lalu mengikuti Yesus.
\par 21 Yesus berjalan terus, lalu melihat pula dua orang bersaudara yang lain, yaitu Yakobus dan Yohanes, anak-anak Zebedeus. Mereka bersama-sama dengan ayah mereka sedang memperbaiki jala di dalam perahu. Yesus memanggil mereka juga,
\par 22 dan mereka langsung meninggalkan perahu dan ayah mereka, lalu mengikuti Yesus.
\par 23 Di Galilea, Yesus berkeliling di seluruh negeri dan mengajar di rumah-rumah ibadat. Ia memberitakan Kabar Baik bahwa Allah akan memerintah. Dan Ia juga menyembuhkan orang-orang yang sakit dan cacat.
\par 24 Kabar tentang Yesus itu tersebar di seluruh negeri Siria, sehingga banyak orang datang kepada-Nya. Mereka membawa orang-orang yang menderita segala macam penyakit dan kesusahan. Orang-orang yang kemasukan roh jahat, yang sakit ayan, dan yang lumpuh, semuanya disembuhkan oleh Yesus.
\par 25 Banyak sekali orang yang mengikuti Yesus pada waktu itu. Ada yang datang dari Galilea, ada yang dari Sepuluh Kota, dari Yerusalem, dari Yudea, dan ada pula yang dari negeri di seberang Yordan.

\chapter{5}

\par 1 Waktu Yesus melihat orang banyak itu, Ia naik ke atas bukit. Sesudah Ia duduk, pengikut-pengikut-Nya datang kepada-Nya,
\par 2 lalu Ia mulai mengajar mereka:
\par 3 "Berbahagialah orang yang merasa tidak berdaya dan hanya bergantung pada Tuhan saja; mereka adalah anggota umat Allah!
\par 4 Berbahagialah orang yang bersedih hati; Allah akan menghibur mereka!
\par 5 Berbahagialah orang yang rendah hati; Allah akan memenuhi janji-Nya kepada mereka!
\par 6 Berbahagialah orang yang rindu melakukan kehendak Allah; Allah akan memuaskan mereka!
\par 7 Berbahagialah orang yang mengasihani orang lain; Allah akan mengasihani mereka juga!
\par 8 Berbahagialah orang yang murni hatinya; mereka akan mengenal Allah.
\par 9 Berbahagialah orang yang membawa damai di antara manusia; Allah akan mengaku mereka sebagai anak-anak-Nya!
\par 10 Berbahagialah orang yang menderita penganiayaan karena melakukan kehendak Allah; mereka adalah anggota umat Allah!
\par 11 Berbahagialah kalian kalau dicela, dianiaya, dan difitnah demi Aku.
\par 12 Nabi-nabi yang hidup sebelum kalian pun sudah dianiaya seperti itu. Bersukacitalah dan bergembiralah, sebab besarlah upah di surga yang disediakan Tuhan untuk kalian."
\par 13 "Kalian adalah garam dunia. Kalau garam menjadi tawar, mungkinkah diasinkan kembali? Tidak ada gunanya lagi, melainkan dibuang dan diinjak-injak orang.
\par 14 Kalian adalah terang dunia. Kota yang terletak di atas bukit tidak dapat disembunyikan.
\par 15 Tidak ada orang yang menyalakan lampu, lalu menutup lampu itu dengan tempayan. Ia malah akan menaruh lampu itu pada tempat lampu, supaya memberi terang kepada setiap orang di dalam rumah.
\par 16 Begitu juga terangmu harus bersinar di hadapan orang, supaya mereka melihat perbuatan-perbuatanmu yang baik, lalu memuji Bapamu di surga."
\par 17 "Janganlah menganggap bahwa Aku datang untuk menghapuskan hukum Musa dan ajaran nabi-nabi. Aku datang bukan untuk menghapuskannya, tetapi untuk menunjukkan arti yang sesungguhnya.
\par 18 Ingatlah! Selama langit dan bumi masih ada, satu huruf atau titik yang terkecil pun di dalam hukum itu, tidak akan dihapuskan, kalau semuanya belum terjadi!
\par 19 Oleh karena itu, barangsiapa melanggar salah satu dari perintah-perintah itu, sekalipun yang terkecil, dan mengajar orang lain berbuat begitu juga, akan menjadi yang paling kecil di antara umat Allah. Sebaliknya, barangsiapa menjalankan perintah-perintah itu dan mengajar orang lain berbuat begitu juga, akan menjadi besar di antara umat Allah.
\par 20 Jadi, ingatlah: Kalian tidak mungkin menjadi umat Allah, kalau tidak melebihi guru-guru agama dan orang-orang Farisi dalam hal melakukan kehendak Allah!"
\par 21 "Kalian tahu bahwa pada nenek moyang kita terdapat ajaran seperti ini: Jangan membunuh; barangsiapa membunuh, harus diadili.
\par 22 Tetapi sekarang Aku berkata kepadamu, barangsiapa marah kepada orang lain, akan diadili; dan barangsiapa memaki orang lain, akan diadili di hadapan Mahkamah Agama. Dan barangsiapa mengatakan kepada orang lain, 'Tolol,' patut dibuang ke dalam api neraka.
\par 23 Oleh sebab itu, kalau salah seorang di antara kalian sedang mempersembahkan pemberiannya kepada Allah, lalu teringat bahwa ada orang yang sakit hati terhadapnya,
\par 24 hendaklah ia meninggalkan dahulu persembahannya itu di depan mezbah, lalu pergi berdamai dengan orang itu. Sesudah itu, dapatlah ia kembali dan mempersembahkan pemberiannya kepada Allah.
\par 25 Seandainya ada orang mengadukan kalian ke mahkamah, berdamailah dengan dia selama masih ada waktu sebelum sampai di mahkamah. Kalau tidak, orang itu akan menyerahkan kalian kepada hakim, yang akan menyerahkan kalian kepada polisi. Lalu polisi akan memasukkan kalian ke dalam penjara.
\par 26 Dan ingatlah: Pasti kalian tidak akan bisa keluar dari penjara itu, sebelum seluruh dendamu lunas sama sekali."
\par 27 "Kalian tahu bahwa ada ajaran seperti ini: Jangan berzinah.
\par 28 Tetapi sekarang Aku berkata kepadamu: barangsiapa memandang seorang wanita dengan nafsu berahi, orang itu sudah berzinah dengan wanita itu di dalam hatinya.
\par 29 Kalau mata kananmu menyebabkan engkau berdosa, cungkillah dan buanglah mata itu! Lebih baik kehilangan salah satu anggota badanmu daripada seluruh badanmu dibuang ke dalam neraka.
\par 30 Kalau tangan kananmu menyebabkan engkau berdosa, potong dan buanglah tangan itu! Lebih baik kehilangan sebelah tanganmu daripada seluruh badanmu masuk ke neraka."
\par 31 "Ada juga ajaran seperti ini: setiap orang yang menceraikan istrinya, harus memberikan surat cerai kepadanya.
\par 32 Tetapi sekarang Aku berkata kepadamu: barangsiapa menceraikan istrinya padahal wanita itu tidak menyeleweng, menyebabkan istrinya itu berzinah, kalau istrinya itu kawin lagi. Dan barangsiapa yang kawin dengan wanita yang diceraikan itu, berzinah juga."
\par 33 "Kalian tahu bahwa pada nenek moyang kita terdapat ajaran seperti ini: jangan mungkir janji. Apa yang sudah kaujanjikan dengan sumpah di hadapan Allah, harus engkau melakukannya.
\par 34 Tetapi sekarang Aku berkata kepadamu: jangan bersumpah sama sekali, baik demi langit, sebab langit adalah takhta Allah,
\par 35 maupun demi bumi, sebab bumi adalah alas kaki-Nya; atau demi Yerusalem, sebab itulah kota Raja besar.
\par 36 Jangan juga bersumpah demi kepalamu, sebab engkau sendiri tidak dapat membuat rambutmu menjadi putih atau hitam, biar hanya sehelai.
\par 37 Katakan saja 'Ya' atau 'Tidak' --lebih dari itu datangnya dari si Iblis."
\par 38 "Kalian tahu bahwa ada juga ajaran seperti ini: mata ganti mata, gigi ganti gigi.
\par 39 Tetapi sekarang Aku berkata kepadamu: jangan membalas dendam terhadap orang yang berbuat jahat kepadamu. Sebaliknya kalau orang menampar pipi kananmu, biarkanlah dia menampar pipi kirimu juga.
\par 40 Dan jikalau orang mengadukan kalian kepada hakim dan menuntut bajumu, berikanlah kepadanya jubahmu juga.
\par 41 Kalau seorang penguasa memaksa kalian memikul barangnya sejauh satu kilometer, pikullah sejauh dua kilometer.
\par 42 Kalau orang minta sesuatu kepadamu, berikanlah kepadanya. Dan jangan juga menolak orang yang mau meminjam sesuatu daripadamu."
\par 43 "Kalian tahu bahwa ada juga ajaran seperti ini: cintailah kawan-kawanmu dan bencilah musuh-musuhmu.
\par 44 Tetapi sekarang Aku berkata kepadamu: cintailah musuh-musuhmu, dan doakanlah orang-orang yang menganiaya kalian,
\par 45 supaya kalian menjadi anak-anak Bapamu yang di surga. Sebab Allah menerbitkan matahari-Nya untuk orang yang baik dan untuk orang yang jahat juga. Ia menurunkan hujan untuk orang yang berbuat benar dan untuk orang yang berbuat jahat juga.
\par 46 Sebab kalau kalian mengasihi hanya orang yang mengasihi kalian saja, untuk apa Allah harus membalas perbuatanmu itu? Bukankah para penagih pajak pun berbuat begitu?
\par 47 Dan kalau kalian memberi salam hanya kepada kawan-kawanmu saja, apakah istimewanya? Orang-orang yang tidak mengenal Allah pun berbuat begitu!
\par 48 Bapamu di surga mengasihi semua orang dengan sempurna. Kalian harus begitu juga."

\chapter{6}

\par 1 "Ingatlah, jangan kalian melakukan kewajiban agama di depan umum supaya dilihat orang. Kalau kalian berbuat begitu, kalian tidak akan diberi upah oleh Bapamu di surga.
\par 2 Jadi, kalau kalian memberi sedekah kepada orang miskin, janganlah menggembar-gemborkan hal itu seperti yang dilakukan oleh orang-orang munafik. Mereka suka melakukan itu di dalam rumah ibadat dan di jalan raya, supaya dipuji orang. Ingatlah, mereka sudah menerima upahnya.
\par 3 Tetapi kalian, kalau kalian memberi sedekah, berikanlah dengan diam-diam, sehingga tidak ada yang tahu.
\par 4 Biarlah perbuatanmu itu tidak diketahui oleh siapa pun, kecuali Bapamu di surga. Ia melihat perbuatanmu yang tersembunyi itu dan akan memberi upah kepadamu."
\par 5 "Kalau kalian berdoa, janganlah seperti orang-orang yang suka berpura-pura. Mereka suka berdoa sambil berdiri di rumah ibadat dan di simpang jalan supaya dilihat orang. Ingatlah, itulah upah yang mereka sudah terima.
\par 6 Tetapi kalau kalian berdoa, masuklah ke kamar dan tutuplah pintu, lalu berdoalah kepada Bapamu yang tidak kelihatan itu. Maka Bapamu yang melihat perbuatanmu yang tersembunyi akan memberi upah kepadamu.
\par 7 Kalau kalian berdoa, janganlah bertele-tele seperti orang-orang yang tak mengenal Tuhan. Mereka menyangka bahwa permintaan mereka akan didengar sebab doa mereka yang panjang itu.
\par 8 Jangan seperti mereka. Bapamu sudah tahu apa yang kalian perlukan, sebelum kalian memintanya.
\par 9 Jadi berdoalah begini, 'Bapa kami di surga: Engkaulah Allah yang Esa. Semoga Engkau disembah dan dihormati.
\par 10 Engkaulah Raja kami. Semoga Engkau memerintah di bumi dan kehendak-Mu ditaati seperti di surga.
\par 11 Berilah pada hari ini makanan yang kami perlukan.
\par 12 Ampunilah kami dari kesalahan kami, seperti kami sudah mengampuni orang yang bersalah kepada kami.
\par 13 Janganlah membiarkan kami kehilangan percaya pada waktu kami dicobai tetapi lepaskanlah kami dari kuasa si Jahat. (Engkaulah Raja yang berkuasa dan mulia untuk selama-lamanya. Amin.)'
\par 14 Kalau kalian mengampuni orang yang bersalah kepadamu, Bapamu di surga pun akan mengampuni kesalahanmu.
\par 15 Tetapi kalau kalian tidak mengampuni kesalahan orang lain, Bapamu di surga juga tidak akan mengampuni kesalahanmu."
\par 16 "Kalau kalian berpuasa, janganlah bermuka muram seperti orang yang suka berpura-pura. Mereka mengubah air mukanya supaya semua orang tahu bahwa mereka berpuasa. Ingatlah, itulah upah yang mereka sudah terima.
\par 17 Tetapi kalau kalian berpuasa, cucilah mukamu dan sisirlah rambutmu,
\par 18 supaya tak ada yang tahu bahwa kalian berpuasa, kecuali Bapamu yang tidak kelihatan itu saja. Dia melihat perbuatanmu yang tersembunyi itu dan akan memberi upah kepadamu."
\par 19 "Janganlah mengumpulkan harta untuk dirimu di dunia, di mana rayap dan karat dapat merusaknya dan pencuri datang mencurinya.
\par 20 Sebaliknya, kumpulkanlah harta di surga, di mana rayap dan karat tidak merusaknya, dan pencuri tidak datang mencurinya.
\par 21 Karena di mana hartamu, di situ juga hatimu!"
\par 22 "Mata adalah lampu untuk badan. Kalau matamu jernih, seluruh badanmu terang-benderang.
\par 23 Tetapi kalau matamu kabur, seluruh badanmu gelap-gulita. Jadi kalau lampu di dalam dirimu itu gelap, alangkah pekatnya kegelapan itu!"
\par 24 "Tidak seorang pun dapat bekerja untuk dua majikan. Sebab ia akan lebih mengasihi yang satu daripada yang lain. Atau ia akan lebih setia kepada majikan yang satu daripada kepada yang lain. Begitulah juga dengan kalian. Kalian tidak dapat bekerja untuk Allah dan untuk harta benda juga.
\par 25 Sebab itu ingatlah; janganlah khawatir tentang hidupmu, yaitu apa yang akan kalian makan dan minum, atau apa yang akan kalian pakai. Bukankah hidup lebih dari makanan, dan badan lebih dari pakaian?
\par 26 Lihatlah burung di udara. Mereka tidak menanam, tidak menuai, dan tidak juga mengumpulkan hasil tanamannya di dalam lumbung. Meskipun begitu Bapamu yang di surga memelihara mereka! Bukankah kalian jauh lebih berharga daripada burung?
\par 27 Siapakah dari kalian yang dengan kekhawatirannya dapat memperpanjang umurnya biarpun sedikit?
\par 28 Mengapa kalian khawatir tentang pakaianmu? Perhatikanlah bunga-bunga bakung yang tumbuh di padang. Bunga-bunga itu tidak bekerja dan tidak menenun;
\par 29 tetapi Raja Salomo yang begitu kaya pun, tidak memakai pakaian yang sebagus bunga-bunga itu!
\par 30 Rumput di padang tumbuh hari ini dan besok dibakar habis. Namun Allah mendandani rumput itu begitu bagus. Apalagi kalian! Tetapi kalian kurang percaya!
\par 31 Janganlah khawatir dan berkata, 'Apa yang akan kita makan', atau 'apa yang akan kita minum', atau 'apa yang akan kita pakai'?
\par 32 Hal-hal itu selalu dikejar oleh orang-orang yang tidak mengenal Allah. Padahal Bapamu yang di surga tahu bahwa kalian memerlukan semuanya itu.
\par 33 Jadi, usahakanlah dahulu supaya Allah memerintah atas hidupmu dan lakukanlah kehendak-Nya. Maka semua yang lain akan diberikan Allah juga kepadamu.
\par 34 Oleh sebab itu, janganlah khawatir tentang hari besok. Sebab besok ada lagi khawatirnya sendiri. Hari ini sudah cukup kesusahannya, tidak usah ditambah lagi."

\chapter{7}

\par 1 "Janganlah menghakimi orang lain, supaya kalian sendiri juga jangan dihakimi oleh Allah.
\par 2 Sebab sebagaimana kalian menghakimi orang lain, begitu juga Allah akan menghakimi kalian. Dan ukuran yang kalian pakai untuk orang lain, akan dipakai juga oleh Allah untuk kalian.
\par 3 Mengapa kalian melihat secukil kayu dalam mata saudaramu, sedangkan balok dalam matamu sendiri tidak kalian perhatikan?
\par 4 Bagaimana kalian dapat mengatakan kepada saudaramu, 'Mari saya keluarkan kayu secukil itu dari matamu,' sedangkan di dalam matamu sendiri ada balok?
\par 5 Hai munafik! Keluarkanlah dahulu balok dari matamu sendiri, barulah engkau melihat dengan jelas, dan dapat mengeluarkan secukil kayu dari mata saudaramu.
\par 6 Jangan berikan barang yang suci kepada anjing, supaya anjing itu jangan berbalik dan menyerangmu. Dan jangan berikan mutiara kepada babi, supaya babi itu jangan menginjak-injak mutiara itu."
\par 7 "Mintalah, maka kalian akan menerima. Carilah, maka kalian akan mendapat. Ketuklah, maka pintu akan dibukakan untukmu.
\par 8 Karena orang yang minta akan menerima; orang yang mencari akan mendapat; dan orang yang mengetuk, akan dibukakan pintu.
\par 9 Di antara kalian apakah ada ayah yang memberikan batu kepada anaknya, kalau ia minta roti?
\par 10 Atau memberikan ular, kalau ia minta ikan?
\par 11 Walaupun kalian jahat, kalian tahu juga memberikan yang baik kepada anak-anakmu. Apalagi Bapamu di surga! Ia lebih lagi akan memberikan yang baik kepada orang yang minta kepada-Nya.
\par 12 Perlakukanlah orang lain seperti kalian ingin diperlakukan oleh mereka. Itulah inti hukum Musa dan ajaran nabi-nabi."
\par 13 "Masuklah melalui pintu yang sempit, sebab pintu dan jalan yang menuju ke neraka besar dan lebar, dan banyak orang yang melaluinya.
\par 14 Tetapi sempit dan sukarlah pintu dan jalan yang membawa orang kepada hidup. Dan hanya sedikit orang yang menemukannya."
\par 15 "Hati-hatilah terhadap nabi-nabi palsu. Mereka datang kepada kalian berkedok domba, tetapi mereka sebenarnya seperti serigala yang buas.
\par 16 Kalian akan mengenal mereka dari hasil perbuatannya. Belukar berduri tidak mengeluarkan buah anggur, dan semak berduri tidak menghasilkan buah ara.
\par 17 Dari pohon yang subur diperoleh buah yang baik, dan dari pohon yang tidak subur buah yang buruk.
\par 18 Pohon yang subur tidak dapat menghasilkan buah yang buruk, dan pohon yang tidak subur tidak dapat menghasilkan buah yang baik.
\par 19 Setiap pohon yang tidak menghasilkan buah yang baik, ditebang dan dibakar.
\par 20 Begitu pula dengan nabi-nabi palsu. Kalian akan mengenal mereka dari hasil perbuatannya."
\par 21 "Tidak semua orang yang memanggil Aku, 'Tuhan, Tuhan,' akan menjadi anggota umat Allah, tetapi hanya orang-orang yang melakukan kehendak Bapa-Ku yang di surga.
\par 22 Pada Hari Kiamat banyak orang akan berkata kepada-Ku, 'Tuhan, Tuhan, bukankah dengan nama-Mu kami sudah menyampaikan pesan Allah? Dan bukankah dengan nama Tuhan juga kami sudah mengusir roh-roh jahat serta mengadakan banyak keajaiban?'
\par 23 Tetapi Aku akan menjawab, 'Aku tidak pernah mengenal kalian! Pergi dari sini, kalian yang melakukan kejahatan!'"
\par 24 "Nah, orang yang mendengar perkataan-Ku ini, dan menurutinya, sama seperti orang bijak yang membangun rumahnya di atas batu.
\par 25 Pada waktu hujan turun, dan air banjir datang serta angin kencang memukul rumah itu, rumah itu tidak roboh sebab telah dibangun di atas batu.
\par 26 Dan orang yang mendengar perkataan-Ku ini, tetapi tidak menurutinya, ia sama seperti orang bodoh yang membangun rumahnya di atas pasir.
\par 27 Pada waktu hujan turun, dan air banjir datang serta angin kencang memukul rumah itu, rumah itu roboh. Dan kerusakannya hebat sekali!"
\par 28 Akhirnya Yesus selesai mengajar, dan orang-orang yang ada di situ heran sekali mendengar pengajaran-Nya.
\par 29 Sebab Ia mengajar dengan wibawa, tidak seperti guru-guru agama mereka.

\chapter{8}

\par 1 Yesus turun dari bukit, dan banyak orang mengikuti Dia.
\par 2 Pada waktu itu datanglah seorang yang berpenyakit kulit yang mengerikan. Ia berlutut di hadapan Yesus, lalu berkata, "Pak, kalau Bapak mau, Bapak dapat menyembuhkan saya."
\par 3 Yesus menjamah orang itu sambil berkata, "Aku mau. Sembuhlah!" Saat itu juga penyakitnya hilang.
\par 4 Lalu kata Yesus kepadanya, "Ingatlah! Jangan ceritakan kepada siapa pun. Tetapi pergilah kepada imam, dan minta dia untuk memastikan engkau sudah sembuh. Sesudah itu persembahkanlah kurban yang diperintahkan Musa, sebagai bukti kepada orang-orang bahwa engkau sungguh-sungguh sudah sembuh!"
\par 5 Waktu Yesus masuk ke Kapernaum, seorang perwira Roma datang menjumpai Dia, dan minta tolong kepada-Nya.
\par 6 "Bapak," kata perwira itu, "pelayan saya sakit di rumah. Ia berbaring lumpuh di tempat tidur dan menderita sekali."
\par 7 Kata Yesus, "Aku akan pergi menyembuhkan dia."
\par 8 "Tidak usah Pak," jawab perwira itu, "Saya tidak patut menerima Bapak di rumah saya. Bapak perintahkan saja. Nanti pelayan saya itu sembuh.
\par 9 Sebab saya pun harus tunduk pada perintah atasan. Dan di bawah saya ada juga prajurit-prajurit yang harus tunduk pada perintah saya. Kalau saya menyuruh seorang prajurit, 'Pergi!' ia pun pergi. Saya mengatakan kepada yang lain, 'Mari sini!' ia pun datang; dan kalau saya memerintahkan hamba saya, 'Buatlah ini!' ia pun membuatnya."
\par 10 Waktu Yesus mendengar apa yang dikatakan oleh perwira itu, Ia kagum sekali. Lalu Ia berkata kepada orang-orang yang sedang mengikuti Dia, "Bukan main orang ini. Di antara orang Israel pun belum pernah Aku menemukan iman sebesar ini!
\par 11 Sungguh! Banyak orang akan datang dari timur dan barat untuk bersukaria bersama-sama Abraham, Ishak, dan Yakub di dalam Dunia Baru Allah.
\par 12 Padahal orang-orang yang seharusnya menjadi umat Allah akan dibuang ke kegelapan di luar. Di situ mereka akan menangis dan menderita."
\par 13 Lalu Yesus berkata kepada perwira itu, "Pulanglah, apa yang engkau percayai itu akan terjadi." Dan saat itu juga pelayannya itu sembuh.
\par 14 Yesus pergi ke rumah Petrus. Di situ Ia melihat ibu mertua Petrus sedang sakit demam di tempat tidur.
\par 15 Yesus menjamah tangannya, lalu demamnya hilang. Ia bangun dan mulai melayani Yesus.
\par 16 Pada waktu mulai petang, orang membawa kepada Yesus banyak orang yang kemasukan roh jahat. Dan dengan sepatah kata saja, Yesus mengusir roh-roh jahat itu dan menyembuhkan juga semua orang yang sakit.
\par 17 Yesus melakukan semuanya itu, dan dengan itu terjadilah apa yang dikatakan oleh Nabi Yesaya, yaitu, "Ia menanggung penderitaan kita dan menyembuhkan penyakit kita."
\par 18 Ada banyak sekali orang di sekeliling Yesus. Waktu Yesus melihat mereka semuanya, Ia menyuruh pengikut-pengikut-Nya menyeberangi danau.
\par 19 Lalu seorang guru agama datang kepada-Nya dan berkata, "Bapak Guru, saya mau mengikuti Bapak ke mana saja!"
\par 20 Yesus menjawab, "Serigala punya liang, dan burung punya sarang, tetapi Anak Manusia tidak punya tempat berbaring."
\par 21 Lalu seorang pengikut-Nya yang lain berkata, "Pak, izinkanlah saya pulang dahulu untuk menguburkan ayah saya."
\par 22 Tetapi Yesus menjawab, "Ikutlah Aku, dan biarkan orang mati menguburkan orang matinya sendiri."
\par 23 Yesus naik ke perahu dan pengikut-pengikut-Nya ikut bersama Dia.
\par 24 Tiba-tiba angin ribut yang hebat sekali melanda danau sehingga perahu dipukul ombak. Pada waktu itu Yesus sedang tidur.
\par 25 Maka pengikut-pengikut-Nya pergi kepada-Nya dan membangunkan Dia. "Pak! Tolong! Kita celaka!" seru mereka.
\par 26 "Mengapa kalian takut?" kata Yesus. "Kalian kurang percaya kepada-Ku!" Kemudian Yesus berdiri dan membentak angin dan danau itu. Lalu danau menjadi sangat tenang.
\par 27 Pengikut-pengikut Yesus heran. Mereka berkata, "Orang apakah Dia ini, sampai angin dan ombak pun menuruti perintah-Nya!"
\par 28 Yesus sampai di seberang danau di daerah Gadara. Di sana dua orang yang kemasukan roh jahat datang kepada-Nya. Kedua orang itu ganas sekali, sehingga tidak seorang pun berani lewat di situ.
\par 29 Mereka keluar dari gua-gua kuburan dan berteriak, "Anak Allah, akan Kauapakan kami? Apakah Engkau sudah mau menyiksa kami walaupun belum waktunya?"
\par 30 Tidak jauh dari situ, ada banyak sekali babi sedang mencari makan.
\par 31 Roh-roh jahat itu memohon kepada Yesus, "Kalau Kau akan menyuruh kami keluar, suruhlah kami masuk ke dalam babi-babi itu."
\par 32 "Pergi!" kata Yesus. Roh-roh jahat itu pergi dari kedua orang itu, lalu masuk ke dalam babi-babi. Dan babi-babi itu pun lari dan terjun dari pinggir jurang ke dalam danau, lalu tenggelam.
\par 33 Penjaga-penjaga babi itu lari ke kota, dan menceritakan semua kejadian itu, juga mengenai apa yang terjadi dengan kedua orang yang kemasukan roh jahat itu.
\par 34 Maka semua orang di kota itu pun pergi menjumpai Yesus. Waktu mereka melihat Yesus, mereka minta dengan sangat supaya Ia meninggalkan daerah mereka.

\chapter{9}

\par 1 Yesus naik ke dalam perahu, lalu menyeberangi danau, kembali ke kampung halaman-Nya.
\par 2 Di situ orang membawa kepada-Nya seorang lumpuh yang terbaring di tikar. Waktu Yesus melihat betapa besar iman orang-orang itu, Ia berkata kepada orang lumpuh itu, "Tabahlah, anak-Ku! Dosa-dosamu sudah diampuni."
\par 3 Beberapa guru agama yang ada di situ berkata dalam hati, "Orang ini menghina Allah!"
\par 4 Yesus tahu pikiran mereka, jadi Ia berkata, "Mengapa pikiranmu sejahat itu?
\par 5 Manakah yang lebih mudah: mengatakan, 'Dosamu sudah diampuni', atau mengatakan, 'Bangunlah dan berjalan'?
\par 6 Tetapi sekarang Aku akan membuktikan kepadamu bahwa di atas bumi ini Anak Manusia berkuasa untuk mengampuni dosa." Lalu Yesus berkata kepada orang lumpuh itu, "Bangun, angkat tikarmu dan pulanglah!"
\par 7 Orang lumpuh itu pun bangun dan pulang ke rumahnya.
\par 8 Waktu orang-orang melihat kejadian itu, mereka ketakutan dan memuji Allah, sebab Allah sudah memberikan kuasa yang begitu besar kepada manusia.
\par 9 Kemudian Yesus meninggalkan tempat itu. Sementara berjalan, Ia melihat seorang penagih pajak, bernama Matius, sedang duduk di kantor pajaknya. Yesus berkata kepadanya, "Mari ikut Aku!" Maka Matius berdiri dan mengikuti Yesus.
\par 10 Waktu Yesus sedang makan di rumah Matius, datanglah banyak penagih pajak dan orang-orang yang dianggap tidak baik oleh masyarakat, ikut makan bersama-sama Yesus dan pengikut-pengikut-Nya.
\par 11 Ada orang-orang Farisi yang melihat hal itu. Dan mereka bertanya kepada pengikut-pengikut Yesus, "Apa sebab gurumu makan bersama-sama dengan penagih pajak dan orang-orang tidak baik?"
\par 12 Yesus mendengar pertanyaan mereka lalu menjawab, "Orang yang sehat tidak memerlukan dokter, hanya orang yang sakit saja.
\par 13 Selidikilah apa artinya ayat Alkitab ini: 'Aku menghendaki belas kasihan, dan bukan kurban binatang'. Sebab Aku datang bukan untuk memanggil orang yang menganggap dirinya sudah baik, melainkan orang yang dianggap hina."
\par 14 Setelah itu pengikut-pengikut Yohanes Pembaptis datang kepada Yesus. Lalu mereka bertanya, "Mengapa kami dan orang-orang Farisi berpuasa sedangkan pengikut-pengikut Bapak tidak?"
\par 15 Yesus menjawab, "Bagaimanakah pendapat kalian? Bisakah tamu-tamu di pesta kawin bersedih hati kalau mempelai laki-laki masih ada bersama-sama mereka? Tentu tidak! Tetapi akan tiba saatnya mempelai laki-laki itu diambil dari mereka. Waktu itu barulah mereka berpuasa.
\par 16 Tidak ada orang yang menambal baju yang sudah tua dengan sepotong kain yang masih baru. Sebab kain penambal itu akan menciut dan menyobek baju itu, sehingga mengakibatkan sobekan yang lebih besar.
\par 17 Begitu juga tidak ada orang yang menuang anggur baru ke dalam kantong kulit yang tua. Sebab kantong itu akan pecah dan rusak, lalu anggurnya terbuang. Anggur baru harus dituang ke dalam kantong yang baru juga, supaya kedua-duanya tetap baik."
\par 18 Sementara Yesus berbicara kepada pengikut-pengikut Yohanes Pembaptis, datanglah seorang pemimpin rumah ibadat. Ia berlutut di depan Yesus dan berkata, "Anak perempuan saya baru saja meninggal. Tetapi, sudilah datang untuk menjamahnya supaya ia hidup lagi."
\par 19 Yesus bangkit dan bersama pengikut-pengikut-Nya pergi dengan orang itu.
\par 20 Di tengah jalan seorang wanita yang sudah dua belas tahun lamanya sakit pendarahan yang berhubungan dengan haidnya, datang mendekati Yesus dari belakang. Ia berpikir, "Asal saja saya menyentuh jubah-Nya, saya akan sembuh." Lalu ia menyentuh ujung jubah Yesus.
\par 21 [9:20]
\par 22 Saat itu Yesus menoleh dan melihat wanita itu lalu berkata kepadanya, "Tabahlah, anak-Ku! Karena engkau percaya kepada-Ku, engkau sembuh!" Pada saat itu juga wanita itu sembuh.
\par 23 Kemudian Yesus sampai di rumah pemimpin rumah ibadat itu. Ketika Ia melihat pemain-pemain musik perkabungan dan banyak orang yang ribut-ribut,
\par 24 Ia berkata kepada mereka, "Keluar kamu semua! Anak ini tidak mati; ia hanya tidur." Mereka semua menertawakan Yesus.
\par 25 Sesudah orang-orang itu keluar, Yesus masuk ke dalam kamar anak itu dan memegang tangannya. Lalu bangkitlah anak perempuan itu.
\par 26 Kabar itu tersebar ke seluruh daerah itu.
\par 27 Yesus pergi dari situ dan di tengah jalan dua orang buta mengikuti Dia. Mereka berteriak, "Anak Daud, kasihanilah kami!"
\par 28 Ketika Yesus masuk ke dalam rumah, kedua orang buta itu datang kepada-Nya. Yesus bertanya kepada mereka, "Apa kalian percaya bahwa Aku dapat menyembuhkan kalian?" "Percaya, Pak!" jawab mereka.
\par 29 Lalu Yesus menjamah mata mereka sambil berkata, "Karena kalian percaya, jadilah apa yang kalian harapkan."
\par 30 Maka mereka bisa melihat. Yesus memperingatkan mereka dengan tegas, supaya jangan memberitahukan hal itu kepada siapa pun.
\par 31 Tetapi mereka pergi dan menyiarkan berita tentang Yesus ke seluruh daerah.
\par 32 Waktu kedua orang itu pergi, seorang bisu yang dikuasai oleh roh jahat dibawa kepada Yesus.
\par 33 Yesus mengusir roh jahat itu dan pada saat itu juga orang itu bisa berbicara lagi. Orang banyak itu heran dan berkata, "Belum pernah kami melihat kejadian serupa ini di Israel!"
\par 34 Tetapi orang-orang Farisi berkata, "Kepala roh-roh jahatlah yang memberi Dia kuasa untuk mengusir roh-roh jahat itu."
\par 35 Demikianlah Yesus pergi berkeliling dari satu kota ke kota yang lain dan dari satu kampung ke kampung yang lain. Ia mengajar di rumah-rumah ibadat, dan mewartakan Kabar Baik tentang bagaimana Allah memerintah sebagai Raja. Ia menyembuhkan orang-orang yang menderita segala macam penyakit dan cacat badan.
\par 36 Waktu Yesus melihat orang banyak itu, Ia kasihan kepada mereka, sebab mereka kebingungan dan tidak berdaya, seperti domba yang tidak punya gembala.
\par 37 Lalu Yesus berkata kepada pengikut-pengikut-Nya, "Panennya banyak, tetapi penuainya hanya sedikit.
\par 38 Sebab itu mintalah kepada pemilik kebun itu supaya mengirim penuai untuk panennya."

\chapter{10}

\par 1 Pada suatu hari Yesus memanggil kedua belas orang pengikut-Nya berkumpul. Lalu Ia memberi kepada mereka kuasa untuk mengusir roh-roh jahat dan menyembuhkan segala macam penyakit dan segala macam cacat badan.
\par 2 Nama kedua belas rasul itu ialah: pertama, Simon (yang disebut juga Petrus) dengan saudaranya Andreas, lalu Yakobus dengan saudaranya Yohanes, yaitu anak-anak Zebedeus.
\par 3 Kemudian Filipus dan Bartolomeus, dengan Tomas, dan Matius, penagih pajak serta Yakobus anak Alfeus, dan Tadeus;
\par 4 lalu akhirnya Simon, si Patriot dan Yudas Iskariot yang mengkhianati Yesus.
\par 5 Kedua belas rasul itu kemudian diutus oleh Yesus dengan mendapat petunjuk-petunjuk ini, "Janganlah pergi ke daerah orang-orang yang bukan Yahudi. Jangan juga ke kota-kota orang Samaria.
\par 6 Tetapi pergilah kepada orang-orang Israel, khususnya kepada mereka yang sesat.
\par 7 Beritahukanlah kepada mereka bahwa Allah akan segera memerintah sebagai Raja.
\par 8 Sembuhkanlah orang-orang yang sakit; hidupkan orang-orang yang mati; sembuhkanlah orang-orang yang berpenyakit kulit yang mengerikan, dan usirlah roh-roh jahat. Kalian sudah menerima semuanya itu dengan cuma-cuma. Jadi, berilah juga dengan cuma-cuma.
\par 9 Jangan membawa uang emas, uang perak, ataupun uang tembaga.
\par 10 Janganlah juga membawa kantong sedekah, atau dua helai pakaian untuk perjalananmu, atau sepatu, ataupun tongkat. Sebab orang yang bekerja, sudah seharusnya dijamin kebutuhannya.
\par 11 Kalau kalian sampai di kota atau kampung, carilah seorang yang mau menerima kalian. Tinggallah dengan dia sampai kalian berangkat lagi.
\par 12 Waktu kalian masuk rumah, katakanlah, 'Semoga Tuhan memberkati kalian.'
\par 13 Kalau orang-orang di rumah itu menerima kalian, biarlah salammu itu tetap pada mereka. Tetapi kalau mereka tidak menerima kalian, tariklah kembali salammu itu.
\par 14 Kalau ada rumah atau kota yang tidak mau menerima kalian atau tidak mau mendengar kalian, tinggalkan tempat itu dan kebaskanlah debu dari tapak kakimu.
\par 15 Ingatlah! Pada Hari Kiamat, orang-orang kota Sodom dan Gomora akan lebih mudah diampuni Allah, daripada orang-orang di kota itu!"
\par 16 "Perhatikan ini: Aku mengutus kalian seperti domba yang tidak berdaya ke tengah-tengah serigala ganas. Kalian harus waspada seperti ular dan tulus hati seperti burung merpati.
\par 17 Berjaga-jagalah, sebab kalian akan ditangkap dan dihadapkan ke mahkamah-mahkamah. Kalian akan disiksa di rumah-rumah ibadat.
\par 18 Kalian akan dibawa ke hadapan penguasa-penguasa dan raja-raja karena kalian pengikut-Ku. Dan itulah kesempatan bagimu untuk memberi kesaksian tentang Aku kepada mereka dan kepada orang-orang yang tidak mengenal Allah.
\par 19 Tetapi kalau kalian dibawa untuk diadili, jangan khawatir mengenai apa yang kalian harus katakan, atau bagaimana kalian harus berbicara. Sebab apa yang kalian harus katakan itu, akan diberitahukan kepadamu pada waktunya.
\par 20 Karena yang berbicara pada waktu itu bukanlah kalian, melainkan Roh Bapa yang di surga, melalui kalian.
\par 21 Akan terjadi bahwa orang akan menyerahkan saudaranya sendiri untuk dibunuh. Dan itu pun yang akan terjadi antara bapak dengan anaknya. Anak-anak akan melawan ibu bapaknya, dan menyerahkan mereka untuk dibunuh.
\par 22 Kalian akan dibenci oleh semua orang karena kalian pengikut-Ku. Tetapi orang yang bertahan sampai akhir akan diselamatkan.
\par 23 Kalau kalian dianiaya di suatu kota, larilah ke kota yang berikut. Aku beritahukan kepadamu: Sebelum kalian selesai mengunjungi semua kota-kota Israel, Anak Manusia sudah datang.
\par 24 Murid tidak lebih besar dari gurunya, dan pelayan tidak lebih besar dari tuannya.
\par 25 Sudah cukup kalau seorang murid menjadi seperti gurunya, dan seorang pelayan seperti tuannya. Kalau kepala keluarga sudah diberi nama Beelzebul, apalagi seisi rumahnya. Mereka akan diberi nama yang lebih buruk lagi!"
\par 26 "Janganlah takut kepada manusia. Tidak ada yang tersembunyi, yang tidak akan kelihatan, dan tidak ada yang dirahasiakan yang tidak akan dibongkar.
\par 27 Apa yang Kukatakan kepadamu dalam gelap haruslah kalian ulangi di dalam terang siang hari. Dan apa yang dibisikkan di telingamu, umumkanlah itu seluas-luasnya!
\par 28 Janganlah takut kepada mereka yang membunuh badan, tetapi tidak berkuasa membunuh jiwa. Takutlah kepada Allah yang berkuasa membinasakan baik badan maupun jiwa di dalam neraka.
\par 29 Dua ekor burung pipit dapat dibeli dengan satu mata uang yang paling kecil. Meskipun begitu tidak ada seekor pun jatuh ke tanah kalau tidak dikehendaki Bapamu.
\par 30 Jumlah rambut di kepalamu pun sudah dihitung semuanya.
\par 31 Sebab itu, janganlah takut! Kalian lebih berharga daripada burung-burung pipit!"
\par 32 "Barangsiapa mengakui di depan umum bahwa ia pengikut-Ku, Aku pun akan mengakui dia di hadapan Bapa-Ku di surga.
\par 33 Tetapi barangsiapa menyangkal di muka umum bahwa ia pengikut-Ku, Aku pun akan menyangkal dia di hadapan Bapa-Ku di surga."
\par 34 "Janganlah menyangka bahwa Aku membawa perdamaian ke dunia ini. Aku tidak membawa perdamaian, tetapi perlawanan.
\par 35 Aku datang menyebabkan anak laki-laki melawan bapaknya, anak perempuan melawan ibunya, dan menantu perempuan melawan ibu mertuanya.
\par 36 Ya, yang akan menjadi musuh terbesar, adalah anggota keluarga sendiri.
\par 37 Orang yang mengasihi bapaknya atau ibunya lebih daripada-Ku tidak patut menjadi pengikut-Ku. Begitu juga orang yang mengasihi anaknya laki-laki atau perempuan lebih daripada-Ku.
\par 38 Dan orang yang tidak mau memikul salibnya dan mengikuti Aku tidak patut menjadi pengikut-Ku.
\par 39 Orang yang mempertahankan hidupnya, akan kehilangan hidupnya, tetapi orang yang kehilangan hidupnya karena setia kepada-Ku, akan mendapat hidupnya."
\par 40 "Orang yang menerima kalian, menerima Aku. Dan orang yang menerima Aku, menerima Dia yang mengutus Aku.
\par 41 Orang yang menerima nabi karena ia nabi, akan menerima upah seorang nabi. Dan siapa yang menerima seorang yang baik karena ia baik, akan menerima upah seorang yang baik.
\par 42 Barangsiapa memberi minum, biar cuma air dingin saja kepada salah seorang dari orang-orang hina dina ini, karena ia pengikut-Ku, percayalah ia pasti akan menerima upahnya!"

\chapter{11}

\par 1 Setelah Yesus memberi petunjuk-petunjuk kepada kedua belas pengikut-Nya, Ia meninggalkan tempat itu. Ia pergi mengajar dan menyiarkan pesan dari Allah di kota-kota yang dekat di situ.
\par 2 Yohanes Pembaptis yang sedang di penjara mendengar tentang pekerjaan Kristus. Lalu ia menyuruh beberapa pengikutnya pergi kepada Yesus untuk menanyakan,
\par 3 "Bapakkah orang yang akan datang menurut janji Allah, atau haruskah kami menunggu orang lain?"
\par 4 Yesus menjawab, "Kembalilah kepada Yohanes dan beritahukanlah apa yang kalian dengar dan lihat:
\par 5 Orang buta melihat, orang lumpuh berjalan, orang berpenyakit kulit yang mengerikan sembuh; orang tuli mendengar, orang mati hidup kembali, dan Kabar Baik dari Allah diberitakan kepada orang-orang miskin.
\par 6 Berbahagialah orang yang tidak ada alasan untuk menolak Aku!"
\par 7 Sesudah utusan-utusan Yohanes itu pergi, Yesus mulai berbicara kepada orang banyak tentang Yohanes, kata-Nya, "Kalian pergi ke padang gurun untuk melihat apa? Sehelai rumput yang ditiup anginkah?
\par 8 Kalian pergi untuk melihat apa? Seorang yang berpakaian baguskah? Orang-orang yang berpakaian begitu tinggal di istana!
\par 9 Jadi mengapa kalian pergi ke padang gurun? Untuk melihat seorang nabikah? Benar, malah lebih dari seorang nabi.
\par 10 Sebab Yohanes itulah yang dimaksudkan dalam ayat Alkitab ini, 'Inilah utusan-Ku, kata Allah, Aku mengutus dia lebih dahulu daripada-Mu supaya ia membuka jalan untuk-Mu.'
\par 11 Ingatlah! Di dunia ini tidak pernah ada orang yang lebih besar daripada Yohanes Pembaptis. Namun demikian, orang yang terkecil di antara umat Allah lebih besar daripada Yohanes.
\par 12 Sejak Yohanes mengabarkan beritanya sampai pada saat ini, umat Allah ditentang oleh orang-orang yang berusaha menguasainya dengan kekerasan.
\par 13 Sampai kedatangan Yohanes, semua hukum Musa dan ajaran nabi-nabi bernubuat tentang hal-hal yang harus terjadi.
\par 14 Dan kalau kalian mau percaya, Yohanes itulah Elia, yang kedatangannya sudah dinubuatkan.
\par 15 Kalau punya telinga, dengarkan!
\par 16 Dengan apa harus Aku bandingkan orang-orang zaman ini? Mereka seperti anak-anak yang duduk di pasar. Sekelompok berseru kepada yang lain,
\par 17 'Kami memainkan lagu gembira untuk kalian, tetapi kalian tidak mau menari! Kami menyanyikan lagu perkabungan, dan kalian tidak mau menangis!'
\par 18 Yohanes datang--ia berpuasa dan tidak minum anggur; dan orang-orang berkata, 'Ia kemasukan setan!'
\par 19 Sekarang Anak Manusia, datang--Ia makan dan minum; lalu orang-orang berkata, 'Lihat orang itu! Rakus, pemabuk, kawan penagih pajak dan kawan orang berdosa.' Meskipun begitu, kebijaksanaan Allah terbukti dari hasil-hasilnya."
\par 20 Lalu Yesus mulai mencela kota-kota, di mana Ia paling banyak membuat keajaiban. Sebab orang-orang di kota-kota itu tidak mau bertobat dari dosa-dosa mereka.
\par 21 "Celaka kamu, Korazim! Dan celaka juga kamu, Betsaida! Seandainya keajaiban-keajaiban yang dibuat di tengah-tengahmu sudah dilakukan di Tirus dan Sidon, pasti orang-orang di sana sudah lama bertobat dari dosa-dosa mereka dan memakai pakaian berkabung serta menaruh abu ke atas kepala.
\par 22 Ingatlah, pada Hari Kiamat, orang-orang Tirus dan Sidon akan lebih mudah diampuni Allah daripada kalian!
\par 23 Dan kamu, Kapernaum! Apakah kamu akan ditinggikan sampai ke surga? Tidak! Malah kamu akan dibuang ke neraka! Sebab seandainya keajaiban-keajaiban yang dibuat di tengah-tengahmu itu sudah dibuat di Sodom, Sodom itu masih ada sampai saat ini!
\par 24 Ingatlah, pada Hari Kiamat, orang Sodom akan lebih mudah diampuni Allah daripada kalian!"
\par 25 Pada waktu itu Yesus berdoa, "Bapa, Tuhan yang menguasai langit dan bumi! Aku mengucap terima kasih kepada-Mu karena semuanya itu Engkau rahasiakan dari orang-orang yang pandai dan berilmu, tetapi Engkau tunjukkan kepada orang-orang yang tidak terpelajar.
\par 26 Itulah yang menyenangkan hati Bapa."
\par 27 Lalu Yesus berkata, "Segala sesuatu sudah diserahkan Bapa kepada-Ku. Tidak seorang pun mengenal Anak, selain Bapa. Tidak ada juga yang mengenal Bapa selain Anak, dan orang-orang kepada siapa Anak itu memperkenalkan Bapa.
\par 28 Datanglah kepada-Ku kamu semua yang lelah, dan merasakan beratnya beban; Aku akan menyegarkan kamu.
\par 29 Ikutlah perintah-Ku dan belajarlah daripada-Ku. Sebab Aku ini lemah lembut dan rendah hati, maka kamu akan merasa segar.
\par 30 Karena perintah-perintah-Ku menyenangkan, dan beban yang Kutanggungkan atasmu ringan."

\chapter{12}

\par 1 Pada suatu hari Sabat, ketika Yesus lewat sebuah ladang gandum, pengikut-pengikut-Nya mulai memetik gandum, lalu memakannya karena lapar.
\par 2 Ketika orang-orang Farisi melihat itu, mereka berkata kepada Yesus, "Lihat! Pengikut-pengikut-Mu melanggar hukum agama kita dengan melakukan yang dilarang pada hari Sabat."
\par 3 Yesus menjawab, "Belum pernahkah kalian membaca tentang apa yang dilakukan Daud waktu ia dan orang-orangnya lapar?
\par 4 Ia masuk ke dalam Rumah Allah; lalu makan roti yang sudah dipersembahkan kepada Allah. Padahal menurut hukum agama kita, ia maupun orang-orangnya tak boleh makan roti itu--hanya imam-imam saja yang boleh.
\par 5 Atau belum pernahkah kalian membaca di dalam hukum Musa bahwa tiap hari Sabat imam-imam yang bertugas di Rumah Tuhan, melanggar peraturan hari Sabat, tetapi tidak disalahkan?
\par 6 Perhatikan apa yang Kukatakan ini: di sini ada yang lebih besar dari Rumah Tuhan.
\par 7 Di dalam Alkitab tertulis: Belas kasihanlah yang Kukehendaki, bukan kurban binatang. Kalau sekiranya kalian benar-benar mengerti perkataan itu, pasti kalian tidak akan menyalahkan orang-orang yang tidak bersalah.
\par 8 Karena Anak Manusia berkuasa atas hari Sabat."
\par 9 Yesus meninggalkan tempat itu lalu pergi ke sebuah rumah ibadat.
\par 10 Di situ ada orang yang tangannya lumpuh sebelah. Beberapa orang yang mau mencari-cari kesalahan Yesus, bertanya kepada-Nya, "Apakah boleh menyembuhkan orang pada hari Sabat?"
\par 11 Yesus menjawab, "Seandainya ada seorang dari antara kalian punya seekor domba, dan pada hari Sabat domba itu jatuh ke dalam lubang yang dalam; apakah pemilik domba itu tidak akan berusaha mengeluarkan domba itu dari dalam lubang itu?
\par 12 Nah, manusia lebih berharga dari domba! Jadi, kalau begitu, boleh menolong orang pada hari Sabat."
\par 13 Kemudian Yesus berkata kepada orang yang tangannya lumpuh sebelah itu, "Ulurkanlah tanganmu." Orang itu mengulurkan tangannya, dan tangan itu sembuh seperti tangannya yang sebelah.
\par 14 Tetapi orang-orang Farisi meninggalkan rumah ibadat itu, lalu bermufakat untuk membunuh Yesus.
\par 15 Yesus tahu bahwa orang-orang Farisi itu berniat jahat terhadap diri-Nya. Jadi Ia pergi dari tempat itu dan banyak orang mengikuti-Nya. Lalu Ia menyembuhkan semua orang yang sakit.
\par 16 Tetapi Ia melarang mereka memberitahukan tentang Dia kepada orang lain.
\par 17 Dengan demikian terjadilah apa yang dikatakan Allah melalui Nabi Yesaya,
\par 18 "Inilah utusan-Ku yang Kupilih, Orang yang Kukasihi dan yang berkenan di hati-Ku. Roh-Ku akan Kuberikan kepada-Nya, keadilan-Ku akan diwartakan-Nya kepada bangsa-bangsa.
\par 19 Ia tidak akan bertengkar atau berteriak, atau berpidato di jalan-jalan raya.
\par 20 Buluh yang terkulai tak akan dipatahkan-Nya pelita yang redup tidak akan dipadamkan-Nya. Ia akan berjuang sampai keadilan tercapai;
\par 21 segala bangsa akan menaruh harapan kepada-Nya."
\par 22 Kemudian dibawa kepada Yesus seorang yang buta dan bisu karena dikuasai oleh roh jahat. Yesus menyembuhkan orang itu sehingga ia dapat berbicara dan melihat.
\par 23 Semua orang heran dan berkata, "Mungkinkah Dia ini Anak Daud yang dijanjikan itu?"
\par 24 Ketika orang-orang Farisi mendengar itu, mereka menjawab, "Orang ini hanya bisa mengusir roh jahat, karena Beelzebul, kepala roh-roh jahat, telah memberi kuasa itu kepada-Nya."
\par 25 Yesus mengetahui pikiran orang-orang Farisi itu. Jadi Ia berkata kepada mereka, "Kalau suatu negara terpecah dalam golongan-golongan yang saling bermusuhan, negara itu tidak akan bertahan. Dan sebuah kota atau keluarga yang terpecah-pecah dan bermusuhan satu sama lain akan hancur.
\par 26 Begitu juga di dalam kerajaan Iblis; kalau satu kelompok mengusir kelompok yang lain, maka kerajaan Iblis itu sudah terpecah-pecah dan akan runtuh.
\par 27 Kalian berkata bahwa Aku mengusir roh jahat karena kuasa Beelzebul. Kalau begitu dengan kuasa siapa pengikut-pengikutmu mengusir roh jahat. Pengikut-pengikutmu itu sendiri yang membuktikan bahwa kalian salah!
\par 28 Tetapi Aku mengusir roh jahat dengan kuasa Roh Allah. Dan itu berarti bahwa Allah sudah mulai memerintah di tengah-tengah kalian.
\par 29 Bagaimana orang dapat masuk ke dalam rumah seorang yang kuat untuk merampas hartanya, kalau ia tidak lebih dahulu mengikat orang kuat itu? Sesudah itu, baru ia dapat merampas hartanya.
\par 30 Orang yang tidak memihak Aku sesungguhnya menentang Aku. Dan orang yang tidak membantu Aku sesungguhnya merusak pekerjaan-Ku!
\par 31 Oleh sebab itu, ketahuilah, apabila orang berbuat dosa dan mengucap penghinaan, ia dapat diampuni! Tetapi kalau ia menghina Roh Allah, ia tidak dapat diampuni!
\par 32 Apabila orang mengatakan sesuatu menentang Anak Manusia, ia dapat diampuni, tetapi apabila ia menghina Roh Allah, ia tidak dapat diampuni, baik sekarang maupun di akhirat!"
\par 33 "Untuk mendapat buah yang baik, pohonnya harus subur. Kalau pohonnya tidak subur, buahnya tidak baik juga. Subur tidaknya suatu pohon diketahui dari buahnya.
\par 34 Kamu orang jahat, bagaimana mungkin kamu dapat mengucapkan hal-hal yang baik kalau kamu jahat? Apa yang diucapkan oleh mulut itulah yang melimpah dari hati!
\par 35 Orang yang baik mengucapkan hal-hal yang baik karena ia penuh kebaikan. Sebaliknya, orang yang jahat mengucapkan hal-hal yang jahat karena ia penuh kejahatan.
\par 36 Jadi, ingatlah: pada Hari Kiamat, setiap orang harus bertanggung jawab atas tiap ucapannya yang tidak berguna.
\par 37 Sebab kata-katamu sendirilah yang akan dipakai untuk memutuskan apakah engkau bersalah atau tidak."
\par 38 Kemudian beberapa guru agama dan orang-orang Farisi berkata, "Pak Guru, kami ingin melihat Bapak membuat keajaiban."
\par 39 "Alangkah jahatnya dan durhakanya orang-orang zaman ini!" jawab Yesus. "Kalian minta Aku membuat keajaiban? Kalian tidak akan diberi satu keajaiban pun, kecuali keajaiban Nabi Yunus.
\par 40 Yunus tinggal tiga hari tiga malam di dalam perut ikan besar. Begitu juga Anak Manusia akan tinggal tiga hari tiga malam di dalam perut bumi.
\par 41 Pada Hari Kiamat, penduduk Niniwe akan bangkit bersama orang-orang zaman ini dan menuduh mereka. Sebab orang-orang Niniwe itu bertobat dari dosa-dosa mereka, ketika Yunus berkhotbah kepada mereka. Tetapi di sini sekarang ada yang lebih besar daripada Yunus!
\par 42 Pada Hari Kiamat, ratu dari negeri Selatan akan bangkit bersama orang-orang zaman ini dan menuduh mereka. Sebab untuk mendengarkan pengajaran Salomo yang bijak, ratu itu membuat perjalanan yang jauh sekali dari ujung bumi. Tetapi di sini sekarang ada yang lebih besar daripada Salomo!"
\par 43 "Apabila roh jahat meninggalkan seseorang, roh itu berkeliling ke tempat-tempat yang kering untuk mencari tempat istirahat, tetapi ia tidak mendapatnya.
\par 44 Oleh sebab itu ia berkata, 'Saya akan kembali ke rumah yang sudah saya tinggalkan.' Waktu ia sampai di sana, rumah itu kosong, bersih dan teratur.
\par 45 Lalu ia pergi dan membawa tujuh roh lain yang lebih jahat dari dia. Kemudian mereka masuk ke dalam orang itu, lalu tinggal di situ. Dan akhirnya keadaan orang itu menjadi lebih buruk dari semula. Itulah juga yang akan terjadi dengan orang-orang jahat zaman ini."
\par 46 Sementara Yesus masih berbicara dengan orang banyak itu, datanglah ibu dan saudara-saudara-Nya. Mereka berdiri di luar sambil berusaha untuk dapat berbicara dengan Dia.
\par 47 Seorang dari orang banyak itu berkata kepada Yesus, "Pak, ibu dan saudara-saudara Bapak ada di luar. Mereka ingin berbicara dengan Bapak."
\par 48 Lalu Yesus menjawab, "Siapakah ibu-Ku? Siapakah saudara-saudara-Ku?"
\par 49 Lalu Ia menunjuk kepada pengikut-pengikut-Nya dan berkata, "Inilah ibu dan saudara-saudara-Ku.
\par 50 Orang yang melakukan kehendak Bapa-Ku yang di surga, dialah saudara laki-laki, saudara perempuan, dan ibu-Ku."

\chapter{13}

\par 1 Pada hari itu juga Yesus meninggalkan rumah itu lalu pergi ke tepi danau dan duduk di situ.
\par 2 Banyak sekali orang yang berkumpul di sekeliling Yesus, karena itu Ia pergi duduk di dalam perahu, sedangkan orang banyak itu berdiri di pantai.
\par 3 Lalu Yesus mengajar banyak hal kepada mereka dengan memakai perumpamaan. "Seorang petani pergi menabur benih," demikianlah Yesus mulai dengan cerita-Nya.
\par 4 "Ketika sedang menabur, ada benih yang jatuh di jalan. Lalu burung datang dan benih itu dimakan habis.
\par 5 Ada juga yang jatuh di tempat berbatu-batu, yang tanahnya sedikit. Benih-benih itu segera tumbuh karena kurang tanah.
\par 6 Tetapi waktu matahari sudah naik, tunas-tunas itu mulai layu, kemudian kering dan mati karena akarnya tidak masuk cukup dalam.
\par 7 Ada pula benih yang jatuh di tengah semak berduri. Semak berduri itu tumbuh dan menghimpit tunas-tunas itu sampai mati.
\par 8 Tetapi ada juga benih yang jatuh di tanah yang subur, lalu berbuah; ada yang seratus, ada yang enam puluh, dan ada juga yang tiga puluh kali lipat."
\par 9 Sesudah menceritakan perumpamaan itu Yesus berkata, "Kalau punya telinga, dengarkan!"
\par 10 Kemudian pengikut-pengikut Yesus datang dan bertanya kepada-Nya, "Mengapa Bapak memakai perumpamaan kalau berbicara dengan orang banyak itu?"
\par 11 Yesus menjawab, "Sebab kalian sudah diberi anugerah untuk mengetahui rahasia tentang bagaimana Allah memerintah, sedangkan mereka tidak.
\par 12 Karena orang yang sudah mempunyai, akan diberi lebih banyak lagi, dan ia akan berkelebihan. Tetapi orang yang tidak mempunyai apa-apa, maka sedikit yang ada padanya malah akan diambil.
\par 13 Itulah sebabnya Aku memakai perumpamaan kalau berbicara dengan orang banyak, karena mereka melihat, tetapi seperti orang yang tidak melihat; mereka mendengar tetapi seperti orang yang tidak mendengar dan tidak mengerti.
\par 14 Dengan itu terjadilah yang dinubuatkan Nabi Yesaya, 'Allah berkata: Mereka akan terus mendengar tetapi tidak mengerti; mereka akan terus memperhatikan tetapi tidak tahu apa yang terjadi.
\par 15 Sebab pikiran orang-orang ini sudah menjadi tumpul, telinga mereka sudah menjadi tuli dan mata mereka sudah dipejamkan. Ini terjadi supaya mata mereka jangan melihat, telinga mereka jangan mendengar, pikiran mereka jangan mengerti dan jangan kembali kepada-Ku, lalu Aku menyembuhkan mereka.'"
\par 16 "Tetapi alangkah beruntungnya kalian," kata Yesus kepada pengikut-pengikut-Nya, "sebab kalian sungguh melihat dan mendengar.
\par 17 Ingatlah, banyak nabi dan orang yang taat kepada Allah ingin melihat yang kalian lihat sekarang ini, tetapi mereka tidak melihatnya. Mereka ingin mendengar apa yang kalian dengar sekarang ini, tetapi mereka tidak mendengarnya."
\par 18 "Dengarlah apa arti perumpamaan tentang penabur itu.
\par 19 Benih yang jatuh di jalan ibarat orang-orang yang mendengar kabar tentang bagaimana Allah memerintah, tetapi tidak mengerti. Si Jahat itu datang dan merampas apa yang sudah ditabur dalam hati mereka.
\par 20 Benih yang jatuh di tempat yang berbatu-batu, ibarat orang-orang yang mendengar kabar itu, dan langsung menerimanya dengan senang hati.
\par 21 Tetapi kabar itu tidak berakar dalam hati mereka, sehingga tidak tahan lama. Begitu mereka menderita kesusahan atau penganiayaan karena kabar itu, langsung mereka murtad.
\par 22 Benih yang jatuh di tengah-tengah semak berduri ibarat orang-orang yang mendengar kabar itu, tetapi khawatir tentang hidup mereka dan ingin hidup mewah. Karena itu kabar dari Allah terhimpit di dalam hati mereka sehingga tidak berbuah.
\par 23 Dan benih yang jatuh di tanah yang subur ibarat orang-orang yang mendengar kabar itu dan memahaminya. Mereka berbuah banyak, ada yang seratus, ada yang enam puluh, dan ada yang tiga puluh kali lipat hasilnya."
\par 24 Yesus menceritakan lagi sebuah perumpamaan kepada orang banyak, kata-Nya, "Apabila Allah memerintah, keadaannya seperti perumpamaan ini: Seorang petani menabur benih yang baik di ladangnya.
\par 25 Pada suatu malam, ketika semua orang sedang tidur, musuh petani itu datang menabur benih alang-alang di antara gandum itu, lalu pergi.
\par 26 Ketika tanaman-tanaman itu tumbuh dan mayang-mayangnya mulai muncul, kelihatanlah juga alang-alang itu.
\par 27 Lalu orang-orang gajian petani itu datang kepada petani itu dan berkata, 'Tuan, bukankah Tuan menanam benih yang baik di ladang Tuan? Bagaimana jadinya sampai ada alang-alang di sana?'
\par 28 Petani itu menjawab, 'Itu perbuatan musuh.' Lalu orang-orang gajian petani itu bertanya lagi, 'Tuan mau kami pergi mencabut alang-alang itu?'
\par 29 'Tidak,' jawabnya, 'sebab kalau alang-alang itu dicabut, nanti gandumnya turut tercabut.
\par 30 Biarkanlah alang-alang itu tumbuh bersama-sama sampai waktu menuai. Nanti saya akan berkata kepada orang-orang yang menuai: Kumpulkan dahulu alang-alangnya, ikat, lalu bakar. Sesudah itu kumpulkan gandumnya, lalu simpan di dalam lumbung.'"
\par 31 Yesus menceritakan lagi sebuah perumpamaan kepada orang banyak, kata-Nya, "Apabila Allah memerintah, keadaannya seperti perumpamaan ini: Sebuah biji sawi diambil oleh seseorang, lalu ditanam di ladangnya.
\par 32 Biji sawi adalah benih yang paling kecil. Tetapi kalau sudah tumbuh, ia menjadi yang terbesar di antara tanaman-tanaman. Ia menjadi pohon, sehingga burung-burung datang bersarang pada cabang-cabangnya."
\par 33 Ada sebuah perumpamaan lain yang diceritakan Yesus kepada orang banyak. "Apabila Allah memerintah, keadaannya seperti ragi yang diambil oleh seorang wanita, lalu diaduk dengan empat puluh liter tepung terigu sampai berkembang semuanya!"
\par 34 Semuanya diajarkan Yesus kepada orang banyak dengan memakai perumpamaan.
\par 35 Dengan demikian terjadilah yang dikatakan oleh nabi, "Aku memakai perumpamaan kalau berbicara dengan mereka; Aku akan memberitakan hal-hal yang tersembunyi semenjak terjadinya dunia ini."
\par 36 Setelah itu Yesus meninggalkan orang banyak itu, lalu masuk rumah. Pengikut-pengikut-Nya datang dan berkata, "Coba Bapak terangkan kepada kami arti perumpamaan tentang alang-alang di antara gandum itu."
\par 37 Yesus menjawab, "Orang yang menabur benih yang baik itu adalah Anak Manusia.
\par 38 Ladang itu ialah dunia ini. Benih yang baik itu adalah orang-orang yang sudah menjadi umat Allah. Alang-alang itu ialah orang-orang yang berpihak kepada Iblis.
\par 39 Musuh yang menanam alang-alang itu ialah Iblis. Masa panen ialah Hari Kiamat, dan orang-orang yang menuai itu ialah malaikat-malaikat.
\par 40 Sebagaimana alang-alang dikumpulkan dan dibakar di dalam api, begitu juga pada Hari Kiamat nanti.
\par 41 Anak Manusia akan menyuruh malaikat-malaikat-Nya mengumpulkan dari antara umat-Nya semua yang menyebabkan orang berbuat dosa, dan semua orang lainnya yang melakukan kejahatan.
\par 42 Mereka semua akan dibuang ke dalam tungku berapi yang bernyala-nyala; di situ mereka akan menangis dan menderita.
\par 43 Dan orang-orang yang melakukan kehendak Allah akan bersinar seperti matahari di dalam Dunia Baru Allah, Bapa mereka. Jadi, kalau punya telinga, dengarkan!"
\par 44 "Apabila Allah memerintah, keadaannya seperti perumpamaan ini: ada harta yang terpendam di dalam tanah lalu ditemukan oleh seseorang, dan dimasukkan kembali ke dalam tanah. Kemudian karena gembiranya, orang itu pergi menjual seluruh miliknya, lalu kembali dan membeli tanah itu."
\par 45 "Apabila Allah memerintah, keadaannya seperti perumpamaan ini: Seorang pedagang mencari mutiara-mutiara yang berharga.
\par 46 Ketika ia menemukan sebutir mutiara yang luar biasa indahnya, segera ia pergi dan menjual semua miliknya, lalu membeli mutiara yang satu itu."
\par 47 "Apabila Allah memerintah, keadaannya diumpamakan dengan jala yang ditebarkan ke danau, lalu mendapat bermacam-macam ikan.
\par 48 Sesudah jala itu penuh, jala itu diangkat ke darat oleh nelayan-nelayan. Kemudian mereka duduk dan memisah-misahkan ikan-ikan itu: Yang baik disimpan dalam tempayan dan yang tidak baik dibuang.
\par 49 Begitulah halnya pada Hari Kiamat, malaikat-malaikat akan pergi memisahkan orang-orang jahat dari orang-orang yang melakukan kehendak Allah.
\par 50 Kemudian orang-orang jahat itu dibuang ke dalam tungku berapi. Di situlah mereka akan menangis dan menderita."
\par 51 "Apakah kalian mengerti semuanya itu?" tanya Yesus. "Mengerti Pak!" jawab mereka.
\par 52 Lalu Yesus berkata, "Itu sebabnya setiap guru agama yang sudah menjadi anggota umat Allah, adalah seperti seorang tuan rumah yang mengeluarkan dari tempat hartanya barang-barang baru dan lama."
\par 53 Setelah Yesus selesai menceritakan perumpamaan-perumpamaan itu, Ia meninggalkan tempat itu,
\par 54 lalu kembali ke kampung halaman-Nya. Di sana Ia pergi mengajar di rumah ibadat, dan orang-orang yang mendengarkan Dia di situ, heran sekali. Mereka berkata, "Dari mana orang ini mendapat hikmat seperti itu? Dan dari mana Ia mendapat kuasa untuk membuat keajaiban?
\par 55 Bukankah Ia anak tukang kayu? Bukankah Maria itu ibu-Nya; dan saudara-saudara-Nya adalah Yakobus, Yusuf, Simon dan Yudas?
\par 56 Dan bukankah saudara-saudara perempuan-Nya tinggal di sini juga? Dari mana Ia mendapat semuanya itu?"
\par 57 Maka mereka menolak Yesus. Lalu Yesus berkata kepada mereka, "Seorang nabi dihormati di mana-mana, kecuali di kampung halamannya dan di rumahnya sendiri."
\par 58 Maka itu Yesus tidak mengerjakan banyak keajaiban di situ sebab mereka tidak percaya.

\chapter{14}

\par 1 Pada masa itu Herodes, penguasa negeri Galilea, mendengar mengenai Yesus.
\par 2 Herodes berkata kepada pejabat-pejabatnya, "Pasti ini Yohanes Pembaptis yang sudah hidup kembali! Itu sebabnya Ia mempunyai kuasa melakukan keajaiban."
\par 3 Sebab sebelum itu Herodes menyuruh menangkap Yohanes dan mengikat dia serta memasukkannya ke dalam penjara. Herodes berbuat begitu karena soal Herodias, istri saudaranya sendiri, yaitu Filipus.
\par 4 Sebab pernah Yohanes mengatakan begini kepada Herodes, "Tidak boleh engkau kawin dengan Herodias!"
\par 5 Sebenarnya Herodes ingin membunuh Yohanes, tetapi ia takut kepada orang-orang, sebab mereka menganggap Yohanes nabi.
\par 6 Pada waktu Herodes merayakan hari ulang tahunnya, anak perempuan Herodias menari di hadapan hadirin. Tariannya sangat menyenangkan hati Herodes,
\par 7 sehingga Herodes berjanji kepadanya dengan sumpah bahwa apa saja yang ia minta, akan diberikan kepadanya.
\par 8 Karena dihasut oleh ibunya, gadis itu berkata, "Saya minta kepala Yohanes Pembaptis diberikan kepada saya sekarang ini juga di atas baki!"
\par 9 Mendengar permintaan itu, Herodes menjadi sedih sekali. Tetapi karena ia sudah bersumpah di hadapan para tamunya, ia memerintahkan supaya permintaan gadis itu dipenuhi.
\par 10 Ia menyuruh orang pergi ke penjara untuk memancung kepala Yohanes.
\par 11 Kemudian kepala Yohanes itu dibawa masuk di atas baki dan diberikan kepada gadis itu. Lalu gadis itu membawanya pula kepada ibunya.
\par 12 Setelah itu pengikut-pengikut Yohanes datang mengambil jenazah Yohanes dan menguburkannya. Lalu mereka pergi dan memberitahukan hal itu kepada Yesus.
\par 13 Waktu Yesus mendengar berita itu, Ia naik perahu sendirian dan meninggalkan tempat itu, untuk pergi ke suatu tempat yang sunyi. Tetapi ketika orang-orang mendengar tentang hal itu, mereka meninggalkan kota-kota mereka dan pergi menyusul Yesus melalui jalan darat.
\par 14 Waktu Yesus turun dari perahu dan melihat orang banyak itu, Ia kasihan kepada mereka. Lalu Ia menyembuhkan orang-orang yang sakit di antara mereka.
\par 15 Sore harinya, pengikut-pengikut Yesus datang dan berkata kepada-Nya, "Hari sudah sore dan tempat ini terpencil. Lebih baik Bapak menyuruh orang-orang ini pergi, supaya dapat membeli makanan di desa-desa."
\par 16 "Tidak usah mereka pergi," kata Yesus, "kalian saja beri mereka makan."
\par 17 "Kami hanya punya lima roti dan dua ikan!" jawab pengikut-pengikut Yesus itu.
\par 18 "Bawa itu kemari," kata Yesus.
\par 19 Kemudian Ia menyuruh orang banyak itu duduk di atas rumput. Lalu Ia mengambil lima roti dan dua ikan itu, lalu menengadah ke langit dan mengucap syukur kepada Allah. Sesudah itu Ia membelah-belah roti itu dengan tangan-Nya dan memberikan-Nya kepada pengikut-pengikut-Nya untuk dibagi-bagikan kepada orang banyak itu.
\par 20 Mereka semua makan sampai kenyang. Sesudah itu pengikut-pengikut Yesus mengumpulkan kelebihan makanan itu; ada dua belas bakul penuh.
\par 21 Yang makan pada waktu itu ada kira-kira lima ribu orang, belum terhitung wanita dan anak-anak.
\par 22 Setelah itu Yesus menyuruh pengikut-pengikut-Nya naik perahu dan lebih dahulu menyeberang danau, sementara Ia menyuruh orang banyak itu pulang.
\par 23 Sesudah orang banyak itu pergi, Yesus naik sebuah bukit sendirian untuk berdoa. Waktu sudah malam, Yesus masih berada di situ sendirian.
\par 24 Sementara itu perahu yang ditumpangi pengikut-pengikut Yesus, sudah jauh di tengah-tengah danau. Perahu itu terhempas-hempas dipukul ombak, karena angin berlawanan arah dengan perahu.
\par 25 Antara pukul tiga dan pukul enam pagi, Yesus datang kepada mereka berjalan di atas air.
\par 26 Ketika pengikut-pengikut-Nya melihat Ia berjalan di atas air, mereka terkejut sekali. "Hantu!" teriak mereka ketakutan.
\par 27 Tetapi Yesus langsung menjawab, "Tenanglah! Aku Yesus. Jangan takut!"
\par 28 Lalu Petrus berkata, "Kalau Engkau memang Yesus, suruhlah saya datang berjalan di atas air."
\par 29 "Datanglah," jawab Yesus. Jadi Petrus turun dari perahu, berjalan di atas air dan datang kepada Yesus.
\par 30 Tetapi waktu Petrus melihat betapa besarnya angin di danau itu, ia takut dan mulai tenggelam. "Tuhan, tolong!" teriaknya.
\par 31 Yesus segera mengulurkan tangan-Nya dan menangkap dia, dan berkata, "Petrus, Petrus, kau ini kurang percaya. Mengapa kau ragu-ragu kepada-Ku?"
\par 32 Lalu keduanya naik ke perahu itu, dan angin pun reda.
\par 33 Maka pengikut-pengikut Yesus sujud menyembah Dia. Mereka berkata, "Sungguh Tuhan adalah Anak Allah!"
\par 34 Waktu mereka sampai di seberang danau, mereka mendarat di Genesaret.
\par 35 Dan ketika orang-orang di situ melihat bahwa yang datang itu Yesus, mereka menyiarkan berita itu ke semua daerah di sekitar kota itu. Lalu semua orang sakit dibawa kepada Yesus.
\par 36 Dan mereka mohon kepada-Nya supaya boleh menyentuh jubah-Nya, biar hanya ujungnya. Lalu semua yang menyentuhnya menjadi sembuh.

\chapter{15}

\par 1 Sekelompok orang Farisi dan beberapa guru agama dari Yerusalem datang kepada Yesus. Mereka bertanya kepada-Nya,
\par 2 "Mengapa pengikut-pengikut-Mu melanggar adat istiadat nenek moyang kita? Waktu akan makan, mereka tidak mencuci tangan lebih dahulu menurut peraturan!"
\par 3 Yesus menjawab, "Dan mengapa kalian juga melanggar perintah Allah, hanya karena mau mengikuti adat istiadat nenek moyangmu?
\par 4 Sebab Allah berkata, 'Hormatilah ayah dan ibumu,' dan 'Barangsiapa mengata-ngatai ayah ibunya, harus dihukum mati.'
\par 5 Tetapi kalian mengajarkan: kalau orang berkata kepada orang tuanya, 'Apa yang seharusnya saya berikan kepada ayah dan ibu, sudah saya persembahkan kepada Allah,'
\par 6 maka orang itu tidak lagi diwajibkan untuk menghormati orang tuanya. Jadi, karena adat istiadat nenek moyangmu, kalian meremehkan perkataan Allah.
\par 7 Kalian orang munafik! Tepat sekali apa yang dinubuatkan Yesaya tentang kalian, yaitu,
\par 8 'Begini kata Allah: Orang-orang itu hanya menyembah Aku dengan kata-kata, tetapi hati mereka jauh dari Aku.
\par 9 Percuma mereka menyembah Aku, sebab peraturan manusia mereka ajarkan seolah-olah itu peraturan-Ku.'"
\par 10 Kemudian Yesus memanggil orang-orang dan berkata kepada mereka, "Dengarlah supaya mengerti!
\par 11 Yang masuk ke mulut tidak membuat orang itu najis; hanya yang keluar dari mulutnya, itulah yang menjadikan dia najis."
\par 12 Lalu pengikut-pengikut Yesus datang dan berkata kepada-Nya, "Tahukah Bapak bahwa orang-orang Farisi itu tersinggung waktu mendengar Bapak berkata begitu?"
\par 13 Yesus menjawab, "Setiap tanaman yang tidak ditanam oleh Bapa-Ku di surga akan dicabut.
\par 14 Tidak usah hiraukan orang-orang Farisi itu. Mereka itu pemimpin-pemimpin buta; dan kalau orang buta memimpin orang buta, kedua-duanya akan jatuh ke dalam parit."
\par 15 Petrus berkata, "Tolong jelaskan perumpamaan itu kepada kami, Pak!"
\par 16 Yesus menjawab mereka, "Apa kalian juga belum mengerti?
\par 17 Tidak tahukah kalian, bahwa yang masuk ke dalam mulut turun ke dalam perut, dan kemudian keluar lagi?
\par 18 Tetapi yang keluar dari mulut, berasal dari hati; dan itulah yang membuat orang menjadi najis.
\par 19 Sebab dari hati timbul pikiran-pikiran jahat, yang menyebabkan orang membunuh, berzinah, berbuat cabul, mencuri, memberi kesaksian palsu dan memfitnah.
\par 20 Hal-hal itulah yang menyebabkan orang menjadi najis, dan bukan makan dengan tangan yang tidak dicuci."
\par 21 Kemudian Yesus meninggalkan tempat itu dan pergi ke daerah dekat kota Tirus dan Sidon.
\par 22 Seorang wanita Kanaan dari daerah itu, datang kepada Yesus sambil berseru-seru, "Anak Daud, kasihanilah saya! Anak perempuan saya kemasukan roh jahat. Keadaannya parah betul."
\par 23 Yesus tidak menjawab wanita itu sama sekali. Lalu pengikut-pengikut Yesus datang kepada-Nya dan memohon, "Pak, suruh wanita itu pergi. Dia hanya ribut-ribut saja di belakang kita!"
\par 24 Yesus menjawab, "Aku diutus hanya kepada bangsa Israel, khususnya kepada mereka yang sesat."
\par 25 Wanita itu datang lalu sujud di hadapan Yesus dan berkata, "Tolonglah saya, Tuan."
\par 26 Yesus menjawab, "Tidak baik mengambil makanan anak-anak dan melemparkannya kepada anjing."
\par 27 "Benar, Tuan," jawab wanita itu, "tetapi anjing pun makan sisa-sisa yang jatuh dari meja tuannya."
\par 28 Lalu Yesus berkata kepadanya, "Ibu, sungguh besar imanmu! Biarlah terjadi apa yang kauinginkan!" Pada saat itu juga anak wanita itu sembuh.
\par 29 Yesus meninggalkan tempat itu, lalu berjalan menyusur pantai Danau Galilea. Kemudian Ia naik ke sebuah bukit lalu duduk di situ.
\par 30 Banyak orang datang kepada-Nya membawa orang-orang yang timpang, yang buta, yang lumpuh, yang bisu, dan banyak lagi orang sakit yang lain. Mereka meletakkan orang-orang itu di depan Yesus, dan Ia menyembuhkan orang-orang itu.
\par 31 Orang banyak yang ada di situ heran sekali, waktu mereka melihat orang bisu dapat berbicara, orang timpang sembuh, orang lumpuh berjalan, dan orang buta melihat. Lalu mereka memuji-muji Allah bangsa Israel.
\par 32 Kemudian Yesus memanggil pengikut-pengikut-Nya lalu berkata, "Aku kasihan kepada orang banyak ini. Sudah tiga hari lamanya mereka bersama-sama Aku, dan sekarang mereka tidak punya makanan. Aku tidak mau membiarkan mereka pulang dengan perut kosong, nanti mereka pingsan di jalan."
\par 33 Lalu pengikut-pengikut Yesus bertanya kepada-Nya, "Di mana kita bisa mendapat makanan di daerah terpencil ini untuk orang sebanyak ini?"
\par 34 "Ada berapa roti pada kalian?" tanya Yesus kepada mereka. "Tujuh," jawab mereka, "dan ikan kecil beberapa ekor."
\par 35 Maka Yesus menyuruh orang banyak itu duduk di atas tanah.
\par 36 Lalu Ia mengambil ketujuh roti dan ikannya, kemudian mengucap syukur kepada Allah. Sesudah itu Ia membelah-belah roti dan ikan itu dengan tangan-Nya, lalu memberikan-Nya kepada pengikut-pengikut-Nya. Dan mereka membagi-bagikannya kepada orang banyak itu.
\par 37 Semua orang itu makan sampai kenyang. Setelah itu pengikut-pengikut Yesus mengumpulkan kelebihan makanan itu. Semuanya ada tujuh bakul penuh.
\par 38 Jumlah orang laki-laki saja yang makan pada waktu itu ada empat ribu orang, belum terhitung wanita dan anak-anak.
\par 39 Kemudian Yesus menyuruh orang banyak itu pulang. Ia sendiri naik sebuah perahu dan pergi ke daerah Magadan.

\chapter{16}

\par 1 Ada beberapa orang Farisi dan orang Saduki datang kepada Yesus hendak menjebak Dia. Mereka minta Dia membuat keajaiban untuk membuktikan bahwa Ia datang dari Allah.
\par 2 Yesus menjawab, ("Kalau matahari sedang terbenam, kalian berkata, 'Cuaca baik, sebab langitnya merah.'
\par 3 Dan pada pagi hari, kalian berkata, 'Hari ini akan hujan, sebab langit merah dan mendung.' Nah, kalian dapat meramalkan cuaca dengan melihat keadaan langit, padahal tanda-tanda zaman tidak bisa kalian ramalkan!")
\par 4 "Alangkah jahatnya dan durhakanya orang-orang zaman ini. Kalian minta Aku membuat keajaiban. Kalian tidak akan diberi satu keajaiban pun kecuali keajaiban Nabi Yunus." Setelah berkata begitu Yesus meninggalkan mereka, lalu pergi.
\par 5 Waktu pengikut-pengikut Yesus sampai di seberang danau, baru ketahuan oleh mereka bahwa mereka lupa membawa roti.
\par 6 Yesus berkata kepada mereka, "Berhati-hatilah terhadap ragi orang-orang Farisi dan Saduki."
\par 7 Mendengar kata-kata Yesus itu, pengikut-pengikut-Nya mulai berkata satu sama lain, "Ia berkata begitu karena kita tidak membawa roti."
\par 8 Yesus tahu apa yang mereka bicarakan. Jadi Ia berkata, "Mengapa kalian persoalkan tentang tidak punya roti? Kalian kurang percaya!
\par 9 Masih belum mengertikah kalian? Apakah kalian tidak ingat akan lima roti yang Aku belah-belah untuk lima ribu orang? Berapa bakul kelebihan roti yang kalian kumpulkan?
\par 10 Dan bagaimana pula dengan tujuh roti untuk empat ribu orang itu? Berapa bakul kelebihan makanan yang kalian kumpulkan?
\par 11 Mengapa sampai kalian tidak mengerti bahwa Aku bukannya berbicara dengan kalian mengenai roti? Berhati-hatilah terhadap ragi orang-orang Farisi dan Saduki!"
\par 12 Akhirnya pengikut-pengikut Yesus itu mengerti bahwa Yesus menyuruh mereka berhati-hati bukannya terhadap ragi yang dipakai untuk membuat roti, tetapi terhadap pengajaran-pengajaran orang-orang Farisi dan Saduki.
\par 13 Yesus pergi ke daerah dekat kota Kaisarea Filipi. Di situ Ia bertanya kepada pengikut-pengikut-Nya, "Menurut kata orang, Anak Manusia itu siapa?"
\par 14 Pengikut-pengikut-Nya menjawab, "Ada yang berkata Yohanes Pembaptis. Ada juga yang berkata Elia. Yang lain lagi berkata Yeremia, atau salah seorang nabi."
\par 15 "Tetapi menurut kalian sendiri, Aku ini siapa?" tanya Yesus kepada mereka.
\par 16 Simon Petrus menjawab, "Bapak adalah Raja Penyelamat, Anak Allah Yang Hidup."
\par 17 "Beruntung sekali engkau, Simon anak Yona!" kata Yesus. "Sebab bukannya manusia yang memberitahukan hal itu kepadamu, melainkan Bapa-Ku yang di surga.
\par 18 Sebab itu ketahuilah, engkau adalah Petrus, batu yang kuat. Dan di atas alas batu inilah Aku akan membangun gereja-Ku, yang tidak dapat dikalahkan; sekalipun oleh maut!
\par 19 Aku akan memberikan kepadamu kunci dari Dunia Baru Allah. Apa yang engkau larang di atas bumi, juga dilarang di surga. Dan apa yang engkau benarkan di atas bumi, juga dibenarkan di surga."
\par 20 Setelah itu Yesus mengingatkan pengikut-pengikut-Nya supaya mereka tidak memberitahukan kepada siapa pun bahwa Dialah Raja Penyelamat.
\par 21 Mulai dari saat itu, Yesus berkata terus terang kepada pengikut-pengikut-bahwa Ia harus pergi ke Yerusalem dan mengalami banyak penderitaan dari pemimpin-pemimpin, imam-imam kepala dan guru-guru agama. Ia akan dibunuh, tetapi pada hari ketiga Ia akan dibangkitkan kembali.
\par 22 Lalu Petrus menarik Yesus ke samping dan menegur Dia, katanya, "Mudah-mudahan Allah menjauhkan hal itu, Tuhan! Jangan sampai hal itu terjadi pada Tuhan!"
\par 23 Yesus menoleh lalu berkata kepada Petrus, "Pergi dari sini, Penggoda! Engkau menghalang-halangi Aku. Pikiranmu itu pikiran manusia; bukan pikiran Allah!"
\par 24 Kemudian Yesus berkata kepada pengikut-pengikut-Nya, "Orang yang mau mengikuti Aku, harus melupakan kepentingannya sendiri, memikul salibnya, dan terus mengikuti Aku.
\par 25 Sebab orang yang mau mempertahankan hidupnya, akan kehilangan hidupnya. Tetapi orang yang mengurbankan hidupnya untuk Aku, akan mendapatkannya.
\par 26 Apa untungnya bagi seseorang, kalau seluruh dunia ini menjadi miliknya tetapi ia kehilangan hidupnya? Dapatkah hidup itu ditukar dengan sesuatu?
\par 27 Tidak lama lagi Anak Manusia, bersama-sama dengan malaikat-malaikat-Nya akan datang dengan kuasa Bapa-Nya. Pada waktu itu Ia akan membalas tiap-tiap orang sesuai dengan perbuatannya.
\par 28 Ketahuilah! Dari antara kalian di sini ada yang tidak akan mati, sebelum melihat Anak Manusia datang sebagai Raja."

\chapter{17}

\par 1 Enam hari kemudian Yesus membawa Petrus dengan Yakobus dan saudaranya Yohanes, menyendiri ke sebuah gunung yang tinggi.
\par 2 Di depan mereka, Yesus berubah rupa: Muka-Nya menjadi terang seperti matahari dan pakaian-Nya putih berkilauan.
\par 3 Kemudian ketiga pengikut-Nya itu melihat Musa dan Elia berbicara dengan Yesus.
\par 4 Lalu Petrus berkata kepada Yesus, "Tuhan, enak sekali kita di sini. Kalau Tuhan suka, saya akan mendirikan tiga kemah di sini: satu untuk Tuhan, satu untuk Musa, dan satu lagi untuk Elia."
\par 5 Sementara Petrus masih berbicara, awan yang terang sekali meliputi mereka dan dari awan itu terdengar suara yang berkata, "Inilah Anak-Ku yang Kukasihi. Ia menyenangkan hati-Ku. Dengarkan Dia!"
\par 6 Waktu pengikut-pengikut Yesus mendengar suara itu, mereka begitu ketakutan sehingga tersungkur ke tanah.
\par 7 Tetapi Yesus datang dan menyentuh mereka. "Bangunlah," kata-Nya, "jangan takut!"
\par 8 Ketika mereka memandang, mereka tidak melihat siapa pun, kecuali Yesus saja.
\par 9 Waktu mereka turun dari gunung itu, Yesus memperingatkan mereka, "Jangan memberitahukan kepada siapa pun apa yang kalian lihat tadi, sebelum Anak Manusia dibangkitkan dari kematian."
\par 10 Pengikut-pengikut Yesus bertanya kepada-Nya, "Mengapa guru-guru agama berkata bahwa Elia harus datang terlebih dahulu?"
\par 11 Yesus menjawab, "Memang Elia datang dan ia akan membereskan segala sesuatu.
\par 12 Tetapi Aku berkata kepadamu, Elia sudah datang, namun orang-orang tidak mengenal dia; mereka memperlakukan dia semau mereka saja. Dan begitu juga mereka akan memperlakukan Anak Manusia."
\par 13 Lalu para pengikut-Nya mengerti bahwa Yesus bicara tentang Yohanes Pembaptis.
\par 14 Ketika Yesus dan ketiga pengikut-Nya kembali kepada orang banyak itu, seorang laki-laki datang, sujud di hadapan Yesus,
\par 15 dan berkata, "Bapak, kasihanilah anak saya. Ia sakit ayan. Serangan ayannya begitu hebat sehingga ia sering sekali jatuh ke dalam api dan sering juga ke dalam air.
\par 16 Saya sudah membawa dia kepada pengikut-pengikut Bapak, tetapi mereka tidak dapat menyembuhkan dia."
\par 17 Yesus menjawab, "Bukan main kalian ini! Kalian sungguh orang-orang yang menyeleweng dan tidak percaya. Sampai kapan Aku harus tinggal bersama kalian dan bersabar terhadap kalian? Bawa anak itu kemari!"
\par 18 Lalu Yesus memerintahkan roh jahat yang di dalam anak itu keluar. Dan roh jahat itu keluar, lalu anak itu sembuh pada saat itu juga.
\par 19 Kemudian pengikut-pengikut Yesus datang kepada-Nya secara tersendiri dan bertanya kepada-Nya, "Pak, apa sebab kami tidak dapat mengusir roh jahat itu?"
\par 20 Yesus menjawab, "Sebab kalian kurang percaya. Ingatlah! Kalau kalian mempunyai iman sebesar biji sawi, kalian dapat berkata kepada bukit ini, 'Pindahlah ke sana!' pasti bukit ini akan pindah. Tidak ada sesuatu pun yang tidak dapat kalian buat!
\par 21 (Tetapi roh jahat yang semacam ini, hanya bisa diusir oleh doa dan puasa saja.)"
\par 22 Ketika pengikut-pengikut Yesus berkumpul di Galilea, Yesus berkata kepada mereka, "Tidak lama lagi, Anak Manusia akan diserahkan kepada kuasa manusia.
\par 23 Ia akan dibunuh, tetapi pada hari ketiga Ia akan dibangkitkan." Maka pengikut-pengikut-Nya menjadi sedih sekali.
\par 24 Waktu Yesus dan pengikut-pengikut-Nya sampai di Kapernaum, penagih-penagih pajak Rumah Tuhan datang kepada Petrus dan bertanya, "Gurumu membayar pajak Rumah Tuhan atau tidak?"
\par 25 Petrus menjawab, "Bayar!" Ketika Petrus pulang, Yesus menyapa dia lebih dahulu, "Simon, bagaimana pendapatmu? Siapa yang harus membayar bea atau pajak kepada raja-raja dunia ini? Rakyatkah atau orang asing?"
\par 26 "Orang asing," jawab Petrus. "Kalau begitu," kata Yesus, "rakyat tidak perlu membayar.
\par 27 Tetapi janganlah kita menyinggung perasaan orang-orang itu. Jadi, pergilah ke danau untuk memancing. Ambillah ikan pertama yang kautangkap. Di dalam mulutnya itu engkau akan menemukan mata uang yang cukup untuk pajak kita berdua. Ambillah uang itu dan bayarlah kepada mereka pajak kita untuk Rumah Tuhan."

\chapter{18}

\par 1 Pada waktu itu pengikut-pengikut Yesus datang kepada-Nya dan bertanya, "Siapa yang dianggap terbesar di antara umat Allah?"
\par 2 Yesus memanggil seorang anak kecil, dan membuat dia berdiri di depan mereka.
\par 3 Lalu Yesus berkata, "Percayalah! Hanya kalau kalian berubah dan menjadi seperti anak-anak, kalian akan menjadi anggota umat Allah.
\par 4 Orang yang merendahkan dirinya dan menjadi seperti anak ini, dialah yang terbesar di antara umat Allah.
\par 5 Dan orang yang menerima anak yang seperti ini karena Aku, berarti menerima Aku."
\par 6 "Siapa menyebabkan salah satu dari orang-orang yang kecil ini tidak percaya lagi kepada-Ku, lebih baik kalau batu penggilingan diikatkan pada lehernya dan ia ditenggelamkan di laut yang dalam.
\par 7 Alangkah celakanya dunia ini karena hal-hal yang menyebabkan orang berdosa. Memang hal-hal seperti itu akan selalu ada, tetapi celakalah orang yang menyebabkannya!
\par 8 Kalau tanganmu atau kakimu membuat engkau berdosa, potonglah dan buanglah. Lebih baik engkau hidup dengan Allah tanpa sebelah tangan atau kaki, daripada engkau dibuang ke dalam api neraka dengan kedua tangan dan kakimu.
\par 9 Dan kalau matamu menyebabkan engkau berbuat dosa, cungkillah dan buanglah. Lebih baik engkau hidup dengan Allah tanpa satu mata daripada dibuang ke dalam api neraka dengan kedua mata."
\par 10 "Awas! Jangan menghina salah satu dari orang-orang yang kecil ini. Sebab ingatlah, malaikat-malaikat mereka selalu ada di hadapan Bapa-Ku di surga.
\par 11 (Sebab Anak Manusia datang untuk menyelamatkan orang yang sesat!)
\par 12 Bagaimanakah pendapatmu? Seandainya ada seorang yang mempunyai seratus ekor domba, lalu seekor dari domba-domba itu hilang, apakah yang akan dibuat oleh orang itu? Pasti ia akan meninggalkan domba yang sembilan puluh sembilan ekor itu di bukit dan pergi mencari yang hilang itu.
\par 13 Dan kalau ia menemukan kembali domba itu--percayalah Aku--ia akan lebih gembira atas domba yang seekor itu daripada atas sembilan puluh sembilan ekor lainnya yang tidak hilang.
\par 14 Begitu juga Bapamu yang di surga tidak mau salah seorang dari orang-orang yang baru percaya kepada-Ku ini sesat."
\par 15 "Kalau saudaramu berdosa terhadapmu, pergilah kepadanya dan tunjukkanlah kesalahannya. Lakukanlah itu dengan diam-diam antara kalian berdua saja. Kalau ia menurut kata-katamu, maka berhasillah engkau mendapat saudaramu itu kembali.
\par 16 Tetapi kalau tidak, bawalah satu atau dua orang lagi. Sebab dalam Alkitab tertulis, 'Sekurang-kurangnya dua atau tiga saksi diperlukan untuk menyatakan seorang tertuduh bersalah.'
\par 17 Kalau ia tidak menerima nasihat orang-orang itu, beritahukanlah hal itu kepada jemaat. Dan akhirnya, kalau ia tidak mau menerima nasihat jemaat, anggaplah ia sebagai penagih pajak dan seorang yang tidak mengenal Allah."
\par 18 "Ketahuilah: Apa yang kalian larang di dunia, juga dilarang di surga. Dan apa yang kalian benarkan di dunia, juga dibenarkan di surga.
\par 19 Dan ketahuilah juga: Kalau di antara kalian di dunia ini dua orang sepakat mengenai apa saja dan mendoakannya, doa itu akan dikabulkan oleh Bapa-Ku di surga.
\par 20 Sebab di mana dua atau tiga orang berkumpul karena Aku, Aku berada di tengah-tengah mereka."
\par 21 Kemudian Petrus datang kepada Yesus dan bertanya, "Tuhan, kalau saudara saya berdosa terhadap saya, sampai berapa kali saya harus mengampuni dia? Sampai tujuh kalikah?"
\par 22 Yesus menjawab, "Tidak, bukan sampai tujuh kali, tetapi tujuh puluh kali tujuh kali!
\par 23 Sebab apabila Allah memerintah, keadaannya seperti dalam perumpamaan ini: Seorang raja mau menyelesaikan utang-utang hamba-hambanya.
\par 24 Waktu ia mulai mengadakan pemeriksaan, dihadapkan kepadanya seorang hamba yang berutang berjuta-juta,
\par 25 dan tidak dapat melunasinya. Jadi, raja itu memerintahkan supaya hamba itu dijual bersama-sama dengan anak istrinya, dan segala harta miliknya untuk membayar utangnya.
\par 26 Hamba itu sujud di depan raja itu, dan memohon, 'Tuan, sabarlah terhadap hamba. Hamba akan melunasi semua utang hamba.'
\par 27 Raja itu kasihan kepadanya, sehingga ia menghapuskan semua utangnya.
\par 28 Ketika hamba itu keluar, ia berjumpa dengan kawannya, seorang hamba juga, yang berutang kepadanya beberapa ribu. Ia menangkap kawannya itu, mencekiknya, dan berkata, 'Bayarlah semua utangmu!'
\par 29 Lalu kawannya itu sujud di depannya sambil memohon, 'Sabarlah dulu kawan, semuanya akan saya bayar!'
\par 30 Tetapi hamba itu menolak. Sebaliknya, ia memasukkan dia ke dalam penjara sampai ia membayar utangnya.
\par 31 Ketika hamba-hamba yang lain melihat apa yang sudah terjadi, mereka sedih dan melaporkan hal itu kepada raja.
\par 32 Maka raja itu memanggil hamba yang jahat itu dan berkata kepadanya, 'Hamba yang jahat! Seluruh utangmu sudah kuhapuskan hanya karena engkau memohon kepadaku.
\par 33 Bukankah engkau pun harus menaruh kasihan kepada kawanmu seperti aku pun sudah menaruh kasihan kepadamu?'
\par 34 Raja itu sangat marah. Hamba yang jahat itu dimasukkannya ke dalam penjara sampai ia melunasi semua utangnya."
\par 35 Yesus mengakhiri cerita-Nya dengan kata-kata ini, "Begitu juga Bapa-Ku di surga akan memperlakukan kalian masing-masing, kalau kalian tidak mengampuni saudaramu dengan ikhlas."

\chapter{19}

\par 1 Sesudah Yesus selesai mengatakan semuanya itu, Ia meninggalkan Galilea lalu pergi ke daerah Yudea di seberang Sungai Yordan.
\par 2 Banyak orang mengikuti Dia, dan Ia menyembuhkan mereka di situ.
\par 3 Lalu orang-orang Farisi datang untuk menjebak Dia. Mereka bertanya, "Menurut hukum agama kita, apakah boleh orang menceraikan istrinya dengan alasan apa saja?"
\par 4 Yesus menjawab, "Apakah kalian belum membaca dalam Alkitab bahwa Pencipta yang membuat manusia, pada mulanya membuat mereka laki-laki dan wanita?
\par 5 Dan sesudah itu Ia berkata, 'Itu sebabnya laki-laki meninggalkan ibu bapaknya dan bersatu dengan istrinya, maka keduanya menjadi satu.'
\par 6 Jadi mereka bukan lagi dua orang, tetapi satu. Itu sebabnya apa yang sudah disatukan oleh Allah tidak boleh diceraikan oleh manusia."
\par 7 Lalu orang-orang Farisi bertanya kepada-Nya, "Kalau begitu mengapa Musa menyuruh orang memberi surat cerai kepada istri yang diceraikannya?"
\par 8 Yesus menjawab, "Musa mengizinkan kalian menceraikan istrimu sebab kalian terlalu susah diajar. Tetapi sebenarnya bukan begitu pada mulanya.
\par 9 Jadi, dengarlah ini: Siapa menceraikan istrinya--padahal wanita itu tidak menyeleweng--kemudian kawin lagi dengan wanita yang lain, orang itu berzinah."
\par 10 Maka pengikut-pengikut Yesus berkata kepada-Nya, "Kalau soal hubungan suami istri adalah seperti itu, lebih baik tidak usah kawin."
\par 11 Yesus menjawab, "Tidak semua orang bisa menerima kata-kata itu, hanya orang-orang yang sudah ditentukan oleh Allah.
\par 12 Karena ada orang yang tidak dapat kawin, sebab mereka memang lahir begitu. Ada juga yang tidak dapat kawin sebab ia dibuat begitu oleh orang lain. Dan ada pula yang memilih sendiri untuk tidak kawin, supaya dapat melayani Allah. Orang yang sanggup menerima pengajaran ini, biarlah ia menerimanya."
\par 13 Ada orang-orang membawa anak-anak kepada Yesus supaya Ia meletakkan tangan-Nya ke atas kepala mereka dan mendoakan mereka. Tetapi pengikut-pengikut Yesus memarahi orang-orang itu.
\par 14 Maka Yesus berkata kepada pengikut-pengikut-Nya, "Biarkan anak-anak itu datang kepada-Ku! Jangan melarang mereka, sebab orang-orang seperti inilah yang menjadi anggota umat Allah."
\par 15 Lalu Yesus meletakkan tangan-Nya di atas kepala anak-anak itu dan memberkati mereka. Kemudian Ia pergi dari situ.
\par 16 Pada suatu hari seorang laki-laki datang kepada Yesus. "Pak Guru," katanya, "perbuatan baik apa yang harus kulakukan supaya dapat menerima hidup sejati dan kekal?"
\par 17 Yesus menjawab, "Mengapa engkau bertanya kepada-Ku mengenai apa yang baik? Hanya ada Satu yang baik. Kalau engkau ingin hidup, engkau harus taat kepada perintah-perintah Allah."
\par 18 "Perintah yang mana itu?" tanya orang itu. Yesus menjawab, "Jangan membunuh, jangan berzinah, jangan mencuri, jangan bersaksi dusta,
\par 19 hormatilah ayah dan ibumu; dan kasihilah sesamamu manusia seperti engkau mengasihi dirimu sendiri."
\par 20 "Semua perintah itu sudah kuturuti," jawab orang muda itu. "Apa lagi yang perlu?"
\par 21 Yesus berkata kepadanya, "Kalau engkau ingin menjadi sempurna, pergilah jual semua milikmu. Berikanlah uangnya kepada orang miskin, dan engkau akan mendapat harta di surga. Sesudah itu, datanglah mengikuti Aku!"
\par 22 Mendengar kata-kata itu, orang muda itu pergi dengan susah hati sebab ia kaya sekali.
\par 23 Lalu Yesus berkata kepada pengikut-pengikut-Nya, "Aku berkata kepadamu, sukar sekali untuk orang kaya menjadi anggota umat Allah.
\par 24 Dan ini lagi yang mau Kukatakan kepadamu; lebih mudah seekor unta masuk lubang jarum, daripada seorang kaya masuk Dunia Baru Allah."
\par 25 Ketika pengikut-pengikut-Nya mendengar perkataan Yesus itu, mereka heran. "Kalau begitu," kata mereka, "siapa yang bisa selamat?"
\par 26 Yesus memandang mereka lalu berkata, "Untuk manusia, itu mustahil! Tetapi untuk Allah, semua mungkin."
\par 27 Lalu Petrus berkata, "Lihatlah, kami sudah meninggalkan segala-galanya untuk mengikuti Bapak. Dan apa yang akan kami terima?"
\par 28 Yesus berkata kepada mereka, "Percayalah: Di Dunia yang Baru, Anak Manusia akan duduk di atas takhta-Nya yang mulia. Pada waktu itu kalian pengikut-pengikut-Ku akan duduk di atas dua belas takhta dan menghakimi kedua belas suku bangsa Israel!
\par 29 Dan setiap orang yang sudah meninggalkan rumah, atau saudara laki-laki atau perempuan, atau ibu bapak, atau anak-anak, atau ladang karena Aku, orang itu akan menerima kembali seratus kali lipat. Dan ia akan menerima juga hidup sejati dan kekal.
\par 30 Tetapi banyak orang yang sekarang ini pertama akan menjadi yang terakhir dan yang sekarang ini terakhir akan menjadi yang pertama."

\chapter{20}

\par 1 "Apabila Allah memerintah, keadaannya seperti perumpamaan ini: Seorang pemilik kebun anggur keluar pada waktu pagi mencari orang untuk bekerja di kebun anggurnya.
\par 2 Sesudah ia setuju membayar mereka satu uang perak sehari, ia menyuruh mereka bekerja di kebun anggurnya.
\par 3 Pukul sembilan pagi, pemilik kebun itu pergi lagi, dan melihat beberapa orang sedang menganggur di pasar.
\par 4 Maka berkatalah ia kepada mereka, 'Pergilah kalian bekerja di kebun anggur saya. Saya akan membayar kalian dengan upah yang layak.'
\par 5 Mereka pun pergi. Pukul dua belas tengah hari dan juga pukul tiga petang, pemilik kebun itu keluar lagi dan berbuat hal yang sama.
\par 6 Dan kira-kira pukul lima sore, ia keluar lagi dan melihat masih ada orang yang menganggur. Lalu ia bertanya kepada mereka, 'Mengapa kalian berdiri terus seharian di sini dengan tidak melakukan apa-apa?'
\par 7 Orang-orang itu menjawab, 'Tidak ada orang yang mempekerjakan kami.' 'Kalau begitu,' kata pemilik kebun itu, 'pergilah kalian bekerja di kebun anggur saya.'
\par 8 Petang hari, pemilik kebun anggur itu berkata kepada mandurnya, 'Panggillah pekerja-pekerja itu dan bayarlah upah mereka masing-masing, mulai dari orang yang masuk terakhir sampai kepada yang masuk pertama.'
\par 9 Lalu datanglah pekerja-pekerja yang mulai bekerja dari pukul lima petang. Mereka masing-masing menerima satu uang perak.
\par 10 Kemudian datang pula pekerja-pekerja yang disewa sejak pagi. Mereka berpikir mereka akan menerima lebih. Padahal mereka hanya diberi satu uang perak juga.
\par 11 Waktu menerima uang itu, mereka menggerutu terhadap pemilik kebun:
\par 12 'Pekerja-pekerja yang datang terakhir itu cuma bekerja satu jam. Sedangkan kami bekerja seharian di bawah panas terik matahari, namun Tuan membayar mereka sama dengan kami!'
\par 13 Pemilik kebun anggur itu menjawab kepada salah seorang dari mereka, 'Kawan, saya tidak bersalah terhadapmu. Bukankah engkau setuju menerima upah satu uang perak untuk pekerjaan sehari?
\par 14 Nah, ambillah upahmu, dan pergilah. Saya memang ingin memberi kepada orang yang masuk terakhir itu upah yang sama dengan yang saya berikan kepadamu.
\par 15 Apakah saya tidak boleh berbuat semau saya dengan kepunyaan saya? Ataukah engkau iri, karena saya bermurah hati?'"
\par 16 Lalu Yesus berkata lagi, "Begitu juga orang-orang yang terakhir akan menjadi yang pertama, dan orang-orang yang pertama akan menjadi yang terakhir."
\par 17 Ketika Yesus menuju ke Yerusalem, di tengah jalan Ia memanggil kedua belas pengikut-Nya berkumpul tersendiri. Lalu Ia berkata kepada mereka,
\par 18 "Dengarkan! Kita sekarang menuju Yerusalem. Di sana Anak Manusia akan diserahkan kepada imam-imam kepala dan guru-guru agama. Lalu Ia akan dihukum mati,
\par 19 kemudian diserahkan kepada orang-orang bukan Yahudi. Mereka akan mengolok-olok Dia, menyiksa Dia, dan menyalibkan Dia. Tetapi pada hari ketiga, Ia akan dibangkitkan kembali."
\par 20 Kemudian istri Zebedeus datang dengan anak-anaknya kepada Yesus. Di hadapan Yesus ia sujud untuk minta sesuatu.
\par 21 "Ibu mau apa?" tanya Yesus. Ibu itu menjawab, "Saya ingin kedua anak saya ini duduk di kiri dan kanan Bapak apabila Bapak menjadi Raja nanti."
\par 22 "Kalian tidak tahu apa yang kalian minta," kata Yesus kepada mereka. "Sanggupkah kalian minum dari piala penderitaan yang harus Aku minum?" "Sanggup," jawab mereka.
\par 23 Yesus berkata, "Memang kalian akan minum juga dari piala-Ku. Tetapi mengenai siapa yang akan duduk di kiri dan kanan-Ku, itu bukan Aku yang berhak menentukan. Tempat-tempat itu adalah untuk orang-orang yang sudah ditentukan oleh Bapa-Ku."
\par 24 Waktu kesepuluh pengikut yang lain mendengar hal itu, mereka marah kepada kedua orang bersaudara itu.
\par 25 Sebab itu Yesus memanggil mereka semua lalu berkata, "Kalian tahu bahwa pemimpin-pemimpin bangsa yang tidak mengenal Allah menindas rakyatnya. Dan pembesar-pembesar mereka menekan mereka.
\par 26 Tetapi kalian tidak boleh begitu. Sebaliknya, orang yang mau menjadi besar di antara kalian, harus menjadi pelayanmu.
\par 27 Dan orang yang mau menjadi yang pertama di antara kalian, harus menjadi hambamu.
\par 28 Sama seperti Anak Manusia itu juga; Ia datang bukan untuk dilayani, tetapi untuk melayani, dan menyerahkan nyawa-Nya untuk membebaskan banyak orang."
\par 29 Waktu mereka meninggalkan Yerikho, banyak orang mengikuti Yesus.
\par 30 Dua orang buta sedang duduk di pinggir jalan. Waktu mereka mendengar bahwa Yesus lewat, mereka berteriak, "Anak Daud, kasihanilah kami!"
\par 31 Orang banyak yang ada di situ memarahi mereka dan menyuruh mereka diam. Tetapi mereka berteriak lebih nyaring lagi, "Tuan, Anak Daud, kasihanilah kami!"
\par 32 Lalu Yesus berhenti dan memanggil mereka. "Apa yang kalian ingin Aku perbuat bagimu?" tanya Yesus.
\par 33 Mereka menjawab, "Tuan, kami ingin melihat!"
\par 34 Yesus kasihan kepada mereka, lalu menjamah mata mereka. Saat itu juga mereka dapat melihat. Lalu mereka mengikuti Yesus.

\chapter{21}

\par 1 Waktu sudah dekat Yerusalem, mereka sampai di Betfage di Bukit Zaitun. Di situ Yesus menyuruh dua orang pengikut-Nya berjalan lebih dahulu.
\par 2 "Pergilah ke kampung yang di depan itu," pesan Yesus kepada mereka. "Segera kalian akan melihat seekor keledai terikat bersama anaknya. Lepaskanlah keduanya dan bawa kemari.
\par 3 Kalau ada orang menanyakan sesuatu, katakan kepada orangnya, 'Tuhan memerlukannya', maka orang itu dengan segera akan membiarkan keledai itu dibawa."
\par 4 Hal itu demikian supaya terjadilah apa yang dikatakan oleh nabi sebagai berikut,
\par 5 "Katakanlah kepada Sion, Rajamu sedang datang kepadamu. Ia rendah hati dan menunggang seekor keledai, seekor anak keledai yang muda."
\par 6 Kemudian kedua pengikut-Nya itu pergi dan melakukan seperti yang dipesankan Yesus kepada mereka.
\par 7 Mereka membawa keledai itu dengan anaknya. Lalu mereka mengalasi punggung keledai-keledai itu dengan jubah mereka. Kemudian Yesus naik.
\par 8 Banyak orang di sana membentangkan jubah-jubah mereka di jalan, sedang orang-orang lain memotong ranting-ranting pohon dan menyebarkannya di tengah jalan.
\par 9 Orang banyak yang berjalan di depan dan di belakang Yesus berseru-seru, "Hidup Anak Daud! Diberkatilah Dia yang datang atas nama Tuhan! Pujilah Allah Yang Mahatinggi!"
\par 10 Lalu, waktu Yesus masuk Yerusalem, seluruh kota itu menjadi gempar. "Ini siapa?" tanya orang-orang di kota itu.
\par 11 "Dia Nabi Yesus, dari Nazaret di Galilea," jawab orang banyak yang mengiringi Yesus.
\par 12 Kemudian Yesus masuk ke Rumah Tuhan, dan mengusir semua orang yang berjual beli di situ. Ia menjungkirbalikkan meja-meja penukar uang, dan bangku-bangku penjual burung merpati.
\par 13 Lalu Ia berkata kepada orang-orang itu, "Di dalam Alkitab tertulis bahwa Allah berkata, 'Rumah-Ku akan disebut rumah tempat berdoa.' Tetapi kalian menjadikannya sarang penyamun!"
\par 14 Orang-orang buta dan lumpuh datang kepada Yesus di Rumah Tuhan, dan Ia menyembuhkan mereka.
\par 15 Tetapi imam-imam kepala dan guru-guru agama marah melihat keajaiban-keajaiban yang dilakukan oleh Yesus. Dan mereka marah juga mendengar anak-anak bersorak-sorak di Rumah Tuhan, "Hidup Anak Daud!"
\par 16 Mereka berkata kepada Yesus, "Engkau dengar apa yang mereka katakan?" "Ya, Aku dengar," jawab Yesus. "Belum pernahkah kalian membaca ayat Alkitab ini: 'Anak-anak dan bayi sudah Kauajar untuk mengucapkan pujian yang sempurna'?"
\par 17 Kemudian Yesus meninggalkan mereka, lalu keluar dari kota itu ke Betania dan bermalam di sana.
\par 18 Pagi-pagi keesokan harinya, dalam perjalanan kembali ke kota, Yesus lapar.
\par 19 Ia melihat sebatang pohon ara di pinggir jalan. Ia pergi ke pohon itu, tetapi tidak menemukan satu buah pun kecuali daun-daun saja. Lalu Yesus berkata kepada pohon itu, "Engkau tidak akan berbuah lagi!" Langsung pohon ara itu layu.
\par 20 Pada waktu pengikut-pengikut Yesus melihat kejadian itu, mereka heran sekali. "Bagaimana pohon ara itu bisa langsung mati seluruhnya?" tanya mereka.
\par 21 "Sungguh," jawab Yesus, "kalau kalian percaya dan tidak ragu-ragu, kalian dapat melakukan apa yang sudah Kulakukan terhadap pohon ara ini. Dan bukan itu saja, malah kalian akan dapat berkata kepada bukit ini, 'Terangkatlah dan terbuanglah ke dalam laut'; maka hal itu akan terjadi.
\par 22 Apa saja yang kalian minta dalam doamu, kalian akan menerimanya, asal kalian percaya."
\par 23 Kemudian Yesus kembali ke Rumah Tuhan, lalu masuk dan mengajar di situ. Waktu Ia sedang mengajar, imam-imam kepala dan pemimpin-pemimpin Yahudi datang kepada-Nya dan bertanya, "Atas dasar apa Engkau melakukan semuanya itu? Siapa yang memberi hak itu kepada-Mu?"
\par 24 Yesus menjawab, "Aku juga mau bertanya kepada kalian. Dan kalau kalian menjawabnya, Aku akan mengatakan kepadamu dengan hak siapa Aku melakukan hal-hal ini.
\par 25 Yohanes membaptis dengan hak siapa? Allah atau manusia?" Lalu imam-imam kepala dan pemimpin-pemimpin Yahudi itu mulai berunding di antara mereka, "Kalau kita katakan, 'Dengan hak Allah,' Ia akan berkata, 'Kalau begitu mengapa kalian tidak percaya kepadanya?'
\par 26 Tetapi kalau kita katakan, 'Dengan hak manusia,' kita takut kepada orang banyak, sebab mereka semua menganggap Yohanes seorang nabi."
\par 27 Jadi mereka menjawab, "Kami tidak tahu." Lalu Yesus berkata kepada mereka, "Kalau begitu Aku pun tidak akan mengatakan kepadamu dengan hak siapa Aku melakukan semuanya ini."
\par 28 "Sekarang bagaimana pendapatmu tentang hal ini?" kata Yesus selanjutnya. "Adalah seorang ayah yang mempunyai dua anak laki-laki, ia datang kepada anaknya yang pertama dan berkata, 'Nak, pergilah bekerja di kebun anggur hari ini.'
\par 29 Anak itu menjawab, 'Baik, Ayah!' Tetapi ia tidak pergi.
\par 30 Lalu ayah itu datang kepada anaknya yang kedua dan mengatakan hal yang sama. Anak itu menjawab, 'Saya tidak mau,' tetapi kemudian berubah pikiran lalu pergi juga.
\par 31 Dari antara kedua anak itu, yang manakah yang melakukan kehendak ayahnya?" "Yang kedua," jawab imam-imam kepala dan pemimpin-pemimpin Yahudi itu.
\par 32 Karena Yohanes Pembaptis datang, dan menunjukkan kepada kalian cara hidup yang dikehendaki Tuhan, namun kalian tidak mau percaya pada ajarannya; tetapi penagih-penagih pajak dan wanita-wanita pelacur percaya kepadanya. Tetapi meskipun kalian sudah melihat semuanya itu, kalian tidak juga mengubah pikiranmu dan tidak percaya kepada Tuhan."
\par 33 "Dengarkan perumpamaan yang satu ini lagi," kata Yesus. "Seorang tuan tanah menanami sebidang kebun anggur. Ia memasang pagar di sekelilingnya, dan menggali lubang untuk alat pemeras anggur, kemudian mendirikan menara jaga. Sesudah itu ia menyewakan kebun anggur itu kepada penggarap-penggarap, lalu berangkat ke negeri lain.
\par 34 Ketika sudah sampai musim petik buah anggur, tuan tanah itu mengirim pelayan-pelayannya kepada penggarap-penggarap kebun itu untuk menerima bagiannya.
\par 35 Tetapi penggarap-penggarap kebun itu menangkap pelayan-pelayan tuan tanah itu: Yang seorang dipukul, yang lain dibunuh, dan yang lain lagi dilempari batu.
\par 36 Tuan tanah itu mengirim lagi pelayan-pelayan lain, lebih banyak dari yang pertama kalinya. Tetapi mereka diperlakukan dengan cara yang sama.
\par 37 Akhirnya tuan tanah itu mengirim kepada mereka anaknya sendiri. 'Pasti anak saya akan dihormati,' pikirnya.
\par 38 Tetapi ketika penggarap-penggarap kebun itu melihat anak tuan tanah itu, mereka berkata satu sama lain, 'Nah, ini dia, ahli warisnya. Mari kita bunuh dia, supaya kita mendapat warisannya!'
\par 39 Maka anak itu ditangkap, dibuang ke luar, lalu dibunuh."
\par 40 Yesus bertanya, "Nah, kalau pemilik kebun anggur itu kembali, ia akan berbuat apa terhadap penggarap-penggarap itu?"
\par 41 Mereka menjawab, "Pasti ia akan membunuh orang-orang jahat itu, lalu menyewakan kebun anggur itu kepada orang lain yang mau memberi bagian hasil tanah itu kepadanya pada waktunya."
\par 42 Maka Yesus berkata kepada mereka, "Belum pernahkah kalian membaca yang tertulis dalam Alkitab? 'Batu yang tidak terpakai oleh tukang bangunan sudah menjadi batu yang terutama. Inilah perbuatan Tuhan; alangkah indahnya!'"
\par 43 "Jadi ingatlah," kata Yesus, "semua hak sebagai umat Allah akan dicabut daripadamu dan diberikan kepada suatu bangsa yang akan menjalankan perintah-perintah Allah.
\par 44 (Orang yang jatuh pada batu itu, akan hancur; dan orang yang ditimpa batu itu, akan tergilas menjadi debu.)"
\par 45 Pada waktu imam-imam kepala dan orang-orang Farisi mendengar perumpamaan-perumpamaan Yesus itu, mereka tahu bahwa Yesus berbicara tentang mereka.
\par 46 Jadi mereka berusaha menangkap Dia. Tetapi mereka takut kepada orang banyak, sebab orang banyak itu menganggap Yesus seorang nabi.

\chapter{22}

\par 1 Yesus berbicara lagi kepada orang banyak dengan memakai perumpamaan,
\par 2 kata-Nya, "Apabila Allah memerintah, keadaannya seperti perumpamaan ini: Seorang raja mengadakan pesta kawin untuk putranya.
\par 3 Raja itu menyuruh pelayan-pelayannya pergi menjemput orang-orang yang diundang ke pesta itu. Tetapi para undangan itu tidak mau datang.
\par 4 Kemudian raja itu mengutus lagi pelayan-pelayannya yang lain. Katanya kepada mereka: 'Beritahukan kepada para undangan itu: Hidangan pesta sudah siap. Sapi, dan anak-anak sapi saya yang terbaik sudah disembelih. Semuanya sudah siap. Silakan datang ke pesta kawin!'
\par 5 Tetapi tamu-tamu yang diundang itu tidak menghiraukannya. Mereka pergi ke pekerjaannya masing-masing--yang seorang ke ladangnya, yang lainnya ke perusahaannya;
\par 6 dan yang lainnya pula menangkap pelayan-pelayan raja itu, lalu memukul dan membunuh mereka.
\par 7 Waktu raja itu mendengar hal itu, ia marah sekali. Ia mengirim tentaranya untuk membunuh pembunuh-pembunuh itu, dan membakar kota mereka.
\par 8 Sesudah itu ia memanggil pelayan-pelayannya, lalu berkata, 'Pesta kawin sudah siap, tetapi para undangan tidak layak.
\par 9 Pergilah sekarang ke jalan-jalan raya, dan undanglah sebanyak mungkin orang ke pesta kawin ini.'
\par 10 Maka pelayan-pelayan itu pun pergilah. Mereka pergi ke jalan-jalan raya lalu mengumpulkan semua orang yang mereka jumpai di sana, yang baik maupun yang jahat. Maka penuhlah ruangan pesta kawin itu dengan tamu-tamu.
\par 11 Kemudian raja itu masuk untuk melihat-lihat para tamu. Ia melihat ada seorang di pesta itu yang tidak memakai pakaian pesta.
\par 12 Lalu ia bertanya kepada orang itu, 'Kawan, bagaimanakah engkau bisa masuk ke sini tanpa memakai pakaian pesta?' Orang itu tidak dapat mengatakan apa-apa.
\par 13 Lalu raja itu berkata kepada pelayan-pelayannya, 'Ikat kaki dan tangan orang ini, dan buang dia ke luar ke tempat yang gelap. Di sana akan ada tangis dan derita.'"
\par 14 Lalu Yesus mengakhiri perumpamaan itu begini, "Banyak yang dipanggil, tetapi sedikit saja yang terpilih."
\par 15 Kemudian orang-orang Farisi pergi berunding bersama-sama mengenai bagaimana mereka bisa menjebak Yesus dengan pertanyaan-pertanyaan.
\par 16 Maka mereka mengutus pengikut-pengikut mereka kepada Yesus bersama beberapa anggota golongan Herodes. Orang-orang itu berkata kepada Yesus, "Pak Guru, kami tahu Bapak jujur. Bapak mengajar dengan terus terang mengenai kehendak Allah untuk manusia, tanpa menghiraukan pendapat siapa pun. Sebab Bapak tidak pandang orang.
\par 17 Karena itu, coba Bapak katakan kepada kami: Menurut peraturan agama kita, bolehkah membayar pajak kepada Kaisar atau tidak?"
\par 18 Yesus tahu maksud mereka yang jahat itu, jadi Ia berkata, "Hai, orang-orang munafik! Mengapa kalian mau menjebak Aku?
\par 19 Coba tunjukkan kepada-Ku mata uang yang kalian pakai untuk membayar pajak!" Lalu mereka memberikan kepada-Nya sekeping mata uang perak.
\par 20 Yesus bertanya kepada mereka, "Gambar dan nama siapakah ini?"
\par 21 "Kaisar," jawab mereka. Maka Yesus berkata kepada mereka, "Kalau begitu, berilah kepada Kaisar apa yang milik Kaisar, dan kepada Allah apa yang milik Allah."
\par 22 Waktu mereka mendengar penjelasan itu, mereka menjadi heran. Maka mereka pergi meninggalkan Yesus.
\par 23 Pada hari itu juga, beberapa orang Saduki datang kepada Yesus. Mereka adalah golongan yang berpendapat bahwa orang mati tidak akan bangkit kembali.
\par 24 "Bapak Guru," kata mereka, "Musa mengajarkan begini: Kalau seorang laki-laki mati, dan ia tidak punya anak, saudaranya harus kawin dengan jandanya supaya memberi keturunan kepada orang yang sudah mati itu.
\par 25 Pernah ada tujuh orang bersaudara yang tinggal di sini. Yang sulung kawin lalu mati tanpa mempunyai anak. Maka jandanya ditinggalkan untuk saudaranya.
\par 26 Saudaranya itu kemudian meninggal juga tanpa mempunyai anak. Hal yang sama terjadi juga dengan saudaranya yang ketiga dan seterusnya sampai yang ketujuh.
\par 27 Akhirnya wanita itu sendiri meninggal juga.
\par 28 Nah, pada waktu orang mati dibangkitkan kembali, istri siapakah wanita itu? Sebab ketujuh-tujuhnya sudah kawin dengan dia."
\par 29 Yesus menjawab, "Kalian keliru sekali, sebab kalian tidak mengerti Alkitab, maupun kuasa Allah.
\par 30 Sebab apabila orang mati nanti bangkit kembali, mereka tidak akan kawin lagi, melainkan mereka akan hidup seperti malaikat di surga.
\par 31 Belum pernahkah kalian membaca apa yang dikatakan Allah tentang orang mati dibangkitkan kembali? Allah berkata,
\par 32 'Akulah Allah Abraham, Allah Ishak dan Allah Yakub.' Allah itu bukan Allah orang mati. Ia Allah orang hidup!"
\par 33 Ketika orang banyak itu mendengar penjelasan Yesus, mereka kagum sekali akan ajaran-Nya.
\par 34 Pada waktu orang-orang Farisi mendengar bahwa Yesus sudah membuat orang-orang Saduki tidak bisa berkata apa-apa lagi, mereka berkumpul.
\par 35 Seorang dari mereka, yaitu seorang guru agama, mencoba menjebak Yesus dengan suatu pertanyaan.
\par 36 "Bapak Guru," katanya, "perintah manakah yang paling utama di dalam hukum agama?"
\par 37 Yesus menjawab, "Cintailah Tuhan Allahmu dengan sepenuh hatimu, dengan segenap jiwamu, dan dengan seluruh akalmu.
\par 38 Itulah perintah yang terutama dan terpenting!
\par 39 Perintah kedua sama dengan yang pertama itu: Cintailah sesamamu seperti engkau mencintai dirimu sendiri.
\par 40 Seluruh hukum agama yang diberikan oleh Musa dan ajaran para nabi berdasar pada kedua perintah itu."
\par 41 Sementara orang-orang Farisi masih berkumpul di situ, Yesus bertanya kepada mereka,
\par 42 "Apa pendapat kalian tentang Raja Penyelamat? Keturunan siapakah Dia?" "Keturunan Daud," jawab mereka.
\par 43 "Kalau begitu," tanya Yesus, "apa sebab Roh Allah mengilhami Daud untuk menyebut Raja Penyelamat 'Tuhan'? Sebab Daud berkata,
\par 44 'Tuhan berkata kepada Tuhanku: duduklah di sebelah kanan-Ku, sampai Aku membuat musuh-musuh-Mu takluk kepada-Mu.'
\par 45 Jadi kalau Daud menyebut Raja Penyelamat itu 'Tuhan', bagaimana mungkin Dia keturunan Daud?"
\par 46 Tidak seorang pun dapat menjawab Yesus. Dan semenjak hari itu, tidak ada yang berani menanyakan apa-apa lagi kepada-Nya.

\chapter{23}

\par 1 Lalu Yesus berkata kepada orang banyak dan kepada pengikut-pengikut-Nya,
\par 2 "Guru-guru agama dan orang-orang Farisi mendapat kekuasaan untuk menafsirkan hukum Musa.
\par 3 Sebab itu taati dan turutilah semuanya yang mereka perintahkan. Tetapi jangan melakukan apa yang mereka lakukan, sebab mereka tidak menjalankan apa yang mereka ajarkan.
\par 4 Mereka menuntut hal-hal yang sulit dan memberi peraturan-peraturan yang berat, tetapi sedikit pun mereka tidak menolong orang menjalankannya.
\par 5 Semua yang mereka lakukan hanyalah untuk dilihat orang saja. Mereka sengaja memakai tali sembahyang yang lebar-lebar dan memanjangkan rumbai-rumbai jubah mereka!
\par 6 Mereka suka tempat yang terbaik pada pesta-pesta, dan kursi istimewa di rumah-rumah ibadat.
\par 7 Mereka senang dihormati orang di pasar-pasar, dan dipanggil 'Bapak Guru'.
\par 8 Tetapi kalian, janganlah mau dipanggil 'Bapak Guru', sebab Gurumu hanya ada satu dan kalian semua bersaudara.
\par 9 Dan janganlah kalian memanggil seorang pun di dunia ini 'Bapak', sebab Bapakmu hanya satu, yaitu Bapa yang di surga.
\par 10 Dan janganlah kalian mau dipanggil 'Pemimpin', sebab pemimpinmu hanya ada satu, yaitu Raja Penyelamat yang dijanjikan oleh Allah.
\par 11 Orang yang terbesar di antara kalian, haruslah menjadi pelayanmu.
\par 12 Orang yang meninggikan dirinya akan direndahkan, dan orang yang merendahkan dirinya akan ditinggikan."
\par 13 "Celakalah kalian guru-guru agama dan orang-orang Farisi! Kalian tukang berpura-pura. Kalian menghalangi orang untuk menjadi anggota umat Allah. Kalian sendiri tidak mau menjadi anggota umat Allah, dan orang lain yang mau, kalian rintangi.
\par 14 (Celakalah kalian guru-guru agama dan orang-orang Farisi: Kalian tukang berpura-pura. Kalian menipu janda-janda dan merampas rumahnya dan untuk menutupi kejahatan itu kalian berdoa panjang-panjang. Itu sebabnya hukuman kalian nanti berat!)
\par 15 Celakalah kalian guru-guru agama dan orang-orang Farisi! Kalian tukang berpura-pura! Kalian pergi jauh-jauh menyeberang lautan, dan menjelajahi daratan hanya untuk membuat satu orang masuk agamamu. Dan sesudah orang itu masuk agamamu, kalian membuat dia calon neraka yang dua kali lebih jahat daripada kalian sendiri!
\par 16 Celakalah kalian pemimpin-pemimpin yang buta! Kalian mengajarkan ini, 'Kalau orang bersumpah demi Rumah Tuhan, orang itu tidak terikat pada sumpahnya; tetapi kalau ia bersumpah demi emas di dalam Rumah Tuhan, ia terikat pada sumpahnya itu.'
\par 17 Kalian orang-orang bodoh yang buta! Mana yang lebih penting: emasnya, atau Rumah Tuhan yang menjadikan emas itu suci?
\par 18 Kalian mengajarkan ini juga, 'Kalau seorang bersumpah demi mezbah, orang itu tidak terikat oleh sumpahnya; tetapi kalau ia bersumpah demi persembahan di atas mezbah itu, ia terikat oleh sumpahnya itu.'
\par 19 Alangkah butanya kalian! Mana yang lebih penting? Persembahannya atau mezbah yang menjadikan persembahan itu suci?
\par 20 Sebab itu, kalau seorang bersumpah demi mezbah, itu berarti ia bersumpah demi mezbah, dan demi semua persembahan yang ada di atasnya.
\par 21 Dan kalau seorang bersumpah demi Rumah Tuhan, itu berarti ia bersumpah demi Rumah Tuhan itu, dan demi Allah yang tinggal di situ.
\par 22 Dan kalau seorang bersumpah demi surga, itu berarti ia bersumpah demi takhta Tuhan, dan demi Allah yang duduk di takhta itu.
\par 23 Celakalah kalian guru-guru agama dan orang-orang Farisi! Kalian tukang berpura-pura. Rempah-rempah seperti selasih, adas manis, dan jintan pun, kalian beri sepersepuluhnya kepada Tuhan. Padahal hal-hal yang terpenting dalam hukum-hukum agama, seperti misalnya: Keadilan, belas kasihan, dan kesetiaan, tidak kalian hiraukan. Padahal itulah yang seharusnya kalian lakukan, tanpa melalaikan yang lain-lainnya juga.
\par 24 Kalian pemimpin-pemimpin yang buta! Lalat dalam minumanmu kalian saring, padahal unta kalian telan!
\par 25 Celakalah kalian guru-guru agama dan orang-orang Farisi! Kalian tukang berpura-pura! Mangkuk-mangkuk dan piring-piringmu kalian cuci bersih-bersih bagian luarnya, padahal bagian dalamnya kotor sekali dengan hal-hal yang kalian dapat dengan kekerasan dan keserakahan.
\par 26 Farisi buta! Cucilah dahulu bersih-bersih bagian dalam dari mangkuk-mangkuk dan piring-piringmu, supaya bagian luarnya menjadi bersih juga!
\par 27 Celakalah kalian guru-guru agama dan orang-orang Farisi! Kalian tukang berpura-pura! Kalian seperti kubur-kubur yang dicat putih; di luarnya kelihatan bagus, tetapi di dalamnya penuh dengan tulang dan semuanya yang busuk-busuk.
\par 28 Begitu juga kalian. Dari luar kalian kelihatan baik kepada orang; tetapi di dalam, kalian penuh dengan kepalsuan dan pelanggaran-pelanggar
\par 29 "Celakalah kalian guru-guru agama dan orang-orang Farisi! Kalian tukang berpura-pura! Kalian membangun makam-makam yang bagus untuk nabi-nabi, dan menghiasi tugu-tugu peringatan dari orang-orang yang hidupnya baik.
\par 30 Dan kalian berkata, 'Seandainya kami hidup di zaman nenek moyang kami dahulu, kami tidak akan turut dengan mereka membunuh nabi-nabi.'
\par 31 Jadi kalian mengaku sendiri bahwa kalianlah keturunan orang-orang yang membunuh nabi-nabi!
\par 32 Kalau begitu, teruskanlah dan selesaikan dosa-dosa yang sudah dimulai oleh nenek moyangmu itu!
\par 33 Kalian jahat dan keturunan orang jahat! Bagaimana kalian bisa menyelamatkan diri dari hukuman di neraka?
\par 34 Dengarlah baik-baik: Aku akan mengirim kepadamu nabi-nabi, orang-orang bijak, dan guru-guru; sebagian dari mereka akan kalian bunuh, dan sebagian yang lain akan kalian salibkan. Ada yang akan kalian siksa di dalam rumah-rumah ibadat, dan kalian kejar-kejar dari satu kota ke kota yang lain.
\par 35 Sebab itu, kalian akan dihukum karena pembunuhan yang dilakukan terhadap semua orang yang tidak bersalah--mulai dari pembunuhan Habel yang tidak bersalah, sampai pembunuhan Zakharia anak Berekhya, yang kalian bunuh di antara Rumah Tuhan dan mezbah.
\par 36 Percayalah: semuanya itu akan ditanggung oleh orang-orang zaman ini!"
\par 37 "Yerusalem, Yerusalem! Nabi-nabi kaubunuh. Utusan-utusan Allah kaulempari batu sampai mati. Sudah berapa kali Aku ingin merangkul semua pendudukmu seperti induk ayam melindungi anak-anaknya di bawah sayapnya, tetapi kamu tidak mau!
\par 38 Karena itu Allah tidak lagi menyertaimu.
\par 39 Ketahuilah: Mulai sekarang ini engkau tidak akan melihat Aku lagi sampai engkau berkata, 'Diberkatilah Dia yang datang atas nama Tuhan.'"

\chapter{24}

\par 1 Ketika Yesus meninggalkan Rumah Tuhan, pengikut-pengikut-Nya datang kepada-Nya dan menunjuk ke bangunan-bangunan Rumah Tuhan itu.
\par 2 Yesus berkata kepada mereka, "Apakah kalian melihat semuanya itu? Ketahuilah, tidak ada satu batu pun dari bangunan-bangunan itu akan tinggal tersusun pada tempatnya. Semuanya akan dirobohkan."
\par 3 Kemudian Yesus pergi ke Bukit Zaitun, dan sedang Ia duduk, pengikut-pengikut-Nya datang untuk berbicara dengan Dia secara pribadi. "Beritahukan kepada kami kapan semuanya itu akan terjadi," kata mereka kepada-Nya. "Tanda-tanda apakah yang menunjukkan kedatangan Bapak dan akhir zaman?"
\par 4 Yesus menjawab, "Waspadalah, jangan sampai kalian tertipu.
\par 5 Sebab banyak orang akan datang dengan memakai nama-Ku dan berkata, 'Akulah Raja Penyelamat!' Mereka akan menipu banyak orang.
\par 6 Kalian akan mendengar bunyi-bunyi pertempuran dan berita-berita peperangan, tetapi jangan takut. Sebab hal-hal itu harus terjadi, tetapi itu tidak berarti bahwa sudah waktunya kiamat.
\par 7 Bangsa yang satu akan berperang melawan bangsa yang lain, dan negara yang satu akan menyerang negara yang lain. Di mana-mana akan terjadi bahaya kelaparan dan gempa bumi.
\par 8 Semuanya itu baru permulaan saja, seperti sakit yang dialami seorang wanita pada waktu mau melahirkan.
\par 9 Kemudian kalian akan ditangkap dan diserahkan untuk disiksa dan dibunuh. Seluruh dunia akan membenci kalian karena kalian pengikut-Ku.
\par 10 Pada waktu itu banyak orang akan murtad, dan mengkhianati serta membenci satu sama lain.
\par 11 Banyak nabi-nabi palsu akan muncul, dan menipu banyak orang.
\par 12 Kejahatan akan menjalar sebegitu hebat sampai banyak orang tidak dapat lagi mengasihi.
\par 13 Tetapi orang yang bertahan sampai akhir, akan diselamatkan.
\par 14 Dan Kabar Baik tentang bagaimana Allah memerintah akan diberitakan ke seluruh dunia, supaya semua orang mendengarnya. Sesudah itu barulah datang kiamat."
\par 15 "Kalian akan melihat 'Kejahatan yang menghancurkan', seperti yang dikatakan oleh Nabi Daniel, berdiri di tempat yang suci. (Catatan kepada pembaca: Perhatikanlah apa artinya!)
\par 16 Pada waktu itu, orang yang berada di Yudea harus lari ke pegunungan.
\par 17 Orang yang berada di atas atap rumah jangan turun untuk mengambil sesuatu dari dalam rumah.
\par 18 Orang yang berada di ladang jangan kembali untuk mengambil jubahnya.
\par 19 Alangkah ngerinya hari-hari itu bagi wanita yang mengandung, dan ibu yang masih menyusui bayi!
\par 20 Berdoalah supaya jangan sampai kalian harus lari pada musim hujan atau pada hari Sabat!
\par 21 Pada hari-hari yang mengerikan itu akan ada kesusahan besar seperti yang belum pernah terjadi sejak permulaan dunia sampai saat ini, dan tidak pula akan terjadi lagi.
\par 22 Sekiranya Allah tidak memperpendek waktunya; maka tidak ada seorang pun yang selamat. Tetapi karena umat-Nya, Allah memperpendek masa itu.
\par 23 Pada waktu itu kalau ada seseorang berkata kepada kalian, 'Lihat, Raja Penyelamat itu ada di sini!' atau 'Ia ada di situ!' --janganlah percaya kepada orang itu.
\par 24 Sebab akan muncul penyelamat-penyelamat palsu dan nabi-nabi palsu. Mereka akan mengerjakan perbuatan-perbuatan yang luar biasa, dan keajaiban-keajaiban untuk menipu, kalau mungkin, umat Allah juga.
\par 25 Jadi, ingatlah! Aku sudah memberitahukannya kepada kalian lebih dahulu sebelum hal itu terjadi.
\par 26 Kalau orang berkata kepadamu, 'Lihat, Dia ada di sana di padang gurun!' --jangan kalian ke sana. Atau kalau mereka berkata, 'Lihat, Ia bersembunyi dalam kamar di sini!' --jangan percaya.
\par 27 Sebab kedatangan Anak Manusia seperti cahaya kilat memancar dari timur, dan bersinar sampai ke barat.
\par 28 Di mana ada bangkai, di situ ada burung pemakan bangkai."
\par 29 "Tidak lama sesudah kesusahan masa itu, matahari akan menjadi gelap, dan bulan tidak lagi bercahaya. Bintang-bintang akan jatuh dari langit, dan para penguasa angkasa raya akan menjadi kacau-balau.
\par 30 Sesudah itu tanda Anak Manusia akan kelihatan di langit. Pada waktu itu semua bangsa di bumi akan menangis. Mereka akan melihat Anak Manusia datang di atas awan dengan kuasa dan kemuliaan yang besar.
\par 31 Trompet besar akan dibunyikan, dan Anak Manusia akan menyuruh malaikat-malaikat-Nya mengumpulkan umat-Nya dari keempat penjuru bumi, dari ujung langit yang satu sampai ujung langit yang lain."
\par 32 "Ambillah pelajaran dari pohon ara. Kalau ranting-rantingnya hijau dan lembut, dan mulai bertunas, kalian tahu bahwa musim panas sudah dekat.
\par 33 Begitu juga kalau kalian melihat hal-hal itu terjadi, kalian tahu bahwa waktunya sudah dekat sekali.
\par 34 Ketahuilah! Hal-hal itu akan terjadi sebelum orang-orang yang hidup sekarang ini mati semuanya.
\par 35 Langit dan bumi akan lenyap, tetapi perkataan-Ku tetap selama-lamanya."
\par 36 "Tidak ada yang tahu kapan harinya dan jamnya, malaikat-malaikat di surga tidak, Anak Allah pun tidak, hanya Bapa saja yang tahu.
\par 37 Apabila Anak Manusia datang nanti, keadaannya seperti pada zaman Nuh dahulu.
\par 38 Pada hari-hari sebelum banjir besar itu, orang-orang makan minum, dan kawin. Begitulah terus-menerus sampai pada hari Nuh masuk ke dalam kapal.
\par 39 Pada waktu banjir itu melanda mereka semua, barulah mereka sadar akan apa yang sedang terjadi. Begitulah juga keadaannya nanti kalau Anak Manusia datang.
\par 40 Pada waktu itu, dua orang sedang bekerja di ladang: Seorang akan dibawa, dan seorang lagi ditinggalkan.
\par 41 Dua wanita sedang menggiling gandum: Seorang akan dibawa, dan seorang lagi ditinggalkan.
\par 42 Jadi, waspadalah, sebab kalian tidak tahu kapan Tuhanmu akan datang.
\par 43 Ingatlah ini! Seandainya tuan rumah tahu jam berapa di malam hari pencuri akan datang, ia tidak akan tidur, supaya pencuri tidak masuk ke dalam rumahnya.
\par 44 Sebab itu, kalian juga harus bersiap-siap. Karena Anak Manusia akan datang pada saat yang tidak kalian sangka-sangka."
\par 45 Kata Yesus lagi, "Kalau begitu, pelayan yang manakah yang setia dan bijaksana? Dialah yang diangkat oleh tuannya menjadi kepala atas pelayan-pelayan lain, supaya ia memberi mereka makan pada waktunya.
\par 46 Alangkah bahagianya pelayan itu apabila tuannya kembali, dan mendapati dia sedang melakukan tugasnya.
\par 47 Percayalah, tuan itu akan mempercayakan segala hartanya kepada pelayan itu.
\par 48 Tetapi kalau pelayan itu jahat, ia akan berkata dalam hatinya,
\par 49 'Tuan saya masih lama baru kembali,' lalu ia mulai memukul pelayan-pelayan yang lain, dan makan minum dengan orang-orang pemabuk.
\par 50 Kemudian tuannya akan kembali pada hari dan jam yang tidak disangka-sangka.
\par 51 Maka pelayan itu akan dihajar habis-habisan oleh tuannya, dan dibuang ke tempat orang-orang munafik. Mereka akan menangis dan menderita di sana."

\chapter{25}

\par 1 "Apabila Anak Manusia datang sebagai Tuhan, keadaannya seperti dalam perumpamaan ini: Sepuluh gadis pengiring pengantin masing-masing mengambil pelita, lalu pergi menyambut pengantin laki-laki.
\par 2 Lima orang dari mereka bodoh, dan lima yang lainnya bijaksana.
\par 3 Kelima gadis yang bodoh membawa pelita, tetapi tidak membawa minyak persediaan.
\par 4 Kelima gadis yang bijaksana membawa pelita bersama-sama dengan minyak persediaan.
\par 5 Pengantin laki-laki itu datang terlambat, jadi gadis-gadis itu mulai mengantuk lalu tertidur.
\par 6 Tengah malam, barulah terdengar suara teriakan, 'Pengantin laki-laki datang! Mari sambut dia!'
\par 7 Sepuluh gadis itu bangun, dan memasang pelita mereka.
\par 8 Gadis-gadis yang bodoh itu berkata kepada yang bijaksana, 'Berikanlah minyakmu sedikit kepada kami, sebab pelita kami sudah mau padam.'
\par 9 'Tidak bisa!' jawab anak-anak gadis yang bijaksana itu, 'sebab nanti kita semua tidak punya cukup minyak. Pergilah beli di toko.'
\par 10 Maka gadis-gadis yang bodoh itu pergi membeli minyak. Sementara mereka pergi, tibalah pengantin laki-laki. Kelima gadis yang sudah siap itu masuk bersama-sama dengan pengantin laki-laki ke tempat pesta, dan pintu pun ditutup!
\par 11 Kemudian gadis-gadis yang lainnya itu tiba. Mereka berseru, 'Tuan, Tuan, bukakan pintu untuk kami.'
\par 12 Tetapi pengantin laki-laki itu menjawab, 'Aku tidak mengenal kalian!'"
\par 13 Lalu Yesus mengakhiri perumpamaan-Nya itu begini, "Oleh sebab itu berjaga-jagalah, sebab kalian tidak tahu harinya ataupun jamnya."
\par 14 "Apabila Anak Manusia datang sebagai Tuhan keadaannya juga seperti dalam perumpamaan ini. Seorang laki-laki hendak berangkat ke tempat yang jauh. Ia memanggil pelayan-pelayannya, lalu mempercayakan hartanya kepada mereka.
\par 15 Kepada setiap pelayan itu ia memberi menurut kesanggupan masing-masing. Kepada yang seorang ia memberi lima ribu uang emas. Kepada yang lainnya ia memberi dua ribu uang emas. Dan kepada seorang lagi ia memberi seribu uang emas. Lalu ia berangkat.
\par 16 Pelayan yang menerima lima ribu uang emas itu segera pergi berdagang, lalu mendapat keuntungan lima ribu uang emas lagi.
\par 17 Begitu juga pelayan yang menerima dua ribu uang emas itu mendapat untung dua ribu lagi.
\par 18 Tetapi pelayan yang menerima seribu uang emas itu pergi menggali lubang di tanah, lalu menyembunyikan uang tuannya di situ.
\par 19 Lama sekali sesudah itu, tuan dari pelayan-pelayan itu pulang, dan mulai mengadakan perhitungan dengan mereka.
\par 20 Pelayan yang menerima lima ribu uang emas itu datang, dan menyerahkan sepuluh ribu. 'Tuan,' katanya, 'Tuan menyerahkan lima ribu uang emas kepada saya. Lihatlah, saya berhasil mendapat keuntungan lima ribu lagi.'
\par 21 'Bagus,' kata tuan itu, 'engkau adalah pelayan yang baik dan setia. Karena engkau dapat dipercayai dengan yang sedikit, saya akan mempercayakan yang banyak kepadamu. Masuklah dan ikutlah bersenang-senang dengan saya!'
\par 22 Lalu pelayan yang menerima dua ribu uang emas itu datang, dan berkata, 'Tuan, Tuan sudah menyerahkan dua ribu uang emas kepada saya. Lihatlah, saya berhasil mendapat keuntungan dua ribu lagi.'
\par 23 'Bagus,' kata tuan itu, 'engkau pelayan yang baik dan setia. Karena engkau dapat dipercayai dengan yang sedikit, saya akan mempercayakan yang banyak kepadamu. Masuklah dan ikutlah bersenang-senang dengan saya!'
\par 24 Kemudian pelayan yang menerima seribu uang emas itu datang, dan berkata, 'Tuan, saya tahu Tuan seorang yang keras. Tuan memetik buah di tempat Tuan tidak menanam, dan memungut hasil di tempat Tuan tidak menabur benih.
\par 25 Saya takut, jadi saya pergi menyembunyikan uang Tuan di dalam tanah. Inilah uang Tuan.'
\par 26 'Engkau pelayan yang jahat dan malas!' kata tuan itu. 'Bukankah engkau sudah tahu bahwa saya memetik buah di tempat saya tidak menanam, dan memungut hasil di tempat saya tidak menabur benih?
\par 27 Kalau begitu, seharusnya engkau menyimpan uang saya itu di bank, supaya pada waktu saya pulang, saya dapat menerima kembali uang saya itu dengan bunganya.
\par 28 Karena itu, ambillah uang itu dari dia, dan berikanlah kepada orang yang mempunyai sepuluh ribu uang emas itu.
\par 29 Karena orang yang sudah mempunyai, akan diberi lebih banyak lagi, dan ia akan berkelebihan. Tetapi orang yang tidak punya, sedikit yang masih ada padanya akan diambil juga.
\par 30 Dan pelayan yang tidak berguna itu, buanglah dalam kegelapan di luar. Di sana ia akan menangis dan menderita!'"
\par 31 "Apabila Anak Manusia datang sebagai Raja diiringi semua malaikat-Nya, Ia akan duduk di atas takhta-Nya yang mulia.
\par 32 Segala bangsa akan dikumpulkan di hadapan-Nya. Lalu Ia akan memisahkan mereka menjadi dua kumpulan seperti gembala memisahkan domba dari kambing.
\par 33 Orang-orang yang melakukan kehendak Allah akan dikumpulkan di sebelah kanan-Nya, dan yang lain di sebelah kiri-Nya.
\par 34 Kemudian Raja itu akan berkata kepada orang-orang di sebelah kanan-Nya, 'Marilah kalian yang diberkati oleh Bapa-Ku. Masuklah ke dalam Kerajaan yang disediakan bagimu sejak permulaan dunia.
\par 35 Sebab pada waktu Aku lapar, kalian memberi Aku makan, dan pada waktu Aku haus, kalian memberi Aku minum. Aku seorang asing, kalian menerima Aku di rumahmu.
\par 36 Aku tidak berpakaian, kalian memberikan Aku pakaian. Aku sakit, kalian merawat Aku. Aku dipenjarakan, kalian menolong Aku.'
\par 37 Lalu orang-orang itu akan berkata, 'Tuhan, kapan kami pernah melihat Tuhan lapar lalu kami memberi Tuhan makan, atau haus lalu kami memberi Tuhan minum?
\par 38 Kapan kami pernah melihat Tuhan sebagai orang asing, lalu kami menyambut Tuhan ke dalam rumah kami? Kapan Tuhan pernah tidak berpakaian, lalu kami memberi Tuhan pakaian?
\par 39 Kapan kami pernah melihat Tuhan sakit atau dipenjarakan, lalu kami menolong Tuhan?'
\par 40 Raja itu akan menjawab, 'Ketahuilah: waktu kalian melakukan hal itu, sekalipun kepada salah seorang dari saudara-saudara-Ku yang terhina, berarti kalian melakukannya kepada-Ku!'
\par 41 Lalu Raja itu akan berkata kepada orang-orang di sebelah kiri-Nya, 'Pergilah dari sini, jahanam! Masuklah ke dalam api yang tidak bisa padam, yang sudah disediakan bagi Iblis dan malaikat-malaikatnya!
\par 42 Sebab pada waktu Aku lapar, kalian tidak memberi Aku makan; pada waktu Aku haus, kalian tidak memberi Aku minum.
\par 43 Aku seorang asing, kalian tidak menerima Aku di dalam rumahmu. Aku tidak berpakaian, kalian tidak memberi Aku pakaian. Aku sakit dan dipenjarakan, kalian tidak merawat Aku.'
\par 44 Lalu mereka akan berkata kepada-Nya, 'Tuhan, kapankah kami melihat Tuhan lapar, atau haus, atau sebagai seorang asing, atau tidak berpakaian, atau sakit, atau dipenjarakan, dan kami tidak menolong Tuhan?'
\par 45 Raja itu akan menjawab, 'Ketahuilah: pada waktu kalian tidak mau menolong salah seorang yang terhina ini, berarti kalian tidak mau menolong Aku.'
\par 46 Maka orang-orang itu akan dihukum dengan hukuman yang kekal, sedangkan orang-orang yang melakukan kehendak Allah akan mengalami hidup sejati dan kekal."

\chapter{26}

\par 1 Waktu Yesus selesai mengajarkan semua hal itu, Ia berkata kepada pengikut-pengikut-Nya,
\par 2 "Kalian tahu dua hari lagi Hari Raya Paskah, dan Anak Manusia akan diserahkan untuk disalibkan!"
\par 3 Pada waktu itu imam-imam kepala dan pemimpin-pemimpin Yahudi berkumpul di istana Imam Agung Kayafas.
\par 4 Mereka berunding untuk menangkap Yesus dengan diam-diam, dan membunuh Dia.
\par 5 "Tetapi," kata mereka, "janganlah hal itu dilakukan pada waktu perayaan, sebab nanti timbul kerusuhan di antara rakyat."
\par 6 Ketika Yesus berada di Betania, di rumah Simon yang dahulu menderita penyakit kulit yang mengerikan,
\par 7 seorang wanita datang kepada Yesus. Ia membawa sebuah botol pualam, berisi minyak wangi yang mahal. Pada waktu Yesus sedang duduk makan, wanita itu menuang minyak wangi itu ke atas kepala Yesus.
\par 8 Pengikut-pengikut Yesus melihat peristiwa itu dan menjadi marah. "Apa gunanya semuanya ini diboroskan?" kata mereka.
\par 9 "Minyak wangi itu dapat dijual dengan harga yang tinggi, dan uangnya diberikan kepada orang miskin!"
\par 10 Yesus tahu pikiran mereka, lalu Ia berkata, "Mengapa kalian menyusahkan wanita ini? Ia melakukan sesuatu yang baik dan terpuji untuk-Ku.
\par 11 Orang miskin selalu ada di antara kalian, tetapi Aku tidak selamanya bersama-sama kalian.
\par 12 Dengan menuang minyak wangi itu ke atas badan-Ku, ia mempersiapkan Aku untuk penguburan-Ku.
\par 13 Percayalah! Di seluruh dunia, di mana saja Kabar Baik dari Allah disiarkan, perbuatan wanita ini akan diceritakan juga sebagai kenangan kepadanya."
\par 14 Lalu seorang dari kedua belas pengikut Yesus, yang bernama Yudas Iskariot, pergi kepada imam-imam kepala.
\par 15 Ia berkata kepada mereka, "Apakah yang akan kalian berikan kepadaku kalau aku menyerahkan Yesus kepadamu?" Maka mereka menghitung tiga puluh uang perak, lalu memberikan uang itu kepadanya.
\par 16 Mulai dari waktu itu Yudas mencari kesempatan yang baik untuk mengkhianati Yesus.
\par 17 Pada hari pertama dalam Perayaan Roti Tidak Beragi, pengikut-pengikut Yesus datang kepada-Nya. Mereka bertanya, "Di mana Bapak ingin kami menyediakan makanan Paskah untuk Bapak?"
\par 18 Yesus menjawab, "Pergilah kepada seseorang di kota dan katakan kepadanya, 'Kata Bapak Guru, sudah sampai waktunya untuk-Ku; Aku mau merayakan Paskah di rumahmu bersama-sama dengan pengikut-pengikut-Ku.'"
\par 19 Pengikut-pengikut Yesus melakukan apa yang disuruh Yesus kepada mereka. Mereka pergi menyiapkan makanan Paskah itu.
\par 20 Setelah malam, Yesus dan kedua belas pengikut-Nya duduk makan.
\par 21 Sementara mereka makan, Yesus berkata, "Dengarkan: seorang dari antara kalian akan mengkhianati Aku."
\par 22 Mendengar itu, pengikut-pengikut Yesus menjadi sangat sedih. Lalu mereka, seorang demi seorang mulai bertanya kepada Yesus, "Tentu bukan saya yang Bapak maksudkan?"
\par 23 Yesus menjawab, "Orang yang mencelup roti ke dalam mangkuk bersama-sama-Ku, dialah yang akan mengkhianati Aku.
\par 24 Memang Anak Manusia akan mati seperti yang tertulis di dalam Alkitab. Tetapi celakalah orang yang mengkhianati Anak Manusia! Lebih baik untuk orang itu kalau ia tidak pernah lahir sama sekali!"
\par 25 Lalu Yudas si pengkhianat itu berkata, "Tentu bukan saya yang Bapak Guru maksudkan?" Yesus menjawab, "Begitulah katamu!"
\par 26 Ketika mereka makan, Yesus mengambil roti, lalu mengucapkan doa syukur. Kemudian Ia membelah-belah roti itu dengan tangan-Nya lalu memberikannya kepada pengikut-pengikut-Nya sambil berkata, "Ambil, dan makanlah; inilah tubuh-Ku."
\par 27 Sesudah itu Ia mengambil sebuah piala anggur, lalu mengucap syukur kepada Allah. Kemudian Ia memberikan piala itu kepada pengikut-pengikut-sambil berkata, "Minumlah, kamu semua.
\par 28 Sebab inilah darah-Ku yang mensahkan perjanjian Allah--darah yang dicurahkan bagi banyak orang untuk pengampunan dosa mereka.
\par 29 Percayalah: Aku tidak akan minum anggur ini lagi sampai pada waktu Aku minum anggur yang baru bersama-sama dengan kalian di Dunia Baru Bapa-Ku."
\par 30 Kemudian mereka menyanyikan sebuah nyanyian pujian. Dan sesudah itu mereka pergi ke Bukit Zaitun.
\par 31 Lalu Yesus berkata kepada pengikut-pengikut-Nya, "Pada malam ini juga kamu semua akan lari meninggalkan Aku; sebab dalam Alkitab tertulis: Allah berkata, 'Aku akan membunuh gembala itu, dan kawanan dombanya akan tercerai-berai.'
\par 32 Tetapi setelah Aku dibangkitkan kembali, Aku akan pergi mendahului kalian ke Galilea."
\par 33 Petrus berkata kepada Yesus, "Biar semua yang lainnya meninggalkan Bapak, saya sekali-kali tidak!"
\par 34 "Ingat," kata Yesus kepadanya, "Malam ini juga, sebelum ayam berkokok, engkau tiga kali mengingkari Aku."
\par 35 Petrus menjawab, "Sekalipun saya harus mati bersama Bapak, saya tidak akan berkata bahwa saya tidak mengenal Bapak!" Dan semua pengikut yang lain berkata begitu juga.
\par 36 Sesudah itu Yesus pergi dengan pengikut-pengikut-Nya ke suatu tempat yang bernama Getsemani. Di sana Ia berkata kepada mereka, "Duduklah di sini sementara Aku pergi berdoa."
\par 37 Lalu Ia mengajak Petrus dan kedua anak Zebedeus pergi bersama-sama dengan Dia. Ia mulai merasa sedih dan gelisah.
\par 38 Ia berkata kepada pengikut-pengikut-Nya, "Hati-Ku sedih sekali, rasanya seperti akan mati saja. Tinggallah kalian di sini, dan turutlah berjaga-jaga dengan Aku."
\par 39 Kemudian Yesus pergi lebih jauh sedikit, lalu Ia tersungkur ke tanah dan berdoa. "Bapa," kata-Nya, "kalau boleh, jauhkanlah daripada-Ku penderitaan yang Aku harus alami ini. Tetapi jangan menurut kemauan-Ku, melainkan menurut kemauan Bapa saja."
\par 40 Sesudah itu Yesus kembali kepada ketiga pengikut-Nya dan mendapati mereka sedang tidur. Ia berkata kepada Petrus, "Hanya satu jam saja kalian bertiga tidak dapat berjaga dengan Aku?
\par 41 Berjaga-jagalah, dan berdoalah supaya kalian jangan mengalami cobaan. Memang rohmu mau melakukan yang benar tetapi kalian tidak sanggup, karena tabiat manusia itu lemah."
\par 42 Sekali lagi Yesus pergi berdoa, kata-Nya, "Bapa, kalau penderitaan ini harus Aku alami, dan tidak dapat dijauhkan, biarlah kemauan Bapa yang jadi."
\par 43 Sesudah itu Yesus kembali lagi, dan mendapati pengikut-pengikut-Nya masih juga tidur, karena mereka terlalu mengantuk.
\par 44 Sekali lagi Yesus meninggalkan mereka dan untuk ketiga kalinya berdoa dengan mengucapkan kata-kata yang sama.
\par 45 Sesudah itu Ia kembali lagi kepada pengikut-pengikut-Nya dan berkata, "Masihkah kalian tidur dan istirahat? Lihat, sudah sampai waktunya Anak Manusia diserahkan kepada kuasa orang-orang berdosa.
\par 46 Bangunlah, mari kita pergi. Lihat! Orang yang mengkhianati Aku sudah datang!"
\par 47 Sementara Yesus masih berbicara, Yudas, seorang dari kedua belas pengikut-Nya itu datang. Bersama-sama dengan dia, datang juga banyak orang yang membawa pedang dan pentungan. Mereka disuruh oleh imam-imam kepala dan pemimpin-pemimpin Yahudi.
\par 48 Si pengkhianat sudah memberitahukan kepada mereka tanda ini, "Orang yang saya cium, itulah orangnya. Tangkap Dia!"
\par 49 Begitu sampai di tempat itu, Yudas langsung pergi kepada Yesus dan berkata, "Salam, Pak Guru!" Lalu ia mencium Yesus.
\par 50 Yesus menjawab, "Saudara, untuk apa saudara datang kemari?" Kemudian orang banyak itu maju, dan menangkap Yesus.
\par 51 Salah seorang pengikut-Nya yang berada di situ dengan Yesus, mencabut pedangnya dan memarang hamba imam agung sampai putus telinganya.
\par 52 Yesus berkata kepada pengikut-Nya itu, "Masukkan kembali pedangmu ke dalam sarungnya, sebab semua orang yang menggunakan pedangnya akan mati oleh pedang.
\par 53 Kaukira Aku tidak dapat minta tolong kepada Bapa-Ku, dan Ia dengan segera akan mengirim lebih dari dua belas pasukan tentara malaikat?
\par 54 Tetapi kalau begitu, mana mungkin terjadi seperti yang sudah dinubuatkan dalam Alkitab bahwa memang harus terjadi seperti yang sekarang ini?"
\par 55 Lalu Yesus berkata kepada orang banyak itu, "Apakah Aku ini penjahat, sampai kalian datang dengan membawa pedang dan pentungan untuk menangkap Aku? Setiap hari Aku mengajar di Rumah Allah, dan kalian tidak menangkap Aku!
\par 56 Tetapi memang sudah seharusnya begitu supaya terjadilah apa yang ditulis oleh nabi-nabi di dalam Alkitab." Setelah itu, semua pengikut-pengikut-Nya lari meninggalkan Yesus.
\par 57 Orang-orang yang menangkap Yesus membawa-Nya ke rumah Imam Agung Kayafas. Di sana guru-guru agama dan pemimpin-pemimpin Yahudi sudah berkumpul.
\par 58 Petrus mengikuti Yesus dari jauh sampai ke halaman rumah imam agung. Lalu Petrus masuk ke dalam halaman itu, dan duduk bersama pengawal-pengawal. Ia ingin tahu bagaimana semuanya itu akan berakhir nanti.
\par 59 Imam-imam kepala dan seluruh Mahkamah Agama berusaha mendapat kesaksian palsu untuk dapat menjatuhkan hukuman mati ke atas Yesus.
\par 60 Tetapi mereka tidak mendapat satu bukti pun, meskipun banyak yang maju sebagai saksi dusta. Akhirnya ada dua orang yang tampil ke depan.
\par 61 Mereka berkata, "Orang ini berkata, 'Aku dapat merobohkan Rumah Allah, dan dalam tiga hari dapat membangunnya kembali.'"
\par 62 Lalu imam agung berdiri, dan berkata kepada Yesus, "Apakah Engkau tidak menjawab tuduhan yang ditujukan kepada-Mu itu?"
\par 63 Tetapi Yesus diam saja. Sekali lagi imam agung berkata kepada-Nya, "Demi Allah yang hidup, katakanlah kepada kami apakah Engkau Raja Penyelamat, Anak Allah?"
\par 64 Yesus menjawab, "Begitulah katamu. Tetapi percayalah: mulai saat ini, kalian akan melihat Anak Manusia duduk di sebelah kanan Allah Yang Mahakuasa, dan datang di atas awan di langit!"
\par 65 Maka imam agung itu menyobek-nyobek pakaiannya, dan berkata, "Ia menghujat Allah! Tidak perlu lagi saksi. Kamu semua sudah mendengar sendiri kata-kata yang menghujat Allah.
\par 66 Sekarang bagaimana pendapat kalian?" Mereka menjawab, "Dia bersalah, dan harus mati."
\par 67 Lalu mereka meludahi muka Yesus, dan memukul Dia. Ada juga yang menampar Dia
\par 68 dan berkata, "Coba tebak dan beritahu kepada kami, hai Raja Penyelamat! Siapa yang menampar Engkau?"
\par 69 Petrus sedang duduk di luar, di halaman. Salah seorang pelayan wanita datang, dan berkata kepada Petrus, "Bukankah engkau juga bersama-sama Yesus orang Galilea itu?"
\par 70 Tetapi Petrus menyangkal di hadapan mereka semuanya. "Saya tidak tahu apa maksudmu," jawab Petrus,
\par 71 lalu ia pergi ke pintu halaman. Seorang pelayan wanita yang lain melihat Petrus, dan berkata kepada orang-orang di situ, "Orang ini tadi juga bersama-sama dengan Yesus dari Nazaret itu."
\par 72 Lalu Petrus menyangkal lagi, dan bersumpah. "Sungguh-sungguh saya tidak kenal orang itu!" kata Petrus.
\par 73 Tidak lama sesudah itu, orang-orang yang berdiri di situ datang kepada Petrus, dan berkata, "Pasti engkau salah seorang dari mereka. Itu kentara sekali dari logatmu."
\par 74 Lalu Petrus mulai menyumpah-nyumpah dan berkata, "Saya tidak kenal orang itu!" Saat itu juga ayam berkokok.
\par 75 Dan Petrus teringat bahwa Yesus sudah berkata kepadanya, "Sebelum ayam berkokok, engkau tiga kali mengingkari Aku." Lalu Petrus ke luar, dan menangis dengan sedih.

\chapter{27}

\par 1 Pagi-pagi sekali, semua imam kepala dan pemimpin Yahudi membuat keputusan untuk membunuh Yesus.
\par 2 Mereka membelenggu Dia, dan membawa Dia, lalu menyerahkan-Nya kepada Pilatus, gubernur pemerintahan Roma.
\par 3 Ketika Yudas si pengkhianat itu melihat bahwa Yesus sudah dijatuhi hukuman, ia menyesal. Lalu ia mengembalikan ketiga puluh uang perak itu kepada imam-imam kepala dan pemimpin-pemimpin Yahudi.
\par 4 Ia berkata, "Saya sudah berdosa mengkhianati orang yang tidak bersalah, sampai Ia dihukum mati!" Tetapi mereka menjawab, "Peduli apa kami? Itu urusanmu!"
\par 5 Yudas melempar uang itu ke dalam Rumah Tuhan, lalu pergi dan menggantung diri.
\par 6 Imam-imam kepala memungut uang itu dan berkata, "Uang ini uang darah. Menurut hukum agama, uang ini tidak boleh dimasukkan ke dalam tempat persembahan di Rumah Tuhan."
\par 7 Lalu sesudah mereka sepakat, mereka memakai uang itu untuk membeli tanah yang disebut Tanah Tukang Periuk. Tanah itu dipakai untuk kuburan orang-orang asing.
\par 8 Itulah sebabnya sampai hari ini tanah itu dinamakan "Tanah Darah".
\par 9 Dengan itu, terjadilah apa yang dikatakan oleh Nabi Yeremia, yaitu, "Mereka menerima tiga puluh uang perak, yaitu harga yang disetujui oleh bangsa Israel sebagai bayaran untuk Dia.
\par 10 Dan uang itu mereka pakai untuk membeli Tanah Tukang Periuk, seperti yang diperintahkan Tuhan kepadaku."
\par 11 Waktu Yesus menghadap Pilatus, gubernur negeri itu, Pilatus bertanya, "Apakah Engkau raja orang Yahudi?" "Begitulah katamu," jawab Yesus.
\par 12 Tetapi waktu imam-imam kepala dan pemimpin-pemimpin Yahudi mengemukakan banyak tuduhan terhadap Yesus, Ia tidak menjawab sama sekali.
\par 13 Sebab itu Pilatus berkata kepada-Nya, "Apakah Engkau tidak mendengar semua yang mereka tuduhkan kepada-Mu itu?"
\par 14 Tetapi Yesus tidak menjawab sedikit pun sehingga gubernur itu heran sekali.
\par 15 Pada setiap Perayaan Paskah, gubernur biasanya melepaskan seorang tahanan menurut pilihan orang banyak.
\par 16 Pada waktu itu ada seorang hukuman yang terkenal. Namanya Yesus Barabas.
\par 17 Jadi, waktu orang banyak sudah berkumpul, Pilatus bertanya kepada mereka, "Siapakah yang kalian mau saya lepaskan untuk kalian? Yesus Barabas atau Yesus yang disebut Kristus?"
\par 18 Pilatus berkata begitu sebab ia tahu, bahwa penguasa-penguasa Yahudi menyerahkan Yesus kepadanya karena mereka iri hati.
\par 19 Pada waktu Pilatus sedang duduk di balai pengadilan, istrinya mengirim pesan ini kepadanya, "Janganlah engkau mencampuri perkara orang yang tidak bersalah itu, sebab oleh karena Dia, saya mendapat mimpi yang ngeri hari ini."
\par 20 Tetapi imam-imam kepala dan pemimpin-pemimpin Yahudi terus saja menghasut orang banyak itu untuk meminta kepada Pilatus supaya Barabas dibebaskan dan Yesus dihukum mati.
\par 21 Lalu gubernur itu bertanya lagi kepada mereka, "Dari kedua orang itu, siapakah yang kalian mau saya bebaskan untuk kalian?" "Barabas," jawab mereka.
\par 22 "Kalau begitu, saya harus buat apa dengan Yesus yang disebut Kristus?" tanya Pilatus kepada mereka. "Salibkan Dia!" jawab mereka semua.
\par 23 "Tetapi apa kejahatan-Nya?" tanya Pilatus. Lalu mereka berteriak lebih keras lagi, "Salibkan Dia!"
\par 24 Akhirnya Pilatus menyadari bahwa ia tidak bisa berbuat apa-apa lagi dan bahwa orang-orang itu mungkin akan memberontak. Jadi ia mengambil air, lalu di hadapan orang banyak itu ia mencuci tangannya dan berkata, "Saya tidak bertanggung jawab atas kematian orang ini! Itu urusan kalian!"
\par 25 Seluruh orang banyak itu menjawab, "Ya, biarlah kami dan anak-anak kami menanggung hukuman atas kematian-Nya!"
\par 26 Lalu Pilatus melepaskan Barabas untuk mereka, dan menyuruh orang mencambuk Yesus; dan menyerahkan Dia untuk disalibkan.
\par 27 Kemudian prajurit-prajurit Pilatus membawa Yesus masuk ke istana gubernur, dan seluruh pasukan berkumpul di sekeliling Yesus.
\par 28 Mereka membuka pakaian Yesus, dan mengenakan kepada-Nya jubah ungu.
\par 29 Mereka membuat sebuah mahkota dari ranting-ranting berduri, dan memasangnya pada kepala Yesus. Kemudian mereka menaruh sebatang tongkat pada tangan kanan-Nya, lalu berlutut di hadapan-Nya dan mengejek Dia. "Daulat Raja Orang Yahudi!" kata mereka.
\par 30 Mereka meludahi Dia, dan mengambil tongkat itu, lalu memukul Dia di kepala-Nya.
\par 31 Sesudah mempermainkan Dia, mereka membuka jubah ungu itu lalu mengenakan kembali pakaian-Nya sendiri. Kemudian Ia dibawa ke luar untuk disalibkan.
\par 32 Di tengah jalan, mereka berjumpa dengan seorang dari Kirene bernama Simon. Mereka memaksa orang itu memikul salib Yesus.
\par 33 Kemudian mereka sampai di suatu tempat yang bernama Golgota, yang artinya "Tempat Tengkorak".
\par 34 Di situ mereka memberi Yesus minum anggur yang bercampur empedu. Tetapi sesudah Yesus mencicipi anggur itu, Ia tidak mau meminumnya.
\par 35 Kemudian mereka menyalibkan Dia, dan membagi-bagikan pakaian-Nya dengan undian.
\par 36 Setelah itu mereka duduk menjaga Dia di sana.
\par 37 Di atas kepala-Nya mereka memasang tulisan mengenai tuduhan terhadap-Nya, yaitu: "Inilah Yesus, Raja Orang Yahudi".
\par 38 Bersama-sama dengan Dia mereka menyalibkan juga dua orang penyamun; seorang di sebelah kanan, seorang lagi di sebelah kiri-Nya.
\par 39 Orang-orang yang lewat di situ menggeleng-gelengkan kepala, dan menghina Yesus.
\par 40 Mereka berkata, "Kau yang mau merobohkan Rumah Allah, dan membangunnya dalam tiga hari! Kalau Kau Anak Allah, turunlah dari salib itu, dan selamatkan diri-Mu!"
\par 41 Begitu juga imam-imam kepala dan guru-guru agama serta pemimpin-pemimpin Yahudi mengejek Yesus. Mereka berkata,
\par 42 "Ia menyelamatkan orang lain, padahal diri-Nya sendiri Ia tidak dapat selamatkan! Kalau Dia raja Israel, baiklah Ia sekarang turun dari salib itu, baru kami mau percaya kepada-Nya.
\par 43 Ia percaya kepada Allah, dan berkata bahwa Ia Anak Allah. Nah, mari kita lihat apakah Allah mau menyelamatkan Dia sekarang."
\par 44 Penyamun-penyamun yang disalibkan dengan Dia itu pun malah menghina Dia juga seperti itu.
\par 45 Pada tengah hari, selama tiga jam, seluruh negeri itu menjadi gelap.
\par 46 Pukul tiga sore, Yesus berteriak dengan suara keras, "Eli, Eli, lama sabakhtani?" yang berarti, "Ya Allah-Ku, ya Allah-Ku, mengapakah Engkau meninggalkan Aku?"
\par 47 Beberapa orang di situ mendengar jeritan itu, dan berkata, "Ia memanggil Elia!"
\par 48 Seorang dari mereka cepat-cepat pergi mengambil bunga karang, dan mencelupkannya ke dalam anggur asam. Kemudian ia mencucukkannya pada ujung sebatang kayu, dan mengulurkannya ke bibir Yesus.
\par 49 Tetapi orang-orang lain berkata, "Tunggu, mari kita lihat apakah Elia datang menyelamatkan Dia!"
\par 50 Kemudian Yesus berteriak lagi dengan suara keras, lalu menghembuskan napas-Nya yang penghabisan.
\par 51 Gorden yang tergantung di dalam Rumah Tuhan sobek menjadi dua dari atas sampai ke bawah. Bumi bergetar dan gunung-gunung batu terbelah.
\par 52 Kuburan-kuburan terbuka, dan banyak umat Allah yang sudah meninggal dihidupkan kembali.
\par 53 Mereka keluar dari kuburan-kuburan sesudah Yesus bangkit dari kematian, dan mereka masuk ke Yerusalem. Dan di sana banyak orang melihat mereka.
\par 54 Kepala pasukan bersama-sama dengan prajurit-prajurit yang sedang menjaga Yesus menjadi ketakutan sekali waktu melihat gempa bumi, dan semua yang terjadi itu. Mereka berkata, "Sungguh, Dia ini Anak Allah!"
\par 55 Di situ ada juga banyak wanita yang sedang melihat dari jauh. Merekalah yang sudah mengikuti Yesus untuk menolong Dia sejak dari Galilea.
\par 56 Di antaranya ialah Maria Magdalena, Maria ibu Yakobus dan Yusuf, dan ibu anak-anak Zebedeus.
\par 57 Malam itu datanglah seorang kaya dari Arimatea, yang bernama Yusuf. Ia juga pengikut Yesus.
\par 58 Ia pergi kepada Pilatus, dan minta jenazah Yesus. Lalu Pilatus memerintahkan supaya jenazah Yesus diberikan kepadanya.
\par 59 Maka Yusuf mengambil jenazah itu, dan membungkusnya dengan kain kapan dari linen yang baru.
\par 60 Lalu ia meletakkan jenazah Yesus di dalam kuburan kepunyaannya sendiri yang dibuat di dalam sebuah bukit batu. Sesudah itu ia menggulingkan sebuah batu besar menutupi pintu kubur itu, lalu pergi.
\par 61 Maria Magdalena dan Maria yang lain tinggal duduk di situ menghadapi kuburan itu.
\par 62 Keesokan harinya, pada hari Sabat, imam-imam kepala dan orang-orang Farisi pergi bersama-sama menghadap Pilatus
\par 63 dan berkata, "Tuan, kami ingat waktu penipu itu masih hidup, Ia pernah berkata, 'Sesudah tiga hari Aku akan bangkit.'
\par 64 Karena itu, suruhlah orang menjaga kuburan itu baik-baik sampai hari yang ketiga, supaya pengikut-pengikut-Nya tidak dapat mencuri mayat-Nya lalu berkata kepada orang-orang bahwa Ia sudah dibangkitkan dari kematian. Dan penipuan yang terakhir ini akan lebih buruk daripada yang pertama."
\par 65 "Kalian punya tentara pengawal," kata Pilatus kepada mereka, "pergilah menjaga kuburan itu seketat mungkin."
\par 66 Lalu mereka pergi ke kuburan, menyegel batu penutupnya dan menempatkan penjagaan di depannya, supaya tidak ada yang mengganggu kuburan itu.

\chapter{28}

\par 1 Ketika hari Sabat sudah lewat, pada hari Minggu pagi-pagi sekali, Maria Magdalena dan Maria yang lain itu pergi melihat kuburan itu.
\par 2 Tiba-tiba terjadi gempa bumi yang hebat. Seorang malaikat Tuhan turun dari surga lalu menggulingkan batu penutup itu, dan duduk di atasnya.
\par 3 Wajah malaikat itu seperti kilat, dan pakaiannya putih sekali.
\par 4 Tentara pengawal yang menjaga di situ begitu ketakutan sampai mereka gemetar, dan menjadi seperti orang mati.
\par 5 Malaikat itu berkata kepada wanita-wanita itu, "Janganlah takut! Aku tahu kalian mencari Yesus yang sudah disalibkan itu.
\par 6 Ia tidak ada di sini. Ia sudah bangkit seperti yang sudah dikatakan-Nya dahulu. Mari lihat tempatnya Ia dibaringkan.
\par 7 Sekarang, pergilah cepat-cepat, beritahukan kepada pengikut-pengikut-Nya, 'Ia sudah bangkit, dan sekarang Ia pergi lebih dahulu dari kalian ke Galilea. Di sana kalian akan melihat Dia!' Ingatlah apa yang sudah kukatakan kepadamu."
\par 8 Cepat-cepat wanita-wanita itu meninggalkan kuburan itu. Dengan perasaan takut bercampur gembira, mereka berlari-lari untuk memberitahukan hal itu kepada pengikut-pengikut Yesus.
\par 9 Tiba-tiba Yesus datang menemui wanita-wanita itu, dan berkata, "Salam!" Lalu mereka datang mendekati Dia, kemudian memeluk kaki-Nya dan menyembah Dia.
\par 10 "Janganlah takut," kata Yesus kepada mereka, "pergi beritahukan kepada saudara-saudara-Ku supaya mereka pergi ke Galilea; di sana mereka akan melihat Aku."
\par 11 Sementara wanita-wanita itu pergi, beberapa dari tentara pengawal yang menjaga kuburan itu kembali ke kota, dan melaporkan kepada imam-imam kepala semua yang sudah terjadi.
\par 12 Imam-imam kepala itu berunding dengan pemimpin-pemimpin Yahudi, lalu memberi sejumlah besar uang kepada tentara pengawal itu,
\par 13 dan berkata, "Kalian harus mengatakan bahwa pengikut-pengikut Yesus datang pada malam hari, dan mencuri mayat-Nya waktu kalian sedang tidur.
\par 14 Dan kalau gubernur mendengar hal itu, kami akan membujuk dia supaya kalian tidak mendapat kesulitan apa-apa."
\par 15 Maka tentara pengawal itu mengambil uang itu, dan melakukan seperti yang dipesankan kepada mereka. Oleh karena itu cerita itu masih tersiar di antara orang Yahudi sampai pada hari ini.
\par 16 Kesebelas pengikut Yesus itu pergi ke bukit di Galilea sesuai dengan yang diperintahkan Yesus kepada mereka.
\par 17 Pada waktu mereka melihat Yesus di sana, mereka sujud menyembah Dia. Tetapi ada di antara mereka yang ragu-ragu.
\par 18 Yesus mendekati mereka, dan berkata, "Seluruh kuasa di surga dan di bumi sudah diserahkan kepada-Ku.
\par 19 Sebab itu pergilah kepada segala bangsa di seluruh dunia, jadikanlah mereka pengikut-pengikut-Ku. Baptiskan mereka dengan menyebut nama Bapa, dan Anak, dan Roh Allah.
\par 20 Ajarkan mereka mentaati semua yang sudah Kuperintahkan kepadamu. Dan ingatlah Aku akan selalu menyertai kalian sampai akhir zaman."


\end{document}