\begin{document}

\title{Ezra}

Ezr 1:1  Pada tahun pertama pemerintahan Kores, raja Persia, TUHAN melaksanakan apa yang telah diucapkan-Nya melalui Nabi Yeremia. TUHAN mendorong hati Kores untuk mengeluarkan sebuah perintah yang dibacakan di seluruh kerajaannya. Bunyinya demikian:
Ezr 1:2  "Inilah perintah Kores, raja Persia. TUHAN, Allah penguasa di surga telah menjadikan aku raja atas seluruh dunia, dan menugaskan aku untuk membangun rumah bagi-Nya di Yerusalem di daerah Yehuda.
Ezr 1:3  Semoga Allah melindungi kamu sekalian yang menjadi umat-Nya. Kamu harus pulang ke Yerusalem dan membangun kembali Rumah TUHAN, Allah Israel, Allah yang diam di Yerusalem.
Ezr 1:4  Jika ada seorang dari umat-Nya di tempat pembuangan ini yang memerlukan bantuan untuk dapat pulang, maka tetangga-tetangganya harus memberikan bantuan itu. Mereka harus memberikan kepadanya perak dan emas, bekal untuk di jalan serta binatang muatan, dan juga persembahan-persembahan untuk Rumah TUHAN di Yerusalem."
Ezr 1:5  Kemudian para kepala kaum dari suku Yehuda dan Benyamin, para imam dan orang-orang Lewi, dan setiap orang yang hatinya digerakkan oleh Allah, berkemas-kemas untuk pulang ke Yerusalem dan membangun kembali Rumah TUHAN.
Ezr 1:6  Semua tetangga mereka ikut membantu dengan memberikan barang-barang, seperti perkakas-perkakas perak dan emas, bekal untuk di jalan, binatang muatan, harta benda lainnya dan persembahan-persembahan untuk Rumah TUHAN.
Ezr 1:7  Kores mengeluarkan mangkuk-mangkuk dan cawan-cawan yang dahulu dipakai dalam Rumah TUHAN di Yerusalem. Barang-barang itu telah diangkut oleh Raja Nebukadnezar dan ditaruh di dalam kuil dewa-dewanya sendiri.
Ezr 1:8  Kores menyerahkan urusan benda-benda itu kepada Mitredat, kepala perbendaharaan kerajaan. Kemudian Mitredat membuat daftar dari barang-barang itu untuk Sesbazar, gubernur Yehuda.
Ezr 1:9  Inilah daftarnya: mangkuk-mangkuk emas untuk upacara persembahan-30; mangkuk-mangkuk perak untuk upacara persembahan-1.000; mangkuk-mangkuk lainnya-29; mangkuk-mangkuk kecil dari emas-30; mangkuk-mangkuk kecil dari perak-410; perkakas lainnya-1.000.
Ezr 1:11  Seluruhnya berjumlah 5.400 mangkuk emas dan perak serta barang-barang lain. Semuanya itu dibawa oleh Sesbazar ketika dia dan orang-orang buangan yang lain berangkat dari Babel menuju ke Yerusalem.
Ezr 2:1  Banyak di antara orang-orang buangan meninggalkan daerah Babel lalu kembali ke Yerusalem dan Yehuda, masing-masing ke kotanya sendiri. Mereka telah hidup dalam pembuangan di Babel sejak mereka diangkut ke sana oleh Raja Nebukadnezar.
Ezr 2:2  Pemimpin-pemimpin mereka adalah Zerubabel, Yesua, Nehemia, Seraya, Reelaya, Mordekhai, Bilsan, Mispar, Bigwai, Rehum dan Baana. Inilah daftar kaum keluarga Israel, dengan jumlah orang dari setiap kaum yang kembali dari pembuangan:
Ezr 2:3  Kaum keluarga Paros-2.172; Sefaca-372; Arah-775; Pahat-Moab (keturunan Yesua dan Yoab) -2.812; Elam-1.254; Zatu-945; Zakai-760; Bani-642; Bebai-623; Azgad-1.222; Adonikam-666; Bigwai-2.056; Adin-454; Ater (disebut juga Hizkia) -98; Bezai-323; Yora-112; Hasum-223; Gibar-95.
Ezr 2:21  Orang-orang yang leluhurnya diam di kota-kota berikut ini juga kembali dari pembuangan: Kota Betlehem-123; Netofa-56; Anatot-128; Asmawet-42; Kiryat-Arim, Kefira dan Beerot-743; Rama dan Gaba-621; Mikhmas-122; Betel dan Ai-223; Nebo-52; Magbis-156; Elam yang lain-1.254; Harim-320; Lod, Hadid dan Ono-725; Yerikho-345; Senaa-3.630.
Ezr 2:36  Inilah daftar kaum keluarga para imam yang pulang dari pembuangan: Yedaya (keturunan Yesua) -973; Imer-1.052; Pasyhur-1.247; Harim-1.017.
Ezr 2:40  Kaum keluarga Lewi yang pulang dari pembuangan ialah: Yesua dan Kadmiel (keturunan Hodawya) -74. Para penyanyi di Rumah TUHAN (keturunan Asaf) -128. Penjaga gerbang di Rumah TUHAN (keturunan Salum, Ater, Talmon, Akub, Hatita dan Sobai) -139.
Ezr 2:43  Inilah daftar kaum keluarga para pekerja di Rumah TUHAN yang pulang dari pembuangan: Ziha, Hasufa, Tabaot, Keros, Siaha, Padon, Lebana, Hagaba, Akub, Hagab, Samlai, Hanan, Gidel, Gahar, Reaya, Rezin, Nekoda, Gazam, Uza, Paseah, Besai, Asna, Meunim, Nefusim, Bakbuk, Hakufa, Harhur, Bazlut, Mehida, Harsa, Barkos, Sisera, Temah, Neziah dan Hatifa.
Ezr 2:55  Kaum keluarga para pelayan Salomo yang pulang dari pembuangan: Sotai, Soferet, Peruda, Yaala, Darkon, Gidel, Sefaca, Hatil, Pokheret-Hazebaim dan Ami.
Ezr 2:58  Kaum keluarga para pekerja di Rumah TUHAN dan para pelayan Salomo yang kembali dari pembuangan berjumlah 392 orang.
Ezr 2:59  Orang-orang yang berangkat dari kota-kota Tel-Melah, Tel-Harsa, Kerub, Adan dan Imer, berjumlah 652 orang. Mereka termasuk kaum keluarga Delaya, Tobia dan Nekoda, tetapi mereka tidak dapat membuktikan bahwa mereka adalah keturunan bangsa Israel.
Ezr 2:61  Beberapa kaum keluarga imam juga tidak dapat menemukan catatan mengenai leluhur mereka, yaitu kaum keluarga Habaya, Hakos, dan Barzilai. (Leluhur kaum Barzilai kawin dengan seorang perempuan keturunan kaum Barzilai di Gilead, dan kemudian ia memakai nama keluarga mertuanya.) Karena mereka tidak dapat membuktikan siapa leluhur mereka, maka mereka tidak diterima sebagai imam.
Ezr 2:63  Gubernur daerah Yehuda melarang mereka makan makanan yang dipersembahkan kepada Allah, sampai ada seorang imam yang dapat minta petunjuk dengan memakai Urim dan Tumim.
Ezr 2:64  Orang-orang buangan yang pulang ke negerinya seluruhnya berjumlah 42.360 orang. Selain itu pulang juga para pembantu mereka sejumlah 7.337 orang dan penyanyi sejumlah 200 orang. Mereka juga membawa binatang-binatang mereka, yaitu: Kuda-736; Bagal-245; Unta-435; Keledai-6.720.
Ezr 2:68  Ketika orang-orang buangan sampai di Rumah TUHAN di Yerusalem, beberapa pemimpin kaum memberikan persembahan sukarela untuk membangun kembali Rumah TUHAN.
Ezr 2:69  Mereka memberi sumbangan menurut kemampuan masing-masing dan hasilnya ialah 500 kg emas, 2.800 kg perak, dan 100 jubah untuk para imam.
Ezr 2:70  Para imam, orang-orang Lewi dan sebagian dari rakyat menetap di Yerusalem dan di kota-kota di dekatnya. Para penyanyi, para penjaga gerbang Rumah TUHAN dan para pekerja di Rumah TUHAN serta orang-orang Israel selebihnya menetap di kota-kota tempat tinggal leluhur mereka dahulu.
Ezr 3:1  Pada bulan tujuh, orang-orang Israel telah menetap di kotanya masing-masing. Pada tanggal satu bulan itu mereka berkumpul di Yerusalem,
Ezr 3:2  lalu Yesua anak Yozadak serta rekan-rekannya para imam dan Zerubabel anak Sealtiel serta sanak saudaranya, membangun kembali mezbah Allah Israel. Dengan demikian mereka dapat membakar kurban di atasnya, sesuai dengan petunjuk-petunjuk yang tertulis dalam buku Hukum Musa, hamba Allah.
Ezr 3:3  Meskipun mereka takut kepada penduduk di negeri itu, namun mereka membangun juga mezbah itu di tempatnya semula. Lalu mulailah mereka mempersembahkan lagi kurban bakaran, di waktu pagi dan petang.
Ezr 3:4  Mereka juga merayakan Hari Raya Pondok Daun, sesuai dengan peraturan; dan setiap hari mempersembahkan kurban yang telah ditetapkan untuk hari itu.
Ezr 3:5  Selain itu mereka secara teratur mempersembahkan kurban bakaran serta kurban pada Hari Raya Bulan Baru, dan pada semua hari raya lain yang diadakan untuk menyembah TUHAN. Mereka juga mempersembahkan persembahan sukarela kepada TUHAN.
Ezr 3:6  Meskipun Rumah TUHAN belum mulai dibangun kembali, namun pada tanggal satu bulan tujuh rakyat telah mulai membakar kurban untuk TUHAN.
Ezr 3:7  Rakyat menyumbangkan uang untuk mengupah tukang batu dan tukang kayu. Mereka juga mengirim makanan, minuman serta minyak zaitun ke kota-kota Tirus dan Sidon untuk ditukar dengan kayu cemara Libanon dari kota-kota itu. Kayu itu dibawa ke Yafo melalui laut. Semua itu dikerjakan oleh rakyat dengan izin Kores, raja Persia.
Ezr 3:8  Maka pada bulan dua, dalam tahun kedua setelah orang Israel kembali ke Yerusalem, mulailah mereka membangun kembali Rumah TUHAN. Zerubabel, Yesua dan teman-teman mereka sebangsa, para imam, orang Lewi, pendek kata, semua orang bekas buangan, ikut bekerja. Orang-orang Lewi yang berumur 20 tahun ke atas ditunjuk untuk mengawasi pekerjaan pembangunan Rumah TUHAN itu.
Ezr 3:9  Maka Yesua, orang Lewi itu, dengan anak-anaknya dan saudara-saudaranya dan juga Kadmiel dengan anak-anaknya (kaum Hodawya) bersama-sama mengawasi pekerjaan pembangunan Rumah TUHAN itu. Mereka dibantu oleh orang Lewi dari kaum Henadad.
Ezr 3:10  Ketika pekerja-pekerja bangunan mulai meletakkan pondasi Rumah TUHAN, para imam berpakaian jubah dan dengan memegang trompet mengambil tempat masing-masing. Orang-orang Lewi dari kaum Asaf juga berdiri di situ dengan memegang gong-gong kecil. Mereka memuji-muji TUHAN sesuai dengan peraturan yang telah ditetapkan sejak masa Raja Daud.
Ezr 3:11  Rakyat menyanyikan puji-pujian bagi TUHAN, sambil mengulang-ulang bagiannya yang terakhir, demikian bunyinya, "TUHAN itu baik; dan kasihnya kepada Israel kekal abadi." Seluruh rakyat ikut bersorak dan memuji TUHAN, sebab pondasi Rumah TUHAN sudah diletakkan.
Ezr 3:12  Banyak di antara para imam, orang-orang Lewi dan kepala-kepala kaum yang sudah lanjut umurnya pernah melihat Rumah TUHAN yang dahulu. Ketika mereka menyaksikan peletakkan pondasi Rumah TUHAN yang ini, menangislah mereka keras-keras. Tetapi orang-orang lain yang hadir di situ bersorak-sorai kegirangan,
Ezr 3:13  sehingga tak dapat dibedakan lagi antara sorak kegirangan dan tangis. Bunyinya begitu nyaring, sehingga terdengar sampai jauh.
Ezr 4:1  Berita bahwa orang-orang bekas buangan itu sedang membangun Rumah TUHAN, Allah Israel, terdengar oleh musuh orang Yehuda dan Benyamin.
Ezr 4:2  Sebab itu mereka menemui Zerubabel dan para kepala kaum, lalu mengusulkan, "Izinkanlah kami ikut membangun Rumah TUHAN itu, sebab kami juga berbakti kepada Allah saudara-saudara, sama seperti saudara sekalian. Kami selalu mempersembahkan kurban kepadanya sejak Esar-Hadon, raja Asyur membawa kami ke mari."
Ezr 4:3  Tetapi Zerubabel, Yesua dan para kepala kaum itu menjawab, "Kami tidak memerlukan bantuan kalian untuk membangun rumah bagi TUHAN Allah kami. Kami akan membangunnya sendiri sesuai dengan perintah Kores, raja Persia, kepada kami."
Ezr 4:4  Kemudian penduduk yang sudah lama tinggal di negeri itu mulai melemahkan semangat orang Yahudi serta menakut-nakuti mereka supaya pembangunan itu dihentikan.
Ezr 4:5  Mereka menyogok pejabat-pejabat pemerintah Persia supaya menggagalkan rencana orang Yahudi itu. Hal itu terus berlangsung selama pemerintahan Raja Kores sampai pemerintahan Raja Darius.
Ezr 4:6  Pada awal pemerintahan Raja Ahasyweros, musuh-musuh penduduk Yerusalem dan Yehuda menulis surat gugatan.
Ezr 4:7  Juga pada zaman pemerintahan Artahsasta, kaisar Persia, kejadian itu berulang: Bislam, Mitredat, Tabeel dan rekan-rekan mereka menulis surat kepada raja. Surat itu ditulis dalam bahasa Aram, dan harus diterjemahkan ketika dibacakan.
Ezr 4:8  Artahsasta juga menerima surat mengenai Yerusalem. Surat itu dikirim oleh Gubernur Rehum dan Simsai sekretaris provinsi dan didukung oleh rekan-rekan mereka, yaitu para hakim dan semua pejabat Persia lain yang berasal dari Erekh, Babel, dan Susan di negeri Elam,
Ezr 4:10  juga oleh bangsa-bangsa lain yang telah diangkut Asnapar Yang Agung dan disuruh menetap di kota Samaria dan di kota-kota lain dalam provinsi Efrat Barat.
Ezr 4:11  Begini bunyi surat itu, "Ke hadapan Raja Artahsasta, dari para hamba Baginda yang tinggal di Efrat Barat. Salam.
Ezr 4:12  Kiranya Baginda maklum, bahwa orang Yahudi yang telah meninggalkan wilayah-wilayah lain dalam kerajaan Baginda, kini menetap di Yerusalem dan sedang membangun kembali kota yang jahat dan suka memberontak itu. Mereka sudah mulai mendirikan lagi tembok-temboknya dan tak lama lagi akan menyelesaikan pekerjaan itu.
Ezr 4:13  Jika kota itu berhasil dibangun kembali dan tembok-temboknya selesai didirikan, penduduknya tidak akan mau lagi membayar pajak, lalu pendapatan kerajaan akan berkurang.
Ezr 4:14  Kami sudah berhutang budi kepada Baginda dan tidak ingin melihat Baginda dirugikan. Karena itu kami mengusulkan,
Ezr 4:15  agar Baginda menyuruh meneliti arsip para leluhur Baginda. Dari isinya Baginda akan mengetahui bahwa kota Yerusalem memang suka memberontak dan sejak dahulu selalu merugikan para raja dan penguasa provinsi. Penduduknya selalu susah dipimpin. Itu sebabnya kota itu telah dihancurkan.
Ezr 4:16  Jadi, kami yakin, bahwa apabila kota itu dibangun kembali dan tembok-temboknya didirikan, Baginda tidak akan dapat lagi menguasai provinsi Efrat Barat ini."
Ezr 4:17  Raja mengirim jawaban ini, "Kepada Gubernur Rehum dan Simsai sekretaris provinsi, serta rekan-rekan mereka yang tinggal di Samaria dan di kota-kota lain di provinsi Efrat Barat. Salam.
Ezr 4:18  Surat yang Saudara-saudara kirimkan telah diterjemahkan dan dibacakan kepadaku.
Ezr 4:19  Lalu atas perintahku sudah diadakan penelitian, dan memang terbukti bahwa sejak zaman dahulu Yerusalem selalu melawan kuasa raja-raja dan banyak di antara penduduknya adalah pemberontak dan perusuh.
Ezr 4:20  Pernah raja-raja yang perkasa memerintah di sana dan menguasai seluruh provinsi Efrat Barat serta menjalankan pajak dan upeti.
Ezr 4:21  Sebab itu, keluarkanlah perintah untuk menghentikan pekerjaan orang-orang itu, supaya kota itu jangan dibangun kembali sampai aku memberi perintah lain.
Ezr 4:22  Laksanakanlah perintahku itu dengan segera supaya jangan bertambah besar lagi kerugianku!"
Ezr 4:23  Segera setelah surat Artahsasta itu dibacakan kepada Rehum, Simsai dan rekan-rekan mereka, berangkatlah mereka ke Yerusalem dan memaksa orang Yahudi menghentikan pekerjaan membangun kembali kota itu.
Ezr 4:24  Pekerjaan Rumah TUHAN telah dihentikan dan tidak diteruskan sampai tahun kedua pemerintahan Darius, raja Persia.
Ezr 5:1  Pada waktu itu dua orang nabi, yaitu Hagai dan Zakharia anak Ido, mulai berbicara atas nama Allah Israel kepada orang Yahudi yang tinggal di Yehuda dan di Yerusalem.
Ezr 5:2  Mendengar pesan kedua nabi itu, Zerubabel anak Sealtiel dan Yesua anak Yozadak segera mulai membangun kembali Rumah TUHAN di Yerusalem, dibantu oleh kedua nabi itu.
Ezr 5:3  Tetapi Tatnai, gubernur provinsi Efrat Barat, bersama dengan Syetar-Boznai dan rekan-rekan mereka segera datang ke Yerusalem dan bertanya kepada orang-orang Yahudi, "Siapa menyuruh kamu membangun dan memperlengkapi Rumah Ibadat ini?"
Ezr 5:4  Mereka menanyakan juga nama semua orang yang membantu mendirikan Rumah TUHAN itu.
Ezr 5:5  Tetapi Allah memperhatikan dan melindungi pemimpin-pemimpin Yahudi itu, sehingga para pejabat Persia memutuskan untuk tidak mengambil tindakan sebelum mereka mengirim surat kepada Darius dan menerima jawabannya.
Ezr 5:6  Inilah laporan yang mereka kirimkan kepada Raja,
Ezr 5:7  "Ke hadapan Raja Darius. Semoga Baginda memerintah dengan sejahtera!
Ezr 5:8  Kiranya Baginda maklum, bahwa kami telah pergi ke provinsi Yehuda dan melihat bahwa Rumah TUHAN, Allah Yang Besar, sedang dibangun kembali dengan batu-batu besar dan tembok-tembok yang dilapisi kayu. Pekerjaan itu dilakukan dengan sangat teliti dan berjalan dengan lancar.
Ezr 5:9  Kemudian kami tanyakan kepada para pemimpin bangsa itu, siapa yang telah memberi izin kepada mereka untuk membangun kembali Rumah Ibadat itu dan memperlengkapinya.
Ezr 5:10  Kami tanyakan juga nama-nama mereka supaya dapat melaporkan kepada Baginda siapa saja yang mengepalai pekerjaan itu.
Ezr 5:11  Mereka menjawab, 'Kami ini hamba Allah penguasa alam semesta. Kami sedang membangun kembali Rumah TUHAN yang didirikan dan diperlengkapi berpuluh-puluh tahun yang lalu oleh seorang raja Israel yang agung.
Ezr 5:12  Tetapi karena para leluhur kami telah menimbulkan kemarahan Allah penguasa di surga, maka Ia menyerahkan mereka kepada kekuasaan Nebukadnezar, raja Babel dari keturunan Kasdim. Ia menghancurkan Rumah itu dan mengangkut bangsa kami ke Babel.
Ezr 5:13  Kemudian Kores menjadi raja Babel dan pada tahun pertama pemerintahannya, ia mengeluarkan perintah untuk membangun kembali Rumah TUHAN itu.
Ezr 5:14  Bahkan dikembalikannya juga perkakas-perkakas emas dan perak yang dipakai dalam Rumah TUHAN dan yang diambil dari Yerusalem oleh Nebukadnezar lalu dimasukkan ke dalam kuil di Babel. Perkakas-perkakas itu diserahkan oleh Raja Kores kepada orang yang bernama Sesbazar, yang telah diangkatnya menjadi gubernur Yehuda.
Ezr 5:15  Raja menyuruh dia mengambil perkakas-perkakas itu dan mengembalikannya ke Rumah TUHAN di Yerusalem, serta membangun kembali Rumah itu.
Ezr 5:16  Lalu datanglah Sesbazar ke Yerusalem dan meletakkan pondasi Rumah TUHAN itu. Sejak waktu itu sampai sekarang pembangunan itu berjalan terus, tetapi belum juga selesai.'
Ezr 5:17  Jika Baginda mengizinkan, kami mengusulkan supaya diadakan penyelidikan di dalam arsip kerajaan di Babel. Dengan demikian dapat diketahui apakah benar ada perintah Raja Kores untuk membangun kembali Rumah TUHAN di Yerusalem ini. Kemudian kami mohon kabar tentang keputusan Baginda mengenai perkara itu."
Ezr 6:1  Setelah itu atas perintah Raja Darius, diadakan penyelidikan dalam arsip kerajaan yang disimpan di Babel.
Ezr 6:2  Tetapi catatan mengenai hal itu ditemukan bukannya di Babel, melainkan di kota Ahmeta di daerah Media. Catatan itu ditulis pada sebuah gulungan dan berbunyi begini,
Ezr 6:3  "Pada tahun pertama pemerintahan Raja Kores, ia mengeluarkan perintah supaya Rumah TUHAN di Yerusalem dibangun kembali sebagai tempat orang membawa persembahan dan membakar kurban. Tinggi Rumah itu harus 27 meter dan lebarnya 27 meter juga.
Ezr 6:4  Dinding-dindingnya harus dibangun seperti berikut: di atas setiap tiga lapis batu harus dipasang satu lapis kayu. Biayanya akan ditanggung oleh kas kerajaan.
Ezr 6:5  Juga perkakas-perkakas emas dan perak yang dipakai di Rumah TUHAN dan telah diambil dari Yerusalem oleh Raja Nebukadnezar serta diangkut ke Babel, harus dikembalikan ke tempatnya semula."
Ezr 6:6  Kemudian Darius mengirimkan jawaban berikut ini, "Kepada Tatnai, gubernur provinsi Efrat Barat, dan Syetar-Boznai, serta para pejabat di Efrat Barat. Jangan mengganggu orang-orang Yahudi itu,
Ezr 6:7  dan jangan menghalang-halangi pembangunan Rumah Ibadat mereka. Biarkanlah gubernur Yehuda dan para pemimpin Yahudi membangun kembali Rumah itu.
Ezr 6:8  Bersama ini aku memerintahkan Saudara-saudara untuk membantu mereka dalam pekerjaan pembangunan itu dengan cara ini: Semua biaya pembangunan harus diambil dari hasil pajak negara di provinsi Efrat Barat dan harus dibayarkan kepada orang-orang itu tanpa ditunda-tunda.
Ezr 6:9  Selain itu apa saja yang diperlukan menurut para imam di Yerusalem, seperti misalnya sapi jantan muda, domba jantan dan anak domba untuk kurban persembahan kepada Allah penguasa di surga, juga gandum, garam, anggur dan minyak zaitun, harus diberikan kepada mereka setiap hari tanpa lalai.
Ezr 6:10  Laksanakan semua itu supaya mereka dapat mempersembahkan kurban yang menyenangkan Allah penguasa di surga, dan supaya mereka memohonkan berkat bagiku dan bagi para putraku.
Ezr 6:11  Aku memerintahkan supaya orang yang melanggar perintah ini dari rumahnya dicabut sebuah tiang, diruncingkan ujungnya, lalu ditusukkan ke dalam badan orang itu sampai tembus. Rumahnya harus dibongkar dan dijadikan tempat sampah.
Ezr 6:12  Semoga Allah yang telah memilih Yerusalem sebagai tempat Ia disembah, membinasakan setiap raja atau bangsa yang berani melanggar perintah ini dan mencoba menghancurkan Rumah TUHAN di Yerusalem itu. Aku, Darius, telah memberikan perintah ini. Laksanakanlah dengan sebaik-baiknya."
Ezr 6:13  Perintah Raja Darius itu dilaksanakan dengan cermat oleh Tatnai, gubernur Efrat Barat, dan Syetar-Boznai serta rekan-rekan mereka.
Ezr 6:14  Didukung oleh nabi-nabi Hagai dan Zakharia, para pemimpin Yahudi melanjutkan pembangunan Rumah TUHAN itu. Mereka cepat menyelesaikan pembangunan itu sesuai dengan perintah Allah Israel dan perintah Kores, Darius dan Artahsasta, raja-raja Persia itu.
Ezr 6:15  Rumah TUHAN itu rampung pada tanggal tiga bulan Adar pada tahun keenam pemerintahan Raja Darius.
Ezr 6:16  Kemudian dengan gembira orang-orang Israel, para imam, orang-orang Lewi dan semua orang bekas buangan meresmikan Rumah TUHAN itu.
Ezr 6:17  Pada peristiwa itu mereka mempersembahkan kepada TUHAN 100 ekor sapi jantan, 200 ekor domba jantan dan 400 ekor anak domba. Untuk kurban pengampunan dosa, dipersembahkan dua belas ekor kambing jantan, seekor bagi setiap suku Israel.
Ezr 6:18  Kemudian untuk pelayanan dalam Rumah TUHAN mereka menyusun penggolongan para imam serta orang-orang Lewi menurut peraturan yang tertulis dalam buku Musa.
Ezr 6:19  Pada tanggal empat belas bulan satu tahun berikutnya, orang-orang yang telah pulang dari pembuangan, merayakan Paskah.
Ezr 6:20  Semua imam dan orang Lewi telah melakukan upacara pembersihan diri sehingga dianggap bersih. Kemudian orang-orang Lewi itu menyembelih binatang-binatang untuk kurban Paskah bagi para imam, bagi mereka sendiri dan semua orang bekas buangan lainnya.
Ezr 6:21  Kurban-kurban itu dimakan oleh semua orang Israel yang telah kembali dari pembuangan dan penduduk negeri itu yang sudah meninggalkan cara hidup yang berdosa dan datang untuk berbakti kepada TUHAN, Allah Israel.
Ezr 6:22  Tujuh hari lamanya mereka dengan gembira merayakan Pesta Roti Tidak Beragi. Mereka sangat senang sebab TUHAN telah membuat raja Asyur bermurah hati kepada mereka sehingga membantu mereka dalam pekerjaan membangun kembali Rumah TUHAN, Allah Israel.
Ezr 7:1  Bertahun-tahun kemudian, ketika Artahsasta memerintah sebagai raja Persia, ada seorang laki-laki bernama Ezra, dari keturunan Imam Agung Harun. Garis keturunannya dari bawah ke atas adalah sebagai berikut: Ezra, Seraya, Azarya, Hilkia,
Ezr 7:2  Salum, Zadok, Ahitub,
Ezr 7:3  Amarya, Azarya, Merayot,
Ezr 7:4  Zerahya, Uzi, Buki,
Ezr 7:5  Abisua, Pinehas, Eleazar, Imam Agung Harun.
Ezr 7:6  Ezra seorang yang terpelajar dan ia tahu banyak tentang Hukum yang diberikan TUHAN, Allah Israel kepada Musa. Karena Ezra diberkati TUHAN Allahnya, maka raja memberi kepadanya segala yang dimintanya. Pada tahun ketujuh pemerintahan Raja Artahsasta, berangkatlah Ezra dari Babel menuju Yerusalem bersama serombongan orang Israel yang terdiri dari sejumlah imam, orang Lewi, penyanyi serta penjaga gerbang di Rumah TUHAN, dan para pekerja.
Ezr 7:8  Mereka meninggalkan Babel pada tanggal satu bulan satu dan dengan pertolongan Allah, sampailah mereka di Yerusalem pada tanggal satu bulan lima.
Ezr 7:10  Ezra telah mencurahkan segala perhatiannya kepada penyelidikan Hukum TUHAN, untuk melakukannya serta mengajarkan segala ketentuan dan peraturannya kepada bangsa Israel.
Ezr 7:11  Inilah dokumen yang diberikan Raja Artahsasta kepada Ezra, seorang imam dan ahli Hukum,
Ezr 7:12  "Dari Raja Artahsasta kepada Imam Ezra, ahli Hukum Allah penguasa di surga.
Ezr 7:13  Aku memerintahkan bahwa semua imam, orang Lewi dan orang Israel lainnya di seluruh kerajaanku, yang ingin pergi ke Yerusalem, diizinkan pergi dengan Saudara.
Ezr 7:14  Sebab aku dan ketujuh penasihatku mengutus Saudara ke Yerusalem dan Yehuda untuk menyelidiki keadaannya dan melihat apakah Hukum Allah yang telah dipercayakan kepada Saudara, benar-benar ditaati.
Ezr 7:15  Aku dan para penasihatku mau mempersembahkan emas dan perak kepada Allah Israel, Allah yang diam di Yerusalem. Bawalah persembahan kami itu.
Ezr 7:16  Bawalah juga semua emas dan perak yang Saudara kumpulkan di seluruh Babel dan segala sumbangan bangsa Israel serta imam-imamnya bagi Rumah TUHAN Allah mereka di Yerusalem.
Ezr 7:17  Pakailah uang itu dengan cermat untuk membeli sapi jantan, domba jantan, anak domba, gandum dan anggur. Persembahkanlah semuanya itu di atas mezbah Rumah TUHAN di Yerusalem.
Ezr 7:18  Perak dan emas yang masih sisa boleh Saudara pakai untuk keperluan Saudara dan rekan-rekan Saudara, sesuai dengan kehendak Allah Saudara.
Ezr 7:19  Serahkanlah kepada Allah di Yerusalem segala perkakas yang diserahkan kepada Saudara untuk kebaktian di Rumah Allah.
Ezr 7:20  Dan segala keperluan lain bagi Rumah TUHAN, boleh Saudara minta dari kas kerajaan.
Ezr 7:21  Kepada semua pengurus kas kerajaan di provinsi Efrat Barat aku perintahkan supaya apa saja yang diminta oleh Imam Ezra, ahli Hukum Allah penguasa di surga, diberikan kepadanya dengan segera,
Ezr 7:22  sampai sebanyak jumlah ini: perak sampai 3.400 kg, gandum sampai 10.000 kg, anggur sampai 2.000 liter, minyak zaitun sampai 2.000 liter, dan garam sebanyak yang diperlukan.
Ezr 7:23  Bertindaklah cermat dalam menyediakan segala sesuatu yang diminta Allah penguasa di surga bagi Rumah-Nya, supaya Ia tidak marah kepadaku atau kepada orang-orang yang memerintah sesudah aku.
Ezr 7:24  Kalian tak boleh memungut pajak dari para imam, orang Lewi, penyanyi, penjaga pintu gerbang, pekerja, atau siapa saja yang ada hubungannya dengan Rumah TUHAN.
Ezr 7:25  Saudara Ezra, pakailah hikmat yang telah diberikan Allah kepada Saudara, dan angkatlah pemimpin-pemimpin dan hakim-hakim untuk memerintah semua orang di provinsi Efrat Barat yang hidup menurut Hukum Allah Saudara. Hendaklah Saudara mengajarkan Hukum itu kepada semua orang yang belum mengetahuinya.
Ezr 7:26  Setiap orang yang tidak mentaati hukum-hukum Allah Saudara, atau hukum-hukum kerajaan, harus segera dihukum mati, atau dibuang, atau disita harta bendanya, atau dipenjarakan."
Ezr 7:27  Lalu kata Ezra, "Pujilah TUHAN, Allah yang disembah nenek moyang kita! Ia telah menggerakkan hati raja sehingga ia mau memberi perhatian khusus kepada Rumah TUHAN di Yerusalem.
Ezr 7:28  Dengan pertolongan Allah aku disenangi oleh raja, oleh penasihat-penasihat dan pejabat-pejabatnya yang berkuasa itu; TUHAN Allahku telah memberi semangat kepadaku, sehingga aku berhasil mengajak sebanyak kepala kaum Israel untuk pulang ke Yerusalem."
Ezr 8:1  Inilah daftar para kepala kaum yang telah dibuang ke Babel dan kembali bersama aku, Ezra, ke Yerusalem ketika Artahsasta memerintah sebagai raja:
Ezr 8:2  Gersom, dari kaum Pinehas; Daniel, dari kaum Itamar; Hatus anak Sekhanya, dari kaum Daud; Zakharia, dari kaum Paros, bersama dengan 150 laki-laki dari kaumnya (ada catatan mengenai silsilah mereka); Elyoenai anak Zerahya, dari kaum Pahat-Moab, bersama dengan 200 laki-laki; Sekhanya anak Yahaziel, dari kaum Zatu, bersama dengan 300 laki-laki; Ebed anak Yonatan, dari kaum Adin, bersama dengan 50 laki-laki; Yesaya anak Atalya, dari kaum Elam, bersama dengan 70 laki-laki; Zebaja anak Mikhael, dari kaum Sefaca, bersama dengan 80 laki-laki; Obaja anak Yehiel, dari kaum Yoab, bersama dengan 218 laki-laki; Selomit anak Yosifya, dari kaum Bani, bersama dengan 160 laki-laki; Zakharia anak Bebai, dari kaum Bebai, bersama dengan 28 laki-laki; Yohanan anak Hakatan, dari kaum Azgad, bersama dengan 110 laki-laki; Elifelet, Yehiel dan Semaya, dari kaum Adonikam, bersama dengan 60 laki-laki (mereka datang kemudian); Utai dan Zabud, dari kaum Bigwai, bersama dengan 70 laki-laki.
Ezr 8:15  Seluruh kelompok itu kukumpulkan di pinggir sungai yang mengalir ke kota Ahawa, lalu kami berkemah di sana selama tiga hari. Setelah kuperiksa, ternyata ada beberapa imam di antara mereka, tetapi tak ada seorang pun dari suku Lewi.
Ezr 8:16  Lalu kupanggil sembilan orang di antara para pemimpin, yaitu: Eliezer, Ariel, Semaya, Elnatan, Yarib, Elnatan, Natan, Zakharia, Mesulam dan juga dua guru, yakni Yoyarib dan Elnatan.
Ezr 8:17  Kemudian kuutus mereka menemui Ido, kepala masyarakat di Kasifya, dengan permintaan supaya Ido dan rekan-rekannya para pekerja di Rumah TUHAN, mengirimkan kepada kami orang-orang untuk melayani di Rumah TUHAN.
Ezr 8:18  Karena kebaikan hati Allah, mereka mengirim kepada kami Serebya, seorang Lewi yang cerdas dari kaum Mahli. Bersama-sama dengan dia datang juga anak-anak dan saudara-saudaranya, sejumlah 18 orang.
Ezr 8:19  Hasabya dan Yesaya dari kaum Merari, dengan dua puluh orang sanak saudara mereka, dikirim juga kepada kami.
Ezr 8:20  Di samping itu ada pula 220 orang pekerja Rumah TUHAN yang terdaftar namanya. Leluhur mereka sejak dahulu ditetapkan oleh Raja Daud dan para pembesar kerajaan untuk membantu orang Lewi.
Ezr 8:21  Lalu di pinggir Sungai Ahawa itu aku mengumumkan bahwa semua harus berpuasa dan merendahkan diri di hadapan Allah. Kami berdoa supaya Allah memimpin kami dalam perjalanan itu, serta melindungi kami, anak-anak kami dan segala milik kami.
Ezr 8:22  Aku segan meminta tentara berkuda kepada raja guna melindungi kami terhadap musuh di perjalanan. Sebab aku sudah mengatakan kepadanya, bahwa Allah kami memberkati setiap orang yang berbakti kepada-Nya dan membenci serta menghukum setiap orang yang tidak setia kepada-Nya.
Ezr 8:23  Maka berpuasalah kami serta berdoa kepada Allah meminta perlindungan dan Ia pun mengabulkan doa kami.
Ezr 8:24  Dari antara para imam yang terkemuka aku memilih 12 orang: Serebya, Hasabya, dan 10 orang lagi.
Ezr 8:25  Kemudian kutimbang bagi mereka perak, emas dan perkakas-perkakas yang telah disumbangkan oleh raja, para penasihat serta pembesar dan oleh semua orang Israel, untuk dipakai di dalam Rumah TUHAN. Semuanya itu kuserahkan kepada para imam itu.
Ezr 8:26  Inilah daftar sumbangan-sumbangan itu: perak-22 ton, 100 perkakas perak-70 kilogram, emas-3.400 kilogram, 20 mangkuk emas-8,4 kilogram, 2 mangkuk perunggu yang bagus sekali, yang sama nilainya dengan mangkuk emas.
Ezr 8:28  Aku berkata kepada para imam itu, "Kalian dikhususkan untuk TUHAN, Allah yang disembah nenek moyang kita, begitu pula segala perkakas perak dan emas yang disumbangkan kepada TUHAN adalah khusus untuk TUHAN.
Ezr 8:29  Jagalah benda-benda itu baik-baik sampai kalian tiba di Rumah TUHAN. Sesudah sampai di sana, kalian harus menimbang benda-benda itu, dan menyerahkannya kepada imam-imam kepala dan pemimpin-pemimpin orang Lewi, serta pemimpin-pemimpin orang Israel di Yerusalem."
Ezr 8:30  Lalu para imam dan orang-orang Lewi itu menerima perak, emas dan perkakas-perkakas itu untuk dibawa ke Rumah TUHAN di Yerusalem.
Ezr 8:31  Pada tanggal dua belas bulan pertama kami meninggalkan Sungai Ahawa dan berangkat ke Yerusalem. Dalam perjalanan itu kami dilindungi Allah kami dari serangan dan penghadangan musuh.
Ezr 8:32  Ketika tiba di Yerusalem, kami beristirahat selama 3 hari.
Ezr 8:33  Kemudian pada hari keempat pergilah kami ke Rumah TUHAN, lalu menimbang perak, emas dan perkakas-perkakas itu. Setelah itu kami menyerahkannya kepada Imam Meremot anak Uria dengan disaksikan oleh Eleazar anak Pinehas dan dua orang Lewi, yaitu Yozabad anak Yesua dan Noaja anak Binui.
Ezr 8:34  Semua benda itu dihitung dan ditimbang, lalu dicatat dengan lengkap.
Ezr 8:35  Pada hari itu juga semua orang yang telah kembali dari pembuangan itu, mempersembahkan kurban bakaran kepada Allah Israel. Mereka mempersembahkan 12 ekor sapi jantan untuk seluruh Israel, 96 ekor domba jantan dan 77 ekor anak domba; selain itu juga 12 ekor kambing untuk kurban pengampunan dosa bagi mereka sendiri. Semua binatang itu dibakar sebagai kurban bagi TUHAN.
Ezr 8:36  Lalu dokumen yang diberikan raja kepada mereka, disampaikan kepada para gubernur dan pejabat provinsi Efrat Barat. Maka pejabat-pejabat itu memberi bantuan kepada rakyat supaya mereka dapat tetap beribadat di Rumah TUHAN.
Ezr 9:1  Setelah semua itu dilaksanakan, beberapa pemuka bangsa Israel datang kepadaku. Mereka memberitahukan bahwa rakyat Israel, termasuk para imam, dan orang-orang Lewi tidak memisahkan diri dari bangsa-bangsa yang tinggal di sekitar situ, yaitu bangsa Amon, Moab, Mesir, Kanaan, Het, Feris, Yebus dan Amori. Bahkan para imam dan orang-orang Lewi pun berbuat begitu. Mereka semua melakukan perbuatan-perbuatan keji yang dilakukan oleh bangsa-bangsa itu.
Ezr 9:2  Orang laki-laki Yahudi kawin dengan wanita bangsa asing, sehingga umat khusus milik TUHAN tidak murni lagi. Malahan yang paling dahulu melakukan itu adalah para pemuka dan pejabat Israel.
Ezr 9:3  Mendengar hal itu, aku merasa sangat kesal sehingga merobek pakaianku dan mencabuti rambut serta jenggotku, lalu duduk dengan hati yang hancur luluh.
Ezr 9:4  Aku terus saja duduk begitu sampai waktu persembahan kurban sore. Kemudian datanglah orang-orang mengelilingi aku. Mereka ketakutan mengingat ancaman Allah Israel terhadap dosa orang-orang yang telah kembali dari pembuangan itu.
Ezr 9:5  Ketika kurban sore mulai dipersembahkan, bangkitlah aku dari tempat aku bersedih itu dan dengan pakaian yang robek, aku sujud dan mengulurkan tanganku kepada TUHAN Allahku.
Ezr 9:6  Aku berkata, "Ya Allahku, aku ini malu untuk mengangkat kepalaku di hadapan-Mu. Dosa kami bertumpuk-tumpuk di atas kepala kami sampai menyentuh langit.
Ezr 9:7  Sejak zaman leluhur kami sampai sekarang, kami umat-Mu berdosa kepada-Mu. Itu sebabnya kami, para raja serta para imam kami telah dikalahkan oleh raja-raja asing. Kami dibunuh, dirampok dan diangkut sebagai tawanan. Kami telah dihina habis-habisan, seperti keadaannya pada hari ini.
Ezr 9:8  Tetapi sekarang, ya TUHAN Allah kami, Engkau baru saja bermurah hati kepada kami. Engkau membebaskan beberapa orang di antara kami dari perbudakan untuk hidup dengan sejahtera di tempat yang khusus ini. Engkau memberikan kepada kami hidup baru.
Ezr 9:9  Pada waktu kami masih dalam perbudakan, Engkau tidak meninggalkan kami. Engkau membuat kami disayangi oleh raja-raja Persia dan diizinkan hidup serta membangun kembali Rumah-Mu yang tinggal puing-puing itu. Engkau memberi kami perlindungan di sini, di Yehuda dan Yerusalem.
Ezr 9:10  Tetapi sekarang, ya TUHAN Allah, sesudah segala kejadian itu, apa yang dapat kami katakan? Kami telah mengabaikan perintah-perintah-Mu lagi,
Ezr 9:11  yang Kauberikan kepada kami melalui para nabi, hamba-hamba-Mu. Mereka memberitahu kepada kami bahwa tanah yang hendak kami diami itu tidak bersih karena seluruh penduduknya dari ujung ke ujung berbuat kotor dan keji.
Ezr 9:12  Nabi-nabi itu melarang kami kawin campur dengan orang-orang itu, ataupun membantu mereka menjadi makmur dan sejahtera. Jika kami taat, kami akan menjadi kuat dan menikmati hasil tanah itu dan mewariskannya kepada keturunan kami sampai selama-lamanya.
Ezr 9:13  Kami sudah berdosa dan melanggar hukum-Mu, dan Engkau sudah menghukum kami. Tapi kami tahu, ya Allah kami, bahwa hukuman yang Kauberikan itu tidak seberat yang patut kami terima, malahan kami masih Kauselamatkan.
Ezr 9:14  Jadi, bagaimana mungkin kami mengabaikan perintah-perintah-Mu lagi dan kawin campur dengan orang-orang yang jahat itu? Kalau kami melakukannya, pastilah Engkau akan begitu marah sehingga menghancurkan kami sama sekali dan tidak membiarkan seorang pun hidup.
Ezr 9:15  TUHAN, Allah Israel Engkau adil, meskipun begitu Engkau membiarkan kami hidup. Kami mengakui kesalahan kami kepada-Mu; kami tidak berhak untuk menghadap ke hadirat-Mu."
Ezr 10:1  Sementara Ezra berdoa sambil menangis dan mengaku dosa di depan Rumah TUHAN, banyak sekali orang Israel datang berkumpul, baik laki-laki, maupun wanita dan anak-anak. Mereka mengelilinginya sambil menangis keras-keras.
Ezr 10:2  Kemudian Sekhanya anak Yehiel dari kaum Elam, berkata kepada Ezra, "Kami berdosa kepada TUHAN karena telah mengawini bangsa asing. Meskipun demikian masih ada harapan bagi Israel.
Ezr 10:3  Marilah kita bersumpah kepada Allah, bahwa kita akan mengusir semua wanita itu bersama dengan anak-anak mereka. Kami akan menuruti segala nasihatmu serta nasihat orang-orang lain yang menghormati perintah Allah. Kami akan mentaati Hukum Allah.
Ezr 10:4  Engkaulah yang harus bertindak. Kami akan membantu engkau sepenuhnya. Bertindaklah dengan tegas sampai tuntas."
Ezr 10:5  Maka mulailah Ezra bertindak. Ia menyuruh imam-imam kepala, orang-orang Lewi dan orang-orang Israel lainnya bersumpah bahwa mereka akan melaksanakan usul Sekhanya itu.
Ezr 10:6  Setelah mereka bersumpah, pergilah Ezra dari depan Rumah TUHAN dan masuk ke dalam tempat tinggal Yohanan anak Elyasib dan bermalam di situ. Ia tidak mau makan atau minum karena sedih memikirkan perbuatan orang-orang buangan itu.
Ezr 10:7  Kemudian disiarkan sebuah maklumat ke seluruh Yerusalem dan Yehuda kepada semua orang yang telah kembali dari pembuangan, supaya berkumpul di Yerusalem.
Ezr 10:8  Barangsiapa tidak datang dalam tiga hari, akan disita seluruh hartanya dan ia tidak lagi dianggap anggota masyarakat. Maklumat itu ditandatangani oleh para pemimpin rakyat.
Ezr 10:9  Sesudah tiga hari, pada tanggal dua puluh bulan sembilan, semua orang laki-laki yang tinggal di daerah Yehuda dan Benyamin datang ke Yerusalem dan berkumpul di halaman Rumah TUHAN. Pada waktu itu hujan turun dengan lebatnya. Semua orang menggigil karena dinginnya udara dan juga karena pentingnya pertemuan itu.
Ezr 10:10  Kemudian Ezra berdiri dan berbicara kepada mereka. Ia berkata, "Kalian telah berdosa dan menambah kesalahan Israel karena mengawini wanita bangsa asing.
Ezr 10:11  Sebab itu, akuilah dosamu kepada TUHAN, Allah yang disembah nenek moyang kita, dan lakukanlah apa yang menyenangkan hati-Nya. Jauhilah orang asing yang tinggal di negeri kita dan usirlah istri-istrimu dari bangsa asing itu."
Ezr 10:12  Orang-orang itu menjawab dengan nyaring, "Kami akan menuruti nasihatmu."
Ezr 10:13  Lalu kata mereka lagi, "Tetapi yang hadir di sini banyak sekali dan hujan begitu lebat. Kami tidak tahan berdiri di luar begini. Lagipula ini bukan perkara yang dapat diselesaikan dalam satu dua hari, sebab banyak sekali dari kami yang terlibat dalam dosa ini.
Ezr 10:14  Izinkanlah para pemimpin kami tinggal di Yerusalem dan mengurus soal itu. Lalu setiap orang di kota-kota kami, yang mempunyai istri dari bangsa asing, harus datang pada waktu yang ditentukan, bersama dengan para pemimpin dan para hakim kotanya masing-masing. Dengan demikian Allah tidak akan marah lagi kepada kita karena perkara itu."
Ezr 10:15  Tidak seorang pun mengajukan keberatan terhadap rencana itu, kecuali Yonatan anak Asael dan Yahzeya anak Tikwa, disokong oleh dua orang Lewi, yaitu Mesulam dan Sabetai.
Ezr 10:16  Orang-orang buangan yang telah kembali itu tetap hendak menjalankan rencana itu, sebab itu Imam Ezra memilih beberapa orang di antara kepala-kepala kaum, lalu mencatat nama mereka. Pada tanggal satu bulan sepuluh mereka mulai bersidang untuk menyelidiki perkara itu.
Ezr 10:17  Dalam tiga bulan berikutnya berhasillah mereka menyelesaikan pemeriksaan terhadap semua pria yang telah mengawini wanita dari bangsa asing.
Ezr 10:18  Inilah daftar dari orang-orang yang mempunyai istri dari bangsa asing: Para Imam, terdaftar dengan kaumnya: Kaum Yesua dengan saudara-saudaranya, anak-anak Yozadak: Maaseya, Eliezer, Yarib dan Gedalya.
Ezr 10:19  Mereka berjanji akan menceraikan istri mereka, lalu mereka mempersembahkan domba jantan untuk kurban pengampunan dosa mereka.
Ezr 10:20  Kaum Imer: Hanani dan Zebaja.
Ezr 10:21  Kaum Harim: Maaseya, Elia, Semaya, Yehiel dan Uzia.
Ezr 10:22  Kaum Pasyhur: Elyoenai, Maaseya, Ismael, Netaneel, Yozabad, dan Elasa.
Ezr 10:23  Orang-orang Lewi: Yozabad, Simei, Kelaya (juga disebut Kelita), Petahya, Yuda dan Eliezer.
Ezr 10:24  Para penyanyi: Elyasib. Para penjaga pintu gerbang Rumah TUHAN: Salum, Telem dan Uri.
Ezr 10:25  Orang-orang Israel yang lain: Kaum Paros: Ramya, Yezia, Malkia, Miyamin, Eleazar, Malkia dan Benaya.
Ezr 10:26  Kaum Elam: Matanya, Zakharia, Yehiel, Abdi, Yeremot dan Elia.
Ezr 10:27  Kaum Zatu: Elyoenai, Elyasib, Matanya, Yeremot, Zabad dan Aziza.
Ezr 10:28  Kaum Bebai: Yohanan, Hananya, Zabai dan Atlai.
Ezr 10:29  Kaum Bani: Mesulam, Malukh, Adaya, Yasub, Seal dan Yeramot.
Ezr 10:30  Kaum Pahat-Moab: Adna, Kelal, Benaya, Maaseya, Matanya, Bezaleel, Binui dan Manasye.
Ezr 10:31  Kaum Harim: Eliezer, Yisia, Malkia, Semaya, Simeon, Benyamin, Malukh dan Semarya.
Ezr 10:33  Kaum Hasum: Matnai, Matata, Zabad, Elifelet, Yeremai, Manasye dan Simei.
Ezr 10:34  Kaum Bani: Maadai, Amram, Uel, Benaya, Bedeya, Keluhu, Wanya, Meremot, Elyasib, Matanya, Matnai, Yaasai,
Ezr 10:38  Kaum Binui: Simei, Selemya, Natan, Adaya, Makhnadbai, Sasai, Sarai, Azareel, Selemya, Semarya, Salum, Amarya dan Yusuf.
Ezr 10:43  Kaum Nebo: Yeiel, Matica, Zabad, Zebina, Yadai, Yoel dan Benaya.
Ezr 10:44  Semua orang itu telah mengawini wanita dari bangsa asing. Maka istri-istri itu diceraikan dan diusir bersama dengan anak-anak mereka.


\end{document}