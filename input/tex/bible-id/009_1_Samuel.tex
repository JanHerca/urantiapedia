\begin{document}

\title{1 Samuel}


\chapter{1}

\par 1 Adalah seorang laki-laki, namanya Elkana, dari suku Efraim. Ia tinggal di kota Rama, di daerah pegunungan Efraim. Elkana adalah anak Yeroham dan cucu Elihu. Ia tergolong keluarga Tohu, cabang dari marga Zuf.
\par 2 Elkana mempunyai dua istri, Hana dan Penina. Penina mempunyai anak, tetapi Hana tidak.
\par 3 Setiap tahun Elkana pergi dari Rama untuk beribadat di Silo dan mempersembahkan kurban kepada TUHAN Yang Mahakuasa. Yang menjadi imam TUHAN di Silo ialah Hofni dan Pinehas, anak-anak Eli.
\par 4 Setiap kali jika Elkana mempersembahkan kurban, Penina dan semua anaknya masing-masing diberinya sebagian dari daging kurban itu.
\par 5 Tetapi Hana diberinya sebagian saja sebab TUHAN tidak memberi anak kepadanya. Meskipun begitu Elkana sangat mencintai Hana.
\par 6 Hana selalu disakiti hatinya dan dihina oleh Penina, madunya itu, karena TUHAN tidak memberi anak kepadanya.
\par 7 Hal itu terjadi dari tahun ke tahun; kalau mereka pergi ke Rumah TUHAN, selalu Hana disakiti hatinya oleh Penina. Seringkali Hana menangis dan tidak mau makan karena ia dihina.
\par 8 Lalu setiap kali pula Elkana, suaminya, bertanya, "Mengapa kau menangis Hana? Mengapa kau tak mau makan dan terus sedih saja? Bukankah aku lebih berharga bagimu daripada sepuluh anak laki-laki?"
\par 9 Pada suatu hari sesudah mereka makan di Rumah TUHAN di Silo, Hana bangkit dari meja makan. Saat itu Imam Eli yang juga ada di Rumah TUHAN, sedang duduk di kursinya dekat pintu. Dengan sangat sedih Hana berdoa kepada TUHAN sambil menangis tersedu-sedu.
\par 10 [1:9]
\par 11 Kemudian Hana mengucapkan janji, katanya, "TUHAN Yang Mahakuasa, perhatikanlah hamba-Mu ini! Lihatlah sengsara hamba. Ingatlah kepada hamba dan jangan lupakan hamba! Jika Engkau memberikan kepada hamba seorang anak laki-laki, hamba berjanji akan memberikan dia kepada-Mu seumur hidupnya. Hamba berjanji juga bahwa rambutnya tidak pernah akan dipotong."
\par 12 Lama sekali Hana berdoa, dan Eli memperhatikan mulut wanita itu.
\par 13 Hana berdoa dalam hati, jadi hanya bibirnya yang komat-kamit, tetapi suaranya tidak terdengar. Sebab itu ia disangka mabuk oleh Eli.
\par 14 Maka Eli pun berkata kepadanya, "Masakan kau mabuk di sini! Jangan minum anggur lagi!"
\par 15 Tetapi Hana menjawab, "Aku tidak mabuk, Pak, aku sama sekali tidak minum anggur! Aku putus asa, dan sedang berdoa menceritakan segala penderitaanku kepada TUHAN.
\par 16 Janganlah aku ini dianggap perempuan jalang. Aku berdoa seperti ini karena aku sangat sedih."
\par 17 Lalu kata Eli, "Kalau begitu pulanglah dengan selamat. Semoga Allah Israel mengabulkan permintaanmu."
\par 18 Jawab Hana, "Semoga aku mendapat restu Bapak." Kemudian ia pergi, lalu makan; dan ia tidak sedih lagi.
\par 19 Keesokan harinya Elkana dan keluarganya bangun pagi-pagi, dan setelah beribadat kepada TUHAN, pulanglah mereka ke Rama. Maka Elkana bersetubuh dengan Hana, istrinya itu, dan TUHAN mengabulkan doa Hana.
\par 20 Wanita itu hamil dan melahirkan anak laki-laki. Ia menamakannya Samuel, katanya, "Aku telah memintanya dari TUHAN."
\par 21 Tibalah waktunya Elkana dan keluarganya pergi ke Silo lagi untuk mempersembahkan kurban tahunan kepada TUHAN. Selain itu Elkana telah berjanji untuk mempersembahkan kurban yang khusus.
\par 22 Tetapi kali ini Hana tidak ikut. Katanya kepada suaminya, "Nanti setelah disapih, Samuel akan kuantarkan ke Rumah TUHAN, dan ia akan tinggal di sana seumur hidupnya."
\par 23 Jawab Elkana, "Baiklah, lakukanlah apa yang kaupandang baik; tinggallah di rumah sampai tiba waktunya engkau menyapih anak kita. Dan semoga TUHAN membuat janjimu menjadi kenyataan." Jadi, Hana tinggal di rumah sampai tiba waktunya anaknya disapih.
\par 24 Sesudah Samuel disapih, ia diantarkan ibunya ke Rumah TUHAN di Silo. Pada waktu itu ia masih sangat kecil. Hana membawa pula seekor sapi jantan yang berumur tiga tahun, gandum sepuluh kilogram, dan sebuah kantong kulit yang penuh berisi anggur.
\par 25 Sapi itu dipotong, lalu Samuel diantarkan kepada Eli.
\par 26 Hana berkata kepada Eli, "Maaf, Pak. Masih ingatkah Bapak kepadaku? Aku ini wanita yang pernah Bapak lihat berdiri di sini, sedang berdoa kepada TUHAN.
\par 27 Anak inilah yang kuminta dari TUHAN. Doaku telah dikabulkan,
\par 28 dan karena itu anak ini kuserahkan untuk menjadi milik TUHAN seumur hidupnya." Setelah itu mereka beribadat kepada TUHAN.

\chapter{2}

\par 1 Lalu Hana berdoa, katanya, "TUHAN menggembirakan hatiku; perbuatan-Nya menyenangkan jiwaku. Dengan rasa bahagia kuejek musuhku, sebab sungguh-sungguh Allah sudah menolongku.
\par 2 Tak ada yang suci seperti TUHAN! Hanya Dia yang dapat memberi perlindungan.
\par 3 Hentikanlah bualmu yang kosong! Akhirilah omongan yang sombong! Sebab TUHAN itu Allah yang mahapaham; yang menghakimi segala perbuatan orang.
\par 4 Busur pahlawan patahlah sudah, tetapi makin kuatlah orang lemah.
\par 5 Orang yang dahulu kenyang karena berlimpah pangan, kini menjadi buruh upahan hanya untuk mendapat makanan. Orang yang dahulu kelaparan, sekarang puas, cukup makanan. Istri mandul, kini bangga, tujuh anak dilahirkannya sudah! Tetapi ibu yang banyak anaknya, ditinggalkan, dibiarkan merana.
\par 6 TUHAN mematikan, Ia pun menghidupkan. Ke dunia orang mati diturunkan-Nya manusia. Dan Dia pula yang mengangkat dari sana.
\par 7 Ada orang yang dibuat-Nya papa, ada juga yang dibuat-Nya kaya. Ada orang yang direndahkan-Nya, ada juga yang ditinggikan-Nya.
\par 8 Dari derita diangkat-Nya orang tak berharta, dari sengsara dibebaskan-Nya orang yang hina. Mereka dijadikan teman para bangsawan, dan diberikan jabatan kehormatan. Alas bumi Tuhanlah yang punya, di atasnya dibangun-Nya dunia.
\par 9 Orang setia selalu aman dalam perlindungan TUHAN. Tapi yang jahat hancur binasa dalam kelam dan gelap gulita. Tenaga sendiri tak dapat diandalkan, kekuatan manusia tidak memberi kemenangan.
\par 10 Musuh-musuh TUHAN hancur lebur, mereka takut; dari langit Dia mengguntur. TUHAN akan menghakimi seluruh dunia ini, dan memberikan kuasa yang mulia kepada raja yang dipilih-Nya."
\par 11 Lalu pulanglah Elkana dan keluarganya ke Rama, tetapi Samuel, anak itu, tinggal di Silo dan melayani TUHAN di bawah pengawasan Imam Eli.
\par 12 Anak-anak Eli jahat sekali. Mereka tidak mengindahkan TUHAN
\par 13 dan sebagai imam mereka bertindak sewenang-wenang terhadap bangsa Israel. Jika misalnya ada orang yang mempersembahkan kurban, pembantu imam datang membawa garpu bergigi tiga. Sementara daging itu direbus,
\par 14 garpu itu ditusukkan ke dalam panci tempat daging itu dimasak, lalu apa saja yang ditarik ke luar dengan garpu itu menjadi milik imam. Semua orang Israel yang datang ke Silo untuk mempersembahkan kurban diperlakukan seperti itu.
\par 15 Bahkan kadang-kadang pembantu itu datang sebelum lemaknya dipisahkan untuk dibakar, lalu ia berkata kepada orang yang mempersembahkan kurban itu, "Berikanlah daging yang masih mentah itu kepada imam supaya dipanggangnya; ia tidak mau menerima daging rebus."
\par 16 Jika orang itu menjawab, "Sebaiknya kita mentaati peraturan, dan membakar lemak itu dahulu; setelah itu boleh kauambil sesukamu," maka pembantu itu berkata, "Tidak, sekarang juga harus kauberikan, jika tidak, akan kuambil dengan paksa."
\par 17 Demikianlah kedua anak Eli itu berdosa besar di mata TUHAN, karena mereka meremehkan persembahan kurban untuk TUHAN.
\par 18 Sementara itu, Samuel yang masih anak-anak itu terus melayani TUHAN. Ia memakai baju khusus dari kain linen, seperti baju ibadat para imam.
\par 19 Setiap tahun ibunya membuat jubah kecil lalu diberikannya kepada Samuel kalau wanita itu bersama-sama dengan suaminya datang mempersembahkan kurban tahunan.
\par 20 Maka Eli memberkati Elkana dan istrinya, dan ia berkata kepada Elkana, "Semoga TUHAN memberikan anak-anak lagi kepadamu dari istrimu ini, untuk menggantikan anak yang telah kamu serahkan kepada TUHAN." Sesudah itu pulanglah mereka ke Rama.
\par 21 Hana diberkati Allah, sehingga melahirkan lagi tiga anak laki-laki dan dua anak perempuan. Samuel pun bertambah besar sementara ia melayani Allah.
\par 22 Eli sudah sangat tua. Ia terus-menerus mendengar pengaduan mengenai kelakuan anak-anaknya terhadap orang Israel. Eli tahu juga bahwa anak-anaknya itu tidur dengan wanita-wanita yang bertugas di depan pintu Kemah Kehadiran TUHAN.
\par 23 Sebab itu berkatalah ia kepada anak-anaknya, "Aku telah menerima laporan dari seluruh umat mengenai tingkah lakumu yang jahat. Mengapa kamu lakukan itu?
\par 24 Jangan berbuat begitu lagi, anakku. Sungguh jelek sekali apa yang dibicarakan umat TUHAN tentang kamu!
\par 25 Jika orang bersalah terhadap manusia, Allah dapat membelanya; tetapi jika orang berdosa kepada TUHAN, siapakah dapat menolongnya?" Tetapi mereka tidak memperdulikan nasihat Eli ayah mereka, sebab TUHAN telah memutuskan untuk membunuh mereka.
\par 26 Sebaliknya Samuel, anak itu, semakin besar dan semakin disukai, baik oleh TUHAN, maupun oleh semua orang.
\par 27 Pada suatu hari datanglah seorang nabi kepada Eli dan menyampaikan pesan dari TUHAN kepadanya, kata-Nya, "Ketika leluhurmu Harun dan keluarganya menjadi hamba raja Mesir, Aku telah menyatakan diri-Ku kepada Harun.
\par 28 Dari segala suku Israel telah Kupilih keluarganya menjadi imam-Ku, untuk melayani di mezbah, membakar dupa, dan memakai baju efod jika berbicara dengan-Ku. Dan kepada mereka serta keturunan mereka telah Kuberi hak untuk mengambil sebagian dari kurban persembahan yang dibakar di mezbah.
\par 29 Jadi, mengapa engkau masih serakah melihat kurban-kurban yang sesuai dengan perintah-Ku, dipersembahkan bangsa-Ku kepada-Ku? Mengapa engkau Eli, lebih menghormati anak-anakmu daripada menghormati Aku, dan membiarkan mereka menggemukkan dirinya dengan bagian yang terbaik dari setiap persembahan bangsa-Ku kepada-Ku?
\par 30 Aku, TUHAN Allah Israel, dahulu telah berjanji, bahwa keluargamu dan margamu akan melayani Aku sebagai imam untuk selamanya. Tetapi sekarang Aku tidak menghendaki hal itu lagi! Sebab yang menghormati Aku, akan Kuhormati tetapi yang menghina Aku, akan Kuhina.
\par 31 Dengarlah, akan datang masanya Aku membunuh semua pemuda dalam keluarga dan margamu, sehingga tak seorang pria pun dalam keluargamu akan mencapai usia lanjut.
\par 32 Engkau akan sedih dan merasa iri melihat segala berkat yang akan diberikan kepada orang-orang lain di Israel, sedangkan dalam keluargamu sendiri semua akan mati muda.
\par 33 Tetapi, seorang dari keturunanmu, akan Kubiarkan hidup dan melayani Aku sebagai imam. Namun ia akan menjadi buta dan hidup dengan putus asa. Semua keturunanmu yang lain akan terbunuh.
\par 34 Sebagai bukti bahwa segala yang Kukatakan itu betul akan terjadi, maka kedua anakmu itu, Hofni dan Pinehas akan mati dalam sehari.
\par 35 Kemudian Aku akan memilih imam yang setia kepada-Ku dan yang melakukan apa yang Kuperintahkan kepadanya. Kepadanya akan Kuberikan keturunan yang selalu akan bertugas sebagai imam di hadapan raja-raja yang Kupilih.
\par 36 Setiap keturunanmu yang masih hidup, akan pergi kepada imam itu untuk meminta uang dan makanan. Ia akan memohon, izinkanlah aku membantu para imam, supaya aku dapat makan biar hanya sesuap saja."

\chapter{3}

\par 1 Sementara itu, Samuel yang masih anak-anak itu melayani TUHAN di bawah pengawasan Eli. Pada masa itu jarang sekali ada pesan dan penglihatan dari TUHAN.
\par 2 Pada suatu malam, Eli yang sudah sangat tua dan hampir buta itu, sedang tidur di kamarnya,
\par 3 dan Samuel tidur di dekat Peti Perjanjian TUHAN, di dalam Kemah Suci. Ketika fajar belum menyingsing dan lampu masih menyala,
\par 4 TUHAN memanggil Samuel, lalu Samuel menjawab, "Ya, Pak!"
\par 5 Kemudian ia lari menemui Eli dan berkata, "Ada apa Pak? Mengapa aku dipanggil?" Tetapi Eli menjawab, "Aku tidak memanggilmu, tidurlah kembali." Lalu Samuel pergi tidur lagi.
\par 6 Kemudian TUHAN memanggil Samuel lagi, tetapi Samuel tidak tahu bahwa yang memanggilnya itu TUHAN, karena belum pernah TUHAN berbicara dengan dia. Sebab itu ia bangun, lalu menjumpai Eli dan bertanya, "Ada apa Pak? Mengapa Bapak memanggil?" Eli menjawab, "Aku tidak memanggilmu, anakku; tidurlah kembali."
\par 7 [3:6]
\par 8 Untuk ketiga kalinya TUHAN memanggil Samuel; lalu bangunlah Samuel dan menemui Eli serta berkata, "Ada apa Pak? Mengapa Bapak memanggil?" Pada saat itulah Eli menyadari bahwa Tuhanlah yang memanggil anak itu.
\par 9 Sebab itu ia berkata kepada Samuel, "Tidurlah kembali, dan jika engkau dipanggil lagi katakanlah, 'Bicaralah, TUHAN, hamba-Mu mendengarkan.'" Lalu pergilah Samuel ke tempat tidurnya lagi.
\par 10 Kemudian datanglah TUHAN, dan berdiri di tempat itu serta memanggil lagi, "Samuel, Samuel!" Dan Samuel menjawab, "Bicaralah, TUHAN, hamba-Mu mendengarkan."
\par 11 Lalu berkatalah TUHAN kepadanya, "Aku akan melakukan sesuatu yang dahsyat terhadap orang Israel, sehingga setiap orang yang mendengarnya akan kebingungan.
\par 12 Pada hari itu Aku akan melakukan segala yang telah Kuancamkan terhadap keluarga Eli dari mula sampai akhir.
\par 13 Telah Kuberitahu kepadanya bahwa keluarganya akan Kuhukum untuk selama-lamanya karena anak-anaknya telah menghina Aku. Eli mengetahui dosa-dosa mereka itu, tetapi mereka tidak dimarahinya.
\par 14 Karena itu Aku bersumpah mengenai keluarga Eli, bahwa dosa mereka yang sangat besar itu tidak akan dapat dihapuskan oleh kurban atau persembahan apapun juga untuk selama-lamanya."
\par 15 Setelah itu Samuel tidur lagi, dan pada pagi harinya ia bangun dan membuka pintu-pintu Rumah TUHAN. Ia takut memberitahukan penglihatan itu kepada Eli,
\par 16 tetapi Eli memanggil dia, katanya, "Samuel, anakku." Jawab Samuel, "Ya, Pak."
\par 17 Eli bertanya, "Apakah yang diberitahukan TUHAN kepadamu? Ceritakanlah semuanya kepadaku. Kalau ada sesuatu yang engkau rahasiakan, pasti engkau dihukum Allah."
\par 18 Lalu Samuel memberitahukan semuanya itu kepada Eli. Tak ada satu pun yang dirahasiakannya. Kemudian berkatalah Eli, "Dia TUHAN, biar Ia melakukan apa yang dianggap-Nya baik."
\par 19 Samuel bertambah besar. TUHAN menyertai dia dan membuat semua yang dikatakan Samuel sungguh terjadi.
\par 20 Itu sebabnya sekalian umat Israel dari seluruh negeri itu mengakui bahwa Samuel adalah nabi TUHAN. Setiap kali Samuel berbicara, semua bangsa Israel mendengarkan. TUHAN masih terus menyatakan diri di Silo, di mana Ia untuk pertama kalinya menampakkan diri kepada Samuel dan berbicara dengan dia.

\chapter{4}

\par 1 Pada masa itu orang Filistin mengerahkan segenap bala tentaranya untuk menggempur Israel; sebab itu tentara Israel maju ke medan peperangan; mereka berkemah dekat Eben-Haezer, sedang tentara Filistin berkemah di Afek.
\par 2 Tentara Filistin mulai menyerang, dan setelah pertempuran berlangsung dengan sengit, tentara Israel kalah dan kira-kira empat ribu orang tewas di medan pertempuran itu.
\par 3 Ketika sisa tentara yang kalah sudah kembali ke perkemahan, para pemimpin Israel berkata, "Mengapa gerangan TUHAN membiarkan kita dikalahkan orang Filistin pada hari ini? Mari kita ambil Peti Perjanjian TUHAN dari Silo, dan kita bawa ke mari, supaya TUHAN mau menolong kita dan menyelamatkan kita dari musuh."
\par 4 Lalu mereka mengutus orang ke Silo untuk mengambil Peti Perjanjian TUHAN yang merupakan takhta TUHAN Yang Mahakuasa. Kedua anak Eli, Hofni dan Pinehas, mengiringi Peti Perjanjian itu.
\par 5 Setelah Peti Perjanjian itu sampai di perkemahan, bersoraklah orang Israel dengan begitu nyaring, sehingga bumi bergetar.
\par 6 Orang Filistin juga mendengar bunyi sorak itu lalu mereka berkata, "Mengapa orang-orang Ibrani itu bersorak-sorak?" Tetapi ketika mereka mengetahui bahwa Peti Perjanjian TUHAN telah tiba di perkemahan Ibrani,
\par 7 mereka menjadi takut, dan berkata, "Ada dewa datang ke perkemahan mereka! Celakalah kita! Belum pernah kita mengalami hal seperti ini!
\par 8 Celakalah betul! Siapakah dapat menyelamatkan kita dari dewa-dewa yang kuat itu? Bukankah dewa-dewa itu pula yang telah membunuh orang Mesir di padang pasir dahulu?
\par 9 Beranikanlah hatimu, hai orang Filistin! Bertempurlah seperti laki-laki, jika tidak, kamu akan menjadi hamba orang-orang Ibrani itu, seperti mereka dahulu menjadi hamba kamu. Sebab itu bertempurlah dengan gagah berani!"
\par 10 Kemudian berperanglah orang Filistin dengan sengit dan mengalahkan tentara Israel. Orang Israel lari ke kemahnya masing-masing. Sungguh besar kekalahan itu: 30.000 prajurit Israel telah gugur.
\par 11 Peti Perjanjian Allah direbut musuh Israel, dan kedua anak Eli, Hofni dan Pinehas, tewas.
\par 12 Seorang dari suku Benyamin lari dari medan pertempuran ke Silo, dan sampai di situ pada hari itu juga. Sebagai tanda dukacita, telah dikoyak-koyaknya pakaiannya dan ditaruhnya tanah di kepalanya.
\par 13 Eli, yang sangat cemas memikirkan keselamatan Peti Perjanjian TUHAN, sedang duduk di kursi di tepi jalan, sambil termenung. Ketika orang itu mengabarkan berita kekalahan Israel, seluruh penduduk kota meratap dengan nyaring.
\par 14 Eli mendengar ratapan itu, lalu bertanya, "Keributan apakah itu?" Sementara itu orang tadi berlari kepada Eli; ketika sampai, ia memberitahukan kabar buruk itu.
\par 15 Adapun Eli sudah sembilan puluh delapan tahun umurnya dan ia hampir buta.
\par 16 Orang itu berkata kepada Eli, "Aku lari hari ini dari medan pertempuran dan baru saja sampai." Eli bertanya kepadanya, "Bagaimana kabarnya, anakku?"
\par 17 Pembawa kabar itu menjawab, "Israel lari dari orang Filistin. Kita menderita kekalahan yang besar sekali. Banyak orang yang tewas, juga kedua anak Bapak, Hofni dan Pinehas. Dan Peti Perjanjian Allah direbut musuh!"
\par 18 Ketika orang itu memberitakan tentang Peti Perjanjian itu, Eli jatuh terlentang dari kursinya di sebelah pintu gerbang. Ia begitu tua dan gemuk sehingga waktu jatuh lehernya patah, dan ia tewas. Ia telah memimpin Israel empat puluh tahun lamanya.
\par 19 Pada waktu itu, menantu Eli, yaitu istri Pinehas, sedang hamil tua, dan saatnya melahirkan sudah dekat. Ketika didengarnya bahwa Peti Perjanjian Allah telah direbut, dan bahwa mertuanya serta suaminya telah meninggal, tiba-tiba ia merasakan sakit beranak, lalu bersalinlah ia tak lama kemudian.
\par 20 Ketika ia sudah hampir meninggal, para wanita yang menolongnya berkata, "Tabahlah! Anakmu laki-laki!" Namun ia tidak menjawab atau memperhatikan mereka.
\par 21 Anak itu dinamainya Ikabod, katanya, "Kehadiran Allah Yang Mulia telah hilang dari Israel." --yang dimaksudkannya ialah bahwa Peti Perjanjian sudah dirampas dan mertuanya serta suaminya telah meninggal. Sebab itu ia berkata, "Kehadiran Allah Yang Mulia telah hilang dari Israel, karena Peti Perjanjian Allah sudah dirampas."

\chapter{5}

\par 1 Setelah orang Filistin merebut Peti Perjanjian Allah, mereka membawanya dari Eben-Haezer ke kota mereka, Asdod.
\par 2 Di situ peti itu dibawa masuk ke kuil dewa mereka, Dagon, dan diletakkan di samping patung dewa itu.
\par 3 Ketika penduduk Asdod besoknya pagi-pagi datang ke kuil itu, mereka melihat bahwa patung Dagon telah jatuh tertelungkup di tanah, di depan Peti Perjanjian TUHAN! Mereka mengangkat patung itu dan mengembalikannya ke tempatnya.
\par 4 Tetapi keesokan harinya, pagi-pagi, mereka melihat bahwa patung itu sudah jatuh lagi di depan Peti Perjanjian itu. Kali ini kepala patung itu dan kedua lengannya terpenggal dan terletak di ambang pintu; hanya badan patung itu yang masih utuh.
\par 5 (Itulah sebabnya, sampai hari ini para imam dewa Dagon dan semua penyembahnya di Asdod melangkahi ambang pintu kuil Dagon itu dan tidak menginjaknya.)
\par 6 Kemudian TUHAN menghukum penduduk Asdod dan daerah sekitarnya dengan benjol-benjol pada tubuh mereka.
\par 7 Ketika mereka melihat apa yang sedang terjadi pada mereka itu, mereka berkata, "Allah Israel menghukum kita dan dewa kita Dagon. Kita tidak boleh membiarkan Peti Perjanjian-Nya tinggal di sini lebih lama lagi."
\par 8 Sebab itu mereka mengundang kelima raja Filistin supaya berkumpul, lalu bertanya, "Peti Perjanjian Allah Israel itu harus kita apakan?" Jawab raja-raja itu, "Pindahkanlah ke kota Gat." Jadi, peti itu dipindahkan ke kota Gat, sebuah kota Filistin yang lain.
\par 9 Tetapi setelah peti itu sampai di situ, TUHAN menghukum kota itu pula dan menimbulkan kegemparan yang sangat besar. Penduduk kota itu, baik yang muda maupun yang tua dihukum dengan benjol-benjol yang tumbuh pada tubuh mereka.
\par 10 Lalu Peti Perjanjian itu diantarkan ke Ekron, kota Filistin yang lain lagi. Setibanya peti itu di situ, penduduknya berteriak, katanya, "Peti Perjanjian Allah Israel itu dibawa ke mari untuk membunuh kita semua."
\par 11 Sebab itu mereka memanggil semua raja Filistin supaya berkumpul, lalu berkata, "Kembalikanlah Peti Perjanjian Allah Israel itu ke tempatnya semula supaya jangan dibunuhnya kita dan keluarga kita." Di seluruh kota terjadi kegemparan karena Allah menghukum penduduknya dengan sangat berat.
\par 12 Orang-orang yang tidak mati, ditimpa penyakit benjol-benjol itu, sehingga penduduk kota itu berteriak kepada dewa-dewa mereka, meminta tolong.

\chapter{6}

\par 1 Setelah Peti Perjanjian TUHAN berada di Filistin tujuh bulan lamanya,
\par 2 orang Filistin memanggil para imam dan para tukang sihir lalu bertanya, "Peti Perjanjian TUHAN itu harus kita apakan? Jika kita kembalikan ke tempatnya, bagaimana caranya?"
\par 3 Lalu jawab mereka, "Peti Perjanjian Allah Israel itu jangan kamu kembalikan begitu saja, tetapi harus disertai hadiah untuk tebusan dosamu. Dengan demikian kamu akan disembuhkan, dan kamu akan tahu mengapa Allah Israel itu terus-menerus menghukum kamu."
\par 4 "Tebusan apa yang harus kita berikan kepada-Nya?" tanya orang-orang itu. Jawab para imam itu, "Lima benjolan emas dan lima tikus emas, sesuai dengan jumlah raja-raja Filistin. Bukankah bencana berupa benjol-benjol dan tikus juga yang telah menimpa kamu semua, baik rakyat maupun raja?
\par 5 Jadi, kamu harus membuat tiruan dari benjol-benjol dan dari tikus yang sedang mengganas di negerimu. Dengan demikian kamu memberi penghormatan kepada Allah Israel. Mungkin Dia akan berhenti menghukum kamu, dan dewa-dewa serta tanahmu.
\par 6 Apa gunanya kamu berkeras kepala seperti raja dan orang Mesir dahulu? Jangan lupa bagaimana Allah mempermain-mainkan mereka, sampai mereka jera dan orang Israel itu mereka biarkan meninggalkan Mesir.
\par 7 Sebab itu, siapkanlah sebuah pedati yang baru dengan dua ekor sapi yang sedang menyusui, dan yang belum pernah dipakai untuk menarik pedati. Pasanglah kedua sapi itu pada pedati, dan giringlah anak-anak sapi itu kembali ke dalam kandang.
\par 8 Kemudian, ambillah Peti Perjanjian TUHAN, naikkanlah ke atas pedati, dan letakkan di sebelahnya kotak yang berisi benda-benda emas yang harus kamu persembahkan kepada-Nya sebagai tebusan dosamu. Biarlah pedati dan sapi itu berjalan dengan sendirinya.
\par 9 Perhatikanlah jalannya; jika Peti Perjanjian itu dibawa ke arah negerinya sendiri, yaitu kota Bet-Semes, itu berarti bahwa Allah orang Israellah yang telah mendatangkan celaka yang hebat itu kepada kita. Tetapi jika pedati itu tidak menuju ke sana, kita akan tahu bahwa musibah itu bukan dari Allah Israel, melainkan kebetulan saja."
\par 10 Orang-orang itu menurut; mereka mengambil dua ekor sapi yang sedang menyusui, lalu dipasang pada pedati, sedangkan anak-anak sapi itu dikurung di dalam kandang.
\par 11 Lalu Peti Perjanjian itu dinaikkan ke atas pedati, juga kotak yang berisi tikus emas dan tiruan benjol-benjol itu.
\par 12 Sapi-sapi itu berjalan langsung ke arah Bet-Semes; binatang itu melenguh sambil berjalan terus, tanpa menyimpang ke kiri atau ke kanan dari jalan raya. Kelima raja Filistin itu terus mengikutinya sampai ke perbatasan Bet-Semes.
\par 13 Pada waktu itu penduduk Bet-Semes sedang panen gandum di lembah. Ketika mereka melihat Peti Perjanjian itu, dengan gembira mereka menyongsongnya.
\par 14 Pedati itu berhenti di dekat sebuah batu besar di ladang Yosua, seorang penduduk Bet-Semes. Penduduk kota itu membelah-belah kayu pedati itu lalu memotong kedua ekor sapi itu untuk kurban bakaran kepada TUHAN.
\par 15 Kemudian orang-orang Lewi menurunkan Peti Perjanjian TUHAN dan kotak yang berisi benda-benda emas itu, lalu meletakkannya di atas batu besar itu. Sesudah itu penduduk Bet-Semes mempersembahkan kurban bakaran dan kurban-kurban lainnya kepada TUHAN.
\par 16 Ketika kelima raja Filistin itu melihat semuanya itu pulanglah mereka ke Ekron pada hari itu juga.
\par 17 Dengan demikian orang Filistin sudah mengirim kepada TUHAN persembahan penebus dosa untuk penduduk lima kota yang diperintah oleh lima raja Filistin. Kota-kota itu ialah: Asdod, Gaza, Askelon, Gat dan Ekron. Di antaranya ada yang berbenteng, ada yang tidak. Setiap kota itu mengirim satu benjolan emas dan satu tikus emas. Bukti dari kejadian itu ialah batu besar tempat mereka meletakkan Peti Perjanjian TUHAN. Sampai hari ini pun batu itu masih ada di ladang Yosua, orang Bet-Semes itu.
\par 18 [6:17]
\par 19 Tetapi karena orang-orang Bet-Semes itu menjenguk ke dalam Peti Perjanjian TUHAN, tujuh puluh orang di antara mereka dibunuh oleh TUHAN. Lalu rakyat berkabung sebab TUHAN menimpa mereka dengan musibah yang begitu besar.
\par 20 Lalu penduduk Bet-Semes berkata, "Siapakah yang bisa tahan di hadapan TUHAN, Allah Yang Suci itu? Ke manakah peti itu harus kita antarkan supaya jauh dari kita?"
\par 21 Lalu mereka mengirim utusan kepada penduduk Kiryat-Yearim dengan pesan sebagai berikut, "Orang Filistin telah mengembalikan Peti Perjanjian TUHAN. Datanglah dan bawalah peti itu ke tempatmu."

\chapter{7}

\par 1 Maka penduduk Kiryat-Yearim mengambil Peti Perjanjian TUHAN itu lalu membawanya ke dalam rumah orang yang bernama Abinadab; rumah itu terletak di atas bukit. Kemudian mereka mentahbiskan Eleazar anak Abinadab, untuk menjaga Peti Perjanjian itu.
\par 2 Lama juga Peti Perjanjian TUHAN itu tinggal di Kiryat-Yearim, kira-kira dua puluh tahun. Selama waktu itu seluruh umat Israel berseru kepada TUHAN meminta tolong.
\par 3 Lalu berkatalah Samuel kepada umat Israel, "Kalau kamu hendak kembali kepada TUHAN dan menyembah Dia dengan sepenuh hatimu, haruslah kamu membuang semua dewa asing dan patung Dewi Asytoret. Serahkanlah dirimu sama sekali kepada TUHAN dan berbakti kepada-Nya saja, maka kamu akan dibebaskan-Nya dari orang Filistin."
\par 4 Jadi orang Israel membuang patung-patung Dewa Baal dan patung Dewi Asytoret, lalu mengabdi kepada TUHAN saja.
\par 5 Setelah itu berkatalah Samuel, "Kumpulkanlah seluruh bangsa Israel di Mizpa, maka aku akan berdoa kepada TUHAN untuk kamu."
\par 6 Lalu berkumpullah mereka semua di Mizpa. Mereka menimba air lalu menuangkannya sebagai persembahan kepada TUHAN, selanjutnya mereka berpuasa sepanjang hari itu. Kata mereka, "Kami telah berdosa kepada TUHAN." (Di Mizpa itu Samuel mulai menyelesaikan perkara-perkara perselisihan orang Israel.)
\par 7 Ketika orang Filistin mendengar bahwa orang Israel telah berkumpul di Mizpa, kelima raja Filistin berangkat ke sana dengan tentara mereka untuk menyerang orang Israel. Orang Israel mendengar hal itu lalu menjadi takut.
\par 8 Kata mereka kepada Samuel, "Janganlah berhenti berdoa kepada TUHAN Allah kita supaya kita diselamatkan-Nya dari orang Filistin."
\par 9 Karena itu Samuel memotong seekor anak domba muda, lalu mempersembahkannya sebagai kurban bakaran kepada TUHAN. Setelah itu ia berdoa supaya TUHAN menolong orang Israel, dan doanya itu dikabulkan.
\par 10 Ketika Samuel sedang mempersembahkan kurban bakaran itu, orang Filistin mulai menyerang; tetapi tepat pada saat itu TUHAN mengguntur dari langit ke atas mereka. Mereka menjadi kacau-balau lalu lari kebingungan.
\par 11 Tentara Israel keluar dari Mizpa lalu mengejar orang Filistin itu hampir sejauh Bet-Kar, dan menghancurkan mereka.
\par 12 Setelah itu Samuel menegakkan sebuah batu, di perbatasan Mizpa dan Sen. Ia berkata, "TUHAN telah menolong kita sepenuhnya." --lalu batu itu dinamainya "Batu Pertolongan".
\par 13 Demikianlah orang Filistin dikalahkan. TUHAN tidak mengizinkan mereka memasuki wilayah Israel lagi, selama Samuel masih hidup.
\par 14 Semua kota Israel yang telah ditaklukkan oleh orang Filistin, mulai dari Ekron sampai Gat, dikembalikan kepada Israel. Jadi Israel mendapat kembali seluruh wilayahnya. Orang Israel dan orang Amori pun hidup dengan damai.
\par 15 Samuel memerintah Israel seumur hidupnya.
\par 16 Setiap tahun ia mengadakan perjalanan keliling ke Betel, Gilgal, dan Mizpa. Dan di tempat-tempat itu ia menyelesaikan perkara-perkara perselisihan.
\par 17 Selalu sesudah itu ia pulang ke rumahnya di Rama. Di sana ia menjadi hakim orang Israel dan juga mendirikan sebuah mezbah bagi TUHAN.

\chapter{8}

\par 1 Setelah Samuel tua, diangkatnya anak-anaknya menjadi hakim di Israel.
\par 2 Anaknya yang sulung bernama Yoel dan yang kedua Abia. Mereka menjadi hakim di Bersyeba.
\par 3 Tetapi mereka tidak mengikuti kelakuan ayah mereka, melainkan hanya mengejar keuntungan sendiri saja. Mereka menerima uang sogok dan menghakimi rakyat secara tidak adil.
\par 4 Sebab itu semua pemimpin Israel berkumpul, lalu menghadap Samuel di Rama,
\par 5 dan berkata kepadanya, "Dengarlah Pak, Bapak sudah tua dan anak-anak Bapak tidak mengikuti kelakuan Bapak. Jadi sebaiknya Bapak mengangkat seorang raja supaya kami mempunyai raja seperti bangsa-bangsa lain."
\par 6 Tetapi Samuel tidak senang dengan usul mereka itu. Lalu ia berdoa kepada TUHAN,
\par 7 dan TUHAN berkata, "Kabulkanlah segala permintaan bangsa itu kepadamu. Sebab bukan engkau yang mereka tolak, melainkan Aku. Mereka tidak menghendaki Aku lagi sebagai raja mereka.
\par 8 Sejak Aku membawa mereka keluar dari Mesir, mereka berpaling daripada-Ku dan menyembah dewa-dewa, dan apa yang sekarang mereka lakukan kepadamu, itulah yang telah mereka lakukan kepada-Ku.
\par 9 Sebab itu kabulkanlah permintaan mereka; tetapi peringatkanlah mereka dengan sungguh-sungguh, dan beritahukanlah bagaimana mereka nanti akan diperlakukan oleh raja."
\par 10 Segala perkataan TUHAN itu disampaikan Samuel kepada orang-orang yang meminta seorang raja kepadanya, katanya,
\par 11 "Beginilah nantinya rajamu akan memperlakukan kamu," demikianlah Samuel menerangkan. "Ia akan memaksa anak-anakmu masuk tentara; sebagian dari mereka sebagai pasukan berkereta, sebagian sebagai pasukan berkuda, dan yang lainnya sebagai pasukan berjalan kaki.
\par 12 Sebagian diangkatnya menjadi perwira atas seribu orang, dan sebagian lain atas lima puluh orang. Raja itu akan memaksa anak-anakmu membajak ladangnya, mengumpulkan hasil panennya, membuat senjata-senjatanya dan perkakas kereta perangnya.
\par 13 Anak-anakmu yang perempuan akan disuruh membuat minyak wangi baginya dan bekerja sebagai tukang masaknya dan tukang rotinya.
\par 14 Ia akan mengambil ladangmu, kebun anggurmu, dan kebun zaitunmu yang paling baik dan memberikannya kepada para pegawainya.
\par 15 Ladang dan kebun anggurmu akan dikenakan pajak sepersepuluh dari hasilnya, lalu akan diberikannya kepada para perwira dan para pegawainya.
\par 16 Ia akan mengambil budakmu, ternakmu yang terbaik dan keledaimu dan memakainya untuk pekerjaannya.
\par 17 Sepersepuluh dari kawanan kambing dombamu akan diambil olehnya. Dan kamu sendiri akan menjadi hambanya.
\par 18 Jika masa itu sudah tiba, kamu akan berkeluh-kesah karena raja yang kamu pilih itu, tetapi TUHAN tidak mau mendengarkan keluhanmu."
\par 19 Tetapi bangsa itu tidak mau menghiraukan perkataan Samuel, malahan mereka berkata, "Biarlah! Bagaimanapun juga kami menginginkan raja.
\par 20 Kami ingin serupa dengan bangsa-bangsa lain, raja kami harus memerintah kami dan memimpin kami dalam peperangan."
\par 21 Setelah Samuel mendengar segala perkataan mereka itu, ia menyampaikannya kepada TUHAN.
\par 22 TUHAN menjawab, "Ikuti saja kemauan mereka dan angkatlah seorang raja bagi mereka." Kemudian Samuel menyuruh orang-orang Israel itu pulang ke rumahnya masing-masing.

\chapter{9}

\par 1 Ada seorang yang kaya dan berwibawa dari suku Benyamin, namanya Kish. Ia anggota keluarga Bekhorat, cabang dari marga Afiah. Ayahnya Abiel dan neneknya Zeror.
\par 2 Kish mempunyai anak laki-laki, namanya Saul, seorang pemuda yang tampan dan tegap. Tak seorang pun di seluruh Israel yang lebih tampan dari dia. Badannya juga lebih tinggi; rata-rata tinggi orang Israel hanya sampai pundaknya.
\par 3 Pada suatu hari beberapa ekor keledai milik Kish hilang. Sebab itu berkatalah Kish kepada Saul, "Bawalah salah seorang pelayan, dan carilah keledai-keledai itu."
\par 4 Lalu Saul dan pelayannya menjelajahi daerah pegunungan Efraim dan tanah Salisa, tetapi tidak menemukan binatang-binatang itu. Kemudian mereka berjalan terus ke tanah Sahalim, tetapi keledai-keledai itu tidak ada juga di situ. Mereka mencarinya pula di wilayah Benyamin, tetapi sia-sia saja.
\par 5 Ketika mereka tiba di tanah Zuf, Saul berkata kepada pelayannya, "Mari kita pulang saja, jangan-jangan ayah lebih cemas memikirkan kita daripada keledai-keledai itu."
\par 6 Pelayan itu menjawab, "Tunggu dulu! Di kota ini ada seorang hamba Allah. Dia sangat dihormati orang karena apa yang dikatakannya selalu benar-benar terjadi. Mari kita pergi kepadanya, barangkali dia dapat memberitahu di mana keledai-keledai itu."
\par 7 Kata Saul, "Jika kita pergi menemui dia, apakah yang akan kita berikan kepadanya? Bekal yang kita bawa sudah habis, dan kita tidak punya apa-apa untuk diberikan kepada hamba Allah itu, bukan?"
\par 8 Pelayan itu menjawab, "Kebetulan masih ada padaku sekeping uang perak. Itu akan kuberikan kepada hamba Allah itu supaya ia mau memberitahu kepada kita di mana keledai-keledai itu."
\par 9 "Baiklah!" jawab Saul. "Mari kita pergi." Lalu pergilah mereka ke kota, ke tempat hamba Allah itu. Ketika mereka mendaki bukit yang menuju ke kota itu, bertemulah mereka dengan beberapa gadis yang hendak menimba air di sumur. Lalu mereka bertanya kepada gadis-gadis itu, "Apakah petenung ada di kota?" (Pada zaman itu seorang nabi disebut petenung, jadi bilamana seorang ingin menanyakan sesuatu kepada Allah, dia berkata begini, "Marilah kita pergi kepada petenung.")
\par 10 [9:9]
\par 11 [9:9]
\par 12 "Ya, ada," jawab gadis-gadis itu, "baru saja ia berjalan mendahului kalian. Jika kalian cepat, mungkin kalian dapat menyusulnya. Ia baru datang ke kota karena hari ini di atas mezbah di bukit ada persembahan kurban untuk rakyat. Para undangan tidak akan makan sebelum ia datang, karena ia harus memberkati kurban itu dahulu. Jika kalian pergi sekarang, kalian akan menjumpainya sebelum ia mendaki bukit untuk makan."
\par 13 [9:12]
\par 14 Jadi Saul dan pelayannya berjalan ke kota, dan ketika mereka masuk pintu gerbang, mereka berpapasan dengan Samuel yang sedang berjalan menuju tempat ibadat di bukit.
\par 15 Adapun sehari sebelum itu TUHAN memberi pesan kepada Samuel begini,
\par 16 "Besok pagi kira-kira waktu begini, Aku akan menyuruh seorang dari suku Benyamin menemuimu. Lantiklah dia dengan upacara peminyakan untuk menjadi raja atas umat-Ku Israel. Ia akan menyelamatkan umat-Ku dari orang Filistin. Aku telah melihat penderitaan umat-Ku dan mendengar tangisan mereka meminta tolong."
\par 17 Ketika Samuel melihat Saul, TUHAN berkata kepadanya, "Inilah orang yang Kusebut kepadamu itu. Dia akan memerintah umat-Ku."
\par 18 Sementara itu Saul mendekati Samuel di dekat pintu gerbang, dan bertanya, "Maaf, Pak, di manakah rumah petenung itu?"
\par 19 Samuel menjawab, "Aku petenung itu. Pergilah mendahuluiku ke tempat ibadat di bukit, karena pada hari ini kamu akan makan bersamaku. Besok pagi-pagi segala pertanyaanmu akan kujawab, setelah itu bolehlah kamu pulang.
\par 20 Tentang keledai-keledaimu yang hilang tiga hari yang lalu itu, jangan khawatir, karena sudah ditemukan. Lagipula, yang paling berharga di seluruh Israel akan diberikan kepadamu dan keluarga ayahmu."
\par 21 Saul menjawab, "Aku ini dari suku Benyamin, suku yang paling kecil di Israel, dan lagi keluargaku yang paling tidak berarti dalam suku itu. Jadi mengapa Bapak berkata begitu kepadaku?"
\par 22 Kemudian Samuel mengajak Saul dan pelayannya masuk ke dalam ruangan yang besar, lalu ia memberi tempat yang utama kepada mereka di depan para undangan yang jumlahnya kira-kira tiga puluh orang.
\par 23 Selanjutnya Samuel berkata kepada juru masak, "Hidangkanlah daging yang tadi kutitipkan kepadamu itu."
\par 24 Kemudian juru masak itu mengambil bagian pilihan dari daging paha, lalu dihidangkannya di depan Saul. Samuel berkata, "Ini bagian yang kusisihkan untukmu supaya dapat kaumakan pada saat ini bersama dengan tamu-tamu lain. Silakan makan." Lalu makanlah Saul. Jadi pada hari itu Saul makan bersama dengan Samuel.
\par 25 Setelah itu mereka turun dari tempat ibadat itu dan masuk ke kota. Saul diberi tempat tidur di atas rumah,
\par 26 lalu ia tidur. Ketika fajar menyingsing, Samuel memanggil Saul yang ada di atap itu, katanya, "Bangunlah, mari kuantar kau pergi." Saul bangun, lalu berangkat bersama-sama Samuel.
\par 27 Ketika mereka sampai di pinggir kota, berkatalah Samuel kepada Saul, "Suruhlah pelayan itu berjalan mendahului kita." Lalu pergilah pelayan itu, dan Samuel berkata lagi, "Berhentilah di sini sebentar, aku akan memberitahukan pesan Allah kepadamu."

\chapter{10}

\par 1 Kemudian Samuel mengambil kendi berisi minyak zaitun, lalu minyak itu dituangkannya ke atas kepala Saul, dan diciumnya Saul sambil berkata, "TUHAN telah melantikmu menjadi raja atas Israel umat-Nya. Engkau akan memerintah umat-Nya dan melindungi mereka dari musuh-musuh mereka. Inilah tanda bagimu bahwa TUHAN telah memilih engkau menjadi raja atas umat-Nya:
\par 2 Setelah engkau meninggalkan aku pada hari ini, engkau akan bertemu dengan dua orang laki-laki di dekat kuburan Rahel di Zelzah di wilayah Benyamin. Mereka akan berkata kepadamu, begini: Keledai-keledai yang kaucari itu telah ditemukan. Sekarang ayahmu tidak khawatir lagi mengenai binatang-binatang itu, tetapi terus-menerus bertanya apa yang harus dilakukannya untuk menemukan engkau."
\par 3 Samuel berkata lagi, "Dari situ engkau berjalan terus sampai ke pohon tempat ibadat di Tabor. Di sana engkau akan bertemu dengan tiga orang laki-laki yang sedang menuju ke Betel untuk mempersembahkan kurban kepada Allah. Seorang menggiring tiga ekor anak kambing, seorang lagi membawa tiga buah roti, dan seorang lagi membawa kantong kulit berisi anggur.
\par 4 Mereka akan memberi salam kepadamu dan engkau akan diberinya dua buah roti yang harus kauterima.
\par 5 Sesudah itu engkau harus pergi ke bukit Allah di Gibea, tempat perkemahan orang Filistin. Di pintu masuk ke kota, engkau akan berpapasan dengan serombongan nabi yang turun dari bukit pengurbanan. Mereka menari-nari dan berteriak-teriak, bermain gambus, rebana, seruling dan kecapi.
\par 6 Pada saat itu juga engkau akan dikuasai oleh Roh TUHAN dan engkau akan mengikuti tari-tarian dan teriakan mereka itu, lalu engkau diberi sifat baru sehingga menjadi manusia lain.
\par 7 Bilamana semuanya itu terjadi, lakukanlah apa saja yang kaurasa baik, karena Allah menyertaimu.
\par 8 Setelah itu engkau harus pergi ke Gilgal dan menunggu di sana tujuh hari. Nanti aku datang untuk mempersembahkan kurban bakaran dan kurban perdamaian, serta memberitahukan kepadamu apa yang harus engkau lakukan."
\par 9 Segera setelah Saul meninggalkan Samuel, Saul menerima sifat baru dari Allah. Dan segala hal yang diberitahukan Samuel kepadanya pun terjadi pada hari itu juga.
\par 10 Ketika Saul dan pelayannya sampai di Gibea, mereka bertemu dengan serombongan nabi. Pada saat itu juga Saul dikuasai oleh Roh Allah, dan ia mengikuti tari-tarian dan teriakan nabi-nabi itu.
\par 11 Orang-orang yang mengenalnya dari dahulu, melihat kelakuannya itu, lalu bertanya seorang kepada yang lain, "Apakah yang telah terjadi dengan anak Kish itu? Apakah Saul juga menjadi nabi?"
\par 12 Lalu seorang yang tinggal di situ bertanya, "Bagaimana dengan nabi-nabi yang lain itu--menurut kalian siapakah ayah mereka?" Itulah asal mulanya peribahasa ini, "Apakah Saul juga menjadi nabi?"
\par 13 Setelah Saul selesai menari-nari dan berteriak-teriak, pergilah ia ke mezbah di atas bukit.
\par 14 Paman Saul melihat dia dan pelayannya, lalu bertanya kepada mereka, "Dari mana kamu?" jawab Saul, "Mencari keledai yang hilang itu, dan ketika kami tidak menemukannya, kami pergi menemui Samuel."
\par 15 "Dan apa yang dikatakannya kepadamu?" tanya pamannya.
\par 16 Saul menjawab, "Dia memberitahukan bahwa binatang-binatang itu telah ditemukan." Tetapi Saul tidak menceritakan kepada pamannya bahwa ia sudah dilantik menjadi raja oleh Samuel.
\par 17 Maka Samuel memanggil rakyat Israel supaya berkumpul di Mizpa,
\par 18 dan ia berkata kepada mereka, "Inilah yang dikatakan TUHAN, Allah Israel, 'Aku telah membawa kamu keluar dari Mesir dan membebaskan kamu dari kuasa orang Mesir dan segala bangsa lain yang menindas kamu.
\par 19 Aku telah melepaskan kamu dari segala sengsara dan penderitaan, tetapi sekarang kamu telah menolak Aku, Allahmu, karena kamu minta supaya Aku menunjuk seorang raja bagimu. Nah, baiklah, berkumpullah di hadapan-Ku, Tuhanmu, menurut suku dan margamu.'"
\par 20 Lalu Samuel menyuruh semua suku maju ke muka dan suku Benyamin yang kena pilih.
\par 21 Ketika keluarga-keluarga dalam suku Benyamin itu disuruh maju ke muka, maka keluarga Matri yang kena pilih. Akhirnya majulah para pria dalam keluarga Matri, dan Saul anak Kish yang kena pilih. Lalu mereka mencari Saul, tetapi ia tidak ada.
\par 22 Sebab itu orang-orang bertanya kepada TUHAN, "Ya TUHAN, apakah orang itu sudah ada di sini?" Jawab TUHAN, "Saul bersembunyi di belakang barang-barang perlengkapan."
\par 23 Lalu dengan segera mereka menjemput Saul dari situ, dan membawanya ke tengah-tengah rakyat. Maka berdirilah Saul di antara mereka dan dialah yang paling tinggi; orang lain hanya sampai pundaknya.
\par 24 Kemudian Samuel berkata kepada rakyat, "Inilah orang yang dipilih TUHAN, tak ada bandingnya di antara kita semua." Seluruh bangsa itu bersorak, "Hiduplah raja!"
\par 25 Kemudian Samuel menerangkan kepada bangsa itu tentang hak dan kewajiban raja, lalu ditulisnya di dalam buku yang disimpan di tempat yang suci. Sesudah itu Samuel menyuruh seluruh bangsa itu pulang ke rumahnya masing-masing.
\par 26 Saul juga pulang ke rumahnya di Gibea, disertai oleh beberapa orang yang gagah berani yang telah digerakkan hatinya oleh TUHAN.
\par 27 Tetapi beberapa orang jahat berkata, "Mana mungkin orang ini akan berguna bagi kita?" Mereka meremehkan Saul dan tidak memberikan hadiah kepadanya. Tetapi ia pura-pura tidak tahu.

\chapter{11}

\par 1 Kemudian Nahas, raja Amon, mengerahkan tentaranya lalu mengepung kota Yabesh di wilayah Gilead. Penduduk Yabesh mengajukan usul begini kepada Nahas, "Buatlah perjanjian dengan kami, maka kami akan takluk kepadamu."
\par 2 Nahas menjawab, "Dengan syarat inilah aku mau membuat perjanjian, 'Mata kananmu masing-masing akan kucungkil sebagai penghinaan terhadap seluruh Israel!'"
\par 3 Kemudian pemimpin-pemimpin kota Yabesh menjawab, "Kami minta tempo selama tujuh hari supaya kami dapat mengirim utusan ke seluruh tanah Israel. Jika tidak ada yang mau menolong kami, maka kami akan menyerah kepadamu."
\par 4 Demikianlah para utusan itu sampai di Gibea, tempat tinggal Saul. Ketika mereka menyampaikan kabar itu, menangislah rakyat karena putus asa.
\par 5 Pada saat itu Saul baru saja datang dari ladangnya dengan membawa sapinya. Ia pun bertanya, "Ada apa? Mengapa semua orang menangis?" Lalu kepadanya diberitahu kabar yang dibawa utusan dari Yabesh itu.
\par 6 Mendengar itu, Saul dikuasai oleh Roh Allah.
\par 7 Dengan sangat marah ia mengambil dua ekor sapi lalu dipotong-potongnya menjadi banyak potongan. Kemudian potongan-potongan itu dikirimkannya ke seluruh tanah Israel dengan perintah begini, "Siapa yang tidak maju berperang mengikuti Saul dan Samuel, sapi-sapinya akan dipotong-potong begini juga!" Bangsa Israel takut kepada apa yang mungkin dilakukan TUHAN, maka semua orang, tanpa kecuali, siap maju berperang bersama-sama.
\par 8 Mereka dikumpulkan dan dihitung di Bezek: Ada 300.000 orang Israel, dan 30.000 orang Yehuda.
\par 9 Setelah itu kepada utusan dari Yabesh diberitahu begini: "Katakanlah kepada penduduk Yabesh, bahwa besok pagi sebelum tengah hari, mereka akan mendapat pertolongan." Ketika penduduk Yabesh menerima pesan itu, sangat senanglah mereka.
\par 10 Kata mereka kepada Nahas, "Besok kami akan menyerah, dan bolehlah kauperlakukan sesuka hatimu."
\par 11 Besoknya, pagi-pagi, Saul membagi orang-orangnya menjadi tiga pasukan. Dan ketika fajar menyingsing mereka menyerbu ke tengah-tengah perkemahan orang Amon itu dan menyerang mereka. Sebelum tengah hari, tentara Saul telah mengalahkan musuh. Orang-orang Amon yang berhasil lolos, tercerai-berai sehingga tak ada dua orang yang dapat lari bersama-sama.
\par 12 Kemudian bertanyalah bangsa Israel kepada Samuel, "Di manakah orang-orang yang tadinya berani mengatakan bahwa Saul tidak layak menjadi raja kita? Hendaknya Bapak serahkan kepada kami, supaya kami bunuh."
\par 13 Tetapi Saul berkata, "Seorang pun tidak boleh dibunuh pada hari ini, karena pada hari ini TUHAN telah menyelamatkan Israel."
\par 14 Lalu berkatalah Samuel kepada mereka, "Marilah kita semua pergi ke Gilgal dan meresmikan Saul sebagai raja kita."
\par 15 Demikianlah mereka semua pergi ke Gilgal, dan di tempat ibadat itu Saul diresmikan sebagai raja. Mereka mempersembahkan kurban perdamaian, dan Saul serta seluruh bangsa Israel merayakan peristiwa itu.

\chapter{12}

\par 1 Lalu berkatalah Samuel kepada seluruh bangsa Israel, "Aku telah memenuhi segala permintaanmu dan mengangkat seorang raja atas kamu.
\par 2 Mulai sekarang dialah yang akan memimpin kamu. Lihatlah, aku ini sudah tua dan beruban. Aku telah lama sekali memimpin kamu--sejak masa mudaku sampai sekarang. Buktinya ialah anak-anakku yang sudah besar itu.
\par 3 Sekarang inilah, aku berdiri di hadapan kamu! Jika aku telah melakukan kesalahan, tuduhlah aku di hadapan TUHAN dan di hadapan raja yang dipilih-Nya. Pernahkah aku mengambil sapi atau keledai orang lain? Pernahkah aku menipu atau menindas atau menerima uang sogok? Katakanlah, supaya apa saja yang kuambil dapat kukembalikan."
\par 4 Orang-orang itu menjawab, "Tidak, tidak pernah Bapak menipu atau menindas kami; tidak pernah Bapak mengambil apa-apa milik siapa pun."
\par 5 Lalu berkatalah Samuel, "Pada hari ini TUHAN dan raja yang dipilih-Nya menjadi saksi bahwa aku ternyata tidak bersalah sedikitpun juga." "Ya benar, TUHAN menjadi saksi kita," jawab mereka.
\par 6 Lalu Samuel berkata lagi, "Tuhanlah yang memilih Musa dan Harun, dan yang membawa leluhurmu keluar dari Mesir.
\par 7 Sekarang, jangan pergi dulu. Aku akan menuduh kamu di hadapan TUHAN. Ingatlah akan segala kemurahan TUHAN yang telah ditunjukkan-Nya kepadamu dan kepada leluhurmu.
\par 8 Ketika Yakub dan keluarganya datang ke Mesir, dan orang Mesir menindas mereka, leluhurmu memohon pertolongan kepada TUHAN, lalu TUHAN mengutus Musa dan Harun. Kedua orang itu membawa leluhurmu keluar dari Mesir dan menolong mereka menetap di negeri ini.
\par 9 Tetapi kemudian mereka melupakan TUHAN, Allah mereka, sebab itu Ia membiarkan mereka diserang dan dikalahkan oleh Sisera, panglima tentara di Hazor, dan oleh orang Filistin serta raja Moab.
\par 10 Kemudian mereka berdoa kepada TUHAN dan mengakui dosa mereka; kata mereka, 'Kami telah berdosa karena berpaling daripada-Mu, ya TUHAN, dan menyembah patung-patung Dewa Baal dan Asytoret. Tetapi, sekarang, selamatkanlah kami dari musuh kami, maka kami akan mengabdi kepada-Mu!'
\par 11 Kemudian TUHAN mengutus Gideon, kemudian Barak, sesudah itu Yefta, dan akhirnya aku. Kami masing-masing telah melepaskan kamu dari musuhmu, sehingga kamu dapat hidup dengan tentram.
\par 12 Namun ketika kamu melihat bahwa kamu hendak diserang oleh Nahas raja Amon, kamu menolak TUHAN sebagai rajamu dan berkata kepadaku, 'Kami ingin diperintah oleh seorang raja.'"
\par 13 "Sekarang, lihat, inilah dia, raja yang kamu pilih itu. Kamu telah memintanya, dan TUHAN telah memberikannya kepadamu.
\par 14 Berbahagialah kamu jika kamu menghormati TUHAN, dan mengabdi kepada-Nya, serta mendengarkan dan mentaati perintah-Nya, apalagi jika kamu dan rajamu itu tetap setia kepada TUHAN Allahmu.
\par 15 Sebaliknya, jika kamu tidak mendengarkan TUHAN, malahan melawan perintah-Nya, pasti kamu dan rajamu akan dilawan TUHAN.
\par 16 Sekarang, dengarkan lagi! Perhatikanlah keajaiban besar yang akan dilakukan TUHAN di depan matamu.
\par 17 Aku akan berseru kepada TUHAN, dan meskipun sekarang musim kemarau, TUHAN akan menurunkan guruh dan hujan sebagai jawaban atas doaku. Pada saat itu, kamu akan sadar betapa besar dosamu terhadap TUHAN karena meminta seorang raja."
\par 18 Lalu berdoalah Samuel, dan pada hari itu juga TUHAN menurunkan guruh dan hujan. Seluruh bangsa itu menjadi takut kepada TUHAN dan kepada Samuel.
\par 19 Lalu berkatalah mereka kepada Samuel, "Tolonglah, Bapak, berdoalah bagi kami ini kepada TUHAN Allahmu, supaya kami jangan mati. Sekarang kami sadar bahwa selain dosa kami dahulu, kami telah berdosa lagi karena meminta seorang raja."
\par 20 Samuel pun menjawab, "Jangan takut, meskipun kamu telah melakukan kejahatan itu, janganlah berpaling dari TUHAN, melainkan tetaplah mengabdi kepada-Nya dengan sepenuh hatimu.
\par 21 Janganlah mengikuti dewa-dewa yang tidak dapat menolong dan menyelamatkan kamu, karena mereka sebenarnya tidak ada.
\par 22 TUHAN telah bersumpah, bahwa Dia tidak akan meninggalkan kamu karena Dia telah memutuskan untuk menjadikan kamu umat-Nya sendiri.
\par 23 Mengenai diriku, semoga aku jangan sampai berdosa terhadap TUHAN karena tidak mendoakan kamu. Sebaliknya aku juga akan terus mengajarkan kepadamu hal-hal yang baik dan yang benar.
\par 24 Hormatilah TUHAN dan mengabdilah kepada-Nya dengan setia dan dengan sepenuh hatimu. Ingatlah akan perbuatan-perbuatan besar yang telah dilakukan-Nya bagimu.
\par 25 Tetapi jika kamu terus juga berbuat dosa, maka kamu dan rajamu akan binasa."

\chapter{13}

\par 1 Pada suatu hari, ketika Saul sudah menjadi raja,
\par 2 ia memilih 3.000 orang Israel; dari antaranya 2.000 tinggal dengan dia di Mikhmas dan di daerah pegunungan Betel, sedang yang 1.000 lagi pergi dengan Yonatan putranya ke Gibea di wilayah suku Benyamin. Selebihnya dari rakyat itu diizinkan pulang ke rumahnya masing-masing.
\par 3 Kemudian Yonatan membunuh panglima Filistin yang di Geba, dan bangsa Filistin mendengar tentang hal itu. Lalu Saul menyuruh membunyikan trompet diseluruh negeri untuk memanggil orang Ibrani supaya maju berperang.
\par 4 Kepada seluruh bangsa Israel diberitahukan bahwa Saul telah membunuh panglima Filistin itu dan bahwa orang Filistin sangat membenci orang Israel. Sebab itu rakyat mematuhi panggilan untuk bergabung dengan Saul di Gilgal.
\par 5 Sementara itu orang Filistin telah berkumpul juga untuk menyerang orang Israel; mereka mempunyai 30.000 kereta perang, 6.000 orang berkuda, dan prajurit-prajurit sebanyak pasir di tepi laut. Mereka berkemah di Mikhmas di sebelah timur Bet-Awen.
\par 6 Kemudian mereka melancarkan serangan sengit terhadap orang Israel, sehingga orang Israel berada dalam keadaan genting. Sejumlah dari orang-orang Israel bersembunyi di dalam gua dan liang gunung atau di dalam sela-sela batu karang, atau dalam sumur dan lubang di tanah,
\par 7 ada juga yang menyeberangi Sungai Yordan dan masuk ke wilayah Gad dan Gilead. Saul masih di Gilgal, dan rakyat yang mengikutinya gemetar ketakutan.
\par 8 Saul menunggu Samuel tujuh hari lamanya, sesuai dengan pesan Samuel kepadanya, tetapi Samuel belum juga sampai di Gilgal, sedangkan Saul mulai ditinggalkan rakyatnya.
\par 9 Sebab itu ia berkata kepada mereka, "Bawalah kemari kurban bakaran dan kurban perdamaian itu." Lalu ia sendiri mempersembahkan kurban bakaran itu.
\par 10 Tetapi baru saja ia selesai melakukannya, datanglah Samuel. Saul menyongsong dia hendak menyalamnya,
\par 11 tetapi Samuel bertanya, "Apa yang telah Baginda lakukan?" Jawab Saul, "Aku melihat bahwa rakyat mulai meninggalkan aku, sedangkan Bapak tidak datang pada waktu yang Bapak tentukan. Lagipula, orang Filistin telah berkumpul di Mikhmas.
\par 12 Lalu pikirku, jangan-jangan tentara Filistin sebentar lagi menyerangku di sini di Gilgal, padahal aku belum memohon belas kasihan TUHAN. Sebab itu aku mengambil keputusan untuk mempersembahkan kurban itu."
\par 13 Mendengar itu Samuel berkata kepada Saul, "Tindakan Baginda itu bodoh sekali. Seandainya Baginda mentaati perintah TUHAN Allah kepada Baginda, pastilah TUHAN akan menetapkan Baginda dan keturunan Baginda menjadi raja atas Israel untuk selama-lamanya.
\par 14 Tetapi sekarang pemerintahan Baginda tidak akan bertahan. Karena Baginda tidak mengikuti perintah TUHAN, maka TUHAN telah memilih orang lain yang dikehendaki-Nya untuk diangkat menjadi raja atas umat-Nya."
\par 15 Kemudian Samuel meninggalkan Gilgal dan meneruskan perjalanannya ke Gibea di wilayah Benyamin. Sementara itu Saul memeriksa pasukannya, jumlahnya kira-kira 600 orang.
\par 16 Demikianlah Saul dan Yonatan putranya, dan anak buah mereka berkemah di Geba di wilayah Benyamin; sedang orang Filistin berkemah di Mikhmas.
\par 17 Pada waktu itu juga tiga rombongan perampok Filistin berangkat dari perkemahan Filistin untuk merampok; rombongan yang satu pergi ke arah Ofra di wilayah Syual,
\par 18 rombongan kedua menuju Bet-Horon, dan rombongan yang satu lagi menuju ke perbatasan yang menghadap ke Lembah Zeboim dan padang gurun.
\par 19 Di seluruh Israel tidak ada tukang besi, karena orang Filistin tidak mengizinkan orang Ibrani membuat pedang dan tombak.
\par 20 Jadi orang Israel harus pergi kepada orang Filistin jika hendak mengasahkan mata bajaknya, cangkulnya, kapaknya dan aritnya;
\par 21 upahnya ialah sekeping uang kecil untuk mengasah kapak dan membetulkan alat pemacu sapi, dan dua keping uang kecil untuk mengasah mata bajak atau cangkul.
\par 22 Itulah sebabnya, pada hari pertempuran di Mikhmas itu, tak seorang pun dari rakyat Israel mempunyai pedang atau tombak, kecuali Saul dan Yonatan putranya.
\par 23 Orang Filistin menempatkan sebuah pos penjagaan untuk mempertahankan sebuah sela gunung di Mikhmas.

\chapter{14}

\par 1 Pada suatu ketika Yonatan berkata kepada pemuda pembawa senjatanya, "Ayolah, kita pergi ke pos Filistin di seberang sana!" Tetapi ia tidak minta izin kepada ayahnya.
\par 2 Pada saat itu Saul sedang berada di bawah pohon delima di Migron, tidak jauh dari Gibea. Rakyat yang mengikutinya kira-kira 600 orang banyaknya.
\par 3 (Imam yang membawa efod ialah Ahiya anak Ahitub, yaitu saudara Ikabod anak Pinehas, dan Pinehas anak Eli, imam TUHAN di Silo.) Rakyat juga tidak tahu bahwa Yonatan telah pergi dari situ.
\par 4 Untuk mencapai pos Filistin itu, Yonatan harus melalui sebuah sela gunung yang diapit oleh dua batu karang yang bergerigi, yang masing-masing bernama Bozes dan Sene.
\par 5 Batu karang yang di sebelah utara menghadap ke Mikhmas, dan yang di sebelah selatan menghadap ke Geba.
\par 6 Maka berkatalah Yonatan kepada pemuda itu, "Mari kita pergi ke pos orang Filistin si kafir itu. Barangkali TUHAN mau menolong kita. Sebab TUHAN dapat memberi kemenangan, baik dengan banyak orang maupun dengan sedikit orang."
\par 7 Pemuda itu menjawab, "Apa saja yang hendak Tuan lakukan, hamba setuju."
\par 8 "Baiklah," kata Yonatan. "Pada waktu menyeberang, kita akan memperlihatkan diri kepada orang-orang Filistin itu.
\par 9 Jika mereka menyuruh kita berhenti dan menunggu sampai mereka datang kepada kita, maka kita akan tetap tinggal di situ saja dan tidak usah menyeberang.
\par 10 Tetapi jika mereka menyuruh kita datang kepada mereka, kita akan ke sana, karena itulah tandanya mereka sudah diserahkan TUHAN kepada kita."
\par 11 Ketika mereka berdua memperlihatkan diri kepada orang Filistin di pos penjagaan itu, berkatalah orang-orang Filistin itu, "Lihatlah, ada orang-orang Ibrani yang keluar dari lubang-lubang tempat mereka bersembunyi!"
\par 12 Kemudian mereka berseru kepada Yonatan dan hambanya, "Naiklah kemari supaya kami perlihatkan sesuatu kepadamu!" Lalu berkatalah Yonatan kepada pemuda itu, "Ikutilah aku, karena mereka sudah diserahkan TUHAN kepada Israel."
\par 13 Dengan diikuti pemuda itu Yonatan merangkak ke atas. Sewaktu Yonatan muncul, berjatuhanlah orang-orang Filistin dipukul olehnya, sedang di belakangnya, pemuda itu membunuhi mereka.
\par 14 Dalam serangan pertama itu, Yonatan dan pemuda itu membunuh kira-kira dua puluh orang di daerah seluas seperempat hektar.
\par 15 Kejadian itu menggemparkan orang Filistin di perkemahan dan di padang-padang. Seluruh rakyat Filistin dengan pos-pos penjagaan serta rombongan-rombongan perampok menjadi ketakutan; bahkan bumi pun bergetar, dan terjadilah hiruk-pikuk yang hebat.
\par 16 Sementara itu tentara Saul yang berkawal di Gibea di wilayah Benyamin, melihat orang Filistin berlari kian kemari dengan sangat kebingungan.
\par 17 Maka berkatalah Saul kepada anak buahnya, "Hitunglah para prajurit dan periksalah siapa yang tidak ada." Mereka melakukan perintah itu lalu ketahuanlah bahwa Yonatan dan pemuda pembawa senjatanya tidak ada.
\par 18 "Bawalah efod itu kemari," kata Saul kepada Imam Ahia. (Pada hari itu Ahia membawa efod itu di hadapan orang Israel.)
\par 19 Ketika Saul sedang berbicara dengan imam itu, kegemparan dalam perkemahan orang Filistin semakin bertambah, sehingga Saul berkata kepada imam itu, "Kita tak sempat lagi untuk meminta petunjuk dari TUHAN!"
\par 20 Lalu Saul dan orang-orangnya serentak maju ke tempat pertempuran. Di situ mereka melihat ada kekacauan di antara orang Filistin; mereka saling menyerang dan saling membunuh.
\par 21 Orang-orang Ibrani yang tinggal di daerah orang Filistin, dan yang telah ikut dengan tentara mereka, berbalik haluan serta bergabung dengan Saul dan Yonatan.
\par 22 Juga orang-orang Israel yang bersembunyi di bukit-bukit Efraim, yang mendengar bahwa orang Filistin telah lari, ikut bergabung dan menyerang orang Filistin itu.
\par 23 Mereka bertempur sepanjang jalan sampai lewat Bet-Awen. TUHAN menyelamatkan Israel pada hari itu.
\par 24 Pada hari itu orang-orang Israel dalam keadaan lemah karena lapar, sebab Saul telah bersumpah demikian, "Terkutuklah orang yang memakan apa pun pada hari ini sebelum aku membalas dendam terhadap musuhku." Karena itu tidak seorang pun yang makan sepanjang hari itu.
\par 25 Kemudian sampailah mereka semua di sebuah hutan dan menemukan banyak sekali madu,
\par 26 tetapi tak seorang pun yang makan sesuatu dari madu itu, karena takut kena kutuk yang diucapkan Saul.
\par 27 Tetapi Yonatan tidak mengetahui bahwa ayahnya telah mengancam rakyat dengan kutuk itu; sebab itu diulurkannya tongkat yang dibawanya, dan dicucukkannya ke dalam sarang madu, lalu diambilnya madu itu dan dimakannya. Segera ia merasa segar dan bertambah kuat.
\par 28 Kemudian seorang dari prajurit-prajurit itu berkata kepadanya, "Kami semua sudah tak berdaya karena lapar. Tetapi ayah Tuan telah mengancam kami dengan sumpah, katanya, 'Terkutuklah orang yang makan pada hari ini.'"
\par 29 Yonatan menjawab, "Sungguh keterlaluan tindakan ayahku itu terhadap bangsa kita! Lihat, keadaanku sekarang jauh lebih baik setelah aku makan sedikit madu.
\par 30 Seandainya rakyat pada hari ini diizinkan makan makanan yang telah dirampasnya ketika mengalahkan musuh, pastilah lebih banyak orang Filistin yang dapat kita bunuh!"
\par 31 Pada hari itu orang Israel mengalahkan orang Filistin setelah bertempur sepanjang jalan dari Mikhmas sampai ke Ayalon. Tetapi setelah itu mereka begitu letih,
\par 32 sehingga mereka berebut-rebutan mengambil barang rampasan yang didapatkan dari musuh; mereka mengambil domba dan sapi lalu memotongnya di tempat itu juga, dan langsung memakan daging yang masih ada darahnya.
\par 33 Lalu diberitahu kepada Saul demikian, "Baginda, rakyat telah berdosa terhadap TUHAN karena makan daging dengan darahnya." "Kamu pengkhianat!" teriak Saul. "Gulingkanlah sebuah batu besar kemari."
\par 34 Kemudian katanya lagi kepada orang-orang yang membawa berita itu, "Berpencarlah kamu di antara rakyat dan suruhlah mereka membawa sapi dan dombanya kemari, supaya dipotong di atas batu ini dan dimakan di sini; dengan demikian mereka tidak makan daging dengan darahnya dan tidak berdosa lagi terhadap TUHAN." Malam itu juga seluruh rakyat membawa binatang rampasannya dan memotongnya di situ.
\par 35 Lalu Saul mendirikan mezbah bagi TUHAN; itulah mezbah pertama yang didirikannya.
\par 36 Kemudian berkatalah Saul kepada anak buahnya, "Marilah kita menyerang orang Filistin pada malam ini lalu merampok mereka sampai pagi dan membunuh mereka semua." Mereka menjawab, "Baiklah Baginda lakukan apa yang Baginda pandang baik." Tetapi imam berkata, "Marilah kita lebih dahulu minta petunjuk dari Allah."
\par 37 Lalu bertanyalah Saul kepada Allah, "Bolehkah aku menyerang orang Filistin? Apakah mereka akan Kauserahkan ke tangan Israel?" Tetapi pada hari itu Allah tidak menjawabnya.
\par 38 Sebab itu berkatalah Saul kepada para pemimpin rakyat, "Mari ke sini semua dan periksalah dosa apa yang telah dilakukan pada hari ini.
\par 39 Aku berjanji demi TUHAN yang hidup, yang telah memberikan kemenangan kepada Israel, bahwa orang yang bersalah akan dihukum mati, meskipun dia adalah Yonatan putraku." Tetapi tak seorang pun berani menjawabnya.
\par 40 Lalu berkatalah Saul kepada seluruh rakyat Israel, "Kamu semua berdiri di sebelah sini." Jawab mereka, "Hendaklah Baginda lakukan apa yang Baginda pandang baik."
\par 41 Kemudian berdoalah Saul, "TUHAN, Allah Israel, mengapa hari ini Engkau tidak menjawab aku? TUHAN, jawablah aku melalui batu-batu yang suci ini. Jika kesalahan itu ada padaku atau pada Yonatan, jawablah dengan batu Urim, tetapi jika kesalahan itu ada pada umat-Mu Israel, jawablah dengan batu Tumim." Maka jawaban TUHAN menunjuk kepada Yonatan dan Saul; dengan demikian rakyat dinyatakan tidak bersalah.
\par 42 Lalu berkatalah Saul, "Ya TUHAN, tentukanlah antara aku dan Yonatan." Maka Yonatanlah yang dinyatakan bersalah.
\par 43 Lalu kata Saul kepada Yonatan, "Apa yang telah kauperbuat?" Yonatan menjawab, "Aku telah memakan sedikit madu yang kuambil dengan ujung tongkatku; aku bersedia untuk mati."
\par 44 Kata Saul kepadanya, "Semoga aku dihukum Allah jika engkau tidak dihukum mati, Yonatan!"
\par 45 Tetapi rakyat berkata kepada Saul, "Haruskah Yonatan dihukum mati? Padahal dialah yang membawa kemenangan besar ini? Tidak! Kami berjanji demi TUHAN yang hidup, bahwa Yonatan tidak boleh kehilangan sehelai rambut pun dari kepalanya. Karena dengan pertolongan Tuhanlah, ia telah mendapat kemenangan pada hari ini." Demikianlah Yonatan diselamatkan rakyat dari hukuman mati.
\par 46 Setelah itu orang Filistin kembali ke wilayah mereka sendiri karena Saul tidak mengejar mereka lagi.
\par 47 Setelah Saul menjadi raja Israel, ia berperang di mana-mana melawan segala musuh Israel, yaitu orang Moab, orang Amon, dan orang Edom, raja-raja negeri Zoba, dan orang Filistin. Di mana pun ia berperang, selalu ia mendapat kemenangan.
\par 48 Ia bertempur dengan gagah berani dan mengalahkan orang Amalek, serta membebaskan Israel dari kuasa perampok.
\par 49 Putra-putra Saul ialah Yonatan, Yiswi, dan Malkisua. Putrinya yang sulung bernama Merab, dan yang kedua bernama Mikhal.
\par 50 Istrinya bernama Ahinoam, anak Ahimaas; panglima tentaranya adalah Abner anak Ner, yaitu paman Saul.
\par 51 Kish ayah Saul, dan Ner ayah Abner, adalah anak-anak Abiel.
\par 52 Selama hidupnya Saul selalu berperang dengan sengit melawan orang Filistin. Sebab itu, apabila Saul melihat orang yang kuat atau berani, ia segera menjadikannya anggota tentaranya.

\chapter{15}

\par 1 Pada suatu hari Samuel berkata kepada Saul, "Tuhanlah yang menyuruh aku melantik Baginda menjadi raja atas Israel umat-Nya. Sebab itu, hendaknya Baginda mendengarkan perintah TUHAN Yang Mahakuasa.
\par 2 Dia akan menghukum orang Amalek, karena leluhur mereka melawan orang Israel ketika orang Israel datang dari Mesir.
\par 3 Jadi, pergilah dan seranglah orang Amalek dan hancurkanlah segala milik mereka. Janganlah tinggalkan sesuatu apa pun; bunuhlah semua orang laki-laki, wanita, anak-anak dan bayi; juga sapi, domba, unta dan keledai."
\par 4 Lalu Saul mempersiapkan tentaranya dan menghitungnya di Telaim; ada 200.000 orang prajurit dari Israel dan 10.000 orang dari Yehuda.
\par 5 Setelah itu pergilah Saul bersama-sama anak buahnya ke kota Amalek dan menghadang musuh di dasar sungai yang kering.
\par 6 Ia juga mengirim pesan peringatan kepada orang Keni katanya, "Pergilah dan tinggalkan orang Amalek, supaya kamu jangan ikut kutumpas bersama mereka, sebab dahulu leluhurmu telah menunjukkan sikap persahabatan kepada orang Israel ketika orang Israel datang dari Mesir." Maka pergilah orang Keni dari situ.
\par 7 Setelah itu Saul bertempur dengan orang Amalek, dan mengalahkan mereka sepanjang jalan dari Hawila sampai ke Syur di sebelah timur Mesir.
\par 8 Seluruh rakyat dan tentara Amalek dibunuh oleh Saul. Tetapi Agag raja orang Amalek ditangkap dan dibiarkan hidup oleh Saul dan tentaranya, demikian juga domba dan sapi yang paling baik, anak sapi serta anak domba yang paling gemuk, dan segala sesuatu yang berharga; hanya ternak yang tidak berguna dan tidak berharga saja yang dibinasakan.
\par 9 [15:8]
\par 10 Lalu berkatalah TUHAN kepada Samuel,
\par 11 "Aku menyesal telah mengangkat Saul menjadi raja, sebab ia telah berbalik daripada-Ku, dan tidak melaksanakan perintah-Ku." Samuel sedih, dan sepanjang malam ia mengeluh kepada TUHAN.
\par 12 Keesokan harinya, pagi-pagi, Samuel berangkat hendak bertemu dengan Saul. Kepadanya diberitahu bahwa Saul telah pergi ke kota Karmel untuk mendirikan batu peringatan bagi dirinya di sana, lalu terus ke Gilgal.
\par 13 Ketika Samuel bertemu dengan Saul, berkatalah Saul kepadanya, "Semoga TUHAN memberkati Bapak! Aku telah melaksanakan perintah TUHAN."
\par 14 Tetapi Samuel bertanya, "Kalau begitu, mengapa kudengar sapi melenguh dan domba mengembik?"
\par 15 Jawab Saul, "Binatang-binatang itu rampasan dari orang Amalek. Domba dan sapi yang paling baik telah diambil rakyat untuk dipersembahkan sebagai kurban kepada TUHAN Allahmu. Tetapi selebihnya telah kami binasakan sama sekali."
\par 16 Mendengar itu Samuel berkata, "Tunggu sebentar, aku akan memberitahukan apa yang dikatakan Allah kepadaku tadi malam." "Katakanlah," kata Saul.
\par 17 Samuel berkata, "Bukankah Baginda telah menjadi pemimpin atas suku-suku Israel, meskipun Baginda menganggap dirinya tidak penting? Baginda telah dipilih TUHAN menjadi raja atas Israel,
\par 18 lagipula Baginda disuruh TUHAN membinasakan orang Amalek yang jahat itu. Baginda disuruhnya berperang melawan mereka sampai mereka habis semuanya.
\par 19 Mengapa Baginda tidak mentaati perintah TUHAN? Mengapa Baginda beramai-ramai mengambil barang rampasan, dan dengan begitu Baginda membuat kesal hati TUHAN?"
\par 20 Saul menjawab, "Aku telah mentaati perintah TUHAN, dan pergi berperang sesuai dengan suruhan-Nya kepadaku; Agag raja Amalek kutawan dan semua orang Amalek telah kubunuh.
\par 21 Tetapi anak buahku tidak membunuh semua ternak; mereka telah memilih domba dan sapi yang paling baik dari hasil rampasan itu untuk mempersembahkannya di Gilgal sebagai kurban kepada TUHAN Allahmu."
\par 22 Tetapi Samuel berkata, "Manakah yang lebih disukai TUHAN, ketaatan atau kurban persembahan? Taat kepada TUHAN lebih baik daripada mempersembahkan kurban. Patuh lebih baik daripada lemak domba.
\par 23 Sebab membangkang terhadap TUHAN sama jahatnya seperti melakukan sihir, dan hati yang sombong sama jahatnya seperti menyembah dewa. Karena Baginda telah melawan perintah TUHAN, maka TUHAN pun tidak mengakui Baginda lagi sebagai raja."
\par 24 Mendengar itu Saul mengaku kepada Samuel, katanya, "Memang aku telah berdosa. Aku melanggar perintah TUHAN dan mengabaikan petunjuk Bapak. Aku takut kepada anak buahku lalu kukabulkan permintaan mereka.
\par 25 Tetapi sekarang, aku mohon kepada Bapak, ampunilah dosaku dan kembalilah bersama-sama dengan aku ke Gilgal, supaya aku dapat beribadat kepada TUHAN."
\par 26 Tetapi Samuel menjawab, "Aku tidak mau kembali bersama-sama dengan Baginda. Baginda telah melawan perintah TUHAN, dan sekarang Dia tidak mengakui Baginda sebagai raja Israel."
\par 27 Lalu Samuel berpaling hendak pergi, tetapi Saul menahan dia dengan memegang jubahnya, sehingga jubah itu sobek.
\par 28 Lalu berkatalah Samuel kepadanya, "Pada hari ini TUHAN telah menyobek kerajaan Israel dari Baginda, dan memberikannya kepada orang lain yang lebih baik daripada Baginda.
\par 29 Yang Mulia Allah Israel tidak berdusta dan tidak pula mengubah pendirian-Nya, karena Dia bukan manusia."
\par 30 Saul menjawab, "Aku telah berdosa. Tetapi tolonglah Pak, kembalilah bersama-sama dengan aku supaya aku dapat beribadat kepada TUHAN Allah Bapa. Dengan demikian Bapak menghormati aku di depan para pemimpin bangsaku dan di depan seluruh Israel."
\par 31 Maka kembalilah Samuel bersama-sama dengan dia ke Gilgal, dan Saul beribadat kepada TUHAN.
\par 32 Kemudian Samuel memerintahkan, "Bawalah raja Agag kemari," Agag datang kepadanya, dengan penuh harapan karena ia berpikir, "Bahaya maut telah lewat."
\par 33 Lalu Samuel berkata, "Seperti pedangmu telah membuat banyak ibu kehilangan anaknya, demikian jugalah ibumu akan kehilangan anaknya." Lalu Samuel mencincang Agag di depan mezbah di Gilgal.
\par 34 Sesudah itu Samuel pergi ke Rama, dan Raja Saul pulang ke rumahnya di Gibea.
\par 35 Sampai akhir hidupnya, Samuel tidak bertemu lagi dengan Saul. Tetapi Samuel sedih karena TUHAN menyesal telah mengangkat Saul menjadi raja Israel.

\chapter{16}

\par 1 Beberapa waktu kemudian TUHAN berkata kepada Samuel, "Berapa lama lagi engkau bersedih hati karena Saul? Bukankah dia telah tak diakui sebagai raja Israel? Sekarang, ambillah minyak zaitun dan pergilah ke Betlehem, kepada seorang yang bernama Isai, karena salah seorang dari anak-anaknya telah Kupilih menjadi raja."
\par 2 Tetapi Samuel menjawab, "Bagaimana aku harus melakukannya? Jika hal itu kedengaran oleh Saul, pastilah aku dibunuhnya!" TUHAN menjawab, "Bawalah seekor sapi muda, dan katakanlah kepada rakyat bahwa engkau datang ke sana untuk mempersembahkan kurban kepada TUHAN.
\par 3 Lalu undanglah Isai ke upacara pengurbanan itu. Nanti akan Kuberitahukan apa yang harus kaulakukan. Orang yang Kutunjukkan kepadamu, harus kaulantik menjadi raja."
\par 4 Samuel melakukan seperti yang dikatakan TUHAN. Ketika ia sampai di Betlehem, para pemimpin kota menyambutnya dengan terkejut dan bertanya, "Apakah kunjungan Bapak membawa selamat?"
\par 5 Jawab Samuel, "Ya, benar. Aku datang untuk mempersembahkan kurban kepada TUHAN. Maka sucikanlah dirimu supaya dapat mengikuti upacara pengurbanan itu." Isai dan anak-anaknya juga disuruhnya menyucikan dirinya masing-masing, dan diundangnya ke upacara itu.
\par 6 Ketika mereka sudah berkumpul dan Samuel melihat Eliab anak Isai, pikirnya, "Pastilah ini yang akan dipilih TUHAN."
\par 7 Tetapi TUHAN berkata kepada Samuel, "Janganlah kau terpikat oleh rupanya yang elok dan tinggi badannya; bukan dia yang Kukehendaki. Aku tidak menilai seperti manusia menilai. Manusia melihat rupa, tetapi Aku melihat hati."
\par 8 Kemudian Isai memanggil Abinadab, anaknya, lalu disuruhnya menghadap Samuel. Tetapi Samuel berkata, "Dia juga tidak dipilih TUHAN."
\par 9 Lalu Syama disuruh maju oleh Isai, tetapi Samuel berkata, "Dia juga tidak dipilih TUHAN."
\par 10 Demikianlah Isai menyuruh ketujuh anaknya berturut-turut menghadap Samuel, tetapi Samuel berkata kepadanya, "Mereka tidak dipilih TUHAN."
\par 11 Lalu bertanyalah Samuel kepadanya, "Hanya inikah semua anak laki-lakimu?" Jawab Isai, "Masih ada seorang lagi, yang bungsu, tetapi ia sedang menggembalakan domba." Samuel berkata, "Suruhlah memanggil dia, karena kita tidak akan makan sebelum ia datang."
\par 12 Lalu Isai menyuruh memanggil anak itu. Ternyata ia seorang pemuda yang tampan dan sehat, dan matanya indah. Lalu berkatalah TUHAN kepada Samuel, "Inilah dia; lantiklah dia!"
\par 13 Segera Samuel mengambil minyak zaitun itu dan melantik Daud dengan upacara peminyakan di hadapan abang-abangnya. Pada saat itu juga Daud dikuasai oleh Roh TUHAN. Sejak hari itu dan seterusnya Roh TUHAN menyertainya. Kemudian pulanglah Samuel ke Rama.
\par 14 Adapun Saul telah ditinggalkan oleh Roh TUHAN, dan kini ia disiksa oleh roh jahat yang diutus TUHAN.
\par 15 Sebab itu hamba-hamba Saul berkata kepadanya, "Kami tahu bahwa roh jahat yang diutus TUHAN menyiksa Baginda.
\par 16 Jika Baginda setuju, kami bersedia mencari seorang yang pandai main kecapi. Bilamana Baginda disiksa roh jahat, orang itu dapat memainkan kecapinya, dan Baginda akan merasa nyaman lagi."
\par 17 Lalu Saul memerintahkan kepada mereka, "Carilah seorang yang pandai main musik, dan bawalah dia kemari."
\par 18 Mendengar itu seorang hambanya berkata, "Isai penduduk kota Betlehem mempunyai seorang anak laki-laki yang pandai main musik. Ia pemberani dan seorang prajurit yang baik. Lagipula ia gagah dan pandai berbicara, dan TUHAN selalu menolongnya."
\par 19 Maka Saul mengirim utusan kepada Isai membawa pesan ini, "Suruhlah anakmu Daud, yang menjaga domba-dombamu itu menghadap kepadaku."
\par 20 Kemudian Isai menyiapkan seekor keledai, lalu memuatinya dengan seekor anak kambing, roti dan kantong kulit berisi penuh anggur. Semuanya itu diberikannya kepada Daud untuk dipersembahkan kepada Saul.
\par 21 Demikianlah Daud datang kepada Saul dan menjadi pelayannya. Ia sangat disayangi Saul dan diangkat menjadi pembawa senjatanya.
\par 22 Kemudian Saul mengirim pesan kepada Isai, "Aku suka kepada Daud. Izinkanlah dia tetap tinggal di sini dan melayani aku."
\par 23 Sejak itu, setiap kali Saul didatangi roh jahat itu, Daud mengambil kecapinya dan memainkannya. Lalu Saul merasa tenang dan nyaman lagi, karena ditinggalkan oleh roh jahat itu.

\chapter{17}

\par 1 Pada suatu ketika orang Filistin mengerahkan tentaranya untuk maju berperang. Mereka mengatur barisannya di kota Sokho, dalam wilayah Yehuda dan memasang perkemahannya di antara Sokho dan Azeka, dekat Efes-Damim.
\par 2 Saul dan orang-orang Israel berkumpul juga dan berkemah di Lembah Ela; mereka bersiap-siap untuk menghadapi serangan orang Filistin.
\par 3 Demikianlah barisan orang Filistin berdiri di sebuah bukit dan barisan orang Israel di bukit yang lain, dan di antaranya ada sebuah lembah.
\par 4 Maka seorang jago berkelahi yang bernama Goliat, dari kota Gat, keluar dari perkemahan Filistin untuk menantang orang Israel. Tingginya kira-kira tiga meter,
\par 5 dan ia memakai topi tembaga dan baju perang tembaga yang beratnya kira-kira lima puluh tujuh kilogram.
\par 6 Kakinya dilindungi oleh penutup kaki dari tembaga, dan di bahunya ia memanggul lembing tembaga.
\par 7 Gagang tombaknya sebesar kayu pada alat tenun, dan mata tombaknya kira-kira tujuh kilogram beratnya. Seorang prajurit berjalan di depannya dengan membawa perisainya.
\par 8 Goliat berhenti lalu berseru kepada tentara Israel, "Apa yang sedang kamu lakukan di situ? Hendak berperangkah kamu? Aku seorang Filistin, hai hamba-hamba Saul! Pilihlah seorang di antara kamu yang berani turun untuk bertempur melawan aku.
\par 9 Jika dalam perang tanding itu, aku terbunuh, kami rela menjadi hambamu, tetapi jika aku yang menang dan membunuhnya, kamulah yang akan menjadi hamba kami.
\par 10 Sekarang juga, kutantang tentara Israel; pilihlah seorang untuk bertanding melawan aku!"
\par 11 Ketika Saul dan orang-orangnya mendengar tantangan itu, terkejutlah mereka dan menjadi sangat ketakutan.
\par 12 Daud adalah anak Isai orang Efrata, dari Betlehem di Yehuda. Isai mempunyai delapan orang anak laki-laki, dan pada zaman pemerintahan Saul, Isai sudah tua sekali.
\par 13 Ketiga anak Isai yang tertua telah pergi berperang mengikuti Saul. Yang sulung bernama Eliab, yang kedua Abinadab, dan yang ketiga Syama.
\par 14 Daud anak yang bungsu. Pada waktu ketiga abangnya yang tertua itu sedang berperang mengikuti Saul,
\par 15 Daud sering meninggalkan Saul dan pulang ke Betlehem untuk menggembalakan domba ayahnya.
\par 16 Selama empat puluh hari, setiap pagi dan petang, Goliat mendekati barisan orang Israel dan menantang mereka.
\par 17 Pada suatu hari Isai berkata kepada Daud, "Ambillah sepuluh kilogram gandum panggang dengan sepuluh roti ini, dan bawalah kepada abang-abangmu di perkemahan tentara.
\par 18 Bawalah juga sepuluh buah keju ini untuk komandan pasukan. Tanyakanlah bagaimana keadaan abang-abangmu, dan bawalah bukti untukku bahwa engkau telah bertemu dengan mereka dan mereka dalam keadaan selamat.
\par 19 Mereka ada di Lembah Ela bersama Raja Saul, dan semua orang Israel sedang bertempur melawan orang Filistin."
\par 20 Keesokan harinya, pagi-pagi, Daud bangun lalu berkemas. Dombanya dititipkannya kepada seorang penjaga, kemudian ia mengambil bawaannya lalu berangkat, sesuai dengan perintah ayahnya. Ia sampai ke perkemahan pada waktu orang Israel berangkat ke medan pertempuran sambil memekikkan sorak perang.
\par 21 Tentara Filistin dan tentara Israel saling berhadapan dan bersiap-siap untuk bertempur.
\par 22 Lalu Daud menitipkan bawaannya itu kepada penjaga perlengkapan tentara, dan lari ke medan perang untuk menemui abang-abangnya.
\par 23 Tetapi ketika ia sedang berbicara dengan mereka, Goliat maju ke depan dan menantang orang Israel, seperti yang biasa dilakukannya. Daud pun mendengar kata-kata tantangannya itu.
\par 24 Segera setelah orang Israel melihat Goliat, mereka lari ketakutan.
\par 25 "Lihatlah dia!" kata mereka sesamanya. "Dengarlah kata-kata tantangannya! Saul raja kita telah berjanji bahwa siapa saja yang membunuh Goliat, akan diberikan hadiah yang besar. Raja juga akan mengawinkan orang itu dengan putrinya. Dan keluarga ayah orang itu akan dibebaskan dari pajak."
\par 26 Lalu Daud berkata, "Berani benar orang Filistin si kafir itu menantang tentara Allah yang hidup!" Kemudian ia bertanya kepada salah seorang prajurit, "Apakah yang akan diberikan kepada orang yang bisa membunuh orang Filistin itu dan menghapus penghinaan dari Israel?"
\par 27 Rakyat memberitahukan kepadanya apa yang telah dijanjikan raja.
\par 28 Eliab abang Daud yang sulung mendengar Daud berbicara dengan prajurit-prajurit. Dia menjadi marah kepada Daud dan berkata, "Mengapa kau datang kemari? Siapa telah kau suruh mengurus domba-dombamu yang beberapa ekor itu di padang gurun? Aku tahu, kau berlagak berani; kau datang kemari hanya untuk melihat pertempuran bukan?"
\par 29 Jawab Daud, "Apa salahku? Aku kan hanya bertanya!"
\par 30 Lalu dia pergi dan menanyakan hal yang sama kepada prajurit-prajurit yang lain; dan ia mendapat jawaban begitu juga.
\par 31 Tetapi beberapa orang yang mendengar perkataan Daud, menyampaikannya kepada Saul, jadi Daud dipanggilnya menghadap.
\par 32 Kata Daud kepada Saul, "Baginda, kita tak perlu takut kepada orang Filistin itu! Hamba bersedia melawan dia."
\par 33 "Jangan," jawab Saul. "Bagaimana mungkin engkau bertanding dengan dia? Engkau masih muda sekali, sedangkan dia sudah biasa berperang sejak masa mudanya."
\par 34 Tetapi Daud berkata, "Baginda, hamba biasa menggembalakan domba ayah hamba. Bilamana ada singa atau beruang datang menerkam domba,
\par 35 binatang buas itu hamba kejar dan hantam, lalu domba itu hamba selamatkan. Dan jika singa atau beruang itu melawan hamba, maka hamba pegang lehernya, lalu hamba pukul sampai mati.
\par 36 Hamba telah membunuh singa maupun beruang, dan orang Filistin si kafir itu juga akan sama seperti binatang-binatang itu, karena ia berani menghina tentara dari Allah yang hidup.
\par 37 TUHAN telah menyelamatkan hamba dari singa dan beruang, Dia juga akan menyelamatkan hamba dari orang Filistin itu." Lalu kata Saul kepadanya, "Baiklah, semoga TUHAN menolongmu."
\par 38 Saul memberikan pakaian perangnya, yaitu sebuah baju besi kepada Daud dan Daud mengenakannya. Lalu Saul memakaikan topi tembaga pada kepala Daud.
\par 39 Akhirnya Daud mengikatkan pedang Saul pada baju besi itu lalu mencoba berjalan, tetapi tidak bisa, karena Daud tidak biasa memakai pakaian perang. "Hamba tidak bisa berjalan dengan pakaian ini," katanya kepada Saul. "Hamba tidak biasa memakainya." Lalu seluruh pakaian perang itu ditanggalkannya.
\par 40 Kemudian ia mengambil tongkat gembalanya, dan memilih lima buah batu yang bulat dari sungai, lalu dimasukkannya ke dalam kantongnya. Dengan umban siap di tangannya, pergilah ia menemui Goliat.
\par 41 Beberapa saat kemudian Goliat yang didahului oleh pembawa perisainya, mulai berjalan mendekati Daud.
\par 42 Tetapi ketika ia melihat Daud dan memperhatikannya, Goliat tertawa mengejek karena Daud masih muda sekali dan tampan.
\par 43 Kata Goliat kepada Daud, "Untuk apa tongkat itu? Apakah kauanggap aku ini anjing?" Lalu Daud dikutukinya demi para dewanya.
\par 44 Lagipula ia menantang Daud, katanya, "Ayo, maju! akan kuberikan tubuhmu kepada burung dan binatang supaya dimakan."
\par 45 Tetapi Daud menjawab, "Engkau datang melawanku dengan pedang, tombak dan lembing, tetapi aku datang melawanmu dengan nama TUHAN Yang Mahakuasa, Allah tentara Israel yang kauhina itu.
\par 46 Hari ini juga TUHAN akan menyerahkan engkau kepadaku; engkau akan kukalahkan dan kepalamu akan kupenggal. Tubuhmu dan tubuh prajurit-prajurit Filistin akan kuberikan kepada burung dan binatang supaya dimakan. Maka seluruh dunia akan tahu bahwa kami bangsa Israel mempunyai Allah yang kami sembah,
\par 47 dan semua orang di sini akan melihat bahwa TUHAN tidak memerlukan pedang atau tombak untuk menyelamatkan umat-Nya. Dialah yang menentukan jalan peperangan ini dan Dia akan menyerahkan kamu ke dalam tangan kami."
\par 48 Goliat mulai maju mendekati Daud, lalu dengan cepat Daud berlari ke arah barisan orang Filistin untuk menghadapi dia.
\par 49 Daud merogoh kantongnya, mengambil sebuah batu lalu diumbankannya kepada Goliat. Batu itu menghantam dahi Goliat sehingga pecahlah tengkoraknya, dan ia roboh dengan mukanya ke tanah.
\par 50 Daud berlari kepada Goliat, lalu berdiri di dekatnya; ia mengambil pedang Goliat dan mencabutnya dari sarungnya, lalu dipenggalnya kepala orang Filistin itu. Demikianlah Daud mengalahkan dan membunuh Goliat, hanya dengan umban dan batu! Ketika orang-orang Filistin melihat bahwa pahlawan mereka sudah mati, larilah mereka.
\par 51 [17:50]
\par 52 Orang-orang Israel dan Yehuda bersorak-sorak dan mengejar orang-orang Filistin sampai ke Gat dan pintu gerbang Ekron. Orang-orang Filistin yang terluka bergelimpangan sepanjang jalan ke Saaraim itu.
\par 53 Setelah itu orang Israel kembali dari mengejar orang Filistin, lalu merampas isi perkemahan mereka.
\par 54 Daud mengambil kepala Goliat, dan dibawanya ke Yerusalem, tetapi senjata-senjata Goliat disimpannya di dalam kemahnya sendiri.
\par 55 Ketika Saul melihat Daud pergi melawan Goliat, bertanyalah ia kepada Abner penglima tentaranya, "Abner, anak siapakah dia?" "Hamba tidak tahu, Baginda," jawab Abner.
\par 56 Lalu perintah Saul, "Pergilah dan tanyakanlah hal itu."
\par 57 Jadi ketika Daud kembali ke perkemahan sesudah membunuh Goliat, ia dibawa Abner menghadap Saul. Daud masih menjinjing kepala Goliat.
\par 58 Lalu bertanyalah Saul kepadanya, "Hai anak muda! anak siapa engkau?" Daud menjawab, "Hamba ini anak Isai dari Betlehem."

\chapter{18}

\par 1 Setelah Saul dan Daud selesai bercakap-cakap, Daud diangkat oleh Saul menjadi pegawainya dan sejak hari itu ia tidak diizinkan pulang ke rumah orang tuanya. Yonatan putra Saul, telah mendengar percakapan itu. Ia merasa tertarik juga kepada Daud, dan mengasihinya seperti dirinya sendiri.
\par 2 [18:1]
\par 3 Karena itu Yonatan bersumpah akan bersahabat dengan Daud selama-lamanya.
\par 4 Yonatan menanggalkan jubahnya lalu diberikannya kepada Daud, juga pakaian perangnya serta pedangnya, busurnya dan ikat pinggangnya.
\par 5 Daud melaksanakan dengan baik segala tugas yang diberikan Saul kepadanya. Sebab itu ia diangkat oleh Saul menjadi perwira dalam tentaranya, dan Daud disukai oleh semua prajurit serta oleh para hamba Saul.
\par 6 Ketika Daud kembali sesudah mengalahkan Goliat orang Filistin itu, dan para prajurit berbaris masuk ke dalam kota, wanita-wanita keluar dari semua kota di Israel untuk menyambut Raja Saul. Mereka menyanyikan lagu-lagu gembira, dan menari-nari dengan memainkan rebana dan kecapi.
\par 7 Sambil menari-nari para wanita bernyanyi demikian, "Saul membunuh beribu-ribu musuh, tetapi Daud berpuluh-puluh ribu."
\par 8 Mendengar itu, Saul menjadi sangat marah. Sebab pikirnya, "Daud dianggap telah menewaskan berpuluh-puluh ribu, sedangkan aku hanya beribu-ribu saja. Sebentar lagi tentulah ia dijadikan raja oleh mereka!"
\par 9 Sejak hari itu ia iri hati kepada Daud.
\par 10 Keesokan harinya Saul ada di dalam rumahnya sedang memegang tombaknya dan tiba-tiba ia didatangi roh jahat yang diutus Allah, sehingga ia mengamuk seperti orang gila. Waktu itu Daud sedang main kecapi seperti biasa.
\par 11 "Kutancapkan tombak ini kepada Daud sampai tertancap ke dinding!" pikir Saul, lalu ia melemparkan tombak itu sampai dua kali kepada Daud, tetapi Daud berhasil mengelak.
\par 12 Saul menyadari bahwa Daud dilindungi TUHAN sedangkan ia sendiri ditinggalkan TUHAN. Karena itu ia menjadi takut terhadap Daud.
\par 13 Maka ia memindahkan Daud dari lingkungan istana, dengan mengangkatnya menjadi komandan atas seribu orang prajurit. Dengan demikian Daud memimpin pasukannya dalam setiap peperangan.
\par 14 Ia berhasil melaksanakan segala tugasnya, sebab TUHAN menolongnya.
\par 15 Ketika Saul melihat bahwa Daud selalu berhasil, makin takutlah ia kepadanya.
\par 16 Tetapi seluruh Israel dan Yehuda sayang kepada Daud karena ia pemimpin yang banyak jasanya.
\par 17 Saul merencanakan supaya Daud dibunuh oleh orang Filistin dalam pertempuran agar bukan dia sendiri yang membunuhnya. Jadi pada suatu hari Saul berkata kepada Daud, "Merab putriku yang sulung, akan kujodohkan dengan engkau, asal saja engkau menunjukkan keberanianmu dalam berperang untuk TUHAN."
\par 18 Daud menjawab, "Siapakah hamba ini, dan apalah arti keluarga ayah hamba di Israel, sehingga hamba menjadi menantu Raja?"
\par 19 Tetapi ketika tiba waktunya Merab hendak dikawinkan dengan Daud, tahu-tahu gadis itu dikawinkan dengan Adriel dari Mehola.
\par 20 Mikhal putri Saul yang lain jatuh cinta kepada Daud. Ketika Saul mendengar hal itu, ia setuju juga.
\par 21 Pikirnya, "Baiklah kutawarkan Mikhal kepada Daud, supaya Daud terjebak dan dapat dibunuh oleh orang Filistin." Jadi untuk kedua kalinya Saul berkata kepada Daud, "Sekarang engkau boleh menjadi menantuku."
\par 22 Lalu ia menyuruh para pegawainya supaya mengatakan kepada Daud dengan diam-diam, demikian, "Baginda sayang kepadamu dan demikian juga semua pegawainya; jadi sekaranglah saat yang tepat bagimu untuk mempersunting putrinya."
\par 23 Tetapi ketika mereka menyampaikan saran itu kepada Daud, ia menjawab, "Kalian kira mudah untuk menjadi menantu raja? Aku ini orang miskin dan tidak berarti!"
\par 24 Para pegawai itu memberitahukan kepada Saul jawaban Daud itu,
\par 25 lalu Saul menyuruh mereka mengatakan kepada Daud, begini, "Yang dikehendaki baginda sebagai emas kawin hanyalah 100 kulit kulup orang Filistin, sebagai pembalasan kepada musuh baginda." (Inilah yang direncanakan Saul untuk menewaskan Daud dengan perantaraan orang Filistin.)
\par 26 Para pegawai Saul menyampaikan pesan itu kepada Daud, dan Daud menerima tawaran raja. Karena itu, sebelum batas waktunya habis, ia berangkat dengan anak buahnya ke daerah Filistin. Dibunuhnya 200 orang Filistin, lalu diambilnya kulit kulup mereka, dan diserahkannya kepada Saul tanpa kurang satu pun. Setelah itu Saul mengawinkan Mikhal dengan Daud.
\par 27 [18:26]
\par 28 Demikianlah Saul menyadari bahwa Daud dilindungi TUHAN dan juga dicintai oleh Mikhal putrinya.
\par 29 Maka makin takutlah ia kepada Daud dan ia membencinya seumur hidupnya.
\par 30 Setiap kali bilamana tentara Filistin datang menyerang, Daud lebih berhasil menumpas mereka daripada para perwira Saul yang lain. Maka makin masyhurlah Daud.

\chapter{19}

\par 1 Saul mengatakan kepada Yonatan putranya, dan kepada semua pegawainya bahwa ia hendak membunuh Daud. Tetapi Yonatan sangat sayang kepada Daud,
\par 2 karena itu ia berkata kepadanya, "Ayahku berusaha membunuhmu; jadi hati-hatilah! Bersembunyilah besok pagi di suatu tempat rahasia, dan tinggallah di situ.
\par 3 Aku akan pergi ke padang bersama dengan ayahku dan berdiri di dekat tempat engkau bersembunyi itu. Lalu aku akan berbicara dengan beliau mengenai maksud beliau terhadapmu, dan hasil pembicaraan itu akan kuberitahukan kepadamu."
\par 4 Lalu Yonatan mengatakan yang baik tentang Daud kepada Saul, katanya, "Ayah, janganlah Ayah mencelakakan hambamu Daud. Sebab ia tidak berbuat kesalahan apa-apa terhadap Ayah; malahan sebaliknya, ia telah sangat berjasa kepada Ayah.
\par 5 Ia telah mempertaruhkan nyawanya ketika membunuh Goliat, dan TUHAN telah memberikan kemenangan besar kepada Israel melalui dia. Ayah sendiri telah melihat kejadian itu, dan Ayah gembira pada waktu itu. Jadi, mengapa sekarang Ayah hendak menganiaya orang yang tidak bersalah dan mau membunuh Daud tanpa alasan apa pun?"
\par 6 Saul menjadi insaf oleh kata-kata Yonatan, dan ia bersumpah demi nama TUHAN bahwa Daud tidak akan dibunuhnya.
\par 7 Setelah itu Yonatan memanggil Daud, dan menceritakan semua percakapan tadi. Kemudian ia membawa Daud kepada Saul, dan Daud melayani dia seperti dahulu.
\par 8 Kemudian pecah lagi perang dengan orang Filistin. Daud menyerang dan mengalahkan mereka sama sekali, sehingga mereka lari.
\par 9 Pada suatu hari Saul sedang duduk di rumahnya dengan tombak di tangannya dan tiba-tiba Saul didatangi lagi oleh roh jahat yang diutus TUHAN. Pada waktu itu Daud sedang memainkan kecapinya.
\par 10 Lalu Saul berusaha menancapkan tombaknya kepada Daud yang berada di dekat dinding. Tetapi Daud mengelak, sehingga tombak itu tertancap di dinding. Daud lari untuk menyelamatkan dirinya.
\par 11 Pada malam itu juga Saul mengirim beberapa orang ke rumah Daud untuk mengintip Daud dan membunuhnya pada waktu pagi. Tetapi Mikhal istrinya berkata kepadanya, "Jika engkau tidak lari malam ini, besok pagi engkau pasti dibunuh."
\par 12 Lalu Mikhal menurunkan Daud dari jendela, dan Daud melarikan diri sehingga lolos.
\par 13 Sesudah itu Mikhal mengambil patung dewa rumah tangganya lalu meletakkannya di tempat tidur Daud. Kemudian di bagian kepalanya ditaruhnya bantal dari bulu kambing, dan selanjutnya diselubunginya patung itu dengan selimut.
\par 14 Ketika para utusan Saul datang untuk menangkap Daud, Mikhal berkata, "Daud sedang sakit."
\par 15 Tetapi Saul menyuruh mereka pergi lagi ke situ untuk melihat Daud dengan mata mereka sendiri. Perintahnya, "Bawalah dia ke mari, sekalian dengan tempat tidurnya, supaya kubunuh."
\par 16 Lalu masuklah utusan itu ke dalam kamar dan ternyata patunglah yang terletak di tempat tidur dengan bantal bulu kambing di bagian kepalanya.
\par 17 Saul berkata kepada Mikhal, "Tega benar kautipu aku begini! Musuhku telah kautolong melarikan diri!" Jawab Mikhal, "Dia mengancam akan membunuhku jika aku tak mau membantunya melarikan diri."
\par 18 Setelah Daud berhasil lolos, ia pergi ke rumah Samuel di Rama, lalu memberitahukan semua yang dilakukan Saul kepadanya. Kemudian Daud dan Samuel bersama-sama pergi ke Nayot dan tinggal di situ.
\par 19 Tetapi kepada Saul diberitahukan bahwa Daud ada di Nayot dekat Rama.
\par 20 Sebab itu Saul mengirim beberapa utusan untuk menangkap Daud. Tetapi mereka mendapati sekumpulan nabi yang sedang menari-nari dan berteriak-teriak bersama-sama, dipimpin oleh Samuel. Lalu para utusan Saul itu dikuasai oleh Roh Allah sehingga mereka juga mulai menari-nari dan berteriak-teriak.
\par 21 Ketika berita itu sampai kepada Saul, ia mengirimkan utusan-utusan yang lain dan mereka itu pun mulai menari-nari dan berteriak-teriak. Untuk ketiga kalinya Saul mengirimkan utusan-utusan dan begitu juga terjadi pada mereka.
\par 22 Sebab itu Saul berangkat ke Rama. Sesampainya di sumur besar di Sekhu, ia bertanya di mana Samuel dan Daud berada, dan orang di situ memberitahukan kepadanya bahwa mereka ada di Nayot.
\par 23 Jadi ia terus ke sana, tetapi ia juga dikuasai oleh Roh Allah, sehingga menari-nari dan berteriak-teriak sepanjang perjalanan ke Nayot.
\par 24 Setelah sampai di sana, ia menanggalkan pakaiannya dan menari-nari serta berteriak-teriak di depan Samuel, lalu rebah ke tanah dengan telanjang sepanjang hari dan malam. (Inilah asal mula peribahasa yang berbunyi, "Apakah Saul juga menjadi nabi?")

\chapter{20}

\par 1 Maka larilah Daud dari Nayot dekat Rama dan pergi menemui Yonatan, lalu bertanya, "Apa salahku? Kejahatan apa yang telah kulakukan dan apa dosaku terhadap ayahmu sehingga ia mau membunuhku?"
\par 2 Jawab Yonatan, "Jangan berkata begitu! Demi TUHAN, engkau tidak akan dibunuh. Ayahku selalu memberitahukan segala rencananya kepadaku, baik yang penting, maupun yang tidak. Jadi tidak mungkin beliau merahasiakan maksudnya itu terhadap aku. Sebab itu, apa yang kaukatakan itu tidak benar!"
\par 3 Daud menjawab, "Ayahmu tahu benar bahwa engkau sayang kepadaku. Jadi pastilah ia tidak mau memberitahukan hal itu kepadamu supaya engkau jangan sedih. Bagaimanapun juga, demi TUHAN yang hidup, sesungguhnyalah hidupku bagaikan telur di ujung tanduk!"
\par 4 Lalu berkatalah Yonatan, "Apa saja yang kauminta akan kulakukan."
\par 5 Daud menjawab, "Begini, besok adalah Pesta Bulan Baru, dan seharusnya aku makan bersama-sama dengan baginda. Tetapi jika engkau setuju, aku akan pergi dan bersembunyi di padang sampai lusa malam.
\par 6 Jika ayahmu menanyakan mengapa aku tidak hadir, katakanlah kepada beliau, bahwa aku telah memohon izin darimu untuk segera pulang ke Betlehem, karena di sana diadakan upacara kurban tahunan bagi seluruh keluargaku.
\par 7 Jika raja berkata, 'Baiklah,' maka aku akan aman, tetapi jika beliau marah, ketahuilah bahwa beliau sudah mengambil keputusan untuk mencelakakan aku.
\par 8 Aku minta pertolonganmu, karena kita telah mengikat perjanjian di hadapan TUHAN. Namun jika aku bersalah, bunuhlah aku, tetapi janganlah kauserahkan aku kepada ayahmu."
\par 9 Jawab Yonatan, "Jangan khawatir! Jika aku tahu dengan pasti bahwa ayahku telah bertekad untuk mencelakakanmu, tentulah akan kuberitahukan kepadamu."
\par 10 Kemudian Daud bertanya, "Siapakah yang akan memberitahukan kepadaku seandainya ayahmu menjawabmu dengan geram?"
\par 11 "Marilah kita pergi ke padang," kata Yonatan. Maka pergilah mereka bersama-sama ke padang.
\par 12 Lalu Yonatan berkata kepada Daud, "Semoga TUHAN, Allah Israel menjadi saksi kita! Besok pagi atau lusa kira-kira pada waktu seperti sekarang ini, aku akan berusaha mengetahui isi hati ayahku. Jika ia bersikap baik terhadapmu, aku akan mengirim kabar kepadamu.
\par 13 Jika ia berniat mencelakakanmu, pasti akan kuberitahukan kepadamu juga supaya engkau dapat menyelamatkan dirimu. Semoga TUHAN membunuhku kalau hal itu tidak kulakukan. Mudah-mudahan TUHAN menolongmu seperti dia telah ditolongnya juga!
\par 14 Dan jika aku masih hidup, ingatlah perjanjian kita di hadapan TUHAN dan setialah kepadaku; tetapi jika aku sudah tiada,
\par 15 tunjukkanlah kesetiaanmu kepada keturunanku untuk selama-lamanya. Dan apabila TUHAN membinasakan semua musuhmu,
\par 16 janganlah perjanjian antara kita berdua terputus. Jika sampai terputus, Tuhanlah yang akan menghukummu."
\par 17 Jadi untuk kedua kalinya Yonatan menyuruh Daud berjanji akan mengasihi dia, sebab ia mengasihi Daud seperti dirinya sendiri.
\par 18 Kemudian berkatalah Yonatan kepadanya, "Besok Pesta Bulan Baru. Pasti akan ketahuan bahwa engkau tidak hadir, sebab tempatmu kosong pada meja makan;
\par 19 apalagi lusa! Sebab itu, pergilah ke tempat persembunyianmu yang dahulu itu, dan tunggulah di belakang timbunan batu di situ.
\par 20 Lusa aku akan melepaskan tiga anak panah ke arah batu itu, seakan-akan membidik sasaran.
\par 21 Lalu akan kusuruh budakku memungut panah-panah itu, dan jika aku berkata kepadanya, 'Ai, panah itu lebih dekat lagi ke mari; ambillah,' itu berarti bahwa engkau aman dan boleh keluar. Aku bersumpah demi TUHAN yang hidup bahwa engkau tidak dalam bahaya apa pun.
\par 22 Tetapi jika aku berkata kepada budakku, 'Ai, panah itu lebih jauh lagi ke sana,' itu berarti TUHAN menyuruh engkau pergi; jadi pergilah.
\par 23 Semoga TUHAN menjaga agar kita selamanya memegang perjanjian antara kita berdua."
\par 24 Maka bersembunyilah Daud di padang. Pada Pesta Bulan Baru, Raja Saul datang ke perjamuan.
\par 25 Seperti biasanya dia duduk dekat dinding; Yonatan mengambil tempat di hadapannya, Abner di sebelah Saul, tetapi kursi Daud tetap kosong.
\par 26 Tetapi Saul tidak menyinggung keadaan itu pada hari itu, sebab pikirnya, "Barangkali ia mengalami sesuatu hal, sehingga ia tidak bersih menurut agama."
\par 27 Pada hari berikutnya, yaitu hari kedua Bulan Baru, kursi Daud masih kosong juga, lalu bertanyalah Saul kepada Yonatan, "Mengapa kemarin dan hari ini Daud tidak hadir pada perjamuan?"
\par 28 Yonatan menjawab, "Ia telah minta izin kepadaku untuk pergi ke Betlehem.
\par 29 Katanya, 'Izinkanlah aku pergi, karena keluargaku sedang merayakan pesta kurban di Betlehem, dan aku disuruh datang oleh abangku. Sebab itu, jika kauizinkan, biar aku pergi untuk bertemu dengan sanak saudaraku.' Itulah sebabnya ia tidak hadir pada perjamuan ini, Ayah."
\par 30 Mendengar itu Saul menjadi marah sekali kepada Yonatan dan berkata kepadanya, "Anak haram jadah! Sekarang aku tahu bahwa engkau memihak kepada Daud dan membuat malu dirimu serta ibumu!
\par 31 Tidak sadarkah engkau bahwa selama Daud hidup engkau tidak akan mungkin menjadi raja atas negeri ini? Sekarang suruhlah orang mencari dia dan membawanya kemari, sebab dia harus mati!"
\par 32 Maka jawab Yonatan, "Mengapa ia harus mati? Dosanya apa?"
\par 33 Mendengar itu Saul melemparkan tombaknya kepada Yonatan hendak membunuhnya. Maka tahulah Yonatan bahwa ayahnya memang sungguh-sungguh berniat hendak membunuh Daud.
\par 34 Dengan marah sekali Yonatan meninggalkan meja perjamuan dan ia tidak makan apa-apa pada hari kedua Pesta Bulan Baru itu. Ia sangat prihatin memikirkan Daud yang telah dihina oleh Saul ayahnya.
\par 35 Keesokan harinya Yonatan pergi ke padang untuk bertemu dengan Daud seperti yang telah mereka rencanakan bersama. Ia membawa seorang anak laki-laki, budaknya,
\par 36 lalu berkata kepadanya, "Ayo, pungutlah panah-panah yang akan kupanahkan." Sedang anak itu berlari, Yonatan memanahkan sebuah panah melewati anak itu.
\par 37 Ketika anak itu sampai ke tempat jatuhnya panah itu, Yonatan berseru kepadanya, "Ai, panah itu lebih jauh lagi ke sana!
\par 38 Cepat sedikit, jangan melamun!" Anak itu memungut panah itu dan kembali kepada tuannya.
\par 39 Ia sama sekali tidak mengerti maksud kata-kata Yonatan itu, hanya Yonatan dengan Daudlah yang mengerti.
\par 40 Kemudian Yonatan memberikan alat pemanahnya kepada anak itu dan menyuruh dia kembali ke kota.
\par 41 Setelah anak itu pergi, Daud keluar dari belakang timbunan batu itu, lalu sujud tiga kali. Baik dia maupun Yonatan, kedua-duanya menangis ketika mereka saling berciuman. Kesedihan Daud lebih besar daripada kesedihan Yonatan.
\par 42 Kemudian berkatalah Yonatan kepada Daud, "Semoga Allah menolongmu. TUHAN akan menjamin bahwa kita dan keturunan kita selalu memegang perjanjian kita berdua di hadapan Allah." Setelah itu pergilah Daud, dan Yonatan pun pulang ke kota.

\chapter{21}

\par 1 Maka pergilah Daud kepada Imam Ahimelekh di Nob. Dengan terkejut Ahimelekh ke luar menemui Daud dan bertanya kepadanya, "Mengapa engkau ke mari sendirian saja?"
\par 2 Daud menjawab, "Aku telah diserahi sebuah tugas oleh baginda dan dilarang memberitahu kepada seorang pun tentang tugas itu. Sebab itu anak buahku telah kusuruh menunggu di suatu tempat.
\par 3 Maaf, sekarang aku ingin bertanya, makanan apa yang ada pada Bapak? Berikanlah kepadaku lima buah roti atau apa saja yang ada di sini."
\par 4 Lalu kata imam itu, "Aku tidak punya roti biasa, hanya roti persembahan yang ada; engkau boleh mengambilnya, asal saja anak buahmu tidak melakukan persetubuhan baru-baru ini."
\par 5 Daud menjawab, "Tentu saja mereka tidak melakukannya. Selalu jika kami bertugas, anak buahku menjaga agar mereka bersih menurut agama, padahal biasanya itu hanya untuk tugas-tugas biasa; apalagi tugas yang khusus seperti sekarang ini!"
\par 6 Lalu imam itu memberikan kepada Daud roti persembahan itu, sebab yang ada padanya hanyalah roti yang dipersembahkan kepada Allah, dan telah diangkat dari mezbah untuk digantikan dengan roti baru.
\par 7 (Kebetulan, Doeg orang Edom, kepala gembala ternak Saul ada di situ pada hari itu untuk memenuhi suatu kewajiban agama.)
\par 8 Kemudian bertanyalah Daud kepada Ahimelekh, "Apakah Bapak menyimpan tombak atau pedang yang dapat kupakai? Perintah baginda itu sangat mendadak sehingga aku berangkat dengan tergesa-gesa sekali dan tak sempat lagi mengambil pedang ataupun senjata yang lain."
\par 9 Lalu Ahimelekh menjawab, "Aku menyimpan pedang Goliat, orang Filistin yang kaubunuh di Lembah Ela itu; pedang itu ada di belakang efod, terbungkus dalam kain. Jika kau mau, ambil saja, sebab hanya itulah senjata yang ada di sini." Kata Daud, "Pedang itu pedang yang terbaik di seluruh negeri, berikanlah itu kepadaku!"
\par 10 Pada hari itu juga Daud melarikan diri dari Saul, lalu sampai kepada Akhis raja kota Gat.
\par 11 Kepada Akhis diberitahu oleh para pegawainya demikian, "Bukankah dia ini Daud, raja di negerinya? Dan bukankah untuk dialah para wanita waktu itu menari-nari sambil bernyanyi begini, 'Saul membunuh beribu-ribu musuh, tetapi Daud berpuluh-puluh ribu?'"
\par 12 Kata-kata itu mengganggu pikiran Daud, dan ia menjadi takut sekali kepada Raja Akhis.
\par 13 Sebab itu, di depan umum Daud berpura-pura gila. Ketika mereka mencoba menahannya, ia seolah-olah kurang waras; ia menggaruk-garuk pintu kota dan membiarkan air liurnya meleleh ke janggutnya.
\par 14 Lalu Akhis berkata kepada para pegawainya, "Apakah kamu tidak melihat bahwa orang itu gila? Mengapa kamu bawa kemari?
\par 15 Masih kurangkah orang tak waras di sekitarku, sehingga kamu bawa seorang gila lagi ke rumahku ini?"

\chapter{22}

\par 1 Kemudian Daud meninggalkan kota Gat dan bersembunyi di gua dekat kota Adulam. Ketika abang-abangnya dan seluruh keluarganya mendengar bahwa ia ada di situ, datanglah mereka kepadanya.
\par 2 Juga orang-orang yang tertindas, yang mempunyai hutang, dan yang merasa tidak puas, semuanya bergabung dengan Daud, dan Daud menjadi pemimpin mereka. Jumlah pengikut Daud ada kira-kira 400 orang.
\par 3 Dari gua itu Daud pergi ke Mizpa di Moab dan berkata kepada raja negeri Moab, "Izinkanlah ayah dan ibu hamba tinggal pada Tuanku, sampai hamba tahu apa maksud Allah terhadap hamba."
\par 4 Kemudian orang tuanya itu dibawanya kepada raja negeri Moab, dan mereka tetap di sana selama Daud bersembunyi di gua.
\par 5 Pada suatu hari seorang nabi yang bernama Gad datang kepada Daud dan berkata, "Jangan tinggal di gua ini, pergilah dengan segera ke tanah Yehuda." Lalu pergilah Daud dan masuk ke hutan Keret.
\par 6 Pada suatu hari Saul sedang duduk di bawah pohon tamariska di atas bukit di Gibea; ia memegang tombaknya sedang semua perwiranya berdiri di sekelilingnya. Ketika ia mendengar bahwa tempat Daud dan anak buahnya telah diketahui,
\par 7 ia berkata kepada para perwiranya itu, "Dengarlah, hai orang-orang Benyamin! Apakah si Daud itu akan memberikan kepadamu ladang dan kebun anggur? Apakah kamu semua akan diangkat menjadi perwira dalam tentaranya?
\par 8 Itukah sebabnya kamu bersepakat melawan aku dan tidak seorang pun memberitahukan kepadaku bahwa anakku sendiri telah memihak kepada si Daud itu? Tak ada seorang pun dari kamu memikirkan diriku atau memberitahukan kepadaku bahwa Daud, anak buahku sendiri, pada saat ini sedang menunggu kesempatan untuk membunuhku, dan bahwa anakku Yonatan telah memberi dorongan kepadanya!"
\par 9 Doeg kepala gembala ternak Saul, juga ada di situ dengan para perwira Saul, lalu Doeg berkata, "Hamba telah melihat Daud mendatangi Ahimelekh anak Ahitub di Nob.
\par 10 Ahimelekh minta petunjuk kepada TUHAN mengenai apa yang harus dilakukan Daud. Selain itu Daud dibekali makanan juga serta diberi kepadanya pedang Goliat, orang Filistin itu."
\par 11 Lalu raja menyuruh memanggil Imam Ahimelekh dan seluruh sanak saudaranya yang menjadi imam di Nob; maka menghadaplah mereka kepadanya.
\par 12 Kata Saul kepada Ahimelekh, "Dengarlah, hai anak Ahitub!" Jawabnya, "Ya, Baginda."
\par 13 Tanya Saul kepadanya, "Mengapa engkau bersepakat dengan Daud melawan aku? Mengapa kaubekali dia dengan roti dan kauberikan pedang serta kaumintakan petunjuk dari Allah? Jadi sekarang ia melawan aku dan sedang menunggu kesempatan untuk membunuh aku."
\par 14 Lalu Ahimelekh menjawab, "Tetapi bukankah Daud perwira Baginda yang paling setia? Dia adalah menantu Baginda sendiri, kepala pengawal pribadi Baginda, dan sangat dihormati oleh semua orang dalam istana Baginda.
\par 15 Memang, hamba telah meminta petunjuk dari Allah untuk dia, tetapi bukan untuk pertama kalinya hamba melakukan itu. Mengenai persepakatan melawan Baginda, janganlah Baginda menuduh hamba atau seorang pun dalam keluarga hamba. Hamba tidak tahu apa-apa tentang perkara itu!"
\par 16 Tetapi raja berkata, "Ahimelekh! Engkau dan seluruh sanak saudaramu mesti mati."
\par 17 Lalu ia memerintahkan kepada para pengawal yang berdiri di dekatnya, "Ayo, bunuhlah imam-imam TUHAN itu! Mereka telah bersekongkol dengan Daud. Mereka tahu bahwa Daud melarikan diri, tetapi mereka tidak mau memberitahukannya kepadaku." Tetapi para pengawal itu tidak mau membunuh imam-imam TUHAN itu.
\par 18 Sebab itu berkatalah Saul kepada Doeg, "Ayo lekas! Engkau saja yang membunuh mereka!" Maka majulah Doeg dan dibunuhnya imam-imam itu. Pada hari itu ia menewaskan delapan puluh lima orang imam yang berhak memakai baju efod.
\par 19 Saul juga memerintahkan untuk membunuh seluruh penduduk Nob, kota imam itu; laki-laki, perempuan, anak-anak dan bayi, juga sapi, keledai dan domba, semuanya dihabisi nyawanya.
\par 20 Hanya Abyatar, seorang dari anak-anak Ahimelekh berhasil luput. Ia melarikan diri kepada Daud,
\par 21 lalu memberitahukan bahwa Saul telah membunuh para imam TUHAN.
\par 22 Maka berkatalah Daud kepadanya, "Ketika kulihat Doeg di sana pada hari itu, tahulah aku dengan pasti bahwa ia akan mengadu kepada Saul. Jadi akulah yang bertanggung jawab atas kematian seluruh sanak saudaramu.
\par 23 Tinggallah bersamaku dan jangan takut. Saul ingin membunuhmu, tetapi ia ingin membunuhku juga. Jadi, engkau akan aman di tempatku."

\chapter{23}

\par 1 Pada suatu hari diberitahukan kepada Daud bahwa orang Filistin menyerang kota Kehila dan merampas gandum yang baru dipotong.
\par 2 Lalu bertanyalah ia kepada TUHAN, "Haruskah aku pergi menyerang orang Filistin?" "Ya," jawab TUHAN, "seranglah mereka dan selamatkanlah penduduk Kehila."
\par 3 Tetapi anak buah Daud berkata kepadanya, "Sedangkan di sini, di Yehuda, kita sudah sangat ketakutan; apalagi kalau kita harus pergi menyerang tentara Filistin di Kehila."
\par 4 Sebab itu sekali lagi Daud meminta petunjuk dari TUHAN, dan TUHAN menjawab, "Jangan khawatir. Seranglah kota Kehila, karena tentara Filistin itu akan Kuserahkan kepadamu."
\par 5 Maka pergilah Daud dan anak buahnya ke Kehila lalu diserangnya serta dibunuhnya banyak orang Filistin, dan dirampasnya ternak mereka. Demikianlah Daud membebaskan penduduk kota itu.
\par 6 Ketika Abyatar anak Ahimelekh melarikan diri kepada Daud, dan menyertai Daud ke Kehila, ia juga membawa efod.
\par 7 Ketika diberitahukan kepada Saul bahwa Daud ada di Kehila, pikir Saul, "Daud telah diserahkan Allah kepadaku. Ia terjebak karena telah masuk ke dalam kota yang bertembok dan berpintu gerbang."
\par 8 Lalu Saul mengerahkan tentaranya dan berangkat ke Kehila untuk mengepung Daud bersama semua anak buahnya.
\par 9 Tetapi Daud sudah tahu bahwa Saul berniat jahat terhadapnya. Sebab itu ia berkata kepada Imam Abyatar, "Bawalah efod itu ke mari."
\par 10 Lalu berdoalah Daud, "TUHAN, Allah Israel, aku mendengar bahwa Saul berniat untuk datang ke Kehila dan menghancurkan kota itu karena aku, hamba-Mu.
\par 11 Benarkah kabar yang kudengar itu? TUHAN, Allah Israel, sudilah beritahukan kepadaku." TUHAN menjawab, "Ya, Saul akan datang." Lalu tanya Daud lagi, "Apakah warga kota Kehila akan menyerahkan aku dan anak buahku kepada Saul?" Maka TUHAN menjawab, "Ya."
\par 12 [23:11]
\par 13 Lalu Daud serta anak buahnya kira-kira 600 orang, segera meninggalkan Kehila dan mengembara dari satu tempat ke tempat yang lain. Ketika Saul mendengar bahwa Daud telah pergi dari Kehila, ia membatalkan rencana penyerangannya.
\par 14 Setelah itu Daud tinggal di gua-gua pegunungan di padang gurun Zif. Ia selalu dikejar-kejar oleh Saul, tetapi TUHAN tidak menyerahkannya kepada raja itu.
\par 15 Meskipun demikian Daud takut juga, karena Saul berniat hendak membunuhnya. Pada suatu waktu Daud ada di Hores, di padang gurun Zif.
\par 16 Yonatan pergi menemuinya ke situ untuk menguatkan kepercayaannya bahwa ia akan dilindungi Allah.
\par 17 Kata Yonatan kepadanya, "Jangan takut, engkau tidak akan jatuh ke tangan ayahku. Beliau tahu benar bahwa engkaulah yang akan menjadi raja Israel dan bahwa aku akan mendapat kedudukan di bawahmu."
\par 18 Kemudian Daud dan Yonatan mengikat perjanjian persahabatan di hadapan TUHAN. Sesudah itu Yonatan pulang ke rumahnya, sedangkan Daud tetap tinggal di Hores.
\par 19 Kemudian beberapa orang dari Zif menghadap Saul di Gibea dan berkata, "Daud bersembunyi di daerah kami dalam gua-gua dekat Hores, di atas Bukit Hakhila sebelah selatan padang gurun Yehuda.
\par 20 Kami tahu, bahwa Baginda ingin sekali menangkap dia; sebab itu, datanglah ke daerah kami. Kami menjamin bahwa Baginda pasti dapat menyergapnya."
\par 21 Saul menjawab, "Semoga kamu diberkati TUHAN karena berbuat baik kepadaku!
\par 22 Pergilah dan pastikan lagi; periksalah dengan sungguh-sungguh di mana Daud berada dan siapa yang telah melihatnya di sana. Aku mendengar bahwa ia sangat cerdik.
\par 23 Jadi periksalah dengan teliti gua-gua yang pernah menjadi tempat persembunyiannya, dan kembalilah ke mari untuk melaporkan kepadaku. Lalu aku akan pergi bersama-sama dengan kamu, dan jika ia masih ada di wilayah itu, pasti akan kucari, walaupun harus kuobrak-abrik seluruh Yehuda!"
\par 24 Maka berangkatlah orang-orang itu kembali ke Zif mendahului Saul. Pada waktu itu Daud dan anak buahnya ada di padang gurun Maon, di lembah yang sunyi di daerah selatan padang gurun Yehuda.
\par 25 Saul dan tentaranya datang hendak mencari Daud, tetapi orang memberitahukan hal itu kepada Daud, sehingga ia pergi ke gunung batu di padang gurun Maon dan tinggal di situ. Setelah Saul mengetahui hal itu, ia segera mengejar Daud.
\par 26 Saul dan tentaranya berjalan di pinggir gunung sebelah sini sedangkan Daud dan anak buahnya di pinggir sebelah sana. Daud cepat-cepat meloloskan diri dari Saul yang sudah mulai mengepung hendak menangkap mereka.
\par 27 Tetapi tiba-tiba datanglah seorang utusan menghadap Saul dan berkata, "Hendaknya Baginda segera kembali! Orang Filistin telah menyerbu negeri kita!"
\par 28 Lalu Saul menghentikan pengejaran terhadap Daud, dan pergi untuk berperang melawan orang Filistin. Itulah sebabnya tempat itu disebut Gunung Pemisahan.

\chapter{24}

\par 1 Dari situ Daud pergi ke wilayah En-Gedi, dan bersembunyi di gua-gua.
\par 2 Ketika Saul baru saja pulang dari peperangan melawan orang Filistin, ia menerima kabar bahwa Daud ada di padang gurun dekat En-Gedi.
\par 3 Segera Saul memilih tiga ribu prajurit terbaik dari seluruh Israel, lalu berangkat mencari Daud dan anak buahnya di sebelah timur Gunung Batu Kambing Hutan.
\par 4 Maka sampailah Saul di dekat beberapa kandang domba di tepi jalan. Di situ ada pula sebuah gua, dan Saul masuk untuk buang hajat. Kebetulan sekali Daud dan anak buahnya sedang bersembunyi di gua itu juga, lebih ke dalam lagi.
\par 5 Lalu berkatalah anak buah Daud kepadanya, "Inilah kesempatan bagi Bapak! Bukankah TUHAN telah berkata kepada Bapak demikian, 'Musuhmu akan Kuserahkan kepadamu, perbuatlah sekehendakmu atas dia!'" Kemudian Daud menyelinap ke tempat Saul dan dengan diam-diam dipotongnya sedikit pinggir jubah Saul.
\par 6 Tetapi setelah itu, Daud merasa bersalah karena ia telah melakukan hal itu.
\par 7 Ia berkata kepada anak buahnya, "Semoga TUHAN menjaga jangan sampai aku berbuat jahat terhadap rajaku yang telah dipilih TUHAN. Sedikit pun tak boleh aku menyakitinya, karena dia raja pilihan TUHAN!"
\par 8 Demikianlah Daud menenangkan anak buahnya, dan tidak mengizinkan mereka menyerang Saul. Saul bangkit dan meninggalkan gua itu, lalu meneruskan perjalanannya.
\par 9 Daud pun keluar dari gua itu dan berseru di belakang Saul, "Baginda!"
\par 10 Saul menoleh ke belakang dan Daud sujud dengan hormat lalu berkata, "Mengapa Baginda mendengarkan omongan orang yang mengatakan bahwa hamba hendak mencelakakan Baginda?
\par 11 Sekarang Baginda mengalami sendiri bahwa di dalam gua tadi, TUHAN telah menyerahkan Baginda kepada hamba. Hamba telah dibujuk oleh seseorang untuk membunuh Baginda, tetapi hamba menolak. Hamba menjawab bahwa hamba tidak mau menyakiti raja hamba yang dipilih TUHAN.
\par 12 Lihatlah, Ayahku! Lihatlah potongan jubah Baginda yang hamba pegang ini! Sebenarnya hamba dapat membunuh Baginda, tetapi hamba hanya memotong sedikit pinggir jubah Baginda. Jadi, nyatalah bahwa hamba tidak bermaksud jahat ataupun durhaka. Baginda mengejar-ngejar hamba untuk membunuh hamba, meskipun hamba tidak bersalah sedikit pun terhadap Baginda.
\par 13 Semoga Tuhanlah yang memutuskan siapa di antara kita yang bersalah! Tuhanlah kiranya yang membalas perbuatan Baginda terhadap hamba, namun sedikit pun hamba tidak akan menyakiti Baginda.
\par 14 Baginda tahu peribahasa kuno ini, 'kejahatan hanya dilakukan oleh orang yang jahat.' Sebab itu hamba tidak akan menyakiti Baginda.
\par 15 Perhatikanlah hamba ini yang hendak dibunuh oleh raja Israel! Lihatlah hamba ini yang dikejar-kejarnya! Hamba hanya seekor anjing mati, seekor kutu saja.
\par 16 TUHAN akan menjadi hakim antara Baginda dan hamba. Semoga Dia memeriksa perkara ini, dan membela hamba serta menyelamatkan hamba dari Baginda."
\par 17 Setelah Daud selesai mengucapkan kata-kata itu, berkatalah Saul, "Engkaukah itu, hai Daud, anakku?" Lalu menangislah ia.
\par 18 Maka katanya kepada Daud, "Engkau yang benar, dan akulah yang salah.
\par 19 Kejahatanku terhadapmu telah kaubalas dengan kebaikan. Pada hari ini engkau telah menunjukkan bahwa engkau bermaksud baik terhadapku, sebab engkau tidak membunuhku, meskipun TUHAN telah menyerahkan aku kepadamu.
\par 20 Jarang sekali seseorang bertemu dengan musuhnya tetapi dibiarkannya musuhnya itu hidup! Jadi, semoga TUHAN memberkati engkau oleh karena perbuatanmu yang baik terhadapku pada hari ini!
\par 21 Sekarang aku yakin bahwa engkau akan menjadi raja atas Israel dan bahwa kerajaan Israel akan tetap kokoh di bawah pemerintahanmu.
\par 22 Sekarang berjanjilah kepadaku demi TUHAN, bahwa engkau tidak akan membunuh keturunanku, dan juga tidak akan menghapus nama keluargaku!"
\par 23 Lalu bersumpahlah Daud akan melaksanakan hal itu. Kemudian pulanglah Saul ke rumahnya, dan Daud serta anak buahnya pun kembali ke tempat persembunyiannya.

\chapter{25}

\par 1 Tak lama kemudian Samuel meninggal. Seluruh Israel berkumpul untuk mengikuti upacara berkabung. Kemudian ia dikuburkan di rumahnya di Rama. Setelah itu Daud pergi ke padang gurun Paran.
\par 2 Di kota Maon ada seorang laki-laki dari kaum Kaleb, namanya Nabal. Ia mempunyai tanah peternakan di dekat kota Karmel. Ia sangat kaya dan memiliki 3.000 ekor domba dan 1.000 ekor kambing. Abigail istrinya, cerdas dan cantik, tetapi Nabal bertabiat kasar dan berkelakuan buruk. Pada suatu hari Nabal sedang menggunting bulu dombanya di Karmel.
\par 3 [25:2]
\par 4 Ketika Daud yang ada di padang gurun, mendengar hal itu,
\par 5 ia menyuruh sepuluh pemuda pergi ke Karmel,
\par 6 menemui Nabal untuk menyampaikan salam Daud kepadanya. Ia menyuruh mereka berkata begini kepada Nabal, "Salam sejahtera kepadamu dan keluargamu serta segala milikmu.
\par 7 Aku mendengar bahwa engkau sedang menggunting bulu dombamu. Ketahuilah bahwa gembala-gembalamu yang ada di daerah kami, tidak pernah kami ganggu, dan juga belum pernah mereka kehilangan apa pun selama ada di Karmel.
\par 8 Coba tanyakan saja kepada mereka, pasti mereka akan membenarkan hal itu. Sebab itu, kuharap agar anak buahku diterima dengan baik di rumahmu, karena kami datang pada hari besar ini. Berikanlah juga kepada hamba-hambamu ini dan kepadaku, Daud, teman baikmu, apa saja menurut kerelaanmu."
\par 9 Setibanya di Karmel, anak buah Daud menyampaikan pesan itu kepada Nabal atas nama Daud, lalu mereka menunggu.
\par 10 Tetapi Nabal menjawab, "Daud? Siapa dia? Siapa itu anak Isai? Belum pernah aku mendengar tentang dia! Akhir-akhir ini banyak hamba yang lari dari majikannya di negeri ini!
\par 11 Mana bisa aku mengambil roti dan air minumku, serta daging yang telah kusediakan bagi para pengguntingku, lalu kuberikan semua itu kepada orang yang tidak jelas asal usulnya!"
\par 12 Maka anak buah Daud itu kembali kepada Daud dan memberitahukan kepadanya kata-kata Nabal itu.
\par 13 "Ikatlah pedangmu di pinggang masing-masing!" perintah Daud. Segera anak buahnya mentaati perintah itu. Daud juga mengikat pedangnya di pinggangnya dan berangkat dengan kira-kira 400 orang, sedang 200 orang tinggal untuk menjaga barang-barang.
\par 14 Sementara itu kepada Abigail istri Nabal, sudah diberitahukan oleh seorang pelayan Nabal, demikian, "Sudah dengarkah Nyonya? Daud telah menyuruh beberapa utusan dari padang gurun untuk memberi salam kepada tuan kita, tetapi beliau mencaci maki mereka.
\par 15 Padahal mereka sangat baik kepada kami. Tidak pernah kami diganggu, dan juga belum pernah kami kehilangan apa pun ketika kami berteman dengan mereka di padang gurun.
\par 16 Siang malam mereka melindungi kami selama kami menggembalakan kambing domba di dekat mereka.
\par 17 Sebab itu, pertimbangkanlah hal ini, dan putuskanlah apa yang dapat Nyonya perbuat. Sebab pastilah akan datang pembalasan terhadap tuan kita dan seluruh keluarganya. Tuan kita begitu keras kepala sehingga beliau tak dapat diajak bicara!"
\par 18 Lalu segera Abigail mengambil dua ratus roti, dan dua kantong kulit penuh berisi air anggur, lima domba panggang, tujuh belas kilogram gandum panggang, seratus rangkai buah anggur kering dan dua ratus kue ara. Semuanya itu dimuat di atas keledai-keledainya.
\par 19 Lalu berkatalah ia kepada pelayan-pelayannya, "Kamu berjalan mendahuluiku; aku akan mengikutimu." Tetapi ia tidak mengatakan apa-apa kepada suaminya.
\par 20 Abigail menunggang keledainya, dan ketika ia sampai pada sebuah belokan di pinggir gunung, berpapasanlah ia dengan Daud dan anak buahnya yang sedang turun menuju ke arahnya.
\par 21 Pada saat itu Daud sedang berpikir, "Percuma saja aku menjaga milik orang itu di padang gurun! Tak ada sedikit pun yang hilang dari miliknya; dan sekarang dibalasnya kebaikanku dengan kejahatan!
\par 22 Kiranya Allah membunuh aku apabila tidak kubunuh semua orang laki-laki yang ada di sana sebelum matahari terbit!"
\par 23 Ketika Abigail melihat Daud, ia segera turun dari keledainya lalu sujud pada kaki Daud,
\par 24 serta berkata kepadanya, "Tuanku, sudilah mendengarkan! Biarlah kesalahan itu hamba tanggung sendiri!
\par 25 Janganlah Tuanku pikirkan Nabal, orang yang tanpa guna itu! Ia orang bodoh--sama dengan arti namanya. Hamba tidak melihat utusan-utusan yang Tuanku suruh itu.
\par 26 Allah telah mencegah Tuanku menuntut balas dan membunuh musuh-musuh Tuan. Dan sekarang Tuanku, demi TUHAN yang hidup, hamba bersumpah, kiranya nasib segala musuh Tuanku dan setiap orang yang ingin mencelakakan Tuanku sama seperti nasib Nabal.
\par 27 Terimalah bingkisan yang hamba bawa ini untuk Tuanku, dan anak buah Tuanku.
\par 28 Maafkanlah kelalaian hamba. TUHAN pasti akan menjadikan Tuanku serta keturunan Tuanku raja atas Israel, sebab Tuanku berperang untuk TUHAN; dan Tuanku tidak akan melakukan kejahatan seumur hidup.
\par 29 Seandainya seorang menyerang Tuanku dan mencoba membunuh Tuanku, maka Tuanku pasti akan dilindungi TUHAN Allah seperti harta yang dijaga oleh pemiliknya. Tetapi musuh Tuanku akan dilemparkan jauh-jauh oleh TUHAN, dilemparkan oleh pengumban.
\par 30 TUHAN akan melakukan segala kebaikan yang dijanjikan-Nya kepada Tuanku, dan tak lama lagi Tuanku dijadikan-Nya raja atas Israel.
\par 31 Pada waktu itu Tuanku akan mengingat bahwa Tuanku tidak membunuh tanpa alasan atau main hakim sendiri sehingga Tuanku tak perlu merasa sedih atau menyesal. Dan apabila TUHAN telah memberkati Tuanku, janganlah melupakan hamba Tuanku ini."
\par 32 Lalu berkatalah Daud kepadanya, "Terpujilah TUHAN, Allah Israel, yang mengutus engkau menemuiku pada hari ini!
\par 33 Terpujilah kebijaksanaanmu dan terpujilah juga engkau sendiri karena telah menahan aku melakukan pembunuhan serta main hakim sendiri.
\par 34 TUHAN telah menahan aku berbuat jahat terhadapmu. Demi TUHAN, Allah Israel yang hidup, seandainya engkau tidak segera menemuiku, pastilah seluruh anak buah Nabal sudah mati pada waktu matahari terbit!"
\par 35 Setelah itu Daud menerima apa yang dibawa oleh Abigail untuknya, lalu berkata kepadanya, "Pulanglah ke rumahmu dan jangan khawatir. Permintaanmu kukabulkan."
\par 36 Ketika Abigail kembali kepada Nabal, ternyata ada pesta besar di rumahnya, seperti pesta raja. Dan karena Nabal mabuk dan bersukaria, kepadanya tidak diberitahu apa pun oleh istrinya sampai besoknya.
\par 37 Paginya, setelah Nabal tidak mabuk lagi, istrinya memberitahukan kepadanya tentang segala kejahatan itu. Lalu Nabal mendapat serangan jantung dan seluruh badannya menjadi lumpuh.
\par 38 Kira-kira sepuluh hari kemudian Nabal mendapat serangan jantung lagi dari TUHAN sehingga ia meninggal.
\par 39 Ketika Daud mendengar bahwa Nabal sudah mati, berkatalah ia, "Terpujilah TUHAN! Ia telah membalas penghinaan yang dilakukan Nabal terhadap aku, dan telah mencegah hamba-Nya ini melakukan kejahatan. TUHAN telah menghukum Nabal karena kejahatannya." Kemudian Daud mengirim utusan kepada Abigail untuk melamarnya menjadi istrinya.
\par 40 Maka sampailah utusan Daud kepada wanita itu di Karmel dan berkata kepadanya, "Kami diutus oleh Daud menghadap Nyonya, sebab ia ingin melamar Nyonya menjadi istrinya."
\par 41 Abigail pun bangkit dan sujud serta berkata, "Aku ini hamba Daud, aku bersedia membasuh kaki para hambanya."
\par 42 Kemudian berkemas-kemaslah ia dengan segera. Ia naik ke atas keledainya dan diiringi oleh lima orang pelayan wanitanya, berangkatlah ia mengikuti utusan Daud, maka ia menjadi istri Daud.
\par 43 Sebelum itu Daud telah menikah dengan Ahinoam dari Yizreel, dan kini Abigail juga menjadi istrinya.
\par 44 Sementara itu Saul telah mengawinkan Mikhal putrinya, yaitu istri Daud, dengan Palti anak Lais dari kota Galim.

\chapter{26}

\par 1 Pada suatu hari orang-orang dari Zif datang kepada Saul di Gibea dan memberitahukan kepadanya bahwa Daud sedang bersembunyi di Gunung Hakhila di pinggir padang gurun Yehuda.
\par 2 Maka dengan segera Saul berangkat diiringi tiga ribu prajurit yang terbaik di Israel, menuju padang gurun Zif untuk mencari Daud.
\par 3 Saul berkemah di tepi jalan dekat Gunung Hakhila itu. Daud yang ada di padang gurun, mendengar bahwa Saul mengikutinya sampai di padang gurun.
\par 4 Lalu ia pun mengirim mata-mata, dan tahulah ia dengan pasti bahwa Saul ada di situ.
\par 5 Dengan segera Daud pergi ke perkemahan Saul dan di situ didapatinya kemah di mana Saul tidur di dekat Abner anak Ner. Kemah Saul itu ada di tengah-tengah, sedang tentaranya berkemah di sekelilingnya.
\par 6 Kemudian Daud bertanya kepada Ahimelekh orang Het itu, dan kepada Abisai saudara Yoab (Ibu mereka adalah Zeruya), "Siapa berani masuk bersamaku ke dalam perkemahan Saul?" Abisai menjawab, "Aku!"
\par 7 Maka pada malam itu Daud dan Abisai menyelinap ke dalam perkemahan itu dan melihat Saul sedang tidur di tengah-tengah perkemahan, tombaknya tertancap di tanah dekat kepalanya. Abner dan pasukannya tidur di sekeliling Saul.
\par 8 Lalu kata Abisai kepada Daud, "Allah telah menyerahkan musuhmu kepadamu pada malam ini. Izinkanlah aku menancapkan tombaknya sendiri ke badannya tembus sampai ke tanah, kubunuh dia dengan sekali tikaman, tidak usah sampai dua kali!"
\par 9 Tetapi Daud berkata, "Jangan kauapa-apakan dia! TUHAN pasti menghukum orang yang berbuat jahat kepada raja yang dipilih-Nya.
\par 10 Demi TUHAN yang hidup, aku tahu bahwa nanti TUHAN sendiri akan membunuh Saul, jika ajalnya sudah sampai, ataupun jika ia tewas dalam pertempuran.
\par 11 Kiranya TUHAN mencegah aku membunuh raja yang dipilih-Nya! Ambil saja tombaknya dan kendinya, lalu kita pergi!"
\par 12 Maka Daud mengambil tombak dan kendi Saul itu dari sebelah kepala Saul, lalu pergi. Tidak seorang pun melihat hal itu atau mengetahui apa yang telah terjadi, ataupun terbangun, sebab TUHAN membuat mereka semuanya tertidur dengan nyenyak.
\par 13 Lalu Daud menyeberang ke pinggir lembah yang lain sampai ke puncak gunung, sehingga dia jauh dari perkemahan Saul.
\par 14 Kemudian berserulah Daud kepada Abner dan tentara Saul katanya, "Abner! Dengarkah engkau suaraku?" Abner pun berseru pula, "Siapakah engkau yang berteriak-teriak itu?"
\par 15 Jawab Daud, "Abner, bukankah engkau orang yang paling hebat di Israel? Mengapa engkau tidak menjaga tuanmu, baginda raja? Baru saja ada orang memasuki perkemahanmu untuk membunuh tuanmu.
\par 16 Engkau tidak sanggup menjalankan tugasmu, Abner! Demi TUHAN yang hidup kamu semua patut mati, karena tidak menjaga tuanmu, raja pilihan TUHAN. Coba lihat! Di manakah tombak raja? Di manakah kendi yang ada di sebelah kepalanya tadi?"
\par 17 Saul mengenal suara Daud dan berkata, "Daud, engkaukah itu, Anakku?" Jawab Daud, "Ya, Baginda."
\par 18 Lalu katanya lagi, "Mengapa Baginda masih juga mengejar-ngejar hamba? Apakah kesalahan hamba?
\par 19 Baginda, dengarkanlah sembah hamba ini. Andaikata TUHAN yang menyuruh Baginda supaya melawan hamba, suatu persembahan kurban akan dapat meredakan kemarahan-Nya terhadap hamba. Tetapi andaikata yang menyuruh Baginda itu adalah manusia, semoga mereka dikutuk TUHAN! Karena akibat dari perbuatan mereka itu, hamba telah diusir dari tanah milik TUHAN dan terpaksa harus pergi ke negeri lain di mana hamba harus beribadat kepada dewa-dewa asing.
\par 20 Janganlah biarkan hamba mati di tanah yang asing, jauh dari TUHAN. Apa gunanya raja Israel datang untuk membunuh seekor kutu seperti hamba ini? Apa gunanya ia memburu hamba seperti orang memburu ayam hutan?"
\par 21 Lalu Saul menjawab, "Aku telah berbuat salah. Kembalilah Daud, Anakku! Aku tak akan berbuat jahat lagi kepadamu karena engkau telah merelakan aku hidup malam tadi. Kelakuanku sungguh bodoh. Aku benar-benar telah keliru!"
\par 22 Tetapi Daud menjawab, "Inilah tombak Baginda, suruhlah seorang prajurit Baginda datang ke mari untuk mengambilnya.
\par 23 TUHAN akan memberkati setiap orang yang setia dan jujur. Sebab meskipun pada hari ini TUHAN menyerahkan Tuanku kepada hamba, namun hamba tidak mau berbuat jahat terhadap Baginda, raja pilihan TUHAN.
\par 24 Jadi, seperti Baginda telah hamba relakan hidup pada hari ini, demikian juga kiranya hamba dilindungi TUHAN dan dilepaskan dari segala bahaya!"
\par 25 Lalu berkatalah Saul kepada Daud, "TUHAN memberkatimu, anakku! Engkau akan berhasil dalam segala pekerjaanmu!" Lalu Daud meneruskan perjalanannya, dan Saul pun pulang ke rumahnya.

\chapter{27}

\par 1 Daud berpikir dalam hatinya, "Bagaimanapun juga pada suatu hari Saul akan membunuhku. Jadi, tak ada jalan lain lagi bagiku, kecuali melarikan diri ke negeri Filistin. Dengan demikian Saul tidak akan mencariku lagi di Israel, dan aku pun akan aman."
\par 2 Lalu Daud dan keenam ratus anak buahnya segera pergi ke Akhis anak Maokh, raja kota Gat.
\par 3 Maka menetaplah ia bersama anak buahnya di Gat, masing-masing dengan keluarga mereka. Daud membawa serta kedua istrinya, yaitu Ahinoam wanita dari Yizreel, dan Abigail, janda Nabal wanita Karmel.
\par 4 Ketika Saul mendengar bahwa Daud telah melarikan diri ke Gat, ia tidak lagi berusaha menangkapnya.
\par 5 Lalu Daud mengusulkan kepada Akhis, "Jika Baginda merasa senang dengan hamba, berikanlah kepada hamba sebuah kota kecil di pedalaman untuk tempat tinggal, sehingga hamba Baginda ini tidak menumpang pada Baginda di ibukota ini."
\par 6 Lalu Akhis memberikan kota Ziklag kepada Daud, dan itulah sebabnya Ziklag menjadi milik raja-raja Yehuda sampai sekarang.
\par 7 Daud tinggal di negeri Filistin selama satu tahun empat bulan.
\par 8 Selama masa itu Daud dan anak buahnya sering kali menyerang orang Gesur, orang Girzi, dan orang Amalek; bangsa-bangsa itu sudah lama tinggal di wilayah itu. Ia menyerbu ke tanah itu sejauh Sur, dan terus ke Mesir.
\par 9 Ia membunuh semua orang laki-laki dan perempuan serta merampas domba, sapi, keledai, unta, bahkan pakaian juga. Kemudian ia kembali dan menghadap Akhis.
\par 10 Jika Akhis bertanya kepadanya, "Ke mana kamu menyerbu kali ini?" Maka Daud mengatakan kepadanya bahwa ia telah pergi ke bagian selatan Yehuda atau ke wilayah orang Yerahmeel atau ke wilayah tempat tinggal orang Keni.
\par 11 Tidak seorang pun di daerah-daerah yang diserbunya itu dibiarkannya hidup, supaya jangan ada orang dari daerah-daerah itu yang dapat pergi ke Gat untuk melaporkan apa yang sesungguhnya telah dilakukan Daud dan anak buahnya. Begitulah tindakan Daud selama ia tinggal di negeri Filistin.
\par 12 Tetapi Akhis mempercayai Daud dan berpikir dalam hatinya, "Dia begitu dibenci oleh bangsanya sendiri, sehingga ia terpaksa menjadi hambaku seumur hidupnya."

\chapter{28}

\par 1 Beberapa waktu kemudian orang Filistin mengerahkan tentaranya untuk menyerang Israel, dan Akhis berkata kepada Daud, "Tentu engkau mengerti bahwa engkau dan anak buahmu harus berperang pada pihakku."
\par 2 Jawab Daud, "Tentu saja, nanti Baginda akan melihat sendiri apa yang dapat dilakukan hamba Baginda itu." Lalu Akhis berkata, "Baik! Engkau kuangkat menjadi pengawalku yang tetap."
\par 3 Samuel sudah meninggal dan ia diratapi oleh seluruh Israel serta dimakamkan di Rama, kotanya sendiri. Beberapa waktu yang lalu Saul telah mengusir dari Israel semua peramal dan dukun pemanggil arwah.
\par 4 Maka ketika tentara Filistin berkumpul dan berkemah di dekat kota Sunem, Saul pun mengumpulkan orang Israel, dan mereka berkemah di Gunung Gilboa.
\par 5 Ketika Saul melihat tentara Filistin itu, ia menjadi sangat takut.
\par 6 Lalu ia meminta petunjuk kepada TUHAN, tetapi TUHAN tidak menjawabnya, baik dengan mimpi, maupun dengan penggunaan Urim dan Tumim ataupun melalui para nabi.
\par 7 Sebab itu Saul berkata kepada para pengawalnya, "Carilah seorang dukun wanita pemanggil arwah, aku akan pergi kepadanya untuk meminta petunjuk." Mereka menjawab, "Di En-Dor ada dukun seperti itu."
\par 8 Lalu Saul menyamar supaya tidak dikenali orang. Dengan diiringi dua orang laki-laki, berangkatlah ia dan sampailah ia di rumah dukun itu pada malam hari. Maka kata Saul, "Panggillah arwah orang yang akan kusebut namanya, dan mintalah supaya ia meramalkan apa yang akan terjadi."
\par 9 Wanita itu menjawab, "Tentu Tuan pun tahu bahwa Raja Saul telah mengusir dari Israel semua peramal dan dukun pemanggil arwah. Mengapa Tuan mencoba mencelakakan aku sehingga aku dibunuh?"
\par 10 Lalu Saul bersumpah, katanya, "Demi TUHAN yang hidup, aku berjanji bahwa engkau tidak akan dihukum karena melakukan hal ini."
\par 11 Wanita itu bertanya, "Arwah siapakah yang harus kupanggil untuk Tuan?" Jawab Saul, "Arwah Samuel."
\par 12 Ketika wanita itu melihat Samuel, ia berteriak dan berkata kepada Saul, "Mengapa Tuan menipu hamba? Tuankulah rupanya Raja Saul!"
\par 13 Jawab raja, "Jangan takut! Katakan saja apa yang kaulihat!" Wanita itu menjawab, "Hamba melihat roh muncul dari bumi."
\par 14 Tanya Saul, "Bagaimana rupanya?" Jawabnya, "Seorang laki-laki yang tua. Ia muncul berpakaian jubah." Lalu tahulah Saul bahwa itu Samuel, dan ia sujud menghormatinya.
\par 15 Setelah itu Samuel berkata kepada Saul, "Mengapa kauganggu aku? Mengapa kaupanggil aku kembali?" Jawab Saul, "Aku sangat bingung karena diserang oleh orang Filistin, sedangkan Allah telah meninggalkanku. Aku tidak dijawabnya lagi, baik melalui para nabi, maupun dengan mimpi. Sebab itu kumohon kepada Bapak, supaya Bapak beritahukan apa yang harus kulakukan."
\par 16 Lalu kata Samuel, "Apa gunanya aku kaupanggil sedangkan TUHAN telah meninggalkanmu dan menjadi musuhmu?
\par 17 TUHAN telah melakukan kepadamu seperti yang sudah dikatakan-Nya melalui aku. TUHAN telah mengambil kerajaan Israel daripadamu dan memberikannya kepada Daud.
\par 18 Engkau telah melawan perintah TUHAN dan tidak mau membinasakan orang Amalek serta segala milik mereka. Itulah sebabnya sekarang engkau diperlakukan begini oleh TUHAN.
\par 19 Engkau dan orang-orang Israel akan diserahkan-Nya kepada orang Filistin. Besok engkau dan putra-putramu akan berkumpul dengan aku; bahkan seluruh tentara Israel pun akan diserahkan TUHAN kepada orang Filistin."
\par 20 Pada saat itu juga, Saul rebah terbujur ke tanah seperti disambar petir, sebab ia sangat terkejut mendengar kata-kata Samuel itu. Lagipula ia lemah sekali, sebab tidak makan apa-apa sehari semalam.
\par 21 Dukun itu mendekati Saul dan melihat bahwa ia sangat kebingungan: sebab itu wanita itu berkata kepadanya, "Tuanku, hamba telah berkurban untuk melakukan apa yang telah Tuan minta.
\par 22 Jadi, hendaknya Tuanku pun mengabulkan permohonan hamba; makanlah sedikit makanan yang akan hamba hidangkan supaya Tuan dapat meneruskan perjalanan Tuan."
\par 23 Saul menolak dan berkata bahwa ia tidak mau makan. Tetapi setelah pegawai-pegawainya juga mendesaknya, ia menurut. Ia bangkit dari tanah dan duduk di balai-balai.
\par 24 Dengan segera wanita itu menyembelih anak sapinya yang gemuk dan mengambil tepung serta meremas-remasnya, lalu dibuatnya menjadi roti yang tidak beragi.
\par 25 Setelah itu dihidangkannya makanan itu kepada Saul dan pegawai-pegawainya lalu mereka makan. Kemudian pergilah mereka pada malam itu juga.

\chapter{29}

\par 1 Orang Filistin mengumpulkan seluruh tentaranya di Afek, sedang orang Israel berkemah di dekat mata air di Lembah Yizreel.
\par 2 Kelima orang raja Filistin maju berbaris dengan kesatuan-kesatuan yang masing-masing terdiri dari seratus dan seribu orang; Daud dan anak buahnya berbaris di belakang bersama-sama dengan raja Akhis.
\par 3 Para panglima orang Filistin melihat mereka lalu bertanya, "Orang-orang Ibrani itu untuk apa di sini?" Akhis menjawab, "Ini Daud, bekas pegawai Raja Saul dari Israel. Ia sudah agak lama tinggal padaku. Sejak ia datang kepadaku sampai hari ini, belum kudapati ia bersalah."
\par 4 Tetapi para panglima Filistin itu marah kepada Akhis dan berkata kepadanya, "Suruh orang itu pulang ke kota yang telah kauberikan kepadanya. Ia tidak boleh berperang bersama-sama dengan kita; jangan-jangan dia mengkhianati kita nanti di tengah-tengah pertempuran. Bukankah ini kesempatan yang paling baik baginya untuk mengambil hati rajanya dengan jalan membunuh anak buah kita?
\par 5 Dia kan Daud? Untuk dialah para wanita dulu menari-nari sambil bernyanyi begini, 'Saul telah membunuh beribu-ribu musuh, tetapi Daud berpuluh-puluh ribu.'"
\par 6 Lalu Akhis memanggil Daud dan berkata kepadanya, "Aku percaya demi TUHAN yang hidup, bahwa kau setia kepadaku. Aku senang jika kau dapat mendampingiku dalam pertempuran ini. Sebab sejak kau datang kepadaku sampai saat ini, tidak kudapati kesalahanmu. Tetapi raja-raja yang lain itu tidak suka kepadamu.
\par 7 Sebab itu pulanglah dengan selamat, dan janganlah melakukan sesuatu yang menimbulkan kemarahan mereka."
\par 8 Daud menjawab, "Kesalahan apakah yang telah hamba lakukan, Baginda? Jika seperti kata Baginda, Baginda tidak mendapati kesalahan apa pun pada hamba sejak hamba mulai melayani Baginda sampai saat ini, mengapa hamba tidak diizinkan berperang melawan musuh Baginda?"
\par 9 Akhis menjawab, "Engkau tahu bahwa engkau kuanggap setia seperti malaikat Allah. Tetapi para panglima itu telah memutuskan bahwa kau tidak boleh ikut berperang bersama-sama dengan kami.
\par 10 Sebab itu Daud, besok kamu semua yang telah meninggalkan Saul dan datang kepadaku, harus bangun pagi-pagi dan berangkat segera setelah matahari terbit."
\par 11 Karena itu keesokan harinya pagi-pagi, berangkatlah Daud dan anak buahnya pulang ke negeri Filistin, sedang tentara Filistin itu berangkat ke Yizreel.

\chapter{30}

\par 1 Dua hari kemudian, Daud dan anak buahnya kembali ke kota Ziklag. Ketika tiba di sana, mereka mendapati bahwa orang Amalek telah menyerbu bagian selatan negeri Yehuda dan menyerang Ziklag. Mereka tidak membunuh seorang pun. Tetapi mereka membakar habis kota itu dan meneruskan perjalanan mereka dengan membawa serta para wanita, anak-anak dan penduduk lainnya. Mereka juga telah menawan keluarga Daud dan keluarga anak buahnya, termasuk kedua istri Daud, Ahinoam dan Abigail. Daud dan anak buahnya begitu sedih, sehingga mereka mulai menangis tak henti-hentinya sampai kepayahan.
\par 2 [30:1]
\par 3 [30:1]
\par 4 [30:1]
\par 5 [30:1]
\par 6 Anak buah Daud sangat susah karena telah kehilangan anak istri mereka. Sebab itu mereka hendak melempari Daud dengan batu. Jadi Daud berada dalam kesulitan besar, tetapi hatinya dikuatkan lagi oleh TUHAN Allahnya.
\par 7 Lalu Daud berkata kepada Imam Abyatar anak Ahimelekh, "Bawalah efod itu ke mari!" Maka Abyatar membawa efod itu kepadanya.
\par 8 Kemudian bertanyalah Daud kepada TUHAN, "Haruskah kukejar gerombolan penyerbu itu? Dapatkah aku menyusul mereka?" Jawab TUHAN, "Ya, kejarlah; engkau akan menyusul mereka dan membebaskan para tawanan."
\par 9 Lalu berangkatlah Daud bersama-sama dengan keenam ratus anak buahnya. Ketika mereka sampai di Sungai Besor, Daud meneruskan perjalanannya bersama-sama dengan 400 orang; sedang yang 200 orang lainnya tinggal di situ. Karena sudah terlalu lelah dan tak sanggup menyeberangi Sungai Besor.
\par 10 [30:9]
\par 11 Dalam perjalanan itu anak buah Daud bertemu dengan seorang anak laki-laki Mesir di padang, lalu ia dibawa kepada Daud. Ia diberi makan dan minum,
\par 12 juga kue buah ara, dan dua rangkai anggur kering. Setelah ia makan, kekuatannya pulih kembali; ternyata sudah tiga hari ia tidak makan dan tidak minum!
\par 13 Kemudian Daud bertanya kepadanya, "Siapa tuanmu, dan dari mana engkau?" Jawab anak itu, "Aku orang Mesir, budak seorang Amalek. Aku ditinggalkan tuanku tiga hari yang lalu sebab aku sakit.
\par 14 Kami telah menyerbu wilayah orang Kreti di bagian selatan Yehuda dan wilayah marga Kaleb, dan kami telah membakar habis kota Ziklag."
\par 15 Lalu tanya Daud kepadanya, "Maukah engkau mengantarkan aku ke tempat gerombolan itu?" Jawabnya, "Aku mau, asal Tuan berjanji kepadaku demi nama Allah, bahwa aku tidak Tuan bunuh atau Tuan serahkan kepada tuanku."
\par 16 Lalu dia mengantarkan mereka ke tempat gerombolan itu. Pada waktu itu gerombolan itu berpencar-pencar di seluruh tempat itu, sambil makan, minum dan berpesta pora karena banyaknya barang rampasan yang telah mereka bawa dari negeri Filistin dan Yehuda.
\par 17 Pada waktu fajar menyingsing keesokan harinya, mereka diserang oleh Daud, dan pertempuran berlangsung sampai malam. Selain dari empat ratus orang muda yang melarikan diri dengan menunggang unta, tak seorang pun dari musuh yang dapat lolos.
\par 18 Daud berhasil membebaskan segalanya yang telah dirampas oleh orang Amalek, termasuk kedua istrinya;
\par 19 tidak ada sesuatu pun yang hilang. Semua anak laki-laki dan perempuan serta segala barang rampasan yang telah diambil oleh orang Amalek, diambil kembali oleh Daud.
\par 20 Di samping itu Daud juga merampas semua domba dan ternak. Binatang-binatang itu digiring oleh anak buah Daud di depan, terpisah dari segala barang rampasan yang lain, dan mereka berkata, "Ini bagian untuk Daud!"
\par 21 Setelah itu Daud kembali kepada kedua ratus orang yang terlalu lelah untuk mengikuti dia, dan yang telah ditinggalkannya di pinggir Sungai Besor. Mereka menyongsong Daud serta anak buahnya, dan Daud mendekati serta menyalami mereka.
\par 22 Tetapi di antara anak buah Daud yang bersama-sama mengikuti dia, ada yang jahat dan tamak. Mereka berkata, "Orang-orang ini tidak ikut; jadi mereka tak berhak mendapat apa-apa dari barang rampasan itu. Mereka hanya boleh mengambil anak istri mereka, dan pergi."
\par 23 Tetapi Daud menjawab, "Saudara-saudaraku, janganlah kamu berbuat demikian dengan apa yang telah diberikan TUHAN kepada kita! Bukankah kita telah dilindunginya dan diberinya kemenangan atas gerombolan perampok itu?
\par 24 Jadi seorang pun tidak boleh menyetujui usul itu! Semuanya harus mendapat bagian yang sama banyaknya: orang yang tinggal untuk menjaga barang-barang harus mendapat bagian yang sama seperti orang yang maju berperang."
\par 25 Kemudian hal itu dibuat menjadi sebuah peraturan oleh Daud, dan ditaati di Israel sampai sekarang.
\par 26 Ketika Daud kembali ke Ziklag, ia mengirim sebagian dari barang-barang rampasan itu kepada para pemimpin kota-kota Yehuda yang telah mendukungnya, katanya, "Inilah hadiah untukmu dari barang rampasan yang kami ambil dari musuh TUHAN."
\par 27 Kota-kota itu ialah: Betel dan Rama di bagian selatan Yehuda, Yatir, Aroer, Sifmot, Estemoa, Rakhal, juga kota-kota orang Yerahmeel, kota-kota orang Keni, Horma, Bor-Asan, Atakh dan Hebron. Semua kota di mana Daud dan anak buahnya pernah mengembara, dikirimnya hadiah itu.

\chapter{31}

\par 1 Sementara itu orang Filistin bertempur melawan orang Israel di pegunungan Gilboa. Banyak orang Israel tewas di situ, dan yang lainnya, termasuk Saul dan putra-putranya melarikan diri.
\par 2 Tetapi mereka disusul oleh orang Filistin dan tiga orang di antara putra-putra Saul itu mati dibunuh; yaitu Yonatan, Abinadab, dan Malkisua.
\par 3 Pertempuran amat sengit di sekitar Saul, dan ia sendiri kena panah-panah musuh sehingga luka parah.
\par 4 Lalu kata Saul kepada pemuda yang membawa senjatanya, "Cabutlah pedangmu dan tikamlah aku, supaya aku jangan dipermain-mainkan dan dibunuh oleh orang-orang yang tak mengenal TUHAN itu." Tetapi pemuda itu tidak mau menikamnya, karena ia sangat menghormatinya. Sebab itu Saul mengambil pedangnya sendiri dan merebahkan dirinya ke atas mata pedang itu.
\par 5 Ketika pemuda itu melihat bahwa Saul sudah wafat, ia pun merebahkan dirinya ke atas mata pedangnya lalu tewas di samping Saul.
\par 6 Demikianlah Saul bersama ketiga putranya, dan semua anak buahnya serta pemuda itu tewas pada satu hari itu.
\par 7 Ketika orang-orang Israel yang tinggal di seberang Lembah Yizreel dan sebelah timur Sungai Yordan mendengar bahwa tentara Israel telah melarikan diri, dan bahwa Saul serta putra-putranya sudah tewas, mereka lari meninggalkan kota-kota mereka. Kemudian orang Filistin menduduki kota-kota itu.
\par 8 Besoknya ketika orang Filistin datang untuk merampoki mayat-mayat, mereka menemukan Saul dan ketiga putranya sudah tewas di pegunungan Gilboa.
\par 9 Mereka memenggal kepala Saul dan mengambil baju perangnya, lalu orang Filistin mengirim utusan ke seluruh negeri Filistin untuk menyampaikan kabar baik itu kepada rakyat dan dewa mereka.
\par 10 Kemudian baju perang Saul itu disimpan di dalam kuil dewi Asytoret, dan badannya dipakukan pada tembok kota Bet-Sean.
\par 11 Tetapi penduduk Yabesh di Gilead mendengar apa yang telah dilakukan orang Filistin kepada Saul,
\par 12 lalu orang-orang yang paling berani berangkat dari situ dan berjalan sepanjang malam ke Bet-Sean. Mereka menurunkan jenazah Saul dan jenazah putra-putranya dari tembok kota itu lalu membawanya ke Yabesh, dan membakarnya.
\par 13 Semua tulang-tulangnya dikumpulkan dan dikebumikan di bawah pohon tamariska di kota itu, kemudian berpuasalah mereka tujuh hari lamanya.


\end{document}