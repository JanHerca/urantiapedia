\begin{document}

\title{1 John}

1Jn 1:1  Isi surat ini mengenai Sabda yang memberi hidup, yaitu Sabda yang sudah ada sejak awal mula. Kami sudah mendengarnya, dan sudah melihatnya dengan mata kami sendiri. Kami sudah memandangnya, dan sudah memegangnya dengan tangan kami.
1Jn 1:2  Ketika hidup itu dinyatakan, kami melihatnya. Itulah sebabnya kami memberi kesaksian mengenainya. Kami memberitakan kepada kalian tentang hidup sejati dan kekal, yang ada bersama Bapa, dan yang sudah dinyatakan kepada kami.
1Jn 1:3  Apa yang sudah kami lihat dan dengar, itulah juga yang kami beritakan kepada kalian, supaya kalian bersama kami ikut menghayati hidup bersatu dengan Bapa dan dengan Anak-Nya Yesus Kristus.
1Jn 1:4  Kami menulis ini supaya hati kita sungguh-sungguh gembira.
1Jn 1:5  Inilah berita yang telah kami dengar dari Anak-Nya dan yang kami sampaikan kepada kalian: Allah itu terang, dan pada-Nya tidak ada kegelapan sama sekali.
1Jn 1:6  Kalau kita berkata bahwa kita hidup bersatu dengan Dia, padahal kita hidup dalam kegelapan, maka kita berdusta baik dengan kata-kata maupun dengan perbuatan.
1Jn 1:7  Tetapi kalau kita hidup di dalam terang sebagaimana Allah ada di dalam terang, maka kita hidup erat, rukun satu sama lain, dan darah Yesus, Anak-Nya, membersihkan kita dari segala dosa.
1Jn 1:8  Kalau kita berkata bahwa kita tidak berdosa, kita menipu diri sendiri; dan Allah tidak berada di dalam hati kita.
1Jn 1:9  Tetapi kalau kita mengakui dosa-dosa kita kepada Allah, Ia akan menepati janji-Nya dan melakukan apa yang adil. Ia akan mengampuni dosa-dosa kita dan membersihkan kita dari segala perbuatan kita yang salah.
1Jn 1:10  Tetapi kalau kita berkata bahwa kita tidak berdosa, kita beranggapan seakan-akan Allah pendusta dan kita tidak menyimpan perkataan Allah di hati kita.
1Jn 2:1  Anak-anakku! Saya menulis ini kepada kalian supaya kalian jangan berbuat dosa. Tetapi kalau ada yang berbuat dosa, maka kita mempunyai seorang pembela, yaitu Yesus Kristus yang adil itu; Ia akan memohon untuk kita di hadapan Bapa.
1Jn 2:2  Dengan perantaraan Yesus Kristus itulah dosa-dosa kita diampuni. Dan bukannya dosa-dosa kita saja, melainkan dosa seluruh umat manusia juga.
1Jn 2:3  Kalau kita taat kepada perintah-perintah Allah, itu tandanya bahwa kita mengenal Allah.
1Jn 2:4  Orang yang berkata, "Saya mengenal Allah," tetapi tidak taat kepada perintah-perintah-Nya, orang itu pendusta, dan Allah tidak berada di dalam hatinya.
1Jn 2:5  Tetapi orang yang taat kepada perkataan Allah, orang itu mengasihi Allah dengan sempurna. Itulah tandanya bahwa kita hidup bersatu dengan Allah.
1Jn 2:6  Barangsiapa berkata bahwa ia hidup bersatu dengan Allah, ia harus hidup mengikuti jejak Kristus.
1Jn 2:7  Saudara-saudara yang tercinta! Yang saya tulis ini bukanlah suatu perintah yang baru. Ini perintah lama yang sudah disampaikan kepadamu sejak kalian mulai percaya kepada Kristus; yaitu berita dari Allah yang sudah kalian dengar sebelumnya.
1Jn 2:8  Meskipun begitu, perintah yang saya tulis ini baru juga; karena kebenarannya dapat dilihat pada diri Kristus dan pada diri kalian. Sebab kegelapan semakin lenyap dan terang yang benar sudah mulai bersinar.
1Jn 2:9  Orang yang berkata bahwa ia berada di dalam terang tetapi orang itu masih membenci saudaranya, orang itu masih berada dalam kegelapan sampai sekarang.
1Jn 2:10  Orang yang mengasihi saudaranya, hidup dalam terang sehingga tidak ada sesuatu pun padanya yang akan menyebabkan orang lain jatuh dalam dosa.
1Jn 2:11  Tetapi orang yang membenci saudaranya, berada dalam kegelapan. Ia berjalan dalam kegelapan, dan tidak tahu ke mana ia pergi, sebab kegelapan itu membuat dia tidak bisa melihat.
1Jn 2:12  Saya menulis kepadamu, Anak-anak, sebab dosamu sudah diampuni karena Kristus.
1Jn 2:13  Saya menulis kepadamu, Bapak-bapak, sebab kalian mengenal Dia yang sudah ada sejak awal mula. Saya menulis kepadamu, Orang-orang muda, sebab kalian sudah mengalahkan Si Jahat.
1Jn 2:14  Saya menulis kepadamu, Anak-anak, sebab kalian mengenal Allah Bapa. Saya menulis kepadamu, Bapak-bapak, sebab kalian mengenal Sabda yang sudah ada sejak awal mula. Saya menulis kepadamu, Orang-orang muda, sebab kalian kuat. Perkataan Allah ada di dalam hatimu, dan kalian sudah mengalahkan Si Jahat.
1Jn 2:15  Janganlah mencintai dunia ini, atau apa saja yang ada di dalam dunia ini. Kalau kalian mencintai dunia, kalian tidak mencintai Bapa.
1Jn 2:16  Segala sesuatu yang ada di dalam dunia ini--yang diinginkan oleh tabiat manusia yang berdosa, yang dilihat lalu diingini dan yang dibangga-banggakan--semuanya adalah hal-hal yang tidak berasal dari Bapa, melainkan dari dunia.
1Jn 2:17  Dunia dan segala sesuatu di dalamnya yang diinginkan oleh manusia, sedang lenyap. Tetapi orang yang menuruti kemauan Allah, tetap hidup sampai selama-lamanya.
1Jn 2:18  Anak-anakku, akhir zaman sudah dekat! Sudah diberitahukan kepadamu sebelumnya bahwa Musuh Kristus akan datang; dan sekarang sudah muncul banyak orang yang memusuhi Kristus. Itu tandanya bahwa akhir zaman sudah dekat.
1Jn 2:19  Musuh-musuh Kristus itu adalah orang-orang yang sudah meninggalkan kita, karena mereka sebenarnya memang bukan orang-orang kita. Kalau mereka orang-orang kita, tentu mereka tetap bersama-sama kita. Tetapi mereka meninggalkan kita, supaya jelaslah bahwa tidak seorang pun dari mereka yang benar-benar termasuk golongan kita.
1Jn 2:20  Tetapi kalian sudah menerima Roh Allah yang dicurahkan oleh Yesus Kristus dan itu sebabnya kamu semuanya mengenal ajaran yang benar.
1Jn 2:21  Jadi saya menulis ini kepadamu bukannya karena kalian tidak mengenal ajaran yang benar, tetapi malah karena kalian sudah mengenalnya, dan kalian juga tahu bahwa tidak ada dusta dalam ajaran yang benar itu.
1Jn 2:22  Yang berdusta adalah orang yang berkata bahwa Yesus bukannya Raja Penyelamat yang dijanjikan Allah. Orang itu Musuh Kristus; ia tidak mengakui Bapa maupun Anak.
1Jn 2:23  Orang yang tidak mengakui Anak, berarti tidak menerima Bapa juga. Dan orang yang mengakui Anak, berarti menerima Bapa juga.
1Jn 2:24  Sebab itu, berita yang sudah kalian dengar sejak kalian mula-mula percaya haruslah kalian jaga baik-baik di dalam hati. Kalau berita itu kalian perhatikan baik-baik, kalian akan selalu hidup bersatu dengan Anak dan dengan Bapa.
1Jn 2:25  Dan inilah yang dijanjikan Kristus sendiri kepada kita: hidup sejati dan kekal.
1Jn 2:26  Saya tulis ini kepadamu mengenai orang-orang yang sedang berusaha menipu kalian.
1Jn 2:27  Tetapi mengenai kalian sendiri, Kristus telah mencurahkan Roh-Nya padamu. Dan selama Roh-Nya ada padamu, tidak perlu ada orang lain mengajar kalian. Sebab Roh-Nya mengajar kalian tentang segala sesuatu; dan apa yang diajarkan-Nya itu benar, bukan dusta. Sebab itu, hendaklah kalian taat kepada apa yang diajarkan oleh Roh itu, dan hendaklah kalian tetap hidup bersatu dengan Kristus.
1Jn 2:28  Jadi, anak-anakku, tetaplah hidup bersatu dengan Kristus, supaya nanti pada waktu Ia datang, kita menghadap Dia dengan penuh keberanian, dan tidak bersembunyi karena malu.
1Jn 2:29  Kalian sudah tahu bahwa Kristus hidup menurut kemauan Allah. Sebab itu kalian harus tahu pula bahwa setiap orang yang hidup menurut kemauan Allah adalah anak Allah.
1Jn 3:1  Lihatlah betapa Allah mengasihi kita, sehingga kita diakui sebagai anak-anak-Nya. Dan memang kita adalah anak-anak Allah. Itulah sebabnya dunia yang jahat ini tidak mengenal kita, sebab dunia tidak mengenal Allah.
1Jn 3:2  Saudara-saudara yang tercinta! Kita sekarang adalah anak-anak Allah, tetapi keadaan kita nanti belum jelas. Namun kita tahu bahwa kalau Kristus datang, kita akan menjadi seperti Dia, sebab kita akan melihat Dia dalam keadaan-Nya yang sebenarnya.
1Jn 3:3  Semua orang yang mempunyai harapan ini terhadap Kristus, menjaga dirinya supaya sungguh-sungguh suci, bersih dari dosa sebagaimana Kristus juga suci.
1Jn 3:4  Orang yang berbuat dosa, melanggar hukum Allah; sebab dosa adalah pelanggaran terhadap hukum Allah.
1Jn 3:5  Kalian tahu bahwa Kristus datang untuk menghapuskan dosa-dosa manusia, dan bahwa tidak ada dosa dalam diri-Nya.
1Jn 3:6  Semua orang yang hidup bersatu dengan Kristus, tidak terus-menerus berbuat dosa. Orang yang terus-menerus berbuat dosa, tidak pernah melihat Kristus atau mengenal-Nya.
1Jn 3:7  Anak-anakku, jangan membiarkan siapapun juga menyesatkan kalian. Orang yang melakukan kehendak Allah adalah anak Allah sebagaimana Kristus adalah Anak Allah.
1Jn 3:8  Tetapi orang yang terus-menerus berbuat dosa adalah anak Iblis, sebab Iblis berdosa sejak semula. Untuk inilah Anak Allah datang, yaitu untuk menghancurkan pekerjaan Iblis.
1Jn 3:9  Orang yang sudah menjadi Anak Allah, tidak terus-menerus berbuat dosa, sebab sifat Allah sendiri ada padanya. Dan karena Allah itu Bapanya, maka ia tidak dapat terus-menerus berbuat dosa.
1Jn 3:10  Inilah bedanya antara anak-anak Allah dengan anak-anak Iblis: barangsiapa tidak melakukan kehendak Allah, atau tidak mengasihi saudaranya, ia bukan anak Allah.
1Jn 3:11  Sejak semula sudah disampaikan berita ini kepadamu: Kita harus mengasihi satu sama lain.
1Jn 3:12  Janganlah kita seperti Kain, yang menjadi anak Iblis dan membunuh saudaranya sendiri. Apa sebab Kain membunuh saudaranya? Sebab hal-hal yang dilakukannya adalah salah, tetapi hal-hal yang dilakukan saudaranya adalah benar.
1Jn 3:13  Sebab itu, Saudara-saudaraku, janganlah heran kalau orang-orang dunia ini membenci kalian.
1Jn 3:14  Kita tahu bahwa kita sudah keluar dari kematian, dan berpindah kepada hidup. Kita tahu itu, sebab kita mengasihi sesama saudara kita. Orang yang tidak mengasihi, berarti masih dikuasai oleh kematian.
1Jn 3:15  Orang yang membenci saudaranya adalah pembunuh, dan kalian tahu bahwa seorang pembunuh tidak mempunyai hidup sejati dan kekal.
1Jn 3:16  Dengan jalan inilah kita mengetahui cara mengasihi sesama: Kristus sudah menyerahkan hidup-Nya untuk kita. Sebab itu, kita juga harus menyerahkan hidup kita untuk saudara-saudara kita!
1Jn 3:17  Kalau seorang yang berkecukupan melihat saudaranya berkekurangan, tetapi tidak mau menolong saudaranya itu, bagaimana orang itu dapat mengatakan bahwa ia mengasihi Allah?
1Jn 3:18  Anak-anakku! Janganlah kita mengasihi hanya di mulut atau hanya dengan perkataan saja. Hendaklah kita mengasihi dengan kasih yang sejati, yang dibuktikan dengan perbuatan kita.
1Jn 3:19  Demikianlah caranya kita mengetahui bahwa kita tergolong anak-anak Allah yang benar, dan hati kita dapat tenang di hadapan Allah.
1Jn 3:20  Kita tahu, bahwa kalau kita disalahkan oleh hati kita, pengetahuan Allah lebih besar dari pengetahuan hati kita, dan bahwa Ia tahu segala-galanya.
1Jn 3:21  Jadi, Saudara-saudaraku yang tercinta, kalau hati kita tidak menyalahkan kita, kita dapat menghadap Allah dengan keberanian.
1Jn 3:22  Dan apa yang kita minta daripada-Nya, kita mendapatnya, karena kita taat kepada perintah-perintah-Nya dan melakukan apa yang menyenangkan hati-Nya.
1Jn 3:23  Yang Ia perintahkan kepada kita ialah: Kita harus percaya kepada Anak-Nya, yaitu Yesus Kristus, dan kita harus saling mengasihi, seperti yang sudah diperintahkan Kristus kepada kita.
1Jn 3:24  Semua orang yang taat kepada perintah-perintah Allah, hidup bersatu dengan Allah, dan Allah bersatu dengan mereka. Dan kita tahu bahwa Allah hidup bersatu dengan kita, karena Ia sudah memberikan Roh-Nya kepada kita.
1Jn 4:1  Saudara-saudara yang tercinta! Janganlah percaya kepada semua orang yang mengaku mempunyai Roh Allah, tetapi ujilah dahulu mereka untuk mengetahui apakah roh yang ada pada mereka itu berasal dari Allah atau tidak. Sebab banyak nabi palsu sudah berkeliaran ke mana-mana.
1Jn 4:2  Beginilah caranya kalian tahu apakah itu Roh Allah atau tidak: Orang yang mengakui bahwa Yesus Kristus datang ke dunia sebagai manusia, orang itu mempunyai Roh yang datang dari Allah.
1Jn 4:3  Tetapi orang yang tidak mengakui hal ini mengenai Yesus, tidak mempunyai Roh Allah. Orang itu mempunyai roh dari Musuh Kristus. Saudara sudah mendengar bahwa roh itu akan datang, dan sekarang ia sudah ada di dalam dunia ini.
1Jn 4:4  Tetapi Anak-anakku, kalian milik Allah. Kalian sudah mengalahkan nabi-nabi palsu, sebab Roh yang ada padamu lebih berkuasa daripada roh yang ada pada orang-orang milik dunia ini.
1Jn 4:5  Nabi-nabi palsu itu berbicara tentang hal-hal dunia, dan dunia mendengar perkataan mereka, sebab mereka milik dunia.
1Jn 4:6  Tetapi kita adalah anak-anak Allah; dan orang yang mengenal Allah, mendengar perkataan kita. Orang yang bukan milik Allah, tidak mendengar perkataan kita. Begitulah caranya kita mengetahui perbedaan antara Roh yang memberi ajaran yang benar, dan roh yang menyesatkan.
1Jn 4:7  Saudara-saudara yang tercinta! Marilah kita mengasihi satu sama lain, sebab kasih berasal dari Allah. Orang yang mengasihi, adalah anak Allah dan ia mengenal Allah.
1Jn 4:8  Orang yang tidak mengasihi, tidak mengenal Allah; sebab Allah adalah kasih.
1Jn 4:9  Allah menyatakan bahwa Ia mengasihi kita dengan mengutus Anak-Nya yang tunggal ke dalam dunia supaya kita memperoleh hidup melalui Anak-Nya itu.
1Jn 4:10  Inilah kasih: Bukan kita yang sudah mengasihi Allah, tetapi Allah yang mengasihi kita dan mengutus Anak-Nya supaya melalui Dia kita mendapat pengampunan atas dosa-dosa kita.
1Jn 4:11  Saudara-saudara yang tercinta, kalau Allah begitu mengasihi kita, kita pun harus mengasihi satu sama lain.
1Jn 4:12  Tidak ada seorang pun yang pernah melihat Allah, tetapi kalau kita saling mengasihi, Allah bersatu dengan kita dan kasih-Nya menjadi sempurna dalam diri kita.
1Jn 4:13  Oleh karena Allah sudah memberikan kepada kita Roh-Nya, maka kita tahu bahwa kita sudah hidup bersatu dengan Allah, dan Allah hidup bersatu dengan kita.
1Jn 4:14  Kami sendiri sudah melihat Anak Allah, dan kami mengabarkan bahwa Ia diutus oleh Bapa untuk menjadi Raja Penyelamat dunia ini.
1Jn 4:15  Barangsiapa mengakui bahwa Yesus itu Anak Allah, Allah hidup bersatu dengan orang itu, dan orang itu pun hidup bersatu dengan Allah.
1Jn 4:16  Kita sendiri tahu dan percaya akan kasih Allah kepada kita. Allah itu kasih. Orang yang hidupnya dikuasai oleh kasih, orang itu bersatu dengan Allah, dan Allah bersatu dengan dia.
1Jn 4:17  Kasih dijadikan sempurna dalam diri kita, agar supaya kita mempunyai keberanian pada Hari Pengadilan. Kita akan mempunyai keberanian, sebab hidup kita di dunia ini sama seperti hidup Kristus.
1Jn 4:18  Orang yang menikmati kasih Allah, tidak mengenal perasaan takut; sebab kasih yang sempurna melenyapkan segala perasaan takut. Jadi nyatalah bahwa orang belum menikmati kasih Allah dengan sempurna kalau orang itu takut menghadapi Hari Pengadilan.
1Jn 4:19  Kita mengasihi, sebab Allah sudah terlebih dahulu mengasihi kita.
1Jn 4:20  Kalau seorang berkata, "Saya mengasihi Allah," tetapi ia tidak mengasihi saudaranya, orang itu pendusta. Sebab orang yang tidak mengasihi saudaranya yang dilihatnya, tidak mungkin bisa mengasihi Allah yang tidak dilihatnya.
1Jn 4:21  Sebab itu, inilah perintah yang diberi Kristus kepada kita: Barangsiapa mengasihi Allah harus mengasihi saudaranya juga.
1Jn 5:1  Orang yang percaya bahwa Yesus adalah Raja Penyelamat yang dijanjikan Allah, orang itu adalah anak Allah. Orang yang mengasihi seorang bapak, mengasihi anaknya juga.
1Jn 5:2  Dengan jalan inilah kita tahu bahwa kita mengasihi anak-anak Allah: Kita mengasihi Allah dan taat kepada perintah-perintah-Nya.
1Jn 5:3  Sebab, mengasihi Allah berarti taat kepada perintah-perintah-Nya. Dan perintah-perintah-Nya tidaklah berat untuk kita,
1Jn 5:4  sebab setiap anak Allah sanggup mengalahkan dunia yang jahat ini. Dan kita mengalahkan dunia dengan iman kita.
1Jn 5:5  Siapakah dapat mengalahkan dunia? Hanya orang yang percaya bahwa Yesus adalah Anak Allah.
1Jn 5:6  Yesus Kristuslah yang datang ke dunia dengan air baptisan-Nya dan dengan darah kematian-Nya. Ia datang bukan hanya dengan air saja, tetapi dengan air dan darah. Dan Roh Allah sendiri memberi kesaksian bahwa itu benar, sebab Roh tidak pernah berdusta.
1Jn 5:7  Ada tiga saksi:
1Jn 5:8  Roh Allah, air dan darah--ketiga-tiganya memberikan kesaksian yang sama.
1Jn 5:9  Kita percaya kepada kesaksian manusia, tetapi kesaksian Allah lebih kuat lagi. Dan itulah kesaksian yang sudah diberikan oleh Allah tentang Anak-Nya.
1Jn 5:10  Sebab itu, orang yang percaya kepada Anak Allah, sudah menerima kesaksian itu di dalam hatinya; tetapi orang yang tidak mempercayai Allah, sudah beranggapan seakan-akan Allah pendusta, sebab ia tidak percaya apa yang dikatakan Allah tentang Anak-Nya.
1Jn 5:11  Inilah kesaksian itu: Allah sudah memberikan kepada kita hidup sejati dan kekal, dan Anak-Nya adalah sumber hidup itu.
1Jn 5:12  Orang yang mempunyai Anak Allah, mempunyai hidup itu; dan orang yang tidak mempunyai Dia, tidak mempunyai hidup itu.
1Jn 5:13  Saya menulis kepada kalian yang percaya kepada Anak Allah, supaya kalian tahu bahwa kalian sudah mempunyai hidup sejati dan kekal.
1Jn 5:14  Dan kita berani menghadap Allah, karena kita yakin Ia mengabulkan doa kita, kalau kita minta apa saja yang sesuai dengan kehendak-Nya.
1Jn 5:15  Karena kita tahu bahwa Ia mendengarkan kita kalau kita memohon kepada-Nya, maka kita tahu juga bahwa Ia memberikan kita apa yang kita minta daripada-Nya.
1Jn 5:16  Kalau kalian melihat salah seorang sesama Kristen melakukan dosa yang tidak mengakibatkan orang itu kehilangan hidup sejati dan kekal, hendaklah kalian berdoa kepada Allah; dan Allah akan memberikan hidup itu kepadanya. Ini berkenaan dengan mereka yang dosanya tidak mengakibatkan mereka kehilangan hidup itu. Tetapi ada dosa yang mengakibatkan orang kehilangan hidup itu. Tentang hal itu saya tidak berkata bahwa kalian harus berdoa kepada Allah.
1Jn 5:17  Semua perbuatan yang salah adalah dosa, tetapi ada dosa yang tidak mengakibatkan orang kehilangan hidup sejati dan kekal itu.
1Jn 5:18  Kita tahu bahwa semua orang yang sudah menjadi anak-anak Allah, tidak terus-menerus berbuat dosa, sebab Anak Allah melindunginya, dan Si Jahat tidak dapat berbuat apa-apa terhadapnya.
1Jn 5:19  Kita tahu bahwa kita milik Allah, meskipun seluruh dunia ini di bawah kekuasaan Si Jahat.
1Jn 5:20  Kita tahu bahwa Anak Allah sudah datang dan sudah memberikan pengertian kepada kita, supaya kita mengenal Allah yang benar. Kita hidup bersatu dengan Allah yang benar dan hidup bersatu dengan Anak-Nya Yesus Kristus. Inilah Allah yang benar, dan inilah hidup sejati dan kekal.
1Jn 5:21  Anak-anakku, jauhkanlah dirimu dari berhala-berhala!


\end{document}