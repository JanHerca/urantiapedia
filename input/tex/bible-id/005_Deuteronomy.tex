\begin{document}

\title{Ulangan}


\chapter{1}

\par 1 Buku ini berisi kata-kata yang disampaikan Musa kepada bangsa Israel ketika mereka berada di padang gurun di Lembah Yordan. Lembah itu ada di sebelah timur Sungai Yordan, dekat kota Suf, antara kota Paran di satu pihak, dan kota-kota Tofel, Laban, Hazerot dan Di-Zahab di pihak lain.
\par 2 (Dari Gunung Sinai sampai ke Kades-Barnea lewat daerah pegunungan Edom, diperlukan sebelas hari perjalanan.)
\par 3 Pada tanggal satu bulan sebelas dalam tahun keempat puluh sesudah mereka meninggalkan Mesir, Musa menyampaikan kepada bangsa itu semua yang diperintahkan TUHAN untuk dikatakan kepada mereka.
\par 4 Itu terjadi sesudah Musa mengalahkan Sihon, raja orang Amori, yang memerintah di kota Hesybon, dan Og, raja Basan, yang memerintah di kota-kota Asytarot dan Edrei.
\par 5 Jadi, di sebelah timur Yordan di daerah Moab, Musa untuk pertama kalinya menerangkan hukum-hukum dan perintah-perintah TUHAN. Kata Musa,
\par 6 "Waktu kita berada di Gunung Sinai, TUHAN Allah kita berkata, 'Kamu sudah cukup lama tinggal di gunung ini.
\par 7 Sekarang bongkarlah perkemahanmu dan pergilah ke daerah pegunungan orang Amori dan ke seluruh daerah di sekitarnya, yaitu Lembah Yordan, daerah berbukit dan dataran rendah, daerah di sebelah selatan dan pantai Laut Tengah. Pergilah menduduki tanah Kanaan dan Pegunungan Libanon sampai ke sungai besar Efrat.
\par 8 Aku, TUHAN, telah menjanjikan seluruh tanah itu kepada leluhurmu Abraham, Ishak dan Yakub serta keturunan mereka. Jadi pergilah menduduki tanah itu.'"
\par 9 Musa berkata kepada bangsa itu, "Waktu kita berada di Gunung Sinai, saya berkata kepadamu: Tanggung jawab untuk memimpin kamu terlalu berat bagi saya. Saya tak dapat melakukannya seorang diri, sebab
\par 10 TUHAN Allahmu telah membuat jumlahmu bertambah sebanyak bintang-bintang di langit.
\par 11 Semoga TUHAN, Allah nenek moyangmu terus menambah jumlahmu sampai seribu kali lipat, dan menjadikan kamu bangsa yang makmur seperti yang telah dijanjikan-Nya!
\par 12 Tetapi bagaimana mungkin saya sendirian dapat memikul tanggung jawab yang berat untuk membereskan semua persoalanmu?
\par 13 Pilihlah dari setiap suku beberapa orang laki-laki yang bijaksana, berpengalaman dan penuh pengertian, supaya saya mengangkat mereka untuk memimpin kamu.
\par 14 Usul itu kamu terima dengan baik.
\par 15 Sebab itu di antara pemimpin-pemimpin yang kamu pilih, beberapa orang yang bijaksana dan berpengalaman saya ambil dan saya angkat untuk mengepalai kamu. Beberapa dijadikan pemimpin atas seribu orang, yang lain atas seratus orang, atas lima puluh orang dan atas sepuluh orang. Saya juga mengangkat pemimpin-pemimpin lain untuk mengurus suku-sukumu.
\par 16 Pada waktu itu saya menugaskan mereka begini: Perhatikanlah perselisihan-perselisihan yang timbul di antara bangsamu. Berilah keputusan yang adil, baik dalam perkara yang menyangkut orang sebangsamu atau melibatkan orang asing yang tinggal di antara kamu.
\par 17 Keputusanmu janganlah berat sebelah; setiap orang harus kamu adili dengan cara yang sama, tanpa membedakan siapa orang itu. Jangan takut terhadap siapa pun, sebab keputusan yang kamu ambil datang dari Allah. Kalau suatu perkara terlalu sulit bagimu, bawalah kepada saya, supaya saya memutuskannya.
\par 18 Pada waktu itu juga saya memberi petunjuk-petunjuk kepadamu tentang segala hal lain yang harus kamu lakukan."
\par 19 "Lalu sesuai dengan perintah TUHAN Allah kita, kita meninggalkan Gunung Sinai dan berjalan melalui padang gurun yang luas dan dahsyat menuju daerah pegunungan orang Amori. Pada waktu kita sampai di Kades-Barnea,
\par 20 saya berkata: Sekarang kamu sudah sampai di daerah pegunungan orang Amori, tanah yang diberikan kepada kita oleh TUHAN Allah kita dan Allah nenek moyang kita. Lihat, di depanmu terbentang tanah itu. Pergilah untuk mendudukinya seperti yang diperintahkan TUHAN. Janganlah bimbang atau takut.
\par 22 Tetapi kamu datang kepada saya dan berkata, 'Lebih baik kita menyuruh beberapa orang mendahului kita untuk menyelidiki negeri itu supaya mereka dapat mengatakan kepada kita jalan mana yang harus kita lalui dan bagaimana kota-kota yang akan kita datangi.'
\par 23 Usul itu saya setujui, sebab itu saya pilih dua belas orang, seorang dari tiap suku.
\par 24 Mereka memasuki daerah pegunungan itu sejauh Lembah Eskol, lalu menjelajahinya.
\par 25 Kemudian mereka kembali kepada kita membawa buah-buahan yang mereka dapat di sana dan melaporkan bahwa negeri yang akan diberikan TUHAN Allah kita itu sangat subur.
\par 26 Tetapi kamu melawan perintah TUHAN Allahmu dan tak mau memasuki negeri itu.
\par 27 Kamu mengomel begini, 'TUHAN benci kepada kita. Maka Ia membawa kita keluar dari Mesir untuk menyerahkan kita ke dalam tangan orang Amori supaya mereka membunuh kita.
\par 28 Jadi untuk apa kita ke sana? Orang-orang yang kita kirim ke sana membuat kita takut dengan melaporkan bahwa bangsa itu lebih kuat dan lebih gagah dari kita, dan bahwa mereka tinggal di kota-kota yang temboknya setinggi langit. Mereka melihat raksasa di sana!'
\par 29 Tetapi saya berkata: Jangan takut kepada orang-orang itu.
\par 30 TUHAN Allahmu akan memimpin kamu dan berjuang untukmu seperti sudah kamu saksikan di Mesir
\par 31 dan di padang gurun. Kamu melihat bagaimana TUHAN Allahmu membawa kamu dengan selamat sepanjang jalan itu sampai ke tempat ini, tiada bedanya seperti seorang ayah menggendong anaknya.
\par 32 Tetapi walaupun saya sudah berkata begitu, kamu belum juga mau percaya kepada TUHAN Allahmu,
\par 33 yang berjalan di depanmu untuk mencari tempat berkemah bagimu. Ia menunjukkan jalan kepadamu di waktu malam dalam tiang api, dan di waktu siang dalam tiang awan."
\par 34 "TUHAN mendengar keluhanmu lalu menjadi marah. Ia bersumpah,
\par 35 'Tak seorang pun dari angkatan jahat ini akan memasuki negeri subur yang Kujanjikan kepada nenek moyangmu.
\par 36 Hanya Kaleb, anak Yefune akan memasukinya, karena ia tetap setia kepada-Ku. Kepada Kaleb dan keturunannya akan Kuberikan negeri yang telah dijelajahinya itu.'
\par 37 Kamu juga menyebabkan TUHAN marah kepadaku dan berkata, 'Musa, engkau pun tidak akan memasuki negeri itu.
\par 38 Tetapi tabahkanlah hati pembantumu, Yosua anak Nun. Dialah yang akan memimpin Israel merebut tanah itu.'
\par 39 Lalu TUHAN berkata kepada kita sekalian, 'Anak-anakmu yang masih terlalu kecil untuk membedakan yang baik dari yang jahat, anak-anak yang menurut katamu akan dirampas oleh musuh-musuhmu, kepada merekalah akan Kuberikan tanah itu, dan mereka akan mendudukinya.
\par 40 Tetapi kamu, kembalilah ke padang gurun lewat jalan yang menuju Teluk Akaba.'
\par 41 Kamu menjawab, 'Musa, kami telah berdosa terhadap TUHAN. Tetapi sekarang kami mau maju berperang seperti yang diperintahkan TUHAN, Allah kita.' Lalu masing-masing di antara kamu bersiap-siap untuk berperang, dengan anggapan bahwa menyerbu daerah pegunungan itu soal mudah.
\par 42 Tetapi TUHAN berkata kepada saya, 'Jangan kaubiarkan mereka berperang, karena Aku tak akan menyertai mereka, nanti mereka dikalahkan oleh musuh mereka.'
\par 43 Saya menyampaikan kepadamu apa yang dikatakan TUHAN, tetapi kamu tidak peduli. Kamu menentang perintah TUHAN, dan dengan congkak pergi ke daerah pegunungan itu.
\par 44 Lalu orang Amori yang mendiami daerah pegunungan itu keluar melawan kamu. Jumlah mereka banyak sekali, seperti kawanan lebah. Mereka mengejar kamu sejauh Horma, lalu mengalahkan kamu di daerah pegunungan Edom.
\par 45 Kemudian kamu kembali dan berseru kepada TUHAN minta tolong, tetapi Ia tak mau mendengarkan atau memperhatikan kamu."
\par 46 "Sesudah itu kita tinggal di Kades lama sekali.

\chapter{2}

\par 1 Akhirnya kita kembali ke padang gurun lewat jalan yang menuju ke Teluk Akaba, seperti yang diperintahkan TUHAN. Lama kita mengembara di daerah pegunungan Edom.
\par 2 Lalu TUHAN berkata kepada saya
\par 3 bahwa sudah cukup lama kita mengembara di daerah itu. Jadi kita harus pergi ke utara.
\par 4 Kemudian TUHAN menyuruh saya memberi petunjuk-petunjuk ini kepadamu, 'Tidak lama lagi kamu akan memasuki negeri Edom, daerah saudara-saudaramu, keturunan Esau. Mereka akan takut kepadamu,
\par 5 tetapi kamu tak boleh menyerang mereka, sebab dari tanah mereka sedikit pun tak akan Kuberikan kepadamu, karena daerah Edom sudah Kuberikan kepada keturunan Esau.
\par 6 Kamu boleh membeli makanan dan minuman dari mereka.'
\par 7 Ingatlah bagaimana TUHAN Allahmu telah memberkati kamu dalam segala yang kamu lakukan. Ia memelihara kamu selama kamu mengembara di padang gurun yang luas ini. Ia melindungi kamu selama empat puluh tahun ini dan memberi segala yang kamu perlukan.
\par 8 Kemudian kita berjalan terus, meninggalkan jalan yang melalui kota-kota Elat dan Ezion-Geber menuju ke Laut Mati, lalu belok ke timur laut menuju Gurun Moab.
\par 9 TUHAN berkata kepada saya, 'Orang Moab, keturunan Lot, tak boleh kamu ganggu. Jangan juga menyerang mereka. Negeri Ar telah Kuberikan kepada mereka dan dari tanah mereka sedikit pun tak akan Kuberikan kepadamu.'"
\par 10 (Dahulu Ar didiami oleh orang Emim. Mereka besar perawakannya, sama dengan orang Enak, juga bangsa raksasa.
\par 11 Seperti orang Enak, mereka juga disebut orang Refaim, akan tetapi orang Moab menamakan mereka orang Emim.
\par 12 Dahulu orang Hori tinggal di Edom, tetapi diusir dan dimusnahkan oleh keturunan Esau, yang kemudian menduduki tanah mereka; begitu juga bangsa Israel di kemudian hari mengusir musuh-musuhnya dari tanah yang diberikan TUHAN kepada mereka.)
\par 13 "Kemudian kita menyeberangi Sungai Zered seperti yang diperintahkan TUHAN kepada kita.
\par 14 Itu terjadi tiga puluh delapan tahun sesudah kita meninggalkan Kades-Barnea. Semua prajurit dari angkatan itu sudah mati seperti yang dikatakan TUHAN.
\par 15 TUHAN terus-menerus melawan mereka sampai akhirnya mereka semua binasa.
\par 16 Sesudah mereka semua mati,
\par 17 TUHAN berkata kepada kita,
\par 18 'Hari ini kamu melalui daerah Moab, lewat negeri Ar.
\par 19 Maka kamu sampai ke dekat tanah orang Amon, keturunan Lot. Janganlah mengganggu atau menyerang mereka, sebab dari tanah yang sudah Kuberikan kepada mereka, sedikit pun tak ada yang akan Kuberikan kepadamu.'"
\par 20 (Daerah itu juga terkenal sebagai negeri Refaim, nama bangsa yang dahulu tinggal di situ; oleh orang Amon mereka dinamakan orang Zamzumim.
\par 21 Seperti orang Enak, mereka juga besar perawakannya. Mereka sangat kuat, dan jumlahnya banyak. Tetapi TUHAN membinasakan mereka, sehingga orang Amon merebut tanah itu, lalu berdiam di situ.
\par 22 Begitu juga dilakukan TUHAN untuk orang Edom, keturunan Esau, yang mendiami daerah pegunungan Edom. TUHAN membinasakan orang Hori, sehingga orang Edom merebut tanah itu dan mendudukinya, lalu menetap di situ sampai sekarang.
\par 23 Tanah di sepanjang pantai Laut Tengah diduduki orang-orang dari pulau Kreta. Mereka telah membinasakan orang Awi, penduduk asli tanah itu, lalu menduduki daerah itu sampai ke selatan sejauh kota Gaza.)
\par 24 "Sesudah kita melewati daerah Moab, TUHAN berkata kepada kita, 'Pergilah sekarang menyeberangi Sungai Arnon. Aku akan menyerahkan kepadamu Sihon, raja Amori yang memerintah di Hesybon, bersama-sama dengan negerinya. Seranglah dia, lalu dudukilah tanahnya.
\par 25 Mulai hari ini Aku akan membuat kamu ditakuti bangsa-bangsa di mana saja. Setiap orang akan gemetar ketakutan bila mendengar tentang kamu.'"
\par 26 "Kemudian saya mengirim beberapa utusan dari padang gurun Kedemot kepada Raja Sihon dari Hesybon untuk menyampaikan usul perdamaian ini:
\par 27 'Izinkan kami melalui negeri Tuanku. Kami akan berjalan terus dan tidak menyimpang dari jalan raya.
\par 28 Kami akan membayar makanan yang kami makan dan air yang kami minum. Kami hanya perlu melalui negeri ini,
\par 29 sampai kami menyeberangi Sungai Yordan untuk masuk ke negeri yang diberikan TUHAN Allah kami kepada kami. Keturunan Esau yang tinggal di Edom dan orang Moab yang tinggal di Ar juga sudah mengizinkan kami melalui daerah mereka.'
\par 30 Tetapi Raja Sihon tidak mengizinkan kita melalui negerinya itu. TUHAN Allahmu membuat dia keras kepala dan nekat supaya kita dapat mengalahkan dia. Daerahnya masih kita duduki sampai sekarang.
\par 31 Kemudian TUHAN berkata kepada saya, 'Raja Sihon dan negerinya Kuserahkan kepadamu; ambillah dan dudukilah negeri itu.'
\par 32 Maka datanglah Sihon dengan seluruh pasukannya untuk memerangi kita dekat kota Yahas.
\par 33 Tetapi TUHAN Allah kita menyerahkan dia kepada kita, lalu kita mengalahkan dia beserta anak-anaknya dan seluruh tentaranya.
\par 34 Pada waktu itu juga kita rebut dan hancurkan setiap kota, dan bunuh seluruh penduduknya, laki-laki, perempuan, dan anak-anak. Tak ada yang dibiarkan hidup.
\par 35 Ternak mereka kita ambil dan kota-kotanya kita rampasi.
\par 36 Dengan pertolongan TUHAN Allah kita, kita dapat merebut semua kota itu, mulai dari Aroer, di ujung Lembah Arnon, dan kota di tengah lembah itu, sampai ke daerah Gilead. Tak ada kota yang temboknya terlalu kuat bagi kita.
\par 37 Tetapi kita tidak mendekati daerah orang Amon atau tepi Sungai Yabok atau kota-kota di daerah pegunungan atau tempat lain mana pun yang dilarang TUHAN Allah kita."

\chapter{3}

\par 1 "Kemudian kita bergerak ke utara menuju daerah Basan. Og, raja Basan maju dengan seluruh tentaranya untuk memerangi kita dekat kota Edrei.
\par 2 Tetapi TUHAN berkata kepada saya, 'Jangan takut kepadanya, Musa. Aku akan menyerahkan dia, tentaranya dan seluruh daerahnya kepadamu. Perlakukanlah dia seperti telah kauperlakukan Sihon, raja Amori yang memerintah di Hesybon.'
\par 3 Jadi TUHAN Allah kita juga menyerahkan Raja Og dan rakyatnya kepada kita, dan mereka semua kita kalahkan.
\par 4 Pada waktu itu kita rebut semua kotanya, tidak satu pun yang tertinggal. Semuanya ada enam puluh kota yang kita rebut di seluruh daerah Argob, yang diperintah Raja Og dari Basan.
\par 5 Semua kota itu diperkuat dengan tembok-tembok yang tinggi, pintu-pintu gerbang dan palang-palang pintu untuk menutup gerbang-gerbang itu. Selain itu ada banyak kota lain yang tidak bertembok.
\par 6 Semua kota itu kita hancurkan, dan semua orang lelaki, perempuan dan anak-anak mereka kita bunuh, seperti yang kita lakukan dengan kota-kota Raja Sihon dari Hesybon.
\par 7 Ternak mereka kita ambil dan kota-kota itu kita rampoki.
\par 8 Pada waktu itu kita ambil dari kedua raja Amori itu daerah di sebelah timur Sungai Yordan dan Sungai Arnon sampai ke Gunung Hermon.
\par 9 (Gunung Hermon disebut Siryon oleh orang Sidon, dan Senir oleh orang Amori.)
\par 10 Kita ambil seluruh wilayah Raja Og dari Basan: kota-kotanya di dataran tinggi, daerah Gilead dan Basan, terus ke timur sejauh kota Salkha dan Edrei."
\par 11 (Raja Og adalah orang Refaim yang terakhir. Peti mayatnya terbuat dari batu, panjangnya empat meter dan lebarnya hampir dua meter menurut ukuran yang ditetapkan. Sampai sekarang masih dapat dilihat di Raba, kota orang Amon.)
\par 12 "Sesudah tanah itu menjadi milik kita, saya berikan kepada suku Ruben dan Gad daerah sebelah utara kota Aroer dekat Sungai Arnon, dan sebagian dari daerah pegunungan Gilead, termasuk kota-kotanya.
\par 13 Kepada setengah suku Manasye saya berikan sisa daerah Gilead, juga daerah Basan bekas jajahan Og, yakni seluruh wilayah Argob." (Basan terkenal sebagai negeri orang Refaim.
\par 14 Yair dari suku Manasye mengambil seluruh wilayah Argob, yaitu dari Basan sampai perbatasan Gesur dan Maakha. Ia menamai desa-desa itu menurut namanya sendiri, dan sampai sekarang masih terkenal sebagai desa-desa Yair.)
\par 15 "Daerah Gilead saya berikan kepada kaum Makhir dari suku Manasye.
\par 16 Dan kepada suku-suku Ruben dan Gad saya berikan daerah antara Gilead dan Sungai Arnon. Tengah sungai itu merupakan perbatasan mereka di sebelah selatan, sedangkan di utara mereka berbatasan dengan Sungai Yabok, yang sebagiannya merupakan batas negeri orang Amon.
\par 17 Di sebelah barat, daerah mereka terbentang sampai ke Sungai Yordan, dari Danau Galilea di utara sampai Laut Mati di sebelah selatan, terus sampai kaki Gunung Pisga di sebelah timur.
\par 18 Pada waktu itu saya memberi kepada mereka perintah ini: TUHAN Allahmu memberikan kepadamu daerah sebelah timur Sungai Yordan untuk tempat tinggalmu. Tetapi prajurit-prajuritmu harus bersenjata dan menyeberangi Sungai Yordan mendahului suku-suku Israel yang lain.
\par 19 Hanya anak istrimu serta ternakmu yang banyak boleh tinggal di kota-kota yang sudah saya berikan kepadamu.
\par 20 Bantulah orang-orang sebangsamu sampai mereka menduduki tanah yang akan diberikan TUHAN Allahmu kepada mereka di sebelah barat Sungai Yordan, dan sampai TUHAN memperkenankan mereka hidup dengan tentram di sana, seperti yang dilakukan-Nya bagimu di sini. Baru sesudah itu kamu boleh kembali ke tanah milikmu yang telah saya berikan kepadamu.
\par 21 Lalu saya berkata kepada Yosua: Engkau telah melihat segala yang dilakukan TUHAN Allahmu terhadap Sihon dan Og, kedua raja itu. Begitu juga Ia akan memperlakukan setiap raja yang kaurebut tanahnya.
\par 22 Jangan takut kepada mereka, karena TUHAN Allahmu akan berjuang untukmu."
\par 23 "Pada waktu itu saya mohon kepada TUHAN:
\par 24 Ya TUHAN Yang Mahatinggi, saya tahu bahwa yang Kautunjukkan kepada saya ini baru merupakan permulaan dari perbuatan-perbuatan-Mu yang besar dan ajaib. Tiada ilah lain di langit maupun di bumi yang seperti Engkau dapat melakukan hal-hal yang hebat.
\par 25 Izinkanlah saya menyeberangi Sungai Yordan, ya TUHAN, untuk melihat negeri yang subur di seberang, daerah pegunungan yang indah dan Pegunungan Libanon.
\par 26 Tetapi kamu telah menyebabkan TUHAN marah kepada saya sehingga Ia tak mau mengabulkan permohonan saya. Sebaliknya TUHAN berkata, 'Cukuplah, Musa, jangan kausebutkan soal itu lagi!
\par 27 Pergilah ke puncak Gunung Pisga dan pandanglah ke arah utara dan selatan, ke timur dan ke barat. Perhatikanlah baik-baik apa yang kaulihat itu, sebab engkau tak akan menyeberangi Sungai Yordan.
\par 28 Berilah petunjuk-petunjuk kepada Yosua. Tabahkanlah hatinya, sebab dialah yang akan memimpin bangsa itu ke seberang untuk menduduki negeri yang kaulihat itu.'
\par 29 Maka tinggallah kita di lembah di seberang kota Bet-Peor."

\chapter{4}

\par 1 Lalu Musa berkata kepada bangsa itu, "Taatilah semua hukum yang saya ajarkan kepadamu, supaya kamu boleh hidup sampai mendiami tanah yang akan diberikan TUHAN, Allah leluhurmu, kepadamu.
\par 2 Apa yang saya perintahkan, jangan ditambah atau dikurangi sedikit pun. Taatilah perintah-perintah TUHAN Allahmu yang saya berikan kepadamu.
\par 3 Kamu melihat sendiri apa yang dilakukan TUHAN Allahmu di Gunung Peor. Setiap orang yang menyembah Baal di tempat itu dibinasakan-Nya,
\par 4 sedangkan kamu yang tetap setia kepada TUHAN Allahmu masih hidup sampai sekarang.
\par 5 Semua hukum itu saya ajarkan kepadamu seperti yang diperintahkan TUHAN Allahku. Taatilah kesemuanya itu di negeri yang akan kamu masuki dan diami itu.
\par 6 Lakukanlah itu dengan setia. Kalau kamu berbuat demikian, maka kebijaksanaanmu menjadi nyata bagi bangsa-bangsa lain. Apabila mereka mendengar tentang hukum-hukum itu, mereka akan berkata, 'Alangkah bijaksana dan cerdasnya bangsa yang besar itu!'
\par 7 Bangsa besar manakah yang mempunyai ilah yang begitu dekat apabila diperlukan seperti TUHAN Allah kita dekat kepada kita? Ia menjawab kapan saja kita berseru minta tolong kepada-Nya.
\par 8 Dan bangsa besar manakah yang mempunyai hukum-hukum yang begitu adil seperti hukum-hukum yang saya ajarkan kepadamu hari ini?
\par 9 Perhatikanlah dan ingatlah baik-baik, supaya seumur hidupmu kamu tidak lupa apa yang sudah kamu saksikan sendiri. Ceritakanlah kepada anak-cucumu
\par 10 tentang hari itu, ketika kamu berdiri di depan TUHAN Allahmu di Gunung Sinai. Pada waktu itu Ia berkata kepadaku, 'Kumpulkanlah bangsa itu. Aku ingin mereka mendengar apa yang hendak Kusampaikan, supaya mereka belajar taat kepada-Ku seumur hidup, dan supaya mereka mengajar anak-anak mereka untuk taat kepada-Ku.'
\par 11 Ceritakanlah kepada anak-anakmu bagaimana kamu datang dan berdiri di kaki gunung yang diliputi asap gelap seperti awan tebal dan apinya menyala sampai ke langit.
\par 12 Kemudian TUHAN berbicara kepadamu dari api dan kamu mendengar suara-Nya, tetapi tidak melihat-Nya dalam bentuk apa pun.
\par 13 Di sana Ia memaklumkan kepadamu apa yang harus kamu lakukan untuk menepati perjanjian yang dibuat-Nya dengan kamu. Sepuluh Perintah yang telah ditulis-Nya pada kedua batu tulis itu harus kamu lakukan.
\par 14 TUHAN menyuruh saya mengajarkan kepadamu semua hukum yang harus ditaati di negeri yang nanti kamu masuki dan diami itu."
\par 15 "Ingatlah baik-baik! Ketika TUHAN berbicara kepadamu dari dalam api di Gunung Sinai, kamu tidak melihat apa-apa.
\par 16 Jadi jagalah supaya kamu jangan sampai berdosa karena membuat bagi dirimu sendiri patung untuk disembah dalam bentuk apa pun, laki-laki atau perempuan,
\par 17 burung atau ikan, binatang melata atau binatang lainnya.
\par 19 Janganlah berdosa karena menyembah dan mengabdi kepada apa yang kamu lihat di langit: matahari, bulan dan bintang-bintang. TUHAN Allahmu membiarkan bangsa-bangsa lain menyembah benda-benda itu.
\par 20 Tetapi kamu adalah bangsa yang diselamatkan-Nya dari Mesir, tempat kamu menderita dengan hebat. Kamu dibawa-Nya keluar dari negeri itu, supaya menjadi bangsa-Nya sendiri seperti keadaanmu sekarang ini.
\par 21 Tetapi TUHAN Allahmu menjadi marah kepada saya oleh karena kamu, dan Ia bersumpah bahwa saya tak akan menyeberangi Sungai Yordan untuk masuk ke tanah subur yang akan diberikan-Nya kepadamu.
\par 22 Tak lama lagi kamu akan menyeberang dan menduduki tanah yang subur itu; tetapi saya tidak ikut, sebab saya akan mati di sini.
\par 23 Ingatlah baik-baik! Jangan lupa akan perjanjian yang dibuat TUHAN Allahmu dengan kamu. Taatilah perintah-Nya; jangan sekali-kali membuat bagi dirimu patung dalam bentuk apa pun untuk disembah,
\par 24 sebab TUHAN Allahmu seperti api yang menghanguskan. Ia tak mau disamakan dengan apa pun.
\par 25 Juga kalau kamu nanti sudah lama tinggal di negeri itu dan sudah beranak cucu, jangan berdosa karena membuat bagi dirimu patung dalam bentuk apa saja. Perbuatan itu suatu kejahatan di mata TUHAN Allahmu dan akan membuat Ia marah.
\par 26 Kiranya langit dan bumi hari ini menjadi saksi bahwa kalau kamu tidak taat kepada saya, kamu akan segera lenyap dari tanah yang kamu diami di seberang Sungai Yordan itu. Kamu tak akan lama tinggal di situ; kamu akan dibinasakan sama sekali.
\par 27 TUHAN akan menceraiberaikan kamu di antara bangsa-bangsa lain dan hanya sedikit di antara kamu yang tetap hidup.
\par 28 Di situ kamu akan berbakti kepada ilah-ilah yang dibuat dengan tangan manusia, ilah-ilah dari kayu dan batu yang tidak dapat melihat atau mendengar, tidak dapat makan atau mencium.
\par 29 Lalu kamu akan mencari TUHAN Allahmu, dan kalau kamu mencari-Nya dengan segenap hatimu, kamu akan menemukan Dia.
\par 30 Apabila kamu ditimpa semua bencana itu dan kamu ada dalam kesukaran, maka kamu akan kembali kepada TUHAN Allahmu dan taat kepada-Nya.
\par 31 TUHAN Allahmu adalah Allah yang sangat berbelaskasihan. Ia tak akan meninggalkan atau membinasakan kamu, atau melupakan perjanjian yang dibuat-Nya sendiri dengan leluhurmu.
\par 32 Selidikilah masa lampau, jauh sebelum kamu lahir, mulai dari hari ketika Allah menciptakan manusia di bumi ini. Carilah di seluruh muka bumi. Pernahkah hal sehebat ini terjadi sebelumnya? Pernahkah ada orang yang mendengar tentang kejadian semacam ini?
\par 33 Pernahkah ada bangsa yang tetap hidup setelah mendengar suatu ilah berbicara kepada mereka dari api, seperti yang kamu alami?
\par 34 Pernahkah ada suatu ilah yang berani mengambil suatu bangsa dari bangsa-bangsa yang ada lalu menjadikan mereka umatnya sendiri, seperti yang dilakukan TUHAN Allahmu untukmu di Mesir? Kamu lihat sendiri bagaimana Allah dengan kekuatan yang besar mendatangkan bencana dan perang, mengerjakan keajaiban-keajaiban dan melakukan hal-hal yang dahsyat.
\par 35 Semua itu ditunjukkan TUHAN kepadamu untuk membuktikan bahwa hanya TUHAN itu Allah, dan tidak ada yang lain.
\par 36 Dari langit Ia memperdengarkan suara-Nya kepadamu untuk mengajar kamu; di bumi ini Ia telah memperlihatkan kepadamu api-Nya yang hebat, dan dari api itu Ia berbicara kepadamu.
\par 37 Karena Ia mencintai leluhurmu, maka kamu telah dipilih-Nya; dengan kuasa-Nya yang besar kamu telah dibawa-Nya keluar dari Mesir.
\par 38 Bangsa-bangsa yang lebih besar dan lebih kuat diusir-Nya dari hadapanmu, untuk mengantar kamu ke negeri mereka yang diberikan-Nya kepadamu, negeri yang sekarang menjadi milikmu.
\par 39 Sebab itu sadarilah hari ini dan jangan lupa: TUHAN satu-satunya Allah di langit dan di bumi; tak ada yang lain.
\par 40 Taatilah semua hukum yang saya berikan kepadamu hari ini, maka kamu dan keturunanmu akan sejahtera dan terus hidup di negeri yang diberikan TUHAN Allahmu menjadi milikmu untuk selama-lamanya."
\par 41 Lalu Musa menunjuk tiga kota di sebelah timur Sungai Yordan
\par 42 menjadi kota suaka, tempat orang dapat menyelamatkan diri kalau ia dengan tidak sengaja telah membunuh orang yang bukan musuhnya. Ia dapat melarikan diri ke salah satu kota itu supaya tidak dibunuh juga.
\par 43 Untuk suku Ruben disediakan kota Bezer di dataran tinggi padang gurun; untuk suku Gad kota Ramot di daerah Gilead, dan untuk orang Manasye kota Golan di daerah Basan.
\par 44 Lalu Musa memberi hukum-hukum dan peraturan-peraturan Allah kepada bangsa Israel.
\par 45 Hal itu dilakukannya sesudah mereka keluar dari Mesir dan berada di lembah sebelah timur Sungai Yordan, berhadapan dengan kota Bet-Peor. Wilayah itu bekas daerah Sihon, raja orang Amori, yang dahulu memerintah di kota Hesybon. Musa dan bangsa Israel mengalahkan dia pada waktu mereka keluar dari Mesir.
\par 47 Mereka menduduki tanah Raja Sihon itu, juga tanah Raja Og dari Basan, seorang raja Amori lain yang tinggal di sebelah timur Sungai Yordan.
\par 48 Tanah itu terbentang dari kota Aroer di tepi Sungai Arnon, terus ke utara sampai Gunung Siryon, yaitu Gunung Hermon,
\par 49 dan seluruh daerah di sebelah timur Sungai Yordan ke selatan sampai Laut Mati di kaki Gunung Pisga.

\chapter{5}

\par 1 Musa mengumpulkan seluruh bangsa Israel, lalu berkata kepada mereka, "Saudara-saudara, dengarlah hukum-hukum yang saya berikan kepadamu hari ini. Pelajarilah hukum-hukum itu, dan taatilah dengan setia.
\par 2 Di Gunung Sinai TUHAN Allah kita membuat suatu perjanjian,
\par 3 bukan saja dengan leluhur kita, tetapi dengan kita semua di sini.
\par 4 Dari api di atas gunung itu TUHAN mau berbicara langsung dengan kamu.
\par 5 Tetapi karena kamu takut kepada api dan tidak berani mendaki gunung, maka saya pada waktu itu berdiri di antara kamu dengan TUHAN untuk menyampaikan perkataan TUHAN kepadamu. TUHAN memaklumkan,
\par 6 'Akulah TUHAN Allahmu yang membawa kamu keluar dari Mesir, tempat kamu diperbudak.
\par 7 Jangan menyembah ilah-ilah lain. Sembahlah Aku saja.
\par 8 Jangan membuat bagi dirimu patung yang menyerupai apa saja yang ada di langit, di bumi atau di dalam air di bawah bumi.
\par 9 Jangan menyembah patung semacam itu karena Aku TUHAN Allahmu dan tak mau disamakan dengan apa pun. Orang-orang yang membenci Aku Kuhukum sampai kepada keturunan yang ketiga dan keempat.
\par 10 Tetapi Aku menunjukkan kasih-Ku kepada beribu-ribu keturunan dari orang-orang yang mencintai Aku dan taat kepada perintah-Ku.
\par 11 Jangan menyebut nama-Ku dengan sembarangan, sebab Aku, TUHAN Allahmu, menghukum siapa saja yang menyalahgunakan nama-Ku.
\par 12 Rayakanlah hari Sabat dan hormatilah sebagai hari yang suci, sebab Akulah TUHAN Allahmu yang memerintahkannya kepadamu.
\par 13 Kamu Kuberi enam hari untuk bekerja,
\par 14 tetapi hari yang ketujuh adalah hari istirahat yang khusus untuk Aku, TUHAN Allahmu. Pada hari itu tak seorang pun boleh bekerja, baik kamu, anak-anakmu, hamba-hambamu, ternakmu atau orang asing yang tinggal di negerimu. Hamba-hambamu harus beristirahat seperti kamu sendiri.
\par 15 Ingatlah bahwa kamu dahulu diperbudak di Mesir, lalu dibebaskan oleh TUHAN Allahmu dengan kekuatan yang besar. Sebab itu kamu Kuperintahkan untuk merayakan hari Sabat.
\par 16 Hormatilah dan peliharalah ayah dan ibumu; itulah perintah-Ku, TUHAN Allahmu, supaya kamu sejahtera dan panjang umur di negeri yang akan Kuberikan kepadamu.
\par 17 Jangan membunuh.
\par 18 Jangan berzinah.
\par 19 Jangan mencuri.
\par 20 Jangan memberi kesaksian palsu tentang orang lain.
\par 21 Jangan menginginkan kepunyaan orang lain: rumahnya, tanahnya, istrinya, hamba-hambanya, ternaknya, keledainya atau apa saja yang dimilikinya.'
\par 22 Itulah perintah-perintah yang diberikan TUHAN kepada kamu ketika kamu berkumpul di gunung. Ketika Ia berbicara dengan suara gemuruh dari api dan awan yang tebal, Ia memberikan perintah-perintah itu, dan bukan yang lain. Lalu Ia menulisnya pada dua batu tulis dan memberikannya kepadaku."
\par 23 "Waktu seluruh gunung menyala dan kamu mendengar suara dari kegelapan, semua pemimpin sukumu dan pemuka masyarakat datang kepada saya.
\par 24 Kata mereka, 'TUHAN Allah kita telah menunjukkan kebesaran dan kemuliaan-Nya ketika kami mendengar Dia berbicara dari api. Hari ini kami telah menyaksikan bahwa memang mungkin bagi manusia untuk tetap hidup, walaupun Allah telah berbicara kepadanya.
\par 25 Tetapi mengapa kami harus mempertaruhkan nyawa kami lagi? Api yang dahsyat itu akan membinasakan kami! Kami yakin bahwa kami akan mati bila mendengar TUHAN Allah berbicara lagi.
\par 26 Pernahkah ada manusia yang tetap hidup setelah mendengar Allah yang hidup berbicara dari api?
\par 27 Sebab itu, pergilah Musa, dengarkan segala yang dikatakan TUHAN Allah kita, lalu sampaikanlah kepada kami apa yang dikatakan-Nya kepadamu. Kami akan mendengarkan dan taat.'
\par 28 Mendengar itu, TUHAN berkata kepada saya, 'Aku telah mendengar apa yang dikatakan bangsa itu, dan benarlah kata mereka.
\par 29 Coba kalau mereka selalu berperasaan demikian! Coba kalau mereka selalu menghormati Aku dan mentaati semua perintah-Ku, sehingga mereka dan keturunan mereka sejahtera untuk selama-lamanya!
\par 30 Pergilah sekarang dan suruhlah bangsa itu kembali ke kemahnya masing-masing.
\par 31 Tetapi engkau, Musa, kembalilah dan tinggallah di sini bersama Aku; kepadamu akan Kuberikan segala hukum dan perintah-Ku. Ajarkanlah kepada mereka, supaya mereka mentaatinya di negeri yang Kuberikan kepada mereka.'
\par 32 Saudara-saudara, taatilah dengan setia segala yang diperintahkan TUHAN Allahmu kepadamu. Jangan melanggar satu pun dari perintah-perintah itu.
\par 33 Lakukanlah semua yang diperintahkan TUHAN Allahmu kepadamu supaya kamu sejahtera dan dapat menetap di negeri yang akan kamu diami."

\chapter{6}

\par 1 "TUHAN Allahmu menyuruh saya mengajarkan semua hukum ini kepadamu supaya kamu melakukannya di negeri yang tak lama lagi kamu diami.
\par 2 Hendaklah kamu dan keturunanmu seumur hidup menghormati TUHAN Allahmu dan mentaati semua hukum-Nya yang saya sampaikan kepadamu, supaya kamu panjang umur di negeri itu.
\par 3 Sebab itu dengarlah saudara-saudara, lakukanlah itu dengan setia, supaya kamu sejahtera dan menjadi bangsa yang besar di negeri yang kaya dan subur, seperti dijanjikan TUHAN Allah leluhur kita.
\par 4 Saudara-saudara, ingatlah! Hanya TUHAN, dan TUHAN saja Allah kita!
\par 5 Cintailah TUHAN Allahmu dengan sepenuh hatimu: Tunjukkanlah itu dalam cara hidupmu dan dalam perbuatanmu.
\par 6 Jangan sekali-kali melupakan perintah-perintah yang saya berikan kepadamu hari ini.
\par 7 Ajarkanlah kepada anak-anakmu. Hendaklah kamu membicarakannya di dalam rumah dan di luar rumah, waktu beristirahat dan waktu bekerja.
\par 8 Ikatkanlah pada lenganmu dan pasanglah pada dahimu untuk diingat-ingat.
\par 9 Tuliskanlah di tiang pintu rumahmu dan di pintu gerbangmu."
\par 10 "TUHAN Allahmu berjanji kepada nenek moyangmu Abraham, Ishak dan Yakub bahwa Ia akan memberikan negeri itu kepadamu. Kota-kotanya besar dan makmur, dan bukan kamu yang mendirikannya.
\par 11 Rumah-rumahnya penuh dengan barang-barang yang baik, dan bukan kamu yang mengisinya. Kamu akan menemukan sumur-sumur, dan bukan kamu yang menggalinya. Kamu akan mendapat kebun-kebun anggur dan zaitun, dan bukan kamu yang menanamnya. Apabila TUHAN membawa kamu ke negeri itu, dan kamu mempunyai segala yang kamu inginkan untuk dimakan,
\par 12 jagalah baik-baik supaya kamu jangan melupakan TUHAN yang sudah membebaskan kamu dari Mesir, tempat kamu diperbudak.
\par 13 Hormatilah TUHAN, Allahmu, dan berbaktilah kepada Dia saja, dan bila kamu bersumpah, lakukanlah demi nama-Nya saja.
\par 14 Jangan menyembah ilah-ilah lain yang disembah bangsa-bangsa yang tinggal di sekitarmu.
\par 15 Bila kamu menyembah ilah-ilah lain, maka kamu akan ditimpa kemarahan TUHAN Allahmu dan dibinasakan sama sekali, sebab TUHAN Allahmu yang ada di tengah-tengahmu tak mau disamakan dengan apa pun.
\par 16 Janganlah mencobai TUHAN Allahmu seperti yang kamu lakukan di Masa.
\par 17 Taatilah semua hukum yang diberikan TUHAN Allahmu kepadamu.
\par 18 Lakukanlah apa yang menurut TUHAN baik dan adil, maka kamu akan sejahtera. Tanah subur yang dijanjikan TUHAN kepada nenek moyangmu, akan menjadi milikmu,
\par 19 dan kamu akan mengusir musuh-musuhmu seperti yang dijanjikan TUHAN.
\par 20 Di kemudian hari anak-anakmu akan bertanya kepadamu, 'Mengapa TUHAN Allah kita menyuruh kita mentaati semua perintah itu?'
\par 21 Maka kamu harus menjawab begini, 'Dahulu kita menjadi hamba raja Mesir, lalu TUHAN membebaskan kita dengan kekuatan besar.
\par 22 Kami sendiri menyaksikan TUHAN melakukan keajaiban-keajaiban dan hal-hal yang mengerikan terhadap orang Mesir dan raja mereka serta semua pejabatnya.
\par 23 Ia membebaskan kita dari Mesir untuk membawa kita ke sini, dan tanah ini diberi-Nya kepada kita seperti yang dijanjikan-Nya kepada nenek moyang kita.
\par 24 TUHAN Allah kita menyuruh kita melakukan semua perintah itu dan menghormati Dia. Kalau kita berbuat begitu, Ia akan selalu memelihara bangsa kita supaya tetap hidup sejahtera.
\par 25 Kalau kita dengan setia mentaati segala yang diperintahkan TUHAN Allah kita kepada kita, maka Ia akan berkenan kepada kita.'"

\chapter{7}

\par 1 "TUHAN Allahmu akan membawa kamu ke negeri yang bakal kamu diami, dan banyak bangsa akan diusir-Nya dari situ. Di depan matamu Ia akan mengusir tujuh bangsa yang lebih besar dan lebih kuat dari kamu, yaitu orang-orang Het, Girgasi, Amori, Kanaan, Feris, Hewi dan Yebus.
\par 2 Apabila TUHAN Allahmu menyerahkan bangsa-bangsa itu ke dalam kuasamu dan kamu mengalahkan mereka, bunuhlah mereka semua. Jangan mengasihani mereka atau membuat perjanjian dengan mereka.
\par 3 Kamu tak boleh kawin dengan mereka, dan jangan izinkan anak-anakmu kawin dengan mereka,
\par 4 supaya mereka jangan menjauhkan anak-anakmu dari TUHAN untuk menyembah ilah-ilah lain. Kalau kamu menyembah ilah-ilah lain, TUHAN marah dan kamu akan segera dibinasakan-Nya.
\par 5 Sebab itu, robohkanlah mezbah-mezbah bangsa-bangsa itu, hancurkan tiang-tiang batu yang mereka anggap keramat, tumbangkan lambang-lambang Asyera, dan bakarlah patung-patung berhala yang mereka sembah.
\par 6 Lakukanlah semuanya itu karena kamu milik TUHAN Allahmu. Dari segala bangsa di muka bumi, kamulah yang dipilih TUHAN Allahmu untuk menjadi umat-Nya yang istimewa.
\par 7 Kamu dicintai dan dipilih TUHAN, bukan karena kamu lebih besar dari bangsa-bangsa lain; sesungguhnya kamu adalah bangsa yang paling kecil di muka bumi.
\par 8 Tetapi TUHAN memilih kamu karena Ia mengasihi kamu dan ingin menepati janji yang dibuat-Nya dengan nenek moyangmu. Itulah sebabnya Ia telah menyelamatkan kamu dengan kuasa yang besar dan membebaskan kamu dari perhambaan kepada raja Mesir.
\par 9 Ingatlah bahwa TUHAN Allahmu adalah satu-satunya Allah, dan Ia Allah yang setia. TUHAN memenuhi janji-Nya dan menunjukkan kasih-Nya yang tetap sampai seribu keturunan kepada orang yang mencintai Dia dan taat kepada perintah-perintah-Nya.
\par 10 Tetapi dengan tidak segan-segan Ia menghukum mereka yang benci kepada-Nya.
\par 11 Jadi, taatilah semua hukum yang saya sampaikan kepadamu hari ini."
\par 12 "Kalau kamu perhatikan perintah-perintah itu, dan melakukannya dengan setia, maka TUHAN Allahmu pun akan setia kepada perjanjian yang dibuat-Nya dengan kamu. Ia akan menunjukkan kasih-Nya yang tetap kepada kamu, seperti yang dijanjikan-Nya kepada leluhurmu.
\par 13 Ia akan mengasihi dan memberkati kamu, sehingga anakmu banyak dan jumlahmu bertambah-tambah. Ia akan memberkati ladang-ladangmu, sehingga menghasilkan gandum, anggur dan minyak zaitun. Ia akan memberkati kamu sehingga banyak sapi dan kambing dombamu. Semua berkat itu akan kamu terima di negeri yang diberikan TUHAN kepadamu, sesuai dengan janji-Nya kepada nenek moyangmu.
\par 14 Dari semua bangsa di dunia, tidak satu pun mendapat berkat begitu berlimpah-limpah seperti kamu. Dari kamu semua dan dari segala ternakmu tidak satu pun akan mandul.
\par 15 Kamu akan dilindungi TUHAN dari segala penyakit dan dijauhkan dari segala bencana yang kamu saksikan di Mesir, tetapi bencana-bencana itu akan ditimpakan-Nya kepada musuh-musuhmu.
\par 16 Setiap bangsa yang diserahkan TUHAN Allahmu ke dalam kuasamu harus kamu binasakan; jangan menunjukkan kasihan kepada mereka. Jangan menyembah ilah-ilah mereka, karena perbuatan itu mendatangkan celaka kepadamu.
\par 17 Barangkali kamu berpikir bangsa-bangsa itu lebih besar jumlahnya sehingga kamu tak dapat mengusir mereka.
\par 18 Tetapi jangan takut kepada mereka; ingatlah apa yang dilakukan TUHAN Allahmu terhadap raja Mesir dan seluruh rakyatnya.
\par 19 Ingatlah bencana-bencana hebat yang kamu saksikan sendiri, keajaiban-keajaiban, dan kekuatan besar yang digunakan TUHAN Allahmu untuk membebaskan kamu. Begitu pula TUHAN Allahmu akan bertindak terhadap semua bangsa yang kamu takuti.
\par 20 Bahkan TUHAN Allahmu akan menimbulkan kegemparan di antara mereka, sehingga orang-orang yang dapat melarikan diri dan bersembunyi juga akan binasa.
\par 21 Jadi janganlah takut kepada bangsa-bangsa itu. TUHAN Allahmu melindungi kamu. Ia adalah Allah yang agung dan dahsyat.
\par 22 Bangsa-bangsa itu akan diusir-Nya sedikit demi sedikit dari hadapanmu. Kamu tak boleh memusnahkan mereka dengan segera, supaya binatang-binatang buas jangan bertambah banyak dan menjadi ancaman bagimu.
\par 23 Begitulah TUHAN Allahmu akan menyerahkan musuh-musuhmu kepadamu. Ia akan mengacaubalaukan mereka sampai mereka binasa.
\par 24 Ia akan menyerahkan raja-raja mereka kepadamu. Mereka akan kamu bunuh, dan nama mereka akan dilupakan. Tak seorang pun dapat bertahan menghadapi kamu; mereka semua akan kamu musnahkan.
\par 25 Bakarlah patung-patung berhala yang mereka sembah. Jangan menginginkan perak atau emas yang melekat padanya dan jangan mengambilnya untuk dirimu sendiri. Kalau kamu melanggar perintah itu kamu akan mati, sebab TUHAN Allahmu membenci penyembahan patung berhala.
\par 26 Jangan menempatkan sebuah patung berhala di dalam rumahmu, supaya kutuk yang menimpanya jangan menimpa kamu juga. Hal itu harus kamu benci dan pandang rendah, karena dikutuk oleh TUHAN."

\chapter{8}

\par 1 "Semua hukum yang saya berikan kepadamu hari ini harus kamu lakukan dengan setia, supaya kamu tetap hidup, bertambah banyak dan mendiami tanah yang dijanjikan TUHAN kepada nenek moyangmu.
\par 2 Ingatlah! Empat puluh tahun lamanya TUHAN Allahmu memimpin kamu dalam perjalanan jauh melewati padang gurun. Perjalanan itu dimaksudkan TUHAN untuk mencobai kamu, supaya Ia dapat melihat apa yang terkandung dalam hatimu dan apakah kamu akan mentaati perintah-perintah-Nya.
\par 3 Ia membiarkan kamu kelaparan, lalu kamu diberi-Nya manna untuk makanan. Kamu dan leluhurmu tak pernah makan makanan itu sebelumnya. TUHAN berbuat begitu untuk mengajar kamu bahwa manusia tidak hidup dari makanan saja, melainkan dari segala sesuatu yang dikatakan TUHAN.
\par 4 Selama empat puluh tahun ini pakaianmu tidak menjadi usang dan kakimu tidak menjadi bengkak.
\par 5 Ingatlah bahwa TUHAN Allahmu mengajari kamu seperti seorang ayah mengajari anak-anaknya.
\par 6 Jadi, lakukanlah segala yang diperintahkan TUHAN Allahmu. Hiduplah menurut hukum-hukum-Nya dan hormatilah Dia.
\par 7 TUHAN Allahmu membawa kamu ke negeri yang subur; negeri yang banyak sungai dan mata airnya, dengan arus air di bawah tanah yang memancur ke lembah-lembah dan bukit-bukit.
\par 8 Negeri itu menghasilkan macam-macam gandum, anggur, buah ara, buah delima, pohon zaitun dan madu.
\par 9 Di situ kamu tak akan kelaparan atau kekurangan. Batu karangnya mengandung besi, dan dari bukit-bukitnya kamu dapat menggali tembaga.
\par 10 Kamu akan makan sepuas hatimu, dan bersyukur kepada TUHAN Allahmu untuk tanah subur yang diberikan-Nya kepadamu."
\par 11 "Jagalah baik-baik jangan sampai kamu melupakan TUHAN Allahmu; jangan abaikan satu pun dari semua hukum-Nya yang saya berikan kepadamu hari ini.
\par 12 Kalau kamu sudah makan sepuas hatimu dan sudah membangun rumah-rumah yang baik untuk tempat kediamanmu,
\par 13 dan kalau sudah bertambah sapi dan kambing dombamu, perak dan emasmu serta segala harta milikmu yang lain,
\par 14 jagalah baik-baik supaya kamu tidak menjadi sombong dan melupakan TUHAN Allahmu yang membebaskan kamu dari Mesir, tempat kamu diperbudak.
\par 15 TUHAN memimpin kamu melalui padang gurun yang luas dan dahsyat, yang banyak ular berbisa dan kalajengkingnya. Di tanah yang kering tanpa air Ia membuat air mengalir dari batu, supaya kamu dapat minum sepuas-puasnya.
\par 16 Di padang gurun Ia memberi manna untuk makananmu, makanan yang tidak dikenal nenek moyangmu. TUHAN melakukan itu untuk mencobai kamu dan merendahkan hatimu, supaya pada akhirnya Ia dapat berbuat baik kepadamu.
\par 17 Jadi, jangan sekali-kali berpikir bahwa dengan usaha dan kekuatanmu sendiri kamu menjadi kaya.
\par 18 Ingat bahwa TUHAN Allahmulah yang memberi kekayaan itu kepadamu. Hal itu dilakukan-Nya karena sampai hari ini Ia masih setia kepada perjanjian yang dibuat-Nya dengan nenek moyangmu.
\par 19 Jangan lupa kepada TUHAN Allahmu dan jangan berbalik kepada ilah-ilah lain untuk memuja dan mengabdi kepada mereka. Hari ini aku mengingatkan kamu bahwa kalau kamu melakukannya juga, kamu pasti dibinasakan TUHAN.
\par 20 Kalau kamu tidak taat kepada TUHAN Allahmu, kamu pun akan dibinasakan-Nya, seperti bangsa-bangsa yang dimusnahkan-Nya di depan matamu."

\chapter{9}

\par 1 "Dengarlah Saudara-saudara! Hari ini kamu menyeberangi Sungai Yordan untuk menduduki daerah bangsa-bangsa yang lebih besar dan lebih kuat dari kamu. Kota-kota mereka besar-besar dengan tembok-temboknya yang menjulang setinggi langit.
\par 2 Orang-orangnya besar dan kuat perawakannya; mereka itu raksasa, dan kamu sudah mendengar bahwa tak seorang pun dapat melawan mereka.
\par 3 Tetapi sekarang, kamu akan menyaksikan sendiri bagaimana TUHAN Allahmu berjalan di depan kamu seperti api yang menghanguskan. Ia akan mengalahkan mereka di depan matamu, sehingga kamu dapat mengusir dan membinasakan mereka dengan cepat seperti yang dijanjikan TUHAN.
\par 4 Sesudah TUHAN Allahmu mengusir mereka untukmu, jangan menyangka bahwa kamu dibawa-Nya ke sana untuk memiliki tanah itu karena kamu baik dan melakukan kehendak TUHAN Allahmu sehingga pantas menerimanya. Sekali-kali tidak! TUHAN mengusir bangsa-bangsa itu untuk kamu karena mereka jahat. TUHAN membiarkan kamu mengambil tanah mereka, karena Ia ingin memenuhi janji-Nya kepada nenek moyangmu Abraham, Ishak dan Yakub.
\par 6 Yakinlah bahwa TUHAN Allahmu tidak menyerahkan tanah subur itu kepadamu oleh karena kamu pantas menerimanya. Tidak! Kamu ini bangsa yang keras kepala.
\par 7 Ingatlah bagaimana kamu membuat TUHAN Allahmu marah di padang gurun. Sejak kamu meninggalkan Mesir sampai kamu tiba di sini, kamu menentang TUHAN.
\par 8 Bahkan di Gunung Sinai kamu membuat TUHAN marah sekali, sehingga Ia mau membinasakan kamu.
\par 9 Saya mendaki gunung itu untuk menerima batu perjanjian yang dibuat TUHAN dengan kamu. Empat puluh hari dan empat puluh malam lamanya saya tinggal di atas gunung itu tanpa makan atau minum.
\par 10 Kemudian TUHAN memberi saya kedua batu yang telah ditulisi oleh Allah sendiri. Pada batu itu tertulis kata-kata yang diucapkan-Nya dari tengah-tengah api kepadamu, ketika kamu berkumpul di kaki gunung.
\par 11 Ya, sesudah lewat empat puluh hari dan empat puluh malam, TUHAN memberikan kepada saya kedua batu perjanjian itu.
\par 12 Lalu TUHAN berkata kepada saya, 'Turunlah segera, sebab bangsamu yang kaubawa keluar dari Mesir telah berbuat jahat. Mereka sudah menyimpang dari perintah-perintah-Ku, dan membuat patung untuk disembah.'
\par 13 TUHAN juga berkata kepada saya, 'Aku tahu bangsa itu amat keras kepala.
\par 14 Jangan coba menghalangi Aku. Aku hendak membinasakan mereka sehingga mereka tidak diingat lagi. Tetapi engkau akan Kujadikan bapak dari suatu bangsa yang lebih besar dan lebih kuat daripada mereka.'
\par 15 Lalu saya berpaling, dan sambil membawa kedua batu perjanjian dengan kedua tangan, saya turuni gunung yang sedang menyala-nyala.
\par 16 Saya lihat bahwa kamu sudah melanggar perintah TUHAN Allahmu. Kamu sudah berdosa terhadap TUHAN karena membuat bagi dirimu sebuah patung sapi dari logam.
\par 17 Maka di depan matamu kedua batu perjanjian itu saya banting sampai hancur berkeping-keping.
\par 18 Lalu sekali lagi saya bersujud di depan TUHAN di puncak gunung selama empat puluh hari dan empat puluh malam, tanpa makan atau minum. Itu saya lakukan karena kamu telah berdosa terhadap TUHAN dengan melakukan apa yang dianggap-Nya jahat, sehingga Ia marah.
\par 19 Saya takut kepada kemarahan TUHAN yang menyala-nyala terhadap kamu sehingga kamu mau dibinasakan-Nya, tetapi kali ini pun TUHAN mendengarkan saya.
\par 20 TUHAN juga marah sekali kepada Harun sehingga Ia mau membunuhnya, maka saya berdoa untuk dia juga.
\par 21 Patung sapi logam buatanmu, saya lemparkan ke dalam api. Lalu saya hancurkan dan tumbuk sampai halus seperti debu, dan debu itu saya lemparkan ke dalam anak sungai yang mengalir dari gunung itu.
\par 22 Juga di Tabera, Masa dan Kibrot-Taawa, kamu membuat TUHAN marah.
\par 23 Lalu pada waktu kamu disuruh TUHAN meninggalkan Kades-Barnea untuk maju dan menduduki tanah yang akan diberikan-Nya kepadamu, kamu menentang perintah TUHAN Allahmu; kamu tidak mau percaya atau taat kepada-Nya.
\par 24 Sejak saya kenal kamu, kamu selalu menentang TUHAN.
\par 25 Saya tahu TUHAN bertekad hendak membinasakan kamu. Maka selama empat puluh hari dan empat puluh malam saya sujud di depan TUHAN
\par 26 dan berdoa begini: Ya TUHAN Yang Mahatinggi, janganlah binasakan umat milik-Mu ini, bangsa yang Kaubebaskan dan Kauantar keluar dari Mesir dengan kekuatan dan kekuasaan-Mu yang besar.
\par 27 Ingatlah akan hamba-hamba-Mu Abraham, Ishak dan Yakub, dan jangan perhatikan sifat keras kepala, kejahatan dan dosa bangsa ini.
\par 28 Kalau umat-Mu Kaumusnahkan juga, maka orang Mesir akan berkata bahwa Engkau tak sanggup membawa bangsa-Mu ke negeri yang Kaujanjikan kepada mereka, dan bahwa Engkau membawa mereka ke padang gurun untuk membunuh mereka, karena Engkau benci kepada mereka.
\par 29 Tetapi ingatlah, TUHAN, bahwa mereka ini bangsa yang Kaupilih menjadi umat-Mu, dan Kaubawa dari Mesir dengan kekuatan dan kekuasaan-Mu yang besar."

\chapter{10}

\par 1 "Kemudian TUHAN berkata kepada saya, 'Musa, pahatlah dua lempeng batu seperti yang pertama, dan buatlah juga sebuah peti kayu untuk tempatnya, lalu datanglah kepada-Ku di puncak gunung.
\par 2 Pada batu-batu itu akan Kutulis kata-kata yang sama seperti pada batu yang sudah kaupecahkan; kemudian batu-batu itu harus kaumasukkan ke dalam peti itu.'
\par 3 Lalu saya membuat peti dari kayu akasia dan memahat dua batu seperti yang dahulu dan membawanya ke atas gunung.
\par 4 Pada batu itu TUHAN menulis Sepuluh Perintah sama dengan yang semula, yaitu perintah-perintah yang diberikan-Nya kepada kamu, ketika Ia berbicara dari tengah-tengah api pada hari kamu berkumpul di kaki gunung. Sesudah itu TUHAN memberi batu tulis itu kepada saya
\par 5 dan saya kembali menuruni gunung. Lalu sesuai dengan perintah TUHAN, kedua batu itu saya masukkan ke dalam peti yang telah saya buat--dan sejak saat itu sampai sekarang kedua batu tulis itu masih ada di situ."
\par 6 (Kemudian orang Israel berangkat dari sumur-sumur orang Yaakan, dan pergi ke Mosera. Di situ Harun meninggal dan dimakamkan. Eleazar anaknya menggantikan dia sebagai imam.
\par 7 Dari situ mereka pergi ke Gudgod, lalu terus ke Yotbata, daerah yang banyak airnya.
\par 8 Di gunung itu TUHAN mengkhususkan suku Lewi untuk mengurus Peti Perjanjian dan melayani TUHAN, serta mengucapkan berkat atas nama-Nya. Itulah tugas mereka sampai sekarang.
\par 9 Oleh karena itu suku Lewi tidak menerima tanah pusaka seperti suku-suku lain, sebab Tuhanlah bagian warisan mereka, seperti yang dikatakan TUHAN Allahmu kepada mereka.)
\par 10 "Empat puluh hari dan empat puluh malam saya tinggal di atas gunung seperti yang saya lakukan pertama kalinya. Sekali lagi TUHAN mendengarkan saya, dan Ia setuju untuk tidak membinasakan kamu.
\par 11 Lalu Ia menyuruh saya pergi untuk memimpin kamu supaya kamu dapat memiliki tanah yang dijanjikan-Nya kepada nenek moyangmu."
\par 12 "Sekarang, Saudara-saudara, perhatikanlah apa yang dituntut TUHAN Allahmu dari kamu: Hormatilah TUHAN Allahmu dan lakukanlah segala perintah-Nya. Cintailah Dia dan beribadatlah kepada-Nya dengan seluruh jiwa ragamu.
\par 13 Taatilah segala perintah TUHAN yang hari ini saya sampaikan kepadamu demi kebaikanmu sendiri.
\par 14 TUHAN Allahmu memiliki langit tertinggi; bumi pun milik-Nya beserta segala yang ada di atasnya.
\par 15 Tetapi kasih TUHAN kepada leluhurmu begitu besar, sehingga dari segala bangsa kamulah yang dipilih-Nya, dan sampai sekarang pun kamu umat-Nya yang terpilih.
\par 16 Jadi mulai saat ini kamu harus taat kepada TUHAN dan janganlah berkeras kepala lagi.
\par 17 TUHAN Allahmu ada di atas segala ilah dan melebihi segala kuasa. Ia Allah yang agung dan berkuasa yang harus ditaati. Ia tidak suka berpihak dan tidak juga menerima suap.
\par 18 Ia membela hak yatim piatu dan janda supaya mereka diperlakukan dengan adil; Ia mengasihi orang asing yang hidup bersama bangsa kita dan memberi mereka makanan dan pakaian.
\par 19 Maka kamu juga harus menunjukkan kasihmu kepada orang-orang asing itu, sebab dahulu kamu pun orang asing di Mesir.
\par 20 Hormatilah TUHAN Allahmu dan beribadatlah kepada Dia saja. Hendaklah kamu tetap setia kepada-Nya dan bersumpah demi nama-Nya saja.
\par 21 Pujilah Dia sebab Ia Allahmu. Dengan mata kepalamu sendiri kamu telah melihat perbuatan-perbuatan hebat dan dahsyat yang dilakukan-Nya untukmu.
\par 22 Ketika nenek moyangmu pergi ke Mesir, jumlah mereka hanya tujuh puluh orang. Tetapi TUHAN Allahmu telah membuat jumlahmu sebanyak bintang-bintang di langit."

\chapter{11}

\par 1 "Cintailah TUHAN Allahmu, dan taatilah selalu segala perintah-Nya.
\par 2 Ingatlah sekarang apa yang kamu ketahui tentang TUHAN dari pengalamanmu. Anak-anakmu tidak mempunyai pengalaman itu. Kamulah yang telah menyaksikan keagungan TUHAN Allahmu, kekuatan-Nya, kekuasaan-Nya
\par 3 dan keajaiban-keajaiban yang dilakukan-Nya. Kamu melihat sendiri apa yang dilakukan-Nya terhadap raja Mesir dan seluruh negerinya.
\par 4 Kamu melihat bagaimana TUHAN memusnahkan tentara Mesir bersama semua kuda dan kereta perangnya. Mereka semua ditenggelamkan di Laut Gelagah ketika sedang mengejar kamu.
\par 5 Kamu tahu apa yang dilakukan TUHAN bagimu di padang gurun sebelum kamu sampai di sini.
\par 6 Kamu juga ingat apa yang dilakukan TUHAN terhadap Datan dan Abiram, anak-anak Eliab dari suku Ruben. Di depan kamu semua, bumi terbelah lalu menelan mereka bersama keluarga, kemah, semua hamba dan ternak mereka.
\par 7 Ya, kamulah yang dengan mata kepala sendiri telah melihat semua keajaiban yang dilakukan TUHAN."
\par 8 "Taatilah semua hukum yang saya sampaikan kepadamu hari ini, maka kamu dapat menyeberangi sungai dan masuk ke negeri yang tak lama lagi kamu diami.
\par 9 Dan kamu akan hidup lama di negeri yang kaya dan subur, yang dijanjikan TUHAN kepada nenek moyangmu dan kepada keturunan mereka.
\par 10 Negeri yang kamu diami nanti tidak seperti tanah Mesir, tempat kamu dahulu tinggal. Di sana kamu harus bekerja keras untuk mengairi ladang-ladang pada waktu menabur benih.
\par 11 Tetapi negeri yang akan kamu masuki terdiri dari pegunungan dan lembah-lembah; tanahnya diairi oleh hujan.
\par 12 TUHAN Allahmu memelihara negeri itu dan menjagainya sepanjang tahun.
\par 13 Jadi taatilah perintah-perintah yang saya sampaikan kepadamu hari ini; cintailah TUHAN Allahmu dan beribadatlah kepada-Nya dengan seluruh jiwa ragamu.
\par 14 Kalau kamu berbuat begitu, maka TUHAN akan menurunkan hujan atas tanahmu pada musimnya, sehabis kamu panen, dan pada waktu kamu mulai menanam. Maka kamu akan mempunyai cukup gandum, anggur dan minyak zaitun,
\par 15 serta rumput untuk ternakmu. Segala makanan yang kamu perlukan akan tersedia bagimu.
\par 16 Jangan kamu mau dijauhkan dari TUHAN untuk menyembah dan mengabdi kepada ilah-ilah lain.
\par 17 Kalau kamu melanggar perintah itu, TUHAN akan marah kepadamu. Ia tidak akan menurunkan hujan; tanahmu menjadi kering kerontang sehingga tanaman tidak memberi hasil. Maka dalam waktu singkat kamu mati di tanah subur yang diberikan TUHAN kepadamu.
\par 18 Ingatlah dan perhatikanlah perintah-perintah itu. Ikatkan pada lenganmu dan pasanglah pada dahimu untuk diingat-ingat.
\par 19 Ajarkanlah kepada anak-anakmu. Bicarakanlah di dalam rumah dan di luar rumah, waktu beristirahat dan waktu bekerja.
\par 20 Tuliskanlah di tiang pintu rumahmu dan di pintu gerbangmu.
\par 21 Maka kamu dan anak-anakmu akan panjang umur di negeri yang dijanjikan TUHAN Allahmu kepada nenek moyangmu. Selama ada langit di atas bumi, selama itu pula kamu akan diam di tanah itu.
\par 22 Taatilah dengan setia semua hukum yang saya sampaikan kepadamu: Cintailah TUHAN Allahmu, lakukanlah segala yang diperintahkan-Nya kepadamu, dan hendaklah kamu tetap setia kepada-Nya.
\par 23 Maka TUHAN akan mengusir semua bangsa itu dari hadapanmu, dan kamu akan mendiami tanah kepunyaan bangsa-bangsa yang lebih besar dan lebih kuat daripadamu.
\par 24 Setiap wilayah yang kamu jejaki akan menjadi milikmu. Wilayahmu terbentang dari padang gurun di selatan sampai ke Pegunungan Libanon di utara, dan dari Sungai Efrat di timur sampai ke Laut Tengah di barat.
\par 25 Di seluruh negeri itu TUHAN Allahmu akan membuat orang-orang ketakutan terhadapmu seperti yang dijanjikan-Nya, sehingga tak seorang pun dapat menghalangi kamu.
\par 26 Hari ini kamu boleh memilih antara berkat dan kutuk.
\par 27 Kamu akan diberkati kalau taat kepada perintah-perintah TUHAN Allahmu, yang hari ini saya sampaikan kepadamu.
\par 28 Tetapi kamu akan dikutuk kalau tidak taat kepada perintah-perintah TUHAN Allahmu, melainkan berbalik untuk menyembah ilah-ilah lain yang tidak pernah kamu sembah.
\par 29 Sesudah TUHAN Allahmu membawa kamu masuk ke negeri yang kamu diami nanti, berkat itu harus kamu ucapkan dari Gunung Gerizim dan kutuk itu dari Gunung Ebal.
\par 30 (Kedua gunung itu terletak di sebelah barat Sungai Yordan di wilayah orang Kanaan yang tinggal di Lembah Yordan, menghadap ke barat tak jauh dari pohon-pohon tempat ibadat di More dekat kota Gilgal.)
\par 31 Tak lama lagi kamu menyeberangi Sungai Yordan dan menduduki tanah yang diberikan TUHAN Allahmu kepadamu. Sesudah kamu merebut tanah itu dan berdiam di situ,
\par 32 kamu harus mentaati semua hukum yang saya sampaikan kepadamu hari ini."

\chapter{12}

\par 1 "Inilah hukum-hukum yang harus kamu taati seumur hidupmu di negeri yang diberikan kepadamu oleh TUHAN, Allah nenek moyangmu.
\par 2 Sesudah masuk ke negeri itu kamu harus menghancurkan tempat-tempat pemujaan di atas gunung-gunung tinggi, di bukit-bukit dan di bawah pohon-pohon rindang, tempat bangsa yang tinggal di situ menyembah ilah-ilah mereka.
\par 3 Robohkanlah mezbah-mezbah mereka dan remukkan tiang-tiang batu yang mereka anggap keramat. Bakarlah lambang-lambang Asyera, dan hancurkan patung-patung berhala mereka, sehingga ilah-ilah itu tak akan disembah lagi di tempat-tempat itu.
\par 4 Pada waktu kamu menyembah TUHAN Allahmu, janganlah meniru cara orang-orang itu menyembah ilah-ilah mereka.
\par 5 Dari seluruh wilayah suku-suku Israel, TUHAN Allahmu akan memilih satu tempat supaya bangsa kita dapat datang ke hadapan-Nya dan menyembah Dia.
\par 6 Ke tempat itulah harus kamu bawa kurban-kurban bakaran serta kurban-kurban yang lain, persembahan sepersepuluhanmu serta persembahan-persemb lain, kurban pembayar kaulmu, kurban sukarelamu dan anak sapi dan kambing dombamu yang pertama lahir.
\par 7 Di situ juga, di hadapan TUHAN Allahmu, kamu bersama-sama dengan keluargamu harus makan dan bergembira atas usaha-usahamu yang berhasil karena diberkati TUHAN Allahmu.
\par 8 Kalau saat itu sudah datang, kamu jangan lagi berbuat seperti sekarang. Sampai sekarang kamu beribadat sesuka hatimu,
\par 9 karena kamu belum masuk ke negeri yang diberikan TUHAN Allahmu kepadamu, tempat kamu dapat menetap dengan aman.
\par 10 Sesudah kamu menyeberangi Sungai Yordan, TUHAN Allahmu akan menolong kamu menduduki dan mendiami tanah di seberang sungai itu. Ia akan melindungi kamu dari segala musuh, dan kamu akan hidup dengan aman dan tentram.
\par 11 Kemudian TUHAN Allahmu akan memilih suatu tempat di mana Ia harus disembah. Ke situlah kamu harus pergi membawa segala yang telah kuperintahkan, yaitu kurban bakaran dan kurban-kurban lain, persembahan sepersepuluhan dan persembahan khusus yang kamu janjikan kepada TUHAN.
\par 12 Bergembiralah di hadapan TUHAN Allahmu, bersama-sama dengan anak-anakmu, hamba-hambamu dan orang-orang Lewi yang tinggal di kota-kotamu, oleh karena orang Lewi tidak mendapat bagian tanah di negerimu.
\par 13 Kamu tak boleh mempersembahkan kurbanmu di sembarang tempat yang kamu pilih sendiri,
\par 14 tetapi hanya di tempat yang dipilih TUHAN di daerah salah satu sukumu. Di situ saja kamu harus mempersembahkan kurban bakaran dan melakukan hal lain yang saya perintahkan kepadamu.
\par 15 Tetapi kamu bebas untuk memotong ternakmu dan makan dagingnya di mana saja kamu tinggal. Kamu boleh makan sebanyak yang diberikan TUHAN Allahmu kepadamu. Setiap orang, baik yang bersih maupun yang najis, boleh makan daging itu, seperti kamu boleh makan daging kijang atau daging rusa.
\par 16 Hanya darahnya tidak boleh dimakan, tetapi harus ditumpahkan ke tanah seperti air.
\par 17 Di tempat kediamanmu kamu tak boleh makan persembahan sepersepuluhan dari gandummu, anggurmu, minyak zaitunmu, sapi atau kambing dombamu yang pertama lahir, kurban pembayar kaulmu, kurban sukarelamu atau persembahan lain.
\par 18 Kamu dan anak-anakmu, bersama-sama dengan hamba-hambamu dan orang-orang Lewi yang tinggal di kotamu, harus makan persembahan itu di hadapan TUHAN Allahmu, di tempat yang dipilih TUHAN Allahmu. Dan di tempat itu kamu harus bergembira karena segala usahamu.
\par 19 Jagalah jangan sampai kamu membiarkan kaum Lewi terlantar selama kamu hidup di negeri itu.
\par 20 Sesudah TUHAN Allahmu memperluas batas-batas negerimu seperti yang dijanjikan-Nya, kamu boleh makan daging sesuka hatimu.
\par 21 Kalau tempat yang dipilih TUHAN Allahmu terlalu jauh dari tempatmu, kamu boleh memotong seekor sapi, domba atau kambing yang diberikan TUHAN kepadamu, dan memakan dagingnya di tempatmu dengan sesuka hatimu, seperti yang sudah saya katakan kepadamu.
\par 22 Setiap orang, baik yang bersih maupun yang najis, boleh makan daging itu, seperti kamu boleh makan daging rusa atau daging kijang.
\par 23 Tetapi daging yang masih ada darahnya tak boleh dimakan, karena nyawa ada di dalam darah, dan kamu tak boleh makan nyawa bersama dengan dagingnya.
\par 24 Darahnya sama sekali tak boleh dimakan, tetapi harus ditumpahkan ke tanah seperti air.
\par 25 Kalau kamu mentaati perintah itu, TUHAN akan senang, dan kamu serta keturunanmu akan sejahtera.
\par 26 Segala kurban dan persembahan yang telah kamu janjikan kepada TUHAN, harus kamu bawa ke tempat yang dipilih TUHAN.
\par 27 Kurban bakaran dan kurban yang dimakan dagingnya harus dipersembahkan di atas mezbah TUHAN Allahmu dan darahnya ditumpahkan ke atas mezbah itu.
\par 28 Perhatikanlah dengan baik segala yang sudah saya perintahkan kepadamu, maka kamu melakukan yang baik dan berkenan kepada TUHAN Allahmu, sehingga kamu dan keturunanmu selamat dan sejahtera untuk selama-lamanya."
\par 29 "TUHAN Allahmu akan membinasakan bangsa-bangsa yang tinggal di negeri itu, sehingga kamu dapat menduduki dan mendiaminya.
\par 30 Sesudah bangsa-bangsa itu dibinasakan, berhati-hatilah jangan sampai kamu meniru mereka menyembah ilah-ilah lain, sebab perbuatan itu mendatangkan bencana.
\par 31 Cara bangsa-bangsa itu menyembah ilah-ilah mereka, janganlah kamu tiru untuk menyembah TUHAN Allahmu, sebab mereka menyembah dengan melakukan perbuatan-perbuatan keji yang dibenci TUHAN. Bahkan anak-anak mereka sendiri mereka bakar di atas mezbah sebagai kurban untuk ilah-ilah mereka.
\par 32 Lakukanlah segala yang saya perintahkan kepadamu; jangan ditambah atau dikurangi.

\chapter{13}

\par 1 Apabila seorang nabi atau tukang mimpi menjanjikan suatu mujizat atau keajaiban
\par 2 dengan maksud supaya kamu menyembah dan mengabdi kepada ilah-ilah yang tidak pernah kamu kenal, biarpun apa yang dijanjikannya itu sungguh-sungguh terjadi,
\par 3 jangan memberi perhatian kepadanya. TUHAN Allahmu memakai orang itu untuk mencobai kamu, untuk melihat apakah kamu betul-betul mencintai TUHAN dengan sepenuhnya atau tidak.
\par 4 Orang semacam itu harus dihukum mati, karena mau menyesatkan kamu dari jalan yang diperintahkan TUHAN Allahmu kepadamu. Ia adalah orang jahat karena menyuruh kamu berontak terhadap TUHAN Allahmu yang membebaskan kamu dari Mesir, tempat kamu diperbudak. Jadi ia harus dibunuh, supaya kejahatan itu diberantas. Tetapi kamu hendaklah mengikuti TUHAN Allahmu dan menghormati-Nya; taatilah Dia dan lakukanlah segala perintah-Nya. Tetaplah setia dan mengabdi kepada-Nya.
\par 6 Apabila saudaramu atau anakmu laki-laki atau perempuan, atau istrimu yang kaukasihi atau kawan karibmu dengan diam-diam membujukmu untuk menyembah ilah-ilah lain yang tidak kamu kenal dan tidak dikenal leluhurmu, ilah-ilah dari bangsa-bangsa yang tinggal dekat dan jauh,
\par 8 jangan dengarkan dia dan jangan biarkan dirimu dibujuk olehnya. Jangan juga mengasihani atau melindungi orang itu.
\par 9 Dia harus dilempari batu sampai mati. Engkau yang harus mulai melempari dia, diikuti oleh seluruh rakyat. Orang yang bersalah itu harus dibunuh, sebab ia mau menjauhkan kamu dari TUHAN Allahmu yang membebaskan kita dari perbudakan di Mesir.
\par 11 Maka seluruh bangsa Israel akan mendengar tentang pembunuhan itu dan menjadi takut, sehingga tak seorang pun berbuat sejahat itu lagi.
\par 12 Apabila kamu nanti tinggal di kota-kota yang diberikan TUHAN Allahmu kepadamu, mungkin tersiar berita
\par 13 bahwa beberapa orang jahat dari bangsamu telah menyesatkan penduduk kota untuk menyembah ilah-ilah yang tidak kamu kenal.
\par 14 Apabila terdengar desas-desus semacam itu, selidikilah dengan teliti. Kalau kejahatan itu benar-benar telah terjadi,
\par 15 maka seluruh penduduk kota itu beserta ternaknya harus dibunuh. Kota itu harus dimusnahkan sama sekali.
\par 16 Seluruh harta benda penduduk kota itu harus dikumpulkan dan ditimbun di tanah lapang, lalu kota dan segala harta bendanya harus dibakar sebagai kurban bagi TUHAN Allahmu. Sesudahnya semua itu harus ditinggalkan menjadi puing dan tak boleh dibangun kembali.
\par 17 Barang-barang terkutuk itu tak boleh disimpan untuk dirimu sendiri, tetapi harus dibinasakan sama sekali. Maka kemarahan TUHAN akan reda dan Ia akan menunjukkan belas kasihan kepadamu. Ia akan bermurah hati kepadamu dan menjadikan kamu bangsa yang besar seperti yang dijanjikan-Nya kepada leluhurmu,
\par 18 asal kamu mentaati segala perintah-Nya yang saya sampaikan kepadamu hari ini, dan melakukan segala yang dikehendaki-Nya."

\chapter{14}

\par 1 "Kamu adalah umat TUHAN Allahmu. Jadi pada waktu kamu berkabung untuk orang mati, janganlah melukai dirimu atau mencukur bagian depan kepalamu seperti yang dilakukan oleh bangsa-bangsa lain.
\par 2 Kamu adalah milik TUHAN Allahmu; dari segala bangsa di atas bumi, kamulah yang dipilih Allah untuk menjadi umat-Nya."
\par 3 "Janganlah makan binatang yang dinyatakan haram oleh TUHAN.
\par 4 Kamu boleh makan binatang-binatang ini: sapi, domba, kambing,
\par 5 rusa, domba hutan, kambing hutan, kijang
\par 6 dan semua binatang yang kukunya terbelah dan memamah biak.
\par 7 Kamu tak boleh makan binatang yang kukunya tidak terbelah dan tidak memamah biak. Jangan makan unta, kelinci atau marmot. Binatang itu haram karena walaupun memamah biak, kukunya tidak terbelah.
\par 8 Jangan makan babi. Binatang itu haram, karena walaupun kukunya terbelah, ia tidak memamah biak. Dagingnya tak boleh dimakan, bangkainya tak boleh disentuh.
\par 9 Kamu boleh makan segala macam ikan yang bersirip dan bersisik.
\par 10 Binatang yang hidup di dalam air tetapi tidak bersirip dan tidak bersisik adalah haram, jadi tak boleh dimakan.
\par 11 Kamu boleh makan segala macam burung yang tidak haram.
\par 12 Tetapi ada beberapa jenis burung yang tidak boleh dimakan, yaitu: burung rajawali, burung hantu, segala jenis elang, nasar, gagak, burung unta, camar, blekok dan segala jenis bangau, undan, burung kasa dan kelelawar.
\par 19 Semua serangga yang bersayap adalah haram, jadi tak boleh dimakan.
\par 20 Binatang bersayap yang tidak haram boleh dimakan.
\par 21 Binatang yang mati dengan sendirinya tak boleh kamu makan, tetapi boleh dimakan oleh orang asing yang tinggal di antara kamu, atau dijual kepada bangsa asing. Kamu adalah umat yang dikhususkan untuk TUHAN Allahmu. Daging anak domba atau anak kambing tak boleh dimasak dengan air susu induknya."
\par 22 "Setiap tahun kamu harus menyisihkan sepersepuluh dari seluruh hasil tanahmu.
\par 23 Di hadapan TUHAN Allahmu, di tempat yang dipilih-Nya, untuk tempat ibadah, kamu harus makan persembahan sepersepuluhan dari gandum, air anggur, minyak zaitun serta anak sapi dan kambing dombamu yang pertama lahir. Lakukanlah hal itu supaya untuk selamanya kamu belajar menghormati TUHAN Allahmu.
\par 24 Kalau tempat yang dipilih TUHAN Allahmu terlalu jauh dari rumahmu, sehingga sepersepuluh dari hasil tanahmu yang diberikan TUHAN Allahmu kepadamu tak dapat kamu bawa ke situ, maka
\par 25 juallah bagian dari hasil tanahmu itu, dan bawalah uangnya ke tempat yang dipilih TUHAN Allahmu.
\par 26 Belanjakanlah uang itu untuk apa saja yang kamu inginkan--sapi atau kambing domba, air anggur atau minuman keras--lalu di tempat itu, di hadapan TUHAN Allahmu, kamu dan keluargamu harus makan bersama dan bersenang-senang.
\par 27 Jangan membiarkan terlantar orang Lewi yang tinggal di kota-kotamu; ingatlah bahwa mereka itu tidak mempunyai tanah sendiri.
\par 28 Pada akhir tiap tahun ketiga, kamu harus membawa sepersepuluh bagian dari hasil tanahmu dan mengumpulkannya di kota-kotamu.
\par 29 Makanan itu untuk orang Lewi karena mereka tidak mempunyai tanah, dan untuk orang asing, anak yatim piatu dan para janda yang hidup di kota-kotamu. Mereka boleh datang dan mengambil segala yang mereka perlukan. Kalau kamu berbuat demikian, maka TUHAN Allahmu akan memberkati segala usahamu."

\chapter{15}

\par 1 "Pada akhir tiap tahun ketujuh kamu harus mengadakan penghapusan hutang.
\par 2 Beginilah caranya. Setiap orang Israel yang meminjamkan uang kepada orang sebangsanya, harus menghapuskan hutang itu, karena menurut keputusan TUHAN sendiri hutang itu sudah dihapuskan, jadi tak boleh ditagih lagi.
\par 3 Kamu boleh menagih uang yang kamu pinjamkan kepada orang asing, tetapi tidak boleh menagih lagi uang yang kamu pinjamkan kepada orang sebangsamu.
\par 4 TUHAN Allahmu akan memberkati kamu di negeri yang diberikan-Nya kepadamu. Maka di antara bangsamu tak akan ada orang miskin,
\par 5 asal kamu taat kepada TUHAN Allahmu, dan dengan teliti melakukan segala yang saya perintahkan kepadamu hari ini.
\par 6 TUHAN Allahmu akan memberkati kamu seperti yang dijanjikan-Nya. Kamu akan meminjamkan uang kepada banyak bangsa, tetapi kamu sendiri tidak akan meminjam; kamu akan menguasai banyak bangsa, tetapi mereka tidak akan menguasai kamu.
\par 7 Apabila kamu nanti sudah berdiam di negeri yang diberikan TUHAN Allahmu kepadamu, lalu di salah satu kotamu terdapat orang sebangsamu yang berkekurangan, janganlah mementingkan dirimu sendiri dan menolak untuk membantu dia.
\par 8 Sebaliknya, kamu harus bermurah hati dan meminjamkan kepada orang itu sebanyak yang diperlukannya.
\par 9 Jangan menolak untuk memberi pinjaman kepadanya dengan alasan bahwa tahun penghapusan hutang sudah dekat. Jangan biarkan pikiran sejahat itu masuk ke dalam hatimu. Kalau kamu tidak mau meminjamkan, orang itu akan berseru kepada TUHAN untuk mengadukan kamu, maka kamu akan dinyatakan bersalah.
\par 10 Berilah kepadanya dengan senang hati dan rela, maka TUHAN Allahmu akan memberkati segala usahamu.
\par 11 Di kalangan orang Israel selalu akan terdapat beberapa orang yang miskin dan berkekurangan. Sebab itu saya memerintahkan kamu untuk bermurah hati kepada mereka."
\par 12 "Apabila seorang laki-laki atau perempuan dari bangsamu menjual dirinya menjadi budakmu, ia harus bekerja untukmu selama enam tahun. Dalam tahun yang ketujuh ia harus dibebaskan.
\par 13 Budak yang dibebaskan itu tak boleh dibiarkan pergi dengan tangan kosong.
\par 14 Berilah kepadanya dengan murah hati dari apa yang diberikan TUHAN Allahmu kepadamu yaitu kambing domba, gandum dan anggur.
\par 15 Ingatlah bahwa kamu dahulu budak di Mesir lalu dibebaskan oleh TUHAN Allahmu; itulah sebabnya saya berikan perintah ini kepadamu.
\par 16 Tetapi kalau budak itu tidak mau pergi karena ia mencintai kamu dan keluargamu dan senang tinggal bersamamu,
\par 17 bawalah dia ke pintu rumahmu dan tindiklah telinganya, maka ia akan menjadi budakmu seumur hidupnya. Begitu juga harus kamu perlakukan budakmu yang perempuan.
\par 18 Kalau kamu membebaskan seorang budak, janganlah merasa kesal, sebab selama enam tahun ia telah bekerja untuk kamu dengan separuh upah seorang pelayan. Lakukanlah perintah itu, maka TUHAN Allahmu akan memberkati segala usahamu."
\par 19 "Anak sapi dan kambing domba jantan yang pertama lahir harus dipersembahkan kepada TUHAN Allahmu. Sapi itu tak boleh dipakai untuk menarik bajak, dan domba itu tak boleh dicukur.
\par 20 Setiap tahun kamu sekeluarga harus makan dagingnya di hadapan TUHAN Allahmu, di tempat yang dipilih TUHAN.
\par 21 Tetapi kalau ternak itu cacat, misalnya pincang atau buta atau cacat berat lainnya, kamu tak boleh mempersembahkannya kepada TUHAN Allahmu.
\par 22 Setiap orang, baik yang bersih maupun yang najis, boleh memakannya di rumah, seperti makan daging rusa atau daging kijang.
\par 23 Darahnya tak boleh dimakan, tetapi harus ditumpahkan ke tanah seperti air."

\chapter{16}

\par 1 "Rayakanlah Paskah dalam bulan Abib untuk menghormati TUHAN Allahmu, sebab dalam bulan itu, pada waktu malam, Ia membebaskan kamu dari Mesir.
\par 2 Pergilah ke tempat yang dipilih TUHAN untuk memotong seekor dari kambing domba atau sapimu sebagai persembahan Paskah untuk menghormati TUHAN Allahmu.
\par 3 Persembahan Paskah itu harus dimakan dengan roti yang tidak beragi. Tujuh hari lamanya kamu harus makan roti yang tidak beragi, seperti pada waktu kamu meninggalkan tanah Mesir dengan tergesa-gesa. Makanlah roti itu, roti yang disebut roti penderitaan, supaya seumur hidupmu kamu teringat akan hari kamu meninggalkan Mesir, tempat kamu menderita.
\par 4 Selama tujuh hari di seluruh negerimu tak seorang pun boleh menyimpan ragi di rumahnya. Daging ternak yang dipotong pada malam hari yang pertama harus dimakan habis pada malam itu juga.
\par 5 Ternak untuk Perayaan Paskah harus dipotong di tempat yang dipilih TUHAN Allahmu, dan tak boleh di tempat lain di seluruh negeri yang diberikan-Nya kepadamu. Hal itu harus kamu lakukan pada waktu matahari terbenam, seperti pada waktu kamu meninggalkan tanah Mesir.
\par 7 Rebuslah daging itu dan makanlah di tempat yang dipilih TUHAN Allahmu, lalu pulanglah ke rumahmu keesokan paginya.
\par 8 Selama enam hari yang berikut kamu harus makan roti tak beragi. Pada hari yang ketujuh kamu harus berkumpul untuk menyembah TUHAN Allahmu. Janganlah bekerja pada hari itu."
\par 9 "Hitunglah tujuh minggu mulai dari hari pertama kamu memotong gandum,
\par 10 lalu rayakanlah Pesta Panen untuk menghormati TUHAN Allahmu. Pada hari itu bawalah persembahan sukarela kepada-Nya menurut banyaknya rezeki yang telah diberikan-Nya kepadamu.
\par 11 Di hadapan TUHAN Allahmu, di tempat yang dipilih-Nya, kamu harus bergembira bersama-sama dengan anak-anakmu, hamba-hambamu dan orang Lewi, orang asing, anak yatim piatu dan para janda yang tinggal di kota-kotamu.
\par 12 Taatilah semua perintah-Nya dengan setia, dan jangan lupa bahwa dahulu kamu budak di Mesir."
\par 13 "Sesudah semua gandummu selesai ditebah dan semua buah anggurmu selesai diperas, kamu harus merayakan Pesta Pondok Daun selama tujuh hari.
\par 14 Bersenang-senanglah bersama anak-anakmu, hamba-hambamu, orang Lewi, orang asing, anak yatim piatu dan para janda yang tinggal di kota-kotamu.
\par 15 Tujuh hari lamanya harus kamu rayakan pesta itu untuk menghormati TUHAN Allahmu di tempat yang dipilih TUHAN. Bergembiralah, karena TUHAN Allahmu telah memberkati hasil tanah dan pekerjaanmu.
\par 16 Semua orang laki-laki Israel harus datang menyembah TUHAN Allahmu di tempat yang dipilih-Nya tiga kali setahun, yaitu pada Hari Raya Paskah, pada Pesta Panen dan Pesta Pondok Daun. Setiap orang harus membawa persembahan
\par 17 menurut kemampuannya, sesuai dengan banyaknya rezeki yang diberikan TUHAN Allahmu kepadanya."
\par 18 "Di setiap kota yang diberikan TUHAN Allahmu kepadamu, kamu harus mengangkat hakim-hakim dan petugas-petugas lain untuk menghakimi rakyat dengan adil.
\par 19 Dalam mengambil keputusan, mereka tak boleh bertindak sewenang-wenang atau berat sebelah. Mereka tak boleh juga menerima suap, karena suap itu membutakan orang, bahkan orang bijaksana dan jujur, sehingga mengambil keputusan yang tidak adil.
\par 20 Kamu harus selalu adil, supaya dapat menetap dan hidup sejahtera di tanah yang diberikan TUHAN Allahmu kepadamu.
\par 21 Kalau kamu membuat mezbah untuk TUHAN Allahmu, jangan menaruh patung berhala kayu lambang Asyera di sebelahnya,
\par 22 dan jangan mendirikan tiang untuk memuja berhala, karena TUHAN Allahmu membenci perbuatan itu.

\chapter{17}

\par 1 Jangan persembahkan sapi atau domba yang cacat kepada TUHAN Allahmu, karena Ia membencinya.
\par 2 Apabila kamu mendengar bahwa di salah satu kotamu ada orang yang berdosa terhadap TUHAN Allahmu karena melanggar perjanjian
\par 3 dan perintah-Nya dengan menyembah dan mengabdi kepada ilah-ilah lain atau kepada matahari, bulan atau bintang-bintang,
\par 4 maka kamu harus menyelidikinya dengan teliti. Sekiranya kejahatan itu benar-benar telah terjadi di Israel,
\par 5 orang yang bersalah harus dibawa ke luar kota dan dilempari batu sampai mati.
\par 6 Tetapi ia hanya boleh dihukum mati kalau ada dua orang saksi atau lebih; seseorang tak boleh dihukum mati berdasarkan keterangan satu orang saksi saja.
\par 7 Para saksi harus lemparkan batu yang pertama kepada orang itu, diikuti seluruh rakyat. Dengan demikian kamu berantas kejahatan itu.
\par 8 Apabila ada perkara yang tak dapat diputuskan oleh hakim-hakim setempat karena terlalu sulit, misalnya perkara pembunuhan yang disengaja atau tidak disengaja, perselisihan tentang hak milik atau luka-melukai badan, kamu harus pergi ke tempat yang dipilih TUHAN Allahmu.
\par 9 Ajukanlah perkaramu itu kepada imam-imam Lewi dan kepada hakim yang sedang bertugas, lalu biarkan mereka memutuskannya.
\par 10 Merekalah yang memberi keputusan, dan kamu harus melakukan dengan teliti apa yang mereka katakan kepadamu.
\par 11 Terimalah keputusan mereka dan laksanakanlah apa yang mereka tetapkan sampai hal yang sekecil-kecilnya.
\par 12 Siapa saja yang berani melawan hakim atau imam yang sedang bertugas, harus dihukum mati; dengan demikian kamu memberantas kejahatan itu di Israel.
\par 13 Semua orang akan mendengar tentang kejadian itu dan menjadi takut, sehingga tak ada lagi yang berani berbuat begitu."
\par 14 "Apabila tanah yang diberikan TUHAN Allahmu kepadamu sudah menjadi milikmu, dan kamu sudah berdiam di situ, mungkin kamu menginginkan seorang raja seperti bangsa-bangsa yang tinggal di sekelilingmu.
\par 15 Perhatikanlah baik-baik supaya orang yang kamu angkat menjadi rajamu adalah orang yang dipilih TUHAN Allahmu. Ia harus seorang dari bangsamu sendiri; jangan mengangkat seorang asing menjadi rajamu.
\par 16 Raja tak boleh memiliki banyak kuda untuk tentaranya dan tak boleh menyuruh bangsa ini kembali ke Mesir untuk membeli kuda, karena TUHAN telah berkata bahwa umat-Nya tak boleh kembali ke negeri itu.
\par 17 Raja tak boleh punya banyak istri, supaya ia jangan membelakangi TUHAN. Ia juga tak boleh memperkaya diri dengan perak dan emas.
\par 18 Kalau seorang dinobatkan menjadi raja, ia harus diberi sebuah buku berisi hukum-hukum Allah yang disalin dari buku asli yang disimpan oleh imam-imam Lewi.
\par 19 Buku itu harus selalu ada di dekatnya, dan seumur hidup ia harus membacanya supaya belajar menghormati TUHAN Allahnya dan dengan setia mentaati segala perintah yang tertulis di dalamnya.
\par 20 Kalau ia membacanya dengan tekun, ia tak akan menganggap dirinya lebih baik dari orang-orang sebangsanya, dan tak akan melanggar perintah TUHAN dengan cara bagaimanapun. Lalu ia dan keturunannya akan memerintah bangsanya sampai turun-temurun."

\chapter{18}

\par 1 "Seluruh suku Lewi diberi tugas imam, karena itu tak boleh menerima bagian tanah di Israel. Mereka harus hidup dari persembahan-persembaha dan kurban-kurban lainnya yang diberikan TUHAN.
\par 2 Mereka tak boleh memiliki tanah seperti suku-suku lainnya, sebab Tuhanlah bagian warisan mereka, seperti yang dijanjikan TUHAN kepada mereka.
\par 3 Apabila seseorang mempersembahkan kurban sapi atau domba, maka paha depan, rahang dan perut ternak itu harus diserahkan kepada imam-imam untuk bagian mereka.
\par 4 Hasil pertama dari gandum, air anggur, minyak zaitun dan bulu domba, juga untuk bagian mereka.
\par 5 Dari semua suku Israel, TUHAN Allahmu telah memilih suku Lewi menjadi imam untuk selama-lamanya.
\par 6 Apabila seorang Lewi yang berasal dari kota mana pun di Israel datang ke tempat yang dipilih TUHAN dan mau tinggal di situ,
\par 7 ia boleh melayani TUHAN Allahnya sebagai imam seperti orang-orang Lewi lain yang bertugas di tempat itu.
\par 8 Ia harus menerima makanan yang sama banyaknya seperti imam-imam lain; apa yang diperolehnya dari keluarganya, juga untuk dia."
\par 9 "Apabila kamu sudah masuk ke negeri yang diberikan TUHAN Allahmu kepadamu, janganlah meniru kejahatan yang dilakukan bangsa-bangsa yang ada di situ.
\par 10 Di antara kamu janganlah ada yang mempersembahkan anak-anaknya sebagai kurban bakaran. Dan janganlah seorang pun menjadi tukang ramal, mencari pertanda-pertanda, memakai jampi-jampi atau
\par 11 ilmu sihir, atau mengadakan hubungan dengan roh-roh orang mati.
\par 12 Orang yang melakukan perbuatan-perbuatan jahat itu dibenci oleh TUHAN Allahmu, dan itulah sebabnya bangsa-bangsa itu disingkirkan-Nya dari hadapanmu.
\par 13 Jadi hendaklah kamu mengabdi kepada TUHAN Allahmu dengan sepenuh hatimu."
\par 14 Kemudian Musa berkata, "Bangsa-bangsa di negeri yang akan kamu diami itu mengikuti nasihat tukang-tukang ramal dan mencari pertanda-pertanda. Tetapi TUHAN Allahmu tidak mengizinkan kamu melakukan hal itu.
\par 15 Sebaliknya dari bangsa kita sendiri Ia akan mengutus kepadamu seorang nabi seperti saya ini, dan kamu harus taat kepadanya.
\par 16 Pada hari kamu berkumpul di Gunung Sinai, kamu mohon supaya kamu jangan lagi mendengar TUHAN Allahmu berbicara dan jangan lagi melihat kehadiran-Nya dalam api yang bernyala-nyala, sebab kamu takut mati.
\par 17 Karena itu TUHAN berkata kepada saya, 'Permintaan mereka itu bijaksana.
\par 18 Dari bangsa mereka sendiri Aku akan mengutus kepada mereka seorang nabi seperti engkau. Aku akan mengatakan kepadanya apa yang harus dikatakannya, lalu ia akan menyampaikan kepada bangsa itu segala yang Kuperintahkan.
\par 19 Ia akan berbicara atas nama-Ku, dan Aku akan menghukum siapa saja yang tidak mau mendengarkan dia.
\par 20 Tetapi kalau seorang nabi berani menyampaikan suatu pesan atas nama-Ku padahal Aku tidak menyuruh dia berbuat begitu, ia harus mati; begitu juga setiap nabi yang berbicara atas nama ilah-ilah lain harus mati.'
\par 21 Mungkin kamu bertanya dalam hati, 'Bagaimana kami tahu apakah pesan seorang nabi itu berasal dari TUHAN atau tidak?'
\par 22 Kalau seorang nabi berbicara atas nama TUHAN, tetapi apa yang dikatakannya itu tidak terjadi, maka ramalan itu bukan dari TUHAN. Nabi itu berbicara atas namanya sendiri dan kamu tak usah takut kepadanya."

\chapter{19}

\par 1 "Sesudah bangsa-bangsa itu dibinasakan TUHAN Allahmu dan negerinya diberikan kepadamu, dan sesudah kamu mengambil kota-kota dan rumah-rumah mereka dan diam di situ,
\par 2 wilayah itu harus kamu bagi menjadi tiga bagian, masing-masing dengan sebuah kota suaka yang mudah dicapai supaya setiap pembunuh dapat melarikan diri ke sana.
\par 4 Kalau seseorang dengan tidak sengaja telah membunuh orang lain yang bukan musuhnya, ia dapat melarikan diri ke salah satu kota itu dan boleh tinggal hidup.
\par 5 Misalnya dua orang bersama-sama masuk hutan untuk menebang kayu. Sementara yang seorang menebang pohon, kepala kapaknya terlepas dari gagangnya dan kena temannya sehingga tewas. Maka orang yang telah membunuh tanpa sengaja itu tidak bersalah dan boleh lari ke salah satu dari tiga kota suaka itu supaya selamat. Sekiranya ada satu kota suaka saja, mungkin jaraknya ke sana terlalu jauh, dan anggota keluarga yang bertanggung jawab untuk membalas pembunuhan itu dapat mengejar dia dan dalam kemarahannya membunuh orang yang tak bersalah.
\par 7 Sebab itu aku memerintahkan kamu untuk menyediakan tiga kota.
\par 8 Apabila TUHAN Allahmu sudah memperluas wilayahmu seperti yang dijanjikan-Nya kepada leluhurmu, dan seluruh negeri itu sudah diberikan kepadamu,
\par 9 kamu harus memilih tiga kota suaka lagi. (Negeri itu akan diberikan kepadamu kalau kamu melakukan segala sesuatu yang saya perintahkan kepadamu hari ini, dan kalau kamu mencintai TUHAN Allahmu dan hidup menurut ajaran-Nya.)
\par 10 Lakukanlah itu supaya kamu jangan berdosa karena membiarkan orang yang tak bersalah dibunuh di negeri yang diberikan TUHAN Allahmu kepadamu."
\par 11 "Tetapi lain halnya kalau seseorang menghadang musuhnya dan dengan sengaja dan rasa benci membunuh dia, lalu melarikan diri ke salah satu kota suaka untuk mendapat perlindungan.
\par 12 Dalam hal itu para pemimpin kotanya harus memanggil orang itu lalu menyerahkan dia kepada anggota keluarga yang bertanggung jawab untuk membalas pembunuhan itu, supaya ia dapat dibunuh.
\par 13 Jangan menaruh kasihan kepadanya. Bebaskanlah Israel dari pembunuh itu, supaya kamu sejahtera."
\par 14 "Apabila kamu sudah sampai di negeri yang diberikan TUHAN Allahmu kepadamu, janganlah menggeser batas tanah tetanggamu yang ditentukan di zaman dahulu."
\par 15 "Seorang saksi saja tidak cukup untuk menyatakan seorang tertuduh bersalah; sekurang-kurangnya dua saksi diperlukan untuk hal itu.
\par 16 Kalau seseorang mau menjatuhkan orang lain dengan tuduhan palsu di pengadilan, maka
\par 17 kedua-duanya harus pergi ke tempat ibadat yang ditentukan TUHAN supaya diadili oleh para imam dan hakim yang sedang bertugas.
\par 18 Para hakim harus menyelidiki perkara itu dengan teliti, dan kalau ternyata orang itu memfitnah sesamanya orang Israel,
\par 19 ia harus menjalani hukuman yang dimaksudkan untuk lawannya itu. Dengan demikian kamu memberantas kejahatan itu.
\par 20 Semua orang akan mendengar tentang hal itu dan menjadi takut, sehingga tak ada lagi yang berbuat sejahat itu.
\par 21 Jangan kasihan dalam perkara-perkara semacam itu. Hukumannya berupa: Nyawa ganti nyawa, mata ganti mata, gigi ganti gigi, tangan ganti tangan dan kaki ganti kaki."

\chapter{20}

\par 1 "Apabila kamu pergi berperang melawan musuh-musuhmu, jangan takut melihat kereta perang dan kuda mereka yang banyak serta tentara mereka yang jumlahnya melebihi jumlahmu. TUHAN Allahmu yang membebaskan kamu dari Mesir, juga akan menolong kamu.
\par 2 Sebelum mulai berperang, seorang imam harus maju dan berbicara kepadamu begini,
\par 3 'Saudara-saudara, dengarlah! Hari ini kamu maju berperang. Jangan takut atau berkecil hati atau bingung.
\par 4 TUHAN Allahmu akan ikut untuk menolong kamu, dan Ia akan memberi kemenangan kepadamu.'
\par 5 Kemudian para perwira harus menyampaikan kata-kata ini kepadamu, 'Adakah di antara kamu orang yang baru saja membangun rumah, tetapi belum mengadakan upacara peresmiannya? Kalau ada, ia boleh pulang. Sebab kalau ia terbunuh dalam peperangan, orang lain akan menempati rumahnya.
\par 6 Adakah di antara kamu orang yang baru saja menanami kebun anggurnya dan belum sempat memetik buah-buah anggurnya? Kalau ada, ia boleh pulang. Sebab kalau ia terbunuh dalam peperangan, orang lain akan menikmati air anggurnya.
\par 7 Adakah di antara kamu orang yang bertunangan dan hendak kawin? Kalau ada, ia boleh pulang. Sebab kalau ia terbunuh dalam peperangan, orang lain akan kawin dengan tunangannya.'
\par 8 Para perwira juga harus mengatakan ini kepadamu, 'Adakah di antara kamu orang gugup dan takut? Kalau ada, ia boleh pulang, supaya ia jangan merusak semangat orang lain.'
\par 9 Sesudah para perwira berbicara kepada tentara, harus ditunjuk pemimpin-pemimpin untuk setiap kesatuan.
\par 10 Apabila kamu pergi untuk menyerang sebuah kota, berilah dahulu kesempatan kepada penduduknya untuk menyerah.
\par 11 Kalau mereka membuka pintu-pintu gerbang dan menyerah, mereka semua harus menjadi hamba-hambamu dan melakukan kerja paksa untukmu.
\par 12 Tetapi kalau penduduk kota itu tidak mau menyerah dan lebih suka berperang, kamu harus mengepung kota itu.
\par 13 Kemudian, apabila TUHAN Allahmu memungkinkan kamu merebut kota itu, kamu harus membunuh seluruh penduduknya yang laki-laki.
\par 14 Tetapi kamu boleh mengambil kaum wanita, anak-anak, ternak dan apa saja yang ada di kota itu. Segala harta benda musuh-musuhmu itu boleh kamu pakai. TUHAN Allahmu menyerahkan itu kepadamu.
\par 15 Begitulah harus kamu perlakukan kota-kota yang jauh dari negeri kediamanmu.
\par 16 Tetapi kalau kota itu ada di dalam wilayah yang diberikan TUHAN Allahmu kepadamu, seluruh penduduknya harus dibunuh.
\par 17 Seperti yang diperintahkan TUHAN Allahmu, kamu harus membinasakan orang-orang Het, Amori, Kanaan, Feris, Hewi dan Yebus.
\par 18 Bunuhlah mereka, supaya mereka tidak membuat kamu berdosa terhadap TUHAN Allahmu dengan mengajar kamu melakukan perbuatan-perbuatan menjijikkan bagi ilah-ilah mereka.
\par 19 Apabila kamu memerangi sebuah kota dan mengepungnya untuk waktu yang lama, janganlah menebang pohon buah-buahan di situ. Makanlah buah-buahnya, tetapi jangan rusakkan pohonnya, sebab pohon-pohon itu bukan musuhmu.
\par 20 Pohon-pohon lain boleh kamu tebang untuk dijadikan pagar pengepungan sampai kota itu sudah jatuh."

\chapter{21}

\par 1 "Apabila di negeri yang diberikan TUHAN Allahmu kepadamu ada orang yang kedapatan mati terbunuh di ladang dan tidak ketahuan siapa pembunuhnya,
\par 2 maka pemimpin dan hakimmu harus pergi mengukur jarak dari tempat mayat itu ditemukan ke tiap-tiap kota yang berdekatan.
\par 3 Lalu para pemuka dari kota yang paling dekat harus memilih seekor sapi muda yang belum pernah dipakai untuk bekerja.
\par 4 Sapi itu harus mereka bawa ke suatu tempat di dekat sungai yang tidak pernah kering dan yang tanahnya belum pernah dibajak atau ditanami. Di situ mereka harus mematahkan leher binatang itu.
\par 5 Para imam Lewi juga harus pergi ke situ, karena mereka harus memberi keputusan dalam tiap perkara tindak kekerasan. Mereka dipilih TUHAN Allahmu untuk mengabdi kepada-Nya dan mengucapkan berkat atas nama-Nya.
\par 6 Lalu semua pemuka dari kota yang paling dekat harus mencuci tangan mereka di atas sapi itu
\par 7 dan mengatakan, 'Bukan kami yang membunuh orang itu, dan kami tidak tahu siapa yang melakukannya.
\par 8 TUHAN, ampunilah Israel umat-Mu yang Kaubebaskan dari Mesir. Ampunilah kami dan jangan menganggap kami bertanggung jawab atas pembunuhan orang yang tidak bersalah.'
\par 9 Maka dengan melakukan apa yang dituntut TUHAN, kamu tidak dianggap bertanggung jawab atas pembunuhan itu."
\par 10 "Apabila kamu berperang dan TUHAN Allahmu memberi kamu kemenangan, lalu kamu mengambil tawanan perang,
\par 11 mungkin di antara mereka ada seorang wanita cantik yang kausukai dan ingin kauperistri.
\par 12 Bawalah wanita itu ke rumahmu. Di situ ia harus menggunting rambutnya, memotong kukunya,
\par 13 dan berganti pakaian. Ia harus tinggal di rumahmu dan berkabung atas kematian orang tuanya selama satu bulan. Sesudah itu engkau boleh kawin dengan dia.
\par 14 Kalau di kemudian hari engkau tidak menginginkan dia lagi, engkau boleh menyuruh dia pergi dengan bebas. Ia tak boleh kauperlakukan sebagai budak atau kaujual, karena ia telah kaupaksa bersetubuh dengan engkau."
\par 15 "Misalkan seorang punya dua istri, dan keduanya melahirkan anak laki-laki, tetapi anak yang lahir lebih dahulu bukan anak dari istri kesayangannya.
\par 16 Kalau orang itu mau menentukan bagaimana ia akan membagi kekayaannya kepada anak-anaknya, ia tak boleh memihak pada anak dari istri kesayangannya dengan memberi kepada anak itu bagian yang menjadi hak anak sulung.
\par 17 Ia harus memberi bagian dua kali lipat dari harta bendanya kepada anak laki-laki yang sulung, walaupun anak itu bukan anak dari istri kesayangannya. Hak anak sulung harus diakui oleh ayahnya, dan kepada anak itu harus diberi warisan yang menjadi haknya menurut hukum."
\par 18 "Misalkan ada anak laki-laki yang keras kepala, suka memberontak dan tak mau menurut kepada orang tuanya walaupun mereka sudah menghukum dia.
\par 19 Maka orang tuanya harus membawa dia kepada para pemuka kota tempat mereka tinggal dan minta supaya anak itu diadili.
\par 20 Mereka harus berkata kepada para pemuka kota itu, 'Anak kami ini keras kepala, suka memberontak dan tak mau taat kepada kami; ia memboroskan uang dan suka mabuk.'
\par 21 Lalu orang-orang lelaki dari kota itu harus melempari anak itu dengan batu sampai mati. Dengan demikian kamu memberantas kejahatan itu. Semua orang di Israel akan mendengar tentang kejadian itu dan menjadi takut."
\par 22 "Apabila seseorang telah dihukum mati karena suatu kejahatan, dan mayatnya digantung pada tiang,
\par 23 mayat itu tak boleh dibiarkan di situ sepanjang malam, tetapi harus dikubur pada hari itu juga. Mayat yang tergantung pada tiang mendatangkan kutuk Allah atas negeri. Jadi, kuburkanlah mayat itu supaya kamu tidak mencemarkan negeri yang diberikan TUHAN Allahmu kepadamu.

\chapter{22}

\par 1 Apabila sapi itu atau domba milik orang sebangsamu sesat dan kamu melihat binatang itu, janganlah pura-pura tidak tahu, tetapi bawalah binatang itu kembali kepada pemiliknya.
\par 2 Kalau pemiliknya jauh rumahnya, atau kamu tidak tahu siapa pemiliknya, bawalah binatang itu ke rumahmu. Apabila pemiliknya datang mencarinya, serahkanlah kepadanya.
\par 3 Buatlah begitu juga kalau kamu menemukan seekor keledai, sepotong pakaian, atau apa saja milik orang sebangsamu.
\par 4 Kalau seekor keledai atau sapi milik orang sebangsamu rebah di jalan, janganlah pura-pura tidak tahu, tetapi tolonglah dia membangunkan binatang itu kembali.
\par 5 Orang perempuan tak boleh berpakaian seperti laki-laki dan orang laki-laki tak boleh berpakaian seperti perempuan, sebab orang yang berbuat begitu dibenci TUHAN Allahmu.
\par 6 Kalau kamu menemukan sebuah sarang burung di pohon kayu atau di tanah dengan induk burung sedang duduk di atas telurnya atau bersama-sama dengan anak-anaknya, janganlah mengambil induk burung itu.
\par 7 Kamu boleh mengambil anak-anak burung itu, tapi biarkan induknya terbang, supaya kamu panjang umur dan hidup makmur.
\par 8 Apabila kamu membangun rumah, kamu harus memasang pagar di sekeliling pinggir atapnya. Maka kamu tidak bertanggung jawab kalau ada orang jatuh dari atap itu lalu mati.
\par 9 Janganlah menanam tanaman lain di kebun anggurmu kecuali pohon anggur. Kalau kamu tanam juga, kamu tak boleh mengambil hasil kebun itu, baik dari anggur, maupun dari tanaman lainnya.
\par 10 Jangan memasang seekor sapi dan seekor keledai pada satu bajak.
\par 11 Jangan memakai pakaian yang dibuat dari wol dan linen dalam satu tenunan.
\par 12 Buatlah rumbai-rumbai pada keempat ujung pakaianmu."
\par 13 "Misalkan seorang laki-laki kawin dengan seorang gadis, dan kemudian tidak menginginkannya lagi.
\par 14 Ia membuat tuduhan palsu bahwa istrinya bukan perawan pada waktu mereka kawin.
\par 15 Dalam hal itu orang tua wanita itu harus mengambil kain pengantin yang ada bekas darahnya, yang membuktikan bahwa anak mereka pada waktu itu masih perawan. Kain itu harus mereka tunjukkan kepada para pemuka kota di pengadilan.
\par 16 Ayah gadis itu harus berkata kepada mereka, 'Saya sudah serahkan anak saya kepada orang ini menjadi istrinya, tetapi sekarang ia tidak menginginkannya lagi.
\par 17 Ia buat tuduhan palsu dengan berkata bahwa istrinya bukan perawan waktu mereka kawin. Tetapi ini buktinya bahwa anak saya waktu itu masih perawan. Perhatikanlah bekas darah pada kain pengantinnya ini!'
\par 18 Maka para pemuka kota harus memanggil orang laki-laki itu dan memukul dia.
\par 19 Mereka juga harus mendenda dia seratus uang perak dan menyerahkan uang itu kepada mertuanya, karena ia telah merusak kehormatan seorang wanita Israel. Lagi pula wanita itu harus tetap menjadi istrinya, dan tak boleh diceraikan seumur hidupnya.
\par 20 Tetapi andaikata tuduhan itu benar dan tidak ada bukti bahwa istrinya itu masih perawan pada waktu kawin,
\par 21 maka wanita itu harus dibawa ke pintu rumah orang tuanya. Di tempat itu orang-orang lelaki dari kota itu harus melemparinya dengan batu sampai mati. Wanita itu telah melakukan sesuatu yang memalukan bangsa kita karena bersetubuh sebelum kawin selagi ia masih tinggal di rumah ayahnya. Dengan menghukum dia kamu memberantas kejahatan itu.
\par 22 Kalau seorang laki-laki tertangkap basah selagi ia bersetubuh dengan istri orang lain, kedua-duanya harus dihukum mati. Dengan demikian kamu memberantas kejahatan itu.
\par 23 Misalkan di dalam kota seorang laki-laki tertangkap basah selagi ia bersetubuh dengan seorang gadis yang sudah bertunangan dengan orang lain.
\par 24 Dalam hal itu mereka harus dibawa ke luar kota dan dilempari dengan batu sampai mati. Gadis itu harus dibunuh karena ia tidak berteriak minta tolong, sedangkan hal itu terjadi di dalam kota, di mana teriakannya dapat didengar. Dan orang laki-laki itu harus dibunuh karena ia bersetubuh dengan gadis yang sudah bertunangan. Dengan menghukum kedua-duanya, kamu memberantas kejahatan itu.
\par 25 Tetapi lain halnya kalau di ladang seorang laki-laki memperkosa seorang gadis yang sudah bertunangan dengan orang lain. Dalam hal itu hanya yang laki-laki harus dihukum mati.
\par 26 Gadis itu tak boleh diapa-apakan karena ia tidak melakukan sesuatu yang pantas dihukum mati. Perkara itu seperti perkara seorang yang menyerang orang lain dan membunuhnya.
\par 27 Pemerkosaan itu terjadi di ladang, dan walaupun gadis itu berteriak minta tolong, namun tak ada orang yang datang menolongnya.
\par 28 Misalkan seorang laki-laki tertangkap basah selagi ia memperkosa seorang gadis yang belum bertunangan.
\par 29 Dalam hal itu ia harus membayar kepada ayah gadis itu mas kawin seharga lima puluh uang perak. Gadis itu harus menjadi istrinya karena ia dipaksa bersetubuh dan selama hidupnya ia tak boleh diceraikan.
\par 30 Tak seorang pun boleh menghina ayahnya dengan meniduri salah seorang istri ayahnya."

\chapter{23}

\par 1 "Orang yang telah dikebiri atau yang dipotong zakarnya tak boleh menjadi warga umat TUHAN.
\par 2 Orang yang lahir di luar pernikahan dan semua keturunannya sampai yang kesepuluh, tak boleh menjadi warga umat TUHAN.
\par 3 Orang Amon dan Moab serta keturunan mereka sampai yang kesepuluh tak boleh menjadi warga umat TUHAN,
\par 4 karena mereka tak mau memberi kamu makanan dan air ketika kamu dalam perjalanan keluar dari Mesir. Bahkan mereka mengupah Bileam, anak Beor dari kota Petor di Mesopotamia, untuk mengutuk kamu.
\par 5 Tetapi TUHAN Allahmu tak mau mendengarkan Bileam. Sebaliknya Ia mengubah kutuk itu menjadi berkat, karena Ia mengasihi kamu.
\par 6 Selama kamu hidup dan sampai selama-lamanya, janganlah menolong bangsa-bangsa itu atau membuat mereka makmur.
\par 7 Jangan memandang rendah orang Edom karena mereka itu saudaramu. Jangan juga memandang rendah orang Mesir, karena kamu pernah tinggal di negeri mereka.
\par 8 Mulai dari angkatan ketiga dan selanjutnya keturunan mereka boleh menjadi warga umat TUHAN."
\par 9 "Pada waktu kamu sedang berperang, kamu harus menghindari segala sesuatu yang menjadikan kamu najis.
\par 10 Kalau seorang menjadi najis karena mengeluarkan mani pada waktu tidur, ia harus meninggalkan perkemahan dan tinggal di luar.
\par 11 Menjelang sore ia harus mandi dan waktu matahari terbenam ia boleh kembali ke perkemahan.
\par 12 Kamu harus menyediakan tempat di luar perkemahan untuk buang air.
\par 13 Kalau mau ke sana, bawalah selain perlengkapanmu juga sepotong kayu untuk menggali lubang tempat membuang hajat dan untuk menimbuninya.
\par 14 TUHAN Allahmu menyertai kamu di dalam perkemahanmu untuk melindungi kamu dan memberi kamu kemenangan atas musuh-musuhmu. Maka jagalah supaya perkemahanmu tetap bersih. Jangan sampai terdapat sesuatu yang tidak senonoh di antara kamu, supaya TUHAN jangan meninggalkan kamu."
\par 15 "Apabila seorang budak melarikan diri dari tuannya lalu datang kepadamu minta perlindungan, janganlah menyuruh dia pulang.
\par 16 Ia boleh tinggal di salah satu kotamu menurut pilihannya, dan kamu tak boleh memperlakukan dia dengan keras.
\par 17 Seorang Israel, baik laki-laki maupun perempuan, tak boleh menjadi pelacur di kuil-kuil pemujaan.
\par 18 Juga uang yang diperoleh dari hasil pelacuran tak boleh dibawa ke Rumah TUHAN Allahmu untuk membayar kaul, sebab TUHAN Allahmu membenci pelacur di kuil-kuil pemujaan dan upah pelacuran.
\par 19 Kalau kamu meminjamkan uang kepada orang asing, kamu boleh minta bunga. Tetapi kalau kamu meminjamkan uang atau makanan atau barang lain kepada orang sebangsamu, pinjaman itu harus diberikan tanpa bunga. Taatilah perintah itu, maka TUHAN Allahmu memberkati segala sesuatu yang kamu lakukan di negeri yang kamu duduki.
\par 21 Kalau kamu menjanjikan sesuatu kepada TUHAN Allahmu, janganlah menunda-nunda untuk menepati janji itu, karena TUHAN Allahmu tetap menuntutnya, dan kamu berdosa jika tidak memenuhinya.
\par 22 Kamu tidak berdosa kalau tidak membuat janji kepada TUHAN.
\par 23 Tetapi kalau kamu dengan sukarela membuat janji kepada TUHAN Allahmu, kamu wajib menepatinya.
\par 24 Apabila kamu sedang berjalan melalui kebun anggur orang lain, kamu boleh makan buah anggur sebanyak yang kamu inginkan, tetapi tak boleh mengumpulkannya di dalam keranjang.
\par 25 Kalau kamu sedang berjalan melalui ladang gandum orang lain, kamu boleh makan gandum yang kamu petik dengan tangan, tetapi tak boleh memotongnya dengan sabit."

\chapter{24}

\par 1 "Misalkan seorang laki-laki kawin dengan seorang gadis, dan kemudian tidak menginginkannya lagi karena ia mendapati sesuatu yang memalukan padanya. Lalu laki-laki itu menyerahkan surat cerai kepadanya dan mengusir dia dari rumahnya.
\par 2 Kemudian wanita itu kawin dengan seorang laki-laki lain.
\par 3 Tetapi sesudah beberapa waktu laki-laki itu tidak suka lagi kepadanya, lalu menyerahkan surat cerai kepadanya dan mengusir dia. Atau mungkin juga suami yang kedua itu meninggal.
\par 4 Dalam kedua hal suami yang pertama tak boleh mengawini wanita itu lagi; ia harus memandangnya sebagai wanita yang sudah dicemarkan. Kalau ia mengawininya juga, perbuatan itu merupakan penghinaan terhadap TUHAN Allahmu. Kamu tak boleh melakukan dosa sejahat itu di negeri yang diberikan TUHAN kepadamu."
\par 5 "Seorang laki-laki yang baru saja kawin tak boleh disuruh masuk tentara dan pergi berperang atau melakukan tugas umum lainnya. Ia harus dibebastugaskan selama satu tahun untuk mengurus rumah tangganya dan menyenangkan hati istrinya.
\par 6 Kalau kamu meminjamkan sesuatu kepada orang lain, janganlah mengambil sebagai jaminan batu yang dipakai orang itu untuk menggiling gandumnya. Dengan mengambil batu itu, kamu merampas alat yang dipakai keluarganya untuk penyambung hidup.
\par 7 Orang yang menculik orang sebangsanya lalu memperlakukan dia sebagai budak atau menjualnya, harus dihukum mati. Dengan demikian kamu memberantas kejahatan itu.
\par 8 Apabila terdapat penyakit kulit yang berbahaya, lakukanlah dengan teliti apa yang diperintahkan imam-imam Lewi kepadamu; ikutilah petunjuk-petunjuk yang saya berikan kepada mereka.
\par 9 Ingatlah apa yang dilakukan TUHAN Allahmu terhadap Miryam dalam perjalananmu keluar dari Mesir.
\par 10 Kalau kamu meminjamkan sesuatu kepada sesamamu, janganlah masuk ke dalam rumahnya untuk mengambil bajunya yang dijadikan jaminan.
\par 11 Tunggulah di luar dan biarkan ia sendiri membawanya kepadamu.
\par 12 Kalau orang itu miskin, baju itu tak boleh kautahan sepanjang malam,
\par 13 tetapi harus dikembalikan kepadanya setiap malam, supaya ia dapat memakainya untuk tidur. Maka ia akan berterima kasih dan TUHAN Allahmu akan berkenan kepadamu.
\par 14 Jangan memeras orang upahan yang miskin dan berkekurangan, baik orang sebangsamu, maupun orang asing yang tinggal di salah satu kotamu.
\par 15 Setiap hari sebelum matahari terbenam kamu harus membayar upahnya untuk kerja hari itu; ia mengharapkan uang itu karena ia memerlukannya. Kalau kamu tidak membayar upahnya, ia akan berseru kepada TUHAN untuk minta tolong terhadap kamu, dan kamu dinyatakan berdosa.
\par 16 Jangan menghukum mati orang tua karena kejahatan yang dilakukan anaknya, dan jangan menghukum mati anak karena kejahatan yang dilakukan orang tuanya. Setiap orang hanya boleh dihukum mati karena kejahatan yang dilakukannya sendiri.
\par 17 Jangan merampas hak orang asing dan anak yatim piatu, dan jangan mengambil pakaian seorang janda untuk jaminan.
\par 18 Ingatlah bahwa kamu pernah menjadi budak di Mesir dan dibebaskan TUHAN Allahmu; itulah sebabnya saya memberi peraturan ini kepadamu.
\par 19 Apabila kamu mengumpulkan hasil tanahmu, janganlah kembali untuk mengambil berkas gandum yang tertinggal. Gandum itu harus dibiarkan untuk orang asing, anak yatim piatu dan para janda, supaya kamu diberkati TUHAN Allahmu dalam segala usahamu.
\par 20 Sesudah kamu sekali memetik buah zaitun dan buah anggurmu, janganlah kembali untuk mengumpulkan buah-buah yang tertinggal. Itu harus kamu biarkan untuk orang asing, anak yatim piatu dan para janda.
\par 22 Ingatlah bahwa kamu pernah menjadi budak di Mesir; itulah sebabnya saya memberi peraturan ini kepadamu.

\chapter{25}

\par 1 Misalkan dua orang Israel pergi ke pengadilan untuk mengadukan suatu perkara, lalu yang seorang dinyatakan tidak bersalah, dan yang lain bersalah.
\par 2 Kalau yang bersalah dijatuhi hukuman cambuk, hakim harus menyuruh dia menelungkup. Jumlah cambukan tergantung dari kejahatan yang telah dilakukannya,
\par 3 tetapi tidak boleh lebih dari empat puluh kali. Kalau lebih, berarti orang itu dihina di depan umum.
\par 4 Jangan memberangus mulut sapi yang sedang menebah gandum."
\par 5 "Kalau orang-orang yang bersaudara tinggal bersama, lalu salah seorang dari mereka mati tanpa meninggalkan anak laki-laki, maka jandanya tak boleh kawin dengan orang lain di luar keluarga mendiang suaminya. Saudara almarhum wajib kawin dengan janda itu.
\par 6 Anak laki-laki sulung mereka harus dianggap anak saudara yang mati itu, supaya ia mempunyai keturunan di antara bangsa Israel.
\par 7 Kalau orang itu tidak mau kawin dengan istri mendiang saudaranya, wanita itu harus pergi menghadap para pemuka kota dan berkata, 'Ipar saya tidak mau melakukan kewajibannya memberi kepada saudaranya seorang keturunan di antara bangsa Israel.'
\par 8 Lalu para pemuka kota harus memanggil orang itu dan bicara dengan dia. Kalau ia tetap menolak,
\par 9 istri mendiang saudaranya harus mendekati dia di depan para pemuka kota, mencabut sandal orang itu, meludahi mukanya dan berkata, 'Begini harus diperlakukan orang yang tak mau memberi keturunan kepada saudaranya.'
\par 10 Keluarga orang itu akan dikenal di Israel sebagai 'Keluarga orang yang dicabut sandalnya.'"
\par 11 "Kalau dua orang laki-laki sedang berkelahi dan istri yang seorang berusaha menolong suaminya dengan meremas kemaluan lawan suaminya,
\par 12 potonglah tangan wanita itu dan jangan kasihani dia.
\par 13 Jangan menipu dengan memakai timbangan dan takaran yang tidak betul.
\par 15 Pakailah timbangan dan takaran yang tepat supaya kamu panjang umur di negeri yang diberikan TUHAN Allahmu kepadamu.
\par 16 TUHAN Allahmu benci kepada orang yang suka menipu."
\par 17 "Ingatlah bagaimana orang Amalek memperlakukan kamu pada waktu kamu dalam perjalanan keluar dari Mesir.
\par 18 Mereka tidak takut kepada Allah dan menyerang kamu dari belakang ketika kamu sedang kelelahan dan kehabisan tenaga, lalu membunuh semua orang-orangmu yang dengan susah payah berjalan di belakang.
\par 19 Oleh sebab itu, kalau TUHAN Allahmu sudah memberi negeri itu kepadamu dan kamu aman dari semua musuh yang tinggal di sekelilingmu, semua orang Amalek itu harus kamu bunuh sehingga mereka tidak diingat lagi. Jangan lupa!"

\chapter{26}

\par 1 "Apabila kamu sudah masuk ke negeri yang diberikan TUHAN Allahmu kepadamu dan sudah berdiam di situ,
\par 2 maka hasil pertama dari segala yang kamu kumpulkan dari tanahmu harus kamu masukkan ke dalam sebuah bakul, dan kamu bawa ke tempat yang dipilih TUHAN Allahmu.
\par 3 Pergilah kepada imam yang bertugas pada waktu itu dan katakanlah kepadanya, 'Saya menyatakan kepada TUHAN Allahmu bahwa saya sudah masuk ke negeri yang dijanjikan-Nya kepada nenek moyang kita untuk diberikan kepada kita.'
\par 4 Imam akan menerima bakul itu dari tanganmu dan meletakkannya di depan mezbah TUHAN Allahmu.
\par 5 Kemudian di hadapan TUHAN Allahmu, engkau harus mengucapkan kata-kata ini: 'Ya TUHAN, nenek moyang saya seorang Aram yang mengembara. Ia membawa keluarganya ke Mesir untuk tinggal di sana sebagai pengungsi. Ketika pergi ke sana jumlah mereka hanya sedikit, tetapi mereka menjadi bangsa yang besar dan kuat.
\par 6 Orang Mesir memperlakukan kami dengan kejam dan memaksa kami bekerja sebagai budak.
\par 7 Lalu kami berseru minta tolong kepada-Mu, TUHAN, Allah nenek moyang kami. Engkau mendengar seruan kami dan melihat penderitaan, kesusahan serta kesengsaraan kami.
\par 8 Dengan kuasa dan kekuatan yang besar Engkau membebaskan kami dari Mesir. Engkau melakukan mujizat-mujizat dan keajaiban-keajaiban serta menimbulkan kejadian-kejadian yang menggemparkan musuh.
\par 9 Kemudian Engkau membawa kami ke sini dan memberikan tanah yang kaya dan subur ini kepada kami.
\par 10 Sebab itu, ya TUHAN, saya persembahkan di sini hasil pertama dari tanah yang Kauberikan kepada saya.' Lalu letakkanlah bakul itu di hadapan TUHAN Allahmu, dan sembahlah Dia.
\par 11 Bersyukurlah untuk segala rezeki yang telah diberikan TUHAN Allahmu kepadamu dan keluargamu. Ajaklah orang Lewi dan orang asing yang tinggal di antara kamu ikut dalam perayaan itu.
\par 12 Tiap tahun yang ketiga kamu harus memberikan persembahan sepersepuluhan, yaitu sepersepuluh dari hasil tanahmu kepada imam Lewi, orang asing, anak yatim piatu dan janda, sehingga di dalam setiap kotamu mereka mendapat makanan yang mereka perlukan. Kalau sudah,
\par 13 berkatalah kepada TUHAN Allahmu, 'Dari sepersepuluh hasil tanah yang harus dipersembahkan kepada-Mu, TUHAN, tidak sedikit pun tertinggal di rumah saya; semuanya sudah saya berikan kepada orang Lewi, orang asing, anak yatim piatu dan janda, seperti yang diperintahkan TUHAN Allahku kepada saya. Dari perintah TUHAN tentang persembahan sepersepuluhan tak ada yang saya langgar.
\par 14 Dari persembahan itu tidak sedikit pun yang saya makan waktu saya sedang berkabung, atau saya bawa ke luar rumah waktu saya sedang najis, atau saya berikan sebagai sajian kepada orang mati. Saya telah taat kepada-Mu, ya TUHAN; segala yang Kauperintahkan sudah saya lakukan.
\par 15 Dari tempat-Mu yang suci di surga sudilah Engkau memandang dan memberkati umat-Mu Israel; berkatilah juga negeri kaya dan subur yang telah Kauberikan kepada kami sesuai dengan janji-Mu kepada nenek moyang kami.'"
\par 16 "Hari ini TUHAN Allahmu memerintahkan kamu untuk melakukan segala perintah itu; lakukanlah semuanya dengan setia dan dengan seluruh jiwa ragamu.
\par 17 Hari ini kamu mengakui TUHAN sebagai Allahmu. Kamu berjanji untuk taat kepada-Nya dan melakukan segala hukum serta perintah-Nya.
\par 18 Hari ini TUHAN menerima kamu sebagai umat-Nya sendiri, seperti yang dijanjikan-Nya kepadamu, dan Ia memerintahkan kamu untuk melakukan segala perintah-Nya.
\par 19 TUHAN akan membuat kamu lebih besar dari semua bangsa lain yang sudah diciptakan-Nya, dan kamu akan dipuji dan dihormati. Kamu menjadi umat TUHAN Allahmu seperti yang dijanjikan-Nya."

\chapter{27}

\par 1 Kemudian bersama para pemimpin Israel, Musa menyampaikan pesan ini kepada bangsa itu, "Taatilah segala perintah yang saya berikan kepadamu hari ini.
\par 2 Pada hari kamu menyeberangi Sungai Yordan untuk masuk ke negeri yang diberikan TUHAN Allahmu kepadamu, kamu harus menegakkan beberapa batu besar lalu mengapurnya.
\par 3 Tuliskanlah pada batu-batu itu setiap perkataan dari hukum-hukum ini. Apabila kamu sudah masuk ke negeri yang kaya dan subur yang dijanjikan TUHAN, Allah nenek moyangmu kepadamu,
\par 4 dan kamu sudah ada di seberang Sungai Yordan, kamu harus pergi ke Gunung Ebal. Di situ kamu harus menegakkan dan mengapur batu-batu itu seperti yang saya perintahkan kepadamu hari ini.
\par 5 Di tempat itu harus juga kamu dirikan bagi TUHAN Allahmu sebuah mezbah dari batu-batu yang belum pernah dikerjakan dengan alat-alat besi,
\par 6 sebab setiap mezbah yang didirikan untuk TUHAN Allahmu harus dibuat dari batu yang utuh. Di atas mezbah itu kamu harus mempersembahkan kurban-kurban bakaran.
\par 7 Persembahkanlah juga kurban perdamaianmu dan makanlah di situ serta bersukacitalah di hadapan TUHAN Allahmu.
\par 8 Di atas batu-batu yang dikapur itu harus kamu tulis dengan jelas setiap perkataan dari hukum-hukum Allah."
\par 9 Kemudian bersama para imam Lewi, Musa berkata kepada seluruh bangsa Israel, "Saudara-saudara, perhatikanlah dan dengarlah! Hari ini kamu sudah menjadi umat TUHAN Allahmu.
\par 10 Sebab itu taatilah TUHAN Allahmu dan lakukanlah semua perintah-Nya yang saya berikan kepadamu hari ini."
\par 11 Lalu Musa berkata kepada bangsa Israel,
\par 12 "Sesudah kamu menyeberangi Sungai Yordan, suku Simeon, Lewi, Yehuda, Isakhar, Yusuf dan Benyamin harus berdiri di atas Gunung Gerizim pada waktu diucapkan berkat atas bangsa ini.
\par 13 Pada waktu diucapkan kutuk, suku Ruben, Gad, Asyer, Zebulon, Dan serta Naftali harus berdiri di atas Gunung Ebal.
\par 14 Orang-orang Lewi harus mengucapkan kata-kata berikut ini dengan suara nyaring dan orang Israel harus menjawab, 'Amin'.
\par 15 Orang-orang Lewi: 'Terkutuklah orang yang membuat patung dari batu, kayu atau logam, dan menyembahnya dengan sembunyi-sembunyi; TUHAN membenci pemujaan berhala!' Seluruh rakyat: 'Amin!'
\par 16 'Terkutuklah orang yang tidak menghormati orang tuanya dan membiarkan mereka terlantar.' 'Amin!'
\par 17 'Terkutuklah orang yang menggeser batas tanah tetangganya.' 'Amin!'
\par 18 'Terkutuklah orang yang menyesatkan orang buta.' 'Amin!'
\par 19 'Terkutuklah orang yang merampas hak orang asing, anak yatim piatu dan janda.' 'Amin!'
\par 20 'Terkutuklah orang yang menghina ayahnya karena bersetubuh dengan salah seorang istri ayahnya.' 'Amin!'
\par 21 'Terkutuklah orang yang bersetubuh dengan binatang.' 'Amin!'
\par 22 'Terkutuklah orang yang bersetubuh dengan saudaranya perempuan yang sekandung, atau saudaranya seayah atau seibu.' 'Amin!'
\par 23 'Terkutuklah orang yang bersetubuh dengan ibu mertuanya.' 'Amin!'
\par 24 'Terkutuklah orang yang melakukan pembunuhan dengan sembunyi-sembunyi.' 'Amin!'
\par 25 'Terkutuklah orang yang menerima suap untuk membunuh orang yang tidak bersalah.' 'Amin!'
\par 26 'Terkutuklah orang yang tidak menepati hukum dan ajaran Allah dengan perbuatan.' 'Amin!'"

\chapter{28}

\par 1 "Kalau kamu mentaati TUHAN Allahmu dan setia melakukan segala perintah yang saya berikan kepadamu hari ini, maka Ia akan menjadikan kamu bangsa yang paling masyhur di dunia.
\par 2 Taatilah TUHAN Allahmu, maka semua berkat ini akan diberikan kepadamu:
\par 3 Diberkatilah kota-kota dan ladang-ladangmu.
\par 4 Diberkatilah kamu, sehingga anak-anakmu banyak, hasil tanahmu berlimpah dan sapi serta kambing dombamu berjumlah besar.
\par 5 Diberkatilah panen gandummu serta makanan yang kamu buat dari gandum itu.
\par 6 Diberkatilah segala usahamu.
\par 7 Jika kamu diserang, TUHAN akan mengalahkan musuhmu. Mereka akan melancarkan serangan dengan teratur, tetapi lari dengan berantakan.
\par 8 TUHAN Allahmu akan memberkati usahamu dan mengisi lumbung-lumbungmu dengan gandum. Ia akan memberkati kamu di negeri yang diberikan-Nya kepadamu.
\par 9 Kalau kamu mentaati TUHAN Allahmu dan melakukan segala perintah-Nya, kamu akan dijadikan umat-Nya sendiri, seperti yang dijanjikan-Nya.
\par 10 Maka semua bangsa di bumi akan melihat bahwa kamu bangsa pilihan TUHAN, dan mereka akan menyegani kamu.
\par 11 Kamu akan dianugerahi banyak anak. Ternakmu akan banyak dan hasil tanahmu berlimpah di negeri yang dijanjikan TUHAN kepada nenek moyangmu untuk diberikan kepadamu.
\par 12 TUHAN akan menurunkan hujan pada musimnya dari tempat penyimpanan-Nya yang berlimpah di langit. Ia akan memberkati segala usahamu, sehingga kamu dapat memberi pinjaman kepada banyak bangsa, tetapi kamu sendiri tidak usah meminjam.
\par 13 TUHAN Allahmu akan menjadikan kamu pemimpin di antara bangsa-bangsa, dan bukan pengikut. Kalau kamu setia mentaati semua perintah TUHAN yang saya berikan kepadamu hari ini, kamu akan semakin makmur dan tak pernah mundur.
\par 14 Tetapi jangan sekali-kali melalaikan perintah-perintah itu dengan cara bagaimanapun juga. Jangan juga memuja dan mengabdi kepada ilah-ilah lain."
\par 15 "Tetapi kalau kamu tidak mentaati TUHAN Allahmu, dan tidak setia melakukan segala perintah dan hukum-Nya yang saya berikan kepadamu hari ini, maka segala kutuk ini akan menimpa kamu:
\par 16 Terkutuklah kota-kota dan ladang-ladangmu.
\par 17 Terkutuklah panen gandummu dan makanan yang kamu buat dari gandum itu.
\par 18 Terkutuklah kamu, sehingga hanya sedikit anakmu, hasil tanahmu, serta sapi dan kambing dombamu.
\par 19 Terkutuklah segala usahamu.
\par 20 Kalau kamu berbuat jahat dan menolak TUHAN, Ia akan mendatangkan bencana, kekacauan dan kesulitan dalam segala yang kamu lakukan, sehingga dalam waktu singkat saja kamu dibinasakan sama sekali.
\par 21 Penyakit demi penyakit akan menimpa kamu, sampai akhirnya tak seorang pun dari kamu masih tinggal di negeri yang kamu duduki itu.
\par 22 TUHAN akan menghukum kamu dengan penyakit-penyakit menular, bengkak-bengkak dan demam. Ia akan mendatangkan kekeringan dan angin yang menghanguskan sehingga panenmu rusak. Bencana-bencana itu terus menimpa kamu sampai kamu binasa.
\par 23 Hujan tak akan turun dan tanahmu menjadi keras seperti besi.
\par 24 TUHAN tak akan menurunkan hujan air, melainkan hujan debu dan pasir sampai kamu habis binasa.
\par 25 Kalau kamu tidak mentaati TUHAN Allahmu, Ia akan membuat kamu dikalahkan musuh-musuhmu. Kamu akan melancarkan serangan dengan teratur, tetapi lari dengan berantakan. Semua bangsa di bumi akan merasa ngeri melihat kemalanganmu.
\par 26 Kalau kamu mati, mayatmu menjadi makanan burung-burung dan binatang buas, dan tidak ada yang mengusir binatang-binatang itu.
\par 27 TUHAN akan menghukum kamu dengan bisul-bisul seperti yang dilakukan-Nya terhadap orang Mesir. Badanmu akan penuh dengan borok dan kudis yang gatal, tetapi tak ada obatnya.
\par 28 TUHAN akan membuat kamu menjadi gila, buta dan kebingungan.
\par 29 Di siang bolong kamu meraba-raba seperti orang buta, tetapi tak dapat menemukan jalan. Segala usahamu akan gagal. Kamu terus-menerus ditindas dan dirampok, dan tidak ada yang menolong kamu.
\par 30 Kalau kamu tidak mentaati TUHAN Allahmu, hal-hal ini akan menimpa dirimu: Kamu akan bertunangan, tetapi orang lain kawin dengan tunanganmu. Kamu akan membangun rumah, tetapi tidak mendiaminya. Kamu akan menanami kebun anggur tetapi tidak makan buah-buahnya.
\par 31 Sapimu akan disembelih di depan matamu, tetapi kamu tidak makan dagingnya. Keledai-keledaimu akan digiring pergi sementara kamu memandanginya dan tak akan dikembalikan. Domba-dombamu akan diberikan kepada musuh-musuhmu dan tak ada yang menolong.
\par 32 Di depan matamu anak-anakmu akan diberikan sebagai budak kepada orang asing. Tiap hari matamu pedih mencari anak-anakmu, tetapi usahamu sia-sia, sebab mereka tidak akan kembali.
\par 33 Bangsa asing akan mengambil semua hasil tanah yang kamu peroleh dengan kerja keras, tetapi kamu sendiri tidak mendapat apa-apa selain penindasan terus-menerus dan perlakuan yang kejam.
\par 34 Kamu akan menjadi gila karena apa yang kamu alami.
\par 35 TUHAN akan membuat kakimu penuh borok-borok yang sakit dan tak dapat sembuh; dari kaki sampai ujung kepala kamu akan penuh bisul-bisul.
\par 36 Kamu dan rajamu akan dibawa TUHAN ke negeri asing yang tidak kamu kenal dan tidak pula dikenal nenek moyangmu. Di situ kamu akan mengabdi kepada ilah-ilah yang dibuat dari kayu dan batu.
\par 37 Di negeri-negeri tempat kamu diceraiberaikan TUHAN, orang-orang akan ngeri melihat kamu; kamu akan diolok-olok dan ditertawakan mereka.
\par 38 Banyak benih yang kamu taburkan, tetapi sedikit hasil yang kamu peroleh, karena tanamanmu habis dimakan belalang.
\par 39 Kamu akan menanami kebun-kebun anggur dan memeliharanya, tetapi tidak memetik buah anggur atau minum air anggurnya, karena pohon-pohon anggur itu habis dimakan ulat.
\par 40 Di seluruh negerimu tumbuh pohon zaitun, tetapi buah-buahnya akan gugur, sehingga kamu tidak memperoleh minyaknya.
\par 41 Kamu akan mendapat anak-anak laki-laki dan perempuan, tetapi mereka diambil sebagai tawanan perang, sehingga kamu kehilangan mereka.
\par 42 Semua pohon dan tanamanmu akan habis dimakan serangga.
\par 43 Orang asing yang tinggal di negerimu akan semakin berkuasa, sedangkan kuasamu sendiri semakin berkurang.
\par 44 Mereka punya uang untuk dipinjamkan kepadamu, tetapi kamu tidak punya apa-apa untuk dipinjamkan kepada mereka. Dan akhirnya kamu dikuasai mereka.
\par 45 Segala malapetaka itu akan menimpa kamu dan terus mengganggu kamu sampai kamu binasa, karena kamu tidak taat kepada TUHAN Allahmu dan tidak melakukan hukum-hukum yang diberikan-Nya kepadamu.
\par 46 Bencana-bencana itu merupakan bukti dari hukuman TUHAN atas kamu dan keturunanmu untuk selama-lamanya.
\par 47 Kamu sudah diberkati TUHAN Allahmu dalam segala hal, tetapi tidak mau mengabdi kepada-Nya dengan hati yang ikhlas dan gembira.
\par 48 Karena itu kamu harus mengabdi kepada musuh-musuh yang dikirim TUHAN untuk melawan kamu. Kamu akan kelaparan, kehausan dan telanjang serta berkekurangan dalam segala hal. TUHAN akan menindas kamu dengan kejam sampai kamu binasa.
\par 49 Suatu bangsa yang tidak kamu mengerti bahasanya akan didatangkan TUHAN dari ujung bumi untuk melawan kamu. Seperti burung rajawali mereka akan menyambar kamu.
\par 50 Mereka kejam dan tidak menaruh kasihan kepada orang tua-tua maupun anak-anak.
\par 51 Mereka akan menghabiskan ternak dan hasil tanahmu, dan tidak meninggalkan bagimu gandum, air anggur, minyak zaitun atau sapi dan kambing dombamu sampai kamu mati kelaparan.
\par 52 Mereka akan menyerang setiap kota di negeri yang diberikan TUHAN Allahmu kepadamu sehingga tembok-temboknya yang tinggi dan diperkuat yang kamu andalkan itu runtuh.
\par 53 Sementara musuh-musuhmu mengepung kota-kotamu, kamu akan putus asa karena kelaparan, sehingga kamu memakan anak-anakmu sendiri, anak-anak yang dianugerahkan TUHAN Allahmu kepadamu.
\par 54 Bahkan orang bangsawan yang paling luhur budinya akan putus asa selama pengepungan itu, sehingga ia makan anak-anaknya sendiri karena tidak ada makanan lain. Tak sedikit pun ia berikan kepada saudaranya atau istrinya yang dicintainya atau anak-anaknya yang masih hidup.
\par 56 Bahkan wanita bangsawan yang paling luhur budinya dan yang sangat kaya sehingga tak pernah harus berjalan kaki, akan berbuat begitu juga. Pada waktu musuh mengepung kota, wanita itu akan putus asa karena kelaparan, sehingga ia dengan sembunyi-sembunyi makan anaknya yang baru lahir dan ari-arinya. Tidak sedikit pun ia berikan kepada suaminya yang dicintainya atau kepada anak-anaknya.
\par 58 Kalau kamu tidak setia mentaati semua hukum TUHAN Allahmu yang tertulis dalam buku ini dan tidak menghormati TUHAN Allahmu yang agung dan menakjubkan,
\par 59 TUHAN akan mendatangkan atas kamu dan keturunanmu penyakit-penyakit yang tak dapat disembuhkan dan wabah-wabah yang tak dapat dihentikan.
\par 60 Penyakit-penyakit mengerikan yang kamu saksikan di Mesir akan menimpa kamu, dan kamu tak dapat sembuh.
\par 61 TUHAN juga akan mendatangkan macam-macam penyakit dan wabah yang tidak disebut dalam buku ini, buku Hukum TUHAN. Lalu kamu akan dibinasakan.
\par 62 Walaupun jumlahmu sudah sebanyak bintang di langit, hanya sedikit saja di antara kamu yang masih hidup, karena kamu tidak taat kepada TUHAN Allahmu.
\par 63 Seperti TUHAN senang memberi kemakmuran kepadamu dan menambah jumlahmu, Ia akan senang membinasakan dan memusnahkan kamu. Kamu akan dicabut dari negeri yang tak lama lagi kamu duduki.
\par 64 TUHAN akan menceraiberaikan kamu di antara bangsa-bangsa di seluruh muka bumi. Di sana kamu akan menyembah ilah-ilah yang dibuat dari kayu dan batu, ilah-ilah yang tidak kamu kenal dan tidak pula dikenal nenek moyangmu.
\par 65 Di antara bangsa-bangsa itu kamu tak akan menemukan ketentraman, dan tak ada tempat yang dapat kamu sebut milikmu; TUHAN akan membuat kamu sangat cemas dan putus asa tanpa harapan.
\par 66 Hidupmu akan selalu terancam bahaya. Siang malam kamu merasa ngeri dan takut mati.
\par 67 Waktu melihat apa saja jantungmu berdebar-debar karena takut. Pagi-pagi kamu mengharapkan malam, dan malam-malam kamu mengharapkan pagi.
\par 68 TUHAN akan mengirim kamu kembali ke Mesir dengan kapal, biarpun Ia telah berkata bahwa kamu tak akan kembali lagi ke sana. Di sana kamu akan berusaha menjual dirimu sebagai budak kepada musuhmu, tetapi tak seorang pun mau membeli kamu."

\chapter{29}

\par 1 Inilah kata-kata perjanjian yang atas perintah TUHAN dibuat Musa dengan bangsa Israel waktu mereka berada di tanah Moab. Perjanjian itu merupakan tambahan dari perjanjian yang dibuat TUHAN dengan mereka di Gunung Sinai.
\par 2 Musa menyuruh semua orang Israel berkumpul, lalu berkata kepada mereka, "Kamu sudah melihat sendiri apa yang dilakukan TUHAN terhadap raja Mesir, para pejabatnya dan seluruh negerinya.
\par 3 Kamu juga sudah melihat bencana-bencana yang dahsyat, mujizat-mujizat dan keajaiban-keajaiban yang dilakukan TUHAN.
\par 4 Tetapi sampai hari ini Ia belum memungkinkan kamu mengerti semua yang kamu alami itu.
\par 5 Empat puluh tahun lamanya TUHAN memimpin kamu melalui padang gurun, namun pakaian dan sandalmu tidak menjadi usang.
\par 6 Kamu tidak mempunyai roti untuk makanan, air anggur atau minuman keras, tetapi TUHAN memberi segala yang kamu perlukan supaya kamu tahu bahwa Dialah TUHAN Allahmu.
\par 7 Dan ketika kita sampai di tempat ini, Raja Sihon dari Hesybon dan Raja Og dari Basan datang memerangi kita. Tetapi kita kalahkan mereka.
\par 8 Tanah mereka kita rebut dan kita bagikan kepada suku Ruben, Gad dan sebagian suku Manasye.
\par 9 Maka hendaklah kamu setia mentaati seluruh perjanjian ini, supaya kamu berhasil dalam segala usahamu.
\par 10 Hari ini kamu semua berdiri di hadapan TUHAN Allahmu, semua kepala suku, pemuka, perwira, semua laki-laki,
\par 11 wanita, anak-anak, dan semua orang asing yang tinggal di antara kamu dan membantu kamu memotong kayu dan memikul air.
\par 12 TUHAN Allahmu membuat perjanjian dengan kamu, dan kamu semua berkumpul di sini untuk mengikat perjanjian itu serta menerima kewajiban-kewajibannya.
\par 13 Dengan perjanjian itu TUHAN hari ini mengukuhkan kamu menjadi bangsa-Nya, dan Ia menjadi Allahmu seperti yang dijanjikan-Nya kepadamu dan kepada nenek moyangmu Abraham, Ishak dan Yakub.
\par 14 Perjanjian itu dengan segala kewajibannya dibuat TUHAN bukan dengan kamu saja,
\par 15 tetapi dengan setiap orang yang pada hari ini berdiri di sini, di hadapan TUHAN Allah kita, dan juga dengan keturunan kita di masa depan."
\par 16 "Kamu masih ingat bagaimana nasib kita di Mesir dan bagaimana kita berjalan melalui negeri bangsa-bangsa lain.
\par 17 Kamu sudah melihat patung-patung berhala mereka yang menjijikkan dari kayu, batu, perak dan emas.
\par 18 Jadi jagalah jangan sampai seorang pun di antara kamu, laki-laki atau perempuan, keluarga atau suku yang berdiri di sini, berbalik dari TUHAN Allah kita untuk menyembah ilah-ilah bangsa lain. Perbuatan itu seperti akar yang tumbuh menjadi tanaman pahit dan beracun.
\par 19 Di antara kamu yang mendengar tuntutan yang keras itu janganlah ada yang berpikir bahwa nasibnya akan tetap baik, sekalipun ia dengan nekat mengikuti jalannya sendiri. Hal itu akan membinasakan kamu semua, orang-orang yang jahat maupun yang baik.
\par 20 TUHAN tidak akan mengampuni orang seperti itu. Sebaliknya, dia akan kena kemarahan TUHAN yang meluap-luap, dan segala bencana yang tertulis di dalam buku ini akan menimpa dia, sampai ia dibinasakan TUHAN.
\par 21 TUHAN akan menjadikan dia sebuah contoh untuk semua suku Israel, dan menghukum dia dengan kutuk yang disebut dalam perjanjian yang tertulis di dalam buku ini, buku Hukum TUHAN.
\par 22 Di kemudian hari, keturunanmu dan orang asing dari negeri jauh akan melihat bencana-bencana dan penderitaan yang didatangkan TUHAN atas negerimu.
\par 23 Ladang-ladangmu akan menjadi padang tandus tertutup belerang dan garam. Tanahnya tak dapat ditanami dan rumput tak mau tumbuh di situ. Negerimu akan menyerupai kota-kota Sodom dan Gomora, Adma dan Zeboim yang dimusnahkan TUHAN ketika kemarahan-Nya berkobar-kobar.
\par 24 Maka seluruh dunia akan bertanya, 'Mengapa TUHAN berbuat begitu dengan negeri mereka? Apa sebab Ia marah sehebat itu?'
\par 25 Dan jawabnya ialah, 'Karena mereka mengingkari perjanjian TUHAN, Allah nenek moyang mereka; perjanjian yang dibuat-Nya dengan mereka pada waktu Ia membawa mereka keluar dari Mesir.
\par 26 Mereka mengabdi kepada ilah-ilah lain yang tidak mereka kenal, padahal TUHAN melarang mereka menyembah ilah-ilah itu.
\par 27 Itulah sebabnya TUHAN marah kepada umat-Nya dan mendatangkan atas negeri mereka semua bencana yang tertulis di dalam buku ini.
\par 28 Kemarahan TUHAN meledak dengan hebatnya, dan dalam kemarahan-Nya yang besar Ia mencabut bangsa Israel dari tanah mereka dan melemparkan mereka ke negeri asing sehingga mereka ada di sana.'
\par 29 Ada beberapa hal yang dirahasiakan TUHAN Allah kita; tetapi hukum-Nya telah dinyatakan-Nya kepada kita, dan kita serta keturunan kita harus mentaatinya untuk selama-lamanya."

\chapter{30}

\par 1 "Sekarang kamu boleh memilih antara berkat dan kutuk seperti yang saya terangkan kepadamu. Kalau kamu sudah mengalami semua penderitaan itu dan hidup di antara bangsa-bangsa asing tempat kamu diceraiberaikan TUHAN Allahmu, kamu akan teringat kepada pilihan yang saya berikan kepadamu.
\par 2 Kalau kamu dan keturunanmu mau kembali kepada TUHAN Allahmu, dan dengan sepenuh hati mentaati perintah-perintah-Nya yang saya berikan kepadamu hari ini,
\par 3 maka TUHAN Allahmu akan mengasihani kamu. Kamu akan dibawa-Nya kembali dari bangsa-bangsa di mana kamu diceraiberaikan, dan dijadikan makmur kembali.
\par 4 Sekalipun kamu terpencar ke mana-mana di seluruh bumi, TUHAN Allahmu akan mengumpulkan kamu dan membawa kamu kembali.
\par 5 Maka kamu dapat memiliki lagi tanah yang dahulu didiami nenek moyangmu. Dan kamu akan dijadikan lebih makmur, dan jumlahmu jauh lebih besar daripada leluhurmu.
\par 6 Kamu dan keturunanmu akan diberi hati yang penurut, sehingga kamu mencintai TUHAN Allahmu dengan sepenuh hatimu dan dapat terus tinggal di negeri itu.
\par 7 Semua kutuk itu akan dijatuhkan-Nya ke atas musuh-musuhmu yang membenci dan menindas kamu.
\par 8 Kamu akan kembali mentaati TUHAN dan menjalankan segala perintah-Nya yang saya berikan kepadamu hari ini.
\par 9 TUHAN Allahmu akan membuat semua usahamu berhasil; kamu akan mempunyai banyak anak dan banyak ternak; ladang-ladangmu akan memberi hasil yang berlimpah-limpah. TUHAN dengan senang hati menjadikan kamu makmur, seperti Ia juga senang memberi kemakmuran kepada nenek moyangmu.
\par 10 Tetapi kamu harus taat kepada TUHAN Allahmu dan mentaati segala hukum-Nya yang tertulis di dalam buku ini, buku Hukum TUHAN. Kamu harus kembali kepada-Nya dengan sepenuh hatimu.
\par 11 Perintah yang saya berikan kepadamu tidak terlalu sulit bagimu dan tidak juga terlalu jauh.
\par 12 Tempatnya bukan di langit, sehingga kamu bertanya, 'Siapa akan naik ke sana untuk mengambilnya bagi kita, supaya kita dapat mendengar dan mentaatinya?'
\par 13 Tempatnya bukan juga di seberang laut, sehingga kamu bertanya, 'Siapa akan ke seberang laut untuk mengambilnya bagi kita, supaya kita dapat mendengar dan mentaatinya?'
\par 14 Jangan bertanya begitu, karena perintah itu sangat dekat padamu. Kamu sudah tahu dan dapat mengucapkannya di luar kepala, jadi tinggal melakukannya saja.
\par 15 Hari ini kamu boleh pilih antara yang baik dan yang jahat, antara hidup dan mati.
\par 16 Hari ini aku memerintahkan kamu untuk mencintai TUHAN Allahmu dan taat kepada-Nya serta melakukan segala hukum dan perintah-Nya supaya kamu makmur dan menjadi bangsa yang besar. TUHAN Allahmu akan memberkati kamu di negeri yang tak lama lagi kamu duduki.
\par 17 Tetapi kalau kamu tidak taat dan tidak mau mendengarkan, dan membiarkan dirimu disesatkan untuk menyembah ilah-ilah lain,
\par 18 pasti kamu dibinasakan, dan umurmu tak akan panjang di negeri yang tak lama lagi kamu duduki. Ingatlah! Hal itu sudah saya beritahukan kepadamu hari ini.
\par 19 Sekarang kamu boleh pilih antara hidup dan mati, antara berkat TUHAN dan kutuk-Nya. Langit dan bumi saya panggil menjadi saksi atas keputusanmu. Pilihlah hidup.
\par 20 Cintailah TUHAN Allahmu, taatilah Dia, dan setialah kepada-Nya. Maka kamu dan keturunanmu akan hidup dan panjang umur di negeri yang dijanjikan TUHAN kepada nenek moyangmu, Abraham, Ishak dan Yakub."

\chapter{31}

\par 1 Musa berbicara lagi kepada bangsa Israel,
\par 2 katanya, "Sekarang umur saya sudah seratus dua puluh tahun dan saya tak mampu lagi menjadi pemimpinmu. Selain itu TUHAN telah berkata bahwa saya tak boleh menyeberangi Sungai Yordan.
\par 3 TUHAN Allahmu sendiri akan berjalan di depanmu dan membinasakan bangsa-bangsa yang tinggal di situ, sehingga kamu dapat menduduki tanah mereka. Yosua akan menjadi pemimpinmu, seperti yang dikatakan TUHAN.
\par 4 TUHAN akan membinasakan bangsa-bangsa itu, seperti Ia mengalahkan Sihon dan Og, raja orang Amori, serta menghancurkan negeri mereka.
\par 5 TUHAN akan memberi kamu kemenangan atas mereka, lalu mereka harus kamu perlakukan tepat seperti yang saya katakan kepadamu.
\par 6 Hendaklah kamu teguh hati dan berani! Jangan takut kepada mereka, sebab TUHAN Allahmu sendiri yang akan menolong kamu. Ia tidak akan mengecewakan atau meninggalkan kamu."
\par 7 Lalu Musa memanggil Yosua, dan di hadapan seluruh bangsa Israel ia berkata kepadanya, "Hendaklah engkau teguh hati dan berani, Yosua! Engkaulah yang akan memimpin bangsa ini untuk menduduki negeri yang dijanjikan TUHAN kepada nenek moyang mereka.
\par 8 TUHAN sendiri membimbing dan menolong engkau. Ia tak akan mengecewakan atau meninggalkan engkau. Sebab itu janganlah takut atau cemas."
\par 9 Lalu Musa menuliskan hukum TUHAN Allah dan memberikannya kepada imam-imam Lewi yang ditugaskan untuk mengurus Peti Perjanjian, dan kepada para pemimpin Israel.
\par 10 Ia memerintahkan kepada mereka, "Pada akhir tiap tahun ketujuh, dalam tahun penghapusan hutang pada pesta Pondok Daun,
\par 11 kamu harus membacakan hukum-hukum ini dengan suara nyaring bagi orang-orang Israel yang datang menyembah TUHAN Allahmu di tempat yang dipilih-Nya.
\par 12 Suruhlah semua orang laki-laki, perempuan dan anak-anak serta orang asing yang tinggal di kota-kotamu berkumpul untuk mendengar pembacaan itu, supaya mereka belajar menghormati TUHAN Allahmu serta setia mentaati perintah-perintah-Nya.
\par 13 Dengan cara itu anak-anakmu yang belum pernah mendengar tentang hukum TUHAN Allahmu dapat mendengarnya dan belajar menghormati TUHAN selama mereka hidup di negeri yang tak lama lagi kamu duduki di seberang Sungai Yordan."
\par 14 Kemudian TUHAN berkata kepada Musa, "Hidupmu tidak lama lagi. Jadi, panggillah Yosua dan bawalah dia ke Kemah-Ku, supaya Aku dapat memberi petunjuk-petunjuk kepadanya." Maka pergilah Musa dan Yosua ke Kemah TUHAN,
\par 15 dan TUHAN menampakkan diri kepada mereka dalam tiang awan dekat pintu Kemah.
\par 16 Kata TUHAN kepada Musa, "Ajalmu sudah dekat. Sesudah kepergianmu, bangsa Israel tidak setia lagi kepada-Ku dan mengingkari perjanjian yang Kubuat dengan mereka. Mereka akan meninggalkan Aku untuk menyembah ilah-ilah yang disembah di negeri yang tak lama lagi mereka masuki.
\par 17 Kalau itu terjadi, kemarahan-Ku akan meluap terhadap mereka. Mereka akan Kutinggalkan dan Kubinasakan. Banyak bencana dahsyat akan menimpa mereka, lalu mereka akan sadar bahwa semua itu terjadi karena Aku, Allah mereka, tidak lagi menyertai mereka.
\par 18 Pada waktu itu Aku tak mau menolong mereka, karena mereka telah menyembah ilah-ilah lain dan berbuat jahat.
\par 19 Sekarang tulislah nyanyian ini dan ajarkanlah kepada bangsa Israel sebagai kesaksian terhadap mereka.
\par 20 Mereka akan Kubawa ke negeri yang kaya dan subur, seperti Kujanjikan kepada nenek moyang mereka. Di situ mereka akan hidup dengan senang dan mendapat segala makanan yang mereka perlukan. Tetapi kemudian mereka akan berbalik dan menyembah ilah-ilah lain. Mereka akan menolak Aku dan mengingkari perjanjian-Ku,
\par 21 dan banyak bencana dahsyat akan menimpa mereka. Tetapi nyanyian ini akan tetap dinyanyikan dan dipakai sebagai kesaksian terhadap mereka. Bahkan sekarang, sebelum mereka Kubawa ke negeri yang Kujanjikan, Aku tahu jalan pikiran mereka."
\par 22 Pada hari itu juga Musa menuliskan nyanyian itu, lalu mengajarkannya kepada bangsa Israel.
\par 23 Kemudian TUHAN berbicara kepada Yosua, anak Nun, kata-Nya, "Hendaklah engkau teguh hati dan berani, Yosua! Engkau akan memimpin bangsa Israel memasuki negeri yang Kujanjikan kepada mereka, dan Aku akan menolong mereka."
\par 24 Lalu Musa menuliskan hukum Allah dalam sebuah buku. Ia menuliskannya dengan teliti dari awal sampai akhir.
\par 25 Ketika selesai, ia berkata kepada para imam Lewi yang ditugaskan untuk mengurus Peti Perjanjian,
\par 26 "Ambillah buku Hukum Allah ini, dan taruhlah di sebelah Peti Perjanjian TUHAN Allahmu, supaya tetap ada di situ sebagai kesaksian terhadap bangsa itu.
\par 27 Saya tahu mereka sangat keras kepala dan suka memberontak. Selagi saya masih hidup pun mereka berontak terhadap TUHAN; apalagi nanti, sesudah saya mati!
\par 28 Sekarang suruhlah semua pemimpin bangsa dan para perwira berkumpul di depan saya, supaya saya dapat menyampaikan hal-hal ini kepada mereka. Langit dan bumi saya panggil sebagai saksi terhadap mereka.
\par 29 Saya tahu bahwa sesudah saya mati, orang-orang itu akan menjadi jahat dan menolak apa yang sudah saya ajarkan kepada mereka. Dan kelak mereka ditimpa bencana, karena membuat TUHAN marah dengan melakukan apa yang dilarang-Nya."
\par 30 Kemudian Musa mengucapkan nyanyian ini sampai selesai sementara seluruh bangsa Israel mendengarkan.

\chapter{32}

\par 1 "Dengarlah, hai langit, aku ingin berbicara; hai bumi, pasanglah telinga!
\par 2 Semoga ajaranku turun seperti hujan, dan kata-kataku menetes seperti embun, laksana hujan rintik-rintik di atas rerumputan, dan hujan deras di atas tanam-tanaman.
\par 3 Aku akan memuji nama TUHAN. Wartakanlah kebesaran Allah kita!
\par 4 TUHAN Pembelamu yang perkasa, karya-Nya sempurna, dan semua jalan-Nya adil. Allahmu setia, tak ada kecurangan pada-Nya, Ia melakukan yang baik dan benar.
\par 5 Tetapi kamu adalah bangsa yang tidak setia, dan tidak pantas lagi menjadi umat-Nya, bangsa yang penuh kecurangan dan dosa.
\par 6 Begitukah kamu balas kebaikan TUHAN, hai orang-orang bodoh dan tidak berakal? Bukankah Ia Bapamu, Penciptamu, yang menjadikan kamu satu bangsa?
\par 7 Ingatlah akan zaman dahulu, perhatikan zaman angkatan-angkatan yang lalu. Tanyakanlah kepada orang tuamu, supaya mereka memberitahukannya kepadamu.
\par 8 Ketika Yang Mahatinggi membagikan tanah, setiap bangsa ditentukan wilayahnya dengan suatu ilah sebagai penguasa.
\par 9 Tetapi keturunan Yakub ini dipilih TUHAN bagi diri-Nya sendiri.
\par 10 Ia mendapati mereka sedang mengembara di tempat yang sepi di padang belantara. Lalu Ia melindungi dan memelihara mereka dan menjaga mereka seperti milik-Nya.
\par 11 Seperti burung elang yang mengajar anaknya terbang menangkapnya di sayapnya yang terbentang dan mendukungnya di atas kepak-kepaknya,
\par 12 demikian TUHAN sendiri memimpin umat-Nya, tak ada ilah asing menyertai dia.
\par 13 Ia membuat mereka menguasai gunung-gunung; mereka makan hasil ladang-ladang dan minum madu liar dari bukit batu. Mereka mendapat minyak dari pohon zaitun yang tumbuh di tanah berbatu.
\par 14 Sapi dan kambing domba mereka menghasilkan banyak susu; ternak mereka paling bermutu, gandum dan air anggur mereka paling baik.
\par 15 Umat TUHAN menjadi kaya, tetapi suka berontak, mereka gemuk-gemuk, kenyang makanan. Lalu mereka meninggalkan Allah, Penciptanya dan menolak Penyelamatnya yang perkasa.
\par 16 Mereka menyembah berhala, sehingga TUHAN cemburu, kemarahan-Nya bangkit karena kejahatan itu.
\par 17 Mereka mempersembahkan kurban kepada roh-roh jahat yang bukan Allah, kepada ilah-ilah yang tidak mereka kenal, ilah-ilah baru yang tidak ditakuti leluhur mereka.
\par 18 Mereka melalaikan Penyelamat mereka yang perkasa, melupakan Allah mereka yang memberi kehidupan.
\par 19 Melihat hal itu, TUHAN menjadi marah, dan menolak anak-anak-Nya.
\par 20 Kata-Nya, 'Aku tidak mau menolong mereka lagi; biar Kulihat bagaimana kesudahan mereka. Sebab bangsa itu keras kepala dan sama sekali tidak setia.
\par 21 Mereka membuat Aku cemburu kepada yang bukan Allah, dan marah kepada patung berhala mereka. Maka Kubuat umat-Ku cemburu kepada yang bukan bangsa, dan marah kepada bangsa yang dungu.
\par 22 Kemarahan-Ku akan berkobar seperti api yang membakar sampai ke dunia orang mati. Api itu membakar bumi dan segala hasilnya, dan gunung-gunung sampai ke akar-akarnya.
\par 23 Terus-menerus Kudatangkan malapetaka, Kutembakkan semua panah-Ku kepada mereka.
\par 24 Mereka akan mati karena demam dan kelaparan, dan karena penyakit yang mengerikan. Kubiarkan mereka diterkam binatang buas, dan dipagut ular berbisa.
\par 25 Peperangan membawa maut di jalan-jalan; kengerian menimpa orang-orang di rumah. Orang muda dan gadis remaja akan dibinasakan, juga anak bayi dan orang lanjut usia.
\par 26 Aku ingin melenyapkan mereka sama sekali, supaya mereka tidak diingat lagi.
\par 27 Tetapi Aku tak mau musuh Israel berbangga; jangan-jangan mereka salah sangka dan berkata bahwa mereka mengalahkan umat-Ku,' padahal Akulah yang melakukannya.
\par 28 Israel suatu bangsa yang tidak punya pertimbangan; pengertian tak ada pada mereka.
\par 29 Sekiranya mereka bijaksana, mereka akan mengerti, dan memikirkan kesudahan mereka.
\par 30 Bagaimana seorang dapat mengejar seribu orang, dan dua orang membuat sepuluh ribu orang lari? TUHAN, Allah mereka, telah menjual mereka; mereka telah ditinggalkan oleh Allah yang perkasa.
\par 31 Musuh Israel tahu bahwa pelindung mereka tidak seperti Allah pelindung Israel.
\par 32 Sebab mereka jahat seperti Sodom dan Gomora, seperti pohon anggur yang pahit dan beracun buahnya.
\par 33 Air anggur mereka adalah racun ular berbisa.
\par 34 TUHAN ingat perbuatan musuh umat-Nya dan akan menghukum mereka pada waktunya.
\par 35 TUHAN akan membalas dan menghukum mereka, tak lama lagi mereka jatuh binasa; saat kehancuran mereka segera tiba.
\par 36 TUHAN akan menyelamatkan umat-Nya dan mengasihani hamba-hamba-Nya, bila Ia melihat mereka tak berdaya dan sudah kehabisan tenaga.
\par 37 Maka TUHAN akan bertanya kepada umat-Nya, 'Di mana ilah-ilah kuat yang kamu andalkan?'
\par 38 Mereka makan lemak yang kamu persembahkan, dan minum air anggur yang kamu kurbankan. Suruhlah mereka datang menolong kamu, biarlah mereka menjadi perlindunganmu.
\par 39 Lihatlah, Aku Allah Yang Esa, tak ada Allah kecuali Aku. Aku membunuh dan menghidupkan, melukai dan menyembuhkan. Bila Aku bertindak, tak seorang pun dapat melawan.
\par 40 Demi Aku sendiri, Allah yang hidup, Kuangkat tangan-Ku dan bersumpah:
\par 41 Aku akan mengasah pedang-Ku yang berkilauan, dan menjalankan penghukuman. Kubalas semua lawan-Ku dan Kuhukum semua yang membenci Aku.
\par 42 Panah-panah-Ku akan dilumuri darah mereka; semua yang menentang Aku Kubunuh dengan pedang-Ku. Tidak Kubiarkan siapa pun melawan Aku; orang tahanan dan yang luka-luka mesti mati juga.
\par 43 Hai bangsa-bangsa, pujilah umat TUHAN, sebab TUHAN menghukum semua yang membunuh mereka. Ia membalas dendam kepada musuh-Nya tetapi mengampuni dosa umat-Nya."
\par 44 Musa dan Yosua anak Nun mengucapkan nyanyian itu, sehingga seluruh bangsa Israel dapat mendengarnya.
\par 45 Sehabis menyampaikan ajaran-ajaran TUHAN kepada bangsa Israel,
\par 46 Musa berkata, "Perhatikanlah semua perintah yang saya berikan kepadamu hari ini. Ajarkanlah kepada anak-anakmu, supaya mereka dengan setia melakukan semua hukum TUHAN.
\par 47 Ajaran itu bukanlah kata-kata kosong, melainkan hidupmu. Taatilah semua perintah itu, supaya kamu panjang umur di negeri yang tak lama lagi kamu duduki di seberang Sungai Yordan."
\par 48 Pada hari itu juga TUHAN berkata kepada Musa,
\par 49 "Pergilah ke Pegunungan Abarim di negeri Moab di seberang kota Yerikho. Lalu naiklah ke Gunung Nebo dan pandanglah tanah Kanaan yang tak lama lagi Kuserahkan kepada bangsa Israel.
\par 50 Engkau akan meninggal di atas gunung itu, seperti Harun saudaramu meninggal di Gunung Hor.
\par 51 Kamu berdua tidak setia kepada-Ku dan tidak menghormati Aku di depan bangsa Israel waktu kamu berada di mata air Meriba, dekat kota Kades, di padang gurun Zin.
\par 52 Karena itu engkau tak boleh masuk ke negeri yang Kuberikan kepada bangsa Israel; engkau hanya boleh memandangnya dari jauh."

\chapter{33}

\par 1 Sebelum meninggal, Musa, utusan Allah, memberkati bangsa Israel dengan kata-kata ini:
\par 2 "TUHAN datang dari Gunung Sinai; Ia terbit di atas Edom laksana matahari dan dari Gunung Paran Ia menyinari umat-Nya. Ia disertai sepuluh ribu malaikat; api menyala di sebelah kanan-Nya.
\par 3 Sungguh, TUHAN mengasihi bangsa-Nya Ia melindungi semua orang yang menjadi milik-Nya. Mari kita sujud di depan kaki-Nya dan mentaati perintah-perintah-Nya.
\par 4 Hukum yang diberi Musa kita taati, milik paling berharga bangsa ini.
\par 5 Waktu Israel berkumpul dengan para pemimpinnya, TUHAN menjadi Raja atas umat-Nya."
\par 6 Tentang suku Ruben, Musa berkata, "Semoga suku Ruben tak pernah binasa, biarpun orangnya sedikit saja."
\par 7 Tentang suku Yehuda ia berkata: "TUHAN, dengarlah seruan Yehuda; satukanlah mereka kembali dengan bangsanya. Berperanglah bagi mereka, ya TUHAN, tolonglah mereka terhadap lawan."
\par 8 Tentang suku Lewi ia berkata: "Dengan Urim dan Tumim Kaunyatakan kehendak-Mu, melalui orang Lewi, hamba-Mu yang setia. Engkau telah mencobai mereka di Masa, dan berbantah dengan mereka di mata air Meriba.
\par 9 Mereka setia kepada-Mu melebihi segalanya, melebihi orang tua dan sanak saudara. Perintah-perintah-Mu mereka taati, perjanjian-Mu mereka tepati.
\par 10 Mereka mengajar umat-Mu mentaati semua perintah-Mu dan mempersembahkan kurban di atas mezbah-Mu.
\par 11 TUHAN, tolonglah suku ini supaya kuat, semoga Engkau berkenan kepada pekerjaan mereka. Binasakanlah semua musuh mereka sehingga tak dapat bangkit lagi."
\par 12 Tentang suku Benyamin ia berkata: "Inilah suku yang dikasihi TUHAN. Ia melindungi mereka siang dan malam dan berdiam di antara mereka."
\par 13 Tentang keturunan Yusuf ia berkata: "Semoga tanah mereka diberkati TUHAN dengan hujan, dan dengan air dari bawah bumi.
\par 14 Semoga setiap musim limpah dengan buah-buahan buah-buahan terbaik yang masak di pohon.
\par 15 Semoga bukit-bukitnya dari zaman purba penuh dengan hasil tanah yang istimewa.
\par 16 Semoga tanahnya dilimpahi dengan segala yang paling baik, dan diberkati oleh kebaikan TUHAN yang berbicara dari semak bernyala. Semoga berkat itu turun ke atas keturunan Yusuf, sebab dialah pemimpin di antara saudara-saudaranya.
\par 17 Yusuf mempunyai kekuatan seekor banteng, dan tanduk sapi jantan hutan itulah ribuan Manasye dan puluhan ribu Efraim. Bangsa-bangsa diseruduknya dan didesaknya sampai ke ujung-ujung bumi."
\par 18 Tentang suku-suku Zebulon dan Isakhar ia berkata: "Semoga Zebulon makmur karena perjalanan-perjalanannya, dan Isakhar sejahtera di kemah-kemahnya.
\par 19 Mereka mengundang bangsa-bangsa ke gunung mereka dan mempersembahkan kurban yang benar di sana. Mereka memperoleh kekayaan dari laut, dan dari pasir di sepanjang pantai."
\par 20 Tentang suku Gad ia berkata: "Terpujilah Allah yang memperluas wilayah Gad. Gad menanti seperti seekor singa hendak mengoyak lengan atau kulit kepala.
\par 21 Tanah yang paling baik diambil untuk dirinya bagian pimpinan diserahkan kepada mereka. Mereka melakukan segala perintah dan hukum TUHAN waktu para pemimpin Israel berkumpul bersama."
\par 22 Tentang suku Dan ia berkata: "Dan seperti seekor singa muda yang melompat keluar dari Basan."
\par 23 Tentang suku Naftali ia berkata: "Naftali dilimpahi dengan berkat dan kasih TUHAN; wilayahnya dari Danau Galilea sampai ke selatan."
\par 24 Tentang suku Asyer ia berkata: "Asyer diberkati melebihi suku-suku lainnya; semoga ia menjadi kesayangan saudara-saudaranya, dan pohon zaitun tumbuh subur di tanahnya.
\par 25 Semoga kota-kotanya dilindungi dengan pintu-pintu gerbang tembaga dan besi agar hidupnya selalu aman dan tentram."
\par 26 "Hai Israel! Tak ada ilah seperti Allahmu; Ia melintasi langit dengan jaya dan menerobos awan untuk menolong kamu.
\par 27 Allah selamanya menjadi perlindunganmu, lengan-Nya yang kekal menopang kamu. Ia mengusir musuh dari hadapanmu, dan menyuruh kamu membinasakan mereka.
\par 28 Maka keturunan Yakub akan hidup dengan tentram di negeri yang limpah gandum dan air anggurnya, dan diairi dengan embun dari langit.
\par 29 Berbahagialah engkau, hai Israel! Tak ada bangsa yang seperti engkau diselamatkan TUHAN. TUHAN seperti pedang dan perisai bagimu; untuk membela engkau dan memberi kemenangan kepadamu. Musuh-musuhmu akan datang mohon kasihan, dan engkau akan menginjak-injak mereka."

\chapter{34}

\par 1 Lalu Musa meninggalkan dataran Moab dan mendaki Gunung Nebo, ke puncak Pisga di sebelah timur Yerikho. Di situ TUHAN menunjukkan kepadanya seluruh negeri itu, yakni: Wilayah Gilead ke utara sejauh kota Dan;
\par 2 seluruh wilayah Naftali; wilayah Efraim dan Manasye, wilayah Yehuda ke barat sejauh Laut Tengah;
\par 3 bagian selatan Yehuda dan dataran yang terbentang dari Zoar ke Yerikho, kota yang penuh dengan pohon kurma.
\par 4 Lalu TUHAN berkata kepada Musa, "Itulah negeri yang Kujanjikan kepada Abraham, Ishak dan Yakub untuk diserahkan kepada keturunan mereka. Aku memperlihatkannya kepadamu, Musa, tetapi tidak mengizinkan engkau pergi ke sana."
\par 5 Lalu meninggallah Musa hamba TUHAN di tanah Moab, seperti dikatakan TUHAN sebelumnya.
\par 6 TUHAN menguburkan dia di sebuah lembah di Moab di seberang kota Bet-Peor, tetapi sampai hari ini tak seorang pun tahu dengan tepat di mana makamnya.
\par 7 Musa meninggal pada usia seratus dua puluh tahun. Kekuatannya tidak berkurang dan penglihatannya masih terang.
\par 8 Di dataran Moab bangsa Israel menangisi Musa dan berkabung selama tiga puluh hari.
\par 9 TUHAN melimpahi Yosua anak Nun dengan kebijaksanaan karena ia telah ditunjuk Musa menjadi penggantinya. Bangsa Israel taat kepada Yosua dan menjalankan perintah-perintah yang diberikan TUHAN kepada mereka melalui Musa.
\par 10 Di Israel tak ada lagi nabi seperti Musa yang berbicara berhadapan muka dengan TUHAN.
\par 11 Tak ada yang melakukan mujizat-mujizat dan keajaiban-keajaiban seperti yang dilakukan Musa terhadap negeri Mesir, rajanya dan para pejabatnya sesuai dengan perintah TUHAN.
\par 12 Tak ada yang melakukan perbuatan-perbuatan hebat dan dahsyat seperti yang dilakukan Musa di depan seluruh bangsa Israel.


\end{document}