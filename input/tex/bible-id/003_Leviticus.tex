\begin{document}

\title{Imamat}


\chapter{1}

\par 1 Dari Kemah-Nya, TUHAN memanggil Musa dan memberi kepadanya peraturan-peraturan
\par 2 yang harus ditaati orang Israel. Apabila seseorang mempersembahkan binatang untuk kurban bakaran bagi TUHAN, ia boleh mempersembahkan sapi, domba atau kambingnya.
\par 3 Kalau yang dipersembahkan seekor sapi, binatang itu harus yang jantan dan tidak ada cacatnya. Sapi itu harus dibawanya ke pintu Kemah TUHAN, supaya TUHAN berkenan kepada orang itu.
\par 4 Lalu ia harus meletakkan tangannya di atas kepala binatang itu, maka persembahannya akan diterima TUHAN untuk pengampunan dosanya.
\par 5 Kemudian sapi itu harus disembelih di depan pintu Kemah. Darahnya harus dipersembahkan oleh imam-imam keturunan Harun kepada TUHAN, lalu disiramkan pada keempat sisi mezbah di depan pintu Kemah.
\par 6 Kemudian orang itu harus menguliti dan memotong-motong binatang itu.
\par 7 Imam-imam harus menyalakan api di atas mezbah lalu menyusun kayu bakar di atasnya.
\par 8 Potongan-potongan kurban, termasuk kepala dan lemaknya, harus mereka letakkan di atas api.
\par 9 Lalu orang itu harus mencuci isi perut serta paha sapi itu, dan imam yang bertugas harus membakar seluruhnya. Bau kurban bakaran itu menyenangkan hati TUHAN.
\par 10 Kalau yang dipersembahkan seekor domba atau kambing, binatang itu harus yang jantan dan tidak ada cacatnya.
\par 11 Orang itu harus menyembelih binatang itu di sebelah utara mezbah, lalu imam-imam harus menyiramkan darahnya pada keempat sisi mezbah.
\par 12 Sesudah binatang itu dipotong-potong, semua potongan, termasuk kepala dan lemaknya, harus diletakkan oleh imam di atas api di mezbah.
\par 13 Lalu orang itu harus mencuci isi perut dan paha binatang itu. Kemudian imam harus mempersembahkan seluruhnya kepada TUHAN, dan membakarnya di atas mezbah. Bau kurban bakaran itu menyenangkan hati TUHAN.
\par 14 Kalau yang dipersembahkan seekor burung, harus burung tekukur atau burung merpati muda.
\par 15 Imam harus membawa burung itu ke mezbah, lalu memuntir lehernya dan membakar kepalanya di atas mezbah. Darah burung itu harus dipencet keluar pada samping mezbah.
\par 16 Tembolok dengan seluruh isinya harus dikeluarkan, lalu dibuang di sebelah timur mezbah di tempat abu.
\par 17 Kemudian burung itu harus dicabik pada pangkal sayapnya, tetapi sayapnya tidak boleh sampai terputus. Lalu burung itu harus dibakar di atas mezbah. Bau kurban bakaran itu menyenangkan hati TUHAN.

\chapter{2}

\par 1 Apabila seseorang mempersembahkan kurban sajian kepada TUHAN, kurban itu harus berupa tepung yang paling baik yang dituangi minyak zaitun dan diberi kemenyan di atasnya.
\par 2 Ia harus membawanya kepada imam-imam keturunan Harun. Imam yang bertugas harus mengambil segenggam dari tepung itu dengan minyak zaitunnya dan semua kemenyannya, lalu membakarnya di atas mezbah sebagai tanda bahwa seluruhnya sudah dipersembahkan kepada TUHAN. Bau kurban makanan itu menyenangkan hati TUHAN.
\par 3 Kurban sajian yang selebihnya adalah untuk para imam. Kurban itu sangat suci karena diambil dari makanan yang sudah dipersembahkan kepada TUHAN.
\par 4 Kalau yang dipersembahkan itu roti yang dibakar di dalam pembakaran, roti itu harus dibuat tanpa ragi. Roti itu boleh tebal, dibuat dari adonan tepung halus dengan minyak zaitun, boleh juga kue yang hanya dioles dengan minyak zaitun.
\par 5 Kalau yang dipersembahkan itu roti yang dipanggang di atas panggangan, roti itu harus dibuat dari tepung halus dengan minyak zaitun, tetapi tidak pakai ragi.
\par 6 Roti yang dipersembahkan itu harus dibelah-belah, lalu dituangi minyak zaitun.
\par 7 Kalau yang dipersembahkan itu roti yang dimasak dalam kuali, roti itu harus dibuat dari tepung halus dan minyak zaitun.
\par 8 Roti yang dibuat dengan cara itu harus dipersembahkan kepada TUHAN dan diserahkan kepada imam yang membawanya ke mezbah.
\par 9 Imam mengambil sebagian dari roti itu lalu membakarnya di atas mezbah sebagai tanda bahwa seluruhnya sudah dipersembahkan kepada TUHAN. Baunya menyenangkan hati TUHAN.
\par 10 Selebihnya adalah untuk para imam. Roti itu sangat suci karena diambil dari makanan yang sudah dipersembahkan kepada TUHAN.
\par 11 Kurban sajian untuk TUHAN tak boleh dicampur dengan madu atau ragi, karena madu dan ragi tak boleh dibakar untuk persembahan bagi TUHAN.
\par 12 Dari hasil tanah yang pertama setiap tahun, sebagian harus dipersembahkan kepada TUHAN, tetapi tak boleh dibakar di atas mezbah.
\par 13 Setiap kurban sajian harus diberi garam, karena garam menandakan perjanjian Allah dengan kamu. Jadi semua persembahan harus diberi garam.
\par 14 Apabila kamu mempersembahkan kepada TUHAN gandum pertama yang kamu tuai, bawalah gandum baru yang sudah ditumbuk halus atau dipanggang.
\par 15 Gandum itu harus dituangi minyak zaitun dan diberi kemenyan di atasnya.
\par 16 Imam harus membakar sebagian dari gandum dan minyak itu serta semua kemenyannya, sebagai tanda bahwa seluruhnya sudah dipersembahkan kepada TUHAN.

\chapter{3}

\par 1 Apabila seseorang mempersembahkan seekor sapi untuk kurban perdamaian, sapi itu boleh jantan, boleh betina, tetapi tak boleh ada cacatnya.
\par 2 Orang yang mempersembahkannya harus meletakkan tangannya di atas kepala binatang itu lalu menyembelihnya di depan pintu Kemah TUHAN. Imam-imam keturunan Harun harus menyiramkan darah sapi itu pada keempat sisi mezbah.
\par 3 Untuk kurban perdamaian kepada TUHAN, imam harus mempersembahkan bagian-bagian ini dari binatang itu: lemak yang membungkus isi perutnya,
\par 4 ginjal dengan lemaknya dan bagian yang paling baik dari hatinya.
\par 5 Semuanya itu bersama-sama dengan kurban bakaran lainnya, harus dibakar di atas mezbah. Bau kurban makanan itu menyenangkan hati TUHAN.
\par 6 Kalau yang dipersembahkan untuk kurban perdamaian itu seekor domba atau kambing, binatang itu boleh jantan, boleh betina, tetapi tak boleh ada cacatnya.
\par 7 Apabila seseorang mempersembahkan seekor domba, ia harus membawanya ke Kemah TUHAN.
\par 8 Di situ ia harus meletakkan tangannya di atas kepala domba itu lalu menyembelihnya di depan Kemah. Lalu imam harus menyiramkan darah domba itu pada keempat sisi mezbah.
\par 9 Untuk kurban perdamaian kepada TUHAN, imam harus mempersembahkan bagian-bagian ini dari domba itu: lemaknya, seluruh ekornya yang berlemak dipotong dekat tulang belakangnya, lemak yang membungkus isi perutnya,
\par 10 ginjal dengan lemaknya dan bagian yang paling baik dari hatinya.
\par 11 Imam yang bertugas harus membakar semua itu di atas mezbah untuk kurban makanan bagi TUHAN.
\par 12 Kalau yang dipersembahkan seekor kambing, orang yang mempersembahkannya harus membawanya ke Kemah TUHAN.
\par 13 Di situ ia harus meletakkan tangannya di atas kepala kambing itu lalu menyembelihnya di depan Kemah. Imam harus menyiramkan darah binatang itu pada keempat sisi mezbah.
\par 14 Untuk kurban makanan bagi TUHAN, imam harus mempersembahkan bagian-bagian ini dari kambing itu: lemak yang membungkus isi perutnya,
\par 15 ginjal dengan lemaknya dan bagian yang paling baik dari hatinya.
\par 16 Imam yang bertugas harus membakar semua itu di atas mezbah. Bau kurban makanan itu menyenangkan hati TUHAN. Semua lemak binatang itu untuk TUHAN.
\par 17 Orang Israel sekali-kali tak boleh makan lemak atau darah. Peraturan itu berlaku untuk selama-lamanya dan harus ditaati oleh setiap orang Israel di mana saja ia tinggal.

\chapter{4}

\par 1 TUHAN menyuruh Musa
\par 2 mengumumkan kepada bangsa Israel bahwa siapa saja yang dengan tidak disengaja berdosa karena melanggar salah satu dari perintah-perintah TUHAN, harus mengikuti peraturan ini.
\par 3 Apabila yang berdosa Imam Agung, sehingga bangsa Israel ikut bersalah, ia harus mengurbankan seekor sapi jantan muda yang tidak ada cacatnya kepada TUHAN supaya dosanya diampuni.
\par 4 Sapi itu harus dibawanya ke pintu Kemah TUHAN. Imam Agung harus meletakkan tangannya di atas kepala binatang itu, lalu menyembelihnya di depan Kemah.
\par 5 Sebagian dari darahnya harus dibawanya masuk ke dalam Kemah.
\par 6 Lalu ia harus mencelupkan jarinya ke dalam darah itu dan memercikkannya tujuh kali di depan tirai yang memisahkan Ruang Suci.
\par 7 Sebagian dari darah itu harus dioleskan pada tanduk-tanduk di sudut-sudut mezbah dupa harum di dalam Kemah, dan selebihnya harus disiramkan pada dasar mezbah kurban bakaran yang ada di depan pintu Kemah.
\par 8 Imam harus mengambil seluruh lemak sapi jantan itu, yaitu lemak yang membungkus isi perutnya,
\par 9 ginjal dengan lemaknya dan bagian yang paling baik dari hatinya.
\par 10 Semua itu harus dibakarnya di atas mezbah kurban bakaran seperti pada kurban perdamaian.
\par 11 Tetapi daging sapi itu, kulitnya, kepalanya, kakinya, isi perut selebihnya termasuk ususnya,
\par 12 jadi seluruhnya kecuali bagian yang sudah dipisahkan, harus dibawa ke luar perkemahan. Di situ sapi itu harus dibakar di atas kayu api di tempat pembuangan abu, yaitu tempat yang dikhususkan untuk itu.
\par 13 Apabila yang berdosa seluruh umat Israel, sebab dengan tidak disengaja melanggar salah satu perintah TUHAN,
\par 14 maka segera setelah dosa itu diketahui, umat harus membawa seekor sapi jantan muda ke depan Kemah TUHAN. Binatang itu harus dipersembahkan untuk kurban pengampunan dosa.
\par 15 Para pemimpin bangsa harus meletakkan tangan mereka di atas kepala binatang itu, lalu menyembelihnya di tempat itu.
\par 16 Imam Agung harus membawa sebagian dari darah binatang itu masuk ke dalam Kemah.
\par 17 Di situ ia harus mencelupkan jarinya ke dalam darah itu lalu memercikkannya tujuh kali di depan tirai yang memisahkan Ruang Suci.
\par 18 Sebagian lagi harus dioleskan pada tanduk-tanduk di sudut-sudut mezbah dupa di dalam Kemah, dan selebihnya harus disiramkan ke dasar mezbah tempat kurban bakaran yang ada di depan pintu Kemah.
\par 19 Seluruh lemak binatang itu harus diambil lalu dibakar di atas mezbah.
\par 20 Caranya seperti pada persembahan sapi jantan untuk kurban pengampunan dosa. Begitulah cara Imam Agung mempersembahkan kurban untuk dosa-dosa umat, maka dosa-dosa itu akan diampuni TUHAN.
\par 21 Kemudian sapi jantan itu harus dibawa ke luar, dan dibakar di luar perkemahan, seperti pada persembahan sapi jantan untuk kurban pengampunan dosa Imam Agung itu sendiri. Begitulah cara mempersembahkan kurban pengampunan dosa umat.
\par 22 Apabila yang berdosa seorang penguasa karena dengan tidak disengaja melanggar salah satu perintah TUHAN,
\par 23 maka segera setelah menyadarinya, ia harus mengurbankan seekor kambing jantan yang tidak ada cacatnya untuk pengampunan dosanya.
\par 24 Ia harus meletakkan tangannya di atas kepala binatang itu, lalu menyembelihnya di sebelah utara mezbah, tempat memotong binatang untuk kurban bakaran.
\par 25 Imam harus mencelupkan jarinya ke dalam darah binatang itu lalu mengoleskannya pada tanduk-tanduk di sudut-sudut mezbah dan menyiramkan sisanya ke dasar mezbah.
\par 26 Kemudian seluruh lemak binatang itu harus dibakar di atas mezbah, seperti pada kurban perdamaian. Begitulah cara imam mempersembahkan kurban untuk seorang penguasa, maka orang itu akan diampuni TUHAN.
\par 27 Apabila seorang dari rakyat biasa berbuat dosa karena dengan tidak disengaja melanggar salah satu perintah TUHAN,
\par 28 maka segera setelah menyadarinya, ia harus mengurbankan seekor kambing betina yang tak ada cacatnya.
\par 29 Ia harus meletakkan tangannya di atas kepala binatang itu, lalu menyembelihnya di sebelah utara, mezbah tempat memotong binatang untuk kurban bakaran.
\par 30 Imam harus mencelupkan jarinya ke dalam darah binatang itu, lalu mengoleskannya pada tanduk-tanduk di sudut-sudut mezbah dan menyiramkan sisanya ke dasar mezbah.
\par 31 Seluruh lemak binatang itu harus diambil seperti pada kurban perdamaian. Lemak itu harus dibakar oleh imam di atas mezbah, supaya baunya menyenangkan hati TUHAN. Begitulah cara imam mempersembahkan kurban untuk dosa seorang dari rakyat biasa, maka orang itu akan diampuni TUHAN.
\par 32 Apabila seseorang mengurbankan seekor domba untuk pengampunan dosanya, domba itu harus betina dan tidak ada cacatnya.
\par 33 Ia harus meletakkan tangannya di atas kepala binatang itu, lalu menyembelihnya di sebelah utara mezbah, tempat memotong binatang untuk kurban bakaran.
\par 34 Imam harus mencelupkan jarinya ke dalam darah binatang itu, lalu mengoleskannya pada tanduk-tanduk di sudut-sudut mezbah, dan menyiramkan sisanya ke dasar mezbah.
\par 35 Kemudian seluruh lemak binatang itu harus diambil seperti pada kurban perdamaian. Lemak itu harus dibakar di atas mezbah, bersama-sama dengan kurban makanan yang dipersembahkan kepada TUHAN. Begitulah cara imam mempersembahkan kurban untuk dosa orang itu, maka ia akan diampuni TUHAN.

\chapter{5}

\par 1 Kurban pengampunan dosa perlu dipersembahkan: Apabila seseorang secara resmi diminta memberi kesaksian di pengadilan, tetapi tidak memberi keterangan tentang apa yang sudah dilihat atau didengarnya sehingga ia harus menanggung akibatnya.
\par 2 Apabila seseorang bersalah karena dengan tidak disengaja menjadi najis, sebab telah menyentuh sesuatu yang najis, misalnya bangkai binatang, dan kemudian menyadari perbuatannya.
\par 3 Apabila seseorang bersalah karena dengan tidak disengaja menyentuh barang najis yang berasal dari manusia, dan kemudian menyadari perbuatannya.
\par 4 Apabila seseorang bersalah karena bersumpah dengan sembarangan untuk melakukan apa saja, dan kemudian menyadari perbuatannya.
\par 5 Dalam setiap perkara itu, orang yang bersalah harus mengakui kesalahannya.
\par 6 Dan untuk tebusan dosanya, ia harus membawa seekor domba atau kambing betina untuk TUHAN. Imam harus mempersembahkan kurban itu supaya dosa orang itu diampuni.
\par 7 Tetapi apabila orang yang bersalah tidak mampu menyediakan domba atau kambing untuk menebus dosanya, ia harus membawa sepasang burung tekukur atau burung merpati muda, seekor untuk kurban pengampunan dosa, dan seekor lagi untuk kurban bakaran.
\par 8 Burung-burung itu harus dibawa kepada imam. Imam harus lebih dahulu mempersembahkan burung yang ditentukan untuk kurban pengampunan dosa. Leher burung itu harus dipuntir, tetapi tak boleh sampai putus kepalanya.
\par 9 Darah burung itu harus dipercikkan pada sisi mezbah, dan selebihnya dipencet keluar pada dasar mezbah untuk kurban pengampunan dosa.
\par 10 Sesudah itu imam mempersembahkan burung yang kedua untuk kurban bakaran, sesuai dengan peraturan. Begitulah cara imam mempersembahkan kurban untuk dosa orang itu, maka ia akan diampuni TUHAN.
\par 11 Kalau orang yang bersalah tidak mampu menyediakan sepasang burung tekukur atau burung merpati untuk kurban pengampunan dosa, ia harus membawa satu kilogram tepung. Tepung itu tidak boleh dicampur minyak zaitun atau disertai kemenyan, sebab kurban itu adalah kurban pengampunan dosa dan bukan kurban sajian.
\par 12 Tepung itu harus dibawa kepada imam. Imam harus mengambil segenggam dari tepung itu sebagai tanda bahwa seluruhnya sudah dikurbankan kepada TUHAN. Tepung yang segenggam itu harus dibakar di atas mezbah sebagai kurban makanan. Itulah kurbannya untuk pengampunan dosa.
\par 13 Begitulah cara imam mempersembahkan kurban untuk dosa orang itu, maka ia akan diampuni TUHAN. Tepung yang selebihnya adalah bagian imam, seperti pada kurban gandum.
\par 14 TUHAN memberi kepada Musa peraturan-peraturan ini.
\par 15 Kalau seseorang dengan tidak disengaja berbuat dosa karena lalai dalam menyerahkan persembahan-persembahan yang diwajibkan untuk TUHAN, maka orang itu harus membawa kurban ganti rugi. Kurban itu harus seekor domba atau kambing jantan yang tidak ada cacatnya, dan dinilai menurut harga yang berlaku di Kemah TUHAN.
\par 16 Orang itu harus membayar apa yang dahulu dilalaikannya, ditambah dengan dua puluh persen. Kemudian ia harus menyerahkan binatang itu kepada imam, dan imam mengurbankannya untuk dosa orang itu, maka ia akan diampuni TUHAN.
\par 17 Kalau seseorang dengan tidak disengaja berbuat dosa karena melanggar salah satu perintah TUHAN, orang itu bersalah dan harus menanggung akibatnya.
\par 18 Untuk kurban ganti rugi ia harus membawa seekor domba atau kambing jantan yang tidak ada cacatnya dan dinilai menurut harga yang berlaku di Kemah TUHAN. Imam mengurbankannya untuk dosa orang itu, maka ia akan diampuni TUHAN.
\par 19 Itulah kurban ganti rugi untuk kesalahan yang dilakukannya terhadap TUHAN.

\chapter{6}

\par 1 TUHAN memberi kepada Musa peraturan-peraturan ini.
\par 2 Seseorang harus mempersembahkan kurban kalau ia berdosa terhadap TUHAN karena tak mau mengembalikan barang yang dititipkan kepadanya oleh sesamanya, atau menipu orang, mencuri barangnya, atau sudah menemukan barang yang hilang, tetapi memungkirinya dengan sumpah.
\par 4 Orang yang bersalah itu harus membayar kembali semua yang diperolehnya dengan tidak jujur. Pada waktu ia kedapatan bersalah, ia harus membayar penuh kepada pemiliknya, ditambah dua puluh persen.
\par 6 Untuk kurban ganti rugi orang itu harus mempersembahkan seekor domba atau kambing jantan yang tidak ada cacatnya dan dinilai menurut harga yang berlaku di Kemah TUHAN.
\par 7 Imam mengurbankannya untuk dosa orang itu, maka ia akan diampuni TUHAN.
\par 8 TUHAN menyuruh Musa
\par 9 memberi kepada Harun dan anak-anaknya peraturan-peraturan ini tentang kurban yang dibakar seluruhnya. Imam harus meletakkan kurban bakaran di atas mezbah dan membiarkannya di situ sepanjang malam, dan apinya harus menyala terus.
\par 10 Sesudah kurban itu menjadi abu, imam berpakaian celana pendek dan jubah dari kain linen, harus mengangkat abu berlemak itu dari atas mezbah dan menaruhnya di samping mezbah.
\par 11 Sesudah itu ia harus ganti pakaian dan membawa abu itu ke luar perkemahan, ke tempat yang dikhususkan untuk itu.
\par 12 Setiap pagi imam harus menaruh kayu api di atas mezbah, mengatur kurban bakaran di atasnya, dan membakar lemak dari kurban perdamaian. Api di atas mezbah harus terus menyala dan tidak boleh dibiarkan padam.
\par 14 Inilah peraturan-peraturan tentang kurban sajian untuk TUHAN. Kurban itu harus dibawa ke depan mezbah oleh seorang dari keturunan Harun.
\par 15 Ia harus mengambil segenggam tepung halus dengan minyak dan semua kemenyannya, lalu membakarnya di atas mezbah sebagai tanda bahwa seluruh kurban itu dipersembahkan kepada TUHAN. Bau kurban itu menyenangkan hati TUHAN.
\par 16 Tepung yang selebihnya harus dijadikan roti tak beragi dan dimakan oleh imam-imam di tempat yang khusus, yaitu di pelataran Kemah TUHAN. Itulah bagian para imam yang diberikan TUHAN kepada mereka dari kurban makanan. Makanan itu sangat suci, seperti kurban pengampunan dosa dan kurban ganti rugi.
\par 18 Untuk selama-lamanya setiap orang laki-laki keturunan Harun boleh makan roti itu; itulah bagian mereka yang tetap dari makanan yang dipersembahkan kepada TUHAN. Orang lain yang menyentuh makanan yang sudah dikurbankan itu, akan mendapat celaka karena kekuatan kesuciannya.
\par 19 TUHAN memberi kepada Musa peraturan-peraturan ini
\par 20 tentang kurban pada pentahbisan seorang imam keturunan Harun. Pada hari ia ditahbiskan, ia harus mempersembahkan satu kilogram tepung gandum kepada TUHAN. Separuhnya harus dipersembahkan pada waktu pagi dan sisanya pada waktu sore.
\par 21 Tepung itu harus dicampur dengan minyak, lalu dibakar di atas panggangan, kemudian dibelah-belah dan dipersembahkan sebagai kurban sajian. Bau kurban itu menyenangkan hati TUHAN.
\par 22 Untuk selama-lamanya setiap keturunan Harun yang ditahbiskan menjadi Imam Agung harus mempersembahkan kurban itu. Seluruhnya harus dibakar untuk persembahan bagi TUHAN.
\par 23 Setiap kurban sajian yang dipersembahkan oleh imam harus dibakar seluruhnya, sedikit pun tidak boleh dimakan.
\par 24 TUHAN menyuruh Musa
\par 25 memberi kepada Harun dan anak-anaknya peraturan-peraturan ini tentang kurban pengampunan dosa. Binatang untuk kurban pengampunan dosa harus disembelih di tempat memotong binatang untuk kurban bakaran. Kurban itu sangat suci.
\par 26 Imam yang mempersembahkan kurban itu harus memakannya di tempat yang khusus, yaitu di pelataran Kemah TUHAN.
\par 27 Siapa pun atau apa pun yang kena daging ternak itu akan mendapat celaka karena kekuatan kesuciannya. Kalau sehelai baju kena percikan darah binatang itu, baju itu harus dicuci di tempat yang khusus.
\par 28 Periuk tanah yang dipakai untuk merebus daging kurban itu harus dipecahkan. Kalau yang dipakai periuk logam, periuk itu harus digosok lalu dibersihkan dengan air.
\par 29 Setiap orang laki-laki dalam keluarga imam-imam boleh makan kurban yang sangat suci itu.
\par 30 Binatang untuk upacara pengampunan dosa yang diambil sebagian darahnya untuk dibawa ke Ruang Suci, harus dibakar habis dan tak boleh dimakan.

\chapter{7}

\par 1 Inilah peraturan-peraturan tentang kurban ganti rugi. Kurban itu sangat suci.
\par 2 Binatang untuk kurban harus disembelih di tempat memotong binatang untuk kurban bakaran, lalu darahnya disiramkan pada keempat sisi mezbah.
\par 3 Imam harus mempersembahkan di atas mezbah semua lemak binatang itu, yaitu: Ekornya yang berlemak, lemak yang menutupi isi perutnya, ginjal dengan lemaknya, dan bagian yang paling baik dari hatinya.
\par 5 Semua itu harus dibakarnya di atas mezbah sebagai kurban makanan bagi TUHAN. Kurban itu adalah untuk kurban ganti rugi.
\par 6 Setiap orang laki-laki dalam keluarga imam-imam boleh makan daging kurban itu, tetapi mereka harus memakannya di tempat yang khusus, karena kurban itu sangat suci.
\par 7 Untuk kurban pengampunan dosa dan kurban ganti rugi berlaku peraturan yang sama, yaitu: Daging ternak yang dikurbankan adalah bagian imam yang mempersembahkannya.
\par 8 Kulit ternak itu juga untuk imam.
\par 9 Setiap kurban sajian yang dibakar dalam pembakaran atau dimasak dalam kuali atau yang dipanggang, menjadi bagian imam yang mempersembahkannya.
\par 10 Tetapi kurban sajian yang tidak dimasak, baik yang dicampur dengan minyak maupun yang kering, seluruhnya untuk imam-imam keturunan Harun, dan harus dibagi rata antara mereka.
\par 11 Inilah peraturan-peraturan tentang kurban perdamaian yang dipersembahkan kepada TUHAN.
\par 12 Kalau kurban itu dipersembahkan untuk berterima kasih, maka dengan binatang itu harus dikurbankan juga roti yang tidak pakai ragi. Roti itu boleh yang tebal dari adonan tepung dengan minyak zaitun, boleh juga roti kering tipis yang dioles dengan minyak, atau kue tepung yang dibuat dengan minyak zaitun.
\par 13 Selain itu harus dipersembahkan juga roti yang dibuat pakai ragi.
\par 14 Dari tiap macam roti sebagian harus dipersembahkan kepada TUHAN, dan itu bagian imam yang mengambil darah binatang itu dan menyiramkannya pada mezbah untuk kurban perdamaian.
\par 15 Dagingnya harus dimakan pada hari binatang itu dipersembahkan, sedikit pun tak boleh ditinggalkan sampai besok paginya.
\par 16 Apabila seseorang membawa kurban perdamaian untuk menepati kaulnya, atau untuk kurban sukarela, tidak perlu seluruhnya dimakan pada hari itu juga. Bagian selebihnya boleh dimakan besoknya.
\par 17 Kalau pada hari yang ketiga masih ada sisanya, daging itu harus dibakar habis.
\par 18 Tetapi kalau pada hari yang ketiga daging itu dimakan juga, maka kurban itu tidak akan diterima dan tidak membawa berkat bagi orang itu, malah menjadi haram. Orang yang memakannya bersalah dan harus menanggung akibatnya.
\par 19 Daging yang kena sesuatu yang najis harus dibakar habis dan tak boleh dimakan. Daging kurban perdamaian hanya boleh dimakan oleh orang yang tidak najis.
\par 20 Kalau orang memakan daging itu pada waktu ia sedang najis, ia tidak lagi dianggap anggota umat TUHAN.
\par 21 Apabila seseorang makan daging kurban itu sesudah menyentuh sesuatu yang najis, baik yang berasal dari manusia maupun dari hewan, ia juga tidak lagi dianggap anggota umat TUHAN.
\par 22 TUHAN memberi kepada Musa peraturan-peraturan ini
\par 23 untuk bangsa Israel. Lemak sapi, domba atau kambing tidak boleh dimakan.
\par 24 Lemak binatang yang mati dengan sendirinya atau karena diterkam oleh binatang buas, tak boleh dimakan tetapi boleh dipakai untuk keperluan lain.
\par 25 Orang yang makan lemak binatang yang bisa dipersembahkan untuk kurban makanan bagi Allah, tidak lagi dianggap anggota umat TUHAN.
\par 26 Orang Israel, di mana pun ia tinggal, tak boleh makan darah burung atau darah binatang lainnya.
\par 27 Siapa yang melanggar peraturan itu tidak lagi dianggap anggota umat TUHAN.
\par 28 TUHAN memberi kepada Musa peraturan-peraturan ini
\par 29 untuk bangsa Israel. Barangsiapa membawa kurban perdamaian, harus memisahkan sebagian untuk persembahan khusus bagi TUHAN.
\par 30 Ia sendiri harus membawa kurban makanan itu. Lemak dan dada binatang itu harus dipersembahkan sebagai persembahan unjukan bagi TUHAN.
\par 31 Imam harus membakar lemak binatang itu di atas mezbah, tetapi dadanya adalah bagian para imam.
\par 32 Paha kanannya harus diberikan sebagai sumbangan khusus
\par 33 kepada imam yang mempersembahkan darah dan lemak kurban perdamaian itu.
\par 34 Dada dan paha kanan adalah pemberian khusus yang diambil TUHAN dari persembahan unjukan bangsa Israel dan diberikan kepada para imam. Itulah sumbangan bangsa Israel bagi para imam untuk selama-lamanya.
\par 35 Dari makanan yang dipersembahkan kepada TUHAN, bagian itulah yang diberikannya kepada Harun dan anak-anaknya pada hari mereka ditahbiskan menjadi imam.
\par 36 Pada hari itu TUHAN memerintahkan bangsa Israel supaya dari persembahan mereka, bagian itu untuk para imam. Peraturan ini harus ditaati bangsa Israel untuk selama-lamanya.
\par 37 Begitulah peraturan-peraturan tentang kurban bakaran, kurban sajian, kurban pengampunan dosa, kurban ganti rugi, kurban pentahbisan dan kurban perdamaian.
\par 38 TUHAN memberi perintah-perintah itu kepada Musa di Gunung Sinai pada waktu Ia menyuruh bangsa Israel mempersembahkan kurban di padang gurun Sinai.

\chapter{8}

\par 1 TUHAN berkata kepada Musa,
\par 2 "Panggillah Harun dan anak-anaknya dan bawalah pakaian imam, minyak upacara, seekor sapi jantan muda untuk kurban pengampunan dosa, dua ekor domba jantan dan sebakul roti yang tak beragi.
\par 3 Lalu suruhlah seluruh umat Israel berkumpul di depan pintu Kemah-Ku."
\par 4 Musa melakukan apa yang diperintahkan TUHAN. Waktu seluruh umat sudah berkumpul di tempat itu,
\par 5 Musa berkata kepada mereka, "Yang saya lakukan sekarang ini adalah perintah TUHAN."
\par 6 Lalu ia menyuruh Harun dan anak-anaknya tampil ke depan dan membasuh diri.
\par 7 Sesudahnya ia mengenakan pakaian imam pada Harun: kemeja, ikat pinggang dan jubah. Juga efod yang diikat ke pinggangnya dengan ikat dari kain halus.
\par 8 Lalu dipasangnya tutup dada pada Harun, dan ke dalam tutup dada itu dimasukkannya Urim dan Tumim.
\par 9 Lalu ditaruhnya serban di kepala Harun, dan pada serban itu disematkannya hiasan emas, yang menandakan bahwa Harun sudah menjadi milik TUHAN. Semua itu dilakukan Musa menurut perintah TUHAN kepadanya.
\par 10 Lalu Musa mengambil minyak upacara dan mengoleskannya pada Kemah TUHAN dan segala perlengkapannya. Dengan demikian semuanya dikhususkannya bagi TUHAN.
\par 11 Sedikit dari minyak itu dipercikkannya tujuh kali ke atas mezbah dan semua perlengkapannya, juga ke atas bak dan dasarnya. Dengan demikian semuanya dikhususkannya bagi TUHAN.
\par 12 Sesudahnya itu Musa mentahbiskan Harun dengan menuangkan sedikit minyak upacara ke atas kepalanya.
\par 13 Lalu ia menyuruh anak-anak Harun tampil ke depan. Dikenakannya pada mereka kemeja dan ikat pinggang. Lalu ditaruhnya serban di kepala mereka seperti yang diperintahkan TUHAN kepadanya.
\par 14 Sesudahnya Musa mengambil seekor sapi jantan muda untuk kurban pengampunan dosa. Harun dan anak-anaknya meletakkan tangan mereka di atas kepala binatang itu.
\par 15 Lalu Musa menyembelihnya dan mengambil sebagian darahnya. Darah itu dioleskannya dengan jarinya pada tanduk-tanduk di sudut-sudut mezbah, supaya mezbah itu dikhususkan bagi Allah. Darah yang selebihnya disiramkannya pada dasar mezbah. Begitulah caranya mezbah itu disucikan dan dikhususkan bagi TUHAN.
\par 16 Dari sapi yang baru dipotong itu, Musa mengambil semua lemak yang menutupi isi perutnya, ginjal dengan lemaknya dan bagian yang paling baik dari hatinya. Semua itu dibakarnya di atas mezbah.
\par 17 Tetapi daging sapi itu, kulitnya, isi perut selebihnya termasuk ususnya, jadi seluruh sapi itu kecuali bagian-bagian yang sudah dipisahkan dibakarnya di luar perkemahan seperti yang diperintahkan TUHAN.
\par 18 Kemudian Musa mengambil domba jantan untuk kurban bakaran, lalu Harun dan anak-anaknya meletakkan tangan mereka di atas kepala domba itu.
\par 19 Musa menyembelih domba itu dan darahnya disiramkannya pada keempat sisi mezbah.
\par 20 Lalu binatang itu dipotong-potongnya dan isi perut serta kakinya dicucinya dengan air. Kepala, lemak, dan selebihnya dari daging domba jantan itu dibakarnya untuk kurban bakaran di atas mezbah seperti yang diperintahkan TUHAN kepadanya. Bau kurban makanan itu menyenangkan hati TUHAN.
\par 22 Lalu Musa mengambil domba jantan yang seekor lagi untuk kurban pentahbisan imam. Harun dan anak-anaknya meletakkan tangan mereka di atas kepala domba itu,
\par 23 lalu Musa menyembelihnya. Sesudah itu diambilnya sedikit darah binatang itu, dan dioleskannya pada cuping telinga kanan Harun, pada ibu jari tangan kanan, dan ibu jari kaki kanannya.
\par 24 Kemudian ia menyuruh anak-anak Harun mendekat, lalu dioleskannya juga sedikit darah pada cuping telinga kanan mereka, pada ibu jari tangan kanan, serta ibu jari kaki kanan mereka. Darah yang selebihnya disiramkannya pada keempat sisi mezbah.
\par 25 Musa mengambil lemak domba itu, ekornya yang berlemak, semua lemak yang menutupi isi perutnya, bagian yang paling baik dari hatinya, ginjal dengan lemaknya serta kaki kanannya.
\par 26 Dari bakul yang berisi roti tak beragi yang sudah dipersembahkan kepada TUHAN, Musa mengambil satu roti tebal yang dibuat dengan minyak dan satu roti tipis. Roti itu diletakkannya di atas lemak dan paha kanan domba.
\par 27 Seluruhnya ditaruhnya di tangan Harun dan anak-anaknya, lalu mereka mempersembahkannya sebagai persembahan unjukan kepada TUHAN.
\par 28 Sesudahnya Musa mengambil makanan itu dari mereka dan membakarnya di atas mezbah, bersama dengan kurban bakaran untuk kurban pentahbisan. Bau kurban makanan itu menyenangkan hati TUHAN.
\par 29 Lalu Musa mengambil dada domba itu dan mempersembahkannya sebagai persembahan unjukan kepada TUHAN. Itulah bagian Musa dari domba jantan yang dikurbankan untuk upacara pentahbisan imam. Semuanya dilakukan Musa seperti yang diperintahkan TUHAN kepadanya.
\par 30 Sesudah itu Musa mengambil sebagian dari minyak upacara dan sebagian dari darah yang ada di atas mezbah, lalu dipercikkannya kepada Harun dan anak-anaknya serta ke pakaian mereka. Dengan cara itu Musa mengkhususkan mereka dan pakaian mereka kepada TUHAN.
\par 31 Kemudian Musa berkata kepada Harun dan anak-anaknya, "Masaklah daging kurban itu di depan pintu Kemah TUHAN. Menurut perintah TUHAN, kalian harus memakannya di situ dengan roti kurban pentahbisan yang ada di dalam bakul.
\par 32 Daging dan roti yang selebihnya harus kalian bakar habis.
\par 33 Selama tujuh hari kalian tak boleh meninggalkan Kemah TUHAN sampai upacara pentahbisanmu selesai.
\par 34 TUHAN memerintahkan supaya apa yang kita lakukan hari ini juga dilakukan selanjutnya. Dengan upacara itu dosa-dosamu diampuni.
\par 35 Selama tujuh hari, siang dan malam, kalian harus tinggal di depan pintu Kemah untuk melakukan apa yang diperintahkan TUHAN. Kalau kalian tidak melakukannya, kalian akan mati. Itulah yang diperintahkan TUHAN kepada saya."
\par 36 Lalu Harun dan anak-anaknya melakukan semua yang diperintahkan TUHAN melalui Musa.

\chapter{9}

\par 1 Sehari sesudah upacara pentahbisan itu selesai, pada hari yang kedelapan, Musa memanggil Harun dan anak-anaknya serta para pemimpin bangsa Israel.
\par 2 Kata Musa kepada Harun, "Ambillah seekor sapi jantan muda dan seekor domba jantan, kedua-duanya yang tak ada cacatnya. Persembahkanlah kedua binatang itu kepada TUHAN, sapi jantan untuk kurban pengampunan dosa, dan domba jantan untuk kurban bakaran.
\par 3 Lalu suruhlah bangsa Israel mempersembahkan seekor kambing jantan untuk kurban pengampunan dosa dan seekor anak sapi serta seekor anak domba yang masing-masing berumur satu tahun dan tak ada cacatnya untuk kurban bakaran.
\par 4 Suruhlah mereka mempersembahkan juga seekor sapi jantan dan seekor kambing jantan untuk kurban perdamaian. Semua itu harus mereka persembahkan kepada TUHAN bersama-sama dengan kurban sajian yang diberi minyak, karena pada hari ini TUHAN datang kepada mereka."
\par 5 Semua yang dipesan Musa mereka bawa ke depan Kemah, dan seluruh umat berkumpul di situ untuk beribadat kepada TUHAN.
\par 6 Lalu Musa berkata, "TUHAN menyuruh kamu melakukan ini supaya kamu dapat melihat cahaya kehadiran TUHAN."
\par 7 Sesudah itu ia berkata kepada Harun, "Pergilah ke mezbah dan siapkanlah kurban pengampunan dosa dan kurban bakaran, supaya dosamu dan dosa umat diampuni TUHAN. Lakukanlah itu seperti yang diperintahkan TUHAN."
\par 8 Maka pergilah Harun ke mezbah dan memotong sapi jantan yang akan dikurbankan untuk pengampunan dosanya sendiri.
\par 9 Anak-anak Harun menyampaikan darah binatang itu kepada Harun, dan ia mencelupkan jarinya ke dalam darah itu, serta mengoleskannya pada tanduk-tanduk di keempat sudut mezbah. Sesudah itu darah yang selebihnya disiramkannya pada dasar mezbah.
\par 10 Lemak, ginjal dan bagian yang paling baik dari binatang itu dibakarnya di atas mezbah, seperti yang diperintahkan TUHAN kepada Musa.
\par 11 Tetapi daging dan kulit binatang itu dibakarnya di luar perkemahan.
\par 12 Kemudian Harun menyembelih binatang untuk kurban bakaran bagi dirinya sendiri. Anak-anak Harun menyampaikan darah binatang itu kepadanya, lalu ia menyiramkannya pada keempat sisi mezbah.
\par 13 Sesudah itu mereka menyerahkan kepadanya kepala dan bagian-bagian lain dari binatang itu, dan ia membakarnya di atas mezbah.
\par 14 Kemudian isi perut dan paha binatang itu dicucinya dan dibakarnya dengan kurban bakaran di atas mezbah.
\par 15 Sesudah itu ia mempersembahkan kambing untuk kurban pengampunan dosa umat. Kambing itu disembelihnya dan dipersembahkannya seperti pada kurban untuk pengampunan dosanya sendiri.
\par 16 Berikut diambilnya juga binatang untuk kurban bakaran, dan dipersembahkannya menurut peraturan.
\par 17 Sesudah itu dipersembahkannya kurban sajian; diambilnya tepung segenggam, dan dibakarnya di atas mezbah. Itu adalah tambahan pada kurban bakaran harian.
\par 18 Lalu dipotongnya juga seekor sapi jantan dan seekor domba jantan untuk kurban perdamaian bagi umat. Anak-anak Harun menyampaikan darah binatang itu kepadanya, lalu ia menyiramkannya pada keempat sisi mezbah.
\par 19 Bagian yang berlemak dari sapi dan kambing itu mereka letakkan
\par 20 di atas dada kedua binatang itu, lalu Harun membakarnya di atas mezbah.
\par 21 Dada dan paha kanan binatang itu dipersembahkannya sebagai persembahan unjukan bagi TUHAN. Itu adalah bagian para imam, seperti yang diperintahkan Musa.
\par 22 Sesudah mempersembahkan semua kurban itu, Harun mengangkat tangannya ke atas umat Israel dan memberkati mereka, lalu ia turun.
\par 23 Kemudian Musa dan Harun masuk ke dalam Kemah TUHAN. Sesudah itu mereka keluar dan memberkati umat. Lalu seluruh umat Israel melihat cahaya kehadiran TUHAN.
\par 24 Tiba-tiba TUHAN menurunkan api yang membakar habis kurban bakaran dan bagian-bagiannya yang berlemak di atas mezbah. Ketika umat Israel melihatnya, mereka bersorak-sorak lalu sujud menyembah.

\chapter{10}

\par 1 Pada suatu hari Nadab dan Abihu, anak-anak Harun, mengambil tempat apinya masing-masing dan mengisinya dengan bara. Lalu mereka menaruh dupa ke dalam api itu dan mempersembahkannya kepada TUHAN. Tetapi api itu tidak halal karena TUHAN tidak menyuruh mereka mempersembahkanny
\par 2 Maka TUHAN mendatangkan api yang membakar mereka sampai mati di Kemah TUHAN.
\par 3 Lalu Musa berkata kepada Harun, "Itulah yang dimaksudkan TUHAN waktu Ia berkata, 'Semua yang melayani Aku harus menghormati kesucian-Ku. Aku akan menyatakan keagungan-Ku kepada seluruh bangsa-Ku.'" Tetapi Harun diam saja.
\par 4 Kemudian Musa memanggil Misael dan Elsafan, anak-anak Uziel, paman Harun. Katanya kepada mereka, "Datanglah ke mari, ambillah jenazah saudara sepupumu dari Kemah TUHAN dan letakkan di luar perkemahan."
\par 5 Mereka datang dan membawa jenazah-jenazah yang masih berpakaian itu ke luar perkemahan, seperti yang diperintahkan Musa.
\par 6 Lalu Musa berkata kepada Harun dan kepada anak-anaknya yang lain, yaitu Eleazar dan Itamar, "Jangan biarkan rambutmu kusut, dan jangan koyakkan pakaianmu untuk menunjukkan bahwa kalian bersedih dan berkabung. Kalau kalian berbuat begitu, kalian akan mati, dan seluruh umat Israel dimarahi TUHAN. Tetapi semua orang Israel yang lainnya boleh berkabung untuk kedua orang yang mati karena api dari TUHAN.
\par 7 Jangan meninggalkan Kemah TUHAN supaya kalian jangan mati, sebab kalian sudah ditahbiskan dengan minyak upacara untuk melayani TUHAN." Mereka berbuat seperti yang dikatakan Musa.
\par 8 TUHAN berkata kepada Harun,
\par 9 "Sehabis minum air anggur atau minuman keras, engkau dan anak-anakmu tidak boleh masuk ke dalam Kemah-Ku. Kalau kamu melanggar peraturan itu, kamu akan mati. Perintah itu harus ditaati oleh semua keturunanmu.
\par 10 Kamu harus dapat membedakan antara yang dikhususkan untuk Allah dan yang tidak dikhususkan, antara yang najis dan yang tidak najis.
\par 11 Semua hukum yang Kuberikan kepadamu melalui Musa, harus dapat kamu ajarkan kepada bangsa Israel."
\par 12 Musa berkata kepada Harun dan kepada kedua anaknya, Eleazar dan Itamar, "Ambillah gandum yang selebihnya dari makanan yang dipersembahkan kepada TUHAN. Dari tepung itu buatlah roti yang tidak pakai ragi, lalu makanlah itu di samping mezbah, karena kurban itu sangat suci.
\par 13 Roti itu harus dimakan di tempat yang suci, sebab dari makanan yang dipersembahkan kepada TUHAN, roti itu menjadi bagianmu dan anak-anakmu yang laki-laki. Itulah perintah TUHAN kepada saya.
\par 14 Tetapi kalian dan anak-anakmu laki-laki dan perempuan boleh makan dada persembahan unjukan dan paha persembahan khusus. Itulah bagian para imam dari kurban perdamaian bangsa Israel. Kalian boleh memakannya di sembarang tempat yang tidak najis.
\par 15 Paha persembahan khusus dan dada persembahan unjukan itu harus dibawa bersama-sama dengan lemak untuk kurban bakaran sebagai persembahan khusus bagi TUHAN. Bagian-bagian itu menjadi milikmu dan anak-anakmu untuk selama-lamanya, seperti yang diperintahkan TUHAN."
\par 16 Kemudian Musa bertanya tentang kambing untuk kurban pengampunan dosa. Ia diberitahukan bahwa kambing itu sudah dibakar. Hal itu membuat Musa marah kepada Eleazar dan Itamar, dan ia berkata kepada mereka,
\par 17 "Mengapa kurban pengampunan dosa itu tidak kalian makan di tempat yang suci? Kurban itu sangat suci; TUHAN memberikannya kepadamu, supaya dosa umat Israel diampuni.
\par 18 Darah kambing itu tidak dibawa ke Kemah TUHAN, jadi seharusnya kurban itu kalian makan di situ seperti yang saya perintahkan."
\par 19 Jawab Harun, "Memang hari ini umat Israel sudah mempersembahkan kurban pengampunan dosa dan kurban bakaran kepada TUHAN. Walaupun begitu, bencana ini telah menimpa saya. Sekiranya hari ini saya makan juga kurban pengampunan dosa itu, apakah TUHAN akan senang?"
\par 20 Musa merasa puas dengan jawaban itu.

\chapter{11}

\par 1 TUHAN memberi kepada Musa dan Harun peraturan-peraturan ini
\par 2 untuk bangsa Israel. Semua binatang yang hidup di darat yang memamah biak dan kukunya terbelah adalah halal dan boleh dimakan.
\par 4 Jangan makan unta, pelanduk atau kelinci. Binatang itu haram karena walaupun memamah biak, kukunya tidak terbelah.
\par 7 Jangan makan babi. Binatang itu haram, karena walaupun kukunya terbelah, ia tidak memamah biak.
\par 8 Dagingnya tak boleh dimakan dan bangkainya pun tak boleh disentuh karena binatang itu haram.
\par 9 Kamu boleh makan segala macam ikan yang bersirip dan bersisik.
\par 10 Binatang yang hidup di dalam air tetapi tidak bersirip dan tidak bersisik adalah haram, jadi tak boleh dimakan, dan bangkainya tak boleh disentuh.
\par 13 Di antara burung-burung, yang berikut ini tak boleh dimakan karena najis: burung rajawali, burung hantu, segala jenis elang, nasar, gagak, burung unta, camar, blekok dan segala jenis bangau, undan, burung kasa dan kelelawar.
\par 20 Semua serangga yang bersayap adalah haram,
\par 21 kecuali yang dapat melompat.
\par 22 Jadi belalang, jangkerik dan belalang hijau boleh dimakan.
\par 23 Tetapi semua serangga lain yang bersayap dan juga merayap adalah haram.
\par 24 Seseorang menjadi najis sampai matahari terbenam kalau ia menyentuh bangkai salah seekor binatang ini: binatang yang berkuku, kecuali yang kukunya terbelah dan memamah biak, dan binatang yang berkaki empat dan ada cakarnya. Barangsiapa membawa bangkai binatang jenis itu harus mencuci pakaiannya, tetapi ia masih najis sampai matahari terbenam.
\par 29 Tikus tanah, tikus besar, tikus kecil, kadal, segala jenis katak, landak, biawak, bengkarung, siput dan bunglon adalah haram.
\par 31 Barangsiapa menyentuh binatang itu atau bangkainya menjadi najis sampai matahari terbenam.
\par 32 Kalau bangkai binatang itu jatuh ke atas suatu benda, maka benda itu menjadi najis. Hal itu berlaku untuk segala macam benda dari kayu, kain, kulit, atau kain karung, dan tidak menjadi soal untuk apa benda itu dipakai. Benda itu harus direndam di dalam air, tetapi tetap najis sampai matahari terbenam.
\par 33 Apabila bangkai itu jatuh ke dalam belanga tanah, segala yang ada di dalam belanga itu menjadi najis, dan belanga itu harus dipecahkan.
\par 34 Makanan yang kena air dari belanga itu menjadi najis, dan minuman yang ada dalam belanga itu juga menjadi najis.
\par 35 Apa saja yang kejatuhan bangkai itu menjadi najis. Anglo atau tempat pembakaran dari tanah yang kena bangkai itu harus dipecahkan karena najis.
\par 36 Kecuali mata air atau sumur, segala yang lain menjadi najis kalau kena bangkai.
\par 37 Kalau bangkai itu jatuh di atas biji yang akan ditaburkan, biji itu tetap bersih.
\par 38 Tetapi kalau biji itu sedang direndam di dalam air lalu kejatuhan bangkai itu, maka biji itu menjadi najis.
\par 39 Barangsiapa menyentuh bangkai binatang yang halal menjadi najis sampai matahari terbenam.
\par 40 Orang yang membawa atau makan bangkai itu harus mencuci pakaiannya, tetapi ia tetap najis sampai matahari terbenam.
\par 41 Semua binatang yang berkeriapan di atas tanah, baik yang merayap, yang berkaki empat atau berkaki banyak, adalah haram dan tak boleh dimakan.
\par 43 Jangan menajiskan dirimu dengan makan binatang jenis itu.
\par 44 Akulah TUHAN Allahmu, sebab itu jagalah agar kamu tetap suci, karena Aku suci.
\par 45 Akulah TUHAN yang membawa kamu keluar dari Mesir, supaya Aku dapat menjadi Allahmu. Hendaklah kamu suci, karena Aku suci.
\par 46 Itulah peraturan tentang binatang termasuk burung, binatang yang hidup di dalam air dan yang berkeriapan di atas tanah.
\par 47 Kamu harus dapat membedakan antara yang bersih dan yang najis, antara binatang yang boleh dimakan dan binatang yang tidak boleh dimakan.

\chapter{12}

\par 1 TUHAN memberi kepada Musa peraturan-peraturan ini
\par 2 untuk bangsa Israel. Apabila seorang wanita melahirkan anak laki-laki, maka selama tujuh hari wanita itu najis, sama seperti waktu ia sedang haid.
\par 3 Pada hari yang kedelapan anaknya harus disunat.
\par 4 Sesudahnya tiga puluh tiga hari lagi wanita itu masih najis karena mengeluarkan darah. Ia tak boleh menyentuh barang-barang yang dipakai untuk ibadat dan tak boleh memasuki Kemah TUHAN sampai masa penyuciannya selesai.
\par 5 Apabila seorang wanita melahirkan anak perempuan, maka selama empat belas hari wanita itu najis, sama seperti waktu ia sedang haid. Sesudahnya enam puluh enam hari lagi wanita itu masih najis karena mengeluarkan darah.
\par 6 Sesudah masa penyuciannya selesai, baik karena telah melahirkan anak laki-laki atau anak perempuan, wanita itu harus membawa persembahan kepada imam di depan pintu Kemah TUHAN. Persembahan itu berupa seekor anak domba yang berumur satu tahun untuk kurban bakaran, dan seekor burung merpati muda atau tekukur muda untuk kurban pengampunan dosa.
\par 7 Imam harus menyerahkan persembahannya kepada TUHAN dan melakukan upacara untuk menghapuskan kenajisan wanita itu, sehingga ia menjadi bersih. Jadi itulah yang harus dilakukan seorang wanita sesudah ia melahirkan.
\par 8 Apabila wanita itu tidak mampu menyediakan seekor anak domba, ia harus membawa dua ekor burung merpati muda atau tekukur muda: seekor untuk kurban bakaran dan seekor lagi untuk kurban pengampunan dosa. Imam harus melakukan upacara untuk menghapuskan kenajisan wanita itu, maka ia menjadi bersih.

\chapter{13}

\par 1 TUHAN memberi kepada Musa dan Harun peraturan-peraturan ini:
\par 2 Apabila seseorang mempunyai bintil-bintil, atau borok atau becak-becak putih pada kulitnya yang bisa menjadi penyakit kulit yang berbahaya, ia harus dibawa kepada imam keturunan Harun.
\par 3 Imam harus memeriksa bagian yang sakit pada kulit orang itu. Kalau bulu yang ada di bagian itu menjadi putih, dan bagian yang sakit nampaknya lebih dalam dari kulit di sekitarnya, orang itu menderita sakit kulit yang berbahaya. Imam harus menyatakan orang itu najis.
\par 4 Tetapi kalau bagian yang sakit berwarna putih dan tidak lebih dalam dari kulit di sekitarnya, dan bulu di bagian itu juga tidak menjadi putih, orang itu harus diasingkan selama tujuh hari.
\par 5 Pada hari yang ketujuh, imam harus memeriksanya kembali. Dan kalau menurut pendapatnya penyakit itu masih sama keadaannya dan tidak menyebar, orang itu harus diasingkan selama tujuh hari lagi.
\par 6 Pada hari yang ketujuh imam harus memeriksanya kembali, dan kalau bagian yang sakit sudah pudar dan tidak menyebar, imam harus menyatakan orang itu bersih. Itu penyakit kulit yang tidak berbahaya. Orang itu harus mencuci pakaiannya dan ia menjadi bersih.
\par 7 Tetapi kalau sesudah dinyatakan bersih oleh imam penyakit orang itu menyebar, ia harus kembali menghadap imam.
\par 8 Imam harus memeriksanya lagi, dan kalau penyakitnya memang sudah menyebar, orang itu harus dinyatakan najis; ia menderita penyakit kulit yang berbahaya.
\par 9 Seseorang yang menderita penyakit kulit yang berbahaya harus dibawa kepada imam.
\par 10 Kalau menurut pemeriksaan imam pada kulit orang itu ada bengkak putih yang memborok dan bulu di bagian itu menjadi putih,
\par 11 orang itu menderita penyakit kulit yang berbahaya. Imam harus menyatakan orang itu najis. Tidak perlu mengasingkan dia, karena sudah jelas bahwa ia najis.
\par 12 Kalau imam melihat bahwa becak-becak putih itu sudah menyebar dari kepala sampai ke kaki,
\par 13 dan menurut pemeriksaannya penyakit itu sudah menyebar ke seluruh badan dan seluruh kulitnya sudah menjadi putih, orang itu dinyatakan bersih.
\par 14 Tetapi kalau timbul borok, imam harus memeriksanya lagi.
\par 15 Kalau itu memang borok, orang itu harus dinyatakan najis, karena ia menderita penyakit kulit yang berbahaya.
\par 16 Tetapi kalau borok itu sembuh dan menjadi putih lagi, orang itu harus menghadap imam
\par 17 untuk diperiksa kembali. Kalau borok itu memang sudah menjadi putih, imam menyatakan orang itu bersih.
\par 18 Kalau seseorang punya bisul yang sudah sembuh,
\par 19 tetapi pada bekasnya timbul bengkak putih atau becak-becak putih kemerah-merahan, orang itu harus menghadap imam
\par 20 untuk diperiksa. Jika becak-becak itu nampaknya lebih dalam dari kulit di sekitarnya dan bulu di bagian itu sudah menjadi putih juga, imam harus menyatakan orang itu najis. Bisul itu permulaan dari penyakit kulit yang berbahaya.
\par 21 Tetapi kalau menurut pemeriksaan imam bulu di bagian itu tidak menjadi putih, dan bagian itu tidak lebih dalam dari kulit di sekitarnya, hanya lebih muda warnanya, orang itu harus diasingkan selama tujuh hari.
\par 22 Kalau becak putih itu menyebar, imam harus menyatakan orang itu najis karena ia menderita penyakit kulit yang berbahaya.
\par 23 Tetapi kalau becak putih itu tidak berubah dan tidak juga menyebar, itu hanya bekas bisul, jadi orang itu dinyatakan bersih.
\par 24 Apabila seseorang kena luka bakar, lalu kulit pada luka itu menjadi putih atau putih kemerah-merahan,
\par 25 imam harus memeriksanya. Kalau bulu di bagian itu menjadi putih dan lukanya lebih dalam dari kulit di sekitarnya, maka penyakit kulit yang berbahaya sudah berjangkit pada luka bakar itu, dan imam harus menyatakan orang itu najis.
\par 26 Tetapi kalau bulu di tempat itu tidak menjadi putih dan bagian itu tidak lebih dalam dari kulit di sekitarnya, hanya lebih muda warnanya, maka orang itu harus diasingkan selama tujuh hari.
\par 27 Pada hari yang ketujuh imam harus memeriksanya kembali. Kalau becak putih itu menyebar, maka orang itu menderita penyakit kulit yang berbahaya, dan harus dinyatakan najis.
\par 28 Tetapi kalau becak putih itu tidak berubah dan tidak juga menyebar, hanya lebih putih warnanya, itu bukan penyakit kulit yang berbahaya. Imam harus menyatakan orang itu bersih, sebab warna putih itu hanya bekas luka bakar.
\par 29 Kalau seorang laki-laki atau wanita menderita penyakit kulit di kepalanya atau di dagunya,
\par 30 ia harus diperiksa oleh imam. Kalau bagian yang sakit itu nampaknya lebih dalam dari kulit di sekitarnya, dan rambut atau jenggot di bagian itu menjadi kekuning-kuningan lalu rontok, maka orang itu menderita penyakit kulit yang berbahaya dan harus dinyatakan najis.
\par 31 Kalau menurut pemeriksaan imam bagian yang sakit tidak nampak lebih dalam dari kulit di sekitarnya, tetapi tidak terdapat rambut atau jenggot yang sehat di bagian itu, orang itu harus diasingkan selama tujuh hari.
\par 32 Pada hari yang ketujuh, imam harus memeriksa kembali bagian yang sakit, dan kalau penyakit itu tidak menyebar serta tidak terdapat rambut atau jenggot yang kekuning-kuningan di bagian itu, lagipula tampaknya tidak lebih dalam dari kulit di sekitarnya,
\par 33 orang itu harus mencukur rambut atau jenggot di sekeliling tempat yang sakit. Lalu ia harus diasingkan lagi selama tujuh hari.
\par 34 Pada hari yang ketujuh, imam harus memeriksanya kembali. Kalau penyakitnya tidak menyebar dan bagian itu nampaknya tidak lebih dalam dari kulit di sekitarnya, orang itu dinyatakan bersih dan harus mencuci pakaiannya.
\par 35 Tetapi kalau sesudah dinyatakan bersih penyakitnya menyebar,
\par 36 imam harus memeriksanya kembali. Kalau penyakit itu memang menyebar, tak perlu lagi imam memeriksa apakah rambut atau jenggot orang itu menjadi kuning, sebab sudah jelas bahwa ia najis.
\par 37 Tetapi kalau menurut pendapat imam penyakitnya tidak menyebar, dan tumbuh rambut atau jenggot yang sehat di tempat itu, maka penyakit itu sudah sembuh, dan imam harus menyatakan dia bersih.
\par 38 Apabila seorang laki-laki atau wanita mempunyai becak-becak putih pada kulitnya,
\par 39 imam harus memeriksa orang itu. Kalau becak-becak itu nampaknya putih kabur, ia hanya dijangkiti kurap, dan dinyatakan bersih.
\par 40 Kalau seorang laki-laki menjadi botak di bagian belakang atau bagian depan kepalanya, botak itu tidak menjadikan dia najis.
\par 42 Tetapi kalau timbul bengkak putih kemerah-merahan di tempat yang botak, ia menderita penyakit kulit yang berbahaya.
\par 43 Imam harus memeriksa orang itu, dan kalau memang ada bengkak putih kemerah-merahan,
\par 44 imam harus menyatakan dia najis karena menderita penyakit kulit yang berbahaya di kepalanya.
\par 45 Seseorang yang menderita penyakit kulit yang berbahaya, harus memakai pakaian yang koyak-koyak dan membiarkan rambutnya kusut. Ia harus menutupi bagian bawah mukanya dan berteriak-teriak, "Najis! Najis!"
\par 46 Selama menderita penyakit itu ia najis dan harus tinggal di luar perkemahan, jauh dari orang-orang lain.
\par 47 Kalau terdapat kelapukan pada pakaian atau kain dari wol atau linen,
\par 48 atau pada kulit atau apa saja yang terbuat dari kulit,
\par 49 dan warna kelapukan itu kehijau-hijauan atau kemerah-merahan, maka kelapukan itu jenis yang menyebar dan harus ditunjukkan kepada imam.
\par 50 Imam harus memeriksa benda itu, lalu mengasingkannya selama tujuh hari.
\par 51 Pada hari yang ketujuh ia harus memeriksanya kembali, dan kalau kelapukan itu memang sudah menyebar, benda itu najis.
\par 52 Imam harus membakar benda itu, karena kelapukan itu jenis yang menyebar, jadi harus dimusnahkan dengan api.
\par 53 Tetapi kalau menurut pemeriksaan imam kelapukan itu tidak menyebar pada benda itu,
\par 54 ia menyuruh supaya benda itu dicuci dan diasingkan selama tujuh hari lagi.
\par 55 Lalu imam harus memeriksanya kembali. Kalau kelapukan itu tidak menyebar tetapi warnanya tetap sama, benda itu masih najis dan harus dibakar, biarpun kelapukan itu terdapat hanya di bagian depan atau bagian belakangnya.
\par 56 Tetapi kalau menurut pemeriksaan imam kelapukan itu sudah menjadi pudar, bagian yang berlapuk dari kain atau kulit itu harus disobek.
\par 57 Kalau kemudian timbul kembali, maka kelapukan itu menyebar, dan pemiliknya harus membakar benda itu.
\par 58 Kalau sesudah dicuci warna kelapukan pada benda itu menghilang, maka benda itu harus dicuci lagi, barulah dianggap bersih.
\par 59 Begitulah peraturan-peraturan tentang kelapukan yang terdapat pada pakaian atau kain dari linen atau wol, atau pada apa saja yang terbuat dari kulit.

\chapter{14}

\par 1 TUHAN memberi kepada Musa
\par 2 peraturan-peraturan ini tentang upacara pembersihan sesudah sembuh dari penyakit kulit yang berbahaya. Pada hari orang itu akan dinyatakan bersih, ia harus dibawa kepada imam,
\par 3 dan imam membawanya ke luar perkemahan. Kalau menurut pemeriksaannya orang itu sudah sembuh,
\par 4 imam menyuruh dia membawa dua ekor burung hidup yang tidak haram, dengan sepotong kayu cemara, seutas tali merah dan setangkai hisop.
\par 5 Burung yang seekor harus disembelih di atas belanga tanah yang berisi air bersih dari mata air.
\par 6 Burung yang seekor lagi dengan kayu cemara, tali merah dan setangkai hisop harus dicelupkan ke dalam darah burung yang sudah disembelih.
\par 7 Imam memercikkan darah burung itu tujuh kali kepada orang yang sudah sembuh, dan menyatakan dia bersih. Lalu burung yang masih hidup harus dilepaskannya di padang.
\par 8 Kemudian orang itu harus mencuci pakaiannya, mencukur semua rambutnya, dan mandi. Maka bersihlah ia, dan boleh masuk ke dalam perkemahan, tetapi selama tujuh hari lagi ia belum boleh masuk ke dalam kemahnya sendiri.
\par 9 Pada hari yang ketujuh ia harus mencukur semua rambutnya, juga jenggotnya, alisnya dan semua bulu yang ada di badannya. Lalu ia harus mencuci pakaiannya dan mandi. Baru sesudah itu ia bersih.
\par 10 Pada hari yang kedelapan orang itu harus membawa dua ekor anak domba jantan dan seekor anak domba betina yang berumur satu tahun, yang masing-masing tak ada cacatnya. Selain itu juga tiga kilogram tepung yang sudah dicampur dengan minyak zaitun dan sepertiga liter minyak zaitun.
\par 11 Lalu imam harus membawa dia beserta persembahannya itu ke depan pintu Kemah TUHAN.
\par 12 Imam harus mengambil anak domba jantan yang seekor dan sepertiga liter minyak zaitun untuk kurban ganti rugi, dan mempersembahkannya sebagai persembahan unjukan bagi TUHAN.
\par 13 Anak domba itu harus disembelihnya di tempat yang khusus untuk memotong binatang bagi kurban pengampunan dosa dan kurban bakaran. Domba itu harus disembelih di situ karena kurban ganti rugi dan kurban pengampunan dosa adalah sangat suci dan menjadi bagian imam.
\par 14 Imam harus mengambil sedikit darah anak domba lalu mengoleskannya pada cuping telinga kanan, pada ibu jari tangan kanan dan pada ibu jari kaki kanan orang yang akan dinyatakan bersih.
\par 15 Kemudian imam mengambil sedikit minyak zaitun dan menuangkannya ke telapak tangannya sendiri yang kiri.
\par 16 Sesudahnya ia mencelupkan jari tangan kanannya ke dalam minyak, dan memercikkannya tujuh kali di dalam Kemah TUHAN.
\par 17 Lalu sedikit minyak yang ada di tangannya harus dioleskan pada tempat-tempat yang sudah diolesi darah, yaitu pada cuping telinga kanan, pada ibu jari tangan kanan, dan ibu jari kaki kanan orang yang akan dinyatakan bersih.
\par 18 Minyak selebihnya harus dituangkan ke atas kepala orang itu. Dengan demikian imam mengadakan upacara penyucian bagi orang itu di hadapan TUHAN.
\par 19 Kemudian imam harus mempersembahkan kurban pengampunan dosa supaya orang yang najis itu menjadi bersih. Sesudahnya ia harus menyembelih binatang untuk kurban bakaran
\par 20 dan mempersembahkannya bersama kurban sajian di atas mezbah. Dengan demikian imam mengadakan upacara penyucian bagi orang itu, dan ia menjadi bersih.
\par 21 Kalau orang itu miskin dan tidak mampu, maka untuk kurban ganti rugi ia boleh membawa seekor anak domba jantan saja untuk persembahan unjukan bagi TUHAN. Selain itu satu kilogram tepung dicampur dengan minyak zaitun untuk kurban sajian, dan sepertiga liter minyak zaitun.
\par 22 Ia harus juga membawa dua ekor burung merpati muda atau tekukur muda: seekor untuk kurban pengampunan dosa dan seekor lagi untuk kurban bakaran.
\par 23 Semua persembahan itu harus dibawanya kepada imam di depan pintu Kemah TUHAN pada hari kedelapan upacara penyucian dirinya.
\par 24 Imam mengambil anak domba dan minyak zaitun itu dan mempersembahkannya sebagai persembahan unjukan bagi TUHAN, dan itu untuk bagian imam.
\par 25 Lalu imam harus menyembelih anak domba itu dan mengoleskan sedikit darahnya pada cuping telinga kanan, ibu jari tangan kanan dan ibu jari kaki kanan orang yang akan dinyatakan bersih.
\par 26 Imam harus menuangkan sedikit minyak ke telapak tangannya yang kiri
\par 27 dan dengan jari tangan kanannya memercikkan minyak itu tujuh kali di dalam Kemah TUHAN.
\par 28 Lalu sedikit minyak yang ada di tangannya harus dioleskan pada tempat-tempat yang sudah diolesi darah, yaitu pada cuping telinga kanan, ibu jari tangan kanan dan ibu jari kaki kanan orang yang akan dinyatakan bersih.
\par 29 Minyak selebihnya harus dituangkan ke atas kepala orang itu untuk upacara penyucian dirinya di hadapan TUHAN.
\par 30 Kemudian imam harus mempersembahkan seekor burung merpati muda atau tekukur muda
\par 31 untuk kurban pengampunan dosa, dan seekor lainnya untuk kurban bakaran bersama kurban sajiannya. Dengan demikian imam mengadakan upacara penyucian orang itu di hadapan TUHAN.
\par 32 Itulah peraturan tentang upacara penyucian sehabis menderita penyakit kulit yang berbahaya bagi orang yang tidak mampu.
\par 33 TUHAN memberi kepada Musa dan Harun
\par 34 peraturan-peraturan tentang rumah yang kejangkitan kelapukan yang menyebar. Peraturan-peraturan ini berlaku sesudah bangsa Israel masuk ke Kanaan, negeri yang diberikan TUHAN kepada mereka.
\par 35 Barangsiapa menemukan kelapukan di dalam rumahnya, harus melaporkan hal itu kepada imam.
\par 36 Sebelum imam datang memeriksa kelapukan itu, ia harus memerintahkan supaya segala sesuatu dikeluarkan dari rumah itu. Kalau tidak, apa saja yang ada di dalam rumah itu menjadi najis. Lalu imam harus datang
\par 37 untuk memeriksa kelapukan itu. Kalau ada becak-becak kehijau-hijauan atau kemerah-merahan yang kelihatannya seperti meresap ke dalam tembok,
\par 38 imam harus meninggalkan rumah itu dan membiarkannya terkunci selama tujuh hari.
\par 39 Pada hari yang ketujuh ia harus datang dan memeriksanya lagi. Kalau kelapukan itu menyebar,
\par 40 ia harus memerintahkan supaya batu-batu tembok yang kena kelapukan itu dibongkar dan seluruh tembok bagian dalam dikikis. Kikisan plester dan batu-batu itu harus dibuang ke suatu tempat yang najis di luar kota.
\par 42 Lalu batu-batu yang dikeluarkan itu harus diganti dengan batu-batu yang baru. Seluruh tembok rumah itu harus dilapisi dengan plester baru.
\par 43 Kalau kelapukan itu timbul lagi, padahal batu-batu rumah sudah diganti dan tembok-temboknya sudah dikikis dan diplester kembali,
\par 44 maka imam harus datang dan memeriksa rumah itu lagi. Kalau kelapukan itu memang menyebar maka rumah itu najis
\par 45 dan harus dibongkar. Batu-batu, kayu-kayu dan seluruh plesternya harus dibawa ke tempat yang najis di luar kota.
\par 46 Sebelumnya siapa saja yang masuk ke dalam rumah itu menjadi najis sampai matahari terbenam.
\par 47 Barangsiapa tidur atau makan di rumah itu harus mencuci pakaiannya.
\par 48 Tetapi kalau imam datang dan menurut pemeriksaannya tidak timbul kelapukan lagi sesudah rumah itu diplester kembali, ia harus menyatakan rumah itu bersih, sebab kelapukan sudah hilang seluruhnya.
\par 49 Untuk penyucian rumah, imam harus mengambil dua ekor burung, kayu cemara, tali merah dan setangkai hisop.
\par 50 Burung yang seekor harus dipotong di atas belanga yang berisi air segar dari mata air.
\par 51 Lalu kayu cemara, tali merah, hisop dan burung yang masih hidup harus dicelupkan ke dalam darah burung yang sudah dipotong dan ke dalam air yang segar. Sesudahnya rumah itu harus dipercikinya tujuh kali.
\par 52 Burung yang masih hidup harus dilepaskannya di padang di luar kota. Dengan demikian upacara penyucian rumah, dan rumah itu menjadi bersih.
\par 54 Itulah peraturan-peraturan tentang penyakit kulit yang berbahaya,
\par 55 tentang bintil-bintil, borok atau becak-becak putih atau bengkak pada badan, dan tentang kelapukan pada pakaian atau rumah.
\par 57 Peraturan-peraturan itu gunanya untuk menentukan apakah sesuatu itu bersih atau najis.

\chapter{15}

\par 1 TUHAN memberi kepada Musa dan Harun peraturan-peraturan ini
\par 2 untuk bangsa Israel. Apabila seorang laki-laki mengeluarkan lelehan karena sakit kelamin, lelehan itu najis,
\par 3 baik itu yang keluar maupun yang tertahan.
\par 4 Tempat yang diduduki atau ditiduri orang itu menjadi najis.
\par 5 Barangsiapa menyentuh orang yang mengeluarkan lelehan atau tempat tidurnya, atau duduk di tempat yang bekas didudukinya atau diludahi orang itu, harus mencuci pakaiannya dan mandi, dan ia najis sampai matahari terbenam.
\par 9 Pelana atau apa saja yang diduduki orang yang mengeluarkan lelehan menjadi najis.
\par 10 Barangsiapa menyentuh sesuatu yang bekas diduduki orang itu, menjadi najis sampai matahari terbenam. Setiap orang yang mengangkat barang itu harus mencuci pakaiannya dan mandi, dan ia najis sampai matahari terbenam.
\par 11 Apabila orang yang mengeluarkan lelehan tanpa mencuci tangannya lebih dahulu menyentuh orang lain, maka orang yang disentuhnya itu harus mencuci pakaiannya dan mandi, dan ia najis sampai matahari terbenam.
\par 12 Belanga tanah yang disentuh orang itu harus dipecahkan dan perkakas yang dipegangnya harus dicuci.
\par 13 Apabila orang itu sembuh, ia harus menunggu tujuh hari lagi. Sesudah itu ia harus mencuci pakaiannya dan mandi di mata air, maka bersihlah ia.
\par 14 Pada hari yang kedelapan ia harus membawa dua ekor burung merpati muda atau burung tekukur muda ke depan pintu Kemah TUHAN lalu menyerahkannya kepada imam.
\par 15 Imam harus mempersembahkan burung yang seekor untuk kurban pengampunan dosa, dan yang seekor lagi untuk kurban bakaran. Dengan demikian imam mengadakan upacara penyucian orang itu di hadapan TUHAN.
\par 16 Apabila seorang laki-laki mengeluarkan mani, ia harus mandi tetapi tetap najis sampai matahari terbenam.
\par 17 Benda apa saja dari kain atau kulit yang kena mani itu harus dicuci; benda itu najis sampai matahari terbenam.
\par 18 Sesudah bersetubuh, baik laki-laki maupun wanita harus mandi, tetapi mereka tetap najis sampai matahari terbenam.
\par 19 Seorang wanita yang sedang haid, najis selama tujuh hari. Barangsiapa menyentuh dia menjadi najis sampai matahari terbenam.
\par 20 Apa saja yang diduduki atau ditiduri wanita selama masa haidnya menjadi najis.
\par 21 Barangsiapa menyentuh tempat yang bekas ditiduri atau diduduki wanita yang sedang haid, harus mencuci pakaiannya dan mandi, dan ia najis sampai matahari terbenam.
\par 24 Apabila seorang laki-laki bersetubuh dengan wanita yang sedang haid, laki-laki itu juga menjadi najis selama tujuh hari, dan setiap tempat tidur yang ditidurinya juga menjadi najis.
\par 25 Apabila seorang wanita mengalami pendarahan selama beberapa hari di luar masa haidnya, atau pendarahannya tidak berhenti sesudah masa haidnya, ia najis selama pendarahan itu seperti pada waktu sedang haid.
\par 26 Tempat yang ditiduri atau didudukinya selama waktu itu menjadi najis.
\par 27 Barangsiapa menyentuh tempat itu juga menjadi najis. Ia harus mencuci pakaiannya dan mandi, dan ia najis sampai matahari terbenam.
\par 28 Sesudah pendarahan itu berhenti, wanita itu harus menunggu selama tujuh hari lagi. Baru sesudah itu ia bersih.
\par 29 Pada hari yang kedelapan ia harus membawa dua ekor burung merpati muda atau tekukur muda kepada imam di depan pintu Kemah TUHAN.
\par 30 Burung yang seekor harus dipersembahkan untuk kurban pengampunan dosa, dan yang seekor lagi untuk kurban bakaran. Dengan cara itu imam mengadakan upacara penyucian wanita itu di hadapan TUHAN.
\par 31 TUHAN menyuruh Musa mengingatkan bangsa Israel supaya memperhatikan kapan mereka najis dan kapan tidak, supaya mereka tidak menajiskan Kemah TUHAN yang ada di tengah-tengah perkemahan mereka. Kalau mereka menajiskan Kemah itu, mereka akan mati.
\par 32 Begitulah peraturan-peraturan tentang orang laki-laki yang mengeluarkan lelehan karena sakit kelamin atau mengeluarkan mani,
\par 33 tentang wanita yang sedang haid, dan tentang laki-laki yang bersetubuh dengan wanita yang sedang haid.

\chapter{16}

\par 1 Sesudah kedua anak Harun mati pada waktu mempersembahkan api yang tidak dikehendaki TUHAN, TUHAN berbicara kepada Musa,
\par 2 kata-Nya, "Sampaikanlah kepada saudaramu Harun, bahwa hanya pada waktu yang ditentukan ia boleh memasuki Ruang Mahasuci yang dipisahkan dengan tirai, karena di situlah Aku menampakkan diri-Ku dalam awan di atas tutup Peti Perjanjian. Kalau Harun melanggar perintah itu, ia akan mati.
\par 3 Hanya dengan cara mempersembahkan sapi jantan muda untuk kurban pengampunan dosa dan seekor domba jantan untuk kurban bakaran, Harun boleh memasuki Ruang Mahasuci."
\par 4 Selanjutnya Harun harus memperhatikan peraturan-peraturan ini. Sebelum memasuki Ruang Mahasuci ia harus mandi lalu memakai pakaian imam, yaitu baju, celana pendek, ikat pinggang dan serban, semuanya dari kain linen.
\par 5 Lalu umat Israel harus menyerahkan kepada Harun dua ekor kambing jantan untuk kurban pengampunan dosa dan seekor domba jantan untuk kurban bakaran.
\par 6 Harun harus mempersembahkan seekor sapi jantan untuk kurban pengampunan dosa bagi dirinya sendiri dan keluarganya.
\par 7 Kedua ekor kambing jantan dari umat Israel harus dibawanya ke depan pintu Kemah TUHAN.
\par 8 Di situ ia harus membuang undi dengan menggunakan dua batu, yang satu ditandai "untuk TUHAN", dan yang lain "untuk Azazel".
\par 9 Kambing yang terpilih bagi TUHAN harus dipersembahkan untuk kurban pengampunan dosa.
\par 10 Kambing yang terpilih bagi Azazel harus ditempatkan hidup-hidup di hadapan TUHAN lalu diusir ke padang gurun bagi Azazel, supaya dosa-dosa bangsa Israel dihapuskan.
\par 11 Setelah Harun mempersembahkan dan menyembelih sapi jantan untuk kurban pengampunan dosa bagi dirinya sendiri dan bagi keluarganya,
\par 12 ia harus mengambil tempat api dari mezbah dan mengisinya penuh dengan bara api dan dua genggam dupa halus, lalu membawanya ke Ruang Mahasuci.
\par 13 Dupa itu harus dibakarnya di depan Peti Perjanjian, sehingga asapnya menyelubungi tutup Peti itu dan Harun tidak dapat melihatnya, sebab kalau ia melihatnya, ia akan mati.
\par 14 Lalu Harun harus mengambil sedikit darah sapi, dan dengan jarinya memercikkan darah itu ke bagian muka tutup Peti Perjanjian dan tujuh kali ke depan Peti.
\par 15 Sesudah itu Harun harus menyembelih kambing untuk kurban pengampunan dosa umat Israel. Darah kambing itu harus dibawanya ke Ruang Mahasuci, lalu dipercikkan ke bagian depan tutup Peti Perjanjian dan ke depan Peti, seperti yang sudah dilakukannya dengan darah sapi jantan.
\par 16 Dengan cara itu ia menyucikan Ruang Mahasuci dari kenajisan bangsa Israel dan dari segala dosa mereka. Kemah TUHAN harus disucikan karena berada di tengah-tengah perkemahan bangsa Israel dan kena kenajisan mereka.
\par 17 Mulai dari saat Harun memasuki Ruang Mahasuci untuk melakukan upacara penyucian sampai ia keluar, seorang pun tak boleh berada di dalam Kemah itu. Sesudah selesai melakukan upacara untuk dirinya sendiri, untuk keluarganya dan untuk seluruh bangsa Israel,
\par 18 Harun harus keluar dan pergi ke mezbah kurban bakaran untuk menyucikan mezbah itu. Ia harus mengambil sedikit darah sapi dan darah kambing, dan mengoleskannya pada tanduk-tanduk di sudut-sudut mezbah.
\par 19 Dengan jarinya ia harus memercikkan sedikit darah itu tujuh kali ke atas mezbah. Dengan cara itu ia menyucikan mezbah dari kenajisan bangsa Israel, supaya menjadi suci.
\par 20 Sesudah mengadakan upacara penyucian Ruang Mahasuci dan bagian-bagian lain dari Kemah TUHAN serta mezbahnya, Harun harus mempersembahkan kepada TUHAN kambing yang hidup yang dipilih bagi Azazel.
\par 21 Ia harus meletakkan kedua tangannya di atas kepala kambing itu sambil mengakui semua kesalahan, dosa dan pelanggaran bangsa Israel. Dengan demikian semua kesalahan bangsa Israel dipindahkan ke atas kepala kambing itu. Lalu seorang yang ditugaskan harus mengusir kambing itu ke padang gurun.
\par 22 Kambing itu membawa semua dosa bangsa Israel ke daerah tandus yang tidak didiami dan ia harus dilepaskan di padang gurun.
\par 23 Sesudah itu Harun harus masuk ke dalam Kemah, membuka pakaian imam yang dipakainya waktu masuk ke Ruang Mahasuci, dan meninggalkan pakaian itu di sana.
\par 24 Lalu ia harus mandi di suatu tempat yang dikhususkan dan memakai pakaiannya sendiri. Sesudahnya ia harus keluar dan mempersembahkan kurban bakaran untuk pengampunan dosanya sendiri dan dosa bangsa Israel.
\par 25 Lemak binatang untuk kurban pengampunan dosa harus dibakar di mezbah.
\par 26 Orang yang melepaskan kambing bagi Azazel di padang gurun harus mencuci pakaiannya dan mandi sebelum pulang ke perkemahan.
\par 27 Sapi jantan dan kambing untuk kurban pengampunan dosa, yang darahnya dibawa masuk ke Ruang Mahasuci untuk pengampunan dosa, harus dibawa ke luar perkemahan dan dibakar. Kulit, daging dan isi perut kedua binatang harus dibakar semuanya.
\par 28 Orang yang membakarnya harus mencuci pakaiannya dan mandi sebelum pulang ke perkemahan.
\par 29 Peraturan-peraturan ini harus ditaati untuk selama-lamanya. Pada tanggal sepuluh bulan tujuh, orang Israel dan orang asing yang menetap di antara mereka harus berpuasa dan dilarang bekerja, karena hari ini hari yang sangat suci. Pada hari itu harus dilakukan upacara untuk menyucikan bangsa Israel dari segala dosa mereka supaya mereka bersih.
\par 32 Imam Agung, yang sudah ditahbiskan dan dikhususkan bagi Allah untuk menggantikan ayahnya, harus melakukan upacara penyucian itu. Ia harus memakai pakaian khusus,
\par 33 lalu melakukan upacara penyucian Ruang Mahasuci, bagian-bagian lain dari Kemah TUHAN, mezbah, imam-imam dan seluruh bangsa Israel.
\par 34 Upacara itu harus dilakukan sekali setahun untuk menyucikan bangsa Israel dari segala dosa mereka. Itulah peraturan-peraturan yang harus ditaati untuk selama-lamanya. Musa melakukan seperti yang diperintahkan TUHAN kepadanya.

\chapter{17}

\par 1 TUHAN menyuruh Musa
\par 2 memberikan peraturan-peraturan ini kepada Harun, anak-anaknya dan seluruh bangsa Israel.
\par 3 Apabila seorang bangsa Israel mempersembahkan sapi, domba atau kambing kepada TUHAN, binatang itu harus disembelihnya di depan pintu Kemah TUHAN. Kalau ia menyembelihnya di tempat lain, ia melanggar hukum karena menumpahkan darah. Ia tidak lagi dianggap anggota umat Allah.
\par 5 Perintah itu dimaksudkan supaya bangsa Israel membawa ke Kemah TUHAN semua binatang persembahan yang biasanya mereka sembelih di ladang. Mereka wajib membawa persembahan itu kepada imam di depan pintu Kemah TUHAN dan menyembelihnya di situ untuk kurban perdamaian.
\par 6 Imam harus menyiramkan darah binatang itu pada sisi-sisi mezbah di depan pintu Kemah TUHAN lalu membakar lemaknya supaya baunya menyenangkan hati TUHAN.
\par 7 Bangsa Israel harus tetap setia kepada TUHAN dan sekali-kali tak boleh menyembelih binatang di ladang-ladang untuk persembahan kepada jin-jin jahat. Peraturan itu harus ditaati untuk selama-lamanya.
\par 8 Apabila seorang Israel atau seorang asing yang tinggal menetap di antara orang Israel mempersembahkan kurban bakaran atau kurban lain
\par 9 kepada TUHAN, ia harus mempersembahkannya di depan pintu Kemah TUHAN. Kalau ia mempersembahkannya di tempat lain, ia tak boleh lagi dianggap anggota umat Allah.
\par 10 Kalau seorang Israel atau seorang asing yang tinggal menetap di antara orang Israel makan darah, ia akan dihukum TUHAN dan tidak lagi dianggap anggota umat-Nya.
\par 11 Nyawa setiap makhluk ada di dalam darahnya. Karena itu TUHAN memerintahkan supaya semua darah disiramkan ke atas mezbah supaya dosa-dosa umat dihapuskan. Darah yaitu nyawa, menghapuskan dosa.
\par 12 Itulah sebabnya bangsa Israel atau orang asing yang menetap di antara mereka dilarang makan darah.
\par 13 Apabila seorang Israel atau seorang asing yang tinggal menetap di antara bangsa Israel menangkap burung atau binatang lain yang tidak haram, ia harus mengeluarkan darah binatang itu dan menimbuninya dengan tanah.
\par 14 Nyawa setiap makhluk ada di dalam darahnya, sebab itu TUHAN melarang bangsa Israel makan darah. Setiap orang yang melakukan perbuatan terlarang itu tidak lagi dianggap anggota umat Allah.
\par 15 Setiap orang Israel atau orang asing yang makan daging binatang yang mati dengan sendirinya atau mati diterkam binatang buas, harus mencuci pakaiannya lalu mandi dan menunggu sampai matahari terbenam, barulah ia menjadi bersih.
\par 16 Kalau ia tidak melakukan itu, ia harus menanggung akibatnya.

\chapter{18}

\par 1 TUHAN menyuruh Musa
\par 2 menyampaikan kepada bangsa Israel perintah ini, "Akulah TUHAN Allahmu.
\par 3 Janganlah meniru perbuatan orang di Mesir, tempat kamu pernah tinggal, atau orang di Kanaan, ke mana Aku sekarang membawa kamu.
\par 4 Taatilah hukum-hukum-Ku dan lakukanlah apa yang Kuperintahkan. Akulah TUHAN Allahmu.
\par 5 Taatilah peraturan-peraturan dan hukum-hukum yang Kuberikan kepadamu. Kalau kamu berbuat begitu, hidupmu selamat. Akulah TUHAN."
\par 6 TUHAN memberikan peraturan-peraturan ini. Jangan bersetubuh dengan salah seorang dari sanak saudaramu.
\par 7 Jangan bersetubuh dengan ibumu, istri ayahmu, sebab dia hak ayahmu.
\par 8 Jangan menghina ayahmu dengan bersetubuh dengan salah seorang istrinya yang lain.
\par 9 Jangan bersetubuh dengan saudaramu perempuan atau saudara tirimu, baik yang dibesarkan serumah dengan engkau maupun yang dibesarkan di rumah lain.
\par 10 Jangan bersetubuh dengan cucumu; itu akan membuat malu dirimu sendiri.
\par 11 Jangan bersetubuh dengan saudaramu perempuan yang seayah lain ibu atau seibu lain ayah, karena dia saudaramu juga.
\par 12 Jangan bersetubuh dengan bibimu, baik ia saudara ayahmu atau saudara ibumu.
\par 14 Jangan bersetubuh dengan istri pamanmu, karena dia bibimu juga.
\par 15 Jangan bersetubuh dengan anak menantumu
\par 16 atau dengan istri abangmu.
\par 17 Jangan bersetubuh dengan anak atau cucu seorang wanita yang pernah kaugauli. Mungkin mereka kerabatmu, dan itu perbuatan yang tidak senonoh.
\par 18 Jangan kawin dengan saudara istrimu selama istrimu sendiri masih hidup.
\par 19 Jangan bersetubuh dengan seorang wanita selama masa haidnya, karena ia dalam keadaan najis.
\par 20 Jangan bersetubuh dengan istri orang lain; perbuatan itu menjadikan engkau najis.
\par 21 Jangan menyerahkan salah seorang anakmu untuk dikurbankan dalam pemujaan Molokh, karena perbuatan itu menghina nama Allah, Tuhanmu.
\par 22 Orang laki-laki tak boleh bersetubuh dengan orang laki-laki; Allah membenci perbuatan itu.
\par 23 Laki-laki maupun wanita, sekali-kali tak boleh bersetubuh dengan binatang; perbuatan jahat itu menajiskan kamu.
\par 24 Jangan menajiskan dirimu dengan perbuatan-perbuatan itu, sebab dengan cara itu bangsa-bangsa yang tinggal di Kanaan sebelum kamu telah menajiskan diri mereka. Karena itu mereka diusir oleh TUHAN, supaya kamu dapat mendiami negeri itu.
\par 25 Mereka telah melakukan semua perbuatan yang menjijikkan itu, sehingga tanah mereka menjadi najis. Maka TUHAN menghukum mereka dan tidak mengizinkan mereka tinggal di sana. Tetapi kamu jangan sekali-kali melakukan perbuatan-perbuatan itu. Kamu semua, orang Israel dan orang asing yang menetap bersama kamu, harus mentaati peraturan dan perintah TUHAN,
\par 28 supaya kamu jangan dihukum oleh-Nya dan diusir dari tanah itu seperti orang-orang yang tidak mengenal TUHAN, yang tinggal di sana sebelum kamu.
\par 29 Kamu tahu bahwa barangsiapa melakukan perbuatan yang menjijikkan itu tidak lagi dianggap anggota umat Allah.
\par 30 Maka TUHAN berkata, "Taatilah peraturan-peraturan yang Kuberikan dan janganlah meniru perbuatan orang-orang yang mendiami tanah Kanaan sebelum kamu. Jangan menajiskan dirimu sendiri dengan melakukan perbuatan-perbuatan itu. Aku TUHAN Allahmu."

\chapter{19}

\par 1 TUHAN menyuruh Musa
\par 2 menyampaikan kepada bangsa Israel perintah ini, "Hendaklah kamu suci, karena Aku TUHAN Allahmu adalah suci.
\par 3 Setiap orang harus menghormati ayah ibunya dan merayakan hari Sabat seperti yang Kuperintahkan. Akulah TUHAN Allahmu.
\par 4 Jangan menyembah berhala. Jangan membuat patung dari logam dan menyembahnya. Akulah TUHAN Allahmu.
\par 5 Kalau kamu mempersembahkan binatang untuk kurban perdamaian, ikutilah peraturan-peraturan yang Kuberikan, supaya kurban itu Kuterima.
\par 6 Dagingnya harus dimakan pada hari binatang itu disembelih atau keesokan harinya. Daging yang masih sisa pada hari yang ketiga harus dibakar habis,
\par 7 karena daging itu haram. Kalau daging itu dimakan juga, kurban itu tidak Kuterima.
\par 8 Barangsiapa makan daging itu bersalah karena tidak menghargai apa yang dikhususkan untuk Aku. Orang itu tidak lagi Kuanggap anggota umat-Ku.
\par 9 Kalau kamu panen, janganlah memotong gandum yang tumbuh di pinggir-pinggir ladangmu, dan jangan kembali untuk mengumpulkan gandum yang tersisa sesudah panen.
\par 10 Jangan kembali ke kebun anggurmu untuk mengumpulkan buah-buah anggur yang tertinggal sesudah kamu memetiknya pertama kali. Juga buah-buahnya yang sudah jatuh jangan kamu ambil. Biarkan itu untuk orang miskin dan orang asing. Akulah TUHAN Allahmu.
\par 11 Jangan mencuri, menipu atau berdusta.
\par 12 Jangan berjanji demi nama-Ku kalau kamu tidak bermaksud menepati janji itu, sebab dengan itu kamu menghina nama-Ku. Akulah TUHAN Allahmu.
\par 13 Jangan memeras sesamamu atau merampas barangnya. Upah seseorang yang bekerja padamu jangan kamu tahan, biar untuk satu malam saja.
\par 14 Jangan mengutuk orang tuli dan jangan menaruh batu sandungan di depan orang buta. Hendaklah kamu hormat dan takut kepada-Ku, sebab Aku TUHAN Allahmu.
\par 15 Kamu harus jujur bila mengadili; jangan berpihak kepada orang miskin dan jangan takut kepada orang kaya.
\par 16 Jangan menyebarkan fitnah di antara orang-orang sebangsamu, supaya kamu tidak menyebabkan kematian sesamamu manusia. Akulah TUHAN.
\par 17 Jangan dendam terhadap siapa pun. Bereskanlah perselisihanmu dengan siapa saja, supaya kamu tidak berdosa karena orang lain.
\par 18 Jangan membalas dendam dan jangan membenci orang lain, tetapi cintailah sesamamu seperti kamu mencintai dirimu sendiri. Akulah TUHAN.
\par 19 Taatilah perintah-perintah-Ku. Jangan mengawinkan ternak yang berlainan jenisnya. Jangan menanam dua macam bibit pada sebidang tanah. Jangan memakai pakaian yang ditenun dari dua macam benang.
\par 20 Apabila seorang laki-laki berjanji akan menjual budaknya yang perempuan kepada laki-laki lain untuk dijadikan selirnya, dan orang yang mau membeli budak itu belum membayarnya, lalu pemilik yang semula bersetubuh dengan budak itu, mereka berdua harus dihukum, tetapi tidak boleh dihukum mati, sebab bagaimanapun juga perempuan itu masih budaknya.
\par 21 Laki-laki itu harus membawa seekor domba jantan ke depan pintu Kemah-Ku untuk kurban ganti rugi.
\par 22 Imam harus mengurbankan domba itu dalam upacara penyucian, supaya dosa orang itu diampuni. Maka Aku, Allah akan mengampuninya.
\par 23 Apabila kamu sudah masuk ke negeri Kanaan dan menanam bermacam-macam pohon buah-buahan di sana, maka selama tiga tahun yang pertama buah-buahnya harus kamu anggap haram dan tak boleh kamu makan.
\par 24 Dalam tahun keempat semua buahnya harus dipersembahkan untuk menyatakan terima kasihmu kepada Aku, TUHAN.
\par 25 Dalam tahun yang kelima barulah kamu boleh makan buahnya. Kalau kamu melakukan semua peraturan itu, pohon buah-buahanmu akan berbuah lebih banyak lagi. Akulah TUHAN Allahmu.
\par 26 Jangan makan darah. Jangan memakai ilmu-ilmu gaib.
\par 27 Jangan menggunting rambut di sisi kepalamu atau mencukur jenggotmu.
\par 28 Jangan mencacah kulitmu untuk membuat tanda-tanda atau melukai badanmu sebagai tanda berkabung atas orang mati. Akulah TUHAN.
\par 29 Jangan merendahkan derajat anak-anakmu yang perempuan dengan menjadikan mereka pelacur di kuil. Dengan berbuat begitu kamu mencemarkan negerimu, dan perbuatan-perbuatan tak senonoh akan merajalela di situ.
\par 30 Rayakanlah hari Sabat dan hormatilah tempat yang dikhususkan untuk menyembah Aku. Akulah TUHAN.
\par 31 Jangan pergi minta nasihat kepada dukun yang mengadakan hubungan dengan roh-roh orang mati. Kalau kamu melakukan itu kamu menjadi najis. Akulah TUHAN Allahmu.
\par 32 Seganilah dan hormatilah orang-orang tua. Hendaklah kamu hormat dan takut kepada-Ku, sebab Akulah TUHAN.
\par 33 Jangan berbuat tidak baik kepada orang asing yang tinggal di negerimu.
\par 34 Perlakukanlah mereka seperti kamu memperlakukan orang-orang sebangsamu dan cintailah mereka seperti kamu mencintai dirimu sendiri. Ingatlah bahwa kamu pun pernah hidup sebagai orang asing di Mesir. Akulah TUHAN Allahmu.
\par 35 Jangan menipu orang lain dengan memakai ukuran, timbangan atau takaran palsu.
\par 36 Pakailah meteran, timbangan dan takaran yang betul. Akulah TUHAN Allahmu, dan Aku telah membawa kamu keluar dari Mesir.
\par 37 Taatilah segala hukum dan perintah-Ku. Akulah TUHAN."

\chapter{20}

\par 1 TUHAN menyuruh Musa
\par 2 mengatakan hal ini kepada bangsa Israel, "Setiap orang di antara kamu atau orang asing yang menetap di antara kamu yang mengurbankan anaknya kepada Molokh, harus dilempari dengan batu sampai mati oleh seluruh masyarakat.
\par 3 Orang yang mengurbankan anaknya kepada Molokh, menajiskan Kemah-Ku dan menghina nama-Ku yang suci. Dia akan Kuhukum dan tidak lagi Kuanggap anggota umat-Ku.
\par 4 Kalau masyarakat menutup mata terhadap perbuatan orang itu dan tidak melaksanakan hukum mati atas dia,
\par 5 maka Aku sendiri yang menghukum dia, seluruh keluarganya, serta semua orang yang bersama dia meninggalkan Aku dan menyembah Molokh. Mereka tidak lagi Kuanggap anggota umat-Ku.
\par 6 Orang yang pergi minta nasihat kepada dukun yang mengadakan hubungan dengan roh-roh orang mati akan Kuhukum dan tidak Kuanggap lagi anggota umat-Ku.
\par 7 Jagalah supaya kamu tetap suci, karena Aku TUHAN Allahmu.
\par 8 Taatilah hukum-hukum-Ku, karena Akulah TUHAN yang mengkhususkan kamu untuk Aku."
\par 9 TUHAN memberikan peraturan-peraturan ini, "Orang yang mengutuk ayahnya atau ibunya harus dihukum mati. Ia mati karena salahnya sendiri.
\par 10 Apabila seorang Israel berzinah dengan istri orang sebangsanya, ia dan wanita itu harus dihukum mati.
\par 11 Seorang laki-laki yang bersetubuh dengan salah seorang istri ayahnya, memperkosa hak ayahnya. Laki-laki dan wanita itu harus dihukum mati. Mereka mati karena salah mereka sendiri.
\par 12 Apabila seorang laki-laki bersetubuh dengan menantunya perempuan, kedua orang itu harus dihukum mati. Mereka telah berzinah, dan mati mereka karena salah mereka sendiri.
\par 13 Apabila seorang laki-laki bersetubuh dengan laki-laki lain, mereka melakukan perbuatan yang keji dan hina, dan kedua-duanya harus dihukum mati. Mereka mati karena salah mereka sendiri.
\par 14 Apabila seorang laki-laki kawin dengan seorang wanita serta ibu wanita itu, ketiga-tiganya harus dibakar sampai mati, karena mereka melakukan perbuatan yang tidak senonoh. Perbuatan semacam itu tak boleh ada di antara kamu.
\par 15 Apabila seorang laki-laki bersetubuh dengan binatang, orang dan binatang itu harus dibunuh.
\par 16 Apabila seorang wanita bersetubuh dengan binatang, wanita dan binatang itu harus dibunuh. Mereka mati karena salah mereka sendiri.
\par 17 Apabila seorang laki-laki bersetubuh dengan saudaranya perempuan atau dengan saudaranya yang seayah lain ibu atau seibu lain ayah, mereka tidak lagi dianggap anggota umat-Ku. Mereka telah melakukan perbuatan yang tidak senonoh dan harus menanggung akibatnya.
\par 18 Apabila seorang laki-laki bersetubuh dengan seorang wanita yang sedang haid, mereka tidak lagi dianggap anggota umat-Ku.
\par 19 Apabila seorang laki-laki bersetubuh dengan bibinya, kedua-duanya harus menanggung akibat dari pelanggaran itu.
\par 20 Apabila seorang laki-laki bersetubuh dengan istri pamannya, ia menghina pamannya. Laki-laki dan wanita itu harus menanggung hukumannya. Sampai mati mereka tidak mendapat anak.
\par 21 Apabila seorang laki-laki merampas istri saudaranya, ia menghina saudaranya dan melakukan perbuatan yang tidak senonoh. Sampai mati mereka tidak mendapat anak."
\par 22 TUHAN berkata, "Taatilah semua hukum dan perintah-Ku, supaya kamu tidak diusir dari negeri Kanaan, ke mana Aku membawa kamu.
\par 23 Jangan meniru adat istiadat bangsa yang tinggal di sana. Mereka memuakkan Aku dengan segala perbuatan yang jahat. Itu sebabnya mereka Kuusir, supaya kamu dapat masuk ke negeri itu.
\par 24 Tanah yang kaya dan subur itu sudah Kujanjikan kepada kamu, jadi Kuberikan kepadamu menjadi milikmu. Akulah TUHAN Allahmu yang memisahkan kamu dari bangsa-bangsa lain.
\par 25 Oleh karena itu kamu harus membedakan dengan tepat binatang yang haram dari yang tidak haram. Jangan makan binatang yang haram. Binatang itu Kunyatakan haram, jadi kalau memakannya kamu menjadi najis.
\par 26 Kamu Kupisahkan dari bangsa-bangsa lain supaya menjadi milik-Ku. Kamu harus suci karena Akulah TUHAN, dan Aku suci.
\par 27 Setiap orang laki-laki atau wanita yang mengadakan hubungan dengan roh-roh orang mati harus dilempari batu sampai mati. Orang itu mati karena salahnya sendiri."

\chapter{21}

\par 1 TUHAN menyuruh Musa menyampaikan perintah ini kepada imam-imam keturunan Harun, "Seorang imam tak boleh menajiskan dirinya karena ikut menghadiri upacara penguburan sanak saudaranya,
\par 2 kecuali kalau yang mati itu ibunya, ayahnya, anaknya, saudaranya laki-laki,
\par 3 atau saudaranya perempuan yang belum kawin, dan tinggal serumah dengan dia.
\par 4 Ia tak boleh menajiskan dirinya dengan mengikuti upacara penguburan sanak saudara istrinya.
\par 5 Seorang imam tak boleh menggunduli sebagian dari kepalanya dan tak boleh memangkas jenggotnya atau melukai badannya dengan benda tajam untuk menunjukkan bahwa ia sedang berkabung.
\par 6 Ia milik-Ku yang khusus dan tak boleh mencemarkan nama-Ku. Ia mempersembahkan kurban makanan kepada-Ku, jadi ia harus suci.
\par 7 Seorang imam tak boleh kawin dengan seorang wanita bekas pelacur atau seorang wanita yang bukan perawan atau yang sudah bercerai, karena imam adalah milik-Ku.
\par 8 Bangsa Israel harus mengakui imam sebagai orang yang dikhususkan bagi-Ku karena ia mempersembahkan kurban makanan kepada-Ku. Akulah TUHAN; Aku ini suci dan Aku menyucikan umat-Ku.
\par 9 Kalau anak perempuan seorang imam menjadi pelacur, ia menghina ayahnya dan harus dibakar sampai mati.
\par 10 Imam Agung telah ditahbiskan dengan cara menuangkan minyak upacara ke atas kepalanya dan dengan mengenakan pakaian imam agung kepadanya. Karena itu ia tak boleh membiarkan rambutnya kusut atau mengoyak pakaiannya sebagai tanda berkabung.
\par 11 Ia telah dipersembahkan kepada-Ku untuk menjadi milik-Ku, jadi tak boleh menajiskan dirinya. Ia tak boleh juga menajiskan Kemah-Ku dengan keluar dari situ untuk masuk ke rumah yang ada jenazahnya, walaupun itu jenazah ayahnya atau ibunya sendiri.
\par 13 Ia hanya boleh kawin dengan seorang perawan dari sukunya sendiri. Ia tak boleh kawin dengan seorang janda atau seorang wanita yang sudah diceraikan, atau seorang wanita bekas pelacur.
\par 15 Kalau ia melanggar peraturan itu, maka anak-anaknya yang seharusnya suci, akan najis. Akulah TUHAN dan Aku telah mengkhususkan dia menjadi Imam Agung."
\par 16 TUHAN memerintahkan Musa
\par 17 untuk mengatakan ini kepada Harun, "Seorang keturunanmu yang cacat badannya tak boleh mempersembahkan kurban makanan kepada-Ku. Itu berlaku untuk selama-lamanya.
\par 18 Orang yang mempersembahkan kurban tak boleh cacat badannya, yaitu orang buta atau lumpuh, orang yang cacat mukanya atau anggota badannya,
\par 19 yang lumpuh tangannya atau timpang kakinya,
\par 20 yang bongkok atau cebol, yang berpenyakit mata atau berpenyakit kulit dan yang dikebiri.
\par 21 Setiap keturunan Imam Harun yang cacat badannya tak boleh mempersembahkan kurban makanan kepada-Ku.
\par 22 Tetapi ia boleh makan makanan yang dipersembahkan, baik kurban makanan yang suci maupun kurban makanan yang sangat suci.
\par 23 Karena orang itu cacat badannya, ia tak boleh mendekati tirai yang memisahkan Ruang Suci atau mezbah. Ia tak boleh menajiskan benda-benda yang sudah dikhususkan untuk Aku, karena Akulah TUHAN, dan Akulah yang mengkhususkan barang-barang itu."
\par 24 Itulah peraturan-peraturan yang disampaikan Musa kepada Harun serta anak-anaknya, dan kepada seluruh bangsa Israel.

\chapter{22}

\par 1 TUHAN menyuruh Musa
\par 2 menyampaikan perintah ini kepada Harun dan anak-anaknya, "Perlakukanlah dengan pantas apa yang dipersembahkan bangsa Israel khusus untuk Aku, supaya kamu tidak mencemarkan nama-Ku yang suci. Akulah TUHAN.
\par 3 Kalau seorang dari keturunanmu dalam keadaan najis mendekati persembahan-persembahan suci yang dikhususkan bangsa Israel bagi-Ku, ia tidak boleh lagi melayani pada mezbah. Peraturan itu berlaku untuk selama-lamanya. Akulah TUHAN.
\par 4 Seorang keturunan Harun yang menderita penyakit kulit yang berbahaya, atau yang sakit kelamin sehingga mengeluarkan lelehan, dilarang makan persembahan suci kalau ia belum dinyatakan bersih. Seorang imam menjadi najis kalau ia menyentuh sesuatu yang najis karena kena mayat. Ia juga najis kalau maninya keluar,
\par 5 atau menyentuh binatang atau orang yang najis.
\par 6 Seorang imam yang dengan cara apa pun menjadi najis, tetap najis sampai matahari terbenam. Dan sesudah matahari terbenam ia belum boleh makan makanan yang dipersembahkan kepada TUHAN kalau ia belum mandi.
\par 7 Sesudah matahari terbenam ia bersih dan boleh memakannya, karena itulah makanannya.
\par 8 Ia dilarang makan daging binatang yang mati dengan sendirinya atau mati diterkam binatang buas. Kalau ia memakannya, ia menjadi najis. Akulah TUHAN.
\par 9 Semua imam harus mentaati peraturan-peraturan yang Kuberikan. Kalau tidak, mereka bersalah dan akan mati, karena melanggar peraturan-peratura yang suci itu. Akulah TUHAN, dan Akulah yang mengkhususkan mereka untuk-Ku.
\par 10 Persembahan yang dikhususkan bagi-Ku hanya boleh dimakan oleh keluarga imam itu sendiri; orang lain tak boleh, sekalipun orang itu serumah dengan imam itu atau orang upahannya.
\par 11 Tetapi budak-budak yang dibeli oleh imam itu dengan uangnya sendiri atau yang dilahirkan di dalam rumahnya, boleh makan makanan bagian imam itu.
\par 12 Anak perempuan seorang imam yang kawin dengan seorang yang bukan imam, tidak boleh makan persembahan itu.
\par 13 Tetapi anak perempuan imam yang sudah menjadi janda atau yang diceraikan dengan tidak mempunyai anak lalu kembali ke rumah ayahnya serta menjadi tanggungannya, boleh makan makanan bagian ayahnya. Hanya keluarga imam saja boleh makan persembahan itu.
\par 14 Kalau seorang yang bukan anggota keluarga imam dengan tidak disengaja makan makanan yang dipersembahkan kepada TUHAN, ia harus memberi gantinya kepada imam ditambah dengan dua puluh persen.
\par 15 Imam-imam tidak boleh menajiskan persembahan yang khusus untuk Aku
\par 16 dengan membiarkannya dimakan oleh seorang yang tidak berhak memakannya. Hal itu menjadikan orang itu bersalah dan mendatangkan hukuman atasnya. Akulah TUHAN, dan Aku yang menyucikan persembahan itu."
\par 17 TUHAN menyuruh Musa
\par 18 memberi kepada Harun, anak-anaknya dan seluruh bangsa Israel peraturan-peraturan ini. Apabila seorang Israel atau seorang asing yang tinggal menetap di Israel mempersembahkan kurban bakaran untuk membayar kaul, atau untuk kurban sukarela, binatang itu harus yang jantan dan tak boleh ada cacatnya supaya dapat diterima.
\par 20 Jangan mempersembahkan binatang yang ada cacatnya, sebab TUHAN tidak mau menerimanya.
\par 21 Kalau seseorang mempersembahkan kepada TUHAN kurban perdamaian untuk membayar kaulnya atau untuk kurban sukarela, binatang itu tak boleh ada cacatnya, supaya dapat diterima.
\par 22 Janganlah mempersembahkan kepada TUHAN binatang yang buta, yang patah tulangnya atau yang luka, yang mempunyai bisul bernanah, yang menderita penyakit kulit atau kudis. Binatang yang begitu jangan dipersembahkan untuk kurban makanan di atas mezbah.
\par 23 Untuk kurban sukarela kamu boleh mempersembahkan binatang yang tidak sempurna bentuknya; tetapi untuk membayar kaul, binatang yang begitu tidak dapat diterima.
\par 24 Janganlah mempersembahkan kepada TUHAN binatang yang buah pelirnya rusak terjepit, dipotong, luka memar atau dibuang. Hal itu tak boleh terjadi di negerimu.
\par 25 Ternak yang diperoleh dari orang asing tidak boleh dipersembahkan untuk kurban makanan. Binatang itu dianggap cacat dan tak dapat diterima.
\par 26 Anak sapi, anak domba atau anak kambing yang baru lahir harus tinggal pada induknya selama tujuh hari. Baru sesudahnya binatang itu dapat diterima untuk kurban makanan.
\par 28 Janganlah mengurbankan induk sapi atau domba atau kambing bersama dengan anaknya pada hari yang sama.
\par 29 Apabila kamu mempersembahkan kurban syukur kepada TUHAN, lakukanlah itu menurut peraturan, supaya kurbanmu diterima.
\par 30 Makanlah kurban itu pada hari itu juga, dan jangan tinggalkan sedikit pun untuk keesokan harinya.
\par 31 TUHAN berkata, "Taatilah perintah-perintah-Ku; Akulah TUHAN.
\par 32 Jangan mencemarkan nama-Ku yang suci. Seluruh bangsa Israel harus menyembah Aku sebagai Yang Mahasuci. Akulah TUHAN dan Aku menjadikan kamu suci.
\par 33 Aku membawa kamu keluar dari Mesir, supaya Aku menjadi Allahmu. Akulah TUHAN."

\chapter{23}

\par 1 TUHAN memberi kepada Musa
\par 2 peraturan-peraturan tentang perayaan agama, yang harus diumumkan sebagai hari raya untuk beribadat.
\par 3 Ada enam hari untuk bekerja, tetapi hari yang ketujuh, hari Sabat, adalah hari istirahat yang dikhususkan bagi TUHAN. Pada hari itu kamu harus mengadakan pertemuan untuk beribadat dan tak boleh bekerja.
\par 4 Inilah hari-hari raya yang ditetapkan TUHAN dan yang harus diumumkan pada waktu yang ditentukan.
\par 5 Hari Raya Paskah yang dirayakan untuk menghormati TUHAN, mulai pada saat matahari terbenam pada tanggal empat belas bulan satu.
\par 6 Besoknya, tanggal lima belas, mulailah Perayaan Roti tak Beragi, dan selama tujuh hari kamu tak boleh makan roti yang dibuat pakai ragi.
\par 7 Selama tujuh hari itu kamu harus mempersembahkan kurban makanan kepada TUHAN. Pada hari yang pertama dan yang ketujuh perayaan itu kamu harus mengadakan pertemuan untuk beribadat dan tak boleh melakukan pekerjaan berat.
\par 9 Kalau kamu sudah sampai di negeri yang diberikan TUHAN kepadamu, dan mengumpulkan hasil tanah, berkas gandum yang pertama kamu tuai harus dibawa kepada imam.
\par 11 Pada hari sesudah hari Sabat, imam harus mempersembahkan gandum itu sebagai persembahan unjukan kepada TUHAN, supaya TUHAN senang kepadamu.
\par 12 Pada hari kamu mempersembahkan gandum itu, kamu harus mempersembahkan juga untuk kurban bakaran seekor anak domba jantan yang berumur satu tahun dan tidak ada cacatnya.
\par 13 Untuk kurban makanan persembahkanlah dua kilogram tepung dicampur minyak zaitun. Bau kurban itu menyenangkan hati TUHAN. Bersama itu persembahkanlah juga satu liter air anggur.
\par 14 Sebelum membawa persembahan itu kepada TUHAN, kamu tak boleh makan sedikit pun dari gandum baru itu, baik mentah, dipanggang atau dibakar menjadi roti. Peraturan itu harus ditaati oleh semua keturunanmu selama-lamanya.
\par 15 Mulai dari hari sesudah Sabat, waktu kamu mempersembahkan kepada TUHAN berkas gandum yang pertama, kamu harus menghitung tujuh minggu penuh.
\par 16 Pada hari yang kelima puluh, yaitu hari sesudah Sabat yang ketujuh kamu harus membawa kurban sajian kepada TUHAN.
\par 17 Setiap keluarga harus mempersembahkan dua buah roti untuk persembahan unjukan bagi TUHAN. Tiap roti harus dibuat dari dua kilogram tepung halus pakai ragi dan harus diserahkan kepada TUHAN sebagai persembahan hasil tanah yang pertama.
\par 18 Bersama roti itu bangsa Israel harus mempersembahkan tujuh ekor anak domba yang berumur satu tahun, seekor sapi jantan muda dan dua ekor kambing. Masing-masing tak boleh ada cacatnya dan harus dipersembahkan untuk kurban bakaran bagi TUHAN, dengan kurban sajian dan kurban air anggur. Bau kurban itu menyenangkan hati TUHAN.
\par 19 Persembahkanlah juga seekor kambing jantan untuk kurban pengampunan dosa dan dua ekor domba jantan yang berumur satu tahun untuk kurban perdamaian.
\par 20 Roti dan kedua anak domba itu harus dipersembahkan oleh imam sebagai persembahan unjukan bagi TUHAN, dan itu untuk bagian imam. Persembahan-persembahan itu adalah suci.
\par 21 Pada hari itu kamu harus mengadakan pertemuan untuk beribadat dan tak boleh melakukan pekerjaan berat. Peraturan itu harus ditaati oleh keturunanmu untuk selama-lamanya, di mana saja mereka tinggal.
\par 22 Kalau kamu panen, janganlah memotong gandum yang tumbuh di pinggir-pinggir ladangmu dan jangan kembali untuk mengumpulkan gandum yang tersisa sesudah panen. Biarkan itu untuk orang miskin dan orang asing. TUHAN adalah Allahmu.
\par 23 Tanggal satu bulan tujuh harus dirayakan sebagai hari yang khusus untuk beristirahat. Hari itu ditandakan dengan bunyi trompet, dan kamu harus mengadakan pertemuan untuk beribadat.
\par 25 Persembahkanlah kurban makanan kepada TUHAN dan jangan melakukan pekerjaan berat pada hari itu.
\par 26 Tanggal sepuluh bulan tujuh adalah hari upacara tahunan untuk pengampunan dosa bangsa Israel. Pada hari itu kamu harus berpuasa dan mengadakan pertemuan untuk beribadat serta mempersembahkan kurban makanan kepada TUHAN.
\par 28 Janganlah bekerja, sebab hari itu adalah hari pengampunan dosa, supaya kamu mendapat ampun di hadapan TUHAN Allahmu.
\par 29 Barangsiapa makan sesuatu pada hari itu, tidak dianggap lagi anggota umat Allah.
\par 30 Dan barangsiapa melakukan pekerjaan pada hari itu, dihukum oleh TUHAN sendiri dengan hukuman mati.
\par 31 Peraturan itu berlaku untuk semua keturunanmu di mana saja mereka tinggal.
\par 32 Mulai dari saat matahari terbenam pada tanggal sembilan bulan itu sampai saat matahari terbenam pada tanggal sepuluh, harus kamu rayakan seperti hari Sabat, hari yang khusus untuk istirahat. Sepanjang hari itu kamu harus berpuasa.
\par 33 Pesta Pondok Daun mulai pada tanggal lima belas bulan tujuh dan lamanya tujuh hari.
\par 35 Selama tujuh hari itu, setiap hari kamu harus mempersembahkan kurban makanan. Pada hari yang pertama dan yang kedelapan kamu harus mengadakan pertemuan untuk beribadat dan mempersembahkan kurban makanan kepada TUHAN. Janganlah melakukan pekerjaan berat pada hari itu.
\par 37 Itulah hari-hari raya agama yang ditetapkan TUHAN dan yang harus kamu umumkan. Pada perayaan-perayaan itu kamu harus menghormati TUHAN dengan mengadakan pertemuan untuk beribadat dan mempersembahkan kurban makanan, kurban bakaran, kurban sajian, kurban binatang dan kurban air anggur, sebanyak yang diperlukan setiap harinya.
\par 38 Perayaan-perayaan itu merupakan tambahan pada hari-hari Sabat biasa, jadi persembahan-persembahan itu adalah juga tambahan pada pemberianmu yang biasa, pada kurbanmu untuk menebus kaul, atau kurban sukarela yang kamu berikan kepada TUHAN.
\par 39 Sehabis panen, mulai tanggal lima belas bulan tujuh, kamu harus mengadakan perayaan selama tujuh hari. Hari yang pertama dan yang kedelapan adalah hari yang khusus untuk beristirahat.
\par 40 Pada hari yang pertama kamu harus memetik buah-buahan yang paling baik dari kebunmu, juga daun-daun palma dan ranting-ranting dari pohon-pohon yang rimbun, lalu mulailah perayaan sukacita untuk menghormati TUHAN Allahmu.
\par 41 Perayaan itu harus kamu langsungkan selama tujuh hari. Peraturan itu harus dilakukan oleh keturunanmu untuk selama-lamanya.
\par 42 Tujuh hari lamanya seluruh bangsa Israel harus tinggal dalam pondok-pondok daun,
\par 43 supaya keturunanmu tahu bahwa TUHAN menyuruh bangsa Israel tinggal dalam pondok-pondok yang sederhana, ketika Ia membawa mereka keluar dari Mesir. Dialah TUHAN Allahmu.
\par 44 Begitulah Musa menyampaikan kepada bangsa Israel peraturan-peraturan tentang cara mengadakan perayaan-perayaan untuk menghormati TUHAN.

\chapter{24}

\par 1 TUHAN menyuruh Musa
\par 2 memberikan perintah-perintah ini kepada bangsa Israel: Bawalah minyak zaitun yang murni dan yang paling baik untuk lampu di dalam Kemah TUHAN, supaya dapat dipasang dan menyala terus.
\par 3 Harun harus mengatur lampu-lampu itu dari petang sampai pagi di tempat TUHAN hadir, di luar tirai yang menutupi Peti Perjanjian di Ruang Mahasuci. Perintah itu harus dilakukan untuk selama-lamanya.
\par 4 Harun harus mengurus lampu-lampu pada kaki lampu dari emas murni supaya tetap menyala di Kemah TUHAN.
\par 5 Ambillah dua belas kilogram tepung yang paling baik dan buatlah dua belas roti bundar, masing-masing dari satu kilogram tepung.
\par 6 Letakkan roti itu dalam dua susun, enam roti setiap susunnya, di atas meja berlapis emas murni yang berada di dalam Kemah TUHAN.
\par 7 Taruhlah kemenyan tulen pada tiap susun roti untuk menandakan bahwa roti itu kurban makanan bagi TUHAN.
\par 8 Setiap hari Sabat harus ada roti yang disajikan di dalam Kemah TUHAN. Itulah kewajiban bangsa Israel untuk selama-lamanya.
\par 9 Roti itu adalah untuk Harun dan keturunannya. Mereka harus memakannya di tempat yang suci, karena roti itu merupakan bagian yang sangat suci dari kurban makanan yang dipersembahkan kepada TUHAN untuk para imam.
\par 10 Di perkemahan orang Israel ada seorang laki-laki. Ayahnya orang Mesir dan ibunya orang Israel bernama Selomit, anak Dibri dari suku Dan. Pada suatu hari orang itu bertengkar dengan seorang Israel di perkemahan. Waktu bertengkar, ia mengutuk Allah. Karena itu orang-orang Israel membawa dia menghadap Musa,
\par 12 dan menahan dia sambil menunggu keputusan TUHAN mengenai apa yang harus mereka lakukan terhadapnya.
\par 13 TUHAN berkata kepada Musa,
\par 14 "Bawalah orang itu ke luar perkemahan. Setiap orang yang telah mendengar dia mengutuk, harus meletakkan tangannya di atas kepala orang itu untuk memberi kesaksian bahwa dia bersalah. Lalu seluruh umat harus melempari dia dengan batu sampai mati.
\par 15 Setelah itu umumkanlah ini kepada bangsa Israel: Barangsiapa mengutuk Allah harus menanggung akibatnya
\par 16 dan dihukum mati. Siapa saja yang mengutuk TUHAN harus dilempari dengan batu sampai mati oleh seluruh jemaat. Hukum itu berlaku untuk orang Israel maupun untuk orang asing yang sudah menetap di Israel.
\par 17 Barangsiapa membunuh orang lain harus dihukum mati.
\par 18 Barangsiapa membunuh binatang orang lain harus menggantinya. Dasarnya ialah 'nyawa ganti nyawa'.
\par 19 Apabila seseorang membuat orang lain cedera, apa saja yang telah dilakukannya, harus dilakukan juga terhadap dia.
\par 20 Kalau ia mematahkan tulang, maka tulangnya pun harus dipatahkan. Kalau ia membuat mata orang lain buta sebelah, maka matanya pun harus dibutakan sebelah. Kalau ia memukul orang lain sampai patah giginya, maka giginya pun harus dipatahkan. Apa saja yang dilakukannya sehingga orang lain cacat, harus juga dilakukan terhadap dia sebagai pembalasan.
\par 21 Barangsiapa membunuh binatang harus menggantinya, tetapi barangsiapa membunuh manusia harus dihukum mati.
\par 22 Hukum itu berlaku untuk kamu semua dan juga untuk orang asing yang menetap di antara kamu, karena Akulah TUHAN, Allahmu."
\par 23 Sesudah Musa menyampaikan perkataan itu kepada bangsa Israel, mereka membawa orang yang telah mengutuk Allah itu ke luar perkemahan lalu melempari dia dengan batu sampai mati. Dengan cara itu bangsa Israel melakukan apa yang telah diperintahkan TUHAN kepada Musa.

\chapter{25}

\par 1 Di atas Gunung Sinai TUHAN berbicara kepada Musa dan menyuruh dia
\par 2 menyampaikan peraturan-peraturan ini kepada bangsa Israel. Apabila kamu masuk ke negeri yang akan diberikan TUHAN kepadamu, kamu harus menghormati TUHAN dengan cara ini: Setiap tahun ketujuh kamu tak boleh mengerjakan tanah untuk bercocok tanam.
\par 3 Selama enam tahun kamu harus menanami ladang-ladangmu, memangkas kebun-kebun anggurmu dan mengumpulkan hasil tanahmu.
\par 4 Tetapi setiap tahun yang ketujuh dikhususkan untuk TUHAN. Jadi sepanjang tahun itu tanah harus dibiarkan dan tak boleh dikerjakan. Jangan menanami ladang-ladang dan jangan memangkas pohon-pohon anggurmu.
\par 5 Bahkan gandum yang tumbuh dengan sendirinya atau anggur dari pohon anggurmu yang tidak dipangkas, tak boleh kamu kumpulkan dan simpan, sebab tahun itu adalah tahun istirahat untuk tanah.
\par 6 Walaupun begitu, tanah itu akan memberi hasil juga untuk dijadikan makanan bagi kamu, budak-budakmu, orang-orang upahanmu, dan orang-orang asing yang tinggal dengan kamu.
\par 7 Juga bagi binatang-binatang peliharaanmu dan binatang-binatang liar di negerimu. Segala yang dihasilkan tanah itu boleh dimakan.
\par 8 Hitunglah tujuh kali tujuh tahun, sebanyak empat puluh sembilan tahun.
\par 9 Lalu pada tanggal sepuluh bulan tujuh, yaitu hari pengampunan dosa, suruhlah orang membunyikan trompet di mana-mana di seluruh negeri.
\par 10 Kamu harus mengkhususkan tahun yang kelima puluh itu, dan mengumumkan kebebasan kepada seluruh penduduk negeri. Dalam tahun itu segala harta milik yang sudah dijual harus dikembalikan kepada pemiliknya yang semula atau kepada keturunannya. Siapa yang dijual sebagai budak harus dikembalikan kepada keluarganya.
\par 11 Janganlah menanami ladangmu. Gandum yang tumbuh dengan sendirinya atau buah anggur di kebun anggurmu yang tidak dipangkas tak boleh kamu kumpulkan dan simpan.
\par 12 Tahun itu merupakan tahun yang suci bagimu. Kamu hanya boleh makan apa yang tumbuh dengan sendirinya di ladangmu.
\par 13 Dalam tahun itu semua tanah yang sudah dijual harus dikembalikan kepada pemiliknya yang semula.
\par 14 Jadi kalau kamu menjual tanah kepada sesamamu orang Israel, atau membeli tanah dari dia, janganlah melakukan jual beli itu dengan tidak jujur.
\par 15 Harga tanah itu harus ditetapkan menurut jumlah tahun selama tanah itu dapat memberi hasil sampai Tahun Pengembalian yang berikut.
\par 16 Kalau jumlah tahunnya banyak, harga tanah itu mahal. Tetapi kalau jumlah tahunnya sedikit, harganya harus murah, karena yang dijual adalah jumlah panen.
\par 17 Janganlah menipu sesamamu orang Israel, tetapi hendaklah kamu takut akan TUHAN, Allahmu.
\par 18 Taatilah semua hukum dan perintah TUHAN, supaya kamu hidup dengan aman di negeri itu.
\par 19 Tanah negeri itu akan memberi hasilnya sehingga kamu mempunyai makanan yang cukup dan tinggal di sana dengan sejahtera.
\par 20 Tetapi barangkali ada yang bertanya apa yang akan dimakan selama tahun ketujuh pada waktu tanah tidak ditanami dan hasilnya tidak dipungut.
\par 21 Dalam tahun keenam TUHAN akan memberkati tanah sehingga hasilnya cukup untuk tiga tahun.
\par 22 Dan pada waktu kamu menanami ladangmu dalam tahun kedelapan, panen dari tahun keenam masih ada untuk kamu makan. Malah persediaan makananmu akan cukup sampai tiba waktunya kamu mengumpulkan hasil tahun itu.
\par 23 Tanahmu tidak boleh dijual untuk seterusnya, sebab tanah itu bukan milikmu, melainkan milik Allah. Kamu hanya seperti orang asing yang mendapat izin untuk memakai tanah itu.
\par 24 Kalau sebidang tanah dijual, harus diakui hak pemiliknya yang semula untuk menebus tanah itu.
\par 25 Kalau seorang Israel jatuh miskin sehingga ia terpaksa menjual tanahnya, sanak saudaranya yang paling dekat wajib menebus tanah itu.
\par 26 Kalau tidak ada orang untuk menebusnya tetapi pemilik itu kemudian hari menjadi mampu dan mempunyai cukup uang untuk menebus tanahnya,
\par 27 maka ia harus membayar kepada orang yang telah membeli tanahnya seharga hasil tanah itu selama tahun-tahun berikutnya sampai Tahun Pengembalian yang akan datang. Dengan demikian ia mendapat kembali tanah miliknya.
\par 28 Tetapi kalau ia tidak mampu menebus tanahnya, orang yang membelinya tetap berkuasa atas tanah itu sampai Tahun Pengembalian yang berikut. Dalam tahun itu tanah itu harus dikembalikan kepada pemiliknya yang semula.
\par 29 Apabila seseorang menjual sebuah rumah dalam kota yang dikelilingi tembok, maka ia berhak menebus rumahnya dalam jangka waktu satu tahun terhitung dari tanggal penjualan.
\par 30 Tetapi kalau rumah itu tidak ditebusnya dalam waktu satu tahun hilanglah haknya untuk menebusnya. Rumah itu tetap milik orang yang membelinya dan keturunannya. Dalam Tahun Pengembalian, rumah itu tidak dikembalikan.
\par 31 Untuk rumah di desa yang tidak dikelilingi tembok berlaku peraturan yang sama seperti untuk tanah: pemilik semula berhak menebusnya dan rumah itu harus dikembalikan dalam Tahun Pengembalian.
\par 32 Akan tetapi orang Lewi berhak untuk kapan saja menebus harta milik mereka di kota-kota yang ditetapkan untuk mereka.
\par 33 Apabila seorang Lewi menjual rumahnya di salah satu kota itu dan ia tidak menebusnya, rumah itu harus dikembalikan kepadanya dalam Tahun Pengembalian. Sebab rumah orang Lewi di kota-kota mereka adalah milik mereka yang tetap di antara bangsa Israel.
\par 34 Tetapi padang rumput di sekitar kota-kota suku Lewi tak boleh dijual. Itu adalah kepunyaan orang Lewi untuk selama-lamanya.
\par 35 Apabila orang sebangsamu yang tinggal di antara kamu jatuh miskin dan tak dapat membiayai hidupnya kamu harus menyokong dia seperti seorang asing atau pendatang, supaya ia dapat hidup terus di antara kamu.
\par 36 Janganlah menuntut bunga dari apa yang kamu pinjamkan kepadanya, tetapi hendaklah kamu takut kepada Allahmu, dan biarkan orang itu hidup di antara kamu. Jangan juga mengambil untung dari makanan yang kamu jual kepadanya.
\par 38 Itulah perintah TUHAN Allahmu yang telah membawa kamu keluar dari Mesir untuk memberikan negeri Kanaan kepadamu, supaya Ia menjadi Allahmu.
\par 39 Apabila orang sebangsamu yang tinggal di antara kamu jatuh miskin sehingga ia menjual dirinya sebagai budak kepadamu, orang itu tak boleh kamu suruh melakukan pekerjaan seorang budak.
\par 40 Ia harus tinggal padamu sebagai seorang upahan dan bekerja untukmu sampai Tahun Pengembalian.
\par 41 Pada waktu itu dia dan anak-anaknya boleh meninggalkan rumahmu dan kembali kepada keluarganya dan ke tanah nenek moyangnya.
\par 42 Orang Israel adalah hamba-hamba TUHAN; Dialah yang membawa mereka keluar dari Mesir, dan karena itu mereka tak boleh dijual sebagai budak.
\par 43 Jadi orang Israel yang menjual dirinya kepadamu tak boleh kamu perlakukan dengan kasar. Hendaklah kamu takut kepada Allahmu.
\par 44 Kalau kamu memerlukan budak, kamu boleh membelinya dari bangsa-bangsa yang tinggal di sekitarmu.
\par 45 Kamu boleh juga membeli anak-anak orang asing yang menetap di antara kamu. Anak-anak orang asing yang lahir di negerimu boleh menjadi milikmu
\par 46 dan boleh juga kamu wariskan kepada anak-anakmu untuk menjadi budak mereka seumur hidup. Tetapi sesamamu orang Israel tak boleh kamu perlakukan dengan kasar.
\par 47 Apabila seorang asing yang tinggal di tengah-tengahmu menjadi kaya, dan ada sesamamu orang Israel jatuh miskin sehingga ia menjual dirinya sebagai budak kepada orang asing itu atau kepada keluarganya,
\par 48 orang Israel itu tetap mempunyai hak untuk ditebus. Seorang dari saudaranya yang laki-laki,
\par 49 atau pamannya, atau saudara sepupunya atau seorang lain dari kerabatnya yang dekat, boleh menebusnya. Atau kalau ia sendiri kemudian mampu, ia boleh menebus dirinya sendiri.
\par 50 Budak itu harus berunding dengan orang yang sudah membelinya. Mereka harus menghitung jumlah tahun sejak dirinya dijual sebagai budak sampai Tahun Pengembalian yang berikut, lalu menentukan harga tebusannya. Harga itu harus ditetapkan menurut perhitungan gaji yang diberikan kepada orang upahan.
\par 51 Budak itu harus mengembalikan harga pembelian dirinya sebesar jumlah gaji seorang upahan selama tahun-tahun yang belum dijalaninya,
\par 53 seolah-olah ia telah bekerja sebagai orang upahan yang digaji tahunan. Tuannya tidak boleh memperlakukan dia dengan kasar.
\par 54 Kalau ia tidak dapat ditebus dengan salah satu cara itu, ia dan anak-anaknya harus dibebaskan dalam Tahun Pengembalian yang berikut.
\par 55 Seorang Israel tak boleh menjadi budak seumur hidup karena bangsa Israel adalah hamba-hamba TUHAN. Tuhanlah yang membawa mereka keluar dari Mesir. Dialah TUHAN, Allah mereka.

\chapter{26}

\par 1 TUHAN berkata, "Janganlah membuat berhala atau mendirikan patung atau tiang batu yang berukir untuk disembah. Akulah TUHAN Allahmu.
\par 2 Rayakanlah hari-hari Sabat dan hari-hari raya agama lainnya dan hormatilah tempat yang dikhususkan untuk menyembah Aku. Akulah TUHAN.
\par 3 Kalau kamu hidup menurut peraturan-peraturan-Ku dan mentaati perintah-perintah-Ku,
\par 4 Aku akan menurunkan hujan pada waktunya, sehingga tanah memberi hasil dan pohon-pohon berbuah.
\par 5 Hasil tanahmu akan berlimpah-limpah, sehingga bila sudah sampai waktunya untuk memetik buah anggur, kamu masih memotong gandum. Dan bila sudah sampai waktunya untuk menanam gandum, kamu masih memetik buah anggur. Kamu akan mempunyai makanan yang cukup dan hidup sejahtera di negerimu.
\par 6 Aku akan menjaga keamanan negerimu sehingga kamu bisa tidur dengan tenang dan tidak takut terhadap siapa pun. Binatang-binatang buas akan Kulenyapkan dari negeri itu, dan tak akan ada lagi peperangan di negerimu.
\par 7 Kamu akan sanggup mengalahkan musuh-musuhmu;
\par 8 dengan lima orang kamu sanggup mengalahkan seratus orang, dan dengan seratus orang kamu sanggup mengalahkan sepuluh ribu orang.
\par 9 Aku akan memberkati kamu dan memberikan banyak anak cucu kepadamu. Apa yang sudah Kujanjikan kepadamu, pasti Kutepati.
\par 10 Hasil tanahmu akan berlimpah-limpah, sehingga cukup untuk satu tahun. Bahkan kamu terpaksa mengeluarkan kelebihan dari panen yang lama supaya ada tempat untuk menyimpan panen yang baru.
\par 11 Aku akan tinggal di antara kamu dalam Kemah-Ku, dan tak akan menolak kamu.
\par 12 Aku akan menyertai kamu; Aku akan menjadi Allahmu dan kamu menjadi umat-Ku.
\par 13 Aku, TUHAN Allahmu, yang membawa kamu keluar dari Mesir supaya kamu tidak menjadi budak lagi. Kekuasaan yang menindas kamu sudah Kupatahkan, dan kamu Kujadikan orang yang merdeka."
\par 14 TUHAN berkata, "Kalau kamu tidak mendengarkan Aku dan tidak melakukan perintah-Ku,
\par 15 kalau kamu melanggar perjanjian yang Kubuat dengan kamu dan tidak mau mentaati ketetapan-ketetapan dan peraturan-peraturan-Ku,
\par 16 kamu akan Kuhukum. Aku akan mendatangkan bencana ke atas kamu, penyakit dan demam yang tak dapat disembuhkan, sehingga matamu menjadi buta dan kamu hidup merana. Kamu akan bercocok tanam tetapi tidak memakan hasilnya, sebab musuh-musuhmu akan datang mengalahkan kamu dan menghabiskan apa yang kamu tanam.
\par 17 Kamu akan Kuhukum, sehingga kamu dikalahkan musuh-musuhmu dan dikuasai orang-orang yang membencimu. Kamu akan menjadi sangat ketakutan sehingga kamu lari, walaupun tak ada yang mengejar.
\par 18 Kalau sesudah mengalami semua hukuman itu kamu belum juga taat kepada-Ku, hukumanmu akan Kutambah tujuh kali lipat.
\par 19 Kamu sangat sombong, tetapi Aku akan menundukkan kamu. Hujan tak akan turun, sehingga tanahmu menjadi kering dan keras seperti besi.
\par 20 Semua jerih payahmu tak ada gunanya, sebab tanahmu tak akan memberi hasil dan tanam-tanamanmu tak akan berbuah.
\par 21 Kalau kamu masih terus juga melawan Aku dan tidak mau taat kepada-Ku, maka hukumanmu akan Kutambah lagi tujuh kali lipat.
\par 22 Aku akan mendatangkan binatang-binatang buas ke tengah-tengahmu. Binatang-binatang itu akan menerkam anak-anakmu, membunuh ternakmu dan menghabiskan kamu, sehingga jumlahmu menjadi sedikit sekali, dan jalan-jalan di negerimu menjadi sunyi.
\par 23 Kalau sesudah mengalami semua hukuman itu kamu belum juga mendengarkan Aku, tetapi masih terus melawan Aku,
\par 24 maka Aku akan menghajar kamu dan menghukum kamu tujuh kali lebih berat dari yang sudah-sudah.
\par 25 Aku akan menimbulkan peperangan di tengah-tengahmu untuk menghukum kamu karena kamu sudah melanggar perjanjian-Ku dengan kamu. Dan kalau kamu berkumpul di kotamu untuk mencari perlindungan, Aku akan mendatangkan penyakit yang tak dapat disembuhkan, sehingga kamu terpaksa menyerah kepada musuhmu.
\par 26 Kamu akan kehabisan makanan, sehingga sepuluh wanita hanya memerlukan satu tempat pembakaran saja untuk membakar roti yang ada pada mereka. Makanan itu akan diransum, dan sesudah kamu memakannya, kamu masih juga merasa lapar.
\par 27 Kalau sesudah mengalami semua hukuman itu kamu masih terus juga melawan Aku dan tidak mau taat kepada-Ku,
\par 28 maka dengan marah Aku akan menghajar kamu, dan menghukum kamu tujuh kali lebih berat dari yang sudah-sudah.
\par 29 Kamu akan kelaparan, sehingga makan anak-anakmu sendiri.
\par 30 Aku akan menghancurkan tempat-tempat pemujaanmu di atas bukit-bukit, merobohkan mezbah-mezbahmu tempat membakar dupa, dan melemparkan mayat-mayatmu ke atas berhala-berhalamu yang sudah roboh itu. Dengan rasa muak
\par 31 kota-kotamu Kurobah menjadi puing-puing, tempat-tempat pemujaanmu Kuhancurkan dan kurban-kurbanmu tidak Kuterima.
\par 32 Negerimu akan Kubinasakan sama sekali, sehingga musuh-musuh yang mendudukinya terkejut melihat hebatnya kehancuran itu.
\par 33 Aku akan menimbulkan peperangan di negerimu dan menceraiberaikan kamu di negeri-negeri asing. Ladang-ladangmu akan terlantar, dan kota-kotamu dibiarkan menjadi puing.
\par 34 Pada waktu itu sementara kamu berada dalam pembuangan di negeri musuh-musuhmu, tanahmu yang tidak kamu biarkan beristirahat sewaktu kamu mendiaminya, akan terlantar dan mendapat kesempatan beristirahat.
\par 36 Orang-orangmu yang berada dalam pembuangan di negeri lain, akan Kubuat sangat ketakutan, sehingga bunyi daun yang ditiup angin pun membuat mereka lari. Kamu akan lari seolah-olah dikejar dalam pertempuran, lalu jatuh walaupun tidak ada musuh di dekatmu.
\par 37 Kamu akan lari dan tersandung yang seorang kepada yang lain, walaupun tidak ada yang mengejar. Dan kamu tak akan mampu melawan musuh-musuhmu.
\par 38 Kamu akan mati dalam pembuangan, dibinasakan oleh negeri musuh-musuhmu.
\par 39 Sebagian kecil dari kamu yang masih hidup di negeri musuhmu akan merana karena dosamu sendiri dan dosa leluhurmu.
\par 40 Tetapi keturunanmu akan mengakui dosa mereka dan dosa nenek moyang mereka yang telah menentang Aku dan memberontak terhadap-Ku,
\par 41 sehingga Aku bertindak melawan mereka dan mengusir mereka ke dalam pembuangan di negeri musuh-musuh mereka. Akhirnya, apabila keturunanmu sudah tunduk dan sudah menjalani hukuman karena dosa dan pemberontakan mereka,
\par 42 Aku akan mengingat perjanjian-Ku dengan Yakub, Ishak dan Abraham. Aku akan memperbaharui janji-Ku untuk memberikan tanah itu kepada bangsa-Ku.
\par 43 Akan tetapi tanah itu harus lebih dahulu dikosongkan dari penduduknya, supaya dapat betul-betul beristirahat. Dan umat-Ku harus menjalani semua hukuman yang Kujatuhkan ke atas mereka karena tidak mau mentaati peraturan-peraturan dan perintah-perintah-Ku.
\par 44 Tetapi selagi mereka masih berada di negeri musuh pun, mereka tidak akan Kutinggalkan atau Kubinasakan sama sekali. Sebab kalau mereka Kubinasakan, berarti Aku memutuskan perjanjian-Ku dengan mereka, sedangkan Aku ini TUHAN, Allah mereka.
\par 45 Aku akan memperbaharui perjanjian yang Kubuat dengan leluhur mereka, ketika Aku menunjukkan kekuasaan-Ku kepada segala bangsa dengan membawa umat-Ku keluar dari Mesir, supaya Aku, TUHAN, menjadi Allah mereka."
\par 46 Itulah hukum-hukum dan perintah-perintah yang diberikan TUHAN di atas Gunung Sinai kepada Musa untuk bangsa Israel.

\chapter{27}

\par 1 TUHAN memberi kepada Musa
\par 2 peraturan ini untuk bangsa Israel. Apabila seorang dipersembahkan kepada TUHAN untuk menebus suatu kaul yang khusus, orang itu boleh ditebus dengan sejumlah uang
\par 3 menurut harga yang berlaku di Kemah TUHAN: di atas 60 tahun: laki-laki-15 uang perak, perempuan-10 uang perak; antara 20-60 tahun: laki-laki-50 uang perak, perempuan-30 uang perak; antara 5-20 tahun: laki-laki-20 uang perak, perempuan-10 uang perak; di bawah 5 tahun: laki-laki-5 uang perak, perempuan-3 uang perak.
\par 8 Kalau orang yang berkaul terlalu miskin untuk membayar harga yang ditetapkan, ia harus membawa orang yang sudah dipersembahkan kepada TUHAN itu kepada imam. Imam harus menetapkan harga yang lebih rendah sesuai dengan kemampuan orang itu.
\par 9 Apabila kaul itu mengenai binatang halal yang dapat diterima sebagai persembahan kepada TUHAN, orang yang berkaul itu tidak boleh menukarnya dengan binatang lain, sebab segala yang dipersembahkan kepada TUHAN menjadi milik TUHAN. Kalau ia menukarnya juga, kedua ekor binatang menjadi milik TUHAN.
\par 11 Tetapi kalau kaul itu mengenai binatang haram yang tak dapat diterima sebagai persembahan kepada TUHAN, orang yang berkaul harus membawa binatang itu kepada imam.
\par 12 Imam akan menentukan harganya menurut keadaan binatang itu. Dan harga yang sudah ditentukan adalah harga mati.
\par 13 Kalau orang itu ingin menebus binatangnya, ia harus membayar harganya ditambah dengan dua puluh persen.
\par 14 Apabila seseorang mempersembahkan rumahnya kepada TUHAN, imam menentukan harganya menurut keadaan rumah itu. Harga yang sudah ditentukan adalah harga mati.
\par 15 Kalau orang itu mau menebus rumahnya, ia harus membayar harganya ditambah dengan dua puluh persen.
\par 16 Apabila seseorang mempersembahkan sebagian dari tanahnya kepada TUHAN, harganya harus ditentukan menurut jumlah bibit yang diperlukan untuk menanami tanah itu. Untuk setiap dua puluh kilogram gandum harganya sepuluh uang perak.
\par 17 Kalau tanah itu dipersembahkan sejak Tahun Pengembalian, harganya harus dibayar penuh.
\par 18 Kalau tanah itu dipersembahkan sesudah Tahun Pengembalian, imam harus menaksir harga kontannya menurut jumlah tahun yang masih ada sampai Tahun Pengembalian yang berikut, lalu menentukan harga yang lebih murah.
\par 19 Kalau orang yang mempersembahkan ladang itu ingin menebusnya, ia harus membayar harganya ditambah dengan dua puluh persen.
\par 20 Kalau ia menjual ladang itu dengan tidak lebih dahulu menebusnya dari TUHAN, ia tidak berhak lagi menebus ladangnya.
\par 21 Dalam Tahun Pengembalian yang berikut, ladang itu menjadi milik TUHAN untuk selama-lamanya, dan diberikan kepada para imam.
\par 22 Apabila seseorang mempersembahkan kepada TUHAN sebuah ladang yang telah dibelinya,
\par 23 imam harus menaksir harganya menurut jumlah tahun yang masih ada sampai Tahun Pengembalian yang berikut. Lalu orang itu harus membayar harganya pada hari itu juga. Uang itu menjadi milik TUHAN.
\par 24 Dalam Tahun Pengembalian, ladang itu harus dikembalikan kepada pemiliknya yang semula atau kepada keturunannya.
\par 25 Semua harga harus ditentukan menurut harga yang berlaku di Kemah TUHAN.
\par 26 Binatang yang pertama lahir adalah milik TUHAN, jadi tak boleh dipersembahkan untuk kurban sukarela. Anak sapi, anak domba atau anak kambing yang pertama lahir adalah kepunyaan TUHAN.
\par 27 Tetapi binatang haram yang pertama lahir boleh ditebus menurut harga yang berlaku di Kemah TUHAN, ditambah dua puluh persen. Kalau tidak ditebus, binatang itu boleh dijual kepada orang lain menurut harga yang berlaku di Kemah TUHAN.
\par 28 Tak seorang pun boleh menjual atau menebus apa yang telah dikhususkannya tanpa syarat kepada TUHAN, baik itu manusia, hewan atau tanah. Itu milik TUHAN untuk selama-lamanya.
\par 29 Bahkan manusia yang telah dikhususkan kepada TUHAN untuk dibinasakan tak boleh ditebus; ia harus dibunuh.
\par 30 Sepersepuluh dari seluruh hasil tanah, baik gandum maupun buah-buahan, adalah untuk TUHAN.
\par 31 Kalau seseorang mau menebus sebagian dari hasil itu, ia harus membayar harganya yang sudah ditentukan ditambah dua puluh persen.
\par 32 Satu dari tiap sepuluh ekor ternak adalah milik TUHAN. Kalau ternak itu dihitung, setiap ekor ternak yang kesepuluh menjadi milik TUHAN.
\par 33 Pemilik ternak itu tak boleh memilih-milih mana yang baik, mana yang jelek. Ia juga tak boleh menukarnya. Kalau ia menukarnya juga, kedua ekor ternak itu menjadi milik TUHAN dan tak boleh ditebus.
\par 34 Itulah perintah-perintah yang diberikan TUHAN di atas Gunung Sinai kepada Musa untuk bangsa Israel.


\end{document}