\begin{document}

\title{Ezekiel}

Eze 1:1  Pada tanggal lima bulan empat pada tahun ketiga puluh, dalam tahun kelima sejak Raja Yoyakhin diangkut ke pembuangan, aku, Imam Yehezkiel anak Busi, berada di tepi Sungai Kebar di Babel bersama orang-orang buangan Yahudi lainnya. Tiba-tiba langit terbuka dan aku mendapat penglihatan tentang Allah. Aku mendengar TUHAN berbicara kepadaku dan aku merasakan kuat kuasa-Nya.
Eze 1:4  Aku menengadah dan melihat badai bertiup dari utara. Petir sambar-menyambar dari segumpal awan yang amat besar, dan langit di sekitarnya terang-benderang. Dalam kilat yang memancar-mancar itu tampak sesuatu yang berkilauan seperti perunggu.
Eze 1:5  Di tengah-tengah awan itu, kulihat empat kerub yaitu makhluk hidup yang menyerupai manusia,
Eze 1:6  tetapi masing-masing mempunyai empat wajah dan dua pasang sayap.
Eze 1:7  Kakinya lurus dan mengkilap seperti perunggu yang digosok. Jari kakinya berkuku seperti kuku banteng.
Eze 1:8  Selain empat wajah dan dua pasang sayap itu, mereka mempunyai tangan manusia di bawah setiap sayap.
Eze 1:9  Sepasang sayap dari setiap makhluk itu terkembang dan ujung-ujungnya bersentuhan dengan sayap makhluk yang lain, sehingga keempat makhluk itu membentuk segi empat. Kalau bergerak, mereka masing-masing bergerak lurus ke depan tanpa membalikkan tubuhnya.
Eze 1:10  Setiap makhluk itu mempunyai empat wajah yang berlain-lainan; wajah manusia di depan, wajah singa di sebelah kanan, wajah banteng sebelah kiri, dan wajah rajawali di sebelah belakang.
Eze 1:11  Sepasang sayap dari setiap makhluk itu dikembangkan ke atas sehingga menyentuh ujung sayap-sayap makhluk di sebelahnya, dan sepasang sayapnya lagi terlipat menutupi tubuhnya.
Eze 1:12  Masing-masing makhluk itu menghadap ke empat jurusan sekaligus, sehingga mereka dapat pergi ke mana saja mereka suka tanpa memutar tubuhnya.
Eze 1:13  Di tengah makhluk-makhluk itu ada sesuatu yang tampaknya seperti bara api atau obor bernyala yang bergerak-gerak tiada hentinya. Api itu berpijar-pijar dan memercikkan kilatan-kilatan petir.
Eze 1:14  Makhluk-makhluk itu sendiri bergerak kian ke mari secepat kilat.
Eze 1:15  Sedang aku memperhatikan semua itu, kulihat empat buah roda di atas tanah, satu roda di samping setiap makhluk itu.
Eze 1:16  Keempat roda itu serupa dan berkilauan seperti batu permata, dan masing-masing mempunyai satu roda lainnya yang melintang di tengah-tengahnya.
Eze 1:17  Dengan demikian roda-roda itu dapat menuju keempat jurusan.
Eze 1:18  Lingkaran-lingkaran roda itu penuh dengan mata.
Eze 1:19  Makhluk-makhluk itu pergi ke mana saja mereka suka, dan roda-roda itu selalu ikut, karena dikuasai oleh makhluk-makhluk itu. Setiap kali makhluk-makhluk itu bergerak naik ke udara atau berhenti, roda-roda itu selalu ikut bersama mereka.
Eze 1:22  Di atas kepala makhluk-makhluk itu terbentang sesuatu yang seperti kubah dari kristal yang kemilau.
Eze 1:23  Di bawah kubah itu setiap makhluk itu merentangkan sepasang sayapnya ke arah makhluk yang di sebelahnya, sedangkan sayapnya yang sepasang lagi menutupi tubuhnya.
Eze 1:24  Kalau mereka terbang, aku mendengar suara kepakan sayapnya; bunyinya seperti deru air terjun, seperti derap langkah pasukan tentara yang besar, seperti suara Allah Yang Mahakuasa. Kalau mereka berhenti terbang, mereka melipat sayap-sayapnya,
Eze 1:25  tetapi dari atas kubah yang di atas kepala mereka, masih terdengar bunyi suara.
Eze 1:26  Di atas kubah itu ada sesuatu yang menyerupai takhta dari batu nilam, dan di atasnya duduk sesuatu yang tampaknya seperti manusia.
Eze 1:27  Bagian atasnya kelihatan bercahaya seperti perunggu di tengah nyala api. Bagian bawahnya bersinar terang-benderang,
Eze 1:28  dan berwarna-warni seperti pelangi. Itulah terang kemilau yang menunjukkan kehadiran TUHAN. Melihat itu, aku jatuh tertelungkup di tanah. Lalu kudengar suara yang
Eze 2:1  berkata, "Hai manusia fana, berdirilah. Aku ingin bicara dengan engkau."
Eze 2:2  Sewaktu suara itu kudengar, masuklah Roh Allah ke dalam diriku, dan ditegakkannya aku. Kemudian kudengar lagi suara itu mengatakan,
Eze 2:3  "Hai manusia fana, engkau Kuutus kepada orang-orang Israel yang memberontak melawan Aku. Mereka itu pemberontak seperti nenek moyang mereka juga.
Eze 2:4  Mereka keras kepala dan tidak menghormati Aku. Engkau Kuutus kepada mereka untuk menyampaikan apa yang Aku, TUHAN Yang Mahatinggi katakan kepada mereka,
Eze 2:5  entah mereka mau mendengarkan atau tidak; yang penting, mereka harus tahu bahwa ada nabi di tengah-tengah mereka.
Eze 2:6  Tetapi engkau, hai manusia fana, janganlah takut kepada para pemberontak itu, atau kepada apa pun yang mereka katakan. Mereka akan melawan dan menghinamu, sehingga seolah-olah engkau hidup di antara kalajengking. Meskipun begitu, janganlah takut.
Eze 2:7  Sampaikanlah semua pesan-Ku kepada mereka, entah mereka mau mendengarkan atau tidak. Ingatlah bahwa mereka itu pemberontak-pemberontak.
Eze 2:8  Hai manusia fana, dengarkan apa yang Kukatakan kepadamu. Jangan ikut memberontak juga. Bukalah mulutmu dan makanlah apa yang Kuberikan kepadamu."
Eze 2:9  Kemudian aku melihat tangan yang terulur kepadaku. Dalam tangan itu ada kitab gulungan,
Eze 2:10  yang ditulisi pada kedua sisinya. Lalu dibukanya gulungan itu dan di situ tertulis ratapan, keluh-kesah dan rintihan.
Eze 3:1  Allah berkata, "Hai manusia fana, makanlah kitab gulungan ini, lalu pergilah dan berbicaralah kepada orang-orang Israel."
Eze 3:2  Lalu kubuka mulutku dan Allah memberikan kitab gulungan itu kepadaku supaya kumakan.
Eze 3:3  Kata-Nya, "Hai manusia fana, makanlah! Isilah perutmu dengan kitab gulungan ini." Lalu kitab itu kumakan dan rasanya manis seperti madu.
Eze 3:4  Allah berkata lagi, "Hai manusia fana, pergilah kepada orang-orang Israel dan sampaikanlah segala pesan-Ku kepada mereka.
Eze 3:5  Kepada merekalah engkau Kuutus, dan bukan kepada bangsa yang bahasanya sulit, asing dan tidak kau mengerti. Seandainya engkau Kuutus kepada bangsa yang seperti itu, maka mereka akan mendengarkan kata-katamu.
Eze 3:7  Tetapi bangsa Israel tidak mau mendengarkan engkau, bahkan Aku pun tak mereka hiraukan. Mereka semua keras kepala dan suka melawan.
Eze 3:8  Sekarang engkau akan Kubuat keras kepala dan ulet seperti mereka juga.
Eze 3:9  Engkau akan Kujadikan seteguh batu karang dan sekeras intan. Janganlah takut kepada para pemberontak itu."
Eze 3:10  Selanjutnya Allah berkata, "Hai manusia fana, perhatikanlah baik-baik dan ingatlah segala yang akan Kukatakan kepadamu.
Eze 3:11  Pergilah kepada bangsamu yang ada dalam pembuangan itu, dan sampaikanlah apa yang Aku, TUHAN Yang Mahatinggi katakan kepada mereka, entah mereka mau mendengarkan atau tidak."
Eze 3:12  Lalu Roh Allah mengangkat aku dan di belakangku kudengar suara gemuruh yang berkata, "Pujilah kemuliaan TUHAN di surga!"
Eze 3:13  Kudengar bunyi sayap makhluk-makhluk itu yang saling bersentuhan, dan bunyi gemertak roda-roda itu yang senyaring bunyi gempa bumi.
Eze 3:14  Kekuasaan TUHAN mendatangi aku dengan hebat, dan ketika aku diangkat dan dibawa oleh Roh-Nya, hatiku terasa pahit dan panas.
Eze 3:15  Maka tibalah aku di tempat pemukiman para buangan di Tel-Abib di tepi Sungai Kebar. Selama tujuh hari aku tinggal di situ; aku termangu-mangu oleh segala yang baru saja kudengar dan kulihat itu.
Eze 3:16  Sesudah tujuh hari, TUHAN berkata kepadaku,
Eze 3:17  "Hai manusia fana, engkau Kuangkat menjadi penjaga bangsa Israel. Sampaikanlah kepada mereka peringatan yang Kuberikan ini.
Eze 3:18  Jika Aku memberitahukan bahwa seorang penjahat akan mati, tetapi engkau tidak memperingatkan dia supaya ia mengubah kelakuannya sehingga ia selamat, maka ia akan mati masih sebagai seorang berdosa, dan tanggung jawab atas kematiannya akan dituntut daripadamu.
Eze 3:19  Jika engkau memperingatkan orang jahat itu dan ia tak mau berhenti berbuat jahat, dia akan mati sebagai orang berdosa, tetapi engkau sendiri akan selamat.
Eze 3:20  Kalau seorang yang baik mulai berbuat dosa, dan dia Kuhadapkan dengan bahaya, maka ia akan mati apabila engkau tidak memperingatkannya. Ia akan mati karena dosa-dosanya, dan perbuatan-perbuatannya yang baik tidak akan Kuingat. Tetapi tanggung jawab atas kematiannya itu akan Kutuntut daripadamu.
Eze 3:21  Jika orang yang baik itu kauperingatkan supaya jangan berbuat dosa, dan ia menurut nasihatmu, maka dia tidak akan Kuhukum dan engkau akan selamat."
Eze 3:22  Kurasakan kuatnya kehadiran TUHAN dan kudengar Ia mengatakan kepadaku, "Bangunlah dan pergilah ke lembah. Di sana Aku akan berbicara kepadamu."
Eze 3:23  Maka pergilah aku ke lembah, dan di situ kulihat terang kemilau yang menunjukkan kehadiran TUHAN seperti yang telah kulihat di tepi Sungai Kebar. Lalu aku jatuh tertelungkup di atas tanah,
Eze 3:24  tetapi Roh Allah memasuki dan menegakkan aku. TUHAN berkata kepadaku, "Pulanglah dan berkurunglah di dalam rumah.
Eze 3:25  Hai manusia fana, engkau akan diikat dengan tali sehingga tak dapat keluar dan tak dapat bertemu dengan orang.
Eze 3:26  Aku akan membuat lidahmu kaku sehingga engkau tak dapat memperingatkan bangsa pemberontak ini.
Eze 3:27  Nanti, kalau Aku berbicara kepadamu lagi dan mengembalikan kemampuanmu untuk berkata-kata, engkau harus menyampaikan apa yang Aku, TUHAN Yang Mahatinggi katakan kepada mereka. Sebagian dari mereka akan mendengarkan, tetapi sebagian lagi tidak, karena mereka memang bangsa pemberontak."
Eze 4:1  Allah berkata, "Hai manusia fana, ambillah sebuah batu bata, letakkan di depanmu dan garislah di atasnya denah kota Yerusalem.
Eze 4:2  Lalu gambarkan suatu pengepungan terhadap kota itu, dengan parit-parit, timbunan-timbunan tanah, perkemahan-perkemahan dan alat-alat pendobrak di sekeliling temboknya.
Eze 4:3  Ambillah panci besi dan anggaplah itu sebuah dinding besi; letakkanlah panci itu di antara engkau dan kota Yerusalem. Kemudian menghadaplah ke kota itu seakan-akan engkaulah yang mengepungnya. Itu akan menjadi lambang bagi bangsa Israel.
Eze 4:4  Lalu berbaringlah miring ke kiri, dan Aku akan menimpakan kesalahan bangsa Israel kepadamu. Selama 390 hari engkau akan berbaring begitu dan menderita sengsara karena kesalahan mereka. Sehari engkau berbaring begitu melambangkan satu tahun hukuman bangsa Israel.
Eze 4:6  Setelah itu, berbaringlah miring ke kanan dan rasakanlah hukuman atas kesalahan Yehuda empat puluh hari lamanya, satu hari melambangkan satu tahun hukuman mereka.
Eze 4:7  Pandanglah pengepungan Yerusalem. Acungkanlah tinjumu kepada kota itu dan sampaikanlah pesan-Ku tentang hukuman bagi mereka.
Eze 4:8  Engkau akan Kuikat sehingga tak dapat membalikkan tubuhmu dari kiri ke kanan atau sebaliknya sampai pengepungan itu berakhir.
Eze 4:9  Sekarang, ambillah gandum, jelai, kacang merah, kacang polong, sekoi dan gandum halus. Campurkanlah itu semua dan buatlah roti. Itulah makananmu selama 390 hari engkau berbaring miring ke kiri itu.
Eze 4:10  Roti itu harus kaubagi sedemikian rupa sehingga setiap hari engkau makan hanya 230 gram. Lebih dari itu tak boleh.
Eze 4:11  Air minummu juga terbatas, yaitu dua cangkir sehari.
Eze 4:12  Buatlah api dengan kotoran manusia sebagai bahan bakarnya. Pangganglah rotimu di atas api itu dan makanlah roti itu di hadapan semua orang."
Eze 4:13  TUHAN berkata, "Begitulah caranya orang Israel kelak memakan makanan yang dilarang oleh hukum Musa, pada waktu mereka Kubuang ke negeri-negeri asing."
Eze 4:14  Tetapi aku menjawab, "Aduh TUHAN Yang Mahatinggi! Sejak kecil sampai sekarang, belum pernah aku menajiskan diriku dengan makanan bangkai binatang atau sisa mangsa binatang buas. Tak pernah aku makan makanan yang dianggap haram."
Eze 4:15  Maka Allah berkata, "Baiklah, sebagai ganti kotoran manusia, engkau boleh memakai kotoran sapi untuk membakar rotimu."
Eze 4:16  Kemudian Ia berkata lagi, "Hai manusia fana, Aku akan memusnahkan persediaan makanan di Yerusalem. Dengan cemas dan khawatir orang-orang akan membatasi makanan yang mereka makan dan air yang mereka minum.
Eze 4:17  Mereka akan berputus asa karena kehabisan roti dan air, dan akhirnya mereka mati karena dosa-dosa mereka."
Eze 5:1  TUHAN berkata, "Hai manusia fana, ambillah pedang yang tajam dan pakailah itu untuk mencukur habis janggut dan rambutmu. Kemudian timbanglah rambut itu pada timbangan lalu bagilah menjadi tiga bagian.
Eze 5:2  Sepertiga harus kaubakar di tengah-tengah kota itu setelah pengepungan berakhir. Sepertiga lagi harus kaubacoki dengan pedangmu sambil mengelilingi kota itu. Dan sepertiga yang terakhir harus kauhamburkan ke udara, lalu Aku akan memburunya dengan pedang-Ku.
Eze 5:3  Tinggalkanlah sedikit dari rambut itu dan bungkuslah dalam lipatan jubahmu.
Eze 5:4  Kemudian ambil sedikit lagi dari rambut itu, lemparkanlah ke dalam api dan biarkan hangus. Dari situ api akan menjalar ke seluruh bangsa Israel."
Eze 5:5  TUHAN Yang Mahatinggi berkata, "Lihat, inilah Yerusalem. Kota ini Kutempatkan di pusat dunia, dengan negara-negara lain di sekelilingnya.
Eze 5:6  Tetapi Yerusalem telah menolak perintah-Ku dan ternyata lebih jahat daripada bangsa-bangsa lain, bahkan lebih tidak taat daripada negara-negara di sekelilingnya. Yerusalem melanggar perintah-perintah-Ku dan tak mau mentaati hukum-hukum-Ku."
Eze 5:7  TUHAN Yang Mahatinggi berkata, "Hai Yerusalem, dengarlah! Engkau tidak mentaati perintah-perintah-Ku dan hukum-hukum-Ku; engkau lebih membangkang daripada bangsa-bangsa lain di sekelilingmu, dan telah mengikuti adat-istiadat bangsa-bangsa itu.
Eze 5:8  Maka Aku, TUHAN Yang Mahatinggi, menyatakan bahwa Aku adalah musuhmu. Engkau akan Kuhukum di hadapan bangsa-bangsa lain.
Eze 5:9  Karena segala perbuatanmu yang Kubenci, pendudukmu akan Kuhukum sehebat yang belum pernah Kulakukan dan tak akan Kulakukan lagi.
Eze 5:10  Orang tua akan memakan anaknya sendiri dan anak akan memakan orang tuanya. Demikianlah mereka akan Kuhukum, dan yang masih hidup Kuceraiberaikan ke segala jurusan.
Eze 5:11  Aku adalah Allah yang hidup, TUHAN Yang Mahatinggi, dan inilah perkataan-Ku: Karena engkau telah mencemarkan Rumah-Ku dengan segala perbuatanmu yang jahat dan menjijikkan, Aku akan menghancurkan engkau tanpa ampun.
Eze 5:12  Sepertiga dari pendudukmu akan mati karena wabah penyakit serta bencana kelaparan yang melanda kota. Sepertiga akan dibantai dengan pedang di luar kota, dan sepertiga lagi akan Kuhamburkan ke segala jurusan dan akan Kukejar-kejar dengan pedang.
Eze 5:13  Engkau akan merasakan kedahsyatan amarah dan amukan-Ku sampai Aku puas. Setelah semua itu terjadi, engkau akan yakin bahwa Aku, TUHAN, telah berbicara kepadamu karena Aku murka atas ketidaksetiaanmu.
Eze 5:14  Engkau akan menjadi puing-puing belaka dan bangsa-bangsa di sekelilingmu akan mengejek engkau; setiap orang yang lewat akan menjauhi engkau.
Eze 5:15  Bila Aku marah dan geram kepadamu serta menghukummu, semua bangsa di sekelilingmu akan merasa ngeri. Mereka akan mengejekmu dan muak melihatmu.
Eze 5:16  Aku akan memusnahkan persediaan makananmu dan membiarkan pendudukmu kelaparan. Mereka akan mengalami pedihnya rasa lapar, seperti anak panah tajam yang menembus dan membinasakan engkau.
Eze 5:17  Selain bencana kelaparan akan Kudatangkan juga wabah penyakit, kekejaman dan perang untuk membunuh pendudukmu, dan binatang-binatang buas untuk membunuh anak-anak mereka. Aku, TUHAN, telah berbicara."
Eze 6:1  TUHAN berkata kepadaku,
Eze 6:2  "Hai manusia fana, pandanglah gunung-gunung Israel, dan sampaikanlah pesan-Ku tentang hukuman bagi mereka.
Eze 6:3  Suruhlah mereka mendengarkan dan memperhatikan apa yang Aku, TUHAN Yang Mahatinggi katakan kepada gunung-gunung itu, juga kepada bukit-bukit, jurang-jurang dan lembah-lembah. Sungguh, Aku akan mengirim pedang untuk menghancurkan tempat-tempat penyembahan berhala.
Eze 6:4  Mezbah-mezbahnya akan dirubuhkan dan pedupaan-pedupaannya akan dipecahkan. Semua orang di situ akan dibunuh di depan berhalanya.
Eze 6:5  Aku akan membuat mayat-mayat orang Israel berserakan; dan tulang-tulang mereka akan Kuhamburkan di sekitar mezbah-mezbah itu.
Eze 6:6  Semua kota Israel akan dimusnahkan, semua mezbah dan berhalanya hancur berantakan, pedupaan-pedupaannya porak-poranda, dan segala apa yang mereka buat akan hilang.
Eze 6:7  Di mana-mana ada orang yang terbunuh, dan mereka yang selamat akan tahu bahwa Akulah TUHAN.
Eze 6:8  Sebagian kecil dari mereka akan Kuluputkan dari pembantaian itu tetapi mereka akan diceraiberaikan di antara bangsa-bangsa;
Eze 6:9  di sana mereka hidup dalam pembuangan. Lalu mereka akan teringat kepada-Ku serta mengerti bahwa Aku menghukum dan menghina mereka sebab hati mereka tidak setia kepada-Ku. Mereka telah meninggalkan Aku dan memilih berhala sebagai ganti-Ku. Lalu mereka akan muak dengan diri mereka sendiri karena segala kejahatan dan kekejian yang telah mereka lakukan.
Eze 6:10  Maka mereka akan tahu bahwa Akulah TUHAN dan bahwa semua peringatan yang Kuberikan bukan ancaman-ancaman kosong belaka."
Eze 6:11  TUHAN Yang Mahatinggi berkata, "Bersedihlah! Merataplah! Menangislah dengan pilu karena segala kejahatan dan kekejian yang telah dilakukan bangsa Israel. Mereka akan mati karena perang, kelaparan dan wabah penyakit.
Eze 6:12  Mereka semua akan merasakan amukan kemarahan-Ku. Orang-orang yang jauh di pembuangan akan jatuh sakit lalu mati; yang dekat akan mati dalam perang, dan sisanya akan mati kelaparan.
Eze 6:13  Mayat-mayat akan berserakan di antara berhala-berhala dan di sekitar mezbah-mezbah, terhampar di setiap bukit yang tinggi, dan di puncak setiap gunung, di bawah setiap pohon yang rimbun dan setiap pohon yang besar; pendek kata, di mana saja mereka membakar kurban bagi berhala mereka. Maka setiap orang akan tahu bahwa Akulah TUHAN.
Eze 6:14  Ya, Aku akan mengacungkan tangan-Ku dan menghancurkan negeri mereka, sehingga menjadi sunyi sepi, mulai dari Gurun Selatan sampai kepada kota Ribla di utara. Dari tempat-tempat yang didiami orang Israel, tak satu pun akan diselamatkan. Maka tahulah semua orang bahwa Akulah TUHAN."
Eze 7:1  TUHAN berbicara kepadaku,
Eze 7:2  "Hai manusia fana, Aku, TUHAN Yang Mahatinggi, hendak mengatakan ini kepada tanah Israel: Inilah akhir riwayat seluruh negeri ini!"
Eze 7:3  TUHAN berkata, "Hai Israel, sekarang sudah tiba kesudahanmu. Engkau akan merasakan murka-Ku. Engkau Kuadili sesuai dengan perbuatanmu yang keji.
Eze 7:4  Aku tidak akan kasihan atau memberi ampun kepadamu. Aku akan menghukum engkau sesuai dengan segala kekejian yang telah kauperbuat. Maka tahulah engkau bahwa Akulah TUHAN."
Eze 7:5  Inilah yang dikatakan TUHAN Yang Mahatinggi kepada bangsa Israel, "Bencana demi bencana akan menimpa kamu.
Eze 7:6  Sekarang segalanya telah berakhir. Inilah kesudahanmu. Tidak ada harapan lagi bagimu.
Eze 7:7  Segalanya telah berakhir bagimu, hai bangsa yang tinggal di tanah ini. Sekarang tidak ada lagi pesta-pesta di mezbah-mezbah pegunungan, yang ada hanyalah kekacauan saja.
Eze 7:8  Tak lama lagi kamu akan merasakan murka-Ku yang dahsyat. Kamu Kuadili sesuai dengan perbuatanmu. Akan Kubalaskan kepadamu segala perbuatanmu yang keji.
Eze 7:9  Aku tidak akan kasihan atau memberi ampun kepadamu. Aku akan menghukum kamu sesuai dengan segala kekejian yang telah kamu perbuat. Maka tahulah kamu bahwa Akulah TUHAN dan Akulah yang menghukum kamu."
Eze 7:10  Hari bencana bagi bangsa Israel telah tiba. Kekerasan merajalela. Kesombongan mencapai puncaknya.
Eze 7:11  Kekejaman mengakibatkan lebih banyak kejahatan. Seluruh kekayaan, keagungan dan kejayaan mereka akan habis tak berbekas.
Eze 7:12  Waktunya segera datang, harinya sudah dekat! Si pembeli dan si penjual tidak mendapat keuntungan apa-apa, sebab hukuman Allah akan menimpa semua orang.
Eze 7:13  Tak ada pedagang yang akan sempat mendapat kembali apa yang telah dijualnya, sebab murka Allah menimpa siapa saja. Orang yang jahat tidak akan selamat.
Eze 7:14  Trompet berbunyi dan semua orang bersiap-siap. Tetapi tak seorang pun berangkat ke medan perang, sebab murka Allah menimpa semua orang.
Eze 7:15  Di jalan-jalan kota ada peperangan, di rumah-rumah ada wabah penyakit dan kelaparan. Siapa saja yang berada di luar kota akan mati dalam peperangan, dan yang tinggal di kota akan menjadi kurban penyakit dan kelaparan.
Eze 7:16  Sebagian akan luput dan melarikan diri ke pegunungan. Di situ mereka akan seperti burung-burung dara yang dihalau dari lembah-lembah; semuanya akan merintih karena dosa-dosa mereka.
Eze 7:17  Setiap orang akan menjadi lemah tangannya dan goyah lututnya.
Eze 7:18  Mereka akan memakai kain karung dan sekujur tubuhnya akan gemetar. Kepala mereka akan digunduli dan mereka dipermalukan.
Eze 7:19  Emas dan perak akan mereka buang di jalan seperti sampah, sebab pada hari TUHAN melampiaskan amarah-Nya, emas dan perak itu tak dapat menyelamatkan mereka, malahan telah menjerumuskan mereka ke dalam dosa. Mereka tak dapat menggunakannya untuk memuaskan nafsu atau untuk mengisi perut mereka.
Eze 7:20  Dahulu mereka bangga akan permata mereka yang indah, tetapi karena mereka memakainya untuk membuat berhala-berhala yang menjijikkan, TUHAN membuat harta mereka itu menjadi sesuatu yang mengerikan.
Eze 7:21  TUHAN berkata kepada bangsa Israel, "Aku akan membiarkan orang-orang asing merampoki kamu. Penjahat-penjahat akan merampas hartamu dan menajiskannya.
Eze 7:22  Aku tak akan turun tangan bila Rumah-Ku dirusakkan dan dicemarkan oleh perampok-perampok.
Eze 7:23  Seluruh negeri menjadi kacau; di mana-mana ada pembunuhan dan kekejaman.
Eze 7:24  Bangsa-bangsa yang paling ganas akan Kudatangkan untuk menduduki rumah-rumahmu. Jagoan-jagoanmu yang paling perkasa akan gentar apabila bangsa-bangsa itu Kubiarkan menghancurkan tempat-tempat ibadahmu.
Eze 7:25  Mereka ditimpa kesedihan dan putus asa, lalu mencari kedamaian, tetapi tidak mereka temukan.
Eze 7:26  Malapetaka dan kabar buruk akan datang terus-menerus. Kamu akan memohon agar para nabi mau meramal, tetapi mereka tak bisa. Juga para imam tidak punya apa-apa untuk diajarkan kepada umat, dan para pemimpin kehabisan nasihat.
Eze 7:27  Raja akan meratap, putra mahkota putus asa dan rakyat akan gemetar ketakutan. Aku akan menghukum kamu sesuai dengan segala perbuatanmu dan menghakimi kamu seperti kamu menghakimi orang lain. Maka tahulah kamu bahwa Akulah TUHAN."
Eze 8:1  Pada tanggal lima bulan enam dalam tahun keenam masa pembuangan kami, para pemimpin orang-orang buangan Yehuda datang ke rumahku dan duduk-duduk bersamaku. Tiba-tiba kekuasaan TUHAN Yang Mahatinggi menguasai aku.
Eze 8:2  Aku menengadah dan melihat sesuatu yang menyerupai manusia. Dari pinggangnya ke bawah ia kelihatan seperti api, dan dari pinggangnya ke atas ia berkilau-kilauan seperti perunggu yang digosok.
Eze 8:3  Ia mengulurkan sesuatu yang menyerupai tangan, lalu menjambak rambutku. Kemudian di dalam penglihatan itu, aku diangkat tinggi ke udara oleh Roh Allah dan dibawa ke Rumah TUHAN di Yerusalem, ke pintu masuk pelataran dalam yang di sebelah utara. Di situ terdapat patung berhala yang telah membangkitkan murka Allah.
Eze 8:4  Lalu aku menyaksikan terang kemilau yang menandakan kehadiran Allah Israel, seperti yang kulihat ketika aku berada di tepi Sungai Kebar.
Eze 8:5  Allah berkata kepadaku, "Hai manusia fana, pandanglah ke arah utara." Aku menurut, dan di sebelah mezbah di dekat jalan masuk, kulihat berhala yang telah membangkitkan murka Allah itu.
Eze 8:6  Allah berkata kepadaku, "Hai manusia fana, kaulihat apa yang sedang terjadi? Perhatikanlah perbuatan-perbuatan menjijikkan yang dilakukan oleh orang Israel, yang menjauhkan Aku dari Rumah-Ku! Mari, akan Kutunjukkan kepadamu hal-hal yang lebih keji lagi."
Eze 8:7  Lalu dibawa-Nya aku ke pintu masuk halaman luar dan diperlihatkan-Nya sebuah lubang di tembok.
Eze 8:8  Kata-Nya, "Hai manusia fana, besarkanlah lubang ini." Setelah kulakukan, kulihat sebuah pintu.
Eze 8:9  Kata-Nya kepadaku, "Masuklah dan lihatlah perbuatan-perbuatan keji dan menjijikkan yang mereka lakukan di situ."
Eze 8:10  Aku pun masuk dan melihat. Dinding-dinding ruangan itu dilukisi dengan segala jenis ular dan binatang-binatang haram lainnya, juga bermacam-macam lukisan berhala yang dipuja oleh orang-orang Israel.
Eze 8:11  Tujuh puluh orang pemimpin bangsa Israel ada di situ, termasuk Yaazanya anak Safan. Masing-masing memegang sebuah pedupaan yang mengepulkan asap.
Eze 8:12  Allah berkata kepadaku, "Hai manusia fana, kaulihat apa yang dilakukan oleh para pemimpin Israel dengan diam-diam? Mereka beribadah di dalam ruangan yang penuh berhala dengan alasan bahwa Aku telah meninggalkan negeri ini dan tidak melihat perbuatan mereka."
Eze 8:13  Lalu TUHAN berkata kepadaku, "Engkau akan melihat mereka melakukan hal yang lebih menjijikkan lagi."
Eze 8:14  Maka dibawa-Nya aku ke gerbang utara Rumah TUHAN dan diperlihatkan-Nya kepadaku perempuan-perempuan yang meratapi kematian Dewa Tamus.
Eze 8:15  TUHAN bertanya kepadaku, "Hai manusia fana, kaulihat hal itu? Masih banyak lagi yang lebih menjijikkan."
Eze 8:16  Dibawa-Nya aku ke pelataran dalam Rumah TUHAN. Di dekat pintu masuk ke ruang suci, di antara mezbah dan lorong, ada dua puluh lima orang. Mereka membelakangi ruang suci, dan sujud ke arah timur menyembah matahari terbit.
Eze 8:17  TUHAN berkata kepadaku, "Hai manusia fana, kaulihat itu? Orang-orang Yehuda ini tak puas bila hanya melakukan hal-hal yang menjijikkan itu dan melakukan tindakan-tindakan yang tak berperikemanusiaan di seluruh negeri. Mereka bahkan melakukan semua itu di sini, di dalam Rumah-Ku dan membuat Aku lebih marah lagi. Lihatlah bagaimana mereka menghina Aku dengan cara yang paling keji ini!
Eze 8:18  Mereka akan merasakan murka-Ku yang dahsyat! Aku tak akan memaafkan dan mengampuni mereka. Meskipun mereka meneriakkan doa kepada-Ku dengan sekeras-kerasnya, Aku tak mau mendengarkan!"
Eze 9:1  Kemudian aku mendengar Allah berseru, "Datanglah ke sini hai orang-orang yang akan menjalankan hukuman-Ku atas kota ini. Bawalah senjata-senjatamu."
Eze 9:2  Serentak muncul enam orang dari sebelah luar gerbang Utara Rumah TUHAN, masing-masing membawa senjata. Bersama dengan mereka ada orang berpakaian linen yang membawa alat tulis. Mereka masuk dan berdiri di dekat mezbah perunggu.
Eze 9:3  Lalu cahaya kemilau yang menandakan kehadiran Allah Israel, memancar dari kerub-kerub, lalu pindah dari tempatnya yang semula itu ke pintu masuk Rumah TUHAN. Maka TUHAN memanggil orang yang berpakaian linen itu, kata-Nya,
Eze 9:4  "Pergilah ke segala penjuru kota dan tandailah dahi setiap orang yang merasa sedih dan susah karena semua perbuatan menjijikkan yang dilakukan di kota itu."
Eze 9:5  Aku mendengar Allah berkata kepada orang-orang lainnya, "Ikutilah dia ke segala penjuru kota dan bunuhlah setiap orang tanpa ampun.
Eze 9:6  Bunuhlah semua orang laki-laki dan perempuan, tua dan muda, orang tua dan anak-anak. Tapi jangan sentuh seorang pun yang ada tanda di dahinya. Mulailah dari sini, dari Rumah-Ku ini." Maka mereka memulai tugas mereka dengan membunuh para pemimpin Israel yang sedang berdiri di Rumah TUHAN.
Eze 9:7  Allah berkata kepada mereka, "Cemarkanlah Rumah-Ku ini! Penuhilah halaman-halamannya dengan mayat-mayat. Mulailah!" Maka mereka pun mulai membunuhi orang-orang di dalam kota.
Eze 9:8  Selama pembantaian itu berlangsung, aku ada di situ sendirian. Aku sujud dan berseru, "TUHAN Yang Mahatinggi, begitu marahkah Engkau kepada Yerusalem sehingga hendak Kaubunuh semua orang yang masih ada di Israel ini?"
Eze 9:9  Allah menjawab, "Orang-orang Israel dan Yehuda telah melakukan dosa yang amat besar. Mereka melakukan pembunuhan di segala tempat, dan Yerusalem dikotori dengan kejahatan. Mereka mengatakan bahwa Aku, TUHAN, telah meninggalkan negeri ini sehingga tak dapat melihat perbuatan mereka.
Eze 9:10  Aku tak akan mengasihani mereka. Akan Kulakukan kepada mereka apa yang mereka lakukan kepada orang lain."
Eze 9:11  Kemudian orang yang berpakaian linen itu kembali dan melapor kepada TUHAN, katanya, "Segala perintah TUHAN sudah terlaksana."
Eze 10:1  Lalu aku memandang ke kubah yang ada di atas kepala kerub-kerub itu, dan kulihat sesuatu yang seperti takhta dari batu nilam.
Eze 10:2  Kemudian Allah berkata kepada orang yang berpakaian linen itu, "Masuklah di antara roda-roda yang ada di bawah kerub-kerub itu, dan ambillah segenggam bara api. Lalu hamburkanlah bara itu ke atas kota." Ia menurut, dan aku memperhatikannya.
Eze 10:3  Ketika ia masuk, kerub-kerub itu sedang berdiri di sebelah selatan Rumah TUHAN. Segumpal awan memenuhi pelataran dalam dan seluruh Rumah TUHAN.
Eze 10:4  Cahaya kemilau yang menandakan kehadiran TUHAN memancar dari kerub-kerub itu dan pindah ke pintu masuk Rumah TUHAN, sehingga pelatarannya menjadi terang-benderang.
Eze 10:5  Bunyi kepakan sayap kerub-kerub itu terdengar sampai ke pelataran luar, seperti suara Allah Yang Mahakuasa.
Eze 10:6  Setelah TUHAN menyuruh orang yang berpakaian linen itu mengambil segenggam bara api dari antara roda-roda di bawah kerub-kerub itu, orang itu masuk dan berdiri di samping salah satu dari roda-roda itu.
Eze 10:7  Lalu salah satu dari kerub itu memasukkan tangannya ke dalam api yang ada di tengah-tengah mereka, memungut beberapa potong bara, dan memberikannya kepada orang yang berpakaian linen itu. Orang itu menerimanya lalu pergi.
Eze 10:8  Aku melihat bahwa di bawah sayap setiap kerub itu ada sesuatu yang berbentuk tangan manusia.
Eze 10:9  Aku melihat juga bahwa kerub-kerub itu masing-masing mempunyai sebuah roda di sampingnya. Keempat roda itu serupa, dan berkilauan seperti permata indah. Masing-masing mempunyai satu roda lainnya yang melintang di tengah-tengahnya.
Eze 10:11  Dengan demikian kerub-kerub itu dapat menuju ke empat jurusan. Mereka dapat pergi ke mana saja mereka suka, tanpa memutar tubuhnya.
Eze 10:12  Seluruh tubuh mereka, punggungnya, tangannya, sayapnya dan roda-roda pun penuh dengan mata.
Eze 10:13  Aku dengar bahwa roda-roda itu disebut "puting beliung".
Eze 10:14  Setiap kerub itu mempunyai empat wajah yang berlainan. Wajah yang pertama ialah wajah banteng, yang kedua wajah manusia, yang ketiga wajah singa, dan yang keempat wajah rajawali.
Eze 10:15  Kerub-kerub itulah yang dahulu kulihat di tepi Sungai Kebar. Setiap kali kerub-kerub itu bergerak atau mengembangkan sayapnya, naik ke udara dan terbang maju atau berhenti, roda-roda itu selalu ikut, karena dikuasai oleh kerub-kerub itu.
Eze 10:18  Lalu cahaya kemilau yang menandakan kehadiran TUHAN meninggalkan pintu masuk Rumah TUHAN dan pindah ke atas kerub-kerub itu.
Eze 10:19  Aku melihat kerub-kerub itu mengembangkan sayapnya dan terbang meninggalkan tanah lalu pergi, dan roda-rodanya ikut. Lalu mereka berhenti di dekat pintu gerbang Rumah TUHAN yang di sebelah timur, sedang cahaya kemilau tetap di atas mereka.
Eze 10:20  Aku teringat bahwa itulah kerub-kerub yang dahulu kulihat ada di bawah Allah Israel, di tepi Sungai Kebar.
Eze 10:21  Kerub-kerub itu masing-masing mempunyai empat wajah, dua pasang sayap, dan juga sesuatu yang seperti tangan manusia di bawah sayap-sayap itu.
Eze 10:22  Wajah mereka serupa pula dengan wajah kerub-kerub yang kulihat di tepi Sungai Kebar. Masing-masing berjalan lurus ke depan.
Eze 11:1  Lalu aku diangkat oleh Roh Allah dan dibawa ke pintu gerbang Timur di Rumah TUHAN. Di situ kulihat 25 orang, di antaranya terdapat dua pemimpin bangsa, yaitu Yaazanya anak Azur dan Pelaca anak Benaya.
Eze 11:2  Allah berkata kepadaku, "Hai manusia fana, inilah orang-orang yang merencanakan kejahatan di kota ini dan memberi nasihat yang menyesatkan.
Eze 11:3  Oleh sebab itu sampaikanlah pesan-Ku tentang hukuman bagi mereka, hai manusia fana. Sebab mereka mengatakan, 'Tak lama lagi rumah-rumah ini akan kita bangun kembali. Kota ini adalah periuk dan kitalah daging di dalamnya. Kita akan aman dari kecamuk api karena dilindungi oleh periuk ini.'"
Eze 11:5  Kemudian Roh TUHAN meliputi aku, dan aku disuruh TUHAN berkata kepada umat Israel, "Hai bangsa Israel, Aku tahu apa yang kamu katakan dan yang kamu rencanakan.
Eze 11:6  Begitu banyak orang yang telah kamu bunuh, sehingga jalan-jalan kota penuh mayat.
Eze 11:7  Sebab itu, Aku, TUHAN Yang Mahatinggi, berkata kepadamu: Memang, kota ini adalah periuk, tetapi daging di dalamnya bukan kamu, melainkan mayat orang-orang yang telah kamu bunuh! Sedangkan kamu akan Kuusir keluar dari kota ini!
Eze 11:8  Takutkah kamu kepada pedang? Akan Kudatangkan orang yang membawa pedang untuk menyerangmu.
Eze 11:9  Kamu akan Kubawa keluar dari kota dan Kuserahkan kepada orang asing. Aku telah menjatuhkan hukuman mati atas kamu.
Eze 11:10  Kamu akan tewas dalam peperangan di negerimu sendiri, maka tahulah kamu bahwa Akulah TUHAN.
Eze 11:11  Jadi, kamu tidak akan aman di dalam kota ini, tidak seperti daging di dalam sebuah periuk. Aku akan mengejar dan menghukum kamu di mana pun kamu berada di negeri Israel.
Eze 11:12  Maka tahulah kamu bahwa Akulah TUHAN dan kamu Kuhukum karena telah melanggar peraturan-peraturan-Ku dan mengabaikan perintah-perintah-Ku dengan mengikuti hukum-hukum bangsa-bangsa di sekelilingmu."
Eze 11:13  Sementara aku mengucapkan perkataan itu, tiba-tiba saja Pelaca jatuh lalu mati. Maka aku pun sujud dan berseru, "Aduh TUHAN Yang Mahatinggi! Apakah seluruh sisa Israel ini hendak Kaubunuh?"
Eze 11:14  TUHAN berkata kepadaku lagi,
Eze 11:15  "Hai manusia fana, engkau dan teman-temanmu dalam pembuangan sedang dipercakapkan oleh penduduk Yerusalem. Kata mereka, 'Orang-orang buangan itu terlalu jauh dari Yerusalem untuk beribadat kepada TUHAN. Tanah Israel sudah diberikan TUHAN kepada kita.'
Eze 11:16  Sebab itu, sampaikanlah pesan-Ku ini kepada teman-temanmu dalam pembuangan, bahwa Akulah yang membawa mereka jauh-jauh ke antara bangsa-bangsa dan menyebarkan mereka di negeri-negeri itu. Namun sementara mereka ada di sana, Aku akan mendampingi mereka.
Eze 11:17  Aku, TUHAN Yang Mahatinggi, menyatakan bahwa Aku akan mengumpulkan mereka dari negeri-negeri di tempat mereka tersebar, lalu mengembalikan tanah Israel kepada mereka.
Eze 11:18  Bila mereka pulang, mereka akan membuang semua berhala yang keji dan menjijikkan yang mereka temukan.
Eze 11:19  Mereka akan Kuberi hati yang baru dan budi yang baru. Hati mereka yang sekeras batu akan Kuambil dan Kuganti dengan hati yang taat kepada-Ku.
Eze 11:20  Maka mereka akan mentaati peraturan-peraturan-Ku dan setia menuruti segala perintah-Ku. Mereka akan menjadi umat-Ku dan Aku menjadi Allah mereka.
Eze 11:21  Tetapi orang-orang yang suka menyembah berhala-berhala yang keji dan menjijikkan akan Kuhukum karena kelakuan mereka yang jahat itu." Demikianlah kata TUHAN Yang Mahatinggi.
Eze 11:22  Kerub-kerub itu mulai terbang dan roda-roda itu ikut juga. Cahaya kemilau yang menandakan kehadiran Allah Israel ada di atas mereka.
Eze 11:23  Kemudian cahaya kemilau itu meninggalkan Yerusalem dan pindah ke gunung di sebelah timur kota itu.
Eze 11:24  Dalam penglihatan itu, Roh Allah mengangkat aku dan mengembalikan aku kepada orang-orang buangan di Babel. Kemudian menghilanglah penglihatan itu.
Eze 11:25  Lalu aku menceritakan kepada orang-orang buangan segala yang diperlihatkan TUHAN kepadaku.
Eze 12:1  TUHAN berbicara lagi kepadaku, kata-Nya,
Eze 12:2  "Hai manusia fana, engkau tinggal di tengah-tengah kaum pemberontak. Mereka punya mata, tetapi tidak melihat; punya telinga, tetapi tidak mendengar, sebab mereka suka memberontak.
Eze 12:3  Sebab itu, hai manusia fana, selagi hari masih siang, kemasi barang-barangmu dalam sebuah bungkusan seperti yang biasanya dilakukan oleh orang buangan. Sebelum malam tiba, tinggalkanlah rumahmu terang-terangan seperti orang yang dibawa ke pembuangan. Buatlah lubang pada dinding rumahmu dengan disaksikan oleh kaum pemberontak itu, dan bawalah bungkusan yang berisi barang-barangmu itu keluar melalui lubang itu. Pikullah bungkusan itu di atas pundakmu. Berangkatlah dalam kegelapan malam; tutupilah matamu sehingga engkau tidak melihat jalan yang kaulalui. Biarlah seluruh bangsa Israel melihat semua yang kaulakukan itu, supaya menjadi peringatan bagi mereka. Barangkali mereka akan insaf bahwa mereka adalah pemberontak."
Eze 12:7  Lalu kulakukan perintah TUHAN kepadaku. Pada hari itu juga, waktu masih siang, kukemasi barang-barangku dalam sebuah bungkusan, seperti yang biasanya dilakukan pengungsi. Ketika hari mulai malam, aku membuat lubang dengan tanganku pada tembok rumahku, lalu keluar melalui lubang itu. Dengan diperhatikan semua orang, bungkusan itu kupikul pada bahuku, kemudian aku berangkat.
Eze 12:8  Keesokan harinya TUHAN berbicara kepadaku, kata-Nya,
Eze 12:9  "Hai manusia fana, bangsa Israel, bangsa pemberontak itu telah menanyakan kepadamu apa yang sedang kaukerjakan.
Eze 12:10  Karena itu sampaikanlah kepada mereka apa yang Aku, TUHAN Yang Mahatinggi katakan kepada mereka. Pesan-Ku itu ditujukan kepada raja di Yerusalem, dan seluruh penduduk kota itu.
Eze 12:11  Terangkanlah bahwa apa yang kaulakukan itu adalah lambang apa yang akan terjadi atas mereka, yaitu: mereka akan menjadi orang buangan dan tawanan.
Eze 12:12  Dalam kegelapan malam raja mereka akan memikul bungkusannya di atas pundaknya lalu berangkat. Orang-orang akan membuat lubang pada tembok supaya ia dapat keluar. Ia akan menutupi matanya sehingga tidak melihat jalan yang dilaluinya.
Eze 12:13  Tetapi Aku akan menebarkan jala-Ku untuk menangkap dia. Lalu akan Kubawa dia ke kota Babel, dan di sana ia akan mati tanpa pernah melihat kota itu.
Eze 12:14  Semua penghuni istana, termasuk para penasihat dan pengawal pribadi raja itu akan Kuceraiberaikan ke segala jurusan. Mereka akan Kukejar dan Kubunuh.
Eze 12:15  Pada waktu mereka Kuceraiberaikan ke tengah-tengah bangsa-bangsa lain di negeri-negeri asing, mereka akan tahu bahwa Akulah TUHAN.
Eze 12:16  Sebagian kecil dari mereka akan Kuselamatkan dari peperangan, dari kelaparan, dan wabah penyakit, supaya di tengah-tengah bangsa-bangsa asing itu mereka akan sadar bahwa mereka telah bertindak keji sekali. Maka mereka akan mengakui bahwa Akulah TUHAN."
Eze 12:17  TUHAN berkata kepadaku,
Eze 12:18  "Hai manusia fana, gemetarlah pada waktu engkau makan, dan menggigillah ketakutan pada waktu engkau minum.
Eze 12:19  Sampaikanlah kepada seluruh bangsa apa yang Aku, TUHAN Yang Mahatinggi katakan kepada penduduk Yerusalem yang masih tinggal di tanah mereka. Mereka akan gemetar pada waktu makan dan menggigil ketakutan pada waktu minum. Tanah mereka akan menjadi tandus, sebab semua orang yang tinggal di situ terus-menerus melanggar hukum.
Eze 12:20  Kota-kota yang sekarang penuh dengan penduduk akan dihancurkan dan ladang-ladang akan menjadi gurun. Maka tahulah mereka bahwa Akulah TUHAN."
Eze 12:21  TUHAN berkata kepadaku,
Eze 12:22  "Hai manusia fana, mengapa orang Israel suka menyebut-nyebut peribahasa ini: 'Masa berganti, tetapi nubuat-nubuat tak kunjung terjadi'?
Eze 12:23  Nah, sampaikanlah kepada mereka bahwa Aku, TUHAN Yang Mahatinggi, memberi jawaban ini: Peribahasa itu akan Kuhapuskan, sehingga tidak diucapkan lagi di Israel. Sebagai gantinya katakan kepada mereka bahwa masanya telah tiba, dan setiap nubuat akan terjadi.
Eze 12:24  Di Israel tak akan ada lagi penglihatan atau ramalan palsu,
Eze 12:25  Aku, TUHAN Yang Mahatinggi akan berbicara kepada bangsa Israel, dan semua yang Kukatakan pasti akan terjadi, tanpa ditunda-tunda lagi. Dengarlah, hai kaum pemberontak! Pada masa hidupmu akan Kulaksanakan segala yang Kuperingatkan kepadamu. Aku, TUHAN Yang Mahatinggi sudah berbicara."
Eze 12:26  Kata TUHAN kepadaku,
Eze 12:27  "Hai manusia fana, orang-orang Israel mengatakan bahwa penglihatan dan nubuatmu itu baru akan terjadi jauh di kemudian hari.
Eze 12:28  Sebab itu, katakanlah kepada mereka bahwa Aku, TUHAN Yang Mahatinggi, mengatakan: Semua yang Kukatakan pasti akan terjadi, tanpa ditunda-tunda lagi. Aku, TUHAN Yang Mahatinggi sudah berbicara!"
Eze 13:1  TUHAN berkata kepadaku,
Eze 13:2  "Hai manusia fana, kecamlah nabi-nabi Israel yang mengucapkan ramalan karangannya sendiri. Suruhlah mereka mendengarkan perkataan-Ku."
Eze 13:3  TUHAN Yang Mahatinggi berkata, "Celakalah nabi-nabi yang bodoh itu! Ilhamnya didapatnya dari khayalannya sendiri dan penglihatannya palsu belaka.
Eze 13:4  Hai Israel, nabi-nabimu itu tak berguna, sama seperti serigala-serigala yang hidup di tengah-tengah puing kota.
Eze 13:5  Mereka tidak menjaga tempat-tempat di mana tembok-temboknya runtuh, dan juga tidak memperbaiki tembok-tembok itu. Jadi, Israel tak dapat dipertahankan apabila pecah perang pada hari Aku, TUHAN datang menghukum.
Eze 13:6  Penglihatan-penglihatan mereka palsu dan ramalan-ramalan mereka hanyalah dusta. Mereka mengaku membawa pesan dari Aku, padahal Aku tak pernah mengutus mereka. Bagaimana mungkin mereka mengharapkan ramalan mereka itu akan terjadi?
Eze 13:7  Aku, TUHAN Yang Mahatinggi berkata kepada mereka, 'Penglihatan-penglihatan itu palsu dan ramalan-ramalanmu itu dusta. Kamu berkata bahwa semua itu adalah pesan-Ku, padahal Aku tak pernah berbicara kepadamu! Karena itu Aku akan menjadi lawanmu.
Eze 13:9  Aku akan menghukum kamu, hai nabi-nabi yang mengucapkan ramalan dusta dan mengkhayalkan penglihatan-penglihatan! Kamu tak boleh lagi ikut memberi keputusan di dalam sidang-sidang umat-Ku. Nama-namamu akan dicoret dari daftar marga Israel, dan kamu tak akan pulang lagi ke tanah airmu. Maka tahulah kamu bahwa Akulah TUHAN Yang Mahatinggi.
Eze 13:10  Hai nabi-nabi palsu! Kamu telah menyesatkan umat-Ku dengan mengatakan bahwa segalanya aman dan beres. Padahal sama sekali tidak! Umat-Ku telah mendirikan tembok dengan batu bata yang longgar pemasangannya, kemudian kamu datang dan mengapurnya.
Eze 13:11  Hai, nabi-nabi palsu, tembokmu akan runtuh. Aku akan menurunkan hujan lebat. Tembok itu akan dilanda angin topan dan ditimpa hujan es
Eze 13:12  sehingga roboh, dan setiap orang akan bertanya apa gunanya tembok itu dikapur.'"
Eze 13:13  TUHAN Yang Mahatinggi berkata, "Dalam kemarahan-Ku, Aku akan mendatangkan angin topan, hujan lebat dan hujan es untuk menghancurkan tembok itu.
Eze 13:14  Aku berniat untuk merubuhkan tembok yang kamu kapur itu, mengobrak-abriknya sehingga tinggal pondamennya saja. Tembok itu akan menimpa dan membunuh kamu semua. Maka tahulah kamu bahwa Akulah TUHAN.
Eze 13:15  Tembok itu dan kamu yang telah mengapurnya akan merasakan hebatnya amarah-Ku. Kemudian akan Kukatakan kepadamu bahwa tembok itu telah hilang lenyap, sama seperti kamu yang mengapurnya,
Eze 13:16  hai nabi-nabi palsu! Kamu meramalkan bahwa Yerusalem aman dan beres, padahal sama sekali tidak!" Begitulah kata TUHAN Yang Mahatinggi.
Eze 13:17  TUHAN berkata, "Sekarang, hai manusia fana, perhatikanlah wanita-wanita yang berlagak nabiah di antara bangsamu. Kecamlah mereka,
Eze 13:18  dan sampaikanlah apa yang Aku, TUHAN Yang Mahatinggi katakan kepada mereka: 'Celakalah kamu! Kamu membuat untuk semua orang gelang-gelang jimat dan kerudung-kerudung sakti supaya orang yang memakainya dapat menguasai hidup orang lain. Kamu ingin berkuasa atas hidup mati umat-Ku, dan kuasa itu kamu pakai demi keuntunganmu sendiri.
Eze 13:19  Kamu telah menghina Aku di depan umat-Ku hanya untuk mendapat beberapa genggam jelai dan beberapa potong roti. Orang-orang yang tak bersalah, kamu bunuh sedangkan orang-orang yang patut mati kamu biarkan hidup. Lalu kamu mendustai umat-Ku dan mereka percaya kepadamu.'"
Eze 13:20  TUHAN Yang Mahatinggi berkata kepada wanita-wanita yang berlagak nabiah itu, "Aku benci melihat gelang-gelang jimat yang kamu pakai untuk menguasai hidup mati umat-Ku. Gelang-gelang itu akan Kurenggut dari pergelanganmu, dan orang-orang yang kamu kuasai akan Kubebaskan.
Eze 13:21  Kerudung-kerudung saktimu akan Kusobek, dan umat-Ku akan Kuhindarkan dari pengaruh jahatmu untuk selama-lamanya. Maka tahulah kamu bahwa Akulah TUHAN.
Eze 13:22  Orang-orang baik yang tidak mau Kuapa-apakan, kamu buat kecil hati dengan kebohonganmu. Orang-orang jahat kamu buat besar hati supaya tetap berdosa dan tidak diselamatkan.
Eze 13:23  Sekarang berakhirlah penglihatan-penglihatan palsu dan ramalan-ramalan dustamu. Kuselamatkan umat-Ku dari pengaruh jahatmu. Maka tahulah kamu bahwa Akulah TUHAN."
Eze 14:1  Beberapa orang dari pemimpin bangsa Israel datang kepadaku untuk menanyakan kehendak TUHAN.
Eze 14:2  Lalu berkatalah TUHAN kepadaku,
Eze 14:3  "Hai manusia fana, orang-orang ini telah mempercayakan dirinya kepada berhala sehingga mereka terjerumus ke dalam dosa. Masakan Aku mau ditanyai oleh mereka?
Eze 14:4  Sebab itu, katakanlah kepada mereka bahwa Aku, TUHAN Yang Mahatinggi berpesan begini, 'Setiap orang Israel yang telah mempercayakan dirinya kepada berhala dan membiarkan dirinya terjerumus ke dalam dosa, lalu datang untuk minta nasihat kepada seorang nabi, akan Kujawab sendiri dengan jawaban yang selayaknya bagi berhala-berhalanya yang banyak itu!
Eze 14:5  Semua berhala itu telah menjauhkan umat-Ku daripada-Ku, tetapi mudah-mudahan dengan jawaban-Ku itu umat-Ku akan kembali kepada-Ku.'
Eze 14:6  Beritahukanlah juga kepada orang Israel bahwa Aku, TUHAN Yang Mahatinggi berkata kepada mereka, 'Bertobatlah dari dosa-dosamu, dan tinggalkanlah berhala-berhalamu yang memuakkan itu.'
Eze 14:7  Setiap orang Israel atau orang asing yang tinggal di antara bangsa Israel, yang menjauhi Aku dan mempercayakan dirinya kepada berhala, lalu minta nasihat kepada seorang nabi, akan Kujawab sendiri.
Eze 14:8  Aku akan melawan dia dan menjadikan dia contoh bagi yang lain. Aku akan mengucilkan dia dari masyarakat umat-Ku, maka tahulah umat-Ku bahwa Akulah TUHAN.
Eze 14:9  Kalau seorang nabi telah ditipu sehingga memberi jawaban yang salah, Akulah yang telah menipunya. Aku, TUHAN Yang Mahatinggi akan mengucilkan dia dari orang-orang Israel.
Eze 14:10  Baik nabi itu maupun mereka yang minta nasihatnya, akan mendapat hukuman yang sama.
Eze 14:11  Semua itu Kulakukan supaya orang Israel jangan lagi menjauhi Aku dan jangan pula menajiskan diri dengan dosa. Dengan demikian mereka akan menjadi umat-Ku, dan Aku menjadi Allah mereka." Aku, TUHAN Yang Mahatinggi telah berbicara.
Eze 14:12  TUHAN berkata kepadaku, "Hai manusia fana,
Eze 14:13  bila suatu negeri berdosa dan tak setia kepada-Ku, Aku akan menghukum negeri itu. Aku akan menghancurkan persediaan makanannya, mendatangkan bencana kelaparan dan membunuh baik manusia maupun binatang.
Eze 14:14  Sekalipun Nuh, Danel dan Ayub tinggal di tempat itu, maka kebaikan mereka bertiga hanya dapat menyelamatkan nyawa mereka sendiri saja. Aku, TUHAN Yang Mahatinggi telah berbicara.
Eze 14:15  Boleh jadi Aku mendatangkan binatang-binatang buas untuk menerkam penduduk negeri itu, sehingga daerah itu menjadi sangat berbahaya dan tak seorang pun berani melewatinya.
Eze 14:16  Sekalipun Nuh, Danel dan Ayub tinggal disitu, maka demi Aku, Allah yang hidup, TUHAN Yang Mahatinggi, mereka tidak akan dapat menyelamatkan siapa saja, bahkan anak mereka sendiri pun tidak. Hanya mereka sendiri saja yang selamat, sedangkan negeri itu akan menjadi gurun.
Eze 14:17  Boleh jadi Aku mendatangkan perang ke negeri itu, dan Kukirimkan senjata-senjata ampuh untuk membinasakan baik manusia maupun binatang.
Eze 14:18  Dan sekalipun ketiga orang tadi tinggal di situ, maka demi Aku, Allah yang hidup, TUHAN Yang Mahatinggi, mereka tidak akan dapat menyelamatkan siapa saja, bahkan anak mereka sendiri pun tidak. Hanya mereka sendiri saja yang selamat.
Eze 14:19  Boleh jadi Aku mendatangkan wabah penyakit ke negeri itu, dan Kuluapkan amarah-Ku terhadapnya sehingga baik manusia maupun binatang akan binasa.
Eze 14:20  Dan sekalipun Nuh, Danel dan Ayub tinggal di situ, maka demi Aku, Allah yang hidup, TUHAN Yang Mahatinggi, mereka tak akan dapat menyelamatkan siapa saja, bahkan anak mereka sendiri pun tidak. Hanya mereka sendiri saja yang selamat karena kebaikan mereka."
Eze 14:21  TUHAN Yang Mahatinggi berkata, "Aku akan menjatuhkan keempat hukuman-Ku yang paling dahsyat ke atas Yerusalem untuk memusnahkan baik manusia maupun binatang. Hukuman itu ialah: perang, bala kelaparan, binatang buas dan wabah penyakit.
Eze 14:22  Seandainya ada yang berhasil lolos dan menyelamatkan anak-anaknya, perhatikanlah mereka apabila mereka datang kepadamu. Lihat bagaimana jahatnya mereka, maka kamu akan sadar bahwa tepatlah tindakan-Ku untuk menghukum Yerusalem.
Eze 14:23  Lalu kamu pun akan mengerti bahwa segala tindakan-Ku itu mempunyai alasan. Aku TUHAN Yang Mahatinggi telah berbicara."
Eze 15:1  TUHAN berkata kepadaku,
Eze 15:2  "Hai manusia fana, apakah kelebihan kayu anggur dari kayu lainnya di hutan? Apakah keunggulan pohon anggur dari pohon-pohon di hutan?
Eze 15:3  Dapatkah kayunya dipakai untuk membuat sesuatu? Dapatkah dijadikan gantungan untuk menyangkutkan perkakas?
Eze 15:4  Tak ada gunanya selain dijadikan kayu bakar. Dan kalau kedua ujungnya habis terbakar, dan bagian tengahnya pun menjadi arang, masihkah ada gunanya?
Eze 15:5  Tidak ada! Bahkan sebelum dibakar, tak dapat dipergunakan. Apalagi sesudah dibakar dan dihanguskan api!"
Eze 15:6  Sebab itu TUHAN Yang Mahatinggi berkata, "Seperti kayu pohon anggur yang diambil dari hutan dan dibakar, begitu juga akan Kuperlakukan penduduk Yerusalem!
Eze 15:7  Aku akan menghukum mereka. Meskipun mereka telah lolos dari api, namun api juga yang akan membinasakan mereka. Maka tahulah kamu bahwa Akulah TUHAN.
Eze 15:8  Mereka tidak setia kepada-Ku, sebab itu, tanah itu Kujadikan gurun. Aku TUHAN Yang Mahatinggi telah berbicara."
Eze 16:1  TUHAN berbicara lagi kepadaku,
Eze 16:2  "Hai manusia fana, tunjukkanlah kepada Yerusalem perbuatan-perbuatan keji yang telah dilakukannya."
Eze 16:3  TUHAN Yang Mahatinggi menyuruh aku menyampaikan pesan ini kepada Yerusalem, "Hai Yerusalem, engkau lahir di tanah Kanaan. Ayahmu orang Amori dan ibumu orang Het.
Eze 16:4  Ketika engkau lahir tak ada yang memotong tali pusatmu, atau memandikan engkau, atau memborehimu dengan garam, atau menyelimutimu dengan kain.
Eze 16:5  Tak seorang pun merasa kasihan kepadamu sehingga mau melakukan hal-hal itu bagimu. Ketika engkau lahir, tak ada yang sayang kepadamu, malahan engkau dibuang di ladang.
Eze 16:6  Kemudian lewatlah Aku dan melihat engkau menggeliat dalam genangan darahmu, tetapi Aku tak membiarkan engkau mati.
Eze 16:7  Kuambil air lalu Kucuci darah dari tubuhmu. Kugosok kulitmu dengan minyak zaitun. Kubesarkan engkau seperti tanaman yang subur. Engkau tumbuh pesat, sehat dan semampai, serta menjadi seorang wanita muda. Buah dadamu mulai montok dan rambutmu panjang, tetapi engkau telanjang. Ketika Aku lewat lagi, Kulihat bahwa telah tiba waktumu untuk bercinta. Kubentangkan jubah-Ku pada tubuhmu yang telanjang, dan Kujanjikan cinta-Ku padamu. Ya, Kuikat janji perkawinan denganmu, dan engkau menjadi milik-Ku. Aku, TUHAN Yang Mahatinggi telah berbicara.
Eze 16:10  Kemudian Kupakaikan kepadamu gaun yang bersulam, sepatu dari kulit yang terhalus, ikat kepala dari kain linen dan jubah sutra.
Eze 16:11  Kurias engkau dengan permata, kalung dan gelang-gelang.
Eze 16:12  Kuberikan kepadamu cincin hidung, anting-anting dan mahkota yang indah.
Eze 16:13  Demikianlah engkau berhiaskan emas dan perak; pakaianmu terbuat dari kain linen dan sutra bersulam. Engkau boleh makan roti dari terigu yang terbaik, dengan madu dan minyak zaitun. Kecantikanmu sungguh luar biasa dan engkau menjadi ratu.
Eze 16:14  Segala bangsa memuji kecantikanmu yang sempurna itu, dan Akulah yang telah membuatmu jelita. Aku TUHAN Yang Mahatinggi telah berbicara.
Eze 16:15  Tetapi engkau menyalahgunakan kecantikanmu serta keharuman namamu. Engkau tidur dengan laki-laki mana pun yang lewat.
Eze 16:16  Sebagian dari pakaianmu kaugunakan untuk menghiasi tempat penyembahan berhalamu. Engkau menjadi pelacur dan memberikan tubuhmu kepada setiap laki-laki.
Eze 16:17  Dari perhiasan emas dan perak yang Kuberikan kepadamu, kaubuat patung-patung orang lelaki, lalu engkau berbuat cabul dengan patung-patung itu.
Eze 16:18  Kaukenakan pada patung-patung itu pakaian bersulam yang telah Kuberikan kepadamu. Kaupersembahkan kepada mereka minyak zaitun dan dupa yang telah kauterima daripada-Ku.
Eze 16:19  Kuberi kepadamu makanan dari terigu yang paling baik, madu dan minyak zaitun yang paling murni. Tetapi semua itu kaupersembahkan kepada berhala-berhala itu untuk menyenangkan hati mereka. Aku TUHAN Yang Mahatinggi telah berbicara.
Eze 16:20  Kemudian kaupersembahkan anak-anak kita, laki-laki dan perempuan, menjadi kurban kepada berhala-berhala. Belum cukupkah bahwa engkau tidak setia kepada-Ku?
Eze 16:21  Haruskah kauambil juga anak-anak-Ku dan mengurbankan mereka kepada berhala-berhala?
Eze 16:22  Dalam hidupmu yang menjijikkan sebagai pelacur itu, sedikit pun engkau tak mengingat masa kecilmu dulu, ketika engkau telanjang dan menggeliat dalam genangan darahmu sendiri."
Eze 16:23  TUHAN Yang Mahatinggi berkata, "Celakalah engkau! Sungguh celaka! Engkau telah melakukan segala kejahatan itu.
Eze 16:24  Lagipula, di alun-alun kota dan di tepi setiap jalan engkau mendirikan tempat-tempat untuk menyembah berhala dan melacur.
Eze 16:25  Engkau mencemarkan kecantikanmu. Tubuhmu kautawarkan kepada setiap laki-laki yang lewat, dan tak henti-hentinya engkau berzinah.
Eze 16:26  Mesir, tetanggamu yang besar nafsunya itu, telah kaukawani tidur. Engkau sengaja melacur untuk membuat Aku marah.
Eze 16:27  Sekarang Aku bertindak untuk menghukum engkau dan mengambil berkat yang telah Kuberikan kepadamu. Kuserahkan engkau kepada orang-orang Filistin yang benci kepadamu dan yang muak melihat kelakuanmu yang kotor itu.
Eze 16:28  Tetapi engkau belum juga merasa puas; sebab itu kaukejar-kejar orang Asyur. Engkau menjadi pelacur mereka, tetapi tidak juga mendapat kepuasan dari mereka.
Eze 16:29  Lalu engkau menjadi pelacur orang Babel, bangsa pedagang itu. Tetapi mereka pun tidak sanggup memberi kepuasan kepadamu."
Eze 16:30  TUHAN Yang Mahatinggi berkata, "Semuanya itu telah kaulakukan seperti pelacur yang tak tahu malu.
Eze 16:31  Di alun-alun kota dan di setiap jalan kaudirikan tempat-tempat untuk menyembah berhala dan melacur. Tetapi engkau bukanlah seperti pelacur biasa yang perlu uang.
Eze 16:32  Engkau seperti istri yang mengkhianati suaminya dan berbuat serong dengan laki-laki lain.
Eze 16:33  Seorang pelacur selalu mendapat bayaran, tetapi sebaliknya engkaulah yang memberi hadiah kepada kekasih-kekasihmu, bahkan kaubujuk mereka dengan uang supaya datang dari mana-mana untuk tidur bersamamu.
Eze 16:34  Engkau pelacur yang lain dari yang lain-lain. Tak ada yang memaksa engkau menjadi pelacur. Engkau tidak dibayar, malahan engkau sendiri yang membayar. Memang, engkau lain dari yang lain-lain."
Eze 16:35  Hai Yerusalem, pelacur, dengarlah apa yang dikatakan TUHAN Yang Mahatinggi, "Engkau telah membuka bajumu, dan sebagai pelacur kauserahkan tubuhmu kepada kekasih-kekasih dan berhala-berhalamu yang menjijikkan itu. Anak-anakmu kaubunuh dan kaukurbankan bagi berhala-berhala itu.
Eze 16:37  Karena itu, semua bekas kekasihmu akan Kukumpulkan, baik yang kausukai maupun yang kaubenci. Mereka akan berdiri mengelilingi engkau dan akan melepaskan seluruh pakaianmu, sehingga mereka dapat memandang tubuhmu yang telanjang.
Eze 16:38  Engkau akan Kuhukum karena perzinahan dan pembunuhan yang telah kaulakukan itu. Dalam kemarahan dan murka-Ku engkau akan Kuhukum mati.
Eze 16:39  Engkau akan Kuserahkan kepada kekasih-kekasihmu itu, dan mereka akan merubuhkan tempat-tempat engkau melacur dan menyembah berhala. Segala pakaian dan perhiasanmu akan mereka rampas, sehingga engkau telanjang.
Eze 16:40  Mereka akan menyuruh orang banyak melempari engkau dengan batu, dan memotong-motong tubuhmu dengan pedang.
Eze 16:41  Mereka akan membakar rumah-rumahmu dan menyuruh para wanita menyaksikan penghukumanmu. Demikianlah akan Kuhentikan perzinahanmu. Engkau tak akan membagi-bagikan hadiah lagi kepada kekasih-kekasihmu.
Eze 16:42  Baru setelah itu kemarahan-Ku akan reda, dan Aku akan menjadi tenang kembali. Aku tak akan marah lagi atau cemburu.
Eze 16:43  Aku menghukum engkau karena engkau tak ingat bagaimana Aku merawat engkau ketika engkau masih muda. Engkau membuat Aku marah dengan segala perbuatanmu yang jahat itu. Mengapa segala kelakuanmu yang menjijikkan itu masih kautambah lagi dengan perzinahan? Aku TUHAN Yang Mahatinggi telah berbicara."
Eze 16:44  TUHAN berkata, "Hai Yerusalem, orang-orang akan mengenakan peribahasa ini kepadamu, 'Seperti ibunya, begitu juga putrinya.'
Eze 16:45  Memang, engkau adalah anak perempuan ibumu. Dia benci kepada suaminya dan kepada anak-anaknya sendiri. Engkau mirip saudara-saudaramu perempuan yang juga membenci suaminya dan anak-anaknya. Engkau dan saudara-saudaramu, yaitu kota-kota di sekitarmu mempunyai ibu orang Het dan ayah orang Amori.
Eze 16:46  Kakakmu adalah Samaria di utara dengan semua desanya. Dan adikmu adalah Sodom di selatan dengan semua desanya.
Eze 16:47  Puaskah engkau mengikuti jejak mereka dan meniru perbuatan mereka yang menjijikkan itu? Belum! Tak berapa lama lagi engkau bahkan lebih bejat lagi daripada mereka.
Eze 16:48  Demi Aku, Allah yang hidup, kata TUHAN Yang Mahatinggi, saudaramu Sodom dengan desa-desanya belum pernah sejahat engkau dengan desa-desamu.
Eze 16:49  Sodom dan desa-desanya sombong karena punya banyak persediaan makanan serta dapat hidup dengan tenang dan damai. Tetapi mereka tidak memperhatikan orang miskin dan orang hina.
Eze 16:50  Mereka sombong dan keras kepala, serta melakukan hal-hal yang Kubenci. Sebab itu mereka Kuhancurkan seperti yang telah kaulihat.
Eze 16:51  Memang Samaria jahat, tetapi dosanya tidak ada separuh dari dosa-dosamu. Kelakuanmu jauh lebih jahat lagi. Saudara-saudaramu perempuan seakan-akan tidak bersalah dibandingkan dengan engkau. Engkau harus merasa malu dan hina!"
Eze 16:53  TUHAN berkata kepada Yerusalem, "Sodom dan Samaria serta desa-desa mereka akan Kujadikan makmur kembali. Setelah itu, barulah engkau Kujadikan makmur juga.
Eze 16:54  Dengan demikian engkau akan malu pada dirimu sendiri, dan kehinaanmu akan menunjukkan kepada saudara-saudaramu betapa baik nasib mereka.
Eze 16:55  Mereka dan desa-desa mereka akan menjadi makmur kembali, juga engkau dan desa-desamu.
Eze 16:56  Bukankah engkau pernah menertawakan Sodom pada waktu engkau masih sombong
Eze 16:57  dan kejahatanmu belum ketahuan? Sekarang engkau sama saja dengan dia; engkau menjadi bahan tertawaan orang Edom, orang Filistin dan tetangga-tetanggamu lainnya yang benci kepadamu.
Eze 16:58  Engkau harus menderita sebagai akibat perbuatanmu yang kotor dan menjijikkan itu. Aku TUHAN telah berbicara."
Eze 16:59  TUHAN Yang Mahatinggi berkata, "Engkau akan Kuperlakukan setimpal dengan perbuatanmu, sebab engkau telah mengingkari sumpahmu serta melanggar perjanjian-Ku dengan engkau.
Eze 16:60  Tetapi Aku akan mengingat perjanjian yang telah Kubuat denganmu di masa mudamu dan membuat perjanjian denganmu yang berlaku untuk selama-lamanya.
Eze 16:61  Engkau akan teringat kepada perbuatan-perbuatanmu dan merasa malu sewaktu saudara-saudaramu Kukembalikan kepadamu. Mereka akan Kujadikan seperti anak-anakmu, meskipun hal itu tidak termasuk dalam perjanjian-Ku denganmu.
Eze 16:62  Aku akan membuat perjanjian yang baru denganmu. Maka tahulah engkau bahwa Akulah TUHAN.
Eze 16:63  Aku akan memaafkan semua kesalahanmu, tetapi engkau tak akan dapat melupakannya dan engkau akan merasa malu untuk mengatakan apa pun. Aku TUHAN Yang Mahatinggi telah berbicara."
Eze 17:1  TUHAN berkata kepadaku,
Eze 17:2  "Hai manusia fana, ceritakan kepada orang-orang Israel perumpamaan,
Eze 17:3  yang Aku, TUHAN Yang Mahatinggi mau sampaikan kepada mereka: Adalah seekor burung rajawali raksasa yang berbulu indah dan bersayap lebar serta kuat. Ia terbang ke gunung Libanon, mematahkan pucuk sebatang pohon cemara,
Eze 17:4  lalu membawanya dan meletakkannya di kota dagang yang ramai.
Eze 17:5  Kemudian diambilnya sebatang tanaman yang masih muda dari tanah Israel, dan ditanamnya di ladang subur, yang tak pernah kekurangan air.
Eze 17:6  Tanaman itu menjadi pohon anggur yang tumbuh rendah tetapi menjalar luas. Dahan-dahannya tertuju kepada rajawali itu, dan akar-akarnya tertancap kuat dalam tanah. Pohon anggur itu rimbun sekali.
Eze 17:7  Sementara itu ada burung rajawali raksasa yang lain; sayapnya lebar dan bulunya tebal. Lalu pohon anggur itu mengarahkan akar-akar dan dahan-dahannya kepada burung rajawali yang baru itu dengan harapan agar burung itu mau memberi kepadanya lebih banyak air daripada yang ada di ladang itu.
Eze 17:8  Padahal pohon anggur itu sudah tertanam di tanah yang subur dengan air yang berlimpah, sehingga ia dapat berdaun rimbun, berbuah lebat, dan menjadi pohon anggur yang baik.
Eze 17:9  Maka, Aku, TUHAN Yang Mahatinggi, bertanya: Dapatkah pohon anggur itu hidup dan berkembang? Tidakkah burung rajawali yang pertama akan mencabut dia dengan akar-akarnya, merenggut buahnya, mematahkan dahan-dahannya dan membiarkan dia layu? Tidak diperlukan banyak tenaga atau bangsa yang perkasa untuk mencabut pohon itu sampai ke akar-akarnya.
Eze 17:10  Memang ia telah tertanam kuat, tetapi dapatkah ia tumbuh terus serta berkembang? Tidakkah ia akan layu bila dihembus oleh angin Timur? Tidakkah ia akan mengering di tempat ia ditanam?"
Eze 17:11  TUHAN berkata kepadaku,
Eze 17:12  "Tanyakan kepada pemberontak-pemberontak itu, apakah mereka mengerti arti perumpamaan itu. Sampaikan kepada mereka bahwa raja Babel telah datang ke Yerusalem, dan mengangkut raja Yehuda serta pembesar-pembesar kota itu ke Babel.
Eze 17:13  Lalu ia memilih salah seorang keluarga raja Yehuda itu, membuat perjanjian dengan dia, dan menyuruh dia bersumpah setia kepadanya. Raja Babel itu menyandera orang-orang penting,
Eze 17:14  supaya kerajaan Yehuda itu tidak akan bangkit kembali, dan untuk menjamin bahwa perjanjian itu sungguh-sungguh ditepati.
Eze 17:15  Tetapi raja Yehuda memberontak serta mengirim utusan ke Mesir untuk mendapat kuda dan tentara yang banyak. Apakah ia akan berhasil? Apakah ia akan dibiarkan lolos? Tidak mungkin dia mengingkari perjanjian tanpa dijatuhi hukuman!
Eze 17:16  Demi Aku, Allah yang hidup, TUHAN Yang Mahatinggi, raja Yehuda akan mati di Babel, karena ia telah mengingkari sumpahnya dan melanggar perjanjiannya dengan raja Babel yang telah mendudukkan dia di atas takhta.
Eze 17:17  Bahkan bala tentara Mesir yang hebat itu, tak sanggup menolong dia dari serbuan orang-orang Babel yang membuat timbunan-timbunan tanah dan menggali parit-parit untuk membunuh banyak orang.
Eze 17:18  Karena ia telah melanggar sumpah dan perjanjiannya, ia tak akan dapat meloloskan diri.
Eze 17:19  Demi Aku, Allah yang hidup, TUHAN Yang Mahatinggi, raja Yehuda akan Kuhukum karena telah melanggar perjanjiannya dengan raja Babel, padahal ia sudah bersumpah atas nama-Ku untuk menepatinya.
Eze 17:20  Aku akan menebarkan jala untuk menangkapnya. Dia akan Kubawa ke Babel dan Kuhukum di sana, karena ia tidak setia kepada-Ku.
Eze 17:21  Tentaranya yang paling baik akan tewas dalam pertempuran, sedangkan yang berhasil lolos akan tercerai-berai ke segala jurusan. Maka tahulah umat-Ku bahwa Aku, TUHAN telah berbicara."
Eze 17:22  Inilah yang dikatakan TUHAN Yang Mahatinggi, "Dari pohon cemara yang tinggi, akan Kupetik pucuknya lalu Kupatahkan sebuah dahannya yang muda; Kutanam dahan itu di gunung yang menjulang ke awan.
Eze 17:23  Di gunung tertinggi di Israel, ia Kutancapkan. Di sana ia akan bertunas, dan menghasilkan buah, lalu menjadi pohon cemara yang megah. Segala jenis unggas akan bersarang di dalamnya dan berlindung di bawah naungannya.
Eze 17:24  Semua pohon di negeri akan faham bahwa Akulah TUHAN. Pohon yang tinggi Kutebang; pohon yang rendah Kubuat tumbuh menjulang. Pohon yang segar Kujadikan kering kerontang; dan pohon yang kering Kuberi tunas dan kembang. Aku, TUHAN, telah berbicara, segala perkataan-Ku akan terlaksana."
Eze 18:1  TUHAN berkata kepadaku,
Eze 18:2  "Mengapa peribahasa ini terus disebut-sebut di negeri Israel? 'Orang tua makan buah anggur yang asam rasanya, tetapi anak-anaklah yang ngilu giginya.'
Eze 18:3  Demi Aku, Allah yang hidup, TUHAN Yang Mahatinggi, peribahasa itu tak akan lagi diucapkan di Israel.
Eze 18:4  Nyawa setiap orang adalah milik-Ku, baik nyawa orang tua maupun nyawa anaknya. Orang yang berdosa, dialah yang akan mati.
Eze 18:5  Misalkan ada orang yang baik, adil dan jujur.
Eze 18:6  Ia tidak menyembah berhala orang Israel atau memakan kurban persembahan di tempat-tempat pemujaan dewa-dewa. Ia tak pernah menggoda istri orang lain atau tidur bersama wanita yang sedang haid.
Eze 18:7  Ia tak pernah menipu atau merampok. Barang-barang jaminan orang yang berhutang kepadanya selalu dikembalikannya. Orang yang kelaparan diberinya makan dan orang yang tidak punya baju, diberinya pakaian.
Eze 18:8  Ia tak pernah memberi pinjaman dengan bunga. Ia tak mau melakukan kejahatan dan dalam setiap pertengkaran ia memberi keputusan yang jujur.
Eze 18:9  Ia mematuhi perintah-Ku dan taat kepada hukum-hukum-Ku. Orang demikianlah yang jujur dan akan tetap hidup. Aku, TUHAN Yang Mahatinggi telah berbicara.
Eze 18:10  Tetapi misalkan orang itu mempunyai anak yang suka merampok, membunuh, dan melakukan banyak kejahatan lainnya
Eze 18:11  yang tidak pernah dilakukan ayahnya. Ia memakan kurban persembahan di tempat-tempat pemujaan dewa-dewa. Ia menggoda dan meniduri istri orang lain.
Eze 18:12  Ia suka menipu orang miskin, merampok, menyita barang jaminan orang yang berhutang kepadanya. Ia pergi ke kuil-kuil orang yang tidak mengenal TUHAN dan menyembah berhala yang menjijikkan.
Eze 18:13  Ia memberi pinjaman dengan bunga. Apakah orang yang begitu akan hidup? Tidak! Ia telah melakukan semua perbuatan yang menjijikkan itu, dan ia akan mati! Kematiannya adalah akibat perbuatannya sendiri.
Eze 18:14  Misalkan orang yang berbuat banyak kejahatan itu mempunyai anak, dan anak itu melihat segala perbuatan ayahnya, tetapi tidak menirunya.
Eze 18:15  Anak itu tak mau menyembah berhala orang Israel, atau memakan kurban persembahan di tempat-tempat pemujaan dewa-dewa. Ia tak pernah menggoda istri orang lain,
Eze 18:16  atau menindas atau merampok siapa pun. Barang-barang jaminan orang yang berhutang kepadanya selalu dikembalikannya. Orang yang kelaparan diberinya makan dan orang yang tak punya baju diberinya pakaian.
Eze 18:17  Ia tak pernah berbuat jahat atau memberi pinjaman dengan bunga. Ia taat kepada hukum-hukum-Ku dan patuh kepada perintah-Ku. Anak itu tidak akan mati karena dosa-dosa ayahnya, melainkan tetap hidup.
Eze 18:18  Sebaliknya, ayahnya yang suka menipu, merampok dan melakukan kejahatan lainnya, akan mati karena dosa-dosa yang dilakukannya sendiri.
Eze 18:19  Tetapi mungkin ada orang yang bertanya mengapa anak itu tidak ikut menanggung akibat dosa-dosa ayahnya. Jawabnya ialah, karena anak itu baik dan jujur. Ia taat kepada hukum-hukum-Ku dan melakukannya dengan setia, maka ia tetap hidup.
Eze 18:20  Orang yang berbuat dosa, dialah yang akan mati. Anak tidak harus menanggung akibat dari kesalahan ayahnya; sebaliknya, ayah pun tidak harus menanggung akibat dari dosa-dosa anaknya. Orang yang baik akan mendapat ganjaran yang baik karena perbuatannya yang baik. Dan orang yang jahat akan menderita akibat dari kejahatannya.
Eze 18:21  Bila seorang yang jahat tidak berbuat dosa lagi, dan mulai mentaati hukum-hukum-Ku, serta melakukan perbuatan yang baik dan benar, ia tidak akan mati.
Eze 18:22  Semua dosanya akan diampuni; ia akan hidup, karena melakukan yang benar.
Eze 18:23  Apakah kaupikir Aku TUHAN Yang Mahatinggi senang kalau orang yang jahat mati? Sama sekali tidak! Aku ingin ia meninggalkan dosa-dosanya supaya ia tetap hidup.
Eze 18:24  Tetapi kalau seorang yang baik tidak berbuat baik lagi, malahan mulai melakukan kejahatan dan segala perbuatan menjijikkan yang biasanya dilakukan oleh orang jahat, apakah ia akan tetap hidup? Tidak! Dari perbuatan-perbuatannya yang baik itu, tidak satu pun diingat lagi. Ia akan mati karena berbuat dosa dan tidak setia kepada-Ku.
Eze 18:25  Barangkali ada yang berkata begini, 'Tindakan TUHAN keliru.' Dengarlah, hai orang Israel! Kelirukah tindakan-Ku? Kamulah yang keliru.
Eze 18:26  Bila orang yang baik tidak berbuat baik lagi, malahan mulai berbuat jahat, ia akan mati, karena kejahatan yang telah dilakukannya.
Eze 18:27  Tetapi bila orang yang jahat tidak berbuat dosa lagi melainkan mulai melakukan perbuatan yang benar dan baik, nyawanya akan selamat.
Eze 18:28  Ia sudah menyadari segala perbuatannya lalu berhenti berbuat dosa. Dengan begitu ia tidak akan mati, melainkan tetap hidup.
Eze 18:29  Dan kamu, hai orang-orang Israel, kamu katakan bahwa tindakan TUHAN itu keliru. Kelirukah tindakan-Ku, hai Israel? Tidak! Kamulah yang keliru!"
Eze 18:30  TUHAN Yang Mahatinggi, berkata, "Hai orang-orang Israel, Aku akan mengadili setiap orang sesuai dengan perbuatannya. Bertobatlah dari dosa-dosamu. Jangan biarkan dosamu menghancurkan dirimu.
Eze 18:31  Hentikan segala kejahatan yang kamu lakukan dan biarlah hati dan pikiranmu menjadi baru. Mengapa kamu mesti mati?
Eze 18:32  Aku tak senang bila seseorang mati. Tinggalkanlah dosa-dosamu supaya kamu tetap hidup. Aku TUHAN Yang Mahatinggi telah berbicara."
Eze 19:1  Allah menyuruh aku menyanyikan ratapan ini untuk dua orang pangeran Israel:
Eze 19:2  Ibumu bagaikan singa betina yang hebat dan luar biasa. Ia membesarkan anak-anaknya di antara singa-singa perkasa.
Eze 19:3  Salah seekor anaknya dididiknya dan diajarinya mencari mangsa. Maka jadilah ia singa dewasa penerkam dan pemakan manusia.
Eze 19:4  Bangsa-bangsa yang mendengar tentang dia berkumpul untuk menangkapnya. Ia dijebak dalam lubang dan ditawan, lalu diseret ke Mesir dengan kaitan.
Eze 19:5  Sia-sia si induk menunggu anaknya; lalu ia tahu bahwa hilanglah harapannya. Maka anaknya yang lain dilatihnya menjadi singa ganas dan perkasa.
Eze 19:6  Anak singa itu menjadi dewasa dan bergaul dengan singa-singa lainnya. Ia belajar mencari mangsa, dan menjadi pemakan orang juga.
Eze 19:7  Benteng-benteng didobraknya, kota-kota dirusaknya. Seluruh penduduk negeri menjadi gentar mendengar aumnya yang bergelegar.
Eze 19:8  Tetapi bangsa-bangsa dari wilayah sekelilingnya menyergapnya dari mana-mana. Mereka tebarkan jala untuk menangkapnya, dan ia pun terjebak dalam perangkap mereka.
Eze 19:9  Ia dimasukkan ke dalam kurungan singa lalu dibawa ke negeri Babel, menghadap raja. Ia dijaga kuat supaya aumnya jangan kedengaran lagi di atas bukit-bukit Israel yang tinggi.
Eze 19:10  Ibumu bagaikan pohon anggur permai yang tumbuh di kebun dekat anak sungai. Ia menjadi rindang dan berbuah sebab air di situ berlimpah-limpah.
Eze 19:11  Cabangnya kuat, tak mudah dipatahkan layak dijadikan tongkat kerajaan. Puncaknya tinggi menjulang menjangkau awan-awan. Ia dikagumi semua orang sebab ia tinggi dan daunnya rindang.
Eze 19:12  Tetapi ia dicabut oleh tangan-tangan yang marah, kemudian dicampakkan ke tanah. Angin Timur bertiup, mengeringkan buahnya yang segar. Dahannya patah, menjadi kering lalu dibakar.
Eze 19:13  Kini ia ditanam di padang pasir, di tanah gersang dan tidak berair.
Eze 19:14  Batang anggur itu dijilat api menjalar; cabang dan buahnya habis terbakar. Dahannya kini tanpa kekuatan; tak mungkin dijadikan tongkat kerajaan. Itulah ratapan yang berulang-ulang dinyanyikan oleh umat.
Eze 20:1  Pada tanggal sepuluh bulan lima, dalam tahun ketujuh masa pembuangan kami, beberapa orang pemimpin Israel datang kepadaku untuk meminta petunjuk TUHAN. Setelah mereka duduk di depanku,
Eze 20:2  aku mendengar TUHAN berkata kepadaku,
Eze 20:3  "Hai manusia fana, bicaralah kepada pemimpin-pemimpin itu dan katakan bahwa Aku, TUHAN Yang Mahatinggi berkata begini, 'Berani sekali kamu minta petunjuk daripada-Ku! Demi Aku, Allah yang hidup, Aku tak sudi dimintai petunjuk olehmu! Aku, TUHAN Yang Mahatinggi telah berbicara.'
Eze 20:4  Hai manusia fana, sudah siapkah engkau menghakimi mereka? Nah, laksanakanlah! Ingatkanlah mereka akan segala kekejian yang telah dilakukan oleh nenek moyang mereka.
Eze 20:5  Sampaikanlah apa yang Kukatakan ini. Aku telah memilih Israel dan mengangkat sumpah bagi mereka. Di Mesir, Aku menyatakan diri kepada mereka serta berkata, 'Akulah TUHAN Allahmu.'
Eze 20:6  Pada waktu itu Aku bersumpah akan membawa mereka keluar dari Mesir menuju ke negeri yang Kupilih untuk mereka. Negeri itu kaya dan subur, tanah yang paling baik di seluruh dunia.
Eze 20:7  Kusuruh mereka membuang patung-patung menjijikkan yang mereka sukai itu, dan Kularang mereka menajiskan diri dengan menyembah berhala-berhala dari Mesir, sebab Akulah TUHAN Allah mereka.
Eze 20:8  Tetapi mereka melawan Aku dan tidak mau mendengarkan Aku. Mereka tidak mau membuang patung-patung yang menjijikkan itu dan juga tidak mau berhenti menyembah berhala-berhala Mesir. Pada waktu mereka masih di sana, Aku sudah siap hendak melepaskan kemarahan-Ku terhadap mereka.
Eze 20:9  Tetapi Aku tidak melakukannya, supaya nama-Ku jangan dihina oleh bangsa-bangsa di tempat mereka berada. Sebab di depan bangsa-bangsa itu, Aku telah mengumumkan bahwa bangsa Israel akan Kubawa keluar dari Mesir.
Eze 20:10  Lalu Kubawa mereka keluar dari Mesir menuju ke padang pasir.
Eze 20:11  Di sana Kuberikan kepada mereka perintah-perintah-Ku dan Kuajarkan hukum-hukum-Ku yang harus mereka taati, sebab orang yang mentaatinya akan tetap hidup.
Eze 20:12  Kusuruh mereka menghormati hari Sabat sebagai tanda perjanjian-Ku dengan mereka, untuk mengingatkan mereka bahwa Aku TUHAN, telah mengkhususkan mereka untuk-Ku.
Eze 20:13  Tetapi di padang pasir itu juga mereka melawan Aku. Mereka melanggar perintah-perintah-Ku, dan menolak hukum-hukum-Ku yang menjamin bahwa orang yang mentaatinya tetap hidup. Mereka sangat mencemarkan hari Sabat. Sebab itu, di padang pasir itu Aku sudah siap hendak melepaskan amukan kemarahan-Ku terhadap mereka dan membinasakan mereka.
Eze 20:14  Tetapi Aku tidak melakukannya demi kehormatan nama-Ku, supaya nama-Ku jangan dihina oleh bangsa-bangsa yang melihat Aku membawa Israel keluar dari Mesir.
Eze 20:15  Di padang pasir itu juga Aku mengancam bahwa bangsa Israel tidak akan Kubawa masuk ke tanah yang sudah disediakan bagi mereka, tanah yang kaya dan subur dan yang paling baik di seluruh dunia.
Eze 20:16  Ancaman itu Kuucapkan oleh karena mereka mengabaikan perintah-perintah-Ku melanggar hukum-hukum-Ku dan menajiskan hari Sabat. Mereka lebih suka menyembah berhala-berhala mereka.
Eze 20:17  Tetapi kemudian Aku kasihan kepada mereka, sehingga mereka tidak jadi Kubunuh di padang pasir itu.
Eze 20:18  Sebaliknya orang-orang muda di antara mereka, Kunasihati begini, 'Janganlah mengikuti peraturan-peraturan yang dibuat oleh nenek moyangmu. Jangan meniru kebiasaan mereka dan jangan juga najiskan dirimu dengan menyembah berhala-berhala mereka.
Eze 20:19  Akulah TUHAN Allahmu. Taatilah hukum-hukum dan perintah-perintah-Ku.
Eze 20:20  Hormatilah hari Sabat supaya menjadi tanda perjanjian antara Aku dan kamu, untuk mengingatkan kamu bahwa Akulah TUHAN Allahmu.'
Eze 20:21  Tetapi angkatan muda itu pun membandel terhadap Aku. Mereka melanggar hukum-hukum-Ku dan mengabaikan perintah-perintah-Ku, yang menjamin bahwa orang yang mentaatinya tetap hidup. Mereka mencemarkan hari Sabat. Sebab itu Aku sudah siap hendak melepaskan amukan kemarahan-Ku terhadap mereka di padang pasir itu dan membunuh mereka semua.
Eze 20:22  Tetapi Aku tidak melakukannya supaya nama-Ku jangan dihina oleh bangsa-bangsa yang melihat Aku membawa Israel keluar dari Mesir.
Eze 20:23  Di padang pasir itu juga Aku bersumpah bahwa Aku akan menceraiberaikan mereka ke seluruh dunia,
Eze 20:24  karena mereka telah mengabaikan perintah-perintah-Ku, melanggar hukum-hukum-Ku, menajiskan hari Sabat, dan tetap menyembah dewa-dewa yang dipuja oleh nenek moyang mereka.
Eze 20:25  Kemudian Kuberikan kepada mereka hukum-hukum yang tidak baik dan perintah-perintah yang tidak dapat memberi hidup.
Eze 20:26  Kubiarkan mereka menajiskan diri dengan kurban-kurban mereka sendiri, bahkan anak-anak lelaki mereka yang sulung mereka kurbankan. Dengan cara itu Aku menghukum mereka supaya mereka tahu bahwa Akulah TUHAN.
Eze 20:27  Nah, manusia fana, sampaikanlah kepada orang Israel, apa yang Aku TUHAN Yang Mahatinggi katakan kepada mereka: Ini contoh lain yang menunjukkan bagaimana nenek moyangmu menghina Aku dan tidak setia kepada-Ku.
Eze 20:28  Aku membawa mereka ke tanah yang telah Kujanjikan kepada mereka. Ketika mereka melihat bukit-bukit yang tinggi dan pohon-pohon yang rindang, mereka mempersembahkan kurban di tempat-tempat itu. Mereka membuat Aku marah karena kurban-kurban bakaran dan persembahan anggur mereka.
Eze 20:29  Aku bertanya kepada mereka, 'Tempat-tempat tinggi apa yang kamu kunjungi itu?' Maka sejak itu tempat-tempat itu disebut 'Tempat-tempat tinggi'.
Eze 20:30  Jadi sampaikanlah kepada orang Israel apa yang Aku, TUHAN Yang Mahatinggi katakan kepada mereka. 'Mengapa kamu melakukan dosa-dosa seperti yang dilakukan nenek moyangmu? Seperti mereka, kamu juga mengikuti berhala-berhala yang menjijikkan itu.
Eze 20:31  Bahkan sampai hari ini pun kamu menajiskan diri dengan membawa persembahan dan membakar anak-anakmu sebagai kurban kepada berhala-berhala. Lalu masih beranikah kamu meminta petunjuk daripada-Ku? Demi Aku, Allah yang hidup, TUHAN Yang Mahatinggi, kamu tidak Kuizinkan meminta petunjuk daripada-Ku.
Eze 20:32  Kamu ingin menyamai bangsa-bangsa lain dan penduduk negeri-negeri lain yang menyembah pohon dan batu. Tetapi keinginanmu itu tidak akan tercapai.'"
Eze 20:33  "Demi Aku, Allah yang hidup, TUHAN Yang Mahatinggi, kamu Kuperingatkan bahwa karena kemarahan-Ku, kamu akan Kuperintah dengan keras dan tegas sehingga kamu merasakan kuasa-Ku. Kamu akan Kukumpulkan dan Kubawa kembali dari segala negeri tempat kamu telah diceraiberaikan.
Eze 20:35  Lalu kamu akan Kubawa ke 'Gurun Bangsa-bangsa'. Di sana kamu akan Kuhakimi dengan berhadapan muka,
Eze 20:36  seperti yang telah Kulakukan terhadap nenek moyangmu di Gurun Sinai. Aku TUHAN Yang Mahatinggi telah berbicara.
Eze 20:37  Kamu akan Kuperlakukan dengan keras supaya kamu mentaati perjanjian-Ku.
Eze 20:38  Para pemberontak dan penjahat akan Kusingkirkan dari tengah-tengahmu. Mereka akan Kubawa keluar dari negeri-negeri tempat mereka sekarang tinggal dan tidak Kuizinkan kembali ke tanah Israel. Maka tahulah kamu bahwa Aku TUHAN."
Eze 20:39  TUHAN Yang Mahatinggi berkata, "Hai orang-orang Israel, sekarang terserah kepadamu. Teruskan saja berbuat dosa dengan menyembah berhala-berhalamu! Tetapi ingatlah bahwa sesudah itu kamu harus mentaati Aku dan tidak lagi mencemarkan nama-Ku yang suci dengan membawa persembahan kepada berhala-berhalamu itu.
Eze 20:40  Sebab di tanah Israel, di puncak gunung-Ku yang suci, kamu semua, bangsa Israel akan menyembah Aku. Di sana Aku akan senang kepadamu dan berkenan menerima kurban-kurbanmu, persembahan-persembahanmu yang paling baik dan pemberian-pemberianmu yang khusus.
Eze 20:41  Setelah kamu Kubawa keluar dari negeri-negeri tempat kamu diceraiberaikan, Aku akan mengumpulkan kamu dan menerima segala kurban bakaranmu. Maka tahulah semua bangsa bahwa Aku ini Allah Yang Mahakudus.
Eze 20:42  Bilamana kamu Kubawa kembali ke tanah Israel, tanah yang Kujanjikan kepada nenek moyangmu, maka tahulah kamu bahwa Akulah TUHAN.
Eze 20:43  Di sana kamu akan teringat akan segala perbuatan menjijikkan yang telah menajiskan dirimu. Maka kamu akan muak mengingat segala kejadian itu.
Eze 20:44  Apabila kamu telah Kuperlakukan begitu demi kehormatan nama-Ku, dan bukan setimpal dengan kelakuanmu yang buruk serta perbuatanmu yang jahat, tahulah kamu orang Israel bahwa Akulah TUHAN. Aku TUHAN Yang Mahatinggi telah berbicara."
Eze 20:45  TUHAN berkata kepadaku, begini,
Eze 20:46  "Hai manusia fana, lihatlah ke sebelah selatan. Tegurlah tanah selatan dan kecamlah hutan di sana.
Eze 20:47  Suruhlah hutan itu mendengarkan apa yang Aku TUHAN Yang Mahatinggi katakan kepadanya, 'Dengarlah! Aku akan menyalakan api yang membakar habis setiap pohon yang ada padamu, baik yang kering maupun yang segar. Api yang menyala-nyala itu tidak dapat dipadamkan. Ia akan menjalar dari selatan ke utara, dan semua orang akan merasakan panasnya.
Eze 20:48  Mereka akan menyadari bahwa Aku, TUHAN telah menyalakan api itu, dan tak seorang pun dapat memadamkannya.'"
Eze 20:49  Tetapi aku Yehezkiel menyatakan keberatanku dan berkata, "TUHAN Yang Mahatinggi, jangan suruh aku mengatakan itu! Semua orang sudah mengeluh sebab aku suka berbicara dengan kiasan sehingga sukar dimengerti."
Eze 21:1  TUHAN berkata kepadaku, kata-Nya,
Eze 21:2  "Hai manusia fana, kutukilah Yerusalem. Kutukilah tempat-tempat ibadat di sana. Peringatkanlah tanah Israel,
Eze 21:3  bahwa Aku, TUHAN, berkata begini, 'Aku ini musuhmu. Aku akan mencabut pedang-Ku dan membunuh kamu semua, yang jahat maupun yang baik.
Eze 21:4  Dengan pedang-Ku akan Kulawan semua orang dari utara sampai ke selatan.
Eze 21:5  Maka semua orang akan tahu bahwa Aku, TUHAN, mencabut pedang-Ku dan tidak akan menyarungkannya kembali.'
Eze 21:6  Hai manusia fana, mengeranglah dan mengaduhlah di depan orang-orang Israel, seolah-olah engkau patah hati dan putus asa.
Eze 21:7  Jika mereka bertanya mengapa engkau mengaduh, katakanlah bahwa engkau sedih karena berita yang akan datang. Berita itu akan membuat hati mereka gentar, tangan mereka lemas, dan lutut mereka gemetar, mereka akan patah semangat. Sungguh saatnya telah tiba, dan semua itu akan terjadi. Aku TUHAN Yang Mahatinggi telah berbicara."
Eze 21:8  TUHAN berkata kepadaku,
Eze 21:9  "Hai manusia fana, meramallah! Beritahukanlah kepada orang-orang bahwa Aku, TUHAN berkata begini, 'Adalah sebuah pedang, pedang yang digosok dan ditajamkan.
Eze 21:10  Ia ditajamkan agar bisa mematikan, dilicinkan agar berkilau-kilauan. Bagi umat-Ku tak ada lagi kegembiraan sebab segala peringatan dan hukuman mereka abaikan.
Eze 21:11  Pedang itu diasah supaya selalu siap dipakai lalu diserahkan kepada seorang pembantai.
Eze 21:12  Hai manusia fana, merataplah dan berseru! Sebab pedang itu siap melawan umat-Ku. Para pemimpin Israel bersama seluruh bangsa akan ditikam pedang itu hingga binasa. Maka tepuklah dadamu dengan pilu,
Eze 21:13  sebab Aku sedang menguji umat-Ku, jika mereka tak mau memperbaiki kelakuannya, semua itu akan terjadi pada mereka.'
Eze 21:14  Nah, manusia fana, meramallah. Tepuklah tanganmu, sebagai tanda bagi pedang itu untuk mulai membantai. Pedang itu sangat menakutkan.
Eze 21:15  Ia membuat bangsa-Ku hilang keberaniannya dan jatuh tersandung. Aku mengancam kota mereka dengan pedang itu yang mengkilap seperti kilat, dan yang siap untuk membunuh.
Eze 21:16  Hai pedang yang tajam, pancunglah ke kanan dan ke kiri, dan ke arah mana saja menurut keinginanmu.
Eze 21:17  Aku akan tepuk tangan juga, lalu kemarahan-Ku akan reda. Aku, TUHAN telah berbicara."
Eze 21:18  TUHAN berbicara kepadaku, kata-Nya,
Eze 21:19  "Hai manusia fana, gambarlah dua jalan yang dapat dilalui oleh raja Babel yang membawa pedangnya. Kedua jalan itu harus berpangkal dari negeri yang sama; jalan yang satu menuju ke Raba, kota Amon; yang satu lagi ke Yerusalem, kota berbenteng di Yehuda. Pasanglah papan penunjuk jalan di mana jalan itu mulai bercabang.
Eze 21:21  Raja Babel berdiri di dekat papan penunjuk jalan itu. Untuk mengetahui jalan mana yang akan dilaluinya, ia meminta petunjuk dari berhala-berhalanya, mengocok panah dan memeriksa hati binatang yang baru dikurbankan.
Eze 21:22  Ternyata panah yang terambil oleh tangan kanannya bertanda Yerusalem! Itu berarti bahwa raja harus menyerang kota Yerusalem, menyerukan pekik pertempuran dan menyuruh tentaranya menyusun alat-alat pendobrak, memasangnya pada pintu-pintu gerbang Yerusalem, menimbun tanah menjadi tembok pengepungan dan menggali parit-parit.
Eze 21:23  Penduduk Yerusalem tidak akan percaya bahwa mereka diserang, karena mereka telah membuat perjanjian dengan Babel. Tetapi ramalan ini akan memperingatkan mereka bahwa mereka akan ditangkap karena dosa-dosa mereka.
Eze 21:24  Aku, TUHAN Yang Mahatinggi berkata kepada mereka, 'Dosa-dosamu sudah ketahuan dan semua orang tahu bahwa kamu bersalah. Kejahatanmu tampak dalam setiap tindakanmu. Sekarang sudah tiba saatnya kamu dihukum dan kamu akan Kuserahkan kepada musuhmu.'
Eze 21:25  Dan kepada raja Israel Aku berkata, 'Hai raja yang jahat dan durhaka! Saat untuk hukumanmu yang terakhir sudah dekat.
Eze 21:26  Aku, TUHAN Yang Mahatinggi sudah berbicara. Lepaskanlah mahkotamu dan bukalah serbanmu. Sekarang, keadaan sudah berubah. Angkatlah kedudukan orang miskin dan berilah mereka kuasa! Turunkan para penguasa dari martabatnya!
Eze 21:27  Kota ini akan Kuhancurkan sama sekali, sehingga di mana-mana ada puing-puing. Tetapi hal itu baru terjadi apabila orang yang Kupilih untuk menghukum kota ini telah datang. Kepada dialah akan Kuserahkan kota ini.'"
Eze 21:28  "Hai manusia fana, meramallah! Beritahukanlah kepada orang Amon yang telah menghina Israel, bahwa Aku, TUHAN Yang Mahatinggi berkata, 'Sebuah pedang sudah siap untuk membantai; pedang itu sudah diasah mengkilap seperti kilat, supaya siap untuk membunuh.
Eze 21:29  Penglihatan yang kamu lihat adalah palsu, dan ramalan yang kamu ucapkan juga bohong dan dusta. Kamu jahat serta durhaka, dan saat untuk hukumanmu yang terakhir sudah dekat. Pedang itu akan memenggal lehermu.
Eze 21:30  Hai orang Amon, sarungkanlah kembali pedangmu! Kamu akan Kuhakimi di tempat kamu diciptakan dan dilahirkan.
Eze 21:31  Aku akan melampiaskan ke atasmu kemarahan-Ku yang menyala-nyala. Dan kamu akan Kuserahkan kepada orang-orang yang kejam yang terlatih untuk membinasakan.
Eze 21:32  Kamu akan habis dimakan api. Darahmu akan tertumpah di negerimu sendiri, dan tak ada seorang pun yang masih ingat kepadamu. Aku TUHAN telah berbicara.'"
Eze 22:1  TUHAN berbicara kepadaku, kata-Nya,
Eze 22:2  "Hai manusia fana, sudah siapkah engkau menghakimi kota yang penuh dengan pembunuh itu. Tunjukkanlah kepadanya segala perbuatannya yang menjijikkan."
Eze 22:3  TUHAN Yang Mahatinggi menyuruh aku menyampaikan pesan ini kepada Yerusalem, "Engkau bersalah karena telah membunuh banyak dari pendudukmu sendiri, dan telah menajiskan dirimu dengan menyembah berhala-berhala buatanmu sendiri. Sebab itu saat penghakimanmu sudah dekat dan riwayatmu segera berakhir. Engkau Kubiarkan menjadi bahan ejekan dan tertawaan bagi bangsa-bangsa dan semua negeri.
Eze 22:5  Engkau diolok-olok oleh negeri-negeri yang jauh dan dekat karena engkau terus-menerus melanggar hukum dan peraturan.
Eze 22:6  Semua pemimpin Israel mengandalkan kekuatannya sendiri dan melakukan pembunuhan.
Eze 22:7  Tak seorang pun dari pendudukmu menghormati orang tuanya. Engkau menipu orang asing dan mengambil untung dari kesusahan janda dan yatim piatu.
Eze 22:8  Tempat-tempat ibadah bagi-Ku kaunodai dan hari Sabat tidak kauindahkan.
Eze 22:9  Di antara pendudukmu ada penghasut-penghasut yang suka memfitnah orang lain supaya dihukum mati. Ada juga yang suka makan daging yang sudah dikurbankan kepada berhala. Ada lagi yang suka berbuat cabul.
Eze 22:10  Ada yang menggoda dan meniduri ibu tirinya, wanita yang sedang haid,
Eze 22:11  istri orang lain atau menantunya atau adiknya dari lain ibu.
Eze 22:12  Ada juga yang menjadi pembunuh bayaran, lintah darat dan pemeras orang-orang sebangsanya. Mereka semua telah melupakan Aku. Aku TUHAN Yang Mahatinggi telah berbicara.
Eze 22:13  Tetapi sekarang Aku akan menghantam para perampok dan pembunuhmu.
Eze 22:14  Masih beranikah atau masih kuatkah engkau untuk mengangkat tanganmu setelah Aku selesai menghukum engkau? Aku, TUHAN Yang Mahatinggi telah berbicara dan Aku akan melaksanakan apa yang telah Kukatakan.
Eze 22:15  Aku akan menceraiberaikan pendudukmu ke semua negeri dan bangsa, dan Aku akan menghentikan perbuatanmu yang jahat itu.
Eze 22:16  Engkau akan dihina oleh bangsa-bangsa lain, maka tahulah engkau bahwa Akulah TUHAN."
Eze 22:17  TUHAN berkata kepadaku,
Eze 22:18  "Hai manusia fana, orang-orang Israel itu tidak berguna bagi-Ku. Mereka ibarat sampah logam campuran, seperti tembaga, timah, besi dan timah hitam yang tersisa setelah perak dimurnikan di dalam perapian.
Eze 22:19  Maka sekarang, Aku, TUHAN Yang Mahatinggi, mengatakan kepada mereka bahwa mereka sama seperti logam yang tidak berguna itu. Sebagaimana bijih perak, perunggu, besi, timah hitam dan timah putih dimasukkan ke dalam perapian untuk dimurnikan, begitu pula mereka akan Kukumpulkan dan Kubawa ke Yerusalem. Kemarahan-Ku yang hebat akan meleburkan mereka seperti api meleburkan bijih-bijih logam.
Eze 22:21  Ya, Aku akan mengumpulkan mereka di Yerusalem, memasang api kemarahan-Ku di bawah mereka, sehingga mereka hancur lebur.
Eze 22:22  Seperti perak dilebur di dalam perapian, begitu pula mereka akan dilebur di Yerusalem. Maka tahulah mereka bahwa Aku, TUHAN, telah melepaskan kemarahan-Ku ke atas mereka."
Eze 22:23  TUHAN berbicara lagi kepadaku, kata-Nya,
Eze 22:24  "Hai manusia fana, katakan kepada orang Israel bahwa tanah mereka najis, dan sebab itu Aku marah dan menghukumnya.
Eze 22:25  Para pemimpinnya seperti singa yang mengaum setelah menerkam mangsanya. Mereka membunuh rakyat, merampas semua harta dan kekayaan yang mereka temukan dan menyebabkan banyak wanita menjadi janda.
Eze 22:26  Para imam Israel melanggar hukum-hukum-Ku dan mencemarkan benda-benda yang dikhususkan untuk-Ku. Mereka tidak membedakan mana yang halal dan mana yang haram. Juga mereka tidak mengajar rakyat untuk membedakan mana yang bersih dan mana yang najis menurut peraturan agama. Mereka tidak menghiraukan hari Sabat. Dengan demikian Aku dihina oleh bangsa Israel.
Eze 22:27  Pejabat-pejabat pemerintahnya seperti serigala yang merobek-robek mangsanya. Mereka melakukan pembunuhan hanya untuk mencari untung.
Eze 22:28  Para nabi menyembunyikan dosa-dosa itu; mereka seperti orang yang mengapuri tembok yang kotor. Penglihatan-penglihatan mereka palsu, begitu juga ramalan-ramalan mereka. Mereka mengaku bahwa pesan yang mereka sampaikan adalah yang mereka terima dari TUHAN Yang Mahatinggi, padahal Aku sama sekali tidak berbicara kepada mereka.
Eze 22:29  Orang-orang kaya menipu dan merampok. Mereka memperlakukan orang miskin dengan sewenang-wenang dan mengambil keuntungan dari orang asing.
Eze 22:30  Di antara orang Israel itu, Aku mencari orang yang dapat mendirikan tembok, atau berdiri di tempat-tempat yang tembok-temboknya runtuh untuk mempertahankan negeri itu apabila Aku hendak menghancurkannya dalam luapan kemarahan-Ku; namun tak seorang pun yang Kutemukan.
Eze 22:31  Sebab itu sekarang Kulepaskan kemarahan-Ku terhadap mereka, dan seperti api Aku akan membinasakan mereka karena segala perbuatan mereka. Aku TUHAN Yang Mahatinggi telah berbicara."
Eze 23:1  TUHAN berbicara kepadaku, kata-Nya,
Eze 23:2  "Hai manusia fana, ada dua wanita, kakak beradik.
Eze 23:3  Pada waktu muda, mereka tinggal di Mesir. Di sana mereka kehilangan keperawanannya, lalu mereka menjadi pelacur.
Eze 23:4  Yang sulung bernama Ohola; ia melambangkan Samaria. Adiknya bernama Oholiba; ia melambangkan Yerusalem. Aku kawin dengan mereka dan mendapat anak dari kedua-duanya.
Eze 23:5  Meskipun Ohola sudah menjadi istri-Ku, ia tetap melacur, dan sangat bergairah melayani kekasih-kekasihnya dari Asyur.
Eze 23:6  Mereka itu adalah tentara yang berseragam merah ungu, para bangsawan dan pejabat-pejabat tinggi; semuanya perwira-perwira pasukan berkuda yang berwajah tampan.
Eze 23:7  Dengan pembesar-pembesar itulah Ohola berzinah. Lagipula karena sangat berahinya, ia menajiskan dirinya dengan menyembah berhala-berhala mereka.
Eze 23:8  Ia terus saja melacur seperti yang dahulu dilakukannya di Mesir, di mana ia kehilangan keperawanannya. Sejak masa kecilnya, ia telah ditiduri dan diperlakukan sebagai pelacur.
Eze 23:9  Sebab itu Kuserahkan dia kepada kekasih-kekasihnya, orang-orang Asyur yang sangat dirindukannya itu.
Eze 23:10  Ia ditelanjangi oleh mereka dengan kasar, lalu dibunuh dengan pedang, dan anak-anaknya dirampas. Di mana-mana nasibnya itu menjadi buah bibir kaum wanita.
Eze 23:11  Oholiba, adiknya melihat kejadian itu, tetapi ia lebih bernafsu melakukan perzinahan daripada Ohola.
Eze 23:12  Oholiba juga sangat menginginkan orang-orang Asyur itu, yaitu pemuda-pemuda tampan dari pasukan berkuda, para bangsawan dan pejabat tinggi, serta tentara yang berseragam merah ungu itu.
Eze 23:13  Aku melihat bahwa Oholiba itu sungguh bejat, sama seperti kakaknya.
Eze 23:14  Makin lama makin bejatlah Oholiba. Ia sangat tertarik kepada gambar-gambar yang terukir pada dinding, dan yang diwarnai dengan cat semerah darah. Gambar-gambar itu melambangkan perwira-perwira Babel dengan ikat pinggang yang indah dan serban yang berjuntai.
Eze 23:16  Ketika Oholiba melihat gambar-gambar itu, berahinya memuncak dan ia mengirim undangan kepada perwira-perwira itu di Babel.
Eze 23:17  Maka datanglah mereka dan tidur dengan dia. Mereka menodai dia sedemikian rupa sehingga akhirnya ia menjadi muak dengan mereka.
Eze 23:18  Setelah itu dengan terang-terangan ia melacur dan mempertontonkan dirinya kepada umum. Aku muak melihat dia, seperti Aku muak terhadap kakaknya.
Eze 23:19  Makin lama makin sering ia berzinah, tak ada bedanya seperti di masa mudanya ketika ia melacur di Mesir.
Eze 23:20  Ia sangat bergairah terhadap laki-laki yang meluap nafsu berahinya, dan yang gejolak berahinya sebesar nafsu kuda jantan.
Eze 23:21  (Hai Oholiba, engkau ingin mengulangi kemesuman yang kaulakukan di masa mudamu, ketika orang Mesir bercumbu-cumbu denganmu.)"
Eze 23:22  TUHAN Yang Mahatinggi berkata, "Hai Oholiba, engkau sudah bosan dengan kekasih-kekasihmu, tetapi Aku akan menghasut mereka supaya mereka marah kepadamu dan mengepungmu.
Eze 23:23  Aku akan mengumpulkan dan membawa kepadamu semua kekasihmu dari Babel, Pekod, Soa, Koa, Asyur, serta orang-orang Kaldea. Kubawa kepadamu pemuda-pemuda tampan yang berpangkat tinggi, bangsawan, pejabat penting dan perwira-perwira pasukan berkuda.
Eze 23:24  Mereka akan menyerangmu dari utara dengan tentara yang besar, dengan kereta-kereta perang dan kereta-kereta pembawa bekal. Dengan dilindungi perisai dan topi-topi baja, mereka akan mengepungmu. Akan Kubiarkan mereka menghakimimu menurut hukum-hukum mereka sendiri.
Eze 23:25  Karena Aku marah kepadamu, Kubiarkan mereka memperlakukanmu dengan kejam. Hidung dan telingamu akan mereka potong dan anak-anakmu akan diambil dan dibakar hidup-hidup.
Eze 23:26  Engkau akan ditelanjangi dan perhiasan-perhiasanmu akan dirampas.
Eze 23:27  Demikianlah Aku akan menghentikan nafsu berahimu dan segala perzinahan yang kaulakukan sejak engkau ada di Mesir. Engkau tak akan lagi berharap kepada berhala mana pun atau teringat kepada Mesir."
Eze 23:28  TUHAN Yang Mahatinggi berkata, "Aku akan menyerahkan engkau kepada orang-orang memuakkan yang kaubenci.
Eze 23:29  Karena mereka benci kepadamu, segala hasil jerih-payahmu akan mereka rampas. Engkau akan ditinggalkan telanjang bulat sebagai pelacur dan menjadi tontonan umum. Nafsu berahi dan kecabulanmulah
Eze 23:30  yang mendatangkan nasib buruk ini kepadamu. Engkau menjadi pelacur bagi bangsa-bangsa, dan menajiskan dirimu dengan berhala-berhala mereka.
Eze 23:31  Engkau mengikuti jejak kakakmu, oleh sebab itu engkau akan Kuhukum dengan hukuman yang sama."
Eze 23:32  TUHAN Yang Mahatinggi berkata, "Isi cangkir kakakmu harus kautelan. Cangkir itu besar dan dalam. Isinya penuh olok-olok dan ejekan, yang harus kautanggung dari setiap orang.
Eze 23:33  Cangkir kakakmu Samaria, berisi ketakutan dan kehancuran. Setelah kauminum seluruh isinya kau akan mabuk dan menderita. Dan dengan pecahan-pecahan cangkir itu engkau akan merobek-robek dadamu. Aku, TUHAN Yang Mahatinggi telah berbicara."
Eze 23:35  TUHAN Yang Mahatinggi berkata, "Karena engkau telah melupakan Aku dan membelakangi Aku, engkau akan menderita akibat nafsu berahimu dan pelacuranmu."
Eze 23:36  TUHAN berkata kepadaku, "Hai manusia fana, sudah siapkah engkau menghakimi Ohola dan Oholiba? Tegurlah mereka mengenai kekejian yang telah mereka perbuat.
Eze 23:37  Mereka telah melakukan perzinahan dan pembunuhan; mereka telah berzinah dengan berhala-berhala dan membunuh anak-anak-Ku yang mereka lahirkan. Anak-anak itu telah mereka persembahkan kepada berhala-berhala mereka.
Eze 23:38  Dan bukan itu saja. Mereka juga menajiskan Rumah-Ku dan hari Sabat yang telah Kutetapkan itu.
Eze 23:39  Pada hari mereka membunuh anak-anak-Ku dan mempersembahkannya kepada berhala-berhala, kedua kakak beradik itu datang ke Rumah-Ku dan mencemarkannya!
Eze 23:40  Mereka bahkan mengirim undangan kepada pria-pria di negeri-negeri yang jauh, dan mereka memenuhi undangan itu. Untuk menyambut tamu-tamu itu, kedua kakak beradik itu mandi, memakai celak mata dan mengenakan perhiasan-perhiasannya.
Eze 23:41  Lalu mereka duduk di atas tempat tidur yang indah. Di depannya ada meja yang penuh dengan hidangan-hidangan, termasuk dupa dan minyak zaitun yang telah Kuberikan kepada mereka.
Eze 23:42  Dari luar terdengar suara gaduh orang banyak yang bersenang-senang. Orang-orang itu telah diundang datang dari padang pasir. Mereka memasukkan gelang pada lengan kakak beradik itu, dan memasang mahkota-mahkota indah pada kepalanya.
Eze 23:43  Lalu pikir-Ku, 'Masih maukah orang-orang itu bermain cinta dengan pelacur-pelacur yang sudah layu karena perzinahan?'
Eze 23:44  Secara bergantian orang-orang mendatangi Ohola dan Oholiba, kedua pelacur yang sudah bejat itu.
Eze 23:45  Tetapi, orang-orang saleh akan mengadili kakak beradik itu atas dasar perzinahan dan pembunuhan, sebab mereka memang bersalah."
Eze 23:46  TUHAN Yang Mahatinggi berkata, "Panggillah segerombolan orang untuk menyiksa dan merampok kedua pelacur itu.
Eze 23:47  Biarlah mereka dilempari dengan batu dan diserang dengan pedang. Biarlah anak-anak mereka dibantai dan rumah-rumah mereka dibakar.
Eze 23:48  Di seluruh negeri akan Kuberantas kebejatan seperti itu. Tindakan-Ku itu akan menjadi peringatan bagi setiap wanita supaya jangan berzinah seperti kedua kakak beradik itu.
Eze 23:49  Hai Ohola dan Oholiba, Aku akan menghukum kamu karena kamu berzinah dan menyembah berhala. Maka tahulah kamu bahwa Akulah TUHAN Yang Mahatinggi."
Eze 24:1  Pada tanggal sepuluh, bulan sepuluh, dalam tahun kesembilan dari masa pembuangan kami, TUHAN berkata kepadaku,
Eze 24:2  "Hai manusia fana, catatlah tanggal hari ini, karena hari ini raja Babel mulai mengepung Yerusalem.
Eze 24:3  Aku, TUHAN, mempunyai sebuah perumpamaan untuk disampaikan kepada umat-Ku yang suka memberontak: Tuangkan air ke dalam panci lalu taruhlah di atas api.
Eze 24:4  Masukkan daging ke dalamnya daging yang paling baik mutunya; daging bahu dan daging paha serta tulang-tulang yang banyak sumsumnya.
Eze 24:5  Pakailah daging domba yang gemuk dan muda tambahlah persediaan kayu bakarnya. Biarlah mendidih dan bergolak airnya rebuslah daging itu beserta tulang-tulangnya."
Eze 24:6  TUHAN Yang Mahatinggi berkata, "Celakalah kota yang penuh dengan pembunuh itu! Kota itu seperti periuk berkarat yang tidak pernah digosok. Daging yang ada di dalamnya dikeluarkan sepotong demi sepotong, sampai tak ada satu pun yang masih tinggal.
Eze 24:7  Di kota itu terjadi pembunuhan, tetapi darah kurbannya tidak tercurah ke tanah, melainkan ke atas batu yang keras, sehingga tak dapat ditimbuni.
Eze 24:8  Darah itu Kubiarkan di situ sehingga dapat dilihat dan menuntut balas."
Eze 24:9  TUHAN Yang Mahatinggi berkata, "Celakalah kota yang penuh dengan pembunuh itu! Aku sendiri yang akan menyusun kayu untuk membakarnya.
Eze 24:10  Tambahlah kayunya! Kipasilah apinya! Rebuslah dagingnya! Buanglah kuahnya. Bakarlah tulang-tulangnya!
Eze 24:11  Sekarang taruhlah periuk perunggu yang kosong itu di atas api dan biarkan sampai merah membara. Sesudah karatnya terbakar, periuk itu akan bersih kembali menurut peraturan agama.
Eze 24:12  Tetapi tidak semua karatnya akan hilang ditelan api.
Eze 24:13  Hai Yerusalem, tindakan-tindakanmu yang bejat telah menajiskan dirimu. Aku telah mencoba membersihkan engkau, namun tidak ada gunanya, sebab engkau tetap najis. Engkau tidak akan bersih kembali sampai Kulepaskan kemarahan-Ku atas engkau.
Eze 24:14  Aku, TUHAN Yang Mahatinggi telah berbicara. Telah tiba saatnya bagi-Ku untuk bertindak. Aku tidak akan melupakan dosamu dan tidak pula mengampuni atau mengasihani engkau. Engkau akan dihukum setimpal dengan segala perbuatanmu. Aku TUHAN Yang Mahatinggi telah berbicara."
Eze 24:15  TUHAN berkata kepadaku,
Eze 24:16  "Hai manusia fana, dengan mendadak akan Kuambil orang yang paling kaucintai. Tetapi engkau tak boleh meratap dan menangis, ataupun membiarkan air matamu berlinang.
Eze 24:17  Jangan sampai sedu sedanmu kedengaran. Janganlah berjalan tanpa tutup kepala, atau tanpa alas kaki, sebagai orang yang berkabung. Jangan pula kaututupi mukamu, dan jangan makan roti perkabungan."
Eze 24:18  Pagi-pagi aku berbicara kepada bangsa itu dan malamnya istriku meninggal. Besoknya aku melakukan apa yang diperintahkan kepadaku.
Eze 24:19  Rakyat bertanya, "Mengapa Bapak berbuat begini? Apa artinya bagi kami?"
Eze 24:20  Lalu kukatakan kepada mereka, "TUHAN menyuruh aku
Eze 24:21  menyampaikan pesan ini kepada kamu: Kamu membanggakan kekuatan Rumah TUHAN. Kamu suka memandangnya dan mengunjunginya, tetapi TUHAN akan mencemarkan Rumah-Nya. Dan anggota-anggota keluargamu yang masih muda, yang tertinggal di Yerusalem, akan dibunuh dalam pertempuran.
Eze 24:22  Lalu kamu akan melakukan seperti yang kulakukan sekarang ini. Kamu tidak akan menutupi mukamu atau makan roti perkabungan.
Eze 24:23  Kamu tidak akan berjalan tanpa tutup kepala atau alas kaki, tidak meratap atau menangis. Kamu akan semakin merana karena dosa-dosamu; kamu akan mengeluh seorang kepada yang lain.
Eze 24:24  Maka aku, Yehezkiel akan menjadi contoh bagimu; kamu akan melakukan seperti yang kulakukan sekarang ini. TUHAN berkata bahwa bilamana hal itu terjadi, kamu akan tahu bahwa Dialah TUHAN Yang Mahatinggi."
Eze 24:25  TUHAN berkata kepadaku, "Nah, manusia fana, akan Kuambil dari mereka Rumah-Ku yang kuat itu, yang telah menjadi kebanggaan dan kesenangan mereka, dan yang mereka pandang dan kunjungi dengan suka hati. Lalu akan Kuambil juga anak-anak mereka.
Eze 24:26  Pada hari Aku melakukan hal itu, seorang yang berhasil luput dari kebinasaan itu akan datang, dan menyampaikan berita itu kepadamu.
Eze 24:27  Pada hari itu juga engkau akan dapat berbicara kembali dan tidak bisu lagi. Lalu engkau akan berbicara dengan orang itu. Demikianlah engkau akan menjadi tanda bagi mereka; maka tahulah mereka bahwa Akulah TUHAN."
Eze 25:1  TUHAN berkata kepadaku,
Eze 25:2  "Hai manusia fana, kutukilah negeri Amon.
Eze 25:3  Suruhlah bangsa itu mendengarkan apa yang Aku, TUHAN Yang Mahatinggi katakan kepada mereka, 'Kamu senang melihat Rumah-Ku dicemarkan dan tanah Israel dirusakkan serta bangsa Yahudi diangkut ke pembuangan.
Eze 25:4  Oleh sebab itu kamu Kubiarkan dikalahkan oleh suku-suku bangsa dari gurun Timur. Mereka akan mendirikan perkemahan-perkemahan di negerimu lalu menetap di situ. Mereka akan memakan buah-buahmu dan minum susu ternakmu.
Eze 25:5  Kota Raba akan Kuubah menjadi tempat menggembala unta, dan seluruh negeri Amon menjadi tempat menggembala domba. Maka tahulah kamu bahwa Akulah TUHAN.'"
Eze 25:6  TUHAN Yang Mahatinggi berkata, "Kamu bertepuk tangan sambil menari gembira. Kamu meremehkan tanah Israel.
Eze 25:7  Oleh sebab itu kamu akan Kuserahkan kepada banyak bangsa yang akan merampok dan merampas segala milikmu. Kamu akan Kubinasakan begitu rupa, sehingga tidak lagi menjadi suatu bangsa dan tidak pula mempunyai negeri sendiri. Maka tahulah kamu bahwa Akulah TUHAN."
Eze 25:8  TUHAN Yang Mahatinggi berkata, "Moab mengatakan bahwa Yehuda itu sama seperti bangsa-bangsa lain.
Eze 25:9  Oleh sebab itu kota-kota yang mempertahankan batas-batas negeri Moab akan Kubiarkan diserang musuh termasuk kota-kota yang paling indah, yaitu: Bet-Yesimot, Baal-Meon dan Kiryataim.
Eze 25:10  Aku akan membiarkan suku-suku bangsa dari gurun timur mengalahkan Moab bersama negeri Amon, sehingga Moab tidak lagi merupakan suatu bangsa.
Eze 25:11  Aku akan menghukum Moab, maka tahulah mereka bahwa Akulah TUHAN."
Eze 25:12  TUHAN Yang Mahatinggi berkata, "Bangsa Edom telah membuat kesalahan besar karena berlaku kejam terhadap Yehuda, sebagai tindakan balas dendam.
Eze 25:13  Sebab itu Aku mengumumkan bahwa Aku akan menghukum Edom dan membunuh setiap manusia dan setiap binatang di sana. Seluruh negeri itu, dari kota Teman sampai ke kota Dedan, Kujadikan sunyi sepi; rakyatnya akan mati dalam pertempuran.
Eze 25:14  Bangsa-Ku Israel akan Kupakai untuk membalas dendam terhadap Edom, dan Edom akan diperlakukan sesuai dengan dahsyatnya kemarahan-Ku. Edom akan tahu bagaimana rasanya menjadi sasaran kemarahan-Ku. Aku TUHAN Yang Mahatinggi telah berbicara."
Eze 25:15  TUHAN Yang Mahatinggi berkata, "Orang-orang Filistin telah membalas dendam dengan kejam terhadap musuh bebuyutan mereka dan membinasakan mereka dengan penuh kebencian.
Eze 25:16  Sebab itu Aku mengumumkan bahwa Aku akan menyerang orang-orang Filistin dan membinasakan mereka. Setiap orang yang masih tersisa di daerah pesisir Filistin akan Kubunuh.
Eze 25:17  Aku akan menghukum mereka dengan sangat berat dan melampiaskan dendam-Ku sampai puas. Bila mereka merasakan kemarahan-Ku, tahulah mereka bahwa Akulah TUHAN."
Eze 26:1  Pada tanggal satu bulan tertentu dalam tahun kesebelas dari masa pembuangan kami, TUHAN berbicara kepadaku, kata-Nya,
Eze 26:2  "Hai manusia fana, penduduk Tirus telah bersorak-sorak sambil mengatakan, 'Yerusalem sudah hancur! Kekuatannya dalam perdagangan sudah hilang! Ia tidak menjadi saingan kita lagi!'"
Eze 26:3  TUHAN Yang Mahatinggi berkata, "Aku ini musuhmu, hai Tirus! Aku akan membawa banyak bangsa untuk menyerangmu, dan mereka akan datang seperti ombak laut.
Eze 26:4  Mereka akan menghancurkan tembok-tembokmu dan merobohkan menara-menaramu. Segala tanah yang ada di sana akan Kusapu sehingga yang tinggal hanyalah sebuah batu tandus di tengah laut.
Eze 26:5  Para nelayan akan menjemur jala mereka di atas batu itu. Aku TUHAN Yang Mahatinggi telah berbicara. Tirus akan menjadi mangsa bagi banyak bangsa.
Eze 26:6  Mereka akan merampok serta membunuh penduduk di kota-kota di tanah daratan Tirus. Maka tahulah penduduk Tirus bahwa Akulah TUHAN."
Eze 26:7  TUHAN Yang Mahatinggi berkata, "Aku akan membawa Raja Nebukadnezar dari Babel untuk menyerang engkau, hai Tirus. Raja yang paling kuat itu akan datang dari utara dengan tentara yang besar, dengan banyak kuda dan kereta-kereta perang serta pasukan berkuda.
Eze 26:8  Para penduduk segala kota di tanah daratan akan dibunuh dalam pertempuran itu. Musuh akan menggali parit-parit pertahanan, menimbun tembok-tembok pengepungan dan menyusun perisai-perisai besar yang mereka pakai sebagai benteng melawan engkau.
Eze 26:9  Mereka akan mendobrak tembok-tembokmu dengan alat-alat pendobrak dan merobohkan menara-menaramu dengan batang-batang besi.
Eze 26:10  Kuda-kuda mereka begitu banyak, sehingga derapnya membuat debu beterbangan yang meliputi engkau. Bunyi gemuruh pasukan berkuda yang menarik kereta perbekalan dan kereta perang akan menggetarkan tembok-tembokmu pada waktu mereka melewati pintu-pintu gerbang dan memasuki kota yang telah menjadi puing-puing itu.
Eze 26:11  Pasukan berkuda itu akan menyerbu di jalan-jalanmu dan membunuh pendudukmu dengan pedang mereka. Tugu-tugumu yang megah dan kuat akan dirobohkan.
Eze 26:12  Musuh-musuhmu itu akan merampas segala kekayaan dan daganganmu. Mereka akan meruntuhkan tembok-tembokmu dan merobohkan rumah-rumahmu yang mewah. Mereka akan mengambil batu-batunya, papan-papan dan tanahnya, lalu melemparkannya ke dalam laut.
Eze 26:13  Aku akan menghentikan segala nyanyianmu dan musik kecapimu.
Eze 26:14  Yang Kutinggalkan hanya sebuah batu yang tandus saja, tempat para nelayan menjemur jala mereka. Kota itu tidak akan dibangun lagi. Aku, TUHAN Yang Mahatinggi telah berbicara."
Eze 26:15  TUHAN Yang Mahatinggi berkata kepada kota Tirus, "Penduduk daerah pantai akan gemetar ketakutan bila mendengar teriak kekalahanmu dan jeritan orang-orang yang sedang dibantai.
Eze 26:16  Melihat nasibmu, raja-raja dari bangsa-bangsa di tepi laut akan bangkit dari takhta, menanggalkan jubah kebesaran dan baju mereka yang bersulam, lalu duduk gemetaran di tanah. Mereka akan sangat ketakutan sehingga tak dapat tenang.
Eze 26:17  Kemudian mereka akan menyanyikan lagu penguburan ini bagimu: Hancurlah sudah kota yang terpuja! Segala kapalnya disapu dari samudra. Penduduknya semula berkuasa di laut membuat orang-orang pantai takut.
Eze 26:18  Kini, pada hari engkau dikalahkan, penduduk pulau-pulau gemetaran. Terkejutlah mereka semua melihat kehancuranmu yang tak terkira."
Eze 26:19  TUHAN Yang Mahatinggi berkata, "Kujadikan engkau sunyi senyap seperti kota yang tidak berpenduduk. Aku akan menutupi engkau dengan air dari dasar samudra.
Eze 26:20  Lalu engkau Kukirim ke dunia orang mati untuk bergabung dengan orang-orang dari zaman purbakala. Kubiarkan engkau menetap di dunia orang mati itu, di tengah-tengah puing-puing abadi, bersama-sama dengan orang mati. Maka tanahmu tak akan didiami lagi, dan engkau tak akan terbilang di antara orang hidup.
Eze 26:21  Engkau Kujadikan contoh yang mengerikan, dan itulah akhir riwayatmu. Orang-orang akan mencari engkau, tetapi engkau tak akan ditemui. Aku TUHAN Yang Mahatinggi telah berbicara."
Eze 27:1  TUHAN Yang Mahatinggi berkata kepadaku, "Hai manusia fana, nyanyikanlah lagu penguburan ini bagi Tirus, kota pesisir yang berdagang dengan penduduk segala pantai: Tirus, dengan sombong engkau berkata, 'Kecantikanku sungguh sempurna!'
Eze 27:4  Bagimu samudra bagaikan rumah. Kau dibangun seperti kapal indah.
Eze 27:5  Dindingmu kayu aras Gunung Hermon tiang layarmu kayu cemara dari Libanon.
Eze 27:6  Dayung-dayungmu dibuat dari pohon-pohon besar di Basan. Geladakmu dibuat dari kayu cemara, dari kepulauan Siprus asalnya, hiasannya tatahan gading berharga.
Eze 27:7  Layar dan tanda pengenalmu dari linen buatan Mesir, dengan aneka sulaman. Halus benar bahan tenda-tendamu, kain dari Siprus berwarna ungu.
Eze 27:8  Pendayungmu dari Sidon dan Arwad. Anak buahmu sendiri pelaut-pelaut yang cakap.
Eze 27:9  Tukang-tukang kayu yang bekerja di kapal adalah para pengrajin dari Gebal. Dari setiap negeri kapal-kapal datang hendak mengadakan hubungan dagang.
Eze 27:10  Prajurit-prajurit dari Persia, Lidia dan Libia masuk tentaramu. Mereka menggantungkan perisai dan topi baja mereka pada tembok-tembok kotamu. Mereka telah membantu mengalahkan musuh-musuh dan menambah kemasyhuranmu.
Eze 27:11  Prajurit-prajurit dari Arwad berjaga di tembok-tembokmu; menara-menaramu dikawal oleh orang-orang dari Gamad. Perisai mereka tergantung berderet-deret pada tembokmu. Merekalah yang melengkapi keindahanmu.
Eze 27:12  Engkau menjual barang-barang daganganmu yang berlimpah-limpah itu di Spanyol, dan sebagai bayarannya engkau menerima perak, besi, timah putih dan timah hitam.
Eze 27:13  Engkau berdagang di Yunani, di Tubal, dan di Mesekh, dan sebagai bayarannya kau terima dari mereka budak-budak dan barang-barang dari perunggu.
Eze 27:14  Engkau menukarkan barang-barangmu dengan kuda kereta, kuda tunggang dan bagal dari Bet-Togarma.
Eze 27:15  Penduduk Rodus berdagang juga dengan engkau dan penduduk daerah pesisir membeli barang-barang daganganmu dengan gading dan kayu hitam.
Eze 27:16  Para pedagang dari negeri Siria membeli barang-barang daganganmu dan hasil-hasilmu yang banyak itu. Mereka membayar dengan permata dari batu zamrud, kain merah ungu, kain bersulam, linen halus, permata dari batu koral dan batu merah delima.
Eze 27:17  Yehuda dan Israel menukarkan barang daganganmu dengan gandum, madu, minyak zaitun dan rempah-rempah.
Eze 27:18  Penduduk Damsyik membeli barang daganganmu dan hasil-hasilmu dan membayarnya dengan anggur dari Helbon, bulu domba dari Sahar, besi tempa dan rempah-rempah.
Eze 27:20  Penduduk Dedan membayar barang-barangmu dengan kain-kain pelana.
Eze 27:21  Orang Arab dan penguasa tanah Kedar membeli barang-barangmu dengan anak domba, domba jantan dan kambing.
Eze 27:22  Untuk membeli barang-barangmu para pedagang dari Syeba dan Raema membawa permata-permata, emas dan rempah-rempah yang paling baik.
Eze 27:23  Kota-kota Haran, Kane, Eden, saudagar-saudagar dari Syeba, kota-kota Asyur dan Kilmad, semuanya berdagang dengan engkau.
Eze 27:24  Mereka menjual kepadamu pakaian mewah, kain merah ungu, kain bersulam, permadani-permadani yang beraneka warna, dan tali-temali yang kuat.
Eze 27:25  Barang daganganmu dibawa dengan iring-iringan kapal dagang yang besar-besar. Engkau bagaikan kapal yang berat muatannya, kapal yang mengarungi samudra.
Eze 27:26  Para pendayung membawamu ke laut lepas. Kau dihantam angin timur yang mengganas.
Eze 27:27  Segala barang dagangan muatanmu, semua pelaut beserta anak buahmu, semua tukang kayu dan saudagarmu, setiap prajurit di kapal itu, mereka semua hilang tenggelam pada waktu engkau karam.
Eze 27:28  Pelaut-pelautmu berteriak karena takut jeritannya bergema di pantai-pantai laut.
Eze 27:29  Kini kapal itu ditinggalkan, semua pelautmu naik ke daratan.
Eze 27:30  Mereka menangisi nasibmu, mengeluh dan meratap dengan pilu. Mereka menaburkan debu di kepala dan berguling dalam abu tanda duka.
Eze 27:31  Demi engkau mereka menggunduli kepala, memakai kain karung penuh dukacita. Mereka menangis pilu dan meratapi nasibmu.
Eze 27:32  Lalu mereka menyanyikan sebuah lagu penguburan, 'Tirus, siapa tandinganmu? Tirus, di laut engkau kini membisu.
Eze 27:33  Sewaktu samudra kauarungi dengan muatan padat berisi, kaupenuhi kebutuhan bangsa-bangsa; raja-raja pun kaujadikan kaya.
Eze 27:34  Kini engkau sudah tenggelam, hancur di dasar lautan. Seluruh anak buahmu dan semua muatan hilang bersama engkau di tengah lautan.'
Eze 27:35  Semua orang yang tinggal di tepi pantai gentar mendengar nasibmu. Raja-raja mereka pun sangat ketakutan; kengerian tampak pada wajah-wajah mereka.
Eze 27:36  Engkau hilang lenyap untuk selama-lamanya. Pedagang-pedagang di seluruh dunia menjadi gempar karena takut akan mengalami nasib seperti itu."
Eze 28:1  TUHAN berkata kepadaku,
Eze 28:2  "Hai manusia fana, sampaikanlah kepada raja Tirus, apa yang Aku TUHAN Yang Mahatinggi katakan kepadanya, 'Dengan sombong engkau menganggap dirimu dewa yang bertakhta di tengah-tengah laut. Boleh saja engkau berlagak sebagai dewa, tetapi ingatlah, engkau hanya manusia biasa.
Eze 28:3  Sangkamu engkau lebih pandai daripada Danel, sehingga tak ada rahasia yang tersembunyi bagimu.
Eze 28:4  Kaupakai hikmat dan pengertianmu untuk mengumpulkan harta, emas dan perak.
Eze 28:5  Memang, engkau pandai berdagang dan sekarang sudah kaya raya, dan kekayaanmu itu membuat engkau sombong!'"
Eze 28:6  TUHAN Yang Mahatinggi berkata, "Hai Tirus, karena engkau menganggap dirimu sama seperti Allah, Aku akan mengirim tentara asing yang sangat kejam untuk menyerang engkau. Mereka akan menghancurkan segala kemegahan yang telah kaucapai berkat hikmat dan pengertianmu.
Eze 28:8  Engkau akan dibunuh dan dikuburkan di dalam laut.
Eze 28:9  Nah, jika tiba saat itu, masih dapatkah engkau mengatakan bahwa engkau Allah? Di hadapan pembunuh-pembunuhmu, nyatalah bahwa engkau hanya manusia belaka.
Eze 28:10  Engkau akan mati seperti orang hina karena dibunuh oleh orang asing yang tidak mengenal Allah. Aku, TUHAN Yang Mahatinggi telah berbicara."
Eze 28:11  TUHAN berkata lagi kepadaku, kata-Nya,
Eze 28:12  "Hai manusia fana, merataplah untuk raja Tirus. Sampaikanlah apa yang Aku, TUHAN Yang Mahatinggi katakan kepadanya, 'Engkau pernah menjadi lambang kesempurnaan; dulu engkau sungguh bijaksana dan tampan!
Eze 28:13  Tempat tinggalmu di Eden, taman Allah. Pakaianmu berhiaskan bermacam-macam permata: batu delima dan intan, topas, batu pirus, batu yakut, ratna, cempaka, batu nilam, zamrud dan batu alkali merah. Perhiasan emasmu pun banyak. Barang-barang itu dibuat untukmu pada hari engkau diciptakan.
Eze 28:14  Malaikat yang menyeramkan Kutempatkan di situ untuk menjaga engkau. Engkau tinggal di atas gunung-Ku yang suci dan berjalan di tengah-tengah batu permata yang gemerlapan.
Eze 28:15  Kelakuanmu sempurna sejak engkau diciptakan sampai engkau mulai berbuat jahat.
Eze 28:16  Engkau sibuk berjual beli, dan itulah yang membawa engkau kepada kekerasan dan dosa. Maka Kuusir engkau dari gunung-Ku yang suci, dan malaikat penjagamu menghalau engkau dari tengah-tengah permata yang gemerlapan itu.
Eze 28:17  Rupamu yang tampan membuat engkau sombong dan kemasyhuran menyebabkan engkau bertindak bodoh. Sebab itu engkau Kubanting ke tanah dan Kujadikan peringatan untuk raja-raja lain.'"
Eze 28:18  TUHAN berkata, "Hai Tirus, engkau mencemarkan tempat-tempat ibadatmu karena kesalahan-kesalahan dan kecurangan waktu berjual beli. Sebab itu Kubakar engkau sehingga menjadi rata dengan tanah. Semua orang yang memandang engkau hanya melihat abu.
Eze 28:19  Engkau telah hilang lenyap untuk selama-lamanya. Semua bangsa yang mengenal engkau menjadi gentar karena takut akan mengalami nasib seperti itu."
Eze 28:20  TUHAN berkata kepadaku,
Eze 28:21  "Hai manusia fana, kutukilah kota Sidon.
Eze 28:22  Sampaikanlah kepada penduduknya bahwa Aku TUHAN Yang Mahatinggi berkata kepada mereka, 'Aku ini musuhmu, hai Sidon. Orang-orang akan memuji Aku karena tindakan-Ku terhadap engkau. Pada waktu Aku menghukum pendudukmu, mereka akan tahu bahwa Akulah TUHAN yang suci.
Eze 28:23  Aku akan mendatangkan wabah penyakit, dan membiarkan darah mengalir di jalan-jalanmu. Engkau akan diserang musuh dari segala jurusan, dan pendudukmu akan dibunuh. Maka tahulah engkau bahwa Akulah TUHAN.'"
Eze 28:24  TUHAN berkata, "Negeri-negeri tetangga yang pernah menghina Israel, tidak lagi merupakan onak dan duri yang menyakiti Israel. Mereka akan sadar bahwa Akulah TUHAN Yang Mahatinggi."
Eze 28:25  TUHAN Yang Mahatinggi berkata, "Aku akan mengumpulkan orang-orang Israel dari negeri-negeri jauh di tempat mereka telah Kuceraiberaikan, dan semua bangsa akan tahu bahwa Aku suci. Bangsa Israel akan tinggal lagi di tanah mereka sendiri, tanah yang telah Kuberikan kepada hamba-Ku Yakub.
Eze 28:26  Mereka akan hidup dengan aman di situ dan membangun rumah-rumah serta membuat kebun-kebun anggur. Aku akan menghukum semua tetangga mereka yang telah menghina mereka, sehingga Israel menjadi aman. Maka tahulah mereka bahwa Akulah TUHAN Allah mereka."
Eze 29:1  Pada tanggal dua belas bulan sepuluh, dalam tahun kesepuluh masa pembuangan kami, TUHAN berkata kepadaku,
Eze 29:2  "Hai manusia fana, kecamlah raja Mesir. Katakanlah kepadanya bahwa dia dan seluruh rakyat Mesir akan dihukum."
Eze 29:3  TUHAN Yang Mahatinggi menyuruh aku menyampaikan kepada raja Mesir pesan ini, "Aku ini musuhmu, hai buaya raksasa yang berbaring di tengah sungai. Engkau membual bahwa Sungai Nil adalah milikmu dan bahwa engkaulah yang membuatnya.
Eze 29:4  Aku akan memasang kaitan pada rahangmu dan melekatkan ikan-ikan pada sisikmu. Lalu engkau Kutarik keluar dari Sungai Nil dengan segala ikan yang melekat pada sisikmu itu.
Eze 29:5  Kemudian engkau dan segala ikan itu Kulemparkan ke padang pasir. Tubuhmu akan terhempas ke tanah dan terkapar di situ. Tak seorang pun akan mengubur engkau. Tubuhmu akan Kuberikan kepada burung-burung dan binatang-binatang lain untuk dimakan.
Eze 29:6  Maka tahulah seluruh rakyat Mesir bahwa Akulah TUHAN." TUHAN berkata, "Hai bangsa Mesir! Orang Israel mengharapkan bantuanmu, tetapi ternyata engkau hanya seperti tongkat yang lemah saja.
Eze 29:7  Ketika mereka bersandar padamu, engkau patah dan menusuk ketiak mereka, sehingga punggung mereka terkilir.
Eze 29:8  Oleh sebab itu Aku, TUHAN Yang Mahatinggi, mengatakan kepadamu, bahwa Aku akan mengirim tentara yang akan menyerang engkau dengan pedang, dan membunuh penduduk serta ternakmu.
Eze 29:9  Mesir akan menjadi tanah yang sunyi sepi. Maka tahulah kamu bahwa Akulah TUHAN." TUHAN berkata, "Hai raja Mesir! Karena engkau mengatakan bahwa Sungai Nil adalah milikmu dan bahwa engkaulah yang membuatnya,
Eze 29:10  maka Aku menjadi musuhmu, dan musuh Sungai Nilmu. Seluruh Mesir akan Kujadikan tanah yang sunyi sepi mulai dari kota Migdol di utara sampai ke kota Aswan di selatan, hingga perbatasan Sudan.
Eze 29:11  Tak ada manusia atau binatang yang akan melaluinya. Selama empat puluh tahun tak akan ada orang yang mendiaminya.
Eze 29:12  Aku akan membuat Mesir sunyi sepi sehingga di antara negeri-negeri sepi lainnya, Mesirlah yang paling sepi. Selama empat puluh tahun kota-kota di Mesir akan merupakan puing-puing yang jauh lebih parah daripada puing-puing kota mana pun juga. Bangsa Mesir akan Kujadikan pengungsi-pengungsi. Mereka akan tersebar ke segala negeri dan tinggal di tengah-tengah bangsa lain."
Eze 29:13  TUHAN Yang Mahatinggi berkata, "Hai bangsa Mesir, sehabis empat puluh tahun itu, kamu akan Kubawa kembali dari negeri-negeri di tempat kamu telah tersebar.
Eze 29:14  Kamu akan Kupulihkan dan Kusuruh tinggal di Mesir bagian selatan, tanah asalmu. Di sana kamu akan membentuk kerajaan yang lemah,
Eze 29:15  yang paling lemah di seluruh dunia. Aku akan membuat kamu sangat lemah sehingga kamu tak akan dapat memerintah bangsa-bangsa lain ataupun menyombongkan diri terhadap mereka.
Eze 29:16  Israel tidak lagi akan meminta bantuan kepadamu. Nasibmu hai bangsa Mesir akan mengingatkan Israel betapa kelirunya mengandalkan Mesir, seperti yang pernah dilakukannya. Maka tahulah Israel bahwa Akulah TUHAN Yang Mahatinggi."
Eze 29:17  Pada tanggal satu bulan satu dalam tahun kedua puluh tujuh masa pembuangan kami, TUHAN berkata kepadaku,
Eze 29:18  "Hai manusia fana, Raja Nebukadnezar dari Babel melancarkan serangan terhadap Tirus. Ia menyuruh prajurit-prajurit memikul beban-beban yang begitu berat, sehingga kepala mereka menjadi botak dan bahu mereka lecet, tetapi baik raja maupun tentaranya tidak mendapat hasil apa pun dengan jerih payah itu.
Eze 29:19  Sebab itu, Aku TUHAN Yang Mahatinggi berkata: Tanah Mesir akan Kuberikan kepada Raja Nebukadnezar. Ia akan merampok dan merampas serta mengangkut semua kekayaan Mesir sebagai upah bagi tentaranya.
Eze 29:20  Tanah Mesir Kuberikan kepadanya sebagai imbalan jasa-jasanya, sebab tentaranya telah bekerja untuk-Ku. Aku, TUHAN Yang Mahatinggi telah berbicara.
Eze 29:21  Bilamana hal itu terjadi, Aku akan menjadikan Israel bangsa yang kuat, dan memungkinkan engkau Yehezkiel, berbicara dengan bebas sehingga semua orang dapat mendengar engkau. Maka tahulah mereka bahwa Akulah TUHAN."
Eze 30:1  TUHAN berbicara lagi, kata-Nya,
Eze 30:2  "Hai manusia fana, meramallah dan sampaikanlah kepada bangsa Mesir apa yang Aku TUHAN Yang Mahatinggi katakan kepada mereka. Berserulah dan merataplah begini: Aduh! Hari TUHAN sudah dekat! Ia akan segera bertindak. Hari itu gelap dan berawan, saat bangsa-bangsa menerima hukuman.
Eze 30:4  Musibah besar akan menimpa Sudan. Di Mesir akan ada peperangan. Kurban-kurban akan berjatuhan; seluruh negeri dirampok dan dijadikan reruntuhan.
Eze 30:5  Dalam pertempuran itu akan terbunuh juga para prajurit sewaan dari Sudan, Lidia, Libia, Arab, Kub dan bahkan dari bangsa-Ku sendiri."
Eze 30:6  TUHAN Yang Mahatinggi berkata, "Dari Migdol di utara sampai ke Aswan di selatan, semua sekutu Mesir akan gugur dalam pertempuran. Dan tentara Mesir yang sombong itu akan hancur berantakan. Aku TUHAN Yang Mahatinggi telah berbicara.
Eze 30:7  Negeri itu akan menjadi negeri yang sunyi sepi di antara negeri-negeri sepi lainnya, dan kota-kotanya akan menjadi puing-puing.
Eze 30:8  Bilamana Aku membakar Mesir dan membunuh semua sekutunya, tahulah mereka bahwa Akulah TUHAN.
Eze 30:9  Bilamana hari itu tiba, dan Mesir telah hancur, Aku akan mengirim utusan-utusan dengan kapal-kapal untuk mengejutkan orang-orang Sudan yang tak curiga itu, maka mereka akan ketakutan. Sungguh, hari itu akan datang!"
Eze 30:10  TUHAN Yang Mahatinggi berkata, "Aku akan memakai Raja Nebukadnezar dari Babel untuk menghancurkan seluruh kekayaan Mesir.
Eze 30:11  Nebukadnezar bersama tentaranya yang tidak kenal ampun akan datang dan menghancurkan tanah itu. Mereka akan menyerang Mesir dengan pedang, dan mayat-mayat akan berserakan di negeri itu.
Eze 30:12  Aku akan mengeringkan Sungai Nil dan menyerahkan seluruh Mesir kepada orang-orang jahat. Orang-orang asing akan Kusuruh memusnahkan seluruh negeri itu. Aku, TUHAN, telah berbicara."
Eze 30:13  TUHAN Yang Mahatinggi berkata, "Aku akan menghancurkan berhala-berhala dan dewa-dewa di Memfis. Tak akan ada lagi raja di Mesir. Seluruh penduduknya akan Kubuat ketakutan.
Eze 30:14  Seluruh Mesir selatan akan Kujadikan sunyi sepi dan kota Soan di utara akan Kubakar. Tebe ibukota itu akan Kuhukum,
Eze 30:15  dan kekayaannya akan Kumusnahkan. Aku akan melepaskan kemarahan-Ku kepada Pelusium, kota yang menjadi benteng Mesir,
Eze 30:16  dan kota itu akan sangat menderita. Sungguh, Aku akan membakar Mesir. Tembok-tembok kota Tebe akan runtuh dan kota itu akan dilanda banjir.
Eze 30:17  Pemuda-pemuda kota Heliopolis dan Bubastis akan tewas dalam pertempuran, dan wanita-wanita akan ditawan.
Eze 30:18  Pada saat Aku mengakhiri kedaulatan Mesir dan mengambil kuasanya yang begitu dibanggakannya, Tahpanhes akan diliputi kegelapan dan wanita-wanitanya akan ditawan.
Eze 30:19  Demikianlah Aku akan menghukum Mesir, maka tahulah mereka bahwa Akulah TUHAN."
Eze 30:20  Pada tanggal tujuh bulan satu dalam tahun kesebelas masa pembuangan kami, TUHAN berkata kepadaku,
Eze 30:21  "Hai manusia fana, Aku telah mematahkan lengan raja Mesir, dan tak ada yang membalut lengannya itu atau mengurutnya supaya sembuh dan menjadi cukup kuat untuk memegang pedang lagi.
Eze 30:22  Aku TUHAN Yang Mahatinggi berkata: Aku ini musuh raja Mesir, dan kedua lengannya akan Kupatahkan, baik yang masih kuat maupun yang sudah patah. Maka pedang itu akan jatuh dari tangannya.
Eze 30:23  Aku akan menceraiberaikan orang Mesir ke seluruh dunia.
Eze 30:24  Lalu akan Kukuatkan lengan raja Babel, dan Kutaruh pedang-Ku dalam tangannya. Tetapi lengan raja Mesir akan Kupatahkan sehingga ia akan mengerang dan mati di hadapan raja Babel, musuhnya.
Eze 30:25  Sungguh, raja Mesir akan Kulemahkan, sedangkan raja Babel akan Kukuatkan. Bilamana Kuberikan pedang-Ku kepadanya, dan ia mengacungkannya ke arah Mesir, mereka akan tahu bahwa Akulah TUHAN.
Eze 30:26  Aku akan menyebarkan orang-orang Mesir ke seluruh dunia. Maka tahulah mereka bahwa Akulah TUHAN."
Eze 31:1  Pada tanggal satu bulan tiga dalam tahun kesebelas masa pembuangan kami, TUHAN berkata kepadaku,
Eze 31:2  "Hai manusia fana, sampaikanlah pesan-Ku ini kepada raja Mesir dan seluruh bangsanya: Sungguh hebat kebesaranmu! Tak ada yang dapat menjadi sainganmu!
Eze 31:3  Engkau seperti cemara Libanon yang gagah penuh cabang yang rimbun dan indah. Batangnya lurus dan lempang puncaknya mencapai awan.
Eze 31:4  Ia tumbuh sebab airnya berkecukupan. Sungai di bawah tanah memberinya makanan, menyirami segala akarnya, dan membasahi pohon-pohon lainnya.
Eze 31:5  Karena airnya tak pernah kurang, tumbuhnya subur, dahannya besar dan panjang. Dari segala pohon di hutan-hutan, dialah yang paling tinggi menjulang.
Eze 31:6  Aneka burung bersarang di rantingnya. Binatang liar beranak dalam naungannya. Semua bangsa besar di seluruh dunia duduk berlindung di bawahnya.
Eze 31:7  Pohon itu indah untuk dipandang; batangnya lurus, dahannya panjang. Akarnya menembus jauh ke bawah mencapai sumber air di dalam tanah.
Eze 31:8  Pohon-pohon cemara di Eden, taman Allah tak dapat disamakan dengan dia. Tak ada cemara yang dapat menyainginya, tak ada pohon yang begitu panjang dahannya. Sungguh, di taman Allah tak ada pohon seindah dia.
Eze 31:9  Akulah yang membuat dia rupawan dengan dahan-dahannya yang panjang dan rindang. Segala pohon di taman Allah merasa cemburu kepadanya.
Eze 31:10  Sebab itu, Aku, TUHAN Yang Mahatinggi, memberitahukan kepadamu apa yang akan terjadi pada pohon yang puncaknya sampai ke awan itu. Sebab semakin tinggi, semakin sombonglah dia.
Eze 31:11  Maka Aku menolak dia dan akan menyerahkannya kepada penguasa asing, yang akan memperlakukan dia setimpal dengan kejahatannya.
Eze 31:12  Tentara asing yang tidak mengenal ampun akan menebang dia lalu membiarkannya. Rantingnya yang telah patah akan berjatuhan di semua gunung dan lembah di negeri itu. Semua bangsa yang tinggal di bawahnya akan pergi meninggalkan dia.
Eze 31:13  Burung-burung akan datang dan hinggap pada pohon yang sudah rebah itu, dan binatang-binatang liar akan menginjak-injak dahan-dahannya.
Eze 31:14  Semua itu terjadi supaya mulai saat ini, tak akan ada lagi pohon yang tumbuh setinggi itu; tak ada lagi puncak pohon yang mencapai awan meskipun mendapat air berlimpah-limpah. Semuanya harus mati seperti manusia yang fana, dan berkumpul dengan orang-orang yang telah turun ke dunia orang mati."
Eze 31:15  Beginilah kata TUHAN Yang Mahatinggi, "Bilamana pohon itu turun ke dunia orang mati, Aku menyuruh air di bawah tanah menggenanginya sebagai tanda kabung. Aku akan menahan arus sungai dan menyumbat anak-anak sungai. Oleh karena pohon itu telah mati, Aku akan mendatangkan kegelapan atas gunung-gunung Libanon dan membuat layu semua pohon di hutan.
Eze 31:16  Pada waktu pohon itu Kuturunkan ke dunia orang mati, bunyi jatuhnya akan menggetarkan bangsa-bangsa. Semua pohon di Eden dan pohon-pohon pilihan di Libanon yang mendapat air berlimpah-limpah, dan yang telah turun ke bumi yang paling bawah akan senang dengan jatuhnya pohon itu.
Eze 31:17  Mereka akan mengiringinya ke dunia orang mati dan berkumpul dengan mereka yang telah jatuh lebih dahulu. Dan semuanya yang dahulu duduk di bawah pohon itu akan disebarkan di antara bangsa-bangsa.
Eze 31:18  Pohon itu melambangkan raja Mesir dan bangsanya. Tak ada pohon di Eden yang begitu tinggi dan mulia. Tetapi sekarang, seperti pohon-pohon di Eden juga, pohon itu pun akan turun ke dunia orang mati dan berkumpul dengan mereka yang tidak mengenal Allah dan dengan mereka yang tewas dalam pertempuran. Aku, TUHAN Yang Mahatinggi telah berbicara."
Eze 32:1  Pada tanggal satu bulan dua belas dalam tahun kedua belas masa pembuangan kami, TUHAN berkata kepadaku,
Eze 32:2  "Hai manusia fana, berilah peringatan keras kepada raja Mesir. Sampaikanlah pesan-Ku ini kepadanya, 'Engkau berlagak seperti singa muda di antara bangsa-bangsa, tetapi engkau lebih menyerupai buaya yang berkecimpung di dalam air. Kakimu mengeruhkan dan mengotorkan segala sungai.
Eze 32:3  Pada waktu bangsa-bangsa berkumpul, Aku akan menangkap engkau dengan jala-Ku, dan membiarkan bangsa-bangsa itu menyeret engkau ke darat.
Eze 32:4  Engkau akan Kuhempaskan ke tanah dan semua burung serta binatang dari seluruh dunia akan Kusuruh memakan engkau.
Eze 32:5  Lalu gunung-gunung dan lembah-lembah akan Kupenuhi dengan bangkaimu yang membusuk itu.
Eze 32:6  Darahmu akan Kutumpahkan sehingga membasahi gunung-gunung dan memenuhi sungai-sungai.
Eze 32:7  Pada saat Aku membinasakan engkau, langit akan Kuselubungi dan bintang-bintang akan Kuredupkan. Matahari akan Kusembunyikan di balik awan dan bulan tidak akan bercahaya.
Eze 32:8  Sungguh, segala terang di langit akan Kupadamkan, dan bumi akan Kujadikan gelap gulita. Aku, TUHAN Yang Mahatinggi telah berbicara.
Eze 32:9  Pada waktu Aku menyiarkan berita tentang kehancuranmu di negeri-negeri yang belum kaukenal, banyak bangsa akan menjadi cemas.
Eze 32:10  Sungguh, Aku akan mengayunkan pedang-Ku dan mengalahkan engkau. Raja-raja dan bangsa-bangsa akan melihat tindakan-Ku itu terhadap engkau. Pada saat itu mereka akan gemetar ketakutan dan menggigil karena takut mati.'"
Eze 32:11  TUHAN Yang Mahatinggi berkata kepada raja Mesir, "Pedang raja Babel akan diacungkan kepadamu.
Eze 32:12  Prajurit-prajurit dari bangsa-bangsa yang kejam akan Kusuruh mencabut pedang mereka dan membantai seluruh rakyatmu. Rakyatmu dan segala apa yang kaubanggakan akan hancur lebur.
Eze 32:13  Ternakmu akan Kubantai di tepi sungai-sungai tempat mereka minum. Baik manusia maupun hewan tak akan mengeruhkan air sungai itu lagi.
Eze 32:14  Maka sungai-sungai di negerimu akan Kujernihkan dan Kubuat mengalir dengan tenang. Aku, TUHAN Yang Mahatinggi telah berbicara.
Eze 32:15  Bilamana Mesir Kujadikan padang gurun yang sepi, serta Kubinasakan semua yang tinggal di situ, mereka akan tahu bahwa Akulah TUHAN.
Eze 32:16  Peringatan yang keras ini akan menjadi nyanyian kabung. Kaum wanita dari bangsa-bangsa lain akan menyanyikannya sebagai tanda berkabung bagi Mesir dan seluruh rakyatnya. Aku, TUHAN Yang Mahatinggi telah berbicara."
Eze 32:17  Pada tanggal lima belas bulan satu dalam tahun kedua belas masa pembuangan kami, TUHAN berkata lagi kepadaku,
Eze 32:18  "Hai manusia fana, berkabunglah bagi rakyat di Mesir. Kirimlah mereka ke dunia orang mati bersama-sama dengan negeri-negeri kuat lainnya.
Eze 32:19  Katakanlah kepada mereka, 'Sangkamu kamukah yang paling jelita? Kamu akan turun ke dunia orang mati, dan dibaringkan di antara orang-orang yang tidak mengenal Allah.'
Eze 32:20  Bangsa Mesir akan mati bersama dengan mereka yang tewas dalam pertempuran. Sebuah pedang sudah siap untuk membunuh mereka semua.
Eze 32:21  Pahlawan-pahlawan yang gagah berani dan orang-orang yang berperang di pihak Mesir menyambut dengan gembira kedatangan orang Mesir ke dunia orang mati. Mereka bersorak begini, 'Orang-orang yang tidak mengenal Allah itu telah tewas dalam pertempuran. Sekarang mereka datang ke mari dan berbaring di sini!'
Eze 32:22  Asyur berbaring di sana, dikelilingi oleh kuburan prajurit-prajuritnya. Mereka semua telah tewas dalam pertempuran. Kuburan mereka ada di bagian yang paling dalam di dunia orang mati. Ketika masih hidup, mereka menimbulkan ketakutan di dunia.
Eze 32:24  Elam ada di sana dengan kuburan prajuritnya di sekelilingnya. Tanpa disunat, mereka semua tewas dalam pertempuran, lalu turun ke dunia orang mati. Ketika hidup, mereka menimbulkan ketakutan, tetapi sekarang mereka telah mati dan harus menanggung malu bersama orang-orang yang telah tewas dalam pertempuran.
Eze 32:26  Mesekh dan Tubal ada di situ, dikelilingi oleh kuburan prajurit-prajurit mereka. Tanpa disunat, mereka semua tewas dalam pertempuran. Ketika hidup, mereka menimbulkan ketakutan di dunia.
Eze 32:27  Mereka tidak mendapat penguburan secara terhormat seperti pahlawan di zaman dahulu yang turun ke dunia orang mati dengan bersenjata lengkap. Pahlawan-pahlawan itu dikuburkan dengan pedang di bawah kepala dan perisai di atas dada. Waktu masih hidup, pahlawan-pahlawan sungguh perkasa sehingga membuat takut semua orang.
Eze 32:28  Demikianlah orang Mesir akan dikuburkan dan dibaringkan di tengah orang-orang yang tidak mengenal Allah dan tewas dalam pertempuran.
Eze 32:29  Edom ada di sana dengan semua raja dan penguasanya. Dahulu mereka adalah prajurit-prajurit yang kuat, tetapi sekarang mereka terbaring di dunia orang mati bersama dengan orang-orang mati lainnya yang tidak mengenal Allah.
Eze 32:30  Semua pangeran dari utara ada di sana, demikian juga orang-orang Sidon. Dahulu kekuasaan mereka menimbulkan ketakutan, tetapi sekarang mereka dipermalukan dan tanpa disunat turun ke dunia orang mati bersama orang-orang yang tewas dalam pertempuran. Di sana mereka berbaring dan menanggung noda bersama orang-orang yang turun ke dunia orang mati.
Eze 32:31  Bilamana raja Mesir dan tentaranya tiba di dunia orang mati dan melihat semua orang itu yang telah tewas dalam pertempuran, mereka akan merasa terhibur. Aku TUHAN Yang Mahatinggi telah berbicara.
Eze 32:32  Akulah yang menyebabkan raja Mesir menimbulkan ketakutan pada orang-orang yang hidup, tetapi dia dan tentaranya akan tewas dan dibaringkan bersama semua orang yang tidak mengenal Allah dan gugur dalam pertempuran. Aku TUHAN Yang Mahatinggi telah berbicara."
Eze 33:1  TUHAN berbicara lagi kepadaku, kata-Nya,
Eze 33:2  "Hai manusia fana, katakanlah kepada bangsamu bahwa bilamana Aku mendatangkan peperangan atas sebuah negeri, inilah yang akan terjadi. Penduduk negeri itu akan memilih di antara mereka seorang penjaga.
Eze 33:3  Jika penjaga itu melihat musuh datang, ia membunyikan tanda bahaya supaya semua orang waspada.
Eze 33:4  Barangsiapa mendengar tanda bahaya itu tetapi tidak mengindahkannya, dan musuh datang lalu membunuhnya, maka dia mati karena salahnya sendiri. Seandainya ia mengindahkan tanda bahaya itu, pasti ia dapat selamat.
Eze 33:6  Tetapi jika penjaga itu melihat musuh datang dan ia tidak membunyikan tanda bahaya, lalu musuh membunuh orang-orang berdosa itu, tanggung jawab atas kematian mereka akan Kutuntut daripadanya.
Eze 33:7  Hai manusia fana, engkau Kuangkat menjadi penjaga bangsa Israel. Sampaikanlah kepada mereka peringatan ini:
Eze 33:8  Jika Aku memberitahukan bahwa seorang penjahat akan mati, tetapi engkau tidak memperingatkan dia supaya ia mengubah kelakuannya sehingga ia selamat, maka ia akan mati masih sebagai seorang berdosa. Tetapi tanggung jawab atas kematiannya akan Kutuntut daripadamu.
Eze 33:9  Jika engkau memperingatkan orang jahat itu dan ia tidak mau berhenti berbuat jahat, dia akan mati sebagai orang berdosa, tetapi engkau sendiri akan selamat."
Eze 33:10  TUHAN berkata kepadaku, "Hai manusia fana, ulangilah kepada orang Israel perkataan mereka ini, 'Kami merana karena dibebani dosa dan kejahatan kami. Bagaimana kami dapat hidup?'
Eze 33:11  Katakanlah kepada mereka bahwa demi Aku, Allah yang hidup, TUHAN Yang Mahatinggi, Aku tidak senang kalau orang jahat mati; sebaliknya Aku ingin ia meninggalkan dosa-dosanya supaya ia tetap hidup. Hai bangsa Israel, berhentilah berbuat jahat! Mengapa kamu mau mati?
Eze 33:12  Nah, manusia fana, katakanlah kepada orang Israel bahwa apabila orang yang baik berbuat dosa, maka kebaikannya dahulu tidak bisa menolong dia. Jika orang jahat berhenti berbuat dosa, ia tidak akan dihukum dan jika orang baik berbuat dosa, ia tidak akan diselamatkan.
Eze 33:13  Boleh jadi Aku telah berjanji akan menyelamatkan orang yang baik, tetapi jika ia mulai berbuat dosa dan berharap bahwa ia akan diselamatkan karena kebaikannya yang dulu itu, ia keliru. Aku tidak akan mengingat lagi kebaikan yang pernah dibuatnya dan ia akan mati karena dosa-dosanya.
Eze 33:14  Boleh jadi Aku telah memperingatkan orang jahat bahwa ia akan mati. Tetapi jika ia berhenti berbuat dosa dan mulai melakukan apa yang baik dan benar, misalnya ia mengembalikan barang gadaian orang, atau barang yang telah dicurinya serta mentaati hukum-hukum yang membawa hidup, maka ia tidak akan mati.
Eze 33:16  Aku tidak akan mengingat lagi dosa-dosa yang telah dilakukannya, dan ia akan hidup karena melakukan apa yang benar dan baik.
Eze 33:17  Tetapi bangsa itu mengatakan bahwa tindakan-Ku itu tidak adil! Padahal tindakan mereka yang tidak adil.
Eze 33:18  Sekali lagi Kutekankan bahwa jika orang yang baik mulai melakukan yang jahat, ia akan mati karena itu.
Eze 33:19  Jika orang yang jahat berhenti berbuat dosa, dan mulai melakukan apa yang baik dan benar, ia akan hidup karena itu.
Eze 33:20  Meskipun begitu bangsa Israel mengatakan bahwa tindakan-Ku tidak adil. Sungguh, Aku akan menghakimi mereka masing-masing menurut perbuatannya."
Eze 33:21  Pada tanggal lima bulan sepuluh, dalam tahun kedua belas masa pembuangan kami, seorang yang berhasil luput dari Yerusalem datang dan memberitahukan kepadaku bahwa kota itu telah jatuh.
Eze 33:22  Pada malam sebelum kedatangannya, aku telah merasakan kuasa TUHAN. Jadi ketika orang itu datang kepadaku, aku sudah bisa berbicara dan tidak bisu lagi.
Eze 33:23  TUHAN berkata kepadaku,
Eze 33:24  "Hai manusia fana, orang-orang yang tinggal di reruntuhan kota-kota di tanah Israel itu mengatakan, 'Abraham mendapat seluruh tanah ini, padahal ia cuma seorang diri. Kita ini banyak jumlahnya, jadi tanah ini menjadi milik kita.'"
Eze 33:25  Karena itu TUHAN Yang Mahatinggi menyuruh aku menyampaikan pesan-Nya ini kepada mereka, "Kamu memakan daging yang masih ada darahnya; kamu menyembah berhala dan melakukan pembunuhan. Berani benar kamu mengaku tanah ini sebagai milikmu.
Eze 33:26  Kamu mengandalkan pedangmu. Tindakanmu menjijikkan. Kamu semua melakukan perzinahan. Masih beranikah kamu mengaku tanah ini sebagai milikmu?
Eze 33:27  Demi Aku, Allah yang hidup, TUHAN Yang Mahatinggi, orang-orang yang tinggal di puing-puing kota itu akan mati terbunuh. Mereka yang tinggal di padang-padang akan dimakan binatang buas. Mereka yang bersembunyi di gunung-gunung dan gua-gua akan mati karena wabah penyakit.
Eze 33:28  Negeri itu akan Kujadikan padang gurun yang sunyi, dan kekuasaan yang mereka banggakan itu akan berakhir. Gunung-gunung di Israel akan menjadi sunyi sepi sehingga tak seorang pun berani melintasinya.
Eze 33:29  Pada waktu Aku menghukum bangsa itu karena dosa-dosa mereka dan menjadikan negeri itu sunyi sepi, tahulah mereka bahwa Akulah TUHAN."
Eze 33:30  TUHAN berkata, "Hai manusia fana, engkau telah menjadi buah bibir orang sebangsamu waktu mereka saling bertemu di dekat tembok-tembok kota atau di depan pintu-pintu rumah. Mereka berkata seorang kepada yang lain, 'Ayo, kita dengarkan apa pesan TUHAN hari ini.'
Eze 33:31  Mereka datang berbondong-bondong dan sebagai umat-Ku mereka duduk mendengar kata-katamu, tetapi apa yang kauperintahkan tidak mereka taati. Bibir mereka mengucapkan kata-kata cinta, tetapi hati mereka hanya memikirkan keuntungan.
Eze 33:32  Engkau dianggap seorang penyanyi lagu-lagu percintaan yang bersuara merdu dan pandai main kecapi. Memang mereka mendengar kata-katamu tetapi tidak mentaatinya.
Eze 33:33  Semua yang kaukatakan itu akan terjadi, dan bilamana waktu itu tiba, mereka akan tahu bahwa pernah ada seorang nabi di tengah-tengah mereka."
Eze 34:1  TUHAN berkata kepadaku,
Eze 34:2  "Hai manusia fana, kecamlah raja-raja Israel, dan sampaikanlah apa yang Aku, TUHAN Yang Mahatinggi katakan kepada mereka, 'Celakalah kamu, hai gembala-gembala Israel. Kamu hanya memikirkan kepentinganmu sendiri, bukannya kepentingan domba-dombamu.
Eze 34:3  Susunya kamu minum, bulu-bulunya kamu jadikan pakaian, dan domba yang paling gemuk kamu potong dan makan. Kamu tak pernah mengurus domba-dombamu.
Eze 34:4  Domba-domba yang lemah tidak kamu pelihara, yang sakit tidak kamu obati, yang luka tidak kamu balut, yang sesat dan hilang tidak kamu cari dan bawa kembali. Malahan semua dombamu itu kamu perlakukan dengan kejam.
Eze 34:5  Karena tidak mempunyai gembala, domba-domba itu tercerai berai dan menjadi mangsa binatang buas.
Eze 34:6  Maka domba-domba-Ku jadi mengembara di bukit-bukit yang tinggi dan di gunung-gunung. Mereka tersebar ke seluruh muka bumi, tak seorang pun memperhatikan atau mencari mereka.
Eze 34:7  Nah, gembala-gembala, dengarlah apa yang Aku TUHAN Yang Mahatinggi katakan kepadamu:
Eze 34:8  Demi Aku, Allah yang hidup, perhatikanlah! Domba-domba-Ku telah diterkam dan dimakan binatang buas, karena tak ada yang menggembalakan mereka. Kamu gembala-gembala-Ku tidak menghiraukan domba-domba itu. Kamu hanya memikirkan dirimu sendiri, bukannya kepentingan domba-dombamu.
Eze 34:9  Jadi, dengarlah hai gembala-gembala,
Eze 34:10  Aku, TUHAN Yang Mahatinggi mengatakan kepadamu, bahwa Aku ini musuhmu. Domba-domba-Ku itu akan Kuambil kembali daripadamu dan tidak lagi Kupercayakan kepadamu; kamu tidak lagi Kuizinkan memikirkan kepentinganmu sendiri. Domba-domba-Ku akan Kuselamatkan daripadamu sehingga mereka tidak lagi menjadi makananmu.'"
Eze 34:11  TUHAN Yang Mahatinggi berkata, "Aku sendiri akan mencari domba-domba-Ku dan memelihara mereka,
Eze 34:12  seperti seorang gembala mencari dan mengumpulkan domba-dombanya yang tercerai-berai lalu memelihara mereka. Mereka akan Kubawa pulang dari segala tempat mereka tersebar pada hari yang gelap dan naas itu.
Eze 34:13  Mereka akan Kukeluarkan dari tengah bangsa-bangsa dan negeri-negeri asing, lalu Kukumpulkan dan Kubawa ke negerinya sendiri. Mereka akan Kugembalakan di gunung-gunung dan di lembah-lembah Israel dan Kubimbing mereka ke padang-padang rumput yang nyaman.
Eze 34:14  Sungguh, Aku akan menyediakan bagi mereka padang-padang rumput yang subur di pegunungan dan di lembah-lembah tanah Israel. Di sana mereka akan istirahat dan makan rumput dengan aman.
Eze 34:15  Aku sendiri akan menjadi gembala domba-domba-Ku dan menyediakan tempat istirahat bagi mereka; Aku, TUHAN Yang Mahatinggi telah berbicara.
Eze 34:16  Yang hilang akan Kucari, yang sesat akan Kubawa pulang, yang luka akan Kubalut, yang sakit akan Kuobati; tetapi yang gemuk dan kuat akan Kubinasakan, sebab Aku gembala yang melakukan apa yang baik.
Eze 34:17  Dengarlah, hai domba-domba-Ku! Aku, TUHAN Yang Mahatinggi berkata kepadamu bahwa Aku akan menghakimi kamu masing-masing dan memisahkan yang baik dari yang jahat, domba-domba dari kambing-kambing.
Eze 34:18  Di antara kamu ada yang tidak puas hanya dengan menghabiskan rumput yang paling baik, tetapi juga menginjak-injak rumput yang tak mereka makan! Mereka minum air yang jernih lalu mengeruhkan air yang tak mereka minum!
Eze 34:19  Domba-domba-Ku yang lain harus makan rumput yang telah diinjak-injak dan minum air yang telah dikeruhkan.
Eze 34:20  Sebab itu, Aku, TUHAN Yang Mahatinggi mengatakan kepadamu, hai domba-domba yang gemuk, bahwa Aku akan datang menjadi hakim di antara domba yang kuat dan domba yang lemah.
Eze 34:21  Kamu mendesak-desak serta menanduk domba yang sakit sehingga mereka terhalau dari kawanan domba.
Eze 34:22  Tetapi Aku sendiri akan menyelamatkan domba-domba-Ku, supaya mereka tidak lagi diperlakukan dengan kejam. Aku akan menghakimi domba-domba-Ku satu per satu dan memisahkan yang baik dari yang jahat.
Eze 34:23  Aku akan mengangkat seorang raja yang seperti hamba-Ku Daud, untuk menjadi satu-satunya gembala mereka, dan dia akan memelihara mereka.
Eze 34:24  Aku, TUHAN, akan menjadi Allah mereka, dan raja yang seperti hamba-Ku Daud itu akan menjadi penguasa mereka. Aku TUHAN telah berbicara.
Eze 34:25  Aku akan membuat perjanjian dengan mereka, yang menjamin keamanan mereka. Aku akan mengusir semua binatang buas dari seluruh negeri, sehingga domba-domba-Ku dapat tinggal dengan aman di padang-padang dan tidur di hutan-hutan.
Eze 34:26  Aku akan memberkati mereka dan mengizinkan mereka tinggal di sekitar bukit-Ku yang suci. Mereka akan Kuberi hujan bila mereka memerlukannya.
Eze 34:27  Pohon-pohon akan berbuah, ladang-ladang akan memberikan hasil, dan semua orang akan tinggal dengan aman di tanahnya sendiri. Jika rantai pengikat umat-Ku telah Kuputuskan dan mereka telah Kubebaskan dari orang-orang yang memperbudak mereka, tahulah mereka bahwa Akulah TUHAN.
Eze 34:28  Mereka tidak akan lagi dirampok oleh bangsa-bangsa yang tidak mengenal Aku, atau diterkam oleh binatang-binatang buas. Mereka akan hidup dengan sejahtera, dan tak ada seorang pun akan membuat mereka takut lagi.
Eze 34:29  Aku akan menyuburkan ladang-ladang mereka sehingga tak ada masa kelaparan lagi di negeri itu. Mereka tidak akan dihina lagi oleh bangsa-bangsa lain.
Eze 34:30  Maka tahulah semua orang bahwa Israel adalah umat-Ku dan bahwa Aku melindunginya. Aku, TUHAN Yang Mahatinggi telah berbicara.
Eze 34:31  Sungguh, kamulah kawanan domba-Ku yang Kuberi makan. Kamu adalah umat-Ku, dan Aku Allahmu. Aku, TUHAN Yang Mahatinggi telah berbicara."
Eze 35:1  TUHAN berkata kepadaku,
Eze 35:2  "Hai manusia fana, kutukilah negeri Edom.
Eze 35:3  Sampaikanlah kepada penduduknya apa yang Aku TUHAN Yang Mahatinggi katakan kepada mereka, 'Aku ini musuhmu, hai penduduk pegunungan Edom! Tanahmu akan Kujadikan sunyi sepi tanpa penghuni, dan kota-kotamu akan menjadi puing-puing. Maka tahulah kamu bahwa Akulah TUHAN.
Eze 35:5  Sejak dahulu kamu memusuhi Israel dan membiarkan mereka dibantai pada waktu mereka ditimpa bencana sebagai hukuman terakhir atas dosa mereka.
Eze 35:6  Oleh karena itu, demi Aku, Allah yang hidup, TUHAN Yang Mahatinggi, maut adalah nasibmu, dan kamu tidak dapat lolos daripadanya. Kamu telah bersalah sebab melakukan pembunuhan, maka pembunuhan pun akan mengejarmu.
Eze 35:7  Kota-kotamu di pegunungan Edom akan Kujadikan sepi dan setiap orang yang melintasinya akan Kubunuh.
Eze 35:8  Gunung-gunung akan Kututupi dengan mayat-mayatmu, bukit dan lembah akan Kupenuhi dengan korban pertempuran.
Eze 35:9  Negerimu akan Kujadikan padang belantara untuk selama-lamanya, dan kota-kotamu tak akan lagi didiami orang. Maka tahulah kamu bahwa Akulah TUHAN.
Eze 35:10  Kamu mengatakan bahwa baik Yehuda maupun Israel dengan tanahnya masing-masing adalah hakmu, dan bahwa kamu akan memilikinya meskipun Aku, TUHAN, adalah Allah mereka.
Eze 35:11  Sebab itu, demi Aku, Allah yang hidup, TUHAN Yang Mahatinggi, Aku akan membalas dendam kepadamu, karena kamu marah, cemburu dan benci kepada umat-Ku. Mereka akan tahu bahwa kamu Kuhukum karena perbuatanmu terhadap mereka.
Eze 35:12  Maka tahulah kamu bahwa Aku, TUHAN, telah mendengar ejekanmu waktu kamu berkata: Gunung-gunung Israel sudah menjadi sunyi sepi dan diserahkan menjadi mangsa kita.
Eze 35:13  Hai bangsa Edom, kamu besar mulut dan berbicara dengan sombong kepada-Ku. Aku telah mendengar semua itu.
Eze 35:14  Aku TUHAN Yang Mahatinggi berkata: Seluruh dunia akan senang pada waktu seluruh negerimu Kujadikan padang yang sunyi sepi,
Eze 35:15  sama seperti kamu pun senang waktu melihat Israel milik-Ku, menjadi sunyi sepi. Gunung-gunung Seir, bahkan seluruh tanah Edom akan menjadi sunyi sepi. Maka semua orang akan tahu bahwa Akulah TUHAN.'"
Eze 36:1  TUHAN berkata, "Hai manusia fana, bicaralah kepada gunung-gunung Israel dan suruhlah mereka mendengarkan amanat TUHAN Yang Mahatinggi,
Eze 36:2  'Ketika bangsa-bangsa di sekitar menyerang dan merampok kamu, mereka telah menghina seluruh Israel. Dengan gembira mereka berkata bahwa sekarang bukit-bukit purbakala itu menjadi milik mereka. Sebab itu, hai gunung-gunung dan bukit-bukit, anak-anak sungai dan lembah, hai kota-kota yang sudah menjadi puing dan yang telah ditinggalkan oleh penghuninya serta dirampok dan diejek oleh bangsa-bangsa di sekitarnya, dengarlah apa yang dikatakan TUHAN Yang Mahatinggi kepadamu: Dalam luapan amarah-Ku Aku mengecam bangsa-bangsa di sekitarmu dan khususnya Edom, karena mereka dengan senang dan bersikap menghina, telah merebut tanah-Ku dan merampasi padang-padang rumputnya. Sekarang Aku marah dan mengamuk karena bangsa-bangsa itu telah sangat menghina serta merendahkan kamu.
Eze 36:7  Aku, TUHAN Yang Mahatinggi bersumpah bahwa bangsa-bangsa di sekitarmu itu akan dihina pula.
Eze 36:8  Tetapi kepadamu Kukatakan hai gunung-gunung Israel, pohon-pohonmu akan bertunas dan berbuah lagi untuk umat-Ku Israel yang segera akan pulang.
Eze 36:9  Aku ada di pihakmu hai Israel, dan Aku memastikan bahwa tanahmu akan dicangkul lagi dan ditaburi benih.
Eze 36:10  Aku akan menambah jumlah umat-Ku Israel. Mereka akan tinggal di kota-kota dan membangun kembali segala puingnya.
Eze 36:11  Penduduk dan ternakmu akan Kutambah, sehingga menjadi sangat banyak, lebih banyak dari semula, dan mereka akan beranak cucu banyak. Mereka akan hidup di sana seperti semula dan Kujadikan makmur, seperti belum pernah kamu alami. Maka tahulah mereka bahwa Akulah TUHAN.
Eze 36:12  Umat-Ku Israel, akan Kubawa pulang untuk mendiami negerinya lagi. Tanah itu akan menjadi milik mereka sendiri dan akan selalu memberi hasilnya sehingga anak-anak mereka tak akan lagi mati kelaparan.'
Eze 36:13  Aku, TUHAN Yang Mahatinggi berkata, 'Memang benar tanah ini diberi julukan pemakan manusia dan perampas anak-anak--bangsa yang mendiaminya.
Eze 36:14  Tetapi mulai sekarang tanah ini tak akan memakan manusia lagi atau merampas anak-anak mereka. Aku TUHAN Yang Mahatinggi telah berbicara.
Eze 36:15  Tak perlu lagi tanah ini mendengar ejekan bangsa-bangsa atau menanggung penghinaan orang. Tanah ini tak lagi akan menumpas bangsa yang mendiaminya. Aku, TUHAN Yang Mahatinggi telah berbicara.'"
Eze 36:16  TUHAN berkata kepadaku,
Eze 36:17  "Hai manusia fana, ketika orang-orang Israel tinggal di negeri mereka sendiri, mereka telah mencemarkannya dengan tindakan dan kelakuan mereka. Aku menganggap kelakuan mereka itu haram, seperti wanita yang sedang haid.
Eze 36:18  Kulimpahkan amarah-Ku kepada mereka karena mereka telah melakukan banyak pembunuhan di tanah itu dan mencemarkannya dengan berhala-berhala mereka.
Eze 36:19  Mereka Kuhukum sesuai dengan cara hidup dan kelakuan mereka, lalu mereka Kuceraiberaikan ke negeri-negeri asing.
Eze 36:20  Ke mana saja mereka pergi, mereka selalu mencemarkan nama-Ku yang suci. Sebab orang-orang mengatakan begini, 'Bangsa ini adalah umat Allah, tetapi mereka harus meninggalkan tanah yang diberikan Allah kepada mereka.'
Eze 36:21  Aku prihatin karena nama-Ku yang suci sudah dicemarkan oleh orang-orang Israel ke mana pun mereka pergi.
Eze 36:22  Oleh sebab itu, sampaikanlah apa yang Aku TUHAN Yang Mahatinggi katakan kepada bangsa Israel, 'Apa yang akan Kulakukan nanti bukanlah karena kamu, hai orang-orang Israel, melainkan karena nama-Ku yang suci, yang telah kamu cemarkan ke mana pun kamu pergi.
Eze 36:23  Aku akan menunjukkan kesucian nama-Ku yang besar itu, nama yang telah kamu cemarkan di tengah bangsa-bangsa, maka tahulah mereka bahwa Aku TUHAN. Aku, TUHAN Yang Mahatinggi telah berbicara. Aku akan memakai kamu untuk memperlihatkan kepada bangsa-bangsa bahwa Aku ini suci, TUHAN Yang Mahasuci.
Eze 36:24  Dari setiap bangsa dan negeri, kamu akan Kukumpulkan dan Kupulangkan ke tanahmu sendiri.
Eze 36:25  Setelah itu kamu akan Kuperciki dengan air jernih, supaya kamu bersih dari segala berhalamu dan dari segala sesuatu yang telah mencemarkan kamu.
Eze 36:26  Maka kamu Kuberikan hati yang baru dan pikiran yang baru. Hatimu yang sekeras batu itu akan Kuganti dengan hati yang taat.
Eze 36:27  Roh-Ku akan Kucurahkan ke dalam hatimu dan kamu akan Kujaga supaya hidup menurut hukum-hukum-Ku serta mentaati segala perintah-Ku.
Eze 36:28  Maka kamu akan tinggal di tanah yang telah Kuberikan kepada nenek moyangmu. Kamu akan menjadi umat-Ku dan Aku pun akan menjadi Allahmu.
Eze 36:29  Kamu akan Kubebaskan dari segala kecemaranmu. Aku akan memberi perintah kepada gandum supaya tumbuh dengan subur sehingga kamu tidak pernah lagi dilanda kelaparan.
Eze 36:30  Aku akan menambah kesuburan pohon buah-buahanmu dan hasil ladang-ladangmu, sehingga kamu tidak akan lagi mengalami masa kelaparan yang memburukkan namamu di antara bangsa-bangsa.
Eze 36:31  Kamu akan teringat kepada kelakuanmu yang jahat dan tindakanmu yang tidak pantas itu, lalu kamu menjadi muak dengan dirimu sendiri karena dosa-dosa dan kekejianmu itu.
Eze 36:32  Ingatlah hai Israel, bahwa semua itu Kulakukan bukan karena kamu. Kamu harus merasa malu dan hina karena kelakuanmu. Aku TUHAN Yang Mahatinggi telah berbicara.'"
Eze 36:33  TUHAN Yang Mahatinggi berkata, "Setelah kamu Kubersihkan dari segala dosamu, kamu akan Kuizinkan tinggal di kota-kotamu lagi, dan membangun kembali puing-puingnya.
Eze 36:34  Dahulu orang-orang yang berjalan lewat padang-padangmu melihat betapa tandusnya tanah itu, tetapi Aku akan memungkinkan kamu mengerjakan tanah itu lagi.
Eze 36:35  Maka semua orang akan menjadi heran, karena tanah yang tadinya tandus itu kini berubah menjadi taman Eden, dan kota-kota yang tadinya sepi, hancur dan menjadi puing-puing, sekarang menjadi kota-kota yang kuat pertahanannya dan padat penduduknya.
Eze 36:36  Maka bangsa-bangsa tetangga yang masih ada, akan tahu bahwa Aku, TUHAN, telah membangun kembali kota-kota yang telah hancur, dan mengerjakan kembali ladang-ladang yang tandus. Aku, TUHAN, telah berbicara dan pasti akan melaksanakannya."
Eze 36:37  TUHAN Yang Mahatinggi berkata, "Hai bangsa Israel, kamu Kuizinkan lagi minta tolong kepada-Ku; Aku akan menambah jumlahmu seperti kawanan domba.
Eze 36:38  Kota-kota yang sekarang hancur, kelak akan penuh penghuninya seperti Yerusalem pada hari-hari raya di masa lampau, penuh dengan domba-domba yang dipersembahkan. Maka tahulah kamu bahwa Akulah TUHAN."
Eze 37:1  Aku merasakan kuasa TUHAN, dan Roh-Nya membawa aku menuruni lembah yang penuh dengan tulang-tulang.
Eze 37:2  Aku dituntun TUHAN berkeliling-keliling di lembah itu, dan kulihat bahwa di seluruh lembah itu berserakan tulang-tulang yang amat banyak dan kering sekali.
Eze 37:3  TUHAN bertanya kepadaku, "Hai manusia fana, dapatkah tulang-tulang ini hidup kembali?" Aku menjawab, "TUHAN Yang Mahatinggi, hanya Engkaulah yang tahu."
Eze 37:4  Lalu TUHAN menyuruh aku menyampaikan pesan-Nya kepada tulang-tulang yang kering itu,
Eze 37:5  "Aku TUHAN Yang Mahatinggi meniupkan napas ke dalam dirimu supaya kamu hidup kembali.
Eze 37:6  Kutaruh urat dan daging padamu serta Kubalut kamu dengan kulit. Kamu akan Kuberi napas sehingga hidup. Maka tahulah kamu bahwa Akulah TUHAN."
Eze 37:7  Lalu aku Yehezkiel berbicara kepada tulang-tulang itu sesuai dengan perintah TUHAN. Dan sedang aku berbicara itu, kudengar suara berderak-derak, karena tulang-tulang itu mulai bersambung satu dengan yang lain.
Eze 37:8  Sementara aku memperhatikannya, tulang-tulang itu mulai berurat dan berdaging, lalu berkulit juga tetapi tubuh-tubuh itu belum bernapas.
Eze 37:9  Allah berkata kepadaku, "Hai manusia fana, bicaralah kepada angin. Katakanlah bahwa Aku, TUHAN Yang Mahatinggi memerintahkan kepadanya supaya datang dari segala arah, untuk meniupkan napas ke dalam tubuh-tubuh yang mati itu sehingga mereka hidup kembali."
Eze 37:10  Maka aku berbicara sesuai dengan perintah TUHAN dan angin pun meniupkan napas ke dalam tubuh-tubuh itu lalu mereka hidup dan berdiri. Jumlah mereka sangat banyak, cukup untuk membentuk kesatuan tentara yang besar.
Eze 37:11  Allah berkata kepadaku, "Hai manusia fana, tulang-tulang ini melambangkan bangsa Israel. Mereka mengeluh bahwa mereka kering, tanpa harapan dan tanpa hari depan.
Eze 37:12  Sebab itu meramallah kepada umat-Ku Israel itu bahwa Aku, TUHAN Yang Mahatinggi akan membuka kuburan mereka. Aku akan mengeluarkan mereka dari situ dan mengembalikan mereka ke tanah Israel.
Eze 37:13  Setelah Aku melakukan semuanya itu, umat-Ku akan tahu bahwa Akulah TUHAN.
Eze 37:14  Aku akan meniupkan napas dalam diri mereka untuk menghidupkan mereka, lalu mereka Kumungkinkan tinggal di negerinya sendiri. Maka tahulah mereka bahwa Akulah TUHAN. Aku TUHAN telah berbicara dan pasti akan melaksanakannya."
Eze 37:15  TUHAN berkata lagi kepadaku,
Eze 37:16  "Hai manusia fana, ambillah sebuah tongkat dan tulislah kata-kata ini di atasnya, 'Kerajaan Yehuda'. Lalu ambil sebuah tongkat yang lain, dan tulislah di atasnya: 'Kerajaan Israel'.
Eze 37:17  Kemudian peganglah kedua tongkat sedemikian rupa sehingga ujung-ujungnya bergabung dan menjadi satu tongkat.
Eze 37:18  Bila bangsamu menanyakan artinya,
Eze 37:19  katakanlah kepada mereka, bahwa Aku, TUHAN Yang Mahatinggi akan mengambil tongkat yang melambangkan Israel dan menggabungkannya dengan tongkat yang melambangkan Yehuda. Kedua tongkat itu akan Kujadikan satu dan Kupegang dalam tangan-Ku.
Eze 37:20  Peganglah kedua tongkat itu dan perlihatkanlah kepada bangsa itu.
Eze 37:21  Lalu katakanlah kepada mereka bahwa Aku, TUHAN Yang Mahatinggi akan mengambil seluruh bangsa-Ku dari negeri-negeri ke mana mereka telah pergi. Mereka akan Kukumpulkan dan Kubawa kembali ke negeri mereka sendiri.
Eze 37:22  Mereka akan Kujadikan satu bangsa di tanah itu, di pegunungan Israel. Mereka akan diperintah oleh satu raja, dan tidak akan lagi terpecah menjadi dua bangsa atau kerajaan.
Eze 37:23  Mereka tidak akan menajiskan diri lagi dengan berbuat dosa dan menyembah berhala-berhala yang menjijikkan. Aku akan membebaskan mereka dari semua penyelewengan yang membuat mereka mengkhianati Aku. Aku akan menyucikan mereka; mereka akan menjadi umat-Ku, dan Aku Allah mereka.
Eze 37:24  Mereka akan dipersatukan di bawah seorang raja seperti raja Daud yang akan memerintah untuk selama-lamanya. Dengan setia mereka akan mentaati hukum-hukum-Ku. Dan untuk selama-lamanya mereka serta keturunan mereka akan mendiami tanah itu, tanah yang Kuberikan kepada Yakub dan didiami oleh nenek moyang mereka.
Eze 37:26  Aku akan membuat suatu perjanjian dengan mereka yang menjamin keamanan mereka sampai kekal. Aku akan memberkati dan menambah jumlah mereka. Di tanah mereka akan Kubangun Rumah-Ku yang tetap di sana untuk selama-lamanya.
Eze 37:27  Aku akan tinggal bersama mereka, Aku akan menjadi Allah mereka dan mereka umat-Ku.
Eze 37:28  Bila Rumah-Ku sudah berada di tengah-tengah mereka untuk selama-lamanya, maka tahulah bangsa-bangsa lain bahwa Aku, TUHAN, telah memilih Israel menjadi umat-Ku sendiri."
Eze 38:1  TUHAN berkata kepadaku,
Eze 38:2  "Hai manusia fana, kutukilah Gog raja agung dari negeri-negeri Mesekh dan Tubal di tanah Magog.
Eze 38:3  Katakanlah kepadanya bahwa Aku, TUHAN Yang Mahatinggi, adalah musuhnya.
Eze 38:4  Aku akan membelokkan dia, memasang kait pada rahangnya, dan merenggut dia keluar beserta seluruh pasukannya. Banyak sekali pasukan berkudanya yang berpakaian seragam itu. Setiap prajurit membawa perisai dan pedang.
Eze 38:5  Raja diiringi oleh prajurit-prajurit dari Persia, Sudan dan Libia, masing-masing membawa pedang dan topi baja.
Eze 38:6  Semua pejuang dari negeri-negeri Gomer dan Bet-Togarma di utara, maju bersama dia, dan banyak bangsa lain ikut juga.
Eze 38:7  Suruhlah dia bersiap-siap dan perintahkanlah seluruh tentara yang bergabung dengan dia supaya siaga menuruti aba-abanya.
Eze 38:8  Beberapa tahun lagi dia akan Kuperintahkan untuk menyerbu sebuah negeri dengan penduduk yang telah dikumpulkan kembali dari antara banyak bangsa dan yang kemudian diam di negeri itu dengan tentram. Gog akan menduduki gunung-gunung Israel yang telah lama sepi dan penuh puing-puing, tetapi yang sekarang didiami oleh rakyat yang hidup dengan damai.
Eze 38:9  Dia dan tentaranya serta banyak bangsa yang mengiringinya akan datang menyerang seperti badai dan menutupi seluruh negeri seperti awan."
Eze 38:10  TUHAN Yang Mahatinggi berkata kepada Gog, "Pada saat itu engkau akan membuat rencana yang jahat.
Eze 38:11  Engkau memutuskan untuk menyerbu negeri yang tak berbenteng itu, di mana penduduknya hidup dengan aman dan tentram di kota-kota yang tidak bertembok dan tidak mempunyai alat-alat pertahanan.
Eze 38:12  Engkau akan merampok dan merampas barang-barang penduduk di kota yang dahulu puing-puing itu. Penduduknya yang dikumpulkan dari berbagai bangsa, sekarang sudah mempunyai banyak harta dan ternak dan mereka tinggal di pusat dunia.
Eze 38:13  Penduduk Syeba dan Dedan serta para pedagang di kota-kota negeri Spanyol akan bertanya kepadamu begini, 'Apakah engkau mengerahkan tentaramu hanya untuk merampas dan merampok emas, perak, ternak dan harta benda, serta mengangkut semuanya itu?'"
Eze 38:14  Sebab itu TUHAN Yang Mahatinggi mengutus aku untuk menyampaikan kepada Gog pesan ini, "Justru pada saat umat-Ku Israel hidup dengan aman dan tentram, engkau datang menyerbu.
Eze 38:15  Engkau berangkat dari tempat kediamanmu, jauh di utara, diiringi tentara yang besar dan hebat, semuanya pasukan berkuda dari segala macam bangsa.
Eze 38:16  Lalu kauserang umat-Ku Israel seperti badai melanda bumi. Jika saatnya tiba, engkau akan Kukirim untuk menduduki tanah-Ku. Dengan tindakan-Ku melalui engkau, hai Gog, bangsa-bangsa akan tahu siapa Aku, yaitu Allah yang suci.
Eze 38:17  Bukankah tentang engkau Aku dahulu berbicara melalui hamba-hamba-Ku para nabi Israel? Mereka telah meramalkan bahwa di masa yang akan datang Aku akan menyuruh seorang untuk menyerang Israel. Aku TUHAN Yang Mahatinggi telah berbicara."
Eze 38:18  TUHAN Yang Mahatinggi berkata, "Tetapi segera sesudah Gog menyerang Israel, Aku akan mengamuk.
Eze 38:19  Dan dengan sangat marah Aku akan mengancam, bahwa pada hari itu ada gempa bumi yang dahsyat di negeri Israel.
Eze 38:20  Semua ikan dan burung, semua binatang besar dan kecil, dan semua orang di muka bumi akan gemetar ketakutan menghadapi Aku. Gunung-gunung akan runtuh, tebing-tebing akan longsor dan semua tembok akan roboh.
Eze 38:21  Lalu Gog akan Kukejutkan dengan segala macam bencana. Aku, TUHAN Yang Mahatinggi telah berbicara. Anak buahnya akan saling menyerang dengan pedang mereka.
Eze 38:22  Dia akan Kuhukum dengan wabah penyakit dan pertumpahan darah. Hujan lebat dan hujan es, api dan belerang akan Kucurahkan atas dia dan tentaranya dan atas banyak bangsa yang telah bergabung dengan dia.
Eze 38:23  Dengan cara itu akan Kutunjukkan kepada segala bangsa bahwa Aku ini agung dan suci. Maka tahulah mereka bahwa Akulah TUHAN."
Eze 39:1  TUHAN Yang Mahatinggi berkata, "Hai manusia fana, ancamlah Gog, raja agung dari negeri-negeri Mesekh dan Tubal, dan katakanlah kepadanya bahwa Aku ini musuhnya.
Eze 39:2  Aku akan membelokkan dia, dan membawa dia keluar dari daerah jauh di utara, ke gunung-gunung Israel.
Eze 39:3  Di situ akan Kupukul busurnya sampai lepas dari tangan kirinya, dan Kujatuhkan anak-anak panahnya dari tangan kanannya.
Eze 39:4  Di atas gunung-gunung Israel, Gog bersama seluruh tentara dan sekutunya akan gugur, dan mayat-mayat mereka akan Kuberikan kepada segala burung dan binatang buas untuk dimakan.
Eze 39:5  Prajurit-prajurit itu akan jatuh dan mati di ladang-ladang. Aku, TUHAN Yang Mahatinggi telah berbicara.
Eze 39:6  Aku akan menimbulkan kebakaran di tanah Magog dan di pantai-pantai, di mana penduduknya hidup dengan tentram. Maka tahulah mereka semua bahwa Akulah TUHAN.
Eze 39:7  Tetapi di tengah-tengah orang Israel, akan Kunyatakan nama-Ku yang suci, dan nama-Ku itu tak akan lagi dicemarkan. Maka tahulah bangsa-bangsa bahwa Akulah TUHAN, Allah yang suci dari Israel."
Eze 39:8  TUHAN Yang Mahatinggi berkata, "Sungguh, hari yang Kusebut itu akan datang.
Eze 39:9  Penduduk kota-kota di Israel akan keluar berbondong-bondong untuk mengumpulkan senjata-senjata yang telah ditinggalkan, yaitu perisai dan pedang, busur dan anak panah serta tongkat pemukul. Semua itu mereka jadikan kayu bakar, dan itu cukup untuk tujuh tahun.
Eze 39:10  Jadi mereka tidak perlu mengumpulkan kayu bakar di ladang atau menebang pohon di hutan, sebab senjata-senjata yang ditinggalkan itu cukup banyak untuk bahan bakar. Mereka akan merampok dan merampas harta orang-orang yang telah merampok dan merampas harta mereka. Aku, TUHAN Yang Mahatinggi telah berbicara."
Eze 39:11  TUHAN berkata, "Bilamana semua itu terjadi, Aku akan memberikan kepada Gog tanah pekuburan di Israel, di Lembah Penyeberangan yang terletak di sebelah timur dari Laut Mati. Di situlah Gog dan seluruh tentaranya akan dikuburkan, dan lembah itu akan disebut 'Lembah Pasukan Gog'.
Eze 39:12  Tujuh bulan lamanya orang-orang Israel akan sibuk menguburkan semua jenazah supaya tanah itu tidak najis lagi.
Eze 39:13  Seluruh penduduk negeri akan membantu pekerjaan penguburan itu; dan untuk itu mereka akan menerima penghargaan pada hari kemenangan-Ku. Aku, TUHAN Yang Mahatinggi telah berbicara.
Eze 39:14  Setelah tujuh bulan itu lewat, beberapa orang akan ditugaskan menjelajahi negeri itu untuk mencari mayat-mayat yang masih terkapar di tanah, dan jika ada, menguburkannya untuk membersihkan negeri itu dari yang najis.
Eze 39:15  Dalam perjalanan itu, setiap kali mereka menemukan tulang manusia, mereka akan memberi tanda di sampingnya, supaya tukang-tukang kubur melihat itu dan menguburkan tulang itu di Lembah Pasukan Gog.
Eze 39:16  Dengan cara itu tanah itu dibersihkan dari yang najis. (Di kemudian hari, di lembah itu akan ada kota yang namanya sama dengan Pasukan Gog.)"
Eze 39:17  TUHAN Yang Mahatinggi berkata kepadaku, "Hai manusia fana, panggillah semua burung dan binatang lainnya, supaya datang dari mana-mana untuk memakan kurban yang Kusediakan bagi mereka. Aku akan mengadakan pesta besar di gunung-gunung Israel, di sana mereka dapat makan daging dan minum darah.
Eze 39:18  Mereka dapat memakan mayat-mayat pahlawan, dan minum darah para penguasa dunia. Para pahlawan dan penguasa itu akan dibunuh seperti kambing-kambing jantan atau anak-anak kambing, atau banteng-banteng gemuk.
Eze 39:19  Sesudah Kubunuh orang-orang itu seperti kurban, burung-burung dan binatang-binatang lain akan memakan semua lemaknya sampai muak, dan meminum darah sampai mabuk.
Eze 39:20  Pada meja-Ku mereka dapat mengenyangkan perut dengan semua kuda dan penunggangnya dan dengan para pejuang. Aku, TUHAN Yang Mahatinggi telah berbicara."
Eze 39:21  TUHAN berkata, "Aku akan memperhatikan kemuliaan-Ku kepada bangsa-bangsa. Mereka semua akan melihat bagaimana Aku mempergunakan kuasa-Ku untuk melaksanakan keputusan-keputusan yang adil.
Eze 39:22  Mulai saat itu orang Israel akan tahu bahwa Akulah TUHAN Allah mereka.
Eze 39:23  Dan semua bangsa akan insaf bahwa orang Israel dikirim ke pembuangan karena dosa mereka terhadap Aku. Mereka sudah Kutinggalkan dan Kubiarkan dikalahkan oleh musuh dan dibunuh dalam pertempuran.
Eze 39:24  Kuperlakukan mereka setimpal dengan kenajisan dan kejahatan mereka, lalu Kutinggalkan mereka."
Eze 39:25  TUHAN Yang Mahatinggi berkata, "Tetapi sekarang Aku akan mengasihani umat Israel keturunan Yakub, dan membuat mereka makmur kembali. Dengan demikian Aku menjaga agar nama-Ku tetap dihormati.
Eze 39:26  Bila mereka tinggal lagi dengan tenang di negeri mereka sendiri, dan tak ada seorang pun yang mengancam mereka, maka mereka akan melupakan bagaimana mereka telah dipermalukan karena mengkhianati Aku.
Eze 39:27  Aku akan membawa mereka kembali dari negeri-negeri musuh mereka, dan kesucian-Ku akan tampak kepada semua bangsa karena perbuatan-Ku itu.
Eze 39:28  Maka tahulah umat-Ku bahwa Akulah TUHAN Allah mereka. Sebab Akulah yang mengirim mereka ke pembuangan, tetapi Aku juga yang membawa mereka pulang ke negeri mereka sendiri, tanpa meninggalkan seorang pun dari mereka di sana.
Eze 39:29  Aku akan mencurahkan Roh-Ku atas bangsa Israel, dan tak akan meninggalkan mereka lagi. Aku, TUHAN Yang Mahatinggi telah berbicara."
Eze 40:1  Waktu itu tanggal sepuluh bulan satu, dalam tahun kedua puluh lima masa pembuangan kami, empat belas tahun sesudah Yerusalem dikalahkan, aku merasakan kuasa TUHAN, lalu aku dibawa-Nya pergi.
Eze 40:2  Dalam sebuah penglihatan, Allah membawa aku ke tanah Israel dan menempatkan aku di gunung yang tinggi. Di hadapanku kulihat sekelompok bangunan yang menyerupai sebuah kota.
Eze 40:3  Aku dibawanya ke situ dan kulihat seorang laki-laki yang berkilauan seperti perunggu berdiri di dekat pintu gerbang. Ia memegang tali pengukur dari linen serta kayu pengukur.
Eze 40:4  Laki-laki itu berkata kepadaku, "Hai manusia fana, dengarlah baik-baik dan perhatikanlah segala sesuatu yang kuperlihatkan kepadamu, sebab untuk itulah engkau dibawa ke mari. Beritahukanlah kepada bangsa Israel semua yang akan kaulihat."
Eze 40:5  Lalu aku melihat Rumah TUHAN yang dikelilingi tembok. Laki-laki itu mulai mengukur tembok itu dengan kayu pengukurnya yang panjangnya tiga meter. Ternyata tembok itu tingginya tiga meter dan tebalnya juga tiga meter.
Eze 40:6  Kemudian ia pergi ke gerbang yang menghadap ke timur. Ia mendaki tangganya dan sesampainya di atas, ia mengukur ambang pintu itu, lebarnya tiga meter.
Eze 40:7  Di belakang gerbang itu ada sebuah lorong dan pada masing-masing sisinya terdapat tiga kamar jaga. Semua kamar itu sama besarnya; bentuknya segi empat, dan panjang setiap sisinya adalah tiga meter. Tembok-tembok pemisah setiap kamar sama tebalnya, yaitu dua setengah meter. Lorong itu memanjang tiga meter lagi dan menuju ke ruang besar yang menghadap ke Rumah TUHAN. Laki-laki itu mengukur ruang besar itu dan ternyata lebarnya empat meter; letaknya di ujung pintu gerbang, sebelah dalam, yaitu yang paling dekat dengan Rumah TUHAN. Tembok-tembok di bagian yang menghadap Rumah TUHAN, tebalnya satu meter.
Eze 40:11  Kemudian laki-laki itu mengukur lebar lorong di pintu gerbang; enam setengah meter, dan juga lebar pintu gerbang itu lima meter.
Eze 40:12  Di depan kamar-kamar jaga yang berukuran tiga kali tiga meter itu, ada tembok rendah yang tingginya setengah meter, dan tebalnya setengah meter juga.
Eze 40:13  Lalu laki-laki itu mengukur jarak dari tembok belakang sebuah kamar jaga sampai ke tembok belakang kamar jaga di seberangnya, yaitu 12,5 meter.
Eze 40:14  Ia juga mengukur ruang besar itu: panjangnya sepuluh meter. Pintu ruang itu menuju ke pelataran luar Rumah TUHAN.
Eze 40:15  Jarak dari pintu gerbang sebelah luar sampai ke tembok ruang besar di bagian yang menghadap ke Rumah TUHAN adalah 25 meter.
Eze 40:16  Pada tembok belakang dan tembok pemisah kamar-kamar itu ada jendela-jendela kecil. Tembok-tembok bagian dalam yang menghadap ke lorong, dihias dengan ukir-ukiran pohon palem.
Eze 40:17  Kemudian laki-laki itu membawa aku melalui pintu gerbang ke pelataran luar Rumah TUHAN. Pada sisi tembok luar di sekeliling Rumah TUHAN itu dibangun 30 kamar. Di depan kamar-kamar itu ada lantai batu
Eze 40:18  yang mengelilingi seluruh pelataran itu. Pelataran luar itu lebih rendah daripada pelataran dalam.
Eze 40:19  Di situ ada pintu gerbang lain yang letaknya lebih tinggi dan yang menuju ke pelataran dalam. Laki-laki itu mengukur jarak antara kedua pintu gerbang itu: 50 meter.
Eze 40:20  Kemudian laki-laki itu mengukur gerbang utara yang menuju ke pelataran dalam.
Eze 40:21  Di gerbang itu, ketiga kamar jaga pada masing-masing sisi lorongnya, tembok-tembok pemisahnya serta ruang besarnya, semuanya sama ukurannya dengan yang ada di gerbang sebelah timur. Seluruh lorong gerbang itu panjangnya 25 meter dan lebarnya 12,5 meter.
Eze 40:22  Juga ruang besarnya, jendela-jendelanya, dan ukir-ukiran pohon palemnya sama dengan yang ada di gerbang timur. Tujuh anak tangga menuju ke gerbang itu, dan di ujungnya ada ruang besar yang menghadap ke pelataran luar Rumah TUHAN.
Eze 40:23  Berhadapan dengan gerbang utara, ada gerbang lain yang menuju ke pelataran dalam, sama seperti yang ada di sebelah gerbang timur. Orang itu mengukur jarak antara kedua gerbang itu: 50 meter.
Eze 40:24  Selanjutnya, laki-laki itu membawa aku ke gerbang selatan dan mengukurnya. Ternyata gerbang itu sama dengan gerbang-gerbang yang lain.
Eze 40:25  Jendela-jendela di kamar-kamarnya sama dengan yang ada di gerbang-gerbang yang lain. Panjang lorong gerbang itu 25 meter, dan lebarnya 12,5 meter.
Eze 40:26  Tujuh anak tangga menuju ke gerbang itu dan ruang besarnya juga ada di ujung, menghadap ke pelataran luar Rumah TUHAN. Tembok-tembok bagian dalam yang menghadap ke lorong, dihias dengan ukir-ukiran pohon palem.
Eze 40:27  Di situ ada juga gerbang yang menuju ke pelataran dalam. Laki-laki itu mengukur jarak antara kedua gerbang itu: 50 meter.
Eze 40:28  Selanjutnya laki-laki itu membawa aku melalui gerbang selatan masuk ke pelataran dalam. Ia mengukur gerbang selatan itu, dan ukurannya sama dengan gerbang-gerbang di sebelah luar.
Eze 40:29  Kamar-kamar jaganya, ruang besarnya, dan tembok-tembok dalamnya sama dengan yang ada di gerbang-gerbang lain. Kamar-kamar di gerbang ini pun mempunyai jendela-jendela. Lorong gerbang itu panjangnya 25 meter dan lebar 12,5 meter.
Eze 40:31  Tetapi ruang besarnya menghadap ke pelataran luar. Tembok-tembok bagian dalam yang menghadap ke lorong, juga dihias dengan ukir-ukiran pohon palem. Delapan anak tangga itu menuju ke gerbang itu.
Eze 40:32  Setelah itu laki-laki itu membawa aku melalui gerbang timur masuk ke halaman dalam. Ia mengukur gerbang itu dan ukurannya sama dengan gerbang-gerbang yang lain.
Eze 40:33  Kamar-kamar jaganya, tembok-tembok dalamnya dan ruang besarnya ukurannya seperti yang ada di gerbang-gerbang yang lain. Di kamar-kamar itu dan di ruang besarnya ada jendela-jendela. Panjang lorong gerbang itu 25 meter dan lebarnya 12,5 meter.
Eze 40:34  Ruang besarnya menghadap ke pelataran luar. Tembok-tembok bagian dalam yang menghadap ke lorong, dihias dengan ukir-ukiran pohon palem. Delapan anak tangga menuju ke gerbang itu.
Eze 40:35  Akhirnya laki-laki itu membawa aku ke gerbang utara. Ia mengukur gerbang itu dan ukurannya sama dengan gerbang-gerbang yang lain.
Eze 40:36  Juga kamar-kamar jaganya, tembok-tembok dalamnya dan ruang besarnya sama dengan yang ada di gerbang-gerbang yang lain. Di kamar-kamarnya dan di ruang besarnya ada jendela-jendela. Panjang lorong gerbang itu 25 meter dan lebarnya 12,5 meter.
Eze 40:37  Ruang depannya juga menghadap ke pelataran luar. Tembok-tembok bagian dalam yang menghadap ke lorong, dihias dengan ukir-ukiran pohon palem. Delapan anak tangga menuju ke gerbang itu.
Eze 40:38  Di pelataran luar ada sebuah kamar tambahan di gerbang dalam sebelah utara. Dari kamar tambahan itu ada sebuah pintu yang menuju ke ruang besar. Di situlah dibasuh binatang-binatang untuk kurban bakaran.
Eze 40:39  Di ruang besar itu, ada empat buah meja, dua di setiap sisinya. Di atas meja-meja itulah dipotong binatang untuk kurban bakaran, atau kurban penghapus dosa atau kurban ganti rugi.
Eze 40:40  Di luar ruang besar itu ada lagi empat meja, dua di setiap sisi gerbang utara itu.
Eze 40:41  Jadi, semuanya ada delapan buah meja tempat memotong kurban: empat di dalam, dan empat di luar pelataran.
Eze 40:42  Keempat buah meja di kamar tambahan yang dipakai untuk mempersiapkan kurban bakaran dipahat dari batu. Meja itu tingginya setengah meter, sedang bagian atasnya persegi empat, panjang dan lebarnya masing-masing tiga perempat meter. Di atas meja-meja itu disimpan perkakas untuk memotong binatang kurban.
Eze 40:43  Tepi meja-meja itu mempunyai pinggiran selebar 75 milimeter. Semua daging yang akan dikurbankan diletakkan di atas meja-meja itu.
Eze 40:44  Kemudian laki-laki itu membawa aku ke pelataran dalam. Di situ ada dua bangsal, yang satu di samping gerbang utara dan menghadap ke selatan, sedang yang satu lagi di samping gerbang selatan dan menghadap ke utara.
Eze 40:45  Laki-laki itu mengatakan kepadaku bahwa bangsal yang menghadap ke selatan itu disediakan untuk para imam yang bertugas di Rumah TUHAN,
Eze 40:46  sedang bangsal yang menghadap ke utara disediakan untuk para imam yang bertugas pada mezbah. Semua imam itu harus dari keturunan Zadok. Sebab dari suku Lewi hanya keturunan Zadok yang boleh menghadap TUHAN untuk melayani Dia dalam ibadat.
Eze 40:47  Laki-laki itu mengukur pelataran dalam. Bentuknya persegi empat: panjang dan lebarnya masing-masing 50 meter. Rumah TUHAN itu ada di sebelah barat, dan di depannya ada sebuah mezbah.
Eze 40:48  Kemudian ia membawa aku ke ruang depan Rumah TUHAN dengan mendaki beberapa anak tangga. Ia mengukur ruang masuk; luasnya tujuh kali dua setengah meter, dengan dinding-dinding pada kedua pinggirnya setebal satu setengah meter. Di kiri kanannya ada sebuah pilar. Ruang depan itu sendiri panjangnya sepuluh meter dan lebarnya enam meter.
Eze 41:1  Laki-laki itu membawa aku ke dalam kamar tengah, yang disebut Tempat Yang Suci. Ia mengukur lorong yang menuju ke situ: panjangnya tiga meter,
Eze 41:2  dan lebarnya lima meter. Dinding pada kedua sisinya tebalnya dua setengah meter. Lalu ia mengukur Tempat Yang Suci itu: panjangnya 20 meter dan lebarnya 10 meter.
Eze 41:3  Kemudian ia memasuki kamar dalam. Ia mengukur lorong yang menuju ke situ: panjangnya satu meter, lebarnya tiga meter, dan dinding di kedua sisinya tebalnya tiga setengah meter.
Eze 41:4  Lalu ia mengukur kamar itu juga: bentuknya persegi empat, panjang dan lebarnya masing-masing sepuluh meter. Kemudian ia berkata kepadaku, "Inilah Tempat Yang Mahasuci."
Eze 41:5  Setelah itu laki-laki itu mengukur tembok dalam dari Rumah TUHAN: tebalnya tiga meter. Pada tembok di sekeliling Rumah TUHAN itu ada kamar-kamar yang lebarnya dua meter.
Eze 41:6  Kamar-kamar itu bertingkat tiga dan pada setiap tingkat ada 30 kamar. Tembok luar setiap tingkat lebih tipis daripada tembok tingkat di bawahnya, sehingga kamar-kamar itu dapat disusun tanpa melubangi tembok Rumah TUHAN itu.
Eze 41:7  Dari luar, tembok Rumah TUHAN itu kelihatan sama tebalnya dari bawah sampai ke atas. Pada tembok luar dari Rumah TUHAN, di sebelah luar kamar-kamar itu, ada dua tangga besar sehingga orang dapat naik dari tingkat bawah ke tingkat dua dan tiga.
Eze 41:8  Tembok luar kamar-kamar itu tebalnya dua setengah meter. Bangunan itu mempunyai pintu yang menuju ke kamar-kamar di sebelah utara Rumah TUHAN, dan pintu lain yang menuju ke kamar-kamar di sebelah selatan. Di sekeliling Rumah TUHAN ada teras yang lebarnya dua setengah meter; letaknya tiga meter dari tanah, dan dibuat sama tingginya dengan fondasi kamar-kamar pada tembok Rumah TUHAN. Di antara teras dan gedung-gedung yang dipakai para imam, ada beranda yang lebarnya sepuluh meter, di kiri kanan Rumah TUHAN.
Eze 41:12  Di seberang pelataran sebelah barat, ada bangunan besar yang panjangnya 45 meter dan lebarnya 35 meter, sedang tebal dinding-dindingnya dua setengah meter.
Eze 41:13  Laki-laki itu mengukur panjangnya Rumah TUHAN: 50 meter. Dari sebelah belakang Rumah TUHAN sampai ke tembok belakang bangunan besar di sebelah barat itu jaraknya juga 50 meter.
Eze 41:14  Bagian depan dari Rumah TUHAN yang menghadap ke timur dengan pelataran di kiri kanannya, lebarnya juga 50 meter.
Eze 41:15  Laki-laki itu juga mengukur bangunan besar yang menghadap pelataran di sebelah barat: lebarnya 50 meter, termasuk dinding-dinding di kiri kanannya. Ruang-ruang di Rumah TUHAN, ruang depan, Tempat Yang Suci, dan Tempat Yang Mahasuci,
Eze 41:16  dihias dengan bingkai kayu, dari lantai sampai ke jendela-jendelanya. Jendela-jendela itu dapat ditutupi dengan tirai.
Eze 41:17  Seluruh dinding bagian dalam dihias dengan ukir-ukiran
Eze 41:18  pohon palem dan kerub, satu pohon palem di antara setiap dua kerub. Kerub itu mempunyai dua wajah,
Eze 41:19  satu wajah manusia yang menghadap ke pohon palem sebelah kiri, dan wajah singa yang menghadap ke pohon palem sebelah kanan. Ukir-ukiran itu terdapat di seluruh Rumah TUHAN,
Eze 41:20  mulai dari lantai sampai ke bagian atas pintu.
Eze 41:21  Tiang-tiang pintu dari Tempat Yang Suci berbentuk persegi empat. Di depan Tempat Yang Mahasuci ada sesuatu yang tampaknya seperti mezbah dari kayu.
Eze 41:22  Tingginya satu setengah meter dan lebarnya satu meter. Sudut-sudutnya, alasnya dan dindingnya terbuat dari kayu. Laki-laki itu berkata kepadaku, "Inilah meja yang ada di depan kehadiran TUHAN."
Eze 41:23  Di ujung lorong yang menuju ke Tempat Yang Suci ada sebuah pintu, begitu juga di ujung lorong yang menuju ke Tempat Yang Mahasuci.
Eze 41:24  Pintu-pintu itu mempunyai dua daun pintu yang membuka di tengah.
Eze 41:25  Pintu-pintu Tempat Yang Suci dihias dengan ukir-ukiran pohon palem dan kerub-kerub, seperti pada dinding-dindingnya. Di muka ruang depan yaitu di sebelah luarnya, ada tirai dari kayu.
Eze 41:26  Pada kedua sisi dinding sampingnya terdapat jendela-jendela, dan dindingnya dihias dengan ukir-ukiran pohon palem.
Eze 42:1  Kemudian laki-laki itu membawa aku ke pelataran luar dan menuntun aku ke bangunan lain di sebelah utara Rumah TUHAN, tidak jauh dari bangunan besar di sebelah barat Rumah TUHAN itu.
Eze 42:2  Bangunan itu panjangnya lima puluh meter, dan lebarnya dua puluh lima meter.
Eze 42:3  Satu sisi dari bangunan itu menghadap ke beranda yang lebarnya sepuluh meter itu, sedang sisi yang lain menghadap ke lantai batu di pelataran luar. Bangunan itu bertingkat tiga; setiap tingkat menjorok lebih ke dalam daripada tingkat di bawahnya.
Eze 42:4  Di depan kamar-kamar itu, di sebelah dalam ada lorong yang lebarnya lima meter dan panjangnya lima puluh meter. Pintu-pintunya menghadap ke utara.
Eze 42:5  Kamar-kamar di tingkat atas lebih sempit daripada kamar-kamar di tingkat-tingkat bawahnya, sebab lorong-lorong di depan kamar-kamar di atas itu lebih lebar daripada yang ada di bawahnya.
Eze 42:6  Kamar-kamar itu bertingkat tiga dan tidak ditopang oleh tiang-tiang seperti pada gedung-gedung di halaman.
Eze 42:7  Separuh dari tingkat bawah, sepanjang dua puluh lima meter merupakan tembok saja, sedangkan pada bagian selebihnya, juga sepanjang dua puluh lima meter, dibangun kamar-kamar. Pada tingkat yang paling atas ada kamar-kamar di sepanjang gedung itu.
Eze 42:9  Di bawah kamar-kamar di ujung timur bangunan ini, dan pada pangkal tembok, ada sebuah pintu yang menuju ke pelataran luar. Pada sisi selatan dari Rumah TUHAN, ada bangunan yang serupa, letaknya tidak jauh dari bangunan di sebelah barat.
Eze 42:11  Di depan kamar-kamarnya ada sebuah lorong yang sama seperti lorong di sebelah utara. Ukurannya sama, modelnya sama, dan pintu-pintunya pun sama.
Eze 42:12  Di bawah kamar-kamar di sisi selatan, yaitu di ujung timur pada pangkal tembok, ada juga sebuah pintu.
Eze 42:13  Laki-laki itu berkata kepadaku, "Kedua bangunan itu suci. Di situ, para imam yang boleh menghadap kehadiran TUHAN, memakan kurban-kurban yang paling suci. Di situ juga harus mereka tempatkan kurban-kurban yang paling suci, yaitu: kurban gandum dan kurban penghapus dosa atau kurban ganti rugi.
Eze 42:14  Jika para imam selesai bertugas di Rumah TUHAN, dan mau pergi ke pelataran luar, maka pakaian khusus yang mereka pakai waktu melayani ibadat harus mereka tinggalkan di kamar-kamar itu. Mereka harus memakai pakaian lain, setelah itu barulah mereka boleh pergi ke tempat rakyat berkumpul."
Eze 42:15  Sesudah laki-laki itu selesai mengukur bangunan Rumah TUHAN sebelah dalam, ia membawa aku keluar melalui gerbang timur, lalu mengukur bagian keliling Rumah TUHAN.
Eze 42:16  Dengan tongkat pengukurnya ia mengukur sisi sebelah timur: 250 meter.
Eze 42:17  Kemudian ia mengukur sisi sebelah utara, sisi sebelah selatan, dan sisi sebelah barat; setiap sisi panjangnya sama, yaitu 250 meter.
Eze 42:20  Jadi, daerah Rumah TUHAN meliputi sebidang tanah persegi empat, berukuran 250 meter pada setiap sisi. Bidang tanah itu dikelilingi sebuah tembok untuk memisahkan daerah yang khusus untuk TUHAN dari daerah yang untuk umum.
Eze 43:1  Kemudian laki-laki itu membawa aku ke gerbang yang menghadap ke timur,
Eze 43:2  dan di situlah kulihat datang dari arah timur cahaya kemilau yang menandakan kehadiran Allah Israel. Suara Allah terdengar seperti gemuruhnya samudra, dan bumi bergemerlapan karena cahaya kehadiran Allah.
Eze 43:3  Penglihatan itu sama seperti yang kulihat ketika Allah datang untuk menghancurkan Yerusalem, dan seperti yang kulihat di Sungai Kebar. Lalu aku merebahkan diri dan sujud.
Eze 43:4  Cahaya kemilau itu bergerak melalui gerbang timur lalu masuk ke dalam Rumah TUHAN.
Eze 43:5  Roh TUHAN mengangkat aku dan membawa aku ke pelataran dalam; di situ kulihat Rumah TUHAN diliputi cahaya kehadiran TUHAN.
Eze 43:6  Ketika laki-laki itu masih berdiri di sampingku, kudengar TUHAN berkata kepadaku dari dalam Rumah TUHAN, kata-Nya,
Eze 43:7  "Hai manusia fana, inilah takhta-Ku. Di sinilah Aku akan tinggal di tengah-tengah umat Israel dan memerintah atas mereka selama-lamanya. Baik mereka, maupun raja-raja mereka, tidak lagi akan mencemarkan nama-Ku yang suci dengan menyembah berhala atau dengan menguburkan jenazah raja-raja mereka di tempat ini.
Eze 43:8  Dahulu raja-raja mereka membuat ambang-ambang dan tiang-tiang pintu istana mereka di samping ambang-ambang dan tiang-tiang pintu Rumah-Ku, sehingga hanya ada sebuah dinding saja di antaranya. Mereka telah mencemarkan nama-Ku yang suci dengan perbuatan mereka yang menjijikkan. Sebab itu Kubinasakan mereka dalam kemarahan-Ku.
Eze 43:9  Tetapi mulai sekarang mereka harus berhenti menyembah berhala dan memindahkan jenazah raja-raja mereka. Setelah itu, Aku akan tinggal di tengah-tengah mereka untuk selama-lamanya."
Eze 43:10  Lalu TUHAN berkata lagi, "Hai manusia fana, beritahukanlah kepada umat Israel tentang Rumah-Ku ini, dan suruhlah mereka meneliti rancangannya. Buatlah mereka menjadi malu karena segala dosa mereka.
Eze 43:11  Dan setelah mereka malu karena perbuatan-perbuatan mereka, terangkanlah rancangan Rumah-Ku ini kepada mereka: modelnya, pintu-pintu masuk dan keluarnya, susunan bangunannya, segala peraturan dan hukum-hukumnya. Tulislah semua itu bagi mereka supaya mereka dapat melihat dan memperhatikan cara mengatur segalanya, serta mentaati semua peraturannya.
Eze 43:12  Inilah sifat khas dari Rumah TUHAN: Seluruh daerah sekitarnya di puncak gunung adalah sangat suci."
Eze 43:13  Inilah ukuran mezbah menurut ukuran yang dipakai untuk mengukur Rumah TUHAN. Di sekeliling kaki mezbah ada sebuah parit yang dalamnya setengah meter dan lebarnya setengah meter juga, dengan pinggiran di tepi luarnya setinggi seperempat meter.
Eze 43:14  Dari parit sampai jalur mezbah yang paling bawah, tingginya satu meter. Bagian tengahnya tingginya dua meter, dan setengah meter lebih sempit daripada keliling kaki mezbah. Bagian ketiga, yaitu puncak mezbah, juga setengah meter lebih sempit daripada keliling bagian tengahnya.
Eze 43:15  Bagian ketiga ini ialah tempat membakar kurban; tingginya dua meter. Pada keempat sudutnya ada bagian yang menonjol seperti tanduk.
Eze 43:16  Puncak mezbah itu berbentuk persegi empat; tiap sisinya berukuran enam meter.
Eze 43:17  Bagian tengahnya juga persegi empat, setiap sisinya berukuran tujuh meter, dengan sebuah tepi di pinggiran luar yang tingginya seperempat meter. (Lebar paritnya setengah meter.) Di sisi sebelah timur ada tangga yang menuju ke puncak mezbah.
Eze 43:18  TUHAN Yang Mahatinggi berkata kepadaku, "Hai manusia fana, dengarkanlah apa yang Kukatakan kepadamu. Jika mezbah itu selesai dibangun, engkau harus mentahbiskannya dengan membakar kurban di atasnya dan memercikkannya dengan darah kurban.
Eze 43:19  Hanya imam keturunan Zadok dari suku Lewi saja yang boleh datang ke hadirat-Ku untuk menyelenggarakan kebaktian. Aku, TUHAN Yang Mahatinggi yang memerintahkan hal itu. Berilah kepada mereka seekor sapi jantan muda untuk kurban pengampunan dosa.
Eze 43:20  Ambillah sedikit darahnya dan sapukanlah pada tanduk-tanduk di keempat sudut puncak mezbah itu, juga pada sudut-sudut bagian tengahnya serta di sekeliling pinggiran mezbah itu. Dengan cara itulah engkau menyucikan mezbah itu dan mengkhususkannya bagi-Ku.
Eze 43:21  Kemudian, ambillah sapi jantan untuk penghapus dosa itu, dan bakarlah di tempat yang sudah ditentukan, yaitu di luar lingkungan suci Rumah TUHAN.
Eze 43:22  Besoknya engkau harus mempersembahkan seekor kambing jantan yang tidak ada cacatnya untuk kurban pengampunan dosa. Sucikanlah mezbah itu dengan darah kambing itu, sama seperti yang kaulakukan dengan sapi jantan muda itu.
Eze 43:23  Jika telah selesai engkau melakukannya, ambillah seekor sapi jantan muda dan seekor kambing jantan muda, kedua-duanya yang tidak ada cacatnya,
Eze 43:24  lalu bawalah kepada-Ku. Para imam harus menaburkan garam ke atas kedua binatang itu dan membakarnya menjadi kurban bagi-Ku.
Eze 43:25  Setiap hari selama tujuh hari, engkau harus mengurbankan seekor kambing, sapi jantan muda, dan kambing jantan untuk kurban pengampunan dosa. Binatang-binatang itu tidak boleh ada cacatnya.
Eze 43:26  Tujuh hari lamanya para imam harus mengadakan upacara pentahbisan mezbah itu dan mempersiapkannya untuk bisa dipakai.
Eze 43:27  Sesudah ketujuh hari itu berakhir, para imam harus mulai mempersembahkan di atas mezbah itu kurban bakaran dan kurban perdamaian untuk umat. Maka Aku akan senang dengan kamu semua. Aku, TUHAN Yang Mahatinggi telah berbicara."
Eze 44:1  Kemudian laki-laki itu menuntun aku lagi ke gerbang luar di sebelah timur lingkungan Rumah TUHAN. Gerbang itu tertutup,
Eze 44:2  TUHAN berkata kepadaku, "Gerbang ini harus tetap tertutup dan tak boleh dibuka. Tak seorang pun boleh memasukinya, sebab Aku, TUHAN Allah Israel, telah masuk ke Rumah-Ku melalui gerbang ini. Jadi, gerbang ini harus tetap tertutup.
Eze 44:3  Akan tetapi penguasa yang memerintah boleh pergi ke gerbang itu untuk memakan makanan suci di hadapan-Ku. Dia harus masuk melalui ruang besar dan kembali melalui jalan itu juga."
Eze 44:4  Kemudian laki-laki itu membawa aku ke gerbang utara yang berhadapan dengan Rumah TUHAN. Aku memperhatikan, dan aku melihat Rumah TUHAN itu diliputi oleh cahaya kemilau yang menandakan kehadiran TUHAN. Lalu aku merebahkan diri dan sujud,
Eze 44:5  dan TUHAN berkata kepadaku, "Hai manusia fana, perhatikanlah baik-baik apa yang kaulihat dan yang kaudengar. Aku akan memberitahu kepadamu segala peraturan dan hukum mengenai Rumah-Ku ini. Catatlah baik-baik siapa-siapa yang diizinkan masuk Rumah-Ku dan siapa-siapa yang dilarang.
Eze 44:6  Katakanlah kepada umat Israel yang suka memberontak itu bahwa Aku, TUHAN Yang Mahatinggi, tidak mau lagi membiarkan mereka berbuat najis.
Eze 44:7  Mereka telah mencemarkan Rumah-Ku, sebab sementara lemak dan darah kurban dipersembahkan kepada-Ku, mereka mengizinkan masuk ke dalamnya orang-orang asing yang tidak disunat, dan yang tidak mentaati Aku. Dengan perbuatan-perbuatan mereka yang menjijikkan itu, mereka telah melanggar perjanjian-Ku dengan mereka.
Eze 44:8  Bukannya mereka sendiri yang mengadakan upacara-upacara suci dalam Rumah-Ku, melainkan orang-orang asing yang mereka tugaskan untuk melakukannya.
Eze 44:9  Aku, TUHAN Yang Mahatinggi, mengumumkan bahwa orang asing yang tidak disunat dan orang yang tidak taat kepada-Ku, tidak diperbolehkan masuk Rumah-Ku. Itu berlaku juga untuk orang asing yang tinggal menetap di tengah-tengah bangsa Israel."
Eze 44:10  TUHAN berkata kepadaku, "Aku akan menghukum orang-orang Lewi yang bersama-sama dengan orang-orang Israel lainnya, telah meninggalkan Aku dan menyembah berhala.
Eze 44:11  Di dalam Rumah-Ku, mereka hanya boleh melayani Aku sebagai karyawan dan penjaga gerbang. Mereka boleh memotong binatang-binatang untuk kurban bakaran dan persembahan, dan mereka wajib untuk selalu melayani umat.
Eze 44:12  Tetapi karena mereka telah melakukan upacara penyembahan berhala untuk bangsa Israel, dan dengan cara itu menyeret bangsa itu ke dalam dosa, maka Aku, TUHAN Yang Mahatinggi, bersumpah bahwa mereka harus dihukum.
Eze 44:13  Mereka tidak boleh lagi melayani Aku sebagai imam atau menyentuh apa yang suci bagi-Ku, atau memasuki tempat Yang Mahasuci. Itulah hukuman atas perbuatan-perbuatan najis yang telah mereka lakukan.
Eze 44:14  Sekarang mereka Kutugaskan untuk melakukan pekerjaan kasar yang perlu dilaksanakan di Rumah-Ku."
Eze 44:15  TUHAN Yang Mahatinggi berkata, "Pada waktu bangsa Israel meninggalkan Aku, imam-imam Lewi dari keturunan Zadok tetap setia melayani Aku di Rumah-Ku. Sebab itu merekalah yang sekarang harus melayani Aku dan datang ke hadapan-Ku untuk mempersembahkan lemak dan darah dari kurban-kurban.
Eze 44:16  Hanya mereka saja yang boleh masuk ke Rumah-Ku, melayani Aku di mezbah-Ku, dan memimpin kebaktian di Rumah-Ku.
Eze 44:17  Pada waktu mereka memasuki pintu gerbang menuju ke pelataran dalam, mereka harus memakai pakaian linen. Tak boleh mereka memakai pakaian dari wol jika sedang bertugas di pelataran dalam, atau di dalam Rumah-Ku.
Eze 44:18  Supaya tidak kepanasan dan berkeringat, mereka harus memakai celana linen tanpa ikat pinggang, dan ikat kepala dari linen.
Eze 44:19  Kalau mereka pergi ke halaman luar, di mana umat berkumpul, imam-imam itu harus menanggalkan pakaian dinas mereka di dalam kamar-kamar suci di Rumah-Ku, lalu memakai pakaian lain, supaya pakaian khusus itu tidak mencelakakan umat.
Eze 44:20  Para imam tidak boleh mencukur gundul kepala mereka, atau membiarkan rambutnya gondrong. Rambut itu harus dipotong pendek menurut aturan yang ditetapkan.
Eze 44:21  Mereka tidak boleh minum anggur kalau mau masuk ke pelataran dalam.
Eze 44:22  Para imam hanya boleh mengawini gadis-gadis Israel yang masih perawan atau seorang janda imam. Mereka tak boleh mengawini wanita yang diceraikan.
Eze 44:23  Para imam harus mengajarkan kepada umat-Ku untuk membedakan mana yang dikhususkan untuk-Ku dan mana yang tidak, mana yang bersih menurut agama dan mana yang najis.
Eze 44:24  Kalau timbul perselisihan perkara hukum, mereka harus menyelesaikannya sesuai dengan hukum-hukum-Ku. Mereka harus merayakan pesta-pesta agama, menurut aturan-aturan dan hukum-hukum-Ku, serta menghormati Sabat sebagai hari yang khusus untuk-Ku.
Eze 44:25  Seorang imam tidak boleh menajiskan diri dengan menyentuh mayat, kecuali jenazah orang tuanya, anaknya, saudaranya laki-laki atau saudaranya perempuan yang belum kawin.
Eze 44:26  Setelah ia bersih kembali, ia harus menunggu tujuh hari lagi,
Eze 44:27  lalu pergi ke pelataran dalam Rumah-Ku, dan mempersembahkan kurban untuk membersihkan dirinya supaya ia boleh bertugas lagi di Rumah-Ku. Aku, TUHAN Yang Mahatinggi telah berbicara.
Eze 44:28  Para imam menerima jabatan imam itu sebagai bagian bagi mereka dari apa yang Kuberikan kepada Israel turun-temurun. Mereka tidak boleh memiliki apa-apa di Israel: Akulah harta mereka.
Eze 44:29  Mereka harus hidup dari persembahan gandum, kurban penghapus dosa, dan kurban ganti rugi; dan segala sesuatu yang telah dikhususkan untuk-Ku, harus diberikan kepada mereka.
Eze 44:30  Para imam harus diberi bagian yang paling baik dari segala macam hasil panen yang pertama, dan dari apa saja yang dipersembahkan kepada-Ku. Setiap kali bangsa Israel membuat roti, maka roti yang pertama adalah untuk para imam. Maka Aku akan memberkati rumah-rumah bangsa Israel.
Eze 44:31  Burung dan binatang-binatang lain yang mati dengan sendirinya, atau yang diterkam oleh binatang buas, tidak boleh dimakan oleh para imam."
Eze 45:1  Sebelum tanah itu dibagi-bagikan untuk menjadi tanah pusaka bagi setiap suku, harus dipisahkan satu bagian untuk TUHAN. Tanah itu panjangnya harus 12,5 kilometer, dan lebarnya 10 kilometer. Seluruh daerah itu dinyatakan suci.
Eze 45:2  Untuk Rumah TUHAN disediakan dari tanah itu sebagian yang persegi empat, berukuran 250 meter setiap sisinya, dan dikelilingi lapangan selebar 25 meter dan di situlah harus didirikan Rumah TUHAN.
Eze 45:3  Separuh dari tanah itu, yang panjangnya 12,5 kilometer, dan lebarnya 5 kilometer harus disediakan untuk membangun Rumah TUHAN, yaitu daerah yang paling suci.
Eze 45:4  Tanah itu suci, dan disediakan bagi para imam yang bertugas menyelenggarakan kebaktian di Rumah TUHAN. Sebagian dari tanah itu akan menjadi tempat untuk rumah mereka, dan sebagian lagi untuk Rumah TUHAN.
Eze 45:5  Separuh lagi dari tanah itu sebesar 12,5 kilometer kali 5 kilometer harus disediakan untuk orang-orang Lewi yang bekerja di dalam Rumah TUHAN. Di situ akan didirikan kota-kota di mana mereka boleh tinggal.
Eze 45:6  Di samping tanah yang suci itu, harus dipisahkan sebidang tanah yang panjangnya dua belas setengah kilometer dan lebarnya dua setengah kilometer, untuk sebuah kota yang boleh didiami siapa saja dari bangsa Israel.
Eze 45:7  Untuk penguasa yang memerintah harus juga disediakan tanah. Tanah itu terbentang dari batas barat daerah yang dikhususkan untuk TUHAN itu, sampai ke Laut Tengah, dan dari batas timur daerah yang dikhususkan itu, sampai ke perbatasan timur negeri itu. Jadi panjangnya sama dengan panjang dari salah satu bagian tanah yang diberikan kepada suku-suku Israel.
Eze 45:8  Tanah itu menjadi milik penguasa yang memerintah di tanah Israel, supaya ia tidak menindas umat-Ku lagi, melainkan memberikan daerah yang selebihnya itu menjadi milik suku-suku Israel.
Eze 45:9  TUHAN Yang Mahatinggi berkata, "Cukuplah sudah dosa-dosamu, hai para penguasa Israel! Hentikanlah tindakanmu yang sewenang-wenang dan menindas rakyat itu. Lakukanlah apa yang benar dan adil. Sekali-kali jangan lagi kamu mengusir umat-Ku dari tanah mereka. Aku, TUHAN Yang Mahatinggi memerintahkan itu kepadamu.
Eze 45:10  Pakailah timbangan dan takaran yang benar:
Eze 45:11  Satu efa untuk menakar bahan kering harus sama isinya dengan satu bat untuk menakar bahan cair. Patokannya haruslah satu homer. Jadi, takaran-takaran itu harus begini: 1 homer sama dengan 10 efa sama dengan 10 bat.
Eze 45:12  Timbanganmu harus begini: 1 syikal sama dengan 20 gera, 1 mina sama dengan 60 syikal.
Eze 45:13  Inilah dasar ukuran untuk menetapkan persembahan-persembahan: gandum-1/60 bagian dari hasil panenanmu, jawawut-1/60 bagian dari hasil panenanmu, minyak zaitun-1/100 bagian dari hasil pohon-pohon, (Ukurlah menurut bat: 10 bat sama dengan 1 homer sama dengan 1 kor) domba-1 ekor domba dari setiap 200 ekor. Kamu harus mempersembahkan kurban gandum, kurban bakaran, dan kurban perdamaian, supaya dosa-dosamu diampuni. Aku, TUHAN Yang Mahatinggi yang memberi perintah untuk itu.
Eze 45:16  Seluruh penduduk negeri harus membawa persembahan-persembahan itu kepada penguasa yang memerintah di Israel.
Eze 45:17  Dialah yang wajib membeli ternak untuk kurban bakaran dan bahan-bahan untuk kurban gandum dan kurban anggur bagi seluruh bangsa Israel pada hari-hari raya, Perayaan Bulan Baru, hari-hari Sabat dan hari-hari raya lain. Dia harus menyediakan kurban pengampunan dosa, kurban gandum, kurban bakaran, dan kurban perdamaian untuk meminta keampunan bagi bangsa Israel."
Eze 45:18  TUHAN Yang Mahatinggi berkata kepada umat-Nya, "Pada tanggal satu bulan satu, kamu harus menyucikan Rumah-Ku dengan mempersembahkan seekor sapi jantan muda yang tidak ada cacatnya.
Eze 45:19  Imam harus mengambil darah dari kurban penghapus dosa itu, lalu menyapukannya pada tiang-tiang pintu di Rumah-Ku, juga pada keempat sudut mezbah, dan pada tiang-tiang pintu gerbang yang menuju ke pelataran dalam.
Eze 45:20  Buatlah begitu juga pada tanggal tujuh bulan itu, untuk setiap orang yang telah berdosa dengan tidak disengaja atau karena tidak tahu. Dengan cara itu kamu akan menjaga kesucian Rumah-Ku.
Eze 45:21  Pada tanggal empat belas bulan satu, kamu harus merayakan Pesta Paskah. Tujuh hari lamanya semua orang harus makan roti yang tidak beragi.
Eze 45:22  Pada hari yang pertama, penguasa yang memerintah harus mempersembahkan seekor sapi jantan untuk kurban penghapus dosa bagi dirinya dan bagi seluruh rakyat.
Eze 45:23  Selama perayaan yang berlangsung tujuh hari itu, setiap hari ia harus mempersembahkan kepada TUHAN tujuh ekor sapi jantan dan tujuh ekor domba jantan yang tidak ada cacatnya, lalu membakarnya utuh-utuh. Setiap hari ia juga harus mempersembahkan seekor kambing jantan untuk kurban pengampunan dosa.
Eze 45:24  Selain itu, harus dipersembahkan gandum sebanyak 17,5 liter untuk setiap sapi jantan serta tiga liter minyak zaitun untuk setiap domba jantan.
Eze 45:25  Pada Pesta Pondok Daun, yang dimulai pada tanggal lima belas bulan tujuh, setiap hari selama tujuh hari penguasa harus mempersembahkan kurban-kurban yang sama, yaitu kurban pengampunan dosa, kurban bakaran, dan kurban gandum serta minyak zaitun."
Eze 46:1  TUHAN Yang Mahatinggi berkata, "Pintu gerbang timur yang menuju ke pelataran dalam harus tetap tertutup selama keenam hari kerja. Gerbang itu hanya boleh dibuka pada hari Sabat dan Perayaan Bulan Baru.
Eze 46:2  Pada hari-hari itu penguasa harus masuk dari pelataran luar melalui ruang besar lalu berdiri di samping tiang gerbang, sementara para imam mempersembahkan kurban bakaran dan kurban perdamaian dari penguasa. Di pintu gerbang itu ia harus sujud dan menyembah TUHAN, lalu pergi. Kemudian sampai malam, gerbang itu tidak boleh ditutup.
Eze 46:3  Di depan gerbang itu juga, seluruh rakyat harus sujud menyembah TUHAN setiap hari Sabat dan Perayaan Bulan Baru.
Eze 46:4  Pada hari Sabat penguasa harus mempersembahkan kepada TUHAN kurban bakaran yang terdiri dari enam ekor anak domba dan seekor domba jantan, semuanya yang tidak ada cacatnya.
Eze 46:5  Bersama dengan domba jantan itu harus dipersembahkan juga kurban gandum sebanyak 17,5 liter, dan bersama dengan setiap anak domba, harus dipersembahkan kurban gandum sebanyak yang diinginkannya. Setiap kurban gandum harus disertai dengan persembahan tiga liter minyak zaitun.
Eze 46:6  Pada Perayaan Bulan Baru persembahan itu harus berupa seekor sapi jantan, enam ekor anak domba, dan seekor domba jantan, semuanya yang tidak ada cacatnya.
Eze 46:7  Bersama dengan setiap sapi jantan dan kambing jantan, ia harus mempersembahkan gandum sebanyak 17,5 liter, dan bersama dengan setiap anak domba, berapa saja menurut keinginannya. Setiap kurban gandum harus disertai dengan persembahan tiga liter minyak zaitun.
Eze 46:8  Jika penguasa hendak beribadat di Rumah-Ku, ia harus masuk melalui ruang besar, dan pulangnya melalui jalan itu juga.
Eze 46:9  Sebaliknya, jika pada hari-hari besar rakyat datang untuk beribadat kepada TUHAN, maka mereka yang masuk melalui gerbang utara, pulangnya harus melalui gerbang selatan setelah mereka selesai beribadat. Dan mereka yang masuk melalui gerbang selatan, pulangnya harus melalui gerbang utara. Orang-orang dilarang pulang lewat jalan yang dilalui pada waktu masuk; mereka harus pulang melalui gerbang yang di seberangnya.
Eze 46:10  Penguasa harus masuk ke Rumah TUHAN bersama-sama dengan rakyat dan pulangnya pun harus bersama-sama juga.
Eze 46:11  Pada hari-hari pesta dan perayaan, bersama setiap sapi jantan dan setiap domba jantan harus dipersembahkan gandum sebanyak 17,5 liter, dan bersama dengan setiap anak domba, berapa saja menurut keinginan orang yang beribadat. Setiap kurban gandum harus disertai dengan persembahan tiga liter minyak zaitun.
Eze 46:12  Jika penguasa ingin memberi persembahan sukarela kepada TUHAN, baik berupa kurban bakaran maupun kurban perdamaian, maka gerbang timur yang menuju ke halaman dalam harus dibuka untuk dia. Dia harus mempersembahkan kurban itu dengan cara yang sama seperti yang dilakukannya pada hari Sabat, tetapi segera setelah ia pulang, gerbang itu harus ditutup lagi."
Eze 46:13  TUHAN berkata, "Setiap hari pada waktu pagi, seekor anak domba yang berumur satu tahun dan yang tidak ada cacatnya harus dipersembahkan untuk kurban bakaran kepada TUHAN.
Eze 46:14  Selain itu, setiap pagi harus dipersembahkan juga dua kilogram gandum dan satu liter minyak zaitun untuk dicampurkan dengan gandum itu. Itulah kurban gandum yang setiap hari harus dipersembahkan kepada TUHAN; aturan itu berlaku untuk selama-lamanya.
Eze 46:15  Setiap pagi, harus selalu dipersembahkan kepada TUHAN: seekor anak domba, kurban gandum dan minyak zaitun."
Eze 46:16  TUHAN Yang Mahatinggi memberi perintah, kata-Nya, "Jika penguasa yang memerintah memberikan sebagian dari tanah pusakanya kepada seorang putranya, tanah itu menjadi milik putranya dan merupakan harta keluarga.
Eze 46:17  Tetapi jika penguasa itu memberikan sebagian dari tanah pusakanya kepada seorang hambanya, tanah itu akan kembali menjadi milik penguasa itu pada Tahun Pengembalian. Tanah itu adalah miliknya dan hanya dia dengan putra-putranya saja yang dapat memiliki tanah itu seterusnya.
Eze 46:18  Penguasa itu tidak boleh mengambil sedikit pun dari harta pusaka rakyatnya. Hanya tanahnya sendiri yang boleh diberikannya kepada putra-putranya sehingga tak seorang pun dari umat-Ku dapat dipisahkan dari hak miliknya."
Eze 46:19  Kemudian laki-laki itu membawa aku melalui pintu di samping gerbang, ke kamar-kamar suci yang menghadap ke utara. Kamar-kamar itu disediakan bagi para imam. Laki-laki itu menunjuk ke ruangan di ujung sebelah barat,
Eze 46:20  dan berkata, "Di tempat ini para imam harus merebus kurban penghapus dosa dan kurban ganti rugi serta memanggang kurban gandum, supaya sedikit pun dari kurban yang suci itu tidak dibawa ke pelataran luar sehingga mencelakakan rakyat."
Eze 46:21  Kemudian aku dibawanya ke pelataran luar dan diperlihatkannya kepadaku bahwa di keempat sudutnya, masing-masing ada pelataran yang lebih kecil, panjangnya 20 meter dan lebarnya 15 meter.
Eze 46:23  Setiap halaman itu dikelilingi oleh tembok batu, dan pada tembok itu ada tungku-tungku.
Eze 46:24  Kata laki-laki itu kepadaku, "Inilah dapur-dapur tempat pelayan-pelayan Rumah TUHAN merebus kurban-kurban persembahan rakyat."
Eze 47:1  Laki-laki itu membawa aku kembali ke pintu masuk Rumah TUHAN. Kulihat air mengalir dari bawah pintu itu ke arah timur, seperti Rumah TUHAN pun menghadap ke timur. Air itu mengalir dari bawah ambang pintu Rumah TUHAN bagian selatan melalui sisi selatan dari mezbah.
Eze 47:2  Kemudian laki-laki itu membawa aku keluar lewat gerbang utara dan menuntun aku ke gerbang timur. Sungai kecil itu mengalir dari sisi selatan gerbang itu.
Eze 47:3  Sambil mengukur dengan tongkat pengukurnya, laki-laki itu membawa aku mengikuti arus sungai itu ke timur sejauh 500 meter, lalu menyuruh aku menyeberang. Pada tempat itu airnya hanya setinggi mata kaki.
Eze 47:4  Kemudian ia mengukur 500 meter lagi, dan di situ airnya sampai ke lutut. Ia mengukur 500 meter lagi, dan airnya mencapai pinggang.
Eze 47:5  Sekali lagi ia mengukur sejauh 500 meter, dan di situ airnya terlalu dalam untuk dilalui dengan berjalan kaki. Jika orang mau menyeberang haruslah ia berenang.
Eze 47:6  Lalu kata orang itu kepadaku, "Hai manusia fana, perhatikanlah ini semua baik-baik." Kemudian ia membawa aku kembali menyusuri sungai,
Eze 47:7  dan aku melihat bahwa pada kedua tepi sungai itu ada banyak sekali pohon.
Eze 47:8  Ia berkata kepadaku, "Sungai ini mengalir ke timur melalui negeri ini dan melalui Lembah Sungai Yordan menuju ke Laut Mati. Jika airnya sampai ke dalam Laut Mati, maka air laut yang asin itu digantinya dengan air tawar.
Eze 47:9  Ke mana pun sungai ini mengalir, di situ pula akan terdapat segala macam ikan dan binatang lain. Air Laut Mati akan dibuatnya menjadi tawar, dan ke mana pun air itu mengalir, di situ akan ada kehidupan.
Eze 47:10  Dari sumber-sumber air di En-Gedi sampai ke sumber-sumber air di En-Eglaim, bagian-bagian tepinya akan ramai dengan para nelayan yang menjemur jala mereka di situ. Segala macam ikan akan ada di dalam air itu, seperti yang ada dalam Laut Tengah.
Eze 47:11  Tetapi air di rawa-rawa dan paya-paya di sepanjang pantai itu tidak akan menjadi tawar, melainkan tetap merupakan sumber garam.
Eze 47:12  Di sepanjang kedua tepi sungai itu tumbuh segala macam pohon buah-buahan. Daun pohon itu tidak pernah layu, dan buah-buahnya tidak pernah habis. Pohon-pohon itu akan menghasilkan buah-buahan yang baru setiap bulan, sebab air yang mengairinya mengalir dari Rumah TUHAN. Buah-buahnya menjadi makanan dan daun-daunnya berguna untuk obat."
Eze 47:13  TUHAN Yang Mahatinggi berkata, "Inilah batas-batas tanah yang akan dibagi-bagikan kepada keduabelas suku Israel. Suku Yusuf menerima dua bagian.
Eze 47:14  Aku telah bersumpah bahwa tanah itu akan Kuberikan kepada nenek moyangmu. Sebab itu, ambillah tanah itu dan bagilah dengan adil di antara kamu.
Eze 47:15  Perbatasan di sebelah utara dimulai dari Laut Tengah menuju ke timur sampai ke kota Hetlon, ke arah Hamat, terus ke kota Zedad,
Eze 47:16  ke Berota dan Sibraim (kota-kota itu terletak di antara kota Damsyik dan kota Hamat), kemudian ke kota Hazar-Enan, di perbatasan kota Hauran.
Eze 47:17  Jadi, perbatasan sebelah utara dimulai dari Laut Tengah sampai ke timur, ke kota Hazar-Enan; maka daerah Damsyik dan Hamat ada di sebelah utara perbatasan itu.
Eze 47:18  Perbatasan di sebelah timur menuju ke selatan, dari suatu tempat antara daerah Damsyik dan daerah Hauran. Jadi Sungai Yordan merupakan batas antara tanah Israel di sebelah barat dan Gilead di sebelah timur, sampai ke Tamar di pantai Laut Mati.
Eze 47:19  Perbatasan di sebelah selatan menuju ke barat daya, dari Tamar ke mata air Kades-Meriba, lalu ke barat laut sepanjang perbatasan Mesir sampai ke Laut Tengah.
Eze 47:20  Perbatasan di sebelah barat dibentuk oleh Laut Tengah, lalu menuju ke utara ke sebuah tempat di sebelah barat jalan masuk ke Hamat.
Eze 47:21  Bagilah tanah itu di antara kamu, menurut suku-suku Israel;
Eze 47:22  dan tanah itu adalah milikmu untuk selama-lamanya. Orang-orang asing yang menetap di tengah-tengahmu, dan mempunyai anak-anak yang lahir di situ, harus juga mendapat bagian dari tanah itu. Mereka harus dianggap sebagai warga penuh dari bangsa Israel, dan mereka harus turut diundi untuk mendapat tanah pada waktu diadakan pembagian tanah di antara suku-suku Israel.
Eze 47:23  Setiap penduduk asing harus diberi tanah bersama suku Israel di mana ia tinggal. Aku, TUHAN Yang Mahatinggi telah berbicara."
Eze 48:1  Perbatasan tanah di sebelah utara memanjang ke timur, mulai dari Laut Tengah ke arah kota Hetlon, jalan masuk ke Hamat, kota Hazar-Enan, sampai ke perbatasan di antara kota Damsyik dan kota Hamat. Suku-suku berikut ini masing-masing harus mendapat satu bagian dari tanah yang meluas dari perbatasan timur menuju ke barat sampai ke Laut Tengah, dalam urutan sebagai berikut dari utara ke selatan: Dan, Asyer, Naftali, Manasye, Efraim, Ruben, Yehuda.
Eze 48:8  Bagian berikut dari tanah itu harus dipisahkan untuk penggunaan khusus. Panjangnya harus 12,5 kilometer dari batas utara sampai ke selatan. Batas-batas timur dan barat harus sama panjangnya dengan bagian-bagian yang diberikan kepada suku-suku Israel. Rumah TUHAN akan didirikan di bagian itu.
Eze 48:9  Sebagian dari daerah itu dikhususkan bagi TUHAN, yaitu sebidang tanah di bagian pusat, panjangnya 12,5 kilometer dan lebarnya 10 kilometer.
Eze 48:10  Di bagian tengah tanah itu akan didirikan Rumah TUHAN. Para imam harus mendapat bagian dari daerah yang suci itu. Bagian mereka ialah 12,5 kilometer dari sisi timur ke barat, dan lima kilometer dari sisi utara ke selatan.
Eze 48:11  Daerah yang khusus itu akan menjadi milik imam-imam keturunan Zadok. Mereka telah melayani TUHAN dengan setia dan tidak seperti orang Lewi yang sesuku dengan mereka, yang turut berbuat dosa bersama dengan orang-orang Israel lainnya.
Eze 48:12  Sebab itu kepada keturunan Zadok itu harus diberikan tanah khusus di samping tanah yang akan diberikan kepada orang Lewi. Dan tanah khusus itu sangat suci.
Eze 48:13  Tanah orang-orang Lewi itu panjangnya 12,5 kilometer dari timur ke barat, dan lebarnya 5 kilometer dari utara ke selatan dan letaknya sejajar dengan tanah para imam.
Eze 48:14  Sejengkal pun dari tanah itu tak boleh mereka jual atau mereka tukarkan; bagian yang paling baik dari tanah Israel tidak boleh pindah ke tangan orang lain, sebab tanah itu sudah dikhususkan bagi TUHAN dan menjadi milik-Nya.
Eze 48:15  Di sebelah selatan dari tanah khusus itu, masih ada sisa yang panjangnya 12,5 kilometer, dan lebarnya dua setengah kilometer. Bagian itu tidak suci dan boleh dipakai rakyat untuk tempat tinggal dan untuk padang rumput, di bagian tengahnya harus dibangun sebuah kota.
Eze 48:16  Kota itu harus berbentuk persegi empat, setiap sisinya berukuran dua seperempat kilometer.
Eze 48:17  Di sekeliling seluruh kota itu harus ada tanah lapang yang lebarnya 125 meter.
Eze 48:18  Di sebelah timur dan barat kota itu ada tanah lapang yang masing-masing berukuran lima kali dua setengah kilometer. Kedua tanah lapang itu harus dipakai untuk tanah pertanian oleh penduduk kota.
Eze 48:19  Orang-orang dari semua suku Israel yang mendiami kota itu boleh mengerjakan tanah itu.
Eze 48:20  Jadi, seluruh daerah itu, termasuk tanah yang khusus dan tanah untuk kota, berbentuk persegi empat berukuran 12,5 kilometer pada setiap sisinya.
Eze 48:21  Tanah di sebelah barat dan timur dari daerah khusus yang disediakan untuk Rumah TUHAN, para imam, orang Lewi dan kota itu, adalah untuk penguasa yang memerintah. Di sebelah timur tanah itu meluas sampai ke perbatasan timur, dan di sebelah barat sampai ke Laut Tengah, dan terletak di antara tanah Yehuda dan tanah Benyamin.
Eze 48:23  Tanah yang ada di sebelah selatan dari daerah khusus itu, adalah untuk suku-suku yang lainnya. Luas tanah itu mulai dari perbatasan timur ke arah barat sampai ke Laut Tengah, dan dibagi menurut susunan ini dari utara ke selatan: Benyamin, Simeon, Isakhar, Zebulon, Gad.
Eze 48:28  Di sebelah selatan tanah Gad terdapat perbatasan yang mulai dari barat daya Tamar lewat mata air Kades-Meriba, lalu ke arah barat laut sepanjang perbatasan Mesir sampai ke Laut Tengah.
Eze 48:29  TUHAN Yang Mahatinggi berkata, "Begitulah tanah itu harus dibagi-bagi kepada suku-suku Israel."
Eze 48:30  Pintu masuk ke kota Yerusalem semuanya ada dua belas. Keempat temboknya masing-masing berukuran dua seperempat kilometer, dan mempunyai tiga gerbang. Tiap-tiap gerbang dinamakan menurut nama salah satu suku Israel. Gerbang-gerbang di utara dinamakan Ruben, Yehuda dan Lewi; gerbang-gerbang di tembok timur dinamakan Yusuf, Benyamin, serta Dan; yang ada di tembok selatan terdapat nama-nama Simeon, Isakhar dan Zebulon; sedangkan yang di tembok barat diberi nama Gad, Asyer, dan Naftali.
Eze 48:35  Seluruh tembok itu panjangnya sembilan kilometer. Dan mulai saat ini orang menamakan kota itu: "TUHAN ADA DI SITU!"


\end{document}