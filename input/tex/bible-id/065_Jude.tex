\begin{document}

\title{Yudas}


\chapter{1}

\par 1 Saudara-saudara yang telah dipanggil dan yang sangat dikasihi oleh Allah Bapa, serta dijaga untuk Yesus Kristus. Sebagai hamba Yesus Kristus, saya--Yudas--saudara dari Yakobus,
\par 2 mengharap: Semoga Allah memberi berkat, rahmat dan sejahtera kepadamu dengan berlimpah-limpah.
\par 3 Saudara-saudara yang tercinta! Saya sungguh-sungguh bermaksud menulis surat kepadamu mengenai keselamatan yang kita sama-sama sudah alami. Tetapi sekarang saya merasa terdorong untuk menasihati kalian dengan surat ini supaya kalian terus berjuang untuk iman yang sudah satu kali diberikan Allah untuk selama-lamanya kepada umat-Nya.
\par 4 Sebab tanpa kita sadari, ada oknum-oknum tertentu yang menyelusup masuk ke tengah-tengah kita. Mereka orang-orang bejat yang memutarbalikkan berita tentang rahmat Allah kita, untuk mendapat kesempatan melampiaskan hawa nafsu mereka. Dan mereka menolak Yesus Kristus, satu-satunya Penguasa dan Tuhan kita. Sejak dahulu hukuman untuk mereka sudah dinubuatkan dalam Alkitab.
\par 5 Semuanya itu kalian sudah tahu. Namun saya ingin mengingatkan kalian mengenai bagaimana Tuhan menyelamatkan umat Israel dari negeri Mesir, tetapi kemudian membinasakan orang-orang yang tidak percaya di antara mereka.
\par 6 Ingatlah juga para malaikat yang melampaui batas-batas kekuasaan mereka, sehingga pergi meninggalkan tempat tinggal mereka. Allah membelenggu malaikat-malaikat itu dengan rantai abadi, di dalam tempat yang gelap di bawah bumi. Di sana mereka ditahan terus sampai hukuman dijatuhkan ke atas mereka pada Hari yang hebat itu nanti.
\par 7 Ingatlah juga kota Sodom dan Gomora serta kota-kota di sekitarnya, yang penduduknya melakukan hal-hal seperti yang dilakukan oleh malaikat-malaikat tersebut. Mereka melakukan hal-hal yang cabul dan bejat, sehingga mereka disiksa dengan hukuman api yang kekal, untuk dijadikan peringatan bagi semua orang.
\par 8 Demikian juga oknum-oknum itu berkhayal-khayal sampai mereka berbuat dosa terhadap badan mereka sendiri. Mereka memandang rendah kekuasaan Allah dan menghina para makhluk yang mulia di surga.
\par 9 Mikhael sendiri, yang mengepalai malaikat-malaikat, tidak pernah berbuat seperti itu. Pada waktu ia bertengkar dengan Iblis dalam perselisihan tentang siapa yang akan mendapat mayat Musa, Mikhael tidak berani menghakimi Iblis dengan kata-kata penghinaan. Mikhael hanya berkata, "Tuhan akan membentak engkau!"
\par 10 Tetapi dengan penghinaan-penghinaan, orang-orang itu menyerang segala sesuatu yang mereka tidak mengerti. Mereka sama seperti binatang-binatang yang tidak berakal, yang mengetahui hal-hal dengan nalurinya. Tetapi justru hal-hal itulah yang menyebabkan kehancuran mereka.
\par 11 Celaka mereka! Mereka telah mengikuti jalan yang ditempuh oleh Kain. Agar mendapat uang, mereka menjerumuskan diri dalam lembah kesesatan, sama seperti Bileam. Dan mereka pun dihancurkan karena memberontak seperti Korah.
\par 12 Pada waktu kalian mengadakan pesta makan, sikap orang-orang itu memuakkan di antaramu. Sebab mereka makan dengan rakus tanpa malu-malu, dan hanya mementingkan diri sendiri. Mereka seperti awan yang ditiup oleh angin, tetapi tidak menurunkan hujan. Mereka juga seperti pohon yang tidak menghasilkan buah walaupun musim buah-buahan; pohon-pohon yang telah dicabut akarnya dan sudah mati sama sekali.
\par 13 Sama seperti ombak yang ganas di laut menimbulkan buih, begitu juga mereka membuihkan perbuatan-perbuatan yang memalukan. Mereka adalah seperti bintang-bintang yang mengembara dan yang sudah disediakan tempatnya oleh Allah di dalam kegelapan yang paling dahsyat untuk selama-lamanya.
\par 14 Henokh, keturunan ketujuh dari Adam, dahulu pernah menubuatkan tentang orang-orang itu. Henokh berkata, "Lihat, Tuhan akan datang dengan beribu-ribu malaikat-Nya yang suci
\par 15 untuk menghakimi semua orang. Tuhan akan menghukum orang-orang jahat karena perbuatan-perbuatan mereka yang bejat. Dan Ia akan menghukum orang berdosa dan durhaka karena semua kata penghinaan yang diucapkannya."
\par 16 Orang-orang itu selalu saja menggerutu dan menyalahkan orang lain. Mereka hidup menurut keinginan-keinginan mereka yang berdosa. Mereka membual mengenai diri sendiri, dan menjilat orang supaya mendapat keuntungan.
\par 17 Tetapi Saudara-saudaraku yang tercinta, ingatlah akan apa yang dahulu dikatakan oleh rasul-rasul Tuhan kita Yesus Kristus.
\par 18 Mereka memberitahukan terlebih dahulu kepadamu bahwa menjelang akhir zaman akan muncul orang-orang yang akan mengejek kalian, yaitu orang-orang yang hidup menurut keinginan-keinginan mereka yang berdosa.
\par 19 Mereka inilah orang-orang yang menimbulkan perpecahan. Mereka tidak dikuasai oleh Roh Allah, melainkan oleh keinginan-keinginan tabiat duniawi.
\par 20 Tetapi kalian, Saudara-saudaraku, binalah terus hidupmu berdasarkan percayamu kepada Yesus Kristus. Imanmu itu sangat suci. Sementara itu berdoalah dengan kuasa Roh Allah,
\par 21 dan hiduplah selalu di dalam naungan kasih Allah selama kalian menantikan rahmat Yesus Kristus Tuhan kita, yang akan memberikan kepadamu hidup sejati dan kekal.
\par 22 Terhadap orang-orang yang bimbang hatinya, hendaklah kalian menunjukkan belas kasihan.
\par 23 Tariklah orang keluar dari dalam api untuk menyelamatkan mereka. Terhadap orang-orang lainnya hendaklah kalian menunjukkan belas kasihan disertai perasaan takut, tetapi bencilah bahkan pakaian mereka pun yang telah dikotori dengan keinginan mereka yang berdosa.
\par 24 Allah, yang sanggup menjaga supaya kalian tidak jatuh, dan yang sanggup membawamu ke hadirat-Nya yang mulia dengan sukacita dan tanpa cela,
\par 25 Dialah satu-satunya Allah, Penyelamat kita melalui Yesus Kristus Tuhan kita. Dialah Allah yang mulia, agung, berkuasa dan berwibawa sejak dahulu kala sampai sekarang dan untuk selama-lamanya! Amin.


\end{document}