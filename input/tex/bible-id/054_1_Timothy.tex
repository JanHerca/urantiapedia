\begin{document}

\title{1 Timothy}

1Ti 1:1  Timotius, anakku yang sejati di dalam Tuhan! Surat ini dari saya, Paulus, rasul Kristus Yesus atas perintah Allah, Penyelamat kita dan atas perintah Kristus Yesus, harapan kita. Semoga Allah Bapa dan Kristus Yesus Tuhan kita memberi berkat, rahmat dan sejahtera kepadamu.
1Ti 1:3  Saya ingin supaya engkau tetap tinggal di Efesus seperti sudah saya tekankan kepadamu pada waktu saya pergi ke Makedonia. Sebab, di Efesus ada beberapa orang yang menyebarkan ajaran yang tidak benar, dan engkau harus menghentikan mereka.
1Ti 1:4  Katakan kepada mereka supaya jangan lagi menaruh perhatian pada dongeng-dongeng dan cerita-cerita asal-usul yang tidak putus-putusnya. Semuanya itu hanya menimbulkan pertengkaran saja, dan tidak memajukan rencana Allah yang hanya dapat dikenal melalui percaya kepada-Nya.
1Ti 1:5  Tujuan nasihat saya itu adalah supaya orang dapat memiliki hati yang murni dan hati nurani yang suci, serta sungguh-sungguh percaya kepada Tuhan, dan dengan demikian mereka dapat mengasihi sesamanya.
1Ti 1:6  Ada sebagian orang yang sudah tidak hidup seperti itu lagi, dan tersesat dalam perdebatan-perdebatan yang tidak ada gunanya.
1Ti 1:7  Mereka mau menjadi guru-guru agama, padahal mereka sendiri tidak memahami kata-kata yang mereka pakai atau hal-hal yang mereka kemukakan dengan begitu yakin.
1Ti 1:8  Kita tahu bahwa hukum agama adalah baik, kalau digunakan sebagaimana mestinya.
1Ti 1:9  Tentunya harus diingat bahwa hukum dibuat bukan terhadap orang baik, melainkan terhadap para pelanggar hukum, para penjahat, orang bejat, terhadap orang berdosa, orang tidak beragama, orang duniawi, terhadap orang yang membunuh ayah atau ibunya, para pembunuh pada umumnya,
1Ti 1:10  terhadap orang-orang cabul, homoseks, penculik, pembohong; saksi-saksi dusta, dan siapa saja yang membuat hal-hal yang bertentangan dengan ajaran yang benar.
1Ti 1:11  Ajaran itu terdapat di dalam Kabar Baik yang penyebarannya dipercayakan kepada saya, yaitu Kabar Baik dari Allah yang agung dan patut dipuji.
1Ti 1:12  Saya mengucap terima kasih kepada Kristus Yesus Tuhan kita. Ia sudah memberikan kekuatan kepada saya untuk melayani Dia, dan Ia menganggap saya layak untuk tugas itu,
1Ti 1:13  walaupun dahulu saya memfitnah dan menganiaya serta menghina Dia. Tetapi Allah mengasihani saya, karena pada waktu itu saya belum percaya, jadi saya tidak tahu apa yang saya lakukan.
1Ti 1:14  Dan saya malah dilimpahi dengan rahmat Tuhan kita. Ia memberikan kepada saya kemampuan untuk percaya kepada-Nya dan mengasihi sesama manusia. Kemampuan itu diberikan kepada kita semua yang sudah bersatu dengan Kristus Yesus.
1Ti 1:15  Sungguh benar perkataan ini--sebab itu patutlah diterima dan dipercayai sepenuhnya--: "Kristus Yesus datang ke dunia ini untuk menyelamatkan orang berdosa." Dan sayalah orang berdosa yang paling jahat.
1Ti 1:16  Tetapi justru oleh sebab itulah Allah mengasihani saya, supaya Kristus Yesus dapat menunjukkan seluruh kesabaran-Nya terhadap saya, orang yang paling berdosa ini. Dan ini menjadi suatu contoh untuk semua orang yang nanti akan percaya kepada-Nya dan mendapat hidup sejati dan kekal.
1Ti 1:17  Allah, Raja segala zaman, patut dihormati dan dipuji senantiasa, sebab Dialah yang kekal, yang tak nampak dan yang esa! Amin.
1Ti 1:18  Timotius, anakku! Tugas ini saya percayakan kepadamu, mengingat akan pesan Allah yang disampaikan oleh nabi-nabi dalam jemaat mengenai dirimu. Hendaklah pesan itu menjadi bekal bagimu dalam perjuangan yang baik.
1Ti 1:19  Berjuanglah dengan berpegang pada kepercayaan yang benar dan dengan hati nurani yang suci. Ada orang-orang yang tidak peduli akan suara hati nuraninya, sehingga hancurlah kepercayaan mereka.
1Ti 1:20  Di antaranya ialah Himeneus dan Aleksander. Kedua orang itu sudah saya hukum dengan menyerahkannya kepada Iblis, supaya mereka belajar untuk berhenti menghujat Allah.
1Ti 2:1  Pertama-tama, saya minta dengan sangat supaya permohonan, sembahyang, dan doa syafaat serta ucapan terima kasih disampaikan kepada Allah untuk semua orang;
1Ti 2:2  untuk raja-raja dan untuk semua orang yang memegang kekuasaan. Mintalah supaya kita dapat hidup tenang dan tentram untuk Allah dengan kelakuan yang patut.
1Ti 2:3  Itulah yang baik dan menyenangkan hati Allah, Penyelamat kita.
1Ti 2:4  Ia mau supaya semua orang diselamatkan dan mengetahui yang benar.
1Ti 2:5  Sebab, hanya ada satu Allah, dan hanya satu penengah antara Allah dengan manusia, yaitu Kristus Yesus.
1Ti 2:6  Ia seorang manusia yang sudah menyerahkan diri-Nya untuk membebaskan semua orang dari dosa-dosa mereka. Dengan itu Allah menunjukkan pada waktu yang tepat, bahwa Ia ingin agar semua orang diselamatkan.
1Ti 2:7  Dan itulah sebabnya saya disuruh pergi kepada orang-orang bukan Yahudi untuk mengabarkan kepada mereka berita tentang kepercayaan yang benar. Yang saya kemukakan ini adalah benar, saya tidak berdusta.
1Ti 2:8  Saya mau supaya di manapun juga kaum pria berdoa dengan hati yang suci, tanpa kemarahan atau perselisihan.
1Ti 2:9  Hendaklah kaum wanita menghias dirinya dengan sederhana dan memakai pakaian yang sopan. Jangan memakai potongan rambut yang menyolok atau perhiasan-perhiasan emas atau mutiara, atau pakaian yang mahal-mahal.
1Ti 2:10  Sebaliknya, hendaklah wanita menghiasi dirinya dengan perbuatan-perbuatan yang baik sebagaimana yang patut bagi wanita yang beribadat kepada Allah.
1Ti 2:11  Wanita harus belajar dengan berdiam diri dan patuh.
1Ti 2:12  Saya tidak membenarkan wanita mengajar ataupun memerintah laki-laki; mereka harus diam.
1Ti 2:13  Sebab yang pertama-tama diciptakan adalah Adam dan kemudian baru Hawa.
1Ti 2:14  Dan bukannya Adam, melainkan wanitalah yang tertipu, sehingga melanggar perintah Allah.
1Ti 2:15  Meskipun begitu, wanita akan selamat dengan melahirkan anak, asal ia dengan kerendahan hati tetap percaya kepada Kristus dan tetap mengasihi orang lain serta hidup khusus untuk Allah.
1Ti 3:1  Sungguh benar perkataan ini, "Orang yang ingin menjadi penilik jemaat, menginginkan suatu pekerjaan yang sangat berharga."
1Ti 3:2  Seorang penilik jemaat haruslah orang yang tanpa cela, hanya satu istrinya, tahu menahan diri, bijaksana, dan tertib; ia suka menerima orang di rumahnya, dan bisa mengajar orang;
1Ti 3:3  jangan orang yang pemabuk, atau yang suka berkelahi. Sebaliknya, ia harus lemah lembut dan suka akan damai. Ia tidak boleh mata duitan.
1Ti 3:4  Ia harus tahu mengatur rumah tangganya dengan baik, dan mendidik anak-anaknya untuk taat dan hormat kepadanya.
1Ti 3:5  Sebab kalau orang tidak tahu mengatur rumah tangganya sendiri, bagaimanakah ia dapat mengatur jemaat Allah?
1Ti 3:6  Seorang penilik jemaat tidak boleh orang yang baru saja menjadi Kristen, sebab nanti ia menjadi sombong lalu terkutuk seperti Iblis dahulu.
1Ti 3:7  Ia haruslah orang yang punya nama baik dalam masyarakat; sebab kalau tidak begitu, maka ia akan dihina orang, sehingga jatuh ke dalam perangkap Iblis.
1Ti 3:8  Pembantu jemaat haruslah juga orang yang baik dan tulus; tidak mata duitan dan tidak suka minum terlalu banyak anggur.
1Ti 3:9  Mereka harus berpegang teguh dengan hati nurani yang murni pada ajaran kepercayaan Kristen yang sudah dinyatakan oleh Allah.
1Ti 3:10  Hendaknya mereka diuji dahulu, dan kalau ternyata mereka tidak bercela, barulah mereka boleh membantu di dalam jemaat.
1Ti 3:11  Istri mereka pun haruslah juga orang yang baik, bukan orang yang suka menyebarkan desas-desus. Mereka harus tahu menahan diri dan jujur dalam segala hal.
1Ti 3:12  Pembantu jemaat harus mempunyai hanya satu orang istri. Ia harus tahu mengatur anak-anaknya dan rumah tangganya dengan baik.
1Ti 3:13  Orang-orang yang membantu dengan baik di dalam jemaat, akan dihormati dan menjadi semakin berani berbicara mengenai kepercayaan mereka kepada Kristus.
1Ti 3:14  Saya mengharap tidak lama lagi saya akan mengunjungimu. Meskipun begitu, saya menulis juga surat ini kepadamu,
1Ti 3:15  supaya kalau saya terlambat, engkau sudah tahu bagaimana kita harus hidup sebagai keluarga Allah, yakni jemaat Allah yang hidup. Jemaat inilah yang merupakan tiang penegak dan pendukung ajaran yang benar dari Allah.
1Ti 3:16  Tidak seorang pun dapat menyangkal, betapa besarnya rahasia agama kita: Ia menampakkan diri dalam rupa manusia, dan dinyatakan benar oleh Roh Allah, serta dilihat oleh para malaikat. Beritanya dikumandangkan di antara bangsa-bangsa, dan Ia dipercaya di seluruh dunia dan diangkat ke surga.
1Ti 4:1  Roh Allah dengan tegas mengatakan bahwa di masa-masa yang akan datang, sebagian orang akan murtad, mengingkari Kristus. Mereka akan patuh kepada roh-roh yang menyesatkan dan mengikuti ajaran-ajaran roh jahat,
1Ti 4:2  yang disampaikan melalui orang-orang munafik yang membohong. Hati nurani orang-orang itu sudah gelap.
1Ti 4:3  Mereka mengajar orang untuk tidak kawin dan tidak makan makanan tertentu. Padahal makanan itu diciptakan Allah untuk dimakan dengan pengucapan terima kasih kepada-Nya oleh orang-orang yang sudah percaya kepada Kristus dan sudah mengenal ajaran yang benar dari Allah.
1Ti 4:4  Segala sesuatu yang diciptakan oleh Allah itu baik, dan tidak ada satu pun yang harus dianggap haram kalau sudah diterima dengan ucapan terima kasih kepada Allah.
1Ti 4:5  Sebab berkat dari Allah dan doa membuat makanan itu menjadi halal.
1Ti 4:6  Kalau engkau mengajarkan semuanya itu kepada saudara-saudara seiman, engkau akan menjadi pelayan Kristus Yesus yang bekerja dengan baik. Dan engkau akan terus memupuk batinmu dengan perkataan Allah yang kita percayai, dan dengan ajaran-ajaran yang benar yang sudah kauikuti selama ini.
1Ti 4:7  Jauhilah cerita-cerita takhayul yang tidak berguna. Hendaklah kaulatih dirimu untuk kehidupan yang beribadat.
1Ti 4:8  Latihan jasmani sedikit saja gunanya, tetapi latihan rohani berguna dalam segala hal, sebab mengandung janji untuk hidup pada masa kini dan masa yang akan datang.
1Ti 4:9  Hal itu benar dan patut diterima serta dipercayai sepenuhnya.
1Ti 4:10  Itulah sebabnya kita berjuang dan bekerja keras, sebab kita berharap sepenuhnya kepada Allah yang hidup; Ialah Penyelamat semua orang, terutama sekali orang-orang yang percaya.
1Ti 4:11  Ajarkanlah semuanya itu dan suruhlah orang-orang menurutinya.
1Ti 4:12  Janganlah membiarkan seorang pun menganggap engkau rendah karena engkau masih muda. Sebaliknya, hendaklah engkau menjadi teladan bagi orang-orang percaya dalam percakapanmu dan kelakuanmu, dalam cara engkau mengasihi sesama dan percaya kepada Yesus Kristus, dan dengan hidupmu yang murni.
1Ti 4:13  Sementara itu, sampai saya datang nanti, engkau harus bersungguh-sungguh membacakan Alkitab kepada orang-orang, dan mendorong serta mengajar mereka.
1Ti 4:14  Janganlah engkau lalai memakai karunia dari Roh Allah yang diberikan kepadamu pada waktu pemimpin-pemimpin jemaat meletakkan tangan mereka di atas kepalamu, dan nabi-nabi menyampaikan pesan Allah mengenai dirimu.
1Ti 4:15  Kerjakanlah semuanya itu dengan bersungguh-sungguh supaya kemajuanmu dilihat oleh semua orang.
1Ti 4:16  Awasilah dirimu dan awasilah juga pengajaranmu. Hendaklah engkau setia melakukan semuanya itu, sebab dengan demikian engkau akan menyelamatkan baik dirimu sendiri maupun orang-orang yang mendengarmu.
1Ti 5:1  Janganlah engkau memarahi dengan keras orang yang lebih tua daripadamu melainkan ajaklah dia mendengarkan kata-katamu seolah-olah ia bapakmu. Perlakukanlah orang-orang muda sebagai saudara,
1Ti 5:2  dan wanita-wanita tua sebagai ibu. Wanita-wanita muda hendaklah engkau perlakukan sebagai adik, dengan sikap yang murni.
1Ti 5:3  Hormatilah janda-janda yang benar-benar hidup seorang diri.
1Ti 5:4  Tetapi kalau seorang janda mempunyai anak-anak atau cucu-cucu, mereka itulah yang pertama-tama harus diberi pengertian bahwa mereka wajib memperhatikan kaum keluarga mereka, dan membalas budi orang tua dan nenek mereka. Sebab hal itulah menyenangkan hati Allah.
1Ti 5:5  Janda yang betul-betul seorang diri, dan tidak mempunyai siapa-siapa yang dapat memeliharanya, hanya berharap kepada Allah. Siang malam ia terus berdoa kepada Allah untuk minta pertolongan-Nya.
1Ti 5:6  Tetapi janda yang suka hidup bersenang-senang, sudah mati meskipun ia masih hidup.
1Ti 5:7  Sampaikanlah petunjuk-petunjuk ini kepada mereka, supaya mereka hidup tanpa cela.
1Ti 5:8  Tetapi kalau ada orang yang tidak memelihara sanak saudaranya, lebih-lebih keluarganya sendiri, orang itu menyangkal kepercayaannya; ia lebih buruk dari orang yang tidak percaya.
1Ti 5:9  Hanya janda yang memenuhi syarat-syarat berikut ini boleh dimasukkan ke dalam daftar janda-janda: umurnya harus tidak kurang dari enam puluh tahun, dahulu hanya satu kali kawin,
1Ti 5:10  terkenal sebagai wanita yang melakukan hal-hal yang baik, seperti misalnya: sudah membesarkan anak-anaknya dengan baik, senang menerima tamu-tamu menginap di rumahnya, melayani orang-orang percaya, membantu orang-orang yang susah, dan melakukan dengan sungguh-sungguh segala macam pekerjaan yang baik.
1Ti 5:11  Jangan masukkan ke dalam daftar itu janda-janda yang masih muda. Sebab kalau rasa berahi mereka menjadi sangat kuat, mereka mau kawin lagi, dan meninggalkan Kristus.
1Ti 5:12  Maka mereka bersalah sebab memungkiri janji yang diberikan kepadanya.
1Ti 5:13  Lagipula mereka membuang-buang waktu dengan keluar masuk rumah orang. Dan yang lebih buruk lagi, ialah bahwa mereka belajar memfitnah nama orang lain dan mencampuri urusan orang serta membicarakan hal-hal yang tidak patut dibicarakan.
1Ti 5:14  Itulah sebabnya saya mau supaya janda-janda muda kawin lagi, mendapat anak dan mengurus rumah tangga, supaya musuh-musuh kita tidak mendapat kesempatan untuk memburuk-burukkan nama kita.
1Ti 5:15  Sebab ada janda-janda yang sudah tersesat mengikuti Iblis.
1Ti 5:16  Tetapi kalau seorang wanita Kristen mempunyai anggota keluarga yang sudah menjadi janda, ia harus membantu janda-janda itu; jangan membiarkan jemaat yang menanggung mereka. Dengan demikian jemaat dapat membantu janda-janda yang sama sekali tidak punya sanak saudara.
1Ti 5:17  Pemimpin jemaat yang melakukan tugasnya dengan baik, patut diberi penghargaan dua kali lipat, terutama sekali mereka yang rajin berkhotbah dan mengajar.
1Ti 5:18  Sebab dalam Alkitab tertulis, "Sapi yang sedang menginjak-injak gandum untuk melepaskan biji gandum dari bulirnya, janganlah diberangus mulutnya", juga, "Orang yang bekerja, berhak menerima upahnya".
1Ti 5:19  Janganlah menerima tuduhan terhadap seorang pemimpin jemaat, kecuali kalau diperkuat oleh dua orang saksi atau lebih.
1Ti 5:20  Orang yang berbuat dosa, haruslah ditegur di depan seluruh jemaat, supaya yang lain menjadi takut.
1Ti 5:21  Di hadapan Allah, Kristus Yesus, dan malaikat-malaikat yang terpilih, saya minta dengan sangat supaya engkau menuruti petunjuk-petunjuk ini tanpa prasangka terhadap siapa pun, dan tanpa memihak-mihak.
1Ti 5:22  Janganlah terlalu cepat meletakkan tangan atas seseorang guna meresmikan pengangkatannya sebagai pelayan Tuhan. Kalau orang lain berdosa, jangan turut terlibat dalam dosa itu. Jagalah dirimu supaya tetap murni.
1Ti 5:23  Jangan lagi minum air saja; minumlah juga sedikit anggur untuk menolong pencernaanmu, sebab engkau sering sakit-sakit.
1Ti 5:24  Dosa sebagian orang, sebelum diadili, sudah kelihatan dengan jelas. Tetapi dosa orang lain baru diketahui sesudahnya.
1Ti 5:25  Begitu juga perbuatan-perbuatan baik langsung kelihatan. Dan yang tidak mudah kelihatan sekalipun, tidak bisa terus tersembunyi.
1Ti 6:1  Orang-orang Kristen yang menjadi hamba, harus menganggap bahwa tuan mereka patut dihormati, supaya orang tidak dapat memburukkan nama Allah atau pengajaran kita.
1Ti 6:2  Hamba-hamba yang tuannya orang Kristen, tidak boleh meremehkan tuannya karena mereka sama-sama orang Kristen. Malah mereka seharusnya melayani tuan mereka itu dengan lebih baik lagi, sebab tuan yang dilayani dengan baik itu adalah sama-sama orang percaya yang dikasihi. Semuanya ini haruslah engkau ajarkan dan nasihatkan.
1Ti 6:3  Barangsiapa mengajarkan ajaran yang lain daripada itu dan tidak setuju dengan ajaran yang benar dari Tuhan kita Yesus Kristus, dan dengan ajaran agama kita,
1Ti 6:4  adalah orang yang sombong dan tidak tahu apa-apa! Penyakitnya ialah suka berdebat dan bertengkar mengenai istilah-istilah sehingga menimbulkan iri hati, perpecahan, fitnahan, curiga yang tidak baik,
1Ti 6:5  dan perselisihan yang tidak habis-habisnya. Jalan pikiran orang-orang itu sudah buntu dan tidak lagi benar. Mereka menyangka bisa menjadi kaya dari agama.
1Ti 6:6  Memang agama memberikan keuntungan yang besar, kalau orang puas dengan apa yang dipunyainya.
1Ti 6:7  Sebab tidak ada sesuatupun yang kita bawa ke dalam dunia ini, dan tidak ada sesuatupun juga yang dapat kita bawa ke luar!
1Ti 6:8  Jadi, kalau ada makanan dan pakaian, itu sudah cukup.
1Ti 6:9  Tetapi orang yang mau menjadi kaya, tergoda dan terjerat oleh bermacam-macam keinginan yang bodoh dan yang merusak. Keinginan-keinginan itu membuat orang menjadi hancur dan celaka.
1Ti 6:10  Sebab dari cinta akan uang, timbul segala macam kejahatan. Ada sebagian orang yang mengejar uang sehingga sudah tidak menuruti lagi ajaran Kristen, lalu mereka tertimpa banyak penderitaan yang menghancurkan hati mereka.
1Ti 6:11  Tetapi engkau adalah orang milik Allah, jadi engkau harus menjauhi semuanya itu. Berusahalah menjadi orang yang benar di mata Allah, yang mengabdi kepada Allah, percaya kepada Kristus, mengasihi sesama, tabah dalam penderitaan, dan bersikap lemah lembut.
1Ti 6:12  Berjuanglah sungguh-sungguh untuk hidup sebagai orang Kristen supaya engkau merebut hadiah hidup sejati dan kekal. Sebab untuk itulah Allah memanggil engkau pada waktu engkau mengakui di hadapan banyak orang bahwa engkau percaya kepada Kristus.
1Ti 6:13  Dan sekarang, di hadapan Allah yang memberikan nyawa kepada segala sesuatu, dan di hadapan Kristus Yesus yang memberi kesaksian yang benar kepada Pontius Pilatus, saya minta ini daripadamu:
1Ti 6:14  Taatilah semua perintah itu secara murni dan tanpa cela sampai pada hari Tuhan kita Yesus Kristus datang kembali.
1Ti 6:15  Kedatangan-Nya akan ditentukan pada waktu yang tepat oleh Allah sendiri. Ialah Penguasa satu-satunya; Ia agung, Raja segala raja, dan Tuhan segala tuan.
1Ti 6:16  Dialah saja yang tidak bisa dikuasai oleh kematian; Ia hidup di dalam cahaya yang tidak dapat didekati oleh siapapun juga. Tidak ada seorang manusia pun yang pernah melihat-Nya, atau dapat melihat-Nya. Bagi Dialah hormat dan kuasa untuk selama-lamanya! Amin.
1Ti 6:17  Kepada orang-orang yang kaya di dunia ini, hendaklah engkau minta supaya mereka jangan sombong dan jangan berharap kepada barang-barang yang tidak tetap--seperti halnya dengan kekayaan. Mereka harus berharap kepada Allah yang memberikan segala sesuatu kepada kita dengan berlimpah supaya kita menikmatinya.
1Ti 6:18  Mintalah kepada mereka untuk menunjukkan kebaikan, untuk banyak melakukan hal-hal yang baik, murah hati dan suka memberi.
1Ti 6:19  Dengan demikian mereka mengumpulkan harta yang menjadi modal yang baik untuk masa yang akan datang. Dan dengan itu mereka akan mendapat hidup kekal, yakni hidup yang sejati.
1Ti 6:20  Apa yang sudah dipercayakan kepadamu, jagalah itu baik-baik, Timotius! Jauhilah percakapan-percakapan yang tidak berguna dan perdebatan-perdebatan mengenai hal-hal yang secara keliru disebut "Pengetahuan".
1Ti 6:21  Sebab ada orang-orang yang mengatakan bahwa mereka memiliki pengetahuan itu, sehingga menyeleweng dari ajaran-ajaran Kristen yang mereka percaya dahulu. Semoga Tuhan memberkati Saudara-saudara. Hormat kami, Paulus.


\end{document}