\begin{document}

\title{Proverbs}


\chapter{1}

\par 1 Ordsprog af Salomo, Davids Søn, Israels Konge.
\par 2 Af dem skal man lære Visdom forstandig Tale,
\par 3 tage mod Tugt, som gør klog, mod Retfærdighed, Ret og Retsind;
\par 4 de skal give tankeløse Klogskab, ungdommen Kundskab og Kløgt;
\par 5 den vise høre og øge sin Viden, den forstandige vinde sig Levekunst;
\par 6 de skal lære at tyde Ordsprog og Billeder, de vises Ord og Gåder.
\par 7 HERRENs Frygt er Kundskabs begyndelse, Dårer ringeagter Visdom og Tugt.
\par 8 Hør, min Søn, på din Faders Tugt, opgiv ikke din Moders Belæring.
\par 9 thi begge er en yndig Krans til dit Hoved og Kæder til din Hals.
\par 10 Min Søn, sig nej, når Syndere lokker!
\par 11 Siger de: "Kom med, lad os lure på den fromme, lægge Baghold for sagesløs, skyldfri Mand!
\par 12 Som Dødsriget sluger vi dem levende, med Hud og Hår, som for de i Graven.
\par 13 Vi vinder os Gods og Guld, vi fylder vore Huse med Rov.
\par 14 Gør fælles Sag med os; vi har alle fælles Pung!"
\par 15 - min Søn, gå da ikke med dem, hold din Fod fra deres Sti;
\par 16 thi deres Fødder løber efter ondt, de haster for at udgyde Blod.
\par 17 Thi det er unyttigt at udspænde Garnet for alle Fugles Øjne;
\par 18 de lurer på eget Blod, lægger Baghold for eget Liv.
\par 19 Så går det enhver, der attrår Rov, det tager sin Herres Liv.
\par 20 Visdommen råber på Gaden, på Torvene løfter den Røsten;
\par 21 oppe på Murene kalder den, tager til Orde i Byen ved Portindgangene:
\par 22 Hvor længe vil I tankeløse elske Tankeløshed, Spotterne finde deres Glæde i Spot og Dårerne hade kundskab?
\par 23 Vend eder til min Revselse! Se, jeg lader min Ånd udvælde for eder, jeg kundgør eder mine Ord:
\par 24 Fordi jeg råbte og I stod imod, jeg vinked og ingen ænsed det,
\par 25 men I lod hånt om alt mit Råd og tog ikke min Revselse til jer,
\par 26 derfor ler jeg ved eders Ulykke, spotter, når det, I frygter, kommer,
\par 27 når det, I frygter, kommer som Uvejr, når eders Ulykke kommer som Storm, når Trængsel og Nød kommer over jer.
\par 28 Da svarer jeg ej, når de kalder, de søger mig uden at finde,
\par 29 fordi de hadede Kundskab og ikke valgte HERRENs Frygt;
\par 30 mit Råd tog de ikke til sig, men lod hånt om al min Revselse.
\par 31 Frugt af deres Færd skal de nyde og mættes med egne Råd;
\par 32 thi tankeløses Egensind bliver deres Død, Tåbers Sorgløshed bliver deres Undergang;
\par 33 men den, der adlyder mig, bor trygt, sikret mod Ulykkens Rædsel.

\chapter{2}

\par 1 Min Søn, når du tager imod mine ord og gemmer mine Pålæg hos dig,
\par 2 idet du låner Visdom Øre og bøjer dit Hjerte til Indsigt,
\par 3 ja, kalder du på Forstanden og løfter din Røst efter Indsigt,
\par 4 søger du den som Sølv og leder den op som Skatte,
\par 5 da nemmer du HERRENs Frygt og vinder dig Kundskab om Gud.
\par 6 Thi HERREN, han giver Visdom, fra hans Mund kommer Kundskab og Indsigt.
\par 7 Til retsindige gemmer han Lykke, han er Skjold for alle med lydefri Vandel,
\par 8 idet han værner Rettens Stier og vogter sine frommes Vej.
\par 9 Da nemmer du Retfærd, Ret og Retsind, hvert et Spor, som er godt.
\par 10 Thi Visdom kommer i dit Hjerte, og Kundskab er liflig for din Sjæl;
\par 11 Kløgt skal våge over dig, Indsigt være din Vogter -
\par 12 idet den frier dig fra den ondes Vej, fra Folk, hvis Ord kun er vrange, -
\par 13 som går fra de lige Stier for at vandre på Mørkets Veje.
\par 14 som glæder sig ved at gøre ondt og jubler over vrangt og ondt,
\par 15 de, som går krogede Stier og følger bugtede Spor -
\par 16 idet den frier dig fra Andenmands Hustru, fra fremmed Kvinde med sleske Ord,
\par 17 der sviger sin Ungdoms Ven og glemmer sin Guds Pagt;
\par 18 thi en Grav til Døden er hendes Hus, til Skyggerne fører hendes Spor;
\par 19 tilbage vender ingen, som går ind til hende, de når ej Livets Stier
\par 20 at du må vandre de godes Vej og holde dig til de retfærdiges Stier;
\par 21 thi retsindige skal bo i Landet, lydefri levnes deri,
\par 22 men gudløse ryddes af Landet, troløse rykkes derfra.

\chapter{3}

\par 1 Min Søn, glem ikke, hvad jeg har lært dig, dit hjerte tage vare på mine bud!
\par 2 Thi en Række af Dage og Leveår og Lykke bringer de dig.
\par 3 Godhed og Troskab vige ej fra dig, bind dem som Bånd om din Hals, skriv dem på dit Hjertes Tavle!
\par 4 Så finder du Nåde og Yndest i Guds og Menneskers Øjne.
\par 5 Stol på HERREN af hele dit Hjerfe, men forlad dig ikke på din Forstand;
\par 6 hav ham i Tanke på alle dine Veje, så jævner han dine Stier.
\par 7 Hold ikke dig selv for viis, frygt HERREN og vig fra det onde;
\par 8 så får du Helse for Legemet, Lindring for dine Ledemod.
\par 9 Ær med din Velstand HERREN med Førstegrøden af al din Avl;
\par 10 da fyldes dine Lader med Korn, dine Perser svømmer over af Most.
\par 11 Min Søn, lad ej hånt om HERRENs Tugt, vær ikke ked af hans Revselse;
\par 12 HERREN revser den, han elsker, han straffer den Søn, han har kær.
\par 13 Lykkelig den, der har opnået Visdom, den, der vinder sig Indsigt;
\par 14 thi den er bedre at købe end Sølv, bedre at vinde end Guld;
\par 15 den er mere værd end Perler, ingen Klenodier opvejer den;
\par 16 en Række af Dage er i dens højre, i dens venstre Rigdom og Ære;
\par 17 dens Veje er liflige Veje, og alle dens Stier er Lykke;
\par 18 den er et Livets Træ for dem, der griber den, lykkelig den, som holder den fast!
\par 19 HERREN grundlagde Jorden med Visdom, grundfæsted Himlen med Indsigt;
\par 20 ved hans Kundskab brød Strømmene frem, lader Skyerne Dug dryppe ned.
\par 21 Min Søn, tag Vare på Snilde og Kløgt, de slippe dig ikke af Syne;
\par 22 så bliver de Liv for din Sjæl og et yndigt Smykke til din Hals.
\par 23 Da vandrer du trygt din Vej, støder ikke imod med din Fod; -
\par 24 sætter du dig, skal du ikke skræmmes, lægger du dig, skal din Søvn vorde sød;
\par 25 du skal ikke frygte uventet Rædsel, Uvejret, når det kommer over gudløse;
\par 26 thi HERREN skal være din Tillid, han vogter din Fod, så den ikke hildes.
\par 27 Nægt ikke den trængende Hjælp, når det står i din Magt at hjælpe;
\par 28 sig ej til din Næste: "Gå og kom igen, jeg vil give i Morgen!" - såfremt du har det.
\par 29 Tænk ikke på ondt mod din Næste, når han tillidsfuldt bor i din Nærhed.
\par 30 Yp ikke Trætte med sagesløs Mand, når han ikke har voldet dig Men.
\par 31 Misund ikke en Voldsmand, græm dig aldrig over hans Veje;
\par 32 thi den falske er HERREN en Gru; mod retsindig er han fortrolig;
\par 33 i den gudløses Hus er HERRENs Forbandelse, men retfærdiges Bolig velsigner han.
\par 34 Over for Spottere bruger han Spot, men ydmyge giver han Nåde.
\par 35 De vise får Ære til Arv, men Tåber høster kun Skam.

\chapter{4}

\par 1 Hør, I sønner, på en Faders lyt til for at vinde Forstand;
\par 2 thi gavnlig Viden giver jeg jer, slip ej hvad jeg har lært jer.
\par 3 Da jeg var min Faders Dreng, min Moders Kælebarn og eneste,
\par 4 lærte han mig og sagde: Lad dit Hjerte gribe om mine Ord, vogt mine Bud, så skal du leve;
\par 5 køb Visdom, køb Forstand, du glemme det ikke, vend dig ej bort fra min Munds Ord;
\par 6 slip den ikke, så vil den vogte dig, elsk den, så vil den værne dig!
\par 7 Køb Visdom for det bedste, du ejer, køb Forstand for alt, hvad du har;
\par 8 hold den højt, så bringer den dig højt til Vejrs, den bringer dig Ære, når du favner den;
\par 9 den sætter en yndig Krans på dit Hoved; den rækker dig en dejlig Krone.
\par 10 Hør, min Søn, tag imod mine Ord, så bliver dine Leveår mange.
\par 11 Jeg viser dig Visdommens Vej, leder dig ad Rettens Spor;
\par 12 når du går, skal din Gang ej hæmmes, og løber du, snubler du ikke;
\par 13 hold fast ved Tugt, lad den ikke fare, tag Vare på den, thi den er dit Liv.
\par 14 Kom ikke på gudløses Sti, skrid ej frem ad de ondes Vej.
\par 15 sky den og følg den ikke, vig fra den, gå udenom;
\par 16 thi de sover ikke, når de ikke har syndet, og Søvnen flyr dem, når de ej har bragt Fald.
\par 17 Thi de æder Gudløsheds Brød og drikker Urettens Vin.
\par 18 men retfærdiges Sti er som strålende Lys, der vokser i Glans til højlys Dag:
\par 19 Gudløses Vej er som Mørket, de skønner ej, hvad de snubler over,
\par 20 Mærk dig, min Søn, mine Ord, bøj Øret til, hvad jeg siger;
\par 21 det slippe dig ikke af Syne, du vogte det dybt i dit Hjerte;
\par 22 thi det er Liv for dem, der finder det, Helse for alt deres Kød.
\par 23 Vogt dit Hjerte mer end alt andet, thi derfra udspringer Livet.
\par 24 Hold dig fra Svig med din Mund, lad Læbernes Falskhed være dig fjern.
\par 25 Lad dine Øjne se lige ud, dit Blik skue lige frem;
\par 26 gå ad det lige Spor, lad alle dine Veje sigte mod Målet;
\par 27 bøj hverken til højre eller venstre, lad Foden vige fra ondt!

\chapter{5}

\par 1 Mærk dig, min Søn, min Visdom, bøj til min Indsigt dit Øre,
\par 2 at Kløgt må våge øver dig, Læbernes kundskab vare på dig.
\par 3 Thi af Honning drypper den fremmedes Læber, glattere end Olie er hendes Gane;
\par 4 men til sidst er hun besk som Malurt, hvas som tveægget Sværd;
\par 5 hendes Fødder styrer nedad mod Døden, til Dødsriget stunder hendes Fjed;
\par 6 hun følger ej Livets Vej, hendes Spor er bugtet, hun ved det ikke.
\par 7 Hør mig da nu, min Søn, vig ikke fra min Munds Ord!
\par 8 Lad din Vej være langt fra hende, kom ej hendes Husdør nær,
\par 9 at du ikke må give andre din Ære, en grusom Mand dine År.
\par 10 at ikke dit Gods skal mætte fremmede, din Vinding ende i Andenmands Hus,
\par 11 så du gribes af Anger til sidst, når dit Kød og Huld svinder hen,
\par 12 og du siger: "Ak, at jeg hadede Tugt, at mit Hjerte lod hånt om Revselse,
\par 13 så jeg ikke lød mine Læreres Røst, ej bøjed mit Øre til dem, som lærte mig!
\par 14 Nær var jeg kommet i alskens Ulykke midt i Forsamling og Menighed!"
\par 15 Drik Vand af din egen Cisterne og rindende Vand af din Brønd;
\par 16 lad ej dine Kilder flyde på Gaden, ej dine Bække på Torvene!
\par 17 Dig skal de tilhøre, dig alene, ingen fremmed ved Siden af dig!
\par 18 Velsignet være dit Væld, og glæd dig ved din Ungdoms Hustru,
\par 19 den elskelige Hind, den yndige Gazel; hendes Elskov fryde dig stedse, berus dig altid i hendes Kærlighed!
\par 20 Hvi beruser du dig, min Søn, i en fremmed og tager en andens Hustru i Favn?
\par 21 Thi for HERRENs Øjne er Menneskets Veje, grant følger han alle dets Spor;
\par 22 den gudløse fanges af egen Brøde og holdes fast i Syndens Reb;
\par 23 han dør af Mangel på Tugt, går til ved sin store Dårskab.

\chapter{6}

\par 1 Min Søn: har du borget for din næste og givet en anden Håndslag,
\par 2 er du fanget ved dine Læber og bundet ved Mundens Ord,
\par 3 gør så dette, min Søn, og red dig, nu du er kommet i Næstens Hånd: Gå hen uden Tøven, træng ind på din Næste;
\par 4 und ikke dine Øjne Søvn, ej heller dine Øjenlåg Hvile,
\par 5 red dig som en Gazel af Snaren, som en Fugl af Fuglefængerens Hånd.
\par 6 Gå hen til Myren, du lade, se dens Færd og bliv viis.
\par 7 Skønt uden Fyrste, Foged og Styrer,
\par 8 sørger den dog om Somren for Æde og sanker sin Føde i Høst.
\par 9 Hvor længe vil du ligge, du lade, når står du op af din Søvn?
\par 10 Lidt Søvn endnu, lidt Blund, lidt Hvile med samlagte Hænder:
\par 11 som en Stimand kommer da Fattigdom over dig, Trang som en skjoldvæbnet Mand.
\par 12 En Nidding, en ussel Mand er den, som vandrer med Falskhed i Munden,
\par 13 som blinker med Øjet, skraber med Foden og giver Tegn med Fingrene,
\par 14 som smeder Rænker i Hjertet og altid kun ypper Kiv;
\par 15 derfor kommer hans Undergang brat, han knuses på Stedet, kan ikke læges.
\par 16 Seks Ting hader HERREN, syv er hans Sjæl en Gru:
\par 17 Stolte Øjne, Løgnetunge, Hænder, der udgyder uskyldigt Blod,
\par 18 et Hjerte, der udtænker onde Råd, Fødder, der haster og iler til ondt,
\par 19 falsk Vidne, der farer med Løgn, og den, som sætter Splid mellem Brødre.
\par 20 Min Søn, tag Vare på din Faders Bud, opgiv ikke din Moders Belæring,
\par 21 bind dem altid på dit Hjerte, knyt dem fast om din Hals;
\par 22 på din Vandring lede den dig, på dit Leje vogte den dig, den tale dig til, når du vågner;
\par 23 thi Budet er en Lygte, Læren Lys, og Tugtens Revselse Livets Vej
\par 24 for at vogte dig for Andenmands Hustru, for fremmed Kvindes sleske Tunge!
\par 25 Attrå ej i dit Hjerte hendes Skønhed, hendes Blik besnære dig ej!
\par 26 Thi en Skøge får man blot for et Brød, men Andenmands Hustru fanger dyrebar Sjæl.
\par 27 Kan nogen bære Ild i sin Brystfold, uden at Klæderne brænder?
\par 28 Kan man vandre på glødende Kul, uden at Fødderne svides?
\par 29 Så er det at gå ind til sin Næstes Hustru; ingen, der rører hende, slipper for Straf.
\par 30 Ringeagter man ikke Tyven, når han stjæler fot at stille sin Sult?
\par 31 Om han gribes, må han syvfold bøde og afgive alt sit Huses Gods.
\par 32 Afsindig er den, der boler med hende, kun en Selvmorder handler så;
\par 33 han opnår Hug og Skændsel, og aldrig udslettes hans Skam.
\par 34 Thi Skinsyge vækker Mandens Vrede, han skåner ikke på Hævnens Dag;
\par 35 ingen Bøde tager han god; store Tilbud rører ham ikke.

\chapter{7}

\par 1 Min Søn, vogt dig mine Ord,mine bud må du gemme hos dig;
\par 2 vogt mine bud, så skal du leve, som din Øjesten vogte du, hvad jeg har lært dig;
\par 3 bind dem om dine Fingre, skriv dem på dit Hjertes Tavle,
\par 4 sig til Visdommen: "Du er min Søster!" og kald Forstanden Veninde,
\par 5 at den må vogte dig for Andenmands Hustru, en fremmed Kvinde med sleske Ord.
\par 6 Thi fra mit Vindue skued jeg ud, jeg kigged igennem mit Gitter;
\par 7 og blandt de tankeløse så jeg en Yngling, en uden Vid blev jeg var blandt de unge;
\par 8 han gik på Gaden tæt ved et Hjørne, skred frem på Vej til hendes Hus
\par 9 i Skumringen henimod Aften, da Nat og Mørke brød frem.
\par 10 Og se, da møder Kvinden ham i Skøgedragt, underfundig i Hjertet;
\par 11 løssluppen, ustyrlig er hun, hjemme fandt hendes Fødder ej Ro;
\par 12 snart på Gader, snart på Torve, ved hvert et Hjørne lurer hun; -
\par 13 hun griber i ham og kysser ham og siger med frække Miner;
\par 14 "Jeg er et Takoffer skyldig og indfrier mit Løfte i Dag,
\par 15 gik derfor ud for at møde dig, søge dig, og nu har jeg fundet dig!
\par 16 Jeg har redt mit Leje med Tæpper, med broget ægyptisk Lærred
\par 17 jeg har stænket min Seng med Myrra, med Aloe og med Kanelbark;
\par 18 kom, lad os svælge til Daggry i Vellyst, beruse os i Elskovs Lyst!
\par 19 Thi Manden er ikke hjemme, - på Langfærd er han draget;
\par 20 Pengepungen tog han med, ved Fuldmåne kommer han hjem!"
\par 21 Hun lokked ham med mange fagre Ord, forførte ham med sleske Læber;
\par 22 tankeløst følger han hende som en Tyr, der føres til Slagtning, som en Hjort, der løber i Nettet,
\par 23 til en Pil gennemborer dens Lever, som en Fugl, der falder i Snaren, uden at vide, det gælder dens Liv.
\par 24 Hør mig da nu, min Søn, og lyt til min Munds Ord!
\par 25 Ej bøje du Hjertet til hendes Veje, far ikke vild på hendes Stier;
\par 26 thi mange ligger slagne, hvem hun har fældet, og stor er Hoben, som hun slog ihjel.
\par 27 Hendes Hus er Dødsrigets Veje, som fører til Dødens Kamre.

\chapter{8}

\par 1 Mon ikke Visdommen kalder, løfter Indsigten ikke sin røst?
\par 2 Oppe på Høje ved Vejen, ved Korsveje træder den frem;
\par 3 ved Porte, ved Byens Udgang, ved Dørenes Indgang råber den:
\par 4 Jeg kalder på eder, I Mænd, løfter min Røst til Menneskens Børn.
\par 5 I tankeløse, vind jer dog Klogskab, I Tåber, så få dog Forstand!
\par 6 Hør, thi jeg fører ædel Tale, åbner mine Læber med retvise Ord;
\par 7 ja, Sandhed taler min Gane, gudløse Læber er mig en Gru.
\par 8 Rette er alle Ord af min Mund, intet er falskt eller vrangt;
\par 9 de er alle ligetil for den kloge, retvise for dem der vandt Indsigt
\par 10 Tag ved Lære, tag ikke mod Sølv, tag mod Kundskab fremfor udsøgt Guld;
\par 11 thi Visdom er bedre end Perler, ingen Skatte opvejer den
\par 12 Jeg, Visdom, er Klogskabs Nabo og råder over Kundskab og Kløgt.
\par 13 HERRENs Frygt er Had til det onde. Jeg hader Hovmod og Stolthed, den onde Vej og den falske Mund.
\par 14 Jeg ejer Råd og Visdom, jeg har Forstand, jeg har Styrke.
\par 15 Ved mig kan Konger styre og Styresmænd give retfærdige Love;
\par 16 ved mig kan Fyrster råde og Stormænd dømme Jorden.
\par 17 Jeg elsker dem, der elsker mig, og de, der søger mig, finder mig.
\par 18 Hos mig er der Rigdom og Ære, ældgammelt Gods og Retfærd.
\par 19 Min Frugt er bedre end Guld og Malme, min Afgrøde bedre end kosteligt Sølv.
\par 20 Jeg vandrer på Retfærds Vej. midt hen ad Rettens Stier
\par 21 for at tildele dem, der elsker mig, Gods og fylde deres Forrådshuse.
\par 22 Mig skabte HERREN først blandt sine Værker, i Urtid, førend han skabte andet;
\par 23 jeg blev frembragt i Evigheden, i Begyndelsen, i Jordens tidligste Tider;
\par 24 jeg fødtes, før Verdensdybet var til, før Kilderne, Vandenes Væld, var til;
\par 25 førend Bjergene sænkedes, før Højene fødtes jeg,
\par 26 førend han skabte Jord og Marker, det første af Jordsmonnets Støv.
\par 27 Da han grundfæsted Himlen, var jeg hos ham, da han satte Hvælv over Verdensdybet.
\par 28 Da han fæstede Skyerne oventil og gav Verdensdybets Kilder deres faste Sted,
\par 29 da han satte Havet en Grænse, at Vandene ej skulde bryde hans Lov, da han lagde Jordens Grundvold,
\par 30 da var jeg Fosterbarn hos ham, hans Glæde Dag efter Dag; for hans Åsyn leged jeg altid,
\par 31 leged på hans vide Jord og havde min Glæde af Menneskens Børn.
\par 32 Og nu, I Sønner, hør mig! Vel den, der vogter på mine Veje!
\par 33 Hør på Tugt og bliv vise, lad ikke hånt derom!
\par 34 Lykkelig den, der hører på mig, så han daglig våger ved mine Døre og vogter på mine Dørstolper.
\par 35 Thi den, der ftnder mig; finder Liv og opnår Yndest hos HERREN;
\par 36 men den, som mister mig, skader sig selv; enhver, som hader mig, elsker Døden.

\chapter{9}

\par 1 Visdommen bygged sig Hus, rejste sig støtter syv,
\par 2 slagted sit Kvæg og blanded sin Vin, hun har også dækket sit Bord;
\par 3 hun har sendt sine Terner ud, byder ind på Byens højeste Steder:
\par 4 Hvo som er tankeløs, han komme hid, jeg taler til dem, som er uden Vid:
\par 5 Kom og smag mit Brød og drik den Vin, jeg har blandet!
\par 6 Lad Tankeløshed fare, så skal I leve, skrid frem ad Forstandens Vej!
\par 7 Tugter man en Spotter, henter man sig Hån; revser man en gudløs, høster man Skam;
\par 8 revs ikke en Spotter, at han ikke skal hade dig, revs den vise, så elsker han dig;
\par 9 giv til den vise, så bliver han visere, lær den retfærdige, så øges hans Viden.
\par 10 HERRENs Frygt er Visdoms Grundlag, at kende den HELLIGE, det er Forstand.
\par 11 Thi mange bliver ved mig dine Dage, dine Livsårs Tal skal øges.
\par 12 Er du viis, er det til Gavn for dig selv; spotter du, bærer du ene Følgen!
\par 13 Dårskaben, hun slår sig løs og lokker og kender ikke til Skam;
\par 14 hun sidder ved sit Huses indgang, troner på Byens Høje
\par 15 og byder dem ind, der kommer forbi, vandrende ad deres slagne Vej:
\par 16 Hvo som er tankeløs, han komme hid, jeg taler til dem, som er uden Vid:
\par 17 Stjålen Drik er sød, lønligt Brød er lækkert!
\par 18 Han ved ej, at Skyggerne dvæler der, hendes Gæster er i Dødsrigets Dyb.

\chapter{10}

\par 1 Salomos ordsprog. Viis søn glæder sin fader, tåbelig søn er sin moders sorg.
\par 2 Gudløsheds skatte gavner intet, men retfærd redder fra død.
\par 3 HERREN lader ej en retfærdig sulte, men gudløses attrå støder han fra sig.
\par 4 Doven hånd skaber fattigdom, flitteges hånd gør rig.
\par 5 En klog søn samler om sommeren, en dårlig sover om høsten.
\par 6 Velsignelse er for retfærdiges hoved, på uret gemmer gudløses mund.
\par 7 Den retfærdiges minde velsignes, gudløses navn smuldrer hen.
\par 8 Den vise tager mod påbud, den brovtende dåre styrtes.
\par 9 Hvo lydefrit vandrer, vandrer trygt; men hvo der går krogveje, ham går det ilde.
\par 10 Blinker man med øjet, volder man ondt, den brovtende dåre styrtes.
\par 11 Den retfærdiges mund er en livsens kilde, på uret gemmer gudløses mund.
\par 12 Had vækker Splid, Kærlighed skjuler alle Synder.
\par 13 På den kloges Læber fnder man Visdom, Stok er til Ryg på Mand uden Vid.
\par 14 De vise gemmer den indsigt, de har, Dårens Mund er truende Våde.
\par 15 Den riges Gods er hans faste Stad, Armod de ringes Våde.
\par 16 Den retfærdiges Vinding tjener til Liv, den gudløses Indtægt til Synd.
\par 17 At vogte påTugt erVej tiILivet, vild farer den, som viser Revselse fra sig.
\par 18 Retfærdige Læber tier om Had, en Tåbe er den, der udspreder Rygter.
\par 19 Ved megen Tale undgås ej Brøde, klog er den, der vogter sin Mund.
\par 20 Den retfærdiges Tunge er udsøgt Sølv, gudløses Hjerte er intet værd.
\par 21 Den retfærdiges Læber nærer mange, Dårerne dør af Mangel på Vid.
\par 22 HERRENs Velsignelse, den gør rig, Slid og Slæb lægger intet til.
\par 23 For Tåben er Skændselsgerning en Leg, Visdom er Leg for Mand med Indsigt.
\par 24 Hvad en gudløs frygter, kommer over hans Hoved, hvad retfærdige ønsker, bliver dem givet.
\par 25 Når Storm farer frem, er den gudløse borte, den retfærdige står på evig Grund.
\par 26 Som Eddike for Tænder og Røg for Øjne så er den lade for dem, der sender ham.
\par 27 HERRENs Frygt lægger dage til, gudløses År kortes af.
\par 28 Retfærdige har Glæde i Vente, gudløses Håb vil briste.
\par 29 For lydefri Vandel er HERREN et Værn, men en Rædsel for Udådsmænd.
\par 30 Den retfærdige rokkes aldrig, ikke skal gudløse bo i Landet.
\par 31 Den retfærdiges Mund bærer Visdoms Frugt, den falske Tunge udryddes.
\par 32 Den retfærdiges Læber søger yndest, gudløses Mund bærer Falskheds Frugt.

\chapter{11}

\par 1 Falske Vægtskåle er HERREN en gru, fuldvvægtigt Lod er efter hans Sind.
\par 2 Kommer Hovmod, kommer og Skændsel, men med ydmyge følger der Visdom.
\par 3 Retsindiges Uskyld leder dem trygt, troløses falskhed lægger dem øde.
\par 4 Ej hjælper Rigdom på Vredens Dag, men Retfærd redder fra Døden.
\par 5 Den lydefris Retfærd jævner hans Vej, for sin Gudløshed falder den gudløse.
\par 6 Retsindiges Retfærd bringer dem Frelse, troløse fanges i egen Attrå.
\par 7 Ved Døden brister den gudløses Håb, Dårers Forventning brister.
\par 8 Den retfærdige fries af Trængsel, den gudløse kommer i hans Sted.
\par 9 Med sin Mund lægger vanhellig Næsten øde, retfærdige fries ved Kundskab.
\par 10 Ved retfærdiges Lykke jubler en By, der er Fryd ved gudløses Undergang.
\par 11 Ved retsindiges Velsignelse rejser en By sig, den styrtes i Grus ved gudløses Mund.
\par 12 Mand uden Vid ser ned på sin Næste, hvo, som har Indsigt, tier.
\par 13 Bagtaleren røber, hvad ham er betroet, den pålidelige skjuler Sagen.
\par 14 Uden Styre står et Folk for Fald, vel står det til, hvor mange giver Råd.
\par 15 Den går det ilde, som borger for andre, tryg er den, der hader Håndslag.
\par 16 Yndefuld Kvinde vinder Manden Ære; hader hun Retsind, volder hun Skændsel. De lade må savne Gods, flittige vinder sig Rigdom.
\par 17 Kærlig Mand gør vel mod sin Sjæl, den grumme er hård ved sit eget Kød.
\par 18 Den gudløse skaber kun skuffende Vinding, hvo Retfærd sår, får virkelig Løn.
\par 19 At hige efter Retfærd er Liv, at jage efter ondt er Død.
\par 20 De svigefulde er HERREN en Gru, hans Velbehag ejer, hvo lydefrit vandrer.
\par 21 Visselig undgår den onde ej Straf, de retfærdiges Æt går fri.
\par 22 Som Guldring i Svinetryne er fager Kvinde, der ikke kan skønne.
\par 23 Retfærdiges Ønske bliver kun til Lykke, gudløse har kun Vrede i Vente.
\par 24 En strør om sig og gør dog Fremgang, en anden nægter sig alt og mangler.
\par 25 Gavmild Sjæl bliver mæt; hvo andre kvæger, kvæges og selv.
\par 26 Hvo Kornet gemmer, ham bander Folket, Velsignelse kommer over den, som sælger.
\par 27 Hvo der jager efter godt, han søger efter Yndest, hvo der higer efter ondt, ham kommer det over.
\par 28 Hvo der stoler på sin Rigdom, falder, retfærdige grønnes som Løv.
\par 29 Den, der øder sit Hus, høster Vind, Dåre bliver Vismands Træl.
\par 30 Retfærds Frugt er et Livets Træ, Vismand indfanger Sjæle.
\par 31 En retfærdig reddes med Nød og næppe, endsige en gudløs, en, der synder.

\chapter{12}

\par 1 At elske Tugt er at elske Kundskab, at hade Revselse er dumt.
\par 2 Den gode vinder Yndest hos HERREN, den rænkefulde dømmer han skyldig.
\par 3 Ingen står fast ved Gudløshed, men retfærdiges Rod skal aldrig rokkes.
\par 4 En duelig Kvinde er sin Ægtemands Krone, en dårlig er som Edder i hans Ben.
\par 5 Retfærdiges Tanker er Ret, gudløses Opspind er Svig.
\par 6 Gudløses Ord er på Lur efter Blod, retsindiges Mund skal bringe dem Frelse.
\par 7 Gudløse styrtes og er ikke mer. retfærdiges Hus står fast.
\par 8 For sin Klogskab prises en Mand, til Spot bliver den, hvis Vid er vrangt.
\par 9 Hellere overses, når man holder Træl, end optræde stort, når man mangler Brød.
\par 10 Den retfærdige føler med sit Kvæg, gudløses Hjerte er grumt.
\par 11 Den mættes med Brød, som dyrker sin Jord, uden Vid er den, der jager efter Tomhed.
\par 12 De ondes Fæstning jævnes med Jorden, de retfærdiges Rod bolder Stand.
\par 13 I Læbernes Brøde hildes den onde, den retfærdige undslipper Nøden.
\par 14 Af sin Munds Frugt mættes en Mand med godt, et Menneske får, som hans Hænder har øvet.
\par 15 Dårens Færd behager ham selv, den vise hører på Råd.
\par 16 En Dåre giver straks sin Krænkelse Luft, den kloge spottes og lader som intet.
\par 17 Den sanddru fremfører, hvad der er ret, det falske Vidne kommer med Svig.
\par 18 Mangens Snak er som Sværdhug, de vises Tunge læger.
\par 19 Sanddru Læbe består for evigt, Løgnetunge et Øjeblik.
\par 20 De, som smeder ondt, har Svig i Hjertet; de, der stifter Fred, har Glæde.
\par 21 Den retfærdige times der intet ondt, - gudløse oplever Vanheld på Vanheld.
\par 22 Løgnelæber er HERREN en Gru, de ærlige har hans Velbebag.
\par 23 Den kloge dølger sin Kundskab, Tåbers Hjerte udråber Dårskab.
\par 24 De flittiges Hånd skal råde, den lade tvinges til Hoveriarbejde.
\par 25 Hjertesorg bøjer til Jorden, et venligt Ord gør glad.
\par 26 Den retfærdige vælger sin Græsgang, gudløses Vej vildleder dem selv.
\par 27 Ladhed opskræmmer intet Vildt, men kosteligt Gods får den flittige tildelt.
\par 28 På Retfærds Sti er der Liv, til Døden fører den onde Vej.

\chapter{13}

\par 1 Viis Søn elsker tugt, spotter hører ikke på skænd.
\par 2 Af sin Munds Frugt nyder en Mand kun godt, til Vold står troløses Hu.
\par 3 Vogter man Munden, bevarer man Sjælen, den åbenmundede falder i Våde.
\par 4 Den lade attrår uden at få, men flittiges Sjæl bliver mæt.
\par 5 Den retfærdige hader Løgnetale, den gudløse spreder Skam og Skændsel.
\par 6 Retfærd skærmer, hvo lydefrit vandrer, Synden fælder de gudløse.
\par 7 Mangen lader rig og ejer dog intet, mangen lader fattig og ejer dog meget.
\par 8 Mands Rigdom er Løsepenge for hans Liv, Fattigmand får ingen Trusel at høre.
\par 9 Retfærdiges Lys bryder frem, gudløses Lampe går ud.
\par 10 Ved Hovmod vækkes kun Splid, hos dem, der lader sig råde, er Visdom.
\par 11 Rigdom, vundet i Hast, smuldrer hen, hvad der samles Håndfuld for Håndfuld, øges.
\par 12 At bie længe gør Hjertet sygt, opfyldt Ønske er et Livets Træ.
\par 13 Den, der lader hånt om Ordet, slås ned, den, der frygter Budet, får Løn.
\par 14 Vismands Lære er en Livsens Kilde, derved undgås Dødens Snarer.
\par 15 God Forstand vinder Yndest, troløses Vej er deres Undergang.
\par 16 Hver, som er klog, går til Værks med Kundskab, Tåben udfolder Dårskab.
\par 17 Gudløs Budbringer går det galt, troværdigt Bud bringer Lægedom.
\par 18 Afvises Tugt, får man Armod og Skam; agtes på Revselse, bliver man æret.
\par 19 Opfyldt Ønske er sødt for Sjælen, at vige fra ondt er Tåber en Gru.
\par 20 Omgås Vismænd, så bliver du viis, ilde faren er Tåbers Ven.
\par 21 Vanheld følger Syndere, Lykken når de retfærdige.
\par 22 Den gode efterlader Børnebrn Arv, til retfærdige gemmes Synderens Gods.
\par 23 På Fattigfolks Nyjord er rigelig Føde, mens mangen rives bort ved Uret.
\par 24 Hvo Riset sparer, hader sin Søn, den, der elsker ham, tugter i Tide.
\par 25 Den retfærdige spiser, til Sulten er stillet, gudløses Bug er tom.

\chapter{14}

\par 1 Visdom bygger sit hus,dårskabs hænder river det ned.
\par 2 Hvo redeligt vandrer, frygter HERREN, men den, som går Krogveje, agter ham ringe.
\par 3 I Dårens Mund er Ris til hans Ryg, for de vise står Læberne Vagt.
\par 4 Når der ikke er Okser, er Laden tom, ved Tyrens Kraft bliver Høsten stor.
\par 5 Sanddru Vidne lyver ikke, det falske Vidne farer med Løgn.
\par 6 Spotter søger Visdom, men finder den ikke, til Kundskab kommer forstandig let.
\par 7 Gå fra en Mand, som er en Tåbe, der mærker du intet til Kundskabs Læber.
\par 8 Den kloge i sin Visdom er klar på sin Vej, men Tåbers Dårskab er Svig.
\par 9 Med Dårer driver Skyldofret Spot, men Velvilje råder iblandt retsindige.
\par 10 Hjertet kender sin egen Kvide, fremmede blander sig ej i dets Glæde.
\par 11 Gudløses Hus lægges øde, retsindiges Telt står i Blomst.
\par 12 Mangen Vej synes Manden ret, og så er dens Ende dog Dødens Veje.
\par 13 Selv under Latter kan Hjertet lide, og Glædens Ende er Kummer.
\par 14 Af sine Veje mættes den frafaldne, af sine Gerninger den, som er god.
\par 15 Den tankeløse tror hvert Ord, den kloge overtænker sine Skridt.
\par 16 Den vise ængstes og skyr det onde, Tåben buser sorgløs på.
\par 17 Den hidsige bærer sig tåbeligt ad, man hader rænkefuld Mand.
\par 18 De tankeløse giver dårskab i Arv, de kloge efterlader sig Kundskab.
\par 19 Onde må bukke for gode, gudløse stå ved retfærdiges Døre.
\par 20 Fattigmand hades endog af sin Ven, men Rigmands Venner er mange.
\par 21 Den, der foragter sin Næste, synder, lykkelig den, der har Medynk med arme.
\par 22 De, som virker ondt, farer visselig vild; de, som virker godt, finder Nåde og Trofasthed.
\par 23 Ved al Slags Møje vindes der noget, Mundsvejr volder kun Tab.
\par 24 De vises Krone er Kløgt, Tåbers Krans er Dårskab.
\par 25 Sanddru Vidne frelser Sjæle; den, som farer med Løgn, bedrager.
\par 26 Den stærkes Tillid er HERRENs Frygt, hans Sønner skal have en Tilflugt.
\par 27 HERRENs Frygt er en Livsens Kilde, derved undgås Dødens Snarer.
\par 28 At Folket er stort, er Kongens Hæder, Brist på Folk er Fyrstens Fald.
\par 29 Den sindige er rig på Indsigt, den heftige driver det vidt i Dårskab.
\par 30 Sagtmodigt Hjerte er Liv for Legemet, Avind er Edder i Benene.
\par 31 At kue den ringe er Hån mod hans Skaber, han æres ved Medynk med fattige.
\par 32 Ved sin Ondskab styrtes den gudløse, ved lydefri Færd er retfærdige trygge.
\par 33 Visdom bor i forstandiges Hjerte, i Tåbers Indre kendes den ikke.
\par 34 Retfærdighed løfter et Folk, men Synd er Folkenes Skændsel.
\par 35 En klog Tjener har Kongens Yndest, en vanartet rammer hans Vrede.

\chapter{15}

\par 1 Mildt svar stiller vrede, sårende ord vækker nag.
\par 2 Vises Tunge drypper af Kundskab, Dårskab strømmer fra Tåbers Mund.
\par 3 Alle Vegne er HERRENs Øjne, de udspejder onde og gode.
\par 4 Et Livets Træ er Tungens Mildhed, dens Falskhed giver Hjertesår.
\par 5 Dåre lader hånt om sin Faders Tugt, klog er den, som tager Vare på Revselse.
\par 6 Den retfærdiges Hus har megen Velstand, den gudløses Høst lægges øde.
\par 7 Vises Læber udstrør Kundskab, Tåbers Hjerte er ikke ret.
\par 8 Gudløses Offer er HERREN en Gru, retsindiges Bøn har han Velbehag i.
\par 9 Den gudløses Færd er HERREN en Gru, han elsker den, der stræber efter Retfærd.
\par 10 Streng Tugt er for den, der forlader Vejen; den, der hader Revselse, dør.
\par 11 Dødsrige og Afgrund ligger åbne for HERREN, endsige da Menneskebørnenes Hjerter.
\par 12 Spotteren ynder ikke at revses, til Vismænd går han ikke.
\par 13 Glad Hjerte giver venligt Ansigt, ved Hjertesorg bliver Modet brudt.
\par 14 Den forstandiges Hjerte søger Kundskab, Tåbers Mund lægger Vind på Dårskab.
\par 15 Alle den armes Dage er onde, glad Hjerte er stadigt Gæstebud.
\par 16 Bedre lidet med HERRENs Frygt end store Skatte med Uro.
\par 17 Bedre en Ret Grønt med Kærlighed end fedet Okse og Had derhos.
\par 18 Vredladen Mand vækker Splid, sindig Mand stiller Trætte.
\par 19 Den lades Vej er spærret af Tjørn, de flittiges Sti er banet.
\par 20 Viis Søn glæder sin Fader, Tåbe til Menneske foragter sin Moder.
\par 21 Dårskab er Glæde for Mand uden Vid, Mand med Indsigt går lige frem.
\par 22 Er der ikke holdt Råd, så mislykkes Planer, de lykkes, når mange rådslår.
\par 23 Mand er glad, når hans Mund kan svare, hvor godt er et Ord i rette Tid.
\par 24 Den kloge går opad på Livets Vej for at undgå Dødsriget nedentil.
\par 25 Hovmodiges Hus river HERREN bort, han fastsætter Enkens Skel.
\par 26 Onde Tanker er HERREN en Gru, men hulde Ord er rene.
\par 27 Den øder sit Hus, hvem Vinding er alt; men leve skal den, der hader Gave.
\par 28 Den retfærdiges Hjerte tænker, før det svarer, gudløses Mund lader ondt strømme ud.
\par 29 HERREN er gudløse fjern, men hører retfærdiges Bøn.
\par 30 Milde Øjne fryder Hjertet, godt Bud giver Marv i Benene.
\par 31 Øret, der lytter til Livsens Revselse, vil gerne dvæle iblandt de vise.
\par 32 Hvo Tugt forsmår, lader hånt om sin Sjæl, men Vid fanger den, der lytter til Revselse.
\par 33 HERRENs Frygt er Tugt til Visdom, Ydmyghed først og siden Ære.

\chapter{16}

\par 1 Hjertets Råd er Menneskets sag. Tungens Svar er fra HERREN.
\par 2 En Mand holder al sin Færd for ren, men HERREN vejer Ånder.
\par 3 Vælt dine Gerninger på HERREN, så skal dine Planer lykkes.
\par 4 Alt skabte HERREN, hvert til sit, den gudløse også for Ulykkens Dag.
\par 5 Hver hovmodig er HERREN en Gru, visselig slipper han ikke for Straf.
\par 6 Ved Mildhed og Troskab sones Brøde, ved HERRENs Frygt undviger man ondt.
\par 7 Når HERREN har Behag i et Menneskes Veje, gør han endog hans Fjender til Venner.
\par 8 Bedre er lidet med Retfærd end megen Vinding med Uret.
\par 9 Menneskets Hjerte udtænker hans Vej, men HERREN styrer hans Fjed.
\par 10 Der er Gudsdom på Kongens Læber, ej fejler hans Mund, når han dømmer.
\par 11 Ret Bismer og Vægtskål er HERRENs, hans Værk er alle Posens Lodder.
\par 12 Gudløs Færd er Konger en Gru, thi ved Retfærd grundfæstes Tronen.
\par 13 Retfærdige Læber har Kongens Yndest, han elsker den, der taler oprigtigt.
\par 14 Kongens Vrede er Dødens Bud, Vismand evner at mildne den.
\par 15 I Kongens Åsyns Lys er der Liv, som Vårregnens Sky er hans Yndest.
\par 16 At vinde Visdom er bedre end Guld, at vinde Indsigt mere end Sølv.
\par 17 De retsindiges Vej er at vige fra ondt; den vogter sit Liv, som agter på sin Vej.
\par 18 Hovmod går forud for Fald, Overmod forud for Snublen.
\par 19 Hellere sagtmodig med ydmyge end dele Bytte med stolte.
\par 20 Vel går det den, der mærker sig Ordet; lykkelig den, der stoler på HERREN.
\par 21 Den vise kaldes forstandig, Læbernes Sødme øger Viden.
\par 22 Kløgt er sin Mand en Livsens Kilde, Dårskab er Dårers Tugt.
\par 23 Den vises Hjerte giver Munden Kløgt, på Læberne lægger det øget Viden.
\par 24 Hulde Ord er som flydende Honning, søde for Sjælen og sunde for Legemet.
\par 25 Mangen Vej synes Manden ret, og så er dens Ende dog Dødens Veje.
\par 26 En Arbejders Hunger arbejder for ham, thi Mundens Krav driver på ham.
\par 27 En Nidding graver Ulykkesgrave, det er, som brændte der Ild på hans Læber.
\par 28 Rænkefuld Mand sætter Splid; den, der bagtaler, skiller Venner.
\par 29 Voldsmand lokker sin Næste og fører ham en Vej, der ikke er god.
\par 30 Den, der stirrer, har Rænker for; knibes Læberne sammen, har man fuldbyrdet ondt.
\par 31 Grå Hår er en dejlig Krone, den vindes på Retfærds Vej.
\par 32 Større end Helt er sindig Mand, større at styre sit Sind end at tage en Stad.
\par 33 I Brystfolden rystes Loddet, det falder, som HERREN vil.

\chapter{17}

\par 1 Bedre en tør Bid Brød med fred end Huset fuldt af Sul med Trætte.
\par 2 Klog Træl bliver Herre over dårlig Søn og får lod og del mellem brødre.
\par 3 Digel til Sølv og Ovn til Guld, men den, der prøver Hjerter, er HERREN.
\par 4 Den onde hører på onde Læber, Løgneren lytter til giftige Tunger.
\par 5 Hvo Fattigmand spotter, håner hans Skaber, den skadefro slipper ikke for Straf.
\par 6 De gamles Krone er Børnebørn, Sønners Stolthed er Fædre.
\par 7 Ypperlig Tale er ej for en Dåre, end mindre da Løgnfor den, som er ædel.
\par 8 Som en Troldsten er Gave i Giverens Øjne; hvorhen den end vender sig, gør den sin Virkning.
\par 9 Den, der dølger en Synd, søger Venskab, men den, der ripper op i en Sag, skiller Venner.
\par 10 Bedre virker Skænd på forstandig end hundrede Slag på en Tåbe.
\par 11 Den onde har kun Genstridigbed for, men et skånselsløst Bud er udsendt imod ham.
\par 12 Man kan møde en Bjørn, hvis Unger er taget, men ikke en Tåbe udi hans Dårskab.
\par 13 Den, der gengælder godt med ondt, fra hans Hus skal Vanheld ej vige.
\par 14 At yppe Strid er at åbne for Vand, hold derfor inde, før Strid bryder løs.
\par 15 At frikende skyldig og dømme uskyldig, begge Dele er HERREN en Gru.
\par 16 Hvad hjælper Penge i Tåbens Hånd til at købe ham Visdom, når Viddet mangler?
\par 17 Ven viser Kærlighed når som helst, Broder fødes til Hjælp i Nød.
\par 18 Mand uden Vid giver Håndslag og går i Borgen for Næsten.
\par 19 Ven af Kiv er Ven af Synd; at højne sin Dør er at attrå Fald.
\par 20 Ej finder man Lykke, når Hjertet er vrangt, man falder i Våde, når Tungen er falsk.
\par 21 Den, der avler en Tåbe, får Sorg, Dårens Fader er ikke glad.
\par 22 Glad Hjerte er godt for Legemet, nedslået Sind suger Marv af Benene.
\par 23 Den gudløse tager Gave i Løn for at bøje Rettens Gænge.
\par 24 Visdom står den forstandige for Øje, Tåbens Blik er ved Jordens Ende.
\par 25 Tåbelig Søn er sin Faders Sorg, Kvide for hende, som fødte ham.
\par 26 At straffe den, der har Ret, er ilde, værre endnu at slå de ædle.
\par 27 Den, som har Kundskab tøjler sin Tale, Mand med Forstand er koldblodig.
\par 28 Selv Dåren, der tier, gælder for viis, forstandig er den, der lukker sine Læber.

\chapter{18}

\par 1 Særlingen søger et påskud, med vold og magt vil han strid.
\par 2 Tåben ynder ej Indsigt, men kun, at hans Tanker kommer for Lyset.
\par 3 Hvor Gudløshed kommer, kommer og Spot, Skam og Skændsel følges.
\par 4 Ord i Mands Mund er dybe Vande, en rindende Bæk, en Visdomskilde.
\par 5 Det er ilde at give en skyldig Medhold, så man afviser skyldfris Sag i Retten.
\par 6 Tåbens Læber fører til Trætte, hans Mund råber højt efter Hug,
\par 7 Tåbens Mund er hans Våde, hans Læber en Snare for hans Liv.
\par 8 Bagtalerens Ord er som Lækkerbidskener, de synker dybt i Bugen.
\par 9 Den, der er efterladen i Gerning, er også Broder til Ødeland.
\par 10 HERRENs Navn er et stærkt Tårn, den retfærdige løber derhen og bjærges.
\par 11 Den riges Gods er hans faste Stad, og tykkes ham en knejsende Mur.
\par 12 Mands Hovmod går forud for Fald, Ydmyghed forud for Ære.
\par 13 Om nogen svarer, førend han hører, regnes det ham til Dårskab og Skændsel.
\par 14 Mands Mod udholder Sygdom, men hvo kan bære en sønderbrudt Ånd?
\par 15 Den forstandiges Hjerte vinder sig Kundskab, de vises Øre attrår Kundskab.
\par 16 Gaver åbner et Menneske Vej og fører ham hen til de store.
\par 17 Den, der taler først i en Trætte har Ret, til den anden kommer og går ham efter.
\par 18 Loddet gør Ende på Trætter og skiller de stærkeste ad.
\par 19 Krænket Broder er som en Fæstning, Trætter som Portslå for Borg.
\par 20 Mands Bug mættes af Mundens Frugt, han mættes af Læbernes Grøde.
\par 21 Død og Liv er i Tungens Vold, hvo der tøjler den, nyder dens Frugt.
\par 22 Fandt man en Hustru, fandt man Lykken og modtog Nåde fra HERREN.
\par 23 Fattigmand beder og trygler, Rigmand svarer med hårde Ord.
\par 24 Med mange Fæller kan Mand gå til Grunde, men Ven kan overgå Broder i Troskab.

\chapter{19}

\par 1 Bedre Fattigmand med lydefri færd end en, som går Krogveje, er han end rig.
\par 2 At mangle Kundskab er ikke godt, men den træder fejl, som har Hastværk.
\par 3 Et Menneskes Dårskab øder hans Vej, men på HERREN vredes hans Hjerte.
\par 4 Gods skaffer mange Venner, den ringe skiller hans Ven sig fra.
\par 5 Det falske Vidne undgår ej Straf; den slipper ikke, som farer med Løgn.
\par 6 Mange bejler til Stormands Yndest, og alle er Venner med gavmild Mand.
\par 7 Fattigmands Frænder hader ham alle, end mere skyr hans Venner ham da. Ej frelses den, som jager efter Ord.
\par 8 Den, der vinder Vid, han elsker sin Sjæl, og den, der vogter på Indsigt, får Lykke.
\par 9 Det falske Vidne undgår ej Straf, og den, der farer med Løgn, går under.
\par 10 Vellevned sømmer sig ikke for Tåbe, end mindre for Træl at herske over Fyrster.
\par 11 Klogskab gør Mennesket sindigt, hans Ære er at overse Brøde.
\par 12 Som Brøl af en Løve er Kongens Vrede, som Dug på Græs er hans Gunst.
\par 13 Tåbelig Søn er sin Faders Ulykke, Kvindekiv er som ustandseligt Tagdryp.
\par 14 Hus og Gods er Arv efter Fædre, en forstandig Hustru er fra HERREN.
\par 15 Dovenskab sænker i Dvale, den lade Sjæl må sulte.
\par 16 Den vogter sin Sjæl, som vogter på Budet, men skødesløs Vandel fører til Død.
\par 17 Er man god mod den ringe, låner man HERREN, han gengælder en, hvad godt man har gjort.
\par 18 Tugt din Søn, imens der er Håb, ellers stiler du efter at slå ham ihjel.
\par 19 Den, som er hidsig, må bøde, ved Skånsel gør man det værre.
\par 20 Hør på Råd og tag ved Lære, så du til sidst bliver viis.
\par 21 I Mands Hjerte er mange Tanker, men HERRENs Råd er det, der står fast.
\par 22 Vinding har man af Godhed, hellere fattig end Løgner.
\par 23 HERRENs Frygt er Vej til Liv, man hviler mæt og frygter ej ondt.
\par 24 Den lade rækker til Fadet, men fører ej Hånden til Munden.
\par 25 Får Spottere Hug, bliver tankeløs klog, ved Revselse får den forstandige Kundskab.
\par 26 Mishandle Fader og bortjage Moder gør kun en dårlig, vanartet Søn.
\par 27 Hør op, min Søn, med at høre på Tugt og så fare vild fra Kundskabsord.
\par 28 Niddingevidne spotter Retten, gudløses Mund er glubsk efter Uret.
\par 29 Slag er rede til Spottere, Hug til Tåbers Ryg.

\chapter{20}

\par 1 En Spotter er Vinen, stærk Drik slår sig løs, og ingen, som raver deraf, er viis.
\par 2 Som Løvebrøl er Rædslen, en Konge vækker, at vække hans Vrede er at vove sit Liv.
\par 3 Mands Ære er det at undgå Trætte, men alle Tåber vil Strid.
\par 4 Om Efteråret pløjer den lade ikke, han søger i Høst, men finder intet.
\par 5 Råd i Mands Hjerte er dybe Vande, men Mand med Indsigt drager det op.
\par 6 Mangen kaldes en velvillig Mand, men hvem kan finde en trofast Mand?
\par 7 Retfærdig er den, som lydefrit vandrer, hans Sønner får Lykke efter ham.
\par 8 Kongen, der sidder i Dommersædet, sigter alt ondt med sit Blik.
\par 9 Hvo kan sige: "Jeg rensed mit Hjerte, og jeg er ren for Synd!"
\par 10 To Slags Vægt og to Slags Mål, begge Dele er HERREN en Gru.
\par 11 Selv Drengen kendes på det, han gør, om han er ren og ret hans Færd.
\par 12 Øret, der hører, og Øjet, der ser, HERREN skabte dem begge.
\par 13 Elsk ikke Søvn, at du ej bliver fattig, luk Øjnene op og bliv mæt.
\par 14 Køberen siger: "Usselt, usselt!" men skryder af Handelen, når han går bort.
\par 15 Har man end Guld og Perler i Mængde, kosteligst Smykke er Kundskabslæber.
\par 16 Tag hans Klæder, han borged for en anden, pant ham for fremmedes Skyld!
\par 17 Sødt smager Løgnens Brød, bagefter fyldes Munden med Grus.
\par 18 Planer, der lægges ved Rådslagning, lykkes; før Krig efter modent Overlæg!
\par 19 Bagtaleren røber, hvad ham er betroet, hav ej med en åbenmundet at gøre!
\par 20 Den, der bander Fader og Moder, i Bælgmørke går hans Lampe ud.
\par 21 Først haster man efter en Arv, men til sidst velsignes den ikke.
\par 22 Sig ikke: "Ondt vil jeg gengælde!" Bi på HERREN, så hjælper han dig.
\par 23 To Slags Lodder er HERREN en Gru, det er ikke godt, at Vægten er falsk.
\par 24 Fra HERREN er Mands Fjed, hvor kan et Menneske fatte sin Skæbne!
\par 25 Det er farligt at sige tankeløst: "Helligt!" og først efter Løftet tænke sig om.
\par 26 Viis Konge sigter de gudløse, lader Tærskehjul gå over dem.
\par 27 Menneskets Ånd er en HERRENs Lampe, den ransager alle hans indres Kamre.
\par 28 Godhed og Troskab vogter Kongen, han støtter sin Trone ved Retfærd.
\par 29 Unges Stolthed er deres Styrke, gamles Smykke er grånet Hår.
\par 30 Blodige Strimer renser den onde og Hug hans Indres Kamre.

\chapter{21}

\par 1 En Konges hjerte er Bække i HERRENs hånd, han leder det hen, hvor han vil.
\par 2 En Mand holder al sin Færd for ret, men HERREN vejer Hjerter.
\par 3 At øve Ret og Skel er mere værd for HERREN end Offer.
\par 4 Hovmodige Øjne, et opblæst Hjerte, selv gudløses Nyjord er Synd.
\par 5 Kun Overflod bringer den flittiges Råd, hver, som har Hastværk, får kun Tab.
\par 6 At skabe sig Rigdom ved Løgnetunge er Jag efter Vind i Dødens Snarer.
\par 7 Gudløses Voldsfærd bortriver dem selv, thi de vægrer sig ved at øve Ret.
\par 8 Skyldtynget Mand går Krogveje, den renes Gerning er ligetil.
\par 9 Hellere bo i en Krog på Taget end fælles Hus med frættekær Kvinde.
\par 10 Den gudløses Sjæl har Lyst til ondt, hans Øjne ynker ikke hans Næste.
\par 11 Må Spotter bøde, bliver tankeløs klog, har Vismand Fremgang, da vinder han kundskab.
\par 12 Den Retfærdige har Øje med den gudløses Hus, han styrter gudløse Folk i Ulykke.
\par 13 Hvo Øret lukker for Småmands Skrig, skal råbe selv og ikke få Svar.
\par 14 Lønlig Gave mildner Vrede, Stikpenge i Brystfolden voldsom Harme.
\par 15 Rettens Gænge er den retfærdiges Glæde, men Udådsmændenes Rædsel.
\par 16 Den, der farer vild fra Kløgtens Vej, skal havne i Skyggers Forsamling.
\par 17 Lyst til Morskab fører i Trang, Lyst til Olie og Vin gør ej rig.
\par 18 Den gudløse bliver Løsepenge for den retfærdige, den troløse kommer i retsindiges Sted.
\par 19 Hellere bo i et Ørkenland end hos en trættekær, arrig Kvinde.
\par 20 I den vises Bolig er kostelig Skat og Olie, en Tåbe af et Menneske øder det.
\par 21 Den, der higer efter Retfærd og Godhed vinder sig Liv og Ære.
\par 22 Vismand stormer Heltes By og styrter Værnet, den stolede på.
\par 23 Den, der vogter sin Mund og sin Tunge, vogter sit Liv for Trængsler. -
\par 24 Den opblæste stolte kaldes en Spotter, han handler frækt i Hovmod.
\par 25 Den lades Attrå bliver hans Død, thi hans Hænder vil intet bestille.
\par 26 Ugerningsmand er stadig i Trang, den retfærdige giver uden at spare.
\par 27 Vederstyggeligt er de gudløses Offer, især når det ofres for Skændselsdåd.
\par 28 Løgnagtigt Vidne går under, Mand, som vil høre, kan tale fremdeles.
\par 29 Den gudløse optræder frækt, den retsindige overtænker sin Vej.
\par 30 Visdom er intet, Indsigt er intet, Råd er intet over for HERREN.
\par 31 Hest holdes rede til Stridens Dag, men Sejren er HERRENs Sag.

\chapter{22}

\par 1 Hellere godt Navn end megen rigdom, Yndest er bedre end Sølv og Guld
\par 2 Rig og fattig mødes, HERREN har skabt dem begge.
\par 3 Den kloge ser Faren og søger i Skjul, tankeløse går videre og bøder.
\par 4 Lønnen for Ydmyghed og HERRENs Frygt er Rigdom, Ære og Liv.
\par 5 På den svigefuldes Vej er der Torne og Snarer; vil man vogte sin Sjæl, må man holde sig fra dem.
\par 6 Væn Drengen til den Vej, han skal følge, da viger han ikke derfra, selv gammel.
\par 7 Over Fattigfolk råder den rige, Låntager bliver Långivers Træl.
\par 8 Hvo Uret sår, vil høste Fortræd, hans Vredes Ris skal slå ham selv.
\par 9 Den vennesæle velsignes, thi han deler sit Brød med den ringe.
\par 10 Driv Spotteren ud, så går Trætten med, og Hiv og Smæden får Ende.
\par 11 HERREN elsker den rene af Hjertet; med Ynde på Læben er man Kongens Ven.
\par 12 HERRENs Øjne agter på Kundskab, men han kuldkaster troløses Ord.
\par 13 Den lade siger: "En Løve på Gaden! Jeg kan let blive revet ihjel på Torvet."
\par 14 Fremmed Kvindes Mund er en bundløs Grav, den, HERREN er vred på, falder deri.
\par 15 Dårskab er knyttet til Ynglingens Hjerte, Tugtens Ris skal tjerne den fra ham.
\par 16 Vold mod den ringe øger hans Eje, Gave til Rigmand gør ham kun fattig. -
\par 17 Bøj Øret og hør de vises Ord, vend Hjertet til og kend deres Liflighed!
\par 18 Vogter du dem i dit Indre, er de alle rede på Læben.
\par 19 For at din Lid skal stå til HERREN, lærer jeg dig i Dag.
\par 20 Alt i Går optegned jeg til dig, alt i Forgårs Råd og Kundskab
\par 21 for at lære dig rammende Sandhedsord, at du kan svare sandt, når du spørges.
\par 22 Røv ej fra den ringe, fordi han er ringe, knus ikke den arme i Porten:
\par 23 thi HERREN fører deres Sag og raner deres Ransmænds Liv.
\par 24 Vær ej Ven med den, der let bliver hidsig, omgås ikke vredladen Mand,
\par 25 at du ikke skal lære hans Stier og hente en Snare for din Sjæl.
\par 26 Hør ikke til dem, der giver Håndslag, dem, som borger for Gæld!
\par 27 Såfremt du ej kan betale, tager man Sengen, du ligger i.
\par 28 Flyt ej ældgamle Skel, dem, dine Fædre satte.
\par 29 Ser du en Mand, som er snar til sin Gerning, da skal han stedes for Konger, ikke for Folk af ringe Stand.

\chapter{23}

\par 1 Når du sidder til bords hos en Stormand, mærk dig da nøje, hvem du har for dig,
\par 2 og sæt dig en Kniv på Struben, i Fald du er alt for sulten.
\par 3 Attrå ikke hans lækre Retter, thi det er svigefuld kost.
\par 4 Slid dig ikke op for at vinde dig Rigdom, brug ej din Forstand dertil!
\par 5 Skal dit Blik flyve efter den uden at finde den? Visselig gør den sig Vinger som Ørnen, der flyver mod Himlen.
\par 6 Spis ej den misundeliges Brød, attrå ikke hans lækre Retter;
\par 7 thi han sidder med karrige Tanker; han siger til dig: "Spis og drik!" men hans Hjerte er ikke med dig.
\par 8 Den Bid, du har spist, må du udspy, du spilder dine fagre Ord.
\par 9 Tal ikke for Tåbens Ører, thi din kloge Tale agter han ringe.
\par 10 Flyt ej ældgamle Skel, kom ikke på faderløses Mark;
\par 11 thi deres Løser er stærk, han fører deres Sag imod dig.
\par 12 Vend dit Hjerte til Tugt, dit Øre til Kundskabs Ord.
\par 13 Spar ej Drengen for Tugt; når du slår ham med Riset, undgår han Døden;
\par 14 du slår ham vel med Riset, men redder hans Liv fra Dødsriget.
\par 15 Min Søn, er dit Hjerte viist, så glæder mit Hjerte sig også,
\par 16 og mine Nyrer jubler, når dine Læber taler, hvad ret er!
\par 17 Dit Hjerte være ikke skinsygt på Syndere, men stadig ivrigt i HERRENs Frygt;
\par 18 en Fremtid har du visselig da, dit Håb bliver ikke til intet.
\par 19 Hør, min Søn, og bliv viis, lad dit Hjerte gå den lige Vej.
\par 20 Hør ikke til dem, der svælger i Vin, eller dem, der frådser i Kød;
\par 21 thi Dranker og Frådser forarmes, Søvn giver lasede Klæder.
\par 22 Hør din Fader, som avlede dig, ringeagt ikke din gamle Moder!
\par 23 Køb Sandhed og sælg den ikke, Visdom, Tugt og Forstand.
\par 24 Den retfærdiges Fader jubler; har man avlet en Vismand, glædes man ved ham;
\par 25 din Fader og Moder glæde sig, hun, der fødte dig, juble!
\par 26 Giv mig dit Hjerte, min Søn, og lad dine Øjne synes om mine Veje!
\par 27 Thi en bundløs Grav er Skøgen, den fremmede Kvinde, en snæver Brønd;
\par 28 ja, som en Stimand ligger hun på Lur og øger de troløses Tal blandt Mennesker.
\par 29 Hvem har Ak, og hvem har Ve, hvem har Kiv, og hvem har Klage? Hvem har Sår uden Grund, hvem har sløve Øjne?
\par 30 De, som sidder sent over Vinen, som kommer for at smage den stærke Drik.
\par 31 Se ikke til Vinen, hvor rød den er, hvorledes den perler i Bægeret; den glider så glat,
\par 32 men bider til sidst som en Slange og spyr sin Gift som en Øgle;
\par 33 dine Øjne skuer de sælsomste Ting, og bagvendt taler dit Hjerte;
\par 34 du har det, som lå du midt i Havet, som lå du oppe på en Mastetop.
\par 35 "De slog mig, jeg følte ej Smerte, gav mig Hug, jeg mærked det ikke; når engang jeg vågner igen, så søger jeg atter til Vinen!"

\chapter{24}

\par 1 Misund ej onde Folk, hav ikke lyst til at være med dem;
\par 2 thi deres Hjerte pønser på Vold, deres Læbers Ord volder Men.
\par 3 Ved Visdom bygges et Hus, ved Indsigt holdes det oppe,
\par 4 ved Kundskab fyldes kamrene med alskens kosteligt, herligt Gods.
\par 5 Vismand er større end Kæmpe, kyndig Mand mer end Kraftkarl.
\par 6 Thi Krig skal du føre efter modent Overlæg, vel står det til, hvor mange giver Råd.
\par 7 Visdom er Dåren for høj, han åbner ej Munden i Porten.
\par 8 Den, der har ondt i Sinde, kaldes en rænkefuld Mand.
\par 9 Hvad en Dåre har for, er Synd, en Spotter er Folk en Gru.
\par 10 Taber du Modet på Trængslens Dag, da er din Kraft kun ringe.
\par 11 Frels dem, der slæbes til Døden, red dem, der vakler hen for at dræbes.
\par 12 Siger du: "Se, jeg vidste det ikke" - mon ej han, der vejer Hjerter, kan skønne? Han, der tager Vare på din Sjæl, han ved det, han gengælder Mennesker, hvad de har gjort.
\par 13 Spis Honning, min Søn, det er godt, og Kubens Saft er sød for din Gane;
\par 14 vid, at så er og Visdom for Sjælen! Når du finder den, har du en Fremtid, dit Håb bliver ikke til intet.
\par 15 Lur ej på den retfærdiges Bolig, du gudløse, ødelæg ikke hans Hjem;
\par 16 thi syv Gange falder en retfærdig og står op, men gudløse styrter i Fordærv.
\par 17 Falder din Fjende, så glæd dig ikke, snubler han, juble dit Hjerte ikke,
\par 18 at ikke HERREN skal se det med Mishag og vende sin Vrede fra ham.
\par 19 Græm dig ej over Ugerningsmænd, misund ikke de gudløse;
\par 20 thi den onde har ingen Fremtid, gudløses Lampe går ud.
\par 21 Frygt HERREN og Kongen, min Søn, indlad dig ikke med Folk, som gør Oprør;
\par 22 thi brat kommer Ulykke fra dem, uventet Fordærv fra begge.
\par 23 Også følgende Ordsprog er af vise Mænd. Partiskhed i Retten er ilde.
\par 24 Mod den, som kender en skyldig fri, er Folkeslags Banden, Folkefærds Vrede;
\par 25 men dem, der dømmer med Ret, går det vel, dem kommer Lykkens Velsignelse over.
\par 26 Et Kys på Læberne giver den, som kommer med ærligt Svar.
\par 27 Fuldfør din Gerning udendørs, gør dig færdig ude på Marken og byg dig siden et Hus!
\par 28 Vidn ikke falsk mod din Næste, vær ikke letsindig med dine Læber;
\par 29 sig ikke: "Jeg gør mod ham, som han gjorde mod mig, jeg gengælder hver hans Gerning."
\par 30 Jeg kom forbi en lad Mands Mark og et uforstandigt Menneskes Vingård;
\par 31 se, den var overgroet af Tidsler, ganske skjult af Nælder; Stendiget om den lå nedbrudt.
\par 32 Jeg skued og skrev mig det bag Øre, jeg så og tog Lære deraf:
\par 33 Lidt Søvn endnu, lidt Blund, lidt Hvile med samlagte Hænder:
\par 34 Som en Stimand kommer da Fattigdom over dig, Trang som en skjoldvæbnet Mand.

\chapter{25}

\par 1 Følgende er også ordsprog af SALOMO, som Kong Ezekias af Judas Mænd samlede.
\par 2 Guds Ære er det at skjule en Sag, Kongers Ære at granske en Sag.
\par 3 Himlens Højde og Jordens Dybde og Kongers Hjerte kan ingen granske.
\par 4 Når Slagger fjernes fra Sølv, så bliver det hele lutret;
\par 5 når gudløse fjernes fra Koogen, grundfæstes hans Trone ved Retfærd.
\par 6 Bryst dig ikke for Kongen og stil dig ikke på de stores Plads;
\par 7 det er bedre, du får Bud: "Kom heropl" end man flytter dig ned for en Stormands Øjne. Hvad end dine Øjne har set,
\par 8 skrid ikke til Trætte straks; thi hvad vil du siden gøre, når din Næste gør dig til Skamme?
\par 9 Før Sagen med din Næste til Ende, men røb ej Andenmands Hemmelighed
\par 10 thi ellers vil den, der bører det, smæde dig og dit onde Rygte aldrig dø hen.
\par 11 Æbler af Guld i Skåle af Sølv er Ord, som tales i rette Tid.
\par 12 En Guldring, et gyldent Smykke er revsende Vismand for lyttende Øre.
\par 13 Som kølende Sne en Dag i Høst er pålideligt Bud for dem, der sender ham; han kvæger sin Herres Sjæl.
\par 14 Som Skyer og Blæst uden Regn er en Mand, der skryder med skrømtet Gavmildhed.
\par 15 Ved Tålmod overtales en Dommer, mild Tunge sønderbryder Ben.
\par 16 Finder du Honning, så spis til Behov, at du ikke bliver mæt og igen spyr den ud.
\par 17 Sæt sjældent din Fod i din Næstes Hus, at han ej får for meget af dig og ledes.
\par 18 Som Stridsøkse, Sværd og hvassen Pil er den, der vidner falsk mod sin Næste.
\par 19 Som ormstukken Tand og vaklende Fod er troløs Mand på Trængselens Dag.
\par 20 Som at lægge Frakken, når det er Frost, og hælde surt over Natron, så er det at synge for mismodig Mand.
\par 21 Sulter din Fjende, så giv ham at spise, tørster han, giv ham at drikke;
\par 22 da sanker du gloende Kul på hans Hoved, og HERREN lønner dig for det.
\par 23 Nordenvind fremkalder Regn, bagtalende Tunge vrede Miner.
\par 24 Hellere bo i en Krog på Taget end fælles Hus med trættekær Kvinde.
\par 25 Hvad koldt Vand er for en vansmægtet Sjæl, er Glædesbud fra et Land i det fjerne.
\par 26 Som grumset Kilde og ødelagt Væld er retfærdig, der vakler i gudløses Påsyn.
\par 27 Ej godt at spise for megen Honning, spar på hædrende Ord.
\par 28 Som åben By uden Mur er en Mand, der ikke kan styre sit Sind.

\chapter{26}

\par 1 Som Sne om Somren og Regn Høsten så lidt hører Ære sig til for en Tåbe.
\par 2 Som en Spurv i Fart, som en Svale i Flugt så rammer ej Banden mod sagesløs Mand.
\par 3 Svøbe for Hest, Bidsel for Æsel og Ris for Tåbers Ryg.
\par 4 Svar ej Tåben efter hans Dårskab, at ikke du selv skal blive som han.
\par 5 Svar Tåben efter hans Dårskab, at han ikke skal tykkes sig viis.
\par 6 Den afhugger Fødderne og inddrikker Vold, som sender Bud ved en Tåbe.
\par 7 Slappe som den lammes Ben er Ordsprog i Tåbers Mund.
\par 8 Som en, der binder Stenen fast i Slyngen, er den, der hædrer en Tåbe.
\par 9 Som en Tornekæp, der falder den drukne i Hænde, er Ordsprog i Tåbers Mund.
\par 10 Som en Skytte, der sårer enhver, som kommer, er den, der lejer en Tåbe og en drukken.
\par 11 Som en Hund, der vender sig om til sit Spy, er en Tåbe, der gentager Dårskab.
\par 12 Ser du en Mand, der tykkes sig viis, for en Tåbe er der mere Håb end for ham.
\par 13 Den lade siger: "Et Rovdyr på Vejen, en Løve ude på Torvene!"
\par 14 Døren drejer sig på sit Hængsel, den lade på sit Leje.
\par 15 Den lade rækker til Fadet, men gider ikke føre Hånden til Munden.
\par 16 Den lade tykkes sig større Vismand end syv, der har kloge Svar.
\par 17 Den griber en Hund i Øret, som blander sig i uvedkommende Strid.
\par 18 Som en vanvittig Mand, der udslynger Gløder, Pile og Død,
\par 19 er den, der sviger sin Næste og siger: "Jeg spøger jo kun."
\par 20 Er der intet Brænde, går Ilden ud, er der ingen Bagtaler, stilles Trætte.
\par 21 Trækul til Gløder og Brænde til Ild og trættekær Mand til at optænde Kiv.
\par 22 Bagtalerens Ord er som Lækkerbidskener, de synker dybt i Legemets Kamre.
\par 23 Som Sølvovertræk på et Lerkar er ondsindet Hjerte bag glatte Læber.
\par 24 Avindsmand hykler med Læben, i sit Indre huser han Svig;
\par 25 gør han Røsten venlig, tro ham dog ikke, thi i hans Hjerte er syvfold Gru.
\par 26 Den, der dølger sit Had med Svig, hans Ondskab kommer frem i Folkets Forsamling.
\par 27 I Graven, man graver, falder man selv, af Stenen, man vælter, rammes man selv.
\par 28 Løgnetunge giver mange Hug, hyklersk Mund volder Fald.

\chapter{27}

\par 1 Ros dig ikke af Dagen i Morgen, du ved jo ikke, hvad Dag kan bringe.
\par 2 Lad en anden rose dig, ikke din Mund, en fremmed, ikke dine egne Læber.
\par 3 Sten er tung, og Sand vejer til, men tung fremfor begge er Dårers Galde.
\par 4 Vrede er grum, og Harme skummer, men Skinsyge, hvo kan stå for den?
\par 5 Hellere åbenlys Revselse end Kærlighed, der skjules.
\par 6 Vennehånds Hug er ærligt mente, Avindsmands Kys er mange.
\par 7 Den mætte vrager Honning, alt beskt er sødt for den sultne.
\par 8 Som Fugl, der må fly fra sin Rede, er Mand, der må fly fra sit Hjem:
\par 9 Olie og Røgelse fryder Sindet, men Sjælen sønderslides af Kummer.
\par 10 Slip ikke din Ven og din Faders Ven, gå ej til din Broders Hus på din Ulykkes Dag. Bedre er Nabo ved Hånden end Broder i det fjerne.
\par 11 Vær viis, min Søn, og glæd mit Hjerte, at jeg kan svare den, der smæder mig.
\par 12 Den kloge ser Faren og søger i Skjul, tankeløse går videre og bøder,
\par 13 Tag hans Klæder, han borged for en anden, pant ham for fremmedes Skyld!
\par 14 Den, som årle højlydt velsigner sin Næste, han får det regnet for Banden.
\par 15 Ustandseligt Tagdryp en Regnvejrsdag og trættekær Kvinde ligner hinanden;
\par 16 den, som vil skjule hende, skjuler Vind, og hans højre griber i Olie.
\par 17 Jern skærpes med Jern, det ene Menneske skærper det andet.
\par 18 Røgter man et Figentræ, spiser man dets Frugt; den, der vogter sin Herre, æres.
\par 19 Som i Vandspejlet Ansigt møder Ansigt, slår Menneskehjerte Menneske i Møde.
\par 20 Dødsrige og Afgrund kan ikke mættes, ej heller kan Menneskens Øjne mættes.
\par 21 Digel til Sølv og Ovn til Guld, efter sit Ry bedømmes en Mand.
\par 22 0m du knuste en Dåre i Morter med Støder midt imellem Gryn, hans Dårskab veg dog ej fra ham.
\par 23 Mærk dig, hvorledes dit Småkvæg ser ud, hav Omhu for dine Hjorde;
\par 24 thi Velstand varer ej evigt, Rigdom ikke fra Slægt til Slægt;
\par 25 er Sommergræsset svundet, Grønt spiret frem, og sankes Bjergenes Urter,
\par 26 da har du Lam til at give dig Klæder og Bukke til at købe en Mark,
\par 27 Gedemælk til Mad for dig og dit Hus, til Livets Ophold for dine Piger.

\chapter{28}

\par 1 Den gudløse flyr, skønt ingen er efter ham; tryg som en Løve er den retfærdige.
\par 2 Ved Voldsmands Brøde opstår Strid, den kvæles af Mand med Forstand.
\par 3 En fattig Tyran, der kuer de ringe, er Regn, der hærger og ej giver Brød.
\par 4 Hvo Loven sviger, roser de gudløse, hvo Loven holder, er på Krigsfod med dem.
\par 5 Ildesindede fatter ej Ret; alt fatter de, som søger HERREN.
\par 6 Hellere en fattig med lydefri Færd end en, som går Krogveje, er han end rig.
\par 7 Forstandig Søn tager Vare på Loven, men Drankeres Fælle gør sin Fader Skam.
\par 8 Hvo Velstand øger ved Åger og Opgæld, samler til en, som er mild mod de ringe.
\par 9 Den, der vender sit Øre fra Loven, endog hans Bøn er en Gru.
\par 10 Leder man retsindige vild på onde Veje, falder man selv i sin Grav; men de lydefri arver Lykke.
\par 11 Rigmand tykkes sig viis, forstandig Småmand gennemskuer ham.
\par 12 Når retfærdige jubler, er Herligheden stor, vinder gudløse frem, skal man lede efter Folk.
\par 13 At dølge sin Synd fører ikke til Held, men bekendes og slippes den, finder man Nåde.
\par 14 Saligt det Menneske, som altid ængstes, men forhærder man sit Hjerte, falder man i Ulykke.
\par 15 En brølende Løve, en grådig Bjørn er en gudløs, som styrer et ringe Folk.
\par 16 Uforstandig Fyrste øver megen Vold, langt Liv får den, der hader Rov.
\par 17 Et Menneske, der tynges af Blodskyld, er på Flugt til sin Grav; man hjælpe ham ikke.
\par 18 Den, som vandrer lydefrit, frelses, men den, som går Krogveje, falder i Graven.
\par 19 Den mættes med brød, som dyrker sin Jord, med Fattigdom den, der jager efter Tomhed.
\par 20 Ærlig Mand velsignes rigt, men Jag efter Rigdom undgår ej Straf.
\par 21 At være partisk er ikke godt, en Mand kan forse sig for en Bid Brød.
\par 22 Misundelig Mand vil i Hast vinde Gods; at Trang kommer over ham, ved han ikke.
\par 23 Den, der revser, får Tak til sidst fremfor den, hvis Tunge er slesk.
\par 24 Stjæle fra Forældre og nægte, at det, er Synd, er at være Fælle med hærgende Mand.
\par 25 Den vindesyge vækker Splid, men den, der stoler på HERREN, kvæges.
\par 26 Den, der stoler på sit Vid, er en Tåbe, men den, der vandrer i Visdom, reddes.
\par 27 Hvo Fattigmand giver, skal intet fattes, men mangefold bandes, hvo Øjnene lukker.
\par 28 Vinder gudløse frem, kryber Folk i Skjul; når de omkommer, bliver de retfærdige mange.

\chapter{29}

\par 1 Hvo Nakken gør stiv, skønt revset tit, han knuses brat uden Lægedom.
\par 2 Er der mange retfærdige, glædes Folket, men råder de gudløse, sukker Folket.
\par 3 Hvo Visdom elsker, glæder sin Fader, hvo Skøger omgås, bortødsler Gods.
\par 4 Kongen grundfæster Landet med Ret, en Udsuger lægger det øde.
\par 5 Mand, der smigrer sin Næste, breder et Net for hans Fod.
\par 6 I sin Brøde hildes den onde, den retfærdige jubler af Glæde.
\par 7 Den retfærdige kender de ringes Retssag; den gudløse skønner intet.
\par 8 Spottere ophidser Byen, men Vismænd, de stiller Vrede.
\par 9 Går Vismand i Rette med Dåre, vredes og ler han, alt preller af.
\par 10 De blodtørstige hader lydefri Mand, de retsindige tager sig af ham.
\par 11 En Tåbe slipper al sin Voldsomhed løs, Vismand stiller den omsider.
\par 12 En Fyrste, som lytter til Løgnetale, får lufter gudløse Tjenere.
\par 13 Fattigmand og Blodsuger mødes, HERREN giver begges Øjne Glans.
\par 14 En Konge, der dømmer de ringe med Ret, hans Trone står fast evindelig.
\par 15 Ris og Revselse, det giver Visdom, uvorn Dreng gør sin Moder Skam.
\par 16 Bliver mange gudløse tiltager Synd; retfærdige ser med Fryd deres Fald.
\par 17 Tugt din Søn, så kvæger han dig og bringer din Sjæl, hvad der smager.
\par 18 Uden Syner forvildes et Folk; salig den, der vogter på Loven.
\par 19 Med Ord lader Træl sig ikke tugte, han fatter dem vel, men adlyder ikke.
\par 20 Ser du en Mand, der er hastig til Tale, for en Tåbe er der snarere Håb end for ham.
\par 21 Forvænner man sin Træl fra ung, vil han til sidst være Herre.
\par 22 Hidsig Mand vækker Strid, vredladen Mand gør megen Synd.
\par 23 Et Menneskes Hovmod ydmyger ham, den ydmyge opnår Ære.
\par 24 Hæleren hader sit Liv, han hører Forbandelsen, men melder intet.
\par 25 Frygt for Mennesker leder i Snare, men den, der stoler på HERREN, er bjærget.
\par 26 Mange søger en Fyrstes Gunst; Mands Ret er dog fra HERREN.
\par 27 Urettens Mand er retfærdiges Gru, hvo redeligt vandrer, gudløses Gru.

\chapter{30}

\par 1 Massaiten Agur, Jakes Søns ord. Manden siger: Træt har jeg slidt mig, Gud, træt har jeg slidt mig, Gud, jeg svandt hen;
\par 2 thi jeg er for dum til at regnes for Mand, Mands Vid er ikke i mig;
\par 3 Visdom lærte jeg ej, den Hellige lærte jeg ikke at kende.
\par 4 Hvo opsteg til Himlen og nedsteg igen, hvo samlede Vinden i sine Næver, hvo bandt Vandet i et Klæde, hvo greb fat om den vide Jord? Hvad er hans Navn og hans Søns Navn? Du kender det jo.
\par 5 Al Guds Tale er ren, han er Skjold for dem, der lider på ham.
\par 6 Læg intet til hans Ord, at han ikke skal stemple dig som Løgner.
\par 7 Tvende Ting har jeg bedet dig om, nægt mig dem ej, før jeg dør:
\par 8 Hold Svig og Løgneord fra mig: giv mig hverken Armod eller Rigdom, men lad mig nyde mit tilmålte Brød,
\par 9 at jeg ikke skal blive for mæt og fornægte og sige: "Hvo er HERREN?" eller blive for fattig og stjæle og volde min Guds Navn Men.
\par 10 Bagtal ikke en Træl for hans Herre, at han ikke forbander dig, så du må bøde.
\par 11 Der findes en Slægt, som forbander sin Fader og ikke velsigner sin Moder,
\par 12 en Slægt, der tykkes sig ren og dog ej har tvættet Snavset af sig,
\par 13 en Slægt med de stolteste Øjne, hvis Blikke er fulde af Hovmod.
\par 14 en Slægt, hvis Tænder er Sværd hvis Kæber er skarpe Knive, så de æder de arme ud af Landet, de fattige ud af Menneskers Samfund.
\par 15 Blodiglen har to Døtre: Givhid, Givhid! Der er tre, som ikke kan mættes, fire, som aldrig får nok:
\par 16 Dødsriget og det golde Moderliv, Jorden, som aldrig mættes af Vand, og Ilden, som aldrig får nok.
\par 17 Den, som håner sin Fader og spotter sin gamle Moder, hans Øje udhakker Bækkens Ravne, Ørneunger får det til Æde.
\par 18 Tre Ting undres jeg over, fire fatter jeg ikke:
\par 19 Ørnens Vej på Himlen, Slangens Vej på Klipper, Skibets Vej på Havet, Mandens Vej til den unge Kvinde.
\par 20 Så er en Ægteskabsbryderskes Færd: Hun spiser og tørrer sig om Munden og siger: "Jeg har ikke gjort noget ondt!"
\par 21 Under tre Ting skælver et Land, fire kan det ikke bære:
\par 22 En Træl, når han gøres til Konge, en Nidding, når han spiser sig mæt,
\par 23 en bortstødt Hustru, når hun bliver gift, en Trælkvinde, når hun arver sin Frue.
\par 24 Fire på Jorden er små, visere dog end Vismænd:
\par 25 Myrerne, de er et Folk uden Styrke, samler dog Føde om Somren;
\par 26 Klippegrævlinger, et Folk uden Magt, bygger dog Bolig i Klipper;
\par 27 Græshopper, de har ej Konge, drager dog ud i Rad og Række;
\par 28 Firbenet, det kan man gribe med Hænder, er dog i Kongers Paladser.
\par 29 Tre skrider stateligt frem, fire har statelig Gang:
\par 30 Løven, Kongen blandt Dyrene, som ikke viger for nogen;
\par 31 en sadlet Stridshest, en Buk, en Konge midt i sin Hær.
\par 32 Har du handlet som Dåre i Overmod, tænker du ondt, da Hånd for Mund!
\par 33 Thi Tryk på Mælk giver Ost, Tryk på Næsen Blod og Tryk på Vrede Trætte.

\chapter{31}

\par 1 Kong Lemuel af Massas Ord; som hans Moder tugtede ham med.
\par 2 Hvad, Lemuel, min Søn, min førstefødte, hvad skal jeg sige dig, hvad, mit Moderlivs Søn, hvad, mine Løfters Søn?
\par 3 Giv ikke din Kraft til Kvinder, din Kærlighed til dem, der ødelægger Konger.
\par 4 Det klæder ej Konger, Lemuel, det klæder ej Konger at drikke Vin eller Fyrster at kræve stærke Drikke,
\par 5 at de ikke skal drikke og glemme Vedtægt og bøje Retten for alle arme.
\par 6 Giv den segnende stærke Drikke, og giv den mismodige Vin;
\par 7 lad ham drikke og glemme sin Fattigdom, ej mer ihukomme sin Møje.
\par 8 Luk Munden op for den stumme, for alle lidendes Sag;
\par 9 luk Munden op og døm retfærdigt, skaf den arme og fattige Ret!
\par 10 Hvo finder en duelig Hustru? Hendes Værd står langt over Perlers.
\par 11 Hendes Husbonds Hjerte stoler på hende, på Vinding skorter det ikke.
\par 12 Hun gør ham godt og intet ondt alle sine Levedage.
\par 13 Hun sørger for Uld og Hør, hun bruger sine Hænder med Lyst.
\par 14 Hun er som en Købmands Skibe, sin Føde henter hun langvejs fra.
\par 15 Endnu før Dag står hun op og giver Huset Mad, sine Piger deres tilmålte Del.
\par 16 Hun tænker på en Mark og får den, hun planter en Vingård, for hvad hun har tjent.
\par 17 Hun bælter sin Hofte med Kraft, lægger Styrke i sine Arme.
\par 18 Hun skønner, hendes Husholdning lykkes, hendes Lampe går ikke ud om Natten.
\par 19 Hun rækker sine Hænder mod Rokken, Fingrene tager om Tenen.
\par 20 Hun rækker sin Hånd til den arme, rækker Armene ud til den fattige.
\par 21 Af Sne har hun intet at frygte for sit Hus, thi hele hendes Hus er klædt i Skarlagen.
\par 22 Tæpper laver hun sig, hun er klædt i Byssus og Purpur.
\par 23 Hendes Husbond er kendt i Portene, når han sidder blandt Landets Ældste.
\par 24 Hun væver Linned til Salg og sælger Bælter til Kræmmeren.
\par 25 Klædt i Styrke og Hæder går hun Morgendagen i Møde med Smil.
\par 26 Hun åbner Munden med Visdom, med mild Vejledning på Tungen.
\par 27 Hun våger over Husets Gænge og spiser ej Ladheds Brød.
\par 28 Hendes Sønner står frem og giver hende Pris, hendes Husbond synger hendes Lov:
\par 29 "Mange duelige Kvinder findes, men du står over dem alle!"
\par 30 Ynde er Svig og Skønhed Skin; en Kvinde, som frygter HERREN, skal roses.
\par 31 Lad hende få sine Hænders Frugt, hendes Gerninger synger hendes Lov i Portene.


\end{document}