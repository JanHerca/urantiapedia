\begin{document}

\title{Titusbrevet}


\chapter{1}

\par 1 Paulus, Guds Tjener og Jesu Kristi Apostel til at virke Tro hos Guds udvalgte og Erkendelse at Sandheden angående Gudsfrygt,
\par 2 i Håb om evigt Liv, hvilket Gud, som ikke lyver, har forjættet fra evige Tider,
\par 3 men i sin Tid har han åbenbaret sit Ord ved den Prædiken, som er bleven mig betroet efter Guds, vor Frelsers Befaling:
\par 4 til Titus, mit ægte Barn i fælles Tro: Nåde og Fred fra Gud Fader og Kristus Jesus vor Frelser!
\par 5 Derfor efterlod jeg dig på Kreta, for at du skulde bringe i Orden, hvad der stod tilbage, og indsætte Ældste i hver By, som jeg pålagde dig,
\par 6 såfremt en er ustraffelig, een Kvindes Mand og har troende Børn, der ikke ere beskyldte for Ryggesløshed eller ere genstridige.
\par 7 Thi en Tilsynsmand bør være ustraffelig som en Guds Husholder, ikke selvbehagelig, ikke vredagtig, ikke hengiven til Vin, ikke til Slagsmål, ikke til slet Vinding,
\par 8 men gæstfri, elskende det gode, sindig, retfærdig, from, afholdende;
\par 9 en Mand, som holder fast ved det troværdige Ord efter Læren, for at han kan være dygtig til både at formane ved den sunde Lære og at gendrive dem, som sige imod.
\par 10 Thi mange ere genstridige, føre intetsigende Snak og dåre Sindet, især de af Omskærelsen;
\par 11 dem bør man stoppe Munden på; thi de forvende hele Huse ved at føre utilbørlig Lære for slet Vindings Skyld.
\par 12 En af dem, en af deres egne Profeter, har sagt: "Kretere ere altid Løgnere, onde Dyr, lade Buge."
\par 13 Dette Vidnesbyrd er sandt. Derfor skal du sætte dem strengelig i Rette, for at de må blive sunde i Troen
\par 14 og ikke agte på jødiske Fabler og Bud af Mennesker, som vende sig bort fra Sandheden.
\par 15 Alt er rent for de rene; men for de besmittede og vantro er intet rent, men både deres Sind og Samvittighed er besmittet.
\par 16 De sige, at de kende Gud, men med deres Gerninger fornægte de ham, vederstyggelige, som de ere, og ulydige og uduelige til al god Gerning.

\chapter{2}

\par 1 Du derimod, tal, hvad der sømmer sig for den sunde Lære:
\par 2 at gamle Mænd skulle være ædruelige, ærbare, sindige, sunde i Troen, i Kærligheden, i Udholdenheden;
\par 3 at gamle Kvinder ligeledes skulle skikke sig, som det sømmer sig hellige, ikke bagtale, ikke være forfaldne til megen Vin, men være Lærere i, hvad godt er,
\par 4 for at de må få de unge Kvinder til at besinde sig på at elske deres Mænd og at elske deres Børn,
\par 5 at være sindige, kyske, huslige, gode, deres egne Mænd undergivne, for at Guds Ord ikke skal bespottes.
\par 6 Forman ligeledes de unge Mænd til at være sindige,
\par 7 idet du i alle Måder viser dig selv som et Forbillede på gode Gerninger og i Læren viser Ufordærvethed, Ærbarhed,
\par 8 sund, ulastelig Tale, for at Modstanderen må blive til Skamme, når han intet ondt har at sige om os.
\par 9 Forman Trælle til at underordne sig under deres egne Herrer, at være dem til Behag i alle Ting, ikke sige imod,
\par 10 ikke besvige, men vise al god Troskab, for at de i alle Måder kunne være en Pryd for Guds, vor Frelsers Lære.
\par 11 Thi Guds Nåde er bleven åbenbaret til Frelse for alle Mennesker
\par 12 og opdrager os til at forsage Ugudeligheden og de verdslige Begæringer og leve sindigt og retfærdigt og gudfrygtigt i den nærværende Verden;
\par 13 forventende det salige Håb og den store Guds og vor Frelsers Jesu Kristi Herligheds Åbenbarelse,
\par 14 han, som gav sig selv for os, for at han måtte forløse os fra al Lovløshed og rense sig selv et Ejendomsfolk, nidkært til gode Gerninger.
\par 15 Tal dette, og forman og irettesæt med al Myndighed; lad ingen ringeagte dig!

\chapter{3}

\par 1 Påmind dem om at underordne sig Øvrigheder og Myndigheder, at adlyde, at være redebonne til al god Gerning.
\par 2 ikke at forhåne nogen, ikke være stridslystne, men milde, og udvise al Sagtmodighed imod alle Mennesker.
\par 3 Thi også vi vare fordum uforstandige, ulydige, vildfarende, Slaver af Begæringer og, mange Hånde Lyster, vi levede i Ondskab og Avind, vare forhadte og hadede hverandre.
\par 4 Men da Guds, vor Frelsers Godhed og Menneskekærlighed åbenbaredes,
\par 5 frelste han os, ikke for de Retfærdigheds Gerningers Skyld, som vi havde gjort, men efter sin Barmhjertighed, ved Igenfødelsens Bad og Fornyelsen i den Helligånd,
\par 6 som han rigeligt udøste over os ved Jesus Kristus, vor Frelser,
\par 7 for at vi, retfærdiggjorte ved hans Nåde, skulde i Håb vorde Arvinger til evigt Liv.
\par 8 Den Tale er troværdig, og derom vil jeg, at du skal forsikre dem, for at de, som ere komne til Tro på Gud, skulle lægge Vind på at øve gode Gerninger. Dette er Menneskene godt og nyttigt.
\par 9 Men hold dig fra tåbelige Stridigheder og Slægtregistre og Kiv og Kampe om Loven; thi de ere unyttige og frugtesløse.
\par 10 Et kættersk Menneske skal du afvise efter een og to Ganges Påmindelse,
\par 11 da du ved, at en sådan er forvendt og synder, domfældt af sig selv.
\par 12 Når jeg sender Artemas til dig eller Tykikus, da gør dig Flid for at komme til mig i Nikopolis;thi der har jeg besluttet at overvintre.
\par 13 Zenas den lovkyndige og Apollos skal du omhyggeligt hjælpe på Vej, for at intet skal fattes dem.
\par 14 Men lad også vore lære at øve gode Gerninger, hvor der er Trang dertil, for at de ikke skulle være uden Frugt.
\par 15 Alle, som ere hos mig, hilse dig. Hils dem, som elske os i Troen.



\end{document}