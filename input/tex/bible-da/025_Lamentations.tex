\begin{document}

\title{Lamentations}


\chapter{1}

\par 1 Hvor sidder hun ene, den by så folkerig før - mægtig blandt Folkene før, men nu som en Enke! Fyrstinden blandt Lande er nu sat til at trælle.
\par 2 Hun græder og græder om Natten med Tårer på Kind; ingen af alle hendes Elskere bringer hende Trøst, alle Vennerne sveg og blev hendes Fjender.
\par 3 Af Trang og tyngende Trældom udvandred Juda; blandt Folkene sidder hun nu og finder ej Ro, alle Forfølgerne nåede hende midt i Trængslerne.
\par 4 Vejene til Zion sørger, uden Højtidsgæster, alle hendes Porte er øde, Præsterne sukker, hendes Jomfruer knuges af Kvide, hun selv er i Vånde.
\par 5 Hendes Avindsmænd er Herrer, hendes Fjender trygge, thi Kvide fik hun af HERREN for Mængden af Synder, hendes Børn drog bort som Fanger for Fjendens Åsyn.
\par 6 Og bort fra Zions Datter drog al hendes Pragt; som Hjorte, der ej finder Græsning, blev hendes Fyrster, de vandrede kraftløse bort for Forfølgernes Åsyn.
\par 7 Jerusalem mindes den Tid, hun blev arm og husvild, (alle sine kostelige Ting fra fordums Dage), i Fjendehånd faldt hendes Folk, og ingen hjalp, Fjender så til og lo, fordi hun gik under.
\par 8 Jerusalem syndede svart, blev derfor til Afsky; hun foragtes af alle sine Beundrere, de så hendes Blusel, derfor sukker hun dybt og vender sig bort.
\par 9 Hendes Urenhed pletter hendes Slæb, hun betænkte ej Enden; hun sank forfærdende dybt, og ingen trøster. Se min Elendighed, HERRE, thi Fjenden hoverer.
\par 10 Avindsmænd bredte deres Hånd over alle hendes Skatte, ja, ind i sin Helligdom så hun Hedninger komme, hvem du havde nægtet Adgang til din Forsamling.
\par 11 Alt hendes Folk måtte sukke, søgende Brød; de gav deres Skatte for Mad for at friste Livet. HERRE, se til og giv Agt på, hvorledes jeg hånes!
\par 12 Alle, som vandrer forbi, giv Agt og se, om det gives en Smerte som den, der er tilføjet mig, hvem HERREN voldte Harm på sin glødende Vredes Dag.
\par 13 Fra det høje sendte han Ild, der for ned i mine Ben; han spændte et Net for min Fod, han drev mig tilbage, han gjorde mig øde, syg både Dag og Nat.
\par 14 Der vogtedes på mine Synder, i hans Hånd blev de flettet, de kom som et Åg om min Hals, han brød min Kraft; Herren gav mig dem i Vold, som, er mig for stærke.
\par 15 Herren forkasted de vældige udi min Midte, han indbød til Fest på mig for at knuse mine unge, trådte Persen til Dom over Jomfruen, Judas Datter.
\par 16 Derover græder mit Øje, det strømmer med Tårer, thi langt har jeg til en Trøster, som kvæger min Sjæl; mine Børn er fortabt, thi Fjenden er blevet for stærk.
\par 17 Zion udrækker Hænderne, ingen trøster; mod Jakob opbød HERREN hans Fjender omkring ham; imellem dem er Jerusalem blevet til Afsky.
\par 18 HERREN, han er retfærdig, jeg modstod hans Mund. Hør dog, alle I Folkeslag, se min Smerte! Mine Jomfruer og unge Mænd drog bort som Fanger.
\par 19 Mine Elskere kaldte jeg ad de svigtede mig; mine Præster og Ældste opgav Ånden i Byen, thi Føde søgte de efter, men intet fandt de.
\par 20 Se, HERRE, hvor jeg er i Vånde, mit Indre i Glød, mit Hjerte er knust i mit Bryst, thi jeg var genstridig; ude mejede Sværdet og inde Døden.
\par 21 Hør, hvor jeg sukker, ingen bringer mig Trøst. De hørte min Ulykke, glæded sig, da du greb ind. Lad komme den Dag, du loved, dem gå det som mig!
\par 22 Læg al deres Ondskab for dig og gør med dem, som du gjorde med mig til Straf for al min Synd! Thi mange er mine Suk, mit Hjerte er sygt.

\chapter{2}

\par 1 Hvor har dog Herren i Vrede lagt mulm over Zion, slængt Israels herlighed ned fra Himmel til Jord og glemt sine Fødders Skammel på sin Vredes Dag.
\par 2 Herren har skånselsløst opslugt hver Bolig i Jakob, han nedbrød i Vrede Judas Datters Borge, slog dem til Jorden, skændede Rige og Fyrster,
\par 3 afhugged i glødende Vrede hvert Horn i Israel; sin højre drog han tilbage for Fjendens Åsyn og brændte i Jakob som en Lue, der åd overalt.
\par 4 På Fjendevis spændte han Buen, stod som en Uven; han dræbte al Øjnenes Lyst i Zions Datters Telt, udgød sin Vrede som Ild.
\par 5 Herren har vist sig som Fjende, opslugt Israel, opslugt alle Paladser, lagt Borgene øde, ophobet Jammer på Jammer i Judas Datter.
\par 6 Han nedrev sin Hytte, lagde sit Feststed øde, HERREN lod Fest og Sabbat gå ad Glemme i Zion, bortstødte i heftig Vrede Konge og Præst.
\par 7 Herren forkasted sit Alter, brød med sin Helligdom, hengav i Fjendens Hånd dets Paladsers Mure; man skreg i HERRENs Hus som på Festens Dag.
\par 8 HERREN fik i Sinde at ødelægge Zions Datters Mur, han udspændte Snoren, holdt ikke sin Hånd fra Fordærv, lod Vold og Mur få Sorg, de vansmægted sammen.
\par 9 I Jorden sank hendes Porte, Slåerne brød han. Blandt Folkene bor uden Lov hendes Konge og Fyrster, og ikke fanger Profeterne Syn fra HERREN.
\par 10 Zions datters Ældste sidder på Jorden i Tavshed; på Hovedet kaster de Støv, de er klædt i Sæk; Jerusalems Jomfruer sænker mod Jord deres Hoved.
\par 11 Mine Øjne hensvinder i Gråd, mit Indre gløder, mit Hjerte er knust, fordi mit Folk er brudt sammen; thi Børn og spæde forsmægter på Byens Torve;
\par 12 hver spørger sin Moder: "Hvor er der Korn og Vin?" forsmægter på Byens Torve som en, der er såret, idet de udånder Sjælen ved Moderens Bryst.
\par 13 Med hvad skal jeg stille dig lige, Jerusalems Datter, hvormed skal jeg ligne og trøste dig, Zions Jomfru? Thi dit Sammenbrud er stort som Havet, hvo læger dig vel?
\par 14 Profeternes Syner om dig var Tomhed og Løgn, de afsløred ikke din Skyld for at vende din Skæbne, Synerne gav dig kun tomme, vildende Udsagn.
\par 15 Over dig slog de Hænderne sammen, de, hvis Vej faldt forbi, de hån fløjted, rysted på Hoved ad Jerusalems Datter: "Er det da Byen, man kaldte den fuldendt skønne, al Jordens Glæde?"
\par 16 De opspærred Munden imod dig, alle dine Fjender, hånfløjted, skar Tænder og sagde: "Vi opslugte hende; ja, det er Dagen, vi vented, vi fik den at se."
\par 17 HERREN har gjort, som han tænkte, fuldbyrdet det Ord, han sendte i fordums Dage, brudt ned uden Skånsel, ladet Fjender glæde sig over dig, rejst Uvenners Horn.
\par 18 Råb højt til Herren, du Jomfru, Zions Datter, lad Tårerne strømme som Bække ved Dag og ved Nat, und dig ej Ro, lad ikke dit Øje få Hvile!
\par 19 Stå op og klag dig om Natten, når Vagterne skifter, udøs dit Hjerte som Vand for Herrens Åsyn, løft dine Hænder til ham for Børnenes Liv, som forsmægter af Hunger ved alle Gadernes Hjørner.
\par 20 HERRE se til og agt på, mod hvem du har gjort det. Skal Kvinder da æde den Livsfrugt, de kælede for, myrdes i Herrens Helligdom Præst og Profet?
\par 21 I Gaderne ligger på Jorden unge og gamle, mine Jomfruer og mine Ynglinge faldt for Sværdet; på din Vredesdag slog du ihjel, hugged ned uden Skånsel.
\par 22 Du bød mine Rædsler til Fest fra alle Sider. På HERRENs Vredes Dag undslap og frelstes ingen; min Fjende tilintetgjorde dem, jeg plejed og fostred.

\chapter{3}

\par 1 Jeg er den, der så nød ved hans vredes ris,
\par 2 mig har han ført og ledt i det tykkeste Mulm,
\par 3 ja, Hånden vender han mod mig Dagen lang.
\par 4 Mit Bød og min Hud har han opslidt, brudt mine Ben,
\par 5 han mured mig inde, omgav mig med Galde og Møje,
\par 6 lod mig bo i Mørke som de, der for længst er døde.
\par 7 Han har spærret mig inde og lagt mig i tunge Lænker.
\par 8 Om jeg end råber og skriger, min Bøn er stængt ude.
\par 9 Han spærred mine Veje med Kvader, gjorde Stierne kroge.
\par 10 Han blev mig en lurende Bjørn, en Løve i Baghold;
\par 11 han ledte mig vild, rev mig sønder og lagde mig øde;
\par 12 han spændte sin Bue; lod mig være Skive for Pilen.
\par 13 Han sendte sit Koggers Sønner i Nyrerne på mig;
\par 14 hvert Folk lo mig ud og smæded mig Dagen lang,
\par 15 med bittert mætted han mig, gav mig Malurt at drikke.
\par 16 Mine Tænder lod han bide i Flint, han trådte mig i Støvet;
\par 17 han skilte min Sjæl fra Freden, jeg glemte Lykken
\par 18 og sagde: "Min Livskraft, mit Håb til HERREN er ude."
\par 19 At mindes min Vånde og Flakken er Malurt og Galde;
\par 20 min Sjæl, den mindes det grant den grubler betynget.
\par 21 Det lægger jeg mig på Sinde, derfor vil jeg håbe:
\par 22 HERRENs Miskundhed er ikke til Ende, ikke brugt op,
\par 23 hans Nåde er ny hver Morgen, hans Trofasthed stor.
\par 24 Min Del er HERREN, (siger min Sjæl,) derfor håber jeg på ham.
\par 25 Dem, der bier på HERREN, er han god, den Sjæl, der ham søger;
\par 26 det er godt at håbe i Stilhed på HERRENs Frelse,
\par 27 godt for en Mand, at han bærer Åg i sin Ungdom.
\par 28 Han sidde ensom og tavs, når han lægger det på ham;
\par 29 han trykke sin Mund mod Støvet, måske er der Håb.
\par 30 Række Kind til den, der slår ham, mættes med Hån.
\par 31 Thi Herren bortstøder ikke for evigt,
\par 32 har han voldt Kvide, så ynkes han, stor er hans Nåde;
\par 33 ej af Hjertet plager og piner han Menneskens Børn.
\par 34 Når Landets Fanger til Hobe trædes under Fod,
\par 35 når Mandens Ret for den Højestes Åsyn bøjes,
\par 36 når en Mand lider Uret i sin Sag mon Herren ej ser det?
\par 37 Hvo taler vel, så det sker, om ej Herren byder?
\par 38 Kommer ikke både ondt og godt fra den Højestes Mund?
\par 39 Over hvad skal den levende sukke? Hver over sin Synd!
\par 40 Lad os ransage, granske vore Veje og vende os til HERREN,
\par 41 løfte Hænder og Hjerte til Gud i Himlen;
\par 42 vi syndede og stod imod, du tilgav ikke,
\par 43 men hylled dig i Vrede, forfulgte os, dræbte uden Skånsel,
\par 44 hylled dig i Skyer, så Bønnen ej nåed frem;
\par 45 til Skarn og til Udskud har du gjort os midt iblandt Folkene.
\par 46 De opspærred Munden imod os, alle vore Fjender.
\par 47 Vor Lod blev Gru og Grav og Sammenbruds Øde;
\par 48 Vandstrømme græder mit Øje, mit Folk brød sammen.
\par 49 Hvileløst strømmer mit Øje, det kender ej Ro,
\par 50 før HERREN skuer ned fra Himlen, før han ser til.
\par 51 Synet af Byens Døtre piner min Sjæl.
\par 52 Jeg joges som en Fugl af Fjender, hvis Had var grundløst,
\par 53 de spærred mig inde i en Grube, de stenede mig;
\par 54 Vand strømmed over mit Hoved, jeg tænkte: "Fortabt!"
\par 55 Dit Navn påkaldte jeg, HERRE, fra Grubens Dyb;
\par 56 du hørte min Røst: "O, gør dig ej døv for mit Skrig!"
\par 57 Nær var du den Dag jeg kaldte, du sagde: "Frygt ikke!"
\par 58 Du førte min Sag, o Herre, genløste mit Liv;
\par 59 HERRE, du ser, jeg lider Uret. skaf mig min Ret!
\par 60 Al deres Hævnlyst ser du, alle deres Rænker,
\par 61 du hører deres Smædeord HERRE, deres Rænker imod mig,
\par 62 mine Fjenders Tale og Tanker imod mig bestandig.
\par 63 Se dem, når de sidder eller står, deres Nidvise er jeg.
\par 64 Dem vil du gengælde, HERRE, deres Hænders Gerning,
\par 65 gør deres Hjerte forhærdet din Forbandelse over dem!
\par 66 forfølg dem i Vrede, udryd dem under din Himmel.

\chapter{4}

\par 1 Hvor Guldet blev sort, og skæmmet det ædle metal, de hellige Stene slængt hen på Gadernes Hjørner!
\par 2 Zions de dyre Sønner, der opvejed Guld, kun regnet for Lerkar, Pottemagerhænders Værk
\par 3 Selv Sjakaler byder Brystet til, giver Ungerne Die, men mit Folks Datter blev grum som Ørkenens Strudse.
\par 4 Den spædes Tunge hang fast ved Ganen af Tørst, Børnene tigged om Brød, og ingen gav dem.
\par 5 Folk, som levede lækkert, omkom på Gaden; Folk, som var båret på Purpur, favnede Skarnet.
\par 6 Mit Folks Datters Brøde var større end Synden i Sodom, som brat blev styrted, så Hænder ej rørtes derinde.
\par 7 Hendes Fyrster var renere end Sne, mer hvide end Mælk, deres Legeme rødere end Koral, som Safir deres Årer;
\par 8 mer sorte end Sod ser de ud, kan ej kendes på Gaden, Huden hænger ved Knoglerne, tør som Træ.
\par 9 Sværdets Ofre var bedre farne end Sultens, som svandt hen, dødsramte, af Mangel på Markens Grøde.
\par 10 Blide kvinders Hænder kogte deres Børn; da mit Folks Datter brød sammen, blev de dem til Føde.
\par 11 HERREN køled sin Vrede, udøste sin Harmglød,. han tændte i Zion en Ild, dets Grundvolde åd den.
\par 12 Ej troede Jordens Konger, ja ingen i Verden, at Uven og Fjende skulde stå i Jerusalems Porte.
\par 13 Det var for Profeternes Synd, for Præsternes Brøde, som i dets Midte udgød retfærdiges Blod.
\par 14 De vanked som blinde på Gaderne, tilsølet af Blod, rørte med Klæderne Ting, som ikke må røres.
\par 15 "Var jer! En uren!" råbte man; "Var jer dog for dem!" Når de flyr og vanker, råber man: "Bliv ikke her!"
\par 16 HERREN spredte dem selv, han så dem ej mer, Præster regned man ej eller ynked Profeter.
\par 17 End smægted vort Blik efter Hjælp, men kun for at skuffes, på Varden spejded vi efter det Folk, der ej hjælper.
\par 18 De lured på vort Fjed, fra Torvene holdt vi os borte; Enden var nær, vore Dage var omme, ja, Enden var kommet.
\par 19 Mer snare end Himlens Ørne var de, som jog os, på Bjergene satte de efter os, lured i Ørkenen,
\par 20 vor Livsånde, HERRENs Salvede, blev fanget i deres Grave, han, i hvis Skygge vi tænkte at leve blandt Folkene.
\par 21 Glæd dig og fryd dig, Edom, som bor i Uz! Også dig skal Bægeret nå, du skal blotte dig drukken.
\par 22 Din Skyld er til Ende, Zion, du forvises ej mer; han hjemsøger, Edom, din Skyld, afslører dine Synder.

\chapter{5}

\par 1 HERRE, kom vor skæbne i Hu, sku ned og se vor skændsel!
\par 2 Vor Arvelod tilfaldt fremmede, Udlændinge fik vore Huse.
\par 3 Forældreløse, faderløse er vi, som Enker er vore Mødre.
\par 4 Vort Drikkevand må vi købe, betale må vi vort Brænde.
\par 5 Åget trykker vor Nakke, vi trættes og finder ej Hvile.
\par 6 Ægypten rakte vi Hånd, Assur, for at mættes med Brød.
\par 7 Vore Fædre, som synded, er borte, og vi må bære deres Skyld.
\par 8 Over os råder Trælle, ingen frier os fra dem.
\par 9 Med Livsfare henter vi vort Brød, udsatte for Ørkenens Sværd.
\par 10 Vor Hud er sværtet som en Ovn af Hungerens svidende Lue.
\par 11 De skændede kvinder i Zion, Jomfruer i Judas Byer.
\par 12 Fyrster greb de og hængte, tog intet Hensyn til gamle.
\par 13 Ynglinge sattes til Kværnen, under Brændeknippet segnede Drenge.
\par 14 De gamle forsvandt fra Porten, de unge fra Strengenes Leg.
\par 15 Vort Hjertes Glæde er borte, vor Dans er vendt til Sorg.
\par 16 Kronen faldt af vort Hoved, ve os, at vi har syndet!
\par 17 Vort Hjerte blev derfor sygt, derfor vort Øje mørkt:
\par 18 For Zions Bjerg, som er øde, Ræve tumler sig der.
\par 19 Du, HERRE, troner for evigt, fra Slægt til Slægt står din trone.
\par 20 Hvi glemmer du os bestandig og svigter os alle dage?
\par 21 Omvend os, HERRE, til dig, så vender vi om, giv os nye Dage, som fordum!
\par 22 Eller har du helt stødt os bort, er din Vrede mod os uden Ende?


\end{document}