\begin{document}

\title{1 Corinthians}


\chapter{1}

\par 1 Paulus, Jesu Kristi kaldede Apostel ved Guds Villie, og Broderen Sosthenes
\par 2 til Guds Menighed, som er i Korinth, helligede i Kristus Jesus, hellige ifølge Kald tillige med alle dem, der på ethvert Sted påkalde vor Herres Jesu Kristi, deres og vor Herres Navn:
\par 3 Nåde være med eder og Fred fra Gud vor Fader og den Herre Jesus Kristus!
\par 4 Jeg takker min Gud altid for eder, for den Guds Nåde, som blev givet eder i Kristus Jesus,
\par 5 at I ved ham ere blevne rige i alt, i al Tale og al Kundskab,
\par 6 ligesom Vidnesbyrdet om Kristus er blevet stadfæstet hos eder,
\par 7 så at I ikke stå tilbage i nogen Nådegave, idet I forvente vor Herres Jesu Kristi Åbenbarelse,
\par 8 han, som også skal stadfæste eder indtil Enden som ustraffelige på vor Herres Jesu Kristi Dag,
\par 9 Trofast er Gud, ved hvem I bleve kaldede til Samfund med hans Søn, Jesus Kristus, vor Herre.
\par 10 Men jeg formaner eder, Brødre! ved vor Herres Jesu Kristi Navn, at I alle skulle føre samme Tale, og at der ikke må findes Splittelser iblandt eder, men at I skulle være forenede i det samme Sind og i den samme Mening.
\par 11 Thi det er blevet mig fortalt om eder, mine Brødre! af Bloes Husfolk, at der er Splidagtighed iblandt eder.
\par 12 Jeg mener dette, at enhver af eder siger: Jeg hører Paulus til, og jeg Apollos, og jeg Kefas, og jeg Kristus.
\par 13 Er Kristus delt? mon Paulus blev korsfæstet for eder? eller bleve I døbte til Paulus's Navn?
\par 14 Jeg takker Gud for, at jeg ikke døbte nogen af eder, uden Krispus og Kajus,
\par 15 for at ikke nogen skal sige, at I bleve døbte til mit Navn.
\par 16 Dog, jeg døbte også Stefanas's Hus; ellers ved jeg ikke, om jeg døbte nogen anden.
\par 17 Thi Kristus sendte mig ikke for at døbe, men for at forkynde Evangeliet, ikke med vise Ord, for at Kristi Kors ikke skulde tabe sin Kraft.
\par 18 Thi Korsets Ord er vel for dem, som fortabes, en Dårskab, men for dem, som frelses, for os er det en Guds Kraft.
\par 19 Thi der er skrevet: "Jeg vil lægge de vises Visdom øde, og de forstandiges Forstand vil jeg gøre til intet."
\par 20 Hvor er der en viis? hvor er der en skriftklog? hvor er der en Ordkæmper al denne verden? har Gud ikke gjort Verdens Visdom til Dårskab?
\par 21 Thi efterdi Verden ved sin Visdom ikke erkendte Gud i hans Visdom, behagede det Gud ved Prædikenens Dårskab at frelse dem, som tro,
\par 22 eftersom både Jøder kræve Tegn, og Grækere søge Visdom,
\par 23 vi derimod prædike Kristus som korsfæstet, for Jøder en Forargelse og for Hedninger en Dårskab,
\par 24 men for selve de kaldede både Jøder og Grækere, Kristus som Guds Kraft og Guds Visdom.
\par 25 Thi Guds Dårskab er visere end Menneskene, og Guds Svaghed er stærkere end Menneskene.
\par 26 Thi ser, Brødre! på eders Kaldelse, at I ere ikke mange vise efter Kødet, ikke mange mægtige, ikke mange fornemme;
\par 27 men det, som var Dårskab for Verden udvalgte Gud for at beskæmme de vise, og det, som var svagt for Verden, udvalgte Gud for at beskæmme det stærke;
\par 28 og det for Verden uædle og det ringeagtede, det, som intet var, udvalgte Gud for at gøre det, som var noget, til intet,
\par 29 for at intet Kød skal rose sig for Gud.
\par 30 Men ud af ham ere I i Kristus Jesus, som blev os Visdom fra Gud, både Retfærdighed og Helliggørelse og Forløsning;
\par 31 for at, som der er skrevet: "Den, som roser sig, rose sig af Herren!"

\chapter{2}

\par 1 Og jeg, Brødre! da jeg kom til eder, kom jeg ikke og forkyndte eder Gud Vidnesbyrd med Stormægtighed i Tale eller i Visdom;
\par 2 thi jeg agtede ikke at vide noget iblandt eder uden Jesus Kristus og ham korsfæstet;
\par 3 og jeg færdedes hos eder i Svaghed og i Frygt og megen Bæven,
\par 4 og min Tale og min Prædiken var ikke med Visdoms overtalende Ord, men med Ånds og Krafts Bevisning,
\par 5 for at eders Tro ikke skulde bero på Menneskers Visdom, men på Guds Kraft.
\par 6 Dog, Visdom tale vi iblandt de fuldkomne, men en Visdom, der ikke stammer fra denne Verden, ikke heller fra denne Verdens Herskere, som blive til intet;
\par 7 men vi tale Visdom fra Gud, den hemmelige, den, som var skjult, som Gud før Verdens Begyndelse forudbestemte til vor Herlighed,
\par 8 hvilken ingen af denne Verdens Herskere har erkendt; thi; havde de erkendt den,havde de ikke korsfæstet Herlighedens Herre;
\par 9 men, som der er skrevet: "Hvad intet Øje har set, og intet Øre har hørt, og ikke er opkommet i noget Menneskes Hjerte, hvad Gud har beredt dem, som elske ham."
\par 10 Men os åbenbarede Gud det ved Ånden; thi Ånden ransager alle Ting, også Guds Dybder.
\par 11 Thi hvilket Menneske ved, hvad der er i Mennesket, uden Menneskets Ånd, som er i ham? Således har heller ingen erkendt, hvad der er i Gud, uden Guds Ånd.
\par 12 Men vi have ikke fået Verdens Ånd, men Ånden fra Gud, for at vi kunne vide, hvad der er os skænket af Gud;
\par 13 og dette tale vi også, ikke med Ord, lærte af menneskelig Visdom, men med Ord, lærte af Ånden, idet vi tolke åndelige Ting med åndelige Ord.
\par 14 Men det sjælelige Menneske tager ikke imod de Ting, som høre Guds Ånd til; thi de ere ham en Dårskab, og han kan ikke erkende dem, thi de bedømmes åndeligt.
\par 15 Men den åndelige bedømmer alle Ting, selv derimod bedømmes han af ingen.
\par 16 Thi hvem har kendt Herrens Sind, så han skulde kunne undervise ham? Men vi have Kristi Sind.

\chapter{3}

\par 1 Og jeg, Brødre! kunde ikke tale til eder som til åndelige, men som til kødelige, som til spæde Børn i Kristus.
\par 2 Mælk gav jeg eder at drikke, ikke fast Føde; thi I kunde endnu ikke tåle det, ja, I kunne det ikke engang nu;
\par 3 thi endnu ere I kødelige. Når der nemlig er Nid og Splid iblandt eder, ere I da ikke kødelige og vandre på Menneskers Vis?
\par 4 Thi når en siger: "Jeg hører Paulus til," og en anden: "Jeg hører Apollos til," ere I så ikke "Mennesker"?
\par 5 Hvad er da Apollos? og hvad er Paulus? Tjenere, ved hvilke I bleve troende og det, efter som Herren gav enhver.
\par 6 Jeg plantede, Apollos vandede, men Gud gav Vækst.
\par 7 Så er da hverken den noget, som planter, ikke heller den, som vander, men Gud, som giver Vækst.
\par 8 Den, som planter, og den, som vander, ere eet; men hver skal få sin egen Løn efter sit eget Arbejde.
\par 9 Thi Guds Medarbejdere ere vi; Guds Ager, Guds Bygning ere I.
\par 10 Efter den Guds Nåde, som blev given mig, har jeg som en viis Bygmester lagt Grundvold, men en anden bygger derpå. Men enhver se til, hvorledes han bygger derpå!
\par 11 thi anden Grundvold kan ingen lægge end den, som er lagt, hvilken er Jesus Kristus.
\par 12 Men dersom nogen på Grundvolden bygger med Guld, Sølv, kostbare Sten, Træ, Hø, Strå,
\par 13 da skal enhvers Arbejde blive åbenbaret; thi Dagen skal gøre det klart, efterdi den åbenbares med Ild, og hvordan enhvers Arbejde er, det skal Ilden prøve.
\par 14 Dersom det Arbejde, som en har bygget derpå, består, da skal han få Løn;
\par 15 dersom ens Arbejde bliver opbrændt, da skal han gå Glip af den; men selv skal han blive frelst, dog som igennem Ild.
\par 16 Vide I ikke, at I ere Guds Tempel, og Guds Ånd bor i eder?
\par 17 Dersom nogen fordærver Guds Tempel, skal Gud fordærve ham; thi Guds Tempel er helligt, og det ere jo I.
\par 18 Ingen bedrage sig selv! Dersom nogen tykkes at være viis iblandt eder i denne Verden, han vorde en Dåre, for at han kan vorde viis.
\par 19 Thi denne Verdens Visdom er Dårskab for Gud; thi der er skrevet: "Han er den, som griber de vise i deres Træskhed;"
\par 20 og atter:"Herren kender de vises Tanker, at de ere forfængelige."
\par 21 Derfor rose ingen sig af Mennesker! Alle Ting ere jo eders,
\par 22 være sig Paulus eller Apollos eller Kefas eller Verden eller Liv eller Død eller det nærværende eller det tilkommende: alle Ting ere eders;
\par 23 men I ere Kristi, og Kristus er Guds.

\chapter{4}

\par 1 Således agte man os; som Kristi Tjenere og Husholdere over Guds Hemmeligheder!
\par 2 I øvrigt kræves her af Husholdere, at man må findes tro,
\par 3 Men mig er det såre lidet at bedømmes af eder eller af en menneskelig Ret; ja, jeg bedømmer end ikke mig selv.
\par 4 Thi vel ved jeg intet med mig selv, dog er jeg ikke dermed retfærdiggjort; men den, som bedømmer mig, er Herren.
\par 5 Derfor dømmer ikke noget før Tiden, førend Herren kommer, som både skal bringe for lyset det, som er skjult i Mørket, og åbenbare Hjerternes Råd; og da skal enhver få sin Ros fra Gud.
\par 6 Men dette, Brødre! har jeg anvendt på mig selv og Apollos for eders Skyld, for at I på os kunne lære dette "ikke ud over, hvad der står skrevet", for at ikke nogen af eder for eens Skyld skal opblæse sig mod en anden.
\par 7 Thi hvem giver dig Fortrin? og hvad har du, som du ikke har fået givet? men når du virkelig har fået det, hvorfor roser du dig da, som om du ikke havde fået det?
\par 8 I ere allerede mættede, I ere allerede blevne rige, I ere blevne Konger uden os, ja, gid I dog vare blevne Konger, for at også vi kunde være Konger med eder!
\par 9 Thi mig synes, at Gud har fremstillet os Apostle som de ringeste, ligesom dødsdømte; thi et Skuespilere vi blevne for Verden, både for Engle og Mennesker.
\par 10 Vi ere Dårer for Kristi Skyld, men I ere kloge i Kristus; vi svage, men I stærke; I hædrede, men vi vanærede.
\par 11 Indtil denne Time lide vi både Hunger og Tørst og Nøgenhed og få Næveslag og have intet blivende Sted
\par 12 og arbejde møjsommeligt med vore egne Hænder. Udskælder man os, velsigne vi; forfølger man os, finde vi os deri;
\par 13 spotter man os, give vi gode Ord; som Verdens Fejeskarn ere vi blevne, et Udskud for alle indtil nu.
\par 14 Ikke for at beskæmme eder skriver jeg dette; men jeg påminder eder som mine elskede Børn.
\par 15 Thi om I end have ti Tusinde Opdragere i Kristus, have I dog ikke mange Fædre; thi jeg har i Kristus Jesus avlet eder ved Evangeliet.
\par 16 Jeg formaner eder altså, vorder mine Efterfølgere!
\par 17 Derfor har jeg sendt Timotheus til eder, som er mit elskede og trofaste Barn i Herren, og han skal minde eder om mine Veje i Kristus, således som jeg lærer alle Vegne i enhver Menighed.
\par 18 Men nogle ere blevne opblæste, i den Tanke, at jeg ikke kommer til eder;
\par 19 men jeg skal snart komme til eder, om Herren vil, og gøre mig bekendt, ikke med de opblæstes Ord, men med deres Kraft.
\par 20 Thi Guds Rige består ikke i Ord, men i Kraft.
\par 21 Hvad ville I? Skal jeg komme til eder med Ris eller med Kærlighed og Sagtmodigheds Ånd?

\chapter{5}

\par 1 I det hele taget høres der om Utugt iblandt eder, og det sådan Utugt, som end ikke findes iblandt Hedningerne, at en lever med sin Faders Hustru.
\par 2 Og I ere opblæste og bleve ikke snarere bedrøvede, for at den, som har gjort denne Gerning, måtte udstødes af eders Midte!
\par 3 Thi jeg for min Del, fraværende med Legemet, men nærværende med Ånden, har allerede, som om jeg var nærværende, fældet den Dom over ham, som på sådan, Vis har bedrevet dette,
\par 4 at, når i vor Herres Jesu Navn I og min Ånd ere forsamlede, så med vor Herres Jesu Kraft
\par 5 at overgive den pågældende til Satan til Kødets Undergang, for af Ånden kan frelses på den Herres Jesu dag.
\par 6 Det er ikke noget smukt, I rose eder af! Vide I ikke, at en liden Surdejg syrer hele Dejgen?
\par 7 Udrenser den gamle Surdejg, for at I kunne være en ny Dejg, ligesom I jo ere usyrede;thi også vort Påskelam er slagtet, nemlig Kristus.
\par 8 Derfor, lader os holde Højtid, ikke med gammel Surdejg, ej heller med Sletheds og Ondskabs Surdejg, men med Renheds og Sandheds usyrede Brød.
\par 9 Jeg skrev eder til i mit Brev, at I ikke skulle have Samkvem med utugtige,
\par 10 ikke i al Almindelighed denne Verdens utugtige eller havesyge og Røvere eller Afgudsdyrkere; ellers måtte I jo gå ud af Verden.
\par 11 Men nu skrev jeg til eder, at I ikke skulle have Samkvem, om nogen, der har Navn af Broder, er en utugtig eller en havesyg eller en Afgudsdyrker eller en Skændegæst eller en Dranker eller en Røver, ja, end ikke spise sammen med en sådan.
\par 12 Thi hvad kommer det mig ved at dømme dem, som ere udenfor? Dømme I ikke dem, som ere indenfor?
\par 13 Men dem udenfor skal Gud dømme.Bortskaffer den onde fra eder selv!

\chapter{6}

\par 1 Kan nogen af eder, når han har Sag med en anden, føre det over sit Sind at søge Dom hos de uretfærdige, og ikke hos de hellige?
\par 2 Eller vide I ikke, at de hellige skulle dømme Verden? og når Verden dømmes ved eder, ere I da uværdige til at sidde til Doms i de ringeste Sager?
\par 3 Vide I ikke, at vi skulle dømme Engle? end sige da i timelige Ting!
\par 4 Når I da have Sager om timelige Ting, sætte I da dem til Dommere, som ere agtede for intet i Menigheden?
\par 5 Til Skam for eder siger jeg det: Er der da slet ingen viis iblandt eder, som kan dømme sine Brødre imellem?
\par 6 Men Broder fører Sag imod Broder, og det for vantro!
\par 7 Overhovedet er jo allerede det en Fejl hos eder, at I have Retssager med hverandre. Hvorfor lide I ikke hellere Uret? hvorfor lade I eder ikke hellere plyndre?
\par 8 Men I gøre Uret og plyndre, og det Brødre!
\par 9 Eller vide I ikke, at uretfærdige skulle ikke arve Guds Rige? Farer ikke vild! Hverken utugtige eller Afgudsdyrkere eller Horkarle eller de som lade sig bruge til unaturlig Utugt, eller de, som øve den,
\par 10 eller Tyve eller havesyge eller Drankere, ingen Skændegæster, ingen Røvere skulle arve Guds Rige.
\par 11 Og sådanne vare I for en Del; men I lode eder aftvætte, ja, I bleve helligede, ja, I bleve retfærdiggjorte ved den Herres Jesu Navn og ved vor Guds Ånd.
\par 12 Alt er mig tilladt, men ikke alt er gavnligt; alt er mig tilladt, men jeg skal ikke lade mig beherske af noget.
\par 13 Maden er for Bugen og Bugen for Maden; men Gud skal tilintetgøre både denne og hin. Legemet derimod er ikke for Utugt, men for Herren, og Herren for Legemet;
\par 14 og Gud har både oprejst Herren og skal oprejse os ved sin Kraft.
\par 15 Vide I ikke, at eders Legemer ere Kristi Lemmer? Skal jeg da tage Kristi Lemmer og gøre Skøgelemmer deraf? Det være langt fra!
\par 16 Eller vide I ikke, at den, som holder sig til Skøgen, er eet Legeme med hende?"Thi de to," hedder det,"skulle blive til eet Kød."
\par 17 Men den, som holder sig til Herren, er een Ånd med ham.
\par 18 Flyr Utugt! Enhver Synd, som et Menneske ellers gør, er uden for Legemet; men den, som bedriver Utugt, synder imod sit eget Legeme.
\par 19 Eller vide I ikke, at eders Legeme er et Tempel for den Helligånd, som er i eder, hvilken I have fra Gud, og at I ikke ere eders egne?
\par 20 Thi I bleve købte dyrt; ærer derfor Gud i eders Legeme!

\chapter{7}

\par 1 Men hvad det angår, hvorom I skreve til mig, da er det godt for en Mand ikke at røre en Kvinde;
\par 2 men for Utugts Skyld have hver Mand sin egen Hustru, og hver Kvinde have sin egen Mand.
\par 3 Manden yde Hustruen sin Skyldighed; ligeledes også Hustruen Manden.
\par 4 Hustruen råder ikke over sit eget Legeme, men Manden; ligeså råder heller ikke Manden over sit eget Legeme, men Hustruen.
\par 5 Unddrager eder ikke hinanden, uden måske med fælles Samtykke, til en Tid, for at I kunne have Ro til Bønnen, og for så atter at være sammen, for at Satan ikke skal friste eder, fordi I ikke formå at være afholdende.
\par 6 Men dette siger jeg som en Indrømmelse, ikke som en Befaling.
\par 7 Jeg ønsker dog, at alle Mennesker måtte være, som jeg selv er; men hver har sin egen Nådegave fra Gud, den ene så, den anden så.
\par 8 Til de ugifte og til Enkerne siger jeg, at det er godt for dem, om de forblive som jeg.
\par 9 Men kunne de ikke være afholdende, da lad dem gifte sig; thi det er bedre at gifte sig end at lide Brynde.
\par 10 Men de gifte byder ikke jeg, men Herren, at en Hustru ikke skal skille sig fra sin Mand;
\par 11 (men om hun virkeligt skiller sig fra ham, da forblive hun ugift eller forlige sig med Manden;) og at en Mand ikke skal forlade sin Hustru.
\par 12 Men til de andre siger jeg, ikke Herren: Dersom nogen Broder har en vantro Hustru, og denne samtykker i at bo hos ham, så forlade han hende ikke!
\par 13 Og dersom en Hustru har en vantro Mand, og denne samtykker i at bo hos hende, så forlade hun ikke Manden!
\par 14 Thi den vantro Mand er helliget ved Hustruen, og den vantro Hustru er helliget ved Manden; ellers vare jo eders Børn urene, men nu ere de hellige.
\par 15 Men skiller den vantro sig, så lad ham skille sig; ingen Broder eller Søster er trælbunden i sådanne Tilfælde; men Gud har kaldet os til Fred.
\par 16 Thi hvad ved du, Hustru! om du kan frelse din Mand? eller hvad ved du, Mand! om du kan frelse din Hustru?
\par 17 Kun vandre enhver således, som Herren har tildelt ham, som Gud har kaldet ham; og således forordner jeg i alle Menighederne.
\par 18 Blev nogen kaldet som omskåren, han lade ikke Forhud drage over; er nogen kaldet som uomskåren, han lade sig ikke omskære!
\par 19 Omskærelse har intet at sige, og Forhud har intet at sige, men det at holde Guds Bud.
\par 20 Hver blive i den Stand, hvori han blev kaldet!
\par 21 Blev du kaldet som Træl, da lad det ikke bekymre dig, men om du også kan blive fri, da gør hellere Brug deraf!
\par 22 Thi den, der er kaldet i Herren som Træl, er Herrens frigivne; ligeså er den, der er kaldet som fri, Kristi Træl.
\par 23 Dyrt bleve I købte, vorde ikke Menneskers Trælle!
\par 24 I den Stand, hvori enhver blev kaldet, Brødre, deri blive han for Gud!
\par 25 Men om Jomfruerne har jeg ikke nogen Befaling fra Herren, men giver min Mening til Hende som den, hvem Herren barmhjertigt har forundt at være troværdig.
\par 26 Jeg mener altså dette, at det på Grund af den forhåndenværende Nød er godt for et Menneske at være således, som han er.
\par 27 Er du bunden til en Kvinde, da søg ikke at blive løst; er du ikke bunden, da søg ikke en Hustru!
\par 28 Men om du også gifter dig, synder du ikke; og om en Jomfru gifter sig, synder hun ikke; dog ville sådanne få Trængsel i Kødet. Men jeg skåner eder.
\par 29 Men dette siger jeg eder, Brødre! at Tiden er kort, for at herefter både de, der have Hustruer, skulle være, som om de ingen have,
\par 30 og de, der græde, som om de ikke græde, og de, der glæde sig, som om de ikke glæde sig, og de, der købe, som om de ikke besidde,
\par 31 og de, der bruge denne Verden, som om de ikke gøre Brug af den; thi denne Verdens Skikkelse forgår.
\par 32 Men jeg ønsker, at I må være uden Bekymring. Den ugifte er bekymret for de Ting, som høre Herren til, hvorledes han kan behage Herren;
\par 33 men den gifte er bekymret for de Ting, som høre Verden til, hvorledes han kan behage Hustruen.
\par 34 Og der er også Forskel imellem Hustruen og Jomfruen. Den ugifte er bekymret for de Ting, som høre Herren til, for at hun kan være hellig både på Legeme og Ånd; men den gifte er bekymret for det, som hører Verden til, hvor ledes hun kan behage Manden.
\par 35 Men dette siger jeg til eders eget Gavn, ikke for at kaste en Snare om eder, men for at bevare Sømmelighed og en urokkelig Vedhængen ved Herren.
\par 36 Men dersom nogen mener at volde sin ugifte Datter Skam, om hun sidder over Tiden, og det må så være, han gøre, hvad han vil, han synder ikke; lad dem gifte sig!
\par 37 Men den, som står fast i sit Hjerte og ikke er tvungen, men har Rådighed over sin Villie og har besluttet dette i sit Hjerte at holde sin Datter ugift, han gør vel.
\par 38 Altså, både den, som bortgifter sin Datter, gør vel, og den, som ikke bortgifter hende, gør bedre.
\par 39 En Hustru er bunden, så længe hendes Mand lever; men dersom Manden sover hen, er hun fri til at gifte sig med hvem hun vil, kun at det sker i Herren.
\par 40 Men lykkeligere er hun, om hun forbliver således, som hun er, efter min Mening; men også jeg mener at have Guds Ånd.

\chapter{8}

\par 1 Men hvad Kødet fra Afgudsofrene angår, da vide vi, fordi vi alle have Kundskab (Kundskaben opblæser, men Kærligheden opbygger.
\par 2 Dersom nogen tykkes at kende noget, han kender endnu ikke således, som man bør kende.
\par 3 Men dersom nogen elsker Gud, han er kendt af ham.)
\par 4 Hvad altså Spisningen af Offerkødet angår, da vide vi, at der er ingen, Afgud i Verden, og at der ingen Gud er uden een.
\par 5 Thi om der end er såkaldte Guder, være sig i Himmelen eller på Jorden, som der jo er mange Guder og mange Herrer,
\par 6 så er der for os dog kun een Gud, Faderen, af hvem alle Ting ere, og vi til ham, og een Herre, Jesus Kristus, ved hvem alle Ting ere, og vi ved ham.
\par 7 Dog ikke alle have den Kundskab. Men der er nogle, som ifølge deres hidtidige Afgudsvane spise det som Afgudsofferkød, og deres Samvittighed, som er skrøbelig, besmittes.
\par 8 Men Mad skal ikke bestemme vor Stilling over for Gud; hverken have vi Fortrin, om vi spise, eller stå tilbage, om vi ikke spise.
\par 9 Men ser, til, at ikke denne eders Frihed skal blive til Anstød for de skrøbelige!
\par 10 Thi dersom nogen ser dig, som har Kundskab, sidde til Bords i et Afgudshus, vil så ikke Samvittigheden hos den, som er skrøbelig, blive opbygget til at spise Afgudsofferkødet?
\par 11 Den skrøbelige går jo til Grunde ved din Kundskab, Broderen, for hvis Skyld Kristus er død.
\par 12 Men når I således Synde imod Brødrene og såre deres skrøbelige Samvittighed, Synde I imod Kristus.
\par 13 Derfor, om Mad forarger min Broder, vil jeg aldrig i Evighed spise Kød, for at jeg ikke skal forarge min Broder.

\chapter{9}

\par 1 Er jeg ikke fri? er jeg ikke Apostel? har jeg ikke set Jesus, vor Herre? er I ikke min Gerning i Herren?
\par 2 Er jeg ikke Apostel for andre, så er jeg det dog i det mindste for eder; thi Seglet på min Apostelgerning ere I i Herren.
\par 3 Dette er mit Forsvar imod dem, som bedømme mig.
\par 4 Have vi ikke Ret til at spise og drikke?
\par 5 Have vi ikke Ret til at føre en Søster med om som Hustru, som også de andre Apostle og Herrens Brødre og Kefas?
\par 6 Eller have alene jeg og Barnabas ingen Ret til at lade være at arbejde?
\par 7 Hvem tjener vel nogen Sinde i Krig på egen Sold? Hvem planter en Vingård og spiser ikke dens Frugt? Eller hvem vogter en Hjord og nyder ikke af Hjordens Mælk?
\par 8 Taler jeg vel dette blot efter menneskelig Vis, eller siger ikke også Loven dette?
\par 9 Thi i Mose Lov er der skrevet: "Du må ikke binde Munden til på en Okse, som tærsker." Er det Okserne, Gud bekymrer sig om,
\par 10 eller siger han det ikke i hvert Tilfælde for vor Skyld? For vor Skyld blev det jo skrevet, fordi den, som pløjer, bør pløje i Håb, og den, som tærsker, bør gøre det i Håb om at få sin Del.
\par 11 Når vi have sået eder de åndelige Ting, er det da noget stort, om vi høste eders timelige?
\par 12 Dersom andre nyde sådan Ret over eder, kunde da vi ikke snarere? Dog have vi ikke brugt denne Ret; men vi tåle alt, for at vi ikke skulle lægge noget i Vejen for Kristi Evangelium.
\par 13 Vide I ikke, at de, som udføre de hellige Tjenester, få deres Føde fra Helligdommen, de, som tjene ved Alteret, dele med Alteret?
\par 14 Således har også Herren forordnet for dem, som forkynde Evangeliet, at de skulle leve af Evangeliet.
\par 15 Jeg derimod har ikke gjort Brug af noget af dette. Jeg skriver dog ikke dette, for at det skal ske således med mig; thi jeg vil hellere dø, end at nogen skulde gøre min Ros til intet.
\par 16 Thi om jeg forkynder Evangeliet, har jeg ikke noget at rose mig af; der påligger mig nemlig en Nødvendighed, thi ve mig, om jeg ikke forkynder det!
\par 17 Gør jeg nemlig dette af fri Villie, så får jeg Løn; men har jeg imod min Villie fået en Husholdning betroet,
\par 18 hvad er da min Løn? For at jeg, når jeg forkynder Evangeliet, skal fremsætte det for intet, så at jeg ikke gør Brug af min Ret i Evangeliet.
\par 19 Thi skønt jeg er fri over for alle, har jeg dog gjort mig selv til Tjener for alle, for at jeg kunde vinde des flere.
\par 20 Og jeg er bleven Jøderne som en Jøde, for at jeg kunde vinde Jøder; dem under Loven som en under Loven, skønt jeg ikke selv er under Loven, for at jeg kunde vinde dem, som ere under Loven;
\par 21 dem uden for Loven som en uden for Loven, skønt jeg ikke er uden Lov for God, men under Kristi Lov, for at jeg kunde vinde dem, som ere uden for Loven.
\par 22 Jeg er bleven skrøbelig for de skrøbelige, for at jeg kunde vinde de skrøbelige; jeg er bleven alt for alle, for at jeg i ethvert Fald kunde frelse nogle.
\par 23 Men alt gør jeg for Evangeliets Skyld, for at jeg kan blive meddelagtig deri.
\par 24 Vide I ikke, at de, som løbe på Banen, løbe vel alle, men ikkun een får Prisen? Således skulle I løbe, for at I kunne vinde den.
\par 25 Enhver, som deltager i Kamplegene, er afholdende i alt; hine nu vel for at få en forkrænkelig Krans, men vi en uforkrænkelig.
\par 26 Jeg løber derfor ikke som på det uvisse jeg fægter som en, der ikke slår i Luften;
\par 27 men jeg bekæmper mit Legeme og holder det i Trældom, for at ikke jeg, som har prædiket for andre, selv skal blive forkastet.

\chapter{10}

\par 1 Thi jeg vil ikke, Brødre, at I skulle være uvidende om, at vore Fædre vare alle under Skyen og gik alle igennem Havet
\par 2 og bleve alle døbte til Moses i Skyen og i Havet
\par 3 og spiste alle den samme åndelige Mad
\par 4 og drak alle den samme åndelige Drik; thi de drak af en åndelig Klippe, som fulgte med; men Klippen var Kristus.
\par 5 Alligevel fandt Gud ikke Behag i de fleste af dem; thi de bleve slagne ned i Ørkenen.
\par 6 Men disse Ting skete som Forbilleder for os, for at vi ikke skulle begære, hvad ondt er, således som hine begærede.
\par 7 Bliver ej heller Afgudsdyrkere som nogle af dem, ligesom der er skrevet: "Folket satte sig ned at spise og drikke, og de stode op at lege."
\par 8 Lader os ej heller bedrive Utugt, som nogle af dem bedreve Utugt, og der faldt på een Dag tre og tyve Tusinde.
\par 9 Lader os ej heller friste Herren, som nogle af dem fristede ham og bleve ødelagte af Slanger.
\par 10 Knurrer ej heller, som nogle af dem knurrede og bleve ødelagte af Ødelæggeren.
\par 11 Men dette skete dem forbilledligt, men det blev skrevet til Advarsel for os, til hvem Tidernes Ende er kommen.
\par 12 Derfor den, som tykkes at stå, se til, at han ikke falder!
\par 13 Der er ikke kommet andre end menneskelige Fristelser over eder, og trofast er Gud, som ikke vil tillade, at I fristes over Evne, men som sammen med Fristelsen vil skabe også Udgangen af den, for at I må kunne udholde den.
\par 14 Derfor, mine elskede, flyr fra Afgudsdyrkelsen!
\par 15 Jeg taler som til forstandige; dømmer selv, hvad jeg siger.
\par 16 Velsignelsens Kalk, som vi velsigne, er den ikke Samfund med Kristi Blod? det Brød, som vi bryde, er det ikke Samfund med Kristi Legeme?
\par 17 Fordi der er eet Brød, ere vi mange eet Legeme; thi vi få alle Del i det ene Brød.
\par 18 Ser til Israel efter Kødet; have de, som spise Ofrene, ikke Samfund med Alteret?
\par 19 Hvad siger jeg da? At Afgudsofferkød er noget? eller at en Afgud er noget?
\par 20 Nej! men hvad Hedningerne ofre, ofre de til onde Ånder og ikke til Gud; men jeg vil ikke,at I skulle få Samfund med de onde Ånder.
\par 21 I kunne ikke drikke Herrens Kalk og onde Ånders Kalk; I kunne ikke være delagtige i Herrens Bord og i onde Ånders Bord.
\par 22 Eller skulle vi vække Herrens Nidkærhed? Mon vi ere stærkere end han?
\par 23 Alt er tilladt, men ikke alt er gavnligt; alt er tilladt, men ikke alt opbygger.
\par 24 Ingen søge sit eget, men Næstens!
\par 25 Alt, hvad der sælges i Slagterbod, spiser det, uden at undersøge noget af Samvittigheds-Hensyn;
\par 26 thi Herrens er Jorden og dens Fylde.
\par 27 Dersom nogen af de vantro indbyder eder, og I ville gå derhen, da spiser alt det, som sættes for eder, uden at undersøge noget af Samvittigheds-Hensyn.
\par 28 Men dersom nogen siger til eder: "Dette er Offerkød," da lad være at spise for hans Skyld, som gav det til Kende, og for Samvittighedens Skyld.
\par 29 Samvittigheden siger jeg, ikke ens egen, men den andens; thi hvorfor skal min Frihed dømmes af en anden Samvittighed?
\par 30 Dersom jeg nyder det med Taksigelse, hvorfor hører jeg da ilde for det, som jeg takker for?
\par 31 Hvad enten I derfor spise eller drikke, eller hvad I gøre, da gører alt til Guds Ære!
\par 32 Værer uden Anstød både for Jøder og Grækere og for Guds Menighed,
\par 33 ligesom også jeg i alt stræber at tækkes alle, idet jeg ikke søger, hvad der gavner mig selv, men hvad der gavner de mange, for at de kunne frelses.

\chapter{11}

\par 1 Vorder mine Efterfølgere, ligesom også jeg er Kristi!
\par 2 Men jeg roser eder, fordi I komme mig i Hu i alt og holde fast ved Overleveringerne, således som jeg har overleveret eder dem.
\par 3 Men jeg vil, at I skulle vide, at Kristus er enhver Mands Hoved; men Manden er Kvindens Hoved; men Gud er Kristi Hoved.
\par 4 Hver Mand, som beder eller profeterer med tildækket Hoved, beskæmmer sit Hoved.
\par 5 Men hver Kvinde, som beder eller profeterer med utildækket Hoved, beskæmmer sit Hoved; thi det er lige det samme, som var hun raget.
\par 6 Thi når en Kvinde ikke tildækker sig, så lad hende også klippe sit Hår af; men er det usømmeligt for en Kvinde at klippes eller rages, da tildække hun sig!
\par 7 Thi en Mand bør ikke tildække sit Hoved, efterdi han er Guds Billede og Ære; men Kvinden er Mandens Ære.
\par 8 Mand er jo ikke af Kvinde, men Kvinde af Mand.
\par 9 Ej heller er jo Mand skabt for Kvindens Skyld, men Kvinde for Mandens Skyld.
\par 10 Derfor bør Kvinden have et Ærbødighedstegn på Hovedet for Englenes Skyld.
\par 11 Dog er hverken Kvinde uden Mand eller Mand uden Kvinde i Herren.
\par 12 Thi ligesom Kvinden er af Manden, således er også Manden ved Kvinden; men alt sammen er det af Gud.
\par 13 Dømmer selv: Er det sømmeligt, at en Kvinde beder til Gud med utildækket Hoved?
\par 14 Lærer ikke også selve Naturen eder, at når en Mand bærer langt Hår, er det ham en Vanære,
\par 15 men når en Kvinde bærer langt Hår, er det hende en Ære; thi det lange Hår er givet hende som et Slør.
\par 16 Men har nogen Lyst til at trættes herom, da have vi ikke sådan Skik, og Guds Menigheder ej heller.
\par 17 Men idet jeg giver følgende Formaning, roser jeg ikke, at I komme sammen, ikke til det bedre, men til det værre.
\par 18 For det første nemlig hører jeg, at når I komme sammen i Menighedsforsamling, er der Splittelser iblandt eder; og for en Del tror jeg det.
\par 19 Thi der må endog være Partier iblandt eder, for at de prøvede kunne blive åbenbare iblandt eder.
\par 20 Når I da komme sammen, er dette ikke at æde en Herrens Nadver.
\par 21 Thi under Spisningen tager enhver sit eget Måltid forud, og den ene hungrer, den anden beruser sig.
\par 22 Have I da ikke Huse til at spise og drikke i? eller foragte I Guds Menighed og beskæmme dem, som intet have? Hvad skal jeg sige eder? Skal jeg rose eder? I dette roser jeg eder ikke.
\par 23 Thi jeg har modtaget fra Herren, hvad jeg også har overleveret eder: At den Herre Jesus i den Nat, da han blev forrådt, tog Brød,
\par 24 takkede og brød det og sagde: "Dette er mit Legeme, som er for eder; gører dette til min Ihukommelse!"
\par 25 Ligeså tog han og,så Kalken efter Aftensmåltidet og sagde: "Denne Kalk er den nye Pagt i mit Blod; gører dette, så ofte som I drikke det, til min Ihukommelse!"
\par 26 Thi så ofte, som I æde dette Brød og drikke Kalken, forkynde I Herrens Død, indtil han kommer.
\par 27 Derfor, den, som æder Brødet eller drikker Herrens Kalk uværdigt, pådrager sig Skyld over for Herrens Legeme og Blod.
\par 28 Men hvert Menneske prøve sig selv, og således æde han af Brødet og drikke af Kalken!
\par 29 Thi den, som æder og drikker, æder og drikker sig selv en Dom til, når han ikke agter på Legenet.
\par 30 Derfor ere mange skrøbelige og sygelige iblandt eder, og en Del sover hen.
\par 31 Men dersom vi bedømte os selv, bleve vi ikke dømte.
\par 32 Men når vi dømmes, tugtes vi af Herren, for at vi ikke skulle fordømmes med Verden.
\par 33 Derfor, mine Brødre! når I komme sammen til Måltid, da venter på hverandre!
\par 34 Når nogen hungrer, han spise hjemme, for at I ikke skulle komme sammen til Dom. Men det øvrige skal jeg forordne, når jeg kommer.

\chapter{12}

\par 1 Men hvad de åndelige Gaver angår, Brødre! vil jeg ikke, at I skulle være uvidende.
\par 2 I vide, at da I vare Hedninger, droges I hen til de stumme Afguder, som man drog eder.
\par 3 Derfor kundgør jeg eder, at ingen, som taler ved Guds Ånd, siger: "Jesus er en Forbandelse," og ingen kan sige: "Jesus er Herre" uden ved den Helligånd.
\par 4 Der er Forskel på Nådegaver, men det er den samme Ånd;
\par 5 og der er Forskel på Tjenester, og det er den samme Herre;
\par 6 og der er Forskel på kraftige Gerninger, men det er den samme Gud, som virker alt i alle.
\par 7 Men til enhver gives Åndens Åbenbarelse til det, som er gavnligt.
\par 8 En gives der nemlig ved Ånden Visdoms Tale; en anden Kundskabs Tale ifølge den samme Ånd;
\par 9 en anden Tro i den samme Ånd; en anden Gaver til at helbrede i den ene Ånd;
\par 10 en anden at udføre kraftige Gerninger; en anden profetisk Gave; en anden at bedømme Ånder; en anden forskellige Slags Tungetale; en anden Udlægning af Tungetale.
\par 11 Men alt dette virker den ene og samme Ånd, som uddeler til enhver især; efter som han vil.
\par 12 Thi ligesom Legemet er eet og har mange Lemmer, men alle Legemets Lemmer, skønt de ere mange, dog ere eet Legeme, således også Kristus.
\par 13 Thi med een Ånd bleve vi jo alle døbte til at være eet Legeme, hvad enten vi ere Jøder eller Grækere, Trælle eller frie; og alle fik vi een Ånd at drikke
\par 14 Legemet er jo heller ikke eet Lem, men mange.
\par 15 Dersom Foden vilde sige: "Fordi jeg ikke er Hånd, hører jeg ikke til Legemet," så ophører den dog ikke derfor at høre til Legemet.
\par 16 Og dersom Øret vilde sige: "Fordi jeg ikke er Øje, hører jeg ikke til Legemet," så ophører det dog ikke derfor at høre til Legemet.
\par 17 Dersom hele Legemet var Øje, hvor blev da Hørelsen? Dersom det helt var Hørelse, hvor blev da Lugten?
\par 18 Men nu har Gud sat Lemmerne, ethvert af dem, på Legemet, efter som han vilde.
\par 19 Men dersom de alle vare eet Lem, hvor blev da Legemet?
\par 20 Nu er der derimod mange Lemmer og dog kun eet Legeme.
\par 21 Øjet kan ikke sige til Hånden: "Jeg har dig ikke nødig," eller atter Hovedet til Fødderne: "Jeg har eder ikke nødig."
\par 22 Nej, langt snarere ere de Lemmer på Legemet nødvendige, som synes at være de svageste,
\par 23 og de, som synes os mindre ærefolde på Legemet, dem klæde vi med des mere Ære; og de Lemmer, vi blues ved, omgives med desto større Blufærdighed;
\par 24 de derimod, som vi ikke blues ved, have det ikke nødig. Men Gud har sammenføjet Legemet således, at han tillagde det ringere mere Ære;
\par 25 for at der ikke skal være Splid i Legemet, men, for at Lemmerne skulle have samme Omsorg for hverandre;
\par 26 og hvad enten eet Lem lider, lide alle Lemmerne med, eller eet Lem bliver hædret, glæde alle Lemmerne sig med.
\par 27 Men I ere Kristi Legeme, og Lemmer enhver især.
\par 28 Og nogle satte Gud i Menigheden for det første til Apostle, for det andet til Profeter, for det tredje til Lærere, dernæst kraftige Gerninger, dernæst Gaver til at helbrede. til at hjælpe, til at styre, og forskellige Slags Tungetale.
\par 29 Mon alle ere Apostle? mon alle ere Profeter? mon alle ere Lærere? mon alle gøre kraftige Gerninger?
\par 30 mon alle have Gaver til at helbrede? mon alle tale i Tunger? mon alle udlægge?
\par 31 Men tragter efter de største Nådegaver! Og yder mere viser jeg eder en ypperlig Vej.

\chapter{13}

\par 1 Taler jeg med Menneskers og Engles Tunger, men ikke har Kærlighed, da er jeg bleven et lydende Malm eller en klingende Bjælde.
\par 2 Og har jeg profetisk Gave og kender alle Hemmelighederne og al Kundskaben, og har jeg al Troen, så at jeg kan flytte Bjerge, men ikke har Kærlighed, da er jeg intet.
\par 3 Og uddeler jeg alt, hvad jeg ejer, til de fattige og giver mit Legeme hen til at brændes, men ikke har Kærlighed, da gavner det mig intet.
\par 4 Kærligheden er langmodig, er velvillig; Kærligheden bærer ikke Nid; Kærligheden praler ikke, opblæses ikke,
\par 5 gør intet usømmeligt, søger ikke sit eget, forbitres ikke, tilregner ikke det onde;
\par 6 glæder sig ikke over Uretfærdigheden, men glæder sig ved Sandheden;
\par 7 den tåler alt, tror alt, håber alt, udholder alt.
\par 8 Kærligheden bortfalder aldrig; men enten det er profetiske Gaver, de skulle forgå, eller Tungetale, den skal ophøre, eller Kundskab, den skal forgå;
\par 9 thi vi kende stykkevis og profetere stykkevis;
\par 10 men når det fuldkomne kommer, da skal det stykkevise forgå.
\par 11 Da jeg var Barn, talte jeg som et Barn, tænkte jeg som et Barn, dømte jeg som et Barn; efter at jeg er bleven Mand, har jeg aflagt det barnagtige.
\par 12 Nu se vi jo i et Spejl, i en Gåde, men da skulle vi se Ansigt til Ansigt; nu kender jeg stykkevis, men da skal jeg erkende, ligesom jeg jo blev erkendt.
\par 13 Så blive da Tro, Håb, Kærlighed disse tre; men størst iblandt disse er Kærligheden.

\chapter{14}

\par 1 Higer efter Kærligheden, og tragter efter de åndelige Gaver men mest efter at profetere.
\par 2 Thi den; som taler i Tunger, taler ikke for Mennesker, men for Gud; thi ingen forstår det, men han taler Hemmeligheder i Ånden.
\par 3 Men den, som profeterer, taler Mennesker til Opbyggelse og Formaning og Trøst.
\par 4 Den, som taler i Tunger, opbygger sig selv; men den, som profeterer, opbygger en Menighed.
\par 5 Men jeg ønsker, at I alle måtte tale i Tunger, men endnu hellere, at I måtte profetere; den, som profeterer, er større end den, som taler i Tunger, med mindre han udlægger det, for at Menigheden kan få Opbyggelse deraf.
\par 6 Men nu, Brødre! dersom jeg kommer til eder og taler i Tunger, hvad vil jeg da gavne eder, hvis jeg ikke taler til eder enten ved Åbenbaring eller ved Kundskab, enten ved Profeti eller ved Lære?
\par 7 Selv de livløse Ting, som give Lyd, være sig en Fløjte eller en Harpe, når de ikke gøre Skel imellem Tonerne, hvorledes skal man så kunne forstå, hvad der spilles på Fløjten eller Harpen?
\par 8 Ja, også når en Basun giver en utydelig Lyd, hvem vil da berede sig til Krig?
\par 9 Således også med eder: dersom I ikke ved Tungen fremføre tydelig Tale, hvorledes skal man da kunne forstå det, som tales? I ville jo tale hen i Vejret.
\par 10 Der er i Verden, lad os sige, så og så mange Slags Sprog, og der er intet af dem, som ikke har sin Betydning.
\par 11 Dersom jeg nu ikke kender Sprogets Betydning, bliver jeg en Barbar for den, som taler, og den, som taler, bliver en Barbar for mig.
\par 12 Således også med eder: når I tragte efter åndelige Gaver, da lad det være til Menighedens Opbyggelse, at I søge at blive rige derpå
\par 13 " Derfor, den, som taler i Tunger, han bede om, at han må kunne udlægge det.
\par 14 Thi dersom: jeg taler i Tunger og beder, da beder. vel min Ånd, men min Forstand er uden Frugt.
\par 15 Hvad da? Jeg vil bede med Ånden, men jeg vil også bede med Forstanden; jeg vil lovsynge med Ånden, men jeg vil også lovsynge med Forstanden.
\par 16 Ellers, når du priser Gud i Ånden, hvorledes vil da den, som indtager den uindviedes Plads, kunne sige sit Amen til din Taksigelse, efterdi han ikke ved, hvad du siger?
\par 17 Thi vel er din Taksigelse smuk, men den anden opbygges ikke.
\par 18 Jeg takker Gud for, at jeg mere end I alle taler i Tunger.
\par 19 Men i en Menighed vil jeg hellere tale fem Ord med min Forstand, for at jeg også kan undervise andre, end ti Tusinde Ord i Tunger.
\par 20 Brødre! vorder ikke Børn i Forstand, men værer Børn i Ondskab, i Forstand derimod vorder fuldvoksne!
\par 21 Der er skrevet i Loven: "Ved Folk med fremmede Tungemål og ved fremmedes Læber vil jeg tale til dette Folk, og de skulle end ikke således høre mig, siger Herren."
\par 22 Således er Tungetalen til et Tegn, ikke for dem, som tro, men for de vantro; men den profetiske Gave er det ikke for de vantro, men for dem, som tro.
\par 23 Når altså den hele Menighed kommer sammen, og alle tale i Tunger, men der kommer uindviede eller vantro ind, ville de da ikke sige, at I rase?
\par 24 Men dersom alle profetere, og der kommer nogen vantro eller uindviet ind, da overbevises han af alle, han bedømmes af alle,
\par 25 hans Hjertes skjulte Tanker åbenbares, og så vil han falde på sit Ansigt og tilbede Gud og forkynde, at Gud er virkelig i eder.
\par 26 Hvad da Brødre? Når I komme sammen, da har enhver en Lovsang, en Lære, en Åbenbaring, en Tungetale, en Udlægning; alt ske til Opbyggelse!
\par 27 Dersom nogen taler i Tunger, da være det to, eller i det højeste tre hver Gang, og den ene efter den anden, og een udlægge det!
\par 28 Men dersom der ingen Udlægger er til Stede, da tie hin i Menigheden, men han tale for sig selv og for Gud!
\par 29 Men af Profeter tale to eller tre, og de andre bedømme det;
\par 30 men dersom en anden, som sidder der, får en Åbenbarelse, da tie den første!
\par 31 Thi I kunne alle profetere, den ene efter den anden, for at alle kunne lære, og alle blive formanede,
\par 32 og Profeters Ånder ere Profeter undergivne.
\par 33 Thi Gud er ikke Forvirringens, men Fredens Gud. Ligesom i alle de helliges Menigheder
\par 34 skulle eders Kvinder tie i Forsamlingerne; thi det tilstedes dem ikke at tale, men lad dem underordne sig, ligesom også Loven siger.
\par 35 Men ville de lære noget, da adspørge de deres egne Mænd hjemme; thi det er usømmeligt for en Kvinde at tale i en Menighedsforsamling.
\par 36 Eller er det fra eder, at Guds Ord er udgået? eller er det til eder alene, at det er kommet?
\par 37 Dersom nogen tykkes, at han er en Profet eller åndelig, han erkende, at hvad jeg skriver til eder, er Herrens Bud.
\par 38 Men er nogen uvidende derom, så får han være uvidende!
\par 39 Altså, mine Brødre! tragter efter at profetere og forhindrer ikke Talen i Tunger!
\par 40 Men alt ske sømmeligt og med Orden!

\chapter{15}

\par 1 Men jeg kundgør eder, Brødre, det Evangelium, som jeg Forkyndte eder, hvilket I også modtoge, i hvilket I også stå,
\par 2 ved hvilket I også frelses, hvis I fastholde, med hvilket Ord jeg forkyndte eder det - ellers troede I forgæves.
\par 3 Jeg overleverede eder nemlig som noget af det første, hvad jeg også har modtaget: at Kristus døde for vore Synder,efter Skrifterne;
\par 4 og at han blev begravet; og at han er bleven oprejst den tredje Dag, efter Skrifterne;
\par 5 og at han blev set af Kefas, derefter af de tolv;
\par 6 derefter blev han set af over fem Hundrede Brødre på een Gang, af hvilke de fleste endnu ere i Live, men nogle ere hensovede;.
\par 7 derefter blev han set af Jakob, dernæst af alle Apostlene;
\par 8 men sidst af alle blev han set også af mig som det ufuldbårne Foster;
\par 9 thi jeg er den ringeste af Apostlene, jeg, som ikke er værd at kaldes Apostel, fordi jeg har forfulgt Guds Menighed.
\par 10 Men af Guds Nåde er jeg det, jeg er, og hans Nåde imod mig har ikke været forgæves; men jeg har arbejdet mere end de alle, dog ikke jeg, men Guds Nåde, som er med mig.
\par 11 Hvad enten det da er mig eller de andre, således prædike vi, og således troede I.
\par 12 Men når der prædikes, at Kristus er oprejst fra de døde, hvorledes sige da nogle iblandt eder, at der ikke er dødes Opstandelse?
\par 13 Dersom der ikke er dødes Opstandelse, da er ikke heller Kristus oprejst.
\par 14 Men er Kristus ikke oprejst, da er vor Prædiken jo tom, og eders Tro også tom.
\par 15 Men vi blive da også fundne som falske Vidner om Gud, fordi vi have vidnet imod Gud, at han oprejste Kristus, hvem han ikke har oprejst, såfremt døde virkelig ikke oprejses.
\par 16 Thi dersom døde ikke oprejses, da er Kristus ikke heller oprejst.
\par 17 Men dersom Kristus ikke er oprejst, da er eders Tro forgæves; så ere I endnu i eders Synder;
\par 18 da gik altså også de, som ere hensovede i Kristus, fortabt.
\par 19 Have vi alene i dette Liv sat vort Håb til Kristus, da ere vi de ynkværdigste af alle Mennesker.
\par 20 Men nu er Kristus oprejst fra de døde, som Førstegrøde af de hensovede.
\par 21 Thi efterdi Død kom ved et Menneske, er også dødes Opstandelse kommen ved et Menneske.
\par 22 Thi ligesom alle dø i Adam, således skulle også alle levendegøres i Kristus.
\par 23 Dog hver i sit Hold: som Førstegrøde Kristus, dernæst de, som tilhøre Kristus, ved hans Tilkommelse.
\par 24 Derpå kommer Enden, når han overgiver Gud og Faderen Riget, når han har tilintetgjort hver Magt og hver Myndighed og Kraft.
\par 25 Thi han bør være Konge, indtil han får lagt alle Fjenderne under sine Fødder.
\par 26 Den sidste Fjende, som tilintetgøres, er Døden.
\par 27 Han har jo "lagt alle Ting under hans Fødder." Men når han" siger: "Alt er underlagt" - åbenbart med Undtagelse af den, som underlagde ham alt -
\par 28 når da alle Ting ere blevne ham underlagte, da skal også Sønnen selv underlægge sig ham, som har underlagt ham alle Ting, for at Gud kan være alt i alle.
\par 29 Hvad ville ellers de udrette, som lade sig døbe for de døde? Dersom døde overhovedet ikke oprejses, hvorfor lade de sig da døbe for dem?
\par 30 Hvorfor udsætte da også vi os hver Time for Fare?
\par 31 Jeg dør daglig, så sandt jeg har eder, Brødre, at rose mig af i Kristus Jesus, vor Herre.
\par 32 Hvis jeg som et almindeligt Menneske har kæmpet med vilde Dyr i Efesus, hvad Gavn har jeg så deraf? Dersom døde ikke oprejses, da "lader os spise og drikke, thi i Morgen dø vi."
\par 33 Farer ikke vild; slet Omgang fordærver gode Sæder!
\par 34 Vorder ædrue, som det bør sig, og synder ikke; thi nogle kende ikke Gud; til Skam for eder siger jeg det.
\par 35 Men man vil sige: "Hvorledes oprejses de døde? hvad Slags Legeme komme de med?"
\par 36 Du Dåre! det, som du sår, bliver ikke levendegjort, dersom det ikke dør.
\par 37 Og hvad du end sår, da sår du ikke det Legeme, der skal vorde, men et nøgent Korn, være sig af Hvede eller af anden Art.
\par 38 Men Gud giver det et Legeme, således som han har villet, og hver Sædart sit eget Legeme.
\par 39 Ikke alt Kød er det samme Kød, men eet er Menneskers, et andet Kvægs Kød, et andet Fugles Kød, et andet Fisks.
\par 40 Og der er himmelske Legemer og jordiske Legemer; men een er de himmelskes Herlighed, en anden de jordiskes.
\par 41 Een er Solens Glans og en anden Månens Glans og en anden Stjernernes Glans; thi den ene Stjerne er forskellig fra den anden i Glans.
\par 42 Således er det også med de dødes Opstandelse: det såes i Forkrænkelighed, det oprejses i Uforkrænkelighed;
\par 43 det såes i Vanære, det oprejses i Herlighed; det såes i Skrøbelighed, det oprejses i Kraft;
\par 44 der såes et sjæleligt Legeme, der oprejses et åndeligt Legeme.
\par 45 Således er der også skrevet: "Det første Menneske, Adam, blev til en levende Sjæl;"den sidste Adam blev til en levendegørende Ånd.
\par 46 Men det åndelige er ikke det første, men det sjælelige; derefter det åndelige.
\par 47 Det første Menneske var af Jord, jordisk; det andet Menneske er fra Himmelen.
\par 48 Sådan som den jordiske var, sådanne ere også de jordiske; og sådan som den himmelske er, sådanne ere også de himmelske.
\par 49 Og ligesom vi have båret den jordiskes Billede, således skulle vi også bære den himmelskes Billede!
\par 50 Men dette siger jeg, Brødre! at Kød og Blod kan ikke arve Guds Rige, ej heller arver Forkrænkeligheden Uforkrænkeligheden.
\par 51 Se, jeg siger eder en Hemmelighed: Alle skulle vi ikke hensove, men vi skulle alle forvandles
\par 52 i et Nu, i et Øjeblik, ved den sidste Basun; thi Basunen skal lyde, og de døde skulle oprejses uforkrænkelige, og vi skulle forvandles.
\par 53 Thi dette forkrænkelige må iføre sig Uforkrænkelighed, og dette dødelige iføre sig Udødelighed.
\par 54 Men når dette forkrænkelige har iført sig Uforkrænkelighed, og dette dødelige har iført sig Udødelighed, da skal det Ord opfyldes, som er skrevet: "Døden er opslugt til Sejr."
\par 55 "Død, hvor er din Sejr? Død, hvor er din Brod?"
\par 56 Men Dødens Brod er Synden, og Syndens Kraft er Loven.
\par 57 Men Gud ske Tak, som giver os Sejren ved vor Herre Jesus Kristus!
\par 58 Derfor, mine elskede Brødre! bliver faste, urokkelige, altid rige i Herrens Gerning, vidende, at eders Arbejde er ikke forgæves i Herren.

\chapter{16}

\par 1 Men hvad Indsamlingen til de hellige angår, da gører også I, ligesom jeg forordnede for Menighederne i Galatien!
\par 2 Hver første Dag i Ugen lægge enhver af eder hjemme hos sig selv noget til Side og samle, hvad han måtte have Lykke til, for at der ikke først skal ske Indsamlinger, når jeg kommer.
\par 3 Men når jeg kommer, vil jeg sende, hvem I måtte finde skikkede dertil, med Breve for at bringe eders Gave til Jerusalem.
\par 4 Men dersom det er værd, at også jeg rejser med, da kunne de rejse med mig.
\par 5 Men jeg vil komme til eder, når jeg er dragen igennem Makedonien; thi jeg drager igennem Makedonien;
\par 6 men hos eder vil jeg måske blive eller endog overvintre, for at I kunne befordre mig videre, hvor jeg så rejser hen.
\par 7 Thi nu vil jeg ikke se eder på Gennemrejse; jeg håber nemlig at forblive nogen Tid hos eder, om Herren vil tilstede det.
\par 8 Men i Efesus vil jeg forblive indtil Pinsen;
\par 9 thi en Dør står mig åben, stor og virksom, og der er mange Modstandere.
\par 10 Men om Timotheus kommer, da ser til, at han kan færdes hos eder uden Frygt; thi han gør Herrens Gerning, såvel som jeg.
\par 11 Derfor må ingen ringeagte ham; befordrer ham videre i Fred, for at han kan komme til mig; thi jeg venter ham med Brødrene.
\par 12 Men hvad Broderen Apollos angår, da har jeg meget opfordret ham til at komme til eder med Brødrene; men det var i hvert Fald ikke hans Villie at komme nu, men han vil komme, når han får belejlig Tid.
\par 13 Våger, står faste i Troen, værer mandige, værer stærke!
\par 14 Alt ske hos eder i Kærlighed!
\par 15 Men jeg formaner eder, Brødre - I kende Stefanas's Hus, at det er Akajas Førstegrøde, og de have hengivet sig selv til at tjene de hellige -
\par 16 til at også I skulle underordne eder under sådanne og enhver, som arbejder med og har Besvær.
\par 17 Men jeg glæder mig ved Stefanas's og Fortunatus's og Akaikus's Nærværelse, fordi disse have udfyldt Savnet af eder;
\par 18 thi de have vederkvæget min Ånd og eders. Skønner derfor på sådanne!
\par 19 Menighederne i Asien hilse eder. Akvila og Priska hilse eder meget i Herren tillige med Menigheden i deres Hus.
\par 20 Alle Brødrene hilse eder. Hilser hverandre med et helligt Kys!
\par 21 Hilsenen med min, Paulus's egen Hånd.
\par 22 Dersom nogen ikke elsker Herren, han være en Forbandelse! Maran Atha.
\par 23 Den Herres Jesu Nåde være med eder!
\par 24 Min Kærlighed med eder alle i Kristus Jesus!



\end{document}