\begin{document}

\title{Nehemiah}


\chapter{1}

\par 1 Nehemias's, Hakaljas Søns, Beretning. I Kislev Måned i det tyvende År, medens jeg var i Borgen i Susan,
\par 2 kom Hanani, en af mine Brødre, sammen med nogle Mænd fra Juda. Og da jeg udspurgte dem om Jøderne, den Rest, der var undsluppet fra Fangenskabet, og om Jerusalem,
\par 3 sagde de til mig: De tiloversblevne, de, som er tilbage fra Fangenskabet der i Landet, lever i stor Nød og Forsmædelse, og Jerusalems Mur er nedrevet og Portene opbrændt.
\par 4 Da jeg hørte denne Tidende, satte jeg mig hen og græd og sørgede i flere Dage, og jeg fastede og bad for Himmelens Guds Åsyn,
\par 5 idet jeg sagde: Ak, HERRE, Himmelens Gud, du store, frygtelige Gud, som tager Vare på Pagten og Miskundheden mod dem, der elsker dig og holder dine Bud!
\par 6 Lad dog dit Øre være lydhørt og dit Øje åbent, så du hører din Tjeners Bøn, som jeg nu beder for dit Åsyn både Nat og Dag for dine Tjenere tsraeliterne, idet jeg bekender deres Synder, som vi også jeg og min Faders Hus har begået imod dig.
\par 7 Såre ilde har vi handlet imod dig, og vi har ikke holdt de Bud, Anordninger og Lovbud, du pålagde din Tjener Moses.
\par 8 Kom det Ord i Hu, som du pålagde din Tjener Moses: Dersom I er troløse, vil jeg adsplitte eder blandt Folkene;
\par 9 men hvis I omvender eder til mig og holder mine Bud og handler efter dem, så vil jeg, om eders fordrevne end er ved Himmelens Ende, dog samle dem derfra og bringe dem til det Sted, jeg udvalgte til Bolig for mit Navn.
\par 10 De er jo dine Tjenere og dit Folk, som du udløste ved din store Kraft og din stærke Hånd.
\par 11 Ak, Herre, lad dit Øre være lydhørt for din Tjeners og dine Tjeneres Bøn, vi, som gerne vil frygte dit Navn, og lad det i Dag lykkes for din Tjener og lad ham finde Nåde for denne Mands Åsyn! Jeg var nemlig Mundskænk hos Kongen.

\chapter{2}

\par 1 I Nisan Måned i Kong Artaxerxes's tyvende Regeringsår, da jeg skulde sørge for Vin, har jeg engang Vinen frem og rakte Kongen den. Jeg havde ikke før set modfalden ud, når jeg stod for hans Ansigt.
\par 2 Og Kongen sagde til mig: Hvorfor ser du så modfalden ud? Du er jo ikke syg; det kan ikke være andet, end at du har en Hjertesorg! Da blev jeg såre bange,
\par 3 og jeg sagde til Kongen: "Kongen leve evindelig! Hvor kan jeg andet end se modfalden ud, når den By, hvor mine Fædres Grave er, ligger øde, og dens Porte er fortæret af Ilden?
\par 4 Kongen spurgte mig da: "Hvad er det, du ønsker? Så bad jeg til Himmelens Gud,"
\par 5 og jeg sagde til Kongen: "Hvis Kongen synes, og hvis din Tjener er dig til Behag, beder jeg om, at du vil lade mig rejse til Juda, til den By, hvor mine Fædres Grave er, og lade mig bygge den op igen!
\par 6 Da sagde Kongen til mig, medens Dronningen sad ved hans Side: "Hvor længe vil den Rejse vare, og hvornår kan du vende tilbage? Og da Kongen således fandt for godt at lade mig rejse, opgav jeg ham en Tid.
\par 7 Og jeg sagde til Kongen: Hvis Kongen synes, så lad mig få Breve med til Statholderne hinsides Floden, så de lader mig drage videre, til jeg når Juda,
\par 8 og et Brev til Asaf, Opsynsmanden.over den kongelige Skov, at han skal give mig Træ til Bjælkeværket i Tempelborgens Porte og til Byens Mur og det Hus, jeg tager ind i! Det gav Kongen mig, eftersom min Guds gode Hånd var over mig.
\par 9 Da jeg kom til Statholderne hinsides Floden, gav jeg dem Kongens Breve; desuden havde Kongen givet mig Hærførere og Ryttere med på Rejsen.
\par 10 Men da Horon'ten Sanballat og den ammonitiske Træl Tobija hørte det, ærgrede de sig højligen over, at der var kommet en Mand for at arbejde på Israeliternes Bedste.
\par 11 Så kom jeg til Jerusalem, og da jeg havde været der i tre Dage,
\par 12 brød jeg op ved Nattetide sammen med nogle få Mænd uden at have sagt noget Menneske, hvad min Gud havde skudt mig i Sinde at gøre for Jerusalem; og der var intet andet Dyr med end det, jeg red på.
\par 13 Jeg red så om Natten ud gennem Dalporten i Retning af Dragekilden og hen til Møgporten, idet jeg undersøgte Jerusalems Mure, der var nedrevet, og Portene, der var fortæret af Ilden;
\par 14 derpå red jeg videre til Kildeporten og Kongedammen, men der var ikke Plads nok til, at mit Ridedyr kunde komme frem med mig.
\par 15 Så red jeg om Natten op igennem Dalen og undersøgte Muren, forandrede Retning og red så ind igennem Dalporten, hvorpå jeg vendte hjem.
\par 16 Forstanderne vidste ikke, hvor jeg var gået hen, eller hvad jeg foretog mig; og hverken Jøderne, Præsterne, Stormændene, Forstanderne eller de andre, der skulde have med Arbejdet at gøre, havde jeg endnu sagt noget.
\par 17 Men nu sagde jeg til dem: I ser den Ulykke, vi er i, hvorledes Jerusalem er ødelagt og Portene opbrændt; kom derforog lad os opbygge Jerusalems Mur, så vi ikke mere skal være til Spot!
\par 18 Og da jeg fortalte dem, hvorledes min Guds gode Hånd havde været over mig, og om de Ord, Kongen havde talt til mig, sagde de: Lad os gøre os rede og bygge! Og de tog sig sammen til det gode Værk.
\par 19 Da Horoniten Sanballat, den ammonitiske Træl Tobija og Araberen Gesjem hørte det, spottede de os og sagde hånligt til os: Hvad er det, I der har for? Sætter I eder op mod Kongen?
\par 20 Men jeg gav dem til Svar: "Himmelens Gud vil lade det lykkes for os, og vi, hans Tjenere, vil gøre os rede og bygge; men I har ingen Del eller Ret eller Ihukommelse i Jerusalem!"

\chapter{3}

\par 1 Ypperstepræsten Eljasjib og hans Brødre Præsterne tog fat på at bygge Fåreporten; de forsynede den med Bjælkeværk og indsatte Portfløjene; derefter byggede de videre hen til Meatårnet og helligede det og igen videre hen til Hanan'eltårnet.
\par 2 Ved Siden af ham byggede Mændene fra Jeriko, ved Siden af dem Zakkur, Imris Søn.
\par 3 Fiskeporten byggede Sena'as Efterkommere; de forsynede den med Bjælkeværk og indsatte Portfløje, Kramper og Portslåer.
\par 4 Ved Siden af dem arbejdede Meremot, en Søn af Hakkoz's Søn Uruja, med at istandsætte Muren. Ved Siden af ham arbejdede Mesjullam, en Søn af Berekja, en Søn af Mesjezab'el, ved Siden af ham Zadok, Ba'anas Søn.
\par 5 Ved Siden af ham arbejdede Folkene fra Tekoa, men de store iblandt dem bøjede ikke deres Nakke under deres Herres Arbejde.
\par 6 Jesjanaporten istandsatte Jojada, Paseas Søn, og Mesjullam, Besodejas Søn; de forsynede den med Bjælkeværk og indsatte Portfløje, Kramper og Portslåer.
\par 7 Ved Siden af dem arbejdede Gibeoniten Melatja og Meronotiten Jadon sammen med Mændene fra Gibeon og Mizpa, der stod under Statholderen hinsides Floden.
\par 8 Ved Siden af ham arbejdede Uzziel, en Søn af Harhaja, en af Guldsmedene.
\par 9 Ved Siden af dem arbejdede Øversten over den ene Halvdel af Jerusalems Område, Refaja, Hurs Søn.
\par 10 Ved Siden af ham arbejdede Jedaja, Harumafs Søn, ud for sit eget Hus. Ved Siden af ham arbejdede Hattusj, Hasjabnejas Søn.
\par 11 En anden Strækningistandsatte Malkija, Harims Søn, og Hassjub, Pahat-Moabs Søn, hen til Ovntårnet.
\par 12 Ved Siden af ham arbejdede Øversten over den anden Halvdel af Jerusalems Område, Sjallum, Hallohesj's Søn, sammen med sine Døtre.
\par 13 Dalporten istandsatte Hanun og Folkene fra Zanoa; de byggede den og indsatte Portfløje, Kramper og Portslåer, og desuden 1000 Alen af Muren hen til Møgporten.
\par 14 Møgporten istandsatte Øversten over Bet-Kerems Område, Malkija, Rekabs Søn; han byggede den og indsatte Portfløje, Kramper og Portslåer.
\par 15 Kildeporten istandsatte Øversten over Mizpas Område, Sjallun, Kol-Hozes Søn; han byggede den, forsynede den med Tag og indsatte Portfløje, Kramper og Portslåer; han byggede også Muren ved Dammen, fra hvilken Vandledningen fører til Kongens Have, og hen til Trinene, der fører ned fra Davidsbyen.
\par 16 Efter ham arbejdede Øversten over den ene Halvdel af Bet-Zurs Område, Nehemja, Azbuks Søn, hen til Pladsen ud for Davids Grave, til den udgravede Dam og til Kærnetroppernes Hus.
\par 17 Efter ham arbejdede Leviterne, ført af Rehum, Banis Søn. Ved Siden af ham arbejdede Øversten over den ene Halvdel af Ke'ilas Område, Hasjabja.
\par 18 Efter ham arbejdede deres Bysbørn, ført af Binnuj, Henadads Søn, Øversten over den anden Halvdel af Ke'ilas Område.
\par 19 Ved Siden af ham istandsatte Øversten over Mizpa, Ezer, Jesuas Søn, en anden Strækning lige ud for Opgangen til Tøjhuset i Hro'en'
\par 20 Efter ham istandsatte Baruk, Zabbajs Søn, op ad Bjerget en Strækning fra Krogen til Indgangen til Ypperstepræsten El-jasjibs Hus.
\par 21 Efter ham istandsatte Meremot, en Søn af Hakkoz's Søn Urija, en Strækning fra Indgangen til Eljasjibs Hus til Gavlen.
\par 22 Efter ham arbejdede Præsterne, Mændene fra Omegnen.
\par 23 Efter dem arbejdede Binjamin og Hassjub ud for deres Huse. Efter ham arbejdede Azarja, en Søn af Ma'aseja, en Søn af Ananja, ved Siden af sit Hus.
\par 24 Efter ham istandsatte Binnuj, Henadads Søn, en Strækning fra Azarjas Hus til Krogen og til Hjørnet.
\par 25 Efter ham arbejdede Palal, Uzajs Søn, lige ud for Krogen og det Tårn, der springer frem fra det øvre kongelige Hus ved Fængselsgården. Efter ham arbejdede Pedaja, Par'osj's Søn,
\par 26 til Stedet ud for Vandporten mod Øst og det fremspringende Tårn.
\par 27 Efter ham istandsatte Folkene fra Tekoa en Strækning fra Stedet ud for det store, fremspringende Tårn til Ofels Mur. (På Ofel boede Tempeltrællene).
\par 28 Over for Hesteporten arbejdede Præsterne, hver lige ud for sit Hus.
\par 29 Efter dem arbejdede Zadok, Immers Søn, ud for sit Hus. Efter ham arbejdede Østportens Vogter Sjemaja, Sjekanjas Søn.
\par 30 Efter ham istandsatte Hananja, Sjelemjas Søn, og Hanun, Zalafs sjette Søn, en Strækning. Efter ham arbejdede Mesjullam, Berekjas Søn, ud for sit Kammer.
\par 31 Efter ham arbejdede Malkija, en af Guldsmedene, hen til Tempeltrællenes Hus. Og Kræmmeme istandsatte Stykket ud for Mifkadporten og hen til Tagbygningen ved Hjørnet.
\par 32 Og mellem Tagbygningen ved Hjørnet og Fåreporten arbejdede Guldsmedene og Kræmmerne.

\chapter{4}

\par 1 Men da Sanballat hørte, at vi byggede på Muren, blev han vred og harmfuld; og han spottede Jøderne
\par 2 og sagde i Påhør af sine Brødre og Samarias Krigsfolk: "Hvad er det, disse usle Jøder har for? Vilo de overlade Gud det? Vil de ofre? Kan de gøre det færdigt endnu i Dag? Kan de kalde Stenene i disse Grusdynger til Live, når de er forbrændt?"
\par 3 Og Ammoniten Tobija, der stod ved siden af ham, sagde: "Lad dem bygge, så meget de vil; en Ræv kan rive deres Stenmur ned, blot den springer derop!"
\par 4 Hør, vor Gud, hvorledes vi er blevet hånet! Lad deres Spot falde tilbage på deres eget Hoved, og lad dem blive hånet som Fanger i et Fremmed Land!
\par 5 Skjul ikke deres Brøde og lad ikke deres Synd blive udslettet for dit Åsyn, thi med deres Ord krænkede de dem, der byggede!
\par 6 Men vi byggede på Muren, og hele Muren blev bygget færdig i halv Højde, og Folket arbejdede med god Vilje.
\par 7 Da nu Sanballat og Tobija og Araberne, Ammoniterne og Asdoditerne børte, at det skred fremad med Istandsættelsen af Jerusalems Mure, og at Hullerne i Muren begyndte at lukkes, blev de meget vrede,
\par 8 og de sammensvor sig alle om at drage hen og angribe Jerusalem og fremkalde Forvirring der.
\par 9 Da bad vi til vor Gud og satte Vagt både Dag og Naf for at værne os imod dem.
\par 10 Men Jøderne sagde: Lastdragernes Kræfter svigter, og Grusdyngerne er for store; vi kan ikke bygge på Muren!
\par 11 Og vore Fjender sagde: De må ikke mærke noget, før vi står midt iblandt dem og hugger dem ned og således får Arbejdet til at gå i Stå!
\par 12 Da nu de Jøder, der boede dem nærmest, Gang på Gang kom og lod os vide, at de rykkede op imod os fra alle Steder, hvor de boede,
\par 13 og da Folkene kun turde stille sig op på Steder, der lå lavere end Pladsen bag Muren, i Kælderrum, så opstillede jeg Folket Slægt for Slægt, væbnet med Sværd, Spyd og Buer;
\par 14 og da jeg så det, trådte jeg frem og sagde til Stormændene og Forstanderne og det øvrige Folk: Frygt ikke for dem! Kom den store, frygtelige Gud i Hu og kæmp for eders Brødre, Sønner og Døtre, eders Hustruer og Huse!
\par 15 Men da vore Fjender hørte, at vi havde fået det at vide, og at Gud gjorde deres Råd til intet, vendte vi alle tilbage til Muren, hver til sit Arbejde.
\par 16 Men fra den Tid af arbejdede kun den ene Halvdel af mine Folk, medens den anden Halvdel stod væbnet med Spyd, Skjolde, Buer og Brynjer; og Øversterne stod bag ved alle de Jøder,
\par 17 der byggede på Muren. Også Lastdragerne var væbnet; med den ene Hånd arbejdede de, og med den anden.holdt de Spydet;
\par 18 og de, der byggede, havde under Arbejdet hver sit Sværd bundet til Lænden.
\par 19 og jeg sagde til de store og Forstanderne og det øvrige Folk: "Arbejdet er stort og omfattende, og vi er spredt på Muren langt fra hverandre;
\par 20 hvor l nu hører Hornet gjalde, skal l samles om os; vor Gud vil stride for os!
\par 21 Således udførte vi Arbejdet, idet Halvdelen af os holdt Spydene rede fra Morgengry til Stjernernes Opgang.
\par 22 Samtidig sagde jeg også til Folket: Enhver skal sammen med sin Dreng overnatte i Jerusalem, for at vi kan have dem til Vagt om Natten og til Arbejde om Dagen!"
\par 23 Og hverken jeg eller mine Brødre eller mine Folk eller Vagten, der fulgte mig, afførte os vore Klæder, og enhver, der blev sendt efter Vand, havde Spydet med.

\chapter{5}

\par 1 Der lød nu højrøstede Klager fra Folket og deres Kvinder mod deres Brødre, Jøderne.
\par 2 Nogle sagde: "Vore Sønner og Døtre må vi give i Pant for at få Horn til Livets Ophold!"
\par 3 Andre sagde: "Vore Marker, Vingårde og Huse må vi give i Pant for at få Korn under Hungersnøden!"
\par 4 Atter andre sagde: "Vi har måttet låne på vore Marker og Vingårde for at kunne udrede de kongelige Skatter!
\par 5 Og vore Legemer er dog lige så gode som vore Brødres og vore Sønner lige så gode som deres; men vi er nødt til at give vore Sønner og Døtre hen til at blive Trælle, ja, nogle af vore Døtre er det allerede, og det stod ikke i vor Magt at hindre det, effersom vore Marker og Vingårde tilhører andre!"
\par 6 Da jeg hørte deres Klager og disse deres Ord, blussede Vreden heftigt op i mig;
\par 7 og efter at have tænkt over Sagen gik jeg i Rette med de store og Forstanderne og sagde til dem: I driver jo Åger over for eders Næste! Så kaldte jeg en stor Folkeforsamling sammen imod dem
\par 8 og sagde til dem: Så vidt vi var i Stand dertil, har vi frikøbt vore jødiske Brødre, der måtte sælge sig til Hedningerne; og I sælger eders Brødre, så de må sælge sig til os! Da tav de og fandt intet at svare.
\par 9 Men jeg fortsatte: Det er ikke ret af eder at handle således! Skulde I ikke vandre i Frygt for vor Gud af Hensyn til Hedningerne, vore Fjenders Spot?
\par 10 Også jeg og mine Brødre og mine Folk har lånt dem Penge og Korn; men lad os nu eftergive dem, hvad de skylder!
\par 11 Giv dem straks dere Marker, Vingårde, Oliventræer og Huse tilbage og eftergiv dem Pengene, Kornet, Moslen og Olien, som I har lånt dem!
\par 12 Da svarede de: Ja, vi vil give det tilbage og ikke afkræve dem noget; som du siger, vil vi gøre! Jeg lod da Præsterne kalde og lod dem sværge på, at de vilde handle således.
\par 13 Og jeg rystede min Brystfold og sagde: Enhver, der ikke holder dette Ord, vil Gud således ryste ud af hans Hus og Ejendom; ja, således skal han blive udrystet og tømt! Da sagde hele Forsamlingen: Amen! Og de lovpriste HERREN; og Folket handlede efter sit Løfte.
\par 14 Desuden skal det nævnes, at fra den Dag Kong Artaxerxes bød mig være deres Statholder i Judas Land, fra Kong Artaxerxes's tyvende til hans to og tredivte Regeriogsår, hele tolv År, spiste hverken jeg eller mine Brødre det Brød, der tilkom Statholderen,
\par 15 medens mine Forgængere, de tidligere Statholdere, lagde Tynge på Folket og for Brød og Vin daglig afkrævede dem fyrretyve Sekel Sølv, ligesom også deres Tjenere optrådte som Folkets Herrer. Det undlod jeg at gøre af Frygt for Gud.
\par 16 Og desuden tog jeg selv fat ved Arbejdet på denne Mur, skønt vi ingen Mark havde købt, og alle mine Folk var samlet der ved Arbejdet.
\par 17 Og Jøderne, både Forstanderne, 150 Mand, og de, der kom til os fra de omboende Hedningefolk, spiste ved mit Bord;
\par 18 og hvad der daglig lavedes til, et Stykke Hornkvæg, seks udsøgte Får og Fjerkræ, afholdt jeg Udgifferne til; dertil kom hver tiende Dag en Masse Vin af alle Sorter. Men alligevel krævede jeg ikke det Brød, der tilkom Statholderen, fordi Arbejdet tyngede hårdt på Folket.
\par 19 Kom i Hu alt, hvad jeg har gjort for dette Folk, og regn mig det til gode, min Gud!

\chapter{6}

\par 1 Da det nu kom Sanballat, Araberen Gesjem og vore andre Fjender for Øre, at jeg havde bygget Muren, og at der ikke var flere Huller i den - dog havde jeg på den Tid ikke sat Fløje i Portene -
\par 2 sendte Sanballat og Gesjem Bud og opfordrede mig til en Sammenkomst i Kefirim i Onodalen. Men de havde ondt i Sinde imod mig.
\par 3 Derfor sendte jeg Bud til dem og lod sige: "Jeg har et stort Arbejde for og kan derfor ikke komme derned; hvorfor skulde Arbejdet standse? Og det vilde ske, hvis jeg lod det ligge for at komme ned til eder.
\par 4 Fire Gange sendte de mig samme Bud, og hver Gang gav jeg dem samme Svar.
\par 5 Da sendte Sanballat for femte Gang sin Tjener til mig med samme Bud, og han havde et åbent Brev med,
\par 6 i hvilket der stod: Det hedder sig blandt Folkene, og Gasjmut bekræfter det, at du og Jøderne pønser på Oprør; derfor er det, du bygger Muren, og at du vil være deres Konge.
\par 7 Og du skal endog have fået Profeter til i Jerusalem at udråbe dig til Konge i Juda. Dette Rygte vil nu komme Kongen for Øre; kom derfor og lad os tales ved!
\par 8 Men jeg sendte ham det Bud: Slige Ting, som du omtaler, er slet ikke sket; det er dit eget Påfund!
\par 9 Thi de havde alle til Hensigt at indjage os Skræk, idet de tænkte, at vi skulde lade Hænderne synke, så Arbejdet ikke blev til noget. Men styrk du nu mine Hænder!
\par 10 Og da jeg gik ind i Sjemajas, Mehetab'els Søn Delajas Søns, Hus, som ved den Tid måtte holde sig inde, sagde han: Lad os tales ved i Guds Hus, i Helligdommens Indre, og stænge Dørene, thi der kommer nogle Folk, som vil dræbe dig; de kommer i Nat for at dræbe dig!"
\par 11 Men jeg svarede: Skulde en Mand som jeg flygte? Og hvorledes skulde en Mand som jeg kunne betræde Helligdommen og blive i Live? Jeg går ikke derind!
\par 12 Thi jeg skønnede, at det ikke var Gud, som havde sendt ham, men at han var kommet med det Udsagn om mig, fordi Tobija og Sanballat havde lejet ham dertil,
\par 13 for at jeg skulde blive bange og forsynde mig ved slig Adfærd og de få Anledning til ilde Omtale, så de kunde bagvaske mig.
\par 14 Kom Tobija og Sanballat i Hu, min Gud, efter deres Gerninger, ligeledes Profetinden Noadja og de andre Profeter, der vilde gøre mig bange!
\par 15 Således blev Muren færdig den fem og tyvende Dag i Elul Måned efter to og halvtredsindstyve Dages Forløb.
\par 16 Og da alle'vore Fjender hørte det, blev alle Hedningerne rundt om os bange og såre nedslåede, idet de skønnede, at dette Værk var udført med vor Guds Hjælp.
\par 17 Men der gik også i de Dage en Mængde Breve frem og tilbage mellem Tobija og de store i Juda;
\par 18 thi mange i Juda stod i Edsforbund med ham, da han var Svigersøn af Sjekanja, Aras Søn, og hans Søn Johanan var gift med en Datter af Mesjullam, Berekjas Søn.
\par 19 Også plejede de både at tale godt om ham til mig og at forebringe ham mine Ord; Tobija sendte også Breve for at gøre mig bange.

\chapter{7}

\par 1 Da Muren var bygget, lod jeg Portfløjene indsætte, og Dørvogterne, Sangerne og Leviterne blev ansat.
\par 2 Overbefalingen over Jerusalem gav jeg min Broder Hanani og Borgøversten Hananja; thi han var en pålidelig Mand og frygtede Gud som få;
\par 3 og jeg sagde til dem: "Jerusalems Porte må ikke åbnes, før Solen står højt på Himmelen; og medens den endnu står der, skal man lukke og stænge Portene og sætte Jerusalems Indbyggere på Vagt, hver på sin Post, hver ud for sit Hus!"
\par 4 Men Byen var udstrakt og stor og dens Indbygere få, og Husene var endnu ikke opbygget.
\par 5 Da skød min Gud mig i Sinde at samle de store, Forstanderne og Folket for at indføre dem i Slægtsfortegnelser. Og da fandt jeg Slægtebogen over dem, der først var draget op, og i den fandt jeg skrevet:
\par 6 Følgende er de Folk fra vor Landsdel, der drog op fra Land flygtigheden og Fangenskabet. Kong Nebukadnezar af Babel havde ført dem bort, men nu vendte de til bage til Jerusalem og Juda, hver til sin By;
\par 7 de kom sammen med Zerubbabel, Jesua, Nehemja, Azarja, Ra'amja, Nahamani, Mordokaj, Bilsjan, Misperet, Bigvaj, Nehum og Ba'ana. Tallet på Mændene i Israels Folk var:
\par 8 Par'osj's Efterkommere 2172,
\par 9 Sjefatjas Efterkommere 372,
\par 10 Aras Efterkommere 652,
\par 11 Pahat-Moabs Efterkommere, Je suas og Joabs Efterkommere, 2818,
\par 12 Elams Efterkommere 1254,
\par 13 Zattus Efterkommere 845,
\par 14 Zakkajs Efterkommere 760,
\par 15 Binnujs Efterkommere 648,
\par 16 Bebajs Efterkommere 628,
\par 17 Azgads Efterkommere 2322,
\par 18 Adonikams Efterkommere 667,
\par 19 Bigvajs Efterkommere 2067,
\par 20 Adins Efterkommere 655,
\par 21 Aters Efterkommere gennem Hizkija 98,
\par 22 Hasjums Efterkommere 328,
\par 23 Bezajs Efterkommere 324,
\par 24 Harifs Efterkommere 112,
\par 25 Gibeons Efterkommere 95,
\par 26 Mændene fra Betlehem og Netofa 188,
\par 27 Mændene fra Anatot 128,
\par 28 Mændene fra Bet-Azmavet 42,
\par 29 Mændene fra Hirjat-Jearim, Kefra og Be'erot 743;
\par 30 Mændene fra Rama og Geba 621,
\par 31 Mændene fra Mikmas 122,
\par 32 Mændene fra Betel og Aj 123,
\par 33 Mændene fra det andet Nebo 52,
\par 34 det andet Elams Efterkommere 1254,
\par 35 Harims Efterkommere 320,
\par 36 Jerikos Efterkommere 345,
\par 37 Lods, Hadids og Onos Efterkommere 721,
\par 38 Sena'as Efterkommere 3930.
\par 39 Præsterne var: Jedajas Efterkommere af Jesuas Hus 973,
\par 40 Immers Efterkommere 1052,
\par 41 Pasjhurs Efterkommere 1247,
\par 42 Harims Efterkommere 1017.
\par 43 Leviterne var: Jesuas og Kadmiels Efterkommere af Hodavjas Efterkommere 74.
\par 44 Tempelsangerne var: Asafs Efterkommere 148.
\par 45 Dørvogterne var: Sjallums, Aters, Talmons, Akkubs, Hatitas og Sjobajs Efterkommere 138.
\par 46 Tempeltrællene var: Zihas, Hasufas, Tabbaots,
\par 47 Keros's, Si'as, Padons,
\par 48 Lebanas, Hagabas, Salmajs,
\par 49 Hanans, Giddels, Gahars,
\par 50 Reajas, Rezins, Nekodas,
\par 51 Gazzams, Uzzas, Paseas,
\par 52 Besajs, Me'uniternes, Nefusiternes,
\par 53 Bakbuks, Hakufas, Harhurs,
\par 54 Bazluts, Mehidas, Harsjas,
\par 55 Barkos's, Siseras, Temas,
\par 56 Nezias og Hatifas Efterkommere.
\par 57 Efterkommerne af Salomos Trælle var: Sotaj s, Soferets, Peridas,
\par 58 Ja'alas, Darkons, Giddels,
\par 59 Sjefatjas, Hattils, Pokeret-Haz-zebajims og Amons Efterkommere.
\par 60 Alle Tempeltrælle og Efferkommerne af Salomos Trælle var tilsammen 392.
\par 61 Følgende, som drog op fra Tel-Mela, Tel-Harsja, Kerub-Addon og Immer, kunde ikke opgive, hvorvidt deres Fædrenehuse og Slægt hørte til Israeliterne:
\par 62 Delajas, Tobijas og Nekodas Efterkommere 642.
\par 63 Og følgende Præster: Habajas, Hakoz's og Barzillajs Efterkommere; denne sidste havde ægtet en af Gileaditen Barzillajs Døtre og var blevet opkaldt efter dem.
\par 64 De ledte efter deres Slægtebøger, men kunde ikke finde dem; derfor blev de som urene udelukket fra Præstestanden.
\par 65 Statholderen forbød dem at spise af det højhellige, indtil der fremstod en Præst med Urim og Tummim.
\par 66 Hele Menigheden udgjorde 42360
\par 67 foruden deres Trælle og Trælkvinder, som udgjorde 7337, hvor til kom 245 Sangere og Sangerinder. Deres Heste udgjorde 736, deres Mulddyr 245,
\par 68 deres Kameler 435 og deres Æsler 6720.
\par 69 En Del af Fædrenehusenes Overhoveder ydede Tilskud til Byggearbejdet.
\par 70 Af Fædrenehusenes Overhoveder gav nogle til Byggesummen 20.000 Drakmer Guld og 2.200 Miner Sølv.
\par 71 Og hvad det øvrige Folk gav, løb op til 20.000 Drakmer Guld, 2.000 Miner Sølv og 67 Præstekjortler.
\par 72 Derpå bosatte Præsterne, Leviterne og en Del af Folket sig i Jerusalem og dets Område, men Sangerne, Dørvogterne og hele det øvrige Israel i deres Byer.
\par 73 Da den syvende Måned indtraf - Israeliterne boede nu i deres Byer -

\chapter{8}

\par 1 samledes hele Folket fuldtalligt på den åbne Plads foran Vandporten og bad Ezra den Skriftlærde om at hente Bogen med Mose Lov, som HERREN havde pålagt Israel.
\par 2 Så hentede Præsten Ezra Loven og fremlagde den for forsamlingen, Mænd, Kvinder og alle, der havde Forstand til at høre; det var den første Dag i den syvende Måned;
\par 3 og vendt mod den åbne Plads foran Vandporten læste han den op fra Daggry til Middag for Mændene, Kvinderne og dem, der kunde fatte del, og alt Folket lyttede til Lovbogens Ord.
\par 4 Ezra den Skriftlærde stod på en Forhøjning af Træ, som var lavet i den Anledning, og ved Siden af ham stod til højre Mattitja, Sjema, Anaja, Urija, Hilkija og Ma'aseja, til venstre Pedaja, Misjael, Malkija, Hasjum, Hasjbaddana, Zekarja og Mesjullam.
\par 5 Og Ezra åbnede Bogen i hele Folkets Påsyn, thi han stod højere end Folket, og da han åbnede den, rejste hele Folket sig op.
\par 6 Derpå lovpriste Ezra HERREN, den store Gud, og hele Folket svarede med oprakte Hænder: Amen, Amen! Og de bøjede sig og kastede sig med Ansigtet til Jorden for HERREN.
\par 7 Og Leviterne Jesua, Bani, Sjerebja, Jamin, Akkub, Sjabbetaj, Hodija, Ma'aseja, Kelita, Azarja, Jozabad, Hanan og Pelaja udlagde Loven for Folket, medens Folket blev på sin Plads,
\par 8 og de oplæste Stykke for Stykke af Bogen med Guds Lov og udlagde det, så man kunde fatte det.
\par 9 Derpå sagde Nehemias, det er Statholderen, og Præsten Ezra den Skriftlærde og Leviterne, der underviste Folket, til alt Folket: Denne Dag er helliget HERREN eders Gud! Sørg ikke og græd ikke! Thi hele Folket græd, da de hørte Lovens Ord.
\par 10 Og han sagde til dem: Gå hen og spis fede Spiser og drik søde brikke og send noget til dem, der intet har, thi denne Dag er helliget vor Herre; vær ikke nedslåede, thi HERRENs Glæde er eders Styrke!
\par 11 Og Leviterne tyssede på alt Folket og sagde: Vær stille, thi denne Dag er hellig, vær ikke nedslåede!
\par 12 Da gik alt Folket hen og spiste og drak og sendte Gaver, og de fejrede en stor Glædesfest; thi de fattede Ordene, man havde kundgjort dem.
\par 13 Næste Dag samledes Overhovederne for alt Folkets Fædrenehuse og Præsterne og Leviterne hos Ezra den Skriftlærde for at mærke sig Lovens Ord,
\par 14 og i Loven som HERREN havde påbudt ved Moses, fandt de skrevet, at Israeliterne på Højtiden i den syvende Måned skulde bo i Løvhytter,
\par 15 og at man i alle deres Byer og i Jerusalem skulde udråbe og kundgøre følgende Budskab: Gå ud i Bjergene og hent Grene af ædle og vilde Oliventræer, Myrter, Palmer og andre Løvtræer for at bygge Løvhytter som foreskrevet!
\par 16 Så gik Folket ud og hentede det og byggede sig Løvhytter på deres Tage og i deres Forgårde og i Guds Hus's Forgårde og på de åbne Pladser ved Vandporten og Efraimsporfen.
\par 17 Og hele Forsamlingen, der var vendt tilbage fra Fangenskabet, byggede Løvbytter og boede i dem. Thi fra Josuas, Nuns Søns, Dage og lige til den Dag havde Israeliterne ikke gjort det, og der herskede såre stor Glæde.
\par 18 Og han læste op af Bogen med Guds Lov Dag for Dag fra den første til den sidste; og de fejrede Højtiden i syv Dage, og på den ottende holdtes der festlig Samling på foreskreven Måde.

\chapter{9}

\par 1 Men på den fire og tyvende Dag i samme Måned samledes Israeliterne under Faste og i Sørgedragt med Jord på Hovedet;
\par 2 og de, der var af Israels Slægt, skilte sig ud fra alle fremmede og trådte frem og bekendte deres Synder og deres Fædres Misgerninger.
\par 3 Så rejste de sig på deres Plads, og der blev læst op af Bogen med HERREN deres Guds Lov i en Fjerdedel af Dagen, og i en anden Fjerdedel bekendte de deres Synder og tilbad HERREN deres Gud.
\par 4 Og Jesua, Bani, Kadmiel, Sjebanja, Bunni, Sjerebja, Bani og Kenani trådte op på Leviternes Forhøjning og råbte med høj Røst til HERREN deres Gud;
\par 5 og Leviterne Jesua, Kadmiel, Bani, Hasjabneja, Sjerebja, Hodija, Sjebanja og Petaja sagde: "Stå op og lov HERREN eders Gud fra Evighed til Evighed!" Da lovede de hans herlige Navn, som er ophøjet over al Lov og Pris.
\par 6 Derpå sagde Ezra: "Du, HERRE, er den eneste; du har skabt Himmelen, Himlenes Himle med al deres Hær, Jorden med alt, hvad der er på den, Havene med alt, hvad der er i dem; du giver dem alle Liv, og Himmelens Hær tilbeder dig.
\par 7 Du er Gud HERREN, der udvalgte Abram og førte ham bort fra Ur-Kasdim og gav ham Navnet Abraham;
\par 8 og du fandt hans Hjerte fast i Troen for dit Åsyn og sluttede Pagt med ham om at give hans Afkom Kana'anæernes, Hetiternes, Amoriternes, Perizziternes, Jebusiternes og Girgasjiternes Land; og du holdt dit Ord; thi du er retfærdig.
\par 9 Og da du så vore Fædres Nød i Ægypten og hørte deres Råb ved det røde Hav,
\par 10 udførte du Tegn og Undere på Farao og alle hans Tjenere og alt Folket i hans Land, fordi du vidste, at de havde handlet overmodigt med dem. Og således skabte du dig et Navn, som du har den Dag i Dag.
\par 11 Du kløvede Havet for deres Øjne, så de vandrede midt igennem Havet på tør Bund, og dem, der forfulgte dem, styrtede du i Dybet som Sten i vældige Vande.
\par 12 I en Skystøtte førte du dem om Dagen og i en Ildstøtte om Natten, så den lyste for dem på Vejen, de skulde vandre.
\par 13 Du steg ned på Sinaj Bjerg, du talede med dem fra Himmelen og gav dem retfærdige Lovbud og tilforladelige Love, gode Anordninger og Bud.
\par 14 Du kundgjorde dem din hellige Sabbat og pålagde dem Bud, Anordninger og Love ved din Tjener Moses.
\par 15 Du gav dem Brød fra Himmelen til at stille deres Sult og lod Vand springe ud af Klippen til at slukke deres Tørst. Og du bød dem drage hen og tage det Land i Besiddelse, som du med løftet Hånd havde lovet dem.
\par 16 Men vore Fædre blev overmodige og halsstarrige og hørte ikke på dine Bud;
\par 17 de vægrede sig ved at lyde og ihukom ikke dine Undergerninger, som du havde gjort iblandt dem; de blev halsstarrige og satte sig i Hovedet at vende tilbage til Trældommen i Ægypten. Men du er Forladelsens Gud, nådig og barmhjertig, langmodig og rig på Miskundhed, og du svigtede dem ikke.
\par 18 Selv da de lavede sig et støbt Billede af en Kalv og sagde: Der er din Gud, som førte dig ud af Ægypten! og gjorde sig skyldige i svare Gudsbespottelser,
\par 19 svigtede du dem i din store Barmhjertighed ikke i Ørkenen; Skystøtten veg ikke fra dem om Dagen, men ledede dem på Vejen, ej heller Ildstøtten om Natten, men lyste for dem på Vejen, de skulde vandre.
\par 20 Du gav dem din gode Ånd for at give dem Indsigt og forholdt ikke deres Mund din Manna, og du gav dem Vand til at slukke deres Tørst.
\par 21 I fyrretyve År sørgede du for dem i Ørkenen, så de ingen Nød led; deres Klæder sledes ikke op, og deres Fødder hovnede ikke.
\par 22 Du gav dem Riger og Folkeslag; som du uddelte Stykke for Stykke. De tog Kong Sihon af Hesjbons Land og Kong Og af Basans Land i Besiddelse.
\par 23 Du gjorde deres Børn talrige som Himmelens Stjerner og førte dem ind i det Land, du havde lovet deres Fædre, at de skulde komme ind i og tage i Besiddelse;
\par 24 og Børnene kom og tog Landet i Besiddelse, og du underlagde dem Landets indbyggere, Kana'anæerne, og gav dem i deres Hånd, både deres Konger og Folkene i Landet, så de kunde handle med dem, som de fandt for godt.
\par 25 De indtog befæstede Byer og frugtbart Land, og Huse, fulde af alle Slags Goder, tog de i Besiddelse og udhugne Cisterner, Vingårde, Oliventræer og Frugttræer i Mængde; og de spiste sig mætte og blev fede og gjorde sig til gode med dine store Rigdomme.
\par 26 Men de var genstridige og satte sig op imod dig; de kastede din Lov bag deres Ryg, dræbte dine Profeter, som talte dem alvorligt til for at lede dem tilbage til dig, og gjorde sig skyldige i svare Gudsbespottelser.
\par 27 Da gav du dem i deres Fjenders Hånd, og de bragte Trængsel over dem. Men når de da i deres Trængsel råbte til dig, hørte du dem i Himmelen og sendte dem i din store Barmhjertighed Befriere, som friede dem af deres Fjenders Hånd.
\par 28 Men når de fik Ro, handlede de atter ilde for dit Åsyn. Da overlod du dem i deres Fjenders Hånd, og de undertvang dem. Men når de atter råbte til dig, hørte du dem i Himmelen og udfriede dem i din Barmhjertighed Gang på Gang.
\par 29 Du talede dem alvorligt til for at lede dem tilbage til din Lov, men de var overmodige og vilde ikke høre på dine Bud, og de syndede mod dine Lovbud, som dog holder det Menneske i Live, der gør efter dem; de vendte i Genstridighed Ryggen til og var halsstarrige og vilde ikke lyde.
\par 30 I mange År var du langmodig imod dem og talede dem alvorligt til ved din Ånd gennem dine Profeter; men da de ikke vilde høre, gav du dem til Pris for Hedningefolkene.
\par 31 Dog har du i din store Barmhjertighed ikke helt tilintetgjort dem eller forladt dem, thi du er en nådig og barmhjertig Gud!
\par 32 Og nu, vor Gud, du store, vældige, frygtelige Gud, som holder fast ved Pagten og Miskundheden! Lad ikke alle de Lidelser, der har ramt os, vore Konger, Øverster, Præster, Profeter, vore Fædre og hele dit Folk fra Assyrerkongernes Dage indtil i Dag, synes ringe for dine Øjne!
\par 33 I alt, hvad der er kommet over os, står du retfærdig, thi du har vist dig trofast, men vi var ugudelige.
\par 34 Vore Konger, Øverster og Præster og vore Fædre handlede ikke efter din Lov og lyttede ikke til dine Bud og de Vidnesbyrd, du lod dem blive til Del.
\par 35 Da de var i Besiddelse af deres Rige og de store Rigdomme, du gav dem, og af det vidtstrakte, frugtbare Land, du oplod for dem, tjente de dig ikke, og de omvendte sig ikke fra deres onde Gerninger.
\par 36 Se, derfor er vi nu Trælle; i det Land, du gav vore Fædre, for at de skulde nyde dets Frugter og Rigdom, er vi Trælle;
\par 37 dets rige Afgrøde tilfalder de Konger, du for vore Synders Skyld har øvet Magten over os, og de gør, hvad de lyster, med vore Kroppe og vort Kvæg. Vi er i stor Nød!"
\par 38 I Henhold til alt dette indgår vi med Navns Underskrift en urokkelig Pagt, og den beseglede Skrivelse er underskrevet af vore Øverster, Leviter og Præster.

\chapter{10}

\par 1 Den beseglede Skrivelse er underskrevet af: Statholderen Nehemias, Hakaljas Søn, og Zidkija,
\par 2 Seraja, Azarja, Jirmeja,
\par 3 Pasjhur, Amarja, Malkija,
\par 4 Hattusj, Sjebanja, Malluk,
\par 5 Harim, Meremot, Obadja,
\par 6 Daniel, Ginneton, Baruk,
\par 7 Mesjullam, Abija, Mijjamin,
\par 8 Ma'azja, Bilgaj og Sjemaja det var Præsterne.
\par 9 Leviterne Jesua, Azanjas Søn, Binnuj af Henadads Sønner, Kadmiel
\par 10 og deres Brødre Sjebanja, Hodija, Kelita, Pelaja, Hanan,
\par 11 Mika, Rehob, Hasjabja,
\par 12 Zakkur, Sjerebja, Sjebanja,
\par 13 Hodija, Bani og Beninu.
\par 14 Folkets Overhoveder Par'osj, Pahat-Moab, Elam, Zattu, Bani,
\par 15 Bunni Azgad, Bebaj,
\par 16 Adonija, Bigvaj, Adin,
\par 17 Ater, Hizkija, Azzur,
\par 18 Hodija, Hasjum, Bezaj,
\par 19 Harif, Anatot, Nebaj,
\par 20 Magpiasj, Mesjullam, Hezir,
\par 21 Mesjezab'el, Zadok Jaddua,
\par 22 Pelatja, Hanan, Anaja,
\par 23 Hosea, Hananja, Hassjub,
\par 24 Hallohesj, Pilha, Sjobek,
\par 25 Rehum, Hasjabna, Ma'aseja,
\par 26 Ahija, Hanan, Anan,
\par 27 Malluk, Harim og Ba'ana.
\par 28 Og det øvrige Folk, Præsterne, Leviterne, Dørvogterne, Sangerne, Tempeltrællene og alle de, der har skilt sig ud fra Hedningerne for at holde sig til Guds Lov, med deres Hustruer, Sønner og Døtre, for så vidt de har Forstand til at fatte det,
\par 29 slutter sig til deres højere stående Brødre og underkaster sig Forbandelsen og Eden om at ville følge Guds Lov, der er givet os ved Guds Tjener Moses, og overholde og udføre alle HERRENs, vor Herres, Bud, Bestemmelser og Anordninger:
\par 30 Vi vil ikke give Hedningerne i Landet vore Døtre eller tage deres Døtre til Hustruer for vore Sønner;
\par 31 vi vil ikke på Sabbaten eller nogen Helligdag købe noget af Hedningerne i Landet, når de på Sabbaten kommer med deres Varer og al Slags Korn og falbyder det; vi vil hvert syvende År lade Landet ligge hen og give Afkald på enhver Fordring;
\par 32 vi vil påtage os en årlig Skat på en Tredjedel Sekel til Tjenesten i vor Guds Hus,
\par 33 til Skuebrødene, det daglige Afgrødeoffer, det daglige Brændoffer, Ofrene på Sabbaterne, Nymånedagene og Højtiderne, Helligofrene og Syndofrene til Soning for Israel og til alt Arbejde ved vor Guds Hus.
\par 34 Hvad Brænde der ydes, har vi, Præsterne, Leviterne og Folket, kastet Lod om at bringe til vor Guds Hus, Fædrenehus for Fædrenehus, til fastsat Tid År efter År for at skaffe Ild på HERREN vor Guds Alter, som det er foreskrevet i Loven.
\par 35 Vi vil År for År bringe Førstegrøden af vor Jord og af alle Frugttræer til HERRENs Hus,
\par 36 og vi vil bringe det førstefødte af vore Sønner og vort Kvæg, som det er foreskrevet i Loven, og det førstefødte af vort Hornkvæg og Småkvæg til vor Guds Hus til Præsterne, som gør Tjeneste i vor Guds Hus;
\par 37 og Førstegrøden af vort Grovmel og af Frugten af alle Slags Træer, af Most og Olie vil vi bringe til Kamrene i vor Guds Hus til Præsterne og Tienden af vore Marker til Leviterne. Leviterne samler selv Tienden ind i alle de Byer, hvor vi har vort Agerbrug;
\par 38 og Præsten, Arons Søn, er til Stede hos Leviterne, når de indsamler Tienden; og Leviterne bringer Tiende af Tienden til vor Guds Hus, til Forrådshusets Kamre.
\par 39 Thi Israeliterne og Levis Efterkommere bringer Offerydelsen af Kornet, Mosten og Olien til Kamrene, hvor Helligdommens Kar og de tjenstgørende Præster, Dørvogterne og Sangerne er. Vi vil således ikke svigte vor Guds Hus.

\chapter{11}

\par 1 Og Folkets Øverster bosatte sig i Jerusalem, medens det øvrige Folk kastede Lod således, at hver tiende Mand skulde bosætte sig i Jerusalem, den hellige By, medens de ni Tiendedele skulde bo i Byerne.
\par 2 Og Folket velsignede alle de Mænd, som frivilligt bosatte sig i Jerusalem.
\par 3 Følgende er de Overhoveder i vor Landsdel, som boede i Jerusalem og i Judas Byer; de boede hver på sin Ejendom i deres Byer, Israel, Præsterne, Leviterne, Tempeltrællene og Efterkommerne af Salomos Trælle.
\par 4 I Jerusalem boede af Judæere og Benjaminiter: Af Judæerne: Ataja, en Søn af Uzzija, en Søn af Zekarja, en Søn af Amarja, en Søn af Sjefatja, en Søn af Mahalal'el af Perez's Efterkommere,
\par 5 og Ma'aseja, en Søn af Baruk, en Søn af Kol-Hoze, en Søn af Hazaja, en Søn af Adaja, en Søn af Jojarib, en Søn af Sjelaniten Zekarja.
\par 6 Alle Perez's Efterkommere, der boede i Jerusalem, udgjorde 468 dygtige Mænd.
\par 7 Følgende Benjaminiter: Sallu, en Søn af Mesjullam, en Søn af Joed, en Søn af Pedaja, en Søn af Kolaja, en Søn af Ma'aseja, en Søn af Itiel, en Søn af Jesja'ja,
\par 8 og hans Brødre, dygtige Krigere. 928.
\par 9 Joel, Zikris Søn, var deres Befalingsmand, og Juda, Hassenuas Søn, var den næstøverste Befalingsmand i Byen.
\par 10 Af Præsterne: Jedaja, Jojarib Jakin,
\par 11 Seraja, en Søn af Hilkija, en Søn af Mesjullam, en Søn af Zadok, en Søn af Merajot, en Søn af Ahitub, Øversten over Guds Hus,
\par 12 og deres Brødre, der udførte Tjenesten i Templet, 822; og Adaja, en Søn af Jeroham, en Søn af Pelalja, en Søn af Amzi, en Søn af Zekarja, en Søn af Pasjhur, en Søn af Malkija,
\par 13 og hans Brødre, Overhovederne for Fædrenehusene, 242; og Amasjsaj, en Søn af Azar'el, en Søn af Azaj, en Søn af Mesjillemot, en Søn af Immer,
\par 14 og hans Brødre, dygtige Mænd 128. Deres Befalingsmand var Zabdiel, Gedolims Søn.
\par 15 Af Leviterne: Sjemaja, en Søn af Hassjub, en Søn af Azrikam, en Søn af Hasjabja, en Søn af Bunni,
\par 16 og Sjabbetaj og Iozabad, som forestod de ydre Arbejder ved Guds Hus og hørte til Leviternes Overhoveder,
\par 17 og Mattanja, en Søn af Mika, en Søn af Zabdi, en Søn af Asaf, Lederen af Lovsangen, der ved Bønnen istemte Ordene lov HERREN! og Bakbukja, den næstøverste af hans Brødre, og Abda, en Søn af Sjammua, en Søn af Galal, en Søn af Jedutun.
\par 18 Alle Leviterne i den hellige By udgjorde 284.
\par 19 Af Dørvogterne: Akkub, Talmon og deres Brødre, der holdt Vagt ved Portene, 172.
\par 20 Resten af Israeliterne, Præsterne og Leviterne boede i alle de andre Byer i Juda, hver på sin Ejendom.
\par 21 Tempeltrællene boede på Ofel; Ziha og Gisjpa var sat over Tempeltrællene.
\par 22 Leviternes foresatte i Jerusalem ved Tjenesten i Guds Hus var Uzzi, en Søn af Bani, en Søn af Hasjabja, en Søn af Mattanja, en Søn af Mika af Asafs Efterkommere, det er Sangerne.
\par 23 Der var nemlig udstedt en kongelig Befaling om dem, og der var tilsikret Sangerne dagligt Underhold.
\par 24 Petaja, Mesjezab'els Søn, af Judas Søn Zeras Efterkommere, forhandlede med Kongen i alle Folkets Sager.
\par 25 Hvad de åbne Byer med deres Marker angår, boede der Judæere i Kirjat-Arba med Småbyer, Dibon med Småbyer, Jekabze'el med Småbyer,
\par 26 Jesua, Molada, Bet-Pelet,
\par 27 Hazar-Sjual, Be'ersjeba med Småbyer,
\par 28 Ziklag, Mekona med Småbyer,
\par 29 En-Rimmon, Zor'a, Jarmut,
\par 30 Zanoa, Adullam med Landsbyer, Lakisj med Marker og Azeka med Småbyer. De bosatte sig fra Be'ersjeba til Hinnoms Dal.
\par 31 Benjaminiterne boede i Geba, Mikmas, Ajja, Betel med Småbyer,
\par 32 Anatot, Nob, Ananja,
\par 33 Hazor, Rama, Gittajim,
\par 34 Hadid, Zebo'im, Neballat,
\par 35 Lod, Ono og Håndværkerdalen.
\par 36 Af Leviterne boede nogle Afdelinger i Juda og Benjamin.

\chapter{12}

\par 1 Følgende er Præsterne og Leviter, der drog op med Zerubbabel, Sjealtiels Søn, og Jesua: Seraja, Jirmeja, Ezra,
\par 2 Amarja, Malluk, Hattusj.
\par 3 Sjekanja, Harim, Meremot,
\par 4 Iddo, Ginnetoj, Abija,
\par 5 Mijjamin, Ma'adja, Bilga,
\par 6 Sjemaja, Jojarib, Jedaja,
\par 7 Sallu, Amok, Hilkija og Jedaja. Det var Overhovederne for Præsterne og deres Brødre på Jesuas Tid.
\par 8 Leviterne: Jesua, Binnuj, Kadmiel, Sjerebja, Juda, Mattanja, der sammen med sine Brødre forestod Lovsangen,
\par 9 medens Bakbukja og Unni sammen med deres Brødre stod over for dem efter deres Afdelinger.
\par 10 Jesua avlede Jojakim, Jojakim avlede Eljasjib, Eljasjib avlede Jojada,
\par 11 Jojada avlede Johanan, og Johanan avlede Jaddua.
\par 12 På Jojakims Tid var Overhovederne for Præsternes Fædrenehuse følgende: Meraja for Seraja, Hananja for Jirmeja,
\par 13 Mesjullam for Ezra, Johanan for Amarja,
\par 14 Jonatan for Malluk, Josef for Sjebanja,
\par 15 Adna for Harim, Helkajtor Merajot,
\par 16 Zekarja for Iddo, Mlesjullam for Ginneton,
\par 17 Zikri for Abija,....... for Minjamin, Piltaj for Ma'adja,
\par 18 Sjammua for Bilga, Jonatan for Sjemaja,
\par 19 Mattenaj for Jojarib, Uzzi for Jedaja,
\par 20 Kallaj for Sallu, Eber for Amok,
\par 21 Hasjabja for Hilkija og Netan'el for Jedaja.
\par 22 Leviterne: I Eljasjibs, Jojadas, Johanans og Jadduas Dage optegnedes Overhovederne for Fædrenehusene og Præsterne indtil Perseren Darius's Regering.
\par 23 Af Levis Efterkommere optegnedes Overhovederne for Fædrenehusene i Krønikebogen ned til Johanans, Eljasjibs Søns, Dage.
\par 24 Og Leviternes Overhoveder var: Hasjabja, Sjerebja, Jesua, Binnuj, Kadmiel og deres Brødte, der stod over for dem for at synge Lovsangen og Takkesangen efter den Guds Mand Davids Bud, den ene Afdeling efter den anden;
\par 25 og Mattanja, Bakbukja og Obadja, Mesjullam, Talmon og Akkub var Dørvogtere og holdt Vagt ved Portenes Forrådskamre.
\par 26 Disse var Overhoveder på Jojakims Tid, en Søn af Jesua, en Søn af Jozadak, og på Statholderen Nehemias's og Præsten Ezra den Skriftlærdes Tid.
\par 27 Da Jerusalems Mur skulde indvies opsøgte man Leviterne alle Vegne, hvor de boede, og bragte dem til Jerusalem, for at de skulde fejre Indvielsen med Fryd og Takkesang, med Sang, Cymbler, Harper og Citre.
\par 28 Da samledes Sangerne fra Egnen om Jerusalem og fra Netofatifernes Landsbyer,
\par 29 fra Bet-Gilgal, fra Gebas og Azmavets Marker; thi Sangerne havde bygget sig Landsbyer rundt om Jerusalem.
\par 30 Da Præsterne og Leviterne havde renset sig, rensede de Folket, Portene og Muren.
\par 31 Så lod jeg Judas Øverster stige op på Muren og opstillede to store Lovprisningstog. Det ene drog til højre oven på Muren ad Møgporten til,
\par 32 og med det fulgte Hosjaja og den ene Halvdel af Judas Øverster;
\par 33 dernæst nogle af Præsterne med Trompeter, Azarja, Ezra, Mesjullam,
\par 34 Juda, Benjamin, Sjemaja og Jirmeja;
\par 35 endvidere Zekarja, en Søn af Jonatan, en Søn af Sjemaja, en Søn af Mattanja, en Søn af Mika, en Søn af Zakkur, en Søn af Asaf,
\par 36 og hans Brødre Sjemaja, Azar'el, Milalaj, Gilalaj, Ma'aj, Netan'el, Juda, Hanani med den Guds Mand Davids Musikinstrumenter, med Ezra den Skriftlærde i Spidsen;
\par 37 og de gik over Kildeporten; derpå gik de lige ud op ad Trinene til Davidsbyen, ad Opgangen på Muren oven for Davids Palads hen til Vandporten mod Øst.
\par 38 Det andet Lovprisningstog, hvor jeg og den anden Halvdel af Folkets Øverster var med, drog til venstre oven på Muren, over Ovntårnet til den brede Mur
\par 39 og videre over Efraimsporten, den gamle Port, Fiskeporten, Hanan'eltårnet og Meatårnet til Fåreporten og stillede sig op i Fængselsporten.
\par 40 Derpå stillede de to Lovprisningstog sig op i Guds Hus, jeg sammen med Halvdelen af Øversterne
\par 41 og Præsterne Eljakim, Ma'aseja, Minjamin, Mika, Eljoenaj, Zekarja, Hananja med Trompeter,
\par 42 endvidere Ma'aseja, Sjemaja, El'azar, Uzzi, Johanan, Malkija, Elam og Ezer. Og Sangerne stemte i, ledede af Jizraja.
\par 43 På den Dag ofrede de store Slagtofre og var glade, thi Gud havde bragt dem stor Glæde; også Kvinderne og Børnene var glade; og Glæden i Jerusalem hørtes langt bort.
\par 44 På den Dag indsattes der Mænd til at have Tilsyn med de Kamre, der brugtes til Forrådene, Offerydelserne, Førstegrøden og Tienden, for i dem at opsamle de i Loven foreskrevne Afgifter til Præsterne og Leviterne fra de forskellige Bymarker, thi Juda glædede sig over Præsterne og Leviterne, der gjorde tjeneste;
\par 45 og disse tog Vare på, hvad der var at varetage for deres Gud og ved Renselsen, ligesom også Sangerne og Dørvogterne gjorde deres Gerning efter Davids og hans Søn Salomos Bud.
\par 46 Thi allerede på Davids Tid var Asaf Leder for Sangerne og for Lov- og Takkesangene til Gud.
\par 47 Hele Israel gav på Zerubbabels og Nehemias's Tid Afgifter til Sangerne og Dørvogterne, efter som det krævedes Dag for Dag; og de gav Leviterne Helliggaver, og Leviterne gav Arons Sønner Helliggaver.

\chapter{13}

\par 1 På den Tid blev der læst op af Moses's Bog for Folket, og man fandt skrevet deri, at ingen Ammonit eller Moabit nogen Sinde måtte få Adgang til Guds Menighed,
\par 2 fordi de ikke kom Israeliterne i Møde med Brød og Vand, og fordi han havde lejet Bileam til at forbande dem, men vor Gud vendte Forbandelsen til Velsignelse.
\par 3 Da de nu hørte Loven, udskilte de alle fremmede af Israel.
\par 4 Nogen Tid i Forvejen havdePræsten Eljasjib, hvem Opsynet med Kamrene i vor Guds Hus var overdraget, og som var i Slægt med Tobija,
\par 5 ladet indrette et stort Kammer til Tobija der, hvor man før henlagde Afgrødeofferet, Røgelsen, Karrene, Tienden af Kornet, Mosten og Olien, de i Loven foreskrevne Afgifter til Leviterne, Sangerne og Dørvogterne såvel som Offerydelsen til Præsterne.
\par 6 Da alt dette fandt Sted, var jeg ikke i Jerusalem, thi i Kong Artaxerxes af Babels to og tredivte Regeringsår var jeg rejst til Kongen. Men nogen Tid efter bad jeg Kongen om Tilladelse til at rejse,
\par 7 og da jeg kom til Jerusalem og opdagede det onde, Eljasjib havde øvet for Tobijas Skyld ved at indrette ham et Kammer i Guds Hus's Forgårde,
\par 8 harmede det mig højligen; og jeg kastede alt Tobijas Bohave ud af Kammeret
\par 9 og bød, at man skulde rense Kammeret, hvorefter jeg atter bragte Guds Hus's Kar, Afgrødeofferet og Røgelsen derind.
\par 10 Da fik jeg at vide, at Afgifterne til Leviterne ikke svaredes dem, og derfor var de Leviter og Sangere, der skulde gøre Tjeneste, flyttet ud hver til sin Landejendom;
\par 11 så gik jeg i Rette med Forstanderne og spurgte dem: Hvorfor er Guds Hus blevet vanrøgtet?" Og jeg fik atter Leviterne samlet og satte dem på deres Pladser.
\par 12 Så bragte hele Juda Tienden af Kornet, Mosten og Olien til Forrådskamrene;
\par 13 og jeg overdrog Tilsynet med Forrådskamrene til Præsten Sjelemja, Skriveren Zadok og Pedaja af Leviterne og gav dem til Medhjælper Hanan, en Søn af Zakkur, en Søn af Mattanja, da de regnedes for pålidelige; og dem pålå det så at uddele Tienden til deres Brødre.
\par 14 Kom mig det i Hu, min Gud, og udslet ikke de Kærlighedsgerninger, jeg har gjort mod min Guds Hus til Gavn for Tjenesten der!
\par 15 I de Dage så jeg i Juda nogle træde Persekarrene på Sabbaten, og andre så jeg bringe Horn i Hus eller læsse det på deres Æsler, ligeledes Vin, Druer, Figener og alle Slags Varer, og bringe det til Jerusalem på Sabbaten. Dem formanede jeg da, når de solgte Levnedsmidler.
\par 16 Også havde Folk fra Tyrus bosat sig der, og de kom med Fisk og alskens Varer og solgte dem på Sabbaten til Jøderne i Jerusalem.
\par 17 Jeg gik derfor i Rette med de store i Juda og sagde til dem: Hvor kan l handle så ilde og vanhellige Sabbatsdagen?
\par 18 Har ikke vor Gud bragt al denne Ulykke over os og over denne By, fordi eders Fædre handlede således? Og I bringer endnu mere Vrede over Israel ved at vanhellige Sabbaten!
\par 19 Og så snart Mørket faldt på i Jerusalems Porte ved Sabbatens Frembrud, bød jeg, at Portene skulde lukkes, og at de ikke måtte åbnes, før Sabbaten var omme; og jeg satte nogle af mine Folk ved Portene for at vogte på, at der ikke førtes Varer ind på Sabbaten.
\par 20 Da nu de handlende og de, der solgte alle Slags Varer, et Par Gange var blevet uden for Jerusalem Natten over,
\par 21 advarede jeg dem og sagde: "Hvorfor bliver I Natten over uden for Muren? Hvis I gør det en anden Gang, lægger jeg Hånd på eder!" Og siden kom de ikke mere på Sabbaten.
\par 22 Fremdeles bød jeg Leviterne, at de skulde rense sig og komme og holde Vagt ved Portene, for at Sabbatsdagen kunde holdes hellig. Kom mig også det i Hu, min Gud, og forbarm dig over mig efter din store Miskundhed!
\par 23 På samme Tid lagde jeg også Mærke til, at hos de Jøder, der havde ægtet asdoditiske, ammonitiske eller moabitiske Kvinder,
\par 24 talte Halvdelen af børnene Asdoditisk eller et af de andre Folks Sprog, men kunde ikke tale Jødisk.
\par 25 Da gik jeg i Rette med dem og forbandede dem, ja, jeg slog nogle af dem og rykkede dem i Håret og besvor dem ved Gud: Giv dog ikke deres Sønner eders Døtre til Ægte og tag ikke deres Døtre til Hustruer for eders Sønner eller eder selv!
\par 26 Syndede ikke Kong Salomo af Israel for slige Kvinders Skyld? Mage til Konge fandtes dog ikke blandt de mange Folk, og han var så elsket af sin Gud, at Gud gjorde ham til Konge over hele Israel; og dog fik de fremmede Kvinder endog ham til at synde!
\par 27 Skal vi da virkelig høre om eder, at I begår al denne svare Misgerning og forbryder eder mod vor Gud ved at ægte fremmede Kvinder?
\par 28 En af Ypperstepræsten Eljasjibs Søn Jojadas Sønner, der var Horoniten Sanballats Svigersøn, jog jeg bort fra min Nærhed.
\par 29 Tilregn dem, min Gud, at de besmittede Præstedømmet og Præsternes og Leviternes Pagt!
\par 30 Således rensede jeg dem for alt fremmed, og jeg ordnede Tjenesten for Præsterne og Leviterne efter det Arbejde, hver især havde.
\par 31 og Ydelsen af Brænde til fastsatte Tider og af Førstegrøderne. Kom mig i Hu, min Gud, og regn mig det til gode!


\end{document}