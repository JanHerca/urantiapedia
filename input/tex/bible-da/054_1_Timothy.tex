\begin{document}

\title{Første Timotheusbrev}


\chapter{1}

\par 1 Paulus, Kristi Jesu Apostel efter Befaling af Gud, vor Frelser, og Kristus Jesus, vort Håb,
\par 2 til Timotheus, sit ægte Barn i Troen: Nåde, Barmhjertig Fred fra Gud Fader og Kristus Jesus vor Herre!
\par 3 Det var derfor, jeg opfordrede dig til at blive i Efesus, da jeg drog til Makedonien, for at du skulde påbyde visse Folk ikke at føre fremmed Lære
\par 4 og ikke at agte på Fabler og Slægtregistre uden Ende, som mere fremme Stridigheden end Guds Husholdning i Tro.
\par 5 Men Påbudets Endemål er Kærlighed af et rent Hjerte og af en god Samvittighed og af en uskrømtet Tro,
\par 6 hvorfra nogle ere afvegne og have vendt sig til intetsigende Snak,
\par 7 idet de ville være Lovlærere uden at forstå, hverken hvad de sige, eller hvorom de udtale sig så sikkert.
\par 8 Men vi vide, at Loven er god, dersom man bruger den lovmæssigt,
\par 9 idet man veed dette, at Loven ikke er sat for den retfærdige, men for lovløse og ulydige, ugudelige og Syndere, ryggesløse og vanhellige, for dem, som øve Vold imod deres Fader og Moder, for Manddrabere,
\par 10 utugtige, Syndere imod Naturen, Menneskerøvere, Løgnere, Menedere, og hvad andet der er imod den sunde Lære,
\par 11 efter den salige Guds Herligheds Evangelium, som er blevet mig betroet.
\par 12 Jeg takker ham, som gjorde mig stærk, Kristus Jesus, vor Herre, fordi han agtede mig for tro, idet han satte mig til en Tjeneste,
\par 13 skønt jeg forhen var en Bespotter og en Forfølger og en Voldsmand; men der blev vist mig Barmhjertighed, thi jeg gjorde det vitterligt i Vantro,
\par 14 Ja, vor Herres Nåde viste sig overvættes rig med Tro og Kærlighed i Kristus Jesus.
\par 15 Den Tale er troværdig og al Modtagelse værd, at Kristus Jesus kom til Verden for at frelse Syndere, iblandt hvilke jeg er den største"
\par 16 Men derfor blev der vist mig Barmhjertighed, for at Jesus Kristus kunde på mig som den første vise hele sin Langmodighed, til et Forbillede på dem, som skulle tro på ham til evigt Liv.
\par 17 Men Evighedens Konge, den uforkrænkelige, usynlige, eneste Gud være Pris og Ære i Evighedernes Evigheder! Amen.
\par 18 Dette Påbud betror jeg dig, mit Barn Timotheus, ifølge de Profetier, som tilforn ere udtalte over dig, at du efter dem strider den gode Strid,
\par 19 idet du har Tro og en god Samvittighed, hvilken nogle have stødt fra sig og lidt Skibbrud på Troen;
\par 20 iblandt dem ere Hymenæus og Aleksander, hvilke jeg har overgivet til Satan, for at de skulle tugtes til ikke at bespotte.

\chapter{2}

\par 1 Jeg formaner da først af alt til, at der holdes Bønner, Påkaldelser, Forbønner, Taksigelser for alle Mennesker,
\par 2 for Konger og alle dem, som ere i Højhed, at vi må leve et roligt og stille Levned i al Gudsfrygt og Ærbarhed;
\par 3 dette er smukt og velbehageligt for Gud, vor Frelser,
\par 4 som vil, at alle Mennesker skulle frelses og komme til Sandheds Erkendelse.
\par 5 Thi der er een Gud, og også een Mellemmand imellem Gud og Mennesker, Mennesket Kristus Jesus,
\par 6 som gav sig selv til en Genløsnings Betaling for alle, hvilket er Vidnesbyrdet i sin Tid,
\par 7 og for dette er jeg bleven sat til Prædiker og Apostel (jeg siger Sandhed, jeg lyver ikke), en Lærer for Hedninger i Tro og Sandhed.
\par 8 Så vil jeg da, at Mændene på ethvert Sted, hvor de bede, skulle opløfte fromme Hænder uden Vrede og Trætte.
\par 9 Ligeså, at Kvinder skulle pryde sig i sømmelig Klædning med Blufærdighed og Ærbarhed, ikke med Fletninger og Guld eller Perler eller kostbar Klædning,
\par 10 men, som det sømmer sig Kvinder, der bekende sig til Gudsfrygt, med gode Gerninger.
\par 11 En Kvinde bør i Stilhed lade sig belære, med al Lydighed;
\par 12 men at være Lærer tilsteder jeg ikke en Kvinde, ikke heller at byde over Manden, men at være i Stilhed.
\par 13 Thi Adam blev dannet først, derefter Eva;
\par 14 og Adam blev ikke bedraget, men Kvinden blev bedraget og er falden i Overtrædelse.
\par 15 Men hun skal frelses igennem sin Barnefødsel, dersom de blive i Tro og Kærlighed og Hellighed med Ærbarhed.

\chapter{3}

\par 1 Den Tale er troværdig; dersom nogen begærer en Tilsynsgerning har han Lyst til en skøn Gerning.
\par 2 En Tilsynsmand bør derfor være ulastelig, een Kvindes Mand, ædruelig, sindig, høvisk, gæstfri, dygtig til at lære andre;
\par 3 ikke hengiven til Vin, ikke til Slagsmål, men mild, ikke kivagtig, ikke pengegridsk;
\par 4 en Mand, som forestår sit eget Hus vel, som har Børn, der ere lydige med al Ærbarhed;
\par 5 (dersom en ikke veed at forestå sit eget Hus, hvorledes vil han da kunne sørge for Guds Menighed?)
\par 6 ikke ny i Troen, som at han ikke skal blive opblæst og falde ind under Djævelens Dom.
\par 7 Men han bør også have et godt Vidnesbyrd af dem, som ere udenfor; for at han ikke skal falde i Forhånelse og Djævelens Snare.
\par 8 Menighedstjenere bør ligeledes være ærbare, ikke tvetungede, ikke hengivne til megen Vin, ikke til slet Vinding,
\par 9 bevarende Troens Hemmelighed i en ren Samvittighed.
\par 10 Men også disse skulle først prøves, og siden gøre Tjeneste, hvis de ere ustrafelige.
\par 11 Kvinder bør ligeledes være ærbare, ikke bagtaleriske, ædruelige, tro i alle Ting.
\par 12 En Menighedstjener skal være een Kvindes Mand og forestå sine Børn og sit eget Hus vel.
\par 13 Thi de, som have tjent vel i Menigheden, de erhverve sig selv en smuk Stilling og megen Frimodighed i Troen på Kristus Jesus.
\par 14 Disse Ting skriver jeg dig til, ihvorvel jeg håber at komme snart til dig;
\par 15 men dersom jeg tøver, da skal du heraf vide, hvorledes man bør færdes i Guds Hus, hvilket jo er den levende Guds Menighed, Sandhedens Søjle og Grundvold.
\par 16 Og uden Modsigelse stor er den Gudsfrygtens Hemmelighed: Han, som blev åbenbaret i Kød, blev retfærdiggjort i Ånd, set af Engle, prædiket iblandt Hedninger, troet i Verden, optagen i Herlighed.

\chapter{4}

\par 1 Men Ånden siger klarlig, at i kommende Tider ville nogle falde fra Troen, idet de agte på forførende Ånder og på Dæmoners Lærdomme,
\par 2 ved Løgnlæreres Hykleri, som ere brændemærkede i deres egen Samvittighed,
\par 3 som byde, at man ikke må gifte sig, og at man skal afholde sig fra Spiser, hvilke Gud har skabt til at nydes med Taksigelse af dem, som tro og have erkendt Sandheden.
\par 4 Thi al Guds Skabning er god, og intet er at forkaste, når det tages med Taksigelse;
\par 5 thi det helliges ved Guds Ord og Bøn.
\par 6 Når du foreholder Brødrene dette, er du en god Kristi Jesu Tjener, idet du næres ved Troens og den gode Læres Ord, den, som du har efterfulgt;
\par 7 men afvis de vanhellige og kælingagtige Fabler! Derimod øv dig selv i Gudsfrygt!
\par 8 Thi den legemlige Øvelse er nyttig til lidet, men Gudsfrygten er nyttig til alle Ting, idet den har Forjættelse for det Liv, som nu er, og for det, som kommer.
\par 9 Den Tale er troværdig og al Modtagelse værd.
\par 10 Thi derfor lide vi Møje og Forhånelser, fordi vi have sat vort Håb til den levende Gud, som er alle Menneskers Frelser, mest deres, som tro.
\par 11 Påbyd og lær dette!
\par 12 Lad ingen ringeagte dig for din Ungdoms Skyld, men bliv et Forbillede for dem, som tro, i Tale, i Vandel, i Kærlighed, i Tro, i Renhed!
\par 13 Indtil jeg kommer, så giv Agt på Oplæsningen, Formaningen, Undervisningen.
\par 14 Forsøm ikke den Nådegave, som er i dig, som blev given dig under Profeti med Håndspålæggelse af de Ældste.
\par 15 Tænk på dette, lev i dette, for at din Fremgang må være åbenbar for alle.
\par 16 Giv Agt på dig selv og på Undervisningen: hold ved dermed; thi når du gør dette,skal du frelse både dig selv og dem, som høre dig.

\chapter{5}

\par 1 En gammel Mand må du ikke skælde på, men forman ham som en Fader, unge Mænd som Brødre,
\par 2 gamle Kvinder som Mødre, unge som Søstre, i al Renhed.
\par 3 Ær Enker, dem, som virkelig ere Enker;
\par 4 men om en Enke har Børn eller Børnebørn, da lad dem først lære at vise deres eget Hus skyldig Kærlighed og gøre Gengæld imod Forældrene; thi dette er velbebageligt for Gud.
\par 5 Men den, som virkelig er Enke og står ene, har sat sit Håb til Gud og bliver ved med sine Bønner og Påkaldelser Nat og Dag;
\par 6 men den, som lever efter sine Lyster, er levende død.
\par 7 Forehold dem også dette, for at de må være ulastelige.
\par 8 Men dersom nogen ikke har Omsorg for sine egne og især for sine Husfæller, han har fornægtet Troen og er værre end en vantro.
\par 9 En Enke kan udnævnes når hun er ikke yngre end tresindstyve År, har været een Mands Hustru,
\par 10 har Vidnesbyrd for gode Gerninger, har opfostret Børn, har vist Gæstfrihed, har toet helliges Fødder, har hjulpet nødlidende, har lagt sig efter al god Gerning.
\par 11 Men afvis unge Enker; thi når de i kødelig Attrå gøre Oprør imod Kristus, ville de giftes
\par 12 og have så den Dom, at de have sveget deres første Tro.
\par 13 Tilmed lære de, idet de løbe omkring i Husene, at være ørkesløse, og ikke alene ørkesløse, men også at være sladderagtige og blande sig i uvedkommende Ting, idet de tale, hvad der er utilbørligt.
\par 14 Derfor vil jeg, at unge Enker skulle giftes, føde Børn, styre Hus, ingen Anledning give Modstanderen til slet Omtale.
\par 15 Thi allerede have nogle vendt sig bort efter Satan.
\par 16 Dersom nogen troende Kvinde har Enker, da lad hende hjælpe dem, og lad ikke Menigheden bebyrdes, for at den kan hjælpe de virkelige Enker.
\par 17 De Ældste, som ere gode Forstandere, skal man holde dobbelt Ære værd, mest dem, som arbejde i Tale og Undervisning.
\par 18 Thi Skriften siger: "Du må ikke binde Munden til på en Okse, som tærsker;" og: "Arbejderen er sin Løn værd."
\par 19 Tag ikke imod noget Klagemål imod en Ældste, uden efter to eller tre Vidner.
\par 20 Dem, som Synde, irettesæt dem for alles Åsyn, for at også de andre må have Frygt.
\par 21 Jeg besværger dig for Guds og Kristi Jesu og de udvalgte Engles Åsyn, at du vogter på dette uden Partiskhed, så du intet gør efter Tilbøjelighed.
\par 22 Vær ikke hastig til at lægge Hænder på nogen, og gør dig ikke delagtig i andres Synder; hold dig selv ren!
\par 23 Drik ikke længere bare Vand, men nyd lidt Vin for din Mave og dine jævnlige Svagheder.
\par 24 Nogle Menneskers Synder ere åbenbare og gå forud til Dom; men for nogle følge de også bagefter.
\par 25 Ligeledes ere også de gode Gerninger åbenbare, og de, som det forholder sig anderledes med, kunne ikke skjules.

\chapter{6}

\par 1 Alle de, som ere Trælle under Åg, skulle holde deres egne Herrer al Ære værd, for at ikke Guds Navn og Læren skal bespottes.
\par 2 Men de, der have troende Herrer, må ikke ringeagte dem, fordi de ere Brødre, men tjene dem desto hellere, fordi de, som nyde godt af deres gode Gerning, ere troende og elskede. Lær dette, og forman dertil!
\par 3 Dersom nogen fører fremmed Lære og ikke holder sig til vor Herres Jesu Kristi sunde Ord og til den Lære, som stemmer med Gudsfrygt.
\par 4 han er opblæst, skønt han intet ved, men er syg for Stridigheder og Ordkampe, hvoraf kommer Avind, Kiv, Forhånelser, ond Mistanke
\par 5 og idelige Rivninger hos Mennesker, som ere fordærvede i Sindet og berøvede Sandheden, idet de mene, at Gudsfrygten er en Vinding.
\par 6 Vist nok er Gudsfrygten sammen med Nøjsomhed en stor Vinding.
\par 7 Thi vi have intet bragt ind i Verden, det er da åbenbart, at vi ej heller kunne bringe noget ud derfra.
\par 8 Men når vi have Føde og Klæder, ville vi dermed lade os nøje.
\par 9 Men de, som ville være rige, falde i Fristelse og Snare og mange ufornuftige og skadelige Begæringer, som nedsænke Menneskene i Undergang og Fortabelse;
\par 10 thi Pengegridskheden er en Rod til alt ondt; og ved at hige derefter ere nogle farne vild fra Troen og have gennemstunget sig selv med mange Smerter.
\par 11 Men du, o Guds Menneske! fly disse Ting; jag derimod efter Retfærdighed, Gudsfrygt, Tro, Kærlighed, Udholdenhed, Sagtmodighed;
\par 12 strid Troens gode Strid, grib det evige Liv, til hvilket du er bleven kaldet og har aflagt den gode Bekendelse for mange Vidner.
\par 13 Jeg byder dig for Guds Åsyn, som holder alle Ting i Live, og for Kristus Jesus, som vidnede den gode Bekendelse for Pontius Pilatus,
\par 14 at du holder Budet uplettet, ulasteligt indtil vor Herres Jesu Kristi Åbenbarelse,
\par 15 hvilken den salige og alene mægtige, Kongernes Konge og Herrernes Herre skal lade til Syne i sin Tid;
\par 16 han, som alene har Udødelighed, som bor i et utilgængeligt Lys, hvem intet Menneske har set, ikke heller kan se; ham være Ære og evig Magt! Amen!
\par 17 Byd dem, som ere rige i den nærværende Verden, at de ikke hovmode sig, ej heller sætte Håb til den usikre Rigdom, men til Gud, som giver os rigeligt alle Ting at nyde;
\par 18 at de gøre godt, ere rige på gode Gerninger, gerne give, meddele
\par 19 og således, opsamle sig selv en god Grundvold for den kommende Tid, for at de kunne gribe det sande Liv.
\par 20 O Timotheus! vogt på den betroede Skat, idet du vender dig bort fra den vanhellige, tomme Snak og Indvendingerne fra den falskelig såkaldte Erkendelse,
\par 21 hvilken nogle have bekendt sig til og ere afvegne fra Troen.


\end{document}