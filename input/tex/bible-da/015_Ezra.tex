\begin{document}

\title{Ezras Bog}


\chapter{1}

\par 1 Og i Perserkongen Kyros's første Regeringsår vakte HERREN, for at hans Ord gennem Jeremias's Mund kunde opfyldes, Perserkongen Kyros's Ånd, så han lod følgende udråbe i hele sit Rige og desuden kundgøre ved en Skrivelse:
\par 2 Perserkongen, Kyros gør vitterligt: Alle Jordens Riger har HERREN, Himmelens Gud, givet mig; og han har pålagt mig at bygge ham et Hus i Jerusalem i Juda.
\par 3 Hvem iblandt eder, der hører til hans Folk, med ham være hans Gud, og han drage op til Jerusalem i Juda og bygge HERRENs, Israels Guds, Hus; han er den Gud, som bor i Jerusalem;
\par 4 og alle Steder, hvor de tiloversblevne bor som fremmede, skal Beboerne støtte dem med Sølv, Guld, Heste og Kvæg, bortset fra de frivillige Gaver til Guds Hus i Jerusalem.
\par 5 Da brød Overhovederne for Judas og Benjamins Fædrenehuse og Præsterne og Leviterne op, alle, hvis Ånd Gud vakte, så de drog op for at bygge HERRENs Hus i Jerusalem;
\par 6 og bortset fra alle de frivillige Gaver kom alle deres Naboer dem til Hjælp med alt, både Sølv, Guld, Heste og Kvæg og Kostbarheder i Mængde.
\par 7 Og Kong Kyros udleverede Karrene fra HERRENs Hus, som Nebudkadnezar havde ført bort fra Jerusalem og ladet opstille i sin Guds Hus;
\par 8 dem gav Perserkongen Kyros ny til Skatmesteren Mitredat, og han talte dem og overgav dem til Sjesjbazzar, Judas Fyrste.
\par 9 Og Tallet på dem var følgende: 30 Guldbækkener, 1000 Sølvbakker, 29 Røgelsesskåle,
\par 10 30 Guldbægre, 410 Sølvbægre af ringere Art og 1000 andre Kar,
\par 11 i alt 5400 Kar, dels af Guld og dels af Sølv. Alt dette bragte Sjesjbazzar med sig, da de landflygtige drog op fra Babel til Jerusalem.

\chapter{2}

\par 1 Følgende er de Folk fra vor Landsdel, der drog op fra Landflygtigheden og Fangenskabet. Kong Nebukadnezar af Babel havde ført dem bort til Babel, men nu vendte de tilbage til Jerusalem og Juda, hver til sin By;
\par 2 de kom i Følge med Zerubbabel, Jesua, Nehemja, Seraja, Re'elaja, Mordokaj, Bilsjan, Mispar, Bigvaj, Rehum og Ba'ana'. Tallet på Mændene i Israels Folk var:
\par 3 Par'osj's Efterkommere 2172,
\par 4 Sjefatjas Efterkommere 372,
\par 5 Aras Efterkommere 775,
\par 6 Pahat-Moabs Efterkommere, Jesuas og Joabs Efterkommere, 2812,
\par 7 Elams Efterkommere 1254,
\par 8 7attus Efterkommere 945,
\par 9 Zakkajs Efterkommere 760,
\par 10 Banis Efterkommere 642,
\par 11 Bebajs Efterkommere 623,
\par 12 Azgads Efterkommere 1222,
\par 13 Adonikams Efterkommere 666,
\par 14 Bigvajs Efterkommere 2056,
\par 15 Adins Efterkommere 454,
\par 16 Aters Efterkommere gennem Hizkija 98,
\par 17 Bezajs Efterkommere 323,
\par 18 Joras Efterkommere 112,
\par 19 Hasjums Efterkommere 223,
\par 20 Gibbars Efterkommere 95,
\par 21 Betlehems Efterkommere 123,
\par 22 Mændene fra Netofa 56,
\par 23 Mændene fra Anatot 128,
\par 24 Azmavets Efterkommere 42,
\par 25 Kirjat-Jearims, Kefiras og Be'erots Efterkommere 743,
\par 26 Ramas og Gebas Efterkommere 621,
\par 27 Mændene fra Mikmas 122,
\par 28 Mændene fra Betel og Aj 223,
\par 29 Nebos Efterkommere 52,
\par 30 Magbisj's Efterkommere 156,
\par 31 det andet Elams Efterkommere 1254,
\par 32 Harims Efterkommere 320,
\par 33 Lods, Hadids og Onos Efterkommere 725,
\par 34 Jerikos Efterkommere 345,
\par 35 Sena'as Efterkommere 3630.
\par 36 Præsterne var: Jedajas Efterkommere af Jesuas Hus 973,
\par 37 Immers Efterkommere 1052,
\par 38 Pasjhurs Efterkommere 1247,
\par 39 Harims Efterkommere 1017.
\par 40 Leviterne var: Jesuas og Kadmiels Efterkommere af Hodavjas Efterkommere 74,
\par 41 Tempelsangerne var: Asafs Sønner 128.
\par 42 Dørvogterne var: Sjallums, Aters, Talmons, Akkubs, Hatitas og Sjobajs Efterkommere, i alt 139.
\par 43 Tempeltrællene var: Zihas, Hasufas, Tabbaots,
\par 44 Keros's, Si'as, Padons,
\par 45 Lebanas, Hagabas, Akkubs,
\par 46 Hagabs, Salmajs, Hanans,
\par 47 Giddels, Gahars, Reajas,
\par 48 Rezins, Nekodas, Gazzams,
\par 49 Uzzas, Paseas, Besajs,
\par 50 Asnas, Me'uniternes, Nefusifernes,
\par 51 Bakbuks, Hakufas, Harhurs,
\par 52 Bazluts, Mehidas, Harsjas,
\par 53 Barkos's, Siseras, Temas,
\par 54 Nezias og Hatifas Efterkommere.
\par 55 Efterkommere af Salomos Trælle var: Sotajs, Soferets, Perudas,
\par 56 Ja'alas, Darkons, Giddels,
\par 57 Sjefatjas, Hattils, Pokeret-Hazzebajims og Amis Efterkommere.
\par 58 Tempeltrællene og Efterkommerne af Salomos Trælle var i alt 392.
\par 59 Følgende, som drog op fra Tel-Mela, Tel-Harsja, Kerub-Addan og Immer, kunde ikke opgive deres Fædrenehuse og Slægt, hvor vidt de hørte til Israel:
\par 60 Delajas, Tobijas og Nekodas Efterkommere 652.
\par 61 Og af Præsterne: Habaj as, Hakkoz's og Barzillajs Efterkommere; denne sidste havde ægtet en af Gileaditen Barzillajs Døtre og var blevet opkaldt efter dem.
\par 62 De ledte efter deres Slægtebøger, men kunde ikke finde dem, derfor blev de som urene udelukket fra Præstestanden.
\par 63 Statholderen forbød dem at spise af det højhellige, indtil der fremstod en Præst med Urim og Tummim.
\par 64 Hele Menigheden udgjorde 42360
\par 65 foruden deres Trælle og Trælkvinder, som udgjorde 7337, hvortil kom 200 Sangere og Sangerinder.
\par 66 Deres Heste udgjorde 736, deres Muldyr 245,
\par 67 deres Kameler 435 og deres Æsler 6720.
\par 68 Af fædrenehusenes Overhoveder gav nogle, da de kom til HERRENs Hus i Jerusalem, frivillige Gaver til Guds Hus, for at det kunde genopbygges på sin Plads;
\par 69 de gav efter deres Evne til Byggesummen 61000 Drakmer Guld, 5000 Miner Sølv og 100 Præstekjortler.
\par 70 Derpå bosatte Præsterne, Leviterne og en Del al Folket sig i Jerusalem og dets Område, men Sangerne, Dørvogterne og Tempeltrællene og hele det øvrige Israel i deres Byer.

\chapter{3}

\par 1 Da den syvende Måned indtraf - Israeliterne boede nu i deres Byer - samledes Folket fuldtalligt i Jerusalem;
\par 2 og Jesua, Jozadaks Søn, og hans Brødre Præsterne og Zerubbabel, Sjealtiels Søn, og hans Brødre skred til at bygge Israels Guds Alter for at ofre Brændofre derpå som foreskrevet i den Guds Mand Moses's Lov.
\par 3 En Del af Hedningerne samlede sig imod dem, men de rejste dog Alteret på dets gamle Plads og ofrede Brændofre derpå til HERREN, Morgen- og Aftenbrændofre.
\par 4 Derpå fejrede de Løvhyttefesten som foreskrevet og ofrede Brændofre Dag for Dag i det rette Tal og på den foreskrevne Måde, hver Dag hvad der hørte sig til,
\par 5 og siden det daglige Brændoffer og de Brændofre, som hørte til Nymånerne og alle HERRENs hellige Højtider, og alle de Brændofre, man frivilligt bragte HERREN.
\par 6 Den første Dag i den syvende Måned begyndte de at ofre Brændofre til HERREN, før Grunden til HERRENs Helligdom endnu var lagt.
\par 7 Derpå gav de Stenhuggerne og Tømmermændene Penge og Zidonierne og Tyrierne Fødevarer, Drikkevarer og Olie, for at de skulde bringe Cederstammer fra Libanon til Havet ud for Jafo, efter den Fuldmagt, Perserkongen Kyros havde givet dem.
\par 8 I den anden Måned i det andet År efter deres Ankomst til Guds Hus i Jerusalem gjorde Zerubbabel, Sjealtiels Søn, og Jesua, Jozadaks Søn, sammen med alle deres Brødre, Præsterne og Leviterne, og alle dem, der var kommet fra Fangenskabet til Jerusalem, Begyndelsen, idet de satte Leviteme fra Tyveårsalderen og opefter til at lede Arbejdet med HERRENs Hus.
\par 9 Og Leviterne Jesua og hans Sønner og Brødre, Kadmiel og hans Sønner, Hodavjas Sønner og Henadads Sønner, deres Sønner og Brødre, trådte til i Endrægtighed for at føre Tilsyn med dem, der arbejdede på Guds Hus.
\par 10 Og da Bygningsmændene lagde Grunden til HERRENs Helligdom, stod Præsterne i Embedsdragt med Trompeter, og Leviterne, Asafs Efterkommere, med Cymbler for at lovprise HERREN efter Kong David af Israels Anordning;
\par 11 og de stemte i med Lov og Pris for HERREN med Ordene thi han er god, og hans Miskundhed mod Israel varer evindelig!" Og hele Folket brød ud i høj Jubel, idet de priste HERREN, fordi Grunden var lagt til HERRENs Hus.
\par 12 Men mange af Præsterne, Leviterne og Overhovederne for Fædrenehusene, de gamle, der havde set det første Tempel, græd højt, da de så Grunden blive lagt til dette Tempel, men mange var også de, der opløftede deres Røst med Jubel og Glæde,
\par 13 og man kunde ikke skelne Glædesjubelen fra Folkets Gråd; thi så højt var Folkefs Jubelråb, at det hørfes langt bort.

\chapter{4}

\par 1 Men da Judas og Benjamins Fjender hørte, at de, der havde været i Landflygtighed, byggede HERREN, Israels Gud, en Helligdom,
\par 2 henvendte de sig til Zerubbabel, Jesua og Overhovederne for Fædrenehusene og sagde til dem: Lad os være med til at bygge, thi vi søger eders Gud såvel som I, og ham har vi ofret til, siden Assyrerkongen Asarhaddon førte os herhen!
\par 3 Men Zerubbabel, Jesua og de andre Overhoveder for Israels Fædrenehuse svarede: I skal ikke være fælles med os om at bygge vor Gud et Hus, men vi vil være ene om at bygge for HERREN, Israels Gud, således som Kong Kyros, Perserkongen, pålagde os!"
\par 4 Så bragte Hedningerne i Landet Judas Folks Hænder til at synke og skræmmede dem fra at bygge;
\par 5 og de købte Folk til med deres Råd at modarbejde dem og bringe deres Planer til at strande; således gik det, så længe Perserkongen Kyros levede, lige til Perserkongen Darius's Regering.
\par 6 Under Ahasverus's Regering, i hans første Regeringstid, skrev de en Klage oer Judas og Jerusalems lodbyggere.
\par 7 I Artaxerxes's Dage affattede Bisjlam, Mitredat, Tabe'el og alle hans andre Embedsbrødre en Skrivelse til Perserkongen Artaxerxes. Skrivelsen var affattet på Aramaisk og oversat.
\par 8 Statholderen Rehum og Skriveren Sjimsjaj skrev et Brev mod Jerusalem til Kong Artaxerxes af følgende Indhold.
\par 9 De, der dengang skrev, var Statholderen Rehum og Skriveren Sjimsjaj og alle deres andre Embedsbrødre, Diniterne, Afarsatkiterne, Tarpeliterne, Afaresiterne, Arkiterne, Babylonerne,
\par 10 Susaniterne, Dehaviterne,Elamiterne og de andre Folk, som den store og navnkundige Asenappar havde ført bort og ladet bosætte sig i Samarias Byer og andensteds hinsides Floden, og så videre.
\par 11 Dette er en Afskrift af Brevet, de sendte ham: "Til Kong Artaxerxes. Dine Trælle, Folkene hinsides Floden, og så videre:
\par 12 Det være Kongen kundgjort, at Jøderne, som drog op til os fra dig, er kommet til Jerusalem; de er i Færd med at genopbygge denne oprørske og onde By; de genopfører Murene og udbedrer Grunden.
\par 13 Men nu være det Kongen kundgjort, at hvis denne By bygges op og Murene genopføres, så vil de ikke svare Skat, Afgift eller Skyld, og der bliver Skår i Kongens Indtægter.
\par 14 Da vi nu spiser Paladsets Salt, og det ikke sømmer sig for os at se på, at Kongen lider Skade, sender vi herved Bud og lader Kongen det vide,
\par 15 for at der kan blive set efter i dine Fædres Krønikebog; i den vil du finde og se, at denne By er en oprørsk By, der har voldt Konger og Lande Skade, og at der fra gammel Tid har fundet Opstande Sted i den. Det er Grunden til, at denne By blev ødelagt.
\par 16 Så lader vi da Kongen vide, at hvis denne By bygges op og Murerne genopføres, har du ikke mere nogen Besiddelse hinsides Floden!"
\par 17 Kongen sendte da følgende Svar til Statholderen Rehum, Skriveren Sjimsjaj og alle deres andre Embeddsbrødre, som boede i Samaria og de andre Lande hinsides Floden: "Hilsen, og så videre.
\par 18 Den Skrivelse, I har sendt mig, er forelæst mig grundigt.
\par 19 Og på mit Bud har man set efter og fundet, at denne By fra gammel Tid har sat sig op mod Konger, og at der har fundet Oprør og Opstande Sted i den;
\par 20 over Jerusalem har der hersket mægtige Konger, som udstrakte deres Magt over alt hinsides Floden, og til hvem der svaredes Skat, Afgift og Skyld.
\par 21 Giv derfor Ordre til at standse disse Mænd og til, at denne By ikke må genopbygges, før der kommer Befaling fra mig;
\par 22 og tag eder vel i Vare for at vise Forsømmelighed i denne Sag, at ikke der skal lides store Tab til Skade for Kongerne!
\par 23 Så snart Afskriften af denne Skrivelse fra Kong Artaxerxes var blevet læst for Rehum, Skriveren Sjimsjaj og deres Embedsbrødre, begav de sig uopholdelig til Jøderne i Jerusalem og tvang dem med Magt til at standse Arbejdet.
\par 24 Så standsede Arbejdet på Guds Hus i Jerusalem, og det hvilede til Perserkongen Darius's andet Regeringsår.

\chapter{5}

\par 1 Men Profeterne Haggaj og Zakarias, Iddos Søn, profeterede for Jøderne i Juda og Jerusalem i Israels Guds Navn, som var over dem.
\par 2 Da tog Zerubbabel, Sjealtiels Søn, og Jesua, Jozadaks Søn, fat og begyndte at bygge på Guds Hus i Jerusalem sammen med Guds Profeter, som støttede dem.
\par 3 Men på den Tid kom Tattenaj, Statholderen hinsides Floden, Sjetar-Bozenaj og deres Embedsbrødre til dem og sagde: "Hvem har givet eder Lov til at bygge dette Tempel og genopføre denne Helligdom,
\par 4 og hvad er Navnene på de Mænd, der bygger denne Bygning?
\par 5 Men over Jødernes Ældste vågede deres Guds Øje, så de ikke standsede dem i Arbejdet, før Sagen var forelagt Darius og der var kommet Svar derpå.
\par 6 Afskrift af det Brev, som Tattenaj, Statholderen hinsides Floden, Sjetar-Bozenaj og hans Embedsbrødre, Afarsekiterne hinsides Floden, sendte Kong Darius;
\par 7 de sendte ham en Skrivelse, hvori der stod: Kong Darius ønsker vi al Fred!
\par 8 Det være Kongen kundgjort, at vi begav os til Landsdelen Judæa til den store Guds Hus; det bliver bygget af Kvadersten, der lægges Bjælker i Muren, og Arbejdet udføres med Omhu og skyder frem under deres Hænder.
\par 9 Vi spurgte da de Ældste der og talte således til dem: "Hvem har givet eder Lov til at, bygge dette Tempel og opføre denne Helligdom?
\par 10 Vi spurgte dem også om deres Navne for at lade dig dem vide, og vi opskrev Navnene på de Mænd, der står i Spidsen for dem.
\par 11 Og Svaret, de gav os, lød således: Vi er Himmelens og Jordens Guds Tjenere, og vi bygger det Tempel, som blev bygget for mange År siden, da en stor Konge i Israel byggede og opførte det.
\par 12 Da imidlertid vore Fædre vakte Himmelens Guds Vrede, gav han dem i Babels Konges, Kaldæeren Nebukadnezars, Hånd, og han nedbrød dette Tempel og førte Folket i Landflygtighed til Babel.
\par 13 Men i sit første Regeringsår gav Kong Kyros af Babel Befaling til at genopbygge dette Gudshus;
\par 14 og Kong Kyros lod tillige de til Gudshusef hørende Guld- og Sølvkar, som Nebukadnezar havde borttaget fra Helligdommen i Jerusalem og ført til sin Helligdom i Babel, tage ud af Helligdommen i Babel, og de overgaves til en Mand ved Navn Sjesjbazzar, som han havde indsat til Statholder;
\par 15 og han sagde til ham: "Tag disse Kar og drag ben og lad dem få deres Plads i Helligdommen i Jerusalem og lad Gudshuset blive genopbygget på sin gamle Plads!"
\par 16 Så kom denne Sjesjbazzar og lagde Grunden til Gudshuset i Jerusalem, og siden den Tid er der bygget derpå, men det er ikke færdigt.
\par 17 Hvis derfor Kongen synes, så lad der blive set efter i det kongelige Skatkammer ovre i Babel, om det har sig således, at der af Kong Kyros er givet Befaling til at bygge dette Gudshus i Jerusalem; og Kongen give os så sin Vilje i denne Sag til Kende!"

\chapter{6}

\par 1 Så gave Kong Darius Befaling til at se efter i Skatkammeret, hvor man i Babel gemte Dokumenterne;
\par 2 og man fandt da i Borgen i Ameta i Landsdelen Medien en Skriftrulle, hvori der stod: "Til Ihukommelse.
\par 3 I sit første Regeringsår udstedte Kong Kyros følgende Befaling: Gudsbuset i Jerusalem skal genopbygges, for at man der kan ofre Slagtofre og frembære Guds Ildofre; det skal være tresindstyve Alen højt og tresindstyve Alen bredt
\par 4 med tre Lag Kvadersten og eet Lag Bjælker; Omkostningerne udredes af Kongens Hus.
\par 5 Desuden skal Gudshusets Guld- og Sølvkar, som Nebukadnezar borttog fra Helligdommen i Jerusalem og førte til Babel, gives tilbage, og de skal bringes tilbage til deres Plads i Helligdommen i Jerusalem, og du skal sætte dem ind i Gudshuset!
\par 6 Derfor skal I, Tattenaj, Statholder hinsides Floden, og SjetarBozenaj med eders Emhedsbrødre, Afarsekiterne hinsides Floden, ikke blande eder deri.
\par 7 Lad Arbejdet med dette Gudshus gå sin Gang, lad Jødernes Statholder og Jødernes Ældste bygge dette Gudshus på den gamle Plads.
\par 8 Og hermed giver jeg Påbud om, hvorledes I skal stille eder over for disse Jødernes Ældste med Hensyn til Opførelsen af dette Gudshus: Af Kongens Skatteindtægter fra Landene hinsides Floden skal Omkostningerne nøjagtigt udredes til disse Mænd, og det ufortøvet;
\par 9 og hvad der ellers er Brug for: Tyre, Vædre og Lam til Brændofre for Himmelens Gud, Hvede, Salt, Vin og Olie, det skal efter Opgivende af Præsterne i Jerusalem udleveres dem Dag for Dag uden Afkortning,
\par 10 for at de kan bringe Ofre til en liflig Duft for Himmelens Gud og bede for Kongens og hans Sønners Liv.
\par 11 Og hermed påbyder jeg, at om nogen overtræder denne For ordning, skal en Bjælke rives ud af hans Hus, og til Straf skal han hænges op og nagles fast på den, og hans Hus skal gøres til en Grusdynge.
\par 12 Og den Gud, der har ladet sit Navn bo der, han slå enhver Kon ge og ethvert Folk til Jorden, som rækker Hånden ud for at over træde denne Forordning og øde lægge dette Gudshus i Jerusalem. Jeg, Darius, giver dette Påbud; lad det blive nøje udført!
\par 13 Da handlede Tattenaj, Statholderen hinsides Floden, Sjetar-Bozenaj og deres Embedsbrødre nøje efter det Påbud, Kong Darius havde sendt dem.
\par 14 Og Jødemes Ældste byggede, og det lykkedes dem i Henhold til Profeterne Haggajs og Zakarias's, Iddos Søns, Profeti; de byggede og fuldførte Værket efter Israels Guds Bud og efter Kyros's og Darius's og Perserkongen Aetaxerxes's Befaling.
\par 15 De fuldendte Templet på den tredje Dag i Adar Måned i Kong Darius's sjette Regeringsår.
\par 16 Så fejrede Israeliterne, Præsterne, Leviterne og de andre, der havde været i Landflygtighed, Gudshusets Indvielse med Glæde;
\par 17 og de ofrede ved Indvielsen 100 Tyre, 200 Vædre og 400 Lam og til Syndofre for hele Israel 12 Gedebukke efter Tallet på Israels Stammer;
\par 18 og de indsatte Præsterne efter deres Afdelinger og Leviterne efter deres Skifter til Gudstjenesten i Jerusalem som foreskrevet i Moses's Bog.
\par 19 Derpå fejrede de, der havde været i Landflygtighed, Påsken den fjortende Dag i den første Måned.
\par 20 Thi Præsterne og Leviterne havde renset sig og var rene alle som een; og de slagtede Påskelam for alle dem, der havde været i Landflygtighed, for deres Brødre Præsterne og for sig selv.
\par 21 Og Israeliterne, der var vendt tilbage fra Landflygtigheden, spiste deraf sammen med alle dem, der havde udskilt sig fra Hedningerne i Landet og deres Urenhed og sluttet sig til dem for at søge HERREN, Israels Gud.
\par 22 Og de fejrede de usyrede Brøds Højtid i syv Dage med Glæde, fordi HERREN havde glædet dem og vendt Assyrerkongens Hjerte til dem, så at han styrkede deres Hænder i Arbejdet på Guds, Israels Guds, Hus.

\chapter{7}

\par 1 Efter disse Tildragelser drog under Perserkongen Artaxerxes's Regering Ezra, en Søn af Seraja, en Søn af Azarja, en Søn af Hilkija,
\par 2 en Søn af Sjallum, en Søn af Zadok, en Søn af Ahitub,
\par 3 en Søn af Amarja, en Søn af Azarja, en Søn af Merajot,
\par 4 en Søn af Zeraja, en Søn af Uzzi, en Søn af Bukki,
\par 5 en Søn af Abisjua, en Søn af Pinehas, en Søn af Eleazar, en Søn af Ypperstepræsten Aron -
\par 6 denne Ezra drog op fra Babel. Han var skriftlærd, hjemme i Mose Lov, som HERREN, Israels Gud, havde givet; og Kongen opfyldte alle hans Ønsker, eftersom HERREN hans Guds Hånd var over ham.
\par 7 Og en Del af Israeliterne og at Præsterne, Leviterne Tempelsangerne, Dørvogterne og Tempeltrællene drog ligeledes op til Jerusalem i Kong Artaxerxes's syvende Regeringsår.
\par 8 De kom til Jerusalem i den femte Måned i Kongens syvende Regeringsår;
\par 9 thi på den første Dag i den første Måned tog han Bestemmelse om Opbruddet fra Babel, og på den første Dag i den femte Måned kom han til Jerusalem, eftersom hans Guds gode Hånd var over ham.
\par 10 Thi Ezra havde vendt sit Hjerte til at granske i HERRNs Lov og handle efter den og undervise Israel i Lov og Ret.
\par 11 Dette er en Afskrift, af den Skrivelse, Kong Artaxerxes medgav Præsten Ezra den Skriftlærde, den skriftlærde Kender af Bøgerne med HERRENs Bud og Anordninger til Israel:
\par 12 Artaxerxes, Kongernes Konge, til Præsten Ezra, den skriftlærde Kender af Himmelens Guds Lov, og så videre:
\par 13 Hermed giver jeg Tilladelse til, at enhver af Israels Folk og dets Præster og Leviter i mit Rige, der er til Sinds at drage til Jerusalem, må drage med dig,
\par 14 al den Stund du af Kongen og hans syv Rådgivere sendes for at undersøge Forholdene i Judæa og Jerusalem på Grundlag af din Guds Lov, som er i din Hånd,
\par 15 og for at bringe det Sølv og Guld derhen, som Kongen og hans Rådgivere frivilligt har givet Israels Gud, hvis Bolig er i Jerusalem,
\par 16 og alt det Sølv og Guld, som du får rundt om i Landsdelen Babel, tillige med de frivillige Gaver fra Folket og Præsterne, der giver frivillige Gaver til deres Guds Hus i Jerusalem.
\par 17 Derfor skal du samvittighedstuldt for disse Penge købe Tyre, Vædre og Lam med tilhørende Afgrøde- og Drikofre og ofre dem på Alteret i eders Guds Hus i Jerusalem;
\par 18 og hvad du og dine Brødre finder for godt at gøre med det Sølv og Guld, der bliver tilovers, det må I gøre efter eders Guds Vilje.
\par 19 De Kar, der skænkes dig til Tjenesten i din Guds Hus, skal du afgive og stille for Israels Guds Åsyn i Jerusalem.
\par 20 Og de andre nødvendige Udgifter til din Guds Hus, som det tilfalder dig at udrede, må du udrede af det kongelige Skatkammer.
\par 21 Jeg, Kong Artaxerxes, giver hermed den Befaling til alle Skatmestre hinsides Floden: Alt, hvad Præsten Ezra, den skriftlærde Kender af Himmelens Guds Lov, kræver af eder, skal nøjagtigt ydes
\par 22 indtil 100 Sølvtalenter, 100 Kor Hvede, 100 Bat Vin, 100 Bat Olie og Salt i ubegrænset Mængde.
\par 23 Alt, hvad der er påbudt af Himmelens Gud, skal punktligt ydes til Himmelens Guds Hus, at der ikke skal komme Vrede over Kongens og hans Sønners Rige.
\par 24 Og det være eder kundgjort, at ingen har Ret til at pålægge nogen af Præsterne, Leviterne, Tempelsangerne, Dørvogterne, Tempeltrællene eller overhovedet nogen, der er sysselsat ved dette Guds Hus, Skat, Afgift eller Skyld!
\par 25 Men du, Ezra, skal i Kraft af Guds Visdom, som er i din Hånd, indsætte Dommere og Retsbetjente til at dømme alt Folket hinsides Floden, alle dem, som kender, din Guds Lov; og hvem der ikke kender den, skal I undervise deri.
\par 26 Og enhver, der ikke handler efter din Guds Lov og Kongens Lov, over ham skal der samvittighedsfuldt fældes Dom, være sig til Død, Landsforvisning, Pengebøde eller Fængsel.
\par 27 Lovet være HERREN, vore Fædres Gud, som indgav Kongen sådanne Tanker for at herliggøre HERRENs Hus i Jerusalem
\par 28 og vandt mig, Nåde hos Kongen og hans Rådgivere og alle Kongens mægtige Fyrster! Så fattede jeg da Mod, eftersom HERREN min Guds Hånd var over mig, og jeg samlede en Del Overhoveder af Israel til at drage op med mig.

\chapter{8}

\par 1 Følgende er de Overhoveder over Fædrenehusene og de i deres Slægtsfortegnelser opførte, som drog op med mig fra Babel under Kong Artaxerxes's Regering:
\par 2 Af Pinehas's Efterkommere Ger som; af Itamars Efterkommere Daniel; af Davids Efterkommere Hattusj,
\par 3 Sjekanjas Søn; af Par'osj's Efterkommere Zekarja, i hvis Slægtsfortegnelse der var opført 150 Mandspersoner;
\par 4 af Pahat-Moabs Efterkommere Eljoenaj, Zerajas Søn, med 200 Mandspersoner;
\par 5 af Zattus Efterkommere Sjekanja, Jahaziels Søn, med 300 Mandspersoner;
\par 6 af Adins Efterkommere Ebed. Jonatans Søn, med 50 Mandspersoner;
\par 7 af Elams Efterkommere Jesja'ja. Ataljas Søn, med 70 Mandspersoner;
\par 8 af Sjefatjas Efterkommere Zebadja, Nikaels Søn, med 80 Mandspersoner;
\par 9 af Joabs Efterkommere 'Obadja. Jehiels' Søn, med 218 Mandspersoner;
\par 10 af Banis Efterkommere Sjelomit, Josifjas Søn, med 160 Mandspersoner;
\par 11 af Bebajs Efterkommere Zekarja, Bebajs Søn, med 28 Mandspersoner;
\par 12 af Azgads Efterkommere Johanan, Hakkatans Søn, med 110 Mandspersoner;
\par 13 af Adonikams Efterkommere de sidst komne, nemlig Elifelet. Je'iel og Sjemaja, med 60 Mandspersoner;
\par 14 af Bigvajs Efterkommere Utaj og Zabud med 70 Mandspersoner.
\par 15 Og jeg samlede dem ved den Flod, der løber ad Ahava til, og vi lå lejret der i tre Dage. Men da jeg tog Folket og Præsterne nærmere i Øjesyn, fandt jeg ingen Leviter der.
\par 16 Jeg sendte derfor Overhovederne Eliezer, Ariel, Sjemaja, Elnatan, Jarib, Elnatan, Natan, Zekarja og Mesjullam og Lærerne Jojarib og Elnatan hen
\par 17 og bød dem gå til Overhovedet Iddo i Byen Kasifja, idet jeg lagde dem de Ord i Munden, hvormed de skulde overtale Iddo og hans Brødre i Byen Kasifja til at sende os Tjenere til vor Guds Hus;
\par 18 og eftersom vor Guds gode Hånd var over os, sendte de os en forstandig Mand afEfterkommerne efter Mali, Israels Søn Levis Søn, Sjerebja med hans Sønner og Brødre, atten Mand
\par 19 og Hasjabja og Jesja'ja af Meraris Efterkommere med deres Brødre og Sønner, tyve Mand,
\par 20 og af Tempeltrællene, som David og Øversterne havde stillet til Leviternes Tjeneste, 220 Tempeltrælle, alle med Navns Nævnelse.
\par 21 Så lod jeg der ved Floden Ahava udråbe en Faste til Ydmygelse for vor Guds Åsyn for hos ham at udvirke en lykkelig Rejse for os og vore Familier og Ejendele;
\par 22 thi jeg undså mig ved at bede Kongen om Krigsfolk og Ryttere til at hjælpe os undervejs mod Fjenden, eftersom vi havde sagt til Kongen: Vor Guds Hånd er over alle; der søger ham, og hjælper dem, men hans Vælde og Vrede kommer over alle dem, der forlader ham.
\par 23 Så fastede vi og bad til vor Gud derom, og han bønbørte os.
\par 24 Derpå udvalgte jeg tolv af Præsternes Øverster og Sjerebja og Hasjabja og ti af deres Brødre;
\par 25 og dem tilvejede jeg Sølvet og Guldet og gav dem Karrene, den Offerydelse til vor Guds Hus, som Kongen, hans Rådgivere og Fyrster og alle de der boende Israeliter havde ydet;
\par 26 jeg tilvejede dem af Sølv 650 Talenter, Sølvkar til en Værdi af 100 Talenter, af Guld 100 Talenter,
\par 27 tyve Guldbægre til 1000 Darejker og to Kar af fint, guldglinsende Kobber, kostbare som Guld.
\par 28 Så sagde jeg til dem: I er helliget HERREN, og Karrene er helliget, og Sølvet og Guldet er en frivillig Gave til HERREN, eders Fædres Gud;
\par 29 våg derfor over det og vogt på det, indtil I i Påsyn af Præsternes og Leviternes Øverster og Israels Fædrenehuses Øverster vejer det ud i Jerusalem i Kamrene i HERRENs Hus!"
\par 30 Da modtog Præsterne og Leviterne det tilvejede, Sølvet og Guldet og Karrene, for at bringe det til vor Guds Hus i Jerusalem.
\par 31 Så brød vi op fra Floden Ahava på den tolvte Dag i den første Måned for at drage til Jerusalem; og vor Guds Hånd var over os, så han frelste os fra Fjendernes og Røvernes Hånd undervejs.
\par 32 Da vi var kommet til Jerusalem, holdt vi os rolige der i tre Dage;
\par 33 og på den fjerde Dag blev Sølvet, Guldet og Karrene afvejet i vor Guds Hus og overgivet Præsten Meremot, Urijas Søn, tillige med El'azar, Pinehas's Søn, og Leviterne Jozabad, Jesuas Søn, og Noadja, Binnujs Søn,
\par 34 alt sammen efter Tal og Vægt, og hele Vægten blev optegnet: På samme Tid
\par 35 bragte de, der kom fra Fangenskabet, de, der havde været i Landflygtighed, Brændofre til Israels Gud: 12 Tyre for hele tsrael, 96 Vædre, 77 Lam og 12 Gedebukke til Syndofre, alt sammen som Brændoffer til HERREN.
\par 36 Og de overgav Kongens Befalinger til Kongens Satraper og Statholderne hinsides Floden, og disse ydede Folket og Guds Hus deres Hjælp.

\chapter{9}

\par 1 Men da alt dette var gjort, kom Øversterne til mig og sagde: "Folket, Israel og Præsterne og Leviterne, har ikke skilt sig ud fra Hedningerne eller fra deres Vederstyggeligheder, Kana'anæerne, Hetiterne, Perizziterne, Jebusiterne, Ammoniterne, Moabiterne, Ægypterne og Amoriterne;
\par 2 thi af deres Døtre har de taget sig selv og deres Sønner Hustruer, så at den hellige Sæd har blandet sig med Hedningerne; og Øversterne og Forstanderne var de første til at øve denne Troløshed!
\par 3 Da jeg hørte den Tale, sønderrev jeg min Kjortel og min Kappe, rev Hår af mit Hoved og Skæg og satte mig hen i stum Smerte.
\par 4 Da samlede sig omkring mig alle de, der bævede for Israels Guds Ord mod Troløsheden hos dem, der havde været i Landflygtighed; og jeg sad i stum Smerte til Aftenafgrødeofferets Tid.
\par 5 Men ved Aftenafgrødeofferets Tid rejste jeg mig af min Selvydmygelse, og idet jeg sønderrev min Kjortel og min Kappe, kastede jeg mig på Knæ og udbredte Hænderne til, HERREN min Gud
\par 6 og sagde: Min Gud, jeg skammer mig og blues ved at løfte mit Ansigt til dig, min Gud, thi vore Misgerninger er vokset os over Hovedet, og vor Skyld er så stor, at den rækker til Himmelen!
\par 7 Fra vore Fædres Tid indtil denne Dag.har vor Skyld været stor, og for vore Misgerninger blev vi, vore Konger og Præster givet til Pris for Landenes Konger, for Sværd, Fangenskab, Udplyndring og Vanære, således som det er den Dag i Dag."
\par 8 Og nu er der en føje Stund blevet os Nåde til Del fra HERREN vor Gud, idet han har ladet os beholde en undsluppet Rest og givet os at slå vor Teltpæl på sit hellige Sted, for at vor Gud kan lade vore Øjne lyse og give os en Smule Livskraft i vor Trældom;
\par 9 thi er vi end Trælle, har vor Gud dog ikke forladt os i vor Trældom, men vundet os Nåde for Perserkongernes Åsyn, så at han har givet os Livskraft til at rejse vor Guds Hus og opbygge dets Grushobe og givet os et Gærde i Juda og Jerusalem.
\par 10 Men hvad skal vi nu sige, vor Gud, efter alt dette? Vi har jo forladt dine Bud,
\par 11 som du gav os ved dine Tjenere Profeterne, da du sagde: Det Land, I drager ind i og tager i Besiddelse, er et urent Land på Grund af Hedningernes Urenhed, på Grund af de Vederstyggeligheder, de i deres Urenhed har fyldt det med fra Ende til anden;
\par 12 derfor må I ikke give deres Sønner eders Døtre eller tage deres Døtre til Hustruer for eders Sønner og ingen Sinde søge deres Velfærd og Lykke, at I kan blive stærke og nyde Landets Goder og sikre eders Sønner Besiddelsen deraf for evigt!
\par 13 Efter alt, hvad der er vederfaret os på Grund af vore onde Gerninger og vor svare Skyld - og endda har du vor Gud ikke i fuldt Mål tilregnet os vore Synder, men skænket os en sådan Flok undslupne -
\par 14 skal vi da på ny krænke dine Bud ved at besvogre os med Folk, der øver slige Vederstyggeligheder? Vil du da ikke vredes således på os, at du ødelægger os aldeles, så der ikke levnes nogen Rest, og ingen undslipper?
\par 15 HERRE, Israels Gud! Du er retfærdig, derfor er vi nu en Rest tilbage, som er undsluppet; se, vi står for dig i vor Syndeskyld; thi det er umuligt at bestå for dit Åsyn, når sligt kan ske!"

\chapter{10}

\par 1 Medens Ezra nu under Bøn og Syndsbekendelse grædende kastede sig ned foran Guds Hus, samlede en stor Skare Israeliter sig om ham, både Mænd, Kvinder og Børn, thi Folket græd heftigt.
\par 2 Derpå tog Sjekanja, Jehiels Søn, af Elams Efterkommere til Orde og sagde til Ezra: Vi har været froløse mod vor Gud ved at hjemføre fremmede Kvinder af Hedningerne i Landet. Men trods alt er der endnu Håb for Israel.
\par 3 Lad os slutte Pagt for vor Gud om at sbille os af med alle vore fremmede Kvinder og deres Børn efter min Herres Bestemmelse og deres, som bæver for Guds Bud, og lad der blive handlet efter Loven!
\par 4 Stå op, thi det er dig, der skal tage dig af Sagen, og vi vil stå dig bi; vær frimodig og tag fat!
\par 5 Da stod Ezra op og tog Præsternes, Leviternes og hele Israels Øverster i Ed på, at de vilde handle således, og de aflagde Eden.
\par 6 Derpå rejste Ezra sig fra Pladsen foran Guds Hus og begav sig til Johanans, Eljasjibs Søns, Kammer, hvor han tilbragte Natten; han hverken spiste eller drak, fordi han græmmede sig over Troløsheden hos dem, der havde været i Landflygtighed.
\par 7 Derpå lod man kundgøre i Juda og Jerusalem forætlle dem, der havde været i Landflygtighed, at de skulde give Møde i Jerusalem;
\par 8 og enhver, som ikke indfandt sig Tredjedagen derefter ifølge Øversternes og de Ældstes Bestemmelse, al hans Ejendom skulde der lægges Band på, og han selv skulde udelukkes fra deres Forsamling, der havde været i Landflygtighed.
\par 9 Så samledes alle Mænd af Juda og Benjamin på Tredjedagen i Jerusalem; det var den tyvende Dag i den niende Måned; og alt Folket stillede sig op på den åbne Plads ved Guds Hus, skælvende både for Sagens Skyld og som Følge af Regnskyllene.
\par 10 Derpå stod Præsten Ezra op og sagde til dem: "I har forbrudt eder ved at hjemføre fremmede Kvinder og således øget Israels Syndeskyld;
\par 11 så bekend da nu eders Synd for HERREN, eders Fædres Gud, og gør hans Vilje; skil eder ud fra Hedningerne i Landet og fra de fremmede Kvinder!"
\par 12 Og hele Forsamlingen svarede med høj Røst: "Som du siger, bør vi gøre!
\par 13 Men Folket er talrigt, og det er Vinterregnens Tid, så vi kan ikke blive stående her ude; og Sagen kan heller ikke afgøres på en Dag eller to, da vi har forbrudt os højligen her.
\par 14 Lad derfor Øversterne for hele vor Forsamling give Møde og lad alle dem, der i vore Byer har hjemført fremmede Kvinder, indfinde sig til en fastsat Tid, ledsaget af de enkelte Byers Ældste og Dommere, for at vi kan blive friet fra vor Guds Vrede i denne Sag!
\par 15 Kun Jonatan, Asa'els Søn, og Jazeja, Tikvas Søn, satte sig derimod med Støtte fra Mesjullam og Leviten Sjabbetaj.
\par 16 Men de, der havde været i Landflygtighed, handlede derefter; og Præsten Ezra udvalgte sig nogle Mænd, Overhovederne for de enkelte Fædrenehuse, alle med Navns Nævnelse. Disse holdt da Møde den første Dag i den tiende Måned for at undersøge Sagen,
\par 17 og de var færdige med alle de Mænd, som havde bjemført fremmede Kvinder, til den første Dag i den første Måned.
\par 18 Blandt Præsterne fandtes følgende, der havde hjemført fremmede Kvinder: Af Jesuas, Jozadaks Søns, Efterkommere og hans Brødre Ma'aseja, Eliezer, Jarib og Gedalja;
\par 19 disse gav deres Hånd på at ville sende deres Hustruer bort, og deres Skyldoffer var en Væder for deres Syndeskyld.
\par 20 Af fmmers Efterkommere: Hanani og Zebadja.
\par 21 Af Harims Efterkommere: Ma'aseja, Elija, Sjemaja, Jehiel og Uzzija.
\par 22 Af Pasjhurs Efterkommere: Eljoenaj, Ma'aseja, Jisjmael, Netan'el, Jozabad og El'asa.
\par 23 Af Leviterne: Jozabad, Sjim'i. Kelaja, det er Kelita, Petaja, Juda og Eliezer.
\par 24 Af Tempelsangerne: Eljasjib og Zakkur. Af Dørvogterne Sjallum, Telem og Uri.
\par 25 Af Israel: Af Par'osj's Efterkommere: Ramja, Jizzija, Malkija, Mijjamin, El'azar, Malkija og Benaja.
\par 26 Af Elams Efterkommere: Mattanja, Zekarja, Jehiel, Abdi, Jeremot og Elija.
\par 27 Af Zattus Efterkommere:Fljoenaj, Eljasjib, Mattanja, Jeremot, Zåd og Aziza.
\par 28 Af Bebajs Efterkommere: Johanan, Hananja, Zabbaj og Atlaj.
\par 29 Af Banis Efterkommere: Mesjullam, Malluk, Adaja, Jasjub, Sjeal og Ramot.
\par 30 Af Pahat-Moabs Efterkommere: Adna, Kelal, Benaja, Ma'aseja, Mattanja, Bezal'el, Binnuj og Menassje.
\par 31 Af Harims Efterkommere: Eliezer, Jissjija, Malkija, Sjemaja, Sjim'on,
\par 32 Binjamin, Malluk og Sjemarja.
\par 33 Af Hasjums Efterkommere: Mattenaj, Mattatta, Zabad, Elifelef, Jeremaj, Menassje og Sjim'i.
\par 34 Af Banis Efterkommere: Ma'adaj, Amram, Uel,
\par 35 Benaja, Bedeja, Keluhu,
\par 36 Vanja, Meremot, Eljasjib,
\par 37 Mattanja, Mattenaj og Ja'asaj.
\par 38 Af Binnujs Efterkommere: Sjim'i,
\par 39 Sjelemja, Natan, Adaja,
\par 40 Maknadbaj, Sjasjaj, Sjaraj,
\par 41 Azar'el, Sjelemja, Sjemarja,
\par 42 Sjallum, Amarja og Josef.
\par 43 Af Nebos Efterkommere: Je'iel, Mattitja, Zåd, Zebina, Jaddaj, Joel og Benaja.
\par 44 Alle disse havde taget fremmede Kvinder til Ægte, men sendte nu Hustruer og Børn bort.


\end{document}