\begin{document}

\title{Filipperbrevet}


\chapter{1}

\par 1 Paulus og Timotheus, Kristi Jesu Tjener, til alle de hellige i Kristus Jesus, som ere i Filippi, med Tilsynsmænd og Menighedstjenere.
\par 2 Nåde være med eder og Fred fra Gud vor Fader og den Herre Jesus Kristus!
\par 3 Jeg takker min Gud, så ofte jeg kommer eder i Hu,
\par 4 idet jeg altid, i hver min Bøn, beder for eder alle med Glæde,
\par 5 for eders Deltagelse i Evangeliet fra den første Dag indtil nu;
\par 6 forvisset om dette, at han, som begyndte en god Gerning i eder, vil fuldføre den indtil Jesu Kristi Dag,
\par 7 således som det jo er ret for mig at mene dette om eder alle, efterdi jeg har eder i Hjertet både under mine Lænker og under Evangeliets Forsvar og Stadfæstelse, fælles som I jo alle ere med mig om Nåden.
\par 8 Thi Gud er mit Vidne, hvorledes jeg længes efter eder alle med Kristi Jesu inderlige Kærlighed.
\par 9 Og derom beder jeg, at eders Kærlighed fremdeles må blive mere og mere rig på Erkendelse og al Skønsomhed,
\par 10 så I kunne værdsætte de forskellige Ting, for at I må være rene og uden Anstød til Kristi Dag,
\par 11 fyldte med Retfærdigheds Frugt, som virkes ved Jesus Kristus, Gud til Ære og Pris.
\par 12 Men jeg vil, I skulle vide, Brødre! at mine Forhold snarere have tjent til Evangeliets Fremme,
\par 13 så at det er blevet åbenbart for hele Livvagten og for alle de øvrige, at mine Lænker bæres for Kristi Skyld,
\par 14 og de fleste af Brødrene fik i Tillid til Herren ved mine Lænker end mere Dristighed til at tale Guds Ord uden Frygt.
\par 15 Nogle prædike vel også Kristus for Avinds og Kivs Skyld, men nogle også i en god Mening.
\par 16 Disse gøre det af Kærlighed, vidende, at jeg er sat til at forsvare Evangeliet;
\par 17 men hine forkynde Kristus af Egennytte, ikke ærligt, men i den Tanke at føje Trængsel til mine Lænker.
\par 18 Hvad så? Kristus forkyndes dog på enhver Måde, være sig på Skrømt eller i Sandhed; og derover glæder jeg mig, og jeg vil også fremdeles glæde mig.
\par 19 Thi jeg ved, at dette skal blive mig til Frelse ved eders Bøn og Jesu Kristi Ånds Hjælp,
\par 20 efter min Længsel og mit Håb, at jeg i intet skal blive til Skamme, men at Kristus skal med al Frimodighed, som altid, så også nu, forherliges i mit Legeme, være sig ved Liv eller ved Død.
\par 21 Thi det at leve er mig Kristus og at dø en Vinding.
\par 22 Men dersom dette at leve i Kødet skaffer mig Frugt af min Gerning, så ved jeg ikke, hvad jeg skal vælge;
\par 23 men jeg står tvivlrådig imellem de to Ting, idet jeg har Lysten til at bryde op og være sammen med Kristus; thi dette var såre meget bedre;
\par 24 men at forblive i Kødet er mere nødvendigt for eders Skyld.
\par 25 Og i Forvisning herom ved jeg, at jeg skal blive i Live og forblive hos eder alle til eders Fremgang og Glæde i Troen,
\par 26 for at eders Ros ved mig kan blive rig i Kristus Jesus, ved at jeg atter kommer til Stede iblandt eder.
\par 27 Kun skulle I leve Kristi Evangelium værdigt, for at, hvad enten jeg kommer og ser eder eller er fraværende, jeg dog kan høre om eder, at I stå faste i een Ånd, så at I med een Sjæl stride tilsammen for Troen på Evangeliet
\par 28 og ikke lade eder forfærde i nogen Ting af Modstanderne; thi dette er for dem et Tegn på Undergang, men for eder på Frelse, og det fra Gud.
\par 29 Thi eder er det forundt for Kristi Skyld - ikke alene at tro på ham, men også at lide for hans Skyld,
\par 30 idet I have den samme Kamp, som I have set på mig og nu høre om mig.

\chapter{2}

\par 1 Er der da nogen Formaning i Kristus, er der nogen Kærlighedens Opmuntring, er der noget Åndens Samfund, er der nogen inderlig Kærlighed og Barmhjertighed:
\par 2 da fuldkommer min Glæde, at I må være enige indbyrdes, så I have den samme Kærlighed, samme Sjæl, een Higen,
\par 3 intet gøre af Egennytte eller Lyst til tom Ære, men i Ydmyghed agte hverandre højere end eder selv
\par 4 og ikke se hver på sit, men enhver også på andres.
\par 5 Det samme Sindelag være i eder, som også var i Kristus Jesus,
\par 6 han, som, da han var i Guds Skikkelse ikke holdt det for et Rov at være Gud lig,
\par 7 men forringede sig selv, idet han tog en Tjeners Skikkelse på og blev Mennesker lig;
\par 8 og da han i Fremtræden fandtes som et Menneske, fornedrede han sig selv, så han blev lydig indtil Døden, ja, Korsdøden.
\par 9 Derfor har også Gud højt ophøjet ham og skænket ham det Navn, som er over alle Navne,
\par 10 for at i Jesu Navn hvert Knæ skal bøje sig, deres i Himmelen og på Jorden og under Jorden,
\par 11 og hver Tunge skal bekende, at Jesus Kristus er Herre, til Gud Faders Ære.
\par 12 Derfor, mine elskede! ligesom I altid have været lydige, så arbejder ikke alene som i min Nærværelse, men nu meget mere i min Fraværelse på eders egen Frelse med Frygt og Bæven;
\par 13 thi Gud er den, som virker i eder både at ville og at virke, efter sit Velbehag.
\par 14 Gører alle Ting uden Knurren og Betænkeligheder,
\par 15 for at I må blive udadlelige og rene, Guds ulastelige Børn, midt i en vanartet og forvendt Slægt, iblandt hvilke I vise eder som Himmellys i Verden,
\par 16 idet I fremholde Livets Ord, mig til Ros på Kristi Dag, at jeg ikke har løbet forgæves, ej heller arbejdet forgæves.
\par 17 Ja, selv om jeg bliver ofret under Ofringen og Betjeningen af eders Tro, så glæder jeg mig og glæder mig med eder alle.
\par 18 Men ligeledes skulle også I glæde eder, og glæde eder med mig!
\par 19 Men jeg håber i den Herre Jesus snart at kunne sende Timotheus til eder, for at også jeg kan blive ved godt Mod ved at erfare, hvorledes det går eder.
\par 20 Thi jeg har ingen ligesindet, der så oprigtig vil have Omsorg for, hvorledes det går eder;
\par 21 thi de søge alle deres eget, ikke hvad der hører Kristus Jesus til.
\par 22 Men hans prøvede Troskab kende I, at, ligesom et Barn tjener sin Fader, således har han tjent med mig for Evangeliet.
\par 23 Ham håber jeg altså at sende straks, når jeg ser Udgangen på min Sag.
\par 24 Men jeg har den Tillid til Herren, at jeg også selv snart skal komme.
\par 25 Men jeg har agtet det nødvendigt at sende Epafroditus til eder, min Broder og Medarbejder og Medstrider, og eders Udsending og Tjener for min Trang,
\par 26 efterdi han længtes efter eder alle og var såre ængstelig, fordi I havde hørt, at han var bleven syg.
\par 27 Ja, han var også syg og Døden nær; men Gud forbarmede sig over ham, ja, ikke alene over ham, men også over mig, for at jeg ikke skulde have Sorg på Sorg.
\par 28 Derfor skynder jeg mig desto mere med at sende ham, for at I og jeg være mere sorgfri.
\par 29 Modtager ham altså i Herren med al Glæde og holder sådanne i Ære;
\par 30 thi for Kristi Gernings Skyld kom han Døden nær, idet han satte sit Liv i Vove for at udfylde Savnet af eder i eders Tjeneste imod mig.

\chapter{3}

\par 1 I øvrigt, mine Brødre glæder eder i Herren! At skrive det samme til eder er ikke til Besvær for mig, men er betryggende for eder.
\par 2 Holder Øje med Hundene, holder Øje med de slette Arbejdere, holder Øje med Sønderskærelsen!
\par 3 Thi vi ere Omskærelsen, vi, som tjene i Guds Ånd og rose os i Kristus Jesus og ikke forlade os på Kødet",
\par 4 endskønt også jeg har det, jeg kunde forlade mig på også i Kødet, Dersom nogen anden synes, han kan forlade sig på Kødet, kan jeg det mere.
\par 5 Jeg er omskåren på den ottende Dag, af Israels Slægt, Benjamins Stamme, en Hebræer af Hebræere, over for Loven en Farisæer,
\par 6 i Nidkærhed en Forfølger af Menigheden, i Retfærdigheden efter Loven udadlelig.
\par 7 Men hvad der var mig Vinding, det har jeg for Kristi Skyld agtet for Tab;
\par 8 ja sandelig, jeg agter endog alt for at være Tab imod det langt højere, at kende Kristus Jesus, min Herre, for hvis Skyld jeg har lidt Tab på alt og agter det for Skarn, for at jeg kan vinde Kristus
\par 9 og findes i ham, så jeg ikke har min Retfærdighed, den af Loven, men den ved Tro på Kristus, Retfærdigheden fra Gud på Grundlag af Troen,
\par 10 for at jeg må kende ham og hans Opstandelses Kraft og hans Lidelsers Samfund, idet jeg bliver ligedannet med hans Død,
\par 11 om jeg dog kunde nå til Opstandelsen fra de døde.
\par 12 Ikke at jeg allerede har grebet det eller allerede er fuldkommen; men jeg jager derefter, om jeg dog kunde gribe det, efterdi jeg også er greben af Kristus Jesus.
\par 13 Brødre! jeg mener ikke om mig selv, at jeg har grebet det.
\par 14 Men eet gør jeg: glemmende, hvad der er bagved, men rækkende efter det, som er foran, jager jeg imod Målet, til den Sejrspris, hvortil Gud fra det høje kaldte os i Kristus Jesus.
\par 15 Lader da os, så mange som ere fuldkomne, have dette Sindelag; og er der noget, hvori I ere anderledes sindede, da skal Gud åbenbare eder også dette.
\par 16 Kun at vi, så vidt vi ere komne, vandre i samme Retning.
\par 17 Vorder mine Efterlignere, Brødre! og agter på dem, der vandre således, som I have os til Forbillede.
\par 18 Thi mange vandre, som jeg ofte har sagt eder, men nu også siger med Tårer, som Kristi Kors's Fjender,
\par 19 hvis Ende er Fortabelse, hvis Gud er Bugen, og hvis Ære er i deres Skændsel, de, som tragte efter de jordiske Ting.
\par 20 Thi vort Borgerskab er i Himlene, hvorfra vi også forvente som Frelser den Herre Jesus Kristus,
\par 21 der skal forvandle vort Fornedrelses-Legeme til at blive ligedannet med hans Herligheds-Legeme, efter den Kraft, ved hvilken han også kan underlægge sig alle Ting.

\chapter{4}

\par 1 Derfor, mine Brødre, elskede og savnede, min Glæde og Krans! står således fast i Herren, I elskede!
\par 2 Evodia formaner jeg, og Syntyke formaner jeg til at være enige i Herren.
\par 3 Ja, jeg beder også dig, min ægte Synzygus! tag dig af dem; thi de have med mig stridt i Evangeliet, tillige med Klemens og mine øvrige Medarbejdere, hvis Navne stå i Livets Bog.
\par 4 Glæder eder i Herren altid; atter siger jeg: glæder eder!
\par 5 Eders milde Sind vorde kendt at alle Mennesker! Herren er nær!
\par 6 Værer ikke bekymrede for noget, men lader i alle Ting eders Begæringer komme frem for Gud i Påkaldelse og Bøn med Taksigelse;
\par 7 og Guds Fred, som overgår al Forstand, skal bevare eders Hjerter og eders Tanker i Kristus Jesus.
\par 8 I øvrigt, Brødre! alt, hvad der er sandt, hvad der er ærbart, hvad der er retfærdigt, hvad der er rent, hvad der er elskeligt, hvad der har godt Lov, enhver Dyd og enhver Hæder: lægger eder det på Sinde!
\par 9 Hvad I både have lært og modtaget og hørt og set på mig, dette skulle I gøre, og Fredens Gud skal være med eder.
\par 10 Men jeg har højlig glædet mig i Herren over, at I nu omsider ere komne til Kræfter, så at I kunne tænke på mit Vel, hvorpå I også forhen tænkte, men I manglede Lejlighed.
\par 11 Dette siger jeg ikke af Trang; thi jeg har lært at nøjes med det, jeg har.
\par 12 Jeg forstår at være i ringe Kår, og jeg forstår også at have Overflod; i alt og hvert er jeg indviet, både i at mættes og i at hungre, både i at have Overflod og i at lide Savn.
\par 13 Alt formår jeg i ham, som gør mig stærk.
\par 14 Dog gjorde I vel i at tage Del i min Trængsel.
\par 15 Men I vide det også selv, Filippensere! at i Evangeliets Begyndelse, da jeg drog ud fra Makedonien, var der ingen Menighed, som havde Regning med mig over givet og modtaget, uden I alene.
\par 16 Thi endog i Thessalonika sendte I mig både een og to Gange, hvad jeg havde nødig.
\par 17 Ikke at jeg attrår Gaven, men jeg attrår den Frugt, som bliver rigelig til eders Fordel.
\par 18 Nu har jeg nok af alt og har Overflod; jeg har fuldt op efter ved Epafroditus at have modtaget eders Gave, en Vellugts-Duft, et velkomment Offer, velbehageligt for Gud.
\par 19 Men min Gud skal efter sin Rigdom fuldelig give eder alt, hvad I have nødig, i Herlighed i Kristus Jesus.
\par 20 Men ham, vor Gud og Fader, være Æren i Evigheders Evigheder! Amen.
\par 21 Hilser hver hellig i Kristus Jesus.
\par 22 De Brødre, som ere hos mig, hilse eder. Alle de hellige hilse eder, men mest de af Kejserens Hus.
\par 23 Den Herres Jesu Kristi Nåde være med eders Ånd!



\end{document}