\begin{document}

\title{Prædikerens Bog}


\chapter{1}

\par 1 Ord af Prædikeren, Davids Søn, Konge i Jerusalem.
\par 2 Endeløs Tomhed, sagde Prædikeren, endeløs Tomhed, alt er Tomhed!
\par 3 Hvad Vinding har Mennesket af al den Flid, han gør sig under Solen?
\par 4 Slægt går, og Slægt kommer, men Jorden står til evig Tid.
\par 5 Sol står op, og Sol går ned og haster igen til sin Opgangs Sted.
\par 6 Vinden går mod Syd og drejer mod Nord, den drejer atter og atter og vender tilbage til samme Kredsløb.
\par 7 Alle Bække løber i Havet, men Havet bliver ikke fuldt; det Sted, til hvilket Bækkene løber, did bliver de ved at løbe.
\par 8 Alting slider sig træt; Mand hører ikke op med at tale, Øjet bliver ikke mæt af at se, Øret ej fuldt af at høre.
\par 9 Det, der kommer, er det, der var, det, der sker, er det, der skete; der er slet intet nyt under Solen.
\par 10 Kommer der noget, om hvilket man siger: "Se, her er da noget nyt!"det har dog for længst været til i Tiderne forud for os
\par 11 Ej mindes de svundne Slægter, og de ny, som kommer engang, skal ej heller mindes af dem, som kommer senere hen.
\par 12 Jeg, Prædikeren, var Konge over Israel i Jerusalem.
\par 13 Jeg vendte min Hu til at ransage og med Visdom udgranske alt, hvad der sker under Himmelen; det er et ondt Slid, som Gud har givet Menneskens Børn at slide med.
\par 14 Jeg så alt, hvad der sker under Solen, og se, det er alt sammen Tomhed og Jag efter Vind.
\par 15 Kroget kan ej blive lige, og halvt kan ej blive helt.
\par 16 Jeg tænkte ved mig selv: "Se, jeg har vundet større og rigere Visdom end alle de, der før mig var over Jerusalem, og mit Hjerte har skuet Visdom og Kundskab i Fylde."
\par 17 Jeg vendte min Hu til at fatte, hvad der er Visdom og Kundskab, og hvad der er Dårskab og Tåbelighed; jeg skønnede, at også det er Jag efter Vind.
\par 18 Thi megen Visdom megen Græmmelse, øget Kundskab øget Smerte.

\chapter{2}

\par 1 Jeg sagde ved mig selv: "Vel,jeg vil prøve med Glæde; så nyd da det gode!" Men se, også det var Tomhed.
\par 2 Om Latteren sagde jeg: "Dårskab!" og om Glæden: "Hvad gavner den?"
\par 3 Jeg kom på den Tanke at kvæge mit Legeme med Vin, medens mit Hjerte dog rådede med Visdom, og at slå mig på Dårskab, indtil jeg så, hvad det båder Menneskens Børn at gøre under Himmelen, det Dagetal de lever.
\par 4 Jeg fuldbyrdede store Værker, byggede mig Huse, plantede mig Vingårde,
\par 5 anlagde mig Haver og Lunde og plantede alle Hånde Frugttræer deri,
\par 6 anlagde mig Damme til at vande en Skov i Opvækst;
\par 7 jeg købte Trælle og Trælkvinder, og jeg havde hjemmefødte Trælle; også Kvæg, Hornkvæg og Småkvæg, havde jeg i større Måder end nogen af dem, der før mig havde været i Jerusalem;
\par 8 jeg samlede mig også Sølv og Guld, Skatte fra Konger og Lande; jeg tog mig Sangere og Sangerinder og Menneskens Børns Lyst: Hustru og Hustruer.
\par 9 Og jeg blev stor, større end nogen af dem, der før mig havde været i Jerusalem; desuden blev min Visdom hos mig.
\par 10 Intet, som mine Øjne attråede, unddrog jeg dem; jeg nægtede ikke mit Hjerte nogen Glæde thi mit Hjerte havde Glæde af al min Flid, og deri lå Lønnen for al min Flid.
\par 11 Men da jeg overskuede alt, hvad mine Hænder havde virket, og den Flid, det havde kostet mig, se, da var det alt sammen Tomhed og Jag efter Vind, og der er ingen Vinding under Solen.
\par 12 Thi hvad gør det Menneske, som kommer efter Kongen? Det samme, som tilforn er gjort? Jeg gav mig da til at sammenligne Visdom med Dårskab og Tåbelighed.
\par 13 Jeg så, at Visdom har samme Fortrin for Tåbelighed som Lys for Mørke:
\par 14 Den vise har Øjne i Hovedet, men Tåben vandrer i Mørke. Men jeg skønnede også, at en og samme Skæbne rammer begge.
\par 15 Da sagde jeg ved mig selv: "Tåbens Skæbne rammer også mig; hvad har jeg da for, at jeg er blevet overvættes viis?" Og jeg sagde ved mig selv, at også det er Tomhed;
\par 16 thi den vises Minde er lige sålidt evigt som Tåbens, fordi nu engang alt glemmes i kommende Dage; ak! den vise må dø så godt som Tåben.
\par 17 Da blev jeg led ved Livet, thi ilde tyktes mig det, som sker under Solen; thi det er alt sammen Tomhed og Jag efter Vind.
\par 18 Og jeg blev led ved al den Flid, jeg, har gjort mig under Solen, fordi jeg må efterlade mit Værk til den, som kommer efter mig.
\par 19 Hvo ved, om det bliver en Vismand eller en Tåbe? Og dog skal han råde over alt, hvad jeg med Flid og Visdom vandt under Solen. Også det er Tomhed.
\par 20 Og jeg var ved at fortvivle over al den Flid, jeg har gjort mig under Solen;
\par 21 thi der har et Menneske gjort sig. Flid med Visdom, Kundskab og Dygtighed, og så må han overlade sit Eje til et Menneske, som ikke har lagt Flid derpå. Også det er Tomhed og et stort Onde.
\par 22 Thi hvad får et Menneske for al sin Flid og sit Hjertes Higen, som han gør sig Flid med under Solen?
\par 23 Alle hans Dage er jo Lidelse, og hans Slid er Græmmelse; end ikke om Natten finder hans Hjerte Hvile. Også det er Tomhed.
\par 24 Intet er bedre for et Menneske end at spise og drikke og give sin Sjæl gode Dage ved sin Flid. Og det skønnede jeg, at også det kommer fra Guds Hånd.
\par 25 Thi hvo kan spise eller drikke uden hans Vilje?
\par 26 Thi det Menneske, som er godt i hans Øjne, giver han Visdom, Kundskab og Glæde; men den, som synder, giver han Slid med at samle og ophobe for så at give det til en, som er god i Guds Øjne. Også det er Tomhed og Jag efter Vind.

\chapter{3}

\par 1 Alt har sin stund og hver en Ting under Himmelen sin Tid:
\par 2 Tid til at fødes og Tid til at dø, Tid til at plante og Tid til at rydde,
\par 3 Tid til at dræbe og Tid til at læge, Tid til at nedrive og Tid til at opbygge,
\par 4 Tid til at græde og Tid til at le, Tid til at sørge og Tid til at danse,
\par 5 Tid til at kaste Sten, og Tid til at sanke Sten, Tid til at favne og Tid til ikke at favne,
\par 6 Tid til at søge og Tid til at miste, Tid til at gemme og Tid til at bortkaste,
\par 7 Tid til at flænge og Tid til at sy, Tid til at tie og Tid til at tale,
\par 8 Tid til at elske og Tid til at hade, Tid til Krig og Tid til Fred.
\par 9 Hvad Løn for sin Flid har da den, der arbejder?
\par 10 Jeg så det Slid, som Gud har givet Menneskens Børn at slide med.
\par 11 Alt har han skabt smukt til rette Tid; også Evigheden har han lagt i deres Hjerte, dog således at Menneskene hverken fatter det første eller det sidste af, hvad Gud har virket.
\par 12 Jeg skønnede, at der ikke gives noget andet Gode for dem end at glæde sig og have det godt, sålænge de lever.
\par 13 Dog også det at spise og drikke og nyde det gode under al sin Flid er for hvert Menneske en Guds Gave.
\par 14 Jeg skønnede, at alt, hvad Gud virker, bliver evindelig, uden at noget kan føjes til eller tages fra; og således har Gud gjort det, for at man skal frygte for hans Åsyn.
\par 15 Hvad der sker, var allerede, og hvad der skal ske, har allerede været; Gud leder det svundne op.
\par 16 Fremdeles så jeg under Solen, at Gudløshed var på Rettens Sted og Gudløshed på Retfærds Sted.
\par 17 Jeg sagde ved mig selv: "Den retfærdige og den gudløse dømmer Gud; thi for hver en Ting og hver en Idræt har han fastsat en Tid.
\par 18 Jeg sagde ved mig selv: "Det er for Menneskenes Skyld, for at Gud kan prøve dem, og for at de selv kan se, at de er Dyr:"
\par 19 Thi Menneskers og Dyrs Skæbne er ens; som den ene dør, dør den anden, og en og samme Ånd har de alle; Mennesket har intet forud for Dyrene, thi alt er Tomhed.
\par 20 Alle går sammesteds hen, alle blev til af Muld, og alle vender tilbage til Mulden.
\par 21 Hvo ved, om Menneskenes Ånd stiger opad, og om Dyrenes Ånd farer nedad til Jorden?
\par 22 Således indså jeg, at intet er bedre for Mennesket end at glæde sig ved sin Gerning, thi det er hans Del; thi hvo kan bringe ham så vidt, at han kan se, hvad der kommer efter hans Død?

\chapter{4}

\par 1 Fremdeles så jeg al den Undertrykkelse, som sker under Solen; jeg så de undertryktes tårer og ingen trøstede dem; de led Vold af deres Undertrykkeres Hånd, og ingen trøstede dem.
\par 2 Da priste jeg de døde, som allerede er døde, lykkeligere end de levende, som endnu er i Live;
\par 3 men lykkeligere end begge den, som slet ikke er til, som ikke har set det onde, der sker under Solen.
\par 4 Og jeg så, at al Flid og alt dygtigt Arbejde udspringer af den enes Misundelse mod den anden. Også det er Tomhed og Jag efter Vind.
\par 5 Dåren lægger Hænderne i Skødet og æder sig selv op
\par 6 Bedre en Håndfuld Hvile end Hænderne fulde af Flid og Jag efter Vind.
\par 7 Og mere Tomhed så jeg under Solen.
\par 8 Mangen står alene og har ikke nogen ved sin Side, hverken Søn eller Broder, og dog er der ingen Ende på al hans Flid og hans Øje bliver ikke mæt af Rigdom. Men, for hvis Skyld gør jeg mig Flid og nægter mig enhver Nydelse? Også det, er Tomhed og ondt Slid.
\par 9 To er bedre faren end een, thi de får god Løn for deres Flid;
\par 10 hvis den ene falder, kan den anden rejse sin Fælle op. Men ve den ensomme! Thi falder han, er der ingen til at rejse ham op.
\par 11 Og når to ligger sammen, bliver de varme; men hvorledes kan den ensomme blive varm?
\par 12 Og når nogen kan overvælde den ensomme, så kan to stå sig imod ham; tretvundet Snor brister ikke i Hast.
\par 13 Bedre faren er en fattig Yngling, som er viis, end en gammel Konge, som er en Tåbe og ikke mere har Forstand til at lade sig råde.
\par 14 Thi hin gik ud af Fængselet for at blive Konge, skønt han var født i Fattigdom under den andens Regering.
\par 15 Jeg så alle, som levede og færdedes under Solen, stille sig ved den Ynglings Side, som skulde træde i Kongens Sted;
\par 16 der var ikke Tal på alle de Mennesker, han stod i Spidsen for; men heller ikke over ham glæder de senere Slægter sig; nej, også det er Tomhed og Jag efter Vind.
\par 17 Var din fod, når du går til Guds Hus! At komme for at høre er bedre, end at Dårer bringer Slagtoffer, thi de har ikke Forstand til andet end at gøre ondt.

\chapter{5}

\par 1 Lad ikke din Mund løbe eller dit Hjerte haste med at udtale et Ord for Guds Åsyn; thi Gud er i Himmelen og du på Jorden, derfor skal dine Ord være få.
\par 2 Thi meget Slid giver Drømme, og mange Ord giver Dåretale.
\par 3 Når du giver Gud et Løfte, så tøv ikke med at holde det! Thi der er ingen Glæde ved Dårer. Hvad du lover, skal du holde.
\par 4 Det er bedre, at du ikke lover, end at du lover uden at holde.
\par 5 Lad ikke din Mund bringe Skyld over dit Legeme og sig ikke til Guds Sendebud, at det var af Vanvare! Hvorfor skal Gud vredes over din Tale og nedbryde dine Hænders Værk?
\par 6 Thi af mange Drømme og Ord kommer mange Skuffelser; nej, frygt Gud!
\par 7 Når du ser den fattige undertrykt og Lov og Ret krænket på din Egn, så undre dig ikke over den Ting; thi på den høje vogter en højere, og andre endnu højere vogter på dem begge.
\par 8 Dog, en Fordel for et Land er det i alt Fald, at der er en Konge over dyrket Jord.
\par 9 Den, der elsker Sølv, mættes aldrig af Sølv, og den, der elsker Rigdom, mættes aldrig af Vinding. Også det er Tomhed.
\par 10 Jo mere Gods, des flere til at fortære det, og hvad Gavn har Ejeren da deraf, ud over at hans Øjne ser det?
\par 11 Sød er Arbejderens Søvn, hvad enten han har lidt eller meget at spise; men den riges Overflod giver ikke ham Lov til at sove.
\par 12 Der er et slemt Onde, som jeg så under Solen: Rigdom gemt hen af sin Ejermand til hans Ulykke;
\par 13 går Rigdommen tabt ved et Uheld, og han har avlet en Søn, så bliver der intet til ham.
\par 14 Som han udgik af sin Moders Liv, skal han atter gå bort, lige så nøgen som han kom, og ved sin Flid vinder han intet, han kan tage med sig.
\par 15 Også det er et slemt Onde: ganske som han kom, går han bort, og hvad Vinding har han så af, at han gør sig Flid hen i Vejret?
\par 16 Og dertil kommer et helt Liv i Mørke, Sorg og stor Kvide, Sygdom og Kummer.
\par 17 Se, hvad der efter mit Skøn er godt og smukt, det er at spise og drikke og nyde det gode under al den Flid, man gør sig under Solen, alle de Levedage Gud giver en; thi det er den Del, man har;
\par 18 og hver Gang Gud giver et Menneske Rigdom og Gods og sætter ham i Stand til at nyde det, og tage sin Del og glæde sig under sin Flid, da er det en Guds Gave;
\par 19 thi da tænker han ikke stort på sine Levedage, idet Gud lader ham være optaget af sit Hjertes Glæde.

\chapter{6}

\par 1 Der er et Onde, jeg så under Solen, og som tynger Menneskene hårdt:
\par 2 Når Gud giver en Mand Rigdom og Gods og Ære, så han intet savner af, hvad han ønsker, og Gud ikke sætter ham i Stand til at nyde det, men en fremmed nyder det, da er dette Tomhed og en slem Lidelse.
\par 3 Om en Mand avler hundrede Børn og lever mange År, så hans Levetid bliver lang, men hans Sjæl ikke mættes af Goder, så siger jeg dog, at et utidigt Foster er bedre faren end han;
\par 4 thi at det kommer, er Tomhed, og det går bort i Mørke, og i Mørke dølges dets Navn;
\par 5 og det har hverken set eller kendt Sol; det får end ikke en Grav; det hviler bedre end han.
\par 6 Om han så levede to Gange tusind År, men ikke skuede Lykke - mon ikke alle farer sammesteds hen?
\par 7 Al Menneskets Flid tjener hans Mund, og dog stilles hans Sult aldrig.
\par 8 Thi hvad har den vise forud for Tåben, hvad båder det den arme, der ved at vandre for de levendes Øjne?
\par 9 Bedre at se med sine Øjne end higende Attrå. Også det er Tomhed og Jag efter Vind.
\par 10 Hvad der bliver til er for længst nævnet ved Navn, og det vides i Forvejen, hvad et Menneske bliver til; det kan ikke gå i Rette med ham, der er den stærkeste.
\par 11 Thi jo flere Ord der bruges, des større bliver Tomheden, og hvad gavner de Mennesket?
\par 12 Thi hvo ved, hvad der båder et Menneske i Livet, det Tal af tomme Levedage han henlever som en Skygge? Thi hvo kan sige et Menneske, hvad der skal ske under Solen efter hans Død?

\chapter{7}

\par 1 Godt Navn er bedre end ypperlig Salve, Dødsdag bedre end Fødselsdag;
\par 2 bedre at gå til et Sørgehus end at gå til et Gildehus; thi hist er alle Menneskers Ende, og de levende bør tage det til Hjerte
\par 3 Bedre Græmmelse end Latter, thi er Minerne mørke, har Hjertet det godt.
\par 4 De vises Hjerte er i Sørgehuset. Tåbernes Hjerte i Glædeshuset.
\par 5 Bedre at høre på Vismands Skænd end at høre på Tåbers Sang.
\par 6 Som Tjørnekvistes Knitren under Gryden er Tåbers Latter; også det er Tomhed.
\par 7 Thi uredelig Vinding gør Vismand til Dåre, og Stikpenge ødelægger Hjertet.
\par 8 En Sags Udgang er bedre end dens Indgang, Tålmod er bedre end Hovmod.
\par 9 Vær ikke hastig i dit Sind til at græmmes, thi Græmmelse bor i Tåbers Bryst.
\par 10 Spørg ikke: "Hvoraf kommer det, at de gamle Dage var bedre end vore?" Thi således spørger du ikke med Visdom.
\par 11 Bedre er Visdom end Arv, en Fordel for dem, som skuer Solen;
\par 12 thi Visdom skygger, som Penge skygger, men Kundskabs Fortrin er dette, at Visdom holder sin Mand i Live.
\par 13 Se på Guds Værk! Hvo kan rette, hvad han har gjort kroget?
\par 14 Vær ved godt Mod på den gode Dag og indse på den onde Dag, at Gud skabte denne såvel som hin, for at Mennesket ikke skal finde noget efter sig.
\par 15 Begge Dele så jeg i mine tomme Dage: Der er retfærdige, som omkommer i deres Retfærdighed, og der er gudløse, som lever længe i deres Ondskab.
\par 16 Vær ikke alt for retfærdig og te dig ikke overvættes viis; hvorfor vil du ødelægge dig selv?
\par 17 Vær ikke alt for gudløs og vær ingen Dåre; hvorfor vil du dø i Utide?
\par 18 Det bedste er, at du fastholder det ene og ikke slipper det andet; thi den, der frygter Gud, vil undgå begge Farer.
\par 19 Visdom gør Vismand stærkere end ti Magthavere i Byen.
\par 20 Thi intet Menneske er så retfærdigt på Jorden, at han kun gør gode Gerninger og aldrig synder.
\par 21 Giv ikke Agt på alle de Ord, Folk siger, at du ikke skal høre din Træl forbande dig;
\par 22 thi du ved med dig selv, at også du mange Gange har forbandet andre.
\par 23 Alt dette ransagede jeg med Visdom; jeg tænkte: "Jeg vil vorde viis." Men Visdom holdt sig langt fra mig;
\par 24 Tingenes Grund er langt borte, så dyb, så dyb; hvem kan finde den?
\par 25 Jeg tog mig for at vende min Hu til Kundskab og Granskning og til at søge efter Visdom og sikker Viden og til at kende, at Gudløshed er Tåbelighed, Dårskab Vanvid.
\par 26 Og beskere end Døden fandt jeg Kvinden, thi hun er et Fangegarn; hendes Hjerfe er et Net og hendes Arme Lænker. Den, som er Gud kær, undslipper hende, men Synderen bliver hendes Fange.
\par 27 Se, det fandt jeg ud, sagde Prædikeren, ved at lægge det ene til det andet for at drage min Slutning.
\par 28 Hvad min Sjæl stadig søgte, men ikke fandt, er dette: Een Mand fandt jeg blandt tusind, men en Kvinde fandt jeg ikke i hele Flokken.
\par 29 Dog se, det fandt jeg, at Gud har skabt Menneskene, som de bør være; men de har så mange sære Ting for.

\chapter{8}

\par 1 Hvo er som den vise, og hvo ved at tyde en Ting? Visdom får et Menneskes Åsyn til at lyse, og Åsynets Hårdhed mildnes.
\par 2 Hold en Konges Bud! Men drejer det sig om en Gudsed, så forhast dig ikke.
\par 3 Gå bort fra hans Åsyn og bliv ikke stående, når Sagen står slet; thi han kan gøre alt, hvad han vil.
\par 4 Thi en Konges Ord er Magtsprog, og hvo kan sige til ham: "Hvad gør du?"
\par 5 Den, som holder Budet, skal ikke mærke til noget ondt, og Dommens Tid skal den vises Hjerte kende.
\par 6 Thi enhver Ting har sin Tid og sin Dom, men det er et tyngende Onde for Mennesket,
\par 7 at han ikke ved, hvad der vil ske; thi hvo kan sige ham hvorledes Fremtiden bliver?
\par 8 Som intet Menneske er Herre over Vinden, så han kan spærre den inde, er ingen Herre over Dødens Dag; Krig kan man ikke unddrage sig, og Gudløshed frier ikke sin Mand.
\par 9 Alt dette så jeg, idet jeg rettede min Tanke på hver en Idræt, som øves under Solen: Der er Tider, da det ene Menneske hersker over det andet til hans Ulykke.
\par 10 Ligeledes så jeg gudløse stedes til Hvile, medens de, som gjorde det rette, måtte gå bort fra det hellige Sted og glemtes i Byen. Også det er Tomhed.
\par 11 Fordi den onde Gerning ikke i Hast rammes af Dommen får Menneskenes Hjerte Mod til at gøre det onde,
\par 12 eftersom Synderen gør det onde fra første Færd og dog lever længe; men også ved jeg, at det skal gå dem godt, som frygter Gud, fordi de frygter for hans Åsyn,
\par 13 og at det ikke skal gå de gudløse godt, og at deres Levetid ikke skal længes som Skyggen, fordi de ikke frygter for Guds Åsyn.
\par 14 Der er en Tomhed, som forekommer på Jorden: at der findes retfærdige, hvem det går, som om de havde gjort de gudløses Gerninger, og gudløse, hvem det går som om de havde gjort de retfærdiges Gerninger. Jeg sagde: Også det er Tomhed.
\par 15 Og jeg priste Glæden, fordi Mennesket ikke har andet Gode under Solen end at spise og drikke og være glad, og at dette ledsager ham under hans Flid i de Levedage, Gud giver ham under Solen.
\par 16 Hver Gang jeg vendte min Hu til at nemme Visdom og granske det Slid, som går for sig på Jorden - thi hverken Dag eller Nat får man Søvn i Øjnene -
\par 17 da indså jeg, at det er således med alt Guds Værk, at Mennesket ikke kan udgrunde det, som sker under Solen; thi trods al den Flid, et Menneske gør sig med at søge, kan han ikke udgrunde det; og selv om den vise mener at kende det, kan han ikke udgrunde det.

\chapter{9}

\par 1 Ja, alt dette lagde jeg mig på Sinde, og mit hjerte indså det alt sammen: at de retfærdige og de vise og deres Gerninger er i Guds Hånd. Hverken om Kærlighed eller Had kan Menneskene vide noget; alt, hvad der er dem for Øje, er Tomhed.
\par 2 Thi alle får en og samme Skæbne, retfærdig og gudløs, god og ond, ren og uren, den, som ofrer, og den, som ikke ofrer; det går den gode som Synderen, den sværgende som den, der skyr at sværge.
\par 3 Det er det, der er Fejlen ved alt, hvad der sker under Solen, at alle får en og samme Skæbne; derfor er også Menneskebørnenes Hjerte fuldt af ondt, og der er Dårskab i deres Hjerte Livet igennem, og tilsidst må de ned til de døde.
\par 4 Kun for den, der hører til de levendes Flok, er der Håb; thi levende Hund er bedre faren end død Løve.
\par 5 Thi de levende ved dog, at de skal dø, men de døde ved ingenting, og Løn har de ikke mere i Vente; thi Mindet om dem slettes ud.
\par 6 Både deres Kærlighed og deres Had og deres Misundelse er forlængst borte, og de får ingen Sinde mere Lod og Del i noget af det, som sker under Solen.
\par 7 Så spis da dit Brød med Glæde, drik vel til Mode din Vin; thi din Id har Gud for længst kendt god.
\par 8 Dine Klæder være altid hvide, lad Olie ikke savnes på dit Hoved!
\par 9 Nyd Livet med den Kvinde, du elsker, alle dine tomme Levedage, som gives dig under Solen; thi det er din Lod og Del af Livet og af den Flid, du gør dig under Solen.
\par 10 Gør efter Evne alt, hvad din Hånd finder Styrke til; thi der er hverken Virke eller Tanke eller Kundskab eller Visdom i Dødsriget, hvor du stævner hen.
\par 11 Og atter så jeg under Solen, at Hurtigløberen ikke er Herre over Løbet eller Heltene over Kampen, ej heller de vise over Brødet, ej heller de kløgtige over Rigdom, ej heller de kloge over Yndest, men alle er de bundet af Tid og Tilfælde.
\par 12 Thi et Menneske kender lige så lidt sin Tid som Fisk, der fanges i det slemme Garn, eller Fugle, der hildes i Snaren; ligesom disse fanges Menneskens Børn i Ulykkens Stund, når den brat falder over dem.
\par 13 Også dette Tilfælde af Visdom så jeg under Solen, og det gjorde dybt Indtryk på mig:
\par 14 Der var en lille By med få Indbyggere, og mod den kom en stor Konge; han omringede den og byggede høje Volde imod den;
\par 15 men der fandtes i Byen en fattig Mand, som var viis, og han frelste den ved sin Visdom. Men ingen mindedes den fattige Mand.
\par 16 Da sagde jeg: "Visdom er bedre end Styrke, men den fattiges Visdom agtes ringe, og hans Ord høres ikke."
\par 17 Vismænds Ord, der høres i Ro, er bedre end en Herskers Råb iblandt Dårer.
\par 18 Visdom er bedre end Våben, men en eneste Synder kan ødelægge meget godt.

\chapter{10}

\par 1 Døde Fluer gør Salveblanderens Olie stinkende, lidt Dårskab ødelægger Visdommens Værd.
\par 2 Den vise har sin Forstand tilhøjre, Tåben har sin til venstre,
\par 3 Hvor Dåren end færdes, svigter hans Forstand, og han røber for alle, at han er en Dåre.
\par 4 Når en Herskers Vrede rejser sig mod dig, forlad ikke derfor din Plads; thi Sagtmodighed hindrer store Synder.
\par 5 Der er et Onde, jeg så under Solen; det ser ud som et Misgreb af ham, som har Magten:
\par 6 Dårskab sættes i Højsædet, nederst sidder de rige.
\par 7 Trælle så jeg højt til Hest og Høvdinger til Fods som Trælle.
\par 8 Den, som graver en Grav, falder selv deri; den, som nedbryder en Mur, ham bider en Slange;
\par 9 den, som bryder Sten, kan såre sig på dem; den, som kløver Træ, er i Fare.
\par 10 Når Øksen er sløv og dens Æg ej hvæsses, må Kraft lægges i; men den dygtiges Fortrin er Visdom.
\par 11 Bider en Slange, før den besværges, har Besværgeren ingen Gavn af sin Kunst.
\par 12 Ord fra Vismands Mund vinder Yndest, en Dåres Læber bringer ham Våde;
\par 13 hans Tale begynder med Dårskab og ender med den værste Galskab.
\par 14 Tåben bruger mange Ord. Ej ved Mennesket, hvad der skal ske; hvad der efter hans Død skal ske, hvo siger ham det?
\par 15 Dårens Flid gør ham træt, thi end ikke til Bys ved han Vej.
\par 16 Ve dig, du Land, hvis Konge er en Dreng og hvis Fyrster holder Gilde ved Gry.
\par 17 Held dig; du Land, hvis Konge er ædelbåren, hvis Fyrster holder Gilde til sømmelig Tid som Mænd og ikke som drankere.
\par 18 Ved Ladhed synker Bjælkelaget; når Hænderne slappes, drypper det i Huset.
\par 19 Til Morskab holder man Gæstebud, og Vin gør de levende glade; men Penge skaffer alt til Veje.
\par 20 End ikke i din Tanke må du bande en Konge, end ikke i dit Sovekammer en, som er rig; thi Himlens Fugle kan udsprede Ordet, de vingede røbe, hvad du siger.

\chapter{11}

\par 1 Kast dit Brød på Vandet, Thi du får det igen, går end lang Tid hen.
\par 2 Del dit Gods i syv otte Dele, thi du ved ej, hvad ondt der kanske på Jorden.
\par 3 Er Skyerne fulde af Regn, så gyder de den ud over Jorden; og falder et Træ mod Syd eller Nord; så bliver det liggende der, hvor det falder.
\par 4 Man får aldrig sået, når man kigger efter Vinden, og aldrig høstet, når man ser efter Skyerne.
\par 5 Som du ikke kender Vindens Vej eller Fostret i Moders Liv, så kender du ej heller Guds Virke, han, som virker alt.
\par 6 Så din Sæd ved Gry og lad Hånden ej hvile ved Kvæld; thi du ved ej, om dette eller hint vil lykkes, eller begge Dele er lige gode.
\par 7 Lyset er lifligt, at skue Solen er godt for Øjnene;
\par 8 ja, lever et Menneske mange År, skal han glæde sig over dem alle og komme Mørkets Dage i Hu, thi af dem er der mange i Vente; alt, hvad der kommer, er Tomhed.
\par 9 Glæd dig, Yngling, i din Ungdom, vær vel til Mode i Livets Vår; gå, hvor dit Hjerte lyster, og nyd, hvad dit Øje skuer; men vid, at for alle disse Ting skal du kræves til Regnskab af Gud.
\par 10 Slå Mismod ud af dit Sind, hold Sygdom fjernt fra din Krop; thi Ungdom og Livsgry er Tomhed!

\chapter{12}

\par 1 Tænk På ding Skaber i Ungdommens Dage, førend de onde Dage kommer og Årene nærmer sig, om hvilke du vil sige: "I dem har jeg ikke Behag!"
\par 2 før Sol og Lys og Måne og Stjerner hylles i Mørke og der atter kommer Skyer efter Regn,
\par 3 Tiden, da Husets Vogtere bæver, de stærke Mænd bliver krumme, da Møllepigerne svigter, fordi de er få, og de bliver mørke, som kigger ved Gluggerne,
\par 4 da begge Gadedørene lukkes, mens Møllen går med dæmpet Lyd, da man står op ved Spurvenes Kvidder og alle Sangens Døtre hvisker,
\par 5 da man også ængstes for Bakker, og Rædsler lurer på Vejen, da Mandeltræet blomstrer; Græshoppen slappes og Kapersbærret svigter, nu Mennesket går til sin evige Bolig og Sørgetoget går gennem Gaden,
\par 6 førend Sølvsnoren brister og Guldskålen brydes itu, før Krukken slås i Stykker ved Kilden og det søndrede Hjul falder ned i Brønden
\par 7 og Støvet vender tilbage til Jorden som før og Ånden til Gud, som gav den.
\par 8 Endeløs Tomhed, sagde Prædikeren, alt er Tomhed.
\par 9 Endnu skal siges, at Prædikeren var viis; han gav også Folket Kundskab; han granskede og ransagede og formede mange Ordsprog.
\par 10 Prædikeren søgte at finde Fyndord og optegnede sanddru Lære, Sandhedsord.
\par 11 Som Pigkæppe er de vises Ord, som inddrevne Søm, der sidder tæt; de er givet af en og samme Hyrde.
\par 12 Endnu skal siges: Min Søn, var dig! Der er ingen Ende på, som der skrives Bøger, og megen Gransken trætter Legemet.
\par 13 Enden på Sagen, når alt er hørt, er: Frygt Gud og hold hans Bud! Thi det bør hvert Menneske gøre.
\par 14 Thi hver en Gerning bringer Gud for Retten, når han dømmer alt, hvad der er skjult, være sig godt eller ondt.


\end{document}