\chapter{Documento 5. Las relaciones de Dios con los individuos}
\par
%\textsuperscript{(62.1)}
\textsuperscript{5:0.1} SI LA mente finita del hombre es incapaz de comprender cómo un Dios tan grande y tan majestuoso como el Padre Universal puede descender de su residencia eterna de perfección infinita para fraternizar con las criaturas humanas individuales, entonces ese intelecto finito debe basar su seguridad de comunión divina en la verdad del hecho de que un fragmento real del Dios viviente reside en el intelecto de cada mortal de Urantia provisto de una mente normal y de una conciencia moral. Los Ajustadores del Pensamiento interiores son una parte de la Deidad eterna del Padre Paradisiaco. El hombre no tiene necesidad de ir más allá de su propia experiencia interior, donde el alma contempla la presencia de esta realidad espiritual, para encontrar a Dios y tratar de comulgar con él.

\par
%\textsuperscript{(62.2)}
\textsuperscript{5:0.2} Dios ha distribuido la infinidad de su naturaleza eterna en todas las realidades existenciales de sus seis coordinados absolutos, pero en cualquier momento puede establecer un contacto directo y personal con cualquier parte, o fase, o tipo de creación por mediación de sus fragmentos prepersonales. Y el Dios eterno también se ha reservado la prerrogativa de conceder la personalidad a los Creadores divinos y a las criaturas vivientes del universo de universos, mientras que además se ha reservado la prerrogativa de mantener un contacto directo y paternal con todos estos seres personales a través del circuito de la personalidad.

\section*{1. El camino de acceso a Dios}
\par
%\textsuperscript{(62.3)}
\textsuperscript{5:1.1} La incapacidad de las criaturas finitas para acercarse al Padre infinito no es inherente a la actitud distante del Padre, sino a la finitud y a las limitaciones materiales de los seres creados. La magnitud de la diferencia espiritual entre la más alta personalidad que existe en el universo y los grupos inferiores de inteligencias creadas es inconcebible. Si a los tipos de inteligencias inferiores les fuera posible ser transportados instantáneamente ante la presencia del Padre mismo, no sabrían que se encuentran allí. Se hallarían allí tan inconscientes de la presencia del Padre Universal como donde se encuentran ahora. El hombre mortal tiene por delante un larguísimo camino antes de que pueda solicitar, de manera coherente y dentro de lo posible, un salvoconducto que le permita llegar ante la presencia paradisiaca del Padre Universal. El hombre ha de ser trasladado espiritualmente muchas veces antes de que pueda alcanzar un plano que le proporcione la visión espiritual adecuada para ver siquiera a uno solo de los Siete Espíritus Maestros.

\par
%\textsuperscript{(62.4)}
\textsuperscript{5:1.2} Nuestro Padre no se oculta; no se encuentra en un retiro arbitrario. Ha movilizado los recursos de su sabiduría divina en un esfuerzo sin fin por revelarse a los hijos de sus dominios universales. La majestad de su amor lleva unidas una grandeza infinita y una generosidad inexpresable que le inducen a anhelar asociarse con cada ser creado que pueda comprenderlo, amarlo o acercarse a él; por consiguiente, vuestras limitaciones inherentes, inseparables de vuestra personalidad finita y de vuestra existencia material, son las que determinan el momento, el lugar y las circunstancias en que podréis alcanzar la meta del viaje de la ascensión humana, y encontraros en la presencia del Padre en el centro de todas las cosas.

\par
%\textsuperscript{(63.1)}
\textsuperscript{5:1.3} Aunque para acercaros a la presencia del Padre en el Paraíso debéis esperar a haber alcanzado los niveles finitos más elevados de la progresión espiritual, deberíais regocijaros en el reconocimiento de la posibilidad siempre presente de poder comulgar inmediatamente con el espíritu otorgado por el Padre, tan íntimamente asociado con vuestra alma interior y con vuestro yo en vías de espiritualización.

\par
%\textsuperscript{(63.2)}
\textsuperscript{5:1.4} Los mortales de los mundos del tiempo y del espacio pueden diferir enormemente en sus capacidades innatas y en sus dones intelectuales, pueden disfrutar de entornos excepcionalmente favorables para el avance social y el progreso moral, o pueden sufrir la carencia de casi toda ayuda humana para cultivarse y avanzar supuestamente en las artes de la civilización; pero las posibilidades para el progreso espiritual en la carrera de la ascensión son iguales para todos; los niveles crecientes de perspicacia espiritual y de significados cósmicos se alcanzan con absoluta independencia de todos los diferenciales sociomorales de los entornos materiales diversificados de los mundos evolutivos.

\par
%\textsuperscript{(63.3)}
\textsuperscript{5:1.5} Por mucho que difieran los mortales de Urantia en sus oportunidades y en sus dones intelectuales, sociales, económicos e incluso morales, no olvidéis que su dotación espiritual es uniforme y única. Todos disfrutan de la misma presencia divina del don procedente del Padre, y todos gozan del mismo privilegio de poder buscar una íntima comunión personal con este espíritu interior de origen divino, mientras que todos pueden elegir igualmente aceptar las directrices espirituales uniformes de estos Monitores de Misterio.

\par
%\textsuperscript{(63.4)}
\textsuperscript{5:1.6} Si un hombre mortal está motivado de manera sincera y espiritual, consagrado sin reservas a hacer la voluntad del Padre, entonces, puesto que está dotado espiritualmente de forma tan cierta y tan eficaz de un Ajustador divino interior, no puede dejar de materializarse en la experiencia de ese individuo la conciencia sublime de conocer a Dios y la seguridad celestial de sobrevivir para encontrar a Dios mediante la experiencia progresiva de volverse cada vez más semejante a él.

\par
%\textsuperscript{(63.5)}
\textsuperscript{5:1.7} El hombre está habitado espiritualmente por un Ajustador del Pensamiento que sobrevive. Si esa mente humana está sincera y espiritualmente motivada, si ese alma humana desea conocer a Dios y volverse semejante a él, si quiere hacer honradamente la voluntad del Padre, no existe ninguna influencia negativa de privaciones mortales ni ningún auténtico poder de interferencia posible que pueda impedir a ese alma divinamente motivada ascender con toda seguridad hasta las puertas del Paraíso.

\par
%\textsuperscript{(63.6)}
\textsuperscript{5:1.8} El Padre desea que todas sus criaturas estén en comunión personal con él. Tiene un lugar en el Paraíso para recibir a todos aquellos cuyo estado de supervivencia y cuya naturaleza espiritual hagan posible esta consecución. Por lo tanto, inscribid en vuestra filosofía, ahora y para siempre, que: para cada uno de vosotros y para todos nosotros, Dios es accesible, el Padre es alcanzable, el camino está abierto; las fuerzas del amor divino y los medios de la administración divina están todos entrelazados en un esfuerzo por facilitar el progreso de todas las inteligencias dignas de todos los universos hasta la presencia del Padre Universal en el Paraíso.

\par
%\textsuperscript{(63.7)}
\textsuperscript{5:1.9} El hecho de que se necesite un tiempo considerable para alcanzar a Dios no hace menos real la presencia y la personalidad del Infinito. Vuestra ascensión es una parte del circuito de los siete superuniversos, y aunque dais la vuelta a su alrededor un número incontable de veces, podéis esperar, en espíritu y en estado, que avanzaréis siempre hacia el interior. Podéis contar con que seréis trasladados de esfera en esfera, desde los circuitos exteriores siempre acercándoos al centro interior, y algún día, no lo dudéis, os encontraréis ante la presencia divina y central, y la veréis, hablando en lenguaje figurado, cara a cara. Es una cuestión de alcanzar los niveles espirituales reales y tangibles; y estos niveles espirituales son accesibles para cualquier ser que haya sido habitado por un Monitor de Misterio, y que haya fusionado posteriormente de manera eterna con ese Ajustador del Pensamiento.

\par
%\textsuperscript{(64.1)}
\textsuperscript{5:1.10} El Padre no se encuentra en un escondite espiritual, pero muchas de sus criaturas se han escondido en las brumas de sus propias decisiones obstinadas, y por el momento se han separado de la comunión con su espíritu y con el espíritu de su Hijo porque han elegido sus propios caminos perversos y porque han dado rienda suelta a la presunción de sus mentes intolerantes y de sus naturalezas no espirituales.

\par
%\textsuperscript{(64.2)}
\textsuperscript{5:1.11} El hombre mortal puede acercarse a Dios y alejarse repetidas veces de la voluntad divina durante tanto tiempo como conserve su poder de elección. El destino final del hombre no se decide hasta que ha perdido el poder de elegir la voluntad del Padre. El Padre no cierra nunca su corazón a las necesidades y a las súplicas de sus hijos. Es su progenitura la que cierra su corazón para siempre al poder de atracción del Padre cuando pierde final y definitivamente el deseo de hacer su voluntad divina ---la de conocerle y ser semejante a él. El destino eterno del hombre está igualmente asegurado cuando su fusión con el Ajustador proclama al universo que este ascendente ha hecho la elección final e irrevocable de vivir la voluntad del Padre.

\par
%\textsuperscript{(64.3)}
\textsuperscript{5:1.12} El gran Dios se pone en contacto directo con el hombre mortal y le concede una parte de su yo infinito, eterno e incomprensible para que viva y resida dentro de él. Dios se ha embarcado en la aventura eterna con el hombre. Si os sometéis a las directrices de las fuerzas espirituales que están en vosotros y alrededor de vosotros, no podréis dejar de alcanzar el alto destino que un Dios amoroso ha establecido como meta universal para sus criaturas ascendentes de los mundos evolutivos del espacio.

\section*{2. La presencia de Dios}
\par
%\textsuperscript{(64.4)}
\textsuperscript{5:2.1} La presencia física del Infinito es la realidad del universo material. La presencia mental de la Deidad ha de estar determinada por la profundidad de la experiencia intelectual individual y por el nivel evolutivo de la personalidad. La presencia espiritual de la Divinidad debe ser forzosamente diferencial en el universo. Está determinada por la capacidad espiritual de receptividad y por el grado en que la voluntad de la criatura está consagrada a hacer la voluntad divina.

\par
%\textsuperscript{(64.5)}
\textsuperscript{5:2.2} Dios vive en cada uno de sus hijos nacidos del espíritu. Los Hijos Paradisiacos siempre tienen acceso a la presencia de Dios, «a la derecha del Padre»\footnote{\textit{A la derecha del Padre}: Sal 110:1; Mt 22:43-44; Mc 12:36; 16:19; Lc 20:42; Hch 7:55-56; Ro 8:34; Col 3:1; Heb 1:3; 8:1; 10:12; 12:2; 1 P 3:22.}, y todas las personalidades de sus criaturas tienen acceso al «seno del Padre»\footnote{\textit{Seno del Padre}: Jn 1:18.}. Esto se refiere al circuito de la personalidad, cuando, dónde y comoquiera que se contacte con él, o suponga por lo demás un contacto y una comunión personal y consciente con el Padre Universal, ya sea en su residencia central o en cualquier otro lugar designado, como por ejemplo una de las siete esferas sagradas del Paraíso.

\par
%\textsuperscript{(64.6)}
\textsuperscript{5:2.3} Sin embargo, la presencia divina no se puede descubrir en ninguna parte de la naturaleza, ni siquiera en la vida de los mortales que conocen a Dios, de una manera tan plena y tan segura como en vuestro intento de comunión con el Monitor de Misterio interior, el Ajustador del Pensamiento del Paraíso. !Qué error soñar con un Dios lejano en los cielos, cuando el espíritu del Padre Universal vive dentro de vuestra propia mente!\footnote{\textit{El espíritu viviendo dentro}: Job 32:8,18; Is 63:10-11; Ez 37:14; Mt 10:20; Lc 17:21; Jn 17:21-23; Ro 8:9-11; 1 Co 3:16-17; 6:19; 2 Co 6:16; Gl 2:20; 1 Jn 3:24; 4:12-15; Ap 21:3.}

\par
%\textsuperscript{(64.7)}
\textsuperscript{5:2.4} Debido a este fragmento de Dios que reside en vosotros, y a medida que os armonicéis progresivamente con las directrices espirituales del Ajustador, podéis esperar discernir más plenamente la presencia y el poder transformador de aquellas otras influencias espirituales que os rodean e inciden en vosotros, pero que no funcionan como una parte integrante de vosotros. El hecho de que no seáis intelectualmente conscientes de un contacto estrecho e íntimo con el Ajustador interior no refuta en lo más mínimo una experiencia tan elevada. La prueba de la fraternidad con el Ajustador divino reside enteramente en la naturaleza y la extensión de los frutos del espíritu que produce la experiencia de la vida del creyente individual. «Por sus frutos los conoceréis»\footnote{\textit{Por sus frutos los conoceréis}: Mt 7:16-20; Lc 6:43-44; Gl 5:22-23; Ef 5:9.}.

\par
%\textsuperscript{(65.1)}
\textsuperscript{5:2.5} A la mente material escasamente espiritualizada del hombre mortal le resulta extremadamente difícil experimentar una conciencia notable de las actividades espirituales de unas entidades divinas tales como los Ajustadores Paradisiacos. A medida que el alma creada conjuntamente por la mente y el Ajustador se vuelve cada vez más real, también se desarrolla una nueva fase de la conciencia del alma que es capaz de experimentar la presencia de los Monitores de Misterio, y de reconocer sus directrices espirituales y sus otras actividades supermateriales.

\par
%\textsuperscript{(65.2)}
\textsuperscript{5:2.6} Toda la experiencia de la comunión con el Ajustador implica poseer un estado moral, una motivación mental y una experiencia espiritual. La conciencia personal de un logro semejante permanece limitada principalmente, aunque no exclusivamente, al ámbito de la conciencia del alma, pero las pruebas aparecen pronto y son abundantes, manifestándose mediante los frutos del espíritu en la vida de todos aquellos que se ponen en contacto con este espíritu interior.

\section*{3. La verdadera adoración}
\par
%\textsuperscript{(65.3)}
\textsuperscript{5:3.1} Desde el punto de vista universal, las Deidades del Paraíso son como una sola, pero en sus relaciones espirituales con los seres como los que viven en Urantia son también tres personas distintas y separadas. Existe una diferencia entre las Divinidades en aquellas cuestiones relacionadas con las súplicas personales, la comunión y otras relaciones íntimas. En el sentido más elevado, adoramos al Padre Universal y sólo a él. Es verdad que podemos adorar y adoramos al Padre tal como se manifiesta en sus Hijos Creadores, pero es el Padre, directa o indirectamente, el que es venerado y adorado.

\par
%\textsuperscript{(65.4)}
\textsuperscript{5:3.2} Las súplicas de todo tipo pertenecen al ámbito del Hijo Eterno y de la organización espiritual del Hijo. Las oraciones, todas las comunicaciones formales, todo, salvo la adoración y la veneración del Padre Universal, son cuestiones que conciernen al universo local; normalmente no sobrepasan el ámbito jurisdiccional de un Hijo Creador. Pero la adoración es incluida sin duda en un circuito y enviada a la persona del Creador por medio del circuito de la personalidad del Padre. Creemos además que este registro del homenaje de una criatura habitada por un Ajustador es facilitado por la presencia del espíritu del Padre. Existe una enorme cantidad de pruebas que justifican esta creencia, y sé que todos los tipos de fragmentos del Padre poseen la facultad de registrar aceptablemente en la presencia del Padre Universal la adoración auténtica de sus súbditos. Los Ajustadores también utilizan indudablemente unos canales prepersonales directos de comunicación con Dios, y son igualmente capaces de utilizar los circuitos de la gravedad espiritual del Hijo Eterno.

\par
%\textsuperscript{(65.5)}
\textsuperscript{5:3.3} La adoración tiene su razón de ser en sí misma; la oración incorpora un elemento de interés personal o para sí mismo; ésta es la gran diferencia entre la adoración y la oración. La verdadera adoración no contiene en absoluto ninguna petición para sí mismo ni ningún otro elemento de interés personal; adoramos simplemente a Dios por lo que comprendemos que él es. La adoración no pide nada ni espera nada para el adorador. No adoramos al Padre porque podamos obtener algo de esa veneración; le rendimos esa devoción y nos dedicamos a esa adoración como reacción espontánea y natural al reconocimiento de la personalidad incomparable del Padre y a causa de su naturaleza encantadora y de sus atributos adorables.

\par
%\textsuperscript{(65.6)}
\textsuperscript{5:3.4} En el momento en que un elemento de interés personal se introduce en la adoración, la devoción pasa de la adoración a la oración, y sería más conveniente dirigirla a la persona del Hijo Eterno o del Hijo Creador. Pero en la experiencia religiosa práctica no existe ninguna razón por la que la oración no pueda dirigirse a Dios Padre como parte de una verdadera adoración.

\par
%\textsuperscript{(66.1)}
\textsuperscript{5:3.5} Cuando os ocupáis de los asuntos prácticos de vuestra vida diaria, estáis en manos de las personalidades espirituales que tienen su origen en la Fuente-Centro Tercera; cooperáis con los agentes del Actor Conjunto. Así es como adoráis a Dios, oráis al Hijo y comulgáis con él, y resolvéis los detalles de vuestra estancia terrestre en conexión con las inteligencias del Espíritu Infinito que trabajan en vuestro mundo y en todo vuestro universo.

\par
%\textsuperscript{(66.2)}
\textsuperscript{5:3.6} Los Hijos Creadores o Hijos Soberanos que presiden los destinos de los universos locales ocupan el lugar tanto del Padre Universal como del Hijo Eterno del Paraíso. Estos Hijos de los Universos reciben en nombre del Padre la adoración del culto, y prestan oído a las súplicas de sus súbditos que oran en todas las partes de sus creaciones respectivas. A efectos prácticos, un Hijo Miguel es Dios para los hijos de su universo local. Es la personificación del Padre Universal y del Hijo Eterno en el universo local. El Espíritu Infinito mantiene un contacto personal con los hijos de esos reinos a través de los Espíritus del Universo, las asociadas administrativas y creativas de los Hijos Creadores Paradisiacos.

\par
%\textsuperscript{(66.3)}
\textsuperscript{5:3.7} La adoración sincera implica la movilización de todos los poderes de la personalidad humana bajo la dominación del alma evolutiva, y sujetos a la dirección divina del Ajustador del Pensamiento asociado. La mente, con sus limitaciones materiales, nunca puede volverse extremadamente consciente del significado real de la verdadera adoración. La comprensión humana de la realidad de la experiencia de la adoración está determinada principalmente por el estado de desarrollo de su alma inmortal en evolución. El crecimiento espiritual del alma tiene lugar de manera totalmente independiente de la conciencia intelectual de sí mismo.

\par
%\textsuperscript{(66.4)}
\textsuperscript{5:3.8} La experiencia de la adoración consiste en el intento sublime del Ajustador prometido por comunicar al Padre divino los anhelos inexpresables y las aspiraciones indecibles del alma humana ---creación conjunta de la mente mortal que busca a Dios y del Ajustador inmortal que revela a Dios. Por consiguiente, la adoración es el acto mediante el cual la mente material consiente que su yo en vías de espiritualizarse intente comunicarse con Dios, bajo la dirección del espíritu asociado, como hijo por la fe del Padre Universal. La mente mortal consiente en adorar; el alma inmortal anhela e inicia la adoración; la presencia divina del Ajustador dirige esta adoración en nombre de la mente mortal y del alma inmortal evolutiva. A fin de cuentas, la verdadera adoración se convierte en una experiencia que se lleva a cabo en cuatro niveles cósmicos: el intelectual, el morontial, el espiritual y el personal ---la conciencia de la mente, del alma y del espíritu, y su unificación en la personalidad.

\section*{4. Dios en la religión}
\par
%\textsuperscript{(66.5)}
\textsuperscript{5:4.1} La moralidad de las religiones evolutivas \textit{empuja} a los hombres hacia adelante en la búsqueda de Dios mediante la fuerza motriz del miedo. Las religiones de la revelación \textit{atraen} a los hombres hacia la búsqueda de un Dios de amor porque anhelan volverse semejantes a él. Pero la religión no es simplemente un sentimiento pasivo de «dependencia absoluta» y de «certeza de la supervivencia»; es una experiencia viviente y dinámica consistente en alcanzar la divinidad, basada en el servicio a la humanidad.

\par
%\textsuperscript{(66.6)}
\textsuperscript{5:4.2} El gran servicio inmediato de la verdadera religión es el establecimiento de una unidad duradera en la experiencia humana, una paz constante y una seguridad profunda\footnote{\textit{Paz duradera}: Sal 119:165; Is 26:3; Nm 6:26; Lc 1:79; 2:14; Jn 14:27; 16:33; Ro 14:17; 1 Co 14:33; Flp 4:7. \textit{Seguridad profunda}: Is 32:17; Ef 3:12; Col 2:2; 1 Ts 1:5; 2 Ti 1:12; Heb 6:11; 10:22; 1 Jn 3:19.}. Entre los hombres primitivos, incluso el politeísmo es una unificación relativa del concepto evolutivo de la Deidad; el politeísmo es el monoteísmo en formación. Tarde o temprano, Dios está destinado a ser comprendido como la realidad de los valores, la sustancia de los significados y la vida de la verdad.

\par
%\textsuperscript{(67.1)}
\textsuperscript{5:4.3} Dios no es solamente el que determina el destino; él \textit{es} el destino eterno del hombre. Todas las actividades humanas no religiosas intentan doblegar el universo al servicio deformante del yo; el individuo verdaderamente religioso intenta identificar su yo con el universo, y luego dedicar las actividades de ese yo unificado al servicio de la familia universal de sus semejantes, humanos y superhumanos.

\par
%\textsuperscript{(67.2)}
\textsuperscript{5:4.4} Los dominios de la filosofía y del arte se interponen entre las actividades religiosas y no religiosas del yo humano. A través del arte y la filosofía, el hombre con mentalidad materialista se siente persuadido a contemplar las realidades espirituales y los valores universales que tienen significados eternos.

\par
%\textsuperscript{(67.3)}
\textsuperscript{5:4.5} Todas las religiones enseñan la adoración de la Deidad y alguna doctrina de salvación humana. La religión budista promete salvar del sufrimiento, una paz sin fin; la religión judía promete salvar de las dificultades, una prosperidad basada en la rectitud; la religión griega prometía salvar de la falta de armonía, de la fealdad, gracias al reconocimiento de la belleza; el cristianismo promete salvar del pecado, la santidad; el mahometismo ofrece liberaros de las rigurosas reglas morales del judaísmo y del cristianismo. La religión de Jesús \textit{salva} del yo, libera de los males del aislamiento de la criatura en el tiempo y en la eternidad.

\par
%\textsuperscript{(67.4)}
\textsuperscript{5:4.6} Los hebreos basaban su religión en la bondad; los griegos, en la belleza; las dos religiones buscaban la verdad. Jesús reveló un Dios de amor, y el amor engloba totalmente a la verdad, la belleza y la bondad.

\par
%\textsuperscript{(67.5)}
\textsuperscript{5:4.7} Los zoroástricos tenían una religión de moralidad; los hindúes, una religión de metafísica; los confucionistas, una religión de ética. Jesús vivió una religión de \textit{servicio.} Todas estas religiones son valiosas en la medida en que se aproximan válidamente a la religión de Jesús. La religión está destinada a convertirse en la realidad de la unificación espiritual de todo lo que es bueno, hermoso y verdadero en la experiencia humana.

\par
%\textsuperscript{(67.6)}
\textsuperscript{5:4.8} La religión griega tenía un lema: «Conócete a ti mismo»; los hebreos centraban su enseñanza en «Conoced a vuestro Dios»\footnote{\textit{Conoce a Dios}: 1 Cr 28:9; Jer 9:24; 31:34; Os 6:6.}; los cristianos predican un evangelio dirigido al «conocimiento del Señor Jesucristo»\footnote{\textit{Conoce a Jesucristo}: Flp 3:8; 2 P 1:8; 3:18.}; Jesús proclamó la buena nueva de «conoce a Dios y conócete a ti mismo como hijo de Dios»\footnote{\textit{Conoce a Dios y a ti como hijo}: Jn 14:7,17,20; Flp 3:9-10; Heb 8:11; 1 Jn 2:3-5. \textit{Somos hijos de Dios}: 1 Cr 22:10; Sal 2:7; Is 56:5; Mt 5:9,16,45; Lc 20:36; Jn 1:12-13; 11:52; Hch 17:28-29; Ro 8:14-17,19,21; 9:26; 2 Co 6:18; Gl 3:26; 4:5-7; Ef 1:5; Flp 2:15; Heb 12:5-8; 1 Jn 3:1-2,10; 5:2; Ap 21:7; 2 Sam 7:14.}. Estos conceptos diferentes sobre la meta de la religión determinan la actitud del individuo en las diversas situaciones de la vida, y presagian la profundidad de su adoración y la naturaleza de sus hábitos personales de oración. El estado espiritual de cualquier religión se puede determinar por la naturaleza de sus oraciones.

\par
%\textsuperscript{(67.7)}
\textsuperscript{5:4.9} El concepto de un Dios semihumano y celoso es una transición inevitable entre el politeísmo y el sublime monoteísmo. Un elevado antropomorfismo es el nivel más alto que puede alcanzar una religión puramente evolutiva. El cristianismo ha elevado el concepto del antropomorfismo desde el ideal de lo humano hasta el concepto trascendente y divino de la persona del Cristo glorificado. Éste es el antropomorfismo más elevado que el hombre pueda concebir jamás.

\par
%\textsuperscript{(67.8)}
\textsuperscript{5:4.10} El concepto cristiano de Dios es un intento por combinar tres enseñanzas diferentes:

\par
%\textsuperscript{(67.9)}
\textsuperscript{5:4.11} 1. \textit{El concepto hebreo} ---Dios como defensor de los valores morales, un Dios justo.

\par
%\textsuperscript{(67.10)}
\textsuperscript{5:4.12} 2. \textit{El concepto griego} ---Dios como unificador, un Dios de sabiduría.

\par
%\textsuperscript{(68.1)}
\textsuperscript{5:4.13} 3. \textit{El concepto de Jesús} ---Dios como amigo viviente, un Padre amoroso, la presencia divina.

\par
%\textsuperscript{(68.2)}
\textsuperscript{5:4.14} Por lo tanto, ha de ser evidente que la teología compuesta cristiana encuentra grandes dificultades para conseguir la coherencia. Estas dificultades se agravan aún más por el hecho de que las doctrinas del cristianismo primitivo estaban basadas generalmente en la experiencia religiosa personal de tres personas diferentes: Filón de Alejandría, Jesús de Nazaret y Pablo de Tarso.

\par
%\textsuperscript{(68.3)}
\textsuperscript{5:4.15} Cuando estudiéis la vida religiosa de Jesús, consideradlo de manera positiva. No penséis tanto en que estaba libre de pecado, sino en su rectitud, en su servicio amoroso. Jesús elevó el amor pasivo, revelado en el concepto hebreo del Padre celestial, hasta el afecto \textit{activo} superior, amoroso por sus criaturas, de un Dios que es el Padre de todos los individuos, incluso de los malhechores.

\section*{5. La conciencia de Dios}
\par
%\textsuperscript{(68.4)}
\textsuperscript{5:5.1} El hecho de ser consciente de sí mismo da origen a la moralidad; ésta es superanimal pero totalmente evolutiva. La evolución humana abarca en su desarrollo todos los dones que preceden a la concesión de los Ajustadores y al derramamiento del Espíritu de la Verdad. Pero alcanzar los niveles de la moralidad no libera al hombre de las luchas reales de su vida como mortal. El entorno físico del hombre implica la lucha por la existencia; el medio ambiente social necesita ajustes éticos; las situaciones morales requieren que se hagan elecciones en las esferas más elevadas de la razón; la experiencia espiritual (una vez que se tiene conciencia de Dios) exige que el hombre lo encuentre y se esfuerce sinceramente por parecerse a él.

\par
%\textsuperscript{(68.5)}
\textsuperscript{5:5.2} La religión no está basada en los hechos de la ciencia, ni en las obligaciones de la sociedad, ni en las suposiciones de la filosofía, ni en los deberes implícitos de la moralidad. La religión es un campo independiente de reacción humana a las situaciones de la vida, y aparece infaliblemente en todas las fases del desarrollo humano posteriores a la moral. La religión puede impregnar los cuatro niveles de la comprensión de los valores y del disfrute de la fraternidad universal: el nivel físico o material de la preservación de sí mismo; el nivel social o emocional de la fraternidad; el nivel moral de la razón o del deber; y el nivel espiritual de la conciencia de la fraternidad universal mediante la adoración divina.

\par
%\textsuperscript{(68.6)}
\textsuperscript{5:5.3} El científico que busca los hechos concibe a Dios como la Causa Primera, un Dios de fuerza. El artista emotivo ve a Dios como el ideal de la belleza, un Dios de estética. El filósofo razonador se siente a veces inclinado a proponer un Dios de unidad universal, e incluso una Deidad panteísta. La persona religiosa que tiene fe cree en un Dios que patrocina la supervivencia, el Padre que está en los cielos, el Dios de amor.

\par
%\textsuperscript{(68.7)}
\textsuperscript{5:5.4} La conducta moral precede siempre a la religión evolutiva e incluso es una parte de la religión revelada, pero nunca es la totalidad de la experiencia religiosa. El servicio social es el resultado de una manera moral de pensar y religiosa de vivir. La moralidad no conduce biológicamente a los niveles espirituales más elevados de la experiencia religiosa. La adoración de la belleza abstracta no es la veneración de Dios; la exaltación de la naturaleza o la veneración de la unidad tampoco son la adoración de Dios.

\par
%\textsuperscript{(68.8)}
\textsuperscript{5:5.5} La religión evolutiva es la madre de la ciencia, del arte y de la filosofía que han elevado al hombre hasta el nivel en que es receptivo a la religión revelada, incluyendo la concesión de los Ajustadores y la venida del Espíritu de la Verdad. El cuadro evolutivo de la existencia humana comienza y termina con la religión, aunque con calidades muy diferentes de religión, una evolutiva y biológica, la otra revelada y periódica. Así pues, aunque la religión es normal y natural para el hombre, es también opcional. El hombre no tiene por qué ser religioso en contra de su voluntad.

\par
%\textsuperscript{(69.1)}
\textsuperscript{5:5.6} Como la experiencia religiosa es esencialmente espiritual, nunca puede ser plenamente comprendida por la mente material; de ahí la función de la teología, que es la psicología de la religión. La doctrina fundamental de la comprensión humana de Dios crea una paradoja en el entendimiento finito. A la lógica humana y a la razón finita les resulta casi imposible armonizar el concepto de la inmanencia divina, un Dios interior que forma parte de cada individuo, con la idea de la trascendencia de Dios, la dominación divina del universo de universos. Estos dos conceptos esenciales de la Deidad deben ser unificados mediante la captación por la fe del concepto de la trascendencia de un Dios personal y la comprensión de la presencia interior de un fragmento de ese Dios, con el objeto de justificar la adoración inteligente y validar la esperanza de la supervivencia de la personalidad. Las dificultades y las paradojas de la religión son inherentes al hecho de que las realidades de la religión sobrepasan por completo la capacidad de comprensión intelectual de los mortales.

\par
%\textsuperscript{(69.2)}
\textsuperscript{5:5.7} El hombre mortal obtiene tres grandes satisfacciones de su experiencia religiosa, incluso durante los días de su estancia temporal en la Tierra:

\par
%\textsuperscript{(69.3)}
\textsuperscript{5:5.8} 1. \textit{Intelectualmente,} adquiere la satisfacción de una conciencia humana más unificada.

\par
%\textsuperscript{(69.4)}
\textsuperscript{5:5.9} 2. \textit{Filosóficamente,} disfruta de la justificación de sus ideales de los valores morales.

\par
%\textsuperscript{(69.5)}
\textsuperscript{5:5.10} 3. \textit{Espiritualmente,} crece en la experiencia del compañerismo divino, en las satisfacciones espirituales de la verdadera adoración.

\par
%\textsuperscript{(69.6)}
\textsuperscript{5:5.11} La conciencia de Dios, tal como la experimentan los mortales evolutivos de los mundos, debe consistir en tres factores variables, en tres niveles diferenciales de comprensión de la realidad. En primer lugar está la conciencia mental ---la comprensión de la \textit{idea} de Dios. Luego le sigue la conciencia del alma ---la comprensión del \textit{ideal} de Dios. Finalmente despunta la conciencia del espíritu ---la comprensión de la \textit{realidad espiritual} de Dios. Mediante la unificación de estos factores de la comprensión divina, por muy incompleta que ésta sea, la personalidad mortal despliega constantemente, sobre todos los niveles conscientes, una comprensión de la \textit{personalidad} de Dios. En aquellos mortales que han alcanzado el Cuerpo de la Finalidad, todo esto conducirá en su momento a la comprensión de la \textit{supremacía} de Dios, y puede traducirse posteriormente en la comprensión de la \textit{ultimidad} de Dios, una fase de la superconciencia absonita del Padre Paradisiaco.

\par
%\textsuperscript{(69.7)}
\textsuperscript{5:5.12} La experiencia de la conciencia de Dios sigue siendo la misma de generación en generación, pero a medida que avanza el conocimiento humano en cada época, el concepto filosófico y las definiciones teológicas de Dios \textit{deben} cambiar. El conocimiento sobre Dios, la conciencia religiosa, es una realidad universal, pero por muy válida (real) que sea la experiencia religiosa, debe estar dispuesta a someterse a la crítica inteligente y a una interpretación filosófica razonable; no debe tratar de ser una cosa separada de la totalidad de la experiencia humana.

\par
%\textsuperscript{(69.8)}
\textsuperscript{5:5.13} La supervivencia eterna de la personalidad depende enteramente de la elección de la mente mortal, cuyas decisiones determinan el potencial de supervivencia del alma inmortal. Cuando la mente cree en Dios y el alma conoce a Dios, cuando con el Ajustador que estimula todos \textit{desean} a Dios, entonces la supervivencia está asegurada. Las limitaciones del intelecto, las restricciones de la educación, la privación de cultura, el empobrecimiento de la posición social e incluso unos criterios morales humanos inferiores ocasionados por la falta desafortunada de ventajas educativas, culturales y sociales, no pueden invalidar la presencia del espíritu divino en esos individuos desafortunados y humanamente perjudicados, pero creyentes. La presencia interior del Monitor de Misterio constituye el comienzo, y asegura la posibilidad, del potencial de crecimiento y de supervivencia del alma inmortal.

\par
%\textsuperscript{(70.1)}
\textsuperscript{5:5.14} La capacidad de los padres mortales para procrear no está basada en su nivel educativo, cultural, social o económico. La unión de los factores parentales en condiciones naturales es completamente suficiente para dar comienzo a una descendencia. Una mente humana que discierne el bien y el mal y que posee la capacidad de adorar a Dios, en unión con un Ajustador divino, es todo lo que necesita ese mortal para dar comienzo y fomentar el nacimiento de su alma inmortal con sus cualidades de supervivencia, si ese individuo dotado de espíritu busca a Dios y desea sinceramente volverse como él, elige honradamente hacer la voluntad del Padre que está en los cielos.

\section*{6. El Dios de la personalidad}
\par
%\textsuperscript{(70.2)}
\textsuperscript{5:6.1} El Padre Universal es el Dios de las personalidades. El campo de la personalidad en el universo, desde las criaturas mortales y materiales más humildes con estatus de personalidad hasta las personas más elevadas con dignidad de creadores y con estatus divino, tiene su centro y su circunferencia en el Padre Universal. Dios Padre es el que concede y conserva cada personalidad. Y el Padre Paradisiaco es igualmente el destino de todas aquellas personalidades finitas que eligen sinceramente hacer la voluntad divina, de aquellos que aman a Dios y anhelan parecerse a él.

\par
%\textsuperscript{(70.3)}
\textsuperscript{5:6.2} La personalidad es uno de los misterios no resueltos de los universos. Podemos formarnos unos conceptos adecuados de los factores que entran en la composición de los diversos tipos y niveles de personalidades, pero no comprendemos plenamente la naturaleza real de la personalidad misma. Percibimos claramente los numerosos factores que, una vez reunidos, constituyen el vehículo de la personalidad humana, pero no comprendemos plenamente la naturaleza y el significado de esa personalidad finita.

\par
%\textsuperscript{(70.4)}
\textsuperscript{5:6.3} La personalidad es potencial en todas las criaturas que poseen una dotación mental comprendida entre el mínimo de conciencia de sí mismo hasta el máximo de conciencia de Dios. Pero la dotación mental por sí sola no es la personalidad, ni tampoco lo es el espíritu ni la energía física. La personalidad es esa cualidad y ese valor, dentro de la realidad cósmica, que es concedida exclusivamente por Dios Padre a aquellos sistemas vivientes donde las energías de la materia, la mente y el espíritu están asociadas y coordinadas. La personalidad tampoco es una consecución progresiva. La personalidad puede ser material o espiritual, pero la personalidad está o no está. Aquello que es distinto a lo personal nunca alcanza el nivel de lo personal, salvo mediante un acto directo del Padre Paradisiaco.

\par
%\textsuperscript{(70.5)}
\textsuperscript{5:6.4} La concesión de la personalidad es una ocupación exclusiva del Padre Universal, es la personalización de los sistemas energéticos vivientes, a los cuales dota de los atributos de una conciencia creativa relativa y del control de la misma por medio del libre albedrío. No hay ninguna personalidad que no provenga de Dios Padre, y no existe ninguna personalidad si no es gracias a Dios Padre. Los atributos fundamentales de la individualidad humana, así como el Ajustador, núcleo absoluto de la personalidad humana, son dones del Padre Universal actuando en su terreno exclusivamente personal de ministerio cósmico.

\par
%\textsuperscript{(70.6)}
\textsuperscript{5:6.5} Los Ajustadores, cuyo estado es prepersonal, residen en numerosos tipos de criaturas mortales, asegurando así a estos mismos seres la posibilidad de sobrevivir a la muerte física para personalizarse como criaturas morontiales, con el potencial de alcanzar el estado espiritual último. Porque, cuando la mente de una criatura dotada de personalidad está habitada por un fragmento del espíritu del Dios eterno, el don prepersonal del Padre personal, entonces esa personalidad finita posee el potencial de lo divino y de lo eterno, y aspira a un destino semejante al del Último, tendiendo incluso hacia la comprensión del Absoluto.

\par
%\textsuperscript{(71.1)}
\textsuperscript{5:6.6} La capacidad para recibir la personalidad divina es inherente al Ajustador prepersonal; la capacidad para recibir la personalidad humana existe en potencia en la dotación mental cósmica del ser humano. Pero la personalidad experiencial del hombre mortal no es observable como realidad activa y funcional hasta después de que el vehículo vital material de la criatura mortal ha sido tocado por la divinidad liberadora del Padre Universal, siendo lanzada así a los mares de la experiencia como una personalidad consciente de sí misma, capaz (relativamente) de determinarse y de crearse a sí misma. El yo material es verdaderamente \textit{personal sin ninguna restricción.}

\par
%\textsuperscript{(71.2)}
\textsuperscript{5:6.7} El yo material posee una personalidad y una identidad, una identidad temporal; el Ajustador espiritual prepersonal posee también una identidad, una identidad eterna. Esta personalidad material y esta prepersonalidad espiritual son capaces de unir sus atributos creadores como para traer a la existencia la identidad sobreviviente del alma inmortal.

\par
%\textsuperscript{(71.3)}
\textsuperscript{5:6.8} Una vez que ha asegurado así el crecimiento del alma inmortal y que ha liberado al yo interior del hombre de las cadenas de la dependencia absoluta a la causalidad precedente, el Padre se retira. Así pues, una vez que el hombre ha sido liberado así de las cadenas de la reacción a la causalidad, al menos en lo relacionado con el destino eterno, y que se ha facilitado el crecimiento del yo inmortal, el alma, queda en manos del hombre mismo el querer o el impedir la creación de ese yo sobreviviente y eterno que será suyo si así lo elige. Ningún otro ser, ninguna fuerza, ningún creador o agente en todo el extenso universo de universos puede interferir en ninguna medida en la soberanía absoluta del libre albedrío humano, tal como éste funciona dentro del campo de la elección, en lo referente al destino eterno de la personalidad del mortal que escoge. En lo que concierne a la supervivencia eterna, Dios ha decretado que la voluntad material y humana es soberana, y este decreto es absoluto.

\par
%\textsuperscript{(71.4)}
\textsuperscript{5:6.9} La concesión de la personalidad a las criaturas les confiere una liberación relativa respecto a la reacción servil a la causalidad precedente, y la personalidad de todos estos seres morales, evolutivos u otros, está centrada en la personalidad del Padre Universal. Siempre es atraída hacia su presencia en el Paraíso por ese parentesco de existencia que constituye el inmenso círculo familiar universal y el circuito fraternal del Dios eterno\footnote{\textit{Gravedad espiritual}: Jer 31:3; Jn 6:44; 12:32.}. Existe un parentesco de espontaneidad divina en toda personalidad.

\par
%\textsuperscript{(71.5)}
\textsuperscript{5:6.10} El circuito de personalidad del universo de universos está centrado en la persona del Padre Universal, y el Padre Paradisiaco es personalmente consciente de todas las personalidades de todos los niveles de existencia consciente, y se mantiene en contacto personal con ellas. Esta conciencia sobre las personalidades de toda la creación existe independientemente de la misión de los Ajustadores del Pensamiento.

\par
%\textsuperscript{(71.6)}
\textsuperscript{5:6.11} Al igual que toda la gravedad está incluida en el circuito de la Isla del Paraíso, toda mente en el circuito del Actor Conjunto y todo espíritu en el Hijo Eterno, del mismo modo toda personalidad está incluida en el circuito de la presencia personal del Padre Universal, y este circuito transmite infaliblemente la adoración de todas las personalidades a la Personalidad Original y Eterna.

\par
%\textsuperscript{(71.7)}
\textsuperscript{5:6.12} En cuanto a aquellas personalidades que no están habitadas por un Ajustador, el Padre Universal también les ha concedido el atributo de la libertad de elección, y estas personas están incluidas igualmente en el gran circuito del amor divino, el circuito de personalidad del Padre Universal. Dios asegura la elección soberana a todas las verdaderas personalidades. Ninguna criatura personal puede ser forzada a emprender la aventura eterna; la puerta de la eternidad sólo se abre en respuesta a la libre elección de los hijos con libre albedrío del Dios del libre albedrío.

\par
%\textsuperscript{(72.1)}
\textsuperscript{5:6.13} Esto representa mis esfuerzos por exponer las relaciones del Dios viviente con los hijos del tiempo. Y cuando todo ha sido dicho y hecho, no puedo hacer nada más útil que reiterar que Dios es vuestro Padre en el universo, y que todos sois sus hijos planetarios.

\par
%\textsuperscript{(72.2)}
\textsuperscript{5:6.14} [Este documento es el quinto y último de la serie que describe al Padre Universal, presentada por un Consejero Divino de Uversa.]