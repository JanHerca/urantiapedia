\begin{document}

\title{1-е послание Иоанна}


\chapter{1}

\par 1 О том, что было от начала, что мы слышали, что видели своими очами, что рассматривали и что осязали руки наши, о Слове жизни, --
\par 2 ибо жизнь явилась, и мы видели и свидетельствуем, и возвещаем вам сию вечную жизнь, которая была у Отца и явилась нам, --
\par 3 о том, что мы видели и слышали, возвещаем вам, чтобы и вы имели общение с нами: а наше общение--с Отцем и Сыном Его, Иисусом Христом.
\par 4 И сие пишем вам, чтобы радость ваша была совершенна.
\par 5 И вот благовестие, которое мы слышали от Него и возвещаем вам: Бог есть свет, и нет в Нем никакой тьмы.
\par 6 Если мы говорим, что имеем общение с Ним, а ходим во тьме, то мы лжем и не поступаем по истине;
\par 7 если же ходим во свете, подобно как Он во свете, то имеем общение друг с другом, и Кровь Иисуса Христа, Сына Его, очищает нас от всякого греха.
\par 8 Если говорим, что не имеем греха, --обманываем самих себя, и истины нет в нас.
\par 9 Если исповедуем грехи наши, то Он, будучи верен и праведен, простит нам грехи наши и очистит нас от всякой неправды.
\par 10 Если говорим, что мы не согрешили, то представляем Его лживым, и слова Его нет в нас.

\chapter{2}

\par 1 Дети мои! сие пишу вам, чтобы вы не согрешали; а если бы кто согрешил, то мы имеем ходатая пред Отцем, Иисуса Христа, праведника;
\par 2 Он есть умилостивление за грехи наши, и не только за наши, но и за [грехи] всего мира.
\par 3 А что мы познали Его, узнаем из того, что соблюдаем Его заповеди.
\par 4 Кто говорит: `я познал Его', но заповедей Его не соблюдает, тот лжец, и нет в нем истины;
\par 5 а кто соблюдает слово Его, в том истинно любовь Божия совершилась: из сего узнаем, что мы в Нем.
\par 6 Кто говорит, что пребывает в Нем, тот должен поступать так, как Он поступал.
\par 7 Возлюбленные! пишу вам не новую заповедь, но заповедь древнюю, которую вы имели от начала. Заповедь древняя есть слово, которое вы слышали от начала.
\par 8 Но притом и новую заповедь пишу вам, что есть истинно и в Нем и в вас: потому что тьма проходит и истинный свет уже светит.
\par 9 Кто говорит, что он во свете, а ненавидит брата своего, тот еще во тьме.
\par 10 Кто любит брата своего, тот пребывает во свете, и нет в нем соблазна.
\par 11 А кто ненавидит брата своего, тот находится во тьме, и во тьме ходит, и не знает, куда идет, потому что тьма ослепила ему глаза.
\par 12 Пишу вам, дети, потому что прощены вам грехи ради имени Его.
\par 13 Пишу вам, отцы, потому что вы познали Сущего от начала. Пишу вам, юноши, потому что вы победили лукавого. Пишу вам, отроки, потому что вы познали Отца.
\par 14 Я написал вам, отцы, потому что вы познали Безначального. Я написал вам, юноши, потому что вы сильны, и слово Божие пребывает в вас, и вы победили лукавого.
\par 15 Не любите мира, ни того, что в мире: кто любит мир, в том нет любви Отчей.
\par 16 Ибо все, что в мире: похоть плоти, похоть очей и гордость житейская, не есть от Отца, но от мира сего.
\par 17 И мир проходит, и похоть его, а исполняющий волю Божию пребывает вовек.
\par 18 Дети! последнее время. И как вы слышали, что придет антихрист, и теперь появилось много антихристов, то мы и познаем из того, что последнее время.
\par 19 Они вышли от нас, но не были наши: ибо если бы они были наши, то остались бы с нами; но [они вышли, и] через то открылось, что не все наши.
\par 20 Впрочем, вы имеете помазание от Святаго и знаете все.
\par 21 Я написал вам не потому, чтобы вы не знали истины, но потому, что вы знаете ее, [равно как] и то, что всякая ложь не от истины.
\par 22 Кто лжец, если не тот, кто отвергает, что Иисус есть Христос? Это антихрист, отвергающий Отца и Сына.
\par 23 Всякий, отвергающий Сына, не имеет и Отца; а исповедующий Сына имеет и Отца.
\par 24 Итак, что вы слышали от начала, то и да пребывает в вас; если пребудет в вас то, что вы слышали от начала, то и вы пребудете в Сыне и в Отце.
\par 25 Обетование же, которое Он обещал нам, есть жизнь вечная.
\par 26 Это я написал вам об обольщающих вас.
\par 27 Впрочем, помазание, которое вы получили от Него, в вас пребывает, и вы не имеете нужды, чтобы кто учил вас; но как самое сие помазание учит вас всему, и оно истинно и неложно, то, чему оно научило вас, в том пребывайте.
\par 28 Итак, дети, пребывайте в Нем, чтобы, когда Он явится, иметь нам дерзновение и не постыдиться пред Ним в пришествие Его.
\par 29 Если вы знаете, что Он праведник, знайте и то, что всякий, делающий правду, рожден от Него.

\chapter{3}

\par 1 Смотрите, какую любовь дал нам Отец, чтобы нам называться и быть детьми Божиими. Мир потому не знает нас, что не познал Его.
\par 2 Возлюбленные! мы теперь дети Божии; но еще не открылось, что будем. Знаем только, что, когда откроется, будем подобны Ему, потому что увидим Его, как Он есть.
\par 3 И всякий, имеющий сию надежду на Него, очищает себя так, как Он чист.
\par 4 Всякий, делающий грех, делает и беззаконие; и грех есть беззаконие.
\par 5 И вы знаете, что Он явился для того, чтобы взять грехи наши, и что в Нем нет греха.
\par 6 Всякий, пребывающий в Нем, не согрешает; всякий согрешающий не видел Его и не познал Его.
\par 7 Дети! да не обольщает вас никто. Кто делает правду, тот праведен, подобно как Он праведен.
\par 8 Кто делает грех, тот от диавола, потому что сначала диавол согрешил. Для сего-то и явился Сын Божий, чтобы разрушить дела диавола.
\par 9 Всякий, рожденный от Бога, не делает греха, потому что семя Его пребывает в нем; и он не может грешить, потому что рожден от Бога.
\par 10 Дети Божии и дети диавола узнаются так: всякий, не делающий правды, не есть от Бога, равно и не любящий брата своего.
\par 11 Ибо таково благовествование, которое вы слышали от начала, чтобы мы любили друг друга,
\par 12 не так, как Каин, [который] был от лукавого и убил брата своего. А за что убил его? За то, что дела его были злы, а дела брата его праведны.
\par 13 Не дивитесь, братия мои, если мир ненавидит вас.
\par 14 Мы знаем, что мы перешли из смерти в жизнь, потому что любим братьев; не любящий брата пребывает в смерти.
\par 15 Всякий, ненавидящий брата своего, есть человекоубийца; а вы знаете, что никакой человекоубийца не имеет жизни вечной, в нем пребывающей.
\par 16 Любовь познали мы в том, что Он положил за нас душу Свою: и мы должны полагать души свои за братьев.
\par 17 А кто имеет достаток в мире, но, видя брата своего в нужде, затворяет от него сердце свое, --как пребывает в том любовь Божия?
\par 18 Дети мои! станем любить не словом или языком, но делом и истиною.
\par 19 И вот по чему узнаем, что мы от истины, и успокаиваем пред Ним сердца наши;
\par 20 ибо если сердце наше осуждает нас, то [кольми паче Бог], потому что Бог больше сердца нашего и знает все.
\par 21 Возлюбленные! если сердце наше не осуждает нас, то мы имеем дерзновение к Богу,
\par 22 и, чего ни попросим, получим от Него, потому что соблюдаем заповеди Его и делаем благоугодное пред Ним.
\par 23 А заповедь Его та, чтобы мы веровали во имя Сына Его Иисуса Христа и любили друг друга, как Он заповедал нам.
\par 24 И кто сохраняет заповеди Его, тот пребывает в Нем, и Он в том. А что Он пребывает в нас, узнаем по духу, который Он дал нам.

\chapter{4}

\par 1 Возлюбленные! не всякому духу верьте, но испытывайте духов, от Бога ли они, потому что много лжепророков появилось в мире.
\par 2 Духа Божия (и духа заблуждения) узнавайте так: всякий дух, который исповедует Иисуса Христа, пришедшего во плоти, есть от Бога;
\par 3 а всякий дух, который не исповедует Иисуса Христа, пришедшего во плоти, не есть от Бога, но это дух антихриста, о котором вы слышали, что он придет и теперь есть уже в мире.
\par 4 Дети! вы от Бога, и победили их; ибо Тот, Кто в вас, больше того, кто в мире.
\par 5 Они от мира, потому и говорят по-мирски, и мир слушает их.
\par 6 Мы от Бога; знающий Бога слушает нас; кто не от Бога, тот не слушает нас. По сему-то узнаем духа истины и духа заблуждения.
\par 7 Возлюбленные! будем любить друг друга, потому что любовь от Бога, и всякий любящий рожден от Бога и знает Бога.
\par 8 Кто не любит, тот не познал Бога, потому что Бог есть любовь.
\par 9 Любовь Божия к нам открылась в том, что Бог послал в мир Единородного Сына Своего, чтобы мы получили жизнь через Него.
\par 10 В том любовь, что не мы возлюбили Бога, но Он возлюбил нас и послал Сына Своего в умилостивление за грехи наши.
\par 11 Возлюбленные! если так возлюбил нас Бог, то и мы должны любить друг друга.
\par 12 Бога никто никогда не видел. Если мы любим друг друга, то Бог в нас пребывает, и любовь Его совершенна есть в нас.
\par 13 Что мы пребываем в Нем и Он в нас, узнаем из того, что Он дал нам от Духа Своего.
\par 14 И мы видели и свидетельствуем, что Отец послал Сына Спасителем миру.
\par 15 Кто исповедует, что Иисус есть Сын Божий, в том пребывает Бог, и он в Боге.
\par 16 И мы познали любовь, которую имеет к нам Бог, и уверовали в нее. Бог есть любовь, и пребывающий в любви пребывает в Боге, и Бог в нем.
\par 17 Любовь до того совершенства достигает в нас, что мы имеем дерзновение в день суда, потому что поступаем в мире сем, как Он.
\par 18 В любви нет страха, но совершенная любовь изгоняет страх, потому что в страхе есть мучение. Боящийся несовершен в любви.
\par 19 Будем любить Его, потому что Он прежде возлюбил нас.
\par 20 Кто говорит: `я люблю Бога', а брата своего ненавидит, тот лжец: ибо не любящий брата своего, которого видит, как может любить Бога, Которого не видит?
\par 21 И мы имеем от Него такую заповедь, чтобы любящий Бога любил и брата своего.

\chapter{5}

\par 1 Всякий верующий, что Иисус есть Христос, от Бога рожден, и всякий, любящий Родившего, любит и Рожденного от Него.
\par 2 Что мы любим детей Божиих, узнаем из того, когда любим Бога и соблюдаем заповеди Его.
\par 3 Ибо это есть любовь к Богу, чтобы мы соблюдали заповеди Его; и заповеди Его нетяжки.
\par 4 Ибо всякий, рожденный от Бога, побеждает мир; и сия есть победа, победившая мир, вера наша.
\par 5 Кто побеждает мир, как не тот, кто верует, что Иисус есть Сын Божий?
\par 6 Сей есть Иисус Христос, пришедший водою и кровию и Духом, не водою только, но водою и кровию, и Дух свидетельствует о [Нем], потому что Дух есть истина.
\par 7 Ибо три свидетельствуют на небе: Отец, Слово и Святый Дух; и Сии три суть едино.
\par 8 И три свидетельствуют на земле: дух, вода и кровь; и сии три об одном.
\par 9 Если мы принимаем свидетельство человеческое, свидетельство Божие--больше, ибо это есть свидетельство Божие, которым Бог свидетельствовал о Сыне Своем.
\par 10 Верующий в Сына Божия имеет свидетельство в себе самом; не верующий Богу представляет Его лживым, потому что не верует в свидетельство, которым Бог свидетельствовал о Сыне Своем.
\par 11 Свидетельство сие состоит в том, что Бог даровал нам жизнь вечную, и сия жизнь в Сыне Его.
\par 12 Имеющий Сына (Божия) имеет жизнь; не имеющий Сына Божия не имеет жизни.
\par 13 Сие написал я вам, верующим во имя Сына Божия, дабы вы знали, что вы, веруя в Сына Божия, имеете жизнь вечную.
\par 14 И вот какое дерзновение мы имеем к Нему, что, когда просим чего по воле Его, Он слушает нас.
\par 15 А когда мы знаем, что Он слушает нас во всем, чего бы мы ни просили, --знаем и то, что получаем просимое от Него.
\par 16 Если кто видит брата своего согрешающего грехом не к смерти, то пусть молится, и [Бог] даст ему жизнь, [то есть] согрешающему [грехом] не к смерти. Есть грех к смерти: не о том говорю, чтобы он молился.
\par 17 Всякая неправда есть грех; но есть грех не к смерти.
\par 18 Мы знаем, что всякий, рожденный от Бога, не грешит; но рожденный от Бога хранит себя, и лукавый не прикасается к нему.
\par 19 Мы знаем, что мы от Бога и что весь мир лежит во зле.
\par 20 Знаем также, что Сын Божий пришел и дал нам свет и разум, да познаем Бога истинного и да будем в истинном Сыне Его Иисусе Христе. Сей есть истинный Бог и жизнь вечная.
\par 21 Дети! храните себя от идолов. Аминь.


\end{document}