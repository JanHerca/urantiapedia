\begin{document}

\title{От Иоанна}


\chapter{1}

\par 1 В начале было Слово, и Слово было у Бога, и Слово было Бог.
\par 2 Оно было в начале у Бога.
\par 3 Все чрез Него начало быть, и без Него ничто не начало быть, что начало быть.
\par 4 В Нем была жизнь, и жизнь была свет человеков.
\par 5 И свет во тьме светит, и тьма не объяла его.
\par 6 Был человек, посланный от Бога; имя ему Иоанн.
\par 7 Он пришел для свидетельства, чтобы свидетельствовать о Свете, дабы все уверовали чрез него.
\par 8 Он не был свет, но [был послан], чтобы свидетельствовать о Свете.
\par 9 Был Свет истинный, Который просвещает всякого человека, приходящего в мир.
\par 10 В мире был, и мир чрез Него начал быть, и мир Его не познал.
\par 11 Пришел к своим, и свои Его не приняли.
\par 12 А тем, которые приняли Его, верующим во имя Его, дал власть быть чадами Божиими,
\par 13 которые ни от крови, ни от хотения плоти, ни от хотения мужа, но от Бога родились.
\par 14 И Слово стало плотию, и обитало с нами, полное благодати и истины; и мы видели славу Его, славу, как Единородного от Отца.
\par 15 Иоанн свидетельствует о Нем и, восклицая, говорит: Сей был Тот, о Котором я сказал, что Идущий за мною стал впереди меня, потому что был прежде меня.
\par 16 И от полноты Его все мы приняли и благодать на благодать,
\par 17 ибо закон дан чрез Моисея; благодать же и истина произошли чрез Иисуса Христа.
\par 18 Бога не видел никто никогда; Единородный Сын, сущий в недре Отчем, Он явил.
\par 19 И вот свидетельство Иоанна, когда Иудеи прислали из Иерусалима священников и левитов спросить его: кто ты?
\par 20 Он объявил, и не отрекся, и объявил, что я не Христос.
\par 21 И спросили его: что же? ты Илия? Он сказал: нет. Пророк? Он отвечал: нет.
\par 22 Сказали ему: кто же ты? чтобы нам дать ответ пославшим нас: что ты скажешь о себе самом?
\par 23 Он сказал: я глас вопиющего в пустыне: исправьте путь Господу, как сказал пророк Исаия.
\par 24 А посланные были из фарисеев;
\par 25 И они спросили его: что же ты крестишь, если ты ни Христос, ни Илия, ни пророк?
\par 26 Иоанн сказал им в ответ: я крещу в воде; но стоит среди вас [Некто], Которого вы не знаете.
\par 27 Он-то Идущий за мною, но Который стал впереди меня. Я недостоин развязать ремень у обуви Его.
\par 28 Это происходило в Вифаваре при Иордане, где крестил Иоанн.
\par 29 На другой день видит Иоанн идущего к нему Иисуса и говорит: вот Агнец Божий, Который берет [на Себя] грех мира.
\par 30 Сей есть, о Котором я сказал: за мною идет Муж, Который стал впереди меня, потому что Он был прежде меня.
\par 31 Я не знал Его; но для того пришел крестить в воде, чтобы Он явлен был Израилю.
\par 32 И свидетельствовал Иоанн, говоря: я видел Духа, сходящего с неба, как голубя, и пребывающего на Нем.
\par 33 Я не знал Его; но Пославший меня крестить в воде сказал мне: на Кого увидишь Духа сходящего и пребывающего на Нем, Тот есть крестящий Духом Святым.
\par 34 И я видел и засвидетельствовал, что Сей есть Сын Божий.
\par 35 На другой день опять стоял Иоанн и двое из учеников его.
\par 36 И, увидев идущего Иисуса, сказал: вот Агнец Божий.
\par 37 Услышав от него сии слова, оба ученика пошли за Иисусом.
\par 38 Иисус же, обратившись и увидев их идущих, говорит им: что вам надобно? Они сказали Ему: Равви, --что значит: учитель, --где живешь?
\par 39 Говорит им: пойдите и увидите. Они пошли и увидели, где Он живет; и пробыли у Него день тот. Было около десятого часа.
\par 40 Один из двух, слышавших от Иоанна [об Иисусе] и последовавших за Ним, был Андрей, брат Симона Петра.
\par 41 Он первый находит брата своего Симона и говорит ему: мы нашли Мессию, что значит: Христос;
\par 42 и привел его к Иисусу. Иисус же, взглянув на него, сказал: ты--Симон, сын Ионин; ты наречешься Кифа, что значит: камень (Петр).
\par 43 На другой день [Иисус] восхотел идти в Галилею, и находит Филиппа и говорит ему: иди за Мною.
\par 44 Филипп же был из Вифсаиды, из [одного] города с Андреем и Петром.
\par 45 Филипп находит Нафанаила и говорит ему: мы нашли Того, о Котором писали Моисей в законе и пророки, Иисуса, сына Иосифова, из Назарета.
\par 46 Но Нафанаил сказал ему: из Назарета может ли быть что доброе? Филипп говорит ему: пойди и посмотри.
\par 47 Иисус, увидев идущего к Нему Нафанаила, говорит о нем: вот подлинно Израильтянин, в котором нет лукавства.
\par 48 Нафанаил говорит Ему: почему Ты знаешь меня? Иисус сказал ему в ответ: прежде нежели позвал тебя Филипп, когда ты был под смоковницею, Я видел тебя.
\par 49 Нафанаил отвечал Ему: Равви! Ты Сын Божий, Ты Царь Израилев.
\par 50 Иисус сказал ему в ответ: ты веришь, потому что Я тебе сказал: Я видел тебя под смоковницею; увидишь больше сего.
\par 51 И говорит ему: истинно, истинно говорю вам: отныне будете видеть небо отверстым и Ангелов Божиих восходящих и нисходящих к Сыну Человеческому.

\chapter{2}

\par 1 На третий день был брак в Кане Галилейской, и Матерь Иисуса была там.
\par 2 Был также зван Иисус и ученики Его на брак.
\par 3 И как недоставало вина, то Матерь Иисуса говорит Ему: вина нет у них.
\par 4 Иисус говорит Ей: что Мне и Тебе, Жено? еще не пришел час Мой.
\par 5 Матерь Его сказала служителям: что скажет Он вам, то сделайте.
\par 6 Было же тут шесть каменных водоносов, стоявших [по обычаю] очищения Иудейского, вмещавших по две или по три меры.
\par 7 Иисус говорит им: наполните сосуды водою. И наполнили их до верха.
\par 8 И говорит им: теперь почерпните и несите к распорядителю пира. И понесли.
\par 9 Когда же распорядитель отведал воды, сделавшейся вином, --а он не знал, откуда [это вино], знали только служители, почерпавшие воду, --тогда распорядитель зовет жениха
\par 10 и говорит ему: всякий человек подает сперва хорошее вино, а когда напьются, тогда худшее; а ты хорошее вино сберег доселе.
\par 11 Так положил Иисус начало чудесам в Кане Галилейской и явил славу Свою; и уверовали в Него ученики Его.
\par 12 После сего пришел Он в Капернаум, Сам и Матерь Его, и братья его, и ученики Его; и там пробыли немного дней.
\par 13 Приближалась Пасха Иудейская, и Иисус пришел в Иерусалим
\par 14 и нашел, что в храме продавали волов, овец и голубей, и сидели меновщики денег.
\par 15 И, сделав бич из веревок, выгнал из храма всех, [также] и овец и волов; и деньги у меновщиков рассыпал, а столы их опрокинул.
\par 16 И сказал продающим голубей: возьмите это отсюда и дома Отца Моего не делайте домом торговли.
\par 17 При сем ученики Его вспомнили, что написано: ревность по доме Твоем снедает Меня.
\par 18 На это Иудеи сказали: каким знамением докажешь Ты нам, что [имеешь] [власть] так поступать?
\par 19 Иисус сказал им в ответ: разрушьте храм сей, и Я в три дня воздвигну его.
\par 20 На это сказали Иудеи: сей храм строился сорок шесть лет, и Ты в три дня воздвигнешь его?
\par 21 А Он говорил о храме тела Своего.
\par 22 Когда же воскрес Он из мертвых, то ученики Его вспомнили, что Он говорил это, и поверили Писанию и слову, которое сказал Иисус.
\par 23 И когда Он был в Иерусалиме на празднике Пасхи, то многие, видя чудеса, которые Он творил, уверовали во имя Его.
\par 24 Но Сам Иисус не вверял Себя им, потому что знал всех
\par 25 и не имел нужды, чтобы кто засвидетельствовал о человеке, ибо Сам знал, что в человеке.

\chapter{3}

\par 1 Между фарисеями был некто, именем Никодим, [один] из начальников Иудейских.
\par 2 Он пришел к Иисусу ночью и сказал Ему: Равви! мы знаем, что Ты учитель, пришедший от Бога; ибо таких чудес, какие Ты творишь, никто не может творить, если не будет с ним Бог.
\par 3 Иисус сказал ему в ответ: истинно, истинно говорю тебе, если кто не родится свыше, не может увидеть Царствия Божия.
\par 4 Никодим говорит Ему: как может человек родиться, будучи стар? неужели может он в другой раз войти в утробу матери своей и родиться?
\par 5 Иисус отвечал: истинно, истинно говорю тебе, если кто не родится от воды и Духа, не может войти в Царствие Божие.
\par 6 Рожденное от плоти есть плоть, а рожденное от Духа есть дух.
\par 7 Не удивляйся тому, что Я сказал тебе: должно вам родиться свыше.
\par 8 Дух дышит, где хочет, и голос его слышишь, а не знаешь, откуда приходит и куда уходит: так бывает со всяким, рожденным от Духа.
\par 9 Никодим сказал Ему в ответ: как это может быть?
\par 10 Иисус отвечал и сказал ему: ты--учитель Израилев, и этого ли не знаешь?
\par 11 Истинно, истинно говорю тебе: мы говорим о том, что знаем, и свидетельствуем о том, что видели, а вы свидетельства Нашего не принимаете.
\par 12 Если Я сказал вам о земном, и вы не верите, --как поверите, если буду говорить вам о небесном?
\par 13 Никто не восходил на небо, как только сшедший с небес Сын Человеческий, сущий на небесах.
\par 14 И как Моисей вознес змию в пустыне, так должно вознесену быть Сыну Человеческому,
\par 15 дабы всякий, верующий в Него, не погиб, но имел жизнь вечную.
\par 16 Ибо так возлюбил Бог мир, что отдал Сына Своего Единородного, дабы всякий верующий в Него, не погиб, но имел жизнь вечную.
\par 17 Ибо не послал Бог Сына Своего в мир, чтобы судить мир, но чтобы мир спасен был чрез Него.
\par 18 Верующий в Него не судится, а неверующий уже осужден, потому что не уверовал во имя Единородного Сына Божия.
\par 19 Суд же состоит в том, что свет пришел в мир; но люди более возлюбили тьму, нежели свет, потому что дела их были злы;
\par 20 ибо всякий, делающий злое, ненавидит свет и не идет к свету, чтобы не обличились дела его, потому что они злы,
\par 21 а поступающий по правде идет к свету, дабы явны были дела его, потому что они в Боге соделаны.
\par 22 После сего пришел Иисус с учениками Своими в землю Иудейскую и там жил с ними и крестил.
\par 23 А Иоанн также крестил в Еноне, близ Салима, потому что там было много воды; и приходили [туда] и крестились,
\par 24 ибо Иоанн еще не был заключен в темницу.
\par 25 Тогда у Иоанновых учеников произошел спор с Иудеями об очищении.
\par 26 И пришли к Иоанну и сказали ему: равви! Тот, Который был с тобою при Иордане и о Котором ты свидетельствовал, вот Он крестит, и все идут к Нему.
\par 27 Иоанн сказал в ответ: не может человек ничего принимать [на] [себя], если не будет дано ему с неба.
\par 28 Вы сами мне свидетели в том, что я сказал: не я Христос, но я послан пред Ним.
\par 29 Имеющий невесту есть жених, а друг жениха, стоящий и внимающий ему, радостью радуется, слыша голос жениха. Сия-то радость моя исполнилась.
\par 30 Ему должно расти, а мне умаляться.
\par 31 Приходящий свыше и есть выше всех; а сущий от земли земной и есть и говорит, как сущий от земли; Приходящий с небес есть выше всех,
\par 32 и что Он видел и слышал, о том и свидетельствует; и никто не принимает свидетельства Его.
\par 33 Принявший Его свидетельство сим запечатлел, что Бог истинен,
\par 34 ибо Тот, Которого послал Бог, говорит слова Божии; ибо не мерою дает Бог Духа.
\par 35 Отец любит Сына и все дал в руку Его.
\par 36 Верующий в Сына имеет жизнь вечную, а не верующий в Сына не увидит жизни, но гнев Божий пребывает на нем.

\chapter{4}

\par 1 Когда же узнал Иисус о [дошедшем до] фарисеев слухе, что Он более приобретает учеников и крестит, нежели Иоанн, --
\par 2 хотя Сам Иисус не крестил, а ученики Его, --
\par 3 то оставил Иудею и пошел опять в Галилею.
\par 4 Надлежало же Ему проходить через Самарию.
\par 5 Итак приходит Он в город Самарийский, называемый Сихарь, близ участка земли, данного Иаковом сыну своему Иосифу.
\par 6 Там был колодезь Иаковлев. Иисус, утрудившись от пути, сел у колодезя. Было около шестого часа.
\par 7 Приходит женщина из Самарии почерпнуть воды. Иисус говорит ей: дай Мне пить.
\par 8 Ибо ученики Его отлучились в город купить пищи.
\par 9 Женщина Самарянская говорит Ему: как ты, будучи Иудей, просишь пить у меня, Самарянки? ибо Иудеи с Самарянами не сообщаются.
\par 10 Иисус сказал ей в ответ: если бы ты знала дар Божий и Кто говорит тебе: дай Мне пить, то ты сама просила бы у Него, и Он дал бы тебе воду живую.
\par 11 Женщина говорит Ему: господин! тебе и почерпнуть нечем, а колодезь глубок; откуда же у тебя вода живая?
\par 12 Неужели ты больше отца нашего Иакова, который дал нам этот колодезь и сам из него пил, и дети его, и скот его?
\par 13 Иисус сказал ей в ответ: всякий, пьющий воду сию, возжаждет опять,
\par 14 а кто будет пить воду, которую Я дам ему, тот не будет жаждать вовек; но вода, которую Я дам ему, сделается в нем источником воды, текущей в жизнь вечную.
\par 15 Женщина говорит Ему: господин! дай мне этой воды, чтобы мне не иметь жажды и не приходить сюда черпать.
\par 16 Иисус говорит ей: пойди, позови мужа твоего и приди сюда.
\par 17 Женщина сказала в ответ: у меня нет мужа. Иисус говорит ей: правду ты сказала, что у тебя нет мужа,
\par 18 ибо у тебя было пять мужей, и тот, которого ныне имеешь, не муж тебе; это справедливо ты сказала.
\par 19 Женщина говорит Ему: Господи! вижу, что Ты пророк.
\par 20 Отцы наши поклонялись на этой горе, а вы говорите, что место, где должно поклоняться, находится в Иерусалиме.
\par 21 Иисус говорит ей: поверь Мне, что наступает время, когда и не на горе сей, и не в Иерусалиме будете поклоняться Отцу.
\par 22 Вы не знаете, чему кланяетесь, а мы знаем, чему кланяемся, ибо спасение от Иудеев.
\par 23 Но настанет время и настало уже, когда истинные поклонники будут поклоняться Отцу в духе и истине, ибо таких поклонников Отец ищет Себе.
\par 24 Бог есть дух, и поклоняющиеся Ему должны поклоняться в духе и истине.
\par 25 Женщина говорит Ему: знаю, что придет Мессия, то есть Христос; когда Он придет, то возвестит нам все.
\par 26 Иисус говорит ей: это Я, Который говорю с тобою.
\par 27 В это время пришли ученики Его, и удивились, что Он разговаривал с женщиною; однакож ни один не сказал: чего Ты требуешь? или: о чем говоришь с нею?
\par 28 Тогда женщина оставила водонос свой и пошла в город, и говорит людям:
\par 29 пойдите, посмотрите Человека, Который сказал мне все, что я сделала: не Он ли Христос?
\par 30 Они вышли из города и пошли к Нему.
\par 31 Между тем ученики просили Его, говоря: Равви! ешь.
\par 32 Но Он сказал им: у Меня есть пища, которой вы не знаете.
\par 33 Посему ученики говорили между собою: разве кто принес Ему есть?
\par 34 Иисус говорит им: Моя пища есть творить волю Пославшего Меня и совершить дело Его.
\par 35 Не говорите ли вы, что еще четыре месяца, и наступит жатва? А Я говорю вам: возведите очи ваши и посмотрите на нивы, как они побелели и поспели к жатве.
\par 36 Жнущий получает награду и собирает плод в жизнь вечную, так что и сеющий и жнущий вместе радоваться будут,
\par 37 ибо в этом случае справедливо изречение: один сеет, а другой жнет.
\par 38 Я послал вас жать то, над чем вы не трудились: другие трудились, а вы вошли в труд их.
\par 39 И многие Самаряне из города того уверовали в Него по слову женщины, свидетельствовавшей, что Он сказал ей все, что она сделала.
\par 40 И потому, когда пришли к Нему Самаряне, то просили Его побыть у них; и Он пробыл там два дня.
\par 41 И еще большее число уверовали по Его слову.
\par 42 А женщине той говорили: уже не по твоим речам веруем, ибо сами слышали и узнали, что Он истинно Спаситель мира, Христос.
\par 43 По прошествии же двух дней Он вышел оттуда и пошел в Галилею,
\par 44 ибо Сам Иисус свидетельствовал, что пророк не имеет чести в своем отечестве.
\par 45 Когда пришел Он в Галилею, то Галилеяне приняли Его, видев все, что Он сделал в Иерусалиме в праздник, --ибо и они ходили на праздник.
\par 46 Итак Иисус опять пришел в Кану Галилейскую, где претворил воду в вино. В Капернауме был некоторый царедворец, у которого сын был болен.
\par 47 Он, услышав, что Иисус пришел из Иудеи в Галилею, пришел к Нему и просил Его придти и исцелить сына его, который был при смерти.
\par 48 Иисус сказал ему: вы не уверуете, если не увидите знамений и чудес.
\par 49 Царедворец говорит Ему: Господи! приди, пока не умер сын мой.
\par 50 Иисус говорит ему: пойди, сын твой здоров. Он поверил слову, которое сказал ему Иисус, и пошел.
\par 51 На дороге встретили его слуги его и сказали: сын твой здоров.
\par 52 Он спросил у них: в котором часу стало ему легче? Ему сказали: вчера в седьмом часу горячка оставила его.
\par 53 Из этого отец узнал, что это был тот час, в который Иисус сказал ему: сын твой здоров, и уверовал сам и весь дом его.
\par 54 Это второе чудо сотворил Иисус, возвратившись из Иудеи в Галилею.

\chapter{5}

\par 1 После сего был праздник Иудейский, и пришел Иисус в Иерусалим.
\par 2 Есть же в Иерусалиме у Овечьих [ворот] купальня, называемая по-еврейски Вифезда, при которой было пять крытых ходов.
\par 3 В них лежало великое множество больных, слепых, хромых, иссохших, ожидающих движения воды,
\par 4 ибо Ангел Господень по временам сходил в купальню и возмущал воду, и кто первый входил [в нее] по возмущении воды, тот выздоравливал, какою бы ни был одержим болезнью.
\par 5 Тут был человек, находившийся в болезни тридцать восемь лет.
\par 6 Иисус, увидев его лежащего и узнав, что он лежит уже долгое время, говорит ему: хочешь ли быть здоров?
\par 7 Больной отвечал Ему: так, Господи; но не имею человека, который опустил бы меня в купальню, когда возмутится вода; когда же я прихожу, другой уже сходит прежде меня.
\par 8 Иисус говорит ему: встань, возьми постель твою и ходи.
\par 9 И он тотчас выздоровел, и взял постель свою и пошел. Было же это в день субботний.
\par 10 Посему Иудеи говорили исцеленному: сегодня суббота; не должно тебе брать постели.
\par 11 Он отвечал им: Кто меня исцелил, Тот мне сказал: возьми постель твою и ходи.
\par 12 Его спросили: кто Тот Человек, Который сказал тебе: возьми постель твою и ходи?
\par 13 Исцеленный же не знал, кто Он, ибо Иисус скрылся в народе, бывшем на том месте.
\par 14 Потом Иисус встретил его в храме и сказал ему: вот, ты выздоровел; не греши больше, чтобы не случилось с тобою чего хуже.
\par 15 Человек сей пошел и объявил Иудеям, что исцеливший его есть Иисус.
\par 16 И стали Иудеи гнать Иисуса и искали убить Его за то, что Он делал такие [дела] в субботу.
\par 17 Иисус же говорил им: Отец Мой доныне делает, и Я делаю.
\par 18 И еще более искали убить Его Иудеи за то, что Он не только нарушал субботу, но и Отцем Своим называл Бога, делая Себя равным Богу.
\par 19 На это Иисус сказал: истинно, истинно говорю вам: Сын ничего не может творить Сам от Себя, если не увидит Отца творящего: ибо, что творит Он, то и Сын творит также.
\par 20 Ибо Отец любит Сына и показывает Ему все, что творит Сам; и покажет Ему дела больше сих, так что вы удивитесь.
\par 21 Ибо, как Отец воскрешает мертвых и оживляет, так и Сын оживляет, кого хочет.
\par 22 Ибо Отец и не судит никого, но весь суд отдал Сыну,
\par 23 дабы все чтили Сына, как чтут Отца. Кто не чтит Сына, тот не чтит и Отца, пославшего Его.
\par 24 Истинно, истинно говорю вам: слушающий слово Мое и верующий в Пославшего Меня имеет жизнь вечную, и на суд не приходит, но перешел от смерти в жизнь.
\par 25 Истинно, истинно говорю вам: наступает время, и настало уже, когда мертвые услышат глас Сына Божия и, услышав, оживут.
\par 26 Ибо, как Отец имеет жизнь в Самом Себе, так и Сыну дал иметь жизнь в Самом Себе.
\par 27 И дал Ему власть производить и суд, потому что Он есть Сын Человеческий.
\par 28 Не дивитесь сему; ибо наступает время, в которое все, находящиеся в гробах, услышат глас Сына Божия;
\par 29 и изыдут творившие добро в воскресение жизни, а делавшие зло--в воскресение осуждения.
\par 30 Я ничего не могу творить Сам от Себя. Как слышу, так и сужу, и суд Мой праведен; ибо не ищу Моей воли, но воли пославшего Меня Отца.
\par 31 Если Я свидетельствую Сам о Себе, то свидетельство Мое не есть истинно.
\par 32 Есть другой, свидетельствующий о Мне; и Я знаю, что истинно то свидетельство, которым он свидетельствует о Мне.
\par 33 Вы посылали к Иоанну, и он засвидетельствовал об истине.
\par 34 Впрочем Я не от человека принимаю свидетельство, но говорю это для того, чтобы вы спаслись.
\par 35 Он был светильник, горящий и светящий; а вы хотели малое время порадоваться при свете его.
\par 36 Я же имею свидетельство больше Иоаннова: ибо дела, которые Отец дал Мне совершить, самые дела сии, Мною творимые, свидетельствуют о Мне, что Отец послал Меня.
\par 37 И пославший Меня Отец Сам засвидетельствовал о Мне. А вы ни гласа Его никогда не слышали, ни лица Его не видели;
\par 38 и не имеете слова Его пребывающего в вас, потому что вы не веруете Тому, Которого Он послал.
\par 39 Исследуйте Писания, ибо вы думаете чрез них иметь жизнь вечную; а они свидетельствуют о Мне.
\par 40 Но вы не хотите придти ко Мне, чтобы иметь жизнь.
\par 41 Не принимаю славы от человеков,
\par 42 но знаю вас: вы не имеете в себе любви к Богу.
\par 43 Я пришел во имя Отца Моего, и не принимаете Меня; а если иной придет во имя свое, его примете.
\par 44 Как вы можете веровать, когда друг от друга принимаете славу, а славы, которая от Единого Бога, не ищете?
\par 45 Не думайте, что Я буду обвинять вас пред Отцем: есть на вас обвинитель Моисей, на которого вы уповаете.
\par 46 Ибо если бы вы верили Моисею, то поверили бы и Мне, потому что он писал о Мне.
\par 47 Если же его писаниям не верите, как поверите Моим словам?

\chapter{6}

\par 1 После сего пошел Иисус на ту сторону моря Галилейского, [в] [окрестности] Тивериады.
\par 2 За Ним последовало множество народа, потому что видели чудеса, которые Он творил над больными.
\par 3 Иисус взошел на гору и там сидел с учениками Своими.
\par 4 Приближалась же Пасха, праздник Иудейский.
\par 5 Иисус, возведя очи и увидев, что множество народа идет к Нему, говорит Филиппу: где нам купить хлебов, чтобы их накормить?
\par 6 Говорил же это, испытывая его; ибо Сам знал, что хотел сделать.
\par 7 Филипп отвечал Ему: им на двести динариев не довольно будет хлеба, чтобы каждому из них досталось хотя понемногу.
\par 8 Один из учеников Его, Андрей, брат Симона Петра, говорит Ему:
\par 9 здесь есть у одного мальчика пять хлебов ячменных и две рыбки; но что это для такого множества?
\par 10 Иисус сказал: велите им возлечь. Было же на том месте много травы. Итак возлегло людей числом около пяти тысяч.
\par 11 Иисус, взяв хлебы и воздав благодарение, роздал ученикам, а ученики возлежавшим, также и рыбы, сколько кто хотел.
\par 12 И когда насытились, то сказал ученикам Своим: соберите оставшиеся куски, чтобы ничего не пропало.
\par 13 И собрали, и наполнили двенадцать коробов кусками от пяти ячменных хлебов, оставшимися у тех, которые ели.
\par 14 Тогда люди, видевшие чудо, сотворенное Иисусом, сказали: это истинно Тот Пророк, Которому должно придти в мир.
\par 15 Иисус же, узнав, что хотят придти, нечаянно взять его и сделать царем, опять удалился на гору один.
\par 16 Когда же настал вечер, то ученики Его сошли к морю
\par 17 и, войдя в лодку, отправились на ту сторону моря, в Капернаум. Становилось темно, а Иисус не приходил к ним.
\par 18 Дул сильный ветер, и море волновалось.
\par 19 Проплыв около двадцати пяти или тридцати стадий, они увидели Иисуса, идущего по морю и приближающегося к лодке, и испугались.
\par 20 Но Он сказал им: это Я; не бойтесь.
\par 21 Они хотели принять Его в лодку; и тотчас лодка пристала к берегу, куда плыли.
\par 22 На другой день народ, стоявший по ту сторону моря, видел, что там, кроме одной лодки, в которую вошли ученики Его, иной не было, и что Иисус не входил в лодку с учениками Своими, а отплыли одни ученики Его.
\par 23 Между тем пришли из Тивериады другие лодки близко к тому месту, где ели хлеб по благословении Господнем.
\par 24 Итак, когда народ увидел, что тут нет Иисуса, ни учеников Его, то вошли в лодки и приплыли в Капернаум, ища Иисуса.
\par 25 И, найдя Его на той стороне моря, сказали Ему: Равви! когда Ты сюда пришел?
\par 26 Иисус сказал им в ответ: истинно, истинно говорю вам: вы ищете Меня не потому, что видели чудеса, но потому, что ели хлеб и насытились.
\par 27 Старайтесь не о пище тленной, но о пище, пребывающей в жизнь вечную, которую даст вам Сын Человеческий, ибо на Нем положил печать [Свою] Отец, Бог.
\par 28 Итак сказали Ему: что нам делать, чтобы творить дела Божии?
\par 29 Иисус сказал им в ответ: вот дело Божие, чтобы вы веровали в Того, Кого Он послал.
\par 30 На это сказали Ему: какое же Ты дашь знамение, чтобы мы увидели и поверили Тебе? что Ты делаешь?
\par 31 Отцы наши ели манну в пустыне, как написано: хлеб с неба дал им есть.
\par 32 Иисус же сказал им: истинно, истинно говорю вам: не Моисей дал вам хлеб с неба, а Отец Мой дает вам истинный хлеб с небес.
\par 33 Ибо хлеб Божий есть тот, который сходит с небес и дает жизнь миру.
\par 34 На это сказали Ему: Господи! подавай нам всегда такой хлеб.
\par 35 Иисус же сказал им: Я есмь хлеб жизни; приходящий ко Мне не будет алкать, и верующий в Меня не будет жаждать никогда.
\par 36 Но Я сказал вам, что вы и видели Меня, и не веруете.
\par 37 Все, что дает Мне Отец, ко Мне придет; и приходящего ко Мне не изгоню вон,
\par 38 ибо Я сошел с небес не для того, чтобы творить волю Мою, но волю пославшего Меня Отца.
\par 39 Воля же пославшего Меня Отца есть та, чтобы из того, что Он Мне дал, ничего не погубить, но все то воскресить в последний день.
\par 40 Воля Пославшего Меня есть та, чтобы всякий, видящий Сына и верующий в Него, имел жизнь вечную; и Я воскрешу его в последний день.
\par 41 Возроптали на Него Иудеи за то, что Он сказал: Я есмь хлеб, сшедший с небес.
\par 42 И говорили: не Иисус ли это, сын Иосифов, Которого отца и Мать мы знаем? Как же говорит Он: я сшел с небес?
\par 43 Иисус сказал им в ответ: не ропщите между собою.
\par 44 Никто не может придти ко Мне, если не привлечет его Отец, пославший Меня; и Я воскрешу его в последний день.
\par 45 У пророков написано: и будут все научены Богом. Всякий, слышавший от Отца и научившийся, приходит ко Мне.
\par 46 Это не то, чтобы кто видел Отца, кроме Того, Кто есть от Бога; Он видел Отца.
\par 47 Истинно, истинно говорю вам: верующий в Меня имеет жизнь вечную.
\par 48 Я есмь хлеб жизни.
\par 49 Отцы ваши ели манну в пустыне и умерли;
\par 50 хлеб же, сходящий с небес, таков, что ядущий его не умрет.
\par 51 Я хлеб живый, сшедший с небес; ядущий хлеб сей будет жить вовек; хлеб же, который Я дам, есть Плоть Моя, которую Я отдам за жизнь мира.
\par 52 Тогда Иудеи стали спорить между собою, говоря: как Он может дать нам есть Плоть Свою?
\par 53 Иисус же сказал им: истинно, истинно говорю вам: если не будете есть Плоти Сына Человеческого и пить Крови Его, то не будете иметь в себе жизни.
\par 54 Ядущий Мою Плоть и пиющий Мою Кровь имеет жизнь вечную, и Я воскрешу его в последний день.
\par 55 Ибо Плоть Моя истинно есть пища, и Кровь Моя истинно есть питие.
\par 56 Ядущий Мою Плоть и пиющий Мою Кровь пребывает во Мне, и Я в нем.
\par 57 Как послал Меня живый Отец, и Я живу Отцем, [так] и ядущий Меня жить будет Мною.
\par 58 Сей-то есть хлеб, сшедший с небес. Не так, как отцы ваши ели манну и умерли: ядущий хлеб сей жить будет вовек.
\par 59 Сие говорил Он в синагоге, уча в Капернауме.
\par 60 Многие из учеников Его, слыша то, говорили: какие странные слова! кто может это слушать?
\par 61 Но Иисус, зная Сам в Себе, что ученики Его ропщут на то, сказал им: это ли соблазняет вас?
\par 62 Что ж, если увидите Сына Человеческого восходящего [туда], где был прежде?
\par 63 Дух животворит; плоть не пользует нимало. Слова, которые говорю Я вам, суть дух и жизнь.
\par 64 Но есть из вас некоторые неверующие. Ибо Иисус от начала знал, кто суть неверующие и кто предаст Его.
\par 65 И сказал: для того-то и говорил Я вам, что никто не может придти ко Мне, если то не дано будет ему от Отца Моего.
\par 66 С этого времени многие из учеников Его отошли от Него и уже не ходили с Ним.
\par 67 Тогда Иисус сказал двенадцати: не хотите ли и вы отойти?
\par 68 Симон Петр отвечал Ему: Господи! к кому нам идти? Ты имеешь глаголы вечной жизни:
\par 69 и мы уверовали и познали, что Ты Христос, Сын Бога живаго.
\par 70 Иисус отвечал им: не двенадцать ли вас избрал Я? но один из вас диавол.
\par 71 Это говорил Он об Иуде Симонове Искариоте, ибо сей хотел предать Его, будучи один из двенадцати.

\chapter{7}

\par 1 После сего Иисус ходил по Галилее, ибо по Иудее не хотел ходить, потому что Иудеи искали убить Его.
\par 2 Приближался праздник Иудейский--поставление кущей.
\par 3 Тогда братья Его сказали Ему: выйди отсюда и пойди в Иудею, чтобы и ученики Твои видели дела, которые Ты делаешь.
\par 4 Ибо никто не делает чего-либо втайне, и ищет сам быть известным. Если Ты творишь такие дела, то яви Себя миру.
\par 5 Ибо и братья Его не веровали в Него.
\par 6 На это Иисус сказал им: Мое время еще не настало, а для вас всегда время.
\par 7 Вас мир не может ненавидеть, а Меня ненавидит, потому что Я свидетельствую о нем, что дела его злы.
\par 8 Вы пойдите на праздник сей; а Я еще не пойду на сей праздник, потому что Мое время еще не исполнилось.
\par 9 Сие сказав им, остался в Галилее.
\par 10 Но когда пришли братья Его, тогда и Он пришел на праздник не явно, а как бы тайно.
\par 11 Иудеи же искали Его на празднике и говорили: где Он?
\par 12 И много толков было о Нем в народе: одни говорили, что Он добр; а другие говорили: нет, но обольщает народ.
\par 13 Впрочем никто не говорил о Нем явно, боясь Иудеев.
\par 14 Но в половине уже праздника вошел Иисус в храм и учил.
\par 15 И дивились Иудеи, говоря: как Он знает Писания, не учившись?
\par 16 Иисус, отвечая им, сказал: Мое учение--не Мое, но Пославшего Меня;
\par 17 кто хочет творить волю Его, тот узнает о сем учении, от Бога ли оно, или Я Сам от Себя говорю.
\par 18 Говорящий сам от себя ищет славы себе; а Кто ищет славы Пославшему Его, Тот истинен, и нет неправды в Нем.
\par 19 Не дал ли вам Моисей закона? и никто из вас не поступает по закону. За что ищете убить Меня?
\par 20 Народ сказал в ответ: не бес ли в Тебе? кто ищет убить Тебя?
\par 21 Иисус, продолжая речь, сказал им: одно дело сделал Я, и все вы дивитесь.
\par 22 Моисей дал вам обрезание--хотя оно не от Моисея, но от отцов, --и в субботу вы обрезываете человека.
\par 23 Если в субботу принимает человек обрезание, чтобы не был нарушен закон Моисеев, --на Меня ли негодуете за то, что Я всего человека исцелил в субботу?
\par 24 Не судите по наружности, но судите судом праведным.
\par 25 Тут некоторые из Иерусалимлян говорили: не Тот ли это, Которого ищут убить?
\par 26 Вот, Он говорит явно, и ничего не говорят Ему: не удостоверились ли начальники, что Он подлинно Христос?
\par 27 Но мы знаем Его, откуда Он; Христос же когда придет, никто не будет знать, откуда Он.
\par 28 Тогда Иисус возгласил в храме, уча и говоря: и знаете Меня, и знаете, откуда Я; и Я пришел не Сам от Себя, но истинен Пославший Меня, Которого вы не знаете.
\par 29 Я знаю Его, потому что Я от Него, и Он послал Меня.
\par 30 И искали схватить Его, но никто не наложил на Него руки, потому что еще не пришел час Его.
\par 31 Многие же из народа уверовали в Него и говорили: когда придет Христос, неужели сотворит больше знамений, нежели сколько Сей сотворил?
\par 32 Услышали фарисеи такие толки о Нем в народе, и послали фарисеи и первосвященники служителей--схватить Его.
\par 33 Иисус же сказал им: еще недолго быть Мне с вами, и пойду к Пославшему Меня;
\par 34 будете искать Меня, и не найдете; и где буду Я, [туда] вы не можете придти.
\par 35 При сем Иудеи говорили между собою: куда Он хочет идти, так что мы не найдем Его? Не хочет ли Он идти в Еллинское рассеяние и учить Еллинов?
\par 36 Что значат сии слова, которые Он сказал: будете искать Меня, и не найдете; и где буду Я, [туда] вы не можете придти?
\par 37 В последний же великий день праздника стоял Иисус и возгласил, говоря: кто жаждет, иди ко Мне и пей.
\par 38 Кто верует в Меня, у того, как сказано в Писании, из чрева потекут реки воды живой.
\par 39 Сие сказал Он о Духе, Которого имели принять верующие в Него: ибо еще не было на них Духа Святаго, потому что Иисус еще не был прославлен.
\par 40 Многие из народа, услышав сии слова, говорили: Он точно пророк.
\par 41 Другие говорили: это Христос. А иные говорили: разве из Галилеи Христос придет?
\par 42 Не сказано ли в Писании, что Христос придет от семени Давидова и из Вифлеема, из того места, откуда был Давид?
\par 43 Итак произошла о Нем распря в народе.
\par 44 Некоторые из них хотели схватить Его; но никто не наложил на Него рук.
\par 45 Итак служители возвратились к первосвященникам и фарисеям, и сии сказали им: для чего вы не привели Его?
\par 46 Служители отвечали: никогда человек не говорил так, как Этот Человек.
\par 47 Фарисеи сказали им: неужели и вы прельстились?
\par 48 Уверовал ли в Него кто из начальников, или из фарисеев?
\par 49 Но этот народ невежда в законе, проклят он.
\par 50 Никодим, приходивший к Нему ночью, будучи один из них, говорит им:
\par 51 судит ли закон наш человека, если прежде не выслушают его и не узнают, что он делает?
\par 52 На это сказали ему: и ты не из Галилеи ли? рассмотри и увидишь, что из Галилеи не приходит пророк.
\par 53 И разошлись все по домам.

\chapter{8}

\par 1 Иисус же пошел на гору Елеонскую.
\par 2 А утром опять пришел в храм, и весь народ шел к Нему. Он сел и учил их.
\par 3 Тут книжники и фарисеи привели к Нему женщину, взятую в прелюбодеянии, и, поставив ее посреди,
\par 4 сказали Ему: Учитель! эта женщина взята в прелюбодеянии;
\par 5 а Моисей в законе заповедал нам побивать таких камнями: Ты что скажешь?
\par 6 Говорили же это, искушая Его, чтобы найти что-нибудь к обвинению Его. Но Иисус, наклонившись низко, писал перстом на земле, не обращая на них внимания.
\par 7 Когда же продолжали спрашивать Его, Он, восклонившись, сказал им: кто из вас без греха, первый брось на нее камень.
\par 8 И опять, наклонившись низко, писал на земле.
\par 9 Они же, услышав [то] и будучи обличаемы совестью, стали уходить один за другим, начиная от старших до последних; и остался один Иисус и женщина, стоящая посреди.
\par 10 Иисус, восклонившись и не видя никого, кроме женщины, сказал ей: женщина! где твои обвинители? никто не осудил тебя?
\par 11 Она отвечала: никто, Господи. Иисус сказал ей: и Я не осуждаю тебя; иди и впредь не греши.
\par 12 Опять говорил Иисус [к народу] и сказал им: Я свет миру; кто последует за Мною, тот не будет ходить во тьме, но будет иметь свет жизни.
\par 13 Тогда фарисеи сказали Ему: Ты Сам о Себе свидетельствуешь, свидетельство Твое не истинно.
\par 14 Иисус сказал им в ответ: если Я и Сам о Себе свидетельствую, свидетельство Мое истинно; потому что Я знаю, откуда пришел и куда иду; а вы не знаете, откуда Я и куда иду.
\par 15 Вы судите по плоти; Я не сужу никого.
\par 16 А если и сужу Я, то суд Мой истинен, потому что Я не один, но Я и Отец, пославший Меня.
\par 17 А и в законе вашем написано, что двух человек свидетельство истинно.
\par 18 Я Сам свидетельствую о Себе, и свидетельствует о Мне Отец, пославший Меня.
\par 19 Тогда сказали Ему: где Твой Отец? Иисус отвечал: вы не знаете ни Меня, ни Отца Моего; если бы вы знали Меня, то знали бы и Отца Моего.
\par 20 Сии слова говорил Иисус у сокровищницы, когда учил в храме; и никто не взял Его, потому что еще не пришел час Его.
\par 21 Опять сказал им Иисус: Я отхожу, и будете искать Меня, и умрете во грехе вашем. Куда Я иду, [туда] вы не можете придти.
\par 22 Тут Иудеи говорили: неужели Он убьет Сам Себя, что говорит: `куда Я иду, вы не можете придти'?
\par 23 Он сказал им: вы от нижних, Я от вышних; вы от мира сего, Я не от сего мира.
\par 24 Потому Я и сказал вам, что вы умрете во грехах ваших; ибо если не уверуете, что это Я, то умрете во грехах ваших.
\par 25 Тогда сказали Ему: кто же Ты? Иисус сказал им: от начала Сущий, как и говорю вам.
\par 26 Много имею говорить и судить о вас; но Пославший Меня есть истинен, и что Я слышал от Него, то и говорю миру.
\par 27 Не поняли, что Он говорил им об Отце.
\par 28 Итак Иисус сказал им: когда вознесете Сына Человеческого, тогда узнаете, что это Я и что ничего не делаю от Себя, но как научил Меня Отец Мой, так и говорю.
\par 29 Пославший Меня есть со Мною; Отец не оставил Меня одного, ибо Я всегда делаю то, что Ему угодно.
\par 30 Когда Он говорил это, многие уверовали в Него.
\par 31 Тогда сказал Иисус к уверовавшим в Него Иудеям: если пребудете в слове Моем, то вы истинно Мои ученики,
\par 32 и познаете истину, и истина сделает вас свободными.
\par 33 Ему отвечали: мы семя Авраамово и не были рабами никому никогда; как же Ты говоришь: сделаетесь свободными?
\par 34 Иисус отвечал им: истинно, истинно говорю вам: всякий, делающий грех, есть раб греха.
\par 35 Но раб не пребывает в доме вечно; сын пребывает вечно.
\par 36 Итак, если Сын освободит вас, то истинно свободны будете.
\par 37 Знаю, что вы семя Авраамово; однако ищете убить Меня, потому что слово Мое не вмещается в вас.
\par 38 Я говорю то, что видел у Отца Моего; а вы делаете то, что видели у отца вашего.
\par 39 Сказали Ему в ответ: отец наш есть Авраам. Иисус сказал им: если бы вы были дети Авраама, то дела Авраамовы делали бы.
\par 40 А теперь ищете убить Меня, Человека, сказавшего вам истину, которую слышал от Бога: Авраам этого не делал.
\par 41 Вы делаете дела отца вашего. На это сказали Ему: мы не от любодеяния рождены; одного Отца имеем, Бога.
\par 42 Иисус сказал им: если бы Бог был Отец ваш, то вы любили бы Меня, потому что Я от Бога исшел и пришел; ибо Я не Сам от Себя пришел, но Он послал Меня.
\par 43 Почему вы не понимаете речи Моей? Потому что не можете слышать слова Моего.
\par 44 Ваш отец диавол; и вы хотите исполнять похоти отца вашего. Он был человекоубийца от начала и не устоял в истине, ибо нет в нем истины. Когда говорит он ложь, говорит свое, ибо он лжец и отец лжи.
\par 45 А как Я истину говорю, то не верите Мне.
\par 46 Кто из вас обличит Меня в неправде? Если же Я говорю истину, почему вы не верите Мне?
\par 47 Кто от Бога, тот слушает слова Божии. Вы потому не слушаете, что вы не от Бога.
\par 48 На это Иудеи отвечали и сказали Ему: не правду ли мы говорим, что Ты Самарянин и что бес в Тебе?
\par 49 Иисус отвечал: во Мне беса нет; но Я чту Отца Моего, а вы бесчестите Меня.
\par 50 Впрочем Я не ищу Моей славы: есть Ищущий и Судящий.
\par 51 Истинно, истинно говорю вам: кто соблюдет слово Мое, тот не увидит смерти вовек.
\par 52 Иудеи сказали Ему: теперь узнали мы, что бес в Тебе. Авраам умер и пророки, а Ты говоришь: кто соблюдет слово Мое, тот не вкусит смерти вовек.
\par 53 Неужели Ты больше отца нашего Авраама, который умер? и пророки умерли: чем Ты Себя делаешь?
\par 54 Иисус отвечал: если Я Сам Себя славлю, то слава Моя ничто. Меня прославляет Отец Мой, о Котором вы говорите, что Он Бог ваш.
\par 55 И вы не познали Его, а Я знаю Его; и если скажу, что не знаю Его, то буду подобный вам лжец. Но Я знаю Его и соблюдаю слово Его.
\par 56 Авраам, отец ваш, рад был увидеть день Мой; и увидел и возрадовался.
\par 57 На это сказали Ему Иудеи: Тебе нет еще пятидесяти лет, --и Ты видел Авраама?
\par 58 Иисус сказал им: истинно, истинно говорю вам: прежде нежели был Авраам, Я есмь.
\par 59 Тогда взяли каменья, чтобы бросить на Него; но Иисус скрылся и вышел из храма, пройдя посреди них, и пошел далее.

\chapter{9}

\par 1 И, проходя, увидел человека, слепого от рождения.
\par 2 Ученики Его спросили у Него: Равви! кто согрешил, он или родители его, что родился слепым?
\par 3 Иисус отвечал: не согрешил ни он, ни родители его, но [это для] [того], чтобы на нем явились дела Божии.
\par 4 Мне должно делать дела Пославшего Меня, доколе есть день; приходит ночь, когда никто не может делать.
\par 5 Доколе Я в мире, Я свет миру.
\par 6 Сказав это, Он плюнул на землю, сделал брение из плюновения и помазал брением глаза слепому,
\par 7 и сказал ему: пойди, умойся в купальне Силоам, что значит: посланный. Он пошел и умылся, и пришел зрячим.
\par 8 Тут соседи и видевшие прежде, что он был слеп, говорили: не тот ли это, который сидел и просил милостыни?
\par 9 Иные говорили: это он, а иные: похож на него. Он же говорил: это я.
\par 10 Тогда спрашивали у него: как открылись у тебя глаза?
\par 11 Он сказал в ответ: Человек, называемый Иисус, сделал брение, помазал глаза мои и сказал мне: пойди на купальню Силоам и умойся. Я пошел, умылся и прозрел.
\par 12 Тогда сказали ему: где Он? Он отвечал: не знаю.
\par 13 Повели сего бывшего слепца к фарисеям.
\par 14 А была суббота, когда Иисус сделал брение и отверз ему очи.
\par 15 Спросили его также и фарисеи, как он прозрел. Он сказал им: брение положил Он на мои глаза, и я умылся, и вижу.
\par 16 Тогда некоторые из фарисеев говорили: не от Бога Этот Человек, потому что не хранит субботы. Другие говорили: как может человек грешный творить такие чудеса? И была между ними распря.
\par 17 Опять говорят слепому: ты что скажешь о Нем, потому что Он отверз тебе очи? Он сказал: это пророк.
\par 18 Тогда Иудеи не поверили, что он был слеп и прозрел, доколе не призвали родителей сего прозревшего
\par 19 и спросили их: это ли сын ваш, о котором вы говорите, что родился слепым? как же он теперь видит?
\par 20 Родители его сказали им в ответ: мы знаем, что это сын наш и что он родился слепым,
\par 21 а как теперь видит, не знаем, или кто отверз ему очи, мы не знаем. Сам в совершенных летах; самого спросите; пусть сам о себе скажет.
\par 22 Так отвечали родители его, потому что боялись Иудеев; ибо Иудеи сговорились уже, чтобы, кто признает Его за Христа, того отлучать от синагоги.
\par 23 Посему-то родители его и сказали: он в совершенных летах; самого спросите.
\par 24 Итак, вторично призвали человека, который был слеп, и сказали ему: воздай славу Богу; мы знаем, что Человек Тот грешник.
\par 25 Он сказал им в ответ: грешник ли Он, не знаю; одно знаю, что я был слеп, а теперь вижу.
\par 26 Снова спросили его: что сделал Он с тобою? как отверз твои очи?
\par 27 Отвечал им: я уже сказал вам, и вы не слушали; что еще хотите слышать? или и вы хотите сделаться Его учениками?
\par 28 Они же укорили его и сказали: ты ученик Его, а мы Моисеевы ученики.
\par 29 Мы знаем, что с Моисеем говорил Бог; Сего же не знаем, откуда Он.
\par 30 Человек [прозревший] сказал им в ответ: это и удивительно, что вы не знаете, откуда Он, а Он отверз мне очи.
\par 31 Но мы знаем, что грешников Бог не слушает; но кто чтит Бога и творит волю Его, того слушает.
\par 32 От века не слыхано, чтобы кто отверз очи слепорожденному.
\par 33 Если бы Он не был от Бога, не мог бы творить ничего.
\par 34 Сказали ему в ответ: во грехах ты весь родился, и ты ли нас учишь? И выгнали его вон.
\par 35 Иисус, услышав, что выгнали его вон, и найдя его, сказал ему: ты веруешь ли в Сына Божия?
\par 36 Он отвечал и сказал: а кто Он, Господи, чтобы мне веровать в Него?
\par 37 Иисус сказал ему: и видел ты Его, и Он говорит с тобою.
\par 38 Он же сказал: верую, Господи! И поклонился Ему.
\par 39 И сказал Иисус: на суд пришел Я в мир сей, чтобы невидящие видели, а видящие стали слепы.
\par 40 Услышав это, некоторые из фарисеев, бывших с Ним, сказали Ему: неужели и мы слепы?
\par 41 Иисус сказал им: если бы вы были слепы, то не имели бы [на] [себе] греха; но как вы говорите, что видите, то грех остается на вас.

\chapter{10}

\par 1 Истинно, истинно говорю вам: кто не дверью входит во двор овчий, но перелазит инуде, тот вор и разбойник;
\par 2 а входящий дверью есть пастырь овцам.
\par 3 Ему придверник отворяет, и овцы слушаются голоса его, и он зовет своих овец по имени и выводит их.
\par 4 И когда выведет своих овец, идет перед ними; а овцы за ним идут, потому что знают голос его.
\par 5 За чужим же не идут, но бегут от него, потому что не знают чужого голоса.
\par 6 Сию притчу сказал им Иисус; но они не поняли, что такое Он говорил им.
\par 7 Итак, опять Иисус сказал им: истинно, истинно говорю вам, что Я дверь овцам.
\par 8 Все, сколько их ни приходило предо Мною, суть воры и разбойники; но овцы не послушали их.
\par 9 Я есмь дверь: кто войдет Мною, тот спасется, и войдет, и выйдет, и пажить найдет.
\par 10 Вор приходит только для того, чтобы украсть, убить и погубить. Я пришел для того, чтобы имели жизнь и имели с избытком.
\par 11 Я есмь пастырь добрый: пастырь добрый полагает жизнь свою за овец.
\par 12 А наемник, не пастырь, которому овцы не свои, видит приходящего волка, и оставляет овец, и бежит; и волк расхищает овец, и разгоняет их.
\par 13 А наемник бежит, потому что наемник, и нерадит об овцах.
\par 14 Я есмь пастырь добрый; и знаю Моих, и Мои знают Меня.
\par 15 Как Отец знает Меня, [так] и Я знаю Отца; и жизнь Мою полагаю за овец.
\par 16 Есть у Меня и другие овцы, которые не сего двора, и тех надлежит Мне привести: и они услышат голос Мой, и будет одно стадо и один Пастырь.
\par 17 Потому любит Меня Отец, что Я отдаю жизнь Мою, чтобы опять принять ее.
\par 18 Никто не отнимает ее у Меня, но Я Сам отдаю ее. Имею власть отдать ее и власть имею опять принять ее. Сию заповедь получил Я от Отца Моего.
\par 19 От этих слов опять произошла между Иудеями распря.
\par 20 Многие из них говорили: Он одержим бесом и безумствует; что слушаете Его?
\par 21 Другие говорили: это слова не бесноватого; может ли бес отверзать очи слепым?
\par 22 Настал же тогда в Иерусалиме [праздник] обновления, и была зима.
\par 23 И ходил Иисус в храме, в притворе Соломоновом.
\par 24 Тут Иудеи обступили Его и говорили Ему: долго ли Тебе держать нас в недоумении? если Ты Христос, скажи нам прямо.
\par 25 Иисус отвечал им: Я сказал вам, и не верите; дела, которые творю Я во имя Отца Моего, они свидетельствуют о Мне.
\par 26 Но вы не верите, ибо вы не из овец Моих, как Я сказал вам.
\par 27 Овцы Мои слушаются голоса Моего, и Я знаю их; и они идут за Мною.
\par 28 И Я даю им жизнь вечную, и не погибнут вовек; и никто не похитит их из руки Моей.
\par 29 Отец Мой, Который дал Мне их, больше всех; и никто не может похитить их из руки Отца Моего.
\par 30 Я и Отец--одно.
\par 31 Тут опять Иудеи схватили каменья, чтобы побить Его.
\par 32 Иисус отвечал им: много добрых дел показал Я вам от Отца Моего; за которое из них хотите побить Меня камнями?
\par 33 Иудеи сказали Ему в ответ: не за доброе дело хотим побить Тебя камнями, но за богохульство и за то, что Ты, будучи человек, делаешь Себя Богом.
\par 34 Иисус отвечал им: не написано ли в законе вашем: Я сказал: вы боги?
\par 35 Если Он назвал богами тех, к которым было слово Божие, и не может нарушиться Писание, --
\par 36 Тому ли, Которого Отец освятил и послал в мир, вы говорите: богохульствуешь, потому что Я сказал: Я Сын Божий?
\par 37 Если Я не творю дел Отца Моего, не верьте Мне;
\par 38 а если творю, то, когда не верите Мне, верьте делам Моим, чтобы узнать и поверить, что Отец во Мне и Я в Нем.
\par 39 Тогда опять искали схватить Его; но Он уклонился от рук их,
\par 40 и пошел опять за Иордан, на то место, где прежде крестил Иоанн, и остался там.
\par 41 Многие пришли к Нему и говорили, что Иоанн не сотворил никакого чуда, но все, что сказал Иоанн о Нем, было истинно.
\par 42 И многие там уверовали в Него.

\chapter{11}

\par 1 Был болен некто Лазарь из Вифании, из селения, [где жили] Мария и Марфа, сестра ее.
\par 2 Мария же, которой брат Лазарь был болен, была [та], которая помазала Господа миром и отерла ноги Его волосами своими.
\par 3 Сестры послали сказать Ему: Господи! вот, кого Ты любишь, болен.
\par 4 Иисус, услышав [то], сказал: эта болезнь не к смерти, но к славе Божией, да прославится через нее Сын Божий.
\par 5 Иисус же любил Марфу и сестру ее и Лазаря.
\par 6 Когда же услышал, что он болен, то пробыл два дня на том месте, где находился.
\par 7 После этого сказал ученикам: пойдем опять в Иудею.
\par 8 Ученики сказали Ему: Равви! давно ли Иудеи искали побить Тебя камнями, и Ты опять идешь туда?
\par 9 Иисус отвечал: не двенадцать ли часов во дне? кто ходит днем, тот не спотыкается, потому что видит свет мира сего;
\par 10 а кто ходит ночью, спотыкается, потому что нет света с ним.
\par 11 Сказав это, говорит им потом: Лазарь, друг наш, уснул; но Я иду разбудить его.
\par 12 Ученики Его сказали: Господи! если уснул, то выздоровеет.
\par 13 Иисус говорил о смерти его, а они думали, что Он говорит о сне обыкновенном.
\par 14 Тогда Иисус сказал им прямо: Лазарь умер;
\par 15 и радуюсь за вас, что Меня не было там, дабы вы уверовали; но пойдем к нему.
\par 16 Тогда Фома, иначе называемый Близнец, сказал ученикам: пойдем и мы умрем с ним.
\par 17 Иисус, придя, нашел, что он уже четыре дня в гробе.
\par 18 Вифания же была близ Иерусалима, стадиях в пятнадцати;
\par 19 и многие из Иудеев пришли к Марфе и Марии утешать их [в] [печали] о брате их.
\par 20 Марфа, услышав, что идет Иисус, пошла навстречу Ему; Мария же сидела дома.
\par 21 Тогда Марфа сказала Иисусу: Господи! если бы Ты был здесь, не умер бы брат мой.
\par 22 Но и теперь знаю, что чего Ты попросишь у Бога, даст Тебе Бог.
\par 23 Иисус говорит ей: воскреснет брат твой.
\par 24 Марфа сказала Ему: знаю, что воскреснет в воскресение, в последний день.
\par 25 Иисус сказал ей: Я есмь воскресение и жизнь; верующий в Меня, если и умрет, оживет.
\par 26 И всякий, живущий и верующий в Меня, не умрет вовек. Веришь ли сему?
\par 27 Она говорит Ему: так, Господи! я верую, что Ты Христос, Сын Божий, грядущий в мир.
\par 28 Сказав это, пошла и позвала тайно Марию, сестру свою, говоря: Учитель здесь и зовет тебя.
\par 29 Она, как скоро услышала, поспешно встала и пошла к Нему.
\par 30 Иисус еще не входил в селение, но был на том месте, где встретила Его Марфа.
\par 31 Иудеи, которые были с нею в доме и утешали ее, видя, что Мария поспешно встала и вышла, пошли за нею, полагая, что она пошла на гроб--плакать там.
\par 32 Мария же, придя туда, где был Иисус, и увидев Его, пала к ногам Его и сказала Ему: Господи! если бы Ты был здесь, не умер бы брат мой.
\par 33 Иисус, когда увидел ее плачущую и пришедших с нею Иудеев плачущих, Сам восскорбел духом и возмутился
\par 34 и сказал: где вы положили его? Говорят Ему: Господи! пойди и посмотри.
\par 35 Иисус прослезился.
\par 36 Тогда Иудеи говорили: смотри, как Он любил его.
\par 37 А некоторые из них сказали: не мог ли Сей, отверзший очи слепому, сделать, чтобы и этот не умер?
\par 38 Иисус же, опять скорбя внутренно, приходит ко гробу. То была пещера, и камень лежал на ней.
\par 39 Иисус говорит: отнимите камень. Сестра умершего, Марфа, говорит Ему: Господи! уже смердит; ибо четыре дня, как он во гробе.
\par 40 Иисус говорит ей: не сказал ли Я тебе, что, если будешь веровать, увидишь славу Божию?
\par 41 Итак отняли камень [от пещеры], где лежал умерший. Иисус же возвел очи к небу и сказал: Отче! благодарю Тебя, что Ты услышал Меня.
\par 42 Я и знал, что Ты всегда услышишь Меня; но сказал [сие] для народа, здесь стоящего, чтобы поверили, что Ты послал Меня.
\par 43 Сказав это, Он воззвал громким голосом: Лазарь! иди вон.
\par 44 И вышел умерший, обвитый по рукам и ногам погребальными пеленами, и лице его обвязано было платком. Иисус говорит им: развяжите его, пусть идет.
\par 45 Тогда многие из Иудеев, пришедших к Марии и видевших, что сотворил Иисус, уверовали в Него.
\par 46 А некоторые из них пошли к фарисеям и сказали им, что сделал Иисус.
\par 47 Тогда первосвященники и фарисеи собрали совет и говорили: что нам делать? Этот Человек много чудес творит.
\par 48 Если оставим Его так, то все уверуют в Него, и придут Римляне и овладеют и местом нашим и народом.
\par 49 Один же из них, некто Каиафа, будучи на тот год первосвященником, сказал им: вы ничего не знаете,
\par 50 и не подумаете, что лучше нам, чтобы один человек умер за людей, нежели чтобы весь народ погиб.
\par 51 Сие же он сказал не от себя, но, будучи на тот год первосвященником, предсказал, что Иисус умрет за народ,
\par 52 и не только за народ, но чтобы и рассеянных чад Божиих собрать воедино.
\par 53 С этого дня положили убить Его.
\par 54 Посему Иисус уже не ходил явно между Иудеями, а пошел оттуда в страну близ пустыни, в город, называемый Ефраим, и там оставался с учениками Своими.
\par 55 Приближалась Пасха Иудейская, и многие из всей страны пришли в Иерусалим перед Пасхою, чтобы очиститься.
\par 56 Тогда искали Иисуса и, стоя в храме, говорили друг другу: как вы думаете? не придет ли Он на праздник?
\par 57 Первосвященники же и фарисеи дали приказание, что если кто узнает, где Он будет, то объявил бы, дабы взять Его.

\chapter{12}

\par 1 За шесть дней до Пасхи пришел Иисус в Вифанию, где был Лазарь умерший, которого Он воскресил из мертвых.
\par 2 Там приготовили Ему вечерю, и Марфа служила, и Лазарь был одним из возлежавших с Ним.
\par 3 Мария же, взяв фунт нардового чистого драгоценного мира, помазала ноги Иисуса и отерла волосами своими ноги Его; и дом наполнился благоуханием от мира.
\par 4 Тогда один из учеников Его, Иуда Симонов Искариот, который хотел предать Его, сказал:
\par 5 Для чего бы не продать это миро за триста динариев и не раздать нищим?
\par 6 Сказал же он это не потому, чтобы заботился о нищих, но потому что был вор. Он имел [при себе денежный] ящик и носил, что туда опускали.
\par 7 Иисус же сказал: оставьте ее; она сберегла это на день погребения Моего.
\par 8 Ибо нищих всегда имеете с собою, а Меня не всегда.
\par 9 Многие из Иудеев узнали, что Он там, и пришли не только для Иисуса, но чтобы видеть и Лазаря, которого Он воскресил из мертвых.
\par 10 Первосвященники же положили убить и Лазаря,
\par 11 потому что ради него многие из Иудеев приходили и веровали в Иисуса.
\par 12 На другой день множество народа, пришедшего на праздник, услышав, что Иисус идет в Иерусалим,
\par 13 взяли пальмовые ветви, вышли навстречу Ему и восклицали: осанна! благословен грядущий во имя Господне, Царь Израилев!
\par 14 Иисус же, найдя молодого осла, сел на него, как написано:
\par 15 Не бойся, дщерь Сионова! се, Царь твой грядет, сидя на молодом осле.
\par 16 Ученики Его сперва не поняли этого; но когда прославился Иисус, тогда вспомнили, что так было о Нем написано, и это сделали Ему.
\par 17 Народ, бывший с Ним прежде, свидетельствовал, что Он вызвал из гроба Лазаря и воскресил его из мертвых.
\par 18 Потому и встретил Его народ, ибо слышал, что Он сотворил это чудо.
\par 19 Фарисеи же говорили между собою: видите ли, что не успеваете ничего? весь мир идет за Ним.
\par 20 Из пришедших на поклонение в праздник были некоторые Еллины.
\par 21 Они подошли к Филиппу, который был из Вифсаиды Галилейской, и просили его, говоря: господин! нам хочется видеть Иисуса.
\par 22 Филипп идет и говорит о том Андрею; и потом Андрей и Филипп сказывают о том Иисусу.
\par 23 Иисус же сказал им в ответ: пришел час прославиться Сыну Человеческому.
\par 24 Истинно, истинно говорю вам: если пшеничное зерно, пав в землю, не умрет, то останется одно; а если умрет, то принесет много плода.
\par 25 Любящий душу свою погубит ее; а ненавидящий душу свою в мире сем сохранит ее в жизнь вечную.
\par 26 Кто Мне служит, Мне да последует; и где Я, там и слуга Мой будет. И кто Мне служит, того почтит Отец Мой.
\par 27 Душа Моя теперь возмутилась; и что Мне сказать? Отче! избавь Меня от часа сего! Но на сей час Я и пришел.
\par 28 Отче! прославь имя Твое. Тогда пришел с неба глас: и прославил и еще прославлю.
\par 29 Народ, стоявший и слышавший [то], говорил: это гром; а другие говорили: Ангел говорил Ему.
\par 30 Иисус на это сказал: не для Меня был глас сей, но для народа.
\par 31 Ныне суд миру сему; ныне князь мира сего изгнан будет вон.
\par 32 И когда Я вознесен буду от земли, всех привлеку к Себе.
\par 33 Сие говорил Он, давая разуметь, какою смертью Он умрет.
\par 34 Народ отвечал Ему: мы слышали из закона, что Христос пребывает вовек; как же Ты говоришь, что должно вознесену быть Сыну Человеческому? кто Этот Сын Человеческий?
\par 35 Тогда Иисус сказал им: еще на малое время свет есть с вами; ходите, пока есть свет, чтобы не объяла вас тьма: а ходящий во тьме не знает, куда идет.
\par 36 Доколе свет с вами, веруйте в свет, да будете сынами света. Сказав это, Иисус отошел и скрылся от них.
\par 37 Столько чудес сотворил Он пред ними, и они не веровали в Него,
\par 38 да сбудется слово Исаии пророка: Господи! кто поверил слышанному от нас? и кому открылась мышца Господня?
\par 39 Потому не могли они веровать, что, как еще сказал Исаия,
\par 40 народ сей ослепил глаза свои и окаменил сердце свое, да не видят глазами, и не уразумеют сердцем, и не обратятся, чтобы Я исцелил их.
\par 41 Сие сказал Исаия, когда видел славу Его и говорил о Нем.
\par 42 Впрочем и из начальников многие уверовали в Него; но ради фарисеев не исповедывали, чтобы не быть отлученными от синагоги,
\par 43 ибо возлюбили больше славу человеческую, нежели славу Божию.
\par 44 Иисус же возгласил и сказал: верующий в Меня не в Меня верует, но в Пославшего Меня.
\par 45 И видящий Меня видит Пославшего Меня.
\par 46 Я свет пришел в мир, чтобы всякий верующий в Меня не оставался во тьме.
\par 47 И если кто услышит Мои слова и не поверит, Я не сужу его, ибо Я пришел не судить мир, но спасти мир.
\par 48 Отвергающий Меня и не принимающий слов Моих имеет судью себе: слово, которое Я говорил, оно будет судить его в последний день.
\par 49 Ибо Я говорил не от Себя; но пославший Меня Отец, Он дал Мне заповедь, что сказать и что говорить.
\par 50 И Я знаю, что заповедь Его есть жизнь вечная. Итак, что Я говорю, говорю, как сказал Мне Отец.

\chapter{13}

\par 1 Перед праздником Пасхи Иисус, зная, что пришел час Его перейти от мира сего к Отцу, [явил делом, что], возлюбив Своих сущих в мире, до конца возлюбил их.
\par 2 И во время вечери, когда диавол уже вложил в сердце Иуде Симонову Искариоту предать Его,
\par 3 Иисус, зная, что Отец все отдал в руки Его, и что Он от Бога исшел и к Богу отходит,
\par 4 встал с вечери, снял [с Себя верхнюю] одежду и, взяв полотенце, препоясался.
\par 5 Потом влил воды в умывальницу и начал умывать ноги ученикам и отирать полотенцем, которым был препоясан.
\par 6 Подходит к Симону Петру, и тот говорит Ему: Господи! Тебе ли умывать мои ноги?
\par 7 Иисус сказал ему в ответ: что Я делаю, теперь ты не знаешь, а уразумеешь после.
\par 8 Петр говорит Ему: не умоешь ног моих вовек. Иисус отвечал ему: если не умою тебя, не имеешь части со Мною.
\par 9 Симон Петр говорит Ему: Господи! не только ноги мои, но и руки и голову.
\par 10 Иисус говорит ему: омытому нужно только ноги умыть, потому что чист весь; и вы чисты, но не все.
\par 11 Ибо знал Он предателя Своего, потому [и] сказал: не все вы чисты.
\par 12 Когда же умыл им ноги и надел одежду Свою, то, возлегши опять, сказал им: знаете ли, что Я сделал вам?
\par 13 Вы называете Меня Учителем и Господом, и правильно говорите, ибо Я точно то.
\par 14 Итак, если Я, Господь и Учитель, умыл ноги вам, то и вы должны умывать ноги друг другу.
\par 15 Ибо Я дал вам пример, чтобы и вы делали то же, что Я сделал вам.
\par 16 Истинно, истинно говорю вам: раб не больше господина своего, и посланник не больше пославшего его.
\par 17 Если это знаете, блаженны вы, когда исполняете.
\par 18 Не о всех вас говорю; Я знаю, которых избрал. Но да сбудется Писание: ядущий со Мною хлеб поднял на Меня пяту свою.
\par 19 Теперь сказываю вам, прежде нежели [то] сбылось, дабы, когда сбудется, вы поверили, что это Я.
\par 20 Истинно, истинно говорю вам: принимающий того, кого Я пошлю, Меня принимает; а принимающий Меня принимает Пославшего Меня.
\par 21 Сказав это, Иисус возмутился духом, и засвидетельствовал, и сказал: истинно, истинно говорю вам, что один из вас предаст Меня.
\par 22 Тогда ученики озирались друг на друга, недоумевая, о ком Он говорит.
\par 23 Один же из учеников Его, которого любил Иисус, возлежал у груди Иисуса.
\par 24 Ему Симон Петр сделал знак, чтобы спросил, кто это, о котором говорит.
\par 25 Он, припав к груди Иисуса, сказал Ему: Господи! кто это?
\par 26 Иисус отвечал: тот, кому Я, обмакнув кусок хлеба, подам. И, обмакнув кусок, подал Иуде Симонову Искариоту.
\par 27 И после сего куска вошел в него сатана. Тогда Иисус сказал ему: что делаешь, делай скорее.
\par 28 Но никто из возлежавших не понял, к чему Он это сказал ему.
\par 29 А как у Иуды был ящик, то некоторые думали, что Иисус говорит ему: купи, что нам нужно к празднику, или чтобы дал что-- нибудь нищим.
\par 30 Он, приняв кусок, тотчас вышел; а была ночь.
\par 31 Когда он вышел, Иисус сказал: ныне прославился Сын Человеческий, и Бог прославился в Нем.
\par 32 Если Бог прославился в Нем, то и Бог прославит Его в Себе, и вскоре прославит Его.
\par 33 Дети! недолго уже быть Мне с вами. Будете искать Меня, и, как сказал Я Иудеям, что, куда Я иду, вы не можете придти, [так] и вам говорю теперь.
\par 34 Заповедь новую даю вам, да любите друг друга; как Я возлюбил вас, [так] и вы да любите друг друга.
\par 35 По тому узнают все, что вы Мои ученики, если будете иметь любовь между собою.
\par 36 Симон Петр сказал Ему: Господи! куда Ты идешь? Иисус отвечал ему: куда Я иду, ты не можешь теперь за Мною идти, а после пойдешь за Мною.
\par 37 Петр сказал Ему: Господи! почему я не могу идти за Тобою теперь? я душу мою положу за Тебя.
\par 38 Иисус отвечал ему: душу твою за Меня положишь? истинно, истинно говорю тебе: не пропоет петух, как отречешься от Меня трижды.

\chapter{14}

\par 1 Да не смущается сердце ваше; веруйте в Бога, и в Меня веруйте.
\par 2 В доме Отца Моего обителей много. А если бы не так, Я сказал бы вам: Я иду приготовить место вам.
\par 3 И когда пойду и приготовлю вам место, приду опять и возьму вас к Себе, чтобы и вы были, где Я.
\par 4 А куда Я иду, вы знаете, и путь знаете.
\par 5 Фома сказал Ему: Господи! не знаем, куда идешь; и как можем знать путь?
\par 6 Иисус сказал ему: Я есмь путь и истина и жизнь; никто не приходит к Отцу, как только через Меня.
\par 7 Если бы вы знали Меня, то знали бы и Отца Моего. И отныне знаете Его и видели Его.
\par 8 Филипп сказал Ему: Господи! покажи нам Отца, и довольно для нас.
\par 9 Иисус сказал ему: столько времени Я с вами, и ты не знаешь Меня, Филипп? Видевший Меня видел Отца; как же ты говоришь, покажи нам Отца?
\par 10 Разве ты не веришь, что Я в Отце и Отец во Мне? Слова, которые говорю Я вам, говорю не от Себя; Отец, пребывающий во Мне, Он творит дела.
\par 11 Верьте Мне, что Я в Отце и Отец во Мне; а если не так, то верьте Мне по самым делам.
\par 12 Истинно, истинно говорю вам: верующий в Меня, дела, которые творю Я, и он сотворит, и больше сих сотворит, потому что Я к Отцу Моему иду.
\par 13 И если чего попросите у Отца во имя Мое, то сделаю, да прославится Отец в Сыне.
\par 14 Если чего попросите во имя Мое, Я то сделаю.
\par 15 Если любите Меня, соблюдите Мои заповеди.
\par 16 И Я умолю Отца, и даст вам другого Утешителя, да пребудет с вами вовек,
\par 17 Духа истины, Которого мир не может принять, потому что не видит Его и не знает Его; а вы знаете Его, ибо Он с вами пребывает и в вас будет.
\par 18 Не оставлю вас сиротами; приду к вам.
\par 19 Еще немного, и мир уже не увидит Меня; а вы увидите Меня, ибо Я живу, и вы будете жить.
\par 20 В тот день узнаете вы, что Я в Отце Моем, и вы во Мне, и Я в вас.
\par 21 Кто имеет заповеди Мои и соблюдает их, тот любит Меня; а кто любит Меня, тот возлюблен будет Отцем Моим; и Я возлюблю его и явлюсь ему Сам.
\par 22 Иуда--не Искариот--говорит Ему: Господи! что это, что Ты хочешь явить Себя нам, а не миру?
\par 23 Иисус сказал ему в ответ: кто любит Меня, тот соблюдет слово Мое; и Отец Мой возлюбит его, и Мы придем к нему и обитель у него сотворим.
\par 24 Нелюбящий Меня не соблюдает слов Моих; слово же, которое вы слышите, не есть Мое, но пославшего Меня Отца.
\par 25 Сие сказал Я вам, находясь с вами.
\par 26 Утешитель же, Дух Святый, Которого пошлет Отец во имя Мое, научит вас всему и напомнит вам все, что Я говорил вам.
\par 27 Мир оставляю вам, мир Мой даю вам; не так, как мир дает, Я даю вам. Да не смущается сердце ваше и да не устрашается.
\par 28 Вы слышали, что Я сказал вам: иду от вас и приду к вам. Если бы вы любили Меня, то возрадовались бы, что Я сказал: иду к Отцу; ибо Отец Мой более Меня.
\par 29 И вот, Я сказал вам [о том], прежде нежели сбылось, дабы вы поверили, когда сбудется.
\par 30 Уже немного Мне говорить с вами; ибо идет князь мира сего, и во Мне не имеет ничего.
\par 31 Но чтобы мир знал, что Я люблю Отца и, как заповедал Мне Отец, так и творю: встаньте, пойдем отсюда.

\chapter{15}

\par 1 Я есмь истинная виноградная лоза, а Отец Мой--виноградарь.
\par 2 Всякую у Меня ветвь, не приносящую плода, Он отсекает; и всякую, приносящую плод, очищает, чтобы более принесла плода.
\par 3 Вы уже очищены через слово, которое Я проповедал вам.
\par 4 Пребудьте во Мне, и Я в вас. Как ветвь не может приносить плода сама собою, если не будет на лозе: так и вы, если не будете во Мне.
\par 5 Я есмь лоза, а вы ветви; кто пребывает во Мне, и Я в нем, тот приносит много плода; ибо без Меня не можете делать ничего.
\par 6 Кто не пребудет во Мне, извергнется вон, как ветвь, и засохнет; а такие [ветви] собирают и бросают в огонь, и они сгорают.
\par 7 Если пребудете во Мне и слова Мои в вас пребудут, то, чего ни пожелаете, просите, и будет вам.
\par 8 Тем прославится Отец Мой, если вы принесете много плода и будете Моими учениками.
\par 9 Как возлюбил Меня Отец, и Я возлюбил вас; пребудьте в любви Моей.
\par 10 Если заповеди Мои соблюдете, пребудете в любви Моей, как и Я соблюл заповеди Отца Моего и пребываю в Его любви.
\par 11 Сие сказал Я вам, да радость Моя в вас пребудет и радость ваша будет совершенна.
\par 12 Сия есть заповедь Моя, да любите друг друга, как Я возлюбил вас.
\par 13 Нет больше той любви, как если кто положит душу свою за друзей своих.
\par 14 Вы друзья Мои, если исполняете то, что Я заповедую вам.
\par 15 Я уже не называю вас рабами, ибо раб не знает, что делает господин его; но Я назвал вас друзьями, потому что сказал вам все, что слышал от Отца Моего.
\par 16 Не вы Меня избрали, а Я вас избрал и поставил вас, чтобы вы шли и приносили плод, и чтобы плод ваш пребывал, дабы, чего ни попросите от Отца во имя Мое, Он дал вам.
\par 17 Сие заповедаю вам, да любите друг друга.
\par 18 Если мир вас ненавидит, знайте, что Меня прежде вас возненавидел.
\par 19 Если бы вы были от мира, то мир любил бы свое; а как вы не от мира, но Я избрал вас от мира, потому ненавидит вас мир.
\par 20 Помните слово, которое Я сказал вам: раб не больше господина своего. Если Меня гнали, будут гнать и вас; если Мое слово соблюдали, будут соблюдать и ваше.
\par 21 Но все то сделают вам за имя Мое, потому что не знают Пославшего Меня.
\par 22 Если бы Я не пришел и не говорил им, то не имели бы греха; а теперь не имеют извинения во грехе своем.
\par 23 Ненавидящий Меня ненавидит и Отца моего.
\par 24 Если бы Я не сотворил между ними дел, каких никто другой не делал, то не имели бы греха; а теперь и видели, и возненавидели и Меня и Отца Моего.
\par 25 Но да сбудется слово, написанное в законе их: возненавидели Меня напрасно.
\par 26 Когда же приидет Утешитель, Которого Я пошлю вам от Отца, Дух истины, Который от Отца исходит, Он будет свидетельствовать о Мне;
\par 27 а также и вы будете свидетельствовать, потому что вы сначала со Мною.

\chapter{16}

\par 1 Сие сказал Я вам, чтобы вы не соблазнились.
\par 2 Изгонят вас из синагог; даже наступает время, когда всякий, убивающий вас, будет думать, что он тем служит Богу.
\par 3 Так будут поступать, потому что не познали ни Отца, ни Меня.
\par 4 Но Я сказал вам сие для того, чтобы вы, когда придет то время вспомнили, что Я сказывал вам о том; не говорил же сего вам сначала, потому что был с вами.
\par 5 А теперь иду к Пославшему Меня, и никто из вас не спрашивает Меня: куда идешь?
\par 6 Но оттого, что Я сказал вам это, печалью исполнилось сердце ваше.
\par 7 Но Я истину говорю вам: лучше для вас, чтобы Я пошел; ибо, если Я не пойду, Утешитель не приидет к вам; а если пойду, то пошлю Его к вам,
\par 8 и Он, придя, обличит мир о грехе и о правде и о суде:
\par 9 о грехе, что не веруют в Меня;
\par 10 о правде, что Я иду к Отцу Моему, и уже не увидите Меня;
\par 11 о суде же, что князь мира сего осужден.
\par 12 Еще многое имею сказать вам; но вы теперь не можете вместить.
\par 13 Когда же приидет Он, Дух истины, то наставит вас на всякую истину: ибо не от Себя говорить будет, но будет говорить, что услышит, и будущее возвестит вам.
\par 14 Он прославит Меня, потому что от Моего возьмет и возвестит вам.
\par 15 Все, что имеет Отец, есть Мое; потому Я сказал, что от Моего возьмет и возвестит вам.
\par 16 Вскоре вы не увидите Меня, и опять вскоре увидите Меня, ибо Я иду к Отцу.
\par 17 Тут [некоторые] из учеников Его сказали один другому: что это Он говорит нам: вскоре не увидите Меня, и опять вскоре увидите Меня, и: Я иду к Отцу?
\par 18 Итак они говорили: что это говорит Он: `вскоре'? Не знаем, что говорит.
\par 19 Иисус, уразумев, что хотят спросить Его, сказал им: о том ли спрашиваете вы один другого, что Я сказал: вскоре не увидите Меня, и опять вскоре увидите Меня?
\par 20 Истинно, истинно говорю вам: вы восплачете и возрыдаете, а мир возрадуется; вы печальны будете, но печаль ваша в радость будет.
\par 21 Женщина, когда рождает, терпит скорбь, потому что пришел час ее; но когда родит младенца, уже не помнит скорби от радости, потому что родился человек в мир.
\par 22 Так и вы теперь имеете печаль; но Я увижу вас опять, и возрадуется сердце ваше, и радости вашей никто не отнимет у вас;
\par 23 и в тот день вы не спросите Меня ни о чем. Истинно, истинно говорю вам: о чем ни попросите Отца во имя Мое, даст вам.
\par 24 Доныне вы ничего не просили во имя Мое; просите, и получите, чтобы радость ваша была совершенна.
\par 25 Доселе Я говорил вам притчами; но наступает время, когда уже не буду говорить вам притчами, но прямо возвещу вам об Отце.
\par 26 В тот день будете просить во имя Мое, и не говорю вам, что Я буду просить Отца о вас:
\par 27 ибо Сам Отец любит вас, потому что вы возлюбили Меня и уверовали, что Я исшел от Бога.
\par 28 Я исшел от Отца и пришел в мир; и опять оставляю мир и иду к Отцу.
\par 29 Ученики Его сказали Ему: вот, теперь Ты прямо говоришь, и притчи не говоришь никакой.
\par 30 Теперь видим, что Ты знаешь все и не имеешь нужды, чтобы кто спрашивал Тебя. Посему веруем, что Ты от Бога исшел.
\par 31 Иисус отвечал им: теперь веруете?
\par 32 Вот, наступает час, и настал уже, что вы рассеетесь каждый в свою [сторону] и Меня оставите одного; но Я не один, потому что Отец со Мною.
\par 33 Сие сказал Я вам, чтобы вы имели во Мне мир. В мире будете иметь скорбь; но мужайтесь: Я победил мир.

\chapter{17}

\par 1 После сих слов Иисус возвел очи Свои на небо и сказал: Отче! пришел час, прославь Сына Твоего, да и Сын Твой прославит Тебя,
\par 2 так как Ты дал Ему власть над всякою плотью, да всему, что Ты дал Ему, даст Он жизнь вечную.
\par 3 Сия же есть жизнь вечная, да знают Тебя, единого истинного Бога, и посланного Тобою Иисуса Христа.
\par 4 Я прославил Тебя на земле, совершил дело, которое Ты поручил Мне исполнить.
\par 5 И ныне прославь Меня Ты, Отче, у Тебя Самого славою, которую Я имел у Тебя прежде бытия мира.
\par 6 Я открыл имя Твое человекам, которых Ты дал Мне от мира; они были Твои, и Ты дал их Мне, и они сохранили слово Твое.
\par 7 Ныне уразумели они, что все, что Ты дал Мне, от Тебя есть,
\par 8 ибо слова, которые Ты дал Мне, Я передал им, и они приняли, и уразумели истинно, что Я исшел от Тебя, и уверовали, что Ты послал Меня.
\par 9 Я о них молю: не о всем мире молю, но о тех, которых Ты дал Мне, потому что они Твои.
\par 10 И все Мое Твое, и Твое Мое; и Я прославился в них.
\par 11 Я уже не в мире, но они в мире, а Я к Тебе иду. Отче Святый! соблюди их во имя Твое, [тех], которых Ты Мне дал, чтобы они были едино, как и Мы.
\par 12 Когда Я был с ними в мире, Я соблюдал их во имя Твое; тех, которых Ты дал Мне, Я сохранил, и никто из них не погиб, кроме сына погибели, да сбудется Писание.
\par 13 Ныне же к Тебе иду, и сие говорю в мире, чтобы они имели в себе радость Мою совершенную.
\par 14 Я передал им слово Твое; и мир возненавидел их, потому что они не от мира, как и Я не от мира.
\par 15 Не молю, чтобы Ты взял их из мира, но чтобы сохранил их от зла.
\par 16 Они не от мира, как и Я не от мира.
\par 17 Освяти их истиною Твоею; слово Твое есть истина.
\par 18 Как Ты послал Меня в мир, [так] и Я послал их в мир.
\par 19 И за них Я посвящаю Себя, чтобы и они были освящены истиною.
\par 20 Не о них же только молю, но и о верующих в Меня по слову их,
\par 21 да будут все едино, как Ты, Отче, во Мне, и Я в Тебе, [так] и они да будут в Нас едино, --да уверует мир, что Ты послал Меня.
\par 22 И славу, которую Ты дал Мне, Я дал им: да будут едино, как Мы едино.
\par 23 Я в них, и Ты во Мне; да будут совершены воедино, и да познает мир, что Ты послал Меня и возлюбил их, как возлюбил Меня.
\par 24 Отче! которых Ты дал Мне, хочу, чтобы там, где Я, и они были со Мною, да видят славу Мою, которую Ты дал Мне, потому что возлюбил Меня прежде основания мира.
\par 25 Отче праведный! и мир Тебя не познал; а Я познал Тебя, и сии познали, что Ты послал Меня.
\par 26 И Я открыл им имя Твое и открою, да любовь, которою Ты возлюбил Меня, в них будет, и Я в них.

\chapter{18}

\par 1 Сказав сие, Иисус вышел с учениками Своими за поток Кедрон, где был сад, в который вошел Сам и ученики Его.
\par 2 Знал же это место и Иуда, предатель Его, потому что Иисус часто собирался там с учениками Своими.
\par 3 Итак Иуда, взяв отряд [воинов] и служителей от первосвященников и фарисеев, приходит туда с фонарями и светильниками и оружием.
\par 4 Иисус же, зная все, что с Ним будет, вышел и сказал им: кого ищете?
\par 5 Ему отвечали: Иисуса Назорея. Иисус говорит им: это Я. Стоял же с ними и Иуда, предатель Его.
\par 6 И когда сказал им: это Я, они отступили назад и пали на землю.
\par 7 Опять спросил их: кого ищете? Они сказали: Иисуса Назорея.
\par 8 Иисус отвечал: Я сказал вам, что это Я; итак, если Меня ищете, оставьте их, пусть идут,
\par 9 да сбудется слово, реченное Им: из тех, которых Ты Мне дал, Я не погубил никого.
\par 10 Симон же Петр, имея меч, извлек его, и ударил первосвященнического раба, и отсек ему правое ухо. Имя рабу было Малх.
\par 11 Но Иисус сказал Петру: вложи меч в ножны; неужели Мне не пить чаши, которую дал Мне Отец?
\par 12 Тогда воины и тысяченачальник и служители Иудейские взяли Иисуса и связали Его,
\par 13 и отвели Его сперва к Анне, ибо он был тесть Каиафе, который был на тот год первосвященником.
\par 14 Это был Каиафа, который подал совет Иудеям, что лучше одному человеку умереть за народ.
\par 15 За Иисусом следовали Симон Петр и другой ученик; ученик же сей был знаком первосвященнику и вошел с Иисусом во двор первосвященнический.
\par 16 А Петр стоял вне за дверями. Потом другой ученик, который был знаком первосвященнику, вышел, и сказал придвернице, и ввел Петра.
\par 17 Тут раба придверница говорит Петру: и ты не из учеников ли Этого Человека? Он сказал: нет.
\par 18 Между тем рабы и служители, разведя огонь, потому что было холодно, стояли и грелись. Петр также стоял с ними и грелся.
\par 19 Первосвященник же спросил Иисуса об учениках Его и об учении Его.
\par 20 Иисус отвечал ему: Я говорил явно миру; Я всегда учил в синагоге и в храме, где всегда Иудеи сходятся, и тайно не говорил ничего.
\par 21 Что спрашиваешь Меня? спроси слышавших, что Я говорил им; вот, они знают, что Я говорил.
\par 22 Когда Он сказал это, один из служителей, стоявший близко, ударил Иисуса по щеке, сказав: так отвечаешь Ты первосвященнику?
\par 23 Иисус отвечал ему: если Я сказал худо, покажи, что худо; а если хорошо, что ты бьешь Меня?
\par 24 Анна послал Его связанного к первосвященнику Каиафе.
\par 25 Симон же Петр стоял и грелся. Тут сказали ему: не из учеников ли Его и ты? Он отрекся и сказал: нет.
\par 26 Один из рабов первосвященнических, родственник тому, которому Петр отсек ухо, говорит: не я ли видел тебя с Ним в саду?
\par 27 Петр опять отрекся; и тотчас запел петух.
\par 28 От Каиафы повели Иисуса в преторию. Было утро; и они не вошли в преторию, чтобы не оскверниться, но чтобы [можно было] есть пасху.
\par 29 Пилат вышел к ним и сказал: в чем вы обвиняете Человека Сего?
\par 30 Они сказали ему в ответ: если бы Он не был злодей, мы не предали бы Его тебе.
\par 31 Пилат сказал им: возьмите Его вы, и по закону вашему судите Его. Иудеи сказали ему: нам не позволено предавать смерти никого, --
\par 32 да сбудется слово Иисусово, которое сказал Он, давая разуметь, какою смертью Он умрет.
\par 33 Тогда Пилат опять вошел в преторию, и призвал Иисуса, и сказал Ему: Ты Царь Иудейский?
\par 34 Иисус отвечал ему: от себя ли ты говоришь это, или другие сказали тебе о Мне?
\par 35 Пилат отвечал: разве я Иудей? Твой народ и первосвященники предали Тебя мне; что Ты сделал?
\par 36 Иисус отвечал: Царство Мое не от мира сего; если бы от мира сего было Царство Мое, то служители Мои подвизались бы за Меня, чтобы Я не был предан Иудеям; но ныне Царство Мое не отсюда.
\par 37 Пилат сказал Ему: итак Ты Царь? Иисус отвечал: ты говоришь, что Я Царь. Я на то родился и на то пришел в мир, чтобы свидетельствовать о истине; всякий, кто от истины, слушает гласа Моего.
\par 38 Пилат сказал Ему: что есть истина? И, сказав это, опять вышел к Иудеям и сказал им: я никакой вины не нахожу в Нем.
\par 39 Есть же у вас обычай, чтобы я одного отпускал вам на Пасху; хотите ли, отпущу вам Царя Иудейского?
\par 40 Тогда опять закричали все, говоря: не Его, но Варавву. Варавва же был разбойник.

\chapter{19}

\par 1 Тогда Пилат взял Иисуса и [велел] бить Его.
\par 2 И воины, сплетши венец из терна, возложили Ему на голову, и одели Его в багряницу,
\par 3 и говорили: радуйся, Царь Иудейский! и били Его по ланитам.
\par 4 Пилат опять вышел и сказал им: вот, я вывожу Его к вам, чтобы вы знали, что я не нахожу в Нем никакой вины.
\par 5 Тогда вышел Иисус в терновом венце и в багрянице. И сказал им [Пилат]: се, Человек!
\par 6 Когда же увидели Его первосвященники и служители, то закричали: распни, распни Его! Пилат говорит им: возьмите Его вы, и распните; ибо я не нахожу в Нем вины.
\par 7 Иудеи отвечали ему: мы имеем закон, и по закону нашему Он должен умереть, потому что сделал Себя Сыном Божиим.
\par 8 Пилат, услышав это слово, больше убоялся.
\par 9 И опять вошел в преторию и сказал Иисусу: откуда Ты? Но Иисус не дал ему ответа.
\par 10 Пилат говорит Ему: мне ли не отвечаешь? не знаешь ли, что я имею власть распять Тебя и власть имею отпустить Тебя?
\par 11 Иисус отвечал: ты не имел бы надо Мною никакой власти, если бы не было дано тебе свыше; посему более греха на том, кто предал Меня тебе.
\par 12 С этого [времени] Пилат искал отпустить Его. Иудеи же кричали: если отпустишь Его, ты не друг кесарю; всякий, делающий себя царем, противник кесарю.
\par 13 Пилат, услышав это слово, вывел вон Иисуса и сел на судилище, на месте, называемом Лифостротон, а по-еврейски Гаввафа.
\par 14 Тогда была пятница перед Пасхою, и час шестый. И сказал [Пилат] Иудеям: се, Царь ваш!
\par 15 Но они закричали: возьми, возьми, распни Его! Пилат говорит им: Царя ли вашего распну? Первосвященники отвечали: нет у нас царя, кроме кесаря.
\par 16 Тогда наконец он предал Его им на распятие. И взяли Иисуса и повели.
\par 17 И, неся крест Свой, Он вышел на место, называемое Лобное, по-еврейски Голгофа;
\par 18 там распяли Его и с Ним двух других, по ту и по другую сторону, а посреди Иисуса.
\par 19 Пилат же написал и надпись, и поставил на кресте. Написано было: Иисус Назорей, Царь Иудейский.
\par 20 Эту надпись читали многие из Иудеев, потому что место, где был распят Иисус, было недалеко от города, и написано было по-- еврейски, по-гречески, по-римски.
\par 21 Первосвященники же Иудейские сказали Пилату: не пиши: Царь Иудейский, но что Он говорил: Я Царь Иудейский.
\par 22 Пилат отвечал: что я написал, то написал.
\par 23 Воины же, когда распяли Иисуса, взяли одежды Его и разделили на четыре части, каждому воину по части, и хитон; хитон же был не сшитый, а весь тканый сверху.
\par 24 Итак сказали друг другу: не станем раздирать его, а бросим о нем жребий, чей будет, --да сбудется реченное в Писании: разделили ризы Мои между собою и об одежде Моей бросали жребий. Так поступили воины.
\par 25 При кресте Иисуса стояли Матерь Его и сестра Матери Его, Мария Клеопова, и Мария Магдалина.
\par 26 Иисус, увидев Матерь и ученика тут стоящего, которого любил, говорит Матери Своей: Жено! се, сын Твой.
\par 27 Потом говорит ученику: се, Матерь твоя! И с этого времени ученик сей взял Ее к себе.
\par 28 После того Иисус, зная, что уже все совершилось, да сбудется Писание, говорит: жажду.
\par 29 Тут стоял сосуд, полный уксуса. [Воины], напоив уксусом губку и наложив на иссоп, поднесли к устам Его.
\par 30 Когда же Иисус вкусил уксуса, сказал: совершилось! И, преклонив главу, предал дух.
\par 31 Но так как [тогда] была пятница, то Иудеи, дабы не оставить тел на кресте в субботу, --ибо та суббота была день великий, --просили Пилата, чтобы перебить у них голени и снять их.
\par 32 Итак пришли воины, и у первого перебили голени, и у другого, распятого с Ним.
\par 33 Но, придя к Иисусу, как увидели Его уже умершим, не перебили у Него голеней,
\par 34 но один из воинов копьем пронзил Ему ребра, и тотчас истекла кровь и вода.
\par 35 И видевший засвидетельствовал, и истинно свидетельство его; он знает, что говорит истину, дабы вы поверили.
\par 36 Ибо сие произошло, да сбудется Писание: кость Его да не сокрушится.
\par 37 Также и в другом [месте] Писание говорит: воззрят на Того, Которого пронзили.
\par 38 После сего Иосиф из Аримафеи--ученик Иисуса, но тайный из страха от Иудеев, --просил Пилата, чтобы снять тело Иисуса; и Пилат позволил. Он пошел и снял тело Иисуса.
\par 39 Пришел также и Никодим, --приходивший прежде к Иисусу ночью, --и принес состав из смирны и алоя, литр около ста.
\par 40 Итак они взяли тело Иисуса и обвили его пеленами с благовониями, как обыкновенно погребают Иудеи.
\par 41 На том месте, где Он распят, был сад, и в саду гроб новый, в котором еще никто не был положен.
\par 42 Там положили Иисуса ради пятницы Иудейской, потому что гроб был близко.

\chapter{20}

\par 1 В первый же [день] недели Мария Магдалина приходит ко гробу рано, когда было еще темно, и видит, что камень отвален от гроба.
\par 2 Итак, бежит и приходит к Симону Петру и к другому ученику, которого любил Иисус, и говорит им: унесли Господа из гроба, и не знаем, где положили Его.
\par 3 Тотчас вышел Петр и другой ученик, и пошли ко гробу.
\par 4 Они побежали оба вместе; но другой ученик бежал скорее Петра, и пришел ко гробу первый.
\par 5 И, наклонившись, увидел лежащие пелены; но не вошел [во гроб].
\par 6 Вслед за ним приходит Симон Петр, и входит во гроб, и видит одни пелены лежащие,
\par 7 и плат, который был на главе Его, не с пеленами лежащий, но особо свитый на другом месте.
\par 8 Тогда вошел и другой ученик, прежде пришедший ко гробу, и увидел, и уверовал.
\par 9 Ибо они еще не знали из Писания, что Ему надлежало воскреснуть из мертвых.
\par 10 Итак ученики опять возвратились к себе.
\par 11 А Мария стояла у гроба и плакала. И, когда плакала, наклонилась во гроб,
\par 12 и видит двух Ангелов, в белом одеянии сидящих, одного у главы и другого у ног, где лежало тело Иисуса.
\par 13 И они говорят ей: жена! что ты плачешь? Говорит им: унесли Господа моего, и не знаю, где положили Его.
\par 14 Сказав сие, обратилась назад и увидела Иисуса стоящего; но не узнала, что это Иисус.
\par 15 Иисус говорит ей: жена! что ты плачешь? кого ищешь? Она, думая, что это садовник, говорит Ему: господин! если ты вынес Его, скажи мне, где ты положил Его, и я возьму Его.
\par 16 Иисус говорит ей: Мария! Она, обратившись, говорит Ему: Раввуни! --что значит: Учитель!
\par 17 Иисус говорит ей: не прикасайся ко Мне, ибо Я еще не восшел к Отцу Моему; а иди к братьям Моим и скажи им: восхожу к Отцу Моему и Отцу вашему, и к Богу Моему и Богу вашему.
\par 18 Мария Магдалина идет и возвещает ученикам, [что] видела Господа и [что] Он это сказал ей.
\par 19 В тот же первый день недели вечером, когда двери [дома], где собирались ученики Его, были заперты из опасения от Иудеев, пришел Иисус, и стал посреди, и говорит им: мир вам!
\par 20 Сказав это, Он показал им руки и ноги и ребра Свои. Ученики обрадовались, увидев Господа.
\par 21 Иисус же сказал им вторично: мир вам! как послал Меня Отец, [так] и Я посылаю вас.
\par 22 Сказав это, дунул, и говорит им: примите Духа Святаго.
\par 23 Кому простите грехи, тому простятся; на ком оставите, на том останутся.
\par 24 Фома же, один из двенадцати, называемый Близнец, не был тут с ними, когда приходил Иисус.
\par 25 Другие ученики сказали ему: мы видели Господа. Но он сказал им: если не увижу на руках Его ран от гвоздей, и не вложу перста моего в раны от гвоздей, и не вложу руки моей в ребра Его, не поверю.
\par 26 После восьми дней опять были в доме ученики Его, и Фома с ними. Пришел Иисус, когда двери были заперты, стал посреди них и сказал: мир вам!
\par 27 Потом говорит Фоме: подай перст твой сюда и посмотри руки Мои; подай руку твою и вложи в ребра Мои; и не будь неверующим, но верующим.
\par 28 Фома сказал Ему в ответ: Господь мой и Бог мой!
\par 29 Иисус говорит ему: ты поверил, потому что увидел Меня; блаженны невидевшие и уверовавшие.
\par 30 Много сотворил Иисус пред учениками Своими и других чудес, о которых не писано в книге сей.
\par 31 Сие же написано, дабы вы уверовали, что Иисус есть Христос, Сын Божий, и, веруя, имели жизнь во имя Его.

\chapter{21}

\par 1 После того опять явился Иисус ученикам Своим при море Тивериадском. Явился же так:
\par 2 были вместе Симон Петр, и Фома, называемый Близнец, и Нафанаил из Каны Галилейской, и сыновья Зеведеевы, и двое других из учеников Его.
\par 3 Симон Петр говорит им: иду ловить рыбу. Говорят ему: идем и мы с тобою. Пошли и тотчас вошли в лодку, и не поймали в ту ночь ничего.
\par 4 А когда уже настало утро, Иисус стоял на берегу; но ученики не узнали, что это Иисус.
\par 5 Иисус говорит им: дети! есть ли у вас какая пища? Они отвечали Ему: нет.
\par 6 Он же сказал им: закиньте сеть по правую сторону лодки, и поймаете. Они закинули, и уже не могли вытащить [сети] от множества рыбы.
\par 7 Тогда ученик, которого любил Иисус, говорит Петру: это Господь. Симон же Петр, услышав, что это Господь, опоясался одеждою, --ибо он был наг, --и бросился в море.
\par 8 А другие ученики приплыли в лодке, --ибо недалеко были от земли, локтей около двухсот, --таща сеть с рыбою.
\par 9 Когда же вышли на землю, видят разложенный огонь и на нем лежащую рыбу и хлеб.
\par 10 Иисус говорит им: принесите рыбы, которую вы теперь поймали.
\par 11 Симон Петр пошел и вытащил на землю сеть, наполненную большими рыбами, [которых было] сто пятьдесят три; и при таком множестве не прорвалась сеть.
\par 12 Иисус говорит им: придите, обедайте. Из учеников же никто не смел спросить Его: кто Ты?, зная, что это Господь.
\par 13 Иисус приходит, берет хлеб и дает им, также и рыбу.
\par 14 Это уже в третий раз явился Иисус ученикам Своим по воскресении Своем из мертвых.
\par 15 Когда же они обедали, Иисус говорит Симону Петру: Симон Ионин! любишь ли ты Меня больше, нежели они? [Петр] говорит Ему: так, Господи! Ты знаешь, что я люблю Тебя. [Иисус] говорит ему: паси агнцев Моих.
\par 16 Еще говорит ему в другой раз: Симон Ионин! любишь ли ты Меня? [Петр] говорит Ему: так, Господи! Ты знаешь, что я люблю Тебя. [Иисус] говорит ему: паси овец Моих.
\par 17 Говорит ему в третий раз: Симон Ионин! любишь ли ты Меня? Петр опечалился, что в третий раз спросил его: любишь ли Меня? и сказал Ему: Господи! Ты все знаешь; Ты знаешь, что я люблю Тебя. Иисус говорит ему: паси овец Моих.
\par 18 Истинно, истинно говорю тебе: когда ты был молод, то препоясывался сам и ходил, куда хотел; а когда состаришься, то прострешь руки твои, и другой препояшет тебя, и поведет, куда не хочешь.
\par 19 Сказал же это, давая разуметь, какою смертью [Петр] прославит Бога. И, сказав сие, говорит ему: иди за Мною.
\par 20 Петр же, обратившись, видит идущего за ним ученика, которого любил Иисус и который на вечери, приклонившись к груди Его, сказал: Господи! кто предаст Тебя?
\par 21 Его увидев, Петр говорит Иисусу: Господи! а он что?
\par 22 Иисус говорит ему: если Я хочу, чтобы он пребыл, пока приду, что тебе [до того]? ты иди за Мною.
\par 23 И пронеслось это слово между братиями, что ученик тот не умрет. Но Иисус не сказал ему, что не умрет, но: если Я хочу, чтобы он пребыл, пока приду, что тебе [до того]?
\par 24 Сей ученик и свидетельствует о сем, и написал сие; и знаем, что истинно свидетельство его.
\par 25 Многое и другое сотворил Иисус; но, если бы писать о том подробно, то, думаю, и самому миру не вместить бы написанных книг. Аминь.


\end{document}