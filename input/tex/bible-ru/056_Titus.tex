\begin{document}

\title{Titus}

Tit 1:1  Павел, раб Божий, Апостол же Иисуса Христа, по вере избранных Божиих и познанию истины, [относящейся] к благочестию,
Tit 1:2  в надежде вечной жизни, которую обещал неизменный в слове Бог прежде вековых времен,
Tit 1:3  а в свое время явил Свое слово в проповеди, вверенной мне по повелению Спасителя нашего, Бога, --
Tit 1:4  Титу, истинному сыну по общей вере: благодать, милость и мир от Бога Отца и Господа Иисуса Христа, Спасителя нашего.
Tit 1:5  Для того я оставил тебя в Крите, чтобы ты довершил недоконченное и поставил по всем городам пресвитеров, как я тебе приказывал:
Tit 1:6  если кто непорочен, муж одной жены, детей имеет верных, не укоряемых в распутстве или непокорности.
Tit 1:7  Ибо епископ должен быть непорочен, как Божий домостроитель, не дерзок, не гневлив, не пьяница, не бийца, не корыстолюбец,
Tit 1:8  но страннолюбив, любящий добро, целомудрен, справедлив, благочестив, воздержан,
Tit 1:9  держащийся истинного слова, согласного с учением, чтобы он был силен и наставлять в здравом учении и противящихся обличать.
Tit 1:10  Ибо есть много и непокорных, пустословов и обманщиков, особенно из обрезанных,
Tit 1:11  каковым должно заграждать уста: они развращают целые домы, уча, чему не должно, из постыдной корысти.
Tit 1:12  Из них же самих один стихотворец сказал: `Критяне всегда лжецы, злые звери, утробы ленивые'.
Tit 1:13  Свидетельство это справедливо. По сей причине обличай их строго, дабы они были здравы в вере,
Tit 1:14  не внимая Иудейским басням и постановлениям людей, отвращающихся от истины.
Tit 1:15  Для чистых все чисто; а для оскверненных и неверных нет ничего чистого, но осквернены и ум их и совесть.
Tit 1:16  Они говорят, что знают Бога, а делами отрекаются, будучи гнусны и непокорны и не способны ни к какому доброму делу.
Tit 2:1  Ты же говори то, что сообразно с здравым учением:
Tit 2:2  чтобы старцы были бдительны, степенны, целомудренны, здравы в вере, в любви, в терпении;
Tit 2:3  чтобы старицы также одевались прилично святым, не были клеветницы, не порабощались пьянству, учили добру;
Tit 2:4  чтобы вразумляли молодых любить мужей, любить детей,
Tit 2:5  быть целомудренными, чистыми, попечительными о доме, добрыми, покорными своим мужьям, да не порицается слово Божие.
Tit 2:6  Юношей также увещевай быть целомудренными.
Tit 2:7  Во всем показывай в себе образец добрых дел, в учительстве чистоту, степенность, неповрежденность,
Tit 2:8  слово здравое, неукоризненное, чтобы противник был посрамлен, не имея ничего сказать о нас худого.
Tit 2:9  Рабов [увещевай] повиноваться своим господам, угождать им во всем, не прекословить,
Tit 2:10  не красть, но оказывать всю добрую верность, дабы они во всем были украшением учению Спасителя нашего, Бога.
Tit 2:11  Ибо явилась благодать Божия, спасительная для всех человеков,
Tit 2:12  научающая нас, чтобы мы, отвергнув нечестие и мирские похоти, целомудренно, праведно и благочестиво жили в нынешнем веке,
Tit 2:13  ожидая блаженного упования и явления славы великого Бога и Спасителя нашего Иисуса Христа,
Tit 2:14  Который дал Себя за нас, чтобы избавить нас от всякого беззакония и очистить Себе народ особенный, ревностный к добрым делам.
Tit 2:15  Сие говори, увещевай и обличай со всякою властью, чтобы никто не пренебрегал тебя.
Tit 3:1  Напоминай им повиноваться и покоряться начальству и властям, быть готовыми на всякое доброе дело,
Tit 3:2  никого не злословить, быть не сварливыми, но тихими, и оказывать всякую кротость ко всем человекам.
Tit 3:3  Ибо и мы были некогда несмысленны, непокорны, заблуждшие, были рабы похотей и различных удовольствий, жили в злобе и зависти, были гнусны, ненавидели друг друга.
Tit 3:4  Когда же явилась благодать и человеколюбие Спасителя нашего, Бога,
Tit 3:5  Он спас нас не по делам праведности, которые бы мы сотворили, а по Своей милости, банею возрождения и обновления Святым Духом,
Tit 3:6  Которого излил на нас обильно через Иисуса Христа, Спасителя нашего,
Tit 3:7  чтобы, оправдавшись Его благодатью, мы по упованию соделались наследниками вечной жизни.
Tit 3:8  Слово это верно; и я желаю, чтобы ты подтверждал о сем, дабы уверовавшие в Бога старались быть прилежными к добрым делам: это хорошо и полезно человекам.
Tit 3:9  Глупых же состязаний и родословий, и споров и распрей о законе удаляйся, ибо они бесполезны и суетны.
Tit 3:10  Еретика, после первого и второго вразумления, отвращайся,
Tit 3:11  зная, что таковой развратился и грешит, будучи самоосужден.
Tit 3:12  Когда пришлю к тебе Артему или Тихика, поспеши придти ко мне в Никополь, ибо я положил там провести зиму.
Tit 3:13  Зину законника и Аполлоса позаботься отправить так, чтобы у них ни в чем не было недостатка.
Tit 3:14  Пусть и наши учатся упражняться в добрых делах, [в] [удовлетворении] необходимым нуждам, дабы не были бесплодны.
Tit 3:15  Приветствуют тебя все находящиеся со мною. Приветствуй любящих нас в вере. Благодать со всеми вами. Аминь.


\end{document}