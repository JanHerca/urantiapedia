\begin{document}

\title{Послание к Титу}


\chapter{1}

\par 1 Павел, раб Божий, Апостол же Иисуса Христа, по вере избранных Божиих и познанию истины, [относящейся] к благочестию,
\par 2 в надежде вечной жизни, которую обещал неизменный в слове Бог прежде вековых времен,
\par 3 а в свое время явил Свое слово в проповеди, вверенной мне по повелению Спасителя нашего, Бога, --
\par 4 Титу, истинному сыну по общей вере: благодать, милость и мир от Бога Отца и Господа Иисуса Христа, Спасителя нашего.
\par 5 Для того я оставил тебя в Крите, чтобы ты довершил недоконченное и поставил по всем городам пресвитеров, как я тебе приказывал:
\par 6 если кто непорочен, муж одной жены, детей имеет верных, не укоряемых в распутстве или непокорности.
\par 7 Ибо епископ должен быть непорочен, как Божий домостроитель, не дерзок, не гневлив, не пьяница, не бийца, не корыстолюбец,
\par 8 но страннолюбив, любящий добро, целомудрен, справедлив, благочестив, воздержан,
\par 9 держащийся истинного слова, согласного с учением, чтобы он был силен и наставлять в здравом учении и противящихся обличать.
\par 10 Ибо есть много и непокорных, пустословов и обманщиков, особенно из обрезанных,
\par 11 каковым должно заграждать уста: они развращают целые домы, уча, чему не должно, из постыдной корысти.
\par 12 Из них же самих один стихотворец сказал: `Критяне всегда лжецы, злые звери, утробы ленивые'.
\par 13 Свидетельство это справедливо. По сей причине обличай их строго, дабы они были здравы в вере,
\par 14 не внимая Иудейским басням и постановлениям людей, отвращающихся от истины.
\par 15 Для чистых все чисто; а для оскверненных и неверных нет ничего чистого, но осквернены и ум их и совесть.
\par 16 Они говорят, что знают Бога, а делами отрекаются, будучи гнусны и непокорны и не способны ни к какому доброму делу.

\chapter{2}

\par 1 Ты же говори то, что сообразно с здравым учением:
\par 2 чтобы старцы были бдительны, степенны, целомудренны, здравы в вере, в любви, в терпении;
\par 3 чтобы старицы также одевались прилично святым, не были клеветницы, не порабощались пьянству, учили добру;
\par 4 чтобы вразумляли молодых любить мужей, любить детей,
\par 5 быть целомудренными, чистыми, попечительными о доме, добрыми, покорными своим мужьям, да не порицается слово Божие.
\par 6 Юношей также увещевай быть целомудренными.
\par 7 Во всем показывай в себе образец добрых дел, в учительстве чистоту, степенность, неповрежденность,
\par 8 слово здравое, неукоризненное, чтобы противник был посрамлен, не имея ничего сказать о нас худого.
\par 9 Рабов [увещевай] повиноваться своим господам, угождать им во всем, не прекословить,
\par 10 не красть, но оказывать всю добрую верность, дабы они во всем были украшением учению Спасителя нашего, Бога.
\par 11 Ибо явилась благодать Божия, спасительная для всех человеков,
\par 12 научающая нас, чтобы мы, отвергнув нечестие и мирские похоти, целомудренно, праведно и благочестиво жили в нынешнем веке,
\par 13 ожидая блаженного упования и явления славы великого Бога и Спасителя нашего Иисуса Христа,
\par 14 Который дал Себя за нас, чтобы избавить нас от всякого беззакония и очистить Себе народ особенный, ревностный к добрым делам.
\par 15 Сие говори, увещевай и обличай со всякою властью, чтобы никто не пренебрегал тебя.

\chapter{3}

\par 1 Напоминай им повиноваться и покоряться начальству и властям, быть готовыми на всякое доброе дело,
\par 2 никого не злословить, быть не сварливыми, но тихими, и оказывать всякую кротость ко всем человекам.
\par 3 Ибо и мы были некогда несмысленны, непокорны, заблуждшие, были рабы похотей и различных удовольствий, жили в злобе и зависти, были гнусны, ненавидели друг друга.
\par 4 Когда же явилась благодать и человеколюбие Спасителя нашего, Бога,
\par 5 Он спас нас не по делам праведности, которые бы мы сотворили, а по Своей милости, банею возрождения и обновления Святым Духом,
\par 6 Которого излил на нас обильно через Иисуса Христа, Спасителя нашего,
\par 7 чтобы, оправдавшись Его благодатью, мы по упованию соделались наследниками вечной жизни.
\par 8 Слово это верно; и я желаю, чтобы ты подтверждал о сем, дабы уверовавшие в Бога старались быть прилежными к добрым делам: это хорошо и полезно человекам.
\par 9 Глупых же состязаний и родословий, и споров и распрей о законе удаляйся, ибо они бесполезны и суетны.
\par 10 Еретика, после первого и второго вразумления, отвращайся,
\par 11 зная, что таковой развратился и грешит, будучи самоосужден.
\par 12 Когда пришлю к тебе Артему или Тихика, поспеши придти ко мне в Никополь, ибо я положил там провести зиму.
\par 13 Зину законника и Аполлоса позаботься отправить так, чтобы у них ни в чем не было недостатка.
\par 14 Пусть и наши учатся упражняться в добрых делах, [в] [удовлетворении] необходимым нуждам, дабы не были бесплодны.
\par 15 Приветствуют тебя все находящиеся со мною. Приветствуй любящих нас в вере. Благодать со всеми вами. Аминь.


\end{document}