\begin{document}

\title{3-е послание Иоанна}


\chapter{1}

\par 1 Старец--возлюбленному Гаию, которого я люблю по истине.
\par 2 Возлюбленный! молюсь, чтобы ты здравствовал и преуспевал во всем, как преуспевает душа твоя.
\par 3 Ибо я весьма обрадовался, когда пришли братия и засвидетельствовали о твоей верности, как ты ходишь в истине.
\par 4 Для меня нет большей радости, как слышать, что дети мои ходят в истине.
\par 5 Возлюбленный! ты как верный поступаешь в том, что делаешь для братьев и для странников.
\par 6 Они засвидетельствовали перед церковью о твоей любви. Ты хорошо поступишь, если отпустишь их, как должно ради Бога,
\par 7 ибо они ради имени Его пошли, не взяв ничего от язычников.
\par 8 Итак мы должны принимать таковых, чтобы сделаться споспешниками истине.
\par 9 Я писал церкви; но любящий первенствовать у них Диотреф не принимает нас.
\par 10 Посему, если я приду, то напомню о делах, которые он делает, понося нас злыми словами, и не довольствуясь тем, и сам не принимает братьев, и запрещает желающим, и изгоняет из церкви.
\par 11 Возлюбленный! не подражай злу, но добру. Кто делает добро, тот от Бога; а делающий зло не видел Бога.
\par 12 О Димитрии засвидетельствовано всеми и самою истиною; свидетельствуем также и мы, и вы знаете, что свидетельство наше истинно.
\par 13 Многое имел я писать; но не хочу писать к тебе чернилами и тростью,
\par 14 а надеюсь скоро увидеть тебя и поговорить устами к устам.
\par 15 Мир тебе. Приветствуют тебя друзья; приветствуй друзей поименно. Аминь.


\end{document}