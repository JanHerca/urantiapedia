\begin{document}

\title{Leviticus}

Lev 1:1  И воззвал Господь к Моисею и сказал ему из скинии собрания, говоря:
Lev 1:2  объяви сынам Израилевым и скажи им: когда кто из вас хочет принести жертву Господу, то, если из скота, приносите жертву вашу из скота крупного и мелкого.
Lev 1:3  Если жертва его есть всесожжение из крупного скота, пусть принесет ее мужеского пола, без порока; пусть приведет ее к дверям скинии собрания, чтобы приобрести ему благоволение пред Господом;
Lev 1:4  и возложит руку свою на голову [жертвы] всесожжения--и приобретет он благоволение, во очищение грехов его;
Lev 1:5  и заколет тельца пред Господом; сыны же Аароновы, священники, принесут кровь и покропят кровью со всех сторон на жертвенник, который у входа скинии собрания;
Lev 1:6  и снимет кожу с [жертвы] всесожжения и рассечет ее на части;
Lev 1:7  сыны же Аароновы, священники, положат на жертвенник огонь и на огне разложат дрова;
Lev 1:8  и разложат сыны Аароновы, священники, части, голову и тук на дровах, которые на огне, на жертвеннике;
Lev 1:9  а внутренности [жертвы] и ноги ее вымоет он водою, и сожжет священник все на жертвеннике: [это] всесожжение, жертва, благоухание, приятное Господу.
Lev 1:10  Если жертва всесожжения его из мелкого скота, из овец, или из коз, пусть принесет ее мужеского пола, без порока.
Lev 1:11  и заколет ее пред Господом на северной стороне жертвенника, и сыны Аароновы, священники, покропят кровью ее на жертвенник со всех сторон;
Lev 1:12  и рассекут ее на части, [отделив] голову ее и тук ее, и разложит их священник на дровах, которые на огне, на жертвеннике,
Lev 1:13  а внутренности и ноги вымоет водою, и принесет священник все и сожжет на жертвеннике: [это] всесожжение, жертва, благоухание, приятное Господу.
Lev 1:14  Если же из птиц приносит он Господу всесожжение, пусть принесет жертву свою из горлиц, или из молодых голубей;
Lev 1:15  священник принесет ее к жертвеннику, и свернет ей голову, и сожжет на жертвеннике, а кровь выцедит к стене жертвенника;
Lev 1:16  зоб ее с перьями ее отнимет и бросит его подле жертвенника на восточную сторону, где пепел;
Lev 1:17  и надломит ее в крыльях ее, не отделяя их, и сожжет ее священник на жертвеннике, на дровах, которые на огне: это всесожжение, жертва, благоухание, приятное Господу.
Lev 2:1  Если какая душа хочет принести Господу жертву приношения хлебного, пусть принесет пшеничной муки, и вольет на нее елея, и положит на нее ливана,
Lev 2:2  и принесет ее к сынам Аароновым, священникам, и возьмет полную горсть муки с елеем и со всем ливаном, и сожжет сие священник в память на жертвеннике; [это] жертва, благоухание, приятное Господу;
Lev 2:3  а остатки от приношения хлебного Аарону и сынам его: [это] великая святыня из жертв Господних.
Lev 2:4  Если же приносишь жертву приношения хлебного из печеного в печи, [то приноси] пшеничные хлебы пресные, смешанные с елеем, и лепешки пресные, помазанные елеем.
Lev 2:5  Если жертва твоя приношение хлебное со сковороды, то это должна быть пшеничная мука, смешанная с елеем, пресная;
Lev 2:6  разломи ее на куски и влей на нее елея: это приношение хлебное.
Lev 2:7  Если жертва твоя приношение хлебное из горшка, то должно сделать оное из пшеничной муки с елеем,
Lev 2:8  и принеси приношение, которое из сего составлено, Господу; представь оное священнику, а он принесет его к жертвеннику;
Lev 2:9  и возьмет священник из сей жертвы часть в память и сожжет на жертвеннике: [это] жертва, благоухание, приятное Господу;
Lev 2:10  а остатки приношения хлебного Аарону и сынам его: [это] великая святыня из жертв Господних.
Lev 2:11  Никакого приношения хлебного, которое приносите Господу, не делайте квасного, ибо ни квасного, ни меду не должны вы сожигать в жертву Господу;
Lev 2:12  как приношение начатков приносите их Господу, а на жертвенник не должно возносить их в приятное благоухание.
Lev 2:13  Всякое приношение твое хлебное соли солью, и не оставляй жертвы твоей без соли завета Бога твоего: при всяком приношении твоем приноси соль.
Lev 2:14  Если приносишь Господу приношение хлебное из первых плодов, приноси в дар от первых плодов твоих из колосьев, высушенных на огне, растолченные зерна,
Lev 2:15  и влей на них елея, и положи на них ливана: это приношение хлебное;
Lev 2:16  и сожжет священник в память часть зерен и елея со всем ливаном: [это] жертва Господу.
Lev 3:1  Если жертва его жертва мирная, и если он приносит из крупного скота, мужеского или женского пола, пусть принесет ее Господу, не имеющую порока,
Lev 3:2  и возложит руку свою на голову жертвы своей, и заколет ее у дверей скинии собрания; сыны же Аароновы, священники, покропят кровью на жертвенник со всех сторон;
Lev 3:3  и принесет он из мирной жертвы в жертву Господу тук, покрывающий внутренности, и весь тук, который на внутренностях,
Lev 3:4  и обе почки и тук, который на них, который на стегнах, и сальник, который на печени; с почками он отделит это;
Lev 3:5  и сыны Аароновы сожгут это на жертвеннике вместе со всесожжением, которое на дровах, на огне: [это] жертва, благоухание, приятное Господу.
Lev 3:6  А если из мелкого скота приносит он мирную жертву Господу, мужеского или женского пола, пусть принесет ее, не имеющую порока.
Lev 3:7  Если из овец приносит он жертву свою, пусть представит ее пред Господа,
Lev 3:8  и возложит руку свою на голову жертвы своей, и заколет ее пред скиниею собрания, и сыны Аароновы покропят кровью ее на жертвенник со всех сторон;
Lev 3:9  и пусть принесет из мирной жертвы в жертву Господу тук ее, весь курдюк, отрезав его по самую хребтовую кость, и тук, покрывающий внутренности, и весь тук, который на внутренностях,
Lev 3:10  и обе почки и тук, который на них, который на стегнах, и сальник, который на печени; с почками он отделит это;
Lev 3:11  священник сожжет это на жертвеннике; [это] пища огня--жертва Господу.
Lev 3:12  А если он приносит жертву из коз, пусть представит ее пред Господа,
Lev 3:13  и возложит руку свою на голову ее, и заколет ее перед скиниею собрания, и покропят сыны Аароновы кровью ее на жертвенник со всех сторон;
Lev 3:14  и принесет из нее в приношение, в жертву Господу тук, покрывающий внутренности, и весь тук, который на внутренностях,
Lev 3:15  и обе почки и тук, который на них, который на стегнах, и сальник, который на печени; с почками он отделит это
Lev 3:16  и сожжет их священник на жертвеннике: [это] пища огня--приятное благоухание [Господу]; весь тук Господу.
Lev 3:17  Это постановление вечное в роды ваши, во всех жилищах ваших; никакого тука и никакой крови не ешьте.
Lev 4:1  И сказал Господь Моисею, говоря:
Lev 4:2  скажи сынам Израилевым: если какая душа согрешит по ошибке против каких-либо заповедей Господних и сделает что-нибудь, чего не должно делать;
Lev 4:3  если священник помазанный согрешит и сделает виновным народ, --то за грех свой, которым согрешил, пусть представит из крупного скота тельца, без порока, Господу в жертву о грехе,
Lev 4:4  и приведет тельца к дверям скинии собрания пред Господа, и возложит руки свои на голову тельца, и заколет тельца пред Господом;
Lev 4:5  и возьмет священник помазанный, крови тельца и внесет ее в скинию собрания,
Lev 4:6  и омочит священник перст свой в кровь и покропит кровью семь раз пред Господом пред завесою святилища;
Lev 4:7  и возложит священник крови [тельца] пред Господом на роги жертвенника благовонных курений, который в скинии собрания, а остальную кровь тельца выльет к подножию жертвенника всесожжений, который у входа скинии собрания;
Lev 4:8  и вынет из тельца за грех весь тук его, тук, покрывающий внутренности, и весь тук, который на внутренностях,
Lev 4:9  и обе почки и тук, который на них, который на стегнах, и сальник на печени; с почками отделит он это,
Lev 4:10  как отделяется из тельца жертвы мирной; и сожжет их священник на жертвеннике всесожжения;
Lev 4:11  а кожу тельца и все мясо его с головою и с ногами его, и внутренности его и нечистоту его,
Lev 4:12  всего тельца пусть вынесет вне стана на чистое место, где высыпается пепел, и сожжет его огнем на дровах; где высыпается пепел, там пусть сожжен будет.
Lev 4:13  Если же все общество Израилево согрешит по ошибке и скрыто будет дело от глаз собрания, и сделает что-нибудь против заповедей Господних, чего не надлежало делать, и будет виновно,
Lev 4:14  то, когда узнан будет грех, которым они согрешили, пусть от всего общества представят они из крупного скота тельца в жертву за грех и приведут его пред скинию собрания;
Lev 4:15  и возложат старейшины общества руки свои на голову тельца пред Господом и заколют тельца пред Господом.
Lev 4:16  И внесет священник помазанный крови тельца в скинию собрания,
Lev 4:17  и омочит священник перст свой в кровь и покропит семь раз пред Господом пред завесою [святилища];
Lev 4:18  и возложит крови на роги жертвенника, который пред лицем Господним в скинии собрания, а остальную кровь выльет к подножию жертвенника всесожжений, который у входа скинии собрания;
Lev 4:19  и весь тук его вынет из него и сожжет на жертвеннике;
Lev 4:20  и сделает с тельцом то, что делается с тельцом за грех; так должен сделать с ним, и так очистит их священник, и прощено будет им;
Lev 4:21  и вынесет тельца вне стана, и сожжет его так, как сожег прежнего тельца. Это жертва за грех общества.
Lev 4:22  А если согрешит начальник, и сделает по ошибке что-нибудь против заповедей Господа, Бога своего, чего не надлежало делать, и будет виновен,
Lev 4:23  то, когда узнан будет им грех, которым он согрешил, пусть приведет он в жертву козла без порока,
Lev 4:24  и возложит руку свою на голову козла, и заколет его на месте, где заколаются всесожжения пред Господом: это жертва за грех;
Lev 4:25  и возьмет священник перстом своим крови от жертвы за грех и возложит на роги жертвенника всесожжения, а остальную кровь его выльет к подножию жертвенника всесожжения;
Lev 4:26  и весь тук его сожжет на жертвеннике, подобно как тук жертвы мирной, и так очистит его священник от греха его, и прощено будет ему.
Lev 4:27  Если же кто из народа земли согрешит по ошибке и сделает что-нибудь против заповедей Господних, чего не надлежало делать, и виновен будет,
Lev 4:28  то, когда узнан будет им грех, которым он согрешил, пусть приведет он в жертву козу без порока за грех свой, которым он согрешил,
Lev 4:29  и возложит руку свою на голову жертвы за грех, и заколют [козу] в жертву за грех на месте, [где заколают] жертву всесожжения;
Lev 4:30  и возьмет священник крови ее перстом своим, и возложит на роги жертвенника всесожжения, а остальную кровь ее выльет к подножию жертвенника;
Lev 4:31  и весь тук ее отделит, подобно как отделяется тук из жертвы мирной, и сожжет [его] священник на жертвеннике в приятное благоухание Господу; и так очистит его священник, и прощено будет ему.
Lev 4:32  А если из стада овец захочет он принести жертву за грех, пусть принесет женского пола, без порока,
Lev 4:33  и возложит руку свою на голову жертвы за грех, и заколет ее в жертву за грех на том месте, где заколают жертву всесожжения;
Lev 4:34  и возьмет священник перстом своим крови от сей жертвы за грех и возложит на роги жертвенника всесожжения, а остальную кровь ее выльет к подножию жертвенника;
Lev 4:35  и весь тук ее отделит, как отделяется тук овцы из жертвы мирной, и сожжет сие священник на жертвеннике в жертву Господу; и так очистит его священник от греха, которым он согрешил, и прощено будет ему.
Lev 5:1  Если кто согрешит тем, что слышал голос проклятия и был свидетелем, или видел, или знал, но не объявил, то он понесет на себе грех.
Lev 5:2  Или если прикоснется к чему-нибудь нечистому, или к трупу зверя нечистого, или к трупу скота нечистого, или к трупу гада нечистого, но не знал того, то он нечист и виновен.
Lev 5:3  Или если прикоснется к нечистоте человеческой, какая бы то ни была нечистота, от которой оскверняются, и он не знал того, но после узнает, то он виновен.
Lev 5:4  Или если кто безрассудно устами своими поклянется сделать что-нибудь худое или доброе, какое бы то ни было дело, в котором люди безрассудно клянутся, и он не знал того, но после узнает, то он виновен в том.
Lev 5:5  Если он виновен в чем-нибудь из сих, и исповедается, в чем он согрешил,
Lev 5:6  то пусть принесет Господу за грех свой, которым он согрешил, жертву повинности из мелкого скота, овцу или козу, за грех, и очистит его священник от греха его.
Lev 5:7  Если же он не в состоянии принести овцы, то в повинность за грех свой пусть принесет Господу двух горлиц или двух молодых голубей, одного в жертву за грех, а другого во всесожжение;
Lev 5:8  пусть принесет их к священнику, и [священник] представит прежде ту [из сих птиц], которая за грех, и надломит голову ее от шеи ее, но не отделит;
Lev 5:9  и покропит кровью сей жертвы за грех на стену жертвенника, а остальную кровь выцедит к подножию жертвенника: это жертва за грех;
Lev 5:10  а другую употребит во всесожжение по установлению; и так очистит его священник от греха его, которым он согрешил, и прощено будет ему.
Lev 5:11  Если же он не в состоянии принести двух горлиц или двух молодых голубей, пусть принесет за то, что согрешил, десятую часть ефы пшеничной муки в жертву за грех; пусть не льет на нее елея, и ливана пусть не кладет на нее, ибо это жертва за грех;
Lev 5:12  и принесет ее к священнику, а священник возьмет из нее полную горсть в память и сожжет на жертвеннике в жертву Господу: это жертва за грех;
Lev 5:13  и так очистит его священник от греха его, которым он согрешил в котором-нибудь из оных [случаев], и прощено будет ему; [остаток] же принадлежит священнику, как приношение хлебное.
Lev 5:14  И сказал Господь Моисею, говоря:
Lev 5:15  если кто сделает преступление и по ошибке согрешит против посвященного Господу, пусть за вину свою принесет Господу из стада овец овна без порока, по твоей оценке, серебряными сиклями по сиклю священному, в жертву повинности;
Lev 5:16  за ту святыню, против которой он согрешил, пусть воздаст и прибавит к тому пятую долю, и отдаст сие священнику, и священник очистит его овном жертвы повинности, и прощено будет ему.
Lev 5:17  Если кто согрешит и сделает что-нибудь против заповедей Господних, чего не надлежало делать, и по неведению сделается виновным и понесет на себе грех,
Lev 5:18  пусть принесет к священнику в жертву повинности овна без порока, по оценке твоей, и загладит священник проступок его, в чем он преступил по неведению, и прощено будет ему.
Lev 5:19  Это жертва повинности, [которою] он провинился пред Господом.
Lev 6:1  И сказал Господь Моисею, говоря:
Lev 6:2  если кто согрешит и сделает преступление пред Господом и запрется пред ближним своим в том, что ему поручено, или у него положено, или им похищено, или обманет ближнего своего,
Lev 6:3  или найдет потерянное и запрется в том, и поклянется ложно в чем-нибудь, что люди делают и тем грешат, --
Lev 6:4  то, согрешив и сделавшись виновным, он должен возвратить похищенное, что похитил, или отнятое, что отнял, или порученное, что ему поручено, или потерянное, что он нашел;
Lev 6:5  или если он в чем поклялся ложно, то должен отдать сполна, и приложить к тому пятую долю и отдать тому, кому принадлежит, в день приношения жертвы повинности;
Lev 6:6  и за вину свою пусть принесет Господу к священнику в жертву повинности из стада овец овна без порока, по оценке твоей;
Lev 6:7  и очистит его священник пред Господом, и прощено будет ему, что бы он ни сделал, все, в чем он сделался виновным.
Lev 6:8  И сказал Господь Моисею, говоря:
Lev 6:9  заповедай Аарону и сынам его: вот закон всесожжения: всесожжение пусть остается на месте сожигания на жертвеннике всю ночь до утра, и огонь жертвенника пусть горит на нем.
Lev 6:10  и пусть священник оденется в льняную одежду свою, и наденет на тело свое льняное нижнее платье, и снимет пепел от всесожжения, которое сжег огонь на жертвеннике, и положит его подле жертвенника;
Lev 6:11  и пусть снимет с себя одежды свои, и наденет другие одежды, и вынесет пепел вне стана на чистое место;
Lev 6:12  а огонь на жертвеннике пусть горит, не угасает; и пусть священник зажигает на нем дрова каждое утро, и раскладывает на нем всесожжение, и сожигает на нем тук мирной жертвы;
Lev 6:13  огонь непрестанно пусть горит на жертвеннике [и] не угасает.
Lev 6:14  Вот закон о приношении хлебном: сыны Аароновы должны приносить его пред Господа к жертвеннику;
Lev 6:15  и пусть возьмет [священник] горстью своею из приношения хлебного и пшеничной муки и елея и весь ливан, который на жертве, и сожжет на жертвеннике: [это] приятное благоухание, в память пред Господом;
Lev 6:16  а остальное из него пусть едят Аарон и сыны его; пресным должно есть его на святом месте, на дворе скинии собрания пусть едят его;
Lev 6:17  не должно печь его квасным. Сие даю Я им в долю из жертв Моих. Это великая святыня, подобно как жертва за грех и жертва повинности.
Lev 6:18  Все потомки Аароновы мужеского пола могут есть ее. Это вечный участок в роды ваши из жертв Господних. Все, прикасающееся к ним, освятится.
Lev 6:19  И сказал Господь Моисею, говоря:
Lev 6:20  вот приношение от Аарона и сынов его, которое принесут они Господу в день помазания его: десятая часть ефы пшеничной муки в жертву постоянную, половина сего для утра и половина для вечера;
Lev 6:21  на сковороде в елее она должна быть приготовлена; напитанную [елеем] приноси ее в кусках, как разламывается в куски приношение хлебное; приноси ее в приятное благоухание Господу;
Lev 6:22  и священник, помазанный на место его из сынов его, должен совершать сие: это вечный устав Господа. Вся она должна быть сожжена;
Lev 6:23  и всякое хлебное приношение от священника все да будет сожигаемо, а не съедаемо.
Lev 6:24  И сказал Господь Моисею, говоря:
Lev 6:25  скажи Аарону и сынам его: вот закон о жертве за грех: жертва за грех должна быть заколаема пред Господом на том месте, где заколается всесожжение; это великая святыня;
Lev 6:26  священник, совершающий жертву за грех, должен есть ее; она должна быть съедаема на святом месте, на дворе скинии собрания;
Lev 6:27  все, что прикоснется к мясу ее, освятится; и если кровью ее обрызгана будет одежда, то обрызганное омой на святом месте;
Lev 6:28  глиняный сосуд, в котором она варилась, должно разбить; если же она варилась в медном сосуде, то должно его вычистить и вымыть водою;
Lev 6:29  весь мужеский пол священнического рода может есть ее: это великая святыня.
Lev 6:30  а всякая жертва за грех, от которой кровь вносится в скинию собрания для очищения во святилище, не должна быть съедаема; ее должно сожигать на огне.
Lev 7:1  Вот закон о жертве повинности: это великая святыня;
Lev 7:2  жертву повинности должно заколать на том месте, где заколается всесожжение, и кровью ее кропить на жертвенник со всех сторон;
Lev 7:3  [приносящий] должен представить из нее весь тук, курдюк и тук, покрывающий внутренности,
Lev 7:4  и обе почки и тук, который на них, который на стегнах, и сальник, который на печени; с почками пусть он отделит сие;
Lev 7:5  и сожжет сие священник на жертвеннике в жертву Господу: это жертва повинности.
Lev 7:6  Весь мужеский пол священнического рода может есть ее; на святом месте должно есть ее: это великая святыня.
Lev 7:7  Как о жертве за грех, так и о жертве повинности закон один: она принадлежит священнику, который очищает посредством ее.
Lev 7:8  И когда священник приносит чью-нибудь жертву всесожжения, кожа от [жертвы] всесожжения, которое он приносит, принадлежит священнику;
Lev 7:9  и всякое приношение хлебное, которое печено в печи, и всякое приготовленное в горшке или на сковороде, принадлежит священнику, приносящему его;
Lev 7:10  и всякое приношение хлебное, смешанное с елеем и сухое, принадлежит всем сынам Аароновым, как одному, так и другому.
Lev 7:11  Вот закон о жертве мирной, которую приносят Господу:
Lev 7:12  если кто в благодарность приносит ее, то при жертве благодарности он должен принести пресные хлебы, смешанные с елеем, и пресные лепешки, помазанные елеем, и пшеничную муку, напитанную [елеем], хлебы, смешанные с елеем;
Lev 7:13  кроме лепешек пусть он приносит в приношение свое квасный хлеб, при мирной жертве благодарной;
Lev 7:14  одно что-нибудь из всего приношения своего пусть принесет он в возношение Господу: это принадлежит священнику, кропящему кровью мирной жертвы;
Lev 7:15  мясо мирной жертвы благодарности должно съесть в день приношения ее, не должно оставлять от него до утра.
Lev 7:16  Если же кто приносит жертву по обету, или от усердия, то жертву его должно есть в день приношения, и на другой день оставшееся от нее есть можно,
Lev 7:17  а оставшееся от жертвенного мяса к третьему дню должно сжечь на огне;
Lev 7:18  если же будут есть мясо мирной жертвы на третий день, то она не будет благоприятна; кто ее принесет, тому ни во что не вменится: это осквернение, и кто будет есть ее, тот понесет на себе грех;
Lev 7:19  мяса сего, если оно прикоснется к чему-либо нечистому, не должно есть, но должно сжечь его на огне; а мясо чистое может есть всякий чистый;
Lev 7:20  если же какая душа, имея на себе нечистоту, будет есть мясо мирной жертвы Господней, то истребится душа та из народа своего;
Lev 7:21  и если какая душа, прикоснувшись к чему-нибудь нечистому, к нечистоте человеческой, или к нечистому скоту, или какому-нибудь нечистому гаду, будет есть мясо мирной жертвы Господней, то истребится душа та из народа своего.
Lev 7:22  И сказал Господь Моисею, говоря:
Lev 7:23  скажи сынам Израилевым: никакого тука ни из вола, ни из овцы, ни из козла не ешьте.
Lev 7:24  Тук из мертвого и тук из растерзанного зверем можно употреблять на всякое дело; а есть не ешьте его;
Lev 7:25  ибо, кто будет есть тук из скота, который приносится в жертву Господу, истребится душа та из народа своего;
Lev 7:26  и никакой крови не ешьте во всех жилищах ваших ни из птиц, ни из скота;
Lev 7:27  а кто будет есть какую-нибудь кровь, истребится душа та из народа своего.
Lev 7:28  И сказал Господь Моисею, говоря:
Lev 7:29  скажи сынам Израилевым: кто представляет мирную жертву свою Господу, тот из мирной жертвы часть должен принести в приношение Господу;
Lev 7:30  своими руками должен он принести в жертву Господу: тук с грудью должен он принести, потрясая грудь пред лицем Господним;
Lev 7:31  тук сожжет священник на жертвеннике, а грудь принадлежит Аарону и сынам его;
Lev 7:32  и правое плечо, как возношение, из мирных жертв ваших отдавайте священнику:
Lev 7:33  кто из сынов Аароновых приносит кровь из мирной жертвы и тук, тому и правое плечо на долю;
Lev 7:34  ибо Я беру от сынов Израилевых из мирных жертв их грудь потрясания и плечо возношения, и отдаю их Аарону священнику и сынам его в вечный участок от сынов Израилевых.
Lev 7:35  Вот участок Аарону и участок сынам его из жертв Господних со дня, когда они предстанут пред Господа для священнодействия,
Lev 7:36  который повелел Господь давать им со дня помазания их от сынов Израилевых. [Это] вечное постановление в роды их. --
Lev 7:37  Вот закон о всесожжении, о приношении хлебном, о жертве за грех, о жертве повинности, о жертве посвящения и о жертве мирной,
Lev 7:38  который дал Господь Моисею на горе Синае, когда повелел сынам Израилевым, в пустыне Синайской, приносить Господу приношения их.
Lev 8:1  И сказал Господь Моисею, говоря:
Lev 8:2  возьми Аарона и сынов его с ним, и одежды и елей помазания, и тельца для жертвы за грех и двух овнов, и корзину опресноков,
Lev 8:3  и собери все общество ко входу скинии собрания.
Lev 8:4  Моисей сделал так, как повелел ему Господь, и собралось общество ко входу скинии собрания.
Lev 8:5  И сказал Моисей к обществу: вот что повелел Господь сделать.
Lev 8:6  И привел Моисей Аарона и сынов его и омыл их водою;
Lev 8:7  и возложил на него хитон, и опоясал его поясом, и надел на него верхнюю ризу, и возложил на него ефод, и опоясал его поясом ефода и прикрепил им ефод на нем,
Lev 8:8  и возложил на него наперсник, и на наперсник положил урим и туммим,
Lev 8:9  и возложил на голову его кидар, а на кидар с передней стороны его возложил полированную дощечку, диадиму святыни, как повелел Господь Моисею.
Lev 8:10  И взял Моисей елей помазания, и помазал скинию и все, что в ней, и освятил это;
Lev 8:11  и покропил им на жертвенник семь раз, и помазал жертвенник и все принадлежности его и умывальницу и подножие ее, чтобы освятить их;
Lev 8:12  и возлил елей помазания на голову Аарона и помазал его, чтоб освятить его.
Lev 8:13  И привел Моисей сынов Аароновых, и одел их в хитоны, и опоясал их поясом, и возложил на них кидары, как повелел Господь Моисею.
Lev 8:14  И привел тельца для жертвы за грех, и Аарон и сыны его возложили руки свои на голову тельца за грех;
Lev 8:15  и заколол [его] и взял крови, и перстом своим возложил на роги жертвенника со всех сторон, и очистил жертвенник, а [остальную] кровь вылил к подножию жертвенника, и освятил его, чтобы сделать его чистым.
Lev 8:16  И взял весь тук, который на внутренностях, и сальник на печени, и обе почки и тук их, и сжег Моисей на жертвеннике;
Lev 8:17  а тельца и кожу его, и мясо его, и нечистоту его сжег на огне вне стана, как повелел Господь Моисею.
Lev 8:18  И привел овна для всесожжения, и возложили Аарон и сыны его руки свои на голову овна;
Lev 8:19  и заколол [его] Моисей и покропил кровью на жертвенник со всех сторон;
Lev 8:20  и рассек овна на части, и сжег Моисей голову и части и тук,
Lev 8:21  а внутренности и ноги вымыл водою, и сжег Моисей всего овна на жертвеннике: это всесожжение в приятное благоухание, это жертва Господу, как повелел Господь Моисею.
Lev 8:22  И привел другого овна, овна посвящения, и возложили Аарон и сыны его руки свои на голову овна;
Lev 8:23  и заколол [его] Моисей, и взял крови его, и возложил на край правого уха Ааронова и на большой палец правой руки его и на большой палец правой ноги его.
Lev 8:24  И привел Моисей сынов Аароновых, и возложил крови на край правого уха их и на большой палец правой руки их и на большой палец правой ноги их, и покропил Моисей кровью на жертвенник со всех сторон.
Lev 8:25  И взял тук и курдюк и весь тук, который на внутренностях, и сальник на печени, и обе почки и тук их и правое плечо;
Lev 8:26  и из корзины с опресноками, которая пред Господом, взял один опреснок и один хлеб с елеем и одну лепешку, и возложил на тук и на правое плечо;
Lev 8:27  и положил все это на руки Аарону и на руки сынам его, и принес это, потрясая пред лицем Господним;
Lev 8:28  и взял это Моисей с рук их и сжег на жертвеннике со всесожжением: это жертва посвящения в приятное благоухание, это жертва Господу.
Lev 8:29  И взял Моисей грудь и принес ее, потрясая пред лицем Господним: это была доля Моисеева от овна посвящения, как повелел Господь Моисею.
Lev 8:30  И взял Моисей елея помазания и крови, которая на жертвеннике, и покропил Аарона и одежды его, и сынов его и одежды сынов его с ним; и так освятил Аарона и одежды его, и сынов его и одежды сынов его с ним.
Lev 8:31  И сказал Моисей Аарону и сынам его: сварите мясо у входа скинии собрания и там ешьте его с хлебом, который в корзине посвящения, как мне повелено и сказано: Аарон и сыны его должны есть его;
Lev 8:32  а остатки мяса и хлеба сожгите на огне.
Lev 8:33  Семь дней не отходите от дверей скинии собрания, пока не исполнятся дни посвящения вашего, ибо семь дней должно совершаться посвящение ваше;
Lev 8:34  как сегодня было сделано, так повелел Господь делать для очищения вас;
Lev 8:35  у входа скинии собрания будьте день и ночь в продолжение семи дней и будьте на страже у Господа, чтобы не умереть, ибо так мне повелено [от Господа Бога].
Lev 8:36  И исполнил Аарон и сыны его все, что повелел Господь чрез Моисея.
Lev 9:1  В восьмой день призвал Моисей Аарона и сынов его и старейшин Израилевых
Lev 9:2  и сказал Аарону: возьми себе из волов тельца в жертву за грех и овна во всесожжение, обоих без порока, и представь пред лице Господне;
Lev 9:3  и сынам Израилевым скажи: возьмите козла в жертву за грех, и тельца, и агнца, однолетних, без порока, во всесожжение,
Lev 9:4  и вола и овна в жертву мирную, чтобы совершить жертвоприношение пред лицем Господним, и приношение хлебное, смешанное с елеем, ибо сегодня Господь явится вам.
Lev 9:5  И принесли то, что приказал Моисей, пред скинию собрания, и пришло все общество и стало пред лицем Господним.
Lev 9:6  И сказал Моисей: вот что повелел Господь сделать, и явится вам слава Господня.
Lev 9:7  И сказал Моисей Аарону: приступи к жертвеннику и соверши жертву твою о грехе и всесожжение твое, и очисти себя и народ, и сделай приношение от народа, и очисти их, как повелел Господь.
Lev 9:8  И приступил Аарон к жертвеннику и заколол тельца, который за него, в жертву за грех:
Lev 9:9  сыны Аарона поднесли ему кровь, и он омочил перст свой в крови и возложил на роги жертвенника, а [остальную] кровь вылил к подножию жертвенника,
Lev 9:10  а тук и почки и сальник на печени от жертвы за грех сжег на жертвеннике, как повелел Господь Моисею;
Lev 9:11  мясо же и кожу сжег на огне вне стана.
Lev 9:12  И заколол всесожжение, и сыны Аарона поднесли ему кровь; он покропил ею на жертвенник со всех сторон;
Lev 9:13  и принесли ему всесожжение в кусках и голову, и он сжег на жертвеннике,
Lev 9:14  а внутренности и ноги омыл и сжег со всесожжением на жертвеннике.
Lev 9:15  И принес приношение от народа, и взял от народа козла за грех, и заколол его, и принес его в жертву за грех, как и прежнего.
Lev 9:16  И принес всесожжение и совершил его по уставу.
Lev 9:17  И принес приношение хлебное, и наполнил им руки свои, и сжег на жертвеннике сверх утреннего всесожжения.
Lev 9:18  И заколол вола и овна, которые от народа, в жертву мирную; и сыны Аарона поднесли ему кровь, и он покропил ею на жертвенник со всех сторон;
Lev 9:19  [поднесли] и тук из вола, и из овна курдюк, и [тук] покрывающий [внутренности], почки и сальник на печени,
Lev 9:20  и положили тук на грудь, и он сжег тук на жертвеннике;
Lev 9:21  грудь же и правое плечо принес Аарон, потрясая пред лицем Господним, как повелел Моисей.
Lev 9:22  И поднял Аарон руки свои, [обратившись] к народу, и благословил его, и сошел, совершив жертву за грех, всесожжение и жертву мирную.
Lev 9:23  И вошли Моисей и Аарон в скинию собрания, и вышли, и благословили народ. И явилась слава Господня всему народу:
Lev 9:24  и вышел огонь от Господа и сжег на жертвеннике всесожжение и тук; и видел весь народ, и воскликнул от радости, и пал на лице свое.
Lev 10:1  Надав и Авиуд, сыны Аароновы, взяли каждый свою кадильницу, и положили в них огня, и вложили в него курений, и принесли пред Господа огонь чуждый, которого Он не велел им;
Lev 10:2  и вышел огонь от Господа и сжег их, и умерли они пред лицем Господним.
Lev 10:3  И сказал Моисей Аарону: вот о чем говорил Господь, когда сказал: в приближающихся ко Мне освящусь и пред всем народом прославлюсь. Аарон молчал.
Lev 10:4  И позвал Моисей Мисаила и Елцафана, сынов Узиила, дяди Ааронова, и сказал им: пойдите, вынесите братьев ваших из святилища за стан.
Lev 10:5  И пошли и вынесли их в хитонах их за стан, как сказал Моисей.
Lev 10:6  Аарону же и Елеазару и Ифамару, сынам его, Моисей сказал: голов ваших не обнажайте и одежд ваших не раздирайте, чтобы вам не умереть и не навести гнева на все общество; но братья ваши, весь дом Израилев, могут плакать о сожженных, которых сожег Господь,
Lev 10:7  и из дверей скинии собрания не выходите, чтобы не умереть вам, ибо на вас елей помазания Господня. И сделали по слову Моисея.
Lev 10:8  И сказал Господь Аарону, говоря:
Lev 10:9  вина и крепких напитков не пей ты и сыны твои с тобою, когда входите в скинию собрания, чтобы не умереть. [Это] вечное постановление в роды ваши,
Lev 10:10  чтобы вы могли отличать священное от несвященного и нечистое от чистого,
Lev 10:11  и научать сынов Израилевых всем уставам, которые изрек им Господь чрез Моисея.
Lev 10:12  И сказал Моисей Аарону и Елеазару и Ифамару, оставшимся сынам его: возьмите приношение хлебное, оставшееся от жертв Господних, и ешьте его пресное у жертвенника, ибо это великая святыня;
Lev 10:13  и ешьте его на святом месте, ибо это участок твой и участок сынов твоих из жертв Господних: так мне повелено [от Господа];
Lev 10:14  и грудь потрясания и плечо возношения ешьте на чистом месте, ты и сыновья твои и дочери твои с тобою, ибо это дано в участок тебе и в участок сынам твоим из мирных жертв сынов Израилевых;
Lev 10:15  плечо возношения и грудь потрясания должны они приносить с жертвами тука, потрясая пред лицем Господним, и да будет это вечным участком тебе и сыновьям твоим с тобою, как повелел Господь.
Lev 10:16  И козла жертвы за грех искал Моисей, и вот, он сожжен. И разгневался на Елеазара и Ифамара, оставшихся сынов Аароновых, и сказал:
Lev 10:17  почему вы не ели жертвы за грех на святом месте? ибо она святыня великая, и она дана вам, чтобы снимать грехи с общества и очищать их пред Господом;
Lev 10:18  вот, кровь ее не внесена внутрь святилища, а вы должны были есть ее на святом месте, как повелено мне.
Lev 10:19  Аарон сказал Моисею: вот, сегодня принесли они жертву свою за грех и всесожжение свое пред Господом, и это случилось со мною; если я сегодня съем жертву за грех, будет ли это угодно Господу?
Lev 10:20  И услышал Моисей и одобрил.
Lev 11:1  И сказал Господь Моисею и Аарону, говоря им:
Lev 11:2  скажите сынам Израилевым: вот животные, которые можно вам есть из всего скота на земле:
Lev 11:3  всякий скот, у которого раздвоены копыта и на копытах глубокий разрез, и который жует жвачку, ешьте;
Lev 11:4  только сих не ешьте из жующих жвачку и имеющих раздвоенные копыта: верблюда, потому что он жует жвачку, но копыта у него не раздвоены, нечист он для вас;
Lev 11:5  и тушканчика, потому что он жует жвачку, но копыта у него не раздвоены, нечист он для вас,
Lev 11:6  и зайца, потому что он жует жвачку, но копыта у него не раздвоены, нечист он для вас;
Lev 11:7  и свиньи, потому что копыта у нее раздвоены и на копытах разрез глубокий, но она не жует жвачки, нечиста она для вас;
Lev 11:8  мяса их не ешьте и к трупам их не прикасайтесь; нечисты они для вас.
Lev 11:9  Из всех [животных], которые в воде, ешьте сих: у которых есть перья и чешуя в воде, в морях ли, или реках, тех ешьте;
Lev 11:10  а все те, у которых нет перьев и чешуи, в морях ли, или реках, из всех плавающих в водах и из всего живущего в водах, скверны для вас;
Lev 11:11  они должны быть скверны для вас: мяса их не ешьте и трупов их гнушайтесь;
Lev 11:12  все [животные], у которых нет перьев и чешуи в воде, скверны для вас.
Lev 11:13  Из птиц же гнушайтесь сих: орла, грифа и морского орла,
Lev 11:14  коршуна и сокола с породою его,
Lev 11:15  всякого ворона с породою его,
Lev 11:16  страуса, совы, чайки и ястреба с породою его,
Lev 11:17  филина, рыболова и ибиса,
Lev 11:18  лебедя, пеликана и сипа,
Lev 11:19  цапли, зуя с породою его, удода и нетопыря.
Lev 11:20  Все [животные] пресмыкающиеся, крылатые, ходящие на четырех [ногах], скверны для нас;
Lev 11:21  из всех пресмыкающихся, крылатых, ходящих на четырех [ногах], тех только ешьте, у которых есть голени выше ног, чтобы скакать ими по земле;
Lev 11:22  сих ешьте из них: саранчу с ее породою, солам с ее породою, харгол с ее породою и хагаб с ее породою.
Lev 11:23  Всякое [другое] пресмыкающееся, крылатое, у которого четыре ноги, скверно для вас;
Lev 11:24  от них вы будете нечисты: всякий, кто прикоснется к трупу их, нечист будет до вечера;
Lev 11:25  и всякий, кто возьмет труп их, должен омыть одежду свою и нечист будет до вечера.
Lev 11:26  Всякий скот, у которого копыта раздвоены, но нет глубокого разреза, и который не жует жвачки, нечист для вас: всякий, кто прикоснется к нему, будет нечист.
Lev 11:27  Из всех зверей четвероногих те, которые ходят на лапах, нечисты для вас: всякий, кто прикоснется к трупу их, нечист будет до вечера;
Lev 11:28  кто возьмет труп их, тот должен омыть одежды свои и нечист будет до вечера: нечисты они для вас.
Lev 11:29  Вот что нечисто для вас из животных, пресмыкающихся по земле: крот, мышь, ящерица с ее породою,
Lev 11:30  анака, хамелеон, летаа, хомет и тиншемет, --
Lev 11:31  сии нечисты для вас из всех пресмыкающихся: всякий, кто прикоснется к ним мертвым, нечист будет до вечера.
Lev 11:32  И все, на что упадет которое-нибудь из них мертвое, всякий деревянный сосуд, или одежда, или кожа, или мешок, и всякая вещь, которая употребляется на дело, будут нечисты: в воду должно положить их, и нечисты будут до вечера, потом будут чисты;
Lev 11:33  если же которое-нибудь из них упадет в какой-нибудь глиняный сосуд, то находящееся в нем будет нечисто, и самый [сосуд] разбейте.
Lev 11:34  Всякая пища, которую едят, на которой была вода [из такого] [сосуда], нечиста будет, и всякое питье, которое пьют, во всяком [таком] сосуде нечисто будет.
Lev 11:35  Все, на что упадет что-нибудь от трупа их, нечисто будет: печь и очаг должно разломать, они нечисты; и они должны быть нечисты для вас;
Lev 11:36  только источник и колодезь, вмещающий воду, остаются чистыми; а кто прикоснется к трупу их, тот нечист.
Lev 11:37  И если что-нибудь от трупа их упадет на какое-либо семя, которое сеют, то оно чисто;
Lev 11:38  если же тогда, как вода налита на семя, упадет на него что-- нибудь от трупа их, то оно нечисто для вас.
Lev 11:39  И когда умрет какой-либо скот, который употребляется вами в пищу, то прикоснувшийся к трупу его нечист будет до вечера;
Lev 11:40  и тот, кто будет есть мертвечину его, должен омыть одежды свои и нечист будет до вечера; и тот, кто понесет труп его, должен омыть одежды свои и нечист будет до вечера.
Lev 11:41  Всякое животное, пресмыкающееся по земле, скверно для вас, не должно есть [его];
Lev 11:42  всего ползающего на чреве и всего ходящего на четырех ногах, и многоножных из животных пресмыкающихся по земле, не ешьте, ибо они скверны;
Lev 11:43  не оскверняйте душ ваших каким-либо животным пресмыкающимся и не делайте себя чрез них нечистыми, чтоб быть чрез них нечистыми,
Lev 11:44  ибо Я--Господь Бог ваш: освящайтесь и будьте святы, ибо Я свят; и не оскверняйте душ ваших каким-либо животным, ползающим по земле,
Lev 11:45  ибо Я--Господь, выведший вас из земли Египетской, чтобы быть вашим Богом. Итак будьте святы, потому что Я свят.
Lev 11:46  Вот закон о скоте, о птицах, о всех животных, живущих в водах, и о всех животных, пресмыкающихся по земле,
Lev 11:47  чтобы отличать нечистое от чистого, и животных, которых можно есть, от животных, которых есть не должно.
Lev 12:1  И сказал Господь Моисею, говоря:
Lev 12:2  скажи сынам Израилевым: если женщина зачнет и родит [младенца] мужеского пола, то она нечиста будет семь дней; как во дни страдания ее очищением, она будет нечиста;
Lev 12:3  в восьмой же день обрежется у него крайняя плоть его;
Lev 12:4  и тридцать три дня должна она сидеть, очищаясь от кровей своих; ни к чему священному не должна прикасаться и к святилищу не должна приходить, пока не исполнятся дни очищения ее.
Lev 12:5  Если же она родит [младенца] женского пола, то во время очищения своего она будет нечиста две недели, и шестьдесят шесть дней должна сидеть, очищаясь от кровей своих.
Lev 12:6  По окончании дней очищения своего за сына или за дочь она должна принести однолетнего агнца во всесожжение и молодого голубя или горлицу в жертву за грех, ко входу скинии собрания к священнику;
Lev 12:7  он принесет это пред Господа и очистит ее, и она будет чиста от течения кровей ее. Вот закон о родившей [младенца] мужеского или женского пола.
Lev 12:8  Если же она не в состоянии принести агнца, то пусть возьмет двух горлиц или двух молодых голубей, одного во всесожжение, а другого в жертву за грех, и очистит ее священник, и она будет чиста.
Lev 13:1  И сказал Господь Моисею и Аарону, говоря:
Lev 13:2  когда у кого появится на коже тела его опухоль, или лишаи, или пятно, и на коже тела его сделается как бы язва проказы, то должно привести его к Аарону священнику, или к одному из сынов его, священников;
Lev 13:3  священник осмотрит язву на коже тела, и если волосы на язве изменились в белые, и язва оказывается углубленною в кожу тела его, то это язва проказы; священник, осмотрев его, объявит его нечистым.
Lev 13:4  А если на коже тела его пятно белое, но оно не окажется углубленным в кожу, и волосы на нем не изменились в белые, то священник [имеющего] язву должен заключить на семь дней;
Lev 13:5  в седьмой день священник осмотрит его, и если язва остается в своем виде и не распространяется язва по коже, то священник должен заключить его на другие семь дней;
Lev 13:6  в седьмой день опять священник осмотрит его, и если язва менее приметна и не распространилась язва по коже, то священник должен объявить его чистым: это лишаи, и пусть он омоет одежды свои, и будет чист.
Lev 13:7  Если же лишаи станут распространяться по коже, после того как он являлся к священнику для очищения, то он вторично должен явиться к священнику;
Lev 13:8  священник, увидев, что лишаи распространяются по коже, объявит его нечистым: это проказа.
Lev 13:9  Если будет на ком язва проказы, то должно привести его к священнику;
Lev 13:10  священник осмотрит, и если опухоль на коже бела, и волос изменился в белый, и на опухоли живое мясо,
Lev 13:11  то это застарелая проказа на коже тела его; и священник объявит его нечистым и заключит его, ибо он нечист.
Lev 13:12  Если же проказа расцветет на коже, и покроет проказа всю кожу больного от головы его до ног, сколько могут видеть глаза священника,
Lev 13:13  и увидит священник, что проказа покрыла все тело его, то он объявит больного чистым, потому что все превратилось в белое: он чист.
Lev 13:14  Когда же окажется на нем живое мясо, то он нечист;
Lev 13:15  священник, увидев живое мясо, объявит его нечистым; живое мясо нечисто: это проказа.
Lev 13:16  Если же живое мясо изменится и обратится в белое, пусть он придет к священнику;
Lev 13:17  священник осмотрит его, и если язва обратилась в белое, священник объявит больного чистым; он чист.
Lev 13:18  Если у кого на коже тела был нарыв и зажил,
Lev 13:19  и на месте нарыва появилась белая опухоль, или пятно белое или красноватое, то он должен явиться к священнику;
Lev 13:20  священник осмотрит его, и если оно окажется ниже кожи, и волос его изменился в белый, то священник объявит его нечистым: это язва проказы, она расцвела на нарыве;
Lev 13:21  если же священник увидит, что волос на ней не бел, и она не ниже кожи, и притом мало приметна, то священник заключит его на семь дней;
Lev 13:22  если она станет очень распространяться по коже, то священник объявит его нечистым: это язва;
Lev 13:23  если же пятно остается на своем месте и не распространяется, то это воспаление нарыва, и священник объявит его чистым.
Lev 13:24  Или если у кого на коже тела будет ожог, и на зажившем ожоге окажется красноватое или белое пятно,
Lev 13:25  и священник увидит, что волос на пятне изменился в белый, и оно окажется углубленным в коже, то это проказа, она расцвела на ожоге; и священник объявит его нечистым: это язва проказы;
Lev 13:26  если же священник увидит, что волос на пятне не бел, и оно не ниже кожи, и притом мало приметно, то священник заключит его на семь дней;
Lev 13:27  в седьмой день священник осмотрит его, и если оно очень распространяется по коже, то священник объявит его нечистым: это язва проказы;
Lev 13:28  если же пятно остается на своем месте и не распространяется по коже, и притом мало приметно, то это опухоль от ожога; священник объявит его чистым, ибо это воспаление от ожога.
Lev 13:29  Если у мужчины или у женщины будет язва на голове или на бороде,
Lev 13:30  и осмотрит священник язву, и она окажется углубленною в коже, и волос на ней желтоватый тонкий, то священник объявит их нечистыми: это паршивость, это проказа на голове или на бороде;
Lev 13:31  если же священник осмотрит язву паршивости и она не окажется углубленною в коже, и волос на ней не черный, то священник [имеющего] язву паршивости заключит на семь дней;
Lev 13:32  в седьмой день священник осмотрит язву, и если паршивость не распространяется, и нет на ней желтоватого волоса, и паршивость не окажется углубленною в коже,
Lev 13:33  то [больного] должно остричь, но паршивого места не остригать, и священник должен паршивого вторично заключить на семь дней;
Lev 13:34  в седьмой день священник осмотрит паршивость, и если паршивость не распространяется по коже и не окажется углубленною в коже, то священник объявит его чистым; пусть он омоет одежды свои, и будет чист.
Lev 13:35  Если же после очищения его будет очень распространяться паршивость по коже,
Lev 13:36  и священник увидит, что паршивость распространяется по коже, то священник пусть не ищет желтоватого волоса: он нечист.
Lev 13:37  Если же паршивость остается в своем виде, и показывается на ней волос черный, то паршивость прошла, он чист; священник объявит его чистым.
Lev 13:38  Если у мужчины или у женщины на коже тела их будут пятна, пятна белые,
Lev 13:39  и священник увидит, что на коже тела их пятна бледно-белые, то это лишай, расцветший на коже: он чист.
Lev 13:40  Если у кого на голове вылезли [волосы], то это плешивый: он чист;
Lev 13:41  а если на передней стороне головы вылезли [волосы], то это лысый: он чист.
Lev 13:42  Если же на плеши или на лысине будет белое или красноватое пятно, то на плеши его или на лысине его расцвела проказа;
Lev 13:43  священник осмотрит его, и если увидит, что опухоль язвы бела [или] красновата на плеши его или на лысине его, видом похожа на проказу кожи тела,
Lev 13:44  то он прокаженный, нечист он; священник должен объявить его нечистым, у него на голове язва.
Lev 13:45  У прокаженного, на котором эта язва, должна быть разодрана одежда, и голова его должна быть не покрыта, и до уст он должен быть закрыт и кричать: нечист! нечист!
Lev 13:46  Во все дни, доколе на нем язва, он должен быть нечист, нечист он; он должен жить отдельно, вне стана жилище его.
Lev 13:47  Если язва проказы будет на одежде, на одежде шерстяной, или на одежде льняной,
Lev 13:48  или на основе, или на утоке из льна или шерсти, или на коже, или на каком-нибудь изделии кожаном,
Lev 13:49  и пятно будет зеленоватое или красноватое на одежде, или на коже, или на основе, или на утоке, или на какой-нибудь кожаной вещи, --то это язва проказы: должно показать ее священнику;
Lev 13:50  священник осмотрит язву и заключит зараженное язвою на семь дней;
Lev 13:51  в седьмой день осмотрит священник зараженное, и если язва распространилась по одежде, или по основе, или по утоку, или по коже, или по какому-либо изделию, сделанному из кожи, то это проказа едкая, язва нечистая;
Lev 13:52  он должен сжечь одежду, или основу, или уток шерстяной или льняной, или какую бы то ни было кожаную вещь, на которой будет язва, ибо это проказа едкая: должно сжечь на огне.
Lev 13:53  Если же священник увидит, что язва не распространилась по одежде, или по основе, или по утоку, или по какой бы то ни было кожаной вещи,
Lev 13:54  то священник прикажет омыть то, на чем язва, и вторично заключит на семь дней;
Lev 13:55  если по омытии зараженной [вещи] священник увидит, что язва не изменила вида своего и не распространилась язва, то она нечиста, сожги ее на огне; это выеденная ямина на лицевой стороне или на изнанке;
Lev 13:56  если же священник увидит, что язва по омытии ее сделалась менее приметна, то священник пусть оторвет ее от одежды, или от кожи, или от основы, или от утока.
Lev 13:57  Если же она опять покажется на одежде, или на основе, или на утоке, или на какой-нибудь кожаной вещи, то это расцветающая язва: сожги на огне то, на чем язва.
Lev 13:58  Если же одежду, или основу, или уток, или какую-нибудь кожаную вещь вымоешь, и сойдет с них язва, то должно вымыть их вторично, и они будут чисты.
Lev 13:59  Вот закон о язве проказы на одежде шерстяной или льняной, или на основе и на утоке, или на какой-нибудь кожаной вещи, как объявлять ее чистою или нечистою.
Lev 14:1  И сказал Господь Моисею, говоря:
Lev 14:2  вот закон о прокаженном, когда надобно его очистить: приведут его к священнику;
Lev 14:3  священник выйдет вон из стана, и если священник увидит, что прокаженный исцелился от болезни прокажения,
Lev 14:4  то священник прикажет взять для очищаемого двух птиц живых чистых, кедрового дерева, червленую нить и иссопа,
Lev 14:5  и прикажет священник заколоть одну птицу над глиняным сосудом, над живою водою;
Lev 14:6  а сам он возьмет живую птицу, кедровое дерево, червленую нить и иссоп, и омочит их и живую птицу в крови птицы заколотой над живою водою,
Lev 14:7  и покропит на очищаемого от проказы семь раз, и объявит его чистым, и пустит живую птицу в поле.
Lev 14:8  Очищаемый омоет одежды свои, острижет все волосы свои, омоется водою, и будет чист; потом войдет в стан и пробудет семь дней вне шатра своего;
Lev 14:9  в седьмой день обреет все волосы свои, голову свою, бороду свою, брови глаз своих, все волосы свои обреет, и омоет одежды свои, и омоет тело свое водою, и будет чист;
Lev 14:10  в восьмой день возьмет он двух овнов без порока, и одну овцу однолетнюю без порока, и три десятых части ефы пшеничной муки, смешанной с елеем, в приношение хлебное, и один лог елея;
Lev 14:11  священник очищающий поставит очищаемого человека с ними пред Господом у входа скинии собрания;
Lev 14:12  и возьмет священник одного овна, и представит его в жертву повинности, и лог елея, и принесет это, потрясая пред Господом;
Lev 14:13  и заколет овна на том месте, где заколают жертву за грех и всесожжение, на месте святом, ибо сия жертва повинности, подобно жертве за грех, принадлежит священнику: это великая святыня;
Lev 14:14  и возьмет священник крови жертвы повинности, и возложит священник на край правого уха очищаемого и на большой палец правой руки его и на большой палец правой ноги его;
Lev 14:15  и возьмет священник из лога елея и польет на левую свою ладонь;
Lev 14:16  и омочит священник правый перст свой в елей, который на левой ладони его, и покропит елеем с перста своего семь раз пред лицем Господа;
Lev 14:17  оставшийся же елей, который на ладони его, возложит священник на край правого уха очищаемого, на большой палец правой руки его и на большой палец правой ноги его, на [места, где] кровь жертвы повинности;
Lev 14:18  а остальной елей, который на ладони священника, возложит он на голову очищаемого, и очистит его священник пред лицем Господа.
Lev 14:19  И совершит священник жертву за грех, и очистит очищаемого от нечистоты его; после того заколет [жертву] всесожжения;
Lev 14:20  и возложит священник всесожжение и приношение хлебное на жертвенник; и очистит его священник, и он будет чист.
Lev 14:21  Если же он беден и не имеет достатка, то пусть возьмет одного овна в жертву повинности для потрясания, чтоб очистить себя, и одну десятую часть [ефы] пшеничной муки, смешанной с елеем, в приношение хлебное, и лог елея,
Lev 14:22  и двух горлиц или двух молодых голубей, что достанет рука его, одну [из птиц] в жертву за грех, а другую во всесожжение;
Lev 14:23  и принесет их в восьмой день очищения своего к священнику ко входу скинии собрания, пред лице Господа;
Lev 14:24  священник возьмет овна жертвы повинности и лог елея, и принесет это священник, потрясая пред Господом;
Lev 14:25  и заколет овна в жертву повинности, и возьмет священник крови жертвы повинности, и возложит на край правого уха очищаемого и на большой палец правой руки его и на большой палец правой ноги его;
Lev 14:26  и нальет священник елея на левую свою ладонь,
Lev 14:27  и елеем, который на левой ладони его, покропит священник с правого перста своего семь раз пред лицем Господним;
Lev 14:28  и возложит священник елея, который на ладони его, на край правого уха очищаемого, на большой палец правой руки его и на большой палец правой ноги его, на места, [где] кровь жертвы повинности;
Lev 14:29  а остальной елей, который на ладони священника, возложит он на голову очищаемого, чтоб очистить его пред лицем Господа;
Lev 14:30  и принесет одну из горлиц или одного из молодых голубей, что достанет рука [очищаемого],
Lev 14:31  из того, что достанет рука его, одну [птицу] в жертву за грех, а другую во всесожжение, вместе с приношением хлебным; и очистит священник очищаемого пред лицем Господа.
Lev 14:32  Вот закон о прокаженном, который во время очищения своего не имеет достатка.
Lev 14:33  И сказал Господь Моисею и Аарону, говоря:
Lev 14:34  когда войдете в землю Ханаанскую, которую Я даю вам во владение, и Я наведу язву проказы на домы в земле владения вашего,
Lev 14:35  тогда тот, чей дом, должен пойти и сказать священнику: у меня на доме показалась как бы язва.
Lev 14:36  Священник прикажет опорожнить дом, прежде нежели войдет священник осматривать язву, чтобы не сделалось нечистым все, что в доме; после сего придет священник осматривать дом.
Lev 14:37  Если он, осмотрев язву, увидит, что язва на стенах дома состоит из зеленоватых или красноватых ямин, которые окажутся углубленными в стене,
Lev 14:38  то священник выйдет из дома к дверям дома и запрет дом на семь дней.
Lev 14:39  В седьмой день опять придет священник, и если увидит, что язва распространилась по стенам дома,
Lev 14:40  то священник прикажет выломать камни, на которых язва, и бросить их вне города на место нечистое;
Lev 14:41  а дом внутри пусть весь оскоблят, и обмазку, которую отскоблят, высыпят вне города на место нечистое;
Lev 14:42  и возьмут другие камни, и вставят вместо тех камней, и возьмут другую обмазку, и обмажут дом.
Lev 14:43  Если язва опять появится и будет цвести на доме после того, как выломали камни и оскоблили дом и обмазали,
Lev 14:44  то священник придет и осмотрит, и если язва на доме распространилась, то это едкая проказа на доме, нечист он;
Lev 14:45  должно разломать сей дом, и камни его и дерево его и всю обмазку дома вынести вне города на место нечистое;
Lev 14:46  кто входит в дом во все время, когда он заперт, тот нечист до вечера;
Lev 14:47  и кто спит в доме том, тот должен вымыть одежды свои; и кто ест в доме том, тот должен вымыть одежды свои.
Lev 14:48  Если же священник придет и увидит, что язва на доме не распространилась после того, как обмазали дом, то священник объявит дом чистым, потому что язва прошла.
Lev 14:49  И чтобы очистить дом, возьмет он две птицы, кедрового дерева, червленую нить и иссопа,
Lev 14:50  и заколет одну птицу над глиняным сосудом, над живою водою;
Lev 14:51  и возьмет кедровое дерево и иссоп, и червленую нить и живую птицу, и омочит их в крови птицы заколотой и в живой воде, и покропит дом семь раз;
Lev 14:52  и очистит дом кровью птицы и живою водою, и живою птицею и кедровым деревом, и иссопом и червленою нитью;
Lev 14:53  и пустит живую птицу вне города в поле и очистит дом, и будет чист.
Lev 14:54  Вот закон о всякой язве проказы и о паршивости,
Lev 14:55  и о проказе на одежде и на доме, и об опухоли, и о лишаях, и о пятнах, --
Lev 14:56  чтобы указать, когда это нечисто и когда чисто. Вот закон о проказе.
Lev 15:1  И сказал Господь Моисею и Аарону, говоря:
Lev 15:2  объявите сынам Израилевым и скажите им: если у кого будет истечение из тела его, то от истечения своего он нечист.
Lev 15:3  И вот [закон] о нечистоте его от истечения его: когда течет из тела его истечение его, и когда задерживается в теле его истечение его, это нечистота его;
Lev 15:4  всякая постель, на которой ляжет имеющий истечение, нечиста, и всякая вещь, на которую сядет, нечиста;
Lev 15:5  и кто прикоснется к постели его, тот должен вымыть одежды свои и омыться водою и нечист будет до вечера;
Lev 15:6  кто сядет на какую-либо вещь, на которой сидел имеющий истечение, тот должен вымыть одежды свои и омыться водою и нечист будет до вечера;
Lev 15:7  и кто прикоснется к телу имеющего истечение, тот должен вымыть одежды свои и омыться водою и нечист будет до вечера;
Lev 15:8  если имеющий истечение плюнет на чистого, то сей должен вымыть одежды свои и омыться водою, и нечист будет до вечера;
Lev 15:9  и всякая повозка, в которой ехал имеющий истечение, нечиста [будет до вечера];
Lev 15:10  и всякий, кто прикоснется к чему-нибудь, что было под ним, нечист будет до вечера; и кто понесет это, должен вымыть одежды свои и омыться водою, и нечист будет до вечера;
Lev 15:11  и всякий, к кому прикоснется имеющий истечение, не омыв рук своих водою, должен вымыть одежды свои и омыться водою, и нечист будет до вечера;
Lev 15:12  глиняный сосуд, к которому прикоснется имеющий истечение, должно разбить, а всякий деревянный сосуд должно вымыть водою.
Lev 15:13  А когда имеющий истечение освободится от истечения своего, тогда должен он отсчитать себе семь дней для очищения своего и вымыть одежды свои, и омыть тело свое живою водою, и будет чист;
Lev 15:14  и в восьмой день возьмет он себе двух горлиц или двух молодых голубей, и придет пред лице Господне ко входу скинии собрания, и отдаст их священнику;
Lev 15:15  и принесет священник из сих [птиц] одну в жертву за грех, а другую во всесожжение, и очистит его священник пред Господом от истечения его.
Lev 15:16  Если у кого случится излияние семени, то он должен омыть водою все тело свое, и нечист будет до вечера;
Lev 15:17  и всякая одежда и всякая кожа, на которую попадет семя, должна быть вымыта водою, и нечиста будет до вечера;
Lev 15:18  если мужчина ляжет с женщиной и [будет] у него излияние семени, то они должны омыться водою, и нечисты будут до вечера.
Lev 15:19  Если женщина имеет истечение крови, текущей из тела ее, то она должна сидеть семь дней во время очищения своего, и всякий, кто прикоснется к ней, нечист будет до вечера;
Lev 15:20  и все, на чем она ляжет в продолжение очищения своего, нечисто; и все, на чем сядет, нечисто;
Lev 15:21  и всякий, кто прикоснется к постели ее, должен вымыть одежды свои и омыться водою и нечист будет до вечера;
Lev 15:22  и всякий, кто прикоснется к какой-нибудь вещи, на которой она сидела, должен вымыть одежды свои и омыться водою, и нечист будет до вечера;
Lev 15:23  и если кто прикоснется к чему-нибудь на постели или на той вещи, на которой она сидела, нечист будет до вечера;
Lev 15:24  если переспит с нею муж, то нечистота ее будет на нем; он нечист будет семь дней, и всякая постель, на которой он ляжет, будет нечиста.
Lev 15:25  Если у женщины течет кровь многие дни не во время очищения ее, или если она имеет истечение долее [обыкновенного] очищения ее, то во все время истечения нечистоты ее, подобно как в продолжение очищения своего, она нечиста;
Lev 15:26  всякая постель, на которой она ляжет во все время истечения своего, будет [нечиста], подобно как постель в продолжение очищения ее; и всякая вещь, на которую она сядет, будет нечиста, как нечисто это во время очищения ее;
Lev 15:27  и всякий, кто прикоснется к ним, будет нечист, и должен вымыть одежды свои и омыться водою, и нечист будет до вечера.
Lev 15:28  А когда она освободится от истечения своего, тогда должна отсчитать себе семь дней, и потом будет чиста;
Lev 15:29  в восьмой день возьмет она себе двух горлиц или двух молодых голубей и принесет их к священнику ко входу скинии собрания;
Lev 15:30  и принесет священник одну [из птиц] в жертву за грех, а другую во всесожжение, и очистит ее священник пред Господом от истечения нечистоты ее.
Lev 15:31  Так предохраняйте сынов Израилевых от нечистоты их, чтоб они не умерли в нечистоте своей, оскверняя жилище Мое, которое среди них:
Lev 15:32  вот закон об имеющем истечение и о том, у кого случится излияние семени, делающее его нечистым,
Lev 15:33  и о страдающей очищением своим, и о имеющих истечение, мужчине или женщине, и о муже, который переспит с нечистою.
Lev 16:1  И говорил Господь Моисею по смерти двух сынов Аароновых, когда они, приступив пред лице Господне, умерли,
Lev 16:2  и сказал Господь Моисею: скажи Аарону, брату твоему, чтоб он не во всякое время входил во святилище за завесу пред крышку, что на ковчеге, дабы ему не умереть, ибо над крышкою Я буду являться в облаке.
Lev 16:3  Вот с чем должен входить Аарон во святилище: с тельцом в жертву за грех и с овном во всесожжение;
Lev 16:4  священный льняной хитон должен надевать он, нижнее платье льняное да будет на теле его, и льняным поясом пусть опоясывается, и льняной кидар надевает: это священные одежды; и пусть омывает он тело свое водою и надевает их;
Lev 16:5  и от общества сынов Израилевых пусть возьмет двух козлов в жертву за грех и одного овна во всесожжение.
Lev 16:6  И принесет Аарон тельца в жертву за грех за себя и очистит себя и дом свой.
Lev 16:7  И возьмет двух козлов и поставит их пред лицем Господним у входа скинии собрания;
Lev 16:8  и бросит Аарон об обоих козлах жребии: один жребий для Господа, а другой жребий для отпущения;
Lev 16:9  и приведет Аарон козла, на которого вышел жребий для Господа, и принесет его в жертву за грех,
Lev 16:10  а козла, на которого вышел жребий для отпущения, поставит живого пред Господом, чтобы совершить над ним очищение и отослать его в пустыню для отпущения.
Lev 16:11  И приведет Аарон тельца в жертву за грех за себя, и очистит себя и дом свой, и заколет тельца в жертву за грех за себя;
Lev 16:12  и возьмет горящих угольев полную кадильницу с жертвенника, который пред лицем Господним, и благовонного мелко истолченного курения полные горсти, и внесет за завесу;
Lev 16:13  и положит курение на огонь пред лицем Господним, и облако курения покроет крышку, которая над [ковчегом] откровения, дабы ему не умереть;
Lev 16:14  и возьмет крови тельца и покропит перстом своим на крышку спереди и пред крышкою, семь раз покропит кровью с перста своего.
Lev 16:15  И заколет козла в жертву за грех за народ, и внесет кровь его за завесу, и сделает с кровью его то же, что делал с кровью тельца и покропит ею на крышку и пред крышкою, --
Lev 16:16  и очистит святилище от нечистот сынов Израилевых и от преступлений их, во всех грехах их. Так должен поступить он и со скиниею собрания, находящеюся у них, среди нечистот их.
Lev 16:17  Ни один человек не должен быть в скинии собрания, когда входит он для очищения святилища, до самого выхода его. И так очистит он себя, дом свой и все общество Израилево.
Lev 16:18  И выйдет он к жертвеннику, который пред лицем Господним, и очистит его, и возьмет крови тельца и крови козла, и возложит на роги жертвенника со всех сторон,
Lev 16:19  и покропит на него кровью с перста своего семь раз, и очистит его, и освятит его от нечистот сынов Израилевых.
Lev 16:20  И совершив очищение святилища, скинии собрания и жертвенника, приведет он живого козла,
Lev 16:21  и возложит Аарон обе руки свои на голову живого козла, и исповедает над ним все беззакония сынов Израилевых и все преступления их и все грехи их, и возложит их на голову козла, и отошлет с нарочным человеком в пустыню:
Lev 16:22  и понесет козел на себе все беззакония их в землю непроходимую, и пустит он козла в пустыню.
Lev 16:23  И войдет Аарон в скинию собрания, и снимет льняные одежды, которые надевал, входя во святилище, и оставит их там,
Lev 16:24  и омоет тело свое водою на святом месте, и наденет одежды свои, и выйдет и совершит всесожжение за себя и всесожжение за народ, и очистит себя и народ;
Lev 16:25  а тук жертвы за грех воскурит на жертвеннике.
Lev 16:26  И тот, кто отводил козла для отпущения, должен вымыть одежды свои, омыть тело свое водою, и потом может войти в стан.
Lev 16:27  А тельца за грех и козла за грех, которых кровь внесена была для очищения святилища, пусть вынесут вон из стана и сожгут на огне кожи их и мясо их и нечистоту их;
Lev 16:28  кто сожжет их, тот должен вымыть одежды свои и омыть тело свое водою, и после того может войти в стан.
Lev 16:29  И да будет сие для вас вечным постановлением: в седьмой месяц, в десятый [день] месяца смиряйте души ваши и никакого дела не делайте, ни туземец, ни пришлец, поселившийся между вами,
Lev 16:30  ибо в сей день очищают вас, чтобы сделать вас чистыми от всех грехов ваших, чтобы вы были чисты пред лицем Господним;
Lev 16:31  это суббота покоя для вас, смиряйте души ваши: это постановление вечное.
Lev 16:32  Очищать же должен священник, который помазан и который посвящен, чтобы священнодействовать ему вместо отца своего: и наденет он льняные одежды, одежды священные,
Lev 16:33  и очистит Святое-святых и скинию собрания, и жертвенник очистит, и священников и весь народ общества очистит.
Lev 16:34  И да будет сие для вас вечным постановлением: очищать сынов Израилевых от всех грехов их однажды в году. И сделал он так, как повелел Господь Моисею.
Lev 17:1  И сказал Господь Моисею, говоря:
Lev 17:2  объяви Аарону и сынам его и всем сынам Израилевым и скажи им: вот что повелевает Господь:
Lev 17:3  если кто из дома Израилева заколет тельца или овцу или козу в стане, или если кто заколет вне стана
Lev 17:4  и не приведет ко входу скинии собрания, чтобы представить в жертву Господу пред жилищем Господним, то человеку тому вменена будет кровь: он пролил кровь, и истребится человек тот из народа своего;
Lev 17:5  [это] для того, чтобы приводили сыны Израилевы жертвы свои, которые они заколают на поле, чтобы приводили их пред Господа ко входу скинии собрания, к священнику, и заколали их Господу в жертвы мирные;
Lev 17:6  и покропит священник кровью на жертвенник Господень у входа скинии собрания и воскурит тук в приятное благоухание Господу,
Lev 17:7  чтоб они впредь не приносили жертв своих идолам, за которыми блудно ходят они. Сие да будет для них постановлением вечным в роды их.
Lev 17:8  [Еще] скажи им: если кто из дома Израилева и из пришельцев, которые живут между вами, приносит всесожжение или жертву
Lev 17:9  и не приведет ко входу скинии собрания, чтобы совершить ее Господу, то истребится человек тот из народа своего.
Lev 17:10  Если кто из дома Израилева и из пришельцев, которые живут между вами, будет есть какую-нибудь кровь, то обращу лице Мое на душу того, кто будет есть кровь, и истреблю ее из народа ее,
Lev 17:11  потому что душа тела в крови, и Я назначил ее вам для жертвенника, чтобы очищать души ваши, ибо кровь сия душу очищает;
Lev 17:12  потому Я и сказал сынам Израилевым: ни одна душа из вас не должна есть крови, и пришлец, живущий между вами, не должен есть крови.
Lev 17:13  Если кто из сынов Израилевых и из пришельцев, живущих между вами, на ловле поймает зверя или птицу, которую можно есть, то он должен дать вытечь крови ее и покрыть ее землею,
Lev 17:14  ибо душа всякого тела [есть] кровь его, она душа его; потому Я сказал сынам Израилевым: не ешьте крови ни из какого тела, потому что душа всякого тела есть кровь его: всякий, кто будет есть ее, истребится.
Lev 17:15  И всякий, кто будет есть мертвечину или растерзанное зверем, туземец или пришлец, должен вымыть одежды свои и омыться водою, и нечист будет до вечера, а [потом] будет чист;
Lev 17:16  если же не вымоет [одежд своих] и не омоет тела своего, то понесет на себе беззаконие свое.
Lev 18:1  И сказал Господь Моисею, говоря:
Lev 18:2  объяви сынам Израилевым и скажи им: Я Господь, Бог ваш.
Lev 18:3  По делам земли Египетской, в которой вы жили, не поступайте, и по делам земли Ханаанской, в которую Я веду вас, не поступайте, и по установлениям их не ходите:
Lev 18:4  Мои законы исполняйте и Мои постановления соблюдайте, поступая по ним. Я Господь, Бог ваш.
Lev 18:5  Соблюдайте постановления Мои и законы Мои, которые исполняя, человек будет жив. Я Господь.
Lev 18:6  Никто ни к какой родственнице по плоти не должен приближаться с тем, чтобы открыть наготу. Я Господь.
Lev 18:7  Наготы отца твоего и наготы матери твоей не открывай: она мать твоя, не открывай наготы ее.
Lev 18:8  Наготы жены отца твоего не открывай: это нагота отца твоего.
Lev 18:9  Наготы сестры твоей, дочери отца твоего или дочери матери твоей, родившейся в доме или вне дома, не открывай наготы их.
Lev 18:10  Наготы дочери сына твоего или дочери дочери твоей, не открывай наготы их, ибо они твоя нагота.
Lev 18:11  Наготы дочери жены отца твоего, родившейся от отца твоего, она сестра твоя [по отцу], не открывай наготы ее.
Lev 18:12  Наготы сестры отца твоего не открывай, она единокровная отцу твоему.
Lev 18:13  Наготы сестры матери твоей не открывай, ибо она единокровная матери твоей.
Lev 18:14  Наготы брата отца твоего не открывай и к жене его не приближайся: она тетка твоя.
Lev 18:15  Наготы невестки твоей не открывай: она жена сына твоего, не открывай наготы ее.
Lev 18:16  Наготы жены брата твоего не открывай, это нагота брата твоего.
Lev 18:17  Наготы жены и дочери ее не открывай; дочери сына ее и дочери дочери ее не бери, чтоб открыть наготу их, они единокровные ее; это беззаконие.
Lev 18:18  Не бери жены вместе с сестрою ее, чтобы сделать ее соперницею, чтоб открыть наготу ее при ней, при жизни ее.
Lev 18:19  И к жене во время очищения нечистот ее не приближайся, чтоб открыть наготу ее.
Lev 18:20  И с женою ближнего твоего не ложись, чтобы излить семя и оскверниться с нею.
Lev 18:21  Из детей твоих не отдавай на служение Молоху и не бесчести имени Бога твоего. Я Господь.
Lev 18:22  Не ложись с мужчиною, как с женщиною: это мерзость.
Lev 18:23  И ни с каким скотом не ложись, чтоб излить [семя] и оскверниться от него; и женщина не должна становиться пред скотом для совокупления с ним: это гнусно.
Lev 18:24  Не оскверняйте себя ничем этим, ибо всем этим осквернили себя народы, которых Я прогоняю от вас:
Lev 18:25  и осквернилась земля, и Я воззрел на беззаконие ее, и свергнула с себя земля живущих на ней.
Lev 18:26  А вы соблюдайте постановления Мои и законы Мои и не делайте всех этих мерзостей, ни туземец, ни пришлец, живущий между вами,
Lev 18:27  ибо все эти мерзости делали люди сей земли, что пред вами, и осквернилась земля;
Lev 18:28  чтоб и вас не свергнула с себя земля, когда вы станете осквернять ее, как она свергнула народы, бывшие прежде вас;
Lev 18:29  ибо если кто будет делать все эти мерзости, то души делающих это истреблены будут из народа своего.
Lev 18:30  Итак соблюдайте повеления Мои, чтобы не поступать по гнусным обычаям, по которым поступали прежде вас, и чтобы не оскверняться ими. Я Господь, Бог ваш.
Lev 19:1  И сказал Господь Моисею, говоря:
Lev 19:2  объяви всему обществу сынов Израилевых и скажи им: святы будьте, ибо свят Я Господь, Бог ваш.
Lev 19:3  Бойтесь каждый матери своей и отца своего и субботы Мои храните. Я Господь, Бог ваш.
Lev 19:4  Не обращайтесь к идолам и богов литых не делайте себе. Я Господь, Бог ваш.
Lev 19:5  Когда будете приносить Господу жертву мирную, то приносите ее, чтобы приобрести себе благоволение:
Lev 19:6  в день жертвоприношения вашего и на другой день должно есть ее, а оставшееся к третьему дню должно сжечь на огне;
Lev 19:7  если же кто станет есть ее на третий день, это гнусно, это не будет благоприятно;
Lev 19:8  кто станет есть ее, тот понесет на себе грех, ибо он осквернил святыню Господню, и истребится душа та из народа своего.
Lev 19:9  Когда будете жать жатву на земле вашей, не дожинай до края поля твоего, и оставшегося от жатвы твоей не подбирай,
Lev 19:10  и виноградника твоего не обирай дочиста, и попадавших ягод в винограднике не подбирай; оставь это бедному и пришельцу. Я Господь, Бог ваш.
Lev 19:11  Не крадите, не лгите и не обманывайте друг друга.
Lev 19:12  Не клянитесь именем Моим во лжи, и не бесчести имени Бога твоего. Я Господь.
Lev 19:13  Не обижай ближнего твоего и не грабительствуй. Плата наемнику не должна оставаться у тебя до утра.
Lev 19:14  Не злословь глухого и пред слепым не клади ничего, чтобы преткнуться ему; бойся Бога твоего. Я Господь.
Lev 19:15  Не делайте неправды на суде; не будь лицеприятен к нищему и не угождай лицу великого; по правде суди ближнего твоего.
Lev 19:16  Не ходи переносчиком в народе твоем и не восставай на жизнь ближнего твоего. Я Господь.
Lev 19:17  Не враждуй на брата твоего в сердце твоем; обличи ближнего твоего, и не понесешь за него греха.
Lev 19:18  Не мсти и не имей злобы на сынов народа твоего, но люби ближнего твоего, как самого себя. Я Господь.
Lev 19:19  Уставы Мои соблюдайте; скота твоего не своди с иною породою; поля твоего не засевай двумя родами [семян]; в одежду из разнородных нитей, из шерсти и льна, не одевайся.
Lev 19:20  Если кто переспит с женщиною, а она раба, обрученная мужу, но еще не выкупленная, или свобода еще не дана ей, то должно наказать их, но не смертью, потому что она несвободная:
Lev 19:21  пусть приведет он Господу ко входу скинии собрания жертву повинности, овна в жертву повинности своей;
Lev 19:22  и очистит его священник овном повинности пред Господом от греха, которым он согрешил, и прощен будет ему грех, которым он согрешил.
Lev 19:23  Когда придете в землю, [которую Господь Бог даст вам], и посадите какое-либо плодовое дерево, то плоды его почитайте за необрезанные: три года должно почитать их за необрезанные, не должно есть их;
Lev 19:24  а в четвертый год все плоды его должны быть посвящены для празднеств Господних;
Lev 19:25  в пятый же год вы можете есть плоды его и собирать себе все произведения его. Я Господь, Бог ваш.
Lev 19:26  Не ешьте с кровью; не ворожите и не гадайте.
Lev 19:27  Не стригите головы вашей кругом, и не порти края бороды твоей.
Lev 19:28  Ради умершего не делайте нарезов на теле вашем и не накалывайте на себе письмен. Я Господь.
Lev 19:29  Не оскверняй дочери твоей, допуская ее до блуда, чтобы не блудодействовала земля и не наполнилась земля развратом.
Lev 19:30  Субботы Мои храните и святилище Мое чтите. Я Господь.
Lev 19:31  Не обращайтесь к вызывающим мертвых, и к волшебникам не ходите, и не доводите себя до осквернения от них. Я Господь, Бог ваш.
Lev 19:32  Пред лицем седого вставай и почитай лице старца, и бойся Бога твоего. Я Господь.
Lev 19:33  Когда поселится пришлец в земле вашей, не притесняйте его:
Lev 19:34  пришлец, поселившийся у вас, да будет для вас то же, что туземец ваш; люби его, как себя; ибо и вы были пришельцами в земле Египетской. Я Господь, Бог ваш.
Lev 19:35  Не делайте неправды в суде, в мере, в весе и в измерении:
Lev 19:36  да будут у вас весы верные, гири верные, ефа верная и гин верный. Я Господь, Бог ваш, Который вывел вас из земли Египетской.
Lev 19:37  Соблюдайте все уставы Мои и все законы Мои и исполняйте их. Я Господь.
Lev 20:1  И сказал Господь Моисею, говоря:
Lev 20:2  скажи сие сынам Израилевым: кто из сынов Израилевых и из пришельцев, живущих между Израильтянами, даст из детей своих Молоху, тот да будет предан смерти: народ земли да побьет его камнями;
Lev 20:3  и Я обращу лице Мое на человека того и истреблю его из народа его за то, что он дал из детей своих Молоху, чтоб осквернить святилище Мое и обесчестить святое имя Мое;
Lev 20:4  и если народ земли не обратит очей своих на человека того, когда он даст из детей своих Молоху, и не умертвит его,
Lev 20:5  то Я обращу лице Мое на человека того и на род его и истреблю его из народа его, и всех блудящих по следам его, чтобы блудно ходить вслед Молоха.
Lev 20:6  И если какая душа обратится к вызывающим мертвых и к волшебникам, чтобы блудно ходить вслед их, то Я обращу лице Мое на ту душу и истреблю ее из народа ее.
Lev 20:7  Освящайте себя и будьте святы, ибо Я Господь, Бог ваш, [свят].
Lev 20:8  Соблюдайте постановления Мои и исполняйте их, ибо Я Господь, освящающий вас.
Lev 20:9  Кто будет злословить отца своего или мать свою, тот да будет предан смерти; отца своего и мать свою он злословил: кровь его на нем.
Lev 20:10  Если кто будет прелюбодействовать с женой замужнею, если кто будет прелюбодействовать с женою ближнего своего, --да будут преданы смерти и прелюбодей и прелюбодейка.
Lev 20:11  Кто ляжет с женою отца своего, тот открыл наготу отца своего: оба они да будут преданы смерти, кровь их на них.
Lev 20:12  Если кто ляжет с невесткою своею, то оба они да будут преданы смерти: мерзость сделали они, кровь их на них.
Lev 20:13  Если кто ляжет с мужчиною, как с женщиною, то оба они сделали мерзость: да будут преданы смерти, кровь их на них.
Lev 20:14  Если кто возьмет себе жену и мать ее: это беззаконие; на огне должно сжечь его и их, чтобы не было беззакония между вами.
Lev 20:15  Кто смесится со скотиною, того предать смерти, и скотину убейте.
Lev 20:16  Если женщина пойдет к какой-нибудь скотине, чтобы совокупиться с нею, то убей женщину и скотину: да будут они преданы смерти, кровь их на них.
Lev 20:17  Если кто возьмет сестру свою, дочь отца своего или дочь матери своей, и увидит наготу ее, и она увидит наготу его: это срам, да будут они истреблены пред глазами сынов народа своего; он открыл наготу сестры своей: грех свой понесет он.
Lev 20:18  Если кто ляжет с женою во время болезни [кровоочищения] и откроет наготу ее, то он обнажил истечения ее, и она открыла течение кровей своих: оба они да будут истреблены из народа своего.
Lev 20:19  Наготы сестры матери твоей и сестры отца твоего не открывай, ибо таковой обнажает плоть свою: грех свой понесут они.
Lev 20:20  Кто ляжет с теткою своею, тот открыл наготу дяди своего; грех свой понесут они, бездетными умрут.
Lev 20:21  Если кто возьмет жену брата своего: это гнусно; он открыл наготу брата своего, бездетны будут они.
Lev 20:22  Соблюдайте все уставы Мои и все законы Мои и исполняйте их, --и не свергнет вас с себя земля, в которую Я веду вас жить.
Lev 20:23  Не поступайте по обычаям народа, который Я прогоняю от вас; ибо они все это делали, и Я вознегодовал на них,
Lev 20:24  и сказал Я вам: вы владейте землею их, и вам отдаю в наследие землю, в которой течет молоко и мед. Я Господь, Бог ваш, Который отделил вас от всех народов.
Lev 20:25  Отличайте скот чистый от нечистого и птицу чистую от нечистой и не оскверняйте душ ваших скотом и птицею и всем пресмыкающимся по земле, что отличил Я, как нечистое.
Lev 20:26  Будьте предо Мною святы, ибо Я свят Господь, и Я отделил вас от народов, чтобы вы были Мои.
Lev 20:27  Мужчина ли или женщина, если будут они вызывать мертвых или волхвовать, да будут преданы смерти: камнями должно побить их, кровь их на них.
Lev 21:1  И сказал Господь Моисею: объяви священникам, сынам Аароновым, и скажи им: да не оскверняют себя [прикосновением] к умершему из народа своего;
Lev 21:2  только к ближнему родственнику своему, к матери своей и к отцу своему, к сыну своему и дочери своей, к брату своему
Lev 21:3  и к сестре своей, девице, живущей при нем и не бывшей замужем, можно ему [прикасаться], не оскверняя себя;
Lev 21:4  и [прикосновением] к кому бы то ни было в народе своем не должен он осквернять себя, чтобы не сделаться нечистым.
Lev 21:5  Они не должны брить головы своей и подстригать края бороды своей и делать нарезы на теле своем.
Lev 21:6  Они должны быть святы Богу своему и не должны бесчестить имени Бога своего, ибо они приносят жертвы Господу, хлеб Богу своему, и потому должны быть святы.
Lev 21:7  Они не должны брать за себя блудницу и опороченную, не должны брать и жену, отверженную мужем своим, ибо они святы Богу своему.
Lev 21:8  Святи его, ибо он приносит хлеб Богу твоему: да будет он у тебя свят, ибо свят Я Господь, освящающий вас.
Lev 21:9  Если дочь священника осквернит себя блудодеянием, то она бесчестит отца своего; огнем должно сжечь ее.
Lev 21:10  Великий же священник из братьев своих, на голову которого возлит елей помазания, и который освящен, чтобы облачаться в [священные] одежды, не должен обнажать головы своей и раздирать одежд своих;
Lev 21:11  и ни к какому умершему не должен он приступать: даже [прикосновением к умершему] отцу своему и матери своей он не должен осквернять себя.
Lev 21:12  И от святилища он не должен отходить и бесчестить святилище Бога своего, ибо освящение елеем помазания Бога его на нем. Я Господь.
Lev 21:13  В жену он должен брать девицу.
Lev 21:14  вдову, или отверженную, или опороченную, [или] блудницу, не должен он брать, но девицу из народа своего должен он брать в жену;
Lev 21:15  он не должен порочить семени своего в народе своем, ибо Я Господь, освящающий его.
Lev 21:16  И сказал Господь Моисею, говоря:
Lev 21:17  скажи Аарону: никто из семени твоего во [все] роды их, у которого [на теле] будет недостаток, не должен приступать, чтобы приносить хлеб Богу своему;
Lev 21:18  никто, у кого на теле есть недостаток, не должен приступать, ни слепой, ни хромой, ни уродливый,
Lev 21:19  ни такой, у которого переломлена нога или переломлена рука,
Lev 21:20  ни горбатый, ни с сухим членом, ни с бельмом на глазу, ни коростовый, ни паршивый, ни с поврежденными ятрами;
Lev 21:21  ни один человек из семени Аарона священника, у которого [на] [теле] есть недостаток, не должен приступать, чтобы приносить жертвы Господу; недостаток [на нем], поэтому не должен он приступать, чтобы приносить хлеб Богу своему;
Lev 21:22  хлеб Бога своего из великих святынь и из святынь он может есть;
Lev 21:23  но к завесе не должен он приходить и к жертвеннику не должен приступать, потому что недостаток на нем: не должен он бесчестить святилища Моего, ибо Я Господь, освящающий их.
Lev 21:24  И объявил [это] Моисей Аарону и сынам его и всем сынам Израилевым.
Lev 22:1  И сказал Господь Моисею, говоря:
Lev 22:2  скажи Аарону и сынам его, чтоб они осторожно поступали со святынями сынов Израилевых и не бесчестили святаго имени Моего в том, что они посвящают Мне. Я Господь.
Lev 22:3  Скажи им: если кто из всего потомства вашего в роды ваши, имея на себе нечистоту, приступит к святыням, которые посвящают сыны Израилевы Господу, то истребится душа та от лица Моего. Я Господь.
Lev 22:4  Кто из семени Ааронова прокажен, или имеет истечение, тот не должен есть святынь, пока не очистится; и кто прикоснется к чему-- нибудь нечистому от мертвого, или у кого случится излияние семени,
Lev 22:5  или кто прикоснется к какому-нибудь гаду, от которого он сделается нечист, или к человеку, от которого он сделается нечист какою бы то ни было нечистотою, --
Lev 22:6  тот, прикоснувшийся к сему, нечист будет до вечера и не должен есть святынь, прежде нежели омоет тело свое водою;
Lev 22:7  но когда зайдет солнце и он очистится, тогда может он есть святыни, ибо это его пища.
Lev 22:8  Мертвечины и звероядины он не должен есть, чтобы не оскверниться этим. Я Господь.
Lev 22:9  Да соблюдают они повеления Мои, чтобы не понести на себе греха и не умереть в нем, когда нарушат сие. Я Господь, освящающий их.
Lev 22:10  Никто посторонний не должен есть святыни; поселившийся у священника и наемник не должен есть святыни;
Lev 22:11  если же священник купит себе человека за серебро, то сей может есть оную; также и домочадцы его могут есть хлеб его.
Lev 22:12  Если дочь священника выйдет в замужество за постороннего, то она не должна есть приносимых святынь;
Lev 22:13  когда же дочь священника будет вдова, или разведенная, и детей нет у нее, и возвратится в дом отца своего, как [была] в юности своей, тогда она может есть хлеб отца своего; а посторонний никто не должен есть его.
Lev 22:14  Кто по ошибке съест [что-нибудь] из святыни, тот должен отдать священнику святыню и приложить к ней пятую ее долю.
Lev 22:15  [Священники] сами не должны порочить святыни сынов Израилевых, которые они приносят Господу,
Lev 22:16  и не должны навлекать на себя вину в преступлении, когда будут есть святыни свои, ибо Я Господь, освящающий их.
Lev 22:17  И сказал Господь Моисею, говоря:
Lev 22:18  объяви Аарону и сынам его и всем сынам Израилевым и скажи им: если кто из дома Израилева, или из пришельцев, [поселившихся] между Израильтянами, по обету ли какому, или по усердию приносит жертву свою, которую приносят Господу во всесожжение,
Lev 22:19  то, чтобы сим приобрести благоволение [от Бога, жертва] [должна быть] без порока, мужеского пола, из крупного скота, из овец и из коз;
Lev 22:20  никакого [животного], на котором есть порок, не приносите; ибо это не приобретет вам благоволения.
Lev 22:21  И если кто приносит мирную жертву Господу, исполняя обет, или по усердию, из крупного скота или из мелкого, то [жертва должна быть] без порока, чтоб быть угодною [Богу]: никакого порока не должно быть на ней;
Lev 22:22  [животного] слепого, или поврежденного, или уродливого, или больного, или коростового, или паршивого, таких не приносите Господу и в жертву не давайте их на жертвенник Господень;
Lev 22:23  тельца и агнца с членами, несоразмерно длинными или короткими, в жертву усердия принести можешь; а если по обету, то это не угодно будет [Богу];
Lev 22:24  [животного], у которого ятра раздавлены, разбиты, оторваны или вырезаны, не приносите Господу и в земле вашей не делайте [сего];
Lev 22:25  и из рук иноземцев не приносите всех таковых [животных] в дар Богу вашему, потому что на них повреждение, порок на них: не приобретут они вам благоволения.
Lev 22:26  И сказал Господь Моисею, говоря:
Lev 22:27  когда родится теленок, или ягненок, или козленок, то семь дней он должен пробыть при матери своей, а от восьмого дня и далее будет благоугоден для приношения в жертву Господу;
Lev 22:28  но ни коровы, ни овцы не заколайте в один день с порождением ее.
Lev 22:29  Если приносите Господу жертву благодарения, то приносите ее так, чтоб она приобрела вам благоволение;
Lev 22:30  в тот же день должно съесть ее, не оставляйте от нее до утра. Я Господь.
Lev 22:31  И соблюдайте заповеди Мои и исполняйте их. Я Господь.
Lev 22:32  Не бесчестите святого имени Моего, чтоб Я был святим среди сынов Израилевых. Я Господь, освящающий вас,
Lev 22:33  Который вывел вас из земли Египетской, чтоб быть вашим Богом. Я Господь.
Lev 23:1  И сказал Господь Моисею, говоря:
Lev 23:2  объяви сынам Израилевым и скажи им о праздниках Господних, в которые должно созывать священные собрания. Вот праздники Мои:
Lev 23:3  шесть дней можно делать дела, а в седьмой день суббота покоя, священное собрание; никакого дела не делайте; это суббота Господня во всех жилищах ваших.
Lev 23:4  Вот праздники Господни, священные собрания, которые вы должны созывать в свое время:
Lev 23:5  в первый месяц, в четырнадцатый [день] месяца вечером Пасха Господня;
Lev 23:6  и в пятнадцатый день того же месяца праздник опресноков Господу; семь дней ешьте опресноки;
Lev 23:7  в первый день да будет у вас священное собрание; никакой работы не работайте;
Lev 23:8  и в течение семи дней приносите жертвы Господу; в седьмой день также священное собрание; никакой работы не работайте.
Lev 23:9  И сказал Господь Моисею, говоря:
Lev 23:10  объяви сынам Израилевым и скажи им: когда придете в землю, которую Я даю вам, и будете жать на ней жатву, то принесите первый сноп жатвы вашей к священнику;
Lev 23:11  он вознесет этот сноп пред Господом, чтобы вам приобрести благоволение; на другой день праздника вознесет его священник;
Lev 23:12  и в день возношения снопа принесите во всесожжение Господу агнца однолетнего, без порока,
Lev 23:13  и с ним хлебного приношения две десятых части [ефы] пшеничной муки, смешанной с елеем, в жертву Господу, в приятное благоухание, и возлияния к нему четверть гина вина;
Lev 23:14  никакого [нового] хлеба, ни сушеных зерен, ни зерен сырых не ешьте до того дня, в который принесете приношения Богу вашему: это вечное постановление в роды ваши во всех жилищах ваших.
Lev 23:15  Отсчитайте себе от первого дня после праздника, от того дня, в который приносите сноп потрясания, семь полных недель,
Lev 23:16  до первого дня после седьмой недели отсчитайте пятьдесят дней, [и] [тогда] принесите новое хлебное приношение Господу:
Lev 23:17  от жилищ ваших приносите два хлеба возношения, которые должны состоять из двух десятых частей [ефы] пшеничной муки и должны быть испечены кислые, [как] первый плод Господу;
Lev 23:18  вместе с хлебами представьте семь агнцев без порока, однолетних, и из крупного скота одного тельца и двух овнов; да будет это во всесожжение Господу, и хлебное приношение и возлияние к ним, в жертву, в приятное благоухание Господу.
Lev 23:19  Приготовьте также из [стада] коз одного козла в жертву за грех и двух однолетних агнцев в жертву мирную.
Lev 23:20  священник должен принести это, потрясая пред Господом, вместе с потрясаемыми хлебами первого плода и с двумя агнцами, и это будет святынею Господу; священнику, [который приносит, это принадлежит];
Lev 23:21  и созывайте [народ] в сей день, священное собрание да будет у вас, никакой работы не работайте: это постановление вечное во всех жилищах ваших в роды ваши.
Lev 23:22  Когда будете жать жатву на земле вашей, не дожинай до края поля твоего, когда жнешь, и оставшегося от жатвы твоей не подбирай; бедному и пришельцу оставь это. Я Господь, Бог ваш.
Lev 23:23  И сказал Господь Моисею, говоря:
Lev 23:24  скажи сынам Израилевым: в седьмой месяц, в первый [день] месяца да будет у вас покой, праздник труб, священное собрание.
Lev 23:25  никакой работы не работайте и приносите жертву Господу.
Lev 23:26  И сказал Господь Моисею, говоря:
Lev 23:27  также в девятый [день] седьмого месяца сего, день очищения, да будет у вас священное собрание; смиряйте души ваши и приносите жертву Господу;
Lev 23:28  никакого дела не делайте в день сей, ибо это день очищения, дабы очистить вас пред лицем Господа, Бога вашего;
Lev 23:29  а всякая душа, которая не смирит себя в этот день, истребится из народа своего;
Lev 23:30  и если какая душа будет делать какое-нибудь дело в день сей, Я истреблю ту душу из народа ее;
Lev 23:31  никакого дела не делайте: это постановление вечное в роды ваши, во всех жилищах ваших;
Lev 23:32  это для вас суббота покоя, и смиряйте души ваши, с вечера девятого [дня] месяца; от вечера до вечера празднуйте субботу вашу.
Lev 23:33  И сказал Господь Моисею, говоря:
Lev 23:34  скажи сынам Израилевым: с пятнадцатого дня того же седьмого месяца праздник кущей, семь дней Господу;
Lev 23:35  в первый день священное собрание, никакой работы не работайте;
Lev 23:36  в [течение] семи дней приносите жертву Господу; в восьмой день священное собрание да будет у вас, и приносите жертву Господу: это отдание праздника, никакой работы не работайте.
Lev 23:37  Вот праздники Господни, в которые должно созывать священные собрания, чтобы приносить в жертву Господу всесожжение, хлебное приношение, заколаемые жертвы и возлияния, каждое в свой день,
Lev 23:38  кроме суббот Господних и кроме даров ваших, и кроме всех обетов ваших и кроме всего [приносимого] по усердию вашему, что вы даете Господу.
Lev 23:39  А в пятнадцатый день седьмого месяца, когда вы собираете произведения земли, празднуйте праздник Господень семь дней: в первый день покой и в восьмой день покой;
Lev 23:40  в первый день возьмите себе ветви красивых дерев, ветви пальмовые и ветви дерев широколиственных и верб речных, и веселитесь пред Господом Богом вашим семь дней;
Lev 23:41  и празднуйте этот праздник Господень семь дней в году: это постановление вечное в роды ваши; в седьмой месяц празднуйте его;
Lev 23:42  в кущах живите семь дней; всякий туземец Израильтянин должен жить в кущах,
Lev 23:43  чтобы знали роды ваши, что в кущах поселил Я сынов Израилевых, когда вывел их из земли Египетской. Я Господь, Бог ваш.
Lev 23:44  И объявил Моисей сынам Израилевым о праздниках Господних.
Lev 24:1  И сказал Господь Моисею, говоря:
Lev 24:2  прикажи сынам Израилевым, чтоб они принесли тебе елея чистого, выбитого, для освещения, чтобы непрестанно горел светильник;
Lev 24:3  вне завесы [ковчега] откровения в скинии собрания Аарон [и сыны его] должны ставить оный пред Господом от вечера до утра всегда: это вечное постановление в роды ваши;
Lev 24:4  на подсвечнике чистом должны они ставить светильник пред Господом всегда.
Lev 24:5  И возьми пшеничной муки и испеки из нее двенадцать хлебов; в каждом хлебе должны быть две десятых [ефы];
Lev 24:6  и положи их в два ряда, по шести в ряд, на чистом столе пред Господом;
Lev 24:7  и положи на [каждый] ряд чистого ливана, и будет это при хлебе, в память, в жертву Господу;
Lev 24:8  в каждый день субботы постоянно должно полагать их пред Господом от сынов Израилевых: это завет вечный;
Lev 24:9  они будут принадлежать Аарону и сынам его, которые будут есть их на святом месте, ибо это великая святыня для них из жертв Господних: [это] постановление вечное.
Lev 24:10  И вышел сын одной Израильтянки, родившейся от Египтянина, к сынам Израилевым, и поссорился в стане сын Израильтянки с Израильтянином;
Lev 24:11  хулил сын Израильтянки имя [Господне] и злословил. И привели его к Моисею;
Lev 24:12  и посадили его под стражу, доколе не будет объявлена им воля Господня.
Lev 24:13  И сказал Господь Моисею, говоря:
Lev 24:14  выведи злословившего вон из стана, и все слышавшие пусть положат руки свои на голову его, и все общество побьет его камнями;
Lev 24:15  и сынам Израилевым скажи: кто будет злословить Бога своего, тот понесет грех свой;
Lev 24:16  и хулитель имени Господня должен умереть, камнями побьет его все общество: пришлец ли, туземец ли станет хулить имя [Господне], предан будет смерти.
Lev 24:17  Кто убьет какого-либо человека, тот предан будет смерти.
Lev 24:18  Кто убьет скотину, должен заплатить за нее, скотину за скотину.
Lev 24:19  Кто сделает повреждение на теле ближнего своего, тому должно сделать то же, что он сделал:
Lev 24:20  перелом за перелом, око за око, зуб за зуб; как он сделал повреждение на [теле] человека, так и ему должно сделать.
Lev 24:21  Кто убьет скотину, должен заплатить за нее; а кто убьет человека, того должно предать смерти.
Lev 24:22  Один суд должен быть у вас, как для пришельца, так и для туземца; ибо Я Господь, Бог ваш.
Lev 24:23  И сказал Моисей сынам Израилевым; и вывели злословившего вон из стана, и побили его камнями, и сделали сыны Израилевы, как повелел Господь Моисею.
Lev 25:1  И сказал Господь Моисею на горе Синае, говоря:
Lev 25:2  объяви сынам Израилевым и скажи им: когда придете в землю, которую Я даю вам, тогда земля должна покоиться в субботу Господню;
Lev 25:3  шесть лет засевай поле твое и шесть лет обрезывай виноградник твой, и собирай произведения их,
Lev 25:4  а в седьмой год да будет суббота покоя земли, суббота Господня: поля твоего не засевай и виноградника твоего не обрезывай;
Lev 25:5  что само вырастет на жатве твоей, не сжинай, и гроздов с необрезанных лоз твоих не снимай; да будет это год покоя земли;
Lev 25:6  и будет это в продолжение субботы земли [всем] вам в пищу, тебе и рабу твоему, и рабе твоей, и наемнику твоему, и поселенцу твоему, поселившемуся у тебя;
Lev 25:7  и скоту твоему и зверям, которые на земле твоей, да будут все произведения ее в пищу.
Lev 25:8  И насчитай себе семь субботних лет, семь раз по семи лет, чтоб было у тебя в семи субботних годах сорок девять лет;
Lev 25:9  и воструби трубою в седьмой месяц, в десятый [день] месяца, в день очищения вострубите трубою по всей земле вашей;
Lev 25:10  и освятите пятидесятый год и объявите свободу на земле всем жителям ее: да будет это у вас юбилей; и возвратитесь каждый во владение свое, и каждый возвратитесь в свое племя.
Lev 25:11  Пятидесятый год да будет у вас юбилей: не сейте и не жните, что само вырастет на земле, и не снимайте ягод с необрезанных [лоз] ее,
Lev 25:12  ибо это юбилей: священным да будет он для вас; с поля ешьте произведения ее.
Lev 25:13  В юбилейный год возвратитесь каждый во владение свое.
Lev 25:14  Если будешь продавать что ближнему твоему, или будешь покупать что у ближнего твоего, не обижайте друг друга;
Lev 25:15  по расчислению лет после юбилея ты должен покупать у ближнего твоего, и по расчислению лет дохода он должен продавать тебе;
Lev 25:16  если много [остается] лет, умножь цену; а если мало лет [остается], уменьши цену, ибо известное число [лет] жатв он продает тебе.
Lev 25:17  Не обижайте один другого; бойся Бога твоего, ибо Я Господь, Бог ваш.
Lev 25:18  Исполняйте постановления Мои, и храните законы Мои и исполняйте их, и будете жить спокойно на земле;
Lev 25:19  и будет земля давать плод свой, и будете есть досыта, и будете жить спокойно на ней.
Lev 25:20  Если скажете: что же нам есть в седьмой год, когда мы не будем ни сеять, ни собирать произведений наших?
Lev 25:21  Я пошлю благословение Мое на вас в шестой год, и он принесет произведений на три года;
Lev 25:22  и будете сеять в восьмой год, но есть будете произведения старые до девятого года; доколе не поспеют произведения его, будете есть старое.
Lev 25:23  Землю не должно продавать навсегда, ибо Моя земля: вы пришельцы и поселенцы у Меня;
Lev 25:24  по всей земле владения вашего дозволяйте выкуп земли.
Lev 25:25  Если брат твой обеднеет и продаст от владения своего, то придет близкий его родственник и выкупит проданное братом его;
Lev 25:26  если же некому за него выкупить, но сам он будет иметь достаток и найдет, сколько нужно на выкуп,
Lev 25:27  то пусть он расчислит годы продажи своей и возвратит остальное тому, кому он продал, и вступит опять во владение свое;
Lev 25:28  если же не найдет рука его, сколько нужно возвратить ему, то проданное им останется в руках покупщика до юбилейного года, а в юбилейный год отойдет оно, и он опять вступит во владение свое.
Lev 25:29  Если кто продаст жилой дом в городе, [огражденном] стеною, то выкупить его можно до истечения года от продажи его: в течение года выкупить его можно;
Lev 25:30  если же не будет он выкуплен до истечения целого года, то дом, который в городе, имеющем стену, останется навсегда у купившего его в роды его, и в юбилей не отойдет [от него].
Lev 25:31  А домы в селениях, вокруг которых нет стены, должно считать наравне с полем земли: выкупать их можно, и в юбилей они отходят.
Lev 25:32  А города левитов, домы в городах владения их, левитам всегда можно выкупать;
Lev 25:33  а кто из левитов не выкупит, то проданный дом в городе владения их в юбилей отойдет, потому что домы в городах левитских составляют их владение среди сынов Израилевых;
Lev 25:34  и полей вокруг городов их продавать нельзя, потому что это вечное владение их.
Lev 25:35  Если брат твой обеднеет и придет в упадок у тебя, то поддержи его, пришлец ли он, или поселенец, чтоб он жил с тобою;
Lev 25:36  не бери от него роста и прибыли и бойся Бога твоего; чтоб жил брат твой с тобою;
Lev 25:37  серебра твоего не отдавай ему в рост и хлеба твоего не отдавай ему для [получения] прибыли.
Lev 25:38  Я Господь, Бог ваш, Который вывел вас из земли Египетской, чтобы дать вам землю Ханаанскую, чтоб быть вашим Богом.
Lev 25:39  Когда обеднеет у тебя брат твой и продан будет тебе, то не налагай на него работы рабской:
Lev 25:40  он должен быть у тебя как наемник, как поселенец; до юбилейного года пусть работает у тебя,
Lev 25:41  а [тогда] пусть отойдет он от тебя, сам и дети его с ним, и возвратится в племя свое, и вступит опять во владение отцов своих,
Lev 25:42  потому что они--Мои рабы, которых Я вывел из земли Египетской: не должно продавать их, как продают рабов;
Lev 25:43  не господствуй над ним с жестокостью и бойся Бога твоего.
Lev 25:44  А чтобы раб твой и рабыня твоя были у тебя, то покупайте себе раба и рабыню у народов, которые вокруг вас;
Lev 25:45  также и из детей поселенцев, поселившихся у вас, можете покупать, и из племени их, которое у вас, которое у них родилось в земле вашей, и они могут быть вашей собственностью;
Lev 25:46  можете передавать их в наследство и сынам вашим по себе, как имение; вечно владейте ими, как рабами. А над братьями вашими, сынами Израилевыми, друг над другом, не господствуйте с жестокостью.
Lev 25:47  Если пришлец или поселенец твой будет иметь достаток, а брат твой пред ним обеднеет и продастся пришельцу, поселившемуся у тебя, или кому-нибудь из племени пришельца,
Lev 25:48  то после продажи можно выкупить его; кто-нибудь из братьев его должен выкупить его,
Lev 25:49  или дядя его, или сын дяди его должен выкупить его, или кто-- нибудь из родства его, из племени его, должен выкупить его; или если будет иметь достаток, сам выкупится.
Lev 25:50  И он должен рассчитаться с купившим его, [начиная] от того года, когда он продал себя, до года юбилейного, и серебро, за которое он продал себя, должно отдать ему по числу лет; как временный наемник он должен быть у него;
Lev 25:51  и если еще много [остается] лет, то по мере их он должен отдать в выкуп за себя серебро, за которое он куплен;
Lev 25:52  если же мало остается лет до юбилейного года, то он должен сосчитать и по мере лет отдать за себя выкуп.
Lev 25:53  Он должен быть у него, как наемник, во все годы; он не должен господствовать над ним с жестокостью в глазах твоих.
Lev 25:54  Если же он не выкупится таким образом, то в юбилейный год отойдет сам и дети его с ним,
Lev 25:55  потому что сыны Израилевы Мои рабы; они Мои рабы, которых Я вывел из земли Египетской. Я Господь, Бог ваш.
Lev 26:1  Не делайте себе кумиров и изваяний, и столбов не ставьте у себя, и камней с изображениями не кладите в земле вашей, чтобы кланяться пред ними, ибо Я Господь Бог ваш.
Lev 26:2  Субботы Мои соблюдайте и святилище Мое чтите: Я Господь.
Lev 26:3  Если вы будете поступать по уставам Моим и заповеди Мои будете хранить и исполнять их,
Lev 26:4  то Я дам вам дожди в свое время, и земля даст произрастения свои, и дерева полевые дадут плод свой;
Lev 26:5  и молотьба [хлеба] будет достигать у вас собирания винограда, собирание винограда будет достигать посева, и будете есть хлеб свой досыта, и будете жить на земле [вашей] безопасно;
Lev 26:6  пошлю мир на землю [вашу], ляжете, и никто вас не обеспокоит, сгоню лютых зверей с земли [вашей], и меч не пройдет по земле вашей;
Lev 26:7  и будете прогонять врагов ваших, и падут они пред вами от меча;
Lev 26:8  пятеро из вас прогонят сто, и сто из вас прогонят тьму, и падут враги ваши пред вами от меча;
Lev 26:9  призрю на вас, и плодородными сделаю вас, и размножу вас, и буду тверд в завете Моем с вами;
Lev 26:10  и будете есть старое прошлогоднее, и выбросите старое ради нового;
Lev 26:11  и поставлю жилище Мое среди вас, и душа Моя не возгнушается вами;
Lev 26:12  и буду ходить среди вас и буду вашим Богом, а вы будете Моим народом.
Lev 26:13  Я Господь Бог ваш, Который вывел вас из земли Египетской, чтоб вы не были там рабами, и сокрушил узы ярма вашего, и повел вас с поднятою головою.
Lev 26:14  Если же не послушаете Меня и не будете исполнять всех заповедей сих,
Lev 26:15  и если презрите Мои постановления, и если душа ваша возгнушается Моими законами, так что вы не будете исполнять всех заповедей Моих, нарушив завет Мой, --
Lev 26:16  то и Я поступлю с вами так: пошлю на вас ужас, чахлость и горячку, от которых истомятся глаза и измучится душа, и будете сеять семена ваши напрасно, и враги ваши съедят их;
Lev 26:17  обращу лице Мое на вас, и падете пред врагами вашими, и будут господствовать над вами неприятели ваши, и побежите, когда никто не гонится за вами.
Lev 26:18  Если и при всем том не послушаете Меня, то Я всемеро увеличу наказание за грехи ваши,
Lev 26:19  и сломлю гордое упорство ваше, и небо ваше сделаю, как железо, и землю вашу, как медь;
Lev 26:20  и напрасно будет истощаться сила ваша, и земля ваша не даст произрастений своих, и дерева земли не дадут плодов своих.
Lev 26:21  Если же [после сего] пойдете против Меня и не захотите слушать Меня, то Я прибавлю вам ударов всемеро за грехи ваши:
Lev 26:22  пошлю на вас зверей полевых, которые лишат вас детей, истребят скот ваш и вас уменьшат, так что опустеют дороги ваши.
Lev 26:23  Если и после сего не исправитесь и пойдете против Меня,
Lev 26:24  то и Я пойду против вас и поражу вас всемеро за грехи ваши,
Lev 26:25  и наведу на вас мстительный меч в отмщение за завет; если же вы укроетесь в города ваши, то пошлю на вас язву, и преданы будете в руки врага;
Lev 26:26  хлеб, подкрепляющий [человека], истреблю у вас; десять женщин будут печь хлеб ваш в одной печи и будут отдавать хлеб ваш весом; вы будете есть и не будете сыты.
Lev 26:27  Если же и после сего не послушаете Меня и пойдете против Меня,
Lev 26:28  то и Я в ярости пойду против вас и накажу вас всемеро за грехи ваши,
Lev 26:29  и будете есть плоть сынов ваших, и плоть дочерей ваших будете есть;
Lev 26:30  разорю высоты ваши и разрушу столбы ваши, и повергну трупы ваши на обломки идолов ваших, и возгнушается душа Моя вами;
Lev 26:31  города ваши сделаю пустынею, и опустошу святилища ваши, и не буду обонять приятного благоухания [жертв] ваших;
Lev 26:32  опустошу землю [вашу], так что изумятся о ней враги ваши, поселившиеся на ней;
Lev 26:33  а вас рассею между народами и обнажу вслед вас меч, и будет земля ваша пуста и города ваши разрушены.
Lev 26:34  Тогда удовлетворит себя земля за субботы свои во все дни запустения [своего]; когда вы будете в земле врагов ваших, тогда будет покоиться земля и удовлетворит себя за субботы свои;
Lev 26:35  во все дни запустения [своего] будет она покоиться, сколько ни покоилась в субботы ваши, когда вы жили на ней.
Lev 26:36  Оставшимся из вас пошлю в сердца робость в земле врагов их, и шум колеблющегося листа погонит их, и побегут, как от меча, и падут, когда никто не преследует,
Lev 26:37  и споткнутся друг на друга, как от меча, между тем как никто не преследует, и не будет у вас силы противостоять врагам вашим;
Lev 26:38  и погибнете между народами, и пожрет вас земля врагов ваших;
Lev 26:39  а оставшиеся из вас исчахнут за свои беззакония в землях врагов ваших и за беззакония отцов своих исчахнут;
Lev 26:40  тогда признаются они в беззаконии своем и в беззаконии отцов своих, как они совершали преступления против Меня и шли против Меня,
Lev 26:41  [за что] и Я шел против них и ввел их в землю врагов их; тогда покорится необрезанное сердце их, и тогда потерпят они за беззакония свои.
Lev 26:42  И Я вспомню завет Мой с Иаковом и завет Мой с Исааком, и завет Мой с Авраамом вспомню, и землю вспомню;
Lev 26:43  тогда как земля оставлена будет ими и будет удовлетворять себя за субботы свои, опустев от них, и они будут терпеть за свое беззаконие, за то, что презирали законы Мои и душа их гнушалась постановлениями Моими,
Lev 26:44  и тогда как они будут в земле врагов их, --Я не презрю их и не возгнушаюсь ими до того, чтоб истребить их, чтоб разрушить завет Мой с ними, ибо Я Господь, Бог их;
Lev 26:45  вспомню для них завет с предками, которых вывел Я из земли Египетской пред глазами народов, чтоб быть их Богом. Я Господь.
Lev 26:46  Вот постановления и определения и законы, которые постановил Господь между Собою и между сынами Израилевыми на горе Синае, чрез Моисея.
Lev 27:1  И сказал Господь Моисею, говоря:
Lev 27:2  объяви сынам Израилевым и скажи им: если кто дает обет посвятить душу Господу по оценке твоей,
Lev 27:3  то оценка твоя мужчине от двадцати лет до шестидесяти должна быть пятьдесят сиклей серебряных, по сиклю священному;
Lev 27:4  если же это женщина, то оценка твоя должна быть тридцать сиклей;
Lev 27:5  от пяти лет до двадцати оценка твоя мужчине должна быть двадцать сиклей, а женщине десять сиклей;
Lev 27:6  а от месяца до пяти лет оценка твоя мужчине должна быть пять сиклей серебра, а женщине оценка твоя три сикля серебра;
Lev 27:7  от шестидесяти лет и выше мужчине оценка твоя должна быть пятнадцать сиклей серебра, а женщине десять сиклей.
Lev 27:8  Если же он беден и не в силах [отдать] по оценке твоей, то пусть представят его священнику, и священник пусть оценит его: соразмерно с состоянием давшего обет пусть оценит его священник.
Lev 27:9  Если же то будет скот, который приносят в жертву Господу, то все, что дано Господу, должно быть свято:
Lev 27:10  не должно выменивать его и заменять хорошее худым, или худое хорошим; если же станет кто заменять скотину скотиною, то и она и замен ее будет святынею.
Lev 27:11  Если же то будет какая-нибудь скотина нечистая, которую не приносят в жертву Господу, то должно представить скотину священнику,
Lev 27:12  и священник оценит ее, хороша ли она, или худа, и как оценит священник, так и должно быть;
Lev 27:13  если же кто хочет выкупить ее, то пусть прибавит пятую долю к оценке твоей.
Lev 27:14  Если кто посвящает дом свой в святыню Господу, то священник должен оценить его, хорош ли он, или худ, и как оценит его священник, так и состоится;
Lev 27:15  если же посвятивший захочет выкупить дом свой, то пусть прибавит пятую часть серебра оценки твоей, и [тогда] будет его.
Lev 27:16  Если поле из своего владения посвятит кто Господу, то оценка твоя должна быть по мере посева: за посев хомера ячменя пятьдесят сиклей серебра;
Lev 27:17  если от юбилейного года посвящает кто поле свое, --должно состояться по оценке твоей;
Lev 27:18  если же после юбилея посвящает кто поле свое, то священник должен рассчитать серебро по мере лет, оставшихся до юбилейного года, и должно убавить из оценки твоей;
Lev 27:19  если же захочет выкупить поле посвятивший его, то пусть он прибавит пятую часть серебра оценки твоей, и оно останется за ним;
Lev 27:20  если же он не выкупит поля, и будет продано поле другому человеку, то уже нельзя выкупить:
Lev 27:21  поле то, когда оно в юбилей отойдет, будет святынею Господу, как бы поле заклятое; священнику достанется оно во владение.
Lev 27:22  А если кто посвятит Господу поле купленное, которое не из полей его владения,
Lev 27:23  то священник должен рассчитать ему количество оценки до юбилейного года, и должен он отдать по расчету в тот же день, [как] святыню Господню;
Lev 27:24  поле же в юбилейный год перейдет опять к тому, у кого куплено, кому принадлежит владение той земли.
Lev 27:25  Всякая оценка твоя должна быть по сиклю священному, двадцать гер должно быть в сикле.
Lev 27:26  Только первенцев из скота, которые по первенству принадлежат Господу, не должен никто посвящать: вол ли то, или мелкий скот, --Господни они.
Lev 27:27  Если же скот нечистый, то должно выкупить по оценке твоей и приложить к тому пятую часть; если не выкупят, то должно продать по оценке твоей.
Lev 27:28  Только все заклятое, что под заклятием отдает человек Господу из своей собственности, --человека ли, скотину ли, поле ли своего владения, --не продается и не выкупается: все заклятое есть великая святыня Господня;
Lev 27:29  все заклятое, что заклято от людей, не выкупается: оно должно быть предано смерти.
Lev 27:30  И всякая десятина на земле из семян земли и из плодов дерева принадлежит Господу: это святыня Господня;
Lev 27:31  если же кто захочет выкупить десятину свою, то пусть приложит к [цене] ее пятую долю.
Lev 27:32  И всякую десятину из крупного и мелкого скота, из всего, что проходит под жезлом десятое, должно посвящать Господу;
Lev 27:33  не должно разбирать, хорошее ли то, или худое, и не должно заменять его; если же кто заменит его, то и само оно и замен его будет святынею и не может быть выкуплено.
Lev 27:34  Вот заповеди, которые заповедал Господь Моисею для сынов Израилевых на горе Синае.


\end{document}