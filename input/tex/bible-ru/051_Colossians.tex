\begin{document}

\title{Послание к Колоссянам}


\chapter{1}

\par 1 Павел, волею Божиею Апостол Иисуса Христа, и Тимофей брат,
\par 2 находящимся в Колоссах святым и верным братиям во Христе Иисусе:
\par 3 благодать вам и мир от Бога Отца нашего и Господа Иисуса Христа. Благодарим Бога и Отца Господа нашего Иисуса Христа, всегда молясь о вас,
\par 4 услышав о вере вашей во Христа Иисуса и о любви ко всем святым,
\par 5 в надежде на уготованное вам на небесах, о чем вы прежде слышали в истинном слове благовествования,
\par 6 которое пребывает у вас, как и во всем мире, и приносит плод, и возрастает, как и между вами, с того дня, как вы услышали и познали благодать Божию в истине,
\par 7 как и научились от Епафраса, возлюбленного сотрудника нашего, верного для вас служителя Христова,
\par 8 который и известил нас о вашей любви в духе.
\par 9 Посему и мы с того дня, как [о сем] услышали, не перестаем молиться о вас и просить, чтобы вы исполнялись познанием воли Его, во всякой премудрости и разумении духовном,
\par 10 чтобы поступали достойно Бога, во всем угождая [Ему], принося плод во всяком деле благом и возрастая в познании Бога,
\par 11 укрепляясь всякою силою по могуществу славы Его, во всяком терпении и великодушии с радостью,
\par 12 благодаря Бога и Отца, призвавшего нас к участию в наследии святых во свете,
\par 13 избавившего нас от власти тьмы и введшего в Царство возлюбленного Сына Своего,
\par 14 в Котором мы имеем искупление Кровию Его и прощение грехов,
\par 15 Который есть образ Бога невидимого, рожденный прежде всякой твари;
\par 16 ибо Им создано все, что на небесах и что на земле, видимое и невидимое: престолы ли, господства ли, начальства ли, власти ли, --все Им и для Него создано;
\par 17 и Он есть прежде всего, и все Им стоит.
\par 18 И Он есть глава тела Церкви; Он--начаток, первенец из мертвых, дабы иметь Ему во всем первенство,
\par 19 ибо благоугодно было [Отцу], чтобы в Нем обитала всякая полнота,
\par 20 и чтобы посредством Его примирить с Собою все, умиротворив через Него, Кровию креста Его, и земное и небесное.
\par 21 И вас, бывших некогда отчужденными и врагами, по расположению к злым делам,
\par 22 ныне примирил в теле Плоти Его, смертью Его, [чтобы] представить вас святыми и непорочными и неповинными пред Собою,
\par 23 если только пребываете тверды и непоколебимы в вере и не отпадаете от надежды благовествования, которое вы слышали, которое возвещено всей твари поднебесной, которого я, Павел, сделался служителем.
\par 24 Ныне радуюсь в страданиях моих за вас и восполняю недостаток в плоти моей скорбей Христовых за Тело Его, которое есть Церковь,
\par 25 которой сделался я служителем по домостроительству Божию, вверенному мне для вас, [чтобы] исполнить слово Божие,
\par 26 тайну, сокрытую от веков и родов, ныне же открытую святым Его,
\par 27 Которым благоволил Бог показать, какое богатство славы в тайне сей для язычников, которая есть Христос в вас, упование славы,
\par 28 Которого мы проповедуем, вразумляя всякого человека и научая всякой премудрости, чтобы представить всякого человека совершенным во Христе Иисусе;
\par 29 для чего я и тружусь и подвизаюсь силою Его, действующею во мне могущественно.

\chapter{2}

\par 1 Желаю, чтобы вы знали, какой подвиг имею я ради вас и ради тех, которые в Лаодикии и Иераполе, и ради всех, кто не видел лица моего в плоти,
\par 2 дабы утешились сердца их, соединенные в любви для всякого богатства совершенного разумения, для познания тайны Бога и Отца и Христа,
\par 3 в Котором сокрыты все сокровища премудрости и ведения.
\par 4 Это говорю я для того, чтобы кто-нибудь не прельстил вас вкрадчивыми словами;
\par 5 ибо хотя я и отсутствую телом, но духом нахожусь с вами, радуясь и видя ваше благоустройство и твердость веры вашей во Христа.
\par 6 Посему, как вы приняли Христа Иисуса Господа, [так] и ходите в Нем,
\par 7 будучи укоренены и утверждены в Нем и укреплены в вере, как вы научены, преуспевая в ней с благодарением.
\par 8 Смотрите, братия, чтобы кто не увлек вас философиею и пустым обольщением, по преданию человеческому, по стихиям мира, а не по Христу;
\par 9 ибо в Нем обитает вся полнота Божества телесно,
\par 10 и вы имеете полноту в Нем, Который есть глава всякого начальства и власти.
\par 11 В Нем вы и обрезаны обрезанием нерукотворенным, совлечением греховного тела плоти, обрезанием Христовым;
\par 12 быв погребены с Ним в крещении, в Нем вы и совоскресли верою в силу Бога, Который воскресил Его из мертвых,
\par 13 и вас, которые были мертвы во грехах и в необрезании плоти вашей, оживил вместе с Ним, простив нам все грехи,
\par 14 истребив учением бывшее о нас рукописание, которое было против нас, и Он взял его от среды и пригвоздил ко кресту;
\par 15 отняв силы у начальств и властей, властно подверг их позору, восторжествовав над ними Собою.
\par 16 Итак никто да не осуждает вас за пищу, или питие, или за какой-нибудь праздник, или новомесячие, или субботу:
\par 17 это есть тень будущего, а тело--во Христе.
\par 18 Никто да не обольщает вас самовольным смиренномудрием и служением Ангелов, вторгаясь в то, чего не видел, безрассудно надмеваясь плотским своим умом
\par 19 и не держась главы, от которой все тело, составами и связями будучи соединяемо и скрепляемо, растет возрастом Божиим.
\par 20 Итак, если вы со Христом умерли для стихий мира, то для чего вы, как живущие в мире, держитесь постановлений:
\par 21 `не прикасайся', `не вкушай', `не дотрагивайся' --
\par 22 что все истлевает от употребления, --по заповедям и учению человеческому?
\par 23 Это имеет только вид мудрости в самовольном служении, смиренномудрии и изнурении тела, в некотором небрежении о насыщении плоти.

\chapter{3}

\par 1 Итак, если вы воскресли со Христом, то ищите горнего, где Христос сидит одесную Бога;
\par 2 о горнем помышляйте, а не о земном.
\par 3 Ибо вы умерли, и жизнь ваша сокрыта со Христом в Боге.
\par 4 Когда же явится Христос, жизнь ваша, тогда и вы явитесь с Ним во славе.
\par 5 Итак, умертвите земные члены ваши: блуд, нечистоту, страсть, злую похоть и любостяжание, которое есть идолослужение,
\par 6 за которые гнев Божий грядет на сынов противления,
\par 7 в которых и вы некогда обращались, когда жили между ними.
\par 8 А теперь вы отложите все: гнев, ярость, злобу, злоречие, сквернословие уст ваших;
\par 9 не говорите лжи друг другу, совлекшись ветхого человека с делами его
\par 10 и облекшись в нового, который обновляется в познании по образу Создавшего его,
\par 11 где нет ни Еллина, ни Иудея, ни обрезания, ни необрезания, варвара, Скифа, раба, свободного, но все и во всем Христос.
\par 12 Итак облекитесь, как избранные Божии, святые и возлюбленные, в милосердие, благость, смиренномудрие, кротость, долготерпение,
\par 13 снисходя друг другу и прощая взаимно, если кто на кого имеет жалобу: как Христос простил вас, так и вы.
\par 14 Более же всего [облекитесь] в любовь, которая есть совокупность совершенства.
\par 15 И да владычествует в сердцах ваших мир Божий, к которому вы и призваны в одном теле, и будьте дружелюбны.
\par 16 Слово Христово да вселяется в вас обильно, со всякою премудростью; научайте и вразумляйте друг друга псалмами, славословием и духовными песнями, во благодати воспевая в сердцах ваших Господу.
\par 17 И все, что вы делаете, словом или делом, все [делайте] во имя Господа Иисуса Христа, благодаря через Него Бога и Отца.
\par 18 Жены, повинуйтесь мужьям своим, как прилично в Господе.
\par 19 Мужья, любите своих жен и не будьте к ним суровы.
\par 20 Дети, будьте послушны родителям вашим во всем, ибо это благоугодно Господу.
\par 21 Отцы, не раздражайте детей ваших, дабы они не унывали.
\par 22 Рабы, во всем повинуйтесь господам вашим по плоти, не в глазах только служа [им], как человекоугодники, но в простоте сердца, боясь Бога.
\par 23 И все, что делаете, делайте от души, как для Господа, а не для человеков,
\par 24 зная, что в воздаяние от Господа получите наследие, ибо вы служите Господу Христу.
\par 25 А кто неправо поступит, тот получит по своей неправде, [у Него] нет лицеприятия.

\chapter{4}

\par 1 Господа, оказывайте рабам должное и справедливое, зная, что и вы имеете Господа на небесах.
\par 2 Будьте постоянны в молитве, бодрствуя в ней с благодарением.
\par 3 Молитесь также и о нас, чтобы Бог отверз нам дверь для слова, возвещать тайну Христову, за которую я и в узах,
\par 4 дабы я открыл ее, как должно мне возвещать.
\par 5 Со внешними обходитесь благоразумно, пользуясь временем.
\par 6 Слово ваше [да будет] всегда с благодатию, приправлено солью, дабы вы знали, как отвечать каждому.
\par 7 О мне все скажет вам Тихик, возлюбленный брат и верный служитель и сотрудник в Господе,
\par 8 которого я для того послал к вам, чтобы он узнал о ваших [обстоятельствах] и утешил сердца ваши,
\par 9 с Онисимом, верным и возлюбленным братом нашим, который от вас. Они расскажут вам о всем здешнем.
\par 10 Приветствует вас Аристарх, заключенный вместе со мною, и Марк, племянник Варнавы--о котором вы получили приказания: если придет к вам, примите его, --
\par 11 также Иисус, прозываемый Иустом, оба из обрезанных. Они--единственные сотрудники для Царствия Божия, бывшие мне отрадою.
\par 12 Приветствует вас Епафрас ваш, раб Иисуса Христа, всегда подвизающийся за вас в молитвах, чтобы вы пребыли совершенны и исполнены всем, что угодно Богу.
\par 13 Свидетельствую о нем, что он имеет великую ревность и заботу о вас и о находящихся в Лаодикии и Иераполе.
\par 14 Приветствует вас Лука, врач возлюбленный, и Димас.
\par 15 Приветствуйте братьев в Лаодикии, и Нимфана, и домашнюю церковь его.
\par 16 Когда это послание прочитано будет у вас, то распорядитесь, чтобы оно было прочитано и в Лаодикийской церкви; а то, которое из Лаодикии, прочитайте и вы.
\par 17 Скажите Архиппу: смотри, чтобы тебе исполнить служение, которое ты принял в Господе.
\par 18 Приветствие моею рукою, Павловою. Помните мои узы. Благодать со всеми вами. Аминь.


\end{document}