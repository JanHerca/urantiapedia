\begin{document}

\title{2 Corinthians}

2Co 1:1  Павел, волею Божиею Апостол Иисуса Христа, и Тимофей брат, церкви Божией, находящейся в Коринфе, со всеми святыми по всей Ахаии:
2Co 1:2  благодать вам и мир от Бога Отца нашего и Господа Иисуса Христа.
2Co 1:3  Благословен Бог и Отец Господа нашего Иисуса Христа, Отец милосердия и Бог всякого утешения,
2Co 1:4  утешающий нас во всякой скорби нашей, чтобы и мы могли утешать находящихся во всякой скорби тем утешением, которым Бог утешает нас самих!
2Co 1:5  Ибо по мере, как умножаются в нас страдания Христовы, умножается Христом и утешение наше.
2Co 1:6  Скорбим ли мы, [скорбим] для вашего утешения и спасения, которое совершается перенесением тех же страданий, какие и мы терпим.
2Co 1:7  И надежда наша о вас тверда. Утешаемся ли, [утешаемся] для вашего утешения и спасения, зная, что вы участвуете как в страданиях наших, так и в утешении.
2Co 1:8  Ибо мы не хотим оставить вас, братия, в неведении о скорби нашей, бывшей с нами в Асии, потому что мы отягчены были чрезмерно и сверх силы, так что не надеялись остаться в живых.
2Co 1:9  Но сами в себе имели приговор к смерти, для того, чтобы надеяться не на самих себя, но на Бога, воскрешающего мертвых,
2Co 1:10  Который и избавил нас от столь [близкой] смерти, и избавляет, и на Которого надеемся, что и еще избавит,
2Co 1:11  при содействии и вашей молитвы за нас, дабы за дарованное нам, по ходатайству многих, многие возблагодарили за нас.
2Co 1:12  Ибо похвала наша сия есть свидетельство совести нашей, что мы в простоте и богоугодной искренности, не по плотской мудрости, но по благодати Божией, жили в мире, особенно же у вас.
2Co 1:13  И мы пишем вам не иное, как то, что вы читаете или разумеете, и что, как надеюсь, до конца уразумеете,
2Co 1:14  так как вы отчасти и уразумели уже, что мы будем вашею похвалою, равно и вы нашею, в день Господа нашего Иисуса Христа.
2Co 1:15  И в этой уверенности я намеревался придти к вам ранее, чтобы вы вторично получили благодать,
2Co 1:16  и через вас пройти в Македонию, из Македонии же опять придти к вам; а вы проводили бы меня в Иудею.
2Co 1:17  Имея такое намерение, легкомысленно ли я поступил? Или, что я предпринимаю, по плоти предпринимаю, так что у меня то `да, да', то `нет, нет'?
2Co 1:18  Верен Бог, что слово наше к вам не было то `да', то `нет'.
2Co 1:19  Ибо Сын Божий, Иисус Христос, проповеданный у вас нами, мною и Силуаном и Тимофеем, не был `да' и `нет'; но в Нем было `да', --
2Co 1:20  ибо все обетования Божии в Нем `да' и в Нем `аминь', --в славу Божию, через нас.
2Co 1:21  Утверждающий же нас с вами во Христе и помазавший нас [есть] Бог,
2Co 1:22  Который и запечатлел нас и дал залог Духа в сердца наши.
2Co 1:23  Бога призываю во свидетели на душу мою, что, щадя вас, я доселе не приходил в Коринф,
2Co 1:24  не потому, будто мы берем власть над верою вашею; но мы споспешествуем радости вашей: ибо верою вы тверды.
2Co 2:1  Итак я рассудил сам в себе не приходить к вам опять с огорчением.
2Co 2:2  Ибо если я огорчаю вас, то кто обрадует меня, как не тот, кто огорчен мною?
2Co 2:3  Это самое и писал я вам, дабы, придя, не иметь огорчения от тех, о которых мне надлежало радоваться: ибо я во всех вас уверен, что моя радость есть [радость] и для всех вас.
2Co 2:4  От великой скорби и стесненного сердца я писал вам со многими слезами, не для того, чтобы огорчить вас, но чтобы вы познали любовь, какую я в избытке имею к вам.
2Co 2:5  Если же кто огорчил, то не меня огорчил, но частью, --чтобы не сказать много, --и всех вас.
2Co 2:6  Для такого довольно сего наказания от многих,
2Co 2:7  так что вам лучше уже простить его и утешить, дабы он не был поглощен чрезмерною печалью.
2Co 2:8  И потому прошу вас оказать ему любовь.
2Co 2:9  Ибо я для того и писал, чтобы узнать на опыте, во всем ли вы послушны.
2Co 2:10  А кого вы в чем прощаете, того и я; ибо и я, если в чем простил кого, простил для вас от лица Христова,
2Co 2:11  чтобы не сделал нам ущерба сатана, ибо нам не безызвестны его умыслы.
2Co 2:12  Придя в Троаду для благовествования о Христе, хотя мне и отверста была дверь Господом,
2Co 2:13  я не имел покоя духу моему, потому что не нашел [там] брата моего Тита; но, простившись с ними, я пошел в Македонию.
2Co 2:14  Но благодарение Богу, Который всегда дает нам торжествовать во Христе и благоухание познания о Себе распространяет нами во всяком месте.
2Co 2:15  Ибо мы Христово благоухание Богу в спасаемых и в погибающих:
2Co 2:16  для одних запах смертоносный на смерть, а для других запах живительный на жизнь. И кто способен к сему?
2Co 2:17  Ибо мы не повреждаем слова Божия, как многие, но проповедуем искренно, как от Бога, пред Богом, во Христе.
2Co 3:1  Неужели нам снова знакомиться с вами? Неужели нужны для нас, как для некоторых, одобрительные письма к вам или от вас?
2Co 3:2  Вы--наше письмо, написанное в сердцах наших, узнаваемое и читаемое всеми человеками;
2Co 3:3  вы показываете собою, что вы--письмо Христово, через служение наше написанное не чернилами, но Духом Бога живаго, не на скрижалях каменных, но на плотяных скрижалях сердца.
2Co 3:4  Такую уверенность мы имеем в Боге через Христа,
2Co 3:5  не потому, чтобы мы сами способны были помыслить что от себя, как бы от себя, но способность наша от Бога.
2Co 3:6  Он дал нам способность быть служителями Нового Завета, не буквы, но духа, потому что буква убивает, а дух животворит.
2Co 3:7  Если же служение смертоносным буквам, начертанное на камнях, было так славно, что сыны Израилевы не могли смотреть на лице Моисеево по причине славы лица его преходящей, --
2Co 3:8  то не гораздо ли более должно быть славно служение духа?
2Co 3:9  Ибо если служение осуждения славно, то тем паче изобилует славою служение оправдания.
2Co 3:10  То прославленное даже не оказывается славным с сей стороны, по причине преимущественной славы [последующего].
2Co 3:11  Ибо, если преходящее славно, тем более славно пребывающее.
2Co 3:12  Имея такую надежду, мы действуем с великим дерзновением,
2Co 3:13  а не так, как Моисей, [который] полагал покрывало на лице свое, чтобы сыны Израилевы не взирали на конец преходящего.
2Co 3:14  Но умы их ослеплены: ибо то же самое покрывало доныне остается неснятым при чтении Ветхого Завета, потому что оно снимается Христом.
2Co 3:15  Доныне, когда они читают Моисея, покрывало лежит на сердце их;
2Co 3:16  но когда обращаются к Господу, тогда это покрывало снимается.
2Co 3:17  Господь есть Дух; а где Дух Господень, там свобода.
2Co 3:18  Мы же все открытым лицем, как в зеркале, взирая на славу Господню, преображаемся в тот же образ от славы в славу, как от Господня Духа.
2Co 4:1  Посему, имея по милости [Божией] такое служение, мы не унываем;
2Co 4:2  но, отвергнув скрытные постыдные [дела], не прибегая к хитрости и не искажая слова Божия, а открывая истину, представляем себя совести всякого человека пред Богом.
2Co 4:3  Если же и закрыто благовествование наше, то закрыто для погибающих,
2Co 4:4  для неверующих, у которых бог века сего ослепил умы, чтобы для них не воссиял свет благовествования о славе Христа, Который есть образ Бога невидимого.
2Co 4:5  Ибо мы не себя проповедуем, но Христа Иисуса, Господа; а мы--рабы ваши для Иисуса,
2Co 4:6  потому что Бог, повелевший из тьмы воссиять свету, озарил наши сердца, дабы просветить [нас] познанием славы Божией в лице Иисуса Христа.
2Co 4:7  Но сокровище сие мы носим в глиняных сосудах, чтобы преизбыточная сила была [приписываема] Богу, а не нам.
2Co 4:8  Мы отовсюду притесняемы, но не стеснены; мы в отчаянных обстоятельствах, но не отчаиваемся;
2Co 4:9  мы гонимы, но не оставлены; низлагаемы, но не погибаем.
2Co 4:10  Всегда носим в теле мертвость Господа Иисуса, чтобы и жизнь Иисусова открылась в теле нашем.
2Co 4:11  Ибо мы живые непрестанно предаемся на смерть ради Иисуса, чтобы и жизнь Иисусова открылась в смертной плоти нашей,
2Co 4:12  так что смерть действует в нас, а жизнь в вас.
2Co 4:13  Но, имея тот же дух веры, как написано: я веровал и потому говорил, и мы веруем, потому и говорим,
2Co 4:14  зная, что Воскресивший Господа Иисуса воскресит через Иисуса и нас и поставит перед [Собою] с вами.
2Co 4:15  Ибо все для вас, дабы обилие благодати тем большую во многих произвело благодарность во славу Божию.
2Co 4:16  Посему мы не унываем; но если внешний наш человек и тлеет, то внутренний со дня на день обновляется.
2Co 4:17  Ибо кратковременное легкое страдание наше производит в безмерном преизбытке вечную славу,
2Co 4:18  когда мы смотрим не на видимое, но на невидимое: ибо видимое временно, а невидимое вечно.
2Co 5:1  Ибо знаем, что, когда земной наш дом, эта хижина, разрушится, мы имеем от Бога жилище на небесах, дом нерукотворенный, вечный.
2Co 5:2  Оттого мы и воздыхаем, желая облечься в небесное наше жилище;
2Co 5:3  только бы нам и одетым не оказаться нагими.
2Co 5:4  Ибо мы, находясь в этой хижине, воздыхаем под бременем, потому что не хотим совлечься, но облечься, чтобы смертное поглощено было жизнью.
2Co 5:5  На сие самое и создал нас Бог и дал нам залог Духа.
2Co 5:6  Итак мы всегда благодушествуем; и как знаем, что, водворяясь в теле, мы устранены от Господа, --
2Co 5:7  ибо мы ходим верою, а не видением, --
2Co 5:8  то мы благодушествуем и желаем лучше выйти из тела и водвориться у Господа.
2Co 5:9  И потому ревностно стараемся, водворяясь ли, выходя ли, быть Ему угодными;
2Co 5:10  ибо всем нам должно явиться пред судилище Христово, чтобы каждому получить [соответственно тому], что он делал, живя в теле, доброе или худое.
2Co 5:11  Итак, зная страх Господень, мы вразумляем людей, Богу же мы открыты; надеюсь, что открыты и вашим совестям.
2Co 5:12  Не снова представляем себя вам, но даем вам повод хвалиться нами, дабы имели вы [что сказать] тем, которые хвалятся лицем, а не сердцем.
2Co 5:13  Если мы выходим из себя, то для Бога; если же скромны, то для вас.
2Co 5:14  Ибо любовь Христова объемлет нас, рассуждающих так: если один умер за всех, то все умерли.
2Co 5:15  А Христос за всех умер, чтобы живущие уже не для себя жили, но для умершего за них и воскресшего.
2Co 5:16  Потому отныне мы никого не знаем по плоти; если же и знали Христа по плоти, то ныне уже не знаем.
2Co 5:17  Итак, кто во Христе, [тот] новая тварь; древнее прошло, теперь все новое.
2Co 5:18  Все же от Бога, Иисусом Христом примирившего нас с Собою и давшего нам служение примирения,
2Co 5:19  потому что Бог во Христе примирил с Собою мир, не вменяя [людям] преступлений их, и дал нам слово примирения.
2Co 5:20  Итак мы--посланники от имени Христова, и как бы Сам Бог увещевает через нас; от имени Христова просим: примиритесь с Богом.
2Co 5:21  Ибо не знавшего греха Он сделал для нас [жертвою за] грех, чтобы мы в Нем сделались праведными пред Богом.
2Co 6:1  Мы же, как споспешники, умоляем вас, чтобы благодать Божия не тщетно была принята вами.
2Co 6:2  Ибо сказано: во время благоприятное Я услышал тебя и в день спасения помог тебе. Вот, теперь время благоприятное, вот, теперь день спасения.
2Co 6:3  Мы никому ни в чем не полагаем претыкания, чтобы не было порицаемо служение,
2Co 6:4  но во всем являем себя, как служители Божии, в великом терпении, в бедствиях, в нуждах, в тесных обстоятельствах,
2Co 6:5  под ударами, в темницах, в изгнаниях, в трудах, в бдениях, в постах,
2Co 6:6  в чистоте, в благоразумии, в великодушии, в благости, в Духе Святом, в нелицемерной любви,
2Co 6:7  в слове истины, в силе Божией, с оружием правды в правой и левой руке,
2Co 6:8  в чести и бесчестии, при порицаниях и похвалах: нас почитают обманщиками, но мы верны;
2Co 6:9  мы неизвестны, но нас узнают; нас почитают умершими, но вот, мы живы; нас наказывают, но мы не умираем;
2Co 6:10  нас огорчают, а мы всегда радуемся; мы нищи, но многих обогащаем; мы ничего не имеем, но всем обладаем.
2Co 6:11  Уста наши отверсты к вам, Коринфяне, сердце наше расширено.
2Co 6:12  Вам не тесно в нас; но в сердцах ваших тесно.
2Co 6:13  В равное возмездие, --говорю, как детям, --распространитесь и вы.
2Co 6:14  Не преклоняйтесь под чужое ярмо с неверными, ибо какое общение праведности с беззаконием? Что общего у света с тьмою?
2Co 6:15  Какое согласие между Христом и Велиаром? Или какое соучастие верного с неверным?
2Co 6:16  Какая совместность храма Божия с идолами? Ибо вы храм Бога живаго, как сказал Бог: вселюсь в них и буду ходить [в них]; и буду их Богом, и они будут Моим народом.
2Co 6:17  И потому выйдите из среды их и отделитесь, говорит Господь, и не прикасайтесь к нечистому; и Я прииму вас.
2Co 6:18  И буду вам Отцем, и вы будете Моими сынами и дщерями, говорит Господь Вседержитель.
2Co 7:1  Итак, возлюбленные, имея такие обетования, очистим себя от всякой скверны плоти и духа, совершая святыню в страхе Божием.
2Co 7:2  Вместите нас. Мы никого не обидели, никому не повредили, ни от кого не искали корысти.
2Co 7:3  Не в осуждение говорю; ибо я прежде сказал, что вы в сердцах наших, так чтобы вместе и умереть и жить.
2Co 7:4  Я много надеюсь на вас, много хвалюсь вами; я исполнен утешением, преизобилую радостью, при всей скорби нашей.
2Co 7:5  Ибо, когда пришли мы в Македонию, плоть наша не имела никакого покоя, но мы были стеснены отовсюду: отвне--нападения, внутри--страхи.
2Co 7:6  Но Бог, утешающий смиренных, утешил нас прибытием Тита,
2Co 7:7  и не только прибытием его, но и утешением, которым он утешался о вас, пересказывая нам о вашем усердии, о вашем плаче, о вашей ревности по мне, так что я еще более обрадовался.
2Co 7:8  Посему, если я опечалил вас посланием, не жалею, хотя и пожалел было; ибо вижу, что послание то опечалило вас, впрочем на время.
2Co 7:9  Теперь я радуюсь не потому, что вы опечалились, но что вы опечалились к покаянию; ибо опечалились ради Бога, так что нисколько не понесли от нас вреда.
2Co 7:10  Ибо печаль ради Бога производит неизменное покаяние ко спасению, а печаль мирская производит смерть.
2Co 7:11  Ибо то самое, что вы опечалились ради Бога, смотрите, какое произвело в вас усердие, какие извинения, какое негодование [на виновного], какой страх, какое желание, какую ревность, какое взыскание! По всему вы показали себя чистыми в этом деле.
2Co 7:12  Итак, если я писал к вам, то не ради оскорбителя и не ради оскорбленного, но чтобы вам открылось попечение наше о вас пред Богом.
2Co 7:13  Посему мы утешились утешением вашим; а еще более обрадованы мы радостью Тита, что вы все успокоили дух его.
2Co 7:14  Итак я не остался в стыде, если чем-либо о вас похвалился перед ним, но как вам мы говорили все истину, так и перед Титом похвала наша оказалась истинною;
2Co 7:15  и сердце его весьма расположено к вам, при воспоминании о послушании всех вас, как вы приняли его со страхом и трепетом.
2Co 7:16  Итак радуюсь, что во всем могу положиться на вас.
2Co 8:1  Уведомляем вас, братия, о благодати Божией, данной церквам Македонским,
2Co 8:2  ибо они среди великого испытания скорбями преизобилуют радостью; и глубокая нищета их преизбыточествует в богатстве их радушия.
2Co 8:3  Ибо они доброхотны по силам и сверх сил--я свидетель:
2Co 8:4  они весьма убедительно просили нас принять дар и участие [их] в служении святым;
2Co 8:5  и не только то, чего мы надеялись, но они отдали самих себя, во-первых, Господу, [потом] и нам по воле Божией;
2Co 8:6  поэтому мы просили Тита, чтобы он, как начал, так и окончил у вас и это доброе дело.
2Co 8:7  А как вы изобилуете всем: верою и словом, и познанием, и всяким усердием, и любовью вашею к нам, --так изобилуйте и сею добродетелью.
2Co 8:8  Говорю это не в виде повеления, но усердием других испытываю искренность и вашей любви.
2Co 8:9  Ибо вы знаете благодать Господа нашего Иисуса Христа, что Он, будучи богат, обнищал ради вас, дабы вы обогатились Его нищетою.
2Co 8:10  Я даю на это совет: ибо это полезно вам, которые не только начали делать сие, но и желали того еще с прошедшего года.
2Co 8:11  Совершите же теперь самое дело, дабы, чего усердно желали, то и исполнено было по достатку.
2Co 8:12  Ибо если есть усердие, то оно принимается смотря по тому, кто что имеет, а не по тому, чего не имеет.
2Co 8:13  Не [требуется], чтобы другим [было] облегчение, а вам тяжесть, но чтобы была равномерность.
2Co 8:14  Ныне ваш избыток в [восполнение] их недостатка; а после их избыток в [восполнение] вашего недостатка, чтобы была равномерность,
2Co 8:15  как написано: кто собрал много, не имел лишнего; и кто мало, не имел недостатка.
2Co 8:16  Благодарение Богу, вложившему в сердце Титово такое усердие к вам.
2Co 8:17  Ибо, хотя и я просил его, впрочем он, будучи очень усерден, пошел к вам добровольно.
2Co 8:18  С ним послали мы также брата, во всех церквах похваляемого за благовествование,
2Co 8:19  и притом избранного от церквей сопутствовать нам для сего благотворения, которому мы служим во славу Самого Господа и [в] [соответствие] вашему усердию,
2Co 8:20  остерегаясь, чтобы нам не подвергнуться от кого нареканию при таком обилии приношений, вверяемых нашему служению;
2Co 8:21  ибо мы стараемся о добром не только пред Господом, но и пред людьми.
2Co 8:22  Мы послали с ними и брата нашего, которого усердие много раз испытали во многом и который ныне еще усерднее по великой уверенности в вас.
2Co 8:23  Что касается до Тита, это--мой товарищ и сотрудник у вас; а что до братьев наших, это--посланники церквей, слава Христова.
2Co 8:24  Итак перед лицем церквей дайте им доказательство любви вашей и того, что мы [справедливо] хвалимся вами.
2Co 9:1  Для меня впрочем излишне писать вам о вспоможении святым,
2Co 9:2  ибо я знаю усердие ваше и хвалюсь вами перед Македонянами, что Ахаия приготовлена еще с прошедшего года; и ревность ваша поощрила многих.
2Co 9:3  Братьев же послал я для того, чтобы похвала моя о вас не оказалась тщетною в сем случае, но чтобы вы, как я говорил, были приготовлены,
2Co 9:4  [и] чтобы, когда придут со мною Македоняне и найдут вас неготовыми, не остались в стыде мы, --не говорю `вы', --похвалившись с такою уверенностью.
2Co 9:5  Посему я почел за нужное упросить братьев, чтобы они наперед пошли к вам и предварительно озаботились, дабы возвещенное уже благословение ваше было готово, как благословение, а не как побор.
2Co 9:6  При сем скажу: кто сеет скупо, тот скупо и пожнет; а кто сеет щедро, тот щедро и пожнет.
2Co 9:7  Каждый [уделяй] по расположению сердца, не с огорчением и не с принуждением; ибо доброхотно дающего любит Бог.
2Co 9:8  Бог же силен обогатить вас всякою благодатью, чтобы вы, всегда и во всем имея всякое довольство, были богаты на всякое доброе дело,
2Co 9:9  как написано: расточил, раздал нищим; правда его пребывает в век.
2Co 9:10  Дающий же семя сеющему и хлеб в пищу подаст обилие посеянному вами и умножит плоды правды вашей,
2Co 9:11  так чтобы вы всем богаты были на всякую щедрость, которая через нас производит благодарение Богу.
2Co 9:12  Ибо дело служения сего не только восполняет скудость святых, но и производит во многих обильные благодарения Богу;
2Co 9:13  ибо, видя опыт сего служения, они прославляют Бога за покорность исповедуемому вами Евангелию Христову и за искреннее общение с ними и со всеми,
2Co 9:14  молясь за вас, по расположению к вам, за преизбыточествующую в вас благодать Божию.
2Co 9:15  Благодарение Богу за неизреченный дар Его!
2Co 10:1  Я же, Павел, который лично между вами скромен, а заочно против вас отважен, убеждаю вас кротостью и снисхождением Христовым.
2Co 10:2  Прошу, чтобы мне по пришествии моем не прибегать к той твердой смелости, которую думаю употребить против некоторых, помышляющих о нас, что мы поступаем по плоти.
2Co 10:3  Ибо мы, ходя во плоти, не по плоти воинствуем.
2Co 10:4  Оружия воинствования нашего не плотские, но сильные Богом на разрушение твердынь: [ими] ниспровергаем замыслы
2Co 10:5  и всякое превозношение, восстающее против познания Божия, и пленяем всякое помышление в послушание Христу,
2Co 10:6  и готовы наказать всякое непослушание, когда ваше послушание исполнится.
2Co 10:7  На личность ли смотрите? Кто уверен в себе, что он Христов, тот сам по себе суди, что, как он Христов, так и мы Христовы.
2Co 10:8  Ибо если бы я и более стал хвалиться нашею властью, которую Господь дал нам к созиданию, а не к расстройству вашему, то не остался бы в стыде.
2Co 10:9  Впрочем, да не покажется, что я устрашаю вас [только] посланиями.
2Co 10:10  Так как [некто] говорит: в посланиях он строг и силен, а в личном присутствии слаб, и речь [его] незначительна, --
2Co 10:11  такой пусть знает, что, каковы мы на словах в посланиях заочно, таковы и на деле лично.
2Co 10:12  Ибо мы не смеем сопоставлять или сравнивать себя с теми, которые сами себя выставляют: они измеряют себя самими собою и сравнивают себя с собою неразумно.
2Co 10:13  А мы не без меры хвалиться будем, но по мере удела, какой назначил нам Бог в такую меру, чтобы достигнуть и до вас.
2Co 10:14  Ибо мы не напрягаем себя, как не достигшие до вас, потому что достигли и до вас благовествованием Христовым.
2Co 10:15  Мы не без меры хвалимся, не чужими трудами, но надеемся, с возрастанием веры вашей, с избытком увеличить в вас удел наш,
2Co 10:16  так чтобы и далее вас проповедывать Евангелие, а не хвалиться готовым в чужом уделе.
2Co 10:17  Хвалящийся хвались о Господе.
2Co 10:18  Ибо не тот достоин, кто сам себя хвалит, но кого хвалит Господь.
2Co 11:1  О, если бы вы несколько были снисходительны к моему неразумию! Но вы и снисходите ко мне.
2Co 11:2  Ибо я ревную о вас ревностью Божиею; потому что я обручил вас единому мужу, чтобы представить Христу чистою девою.
2Co 11:3  Но боюсь, чтобы, как змий хитростью своею прельстил Еву, так и ваши умы не повредились, [уклонившись] от простоты во Христе.
2Co 11:4  Ибо если бы кто, придя, начал проповедывать другого Иисуса, которого мы не проповедывали, или если бы вы получили иного Духа, которого не получили, или иное благовестие, которого не принимали, --то вы были бы очень снисходительны [к тому].
2Co 11:5  Но я думаю, что у меня ни в чем нет недостатка против высших Апостолов:
2Co 11:6  хотя я и невежда в слове, но не в познании. Впрочем мы во всем совершенно известны вам.
2Co 11:7  Согрешил ли я тем, что унижал себя, чтобы возвысить вас, потому что безмездно проповедывал вам Евангелие Божие?
2Co 11:8  Другим церквам я причинял издержки, получая [от них] содержание для служения вам; и, будучи у вас, хотя терпел недостаток, никому не докучал,
2Co 11:9  ибо недостаток мой восполнили братия, пришедшие из Македонии; да и во всем я старался и постараюсь не быть вам в тягость.
2Co 11:10  По истине Христовой во мне [скажу], что похвала сия не отнимется у меня в странах Ахаии.
2Co 11:11  Почему же [так поступаю]? Потому ли, что не люблю вас? Богу известно! Но как поступаю, так и буду поступать,
2Co 11:12  чтобы не дать повода ищущим повода, дабы они, чем хвалятся, в том оказались [такими же], как и мы.
2Co 11:13  Ибо таковые лжеапостолы, лукавые делатели, принимают вид Апостолов Христовых.
2Co 11:14  И неудивительно: потому что сам сатана принимает вид Ангела света,
2Co 11:15  а потому не великое дело, если и служители его принимают вид служителей правды; но конец их будет по делам их.
2Co 11:16  Еще скажу: не почти кто-нибудь меня неразумным; а если не так, то примите меня, хотя как неразумного, чтобы и мне сколько-- нибудь похвалиться.
2Co 11:17  Что скажу, то скажу не в Господе, но как бы в неразумии при такой отважности на похвалу.
2Co 11:18  Как многие хвалятся по плоти, то и я буду хвалиться.
2Co 11:19  Ибо вы, люди разумные, охотно терпите неразумных:
2Co 11:20  вы терпите, когда кто вас порабощает, когда кто объедает, когда кто обирает, когда кто превозносится, когда кто бьет вас в лицо.
2Co 11:21  К стыду говорю, что [на это] у нас недоставало сил. А если кто смеет [хвалиться] чем-либо, то (скажу по неразумию) смею и я.
2Co 11:22  Они Евреи? и я. Израильтяне? и я. Семя Авраамово? и я.
2Co 11:23  Христовы служители? (в безумии говорю:) я больше. Я гораздо более [был] в трудах, безмерно в ранах, более в темницах и многократно при смерти.
2Co 11:24  От Иудеев пять раз дано мне было по сорока [ударов] без одного;
2Co 11:25  три раза меня били палками, однажды камнями побивали, три раза я терпел кораблекрушение, ночь и день пробыл во глубине [морской];
2Co 11:26  много раз [был] в путешествиях, в опасностях на реках, в опасностях от разбойников, в опасностях от единоплеменников, в опасностях от язычников, в опасностях в городе, в опасностях в пустыне, в опасностях на море, в опасностях между лжебратиями,
2Co 11:27  в труде и в изнурении, часто в бдении, в голоде и жажде, часто в посте, на стуже и в наготе.
2Co 11:28  Кроме посторонних [приключений], у меня ежедневно стечение [людей], забота о всех церквах.
2Co 11:29  Кто изнемогает, с кем бы и я не изнемогал? Кто соблазняется, за кого бы я не воспламенялся?
2Co 11:30  Если должно мне хвалиться, то буду хвалиться немощью моею.
2Co 11:31  Бог и Отец Господа нашего Иисуса Христа, благословенный во веки, знает, что я не лгу.
2Co 11:32  В Дамаске областной правитель царя Ареты стерег город Дамаск, чтобы схватить меня;
2Co 11:33  и я в корзине был спущен из окна по стене и избежал его рук.
2Co 12:1  Не полезно хвалиться мне, ибо я приду к видениям и откровениям Господним.
2Co 12:2  Знаю человека во Христе, который назад тому четырнадцать лет (в теле ли--не знаю, вне ли тела--не знаю: Бог знает) восхищен был до третьего неба.
2Co 12:3  И знаю о таком человеке ([только] не знаю--в теле, или вне тела: Бог знает),
2Co 12:4  что он был восхищен в рай и слышал неизреченные слова, которых человеку нельзя пересказать.
2Co 12:5  Таким [человеком] могу хвалиться; собою же не похвалюсь, разве только немощами моими.
2Co 12:6  Впрочем, если захочу хвалиться, не буду неразумен, потому что скажу истину; но я удерживаюсь, чтобы кто не подумал о мне более, нежели сколько во мне видит или слышит от меня.
2Co 12:7  И чтобы я не превозносился чрезвычайностью откровений, дано мне жало в плоть, ангел сатаны, удручать меня, чтобы я не превозносился.
2Co 12:8  Трижды молил я Господа о том, чтобы удалил его от меня.
2Co 12:9  Но [Господь] сказал мне: `довольно для тебя благодати Моей, ибо сила Моя совершается в немощи'. И потому я гораздо охотнее буду хвалиться своими немощами, чтобы обитала во мне сила Христова.
2Co 12:10  Посему я благодушествую в немощах, в обидах, в нуждах, в гонениях, в притеснениях за Христа, ибо, когда я немощен, тогда силен.
2Co 12:11  Я дошел до неразумия, хвалясь; вы меня [к сему] принудили. Вам бы надлежало хвалить меня, ибо у меня ни в чем нет недостатка против высших Апостолов, хотя я и ничто.
2Co 12:12  Признаки Апостола оказались перед вами всяким терпением, знамениями, чудесами и силами.
2Co 12:13  Ибо чего у вас недостает перед прочими церквами, разве только того, что сам я не был вам в тягость? Простите мне такую вину.
2Co 12:14  Вот, в третий раз я готов идти к вам, и не буду отягощать вас, ибо я ищу не вашего, а вас. Не дети должны собирать имение для родителей, но родители для детей.
2Co 12:15  Я охотно буду издерживать [свое] и истощать себя за души ваши, несмотря на то, что, чрезвычайно любя вас, я менее любим вами.
2Co 12:16  Положим, [что] сам я не обременял вас, но, будучи хитр, лукавством брал с вас.
2Co 12:17  Но пользовался ли я [чем] от вас через кого-нибудь из тех, кого посылал к вам?
2Co 12:18  Я упросил Тита и послал с ним одного из братьев: Тит воспользовался ли чем от вас? Не в одном ли духе мы действовали? Не одним ли путем ходили?
2Co 12:19  Не думаете ли еще, что мы [только] оправдываемся перед вами? Мы говорим пред Богом, во Христе, и все это, возлюбленные, к вашему назиданию.
2Co 12:20  Ибо я опасаюсь, чтобы мне, по пришествии моем, не найти вас такими, какими не желаю, также чтобы и вам не найти меня таким, каким не желаете: чтобы [не найти у вас] раздоров, зависти, гнева, ссор, клевет, ябед, гордости, беспорядков,
2Co 12:21  чтобы опять, когда приду, не уничижил меня у вас Бог мой и [чтобы] не оплакивать мне многих, которые согрешили прежде и не покаялись в нечистоте, блудодеянии и непотребстве, какое делали.
2Co 13:1  В третий уже раз иду к вам. При устах двух или трех свидетелей будет твердо всякое слово.
2Co 13:2  Я предварял и предваряю, как бы находясь [у вас] во второй раз, и теперь, отсутствуя, пишу прежде согрешившим и всем прочим, что, когда опять приду, не пощажу.
2Co 13:3  Вы ищете доказательства на то, Христос ли говорит во мне: Он не бессилен для вас, но силен в вас.
2Co 13:4  Ибо, хотя Он и распят в немощи, но жив силою Божиею; и мы также, [хотя] немощны в Нем, но будем живы с Ним силою Божиею в вас.
2Co 13:5  Испытывайте самих себя, в вере ли вы; самих себя исследывайте. Или вы не знаете самих себя, что Иисус Христос в вас? Разве только вы не то, чем должны быть.
2Co 13:6  О нас же, надеюсь, узнаете, что мы то, чем быть должны.
2Co 13:7  Молим Бога, чтобы вы не делали никакого зла, не для того, чтобы нам показаться, чем должны быть; но чтобы вы делали добро, хотя бы мы казались и не тем, чем должны быть.
2Co 13:8  Ибо мы не сильны против истины, но сильны за истину.
2Co 13:9  Мы радуемся, когда мы немощны, а вы сильны; о сем-то и молимся, о вашем совершенстве.
2Co 13:10  Для того я и пишу сие в отсутствии, чтобы в присутствии не употребить строгости по власти, данной мне Господом к созиданию, а не к разорению.
2Co 13:11  Впрочем, братия, радуйтесь, усовершайтесь, утешайтесь, будьте единомысленны, мирны, --и Бог любви и мира будет с вами.
2Co 13:12  Приветствуйте друг друга лобзанием святым.
2Co 13:12  Приветствуйте друг друга лобзанием святым.
2Co 13:13  Благодать Господа нашего Иисуса Христа, и любовь Бога Отца, и общение Святаго Духа со всеми вами. Аминь.


\end{document}