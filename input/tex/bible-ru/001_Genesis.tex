\begin{document}

\title{Бытие}


\chapter{1}

\par 1 В начале сотворил Бог небо и землю.
\par 2 Земля же была безвидна и пуста, и тьма над бездною, и Дух Божий носился над водою.
\par 3 И сказал Бог: да будет свет. И стал свет.
\par 4 И увидел Бог свет, что он хорош, и отделил Бог свет от тьмы.
\par 5 И назвал Бог свет днем, а тьму ночью. И был вечер, и было утро: день один.
\par 6 И сказал Бог: да будет твердь посреди воды, и да отделяет она воду от воды.
\par 7 И создал Бог твердь, и отделил воду, которая под твердью, от воды, которая над твердью. И стало так.
\par 8 И назвал Бог твердь небом. И был вечер, и было утро: день второй.
\par 9 И сказал Бог: да соберется вода, которая под небом, в одно место, и да явится суша. И стало так.
\par 10 И назвал Бог сушу землею, а собрание вод назвал морями. И увидел Бог, что [это] хорошо.
\par 11 И сказал Бог: да произрастит земля зелень, траву, сеющую семя дерево плодовитое, приносящее по роду своему плод, в котором семя его на земле. И стало так.
\par 12 И произвела земля зелень, траву, сеющую семя по роду ее, и дерево, приносящее плод, в котором семя его по роду его. И увидел Бог, что [это] хорошо.
\par 13 И был вечер, и было утро: день третий.
\par 14 И сказал Бог: да будут светила на тверди небесной для отделения дня от ночи, и для знамений, и времен, и дней, и годов;
\par 15 и да будут они светильниками на тверди небесной, чтобы светить на землю. И стало так.
\par 16 И создал Бог два светила великие: светило большее, для управления днем, и светило меньшее, для управления ночью, и звезды;
\par 17 и поставил их Бог на тверди небесной, чтобы светить на землю,
\par 18 и управлять днем и ночью, и отделять свет от тьмы. И увидел Бог, что [это] хорошо.
\par 19 И был вечер, и было утро: день четвертый.
\par 20 И сказал Бог: да произведет вода пресмыкающихся, душу живую; и птицы да полетят над землею, по тверди небесной.
\par 21 И сотворил Бог рыб больших и всякую душу животных пресмыкающихся, которых произвела вода, по роду их, и всякую птицу пернатую по роду ее. И увидел Бог, что [это] хорошо.
\par 22 И благословил их Бог, говоря: плодитесь и размножайтесь, и наполняйте воды в морях, и птицы да размножаются на земле.
\par 23 И был вечер, и было утро: день пятый.
\par 24 И сказал Бог: да произведет земля душу живую по роду ее, скотов, и гадов, и зверей земных по роду их. И стало так.
\par 25 И создал Бог зверей земных по роду их, и скот по роду его, и всех гадов земных по роду их. И увидел Бог, что [это] хорошо.
\par 26 И сказал Бог: сотворим человека по образу Нашему по подобию Нашему, и да владычествуют они над рыбами морскими, и над птицами небесными, и над скотом, и над всею землею, и над всеми гадами, пресмыкающимися по земле.
\par 27 И сотворил Бог человека по образу Своему, по образу Божию сотворил его; мужчину и женщину сотворил их.
\par 28 И благословил их Бог, и сказал им Бог: плодитесь и размножайтесь, и наполняйте землю, и обладайте ею, и владычествуйте над рыбами морскими и над птицами небесными, и над всяким животным, пресмыкающимся по земле.
\par 29 И сказал Бог: вот, Я дал вам всякую траву, сеющую семя, какая есть на всей земле, и всякое дерево, у которого плод древесный, сеющий семя; --вам [сие] будет в пищу;
\par 30 а всем зверям земным, и всем птицам небесным, и всякому пресмыкающемуся по земле, в котором душа живая, [дал] Я всю зелень травную в пищу. И стало так.
\par 31 И увидел Бог все, что Он создал, и вот, хорошо весьма. И был вечер, и было утро: день шестой.

\chapter{2}

\par 1 Так совершены небо и земля и все воинство их.
\par 2 И совершил Бог к седьмому дню дела Свои, которые Он делал, и почил в день седьмый от всех дел Своих, которые делал.
\par 3 И благословил Бог седьмой день, и освятил его, ибо в оный почил от всех дел Своих, которые Бог творил и созидал.
\par 4 Вот происхождение неба и земли, при сотворении их, в то время, когда Господь Бог создал землю и небо,
\par 5 и всякий полевой кустарник, которого еще не было на земле, и всякую полевую траву, которая еще не росла, ибо Господь Бог не посылал дождя на землю, и не было человека для возделывания земли,
\par 6 но пар поднимался с земли и орошал все лице земли.
\par 7 И создал Господь Бог человека из праха земного, и вдунул в лице его дыхание жизни, и стал человек душею живою.
\par 8 И насадил Господь Бог рай в Едеме на востоке, и поместил там человека, которого создал.
\par 9 И произрастил Господь Бог из земли всякое дерево, приятное на вид и хорошее для пищи, и дерево жизни посреди рая, и дерево познания добра и зла.
\par 10 Из Едема выходила река для орошения рая; и потом разделялась на четыре реки.
\par 11 Имя одной Фисон: она обтекает всю землю Хавила, ту, где золото;
\par 12 и золото той земли хорошее; там бдолах и камень оникс.
\par 13 Имя второй реки Гихон: она обтекает всю землю Куш.
\par 14 Имя третьей реки Хиддекель: она протекает пред Ассириею. Четвертая река Евфрат.
\par 15 И взял Господь Бог человека, и поселил его в саду Едемском, чтобы возделывать его и хранить его.
\par 16 И заповедал Господь Бог человеку, говоря: от всякого дерева в саду ты будешь есть,
\par 17 а от дерева познания добра и зла не ешь от него, ибо в день, в который ты вкусишь от него, смертью умрешь.
\par 18 И сказал Господь Бог: не хорошо быть человеку одному; сотворим ему помощника, соответственного ему.
\par 19 Господь Бог образовал из земли всех животных полевых и всех птиц небесных, и привел к человеку, чтобы видеть, как он назовет их, и чтобы, как наречет человек всякую душу живую, так и было имя ей.
\par 20 И нарек человек имена всем скотам и птицам небесным и всем зверям полевым; но для человека не нашлось помощника, подобного ему.
\par 21 И навел Господь Бог на человека крепкий сон; и, когда он уснул, взял одно из ребр его, и закрыл то место плотию.
\par 22 И создал Господь Бог из ребра, взятого у человека, жену, и привел ее к человеку.
\par 23 И сказал человек: вот, это кость от костей моих и плоть от плоти моей; она будет называться женою, ибо взята от мужа.
\par 24 Потому оставит человек отца своего и мать свою и прилепится к жене своей; и будут одна плоть.
\par 25 И были оба наги, Адам и жена его, и не стыдились.

\chapter{3}

\par 1 Змей был хитрее всех зверей полевых, которых создал Господь Бог. И сказал змей жене: подлинно ли сказал Бог: не ешьте ни от какого дерева в раю?
\par 2 И сказала жена змею: плоды с дерев мы можем есть,
\par 3 только плодов дерева, которое среди рая, сказал Бог, не ешьте их и не прикасайтесь к ним, чтобы вам не умереть.
\par 4 И сказал змей жене: нет, не умрете,
\par 5 но знает Бог, что в день, в который вы вкусите их, откроются глаза ваши, и вы будете, как боги, знающие добро и зло.
\par 6 И увидела жена, что дерево хорошо для пищи, и что оно приятно для глаз и вожделенно, потому что дает знание; и взяла плодов его и ела; и дала также мужу своему, и он ел.
\par 7 И открылись глаза у них обоих, и узнали они, что наги, и сшили смоковные листья, и сделали себе опоясания.
\par 8 И услышали голос Господа Бога, ходящего в раю во время прохлады дня; и скрылся Адам и жена его от лица Господа Бога между деревьями рая.
\par 9 И воззвал Господь Бог к Адаму и сказал ему: где ты?
\par 10 Он сказал: голос Твой я услышал в раю, и убоялся, потому что я наг, и скрылся.
\par 11 И сказал: кто сказал тебе, что ты наг? не ел ли ты от дерева, с которого Я запретил тебе есть?
\par 12 Адам сказал: жена, которую Ты мне дал, она дала мне от дерева, и я ел.
\par 13 И сказал Господь Бог жене: что ты это сделала? Жена сказала: змей обольстил меня, и я ела.
\par 14 И сказал Господь Бог змею: за то, что ты сделал это, проклят ты пред всеми скотами и пред всеми зверями полевыми; ты будешь ходить на чреве твоем, и будешь есть прах во все дни жизни твоей;
\par 15 и вражду положу между тобою и между женою, и между семенем твоим и между семенем ее; оно будет поражать тебя в голову, а ты будешь жалить его в пяту.
\par 16 Жене сказал: умножая умножу скорбь твою в беременности твоей; в болезни будешь рождать детей; и к мужу твоему влечение твое, и он будет господствовать над тобою.
\par 17 Адаму же сказал: за то, что ты послушал голоса жены твоей и ел от дерева, о котором Я заповедал тебе, сказав: не ешь от него, проклята земля за тебя; со скорбью будешь питаться от нее во все дни жизни твоей;
\par 18 терния и волчцы произрастит она тебе; и будешь питаться полевою травою;
\par 19 в поте лица твоего будешь есть хлеб, доколе не возвратишься в землю, из которой ты взят, ибо прах ты и в прах возвратишься.
\par 20 И нарек Адам имя жене своей: Ева, ибо она стала матерью всех живущих.
\par 21 И сделал Господь Бог Адаму и жене его одежды кожаные и одел их.
\par 22 И сказал Господь Бог: вот, Адам стал как один из Нас, зная добро и зло; и теперь как бы не простер он руки своей, и не взял также от дерева жизни, и не вкусил, и не стал жить вечно.
\par 23 И выслал его Господь Бог из сада Едемского, чтобы возделывать землю, из которой он взят.
\par 24 И изгнал Адама, и поставил на востоке у сада Едемского Херувима и пламенный меч обращающийся, чтобы охранять путь к дереву жизни.

\chapter{4}

\par 1 Адам познал Еву, жену свою; и она зачала, и родила Каина, и сказала: приобрела я человека от Господа.
\par 2 И еще родила брата его, Авеля. И был Авель пастырь овец, а Каин был земледелец.
\par 3 Спустя несколько времени, Каин принес от плодов земли дар Господу,
\par 4 и Авель также принес от первородных стада своего и от тука их. И призрел Господь на Авеля и на дар его,
\par 5 а на Каина и на дар его не призрел. Каин сильно огорчился, и поникло лице его.
\par 6 И сказал Господь Каину: почему ты огорчился? и отчего поникло лице твое?
\par 7 если делаешь доброе, то не поднимаешь ли лица? а если не делаешь доброго, то у дверей грех лежит; он влечет тебя к себе, но ты господствуй над ним.
\par 8 И сказал Каин Авелю, брату своему. И когда они были в поле, восстал Каин на Авеля, брата своего, и убил его.
\par 9 И сказал Господь Каину: где Авель, брат твой? Он сказал: не знаю; разве я сторож брату моему?
\par 10 И сказал: что ты сделал? голос крови брата твоего вопиет ко Мне от земли;
\par 11 и ныне проклят ты от земли, которая отверзла уста свои принять кровь брата твоего от руки твоей;
\par 12 когда ты будешь возделывать землю, она не станет более давать силы своей для тебя; ты будешь изгнанником и скитальцем на земле.
\par 13 И сказал Каин Господу: наказание мое больше, нежели снести можно;
\par 14 вот, Ты теперь сгоняешь меня с лица земли, и от лица Твоего я скроюсь, и буду изгнанником и скитальцем на земле; и всякий, кто встретится со мною, убьет меня.
\par 15 И сказал ему Господь: за то всякому, кто убьет Каина, отмстится всемеро. И сделал Господь Каину знамение, чтобы никто, встретившись с ним, не убил его.
\par 16 И пошел Каин от лица Господня и поселился в земле Нод, на восток от Едема.
\par 17 И познал Каин жену свою; и она зачала и родила Еноха. И построил он город; и назвал город по имени сына своего: Енох.
\par 18 У Еноха родился Ирад; Ирад родил Мехиаеля; Мехиаель родил Мафусала; Мафусал родил Ламеха.
\par 19 И взял себе Ламех две жены: имя одной: Ада, и имя второй: Цилла.
\par 20 Ада родила Иавала: он был отец живущих в шатрах со стадами.
\par 21 Имя брату его Иувал: он был отец всех играющих на гуслях и свирели.
\par 22 Цилла также родила Тувалкаина, который был ковачом всех орудий из меди и железа. И сестра Тувалкаина Ноема.
\par 23 И сказал Ламех женам своим: Ада и Цилла! послушайте голоса моего; жены Ламеховы! внимайте словам моим: я убил мужа в язву мне и отрока в рану мне;
\par 24 если за Каина отмстится всемеро, то за Ламеха в семьдесят раз всемеро.
\par 25 И познал Адам еще жену свою, и она родила сына, и нарекла ему имя: Сиф, потому что, [говорила она], Бог положил мне другое семя, вместо Авеля, которого убил Каин.
\par 26 У Сифа также родился сын, и он нарек ему имя: Енос; тогда начали призывать имя Господа.

\chapter{5}

\par 1 Вот родословие Адама: когда Бог сотворил человека, по подобию Божию создал его,
\par 2 мужчину и женщину сотворил их, и благословил их, и нарек им имя: человек, в день сотворения их.
\par 3 Адам жил сто тридцать лет и родил [сына] по подобию своему по образу своему, и нарек ему имя: Сиф.
\par 4 Дней Адама по рождении им Сифа было восемьсот лет, и родил он сынов и дочерей.
\par 5 Всех же дней жизни Адамовой было девятьсот тридцать лет; и он умер.
\par 6 Сиф жил сто пять лет и родил Еноса.
\par 7 По рождении Еноса Сиф жил восемьсот семь лет и родил сынов и дочерей.
\par 8 Всех же дней Сифовых было девятьсот двенадцать лет; и он умер.
\par 9 Енос жил девяносто лет и родил Каинана.
\par 10 По рождении Каинана Енос жил восемьсот пятнадцать лет и родил сынов и дочерей.
\par 11 Всех же дней Еноса было девятьсот пять лет; и он умер.
\par 12 Каинан жил семьдесят лет и родил Малелеила.
\par 13 По рождении Малелеила Каинан жил восемьсот сорок лет и родил сынов и дочерей.
\par 14 Всех же дней Каинана было девятьсот десять лет; и он умер.
\par 15 Малелеил жил шестьдесят пять лет и родил Иареда.
\par 16 По рождении Иареда Малелеил жил восемьсот тридцать лет и родил сынов и дочерей.
\par 17 Всех же дней Малелеила было восемьсот девяносто пять лет; и он умер.
\par 18 Иаред жил сто шестьдесят два года и родил Еноха.
\par 19 По рождении Еноха Иаред жил восемьсот лет и родил сынов и дочерей.
\par 20 Всех же дней Иареда было девятьсот шестьдесят два года; и он умер.
\par 21 Енох жил шестьдесят пять лет и родил Мафусала.
\par 22 И ходил Енох пред Богом, по рождении Мафусала, триста лет и родил сынов и дочерей.
\par 23 Всех же дней Еноха было триста шестьдесят пять лет.
\par 24 И ходил Енох пред Богом; и не стало его, потому что Бог взял его.
\par 25 Мафусал жил сто восемьдесят семь лет и родил Ламеха.
\par 26 По рождении Ламеха Мафусал жил семьсот восемьдесят два года и родил сынов и дочерей.
\par 27 Всех же дней Мафусала было девятьсот шестьдесят девять лет; и он умер.
\par 28 Ламех жил сто восемьдесят два года и родил сына,
\par 29 и нарек ему имя: Ной, сказав: он утешит нас в работе нашей и в трудах рук наших при [возделывании] земли, которую проклял Господь.
\par 30 И жил Ламех по рождении Ноя пятьсот девяносто пять лет и родил сынов и дочерей.
\par 31 Всех же дней Ламеха было семьсот семьдесят семь лет; и он умер.
\par 32 Ною было пятьсот лет и родил Ной Сима, Хама и Иафета.

\chapter{6}

\par 1 Когда люди начали умножаться на земле и родились у них дочери,
\par 2 тогда сыны Божии увидели дочерей человеческих, что они красивы, и брали [их] себе в жены, какую кто избрал.
\par 3 И сказал Господь: не вечно Духу Моему быть пренебрегаемым человеками; потому что они плоть; пусть будут дни их сто двадцать лет.
\par 4 В то время были на земле исполины, особенно же с того времени, как сыны Божии стали входить к дочерям человеческим, и они стали рождать им: это сильные, издревле славные люди.
\par 5 И увидел Господь, что велико развращение человеков на земле, и что все мысли и помышления сердца их были зло во всякое время;
\par 6 и раскаялся Господь, что создал человека на земле, и восскорбел в сердце Своем.
\par 7 И сказал Господь: истреблю с лица земли человеков, которых Я сотворил, от человека до скотов, и гадов и птиц небесных истреблю, ибо Я раскаялся, что создал их.
\par 8 Ной же обрел благодать пред очами Господа.
\par 9 Вот житие Ноя: Ной был человек праведный и непорочный в роде своем; Ной ходил пред Богом.
\par 10 Ной родил трех сынов: Сима, Хама и Иафета.
\par 11 Но земля растлилась пред лицем Божиим, и наполнилась земля злодеяниями.
\par 12 И воззрел Бог на землю, и вот, она растленна, ибо всякая плоть извратила путь свой на земле.
\par 13 И сказал Бог Ною: конец всякой плоти пришел пред лице Мое, ибо земля наполнилась от них злодеяниями; и вот, Я истреблю их с земли.
\par 14 Сделай себе ковчег из дерева гофер; отделения сделай в ковчеге и осмоли его смолою внутри и снаружи.
\par 15 И сделай его так: длина ковчега триста локтей; ширина его пятьдесят локтей, а высота его тридцать локтей.
\par 16 И сделай отверстие в ковчеге, и в локоть сведи его вверху, и дверь в ковчег сделай с боку его; устрой в нем нижнее, второе и третье [жилье].
\par 17 И вот, Я наведу на землю потоп водный, чтоб истребить всякую плоть, в которой есть дух жизни, под небесами; все, что есть на земле, лишится жизни.
\par 18 Но с тобою Я поставлю завет Мой, и войдешь в ковчег ты, и сыновья твои, и жена твоя, и жены сынов твоих с тобою.
\par 19 Введи также в ковчег из всех животных, и от всякой плоти по паре, чтоб они остались с тобою в живых; мужеского пола и женского пусть они будут.
\par 20 Из птиц по роду их, и из скотов по роду их, и из всех пресмыкающихся по земле по роду их, из всех по паре войдут к тебе, чтобы остались в живых.
\par 21 Ты же возьми себе всякой пищи, какою питаются, и собери к себе; и будет она для тебя и для них пищею.
\par 22 И сделал Ной все: как повелел ему Бог, так он и сделал.

\chapter{7}

\par 1 И сказал Господь Ною: войди ты и все семейство твое в ковчег, ибо тебя увидел Я праведным предо Мною в роде сем;
\par 2 и всякого скота чистого возьми по семи, мужеского пола и женского, а из скота нечистого по два, мужеского пола и женского;
\par 3 также и из птиц небесных по семи, мужеского пола и женского, чтобы сохранить племя для всей земли,
\par 4 ибо чрез семь дней Я буду изливать дождь на землю сорок дней и сорок ночей; и истреблю все существующее, что Я создал, с лица земли.
\par 5 Ной сделал все, что Господь повелел ему.
\par 6 Ной же был шестисот лет, как потоп водный пришел на землю.
\par 7 И вошел Ной и сыновья его, и жена его, и жены сынов его с ним в ковчег от вод потопа.
\par 8 И из скотов чистых и из скотов нечистых, и из всех пресмыкающихся по земле
\par 9 по паре, мужеского пола и женского, вошли к Ною в ковчег, как Бог повелел Ною.
\par 10 Чрез семь дней воды потопа пришли на землю.
\par 11 В шестисотый год жизни Ноевой, во второй месяц, в семнадцатый день месяца, в сей день разверзлись все источники великой бездны, и окна небесные отворились;
\par 12 и лился на землю дождь сорок дней и сорок ночей.
\par 13 В сей самый день вошел в ковчег Ной, и Сим, Хам и Иафет, сыновья Ноевы, и жена Ноева, и три жены сынов его с ними.
\par 14 Они, и все звери по роду их, и всякий скот по роду его, и все гады, пресмыкающиеся по земле, по роду их, и все летающие по роду их, все птицы, все крылатые,
\par 15 и вошли к Ною в ковчег по паре от всякой плоти, в которой есть дух жизни;
\par 16 и вошедшие мужеский и женский пол всякой плоти вошли, как повелел ему Бог. И затворил Господь за ним.
\par 17 И продолжалось на земле наводнение сорок дней, и умножилась вода, и подняла ковчег, и он возвысился над землею;
\par 18 вода же усиливалась и весьма умножалась на земле, и ковчег плавал по поверхности вод.
\par 19 И усилилась вода на земле чрезвычайно, так что покрылись все высокие горы, какие есть под всем небом;
\par 20 на пятнадцать локтей поднялась над ними вода, и покрылись горы.
\par 21 И лишилась жизни всякая плоть, движущаяся по земле, и птицы, и скоты, и звери, и все гады, ползающие по земле, и все люди;
\par 22 все, что имело дыхание духа жизни в ноздрях своих на суше, умерло.
\par 23 Истребилось всякое существо, которое было на поверхности земли; от человека до скота, и гадов, и птиц небесных, --все истребилось с земли, остался только Ной и что [было] с ним в ковчеге.
\par 24 Вода же усиливалась на земле сто пятьдесят дней.

\chapter{8}

\par 1 И вспомнил Бог о Ное, и о всех зверях, и о всех скотах, (и о всех птицах, и о всех гадах пресмыкающихся,) бывших с ним в ковчеге; и навел Бог ветер на землю, и воды остановились.
\par 2 И закрылись источники бездны и окна небесные, и перестал дождь с неба.
\par 3 Вода же постепенно возвращалась с земли, и стала убывать вода по окончании ста пятидесяти дней.
\par 4 И остановился ковчег в седьмом месяце, в семнадцатый день месяца, на горах Араратских.
\par 5 Вода постоянно убывала до десятого месяца; в первый день десятого месяца показались верхи гор.
\par 6 По прошествии сорока дней Ной открыл сделанное им окно ковчега
\par 7 и выпустил ворона, который, вылетев, отлетал и прилетал, пока осушилась земля от воды.
\par 8 Потом выпустил от себя голубя, чтобы видеть, сошла ли вода с лица земли,
\par 9 но голубь не нашел места покоя для ног своих и возвратился к нему в ковчег, ибо вода была еще на поверхности всей земли; и он простер руку свою, и взял его, и принял к себе в ковчег.
\par 10 И помедлил еще семь дней других и опять выпустил голубя из ковчега.
\par 11 Голубь возвратился к нему в вечернее время, и вот, свежий масличный лист во рту у него, и Ной узнал, что вода сошла с земли.
\par 12 Он помедлил еще семь дней других и выпустил голубя; и он уже не возвратился к нему.
\par 13 Шестьсот первого года к первому [дню] первого месяца иссякла вода на земле; и открыл Ной кровлю ковчега и посмотрел, и вот, обсохла поверхность земли.
\par 14 И во втором месяце, к двадцать седьмому дню месяца, земля высохла.
\par 15 И сказал Бог Ною:
\par 16 выйди из ковчега ты и жена твоя, и сыновья твои, и жены сынов твоих с тобою;
\par 17 выведи с собою всех животных, которые с тобою, от всякой плоти, из птиц, и скотов, и всех гадов, пресмыкающихся по земле: пусть разойдутся они по земле, и пусть плодятся и размножаются на земле.
\par 18 И вышел Ной и сыновья его, и жена его, и жены сынов его с ним;
\par 19 все звери, и все гады, и все птицы, все движущееся по земле, по родам своим, вышли из ковчега.
\par 20 И устроил Ной жертвенник Господу; и взял из всякого скота чистого и из всех птиц чистых и принес во всесожжение на жертвеннике.
\par 21 И обонял Господь приятное благоухание, и сказал Господь в сердце Своем: не буду больше проклинать землю за человека, потому что помышление сердца человеческого--зло от юности его; и не буду больше поражать всего живущего, как Я сделал:
\par 22 впредь во все дни земли сеяние и жатва, холод и зной, лето и зима, день и ночь не прекратятся.

\chapter{9}

\par 1 И благословил Бог Ноя и сынов его и сказал им: плодитесь и размножайтесь, и наполняйте землю.
\par 2 да страшатся и да трепещут вас все звери земные, и все птицы небесные, все, что движется на земле, и все рыбы морские: в ваши руки отданы они;
\par 3 все движущееся, что живет, будет вам в пищу; как зелень травную даю вам все;
\par 4 только плоти с душею ее, с кровью ее, не ешьте;
\par 5 Я взыщу и вашу кровь, [в которой] жизнь ваша, взыщу ее от всякого зверя, взыщу также душу человека от руки человека, от руки брата его;
\par 6 кто прольет кровь человеческую, того кровь прольется рукою человека: ибо человек создан по образу Божию;
\par 7 вы же плодитесь и размножайтесь, и распространяйтесь по земле, и умножайтесь на ней.
\par 8 И сказал Бог Ною и сынам его с ним:
\par 9 вот, Я поставляю завет Мой с вами и с потомством вашим после вас,
\par 10 и со всякою душею живою, которая с вами, с птицами и со скотами, и со всеми зверями земными, которые у вас, со всеми вышедшими из ковчега, со всеми животными земными;
\par 11 поставляю завет Мой с вами, что не будет более истреблена всякая плоть водами потопа, и не будет уже потопа на опустошение земли.
\par 12 И сказал Бог: вот знамение завета, который Я поставляю между Мною и между вами и между всякою душею живою, которая с вами, в роды навсегда:
\par 13 Я полагаю радугу Мою в облаке, чтоб она была знамением завета между Мною и между землею.
\par 14 И будет, когда Я наведу облако на землю, то явится радуга в облаке;
\par 15 и Я вспомню завет Мой, который между Мною и между вами и между всякою душею живою во всякой плоти; и не будет более вода потопом на истребление всякой плоти.
\par 16 И будет радуга в облаке, и Я увижу ее, и вспомню завет вечный между Богом и между всякою душею живою во всякой плоти, которая на земле.
\par 17 И сказал Бог Ною: вот знамение завета, который Я поставил между Мною и между всякою плотью, которая на земле.
\par 18 Сыновья Ноя, вышедшие из ковчега, были: Сим, Хам и Иафет. Хам же был отец Ханаана.
\par 19 Сии трое были сыновья Ноевы, и от них населилась вся земля.
\par 20 Ной начал возделывать землю и насадил виноградник;
\par 21 и выпил он вина, и опьянел, и [лежал] обнаженным в шатре своем.
\par 22 И увидел Хам, отец Ханаана, наготу отца своего, и выйдя рассказал двум братьям своим.
\par 23 Сим же и Иафет взяли одежду и, положив ее на плечи свои, пошли задом и покрыли наготу отца своего; лица их были обращены назад, и они не видали наготы отца своего.
\par 24 Ной проспался от вина своего и узнал, что сделал над ним меньший сын его,
\par 25 и сказал: проклят Ханаан; раб рабов будет он у братьев своих.
\par 26 Потом сказал: благословен Господь Бог Симов; Ханаан же будет рабом ему;
\par 27 да распространит Бог Иафета, и да вселится он в шатрах Симовых; Ханаан же будет рабом ему.
\par 28 И жил Ной после потопа триста пятьдесят лет.
\par 29 Всех же дней Ноевых было девятьсот пятьдесят лет, и он умер.

\chapter{10}

\par 1 Вот родословие сынов Ноевых: Сима, Хама и Иафета. После потопа родились у них дети.
\par 2 Сыны Иафета: Гомер, Магог, Мадай, Иаван, Фувал, Мешех и Фирас.
\par 3 Сыны Гомера: Аскеназ, Рифат и Фогарма.
\par 4 Сыны Иавана: Елиса, Фарсис, Киттим и Доданим.
\par 5 От сих населились острова народов в землях их, каждый по языку своему, по племенам своим, в народах своих.
\par 6 Сыны Хама: Хуш, Мицраим, Фут и Ханаан.
\par 7 Сыны Хуша: Сева, Хавила, Савта, Раама и Савтеха. Сыны Раамы: Шева и дедан.
\par 8 Хуш родил также Нимрода: сей начал быть силен на земле.
\par 9 Он был сильный зверолов пред Господом; потому и говориться: сильный зверолов, как Нимрод, пред Господом.
\par 10 Царство его вначале сщставляли: Вавилон, Эрех, аккад и Халне, в земле Сеннаар.
\par 11 Из сей земли вышел Ассур, и построил Ниневию, Реховофир, Калах.
\par 12 И ресен между Ниневию и между Калахом; это город великий.
\par 13 От Мицраима произщшли Лудим, Анамим, Легавим, Нафтухим,
\par 14 Патрусим, Каслухим, откуда вышли Филистимляне, и Кафторим.
\par 15 От Ханаана родились: Сидон, первенец его, Хет,
\par 16 Иевусей, Аморей, Гергесей,
\par 17 Евей, Аркей, Синей,
\par 18 Арвадей, Цемарей и Химарей. В последствии племена Ханаанские рассеялись.
\par 19 И были пределы Хананеев от Сидона к Герару до Газы, Отсюда к Садому, Гаморре, Адме и Цевоиму до Лаши.
\par 20 Это сыны Хамовы, по племенам их, по языкам их, в землях их, в народах их.
\par 21 Были дети и у Сима, отца всех сынов Еверовых, старшего брата Иафетова.
\par 22 Сыны Сима: Елам, Асур, Арфаксад, Луд, Арам.
\par 23 Сыны Арама: Уц, Хул, Гефер и Маш.
\par 24 Арфаксад родил Салу, Сала родил Евера.
\par 25 У Евера родились два сына; имя одному: Фалек, потому что во дни его земля разделена; имя брата его: Иоктан.
\par 26 Иоктан родил Алмодада, Шалефа, Хацармавефа, Иераха,
\par 27 Гадорама, Узала, Диклу,
\par 28 Овала, Авимаила, Шеву,
\par 29 Офира, Хавилу и Иовава. Все эти сыновья Иоктана.
\par 30 Поселения их были от Меши до Сефара, горы восточной.
\par 31 Это сыновья Симовы по племенам их, по языкам их, в землях их, по народам их.
\par 32 Вот племена сынов Ноевых, по Родословию их, в народах их. От них распространились народы по земле после потопа.

\chapter{11}

\par 1 На всей земле был один язык и одно наречие.
\par 2 Двинувшись с востока, они нашли в земле Сеннаар равнину и поселились там.
\par 3 И сказали друг другу: наделаем кирпичей и обожжем огнем. И стали у них кирпичи вместо камней, а земляная смола вместо извести.
\par 4 И сказали они: построим себе город и башню, высотою до небес, и сделаем себе имя, прежде нежели рассеемся по лицу всей земли.
\par 5 И сошел Господь посмотреть город и башню, которые строили сыны человеческие.
\par 6 И сказал Господь: вот, один народ, и один у всех язык; и вот что начали они делать, и не отстанут они от того, что задумали делать;
\par 7 сойдем же и смешаем там язык их, так чтобы один не понимал речи другого.
\par 8 И рассеял их Господь оттуда по всей земле; и они перестали строить город.
\par 9 Посему дано ему имя: Вавилон, ибо там смешал Господь язык всей земли, и оттуда рассеял их Господь по всей земле.
\par 10 Вот родословие Сима: Сим был ста лет и родил Арфаксада, чрез два года после потопа;
\par 11 по рождении Арфаксада Сим жил пятьсот лет и родил сынов и дочерей.
\par 12 Арфаксад жил тридцать пять лет и родил Салу.
\par 13 По рождении Салы Арфаксад жил четыреста три года и родил сынов и дочерей.
\par 14 Сала жил тридцать лет и родил Евера.
\par 15 По рождении Евера Сала жил четыреста три года и родил сынов и дочерей.
\par 16 Евер жил тридцать четыре года и родил Фалека.
\par 17 По рождении Фалека Евер жил четыреста тридцать лет и родил сынов и дочерей.
\par 18 Фалек жил тридцать лет и родил Рагава.
\par 19 По рождении Рагава Фалек жил двести девять лет и родил сынов и дочерей.
\par 20 Рагав жил тридцать два года и родил Серуха.
\par 21 По рождении Серуха Рагав жил двести семь лет и родил сынов и дочерей.
\par 22 Серух жил тридцать лет и родил Нахора.
\par 23 По рождении Нахора Серух жил двести лет и родил сынов и дочерей.
\par 24 Нахор жил двадцать девять лет и родил Фарру.
\par 25 По рождении Фарры Нахор жил сто девятнадцать лет и родил сынов и дочерей.
\par 26 Фарра жил семьдесят лет и родил Аврама, Нахора и Арана.
\par 27 Вот родословие Фарры: Фарра родил Аврама, Нахора и Арана. Аран родил Лота.
\par 28 И умер Аран при Фарре, отце своем, в земле рождения своего, в Уре Халдейском.
\par 29 Аврам и Нахор взяли себе жен; имя жены Аврамовой: Сара; имя жены Нахоровой: Милка, дочь Арана, отца Милки и отца Иски.
\par 30 И Сара была неплодна и бездетна.
\par 31 И взял Фарра Аврама, сына своего, и Лота, сына Аранова, внука своего, и Сару, невестку свою, жену Аврама, сына своего, и вышел с ними из Ура Халдейского, чтобы идти в землю Ханаанскую; но, дойдя до Харрана, они остановились там.
\par 32 И было дней [жизни] Фарры двести пять лет, и умер Фарра в Харране.

\chapter{12}

\par 1 И сказал Господь Авраму: пойди из земли твоей, от родства твоего и из дома отца твоего, в землю, которую Я укажу тебе;
\par 2 и Я произведу от тебя великий народ, и благословлю тебя, и возвеличу имя твое, и будешь ты в благословение;
\par 3 Я благословлю благословляющих тебя, и злословящих тебя прокляну; и благословятся в тебе все племена земные.
\par 4 И пошел Аврам, как сказал ему Господь; и с ним пошел Лот. Аврам был семидесяти пяти лет, когда вышел из Харрана.
\par 5 И взял Аврам с собою Сару, жену свою, Лота, сына брата своего, и все имение, которое они приобрели, и всех людей, которых они имели в Харране; и вышли, чтобы идти в землю Ханаанскую; и пришли в землю Ханаанскую.
\par 6 И прошел Аврам по земле сей до места Сихема, до дубравы Море. В этой земле тогда [жили] Хананеи.
\par 7 И явился Господь Авраму и сказал: потомству твоему отдам Я землю сию. И создал [он] там жертвенник Господу, Который явился ему.
\par 8 Оттуда двинулся он к горе, на восток от Вефиля; и поставил шатер свой [так, что от него] Вефиль [был] на запад, а Гай на восток; и создал там жертвенник Господу и призвал имя Господа.
\par 9 И поднялся Аврам и продолжал идти к югу.
\par 10 И был голод в той земле. И сошел Аврам в Египет, пожить там, потому что усилился голод в земле той.
\par 11 Когда же он приближался к Египту, то сказал Саре, жене своей: вот, я знаю, что ты женщина, прекрасная видом;
\par 12 и когда Египтяне увидят тебя, то скажут: это жена его; и убьют меня, а тебя оставят в живых;
\par 13 скажи же, что ты мне сестра, дабы мне хорошо было ради тебя, и дабы жива была душа моя чрез тебя.
\par 14 И было, когда пришел Аврам в Египет, Египтяне увидели, что она женщина весьма красивая;
\par 15 увидели ее и вельможи фараоновы и похвалили ее фараону; и взята была она в дом фараонов.
\par 16 И Авраму хорошо было ради ее; и был у него мелкий и крупный скот и ослы, и рабы и рабыни, и лошаки и верблюды.
\par 17 Но Господь поразил тяжкими ударами фараона и дом его за Сару, жену Аврамову.
\par 18 И призвал фараон Аврама и сказал: что ты это сделал со мною? для чего не сказал мне, что она жена твоя?
\par 19 для чего ты сказал: она сестра моя? и я взял было ее себе в жену. И теперь вот жена твоя; возьми и пойди.
\par 20 И дал о нем фараон повеление людям, и проводили его, и жену его, и все, что у него было.

\chapter{13}

\par 1 И поднялся Аврам из Египта, сам и жена его, и все, что у него было, и Лот с ним, на юг.
\par 2 И был Аврам очень богат скотом, и серебром, и золотом.
\par 3 И продолжал он переходы свои от юга до Вефиля, до места, где прежде был шатер его между Вефилем и между Гаем,
\par 4 до места жертвенника, который он сделал там вначале; и там призвал Аврам имя Господа.
\par 5 И у Лота, который ходил с Аврамом, также был мелкий и крупный скот и шатры.
\par 6 И непоместительна была земля для них, чтобы жить вместе, ибо имущество их было так велико, что они не могли жить вместе.
\par 7 И был спор между пастухами скота Аврамова и между пастухами скота Лотова; и Хананеи и Ферезеи жили тогда в той земле.
\par 8 И сказал Аврам Лоту: да не будет раздора между мною и тобою, и между пастухами моими и пастухами твоими, ибо мы родственники;
\par 9 не вся ли земля пред тобою? отделись же от меня: если ты налево, то я направо; а если ты направо, то я налево.
\par 10 Лот возвел очи свои и увидел всю окрестность Иорданскую, что она, прежде нежели истребил Господь Содом и Гоморру, вся до Сигора орошалась водою, как сад Господень, как земля Египетская;
\par 11 и избрал себе Лот всю окрестность Иорданскую; и двинулся Лот к востоку. И отделились они друг от друга.
\par 12 Аврам стал жить на земле Ханаанской; а Лот стал жить в городах окрестности и раскинул шатры до Содома.
\par 13 Жители же Содомские были злы и весьма грешны пред Господом.
\par 14 И сказал Господь Авраму, после того как Лот отделился от него: возведи очи твои и с места, на котором ты теперь, посмотри к северу и к югу, и к востоку и к западу;
\par 15 ибо всю землю, которую ты видишь, тебе дам Я и потомству твоему навеки,
\par 16 и сделаю потомство твое, как песок земной; если кто может сосчитать песок земной, то и потомство твое сочтено будет;
\par 17 встань, пройди по земле сей в долготу и в широту ее, ибо Я тебе дам ее.
\par 18 И двинул Аврам шатер, и пошел, и поселился у дубравы Мамре, что в Хевроне; и создал там жертвенник Господу.

\chapter{14}

\par 1 И было во дни Амрафела, царя Сеннаарского, Ариоха, царя Елласарского, Кедорлаомера, царя Еламского, и Фидала, царя Гоимского,
\par 2 пошли они войною против Беры, царя Содомского, против Бирши, царя Гоморрского, Шинава, царя Адмы, Шемевера, царя Севоимского, и против царя Белы, которая есть Сигор.
\par 3 Все сии соединились в долине Сиддим, где [ныне] море Соленое.
\par 4 Двенадцать лет были они в порабощении у Кедорлаомера, а в тринадцатом году возмутились.
\par 5 В четырнадцатом году пришел Кедорлаомер и цари, которые с ним, и поразили Рефаимов в Аштероф-Карнаиме, Зузимов в Гаме, Эмимов в Шаве-Кириафаиме,
\par 6 и Хорреев в горе их Сеире, до Эл-Фарана, что при пустыне.
\par 7 И возвратившись оттуда, они пришли к источнику Мишпат, который есть Кадес, и поразили всю страну Амаликитян, и также Аморреев, живущих в Хацацон-Фамаре.
\par 8 И вышли царь Содомский, царь Гоморрский, царь Адмы, царь Севоимский и царь Белы, которая есть Сигор; и вступили в сражение с ними в долине Сиддим,
\par 9 с Кедорлаомером, царем Еламским, Фидалом, царем Гоимским, Амрафелом, царем Сеннаарским, Ариохом, царем Елласарским, --четыре царя против пяти.
\par 10 В долине же Сиддим было много смоляных ям. И цари Содомский и Гоморрский, обратившись в бегство, упали в них, а остальные убежали в горы.
\par 11 [Победители] взяли все имущество Содома и Гоморры и весь запас их и ушли.
\par 12 И взяли Лота, племянника Аврамова, жившего в Содоме, и имущество его и ушли.
\par 13 И пришел один из уцелевших и известил Аврама Еврея, жившего тогда у дубравы Мамре, Аморреянина, брата Эшколу и брата Анеру, которые были союзники Аврамовы.
\par 14 Аврам, услышав, что сродник его взят в плен, вооружил рабов своих, рожденных в доме его, триста восемнадцать, и преследовал [неприятелей] до Дана;
\par 15 и, разделившись, [напал] на них ночью, сам и рабы его, и поразил их, и преследовал их до Ховы, что по левую сторону Дамаска;
\par 16 и возвратил все имущество и Лота, сродника своего, и имущество его возвратил, также и женщин и народ.
\par 17 Когда он возвращался после поражения Кедорлаомера и царей, бывших с ним, царь Содомский вышел ему навстречу в долину Шаве, что [ныне] долина царская;
\par 18 и Мелхиседек, царь Салимский, вынес хлеб и вино, --он был священник Бога Всевышнего, --
\par 19 и благословил его, и сказал: благословен Аврам от Бога Всевышнего, Владыки неба и земли;
\par 20 и благословен Бог Всевышний, Который предал врагов твоих в руки твои. [Аврам] дал ему десятую часть из всего.
\par 21 И сказал царь Содомский Авраму: отдай мне людей, а имение возьми себе.
\par 22 Но Аврам сказал царю Содомскому: поднимаю руку мою к Господу Богу Всевышнему, Владыке неба и земли,
\par 23 что даже нитки и ремня от обуви не возьму из всего твоего, чтобы ты не сказал: я обогатил Аврама;
\par 24 кроме того, что съели отроки, и кроме доли, принадлежащей людям, которые ходили со мною; Анер, Эшкол и Мамрий пусть возьмут свою долю.

\chapter{15}

\par 1 После сих происшествий было слово Господа к Авраму в видении, и сказано: не бойся, Аврам; Я твой щит; награда твоя весьма велика.
\par 2 Аврам сказал: Владыка Господи! что Ты дашь мне? я остаюсь бездетным; распорядитель в доме моем этот Елиезер из Дамаска.
\par 3 И сказал Аврам: вот, Ты не дал мне потомства, и вот, домочадец мой наследник мой.
\par 4 И было слово Господа к нему, и сказано: не будет он твоим наследником, но тот, кто произойдет из чресл твоих, будет твоим наследником.
\par 5 И вывел его вон и сказал: посмотри на небо и сосчитай звезды, если ты можешь счесть их. И сказал ему: столько будет у тебя потомков.
\par 6 Аврам поверил Господу, и Он вменил ему это в праведность.
\par 7 И сказал ему: Я Господь, Который вывел тебя из Ура Халдейского, чтобы дать тебе землю сию во владение.
\par 8 Он сказал: Владыка Господи! по чему мне узнать, что я буду владеть ею?
\par 9 [Господь] сказал ему: возьми Мне трехлетнюю телицу, трехлетнюю козу, трехлетнего овна, горлицу и молодого голубя.
\par 10 Он взял всех их, рассек их пополам и положил одну часть против другой; только птиц не рассек.
\par 11 И налетели на трупы хищные птицы; но Аврам отгонял их.
\par 12 При захождении солнца крепкий сон напал на Аврама, и вот, напал на него ужас и мрак великий.
\par 13 И сказал [Господь] Авраму: знай, что потомки твои будут пришельцами в земле не своей, и поработят их, и будут угнетать их четыреста лет,
\par 14 но Я произведу суд над народом, у которого они будут в порабощении; после сего они выйдут с большим имуществом,
\par 15 а ты отойдешь к отцам твоим в мире [и] будешь погребен в старости доброй;
\par 16 в четвертом роде возвратятся они сюда: ибо [мера] беззаконий Аморреев доселе еще не наполнилась.
\par 17 Когда зашло солнце и наступила тьма, вот, дым [как бы из] печи и пламя огня прошли между рассеченными [животными].
\par 18 В этот день заключил Господь завет с Аврамом, сказав: потомству твоему даю Я землю сию, от реки Египетской до великой реки, реки Евфрата:
\par 19 Кенеев, Кенезеев, Кедмонеев,
\par 20 Хеттеев, Ферезеев, Рефаимов,
\par 21 Аморреев, Хананеев, Гергесеев и Иевусеев.

\chapter{16}

\par 1 Но Сара, жена Аврамова, не рождала ему. У ней была служанка Египтянка, именем Агарь.
\par 2 И сказала Сара Авраму: вот, Господь заключил чрево мое, чтобы мне не рождать; войди же к служанке моей: может быть, я буду иметь детей от нее. Аврам послушался слов Сары.
\par 3 И взяла Сара, жена Аврамова, служанку свою, Египтянку Агарь, по истечении десяти лет пребывания Аврамова в земле Ханаанской, и дала ее Авраму, мужу своему, в жену.
\par 4 Он вошел к Агари, и она зачала. Увидев же, что зачала, она стала презирать госпожу свою.
\par 5 И сказала Сара Авраму: в обиде моей ты виновен; я отдала служанку мою в недро твое; а она, увидев, что зачала, стала презирать меня; Господь пусть будет судьею между мною и между тобою.
\par 6 Аврам сказал Саре: вот, служанка твоя в твоих руках; делай с нею, что тебе угодно. И Сара стала притеснять ее, и она убежала от нее.
\par 7 И нашел ее Ангел Господень у источника воды в пустыне, у источника на дороге к Суру.
\par 8 И сказал ей: Агарь, служанка Сарина! откуда ты пришла и куда идешь? Она сказала: я бегу от лица Сары, госпожи моей.
\par 9 Ангел Господень сказал ей: возвратись к госпоже своей и покорись ей.
\par 10 И сказал ей Ангел Господень: умножая умножу потомство твое, так что нельзя будет и счесть его от множества.
\par 11 И еще сказал ей Ангел Господень: вот, ты беременна, и родишь сына, и наречешь ему имя Измаил, ибо услышал Господь страдание твое;
\par 12 он будет [между] людьми, [как] дикий осел; руки его на всех, и руки всех на него; жить будет он пред лицем всех братьев своих.
\par 13 И нарекла [Агарь] Господа, Который говорил к ней, [сим] именем: Ты Бог видящий меня. Ибо сказала она: точно я видела здесь в след видящего меня.
\par 14 Посему источник [тот] называется: Беэр-лахай-рои. Он находится между Кадесом и между Баредом.
\par 15 Агарь родила Авраму сына; и нарек [Аврам] имя сыну своему, рожденному от Агари: Измаил.
\par 16 Аврам был восьмидесяти шести лет, когда Агарь родила Авраму Измаила.

\chapter{17}

\par 1 Аврам был девяноста девяти лет, и Господь явился Авраму и сказал ему: Я Бог Всемогущий; ходи предо Мною и будь непорочен;
\par 2 и поставлю завет Мой между Мною и тобою, и весьма, весьма размножу тебя.
\par 3 И пал Аврам на лице свое. Бог продолжал говорить с ним и сказал:
\par 4 Я--вот завет Мой с тобою: ты будешь отцом множества народов,
\par 5 и не будешь ты больше называться Аврамом, но будет тебе имя: Авраам, ибо Я сделаю тебя отцом множества народов;
\par 6 и весьма, весьма распложу тебя, и произведу от тебя народы, и цари произойдут от тебя;
\par 7 и поставлю завет Мой между Мною и тобою и между потомками твоими после тебя в роды их, завет вечный в том, что Я буду Богом твоим и потомков твоих после тебя;
\par 8 и дам тебе и потомкам твоим после тебя землю, по которой ты странствуешь, всю землю Ханаанскую, во владение вечное; и буду им Богом.
\par 9 И сказал Бог Аврааму: ты же соблюди завет Мой, ты и потомки твои после тебя в роды их.
\par 10 Сей есть завет Мой, который вы [должны] соблюдать между Мною и между вами и между потомками твоими после тебя: да будет у вас обрезан весь мужеский пол;
\par 11 обрезывайте крайнюю плоть вашу: и сие будет знамением завета между Мною и вами.
\par 12 Восьми дней от рождения да будет обрезан у вас в роды ваши всякий [младенец] мужеского пола, рожденный в доме и купленный за серебро у какого-нибудь иноплеменника, который не от твоего семени.
\par 13 Непременно да будет обрезан рожденный в доме твоем и купленный за серебро твое, и будет завет Мой на теле вашем заветом вечным.
\par 14 Необрезанный же мужеского пола, который не обрежет крайней плоти своей, истребится душа та из народа своего, [ибо] он нарушил завет Мой.
\par 15 И сказал Бог Аврааму: Сару, жену твою, не называй Сарою, но да будет имя ей: Сарра;
\par 16 Я благословлю ее и дам тебе от нее сына; благословлю ее, и произойдут от нее народы, и цари народов произойдут от нее.
\par 17 И пал Авраам на лице свое, и рассмеялся, и сказал сам в себе: неужели от столетнего будет сын? и Сарра, девяностолетняя, неужели родит?
\par 18 И сказал Авраам Богу: о, хотя бы Измаил был жив пред лицем Твоим!
\par 19 Бог же сказал: именно Сарра, жена твоя, родит тебе сына, и ты наречешь ему имя: Исаак; и поставлю завет Мой с ним заветом вечным [и] потомству его после него.
\par 20 И о Измаиле Я услышал тебя: вот, Я благословлю его, и возращу его, и весьма, весьма размножу; двенадцать князей родятся от него; и Я произведу от него великий народ.
\par 21 Но завет Мой поставлю с Исааком, которого родит тебе Сарра в сие самое время на другой год.
\par 22 И Бог перестал говорить с Авраамом и восшел от него.
\par 23 И взял Авраам Измаила, сына своего, и всех рожденных в доме своем и всех купленных за серебро свое, весь мужеский пол людей дома Авраамова; и обрезал крайнюю плоть их в тот самый день, как сказал ему Бог.
\par 24 Авраам был девяноста девяти лет, когда была обрезана крайняя плоть его.
\par 25 А Измаил, сын его, был тринадцати лет, когда была обрезана крайняя плоть его.
\par 26 В тот же самый день обрезаны были Авраам и Измаил, сын его,
\par 27 и с ним обрезан был весь мужеский пол дома его, рожденные в доме и купленные за серебро у иноплеменников.

\chapter{18}

\par 1 И явился ему Господь у дубравы Мамре, когда он сидел при входе в шатер, во время зноя дневного.
\par 2 Он возвел очи свои и взглянул, и вот, три мужа стоят против него. Увидев, он побежал навстречу им от входа в шатер и поклонился до земли,
\par 3 и сказал: Владыка! если я обрел благоволение пред очами Твоими, не пройди мимо раба Твоего;
\par 4 и принесут немного воды, и омоют ноги ваши; и отдохните под сим деревом,
\par 5 а я принесу хлеба, и вы подкрепите сердца ваши; потом пойдите; так как вы идете мимо раба вашего. Они сказали: сделай так, как говоришь.
\par 6 И поспешил Авраам в шатер к Сарре и сказал: поскорее замеси три саты лучшей муки и сделай пресные хлебы.
\par 7 И побежал Авраам к стаду, и взял теленка нежного и хорошего, и дал отроку, и тот поспешил приготовить его.
\par 8 И взял масла и молока и теленка приготовленного, и поставил перед ними, а сам стоял подле них под деревом. И они ели.
\par 9 И сказали ему: где Сарра, жена твоя? Он отвечал: здесь, в шатре.
\par 10 И сказал [один из них]: Я опять буду у тебя в это же время, и будет сын у Сарры, жены твоей. А Сарра слушала у входа в шатер, сзади его.
\par 11 Авраам же и Сарра были стары и в летах преклонных, и обыкновенное у женщин у Сарры прекратилось.
\par 12 Сарра внутренно рассмеялась, сказав: мне ли, когда я состарилась, иметь сие утешение? и господин мой стар.
\par 13 И сказал Господь Аврааму: отчего это рассмеялась Сарра, сказав: `неужели я действительно могу родить, когда я состарилась'?
\par 14 Есть ли что трудное для Господа? В назначенный срок буду Я у тебя в следующем году, и у Сарры [будет] сын.
\par 15 Сарра же не призналась, а сказала: я не смеялась. Ибо она испугалась. Но Он сказал: нет, ты рассмеялась.
\par 16 И встали те мужи и оттуда отправились к Содому; Авраам же пошел с ними, проводить их.
\par 17 И сказал Господь: утаю ли Я от Авраама, что хочу делать!
\par 18 От Авраама точно произойдет народ великий и сильный, и благословятся в нем все народы земли,
\par 19 ибо Я избрал его для того, чтобы он заповедал сынам своим и дому своему после себя, ходить путем Господним, творя правду и суд; и исполнит Господь над Авраамом, что сказал о нем.
\par 20 И сказал Господь: вопль Содомский и Гоморрский, велик он, и грех их, тяжел он весьма;
\par 21 сойду и посмотрю, точно ли они поступают так, каков вопль на них, восходящий ко Мне, или нет; узнаю.
\par 22 И обратились мужи оттуда и пошли в Содом; Авраам же еще стоял пред лицем Господа.
\par 23 И подошел Авраам и сказал: неужели Ты погубишь праведного с нечестивым?
\par 24 может быть, есть в этом городе пятьдесят праведников? неужели Ты погубишь, и не пощадишь места сего ради пятидесяти праведников, в нем?
\par 25 не может быть, чтобы Ты поступил так, чтобы Ты погубил праведного с нечестивым, чтобы то же было с праведником, что с нечестивым; не может быть от Тебя! Судия всей земли поступит ли неправосудно?
\par 26 Господь сказал: если Я найду в городе Содоме пятьдесят праведников, то Я ради них пощажу все место сие.
\par 27 Авраам сказал в ответ: вот, я решился говорить Владыке, я, прах и пепел:
\par 28 может быть, до пятидесяти праведников недостанет пяти, неужели за [недостатком] пяти Ты истребишь весь город? Он сказал: не истреблю, если найду там сорок пять.
\par 29 [Авраам] продолжал говорить с Ним и сказал: может быть, найдется там сорок? Он сказал: не сделаю [того] и ради сорока.
\par 30 И сказал [Авраам]: да не прогневается Владыка, что я буду говорить: может быть, найдется там тридцать? Он сказал: не сделаю, если найдется там тридцать.
\par 31 [Авраам] сказал: вот, я решился говорить Владыке: может быть, найдется там двадцать? Он сказал: не истреблю ради двадцати.
\par 32 [Авраам] сказал: да не прогневается Владыка, что я скажу еще однажды: может быть, найдется там десять? Он сказал: не истреблю ради десяти.
\par 33 И пошел Господь, перестав говорить с Авраамом; Авраам же возвратился в свое место.

\chapter{19}

\par 1 И пришли те два Ангела в Содом вечером, когда Лот сидел у ворот Содома. Лот увидел, и встал, чтобы встретить их, и поклонился лицем до земли
\par 2 и сказал: государи мои! зайдите в дом раба вашего и ночуйте, и умойте ноги ваши, и встаньте поутру и пойдете в путь свой. Но они сказали: нет, мы ночуем на улице.
\par 3 Он же сильно упрашивал их; и они пошли к нему и пришли в дом его. Он сделал им угощение и испек пресные хлебы, и они ели.
\par 4 Еще не легли они спать, как городские жители, Содомляне, от молодого до старого, весь народ со [всех] концов [города], окружили дом
\par 5 и вызвали Лота и говорили ему: где люди, пришедшие к тебе на ночь? выведи их к нам; мы познаем их.
\par 6 Лот вышел к ним ко входу, и запер за собою дверь,
\par 7 и сказал: братья мои, не делайте зла;
\par 8 вот у меня две дочери, которые не познали мужа; лучше я выведу их к вам, делайте с ними, что вам угодно, только людям сим не делайте ничего, так как они пришли под кров дома моего.
\par 9 Но они сказали: пойди сюда. И сказали: вот пришлец, и хочет судить? теперь мы хуже поступим с тобою, нежели с ними. И очень приступали к человеку сему, к Лоту, и подошли, чтобы выломать дверь.
\par 10 Тогда мужи те простерли руки свои и ввели Лота к себе в дом, и дверь заперли;
\par 11 а людей, бывших при входе в дом, поразили слепотою, от малого до большого, так что они измучились, искав входа.
\par 12 Сказали мужи те Лоту: кто у тебя есть еще здесь? зять ли, сыновья ли твои, дочери ли твои, и кто бы ни был у тебя в городе, всех выведи из сего места,
\par 13 ибо мы истребим сие место, потому что велик вопль на жителей его к Господу, и Господь послал нас истребить его.
\par 14 И вышел Лот, и говорил с зятьями своими, которые брали за себя дочерей его, и сказал: встаньте, выйдите из сего места, ибо Господь истребит сей город. Но зятьям его показалось, что он шутит.
\par 15 Когда взошла заря, Ангелы начали торопить Лота, говоря: встань, возьми жену твою и двух дочерей твоих, которые у тебя, чтобы не погибнуть тебе за беззакония города.
\par 16 И как он медлил, то мужи те, по милости к нему Господней, взяли за руку его и жену его, и двух дочерей его, и вывели его и поставили его вне города.
\par 17 Когда же вывели их вон, [то один из них] сказал: спасай душу свою; не оглядывайся назад и нигде не останавливайся в окрестности сей; спасайся на гору, чтобы тебе не погибнуть.
\par 18 Но Лот сказал им: нет, Владыка!
\par 19 вот, раб Твой обрел благоволение пред очами Твоими, и велика милость Твоя, которую Ты сделал со мною, что спас жизнь мою; но я не могу спасаться на гору, чтоб не застигла меня беда и мне не умереть;
\par 20 вот, ближе бежать в сей город, он же мал; побегу я туда, --он же мал; и сохранится жизнь моя.
\par 21 И сказал ему: вот, в угодность тебе Я сделаю и это: не ниспровергну города, о котором ты говоришь;
\par 22 поспешай, спасайся туда, ибо Я не могу сделать дела, доколе ты не придешь туда. Потому и назван город сей: Сигор.
\par 23 Солнце взошло над землею, и Лот пришел в Сигор.
\par 24 И пролил Господь на Содом и Гоморру дождем серу и огонь от Господа с неба,
\par 25 и ниспроверг города сии, и всю окрестность сию, и всех жителей городов сих, и произрастания земли.
\par 26 Жена же [Лотова] оглянулась позади его, и стала соляным столпом.
\par 27 И встал Авраам рано утром и [пошел] на место, где стоял пред лицем Господа,
\par 28 и посмотрел к Содому и Гоморре и на все пространство окрестности и увидел: вот, дым поднимается с земли, как дым из печи.
\par 29 И было, когда Бог истреблял города окрестности сей, вспомнил Бог об Аврааме и выслал Лота из среды истребления, когда ниспровергал города, в которых жил Лот.
\par 30 И вышел Лот из Сигора и стал жить в горе, и с ним две дочери его, ибо он боялся жить в Сигоре. И жил в пещере, и с ним две дочери его.
\par 31 И сказала старшая младшей: отец наш стар, и нет человека на земле, который вошел бы к нам по обычаю всей земли;
\par 32 итак напоим отца нашего вином, и переспим с ним, и восставим от отца нашего племя.
\par 33 И напоили отца своего вином в ту ночь; и вошла старшая и спала с отцом своим: а он не знал, когда она легла и когда встала.
\par 34 На другой день старшая сказала младшей: вот, я спала вчера с отцом моим; напоим его вином и в эту ночь; и ты войди, спи с ним, и восставим от отца нашего племя.
\par 35 И напоили отца своего вином и в эту ночь; и вошла младшая и спала с ним; и он не знал, когда она легла и когда встала.
\par 36 И сделались обе дочери Лотовы беременными от отца своего,
\par 37 и родила старшая сына, и нарекла ему имя: Моав. Он отец Моавитян доныне.
\par 38 И младшая также родила сына, и нарекла ему имя: Бен-Амми. Он отец Аммонитян доныне.

\chapter{20}

\par 1 Авраам поднялся оттуда к югу и поселился между Кадесом и между Суром; и был на время в Гераре.
\par 2 И сказал Авраам о Сарре, жене своей: она сестра моя. И послал Авимелех, царь Герарский, и взял Сарру.
\par 3 И пришел Бог к Авимелеху ночью во сне и сказал ему: вот, ты умрешь за женщину, которую ты взял, ибо она имеет мужа.
\par 4 Авимелех же не прикасался к ней и сказал: Владыка! неужели ты погубишь и невинный народ?
\par 5 Не сам ли он сказал мне: она сестра моя? И она сама сказала: он брат мой. Я сделал это в простоте сердца моего и в чистоте рук моих.
\par 6 И сказал ему Бог во сне: и Я знаю, что ты сделал сие в простоте сердца твоего, и удержал тебя от греха предо Мною, потому и не допустил тебя прикоснуться к ней;
\par 7 теперь же возврати жену мужу, ибо он пророк и помолится о тебе, и ты будешь жив; а если не возвратишь, то знай, что непременно умрешь ты и все твои.
\par 8 И встал Авимелех утром рано, и призвал всех рабов своих, и пересказал все слова сии в уши их; и люди сии весьма испугались.
\par 9 И призвал Авимелех Авраама и сказал ему: что ты с нами сделал? чем согрешил я против тебя, что ты навел было на меня и на царство мое великий грех? Ты сделал со мною дела, каких не делают.
\par 10 И сказал Авимелех Аврааму: что ты имел в виду, когда делал это дело?
\par 11 Авраам сказал: я подумал, что нет на месте сем страха Божия, и убьют меня за жену мою;
\par 12 да она и подлинно сестра мне: она дочь отца моего, только не дочь матери моей; и сделалась моею женою;
\par 13 когда Бог повел меня странствовать из дома отца моего, то я сказал ей: сделай со мною сию милость, в какое ни придем мы место, везде говори обо мне: это брат мой.
\par 14 И взял Авимелех мелкого и крупного скота, и рабов и рабынь, и дал Аврааму; и возвратил ему Сарру, жену его.
\par 15 И сказал Авимелех: вот, земля моя пред тобою; живи, где тебе угодно.
\par 16 И Сарре сказал: вот, я дал брату твоему тысячу [сиклей] серебра; вот, это тебе покрывало для очей пред всеми, которые с тобою, и пред всеми ты оправдана.
\par 17 И помолился Авраам Богу, и исцелил Бог Авимелеха, и жену его, и рабынь его, и они стали рождать;
\par 18 ибо заключил Господь всякое чрево в доме Авимелеха за Сарру, жену Авраамову.

\chapter{21}

\par 1 И призрел Господь на Сарру, как сказал; и сделал Господь Сарре, как говорил.
\par 2 Сарра зачала и родила Аврааму сына в старости его во время, о котором говорил ему Бог;
\par 3 и нарек Авраам имя сыну своему, родившемуся у него, которого родила ему Сарра, Исаак;
\par 4 и обрезал Авраам Исаака, сына своего, в восьмой день, как заповедал ему Бог.
\par 5 Авраам был ста лет, когда родился у него Исаак, сын его.
\par 6 И сказала Сарра: смех сделал мне Бог; кто ни услышит обо мне, рассмеется.
\par 7 И сказала: кто сказал бы Аврааму: Сарра будет кормить детей грудью? ибо в старости его я родила сына.
\par 8 Дитя выросло и отнято от груди; и Авраам сделал большой пир в тот день, когда Исаак отнят был от груди.
\par 9 И увидела Сарра, что сын Агари Египтянки, которого она родила Аврааму, насмехается,
\par 10 и сказала Аврааму: выгони эту рабыню и сына ее, ибо не наследует сын рабыни сей с сыном моим Исааком.
\par 11 И показалось это Аврааму весьма неприятным ради сына его.
\par 12 Но Бог сказал Аврааму: не огорчайся ради отрока и рабыни твоей; во всем, что скажет тебе Сарра, слушайся голоса ее, ибо в Исааке наречется тебе семя;
\par 13 и от сына рабыни Я произведу народ, потому что он семя твое.
\par 14 Авраам встал рано утром, и взял хлеба и мех воды, и дал Агари, положив ей на плечи, и отрока, и отпустил ее. Она пошла, и заблудилась в пустыне Вирсавии;
\par 15 и не стало воды в мехе, и она оставила отрока под одним кустом
\par 16 и пошла, села вдали, в расстоянии на [один] выстрел из лука. Ибо она сказала: не [хочу] видеть смерти отрока. И она села против, и подняла вопль, и плакала;
\par 17 и услышал Бог голос отрока; и Ангел Божий с неба воззвал к Агари и сказал ей: что с тобою, Агарь? не бойся; Бог услышал голос отрока оттуда, где он находится;
\par 18 встань, подними отрока и возьми его за руку, ибо Я произведу от него великий народ.
\par 19 И Бог открыл глаза ее, и она увидела колодезь с водою, и пошла, наполнила мех водою и напоила отрока.
\par 20 И Бог был с отроком; и он вырос, и стал жить в пустыне, и сделался стрелком из лука.
\par 21 Он жил в пустыне Фаран; и мать его взяла ему жену из земли Египетской.
\par 22 И было в то время, Авимелех с Фихолом, военачальником своим, сказал Аврааму: с тобою Бог во всем, что ты ни делаешь;
\par 23 и теперь поклянись мне здесь Богом, что ты не обидишь ни меня, ни сына моего, ни внука моего; и как я хорошо поступал с тобою, так и ты будешь поступать со мною и землею, в которой ты гостишь.
\par 24 И сказал Авраам: я клянусь.
\par 25 И Авраам упрекал Авимелеха за колодезь с водою, который отняли рабы Авимелеховы.
\par 26 Авимелех же сказал: не знаю, кто это сделал, и ты не сказал мне; я даже и не слыхал [о том] доныне.
\par 27 И взял Авраам мелкого и крупного скота и дал Авимелеху, и они оба заключили союз.
\par 28 И поставил Авраам семь агниц из [стада] мелкого скота особо.
\par 29 Авимелех же сказал Аврааму: на что здесь сии семь агниц, которых ты поставил особо?
\par 30 [он] сказал: семь агниц сих возьми от руки моей, чтобы они были мне свидетельством, что я выкопал этот колодезь.
\par 31 Потому и назвал он сие место: Вирсавия, ибо тут оба они клялись
\par 32 и заключили союз в Вирсавии. И встал Авимелех, и Фихол, военачальник его, и возвратились в землю Филистимскую.
\par 33 И насадил [Авраам] при Вирсавии рощу и призвал там имя Господа, Бога вечного.
\par 34 И жил Авраам в земле Филистимской, как странник, дни многие.

\chapter{22}

\par 1 И было, после сих происшествий Бог искушал Авраама и сказал ему: Авраам! Он сказал: вот я.
\par 2 [Бог] сказал: возьми сына твоего, единственного твоего, которого ты любишь, Исаака; и пойди в землю Мориа и там принеси его во всесожжение на одной из гор, о которой Я скажу тебе.
\par 3 Авраам встал рано утром, оседлал осла своего, взял с собою двоих из отроков своих и Исаака, сына своего; наколол дров для всесожжения, и встав пошел на место, о котором сказал ему Бог.
\par 4 На третий день Авраам возвел очи свои, и увидел то место издалека.
\par 5 И сказал Авраам отрокам своим: останьтесь вы здесь с ослом, а я и сын пойдем туда и поклонимся, и возвратимся к вам.
\par 6 И взял Авраам дрова для всесожжения, и возложил на Исаака, сына своего; взял в руки огонь и нож, и пошли оба вместе.
\par 7 И начал Исаак говорить Аврааму, отцу своему, и сказал: отец мой! Он отвечал: вот я, сын мой. Он сказал: вот огонь и дрова, где же агнец для всесожжения?
\par 8 Авраам сказал: Бог усмотрит Себе агнца для всесожжения, сын мой. И шли [далее] оба вместе.
\par 9 И пришли на место, о котором сказал ему Бог; и устроил там Авраам жертвенник, разложил дрова и, связав сына своего Исаака, положил его на жертвенник поверх дров.
\par 10 И простер Авраам руку свою и взял нож, чтобы заколоть сына своего.
\par 11 Но Ангел Господень воззвал к нему с неба и сказал: Авраам! Авраам! Он сказал: вот я.
\par 12 [Ангел] сказал: не поднимай руки твоей на отрока и не делай над ним ничего, ибо теперь Я знаю, что боишься ты Бога и не пожалел сына твоего, единственного твоего, для Меня.
\par 13 И возвел Авраам очи свои и увидел: и вот, позади овен, запутавшийся в чаще рогами своими. Авраам пошел, взял овна и принес его во всесожжение вместо сына своего.
\par 14 И нарек Авраам имя месту тому: Иегова-ире. Посему [и] ныне говорится: на горе Иеговы усмотрится.
\par 15 И вторично воззвал к Аврааму Ангел Господень с неба
\par 16 и сказал: Мною клянусь, говорит Господь, что, так как ты сделал сие дело, и не пожалел сына твоего, единственного твоего,
\par 17 то Я благословляя благословлю тебя и умножая умножу семя твое, как звезды небесные и как песок на берегу моря; и овладеет семя твое городами врагов своих;
\par 18 и благословятся в семени твоем все народы земли за то, что ты послушался гласа Моего.
\par 19 И возвратился Авраам к отрокам своим, и встали и пошли вместе в Вирсавию; и жил Авраам в Вирсавии.
\par 20 После сих происшествий Аврааму возвестили, сказав: вот, и Милка родила Нахору, брату твоему, сынов:
\par 21 Уца, первенца его, Вуза, брата сему, Кемуила, отца Арамова,
\par 22 Кеседа, Хазо, Пилдаша, Идлафа и Вафуила;
\par 23 от Вафуила родилась Ревекка. Восьмерых сих родила Милка Нахору, брату Авраамову;
\par 24 и наложница его, именем Реума, также родила Теваха, Гахама, Тахаша и Мааху.

\chapter{23}

\par 1 Жизни Сарриной было сто двадцать семь лет: [вот] лета жизни Сарриной;
\par 2 и умерла Сарра в Кириаф-Арбе, что [ныне] Хеврон, в земле Ханаанской. И пришел Авраам рыдать по Сарре и оплакивать ее.
\par 3 И отошел Авраам от умершей своей, и говорил сынам Хетовым, и сказал:
\par 4 я у вас пришлец и поселенец; дайте мне в собственность [место] [для] гроба между вами, чтобы мне умершую мою схоронить от глаз моих.
\par 5 Сыны Хета отвечали Аврааму и сказали ему:
\par 6 послушай нас, господин наш; ты князь Божий посреди нас; в лучшем из наших погребальных мест похорони умершую твою; никто из нас не откажет тебе в погребальном месте, для погребения умершей твоей.
\par 7 Авраам встал и поклонился народу земли той, сынам Хетовым;
\par 8 и говорил им и сказал: если вы согласны, чтобы я похоронил умершую мою, то послушайте меня, попросите за меня Ефрона, сына Цохарова,
\par 9 чтобы он отдал мне пещеру Махпелу, которая у него на конце поля его, чтобы за довольную цену отдал ее мне посреди вас, в собственность для погребения.
\par 10 Ефрон же сидел посреди сынов Хетовых; и отвечал Ефрон Хеттеянин Аврааму вслух сынов Хета, всех входящих во врата города его, и сказал:
\par 11 нет, господин мой, послушай меня: я даю тебе поле и пещеру, которая на нем, даю тебе, пред очами сынов народа моего дарю тебе ее, похорони умершую твою.
\par 12 Авраам поклонился пред народом земли той
\par 13 и говорил Ефрону вслух народа земли той и сказал: если послушаешь, я даю тебе за поле серебро; возьми у меня, и я похороню там умершую мою.
\par 14 Ефрон отвечал Аврааму и сказал ему:
\par 15 господин мой! послушай меня: земля [стоит] четыреста сиклей серебра; для меня и для тебя что это? похорони умершую твою.
\par 16 Авраам выслушал Ефрона; и отвесил Авраам Ефрону серебра, сколько он объявил вслух сынов Хетовых, четыреста сиклей серебра, какое ходит у купцов.
\par 17 И стало поле Ефроново, которое при Махпеле, против Мамре, поле и пещера, которая на нем, и все деревья, которые на поле, во всех пределах его вокруг,
\par 18 владением Авраамовым пред очами сынов Хета, всех входящих во врата города его.
\par 19 После сего Авраам похоронил Сарру, жену свою, в пещере поля в Махпеле, против Мамре, что [ныне] Хеврон, в земле Ханаанской.
\par 20 Так достались Аврааму от сынов Хетовых поле и пещера, которая на нем, в собственность для погребения.

\chapter{24}

\par 1 Авраам был уже стар и в летах преклонных. Господь благословил Авраама всем.
\par 2 И сказал Авраам рабу своему, старшему в доме его, управлявшему всем, что у него было: положи руку твою под стегно мое
\par 3 и клянись мне Господом, Богом неба и Богом земли, что ты не возьмешь сыну моему жены из дочерей Хананеев, среди которых я живу,
\par 4 но пойдешь в землю мою, на родину мою, и возьмешь жену сыну моему Исааку.
\par 5 Раб сказал ему: может быть, не захочет женщина идти со мною в эту землю, должен ли я возвратить сына твоего в землю, из которой ты вышел?
\par 6 Авраам сказал ему: берегись, не возвращай сына моего туда;
\par 7 Господь, Бог неба, Который взял меня из дома отца моего и из земли рождения моего, Который говорил мне и Который клялся мне, говоря: `потомству твоему дам сию землю', --Он пошлет Ангела Своего пред тобою, и ты возьмешь жену сыну моему оттуда;
\par 8 если же не захочет женщина идти с тобою, ты будешь свободен от сей клятвы моей; только сына моего не возвращай туда.
\par 9 И положил раб руку свою под стегно Авраама, господина своего, и клялся ему в сем.
\par 10 И взял раб из верблюдов господина своего десять верблюдов и пошел. В руках у него были также всякие сокровища господина его. Он встал и пошел в Месопотамию, в город Нахора,
\par 11 и остановил верблюдов вне города, у колодезя воды, под вечер, в то время, когда выходят женщины черпать,
\par 12 и сказал: Господи, Боже господина моего Авраама! пошли [ее] сегодня навстречу мне и сотвори милость с господином моим Авраамом;
\par 13 вот, я стою у источника воды, и дочери жителей города выходят черпать воду;
\par 14 и девица, которой я скажу: `наклони кувшин твой, я напьюсь', и которая скажет: `пей, я и верблюдам твоим дам пить', --вот та, которую Ты назначил рабу Твоему Исааку; и по сему узнаю я, что Ты творишь милость с господином моим.
\par 15 Еще не перестал он говорить, и вот, вышла Ревекка, которая родилась от Вафуила, сына Милки, жены Нахора, брата Авраамова, и кувшин ее на плече ее;
\par 16 девица [была] прекрасна видом, дева, которой не познал муж. Она сошла к источнику, наполнила кувшин свой и пошла вверх.
\par 17 И побежал раб навстречу ей и сказал: дай мне испить немного воды из кувшина твоего.
\par 18 Она сказала: пей, господин мой. И тотчас спустила кувшин свой на руку свою и напоила его.
\par 19 И, когда напоила его, сказала: я стану черпать и для верблюдов твоих, пока не напьются.
\par 20 И тотчас вылила воду из кувшина своего в поило и побежала опять к колодезю почерпнуть, и начерпала для всех верблюдов его.
\par 21 Человек тот смотрел на нее с изумлением в молчании, желая уразуметь, благословил ли Господь путь его, или нет.
\par 22 Когда верблюды перестали пить, тогда человек тот взял золотую серьгу, весом полсикля, и два запястья на руки ей, весом в десять [сиклей] золота;
\par 23 И сказал: чья ты дочь? скажи мне, есть ли в доме отца твоего место нам ночевать?
\par 24 Она сказала ему: я дочь Вафуила, сына Милки, которого она родила Нахору.
\par 25 И еще сказала ему: у нас много соломы и корму, и [есть] место для ночлега.
\par 26 И преклонился человек тот и поклонился Господу,
\par 27 и сказал: благословен Господь Бог господина моего Авраама, Который не оставил господина моего милостью Своею и истиною Своею! Господь прямым путем привел меня к дому брата господина моего.
\par 28 Девица побежала и рассказала об этом в доме матери своей.
\par 29 У Ревекки был брат, именем Лаван. Лаван выбежал к тому человеку, к источнику.
\par 30 И когда он увидел серьгу и запястья на руках у сестры своей и услышал слова Ревекки, сестры своей, которая говорила: так говорил со мною этот человек, --то пришел к человеку, и вот, он стоит при верблюдах у источника;
\par 31 и сказал: войди, благословенный Господом; зачем ты стоишь вне? я приготовил дом и место для верблюдов.
\par 32 И вошел человек. [Лаван] расседлал верблюдов и дал соломы и корму верблюдам, и воды умыть ноги ему и людям, которые были с ним;
\par 33 и предложена была ему пища; но он сказал: не стану есть, доколе не скажу дела своего. И сказали: говори.
\par 34 Он сказал: я раб Авраамов;
\par 35 Господь весьма благословил господина моего, и он сделался великим: Он дал ему овец и волов, серебро и золото, рабов и рабынь, верблюдов и ослов;
\par 36 Сарра, жена господина моего, уже состарившись, родила господину моему сына, которому он отдал все, что у него;
\par 37 и взял с меня клятву господин мой, сказав: не бери жены сыну моему из дочерей Хананеев, в земле которых я живу,
\par 38 а пойди в дом отца моего и к родственникам моим, и возьмешь жену сыну моему.
\par 39 Я сказал господину моему: может быть, не пойдет женщина со мною.
\par 40 Он сказал мне: Господь, пред лицем Которого я хожу, пошлет с тобою Ангела Своего и благоустроит путь твой, и возьмешь жену сыну моему из родных моих и из дома отца моего;
\par 41 тогда будешь ты свободен от клятвы моей, когда сходишь к родственникам моим; и если они не дадут тебе, то будешь свободен от клятвы моей.
\par 42 И пришел я ныне к источнику, и сказал: Господи, Боже господина моего Авраама! Если Ты благоустроишь путь, который я совершаю,
\par 43 то вот, я стою у источника воды, и девица, которая выйдет почерпать, и которой я скажу: дай мне испить немного из кувшина твоего,
\par 44 и которая скажет мне: `и ты пей, и верблюдам твоим я начерпаю' --вот жена, которую Господь назначил сыну господина моего.
\par 45 Еще не перестал я говорить в уме моем, и вот вышла Ревекка, и кувшин ее на плече ее, и сошла к источнику и почерпнула; и я сказал ей: напой меня.
\par 46 Она тотчас спустила с себя кувшин свой и сказала: пей, и верблюдов твоих я напою. И я пил, и верблюдов она напоила.
\par 47 Я спросил ее и сказал: чья ты дочь? Она сказала: дочь Вафуила, сына Нахорова, которого родила ему Милка. И дал я серьги ей и запястья на руки ее.
\par 48 И преклонился я и поклонился Господу, и благословил Господа, Бога господина моего Авраама, Который прямым путем привел меня, чтобы взять дочь брата господина моего за сына его.
\par 49 И ныне скажите мне: намерены ли вы оказать милость и правду господину моему или нет? скажите мне, и я обращусь направо, или налево.
\par 50 И отвечали Лаван и Вафуил и сказали: от Господа пришло это дело; мы не можем сказать тебе вопреки ни худого, ни доброго;
\par 51 вот Ревекка пред тобою; возьми и пойди; пусть будет она женою сыну господина твоего, как сказал Господь.
\par 52 Когда раб Авраамов услышал слова их, то поклонился Господу до земли.
\par 53 И вынул раб серебряные вещи и золотые вещи и одежды и дал Ревекке; также и брату ее и матери ее дал богатые подарки.
\par 54 И ели и пили он и люди, бывшие с ним, и переночевали. Когда же встали поутру, то он сказал: отпустите меня к господину моему.
\par 55 Но брат ее и мать ее сказали: пусть побудет с нами девица дней хотя десять, потом пойдешь.
\par 56 Он сказал им: не удерживайте меня, ибо Господь благоустроил путь мой; отпустите меня, и я пойду к господину моему.
\par 57 Они сказали: призовем девицу и спросим, что она скажет.
\par 58 И призвали Ревекку и сказали ей: пойдешь ли с этим человеком? Она сказала: пойду.
\par 59 И отпустили Ревекку, сестру свою, и кормилицу ее, и раба Авраамова, и людей его.
\par 60 И благословили Ревекку и сказали ей: сестра наша! да родятся от тебя тысячи тысяч, и да владеет потомство твое жилищами врагов твоих!
\par 61 И встала Ревекка и служанки ее, и сели на верблюдов, и поехали за тем человеком. И раб взял Ревекку и пошел.
\par 62 А Исаак пришел из Беэр-лахай-рои, ибо жил он в земле полуденной.
\par 63 При наступлении вечера Исаак вышел в поле поразмыслить, и возвел очи свои, и увидел: вот, идут верблюды.
\par 64 Ревекка взглянула, и увидела Исаака, и спустилась с верблюда.
\par 65 И сказала рабу: кто этот человек, который идет по полю навстречу нам? Раб сказал: это господин мой. И она взяла покрывало и покрылась.
\par 66 Раб же сказал Исааку все, что сделал.
\par 67 И ввел ее Исаак в шатер Сарры, матери своей, и взял Ревекку, и она сделалась ему женою, и он возлюбил ее; и утешился Исаак в [печали] по матери своей.

\chapter{25}

\par 1 И взял Авраам еще жену, именем Хеттуру.
\par 2 Она родила ему Зимрана, Иокшана, Медана, Мадиана, Ишбака и Шуаха.
\par 3 Иокшан родил Шеву и Дедана. Сыны Дедана были: Ашурим, Летушим и Леюмим.
\par 4 Сыны Мадиана: Ефа, Ефер, Ханох, Авида и Елдага. Все сии сыны Хеттуры.
\par 5 И отдал Авраам все, что было у него, Исааку,
\par 6 а сынам наложниц, которые были у Авраама, дал Авраам подарки и отослал их от Исаака, сына своего, еще при жизни своей, на восток, в землю восточную.
\par 7 Дней жизни Авраамовой, которые он прожил, было сто семьдесят пять лет;
\par 8 и скончался Авраам, и умер в старости доброй, престарелый и насыщенный [жизнью], и приложился к народу своему.
\par 9 И погребли его Исаак и Измаил, сыновья его, в пещере Махпеле, на поле Ефрона, сына Цохара, Хеттеянина, которое против Мамре,
\par 10 на поле, которые Авраам приобрел от сынов Хетовых. Там погребены Авраам и Сарра, жена его.
\par 11 По смерти Авраама Бог благословил Исаака, сына его. Исаак жил при Беэр-лахай-рои.
\par 12 Вот родословие Измаила, сына Авраамова, которого родила Аврааму Агарь Египтянка, служанка Саррина;
\par 13 и вот имена сынов Измаиловых, имена их по родословию их: первенец Измаилов Наваиоф, [за ним] Кедар, Адбеел, Мивсам,
\par 14 Мишма, Дума, Масса,
\par 15 Хадад, Фема, Иетур, Нафиш и Кедма.
\par 16 Сии суть сыны Измаиловы, и сии имена их, в селениях их, в кочевьях их. [Это] двенадцать князей племен их.
\par 17 Лет же жизни Измаиловой было сто тридцать семь лет; и скончался он, и умер, и приложился к народу своему.
\par 18 Они жили от Хавилы до Сура, что пред Египтом, как идешь к Ассирии. Они поселились пред лицем всех братьев своих.
\par 19 Вот родословие Исаака, сына Авраамова. Авраам родил Исаака.
\par 20 Исаак был сорока лет, когда он взял себе в жену Ревекку, дочь Вафуила Арамеянина из Месопотамии, сестру Лавана Арамеянина.
\par 21 И молился Исаак Господу о жене своей, потому что она была неплодна; и Господь услышал его, и зачала Ревекка, жена его.
\par 22 Сыновья в утробе ее стали биться, и она сказала: если так будет, то для чего мне это? И пошла вопросить Господа.
\par 23 Господь сказал ей: два племени во чреве твоем, и два различных народа произойдут из утробы твоей; один народ сделается сильнее другого, и больший будет служить меньшему.
\par 24 И настало время родить ей: и вот близнецы в утробе ее.
\par 25 Первый вышел красный, весь, как кожа, косматый; и нарекли ему имя Исав.
\par 26 Потом вышел брат его, держась рукою своею за пяту Исава; и наречено ему имя Иаков. Исаак же был шестидесяти лет, когда они родились.
\par 27 Дети выросли, и стал Исав человеком искусным в звероловстве, человеком полей; а Иаков человеком кротким, живущим в шатрах.
\par 28 Исаак любил Исава, потому что дичь его была по вкусу его, а Ревекка любила Иакова.
\par 29 И сварил Иаков кушанье; а Исав пришел с поля усталый.
\par 30 И сказал Исав Иакову: дай мне поесть красного, красного этого, ибо я устал. От сего дано ему прозвание: Едом.
\par 31 Но Иаков сказал: продай мне теперь же свое первородство.
\par 32 Исав сказал: вот, я умираю, что мне в этом первородстве?
\par 33 Иаков сказал: поклянись мне теперь же. Он поклялся ему, и продал первородство свое Иакову.
\par 34 И дал Иаков Исаву хлеба и кушанья из чечевицы; и он ел и пил, и встал и пошел; и пренебрег Исав первородство.

\chapter{26}

\par 1 Был голод в земле, сверх прежнего голода, который был во дни Авраама; и пошел Исаак к Авимелеху, царю Филистимскому, в Герар.
\par 2 Господь явился ему и сказал: не ходи в Египет; живи в земле, о которой Я скажу тебе,
\par 3 странствуй по сей земле, и Я буду с тобою и благословлю тебя, ибо тебе и потомству твоему дам все земли сии и исполню клятву, которою Я клялся Аврааму, отцу твоему;
\par 4 умножу потомство твое, как звезды небесные, и дам потомству твоему все земли сии; благословятся в семени твоем все народы земные,
\par 5 за то, что Авраам послушался гласа Моего и соблюдал, что Мною [заповедано] было соблюдать: повеления Мои, уставы Мои и законы Мои.
\par 6 Исаак поселился в Гераре.
\par 7 Жители места того спросили о жене его, и он сказал: это сестра моя; потому что боялся сказать: жена моя, чтобы не убили меня, [думал он], жители места сего за Ревекку, потому что она прекрасна видом.
\par 8 Но когда уже много времени он там прожил, Авимелех, царь Филистимский, посмотрев в окно, увидел, что Исаак играет с Ревеккою, женою своею.
\par 9 И призвал Авимелех Исаака и сказал: вот, это жена твоя; как же ты сказал: она сестра моя? Исаак сказал ему: потому что я думал, не умереть бы мне ради ее.
\par 10 Но Авимелех сказал: что это ты сделал с нами? едва один из народа не совокупился с женою твоею, и ты ввел бы нас в грех.
\par 11 И дал Авимелех повеление всему народу, сказав: кто прикоснется к сему человеку и к жене его, тот предан будет смерти.
\par 12 И сеял Исаак в земле той и получил в тот год ячменя во сто крат: так благословил его Господь.
\par 13 И стал великим человек сей и возвеличивался больше и больше до того, что стал весьма великим.
\par 14 У него были стада мелкого и стада крупного скота и множество пахотных полей, и Филистимляне стали завидовать ему.
\par 15 И все колодези, которые выкопали рабы отца его при жизни отца его Авраама, Филистимляне завалили и засыпали землею.
\par 16 И Авимелех сказал Исааку: удались от нас, ибо ты сделался гораздо сильнее нас.
\par 17 И Исаак удалился оттуда, и расположился шатрами в долине Герарской, и поселился там.
\par 18 И вновь выкопал Исаак колодези воды, которые выкопаны были во дни Авраама, отца его, и которые завалили Филистимляне по смерти Авраама; и назвал их теми же именами, которыми назвал их отец его.
\par 19 И копали рабы Исааковы в долине и нашли там колодезь воды живой.
\par 20 И спорили пастухи Герарские с пастухами Исаака, говоря: наша вода. И он нарек колодезю имя: Есек, потому что спорили с ним.
\par 21 выкопали другой колодезь; спорили также и о нем; и он нарек ему имя: Ситна.
\par 22 И он двинулся отсюда и выкопал иной колодезь, о котором уже не спорили, и нарек ему имя: Реховоф, ибо, сказал он, теперь Господь дал нам пространное место, и мы размножимся на земле.
\par 23 Оттуда перешел он в Вирсавию.
\par 24 И в ту ночь явился ему Господь и сказал: Я Бог Авраама, отца твоего; не бойся, ибо Я с тобою; и благословлю тебя и умножу потомство твое, ради Авраама, раба Моего.
\par 25 И он устроил там жертвенник и призвал имя Господа. И раскинул там шатер свой, и выкопали там рабы Исааковы колодезь.
\par 26 Пришел к нему из Герара Авимелех и Ахузаф, друг его, и Фихол, военачальник его.
\par 27 Исаак сказал им: для чего вы пришли ко мне, когда вы возненавидели меня и выслали меня от себя?
\par 28 Они сказали: мы ясно увидели, что Господь с тобою, и потому мы сказали: поставим между нами и тобою клятву и заключим с тобою союз,
\par 29 чтобы ты не делал нам зла, как и мы не коснулись до тебя, а делали тебе одно доброе и отпустили тебя с миром; теперь ты благословен Господом.
\par 30 Он сделал им пиршество, и они ели и пили.
\par 31 И встав рано утром, поклялись друг другу; и отпустил их Исаак, и они пошли от него с миром.
\par 32 В тот же день пришли рабы Исааковы и известили его о колодезе, который копали они, и сказали ему: мы нашли воду.
\par 33 И он назвал его: Шива. Посему имя городу тому Беэршива до сего дня.
\par 34 И был Исав сорока лет, и взял себе в жены Иегудифу, дочь Беэра Хеттеянина, и Васемафу, дочь Елона Хеттеянина;
\par 35 и они были в тягость Исааку и Ревекке.

\chapter{27}

\par 1 Когда Исаак состарился и притупилось зрение глаз его, он призвал старшего сына своего Исава и сказал ему: сын мой! Тот сказал ему: вот я.
\par 2 Он сказал: вот, я состарился; не знаю дня смерти моей;
\par 3 возьми теперь орудия твои, колчан твой и лук твой, пойди в поле, и налови мне дичи,
\par 4 и приготовь мне кушанье, какое я люблю, и принеси мне есть, чтобы благословила тебя душа моя, прежде нежели я умру.
\par 5 Ревекка слышала, когда Исаак говорил сыну своему Исаву. И пошел Исав в поле достать и принести дичи;
\par 6 а Ревекка сказала сыну своему Иакову: вот, я слышала, как отец твой говорил брату твоему Исаву:
\par 7 принеси мне дичи и приготовь мне кушанье; я поем и благословлю тебя пред лицем Господним, пред смертью моею.
\par 8 Теперь, сын мой, послушайся слов моих в том, что я прикажу тебе:
\par 9 пойди в [стадо] и возьми мне оттуда два козленка хороших, и я приготовлю из них отцу твоему кушанье, какое он любит,
\par 10 а ты принесешь отцу твоему, и он поест, чтобы благословить тебя пред смертью своею.
\par 11 Иаков сказал Ревекке, матери своей: Исав, брат мой, человек косматый, а я человек гладкий;
\par 12 может статься, ощупает меня отец мой, и я буду в глазах его обманщиком и наведу на себя проклятие, а не благословение.
\par 13 Мать его сказала ему: на мне пусть будет проклятие твое, сын мой, только послушайся слов моих и пойди, принеси мне.
\par 14 Он пошел, и взял, и принес матери своей; и мать его сделала кушанье, какое любил отец его.
\par 15 И взяла Ревекка богатую одежду старшего сына своего Исава, бывшую у ней в доме, и одела [в нее] младшего сына своего Иакова;
\par 16 а руки его и гладкую шею его обложила кожею козлят;
\par 17 и дала кушанье и хлеб, которые она приготовила, в руки Иакову, сыну своему.
\par 18 Он вошел к отцу своему и сказал: отец мой! Тот сказал: вот я; кто ты, сын мой?
\par 19 Иаков сказал отцу своему: я Исав, первенец твой; я сделал, как ты сказал мне; встань, сядь и поешь дичи моей, чтобы благословила меня душа твоя.
\par 20 И сказал Исаак сыну своему: что так скоро нашел ты, сын мой? Он сказал: потому что Господь Бог твой послал мне навстречу.
\par 21 И сказал Исаак Иакову: подойди, я ощупаю тебя, сын мой, ты ли сын мой Исав, или нет?
\par 22 Иаков подошел к Исааку, отцу своему, и он ощупал его и сказал: голос, голос Иакова; а руки, руки Исавовы.
\par 23 И не узнал его, потому что руки его были, как руки Исава, брата его, косматые; и благословил его
\par 24 и сказал: ты ли сын мой Исав? Он отвечал: я.
\par 25 [Исаак] сказал: подай мне, я поем дичи сына моего, чтобы благословила тебя душа моя. [Иаков] подал ему, и он ел; принес ему и вина, и он пил.
\par 26 Исаак, отец его, сказал ему: подойди, поцелуй меня, сын мой.
\par 27 Он подошел и поцеловал его. И ощутил [Исаак] запах от одежды его и благословил его и сказал: вот, запах от сына моего, как запах от поля, которое благословил Господь;
\par 28 да даст тебе Бог от росы небесной и от тука земли, и множество хлеба и вина;
\par 29 да послужат тебе народы, и да поклонятся тебе племена; будь господином над братьями твоими, и да поклонятся тебе сыны матери твоей; проклинающие тебя--прокляты; благословляющие тебя--благословенны!
\par 30 Как скоро совершил Исаак благословение над Иаковом, и как только вышел Иаков от лица Исаака, отца своего, Исав, брат его, пришел с ловли своей.
\par 31 Приготовил и он кушанье, и принес отцу своему, и сказал отцу своему: встань, отец мой, и поешь дичи сына твоего, чтобы благословила меня душа твоя.
\par 32 Исаак же, отец его, сказал ему: кто ты? Он сказал: я сын твой, первенец твой, Исав.
\par 33 И вострепетал Исаак весьма великим трепетом, и сказал: кто ж это, который достал дичи и принес мне, и я ел от всего, прежде нежели ты пришел, и я благословил его? он и будет благословен.
\par 34 Исав, выслушав слова отца своего, поднял громкий и весьма горький вопль и сказал отцу своему: отец мой! благослови и меня.
\par 35 Но он сказал: брат твой пришел с хитростью и взял благословение твое.
\par 36 И сказал он: не потому ли дано ему имя: Иаков, что он запнул меня уже два раза? Он взял первородство мое, и вот, теперь взял благословение мое. И [еще] сказал: неужели ты не оставил мне благословения?
\par 37 Исаак отвечал Исаву: вот, я поставил его господином над тобою и всех братьев его отдал ему в рабы; одарил его хлебом и вином; что же я сделаю для тебя, сын мой?
\par 38 Но Исав сказал отцу своему: неужели, отец мой, одно у тебя благословение? благослови и меня, отец мой! И возвысил Исав голос свой и заплакал.
\par 39 И отвечал Исаак, отец его, и сказал ему: вот, от тука земли будет обитание твое и от росы небесной свыше;
\par 40 и ты будешь жить мечом твоим и будешь служить брату твоему; будет же [время], когда воспротивишься и свергнешь иго его с выи твоей.
\par 41 И возненавидел Исав Иакова за благословение, которым благословил его отец его; и сказал Исав в сердце своем: приближаются дни плача по отце моем, и я убью Иакова, брата моего.
\par 42 И пересказаны были Ревекке слова Исава, старшего сына ее; и она послала, и призвала младшего сына своего Иакова, и сказала ему: вот, Исав, брат твой, грозит убить тебя;
\par 43 и теперь, сын мой, послушайся слов моих, встань, беги к Лавану, брату моему, в Харран,
\par 44 и поживи у него несколько времени, пока утолится ярость брата твоего,
\par 45 пока утолится гнев брата твоего на тебя, и он позабудет, что ты сделал ему: тогда я пошлю и возьму тебя оттуда; для чего мне в один день лишиться обоих вас?
\par 46 И сказала Ревекка Исааку: я жизни не рада от дочерей Хеттейских; если Иаков возьмет жену из дочерей Хеттейских, каковы эти, из дочерей этой земли, то к чему мне и жизнь?

\chapter{28}

\par 1 И призвал Исаак Иакова и благословил его, и заповедал ему и сказал: не бери себе жены из дочерей Ханаанских;
\par 2 встань, пойди в Месопотамию, в дом Вафуила, отца матери твоей, и возьми себе жену оттуда, из дочерей Лавана, брата матери твоей;
\par 3 Бог же Всемогущий да благословит тебя, да расплодит тебя и да размножит тебя, и да будет от тебя множество народов,
\par 4 и да даст тебе благословение Авраама, тебе и потомству твоему с тобою, чтобы тебе наследовать землю странствования твоего, которую Бог дал Аврааму!
\par 5 И отпустил Исаак Иакова, и он пошел в Месопотамию к Лавану, сыну Вафуила Арамеянина, к брату Ревекки, матери Иакова и Исава.
\par 6 Исав увидел, что Исаак благословил Иакова и благословляя послал его в Месопотамию, взять себе жену оттуда, и заповедал ему, сказав: не бери жены из дочерей Ханаанских;
\par 7 и что Иаков послушался отца своего и матери своей и пошел в Месопотамию.
\par 8 И увидел Исав, что дочери Ханаанские не угодны Исааку, отцу его;
\par 9 и пошел Исав к Измаилу и взял себе жену Махалафу, дочь Измаила, сына Авраамова, сестру Наваиофову, сверх [других] жен своих.
\par 10 Иаков же вышел из Вирсавии и пошел в Харран,
\par 11 и пришел на [одно] место, и [остался] там ночевать, потому что зашло солнце. И взял [один] из камней того места, и положил себе изголовьем, и лег на том месте.
\par 12 И увидел во сне: вот, лестница стоит на земле, а верх ее касается неба; и вот, Ангелы Божии восходят и нисходят по ней.
\par 13 И вот, Господь стоит на ней и говорит: Я Господь, Бог Авраама, отца твоего, и Бог Исаака. Землю, на которой ты лежишь, Я дам тебе и потомству твоему;
\par 14 и будет потомство твое, как песок земной; и распространишься к морю и к востоку, и к северу и к полудню; и благословятся в тебе и в семени твоем все племена земные;
\par 15 и вот Я с тобою, и сохраню тебя везде, куда ты ни пойдешь; и возвращу тебя в сию землю, ибо Я не оставлю тебя, доколе не исполню того, что Я сказал тебе.
\par 16 Иаков пробудился от сна своего и сказал: истинно Господь присутствует на месте сем; а я не знал!
\par 17 И убоялся и сказал: как страшно сие место! это не иное что, как дом Божий, это врата небесные.
\par 18 И встал Иаков рано утром, и взял камень, который он положил себе изголовьем, и поставил его памятником, и возлил елей на верх его.
\par 19 И нарек имя месту тому: Вефиль, а прежнее имя того города было: Луз.
\par 20 И положил Иаков обет, сказав: если Бог будет со мною и сохранит меня в пути сем, в который я иду, и даст мне хлеб есть и одежду одеться,
\par 21 и я в мире возвращусь в дом отца моего, и будет Господь моим Богом, --
\par 22 то этот камень, который я поставил памятником, будет домом Божиим; и из всего, что Ты, [Боже], даруешь мне, я дам Тебе десятую часть.

\chapter{29}

\par 1 И встал Иаков и пошел в землю сынов востока.
\par 2 И увидел: вот, на поле колодезь, и там три стада мелкого скота, лежавшие около него, потому что из того колодезя поили стада. Над устьем колодезя был большой камень.
\par 3 Когда собирались туда все стада, отваливали камень от устья колодезя и поили овец; потом опять клали камень на свое место, на устье колодезя.
\par 4 Иаков сказал им: братья мои! откуда вы? Они сказали: мы из Харрана.
\par 5 Он сказал им: знаете ли вы Лавана, сына Нахорова? Они сказали: знаем.
\par 6 Он еще сказал им: здравствует ли он? Они сказали: здравствует; и вот, Рахиль, дочь его, идет с овцами.
\par 7 И сказал: вот, дня еще много; не время собирать скот; напойте овец и пойдите, пасите.
\par 8 Они сказали: не можем, пока не соберутся все стада, и не отвалят камня от устья колодезя; тогда будем мы поить овец.
\par 9 Еще он говорил с ними, как пришла Рахиль с мелким скотом отца своего, потому что она пасла.
\par 10 Когда Иаков увидел Рахиль, дочь Лавана, брата матери своей, и овец Лавана, брата матери своей, то подошел Иаков, отвалил камень от устья колодезя и напоил овец Лавана, брата матери своей.
\par 11 И поцеловал Иаков Рахиль и возвысил голос свой и заплакал.
\par 12 И сказал Иаков Рахили, что он родственник отцу ее и что он сын Ревеккин. А она побежала и сказала отцу своему.
\par 13 Лаван, услышав о Иакове, сыне сестры своей, выбежал ему навстречу, обнял его и поцеловал его, и ввел его в дом свой; и он рассказал Лавану все сие.
\par 14 Лаван же сказал ему: подлинно ты кость моя и плоть моя. И жил у него [Иаков] целый месяц.
\par 15 И Лаван сказал Иакову: неужели ты даром будешь служить мне, потому что ты родственник? скажи мне, что заплатить тебе?
\par 16 У Лавана же было две дочери; имя старшей: Лия; имя младшей: Рахиль.
\par 17 Лия была слаба глазами, а Рахиль была красива станом и красива лицем.
\par 18 Иаков полюбил Рахиль и сказал: я буду служить тебе семь лет за Рахиль, младшую дочь твою.
\par 19 Лаван сказал: лучше отдать мне ее за тебя, нежели отдать ее за другого кого; живи у меня.
\par 20 И служил Иаков за Рахиль семь лет; и они показались ему за несколько дней, потому что он любил ее.
\par 21 И сказал Иаков Лавану: дай жену мою, потому что мне уже исполнилось время, чтобы войти к ней.
\par 22 Лаван созвал всех людей того места и сделал пир.
\par 23 Вечером же взял дочь свою Лию и ввел ее к нему; и вошел к ней [Иаков].
\par 24 И дал Лаван служанку свою Зелфу в служанки дочери своей Лии.
\par 25 Утром же оказалось, что это Лия. И сказал Лавану: что это сделал ты со мною? не за Рахиль ли я служил у тебя? зачем ты обманул меня?
\par 26 Лаван сказал: в нашем месте так не делают, чтобы младшую выдать прежде старшей;
\par 27 окончи неделю этой, потом дадим тебе и ту за службу, которую ты будешь служить у меня еще семь лет других.
\par 28 Иаков так и сделал и окончил неделю этой. И [Лаван] дал Рахиль, дочь свою, ему в жену.
\par 29 И дал Лаван служанку свою Валлу в служанки дочери своей Рахили.
\par 30 [Иаков] вошел и к Рахили, и любил Рахиль больше, нежели Лию; и служил у него еще семь лет других.
\par 31 Господь узрел, что Лия была нелюбима, и отверз утробу ее, а Рахиль была неплодна.
\par 32 Лия зачала и родила сына, и нарекла ему имя: Рувим, потому что сказала она: Господь призрел на мое бедствие; ибо теперь будет любить меня муж мой.
\par 33 И зачала опять и родила сына, и сказала: Господь услышал, что я нелюбима, и дал мне и сего. И нарекла ему имя: Симеон.
\par 34 И зачала еще и родила сына, и сказала: теперь-то прилепится ко мне муж мой, ибо я родила ему трех сынов. От сего наречено ему имя: Левий.
\par 35 И еще зачала и родила сына, и сказала: теперь-то я восхвалю Господа. Посему нарекла ему имя Иуда. И перестала рождать.

\chapter{30}

\par 1 И увидела Рахиль, что она не рождает детей Иакову, и позавидовала Рахиль сестре своей, и сказала Иакову: дай мне детей, а если не так, я умираю.
\par 2 Иаков разгневался на Рахиль и сказал: разве я Бог, Который не дал тебе плода чрева?
\par 3 Она сказала: вот служанка моя Валла; войди к ней; пусть она родит на колени мои, чтобы и я имела детей от нее.
\par 4 И дала она Валлу, служанку свою, в жену ему; и вошел к ней Иаков.
\par 5 Валла зачала и родила Иакову сына.
\par 6 И сказала Рахиль: судил мне Бог, и услышал голос мой, и дал мне сына. Посему нарекла ему имя: Дан.
\par 7 И еще зачала и родила Валла, служанка Рахилина, другого сына Иакову.
\par 8 И сказала Рахиль: борьбою сильною боролась я с сестрою моею и превозмогла. И нарекла ему имя: Неффалим.
\par 9 Лия увидела, что перестала рождать, и взяла служанку свою Зелфу, и дала ее Иакову в жену.
\par 10 И Зелфа, служанка Лиина, родила Иакову сына.
\par 11 И сказала Лия: прибавилось. И нарекла ему имя: Гад.
\par 12 И родила Зелфа, служанка Лии, другого сына Иакову.
\par 13 И сказала Лия: к благу моему, ибо блаженною будут называть меня женщины. И нарекла ему имя: Асир.
\par 14 Рувим пошел во время жатвы пшеницы, и нашел мандрагоровые яблоки в поле, и принес их Лии, матери своей. И Рахиль сказала Лии: дай мне мандрагоров сына твоего.
\par 15 Но она сказала ей: неужели мало тебе завладеть мужем моим, что ты домогаешься и мандрагоров сына моего? Рахиль сказала: так пусть он ляжет с тобою эту ночь, за мандрагоры сына твоего.
\par 16 Иаков пришел с поля вечером, и Лия вышла ему навстречу и сказала: войди ко мне; ибо я купила тебя за мандрагоры сына моего. И лег он с нею в ту ночь.
\par 17 И услышал Бог Лию, и она зачала и родила Иакову пятого сына.
\par 18 И сказала Лия: Бог дал возмездие мне за то, что я отдала служанку мою мужу моему. И нарекла ему имя: Иссахар.
\par 19 И еще зачала Лия и родила Иакову шестого сына.
\par 20 И сказала Лия: Бог дал мне прекрасный дар; теперь будет жить у меня муж мой, ибо я родила ему шесть сынов. И нарекла ему имя: Завулон.
\par 21 Потом родила дочь и нарекла ей имя: Дина.
\par 22 И вспомнил Бог о Рахили, и услышал ее Бог, и отверз утробу ее.
\par 23 Она зачала и родила сына, и сказала: снял Бог позор мой.
\par 24 И нарекла ему имя: Иосиф, сказав: Господь даст мне и другого сына.
\par 25 После того, как Рахиль родила Иосифа, Иаков сказал Лавану: отпусти меня, и пойду я в свое место, и в свою землю;
\par 26 отдай жен моих и детей моих, за которых я служил тебе, и я пойду, ибо ты знаешь службу мою, какую я служил тебе.
\par 27 И сказал ему Лаван: о, если бы я нашел благоволение пред очами твоими! я примечаю, что за тебя Господь благословил меня.
\par 28 И сказал: назначь себе награду от меня, и я дам.
\par 29 И сказал ему [Иаков]: ты знаешь, как я служил тебе, и каков стал скот твой при мне;
\par 30 ибо мало было у тебя до меня, а стало много; Господь благословил тебя с приходом моим; когда же я буду работать для своего дома?
\par 31 И сказал [Лаван]: что дать тебе? Иаков сказал: не давай мне ничего. Если только сделаешь мне, что я скажу, то я опять буду пасти и стеречь овец твоих.
\par 32 Я пройду сегодня по всему [стаду] овец твоих; отдели из него всякий скот с крапинами и с пятнами, всякую скотину черную из овец, также с пятнами и с крапинами из коз. [Такой скот] будет наградою мне.
\par 33 И будет говорить за меня пред тобою справедливость моя в следующее время, когда придешь посмотреть награду мою. Всякая из коз не с крапинами и не с пятнами, и из овец не черная, краденое это у меня.
\par 34 Лаван сказал: хорошо, пусть будет по твоему слову.
\par 35 И отделил в тот день козлов пестрых и с пятнами, и всех коз с крапинами и с пятнами, всех, на которых было [несколько] белого, и всех черных овец, и отдал на руки сыновьям своим;
\par 36 и назначил расстояние между собою и между Иаковом на три дня пути. Иаков же пас остальной мелкий скот Лаванов.
\par 37 И взял Иаков свежих прутьев тополевых, миндальных и яворовых, и вырезал на них белые полосы, сняв кору до белизны, которая на прутьях,
\par 38 и положил прутья с нарезкою перед скотом в водопойных корытах, куда скот приходил пить, и где, приходя пить, зачинал пред прутьями.
\par 39 И зачинал скот пред прутьями, и рождался скот пестрый, и с крапинами, и с пятнами.
\par 40 И отделял Иаков ягнят и ставил скот лицем к пестрому и всему черному скоту Лаванову; и держал свои стада особо и не ставил их вместе со скотом Лавана.
\par 41 Каждый раз, когда зачинал скот крепкий, Иаков клал прутья в корытах пред глазами скота, чтобы он зачинал пред прутьями.
\par 42 А когда зачинал скот слабый, тогда он не клал. И доставался слабый [скот] Лавану, а крепкий Иакову.
\par 43 И сделался этот человек весьма, весьма богатым, и было у него множество мелкого скота, и рабынь, и рабов, и верблюдов, и ослов.

\chapter{31}

\par 1 И услышал [Иаков] слова сынов Лавановых, которые говорили: Иаков завладел всем, что было у отца нашего, и из имения отца нашего составил все богатство сие.
\par 2 И увидел Иаков лице Лавана, и вот, оно не таково к нему, как было вчера и третьего дня.
\par 3 И сказал Господь Иакову: возвратись в землю отцов твоих и на родину твою; и Я буду с тобою.
\par 4 И послал Иаков, и призвал Рахиль и Лию в поле, к [стаду] мелкого скота своего,
\par 5 и сказал им: я вижу лице отца вашего, что оно ко мне не таково, как было вчера и третьего дня; но Бог отца моего был со мною;
\par 6 вы сами знаете, что я всеми силами служил отцу вашему,
\par 7 а отец ваш обманывал меня и раз десять переменял награду мою; но Бог не попустил ему сделать мне зло.
\par 8 Когда сказал он, что [скот] с крапинами будет тебе в награду, то скот весь родил с крапинами. А когда он сказал: пестрые будут тебе в награду, то скот весь и родил пестрых.
\par 9 И отнял Бог скот у отца вашего и дал мне.
\par 10 Однажды в такое время, когда скот зачинает, я взглянул и увидел во сне, и вот козлы, поднявшиеся на скот, пестрые с крапинами и пятнами.
\par 11 Ангел Божий сказал мне во сне: Иаков! Я сказал: вот я.
\par 12 Он сказал: возведи очи твои и посмотри: все козлы, поднявшиеся на скот, пестрые, с крапинами и с пятнами, ибо Я вижу все, что Лаван делает с тобою;
\par 13 Я Бог [явившийся тебе] в Вефиле, где ты возлил елей на памятник и где ты дал Мне обет; теперь встань, выйди из земли сей и возвратись в землю родины твоей.
\par 14 Рахиль и Лия сказали ему в ответ: есть ли еще нам доля и наследство в доме отца нашего?
\par 15 не за чужих ли он нас почитает? ибо он продал нас и съел даже серебро наше;
\par 16 посему все богатство, которое Бог отнял у отца нашего, есть наше и детей наших; итак делай все, что Бог сказал тебе.
\par 17 И встал Иаков, и посадил детей своих и жен своих на верблюдов,
\par 18 и взял с собою весь скот свой и все богатство свое, которое приобрел, скот собственный его, который он приобрел в Месопотамии, чтобы идти к Исааку, отцу своему, в землю Ханаанскую.
\par 19 И как Лаван пошел стричь скот свой, то Рахиль похитила идолов, которые были у отца ее.
\par 20 Иаков же похитил сердце у Лавана Арамеянина, потому что не известил его, что удаляется.
\par 21 И ушел со всем, что у него было; и, встав, перешел реку и направился к горе Галаад.
\par 22 На третий день сказали Лавану, что Иаков ушел.
\par 23 Тогда он взял с собою родственников своих, и гнался за ним семь дней, и догнал его на горе Галаад.
\par 24 И пришел Бог к Лавану Арамеянину ночью во сне и сказал ему: берегись, не говори Иакову ни доброго, ни худого.
\par 25 И догнал Лаван Иакова; Иаков же поставил шатер свой на горе, и Лаван со сродниками своими поставил на горе Галаад.
\par 26 И сказал Лаван Иакову: что ты сделал? для чего ты обманул меня, и увел дочерей моих, как плененных оружием?
\par 27 зачем ты убежал тайно, и укрылся от меня, и не сказал мне? я отпустил бы тебя с веселием и с песнями, с тимпаном и с гуслями;
\par 28 ты не позволил мне даже поцеловать внуков моих и дочерей моих; безрассудно ты сделал.
\par 29 Есть в руке моей сила сделать вам зло; но Бог отца вашего вчера говорил ко мне и сказал: берегись, не говори Иакову ни хорошего, ни худого.
\par 30 Но пусть бы ты ушел, потому что ты нетерпеливо захотел быть в доме отца твоего, --зачем ты украл богов моих?
\par 31 Иаков отвечал Лавану и сказал: [я] боялся, ибо я думал, не отнял бы ты у меня дочерей своих.
\par 32 у кого найдешь богов твоих, тот не будет жив; при родственниках наших узнавай, что у меня, и возьми себе. Иаков не знал, что Рахиль украла их.
\par 33 И ходил Лаван в шатер Иакова, и в шатер Лии, и в шатер двух рабынь, но не нашел. И, выйдя из шатра Лии, вошел в шатер Рахили.
\par 34 Рахиль же взяла идолов, и положила их под верблюжье седло и села на них. И обыскал Лаван весь шатер; но не нашел.
\par 35 Она же сказала отцу своему: да не прогневается господин мой, что я не могу встать пред тобою, ибо у меня обыкновенное женское. И он искал, но не нашел идолов.
\par 36 Иаков рассердился и вступил в спор с Лаваном. И начал Иаков говорить и сказал Лавану: какая вина моя, какой грех мой, что ты преследуешь меня?
\par 37 ты осмотрел у меня все вещи, что нашел ты из всех вещей твоего дома? покажи здесь пред родственниками моими и пред родственниками твоими; пусть они рассудят между нами обоими.
\par 38 Вот, двадцать лет я [был] у тебя; овцы твои и козы твои не выкидывали; овнов стада твоего я не ел;
\par 39 растерзанного зверем я не приносил к тебе, это был мой убыток; ты с меня взыскивал, днем ли что пропадало, ночью ли пропадало;
\par 40 я томился днем от жара, а ночью от стужи, и сон мой убегал от глаз моих.
\par 41 Таковы мои двадцать лет в доме твоем. Я служил тебе четырнадцать лет за двух дочерей твоих и шесть лет за скот твой, а ты десять раз переменял награду мою.
\par 42 Если бы не был со мною Бог отца моего, Бог Авраама и страх Исаака, ты бы теперь отпустил меня ни с чем. Бог увидел бедствие мое и труд рук моих и вступился [за меня] вчера.
\par 43 И отвечал Лаван и сказал Иакову: дочери--мои дочери; дети--мои дети; скот--мой скот, и все, что ты видишь, это мое: могу ли я что сделать теперь с дочерями моими и с детьми их, которые рождены ими?
\par 44 Теперь заключим союз я и ты, и это будет свидетельством между мною и тобою.
\par 45 И взял Иаков камень и поставил его памятником.
\par 46 И сказал Иаков родственникам своим: наберите камней. Они взяли камни, и сделали холм, и ели там на холме.
\par 47 И назвал его Лаван: Иегар-Сагадуфа; а Иаков назвал его Галаадом.
\par 48 И сказал Лаван: сегодня этот холм между мною и тобою свидетель. Посему и наречено ему имя: Галаад,
\par 49 [также]: Мицпа, оттого, что Лаван сказал: да надзирает Господь надо мною и над тобою, когда мы скроемся друг от друга;
\par 50 если ты будешь худо поступать с дочерями моими, или если возьмешь жен сверх дочерей моих, то, хотя нет человека между нами, но смотри, Бог свидетель между мною и между тобою.
\par 51 И сказал Лаван Иакову: вот холм сей и вот памятник, который я поставил между мною и тобою;
\par 52 этот холм свидетель, и этот памятник свидетель, что ни я не перейду к тебе за этот холм, ни ты не перейдешь ко мне за этот холм и за этот памятник, для зла;
\par 53 Бог Авраамов и Бог Нахоров да судит между нами, Бог отца их. Иаков поклялся страхом отца своего Исаака.
\par 54 И заколол Иаков жертву на горе и позвал родственников своих есть хлеб; и они ели хлеб и ночевали на горе.
\par 55 И встал Лаван рано утром и поцеловал внуков своих и дочерей своих, и благословил их. И пошел и возвратился Лаван в свое место.

\chapter{32}

\par 1 А Иаков пошел путем своим. И встретили его Ангелы Божии.
\par 2 Иаков, увидев их, сказал: это ополчение Божие. И нарек имя месту тому: Маханаим.
\par 3 И послал Иаков пред собою вестников к брату своему Исаву в землю Сеир, в область Едом,
\par 4 и приказал им, сказав: так скажите господину моему Исаву: вот что говорит раб твой Иаков: я жил у Лавана и прожил доныне;
\par 5 и есть у меня волы и ослы и мелкий скот, и рабы и рабыни; и я послал известить [о себе] господина моего, дабы приобрести благоволение пред очами твоими.
\par 6 И возвратились вестники к Иакову и сказали: мы ходили к брату твоему Исаву; он идет навстречу тебе, и с ним четыреста человек.
\par 7 Иаков очень испугался и смутился; и разделил людей, бывших с ним, и скот мелкий и крупный и верблюдов на два стана.
\par 8 И сказал: если Исав нападет на один стан и побьет его, то остальной стан может спастись.
\par 9 И сказал Иаков: Боже отца моего Авраама и Боже отца моего Исаака, Господи, сказавший мне: возвратись в землю твою, на родину твою, и Я буду благотворить тебе!
\par 10 Недостоин я всех милостей и всех благодеяний, которые Ты сотворил рабу Твоему, ибо я с посохом моим перешел этот Иордан, а теперь у меня два стана.
\par 11 Избавь меня от руки брата моего, от руки Исава, ибо я боюсь его, чтобы он, придя, не убил меня [и] матери с детьми.
\par 12 Ты сказал: Я буду благотворить тебе и сделаю потомство твое, как песок морской, которого не исчислить от множества.
\par 13 И ночевал там [Иаков] в ту ночь. И взял из того, что у него было, в подарок Исаву, брату своему:
\par 14 двести коз, двадцать козлов, двести овец, двадцать овнов,
\par 15 тридцать верблюдиц дойных с жеребятами их, сорок коров, десять волов, двадцать ослиц, десять ослов.
\par 16 И дал в руки рабам своим каждое стадо особо и сказал рабам своим: пойдите предо мною и оставляйте расстояние от стада до стада.
\par 17 И приказал первому, сказав: когда брат мой Исав встретится тебе и спросит тебя, говоря: чей ты? и куда идешь? и чье это [стадо] пред тобою?
\par 18 то скажи: раба твоего Иакова; это подарок, посланный господину моему Исаву; вот, и сам он за нами.
\par 19 То же приказал он и второму, и третьему, и всем, которые шли за стадами, говоря: так скажите Исаву, когда встретите его;
\par 20 и скажите: вот, и раб твой Иаков за нами. Ибо он сказал [сам в себе]: умилостивлю его дарами, которые идут предо мною, и потом увижу лице его; может быть, и примет меня.
\par 21 И пошли дары пред ним, а он ту ночь ночевал в стане.
\par 22 И встал в ту ночь, и, взяв двух жен своих и двух рабынь своих, и одиннадцать сынов своих, перешел через Иавок вброд;
\par 23 и, взяв их, перевел через поток, и перевел все, что у него [было].
\par 24 И остался Иаков один. И боролся Некто с ним до появления зари;
\par 25 и, увидев, что не одолевает его, коснулся состава бедра его и повредил состав бедра у Иакова, когда он боролся с Ним.
\par 26 И сказал: отпусти Меня, ибо взошла заря. Иаков сказал: не отпущу Тебя, пока не благословишь меня.
\par 27 И сказал: как имя твое? Он сказал: Иаков.
\par 28 И сказал: отныне имя тебе будет не Иаков, а Израиль, ибо ты боролся с Богом, и человеков одолевать будешь.
\par 29 Спросил и Иаков, говоря: скажи имя Твое. И Он сказал: на что ты спрашиваешь о имени Моем? И благословил его там.
\par 30 И нарек Иаков имя месту тому: Пенуэл; ибо, [говорил он], я видел Бога лицем к лицу, и сохранилась душа моя.
\par 31 И взошло солнце, когда он проходил Пенуэл; и хромал он на бедро свое.
\par 32 Поэтому и доныне сыны Израилевы не едят жилы, которая на составе бедра, потому что [Боровшийся] коснулся жилы на составе бедра Иакова.

\chapter{33}

\par 1 Взглянул Иаков и увидел, и вот, идет Исав, и с ним четыреста человек. И разделил детей Лии, Рахили и двух служанок.
\par 2 И поставил служанок и детей их впереди, Лию и детей ее за ними, а Рахиль и Иосифа позади.
\par 3 А сам пошел пред ними и поклонился до земли семь раз, подходя к брату своему.
\par 4 И побежал Исав к нему навстречу и обнял его, и пал на шею его и целовал его, и плакали.
\par 5 И взглянул и увидел жен и детей и сказал: кто это у тебя? [Иаков] сказал: дети, которых Бог даровал рабу твоему.
\par 6 И подошли служанки и дети их и поклонились;
\par 7 подошла и Лия и дети ее и поклонились; наконец подошли Иосиф и Рахиль и поклонились.
\par 8 И сказал Исав: для чего у тебя это множество, которое я встретил? И сказал Иаков: дабы приобрести благоволение в очах господина моего.
\par 9 Исав сказал: у меня много, брат мой; пусть будет твое у тебя.
\par 10 Иаков сказал: нет, если я приобрел благоволение в очах твоих, прими дар мой от руки моей, ибо я увидел лице твое, как бы кто увидел лице Божие, и ты был благосклонен ко мне;
\par 11 прими благословение мое, которое я принес тебе, потому что Бог даровал мне, и есть у меня все. И упросил его, и тот взял
\par 12 и сказал: поднимемся и пойдем; и я пойду пред тобою.
\par 13 Иаков сказал ему: господин мой знает, что дети нежны, а мелкий и крупный скот у меня дойный: если погнать его один день, то помрет весь скот;
\par 14 пусть господин мой пойдет впереди раба своего, а я пойду медленно, как пойдет скот, который предо мною, и как пойдут дети, и приду к господину моему в Сеир.
\par 15 Исав сказал: оставлю я с тобою [несколько] из людей, которые при мне. Иаков сказал: к чему это? только бы мне приобрести благоволение в очах господина моего!
\par 16 И возвратился Исав в тот же день путем своим в Сеир.
\par 17 А Иаков двинулся в Сокхоф, и построил себе дом, и для скота своего сделал шалаши. От сего он нарек имя месту: Сокхоф.
\par 18 Иаков, возвратившись из Месопотамии, благополучно пришел в город Сихем, который в земле Ханаанской, и расположился пред городом.
\par 19 И купил часть поля, на котором раскинул шатер свой, у сынов Еммора, отца Сихемова, за сто монет.
\par 20 И поставил там жертвенник, и призвал имя Господа Бога Израилева.

\chapter{34}

\par 1 Дина, дочь Лии, которую она родила Иакову, вышла посмотреть на дочерей земли той.
\par 2 И увидел ее Сихем, сын Еммора Евеянина, князя земли той, и взял ее, и спал с нею, и сделал ей насилие.
\par 3 И прилепилась душа его в Дине, дочери Иакова, и он полюбил девицу и говорил по сердцу девицы.
\par 4 И сказал Сихем Еммору, отцу своему, говоря: возьми мне эту девицу в жену.
\par 5 Иаков слышал, что [сын Емморов] обесчестил Дину, дочь его, но как сыновья его были со скотом его в поле, то Иаков молчал, пока не пришли они.
\par 6 И вышел Еммор, отец Сихемов, к Иакову, поговорить с ним.
\par 7 Сыновья же Иакова пришли с поля, и когда услышали, то огорчились мужи те и воспылали гневом, потому что бесчестие сделал он Израилю, переспав с дочерью Иакова, а так не надлежало делать.
\par 8 Еммор стал говорить им, и сказал: Сихем, сын мой, прилепился душею к дочери вашей; дайте же ее в жену ему;
\par 9 породнитесь с нами; отдавайте за нас дочерей ваших, а наших дочерей берите себе.
\par 10 и живите с нами; земля сия пред вами, живите и промышляйте на ней и приобретайте ее во владение.
\par 11 Сихем же сказал отцу ее и братьям ее: только бы мне найти благоволение в очах ваших, я дам, что ни скажете мне;
\par 12 назначьте самое большое вено и дары; я дам, что ни скажете мне, только отдайте мне девицу в жену.
\par 13 И отвечали сыновья Иакова Сихему и Еммору, отцу его, с лукавством; а говорили так потому, что он обесчестил Дину, сестру их;
\par 14 и сказали им: не можем этого сделать, выдать сестру нашу за человека, который необрезан, ибо это бесчестно для нас;
\par 15 только на том условии мы согласимся с вами, если вы будете как мы, чтобы и у вас весь мужеский пол был обрезан;
\par 16 и будем отдавать за вас дочерей наших и брать за себя ваших дочерей, и будем жить с вами, и составим один народ;
\par 17 а если не послушаетесь нас в том, чтобы обрезаться, то мы возьмем дочь нашу и удалимся.
\par 18 И понравились слова сии Еммору и Сихему, сыну Емморову.
\par 19 Юноша не умедлил исполнить это, потому что любил дочь Иакова. А он более всех уважаем был из дома отца своего.
\par 20 И пришел Еммор и Сихем, сын его, к воротам города своего, и стали говорить жителям города своего и сказали:
\par 21 сии люди мирны с нами; пусть они селятся на земле и промышляют на ней; земля же вот пространна пред ними. Станем брать дочерей их себе в жены и наших дочерей выдавать за них.
\par 22 Только на том условии сии люди соглашаются жить с нами и быть одним народом, чтобы и у нас обрезан был весь мужеский пол, как они обрезаны.
\par 23 Не для нас ли стада их, и имение их, и весь скот их? Только согласимся с ними, и будут жить с нами.
\par 24 И послушались Еммора и Сихема, сына его, все выходящие из ворот города его: и обрезан был весь мужеский пол, --все выходящие из ворот города его.
\par 25 На третий день, когда они были в болезни, два сына Иакова, Симеон и Левий, братья Динины, взяли каждый свой меч, и смело напали на город, и умертвили весь мужеский пол;
\par 26 и самого Еммора и Сихема, сына его, убили мечом; и взяли Дину из дома Сихемова и вышли.
\par 27 Сыновья Иакова пришли к убитым и разграбили город за то, что обесчестили сестру их.
\par 28 Они взяли мелкий и крупный скот их, и ослов их, и что ни было в городе, и что ни было в поле;
\par 29 и все богатство их, и всех детей их, и жен их взяли в плен, и разграбили все, что было в домах.
\par 30 И сказал Иаков Симеону и Левию: вы возмутили меня, сделав меня ненавистным для жителей сей земли, для Хананеев и Ферезеев. У меня людей мало; соберутся против меня, поразят меня, и истреблен буду я и дом мой.
\par 31 Они же сказали: а разве можно поступать с сестрою нашею, как с блудницею!

\chapter{35}

\par 1 Бог сказал Иакову: встань, пойди в Вефиль и живи там, и устрой там жертвенник Богу, явившемуся тебе, когда ты бежал от лица Исава, брата твоего.
\par 2 И сказал Иаков дому своему и всем бывшим с ним: бросьте богов чужих, находящихся у вас, и очиститесь, и перемените одежды ваши;
\par 3 встанем и пойдем в Вефиль; там устрою я жертвенник Богу, Который услышал меня в день бедствия моего и был со мною в пути, которым я ходил.
\par 4 И отдали Иакову всех богов чужих, бывших в руках их, и серьги, бывшие в ушах у них, и закопал их Иаков под дубом, который близ Сихема.
\par 5 И отправились они. И был ужас Божий на окрестных городах, и не преследовали сынов Иаковлевых.
\par 6 И пришел Иаков в Луз, что в земле Ханаанской, то есть в Вефиль, сам и все люди, бывшие с ним,
\par 7 и устроил там жертвенник, и назвал сие место: Эл-Вефиль, ибо тут явился ему Бог, когда он бежал от лица брата своего.
\par 8 И умерла Девора, кормилица Ревеккина, и погребена ниже Вефиля под дубом, который и назвал [Иаков] дубом плача.
\par 9 И явился Бог Иакову по возвращении его из Месопотамии, и благословил его,
\par 10 и сказал ему Бог: имя твое Иаков; отныне ты не будешь называться Иаковом, но будет имя тебе: Израиль. И нарек ему имя: Израиль.
\par 11 И сказал ему Бог: Я Бог Всемогущий; плодись и умножайся; народ и множество народов будет от тебя, и цари произойдут из чресл твоих;
\par 12 землю, которую Я дал Аврааму и Исааку, Я дам тебе, и потомству твоему по тебе дам землю сию.
\par 13 И восшел от него Бог с места, на котором говорил ему.
\par 14 И поставил Иаков памятник на месте, на котором говорил ему [Бог], памятник каменный, и возлил на него возлияние, и возлил на него елей;
\par 15 и нарек Иаков имя месту, на котором Бог говорил ему: Вефиль.
\par 16 И отправились из Вефиля. И когда еще оставалось некоторое расстояние земли до Ефрафы, Рахиль родила, и роды ее были трудны.
\par 17 Когда же она страдала в родах, повивальная бабка сказала ей: не бойся, ибо и это тебе сын.
\par 18 И когда выходила из нее душа, ибо она умирала, то нарекла ему имя: Бенони. Но отец его назвал его Вениамином.
\par 19 И умерла Рахиль, и погребена на дороге в Ефрафу, то есть Вифлеем.
\par 20 Иаков поставил над гробом ее памятник. Это надгробный памятник Рахили до сего дня.
\par 21 И отправился Израиль и раскинул шатер свой за башнею Гадер.
\par 22 Во время пребывания Израиля в той стране, Рувим пошел и переспал с Валлою, наложницею отца своего. И услышал Израиль. Сынов же у Иакова было двенадцать.
\par 23 Сыновья Лии: первенец Иакова Рувим, [по нем] Симеон, Левий, Иуда, Иссахар и Завулон.
\par 24 Сыновья Рахили: Иосиф и Вениамин.
\par 25 Сыновья Валлы, служанки Рахилиной: Дан и Неффалим.
\par 26 Сыновья Зелфы, служанки Лииной: Гад и Асир. Сии сыновья Иакова, родившиеся ему в Месопотамии.
\par 27 И пришел Иаков к Исааку, отцу своему, в Мамре, в Кириаф-Арбу, то есть Хеврон где странствовал Авраам и Исаак.
\par 28 И было дней [жизни] Исааковой сто восемьдесят лет.
\par 29 И испустил Исаак дух и умер, и приложился к народу своему, будучи стар и насыщен жизнью; и погребли его Исав и Иаков, сыновья его.

\chapter{36}

\par 1 Вот родословие Исава, он же Едом.
\par 2 Исав взял себе жен из дочерей Ханаанских: Аду, дочь Елона Хеттеянина, и Оливему, дочь Аны, сына Цивеона Евеянина,
\par 3 и Васемафу, дочь Измаила, сестру Наваиофа.
\par 4 Ада родила Исаву Елифаза, Васемафа родила Рагуила,
\par 5 Оливема родила Иеуса, Иеглома и Корея. Это сыновья Исава, родившиеся ему в земле Ханаанской.
\par 6 И взял Исав жен своих и сыновей своих, и дочерей своих, и всех людей дома своего, и стада свои, и весь скот свой, и все имение свое, которое он приобрел в земле Ханаанской, и пошел в [другую] землю от лица Иакова, брата своего,
\par 7 ибо имение их было так велико, что они не могли жить вместе, и земля странствования их не вмещала их, по множеству стад их.
\par 8 И поселился Исав на горе Сеир, Исав, он же Едом.
\par 9 И вот родословие Исава, отца Идумеев, на горе Сеир.
\par 10 Вот имена сынов Исава: Елифаз, сын Ады, жены Исавовой, и Рагуил, сын Васемафы, жены Исавовой.
\par 11 У Елифаза были сыновья: Феман, Омар, Цефо, Гафам и Кеназ.
\par 12 Фамна же была наложница Елифаза, сына Исавова, и родила Елифазу Амалика. Вот сыновья Ады, жены Исавовой.
\par 13 И вот сыновья Рагуила: Нахаф и Зерах, Шамма и Миза. Это сыновья Васемафы, жены Исавовой.
\par 14 И сии были сыновья Оливемы, дочери Аны, сына Цивеонова, жены Исавовой: она родила Исаву Иеуса, Иеглома и Корея.
\par 15 Вот старейшины сынов Исавовых. Сыновья Елифаза, первенца Исавова: старейшина Феман, старейшина Омар, старейшина Цефо, старейшина Кеназ,
\par 16 старейшина Корей, старейшина Гафам, старейшина Амалик. Сии старейшины Елифазовы в земле Едома; сии сыновья Ады.
\par 17 Сии сыновья Рагуила, сына Исавова: старейшина Нахаф, старейшина Зерах, старейшина Шамма, старейшина Миза. Сии старейшины Рагуиловы в земле Едома; сии сыновья Васемафы, жены Исавовой.
\par 18 Сии сыновья Оливемы, жены Исавовой: старейшина Иеус, старейшина Иеглом, старейшина Корей. Сии старейшины Оливемы, дочери Аны, жены Исавовой.
\par 19 Вот сыновья Исава, и вот старейшины их. Это Едом.
\par 20 Сии сыновья Сеира Хорреянина, жившие в земле той: Лотан, Шовал, Цивеон, Ана,
\par 21 Дишон, Эцер и Дишан. Сии старейшины Хорреев, сынов Сеира, в земле Едома.
\par 22 Сыновья Лотана были: Хори и Геман; а сестра у Лотана: Фамна.
\par 23 Сии сыновья Шовала: Алван, Манахаф, Эвал, Шефо и Онам.
\par 24 Сии сыновья Цивеона: Аиа и Ана. Это тот Ана, который нашел теплые воды в пустыне, когда пас ослов Цивеона, отца своего.
\par 25 Сии дети Аны: Дишон и Оливема, дочь Аны.
\par 26 Сии сыновья Дишона: Хемдан, Эшбан, Ифран и Херан.
\par 27 Сии сыновья Эцера: Билган, Зааван, и Акан.
\par 28 Сии сыновья Дишана: Уц и Аран.
\par 29 Сии старейшины Хорреев: старейшина Лотан, старейшина Шовал, старейшина Цивеон, старейшина Ана,
\par 30 старейшина Дишон, старейшина Эцер, старейшина Дишан. Вот старейшины Хорреев, по старшинствам их в земле Сеир.
\par 31 Вот цари, царствовавшие в земле Едома, прежде царствования царей у сынов Израилевых:
\par 32 царствовал в Едоме Бела, сын Веоров, а имя городу его Дингава.
\par 33 И умер Бела, и воцарился по нем Иовав, сын Зераха, из Восоры.
\par 34 Умер Иовав, и воцарился по нем Хушам, из земли Феманитян.
\par 35 И умер Хушам, и воцарился по нем Гадад, сын Бедадов, который поразил Мадианитян на поле Моава; имя городу его Авиф.
\par 36 И умер Гадад, и воцарился по нем Самла из Масреки.
\par 37 И умер Самла, и воцарился по нем Саул из Реховофа, что при реке.
\par 38 И умер Саул, и воцарился по нем Баал-Ханан, сын Ахбора.
\par 39 И умер Баал-Ханан, сын Ахбора, и воцарился по нем Гадар: имя городу его Пау; имя жене его Мегетавеель, дочь Матреды, сына Мезагава.
\par 40 Сии имена старейшин Исавовых, по племенам их, по местам их, по именам их: старейшина Фимна, старейшина Алва, старейшина Иетеф,
\par 41 старейшина Оливема, старейшина Эла, старейшина Пинон,
\par 42 старейшина Кеназ, старейшина Феман, старейшина Мивцар,
\par 43 старейшина Магдиил, старейшина Ирам. Вот старейшины Идумейские, по их селениям, в земле обладания их. Вот Исав, отец Идумеев.

\chapter{37}

\par 1 Иаков жил в земле странствования отца своего, в земле Ханаанской.
\par 2 Вот житие Иакова. Иосиф, семнадцати лет, пас скот вместе с братьями своими, будучи отроком, с сыновьями Валлы и с сыновьями Зелфы, жен отца своего. И доводил Иосиф худые о них слухи до отца их.
\par 3 Израиль любил Иосифа более всех сыновей своих, потому что он был сын старости его, --и сделал ему разноцветную одежду.
\par 4 И увидели братья его, что отец их любит его более всех братьев его; и возненавидели его и не могли говорить с ним дружелюбно.
\par 5 И видел Иосиф сон, и рассказал братьям своим: и они возненавидели его еще более.
\par 6 Он сказал им: выслушайте сон, который я видел:
\par 7 вот, мы вяжем снопы посреди поля; и вот, мой сноп встал и стал прямо; и вот, ваши снопы стали кругом и поклонились моему снопу.
\par 8 И сказали ему братья его: неужели ты будешь царствовать над нами? неужели будешь владеть нами? И возненавидели его еще более за сны его и за слова его.
\par 9 И видел он еще другой сон и рассказал его братьям своим, говоря: вот, я видел еще сон: вот, солнце и луна и одиннадцать звезд поклоняются мне.
\par 10 И он рассказал отцу своему и братьям своим; и побранил его отец его и сказал ему: что это за сон, который ты видел? неужели я и твоя мать, и твои братья придем поклониться тебе до земли?
\par 11 Братья его досадовали на него, а отец его заметил это слово.
\par 12 Братья его пошли пасти скот отца своего в Сихем.
\par 13 И сказал Израиль Иосифу: братья твои не пасут ли в Сихеме? пойди, я пошлю тебя к ним. Он отвечал ему: вот я.
\par 14 И сказал ему: пойди, посмотри, здоровы ли братья твои и цел ли скот, и принеси мне ответ. И послал его из долины Хевронской; и он пришел в Сихем.
\par 15 И нашел его некто блуждающим в поле, и спросил его тот человек, говоря: чего ты ищешь?
\par 16 Он сказал: я ищу братьев моих; скажи мне, где они пасут?
\par 17 И сказал тот человек: они ушли отсюда, ибо я слышал, как они говорили: пойдем в Дофан. И пошел Иосиф за братьями своими и нашел их в Дофане.
\par 18 И увидели они его издали, и прежде нежели он приблизился к ним, стали умышлять против него, чтобы убить его.
\par 19 И сказали друг другу: вот, идет сновидец;
\par 20 пойдем теперь, и убьем его, и бросим его в какой-нибудь ров, и скажем, что хищный зверь съел его; и увидим, что будет из его снов.
\par 21 И услышал [сие] Рувим и избавил его от рук их, сказав: не убьем его.
\par 22 И сказал им Рувим: не проливайте крови; бросьте его в ров, который в пустыне, а руки не налагайте на него. [Сие говорил он], чтобы избавить его от рук их и возвратить его к отцу его.
\par 23 Когда Иосиф пришел к братьям своим, они сняли с Иосифа одежду его, одежду разноцветную, которая была на нем,
\par 24 и взяли его и бросили его в ров; ров же тот был пуст; воды в нем не было.
\par 25 И сели они есть хлеб, и, взглянув, увидели, вот, идет из Галаада караван Измаильтян, и верблюды их несут стираксу, бальзам и ладан: идут они отвезти это в Египет.
\par 26 И сказал Иуда братьям своим: что пользы, если мы убьем брата нашего и скроем кровь его?
\par 27 Пойдем, продадим его Измаильтянам, а руки наши да не будут на нем, ибо он брат наш, плоть наша. Братья его послушались
\par 28 и, когда проходили купцы Мадиамские, вытащили Иосифа изо рва и продали Иосифа Измаильтянам за двадцать сребренников; а они отвели Иосифа в Египет.
\par 29 Рувим же пришел опять ко рву; и вот, нет Иосифа во рве. И разодрал он одежды свои,
\par 30 и возвратился к братьям своим, и сказал: отрока нет, а я, куда я денусь?
\par 31 И взяли одежду Иосифа, и закололи козла, и вымарали одежду кровью;
\par 32 и послали разноцветную одежду, и доставили к отцу своему, и сказали: мы это нашли; посмотри, сына ли твоего эта одежда, или нет.
\par 33 Он узнал ее и сказал: [это] одежда сына моего; хищный зверь съел его; верно, растерзан Иосиф.
\par 34 И разодрал Иаков одежды свои, и возложил вретище на чресла свои, и оплакивал сына своего многие дни.
\par 35 И собрались все сыновья его и все дочери его, чтобы утешить его; но он не хотел утешиться и сказал: с печалью сойду к сыну моему в преисподнюю. Так оплакивал его отец его.
\par 36 Мадианитяне же продали его в Египте Потифару, царедворцу фараонову, начальнику телохранителей.

\chapter{38}

\par 1 В то время Иуда отошел от братьев своих и поселился близ одного Одолламитянина, которому имя: Хира.
\par 2 И увидел там Иуда дочь одного Хананеянина, которому имя: Шуа; и взял ее и вошел к ней.
\par 3 Она зачала и родила сына; и он нарек ему имя: Ир.
\par 4 И зачала опять, и родила сына, и нарекла ему имя: Онан.
\par 5 И еще родила сына и нарекла ему имя: Шела. Иуда был в Хезиве, когда она родила его.
\par 6 И взял Иуда жену Иру, первенцу своему; имя ей Фамарь.
\par 7 Ир, первенец Иудин, был неугоден пред очами Господа, и умертвил его Господь.
\par 8 И сказал Иуда Онану: войди к жене брата твоего, женись на ней, как деверь, и восстанови семя брату твоему.
\par 9 Онан знал, что семя будет не ему, и потому, когда входил к жене брата своего, изливал на землю, чтобы не дать семени брату своему.
\par 10 Зло было пред очами Господа то, что он делал; и Он умертвил и его.
\par 11 И сказал Иуда Фамари, невестке своей: живи вдовою в доме отца твоего, пока подрастет Шела, сын мой. Ибо он сказал: не умер бы и он подобно братьям его. Фамарь пошла и стала жить в доме отца своего.
\par 12 Прошло много времени, и умерла дочь Шуи, жена Иудина. Иуда, утешившись, пошел в Фамну к стригущим скот его, сам и Хира, друг его, Одолламитянин.
\par 13 И уведомили Фамарь, говоря: вот, свекор твой идет в Фамну стричь скот свой.
\par 14 И сняла она с себя одежду вдовства своего, покрыла себя покрывалом и, закрывшись, села у ворот Енаима, что на дороге в Фамну. Ибо видела, что Шела вырос, и она не дана ему в жену.
\par 15 И увидел ее Иуда и почел ее за блудницу, потому что она закрыла лице свое.
\par 16 Он поворотил к ней и сказал: войду я к тебе. Ибо не знал, что это невестка его. Она сказала: что ты дашь мне, если войдешь ко мне?
\par 17 Он сказал: я пришлю тебе козленка из стада. Она сказала: дашь ли ты мне залог, пока пришлешь?
\par 18 Он сказал: какой дать тебе залог? Она сказала: печать твою, и перевязь твою, и трость твою, которая в руке твоей. И дал он ей и вошел к ней; и она зачала от него.
\par 19 И, встав, пошла, сняла с себя покрывало свое и оделась в одежду вдовства своего.
\par 20 Иуда же послал козленка чрез друга своего Одолламитянина, чтобы взять залог из руки женщины, но он не нашел ее.
\par 21 И спросил жителей того места, говоря: где блудница, [которая] [была] в Енаиме при дороге? Но они сказали: здесь не было блудницы.
\par 22 И возвратился он к Иуде и сказал: я не нашел ее; да и жители места того сказали: здесь не было блудницы.
\par 23 Иуда сказал: пусть она возьмет себе, чтобы только не стали над нами смеяться; вот, я посылал этого козленка, но ты не нашел ее.
\par 24 Прошло около трех месяцев, и сказали Иуде, говоря: Фамарь, невестка твоя, впала в блуд, и вот, она беременна от блуда. Иуда сказал: выведите ее, и пусть она будет сожжена.
\par 25 Но когда повели ее, она послала сказать свекру своему: я беременна от того, чьи эти вещи. И сказала: узнавай, чья эта печать и перевязь и трость.
\par 26 Иуда узнал и сказал: она правее меня, потому что я не дал ее Шеле, сыну моему. И не познавал ее более.
\par 27 Во время родов ее оказалось, что близнецы в утробе ее.
\par 28 И во время родов ее показалась рука; и взяла повивальная бабка и навязала ему на руку красную нить, сказав: этот вышел первый.
\par 29 Но он возвратил руку свою; и вот, вышел брат его. И она сказала: как ты расторг себе преграду? И наречено ему имя: Фарес.
\par 30 Потом вышел брат его с красной нитью на руке. И наречено ему имя: Зара.

\chapter{39}

\par 1 Иосиф же отведен был в Египет, и купил его из рук Измаильтян, приведших его туда, Египтянин Потифар, царедворец фараонов, начальник телохранителей.
\par 2 И был Господь с Иосифом: он был успешен в делах и жил в доме господина своего, Египтянина.
\par 3 И увидел господин его, что Господь с ним и что всему, что он делает, Господь в руках его дает успех.
\par 4 И снискал Иосиф благоволение в очах его и служил ему. И он поставил его над домом своим, и все, что имел, отдал на руки его.
\par 5 И с того времени, как он поставил его над домом своим и над всем, что имел, Господь благословил дом Египтянина ради Иосифа, и было благословение Господне на всем, что имел он в доме и в поле.
\par 6 И оставил он все, что имел, в руках Иосифа и не знал при нем ничего, кроме хлеба, который он ел. Иосиф же был красив станом и красив лицем.
\par 7 И обратила взоры на Иосифа жена господина его и сказала: спи со мною.
\par 8 Но он отказался и сказал жене господина своего: вот, господин мой не знает при мне ничего в доме, и все, что имеет, отдал в мои руки;
\par 9 нет больше меня в доме сем; и он не запретил мне ничего, кроме тебя, потому что ты жена ему; как же сделаю я сие великое зло и согрешу пред Богом?
\par 10 Когда так она ежедневно говорила Иосифу, а он не слушался ее, чтобы спать с нею и быть с нею,
\par 11 случилось в один день, что он вошел в дом делать дело свое, а никого из домашних тут в доме не было;
\par 12 она схватила его за одежду его и сказала: ложись со мной. Но он, оставив одежду свою в руках ее, побежал и выбежал вон.
\par 13 Она же, увидев, что он оставил одежду свою в руках ее и побежал вон,
\par 14 кликнула домашних своих и сказала им так: посмотрите, он привел к нам Еврея ругаться над нами. Он пришел ко мне, чтобы лечь со мною, но я закричала громким голосом,
\par 15 и он, услышав, что я подняла вопль и закричала, оставил у меня одежду свою, и побежал, и выбежал вон.
\par 16 И оставила одежду его у себя до прихода господина его в дом свой.
\par 17 И пересказала ему те же слова, говоря: раб Еврей, которого ты привел к нам, приходил ко мне ругаться надо мною.
\par 18 но, когда я подняла вопль и закричала, он оставил у меня одежду свою и убежал вон.
\par 19 Когда господин его услышал слова жены своей, которые она сказала ему, говоря: так поступил со мною раб твой, то воспылал гневом;
\par 20 и взял Иосифа господин его и отдал его в темницу, где заключены узники царя. И был он там в темнице.
\par 21 И Господь был с Иосифом, и простер к нему милость, и даровал ему благоволение в очах начальника темницы.
\par 22 И отдал начальник темницы в руки Иосифу всех узников, находившихся в темнице, и во всем, что они там ни делали, он был распорядителем.
\par 23 Начальник темницы и не смотрел ни за чем, что было у него в руках, потому что Господь был с [Иосифом], и во всем, что он делал, Господь давал успех.

\chapter{40}

\par 1 После сего виночерпий царя Египетского и хлебодар провинились пред господином своим, царем Египетским.
\par 2 И прогневался фараон на двух царедворцев своих, на главного виночерпия и на главного хлебодара,
\par 3 и отдал их под стражу в дом начальника телохранителей, в темницу, в место, где заключен был Иосиф.
\par 4 Начальник телохранителей приставил к ним Иосифа, и он служил им. И пробыли они под стражею несколько времени.
\par 5 Однажды виночерпию и хлебодару царя Египетского, заключенным в темнице, виделись сны, каждому свой сон, обоим в одну ночь, каждому сон особенного значения.
\par 6 И пришел к ним Иосиф поутру, увидел их, и вот, они в смущении.
\par 7 И спросил он царедворцев фараоновых, находившихся с ним в доме господина его под стражею, говоря: отчего у вас сегодня печальные лица?
\par 8 Они сказали ему: нам виделись сны; а истолковать их некому. Иосиф сказал им: не от Бога ли истолкования? расскажите мне.
\par 9 И рассказал главный виночерпий Иосифу сон свой и сказал ему: мне снилось, вот виноградная лоза предо мною;
\par 10 на лозе три ветви; она развилась, показался на ней цвет, выросли и созрели на ней ягоды;
\par 11 и чаша фараонова в руке у меня; я взял ягод, выжал их в чашу фараонову и подал чашу в руку фараону.
\par 12 И сказал ему Иосиф: вот истолкование его: три ветви--это три дня;
\par 13 через три дня фараон вознесет главу твою и возвратит тебя на место твое, и ты подашь чашу фараонову в руку его, по прежнему обыкновению, когда ты был у него виночерпием;
\par 14 вспомни же меня, когда хорошо тебе будет, и сделай мне благодеяние, и упомяни обо мне фараону, и выведи меня из этого дома,
\par 15 ибо я украден из земли Евреев; а также и здесь ничего не сделал, за что бы бросить меня в темницу.
\par 16 Главный хлебодар увидел, что истолковал он хорошо, и сказал Иосифу: мне также снилось: вот на голове у меня три корзины решетчатых;
\par 17 в верхней корзине всякая пища фараонова, изделие пекаря, и птицы клевали ее из корзины на голове моей.
\par 18 И отвечал Иосиф и сказал: вот истолкование его: три корзины--это три дня;
\par 19 через три дня фараон снимет с тебя голову твою и повесит тебя на дереве, и птицы будут клевать плоть твою с тебя.
\par 20 На третий день, день рождения фараонова, сделал он пир для всех слуг своих и вспомнил о главном виночерпии и главном хлебодаре среди слуг своих;
\par 21 и возвратил главного виночерпия на прежнее место, и он подал чашу в руку фараону,
\par 22 а главного хлебодара повесил, как истолковал им Иосиф.
\par 23 И не вспомнил главный виночерпий об Иосифе, но забыл его.

\chapter{41}

\par 1 По прошествии двух лет фараону снилось: вот, он стоит у реки;
\par 2 и вот, вышли из реки семь коров, хороших видом и тучных плотью, и паслись в тростнике;
\par 3 но вот, после них вышли из реки семь коров других, худых видом и тощих плотью, и стали подле тех коров, на берегу реки;
\par 4 и съели коровы худые видом и тощие плотью семь коров хороших видом и тучных. И проснулся фараон,
\par 5 и заснул опять, и снилось ему в другой раз: вот, на одном стебле поднялось семь колосьев тучных и хороших;
\par 6 но вот, после них выросло семь колосьев тощих и иссушенных восточным ветром;
\par 7 и пожрали тощие колосья семь колосьев тучных и полных. И проснулся фараон и [понял, что] это сон.
\par 8 Утром смутился дух его, и послал он, и призвал всех волхвов Египта и всех мудрецов его, и рассказал им фараон сон свой; но не было никого, кто бы истолковал его фараону.
\par 9 И стал говорить главный виночерпий фараону и сказал: грехи мои вспоминаю я ныне;
\par 10 фараон прогневался на рабов своих и отдал меня и главного хлебодара под стражу в дом начальника телохранителей;
\par 11 и снился нам сон в одну ночь, мне и ему, каждому снился сон особенного значения;
\par 12 там же был с нами молодой Еврей, раб начальника телохранителей; мы рассказали ему сны наши, и он истолковал нам каждому соответственно с его сновидением;
\par 13 и как он истолковал нам, так и сбылось: я возвращен на место мое, а тот повешен.
\par 14 И послал фараон и позвал Иосифа. И поспешно вывели его из темницы. Он остригся и переменил одежду свою и пришел к фараону.
\par 15 Фараон сказал Иосифу: мне снился сон, и нет никого, кто бы истолковал его, а о тебе я слышал, что ты умеешь толковать сны.
\par 16 И отвечал Иосиф фараону, говоря: это не мое; Бог даст ответ во благо фараону.
\par 17 И сказал фараон Иосифу: мне снилось: вот, стою я на берегу реки;
\par 18 и вот, вышли из реки семь коров тучных плотью и хороших видом и паслись в тростнике;
\par 19 но вот, после них вышли семь коров других, худых, очень дурных видом и тощих плотью: я не видывал во всей земле Египетской таких худых, как они;
\par 20 и съели тощие и худые коровы прежних семь коров тучных;
\par 21 и вошли [тучные] в утробу их, но не приметно было, что они вошли в утробу их: они были так же худы видом, как и сначала. И я проснулся.
\par 22 [Потом] снилось мне: вот, на одном стебле поднялись семь колосьев полных и хороших;
\par 23 но вот, после них выросло семь колосьев тонких, тощих и иссушенных восточным ветром;
\par 24 и пожрали тощие колосья семь колосьев хороших. Я рассказал это волхвам, но никто не изъяснил мне.
\par 25 И сказал Иосиф фараону: сон фараонов один: что Бог сделает, то Он возвестил фараону.
\par 26 Семь коров хороших, это семь лет; и семь колосьев хороших, это семь лет: сон один;
\par 27 и семь коров тощих и худых, вышедших после тех, это семь лет, также и семь колосьев тощих и иссушенных восточным ветром, это семь лет голода.
\par 28 Вот почему сказал я фараону: что Бог сделает, то Он показал фараону.
\par 29 Вот, наступает семь лет великого изобилия во всей земле Египетской;
\par 30 после них настанут семь лет голода, и забудется все то изобилие в земле Египетской, и истощит голод землю,
\par 31 и неприметно будет прежнее изобилие на земле, по причине голода, который последует, ибо он будет очень тяжел.
\par 32 А что сон повторился фараону дважды, [это значит], что сие истинно слово Божие, и что вскоре Бог исполнит сие.
\par 33 И ныне да усмотрит фараон мужа разумного и мудрого и да поставит его над землею Египетскою.
\par 34 Да повелит фараон поставить над землею надзирателей и собирать в семь лет изобилия пятую часть с земли Египетской;
\par 35 пусть они берут всякий хлеб этих наступающих хороших годов и соберут в городах хлеб под ведение фараона в пищу, и пусть берегут;
\par 36 и будет сия пища в запас для земли на семь лет голода, которые будут в земле Египетской, дабы земля не погибла от голода.
\par 37 Сие понравилось фараону и всем слугам его.
\par 38 И сказал фараон слугам своим: найдем ли мы такого, как он, человека, в котором был бы Дух Божий?
\par 39 И сказал фараон Иосифу: так как Бог открыл тебе все сие, то нет столь разумного и мудрого, как ты;
\par 40 ты будешь над домом моим, и твоего слова держаться будет весь народ мой; только престолом я буду больше тебя.
\par 41 И сказал фараон Иосифу: вот, я поставляю тебя над всею землею Египетскою.
\par 42 И снял фараон перстень свой с руки своей и надел его на руку Иосифа; одел его в виссонные одежды, возложил золотую цепь на шею ему;
\par 43 велел везти его на второй из своих колесниц и провозглашать пред ним: преклоняйтесь! И поставил его над всею землею Египетскою.
\par 44 И сказал фараон Иосифу: я фараон; без тебя никто не двинет ни руки своей, ни ноги своей во всей земле Египетской.
\par 45 И нарек фараон Иосифу имя: Цафнаф-панеах, и дал ему в жену Асенефу, дочь Потифера, жреца Илиопольского. И пошел Иосиф по земле Египетской.
\par 46 Иосифу было тридцать лет от рождения, когда он предстал пред лице фараона, царя Египетского. И вышел Иосиф от лица фараонова и прошел по всей земле Египетской.
\par 47 Земля же в семь лет изобилия приносила [из зерна] по горсти.
\par 48 И собрал он всякий хлеб семи лет, которые были [плодородны] в земле Египетской, и положил хлеб в городах; в [каждом] городе положил хлеб полей, окружающих его.
\par 49 И скопил Иосиф хлеба весьма много, как песку морского, так что перестал и считать, ибо не стало счета.
\par 50 До наступления годов голода, у Иосифа родились два сына, которых родила ему Асенефа, дочь Потифера, жреца Илиопольского.
\par 51 И нарек Иосиф имя первенцу: Манассия, потому что [говорил он] Бог дал мне забыть все несчастья мои и весь дом отца моего.
\par 52 А другому нарек имя: Ефрем, потому что [говорил он] Бог сделал меня плодовитым в земле страдания моего.
\par 53 И прошли семь лет изобилия, которое было в земле Египетской,
\par 54 и наступили семь лет голода, как сказал Иосиф. И был голод во всех землях, а во всей земле Египетской был хлеб.
\par 55 Но когда и вся земля Египетская начала терпеть голод, то народ начал вопиять к фараону о хлебе. И сказал фараон всем Египтянам: пойдите к Иосифу и делайте, что он вам скажет.
\par 56 И был голод по всей земле; и отворил Иосиф все житницы, и стал продавать хлеб Египтянам. Голод же усиливался в земле Египетской.
\par 57 И из всех стран приходили в Египет покупать хлеб у Иосифа, ибо голод усилился по всей земле.

\chapter{42}

\par 1 И узнал Иаков, что в Египте есть хлеб, и сказал Иаков сыновьям своим: что вы смотрите?
\par 2 И сказал: вот, я слышал, что есть хлеб в Египте; пойдите туда и купите нам оттуда хлеба, чтобы нам жить и не умереть.
\par 3 Десять братьев Иосифовых пошли купить хлеба в Египте,
\par 4 а Вениамина, брата Иосифова, не послал Иаков с братьями его, ибо сказал: не случилось бы с ним беды.
\par 5 И пришли сыны Израилевы покупать хлеб, вместе с другими пришедшими, ибо в земле Ханаанской был голод.
\par 6 Иосиф же был начальником в земле той; он и продавал хлеб всему народу земли. Братья Иосифа пришли и поклонились ему лицем до земли.
\par 7 И увидел Иосиф братьев своих и узнал их; но показал, будто не знает их, и говорил с ними сурово и сказал им: откуда вы пришли? Они сказали: из земли Ханаанской, купить пищи.
\par 8 Иосиф узнал братьев своих, но они не узнали его.
\par 9 И вспомнил Иосиф сны, которые снились ему о них; и сказал им: вы соглядатаи, вы пришли высмотреть наготу земли сей.
\par 10 Они сказали ему: нет, господин наш; рабы твои пришли купить пищи;
\par 11 мы все дети одного человека; мы люди честные; рабы твои не бывали соглядатаями.
\par 12 Он сказал им: нет, вы пришли высмотреть наготу земли сей.
\par 13 Они сказали: нас, рабов твоих, двенадцать братьев; мы сыновья одного человека в земле Ханаанской, и вот, меньший теперь с отцом нашим, а одного не стало.
\par 14 И сказал им Иосиф: это самое я и говорил вам, сказав: вы соглядатаи;
\par 15 вот как вы будете испытаны: [клянусь] жизнью фараона, вы не выйдете отсюда, если не придет сюда меньший брат ваш;
\par 16 пошлите одного из вас, и пусть он приведет брата вашего, а вы будете задержаны; и откроется, правда ли у вас; и если нет, [то клянусь] жизнью фараона, что вы соглядатаи.
\par 17 И отдал их под стражу на три дня.
\par 18 И сказал им Иосиф в третий день: вот что сделайте, и останетесь живы, ибо я боюсь Бога:
\par 19 если вы люди честные, то один брат из вас пусть содержится в доме, где вы заключены; а вы пойдите, отвезите хлеб, ради голода семейств ваших;
\par 20 брата же вашего меньшого приведите ко мне, чтобы оправдались слова ваши и чтобы не умереть вам. Так они и сделали.
\par 21 И говорили они друг другу: точно мы наказываемся за грех против брата нашего; мы видели страдание души его, когда он умолял нас, но не послушали; за то и постигло нас горе сие.
\par 22 Рувим отвечал им и сказал: не говорил ли я вам: не грешите против отрока? но вы не послушались; вот, кровь его взыскивается.
\par 23 А того не знали они, что Иосиф понимает; ибо между ними был переводчик.
\par 24 И отошел от них, и заплакал. И возвратился к ним, и говорил с ними, и, взяв из них Симеона, связал его пред глазами их.
\par 25 И приказал Иосиф наполнить мешки их хлебом, а серебро их возвратить каждому в мешок его, и дать им запасов на дорогу. Так и сделано с ними.
\par 26 Они положили хлеб свой на ослов своих, и пошли оттуда.
\par 27 И открыл один [из них] мешок свой, чтобы дать корму ослу своему на ночлеге, и увидел серебро свое в отверстии мешка его,
\par 28 и сказал своим братьям: серебро мое возвращено; вот оно в мешке у меня. И смутилось сердце их, и они с трепетом друг другу говорили: что это Бог сделал с нами?
\par 29 И пришли к Иакову, отцу своему, в землю Ханаанскую и рассказали ему все случившееся с ними, говоря:
\par 30 начальствующий над тою землею говорил с нами сурово и принял нас за соглядатаев земли той.
\par 31 И сказали мы ему: мы люди честные; мы не бывали соглядатаями;
\par 32 нас двенадцать братьев, сыновей у отца нашего; одного не стало, а меньший теперь с отцом нашим в земле Ханаанской.
\par 33 И сказал нам начальствующий над тою землею: вот как узнаю я, честные ли вы люди: оставьте у меня одного брата из вас, а вы возьмите хлеб ради голода семейств ваших и пойдите,
\par 34 и приведите ко мне меньшого брата вашего; и узнаю я, что вы не соглядатаи, но люди честные; отдам вам брата вашего, и вы можете промышлять в этой земле.
\par 35 Когда же они опорожняли мешки свои, вот, у каждого узел серебра его в мешке его. И увидели они узлы серебра своего, они и отец их, и испугались.
\par 36 И сказал им Иаков, отец их: вы лишили меня детей: Иосифа нет, и Симеона нет, и Вениамина взять хотите, --все это на меня!
\par 37 И сказал Рувим отцу своему, говоря: убей двух моих сыновей, если я не приведу его к тебе; отдай его на мои руки; я возвращу его тебе.
\par 38 Он сказал: не пойдет сын мой с вами; потому что брат его умер, и он один остался; если случится с ним несчастье на пути, в который вы пойдете, то сведете вы седину мою с печалью во гроб.

\chapter{43}

\par 1 Голод усилился на земле.
\par 2 И когда они съели хлеб, который привезли из Египта, тогда отец их сказал им: пойдите опять, купите нам немного пищи.
\par 3 И сказал ему Иуда, говоря: тот человек решительно объявил нам, сказав: не являйтесь ко мне на лице, если брата вашего не будет с вами.
\par 4 Если пошлешь с нами брата нашего, то пойдем и купим тебе пищи,
\par 5 а если не пошлешь, то не пойдем, ибо тот человек сказал нам: не являйтесь ко мне на лице, если брата вашего не будет с вами.
\par 6 Израиль сказал: для чего вы сделали мне такое зло, сказав тому человеку, что у вас есть еще брат?
\par 7 Они сказали: расспрашивал тот человек о нас и о родстве нашем, говоря: жив ли еще отец ваш? есть ли у вас брат? Мы и рассказали ему по этим расспросам. Могли ли мы знать, что он скажет: приведите брата вашего?
\par 8 Иуда же сказал Израилю, отцу своему: отпусти отрока со мною, и мы встанем и пойдем, и живы будем и не умрем и мы, и ты, и дети наши;
\par 9 я отвечаю за него, из моих рук потребуешь его; если я не приведу его к тебе и не поставлю его пред лицем твоим, то останусь я виновным пред тобою во все дни жизни;
\par 10 если бы мы не медлили, то уже сходили бы два раза.
\par 11 Израиль, отец их, сказал им: если так, то вот что сделайте: возьмите с собою плодов земли сей и отнесите в дар тому человеку несколько бальзама и несколько меду, стираксы и ладану, фисташков и миндальных орехов;
\par 12 возьмите и другое серебро в руки ваши; а серебро, обратно положенное в отверстие мешков ваших, возвратите руками вашими: может быть, это недосмотр;
\par 13 и брата вашего возьмите и, встав, пойдите опять к человеку тому;
\par 14 Бог же Всемогущий да даст вам найти милость у человека того, чтобы он отпустил вам и другого брата вашего и Вениамина, а мне если уже быть бездетным, то пусть буду бездетным.
\par 15 И взяли те люди дары эти, и серебра вдвое взяли в руки свои, и Вениамина, и встали, пошли в Египет и предстали пред лице Иосифа.
\par 16 Иосиф, увидев между ними Вениамина, сказал начальнику дома своего: введи сих людей в дом и заколи что-нибудь из скота, и приготовь, потому что со мною будут есть эти люди в полдень.
\par 17 И сделал человек тот, как сказал Иосиф, и ввел человек тот людей сих в дом Иосифов.
\par 18 И испугались люди эти, что ввели их в дом Иосифов, и сказали: это за серебро, возвращенное прежде в мешки наши, ввели нас, чтобы придраться к нам и напасть на нас, и взять нас в рабство, и ослов наших.
\par 19 И подошли они к начальнику дома Иосифова, и стали говорить ему у дверей дома,
\par 20 и сказали: послушай, господин наш, мы приходили уже прежде покупать пищи,
\par 21 и случилось, что, когда пришли мы на ночлег и открыли мешки наши, --вот серебро каждого в отверстии мешка его, серебро наше по весу его, и мы возвращаем его своими руками;
\par 22 а для покупки пищи мы принесли другое серебро в руках наших, мы не знаем, кто положил серебро наше в мешки наши.
\par 23 Он сказал: будьте спокойны, не бойтесь; Бог ваш и Бог отца вашего дал вам клад в мешках ваших; серебро ваше дошло до меня. И привел к ним Симеона.
\par 24 И ввел тот человек людей сих в дом Иосифов и дал воды, и они омыли ноги свои; и дал корму ослам их.
\par 25 И они приготовили дары к приходу Иосифа в полдень, ибо слышали, что там будут есть хлеб.
\par 26 И пришел Иосиф домой; и они принесли ему в дом дары, которые были на руках их, и поклонились ему до земли.
\par 27 Он спросил их о здоровье и сказал: здоров ли отец ваш старец, о котором вы говорили? жив ли еще он?
\par 28 Они сказали: здоров раб твой, отец наш; еще жив. И преклонились они и поклонились.
\par 29 И поднял глаза свои, и увидел Вениамина, брата своего, сына матери своей, и сказал: это брат ваш меньший, о котором вы сказывали мне? И сказал: да будет милость Божия с тобою, сын мой!
\par 30 И поспешно удалился Иосиф, потому что воскипела любовь к брату его, и он готов был заплакать, и вошел он во внутреннюю комнату и плакал там.
\par 31 И умыв лице свое, вышел, и скрепился и сказал: подавайте кушанье.
\par 32 И подали ему особо, и им особо, и Египтянам, обедавшим с ним, особо, ибо Египтяне не могут есть с Евреями, потому что это мерзость для Египтян.
\par 33 И сели они пред ним, первородный по первородству его, и младший по молодости его, и дивились эти люди друг пред другом.
\par 34 И посылались им кушанья от него, и доля Вениамина была впятеро больше долей каждого из них. И пили, и довольно пили они с ним.

\chapter{44}

\par 1 И приказал [Иосиф] начальнику дома своего, говоря: наполни мешки этих людей пищею, сколько они могут нести, и серебро каждого положи в отверстие мешка его,
\par 2 а чашу мою, чашу серебряную, положи в отверстие мешка к младшему вместе с серебром за купленный им хлеб. И сделал тот по слову Иосифа, которое сказал он.
\par 3 Утром, когда рассвело, эти люди были отпущены, они и ослы их.
\par 4 Еще не далеко отошли они от города, как Иосиф сказал начальнику дома своего: ступай, догоняй этих людей и, когда догонишь, скажи им: для чего вы заплатили злом за добро?
\par 5 Не та ли это, из которой пьет господин мой и он гадает на ней? Худо это вы сделали.
\par 6 Он догнал их и сказал им эти слова.
\par 7 Они сказали ему: для чего господин наш говорит такие слова? Нет, рабы твои не сделают такого дела.
\par 8 Вот, серебро, найденное нами в отверстии мешков наших, мы обратно принесли тебе из земли Ханаанской: как же нам украсть из дома господина твоего серебро или золото?
\par 9 У кого из рабов твоих найдется, тому смерть, и мы будем рабами господину нашему.
\par 10 Он сказал: хорошо; как вы сказали, так пусть и будет: у кого найдется [чаша], тот будет мне рабом, а вы будете не виноваты.
\par 11 Они поспешно спустили каждый свой мешок на землю и открыли каждый свой мешок.
\par 12 Он обыскал, начал со старшего и окончил младшим; и нашлась чаша в мешке Вениаминовом.
\par 13 И разодрали они одежды свои, и, возложив каждый на осла своего ношу, возвратились в город.
\par 14 И пришли Иуда и братья его в дом Иосифа, который был еще дома, и пали пред ним на землю.
\par 15 Иосиф сказал им: что это вы сделали? разве вы не знали, что такой человек, как я, конечно угадает?
\par 16 Иуда сказал: что нам сказать господину нашему? что говорить? чем оправдываться? Бог нашел неправду рабов твоих; вот, мы рабы господину нашему, и мы, и тот, в чьих руках нашлась чаша.
\par 17 Но [Иосиф] сказал: нет, я этого не сделаю; тот, в чьих руках нашлась чаша, будет мне рабом, а вы пойдите с миром к отцу вашему.
\par 18 И подошел Иуда к нему и сказал: господин мой, позволь рабу твоему сказать слово в уши господина моего, и не прогневайся на раба твоего, ибо ты то же, что фараон.
\par 19 Господин мой спрашивал рабов своих, говоря: есть ли у вас отец или брат?
\par 20 Мы сказали господину нашему, что у нас есть отец престарелый, и младший сын, сын старости, которого брат умер, а он остался один [от] матери своей, и отец любит его.
\par 21 Ты же сказал рабам твоим: приведите его ко мне, чтобы мне взглянуть на него.
\par 22 Мы сказали господину нашему: отрок не может оставить отца своего, и если он оставит отца своего, то сей умрет.
\par 23 Но ты сказал рабам твоим: если не придет с вами меньший брат ваш, то вы более не являйтесь ко мне на лице.
\par 24 Когда мы пришли к рабу твоему, отцу нашему, то пересказали ему слова господина моего.
\par 25 И сказал отец наш: пойдите опять, купите нам немного пищи.
\par 26 Мы сказали: нельзя нам идти; а если будет с нами меньший брат наш, то пойдем; потому что нельзя нам видеть лица того человека, если не будет с нами меньшого брата нашего.
\par 27 И сказал нам раб твой, отец наш: вы знаете, что жена моя родила мне двух [сынов];
\par 28 один пошел от меня, и я сказал: верно он растерзан; и я не видал его доныне;
\par 29 если и сего возьмете от глаз моих, и случится с ним несчастье, то сведете вы седину мою с горестью во гроб.
\par 30 Теперь если я приду к рабу твоему, отцу нашему, и не будет с нами отрока, с душею которого связана душа его,
\par 31 то он, увидев, что нет отрока, умрет; и сведут рабы твои седину раба твоего, отца нашего, с печалью во гроб.
\par 32 Притом я, раб твой, взялся отвечать за отрока отцу моему, сказав: если не приведу его к тебе, то останусь я виновным пред отцом моим во все дни жизни.
\par 33 Итак пусть я, раб твой, вместо отрока останусь рабом у господина моего, а отрок пусть идет с братьями своими:
\par 34 ибо как пойду я к отцу моему, когда отрока не будет со мною? я увидел бы бедствие, которое постигло бы отца моего.

\chapter{45}

\par 1 Иосиф не мог более удерживаться при всех стоявших около него и закричал: удалите от меня всех. И не оставалось при Иосифе никого, когда он открылся братьям своим.
\par 2 И громко зарыдал он, и услышали Египтяне, и услышал дом фараонов.
\par 3 И сказал Иосиф братьям своим: я--Иосиф, жив ли еще отец мой? Но братья его не могли отвечать ему, потому что они смутились пред ним.
\par 4 И сказал Иосиф братьям своим: подойдите ко мне. Они подошли. Он сказал: я--Иосиф, брат ваш, которого вы продали в Египет;
\par 5 но теперь не печальтесь и не жалейте о том, что вы продали меня сюда, потому что Бог послал меня перед вами для сохранения вашей жизни;
\par 6 ибо теперь два года голода на земле: еще пять лет, в которые ни орать, ни жать не будут;
\par 7 Бог послал меня перед вами, чтобы оставить вас на земле и сохранить вашу жизнь великим избавлением.
\par 8 Итак не вы послали меня сюда, но Бог, Который и поставил меня отцом фараону и господином во всем доме его и владыкою во всей земле Египетской.
\par 9 Идите скорее к отцу моему и скажите ему: так говорит сын твой Иосиф: Бог поставил меня господином над всем Египтом; приди ко мне, не медли;
\par 10 ты будешь жить в земле Гесем; и будешь близ меня, ты, и сыны твои, и сыны сынов твоих, и мелкий и крупный скот твой, и все твое;
\par 11 и прокормлю тебя там, ибо голод будет еще пять лет, чтобы не обнищал ты и дом твой и все твое.
\par 12 И вот, очи ваши и очи брата моего Вениамина видят, что это мои уста говорят с вами;
\par 13 скажите же отцу моему о всей славе моей в Египте и о всем, что вы видели, и приведите скорее отца моего сюда.
\par 14 И пал он на шею Вениамину, брату своему, и плакал; и Вениамин плакал на шее его.
\par 15 И целовал всех братьев своих и плакал, обнимая их. Потом говорили с ним братья его.
\par 16 Дошел в дом фараона слух, что пришли братья Иосифа; и приятно было фараону и рабам его.
\par 17 И сказал фараон Иосифу: скажи братьям твоим: вот что сделайте: навьючьте скот ваш, и ступайте в землю Ханаанскую;
\par 18 и возьмите отца вашего и семейства ваши и придите ко мне; я дам вам лучшее в земле Египетской, и вы будете есть тук земли.
\par 19 Тебе же повелеваю сказать им: сделайте сие: возьмите себе из земли Египетской колесниц для детей ваших и для жен ваших, и привезите отца вашего и придите;
\par 20 и не жалейте вещей ваших, ибо лучшее из всей земли Египетской [дам] вам.
\par 21 Так и сделали сыны Израилевы. И дал им Иосиф колесницы по приказанию фараона, и дал им путевой запас,
\par 22 каждому из них он дал перемену одежд, а Вениамину дал триста сребренников и пять перемен одежд;
\par 23 также и отцу своему послал десять ослов, навьюченных лучшими [произведениями] Египетскими, и десять ослиц, навьюченных зерном, хлебом и припасами отцу своему на путь.
\par 24 И отпустил братьев своих, и они пошли. И сказал им: не ссорьтесь на дороге.
\par 25 И пошли они из Египта, и пришли в землю Ханаанскую к Иакову, отцу своему,
\par 26 и известили его, сказав: Иосиф жив, и теперь владычествует над всею землею Египетскою. Но сердце его смутилось, ибо он не верил им.
\par 27 Когда же они пересказали ему все слова Иосифа, которые он говорил им, и когда увидел колесницы, которые прислал Иосиф, чтобы везти его, тогда ожил дух Иакова, отца их,
\par 28 и сказал Израиль: довольно, еще жив сын мой Иосиф; пойду и увижу его, пока не умру.

\chapter{46}

\par 1 И отправился Израиль со всем, что у него было, и пришел в Вирсавию, и принес жертвы Богу отца своего Исаака.
\par 2 И сказал Бог Израилю в видении ночном: Иаков! Иаков! Он сказал: вот я.
\par 3 [Бог] сказал: Я Бог, Бог отца твоего; не бойся идти в Египет, ибо там произведу от тебя народ великий;
\par 4 Я пойду с тобою в Египет, Я и выведу тебя обратно. Иосиф своею рукою закроет глаза [твои].
\par 5 Иаков отправился из Вирсавии; и повезли сыны Израилевы Иакова отца своего, и детей своих, и жен своих на колесницах, которые послал фараон, чтобы привезти его.
\par 6 И взяли они скот свой и имущество свое, которое приобрели в земле Ханаанской, и пришли в Египет, --Иаков и весь род его с ним.
\par 7 Сынов своих и внуков своих с собою, дочерей своих и внучек своих и весь род свой привел он с собою в Египет.
\par 8 Вот имена сынов Израилевых, пришедших в Египет: Иаков и сыновья его. Первенец Иакова Рувим.
\par 9 Сыны Рувима: Ханох и Фаллу, Хецрон и Харми.
\par 10 Сыны Симеона: Иемуил и Иамин, и Огад, и Иахин, и Цохар, и Саул, сын Хананеянки.
\par 11 Сыны Левия: Гирсон, Кааф и Мерари.
\par 12 Сыны Иуды: Ир и Онан, и Шела, и Фарес, и Зара; но Ир и Онан умерли в земле Ханаанской. Сыны Фареса были: Есром и Хамул.
\par 13 Сыны Иссахара: Фола и Фува, Иов и Шимрон.
\par 14 Сыны Завулона: Серед и Елон, и Иахлеил.
\par 15 Это сыны Лии, которых она родила Иакову в Месопотамии, и Дину, дочь его. Всех душ сынов его и дочерей его--тридцать три.
\par 16 Сыны Гада: Цифион и Хагги, Шуни и Эцбон, Ери и Ароди и Арели.
\par 17 Сыны Асира: Имна и Ишва, и Ишви, и Бриа, и Серах, сестра их. Сыны Брии: Хевер и Малхиил.
\par 18 Это сыны Зелфы, которую Лаван дал Лии, дочери своей; она родила их Иакову шестнадцать душ.
\par 19 Сыны Рахили, жены Иакова: Иосиф и Вениамин.
\par 20 И родились у Иосифа в земле Египетской Манассия и Ефрем, которых родила ему Асенефа, дочь Потифера, жреца Илиопольского.
\par 21 Сыны Вениамина: Бела и Бехер и Ашбел; Гера и Нааман, Эхи и Рош, Муппим и Хуппим и Ард.
\par 22 Это сыны Рахили, которые родились у Иакова, всего четырнадцать душ.
\par 23 Сын Дана: Хушим.
\par 24 Сыны Неффалима: Иахцеил и Гуни, и Иецер, и Шиллем.
\par 25 Это сыны Валлы, которую дал Лаван дочери своей Рахили; она родила их Иакову всего семь душ.
\par 26 Всех душ, пришедших с Иаковом в Египет, которые произошли из чресл его, кроме жен сынов Иаковлевых, всего шестьдесят шесть душ.
\par 27 Сынов Иосифа, которые родились у него в Египте, две души. Всех душ дома Иаковлева, перешедших в Египет, семьдесят.
\par 28 Иуду послал он пред собою к Иосифу, чтобы он указал [путь] в Гесем. И пришли в землю Гесем.
\par 29 Иосиф запряг колесницу свою и выехал навстречу Израилю, отцу своему, в Гесем, и, увидев его, пал на шею его, и долго плакал на шее его.
\par 30 И сказал Израиль Иосифу: умру я теперь, увидев лице твое, ибо ты еще жив.
\par 31 И сказал Иосиф братьям своим и дому отца своего: я пойду, извещу фараона и скажу ему: братья мои и дом отца моего, которые были в земле Ханаанской, пришли ко мне;
\par 32 эти люди пастухи овец, ибо скотоводы они; и мелкий и крупный скот свой, и все, что у них, привели они.
\par 33 Если фараон призовет вас и скажет: какое занятие ваше?
\par 34 то вы скажите: [мы], рабы твои, скотоводами были от юности нашей доныне, и мы и отцы наши, чтобы вас поселили в земле Гесем. Ибо мерзость для Египтян всякий пастух овец.

\chapter{47}

\par 1 И пришел Иосиф и известил фараона и сказал: отец мой и братья мои, с мелким и крупным скотом своим и со всем, что у них, пришли из земли Ханаанской; и вот, они в земле Гесем.
\par 2 И из братьев своих он взял пять человек и представил их фараону.
\par 3 И сказал фараон братьям его: какое ваше занятие? Они сказали фараону: пастухи овец рабы твои, и мы и отцы наши.
\par 4 И сказали они фараону: мы пришли пожить в этой земле, потому что нет пажити для скота рабов твоих, ибо в земле Ханаанской сильный голод; итак позволь поселиться рабам твоим в земле Гесем.
\par 5 И сказал фараон Иосифу: отец твой и братья твои пришли к тебе;
\par 6 земля Египетская пред тобою; на лучшем месте земли посели отца твоего и братьев твоих; пусть живут они в земле Гесем; и если знаешь, что между ними есть способные люди, поставь их смотрителями над моим скотом.
\par 7 И привел Иосиф Иакова, отца своего, и представил его фараону; и благословил Иаков фараона.
\par 8 Фараон сказал Иакову: сколько лет жизни твоей?
\par 9 Иаков сказал фараону: дней странствования моего сто тридцать лет; малы и несчастны дни жизни моей и не достигли до лет жизни отцов моих во днях странствования их.
\par 10 И благословил фараона Иаков и вышел от фараона.
\par 11 И поселил Иосиф отца своего и братьев своих, и дал им владение в земле Египетской, в лучшей части земли, в земле Раамсес, как повелел фараон.
\par 12 И снабжал Иосиф отца своего и братьев своих и весь дом отца своего хлебом, по потребностям каждого семейства.
\par 13 И не было хлеба по всей земле, потому что голод весьма усилился, и изнурены были от голода земля Египетская и земля Ханаанская.
\par 14 Иосиф собрал все серебро, какое было в земле Египетской и в земле Ханаанской, за хлеб, который покупали, и внес Иосиф серебро в дом фараонов.
\par 15 И серебро истощилось в земле Египетской и в земле Ханаанской. Все Египтяне пришли к Иосифу и говорили: дай нам хлеба; зачем нам умирать пред тобою, потому что серебро вышло у нас?
\par 16 Иосиф сказал: пригоняйте скот ваш, и я буду давать вам за скот ваш, если серебро вышло у вас.
\par 17 И пригоняли они к Иосифу скот свой; и давал им Иосиф хлеб за лошадей, и за стада мелкого скота, и за стада крупного скота, и за ослов; и снабжал их хлебом в тот год за весь скот их.
\par 18 И прошел этот год; и пришли к нему на другой год и сказали ему: не скроем от господина нашего, что серебро истощилось и стада скота нашего у господина нашего; ничего не осталось у нас пред господином нашим, кроме тел наших и земель наших;
\par 19 для чего нам погибать в глазах твоих, и нам и землям нашим? купи нас и земли наши за хлеб, и мы с землями нашими будем рабами фараону, а ты дай нам семян, чтобы нам быть живыми и не умереть, и чтобы не опустела земля.
\par 20 И купил Иосиф всю землю Египетскую для фараона, потому что продали Египтяне каждый свое поле, ибо голод одолевал их. И досталась земля фараону.
\par 21 И народ сделал он рабами от одного конца Египта до другого.
\par 22 Только земли жрецов не купил, ибо жрецам от фараона положен был участок, и они питались своим участком, который дал им фараон; посему и не продали земли своей.
\par 23 И сказал Иосиф народу: вот, я купил теперь для фараона вас и землю вашу; вот вам семена, и засевайте землю;
\par 24 когда будет жатва, давайте пятую часть фараону, а четыре части останутся вам на засеяние полей, на пропитание вам и тем, кто в домах ваших, и на пропитание детям вашим.
\par 25 Они сказали: ты спас нам жизнь; да обретем милость в очах господина нашего и да будем рабами фараону.
\par 26 И поставил Иосиф в закон земле Египетской, даже до сего дня: пятую часть давать фараону, исключая только землю жрецов, которая не принадлежала фараону.
\par 27 И жил Израиль в земле Египетской, в земле Гесем, и владели они ею, и плодились, и весьма умножились.
\par 28 И жил Иаков в земле Египетской семнадцать лет; и было дней Иакова, годов жизни его, сто сорок семь лет.
\par 29 И пришло время Израилю умереть, и призвал он сына своего Иосифа и сказал ему: если я нашел благоволение в очах твоих, положи руку твою под стегно мое и [клянись], что ты окажешь мне милость и правду, не похоронишь меня в Египте,
\par 30 дабы мне лечь с отцами моими; вынесешь меня из Египта и похоронишь меня в их гробнице. [Иосиф] сказал: сделаю по слову твоему.
\par 31 И сказал: клянись мне. И клялся ему. И поклонился Израиль на возглавие постели.

\chapter{48}

\par 1 После того Иосифу сказали: вот, отец твой болен. И он взял с собою двух сынов своих, Манассию и Ефрема.
\par 2 Иакова известили и сказали: вот, сын твой Иосиф идет к тебе. Израиль собрал силы свои и сел на постели.
\par 3 И сказал Иаков Иосифу: Бог Всемогущий явился мне в Лузе, в земле Ханаанской, и благословил меня,
\par 4 и сказал мне: вот, Я распложу тебя, и размножу тебя, и произведу от тебя множество народов, и дам землю сию потомству твоему после тебя, в вечное владение.
\par 5 И ныне два сына твои, родившиеся тебе в земле Египетской, до моего прибытия к тебе в Египет, мои они; Ефрем и Манассия, как Рувим и Симеон, будут мои;
\par 6 дети же твои, которые родятся от тебя после них, будут твои; они под именем братьев своих будут именоваться в их уделе.
\par 7 Когда я шел из Месопотамии, умерла у меня Рахиль в земле Ханаанской, по дороге, не доходя несколько до Ефрафы, и я похоронил ее там на дороге к Ефрафе, что [ныне] Вифлеем.
\par 8 И увидел Израиль сыновей Иосифа и сказал: кто это?
\par 9 И сказал Иосиф отцу своему: это сыновья мои, которых Бог дал мне здесь. Иаков сказал: подведи их ко мне, и я благословлю их.
\par 10 Глаза же Израилевы притупились от старости; не мог он видеть [ясно. Иосиф] подвел их к нему, и он поцеловал их и обнял их.
\par 11 И сказал Израиль Иосифу: не надеялся я видеть твое лице; но вот, Бог показал мне и детей твоих.
\par 12 И отвел их Иосиф от колен его и поклонился ему лицем своим до земли.
\par 13 И взял Иосиф обоих, Ефрема в правую свою руку против левой Израиля, а Манассию в левую против правой Израиля, и подвел к нему.
\par 14 Но Израиль простер правую руку свою и положил на голову Ефрему, хотя сей был меньший, а левую на голову Манассии. С намерением положил он так руки свои, хотя Манассия был первенец.
\par 15 И благословил Иосифа и сказал: Бог, пред Которым ходили отцы мои Авраам и Исаак, Бог, пасущий меня с тех пор, как я существую, до сего дня,
\par 16 Ангел, избавляющий меня от всякого зла, да благословит отроков сих; да будет на них наречено имя мое и имя отцов моих Авраама и Исаака, и да возрастут они во множество посреди земли.
\par 17 И увидел Иосиф, что отец его положил правую руку свою на голову Ефрема; и прискорбно было ему это. И взял он руку отца своего, чтобы переложить ее с головы Ефрема на голову Манассии,
\par 18 и сказал Иосиф отцу своему: не так, отец мой, ибо это--первенец; положи на его голову правую руку твою.
\par 19 Но отец его не согласился и сказал: знаю, сын мой, знаю; и от него произойдет народ, и он будет велик; но меньший его брат будет больше его, и от семени его произойдет многочисленный народ.
\par 20 И благословил их в тот день, говоря: тобою будет благословлять Израиль, говоря: Бог да сотворит тебе, как Ефрему и Манассии. И поставил Ефрема выше Манассии.
\par 21 И сказал Израиль Иосифу: вот, я умираю; и Бог будет с вами и возвратит вас в землю отцов ваших;
\par 22 я даю тебе, преимущественно пред братьями твоими, один участок, который я взял из рук Аморреев мечом моим и луком моим.

\chapter{49}

\par 1 И призвал Иаков сыновей своих и сказал: соберитесь, и я возвещу вам, что будет с вами в грядущие дни;
\par 2 сойдитесь и послушайте, сыны Иакова, послушайте Израиля, отца вашего.
\par 3 Рувим, первенец мой! ты--крепость моя и начаток силы моей, верх достоинства и верх могущества;
\par 4 но ты бушевал, как вода, --не будешь преимуществовать, ибо ты взошел на ложе отца твоего, ты осквернил постель мою, взошел.
\par 5 Симеон и Левий братья, орудия жестокости мечи их;
\par 6 в совет их да не внидет душа моя, и к собранию их да не приобщится слава моя, ибо они во гневе своем убили мужа и по прихоти своей перерезали жилы тельца;
\par 7 проклят гнев их, ибо жесток, и ярость их, ибо свирепа; разделю их в Иакове и рассею их в Израиле.
\par 8 Иуда! тебя восхвалят братья твои. Рука твоя на хребте врагов твоих; поклонятся тебе сыны отца твоего.
\par 9 Молодой лев Иуда, с добычи, сын мой, поднимается. Преклонился он, лег, как лев и как львица: кто поднимет его?
\par 10 Не отойдет скипетр от Иуды и законодатель от чресл его, доколе не приидет Примиритель, и Ему покорность народов.
\par 11 Он привязывает к виноградной лозе осленка своего и к лозе лучшего винограда сына ослицы своей; моет в вине одежду свою и в крови гроздов одеяние свое;
\par 12 блестящи очи [его] от вина, и белы зубы от молока.
\par 13 Завулон при береге морском будет жить и у пристани корабельной, и предел его до Сидона.
\par 14 Иссахар осел крепкий, лежащий между протоками вод;
\par 15 и увидел он, что покой хорош, и что земля приятна: и преклонил плечи свои для ношения бремени и стал работать в уплату дани.
\par 16 Дан будет судить народ свой, как одно из колен Израиля;
\par 17 Дан будет змеем на дороге, аспидом на пути, уязвляющим ногу коня, так что всадник его упадет назад.
\par 18 На помощь твою надеюсь, Господи!
\par 19 Гад, --толпа будет теснить его, но он оттеснит ее по пятам.
\par 20 Для Асира--слишком тучен хлеб его, и он будет доставлять царские яства.
\par 21 Неффалим--теревинф рослый, распускающий прекрасные ветви.
\par 22 Иосиф--отрасль плодоносного [дерева], отрасль плодоносного [дерева] над источником; ветви его простираются над стеною;
\par 23 огорчали его, и стреляли и враждовали на него стрельцы,
\par 24 но тверд остался лук его, и крепки мышцы рук его, от рук мощного [Бога] Иаковлева. Оттуда Пастырь и твердыня Израилева,
\par 25 от Бога отца твоего, [Который] и да поможет тебе, и от Всемогущего, Который и да благословит тебя благословениями небесными свыше, благословениями бездны, лежащей долу, благословениями сосцов и утробы,
\par 26 благословениями отца твоего, которые превышают благословения гор древних и приятности холмов вечных; да будут они на голове Иосифа и на темени избранного между братьями своими.
\par 27 Вениамин, хищный волк, утром будет есть ловитву и вечером будет делить добычу.
\par 28 Вот все двенадцать колен Израилевых; и вот что сказал им отец их; и благословил их, и дал им благословение, каждому свое.
\par 29 И заповедал он им и сказал им: я прилагаюсь к народу моему; похороните меня с отцами моими в пещере, которая на поле Ефрона Хеттеянина,
\par 30 в пещере, которая на поле Махпела, что пред Мамре, в земле Ханаанской, которую купил Авраам с полем у Ефрона Хеттеянина в собственность для погребения;
\par 31 там похоронили Авраама и Сарру, жену его; там похоронили Исаака и Ревекку, жену его; и там похоронил я Лию;
\par 32 это поле и пещера, которая на нем, куплена у сынов Хеттеевых.
\par 33 И окончил Иаков завещание сыновьям своим, и положил ноги свои на постель, и скончался, и приложился к народу своему.

\chapter{50}

\par 1 Иосиф пал на лице отца своего, и плакал над ним, и целовал его.
\par 2 И повелел Иосиф слугам своим--врачам, бальзамировать отца его; и врачи набальзамировали Израиля.
\par 3 И исполнилось ему сорок дней, ибо столько дней употребляется на бальзамирование, и оплакивали его Египтяне семьдесят дней.
\par 4 Когда же прошли дни плача по нем, Иосиф сказал придворным фараона, говоря: если я обрел благоволение в очах ваших, то скажите фараону так:
\par 5 отец мой заклял меня, сказав: вот, я умираю; во гробе моем, который я выкопал себе в земле Ханаанской, там похорони меня. И теперь хотел бы я пойти и похоронить отца моего и возвратиться.
\par 6 И сказал фараон: пойди и похорони отца твоего, как он заклял тебя.
\par 7 И пошел Иосиф хоронить отца своего. И пошли с ним все слуги фараона, старейшины дома его и все старейшины земли Египетской,
\par 8 и весь дом Иосифа, и братья его, и дом отца его. Только детей своих и мелкий и крупный скот свой оставили в земле Гесем.
\par 9 С ним отправились также колесницы и всадники, так что сонм был весьма велик.
\par 10 И дошли они до Горен-гаатада при Иордане и плакали там плачем великим и весьма сильным; и сделал [Иосиф] плач по отце своем семь дней.
\par 11 И видели жители земли той, Хананеи, плач в Горен-гаатаде, и сказали: велик плач этот у Египтян! Посему наречено имя [месту] тому: плач Египтян, что при Иордане.
\par 12 И сделали сыновья [Иакова] с ним, как он заповедал им;
\par 13 и отнесли его сыновья его в землю Ханаанскую и похоронили его в пещере на поле Махпела, которую купил Авраам с полем в собственность для погребения у Ефрона Хеттеянина, пред Мамре.
\par 14 И возвратился Иосиф в Египет, сам и братья его и все ходившие с ним хоронить отца его, после погребения им отца своего.
\par 15 И увидели братья Иосифовы, что умер отец их, и сказали: что, если Иосиф возненавидит нас и захочет отмстить нам за все зло, которое мы ему сделали?
\par 16 И послали они сказать Иосифу: отец твой пред смертью своею завещал, говоря:
\par 17 так скажите Иосифу: прости братьям твоим вину и грех их, так как они сделали тебе зло. И ныне прости вины рабов Бога отца твоего. Иосиф плакал, когда ему говорили это.
\par 18 Пришли и сами братья его, и пали пред лицем его, и сказали: вот, мы рабы тебе.
\par 19 И сказал Иосиф: не бойтесь, ибо я боюсь Бога;
\par 20 вот, вы умышляли против меня зло; но Бог обратил это в добро, чтобы сделать то, что теперь есть: сохранить жизнь великому числу людей;
\par 21 итак не бойтесь: я буду питать вас и детей ваших. И успокоил их и говорил по сердцу их.
\par 22 И жил Иосиф в Египте сам и дом отца его; жил же Иосиф всего сто десять лет.
\par 23 И видел Иосиф детей у Ефрема до третьего рода, также и сыновья Махира, сына Манассиина, родились на колени Иосифа.
\par 24 И сказал Иосиф братьям своим: я умираю, но Бог посетит вас и выведет вас из земли сей в землю, о которой клялся Аврааму, Исааку и Иакову.
\par 25 И заклял Иосиф сынов Израилевых, говоря: Бог посетит вас, и вынесите кости мои отсюда.
\par 26 И умер Иосиф ста десяти лет. И набальзамировали его и положили в ковчег в Египте.


\end{document}