\begin{document}

\title{Song of Solomon}

Son 1:1  Да лобзает он меня лобзанием уст своих! Ибо ласки твои лучше вина.
Son 1:2  От благовония мастей твоих имя твое--как разлитое миро; поэтому девицы любят тебя.
Son 1:3  Влеки меня, мы побежим за тобою; --царь ввел меня в чертоги свои, --будем восхищаться и радоваться тобою, превозносить ласки твои больше, нежели вино; достойно любят тебя!
Son 1:4  Дщери Иерусалимские! черна я, но красива, как шатры Кидарские, как завесы Соломоновы.
Son 1:5  Не смотрите на меня, что я смугла, ибо солнце опалило меня: сыновья матери моей разгневались на меня, поставили меня стеречь виноградники, --моего собственного виноградника я не стерегла.
Son 1:6  Скажи мне, ты, которого любит душа моя: где пасешь ты? где отдыхаешь в полдень? к чему мне быть скиталицею возле стад товарищей твоих?
Son 1:7  Если ты не знаешь этого, прекраснейшая из женщин, то иди себе по следам овец и паси козлят твоих подле шатров пастушеских.
Son 1:8  Кобылице моей в колеснице фараоновой я уподобил тебя, возлюбленная моя.
Son 1:9  Прекрасны ланиты твои под подвесками, шея твоя в ожерельях;
Son 1:10  золотые подвески мы сделаем тебе с серебряными блестками.
Son 1:11  Доколе царь был за столом своим, нард мой издавал благовоние свое.
Son 1:12  Мирровый пучок--возлюбленный мой у меня, у грудей моих пребывает.
Son 1:13  Как кисть кипера, возлюбленный мой у меня в виноградниках Енгедских.
Son 1:14  О, ты прекрасна, возлюбленная моя, ты прекрасна! глаза твои голубиные.
Son 1:15  О, ты прекрасен, возлюбленный мой, и любезен! и ложе у нас--зелень;
Son 1:16  кровли домов наших--кедры,
Son 1:17  потолки наши--кипарисы.
Son 2:1  Я нарцисс Саронский, лилия долин!
Son 2:2  Что лилия между тернами, то возлюбленная моя между девицами.
Son 2:3  Что яблоня между лесными деревьями, то возлюбленный мой между юношами. В тени ее люблю я сидеть, и плоды ее сладки для гортани моей.
Son 2:4  Он ввел меня в дом пира, и знамя его надо мною--любовь.
Son 2:5  Подкрепите меня вином, освежите меня яблоками, ибо я изнемогаю от любви.
Son 2:6  Левая рука его у меня под головою, а правая обнимает меня.
Son 2:7  Заклинаю вас, дщери Иерусалимские, сернами или полевыми ланями: не будите и не тревожьте возлюбленной, доколе ей угодно.
Son 2:8  Голос возлюбленного моего! вот, он идет, скачет по горам, прыгает по холмам.
Son 2:9  Друг мой похож на серну или на молодого оленя. Вот, он стоит у нас за стеною, заглядывает в окно, мелькает сквозь решетку.
Son 2:10  Возлюбленный мой начал говорить мне: встань, возлюбленная моя, прекрасная моя, выйди!
Son 2:11  Вот, зима уже прошла; дождь миновал, перестал;
Son 2:12  цветы показались на земле; время пения настало, и голос горлицы слышен в стране нашей;
Son 2:13  смоковницы распустили свои почки, и виноградные лозы, расцветая, издают благовоние. Встань, возлюбленная моя, прекрасная моя, выйди!
Son 2:14  Голубица моя в ущелье скалы под кровом утеса! покажи мне лице твое, дай мне услышать голос твой, потому что голос твой сладок и лице твое приятно.
Son 2:15  Ловите нам лисиц, лисенят, которые портят виноградники, а виноградники наши в цвете.
Son 2:16  Возлюбленный мой принадлежит мне, а я ему; он пасет между лилиями.
Son 2:17  Доколе день дышит [прохладою], и убегают тени, возвратись, будь подобен серне или молодому оленю на расселинах гор.
Son 3:1  На ложе моем ночью искала я того, которого любит душа моя, искала его и не нашла его.
Son 3:2  Встану же я, пойду по городу, по улицам и площадям, и буду искать того, которого любит душа моя; искала я его и не нашла его.
Son 3:3  Встретили меня стражи, обходящие город: `не видали ли вы того, которого любит душа моя?'
Son 3:4  Но едва я отошла от них, как нашла того, которого любит душа моя, ухватилась за него, и не отпустила его, доколе не привела его в дом матери моей и во внутренние комнаты родительницы моей.
Son 3:5  Заклинаю вас, дщери Иерусалимские, сернами или полевыми ланями: не будите и не тревожьте возлюбленной, доколе ей угодно.
Son 3:6  Кто эта, восходящая от пустыни как бы столбы дыма, окуриваемая миррою и фимиамом, всякими порошками мироварника?
Son 3:7  Вот одр его--Соломона: шестьдесят сильных вокруг него, из сильных Израилевых.
Son 3:8  Все они держат по мечу, опытны в бою; у каждого меч при бедре его ради страха ночного.
Son 3:9  Носильный одр сделал себе царь Соломон из дерев Ливанских;
Son 3:10  столпцы его сделал из серебра, локотники его из золота, седалище его из пурпуровой ткани; внутренность его убрана с любовью дщерями Иерусалимскими.
Son 3:11  Пойдите и посмотрите, дщери Сионские, на царя Соломона в венце, которым увенчала его мать его в день бракосочетания его, в день, радостный для сердца его.
Son 4:1  О, ты прекрасна, возлюбленная моя, ты прекрасна! глаза твои голубиные под кудрями твоими; волосы твои--как стадо коз, сходящих с горы Галаадской;
Son 4:2  зубы твои--как стадо выстриженных овец, выходящих из купальни, из которых у каждой пара ягнят, и бесплодной нет между ними;
Son 4:3  как лента алая губы твои, и уста твои любезны; как половинки гранатового яблока--ланиты твои под кудрями твоими;
Son 4:4  шея твоя--как столп Давидов, сооруженный для оружий, тысяча щитов висит на нем--все щиты сильных;
Son 4:5  два сосца твои--как двойни молодой серны, пасущиеся между лилиями.
Son 4:6  Доколе день дышит [прохладою], и убегают тени, пойду я на гору мирровую и на холм фимиама.
Son 4:7  Вся ты прекрасна, возлюбленная моя, и пятна нет на тебе!
Son 4:8  Со мною с Ливана, невеста! со мною иди с Ливана! спеши с вершины Аманы, с вершины Сенира и Ермона, от логовищ львиных, от гор барсовых!
Son 4:9  Пленила ты сердце мое, сестра моя, невеста! пленила ты сердце мое одним взглядом очей твоих, одним ожерельем на шее твоей.
Son 4:10  О, как любезны ласки твои, сестра моя, невеста! о, как много ласки твои лучше вина, и благовоние мастей твоих лучше всех ароматов!
Son 4:11  Сотовый мед каплет из уст твоих, невеста; мед и молоко под языком твоим, и благоухание одежды твоей подобно благоуханию Ливана!
Son 4:12  Запертый сад--сестра моя, невеста, заключенный колодезь, запечатанный источник:
Son 4:13  рассадники твои--сад с гранатовыми яблоками, с превосходными плодами, киперы с нардами,
Son 4:14  нард и шафран, аир и корица со всякими благовонными деревами, мирра и алой со всякими лучшими ароматами;
Son 4:15  садовый источник--колодезь живых вод и потоки с Ливана.
Son 4:16  Поднимись [ветер] с севера и принесись с юга, повей на сад мой, --и польются ароматы его! --Пусть придет возлюбленный мой в сад свой и вкушает сладкие плоды его.
Son 5:1  Пришел я в сад мой, сестра моя, невеста; набрал мирры моей с ароматами моими, поел сотов моих с медом моим, напился вина моего с молоком моим. Ешьте, друзья, пейте и насыщайтесь, возлюбленные!
Son 5:2  Я сплю, а сердце мое бодрствует; [вот], голос моего возлюбленного, который стучится: `отвори мне, сестра моя, возлюбленная моя, голубица моя, чистая моя! потому что голова моя вся покрыта росою, кудри мои--ночною влагою'.
Son 5:3  Я скинула хитон мой; как же мне опять надевать его? Я вымыла ноги мои; как же мне марать их?
Son 5:4  Возлюбленный мой протянул руку свою сквозь скважину, и внутренность моя взволновалась от него.
Son 5:5  Я встала, чтобы отпереть возлюбленному моему, и с рук моих капала мирра, и с перстов моих мирра капала на ручки замка.
Son 5:6  Отперла я возлюбленному моему, а возлюбленный мой повернулся и ушел. Души во мне не стало, когда он говорил; я искала его и не находила его; звала его, и он не отзывался мне.
Son 5:7  Встретили меня стражи, обходящие город, избили меня, изранили меня; сняли с меня покрывало стерегущие стены.
Son 5:8  Заклинаю вас, дщери Иерусалимские: если вы встретите возлюбленного моего, что скажете вы ему? что я изнемогаю от любви.
Son 5:9  `Чем возлюбленный твой лучше других возлюбленных, прекраснейшая из женщин? Чем возлюбленный твой лучше других, что ты так заклинаешь нас?'
Son 5:10  Возлюбленный мой бел и румян, лучше десяти тысяч других:
Son 5:11  голова его--чистое золото; кудри его волнистые, черные, как ворон;
Son 5:12  глаза его--как голуби при потоках вод, купающиеся в молоке, сидящие в довольстве;
Son 5:13  щеки его--цветник ароматный, гряды благовонных растений; губы его--лилии, источают текучую мирру;
Son 5:14  руки его--золотые кругляки, усаженные топазами; живот его--как изваяние из слоновой кости, обложенное сапфирами;
Son 5:15  голени его--мраморные столбы, поставленные на золотых подножиях; вид его подобен Ливану, величествен, как кедры;
Son 5:16  уста его--сладость, и весь он--любезность. Вот кто возлюбленный мой, и вот кто друг мой, дщери Иерусалимские!
Son 6:1  `Куда пошел возлюбленный твой, прекраснейшая из женщин? куда обратился возлюбленный твой? мы поищем его с тобою'.
Son 6:2  Мой возлюбленный пошел в сад свой, в цветники ароматные, чтобы пасти в садах и собирать лилии.
Son 6:3  Я принадлежу возлюбленному моему, а возлюбленный мой--мне; он пасет между лилиями.
Son 6:4  Прекрасна ты, возлюбленная моя, как Фирца, любезна, как Иерусалим, грозна, как полки со знаменами.
Son 6:5  Уклони очи твои от меня, потому что они волнуют меня.
Son 6:6  Волосы твои--как стадо коз, сходящих с Галаада; зубы твои--как стадо овец, выходящих из купальни, из которых у каждой пара ягнят, и бесплодной нет между ними;
Son 6:7  как половинки гранатового яблока--ланиты твои под кудрями твоими.
Son 6:8  Есть шестьдесят цариц и восемьдесят наложниц и девиц без числа,
Son 6:9  но единственная--она, голубица моя, чистая моя; единственная она у матери своей, отличенная у родительницы своей. Увидели ее девицы, и--превознесли ее, царицы и наложницы, и--восхвалили ее.
Son 6:10  Кто эта, блистающая, как заря, прекрасная, как луна, светлая, как солнце, грозная, как полки со знаменами?
Son 6:11  Я сошла в ореховый сад посмотреть на зелень долины, поглядеть, распустилась ли виноградная лоза, расцвели ли гранатовые яблоки?
Son 6:12  Не знаю, как душа моя влекла меня к колесницам знатных народа моего.
Son 7:1  `Оглянись, оглянись, Суламита! оглянись, оглянись, --и мы посмотрим на тебя'. Что вам смотреть на Суламиту, как на хоровод Манаимский?
Son 7:2  О, как прекрасны ноги твои в сандалиях, дщерь именитая! Округление бедр твоих, как ожерелье, дело рук искусного художника;
Son 7:3  живот твой--круглая чаша, [в которой] не истощается ароматное вино; чрево твое--ворох пшеницы, обставленный лилиями;
Son 7:4  два сосца твои--как два козленка, двойни серны;
Son 7:5  шея твоя--как столп из слоновой кости; глаза твои--озерки Есевонские, что у ворот Батраббима; нос твой--башня Ливанская, обращенная к Дамаску;
Son 7:6  голова твоя на тебе, как Кармил, и волосы на голове твоей, как пурпур; царь увлечен [твоими] кудрями.
Son 7:7  Как ты прекрасна, как привлекательна, возлюбленная, твоею миловидностью!
Son 7:8  Этот стан твой похож на пальму, и груди твои на виноградные кисти.
Son 7:9  Подумал я: влез бы я на пальму, ухватился бы за ветви ее; и груди твои были бы вместо кистей винограда, и запах от ноздрей твоих, как от яблоков;
Son 7:10  уста твои--как отличное вино. Оно течет прямо к другу моему, услаждает уста утомленных.
Son 7:11  Я принадлежу другу моему, и ко мне [обращено] желание его.
Son 7:12  Приди, возлюбленный мой, выйдем в поле, побудем в селах;
Son 7:13  поутру пойдем в виноградники, посмотрим, распустилась ли виноградная лоза, раскрылись ли почки, расцвели ли гранатовые яблоки; там я окажу ласки мои тебе.
Son 7:14  Мандрагоры уже пустили благовоние, и у дверей наших всякие превосходные плоды, новые и старые: [это] сберегла я для тебя, мой возлюбленный!
Son 8:1  О, если бы ты был мне брат, сосавший груди матери моей! тогда я, встретив тебя на улице, целовала бы тебя, и меня не осуждали бы.
Son 8:2  Повела бы я тебя, привела бы тебя в дом матери моей. Ты учил бы меня, а я поила бы тебя ароматным вином, соком гранатовых яблоков моих.
Son 8:3  Левая рука его у меня под головою, а правая обнимает меня.
Son 8:4  Заклинаю вас, дщери Иерусалимские, --не будите и не тревожьте возлюбленной, доколе ей угодно.
Son 8:5  Кто это восходит от пустыни, опираясь на своего возлюбленного? Под яблоней разбудила я тебя: там родила тебя мать твоя, там родила тебя родительница твоя.
Son 8:6  Положи меня, как печать, на сердце твое, как перстень, на руку твою: ибо крепка, как смерть, любовь; люта, как преисподняя, ревность; стрелы ее--стрелы огненные; она пламень весьма сильный.
Son 8:7  Большие воды не могут потушить любви, и реки не зальют ее. Если бы кто давал все богатство дома своего за любовь, то он был бы отвергнут с презреньем.
Son 8:8  Есть у нас сестра, которая еще мала, и сосцов нет у нее; что нам будет делать с сестрою нашею, когда будут свататься за нее?
Son 8:9  Если бы она была стена, то мы построили бы на ней палаты из серебра; если бы она была дверь, то мы обложили бы ее кедровыми досками.
Son 8:10  Я--стена, и сосцы у меня, как башни; потому я буду в глазах его, как достигшая полноты.
Son 8:11  Виноградник был у Соломона в Ваал-Гамоне; он отдал этот виноградник сторожам; каждый должен был доставлять за плоды его тысячу сребренников.
Son 8:12  А мой виноградник у меня при себе. Тысяча пусть тебе, Соломон, а двести--стерегущим плоды его.
Son 8:13  Жительница садов! товарищи внимают голосу твоему, дай и мне послушать его.
Son 8:14  Беги, возлюбленный мой; будь подобен серне или молодому оленю на горах бальзамических!


\end{document}