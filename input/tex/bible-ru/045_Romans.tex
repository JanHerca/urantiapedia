\begin{document}

\title{Послание к Римлянам}


\chapter{1}

\par 1 Павел, раб Иисуса Христа, призванный Апостол, избранный к благовестию Божию,
\par 2 которое Бог прежде обещал через пророков Своих, в святых писаниях,
\par 3 о Сыне Своем, Который родился от семени Давидова по плоти
\par 4 и открылся Сыном Божиим в силе, по духу святыни, через воскресение из мертвых, о Иисусе Христе Господе нашем,
\par 5 через Которого мы получили благодать и апостольство, чтобы во имя Его покорять вере все народы,
\par 6 между которыми находитесь и вы, призванные Иисусом Христом, --
\par 7 всем находящимся в Риме возлюбленным Божиим, призванным святым: благодать вам и мир от Бога отца нашего и Господа Иисуса Христа.
\par 8 Прежде всего благодарю Бога моего через Иисуса Христа за всех вас, что вера ваша возвещается во всем мире.
\par 9 Свидетель мне Бог, Которому служу духом моим в благовествовании Сына Его, что непрестанно воспоминаю о вас,
\par 10 всегда прося в молитвах моих, чтобы воля Божия когда-нибудь благопоспешила мне придти к вам,
\par 11 ибо я весьма желаю увидеть вас, чтобы преподать вам некое дарование духовное к утверждению вашему,
\par 12 то есть утешиться с вами верою общею, вашею и моею.
\par 13 Не хочу, братия, [оставить] вас в неведении, что я многократно намеревался придти к вам (но встречал препятствия даже доныне), чтобы иметь некий плод и у вас, как и у прочих народов.
\par 14 Я должен и Еллинам и варварам, мудрецам и невеждам.
\par 15 Итак, что до меня, я готов благовествовать и вам, находящимся в Риме.
\par 16 Ибо я не стыжусь благовествования Христова, потому что [оно] есть сила Божия ко спасению всякому верующему, во-первых, Иудею, [потом] и Еллину.
\par 17 В нем открывается правда Божия от веры в веру, как написано: праведный верою жив будет.
\par 18 Ибо открывается гнев Божий с неба на всякое нечестие и неправду человеков, подавляющих истину неправдою.
\par 19 Ибо, что можно знать о Боге, явно для них, потому что Бог явил им.
\par 20 Ибо невидимое Его, вечная сила Его и Божество, от создания мира через рассматривание творений видимы, так что они безответны.
\par 21 Но как они, познав Бога, не прославили Его, как Бога, и не возблагодарили, но осуетились в умствованиях своих, и омрачилось несмысленное их сердце;
\par 22 называя себя мудрыми, обезумели,
\par 23 и славу нетленного Бога изменили в образ, подобный тленному человеку, и птицам, и четвероногим, и пресмыкающимся, --
\par 24 то и предал их Бог в похотях сердец их нечистоте, так что они сквернили сами свои тела.
\par 25 Они заменили истину Божию ложью, и поклонялись, и служили твари вместо Творца, Который благословен во веки, аминь.
\par 26 Потому предал их Бог постыдным страстям: женщины их заменили естественное употребление противоестественным;
\par 27 подобно и мужчины, оставив естественное употребление женского пола, разжигались похотью друг на друга, мужчины на мужчинах делая срам и получая в самих себе должное возмездие за свое заблуждение.
\par 28 И как они не заботились иметь Бога в разуме, то предал их Бог превратному уму--делать непотребства,
\par 29 так что они исполнены всякой неправды, блуда, лукавства, корыстолюбия, злобы, исполнены зависти, убийства, распрей, обмана, злонравия,
\par 30 злоречивы, клеветники, богоненавистники, обидчики, самохвалы, горды, изобретательны на зло, непослушны родителям,
\par 31 безрассудны, вероломны, нелюбовны, непримиримы, немилостивы.
\par 32 Они знают праведный [суд] Божий, что делающие такие [дела] достойны смерти; однако не только [их] делают, но и делающих одобряют.

\chapter{2}

\par 1 Итак, неизвинителен ты, всякий человек, судящий [другого], ибо тем же судом, каким судишь другого, осуждаешь себя, потому что, судя [другого], делаешь то же.
\par 2 А мы знаем, что поистине есть суд Божий на делающих такие [дела].
\par 3 Неужели думаешь ты, человек, что избежишь суда Божия, осуждая делающих такие [дела] и (сам) делая то же?
\par 4 Или пренебрегаешь богатство благости, кротости и долготерпения Божия, не разумея, что благость Божия ведет тебя к покаянию?
\par 5 Но, по упорству твоему и нераскаянному сердцу, ты сам себе собираешь гнев на день гнева и откровения праведного суда от Бога,
\par 6 Который воздаст каждому по делам его:
\par 7 тем, которые постоянством в добром деле ищут славы, чести и бессмертия, --жизнь вечную;
\par 8 а тем, которые упорствуют и не покоряются истине, но предаются неправде, --ярость и гнев.
\par 9 Скорбь и теснота всякой душе человека, делающего злое, во-первых, Иудея, [потом] и Еллина!
\par 10 Напротив, слава и честь и мир всякому, делающему доброе, во-- первых, Иудею, [потом] и Еллину!
\par 11 Ибо нет лицеприятия у Бога.
\par 12 Те, которые, не [имея] закона, согрешили, вне закона и погибнут; а те, которые под законом согрешили, по закону осудятся
\par 13 (потому что не слушатели закона праведны пред Богом, но исполнители закона оправданы будут,
\par 14 ибо когда язычники, не имеющие закона, по природе законное делают, то, не имея закона, они сами себе закон:
\par 15 они показывают, что дело закона у них написано в сердцах, о чем свидетельствует совесть их и мысли их, то обвиняющие, то оправдывающие одна другую)
\par 16 в день, когда, по благовествованию моему, Бог будет судить тайные [дела] человеков через Иисуса Христа.
\par 17 Вот, ты называешься Иудеем, и успокаиваешь себя законом, и хвалишься Богом,
\par 18 и знаешь волю [Его], и разумеешь лучшее, научаясь из закона,
\par 19 и уверен о себе, что ты путеводитель слепых, свет для находящихся во тьме,
\par 20 наставник невежд, учитель младенцев, имеющий в законе образец ведения и истины:
\par 21 как же ты, уча другого, не учишь себя самого?
\par 22 Проповедуя не красть, крадешь? говоря: `не прелюбодействуй', прелюбодействуешь? гнушаясь идолов, святотатствуешь?
\par 23 Хвалишься законом, а преступлением закона бесчестишь Бога?
\par 24 Ибо ради вас, как написано, имя Божие хулится у язычников.
\par 25 Обрезание полезно, если исполняешь закон; а если ты преступник закона, то обрезание твое стало необрезанием.
\par 26 Итак, если необрезанный соблюдает постановления закона, то его необрезание не вменится ли ему в обрезание?
\par 27 И необрезанный по природе, исполняющий закон, не осудит ли тебя, преступника закона при Писании и обрезании?
\par 28 Ибо не тот Иудей, кто [таков] по наружности, и не то обрезание, которое наружно, на плоти;
\par 29 но [тот] Иудей, кто внутренно [таков], и [то] обрезание, [которое] в сердце, по духу, [а] не по букве: ему и похвала не от людей, но от Бога.

\chapter{3}

\par 1 Итак, какое преимущество [быть] Иудеем, или какая польза от обрезания?
\par 2 Великое преимущество во всех отношениях, а наипаче [в том], что им вверено слово Божие.
\par 3 Ибо что же? если некоторые и неверны были, неверность их уничтожит ли верность Божию?
\par 4 Никак. Бог верен, а всякий человек лжив, как написано: Ты праведен в словах Твоих и победишь в суде Твоем.
\par 5 Если же наша неправда открывает правду Божию, то что скажем? не будет ли Бог несправедлив, когда изъявляет гнев? (говорю по человеческому [рассуждению]).
\par 6 Никак. Ибо [иначе] как Богу судить мир?
\par 7 Ибо, если верность Божия возвышается моею неверностью к славе Божией, за что еще меня же судить, как грешника?
\par 8 И не делать ли нам зло, чтобы вышло добро, как некоторые злословят нас и говорят, будто мы так учим? Праведен суд на таковых.
\par 9 Итак, что же? имеем ли мы преимущество? Нисколько. Ибо мы уже доказали, что как Иудеи, так и Еллины, все под грехом,
\par 10 как написано: нет праведного ни одного;
\par 11 нет разумевающего; никто не ищет Бога;
\par 12 все совратились с пути, до одного негодны; нет делающего добро, нет ни одного.
\par 13 Гортань их--открытый гроб; языком своим обманывают; яд аспидов на губах их.
\par 14 Уста их полны злословия и горечи.
\par 15 Ноги их быстры на пролитие крови;
\par 16 разрушение и пагуба на путях их;
\par 17 они не знают пути мира.
\par 18 Нет страха Божия перед глазами их.
\par 19 Но мы знаем, что закон, если что говорит, говорит к состоящим под законом, так что заграждаются всякие уста, и весь мир становится виновен пред Богом,
\par 20 потому что делами закона не оправдается пред Ним никакая плоть; ибо законом познается грех.
\par 21 Но ныне, независимо от закона, явилась правда Божия, о которой свидетельствуют закон и пророки,
\par 22 правда Божия через веру в Иисуса Христа во всех и на всех верующих, ибо нет различия,
\par 23 потому что все согрешили и лишены славы Божией,
\par 24 получая оправдание даром, по благодати Его, искуплением во Христе Иисусе,
\par 25 которого Бог предложил в жертву умилостивления в Крови Его через веру, для показания правды Его в прощении грехов, соделанных прежде,
\par 26 во [время] долготерпения Божия, к показанию правды Его в настоящее время, да [явится] Он праведным и оправдывающим верующего в Иисуса.
\par 27 Где же то, чем бы хвалиться? уничтожено. Каким законом? [законом] дел? Нет, но законом веры.
\par 28 Ибо мы признаем, что человек оправдывается верою, независимо от дел закона.
\par 29 Неужели Бог [есть Бог] Иудеев только, а не и язычников? Конечно, и язычников,
\par 30 потому что один Бог, Который оправдает обрезанных по вере и необрезанных через веру.
\par 31 Итак, мы уничтожаем закон верою? Никак; но закон утверждаем.

\chapter{4}

\par 1 Что же, скажем, Авраам, отец наш, приобрел по плоти?
\par 2 Если Авраам оправдался делами, он имеет похвалу, но не пред Богом.
\par 3 Ибо что говорит Писание? Поверил Авраам Богу, и это вменилось ему в праведность.
\par 4 Воздаяние делающему вменяется не по милости, но по долгу.
\par 5 А не делающему, но верующему в Того, Кто оправдывает нечестивого, вера его вменяется в праведность.
\par 6 Так и Давид называет блаженным человека, которому Бог вменяет праведность независимо от дел:
\par 7 Блаженны, чьи беззакония прощены и чьи грехи покрыты.
\par 8 Блажен человек, которому Господь не вменит греха.
\par 9 Блаженство сие [относится] к обрезанию, или к необрезанию? Мы говорим, что Аврааму вера вменилась в праведность.
\par 10 Когда вменилась? по обрезании или до обрезания? Не по обрезании, а до обрезания.
\par 11 И знак обрезания он получил, [как] печать праведности через веру, которую [имел] в необрезании, так что он стал отцом всех верующих в необрезании, чтобы и им вменилась праведность,
\par 12 и отцом обрезанных, не только [принявших] обрезание, но и ходящих по следам веры отца нашего Авраама, которую [имел он] в необрезании.
\par 13 Ибо не законом [даровано] Аврааму, или семени его, обетование--быть наследником мира, но праведностью веры.
\par 14 Если утверждающиеся на законе суть наследники, то тщетна вера, бездейственно обетование;
\par 15 ибо закон производит гнев, потому что, где нет закона, нет и преступления.
\par 16 Итак по вере, чтобы [было] по милости, дабы обетование было непреложно для всех, не только по закону, но и по вере потомков Авраама, который есть отец всем нам
\par 17 (как написано: Я поставил тебя отцом многих народов) пред Богом, Которому он поверил, животворящим мертвых и называющим несуществующее, как существующее.
\par 18 Он, сверх надежды, поверил с надеждою, через что сделался отцом многих народов, по сказанному: `так [многочисленно] будет семя твое'.
\par 19 И, не изнемогши в вере, он не помышлял, что тело его, почти столетнего, уже омертвело, и утроба Саррина в омертвении;
\par 20 не поколебался в обетовании Божием неверием, но пребыл тверд в вере, воздав славу Богу
\par 21 и будучи вполне уверен, что Он силен и исполнить обещанное.
\par 22 Потому и вменилось ему в праведность.
\par 23 А впрочем не в отношении к нему одному написано, что вменилось ему,
\par 24 но и в отношении к нам; вменится и нам, верующим в Того, Кто воскресил из мертвых Иисуса Христа, Господа нашего,
\par 25 Который предан за грехи наши и воскрес для оправдания нашего.

\chapter{5}

\par 1 Итак, оправдавшись верою, мы имеем мир с Богом через Господа нашего Иисуса Христа,
\par 2 через Которого верою и получили мы доступ к той благодати, в которой стоим и хвалимся надеждою славы Божией.
\par 3 И не сим только, но хвалимся и скорбями, зная, что от скорби происходит терпение,
\par 4 от терпения опытность, от опытности надежда,
\par 5 а надежда не постыжает, потому что любовь Божия излилась в сердца наши Духом Святым, данным нам.
\par 6 Ибо Христос, когда еще мы были немощны, в определенное время умер за нечестивых.
\par 7 Ибо едва ли кто умрет за праведника; разве за благодетеля, может быть, кто и решится умереть.
\par 8 Но Бог Свою любовь к нам доказывает тем, что Христос умер за нас, когда мы были еще грешниками.
\par 9 Посему тем более ныне, будучи оправданы Кровию Его, спасемся Им от гнева.
\par 10 Ибо если, будучи врагами, мы примирились с Богом смертью Сына Его, то тем более, примирившись, спасемся жизнью Его.
\par 11 И не довольно сего, но и хвалимся Богом чрез Господа нашего Иисуса Христа, посредством Которого мы получили ныне примирение.
\par 12 Посему, как одним человеком грех вошел в мир, и грехом смерть, так и смерть перешла во всех человеков, [потому что] в нем все согрешили.
\par 13 Ибо [и] до закона грех был в мире; но грех не вменяется, когда нет закона.
\par 14 Однако же смерть царствовала от Адама до Моисея и над несогрешившими подобно преступлению Адама, который есть образ будущего.
\par 15 Но дар благодати не как преступление. Ибо если преступлением одного подверглись смерти многие, то тем более благодать Божия и дар по благодати одного Человека, Иисуса Христа, преизбыточествуют для многих.
\par 16 И дар не как [суд] за одного согрешившего; ибо суд за одно [преступление] --к осуждению; а дар благодати--к оправданию от многих преступлений.
\par 17 Ибо если преступлением одного смерть царствовала посредством одного, то тем более приемлющие обилие благодати и дар праведности будут царствовать в жизни посредством единого Иисуса Христа.
\par 18 Посему, как преступлением одного всем человекам осуждение, так правдою одного всем человекам оправдание к жизни.
\par 19 Ибо, как непослушанием одного человека сделались многие грешными, так и послушанием одного сделаются праведными многие.
\par 20 Закон же пришел после, и таким образом умножилось преступление. А когда умножился грех, стала преизобиловать благодать,
\par 21 дабы, как грех царствовал к смерти, так и благодать воцарилась через праведность к жизни вечной Иисусом Христом, Господом нашим.

\chapter{6}

\par 1 Что же скажем? оставаться ли нам в грехе, чтобы умножилась благодать? Никак.
\par 2 Мы умерли для греха: как же нам жить в нем?
\par 3 Неужели не знаете, что все мы, крестившиеся во Христа Иисуса, в смерть Его крестились?
\par 4 Итак мы погреблись с Ним крещением в смерть, дабы, как Христос воскрес из мертвых славою Отца, так и нам ходить в обновленной жизни.
\par 5 Ибо если мы соединены с Ним подобием смерти Его, то должны быть [соединены] и [подобием] воскресения,
\par 6 зная то, что ветхий наш человек распят с Ним, чтобы упразднено было тело греховное, дабы нам не быть уже рабами греху;
\par 7 ибо умерший освободился от греха.
\par 8 Если же мы умерли со Христом, то веруем, что и жить будем с Ним,
\par 9 зная, что Христос, воскреснув из мертвых, уже не умирает: смерть уже не имеет над Ним власти.
\par 10 Ибо, что Он умер, то умер однажды для греха; а что живет, то живет для Бога.
\par 11 Так и вы почитайте себя мертвыми для греха, живыми же для Бога во Христе Иисусе, Господе нашем.
\par 12 Итак да не царствует грех в смертном вашем теле, чтобы вам повиноваться ему в похотях его;
\par 13 и не предавайте членов ваших греху в орудия неправды, но представьте себя Богу, как оживших из мертвых, и члены ваши Богу в орудия праведности.
\par 14 Грех не должен над вами господствовать, ибо вы не под законом, но под благодатью.
\par 15 Что же? станем ли грешить, потому что мы не под законом, а под благодатью? Никак.
\par 16 Неужели вы не знаете, что, кому вы отдаете себя в рабы для послушания, того вы и рабы, кому повинуетесь, или [рабы] греха к смерти, или послушания к праведности?
\par 17 Благодарение Богу, что вы, быв прежде рабами греха, от сердца стали послушны тому образу учения, которому предали себя.
\par 18 Освободившись же от греха, вы стали рабами праведности.
\par 19 Говорю по [рассуждению] человеческому, ради немощи плоти вашей. Как предавали вы члены ваши в рабы нечистоте и беззаконию на [дела] беззаконные, так ныне представьте члены ваши в рабы праведности на [дела] святые.
\par 20 Ибо, когда вы были рабами греха, тогда были свободны от праведности.
\par 21 Какой же плод вы имели тогда? [Такие дела], каких ныне сами стыдитесь, потому что конец их--смерть.
\par 22 Но ныне, когда вы освободились от греха и стали рабами Богу, плод ваш есть святость, а конец--жизнь вечная.
\par 23 Ибо возмездие за грех--смерть, а дар Божий--жизнь вечная во Христе Иисусе, Господе нашем.

\chapter{7}

\par 1 Разве вы не знаете, братия (ибо говорю знающим закон), что закон имеет власть над человеком, пока он жив?
\par 2 Замужняя женщина привязана законом к живому мужу; а если умрет муж, она освобождается от закона замужества.
\par 3 Посему, если при живом муже выйдет за другого, называется прелюбодейцею; если же умрет муж, она свободна от закона, и не будет прелюбодейцею, выйдя за другого мужа.
\par 4 Так и вы, братия мои, умерли для закона телом Христовым, чтобы принадлежать другому, Воскресшему из мертвых, да приносим плод Богу.
\par 5 Ибо, когда мы жили по плоти, тогда страсти греховные, [обнаруживаемые] законом, действовали в членах наших, чтобы приносить плод смерти;
\par 6 но ныне, умерши для закона, которым были связаны, мы освободились от него, чтобы нам служить Богу в обновлении духа, а не по ветхой букве.
\par 7 Что же скажем? Неужели [от] закона грех? Никак. Но я не иначе узнал грех, как посредством закона. Ибо я не понимал бы и пожелания, если бы закон не говорил: не пожелай.
\par 8 Но грех, взяв повод от заповеди, произвел во мне всякое пожелание: ибо без закона грех мертв.
\par 9 Я жил некогда без закона; но когда пришла заповедь, то грех ожил,
\par 10 а я умер; и таким образом заповедь, [данная] для жизни, послужила мне к смерти,
\par 11 потому что грех, взяв повод от заповеди, обольстил меня и умертвил ею.
\par 12 Посему закон свят, и заповедь свята и праведна и добра.
\par 13 Итак, неужели доброе сделалось мне смертоносным? Никак; но грех, оказывающийся грехом потому, что посредством доброго причиняет мне смерть, так что грех становится крайне грешен посредством заповеди.
\par 14 Ибо мы знаем, что закон духовен, а я плотян, продан греху.
\par 15 Ибо не понимаю, что делаю: потому что не то делаю, что хочу, а что ненавижу, то делаю.
\par 16 Если же делаю то, чего не хочу, то соглашаюсь с законом, что он добр,
\par 17 а потому уже не я делаю то, но живущий во мне грех.
\par 18 Ибо знаю, что не живет во мне, то есть в плоти моей, доброе; потому что желание добра есть во мне, но чтобы сделать оное, того не нахожу.
\par 19 Доброго, которого хочу, не делаю, а злое, которого не хочу, делаю.
\par 20 Если же делаю то, чего не хочу, уже не я делаю то, но живущий во мне грех.
\par 21 Итак я нахожу закон, что, когда хочу делать доброе, прилежит мне злое.
\par 22 Ибо по внутреннему человеку нахожу удовольствие в законе Божием;
\par 23 но в членах моих вижу иной закон, противоборствующий закону ума моего и делающий меня пленником закона греховного, находящегося в членах моих.
\par 24 Бедный я человек! кто избавит меня от сего тела смерти?
\par 25 Благодарю Бога моего Иисусом Христом, Господом нашим. Итак тот же самый я умом моим служу закону Божию, а плотию закону греха.

\chapter{8}

\par 1 Итак нет ныне никакого осуждения тем, которые во Христе Иисусе живут не по плоти, но по духу,
\par 2 потому что закон духа жизни во Христе Иисусе освободил меня от закона греха и смерти.
\par 3 Как закон, ослабленный плотию, был бессилен, то Бог послал Сына Своего в подобии плоти греховной [в жертву] за грех и осудил грех во плоти,
\par 4 чтобы оправдание закона исполнилось в нас, живущих не по плоти, но по духу.
\par 5 Ибо живущие по плоти о плотском помышляют, а живущие по духу--о духовном.
\par 6 Помышления плотские суть смерть, а помышления духовные--жизнь и мир,
\par 7 потому что плотские помышления суть вражда против Бога; ибо закону Божию не покоряются, да и не могут.
\par 8 Посему живущие по плоти Богу угодить не могут.
\par 9 Но вы не по плоти живете, а по духу, если только Дух Божий живет в вас. Если же кто Духа Христова не имеет, тот [и] не Его.
\par 10 А если Христос в вас, то тело мертво для греха, но дух жив для праведности.
\par 11 Если же Дух Того, Кто воскресил из мертвых Иисуса, живет в вас, то Воскресивший Христа из мертвых оживит и ваши смертные тела Духом Своим, живущим в вас.
\par 12 Итак, братия, мы не должники плоти, чтобы жить по плоти;
\par 13 ибо если живете по плоти, то умрете, а если духом умерщвляете дела плотские, то живы будете.
\par 14 Ибо все, водимые Духом Божиим, суть сыны Божии.
\par 15 Потому что вы не приняли духа рабства, [чтобы] опять [жить] в страхе, но приняли Духа усыновления, Которым взываем: `Авва, Отче!'
\par 16 Сей самый Дух свидетельствует духу нашему, что мы--дети Божии.
\par 17 А если дети, то и наследники, наследники Божии, сонаследники же Христу, если только с Ним страдаем, чтобы с Ним и прославиться.
\par 18 Ибо думаю, что нынешние временные страдания ничего не стоят в сравнении с тою славою, которая откроется в нас.
\par 19 Ибо тварь с надеждою ожидает откровения сынов Божиих,
\par 20 потому что тварь покорилась суете не добровольно, но по воле покорившего ее, в надежде,
\par 21 что и сама тварь освобождена будет от рабства тлению в свободу славы детей Божиих.
\par 22 Ибо знаем, что вся тварь совокупно стенает и мучится доныне;
\par 23 и не только [она], но и мы сами, имея начаток Духа, и мы в себе стенаем, ожидая усыновления, искупления тела нашего.
\par 24 Ибо мы спасены в надежде. Надежда же, когда видит, не есть надежда; ибо если кто видит, то чего ему и надеяться?
\par 25 Но когда надеемся того, чего не видим, тогда ожидаем в терпении.
\par 26 Также и Дух подкрепляет нас в немощах наших; ибо мы не знаем, о чем молиться, как должно, но Сам Дух ходатайствует за нас воздыханиями неизреченными.
\par 27 Испытующий же сердца знает, какая мысль у Духа, потому что Он ходатайствует за святых по [воле] Божией.
\par 28 Притом знаем, что любящим Бога, призванным по [Его] изволению, все содействует ко благу.
\par 29 Ибо кого Он предузнал, тем и предопределил быть подобными образу Сына Своего, дабы Он был первородным между многими братиями.
\par 30 А кого Он предопределил, тех и призвал, а кого призвал, тех и оправдал; а кого оправдал, тех и прославил.
\par 31 Что же сказать на это? Если Бог за нас, кто против нас?
\par 32 Тот, Который Сына Своего не пощадил, но предал Его за всех нас, как с Ним не дарует нам и всего?
\par 33 Кто будет обвинять избранных Божиих? Бог оправдывает [их].
\par 34 Кто осуждает? Христос Иисус умер, но и воскрес: Он и одесную Бога, Он и ходатайствует за нас.
\par 35 Кто отлучит нас от любви Божией: скорбь, или теснота, или гонение, или голод, или нагота, или опасность, или меч? как написано:
\par 36 за Тебя умерщвляют нас всякий день, считают нас за овец, [обреченных] на заклание.
\par 37 Но все сие преодолеваем силою Возлюбившего нас.
\par 38 Ибо я уверен, что ни смерть, ни жизнь, ни Ангелы, ни Начала, ни Силы, ни настоящее, ни будущее,
\par 39 ни высота, ни глубина, ни другая какая тварь не может отлучить нас от любви Божией во Христе Иисусе, Господе нашем.

\chapter{9}

\par 1 Истину говорю во Христе, не лгу, свидетельствует мне совесть моя в Духе Святом,
\par 2 что великая для меня печаль и непрестанное мучение сердцу моему:
\par 3 я желал бы сам быть отлученным от Христа за братьев моих, родных мне по плоти,
\par 4 то есть Израильтян, которым принадлежат усыновление и слава, и заветы, и законоположение, и богослужение, и обетования;
\par 5 их и отцы, и от них Христос по плоти, сущий над всем Бог, благословенный во веки, аминь.
\par 6 Но не то, чтобы слово Божие не сбылось: ибо не все те Израильтяне, которые от Израиля;
\par 7 и не все дети Авраама, которые от семени его, но сказано: в Исааке наречется тебе семя.
\par 8 То есть не плотские дети суть дети Божии, но дети обетования признаются за семя.
\par 9 А слово обетования таково: в это же время приду, и у Сарры будет сын.
\par 10 И не одно это; но [так было] и с Ревеккою, когда она зачала в одно время [двух сыновей] от Исаака, отца нашего.
\par 11 Ибо, когда они еще не родились и не сделали ничего доброго или худого (дабы изволение Божие в избрании происходило
\par 12 не от дел, но от Призывающего), сказано было ей: больший будет в порабощении у меньшего,
\par 13 как и написано: Иакова Я возлюбил, а Исава возненавидел.
\par 14 Что же скажем? Неужели неправда у Бога? Никак.
\par 15 Ибо Он говорит Моисею: кого миловать, помилую; кого жалеть, пожалею.
\par 16 Итак [помилование зависит] не от желающего и не от подвизающегося, но от Бога милующего.
\par 17 Ибо Писание говорит фараону: для того самого Я и поставил тебя, чтобы показать над тобою силу Мою и чтобы проповедано было имя Мое по всей земле.
\par 18 Итак, кого хочет, милует; а кого хочет, ожесточает.
\par 19 Ты скажешь мне: `за что же еще обвиняет? Ибо кто противостанет воле Его?'
\par 20 А ты кто, человек, что споришь с Богом? Изделие скажет ли сделавшему его: `зачем ты меня так сделал?'
\par 21 Не властен ли горшечник над глиною, чтобы из той же смеси сделать один сосуд для почетного [употребления], а другой для низкого?
\par 22 Что же, если Бог, желая показать гнев и явить могущество Свое, с великим долготерпением щадил сосуды гнева, готовые к погибели,
\par 23 дабы вместе явить богатство славы Своей над сосудами милосердия, которые Он приготовил к славе,
\par 24 над нами, которых Он призвал не только из Иудеев, но и из язычников?
\par 25 Как и у Осии говорит: не Мой народ назову Моим народом, и не возлюбленную--возлюбленною.
\par 26 И на том месте, где сказано им: вы не Мой народ, там названы будут сынами Бога живаго.
\par 27 А Исаия провозглашает об Израиле: хотя бы сыны Израилевы были числом, как песок морской, [только] остаток спасется;
\par 28 ибо дело оканчивает и скоро решит по правде, дело решительное совершит Господь на земле.
\par 29 И, как предсказал Исаия: если бы Господь Саваоф не оставил нам семени, то мы сделались бы, как Содом, и были бы подобны Гоморре.
\par 30 Что же скажем? Язычники, не искавшие праведности, получили праведность, праведность от веры.
\par 31 А Израиль, искавший закона праведности, не достиг до закона праведности.
\par 32 Почему? потому что [искали] не в вере, а в делах закона. Ибо преткнулись о камень преткновения,
\par 33 как написано: вот, полагаю в Сионе камень преткновения и камень соблазна; но всякий, верующий в Него, не постыдится.

\chapter{10}

\par 1 Братия! желание моего сердца и молитва к Богу об Израиле во спасение.
\par 2 Ибо свидетельствую им, что имеют ревность по Боге, но не по рассуждению.
\par 3 Ибо, не разумея праведности Божией и усиливаясь поставить собственную праведность, они не покорились праведности Божией,
\par 4 потому что конец закона--Христос, к праведности всякого верующего.
\par 5 Моисей пишет о праведности от закона: исполнивший его человек жив будет им.
\par 6 А праведность от веры так говорит: не говори в сердце твоем: кто взойдет на небо? то есть Христа свести.
\par 7 Или кто сойдет в бездну? то есть Христа из мертвых возвести.
\par 8 Но что говорит Писание? Близко к тебе слово, в устах твоих и в сердце твоем, то есть слово веры, которое проповедуем.
\par 9 Ибо если устами твоими будешь исповедывать Иисуса Господом и сердцем твоим веровать, что Бог воскресил Его из мертвых, то спасешься,
\par 10 потому что сердцем веруют к праведности, а устами исповедуют ко спасению.
\par 11 Ибо Писание говорит: всякий, верующий в Него, не постыдится.
\par 12 Здесь нет различия между Иудеем и Еллином, потому что один Господь у всех, богатый для всех, призывающих Его.
\par 13 Ибо всякий, кто призовет имя Господне, спасется.
\par 14 Но как призывать [Того], в Кого не уверовали? как веровать [в] [Того], о Ком не слыхали? как слышать без проповедующего?
\par 15 И как проповедывать, если не будут посланы? как написано: как прекрасны ноги благовествующих мир, благовествующих благое!
\par 16 Но не все послушались благовествования. Ибо Исаия говорит: Господи! кто поверил слышанному от нас?
\par 17 Итак вера от слышания, а слышание от слова Божия.
\par 18 Но спрашиваю: разве они не слышали? Напротив, по всей земле прошел голос их, и до пределов вселенной слова их.
\par 19 Еще спрашиваю: разве Израиль не знал? Но первый Моисей говорит: Я возбужу в вас ревность не народом, раздражу вас народом несмысленным.
\par 20 А Исаия смело говорит: Меня нашли не искавшие Меня; Я открылся не вопрошавшим о Мне.
\par 21 Об Израиле же говорит: целый день Я простирал руки Мои к народу непослушному и упорному.

\chapter{11}

\par 1 Итак, спрашиваю: неужели Бог отверг народ Свой? Никак. Ибо и я Израильтянин, от семени Авраамова, из колена Вениаминова.
\par 2 Не отверг Бог народа Своего, который Он наперед знал. Или не знаете, что говорит Писание в [повествовании об] Илии? как он жалуется Богу на Израиля, говоря:
\par 3 Господи! пророков Твоих убили, жертвенники Твои разрушили; остался я один, и моей души ищут.
\par 4 Что же говорит ему Божеский ответ? Я соблюл Себе семь тысяч человек, которые не преклонили колени перед Ваалом.
\par 5 Так и в нынешнее время, по избранию благодати, сохранился остаток.
\par 6 Но если по благодати, то не по делам; иначе благодать не была бы уже благодатью. А если по делам, то это уже не благодать; иначе дело не есть уже дело.
\par 7 Что же? Израиль, чего искал, того не получил; избранные же получили, а прочие ожесточились,
\par 8 как написано: Бог дал им дух усыпления, глаза, которыми не видят, и уши, которыми не слышат, даже до сего дня.
\par 9 И Давид говорит: да будет трапеза их сетью, тенетами и петлею в возмездие им;
\par 10 да помрачатся глаза их, чтобы не видеть, и хребет их да будет согбен навсегда.
\par 11 Итак спрашиваю: неужели они преткнулись, чтобы [совсем] пасть? Никак. Но от их падения спасение язычникам, чтобы возбудить в них ревность.
\par 12 Если же падение их--богатство миру, и оскудение их--богатство язычникам, то тем более полнота их.
\par 13 Вам говорю, язычникам. Как Апостол язычников, я прославляю служение мое.
\par 14 Не возбужу ли ревность в [сродниках] моих по плоти и не спасу ли некоторых из них?
\par 15 Ибо если отвержение их--примирение мира, то что [будет] принятие, как не жизнь из мертвых?
\par 16 Если начаток свят, то и целое; и если корень свят, то и ветви.
\par 17 Если же некоторые из ветвей отломились, а ты, дикая маслина, привился на место их и стал общником корня и сока маслины,
\par 18 то не превозносись перед ветвями. Если же превозносишься, [то] [вспомни, что] не ты корень держишь, но корень тебя.
\par 19 Скажешь: `ветви отломились, чтобы мне привиться'.
\par 20 Хорошо. Они отломились неверием, а ты держишься верою: не гордись, но бойся.
\par 21 Ибо если Бог не пощадил природных ветвей, то смотри, пощадит ли и тебя.
\par 22 Итак видишь благость и строгость Божию: строгость к отпадшим, а благость к тебе, если пребудешь в благости [Божией]; иначе и ты будешь отсечен.
\par 23 Но и те, если не пребудут в неверии, привьются, потому что Бог силен опять привить их.
\par 24 Ибо если ты отсечен от дикой по природе маслины и не по природе привился к хорошей маслине, то тем более сии природные привьются к своей маслине.
\par 25 Ибо не хочу оставить вас, братия, в неведении о тайне сей, --чтобы вы не мечтали о себе, --что ожесточение произошло в Израиле отчасти, [до времени], пока войдет полное [число] язычников;
\par 26 и так весь Израиль спасется, как написано: придет от Сиона Избавитель, и отвратит нечестие от Иакова.
\par 27 И сей завет им от Меня, когда сниму с них грехи их.
\par 28 В отношении к благовестию, они враги ради вас; а в отношении к избранию, возлюбленные [Божии] ради отцов.
\par 29 Ибо дары и призвание Божие непреложны.
\par 30 Как и вы некогда были непослушны Богу, а ныне помилованы, по непослушанию их,
\par 31 так и они теперь непослушны для помилования вас, чтобы и сами они были помилованы.
\par 32 Ибо всех заключил Бог в непослушание, чтобы всех помиловать.
\par 33 О, бездна богатства и премудрости и ведения Божия! Как непостижимы судьбы Его и неисследимы пути Его!
\par 34 Ибо кто познал ум Господень? Или кто был советником Ему?
\par 35 Или кто дал Ему наперед, чтобы Он должен был воздать?
\par 36 Ибо все из Него, Им и к Нему. Ему слава во веки, аминь.

\chapter{12}

\par 1 Итак умоляю вас, братия, милосердием Божиим, представьте тела ваши в жертву живую, святую, благоугодную Богу, [для] разумного служения вашего,
\par 2 и не сообразуйтесь с веком сим, но преобразуйтесь обновлением ума вашего, чтобы вам познавать, что есть воля Божия, благая, угодная и совершенная.
\par 3 По данной мне благодати, всякому из вас говорю: не думайте [о] [себе] более, нежели должно думать; но думайте скромно, по мере веры, какую каждому Бог уделил.
\par 4 Ибо, как в одном теле у нас много членов, но не у всех членов одно и то же дело,
\par 5 так мы, многие, составляем одно тело во Христе, а порознь один для другого члены.
\par 6 И как, по данной нам благодати, имеем различные дарования, [то], [имеешь ли] пророчество, [пророчествуй] по мере веры;
\par 7 [имеешь ли] служение, [пребывай] в служении; учитель ли, --в учении;
\par 8 увещатель ли, увещевай; раздаватель ли, [раздавай] в простоте; начальник ли, [начальствуй] с усердием; благотворитель ли, [благотвори] с радушием.
\par 9 Любовь [да будет] непритворна; отвращайтесь зла, прилепляйтесь к добру;
\par 10 будьте братолюбивы друг к другу с нежностью; в почтительности друг друга предупреждайте;
\par 11 в усердии не ослабевайте; духом пламенейте; Господу служите;
\par 12 утешайтесь надеждою; в скорби [будьте] терпеливы, в молитве постоянны;
\par 13 в нуждах святых принимайте участие; ревнуйте о странноприимстве.
\par 14 Благословляйте гонителей ваших; благословляйте, а не проклинайте.
\par 15 Радуйтесь с радующимися и плачьте с плачущими.
\par 16 Будьте единомысленны между собою; не высокомудрствуйте, но последуйте смиренным; не мечтайте о себе;
\par 17 никому не воздавайте злом за зло, но пекитесь о добром перед всеми человеками.
\par 18 Если возможно с вашей стороны, будьте в мире со всеми людьми.
\par 19 Не мстите за себя, возлюбленные, но дайте место гневу [Божию]. Ибо написано: Мне отмщение, Я воздам, говорит Господь.
\par 20 Итак, если враг твой голоден, накорми его; если жаждет, напой его: ибо, делая сие, ты соберешь ему на голову горящие уголья.
\par 21 Не будь побежден злом, но побеждай зло добром.

\chapter{13}

\par 1 Всякая душа да будет покорна высшим властям, ибо нет власти не от Бога; существующие же власти от Бога установлены.
\par 2 Посему противящийся власти противится Божию установлению. А противящиеся сами навлекут на себя осуждение.
\par 3 Ибо начальствующие страшны не для добрых дел, но для злых. Хочешь ли не бояться власти? Делай добро, и получишь похвалу от нее,
\par 4 ибо [начальник] есть Божий слуга, тебе на добро. Если же делаешь зло, бойся, ибо он не напрасно носит меч: он Божий слуга, отмститель в наказание делающему злое.
\par 5 И потому надобно повиноваться не только из [страха] наказания, но и по совести.
\par 6 Для сего вы и подати платите, ибо они Божии служители, сим самым постоянно занятые.
\par 7 Итак отдавайте всякому должное: кому подать, подать; кому оброк, оброк; кому страх, страх; кому честь, честь.
\par 8 Не оставайтесь должными никому ничем, кроме взаимной любви; ибо любящий другого исполнил закон.
\par 9 Ибо заповеди: не прелюбодействуй, не убивай, не кради, не лжесвидетельствуй, не пожелай [чужого] и все другие заключаются в сем слове: люби ближнего твоего, как самого себя.
\par 10 Любовь не делает ближнему зла; итак любовь есть исполнение закона.
\par 11 Так [поступайте], зная время, что наступил уже час пробудиться нам от сна. Ибо ныне ближе к нам спасение, нежели когда мы уверовали.
\par 12 Ночь прошла, а день приблизился: итак отвергнем дела тьмы и облечемся в оружия света.
\par 13 Как днем, будем вести себя благочинно, не [предаваясь] ни пированиям и пьянству, ни сладострастию и распутству, ни ссорам и зависти;
\par 14 но облекитесь в Господа нашего Иисуса Христа, и попечения о плоти не превращайте в похоти.

\chapter{14}

\par 1 Немощного в вере принимайте без споров о мнениях.
\par 2 Ибо иной уверен, [что можно] есть все, а немощный ест овощи.
\par 3 Кто ест, не уничижай того, кто не ест; и кто не ест, не осуждай того, кто ест, потому что Бог принял его.
\par 4 Кто ты, осуждающий чужого раба? Перед своим Господом стоит он, или падает. И будет восставлен, ибо силен Бог восставить его.
\par 5 Иной отличает день от дня, а другой судит о всяком дне [равно]. Всякий [поступай] по удостоверению своего ума.
\par 6 Кто различает дни, для Господа различает; и кто не различает дней, для Господа не различает. Кто ест, для Господа ест, ибо благодарит Бога; и кто не ест, для Господа не ест, и благодарит Бога.
\par 7 Ибо никто из нас не живет для себя, и никто не умирает для себя;
\par 8 а живем ли--для Господа живем; умираем ли--для Господа умираем: и потому, живем ли или умираем, --[всегда] Господни.
\par 9 Ибо Христос для того и умер, и воскрес, и ожил, чтобы владычествовать и над мертвыми и над живыми.
\par 10 А ты что осуждаешь брата твоего? Или и ты, что унижаешь брата твоего? Все мы предстанем на суд Христов.
\par 11 Ибо написано: живу Я, говорит Господь, предо Мною преклонится всякое колено, и всякий язык будет исповедывать Бога.
\par 12 Итак каждый из нас за себя даст отчет Богу.
\par 13 Не станем же более судить друг друга, а лучше судите о том, как бы не подавать брату [случая к] преткновению или соблазну.
\par 14 Я знаю и уверен в Господе Иисусе, что нет ничего в себе самом нечистого; только почитающему что-либо нечистым, тому нечисто.
\par 15 Если же за пищу огорчается брат твой, то ты уже не по любви поступаешь. Не губи твоею пищею того, за кого Христос умер.
\par 16 Да не хулится ваше доброе.
\par 17 Ибо Царствие Божие не пища и питие, но праведность и мир и радость во Святом Духе.
\par 18 Кто сим служит Христу, тот угоден Богу и [достоин] одобрения от людей.
\par 19 Итак будем искать того, что служит к миру и ко взаимному назиданию.
\par 20 Ради пищи не разрушай дела Божия. Все чисто, но худо человеку, который ест на соблазн.
\par 21 Лучше не есть мяса, не пить вина и не [делать] ничего [такого], отчего брат твой претыкается, или соблазняется, или изнемогает.
\par 22 Ты имеешь веру? имей ее сам в себе, пред Богом. Блажен, кто не осуждает себя в том, что избирает.
\par 23 А сомневающийся, если ест, осуждается, потому что не по вере; а все, что не по вере, грех.
\par 24 Могущему же утвердить вас, по благовествованию моему и проповеди Иисуса Христа, по откровению тайны, о которой от вечных времен было умолчано,
\par 25 но которая ныне явлена, и через писания пророческие, по повелению вечного Бога, возвещена всем народам для покорения их вере,
\par 26 Единому Премудрому Богу, через Иисуса Христа, слава во веки. Аминь.

\chapter{15}

\par 1 Мы, сильные, должны сносить немощи бессильных и не себе угождать.
\par 2 Каждый из нас должен угождать ближнему, во благо, к назиданию.
\par 3 Ибо и Христос не Себе угождал, но, как написано: злословия злословящих Тебя пали на Меня.
\par 4 А все, что писано было прежде, написано нам в наставление, чтобы мы терпением и утешением из Писаний сохраняли надежду.
\par 5 Бог же терпения и утешения да дарует вам быть в единомыслии между собою, по [учению] Христа Иисуса,
\par 6 дабы вы единодушно, едиными устами славили Бога и Отца Господа нашего Иисуса Христа.
\par 7 Посему принимайте друг друга, как и Христос принял вас в славу Божию.
\par 8 Разумею то, что Иисус Христос сделался служителем для обрезанных--ради истины Божией, чтобы исполнить обещанное отцам,
\par 9 а для язычников--из милости, чтобы славили Бога, как написано: за то буду славить Тебя, (Господи,) между язычниками, и буду петь имени Твоему.
\par 10 И еще сказано: возвеселитесь, язычники, с народом Его.
\par 11 И еще: хвалите Господа, все язычники, и прославляйте Его, все народы.
\par 12 Исаия также говорит: будет корень Иессеев, и восстанет владеть народами; на Него язычники надеяться будут.
\par 13 Бог же надежды да исполнит вас всякой радости и мира в вере, дабы вы, силою Духа Святаго, обогатились надеждою.
\par 14 И сам я уверен о вас, братия мои, что и вы полны благости, исполнены всякого познания и можете наставлять друг друга;
\par 15 но писал вам, братия, с некоторою смелостью, отчасти как бы в напоминание вам, по данной мне от Бога благодати
\par 16 быть служителем Иисуса Христа у язычников и [совершать] священнодействие благовествования Божия, дабы сие приношение язычников, будучи освящено Духом Святым, было благоприятно [Богу].
\par 17 Итак я могу похвалиться в Иисусе Христе в том, что [относится] к Богу,
\par 18 ибо не осмелюсь сказать что-нибудь такое, чего не совершил Христос через меня, в покорении язычников [вере], словом и делом,
\par 19 силою знамений и чудес, силою Духа Божия, так что благовествование Христово распространено мною от Иерусалима и окрестности до Иллирика.
\par 20 Притом я старался благовествовать не там, где [уже] было известно имя Христово, дабы не созидать на чужом основании,
\par 21 но как написано: не имевшие о Нем известия увидят, и не слышавшие узнают.
\par 22 Сие-то много раз и препятствовало мне придти к вам.
\par 23 Ныне же, не имея [такого] места в сих странах, а с давних лет имея желание придти к вам,
\par 24 как только предприму путь в Испанию, приду к вам. Ибо надеюсь, что, проходя, увижусь с вами и что вы проводите меня туда, как скоро наслажусь [общением] с вами, хотя отчасти.
\par 25 А теперь я иду в Иерусалим, чтобы послужить святым,
\par 26 ибо Македония и Ахаия усердствуют некоторым подаянием для бедных между святыми в Иерусалиме.
\par 27 Усердствуют, да и должники они перед ними. Ибо если язычники сделались участниками в их духовном, то должны и им послужить в телесном.
\par 28 Исполнив это и верно доставив им сей плод [усердия], я отправлюсь через ваши [места] в Испанию,
\par 29 и уверен, что когда приду к вам, то приду с полным благословением благовествования Христова.
\par 30 Между тем умоляю вас, братия, Господом нашим Иисусом Христом и любовью Духа, подвизаться со мною в молитвах за меня к Богу,
\par 31 чтобы избавиться мне от неверующих в Иудее и чтобы служение мое для Иерусалима было благоприятно святым,
\par 32 дабы мне в радости, если Богу угодно, придти к вам и успокоиться с вами.
\par 33 Бог же мира да будет со всеми вами, аминь.

\chapter{16}

\par 1 Представляю вам Фиву, сестру нашу, диакониссу церкви Кенхрейской.
\par 2 Примите ее для Господа, как прилично святым, и помогите ей, в чем она будет иметь нужду у вас, ибо и она была помощницею многим и мне самому.
\par 3 Приветствуйте Прискиллу и Акилу, сотрудников моих во Христе Иисусе
\par 4 (которые голову свою полагали за мою душу, которых не я один благодарю, но и все церкви из язычников), и домашнюю их церковь.
\par 5 Приветствуйте возлюбленного моего Епенета, который есть начаток Ахаии для Христа.
\par 6 Приветствуйте Мариам, которая много трудилась для нас.
\par 7 Приветствуйте Андроника и Юнию, сродников моих и узников со мною, прославившихся между Апостолами и прежде меня еще уверовавших во Христа.
\par 8 Приветствуйте Амплия, возлюбленного мне в Господе.
\par 9 Приветствуйте Урбана, сотрудника нашего во Христе, и Стахия, возлюбленного мне.
\par 10 Приветствуйте Апеллеса, испытанного во Христе. Приветствуйте [верных] из дома Аристовулова.
\par 11 Приветствуйте Иродиона, сродника моего. Приветствуйте из домашних Наркисса тех, которые в Господе.
\par 12 Приветствуйте Трифену и Трифосу, трудящихся о Господе. Приветствуйте Персиду возлюбленную, которая много потрудилась о Господе.
\par 13 Приветствуйте Руфа, избранного в Господе, и матерь его и мою.
\par 14 Приветствуйте Асинкрита, Флегонта, Ерма, Патрова, Ермия и других с ними братьев.
\par 15 Приветствуйте Филолога и Юлию, Нирея и сестру его, и Олимпана, и всех с ними святых.
\par 16 Приветствуйте друг друга с целованием святым. Приветствуют вас все церкви Христовы.
\par 17 Умоляю вас, братия, остерегайтесь производящих разделения и соблазны, вопреки учению, которому вы научились, и уклоняйтесь от них;
\par 18 ибо такие [люди] служат не Господу нашему Иисусу Христу, а своему чреву, и ласкательством и красноречием обольщают сердца простодушных.
\par 19 Ваша покорность [вере] всем известна; посему я радуюсь за вас, но желаю, чтобы вы были мудры на добро и просты на зло.
\par 20 Бог же мира сокрушит сатану под ногами вашими вскоре. Благодать Господа нашего Иисуса Христа с вами! Аминь.
\par 21 Приветствуют вас Тимофей, сотрудник мой, и Луций, Иасон и Сосипатр, сродники мои.
\par 22 Приветствую вас в Господе и я, Тертий, писавший сие послание.
\par 23 Приветствует вас Гаий, странноприимец мой и всей церкви. Приветствует вас Ераст, городской казнохранитель, и брат Кварт.
\par 24 Благодать Господа нашего Иисуса Христа со всеми вами. Аминь.


\end{document}